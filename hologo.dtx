% \iffalse meta-comment
%
% File: hologo.dtx
% Version: 2016/05/12 v1.11
% Info: A logo collection with bookmark support
%
% Copyright (C) 2010-2012 by
%    Heiko Oberdiek <heiko.oberdiek at googlemail.com>
%
% This work may be distributed and/or modified under the
% conditions of the LaTeX Project Public License, either
% version 1.3c of this license or (at your option) any later
% version. This version of this license is in
%    http://www.latex-project.org/lppl/lppl-1-3c.txt
% and the latest version of this license is in
%    http://www.latex-project.org/lppl.txt
% and version 1.3 or later is part of all distributions of
% LaTeX version 2005/12/01 or later.
%
% This work has the LPPL maintenance status "maintained".
%
% This Current Maintainer of this work is Heiko Oberdiek.
%
% The Base Interpreter refers to any `TeX-Format',
% because some files are installed in TDS:tex/generic//.
%
% This work consists of the main source file hologo.dtx
% and the derived files
%    hologo.sty, hologo.pdf, hologo.ins, hologo.drv, hologo-example.tex,
%    hologo-test1.tex, hologo-test-spacefactor.tex,
%    hologo-test-list.tex.
%
% Distribution:
%    CTAN:macros/latex/contrib/oberdiek/hologo.dtx
%    CTAN:macros/latex/contrib/oberdiek/hologo.pdf
%
% Unpacking:
%    (a) If hologo.ins is present:
%           tex hologo.ins
%    (b) Without hologo.ins:
%           tex hologo.dtx
%    (c) If you insist on using LaTeX
%           latex \let\install=y% \iffalse meta-comment
%
% File: hologo.dtx
% Version: 2016/05/12 v1.11
% Info: A logo collection with bookmark support
%
% Copyright (C) 2010-2012 by
%    Heiko Oberdiek <heiko.oberdiek at googlemail.com>
%
% This work may be distributed and/or modified under the
% conditions of the LaTeX Project Public License, either
% version 1.3c of this license or (at your option) any later
% version. This version of this license is in
%    http://www.latex-project.org/lppl/lppl-1-3c.txt
% and the latest version of this license is in
%    http://www.latex-project.org/lppl.txt
% and version 1.3 or later is part of all distributions of
% LaTeX version 2005/12/01 or later.
%
% This work has the LPPL maintenance status "maintained".
%
% This Current Maintainer of this work is Heiko Oberdiek.
%
% The Base Interpreter refers to any `TeX-Format',
% because some files are installed in TDS:tex/generic//.
%
% This work consists of the main source file hologo.dtx
% and the derived files
%    hologo.sty, hologo.pdf, hologo.ins, hologo.drv, hologo-example.tex,
%    hologo-test1.tex, hologo-test-spacefactor.tex,
%    hologo-test-list.tex.
%
% Distribution:
%    CTAN:macros/latex/contrib/oberdiek/hologo.dtx
%    CTAN:macros/latex/contrib/oberdiek/hologo.pdf
%
% Unpacking:
%    (a) If hologo.ins is present:
%           tex hologo.ins
%    (b) Without hologo.ins:
%           tex hologo.dtx
%    (c) If you insist on using LaTeX
%           latex \let\install=y% \iffalse meta-comment
%
% File: hologo.dtx
% Version: 2016/05/12 v1.11
% Info: A logo collection with bookmark support
%
% Copyright (C) 2010-2012 by
%    Heiko Oberdiek <heiko.oberdiek at googlemail.com>
%
% This work may be distributed and/or modified under the
% conditions of the LaTeX Project Public License, either
% version 1.3c of this license or (at your option) any later
% version. This version of this license is in
%    http://www.latex-project.org/lppl/lppl-1-3c.txt
% and the latest version of this license is in
%    http://www.latex-project.org/lppl.txt
% and version 1.3 or later is part of all distributions of
% LaTeX version 2005/12/01 or later.
%
% This work has the LPPL maintenance status "maintained".
%
% This Current Maintainer of this work is Heiko Oberdiek.
%
% The Base Interpreter refers to any `TeX-Format',
% because some files are installed in TDS:tex/generic//.
%
% This work consists of the main source file hologo.dtx
% and the derived files
%    hologo.sty, hologo.pdf, hologo.ins, hologo.drv, hologo-example.tex,
%    hologo-test1.tex, hologo-test-spacefactor.tex,
%    hologo-test-list.tex.
%
% Distribution:
%    CTAN:macros/latex/contrib/oberdiek/hologo.dtx
%    CTAN:macros/latex/contrib/oberdiek/hologo.pdf
%
% Unpacking:
%    (a) If hologo.ins is present:
%           tex hologo.ins
%    (b) Without hologo.ins:
%           tex hologo.dtx
%    (c) If you insist on using LaTeX
%           latex \let\install=y% \iffalse meta-comment
%
% File: hologo.dtx
% Version: 2016/05/12 v1.11
% Info: A logo collection with bookmark support
%
% Copyright (C) 2010-2012 by
%    Heiko Oberdiek <heiko.oberdiek at googlemail.com>
%
% This work may be distributed and/or modified under the
% conditions of the LaTeX Project Public License, either
% version 1.3c of this license or (at your option) any later
% version. This version of this license is in
%    http://www.latex-project.org/lppl/lppl-1-3c.txt
% and the latest version of this license is in
%    http://www.latex-project.org/lppl.txt
% and version 1.3 or later is part of all distributions of
% LaTeX version 2005/12/01 or later.
%
% This work has the LPPL maintenance status "maintained".
%
% This Current Maintainer of this work is Heiko Oberdiek.
%
% The Base Interpreter refers to any `TeX-Format',
% because some files are installed in TDS:tex/generic//.
%
% This work consists of the main source file hologo.dtx
% and the derived files
%    hologo.sty, hologo.pdf, hologo.ins, hologo.drv, hologo-example.tex,
%    hologo-test1.tex, hologo-test-spacefactor.tex,
%    hologo-test-list.tex.
%
% Distribution:
%    CTAN:macros/latex/contrib/oberdiek/hologo.dtx
%    CTAN:macros/latex/contrib/oberdiek/hologo.pdf
%
% Unpacking:
%    (a) If hologo.ins is present:
%           tex hologo.ins
%    (b) Without hologo.ins:
%           tex hologo.dtx
%    (c) If you insist on using LaTeX
%           latex \let\install=y\input{hologo.dtx}
%        (quote the arguments according to the demands of your shell)
%
% Documentation:
%    (a) If hologo.drv is present:
%           latex hologo.drv
%    (b) Without hologo.drv:
%           latex hologo.dtx; ...
%    The class ltxdoc loads the configuration file ltxdoc.cfg
%    if available. Here you can specify further options, e.g.
%    use A4 as paper format:
%       \PassOptionsToClass{a4paper}{article}
%
%    Programm calls to get the documentation (example):
%       pdflatex hologo.dtx
%       makeindex -s gind.ist hologo.idx
%       pdflatex hologo.dtx
%       makeindex -s gind.ist hologo.idx
%       pdflatex hologo.dtx
%
% Installation:
%    TDS:tex/generic/oberdiek/hologo.sty
%    TDS:doc/latex/oberdiek/hologo.pdf
%    TDS:doc/latex/oberdiek/example/hologo-example.tex
%    TDS:doc/latex/oberdiek/test/hologo-test1.tex
%    TDS:doc/latex/oberdiek/test/hologo-test-spacefactor.tex
%    TDS:doc/latex/oberdiek/test/hologo-test-list.tex
%    TDS:source/latex/oberdiek/hologo.dtx
%
%<*ignore>
\begingroup
  \catcode123=1 %
  \catcode125=2 %
  \def\x{LaTeX2e}%
\expandafter\endgroup
\ifcase 0\ifx\install y1\fi\expandafter
         \ifx\csname processbatchFile\endcsname\relax\else1\fi
         \ifx\fmtname\x\else 1\fi\relax
\else\csname fi\endcsname
%</ignore>
%<*install>
\input docstrip.tex
\Msg{************************************************************************}
\Msg{* Installation}
\Msg{* Package: hologo 2016/05/12 v1.11 A logo collection with bookmark support (HO)}
\Msg{************************************************************************}

\keepsilent
\askforoverwritefalse

\let\MetaPrefix\relax
\preamble

This is a generated file.

Project: hologo
Version: 2016/05/12 v1.11

Copyright (C) 2010-2012 by
   Heiko Oberdiek <heiko.oberdiek at googlemail.com>

This work may be distributed and/or modified under the
conditions of the LaTeX Project Public License, either
version 1.3c of this license or (at your option) any later
version. This version of this license is in
   http://www.latex-project.org/lppl/lppl-1-3c.txt
and the latest version of this license is in
   http://www.latex-project.org/lppl.txt
and version 1.3 or later is part of all distributions of
LaTeX version 2005/12/01 or later.

This work has the LPPL maintenance status "maintained".

This Current Maintainer of this work is Heiko Oberdiek.

The Base Interpreter refers to any `TeX-Format',
because some files are installed in TDS:tex/generic//.

This work consists of the main source file hologo.dtx
and the derived files
   hologo.sty, hologo.pdf, hologo.ins, hologo.drv, hologo-example.tex,
   hologo-test1.tex, hologo-test-spacefactor.tex,
   hologo-test-list.tex.

\endpreamble
\let\MetaPrefix\DoubleperCent

\generate{%
  \file{hologo.ins}{\from{hologo.dtx}{install}}%
  \file{hologo.drv}{\from{hologo.dtx}{driver}}%
  \usedir{tex/generic/oberdiek}%
  \file{hologo.sty}{\from{hologo.dtx}{package}}%
  \usedir{doc/latex/oberdiek/example}%
  \file{hologo-example.tex}{\from{hologo.dtx}{example}}%
  \usedir{doc/latex/oberdiek/test}%
  \file{hologo-test1.tex}{\from{hologo.dtx}{test1}}%
  \file{hologo-test-spacefactor.tex}{\from{hologo.dtx}{test-spacefactor}}%
  \file{hologo-test-list.tex}{\from{hologo.dtx}{test-list}}%
  \nopreamble
  \nopostamble
  \usedir{source/latex/oberdiek/catalogue}%
  \file{hologo.xml}{\from{hologo.dtx}{catalogue}}%
}

\catcode32=13\relax% active space
\let =\space%
\Msg{************************************************************************}
\Msg{*}
\Msg{* To finish the installation you have to move the following}
\Msg{* file into a directory searched by TeX:}
\Msg{*}
\Msg{*     hologo.sty}
\Msg{*}
\Msg{* To produce the documentation run the file `hologo.drv'}
\Msg{* through LaTeX.}
\Msg{*}
\Msg{* Happy TeXing!}
\Msg{*}
\Msg{************************************************************************}

\endbatchfile
%</install>
%<*ignore>
\fi
%</ignore>
%<*driver>
\NeedsTeXFormat{LaTeX2e}
\ProvidesFile{hologo.drv}%
  [2016/05/12 v1.11 A logo collection with bookmark support (HO)]%
\documentclass{ltxdoc}
\usepackage{holtxdoc}[2011/11/22]
\usepackage{hologo}[2016/05/12]
\usepackage{longtable}
\usepackage{array}
\usepackage{paralist}
%\usepackage[T1]{fontenc}
%\usepackage{lmodern}
\begin{document}
  \DocInput{hologo.dtx}%
\end{document}
%</driver>
% \fi
%
%
% \CharacterTable
%  {Upper-case    \A\B\C\D\E\F\G\H\I\J\K\L\M\N\O\P\Q\R\S\T\U\V\W\X\Y\Z
%   Lower-case    \a\b\c\d\e\f\g\h\i\j\k\l\m\n\o\p\q\r\s\t\u\v\w\x\y\z
%   Digits        \0\1\2\3\4\5\6\7\8\9
%   Exclamation   \!     Double quote  \"     Hash (number) \#
%   Dollar        \$     Percent       \%     Ampersand     \&
%   Acute accent  \'     Left paren    \(     Right paren   \)
%   Asterisk      \*     Plus          \+     Comma         \,
%   Minus         \-     Point         \.     Solidus       \/
%   Colon         \:     Semicolon     \;     Less than     \<
%   Equals        \=     Greater than  \>     Question mark \?
%   Commercial at \@     Left bracket  \[     Backslash     \\
%   Right bracket \]     Circumflex    \^     Underscore    \_
%   Grave accent  \`     Left brace    \{     Vertical bar  \|
%   Right brace   \}     Tilde         \~}
%
% \GetFileInfo{hologo.drv}
%
% \title{The \xpackage{hologo} package}
% \date{2016/05/12 v1.11}
% \author{Heiko Oberdiek\\\xemail{heiko.oberdiek at googlemail.com}}
%
% \maketitle
%
% \begin{abstract}
% This package starts a collection of logos with support for bookmarks
% strings.
% \end{abstract}
%
% \tableofcontents
%
% \section{Documentation}
%
% \subsection{Logo macros}
%
% \begin{declcs}{hologo} \M{name}
% \end{declcs}
% Macro \cs{hologo} sets the logo with name \meta{name}.
% The following table shows the supported names.
%
% \begingroup
%   \def\hologoEntry#1#2#3{^^A
%     #1&#2&\hologoLogoSetup{#1}{variant=#2}\hologo{#1}&#3\tabularnewline
%   }
%   \begin{longtable}{>{\ttfamily}l>{\ttfamily}lll}
%     \rmfamily\bfseries{name} & \rmfamily\bfseries variant
%     & \bfseries logo & \bfseries since\\
%     \hline
%     \endhead
%     \hologoList
%   \end{longtable}
% \endgroup
%
% \begin{declcs}{Hologo} \M{name}
% \end{declcs}
% Macro \cs{Hologo} starts the logo \meta{name} with an uppercase
% letter. As an exception small greek letters are not converted
% to uppercase. Examples, see \hologo{eTeX} and \hologo{ExTeX}.
%
% \subsection{Setup macros}
%
% The package does not support package options, but the following
% setup macros can be used to set options.
%
% \begin{declcs}{hologoSetup} \M{key value list}
% \end{declcs}
% Macro \cs{hologoSetup} sets global options.
%
% \begin{declcs}{hologoLogoSetup} \M{logo} \M{key value list}
% \end{declcs}
% Some options can also be used to configure a logo.
% These settings take precedence over global option settings.
%
% \subsection{Options}\label{sec:options}
%
% There are boolean and string options:
% \begin{description}
% \item[Boolean option:]
% It takes |true| or |false|
% as value. If the value is omitted, then |true| is used.
% \item[String option:]
% A value must be given as string. (But the string might be empty.)
% \end{description}
% The following options can be used both in \cs{hologoSetup}
% and \cs{hologoLogoSetup}:
% \begin{description}
% \def\entry#1{\item[\xoption{#1}:]}
% \entry{break}
%   enables or disables line breaks inside the logo. This setting is
%   refined by options \xoption{hyphenbreak}, \xoption{spacebreak}
%   or \xoption{discretionarybreak}.
%   Default is |false|.
% \entry{hyphenbreak}
%   enables or disables the line break right after the hyphen character.
% \entry{spacebreak}
%   enables or disables line breaks at space characters.
% \entry{discretionarybreak}
%   enables or disables line breaks at hyphenation points
%   (inserted by \cs{-}).
% \end{description}
% Macro \cs{hologoLogoSetup} also knows:
% \begin{description}
% \item[\xoption{variant}:]
%   This is a string option. It specifies a variant of a logo that
%   must exist. An empty string selects the package default variant.
% \end{description}
% Example:
% \begin{quote}
%   |\hologoSetup{break=false}|\\
%   |\hologoLogoSetup{plainTeX}{variant=hyphen,hyphenbreak}|\\
%   Then ``plain-\TeX'' contains one break point after the hyphen.
% \end{quote}
%
% \subsection{Driver options}
%
% Sometimes graphical operations are needed to construct some
% glyphs (e.g.\ \hologo{XeTeX}). If package \xpackage{graphics}
% or package \xpackage{pgf} are found, then the macros are taken
% from there. Otherwise the packge defines its own operations
% and therefore needs the driver information. Many drivers are
% detected automatically (\hologo{pdfTeX}/\hologo{LuaTeX}
% in PDF mode, \hologo{XeTeX}, \hologo{VTeX}). These have precedence
% over a driver option. The driver can be given as package option
% or using \cs{hologoDriverSetup}.
% The following list contains the recognized driver options:
% \begin{itemize}
% \item \xoption{pdftex}, \xoption{luatex}
% \item \xoption{dvipdfm}, \xoption{dvipdfmx}
% \item \xoption{dvips}, \xoption{dvipsone}, \xoption{xdvi}
% \item \xoption{xetex}
% \item \xoption{vtex}
% \end{itemize}
% The left driver of a line is the driver name that is used internally.
% The following names are aliases for drivers that use the
% same method. Therefore the entry in the \xext{log} file for
% the used driver prints the internally used driver name.
% \begin{description}
% \item[\xoption{driverfallback}:]
%   This option expects a driver that is used,
%   if the driver could not be detected automatically.
% \end{description}
%
% \begin{declcs}{hologoDriverSetup} \M{driver option}
% \end{declcs}
% The driver can also be configured after package loading
% using \cs{hologoDriverSetup}, also the way for \hologo{plainTeX}
% to setup the driver.
%
% \subsection{Font setup}
%
% Some logos require a special font, but should also be usable by
% \hologo{plainTeX}. Therefore the package provides some ways
% to influence the font settings. The options below
% take font settings as values. Both font commands
% such as \cs{sffamily} and macros that take one argument
% like \cs{textsf} can be used.
%
% \begin{declcs}{hologoFontSetup} \M{key value list}
% \end{declcs}
% Macro \cs{hologoFontSetup} sets the fonts for all logos.
% Supported keys:
% \begin{description}
% \def\entry#1{\item[\xoption{#1}:]}
% \entry{general}
%   This font is used for all logos. The default is empty.
%   That means no special font is used.
% \entry{bibsf}
%   This font is used for
%   {\hologoLogoSetup{BibTeX}{variant=sf}\hologo{BibTeX}}
%   with variant \xoption{sf}.
% \entry{rm}
%   This font is a serif font. It is used for \hologo{ExTeX}.
% \entry{sc}
%   This font specifies a small caps font. It is used for
%   {\hologoLogoSetup{BibTeX}{variant=sc}\hologo{BibTeX}}
%   with variant \xoption{sc}.
% \entry{sf}
%   This font specifies a sans serif font. The default
%   is \cs{sffamily}, then \cs{sf} is tried. Otherwise
%   a warning is given. It is used by \hologo{KOMAScript}.
% \entry{sy}
%   This is the font for math symbols (e.g. cmsy).
%   It is used by \hologo{AmS}, \hologo{NTS}, \hologo{ExTeX}.
% \entry{logo}
%   \hologo{METAFONT} and \hologo{METAPOST} are using that font.
%   In \hologo{LaTeX} \cs{logofamily} is used and
%   the definitions of package \xpackage{mflogo} are used
%   if the package is not loaded.
%   Otherwise the \cs{tenlogo} is used and defined
%   if it does not already exists.
% \end{description}
%
% \begin{declcs}{hologoLogoFontSetup} \M{logo} \M{key value list}
% \end{declcs}
% Fonts can also be set for a logo or logo component separately,
% see the following list.
% The keys are the same as for \cs{hologoFontSetup}.
%
% \begin{longtable}{>{\ttfamily}l>{\sffamily}ll}
%   \meta{logo} & keys & result\\
%   \hline
%   \endhead
%   BibTeX & bibsf & {\hologoLogoSetup{BibTeX}{variant=sf}\hologo{BibTeX}}\\[.5ex]
%   BibTeX & sc & {\hologoLogoSetup{BibTeX}{variant=sc}\hologo{BibTeX}}\\[.5ex]
%   ExTeX & rm & \hologo{ExTeX}\\
%   SliTeX & rm & \hologo{SliTeX}\\[.5ex]
%   AmS & sy & \hologo{AmS}\\
%   ExTeX & sy & \hologo{ExTeX}\\
%   NTS & sy & \hologo{NTS}\\[.5ex]
%   KOMAScript & sf & \hologo{KOMAScript}\\[.5ex]
%   METAFONT & logo & \hologo{METAFONT}\\
%   METAPOST & logo & \hologo{METAPOST}\\[.5ex]
%   SliTeX & sc \hologo{SliTeX}
% \end{longtable}
%
% \subsubsection{Font order}
%
% For all logos the font \xoption{general} is applied first.
% Example:
%\begin{quote}
%|\hologoFontSetup{general=\color{red}}|
%\end{quote}
% will print red logos.
% Then if the font uses a special font \xoption{sf}, for example,
% the font is applied that is setup by \cs{hologoLogoFontSetup}.
% If this font is not setup, then the common font setup
% by \cs{hologoFontSetup} is used. Otherwise a warning is given,
% that there is no font configured.
%
% \subsection{Additional user macros}
%
% Usually a variant of a logo is configured by using
% \cs{hologoLogoSetup}, because it is bad style to mix
% different variants of the same logo in the same text.
% There the following macros are a convenience for testing.
%
% \begin{declcs}{hologoVariant} \M{name} \M{variant}\\
%   \cs{HologoVariant} \M{name} \M{variant}
% \end{declcs}
% Logo \meta{name} is set using \meta{variant} that specifies
% explicitely which variant of the macro is used. If the argument
% is empty, then the default form of the logo is used
% (configurable by \cs{hologoLogoSetup}).
%
% \cs{HologoVariant} is used if the logo is set in a context
% that needs an uppercase first letter (beginning of a sentence, \dots).
%
% \begin{declcs}{hologoList}\\
%   \cs{hologoEntry} \M{logo} \M{variant} \M{since}
% \end{declcs}
% Macro \cs{hologoList} contains all logos that are provided
% by the package including variants. The list consists of calls
% of \cs{hologoEntry} with three arguments starting with the
% logo name \meta{logo} and its variant \meta{variant}. An empty
% variant means the current default. Argument \meta{since} specifies
% with version of the package \xpackage{hologo} is needed to get
% the logo. If the logo is fixed, then the date gets updated.
% Therefore the date \meta{since} is not exactly the date of
% the first introduction, but rather the date of the latest fix.
%
% Before \cs{hologoList} can be used, macro \cs{hologoEntry} needs
% a definition. The example file in section \ref{sec:example}
% shows applications of \cs{hologoList}.
%
% \subsection{Supported contexts}
%
% Macros \cs{hologo} and friends support special contexts:
% \begin{itemize}
% \item \hologo{LaTeX}'s protection mechanism.
% \item Bookmarks of package \xpackage{hyperref}.
% \item Package \xpackage{tex4ht}.
% \item The macros can be used inside \cs{csname} constructs,
%   if \cs{ifincsname} is available (\hologo{pdfTeX}, \hologo{XeTeX},
%   \hologo{LuaTeX}).
% \end{itemize}
%
% \subsection{Example}
% \label{sec:example}
%
% The following example prints the logos in different fonts.
%    \begin{macrocode}
%<*example>
%<<verbatim
\NeedsTeXFormat{LaTeX2e}
\documentclass[a4paper]{article}
\usepackage[
  hmargin=20mm,
  vmargin=20mm,
]{geometry}
\pagestyle{empty}
\usepackage{hologo}[2016/05/12]
\usepackage{longtable}
\usepackage{array}
\setlength{\extrarowheight}{2pt}
\usepackage[T1]{fontenc}
\usepackage{lmodern}
\usepackage{pdflscape}
\usepackage[
  pdfencoding=auto,
]{hyperref}
\hypersetup{
  pdfauthor={Heiko Oberdiek},
  pdftitle={Example for package `hologo'},
  pdfsubject={Logos with fonts lmr, lmss, qtm, qpl, qhv},
}
\usepackage{bookmark}

% Print the logo list on the console

\begingroup
  \typeout{}%
  \typeout{*** Begin of logo list ***}%
  \newcommand*{\hologoEntry}[3]{%
    \typeout{#1 \ifx\\#2\\\else(#2) \fi[#3]}%
  }%
  \hologoList
  \typeout{*** End of logo list ***}%
  \typeout{}%
\endgroup

\begin{document}
\begin{landscape}

  \section{Example file for package `hologo'}

  % Table for font names

  \begin{longtable}{>{\bfseries}ll}
    \textbf{font} & \textbf{Font name}\\
    \hline
    lmr & Latin Modern Roman\\
    lmss & Latin Modern Sans\\
    qtm & \TeX\ Gyre Termes\\
    qhv & \TeX\ Gyre Heros\\
    qpl & \TeX\ Gyre Pagella\\
  \end{longtable}

  % Logo list with logos in different fonts

  \begingroup
    \newcommand*{\SetVariant}[2]{%
      \ifx\\#2\\%
      \else
        \hologoLogoSetup{#1}{variant=#2}%
      \fi
    }%
    \newcommand*{\hologoEntry}[3]{%
      \SetVariant{#1}{#2}%
      \raisebox{1em}[0pt][0pt]{\hypertarget{#1@#2}{}}%
      \bookmark[%
        dest={#1@#2},%
      ]{%
        #1\ifx\\#2\\\else\space(#2)\fi: \Hologo{#1}, \hologo{#1} %
        [Unicode]%
      }%
      \hypersetup{unicode=false}%
      \bookmark[%
        dest={#1@#2},%
      ]{%
        #1\ifx\\#2\\\else\space(#2)\fi: \Hologo{#1}, \hologo{#1} %
        [PDFDocEncoding]%
      }%
      \texttt{#1}%
      &%
      \texttt{#2}%
      &%
      \Hologo{#1}%
      &%
      \SetVariant{#1}{#2}%
      \hologo{#1}%
      &%
      \SetVariant{#1}{#2}%
      \fontfamily{qtm}\selectfont
      \hologo{#1}%
      &%
      \SetVariant{#1}{#2}%
      \fontfamily{qpl}\selectfont
      \hologo{#1}%
      &%
      \SetVariant{#1}{#2}%
      \textsf{\hologo{#1}}%
      &%
      \SetVariant{#1}{#2}%
      \fontfamily{qhv}\selectfont
      \hologo{#1}%
      \tabularnewline
    }%
    \begin{longtable}{llllllll}%
      \textbf{\textit{logo}} & \textbf{\textit{variant}} &
      \texttt{\string\Hologo} &
      \textbf{lmr} & \textbf{qtm} & \textbf{qpl} &
      \textbf{lmss} & \textbf{qhv}
      \tabularnewline
      \hline
      \endhead
      \hologoList
    \end{longtable}%
  \endgroup

\end{landscape}
\end{document}
%verbatim
%</example>
%    \end{macrocode}
%
% \StopEventually{
% }
%
% \section{Implementation}
%    \begin{macrocode}
%<*package>
%    \end{macrocode}
%    Reload check, especially if the package is not used with \LaTeX.
%    \begin{macrocode}
\begingroup\catcode61\catcode48\catcode32=10\relax%
  \catcode13=5 % ^^M
  \endlinechar=13 %
  \catcode35=6 % #
  \catcode39=12 % '
  \catcode44=12 % ,
  \catcode45=12 % -
  \catcode46=12 % .
  \catcode58=12 % :
  \catcode64=11 % @
  \catcode123=1 % {
  \catcode125=2 % }
  \expandafter\let\expandafter\x\csname ver@hologo.sty\endcsname
  \ifx\x\relax % plain-TeX, first loading
  \else
    \def\empty{}%
    \ifx\x\empty % LaTeX, first loading,
      % variable is initialized, but \ProvidesPackage not yet seen
    \else
      \expandafter\ifx\csname PackageInfo\endcsname\relax
        \def\x#1#2{%
          \immediate\write-1{Package #1 Info: #2.}%
        }%
      \else
        \def\x#1#2{\PackageInfo{#1}{#2, stopped}}%
      \fi
      \x{hologo}{The package is already loaded}%
      \aftergroup\endinput
    \fi
  \fi
\endgroup%
%    \end{macrocode}
%    Package identification:
%    \begin{macrocode}
\begingroup\catcode61\catcode48\catcode32=10\relax%
  \catcode13=5 % ^^M
  \endlinechar=13 %
  \catcode35=6 % #
  \catcode39=12 % '
  \catcode40=12 % (
  \catcode41=12 % )
  \catcode44=12 % ,
  \catcode45=12 % -
  \catcode46=12 % .
  \catcode47=12 % /
  \catcode58=12 % :
  \catcode64=11 % @
  \catcode91=12 % [
  \catcode93=12 % ]
  \catcode123=1 % {
  \catcode125=2 % }
  \expandafter\ifx\csname ProvidesPackage\endcsname\relax
    \def\x#1#2#3[#4]{\endgroup
      \immediate\write-1{Package: #3 #4}%
      \xdef#1{#4}%
    }%
  \else
    \def\x#1#2[#3]{\endgroup
      #2[{#3}]%
      \ifx#1\@undefined
        \xdef#1{#3}%
      \fi
      \ifx#1\relax
        \xdef#1{#3}%
      \fi
    }%
  \fi
\expandafter\x\csname ver@hologo.sty\endcsname
\ProvidesPackage{hologo}%
  [2016/05/12 v1.11 A logo collection with bookmark support (HO)]%
%    \end{macrocode}
%
%    \begin{macrocode}
\begingroup\catcode61\catcode48\catcode32=10\relax%
  \catcode13=5 % ^^M
  \endlinechar=13 %
  \catcode123=1 % {
  \catcode125=2 % }
  \catcode64=11 % @
  \def\x{\endgroup
    \expandafter\edef\csname HOLOGO@AtEnd\endcsname{%
      \endlinechar=\the\endlinechar\relax
      \catcode13=\the\catcode13\relax
      \catcode32=\the\catcode32\relax
      \catcode35=\the\catcode35\relax
      \catcode61=\the\catcode61\relax
      \catcode64=\the\catcode64\relax
      \catcode123=\the\catcode123\relax
      \catcode125=\the\catcode125\relax
    }%
  }%
\x\catcode61\catcode48\catcode32=10\relax%
\catcode13=5 % ^^M
\endlinechar=13 %
\catcode35=6 % #
\catcode64=11 % @
\catcode123=1 % {
\catcode125=2 % }
\def\TMP@EnsureCode#1#2{%
  \edef\HOLOGO@AtEnd{%
    \HOLOGO@AtEnd
    \catcode#1=\the\catcode#1\relax
  }%
  \catcode#1=#2\relax
}
\TMP@EnsureCode{10}{12}% ^^J
\TMP@EnsureCode{33}{12}% !
\TMP@EnsureCode{34}{12}% "
\TMP@EnsureCode{36}{3}% $
\TMP@EnsureCode{38}{4}% &
\TMP@EnsureCode{39}{12}% '
\TMP@EnsureCode{40}{12}% (
\TMP@EnsureCode{41}{12}% )
\TMP@EnsureCode{42}{12}% *
\TMP@EnsureCode{43}{12}% +
\TMP@EnsureCode{44}{12}% ,
\TMP@EnsureCode{45}{12}% -
\TMP@EnsureCode{46}{12}% .
\TMP@EnsureCode{47}{12}% /
\TMP@EnsureCode{58}{12}% :
\TMP@EnsureCode{59}{12}% ;
\TMP@EnsureCode{60}{12}% <
\TMP@EnsureCode{62}{12}% >
\TMP@EnsureCode{63}{12}% ?
\TMP@EnsureCode{91}{12}% [
\TMP@EnsureCode{93}{12}% ]
\TMP@EnsureCode{94}{7}% ^ (superscript)
\TMP@EnsureCode{95}{8}% _ (subscript)
\TMP@EnsureCode{96}{12}% `
\TMP@EnsureCode{124}{12}% |
\edef\HOLOGO@AtEnd{%
  \HOLOGO@AtEnd
  \escapechar\the\escapechar\relax
  \noexpand\endinput
}
\escapechar=92 %
%    \end{macrocode}
%
% \subsection{Logo list}
%
%    \begin{macro}{\hologoList}
%    \begin{macrocode}
\def\hologoList{%
  \hologoEntry{(La)TeX}{}{2011/10/01}%
  \hologoEntry{AmSLaTeX}{}{2010/04/16}%
  \hologoEntry{AmSTeX}{}{2010/04/16}%
  \hologoEntry{biber}{}{2011/10/01}%
  \hologoEntry{BibTeX}{}{2011/10/01}%
  \hologoEntry{BibTeX}{sf}{2011/10/01}%
  \hologoEntry{BibTeX}{sc}{2011/10/01}%
  \hologoEntry{BibTeX8}{}{2011/11/22}%
  \hologoEntry{ConTeXt}{}{2011/03/25}%
  \hologoEntry{ConTeXt}{narrow}{2011/03/25}%
  \hologoEntry{ConTeXt}{simple}{2011/03/25}%
  \hologoEntry{emTeX}{}{2010/04/26}%
  \hologoEntry{eTeX}{}{2010/04/08}%
  \hologoEntry{ExTeX}{}{2011/10/01}%
  \hologoEntry{HanTheThanh}{}{2011/11/29}%
  \hologoEntry{iniTeX}{}{2011/10/01}%
  \hologoEntry{KOMAScript}{}{2011/10/01}%
  \hologoEntry{La}{}{2010/05/08}%
  \hologoEntry{LaTeX}{}{2010/04/08}%
  \hologoEntry{LaTeX2e}{}{2010/04/08}%
  \hologoEntry{LaTeX3}{}{2010/04/24}%
  \hologoEntry{LaTeXe}{}{2010/04/08}%
  \hologoEntry{LaTeXML}{}{2011/11/22}%
  \hologoEntry{LaTeXTeX}{}{2011/10/01}%
  \hologoEntry{LuaLaTeX}{}{2010/04/08}%
  \hologoEntry{LuaTeX}{}{2010/04/08}%
  \hologoEntry{LyX}{}{2011/10/01}%
  \hologoEntry{METAFONT}{}{2011/10/01}%
  \hologoEntry{MetaFun}{}{2011/10/01}%
  \hologoEntry{METAPOST}{}{2011/10/01}%
  \hologoEntry{MetaPost}{}{2011/10/01}%
  \hologoEntry{MiKTeX}{}{2011/10/01}%
  \hologoEntry{NTS}{}{2011/10/01}%
  \hologoEntry{OzMF}{}{2011/10/01}%
  \hologoEntry{OzMP}{}{2011/10/01}%
  \hologoEntry{OzTeX}{}{2011/10/01}%
  \hologoEntry{OzTtH}{}{2011/10/01}%
  \hologoEntry{PCTeX}{}{2011/10/01}%
  \hologoEntry{pdfTeX}{}{2011/10/01}%
  \hologoEntry{pdfLaTeX}{}{2011/10/01}%
  \hologoEntry{PiC}{}{2011/10/01}%
  \hologoEntry{PiCTeX}{}{2011/10/01}%
  \hologoEntry{plainTeX}{}{2010/04/08}%
  \hologoEntry{plainTeX}{space}{2010/04/16}%
  \hologoEntry{plainTeX}{hyphen}{2010/04/16}%
  \hologoEntry{plainTeX}{runtogether}{2010/04/16}%
  \hologoEntry{SageTeX}{}{2011/11/22}%
  \hologoEntry{SLiTeX}{}{2011/10/01}%
  \hologoEntry{SLiTeX}{lift}{2011/10/01}%
  \hologoEntry{SLiTeX}{narrow}{2011/10/01}%
  \hologoEntry{SLiTeX}{simple}{2011/10/01}%
  \hologoEntry{SliTeX}{}{2011/10/01}%
  \hologoEntry{SliTeX}{narrow}{2011/10/01}%
  \hologoEntry{SliTeX}{simple}{2011/10/01}%
  \hologoEntry{SliTeX}{lift}{2011/10/01}%
  \hologoEntry{teTeX}{}{2011/10/01}%
  \hologoEntry{TeX}{}{2010/04/08}%
  \hologoEntry{TeX4ht}{}{2011/11/22}%
  \hologoEntry{TTH}{}{2011/11/22}%
  \hologoEntry{virTeX}{}{2011/10/01}%
  \hologoEntry{VTeX}{}{2010/04/24}%
  \hologoEntry{Xe}{}{2010/04/08}%
  \hologoEntry{XeLaTeX}{}{2010/04/08}%
  \hologoEntry{XeTeX}{}{2010/04/08}%
}
%    \end{macrocode}
%    \end{macro}
%
% \subsection{Load resources}
%
%    \begin{macrocode}
\begingroup\expandafter\expandafter\expandafter\endgroup
\expandafter\ifx\csname RequirePackage\endcsname\relax
  \def\TMP@RequirePackage#1[#2]{%
    \begingroup\expandafter\expandafter\expandafter\endgroup
    \expandafter\ifx\csname ver@#1.sty\endcsname\relax
      \input #1.sty\relax
    \fi
  }%
  \TMP@RequirePackage{ltxcmds}[2011/02/04]%
  \TMP@RequirePackage{infwarerr}[2010/04/08]%
  \TMP@RequirePackage{kvsetkeys}[2010/03/01]%
  \TMP@RequirePackage{kvdefinekeys}[2010/03/01]%
  \TMP@RequirePackage{pdftexcmds}[2010/04/01]%
  \TMP@RequirePackage{ifpdf}[2010/01/28]%
  \TMP@RequirePackage{ifluatex}[2010/03/01]%
  \ltx@IfUndefined{newif}{%
    \expandafter\let\csname newif\endcsname\ltx@newif
  }{}%
  \TMP@RequirePackage{ifxetex}[2009/01/23]%
  \TMP@RequirePackage{ifvtex}[2010/03/01]%
\else
  \RequirePackage{ltxcmds}[2011/02/04]%
  \RequirePackage{infwarerr}[2010/04/08]%
  \RequirePackage{kvsetkeys}[2010/03/01]%
  \RequirePackage{kvdefinekeys}[2010/03/01]%
  \RequirePackage{pdftexcmds}[2010/04/01]%
  \RequirePackage{ifpdf}[2010/01/28]%
  \RequirePackage{ifluatex}[2010/03/01]%
  \RequirePackage{ifxetex}[2009/01/23]%
  \RequirePackage{ifvtex}[2010/03/01]%
\fi
%    \end{macrocode}
%
%    \begin{macro}{\HOLOGO@IfDefined}
%    \begin{macrocode}
\def\HOLOGO@IfExists#1{%
  \ifx\@undefined#1%
    \expandafter\ltx@secondoftwo
  \else
    \ifx\relax#1%
      \expandafter\ltx@secondoftwo
    \else
      \expandafter\expandafter\expandafter\ltx@firstoftwo
    \fi
  \fi
}
%    \end{macrocode}
%    \end{macro}
%
% \subsection{Setup macros}
%
%    \begin{macro}{\hologoSetup}
%    \begin{macrocode}
\def\hologoSetup{%
  \let\HOLOGO@name\relax
  \HOLOGO@Setup
}
%    \end{macrocode}
%    \end{macro}
%
%    \begin{macro}{\hologoLogoSetup}
%    \begin{macrocode}
\def\hologoLogoSetup#1{%
  \edef\HOLOGO@name{#1}%
  \ltx@IfUndefined{HoLogo@\HOLOGO@name}{%
    \@PackageError{hologo}{%
      Unknown logo `\HOLOGO@name'%
    }\@ehc
    \ltx@gobble
  }{%
    \HOLOGO@Setup
  }%
}
%    \end{macrocode}
%    \end{macro}
%
%    \begin{macro}{\HOLOGO@Setup}
%    \begin{macrocode}
\def\HOLOGO@Setup{%
  \kvsetkeys{HoLogo}%
}
%    \end{macrocode}
%    \end{macro}
%
% \subsection{Options}
%
%    \begin{macro}{\HOLOGO@DeclareBoolOption}
%    \begin{macrocode}
\def\HOLOGO@DeclareBoolOption#1{%
  \expandafter\chardef\csname HOLOGOOPT@#1\endcsname\ltx@zero
  \kv@define@key{HoLogo}{#1}[true]{%
    \def\HOLOGO@temp{##1}%
    \ifx\HOLOGO@temp\HOLOGO@true
      \ifx\HOLOGO@name\relax
        \expandafter\chardef\csname HOLOGOOPT@#1\endcsname=\ltx@one
      \else
        \expandafter\chardef\csname
        HoLogoOpt@#1@\HOLOGO@name\endcsname\ltx@one
      \fi
      \HOLOGO@SetBreakAll{#1}%
    \else
      \ifx\HOLOGO@temp\HOLOGO@false
        \ifx\HOLOGO@name\relax
          \expandafter\chardef\csname HOLOGOOPT@#1\endcsname=\ltx@zero
        \else
          \expandafter\chardef\csname
          HoLogoOpt@#1@\HOLOGO@name\endcsname=\ltx@zero
        \fi
        \HOLOGO@SetBreakAll{#1}%
      \else
        \@PackageError{hologo}{%
          Unknown value `##1' for boolean option `#1'.\MessageBreak
          Known values are `true' and `false'%
        }\@ehc
      \fi
    \fi
  }%
}
%    \end{macrocode}
%    \end{macro}
%
%    \begin{macro}{\HOLOGO@SetBreakAll}
%    \begin{macrocode}
\def\HOLOGO@SetBreakAll#1{%
  \def\HOLOGO@temp{#1}%
  \ifx\HOLOGO@temp\HOLOGO@break
    \ifx\HOLOGO@name\relax
      \chardef\HOLOGOOPT@hyphenbreak=\HOLOGOOPT@break
      \chardef\HOLOGOOPT@spacebreak=\HOLOGOOPT@break
      \chardef\HOLOGOOPT@discretionarybreak=\HOLOGOOPT@break
    \else
      \expandafter\chardef
         \csname HoLogoOpt@hyphenbreak@\HOLOGO@name\endcsname=%
         \csname HoLogoOpt@break@\HOLOGO@name\endcsname
      \expandafter\chardef
         \csname HoLogoOpt@spacebreak@\HOLOGO@name\endcsname=%
         \csname HoLogoOpt@break@\HOLOGO@name\endcsname
      \expandafter\chardef
         \csname HoLogoOpt@discretionarybreak@\HOLOGO@name
             \endcsname=%
         \csname HoLogoOpt@break@\HOLOGO@name\endcsname
    \fi
  \fi
}
%    \end{macrocode}
%    \end{macro}
%
%    \begin{macro}{\HOLOGO@true}
%    \begin{macrocode}
\def\HOLOGO@true{true}
%    \end{macrocode}
%    \end{macro}
%    \begin{macro}{\HOLOGO@false}
%    \begin{macrocode}
\def\HOLOGO@false{false}
%    \end{macrocode}
%    \end{macro}
%    \begin{macro}{\HOLOGO@break}
%    \begin{macrocode}
\def\HOLOGO@break{break}
%    \end{macrocode}
%    \end{macro}
%
%    \begin{macrocode}
\HOLOGO@DeclareBoolOption{break}
\HOLOGO@DeclareBoolOption{hyphenbreak}
\HOLOGO@DeclareBoolOption{spacebreak}
\HOLOGO@DeclareBoolOption{discretionarybreak}
%    \end{macrocode}
%
%    \begin{macrocode}
\kv@define@key{HoLogo}{variant}{%
  \ifx\HOLOGO@name\relax
    \@PackageError{hologo}{%
      Option `variant' is not available in \string\hologoSetup,%
      \MessageBreak
      Use \string\hologoLogoSetup\space instead%
    }\@ehc
  \else
    \edef\HOLOGO@temp{#1}%
    \ifx\HOLOGO@temp\ltx@empty
      \expandafter
      \let\csname HoLogoOpt@variant@\HOLOGO@name\endcsname\@undefined
    \else
      \ltx@IfUndefined{HoLogo@\HOLOGO@name @\HOLOGO@temp}{%
        \@PackageError{hologo}{%
          Unknown variant `\HOLOGO@temp' of logo `\HOLOGO@name'%
        }\@ehc
      }{%
        \expandafter
        \let\csname HoLogoOpt@variant@\HOLOGO@name\endcsname
            \HOLOGO@temp
      }%
    \fi
  \fi
}
%    \end{macrocode}
%
%    \begin{macro}{\HOLOGO@Variant}
%    \begin{macrocode}
\def\HOLOGO@Variant#1{%
  #1%
  \ltx@ifundefined{HoLogoOpt@variant@#1}{%
  }{%
    @\csname HoLogoOpt@variant@#1\endcsname
  }%
}
%    \end{macrocode}
%    \end{macro}
%
% \subsection{Break/no-break support}
%
%    \begin{macro}{\HOLOGO@space}
%    \begin{macrocode}
\def\HOLOGO@space{%
  \ltx@ifundefined{HoLogoOpt@spacebreak@\HOLOGO@name}{%
    \ltx@ifundefined{HoLogoOpt@break@\HOLOGO@name}{%
      \chardef\HOLOGO@temp=\HOLOGOOPT@spacebreak
    }{%
      \chardef\HOLOGO@temp=%
        \csname HoLogoOpt@break@\HOLOGO@name\endcsname
    }%
  }{%
    \chardef\HOLOGO@temp=%
      \csname HoLogoOpt@spacebreak@\HOLOGO@name\endcsname
  }%
  \ifcase\HOLOGO@temp
    \penalty10000 %
  \fi
  \ltx@space
}
%    \end{macrocode}
%    \end{macro}
%
%    \begin{macro}{\HOLOGO@hyphen}
%    \begin{macrocode}
\def\HOLOGO@hyphen{%
  \ltx@ifundefined{HoLogoOpt@hyphenbreak@\HOLOGO@name}{%
    \ltx@ifundefined{HoLogoOpt@break@\HOLOGO@name}{%
      \chardef\HOLOGO@temp=\HOLOGOOPT@hyphenbreak
    }{%
      \chardef\HOLOGO@temp=%
        \csname HoLogoOpt@break@\HOLOGO@name\endcsname
    }%
  }{%
    \chardef\HOLOGO@temp=%
      \csname HoLogoOpt@hyphenbreak@\HOLOGO@name\endcsname
  }%
  \ifcase\HOLOGO@temp
    \ltx@mbox{-}%
  \else
    -%
  \fi
}
%    \end{macrocode}
%    \end{macro}
%
%    \begin{macro}{\HOLOGO@discretionary}
%    \begin{macrocode}
\def\HOLOGO@discretionary{%
  \ltx@ifundefined{HoLogoOpt@discretionarybreak@\HOLOGO@name}{%
    \ltx@ifundefined{HoLogoOpt@break@\HOLOGO@name}{%
      \chardef\HOLOGO@temp=\HOLOGOOPT@discretionarybreak
    }{%
      \chardef\HOLOGO@temp=%
        \csname HoLogoOpt@break@\HOLOGO@name\endcsname
    }%
  }{%
    \chardef\HOLOGO@temp=%
      \csname HoLogoOpt@discretionarybreak@\HOLOGO@name\endcsname
  }%
  \ifcase\HOLOGO@temp
  \else
    \-%
  \fi
}
%    \end{macrocode}
%    \end{macro}
%
%    \begin{macro}{\HOLOGO@mbox}
%    \begin{macrocode}
\def\HOLOGO@mbox#1{%
  \ltx@ifundefined{HoLogoOpt@break@\HOLOGO@name}{%
    \chardef\HOLOGO@temp=\HOLOGOOPT@hyphenbreak
  }{%
    \chardef\HOLOGO@temp=%
      \csname HoLogoOpt@break@\HOLOGO@name\endcsname
  }%
  \ifcase\HOLOGO@temp
    \ltx@mbox{#1}%
  \else
    #1%
  \fi
}
%    \end{macrocode}
%    \end{macro}
%
% \subsection{Font support}
%
%    \begin{macro}{\HoLogoFont@font}
%    \begin{tabular}{@{}ll@{}}
%    |#1|:& logo name\\
%    |#2|:& font short name\\
%    |#3|:& text
%    \end{tabular}
%    \begin{macrocode}
\def\HoLogoFont@font#1#2#3{%
  \begingroup
    \ltx@IfUndefined{HoLogoFont@logo@#1.#2}{%
      \ltx@IfUndefined{HoLogoFont@font@#2}{%
        \@PackageWarning{hologo}{%
          Missing font `#2' for logo `#1'%
        }%
        #3%
      }{%
        \csname HoLogoFont@font@#2\endcsname{#3}%
      }%
    }{%
      \csname HoLogoFont@logo@#1.#2\endcsname{#3}%
    }%
  \endgroup
}
%    \end{macrocode}
%    \end{macro}
%
%    \begin{macro}{\HoLogoFont@Def}
%    \begin{macrocode}
\def\HoLogoFont@Def#1{%
  \expandafter\def\csname HoLogoFont@font@#1\endcsname
}
%    \end{macrocode}
%    \end{macro}
%    \begin{macro}{\HoLogoFont@LogoDef}
%    \begin{macrocode}
\def\HoLogoFont@LogoDef#1#2{%
  \expandafter\def\csname HoLogoFont@logo@#1.#2\endcsname
}
%    \end{macrocode}
%    \end{macro}
%
% \subsubsection{Font defaults}
%
%    \begin{macro}{\HoLogoFont@font@general}
%    \begin{macrocode}
\HoLogoFont@Def{general}{}%
%    \end{macrocode}
%    \end{macro}
%
%    \begin{macro}{\HoLogoFont@font@rm}
%    \begin{macrocode}
\ltx@IfUndefined{rmfamily}{%
  \ltx@IfUndefined{rm}{%
  }{%
    \HoLogoFont@Def{rm}{\rm}%
  }%
}{%
  \HoLogoFont@Def{rm}{\rmfamily}%
}
%    \end{macrocode}
%    \end{macro}
%
%    \begin{macro}{\HoLogoFont@font@sf}
%    \begin{macrocode}
\ltx@IfUndefined{sffamily}{%
  \ltx@IfUndefined{sf}{%
  }{%
    \HoLogoFont@Def{sf}{\sf}%
  }%
}{%
  \HoLogoFont@Def{sf}{\sffamily}%
}
%    \end{macrocode}
%    \end{macro}
%
%    \begin{macro}{\HoLogoFont@font@bibsf}
%    In case of \hologo{plainTeX} the original small caps
%    variant is used as default. In \hologo{LaTeX}
%    the definition of package \xpackage{dtklogos} \cite{dtklogos}
%    is used.
%\begin{quote}
%\begin{verbatim}
%\DeclareRobustCommand{\BibTeX}{%
%  B%
%  \kern-.05em%
%  \hbox{%
%    $\m@th$% %% force math size calculations
%    \csname S@\f@size\endcsname
%    \fontsize\sf@size\z@
%    \math@fontsfalse
%    \selectfont
%    I%
%    \kern-.025em%
%    B
%  }%
%  \kern-.08em%
%  \-%
%  \TeX
%}
%\end{verbatim}
%\end{quote}
%    \begin{macrocode}
\ltx@IfUndefined{selectfont}{%
  \ltx@IfUndefined{tensc}{%
    \font\tensc=cmcsc10\relax
  }{}%
  \HoLogoFont@Def{bibsf}{\tensc}%
}{%
  \HoLogoFont@Def{bibsf}{%
    $\mathsurround=0pt$%
    \csname S@\f@size\endcsname
    \fontsize\sf@size{0pt}%
    \math@fontsfalse
    \selectfont
  }%
}
%    \end{macrocode}
%    \end{macro}
%
%    \begin{macro}{\HoLogoFont@font@sc}
%    \begin{macrocode}
\ltx@IfUndefined{scshape}{%
  \ltx@IfUndefined{tensc}{%
    \font\tensc=cmcsc10\relax
  }{}%
  \HoLogoFont@Def{sc}{\tensc}%
}{%
  \HoLogoFont@Def{sc}{\scshape}%
}
%    \end{macrocode}
%    \end{macro}
%
%    \begin{macro}{\HoLogoFont@font@sy}
%    \begin{macrocode}
\ltx@IfUndefined{usefont}{%
  \ltx@IfUndefined{tensy}{%
  }{%
    \HoLogoFont@Def{sy}{\tensy}%
  }%
}{%
  \HoLogoFont@Def{sy}{%
    \usefont{OMS}{cmsy}{m}{n}%
  }%
}
%    \end{macrocode}
%    \end{macro}
%
%    \begin{macro}{\HoLogoFont@font@logo}
%    \begin{macrocode}
\begingroup
  \def\x{LaTeX2e}%
\expandafter\endgroup
\ifx\fmtname\x
  \ltx@IfUndefined{logofamily}{%
    \DeclareRobustCommand\logofamily{%
      \not@math@alphabet\logofamily\relax
      \fontencoding{U}%
      \fontfamily{logo}%
      \selectfont
    }%
  }{}%
  \ltx@IfUndefined{logofamily}{%
  }{%
    \HoLogoFont@Def{logo}{\logofamily}%
  }%
\else
  \ltx@IfUndefined{tenlogo}{%
    \font\tenlogo=logo10\relax
  }{}%
  \HoLogoFont@Def{logo}{\tenlogo}%
\fi
%    \end{macrocode}
%    \end{macro}
%
% \subsubsection{Font setup}
%
%    \begin{macro}{\hologoFontSetup}
%    \begin{macrocode}
\def\hologoFontSetup{%
  \let\HOLOGO@name\relax
  \HOLOGO@FontSetup
}
%    \end{macrocode}
%    \end{macro}
%
%    \begin{macro}{\hologoLogoFontSetup}
%    \begin{macrocode}
\def\hologoLogoFontSetup#1{%
  \edef\HOLOGO@name{#1}%
  \ltx@IfUndefined{HoLogo@\HOLOGO@name}{%
    \@PackageError{hologo}{%
      Unknown logo `\HOLOGO@name'%
    }\@ehc
    \ltx@gobble
  }{%
    \HOLOGO@FontSetup
  }%
}
%    \end{macrocode}
%    \end{macro}
%
%    \begin{macro}{\HOLOGO@FontSetup}
%    \begin{macrocode}
\def\HOLOGO@FontSetup{%
  \kvsetkeys{HoLogoFont}%
}
%    \end{macrocode}
%    \end{macro}
%
%    \begin{macrocode}
\def\HOLOGO@temp#1{%
  \kv@define@key{HoLogoFont}{#1}{%
    \ifx\HOLOGO@name\relax
      \HoLogoFont@Def{#1}{##1}%
    \else
      \HoLogoFont@LogoDef\HOLOGO@name{#1}{##1}%
    \fi
  }%
}
\HOLOGO@temp{general}
\HOLOGO@temp{sf}
%    \end{macrocode}
%
% \subsection{Generic logo commands}
%
%    \begin{macrocode}
\HOLOGO@IfExists\hologo{%
  \@PackageError{hologo}{%
    \string\hologo\ltx@space is already defined.\MessageBreak
    Package loading is aborted%
  }\@ehc
  \HOLOGO@AtEnd
}%
\HOLOGO@IfExists\hologoRobust{%
  \@PackageError{hologo}{%
    \string\hologoRobust\ltx@space is already defined.\MessageBreak
    Package loading is aborted%
  }\@ehc
  \HOLOGO@AtEnd
}%
%    \end{macrocode}
%
% \subsubsection{\cs{hologo} and friends}
%
%    \begin{macrocode}
\ifluatex
  \expandafter\ltx@firstofone
\else
  \expandafter\ltx@gobble
\fi
{%
  \ltx@IfUndefined{ifincsname}{%
    \ifnum\luatexversion<36 %
      \expandafter\ltx@gobble
    \else
      \expandafter\ltx@firstofone
    \fi
    {%
      \begingroup
        \ifcase0%
            \directlua{%
              if tex.enableprimitives then %
                tex.enableprimitives('HOLOGO@', {'ifincsname'})%
              else %
                tex.print('1')%
              end%
            }%
            \ifx\HOLOGO@ifincsname\@undefined 1\fi%
            \relax
          \expandafter\ltx@firstofone
        \else
          \endgroup
          \expandafter\ltx@gobble
        \fi
        {%
          \global\let\ifincsname\HOLOGO@ifincsname
        }%
      \HOLOGO@temp
    }%
  }{}%
}
%    \end{macrocode}
%    \begin{macrocode}
\ltx@IfUndefined{ifincsname}{%
  \catcode`$=14 %
}{%
  \catcode`$=9 %
}
%    \end{macrocode}
%
%    \begin{macro}{\hologo}
%    \begin{macrocode}
\def\hologo#1{%
$ \ifincsname
$   \ltx@ifundefined{HoLogoCs@\HOLOGO@Variant{#1}}{%
$     #1%
$   }{%
$     \csname HoLogoCs@\HOLOGO@Variant{#1}\endcsname\ltx@firstoftwo
$   }%
$ \else
    \HOLOGO@IfExists\texorpdfstring\texorpdfstring\ltx@firstoftwo
    {%
      \hologoRobust{#1}%
    }{%
      \ltx@ifundefined{HoLogoBkm@\HOLOGO@Variant{#1}}{%
        \ltx@ifundefined{HoLogo@#1}{?#1?}{#1}%
      }{%
        \csname HoLogoBkm@\HOLOGO@Variant{#1}\endcsname
        \ltx@firstoftwo
      }%
    }%
$ \fi
}
%    \end{macrocode}
%    \end{macro}
%    \begin{macro}{\Hologo}
%    \begin{macrocode}
\def\Hologo#1{%
$ \ifincsname
$   \ltx@ifundefined{HoLogoCs@\HOLOGO@Variant{#1}}{%
$     #1%
$   }{%
$     \csname HoLogoCs@\HOLOGO@Variant{#1}\endcsname\ltx@secondoftwo
$   }%
$ \else
    \HOLOGO@IfExists\texorpdfstring\texorpdfstring\ltx@firstoftwo
    {%
      \HologoRobust{#1}%
    }{%
      \ltx@ifundefined{HoLogoBkm@\HOLOGO@Variant{#1}}{%
        \ltx@ifundefined{HoLogo@#1}{?#1?}{#1}%
      }{%
        \csname HoLogoBkm@\HOLOGO@Variant{#1}\endcsname
        \ltx@secondoftwo
      }%
    }%
$ \fi
}
%    \end{macrocode}
%    \end{macro}
%
%    \begin{macro}{\hologoVariant}
%    \begin{macrocode}
\def\hologoVariant#1#2{%
  \ifx\relax#2\relax
    \hologo{#1}%
  \else
$   \ifincsname
$     \ltx@ifundefined{HoLogoCs@#1@#2}{%
$       #1%
$     }{%
$       \csname HoLogoCs@#1@#2\endcsname\ltx@firstoftwo
$     }%
$   \else
      \HOLOGO@IfExists\texorpdfstring\texorpdfstring\ltx@firstoftwo
      {%
        \hologoVariantRobust{#1}{#2}%
      }{%
        \ltx@ifundefined{HoLogoBkm@#1@#2}{%
          \ltx@ifundefined{HoLogo@#1}{?#1?}{#1}%
        }{%
          \csname HoLogoBkm@#1@#2\endcsname
          \ltx@firstoftwo
        }%
      }%
$   \fi
  \fi
}
%    \end{macrocode}
%    \end{macro}
%    \begin{macro}{\HologoVariant}
%    \begin{macrocode}
\def\HologoVariant#1#2{%
  \ifx\relax#2\relax
    \Hologo{#1}%
  \else
$   \ifincsname
$     \ltx@ifundefined{HoLogoCs@#1@#2}{%
$       #1%
$     }{%
$       \csname HoLogoCs@#1@#2\endcsname\ltx@secondoftwo
$     }%
$   \else
      \HOLOGO@IfExists\texorpdfstring\texorpdfstring\ltx@firstoftwo
      {%
        \HologoVariantRobust{#1}{#2}%
      }{%
        \ltx@ifundefined{HoLogoBkm@#1@#2}{%
          \ltx@ifundefined{HoLogo@#1}{?#1?}{#1}%
        }{%
          \csname HoLogoBkm@#1@#2\endcsname
          \ltx@secondoftwo
        }%
      }%
$   \fi
  \fi
}
%    \end{macrocode}
%    \end{macro}
%
%    \begin{macrocode}
\catcode`\$=3 %
%    \end{macrocode}
%
% \subsubsection{\cs{hologoRobust} and friends}
%
%    \begin{macro}{\hologoRobust}
%    \begin{macrocode}
\ltx@IfUndefined{protected}{%
  \ltx@IfUndefined{DeclareRobustCommand}{%
    \def\hologoRobust#1%
  }{%
    \DeclareRobustCommand*\hologoRobust[1]%
  }%
}{%
  \protected\def\hologoRobust#1%
}%
{%
  \edef\HOLOGO@name{#1}%
  \ltx@IfUndefined{HoLogo@\HOLOGO@Variant\HOLOGO@name}{%
    \@PackageError{hologo}{%
      Unknown logo `\HOLOGO@name'%
    }\@ehc
    ?\HOLOGO@name?%
  }{%
    \ltx@IfUndefined{ver@tex4ht.sty}{%
      \HoLogoFont@font\HOLOGO@name{general}{%
        \csname HoLogo@\HOLOGO@Variant\HOLOGO@name\endcsname
        \ltx@firstoftwo
      }%
    }{%
      \ltx@IfUndefined{HoLogoHtml@\HOLOGO@Variant\HOLOGO@name}{%
        \HOLOGO@name
      }{%
        \csname HoLogoHtml@\HOLOGO@Variant\HOLOGO@name\endcsname
        \ltx@firstoftwo
      }%
    }%
  }%
}
%    \end{macrocode}
%    \end{macro}
%    \begin{macro}{\HologoRobust}
%    \begin{macrocode}
\ltx@IfUndefined{protected}{%
  \ltx@IfUndefined{DeclareRobustCommand}{%
    \def\HologoRobust#1%
  }{%
    \DeclareRobustCommand*\HologoRobust[1]%
  }%
}{%
  \protected\def\HologoRobust#1%
}%
{%
  \edef\HOLOGO@name{#1}%
  \ltx@IfUndefined{HoLogo@\HOLOGO@Variant\HOLOGO@name}{%
    \@PackageError{hologo}{%
      Unknown logo `\HOLOGO@name'%
    }\@ehc
    ?\HOLOGO@name?%
  }{%
    \ltx@IfUndefined{ver@tex4ht.sty}{%
      \HoLogoFont@font\HOLOGO@name{general}{%
        \csname HoLogo@\HOLOGO@Variant\HOLOGO@name\endcsname
        \ltx@secondoftwo
      }%
    }{%
      \ltx@IfUndefined{HoLogoHtml@\HOLOGO@Variant\HOLOGO@name}{%
        \expandafter\HOLOGO@Uppercase\HOLOGO@name
      }{%
        \csname HoLogoHtml@\HOLOGO@Variant\HOLOGO@name\endcsname
        \ltx@secondoftwo
      }%
    }%
  }%
}
%    \end{macrocode}
%    \end{macro}
%    \begin{macro}{\hologoVariantRobust}
%    \begin{macrocode}
\ltx@IfUndefined{protected}{%
  \ltx@IfUndefined{DeclareRobustCommand}{%
    \def\hologoVariantRobust#1#2%
  }{%
    \DeclareRobustCommand*\hologoVariantRobust[2]%
  }%
}{%
  \protected\def\hologoVariantRobust#1#2%
}%
{%
  \begingroup
    \hologoLogoSetup{#1}{variant={#2}}%
    \hologoRobust{#1}%
  \endgroup
}
%    \end{macrocode}
%    \end{macro}
%    \begin{macro}{\HologoVariantRobust}
%    \begin{macrocode}
\ltx@IfUndefined{protected}{%
  \ltx@IfUndefined{DeclareRobustCommand}{%
    \def\HologoVariantRobust#1#2%
  }{%
    \DeclareRobustCommand*\HologoVariantRobust[2]%
  }%
}{%
  \protected\def\HologoVariantRobust#1#2%
}%
{%
  \begingroup
    \hologoLogoSetup{#1}{variant={#2}}%
    \HologoRobust{#1}%
  \endgroup
}
%    \end{macrocode}
%    \end{macro}
%
%    \begin{macro}{\hologorobust}
%    Macro \cs{hologorobust} is only defined for compatibility.
%    Its use is deprecated.
%    \begin{macrocode}
\def\hologorobust{\hologoRobust}
%    \end{macrocode}
%    \end{macro}
%
% \subsection{Helpers}
%
%    \begin{macro}{\HOLOGO@Uppercase}
%    Macro \cs{HOLOGO@Uppercase} is restricted to \cs{uppercase},
%    because \hologo{plainTeX} or \hologo{iniTeX} do not provide
%    \cs{MakeUppercase}.
%    \begin{macrocode}
\def\HOLOGO@Uppercase#1{\uppercase{#1}}
%    \end{macrocode}
%    \end{macro}
%
%    \begin{macro}{\HOLOGO@PdfdocUnicode}
%    \begin{macrocode}
\def\HOLOGO@PdfdocUnicode{%
  \ifx\ifHy@unicode\iftrue
    \expandafter\ltx@secondoftwo
  \else
    \expandafter\ltx@firstoftwo
  \fi
}
%    \end{macrocode}
%    \end{macro}
%
%    \begin{macro}{\HOLOGO@Math}
%    \begin{macrocode}
\def\HOLOGO@MathSetup{%
  \mathsurround0pt\relax
  \HOLOGO@IfExists\f@series{%
    \if b\expandafter\ltx@car\f@series x\@nil
      \csname boldmath\endcsname
   \fi
  }{}%
}
%    \end{macrocode}
%    \end{macro}
%
%    \begin{macro}{\HOLOGO@TempDimen}
%    \begin{macrocode}
\dimendef\HOLOGO@TempDimen=\ltx@zero
%    \end{macrocode}
%    \end{macro}
%    \begin{macro}{\HOLOGO@NegativeKerning}
%    \begin{macrocode}
\def\HOLOGO@NegativeKerning#1{%
  \begingroup
    \HOLOGO@TempDimen=0pt\relax
    \comma@parse@normalized{#1}{%
      \ifdim\HOLOGO@TempDimen=0pt %
        \expandafter\HOLOGO@@NegativeKerning\comma@entry
      \fi
      \ltx@gobble
    }%
    \ifdim\HOLOGO@TempDimen<0pt %
      \kern\HOLOGO@TempDimen
    \fi
  \endgroup
}
%    \end{macrocode}
%    \end{macro}
%    \begin{macro}{\HOLOGO@@NegativeKerning}
%    \begin{macrocode}
\def\HOLOGO@@NegativeKerning#1#2{%
  \setbox\ltx@zero\hbox{#1#2}%
  \HOLOGO@TempDimen=\wd\ltx@zero
  \setbox\ltx@zero\hbox{#1\kern0pt#2}%
  \advance\HOLOGO@TempDimen by -\wd\ltx@zero
}
%    \end{macrocode}
%    \end{macro}
%
%    \begin{macro}{\HOLOGO@SpaceFactor}
%    \begin{macrocode}
\def\HOLOGO@SpaceFactor{%
  \spacefactor1000 %
}
%    \end{macrocode}
%    \end{macro}
%
%    \begin{macro}{\HOLOGO@Span}
%    \begin{macrocode}
\def\HOLOGO@Span#1#2{%
  \HCode{<span class="HoLogo-#1">}%
  #2%
  \HCode{</span>}%
}
%    \end{macrocode}
%    \end{macro}
%
% \subsubsection{Text subscript}
%
%    \begin{macro}{\HOLOGO@SubScript}%
%    \begin{macrocode}
\def\HOLOGO@SubScript#1{%
  \ltx@IfUndefined{textsubscript}{%
    \ltx@IfUndefined{text}{%
      \ltx@mbox{%
        \mathsurround=0pt\relax
        $%
          _{%
            \ltx@IfUndefined{sf@size}{%
              \mathrm{#1}%
            }{%
              \mbox{%
                \fontsize\sf@size{0pt}\selectfont
                #1%
              }%
            }%
          }%
        $%
      }%
    }{%
      \ltx@mbox{%
        \mathsurround=0pt\relax
        $_{\text{#1}}$%
      }%
    }%
  }{%
    \textsubscript{#1}%
  }%
}
%    \end{macrocode}
%    \end{macro}
%
% \subsection{\hologo{TeX} and friends}
%
% \subsubsection{\hologo{TeX}}
%
%    \begin{macro}{\HoLogo@TeX}
%    Source: \hologo{LaTeX} kernel.
%    \begin{macrocode}
\def\HoLogo@TeX#1{%
  T\kern-.1667em\lower.5ex\hbox{E}\kern-.125emX\HOLOGO@SpaceFactor
}
%    \end{macrocode}
%    \end{macro}
%    \begin{macro}{\HoLogoHtml@TeX}
%    \begin{macrocode}
\def\HoLogoHtml@TeX#1{%
  \HoLogoCss@TeX
  \HOLOGO@Span{TeX}{%
    T%
    \HOLOGO@Span{e}{%
      E%
    }%
    X%
  }%
}
%    \end{macrocode}
%    \end{macro}
%    \begin{macro}{\HoLogoCss@TeX}
%    \begin{macrocode}
\def\HoLogoCss@TeX{%
  \Css{%
    span.HoLogo-TeX span.HoLogo-e{%
      position:relative;%
      top:.5ex;%
      margin-left:-.1667em;%
      margin-right:-.125em;%
    }%
  }%
  \Css{%
    a span.HoLogo-TeX span.HoLogo-e{%
      text-decoration:none;%
    }%
  }%
  \global\let\HoLogoCss@TeX\relax
}
%    \end{macrocode}
%    \end{macro}
%
% \subsubsection{\hologo{plainTeX}}
%
%    \begin{macro}{\HoLogo@plainTeX@space}
%    Source: ``The \hologo{TeX}book''
%    \begin{macrocode}
\def\HoLogo@plainTeX@space#1{%
  \HOLOGO@mbox{#1{p}{P}lain}\HOLOGO@space\hologo{TeX}%
}
%    \end{macrocode}
%    \end{macro}
%    \begin{macro}{\HoLogoCs@plainTeX@space}
%    \begin{macrocode}
\def\HoLogoCs@plainTeX@space#1{#1{p}{P}lain TeX}%
%    \end{macrocode}
%    \end{macro}
%    \begin{macro}{\HoLogoBkm@plainTeX@space}
%    \begin{macrocode}
\def\HoLogoBkm@plainTeX@space#1{%
  #1{p}{P}lain \hologo{TeX}%
}
%    \end{macrocode}
%    \end{macro}
%    \begin{macro}{\HoLogoHtml@plainTeX@space}
%    \begin{macrocode}
\def\HoLogoHtml@plainTeX@space#1{%
  #1{p}{P}lain \hologo{TeX}%
}
%    \end{macrocode}
%    \end{macro}
%
%    \begin{macro}{\HoLogo@plainTeX@hyphen}
%    \begin{macrocode}
\def\HoLogo@plainTeX@hyphen#1{%
  \HOLOGO@mbox{#1{p}{P}lain}\HOLOGO@hyphen\hologo{TeX}%
}
%    \end{macrocode}
%    \end{macro}
%    \begin{macro}{\HoLogoCs@plainTeX@hyphen}
%    \begin{macrocode}
\def\HoLogoCs@plainTeX@hyphen#1{#1{p}{P}lain-TeX}
%    \end{macrocode}
%    \end{macro}
%    \begin{macro}{\HoLogoBkm@plainTeX@hyphen}
%    \begin{macrocode}
\def\HoLogoBkm@plainTeX@hyphen#1{%
  #1{p}{P}lain-\hologo{TeX}%
}
%    \end{macrocode}
%    \end{macro}
%    \begin{macro}{\HoLogoHtml@plainTeX@hyphen}
%    \begin{macrocode}
\def\HoLogoHtml@plainTeX@hyphen#1{%
  #1{p}{P}lain-\hologo{TeX}%
}
%    \end{macrocode}
%    \end{macro}
%
%    \begin{macro}{\HoLogo@plainTeX@runtogether}
%    \begin{macrocode}
\def\HoLogo@plainTeX@runtogether#1{%
  \HOLOGO@mbox{#1{p}{P}lain\hologo{TeX}}%
}
%    \end{macrocode}
%    \end{macro}
%    \begin{macro}{\HoLogoCs@plainTeX@runtogether}
%    \begin{macrocode}
\def\HoLogoCs@plainTeX@runtogether#1{#1{p}{P}lainTeX}
%    \end{macrocode}
%    \end{macro}
%    \begin{macro}{\HoLogoBkm@plainTeX@runtogether}
%    \begin{macrocode}
\def\HoLogoBkm@plainTeX@runtogether#1{%
  #1{p}{P}lain\hologo{TeX}%
}
%    \end{macrocode}
%    \end{macro}
%    \begin{macro}{\HoLogoHtml@plainTeX@runtogether}
%    \begin{macrocode}
\def\HoLogoHtml@plainTeX@runtogether#1{%
  #1{p}{P}lain\hologo{TeX}%
}
%    \end{macrocode}
%    \end{macro}
%
%    \begin{macro}{\HoLogo@plainTeX}
%    \begin{macrocode}
\def\HoLogo@plainTeX{\HoLogo@plainTeX@space}
%    \end{macrocode}
%    \end{macro}
%    \begin{macro}{\HoLogoCs@plainTeX}
%    \begin{macrocode}
\def\HoLogoCs@plainTeX{\HoLogoCs@plainTeX@space}
%    \end{macrocode}
%    \end{macro}
%    \begin{macro}{\HoLogoBkm@plainTeX}
%    \begin{macrocode}
\def\HoLogoBkm@plainTeX{\HoLogoBkm@plainTeX@space}
%    \end{macrocode}
%    \end{macro}
%    \begin{macro}{\HoLogoHtml@plainTeX}
%    \begin{macrocode}
\def\HoLogoHtml@plainTeX{\HoLogoHtml@plainTeX@space}
%    \end{macrocode}
%    \end{macro}
%
% \subsubsection{\hologo{LaTeX}}
%
%    Source: \hologo{LaTeX} kernel.
%\begin{quote}
%\begin{verbatim}
%\DeclareRobustCommand{\LaTeX}{%
%  L%
%  \kern-.36em%
%  {%
%    \sbox\z@ T%
%    \vbox to\ht\z@{%
%      \hbox{%
%        \check@mathfonts
%        \fontsize\sf@size\z@
%        \math@fontsfalse
%        \selectfont
%        A%
%      }%
%      \vss
%    }%
%  }%
%  \kern-.15em%
%  \TeX
%}
%\end{verbatim}
%\end{quote}
%
%    \begin{macro}{\HoLogo@La}
%    \begin{macrocode}
\def\HoLogo@La#1{%
  L%
  \kern-.36em%
  \begingroup
    \setbox\ltx@zero\hbox{T}%
    \vbox to\ht\ltx@zero{%
      \hbox{%
        \ltx@ifundefined{check@mathfonts}{%
          \csname sevenrm\endcsname
        }{%
          \check@mathfonts
          \fontsize\sf@size{0pt}%
          \math@fontsfalse\selectfont
        }%
        A%
      }%
      \vss
    }%
  \endgroup
}
%    \end{macrocode}
%    \end{macro}
%
%    \begin{macro}{\HoLogo@LaTeX}
%    Source: \hologo{LaTeX} kernel.
%    \begin{macrocode}
\def\HoLogo@LaTeX#1{%
  \hologo{La}%
  \kern-.15em%
  \hologo{TeX}%
}
%    \end{macrocode}
%    \end{macro}
%    \begin{macro}{\HoLogoHtml@LaTeX}
%    \begin{macrocode}
\def\HoLogoHtml@LaTeX#1{%
  \HoLogoCss@LaTeX
  \HOLOGO@Span{LaTeX}{%
    L%
    \HOLOGO@Span{a}{%
      A%
    }%
    \hologo{TeX}%
  }%
}
%    \end{macrocode}
%    \end{macro}
%    \begin{macro}{\HoLogoCss@LaTeX}
%    \begin{macrocode}
\def\HoLogoCss@LaTeX{%
  \Css{%
    span.HoLogo-LaTeX span.HoLogo-a{%
      position:relative;%
      top:-.5ex;%
      margin-left:-.36em;%
      margin-right:-.15em;%
      font-size:85\%;%
    }%
  }%
  \global\let\HoLogoCss@LaTeX\relax
}
%    \end{macrocode}
%    \end{macro}
%
% \subsubsection{\hologo{(La)TeX}}
%
%    \begin{macro}{\HoLogo@LaTeXTeX}
%    The kerning around the parentheses is taken
%    from package \xpackage{dtklogos} \cite{dtklogos}.
%\begin{quote}
%\begin{verbatim}
%\DeclareRobustCommand{\LaTeXTeX}{%
%  (%
%  \kern-.15em%
%  L%
%  \kern-.36em%
%  {%
%    \sbox\z@ T%
%    \vbox to\ht0{%
%      \hbox{%
%        $\m@th$%
%        \csname S@\f@size\endcsname
%        \fontsize\sf@size\z@
%        \math@fontsfalse
%        \selectfont
%        A%
%      }%
%      \vss
%    }%
%  }%
%  \kern-.2em%
%  )%
%  \kern-.15em%
%  \TeX
%}
%\end{verbatim}
%\end{quote}
%    \begin{macrocode}
\def\HoLogo@LaTeXTeX#1{%
  (%
  \kern-.15em%
  \hologo{La}%
  \kern-.2em%
  )%
  \kern-.15em%
  \hologo{TeX}%
}
%    \end{macrocode}
%    \end{macro}
%    \begin{macro}{\HoLogoBkm@LaTeXTeX}
%    \begin{macrocode}
\def\HoLogoBkm@LaTeXTeX#1{(La)TeX}
%    \end{macrocode}
%    \end{macro}
%
%    \begin{macro}{\HoLogo@(La)TeX}
%    \begin{macrocode}
\expandafter
\let\csname HoLogo@(La)TeX\endcsname\HoLogo@LaTeXTeX
%    \end{macrocode}
%    \end{macro}
%    \begin{macro}{\HoLogoBkm@(La)TeX}
%    \begin{macrocode}
\expandafter
\let\csname HoLogoBkm@(La)TeX\endcsname\HoLogoBkm@LaTeXTeX
%    \end{macrocode}
%    \end{macro}
%    \begin{macro}{\HoLogoHtml@LaTeXTeX}
%    \begin{macrocode}
\def\HoLogoHtml@LaTeXTeX#1{%
  \HoLogoCss@LaTeXTeX
  \HOLOGO@Span{LaTeXTeX}{%
    (%
    \HOLOGO@Span{L}{L}%
    \HOLOGO@Span{a}{A}%
    \HOLOGO@Span{ParenRight}{)}%
    \hologo{TeX}%
  }%
}
%    \end{macrocode}
%    \end{macro}
%    \begin{macro}{\HoLogoHtml@(La)TeX}
%    Kerning after opening parentheses and before closing parentheses
%    is $-0.1$\,em. The original values $-0.15$\,em
%    looked too ugly for a serif font.
%    \begin{macrocode}
\expandafter
\let\csname HoLogoHtml@(La)TeX\endcsname\HoLogoHtml@LaTeXTeX
%    \end{macrocode}
%    \end{macro}
%    \begin{macro}{\HoLogoCss@LaTeXTeX}
%    \begin{macrocode}
\def\HoLogoCss@LaTeXTeX{%
  \Css{%
    span.HoLogo-LaTeXTeX span.HoLogo-L{%
      margin-left:-.1em;%
    }%
  }%
  \Css{%
    span.HoLogo-LaTeXTeX span.HoLogo-a{%
      position:relative;%
      top:-.5ex;%
      margin-left:-.36em;%
      margin-right:-.1em;%
      font-size:85\%;%
    }%
  }%
  \Css{%
    span.HoLogo-LaTeXTeX span.HoLogo-ParenRight{%
      margin-right:-.15em;%
    }%
  }%
  \global\let\HoLogoCss@LaTeXTeX\relax
}
%    \end{macrocode}
%    \end{macro}
%
% \subsubsection{\hologo{LaTeXe}}
%
%    \begin{macro}{\HoLogo@LaTeXe}
%    Source: \hologo{LaTeX} kernel
%    \begin{macrocode}
\def\HoLogo@LaTeXe#1{%
  \hologo{LaTeX}%
  \kern.15em%
  \hbox{%
    \HOLOGO@MathSetup
    2%
    $_{\textstyle\varepsilon}$%
  }%
}
%    \end{macrocode}
%    \end{macro}
%
%    \begin{macro}{\HoLogoCs@LaTeXe}
%    \begin{macrocode}
\ifnum64=`\^^^^0040\relax % test for big chars of LuaTeX/XeTeX
  \catcode`\$=9 %
  \catcode`\&=14 %
\else
  \catcode`\$=14 %
  \catcode`\&=9 %
\fi
\def\HoLogoCs@LaTeXe#1{%
  LaTeX2%
$ \string ^^^^0395%
& e%
}%
\catcode`\$=3 %
\catcode`\&=4 %
%    \end{macrocode}
%    \end{macro}
%
%    \begin{macro}{\HoLogoBkm@LaTeXe}
%    \begin{macrocode}
\def\HoLogoBkm@LaTeXe#1{%
  \hologo{LaTeX}%
  2%
  \HOLOGO@PdfdocUnicode{e}{\textepsilon}%
}
%    \end{macrocode}
%    \end{macro}
%
%    \begin{macro}{\HoLogoHtml@LaTeXe}
%    \begin{macrocode}
\def\HoLogoHtml@LaTeXe#1{%
  \HoLogoCss@LaTeXe
  \HOLOGO@Span{LaTeX2e}{%
    \hologo{LaTeX}%
    \HOLOGO@Span{2}{2}%
    \HOLOGO@Span{e}{%
      \HOLOGO@MathSetup
      \ensuremath{\textstyle\varepsilon}%
    }%
  }%
}
%    \end{macrocode}
%    \end{macro}
%    \begin{macro}{\HoLogoCss@LaTeXe}
%    \begin{macrocode}
\def\HoLogoCss@LaTeXe{%
  \Css{%
    span.HoLogo-LaTeX2e span.HoLogo-2{%
      padding-left:.15em;%
    }%
  }%
  \Css{%
    span.HoLogo-LaTeX2e span.HoLogo-e{%
      position:relative;%
      top:.35ex;%
      text-decoration:none;%
    }%
  }%
  \global\let\HoLogoCss@LaTeXe\relax
}
%    \end{macrocode}
%    \end{macro}
%
%    \begin{macro}{\HoLogo@LaTeX2e}
%    \begin{macrocode}
\expandafter
\let\csname HoLogo@LaTeX2e\endcsname\HoLogo@LaTeXe
%    \end{macrocode}
%    \end{macro}
%    \begin{macro}{\HoLogoCs@LaTeX2e}
%    \begin{macrocode}
\expandafter
\let\csname HoLogoCs@LaTeX2e\endcsname\HoLogoCs@LaTeXe
%    \end{macrocode}
%    \end{macro}
%    \begin{macro}{\HoLogoBkm@LaTeX2e}
%    \begin{macrocode}
\expandafter
\let\csname HoLogoBkm@LaTeX2e\endcsname\HoLogoBkm@LaTeXe
%    \end{macrocode}
%    \end{macro}
%    \begin{macro}{\HoLogoHtml@LaTeX2e}
%    \begin{macrocode}
\expandafter
\let\csname HoLogoHtml@LaTeX2e\endcsname\HoLogoHtml@LaTeXe
%    \end{macrocode}
%    \end{macro}
%
% \subsubsection{\hologo{LaTeX3}}
%
%    \begin{macro}{\HoLogo@LaTeX3}
%    Source: \hologo{LaTeX} kernel
%    \begin{macrocode}
\expandafter\def\csname HoLogo@LaTeX3\endcsname#1{%
  \hologo{LaTeX}%
  3%
}
%    \end{macrocode}
%    \end{macro}
%
%    \begin{macro}{\HoLogoBkm@LaTeX3}
%    \begin{macrocode}
\expandafter\def\csname HoLogoBkm@LaTeX3\endcsname#1{%
  \hologo{LaTeX}%
  3%
}
%    \end{macrocode}
%    \end{macro}
%    \begin{macro}{\HoLogoHtml@LaTeX3}
%    \begin{macrocode}
\expandafter
\let\csname HoLogoHtml@LaTeX3\expandafter\endcsname
\csname HoLogo@LaTeX3\endcsname
%    \end{macrocode}
%    \end{macro}
%
% \subsubsection{\hologo{LaTeXML}}
%
%    \begin{macro}{\HoLogo@LaTeXML}
%    \begin{macrocode}
\def\HoLogo@LaTeXML#1{%
  \HOLOGO@mbox{%
    \hologo{La}%
    \kern-.15em%
    T%
    \kern-.1667em%
    \lower.5ex\hbox{E}%
    \kern-.125em%
    \HoLogoFont@font{LaTeXML}{sc}{xml}%
  }%
}
%    \end{macrocode}
%    \end{macro}
%    \begin{macro}{\HoLogoHtml@pdfLaTeX}
%    \begin{macrocode}
\def\HoLogoHtml@LaTeXML#1{%
  \HOLOGO@Span{LaTeXML}{%
    \HoLogoCss@LaTeX
    \HoLogoCss@TeX
    \HOLOGO@Span{LaTeX}{%
      L%
      \HOLOGO@Span{a}{%
        A%
      }%
    }%
    \HOLOGO@Span{TeX}{%
      T%
      \HOLOGO@Span{e}{%
        E%
      }%
    }%
    \HCode{<span style="font-variant: small-caps;">}%
    xml%
    \HCode{</span>}%
  }%
}
%    \end{macrocode}
%    \end{macro}
%
% \subsubsection{\hologo{eTeX}}
%
%    \begin{macro}{\HoLogo@eTeX}
%    Source: package \xpackage{etex}
%    \begin{macrocode}
\def\HoLogo@eTeX#1{%
  \ltx@mbox{%
    \HOLOGO@MathSetup
    $\varepsilon$%
    -%
    \HOLOGO@NegativeKerning{-T,T-,To}%
    \hologo{TeX}%
  }%
}
%    \end{macrocode}
%    \end{macro}
%    \begin{macro}{\HoLogoCs@eTeX}
%    \begin{macrocode}
\ifnum64=`\^^^^0040\relax % test for big chars of LuaTeX/XeTeX
  \catcode`\$=9 %
  \catcode`\&=14 %
\else
  \catcode`\$=14 %
  \catcode`\&=9 %
\fi
\def\HoLogoCs@eTeX#1{%
$ #1{\string ^^^^0395}{\string ^^^^03b5}%
& #1{e}{E}%
  TeX%
}%
\catcode`\$=3 %
\catcode`\&=4 %
%    \end{macrocode}
%    \end{macro}
%    \begin{macro}{\HoLogoBkm@eTeX}
%    \begin{macrocode}
\def\HoLogoBkm@eTeX#1{%
  \HOLOGO@PdfdocUnicode{#1{e}{E}}{\textepsilon}%
  -%
  \hologo{TeX}%
}
%    \end{macrocode}
%    \end{macro}
%    \begin{macro}{\HoLogoHtml@eTeX}
%    \begin{macrocode}
\def\HoLogoHtml@eTeX#1{%
  \ltx@mbox{%
    \HOLOGO@MathSetup
    $\varepsilon$%
    -%
    \hologo{TeX}%
  }%
}
%    \end{macrocode}
%    \end{macro}
%
% \subsubsection{\hologo{iniTeX}}
%
%    \begin{macro}{\HoLogo@iniTeX}
%    \begin{macrocode}
\def\HoLogo@iniTeX#1{%
  \HOLOGO@mbox{%
    #1{i}{I}ni\hologo{TeX}%
  }%
}
%    \end{macrocode}
%    \end{macro}
%    \begin{macro}{\HoLogoCs@iniTeX}
%    \begin{macrocode}
\def\HoLogoCs@iniTeX#1{#1{i}{I}niTeX}
%    \end{macrocode}
%    \end{macro}
%    \begin{macro}{\HoLogoBkm@iniTeX}
%    \begin{macrocode}
\def\HoLogoBkm@iniTeX#1{%
  #1{i}{I}ni\hologo{TeX}%
}
%    \end{macrocode}
%    \end{macro}
%    \begin{macro}{\HoLogoHtml@iniTeX}
%    \begin{macrocode}
\let\HoLogoHtml@iniTeX\HoLogo@iniTeX
%    \end{macrocode}
%    \end{macro}
%
% \subsubsection{\hologo{virTeX}}
%
%    \begin{macro}{\HoLogo@virTeX}
%    \begin{macrocode}
\def\HoLogo@virTeX#1{%
  \HOLOGO@mbox{%
    #1{v}{V}ir\hologo{TeX}%
  }%
}
%    \end{macrocode}
%    \end{macro}
%    \begin{macro}{\HoLogoCs@virTeX}
%    \begin{macrocode}
\def\HoLogoCs@virTeX#1{#1{v}{V}irTeX}
%    \end{macrocode}
%    \end{macro}
%    \begin{macro}{\HoLogoBkm@virTeX}
%    \begin{macrocode}
\def\HoLogoBkm@virTeX#1{%
  #1{v}{V}ir\hologo{TeX}%
}
%    \end{macrocode}
%    \end{macro}
%    \begin{macro}{\HoLogoHtml@virTeX}
%    \begin{macrocode}
\let\HoLogoHtml@virTeX\HoLogo@virTeX
%    \end{macrocode}
%    \end{macro}
%
% \subsubsection{\hologo{SliTeX}}
%
% \paragraph{Definitions of the three variants.}
%
%    \begin{macro}{\HoLogo@SLiTeX@lift}
%    \begin{macrocode}
\def\HoLogo@SLiTeX@lift#1{%
  \HoLogoFont@font{SliTeX}{rm}{%
    S%
    \kern-.06em%
    L%
    \kern-.18em%
    \raise.32ex\hbox{\HoLogoFont@font{SliTeX}{sc}{i}}%
    \HOLOGO@discretionary
    \kern-.06em%
    \hologo{TeX}%
  }%
}
%    \end{macrocode}
%    \end{macro}
%    \begin{macro}{\HoLogoBkm@SLiTeX@lift}
%    \begin{macrocode}
\def\HoLogoBkm@SLiTeX@lift#1{SLiTeX}
%    \end{macrocode}
%    \end{macro}
%    \begin{macro}{\HoLogoHtml@SLiTeX@lift}
%    \begin{macrocode}
\def\HoLogoHtml@SLiTeX@lift#1{%
  \HoLogoCss@SLiTeX@lift
  \HOLOGO@Span{SLiTeX-lift}{%
    \HoLogoFont@font{SliTeX}{rm}{%
      S%
      \HOLOGO@Span{L}{L}%
      \HOLOGO@Span{i}{i}%
      \hologo{TeX}%
    }%
  }%
}
%    \end{macrocode}
%    \end{macro}
%    \begin{macro}{\HoLogoCss@SLiTeX@lift}
%    \begin{macrocode}
\def\HoLogoCss@SLiTeX@lift{%
  \Css{%
    span.HoLogo-SLiTeX-lift span.HoLogo-L{%
      margin-left:-.06em;%
      margin-right:-.18em;%
    }%
  }%
  \Css{%
    span.HoLogo-SLiTeX-lift span.HoLogo-i{%
      position:relative;%
      top:-.32ex;%
      margin-right:-.06em;%
      font-variant:small-caps;%
    }%
  }%
  \global\let\HoLogoCss@SLiTeX@lift\relax
}
%    \end{macrocode}
%    \end{macro}
%
%    \begin{macro}{\HoLogo@SliTeX@simple}
%    \begin{macrocode}
\def\HoLogo@SliTeX@simple#1{%
  \HoLogoFont@font{SliTeX}{rm}{%
    \ltx@mbox{%
      \HoLogoFont@font{SliTeX}{sc}{Sli}%
    }%
    \HOLOGO@discretionary
    \hologo{TeX}%
  }%
}
%    \end{macrocode}
%    \end{macro}
%    \begin{macro}{\HoLogoBkm@SliTeX@simple}
%    \begin{macrocode}
\def\HoLogoBkm@SliTeX@simple#1{SliTeX}
%    \end{macrocode}
%    \end{macro}
%    \begin{macro}{\HoLogoHtml@SliTeX@simple}
%    \begin{macrocode}
\let\HoLogoHtml@SliTeX@simple\HoLogo@SliTeX@simple
%    \end{macrocode}
%    \end{macro}
%
%    \begin{macro}{\HoLogo@SliTeX@narrow}
%    \begin{macrocode}
\def\HoLogo@SliTeX@narrow#1{%
  \HoLogoFont@font{SliTeX}{rm}{%
    \ltx@mbox{%
      S%
      \kern-.06em%
      \HoLogoFont@font{SliTeX}{sc}{%
        l%
        \kern-.035em%
        i%
      }%
    }%
    \HOLOGO@discretionary
    \kern-.06em%
    \hologo{TeX}%
  }%
}
%    \end{macrocode}
%    \end{macro}
%    \begin{macro}{\HoLogoBkm@SliTeX@narrow}
%    \begin{macrocode}
\def\HoLogoBkm@SliTeX@narrow#1{SliTeX}
%    \end{macrocode}
%    \end{macro}
%    \begin{macro}{\HoLogoHtml@SliTeX@narrow}
%    \begin{macrocode}
\def\HoLogoHtml@SliTeX@narrow#1{%
  \HoLogoCss@SliTeX@narrow
  \HOLOGO@Span{SliTeX-narrow}{%
    \HoLogoFont@font{SliTeX}{rm}{%
      S%
        \HOLOGO@Span{l}{l}%
        \HOLOGO@Span{i}{i}%
      \hologo{TeX}%
    }%
  }%
}
%    \end{macrocode}
%    \end{macro}
%    \begin{macro}{\HoLogoCss@SliTeX@narrow}
%    \begin{macrocode}
\def\HoLogoCss@SliTeX@narrow{%
  \Css{%
    span.HoLogo-SliTeX-narrow span.HoLogo-l{%
      margin-left:-.06em;%
      margin-right:-.035em;%
      font-variant:small-caps;%
    }%
  }%
  \Css{%
    span.HoLogo-SliTeX-narrow span.HoLogo-i{%
      margin-right:-.06em;%
      font-variant:small-caps;%
    }%
  }%
  \global\let\HoLogoCss@SliTeX@narrow\relax
}
%    \end{macrocode}
%    \end{macro}
%
% \paragraph{Macro set completion.}
%
%    \begin{macro}{\HoLogo@SLiTeX@simple}
%    \begin{macrocode}
\def\HoLogo@SLiTeX@simple{\HoLogo@SliTeX@simple}
%    \end{macrocode}
%    \end{macro}
%    \begin{macro}{\HoLogoBkm@SLiTeX@simple}
%    \begin{macrocode}
\def\HoLogoBkm@SLiTeX@simple{\HoLogoBkm@SliTeX@simple}
%    \end{macrocode}
%    \end{macro}
%    \begin{macro}{\HoLogoHtml@SLiTeX@simple}
%    \begin{macrocode}
\def\HoLogoHtml@SLiTeX@simple{\HoLogoHtml@SliTeX@simple}
%    \end{macrocode}
%    \end{macro}
%
%    \begin{macro}{\HoLogo@SLiTeX@narrow}
%    \begin{macrocode}
\def\HoLogo@SLiTeX@narrow{\HoLogo@SliTeX@narrow}
%    \end{macrocode}
%    \end{macro}
%    \begin{macro}{\HoLogoBkm@SLiTeX@narrow}
%    \begin{macrocode}
\def\HoLogoBkm@SLiTeX@narrow{\HoLogoBkm@SliTeX@narrow}
%    \end{macrocode}
%    \end{macro}
%    \begin{macro}{\HoLogoHtml@SLiTeX@narrow}
%    \begin{macrocode}
\def\HoLogoHtml@SLiTeX@narrow{\HoLogoHtml@SliTeX@narrow}
%    \end{macrocode}
%    \end{macro}
%
%    \begin{macro}{\HoLogo@SliTeX@lift}
%    \begin{macrocode}
\def\HoLogo@SliTeX@lift{\HoLogo@SLiTeX@lift}
%    \end{macrocode}
%    \end{macro}
%    \begin{macro}{\HoLogoBkm@SliTeX@lift}
%    \begin{macrocode}
\def\HoLogoBkm@SliTeX@lift{\HoLogoBkm@SLiTeX@lift}
%    \end{macrocode}
%    \end{macro}
%    \begin{macro}{\HoLogoHtml@SliTeX@lift}
%    \begin{macrocode}
\def\HoLogoHtml@SliTeX@lift{\HoLogoHtml@SLiTeX@lift}
%    \end{macrocode}
%    \end{macro}
%
% \paragraph{Defaults.}
%
%    \begin{macro}{\HoLogo@SLiTeX}
%    \begin{macrocode}
\def\HoLogo@SLiTeX{\HoLogo@SLiTeX@lift}
%    \end{macrocode}
%    \end{macro}
%    \begin{macro}{\HoLogoBkm@SLiTeX}
%    \begin{macrocode}
\def\HoLogoBkm@SLiTeX{\HoLogoBkm@SLiTeX@lift}
%    \end{macrocode}
%    \end{macro}
%    \begin{macro}{\HoLogoHtml@SLiTeX}
%    \begin{macrocode}
\def\HoLogoHtml@SLiTeX{\HoLogoHtml@SLiTeX@lift}
%    \end{macrocode}
%    \end{macro}
%
%    \begin{macro}{\HoLogo@SliTeX}
%    \begin{macrocode}
\def\HoLogo@SliTeX{\HoLogo@SliTeX@narrow}
%    \end{macrocode}
%    \end{macro}
%    \begin{macro}{\HoLogoBkm@SliTeX}
%    \begin{macrocode}
\def\HoLogoBkm@SliTeX{\HoLogoBkm@SliTeX@narrow}
%    \end{macrocode}
%    \end{macro}
%    \begin{macro}{\HoLogoHtml@SliTeX}
%    \begin{macrocode}
\def\HoLogoHtml@SliTeX{\HoLogoHtml@SliTeX@narrow}
%    \end{macrocode}
%    \end{macro}
%
% \subsubsection{\hologo{LuaTeX}}
%
%    \begin{macro}{\HoLogo@LuaTeX}
%    The kerning is an idea of Hans Hagen, see mailing list
%    `luatex at tug dot org' in March 2010.
%    \begin{macrocode}
\def\HoLogo@LuaTeX#1{%
  \HOLOGO@mbox{%
    Lua%
    \HOLOGO@NegativeKerning{aT,oT,To}%
    \hologo{TeX}%
  }%
}
%    \end{macrocode}
%    \end{macro}
%    \begin{macro}{\HoLogoHtml@LuaTeX}
%    \begin{macrocode}
\let\HoLogoHtml@LuaTeX\HoLogo@LuaTeX
%    \end{macrocode}
%    \end{macro}
%
% \subsubsection{\hologo{LuaLaTeX}}
%
%    \begin{macro}{\HoLogo@LuaLaTeX}
%    \begin{macrocode}
\def\HoLogo@LuaLaTeX#1{%
  \HOLOGO@mbox{%
    Lua%
    \hologo{LaTeX}%
  }%
}
%    \end{macrocode}
%    \end{macro}
%    \begin{macro}{\HoLogoHtml@LuaLaTeX}
%    \begin{macrocode}
\let\HoLogoHtml@LuaLaTeX\HoLogo@LuaLaTeX
%    \end{macrocode}
%    \end{macro}
%
% \subsubsection{\hologo{XeTeX}, \hologo{XeLaTeX}}
%
%    \begin{macro}{\HOLOGO@IfCharExists}
%    \begin{macrocode}
\ifluatex
  \ifnum\luatexversion<36 %
  \else
    \def\HOLOGO@IfCharExists#1{%
      \ifnum
        \directlua{%
           if luaotfload and luaotfload.aux then
             if luaotfload.aux.font_has_glyph(%
                    font.current(), \number#1) then % 	 
	       tex.print("1") % 	 
	     end % 	 
	   elseif font and font.fonts and font.current then %
            local f = font.fonts[font.current()]%
            if f.characters and f.characters[\number#1] then %
              tex.print("1")%
            end %
          end%
        }0=\ltx@zero
        \expandafter\ltx@secondoftwo
      \else
        \expandafter\ltx@firstoftwo
      \fi
    }%
  \fi
\fi
\ltx@IfUndefined{HOLOGO@IfCharExists}{%
  \def\HOLOGO@@IfCharExists#1{%
    \begingroup
      \tracinglostchars=\ltx@zero
      \setbox\ltx@zero=\hbox{%
        \kern7sp\char#1\relax
        \ifnum\lastkern>\ltx@zero
          \expandafter\aftergroup\csname iffalse\endcsname
        \else
          \expandafter\aftergroup\csname iftrue\endcsname
        \fi
      }%
      % \if{true|false} from \aftergroup
      \endgroup
      \expandafter\ltx@firstoftwo
    \else
      \endgroup
      \expandafter\ltx@secondoftwo
    \fi
  }%
  \ifxetex
    \ltx@IfUndefined{XeTeXfonttype}{}{%
      \ltx@IfUndefined{XeTeXcharglyph}{}{%
        \def\HOLOGO@IfCharExists#1{%
          \ifnum\XeTeXfonttype\font>\ltx@zero
            \expandafter\ltx@firstofthree
          \else
            \expandafter\ltx@gobble
          \fi
          {%
            \ifnum\XeTeXcharglyph#1>\ltx@zero
              \expandafter\ltx@firstoftwo
            \else
              \expandafter\ltx@secondoftwo
            \fi
          }%
          \HOLOGO@@IfCharExists{#1}%
        }%
      }%
    }%
  \fi
}{}
\ltx@ifundefined{HOLOGO@IfCharExists}{%
  \ifnum64=`\^^^^0040\relax % test for big chars of LuaTeX/XeTeX
    \let\HOLOGO@IfCharExists\HOLOGO@@IfCharExists
  \else
    \def\HOLOGO@IfCharExists#1{%
      \ifnum#1>255 %
        \expandafter\ltx@fourthoffour
      \fi
      \HOLOGO@@IfCharExists{#1}%
    }%
  \fi
}{}
%    \end{macrocode}
%    \end{macro}
%
%    \begin{macro}{\HoLogo@Xe}
%    Source: package \xpackage{dtklogos}
%    \begin{macrocode}
\def\HoLogo@Xe#1{%
  X%
  \kern-.1em\relax
  \HOLOGO@IfCharExists{"018E}{%
    \lower.5ex\hbox{\char"018E}%
  }{%
    \chardef\HOLOGO@choice=\ltx@zero
    \ifdim\fontdimen\ltx@one\font>0pt %
      \ltx@IfUndefined{rotatebox}{%
        \ltx@IfUndefined{pgftext}{%
          \ltx@IfUndefined{psscalebox}{%
            \ltx@IfUndefined{HOLOGO@ScaleBox@\hologoDriver}{%
            }{%
              \chardef\HOLOGO@choice=4 %
            }%
          }{%
            \chardef\HOLOGO@choice=3 %
          }%
        }{%
          \chardef\HOLOGO@choice=2 %
        }%
      }{%
        \chardef\HOLOGO@choice=1 %
      }%
      \ifcase\HOLOGO@choice
        \HOLOGO@WarningUnsupportedDriver{Xe}%
        e%
      \or % 1: \rotatebox
        \begingroup
          \setbox\ltx@zero\hbox{\rotatebox{180}{E}}%
          \ltx@LocDimenA=\dp\ltx@zero
          \advance\ltx@LocDimenA by -.5ex\relax
          \raise\ltx@LocDimenA\box\ltx@zero
        \endgroup
      \or % 2: \pgftext
        \lower.5ex\hbox{%
          \pgfpicture
            \pgftext[rotate=180]{E}%
          \endpgfpicture
        }%
      \or % 3: \psscalebox
        \begingroup
          \setbox\ltx@zero\hbox{\psscalebox{-1 -1}{E}}%
          \ltx@LocDimenA=\dp\ltx@zero
          \advance\ltx@LocDimenA by -.5ex\relax
          \raise\ltx@LocDimenA\box\ltx@zero
        \endgroup
      \or % 4: \HOLOGO@PointReflectBox
        \lower.5ex\hbox{\HOLOGO@PointReflectBox{E}}%
      \else
        \@PackageError{hologo}{Internal error (choice/it}\@ehc
      \fi
    \else
      \ltx@IfUndefined{reflectbox}{%
        \ltx@IfUndefined{pgftext}{%
          \ltx@IfUndefined{psscalebox}{%
            \ltx@IfUndefined{HOLOGO@ScaleBox@\hologoDriver}{%
            }{%
              \chardef\HOLOGO@choice=4 %
            }%
          }{%
            \chardef\HOLOGO@choice=3 %
          }%
        }{%
          \chardef\HOLOGO@choice=2 %
        }%
      }{%
        \chardef\HOLOGO@choice=1 %
      }%
      \ifcase\HOLOGO@choice
        \HOLOGO@WarningUnsupportedDriver{Xe}%
        e%
      \or % 1: reflectbox
        \lower.5ex\hbox{%
          \reflectbox{E}%
        }%
      \or % 2: \pgftext
        \lower.5ex\hbox{%
          \pgfpicture
            \pgftransformxscale{-1}%
            \pgftext{E}%
          \endpgfpicture
        }%
      \or % 3: \psscalebox
        \lower.5ex\hbox{%
          \psscalebox{-1 1}{E}%
        }%
      \or % 4: \HOLOGO@Reflectbox
        \lower.5ex\hbox{%
          \HOLOGO@ReflectBox{E}%
        }%
      \else
        \@PackageError{hologo}{Internal error (choice/up)}\@ehc
      \fi
    \fi
  }%
}
%    \end{macrocode}
%    \end{macro}
%    \begin{macro}{\HoLogoHtml@Xe}
%    \begin{macrocode}
\def\HoLogoHtml@Xe#1{%
  \HoLogoCss@Xe
  \HOLOGO@Span{Xe}{%
    X%
    \HOLOGO@Span{e}{%
      \HCode{&\ltx@hashchar x018e;}%
    }%
  }%
}
%    \end{macrocode}
%    \end{macro}
%    \begin{macro}{\HoLogoCss@Xe}
%    \begin{macrocode}
\def\HoLogoCss@Xe{%
  \Css{%
    span.HoLogo-Xe span.HoLogo-e{%
      position:relative;%
      top:.5ex;%
      left-margin:-.1em;%
    }%
  }%
  \global\let\HoLogoCss@Xe\relax
}
%    \end{macrocode}
%    \end{macro}
%
%    \begin{macro}{\HoLogo@XeTeX}
%    \begin{macrocode}
\def\HoLogo@XeTeX#1{%
  \hologo{Xe}%
  \kern-.15em\relax
  \hologo{TeX}%
}
%    \end{macrocode}
%    \end{macro}
%
%    \begin{macro}{\HoLogoHtml@XeTeX}
%    \begin{macrocode}
\def\HoLogoHtml@XeTeX#1{%
  \HoLogoCss@XeTeX
  \HOLOGO@Span{XeTeX}{%
    \hologo{Xe}%
    \hologo{TeX}%
  }%
}
%    \end{macrocode}
%    \end{macro}
%    \begin{macro}{\HoLogoCss@XeTeX}
%    \begin{macrocode}
\def\HoLogoCss@XeTeX{%
  \Css{%
    span.HoLogo-XeTeX span.HoLogo-TeX{%
      margin-left:-.15em;%
    }%
  }%
  \global\let\HoLogoCss@XeTeX\relax
}
%    \end{macrocode}
%    \end{macro}
%
%    \begin{macro}{\HoLogo@XeLaTeX}
%    \begin{macrocode}
\def\HoLogo@XeLaTeX#1{%
  \hologo{Xe}%
  \kern-.13em%
  \hologo{LaTeX}%
}
%    \end{macrocode}
%    \end{macro}
%    \begin{macro}{\HoLogoHtml@XeLaTeX}
%    \begin{macrocode}
\def\HoLogoHtml@XeLaTeX#1{%
  \HoLogoCss@XeLaTeX
  \HOLOGO@Span{XeLaTeX}{%
    \hologo{Xe}%
    \hologo{LaTeX}%
  }%
}
%    \end{macrocode}
%    \end{macro}
%    \begin{macro}{\HoLogoCss@XeLaTeX}
%    \begin{macrocode}
\def\HoLogoCss@XeLaTeX{%
  \Css{%
    span.HoLogo-XeLaTeX span.HoLogo-Xe{%
      margin-right:-.13em;%
    }%
  }%
  \global\let\HoLogoCss@XeLaTeX\relax
}
%    \end{macrocode}
%    \end{macro}
%
% \subsubsection{\hologo{pdfTeX}, \hologo{pdfLaTeX}}
%
%    \begin{macro}{\HoLogo@pdfTeX}
%    \begin{macrocode}
\def\HoLogo@pdfTeX#1{%
  \HOLOGO@mbox{%
    #1{p}{P}df\hologo{TeX}%
  }%
}
%    \end{macrocode}
%    \end{macro}
%    \begin{macro}{\HoLogoCs@pdfTeX}
%    \begin{macrocode}
\def\HoLogoCs@pdfTeX#1{#1{p}{P}dfTeX}
%    \end{macrocode}
%    \end{macro}
%    \begin{macro}{\HoLogoBkm@pdfTeX}
%    \begin{macrocode}
\def\HoLogoBkm@pdfTeX#1{%
  #1{p}{P}df\hologo{TeX}%
}
%    \end{macrocode}
%    \end{macro}
%    \begin{macro}{\HoLogoHtml@pdfTeX}
%    \begin{macrocode}
\let\HoLogoHtml@pdfTeX\HoLogo@pdfTeX
%    \end{macrocode}
%    \end{macro}
%
%    \begin{macro}{\HoLogo@pdfLaTeX}
%    \begin{macrocode}
\def\HoLogo@pdfLaTeX#1{%
  \HOLOGO@mbox{%
    #1{p}{P}df\hologo{LaTeX}%
  }%
}
%    \end{macrocode}
%    \end{macro}
%    \begin{macro}{\HoLogoCs@pdfLaTeX}
%    \begin{macrocode}
\def\HoLogoCs@pdfLaTeX#1{#1{p}{P}dfLaTeX}
%    \end{macrocode}
%    \end{macro}
%    \begin{macro}{\HoLogoBkm@pdfLaTeX}
%    \begin{macrocode}
\def\HoLogoBkm@pdfLaTeX#1{%
  #1{p}{P}df\hologo{LaTeX}%
}
%    \end{macrocode}
%    \end{macro}
%    \begin{macro}{\HoLogoHtml@pdfLaTeX}
%    \begin{macrocode}
\let\HoLogoHtml@pdfLaTeX\HoLogo@pdfLaTeX
%    \end{macrocode}
%    \end{macro}
%
% \subsubsection{\hologo{VTeX}}
%
%    \begin{macro}{\HoLogo@VTeX}
%    \begin{macrocode}
\def\HoLogo@VTeX#1{%
  \HOLOGO@mbox{%
    V\hologo{TeX}%
  }%
}
%    \end{macrocode}
%    \end{macro}
%    \begin{macro}{\HoLogoHtml@VTeX}
%    \begin{macrocode}
\let\HoLogoHtml@VTeX\HoLogo@VTeX
%    \end{macrocode}
%    \end{macro}
%
% \subsubsection{\hologo{AmS}, \dots}
%
%    Source: class \xclass{amsdtx}
%
%    \begin{macro}{\HoLogo@AmS}
%    \begin{macrocode}
\def\HoLogo@AmS#1{%
  \HoLogoFont@font{AmS}{sy}{%
    A%
    \kern-.1667em%
    \lower.5ex\hbox{M}%
    \kern-.125em%
    S%
  }%
}
%    \end{macrocode}
%    \end{macro}
%    \begin{macro}{\HoLogoBkm@AmS}
%    \begin{macrocode}
\def\HoLogoBkm@AmS#1{AmS}
%    \end{macrocode}
%    \end{macro}
%    \begin{macro}{\HoLogoHtml@AmS}
%    \begin{macrocode}
\def\HoLogoHtml@AmS#1{%
  \HoLogoCss@AmS
%  \HoLogoFont@font{AmS}{sy}{%
    \HOLOGO@Span{AmS}{%
      A%
      \HOLOGO@Span{M}{M}%
      S%
    }%
%   }%
}
%    \end{macrocode}
%    \end{macro}
%    \begin{macro}{\HoLogoCss@AmS}
%    \begin{macrocode}
\def\HoLogoCss@AmS{%
  \Css{%
    span.HoLogo-AmS span.HoLogo-M{%
      position:relative;%
      top:.5ex;%
      margin-left:-.1667em;%
      margin-right:-.125em;%
      text-decoration:none;%
    }%
  }%
  \global\let\HoLogoCss@AmS\relax
}
%    \end{macrocode}
%    \end{macro}
%
%    \begin{macro}{\HoLogo@AmSTeX}
%    \begin{macrocode}
\def\HoLogo@AmSTeX#1{%
  \hologo{AmS}%
  \HOLOGO@hyphen
  \hologo{TeX}%
}
%    \end{macrocode}
%    \end{macro}
%    \begin{macro}{\HoLogoBkm@AmSTeX}
%    \begin{macrocode}
\def\HoLogoBkm@AmSTeX#1{AmS-TeX}%
%    \end{macrocode}
%    \end{macro}
%    \begin{macro}{\HoLogoHtml@AmSTeX}
%    \begin{macrocode}
\let\HoLogoHtml@AmSTeX\HoLogo@AmSTeX
%    \end{macrocode}
%    \end{macro}
%
%    \begin{macro}{\HoLogo@AmSLaTeX}
%    \begin{macrocode}
\def\HoLogo@AmSLaTeX#1{%
  \hologo{AmS}%
  \HOLOGO@hyphen
  \hologo{LaTeX}%
}
%    \end{macrocode}
%    \end{macro}
%    \begin{macro}{\HoLogoBkm@AmSLaTeX}
%    \begin{macrocode}
\def\HoLogoBkm@AmSLaTeX#1{AmS-LaTeX}%
%    \end{macrocode}
%    \end{macro}
%    \begin{macro}{\HoLogoHtml@AmSLaTeX}
%    \begin{macrocode}
\let\HoLogoHtml@AmSLaTeX\HoLogo@AmSLaTeX
%    \end{macrocode}
%    \end{macro}
%
% \subsubsection{\hologo{BibTeX}}
%
%    \begin{macro}{\HoLogo@BibTeX@sc}
%    A definition of \hologo{BibTeX} is provided in
%    the documentation source for the manual of \hologo{BibTeX}
%    \cite{btxdoc}.
%\begin{quote}
%\begin{verbatim}
%\def\BibTeX{%
%  {%
%    \rm
%    B%
%    \kern-.05em%
%    {%
%      \sc
%      i%
%      \kern-.025em %
%      b%
%    }%
%    \kern-.08em
%    T%
%    \kern-.1667em%
%    \lower.7ex\hbox{E}%
%    \kern-.125em%
%    X%
%  }%
%}
%\end{verbatim}
%\end{quote}
%    \begin{macrocode}
\def\HoLogo@BibTeX@sc#1{%
  B%
  \kern-.05em%
  \HoLogoFont@font{BibTeX}{sc}{%
    i%
    \kern-.025em%
    b%
  }%
  \HOLOGO@discretionary
  \kern-.08em%
  \hologo{TeX}%
}
%    \end{macrocode}
%    \end{macro}
%    \begin{macro}{\HoLogoHtml@BibTeX@sc}
%    \begin{macrocode}
\def\HoLogoHtml@BibTeX@sc#1{%
  \HoLogoCss@BibTeX@sc
  \HOLOGO@Span{BibTeX-sc}{%
    B%
    \HOLOGO@Span{i}{i}%
    \HOLOGO@Span{b}{b}%
    \hologo{TeX}%
  }%
}
%    \end{macrocode}
%    \end{macro}
%    \begin{macro}{\HoLogoCss@BibTeX@sc}
%    \begin{macrocode}
\def\HoLogoCss@BibTeX@sc{%
  \Css{%
    span.HoLogo-BibTeX-sc span.HoLogo-i{%
      margin-left:-.05em;%
      margin-right:-.025em;%
      font-variant:small-caps;%
    }%
  }%
  \Css{%
    span.HoLogo-BibTeX-sc span.HoLogo-b{%
      margin-right:-.08em;%
      font-variant:small-caps;%
    }%
  }%
  \global\let\HoLogoCss@BibTeX@sc\relax
}
%    \end{macrocode}
%    \end{macro}
%
%    \begin{macro}{\HoLogo@BibTeX@sf}
%    Variant \xoption{sf} avoids trouble with unavailable
%    small caps fonts (e.g., bold versions of Computer Modern or
%    Latin Modern). The definition is taken from
%    package \xpackage{dtklogos} \cite{dtklogos}.
%\begin{quote}
%\begin{verbatim}
%\DeclareRobustCommand{\BibTeX}{%
%  B%
%  \kern-.05em%
%  \hbox{%
%    $\m@th$% %% force math size calculations
%    \csname S@\f@size\endcsname
%    \fontsize\sf@size\z@
%    \math@fontsfalse
%    \selectfont
%    I%
%    \kern-.025em%
%    B
%  }%
%  \kern-.08em%
%  \-%
%  \TeX
%}
%\end{verbatim}
%\end{quote}
%    \begin{macrocode}
\def\HoLogo@BibTeX@sf#1{%
  B%
  \kern-.05em%
  \HoLogoFont@font{BibTeX}{bibsf}{%
    I%
    \kern-.025em%
    B%
  }%
  \HOLOGO@discretionary
  \kern-.08em%
  \hologo{TeX}%
}
%    \end{macrocode}
%    \end{macro}
%    \begin{macro}{\HoLogoHtml@BibTeX@sf}
%    \begin{macrocode}
\def\HoLogoHtml@BibTeX@sf#1{%
  \HoLogoCss@BibTeX@sf
  \HOLOGO@Span{BibTeX-sf}{%
    B%
    \HoLogoFont@font{BibTeX}{bibsf}{%
      \HOLOGO@Span{i}{I}%
      B%
    }%
    \hologo{TeX}%
  }%
}
%    \end{macrocode}
%    \end{macro}
%    \begin{macro}{\HoLogoCss@BibTeX@sf}
%    \begin{macrocode}
\def\HoLogoCss@BibTeX@sf{%
  \Css{%
    span.HoLogo-BibTeX-sf span.HoLogo-i{%
      margin-left:-.05em;%
      margin-right:-.025em;%
    }%
  }%
  \Css{%
    span.HoLogo-BibTeX-sf span.HoLogo-TeX{%
      margin-left:-.08em;%
    }%
  }%
  \global\let\HoLogoCss@BibTeX@sf\relax
}
%    \end{macrocode}
%    \end{macro}
%
%    \begin{macro}{\HoLogo@BibTeX}
%    \begin{macrocode}
\def\HoLogo@BibTeX{\HoLogo@BibTeX@sf}
%    \end{macrocode}
%    \end{macro}
%    \begin{macro}{\HoLogoHtml@BibTeX}
%    \begin{macrocode}
\def\HoLogoHtml@BibTeX{\HoLogoHtml@BibTeX@sf}
%    \end{macrocode}
%    \end{macro}
%
% \subsubsection{\hologo{BibTeX8}}
%
%    \begin{macro}{\HoLogo@BibTeX8}
%    \begin{macrocode}
\expandafter\def\csname HoLogo@BibTeX8\endcsname#1{%
  \hologo{BibTeX}%
  8%
}
%    \end{macrocode}
%    \end{macro}
%
%    \begin{macro}{\HoLogoBkm@BibTeX8}
%    \begin{macrocode}
\expandafter\def\csname HoLogoBkm@BibTeX8\endcsname#1{%
  \hologo{BibTeX}%
  8%
}
%    \end{macrocode}
%    \end{macro}
%    \begin{macro}{\HoLogoHtml@BibTeX8}
%    \begin{macrocode}
\expandafter
\let\csname HoLogoHtml@BibTeX8\expandafter\endcsname
\csname HoLogo@BibTeX8\endcsname
%    \end{macrocode}
%    \end{macro}
%
% \subsubsection{\hologo{ConTeXt}}
%
%    \begin{macro}{\HoLogo@ConTeXt@simple}
%    \begin{macrocode}
\def\HoLogo@ConTeXt@simple#1{%
  \HOLOGO@mbox{Con}%
  \HOLOGO@discretionary
  \HOLOGO@mbox{\hologo{TeX}t}%
}
%    \end{macrocode}
%    \end{macro}
%    \begin{macro}{\HoLogoHtml@ConTeXt@simple}
%    \begin{macrocode}
\let\HoLogoHtml@ConTeXt@simple\HoLogo@ConTeXt@simple
%    \end{macrocode}
%    \end{macro}
%
%    \begin{macro}{\HoLogo@ConTeXt@narrow}
%    This definition of logo \hologo{ConTeXt} with variant \xoption{narrow}
%    comes from TUGboat's class \xclass{ltugboat} (version 2010/11/15 v2.8).
%    \begin{macrocode}
\def\HoLogo@ConTeXt@narrow#1{%
  \HOLOGO@mbox{C\kern-.0333emon}%
  \HOLOGO@discretionary
  \kern-.0667em%
  \HOLOGO@mbox{\hologo{TeX}\kern-.0333emt}%
}
%    \end{macrocode}
%    \end{macro}
%    \begin{macro}{\HoLogoHtml@ConTeXt@narrow}
%    \begin{macrocode}
\def\HoLogoHtml@ConTeXt@narrow#1{%
  \HoLogoCss@ConTeXt@narrow
  \HOLOGO@Span{ConTeXt-narrow}{%
    \HOLOGO@Span{C}{C}%
    on%
    \hologo{TeX}%
    t%
  }%
}
%    \end{macrocode}
%    \end{macro}
%    \begin{macro}{\HoLogoCss@ConTeXt@narrow}
%    \begin{macrocode}
\def\HoLogoCss@ConTeXt@narrow{%
  \Css{%
    span.HoLogo-ConTeXt-narrow span.HoLogo-C{%
      margin-left:-.0333em;%
    }%
  }%
  \Css{%
    span.HoLogo-ConTeXt-narrow span.HoLogo-TeX{%
      margin-left:-.0667em;%
      margin-right:-.0333em;%
    }%
  }%
  \global\let\HoLogoCss@ConTeXt@narrow\relax
}
%    \end{macrocode}
%    \end{macro}
%
%    \begin{macro}{\HoLogo@ConTeXt}
%    \begin{macrocode}
\def\HoLogo@ConTeXt{\HoLogo@ConTeXt@narrow}
%    \end{macrocode}
%    \end{macro}
%    \begin{macro}{\HoLogoHtml@ConTeXt}
%    \begin{macrocode}
\def\HoLogoHtml@ConTeXt{\HoLogoHtml@ConTeXt@narrow}
%    \end{macrocode}
%    \end{macro}
%
% \subsubsection{\hologo{emTeX}}
%
%    \begin{macro}{\HoLogo@emTeX}
%    \begin{macrocode}
\def\HoLogo@emTeX#1{%
  \HOLOGO@mbox{#1{e}{E}m}%
  \HOLOGO@discretionary
  \hologo{TeX}%
}
%    \end{macrocode}
%    \end{macro}
%    \begin{macro}{\HoLogoCs@emTeX}
%    \begin{macrocode}
\def\HoLogoCs@emTeX#1{#1{e}{E}mTeX}%
%    \end{macrocode}
%    \end{macro}
%    \begin{macro}{\HoLogoBkm@emTeX}
%    \begin{macrocode}
\def\HoLogoBkm@emTeX#1{%
  #1{e}{E}m\hologo{TeX}%
}
%    \end{macrocode}
%    \end{macro}
%    \begin{macro}{\HoLogoHtml@emTeX}
%    \begin{macrocode}
\let\HoLogoHtml@emTeX\HoLogo@emTeX
%    \end{macrocode}
%    \end{macro}
%
% \subsubsection{\hologo{ExTeX}}
%
%    \begin{macro}{\HoLogo@ExTeX}
%    The definition is taken from the FAQ of the
%    project \hologo{ExTeX}
%    \cite{ExTeX-FAQ}.
%\begin{quote}
%\begin{verbatim}
%\def\ExTeX{%
%  \textrm{% Logo always with serifs
%    \ensuremath{%
%      \textstyle
%      \varepsilon_{%
%        \kern-0.15em%
%        \mathcal{X}%
%      }%
%    }%
%    \kern-.15em%
%    \TeX
%  }%
%}
%\end{verbatim}
%\end{quote}
%    \begin{macrocode}
\def\HoLogo@ExTeX#1{%
  \HoLogoFont@font{ExTeX}{rm}{%
    \ltx@mbox{%
      \HOLOGO@MathSetup
      $%
        \textstyle
        \varepsilon_{%
          \kern-0.15em%
          \HoLogoFont@font{ExTeX}{sy}{X}%
        }%
      $%
    }%
    \HOLOGO@discretionary
    \kern-.15em%
    \hologo{TeX}%
  }%
}
%    \end{macrocode}
%    \end{macro}
%    \begin{macro}{\HoLogoHtml@ExTeX}
%    \begin{macrocode}
\def\HoLogoHtml@ExTeX#1{%
  \HoLogoCss@ExTeX
  \HoLogoFont@font{ExTeX}{rm}{%
    \HOLOGO@Span{ExTeX}{%
      \ltx@mbox{%
        \HOLOGO@MathSetup
        $\textstyle\varepsilon$%
        \HOLOGO@Span{X}{$\textstyle\chi$}%
        \hologo{TeX}%
      }%
    }%
  }%
}
%    \end{macrocode}
%    \end{macro}
%    \begin{macro}{\HoLogoBkm@ExTeX}
%    \begin{macrocode}
\def\HoLogoBkm@ExTeX#1{%
  \HOLOGO@PdfdocUnicode{#1{e}{E}x}{\textepsilon\textchi}%
  \hologo{TeX}%
}
%    \end{macrocode}
%    \end{macro}
%    \begin{macro}{\HoLogoCss@ExTeX}
%    \begin{macrocode}
\def\HoLogoCss@ExTeX{%
  \Css{%
    span.HoLogo-ExTeX{%
      font-family:serif;%
    }%
  }%
  \Css{%
    span.HoLogo-ExTeX span.HoLogo-TeX{%
      margin-left:-.15em;%
    }%
  }%
  \global\let\HoLogoCss@ExTeX\relax
}
%    \end{macrocode}
%    \end{macro}
%
% \subsubsection{\hologo{MiKTeX}}
%
%    \begin{macro}{\HoLogo@MiKTeX}
%    \begin{macrocode}
\def\HoLogo@MiKTeX#1{%
  \HOLOGO@mbox{MiK}%
  \HOLOGO@discretionary
  \hologo{TeX}%
}
%    \end{macrocode}
%    \end{macro}
%    \begin{macro}{\HoLogoHtml@MiKTeX}
%    \begin{macrocode}
\let\HoLogoHtml@MiKTeX\HoLogo@MiKTeX
%    \end{macrocode}
%    \end{macro}
%
% \subsubsection{\hologo{OzTeX} and friends}
%
%    Source: \hologo{OzTeX} FAQ \cite{OzTeX}:
%    \begin{quote}
%      |\def\OzTeX{O\kern-.03em z\kern-.15em\TeX}|\\
%      (There is no kerning in OzMF, OzMP and OzTtH.)
%    \end{quote}
%
%    \begin{macro}{\HoLogo@OzTeX}
%    \begin{macrocode}
\def\HoLogo@OzTeX#1{%
  O%
  \kern-.03em %
  z%
  \kern-.15em %
  \hologo{TeX}%
}
%    \end{macrocode}
%    \end{macro}
%    \begin{macro}{\HoLogoHtml@OzTeX}
%    \begin{macrocode}
\def\HoLogoHtml@OzTeX#1{%
  \HoLogoCss@OzTeX
  \HOLOGO@Span{OzTeX}{%
    O%
    \HOLOGO@Span{z}{z}%
    \hologo{TeX}%
  }%
}
%    \end{macrocode}
%    \end{macro}
%    \begin{macro}{\HoLogoCss@OzTeX}
%    \begin{macrocode}
\def\HoLogoCss@OzTeX{%
  \Css{%
    span.HoLogo-OzTeX span.HoLogo-z{%
      margin-left:-.03em;%
      margin-right:-.15em;%
    }%
  }%
  \global\let\HoLogoCss@OzTeX\relax
}
%    \end{macrocode}
%    \end{macro}
%
%    \begin{macro}{\HoLogo@OzMF}
%    \begin{macrocode}
\def\HoLogo@OzMF#1{%
  \HOLOGO@mbox{OzMF}%
}
%    \end{macrocode}
%    \end{macro}
%    \begin{macro}{\HoLogo@OzMP}
%    \begin{macrocode}
\def\HoLogo@OzMP#1{%
  \HOLOGO@mbox{OzMP}%
}
%    \end{macrocode}
%    \end{macro}
%    \begin{macro}{\HoLogo@OzTtH}
%    \begin{macrocode}
\def\HoLogo@OzTtH#1{%
  \HOLOGO@mbox{OzTtH}%
}
%    \end{macrocode}
%    \end{macro}
%
% \subsubsection{\hologo{PCTeX}}
%
%    \begin{macro}{\HoLogo@PCTeX}
%    \begin{macrocode}
\def\HoLogo@PCTeX#1{%
  \HOLOGO@mbox{PC}%
  \hologo{TeX}%
}
%    \end{macrocode}
%    \end{macro}
%    \begin{macro}{\HoLogoHtml@PCTeX}
%    \begin{macrocode}
\let\HoLogoHtml@PCTeX\HoLogo@PCTeX
%    \end{macrocode}
%    \end{macro}
%
% \subsubsection{\hologo{PiCTeX}}
%
%    The original definitions from \xfile{pictex.tex} \cite{PiCTeX}:
%\begin{quote}
%\begin{verbatim}
%\def\PiC{%
%  P%
%  \kern-.12em%
%  \lower.5ex\hbox{I}%
%  \kern-.075em%
%  C%
%}
%\def\PiCTeX{%
%  \PiC
%  \kern-.11em%
%  \TeX
%}
%\end{verbatim}
%\end{quote}
%
%    \begin{macro}{\HoLogo@PiC}
%    \begin{macrocode}
\def\HoLogo@PiC#1{%
  P%
  \kern-.12em%
  \lower.5ex\hbox{I}%
  \kern-.075em%
  C%
  \HOLOGO@SpaceFactor
}
%    \end{macrocode}
%    \end{macro}
%    \begin{macro}{\HoLogoHtml@PiC}
%    \begin{macrocode}
\def\HoLogoHtml@PiC#1{%
  \HoLogoCss@PiC
  \HOLOGO@Span{PiC}{%
    P%
    \HOLOGO@Span{i}{I}%
    C%
  }%
}
%    \end{macrocode}
%    \end{macro}
%    \begin{macro}{\HoLogoCss@PiC}
%    \begin{macrocode}
\def\HoLogoCss@PiC{%
  \Css{%
    span.HoLogo-PiC span.HoLogo-i{%
      position:relative;%
      top:.5ex;%
      margin-left:-.12em;%
      margin-right:-.075em;%
      text-decoration:none;%
    }%
  }%
  \global\let\HoLogoCss@PiC\relax
}
%    \end{macrocode}
%    \end{macro}
%
%    \begin{macro}{\HoLogo@PiCTeX}
%    \begin{macrocode}
\def\HoLogo@PiCTeX#1{%
  \hologo{PiC}%
  \HOLOGO@discretionary
  \kern-.11em%
  \hologo{TeX}%
}
%    \end{macrocode}
%    \end{macro}
%    \begin{macro}{\HoLogoHtml@PiCTeX}
%    \begin{macrocode}
\def\HoLogoHtml@PiCTeX#1{%
  \HoLogoCss@PiCTeX
  \HOLOGO@Span{PiCTeX}{%
    \hologo{PiC}%
    \hologo{TeX}%
  }%
}
%    \end{macrocode}
%    \end{macro}
%    \begin{macro}{\HoLogoCss@PiCTeX}
%    \begin{macrocode}
\def\HoLogoCss@PiCTeX{%
  \Css{%
    span.HoLogo-PiCTeX span.HoLogo-PiC{%
      margin-right:-.11em;%
    }%
  }%
  \global\let\HoLogoCss@PiCTeX\relax
}
%    \end{macrocode}
%    \end{macro}
%
% \subsubsection{\hologo{teTeX}}
%
%    \begin{macro}{\HoLogo@teTeX}
%    \begin{macrocode}
\def\HoLogo@teTeX#1{%
  \HOLOGO@mbox{#1{t}{T}e}%
  \HOLOGO@discretionary
  \hologo{TeX}%
}
%    \end{macrocode}
%    \end{macro}
%    \begin{macro}{\HoLogoCs@teTeX}
%    \begin{macrocode}
\def\HoLogoCs@teTeX#1{#1{t}{T}dfTeX}
%    \end{macrocode}
%    \end{macro}
%    \begin{macro}{\HoLogoBkm@teTeX}
%    \begin{macrocode}
\def\HoLogoBkm@teTeX#1{%
  #1{t}{T}e\hologo{TeX}%
}
%    \end{macrocode}
%    \end{macro}
%    \begin{macro}{\HoLogoHtml@teTeX}
%    \begin{macrocode}
\let\HoLogoHtml@teTeX\HoLogo@teTeX
%    \end{macrocode}
%    \end{macro}
%
% \subsubsection{\hologo{TeX4ht}}
%
%    \begin{macro}{\HoLogo@TeX4ht}
%    \begin{macrocode}
\expandafter\def\csname HoLogo@TeX4ht\endcsname#1{%
  \HOLOGO@mbox{\hologo{TeX}4ht}%
}
%    \end{macrocode}
%    \end{macro}
%    \begin{macro}{\HoLogoHtml@TeX4ht}
%    \begin{macrocode}
\expandafter
\let\csname HoLogoHtml@TeX4ht\expandafter\endcsname
\csname HoLogo@TeX4ht\endcsname
%    \end{macrocode}
%    \end{macro}
%
%
% \subsubsection{\hologo{SageTeX}}
%
%    \begin{macro}{\HoLogo@SageTeX}
%    \begin{macrocode}
\def\HoLogo@SageTeX#1{%
  \HOLOGO@mbox{Sage}%
  \HOLOGO@discretionary
  \HOLOGO@NegativeKerning{eT,oT,To}%
  \hologo{TeX}%
}
%    \end{macrocode}
%    \end{macro}
%    \begin{macro}{\HoLogoHtml@SageTeX}
%    \begin{macrocode}
\let\HoLogoHtml@SageTeX\HoLogo@SageTeX
%    \end{macrocode}
%    \end{macro}
%
% \subsection{\hologo{METAFONT} and friends}
%
%    \begin{macro}{\HoLogo@METAFONT}
%    \begin{macrocode}
\def\HoLogo@METAFONT#1{%
  \HoLogoFont@font{METAFONT}{logo}{%
    \HOLOGO@mbox{META}%
    \HOLOGO@discretionary
    \HOLOGO@mbox{FONT}%
  }%
}
%    \end{macrocode}
%    \end{macro}
%
%    \begin{macro}{\HoLogo@METAPOST}
%    \begin{macrocode}
\def\HoLogo@METAPOST#1{%
  \HoLogoFont@font{METAPOST}{logo}{%
    \HOLOGO@mbox{META}%
    \HOLOGO@discretionary
    \HOLOGO@mbox{POST}%
  }%
}
%    \end{macrocode}
%    \end{macro}
%
%    \begin{macro}{\HoLogo@MetaFun}
%    \begin{macrocode}
\def\HoLogo@MetaFun#1{%
  \HOLOGO@mbox{Meta}%
  \HOLOGO@discretionary
  \HOLOGO@mbox{Fun}%
}
%    \end{macrocode}
%    \end{macro}
%
%    \begin{macro}{\HoLogo@MetaPost}
%    \begin{macrocode}
\def\HoLogo@MetaPost#1{%
  \HOLOGO@mbox{Meta}%
  \HOLOGO@discretionary
  \HOLOGO@mbox{Post}%
}
%    \end{macrocode}
%    \end{macro}
%
% \subsection{Others}
%
% \subsubsection{\hologo{biber}}
%
%    \begin{macro}{\HoLogo@biber}
%    \begin{macrocode}
\def\HoLogo@biber#1{%
  \HOLOGO@mbox{#1{b}{B}i}%
  \HOLOGO@discretionary
  \HOLOGO@mbox{ber}%
}
%    \end{macrocode}
%    \end{macro}
%    \begin{macro}{\HoLogoCs@biber}
%    \begin{macrocode}
\def\HoLogoCs@biber#1{#1{b}{B}iber}
%    \end{macrocode}
%    \end{macro}
%    \begin{macro}{\HoLogoBkm@biber}
%    \begin{macrocode}
\def\HoLogoBkm@biber#1{%
  #1{b}{B}iber%
}
%    \end{macrocode}
%    \end{macro}
%    \begin{macro}{\HoLogoHtml@biber}
%    \begin{macrocode}
\let\HoLogoHtml@biber\HoLogo@biber
%    \end{macrocode}
%    \end{macro}
%
% \subsubsection{\hologo{KOMAScript}}
%
%    \begin{macro}{\HoLogo@KOMAScript}
%    The definition for \hologo{KOMAScript} is taken
%    from \hologo{KOMAScript} (\xfile{scrlogo.dtx}, reformatted) \cite{scrlogo}:
%\begin{quote}
%\begin{verbatim}
%\@ifundefined{KOMAScript}{%
%  \DeclareRobustCommand{\KOMAScript}{%
%    \textsf{%
%      K\kern.05em O\kern.05emM\kern.05em A%
%      \kern.1em-\kern.1em %
%      Script%
%    }%
%  }%
%}{}
%\end{verbatim}
%\end{quote}
%    \begin{macrocode}
\def\HoLogo@KOMAScript#1{%
  \HoLogoFont@font{KOMAScript}{sf}{%
    \HOLOGO@mbox{%
      K\kern.05em%
      O\kern.05em%
      M\kern.05em%
      A%
    }%
    \kern.1em%
    \HOLOGO@hyphen
    \kern.1em%
    \HOLOGO@mbox{Script}%
  }%
}
%    \end{macrocode}
%    \end{macro}
%    \begin{macro}{\HoLogoBkm@KOMAScript}
%    \begin{macrocode}
\def\HoLogoBkm@KOMAScript#1{%
  KOMA-Script%
}
%    \end{macrocode}
%    \end{macro}
%    \begin{macro}{\HoLogoHtml@KOMAScript}
%    \begin{macrocode}
\def\HoLogoHtml@KOMAScript#1{%
  \HoLogoCss@KOMAScript
  \HoLogoFont@font{KOMAScript}{sf}{%
    \HOLOGO@Span{KOMAScript}{%
      K%
      \HOLOGO@Span{O}{O}%
      M%
      \HOLOGO@Span{A}{A}%
      \HOLOGO@Span{hyphen}{-}%
      Script%
    }%
  }%
}
%    \end{macrocode}
%    \end{macro}
%    \begin{macro}{\HoLogoCss@KOMAScript}
%    \begin{macrocode}
\def\HoLogoCss@KOMAScript{%
  \Css{%
    span.HoLogo-KOMAScript{%
      font-family:sans-serif;%
    }%
  }%
  \Css{%
    span.HoLogo-KOMAScript span.HoLogo-O{%
      padding-left:.05em;%
      padding-right:.05em;%
    }%
  }%
  \Css{%
    span.HoLogo-KOMAScript span.HoLogo-A{%
      padding-left:.05em;%
    }%
  }%
  \Css{%
    span.HoLogo-KOMAScript span.HoLogo-hyphen{%
      padding-left:.1em;%
      padding-right:.1em;%
    }%
  }%
  \global\let\HoLogoCss@KOMAScript\relax
}
%    \end{macrocode}
%    \end{macro}
%
% \subsubsection{\hologo{LyX}}
%
%    \begin{macro}{\HoLogo@LyX}
%    The definition is taken from the documentation source files
%    of \hologo{LyX}, \xfile{Intro.lyx} \cite{LyX}:
%\begin{quote}
%\begin{verbatim}
%\def\LyX{%
%  \texorpdfstring{%
%    L\kern-.1667em\lower.25em\hbox{Y}\kern-.125emX\@%
%  }{%
%    LyX%
%  }%
%}
%\end{verbatim}
%\end{quote}
%    \begin{macrocode}
\def\HoLogo@LyX#1{%
  L%
  \kern-.1667em%
  \lower.25em\hbox{Y}%
  \kern-.125em%
  X%
  \HOLOGO@SpaceFactor
}
%    \end{macrocode}
%    \end{macro}
%    \begin{macro}{\HoLogoHtml@LyX}
%    \begin{macrocode}
\def\HoLogoHtml@LyX#1{%
  \HoLogoCss@LyX
  \HOLOGO@Span{LyX}{%
    L%
    \HOLOGO@Span{y}{Y}%
    X%
  }%
}
%    \end{macrocode}
%    \end{macro}
%    \begin{macro}{\HoLogoCss@LyX}
%    \begin{macrocode}
\def\HoLogoCss@LyX{%
  \Css{%
    span.HoLogo-LyX span.HoLogo-y{%
      position:relative;%
      top:.25em;%
      margin-left:-.1667em;%
      margin-right:-.125em;%
      text-decoration:none;%
    }%
  }%
  \global\let\HoLogoCss@LyX\relax
}
%    \end{macrocode}
%    \end{macro}
%
% \subsubsection{\hologo{NTS}}
%
%    \begin{macro}{\HoLogo@NTS}
%    Definition for \hologo{NTS} can be found in
%    package \xpackage{etex\textunderscore man} for the \hologo{eTeX} manual \cite{etexman}
%    and in package \xpackage{dtklogos} \cite{dtklogos}:
%\begin{quote}
%\begin{verbatim}
%\def\NTS{%
%  \leavevmode
%  \hbox{%
%    $%
%      \cal N%
%      \kern-0.35em%
%      \lower0.5ex\hbox{$\cal T$}%
%      \kern-0.2em%
%      S%
%    $%
%  }%
%}
%\end{verbatim}
%\end{quote}
%    \begin{macrocode}
\def\HoLogo@NTS#1{%
  \HoLogoFont@font{NTS}{sy}{%
    N\/%
    \kern-.35em%
    \lower.5ex\hbox{T\/}%
    \kern-.2em%
    S\/%
  }%
  \HOLOGO@SpaceFactor
}
%    \end{macrocode}
%    \end{macro}
%
% \subsubsection{\Hologo{TTH} (\hologo{TeX} to HTML translator)}
%
%    Source: \url{http://hutchinson.belmont.ma.us/tth/}
%    In the HTML source the second `T' is printed as subscript.
%\begin{quote}
%\begin{verbatim}
%T<sub>T</sub>H
%\end{verbatim}
%\end{quote}
%    \begin{macro}{\HoLogo@TTH}
%    \begin{macrocode}
\def\HoLogo@TTH#1{%
  \ltx@mbox{%
    T\HOLOGO@SubScript{T}H%
  }%
  \HOLOGO@SpaceFactor
}
%    \end{macrocode}
%    \end{macro}
%
%    \begin{macro}{\HoLogoHtml@TTH}
%    \begin{macrocode}
\def\HoLogoHtml@TTH#1{%
  T\HCode{<sub>}T\HCode{</sub>}H%
}
%    \end{macrocode}
%    \end{macro}
%
% \subsubsection{\Hologo{HanTheThanh}}
%
%    Partial source: Package \xpackage{dtklogos}.
%    The double accent is U+1EBF (latin small letter e with circumflex
%    and acute).
%    \begin{macro}{\HoLogo@HanTheThanh}
%    \begin{macrocode}
\def\HoLogo@HanTheThanh#1{%
  \ltx@mbox{H\`an}%
  \HOLOGO@space
  \ltx@mbox{%
    Th%
    \HOLOGO@IfCharExists{"1EBF}{%
      \char"1EBF\relax
    }{%
      \^e\hbox to 0pt{\hss\raise .5ex\hbox{\'{}}}%
    }%
  }%
  \HOLOGO@space
  \ltx@mbox{Th\`anh}%
}
%    \end{macrocode}
%    \end{macro}
%    \begin{macro}{\HoLogoBkm@HanTheThanh}
%    \begin{macrocode}
\def\HoLogoBkm@HanTheThanh#1{%
  H\`an %
  Th\HOLOGO@PdfdocUnicode{\^e}{\9036\277} %
  Th\`anh%
}
%    \end{macrocode}
%    \end{macro}
%    \begin{macro}{\HoLogoHtml@HanTheThanh}
%    \begin{macrocode}
\def\HoLogoHtml@HanTheThanh#1{%
  H\`an %
  Th\HCode{&\ltx@hashchar x1ebf;} %
  Th\`anh%
}
%    \end{macrocode}
%    \end{macro}
%
% \subsection{Driver detection}
%
%    \begin{macrocode}
\HOLOGO@IfExists\InputIfFileExists{%
  \InputIfFileExists{hologo.cfg}{}{}%
}{%
  \ltx@IfUndefined{pdf@filesize}{%
    \def\HOLOGO@InputIfExists{%
      \openin\HOLOGO@temp=hologo.cfg\relax
      \ifeof\HOLOGO@temp
        \closein\HOLOGO@temp
      \else
        \closein\HOLOGO@temp
        \begingroup
          \def\x{LaTeX2e}%
        \expandafter\endgroup
        \ifx\fmtname\x
          \input{hologo.cfg}%
        \else
          \input hologo.cfg\relax
        \fi
      \fi
    }%
    \ltx@IfUndefined{newread}{%
      \chardef\HOLOGO@temp=15 %
      \def\HOLOGO@CheckRead{%
        \ifeof\HOLOGO@temp
          \HOLOGO@InputIfExists
        \else
          \ifcase\HOLOGO@temp
            \@PackageWarningNoLine{hologo}{%
              Configuration file ignored, because\MessageBreak
              a free read register could not be found%
            }%
          \else
            \begingroup
              \count\ltx@cclv=\HOLOGO@temp
              \advance\ltx@cclv by \ltx@minusone
              \edef\x{\endgroup
                \chardef\noexpand\HOLOGO@temp=\the\count\ltx@cclv
                \relax
              }%
            \x
          \fi
        \fi
      }%
    }{%
      \csname newread\endcsname\HOLOGO@temp
      \HOLOGO@InputIfExists
    }%
  }{%
    \edef\HOLOGO@temp{\pdf@filesize{hologo.cfg}}%
    \ifx\HOLOGO@temp\ltx@empty
    \else
      \ifnum\HOLOGO@temp>0 %
        \begingroup
          \def\x{LaTeX2e}%
        \expandafter\endgroup
        \ifx\fmtname\x
          \input{hologo.cfg}%
        \else
          \input hologo.cfg\relax
        \fi
      \else
        \@PackageInfoNoLine{hologo}{%
          Empty configuration file `hologo.cfg' ignored%
        }%
      \fi
    \fi
  }%
}
%    \end{macrocode}
%
%    \begin{macrocode}
\def\HOLOGO@temp#1#2{%
  \kv@define@key{HoLogoDriver}{#1}[]{%
    \begingroup
      \def\HOLOGO@temp{##1}%
      \ltx@onelevel@sanitize\HOLOGO@temp
      \ifx\HOLOGO@temp\ltx@empty
      \else
        \@PackageError{hologo}{%
          Value (\HOLOGO@temp) not permitted for option `#1'%
        }%
        \@ehc
      \fi
    \endgroup
    \def\hologoDriver{#2}%
  }%
}%
\def\HOLOGO@@temp#1#2{%
  \ifx\kv@value\relax
    \HOLOGO@temp{#1}{#1}%
  \else
    \HOLOGO@temp{#1}{#2}%
  \fi
}%
\kv@parse@normalized{%
  pdftex,%
  luatex=pdftex,%
  dvipdfm,%
  dvipdfmx=dvipdfm,%
  dvips,%
  dvipsone=dvips,%
  xdvi=dvips,%
  xetex,%
  vtex,%
}\HOLOGO@@temp
%    \end{macrocode}
%
%    \begin{macrocode}
\kv@define@key{HoLogoDriver}{driverfallback}{%
  \def\HOLOGO@DriverFallback{#1}%
}
%    \end{macrocode}
%
%    \begin{macro}{\HOLOGO@DriverFallback}
%    \begin{macrocode}
\def\HOLOGO@DriverFallback{dvips}
%    \end{macrocode}
%    \end{macro}
%
%    \begin{macro}{\hologoDriverSetup}
%    \begin{macrocode}
\def\hologoDriverSetup{%
  \let\hologoDriver\ltx@undefined
  \HOLOGO@DriverSetup
}
%    \end{macrocode}
%    \end{macro}
%
%    \begin{macro}{\HOLOGO@DriverSetup}
%    \begin{macrocode}
\def\HOLOGO@DriverSetup#1{%
  \kvsetkeys{HoLogoDriver}{#1}%
  \HOLOGO@CheckDriver
  \ltx@ifundefined{hologoDriver}{%
    \begingroup
    \edef\x{\endgroup
      \noexpand\kvsetkeys{HoLogoDriver}{\HOLOGO@DriverFallback}%
    }\x
  }{}%
  \@PackageInfoNoLine{hologo}{Using driver `\hologoDriver'}%
}
%    \end{macrocode}
%    \end{macro}
%
%    \begin{macro}{\HOLOGO@CheckDriver}
%    \begin{macrocode}
\def\HOLOGO@CheckDriver{%
  \ifpdf
    \def\hologoDriver{pdftex}%
    \let\HOLOGO@pdfliteral\pdfliteral
    \ifluatex
      \ifx\pdfextension\@undefined\else
        \protected\def\pdfliteral{\pdfextension literal}%
        \let\HOLOGO@pdfliteral\pdfliteral
      \fi
      \ltx@IfUndefined{HOLOGO@pdfliteral}{%
        \ifnum\luatexversion<36 %
        \else
          \begingroup
            \let\HOLOGO@temp\endgroup
            \ifcase0%
                \directlua{%
                  if tex.enableprimitives then %
                    tex.enableprimitives('HOLOGO@', {'pdfliteral'})%
                  else %
                    tex.print('1')%
                  end%
                }%
                \ifx\HOLOGO@pdfliteral\@undefined 1\fi%
                \relax%
              \endgroup
              \let\HOLOGO@temp\relax
              \global\let\HOLOGO@pdfliteral\HOLOGO@pdfliteral
            \fi%
          \HOLOGO@temp
        \fi
      }{}%
    \fi
    \ltx@IfUndefined{HOLOGO@pdfliteral}{%
      \@PackageWarningNoLine{hologo}{%
        Cannot find \string\pdfliteral
      }%
    }{}%
  \else
    \ifxetex
      \def\hologoDriver{xetex}%
    \else
      \ifvtex
        \def\hologoDriver{vtex}%
      \fi
    \fi
  \fi
}
%    \end{macrocode}
%    \end{macro}
%
%    \begin{macro}{\HOLOGO@WarningUnsupportedDriver}
%    \begin{macrocode}
\def\HOLOGO@WarningUnsupportedDriver#1{%
  \@PackageWarningNoLine{hologo}{%
    Logo `#1' needs driver specific macros,\MessageBreak
    but driver `\hologoDriver' is not supported.\MessageBreak
    Use a different driver or\MessageBreak
    load package `graphics' or `pgf'%
  }%
}
%    \end{macrocode}
%    \end{macro}
%
% \subsubsection{Reflect box macros}
%
%    Skip driver part if not needed.
%    \begin{macrocode}
\ltx@IfUndefined{reflectbox}{}{%
  \ltx@IfUndefined{rotatebox}{}{%
    \HOLOGO@AtEnd
  }%
}
\ltx@IfUndefined{pgftext}{}{%
  \HOLOGO@AtEnd
}
\ltx@IfUndefined{psscalebox}{}{%
  \HOLOGO@AtEnd
}
%    \end{macrocode}
%
%    \begin{macrocode}
\def\HOLOGO@temp{LaTeX2e}
\ifx\fmtname\HOLOGO@temp
  \RequirePackage{kvoptions}[2011/06/30]%
  \ProcessKeyvalOptions{HoLogoDriver}%
\fi
\HOLOGO@DriverSetup{}
%    \end{macrocode}
%
%    \begin{macro}{\HOLOGO@ReflectBox}
%    \begin{macrocode}
\def\HOLOGO@ReflectBox#1{%
  \begingroup
    \setbox\ltx@zero\hbox{\begingroup#1\endgroup}%
    \setbox\ltx@two\hbox{%
      \kern\wd\ltx@zero
      \csname HOLOGO@ScaleBox@\hologoDriver\endcsname{-1}{1}{%
        \hbox to 0pt{\copy\ltx@zero\hss}%
      }%
    }%
    \wd\ltx@two=\wd\ltx@zero
    \box\ltx@two
  \endgroup
}
%    \end{macrocode}
%    \end{macro}
%
%    \begin{macro}{\HOLOGO@PointReflectBox}
%    \begin{macrocode}
\def\HOLOGO@PointReflectBox#1{%
  \begingroup
    \setbox\ltx@zero\hbox{\begingroup#1\endgroup}%
    \setbox\ltx@two\hbox{%
      \kern\wd\ltx@zero
      \raise\ht\ltx@zero\hbox{%
        \csname HOLOGO@ScaleBox@\hologoDriver\endcsname{-1}{-1}{%
          \hbox to 0pt{\copy\ltx@zero\hss}%
        }%
      }%
    }%
    \wd\ltx@two=\wd\ltx@zero
    \box\ltx@two
  \endgroup
}
%    \end{macrocode}
%    \end{macro}
%
%    We must define all variants because of dynamic driver setup.
%    \begin{macrocode}
\def\HOLOGO@temp#1#2{#2}
%    \end{macrocode}
%
%    \begin{macro}{\HOLOGO@ScaleBox@pdftex}
%    \begin{macrocode}
\HOLOGO@temp{pdftex}{%
  \def\HOLOGO@ScaleBox@pdftex#1#2#3{%
    \HOLOGO@pdfliteral{%
      q #1 0 0 #2 0 0 cm%
    }%
    #3%
    \HOLOGO@pdfliteral{%
      Q%
    }%
  }%
}
%    \end{macrocode}
%    \end{macro}
%    \begin{macro}{\HOLOGO@ScaleBox@dvips}
%    \begin{macrocode}
\HOLOGO@temp{dvips}{%
  \def\HOLOGO@ScaleBox@dvips#1#2#3{%
    \special{ps:%
      gsave %
      currentpoint %
      currentpoint translate %
      #1 #2 scale %
      neg exch neg exch translate%
    }%
    #3%
    \special{ps:%
      currentpoint %
      grestore %
      moveto%
    }%
  }%
}
%    \end{macrocode}
%    \end{macro}
%    \begin{macro}{\HOLOGO@ScaleBox@dvipdfm}
%    \begin{macrocode}
\HOLOGO@temp{dvipdfm}{%
  \let\HOLOGO@ScaleBox@dvipdfm\HOLOGO@ScaleBox@dvips
}
%    \end{macrocode}
%    \end{macro}
%    Since \hologo{XeTeX} v0.6.
%    \begin{macro}{\HOLOGO@ScaleBox@xetex}
%    \begin{macrocode}
\HOLOGO@temp{xetex}{%
  \def\HOLOGO@ScaleBox@xetex#1#2#3{%
    \special{x:gsave}%
    \special{x:scale #1 #2}%
    #3%
    \special{x:grestore}%
  }%
}
%    \end{macrocode}
%    \end{macro}
%    \begin{macro}{\HOLOGO@ScaleBox@vtex}
%    \begin{macrocode}
\HOLOGO@temp{vtex}{%
  \def\HOLOGO@ScaleBox@vtex#1#2#3{%
    \special{r(#1,0,0,#2,0,0}%
    #3%
    \special{r)}%
  }%
}
%    \end{macrocode}
%    \end{macro}
%
%    \begin{macrocode}
\HOLOGO@AtEnd%
%</package>
%    \end{macrocode}
%
% \section{Test}
%
% \subsection{Catcode checks for loading}
%
%    \begin{macrocode}
%<*test1>
%    \end{macrocode}
%    \begin{macrocode}
\catcode`\{=1 %
\catcode`\}=2 %
\catcode`\#=6 %
\catcode`\@=11 %
\expandafter\ifx\csname count@\endcsname\relax
  \countdef\count@=255 %
\fi
\expandafter\ifx\csname @gobble\endcsname\relax
  \long\def\@gobble#1{}%
\fi
\expandafter\ifx\csname @firstofone\endcsname\relax
  \long\def\@firstofone#1{#1}%
\fi
\expandafter\ifx\csname loop\endcsname\relax
  \expandafter\@firstofone
\else
  \expandafter\@gobble
\fi
{%
  \def\loop#1\repeat{%
    \def\body{#1}%
    \iterate
  }%
  \def\iterate{%
    \body
      \let\next\iterate
    \else
      \let\next\relax
    \fi
    \next
  }%
  \let\repeat=\fi
}%
\def\RestoreCatcodes{}
\count@=0 %
\loop
  \edef\RestoreCatcodes{%
    \RestoreCatcodes
    \catcode\the\count@=\the\catcode\count@\relax
  }%
\ifnum\count@<255 %
  \advance\count@ 1 %
\repeat

\def\RangeCatcodeInvalid#1#2{%
  \count@=#1\relax
  \loop
    \catcode\count@=15 %
  \ifnum\count@<#2\relax
    \advance\count@ 1 %
  \repeat
}
\def\RangeCatcodeCheck#1#2#3{%
  \count@=#1\relax
  \loop
    \ifnum#3=\catcode\count@
    \else
      \errmessage{%
        Character \the\count@\space
        with wrong catcode \the\catcode\count@\space
        instead of \number#3%
      }%
    \fi
  \ifnum\count@<#2\relax
    \advance\count@ 1 %
  \repeat
}
\def\space{ }
\expandafter\ifx\csname LoadCommand\endcsname\relax
  \def\LoadCommand{\input hologo.sty\relax}%
\fi
\def\Test{%
  \RangeCatcodeInvalid{0}{47}%
  \RangeCatcodeInvalid{58}{64}%
  \RangeCatcodeInvalid{91}{96}%
  \RangeCatcodeInvalid{123}{255}%
  \catcode`\@=12 %
  \catcode`\\=0 %
  \catcode`\%=14 %
  \LoadCommand
  \RangeCatcodeCheck{0}{36}{15}%
  \RangeCatcodeCheck{37}{37}{14}%
  \RangeCatcodeCheck{38}{47}{15}%
  \RangeCatcodeCheck{48}{57}{12}%
  \RangeCatcodeCheck{58}{63}{15}%
  \RangeCatcodeCheck{64}{64}{12}%
  \RangeCatcodeCheck{65}{90}{11}%
  \RangeCatcodeCheck{91}{91}{15}%
  \RangeCatcodeCheck{92}{92}{0}%
  \RangeCatcodeCheck{93}{96}{15}%
  \RangeCatcodeCheck{97}{122}{11}%
  \RangeCatcodeCheck{123}{255}{15}%
  \RestoreCatcodes
}
\Test
\csname @@end\endcsname
\end
%    \end{macrocode}
%    \begin{macrocode}
%</test1>
%    \end{macrocode}
%
% \subsection{Spacefactor}
%
%    The space factor must be 1000 after a logo. If it is greater 1000
%    then the following space is a space after a sentence closing point.
%    If the space factor is smaller 1000 then an immediate following
%    dot is interpreted as abbreviation, not sentence closing point.
%
%    \begin{macrocode}
%<*test-spacefactor>
\NeedsTeXFormat{LaTeX2e}
\documentclass{article}
\usepackage{hologo}[2016/05/12]
\usepackage{kvsetkeys}
\usepackage{qstest}
\IncludeTests{*}
\LogTests{log}{*}{*}
\begin{document}
\begin{qstest}{spacefactor}{spacefactor}
\newcommand*{\Test}[1]{%
  \sbox0{%
    \hologo{#1}%
    \Expect*{1000 (#1)}*{\the\spacefactor\space(#1)}%
  }%
}%
\makeatletter
\def\TestList{}
\def\hologoEntry#1#2#3{%
  \edef\TestList{%
    \ifx\TestList\@empty
    \else
      \TestList,%
    \fi
    #1%
    \ifx\\#2\\%
    \else
      ={variant=#2}%
    \fi
  }%
}
\hologoList
\expandafter\kv@parse@normalized\expandafter{%
  \TestList
}{%
  \begingroup
    \let\@logo=\kv@key
    \ifx\kv@value\relax
    \else
      \expandafter\hologoLogoSetup\expandafter\@logo\expandafter{%
        \kv@value
      }%
    \fi
    \Test\@logo
  \endgroup
  \@gobbletwo
}
\end{qstest}
\end{document}
%</test-spacefactor>
%    \end{macrocode}
%
% \subsection{Complete list}
%
%    \begin{macrocode}
%<*test-list>
\NeedsTeXFormat{LaTeX2e}
\documentclass[12pt,a4paper]{article}
\usepackage{hologo}[2016/05/12]
\usepackage[T1]{fontenc}
\usepackage{lmodern}
\usepackage{parskip}
\usepackage[unicode]{hyperref}[2011/09/28]
\usepackage{bookmark}[2011/09/19]
\bookmarksetup{%
  numbered,%
  open,%
  openlevel=2,%
}
\renewcommand*{\contentsname}{List of logos}
\begin{document}
\tableofcontents
\def\TestFont#1#2#3#4#5#6{%
  \begingroup
    \usefont{#3}{#4}{#5}{#6}%
    \HologoVariant{#1}{#2}/\hologoVariant{#1}{#2}%
    \quad
    \begingroup\scriptsize\hologoVariant{#1}{#2}\endgroup
    \quad
  \endgroup
  (#3/#4/#5/#6)%
  \par
}
\makeatletter
\def\hologoEntry#1#2#3{%
  \section{%
    \HologoVariant{#1}{#2}/\hologoVariant{#1}{#2} %
    {[#1\ifx\\#2\\\else\space(#2)\fi]}% hash-ok
  }% braces around [] because of bug in tex4ht
  \begingroup
    \hypersetup{unicode=false}%
    \bookmark[%
      dest=\@currentHref,%
      rellevel=1,%
      keeplevel,%
    ]{%
      \HologoVariant{#1}{#2}/\hologoVariant{#1}{#2} %
      (PDFDocEncoding)%
    }%
  \endgroup
  \TestFont{#1}{#2}{OT1}{cmr}{m}{n}%
  \TestFont{#1}{#2}{OT1}{cmss}{m}{n}%
  \TestFont{#1}{#2}{OT1}{cmr}{b}{n}%
  \TestFont{#1}{#2}{OT1}{cmr}{m}{it}%
  \TestFont{#1}{#2}{OT1}{cmtt}{m}{n}%
  \TestFont{#1}{#2}{T1}{lmr}{m}{n}%
  \TestFont{#1}{#2}{T1}{lmss}{m}{n}%
  \TestFont{#1}{#2}{T1}{lmr}{b}{n}%
  \TestFont{#1}{#2}{T1}{lmr}{m}{it}%
  \TestFont{#1}{#2}{T1}{lmtt}{m}{n}%
  \TestFont{#1}{#2}{T1}{lmvtt}{m}{n}%
  \TestFont{#1}{#2}{T1}{qtm}{m}{n}%
  \TestFont{#1}{#2}{T1}{qhv}{m}{n}%
  \TestFont{#1}{#2}{T1}{qtm}{b}{n}%
  \TestFont{#1}{#2}{T1}{qtm}{m}{it}%
  \TestFont{#1}{#2}{T1}{qcr}{m}{n}%
  \newpage
}
\makeatother
\hologoList
\end{document}
%</test-list>
%    \end{macrocode}
%
% \section{Installation}
%
% \subsection{Download}
%
% \paragraph{Package.} This package is available on
% CTAN\footnote{\url{ftp://ftp.ctan.org/tex-archive/}}:
% \begin{description}
% \item[\CTAN{macros/latex/contrib/oberdiek/hologo.dtx}] The source file.
% \item[\CTAN{macros/latex/contrib/oberdiek/hologo.pdf}] Documentation.
% \end{description}
%
%
% \paragraph{Bundle.} All the packages of the bundle `oberdiek'
% are also available in a TDS compliant ZIP archive. There
% the packages are already unpacked and the documentation files
% are generated. The files and directories obey the TDS standard.
% \begin{description}
% \item[\CTAN{install/macros/latex/contrib/oberdiek.tds.zip}]
% \end{description}
% \emph{TDS} refers to the standard ``A Directory Structure
% for \TeX\ Files'' (\CTAN{tds/tds.pdf}). Directories
% with \xfile{texmf} in their name are usually organized this way.
%
% \subsection{Bundle installation}
%
% \paragraph{Unpacking.} Unpack the \xfile{oberdiek.tds.zip} in the
% TDS tree (also known as \xfile{texmf} tree) of your choice.
% Example (linux):
% \begin{quote}
%   |unzip oberdiek.tds.zip -d ~/texmf|
% \end{quote}
%
% \paragraph{Script installation.}
% Check the directory \xfile{TDS:scripts/oberdiek/} for
% scripts that need further installation steps.
% Package \xpackage{attachfile2} comes with the Perl script
% \xfile{pdfatfi.pl} that should be installed in such a way
% that it can be called as \texttt{pdfatfi}.
% Example (linux):
% \begin{quote}
%   |chmod +x scripts/oberdiek/pdfatfi.pl|\\
%   |cp scripts/oberdiek/pdfatfi.pl /usr/local/bin/|
% \end{quote}
%
% \subsection{Package installation}
%
% \paragraph{Unpacking.} The \xfile{.dtx} file is a self-extracting
% \docstrip\ archive. The files are extracted by running the
% \xfile{.dtx} through \plainTeX:
% \begin{quote}
%   \verb|tex hologo.dtx|
% \end{quote}
%
% \paragraph{TDS.} Now the different files must be moved into
% the different directories in your installation TDS tree
% (also known as \xfile{texmf} tree):
% \begin{quote}
% \def\t{^^A
% \begin{tabular}{@{}>{\ttfamily}l@{ $\rightarrow$ }>{\ttfamily}l@{}}
%   hologo.sty & tex/generic/oberdiek/hologo.sty\\
%   hologo.pdf & doc/latex/oberdiek/hologo.pdf\\
%   example/hologo-example.tex & doc/latex/oberdiek/example/hologo-example.tex\\
%   test/hologo-test1.tex & doc/latex/oberdiek/test/hologo-test1.tex\\
%   test/hologo-test-spacefactor.tex & doc/latex/oberdiek/test/hologo-test-spacefactor.tex\\
%   test/hologo-test-list.tex & doc/latex/oberdiek/test/hologo-test-list.tex\\
%   hologo.dtx & source/latex/oberdiek/hologo.dtx\\
% \end{tabular}^^A
% }^^A
% \sbox0{\t}^^A
% \ifdim\wd0>\linewidth
%   \begingroup
%     \advance\linewidth by\leftmargin
%     \advance\linewidth by\rightmargin
%   \edef\x{\endgroup
%     \def\noexpand\lw{\the\linewidth}^^A
%   }\x
%   \def\lwbox{^^A
%     \leavevmode
%     \hbox to \linewidth{^^A
%       \kern-\leftmargin\relax
%       \hss
%       \usebox0
%       \hss
%       \kern-\rightmargin\relax
%     }^^A
%   }^^A
%   \ifdim\wd0>\lw
%     \sbox0{\small\t}^^A
%     \ifdim\wd0>\linewidth
%       \ifdim\wd0>\lw
%         \sbox0{\footnotesize\t}^^A
%         \ifdim\wd0>\linewidth
%           \ifdim\wd0>\lw
%             \sbox0{\scriptsize\t}^^A
%             \ifdim\wd0>\linewidth
%               \ifdim\wd0>\lw
%                 \sbox0{\tiny\t}^^A
%                 \ifdim\wd0>\linewidth
%                   \lwbox
%                 \else
%                   \usebox0
%                 \fi
%               \else
%                 \lwbox
%               \fi
%             \else
%               \usebox0
%             \fi
%           \else
%             \lwbox
%           \fi
%         \else
%           \usebox0
%         \fi
%       \else
%         \lwbox
%       \fi
%     \else
%       \usebox0
%     \fi
%   \else
%     \lwbox
%   \fi
% \else
%   \usebox0
% \fi
% \end{quote}
% If you have a \xfile{docstrip.cfg} that configures and enables \docstrip's
% TDS installing feature, then some files can already be in the right
% place, see the documentation of \docstrip.
%
% \subsection{Refresh file name databases}
%
% If your \TeX~distribution
% (\teTeX, \mikTeX, \dots) relies on file name databases, you must refresh
% these. For example, \teTeX\ users run \verb|texhash| or
% \verb|mktexlsr|.
%
% \subsection{Some details for the interested}
%
% \paragraph{Attached source.}
%
% The PDF documentation on CTAN also includes the
% \xfile{.dtx} source file. It can be extracted by
% AcrobatReader 6 or higher. Another option is \textsf{pdftk},
% e.g. unpack the file into the current directory:
% \begin{quote}
%   \verb|pdftk hologo.pdf unpack_files output .|
% \end{quote}
%
% \paragraph{Unpacking with \LaTeX.}
% The \xfile{.dtx} chooses its action depending on the format:
% \begin{description}
% \item[\plainTeX:] Run \docstrip\ and extract the files.
% \item[\LaTeX:] Generate the documentation.
% \end{description}
% If you insist on using \LaTeX\ for \docstrip\ (really,
% \docstrip\ does not need \LaTeX), then inform the autodetect routine
% about your intention:
% \begin{quote}
%   \verb|latex \let\install=y\input{hologo.dtx}|
% \end{quote}
% Do not forget to quote the argument according to the demands
% of your shell.
%
% \paragraph{Generating the documentation.}
% You can use both the \xfile{.dtx} or the \xfile{.drv} to generate
% the documentation. The process can be configured by the
% configuration file \xfile{ltxdoc.cfg}. For instance, put this
% line into this file, if you want to have A4 as paper format:
% \begin{quote}
%   \verb|\PassOptionsToClass{a4paper}{article}|
% \end{quote}
% An example follows how to generate the
% documentation with pdf\LaTeX:
% \begin{quote}
%\begin{verbatim}
%pdflatex hologo.dtx
%makeindex -s gind.ist hologo.idx
%pdflatex hologo.dtx
%makeindex -s gind.ist hologo.idx
%pdflatex hologo.dtx
%\end{verbatim}
% \end{quote}
%
% \section{Catalogue}
%
% The following XML file can be used as source for the
% \href{http://mirror.ctan.org/help/Catalogue/catalogue.html}{\TeX\ Catalogue}.
% The elements \texttt{caption} and \texttt{description} are imported
% from the original XML file from the Catalogue.
% The name of the XML file in the Catalogue is \xfile{hologo.xml}.
%    \begin{macrocode}
%<*catalogue>
<?xml version='1.0' encoding='us-ascii'?>
<!DOCTYPE entry SYSTEM 'catalogue.dtd'>
<entry datestamp='$Date$' modifier='$Author$' id='hologo'>
  <name>hologo</name>
  <caption>A collection of logos with bookmark support.</caption>
  <authorref id='auth:oberdiek'/>
  <copyright owner='Heiko Oberdiek' year='2010-2012'/>
  <license type='lppl1.3'/>
  <version number='1.10'/>
  <description>
    The package defines a single command <tt>\hologo</tt>, whose
    argument is the usual case-confused ASCII version of the logo.
    The command is bookmark-enabled, so that every logo becomes
    available in bookmarks without further work.
    <p/>
    The package is part of the <xref refid='oberdiek'>oberdiek</xref>
    bundle.
  </description>
  <documentation details='Package documentation'
      href='ctan:/macros/latex/contrib/oberdiek/hologo.pdf'/>
  <ctan file='true' path='/macros/latex/contrib/oberdiek/hologo.dtx'/>
  <miktex location='oberdiek'/>
  <texlive location='oberdiek'/>
  <install path='/macros/latex/contrib/oberdiek/oberdiek.tds.zip'/>
</entry>
%</catalogue>
%    \end{macrocode}
%
% \begin{thebibliography}{9}
% \raggedright
%
% \bibitem{btxdoc}
% Oren Patashnik,
% \textit{\hologo{BibTeX}ing},
% 1988-02-08.\\
% \CTAN{biblio/bibtex/base/}
%
% \bibitem{dtklogos}
% Gerd Neugebauer, DANTE,
% \textit{Package \xpackage{dtklogos}},
% 2011-04-25.\\
% \CTAN{usergrps/dante/dtk/dtklogos.sty}
%
% \bibitem{etexman}
% The \hologo{NTS} Team,
% \textit{The \hologo{eTeX} manual},
% 1998-02.\\
% \CTAN{systems/e-tex/v2/doc/}
%
% \bibitem{ExTeX-FAQ}
% The \hologo{ExTeX} group,
% \textit{\hologo{ExTeX}: FAQ -- How is \hologo{ExTeX} typeset?},
% 2007-04-14.\\
% \url{http://www.extex.org/documentation/faq.html}
%
% \bibitem{LyX}
% %@MISC{ LyX,
% %  title = {{LyX 2.0.0 -- The Document Processor [Computer software and manual]}},
% %  author = {{The LyX Team}},
% %  howpublished = {Internet: http://www.lyx.org},
% %  year = {2011-05-08},
% %  note = {Retrieved May 10, 2011, from http://www.lyx.org},
% %  url = {http://www.lyx.org/}
% %}
% The \hologo{LyX} Team,
% \textit{\hologo{LyX} -- The Document Processor},
% 2011-05-08.\\
% \url{http://www.lyx.org/}
%
% \bibitem{OzTeX}
% Andrew Trevorrow,
% \hologo{OzTeX} FAQ: What is the correct way to typeset ``\hologo{OzTeX}''?,
% 2011-09-15 (visited).
% \url{http://www.trevorrow.com/oztex/ozfaq.html#oztex-logo}
%
% \bibitem{PiCTeX}
% Michael Wichura,
% \textit{The \hologo{PiCTeX} macro package},
% 1987-09-21.
% \CTAN{graphics/pictex/}
%
% \bibitem{scrlogo}
% Markus Kohm,
% \textit{\hologo{KOMAScript} Datei \xfile{scrlogo.dtx}},
% 2009-01-30.\\
% \CTAN{install/macros/latex/contrib/komascript.tds.zip}
%
% \end{thebibliography}
%
% \begin{History}
%   \begin{Version}{2010/04/08 v1.0}
%   \item
%     The first version.
%   \end{Version}
%   \begin{Version}{2010/04/16 v1.1}
%   \item
%     \cs{Hologo} added for support of logos at start of a sentence.
%   \item
%     \cs{hologoSetup} and \cs{hologoLogoSetup} added.
%   \item
%     Options \xoption{break}, \xoption{hyphenbreak}, \xoption{spacebreak}
%     added.
%   \item
%     Variant support added by option \xoption{variant}.
%   \end{Version}
%   \begin{Version}{2010/04/24 v1.2}
%   \item
%     \hologo{LaTeX3} added.
%   \item
%     \hologo{VTeX} added.
%   \end{Version}
%   \begin{Version}{2010/11/21 v1.3}
%   \item
%     \hologo{iniTeX}, \hologo{virTeX} added.
%   \end{Version}
%   \begin{Version}{2011/03/25 v1.4}
%   \item
%     \hologo{ConTeXt} with variants added.
%   \item
%     Option \xoption{discretionarybreak} added as refinement for
%     option \xoption{break}.
%   \end{Version}
%   \begin{Version}{2011/04/21 v1.5}
%   \item
%     Wrong TDS directory for test files fixed.
%   \end{Version}
%   \begin{Version}{2011/10/01 v1.6}
%   \item
%     Support for package \xpackage{tex4ht} added.
%   \item
%     Support for \cs{csname} added if \cs{ifincsname} is available.
%   \item
%     New logos:
%     \hologo{(La)TeX},
%     \hologo{biber},
%     \hologo{BibTeX} (\xoption{sc}, \xoption{sf}),
%     \hologo{emTeX},
%     \hologo{ExTeX},
%     \hologo{KOMAScript},
%     \hologo{La},
%     \hologo{LyX},
%     \hologo{MiKTeX},
%     \hologo{NTS},
%     \hologo{OzMF},
%     \hologo{OzMP},
%     \hologo{OzTeX},
%     \hologo{OzTtH},
%     \hologo{PCTeX},
%     \hologo{PiC},
%     \hologo{PiCTeX},
%     \hologo{METAFONT},
%     \hologo{MetaFun},
%     \hologo{METAPOST},
%     \hologo{MetaPost},
%     \hologo{SLiTeX} (\xoption{lift}, \xoption{narrow}, \xoption{simple}),
%     \hologo{SliTeX} (\xoption{narrow}, \xoption{simple}, \xoption{lift}),
%     \hologo{teTeX}.
%   \item
%     Fixes:
%     \hologo{iniTeX},
%     \hologo{pdfLaTeX},
%     \hologo{pdfTeX},
%     \hologo{virTeX}.
%   \item
%     \cs{hologoFontSetup} and \cs{hologoLogoFontSetup} added.
%   \item
%     \cs{hologoVariant} and \cs{HologoVariant} added.
%   \end{Version}
%   \begin{Version}{2011/11/22 v1.7}
%   \item
%     New logos:
%     \hologo{BibTeX8},
%     \hologo{LaTeXML},
%     \hologo{SageTeX},
%     \hologo{TeX4ht},
%     \hologo{TTH}.
%   \item
%     \hologo{Xe} and friends: Driver stuff fixed.
%   \item
%     \hologo{Xe} and friends: Support for italic added.
%   \item
%     \hologo{Xe} and friends: Package support for \xpackage{pgf}
%     and \xpackage{pstricks} added.
%   \end{Version}
%   \begin{Version}{2011/11/29 v1.8}
%   \item
%     New logos:
%     \hologo{HanTheThanh}.
%   \end{Version}
%   \begin{Version}{2011/12/21 v1.9}
%   \item
%     Patch for package \xpackage{ifxetex} added for the case that
%     \cs{newif} is undefined in \hologo{iniTeX}.
%   \item
%     Some fixes for \hologo{iniTeX}.
%   \end{Version}
%   \begin{Version}{2012/04/26 v1.10}
%   \item
%     Fix in bookmark version of logo ``\hologo{HanTheThanh}''.
%   \end{Version}
%   \begin{Version}{2016/05/12 v1.11}
%   \item
%     Update HOLOGO@IfCharExists (previously in texlive)
%   \item define pdfliteral in current luatex.
%   \end{Version}
% \end{History}
%
% \PrintIndex
%
% \Finale
\endinput

%        (quote the arguments according to the demands of your shell)
%
% Documentation:
%    (a) If hologo.drv is present:
%           latex hologo.drv
%    (b) Without hologo.drv:
%           latex hologo.dtx; ...
%    The class ltxdoc loads the configuration file ltxdoc.cfg
%    if available. Here you can specify further options, e.g.
%    use A4 as paper format:
%       \PassOptionsToClass{a4paper}{article}
%
%    Programm calls to get the documentation (example):
%       pdflatex hologo.dtx
%       makeindex -s gind.ist hologo.idx
%       pdflatex hologo.dtx
%       makeindex -s gind.ist hologo.idx
%       pdflatex hologo.dtx
%
% Installation:
%    TDS:tex/generic/oberdiek/hologo.sty
%    TDS:doc/latex/oberdiek/hologo.pdf
%    TDS:doc/latex/oberdiek/example/hologo-example.tex
%    TDS:doc/latex/oberdiek/test/hologo-test1.tex
%    TDS:doc/latex/oberdiek/test/hologo-test-spacefactor.tex
%    TDS:doc/latex/oberdiek/test/hologo-test-list.tex
%    TDS:source/latex/oberdiek/hologo.dtx
%
%<*ignore>
\begingroup
  \catcode123=1 %
  \catcode125=2 %
  \def\x{LaTeX2e}%
\expandafter\endgroup
\ifcase 0\ifx\install y1\fi\expandafter
         \ifx\csname processbatchFile\endcsname\relax\else1\fi
         \ifx\fmtname\x\else 1\fi\relax
\else\csname fi\endcsname
%</ignore>
%<*install>
\input docstrip.tex
\Msg{************************************************************************}
\Msg{* Installation}
\Msg{* Package: hologo 2016/05/12 v1.11 A logo collection with bookmark support (HO)}
\Msg{************************************************************************}

\keepsilent
\askforoverwritefalse

\let\MetaPrefix\relax
\preamble

This is a generated file.

Project: hologo
Version: 2016/05/12 v1.11

Copyright (C) 2010-2012 by
   Heiko Oberdiek <heiko.oberdiek at googlemail.com>

This work may be distributed and/or modified under the
conditions of the LaTeX Project Public License, either
version 1.3c of this license or (at your option) any later
version. This version of this license is in
   http://www.latex-project.org/lppl/lppl-1-3c.txt
and the latest version of this license is in
   http://www.latex-project.org/lppl.txt
and version 1.3 or later is part of all distributions of
LaTeX version 2005/12/01 or later.

This work has the LPPL maintenance status "maintained".

This Current Maintainer of this work is Heiko Oberdiek.

The Base Interpreter refers to any `TeX-Format',
because some files are installed in TDS:tex/generic//.

This work consists of the main source file hologo.dtx
and the derived files
   hologo.sty, hologo.pdf, hologo.ins, hologo.drv, hologo-example.tex,
   hologo-test1.tex, hologo-test-spacefactor.tex,
   hologo-test-list.tex.

\endpreamble
\let\MetaPrefix\DoubleperCent

\generate{%
  \file{hologo.ins}{\from{hologo.dtx}{install}}%
  \file{hologo.drv}{\from{hologo.dtx}{driver}}%
  \usedir{tex/generic/oberdiek}%
  \file{hologo.sty}{\from{hologo.dtx}{package}}%
  \usedir{doc/latex/oberdiek/example}%
  \file{hologo-example.tex}{\from{hologo.dtx}{example}}%
  \usedir{doc/latex/oberdiek/test}%
  \file{hologo-test1.tex}{\from{hologo.dtx}{test1}}%
  \file{hologo-test-spacefactor.tex}{\from{hologo.dtx}{test-spacefactor}}%
  \file{hologo-test-list.tex}{\from{hologo.dtx}{test-list}}%
  \nopreamble
  \nopostamble
  \usedir{source/latex/oberdiek/catalogue}%
  \file{hologo.xml}{\from{hologo.dtx}{catalogue}}%
}

\catcode32=13\relax% active space
\let =\space%
\Msg{************************************************************************}
\Msg{*}
\Msg{* To finish the installation you have to move the following}
\Msg{* file into a directory searched by TeX:}
\Msg{*}
\Msg{*     hologo.sty}
\Msg{*}
\Msg{* To produce the documentation run the file `hologo.drv'}
\Msg{* through LaTeX.}
\Msg{*}
\Msg{* Happy TeXing!}
\Msg{*}
\Msg{************************************************************************}

\endbatchfile
%</install>
%<*ignore>
\fi
%</ignore>
%<*driver>
\NeedsTeXFormat{LaTeX2e}
\ProvidesFile{hologo.drv}%
  [2016/05/12 v1.11 A logo collection with bookmark support (HO)]%
\documentclass{ltxdoc}
\usepackage{holtxdoc}[2011/11/22]
\usepackage{hologo}[2016/05/12]
\usepackage{longtable}
\usepackage{array}
\usepackage{paralist}
%\usepackage[T1]{fontenc}
%\usepackage{lmodern}
\begin{document}
  \DocInput{hologo.dtx}%
\end{document}
%</driver>
% \fi
%
%
% \CharacterTable
%  {Upper-case    \A\B\C\D\E\F\G\H\I\J\K\L\M\N\O\P\Q\R\S\T\U\V\W\X\Y\Z
%   Lower-case    \a\b\c\d\e\f\g\h\i\j\k\l\m\n\o\p\q\r\s\t\u\v\w\x\y\z
%   Digits        \0\1\2\3\4\5\6\7\8\9
%   Exclamation   \!     Double quote  \"     Hash (number) \#
%   Dollar        \$     Percent       \%     Ampersand     \&
%   Acute accent  \'     Left paren    \(     Right paren   \)
%   Asterisk      \*     Plus          \+     Comma         \,
%   Minus         \-     Point         \.     Solidus       \/
%   Colon         \:     Semicolon     \;     Less than     \<
%   Equals        \=     Greater than  \>     Question mark \?
%   Commercial at \@     Left bracket  \[     Backslash     \\
%   Right bracket \]     Circumflex    \^     Underscore    \_
%   Grave accent  \`     Left brace    \{     Vertical bar  \|
%   Right brace   \}     Tilde         \~}
%
% \GetFileInfo{hologo.drv}
%
% \title{The \xpackage{hologo} package}
% \date{2016/05/12 v1.11}
% \author{Heiko Oberdiek\\\xemail{heiko.oberdiek at googlemail.com}}
%
% \maketitle
%
% \begin{abstract}
% This package starts a collection of logos with support for bookmarks
% strings.
% \end{abstract}
%
% \tableofcontents
%
% \section{Documentation}
%
% \subsection{Logo macros}
%
% \begin{declcs}{hologo} \M{name}
% \end{declcs}
% Macro \cs{hologo} sets the logo with name \meta{name}.
% The following table shows the supported names.
%
% \begingroup
%   \def\hologoEntry#1#2#3{^^A
%     #1&#2&\hologoLogoSetup{#1}{variant=#2}\hologo{#1}&#3\tabularnewline
%   }
%   \begin{longtable}{>{\ttfamily}l>{\ttfamily}lll}
%     \rmfamily\bfseries{name} & \rmfamily\bfseries variant
%     & \bfseries logo & \bfseries since\\
%     \hline
%     \endhead
%     \hologoList
%   \end{longtable}
% \endgroup
%
% \begin{declcs}{Hologo} \M{name}
% \end{declcs}
% Macro \cs{Hologo} starts the logo \meta{name} with an uppercase
% letter. As an exception small greek letters are not converted
% to uppercase. Examples, see \hologo{eTeX} and \hologo{ExTeX}.
%
% \subsection{Setup macros}
%
% The package does not support package options, but the following
% setup macros can be used to set options.
%
% \begin{declcs}{hologoSetup} \M{key value list}
% \end{declcs}
% Macro \cs{hologoSetup} sets global options.
%
% \begin{declcs}{hologoLogoSetup} \M{logo} \M{key value list}
% \end{declcs}
% Some options can also be used to configure a logo.
% These settings take precedence over global option settings.
%
% \subsection{Options}\label{sec:options}
%
% There are boolean and string options:
% \begin{description}
% \item[Boolean option:]
% It takes |true| or |false|
% as value. If the value is omitted, then |true| is used.
% \item[String option:]
% A value must be given as string. (But the string might be empty.)
% \end{description}
% The following options can be used both in \cs{hologoSetup}
% and \cs{hologoLogoSetup}:
% \begin{description}
% \def\entry#1{\item[\xoption{#1}:]}
% \entry{break}
%   enables or disables line breaks inside the logo. This setting is
%   refined by options \xoption{hyphenbreak}, \xoption{spacebreak}
%   or \xoption{discretionarybreak}.
%   Default is |false|.
% \entry{hyphenbreak}
%   enables or disables the line break right after the hyphen character.
% \entry{spacebreak}
%   enables or disables line breaks at space characters.
% \entry{discretionarybreak}
%   enables or disables line breaks at hyphenation points
%   (inserted by \cs{-}).
% \end{description}
% Macro \cs{hologoLogoSetup} also knows:
% \begin{description}
% \item[\xoption{variant}:]
%   This is a string option. It specifies a variant of a logo that
%   must exist. An empty string selects the package default variant.
% \end{description}
% Example:
% \begin{quote}
%   |\hologoSetup{break=false}|\\
%   |\hologoLogoSetup{plainTeX}{variant=hyphen,hyphenbreak}|\\
%   Then ``plain-\TeX'' contains one break point after the hyphen.
% \end{quote}
%
% \subsection{Driver options}
%
% Sometimes graphical operations are needed to construct some
% glyphs (e.g.\ \hologo{XeTeX}). If package \xpackage{graphics}
% or package \xpackage{pgf} are found, then the macros are taken
% from there. Otherwise the packge defines its own operations
% and therefore needs the driver information. Many drivers are
% detected automatically (\hologo{pdfTeX}/\hologo{LuaTeX}
% in PDF mode, \hologo{XeTeX}, \hologo{VTeX}). These have precedence
% over a driver option. The driver can be given as package option
% or using \cs{hologoDriverSetup}.
% The following list contains the recognized driver options:
% \begin{itemize}
% \item \xoption{pdftex}, \xoption{luatex}
% \item \xoption{dvipdfm}, \xoption{dvipdfmx}
% \item \xoption{dvips}, \xoption{dvipsone}, \xoption{xdvi}
% \item \xoption{xetex}
% \item \xoption{vtex}
% \end{itemize}
% The left driver of a line is the driver name that is used internally.
% The following names are aliases for drivers that use the
% same method. Therefore the entry in the \xext{log} file for
% the used driver prints the internally used driver name.
% \begin{description}
% \item[\xoption{driverfallback}:]
%   This option expects a driver that is used,
%   if the driver could not be detected automatically.
% \end{description}
%
% \begin{declcs}{hologoDriverSetup} \M{driver option}
% \end{declcs}
% The driver can also be configured after package loading
% using \cs{hologoDriverSetup}, also the way for \hologo{plainTeX}
% to setup the driver.
%
% \subsection{Font setup}
%
% Some logos require a special font, but should also be usable by
% \hologo{plainTeX}. Therefore the package provides some ways
% to influence the font settings. The options below
% take font settings as values. Both font commands
% such as \cs{sffamily} and macros that take one argument
% like \cs{textsf} can be used.
%
% \begin{declcs}{hologoFontSetup} \M{key value list}
% \end{declcs}
% Macro \cs{hologoFontSetup} sets the fonts for all logos.
% Supported keys:
% \begin{description}
% \def\entry#1{\item[\xoption{#1}:]}
% \entry{general}
%   This font is used for all logos. The default is empty.
%   That means no special font is used.
% \entry{bibsf}
%   This font is used for
%   {\hologoLogoSetup{BibTeX}{variant=sf}\hologo{BibTeX}}
%   with variant \xoption{sf}.
% \entry{rm}
%   This font is a serif font. It is used for \hologo{ExTeX}.
% \entry{sc}
%   This font specifies a small caps font. It is used for
%   {\hologoLogoSetup{BibTeX}{variant=sc}\hologo{BibTeX}}
%   with variant \xoption{sc}.
% \entry{sf}
%   This font specifies a sans serif font. The default
%   is \cs{sffamily}, then \cs{sf} is tried. Otherwise
%   a warning is given. It is used by \hologo{KOMAScript}.
% \entry{sy}
%   This is the font for math symbols (e.g. cmsy).
%   It is used by \hologo{AmS}, \hologo{NTS}, \hologo{ExTeX}.
% \entry{logo}
%   \hologo{METAFONT} and \hologo{METAPOST} are using that font.
%   In \hologo{LaTeX} \cs{logofamily} is used and
%   the definitions of package \xpackage{mflogo} are used
%   if the package is not loaded.
%   Otherwise the \cs{tenlogo} is used and defined
%   if it does not already exists.
% \end{description}
%
% \begin{declcs}{hologoLogoFontSetup} \M{logo} \M{key value list}
% \end{declcs}
% Fonts can also be set for a logo or logo component separately,
% see the following list.
% The keys are the same as for \cs{hologoFontSetup}.
%
% \begin{longtable}{>{\ttfamily}l>{\sffamily}ll}
%   \meta{logo} & keys & result\\
%   \hline
%   \endhead
%   BibTeX & bibsf & {\hologoLogoSetup{BibTeX}{variant=sf}\hologo{BibTeX}}\\[.5ex]
%   BibTeX & sc & {\hologoLogoSetup{BibTeX}{variant=sc}\hologo{BibTeX}}\\[.5ex]
%   ExTeX & rm & \hologo{ExTeX}\\
%   SliTeX & rm & \hologo{SliTeX}\\[.5ex]
%   AmS & sy & \hologo{AmS}\\
%   ExTeX & sy & \hologo{ExTeX}\\
%   NTS & sy & \hologo{NTS}\\[.5ex]
%   KOMAScript & sf & \hologo{KOMAScript}\\[.5ex]
%   METAFONT & logo & \hologo{METAFONT}\\
%   METAPOST & logo & \hologo{METAPOST}\\[.5ex]
%   SliTeX & sc \hologo{SliTeX}
% \end{longtable}
%
% \subsubsection{Font order}
%
% For all logos the font \xoption{general} is applied first.
% Example:
%\begin{quote}
%|\hologoFontSetup{general=\color{red}}|
%\end{quote}
% will print red logos.
% Then if the font uses a special font \xoption{sf}, for example,
% the font is applied that is setup by \cs{hologoLogoFontSetup}.
% If this font is not setup, then the common font setup
% by \cs{hologoFontSetup} is used. Otherwise a warning is given,
% that there is no font configured.
%
% \subsection{Additional user macros}
%
% Usually a variant of a logo is configured by using
% \cs{hologoLogoSetup}, because it is bad style to mix
% different variants of the same logo in the same text.
% There the following macros are a convenience for testing.
%
% \begin{declcs}{hologoVariant} \M{name} \M{variant}\\
%   \cs{HologoVariant} \M{name} \M{variant}
% \end{declcs}
% Logo \meta{name} is set using \meta{variant} that specifies
% explicitely which variant of the macro is used. If the argument
% is empty, then the default form of the logo is used
% (configurable by \cs{hologoLogoSetup}).
%
% \cs{HologoVariant} is used if the logo is set in a context
% that needs an uppercase first letter (beginning of a sentence, \dots).
%
% \begin{declcs}{hologoList}\\
%   \cs{hologoEntry} \M{logo} \M{variant} \M{since}
% \end{declcs}
% Macro \cs{hologoList} contains all logos that are provided
% by the package including variants. The list consists of calls
% of \cs{hologoEntry} with three arguments starting with the
% logo name \meta{logo} and its variant \meta{variant}. An empty
% variant means the current default. Argument \meta{since} specifies
% with version of the package \xpackage{hologo} is needed to get
% the logo. If the logo is fixed, then the date gets updated.
% Therefore the date \meta{since} is not exactly the date of
% the first introduction, but rather the date of the latest fix.
%
% Before \cs{hologoList} can be used, macro \cs{hologoEntry} needs
% a definition. The example file in section \ref{sec:example}
% shows applications of \cs{hologoList}.
%
% \subsection{Supported contexts}
%
% Macros \cs{hologo} and friends support special contexts:
% \begin{itemize}
% \item \hologo{LaTeX}'s protection mechanism.
% \item Bookmarks of package \xpackage{hyperref}.
% \item Package \xpackage{tex4ht}.
% \item The macros can be used inside \cs{csname} constructs,
%   if \cs{ifincsname} is available (\hologo{pdfTeX}, \hologo{XeTeX},
%   \hologo{LuaTeX}).
% \end{itemize}
%
% \subsection{Example}
% \label{sec:example}
%
% The following example prints the logos in different fonts.
%    \begin{macrocode}
%<*example>
%<<verbatim
\NeedsTeXFormat{LaTeX2e}
\documentclass[a4paper]{article}
\usepackage[
  hmargin=20mm,
  vmargin=20mm,
]{geometry}
\pagestyle{empty}
\usepackage{hologo}[2016/05/12]
\usepackage{longtable}
\usepackage{array}
\setlength{\extrarowheight}{2pt}
\usepackage[T1]{fontenc}
\usepackage{lmodern}
\usepackage{pdflscape}
\usepackage[
  pdfencoding=auto,
]{hyperref}
\hypersetup{
  pdfauthor={Heiko Oberdiek},
  pdftitle={Example for package `hologo'},
  pdfsubject={Logos with fonts lmr, lmss, qtm, qpl, qhv},
}
\usepackage{bookmark}

% Print the logo list on the console

\begingroup
  \typeout{}%
  \typeout{*** Begin of logo list ***}%
  \newcommand*{\hologoEntry}[3]{%
    \typeout{#1 \ifx\\#2\\\else(#2) \fi[#3]}%
  }%
  \hologoList
  \typeout{*** End of logo list ***}%
  \typeout{}%
\endgroup

\begin{document}
\begin{landscape}

  \section{Example file for package `hologo'}

  % Table for font names

  \begin{longtable}{>{\bfseries}ll}
    \textbf{font} & \textbf{Font name}\\
    \hline
    lmr & Latin Modern Roman\\
    lmss & Latin Modern Sans\\
    qtm & \TeX\ Gyre Termes\\
    qhv & \TeX\ Gyre Heros\\
    qpl & \TeX\ Gyre Pagella\\
  \end{longtable}

  % Logo list with logos in different fonts

  \begingroup
    \newcommand*{\SetVariant}[2]{%
      \ifx\\#2\\%
      \else
        \hologoLogoSetup{#1}{variant=#2}%
      \fi
    }%
    \newcommand*{\hologoEntry}[3]{%
      \SetVariant{#1}{#2}%
      \raisebox{1em}[0pt][0pt]{\hypertarget{#1@#2}{}}%
      \bookmark[%
        dest={#1@#2},%
      ]{%
        #1\ifx\\#2\\\else\space(#2)\fi: \Hologo{#1}, \hologo{#1} %
        [Unicode]%
      }%
      \hypersetup{unicode=false}%
      \bookmark[%
        dest={#1@#2},%
      ]{%
        #1\ifx\\#2\\\else\space(#2)\fi: \Hologo{#1}, \hologo{#1} %
        [PDFDocEncoding]%
      }%
      \texttt{#1}%
      &%
      \texttt{#2}%
      &%
      \Hologo{#1}%
      &%
      \SetVariant{#1}{#2}%
      \hologo{#1}%
      &%
      \SetVariant{#1}{#2}%
      \fontfamily{qtm}\selectfont
      \hologo{#1}%
      &%
      \SetVariant{#1}{#2}%
      \fontfamily{qpl}\selectfont
      \hologo{#1}%
      &%
      \SetVariant{#1}{#2}%
      \textsf{\hologo{#1}}%
      &%
      \SetVariant{#1}{#2}%
      \fontfamily{qhv}\selectfont
      \hologo{#1}%
      \tabularnewline
    }%
    \begin{longtable}{llllllll}%
      \textbf{\textit{logo}} & \textbf{\textit{variant}} &
      \texttt{\string\Hologo} &
      \textbf{lmr} & \textbf{qtm} & \textbf{qpl} &
      \textbf{lmss} & \textbf{qhv}
      \tabularnewline
      \hline
      \endhead
      \hologoList
    \end{longtable}%
  \endgroup

\end{landscape}
\end{document}
%verbatim
%</example>
%    \end{macrocode}
%
% \StopEventually{
% }
%
% \section{Implementation}
%    \begin{macrocode}
%<*package>
%    \end{macrocode}
%    Reload check, especially if the package is not used with \LaTeX.
%    \begin{macrocode}
\begingroup\catcode61\catcode48\catcode32=10\relax%
  \catcode13=5 % ^^M
  \endlinechar=13 %
  \catcode35=6 % #
  \catcode39=12 % '
  \catcode44=12 % ,
  \catcode45=12 % -
  \catcode46=12 % .
  \catcode58=12 % :
  \catcode64=11 % @
  \catcode123=1 % {
  \catcode125=2 % }
  \expandafter\let\expandafter\x\csname ver@hologo.sty\endcsname
  \ifx\x\relax % plain-TeX, first loading
  \else
    \def\empty{}%
    \ifx\x\empty % LaTeX, first loading,
      % variable is initialized, but \ProvidesPackage not yet seen
    \else
      \expandafter\ifx\csname PackageInfo\endcsname\relax
        \def\x#1#2{%
          \immediate\write-1{Package #1 Info: #2.}%
        }%
      \else
        \def\x#1#2{\PackageInfo{#1}{#2, stopped}}%
      \fi
      \x{hologo}{The package is already loaded}%
      \aftergroup\endinput
    \fi
  \fi
\endgroup%
%    \end{macrocode}
%    Package identification:
%    \begin{macrocode}
\begingroup\catcode61\catcode48\catcode32=10\relax%
  \catcode13=5 % ^^M
  \endlinechar=13 %
  \catcode35=6 % #
  \catcode39=12 % '
  \catcode40=12 % (
  \catcode41=12 % )
  \catcode44=12 % ,
  \catcode45=12 % -
  \catcode46=12 % .
  \catcode47=12 % /
  \catcode58=12 % :
  \catcode64=11 % @
  \catcode91=12 % [
  \catcode93=12 % ]
  \catcode123=1 % {
  \catcode125=2 % }
  \expandafter\ifx\csname ProvidesPackage\endcsname\relax
    \def\x#1#2#3[#4]{\endgroup
      \immediate\write-1{Package: #3 #4}%
      \xdef#1{#4}%
    }%
  \else
    \def\x#1#2[#3]{\endgroup
      #2[{#3}]%
      \ifx#1\@undefined
        \xdef#1{#3}%
      \fi
      \ifx#1\relax
        \xdef#1{#3}%
      \fi
    }%
  \fi
\expandafter\x\csname ver@hologo.sty\endcsname
\ProvidesPackage{hologo}%
  [2016/05/12 v1.11 A logo collection with bookmark support (HO)]%
%    \end{macrocode}
%
%    \begin{macrocode}
\begingroup\catcode61\catcode48\catcode32=10\relax%
  \catcode13=5 % ^^M
  \endlinechar=13 %
  \catcode123=1 % {
  \catcode125=2 % }
  \catcode64=11 % @
  \def\x{\endgroup
    \expandafter\edef\csname HOLOGO@AtEnd\endcsname{%
      \endlinechar=\the\endlinechar\relax
      \catcode13=\the\catcode13\relax
      \catcode32=\the\catcode32\relax
      \catcode35=\the\catcode35\relax
      \catcode61=\the\catcode61\relax
      \catcode64=\the\catcode64\relax
      \catcode123=\the\catcode123\relax
      \catcode125=\the\catcode125\relax
    }%
  }%
\x\catcode61\catcode48\catcode32=10\relax%
\catcode13=5 % ^^M
\endlinechar=13 %
\catcode35=6 % #
\catcode64=11 % @
\catcode123=1 % {
\catcode125=2 % }
\def\TMP@EnsureCode#1#2{%
  \edef\HOLOGO@AtEnd{%
    \HOLOGO@AtEnd
    \catcode#1=\the\catcode#1\relax
  }%
  \catcode#1=#2\relax
}
\TMP@EnsureCode{10}{12}% ^^J
\TMP@EnsureCode{33}{12}% !
\TMP@EnsureCode{34}{12}% "
\TMP@EnsureCode{36}{3}% $
\TMP@EnsureCode{38}{4}% &
\TMP@EnsureCode{39}{12}% '
\TMP@EnsureCode{40}{12}% (
\TMP@EnsureCode{41}{12}% )
\TMP@EnsureCode{42}{12}% *
\TMP@EnsureCode{43}{12}% +
\TMP@EnsureCode{44}{12}% ,
\TMP@EnsureCode{45}{12}% -
\TMP@EnsureCode{46}{12}% .
\TMP@EnsureCode{47}{12}% /
\TMP@EnsureCode{58}{12}% :
\TMP@EnsureCode{59}{12}% ;
\TMP@EnsureCode{60}{12}% <
\TMP@EnsureCode{62}{12}% >
\TMP@EnsureCode{63}{12}% ?
\TMP@EnsureCode{91}{12}% [
\TMP@EnsureCode{93}{12}% ]
\TMP@EnsureCode{94}{7}% ^ (superscript)
\TMP@EnsureCode{95}{8}% _ (subscript)
\TMP@EnsureCode{96}{12}% `
\TMP@EnsureCode{124}{12}% |
\edef\HOLOGO@AtEnd{%
  \HOLOGO@AtEnd
  \escapechar\the\escapechar\relax
  \noexpand\endinput
}
\escapechar=92 %
%    \end{macrocode}
%
% \subsection{Logo list}
%
%    \begin{macro}{\hologoList}
%    \begin{macrocode}
\def\hologoList{%
  \hologoEntry{(La)TeX}{}{2011/10/01}%
  \hologoEntry{AmSLaTeX}{}{2010/04/16}%
  \hologoEntry{AmSTeX}{}{2010/04/16}%
  \hologoEntry{biber}{}{2011/10/01}%
  \hologoEntry{BibTeX}{}{2011/10/01}%
  \hologoEntry{BibTeX}{sf}{2011/10/01}%
  \hologoEntry{BibTeX}{sc}{2011/10/01}%
  \hologoEntry{BibTeX8}{}{2011/11/22}%
  \hologoEntry{ConTeXt}{}{2011/03/25}%
  \hologoEntry{ConTeXt}{narrow}{2011/03/25}%
  \hologoEntry{ConTeXt}{simple}{2011/03/25}%
  \hologoEntry{emTeX}{}{2010/04/26}%
  \hologoEntry{eTeX}{}{2010/04/08}%
  \hologoEntry{ExTeX}{}{2011/10/01}%
  \hologoEntry{HanTheThanh}{}{2011/11/29}%
  \hologoEntry{iniTeX}{}{2011/10/01}%
  \hologoEntry{KOMAScript}{}{2011/10/01}%
  \hologoEntry{La}{}{2010/05/08}%
  \hologoEntry{LaTeX}{}{2010/04/08}%
  \hologoEntry{LaTeX2e}{}{2010/04/08}%
  \hologoEntry{LaTeX3}{}{2010/04/24}%
  \hologoEntry{LaTeXe}{}{2010/04/08}%
  \hologoEntry{LaTeXML}{}{2011/11/22}%
  \hologoEntry{LaTeXTeX}{}{2011/10/01}%
  \hologoEntry{LuaLaTeX}{}{2010/04/08}%
  \hologoEntry{LuaTeX}{}{2010/04/08}%
  \hologoEntry{LyX}{}{2011/10/01}%
  \hologoEntry{METAFONT}{}{2011/10/01}%
  \hologoEntry{MetaFun}{}{2011/10/01}%
  \hologoEntry{METAPOST}{}{2011/10/01}%
  \hologoEntry{MetaPost}{}{2011/10/01}%
  \hologoEntry{MiKTeX}{}{2011/10/01}%
  \hologoEntry{NTS}{}{2011/10/01}%
  \hologoEntry{OzMF}{}{2011/10/01}%
  \hologoEntry{OzMP}{}{2011/10/01}%
  \hologoEntry{OzTeX}{}{2011/10/01}%
  \hologoEntry{OzTtH}{}{2011/10/01}%
  \hologoEntry{PCTeX}{}{2011/10/01}%
  \hologoEntry{pdfTeX}{}{2011/10/01}%
  \hologoEntry{pdfLaTeX}{}{2011/10/01}%
  \hologoEntry{PiC}{}{2011/10/01}%
  \hologoEntry{PiCTeX}{}{2011/10/01}%
  \hologoEntry{plainTeX}{}{2010/04/08}%
  \hologoEntry{plainTeX}{space}{2010/04/16}%
  \hologoEntry{plainTeX}{hyphen}{2010/04/16}%
  \hologoEntry{plainTeX}{runtogether}{2010/04/16}%
  \hologoEntry{SageTeX}{}{2011/11/22}%
  \hologoEntry{SLiTeX}{}{2011/10/01}%
  \hologoEntry{SLiTeX}{lift}{2011/10/01}%
  \hologoEntry{SLiTeX}{narrow}{2011/10/01}%
  \hologoEntry{SLiTeX}{simple}{2011/10/01}%
  \hologoEntry{SliTeX}{}{2011/10/01}%
  \hologoEntry{SliTeX}{narrow}{2011/10/01}%
  \hologoEntry{SliTeX}{simple}{2011/10/01}%
  \hologoEntry{SliTeX}{lift}{2011/10/01}%
  \hologoEntry{teTeX}{}{2011/10/01}%
  \hologoEntry{TeX}{}{2010/04/08}%
  \hologoEntry{TeX4ht}{}{2011/11/22}%
  \hologoEntry{TTH}{}{2011/11/22}%
  \hologoEntry{virTeX}{}{2011/10/01}%
  \hologoEntry{VTeX}{}{2010/04/24}%
  \hologoEntry{Xe}{}{2010/04/08}%
  \hologoEntry{XeLaTeX}{}{2010/04/08}%
  \hologoEntry{XeTeX}{}{2010/04/08}%
}
%    \end{macrocode}
%    \end{macro}
%
% \subsection{Load resources}
%
%    \begin{macrocode}
\begingroup\expandafter\expandafter\expandafter\endgroup
\expandafter\ifx\csname RequirePackage\endcsname\relax
  \def\TMP@RequirePackage#1[#2]{%
    \begingroup\expandafter\expandafter\expandafter\endgroup
    \expandafter\ifx\csname ver@#1.sty\endcsname\relax
      \input #1.sty\relax
    \fi
  }%
  \TMP@RequirePackage{ltxcmds}[2011/02/04]%
  \TMP@RequirePackage{infwarerr}[2010/04/08]%
  \TMP@RequirePackage{kvsetkeys}[2010/03/01]%
  \TMP@RequirePackage{kvdefinekeys}[2010/03/01]%
  \TMP@RequirePackage{pdftexcmds}[2010/04/01]%
  \TMP@RequirePackage{ifpdf}[2010/01/28]%
  \TMP@RequirePackage{ifluatex}[2010/03/01]%
  \ltx@IfUndefined{newif}{%
    \expandafter\let\csname newif\endcsname\ltx@newif
  }{}%
  \TMP@RequirePackage{ifxetex}[2009/01/23]%
  \TMP@RequirePackage{ifvtex}[2010/03/01]%
\else
  \RequirePackage{ltxcmds}[2011/02/04]%
  \RequirePackage{infwarerr}[2010/04/08]%
  \RequirePackage{kvsetkeys}[2010/03/01]%
  \RequirePackage{kvdefinekeys}[2010/03/01]%
  \RequirePackage{pdftexcmds}[2010/04/01]%
  \RequirePackage{ifpdf}[2010/01/28]%
  \RequirePackage{ifluatex}[2010/03/01]%
  \RequirePackage{ifxetex}[2009/01/23]%
  \RequirePackage{ifvtex}[2010/03/01]%
\fi
%    \end{macrocode}
%
%    \begin{macro}{\HOLOGO@IfDefined}
%    \begin{macrocode}
\def\HOLOGO@IfExists#1{%
  \ifx\@undefined#1%
    \expandafter\ltx@secondoftwo
  \else
    \ifx\relax#1%
      \expandafter\ltx@secondoftwo
    \else
      \expandafter\expandafter\expandafter\ltx@firstoftwo
    \fi
  \fi
}
%    \end{macrocode}
%    \end{macro}
%
% \subsection{Setup macros}
%
%    \begin{macro}{\hologoSetup}
%    \begin{macrocode}
\def\hologoSetup{%
  \let\HOLOGO@name\relax
  \HOLOGO@Setup
}
%    \end{macrocode}
%    \end{macro}
%
%    \begin{macro}{\hologoLogoSetup}
%    \begin{macrocode}
\def\hologoLogoSetup#1{%
  \edef\HOLOGO@name{#1}%
  \ltx@IfUndefined{HoLogo@\HOLOGO@name}{%
    \@PackageError{hologo}{%
      Unknown logo `\HOLOGO@name'%
    }\@ehc
    \ltx@gobble
  }{%
    \HOLOGO@Setup
  }%
}
%    \end{macrocode}
%    \end{macro}
%
%    \begin{macro}{\HOLOGO@Setup}
%    \begin{macrocode}
\def\HOLOGO@Setup{%
  \kvsetkeys{HoLogo}%
}
%    \end{macrocode}
%    \end{macro}
%
% \subsection{Options}
%
%    \begin{macro}{\HOLOGO@DeclareBoolOption}
%    \begin{macrocode}
\def\HOLOGO@DeclareBoolOption#1{%
  \expandafter\chardef\csname HOLOGOOPT@#1\endcsname\ltx@zero
  \kv@define@key{HoLogo}{#1}[true]{%
    \def\HOLOGO@temp{##1}%
    \ifx\HOLOGO@temp\HOLOGO@true
      \ifx\HOLOGO@name\relax
        \expandafter\chardef\csname HOLOGOOPT@#1\endcsname=\ltx@one
      \else
        \expandafter\chardef\csname
        HoLogoOpt@#1@\HOLOGO@name\endcsname\ltx@one
      \fi
      \HOLOGO@SetBreakAll{#1}%
    \else
      \ifx\HOLOGO@temp\HOLOGO@false
        \ifx\HOLOGO@name\relax
          \expandafter\chardef\csname HOLOGOOPT@#1\endcsname=\ltx@zero
        \else
          \expandafter\chardef\csname
          HoLogoOpt@#1@\HOLOGO@name\endcsname=\ltx@zero
        \fi
        \HOLOGO@SetBreakAll{#1}%
      \else
        \@PackageError{hologo}{%
          Unknown value `##1' for boolean option `#1'.\MessageBreak
          Known values are `true' and `false'%
        }\@ehc
      \fi
    \fi
  }%
}
%    \end{macrocode}
%    \end{macro}
%
%    \begin{macro}{\HOLOGO@SetBreakAll}
%    \begin{macrocode}
\def\HOLOGO@SetBreakAll#1{%
  \def\HOLOGO@temp{#1}%
  \ifx\HOLOGO@temp\HOLOGO@break
    \ifx\HOLOGO@name\relax
      \chardef\HOLOGOOPT@hyphenbreak=\HOLOGOOPT@break
      \chardef\HOLOGOOPT@spacebreak=\HOLOGOOPT@break
      \chardef\HOLOGOOPT@discretionarybreak=\HOLOGOOPT@break
    \else
      \expandafter\chardef
         \csname HoLogoOpt@hyphenbreak@\HOLOGO@name\endcsname=%
         \csname HoLogoOpt@break@\HOLOGO@name\endcsname
      \expandafter\chardef
         \csname HoLogoOpt@spacebreak@\HOLOGO@name\endcsname=%
         \csname HoLogoOpt@break@\HOLOGO@name\endcsname
      \expandafter\chardef
         \csname HoLogoOpt@discretionarybreak@\HOLOGO@name
             \endcsname=%
         \csname HoLogoOpt@break@\HOLOGO@name\endcsname
    \fi
  \fi
}
%    \end{macrocode}
%    \end{macro}
%
%    \begin{macro}{\HOLOGO@true}
%    \begin{macrocode}
\def\HOLOGO@true{true}
%    \end{macrocode}
%    \end{macro}
%    \begin{macro}{\HOLOGO@false}
%    \begin{macrocode}
\def\HOLOGO@false{false}
%    \end{macrocode}
%    \end{macro}
%    \begin{macro}{\HOLOGO@break}
%    \begin{macrocode}
\def\HOLOGO@break{break}
%    \end{macrocode}
%    \end{macro}
%
%    \begin{macrocode}
\HOLOGO@DeclareBoolOption{break}
\HOLOGO@DeclareBoolOption{hyphenbreak}
\HOLOGO@DeclareBoolOption{spacebreak}
\HOLOGO@DeclareBoolOption{discretionarybreak}
%    \end{macrocode}
%
%    \begin{macrocode}
\kv@define@key{HoLogo}{variant}{%
  \ifx\HOLOGO@name\relax
    \@PackageError{hologo}{%
      Option `variant' is not available in \string\hologoSetup,%
      \MessageBreak
      Use \string\hologoLogoSetup\space instead%
    }\@ehc
  \else
    \edef\HOLOGO@temp{#1}%
    \ifx\HOLOGO@temp\ltx@empty
      \expandafter
      \let\csname HoLogoOpt@variant@\HOLOGO@name\endcsname\@undefined
    \else
      \ltx@IfUndefined{HoLogo@\HOLOGO@name @\HOLOGO@temp}{%
        \@PackageError{hologo}{%
          Unknown variant `\HOLOGO@temp' of logo `\HOLOGO@name'%
        }\@ehc
      }{%
        \expandafter
        \let\csname HoLogoOpt@variant@\HOLOGO@name\endcsname
            \HOLOGO@temp
      }%
    \fi
  \fi
}
%    \end{macrocode}
%
%    \begin{macro}{\HOLOGO@Variant}
%    \begin{macrocode}
\def\HOLOGO@Variant#1{%
  #1%
  \ltx@ifundefined{HoLogoOpt@variant@#1}{%
  }{%
    @\csname HoLogoOpt@variant@#1\endcsname
  }%
}
%    \end{macrocode}
%    \end{macro}
%
% \subsection{Break/no-break support}
%
%    \begin{macro}{\HOLOGO@space}
%    \begin{macrocode}
\def\HOLOGO@space{%
  \ltx@ifundefined{HoLogoOpt@spacebreak@\HOLOGO@name}{%
    \ltx@ifundefined{HoLogoOpt@break@\HOLOGO@name}{%
      \chardef\HOLOGO@temp=\HOLOGOOPT@spacebreak
    }{%
      \chardef\HOLOGO@temp=%
        \csname HoLogoOpt@break@\HOLOGO@name\endcsname
    }%
  }{%
    \chardef\HOLOGO@temp=%
      \csname HoLogoOpt@spacebreak@\HOLOGO@name\endcsname
  }%
  \ifcase\HOLOGO@temp
    \penalty10000 %
  \fi
  \ltx@space
}
%    \end{macrocode}
%    \end{macro}
%
%    \begin{macro}{\HOLOGO@hyphen}
%    \begin{macrocode}
\def\HOLOGO@hyphen{%
  \ltx@ifundefined{HoLogoOpt@hyphenbreak@\HOLOGO@name}{%
    \ltx@ifundefined{HoLogoOpt@break@\HOLOGO@name}{%
      \chardef\HOLOGO@temp=\HOLOGOOPT@hyphenbreak
    }{%
      \chardef\HOLOGO@temp=%
        \csname HoLogoOpt@break@\HOLOGO@name\endcsname
    }%
  }{%
    \chardef\HOLOGO@temp=%
      \csname HoLogoOpt@hyphenbreak@\HOLOGO@name\endcsname
  }%
  \ifcase\HOLOGO@temp
    \ltx@mbox{-}%
  \else
    -%
  \fi
}
%    \end{macrocode}
%    \end{macro}
%
%    \begin{macro}{\HOLOGO@discretionary}
%    \begin{macrocode}
\def\HOLOGO@discretionary{%
  \ltx@ifundefined{HoLogoOpt@discretionarybreak@\HOLOGO@name}{%
    \ltx@ifundefined{HoLogoOpt@break@\HOLOGO@name}{%
      \chardef\HOLOGO@temp=\HOLOGOOPT@discretionarybreak
    }{%
      \chardef\HOLOGO@temp=%
        \csname HoLogoOpt@break@\HOLOGO@name\endcsname
    }%
  }{%
    \chardef\HOLOGO@temp=%
      \csname HoLogoOpt@discretionarybreak@\HOLOGO@name\endcsname
  }%
  \ifcase\HOLOGO@temp
  \else
    \-%
  \fi
}
%    \end{macrocode}
%    \end{macro}
%
%    \begin{macro}{\HOLOGO@mbox}
%    \begin{macrocode}
\def\HOLOGO@mbox#1{%
  \ltx@ifundefined{HoLogoOpt@break@\HOLOGO@name}{%
    \chardef\HOLOGO@temp=\HOLOGOOPT@hyphenbreak
  }{%
    \chardef\HOLOGO@temp=%
      \csname HoLogoOpt@break@\HOLOGO@name\endcsname
  }%
  \ifcase\HOLOGO@temp
    \ltx@mbox{#1}%
  \else
    #1%
  \fi
}
%    \end{macrocode}
%    \end{macro}
%
% \subsection{Font support}
%
%    \begin{macro}{\HoLogoFont@font}
%    \begin{tabular}{@{}ll@{}}
%    |#1|:& logo name\\
%    |#2|:& font short name\\
%    |#3|:& text
%    \end{tabular}
%    \begin{macrocode}
\def\HoLogoFont@font#1#2#3{%
  \begingroup
    \ltx@IfUndefined{HoLogoFont@logo@#1.#2}{%
      \ltx@IfUndefined{HoLogoFont@font@#2}{%
        \@PackageWarning{hologo}{%
          Missing font `#2' for logo `#1'%
        }%
        #3%
      }{%
        \csname HoLogoFont@font@#2\endcsname{#3}%
      }%
    }{%
      \csname HoLogoFont@logo@#1.#2\endcsname{#3}%
    }%
  \endgroup
}
%    \end{macrocode}
%    \end{macro}
%
%    \begin{macro}{\HoLogoFont@Def}
%    \begin{macrocode}
\def\HoLogoFont@Def#1{%
  \expandafter\def\csname HoLogoFont@font@#1\endcsname
}
%    \end{macrocode}
%    \end{macro}
%    \begin{macro}{\HoLogoFont@LogoDef}
%    \begin{macrocode}
\def\HoLogoFont@LogoDef#1#2{%
  \expandafter\def\csname HoLogoFont@logo@#1.#2\endcsname
}
%    \end{macrocode}
%    \end{macro}
%
% \subsubsection{Font defaults}
%
%    \begin{macro}{\HoLogoFont@font@general}
%    \begin{macrocode}
\HoLogoFont@Def{general}{}%
%    \end{macrocode}
%    \end{macro}
%
%    \begin{macro}{\HoLogoFont@font@rm}
%    \begin{macrocode}
\ltx@IfUndefined{rmfamily}{%
  \ltx@IfUndefined{rm}{%
  }{%
    \HoLogoFont@Def{rm}{\rm}%
  }%
}{%
  \HoLogoFont@Def{rm}{\rmfamily}%
}
%    \end{macrocode}
%    \end{macro}
%
%    \begin{macro}{\HoLogoFont@font@sf}
%    \begin{macrocode}
\ltx@IfUndefined{sffamily}{%
  \ltx@IfUndefined{sf}{%
  }{%
    \HoLogoFont@Def{sf}{\sf}%
  }%
}{%
  \HoLogoFont@Def{sf}{\sffamily}%
}
%    \end{macrocode}
%    \end{macro}
%
%    \begin{macro}{\HoLogoFont@font@bibsf}
%    In case of \hologo{plainTeX} the original small caps
%    variant is used as default. In \hologo{LaTeX}
%    the definition of package \xpackage{dtklogos} \cite{dtklogos}
%    is used.
%\begin{quote}
%\begin{verbatim}
%\DeclareRobustCommand{\BibTeX}{%
%  B%
%  \kern-.05em%
%  \hbox{%
%    $\m@th$% %% force math size calculations
%    \csname S@\f@size\endcsname
%    \fontsize\sf@size\z@
%    \math@fontsfalse
%    \selectfont
%    I%
%    \kern-.025em%
%    B
%  }%
%  \kern-.08em%
%  \-%
%  \TeX
%}
%\end{verbatim}
%\end{quote}
%    \begin{macrocode}
\ltx@IfUndefined{selectfont}{%
  \ltx@IfUndefined{tensc}{%
    \font\tensc=cmcsc10\relax
  }{}%
  \HoLogoFont@Def{bibsf}{\tensc}%
}{%
  \HoLogoFont@Def{bibsf}{%
    $\mathsurround=0pt$%
    \csname S@\f@size\endcsname
    \fontsize\sf@size{0pt}%
    \math@fontsfalse
    \selectfont
  }%
}
%    \end{macrocode}
%    \end{macro}
%
%    \begin{macro}{\HoLogoFont@font@sc}
%    \begin{macrocode}
\ltx@IfUndefined{scshape}{%
  \ltx@IfUndefined{tensc}{%
    \font\tensc=cmcsc10\relax
  }{}%
  \HoLogoFont@Def{sc}{\tensc}%
}{%
  \HoLogoFont@Def{sc}{\scshape}%
}
%    \end{macrocode}
%    \end{macro}
%
%    \begin{macro}{\HoLogoFont@font@sy}
%    \begin{macrocode}
\ltx@IfUndefined{usefont}{%
  \ltx@IfUndefined{tensy}{%
  }{%
    \HoLogoFont@Def{sy}{\tensy}%
  }%
}{%
  \HoLogoFont@Def{sy}{%
    \usefont{OMS}{cmsy}{m}{n}%
  }%
}
%    \end{macrocode}
%    \end{macro}
%
%    \begin{macro}{\HoLogoFont@font@logo}
%    \begin{macrocode}
\begingroup
  \def\x{LaTeX2e}%
\expandafter\endgroup
\ifx\fmtname\x
  \ltx@IfUndefined{logofamily}{%
    \DeclareRobustCommand\logofamily{%
      \not@math@alphabet\logofamily\relax
      \fontencoding{U}%
      \fontfamily{logo}%
      \selectfont
    }%
  }{}%
  \ltx@IfUndefined{logofamily}{%
  }{%
    \HoLogoFont@Def{logo}{\logofamily}%
  }%
\else
  \ltx@IfUndefined{tenlogo}{%
    \font\tenlogo=logo10\relax
  }{}%
  \HoLogoFont@Def{logo}{\tenlogo}%
\fi
%    \end{macrocode}
%    \end{macro}
%
% \subsubsection{Font setup}
%
%    \begin{macro}{\hologoFontSetup}
%    \begin{macrocode}
\def\hologoFontSetup{%
  \let\HOLOGO@name\relax
  \HOLOGO@FontSetup
}
%    \end{macrocode}
%    \end{macro}
%
%    \begin{macro}{\hologoLogoFontSetup}
%    \begin{macrocode}
\def\hologoLogoFontSetup#1{%
  \edef\HOLOGO@name{#1}%
  \ltx@IfUndefined{HoLogo@\HOLOGO@name}{%
    \@PackageError{hologo}{%
      Unknown logo `\HOLOGO@name'%
    }\@ehc
    \ltx@gobble
  }{%
    \HOLOGO@FontSetup
  }%
}
%    \end{macrocode}
%    \end{macro}
%
%    \begin{macro}{\HOLOGO@FontSetup}
%    \begin{macrocode}
\def\HOLOGO@FontSetup{%
  \kvsetkeys{HoLogoFont}%
}
%    \end{macrocode}
%    \end{macro}
%
%    \begin{macrocode}
\def\HOLOGO@temp#1{%
  \kv@define@key{HoLogoFont}{#1}{%
    \ifx\HOLOGO@name\relax
      \HoLogoFont@Def{#1}{##1}%
    \else
      \HoLogoFont@LogoDef\HOLOGO@name{#1}{##1}%
    \fi
  }%
}
\HOLOGO@temp{general}
\HOLOGO@temp{sf}
%    \end{macrocode}
%
% \subsection{Generic logo commands}
%
%    \begin{macrocode}
\HOLOGO@IfExists\hologo{%
  \@PackageError{hologo}{%
    \string\hologo\ltx@space is already defined.\MessageBreak
    Package loading is aborted%
  }\@ehc
  \HOLOGO@AtEnd
}%
\HOLOGO@IfExists\hologoRobust{%
  \@PackageError{hologo}{%
    \string\hologoRobust\ltx@space is already defined.\MessageBreak
    Package loading is aborted%
  }\@ehc
  \HOLOGO@AtEnd
}%
%    \end{macrocode}
%
% \subsubsection{\cs{hologo} and friends}
%
%    \begin{macrocode}
\ifluatex
  \expandafter\ltx@firstofone
\else
  \expandafter\ltx@gobble
\fi
{%
  \ltx@IfUndefined{ifincsname}{%
    \ifnum\luatexversion<36 %
      \expandafter\ltx@gobble
    \else
      \expandafter\ltx@firstofone
    \fi
    {%
      \begingroup
        \ifcase0%
            \directlua{%
              if tex.enableprimitives then %
                tex.enableprimitives('HOLOGO@', {'ifincsname'})%
              else %
                tex.print('1')%
              end%
            }%
            \ifx\HOLOGO@ifincsname\@undefined 1\fi%
            \relax
          \expandafter\ltx@firstofone
        \else
          \endgroup
          \expandafter\ltx@gobble
        \fi
        {%
          \global\let\ifincsname\HOLOGO@ifincsname
        }%
      \HOLOGO@temp
    }%
  }{}%
}
%    \end{macrocode}
%    \begin{macrocode}
\ltx@IfUndefined{ifincsname}{%
  \catcode`$=14 %
}{%
  \catcode`$=9 %
}
%    \end{macrocode}
%
%    \begin{macro}{\hologo}
%    \begin{macrocode}
\def\hologo#1{%
$ \ifincsname
$   \ltx@ifundefined{HoLogoCs@\HOLOGO@Variant{#1}}{%
$     #1%
$   }{%
$     \csname HoLogoCs@\HOLOGO@Variant{#1}\endcsname\ltx@firstoftwo
$   }%
$ \else
    \HOLOGO@IfExists\texorpdfstring\texorpdfstring\ltx@firstoftwo
    {%
      \hologoRobust{#1}%
    }{%
      \ltx@ifundefined{HoLogoBkm@\HOLOGO@Variant{#1}}{%
        \ltx@ifundefined{HoLogo@#1}{?#1?}{#1}%
      }{%
        \csname HoLogoBkm@\HOLOGO@Variant{#1}\endcsname
        \ltx@firstoftwo
      }%
    }%
$ \fi
}
%    \end{macrocode}
%    \end{macro}
%    \begin{macro}{\Hologo}
%    \begin{macrocode}
\def\Hologo#1{%
$ \ifincsname
$   \ltx@ifundefined{HoLogoCs@\HOLOGO@Variant{#1}}{%
$     #1%
$   }{%
$     \csname HoLogoCs@\HOLOGO@Variant{#1}\endcsname\ltx@secondoftwo
$   }%
$ \else
    \HOLOGO@IfExists\texorpdfstring\texorpdfstring\ltx@firstoftwo
    {%
      \HologoRobust{#1}%
    }{%
      \ltx@ifundefined{HoLogoBkm@\HOLOGO@Variant{#1}}{%
        \ltx@ifundefined{HoLogo@#1}{?#1?}{#1}%
      }{%
        \csname HoLogoBkm@\HOLOGO@Variant{#1}\endcsname
        \ltx@secondoftwo
      }%
    }%
$ \fi
}
%    \end{macrocode}
%    \end{macro}
%
%    \begin{macro}{\hologoVariant}
%    \begin{macrocode}
\def\hologoVariant#1#2{%
  \ifx\relax#2\relax
    \hologo{#1}%
  \else
$   \ifincsname
$     \ltx@ifundefined{HoLogoCs@#1@#2}{%
$       #1%
$     }{%
$       \csname HoLogoCs@#1@#2\endcsname\ltx@firstoftwo
$     }%
$   \else
      \HOLOGO@IfExists\texorpdfstring\texorpdfstring\ltx@firstoftwo
      {%
        \hologoVariantRobust{#1}{#2}%
      }{%
        \ltx@ifundefined{HoLogoBkm@#1@#2}{%
          \ltx@ifundefined{HoLogo@#1}{?#1?}{#1}%
        }{%
          \csname HoLogoBkm@#1@#2\endcsname
          \ltx@firstoftwo
        }%
      }%
$   \fi
  \fi
}
%    \end{macrocode}
%    \end{macro}
%    \begin{macro}{\HologoVariant}
%    \begin{macrocode}
\def\HologoVariant#1#2{%
  \ifx\relax#2\relax
    \Hologo{#1}%
  \else
$   \ifincsname
$     \ltx@ifundefined{HoLogoCs@#1@#2}{%
$       #1%
$     }{%
$       \csname HoLogoCs@#1@#2\endcsname\ltx@secondoftwo
$     }%
$   \else
      \HOLOGO@IfExists\texorpdfstring\texorpdfstring\ltx@firstoftwo
      {%
        \HologoVariantRobust{#1}{#2}%
      }{%
        \ltx@ifundefined{HoLogoBkm@#1@#2}{%
          \ltx@ifundefined{HoLogo@#1}{?#1?}{#1}%
        }{%
          \csname HoLogoBkm@#1@#2\endcsname
          \ltx@secondoftwo
        }%
      }%
$   \fi
  \fi
}
%    \end{macrocode}
%    \end{macro}
%
%    \begin{macrocode}
\catcode`\$=3 %
%    \end{macrocode}
%
% \subsubsection{\cs{hologoRobust} and friends}
%
%    \begin{macro}{\hologoRobust}
%    \begin{macrocode}
\ltx@IfUndefined{protected}{%
  \ltx@IfUndefined{DeclareRobustCommand}{%
    \def\hologoRobust#1%
  }{%
    \DeclareRobustCommand*\hologoRobust[1]%
  }%
}{%
  \protected\def\hologoRobust#1%
}%
{%
  \edef\HOLOGO@name{#1}%
  \ltx@IfUndefined{HoLogo@\HOLOGO@Variant\HOLOGO@name}{%
    \@PackageError{hologo}{%
      Unknown logo `\HOLOGO@name'%
    }\@ehc
    ?\HOLOGO@name?%
  }{%
    \ltx@IfUndefined{ver@tex4ht.sty}{%
      \HoLogoFont@font\HOLOGO@name{general}{%
        \csname HoLogo@\HOLOGO@Variant\HOLOGO@name\endcsname
        \ltx@firstoftwo
      }%
    }{%
      \ltx@IfUndefined{HoLogoHtml@\HOLOGO@Variant\HOLOGO@name}{%
        \HOLOGO@name
      }{%
        \csname HoLogoHtml@\HOLOGO@Variant\HOLOGO@name\endcsname
        \ltx@firstoftwo
      }%
    }%
  }%
}
%    \end{macrocode}
%    \end{macro}
%    \begin{macro}{\HologoRobust}
%    \begin{macrocode}
\ltx@IfUndefined{protected}{%
  \ltx@IfUndefined{DeclareRobustCommand}{%
    \def\HologoRobust#1%
  }{%
    \DeclareRobustCommand*\HologoRobust[1]%
  }%
}{%
  \protected\def\HologoRobust#1%
}%
{%
  \edef\HOLOGO@name{#1}%
  \ltx@IfUndefined{HoLogo@\HOLOGO@Variant\HOLOGO@name}{%
    \@PackageError{hologo}{%
      Unknown logo `\HOLOGO@name'%
    }\@ehc
    ?\HOLOGO@name?%
  }{%
    \ltx@IfUndefined{ver@tex4ht.sty}{%
      \HoLogoFont@font\HOLOGO@name{general}{%
        \csname HoLogo@\HOLOGO@Variant\HOLOGO@name\endcsname
        \ltx@secondoftwo
      }%
    }{%
      \ltx@IfUndefined{HoLogoHtml@\HOLOGO@Variant\HOLOGO@name}{%
        \expandafter\HOLOGO@Uppercase\HOLOGO@name
      }{%
        \csname HoLogoHtml@\HOLOGO@Variant\HOLOGO@name\endcsname
        \ltx@secondoftwo
      }%
    }%
  }%
}
%    \end{macrocode}
%    \end{macro}
%    \begin{macro}{\hologoVariantRobust}
%    \begin{macrocode}
\ltx@IfUndefined{protected}{%
  \ltx@IfUndefined{DeclareRobustCommand}{%
    \def\hologoVariantRobust#1#2%
  }{%
    \DeclareRobustCommand*\hologoVariantRobust[2]%
  }%
}{%
  \protected\def\hologoVariantRobust#1#2%
}%
{%
  \begingroup
    \hologoLogoSetup{#1}{variant={#2}}%
    \hologoRobust{#1}%
  \endgroup
}
%    \end{macrocode}
%    \end{macro}
%    \begin{macro}{\HologoVariantRobust}
%    \begin{macrocode}
\ltx@IfUndefined{protected}{%
  \ltx@IfUndefined{DeclareRobustCommand}{%
    \def\HologoVariantRobust#1#2%
  }{%
    \DeclareRobustCommand*\HologoVariantRobust[2]%
  }%
}{%
  \protected\def\HologoVariantRobust#1#2%
}%
{%
  \begingroup
    \hologoLogoSetup{#1}{variant={#2}}%
    \HologoRobust{#1}%
  \endgroup
}
%    \end{macrocode}
%    \end{macro}
%
%    \begin{macro}{\hologorobust}
%    Macro \cs{hologorobust} is only defined for compatibility.
%    Its use is deprecated.
%    \begin{macrocode}
\def\hologorobust{\hologoRobust}
%    \end{macrocode}
%    \end{macro}
%
% \subsection{Helpers}
%
%    \begin{macro}{\HOLOGO@Uppercase}
%    Macro \cs{HOLOGO@Uppercase} is restricted to \cs{uppercase},
%    because \hologo{plainTeX} or \hologo{iniTeX} do not provide
%    \cs{MakeUppercase}.
%    \begin{macrocode}
\def\HOLOGO@Uppercase#1{\uppercase{#1}}
%    \end{macrocode}
%    \end{macro}
%
%    \begin{macro}{\HOLOGO@PdfdocUnicode}
%    \begin{macrocode}
\def\HOLOGO@PdfdocUnicode{%
  \ifx\ifHy@unicode\iftrue
    \expandafter\ltx@secondoftwo
  \else
    \expandafter\ltx@firstoftwo
  \fi
}
%    \end{macrocode}
%    \end{macro}
%
%    \begin{macro}{\HOLOGO@Math}
%    \begin{macrocode}
\def\HOLOGO@MathSetup{%
  \mathsurround0pt\relax
  \HOLOGO@IfExists\f@series{%
    \if b\expandafter\ltx@car\f@series x\@nil
      \csname boldmath\endcsname
   \fi
  }{}%
}
%    \end{macrocode}
%    \end{macro}
%
%    \begin{macro}{\HOLOGO@TempDimen}
%    \begin{macrocode}
\dimendef\HOLOGO@TempDimen=\ltx@zero
%    \end{macrocode}
%    \end{macro}
%    \begin{macro}{\HOLOGO@NegativeKerning}
%    \begin{macrocode}
\def\HOLOGO@NegativeKerning#1{%
  \begingroup
    \HOLOGO@TempDimen=0pt\relax
    \comma@parse@normalized{#1}{%
      \ifdim\HOLOGO@TempDimen=0pt %
        \expandafter\HOLOGO@@NegativeKerning\comma@entry
      \fi
      \ltx@gobble
    }%
    \ifdim\HOLOGO@TempDimen<0pt %
      \kern\HOLOGO@TempDimen
    \fi
  \endgroup
}
%    \end{macrocode}
%    \end{macro}
%    \begin{macro}{\HOLOGO@@NegativeKerning}
%    \begin{macrocode}
\def\HOLOGO@@NegativeKerning#1#2{%
  \setbox\ltx@zero\hbox{#1#2}%
  \HOLOGO@TempDimen=\wd\ltx@zero
  \setbox\ltx@zero\hbox{#1\kern0pt#2}%
  \advance\HOLOGO@TempDimen by -\wd\ltx@zero
}
%    \end{macrocode}
%    \end{macro}
%
%    \begin{macro}{\HOLOGO@SpaceFactor}
%    \begin{macrocode}
\def\HOLOGO@SpaceFactor{%
  \spacefactor1000 %
}
%    \end{macrocode}
%    \end{macro}
%
%    \begin{macro}{\HOLOGO@Span}
%    \begin{macrocode}
\def\HOLOGO@Span#1#2{%
  \HCode{<span class="HoLogo-#1">}%
  #2%
  \HCode{</span>}%
}
%    \end{macrocode}
%    \end{macro}
%
% \subsubsection{Text subscript}
%
%    \begin{macro}{\HOLOGO@SubScript}%
%    \begin{macrocode}
\def\HOLOGO@SubScript#1{%
  \ltx@IfUndefined{textsubscript}{%
    \ltx@IfUndefined{text}{%
      \ltx@mbox{%
        \mathsurround=0pt\relax
        $%
          _{%
            \ltx@IfUndefined{sf@size}{%
              \mathrm{#1}%
            }{%
              \mbox{%
                \fontsize\sf@size{0pt}\selectfont
                #1%
              }%
            }%
          }%
        $%
      }%
    }{%
      \ltx@mbox{%
        \mathsurround=0pt\relax
        $_{\text{#1}}$%
      }%
    }%
  }{%
    \textsubscript{#1}%
  }%
}
%    \end{macrocode}
%    \end{macro}
%
% \subsection{\hologo{TeX} and friends}
%
% \subsubsection{\hologo{TeX}}
%
%    \begin{macro}{\HoLogo@TeX}
%    Source: \hologo{LaTeX} kernel.
%    \begin{macrocode}
\def\HoLogo@TeX#1{%
  T\kern-.1667em\lower.5ex\hbox{E}\kern-.125emX\HOLOGO@SpaceFactor
}
%    \end{macrocode}
%    \end{macro}
%    \begin{macro}{\HoLogoHtml@TeX}
%    \begin{macrocode}
\def\HoLogoHtml@TeX#1{%
  \HoLogoCss@TeX
  \HOLOGO@Span{TeX}{%
    T%
    \HOLOGO@Span{e}{%
      E%
    }%
    X%
  }%
}
%    \end{macrocode}
%    \end{macro}
%    \begin{macro}{\HoLogoCss@TeX}
%    \begin{macrocode}
\def\HoLogoCss@TeX{%
  \Css{%
    span.HoLogo-TeX span.HoLogo-e{%
      position:relative;%
      top:.5ex;%
      margin-left:-.1667em;%
      margin-right:-.125em;%
    }%
  }%
  \Css{%
    a span.HoLogo-TeX span.HoLogo-e{%
      text-decoration:none;%
    }%
  }%
  \global\let\HoLogoCss@TeX\relax
}
%    \end{macrocode}
%    \end{macro}
%
% \subsubsection{\hologo{plainTeX}}
%
%    \begin{macro}{\HoLogo@plainTeX@space}
%    Source: ``The \hologo{TeX}book''
%    \begin{macrocode}
\def\HoLogo@plainTeX@space#1{%
  \HOLOGO@mbox{#1{p}{P}lain}\HOLOGO@space\hologo{TeX}%
}
%    \end{macrocode}
%    \end{macro}
%    \begin{macro}{\HoLogoCs@plainTeX@space}
%    \begin{macrocode}
\def\HoLogoCs@plainTeX@space#1{#1{p}{P}lain TeX}%
%    \end{macrocode}
%    \end{macro}
%    \begin{macro}{\HoLogoBkm@plainTeX@space}
%    \begin{macrocode}
\def\HoLogoBkm@plainTeX@space#1{%
  #1{p}{P}lain \hologo{TeX}%
}
%    \end{macrocode}
%    \end{macro}
%    \begin{macro}{\HoLogoHtml@plainTeX@space}
%    \begin{macrocode}
\def\HoLogoHtml@plainTeX@space#1{%
  #1{p}{P}lain \hologo{TeX}%
}
%    \end{macrocode}
%    \end{macro}
%
%    \begin{macro}{\HoLogo@plainTeX@hyphen}
%    \begin{macrocode}
\def\HoLogo@plainTeX@hyphen#1{%
  \HOLOGO@mbox{#1{p}{P}lain}\HOLOGO@hyphen\hologo{TeX}%
}
%    \end{macrocode}
%    \end{macro}
%    \begin{macro}{\HoLogoCs@plainTeX@hyphen}
%    \begin{macrocode}
\def\HoLogoCs@plainTeX@hyphen#1{#1{p}{P}lain-TeX}
%    \end{macrocode}
%    \end{macro}
%    \begin{macro}{\HoLogoBkm@plainTeX@hyphen}
%    \begin{macrocode}
\def\HoLogoBkm@plainTeX@hyphen#1{%
  #1{p}{P}lain-\hologo{TeX}%
}
%    \end{macrocode}
%    \end{macro}
%    \begin{macro}{\HoLogoHtml@plainTeX@hyphen}
%    \begin{macrocode}
\def\HoLogoHtml@plainTeX@hyphen#1{%
  #1{p}{P}lain-\hologo{TeX}%
}
%    \end{macrocode}
%    \end{macro}
%
%    \begin{macro}{\HoLogo@plainTeX@runtogether}
%    \begin{macrocode}
\def\HoLogo@plainTeX@runtogether#1{%
  \HOLOGO@mbox{#1{p}{P}lain\hologo{TeX}}%
}
%    \end{macrocode}
%    \end{macro}
%    \begin{macro}{\HoLogoCs@plainTeX@runtogether}
%    \begin{macrocode}
\def\HoLogoCs@plainTeX@runtogether#1{#1{p}{P}lainTeX}
%    \end{macrocode}
%    \end{macro}
%    \begin{macro}{\HoLogoBkm@plainTeX@runtogether}
%    \begin{macrocode}
\def\HoLogoBkm@plainTeX@runtogether#1{%
  #1{p}{P}lain\hologo{TeX}%
}
%    \end{macrocode}
%    \end{macro}
%    \begin{macro}{\HoLogoHtml@plainTeX@runtogether}
%    \begin{macrocode}
\def\HoLogoHtml@plainTeX@runtogether#1{%
  #1{p}{P}lain\hologo{TeX}%
}
%    \end{macrocode}
%    \end{macro}
%
%    \begin{macro}{\HoLogo@plainTeX}
%    \begin{macrocode}
\def\HoLogo@plainTeX{\HoLogo@plainTeX@space}
%    \end{macrocode}
%    \end{macro}
%    \begin{macro}{\HoLogoCs@plainTeX}
%    \begin{macrocode}
\def\HoLogoCs@plainTeX{\HoLogoCs@plainTeX@space}
%    \end{macrocode}
%    \end{macro}
%    \begin{macro}{\HoLogoBkm@plainTeX}
%    \begin{macrocode}
\def\HoLogoBkm@plainTeX{\HoLogoBkm@plainTeX@space}
%    \end{macrocode}
%    \end{macro}
%    \begin{macro}{\HoLogoHtml@plainTeX}
%    \begin{macrocode}
\def\HoLogoHtml@plainTeX{\HoLogoHtml@plainTeX@space}
%    \end{macrocode}
%    \end{macro}
%
% \subsubsection{\hologo{LaTeX}}
%
%    Source: \hologo{LaTeX} kernel.
%\begin{quote}
%\begin{verbatim}
%\DeclareRobustCommand{\LaTeX}{%
%  L%
%  \kern-.36em%
%  {%
%    \sbox\z@ T%
%    \vbox to\ht\z@{%
%      \hbox{%
%        \check@mathfonts
%        \fontsize\sf@size\z@
%        \math@fontsfalse
%        \selectfont
%        A%
%      }%
%      \vss
%    }%
%  }%
%  \kern-.15em%
%  \TeX
%}
%\end{verbatim}
%\end{quote}
%
%    \begin{macro}{\HoLogo@La}
%    \begin{macrocode}
\def\HoLogo@La#1{%
  L%
  \kern-.36em%
  \begingroup
    \setbox\ltx@zero\hbox{T}%
    \vbox to\ht\ltx@zero{%
      \hbox{%
        \ltx@ifundefined{check@mathfonts}{%
          \csname sevenrm\endcsname
        }{%
          \check@mathfonts
          \fontsize\sf@size{0pt}%
          \math@fontsfalse\selectfont
        }%
        A%
      }%
      \vss
    }%
  \endgroup
}
%    \end{macrocode}
%    \end{macro}
%
%    \begin{macro}{\HoLogo@LaTeX}
%    Source: \hologo{LaTeX} kernel.
%    \begin{macrocode}
\def\HoLogo@LaTeX#1{%
  \hologo{La}%
  \kern-.15em%
  \hologo{TeX}%
}
%    \end{macrocode}
%    \end{macro}
%    \begin{macro}{\HoLogoHtml@LaTeX}
%    \begin{macrocode}
\def\HoLogoHtml@LaTeX#1{%
  \HoLogoCss@LaTeX
  \HOLOGO@Span{LaTeX}{%
    L%
    \HOLOGO@Span{a}{%
      A%
    }%
    \hologo{TeX}%
  }%
}
%    \end{macrocode}
%    \end{macro}
%    \begin{macro}{\HoLogoCss@LaTeX}
%    \begin{macrocode}
\def\HoLogoCss@LaTeX{%
  \Css{%
    span.HoLogo-LaTeX span.HoLogo-a{%
      position:relative;%
      top:-.5ex;%
      margin-left:-.36em;%
      margin-right:-.15em;%
      font-size:85\%;%
    }%
  }%
  \global\let\HoLogoCss@LaTeX\relax
}
%    \end{macrocode}
%    \end{macro}
%
% \subsubsection{\hologo{(La)TeX}}
%
%    \begin{macro}{\HoLogo@LaTeXTeX}
%    The kerning around the parentheses is taken
%    from package \xpackage{dtklogos} \cite{dtklogos}.
%\begin{quote}
%\begin{verbatim}
%\DeclareRobustCommand{\LaTeXTeX}{%
%  (%
%  \kern-.15em%
%  L%
%  \kern-.36em%
%  {%
%    \sbox\z@ T%
%    \vbox to\ht0{%
%      \hbox{%
%        $\m@th$%
%        \csname S@\f@size\endcsname
%        \fontsize\sf@size\z@
%        \math@fontsfalse
%        \selectfont
%        A%
%      }%
%      \vss
%    }%
%  }%
%  \kern-.2em%
%  )%
%  \kern-.15em%
%  \TeX
%}
%\end{verbatim}
%\end{quote}
%    \begin{macrocode}
\def\HoLogo@LaTeXTeX#1{%
  (%
  \kern-.15em%
  \hologo{La}%
  \kern-.2em%
  )%
  \kern-.15em%
  \hologo{TeX}%
}
%    \end{macrocode}
%    \end{macro}
%    \begin{macro}{\HoLogoBkm@LaTeXTeX}
%    \begin{macrocode}
\def\HoLogoBkm@LaTeXTeX#1{(La)TeX}
%    \end{macrocode}
%    \end{macro}
%
%    \begin{macro}{\HoLogo@(La)TeX}
%    \begin{macrocode}
\expandafter
\let\csname HoLogo@(La)TeX\endcsname\HoLogo@LaTeXTeX
%    \end{macrocode}
%    \end{macro}
%    \begin{macro}{\HoLogoBkm@(La)TeX}
%    \begin{macrocode}
\expandafter
\let\csname HoLogoBkm@(La)TeX\endcsname\HoLogoBkm@LaTeXTeX
%    \end{macrocode}
%    \end{macro}
%    \begin{macro}{\HoLogoHtml@LaTeXTeX}
%    \begin{macrocode}
\def\HoLogoHtml@LaTeXTeX#1{%
  \HoLogoCss@LaTeXTeX
  \HOLOGO@Span{LaTeXTeX}{%
    (%
    \HOLOGO@Span{L}{L}%
    \HOLOGO@Span{a}{A}%
    \HOLOGO@Span{ParenRight}{)}%
    \hologo{TeX}%
  }%
}
%    \end{macrocode}
%    \end{macro}
%    \begin{macro}{\HoLogoHtml@(La)TeX}
%    Kerning after opening parentheses and before closing parentheses
%    is $-0.1$\,em. The original values $-0.15$\,em
%    looked too ugly for a serif font.
%    \begin{macrocode}
\expandafter
\let\csname HoLogoHtml@(La)TeX\endcsname\HoLogoHtml@LaTeXTeX
%    \end{macrocode}
%    \end{macro}
%    \begin{macro}{\HoLogoCss@LaTeXTeX}
%    \begin{macrocode}
\def\HoLogoCss@LaTeXTeX{%
  \Css{%
    span.HoLogo-LaTeXTeX span.HoLogo-L{%
      margin-left:-.1em;%
    }%
  }%
  \Css{%
    span.HoLogo-LaTeXTeX span.HoLogo-a{%
      position:relative;%
      top:-.5ex;%
      margin-left:-.36em;%
      margin-right:-.1em;%
      font-size:85\%;%
    }%
  }%
  \Css{%
    span.HoLogo-LaTeXTeX span.HoLogo-ParenRight{%
      margin-right:-.15em;%
    }%
  }%
  \global\let\HoLogoCss@LaTeXTeX\relax
}
%    \end{macrocode}
%    \end{macro}
%
% \subsubsection{\hologo{LaTeXe}}
%
%    \begin{macro}{\HoLogo@LaTeXe}
%    Source: \hologo{LaTeX} kernel
%    \begin{macrocode}
\def\HoLogo@LaTeXe#1{%
  \hologo{LaTeX}%
  \kern.15em%
  \hbox{%
    \HOLOGO@MathSetup
    2%
    $_{\textstyle\varepsilon}$%
  }%
}
%    \end{macrocode}
%    \end{macro}
%
%    \begin{macro}{\HoLogoCs@LaTeXe}
%    \begin{macrocode}
\ifnum64=`\^^^^0040\relax % test for big chars of LuaTeX/XeTeX
  \catcode`\$=9 %
  \catcode`\&=14 %
\else
  \catcode`\$=14 %
  \catcode`\&=9 %
\fi
\def\HoLogoCs@LaTeXe#1{%
  LaTeX2%
$ \string ^^^^0395%
& e%
}%
\catcode`\$=3 %
\catcode`\&=4 %
%    \end{macrocode}
%    \end{macro}
%
%    \begin{macro}{\HoLogoBkm@LaTeXe}
%    \begin{macrocode}
\def\HoLogoBkm@LaTeXe#1{%
  \hologo{LaTeX}%
  2%
  \HOLOGO@PdfdocUnicode{e}{\textepsilon}%
}
%    \end{macrocode}
%    \end{macro}
%
%    \begin{macro}{\HoLogoHtml@LaTeXe}
%    \begin{macrocode}
\def\HoLogoHtml@LaTeXe#1{%
  \HoLogoCss@LaTeXe
  \HOLOGO@Span{LaTeX2e}{%
    \hologo{LaTeX}%
    \HOLOGO@Span{2}{2}%
    \HOLOGO@Span{e}{%
      \HOLOGO@MathSetup
      \ensuremath{\textstyle\varepsilon}%
    }%
  }%
}
%    \end{macrocode}
%    \end{macro}
%    \begin{macro}{\HoLogoCss@LaTeXe}
%    \begin{macrocode}
\def\HoLogoCss@LaTeXe{%
  \Css{%
    span.HoLogo-LaTeX2e span.HoLogo-2{%
      padding-left:.15em;%
    }%
  }%
  \Css{%
    span.HoLogo-LaTeX2e span.HoLogo-e{%
      position:relative;%
      top:.35ex;%
      text-decoration:none;%
    }%
  }%
  \global\let\HoLogoCss@LaTeXe\relax
}
%    \end{macrocode}
%    \end{macro}
%
%    \begin{macro}{\HoLogo@LaTeX2e}
%    \begin{macrocode}
\expandafter
\let\csname HoLogo@LaTeX2e\endcsname\HoLogo@LaTeXe
%    \end{macrocode}
%    \end{macro}
%    \begin{macro}{\HoLogoCs@LaTeX2e}
%    \begin{macrocode}
\expandafter
\let\csname HoLogoCs@LaTeX2e\endcsname\HoLogoCs@LaTeXe
%    \end{macrocode}
%    \end{macro}
%    \begin{macro}{\HoLogoBkm@LaTeX2e}
%    \begin{macrocode}
\expandafter
\let\csname HoLogoBkm@LaTeX2e\endcsname\HoLogoBkm@LaTeXe
%    \end{macrocode}
%    \end{macro}
%    \begin{macro}{\HoLogoHtml@LaTeX2e}
%    \begin{macrocode}
\expandafter
\let\csname HoLogoHtml@LaTeX2e\endcsname\HoLogoHtml@LaTeXe
%    \end{macrocode}
%    \end{macro}
%
% \subsubsection{\hologo{LaTeX3}}
%
%    \begin{macro}{\HoLogo@LaTeX3}
%    Source: \hologo{LaTeX} kernel
%    \begin{macrocode}
\expandafter\def\csname HoLogo@LaTeX3\endcsname#1{%
  \hologo{LaTeX}%
  3%
}
%    \end{macrocode}
%    \end{macro}
%
%    \begin{macro}{\HoLogoBkm@LaTeX3}
%    \begin{macrocode}
\expandafter\def\csname HoLogoBkm@LaTeX3\endcsname#1{%
  \hologo{LaTeX}%
  3%
}
%    \end{macrocode}
%    \end{macro}
%    \begin{macro}{\HoLogoHtml@LaTeX3}
%    \begin{macrocode}
\expandafter
\let\csname HoLogoHtml@LaTeX3\expandafter\endcsname
\csname HoLogo@LaTeX3\endcsname
%    \end{macrocode}
%    \end{macro}
%
% \subsubsection{\hologo{LaTeXML}}
%
%    \begin{macro}{\HoLogo@LaTeXML}
%    \begin{macrocode}
\def\HoLogo@LaTeXML#1{%
  \HOLOGO@mbox{%
    \hologo{La}%
    \kern-.15em%
    T%
    \kern-.1667em%
    \lower.5ex\hbox{E}%
    \kern-.125em%
    \HoLogoFont@font{LaTeXML}{sc}{xml}%
  }%
}
%    \end{macrocode}
%    \end{macro}
%    \begin{macro}{\HoLogoHtml@pdfLaTeX}
%    \begin{macrocode}
\def\HoLogoHtml@LaTeXML#1{%
  \HOLOGO@Span{LaTeXML}{%
    \HoLogoCss@LaTeX
    \HoLogoCss@TeX
    \HOLOGO@Span{LaTeX}{%
      L%
      \HOLOGO@Span{a}{%
        A%
      }%
    }%
    \HOLOGO@Span{TeX}{%
      T%
      \HOLOGO@Span{e}{%
        E%
      }%
    }%
    \HCode{<span style="font-variant: small-caps;">}%
    xml%
    \HCode{</span>}%
  }%
}
%    \end{macrocode}
%    \end{macro}
%
% \subsubsection{\hologo{eTeX}}
%
%    \begin{macro}{\HoLogo@eTeX}
%    Source: package \xpackage{etex}
%    \begin{macrocode}
\def\HoLogo@eTeX#1{%
  \ltx@mbox{%
    \HOLOGO@MathSetup
    $\varepsilon$%
    -%
    \HOLOGO@NegativeKerning{-T,T-,To}%
    \hologo{TeX}%
  }%
}
%    \end{macrocode}
%    \end{macro}
%    \begin{macro}{\HoLogoCs@eTeX}
%    \begin{macrocode}
\ifnum64=`\^^^^0040\relax % test for big chars of LuaTeX/XeTeX
  \catcode`\$=9 %
  \catcode`\&=14 %
\else
  \catcode`\$=14 %
  \catcode`\&=9 %
\fi
\def\HoLogoCs@eTeX#1{%
$ #1{\string ^^^^0395}{\string ^^^^03b5}%
& #1{e}{E}%
  TeX%
}%
\catcode`\$=3 %
\catcode`\&=4 %
%    \end{macrocode}
%    \end{macro}
%    \begin{macro}{\HoLogoBkm@eTeX}
%    \begin{macrocode}
\def\HoLogoBkm@eTeX#1{%
  \HOLOGO@PdfdocUnicode{#1{e}{E}}{\textepsilon}%
  -%
  \hologo{TeX}%
}
%    \end{macrocode}
%    \end{macro}
%    \begin{macro}{\HoLogoHtml@eTeX}
%    \begin{macrocode}
\def\HoLogoHtml@eTeX#1{%
  \ltx@mbox{%
    \HOLOGO@MathSetup
    $\varepsilon$%
    -%
    \hologo{TeX}%
  }%
}
%    \end{macrocode}
%    \end{macro}
%
% \subsubsection{\hologo{iniTeX}}
%
%    \begin{macro}{\HoLogo@iniTeX}
%    \begin{macrocode}
\def\HoLogo@iniTeX#1{%
  \HOLOGO@mbox{%
    #1{i}{I}ni\hologo{TeX}%
  }%
}
%    \end{macrocode}
%    \end{macro}
%    \begin{macro}{\HoLogoCs@iniTeX}
%    \begin{macrocode}
\def\HoLogoCs@iniTeX#1{#1{i}{I}niTeX}
%    \end{macrocode}
%    \end{macro}
%    \begin{macro}{\HoLogoBkm@iniTeX}
%    \begin{macrocode}
\def\HoLogoBkm@iniTeX#1{%
  #1{i}{I}ni\hologo{TeX}%
}
%    \end{macrocode}
%    \end{macro}
%    \begin{macro}{\HoLogoHtml@iniTeX}
%    \begin{macrocode}
\let\HoLogoHtml@iniTeX\HoLogo@iniTeX
%    \end{macrocode}
%    \end{macro}
%
% \subsubsection{\hologo{virTeX}}
%
%    \begin{macro}{\HoLogo@virTeX}
%    \begin{macrocode}
\def\HoLogo@virTeX#1{%
  \HOLOGO@mbox{%
    #1{v}{V}ir\hologo{TeX}%
  }%
}
%    \end{macrocode}
%    \end{macro}
%    \begin{macro}{\HoLogoCs@virTeX}
%    \begin{macrocode}
\def\HoLogoCs@virTeX#1{#1{v}{V}irTeX}
%    \end{macrocode}
%    \end{macro}
%    \begin{macro}{\HoLogoBkm@virTeX}
%    \begin{macrocode}
\def\HoLogoBkm@virTeX#1{%
  #1{v}{V}ir\hologo{TeX}%
}
%    \end{macrocode}
%    \end{macro}
%    \begin{macro}{\HoLogoHtml@virTeX}
%    \begin{macrocode}
\let\HoLogoHtml@virTeX\HoLogo@virTeX
%    \end{macrocode}
%    \end{macro}
%
% \subsubsection{\hologo{SliTeX}}
%
% \paragraph{Definitions of the three variants.}
%
%    \begin{macro}{\HoLogo@SLiTeX@lift}
%    \begin{macrocode}
\def\HoLogo@SLiTeX@lift#1{%
  \HoLogoFont@font{SliTeX}{rm}{%
    S%
    \kern-.06em%
    L%
    \kern-.18em%
    \raise.32ex\hbox{\HoLogoFont@font{SliTeX}{sc}{i}}%
    \HOLOGO@discretionary
    \kern-.06em%
    \hologo{TeX}%
  }%
}
%    \end{macrocode}
%    \end{macro}
%    \begin{macro}{\HoLogoBkm@SLiTeX@lift}
%    \begin{macrocode}
\def\HoLogoBkm@SLiTeX@lift#1{SLiTeX}
%    \end{macrocode}
%    \end{macro}
%    \begin{macro}{\HoLogoHtml@SLiTeX@lift}
%    \begin{macrocode}
\def\HoLogoHtml@SLiTeX@lift#1{%
  \HoLogoCss@SLiTeX@lift
  \HOLOGO@Span{SLiTeX-lift}{%
    \HoLogoFont@font{SliTeX}{rm}{%
      S%
      \HOLOGO@Span{L}{L}%
      \HOLOGO@Span{i}{i}%
      \hologo{TeX}%
    }%
  }%
}
%    \end{macrocode}
%    \end{macro}
%    \begin{macro}{\HoLogoCss@SLiTeX@lift}
%    \begin{macrocode}
\def\HoLogoCss@SLiTeX@lift{%
  \Css{%
    span.HoLogo-SLiTeX-lift span.HoLogo-L{%
      margin-left:-.06em;%
      margin-right:-.18em;%
    }%
  }%
  \Css{%
    span.HoLogo-SLiTeX-lift span.HoLogo-i{%
      position:relative;%
      top:-.32ex;%
      margin-right:-.06em;%
      font-variant:small-caps;%
    }%
  }%
  \global\let\HoLogoCss@SLiTeX@lift\relax
}
%    \end{macrocode}
%    \end{macro}
%
%    \begin{macro}{\HoLogo@SliTeX@simple}
%    \begin{macrocode}
\def\HoLogo@SliTeX@simple#1{%
  \HoLogoFont@font{SliTeX}{rm}{%
    \ltx@mbox{%
      \HoLogoFont@font{SliTeX}{sc}{Sli}%
    }%
    \HOLOGO@discretionary
    \hologo{TeX}%
  }%
}
%    \end{macrocode}
%    \end{macro}
%    \begin{macro}{\HoLogoBkm@SliTeX@simple}
%    \begin{macrocode}
\def\HoLogoBkm@SliTeX@simple#1{SliTeX}
%    \end{macrocode}
%    \end{macro}
%    \begin{macro}{\HoLogoHtml@SliTeX@simple}
%    \begin{macrocode}
\let\HoLogoHtml@SliTeX@simple\HoLogo@SliTeX@simple
%    \end{macrocode}
%    \end{macro}
%
%    \begin{macro}{\HoLogo@SliTeX@narrow}
%    \begin{macrocode}
\def\HoLogo@SliTeX@narrow#1{%
  \HoLogoFont@font{SliTeX}{rm}{%
    \ltx@mbox{%
      S%
      \kern-.06em%
      \HoLogoFont@font{SliTeX}{sc}{%
        l%
        \kern-.035em%
        i%
      }%
    }%
    \HOLOGO@discretionary
    \kern-.06em%
    \hologo{TeX}%
  }%
}
%    \end{macrocode}
%    \end{macro}
%    \begin{macro}{\HoLogoBkm@SliTeX@narrow}
%    \begin{macrocode}
\def\HoLogoBkm@SliTeX@narrow#1{SliTeX}
%    \end{macrocode}
%    \end{macro}
%    \begin{macro}{\HoLogoHtml@SliTeX@narrow}
%    \begin{macrocode}
\def\HoLogoHtml@SliTeX@narrow#1{%
  \HoLogoCss@SliTeX@narrow
  \HOLOGO@Span{SliTeX-narrow}{%
    \HoLogoFont@font{SliTeX}{rm}{%
      S%
        \HOLOGO@Span{l}{l}%
        \HOLOGO@Span{i}{i}%
      \hologo{TeX}%
    }%
  }%
}
%    \end{macrocode}
%    \end{macro}
%    \begin{macro}{\HoLogoCss@SliTeX@narrow}
%    \begin{macrocode}
\def\HoLogoCss@SliTeX@narrow{%
  \Css{%
    span.HoLogo-SliTeX-narrow span.HoLogo-l{%
      margin-left:-.06em;%
      margin-right:-.035em;%
      font-variant:small-caps;%
    }%
  }%
  \Css{%
    span.HoLogo-SliTeX-narrow span.HoLogo-i{%
      margin-right:-.06em;%
      font-variant:small-caps;%
    }%
  }%
  \global\let\HoLogoCss@SliTeX@narrow\relax
}
%    \end{macrocode}
%    \end{macro}
%
% \paragraph{Macro set completion.}
%
%    \begin{macro}{\HoLogo@SLiTeX@simple}
%    \begin{macrocode}
\def\HoLogo@SLiTeX@simple{\HoLogo@SliTeX@simple}
%    \end{macrocode}
%    \end{macro}
%    \begin{macro}{\HoLogoBkm@SLiTeX@simple}
%    \begin{macrocode}
\def\HoLogoBkm@SLiTeX@simple{\HoLogoBkm@SliTeX@simple}
%    \end{macrocode}
%    \end{macro}
%    \begin{macro}{\HoLogoHtml@SLiTeX@simple}
%    \begin{macrocode}
\def\HoLogoHtml@SLiTeX@simple{\HoLogoHtml@SliTeX@simple}
%    \end{macrocode}
%    \end{macro}
%
%    \begin{macro}{\HoLogo@SLiTeX@narrow}
%    \begin{macrocode}
\def\HoLogo@SLiTeX@narrow{\HoLogo@SliTeX@narrow}
%    \end{macrocode}
%    \end{macro}
%    \begin{macro}{\HoLogoBkm@SLiTeX@narrow}
%    \begin{macrocode}
\def\HoLogoBkm@SLiTeX@narrow{\HoLogoBkm@SliTeX@narrow}
%    \end{macrocode}
%    \end{macro}
%    \begin{macro}{\HoLogoHtml@SLiTeX@narrow}
%    \begin{macrocode}
\def\HoLogoHtml@SLiTeX@narrow{\HoLogoHtml@SliTeX@narrow}
%    \end{macrocode}
%    \end{macro}
%
%    \begin{macro}{\HoLogo@SliTeX@lift}
%    \begin{macrocode}
\def\HoLogo@SliTeX@lift{\HoLogo@SLiTeX@lift}
%    \end{macrocode}
%    \end{macro}
%    \begin{macro}{\HoLogoBkm@SliTeX@lift}
%    \begin{macrocode}
\def\HoLogoBkm@SliTeX@lift{\HoLogoBkm@SLiTeX@lift}
%    \end{macrocode}
%    \end{macro}
%    \begin{macro}{\HoLogoHtml@SliTeX@lift}
%    \begin{macrocode}
\def\HoLogoHtml@SliTeX@lift{\HoLogoHtml@SLiTeX@lift}
%    \end{macrocode}
%    \end{macro}
%
% \paragraph{Defaults.}
%
%    \begin{macro}{\HoLogo@SLiTeX}
%    \begin{macrocode}
\def\HoLogo@SLiTeX{\HoLogo@SLiTeX@lift}
%    \end{macrocode}
%    \end{macro}
%    \begin{macro}{\HoLogoBkm@SLiTeX}
%    \begin{macrocode}
\def\HoLogoBkm@SLiTeX{\HoLogoBkm@SLiTeX@lift}
%    \end{macrocode}
%    \end{macro}
%    \begin{macro}{\HoLogoHtml@SLiTeX}
%    \begin{macrocode}
\def\HoLogoHtml@SLiTeX{\HoLogoHtml@SLiTeX@lift}
%    \end{macrocode}
%    \end{macro}
%
%    \begin{macro}{\HoLogo@SliTeX}
%    \begin{macrocode}
\def\HoLogo@SliTeX{\HoLogo@SliTeX@narrow}
%    \end{macrocode}
%    \end{macro}
%    \begin{macro}{\HoLogoBkm@SliTeX}
%    \begin{macrocode}
\def\HoLogoBkm@SliTeX{\HoLogoBkm@SliTeX@narrow}
%    \end{macrocode}
%    \end{macro}
%    \begin{macro}{\HoLogoHtml@SliTeX}
%    \begin{macrocode}
\def\HoLogoHtml@SliTeX{\HoLogoHtml@SliTeX@narrow}
%    \end{macrocode}
%    \end{macro}
%
% \subsubsection{\hologo{LuaTeX}}
%
%    \begin{macro}{\HoLogo@LuaTeX}
%    The kerning is an idea of Hans Hagen, see mailing list
%    `luatex at tug dot org' in March 2010.
%    \begin{macrocode}
\def\HoLogo@LuaTeX#1{%
  \HOLOGO@mbox{%
    Lua%
    \HOLOGO@NegativeKerning{aT,oT,To}%
    \hologo{TeX}%
  }%
}
%    \end{macrocode}
%    \end{macro}
%    \begin{macro}{\HoLogoHtml@LuaTeX}
%    \begin{macrocode}
\let\HoLogoHtml@LuaTeX\HoLogo@LuaTeX
%    \end{macrocode}
%    \end{macro}
%
% \subsubsection{\hologo{LuaLaTeX}}
%
%    \begin{macro}{\HoLogo@LuaLaTeX}
%    \begin{macrocode}
\def\HoLogo@LuaLaTeX#1{%
  \HOLOGO@mbox{%
    Lua%
    \hologo{LaTeX}%
  }%
}
%    \end{macrocode}
%    \end{macro}
%    \begin{macro}{\HoLogoHtml@LuaLaTeX}
%    \begin{macrocode}
\let\HoLogoHtml@LuaLaTeX\HoLogo@LuaLaTeX
%    \end{macrocode}
%    \end{macro}
%
% \subsubsection{\hologo{XeTeX}, \hologo{XeLaTeX}}
%
%    \begin{macro}{\HOLOGO@IfCharExists}
%    \begin{macrocode}
\ifluatex
  \ifnum\luatexversion<36 %
  \else
    \def\HOLOGO@IfCharExists#1{%
      \ifnum
        \directlua{%
           if luaotfload and luaotfload.aux then
             if luaotfload.aux.font_has_glyph(%
                    font.current(), \number#1) then % 	 
	       tex.print("1") % 	 
	     end % 	 
	   elseif font and font.fonts and font.current then %
            local f = font.fonts[font.current()]%
            if f.characters and f.characters[\number#1] then %
              tex.print("1")%
            end %
          end%
        }0=\ltx@zero
        \expandafter\ltx@secondoftwo
      \else
        \expandafter\ltx@firstoftwo
      \fi
    }%
  \fi
\fi
\ltx@IfUndefined{HOLOGO@IfCharExists}{%
  \def\HOLOGO@@IfCharExists#1{%
    \begingroup
      \tracinglostchars=\ltx@zero
      \setbox\ltx@zero=\hbox{%
        \kern7sp\char#1\relax
        \ifnum\lastkern>\ltx@zero
          \expandafter\aftergroup\csname iffalse\endcsname
        \else
          \expandafter\aftergroup\csname iftrue\endcsname
        \fi
      }%
      % \if{true|false} from \aftergroup
      \endgroup
      \expandafter\ltx@firstoftwo
    \else
      \endgroup
      \expandafter\ltx@secondoftwo
    \fi
  }%
  \ifxetex
    \ltx@IfUndefined{XeTeXfonttype}{}{%
      \ltx@IfUndefined{XeTeXcharglyph}{}{%
        \def\HOLOGO@IfCharExists#1{%
          \ifnum\XeTeXfonttype\font>\ltx@zero
            \expandafter\ltx@firstofthree
          \else
            \expandafter\ltx@gobble
          \fi
          {%
            \ifnum\XeTeXcharglyph#1>\ltx@zero
              \expandafter\ltx@firstoftwo
            \else
              \expandafter\ltx@secondoftwo
            \fi
          }%
          \HOLOGO@@IfCharExists{#1}%
        }%
      }%
    }%
  \fi
}{}
\ltx@ifundefined{HOLOGO@IfCharExists}{%
  \ifnum64=`\^^^^0040\relax % test for big chars of LuaTeX/XeTeX
    \let\HOLOGO@IfCharExists\HOLOGO@@IfCharExists
  \else
    \def\HOLOGO@IfCharExists#1{%
      \ifnum#1>255 %
        \expandafter\ltx@fourthoffour
      \fi
      \HOLOGO@@IfCharExists{#1}%
    }%
  \fi
}{}
%    \end{macrocode}
%    \end{macro}
%
%    \begin{macro}{\HoLogo@Xe}
%    Source: package \xpackage{dtklogos}
%    \begin{macrocode}
\def\HoLogo@Xe#1{%
  X%
  \kern-.1em\relax
  \HOLOGO@IfCharExists{"018E}{%
    \lower.5ex\hbox{\char"018E}%
  }{%
    \chardef\HOLOGO@choice=\ltx@zero
    \ifdim\fontdimen\ltx@one\font>0pt %
      \ltx@IfUndefined{rotatebox}{%
        \ltx@IfUndefined{pgftext}{%
          \ltx@IfUndefined{psscalebox}{%
            \ltx@IfUndefined{HOLOGO@ScaleBox@\hologoDriver}{%
            }{%
              \chardef\HOLOGO@choice=4 %
            }%
          }{%
            \chardef\HOLOGO@choice=3 %
          }%
        }{%
          \chardef\HOLOGO@choice=2 %
        }%
      }{%
        \chardef\HOLOGO@choice=1 %
      }%
      \ifcase\HOLOGO@choice
        \HOLOGO@WarningUnsupportedDriver{Xe}%
        e%
      \or % 1: \rotatebox
        \begingroup
          \setbox\ltx@zero\hbox{\rotatebox{180}{E}}%
          \ltx@LocDimenA=\dp\ltx@zero
          \advance\ltx@LocDimenA by -.5ex\relax
          \raise\ltx@LocDimenA\box\ltx@zero
        \endgroup
      \or % 2: \pgftext
        \lower.5ex\hbox{%
          \pgfpicture
            \pgftext[rotate=180]{E}%
          \endpgfpicture
        }%
      \or % 3: \psscalebox
        \begingroup
          \setbox\ltx@zero\hbox{\psscalebox{-1 -1}{E}}%
          \ltx@LocDimenA=\dp\ltx@zero
          \advance\ltx@LocDimenA by -.5ex\relax
          \raise\ltx@LocDimenA\box\ltx@zero
        \endgroup
      \or % 4: \HOLOGO@PointReflectBox
        \lower.5ex\hbox{\HOLOGO@PointReflectBox{E}}%
      \else
        \@PackageError{hologo}{Internal error (choice/it}\@ehc
      \fi
    \else
      \ltx@IfUndefined{reflectbox}{%
        \ltx@IfUndefined{pgftext}{%
          \ltx@IfUndefined{psscalebox}{%
            \ltx@IfUndefined{HOLOGO@ScaleBox@\hologoDriver}{%
            }{%
              \chardef\HOLOGO@choice=4 %
            }%
          }{%
            \chardef\HOLOGO@choice=3 %
          }%
        }{%
          \chardef\HOLOGO@choice=2 %
        }%
      }{%
        \chardef\HOLOGO@choice=1 %
      }%
      \ifcase\HOLOGO@choice
        \HOLOGO@WarningUnsupportedDriver{Xe}%
        e%
      \or % 1: reflectbox
        \lower.5ex\hbox{%
          \reflectbox{E}%
        }%
      \or % 2: \pgftext
        \lower.5ex\hbox{%
          \pgfpicture
            \pgftransformxscale{-1}%
            \pgftext{E}%
          \endpgfpicture
        }%
      \or % 3: \psscalebox
        \lower.5ex\hbox{%
          \psscalebox{-1 1}{E}%
        }%
      \or % 4: \HOLOGO@Reflectbox
        \lower.5ex\hbox{%
          \HOLOGO@ReflectBox{E}%
        }%
      \else
        \@PackageError{hologo}{Internal error (choice/up)}\@ehc
      \fi
    \fi
  }%
}
%    \end{macrocode}
%    \end{macro}
%    \begin{macro}{\HoLogoHtml@Xe}
%    \begin{macrocode}
\def\HoLogoHtml@Xe#1{%
  \HoLogoCss@Xe
  \HOLOGO@Span{Xe}{%
    X%
    \HOLOGO@Span{e}{%
      \HCode{&\ltx@hashchar x018e;}%
    }%
  }%
}
%    \end{macrocode}
%    \end{macro}
%    \begin{macro}{\HoLogoCss@Xe}
%    \begin{macrocode}
\def\HoLogoCss@Xe{%
  \Css{%
    span.HoLogo-Xe span.HoLogo-e{%
      position:relative;%
      top:.5ex;%
      left-margin:-.1em;%
    }%
  }%
  \global\let\HoLogoCss@Xe\relax
}
%    \end{macrocode}
%    \end{macro}
%
%    \begin{macro}{\HoLogo@XeTeX}
%    \begin{macrocode}
\def\HoLogo@XeTeX#1{%
  \hologo{Xe}%
  \kern-.15em\relax
  \hologo{TeX}%
}
%    \end{macrocode}
%    \end{macro}
%
%    \begin{macro}{\HoLogoHtml@XeTeX}
%    \begin{macrocode}
\def\HoLogoHtml@XeTeX#1{%
  \HoLogoCss@XeTeX
  \HOLOGO@Span{XeTeX}{%
    \hologo{Xe}%
    \hologo{TeX}%
  }%
}
%    \end{macrocode}
%    \end{macro}
%    \begin{macro}{\HoLogoCss@XeTeX}
%    \begin{macrocode}
\def\HoLogoCss@XeTeX{%
  \Css{%
    span.HoLogo-XeTeX span.HoLogo-TeX{%
      margin-left:-.15em;%
    }%
  }%
  \global\let\HoLogoCss@XeTeX\relax
}
%    \end{macrocode}
%    \end{macro}
%
%    \begin{macro}{\HoLogo@XeLaTeX}
%    \begin{macrocode}
\def\HoLogo@XeLaTeX#1{%
  \hologo{Xe}%
  \kern-.13em%
  \hologo{LaTeX}%
}
%    \end{macrocode}
%    \end{macro}
%    \begin{macro}{\HoLogoHtml@XeLaTeX}
%    \begin{macrocode}
\def\HoLogoHtml@XeLaTeX#1{%
  \HoLogoCss@XeLaTeX
  \HOLOGO@Span{XeLaTeX}{%
    \hologo{Xe}%
    \hologo{LaTeX}%
  }%
}
%    \end{macrocode}
%    \end{macro}
%    \begin{macro}{\HoLogoCss@XeLaTeX}
%    \begin{macrocode}
\def\HoLogoCss@XeLaTeX{%
  \Css{%
    span.HoLogo-XeLaTeX span.HoLogo-Xe{%
      margin-right:-.13em;%
    }%
  }%
  \global\let\HoLogoCss@XeLaTeX\relax
}
%    \end{macrocode}
%    \end{macro}
%
% \subsubsection{\hologo{pdfTeX}, \hologo{pdfLaTeX}}
%
%    \begin{macro}{\HoLogo@pdfTeX}
%    \begin{macrocode}
\def\HoLogo@pdfTeX#1{%
  \HOLOGO@mbox{%
    #1{p}{P}df\hologo{TeX}%
  }%
}
%    \end{macrocode}
%    \end{macro}
%    \begin{macro}{\HoLogoCs@pdfTeX}
%    \begin{macrocode}
\def\HoLogoCs@pdfTeX#1{#1{p}{P}dfTeX}
%    \end{macrocode}
%    \end{macro}
%    \begin{macro}{\HoLogoBkm@pdfTeX}
%    \begin{macrocode}
\def\HoLogoBkm@pdfTeX#1{%
  #1{p}{P}df\hologo{TeX}%
}
%    \end{macrocode}
%    \end{macro}
%    \begin{macro}{\HoLogoHtml@pdfTeX}
%    \begin{macrocode}
\let\HoLogoHtml@pdfTeX\HoLogo@pdfTeX
%    \end{macrocode}
%    \end{macro}
%
%    \begin{macro}{\HoLogo@pdfLaTeX}
%    \begin{macrocode}
\def\HoLogo@pdfLaTeX#1{%
  \HOLOGO@mbox{%
    #1{p}{P}df\hologo{LaTeX}%
  }%
}
%    \end{macrocode}
%    \end{macro}
%    \begin{macro}{\HoLogoCs@pdfLaTeX}
%    \begin{macrocode}
\def\HoLogoCs@pdfLaTeX#1{#1{p}{P}dfLaTeX}
%    \end{macrocode}
%    \end{macro}
%    \begin{macro}{\HoLogoBkm@pdfLaTeX}
%    \begin{macrocode}
\def\HoLogoBkm@pdfLaTeX#1{%
  #1{p}{P}df\hologo{LaTeX}%
}
%    \end{macrocode}
%    \end{macro}
%    \begin{macro}{\HoLogoHtml@pdfLaTeX}
%    \begin{macrocode}
\let\HoLogoHtml@pdfLaTeX\HoLogo@pdfLaTeX
%    \end{macrocode}
%    \end{macro}
%
% \subsubsection{\hologo{VTeX}}
%
%    \begin{macro}{\HoLogo@VTeX}
%    \begin{macrocode}
\def\HoLogo@VTeX#1{%
  \HOLOGO@mbox{%
    V\hologo{TeX}%
  }%
}
%    \end{macrocode}
%    \end{macro}
%    \begin{macro}{\HoLogoHtml@VTeX}
%    \begin{macrocode}
\let\HoLogoHtml@VTeX\HoLogo@VTeX
%    \end{macrocode}
%    \end{macro}
%
% \subsubsection{\hologo{AmS}, \dots}
%
%    Source: class \xclass{amsdtx}
%
%    \begin{macro}{\HoLogo@AmS}
%    \begin{macrocode}
\def\HoLogo@AmS#1{%
  \HoLogoFont@font{AmS}{sy}{%
    A%
    \kern-.1667em%
    \lower.5ex\hbox{M}%
    \kern-.125em%
    S%
  }%
}
%    \end{macrocode}
%    \end{macro}
%    \begin{macro}{\HoLogoBkm@AmS}
%    \begin{macrocode}
\def\HoLogoBkm@AmS#1{AmS}
%    \end{macrocode}
%    \end{macro}
%    \begin{macro}{\HoLogoHtml@AmS}
%    \begin{macrocode}
\def\HoLogoHtml@AmS#1{%
  \HoLogoCss@AmS
%  \HoLogoFont@font{AmS}{sy}{%
    \HOLOGO@Span{AmS}{%
      A%
      \HOLOGO@Span{M}{M}%
      S%
    }%
%   }%
}
%    \end{macrocode}
%    \end{macro}
%    \begin{macro}{\HoLogoCss@AmS}
%    \begin{macrocode}
\def\HoLogoCss@AmS{%
  \Css{%
    span.HoLogo-AmS span.HoLogo-M{%
      position:relative;%
      top:.5ex;%
      margin-left:-.1667em;%
      margin-right:-.125em;%
      text-decoration:none;%
    }%
  }%
  \global\let\HoLogoCss@AmS\relax
}
%    \end{macrocode}
%    \end{macro}
%
%    \begin{macro}{\HoLogo@AmSTeX}
%    \begin{macrocode}
\def\HoLogo@AmSTeX#1{%
  \hologo{AmS}%
  \HOLOGO@hyphen
  \hologo{TeX}%
}
%    \end{macrocode}
%    \end{macro}
%    \begin{macro}{\HoLogoBkm@AmSTeX}
%    \begin{macrocode}
\def\HoLogoBkm@AmSTeX#1{AmS-TeX}%
%    \end{macrocode}
%    \end{macro}
%    \begin{macro}{\HoLogoHtml@AmSTeX}
%    \begin{macrocode}
\let\HoLogoHtml@AmSTeX\HoLogo@AmSTeX
%    \end{macrocode}
%    \end{macro}
%
%    \begin{macro}{\HoLogo@AmSLaTeX}
%    \begin{macrocode}
\def\HoLogo@AmSLaTeX#1{%
  \hologo{AmS}%
  \HOLOGO@hyphen
  \hologo{LaTeX}%
}
%    \end{macrocode}
%    \end{macro}
%    \begin{macro}{\HoLogoBkm@AmSLaTeX}
%    \begin{macrocode}
\def\HoLogoBkm@AmSLaTeX#1{AmS-LaTeX}%
%    \end{macrocode}
%    \end{macro}
%    \begin{macro}{\HoLogoHtml@AmSLaTeX}
%    \begin{macrocode}
\let\HoLogoHtml@AmSLaTeX\HoLogo@AmSLaTeX
%    \end{macrocode}
%    \end{macro}
%
% \subsubsection{\hologo{BibTeX}}
%
%    \begin{macro}{\HoLogo@BibTeX@sc}
%    A definition of \hologo{BibTeX} is provided in
%    the documentation source for the manual of \hologo{BibTeX}
%    \cite{btxdoc}.
%\begin{quote}
%\begin{verbatim}
%\def\BibTeX{%
%  {%
%    \rm
%    B%
%    \kern-.05em%
%    {%
%      \sc
%      i%
%      \kern-.025em %
%      b%
%    }%
%    \kern-.08em
%    T%
%    \kern-.1667em%
%    \lower.7ex\hbox{E}%
%    \kern-.125em%
%    X%
%  }%
%}
%\end{verbatim}
%\end{quote}
%    \begin{macrocode}
\def\HoLogo@BibTeX@sc#1{%
  B%
  \kern-.05em%
  \HoLogoFont@font{BibTeX}{sc}{%
    i%
    \kern-.025em%
    b%
  }%
  \HOLOGO@discretionary
  \kern-.08em%
  \hologo{TeX}%
}
%    \end{macrocode}
%    \end{macro}
%    \begin{macro}{\HoLogoHtml@BibTeX@sc}
%    \begin{macrocode}
\def\HoLogoHtml@BibTeX@sc#1{%
  \HoLogoCss@BibTeX@sc
  \HOLOGO@Span{BibTeX-sc}{%
    B%
    \HOLOGO@Span{i}{i}%
    \HOLOGO@Span{b}{b}%
    \hologo{TeX}%
  }%
}
%    \end{macrocode}
%    \end{macro}
%    \begin{macro}{\HoLogoCss@BibTeX@sc}
%    \begin{macrocode}
\def\HoLogoCss@BibTeX@sc{%
  \Css{%
    span.HoLogo-BibTeX-sc span.HoLogo-i{%
      margin-left:-.05em;%
      margin-right:-.025em;%
      font-variant:small-caps;%
    }%
  }%
  \Css{%
    span.HoLogo-BibTeX-sc span.HoLogo-b{%
      margin-right:-.08em;%
      font-variant:small-caps;%
    }%
  }%
  \global\let\HoLogoCss@BibTeX@sc\relax
}
%    \end{macrocode}
%    \end{macro}
%
%    \begin{macro}{\HoLogo@BibTeX@sf}
%    Variant \xoption{sf} avoids trouble with unavailable
%    small caps fonts (e.g., bold versions of Computer Modern or
%    Latin Modern). The definition is taken from
%    package \xpackage{dtklogos} \cite{dtklogos}.
%\begin{quote}
%\begin{verbatim}
%\DeclareRobustCommand{\BibTeX}{%
%  B%
%  \kern-.05em%
%  \hbox{%
%    $\m@th$% %% force math size calculations
%    \csname S@\f@size\endcsname
%    \fontsize\sf@size\z@
%    \math@fontsfalse
%    \selectfont
%    I%
%    \kern-.025em%
%    B
%  }%
%  \kern-.08em%
%  \-%
%  \TeX
%}
%\end{verbatim}
%\end{quote}
%    \begin{macrocode}
\def\HoLogo@BibTeX@sf#1{%
  B%
  \kern-.05em%
  \HoLogoFont@font{BibTeX}{bibsf}{%
    I%
    \kern-.025em%
    B%
  }%
  \HOLOGO@discretionary
  \kern-.08em%
  \hologo{TeX}%
}
%    \end{macrocode}
%    \end{macro}
%    \begin{macro}{\HoLogoHtml@BibTeX@sf}
%    \begin{macrocode}
\def\HoLogoHtml@BibTeX@sf#1{%
  \HoLogoCss@BibTeX@sf
  \HOLOGO@Span{BibTeX-sf}{%
    B%
    \HoLogoFont@font{BibTeX}{bibsf}{%
      \HOLOGO@Span{i}{I}%
      B%
    }%
    \hologo{TeX}%
  }%
}
%    \end{macrocode}
%    \end{macro}
%    \begin{macro}{\HoLogoCss@BibTeX@sf}
%    \begin{macrocode}
\def\HoLogoCss@BibTeX@sf{%
  \Css{%
    span.HoLogo-BibTeX-sf span.HoLogo-i{%
      margin-left:-.05em;%
      margin-right:-.025em;%
    }%
  }%
  \Css{%
    span.HoLogo-BibTeX-sf span.HoLogo-TeX{%
      margin-left:-.08em;%
    }%
  }%
  \global\let\HoLogoCss@BibTeX@sf\relax
}
%    \end{macrocode}
%    \end{macro}
%
%    \begin{macro}{\HoLogo@BibTeX}
%    \begin{macrocode}
\def\HoLogo@BibTeX{\HoLogo@BibTeX@sf}
%    \end{macrocode}
%    \end{macro}
%    \begin{macro}{\HoLogoHtml@BibTeX}
%    \begin{macrocode}
\def\HoLogoHtml@BibTeX{\HoLogoHtml@BibTeX@sf}
%    \end{macrocode}
%    \end{macro}
%
% \subsubsection{\hologo{BibTeX8}}
%
%    \begin{macro}{\HoLogo@BibTeX8}
%    \begin{macrocode}
\expandafter\def\csname HoLogo@BibTeX8\endcsname#1{%
  \hologo{BibTeX}%
  8%
}
%    \end{macrocode}
%    \end{macro}
%
%    \begin{macro}{\HoLogoBkm@BibTeX8}
%    \begin{macrocode}
\expandafter\def\csname HoLogoBkm@BibTeX8\endcsname#1{%
  \hologo{BibTeX}%
  8%
}
%    \end{macrocode}
%    \end{macro}
%    \begin{macro}{\HoLogoHtml@BibTeX8}
%    \begin{macrocode}
\expandafter
\let\csname HoLogoHtml@BibTeX8\expandafter\endcsname
\csname HoLogo@BibTeX8\endcsname
%    \end{macrocode}
%    \end{macro}
%
% \subsubsection{\hologo{ConTeXt}}
%
%    \begin{macro}{\HoLogo@ConTeXt@simple}
%    \begin{macrocode}
\def\HoLogo@ConTeXt@simple#1{%
  \HOLOGO@mbox{Con}%
  \HOLOGO@discretionary
  \HOLOGO@mbox{\hologo{TeX}t}%
}
%    \end{macrocode}
%    \end{macro}
%    \begin{macro}{\HoLogoHtml@ConTeXt@simple}
%    \begin{macrocode}
\let\HoLogoHtml@ConTeXt@simple\HoLogo@ConTeXt@simple
%    \end{macrocode}
%    \end{macro}
%
%    \begin{macro}{\HoLogo@ConTeXt@narrow}
%    This definition of logo \hologo{ConTeXt} with variant \xoption{narrow}
%    comes from TUGboat's class \xclass{ltugboat} (version 2010/11/15 v2.8).
%    \begin{macrocode}
\def\HoLogo@ConTeXt@narrow#1{%
  \HOLOGO@mbox{C\kern-.0333emon}%
  \HOLOGO@discretionary
  \kern-.0667em%
  \HOLOGO@mbox{\hologo{TeX}\kern-.0333emt}%
}
%    \end{macrocode}
%    \end{macro}
%    \begin{macro}{\HoLogoHtml@ConTeXt@narrow}
%    \begin{macrocode}
\def\HoLogoHtml@ConTeXt@narrow#1{%
  \HoLogoCss@ConTeXt@narrow
  \HOLOGO@Span{ConTeXt-narrow}{%
    \HOLOGO@Span{C}{C}%
    on%
    \hologo{TeX}%
    t%
  }%
}
%    \end{macrocode}
%    \end{macro}
%    \begin{macro}{\HoLogoCss@ConTeXt@narrow}
%    \begin{macrocode}
\def\HoLogoCss@ConTeXt@narrow{%
  \Css{%
    span.HoLogo-ConTeXt-narrow span.HoLogo-C{%
      margin-left:-.0333em;%
    }%
  }%
  \Css{%
    span.HoLogo-ConTeXt-narrow span.HoLogo-TeX{%
      margin-left:-.0667em;%
      margin-right:-.0333em;%
    }%
  }%
  \global\let\HoLogoCss@ConTeXt@narrow\relax
}
%    \end{macrocode}
%    \end{macro}
%
%    \begin{macro}{\HoLogo@ConTeXt}
%    \begin{macrocode}
\def\HoLogo@ConTeXt{\HoLogo@ConTeXt@narrow}
%    \end{macrocode}
%    \end{macro}
%    \begin{macro}{\HoLogoHtml@ConTeXt}
%    \begin{macrocode}
\def\HoLogoHtml@ConTeXt{\HoLogoHtml@ConTeXt@narrow}
%    \end{macrocode}
%    \end{macro}
%
% \subsubsection{\hologo{emTeX}}
%
%    \begin{macro}{\HoLogo@emTeX}
%    \begin{macrocode}
\def\HoLogo@emTeX#1{%
  \HOLOGO@mbox{#1{e}{E}m}%
  \HOLOGO@discretionary
  \hologo{TeX}%
}
%    \end{macrocode}
%    \end{macro}
%    \begin{macro}{\HoLogoCs@emTeX}
%    \begin{macrocode}
\def\HoLogoCs@emTeX#1{#1{e}{E}mTeX}%
%    \end{macrocode}
%    \end{macro}
%    \begin{macro}{\HoLogoBkm@emTeX}
%    \begin{macrocode}
\def\HoLogoBkm@emTeX#1{%
  #1{e}{E}m\hologo{TeX}%
}
%    \end{macrocode}
%    \end{macro}
%    \begin{macro}{\HoLogoHtml@emTeX}
%    \begin{macrocode}
\let\HoLogoHtml@emTeX\HoLogo@emTeX
%    \end{macrocode}
%    \end{macro}
%
% \subsubsection{\hologo{ExTeX}}
%
%    \begin{macro}{\HoLogo@ExTeX}
%    The definition is taken from the FAQ of the
%    project \hologo{ExTeX}
%    \cite{ExTeX-FAQ}.
%\begin{quote}
%\begin{verbatim}
%\def\ExTeX{%
%  \textrm{% Logo always with serifs
%    \ensuremath{%
%      \textstyle
%      \varepsilon_{%
%        \kern-0.15em%
%        \mathcal{X}%
%      }%
%    }%
%    \kern-.15em%
%    \TeX
%  }%
%}
%\end{verbatim}
%\end{quote}
%    \begin{macrocode}
\def\HoLogo@ExTeX#1{%
  \HoLogoFont@font{ExTeX}{rm}{%
    \ltx@mbox{%
      \HOLOGO@MathSetup
      $%
        \textstyle
        \varepsilon_{%
          \kern-0.15em%
          \HoLogoFont@font{ExTeX}{sy}{X}%
        }%
      $%
    }%
    \HOLOGO@discretionary
    \kern-.15em%
    \hologo{TeX}%
  }%
}
%    \end{macrocode}
%    \end{macro}
%    \begin{macro}{\HoLogoHtml@ExTeX}
%    \begin{macrocode}
\def\HoLogoHtml@ExTeX#1{%
  \HoLogoCss@ExTeX
  \HoLogoFont@font{ExTeX}{rm}{%
    \HOLOGO@Span{ExTeX}{%
      \ltx@mbox{%
        \HOLOGO@MathSetup
        $\textstyle\varepsilon$%
        \HOLOGO@Span{X}{$\textstyle\chi$}%
        \hologo{TeX}%
      }%
    }%
  }%
}
%    \end{macrocode}
%    \end{macro}
%    \begin{macro}{\HoLogoBkm@ExTeX}
%    \begin{macrocode}
\def\HoLogoBkm@ExTeX#1{%
  \HOLOGO@PdfdocUnicode{#1{e}{E}x}{\textepsilon\textchi}%
  \hologo{TeX}%
}
%    \end{macrocode}
%    \end{macro}
%    \begin{macro}{\HoLogoCss@ExTeX}
%    \begin{macrocode}
\def\HoLogoCss@ExTeX{%
  \Css{%
    span.HoLogo-ExTeX{%
      font-family:serif;%
    }%
  }%
  \Css{%
    span.HoLogo-ExTeX span.HoLogo-TeX{%
      margin-left:-.15em;%
    }%
  }%
  \global\let\HoLogoCss@ExTeX\relax
}
%    \end{macrocode}
%    \end{macro}
%
% \subsubsection{\hologo{MiKTeX}}
%
%    \begin{macro}{\HoLogo@MiKTeX}
%    \begin{macrocode}
\def\HoLogo@MiKTeX#1{%
  \HOLOGO@mbox{MiK}%
  \HOLOGO@discretionary
  \hologo{TeX}%
}
%    \end{macrocode}
%    \end{macro}
%    \begin{macro}{\HoLogoHtml@MiKTeX}
%    \begin{macrocode}
\let\HoLogoHtml@MiKTeX\HoLogo@MiKTeX
%    \end{macrocode}
%    \end{macro}
%
% \subsubsection{\hologo{OzTeX} and friends}
%
%    Source: \hologo{OzTeX} FAQ \cite{OzTeX}:
%    \begin{quote}
%      |\def\OzTeX{O\kern-.03em z\kern-.15em\TeX}|\\
%      (There is no kerning in OzMF, OzMP and OzTtH.)
%    \end{quote}
%
%    \begin{macro}{\HoLogo@OzTeX}
%    \begin{macrocode}
\def\HoLogo@OzTeX#1{%
  O%
  \kern-.03em %
  z%
  \kern-.15em %
  \hologo{TeX}%
}
%    \end{macrocode}
%    \end{macro}
%    \begin{macro}{\HoLogoHtml@OzTeX}
%    \begin{macrocode}
\def\HoLogoHtml@OzTeX#1{%
  \HoLogoCss@OzTeX
  \HOLOGO@Span{OzTeX}{%
    O%
    \HOLOGO@Span{z}{z}%
    \hologo{TeX}%
  }%
}
%    \end{macrocode}
%    \end{macro}
%    \begin{macro}{\HoLogoCss@OzTeX}
%    \begin{macrocode}
\def\HoLogoCss@OzTeX{%
  \Css{%
    span.HoLogo-OzTeX span.HoLogo-z{%
      margin-left:-.03em;%
      margin-right:-.15em;%
    }%
  }%
  \global\let\HoLogoCss@OzTeX\relax
}
%    \end{macrocode}
%    \end{macro}
%
%    \begin{macro}{\HoLogo@OzMF}
%    \begin{macrocode}
\def\HoLogo@OzMF#1{%
  \HOLOGO@mbox{OzMF}%
}
%    \end{macrocode}
%    \end{macro}
%    \begin{macro}{\HoLogo@OzMP}
%    \begin{macrocode}
\def\HoLogo@OzMP#1{%
  \HOLOGO@mbox{OzMP}%
}
%    \end{macrocode}
%    \end{macro}
%    \begin{macro}{\HoLogo@OzTtH}
%    \begin{macrocode}
\def\HoLogo@OzTtH#1{%
  \HOLOGO@mbox{OzTtH}%
}
%    \end{macrocode}
%    \end{macro}
%
% \subsubsection{\hologo{PCTeX}}
%
%    \begin{macro}{\HoLogo@PCTeX}
%    \begin{macrocode}
\def\HoLogo@PCTeX#1{%
  \HOLOGO@mbox{PC}%
  \hologo{TeX}%
}
%    \end{macrocode}
%    \end{macro}
%    \begin{macro}{\HoLogoHtml@PCTeX}
%    \begin{macrocode}
\let\HoLogoHtml@PCTeX\HoLogo@PCTeX
%    \end{macrocode}
%    \end{macro}
%
% \subsubsection{\hologo{PiCTeX}}
%
%    The original definitions from \xfile{pictex.tex} \cite{PiCTeX}:
%\begin{quote}
%\begin{verbatim}
%\def\PiC{%
%  P%
%  \kern-.12em%
%  \lower.5ex\hbox{I}%
%  \kern-.075em%
%  C%
%}
%\def\PiCTeX{%
%  \PiC
%  \kern-.11em%
%  \TeX
%}
%\end{verbatim}
%\end{quote}
%
%    \begin{macro}{\HoLogo@PiC}
%    \begin{macrocode}
\def\HoLogo@PiC#1{%
  P%
  \kern-.12em%
  \lower.5ex\hbox{I}%
  \kern-.075em%
  C%
  \HOLOGO@SpaceFactor
}
%    \end{macrocode}
%    \end{macro}
%    \begin{macro}{\HoLogoHtml@PiC}
%    \begin{macrocode}
\def\HoLogoHtml@PiC#1{%
  \HoLogoCss@PiC
  \HOLOGO@Span{PiC}{%
    P%
    \HOLOGO@Span{i}{I}%
    C%
  }%
}
%    \end{macrocode}
%    \end{macro}
%    \begin{macro}{\HoLogoCss@PiC}
%    \begin{macrocode}
\def\HoLogoCss@PiC{%
  \Css{%
    span.HoLogo-PiC span.HoLogo-i{%
      position:relative;%
      top:.5ex;%
      margin-left:-.12em;%
      margin-right:-.075em;%
      text-decoration:none;%
    }%
  }%
  \global\let\HoLogoCss@PiC\relax
}
%    \end{macrocode}
%    \end{macro}
%
%    \begin{macro}{\HoLogo@PiCTeX}
%    \begin{macrocode}
\def\HoLogo@PiCTeX#1{%
  \hologo{PiC}%
  \HOLOGO@discretionary
  \kern-.11em%
  \hologo{TeX}%
}
%    \end{macrocode}
%    \end{macro}
%    \begin{macro}{\HoLogoHtml@PiCTeX}
%    \begin{macrocode}
\def\HoLogoHtml@PiCTeX#1{%
  \HoLogoCss@PiCTeX
  \HOLOGO@Span{PiCTeX}{%
    \hologo{PiC}%
    \hologo{TeX}%
  }%
}
%    \end{macrocode}
%    \end{macro}
%    \begin{macro}{\HoLogoCss@PiCTeX}
%    \begin{macrocode}
\def\HoLogoCss@PiCTeX{%
  \Css{%
    span.HoLogo-PiCTeX span.HoLogo-PiC{%
      margin-right:-.11em;%
    }%
  }%
  \global\let\HoLogoCss@PiCTeX\relax
}
%    \end{macrocode}
%    \end{macro}
%
% \subsubsection{\hologo{teTeX}}
%
%    \begin{macro}{\HoLogo@teTeX}
%    \begin{macrocode}
\def\HoLogo@teTeX#1{%
  \HOLOGO@mbox{#1{t}{T}e}%
  \HOLOGO@discretionary
  \hologo{TeX}%
}
%    \end{macrocode}
%    \end{macro}
%    \begin{macro}{\HoLogoCs@teTeX}
%    \begin{macrocode}
\def\HoLogoCs@teTeX#1{#1{t}{T}dfTeX}
%    \end{macrocode}
%    \end{macro}
%    \begin{macro}{\HoLogoBkm@teTeX}
%    \begin{macrocode}
\def\HoLogoBkm@teTeX#1{%
  #1{t}{T}e\hologo{TeX}%
}
%    \end{macrocode}
%    \end{macro}
%    \begin{macro}{\HoLogoHtml@teTeX}
%    \begin{macrocode}
\let\HoLogoHtml@teTeX\HoLogo@teTeX
%    \end{macrocode}
%    \end{macro}
%
% \subsubsection{\hologo{TeX4ht}}
%
%    \begin{macro}{\HoLogo@TeX4ht}
%    \begin{macrocode}
\expandafter\def\csname HoLogo@TeX4ht\endcsname#1{%
  \HOLOGO@mbox{\hologo{TeX}4ht}%
}
%    \end{macrocode}
%    \end{macro}
%    \begin{macro}{\HoLogoHtml@TeX4ht}
%    \begin{macrocode}
\expandafter
\let\csname HoLogoHtml@TeX4ht\expandafter\endcsname
\csname HoLogo@TeX4ht\endcsname
%    \end{macrocode}
%    \end{macro}
%
%
% \subsubsection{\hologo{SageTeX}}
%
%    \begin{macro}{\HoLogo@SageTeX}
%    \begin{macrocode}
\def\HoLogo@SageTeX#1{%
  \HOLOGO@mbox{Sage}%
  \HOLOGO@discretionary
  \HOLOGO@NegativeKerning{eT,oT,To}%
  \hologo{TeX}%
}
%    \end{macrocode}
%    \end{macro}
%    \begin{macro}{\HoLogoHtml@SageTeX}
%    \begin{macrocode}
\let\HoLogoHtml@SageTeX\HoLogo@SageTeX
%    \end{macrocode}
%    \end{macro}
%
% \subsection{\hologo{METAFONT} and friends}
%
%    \begin{macro}{\HoLogo@METAFONT}
%    \begin{macrocode}
\def\HoLogo@METAFONT#1{%
  \HoLogoFont@font{METAFONT}{logo}{%
    \HOLOGO@mbox{META}%
    \HOLOGO@discretionary
    \HOLOGO@mbox{FONT}%
  }%
}
%    \end{macrocode}
%    \end{macro}
%
%    \begin{macro}{\HoLogo@METAPOST}
%    \begin{macrocode}
\def\HoLogo@METAPOST#1{%
  \HoLogoFont@font{METAPOST}{logo}{%
    \HOLOGO@mbox{META}%
    \HOLOGO@discretionary
    \HOLOGO@mbox{POST}%
  }%
}
%    \end{macrocode}
%    \end{macro}
%
%    \begin{macro}{\HoLogo@MetaFun}
%    \begin{macrocode}
\def\HoLogo@MetaFun#1{%
  \HOLOGO@mbox{Meta}%
  \HOLOGO@discretionary
  \HOLOGO@mbox{Fun}%
}
%    \end{macrocode}
%    \end{macro}
%
%    \begin{macro}{\HoLogo@MetaPost}
%    \begin{macrocode}
\def\HoLogo@MetaPost#1{%
  \HOLOGO@mbox{Meta}%
  \HOLOGO@discretionary
  \HOLOGO@mbox{Post}%
}
%    \end{macrocode}
%    \end{macro}
%
% \subsection{Others}
%
% \subsubsection{\hologo{biber}}
%
%    \begin{macro}{\HoLogo@biber}
%    \begin{macrocode}
\def\HoLogo@biber#1{%
  \HOLOGO@mbox{#1{b}{B}i}%
  \HOLOGO@discretionary
  \HOLOGO@mbox{ber}%
}
%    \end{macrocode}
%    \end{macro}
%    \begin{macro}{\HoLogoCs@biber}
%    \begin{macrocode}
\def\HoLogoCs@biber#1{#1{b}{B}iber}
%    \end{macrocode}
%    \end{macro}
%    \begin{macro}{\HoLogoBkm@biber}
%    \begin{macrocode}
\def\HoLogoBkm@biber#1{%
  #1{b}{B}iber%
}
%    \end{macrocode}
%    \end{macro}
%    \begin{macro}{\HoLogoHtml@biber}
%    \begin{macrocode}
\let\HoLogoHtml@biber\HoLogo@biber
%    \end{macrocode}
%    \end{macro}
%
% \subsubsection{\hologo{KOMAScript}}
%
%    \begin{macro}{\HoLogo@KOMAScript}
%    The definition for \hologo{KOMAScript} is taken
%    from \hologo{KOMAScript} (\xfile{scrlogo.dtx}, reformatted) \cite{scrlogo}:
%\begin{quote}
%\begin{verbatim}
%\@ifundefined{KOMAScript}{%
%  \DeclareRobustCommand{\KOMAScript}{%
%    \textsf{%
%      K\kern.05em O\kern.05emM\kern.05em A%
%      \kern.1em-\kern.1em %
%      Script%
%    }%
%  }%
%}{}
%\end{verbatim}
%\end{quote}
%    \begin{macrocode}
\def\HoLogo@KOMAScript#1{%
  \HoLogoFont@font{KOMAScript}{sf}{%
    \HOLOGO@mbox{%
      K\kern.05em%
      O\kern.05em%
      M\kern.05em%
      A%
    }%
    \kern.1em%
    \HOLOGO@hyphen
    \kern.1em%
    \HOLOGO@mbox{Script}%
  }%
}
%    \end{macrocode}
%    \end{macro}
%    \begin{macro}{\HoLogoBkm@KOMAScript}
%    \begin{macrocode}
\def\HoLogoBkm@KOMAScript#1{%
  KOMA-Script%
}
%    \end{macrocode}
%    \end{macro}
%    \begin{macro}{\HoLogoHtml@KOMAScript}
%    \begin{macrocode}
\def\HoLogoHtml@KOMAScript#1{%
  \HoLogoCss@KOMAScript
  \HoLogoFont@font{KOMAScript}{sf}{%
    \HOLOGO@Span{KOMAScript}{%
      K%
      \HOLOGO@Span{O}{O}%
      M%
      \HOLOGO@Span{A}{A}%
      \HOLOGO@Span{hyphen}{-}%
      Script%
    }%
  }%
}
%    \end{macrocode}
%    \end{macro}
%    \begin{macro}{\HoLogoCss@KOMAScript}
%    \begin{macrocode}
\def\HoLogoCss@KOMAScript{%
  \Css{%
    span.HoLogo-KOMAScript{%
      font-family:sans-serif;%
    }%
  }%
  \Css{%
    span.HoLogo-KOMAScript span.HoLogo-O{%
      padding-left:.05em;%
      padding-right:.05em;%
    }%
  }%
  \Css{%
    span.HoLogo-KOMAScript span.HoLogo-A{%
      padding-left:.05em;%
    }%
  }%
  \Css{%
    span.HoLogo-KOMAScript span.HoLogo-hyphen{%
      padding-left:.1em;%
      padding-right:.1em;%
    }%
  }%
  \global\let\HoLogoCss@KOMAScript\relax
}
%    \end{macrocode}
%    \end{macro}
%
% \subsubsection{\hologo{LyX}}
%
%    \begin{macro}{\HoLogo@LyX}
%    The definition is taken from the documentation source files
%    of \hologo{LyX}, \xfile{Intro.lyx} \cite{LyX}:
%\begin{quote}
%\begin{verbatim}
%\def\LyX{%
%  \texorpdfstring{%
%    L\kern-.1667em\lower.25em\hbox{Y}\kern-.125emX\@%
%  }{%
%    LyX%
%  }%
%}
%\end{verbatim}
%\end{quote}
%    \begin{macrocode}
\def\HoLogo@LyX#1{%
  L%
  \kern-.1667em%
  \lower.25em\hbox{Y}%
  \kern-.125em%
  X%
  \HOLOGO@SpaceFactor
}
%    \end{macrocode}
%    \end{macro}
%    \begin{macro}{\HoLogoHtml@LyX}
%    \begin{macrocode}
\def\HoLogoHtml@LyX#1{%
  \HoLogoCss@LyX
  \HOLOGO@Span{LyX}{%
    L%
    \HOLOGO@Span{y}{Y}%
    X%
  }%
}
%    \end{macrocode}
%    \end{macro}
%    \begin{macro}{\HoLogoCss@LyX}
%    \begin{macrocode}
\def\HoLogoCss@LyX{%
  \Css{%
    span.HoLogo-LyX span.HoLogo-y{%
      position:relative;%
      top:.25em;%
      margin-left:-.1667em;%
      margin-right:-.125em;%
      text-decoration:none;%
    }%
  }%
  \global\let\HoLogoCss@LyX\relax
}
%    \end{macrocode}
%    \end{macro}
%
% \subsubsection{\hologo{NTS}}
%
%    \begin{macro}{\HoLogo@NTS}
%    Definition for \hologo{NTS} can be found in
%    package \xpackage{etex\textunderscore man} for the \hologo{eTeX} manual \cite{etexman}
%    and in package \xpackage{dtklogos} \cite{dtklogos}:
%\begin{quote}
%\begin{verbatim}
%\def\NTS{%
%  \leavevmode
%  \hbox{%
%    $%
%      \cal N%
%      \kern-0.35em%
%      \lower0.5ex\hbox{$\cal T$}%
%      \kern-0.2em%
%      S%
%    $%
%  }%
%}
%\end{verbatim}
%\end{quote}
%    \begin{macrocode}
\def\HoLogo@NTS#1{%
  \HoLogoFont@font{NTS}{sy}{%
    N\/%
    \kern-.35em%
    \lower.5ex\hbox{T\/}%
    \kern-.2em%
    S\/%
  }%
  \HOLOGO@SpaceFactor
}
%    \end{macrocode}
%    \end{macro}
%
% \subsubsection{\Hologo{TTH} (\hologo{TeX} to HTML translator)}
%
%    Source: \url{http://hutchinson.belmont.ma.us/tth/}
%    In the HTML source the second `T' is printed as subscript.
%\begin{quote}
%\begin{verbatim}
%T<sub>T</sub>H
%\end{verbatim}
%\end{quote}
%    \begin{macro}{\HoLogo@TTH}
%    \begin{macrocode}
\def\HoLogo@TTH#1{%
  \ltx@mbox{%
    T\HOLOGO@SubScript{T}H%
  }%
  \HOLOGO@SpaceFactor
}
%    \end{macrocode}
%    \end{macro}
%
%    \begin{macro}{\HoLogoHtml@TTH}
%    \begin{macrocode}
\def\HoLogoHtml@TTH#1{%
  T\HCode{<sub>}T\HCode{</sub>}H%
}
%    \end{macrocode}
%    \end{macro}
%
% \subsubsection{\Hologo{HanTheThanh}}
%
%    Partial source: Package \xpackage{dtklogos}.
%    The double accent is U+1EBF (latin small letter e with circumflex
%    and acute).
%    \begin{macro}{\HoLogo@HanTheThanh}
%    \begin{macrocode}
\def\HoLogo@HanTheThanh#1{%
  \ltx@mbox{H\`an}%
  \HOLOGO@space
  \ltx@mbox{%
    Th%
    \HOLOGO@IfCharExists{"1EBF}{%
      \char"1EBF\relax
    }{%
      \^e\hbox to 0pt{\hss\raise .5ex\hbox{\'{}}}%
    }%
  }%
  \HOLOGO@space
  \ltx@mbox{Th\`anh}%
}
%    \end{macrocode}
%    \end{macro}
%    \begin{macro}{\HoLogoBkm@HanTheThanh}
%    \begin{macrocode}
\def\HoLogoBkm@HanTheThanh#1{%
  H\`an %
  Th\HOLOGO@PdfdocUnicode{\^e}{\9036\277} %
  Th\`anh%
}
%    \end{macrocode}
%    \end{macro}
%    \begin{macro}{\HoLogoHtml@HanTheThanh}
%    \begin{macrocode}
\def\HoLogoHtml@HanTheThanh#1{%
  H\`an %
  Th\HCode{&\ltx@hashchar x1ebf;} %
  Th\`anh%
}
%    \end{macrocode}
%    \end{macro}
%
% \subsection{Driver detection}
%
%    \begin{macrocode}
\HOLOGO@IfExists\InputIfFileExists{%
  \InputIfFileExists{hologo.cfg}{}{}%
}{%
  \ltx@IfUndefined{pdf@filesize}{%
    \def\HOLOGO@InputIfExists{%
      \openin\HOLOGO@temp=hologo.cfg\relax
      \ifeof\HOLOGO@temp
        \closein\HOLOGO@temp
      \else
        \closein\HOLOGO@temp
        \begingroup
          \def\x{LaTeX2e}%
        \expandafter\endgroup
        \ifx\fmtname\x
          % \iffalse meta-comment
%
% File: hologo.dtx
% Version: 2016/05/12 v1.11
% Info: A logo collection with bookmark support
%
% Copyright (C) 2010-2012 by
%    Heiko Oberdiek <heiko.oberdiek at googlemail.com>
%
% This work may be distributed and/or modified under the
% conditions of the LaTeX Project Public License, either
% version 1.3c of this license or (at your option) any later
% version. This version of this license is in
%    http://www.latex-project.org/lppl/lppl-1-3c.txt
% and the latest version of this license is in
%    http://www.latex-project.org/lppl.txt
% and version 1.3 or later is part of all distributions of
% LaTeX version 2005/12/01 or later.
%
% This work has the LPPL maintenance status "maintained".
%
% This Current Maintainer of this work is Heiko Oberdiek.
%
% The Base Interpreter refers to any `TeX-Format',
% because some files are installed in TDS:tex/generic//.
%
% This work consists of the main source file hologo.dtx
% and the derived files
%    hologo.sty, hologo.pdf, hologo.ins, hologo.drv, hologo-example.tex,
%    hologo-test1.tex, hologo-test-spacefactor.tex,
%    hologo-test-list.tex.
%
% Distribution:
%    CTAN:macros/latex/contrib/oberdiek/hologo.dtx
%    CTAN:macros/latex/contrib/oberdiek/hologo.pdf
%
% Unpacking:
%    (a) If hologo.ins is present:
%           tex hologo.ins
%    (b) Without hologo.ins:
%           tex hologo.dtx
%    (c) If you insist on using LaTeX
%           latex \let\install=y\input{hologo.dtx}
%        (quote the arguments according to the demands of your shell)
%
% Documentation:
%    (a) If hologo.drv is present:
%           latex hologo.drv
%    (b) Without hologo.drv:
%           latex hologo.dtx; ...
%    The class ltxdoc loads the configuration file ltxdoc.cfg
%    if available. Here you can specify further options, e.g.
%    use A4 as paper format:
%       \PassOptionsToClass{a4paper}{article}
%
%    Programm calls to get the documentation (example):
%       pdflatex hologo.dtx
%       makeindex -s gind.ist hologo.idx
%       pdflatex hologo.dtx
%       makeindex -s gind.ist hologo.idx
%       pdflatex hologo.dtx
%
% Installation:
%    TDS:tex/generic/oberdiek/hologo.sty
%    TDS:doc/latex/oberdiek/hologo.pdf
%    TDS:doc/latex/oberdiek/example/hologo-example.tex
%    TDS:doc/latex/oberdiek/test/hologo-test1.tex
%    TDS:doc/latex/oberdiek/test/hologo-test-spacefactor.tex
%    TDS:doc/latex/oberdiek/test/hologo-test-list.tex
%    TDS:source/latex/oberdiek/hologo.dtx
%
%<*ignore>
\begingroup
  \catcode123=1 %
  \catcode125=2 %
  \def\x{LaTeX2e}%
\expandafter\endgroup
\ifcase 0\ifx\install y1\fi\expandafter
         \ifx\csname processbatchFile\endcsname\relax\else1\fi
         \ifx\fmtname\x\else 1\fi\relax
\else\csname fi\endcsname
%</ignore>
%<*install>
\input docstrip.tex
\Msg{************************************************************************}
\Msg{* Installation}
\Msg{* Package: hologo 2016/05/12 v1.11 A logo collection with bookmark support (HO)}
\Msg{************************************************************************}

\keepsilent
\askforoverwritefalse

\let\MetaPrefix\relax
\preamble

This is a generated file.

Project: hologo
Version: 2016/05/12 v1.11

Copyright (C) 2010-2012 by
   Heiko Oberdiek <heiko.oberdiek at googlemail.com>

This work may be distributed and/or modified under the
conditions of the LaTeX Project Public License, either
version 1.3c of this license or (at your option) any later
version. This version of this license is in
   http://www.latex-project.org/lppl/lppl-1-3c.txt
and the latest version of this license is in
   http://www.latex-project.org/lppl.txt
and version 1.3 or later is part of all distributions of
LaTeX version 2005/12/01 or later.

This work has the LPPL maintenance status "maintained".

This Current Maintainer of this work is Heiko Oberdiek.

The Base Interpreter refers to any `TeX-Format',
because some files are installed in TDS:tex/generic//.

This work consists of the main source file hologo.dtx
and the derived files
   hologo.sty, hologo.pdf, hologo.ins, hologo.drv, hologo-example.tex,
   hologo-test1.tex, hologo-test-spacefactor.tex,
   hologo-test-list.tex.

\endpreamble
\let\MetaPrefix\DoubleperCent

\generate{%
  \file{hologo.ins}{\from{hologo.dtx}{install}}%
  \file{hologo.drv}{\from{hologo.dtx}{driver}}%
  \usedir{tex/generic/oberdiek}%
  \file{hologo.sty}{\from{hologo.dtx}{package}}%
  \usedir{doc/latex/oberdiek/example}%
  \file{hologo-example.tex}{\from{hologo.dtx}{example}}%
  \usedir{doc/latex/oberdiek/test}%
  \file{hologo-test1.tex}{\from{hologo.dtx}{test1}}%
  \file{hologo-test-spacefactor.tex}{\from{hologo.dtx}{test-spacefactor}}%
  \file{hologo-test-list.tex}{\from{hologo.dtx}{test-list}}%
  \nopreamble
  \nopostamble
  \usedir{source/latex/oberdiek/catalogue}%
  \file{hologo.xml}{\from{hologo.dtx}{catalogue}}%
}

\catcode32=13\relax% active space
\let =\space%
\Msg{************************************************************************}
\Msg{*}
\Msg{* To finish the installation you have to move the following}
\Msg{* file into a directory searched by TeX:}
\Msg{*}
\Msg{*     hologo.sty}
\Msg{*}
\Msg{* To produce the documentation run the file `hologo.drv'}
\Msg{* through LaTeX.}
\Msg{*}
\Msg{* Happy TeXing!}
\Msg{*}
\Msg{************************************************************************}

\endbatchfile
%</install>
%<*ignore>
\fi
%</ignore>
%<*driver>
\NeedsTeXFormat{LaTeX2e}
\ProvidesFile{hologo.drv}%
  [2016/05/12 v1.11 A logo collection with bookmark support (HO)]%
\documentclass{ltxdoc}
\usepackage{holtxdoc}[2011/11/22]
\usepackage{hologo}[2016/05/12]
\usepackage{longtable}
\usepackage{array}
\usepackage{paralist}
%\usepackage[T1]{fontenc}
%\usepackage{lmodern}
\begin{document}
  \DocInput{hologo.dtx}%
\end{document}
%</driver>
% \fi
%
%
% \CharacterTable
%  {Upper-case    \A\B\C\D\E\F\G\H\I\J\K\L\M\N\O\P\Q\R\S\T\U\V\W\X\Y\Z
%   Lower-case    \a\b\c\d\e\f\g\h\i\j\k\l\m\n\o\p\q\r\s\t\u\v\w\x\y\z
%   Digits        \0\1\2\3\4\5\6\7\8\9
%   Exclamation   \!     Double quote  \"     Hash (number) \#
%   Dollar        \$     Percent       \%     Ampersand     \&
%   Acute accent  \'     Left paren    \(     Right paren   \)
%   Asterisk      \*     Plus          \+     Comma         \,
%   Minus         \-     Point         \.     Solidus       \/
%   Colon         \:     Semicolon     \;     Less than     \<
%   Equals        \=     Greater than  \>     Question mark \?
%   Commercial at \@     Left bracket  \[     Backslash     \\
%   Right bracket \]     Circumflex    \^     Underscore    \_
%   Grave accent  \`     Left brace    \{     Vertical bar  \|
%   Right brace   \}     Tilde         \~}
%
% \GetFileInfo{hologo.drv}
%
% \title{The \xpackage{hologo} package}
% \date{2016/05/12 v1.11}
% \author{Heiko Oberdiek\\\xemail{heiko.oberdiek at googlemail.com}}
%
% \maketitle
%
% \begin{abstract}
% This package starts a collection of logos with support for bookmarks
% strings.
% \end{abstract}
%
% \tableofcontents
%
% \section{Documentation}
%
% \subsection{Logo macros}
%
% \begin{declcs}{hologo} \M{name}
% \end{declcs}
% Macro \cs{hologo} sets the logo with name \meta{name}.
% The following table shows the supported names.
%
% \begingroup
%   \def\hologoEntry#1#2#3{^^A
%     #1&#2&\hologoLogoSetup{#1}{variant=#2}\hologo{#1}&#3\tabularnewline
%   }
%   \begin{longtable}{>{\ttfamily}l>{\ttfamily}lll}
%     \rmfamily\bfseries{name} & \rmfamily\bfseries variant
%     & \bfseries logo & \bfseries since\\
%     \hline
%     \endhead
%     \hologoList
%   \end{longtable}
% \endgroup
%
% \begin{declcs}{Hologo} \M{name}
% \end{declcs}
% Macro \cs{Hologo} starts the logo \meta{name} with an uppercase
% letter. As an exception small greek letters are not converted
% to uppercase. Examples, see \hologo{eTeX} and \hologo{ExTeX}.
%
% \subsection{Setup macros}
%
% The package does not support package options, but the following
% setup macros can be used to set options.
%
% \begin{declcs}{hologoSetup} \M{key value list}
% \end{declcs}
% Macro \cs{hologoSetup} sets global options.
%
% \begin{declcs}{hologoLogoSetup} \M{logo} \M{key value list}
% \end{declcs}
% Some options can also be used to configure a logo.
% These settings take precedence over global option settings.
%
% \subsection{Options}\label{sec:options}
%
% There are boolean and string options:
% \begin{description}
% \item[Boolean option:]
% It takes |true| or |false|
% as value. If the value is omitted, then |true| is used.
% \item[String option:]
% A value must be given as string. (But the string might be empty.)
% \end{description}
% The following options can be used both in \cs{hologoSetup}
% and \cs{hologoLogoSetup}:
% \begin{description}
% \def\entry#1{\item[\xoption{#1}:]}
% \entry{break}
%   enables or disables line breaks inside the logo. This setting is
%   refined by options \xoption{hyphenbreak}, \xoption{spacebreak}
%   or \xoption{discretionarybreak}.
%   Default is |false|.
% \entry{hyphenbreak}
%   enables or disables the line break right after the hyphen character.
% \entry{spacebreak}
%   enables or disables line breaks at space characters.
% \entry{discretionarybreak}
%   enables or disables line breaks at hyphenation points
%   (inserted by \cs{-}).
% \end{description}
% Macro \cs{hologoLogoSetup} also knows:
% \begin{description}
% \item[\xoption{variant}:]
%   This is a string option. It specifies a variant of a logo that
%   must exist. An empty string selects the package default variant.
% \end{description}
% Example:
% \begin{quote}
%   |\hologoSetup{break=false}|\\
%   |\hologoLogoSetup{plainTeX}{variant=hyphen,hyphenbreak}|\\
%   Then ``plain-\TeX'' contains one break point after the hyphen.
% \end{quote}
%
% \subsection{Driver options}
%
% Sometimes graphical operations are needed to construct some
% glyphs (e.g.\ \hologo{XeTeX}). If package \xpackage{graphics}
% or package \xpackage{pgf} are found, then the macros are taken
% from there. Otherwise the packge defines its own operations
% and therefore needs the driver information. Many drivers are
% detected automatically (\hologo{pdfTeX}/\hologo{LuaTeX}
% in PDF mode, \hologo{XeTeX}, \hologo{VTeX}). These have precedence
% over a driver option. The driver can be given as package option
% or using \cs{hologoDriverSetup}.
% The following list contains the recognized driver options:
% \begin{itemize}
% \item \xoption{pdftex}, \xoption{luatex}
% \item \xoption{dvipdfm}, \xoption{dvipdfmx}
% \item \xoption{dvips}, \xoption{dvipsone}, \xoption{xdvi}
% \item \xoption{xetex}
% \item \xoption{vtex}
% \end{itemize}
% The left driver of a line is the driver name that is used internally.
% The following names are aliases for drivers that use the
% same method. Therefore the entry in the \xext{log} file for
% the used driver prints the internally used driver name.
% \begin{description}
% \item[\xoption{driverfallback}:]
%   This option expects a driver that is used,
%   if the driver could not be detected automatically.
% \end{description}
%
% \begin{declcs}{hologoDriverSetup} \M{driver option}
% \end{declcs}
% The driver can also be configured after package loading
% using \cs{hologoDriverSetup}, also the way for \hologo{plainTeX}
% to setup the driver.
%
% \subsection{Font setup}
%
% Some logos require a special font, but should also be usable by
% \hologo{plainTeX}. Therefore the package provides some ways
% to influence the font settings. The options below
% take font settings as values. Both font commands
% such as \cs{sffamily} and macros that take one argument
% like \cs{textsf} can be used.
%
% \begin{declcs}{hologoFontSetup} \M{key value list}
% \end{declcs}
% Macro \cs{hologoFontSetup} sets the fonts for all logos.
% Supported keys:
% \begin{description}
% \def\entry#1{\item[\xoption{#1}:]}
% \entry{general}
%   This font is used for all logos. The default is empty.
%   That means no special font is used.
% \entry{bibsf}
%   This font is used for
%   {\hologoLogoSetup{BibTeX}{variant=sf}\hologo{BibTeX}}
%   with variant \xoption{sf}.
% \entry{rm}
%   This font is a serif font. It is used for \hologo{ExTeX}.
% \entry{sc}
%   This font specifies a small caps font. It is used for
%   {\hologoLogoSetup{BibTeX}{variant=sc}\hologo{BibTeX}}
%   with variant \xoption{sc}.
% \entry{sf}
%   This font specifies a sans serif font. The default
%   is \cs{sffamily}, then \cs{sf} is tried. Otherwise
%   a warning is given. It is used by \hologo{KOMAScript}.
% \entry{sy}
%   This is the font for math symbols (e.g. cmsy).
%   It is used by \hologo{AmS}, \hologo{NTS}, \hologo{ExTeX}.
% \entry{logo}
%   \hologo{METAFONT} and \hologo{METAPOST} are using that font.
%   In \hologo{LaTeX} \cs{logofamily} is used and
%   the definitions of package \xpackage{mflogo} are used
%   if the package is not loaded.
%   Otherwise the \cs{tenlogo} is used and defined
%   if it does not already exists.
% \end{description}
%
% \begin{declcs}{hologoLogoFontSetup} \M{logo} \M{key value list}
% \end{declcs}
% Fonts can also be set for a logo or logo component separately,
% see the following list.
% The keys are the same as for \cs{hologoFontSetup}.
%
% \begin{longtable}{>{\ttfamily}l>{\sffamily}ll}
%   \meta{logo} & keys & result\\
%   \hline
%   \endhead
%   BibTeX & bibsf & {\hologoLogoSetup{BibTeX}{variant=sf}\hologo{BibTeX}}\\[.5ex]
%   BibTeX & sc & {\hologoLogoSetup{BibTeX}{variant=sc}\hologo{BibTeX}}\\[.5ex]
%   ExTeX & rm & \hologo{ExTeX}\\
%   SliTeX & rm & \hologo{SliTeX}\\[.5ex]
%   AmS & sy & \hologo{AmS}\\
%   ExTeX & sy & \hologo{ExTeX}\\
%   NTS & sy & \hologo{NTS}\\[.5ex]
%   KOMAScript & sf & \hologo{KOMAScript}\\[.5ex]
%   METAFONT & logo & \hologo{METAFONT}\\
%   METAPOST & logo & \hologo{METAPOST}\\[.5ex]
%   SliTeX & sc \hologo{SliTeX}
% \end{longtable}
%
% \subsubsection{Font order}
%
% For all logos the font \xoption{general} is applied first.
% Example:
%\begin{quote}
%|\hologoFontSetup{general=\color{red}}|
%\end{quote}
% will print red logos.
% Then if the font uses a special font \xoption{sf}, for example,
% the font is applied that is setup by \cs{hologoLogoFontSetup}.
% If this font is not setup, then the common font setup
% by \cs{hologoFontSetup} is used. Otherwise a warning is given,
% that there is no font configured.
%
% \subsection{Additional user macros}
%
% Usually a variant of a logo is configured by using
% \cs{hologoLogoSetup}, because it is bad style to mix
% different variants of the same logo in the same text.
% There the following macros are a convenience for testing.
%
% \begin{declcs}{hologoVariant} \M{name} \M{variant}\\
%   \cs{HologoVariant} \M{name} \M{variant}
% \end{declcs}
% Logo \meta{name} is set using \meta{variant} that specifies
% explicitely which variant of the macro is used. If the argument
% is empty, then the default form of the logo is used
% (configurable by \cs{hologoLogoSetup}).
%
% \cs{HologoVariant} is used if the logo is set in a context
% that needs an uppercase first letter (beginning of a sentence, \dots).
%
% \begin{declcs}{hologoList}\\
%   \cs{hologoEntry} \M{logo} \M{variant} \M{since}
% \end{declcs}
% Macro \cs{hologoList} contains all logos that are provided
% by the package including variants. The list consists of calls
% of \cs{hologoEntry} with three arguments starting with the
% logo name \meta{logo} and its variant \meta{variant}. An empty
% variant means the current default. Argument \meta{since} specifies
% with version of the package \xpackage{hologo} is needed to get
% the logo. If the logo is fixed, then the date gets updated.
% Therefore the date \meta{since} is not exactly the date of
% the first introduction, but rather the date of the latest fix.
%
% Before \cs{hologoList} can be used, macro \cs{hologoEntry} needs
% a definition. The example file in section \ref{sec:example}
% shows applications of \cs{hologoList}.
%
% \subsection{Supported contexts}
%
% Macros \cs{hologo} and friends support special contexts:
% \begin{itemize}
% \item \hologo{LaTeX}'s protection mechanism.
% \item Bookmarks of package \xpackage{hyperref}.
% \item Package \xpackage{tex4ht}.
% \item The macros can be used inside \cs{csname} constructs,
%   if \cs{ifincsname} is available (\hologo{pdfTeX}, \hologo{XeTeX},
%   \hologo{LuaTeX}).
% \end{itemize}
%
% \subsection{Example}
% \label{sec:example}
%
% The following example prints the logos in different fonts.
%    \begin{macrocode}
%<*example>
%<<verbatim
\NeedsTeXFormat{LaTeX2e}
\documentclass[a4paper]{article}
\usepackage[
  hmargin=20mm,
  vmargin=20mm,
]{geometry}
\pagestyle{empty}
\usepackage{hologo}[2016/05/12]
\usepackage{longtable}
\usepackage{array}
\setlength{\extrarowheight}{2pt}
\usepackage[T1]{fontenc}
\usepackage{lmodern}
\usepackage{pdflscape}
\usepackage[
  pdfencoding=auto,
]{hyperref}
\hypersetup{
  pdfauthor={Heiko Oberdiek},
  pdftitle={Example for package `hologo'},
  pdfsubject={Logos with fonts lmr, lmss, qtm, qpl, qhv},
}
\usepackage{bookmark}

% Print the logo list on the console

\begingroup
  \typeout{}%
  \typeout{*** Begin of logo list ***}%
  \newcommand*{\hologoEntry}[3]{%
    \typeout{#1 \ifx\\#2\\\else(#2) \fi[#3]}%
  }%
  \hologoList
  \typeout{*** End of logo list ***}%
  \typeout{}%
\endgroup

\begin{document}
\begin{landscape}

  \section{Example file for package `hologo'}

  % Table for font names

  \begin{longtable}{>{\bfseries}ll}
    \textbf{font} & \textbf{Font name}\\
    \hline
    lmr & Latin Modern Roman\\
    lmss & Latin Modern Sans\\
    qtm & \TeX\ Gyre Termes\\
    qhv & \TeX\ Gyre Heros\\
    qpl & \TeX\ Gyre Pagella\\
  \end{longtable}

  % Logo list with logos in different fonts

  \begingroup
    \newcommand*{\SetVariant}[2]{%
      \ifx\\#2\\%
      \else
        \hologoLogoSetup{#1}{variant=#2}%
      \fi
    }%
    \newcommand*{\hologoEntry}[3]{%
      \SetVariant{#1}{#2}%
      \raisebox{1em}[0pt][0pt]{\hypertarget{#1@#2}{}}%
      \bookmark[%
        dest={#1@#2},%
      ]{%
        #1\ifx\\#2\\\else\space(#2)\fi: \Hologo{#1}, \hologo{#1} %
        [Unicode]%
      }%
      \hypersetup{unicode=false}%
      \bookmark[%
        dest={#1@#2},%
      ]{%
        #1\ifx\\#2\\\else\space(#2)\fi: \Hologo{#1}, \hologo{#1} %
        [PDFDocEncoding]%
      }%
      \texttt{#1}%
      &%
      \texttt{#2}%
      &%
      \Hologo{#1}%
      &%
      \SetVariant{#1}{#2}%
      \hologo{#1}%
      &%
      \SetVariant{#1}{#2}%
      \fontfamily{qtm}\selectfont
      \hologo{#1}%
      &%
      \SetVariant{#1}{#2}%
      \fontfamily{qpl}\selectfont
      \hologo{#1}%
      &%
      \SetVariant{#1}{#2}%
      \textsf{\hologo{#1}}%
      &%
      \SetVariant{#1}{#2}%
      \fontfamily{qhv}\selectfont
      \hologo{#1}%
      \tabularnewline
    }%
    \begin{longtable}{llllllll}%
      \textbf{\textit{logo}} & \textbf{\textit{variant}} &
      \texttt{\string\Hologo} &
      \textbf{lmr} & \textbf{qtm} & \textbf{qpl} &
      \textbf{lmss} & \textbf{qhv}
      \tabularnewline
      \hline
      \endhead
      \hologoList
    \end{longtable}%
  \endgroup

\end{landscape}
\end{document}
%verbatim
%</example>
%    \end{macrocode}
%
% \StopEventually{
% }
%
% \section{Implementation}
%    \begin{macrocode}
%<*package>
%    \end{macrocode}
%    Reload check, especially if the package is not used with \LaTeX.
%    \begin{macrocode}
\begingroup\catcode61\catcode48\catcode32=10\relax%
  \catcode13=5 % ^^M
  \endlinechar=13 %
  \catcode35=6 % #
  \catcode39=12 % '
  \catcode44=12 % ,
  \catcode45=12 % -
  \catcode46=12 % .
  \catcode58=12 % :
  \catcode64=11 % @
  \catcode123=1 % {
  \catcode125=2 % }
  \expandafter\let\expandafter\x\csname ver@hologo.sty\endcsname
  \ifx\x\relax % plain-TeX, first loading
  \else
    \def\empty{}%
    \ifx\x\empty % LaTeX, first loading,
      % variable is initialized, but \ProvidesPackage not yet seen
    \else
      \expandafter\ifx\csname PackageInfo\endcsname\relax
        \def\x#1#2{%
          \immediate\write-1{Package #1 Info: #2.}%
        }%
      \else
        \def\x#1#2{\PackageInfo{#1}{#2, stopped}}%
      \fi
      \x{hologo}{The package is already loaded}%
      \aftergroup\endinput
    \fi
  \fi
\endgroup%
%    \end{macrocode}
%    Package identification:
%    \begin{macrocode}
\begingroup\catcode61\catcode48\catcode32=10\relax%
  \catcode13=5 % ^^M
  \endlinechar=13 %
  \catcode35=6 % #
  \catcode39=12 % '
  \catcode40=12 % (
  \catcode41=12 % )
  \catcode44=12 % ,
  \catcode45=12 % -
  \catcode46=12 % .
  \catcode47=12 % /
  \catcode58=12 % :
  \catcode64=11 % @
  \catcode91=12 % [
  \catcode93=12 % ]
  \catcode123=1 % {
  \catcode125=2 % }
  \expandafter\ifx\csname ProvidesPackage\endcsname\relax
    \def\x#1#2#3[#4]{\endgroup
      \immediate\write-1{Package: #3 #4}%
      \xdef#1{#4}%
    }%
  \else
    \def\x#1#2[#3]{\endgroup
      #2[{#3}]%
      \ifx#1\@undefined
        \xdef#1{#3}%
      \fi
      \ifx#1\relax
        \xdef#1{#3}%
      \fi
    }%
  \fi
\expandafter\x\csname ver@hologo.sty\endcsname
\ProvidesPackage{hologo}%
  [2016/05/12 v1.11 A logo collection with bookmark support (HO)]%
%    \end{macrocode}
%
%    \begin{macrocode}
\begingroup\catcode61\catcode48\catcode32=10\relax%
  \catcode13=5 % ^^M
  \endlinechar=13 %
  \catcode123=1 % {
  \catcode125=2 % }
  \catcode64=11 % @
  \def\x{\endgroup
    \expandafter\edef\csname HOLOGO@AtEnd\endcsname{%
      \endlinechar=\the\endlinechar\relax
      \catcode13=\the\catcode13\relax
      \catcode32=\the\catcode32\relax
      \catcode35=\the\catcode35\relax
      \catcode61=\the\catcode61\relax
      \catcode64=\the\catcode64\relax
      \catcode123=\the\catcode123\relax
      \catcode125=\the\catcode125\relax
    }%
  }%
\x\catcode61\catcode48\catcode32=10\relax%
\catcode13=5 % ^^M
\endlinechar=13 %
\catcode35=6 % #
\catcode64=11 % @
\catcode123=1 % {
\catcode125=2 % }
\def\TMP@EnsureCode#1#2{%
  \edef\HOLOGO@AtEnd{%
    \HOLOGO@AtEnd
    \catcode#1=\the\catcode#1\relax
  }%
  \catcode#1=#2\relax
}
\TMP@EnsureCode{10}{12}% ^^J
\TMP@EnsureCode{33}{12}% !
\TMP@EnsureCode{34}{12}% "
\TMP@EnsureCode{36}{3}% $
\TMP@EnsureCode{38}{4}% &
\TMP@EnsureCode{39}{12}% '
\TMP@EnsureCode{40}{12}% (
\TMP@EnsureCode{41}{12}% )
\TMP@EnsureCode{42}{12}% *
\TMP@EnsureCode{43}{12}% +
\TMP@EnsureCode{44}{12}% ,
\TMP@EnsureCode{45}{12}% -
\TMP@EnsureCode{46}{12}% .
\TMP@EnsureCode{47}{12}% /
\TMP@EnsureCode{58}{12}% :
\TMP@EnsureCode{59}{12}% ;
\TMP@EnsureCode{60}{12}% <
\TMP@EnsureCode{62}{12}% >
\TMP@EnsureCode{63}{12}% ?
\TMP@EnsureCode{91}{12}% [
\TMP@EnsureCode{93}{12}% ]
\TMP@EnsureCode{94}{7}% ^ (superscript)
\TMP@EnsureCode{95}{8}% _ (subscript)
\TMP@EnsureCode{96}{12}% `
\TMP@EnsureCode{124}{12}% |
\edef\HOLOGO@AtEnd{%
  \HOLOGO@AtEnd
  \escapechar\the\escapechar\relax
  \noexpand\endinput
}
\escapechar=92 %
%    \end{macrocode}
%
% \subsection{Logo list}
%
%    \begin{macro}{\hologoList}
%    \begin{macrocode}
\def\hologoList{%
  \hologoEntry{(La)TeX}{}{2011/10/01}%
  \hologoEntry{AmSLaTeX}{}{2010/04/16}%
  \hologoEntry{AmSTeX}{}{2010/04/16}%
  \hologoEntry{biber}{}{2011/10/01}%
  \hologoEntry{BibTeX}{}{2011/10/01}%
  \hologoEntry{BibTeX}{sf}{2011/10/01}%
  \hologoEntry{BibTeX}{sc}{2011/10/01}%
  \hologoEntry{BibTeX8}{}{2011/11/22}%
  \hologoEntry{ConTeXt}{}{2011/03/25}%
  \hologoEntry{ConTeXt}{narrow}{2011/03/25}%
  \hologoEntry{ConTeXt}{simple}{2011/03/25}%
  \hologoEntry{emTeX}{}{2010/04/26}%
  \hologoEntry{eTeX}{}{2010/04/08}%
  \hologoEntry{ExTeX}{}{2011/10/01}%
  \hologoEntry{HanTheThanh}{}{2011/11/29}%
  \hologoEntry{iniTeX}{}{2011/10/01}%
  \hologoEntry{KOMAScript}{}{2011/10/01}%
  \hologoEntry{La}{}{2010/05/08}%
  \hologoEntry{LaTeX}{}{2010/04/08}%
  \hologoEntry{LaTeX2e}{}{2010/04/08}%
  \hologoEntry{LaTeX3}{}{2010/04/24}%
  \hologoEntry{LaTeXe}{}{2010/04/08}%
  \hologoEntry{LaTeXML}{}{2011/11/22}%
  \hologoEntry{LaTeXTeX}{}{2011/10/01}%
  \hologoEntry{LuaLaTeX}{}{2010/04/08}%
  \hologoEntry{LuaTeX}{}{2010/04/08}%
  \hologoEntry{LyX}{}{2011/10/01}%
  \hologoEntry{METAFONT}{}{2011/10/01}%
  \hologoEntry{MetaFun}{}{2011/10/01}%
  \hologoEntry{METAPOST}{}{2011/10/01}%
  \hologoEntry{MetaPost}{}{2011/10/01}%
  \hologoEntry{MiKTeX}{}{2011/10/01}%
  \hologoEntry{NTS}{}{2011/10/01}%
  \hologoEntry{OzMF}{}{2011/10/01}%
  \hologoEntry{OzMP}{}{2011/10/01}%
  \hologoEntry{OzTeX}{}{2011/10/01}%
  \hologoEntry{OzTtH}{}{2011/10/01}%
  \hologoEntry{PCTeX}{}{2011/10/01}%
  \hologoEntry{pdfTeX}{}{2011/10/01}%
  \hologoEntry{pdfLaTeX}{}{2011/10/01}%
  \hologoEntry{PiC}{}{2011/10/01}%
  \hologoEntry{PiCTeX}{}{2011/10/01}%
  \hologoEntry{plainTeX}{}{2010/04/08}%
  \hologoEntry{plainTeX}{space}{2010/04/16}%
  \hologoEntry{plainTeX}{hyphen}{2010/04/16}%
  \hologoEntry{plainTeX}{runtogether}{2010/04/16}%
  \hologoEntry{SageTeX}{}{2011/11/22}%
  \hologoEntry{SLiTeX}{}{2011/10/01}%
  \hologoEntry{SLiTeX}{lift}{2011/10/01}%
  \hologoEntry{SLiTeX}{narrow}{2011/10/01}%
  \hologoEntry{SLiTeX}{simple}{2011/10/01}%
  \hologoEntry{SliTeX}{}{2011/10/01}%
  \hologoEntry{SliTeX}{narrow}{2011/10/01}%
  \hologoEntry{SliTeX}{simple}{2011/10/01}%
  \hologoEntry{SliTeX}{lift}{2011/10/01}%
  \hologoEntry{teTeX}{}{2011/10/01}%
  \hologoEntry{TeX}{}{2010/04/08}%
  \hologoEntry{TeX4ht}{}{2011/11/22}%
  \hologoEntry{TTH}{}{2011/11/22}%
  \hologoEntry{virTeX}{}{2011/10/01}%
  \hologoEntry{VTeX}{}{2010/04/24}%
  \hologoEntry{Xe}{}{2010/04/08}%
  \hologoEntry{XeLaTeX}{}{2010/04/08}%
  \hologoEntry{XeTeX}{}{2010/04/08}%
}
%    \end{macrocode}
%    \end{macro}
%
% \subsection{Load resources}
%
%    \begin{macrocode}
\begingroup\expandafter\expandafter\expandafter\endgroup
\expandafter\ifx\csname RequirePackage\endcsname\relax
  \def\TMP@RequirePackage#1[#2]{%
    \begingroup\expandafter\expandafter\expandafter\endgroup
    \expandafter\ifx\csname ver@#1.sty\endcsname\relax
      \input #1.sty\relax
    \fi
  }%
  \TMP@RequirePackage{ltxcmds}[2011/02/04]%
  \TMP@RequirePackage{infwarerr}[2010/04/08]%
  \TMP@RequirePackage{kvsetkeys}[2010/03/01]%
  \TMP@RequirePackage{kvdefinekeys}[2010/03/01]%
  \TMP@RequirePackage{pdftexcmds}[2010/04/01]%
  \TMP@RequirePackage{ifpdf}[2010/01/28]%
  \TMP@RequirePackage{ifluatex}[2010/03/01]%
  \ltx@IfUndefined{newif}{%
    \expandafter\let\csname newif\endcsname\ltx@newif
  }{}%
  \TMP@RequirePackage{ifxetex}[2009/01/23]%
  \TMP@RequirePackage{ifvtex}[2010/03/01]%
\else
  \RequirePackage{ltxcmds}[2011/02/04]%
  \RequirePackage{infwarerr}[2010/04/08]%
  \RequirePackage{kvsetkeys}[2010/03/01]%
  \RequirePackage{kvdefinekeys}[2010/03/01]%
  \RequirePackage{pdftexcmds}[2010/04/01]%
  \RequirePackage{ifpdf}[2010/01/28]%
  \RequirePackage{ifluatex}[2010/03/01]%
  \RequirePackage{ifxetex}[2009/01/23]%
  \RequirePackage{ifvtex}[2010/03/01]%
\fi
%    \end{macrocode}
%
%    \begin{macro}{\HOLOGO@IfDefined}
%    \begin{macrocode}
\def\HOLOGO@IfExists#1{%
  \ifx\@undefined#1%
    \expandafter\ltx@secondoftwo
  \else
    \ifx\relax#1%
      \expandafter\ltx@secondoftwo
    \else
      \expandafter\expandafter\expandafter\ltx@firstoftwo
    \fi
  \fi
}
%    \end{macrocode}
%    \end{macro}
%
% \subsection{Setup macros}
%
%    \begin{macro}{\hologoSetup}
%    \begin{macrocode}
\def\hologoSetup{%
  \let\HOLOGO@name\relax
  \HOLOGO@Setup
}
%    \end{macrocode}
%    \end{macro}
%
%    \begin{macro}{\hologoLogoSetup}
%    \begin{macrocode}
\def\hologoLogoSetup#1{%
  \edef\HOLOGO@name{#1}%
  \ltx@IfUndefined{HoLogo@\HOLOGO@name}{%
    \@PackageError{hologo}{%
      Unknown logo `\HOLOGO@name'%
    }\@ehc
    \ltx@gobble
  }{%
    \HOLOGO@Setup
  }%
}
%    \end{macrocode}
%    \end{macro}
%
%    \begin{macro}{\HOLOGO@Setup}
%    \begin{macrocode}
\def\HOLOGO@Setup{%
  \kvsetkeys{HoLogo}%
}
%    \end{macrocode}
%    \end{macro}
%
% \subsection{Options}
%
%    \begin{macro}{\HOLOGO@DeclareBoolOption}
%    \begin{macrocode}
\def\HOLOGO@DeclareBoolOption#1{%
  \expandafter\chardef\csname HOLOGOOPT@#1\endcsname\ltx@zero
  \kv@define@key{HoLogo}{#1}[true]{%
    \def\HOLOGO@temp{##1}%
    \ifx\HOLOGO@temp\HOLOGO@true
      \ifx\HOLOGO@name\relax
        \expandafter\chardef\csname HOLOGOOPT@#1\endcsname=\ltx@one
      \else
        \expandafter\chardef\csname
        HoLogoOpt@#1@\HOLOGO@name\endcsname\ltx@one
      \fi
      \HOLOGO@SetBreakAll{#1}%
    \else
      \ifx\HOLOGO@temp\HOLOGO@false
        \ifx\HOLOGO@name\relax
          \expandafter\chardef\csname HOLOGOOPT@#1\endcsname=\ltx@zero
        \else
          \expandafter\chardef\csname
          HoLogoOpt@#1@\HOLOGO@name\endcsname=\ltx@zero
        \fi
        \HOLOGO@SetBreakAll{#1}%
      \else
        \@PackageError{hologo}{%
          Unknown value `##1' for boolean option `#1'.\MessageBreak
          Known values are `true' and `false'%
        }\@ehc
      \fi
    \fi
  }%
}
%    \end{macrocode}
%    \end{macro}
%
%    \begin{macro}{\HOLOGO@SetBreakAll}
%    \begin{macrocode}
\def\HOLOGO@SetBreakAll#1{%
  \def\HOLOGO@temp{#1}%
  \ifx\HOLOGO@temp\HOLOGO@break
    \ifx\HOLOGO@name\relax
      \chardef\HOLOGOOPT@hyphenbreak=\HOLOGOOPT@break
      \chardef\HOLOGOOPT@spacebreak=\HOLOGOOPT@break
      \chardef\HOLOGOOPT@discretionarybreak=\HOLOGOOPT@break
    \else
      \expandafter\chardef
         \csname HoLogoOpt@hyphenbreak@\HOLOGO@name\endcsname=%
         \csname HoLogoOpt@break@\HOLOGO@name\endcsname
      \expandafter\chardef
         \csname HoLogoOpt@spacebreak@\HOLOGO@name\endcsname=%
         \csname HoLogoOpt@break@\HOLOGO@name\endcsname
      \expandafter\chardef
         \csname HoLogoOpt@discretionarybreak@\HOLOGO@name
             \endcsname=%
         \csname HoLogoOpt@break@\HOLOGO@name\endcsname
    \fi
  \fi
}
%    \end{macrocode}
%    \end{macro}
%
%    \begin{macro}{\HOLOGO@true}
%    \begin{macrocode}
\def\HOLOGO@true{true}
%    \end{macrocode}
%    \end{macro}
%    \begin{macro}{\HOLOGO@false}
%    \begin{macrocode}
\def\HOLOGO@false{false}
%    \end{macrocode}
%    \end{macro}
%    \begin{macro}{\HOLOGO@break}
%    \begin{macrocode}
\def\HOLOGO@break{break}
%    \end{macrocode}
%    \end{macro}
%
%    \begin{macrocode}
\HOLOGO@DeclareBoolOption{break}
\HOLOGO@DeclareBoolOption{hyphenbreak}
\HOLOGO@DeclareBoolOption{spacebreak}
\HOLOGO@DeclareBoolOption{discretionarybreak}
%    \end{macrocode}
%
%    \begin{macrocode}
\kv@define@key{HoLogo}{variant}{%
  \ifx\HOLOGO@name\relax
    \@PackageError{hologo}{%
      Option `variant' is not available in \string\hologoSetup,%
      \MessageBreak
      Use \string\hologoLogoSetup\space instead%
    }\@ehc
  \else
    \edef\HOLOGO@temp{#1}%
    \ifx\HOLOGO@temp\ltx@empty
      \expandafter
      \let\csname HoLogoOpt@variant@\HOLOGO@name\endcsname\@undefined
    \else
      \ltx@IfUndefined{HoLogo@\HOLOGO@name @\HOLOGO@temp}{%
        \@PackageError{hologo}{%
          Unknown variant `\HOLOGO@temp' of logo `\HOLOGO@name'%
        }\@ehc
      }{%
        \expandafter
        \let\csname HoLogoOpt@variant@\HOLOGO@name\endcsname
            \HOLOGO@temp
      }%
    \fi
  \fi
}
%    \end{macrocode}
%
%    \begin{macro}{\HOLOGO@Variant}
%    \begin{macrocode}
\def\HOLOGO@Variant#1{%
  #1%
  \ltx@ifundefined{HoLogoOpt@variant@#1}{%
  }{%
    @\csname HoLogoOpt@variant@#1\endcsname
  }%
}
%    \end{macrocode}
%    \end{macro}
%
% \subsection{Break/no-break support}
%
%    \begin{macro}{\HOLOGO@space}
%    \begin{macrocode}
\def\HOLOGO@space{%
  \ltx@ifundefined{HoLogoOpt@spacebreak@\HOLOGO@name}{%
    \ltx@ifundefined{HoLogoOpt@break@\HOLOGO@name}{%
      \chardef\HOLOGO@temp=\HOLOGOOPT@spacebreak
    }{%
      \chardef\HOLOGO@temp=%
        \csname HoLogoOpt@break@\HOLOGO@name\endcsname
    }%
  }{%
    \chardef\HOLOGO@temp=%
      \csname HoLogoOpt@spacebreak@\HOLOGO@name\endcsname
  }%
  \ifcase\HOLOGO@temp
    \penalty10000 %
  \fi
  \ltx@space
}
%    \end{macrocode}
%    \end{macro}
%
%    \begin{macro}{\HOLOGO@hyphen}
%    \begin{macrocode}
\def\HOLOGO@hyphen{%
  \ltx@ifundefined{HoLogoOpt@hyphenbreak@\HOLOGO@name}{%
    \ltx@ifundefined{HoLogoOpt@break@\HOLOGO@name}{%
      \chardef\HOLOGO@temp=\HOLOGOOPT@hyphenbreak
    }{%
      \chardef\HOLOGO@temp=%
        \csname HoLogoOpt@break@\HOLOGO@name\endcsname
    }%
  }{%
    \chardef\HOLOGO@temp=%
      \csname HoLogoOpt@hyphenbreak@\HOLOGO@name\endcsname
  }%
  \ifcase\HOLOGO@temp
    \ltx@mbox{-}%
  \else
    -%
  \fi
}
%    \end{macrocode}
%    \end{macro}
%
%    \begin{macro}{\HOLOGO@discretionary}
%    \begin{macrocode}
\def\HOLOGO@discretionary{%
  \ltx@ifundefined{HoLogoOpt@discretionarybreak@\HOLOGO@name}{%
    \ltx@ifundefined{HoLogoOpt@break@\HOLOGO@name}{%
      \chardef\HOLOGO@temp=\HOLOGOOPT@discretionarybreak
    }{%
      \chardef\HOLOGO@temp=%
        \csname HoLogoOpt@break@\HOLOGO@name\endcsname
    }%
  }{%
    \chardef\HOLOGO@temp=%
      \csname HoLogoOpt@discretionarybreak@\HOLOGO@name\endcsname
  }%
  \ifcase\HOLOGO@temp
  \else
    \-%
  \fi
}
%    \end{macrocode}
%    \end{macro}
%
%    \begin{macro}{\HOLOGO@mbox}
%    \begin{macrocode}
\def\HOLOGO@mbox#1{%
  \ltx@ifundefined{HoLogoOpt@break@\HOLOGO@name}{%
    \chardef\HOLOGO@temp=\HOLOGOOPT@hyphenbreak
  }{%
    \chardef\HOLOGO@temp=%
      \csname HoLogoOpt@break@\HOLOGO@name\endcsname
  }%
  \ifcase\HOLOGO@temp
    \ltx@mbox{#1}%
  \else
    #1%
  \fi
}
%    \end{macrocode}
%    \end{macro}
%
% \subsection{Font support}
%
%    \begin{macro}{\HoLogoFont@font}
%    \begin{tabular}{@{}ll@{}}
%    |#1|:& logo name\\
%    |#2|:& font short name\\
%    |#3|:& text
%    \end{tabular}
%    \begin{macrocode}
\def\HoLogoFont@font#1#2#3{%
  \begingroup
    \ltx@IfUndefined{HoLogoFont@logo@#1.#2}{%
      \ltx@IfUndefined{HoLogoFont@font@#2}{%
        \@PackageWarning{hologo}{%
          Missing font `#2' for logo `#1'%
        }%
        #3%
      }{%
        \csname HoLogoFont@font@#2\endcsname{#3}%
      }%
    }{%
      \csname HoLogoFont@logo@#1.#2\endcsname{#3}%
    }%
  \endgroup
}
%    \end{macrocode}
%    \end{macro}
%
%    \begin{macro}{\HoLogoFont@Def}
%    \begin{macrocode}
\def\HoLogoFont@Def#1{%
  \expandafter\def\csname HoLogoFont@font@#1\endcsname
}
%    \end{macrocode}
%    \end{macro}
%    \begin{macro}{\HoLogoFont@LogoDef}
%    \begin{macrocode}
\def\HoLogoFont@LogoDef#1#2{%
  \expandafter\def\csname HoLogoFont@logo@#1.#2\endcsname
}
%    \end{macrocode}
%    \end{macro}
%
% \subsubsection{Font defaults}
%
%    \begin{macro}{\HoLogoFont@font@general}
%    \begin{macrocode}
\HoLogoFont@Def{general}{}%
%    \end{macrocode}
%    \end{macro}
%
%    \begin{macro}{\HoLogoFont@font@rm}
%    \begin{macrocode}
\ltx@IfUndefined{rmfamily}{%
  \ltx@IfUndefined{rm}{%
  }{%
    \HoLogoFont@Def{rm}{\rm}%
  }%
}{%
  \HoLogoFont@Def{rm}{\rmfamily}%
}
%    \end{macrocode}
%    \end{macro}
%
%    \begin{macro}{\HoLogoFont@font@sf}
%    \begin{macrocode}
\ltx@IfUndefined{sffamily}{%
  \ltx@IfUndefined{sf}{%
  }{%
    \HoLogoFont@Def{sf}{\sf}%
  }%
}{%
  \HoLogoFont@Def{sf}{\sffamily}%
}
%    \end{macrocode}
%    \end{macro}
%
%    \begin{macro}{\HoLogoFont@font@bibsf}
%    In case of \hologo{plainTeX} the original small caps
%    variant is used as default. In \hologo{LaTeX}
%    the definition of package \xpackage{dtklogos} \cite{dtklogos}
%    is used.
%\begin{quote}
%\begin{verbatim}
%\DeclareRobustCommand{\BibTeX}{%
%  B%
%  \kern-.05em%
%  \hbox{%
%    $\m@th$% %% force math size calculations
%    \csname S@\f@size\endcsname
%    \fontsize\sf@size\z@
%    \math@fontsfalse
%    \selectfont
%    I%
%    \kern-.025em%
%    B
%  }%
%  \kern-.08em%
%  \-%
%  \TeX
%}
%\end{verbatim}
%\end{quote}
%    \begin{macrocode}
\ltx@IfUndefined{selectfont}{%
  \ltx@IfUndefined{tensc}{%
    \font\tensc=cmcsc10\relax
  }{}%
  \HoLogoFont@Def{bibsf}{\tensc}%
}{%
  \HoLogoFont@Def{bibsf}{%
    $\mathsurround=0pt$%
    \csname S@\f@size\endcsname
    \fontsize\sf@size{0pt}%
    \math@fontsfalse
    \selectfont
  }%
}
%    \end{macrocode}
%    \end{macro}
%
%    \begin{macro}{\HoLogoFont@font@sc}
%    \begin{macrocode}
\ltx@IfUndefined{scshape}{%
  \ltx@IfUndefined{tensc}{%
    \font\tensc=cmcsc10\relax
  }{}%
  \HoLogoFont@Def{sc}{\tensc}%
}{%
  \HoLogoFont@Def{sc}{\scshape}%
}
%    \end{macrocode}
%    \end{macro}
%
%    \begin{macro}{\HoLogoFont@font@sy}
%    \begin{macrocode}
\ltx@IfUndefined{usefont}{%
  \ltx@IfUndefined{tensy}{%
  }{%
    \HoLogoFont@Def{sy}{\tensy}%
  }%
}{%
  \HoLogoFont@Def{sy}{%
    \usefont{OMS}{cmsy}{m}{n}%
  }%
}
%    \end{macrocode}
%    \end{macro}
%
%    \begin{macro}{\HoLogoFont@font@logo}
%    \begin{macrocode}
\begingroup
  \def\x{LaTeX2e}%
\expandafter\endgroup
\ifx\fmtname\x
  \ltx@IfUndefined{logofamily}{%
    \DeclareRobustCommand\logofamily{%
      \not@math@alphabet\logofamily\relax
      \fontencoding{U}%
      \fontfamily{logo}%
      \selectfont
    }%
  }{}%
  \ltx@IfUndefined{logofamily}{%
  }{%
    \HoLogoFont@Def{logo}{\logofamily}%
  }%
\else
  \ltx@IfUndefined{tenlogo}{%
    \font\tenlogo=logo10\relax
  }{}%
  \HoLogoFont@Def{logo}{\tenlogo}%
\fi
%    \end{macrocode}
%    \end{macro}
%
% \subsubsection{Font setup}
%
%    \begin{macro}{\hologoFontSetup}
%    \begin{macrocode}
\def\hologoFontSetup{%
  \let\HOLOGO@name\relax
  \HOLOGO@FontSetup
}
%    \end{macrocode}
%    \end{macro}
%
%    \begin{macro}{\hologoLogoFontSetup}
%    \begin{macrocode}
\def\hologoLogoFontSetup#1{%
  \edef\HOLOGO@name{#1}%
  \ltx@IfUndefined{HoLogo@\HOLOGO@name}{%
    \@PackageError{hologo}{%
      Unknown logo `\HOLOGO@name'%
    }\@ehc
    \ltx@gobble
  }{%
    \HOLOGO@FontSetup
  }%
}
%    \end{macrocode}
%    \end{macro}
%
%    \begin{macro}{\HOLOGO@FontSetup}
%    \begin{macrocode}
\def\HOLOGO@FontSetup{%
  \kvsetkeys{HoLogoFont}%
}
%    \end{macrocode}
%    \end{macro}
%
%    \begin{macrocode}
\def\HOLOGO@temp#1{%
  \kv@define@key{HoLogoFont}{#1}{%
    \ifx\HOLOGO@name\relax
      \HoLogoFont@Def{#1}{##1}%
    \else
      \HoLogoFont@LogoDef\HOLOGO@name{#1}{##1}%
    \fi
  }%
}
\HOLOGO@temp{general}
\HOLOGO@temp{sf}
%    \end{macrocode}
%
% \subsection{Generic logo commands}
%
%    \begin{macrocode}
\HOLOGO@IfExists\hologo{%
  \@PackageError{hologo}{%
    \string\hologo\ltx@space is already defined.\MessageBreak
    Package loading is aborted%
  }\@ehc
  \HOLOGO@AtEnd
}%
\HOLOGO@IfExists\hologoRobust{%
  \@PackageError{hologo}{%
    \string\hologoRobust\ltx@space is already defined.\MessageBreak
    Package loading is aborted%
  }\@ehc
  \HOLOGO@AtEnd
}%
%    \end{macrocode}
%
% \subsubsection{\cs{hologo} and friends}
%
%    \begin{macrocode}
\ifluatex
  \expandafter\ltx@firstofone
\else
  \expandafter\ltx@gobble
\fi
{%
  \ltx@IfUndefined{ifincsname}{%
    \ifnum\luatexversion<36 %
      \expandafter\ltx@gobble
    \else
      \expandafter\ltx@firstofone
    \fi
    {%
      \begingroup
        \ifcase0%
            \directlua{%
              if tex.enableprimitives then %
                tex.enableprimitives('HOLOGO@', {'ifincsname'})%
              else %
                tex.print('1')%
              end%
            }%
            \ifx\HOLOGO@ifincsname\@undefined 1\fi%
            \relax
          \expandafter\ltx@firstofone
        \else
          \endgroup
          \expandafter\ltx@gobble
        \fi
        {%
          \global\let\ifincsname\HOLOGO@ifincsname
        }%
      \HOLOGO@temp
    }%
  }{}%
}
%    \end{macrocode}
%    \begin{macrocode}
\ltx@IfUndefined{ifincsname}{%
  \catcode`$=14 %
}{%
  \catcode`$=9 %
}
%    \end{macrocode}
%
%    \begin{macro}{\hologo}
%    \begin{macrocode}
\def\hologo#1{%
$ \ifincsname
$   \ltx@ifundefined{HoLogoCs@\HOLOGO@Variant{#1}}{%
$     #1%
$   }{%
$     \csname HoLogoCs@\HOLOGO@Variant{#1}\endcsname\ltx@firstoftwo
$   }%
$ \else
    \HOLOGO@IfExists\texorpdfstring\texorpdfstring\ltx@firstoftwo
    {%
      \hologoRobust{#1}%
    }{%
      \ltx@ifundefined{HoLogoBkm@\HOLOGO@Variant{#1}}{%
        \ltx@ifundefined{HoLogo@#1}{?#1?}{#1}%
      }{%
        \csname HoLogoBkm@\HOLOGO@Variant{#1}\endcsname
        \ltx@firstoftwo
      }%
    }%
$ \fi
}
%    \end{macrocode}
%    \end{macro}
%    \begin{macro}{\Hologo}
%    \begin{macrocode}
\def\Hologo#1{%
$ \ifincsname
$   \ltx@ifundefined{HoLogoCs@\HOLOGO@Variant{#1}}{%
$     #1%
$   }{%
$     \csname HoLogoCs@\HOLOGO@Variant{#1}\endcsname\ltx@secondoftwo
$   }%
$ \else
    \HOLOGO@IfExists\texorpdfstring\texorpdfstring\ltx@firstoftwo
    {%
      \HologoRobust{#1}%
    }{%
      \ltx@ifundefined{HoLogoBkm@\HOLOGO@Variant{#1}}{%
        \ltx@ifundefined{HoLogo@#1}{?#1?}{#1}%
      }{%
        \csname HoLogoBkm@\HOLOGO@Variant{#1}\endcsname
        \ltx@secondoftwo
      }%
    }%
$ \fi
}
%    \end{macrocode}
%    \end{macro}
%
%    \begin{macro}{\hologoVariant}
%    \begin{macrocode}
\def\hologoVariant#1#2{%
  \ifx\relax#2\relax
    \hologo{#1}%
  \else
$   \ifincsname
$     \ltx@ifundefined{HoLogoCs@#1@#2}{%
$       #1%
$     }{%
$       \csname HoLogoCs@#1@#2\endcsname\ltx@firstoftwo
$     }%
$   \else
      \HOLOGO@IfExists\texorpdfstring\texorpdfstring\ltx@firstoftwo
      {%
        \hologoVariantRobust{#1}{#2}%
      }{%
        \ltx@ifundefined{HoLogoBkm@#1@#2}{%
          \ltx@ifundefined{HoLogo@#1}{?#1?}{#1}%
        }{%
          \csname HoLogoBkm@#1@#2\endcsname
          \ltx@firstoftwo
        }%
      }%
$   \fi
  \fi
}
%    \end{macrocode}
%    \end{macro}
%    \begin{macro}{\HologoVariant}
%    \begin{macrocode}
\def\HologoVariant#1#2{%
  \ifx\relax#2\relax
    \Hologo{#1}%
  \else
$   \ifincsname
$     \ltx@ifundefined{HoLogoCs@#1@#2}{%
$       #1%
$     }{%
$       \csname HoLogoCs@#1@#2\endcsname\ltx@secondoftwo
$     }%
$   \else
      \HOLOGO@IfExists\texorpdfstring\texorpdfstring\ltx@firstoftwo
      {%
        \HologoVariantRobust{#1}{#2}%
      }{%
        \ltx@ifundefined{HoLogoBkm@#1@#2}{%
          \ltx@ifundefined{HoLogo@#1}{?#1?}{#1}%
        }{%
          \csname HoLogoBkm@#1@#2\endcsname
          \ltx@secondoftwo
        }%
      }%
$   \fi
  \fi
}
%    \end{macrocode}
%    \end{macro}
%
%    \begin{macrocode}
\catcode`\$=3 %
%    \end{macrocode}
%
% \subsubsection{\cs{hologoRobust} and friends}
%
%    \begin{macro}{\hologoRobust}
%    \begin{macrocode}
\ltx@IfUndefined{protected}{%
  \ltx@IfUndefined{DeclareRobustCommand}{%
    \def\hologoRobust#1%
  }{%
    \DeclareRobustCommand*\hologoRobust[1]%
  }%
}{%
  \protected\def\hologoRobust#1%
}%
{%
  \edef\HOLOGO@name{#1}%
  \ltx@IfUndefined{HoLogo@\HOLOGO@Variant\HOLOGO@name}{%
    \@PackageError{hologo}{%
      Unknown logo `\HOLOGO@name'%
    }\@ehc
    ?\HOLOGO@name?%
  }{%
    \ltx@IfUndefined{ver@tex4ht.sty}{%
      \HoLogoFont@font\HOLOGO@name{general}{%
        \csname HoLogo@\HOLOGO@Variant\HOLOGO@name\endcsname
        \ltx@firstoftwo
      }%
    }{%
      \ltx@IfUndefined{HoLogoHtml@\HOLOGO@Variant\HOLOGO@name}{%
        \HOLOGO@name
      }{%
        \csname HoLogoHtml@\HOLOGO@Variant\HOLOGO@name\endcsname
        \ltx@firstoftwo
      }%
    }%
  }%
}
%    \end{macrocode}
%    \end{macro}
%    \begin{macro}{\HologoRobust}
%    \begin{macrocode}
\ltx@IfUndefined{protected}{%
  \ltx@IfUndefined{DeclareRobustCommand}{%
    \def\HologoRobust#1%
  }{%
    \DeclareRobustCommand*\HologoRobust[1]%
  }%
}{%
  \protected\def\HologoRobust#1%
}%
{%
  \edef\HOLOGO@name{#1}%
  \ltx@IfUndefined{HoLogo@\HOLOGO@Variant\HOLOGO@name}{%
    \@PackageError{hologo}{%
      Unknown logo `\HOLOGO@name'%
    }\@ehc
    ?\HOLOGO@name?%
  }{%
    \ltx@IfUndefined{ver@tex4ht.sty}{%
      \HoLogoFont@font\HOLOGO@name{general}{%
        \csname HoLogo@\HOLOGO@Variant\HOLOGO@name\endcsname
        \ltx@secondoftwo
      }%
    }{%
      \ltx@IfUndefined{HoLogoHtml@\HOLOGO@Variant\HOLOGO@name}{%
        \expandafter\HOLOGO@Uppercase\HOLOGO@name
      }{%
        \csname HoLogoHtml@\HOLOGO@Variant\HOLOGO@name\endcsname
        \ltx@secondoftwo
      }%
    }%
  }%
}
%    \end{macrocode}
%    \end{macro}
%    \begin{macro}{\hologoVariantRobust}
%    \begin{macrocode}
\ltx@IfUndefined{protected}{%
  \ltx@IfUndefined{DeclareRobustCommand}{%
    \def\hologoVariantRobust#1#2%
  }{%
    \DeclareRobustCommand*\hologoVariantRobust[2]%
  }%
}{%
  \protected\def\hologoVariantRobust#1#2%
}%
{%
  \begingroup
    \hologoLogoSetup{#1}{variant={#2}}%
    \hologoRobust{#1}%
  \endgroup
}
%    \end{macrocode}
%    \end{macro}
%    \begin{macro}{\HologoVariantRobust}
%    \begin{macrocode}
\ltx@IfUndefined{protected}{%
  \ltx@IfUndefined{DeclareRobustCommand}{%
    \def\HologoVariantRobust#1#2%
  }{%
    \DeclareRobustCommand*\HologoVariantRobust[2]%
  }%
}{%
  \protected\def\HologoVariantRobust#1#2%
}%
{%
  \begingroup
    \hologoLogoSetup{#1}{variant={#2}}%
    \HologoRobust{#1}%
  \endgroup
}
%    \end{macrocode}
%    \end{macro}
%
%    \begin{macro}{\hologorobust}
%    Macro \cs{hologorobust} is only defined for compatibility.
%    Its use is deprecated.
%    \begin{macrocode}
\def\hologorobust{\hologoRobust}
%    \end{macrocode}
%    \end{macro}
%
% \subsection{Helpers}
%
%    \begin{macro}{\HOLOGO@Uppercase}
%    Macro \cs{HOLOGO@Uppercase} is restricted to \cs{uppercase},
%    because \hologo{plainTeX} or \hologo{iniTeX} do not provide
%    \cs{MakeUppercase}.
%    \begin{macrocode}
\def\HOLOGO@Uppercase#1{\uppercase{#1}}
%    \end{macrocode}
%    \end{macro}
%
%    \begin{macro}{\HOLOGO@PdfdocUnicode}
%    \begin{macrocode}
\def\HOLOGO@PdfdocUnicode{%
  \ifx\ifHy@unicode\iftrue
    \expandafter\ltx@secondoftwo
  \else
    \expandafter\ltx@firstoftwo
  \fi
}
%    \end{macrocode}
%    \end{macro}
%
%    \begin{macro}{\HOLOGO@Math}
%    \begin{macrocode}
\def\HOLOGO@MathSetup{%
  \mathsurround0pt\relax
  \HOLOGO@IfExists\f@series{%
    \if b\expandafter\ltx@car\f@series x\@nil
      \csname boldmath\endcsname
   \fi
  }{}%
}
%    \end{macrocode}
%    \end{macro}
%
%    \begin{macro}{\HOLOGO@TempDimen}
%    \begin{macrocode}
\dimendef\HOLOGO@TempDimen=\ltx@zero
%    \end{macrocode}
%    \end{macro}
%    \begin{macro}{\HOLOGO@NegativeKerning}
%    \begin{macrocode}
\def\HOLOGO@NegativeKerning#1{%
  \begingroup
    \HOLOGO@TempDimen=0pt\relax
    \comma@parse@normalized{#1}{%
      \ifdim\HOLOGO@TempDimen=0pt %
        \expandafter\HOLOGO@@NegativeKerning\comma@entry
      \fi
      \ltx@gobble
    }%
    \ifdim\HOLOGO@TempDimen<0pt %
      \kern\HOLOGO@TempDimen
    \fi
  \endgroup
}
%    \end{macrocode}
%    \end{macro}
%    \begin{macro}{\HOLOGO@@NegativeKerning}
%    \begin{macrocode}
\def\HOLOGO@@NegativeKerning#1#2{%
  \setbox\ltx@zero\hbox{#1#2}%
  \HOLOGO@TempDimen=\wd\ltx@zero
  \setbox\ltx@zero\hbox{#1\kern0pt#2}%
  \advance\HOLOGO@TempDimen by -\wd\ltx@zero
}
%    \end{macrocode}
%    \end{macro}
%
%    \begin{macro}{\HOLOGO@SpaceFactor}
%    \begin{macrocode}
\def\HOLOGO@SpaceFactor{%
  \spacefactor1000 %
}
%    \end{macrocode}
%    \end{macro}
%
%    \begin{macro}{\HOLOGO@Span}
%    \begin{macrocode}
\def\HOLOGO@Span#1#2{%
  \HCode{<span class="HoLogo-#1">}%
  #2%
  \HCode{</span>}%
}
%    \end{macrocode}
%    \end{macro}
%
% \subsubsection{Text subscript}
%
%    \begin{macro}{\HOLOGO@SubScript}%
%    \begin{macrocode}
\def\HOLOGO@SubScript#1{%
  \ltx@IfUndefined{textsubscript}{%
    \ltx@IfUndefined{text}{%
      \ltx@mbox{%
        \mathsurround=0pt\relax
        $%
          _{%
            \ltx@IfUndefined{sf@size}{%
              \mathrm{#1}%
            }{%
              \mbox{%
                \fontsize\sf@size{0pt}\selectfont
                #1%
              }%
            }%
          }%
        $%
      }%
    }{%
      \ltx@mbox{%
        \mathsurround=0pt\relax
        $_{\text{#1}}$%
      }%
    }%
  }{%
    \textsubscript{#1}%
  }%
}
%    \end{macrocode}
%    \end{macro}
%
% \subsection{\hologo{TeX} and friends}
%
% \subsubsection{\hologo{TeX}}
%
%    \begin{macro}{\HoLogo@TeX}
%    Source: \hologo{LaTeX} kernel.
%    \begin{macrocode}
\def\HoLogo@TeX#1{%
  T\kern-.1667em\lower.5ex\hbox{E}\kern-.125emX\HOLOGO@SpaceFactor
}
%    \end{macrocode}
%    \end{macro}
%    \begin{macro}{\HoLogoHtml@TeX}
%    \begin{macrocode}
\def\HoLogoHtml@TeX#1{%
  \HoLogoCss@TeX
  \HOLOGO@Span{TeX}{%
    T%
    \HOLOGO@Span{e}{%
      E%
    }%
    X%
  }%
}
%    \end{macrocode}
%    \end{macro}
%    \begin{macro}{\HoLogoCss@TeX}
%    \begin{macrocode}
\def\HoLogoCss@TeX{%
  \Css{%
    span.HoLogo-TeX span.HoLogo-e{%
      position:relative;%
      top:.5ex;%
      margin-left:-.1667em;%
      margin-right:-.125em;%
    }%
  }%
  \Css{%
    a span.HoLogo-TeX span.HoLogo-e{%
      text-decoration:none;%
    }%
  }%
  \global\let\HoLogoCss@TeX\relax
}
%    \end{macrocode}
%    \end{macro}
%
% \subsubsection{\hologo{plainTeX}}
%
%    \begin{macro}{\HoLogo@plainTeX@space}
%    Source: ``The \hologo{TeX}book''
%    \begin{macrocode}
\def\HoLogo@plainTeX@space#1{%
  \HOLOGO@mbox{#1{p}{P}lain}\HOLOGO@space\hologo{TeX}%
}
%    \end{macrocode}
%    \end{macro}
%    \begin{macro}{\HoLogoCs@plainTeX@space}
%    \begin{macrocode}
\def\HoLogoCs@plainTeX@space#1{#1{p}{P}lain TeX}%
%    \end{macrocode}
%    \end{macro}
%    \begin{macro}{\HoLogoBkm@plainTeX@space}
%    \begin{macrocode}
\def\HoLogoBkm@plainTeX@space#1{%
  #1{p}{P}lain \hologo{TeX}%
}
%    \end{macrocode}
%    \end{macro}
%    \begin{macro}{\HoLogoHtml@plainTeX@space}
%    \begin{macrocode}
\def\HoLogoHtml@plainTeX@space#1{%
  #1{p}{P}lain \hologo{TeX}%
}
%    \end{macrocode}
%    \end{macro}
%
%    \begin{macro}{\HoLogo@plainTeX@hyphen}
%    \begin{macrocode}
\def\HoLogo@plainTeX@hyphen#1{%
  \HOLOGO@mbox{#1{p}{P}lain}\HOLOGO@hyphen\hologo{TeX}%
}
%    \end{macrocode}
%    \end{macro}
%    \begin{macro}{\HoLogoCs@plainTeX@hyphen}
%    \begin{macrocode}
\def\HoLogoCs@plainTeX@hyphen#1{#1{p}{P}lain-TeX}
%    \end{macrocode}
%    \end{macro}
%    \begin{macro}{\HoLogoBkm@plainTeX@hyphen}
%    \begin{macrocode}
\def\HoLogoBkm@plainTeX@hyphen#1{%
  #1{p}{P}lain-\hologo{TeX}%
}
%    \end{macrocode}
%    \end{macro}
%    \begin{macro}{\HoLogoHtml@plainTeX@hyphen}
%    \begin{macrocode}
\def\HoLogoHtml@plainTeX@hyphen#1{%
  #1{p}{P}lain-\hologo{TeX}%
}
%    \end{macrocode}
%    \end{macro}
%
%    \begin{macro}{\HoLogo@plainTeX@runtogether}
%    \begin{macrocode}
\def\HoLogo@plainTeX@runtogether#1{%
  \HOLOGO@mbox{#1{p}{P}lain\hologo{TeX}}%
}
%    \end{macrocode}
%    \end{macro}
%    \begin{macro}{\HoLogoCs@plainTeX@runtogether}
%    \begin{macrocode}
\def\HoLogoCs@plainTeX@runtogether#1{#1{p}{P}lainTeX}
%    \end{macrocode}
%    \end{macro}
%    \begin{macro}{\HoLogoBkm@plainTeX@runtogether}
%    \begin{macrocode}
\def\HoLogoBkm@plainTeX@runtogether#1{%
  #1{p}{P}lain\hologo{TeX}%
}
%    \end{macrocode}
%    \end{macro}
%    \begin{macro}{\HoLogoHtml@plainTeX@runtogether}
%    \begin{macrocode}
\def\HoLogoHtml@plainTeX@runtogether#1{%
  #1{p}{P}lain\hologo{TeX}%
}
%    \end{macrocode}
%    \end{macro}
%
%    \begin{macro}{\HoLogo@plainTeX}
%    \begin{macrocode}
\def\HoLogo@plainTeX{\HoLogo@plainTeX@space}
%    \end{macrocode}
%    \end{macro}
%    \begin{macro}{\HoLogoCs@plainTeX}
%    \begin{macrocode}
\def\HoLogoCs@plainTeX{\HoLogoCs@plainTeX@space}
%    \end{macrocode}
%    \end{macro}
%    \begin{macro}{\HoLogoBkm@plainTeX}
%    \begin{macrocode}
\def\HoLogoBkm@plainTeX{\HoLogoBkm@plainTeX@space}
%    \end{macrocode}
%    \end{macro}
%    \begin{macro}{\HoLogoHtml@plainTeX}
%    \begin{macrocode}
\def\HoLogoHtml@plainTeX{\HoLogoHtml@plainTeX@space}
%    \end{macrocode}
%    \end{macro}
%
% \subsubsection{\hologo{LaTeX}}
%
%    Source: \hologo{LaTeX} kernel.
%\begin{quote}
%\begin{verbatim}
%\DeclareRobustCommand{\LaTeX}{%
%  L%
%  \kern-.36em%
%  {%
%    \sbox\z@ T%
%    \vbox to\ht\z@{%
%      \hbox{%
%        \check@mathfonts
%        \fontsize\sf@size\z@
%        \math@fontsfalse
%        \selectfont
%        A%
%      }%
%      \vss
%    }%
%  }%
%  \kern-.15em%
%  \TeX
%}
%\end{verbatim}
%\end{quote}
%
%    \begin{macro}{\HoLogo@La}
%    \begin{macrocode}
\def\HoLogo@La#1{%
  L%
  \kern-.36em%
  \begingroup
    \setbox\ltx@zero\hbox{T}%
    \vbox to\ht\ltx@zero{%
      \hbox{%
        \ltx@ifundefined{check@mathfonts}{%
          \csname sevenrm\endcsname
        }{%
          \check@mathfonts
          \fontsize\sf@size{0pt}%
          \math@fontsfalse\selectfont
        }%
        A%
      }%
      \vss
    }%
  \endgroup
}
%    \end{macrocode}
%    \end{macro}
%
%    \begin{macro}{\HoLogo@LaTeX}
%    Source: \hologo{LaTeX} kernel.
%    \begin{macrocode}
\def\HoLogo@LaTeX#1{%
  \hologo{La}%
  \kern-.15em%
  \hologo{TeX}%
}
%    \end{macrocode}
%    \end{macro}
%    \begin{macro}{\HoLogoHtml@LaTeX}
%    \begin{macrocode}
\def\HoLogoHtml@LaTeX#1{%
  \HoLogoCss@LaTeX
  \HOLOGO@Span{LaTeX}{%
    L%
    \HOLOGO@Span{a}{%
      A%
    }%
    \hologo{TeX}%
  }%
}
%    \end{macrocode}
%    \end{macro}
%    \begin{macro}{\HoLogoCss@LaTeX}
%    \begin{macrocode}
\def\HoLogoCss@LaTeX{%
  \Css{%
    span.HoLogo-LaTeX span.HoLogo-a{%
      position:relative;%
      top:-.5ex;%
      margin-left:-.36em;%
      margin-right:-.15em;%
      font-size:85\%;%
    }%
  }%
  \global\let\HoLogoCss@LaTeX\relax
}
%    \end{macrocode}
%    \end{macro}
%
% \subsubsection{\hologo{(La)TeX}}
%
%    \begin{macro}{\HoLogo@LaTeXTeX}
%    The kerning around the parentheses is taken
%    from package \xpackage{dtklogos} \cite{dtklogos}.
%\begin{quote}
%\begin{verbatim}
%\DeclareRobustCommand{\LaTeXTeX}{%
%  (%
%  \kern-.15em%
%  L%
%  \kern-.36em%
%  {%
%    \sbox\z@ T%
%    \vbox to\ht0{%
%      \hbox{%
%        $\m@th$%
%        \csname S@\f@size\endcsname
%        \fontsize\sf@size\z@
%        \math@fontsfalse
%        \selectfont
%        A%
%      }%
%      \vss
%    }%
%  }%
%  \kern-.2em%
%  )%
%  \kern-.15em%
%  \TeX
%}
%\end{verbatim}
%\end{quote}
%    \begin{macrocode}
\def\HoLogo@LaTeXTeX#1{%
  (%
  \kern-.15em%
  \hologo{La}%
  \kern-.2em%
  )%
  \kern-.15em%
  \hologo{TeX}%
}
%    \end{macrocode}
%    \end{macro}
%    \begin{macro}{\HoLogoBkm@LaTeXTeX}
%    \begin{macrocode}
\def\HoLogoBkm@LaTeXTeX#1{(La)TeX}
%    \end{macrocode}
%    \end{macro}
%
%    \begin{macro}{\HoLogo@(La)TeX}
%    \begin{macrocode}
\expandafter
\let\csname HoLogo@(La)TeX\endcsname\HoLogo@LaTeXTeX
%    \end{macrocode}
%    \end{macro}
%    \begin{macro}{\HoLogoBkm@(La)TeX}
%    \begin{macrocode}
\expandafter
\let\csname HoLogoBkm@(La)TeX\endcsname\HoLogoBkm@LaTeXTeX
%    \end{macrocode}
%    \end{macro}
%    \begin{macro}{\HoLogoHtml@LaTeXTeX}
%    \begin{macrocode}
\def\HoLogoHtml@LaTeXTeX#1{%
  \HoLogoCss@LaTeXTeX
  \HOLOGO@Span{LaTeXTeX}{%
    (%
    \HOLOGO@Span{L}{L}%
    \HOLOGO@Span{a}{A}%
    \HOLOGO@Span{ParenRight}{)}%
    \hologo{TeX}%
  }%
}
%    \end{macrocode}
%    \end{macro}
%    \begin{macro}{\HoLogoHtml@(La)TeX}
%    Kerning after opening parentheses and before closing parentheses
%    is $-0.1$\,em. The original values $-0.15$\,em
%    looked too ugly for a serif font.
%    \begin{macrocode}
\expandafter
\let\csname HoLogoHtml@(La)TeX\endcsname\HoLogoHtml@LaTeXTeX
%    \end{macrocode}
%    \end{macro}
%    \begin{macro}{\HoLogoCss@LaTeXTeX}
%    \begin{macrocode}
\def\HoLogoCss@LaTeXTeX{%
  \Css{%
    span.HoLogo-LaTeXTeX span.HoLogo-L{%
      margin-left:-.1em;%
    }%
  }%
  \Css{%
    span.HoLogo-LaTeXTeX span.HoLogo-a{%
      position:relative;%
      top:-.5ex;%
      margin-left:-.36em;%
      margin-right:-.1em;%
      font-size:85\%;%
    }%
  }%
  \Css{%
    span.HoLogo-LaTeXTeX span.HoLogo-ParenRight{%
      margin-right:-.15em;%
    }%
  }%
  \global\let\HoLogoCss@LaTeXTeX\relax
}
%    \end{macrocode}
%    \end{macro}
%
% \subsubsection{\hologo{LaTeXe}}
%
%    \begin{macro}{\HoLogo@LaTeXe}
%    Source: \hologo{LaTeX} kernel
%    \begin{macrocode}
\def\HoLogo@LaTeXe#1{%
  \hologo{LaTeX}%
  \kern.15em%
  \hbox{%
    \HOLOGO@MathSetup
    2%
    $_{\textstyle\varepsilon}$%
  }%
}
%    \end{macrocode}
%    \end{macro}
%
%    \begin{macro}{\HoLogoCs@LaTeXe}
%    \begin{macrocode}
\ifnum64=`\^^^^0040\relax % test for big chars of LuaTeX/XeTeX
  \catcode`\$=9 %
  \catcode`\&=14 %
\else
  \catcode`\$=14 %
  \catcode`\&=9 %
\fi
\def\HoLogoCs@LaTeXe#1{%
  LaTeX2%
$ \string ^^^^0395%
& e%
}%
\catcode`\$=3 %
\catcode`\&=4 %
%    \end{macrocode}
%    \end{macro}
%
%    \begin{macro}{\HoLogoBkm@LaTeXe}
%    \begin{macrocode}
\def\HoLogoBkm@LaTeXe#1{%
  \hologo{LaTeX}%
  2%
  \HOLOGO@PdfdocUnicode{e}{\textepsilon}%
}
%    \end{macrocode}
%    \end{macro}
%
%    \begin{macro}{\HoLogoHtml@LaTeXe}
%    \begin{macrocode}
\def\HoLogoHtml@LaTeXe#1{%
  \HoLogoCss@LaTeXe
  \HOLOGO@Span{LaTeX2e}{%
    \hologo{LaTeX}%
    \HOLOGO@Span{2}{2}%
    \HOLOGO@Span{e}{%
      \HOLOGO@MathSetup
      \ensuremath{\textstyle\varepsilon}%
    }%
  }%
}
%    \end{macrocode}
%    \end{macro}
%    \begin{macro}{\HoLogoCss@LaTeXe}
%    \begin{macrocode}
\def\HoLogoCss@LaTeXe{%
  \Css{%
    span.HoLogo-LaTeX2e span.HoLogo-2{%
      padding-left:.15em;%
    }%
  }%
  \Css{%
    span.HoLogo-LaTeX2e span.HoLogo-e{%
      position:relative;%
      top:.35ex;%
      text-decoration:none;%
    }%
  }%
  \global\let\HoLogoCss@LaTeXe\relax
}
%    \end{macrocode}
%    \end{macro}
%
%    \begin{macro}{\HoLogo@LaTeX2e}
%    \begin{macrocode}
\expandafter
\let\csname HoLogo@LaTeX2e\endcsname\HoLogo@LaTeXe
%    \end{macrocode}
%    \end{macro}
%    \begin{macro}{\HoLogoCs@LaTeX2e}
%    \begin{macrocode}
\expandafter
\let\csname HoLogoCs@LaTeX2e\endcsname\HoLogoCs@LaTeXe
%    \end{macrocode}
%    \end{macro}
%    \begin{macro}{\HoLogoBkm@LaTeX2e}
%    \begin{macrocode}
\expandafter
\let\csname HoLogoBkm@LaTeX2e\endcsname\HoLogoBkm@LaTeXe
%    \end{macrocode}
%    \end{macro}
%    \begin{macro}{\HoLogoHtml@LaTeX2e}
%    \begin{macrocode}
\expandafter
\let\csname HoLogoHtml@LaTeX2e\endcsname\HoLogoHtml@LaTeXe
%    \end{macrocode}
%    \end{macro}
%
% \subsubsection{\hologo{LaTeX3}}
%
%    \begin{macro}{\HoLogo@LaTeX3}
%    Source: \hologo{LaTeX} kernel
%    \begin{macrocode}
\expandafter\def\csname HoLogo@LaTeX3\endcsname#1{%
  \hologo{LaTeX}%
  3%
}
%    \end{macrocode}
%    \end{macro}
%
%    \begin{macro}{\HoLogoBkm@LaTeX3}
%    \begin{macrocode}
\expandafter\def\csname HoLogoBkm@LaTeX3\endcsname#1{%
  \hologo{LaTeX}%
  3%
}
%    \end{macrocode}
%    \end{macro}
%    \begin{macro}{\HoLogoHtml@LaTeX3}
%    \begin{macrocode}
\expandafter
\let\csname HoLogoHtml@LaTeX3\expandafter\endcsname
\csname HoLogo@LaTeX3\endcsname
%    \end{macrocode}
%    \end{macro}
%
% \subsubsection{\hologo{LaTeXML}}
%
%    \begin{macro}{\HoLogo@LaTeXML}
%    \begin{macrocode}
\def\HoLogo@LaTeXML#1{%
  \HOLOGO@mbox{%
    \hologo{La}%
    \kern-.15em%
    T%
    \kern-.1667em%
    \lower.5ex\hbox{E}%
    \kern-.125em%
    \HoLogoFont@font{LaTeXML}{sc}{xml}%
  }%
}
%    \end{macrocode}
%    \end{macro}
%    \begin{macro}{\HoLogoHtml@pdfLaTeX}
%    \begin{macrocode}
\def\HoLogoHtml@LaTeXML#1{%
  \HOLOGO@Span{LaTeXML}{%
    \HoLogoCss@LaTeX
    \HoLogoCss@TeX
    \HOLOGO@Span{LaTeX}{%
      L%
      \HOLOGO@Span{a}{%
        A%
      }%
    }%
    \HOLOGO@Span{TeX}{%
      T%
      \HOLOGO@Span{e}{%
        E%
      }%
    }%
    \HCode{<span style="font-variant: small-caps;">}%
    xml%
    \HCode{</span>}%
  }%
}
%    \end{macrocode}
%    \end{macro}
%
% \subsubsection{\hologo{eTeX}}
%
%    \begin{macro}{\HoLogo@eTeX}
%    Source: package \xpackage{etex}
%    \begin{macrocode}
\def\HoLogo@eTeX#1{%
  \ltx@mbox{%
    \HOLOGO@MathSetup
    $\varepsilon$%
    -%
    \HOLOGO@NegativeKerning{-T,T-,To}%
    \hologo{TeX}%
  }%
}
%    \end{macrocode}
%    \end{macro}
%    \begin{macro}{\HoLogoCs@eTeX}
%    \begin{macrocode}
\ifnum64=`\^^^^0040\relax % test for big chars of LuaTeX/XeTeX
  \catcode`\$=9 %
  \catcode`\&=14 %
\else
  \catcode`\$=14 %
  \catcode`\&=9 %
\fi
\def\HoLogoCs@eTeX#1{%
$ #1{\string ^^^^0395}{\string ^^^^03b5}%
& #1{e}{E}%
  TeX%
}%
\catcode`\$=3 %
\catcode`\&=4 %
%    \end{macrocode}
%    \end{macro}
%    \begin{macro}{\HoLogoBkm@eTeX}
%    \begin{macrocode}
\def\HoLogoBkm@eTeX#1{%
  \HOLOGO@PdfdocUnicode{#1{e}{E}}{\textepsilon}%
  -%
  \hologo{TeX}%
}
%    \end{macrocode}
%    \end{macro}
%    \begin{macro}{\HoLogoHtml@eTeX}
%    \begin{macrocode}
\def\HoLogoHtml@eTeX#1{%
  \ltx@mbox{%
    \HOLOGO@MathSetup
    $\varepsilon$%
    -%
    \hologo{TeX}%
  }%
}
%    \end{macrocode}
%    \end{macro}
%
% \subsubsection{\hologo{iniTeX}}
%
%    \begin{macro}{\HoLogo@iniTeX}
%    \begin{macrocode}
\def\HoLogo@iniTeX#1{%
  \HOLOGO@mbox{%
    #1{i}{I}ni\hologo{TeX}%
  }%
}
%    \end{macrocode}
%    \end{macro}
%    \begin{macro}{\HoLogoCs@iniTeX}
%    \begin{macrocode}
\def\HoLogoCs@iniTeX#1{#1{i}{I}niTeX}
%    \end{macrocode}
%    \end{macro}
%    \begin{macro}{\HoLogoBkm@iniTeX}
%    \begin{macrocode}
\def\HoLogoBkm@iniTeX#1{%
  #1{i}{I}ni\hologo{TeX}%
}
%    \end{macrocode}
%    \end{macro}
%    \begin{macro}{\HoLogoHtml@iniTeX}
%    \begin{macrocode}
\let\HoLogoHtml@iniTeX\HoLogo@iniTeX
%    \end{macrocode}
%    \end{macro}
%
% \subsubsection{\hologo{virTeX}}
%
%    \begin{macro}{\HoLogo@virTeX}
%    \begin{macrocode}
\def\HoLogo@virTeX#1{%
  \HOLOGO@mbox{%
    #1{v}{V}ir\hologo{TeX}%
  }%
}
%    \end{macrocode}
%    \end{macro}
%    \begin{macro}{\HoLogoCs@virTeX}
%    \begin{macrocode}
\def\HoLogoCs@virTeX#1{#1{v}{V}irTeX}
%    \end{macrocode}
%    \end{macro}
%    \begin{macro}{\HoLogoBkm@virTeX}
%    \begin{macrocode}
\def\HoLogoBkm@virTeX#1{%
  #1{v}{V}ir\hologo{TeX}%
}
%    \end{macrocode}
%    \end{macro}
%    \begin{macro}{\HoLogoHtml@virTeX}
%    \begin{macrocode}
\let\HoLogoHtml@virTeX\HoLogo@virTeX
%    \end{macrocode}
%    \end{macro}
%
% \subsubsection{\hologo{SliTeX}}
%
% \paragraph{Definitions of the three variants.}
%
%    \begin{macro}{\HoLogo@SLiTeX@lift}
%    \begin{macrocode}
\def\HoLogo@SLiTeX@lift#1{%
  \HoLogoFont@font{SliTeX}{rm}{%
    S%
    \kern-.06em%
    L%
    \kern-.18em%
    \raise.32ex\hbox{\HoLogoFont@font{SliTeX}{sc}{i}}%
    \HOLOGO@discretionary
    \kern-.06em%
    \hologo{TeX}%
  }%
}
%    \end{macrocode}
%    \end{macro}
%    \begin{macro}{\HoLogoBkm@SLiTeX@lift}
%    \begin{macrocode}
\def\HoLogoBkm@SLiTeX@lift#1{SLiTeX}
%    \end{macrocode}
%    \end{macro}
%    \begin{macro}{\HoLogoHtml@SLiTeX@lift}
%    \begin{macrocode}
\def\HoLogoHtml@SLiTeX@lift#1{%
  \HoLogoCss@SLiTeX@lift
  \HOLOGO@Span{SLiTeX-lift}{%
    \HoLogoFont@font{SliTeX}{rm}{%
      S%
      \HOLOGO@Span{L}{L}%
      \HOLOGO@Span{i}{i}%
      \hologo{TeX}%
    }%
  }%
}
%    \end{macrocode}
%    \end{macro}
%    \begin{macro}{\HoLogoCss@SLiTeX@lift}
%    \begin{macrocode}
\def\HoLogoCss@SLiTeX@lift{%
  \Css{%
    span.HoLogo-SLiTeX-lift span.HoLogo-L{%
      margin-left:-.06em;%
      margin-right:-.18em;%
    }%
  }%
  \Css{%
    span.HoLogo-SLiTeX-lift span.HoLogo-i{%
      position:relative;%
      top:-.32ex;%
      margin-right:-.06em;%
      font-variant:small-caps;%
    }%
  }%
  \global\let\HoLogoCss@SLiTeX@lift\relax
}
%    \end{macrocode}
%    \end{macro}
%
%    \begin{macro}{\HoLogo@SliTeX@simple}
%    \begin{macrocode}
\def\HoLogo@SliTeX@simple#1{%
  \HoLogoFont@font{SliTeX}{rm}{%
    \ltx@mbox{%
      \HoLogoFont@font{SliTeX}{sc}{Sli}%
    }%
    \HOLOGO@discretionary
    \hologo{TeX}%
  }%
}
%    \end{macrocode}
%    \end{macro}
%    \begin{macro}{\HoLogoBkm@SliTeX@simple}
%    \begin{macrocode}
\def\HoLogoBkm@SliTeX@simple#1{SliTeX}
%    \end{macrocode}
%    \end{macro}
%    \begin{macro}{\HoLogoHtml@SliTeX@simple}
%    \begin{macrocode}
\let\HoLogoHtml@SliTeX@simple\HoLogo@SliTeX@simple
%    \end{macrocode}
%    \end{macro}
%
%    \begin{macro}{\HoLogo@SliTeX@narrow}
%    \begin{macrocode}
\def\HoLogo@SliTeX@narrow#1{%
  \HoLogoFont@font{SliTeX}{rm}{%
    \ltx@mbox{%
      S%
      \kern-.06em%
      \HoLogoFont@font{SliTeX}{sc}{%
        l%
        \kern-.035em%
        i%
      }%
    }%
    \HOLOGO@discretionary
    \kern-.06em%
    \hologo{TeX}%
  }%
}
%    \end{macrocode}
%    \end{macro}
%    \begin{macro}{\HoLogoBkm@SliTeX@narrow}
%    \begin{macrocode}
\def\HoLogoBkm@SliTeX@narrow#1{SliTeX}
%    \end{macrocode}
%    \end{macro}
%    \begin{macro}{\HoLogoHtml@SliTeX@narrow}
%    \begin{macrocode}
\def\HoLogoHtml@SliTeX@narrow#1{%
  \HoLogoCss@SliTeX@narrow
  \HOLOGO@Span{SliTeX-narrow}{%
    \HoLogoFont@font{SliTeX}{rm}{%
      S%
        \HOLOGO@Span{l}{l}%
        \HOLOGO@Span{i}{i}%
      \hologo{TeX}%
    }%
  }%
}
%    \end{macrocode}
%    \end{macro}
%    \begin{macro}{\HoLogoCss@SliTeX@narrow}
%    \begin{macrocode}
\def\HoLogoCss@SliTeX@narrow{%
  \Css{%
    span.HoLogo-SliTeX-narrow span.HoLogo-l{%
      margin-left:-.06em;%
      margin-right:-.035em;%
      font-variant:small-caps;%
    }%
  }%
  \Css{%
    span.HoLogo-SliTeX-narrow span.HoLogo-i{%
      margin-right:-.06em;%
      font-variant:small-caps;%
    }%
  }%
  \global\let\HoLogoCss@SliTeX@narrow\relax
}
%    \end{macrocode}
%    \end{macro}
%
% \paragraph{Macro set completion.}
%
%    \begin{macro}{\HoLogo@SLiTeX@simple}
%    \begin{macrocode}
\def\HoLogo@SLiTeX@simple{\HoLogo@SliTeX@simple}
%    \end{macrocode}
%    \end{macro}
%    \begin{macro}{\HoLogoBkm@SLiTeX@simple}
%    \begin{macrocode}
\def\HoLogoBkm@SLiTeX@simple{\HoLogoBkm@SliTeX@simple}
%    \end{macrocode}
%    \end{macro}
%    \begin{macro}{\HoLogoHtml@SLiTeX@simple}
%    \begin{macrocode}
\def\HoLogoHtml@SLiTeX@simple{\HoLogoHtml@SliTeX@simple}
%    \end{macrocode}
%    \end{macro}
%
%    \begin{macro}{\HoLogo@SLiTeX@narrow}
%    \begin{macrocode}
\def\HoLogo@SLiTeX@narrow{\HoLogo@SliTeX@narrow}
%    \end{macrocode}
%    \end{macro}
%    \begin{macro}{\HoLogoBkm@SLiTeX@narrow}
%    \begin{macrocode}
\def\HoLogoBkm@SLiTeX@narrow{\HoLogoBkm@SliTeX@narrow}
%    \end{macrocode}
%    \end{macro}
%    \begin{macro}{\HoLogoHtml@SLiTeX@narrow}
%    \begin{macrocode}
\def\HoLogoHtml@SLiTeX@narrow{\HoLogoHtml@SliTeX@narrow}
%    \end{macrocode}
%    \end{macro}
%
%    \begin{macro}{\HoLogo@SliTeX@lift}
%    \begin{macrocode}
\def\HoLogo@SliTeX@lift{\HoLogo@SLiTeX@lift}
%    \end{macrocode}
%    \end{macro}
%    \begin{macro}{\HoLogoBkm@SliTeX@lift}
%    \begin{macrocode}
\def\HoLogoBkm@SliTeX@lift{\HoLogoBkm@SLiTeX@lift}
%    \end{macrocode}
%    \end{macro}
%    \begin{macro}{\HoLogoHtml@SliTeX@lift}
%    \begin{macrocode}
\def\HoLogoHtml@SliTeX@lift{\HoLogoHtml@SLiTeX@lift}
%    \end{macrocode}
%    \end{macro}
%
% \paragraph{Defaults.}
%
%    \begin{macro}{\HoLogo@SLiTeX}
%    \begin{macrocode}
\def\HoLogo@SLiTeX{\HoLogo@SLiTeX@lift}
%    \end{macrocode}
%    \end{macro}
%    \begin{macro}{\HoLogoBkm@SLiTeX}
%    \begin{macrocode}
\def\HoLogoBkm@SLiTeX{\HoLogoBkm@SLiTeX@lift}
%    \end{macrocode}
%    \end{macro}
%    \begin{macro}{\HoLogoHtml@SLiTeX}
%    \begin{macrocode}
\def\HoLogoHtml@SLiTeX{\HoLogoHtml@SLiTeX@lift}
%    \end{macrocode}
%    \end{macro}
%
%    \begin{macro}{\HoLogo@SliTeX}
%    \begin{macrocode}
\def\HoLogo@SliTeX{\HoLogo@SliTeX@narrow}
%    \end{macrocode}
%    \end{macro}
%    \begin{macro}{\HoLogoBkm@SliTeX}
%    \begin{macrocode}
\def\HoLogoBkm@SliTeX{\HoLogoBkm@SliTeX@narrow}
%    \end{macrocode}
%    \end{macro}
%    \begin{macro}{\HoLogoHtml@SliTeX}
%    \begin{macrocode}
\def\HoLogoHtml@SliTeX{\HoLogoHtml@SliTeX@narrow}
%    \end{macrocode}
%    \end{macro}
%
% \subsubsection{\hologo{LuaTeX}}
%
%    \begin{macro}{\HoLogo@LuaTeX}
%    The kerning is an idea of Hans Hagen, see mailing list
%    `luatex at tug dot org' in March 2010.
%    \begin{macrocode}
\def\HoLogo@LuaTeX#1{%
  \HOLOGO@mbox{%
    Lua%
    \HOLOGO@NegativeKerning{aT,oT,To}%
    \hologo{TeX}%
  }%
}
%    \end{macrocode}
%    \end{macro}
%    \begin{macro}{\HoLogoHtml@LuaTeX}
%    \begin{macrocode}
\let\HoLogoHtml@LuaTeX\HoLogo@LuaTeX
%    \end{macrocode}
%    \end{macro}
%
% \subsubsection{\hologo{LuaLaTeX}}
%
%    \begin{macro}{\HoLogo@LuaLaTeX}
%    \begin{macrocode}
\def\HoLogo@LuaLaTeX#1{%
  \HOLOGO@mbox{%
    Lua%
    \hologo{LaTeX}%
  }%
}
%    \end{macrocode}
%    \end{macro}
%    \begin{macro}{\HoLogoHtml@LuaLaTeX}
%    \begin{macrocode}
\let\HoLogoHtml@LuaLaTeX\HoLogo@LuaLaTeX
%    \end{macrocode}
%    \end{macro}
%
% \subsubsection{\hologo{XeTeX}, \hologo{XeLaTeX}}
%
%    \begin{macro}{\HOLOGO@IfCharExists}
%    \begin{macrocode}
\ifluatex
  \ifnum\luatexversion<36 %
  \else
    \def\HOLOGO@IfCharExists#1{%
      \ifnum
        \directlua{%
           if luaotfload and luaotfload.aux then
             if luaotfload.aux.font_has_glyph(%
                    font.current(), \number#1) then % 	 
	       tex.print("1") % 	 
	     end % 	 
	   elseif font and font.fonts and font.current then %
            local f = font.fonts[font.current()]%
            if f.characters and f.characters[\number#1] then %
              tex.print("1")%
            end %
          end%
        }0=\ltx@zero
        \expandafter\ltx@secondoftwo
      \else
        \expandafter\ltx@firstoftwo
      \fi
    }%
  \fi
\fi
\ltx@IfUndefined{HOLOGO@IfCharExists}{%
  \def\HOLOGO@@IfCharExists#1{%
    \begingroup
      \tracinglostchars=\ltx@zero
      \setbox\ltx@zero=\hbox{%
        \kern7sp\char#1\relax
        \ifnum\lastkern>\ltx@zero
          \expandafter\aftergroup\csname iffalse\endcsname
        \else
          \expandafter\aftergroup\csname iftrue\endcsname
        \fi
      }%
      % \if{true|false} from \aftergroup
      \endgroup
      \expandafter\ltx@firstoftwo
    \else
      \endgroup
      \expandafter\ltx@secondoftwo
    \fi
  }%
  \ifxetex
    \ltx@IfUndefined{XeTeXfonttype}{}{%
      \ltx@IfUndefined{XeTeXcharglyph}{}{%
        \def\HOLOGO@IfCharExists#1{%
          \ifnum\XeTeXfonttype\font>\ltx@zero
            \expandafter\ltx@firstofthree
          \else
            \expandafter\ltx@gobble
          \fi
          {%
            \ifnum\XeTeXcharglyph#1>\ltx@zero
              \expandafter\ltx@firstoftwo
            \else
              \expandafter\ltx@secondoftwo
            \fi
          }%
          \HOLOGO@@IfCharExists{#1}%
        }%
      }%
    }%
  \fi
}{}
\ltx@ifundefined{HOLOGO@IfCharExists}{%
  \ifnum64=`\^^^^0040\relax % test for big chars of LuaTeX/XeTeX
    \let\HOLOGO@IfCharExists\HOLOGO@@IfCharExists
  \else
    \def\HOLOGO@IfCharExists#1{%
      \ifnum#1>255 %
        \expandafter\ltx@fourthoffour
      \fi
      \HOLOGO@@IfCharExists{#1}%
    }%
  \fi
}{}
%    \end{macrocode}
%    \end{macro}
%
%    \begin{macro}{\HoLogo@Xe}
%    Source: package \xpackage{dtklogos}
%    \begin{macrocode}
\def\HoLogo@Xe#1{%
  X%
  \kern-.1em\relax
  \HOLOGO@IfCharExists{"018E}{%
    \lower.5ex\hbox{\char"018E}%
  }{%
    \chardef\HOLOGO@choice=\ltx@zero
    \ifdim\fontdimen\ltx@one\font>0pt %
      \ltx@IfUndefined{rotatebox}{%
        \ltx@IfUndefined{pgftext}{%
          \ltx@IfUndefined{psscalebox}{%
            \ltx@IfUndefined{HOLOGO@ScaleBox@\hologoDriver}{%
            }{%
              \chardef\HOLOGO@choice=4 %
            }%
          }{%
            \chardef\HOLOGO@choice=3 %
          }%
        }{%
          \chardef\HOLOGO@choice=2 %
        }%
      }{%
        \chardef\HOLOGO@choice=1 %
      }%
      \ifcase\HOLOGO@choice
        \HOLOGO@WarningUnsupportedDriver{Xe}%
        e%
      \or % 1: \rotatebox
        \begingroup
          \setbox\ltx@zero\hbox{\rotatebox{180}{E}}%
          \ltx@LocDimenA=\dp\ltx@zero
          \advance\ltx@LocDimenA by -.5ex\relax
          \raise\ltx@LocDimenA\box\ltx@zero
        \endgroup
      \or % 2: \pgftext
        \lower.5ex\hbox{%
          \pgfpicture
            \pgftext[rotate=180]{E}%
          \endpgfpicture
        }%
      \or % 3: \psscalebox
        \begingroup
          \setbox\ltx@zero\hbox{\psscalebox{-1 -1}{E}}%
          \ltx@LocDimenA=\dp\ltx@zero
          \advance\ltx@LocDimenA by -.5ex\relax
          \raise\ltx@LocDimenA\box\ltx@zero
        \endgroup
      \or % 4: \HOLOGO@PointReflectBox
        \lower.5ex\hbox{\HOLOGO@PointReflectBox{E}}%
      \else
        \@PackageError{hologo}{Internal error (choice/it}\@ehc
      \fi
    \else
      \ltx@IfUndefined{reflectbox}{%
        \ltx@IfUndefined{pgftext}{%
          \ltx@IfUndefined{psscalebox}{%
            \ltx@IfUndefined{HOLOGO@ScaleBox@\hologoDriver}{%
            }{%
              \chardef\HOLOGO@choice=4 %
            }%
          }{%
            \chardef\HOLOGO@choice=3 %
          }%
        }{%
          \chardef\HOLOGO@choice=2 %
        }%
      }{%
        \chardef\HOLOGO@choice=1 %
      }%
      \ifcase\HOLOGO@choice
        \HOLOGO@WarningUnsupportedDriver{Xe}%
        e%
      \or % 1: reflectbox
        \lower.5ex\hbox{%
          \reflectbox{E}%
        }%
      \or % 2: \pgftext
        \lower.5ex\hbox{%
          \pgfpicture
            \pgftransformxscale{-1}%
            \pgftext{E}%
          \endpgfpicture
        }%
      \or % 3: \psscalebox
        \lower.5ex\hbox{%
          \psscalebox{-1 1}{E}%
        }%
      \or % 4: \HOLOGO@Reflectbox
        \lower.5ex\hbox{%
          \HOLOGO@ReflectBox{E}%
        }%
      \else
        \@PackageError{hologo}{Internal error (choice/up)}\@ehc
      \fi
    \fi
  }%
}
%    \end{macrocode}
%    \end{macro}
%    \begin{macro}{\HoLogoHtml@Xe}
%    \begin{macrocode}
\def\HoLogoHtml@Xe#1{%
  \HoLogoCss@Xe
  \HOLOGO@Span{Xe}{%
    X%
    \HOLOGO@Span{e}{%
      \HCode{&\ltx@hashchar x018e;}%
    }%
  }%
}
%    \end{macrocode}
%    \end{macro}
%    \begin{macro}{\HoLogoCss@Xe}
%    \begin{macrocode}
\def\HoLogoCss@Xe{%
  \Css{%
    span.HoLogo-Xe span.HoLogo-e{%
      position:relative;%
      top:.5ex;%
      left-margin:-.1em;%
    }%
  }%
  \global\let\HoLogoCss@Xe\relax
}
%    \end{macrocode}
%    \end{macro}
%
%    \begin{macro}{\HoLogo@XeTeX}
%    \begin{macrocode}
\def\HoLogo@XeTeX#1{%
  \hologo{Xe}%
  \kern-.15em\relax
  \hologo{TeX}%
}
%    \end{macrocode}
%    \end{macro}
%
%    \begin{macro}{\HoLogoHtml@XeTeX}
%    \begin{macrocode}
\def\HoLogoHtml@XeTeX#1{%
  \HoLogoCss@XeTeX
  \HOLOGO@Span{XeTeX}{%
    \hologo{Xe}%
    \hologo{TeX}%
  }%
}
%    \end{macrocode}
%    \end{macro}
%    \begin{macro}{\HoLogoCss@XeTeX}
%    \begin{macrocode}
\def\HoLogoCss@XeTeX{%
  \Css{%
    span.HoLogo-XeTeX span.HoLogo-TeX{%
      margin-left:-.15em;%
    }%
  }%
  \global\let\HoLogoCss@XeTeX\relax
}
%    \end{macrocode}
%    \end{macro}
%
%    \begin{macro}{\HoLogo@XeLaTeX}
%    \begin{macrocode}
\def\HoLogo@XeLaTeX#1{%
  \hologo{Xe}%
  \kern-.13em%
  \hologo{LaTeX}%
}
%    \end{macrocode}
%    \end{macro}
%    \begin{macro}{\HoLogoHtml@XeLaTeX}
%    \begin{macrocode}
\def\HoLogoHtml@XeLaTeX#1{%
  \HoLogoCss@XeLaTeX
  \HOLOGO@Span{XeLaTeX}{%
    \hologo{Xe}%
    \hologo{LaTeX}%
  }%
}
%    \end{macrocode}
%    \end{macro}
%    \begin{macro}{\HoLogoCss@XeLaTeX}
%    \begin{macrocode}
\def\HoLogoCss@XeLaTeX{%
  \Css{%
    span.HoLogo-XeLaTeX span.HoLogo-Xe{%
      margin-right:-.13em;%
    }%
  }%
  \global\let\HoLogoCss@XeLaTeX\relax
}
%    \end{macrocode}
%    \end{macro}
%
% \subsubsection{\hologo{pdfTeX}, \hologo{pdfLaTeX}}
%
%    \begin{macro}{\HoLogo@pdfTeX}
%    \begin{macrocode}
\def\HoLogo@pdfTeX#1{%
  \HOLOGO@mbox{%
    #1{p}{P}df\hologo{TeX}%
  }%
}
%    \end{macrocode}
%    \end{macro}
%    \begin{macro}{\HoLogoCs@pdfTeX}
%    \begin{macrocode}
\def\HoLogoCs@pdfTeX#1{#1{p}{P}dfTeX}
%    \end{macrocode}
%    \end{macro}
%    \begin{macro}{\HoLogoBkm@pdfTeX}
%    \begin{macrocode}
\def\HoLogoBkm@pdfTeX#1{%
  #1{p}{P}df\hologo{TeX}%
}
%    \end{macrocode}
%    \end{macro}
%    \begin{macro}{\HoLogoHtml@pdfTeX}
%    \begin{macrocode}
\let\HoLogoHtml@pdfTeX\HoLogo@pdfTeX
%    \end{macrocode}
%    \end{macro}
%
%    \begin{macro}{\HoLogo@pdfLaTeX}
%    \begin{macrocode}
\def\HoLogo@pdfLaTeX#1{%
  \HOLOGO@mbox{%
    #1{p}{P}df\hologo{LaTeX}%
  }%
}
%    \end{macrocode}
%    \end{macro}
%    \begin{macro}{\HoLogoCs@pdfLaTeX}
%    \begin{macrocode}
\def\HoLogoCs@pdfLaTeX#1{#1{p}{P}dfLaTeX}
%    \end{macrocode}
%    \end{macro}
%    \begin{macro}{\HoLogoBkm@pdfLaTeX}
%    \begin{macrocode}
\def\HoLogoBkm@pdfLaTeX#1{%
  #1{p}{P}df\hologo{LaTeX}%
}
%    \end{macrocode}
%    \end{macro}
%    \begin{macro}{\HoLogoHtml@pdfLaTeX}
%    \begin{macrocode}
\let\HoLogoHtml@pdfLaTeX\HoLogo@pdfLaTeX
%    \end{macrocode}
%    \end{macro}
%
% \subsubsection{\hologo{VTeX}}
%
%    \begin{macro}{\HoLogo@VTeX}
%    \begin{macrocode}
\def\HoLogo@VTeX#1{%
  \HOLOGO@mbox{%
    V\hologo{TeX}%
  }%
}
%    \end{macrocode}
%    \end{macro}
%    \begin{macro}{\HoLogoHtml@VTeX}
%    \begin{macrocode}
\let\HoLogoHtml@VTeX\HoLogo@VTeX
%    \end{macrocode}
%    \end{macro}
%
% \subsubsection{\hologo{AmS}, \dots}
%
%    Source: class \xclass{amsdtx}
%
%    \begin{macro}{\HoLogo@AmS}
%    \begin{macrocode}
\def\HoLogo@AmS#1{%
  \HoLogoFont@font{AmS}{sy}{%
    A%
    \kern-.1667em%
    \lower.5ex\hbox{M}%
    \kern-.125em%
    S%
  }%
}
%    \end{macrocode}
%    \end{macro}
%    \begin{macro}{\HoLogoBkm@AmS}
%    \begin{macrocode}
\def\HoLogoBkm@AmS#1{AmS}
%    \end{macrocode}
%    \end{macro}
%    \begin{macro}{\HoLogoHtml@AmS}
%    \begin{macrocode}
\def\HoLogoHtml@AmS#1{%
  \HoLogoCss@AmS
%  \HoLogoFont@font{AmS}{sy}{%
    \HOLOGO@Span{AmS}{%
      A%
      \HOLOGO@Span{M}{M}%
      S%
    }%
%   }%
}
%    \end{macrocode}
%    \end{macro}
%    \begin{macro}{\HoLogoCss@AmS}
%    \begin{macrocode}
\def\HoLogoCss@AmS{%
  \Css{%
    span.HoLogo-AmS span.HoLogo-M{%
      position:relative;%
      top:.5ex;%
      margin-left:-.1667em;%
      margin-right:-.125em;%
      text-decoration:none;%
    }%
  }%
  \global\let\HoLogoCss@AmS\relax
}
%    \end{macrocode}
%    \end{macro}
%
%    \begin{macro}{\HoLogo@AmSTeX}
%    \begin{macrocode}
\def\HoLogo@AmSTeX#1{%
  \hologo{AmS}%
  \HOLOGO@hyphen
  \hologo{TeX}%
}
%    \end{macrocode}
%    \end{macro}
%    \begin{macro}{\HoLogoBkm@AmSTeX}
%    \begin{macrocode}
\def\HoLogoBkm@AmSTeX#1{AmS-TeX}%
%    \end{macrocode}
%    \end{macro}
%    \begin{macro}{\HoLogoHtml@AmSTeX}
%    \begin{macrocode}
\let\HoLogoHtml@AmSTeX\HoLogo@AmSTeX
%    \end{macrocode}
%    \end{macro}
%
%    \begin{macro}{\HoLogo@AmSLaTeX}
%    \begin{macrocode}
\def\HoLogo@AmSLaTeX#1{%
  \hologo{AmS}%
  \HOLOGO@hyphen
  \hologo{LaTeX}%
}
%    \end{macrocode}
%    \end{macro}
%    \begin{macro}{\HoLogoBkm@AmSLaTeX}
%    \begin{macrocode}
\def\HoLogoBkm@AmSLaTeX#1{AmS-LaTeX}%
%    \end{macrocode}
%    \end{macro}
%    \begin{macro}{\HoLogoHtml@AmSLaTeX}
%    \begin{macrocode}
\let\HoLogoHtml@AmSLaTeX\HoLogo@AmSLaTeX
%    \end{macrocode}
%    \end{macro}
%
% \subsubsection{\hologo{BibTeX}}
%
%    \begin{macro}{\HoLogo@BibTeX@sc}
%    A definition of \hologo{BibTeX} is provided in
%    the documentation source for the manual of \hologo{BibTeX}
%    \cite{btxdoc}.
%\begin{quote}
%\begin{verbatim}
%\def\BibTeX{%
%  {%
%    \rm
%    B%
%    \kern-.05em%
%    {%
%      \sc
%      i%
%      \kern-.025em %
%      b%
%    }%
%    \kern-.08em
%    T%
%    \kern-.1667em%
%    \lower.7ex\hbox{E}%
%    \kern-.125em%
%    X%
%  }%
%}
%\end{verbatim}
%\end{quote}
%    \begin{macrocode}
\def\HoLogo@BibTeX@sc#1{%
  B%
  \kern-.05em%
  \HoLogoFont@font{BibTeX}{sc}{%
    i%
    \kern-.025em%
    b%
  }%
  \HOLOGO@discretionary
  \kern-.08em%
  \hologo{TeX}%
}
%    \end{macrocode}
%    \end{macro}
%    \begin{macro}{\HoLogoHtml@BibTeX@sc}
%    \begin{macrocode}
\def\HoLogoHtml@BibTeX@sc#1{%
  \HoLogoCss@BibTeX@sc
  \HOLOGO@Span{BibTeX-sc}{%
    B%
    \HOLOGO@Span{i}{i}%
    \HOLOGO@Span{b}{b}%
    \hologo{TeX}%
  }%
}
%    \end{macrocode}
%    \end{macro}
%    \begin{macro}{\HoLogoCss@BibTeX@sc}
%    \begin{macrocode}
\def\HoLogoCss@BibTeX@sc{%
  \Css{%
    span.HoLogo-BibTeX-sc span.HoLogo-i{%
      margin-left:-.05em;%
      margin-right:-.025em;%
      font-variant:small-caps;%
    }%
  }%
  \Css{%
    span.HoLogo-BibTeX-sc span.HoLogo-b{%
      margin-right:-.08em;%
      font-variant:small-caps;%
    }%
  }%
  \global\let\HoLogoCss@BibTeX@sc\relax
}
%    \end{macrocode}
%    \end{macro}
%
%    \begin{macro}{\HoLogo@BibTeX@sf}
%    Variant \xoption{sf} avoids trouble with unavailable
%    small caps fonts (e.g., bold versions of Computer Modern or
%    Latin Modern). The definition is taken from
%    package \xpackage{dtklogos} \cite{dtklogos}.
%\begin{quote}
%\begin{verbatim}
%\DeclareRobustCommand{\BibTeX}{%
%  B%
%  \kern-.05em%
%  \hbox{%
%    $\m@th$% %% force math size calculations
%    \csname S@\f@size\endcsname
%    \fontsize\sf@size\z@
%    \math@fontsfalse
%    \selectfont
%    I%
%    \kern-.025em%
%    B
%  }%
%  \kern-.08em%
%  \-%
%  \TeX
%}
%\end{verbatim}
%\end{quote}
%    \begin{macrocode}
\def\HoLogo@BibTeX@sf#1{%
  B%
  \kern-.05em%
  \HoLogoFont@font{BibTeX}{bibsf}{%
    I%
    \kern-.025em%
    B%
  }%
  \HOLOGO@discretionary
  \kern-.08em%
  \hologo{TeX}%
}
%    \end{macrocode}
%    \end{macro}
%    \begin{macro}{\HoLogoHtml@BibTeX@sf}
%    \begin{macrocode}
\def\HoLogoHtml@BibTeX@sf#1{%
  \HoLogoCss@BibTeX@sf
  \HOLOGO@Span{BibTeX-sf}{%
    B%
    \HoLogoFont@font{BibTeX}{bibsf}{%
      \HOLOGO@Span{i}{I}%
      B%
    }%
    \hologo{TeX}%
  }%
}
%    \end{macrocode}
%    \end{macro}
%    \begin{macro}{\HoLogoCss@BibTeX@sf}
%    \begin{macrocode}
\def\HoLogoCss@BibTeX@sf{%
  \Css{%
    span.HoLogo-BibTeX-sf span.HoLogo-i{%
      margin-left:-.05em;%
      margin-right:-.025em;%
    }%
  }%
  \Css{%
    span.HoLogo-BibTeX-sf span.HoLogo-TeX{%
      margin-left:-.08em;%
    }%
  }%
  \global\let\HoLogoCss@BibTeX@sf\relax
}
%    \end{macrocode}
%    \end{macro}
%
%    \begin{macro}{\HoLogo@BibTeX}
%    \begin{macrocode}
\def\HoLogo@BibTeX{\HoLogo@BibTeX@sf}
%    \end{macrocode}
%    \end{macro}
%    \begin{macro}{\HoLogoHtml@BibTeX}
%    \begin{macrocode}
\def\HoLogoHtml@BibTeX{\HoLogoHtml@BibTeX@sf}
%    \end{macrocode}
%    \end{macro}
%
% \subsubsection{\hologo{BibTeX8}}
%
%    \begin{macro}{\HoLogo@BibTeX8}
%    \begin{macrocode}
\expandafter\def\csname HoLogo@BibTeX8\endcsname#1{%
  \hologo{BibTeX}%
  8%
}
%    \end{macrocode}
%    \end{macro}
%
%    \begin{macro}{\HoLogoBkm@BibTeX8}
%    \begin{macrocode}
\expandafter\def\csname HoLogoBkm@BibTeX8\endcsname#1{%
  \hologo{BibTeX}%
  8%
}
%    \end{macrocode}
%    \end{macro}
%    \begin{macro}{\HoLogoHtml@BibTeX8}
%    \begin{macrocode}
\expandafter
\let\csname HoLogoHtml@BibTeX8\expandafter\endcsname
\csname HoLogo@BibTeX8\endcsname
%    \end{macrocode}
%    \end{macro}
%
% \subsubsection{\hologo{ConTeXt}}
%
%    \begin{macro}{\HoLogo@ConTeXt@simple}
%    \begin{macrocode}
\def\HoLogo@ConTeXt@simple#1{%
  \HOLOGO@mbox{Con}%
  \HOLOGO@discretionary
  \HOLOGO@mbox{\hologo{TeX}t}%
}
%    \end{macrocode}
%    \end{macro}
%    \begin{macro}{\HoLogoHtml@ConTeXt@simple}
%    \begin{macrocode}
\let\HoLogoHtml@ConTeXt@simple\HoLogo@ConTeXt@simple
%    \end{macrocode}
%    \end{macro}
%
%    \begin{macro}{\HoLogo@ConTeXt@narrow}
%    This definition of logo \hologo{ConTeXt} with variant \xoption{narrow}
%    comes from TUGboat's class \xclass{ltugboat} (version 2010/11/15 v2.8).
%    \begin{macrocode}
\def\HoLogo@ConTeXt@narrow#1{%
  \HOLOGO@mbox{C\kern-.0333emon}%
  \HOLOGO@discretionary
  \kern-.0667em%
  \HOLOGO@mbox{\hologo{TeX}\kern-.0333emt}%
}
%    \end{macrocode}
%    \end{macro}
%    \begin{macro}{\HoLogoHtml@ConTeXt@narrow}
%    \begin{macrocode}
\def\HoLogoHtml@ConTeXt@narrow#1{%
  \HoLogoCss@ConTeXt@narrow
  \HOLOGO@Span{ConTeXt-narrow}{%
    \HOLOGO@Span{C}{C}%
    on%
    \hologo{TeX}%
    t%
  }%
}
%    \end{macrocode}
%    \end{macro}
%    \begin{macro}{\HoLogoCss@ConTeXt@narrow}
%    \begin{macrocode}
\def\HoLogoCss@ConTeXt@narrow{%
  \Css{%
    span.HoLogo-ConTeXt-narrow span.HoLogo-C{%
      margin-left:-.0333em;%
    }%
  }%
  \Css{%
    span.HoLogo-ConTeXt-narrow span.HoLogo-TeX{%
      margin-left:-.0667em;%
      margin-right:-.0333em;%
    }%
  }%
  \global\let\HoLogoCss@ConTeXt@narrow\relax
}
%    \end{macrocode}
%    \end{macro}
%
%    \begin{macro}{\HoLogo@ConTeXt}
%    \begin{macrocode}
\def\HoLogo@ConTeXt{\HoLogo@ConTeXt@narrow}
%    \end{macrocode}
%    \end{macro}
%    \begin{macro}{\HoLogoHtml@ConTeXt}
%    \begin{macrocode}
\def\HoLogoHtml@ConTeXt{\HoLogoHtml@ConTeXt@narrow}
%    \end{macrocode}
%    \end{macro}
%
% \subsubsection{\hologo{emTeX}}
%
%    \begin{macro}{\HoLogo@emTeX}
%    \begin{macrocode}
\def\HoLogo@emTeX#1{%
  \HOLOGO@mbox{#1{e}{E}m}%
  \HOLOGO@discretionary
  \hologo{TeX}%
}
%    \end{macrocode}
%    \end{macro}
%    \begin{macro}{\HoLogoCs@emTeX}
%    \begin{macrocode}
\def\HoLogoCs@emTeX#1{#1{e}{E}mTeX}%
%    \end{macrocode}
%    \end{macro}
%    \begin{macro}{\HoLogoBkm@emTeX}
%    \begin{macrocode}
\def\HoLogoBkm@emTeX#1{%
  #1{e}{E}m\hologo{TeX}%
}
%    \end{macrocode}
%    \end{macro}
%    \begin{macro}{\HoLogoHtml@emTeX}
%    \begin{macrocode}
\let\HoLogoHtml@emTeX\HoLogo@emTeX
%    \end{macrocode}
%    \end{macro}
%
% \subsubsection{\hologo{ExTeX}}
%
%    \begin{macro}{\HoLogo@ExTeX}
%    The definition is taken from the FAQ of the
%    project \hologo{ExTeX}
%    \cite{ExTeX-FAQ}.
%\begin{quote}
%\begin{verbatim}
%\def\ExTeX{%
%  \textrm{% Logo always with serifs
%    \ensuremath{%
%      \textstyle
%      \varepsilon_{%
%        \kern-0.15em%
%        \mathcal{X}%
%      }%
%    }%
%    \kern-.15em%
%    \TeX
%  }%
%}
%\end{verbatim}
%\end{quote}
%    \begin{macrocode}
\def\HoLogo@ExTeX#1{%
  \HoLogoFont@font{ExTeX}{rm}{%
    \ltx@mbox{%
      \HOLOGO@MathSetup
      $%
        \textstyle
        \varepsilon_{%
          \kern-0.15em%
          \HoLogoFont@font{ExTeX}{sy}{X}%
        }%
      $%
    }%
    \HOLOGO@discretionary
    \kern-.15em%
    \hologo{TeX}%
  }%
}
%    \end{macrocode}
%    \end{macro}
%    \begin{macro}{\HoLogoHtml@ExTeX}
%    \begin{macrocode}
\def\HoLogoHtml@ExTeX#1{%
  \HoLogoCss@ExTeX
  \HoLogoFont@font{ExTeX}{rm}{%
    \HOLOGO@Span{ExTeX}{%
      \ltx@mbox{%
        \HOLOGO@MathSetup
        $\textstyle\varepsilon$%
        \HOLOGO@Span{X}{$\textstyle\chi$}%
        \hologo{TeX}%
      }%
    }%
  }%
}
%    \end{macrocode}
%    \end{macro}
%    \begin{macro}{\HoLogoBkm@ExTeX}
%    \begin{macrocode}
\def\HoLogoBkm@ExTeX#1{%
  \HOLOGO@PdfdocUnicode{#1{e}{E}x}{\textepsilon\textchi}%
  \hologo{TeX}%
}
%    \end{macrocode}
%    \end{macro}
%    \begin{macro}{\HoLogoCss@ExTeX}
%    \begin{macrocode}
\def\HoLogoCss@ExTeX{%
  \Css{%
    span.HoLogo-ExTeX{%
      font-family:serif;%
    }%
  }%
  \Css{%
    span.HoLogo-ExTeX span.HoLogo-TeX{%
      margin-left:-.15em;%
    }%
  }%
  \global\let\HoLogoCss@ExTeX\relax
}
%    \end{macrocode}
%    \end{macro}
%
% \subsubsection{\hologo{MiKTeX}}
%
%    \begin{macro}{\HoLogo@MiKTeX}
%    \begin{macrocode}
\def\HoLogo@MiKTeX#1{%
  \HOLOGO@mbox{MiK}%
  \HOLOGO@discretionary
  \hologo{TeX}%
}
%    \end{macrocode}
%    \end{macro}
%    \begin{macro}{\HoLogoHtml@MiKTeX}
%    \begin{macrocode}
\let\HoLogoHtml@MiKTeX\HoLogo@MiKTeX
%    \end{macrocode}
%    \end{macro}
%
% \subsubsection{\hologo{OzTeX} and friends}
%
%    Source: \hologo{OzTeX} FAQ \cite{OzTeX}:
%    \begin{quote}
%      |\def\OzTeX{O\kern-.03em z\kern-.15em\TeX}|\\
%      (There is no kerning in OzMF, OzMP and OzTtH.)
%    \end{quote}
%
%    \begin{macro}{\HoLogo@OzTeX}
%    \begin{macrocode}
\def\HoLogo@OzTeX#1{%
  O%
  \kern-.03em %
  z%
  \kern-.15em %
  \hologo{TeX}%
}
%    \end{macrocode}
%    \end{macro}
%    \begin{macro}{\HoLogoHtml@OzTeX}
%    \begin{macrocode}
\def\HoLogoHtml@OzTeX#1{%
  \HoLogoCss@OzTeX
  \HOLOGO@Span{OzTeX}{%
    O%
    \HOLOGO@Span{z}{z}%
    \hologo{TeX}%
  }%
}
%    \end{macrocode}
%    \end{macro}
%    \begin{macro}{\HoLogoCss@OzTeX}
%    \begin{macrocode}
\def\HoLogoCss@OzTeX{%
  \Css{%
    span.HoLogo-OzTeX span.HoLogo-z{%
      margin-left:-.03em;%
      margin-right:-.15em;%
    }%
  }%
  \global\let\HoLogoCss@OzTeX\relax
}
%    \end{macrocode}
%    \end{macro}
%
%    \begin{macro}{\HoLogo@OzMF}
%    \begin{macrocode}
\def\HoLogo@OzMF#1{%
  \HOLOGO@mbox{OzMF}%
}
%    \end{macrocode}
%    \end{macro}
%    \begin{macro}{\HoLogo@OzMP}
%    \begin{macrocode}
\def\HoLogo@OzMP#1{%
  \HOLOGO@mbox{OzMP}%
}
%    \end{macrocode}
%    \end{macro}
%    \begin{macro}{\HoLogo@OzTtH}
%    \begin{macrocode}
\def\HoLogo@OzTtH#1{%
  \HOLOGO@mbox{OzTtH}%
}
%    \end{macrocode}
%    \end{macro}
%
% \subsubsection{\hologo{PCTeX}}
%
%    \begin{macro}{\HoLogo@PCTeX}
%    \begin{macrocode}
\def\HoLogo@PCTeX#1{%
  \HOLOGO@mbox{PC}%
  \hologo{TeX}%
}
%    \end{macrocode}
%    \end{macro}
%    \begin{macro}{\HoLogoHtml@PCTeX}
%    \begin{macrocode}
\let\HoLogoHtml@PCTeX\HoLogo@PCTeX
%    \end{macrocode}
%    \end{macro}
%
% \subsubsection{\hologo{PiCTeX}}
%
%    The original definitions from \xfile{pictex.tex} \cite{PiCTeX}:
%\begin{quote}
%\begin{verbatim}
%\def\PiC{%
%  P%
%  \kern-.12em%
%  \lower.5ex\hbox{I}%
%  \kern-.075em%
%  C%
%}
%\def\PiCTeX{%
%  \PiC
%  \kern-.11em%
%  \TeX
%}
%\end{verbatim}
%\end{quote}
%
%    \begin{macro}{\HoLogo@PiC}
%    \begin{macrocode}
\def\HoLogo@PiC#1{%
  P%
  \kern-.12em%
  \lower.5ex\hbox{I}%
  \kern-.075em%
  C%
  \HOLOGO@SpaceFactor
}
%    \end{macrocode}
%    \end{macro}
%    \begin{macro}{\HoLogoHtml@PiC}
%    \begin{macrocode}
\def\HoLogoHtml@PiC#1{%
  \HoLogoCss@PiC
  \HOLOGO@Span{PiC}{%
    P%
    \HOLOGO@Span{i}{I}%
    C%
  }%
}
%    \end{macrocode}
%    \end{macro}
%    \begin{macro}{\HoLogoCss@PiC}
%    \begin{macrocode}
\def\HoLogoCss@PiC{%
  \Css{%
    span.HoLogo-PiC span.HoLogo-i{%
      position:relative;%
      top:.5ex;%
      margin-left:-.12em;%
      margin-right:-.075em;%
      text-decoration:none;%
    }%
  }%
  \global\let\HoLogoCss@PiC\relax
}
%    \end{macrocode}
%    \end{macro}
%
%    \begin{macro}{\HoLogo@PiCTeX}
%    \begin{macrocode}
\def\HoLogo@PiCTeX#1{%
  \hologo{PiC}%
  \HOLOGO@discretionary
  \kern-.11em%
  \hologo{TeX}%
}
%    \end{macrocode}
%    \end{macro}
%    \begin{macro}{\HoLogoHtml@PiCTeX}
%    \begin{macrocode}
\def\HoLogoHtml@PiCTeX#1{%
  \HoLogoCss@PiCTeX
  \HOLOGO@Span{PiCTeX}{%
    \hologo{PiC}%
    \hologo{TeX}%
  }%
}
%    \end{macrocode}
%    \end{macro}
%    \begin{macro}{\HoLogoCss@PiCTeX}
%    \begin{macrocode}
\def\HoLogoCss@PiCTeX{%
  \Css{%
    span.HoLogo-PiCTeX span.HoLogo-PiC{%
      margin-right:-.11em;%
    }%
  }%
  \global\let\HoLogoCss@PiCTeX\relax
}
%    \end{macrocode}
%    \end{macro}
%
% \subsubsection{\hologo{teTeX}}
%
%    \begin{macro}{\HoLogo@teTeX}
%    \begin{macrocode}
\def\HoLogo@teTeX#1{%
  \HOLOGO@mbox{#1{t}{T}e}%
  \HOLOGO@discretionary
  \hologo{TeX}%
}
%    \end{macrocode}
%    \end{macro}
%    \begin{macro}{\HoLogoCs@teTeX}
%    \begin{macrocode}
\def\HoLogoCs@teTeX#1{#1{t}{T}dfTeX}
%    \end{macrocode}
%    \end{macro}
%    \begin{macro}{\HoLogoBkm@teTeX}
%    \begin{macrocode}
\def\HoLogoBkm@teTeX#1{%
  #1{t}{T}e\hologo{TeX}%
}
%    \end{macrocode}
%    \end{macro}
%    \begin{macro}{\HoLogoHtml@teTeX}
%    \begin{macrocode}
\let\HoLogoHtml@teTeX\HoLogo@teTeX
%    \end{macrocode}
%    \end{macro}
%
% \subsubsection{\hologo{TeX4ht}}
%
%    \begin{macro}{\HoLogo@TeX4ht}
%    \begin{macrocode}
\expandafter\def\csname HoLogo@TeX4ht\endcsname#1{%
  \HOLOGO@mbox{\hologo{TeX}4ht}%
}
%    \end{macrocode}
%    \end{macro}
%    \begin{macro}{\HoLogoHtml@TeX4ht}
%    \begin{macrocode}
\expandafter
\let\csname HoLogoHtml@TeX4ht\expandafter\endcsname
\csname HoLogo@TeX4ht\endcsname
%    \end{macrocode}
%    \end{macro}
%
%
% \subsubsection{\hologo{SageTeX}}
%
%    \begin{macro}{\HoLogo@SageTeX}
%    \begin{macrocode}
\def\HoLogo@SageTeX#1{%
  \HOLOGO@mbox{Sage}%
  \HOLOGO@discretionary
  \HOLOGO@NegativeKerning{eT,oT,To}%
  \hologo{TeX}%
}
%    \end{macrocode}
%    \end{macro}
%    \begin{macro}{\HoLogoHtml@SageTeX}
%    \begin{macrocode}
\let\HoLogoHtml@SageTeX\HoLogo@SageTeX
%    \end{macrocode}
%    \end{macro}
%
% \subsection{\hologo{METAFONT} and friends}
%
%    \begin{macro}{\HoLogo@METAFONT}
%    \begin{macrocode}
\def\HoLogo@METAFONT#1{%
  \HoLogoFont@font{METAFONT}{logo}{%
    \HOLOGO@mbox{META}%
    \HOLOGO@discretionary
    \HOLOGO@mbox{FONT}%
  }%
}
%    \end{macrocode}
%    \end{macro}
%
%    \begin{macro}{\HoLogo@METAPOST}
%    \begin{macrocode}
\def\HoLogo@METAPOST#1{%
  \HoLogoFont@font{METAPOST}{logo}{%
    \HOLOGO@mbox{META}%
    \HOLOGO@discretionary
    \HOLOGO@mbox{POST}%
  }%
}
%    \end{macrocode}
%    \end{macro}
%
%    \begin{macro}{\HoLogo@MetaFun}
%    \begin{macrocode}
\def\HoLogo@MetaFun#1{%
  \HOLOGO@mbox{Meta}%
  \HOLOGO@discretionary
  \HOLOGO@mbox{Fun}%
}
%    \end{macrocode}
%    \end{macro}
%
%    \begin{macro}{\HoLogo@MetaPost}
%    \begin{macrocode}
\def\HoLogo@MetaPost#1{%
  \HOLOGO@mbox{Meta}%
  \HOLOGO@discretionary
  \HOLOGO@mbox{Post}%
}
%    \end{macrocode}
%    \end{macro}
%
% \subsection{Others}
%
% \subsubsection{\hologo{biber}}
%
%    \begin{macro}{\HoLogo@biber}
%    \begin{macrocode}
\def\HoLogo@biber#1{%
  \HOLOGO@mbox{#1{b}{B}i}%
  \HOLOGO@discretionary
  \HOLOGO@mbox{ber}%
}
%    \end{macrocode}
%    \end{macro}
%    \begin{macro}{\HoLogoCs@biber}
%    \begin{macrocode}
\def\HoLogoCs@biber#1{#1{b}{B}iber}
%    \end{macrocode}
%    \end{macro}
%    \begin{macro}{\HoLogoBkm@biber}
%    \begin{macrocode}
\def\HoLogoBkm@biber#1{%
  #1{b}{B}iber%
}
%    \end{macrocode}
%    \end{macro}
%    \begin{macro}{\HoLogoHtml@biber}
%    \begin{macrocode}
\let\HoLogoHtml@biber\HoLogo@biber
%    \end{macrocode}
%    \end{macro}
%
% \subsubsection{\hologo{KOMAScript}}
%
%    \begin{macro}{\HoLogo@KOMAScript}
%    The definition for \hologo{KOMAScript} is taken
%    from \hologo{KOMAScript} (\xfile{scrlogo.dtx}, reformatted) \cite{scrlogo}:
%\begin{quote}
%\begin{verbatim}
%\@ifundefined{KOMAScript}{%
%  \DeclareRobustCommand{\KOMAScript}{%
%    \textsf{%
%      K\kern.05em O\kern.05emM\kern.05em A%
%      \kern.1em-\kern.1em %
%      Script%
%    }%
%  }%
%}{}
%\end{verbatim}
%\end{quote}
%    \begin{macrocode}
\def\HoLogo@KOMAScript#1{%
  \HoLogoFont@font{KOMAScript}{sf}{%
    \HOLOGO@mbox{%
      K\kern.05em%
      O\kern.05em%
      M\kern.05em%
      A%
    }%
    \kern.1em%
    \HOLOGO@hyphen
    \kern.1em%
    \HOLOGO@mbox{Script}%
  }%
}
%    \end{macrocode}
%    \end{macro}
%    \begin{macro}{\HoLogoBkm@KOMAScript}
%    \begin{macrocode}
\def\HoLogoBkm@KOMAScript#1{%
  KOMA-Script%
}
%    \end{macrocode}
%    \end{macro}
%    \begin{macro}{\HoLogoHtml@KOMAScript}
%    \begin{macrocode}
\def\HoLogoHtml@KOMAScript#1{%
  \HoLogoCss@KOMAScript
  \HoLogoFont@font{KOMAScript}{sf}{%
    \HOLOGO@Span{KOMAScript}{%
      K%
      \HOLOGO@Span{O}{O}%
      M%
      \HOLOGO@Span{A}{A}%
      \HOLOGO@Span{hyphen}{-}%
      Script%
    }%
  }%
}
%    \end{macrocode}
%    \end{macro}
%    \begin{macro}{\HoLogoCss@KOMAScript}
%    \begin{macrocode}
\def\HoLogoCss@KOMAScript{%
  \Css{%
    span.HoLogo-KOMAScript{%
      font-family:sans-serif;%
    }%
  }%
  \Css{%
    span.HoLogo-KOMAScript span.HoLogo-O{%
      padding-left:.05em;%
      padding-right:.05em;%
    }%
  }%
  \Css{%
    span.HoLogo-KOMAScript span.HoLogo-A{%
      padding-left:.05em;%
    }%
  }%
  \Css{%
    span.HoLogo-KOMAScript span.HoLogo-hyphen{%
      padding-left:.1em;%
      padding-right:.1em;%
    }%
  }%
  \global\let\HoLogoCss@KOMAScript\relax
}
%    \end{macrocode}
%    \end{macro}
%
% \subsubsection{\hologo{LyX}}
%
%    \begin{macro}{\HoLogo@LyX}
%    The definition is taken from the documentation source files
%    of \hologo{LyX}, \xfile{Intro.lyx} \cite{LyX}:
%\begin{quote}
%\begin{verbatim}
%\def\LyX{%
%  \texorpdfstring{%
%    L\kern-.1667em\lower.25em\hbox{Y}\kern-.125emX\@%
%  }{%
%    LyX%
%  }%
%}
%\end{verbatim}
%\end{quote}
%    \begin{macrocode}
\def\HoLogo@LyX#1{%
  L%
  \kern-.1667em%
  \lower.25em\hbox{Y}%
  \kern-.125em%
  X%
  \HOLOGO@SpaceFactor
}
%    \end{macrocode}
%    \end{macro}
%    \begin{macro}{\HoLogoHtml@LyX}
%    \begin{macrocode}
\def\HoLogoHtml@LyX#1{%
  \HoLogoCss@LyX
  \HOLOGO@Span{LyX}{%
    L%
    \HOLOGO@Span{y}{Y}%
    X%
  }%
}
%    \end{macrocode}
%    \end{macro}
%    \begin{macro}{\HoLogoCss@LyX}
%    \begin{macrocode}
\def\HoLogoCss@LyX{%
  \Css{%
    span.HoLogo-LyX span.HoLogo-y{%
      position:relative;%
      top:.25em;%
      margin-left:-.1667em;%
      margin-right:-.125em;%
      text-decoration:none;%
    }%
  }%
  \global\let\HoLogoCss@LyX\relax
}
%    \end{macrocode}
%    \end{macro}
%
% \subsubsection{\hologo{NTS}}
%
%    \begin{macro}{\HoLogo@NTS}
%    Definition for \hologo{NTS} can be found in
%    package \xpackage{etex\textunderscore man} for the \hologo{eTeX} manual \cite{etexman}
%    and in package \xpackage{dtklogos} \cite{dtklogos}:
%\begin{quote}
%\begin{verbatim}
%\def\NTS{%
%  \leavevmode
%  \hbox{%
%    $%
%      \cal N%
%      \kern-0.35em%
%      \lower0.5ex\hbox{$\cal T$}%
%      \kern-0.2em%
%      S%
%    $%
%  }%
%}
%\end{verbatim}
%\end{quote}
%    \begin{macrocode}
\def\HoLogo@NTS#1{%
  \HoLogoFont@font{NTS}{sy}{%
    N\/%
    \kern-.35em%
    \lower.5ex\hbox{T\/}%
    \kern-.2em%
    S\/%
  }%
  \HOLOGO@SpaceFactor
}
%    \end{macrocode}
%    \end{macro}
%
% \subsubsection{\Hologo{TTH} (\hologo{TeX} to HTML translator)}
%
%    Source: \url{http://hutchinson.belmont.ma.us/tth/}
%    In the HTML source the second `T' is printed as subscript.
%\begin{quote}
%\begin{verbatim}
%T<sub>T</sub>H
%\end{verbatim}
%\end{quote}
%    \begin{macro}{\HoLogo@TTH}
%    \begin{macrocode}
\def\HoLogo@TTH#1{%
  \ltx@mbox{%
    T\HOLOGO@SubScript{T}H%
  }%
  \HOLOGO@SpaceFactor
}
%    \end{macrocode}
%    \end{macro}
%
%    \begin{macro}{\HoLogoHtml@TTH}
%    \begin{macrocode}
\def\HoLogoHtml@TTH#1{%
  T\HCode{<sub>}T\HCode{</sub>}H%
}
%    \end{macrocode}
%    \end{macro}
%
% \subsubsection{\Hologo{HanTheThanh}}
%
%    Partial source: Package \xpackage{dtklogos}.
%    The double accent is U+1EBF (latin small letter e with circumflex
%    and acute).
%    \begin{macro}{\HoLogo@HanTheThanh}
%    \begin{macrocode}
\def\HoLogo@HanTheThanh#1{%
  \ltx@mbox{H\`an}%
  \HOLOGO@space
  \ltx@mbox{%
    Th%
    \HOLOGO@IfCharExists{"1EBF}{%
      \char"1EBF\relax
    }{%
      \^e\hbox to 0pt{\hss\raise .5ex\hbox{\'{}}}%
    }%
  }%
  \HOLOGO@space
  \ltx@mbox{Th\`anh}%
}
%    \end{macrocode}
%    \end{macro}
%    \begin{macro}{\HoLogoBkm@HanTheThanh}
%    \begin{macrocode}
\def\HoLogoBkm@HanTheThanh#1{%
  H\`an %
  Th\HOLOGO@PdfdocUnicode{\^e}{\9036\277} %
  Th\`anh%
}
%    \end{macrocode}
%    \end{macro}
%    \begin{macro}{\HoLogoHtml@HanTheThanh}
%    \begin{macrocode}
\def\HoLogoHtml@HanTheThanh#1{%
  H\`an %
  Th\HCode{&\ltx@hashchar x1ebf;} %
  Th\`anh%
}
%    \end{macrocode}
%    \end{macro}
%
% \subsection{Driver detection}
%
%    \begin{macrocode}
\HOLOGO@IfExists\InputIfFileExists{%
  \InputIfFileExists{hologo.cfg}{}{}%
}{%
  \ltx@IfUndefined{pdf@filesize}{%
    \def\HOLOGO@InputIfExists{%
      \openin\HOLOGO@temp=hologo.cfg\relax
      \ifeof\HOLOGO@temp
        \closein\HOLOGO@temp
      \else
        \closein\HOLOGO@temp
        \begingroup
          \def\x{LaTeX2e}%
        \expandafter\endgroup
        \ifx\fmtname\x
          \input{hologo.cfg}%
        \else
          \input hologo.cfg\relax
        \fi
      \fi
    }%
    \ltx@IfUndefined{newread}{%
      \chardef\HOLOGO@temp=15 %
      \def\HOLOGO@CheckRead{%
        \ifeof\HOLOGO@temp
          \HOLOGO@InputIfExists
        \else
          \ifcase\HOLOGO@temp
            \@PackageWarningNoLine{hologo}{%
              Configuration file ignored, because\MessageBreak
              a free read register could not be found%
            }%
          \else
            \begingroup
              \count\ltx@cclv=\HOLOGO@temp
              \advance\ltx@cclv by \ltx@minusone
              \edef\x{\endgroup
                \chardef\noexpand\HOLOGO@temp=\the\count\ltx@cclv
                \relax
              }%
            \x
          \fi
        \fi
      }%
    }{%
      \csname newread\endcsname\HOLOGO@temp
      \HOLOGO@InputIfExists
    }%
  }{%
    \edef\HOLOGO@temp{\pdf@filesize{hologo.cfg}}%
    \ifx\HOLOGO@temp\ltx@empty
    \else
      \ifnum\HOLOGO@temp>0 %
        \begingroup
          \def\x{LaTeX2e}%
        \expandafter\endgroup
        \ifx\fmtname\x
          \input{hologo.cfg}%
        \else
          \input hologo.cfg\relax
        \fi
      \else
        \@PackageInfoNoLine{hologo}{%
          Empty configuration file `hologo.cfg' ignored%
        }%
      \fi
    \fi
  }%
}
%    \end{macrocode}
%
%    \begin{macrocode}
\def\HOLOGO@temp#1#2{%
  \kv@define@key{HoLogoDriver}{#1}[]{%
    \begingroup
      \def\HOLOGO@temp{##1}%
      \ltx@onelevel@sanitize\HOLOGO@temp
      \ifx\HOLOGO@temp\ltx@empty
      \else
        \@PackageError{hologo}{%
          Value (\HOLOGO@temp) not permitted for option `#1'%
        }%
        \@ehc
      \fi
    \endgroup
    \def\hologoDriver{#2}%
  }%
}%
\def\HOLOGO@@temp#1#2{%
  \ifx\kv@value\relax
    \HOLOGO@temp{#1}{#1}%
  \else
    \HOLOGO@temp{#1}{#2}%
  \fi
}%
\kv@parse@normalized{%
  pdftex,%
  luatex=pdftex,%
  dvipdfm,%
  dvipdfmx=dvipdfm,%
  dvips,%
  dvipsone=dvips,%
  xdvi=dvips,%
  xetex,%
  vtex,%
}\HOLOGO@@temp
%    \end{macrocode}
%
%    \begin{macrocode}
\kv@define@key{HoLogoDriver}{driverfallback}{%
  \def\HOLOGO@DriverFallback{#1}%
}
%    \end{macrocode}
%
%    \begin{macro}{\HOLOGO@DriverFallback}
%    \begin{macrocode}
\def\HOLOGO@DriverFallback{dvips}
%    \end{macrocode}
%    \end{macro}
%
%    \begin{macro}{\hologoDriverSetup}
%    \begin{macrocode}
\def\hologoDriverSetup{%
  \let\hologoDriver\ltx@undefined
  \HOLOGO@DriverSetup
}
%    \end{macrocode}
%    \end{macro}
%
%    \begin{macro}{\HOLOGO@DriverSetup}
%    \begin{macrocode}
\def\HOLOGO@DriverSetup#1{%
  \kvsetkeys{HoLogoDriver}{#1}%
  \HOLOGO@CheckDriver
  \ltx@ifundefined{hologoDriver}{%
    \begingroup
    \edef\x{\endgroup
      \noexpand\kvsetkeys{HoLogoDriver}{\HOLOGO@DriverFallback}%
    }\x
  }{}%
  \@PackageInfoNoLine{hologo}{Using driver `\hologoDriver'}%
}
%    \end{macrocode}
%    \end{macro}
%
%    \begin{macro}{\HOLOGO@CheckDriver}
%    \begin{macrocode}
\def\HOLOGO@CheckDriver{%
  \ifpdf
    \def\hologoDriver{pdftex}%
    \let\HOLOGO@pdfliteral\pdfliteral
    \ifluatex
      \ifx\pdfextension\@undefined\else
        \protected\def\pdfliteral{\pdfextension literal}%
        \let\HOLOGO@pdfliteral\pdfliteral
      \fi
      \ltx@IfUndefined{HOLOGO@pdfliteral}{%
        \ifnum\luatexversion<36 %
        \else
          \begingroup
            \let\HOLOGO@temp\endgroup
            \ifcase0%
                \directlua{%
                  if tex.enableprimitives then %
                    tex.enableprimitives('HOLOGO@', {'pdfliteral'})%
                  else %
                    tex.print('1')%
                  end%
                }%
                \ifx\HOLOGO@pdfliteral\@undefined 1\fi%
                \relax%
              \endgroup
              \let\HOLOGO@temp\relax
              \global\let\HOLOGO@pdfliteral\HOLOGO@pdfliteral
            \fi%
          \HOLOGO@temp
        \fi
      }{}%
    \fi
    \ltx@IfUndefined{HOLOGO@pdfliteral}{%
      \@PackageWarningNoLine{hologo}{%
        Cannot find \string\pdfliteral
      }%
    }{}%
  \else
    \ifxetex
      \def\hologoDriver{xetex}%
    \else
      \ifvtex
        \def\hologoDriver{vtex}%
      \fi
    \fi
  \fi
}
%    \end{macrocode}
%    \end{macro}
%
%    \begin{macro}{\HOLOGO@WarningUnsupportedDriver}
%    \begin{macrocode}
\def\HOLOGO@WarningUnsupportedDriver#1{%
  \@PackageWarningNoLine{hologo}{%
    Logo `#1' needs driver specific macros,\MessageBreak
    but driver `\hologoDriver' is not supported.\MessageBreak
    Use a different driver or\MessageBreak
    load package `graphics' or `pgf'%
  }%
}
%    \end{macrocode}
%    \end{macro}
%
% \subsubsection{Reflect box macros}
%
%    Skip driver part if not needed.
%    \begin{macrocode}
\ltx@IfUndefined{reflectbox}{}{%
  \ltx@IfUndefined{rotatebox}{}{%
    \HOLOGO@AtEnd
  }%
}
\ltx@IfUndefined{pgftext}{}{%
  \HOLOGO@AtEnd
}
\ltx@IfUndefined{psscalebox}{}{%
  \HOLOGO@AtEnd
}
%    \end{macrocode}
%
%    \begin{macrocode}
\def\HOLOGO@temp{LaTeX2e}
\ifx\fmtname\HOLOGO@temp
  \RequirePackage{kvoptions}[2011/06/30]%
  \ProcessKeyvalOptions{HoLogoDriver}%
\fi
\HOLOGO@DriverSetup{}
%    \end{macrocode}
%
%    \begin{macro}{\HOLOGO@ReflectBox}
%    \begin{macrocode}
\def\HOLOGO@ReflectBox#1{%
  \begingroup
    \setbox\ltx@zero\hbox{\begingroup#1\endgroup}%
    \setbox\ltx@two\hbox{%
      \kern\wd\ltx@zero
      \csname HOLOGO@ScaleBox@\hologoDriver\endcsname{-1}{1}{%
        \hbox to 0pt{\copy\ltx@zero\hss}%
      }%
    }%
    \wd\ltx@two=\wd\ltx@zero
    \box\ltx@two
  \endgroup
}
%    \end{macrocode}
%    \end{macro}
%
%    \begin{macro}{\HOLOGO@PointReflectBox}
%    \begin{macrocode}
\def\HOLOGO@PointReflectBox#1{%
  \begingroup
    \setbox\ltx@zero\hbox{\begingroup#1\endgroup}%
    \setbox\ltx@two\hbox{%
      \kern\wd\ltx@zero
      \raise\ht\ltx@zero\hbox{%
        \csname HOLOGO@ScaleBox@\hologoDriver\endcsname{-1}{-1}{%
          \hbox to 0pt{\copy\ltx@zero\hss}%
        }%
      }%
    }%
    \wd\ltx@two=\wd\ltx@zero
    \box\ltx@two
  \endgroup
}
%    \end{macrocode}
%    \end{macro}
%
%    We must define all variants because of dynamic driver setup.
%    \begin{macrocode}
\def\HOLOGO@temp#1#2{#2}
%    \end{macrocode}
%
%    \begin{macro}{\HOLOGO@ScaleBox@pdftex}
%    \begin{macrocode}
\HOLOGO@temp{pdftex}{%
  \def\HOLOGO@ScaleBox@pdftex#1#2#3{%
    \HOLOGO@pdfliteral{%
      q #1 0 0 #2 0 0 cm%
    }%
    #3%
    \HOLOGO@pdfliteral{%
      Q%
    }%
  }%
}
%    \end{macrocode}
%    \end{macro}
%    \begin{macro}{\HOLOGO@ScaleBox@dvips}
%    \begin{macrocode}
\HOLOGO@temp{dvips}{%
  \def\HOLOGO@ScaleBox@dvips#1#2#3{%
    \special{ps:%
      gsave %
      currentpoint %
      currentpoint translate %
      #1 #2 scale %
      neg exch neg exch translate%
    }%
    #3%
    \special{ps:%
      currentpoint %
      grestore %
      moveto%
    }%
  }%
}
%    \end{macrocode}
%    \end{macro}
%    \begin{macro}{\HOLOGO@ScaleBox@dvipdfm}
%    \begin{macrocode}
\HOLOGO@temp{dvipdfm}{%
  \let\HOLOGO@ScaleBox@dvipdfm\HOLOGO@ScaleBox@dvips
}
%    \end{macrocode}
%    \end{macro}
%    Since \hologo{XeTeX} v0.6.
%    \begin{macro}{\HOLOGO@ScaleBox@xetex}
%    \begin{macrocode}
\HOLOGO@temp{xetex}{%
  \def\HOLOGO@ScaleBox@xetex#1#2#3{%
    \special{x:gsave}%
    \special{x:scale #1 #2}%
    #3%
    \special{x:grestore}%
  }%
}
%    \end{macrocode}
%    \end{macro}
%    \begin{macro}{\HOLOGO@ScaleBox@vtex}
%    \begin{macrocode}
\HOLOGO@temp{vtex}{%
  \def\HOLOGO@ScaleBox@vtex#1#2#3{%
    \special{r(#1,0,0,#2,0,0}%
    #3%
    \special{r)}%
  }%
}
%    \end{macrocode}
%    \end{macro}
%
%    \begin{macrocode}
\HOLOGO@AtEnd%
%</package>
%    \end{macrocode}
%
% \section{Test}
%
% \subsection{Catcode checks for loading}
%
%    \begin{macrocode}
%<*test1>
%    \end{macrocode}
%    \begin{macrocode}
\catcode`\{=1 %
\catcode`\}=2 %
\catcode`\#=6 %
\catcode`\@=11 %
\expandafter\ifx\csname count@\endcsname\relax
  \countdef\count@=255 %
\fi
\expandafter\ifx\csname @gobble\endcsname\relax
  \long\def\@gobble#1{}%
\fi
\expandafter\ifx\csname @firstofone\endcsname\relax
  \long\def\@firstofone#1{#1}%
\fi
\expandafter\ifx\csname loop\endcsname\relax
  \expandafter\@firstofone
\else
  \expandafter\@gobble
\fi
{%
  \def\loop#1\repeat{%
    \def\body{#1}%
    \iterate
  }%
  \def\iterate{%
    \body
      \let\next\iterate
    \else
      \let\next\relax
    \fi
    \next
  }%
  \let\repeat=\fi
}%
\def\RestoreCatcodes{}
\count@=0 %
\loop
  \edef\RestoreCatcodes{%
    \RestoreCatcodes
    \catcode\the\count@=\the\catcode\count@\relax
  }%
\ifnum\count@<255 %
  \advance\count@ 1 %
\repeat

\def\RangeCatcodeInvalid#1#2{%
  \count@=#1\relax
  \loop
    \catcode\count@=15 %
  \ifnum\count@<#2\relax
    \advance\count@ 1 %
  \repeat
}
\def\RangeCatcodeCheck#1#2#3{%
  \count@=#1\relax
  \loop
    \ifnum#3=\catcode\count@
    \else
      \errmessage{%
        Character \the\count@\space
        with wrong catcode \the\catcode\count@\space
        instead of \number#3%
      }%
    \fi
  \ifnum\count@<#2\relax
    \advance\count@ 1 %
  \repeat
}
\def\space{ }
\expandafter\ifx\csname LoadCommand\endcsname\relax
  \def\LoadCommand{\input hologo.sty\relax}%
\fi
\def\Test{%
  \RangeCatcodeInvalid{0}{47}%
  \RangeCatcodeInvalid{58}{64}%
  \RangeCatcodeInvalid{91}{96}%
  \RangeCatcodeInvalid{123}{255}%
  \catcode`\@=12 %
  \catcode`\\=0 %
  \catcode`\%=14 %
  \LoadCommand
  \RangeCatcodeCheck{0}{36}{15}%
  \RangeCatcodeCheck{37}{37}{14}%
  \RangeCatcodeCheck{38}{47}{15}%
  \RangeCatcodeCheck{48}{57}{12}%
  \RangeCatcodeCheck{58}{63}{15}%
  \RangeCatcodeCheck{64}{64}{12}%
  \RangeCatcodeCheck{65}{90}{11}%
  \RangeCatcodeCheck{91}{91}{15}%
  \RangeCatcodeCheck{92}{92}{0}%
  \RangeCatcodeCheck{93}{96}{15}%
  \RangeCatcodeCheck{97}{122}{11}%
  \RangeCatcodeCheck{123}{255}{15}%
  \RestoreCatcodes
}
\Test
\csname @@end\endcsname
\end
%    \end{macrocode}
%    \begin{macrocode}
%</test1>
%    \end{macrocode}
%
% \subsection{Spacefactor}
%
%    The space factor must be 1000 after a logo. If it is greater 1000
%    then the following space is a space after a sentence closing point.
%    If the space factor is smaller 1000 then an immediate following
%    dot is interpreted as abbreviation, not sentence closing point.
%
%    \begin{macrocode}
%<*test-spacefactor>
\NeedsTeXFormat{LaTeX2e}
\documentclass{article}
\usepackage{hologo}[2016/05/12]
\usepackage{kvsetkeys}
\usepackage{qstest}
\IncludeTests{*}
\LogTests{log}{*}{*}
\begin{document}
\begin{qstest}{spacefactor}{spacefactor}
\newcommand*{\Test}[1]{%
  \sbox0{%
    \hologo{#1}%
    \Expect*{1000 (#1)}*{\the\spacefactor\space(#1)}%
  }%
}%
\makeatletter
\def\TestList{}
\def\hologoEntry#1#2#3{%
  \edef\TestList{%
    \ifx\TestList\@empty
    \else
      \TestList,%
    \fi
    #1%
    \ifx\\#2\\%
    \else
      ={variant=#2}%
    \fi
  }%
}
\hologoList
\expandafter\kv@parse@normalized\expandafter{%
  \TestList
}{%
  \begingroup
    \let\@logo=\kv@key
    \ifx\kv@value\relax
    \else
      \expandafter\hologoLogoSetup\expandafter\@logo\expandafter{%
        \kv@value
      }%
    \fi
    \Test\@logo
  \endgroup
  \@gobbletwo
}
\end{qstest}
\end{document}
%</test-spacefactor>
%    \end{macrocode}
%
% \subsection{Complete list}
%
%    \begin{macrocode}
%<*test-list>
\NeedsTeXFormat{LaTeX2e}
\documentclass[12pt,a4paper]{article}
\usepackage{hologo}[2016/05/12]
\usepackage[T1]{fontenc}
\usepackage{lmodern}
\usepackage{parskip}
\usepackage[unicode]{hyperref}[2011/09/28]
\usepackage{bookmark}[2011/09/19]
\bookmarksetup{%
  numbered,%
  open,%
  openlevel=2,%
}
\renewcommand*{\contentsname}{List of logos}
\begin{document}
\tableofcontents
\def\TestFont#1#2#3#4#5#6{%
  \begingroup
    \usefont{#3}{#4}{#5}{#6}%
    \HologoVariant{#1}{#2}/\hologoVariant{#1}{#2}%
    \quad
    \begingroup\scriptsize\hologoVariant{#1}{#2}\endgroup
    \quad
  \endgroup
  (#3/#4/#5/#6)%
  \par
}
\makeatletter
\def\hologoEntry#1#2#3{%
  \section{%
    \HologoVariant{#1}{#2}/\hologoVariant{#1}{#2} %
    {[#1\ifx\\#2\\\else\space(#2)\fi]}% hash-ok
  }% braces around [] because of bug in tex4ht
  \begingroup
    \hypersetup{unicode=false}%
    \bookmark[%
      dest=\@currentHref,%
      rellevel=1,%
      keeplevel,%
    ]{%
      \HologoVariant{#1}{#2}/\hologoVariant{#1}{#2} %
      (PDFDocEncoding)%
    }%
  \endgroup
  \TestFont{#1}{#2}{OT1}{cmr}{m}{n}%
  \TestFont{#1}{#2}{OT1}{cmss}{m}{n}%
  \TestFont{#1}{#2}{OT1}{cmr}{b}{n}%
  \TestFont{#1}{#2}{OT1}{cmr}{m}{it}%
  \TestFont{#1}{#2}{OT1}{cmtt}{m}{n}%
  \TestFont{#1}{#2}{T1}{lmr}{m}{n}%
  \TestFont{#1}{#2}{T1}{lmss}{m}{n}%
  \TestFont{#1}{#2}{T1}{lmr}{b}{n}%
  \TestFont{#1}{#2}{T1}{lmr}{m}{it}%
  \TestFont{#1}{#2}{T1}{lmtt}{m}{n}%
  \TestFont{#1}{#2}{T1}{lmvtt}{m}{n}%
  \TestFont{#1}{#2}{T1}{qtm}{m}{n}%
  \TestFont{#1}{#2}{T1}{qhv}{m}{n}%
  \TestFont{#1}{#2}{T1}{qtm}{b}{n}%
  \TestFont{#1}{#2}{T1}{qtm}{m}{it}%
  \TestFont{#1}{#2}{T1}{qcr}{m}{n}%
  \newpage
}
\makeatother
\hologoList
\end{document}
%</test-list>
%    \end{macrocode}
%
% \section{Installation}
%
% \subsection{Download}
%
% \paragraph{Package.} This package is available on
% CTAN\footnote{\url{ftp://ftp.ctan.org/tex-archive/}}:
% \begin{description}
% \item[\CTAN{macros/latex/contrib/oberdiek/hologo.dtx}] The source file.
% \item[\CTAN{macros/latex/contrib/oberdiek/hologo.pdf}] Documentation.
% \end{description}
%
%
% \paragraph{Bundle.} All the packages of the bundle `oberdiek'
% are also available in a TDS compliant ZIP archive. There
% the packages are already unpacked and the documentation files
% are generated. The files and directories obey the TDS standard.
% \begin{description}
% \item[\CTAN{install/macros/latex/contrib/oberdiek.tds.zip}]
% \end{description}
% \emph{TDS} refers to the standard ``A Directory Structure
% for \TeX\ Files'' (\CTAN{tds/tds.pdf}). Directories
% with \xfile{texmf} in their name are usually organized this way.
%
% \subsection{Bundle installation}
%
% \paragraph{Unpacking.} Unpack the \xfile{oberdiek.tds.zip} in the
% TDS tree (also known as \xfile{texmf} tree) of your choice.
% Example (linux):
% \begin{quote}
%   |unzip oberdiek.tds.zip -d ~/texmf|
% \end{quote}
%
% \paragraph{Script installation.}
% Check the directory \xfile{TDS:scripts/oberdiek/} for
% scripts that need further installation steps.
% Package \xpackage{attachfile2} comes with the Perl script
% \xfile{pdfatfi.pl} that should be installed in such a way
% that it can be called as \texttt{pdfatfi}.
% Example (linux):
% \begin{quote}
%   |chmod +x scripts/oberdiek/pdfatfi.pl|\\
%   |cp scripts/oberdiek/pdfatfi.pl /usr/local/bin/|
% \end{quote}
%
% \subsection{Package installation}
%
% \paragraph{Unpacking.} The \xfile{.dtx} file is a self-extracting
% \docstrip\ archive. The files are extracted by running the
% \xfile{.dtx} through \plainTeX:
% \begin{quote}
%   \verb|tex hologo.dtx|
% \end{quote}
%
% \paragraph{TDS.} Now the different files must be moved into
% the different directories in your installation TDS tree
% (also known as \xfile{texmf} tree):
% \begin{quote}
% \def\t{^^A
% \begin{tabular}{@{}>{\ttfamily}l@{ $\rightarrow$ }>{\ttfamily}l@{}}
%   hologo.sty & tex/generic/oberdiek/hologo.sty\\
%   hologo.pdf & doc/latex/oberdiek/hologo.pdf\\
%   example/hologo-example.tex & doc/latex/oberdiek/example/hologo-example.tex\\
%   test/hologo-test1.tex & doc/latex/oberdiek/test/hologo-test1.tex\\
%   test/hologo-test-spacefactor.tex & doc/latex/oberdiek/test/hologo-test-spacefactor.tex\\
%   test/hologo-test-list.tex & doc/latex/oberdiek/test/hologo-test-list.tex\\
%   hologo.dtx & source/latex/oberdiek/hologo.dtx\\
% \end{tabular}^^A
% }^^A
% \sbox0{\t}^^A
% \ifdim\wd0>\linewidth
%   \begingroup
%     \advance\linewidth by\leftmargin
%     \advance\linewidth by\rightmargin
%   \edef\x{\endgroup
%     \def\noexpand\lw{\the\linewidth}^^A
%   }\x
%   \def\lwbox{^^A
%     \leavevmode
%     \hbox to \linewidth{^^A
%       \kern-\leftmargin\relax
%       \hss
%       \usebox0
%       \hss
%       \kern-\rightmargin\relax
%     }^^A
%   }^^A
%   \ifdim\wd0>\lw
%     \sbox0{\small\t}^^A
%     \ifdim\wd0>\linewidth
%       \ifdim\wd0>\lw
%         \sbox0{\footnotesize\t}^^A
%         \ifdim\wd0>\linewidth
%           \ifdim\wd0>\lw
%             \sbox0{\scriptsize\t}^^A
%             \ifdim\wd0>\linewidth
%               \ifdim\wd0>\lw
%                 \sbox0{\tiny\t}^^A
%                 \ifdim\wd0>\linewidth
%                   \lwbox
%                 \else
%                   \usebox0
%                 \fi
%               \else
%                 \lwbox
%               \fi
%             \else
%               \usebox0
%             \fi
%           \else
%             \lwbox
%           \fi
%         \else
%           \usebox0
%         \fi
%       \else
%         \lwbox
%       \fi
%     \else
%       \usebox0
%     \fi
%   \else
%     \lwbox
%   \fi
% \else
%   \usebox0
% \fi
% \end{quote}
% If you have a \xfile{docstrip.cfg} that configures and enables \docstrip's
% TDS installing feature, then some files can already be in the right
% place, see the documentation of \docstrip.
%
% \subsection{Refresh file name databases}
%
% If your \TeX~distribution
% (\teTeX, \mikTeX, \dots) relies on file name databases, you must refresh
% these. For example, \teTeX\ users run \verb|texhash| or
% \verb|mktexlsr|.
%
% \subsection{Some details for the interested}
%
% \paragraph{Attached source.}
%
% The PDF documentation on CTAN also includes the
% \xfile{.dtx} source file. It can be extracted by
% AcrobatReader 6 or higher. Another option is \textsf{pdftk},
% e.g. unpack the file into the current directory:
% \begin{quote}
%   \verb|pdftk hologo.pdf unpack_files output .|
% \end{quote}
%
% \paragraph{Unpacking with \LaTeX.}
% The \xfile{.dtx} chooses its action depending on the format:
% \begin{description}
% \item[\plainTeX:] Run \docstrip\ and extract the files.
% \item[\LaTeX:] Generate the documentation.
% \end{description}
% If you insist on using \LaTeX\ for \docstrip\ (really,
% \docstrip\ does not need \LaTeX), then inform the autodetect routine
% about your intention:
% \begin{quote}
%   \verb|latex \let\install=y\input{hologo.dtx}|
% \end{quote}
% Do not forget to quote the argument according to the demands
% of your shell.
%
% \paragraph{Generating the documentation.}
% You can use both the \xfile{.dtx} or the \xfile{.drv} to generate
% the documentation. The process can be configured by the
% configuration file \xfile{ltxdoc.cfg}. For instance, put this
% line into this file, if you want to have A4 as paper format:
% \begin{quote}
%   \verb|\PassOptionsToClass{a4paper}{article}|
% \end{quote}
% An example follows how to generate the
% documentation with pdf\LaTeX:
% \begin{quote}
%\begin{verbatim}
%pdflatex hologo.dtx
%makeindex -s gind.ist hologo.idx
%pdflatex hologo.dtx
%makeindex -s gind.ist hologo.idx
%pdflatex hologo.dtx
%\end{verbatim}
% \end{quote}
%
% \section{Catalogue}
%
% The following XML file can be used as source for the
% \href{http://mirror.ctan.org/help/Catalogue/catalogue.html}{\TeX\ Catalogue}.
% The elements \texttt{caption} and \texttt{description} are imported
% from the original XML file from the Catalogue.
% The name of the XML file in the Catalogue is \xfile{hologo.xml}.
%    \begin{macrocode}
%<*catalogue>
<?xml version='1.0' encoding='us-ascii'?>
<!DOCTYPE entry SYSTEM 'catalogue.dtd'>
<entry datestamp='$Date$' modifier='$Author$' id='hologo'>
  <name>hologo</name>
  <caption>A collection of logos with bookmark support.</caption>
  <authorref id='auth:oberdiek'/>
  <copyright owner='Heiko Oberdiek' year='2010-2012'/>
  <license type='lppl1.3'/>
  <version number='1.10'/>
  <description>
    The package defines a single command <tt>\hologo</tt>, whose
    argument is the usual case-confused ASCII version of the logo.
    The command is bookmark-enabled, so that every logo becomes
    available in bookmarks without further work.
    <p/>
    The package is part of the <xref refid='oberdiek'>oberdiek</xref>
    bundle.
  </description>
  <documentation details='Package documentation'
      href='ctan:/macros/latex/contrib/oberdiek/hologo.pdf'/>
  <ctan file='true' path='/macros/latex/contrib/oberdiek/hologo.dtx'/>
  <miktex location='oberdiek'/>
  <texlive location='oberdiek'/>
  <install path='/macros/latex/contrib/oberdiek/oberdiek.tds.zip'/>
</entry>
%</catalogue>
%    \end{macrocode}
%
% \begin{thebibliography}{9}
% \raggedright
%
% \bibitem{btxdoc}
% Oren Patashnik,
% \textit{\hologo{BibTeX}ing},
% 1988-02-08.\\
% \CTAN{biblio/bibtex/base/}
%
% \bibitem{dtklogos}
% Gerd Neugebauer, DANTE,
% \textit{Package \xpackage{dtklogos}},
% 2011-04-25.\\
% \CTAN{usergrps/dante/dtk/dtklogos.sty}
%
% \bibitem{etexman}
% The \hologo{NTS} Team,
% \textit{The \hologo{eTeX} manual},
% 1998-02.\\
% \CTAN{systems/e-tex/v2/doc/}
%
% \bibitem{ExTeX-FAQ}
% The \hologo{ExTeX} group,
% \textit{\hologo{ExTeX}: FAQ -- How is \hologo{ExTeX} typeset?},
% 2007-04-14.\\
% \url{http://www.extex.org/documentation/faq.html}
%
% \bibitem{LyX}
% %@MISC{ LyX,
% %  title = {{LyX 2.0.0 -- The Document Processor [Computer software and manual]}},
% %  author = {{The LyX Team}},
% %  howpublished = {Internet: http://www.lyx.org},
% %  year = {2011-05-08},
% %  note = {Retrieved May 10, 2011, from http://www.lyx.org},
% %  url = {http://www.lyx.org/}
% %}
% The \hologo{LyX} Team,
% \textit{\hologo{LyX} -- The Document Processor},
% 2011-05-08.\\
% \url{http://www.lyx.org/}
%
% \bibitem{OzTeX}
% Andrew Trevorrow,
% \hologo{OzTeX} FAQ: What is the correct way to typeset ``\hologo{OzTeX}''?,
% 2011-09-15 (visited).
% \url{http://www.trevorrow.com/oztex/ozfaq.html#oztex-logo}
%
% \bibitem{PiCTeX}
% Michael Wichura,
% \textit{The \hologo{PiCTeX} macro package},
% 1987-09-21.
% \CTAN{graphics/pictex/}
%
% \bibitem{scrlogo}
% Markus Kohm,
% \textit{\hologo{KOMAScript} Datei \xfile{scrlogo.dtx}},
% 2009-01-30.\\
% \CTAN{install/macros/latex/contrib/komascript.tds.zip}
%
% \end{thebibliography}
%
% \begin{History}
%   \begin{Version}{2010/04/08 v1.0}
%   \item
%     The first version.
%   \end{Version}
%   \begin{Version}{2010/04/16 v1.1}
%   \item
%     \cs{Hologo} added for support of logos at start of a sentence.
%   \item
%     \cs{hologoSetup} and \cs{hologoLogoSetup} added.
%   \item
%     Options \xoption{break}, \xoption{hyphenbreak}, \xoption{spacebreak}
%     added.
%   \item
%     Variant support added by option \xoption{variant}.
%   \end{Version}
%   \begin{Version}{2010/04/24 v1.2}
%   \item
%     \hologo{LaTeX3} added.
%   \item
%     \hologo{VTeX} added.
%   \end{Version}
%   \begin{Version}{2010/11/21 v1.3}
%   \item
%     \hologo{iniTeX}, \hologo{virTeX} added.
%   \end{Version}
%   \begin{Version}{2011/03/25 v1.4}
%   \item
%     \hologo{ConTeXt} with variants added.
%   \item
%     Option \xoption{discretionarybreak} added as refinement for
%     option \xoption{break}.
%   \end{Version}
%   \begin{Version}{2011/04/21 v1.5}
%   \item
%     Wrong TDS directory for test files fixed.
%   \end{Version}
%   \begin{Version}{2011/10/01 v1.6}
%   \item
%     Support for package \xpackage{tex4ht} added.
%   \item
%     Support for \cs{csname} added if \cs{ifincsname} is available.
%   \item
%     New logos:
%     \hologo{(La)TeX},
%     \hologo{biber},
%     \hologo{BibTeX} (\xoption{sc}, \xoption{sf}),
%     \hologo{emTeX},
%     \hologo{ExTeX},
%     \hologo{KOMAScript},
%     \hologo{La},
%     \hologo{LyX},
%     \hologo{MiKTeX},
%     \hologo{NTS},
%     \hologo{OzMF},
%     \hologo{OzMP},
%     \hologo{OzTeX},
%     \hologo{OzTtH},
%     \hologo{PCTeX},
%     \hologo{PiC},
%     \hologo{PiCTeX},
%     \hologo{METAFONT},
%     \hologo{MetaFun},
%     \hologo{METAPOST},
%     \hologo{MetaPost},
%     \hologo{SLiTeX} (\xoption{lift}, \xoption{narrow}, \xoption{simple}),
%     \hologo{SliTeX} (\xoption{narrow}, \xoption{simple}, \xoption{lift}),
%     \hologo{teTeX}.
%   \item
%     Fixes:
%     \hologo{iniTeX},
%     \hologo{pdfLaTeX},
%     \hologo{pdfTeX},
%     \hologo{virTeX}.
%   \item
%     \cs{hologoFontSetup} and \cs{hologoLogoFontSetup} added.
%   \item
%     \cs{hologoVariant} and \cs{HologoVariant} added.
%   \end{Version}
%   \begin{Version}{2011/11/22 v1.7}
%   \item
%     New logos:
%     \hologo{BibTeX8},
%     \hologo{LaTeXML},
%     \hologo{SageTeX},
%     \hologo{TeX4ht},
%     \hologo{TTH}.
%   \item
%     \hologo{Xe} and friends: Driver stuff fixed.
%   \item
%     \hologo{Xe} and friends: Support for italic added.
%   \item
%     \hologo{Xe} and friends: Package support for \xpackage{pgf}
%     and \xpackage{pstricks} added.
%   \end{Version}
%   \begin{Version}{2011/11/29 v1.8}
%   \item
%     New logos:
%     \hologo{HanTheThanh}.
%   \end{Version}
%   \begin{Version}{2011/12/21 v1.9}
%   \item
%     Patch for package \xpackage{ifxetex} added for the case that
%     \cs{newif} is undefined in \hologo{iniTeX}.
%   \item
%     Some fixes for \hologo{iniTeX}.
%   \end{Version}
%   \begin{Version}{2012/04/26 v1.10}
%   \item
%     Fix in bookmark version of logo ``\hologo{HanTheThanh}''.
%   \end{Version}
%   \begin{Version}{2016/05/12 v1.11}
%   \item
%     Update HOLOGO@IfCharExists (previously in texlive)
%   \item define pdfliteral in current luatex.
%   \end{Version}
% \end{History}
%
% \PrintIndex
%
% \Finale
\endinput
%
        \else
          \input hologo.cfg\relax
        \fi
      \fi
    }%
    \ltx@IfUndefined{newread}{%
      \chardef\HOLOGO@temp=15 %
      \def\HOLOGO@CheckRead{%
        \ifeof\HOLOGO@temp
          \HOLOGO@InputIfExists
        \else
          \ifcase\HOLOGO@temp
            \@PackageWarningNoLine{hologo}{%
              Configuration file ignored, because\MessageBreak
              a free read register could not be found%
            }%
          \else
            \begingroup
              \count\ltx@cclv=\HOLOGO@temp
              \advance\ltx@cclv by \ltx@minusone
              \edef\x{\endgroup
                \chardef\noexpand\HOLOGO@temp=\the\count\ltx@cclv
                \relax
              }%
            \x
          \fi
        \fi
      }%
    }{%
      \csname newread\endcsname\HOLOGO@temp
      \HOLOGO@InputIfExists
    }%
  }{%
    \edef\HOLOGO@temp{\pdf@filesize{hologo.cfg}}%
    \ifx\HOLOGO@temp\ltx@empty
    \else
      \ifnum\HOLOGO@temp>0 %
        \begingroup
          \def\x{LaTeX2e}%
        \expandafter\endgroup
        \ifx\fmtname\x
          % \iffalse meta-comment
%
% File: hologo.dtx
% Version: 2016/05/12 v1.11
% Info: A logo collection with bookmark support
%
% Copyright (C) 2010-2012 by
%    Heiko Oberdiek <heiko.oberdiek at googlemail.com>
%
% This work may be distributed and/or modified under the
% conditions of the LaTeX Project Public License, either
% version 1.3c of this license or (at your option) any later
% version. This version of this license is in
%    http://www.latex-project.org/lppl/lppl-1-3c.txt
% and the latest version of this license is in
%    http://www.latex-project.org/lppl.txt
% and version 1.3 or later is part of all distributions of
% LaTeX version 2005/12/01 or later.
%
% This work has the LPPL maintenance status "maintained".
%
% This Current Maintainer of this work is Heiko Oberdiek.
%
% The Base Interpreter refers to any `TeX-Format',
% because some files are installed in TDS:tex/generic//.
%
% This work consists of the main source file hologo.dtx
% and the derived files
%    hologo.sty, hologo.pdf, hologo.ins, hologo.drv, hologo-example.tex,
%    hologo-test1.tex, hologo-test-spacefactor.tex,
%    hologo-test-list.tex.
%
% Distribution:
%    CTAN:macros/latex/contrib/oberdiek/hologo.dtx
%    CTAN:macros/latex/contrib/oberdiek/hologo.pdf
%
% Unpacking:
%    (a) If hologo.ins is present:
%           tex hologo.ins
%    (b) Without hologo.ins:
%           tex hologo.dtx
%    (c) If you insist on using LaTeX
%           latex \let\install=y\input{hologo.dtx}
%        (quote the arguments according to the demands of your shell)
%
% Documentation:
%    (a) If hologo.drv is present:
%           latex hologo.drv
%    (b) Without hologo.drv:
%           latex hologo.dtx; ...
%    The class ltxdoc loads the configuration file ltxdoc.cfg
%    if available. Here you can specify further options, e.g.
%    use A4 as paper format:
%       \PassOptionsToClass{a4paper}{article}
%
%    Programm calls to get the documentation (example):
%       pdflatex hologo.dtx
%       makeindex -s gind.ist hologo.idx
%       pdflatex hologo.dtx
%       makeindex -s gind.ist hologo.idx
%       pdflatex hologo.dtx
%
% Installation:
%    TDS:tex/generic/oberdiek/hologo.sty
%    TDS:doc/latex/oberdiek/hologo.pdf
%    TDS:doc/latex/oberdiek/example/hologo-example.tex
%    TDS:doc/latex/oberdiek/test/hologo-test1.tex
%    TDS:doc/latex/oberdiek/test/hologo-test-spacefactor.tex
%    TDS:doc/latex/oberdiek/test/hologo-test-list.tex
%    TDS:source/latex/oberdiek/hologo.dtx
%
%<*ignore>
\begingroup
  \catcode123=1 %
  \catcode125=2 %
  \def\x{LaTeX2e}%
\expandafter\endgroup
\ifcase 0\ifx\install y1\fi\expandafter
         \ifx\csname processbatchFile\endcsname\relax\else1\fi
         \ifx\fmtname\x\else 1\fi\relax
\else\csname fi\endcsname
%</ignore>
%<*install>
\input docstrip.tex
\Msg{************************************************************************}
\Msg{* Installation}
\Msg{* Package: hologo 2016/05/12 v1.11 A logo collection with bookmark support (HO)}
\Msg{************************************************************************}

\keepsilent
\askforoverwritefalse

\let\MetaPrefix\relax
\preamble

This is a generated file.

Project: hologo
Version: 2016/05/12 v1.11

Copyright (C) 2010-2012 by
   Heiko Oberdiek <heiko.oberdiek at googlemail.com>

This work may be distributed and/or modified under the
conditions of the LaTeX Project Public License, either
version 1.3c of this license or (at your option) any later
version. This version of this license is in
   http://www.latex-project.org/lppl/lppl-1-3c.txt
and the latest version of this license is in
   http://www.latex-project.org/lppl.txt
and version 1.3 or later is part of all distributions of
LaTeX version 2005/12/01 or later.

This work has the LPPL maintenance status "maintained".

This Current Maintainer of this work is Heiko Oberdiek.

The Base Interpreter refers to any `TeX-Format',
because some files are installed in TDS:tex/generic//.

This work consists of the main source file hologo.dtx
and the derived files
   hologo.sty, hologo.pdf, hologo.ins, hologo.drv, hologo-example.tex,
   hologo-test1.tex, hologo-test-spacefactor.tex,
   hologo-test-list.tex.

\endpreamble
\let\MetaPrefix\DoubleperCent

\generate{%
  \file{hologo.ins}{\from{hologo.dtx}{install}}%
  \file{hologo.drv}{\from{hologo.dtx}{driver}}%
  \usedir{tex/generic/oberdiek}%
  \file{hologo.sty}{\from{hologo.dtx}{package}}%
  \usedir{doc/latex/oberdiek/example}%
  \file{hologo-example.tex}{\from{hologo.dtx}{example}}%
  \usedir{doc/latex/oberdiek/test}%
  \file{hologo-test1.tex}{\from{hologo.dtx}{test1}}%
  \file{hologo-test-spacefactor.tex}{\from{hologo.dtx}{test-spacefactor}}%
  \file{hologo-test-list.tex}{\from{hologo.dtx}{test-list}}%
  \nopreamble
  \nopostamble
  \usedir{source/latex/oberdiek/catalogue}%
  \file{hologo.xml}{\from{hologo.dtx}{catalogue}}%
}

\catcode32=13\relax% active space
\let =\space%
\Msg{************************************************************************}
\Msg{*}
\Msg{* To finish the installation you have to move the following}
\Msg{* file into a directory searched by TeX:}
\Msg{*}
\Msg{*     hologo.sty}
\Msg{*}
\Msg{* To produce the documentation run the file `hologo.drv'}
\Msg{* through LaTeX.}
\Msg{*}
\Msg{* Happy TeXing!}
\Msg{*}
\Msg{************************************************************************}

\endbatchfile
%</install>
%<*ignore>
\fi
%</ignore>
%<*driver>
\NeedsTeXFormat{LaTeX2e}
\ProvidesFile{hologo.drv}%
  [2016/05/12 v1.11 A logo collection with bookmark support (HO)]%
\documentclass{ltxdoc}
\usepackage{holtxdoc}[2011/11/22]
\usepackage{hologo}[2016/05/12]
\usepackage{longtable}
\usepackage{array}
\usepackage{paralist}
%\usepackage[T1]{fontenc}
%\usepackage{lmodern}
\begin{document}
  \DocInput{hologo.dtx}%
\end{document}
%</driver>
% \fi
%
%
% \CharacterTable
%  {Upper-case    \A\B\C\D\E\F\G\H\I\J\K\L\M\N\O\P\Q\R\S\T\U\V\W\X\Y\Z
%   Lower-case    \a\b\c\d\e\f\g\h\i\j\k\l\m\n\o\p\q\r\s\t\u\v\w\x\y\z
%   Digits        \0\1\2\3\4\5\6\7\8\9
%   Exclamation   \!     Double quote  \"     Hash (number) \#
%   Dollar        \$     Percent       \%     Ampersand     \&
%   Acute accent  \'     Left paren    \(     Right paren   \)
%   Asterisk      \*     Plus          \+     Comma         \,
%   Minus         \-     Point         \.     Solidus       \/
%   Colon         \:     Semicolon     \;     Less than     \<
%   Equals        \=     Greater than  \>     Question mark \?
%   Commercial at \@     Left bracket  \[     Backslash     \\
%   Right bracket \]     Circumflex    \^     Underscore    \_
%   Grave accent  \`     Left brace    \{     Vertical bar  \|
%   Right brace   \}     Tilde         \~}
%
% \GetFileInfo{hologo.drv}
%
% \title{The \xpackage{hologo} package}
% \date{2016/05/12 v1.11}
% \author{Heiko Oberdiek\\\xemail{heiko.oberdiek at googlemail.com}}
%
% \maketitle
%
% \begin{abstract}
% This package starts a collection of logos with support for bookmarks
% strings.
% \end{abstract}
%
% \tableofcontents
%
% \section{Documentation}
%
% \subsection{Logo macros}
%
% \begin{declcs}{hologo} \M{name}
% \end{declcs}
% Macro \cs{hologo} sets the logo with name \meta{name}.
% The following table shows the supported names.
%
% \begingroup
%   \def\hologoEntry#1#2#3{^^A
%     #1&#2&\hologoLogoSetup{#1}{variant=#2}\hologo{#1}&#3\tabularnewline
%   }
%   \begin{longtable}{>{\ttfamily}l>{\ttfamily}lll}
%     \rmfamily\bfseries{name} & \rmfamily\bfseries variant
%     & \bfseries logo & \bfseries since\\
%     \hline
%     \endhead
%     \hologoList
%   \end{longtable}
% \endgroup
%
% \begin{declcs}{Hologo} \M{name}
% \end{declcs}
% Macro \cs{Hologo} starts the logo \meta{name} with an uppercase
% letter. As an exception small greek letters are not converted
% to uppercase. Examples, see \hologo{eTeX} and \hologo{ExTeX}.
%
% \subsection{Setup macros}
%
% The package does not support package options, but the following
% setup macros can be used to set options.
%
% \begin{declcs}{hologoSetup} \M{key value list}
% \end{declcs}
% Macro \cs{hologoSetup} sets global options.
%
% \begin{declcs}{hologoLogoSetup} \M{logo} \M{key value list}
% \end{declcs}
% Some options can also be used to configure a logo.
% These settings take precedence over global option settings.
%
% \subsection{Options}\label{sec:options}
%
% There are boolean and string options:
% \begin{description}
% \item[Boolean option:]
% It takes |true| or |false|
% as value. If the value is omitted, then |true| is used.
% \item[String option:]
% A value must be given as string. (But the string might be empty.)
% \end{description}
% The following options can be used both in \cs{hologoSetup}
% and \cs{hologoLogoSetup}:
% \begin{description}
% \def\entry#1{\item[\xoption{#1}:]}
% \entry{break}
%   enables or disables line breaks inside the logo. This setting is
%   refined by options \xoption{hyphenbreak}, \xoption{spacebreak}
%   or \xoption{discretionarybreak}.
%   Default is |false|.
% \entry{hyphenbreak}
%   enables or disables the line break right after the hyphen character.
% \entry{spacebreak}
%   enables or disables line breaks at space characters.
% \entry{discretionarybreak}
%   enables or disables line breaks at hyphenation points
%   (inserted by \cs{-}).
% \end{description}
% Macro \cs{hologoLogoSetup} also knows:
% \begin{description}
% \item[\xoption{variant}:]
%   This is a string option. It specifies a variant of a logo that
%   must exist. An empty string selects the package default variant.
% \end{description}
% Example:
% \begin{quote}
%   |\hologoSetup{break=false}|\\
%   |\hologoLogoSetup{plainTeX}{variant=hyphen,hyphenbreak}|\\
%   Then ``plain-\TeX'' contains one break point after the hyphen.
% \end{quote}
%
% \subsection{Driver options}
%
% Sometimes graphical operations are needed to construct some
% glyphs (e.g.\ \hologo{XeTeX}). If package \xpackage{graphics}
% or package \xpackage{pgf} are found, then the macros are taken
% from there. Otherwise the packge defines its own operations
% and therefore needs the driver information. Many drivers are
% detected automatically (\hologo{pdfTeX}/\hologo{LuaTeX}
% in PDF mode, \hologo{XeTeX}, \hologo{VTeX}). These have precedence
% over a driver option. The driver can be given as package option
% or using \cs{hologoDriverSetup}.
% The following list contains the recognized driver options:
% \begin{itemize}
% \item \xoption{pdftex}, \xoption{luatex}
% \item \xoption{dvipdfm}, \xoption{dvipdfmx}
% \item \xoption{dvips}, \xoption{dvipsone}, \xoption{xdvi}
% \item \xoption{xetex}
% \item \xoption{vtex}
% \end{itemize}
% The left driver of a line is the driver name that is used internally.
% The following names are aliases for drivers that use the
% same method. Therefore the entry in the \xext{log} file for
% the used driver prints the internally used driver name.
% \begin{description}
% \item[\xoption{driverfallback}:]
%   This option expects a driver that is used,
%   if the driver could not be detected automatically.
% \end{description}
%
% \begin{declcs}{hologoDriverSetup} \M{driver option}
% \end{declcs}
% The driver can also be configured after package loading
% using \cs{hologoDriverSetup}, also the way for \hologo{plainTeX}
% to setup the driver.
%
% \subsection{Font setup}
%
% Some logos require a special font, but should also be usable by
% \hologo{plainTeX}. Therefore the package provides some ways
% to influence the font settings. The options below
% take font settings as values. Both font commands
% such as \cs{sffamily} and macros that take one argument
% like \cs{textsf} can be used.
%
% \begin{declcs}{hologoFontSetup} \M{key value list}
% \end{declcs}
% Macro \cs{hologoFontSetup} sets the fonts for all logos.
% Supported keys:
% \begin{description}
% \def\entry#1{\item[\xoption{#1}:]}
% \entry{general}
%   This font is used for all logos. The default is empty.
%   That means no special font is used.
% \entry{bibsf}
%   This font is used for
%   {\hologoLogoSetup{BibTeX}{variant=sf}\hologo{BibTeX}}
%   with variant \xoption{sf}.
% \entry{rm}
%   This font is a serif font. It is used for \hologo{ExTeX}.
% \entry{sc}
%   This font specifies a small caps font. It is used for
%   {\hologoLogoSetup{BibTeX}{variant=sc}\hologo{BibTeX}}
%   with variant \xoption{sc}.
% \entry{sf}
%   This font specifies a sans serif font. The default
%   is \cs{sffamily}, then \cs{sf} is tried. Otherwise
%   a warning is given. It is used by \hologo{KOMAScript}.
% \entry{sy}
%   This is the font for math symbols (e.g. cmsy).
%   It is used by \hologo{AmS}, \hologo{NTS}, \hologo{ExTeX}.
% \entry{logo}
%   \hologo{METAFONT} and \hologo{METAPOST} are using that font.
%   In \hologo{LaTeX} \cs{logofamily} is used and
%   the definitions of package \xpackage{mflogo} are used
%   if the package is not loaded.
%   Otherwise the \cs{tenlogo} is used and defined
%   if it does not already exists.
% \end{description}
%
% \begin{declcs}{hologoLogoFontSetup} \M{logo} \M{key value list}
% \end{declcs}
% Fonts can also be set for a logo or logo component separately,
% see the following list.
% The keys are the same as for \cs{hologoFontSetup}.
%
% \begin{longtable}{>{\ttfamily}l>{\sffamily}ll}
%   \meta{logo} & keys & result\\
%   \hline
%   \endhead
%   BibTeX & bibsf & {\hologoLogoSetup{BibTeX}{variant=sf}\hologo{BibTeX}}\\[.5ex]
%   BibTeX & sc & {\hologoLogoSetup{BibTeX}{variant=sc}\hologo{BibTeX}}\\[.5ex]
%   ExTeX & rm & \hologo{ExTeX}\\
%   SliTeX & rm & \hologo{SliTeX}\\[.5ex]
%   AmS & sy & \hologo{AmS}\\
%   ExTeX & sy & \hologo{ExTeX}\\
%   NTS & sy & \hologo{NTS}\\[.5ex]
%   KOMAScript & sf & \hologo{KOMAScript}\\[.5ex]
%   METAFONT & logo & \hologo{METAFONT}\\
%   METAPOST & logo & \hologo{METAPOST}\\[.5ex]
%   SliTeX & sc \hologo{SliTeX}
% \end{longtable}
%
% \subsubsection{Font order}
%
% For all logos the font \xoption{general} is applied first.
% Example:
%\begin{quote}
%|\hologoFontSetup{general=\color{red}}|
%\end{quote}
% will print red logos.
% Then if the font uses a special font \xoption{sf}, for example,
% the font is applied that is setup by \cs{hologoLogoFontSetup}.
% If this font is not setup, then the common font setup
% by \cs{hologoFontSetup} is used. Otherwise a warning is given,
% that there is no font configured.
%
% \subsection{Additional user macros}
%
% Usually a variant of a logo is configured by using
% \cs{hologoLogoSetup}, because it is bad style to mix
% different variants of the same logo in the same text.
% There the following macros are a convenience for testing.
%
% \begin{declcs}{hologoVariant} \M{name} \M{variant}\\
%   \cs{HologoVariant} \M{name} \M{variant}
% \end{declcs}
% Logo \meta{name} is set using \meta{variant} that specifies
% explicitely which variant of the macro is used. If the argument
% is empty, then the default form of the logo is used
% (configurable by \cs{hologoLogoSetup}).
%
% \cs{HologoVariant} is used if the logo is set in a context
% that needs an uppercase first letter (beginning of a sentence, \dots).
%
% \begin{declcs}{hologoList}\\
%   \cs{hologoEntry} \M{logo} \M{variant} \M{since}
% \end{declcs}
% Macro \cs{hologoList} contains all logos that are provided
% by the package including variants. The list consists of calls
% of \cs{hologoEntry} with three arguments starting with the
% logo name \meta{logo} and its variant \meta{variant}. An empty
% variant means the current default. Argument \meta{since} specifies
% with version of the package \xpackage{hologo} is needed to get
% the logo. If the logo is fixed, then the date gets updated.
% Therefore the date \meta{since} is not exactly the date of
% the first introduction, but rather the date of the latest fix.
%
% Before \cs{hologoList} can be used, macro \cs{hologoEntry} needs
% a definition. The example file in section \ref{sec:example}
% shows applications of \cs{hologoList}.
%
% \subsection{Supported contexts}
%
% Macros \cs{hologo} and friends support special contexts:
% \begin{itemize}
% \item \hologo{LaTeX}'s protection mechanism.
% \item Bookmarks of package \xpackage{hyperref}.
% \item Package \xpackage{tex4ht}.
% \item The macros can be used inside \cs{csname} constructs,
%   if \cs{ifincsname} is available (\hologo{pdfTeX}, \hologo{XeTeX},
%   \hologo{LuaTeX}).
% \end{itemize}
%
% \subsection{Example}
% \label{sec:example}
%
% The following example prints the logos in different fonts.
%    \begin{macrocode}
%<*example>
%<<verbatim
\NeedsTeXFormat{LaTeX2e}
\documentclass[a4paper]{article}
\usepackage[
  hmargin=20mm,
  vmargin=20mm,
]{geometry}
\pagestyle{empty}
\usepackage{hologo}[2016/05/12]
\usepackage{longtable}
\usepackage{array}
\setlength{\extrarowheight}{2pt}
\usepackage[T1]{fontenc}
\usepackage{lmodern}
\usepackage{pdflscape}
\usepackage[
  pdfencoding=auto,
]{hyperref}
\hypersetup{
  pdfauthor={Heiko Oberdiek},
  pdftitle={Example for package `hologo'},
  pdfsubject={Logos with fonts lmr, lmss, qtm, qpl, qhv},
}
\usepackage{bookmark}

% Print the logo list on the console

\begingroup
  \typeout{}%
  \typeout{*** Begin of logo list ***}%
  \newcommand*{\hologoEntry}[3]{%
    \typeout{#1 \ifx\\#2\\\else(#2) \fi[#3]}%
  }%
  \hologoList
  \typeout{*** End of logo list ***}%
  \typeout{}%
\endgroup

\begin{document}
\begin{landscape}

  \section{Example file for package `hologo'}

  % Table for font names

  \begin{longtable}{>{\bfseries}ll}
    \textbf{font} & \textbf{Font name}\\
    \hline
    lmr & Latin Modern Roman\\
    lmss & Latin Modern Sans\\
    qtm & \TeX\ Gyre Termes\\
    qhv & \TeX\ Gyre Heros\\
    qpl & \TeX\ Gyre Pagella\\
  \end{longtable}

  % Logo list with logos in different fonts

  \begingroup
    \newcommand*{\SetVariant}[2]{%
      \ifx\\#2\\%
      \else
        \hologoLogoSetup{#1}{variant=#2}%
      \fi
    }%
    \newcommand*{\hologoEntry}[3]{%
      \SetVariant{#1}{#2}%
      \raisebox{1em}[0pt][0pt]{\hypertarget{#1@#2}{}}%
      \bookmark[%
        dest={#1@#2},%
      ]{%
        #1\ifx\\#2\\\else\space(#2)\fi: \Hologo{#1}, \hologo{#1} %
        [Unicode]%
      }%
      \hypersetup{unicode=false}%
      \bookmark[%
        dest={#1@#2},%
      ]{%
        #1\ifx\\#2\\\else\space(#2)\fi: \Hologo{#1}, \hologo{#1} %
        [PDFDocEncoding]%
      }%
      \texttt{#1}%
      &%
      \texttt{#2}%
      &%
      \Hologo{#1}%
      &%
      \SetVariant{#1}{#2}%
      \hologo{#1}%
      &%
      \SetVariant{#1}{#2}%
      \fontfamily{qtm}\selectfont
      \hologo{#1}%
      &%
      \SetVariant{#1}{#2}%
      \fontfamily{qpl}\selectfont
      \hologo{#1}%
      &%
      \SetVariant{#1}{#2}%
      \textsf{\hologo{#1}}%
      &%
      \SetVariant{#1}{#2}%
      \fontfamily{qhv}\selectfont
      \hologo{#1}%
      \tabularnewline
    }%
    \begin{longtable}{llllllll}%
      \textbf{\textit{logo}} & \textbf{\textit{variant}} &
      \texttt{\string\Hologo} &
      \textbf{lmr} & \textbf{qtm} & \textbf{qpl} &
      \textbf{lmss} & \textbf{qhv}
      \tabularnewline
      \hline
      \endhead
      \hologoList
    \end{longtable}%
  \endgroup

\end{landscape}
\end{document}
%verbatim
%</example>
%    \end{macrocode}
%
% \StopEventually{
% }
%
% \section{Implementation}
%    \begin{macrocode}
%<*package>
%    \end{macrocode}
%    Reload check, especially if the package is not used with \LaTeX.
%    \begin{macrocode}
\begingroup\catcode61\catcode48\catcode32=10\relax%
  \catcode13=5 % ^^M
  \endlinechar=13 %
  \catcode35=6 % #
  \catcode39=12 % '
  \catcode44=12 % ,
  \catcode45=12 % -
  \catcode46=12 % .
  \catcode58=12 % :
  \catcode64=11 % @
  \catcode123=1 % {
  \catcode125=2 % }
  \expandafter\let\expandafter\x\csname ver@hologo.sty\endcsname
  \ifx\x\relax % plain-TeX, first loading
  \else
    \def\empty{}%
    \ifx\x\empty % LaTeX, first loading,
      % variable is initialized, but \ProvidesPackage not yet seen
    \else
      \expandafter\ifx\csname PackageInfo\endcsname\relax
        \def\x#1#2{%
          \immediate\write-1{Package #1 Info: #2.}%
        }%
      \else
        \def\x#1#2{\PackageInfo{#1}{#2, stopped}}%
      \fi
      \x{hologo}{The package is already loaded}%
      \aftergroup\endinput
    \fi
  \fi
\endgroup%
%    \end{macrocode}
%    Package identification:
%    \begin{macrocode}
\begingroup\catcode61\catcode48\catcode32=10\relax%
  \catcode13=5 % ^^M
  \endlinechar=13 %
  \catcode35=6 % #
  \catcode39=12 % '
  \catcode40=12 % (
  \catcode41=12 % )
  \catcode44=12 % ,
  \catcode45=12 % -
  \catcode46=12 % .
  \catcode47=12 % /
  \catcode58=12 % :
  \catcode64=11 % @
  \catcode91=12 % [
  \catcode93=12 % ]
  \catcode123=1 % {
  \catcode125=2 % }
  \expandafter\ifx\csname ProvidesPackage\endcsname\relax
    \def\x#1#2#3[#4]{\endgroup
      \immediate\write-1{Package: #3 #4}%
      \xdef#1{#4}%
    }%
  \else
    \def\x#1#2[#3]{\endgroup
      #2[{#3}]%
      \ifx#1\@undefined
        \xdef#1{#3}%
      \fi
      \ifx#1\relax
        \xdef#1{#3}%
      \fi
    }%
  \fi
\expandafter\x\csname ver@hologo.sty\endcsname
\ProvidesPackage{hologo}%
  [2016/05/12 v1.11 A logo collection with bookmark support (HO)]%
%    \end{macrocode}
%
%    \begin{macrocode}
\begingroup\catcode61\catcode48\catcode32=10\relax%
  \catcode13=5 % ^^M
  \endlinechar=13 %
  \catcode123=1 % {
  \catcode125=2 % }
  \catcode64=11 % @
  \def\x{\endgroup
    \expandafter\edef\csname HOLOGO@AtEnd\endcsname{%
      \endlinechar=\the\endlinechar\relax
      \catcode13=\the\catcode13\relax
      \catcode32=\the\catcode32\relax
      \catcode35=\the\catcode35\relax
      \catcode61=\the\catcode61\relax
      \catcode64=\the\catcode64\relax
      \catcode123=\the\catcode123\relax
      \catcode125=\the\catcode125\relax
    }%
  }%
\x\catcode61\catcode48\catcode32=10\relax%
\catcode13=5 % ^^M
\endlinechar=13 %
\catcode35=6 % #
\catcode64=11 % @
\catcode123=1 % {
\catcode125=2 % }
\def\TMP@EnsureCode#1#2{%
  \edef\HOLOGO@AtEnd{%
    \HOLOGO@AtEnd
    \catcode#1=\the\catcode#1\relax
  }%
  \catcode#1=#2\relax
}
\TMP@EnsureCode{10}{12}% ^^J
\TMP@EnsureCode{33}{12}% !
\TMP@EnsureCode{34}{12}% "
\TMP@EnsureCode{36}{3}% $
\TMP@EnsureCode{38}{4}% &
\TMP@EnsureCode{39}{12}% '
\TMP@EnsureCode{40}{12}% (
\TMP@EnsureCode{41}{12}% )
\TMP@EnsureCode{42}{12}% *
\TMP@EnsureCode{43}{12}% +
\TMP@EnsureCode{44}{12}% ,
\TMP@EnsureCode{45}{12}% -
\TMP@EnsureCode{46}{12}% .
\TMP@EnsureCode{47}{12}% /
\TMP@EnsureCode{58}{12}% :
\TMP@EnsureCode{59}{12}% ;
\TMP@EnsureCode{60}{12}% <
\TMP@EnsureCode{62}{12}% >
\TMP@EnsureCode{63}{12}% ?
\TMP@EnsureCode{91}{12}% [
\TMP@EnsureCode{93}{12}% ]
\TMP@EnsureCode{94}{7}% ^ (superscript)
\TMP@EnsureCode{95}{8}% _ (subscript)
\TMP@EnsureCode{96}{12}% `
\TMP@EnsureCode{124}{12}% |
\edef\HOLOGO@AtEnd{%
  \HOLOGO@AtEnd
  \escapechar\the\escapechar\relax
  \noexpand\endinput
}
\escapechar=92 %
%    \end{macrocode}
%
% \subsection{Logo list}
%
%    \begin{macro}{\hologoList}
%    \begin{macrocode}
\def\hologoList{%
  \hologoEntry{(La)TeX}{}{2011/10/01}%
  \hologoEntry{AmSLaTeX}{}{2010/04/16}%
  \hologoEntry{AmSTeX}{}{2010/04/16}%
  \hologoEntry{biber}{}{2011/10/01}%
  \hologoEntry{BibTeX}{}{2011/10/01}%
  \hologoEntry{BibTeX}{sf}{2011/10/01}%
  \hologoEntry{BibTeX}{sc}{2011/10/01}%
  \hologoEntry{BibTeX8}{}{2011/11/22}%
  \hologoEntry{ConTeXt}{}{2011/03/25}%
  \hologoEntry{ConTeXt}{narrow}{2011/03/25}%
  \hologoEntry{ConTeXt}{simple}{2011/03/25}%
  \hologoEntry{emTeX}{}{2010/04/26}%
  \hologoEntry{eTeX}{}{2010/04/08}%
  \hologoEntry{ExTeX}{}{2011/10/01}%
  \hologoEntry{HanTheThanh}{}{2011/11/29}%
  \hologoEntry{iniTeX}{}{2011/10/01}%
  \hologoEntry{KOMAScript}{}{2011/10/01}%
  \hologoEntry{La}{}{2010/05/08}%
  \hologoEntry{LaTeX}{}{2010/04/08}%
  \hologoEntry{LaTeX2e}{}{2010/04/08}%
  \hologoEntry{LaTeX3}{}{2010/04/24}%
  \hologoEntry{LaTeXe}{}{2010/04/08}%
  \hologoEntry{LaTeXML}{}{2011/11/22}%
  \hologoEntry{LaTeXTeX}{}{2011/10/01}%
  \hologoEntry{LuaLaTeX}{}{2010/04/08}%
  \hologoEntry{LuaTeX}{}{2010/04/08}%
  \hologoEntry{LyX}{}{2011/10/01}%
  \hologoEntry{METAFONT}{}{2011/10/01}%
  \hologoEntry{MetaFun}{}{2011/10/01}%
  \hologoEntry{METAPOST}{}{2011/10/01}%
  \hologoEntry{MetaPost}{}{2011/10/01}%
  \hologoEntry{MiKTeX}{}{2011/10/01}%
  \hologoEntry{NTS}{}{2011/10/01}%
  \hologoEntry{OzMF}{}{2011/10/01}%
  \hologoEntry{OzMP}{}{2011/10/01}%
  \hologoEntry{OzTeX}{}{2011/10/01}%
  \hologoEntry{OzTtH}{}{2011/10/01}%
  \hologoEntry{PCTeX}{}{2011/10/01}%
  \hologoEntry{pdfTeX}{}{2011/10/01}%
  \hologoEntry{pdfLaTeX}{}{2011/10/01}%
  \hologoEntry{PiC}{}{2011/10/01}%
  \hologoEntry{PiCTeX}{}{2011/10/01}%
  \hologoEntry{plainTeX}{}{2010/04/08}%
  \hologoEntry{plainTeX}{space}{2010/04/16}%
  \hologoEntry{plainTeX}{hyphen}{2010/04/16}%
  \hologoEntry{plainTeX}{runtogether}{2010/04/16}%
  \hologoEntry{SageTeX}{}{2011/11/22}%
  \hologoEntry{SLiTeX}{}{2011/10/01}%
  \hologoEntry{SLiTeX}{lift}{2011/10/01}%
  \hologoEntry{SLiTeX}{narrow}{2011/10/01}%
  \hologoEntry{SLiTeX}{simple}{2011/10/01}%
  \hologoEntry{SliTeX}{}{2011/10/01}%
  \hologoEntry{SliTeX}{narrow}{2011/10/01}%
  \hologoEntry{SliTeX}{simple}{2011/10/01}%
  \hologoEntry{SliTeX}{lift}{2011/10/01}%
  \hologoEntry{teTeX}{}{2011/10/01}%
  \hologoEntry{TeX}{}{2010/04/08}%
  \hologoEntry{TeX4ht}{}{2011/11/22}%
  \hologoEntry{TTH}{}{2011/11/22}%
  \hologoEntry{virTeX}{}{2011/10/01}%
  \hologoEntry{VTeX}{}{2010/04/24}%
  \hologoEntry{Xe}{}{2010/04/08}%
  \hologoEntry{XeLaTeX}{}{2010/04/08}%
  \hologoEntry{XeTeX}{}{2010/04/08}%
}
%    \end{macrocode}
%    \end{macro}
%
% \subsection{Load resources}
%
%    \begin{macrocode}
\begingroup\expandafter\expandafter\expandafter\endgroup
\expandafter\ifx\csname RequirePackage\endcsname\relax
  \def\TMP@RequirePackage#1[#2]{%
    \begingroup\expandafter\expandafter\expandafter\endgroup
    \expandafter\ifx\csname ver@#1.sty\endcsname\relax
      \input #1.sty\relax
    \fi
  }%
  \TMP@RequirePackage{ltxcmds}[2011/02/04]%
  \TMP@RequirePackage{infwarerr}[2010/04/08]%
  \TMP@RequirePackage{kvsetkeys}[2010/03/01]%
  \TMP@RequirePackage{kvdefinekeys}[2010/03/01]%
  \TMP@RequirePackage{pdftexcmds}[2010/04/01]%
  \TMP@RequirePackage{ifpdf}[2010/01/28]%
  \TMP@RequirePackage{ifluatex}[2010/03/01]%
  \ltx@IfUndefined{newif}{%
    \expandafter\let\csname newif\endcsname\ltx@newif
  }{}%
  \TMP@RequirePackage{ifxetex}[2009/01/23]%
  \TMP@RequirePackage{ifvtex}[2010/03/01]%
\else
  \RequirePackage{ltxcmds}[2011/02/04]%
  \RequirePackage{infwarerr}[2010/04/08]%
  \RequirePackage{kvsetkeys}[2010/03/01]%
  \RequirePackage{kvdefinekeys}[2010/03/01]%
  \RequirePackage{pdftexcmds}[2010/04/01]%
  \RequirePackage{ifpdf}[2010/01/28]%
  \RequirePackage{ifluatex}[2010/03/01]%
  \RequirePackage{ifxetex}[2009/01/23]%
  \RequirePackage{ifvtex}[2010/03/01]%
\fi
%    \end{macrocode}
%
%    \begin{macro}{\HOLOGO@IfDefined}
%    \begin{macrocode}
\def\HOLOGO@IfExists#1{%
  \ifx\@undefined#1%
    \expandafter\ltx@secondoftwo
  \else
    \ifx\relax#1%
      \expandafter\ltx@secondoftwo
    \else
      \expandafter\expandafter\expandafter\ltx@firstoftwo
    \fi
  \fi
}
%    \end{macrocode}
%    \end{macro}
%
% \subsection{Setup macros}
%
%    \begin{macro}{\hologoSetup}
%    \begin{macrocode}
\def\hologoSetup{%
  \let\HOLOGO@name\relax
  \HOLOGO@Setup
}
%    \end{macrocode}
%    \end{macro}
%
%    \begin{macro}{\hologoLogoSetup}
%    \begin{macrocode}
\def\hologoLogoSetup#1{%
  \edef\HOLOGO@name{#1}%
  \ltx@IfUndefined{HoLogo@\HOLOGO@name}{%
    \@PackageError{hologo}{%
      Unknown logo `\HOLOGO@name'%
    }\@ehc
    \ltx@gobble
  }{%
    \HOLOGO@Setup
  }%
}
%    \end{macrocode}
%    \end{macro}
%
%    \begin{macro}{\HOLOGO@Setup}
%    \begin{macrocode}
\def\HOLOGO@Setup{%
  \kvsetkeys{HoLogo}%
}
%    \end{macrocode}
%    \end{macro}
%
% \subsection{Options}
%
%    \begin{macro}{\HOLOGO@DeclareBoolOption}
%    \begin{macrocode}
\def\HOLOGO@DeclareBoolOption#1{%
  \expandafter\chardef\csname HOLOGOOPT@#1\endcsname\ltx@zero
  \kv@define@key{HoLogo}{#1}[true]{%
    \def\HOLOGO@temp{##1}%
    \ifx\HOLOGO@temp\HOLOGO@true
      \ifx\HOLOGO@name\relax
        \expandafter\chardef\csname HOLOGOOPT@#1\endcsname=\ltx@one
      \else
        \expandafter\chardef\csname
        HoLogoOpt@#1@\HOLOGO@name\endcsname\ltx@one
      \fi
      \HOLOGO@SetBreakAll{#1}%
    \else
      \ifx\HOLOGO@temp\HOLOGO@false
        \ifx\HOLOGO@name\relax
          \expandafter\chardef\csname HOLOGOOPT@#1\endcsname=\ltx@zero
        \else
          \expandafter\chardef\csname
          HoLogoOpt@#1@\HOLOGO@name\endcsname=\ltx@zero
        \fi
        \HOLOGO@SetBreakAll{#1}%
      \else
        \@PackageError{hologo}{%
          Unknown value `##1' for boolean option `#1'.\MessageBreak
          Known values are `true' and `false'%
        }\@ehc
      \fi
    \fi
  }%
}
%    \end{macrocode}
%    \end{macro}
%
%    \begin{macro}{\HOLOGO@SetBreakAll}
%    \begin{macrocode}
\def\HOLOGO@SetBreakAll#1{%
  \def\HOLOGO@temp{#1}%
  \ifx\HOLOGO@temp\HOLOGO@break
    \ifx\HOLOGO@name\relax
      \chardef\HOLOGOOPT@hyphenbreak=\HOLOGOOPT@break
      \chardef\HOLOGOOPT@spacebreak=\HOLOGOOPT@break
      \chardef\HOLOGOOPT@discretionarybreak=\HOLOGOOPT@break
    \else
      \expandafter\chardef
         \csname HoLogoOpt@hyphenbreak@\HOLOGO@name\endcsname=%
         \csname HoLogoOpt@break@\HOLOGO@name\endcsname
      \expandafter\chardef
         \csname HoLogoOpt@spacebreak@\HOLOGO@name\endcsname=%
         \csname HoLogoOpt@break@\HOLOGO@name\endcsname
      \expandafter\chardef
         \csname HoLogoOpt@discretionarybreak@\HOLOGO@name
             \endcsname=%
         \csname HoLogoOpt@break@\HOLOGO@name\endcsname
    \fi
  \fi
}
%    \end{macrocode}
%    \end{macro}
%
%    \begin{macro}{\HOLOGO@true}
%    \begin{macrocode}
\def\HOLOGO@true{true}
%    \end{macrocode}
%    \end{macro}
%    \begin{macro}{\HOLOGO@false}
%    \begin{macrocode}
\def\HOLOGO@false{false}
%    \end{macrocode}
%    \end{macro}
%    \begin{macro}{\HOLOGO@break}
%    \begin{macrocode}
\def\HOLOGO@break{break}
%    \end{macrocode}
%    \end{macro}
%
%    \begin{macrocode}
\HOLOGO@DeclareBoolOption{break}
\HOLOGO@DeclareBoolOption{hyphenbreak}
\HOLOGO@DeclareBoolOption{spacebreak}
\HOLOGO@DeclareBoolOption{discretionarybreak}
%    \end{macrocode}
%
%    \begin{macrocode}
\kv@define@key{HoLogo}{variant}{%
  \ifx\HOLOGO@name\relax
    \@PackageError{hologo}{%
      Option `variant' is not available in \string\hologoSetup,%
      \MessageBreak
      Use \string\hologoLogoSetup\space instead%
    }\@ehc
  \else
    \edef\HOLOGO@temp{#1}%
    \ifx\HOLOGO@temp\ltx@empty
      \expandafter
      \let\csname HoLogoOpt@variant@\HOLOGO@name\endcsname\@undefined
    \else
      \ltx@IfUndefined{HoLogo@\HOLOGO@name @\HOLOGO@temp}{%
        \@PackageError{hologo}{%
          Unknown variant `\HOLOGO@temp' of logo `\HOLOGO@name'%
        }\@ehc
      }{%
        \expandafter
        \let\csname HoLogoOpt@variant@\HOLOGO@name\endcsname
            \HOLOGO@temp
      }%
    \fi
  \fi
}
%    \end{macrocode}
%
%    \begin{macro}{\HOLOGO@Variant}
%    \begin{macrocode}
\def\HOLOGO@Variant#1{%
  #1%
  \ltx@ifundefined{HoLogoOpt@variant@#1}{%
  }{%
    @\csname HoLogoOpt@variant@#1\endcsname
  }%
}
%    \end{macrocode}
%    \end{macro}
%
% \subsection{Break/no-break support}
%
%    \begin{macro}{\HOLOGO@space}
%    \begin{macrocode}
\def\HOLOGO@space{%
  \ltx@ifundefined{HoLogoOpt@spacebreak@\HOLOGO@name}{%
    \ltx@ifundefined{HoLogoOpt@break@\HOLOGO@name}{%
      \chardef\HOLOGO@temp=\HOLOGOOPT@spacebreak
    }{%
      \chardef\HOLOGO@temp=%
        \csname HoLogoOpt@break@\HOLOGO@name\endcsname
    }%
  }{%
    \chardef\HOLOGO@temp=%
      \csname HoLogoOpt@spacebreak@\HOLOGO@name\endcsname
  }%
  \ifcase\HOLOGO@temp
    \penalty10000 %
  \fi
  \ltx@space
}
%    \end{macrocode}
%    \end{macro}
%
%    \begin{macro}{\HOLOGO@hyphen}
%    \begin{macrocode}
\def\HOLOGO@hyphen{%
  \ltx@ifundefined{HoLogoOpt@hyphenbreak@\HOLOGO@name}{%
    \ltx@ifundefined{HoLogoOpt@break@\HOLOGO@name}{%
      \chardef\HOLOGO@temp=\HOLOGOOPT@hyphenbreak
    }{%
      \chardef\HOLOGO@temp=%
        \csname HoLogoOpt@break@\HOLOGO@name\endcsname
    }%
  }{%
    \chardef\HOLOGO@temp=%
      \csname HoLogoOpt@hyphenbreak@\HOLOGO@name\endcsname
  }%
  \ifcase\HOLOGO@temp
    \ltx@mbox{-}%
  \else
    -%
  \fi
}
%    \end{macrocode}
%    \end{macro}
%
%    \begin{macro}{\HOLOGO@discretionary}
%    \begin{macrocode}
\def\HOLOGO@discretionary{%
  \ltx@ifundefined{HoLogoOpt@discretionarybreak@\HOLOGO@name}{%
    \ltx@ifundefined{HoLogoOpt@break@\HOLOGO@name}{%
      \chardef\HOLOGO@temp=\HOLOGOOPT@discretionarybreak
    }{%
      \chardef\HOLOGO@temp=%
        \csname HoLogoOpt@break@\HOLOGO@name\endcsname
    }%
  }{%
    \chardef\HOLOGO@temp=%
      \csname HoLogoOpt@discretionarybreak@\HOLOGO@name\endcsname
  }%
  \ifcase\HOLOGO@temp
  \else
    \-%
  \fi
}
%    \end{macrocode}
%    \end{macro}
%
%    \begin{macro}{\HOLOGO@mbox}
%    \begin{macrocode}
\def\HOLOGO@mbox#1{%
  \ltx@ifundefined{HoLogoOpt@break@\HOLOGO@name}{%
    \chardef\HOLOGO@temp=\HOLOGOOPT@hyphenbreak
  }{%
    \chardef\HOLOGO@temp=%
      \csname HoLogoOpt@break@\HOLOGO@name\endcsname
  }%
  \ifcase\HOLOGO@temp
    \ltx@mbox{#1}%
  \else
    #1%
  \fi
}
%    \end{macrocode}
%    \end{macro}
%
% \subsection{Font support}
%
%    \begin{macro}{\HoLogoFont@font}
%    \begin{tabular}{@{}ll@{}}
%    |#1|:& logo name\\
%    |#2|:& font short name\\
%    |#3|:& text
%    \end{tabular}
%    \begin{macrocode}
\def\HoLogoFont@font#1#2#3{%
  \begingroup
    \ltx@IfUndefined{HoLogoFont@logo@#1.#2}{%
      \ltx@IfUndefined{HoLogoFont@font@#2}{%
        \@PackageWarning{hologo}{%
          Missing font `#2' for logo `#1'%
        }%
        #3%
      }{%
        \csname HoLogoFont@font@#2\endcsname{#3}%
      }%
    }{%
      \csname HoLogoFont@logo@#1.#2\endcsname{#3}%
    }%
  \endgroup
}
%    \end{macrocode}
%    \end{macro}
%
%    \begin{macro}{\HoLogoFont@Def}
%    \begin{macrocode}
\def\HoLogoFont@Def#1{%
  \expandafter\def\csname HoLogoFont@font@#1\endcsname
}
%    \end{macrocode}
%    \end{macro}
%    \begin{macro}{\HoLogoFont@LogoDef}
%    \begin{macrocode}
\def\HoLogoFont@LogoDef#1#2{%
  \expandafter\def\csname HoLogoFont@logo@#1.#2\endcsname
}
%    \end{macrocode}
%    \end{macro}
%
% \subsubsection{Font defaults}
%
%    \begin{macro}{\HoLogoFont@font@general}
%    \begin{macrocode}
\HoLogoFont@Def{general}{}%
%    \end{macrocode}
%    \end{macro}
%
%    \begin{macro}{\HoLogoFont@font@rm}
%    \begin{macrocode}
\ltx@IfUndefined{rmfamily}{%
  \ltx@IfUndefined{rm}{%
  }{%
    \HoLogoFont@Def{rm}{\rm}%
  }%
}{%
  \HoLogoFont@Def{rm}{\rmfamily}%
}
%    \end{macrocode}
%    \end{macro}
%
%    \begin{macro}{\HoLogoFont@font@sf}
%    \begin{macrocode}
\ltx@IfUndefined{sffamily}{%
  \ltx@IfUndefined{sf}{%
  }{%
    \HoLogoFont@Def{sf}{\sf}%
  }%
}{%
  \HoLogoFont@Def{sf}{\sffamily}%
}
%    \end{macrocode}
%    \end{macro}
%
%    \begin{macro}{\HoLogoFont@font@bibsf}
%    In case of \hologo{plainTeX} the original small caps
%    variant is used as default. In \hologo{LaTeX}
%    the definition of package \xpackage{dtklogos} \cite{dtklogos}
%    is used.
%\begin{quote}
%\begin{verbatim}
%\DeclareRobustCommand{\BibTeX}{%
%  B%
%  \kern-.05em%
%  \hbox{%
%    $\m@th$% %% force math size calculations
%    \csname S@\f@size\endcsname
%    \fontsize\sf@size\z@
%    \math@fontsfalse
%    \selectfont
%    I%
%    \kern-.025em%
%    B
%  }%
%  \kern-.08em%
%  \-%
%  \TeX
%}
%\end{verbatim}
%\end{quote}
%    \begin{macrocode}
\ltx@IfUndefined{selectfont}{%
  \ltx@IfUndefined{tensc}{%
    \font\tensc=cmcsc10\relax
  }{}%
  \HoLogoFont@Def{bibsf}{\tensc}%
}{%
  \HoLogoFont@Def{bibsf}{%
    $\mathsurround=0pt$%
    \csname S@\f@size\endcsname
    \fontsize\sf@size{0pt}%
    \math@fontsfalse
    \selectfont
  }%
}
%    \end{macrocode}
%    \end{macro}
%
%    \begin{macro}{\HoLogoFont@font@sc}
%    \begin{macrocode}
\ltx@IfUndefined{scshape}{%
  \ltx@IfUndefined{tensc}{%
    \font\tensc=cmcsc10\relax
  }{}%
  \HoLogoFont@Def{sc}{\tensc}%
}{%
  \HoLogoFont@Def{sc}{\scshape}%
}
%    \end{macrocode}
%    \end{macro}
%
%    \begin{macro}{\HoLogoFont@font@sy}
%    \begin{macrocode}
\ltx@IfUndefined{usefont}{%
  \ltx@IfUndefined{tensy}{%
  }{%
    \HoLogoFont@Def{sy}{\tensy}%
  }%
}{%
  \HoLogoFont@Def{sy}{%
    \usefont{OMS}{cmsy}{m}{n}%
  }%
}
%    \end{macrocode}
%    \end{macro}
%
%    \begin{macro}{\HoLogoFont@font@logo}
%    \begin{macrocode}
\begingroup
  \def\x{LaTeX2e}%
\expandafter\endgroup
\ifx\fmtname\x
  \ltx@IfUndefined{logofamily}{%
    \DeclareRobustCommand\logofamily{%
      \not@math@alphabet\logofamily\relax
      \fontencoding{U}%
      \fontfamily{logo}%
      \selectfont
    }%
  }{}%
  \ltx@IfUndefined{logofamily}{%
  }{%
    \HoLogoFont@Def{logo}{\logofamily}%
  }%
\else
  \ltx@IfUndefined{tenlogo}{%
    \font\tenlogo=logo10\relax
  }{}%
  \HoLogoFont@Def{logo}{\tenlogo}%
\fi
%    \end{macrocode}
%    \end{macro}
%
% \subsubsection{Font setup}
%
%    \begin{macro}{\hologoFontSetup}
%    \begin{macrocode}
\def\hologoFontSetup{%
  \let\HOLOGO@name\relax
  \HOLOGO@FontSetup
}
%    \end{macrocode}
%    \end{macro}
%
%    \begin{macro}{\hologoLogoFontSetup}
%    \begin{macrocode}
\def\hologoLogoFontSetup#1{%
  \edef\HOLOGO@name{#1}%
  \ltx@IfUndefined{HoLogo@\HOLOGO@name}{%
    \@PackageError{hologo}{%
      Unknown logo `\HOLOGO@name'%
    }\@ehc
    \ltx@gobble
  }{%
    \HOLOGO@FontSetup
  }%
}
%    \end{macrocode}
%    \end{macro}
%
%    \begin{macro}{\HOLOGO@FontSetup}
%    \begin{macrocode}
\def\HOLOGO@FontSetup{%
  \kvsetkeys{HoLogoFont}%
}
%    \end{macrocode}
%    \end{macro}
%
%    \begin{macrocode}
\def\HOLOGO@temp#1{%
  \kv@define@key{HoLogoFont}{#1}{%
    \ifx\HOLOGO@name\relax
      \HoLogoFont@Def{#1}{##1}%
    \else
      \HoLogoFont@LogoDef\HOLOGO@name{#1}{##1}%
    \fi
  }%
}
\HOLOGO@temp{general}
\HOLOGO@temp{sf}
%    \end{macrocode}
%
% \subsection{Generic logo commands}
%
%    \begin{macrocode}
\HOLOGO@IfExists\hologo{%
  \@PackageError{hologo}{%
    \string\hologo\ltx@space is already defined.\MessageBreak
    Package loading is aborted%
  }\@ehc
  \HOLOGO@AtEnd
}%
\HOLOGO@IfExists\hologoRobust{%
  \@PackageError{hologo}{%
    \string\hologoRobust\ltx@space is already defined.\MessageBreak
    Package loading is aborted%
  }\@ehc
  \HOLOGO@AtEnd
}%
%    \end{macrocode}
%
% \subsubsection{\cs{hologo} and friends}
%
%    \begin{macrocode}
\ifluatex
  \expandafter\ltx@firstofone
\else
  \expandafter\ltx@gobble
\fi
{%
  \ltx@IfUndefined{ifincsname}{%
    \ifnum\luatexversion<36 %
      \expandafter\ltx@gobble
    \else
      \expandafter\ltx@firstofone
    \fi
    {%
      \begingroup
        \ifcase0%
            \directlua{%
              if tex.enableprimitives then %
                tex.enableprimitives('HOLOGO@', {'ifincsname'})%
              else %
                tex.print('1')%
              end%
            }%
            \ifx\HOLOGO@ifincsname\@undefined 1\fi%
            \relax
          \expandafter\ltx@firstofone
        \else
          \endgroup
          \expandafter\ltx@gobble
        \fi
        {%
          \global\let\ifincsname\HOLOGO@ifincsname
        }%
      \HOLOGO@temp
    }%
  }{}%
}
%    \end{macrocode}
%    \begin{macrocode}
\ltx@IfUndefined{ifincsname}{%
  \catcode`$=14 %
}{%
  \catcode`$=9 %
}
%    \end{macrocode}
%
%    \begin{macro}{\hologo}
%    \begin{macrocode}
\def\hologo#1{%
$ \ifincsname
$   \ltx@ifundefined{HoLogoCs@\HOLOGO@Variant{#1}}{%
$     #1%
$   }{%
$     \csname HoLogoCs@\HOLOGO@Variant{#1}\endcsname\ltx@firstoftwo
$   }%
$ \else
    \HOLOGO@IfExists\texorpdfstring\texorpdfstring\ltx@firstoftwo
    {%
      \hologoRobust{#1}%
    }{%
      \ltx@ifundefined{HoLogoBkm@\HOLOGO@Variant{#1}}{%
        \ltx@ifundefined{HoLogo@#1}{?#1?}{#1}%
      }{%
        \csname HoLogoBkm@\HOLOGO@Variant{#1}\endcsname
        \ltx@firstoftwo
      }%
    }%
$ \fi
}
%    \end{macrocode}
%    \end{macro}
%    \begin{macro}{\Hologo}
%    \begin{macrocode}
\def\Hologo#1{%
$ \ifincsname
$   \ltx@ifundefined{HoLogoCs@\HOLOGO@Variant{#1}}{%
$     #1%
$   }{%
$     \csname HoLogoCs@\HOLOGO@Variant{#1}\endcsname\ltx@secondoftwo
$   }%
$ \else
    \HOLOGO@IfExists\texorpdfstring\texorpdfstring\ltx@firstoftwo
    {%
      \HologoRobust{#1}%
    }{%
      \ltx@ifundefined{HoLogoBkm@\HOLOGO@Variant{#1}}{%
        \ltx@ifundefined{HoLogo@#1}{?#1?}{#1}%
      }{%
        \csname HoLogoBkm@\HOLOGO@Variant{#1}\endcsname
        \ltx@secondoftwo
      }%
    }%
$ \fi
}
%    \end{macrocode}
%    \end{macro}
%
%    \begin{macro}{\hologoVariant}
%    \begin{macrocode}
\def\hologoVariant#1#2{%
  \ifx\relax#2\relax
    \hologo{#1}%
  \else
$   \ifincsname
$     \ltx@ifundefined{HoLogoCs@#1@#2}{%
$       #1%
$     }{%
$       \csname HoLogoCs@#1@#2\endcsname\ltx@firstoftwo
$     }%
$   \else
      \HOLOGO@IfExists\texorpdfstring\texorpdfstring\ltx@firstoftwo
      {%
        \hologoVariantRobust{#1}{#2}%
      }{%
        \ltx@ifundefined{HoLogoBkm@#1@#2}{%
          \ltx@ifundefined{HoLogo@#1}{?#1?}{#1}%
        }{%
          \csname HoLogoBkm@#1@#2\endcsname
          \ltx@firstoftwo
        }%
      }%
$   \fi
  \fi
}
%    \end{macrocode}
%    \end{macro}
%    \begin{macro}{\HologoVariant}
%    \begin{macrocode}
\def\HologoVariant#1#2{%
  \ifx\relax#2\relax
    \Hologo{#1}%
  \else
$   \ifincsname
$     \ltx@ifundefined{HoLogoCs@#1@#2}{%
$       #1%
$     }{%
$       \csname HoLogoCs@#1@#2\endcsname\ltx@secondoftwo
$     }%
$   \else
      \HOLOGO@IfExists\texorpdfstring\texorpdfstring\ltx@firstoftwo
      {%
        \HologoVariantRobust{#1}{#2}%
      }{%
        \ltx@ifundefined{HoLogoBkm@#1@#2}{%
          \ltx@ifundefined{HoLogo@#1}{?#1?}{#1}%
        }{%
          \csname HoLogoBkm@#1@#2\endcsname
          \ltx@secondoftwo
        }%
      }%
$   \fi
  \fi
}
%    \end{macrocode}
%    \end{macro}
%
%    \begin{macrocode}
\catcode`\$=3 %
%    \end{macrocode}
%
% \subsubsection{\cs{hologoRobust} and friends}
%
%    \begin{macro}{\hologoRobust}
%    \begin{macrocode}
\ltx@IfUndefined{protected}{%
  \ltx@IfUndefined{DeclareRobustCommand}{%
    \def\hologoRobust#1%
  }{%
    \DeclareRobustCommand*\hologoRobust[1]%
  }%
}{%
  \protected\def\hologoRobust#1%
}%
{%
  \edef\HOLOGO@name{#1}%
  \ltx@IfUndefined{HoLogo@\HOLOGO@Variant\HOLOGO@name}{%
    \@PackageError{hologo}{%
      Unknown logo `\HOLOGO@name'%
    }\@ehc
    ?\HOLOGO@name?%
  }{%
    \ltx@IfUndefined{ver@tex4ht.sty}{%
      \HoLogoFont@font\HOLOGO@name{general}{%
        \csname HoLogo@\HOLOGO@Variant\HOLOGO@name\endcsname
        \ltx@firstoftwo
      }%
    }{%
      \ltx@IfUndefined{HoLogoHtml@\HOLOGO@Variant\HOLOGO@name}{%
        \HOLOGO@name
      }{%
        \csname HoLogoHtml@\HOLOGO@Variant\HOLOGO@name\endcsname
        \ltx@firstoftwo
      }%
    }%
  }%
}
%    \end{macrocode}
%    \end{macro}
%    \begin{macro}{\HologoRobust}
%    \begin{macrocode}
\ltx@IfUndefined{protected}{%
  \ltx@IfUndefined{DeclareRobustCommand}{%
    \def\HologoRobust#1%
  }{%
    \DeclareRobustCommand*\HologoRobust[1]%
  }%
}{%
  \protected\def\HologoRobust#1%
}%
{%
  \edef\HOLOGO@name{#1}%
  \ltx@IfUndefined{HoLogo@\HOLOGO@Variant\HOLOGO@name}{%
    \@PackageError{hologo}{%
      Unknown logo `\HOLOGO@name'%
    }\@ehc
    ?\HOLOGO@name?%
  }{%
    \ltx@IfUndefined{ver@tex4ht.sty}{%
      \HoLogoFont@font\HOLOGO@name{general}{%
        \csname HoLogo@\HOLOGO@Variant\HOLOGO@name\endcsname
        \ltx@secondoftwo
      }%
    }{%
      \ltx@IfUndefined{HoLogoHtml@\HOLOGO@Variant\HOLOGO@name}{%
        \expandafter\HOLOGO@Uppercase\HOLOGO@name
      }{%
        \csname HoLogoHtml@\HOLOGO@Variant\HOLOGO@name\endcsname
        \ltx@secondoftwo
      }%
    }%
  }%
}
%    \end{macrocode}
%    \end{macro}
%    \begin{macro}{\hologoVariantRobust}
%    \begin{macrocode}
\ltx@IfUndefined{protected}{%
  \ltx@IfUndefined{DeclareRobustCommand}{%
    \def\hologoVariantRobust#1#2%
  }{%
    \DeclareRobustCommand*\hologoVariantRobust[2]%
  }%
}{%
  \protected\def\hologoVariantRobust#1#2%
}%
{%
  \begingroup
    \hologoLogoSetup{#1}{variant={#2}}%
    \hologoRobust{#1}%
  \endgroup
}
%    \end{macrocode}
%    \end{macro}
%    \begin{macro}{\HologoVariantRobust}
%    \begin{macrocode}
\ltx@IfUndefined{protected}{%
  \ltx@IfUndefined{DeclareRobustCommand}{%
    \def\HologoVariantRobust#1#2%
  }{%
    \DeclareRobustCommand*\HologoVariantRobust[2]%
  }%
}{%
  \protected\def\HologoVariantRobust#1#2%
}%
{%
  \begingroup
    \hologoLogoSetup{#1}{variant={#2}}%
    \HologoRobust{#1}%
  \endgroup
}
%    \end{macrocode}
%    \end{macro}
%
%    \begin{macro}{\hologorobust}
%    Macro \cs{hologorobust} is only defined for compatibility.
%    Its use is deprecated.
%    \begin{macrocode}
\def\hologorobust{\hologoRobust}
%    \end{macrocode}
%    \end{macro}
%
% \subsection{Helpers}
%
%    \begin{macro}{\HOLOGO@Uppercase}
%    Macro \cs{HOLOGO@Uppercase} is restricted to \cs{uppercase},
%    because \hologo{plainTeX} or \hologo{iniTeX} do not provide
%    \cs{MakeUppercase}.
%    \begin{macrocode}
\def\HOLOGO@Uppercase#1{\uppercase{#1}}
%    \end{macrocode}
%    \end{macro}
%
%    \begin{macro}{\HOLOGO@PdfdocUnicode}
%    \begin{macrocode}
\def\HOLOGO@PdfdocUnicode{%
  \ifx\ifHy@unicode\iftrue
    \expandafter\ltx@secondoftwo
  \else
    \expandafter\ltx@firstoftwo
  \fi
}
%    \end{macrocode}
%    \end{macro}
%
%    \begin{macro}{\HOLOGO@Math}
%    \begin{macrocode}
\def\HOLOGO@MathSetup{%
  \mathsurround0pt\relax
  \HOLOGO@IfExists\f@series{%
    \if b\expandafter\ltx@car\f@series x\@nil
      \csname boldmath\endcsname
   \fi
  }{}%
}
%    \end{macrocode}
%    \end{macro}
%
%    \begin{macro}{\HOLOGO@TempDimen}
%    \begin{macrocode}
\dimendef\HOLOGO@TempDimen=\ltx@zero
%    \end{macrocode}
%    \end{macro}
%    \begin{macro}{\HOLOGO@NegativeKerning}
%    \begin{macrocode}
\def\HOLOGO@NegativeKerning#1{%
  \begingroup
    \HOLOGO@TempDimen=0pt\relax
    \comma@parse@normalized{#1}{%
      \ifdim\HOLOGO@TempDimen=0pt %
        \expandafter\HOLOGO@@NegativeKerning\comma@entry
      \fi
      \ltx@gobble
    }%
    \ifdim\HOLOGO@TempDimen<0pt %
      \kern\HOLOGO@TempDimen
    \fi
  \endgroup
}
%    \end{macrocode}
%    \end{macro}
%    \begin{macro}{\HOLOGO@@NegativeKerning}
%    \begin{macrocode}
\def\HOLOGO@@NegativeKerning#1#2{%
  \setbox\ltx@zero\hbox{#1#2}%
  \HOLOGO@TempDimen=\wd\ltx@zero
  \setbox\ltx@zero\hbox{#1\kern0pt#2}%
  \advance\HOLOGO@TempDimen by -\wd\ltx@zero
}
%    \end{macrocode}
%    \end{macro}
%
%    \begin{macro}{\HOLOGO@SpaceFactor}
%    \begin{macrocode}
\def\HOLOGO@SpaceFactor{%
  \spacefactor1000 %
}
%    \end{macrocode}
%    \end{macro}
%
%    \begin{macro}{\HOLOGO@Span}
%    \begin{macrocode}
\def\HOLOGO@Span#1#2{%
  \HCode{<span class="HoLogo-#1">}%
  #2%
  \HCode{</span>}%
}
%    \end{macrocode}
%    \end{macro}
%
% \subsubsection{Text subscript}
%
%    \begin{macro}{\HOLOGO@SubScript}%
%    \begin{macrocode}
\def\HOLOGO@SubScript#1{%
  \ltx@IfUndefined{textsubscript}{%
    \ltx@IfUndefined{text}{%
      \ltx@mbox{%
        \mathsurround=0pt\relax
        $%
          _{%
            \ltx@IfUndefined{sf@size}{%
              \mathrm{#1}%
            }{%
              \mbox{%
                \fontsize\sf@size{0pt}\selectfont
                #1%
              }%
            }%
          }%
        $%
      }%
    }{%
      \ltx@mbox{%
        \mathsurround=0pt\relax
        $_{\text{#1}}$%
      }%
    }%
  }{%
    \textsubscript{#1}%
  }%
}
%    \end{macrocode}
%    \end{macro}
%
% \subsection{\hologo{TeX} and friends}
%
% \subsubsection{\hologo{TeX}}
%
%    \begin{macro}{\HoLogo@TeX}
%    Source: \hologo{LaTeX} kernel.
%    \begin{macrocode}
\def\HoLogo@TeX#1{%
  T\kern-.1667em\lower.5ex\hbox{E}\kern-.125emX\HOLOGO@SpaceFactor
}
%    \end{macrocode}
%    \end{macro}
%    \begin{macro}{\HoLogoHtml@TeX}
%    \begin{macrocode}
\def\HoLogoHtml@TeX#1{%
  \HoLogoCss@TeX
  \HOLOGO@Span{TeX}{%
    T%
    \HOLOGO@Span{e}{%
      E%
    }%
    X%
  }%
}
%    \end{macrocode}
%    \end{macro}
%    \begin{macro}{\HoLogoCss@TeX}
%    \begin{macrocode}
\def\HoLogoCss@TeX{%
  \Css{%
    span.HoLogo-TeX span.HoLogo-e{%
      position:relative;%
      top:.5ex;%
      margin-left:-.1667em;%
      margin-right:-.125em;%
    }%
  }%
  \Css{%
    a span.HoLogo-TeX span.HoLogo-e{%
      text-decoration:none;%
    }%
  }%
  \global\let\HoLogoCss@TeX\relax
}
%    \end{macrocode}
%    \end{macro}
%
% \subsubsection{\hologo{plainTeX}}
%
%    \begin{macro}{\HoLogo@plainTeX@space}
%    Source: ``The \hologo{TeX}book''
%    \begin{macrocode}
\def\HoLogo@plainTeX@space#1{%
  \HOLOGO@mbox{#1{p}{P}lain}\HOLOGO@space\hologo{TeX}%
}
%    \end{macrocode}
%    \end{macro}
%    \begin{macro}{\HoLogoCs@plainTeX@space}
%    \begin{macrocode}
\def\HoLogoCs@plainTeX@space#1{#1{p}{P}lain TeX}%
%    \end{macrocode}
%    \end{macro}
%    \begin{macro}{\HoLogoBkm@plainTeX@space}
%    \begin{macrocode}
\def\HoLogoBkm@plainTeX@space#1{%
  #1{p}{P}lain \hologo{TeX}%
}
%    \end{macrocode}
%    \end{macro}
%    \begin{macro}{\HoLogoHtml@plainTeX@space}
%    \begin{macrocode}
\def\HoLogoHtml@plainTeX@space#1{%
  #1{p}{P}lain \hologo{TeX}%
}
%    \end{macrocode}
%    \end{macro}
%
%    \begin{macro}{\HoLogo@plainTeX@hyphen}
%    \begin{macrocode}
\def\HoLogo@plainTeX@hyphen#1{%
  \HOLOGO@mbox{#1{p}{P}lain}\HOLOGO@hyphen\hologo{TeX}%
}
%    \end{macrocode}
%    \end{macro}
%    \begin{macro}{\HoLogoCs@plainTeX@hyphen}
%    \begin{macrocode}
\def\HoLogoCs@plainTeX@hyphen#1{#1{p}{P}lain-TeX}
%    \end{macrocode}
%    \end{macro}
%    \begin{macro}{\HoLogoBkm@plainTeX@hyphen}
%    \begin{macrocode}
\def\HoLogoBkm@plainTeX@hyphen#1{%
  #1{p}{P}lain-\hologo{TeX}%
}
%    \end{macrocode}
%    \end{macro}
%    \begin{macro}{\HoLogoHtml@plainTeX@hyphen}
%    \begin{macrocode}
\def\HoLogoHtml@plainTeX@hyphen#1{%
  #1{p}{P}lain-\hologo{TeX}%
}
%    \end{macrocode}
%    \end{macro}
%
%    \begin{macro}{\HoLogo@plainTeX@runtogether}
%    \begin{macrocode}
\def\HoLogo@plainTeX@runtogether#1{%
  \HOLOGO@mbox{#1{p}{P}lain\hologo{TeX}}%
}
%    \end{macrocode}
%    \end{macro}
%    \begin{macro}{\HoLogoCs@plainTeX@runtogether}
%    \begin{macrocode}
\def\HoLogoCs@plainTeX@runtogether#1{#1{p}{P}lainTeX}
%    \end{macrocode}
%    \end{macro}
%    \begin{macro}{\HoLogoBkm@plainTeX@runtogether}
%    \begin{macrocode}
\def\HoLogoBkm@plainTeX@runtogether#1{%
  #1{p}{P}lain\hologo{TeX}%
}
%    \end{macrocode}
%    \end{macro}
%    \begin{macro}{\HoLogoHtml@plainTeX@runtogether}
%    \begin{macrocode}
\def\HoLogoHtml@plainTeX@runtogether#1{%
  #1{p}{P}lain\hologo{TeX}%
}
%    \end{macrocode}
%    \end{macro}
%
%    \begin{macro}{\HoLogo@plainTeX}
%    \begin{macrocode}
\def\HoLogo@plainTeX{\HoLogo@plainTeX@space}
%    \end{macrocode}
%    \end{macro}
%    \begin{macro}{\HoLogoCs@plainTeX}
%    \begin{macrocode}
\def\HoLogoCs@plainTeX{\HoLogoCs@plainTeX@space}
%    \end{macrocode}
%    \end{macro}
%    \begin{macro}{\HoLogoBkm@plainTeX}
%    \begin{macrocode}
\def\HoLogoBkm@plainTeX{\HoLogoBkm@plainTeX@space}
%    \end{macrocode}
%    \end{macro}
%    \begin{macro}{\HoLogoHtml@plainTeX}
%    \begin{macrocode}
\def\HoLogoHtml@plainTeX{\HoLogoHtml@plainTeX@space}
%    \end{macrocode}
%    \end{macro}
%
% \subsubsection{\hologo{LaTeX}}
%
%    Source: \hologo{LaTeX} kernel.
%\begin{quote}
%\begin{verbatim}
%\DeclareRobustCommand{\LaTeX}{%
%  L%
%  \kern-.36em%
%  {%
%    \sbox\z@ T%
%    \vbox to\ht\z@{%
%      \hbox{%
%        \check@mathfonts
%        \fontsize\sf@size\z@
%        \math@fontsfalse
%        \selectfont
%        A%
%      }%
%      \vss
%    }%
%  }%
%  \kern-.15em%
%  \TeX
%}
%\end{verbatim}
%\end{quote}
%
%    \begin{macro}{\HoLogo@La}
%    \begin{macrocode}
\def\HoLogo@La#1{%
  L%
  \kern-.36em%
  \begingroup
    \setbox\ltx@zero\hbox{T}%
    \vbox to\ht\ltx@zero{%
      \hbox{%
        \ltx@ifundefined{check@mathfonts}{%
          \csname sevenrm\endcsname
        }{%
          \check@mathfonts
          \fontsize\sf@size{0pt}%
          \math@fontsfalse\selectfont
        }%
        A%
      }%
      \vss
    }%
  \endgroup
}
%    \end{macrocode}
%    \end{macro}
%
%    \begin{macro}{\HoLogo@LaTeX}
%    Source: \hologo{LaTeX} kernel.
%    \begin{macrocode}
\def\HoLogo@LaTeX#1{%
  \hologo{La}%
  \kern-.15em%
  \hologo{TeX}%
}
%    \end{macrocode}
%    \end{macro}
%    \begin{macro}{\HoLogoHtml@LaTeX}
%    \begin{macrocode}
\def\HoLogoHtml@LaTeX#1{%
  \HoLogoCss@LaTeX
  \HOLOGO@Span{LaTeX}{%
    L%
    \HOLOGO@Span{a}{%
      A%
    }%
    \hologo{TeX}%
  }%
}
%    \end{macrocode}
%    \end{macro}
%    \begin{macro}{\HoLogoCss@LaTeX}
%    \begin{macrocode}
\def\HoLogoCss@LaTeX{%
  \Css{%
    span.HoLogo-LaTeX span.HoLogo-a{%
      position:relative;%
      top:-.5ex;%
      margin-left:-.36em;%
      margin-right:-.15em;%
      font-size:85\%;%
    }%
  }%
  \global\let\HoLogoCss@LaTeX\relax
}
%    \end{macrocode}
%    \end{macro}
%
% \subsubsection{\hologo{(La)TeX}}
%
%    \begin{macro}{\HoLogo@LaTeXTeX}
%    The kerning around the parentheses is taken
%    from package \xpackage{dtklogos} \cite{dtklogos}.
%\begin{quote}
%\begin{verbatim}
%\DeclareRobustCommand{\LaTeXTeX}{%
%  (%
%  \kern-.15em%
%  L%
%  \kern-.36em%
%  {%
%    \sbox\z@ T%
%    \vbox to\ht0{%
%      \hbox{%
%        $\m@th$%
%        \csname S@\f@size\endcsname
%        \fontsize\sf@size\z@
%        \math@fontsfalse
%        \selectfont
%        A%
%      }%
%      \vss
%    }%
%  }%
%  \kern-.2em%
%  )%
%  \kern-.15em%
%  \TeX
%}
%\end{verbatim}
%\end{quote}
%    \begin{macrocode}
\def\HoLogo@LaTeXTeX#1{%
  (%
  \kern-.15em%
  \hologo{La}%
  \kern-.2em%
  )%
  \kern-.15em%
  \hologo{TeX}%
}
%    \end{macrocode}
%    \end{macro}
%    \begin{macro}{\HoLogoBkm@LaTeXTeX}
%    \begin{macrocode}
\def\HoLogoBkm@LaTeXTeX#1{(La)TeX}
%    \end{macrocode}
%    \end{macro}
%
%    \begin{macro}{\HoLogo@(La)TeX}
%    \begin{macrocode}
\expandafter
\let\csname HoLogo@(La)TeX\endcsname\HoLogo@LaTeXTeX
%    \end{macrocode}
%    \end{macro}
%    \begin{macro}{\HoLogoBkm@(La)TeX}
%    \begin{macrocode}
\expandafter
\let\csname HoLogoBkm@(La)TeX\endcsname\HoLogoBkm@LaTeXTeX
%    \end{macrocode}
%    \end{macro}
%    \begin{macro}{\HoLogoHtml@LaTeXTeX}
%    \begin{macrocode}
\def\HoLogoHtml@LaTeXTeX#1{%
  \HoLogoCss@LaTeXTeX
  \HOLOGO@Span{LaTeXTeX}{%
    (%
    \HOLOGO@Span{L}{L}%
    \HOLOGO@Span{a}{A}%
    \HOLOGO@Span{ParenRight}{)}%
    \hologo{TeX}%
  }%
}
%    \end{macrocode}
%    \end{macro}
%    \begin{macro}{\HoLogoHtml@(La)TeX}
%    Kerning after opening parentheses and before closing parentheses
%    is $-0.1$\,em. The original values $-0.15$\,em
%    looked too ugly for a serif font.
%    \begin{macrocode}
\expandafter
\let\csname HoLogoHtml@(La)TeX\endcsname\HoLogoHtml@LaTeXTeX
%    \end{macrocode}
%    \end{macro}
%    \begin{macro}{\HoLogoCss@LaTeXTeX}
%    \begin{macrocode}
\def\HoLogoCss@LaTeXTeX{%
  \Css{%
    span.HoLogo-LaTeXTeX span.HoLogo-L{%
      margin-left:-.1em;%
    }%
  }%
  \Css{%
    span.HoLogo-LaTeXTeX span.HoLogo-a{%
      position:relative;%
      top:-.5ex;%
      margin-left:-.36em;%
      margin-right:-.1em;%
      font-size:85\%;%
    }%
  }%
  \Css{%
    span.HoLogo-LaTeXTeX span.HoLogo-ParenRight{%
      margin-right:-.15em;%
    }%
  }%
  \global\let\HoLogoCss@LaTeXTeX\relax
}
%    \end{macrocode}
%    \end{macro}
%
% \subsubsection{\hologo{LaTeXe}}
%
%    \begin{macro}{\HoLogo@LaTeXe}
%    Source: \hologo{LaTeX} kernel
%    \begin{macrocode}
\def\HoLogo@LaTeXe#1{%
  \hologo{LaTeX}%
  \kern.15em%
  \hbox{%
    \HOLOGO@MathSetup
    2%
    $_{\textstyle\varepsilon}$%
  }%
}
%    \end{macrocode}
%    \end{macro}
%
%    \begin{macro}{\HoLogoCs@LaTeXe}
%    \begin{macrocode}
\ifnum64=`\^^^^0040\relax % test for big chars of LuaTeX/XeTeX
  \catcode`\$=9 %
  \catcode`\&=14 %
\else
  \catcode`\$=14 %
  \catcode`\&=9 %
\fi
\def\HoLogoCs@LaTeXe#1{%
  LaTeX2%
$ \string ^^^^0395%
& e%
}%
\catcode`\$=3 %
\catcode`\&=4 %
%    \end{macrocode}
%    \end{macro}
%
%    \begin{macro}{\HoLogoBkm@LaTeXe}
%    \begin{macrocode}
\def\HoLogoBkm@LaTeXe#1{%
  \hologo{LaTeX}%
  2%
  \HOLOGO@PdfdocUnicode{e}{\textepsilon}%
}
%    \end{macrocode}
%    \end{macro}
%
%    \begin{macro}{\HoLogoHtml@LaTeXe}
%    \begin{macrocode}
\def\HoLogoHtml@LaTeXe#1{%
  \HoLogoCss@LaTeXe
  \HOLOGO@Span{LaTeX2e}{%
    \hologo{LaTeX}%
    \HOLOGO@Span{2}{2}%
    \HOLOGO@Span{e}{%
      \HOLOGO@MathSetup
      \ensuremath{\textstyle\varepsilon}%
    }%
  }%
}
%    \end{macrocode}
%    \end{macro}
%    \begin{macro}{\HoLogoCss@LaTeXe}
%    \begin{macrocode}
\def\HoLogoCss@LaTeXe{%
  \Css{%
    span.HoLogo-LaTeX2e span.HoLogo-2{%
      padding-left:.15em;%
    }%
  }%
  \Css{%
    span.HoLogo-LaTeX2e span.HoLogo-e{%
      position:relative;%
      top:.35ex;%
      text-decoration:none;%
    }%
  }%
  \global\let\HoLogoCss@LaTeXe\relax
}
%    \end{macrocode}
%    \end{macro}
%
%    \begin{macro}{\HoLogo@LaTeX2e}
%    \begin{macrocode}
\expandafter
\let\csname HoLogo@LaTeX2e\endcsname\HoLogo@LaTeXe
%    \end{macrocode}
%    \end{macro}
%    \begin{macro}{\HoLogoCs@LaTeX2e}
%    \begin{macrocode}
\expandafter
\let\csname HoLogoCs@LaTeX2e\endcsname\HoLogoCs@LaTeXe
%    \end{macrocode}
%    \end{macro}
%    \begin{macro}{\HoLogoBkm@LaTeX2e}
%    \begin{macrocode}
\expandafter
\let\csname HoLogoBkm@LaTeX2e\endcsname\HoLogoBkm@LaTeXe
%    \end{macrocode}
%    \end{macro}
%    \begin{macro}{\HoLogoHtml@LaTeX2e}
%    \begin{macrocode}
\expandafter
\let\csname HoLogoHtml@LaTeX2e\endcsname\HoLogoHtml@LaTeXe
%    \end{macrocode}
%    \end{macro}
%
% \subsubsection{\hologo{LaTeX3}}
%
%    \begin{macro}{\HoLogo@LaTeX3}
%    Source: \hologo{LaTeX} kernel
%    \begin{macrocode}
\expandafter\def\csname HoLogo@LaTeX3\endcsname#1{%
  \hologo{LaTeX}%
  3%
}
%    \end{macrocode}
%    \end{macro}
%
%    \begin{macro}{\HoLogoBkm@LaTeX3}
%    \begin{macrocode}
\expandafter\def\csname HoLogoBkm@LaTeX3\endcsname#1{%
  \hologo{LaTeX}%
  3%
}
%    \end{macrocode}
%    \end{macro}
%    \begin{macro}{\HoLogoHtml@LaTeX3}
%    \begin{macrocode}
\expandafter
\let\csname HoLogoHtml@LaTeX3\expandafter\endcsname
\csname HoLogo@LaTeX3\endcsname
%    \end{macrocode}
%    \end{macro}
%
% \subsubsection{\hologo{LaTeXML}}
%
%    \begin{macro}{\HoLogo@LaTeXML}
%    \begin{macrocode}
\def\HoLogo@LaTeXML#1{%
  \HOLOGO@mbox{%
    \hologo{La}%
    \kern-.15em%
    T%
    \kern-.1667em%
    \lower.5ex\hbox{E}%
    \kern-.125em%
    \HoLogoFont@font{LaTeXML}{sc}{xml}%
  }%
}
%    \end{macrocode}
%    \end{macro}
%    \begin{macro}{\HoLogoHtml@pdfLaTeX}
%    \begin{macrocode}
\def\HoLogoHtml@LaTeXML#1{%
  \HOLOGO@Span{LaTeXML}{%
    \HoLogoCss@LaTeX
    \HoLogoCss@TeX
    \HOLOGO@Span{LaTeX}{%
      L%
      \HOLOGO@Span{a}{%
        A%
      }%
    }%
    \HOLOGO@Span{TeX}{%
      T%
      \HOLOGO@Span{e}{%
        E%
      }%
    }%
    \HCode{<span style="font-variant: small-caps;">}%
    xml%
    \HCode{</span>}%
  }%
}
%    \end{macrocode}
%    \end{macro}
%
% \subsubsection{\hologo{eTeX}}
%
%    \begin{macro}{\HoLogo@eTeX}
%    Source: package \xpackage{etex}
%    \begin{macrocode}
\def\HoLogo@eTeX#1{%
  \ltx@mbox{%
    \HOLOGO@MathSetup
    $\varepsilon$%
    -%
    \HOLOGO@NegativeKerning{-T,T-,To}%
    \hologo{TeX}%
  }%
}
%    \end{macrocode}
%    \end{macro}
%    \begin{macro}{\HoLogoCs@eTeX}
%    \begin{macrocode}
\ifnum64=`\^^^^0040\relax % test for big chars of LuaTeX/XeTeX
  \catcode`\$=9 %
  \catcode`\&=14 %
\else
  \catcode`\$=14 %
  \catcode`\&=9 %
\fi
\def\HoLogoCs@eTeX#1{%
$ #1{\string ^^^^0395}{\string ^^^^03b5}%
& #1{e}{E}%
  TeX%
}%
\catcode`\$=3 %
\catcode`\&=4 %
%    \end{macrocode}
%    \end{macro}
%    \begin{macro}{\HoLogoBkm@eTeX}
%    \begin{macrocode}
\def\HoLogoBkm@eTeX#1{%
  \HOLOGO@PdfdocUnicode{#1{e}{E}}{\textepsilon}%
  -%
  \hologo{TeX}%
}
%    \end{macrocode}
%    \end{macro}
%    \begin{macro}{\HoLogoHtml@eTeX}
%    \begin{macrocode}
\def\HoLogoHtml@eTeX#1{%
  \ltx@mbox{%
    \HOLOGO@MathSetup
    $\varepsilon$%
    -%
    \hologo{TeX}%
  }%
}
%    \end{macrocode}
%    \end{macro}
%
% \subsubsection{\hologo{iniTeX}}
%
%    \begin{macro}{\HoLogo@iniTeX}
%    \begin{macrocode}
\def\HoLogo@iniTeX#1{%
  \HOLOGO@mbox{%
    #1{i}{I}ni\hologo{TeX}%
  }%
}
%    \end{macrocode}
%    \end{macro}
%    \begin{macro}{\HoLogoCs@iniTeX}
%    \begin{macrocode}
\def\HoLogoCs@iniTeX#1{#1{i}{I}niTeX}
%    \end{macrocode}
%    \end{macro}
%    \begin{macro}{\HoLogoBkm@iniTeX}
%    \begin{macrocode}
\def\HoLogoBkm@iniTeX#1{%
  #1{i}{I}ni\hologo{TeX}%
}
%    \end{macrocode}
%    \end{macro}
%    \begin{macro}{\HoLogoHtml@iniTeX}
%    \begin{macrocode}
\let\HoLogoHtml@iniTeX\HoLogo@iniTeX
%    \end{macrocode}
%    \end{macro}
%
% \subsubsection{\hologo{virTeX}}
%
%    \begin{macro}{\HoLogo@virTeX}
%    \begin{macrocode}
\def\HoLogo@virTeX#1{%
  \HOLOGO@mbox{%
    #1{v}{V}ir\hologo{TeX}%
  }%
}
%    \end{macrocode}
%    \end{macro}
%    \begin{macro}{\HoLogoCs@virTeX}
%    \begin{macrocode}
\def\HoLogoCs@virTeX#1{#1{v}{V}irTeX}
%    \end{macrocode}
%    \end{macro}
%    \begin{macro}{\HoLogoBkm@virTeX}
%    \begin{macrocode}
\def\HoLogoBkm@virTeX#1{%
  #1{v}{V}ir\hologo{TeX}%
}
%    \end{macrocode}
%    \end{macro}
%    \begin{macro}{\HoLogoHtml@virTeX}
%    \begin{macrocode}
\let\HoLogoHtml@virTeX\HoLogo@virTeX
%    \end{macrocode}
%    \end{macro}
%
% \subsubsection{\hologo{SliTeX}}
%
% \paragraph{Definitions of the three variants.}
%
%    \begin{macro}{\HoLogo@SLiTeX@lift}
%    \begin{macrocode}
\def\HoLogo@SLiTeX@lift#1{%
  \HoLogoFont@font{SliTeX}{rm}{%
    S%
    \kern-.06em%
    L%
    \kern-.18em%
    \raise.32ex\hbox{\HoLogoFont@font{SliTeX}{sc}{i}}%
    \HOLOGO@discretionary
    \kern-.06em%
    \hologo{TeX}%
  }%
}
%    \end{macrocode}
%    \end{macro}
%    \begin{macro}{\HoLogoBkm@SLiTeX@lift}
%    \begin{macrocode}
\def\HoLogoBkm@SLiTeX@lift#1{SLiTeX}
%    \end{macrocode}
%    \end{macro}
%    \begin{macro}{\HoLogoHtml@SLiTeX@lift}
%    \begin{macrocode}
\def\HoLogoHtml@SLiTeX@lift#1{%
  \HoLogoCss@SLiTeX@lift
  \HOLOGO@Span{SLiTeX-lift}{%
    \HoLogoFont@font{SliTeX}{rm}{%
      S%
      \HOLOGO@Span{L}{L}%
      \HOLOGO@Span{i}{i}%
      \hologo{TeX}%
    }%
  }%
}
%    \end{macrocode}
%    \end{macro}
%    \begin{macro}{\HoLogoCss@SLiTeX@lift}
%    \begin{macrocode}
\def\HoLogoCss@SLiTeX@lift{%
  \Css{%
    span.HoLogo-SLiTeX-lift span.HoLogo-L{%
      margin-left:-.06em;%
      margin-right:-.18em;%
    }%
  }%
  \Css{%
    span.HoLogo-SLiTeX-lift span.HoLogo-i{%
      position:relative;%
      top:-.32ex;%
      margin-right:-.06em;%
      font-variant:small-caps;%
    }%
  }%
  \global\let\HoLogoCss@SLiTeX@lift\relax
}
%    \end{macrocode}
%    \end{macro}
%
%    \begin{macro}{\HoLogo@SliTeX@simple}
%    \begin{macrocode}
\def\HoLogo@SliTeX@simple#1{%
  \HoLogoFont@font{SliTeX}{rm}{%
    \ltx@mbox{%
      \HoLogoFont@font{SliTeX}{sc}{Sli}%
    }%
    \HOLOGO@discretionary
    \hologo{TeX}%
  }%
}
%    \end{macrocode}
%    \end{macro}
%    \begin{macro}{\HoLogoBkm@SliTeX@simple}
%    \begin{macrocode}
\def\HoLogoBkm@SliTeX@simple#1{SliTeX}
%    \end{macrocode}
%    \end{macro}
%    \begin{macro}{\HoLogoHtml@SliTeX@simple}
%    \begin{macrocode}
\let\HoLogoHtml@SliTeX@simple\HoLogo@SliTeX@simple
%    \end{macrocode}
%    \end{macro}
%
%    \begin{macro}{\HoLogo@SliTeX@narrow}
%    \begin{macrocode}
\def\HoLogo@SliTeX@narrow#1{%
  \HoLogoFont@font{SliTeX}{rm}{%
    \ltx@mbox{%
      S%
      \kern-.06em%
      \HoLogoFont@font{SliTeX}{sc}{%
        l%
        \kern-.035em%
        i%
      }%
    }%
    \HOLOGO@discretionary
    \kern-.06em%
    \hologo{TeX}%
  }%
}
%    \end{macrocode}
%    \end{macro}
%    \begin{macro}{\HoLogoBkm@SliTeX@narrow}
%    \begin{macrocode}
\def\HoLogoBkm@SliTeX@narrow#1{SliTeX}
%    \end{macrocode}
%    \end{macro}
%    \begin{macro}{\HoLogoHtml@SliTeX@narrow}
%    \begin{macrocode}
\def\HoLogoHtml@SliTeX@narrow#1{%
  \HoLogoCss@SliTeX@narrow
  \HOLOGO@Span{SliTeX-narrow}{%
    \HoLogoFont@font{SliTeX}{rm}{%
      S%
        \HOLOGO@Span{l}{l}%
        \HOLOGO@Span{i}{i}%
      \hologo{TeX}%
    }%
  }%
}
%    \end{macrocode}
%    \end{macro}
%    \begin{macro}{\HoLogoCss@SliTeX@narrow}
%    \begin{macrocode}
\def\HoLogoCss@SliTeX@narrow{%
  \Css{%
    span.HoLogo-SliTeX-narrow span.HoLogo-l{%
      margin-left:-.06em;%
      margin-right:-.035em;%
      font-variant:small-caps;%
    }%
  }%
  \Css{%
    span.HoLogo-SliTeX-narrow span.HoLogo-i{%
      margin-right:-.06em;%
      font-variant:small-caps;%
    }%
  }%
  \global\let\HoLogoCss@SliTeX@narrow\relax
}
%    \end{macrocode}
%    \end{macro}
%
% \paragraph{Macro set completion.}
%
%    \begin{macro}{\HoLogo@SLiTeX@simple}
%    \begin{macrocode}
\def\HoLogo@SLiTeX@simple{\HoLogo@SliTeX@simple}
%    \end{macrocode}
%    \end{macro}
%    \begin{macro}{\HoLogoBkm@SLiTeX@simple}
%    \begin{macrocode}
\def\HoLogoBkm@SLiTeX@simple{\HoLogoBkm@SliTeX@simple}
%    \end{macrocode}
%    \end{macro}
%    \begin{macro}{\HoLogoHtml@SLiTeX@simple}
%    \begin{macrocode}
\def\HoLogoHtml@SLiTeX@simple{\HoLogoHtml@SliTeX@simple}
%    \end{macrocode}
%    \end{macro}
%
%    \begin{macro}{\HoLogo@SLiTeX@narrow}
%    \begin{macrocode}
\def\HoLogo@SLiTeX@narrow{\HoLogo@SliTeX@narrow}
%    \end{macrocode}
%    \end{macro}
%    \begin{macro}{\HoLogoBkm@SLiTeX@narrow}
%    \begin{macrocode}
\def\HoLogoBkm@SLiTeX@narrow{\HoLogoBkm@SliTeX@narrow}
%    \end{macrocode}
%    \end{macro}
%    \begin{macro}{\HoLogoHtml@SLiTeX@narrow}
%    \begin{macrocode}
\def\HoLogoHtml@SLiTeX@narrow{\HoLogoHtml@SliTeX@narrow}
%    \end{macrocode}
%    \end{macro}
%
%    \begin{macro}{\HoLogo@SliTeX@lift}
%    \begin{macrocode}
\def\HoLogo@SliTeX@lift{\HoLogo@SLiTeX@lift}
%    \end{macrocode}
%    \end{macro}
%    \begin{macro}{\HoLogoBkm@SliTeX@lift}
%    \begin{macrocode}
\def\HoLogoBkm@SliTeX@lift{\HoLogoBkm@SLiTeX@lift}
%    \end{macrocode}
%    \end{macro}
%    \begin{macro}{\HoLogoHtml@SliTeX@lift}
%    \begin{macrocode}
\def\HoLogoHtml@SliTeX@lift{\HoLogoHtml@SLiTeX@lift}
%    \end{macrocode}
%    \end{macro}
%
% \paragraph{Defaults.}
%
%    \begin{macro}{\HoLogo@SLiTeX}
%    \begin{macrocode}
\def\HoLogo@SLiTeX{\HoLogo@SLiTeX@lift}
%    \end{macrocode}
%    \end{macro}
%    \begin{macro}{\HoLogoBkm@SLiTeX}
%    \begin{macrocode}
\def\HoLogoBkm@SLiTeX{\HoLogoBkm@SLiTeX@lift}
%    \end{macrocode}
%    \end{macro}
%    \begin{macro}{\HoLogoHtml@SLiTeX}
%    \begin{macrocode}
\def\HoLogoHtml@SLiTeX{\HoLogoHtml@SLiTeX@lift}
%    \end{macrocode}
%    \end{macro}
%
%    \begin{macro}{\HoLogo@SliTeX}
%    \begin{macrocode}
\def\HoLogo@SliTeX{\HoLogo@SliTeX@narrow}
%    \end{macrocode}
%    \end{macro}
%    \begin{macro}{\HoLogoBkm@SliTeX}
%    \begin{macrocode}
\def\HoLogoBkm@SliTeX{\HoLogoBkm@SliTeX@narrow}
%    \end{macrocode}
%    \end{macro}
%    \begin{macro}{\HoLogoHtml@SliTeX}
%    \begin{macrocode}
\def\HoLogoHtml@SliTeX{\HoLogoHtml@SliTeX@narrow}
%    \end{macrocode}
%    \end{macro}
%
% \subsubsection{\hologo{LuaTeX}}
%
%    \begin{macro}{\HoLogo@LuaTeX}
%    The kerning is an idea of Hans Hagen, see mailing list
%    `luatex at tug dot org' in March 2010.
%    \begin{macrocode}
\def\HoLogo@LuaTeX#1{%
  \HOLOGO@mbox{%
    Lua%
    \HOLOGO@NegativeKerning{aT,oT,To}%
    \hologo{TeX}%
  }%
}
%    \end{macrocode}
%    \end{macro}
%    \begin{macro}{\HoLogoHtml@LuaTeX}
%    \begin{macrocode}
\let\HoLogoHtml@LuaTeX\HoLogo@LuaTeX
%    \end{macrocode}
%    \end{macro}
%
% \subsubsection{\hologo{LuaLaTeX}}
%
%    \begin{macro}{\HoLogo@LuaLaTeX}
%    \begin{macrocode}
\def\HoLogo@LuaLaTeX#1{%
  \HOLOGO@mbox{%
    Lua%
    \hologo{LaTeX}%
  }%
}
%    \end{macrocode}
%    \end{macro}
%    \begin{macro}{\HoLogoHtml@LuaLaTeX}
%    \begin{macrocode}
\let\HoLogoHtml@LuaLaTeX\HoLogo@LuaLaTeX
%    \end{macrocode}
%    \end{macro}
%
% \subsubsection{\hologo{XeTeX}, \hologo{XeLaTeX}}
%
%    \begin{macro}{\HOLOGO@IfCharExists}
%    \begin{macrocode}
\ifluatex
  \ifnum\luatexversion<36 %
  \else
    \def\HOLOGO@IfCharExists#1{%
      \ifnum
        \directlua{%
           if luaotfload and luaotfload.aux then
             if luaotfload.aux.font_has_glyph(%
                    font.current(), \number#1) then % 	 
	       tex.print("1") % 	 
	     end % 	 
	   elseif font and font.fonts and font.current then %
            local f = font.fonts[font.current()]%
            if f.characters and f.characters[\number#1] then %
              tex.print("1")%
            end %
          end%
        }0=\ltx@zero
        \expandafter\ltx@secondoftwo
      \else
        \expandafter\ltx@firstoftwo
      \fi
    }%
  \fi
\fi
\ltx@IfUndefined{HOLOGO@IfCharExists}{%
  \def\HOLOGO@@IfCharExists#1{%
    \begingroup
      \tracinglostchars=\ltx@zero
      \setbox\ltx@zero=\hbox{%
        \kern7sp\char#1\relax
        \ifnum\lastkern>\ltx@zero
          \expandafter\aftergroup\csname iffalse\endcsname
        \else
          \expandafter\aftergroup\csname iftrue\endcsname
        \fi
      }%
      % \if{true|false} from \aftergroup
      \endgroup
      \expandafter\ltx@firstoftwo
    \else
      \endgroup
      \expandafter\ltx@secondoftwo
    \fi
  }%
  \ifxetex
    \ltx@IfUndefined{XeTeXfonttype}{}{%
      \ltx@IfUndefined{XeTeXcharglyph}{}{%
        \def\HOLOGO@IfCharExists#1{%
          \ifnum\XeTeXfonttype\font>\ltx@zero
            \expandafter\ltx@firstofthree
          \else
            \expandafter\ltx@gobble
          \fi
          {%
            \ifnum\XeTeXcharglyph#1>\ltx@zero
              \expandafter\ltx@firstoftwo
            \else
              \expandafter\ltx@secondoftwo
            \fi
          }%
          \HOLOGO@@IfCharExists{#1}%
        }%
      }%
    }%
  \fi
}{}
\ltx@ifundefined{HOLOGO@IfCharExists}{%
  \ifnum64=`\^^^^0040\relax % test for big chars of LuaTeX/XeTeX
    \let\HOLOGO@IfCharExists\HOLOGO@@IfCharExists
  \else
    \def\HOLOGO@IfCharExists#1{%
      \ifnum#1>255 %
        \expandafter\ltx@fourthoffour
      \fi
      \HOLOGO@@IfCharExists{#1}%
    }%
  \fi
}{}
%    \end{macrocode}
%    \end{macro}
%
%    \begin{macro}{\HoLogo@Xe}
%    Source: package \xpackage{dtklogos}
%    \begin{macrocode}
\def\HoLogo@Xe#1{%
  X%
  \kern-.1em\relax
  \HOLOGO@IfCharExists{"018E}{%
    \lower.5ex\hbox{\char"018E}%
  }{%
    \chardef\HOLOGO@choice=\ltx@zero
    \ifdim\fontdimen\ltx@one\font>0pt %
      \ltx@IfUndefined{rotatebox}{%
        \ltx@IfUndefined{pgftext}{%
          \ltx@IfUndefined{psscalebox}{%
            \ltx@IfUndefined{HOLOGO@ScaleBox@\hologoDriver}{%
            }{%
              \chardef\HOLOGO@choice=4 %
            }%
          }{%
            \chardef\HOLOGO@choice=3 %
          }%
        }{%
          \chardef\HOLOGO@choice=2 %
        }%
      }{%
        \chardef\HOLOGO@choice=1 %
      }%
      \ifcase\HOLOGO@choice
        \HOLOGO@WarningUnsupportedDriver{Xe}%
        e%
      \or % 1: \rotatebox
        \begingroup
          \setbox\ltx@zero\hbox{\rotatebox{180}{E}}%
          \ltx@LocDimenA=\dp\ltx@zero
          \advance\ltx@LocDimenA by -.5ex\relax
          \raise\ltx@LocDimenA\box\ltx@zero
        \endgroup
      \or % 2: \pgftext
        \lower.5ex\hbox{%
          \pgfpicture
            \pgftext[rotate=180]{E}%
          \endpgfpicture
        }%
      \or % 3: \psscalebox
        \begingroup
          \setbox\ltx@zero\hbox{\psscalebox{-1 -1}{E}}%
          \ltx@LocDimenA=\dp\ltx@zero
          \advance\ltx@LocDimenA by -.5ex\relax
          \raise\ltx@LocDimenA\box\ltx@zero
        \endgroup
      \or % 4: \HOLOGO@PointReflectBox
        \lower.5ex\hbox{\HOLOGO@PointReflectBox{E}}%
      \else
        \@PackageError{hologo}{Internal error (choice/it}\@ehc
      \fi
    \else
      \ltx@IfUndefined{reflectbox}{%
        \ltx@IfUndefined{pgftext}{%
          \ltx@IfUndefined{psscalebox}{%
            \ltx@IfUndefined{HOLOGO@ScaleBox@\hologoDriver}{%
            }{%
              \chardef\HOLOGO@choice=4 %
            }%
          }{%
            \chardef\HOLOGO@choice=3 %
          }%
        }{%
          \chardef\HOLOGO@choice=2 %
        }%
      }{%
        \chardef\HOLOGO@choice=1 %
      }%
      \ifcase\HOLOGO@choice
        \HOLOGO@WarningUnsupportedDriver{Xe}%
        e%
      \or % 1: reflectbox
        \lower.5ex\hbox{%
          \reflectbox{E}%
        }%
      \or % 2: \pgftext
        \lower.5ex\hbox{%
          \pgfpicture
            \pgftransformxscale{-1}%
            \pgftext{E}%
          \endpgfpicture
        }%
      \or % 3: \psscalebox
        \lower.5ex\hbox{%
          \psscalebox{-1 1}{E}%
        }%
      \or % 4: \HOLOGO@Reflectbox
        \lower.5ex\hbox{%
          \HOLOGO@ReflectBox{E}%
        }%
      \else
        \@PackageError{hologo}{Internal error (choice/up)}\@ehc
      \fi
    \fi
  }%
}
%    \end{macrocode}
%    \end{macro}
%    \begin{macro}{\HoLogoHtml@Xe}
%    \begin{macrocode}
\def\HoLogoHtml@Xe#1{%
  \HoLogoCss@Xe
  \HOLOGO@Span{Xe}{%
    X%
    \HOLOGO@Span{e}{%
      \HCode{&\ltx@hashchar x018e;}%
    }%
  }%
}
%    \end{macrocode}
%    \end{macro}
%    \begin{macro}{\HoLogoCss@Xe}
%    \begin{macrocode}
\def\HoLogoCss@Xe{%
  \Css{%
    span.HoLogo-Xe span.HoLogo-e{%
      position:relative;%
      top:.5ex;%
      left-margin:-.1em;%
    }%
  }%
  \global\let\HoLogoCss@Xe\relax
}
%    \end{macrocode}
%    \end{macro}
%
%    \begin{macro}{\HoLogo@XeTeX}
%    \begin{macrocode}
\def\HoLogo@XeTeX#1{%
  \hologo{Xe}%
  \kern-.15em\relax
  \hologo{TeX}%
}
%    \end{macrocode}
%    \end{macro}
%
%    \begin{macro}{\HoLogoHtml@XeTeX}
%    \begin{macrocode}
\def\HoLogoHtml@XeTeX#1{%
  \HoLogoCss@XeTeX
  \HOLOGO@Span{XeTeX}{%
    \hologo{Xe}%
    \hologo{TeX}%
  }%
}
%    \end{macrocode}
%    \end{macro}
%    \begin{macro}{\HoLogoCss@XeTeX}
%    \begin{macrocode}
\def\HoLogoCss@XeTeX{%
  \Css{%
    span.HoLogo-XeTeX span.HoLogo-TeX{%
      margin-left:-.15em;%
    }%
  }%
  \global\let\HoLogoCss@XeTeX\relax
}
%    \end{macrocode}
%    \end{macro}
%
%    \begin{macro}{\HoLogo@XeLaTeX}
%    \begin{macrocode}
\def\HoLogo@XeLaTeX#1{%
  \hologo{Xe}%
  \kern-.13em%
  \hologo{LaTeX}%
}
%    \end{macrocode}
%    \end{macro}
%    \begin{macro}{\HoLogoHtml@XeLaTeX}
%    \begin{macrocode}
\def\HoLogoHtml@XeLaTeX#1{%
  \HoLogoCss@XeLaTeX
  \HOLOGO@Span{XeLaTeX}{%
    \hologo{Xe}%
    \hologo{LaTeX}%
  }%
}
%    \end{macrocode}
%    \end{macro}
%    \begin{macro}{\HoLogoCss@XeLaTeX}
%    \begin{macrocode}
\def\HoLogoCss@XeLaTeX{%
  \Css{%
    span.HoLogo-XeLaTeX span.HoLogo-Xe{%
      margin-right:-.13em;%
    }%
  }%
  \global\let\HoLogoCss@XeLaTeX\relax
}
%    \end{macrocode}
%    \end{macro}
%
% \subsubsection{\hologo{pdfTeX}, \hologo{pdfLaTeX}}
%
%    \begin{macro}{\HoLogo@pdfTeX}
%    \begin{macrocode}
\def\HoLogo@pdfTeX#1{%
  \HOLOGO@mbox{%
    #1{p}{P}df\hologo{TeX}%
  }%
}
%    \end{macrocode}
%    \end{macro}
%    \begin{macro}{\HoLogoCs@pdfTeX}
%    \begin{macrocode}
\def\HoLogoCs@pdfTeX#1{#1{p}{P}dfTeX}
%    \end{macrocode}
%    \end{macro}
%    \begin{macro}{\HoLogoBkm@pdfTeX}
%    \begin{macrocode}
\def\HoLogoBkm@pdfTeX#1{%
  #1{p}{P}df\hologo{TeX}%
}
%    \end{macrocode}
%    \end{macro}
%    \begin{macro}{\HoLogoHtml@pdfTeX}
%    \begin{macrocode}
\let\HoLogoHtml@pdfTeX\HoLogo@pdfTeX
%    \end{macrocode}
%    \end{macro}
%
%    \begin{macro}{\HoLogo@pdfLaTeX}
%    \begin{macrocode}
\def\HoLogo@pdfLaTeX#1{%
  \HOLOGO@mbox{%
    #1{p}{P}df\hologo{LaTeX}%
  }%
}
%    \end{macrocode}
%    \end{macro}
%    \begin{macro}{\HoLogoCs@pdfLaTeX}
%    \begin{macrocode}
\def\HoLogoCs@pdfLaTeX#1{#1{p}{P}dfLaTeX}
%    \end{macrocode}
%    \end{macro}
%    \begin{macro}{\HoLogoBkm@pdfLaTeX}
%    \begin{macrocode}
\def\HoLogoBkm@pdfLaTeX#1{%
  #1{p}{P}df\hologo{LaTeX}%
}
%    \end{macrocode}
%    \end{macro}
%    \begin{macro}{\HoLogoHtml@pdfLaTeX}
%    \begin{macrocode}
\let\HoLogoHtml@pdfLaTeX\HoLogo@pdfLaTeX
%    \end{macrocode}
%    \end{macro}
%
% \subsubsection{\hologo{VTeX}}
%
%    \begin{macro}{\HoLogo@VTeX}
%    \begin{macrocode}
\def\HoLogo@VTeX#1{%
  \HOLOGO@mbox{%
    V\hologo{TeX}%
  }%
}
%    \end{macrocode}
%    \end{macro}
%    \begin{macro}{\HoLogoHtml@VTeX}
%    \begin{macrocode}
\let\HoLogoHtml@VTeX\HoLogo@VTeX
%    \end{macrocode}
%    \end{macro}
%
% \subsubsection{\hologo{AmS}, \dots}
%
%    Source: class \xclass{amsdtx}
%
%    \begin{macro}{\HoLogo@AmS}
%    \begin{macrocode}
\def\HoLogo@AmS#1{%
  \HoLogoFont@font{AmS}{sy}{%
    A%
    \kern-.1667em%
    \lower.5ex\hbox{M}%
    \kern-.125em%
    S%
  }%
}
%    \end{macrocode}
%    \end{macro}
%    \begin{macro}{\HoLogoBkm@AmS}
%    \begin{macrocode}
\def\HoLogoBkm@AmS#1{AmS}
%    \end{macrocode}
%    \end{macro}
%    \begin{macro}{\HoLogoHtml@AmS}
%    \begin{macrocode}
\def\HoLogoHtml@AmS#1{%
  \HoLogoCss@AmS
%  \HoLogoFont@font{AmS}{sy}{%
    \HOLOGO@Span{AmS}{%
      A%
      \HOLOGO@Span{M}{M}%
      S%
    }%
%   }%
}
%    \end{macrocode}
%    \end{macro}
%    \begin{macro}{\HoLogoCss@AmS}
%    \begin{macrocode}
\def\HoLogoCss@AmS{%
  \Css{%
    span.HoLogo-AmS span.HoLogo-M{%
      position:relative;%
      top:.5ex;%
      margin-left:-.1667em;%
      margin-right:-.125em;%
      text-decoration:none;%
    }%
  }%
  \global\let\HoLogoCss@AmS\relax
}
%    \end{macrocode}
%    \end{macro}
%
%    \begin{macro}{\HoLogo@AmSTeX}
%    \begin{macrocode}
\def\HoLogo@AmSTeX#1{%
  \hologo{AmS}%
  \HOLOGO@hyphen
  \hologo{TeX}%
}
%    \end{macrocode}
%    \end{macro}
%    \begin{macro}{\HoLogoBkm@AmSTeX}
%    \begin{macrocode}
\def\HoLogoBkm@AmSTeX#1{AmS-TeX}%
%    \end{macrocode}
%    \end{macro}
%    \begin{macro}{\HoLogoHtml@AmSTeX}
%    \begin{macrocode}
\let\HoLogoHtml@AmSTeX\HoLogo@AmSTeX
%    \end{macrocode}
%    \end{macro}
%
%    \begin{macro}{\HoLogo@AmSLaTeX}
%    \begin{macrocode}
\def\HoLogo@AmSLaTeX#1{%
  \hologo{AmS}%
  \HOLOGO@hyphen
  \hologo{LaTeX}%
}
%    \end{macrocode}
%    \end{macro}
%    \begin{macro}{\HoLogoBkm@AmSLaTeX}
%    \begin{macrocode}
\def\HoLogoBkm@AmSLaTeX#1{AmS-LaTeX}%
%    \end{macrocode}
%    \end{macro}
%    \begin{macro}{\HoLogoHtml@AmSLaTeX}
%    \begin{macrocode}
\let\HoLogoHtml@AmSLaTeX\HoLogo@AmSLaTeX
%    \end{macrocode}
%    \end{macro}
%
% \subsubsection{\hologo{BibTeX}}
%
%    \begin{macro}{\HoLogo@BibTeX@sc}
%    A definition of \hologo{BibTeX} is provided in
%    the documentation source for the manual of \hologo{BibTeX}
%    \cite{btxdoc}.
%\begin{quote}
%\begin{verbatim}
%\def\BibTeX{%
%  {%
%    \rm
%    B%
%    \kern-.05em%
%    {%
%      \sc
%      i%
%      \kern-.025em %
%      b%
%    }%
%    \kern-.08em
%    T%
%    \kern-.1667em%
%    \lower.7ex\hbox{E}%
%    \kern-.125em%
%    X%
%  }%
%}
%\end{verbatim}
%\end{quote}
%    \begin{macrocode}
\def\HoLogo@BibTeX@sc#1{%
  B%
  \kern-.05em%
  \HoLogoFont@font{BibTeX}{sc}{%
    i%
    \kern-.025em%
    b%
  }%
  \HOLOGO@discretionary
  \kern-.08em%
  \hologo{TeX}%
}
%    \end{macrocode}
%    \end{macro}
%    \begin{macro}{\HoLogoHtml@BibTeX@sc}
%    \begin{macrocode}
\def\HoLogoHtml@BibTeX@sc#1{%
  \HoLogoCss@BibTeX@sc
  \HOLOGO@Span{BibTeX-sc}{%
    B%
    \HOLOGO@Span{i}{i}%
    \HOLOGO@Span{b}{b}%
    \hologo{TeX}%
  }%
}
%    \end{macrocode}
%    \end{macro}
%    \begin{macro}{\HoLogoCss@BibTeX@sc}
%    \begin{macrocode}
\def\HoLogoCss@BibTeX@sc{%
  \Css{%
    span.HoLogo-BibTeX-sc span.HoLogo-i{%
      margin-left:-.05em;%
      margin-right:-.025em;%
      font-variant:small-caps;%
    }%
  }%
  \Css{%
    span.HoLogo-BibTeX-sc span.HoLogo-b{%
      margin-right:-.08em;%
      font-variant:small-caps;%
    }%
  }%
  \global\let\HoLogoCss@BibTeX@sc\relax
}
%    \end{macrocode}
%    \end{macro}
%
%    \begin{macro}{\HoLogo@BibTeX@sf}
%    Variant \xoption{sf} avoids trouble with unavailable
%    small caps fonts (e.g., bold versions of Computer Modern or
%    Latin Modern). The definition is taken from
%    package \xpackage{dtklogos} \cite{dtklogos}.
%\begin{quote}
%\begin{verbatim}
%\DeclareRobustCommand{\BibTeX}{%
%  B%
%  \kern-.05em%
%  \hbox{%
%    $\m@th$% %% force math size calculations
%    \csname S@\f@size\endcsname
%    \fontsize\sf@size\z@
%    \math@fontsfalse
%    \selectfont
%    I%
%    \kern-.025em%
%    B
%  }%
%  \kern-.08em%
%  \-%
%  \TeX
%}
%\end{verbatim}
%\end{quote}
%    \begin{macrocode}
\def\HoLogo@BibTeX@sf#1{%
  B%
  \kern-.05em%
  \HoLogoFont@font{BibTeX}{bibsf}{%
    I%
    \kern-.025em%
    B%
  }%
  \HOLOGO@discretionary
  \kern-.08em%
  \hologo{TeX}%
}
%    \end{macrocode}
%    \end{macro}
%    \begin{macro}{\HoLogoHtml@BibTeX@sf}
%    \begin{macrocode}
\def\HoLogoHtml@BibTeX@sf#1{%
  \HoLogoCss@BibTeX@sf
  \HOLOGO@Span{BibTeX-sf}{%
    B%
    \HoLogoFont@font{BibTeX}{bibsf}{%
      \HOLOGO@Span{i}{I}%
      B%
    }%
    \hologo{TeX}%
  }%
}
%    \end{macrocode}
%    \end{macro}
%    \begin{macro}{\HoLogoCss@BibTeX@sf}
%    \begin{macrocode}
\def\HoLogoCss@BibTeX@sf{%
  \Css{%
    span.HoLogo-BibTeX-sf span.HoLogo-i{%
      margin-left:-.05em;%
      margin-right:-.025em;%
    }%
  }%
  \Css{%
    span.HoLogo-BibTeX-sf span.HoLogo-TeX{%
      margin-left:-.08em;%
    }%
  }%
  \global\let\HoLogoCss@BibTeX@sf\relax
}
%    \end{macrocode}
%    \end{macro}
%
%    \begin{macro}{\HoLogo@BibTeX}
%    \begin{macrocode}
\def\HoLogo@BibTeX{\HoLogo@BibTeX@sf}
%    \end{macrocode}
%    \end{macro}
%    \begin{macro}{\HoLogoHtml@BibTeX}
%    \begin{macrocode}
\def\HoLogoHtml@BibTeX{\HoLogoHtml@BibTeX@sf}
%    \end{macrocode}
%    \end{macro}
%
% \subsubsection{\hologo{BibTeX8}}
%
%    \begin{macro}{\HoLogo@BibTeX8}
%    \begin{macrocode}
\expandafter\def\csname HoLogo@BibTeX8\endcsname#1{%
  \hologo{BibTeX}%
  8%
}
%    \end{macrocode}
%    \end{macro}
%
%    \begin{macro}{\HoLogoBkm@BibTeX8}
%    \begin{macrocode}
\expandafter\def\csname HoLogoBkm@BibTeX8\endcsname#1{%
  \hologo{BibTeX}%
  8%
}
%    \end{macrocode}
%    \end{macro}
%    \begin{macro}{\HoLogoHtml@BibTeX8}
%    \begin{macrocode}
\expandafter
\let\csname HoLogoHtml@BibTeX8\expandafter\endcsname
\csname HoLogo@BibTeX8\endcsname
%    \end{macrocode}
%    \end{macro}
%
% \subsubsection{\hologo{ConTeXt}}
%
%    \begin{macro}{\HoLogo@ConTeXt@simple}
%    \begin{macrocode}
\def\HoLogo@ConTeXt@simple#1{%
  \HOLOGO@mbox{Con}%
  \HOLOGO@discretionary
  \HOLOGO@mbox{\hologo{TeX}t}%
}
%    \end{macrocode}
%    \end{macro}
%    \begin{macro}{\HoLogoHtml@ConTeXt@simple}
%    \begin{macrocode}
\let\HoLogoHtml@ConTeXt@simple\HoLogo@ConTeXt@simple
%    \end{macrocode}
%    \end{macro}
%
%    \begin{macro}{\HoLogo@ConTeXt@narrow}
%    This definition of logo \hologo{ConTeXt} with variant \xoption{narrow}
%    comes from TUGboat's class \xclass{ltugboat} (version 2010/11/15 v2.8).
%    \begin{macrocode}
\def\HoLogo@ConTeXt@narrow#1{%
  \HOLOGO@mbox{C\kern-.0333emon}%
  \HOLOGO@discretionary
  \kern-.0667em%
  \HOLOGO@mbox{\hologo{TeX}\kern-.0333emt}%
}
%    \end{macrocode}
%    \end{macro}
%    \begin{macro}{\HoLogoHtml@ConTeXt@narrow}
%    \begin{macrocode}
\def\HoLogoHtml@ConTeXt@narrow#1{%
  \HoLogoCss@ConTeXt@narrow
  \HOLOGO@Span{ConTeXt-narrow}{%
    \HOLOGO@Span{C}{C}%
    on%
    \hologo{TeX}%
    t%
  }%
}
%    \end{macrocode}
%    \end{macro}
%    \begin{macro}{\HoLogoCss@ConTeXt@narrow}
%    \begin{macrocode}
\def\HoLogoCss@ConTeXt@narrow{%
  \Css{%
    span.HoLogo-ConTeXt-narrow span.HoLogo-C{%
      margin-left:-.0333em;%
    }%
  }%
  \Css{%
    span.HoLogo-ConTeXt-narrow span.HoLogo-TeX{%
      margin-left:-.0667em;%
      margin-right:-.0333em;%
    }%
  }%
  \global\let\HoLogoCss@ConTeXt@narrow\relax
}
%    \end{macrocode}
%    \end{macro}
%
%    \begin{macro}{\HoLogo@ConTeXt}
%    \begin{macrocode}
\def\HoLogo@ConTeXt{\HoLogo@ConTeXt@narrow}
%    \end{macrocode}
%    \end{macro}
%    \begin{macro}{\HoLogoHtml@ConTeXt}
%    \begin{macrocode}
\def\HoLogoHtml@ConTeXt{\HoLogoHtml@ConTeXt@narrow}
%    \end{macrocode}
%    \end{macro}
%
% \subsubsection{\hologo{emTeX}}
%
%    \begin{macro}{\HoLogo@emTeX}
%    \begin{macrocode}
\def\HoLogo@emTeX#1{%
  \HOLOGO@mbox{#1{e}{E}m}%
  \HOLOGO@discretionary
  \hologo{TeX}%
}
%    \end{macrocode}
%    \end{macro}
%    \begin{macro}{\HoLogoCs@emTeX}
%    \begin{macrocode}
\def\HoLogoCs@emTeX#1{#1{e}{E}mTeX}%
%    \end{macrocode}
%    \end{macro}
%    \begin{macro}{\HoLogoBkm@emTeX}
%    \begin{macrocode}
\def\HoLogoBkm@emTeX#1{%
  #1{e}{E}m\hologo{TeX}%
}
%    \end{macrocode}
%    \end{macro}
%    \begin{macro}{\HoLogoHtml@emTeX}
%    \begin{macrocode}
\let\HoLogoHtml@emTeX\HoLogo@emTeX
%    \end{macrocode}
%    \end{macro}
%
% \subsubsection{\hologo{ExTeX}}
%
%    \begin{macro}{\HoLogo@ExTeX}
%    The definition is taken from the FAQ of the
%    project \hologo{ExTeX}
%    \cite{ExTeX-FAQ}.
%\begin{quote}
%\begin{verbatim}
%\def\ExTeX{%
%  \textrm{% Logo always with serifs
%    \ensuremath{%
%      \textstyle
%      \varepsilon_{%
%        \kern-0.15em%
%        \mathcal{X}%
%      }%
%    }%
%    \kern-.15em%
%    \TeX
%  }%
%}
%\end{verbatim}
%\end{quote}
%    \begin{macrocode}
\def\HoLogo@ExTeX#1{%
  \HoLogoFont@font{ExTeX}{rm}{%
    \ltx@mbox{%
      \HOLOGO@MathSetup
      $%
        \textstyle
        \varepsilon_{%
          \kern-0.15em%
          \HoLogoFont@font{ExTeX}{sy}{X}%
        }%
      $%
    }%
    \HOLOGO@discretionary
    \kern-.15em%
    \hologo{TeX}%
  }%
}
%    \end{macrocode}
%    \end{macro}
%    \begin{macro}{\HoLogoHtml@ExTeX}
%    \begin{macrocode}
\def\HoLogoHtml@ExTeX#1{%
  \HoLogoCss@ExTeX
  \HoLogoFont@font{ExTeX}{rm}{%
    \HOLOGO@Span{ExTeX}{%
      \ltx@mbox{%
        \HOLOGO@MathSetup
        $\textstyle\varepsilon$%
        \HOLOGO@Span{X}{$\textstyle\chi$}%
        \hologo{TeX}%
      }%
    }%
  }%
}
%    \end{macrocode}
%    \end{macro}
%    \begin{macro}{\HoLogoBkm@ExTeX}
%    \begin{macrocode}
\def\HoLogoBkm@ExTeX#1{%
  \HOLOGO@PdfdocUnicode{#1{e}{E}x}{\textepsilon\textchi}%
  \hologo{TeX}%
}
%    \end{macrocode}
%    \end{macro}
%    \begin{macro}{\HoLogoCss@ExTeX}
%    \begin{macrocode}
\def\HoLogoCss@ExTeX{%
  \Css{%
    span.HoLogo-ExTeX{%
      font-family:serif;%
    }%
  }%
  \Css{%
    span.HoLogo-ExTeX span.HoLogo-TeX{%
      margin-left:-.15em;%
    }%
  }%
  \global\let\HoLogoCss@ExTeX\relax
}
%    \end{macrocode}
%    \end{macro}
%
% \subsubsection{\hologo{MiKTeX}}
%
%    \begin{macro}{\HoLogo@MiKTeX}
%    \begin{macrocode}
\def\HoLogo@MiKTeX#1{%
  \HOLOGO@mbox{MiK}%
  \HOLOGO@discretionary
  \hologo{TeX}%
}
%    \end{macrocode}
%    \end{macro}
%    \begin{macro}{\HoLogoHtml@MiKTeX}
%    \begin{macrocode}
\let\HoLogoHtml@MiKTeX\HoLogo@MiKTeX
%    \end{macrocode}
%    \end{macro}
%
% \subsubsection{\hologo{OzTeX} and friends}
%
%    Source: \hologo{OzTeX} FAQ \cite{OzTeX}:
%    \begin{quote}
%      |\def\OzTeX{O\kern-.03em z\kern-.15em\TeX}|\\
%      (There is no kerning in OzMF, OzMP and OzTtH.)
%    \end{quote}
%
%    \begin{macro}{\HoLogo@OzTeX}
%    \begin{macrocode}
\def\HoLogo@OzTeX#1{%
  O%
  \kern-.03em %
  z%
  \kern-.15em %
  \hologo{TeX}%
}
%    \end{macrocode}
%    \end{macro}
%    \begin{macro}{\HoLogoHtml@OzTeX}
%    \begin{macrocode}
\def\HoLogoHtml@OzTeX#1{%
  \HoLogoCss@OzTeX
  \HOLOGO@Span{OzTeX}{%
    O%
    \HOLOGO@Span{z}{z}%
    \hologo{TeX}%
  }%
}
%    \end{macrocode}
%    \end{macro}
%    \begin{macro}{\HoLogoCss@OzTeX}
%    \begin{macrocode}
\def\HoLogoCss@OzTeX{%
  \Css{%
    span.HoLogo-OzTeX span.HoLogo-z{%
      margin-left:-.03em;%
      margin-right:-.15em;%
    }%
  }%
  \global\let\HoLogoCss@OzTeX\relax
}
%    \end{macrocode}
%    \end{macro}
%
%    \begin{macro}{\HoLogo@OzMF}
%    \begin{macrocode}
\def\HoLogo@OzMF#1{%
  \HOLOGO@mbox{OzMF}%
}
%    \end{macrocode}
%    \end{macro}
%    \begin{macro}{\HoLogo@OzMP}
%    \begin{macrocode}
\def\HoLogo@OzMP#1{%
  \HOLOGO@mbox{OzMP}%
}
%    \end{macrocode}
%    \end{macro}
%    \begin{macro}{\HoLogo@OzTtH}
%    \begin{macrocode}
\def\HoLogo@OzTtH#1{%
  \HOLOGO@mbox{OzTtH}%
}
%    \end{macrocode}
%    \end{macro}
%
% \subsubsection{\hologo{PCTeX}}
%
%    \begin{macro}{\HoLogo@PCTeX}
%    \begin{macrocode}
\def\HoLogo@PCTeX#1{%
  \HOLOGO@mbox{PC}%
  \hologo{TeX}%
}
%    \end{macrocode}
%    \end{macro}
%    \begin{macro}{\HoLogoHtml@PCTeX}
%    \begin{macrocode}
\let\HoLogoHtml@PCTeX\HoLogo@PCTeX
%    \end{macrocode}
%    \end{macro}
%
% \subsubsection{\hologo{PiCTeX}}
%
%    The original definitions from \xfile{pictex.tex} \cite{PiCTeX}:
%\begin{quote}
%\begin{verbatim}
%\def\PiC{%
%  P%
%  \kern-.12em%
%  \lower.5ex\hbox{I}%
%  \kern-.075em%
%  C%
%}
%\def\PiCTeX{%
%  \PiC
%  \kern-.11em%
%  \TeX
%}
%\end{verbatim}
%\end{quote}
%
%    \begin{macro}{\HoLogo@PiC}
%    \begin{macrocode}
\def\HoLogo@PiC#1{%
  P%
  \kern-.12em%
  \lower.5ex\hbox{I}%
  \kern-.075em%
  C%
  \HOLOGO@SpaceFactor
}
%    \end{macrocode}
%    \end{macro}
%    \begin{macro}{\HoLogoHtml@PiC}
%    \begin{macrocode}
\def\HoLogoHtml@PiC#1{%
  \HoLogoCss@PiC
  \HOLOGO@Span{PiC}{%
    P%
    \HOLOGO@Span{i}{I}%
    C%
  }%
}
%    \end{macrocode}
%    \end{macro}
%    \begin{macro}{\HoLogoCss@PiC}
%    \begin{macrocode}
\def\HoLogoCss@PiC{%
  \Css{%
    span.HoLogo-PiC span.HoLogo-i{%
      position:relative;%
      top:.5ex;%
      margin-left:-.12em;%
      margin-right:-.075em;%
      text-decoration:none;%
    }%
  }%
  \global\let\HoLogoCss@PiC\relax
}
%    \end{macrocode}
%    \end{macro}
%
%    \begin{macro}{\HoLogo@PiCTeX}
%    \begin{macrocode}
\def\HoLogo@PiCTeX#1{%
  \hologo{PiC}%
  \HOLOGO@discretionary
  \kern-.11em%
  \hologo{TeX}%
}
%    \end{macrocode}
%    \end{macro}
%    \begin{macro}{\HoLogoHtml@PiCTeX}
%    \begin{macrocode}
\def\HoLogoHtml@PiCTeX#1{%
  \HoLogoCss@PiCTeX
  \HOLOGO@Span{PiCTeX}{%
    \hologo{PiC}%
    \hologo{TeX}%
  }%
}
%    \end{macrocode}
%    \end{macro}
%    \begin{macro}{\HoLogoCss@PiCTeX}
%    \begin{macrocode}
\def\HoLogoCss@PiCTeX{%
  \Css{%
    span.HoLogo-PiCTeX span.HoLogo-PiC{%
      margin-right:-.11em;%
    }%
  }%
  \global\let\HoLogoCss@PiCTeX\relax
}
%    \end{macrocode}
%    \end{macro}
%
% \subsubsection{\hologo{teTeX}}
%
%    \begin{macro}{\HoLogo@teTeX}
%    \begin{macrocode}
\def\HoLogo@teTeX#1{%
  \HOLOGO@mbox{#1{t}{T}e}%
  \HOLOGO@discretionary
  \hologo{TeX}%
}
%    \end{macrocode}
%    \end{macro}
%    \begin{macro}{\HoLogoCs@teTeX}
%    \begin{macrocode}
\def\HoLogoCs@teTeX#1{#1{t}{T}dfTeX}
%    \end{macrocode}
%    \end{macro}
%    \begin{macro}{\HoLogoBkm@teTeX}
%    \begin{macrocode}
\def\HoLogoBkm@teTeX#1{%
  #1{t}{T}e\hologo{TeX}%
}
%    \end{macrocode}
%    \end{macro}
%    \begin{macro}{\HoLogoHtml@teTeX}
%    \begin{macrocode}
\let\HoLogoHtml@teTeX\HoLogo@teTeX
%    \end{macrocode}
%    \end{macro}
%
% \subsubsection{\hologo{TeX4ht}}
%
%    \begin{macro}{\HoLogo@TeX4ht}
%    \begin{macrocode}
\expandafter\def\csname HoLogo@TeX4ht\endcsname#1{%
  \HOLOGO@mbox{\hologo{TeX}4ht}%
}
%    \end{macrocode}
%    \end{macro}
%    \begin{macro}{\HoLogoHtml@TeX4ht}
%    \begin{macrocode}
\expandafter
\let\csname HoLogoHtml@TeX4ht\expandafter\endcsname
\csname HoLogo@TeX4ht\endcsname
%    \end{macrocode}
%    \end{macro}
%
%
% \subsubsection{\hologo{SageTeX}}
%
%    \begin{macro}{\HoLogo@SageTeX}
%    \begin{macrocode}
\def\HoLogo@SageTeX#1{%
  \HOLOGO@mbox{Sage}%
  \HOLOGO@discretionary
  \HOLOGO@NegativeKerning{eT,oT,To}%
  \hologo{TeX}%
}
%    \end{macrocode}
%    \end{macro}
%    \begin{macro}{\HoLogoHtml@SageTeX}
%    \begin{macrocode}
\let\HoLogoHtml@SageTeX\HoLogo@SageTeX
%    \end{macrocode}
%    \end{macro}
%
% \subsection{\hologo{METAFONT} and friends}
%
%    \begin{macro}{\HoLogo@METAFONT}
%    \begin{macrocode}
\def\HoLogo@METAFONT#1{%
  \HoLogoFont@font{METAFONT}{logo}{%
    \HOLOGO@mbox{META}%
    \HOLOGO@discretionary
    \HOLOGO@mbox{FONT}%
  }%
}
%    \end{macrocode}
%    \end{macro}
%
%    \begin{macro}{\HoLogo@METAPOST}
%    \begin{macrocode}
\def\HoLogo@METAPOST#1{%
  \HoLogoFont@font{METAPOST}{logo}{%
    \HOLOGO@mbox{META}%
    \HOLOGO@discretionary
    \HOLOGO@mbox{POST}%
  }%
}
%    \end{macrocode}
%    \end{macro}
%
%    \begin{macro}{\HoLogo@MetaFun}
%    \begin{macrocode}
\def\HoLogo@MetaFun#1{%
  \HOLOGO@mbox{Meta}%
  \HOLOGO@discretionary
  \HOLOGO@mbox{Fun}%
}
%    \end{macrocode}
%    \end{macro}
%
%    \begin{macro}{\HoLogo@MetaPost}
%    \begin{macrocode}
\def\HoLogo@MetaPost#1{%
  \HOLOGO@mbox{Meta}%
  \HOLOGO@discretionary
  \HOLOGO@mbox{Post}%
}
%    \end{macrocode}
%    \end{macro}
%
% \subsection{Others}
%
% \subsubsection{\hologo{biber}}
%
%    \begin{macro}{\HoLogo@biber}
%    \begin{macrocode}
\def\HoLogo@biber#1{%
  \HOLOGO@mbox{#1{b}{B}i}%
  \HOLOGO@discretionary
  \HOLOGO@mbox{ber}%
}
%    \end{macrocode}
%    \end{macro}
%    \begin{macro}{\HoLogoCs@biber}
%    \begin{macrocode}
\def\HoLogoCs@biber#1{#1{b}{B}iber}
%    \end{macrocode}
%    \end{macro}
%    \begin{macro}{\HoLogoBkm@biber}
%    \begin{macrocode}
\def\HoLogoBkm@biber#1{%
  #1{b}{B}iber%
}
%    \end{macrocode}
%    \end{macro}
%    \begin{macro}{\HoLogoHtml@biber}
%    \begin{macrocode}
\let\HoLogoHtml@biber\HoLogo@biber
%    \end{macrocode}
%    \end{macro}
%
% \subsubsection{\hologo{KOMAScript}}
%
%    \begin{macro}{\HoLogo@KOMAScript}
%    The definition for \hologo{KOMAScript} is taken
%    from \hologo{KOMAScript} (\xfile{scrlogo.dtx}, reformatted) \cite{scrlogo}:
%\begin{quote}
%\begin{verbatim}
%\@ifundefined{KOMAScript}{%
%  \DeclareRobustCommand{\KOMAScript}{%
%    \textsf{%
%      K\kern.05em O\kern.05emM\kern.05em A%
%      \kern.1em-\kern.1em %
%      Script%
%    }%
%  }%
%}{}
%\end{verbatim}
%\end{quote}
%    \begin{macrocode}
\def\HoLogo@KOMAScript#1{%
  \HoLogoFont@font{KOMAScript}{sf}{%
    \HOLOGO@mbox{%
      K\kern.05em%
      O\kern.05em%
      M\kern.05em%
      A%
    }%
    \kern.1em%
    \HOLOGO@hyphen
    \kern.1em%
    \HOLOGO@mbox{Script}%
  }%
}
%    \end{macrocode}
%    \end{macro}
%    \begin{macro}{\HoLogoBkm@KOMAScript}
%    \begin{macrocode}
\def\HoLogoBkm@KOMAScript#1{%
  KOMA-Script%
}
%    \end{macrocode}
%    \end{macro}
%    \begin{macro}{\HoLogoHtml@KOMAScript}
%    \begin{macrocode}
\def\HoLogoHtml@KOMAScript#1{%
  \HoLogoCss@KOMAScript
  \HoLogoFont@font{KOMAScript}{sf}{%
    \HOLOGO@Span{KOMAScript}{%
      K%
      \HOLOGO@Span{O}{O}%
      M%
      \HOLOGO@Span{A}{A}%
      \HOLOGO@Span{hyphen}{-}%
      Script%
    }%
  }%
}
%    \end{macrocode}
%    \end{macro}
%    \begin{macro}{\HoLogoCss@KOMAScript}
%    \begin{macrocode}
\def\HoLogoCss@KOMAScript{%
  \Css{%
    span.HoLogo-KOMAScript{%
      font-family:sans-serif;%
    }%
  }%
  \Css{%
    span.HoLogo-KOMAScript span.HoLogo-O{%
      padding-left:.05em;%
      padding-right:.05em;%
    }%
  }%
  \Css{%
    span.HoLogo-KOMAScript span.HoLogo-A{%
      padding-left:.05em;%
    }%
  }%
  \Css{%
    span.HoLogo-KOMAScript span.HoLogo-hyphen{%
      padding-left:.1em;%
      padding-right:.1em;%
    }%
  }%
  \global\let\HoLogoCss@KOMAScript\relax
}
%    \end{macrocode}
%    \end{macro}
%
% \subsubsection{\hologo{LyX}}
%
%    \begin{macro}{\HoLogo@LyX}
%    The definition is taken from the documentation source files
%    of \hologo{LyX}, \xfile{Intro.lyx} \cite{LyX}:
%\begin{quote}
%\begin{verbatim}
%\def\LyX{%
%  \texorpdfstring{%
%    L\kern-.1667em\lower.25em\hbox{Y}\kern-.125emX\@%
%  }{%
%    LyX%
%  }%
%}
%\end{verbatim}
%\end{quote}
%    \begin{macrocode}
\def\HoLogo@LyX#1{%
  L%
  \kern-.1667em%
  \lower.25em\hbox{Y}%
  \kern-.125em%
  X%
  \HOLOGO@SpaceFactor
}
%    \end{macrocode}
%    \end{macro}
%    \begin{macro}{\HoLogoHtml@LyX}
%    \begin{macrocode}
\def\HoLogoHtml@LyX#1{%
  \HoLogoCss@LyX
  \HOLOGO@Span{LyX}{%
    L%
    \HOLOGO@Span{y}{Y}%
    X%
  }%
}
%    \end{macrocode}
%    \end{macro}
%    \begin{macro}{\HoLogoCss@LyX}
%    \begin{macrocode}
\def\HoLogoCss@LyX{%
  \Css{%
    span.HoLogo-LyX span.HoLogo-y{%
      position:relative;%
      top:.25em;%
      margin-left:-.1667em;%
      margin-right:-.125em;%
      text-decoration:none;%
    }%
  }%
  \global\let\HoLogoCss@LyX\relax
}
%    \end{macrocode}
%    \end{macro}
%
% \subsubsection{\hologo{NTS}}
%
%    \begin{macro}{\HoLogo@NTS}
%    Definition for \hologo{NTS} can be found in
%    package \xpackage{etex\textunderscore man} for the \hologo{eTeX} manual \cite{etexman}
%    and in package \xpackage{dtklogos} \cite{dtklogos}:
%\begin{quote}
%\begin{verbatim}
%\def\NTS{%
%  \leavevmode
%  \hbox{%
%    $%
%      \cal N%
%      \kern-0.35em%
%      \lower0.5ex\hbox{$\cal T$}%
%      \kern-0.2em%
%      S%
%    $%
%  }%
%}
%\end{verbatim}
%\end{quote}
%    \begin{macrocode}
\def\HoLogo@NTS#1{%
  \HoLogoFont@font{NTS}{sy}{%
    N\/%
    \kern-.35em%
    \lower.5ex\hbox{T\/}%
    \kern-.2em%
    S\/%
  }%
  \HOLOGO@SpaceFactor
}
%    \end{macrocode}
%    \end{macro}
%
% \subsubsection{\Hologo{TTH} (\hologo{TeX} to HTML translator)}
%
%    Source: \url{http://hutchinson.belmont.ma.us/tth/}
%    In the HTML source the second `T' is printed as subscript.
%\begin{quote}
%\begin{verbatim}
%T<sub>T</sub>H
%\end{verbatim}
%\end{quote}
%    \begin{macro}{\HoLogo@TTH}
%    \begin{macrocode}
\def\HoLogo@TTH#1{%
  \ltx@mbox{%
    T\HOLOGO@SubScript{T}H%
  }%
  \HOLOGO@SpaceFactor
}
%    \end{macrocode}
%    \end{macro}
%
%    \begin{macro}{\HoLogoHtml@TTH}
%    \begin{macrocode}
\def\HoLogoHtml@TTH#1{%
  T\HCode{<sub>}T\HCode{</sub>}H%
}
%    \end{macrocode}
%    \end{macro}
%
% \subsubsection{\Hologo{HanTheThanh}}
%
%    Partial source: Package \xpackage{dtklogos}.
%    The double accent is U+1EBF (latin small letter e with circumflex
%    and acute).
%    \begin{macro}{\HoLogo@HanTheThanh}
%    \begin{macrocode}
\def\HoLogo@HanTheThanh#1{%
  \ltx@mbox{H\`an}%
  \HOLOGO@space
  \ltx@mbox{%
    Th%
    \HOLOGO@IfCharExists{"1EBF}{%
      \char"1EBF\relax
    }{%
      \^e\hbox to 0pt{\hss\raise .5ex\hbox{\'{}}}%
    }%
  }%
  \HOLOGO@space
  \ltx@mbox{Th\`anh}%
}
%    \end{macrocode}
%    \end{macro}
%    \begin{macro}{\HoLogoBkm@HanTheThanh}
%    \begin{macrocode}
\def\HoLogoBkm@HanTheThanh#1{%
  H\`an %
  Th\HOLOGO@PdfdocUnicode{\^e}{\9036\277} %
  Th\`anh%
}
%    \end{macrocode}
%    \end{macro}
%    \begin{macro}{\HoLogoHtml@HanTheThanh}
%    \begin{macrocode}
\def\HoLogoHtml@HanTheThanh#1{%
  H\`an %
  Th\HCode{&\ltx@hashchar x1ebf;} %
  Th\`anh%
}
%    \end{macrocode}
%    \end{macro}
%
% \subsection{Driver detection}
%
%    \begin{macrocode}
\HOLOGO@IfExists\InputIfFileExists{%
  \InputIfFileExists{hologo.cfg}{}{}%
}{%
  \ltx@IfUndefined{pdf@filesize}{%
    \def\HOLOGO@InputIfExists{%
      \openin\HOLOGO@temp=hologo.cfg\relax
      \ifeof\HOLOGO@temp
        \closein\HOLOGO@temp
      \else
        \closein\HOLOGO@temp
        \begingroup
          \def\x{LaTeX2e}%
        \expandafter\endgroup
        \ifx\fmtname\x
          \input{hologo.cfg}%
        \else
          \input hologo.cfg\relax
        \fi
      \fi
    }%
    \ltx@IfUndefined{newread}{%
      \chardef\HOLOGO@temp=15 %
      \def\HOLOGO@CheckRead{%
        \ifeof\HOLOGO@temp
          \HOLOGO@InputIfExists
        \else
          \ifcase\HOLOGO@temp
            \@PackageWarningNoLine{hologo}{%
              Configuration file ignored, because\MessageBreak
              a free read register could not be found%
            }%
          \else
            \begingroup
              \count\ltx@cclv=\HOLOGO@temp
              \advance\ltx@cclv by \ltx@minusone
              \edef\x{\endgroup
                \chardef\noexpand\HOLOGO@temp=\the\count\ltx@cclv
                \relax
              }%
            \x
          \fi
        \fi
      }%
    }{%
      \csname newread\endcsname\HOLOGO@temp
      \HOLOGO@InputIfExists
    }%
  }{%
    \edef\HOLOGO@temp{\pdf@filesize{hologo.cfg}}%
    \ifx\HOLOGO@temp\ltx@empty
    \else
      \ifnum\HOLOGO@temp>0 %
        \begingroup
          \def\x{LaTeX2e}%
        \expandafter\endgroup
        \ifx\fmtname\x
          \input{hologo.cfg}%
        \else
          \input hologo.cfg\relax
        \fi
      \else
        \@PackageInfoNoLine{hologo}{%
          Empty configuration file `hologo.cfg' ignored%
        }%
      \fi
    \fi
  }%
}
%    \end{macrocode}
%
%    \begin{macrocode}
\def\HOLOGO@temp#1#2{%
  \kv@define@key{HoLogoDriver}{#1}[]{%
    \begingroup
      \def\HOLOGO@temp{##1}%
      \ltx@onelevel@sanitize\HOLOGO@temp
      \ifx\HOLOGO@temp\ltx@empty
      \else
        \@PackageError{hologo}{%
          Value (\HOLOGO@temp) not permitted for option `#1'%
        }%
        \@ehc
      \fi
    \endgroup
    \def\hologoDriver{#2}%
  }%
}%
\def\HOLOGO@@temp#1#2{%
  \ifx\kv@value\relax
    \HOLOGO@temp{#1}{#1}%
  \else
    \HOLOGO@temp{#1}{#2}%
  \fi
}%
\kv@parse@normalized{%
  pdftex,%
  luatex=pdftex,%
  dvipdfm,%
  dvipdfmx=dvipdfm,%
  dvips,%
  dvipsone=dvips,%
  xdvi=dvips,%
  xetex,%
  vtex,%
}\HOLOGO@@temp
%    \end{macrocode}
%
%    \begin{macrocode}
\kv@define@key{HoLogoDriver}{driverfallback}{%
  \def\HOLOGO@DriverFallback{#1}%
}
%    \end{macrocode}
%
%    \begin{macro}{\HOLOGO@DriverFallback}
%    \begin{macrocode}
\def\HOLOGO@DriverFallback{dvips}
%    \end{macrocode}
%    \end{macro}
%
%    \begin{macro}{\hologoDriverSetup}
%    \begin{macrocode}
\def\hologoDriverSetup{%
  \let\hologoDriver\ltx@undefined
  \HOLOGO@DriverSetup
}
%    \end{macrocode}
%    \end{macro}
%
%    \begin{macro}{\HOLOGO@DriverSetup}
%    \begin{macrocode}
\def\HOLOGO@DriverSetup#1{%
  \kvsetkeys{HoLogoDriver}{#1}%
  \HOLOGO@CheckDriver
  \ltx@ifundefined{hologoDriver}{%
    \begingroup
    \edef\x{\endgroup
      \noexpand\kvsetkeys{HoLogoDriver}{\HOLOGO@DriverFallback}%
    }\x
  }{}%
  \@PackageInfoNoLine{hologo}{Using driver `\hologoDriver'}%
}
%    \end{macrocode}
%    \end{macro}
%
%    \begin{macro}{\HOLOGO@CheckDriver}
%    \begin{macrocode}
\def\HOLOGO@CheckDriver{%
  \ifpdf
    \def\hologoDriver{pdftex}%
    \let\HOLOGO@pdfliteral\pdfliteral
    \ifluatex
      \ifx\pdfextension\@undefined\else
        \protected\def\pdfliteral{\pdfextension literal}%
        \let\HOLOGO@pdfliteral\pdfliteral
      \fi
      \ltx@IfUndefined{HOLOGO@pdfliteral}{%
        \ifnum\luatexversion<36 %
        \else
          \begingroup
            \let\HOLOGO@temp\endgroup
            \ifcase0%
                \directlua{%
                  if tex.enableprimitives then %
                    tex.enableprimitives('HOLOGO@', {'pdfliteral'})%
                  else %
                    tex.print('1')%
                  end%
                }%
                \ifx\HOLOGO@pdfliteral\@undefined 1\fi%
                \relax%
              \endgroup
              \let\HOLOGO@temp\relax
              \global\let\HOLOGO@pdfliteral\HOLOGO@pdfliteral
            \fi%
          \HOLOGO@temp
        \fi
      }{}%
    \fi
    \ltx@IfUndefined{HOLOGO@pdfliteral}{%
      \@PackageWarningNoLine{hologo}{%
        Cannot find \string\pdfliteral
      }%
    }{}%
  \else
    \ifxetex
      \def\hologoDriver{xetex}%
    \else
      \ifvtex
        \def\hologoDriver{vtex}%
      \fi
    \fi
  \fi
}
%    \end{macrocode}
%    \end{macro}
%
%    \begin{macro}{\HOLOGO@WarningUnsupportedDriver}
%    \begin{macrocode}
\def\HOLOGO@WarningUnsupportedDriver#1{%
  \@PackageWarningNoLine{hologo}{%
    Logo `#1' needs driver specific macros,\MessageBreak
    but driver `\hologoDriver' is not supported.\MessageBreak
    Use a different driver or\MessageBreak
    load package `graphics' or `pgf'%
  }%
}
%    \end{macrocode}
%    \end{macro}
%
% \subsubsection{Reflect box macros}
%
%    Skip driver part if not needed.
%    \begin{macrocode}
\ltx@IfUndefined{reflectbox}{}{%
  \ltx@IfUndefined{rotatebox}{}{%
    \HOLOGO@AtEnd
  }%
}
\ltx@IfUndefined{pgftext}{}{%
  \HOLOGO@AtEnd
}
\ltx@IfUndefined{psscalebox}{}{%
  \HOLOGO@AtEnd
}
%    \end{macrocode}
%
%    \begin{macrocode}
\def\HOLOGO@temp{LaTeX2e}
\ifx\fmtname\HOLOGO@temp
  \RequirePackage{kvoptions}[2011/06/30]%
  \ProcessKeyvalOptions{HoLogoDriver}%
\fi
\HOLOGO@DriverSetup{}
%    \end{macrocode}
%
%    \begin{macro}{\HOLOGO@ReflectBox}
%    \begin{macrocode}
\def\HOLOGO@ReflectBox#1{%
  \begingroup
    \setbox\ltx@zero\hbox{\begingroup#1\endgroup}%
    \setbox\ltx@two\hbox{%
      \kern\wd\ltx@zero
      \csname HOLOGO@ScaleBox@\hologoDriver\endcsname{-1}{1}{%
        \hbox to 0pt{\copy\ltx@zero\hss}%
      }%
    }%
    \wd\ltx@two=\wd\ltx@zero
    \box\ltx@two
  \endgroup
}
%    \end{macrocode}
%    \end{macro}
%
%    \begin{macro}{\HOLOGO@PointReflectBox}
%    \begin{macrocode}
\def\HOLOGO@PointReflectBox#1{%
  \begingroup
    \setbox\ltx@zero\hbox{\begingroup#1\endgroup}%
    \setbox\ltx@two\hbox{%
      \kern\wd\ltx@zero
      \raise\ht\ltx@zero\hbox{%
        \csname HOLOGO@ScaleBox@\hologoDriver\endcsname{-1}{-1}{%
          \hbox to 0pt{\copy\ltx@zero\hss}%
        }%
      }%
    }%
    \wd\ltx@two=\wd\ltx@zero
    \box\ltx@two
  \endgroup
}
%    \end{macrocode}
%    \end{macro}
%
%    We must define all variants because of dynamic driver setup.
%    \begin{macrocode}
\def\HOLOGO@temp#1#2{#2}
%    \end{macrocode}
%
%    \begin{macro}{\HOLOGO@ScaleBox@pdftex}
%    \begin{macrocode}
\HOLOGO@temp{pdftex}{%
  \def\HOLOGO@ScaleBox@pdftex#1#2#3{%
    \HOLOGO@pdfliteral{%
      q #1 0 0 #2 0 0 cm%
    }%
    #3%
    \HOLOGO@pdfliteral{%
      Q%
    }%
  }%
}
%    \end{macrocode}
%    \end{macro}
%    \begin{macro}{\HOLOGO@ScaleBox@dvips}
%    \begin{macrocode}
\HOLOGO@temp{dvips}{%
  \def\HOLOGO@ScaleBox@dvips#1#2#3{%
    \special{ps:%
      gsave %
      currentpoint %
      currentpoint translate %
      #1 #2 scale %
      neg exch neg exch translate%
    }%
    #3%
    \special{ps:%
      currentpoint %
      grestore %
      moveto%
    }%
  }%
}
%    \end{macrocode}
%    \end{macro}
%    \begin{macro}{\HOLOGO@ScaleBox@dvipdfm}
%    \begin{macrocode}
\HOLOGO@temp{dvipdfm}{%
  \let\HOLOGO@ScaleBox@dvipdfm\HOLOGO@ScaleBox@dvips
}
%    \end{macrocode}
%    \end{macro}
%    Since \hologo{XeTeX} v0.6.
%    \begin{macro}{\HOLOGO@ScaleBox@xetex}
%    \begin{macrocode}
\HOLOGO@temp{xetex}{%
  \def\HOLOGO@ScaleBox@xetex#1#2#3{%
    \special{x:gsave}%
    \special{x:scale #1 #2}%
    #3%
    \special{x:grestore}%
  }%
}
%    \end{macrocode}
%    \end{macro}
%    \begin{macro}{\HOLOGO@ScaleBox@vtex}
%    \begin{macrocode}
\HOLOGO@temp{vtex}{%
  \def\HOLOGO@ScaleBox@vtex#1#2#3{%
    \special{r(#1,0,0,#2,0,0}%
    #3%
    \special{r)}%
  }%
}
%    \end{macrocode}
%    \end{macro}
%
%    \begin{macrocode}
\HOLOGO@AtEnd%
%</package>
%    \end{macrocode}
%
% \section{Test}
%
% \subsection{Catcode checks for loading}
%
%    \begin{macrocode}
%<*test1>
%    \end{macrocode}
%    \begin{macrocode}
\catcode`\{=1 %
\catcode`\}=2 %
\catcode`\#=6 %
\catcode`\@=11 %
\expandafter\ifx\csname count@\endcsname\relax
  \countdef\count@=255 %
\fi
\expandafter\ifx\csname @gobble\endcsname\relax
  \long\def\@gobble#1{}%
\fi
\expandafter\ifx\csname @firstofone\endcsname\relax
  \long\def\@firstofone#1{#1}%
\fi
\expandafter\ifx\csname loop\endcsname\relax
  \expandafter\@firstofone
\else
  \expandafter\@gobble
\fi
{%
  \def\loop#1\repeat{%
    \def\body{#1}%
    \iterate
  }%
  \def\iterate{%
    \body
      \let\next\iterate
    \else
      \let\next\relax
    \fi
    \next
  }%
  \let\repeat=\fi
}%
\def\RestoreCatcodes{}
\count@=0 %
\loop
  \edef\RestoreCatcodes{%
    \RestoreCatcodes
    \catcode\the\count@=\the\catcode\count@\relax
  }%
\ifnum\count@<255 %
  \advance\count@ 1 %
\repeat

\def\RangeCatcodeInvalid#1#2{%
  \count@=#1\relax
  \loop
    \catcode\count@=15 %
  \ifnum\count@<#2\relax
    \advance\count@ 1 %
  \repeat
}
\def\RangeCatcodeCheck#1#2#3{%
  \count@=#1\relax
  \loop
    \ifnum#3=\catcode\count@
    \else
      \errmessage{%
        Character \the\count@\space
        with wrong catcode \the\catcode\count@\space
        instead of \number#3%
      }%
    \fi
  \ifnum\count@<#2\relax
    \advance\count@ 1 %
  \repeat
}
\def\space{ }
\expandafter\ifx\csname LoadCommand\endcsname\relax
  \def\LoadCommand{\input hologo.sty\relax}%
\fi
\def\Test{%
  \RangeCatcodeInvalid{0}{47}%
  \RangeCatcodeInvalid{58}{64}%
  \RangeCatcodeInvalid{91}{96}%
  \RangeCatcodeInvalid{123}{255}%
  \catcode`\@=12 %
  \catcode`\\=0 %
  \catcode`\%=14 %
  \LoadCommand
  \RangeCatcodeCheck{0}{36}{15}%
  \RangeCatcodeCheck{37}{37}{14}%
  \RangeCatcodeCheck{38}{47}{15}%
  \RangeCatcodeCheck{48}{57}{12}%
  \RangeCatcodeCheck{58}{63}{15}%
  \RangeCatcodeCheck{64}{64}{12}%
  \RangeCatcodeCheck{65}{90}{11}%
  \RangeCatcodeCheck{91}{91}{15}%
  \RangeCatcodeCheck{92}{92}{0}%
  \RangeCatcodeCheck{93}{96}{15}%
  \RangeCatcodeCheck{97}{122}{11}%
  \RangeCatcodeCheck{123}{255}{15}%
  \RestoreCatcodes
}
\Test
\csname @@end\endcsname
\end
%    \end{macrocode}
%    \begin{macrocode}
%</test1>
%    \end{macrocode}
%
% \subsection{Spacefactor}
%
%    The space factor must be 1000 after a logo. If it is greater 1000
%    then the following space is a space after a sentence closing point.
%    If the space factor is smaller 1000 then an immediate following
%    dot is interpreted as abbreviation, not sentence closing point.
%
%    \begin{macrocode}
%<*test-spacefactor>
\NeedsTeXFormat{LaTeX2e}
\documentclass{article}
\usepackage{hologo}[2016/05/12]
\usepackage{kvsetkeys}
\usepackage{qstest}
\IncludeTests{*}
\LogTests{log}{*}{*}
\begin{document}
\begin{qstest}{spacefactor}{spacefactor}
\newcommand*{\Test}[1]{%
  \sbox0{%
    \hologo{#1}%
    \Expect*{1000 (#1)}*{\the\spacefactor\space(#1)}%
  }%
}%
\makeatletter
\def\TestList{}
\def\hologoEntry#1#2#3{%
  \edef\TestList{%
    \ifx\TestList\@empty
    \else
      \TestList,%
    \fi
    #1%
    \ifx\\#2\\%
    \else
      ={variant=#2}%
    \fi
  }%
}
\hologoList
\expandafter\kv@parse@normalized\expandafter{%
  \TestList
}{%
  \begingroup
    \let\@logo=\kv@key
    \ifx\kv@value\relax
    \else
      \expandafter\hologoLogoSetup\expandafter\@logo\expandafter{%
        \kv@value
      }%
    \fi
    \Test\@logo
  \endgroup
  \@gobbletwo
}
\end{qstest}
\end{document}
%</test-spacefactor>
%    \end{macrocode}
%
% \subsection{Complete list}
%
%    \begin{macrocode}
%<*test-list>
\NeedsTeXFormat{LaTeX2e}
\documentclass[12pt,a4paper]{article}
\usepackage{hologo}[2016/05/12]
\usepackage[T1]{fontenc}
\usepackage{lmodern}
\usepackage{parskip}
\usepackage[unicode]{hyperref}[2011/09/28]
\usepackage{bookmark}[2011/09/19]
\bookmarksetup{%
  numbered,%
  open,%
  openlevel=2,%
}
\renewcommand*{\contentsname}{List of logos}
\begin{document}
\tableofcontents
\def\TestFont#1#2#3#4#5#6{%
  \begingroup
    \usefont{#3}{#4}{#5}{#6}%
    \HologoVariant{#1}{#2}/\hologoVariant{#1}{#2}%
    \quad
    \begingroup\scriptsize\hologoVariant{#1}{#2}\endgroup
    \quad
  \endgroup
  (#3/#4/#5/#6)%
  \par
}
\makeatletter
\def\hologoEntry#1#2#3{%
  \section{%
    \HologoVariant{#1}{#2}/\hologoVariant{#1}{#2} %
    {[#1\ifx\\#2\\\else\space(#2)\fi]}% hash-ok
  }% braces around [] because of bug in tex4ht
  \begingroup
    \hypersetup{unicode=false}%
    \bookmark[%
      dest=\@currentHref,%
      rellevel=1,%
      keeplevel,%
    ]{%
      \HologoVariant{#1}{#2}/\hologoVariant{#1}{#2} %
      (PDFDocEncoding)%
    }%
  \endgroup
  \TestFont{#1}{#2}{OT1}{cmr}{m}{n}%
  \TestFont{#1}{#2}{OT1}{cmss}{m}{n}%
  \TestFont{#1}{#2}{OT1}{cmr}{b}{n}%
  \TestFont{#1}{#2}{OT1}{cmr}{m}{it}%
  \TestFont{#1}{#2}{OT1}{cmtt}{m}{n}%
  \TestFont{#1}{#2}{T1}{lmr}{m}{n}%
  \TestFont{#1}{#2}{T1}{lmss}{m}{n}%
  \TestFont{#1}{#2}{T1}{lmr}{b}{n}%
  \TestFont{#1}{#2}{T1}{lmr}{m}{it}%
  \TestFont{#1}{#2}{T1}{lmtt}{m}{n}%
  \TestFont{#1}{#2}{T1}{lmvtt}{m}{n}%
  \TestFont{#1}{#2}{T1}{qtm}{m}{n}%
  \TestFont{#1}{#2}{T1}{qhv}{m}{n}%
  \TestFont{#1}{#2}{T1}{qtm}{b}{n}%
  \TestFont{#1}{#2}{T1}{qtm}{m}{it}%
  \TestFont{#1}{#2}{T1}{qcr}{m}{n}%
  \newpage
}
\makeatother
\hologoList
\end{document}
%</test-list>
%    \end{macrocode}
%
% \section{Installation}
%
% \subsection{Download}
%
% \paragraph{Package.} This package is available on
% CTAN\footnote{\url{ftp://ftp.ctan.org/tex-archive/}}:
% \begin{description}
% \item[\CTAN{macros/latex/contrib/oberdiek/hologo.dtx}] The source file.
% \item[\CTAN{macros/latex/contrib/oberdiek/hologo.pdf}] Documentation.
% \end{description}
%
%
% \paragraph{Bundle.} All the packages of the bundle `oberdiek'
% are also available in a TDS compliant ZIP archive. There
% the packages are already unpacked and the documentation files
% are generated. The files and directories obey the TDS standard.
% \begin{description}
% \item[\CTAN{install/macros/latex/contrib/oberdiek.tds.zip}]
% \end{description}
% \emph{TDS} refers to the standard ``A Directory Structure
% for \TeX\ Files'' (\CTAN{tds/tds.pdf}). Directories
% with \xfile{texmf} in their name are usually organized this way.
%
% \subsection{Bundle installation}
%
% \paragraph{Unpacking.} Unpack the \xfile{oberdiek.tds.zip} in the
% TDS tree (also known as \xfile{texmf} tree) of your choice.
% Example (linux):
% \begin{quote}
%   |unzip oberdiek.tds.zip -d ~/texmf|
% \end{quote}
%
% \paragraph{Script installation.}
% Check the directory \xfile{TDS:scripts/oberdiek/} for
% scripts that need further installation steps.
% Package \xpackage{attachfile2} comes with the Perl script
% \xfile{pdfatfi.pl} that should be installed in such a way
% that it can be called as \texttt{pdfatfi}.
% Example (linux):
% \begin{quote}
%   |chmod +x scripts/oberdiek/pdfatfi.pl|\\
%   |cp scripts/oberdiek/pdfatfi.pl /usr/local/bin/|
% \end{quote}
%
% \subsection{Package installation}
%
% \paragraph{Unpacking.} The \xfile{.dtx} file is a self-extracting
% \docstrip\ archive. The files are extracted by running the
% \xfile{.dtx} through \plainTeX:
% \begin{quote}
%   \verb|tex hologo.dtx|
% \end{quote}
%
% \paragraph{TDS.} Now the different files must be moved into
% the different directories in your installation TDS tree
% (also known as \xfile{texmf} tree):
% \begin{quote}
% \def\t{^^A
% \begin{tabular}{@{}>{\ttfamily}l@{ $\rightarrow$ }>{\ttfamily}l@{}}
%   hologo.sty & tex/generic/oberdiek/hologo.sty\\
%   hologo.pdf & doc/latex/oberdiek/hologo.pdf\\
%   example/hologo-example.tex & doc/latex/oberdiek/example/hologo-example.tex\\
%   test/hologo-test1.tex & doc/latex/oberdiek/test/hologo-test1.tex\\
%   test/hologo-test-spacefactor.tex & doc/latex/oberdiek/test/hologo-test-spacefactor.tex\\
%   test/hologo-test-list.tex & doc/latex/oberdiek/test/hologo-test-list.tex\\
%   hologo.dtx & source/latex/oberdiek/hologo.dtx\\
% \end{tabular}^^A
% }^^A
% \sbox0{\t}^^A
% \ifdim\wd0>\linewidth
%   \begingroup
%     \advance\linewidth by\leftmargin
%     \advance\linewidth by\rightmargin
%   \edef\x{\endgroup
%     \def\noexpand\lw{\the\linewidth}^^A
%   }\x
%   \def\lwbox{^^A
%     \leavevmode
%     \hbox to \linewidth{^^A
%       \kern-\leftmargin\relax
%       \hss
%       \usebox0
%       \hss
%       \kern-\rightmargin\relax
%     }^^A
%   }^^A
%   \ifdim\wd0>\lw
%     \sbox0{\small\t}^^A
%     \ifdim\wd0>\linewidth
%       \ifdim\wd0>\lw
%         \sbox0{\footnotesize\t}^^A
%         \ifdim\wd0>\linewidth
%           \ifdim\wd0>\lw
%             \sbox0{\scriptsize\t}^^A
%             \ifdim\wd0>\linewidth
%               \ifdim\wd0>\lw
%                 \sbox0{\tiny\t}^^A
%                 \ifdim\wd0>\linewidth
%                   \lwbox
%                 \else
%                   \usebox0
%                 \fi
%               \else
%                 \lwbox
%               \fi
%             \else
%               \usebox0
%             \fi
%           \else
%             \lwbox
%           \fi
%         \else
%           \usebox0
%         \fi
%       \else
%         \lwbox
%       \fi
%     \else
%       \usebox0
%     \fi
%   \else
%     \lwbox
%   \fi
% \else
%   \usebox0
% \fi
% \end{quote}
% If you have a \xfile{docstrip.cfg} that configures and enables \docstrip's
% TDS installing feature, then some files can already be in the right
% place, see the documentation of \docstrip.
%
% \subsection{Refresh file name databases}
%
% If your \TeX~distribution
% (\teTeX, \mikTeX, \dots) relies on file name databases, you must refresh
% these. For example, \teTeX\ users run \verb|texhash| or
% \verb|mktexlsr|.
%
% \subsection{Some details for the interested}
%
% \paragraph{Attached source.}
%
% The PDF documentation on CTAN also includes the
% \xfile{.dtx} source file. It can be extracted by
% AcrobatReader 6 or higher. Another option is \textsf{pdftk},
% e.g. unpack the file into the current directory:
% \begin{quote}
%   \verb|pdftk hologo.pdf unpack_files output .|
% \end{quote}
%
% \paragraph{Unpacking with \LaTeX.}
% The \xfile{.dtx} chooses its action depending on the format:
% \begin{description}
% \item[\plainTeX:] Run \docstrip\ and extract the files.
% \item[\LaTeX:] Generate the documentation.
% \end{description}
% If you insist on using \LaTeX\ for \docstrip\ (really,
% \docstrip\ does not need \LaTeX), then inform the autodetect routine
% about your intention:
% \begin{quote}
%   \verb|latex \let\install=y\input{hologo.dtx}|
% \end{quote}
% Do not forget to quote the argument according to the demands
% of your shell.
%
% \paragraph{Generating the documentation.}
% You can use both the \xfile{.dtx} or the \xfile{.drv} to generate
% the documentation. The process can be configured by the
% configuration file \xfile{ltxdoc.cfg}. For instance, put this
% line into this file, if you want to have A4 as paper format:
% \begin{quote}
%   \verb|\PassOptionsToClass{a4paper}{article}|
% \end{quote}
% An example follows how to generate the
% documentation with pdf\LaTeX:
% \begin{quote}
%\begin{verbatim}
%pdflatex hologo.dtx
%makeindex -s gind.ist hologo.idx
%pdflatex hologo.dtx
%makeindex -s gind.ist hologo.idx
%pdflatex hologo.dtx
%\end{verbatim}
% \end{quote}
%
% \section{Catalogue}
%
% The following XML file can be used as source for the
% \href{http://mirror.ctan.org/help/Catalogue/catalogue.html}{\TeX\ Catalogue}.
% The elements \texttt{caption} and \texttt{description} are imported
% from the original XML file from the Catalogue.
% The name of the XML file in the Catalogue is \xfile{hologo.xml}.
%    \begin{macrocode}
%<*catalogue>
<?xml version='1.0' encoding='us-ascii'?>
<!DOCTYPE entry SYSTEM 'catalogue.dtd'>
<entry datestamp='$Date$' modifier='$Author$' id='hologo'>
  <name>hologo</name>
  <caption>A collection of logos with bookmark support.</caption>
  <authorref id='auth:oberdiek'/>
  <copyright owner='Heiko Oberdiek' year='2010-2012'/>
  <license type='lppl1.3'/>
  <version number='1.10'/>
  <description>
    The package defines a single command <tt>\hologo</tt>, whose
    argument is the usual case-confused ASCII version of the logo.
    The command is bookmark-enabled, so that every logo becomes
    available in bookmarks without further work.
    <p/>
    The package is part of the <xref refid='oberdiek'>oberdiek</xref>
    bundle.
  </description>
  <documentation details='Package documentation'
      href='ctan:/macros/latex/contrib/oberdiek/hologo.pdf'/>
  <ctan file='true' path='/macros/latex/contrib/oberdiek/hologo.dtx'/>
  <miktex location='oberdiek'/>
  <texlive location='oberdiek'/>
  <install path='/macros/latex/contrib/oberdiek/oberdiek.tds.zip'/>
</entry>
%</catalogue>
%    \end{macrocode}
%
% \begin{thebibliography}{9}
% \raggedright
%
% \bibitem{btxdoc}
% Oren Patashnik,
% \textit{\hologo{BibTeX}ing},
% 1988-02-08.\\
% \CTAN{biblio/bibtex/base/}
%
% \bibitem{dtklogos}
% Gerd Neugebauer, DANTE,
% \textit{Package \xpackage{dtklogos}},
% 2011-04-25.\\
% \CTAN{usergrps/dante/dtk/dtklogos.sty}
%
% \bibitem{etexman}
% The \hologo{NTS} Team,
% \textit{The \hologo{eTeX} manual},
% 1998-02.\\
% \CTAN{systems/e-tex/v2/doc/}
%
% \bibitem{ExTeX-FAQ}
% The \hologo{ExTeX} group,
% \textit{\hologo{ExTeX}: FAQ -- How is \hologo{ExTeX} typeset?},
% 2007-04-14.\\
% \url{http://www.extex.org/documentation/faq.html}
%
% \bibitem{LyX}
% %@MISC{ LyX,
% %  title = {{LyX 2.0.0 -- The Document Processor [Computer software and manual]}},
% %  author = {{The LyX Team}},
% %  howpublished = {Internet: http://www.lyx.org},
% %  year = {2011-05-08},
% %  note = {Retrieved May 10, 2011, from http://www.lyx.org},
% %  url = {http://www.lyx.org/}
% %}
% The \hologo{LyX} Team,
% \textit{\hologo{LyX} -- The Document Processor},
% 2011-05-08.\\
% \url{http://www.lyx.org/}
%
% \bibitem{OzTeX}
% Andrew Trevorrow,
% \hologo{OzTeX} FAQ: What is the correct way to typeset ``\hologo{OzTeX}''?,
% 2011-09-15 (visited).
% \url{http://www.trevorrow.com/oztex/ozfaq.html#oztex-logo}
%
% \bibitem{PiCTeX}
% Michael Wichura,
% \textit{The \hologo{PiCTeX} macro package},
% 1987-09-21.
% \CTAN{graphics/pictex/}
%
% \bibitem{scrlogo}
% Markus Kohm,
% \textit{\hologo{KOMAScript} Datei \xfile{scrlogo.dtx}},
% 2009-01-30.\\
% \CTAN{install/macros/latex/contrib/komascript.tds.zip}
%
% \end{thebibliography}
%
% \begin{History}
%   \begin{Version}{2010/04/08 v1.0}
%   \item
%     The first version.
%   \end{Version}
%   \begin{Version}{2010/04/16 v1.1}
%   \item
%     \cs{Hologo} added for support of logos at start of a sentence.
%   \item
%     \cs{hologoSetup} and \cs{hologoLogoSetup} added.
%   \item
%     Options \xoption{break}, \xoption{hyphenbreak}, \xoption{spacebreak}
%     added.
%   \item
%     Variant support added by option \xoption{variant}.
%   \end{Version}
%   \begin{Version}{2010/04/24 v1.2}
%   \item
%     \hologo{LaTeX3} added.
%   \item
%     \hologo{VTeX} added.
%   \end{Version}
%   \begin{Version}{2010/11/21 v1.3}
%   \item
%     \hologo{iniTeX}, \hologo{virTeX} added.
%   \end{Version}
%   \begin{Version}{2011/03/25 v1.4}
%   \item
%     \hologo{ConTeXt} with variants added.
%   \item
%     Option \xoption{discretionarybreak} added as refinement for
%     option \xoption{break}.
%   \end{Version}
%   \begin{Version}{2011/04/21 v1.5}
%   \item
%     Wrong TDS directory for test files fixed.
%   \end{Version}
%   \begin{Version}{2011/10/01 v1.6}
%   \item
%     Support for package \xpackage{tex4ht} added.
%   \item
%     Support for \cs{csname} added if \cs{ifincsname} is available.
%   \item
%     New logos:
%     \hologo{(La)TeX},
%     \hologo{biber},
%     \hologo{BibTeX} (\xoption{sc}, \xoption{sf}),
%     \hologo{emTeX},
%     \hologo{ExTeX},
%     \hologo{KOMAScript},
%     \hologo{La},
%     \hologo{LyX},
%     \hologo{MiKTeX},
%     \hologo{NTS},
%     \hologo{OzMF},
%     \hologo{OzMP},
%     \hologo{OzTeX},
%     \hologo{OzTtH},
%     \hologo{PCTeX},
%     \hologo{PiC},
%     \hologo{PiCTeX},
%     \hologo{METAFONT},
%     \hologo{MetaFun},
%     \hologo{METAPOST},
%     \hologo{MetaPost},
%     \hologo{SLiTeX} (\xoption{lift}, \xoption{narrow}, \xoption{simple}),
%     \hologo{SliTeX} (\xoption{narrow}, \xoption{simple}, \xoption{lift}),
%     \hologo{teTeX}.
%   \item
%     Fixes:
%     \hologo{iniTeX},
%     \hologo{pdfLaTeX},
%     \hologo{pdfTeX},
%     \hologo{virTeX}.
%   \item
%     \cs{hologoFontSetup} and \cs{hologoLogoFontSetup} added.
%   \item
%     \cs{hologoVariant} and \cs{HologoVariant} added.
%   \end{Version}
%   \begin{Version}{2011/11/22 v1.7}
%   \item
%     New logos:
%     \hologo{BibTeX8},
%     \hologo{LaTeXML},
%     \hologo{SageTeX},
%     \hologo{TeX4ht},
%     \hologo{TTH}.
%   \item
%     \hologo{Xe} and friends: Driver stuff fixed.
%   \item
%     \hologo{Xe} and friends: Support for italic added.
%   \item
%     \hologo{Xe} and friends: Package support for \xpackage{pgf}
%     and \xpackage{pstricks} added.
%   \end{Version}
%   \begin{Version}{2011/11/29 v1.8}
%   \item
%     New logos:
%     \hologo{HanTheThanh}.
%   \end{Version}
%   \begin{Version}{2011/12/21 v1.9}
%   \item
%     Patch for package \xpackage{ifxetex} added for the case that
%     \cs{newif} is undefined in \hologo{iniTeX}.
%   \item
%     Some fixes for \hologo{iniTeX}.
%   \end{Version}
%   \begin{Version}{2012/04/26 v1.10}
%   \item
%     Fix in bookmark version of logo ``\hologo{HanTheThanh}''.
%   \end{Version}
%   \begin{Version}{2016/05/12 v1.11}
%   \item
%     Update HOLOGO@IfCharExists (previously in texlive)
%   \item define pdfliteral in current luatex.
%   \end{Version}
% \end{History}
%
% \PrintIndex
%
% \Finale
\endinput
%
        \else
          \input hologo.cfg\relax
        \fi
      \else
        \@PackageInfoNoLine{hologo}{%
          Empty configuration file `hologo.cfg' ignored%
        }%
      \fi
    \fi
  }%
}
%    \end{macrocode}
%
%    \begin{macrocode}
\def\HOLOGO@temp#1#2{%
  \kv@define@key{HoLogoDriver}{#1}[]{%
    \begingroup
      \def\HOLOGO@temp{##1}%
      \ltx@onelevel@sanitize\HOLOGO@temp
      \ifx\HOLOGO@temp\ltx@empty
      \else
        \@PackageError{hologo}{%
          Value (\HOLOGO@temp) not permitted for option `#1'%
        }%
        \@ehc
      \fi
    \endgroup
    \def\hologoDriver{#2}%
  }%
}%
\def\HOLOGO@@temp#1#2{%
  \ifx\kv@value\relax
    \HOLOGO@temp{#1}{#1}%
  \else
    \HOLOGO@temp{#1}{#2}%
  \fi
}%
\kv@parse@normalized{%
  pdftex,%
  luatex=pdftex,%
  dvipdfm,%
  dvipdfmx=dvipdfm,%
  dvips,%
  dvipsone=dvips,%
  xdvi=dvips,%
  xetex,%
  vtex,%
}\HOLOGO@@temp
%    \end{macrocode}
%
%    \begin{macrocode}
\kv@define@key{HoLogoDriver}{driverfallback}{%
  \def\HOLOGO@DriverFallback{#1}%
}
%    \end{macrocode}
%
%    \begin{macro}{\HOLOGO@DriverFallback}
%    \begin{macrocode}
\def\HOLOGO@DriverFallback{dvips}
%    \end{macrocode}
%    \end{macro}
%
%    \begin{macro}{\hologoDriverSetup}
%    \begin{macrocode}
\def\hologoDriverSetup{%
  \let\hologoDriver\ltx@undefined
  \HOLOGO@DriverSetup
}
%    \end{macrocode}
%    \end{macro}
%
%    \begin{macro}{\HOLOGO@DriverSetup}
%    \begin{macrocode}
\def\HOLOGO@DriverSetup#1{%
  \kvsetkeys{HoLogoDriver}{#1}%
  \HOLOGO@CheckDriver
  \ltx@ifundefined{hologoDriver}{%
    \begingroup
    \edef\x{\endgroup
      \noexpand\kvsetkeys{HoLogoDriver}{\HOLOGO@DriverFallback}%
    }\x
  }{}%
  \@PackageInfoNoLine{hologo}{Using driver `\hologoDriver'}%
}
%    \end{macrocode}
%    \end{macro}
%
%    \begin{macro}{\HOLOGO@CheckDriver}
%    \begin{macrocode}
\def\HOLOGO@CheckDriver{%
  \ifpdf
    \def\hologoDriver{pdftex}%
    \let\HOLOGO@pdfliteral\pdfliteral
    \ifluatex
      \ifx\pdfextension\@undefined\else
        \protected\def\pdfliteral{\pdfextension literal}%
        \let\HOLOGO@pdfliteral\pdfliteral
      \fi
      \ltx@IfUndefined{HOLOGO@pdfliteral}{%
        \ifnum\luatexversion<36 %
        \else
          \begingroup
            \let\HOLOGO@temp\endgroup
            \ifcase0%
                \directlua{%
                  if tex.enableprimitives then %
                    tex.enableprimitives('HOLOGO@', {'pdfliteral'})%
                  else %
                    tex.print('1')%
                  end%
                }%
                \ifx\HOLOGO@pdfliteral\@undefined 1\fi%
                \relax%
              \endgroup
              \let\HOLOGO@temp\relax
              \global\let\HOLOGO@pdfliteral\HOLOGO@pdfliteral
            \fi%
          \HOLOGO@temp
        \fi
      }{}%
    \fi
    \ltx@IfUndefined{HOLOGO@pdfliteral}{%
      \@PackageWarningNoLine{hologo}{%
        Cannot find \string\pdfliteral
      }%
    }{}%
  \else
    \ifxetex
      \def\hologoDriver{xetex}%
    \else
      \ifvtex
        \def\hologoDriver{vtex}%
      \fi
    \fi
  \fi
}
%    \end{macrocode}
%    \end{macro}
%
%    \begin{macro}{\HOLOGO@WarningUnsupportedDriver}
%    \begin{macrocode}
\def\HOLOGO@WarningUnsupportedDriver#1{%
  \@PackageWarningNoLine{hologo}{%
    Logo `#1' needs driver specific macros,\MessageBreak
    but driver `\hologoDriver' is not supported.\MessageBreak
    Use a different driver or\MessageBreak
    load package `graphics' or `pgf'%
  }%
}
%    \end{macrocode}
%    \end{macro}
%
% \subsubsection{Reflect box macros}
%
%    Skip driver part if not needed.
%    \begin{macrocode}
\ltx@IfUndefined{reflectbox}{}{%
  \ltx@IfUndefined{rotatebox}{}{%
    \HOLOGO@AtEnd
  }%
}
\ltx@IfUndefined{pgftext}{}{%
  \HOLOGO@AtEnd
}
\ltx@IfUndefined{psscalebox}{}{%
  \HOLOGO@AtEnd
}
%    \end{macrocode}
%
%    \begin{macrocode}
\def\HOLOGO@temp{LaTeX2e}
\ifx\fmtname\HOLOGO@temp
  \RequirePackage{kvoptions}[2011/06/30]%
  \ProcessKeyvalOptions{HoLogoDriver}%
\fi
\HOLOGO@DriverSetup{}
%    \end{macrocode}
%
%    \begin{macro}{\HOLOGO@ReflectBox}
%    \begin{macrocode}
\def\HOLOGO@ReflectBox#1{%
  \begingroup
    \setbox\ltx@zero\hbox{\begingroup#1\endgroup}%
    \setbox\ltx@two\hbox{%
      \kern\wd\ltx@zero
      \csname HOLOGO@ScaleBox@\hologoDriver\endcsname{-1}{1}{%
        \hbox to 0pt{\copy\ltx@zero\hss}%
      }%
    }%
    \wd\ltx@two=\wd\ltx@zero
    \box\ltx@two
  \endgroup
}
%    \end{macrocode}
%    \end{macro}
%
%    \begin{macro}{\HOLOGO@PointReflectBox}
%    \begin{macrocode}
\def\HOLOGO@PointReflectBox#1{%
  \begingroup
    \setbox\ltx@zero\hbox{\begingroup#1\endgroup}%
    \setbox\ltx@two\hbox{%
      \kern\wd\ltx@zero
      \raise\ht\ltx@zero\hbox{%
        \csname HOLOGO@ScaleBox@\hologoDriver\endcsname{-1}{-1}{%
          \hbox to 0pt{\copy\ltx@zero\hss}%
        }%
      }%
    }%
    \wd\ltx@two=\wd\ltx@zero
    \box\ltx@two
  \endgroup
}
%    \end{macrocode}
%    \end{macro}
%
%    We must define all variants because of dynamic driver setup.
%    \begin{macrocode}
\def\HOLOGO@temp#1#2{#2}
%    \end{macrocode}
%
%    \begin{macro}{\HOLOGO@ScaleBox@pdftex}
%    \begin{macrocode}
\HOLOGO@temp{pdftex}{%
  \def\HOLOGO@ScaleBox@pdftex#1#2#3{%
    \HOLOGO@pdfliteral{%
      q #1 0 0 #2 0 0 cm%
    }%
    #3%
    \HOLOGO@pdfliteral{%
      Q%
    }%
  }%
}
%    \end{macrocode}
%    \end{macro}
%    \begin{macro}{\HOLOGO@ScaleBox@dvips}
%    \begin{macrocode}
\HOLOGO@temp{dvips}{%
  \def\HOLOGO@ScaleBox@dvips#1#2#3{%
    \special{ps:%
      gsave %
      currentpoint %
      currentpoint translate %
      #1 #2 scale %
      neg exch neg exch translate%
    }%
    #3%
    \special{ps:%
      currentpoint %
      grestore %
      moveto%
    }%
  }%
}
%    \end{macrocode}
%    \end{macro}
%    \begin{macro}{\HOLOGO@ScaleBox@dvipdfm}
%    \begin{macrocode}
\HOLOGO@temp{dvipdfm}{%
  \let\HOLOGO@ScaleBox@dvipdfm\HOLOGO@ScaleBox@dvips
}
%    \end{macrocode}
%    \end{macro}
%    Since \hologo{XeTeX} v0.6.
%    \begin{macro}{\HOLOGO@ScaleBox@xetex}
%    \begin{macrocode}
\HOLOGO@temp{xetex}{%
  \def\HOLOGO@ScaleBox@xetex#1#2#3{%
    \special{x:gsave}%
    \special{x:scale #1 #2}%
    #3%
    \special{x:grestore}%
  }%
}
%    \end{macrocode}
%    \end{macro}
%    \begin{macro}{\HOLOGO@ScaleBox@vtex}
%    \begin{macrocode}
\HOLOGO@temp{vtex}{%
  \def\HOLOGO@ScaleBox@vtex#1#2#3{%
    \special{r(#1,0,0,#2,0,0}%
    #3%
    \special{r)}%
  }%
}
%    \end{macrocode}
%    \end{macro}
%
%    \begin{macrocode}
\HOLOGO@AtEnd%
%</package>
%    \end{macrocode}
%
% \section{Test}
%
% \subsection{Catcode checks for loading}
%
%    \begin{macrocode}
%<*test1>
%    \end{macrocode}
%    \begin{macrocode}
\catcode`\{=1 %
\catcode`\}=2 %
\catcode`\#=6 %
\catcode`\@=11 %
\expandafter\ifx\csname count@\endcsname\relax
  \countdef\count@=255 %
\fi
\expandafter\ifx\csname @gobble\endcsname\relax
  \long\def\@gobble#1{}%
\fi
\expandafter\ifx\csname @firstofone\endcsname\relax
  \long\def\@firstofone#1{#1}%
\fi
\expandafter\ifx\csname loop\endcsname\relax
  \expandafter\@firstofone
\else
  \expandafter\@gobble
\fi
{%
  \def\loop#1\repeat{%
    \def\body{#1}%
    \iterate
  }%
  \def\iterate{%
    \body
      \let\next\iterate
    \else
      \let\next\relax
    \fi
    \next
  }%
  \let\repeat=\fi
}%
\def\RestoreCatcodes{}
\count@=0 %
\loop
  \edef\RestoreCatcodes{%
    \RestoreCatcodes
    \catcode\the\count@=\the\catcode\count@\relax
  }%
\ifnum\count@<255 %
  \advance\count@ 1 %
\repeat

\def\RangeCatcodeInvalid#1#2{%
  \count@=#1\relax
  \loop
    \catcode\count@=15 %
  \ifnum\count@<#2\relax
    \advance\count@ 1 %
  \repeat
}
\def\RangeCatcodeCheck#1#2#3{%
  \count@=#1\relax
  \loop
    \ifnum#3=\catcode\count@
    \else
      \errmessage{%
        Character \the\count@\space
        with wrong catcode \the\catcode\count@\space
        instead of \number#3%
      }%
    \fi
  \ifnum\count@<#2\relax
    \advance\count@ 1 %
  \repeat
}
\def\space{ }
\expandafter\ifx\csname LoadCommand\endcsname\relax
  \def\LoadCommand{\input hologo.sty\relax}%
\fi
\def\Test{%
  \RangeCatcodeInvalid{0}{47}%
  \RangeCatcodeInvalid{58}{64}%
  \RangeCatcodeInvalid{91}{96}%
  \RangeCatcodeInvalid{123}{255}%
  \catcode`\@=12 %
  \catcode`\\=0 %
  \catcode`\%=14 %
  \LoadCommand
  \RangeCatcodeCheck{0}{36}{15}%
  \RangeCatcodeCheck{37}{37}{14}%
  \RangeCatcodeCheck{38}{47}{15}%
  \RangeCatcodeCheck{48}{57}{12}%
  \RangeCatcodeCheck{58}{63}{15}%
  \RangeCatcodeCheck{64}{64}{12}%
  \RangeCatcodeCheck{65}{90}{11}%
  \RangeCatcodeCheck{91}{91}{15}%
  \RangeCatcodeCheck{92}{92}{0}%
  \RangeCatcodeCheck{93}{96}{15}%
  \RangeCatcodeCheck{97}{122}{11}%
  \RangeCatcodeCheck{123}{255}{15}%
  \RestoreCatcodes
}
\Test
\csname @@end\endcsname
\end
%    \end{macrocode}
%    \begin{macrocode}
%</test1>
%    \end{macrocode}
%
% \subsection{Spacefactor}
%
%    The space factor must be 1000 after a logo. If it is greater 1000
%    then the following space is a space after a sentence closing point.
%    If the space factor is smaller 1000 then an immediate following
%    dot is interpreted as abbreviation, not sentence closing point.
%
%    \begin{macrocode}
%<*test-spacefactor>
\NeedsTeXFormat{LaTeX2e}
\documentclass{article}
\usepackage{hologo}[2016/05/12]
\usepackage{kvsetkeys}
\usepackage{qstest}
\IncludeTests{*}
\LogTests{log}{*}{*}
\begin{document}
\begin{qstest}{spacefactor}{spacefactor}
\newcommand*{\Test}[1]{%
  \sbox0{%
    \hologo{#1}%
    \Expect*{1000 (#1)}*{\the\spacefactor\space(#1)}%
  }%
}%
\makeatletter
\def\TestList{}
\def\hologoEntry#1#2#3{%
  \edef\TestList{%
    \ifx\TestList\@empty
    \else
      \TestList,%
    \fi
    #1%
    \ifx\\#2\\%
    \else
      ={variant=#2}%
    \fi
  }%
}
\hologoList
\expandafter\kv@parse@normalized\expandafter{%
  \TestList
}{%
  \begingroup
    \let\@logo=\kv@key
    \ifx\kv@value\relax
    \else
      \expandafter\hologoLogoSetup\expandafter\@logo\expandafter{%
        \kv@value
      }%
    \fi
    \Test\@logo
  \endgroup
  \@gobbletwo
}
\end{qstest}
\end{document}
%</test-spacefactor>
%    \end{macrocode}
%
% \subsection{Complete list}
%
%    \begin{macrocode}
%<*test-list>
\NeedsTeXFormat{LaTeX2e}
\documentclass[12pt,a4paper]{article}
\usepackage{hologo}[2016/05/12]
\usepackage[T1]{fontenc}
\usepackage{lmodern}
\usepackage{parskip}
\usepackage[unicode]{hyperref}[2011/09/28]
\usepackage{bookmark}[2011/09/19]
\bookmarksetup{%
  numbered,%
  open,%
  openlevel=2,%
}
\renewcommand*{\contentsname}{List of logos}
\begin{document}
\tableofcontents
\def\TestFont#1#2#3#4#5#6{%
  \begingroup
    \usefont{#3}{#4}{#5}{#6}%
    \HologoVariant{#1}{#2}/\hologoVariant{#1}{#2}%
    \quad
    \begingroup\scriptsize\hologoVariant{#1}{#2}\endgroup
    \quad
  \endgroup
  (#3/#4/#5/#6)%
  \par
}
\makeatletter
\def\hologoEntry#1#2#3{%
  \section{%
    \HologoVariant{#1}{#2}/\hologoVariant{#1}{#2} %
    {[#1\ifx\\#2\\\else\space(#2)\fi]}% hash-ok
  }% braces around [] because of bug in tex4ht
  \begingroup
    \hypersetup{unicode=false}%
    \bookmark[%
      dest=\@currentHref,%
      rellevel=1,%
      keeplevel,%
    ]{%
      \HologoVariant{#1}{#2}/\hologoVariant{#1}{#2} %
      (PDFDocEncoding)%
    }%
  \endgroup
  \TestFont{#1}{#2}{OT1}{cmr}{m}{n}%
  \TestFont{#1}{#2}{OT1}{cmss}{m}{n}%
  \TestFont{#1}{#2}{OT1}{cmr}{b}{n}%
  \TestFont{#1}{#2}{OT1}{cmr}{m}{it}%
  \TestFont{#1}{#2}{OT1}{cmtt}{m}{n}%
  \TestFont{#1}{#2}{T1}{lmr}{m}{n}%
  \TestFont{#1}{#2}{T1}{lmss}{m}{n}%
  \TestFont{#1}{#2}{T1}{lmr}{b}{n}%
  \TestFont{#1}{#2}{T1}{lmr}{m}{it}%
  \TestFont{#1}{#2}{T1}{lmtt}{m}{n}%
  \TestFont{#1}{#2}{T1}{lmvtt}{m}{n}%
  \TestFont{#1}{#2}{T1}{qtm}{m}{n}%
  \TestFont{#1}{#2}{T1}{qhv}{m}{n}%
  \TestFont{#1}{#2}{T1}{qtm}{b}{n}%
  \TestFont{#1}{#2}{T1}{qtm}{m}{it}%
  \TestFont{#1}{#2}{T1}{qcr}{m}{n}%
  \newpage
}
\makeatother
\hologoList
\end{document}
%</test-list>
%    \end{macrocode}
%
% \section{Installation}
%
% \subsection{Download}
%
% \paragraph{Package.} This package is available on
% CTAN\footnote{\url{ftp://ftp.ctan.org/tex-archive/}}:
% \begin{description}
% \item[\CTAN{macros/latex/contrib/oberdiek/hologo.dtx}] The source file.
% \item[\CTAN{macros/latex/contrib/oberdiek/hologo.pdf}] Documentation.
% \end{description}
%
%
% \paragraph{Bundle.} All the packages of the bundle `oberdiek'
% are also available in a TDS compliant ZIP archive. There
% the packages are already unpacked and the documentation files
% are generated. The files and directories obey the TDS standard.
% \begin{description}
% \item[\CTAN{install/macros/latex/contrib/oberdiek.tds.zip}]
% \end{description}
% \emph{TDS} refers to the standard ``A Directory Structure
% for \TeX\ Files'' (\CTAN{tds/tds.pdf}). Directories
% with \xfile{texmf} in their name are usually organized this way.
%
% \subsection{Bundle installation}
%
% \paragraph{Unpacking.} Unpack the \xfile{oberdiek.tds.zip} in the
% TDS tree (also known as \xfile{texmf} tree) of your choice.
% Example (linux):
% \begin{quote}
%   |unzip oberdiek.tds.zip -d ~/texmf|
% \end{quote}
%
% \paragraph{Script installation.}
% Check the directory \xfile{TDS:scripts/oberdiek/} for
% scripts that need further installation steps.
% Package \xpackage{attachfile2} comes with the Perl script
% \xfile{pdfatfi.pl} that should be installed in such a way
% that it can be called as \texttt{pdfatfi}.
% Example (linux):
% \begin{quote}
%   |chmod +x scripts/oberdiek/pdfatfi.pl|\\
%   |cp scripts/oberdiek/pdfatfi.pl /usr/local/bin/|
% \end{quote}
%
% \subsection{Package installation}
%
% \paragraph{Unpacking.} The \xfile{.dtx} file is a self-extracting
% \docstrip\ archive. The files are extracted by running the
% \xfile{.dtx} through \plainTeX:
% \begin{quote}
%   \verb|tex hologo.dtx|
% \end{quote}
%
% \paragraph{TDS.} Now the different files must be moved into
% the different directories in your installation TDS tree
% (also known as \xfile{texmf} tree):
% \begin{quote}
% \def\t{^^A
% \begin{tabular}{@{}>{\ttfamily}l@{ $\rightarrow$ }>{\ttfamily}l@{}}
%   hologo.sty & tex/generic/oberdiek/hologo.sty\\
%   hologo.pdf & doc/latex/oberdiek/hologo.pdf\\
%   example/hologo-example.tex & doc/latex/oberdiek/example/hologo-example.tex\\
%   test/hologo-test1.tex & doc/latex/oberdiek/test/hologo-test1.tex\\
%   test/hologo-test-spacefactor.tex & doc/latex/oberdiek/test/hologo-test-spacefactor.tex\\
%   test/hologo-test-list.tex & doc/latex/oberdiek/test/hologo-test-list.tex\\
%   hologo.dtx & source/latex/oberdiek/hologo.dtx\\
% \end{tabular}^^A
% }^^A
% \sbox0{\t}^^A
% \ifdim\wd0>\linewidth
%   \begingroup
%     \advance\linewidth by\leftmargin
%     \advance\linewidth by\rightmargin
%   \edef\x{\endgroup
%     \def\noexpand\lw{\the\linewidth}^^A
%   }\x
%   \def\lwbox{^^A
%     \leavevmode
%     \hbox to \linewidth{^^A
%       \kern-\leftmargin\relax
%       \hss
%       \usebox0
%       \hss
%       \kern-\rightmargin\relax
%     }^^A
%   }^^A
%   \ifdim\wd0>\lw
%     \sbox0{\small\t}^^A
%     \ifdim\wd0>\linewidth
%       \ifdim\wd0>\lw
%         \sbox0{\footnotesize\t}^^A
%         \ifdim\wd0>\linewidth
%           \ifdim\wd0>\lw
%             \sbox0{\scriptsize\t}^^A
%             \ifdim\wd0>\linewidth
%               \ifdim\wd0>\lw
%                 \sbox0{\tiny\t}^^A
%                 \ifdim\wd0>\linewidth
%                   \lwbox
%                 \else
%                   \usebox0
%                 \fi
%               \else
%                 \lwbox
%               \fi
%             \else
%               \usebox0
%             \fi
%           \else
%             \lwbox
%           \fi
%         \else
%           \usebox0
%         \fi
%       \else
%         \lwbox
%       \fi
%     \else
%       \usebox0
%     \fi
%   \else
%     \lwbox
%   \fi
% \else
%   \usebox0
% \fi
% \end{quote}
% If you have a \xfile{docstrip.cfg} that configures and enables \docstrip's
% TDS installing feature, then some files can already be in the right
% place, see the documentation of \docstrip.
%
% \subsection{Refresh file name databases}
%
% If your \TeX~distribution
% (\teTeX, \mikTeX, \dots) relies on file name databases, you must refresh
% these. For example, \teTeX\ users run \verb|texhash| or
% \verb|mktexlsr|.
%
% \subsection{Some details for the interested}
%
% \paragraph{Attached source.}
%
% The PDF documentation on CTAN also includes the
% \xfile{.dtx} source file. It can be extracted by
% AcrobatReader 6 or higher. Another option is \textsf{pdftk},
% e.g. unpack the file into the current directory:
% \begin{quote}
%   \verb|pdftk hologo.pdf unpack_files output .|
% \end{quote}
%
% \paragraph{Unpacking with \LaTeX.}
% The \xfile{.dtx} chooses its action depending on the format:
% \begin{description}
% \item[\plainTeX:] Run \docstrip\ and extract the files.
% \item[\LaTeX:] Generate the documentation.
% \end{description}
% If you insist on using \LaTeX\ for \docstrip\ (really,
% \docstrip\ does not need \LaTeX), then inform the autodetect routine
% about your intention:
% \begin{quote}
%   \verb|latex \let\install=y% \iffalse meta-comment
%
% File: hologo.dtx
% Version: 2016/05/12 v1.11
% Info: A logo collection with bookmark support
%
% Copyright (C) 2010-2012 by
%    Heiko Oberdiek <heiko.oberdiek at googlemail.com>
%
% This work may be distributed and/or modified under the
% conditions of the LaTeX Project Public License, either
% version 1.3c of this license or (at your option) any later
% version. This version of this license is in
%    http://www.latex-project.org/lppl/lppl-1-3c.txt
% and the latest version of this license is in
%    http://www.latex-project.org/lppl.txt
% and version 1.3 or later is part of all distributions of
% LaTeX version 2005/12/01 or later.
%
% This work has the LPPL maintenance status "maintained".
%
% This Current Maintainer of this work is Heiko Oberdiek.
%
% The Base Interpreter refers to any `TeX-Format',
% because some files are installed in TDS:tex/generic//.
%
% This work consists of the main source file hologo.dtx
% and the derived files
%    hologo.sty, hologo.pdf, hologo.ins, hologo.drv, hologo-example.tex,
%    hologo-test1.tex, hologo-test-spacefactor.tex,
%    hologo-test-list.tex.
%
% Distribution:
%    CTAN:macros/latex/contrib/oberdiek/hologo.dtx
%    CTAN:macros/latex/contrib/oberdiek/hologo.pdf
%
% Unpacking:
%    (a) If hologo.ins is present:
%           tex hologo.ins
%    (b) Without hologo.ins:
%           tex hologo.dtx
%    (c) If you insist on using LaTeX
%           latex \let\install=y\input{hologo.dtx}
%        (quote the arguments according to the demands of your shell)
%
% Documentation:
%    (a) If hologo.drv is present:
%           latex hologo.drv
%    (b) Without hologo.drv:
%           latex hologo.dtx; ...
%    The class ltxdoc loads the configuration file ltxdoc.cfg
%    if available. Here you can specify further options, e.g.
%    use A4 as paper format:
%       \PassOptionsToClass{a4paper}{article}
%
%    Programm calls to get the documentation (example):
%       pdflatex hologo.dtx
%       makeindex -s gind.ist hologo.idx
%       pdflatex hologo.dtx
%       makeindex -s gind.ist hologo.idx
%       pdflatex hologo.dtx
%
% Installation:
%    TDS:tex/generic/oberdiek/hologo.sty
%    TDS:doc/latex/oberdiek/hologo.pdf
%    TDS:doc/latex/oberdiek/example/hologo-example.tex
%    TDS:doc/latex/oberdiek/test/hologo-test1.tex
%    TDS:doc/latex/oberdiek/test/hologo-test-spacefactor.tex
%    TDS:doc/latex/oberdiek/test/hologo-test-list.tex
%    TDS:source/latex/oberdiek/hologo.dtx
%
%<*ignore>
\begingroup
  \catcode123=1 %
  \catcode125=2 %
  \def\x{LaTeX2e}%
\expandafter\endgroup
\ifcase 0\ifx\install y1\fi\expandafter
         \ifx\csname processbatchFile\endcsname\relax\else1\fi
         \ifx\fmtname\x\else 1\fi\relax
\else\csname fi\endcsname
%</ignore>
%<*install>
\input docstrip.tex
\Msg{************************************************************************}
\Msg{* Installation}
\Msg{* Package: hologo 2016/05/12 v1.11 A logo collection with bookmark support (HO)}
\Msg{************************************************************************}

\keepsilent
\askforoverwritefalse

\let\MetaPrefix\relax
\preamble

This is a generated file.

Project: hologo
Version: 2016/05/12 v1.11

Copyright (C) 2010-2012 by
   Heiko Oberdiek <heiko.oberdiek at googlemail.com>

This work may be distributed and/or modified under the
conditions of the LaTeX Project Public License, either
version 1.3c of this license or (at your option) any later
version. This version of this license is in
   http://www.latex-project.org/lppl/lppl-1-3c.txt
and the latest version of this license is in
   http://www.latex-project.org/lppl.txt
and version 1.3 or later is part of all distributions of
LaTeX version 2005/12/01 or later.

This work has the LPPL maintenance status "maintained".

This Current Maintainer of this work is Heiko Oberdiek.

The Base Interpreter refers to any `TeX-Format',
because some files are installed in TDS:tex/generic//.

This work consists of the main source file hologo.dtx
and the derived files
   hologo.sty, hologo.pdf, hologo.ins, hologo.drv, hologo-example.tex,
   hologo-test1.tex, hologo-test-spacefactor.tex,
   hologo-test-list.tex.

\endpreamble
\let\MetaPrefix\DoubleperCent

\generate{%
  \file{hologo.ins}{\from{hologo.dtx}{install}}%
  \file{hologo.drv}{\from{hologo.dtx}{driver}}%
  \usedir{tex/generic/oberdiek}%
  \file{hologo.sty}{\from{hologo.dtx}{package}}%
  \usedir{doc/latex/oberdiek/example}%
  \file{hologo-example.tex}{\from{hologo.dtx}{example}}%
  \usedir{doc/latex/oberdiek/test}%
  \file{hologo-test1.tex}{\from{hologo.dtx}{test1}}%
  \file{hologo-test-spacefactor.tex}{\from{hologo.dtx}{test-spacefactor}}%
  \file{hologo-test-list.tex}{\from{hologo.dtx}{test-list}}%
  \nopreamble
  \nopostamble
  \usedir{source/latex/oberdiek/catalogue}%
  \file{hologo.xml}{\from{hologo.dtx}{catalogue}}%
}

\catcode32=13\relax% active space
\let =\space%
\Msg{************************************************************************}
\Msg{*}
\Msg{* To finish the installation you have to move the following}
\Msg{* file into a directory searched by TeX:}
\Msg{*}
\Msg{*     hologo.sty}
\Msg{*}
\Msg{* To produce the documentation run the file `hologo.drv'}
\Msg{* through LaTeX.}
\Msg{*}
\Msg{* Happy TeXing!}
\Msg{*}
\Msg{************************************************************************}

\endbatchfile
%</install>
%<*ignore>
\fi
%</ignore>
%<*driver>
\NeedsTeXFormat{LaTeX2e}
\ProvidesFile{hologo.drv}%
  [2016/05/12 v1.11 A logo collection with bookmark support (HO)]%
\documentclass{ltxdoc}
\usepackage{holtxdoc}[2011/11/22]
\usepackage{hologo}[2016/05/12]
\usepackage{longtable}
\usepackage{array}
\usepackage{paralist}
%\usepackage[T1]{fontenc}
%\usepackage{lmodern}
\begin{document}
  \DocInput{hologo.dtx}%
\end{document}
%</driver>
% \fi
%
%
% \CharacterTable
%  {Upper-case    \A\B\C\D\E\F\G\H\I\J\K\L\M\N\O\P\Q\R\S\T\U\V\W\X\Y\Z
%   Lower-case    \a\b\c\d\e\f\g\h\i\j\k\l\m\n\o\p\q\r\s\t\u\v\w\x\y\z
%   Digits        \0\1\2\3\4\5\6\7\8\9
%   Exclamation   \!     Double quote  \"     Hash (number) \#
%   Dollar        \$     Percent       \%     Ampersand     \&
%   Acute accent  \'     Left paren    \(     Right paren   \)
%   Asterisk      \*     Plus          \+     Comma         \,
%   Minus         \-     Point         \.     Solidus       \/
%   Colon         \:     Semicolon     \;     Less than     \<
%   Equals        \=     Greater than  \>     Question mark \?
%   Commercial at \@     Left bracket  \[     Backslash     \\
%   Right bracket \]     Circumflex    \^     Underscore    \_
%   Grave accent  \`     Left brace    \{     Vertical bar  \|
%   Right brace   \}     Tilde         \~}
%
% \GetFileInfo{hologo.drv}
%
% \title{The \xpackage{hologo} package}
% \date{2016/05/12 v1.11}
% \author{Heiko Oberdiek\\\xemail{heiko.oberdiek at googlemail.com}}
%
% \maketitle
%
% \begin{abstract}
% This package starts a collection of logos with support for bookmarks
% strings.
% \end{abstract}
%
% \tableofcontents
%
% \section{Documentation}
%
% \subsection{Logo macros}
%
% \begin{declcs}{hologo} \M{name}
% \end{declcs}
% Macro \cs{hologo} sets the logo with name \meta{name}.
% The following table shows the supported names.
%
% \begingroup
%   \def\hologoEntry#1#2#3{^^A
%     #1&#2&\hologoLogoSetup{#1}{variant=#2}\hologo{#1}&#3\tabularnewline
%   }
%   \begin{longtable}{>{\ttfamily}l>{\ttfamily}lll}
%     \rmfamily\bfseries{name} & \rmfamily\bfseries variant
%     & \bfseries logo & \bfseries since\\
%     \hline
%     \endhead
%     \hologoList
%   \end{longtable}
% \endgroup
%
% \begin{declcs}{Hologo} \M{name}
% \end{declcs}
% Macro \cs{Hologo} starts the logo \meta{name} with an uppercase
% letter. As an exception small greek letters are not converted
% to uppercase. Examples, see \hologo{eTeX} and \hologo{ExTeX}.
%
% \subsection{Setup macros}
%
% The package does not support package options, but the following
% setup macros can be used to set options.
%
% \begin{declcs}{hologoSetup} \M{key value list}
% \end{declcs}
% Macro \cs{hologoSetup} sets global options.
%
% \begin{declcs}{hologoLogoSetup} \M{logo} \M{key value list}
% \end{declcs}
% Some options can also be used to configure a logo.
% These settings take precedence over global option settings.
%
% \subsection{Options}\label{sec:options}
%
% There are boolean and string options:
% \begin{description}
% \item[Boolean option:]
% It takes |true| or |false|
% as value. If the value is omitted, then |true| is used.
% \item[String option:]
% A value must be given as string. (But the string might be empty.)
% \end{description}
% The following options can be used both in \cs{hologoSetup}
% and \cs{hologoLogoSetup}:
% \begin{description}
% \def\entry#1{\item[\xoption{#1}:]}
% \entry{break}
%   enables or disables line breaks inside the logo. This setting is
%   refined by options \xoption{hyphenbreak}, \xoption{spacebreak}
%   or \xoption{discretionarybreak}.
%   Default is |false|.
% \entry{hyphenbreak}
%   enables or disables the line break right after the hyphen character.
% \entry{spacebreak}
%   enables or disables line breaks at space characters.
% \entry{discretionarybreak}
%   enables or disables line breaks at hyphenation points
%   (inserted by \cs{-}).
% \end{description}
% Macro \cs{hologoLogoSetup} also knows:
% \begin{description}
% \item[\xoption{variant}:]
%   This is a string option. It specifies a variant of a logo that
%   must exist. An empty string selects the package default variant.
% \end{description}
% Example:
% \begin{quote}
%   |\hologoSetup{break=false}|\\
%   |\hologoLogoSetup{plainTeX}{variant=hyphen,hyphenbreak}|\\
%   Then ``plain-\TeX'' contains one break point after the hyphen.
% \end{quote}
%
% \subsection{Driver options}
%
% Sometimes graphical operations are needed to construct some
% glyphs (e.g.\ \hologo{XeTeX}). If package \xpackage{graphics}
% or package \xpackage{pgf} are found, then the macros are taken
% from there. Otherwise the packge defines its own operations
% and therefore needs the driver information. Many drivers are
% detected automatically (\hologo{pdfTeX}/\hologo{LuaTeX}
% in PDF mode, \hologo{XeTeX}, \hologo{VTeX}). These have precedence
% over a driver option. The driver can be given as package option
% or using \cs{hologoDriverSetup}.
% The following list contains the recognized driver options:
% \begin{itemize}
% \item \xoption{pdftex}, \xoption{luatex}
% \item \xoption{dvipdfm}, \xoption{dvipdfmx}
% \item \xoption{dvips}, \xoption{dvipsone}, \xoption{xdvi}
% \item \xoption{xetex}
% \item \xoption{vtex}
% \end{itemize}
% The left driver of a line is the driver name that is used internally.
% The following names are aliases for drivers that use the
% same method. Therefore the entry in the \xext{log} file for
% the used driver prints the internally used driver name.
% \begin{description}
% \item[\xoption{driverfallback}:]
%   This option expects a driver that is used,
%   if the driver could not be detected automatically.
% \end{description}
%
% \begin{declcs}{hologoDriverSetup} \M{driver option}
% \end{declcs}
% The driver can also be configured after package loading
% using \cs{hologoDriverSetup}, also the way for \hologo{plainTeX}
% to setup the driver.
%
% \subsection{Font setup}
%
% Some logos require a special font, but should also be usable by
% \hologo{plainTeX}. Therefore the package provides some ways
% to influence the font settings. The options below
% take font settings as values. Both font commands
% such as \cs{sffamily} and macros that take one argument
% like \cs{textsf} can be used.
%
% \begin{declcs}{hologoFontSetup} \M{key value list}
% \end{declcs}
% Macro \cs{hologoFontSetup} sets the fonts for all logos.
% Supported keys:
% \begin{description}
% \def\entry#1{\item[\xoption{#1}:]}
% \entry{general}
%   This font is used for all logos. The default is empty.
%   That means no special font is used.
% \entry{bibsf}
%   This font is used for
%   {\hologoLogoSetup{BibTeX}{variant=sf}\hologo{BibTeX}}
%   with variant \xoption{sf}.
% \entry{rm}
%   This font is a serif font. It is used for \hologo{ExTeX}.
% \entry{sc}
%   This font specifies a small caps font. It is used for
%   {\hologoLogoSetup{BibTeX}{variant=sc}\hologo{BibTeX}}
%   with variant \xoption{sc}.
% \entry{sf}
%   This font specifies a sans serif font. The default
%   is \cs{sffamily}, then \cs{sf} is tried. Otherwise
%   a warning is given. It is used by \hologo{KOMAScript}.
% \entry{sy}
%   This is the font for math symbols (e.g. cmsy).
%   It is used by \hologo{AmS}, \hologo{NTS}, \hologo{ExTeX}.
% \entry{logo}
%   \hologo{METAFONT} and \hologo{METAPOST} are using that font.
%   In \hologo{LaTeX} \cs{logofamily} is used and
%   the definitions of package \xpackage{mflogo} are used
%   if the package is not loaded.
%   Otherwise the \cs{tenlogo} is used and defined
%   if it does not already exists.
% \end{description}
%
% \begin{declcs}{hologoLogoFontSetup} \M{logo} \M{key value list}
% \end{declcs}
% Fonts can also be set for a logo or logo component separately,
% see the following list.
% The keys are the same as for \cs{hologoFontSetup}.
%
% \begin{longtable}{>{\ttfamily}l>{\sffamily}ll}
%   \meta{logo} & keys & result\\
%   \hline
%   \endhead
%   BibTeX & bibsf & {\hologoLogoSetup{BibTeX}{variant=sf}\hologo{BibTeX}}\\[.5ex]
%   BibTeX & sc & {\hologoLogoSetup{BibTeX}{variant=sc}\hologo{BibTeX}}\\[.5ex]
%   ExTeX & rm & \hologo{ExTeX}\\
%   SliTeX & rm & \hologo{SliTeX}\\[.5ex]
%   AmS & sy & \hologo{AmS}\\
%   ExTeX & sy & \hologo{ExTeX}\\
%   NTS & sy & \hologo{NTS}\\[.5ex]
%   KOMAScript & sf & \hologo{KOMAScript}\\[.5ex]
%   METAFONT & logo & \hologo{METAFONT}\\
%   METAPOST & logo & \hologo{METAPOST}\\[.5ex]
%   SliTeX & sc \hologo{SliTeX}
% \end{longtable}
%
% \subsubsection{Font order}
%
% For all logos the font \xoption{general} is applied first.
% Example:
%\begin{quote}
%|\hologoFontSetup{general=\color{red}}|
%\end{quote}
% will print red logos.
% Then if the font uses a special font \xoption{sf}, for example,
% the font is applied that is setup by \cs{hologoLogoFontSetup}.
% If this font is not setup, then the common font setup
% by \cs{hologoFontSetup} is used. Otherwise a warning is given,
% that there is no font configured.
%
% \subsection{Additional user macros}
%
% Usually a variant of a logo is configured by using
% \cs{hologoLogoSetup}, because it is bad style to mix
% different variants of the same logo in the same text.
% There the following macros are a convenience for testing.
%
% \begin{declcs}{hologoVariant} \M{name} \M{variant}\\
%   \cs{HologoVariant} \M{name} \M{variant}
% \end{declcs}
% Logo \meta{name} is set using \meta{variant} that specifies
% explicitely which variant of the macro is used. If the argument
% is empty, then the default form of the logo is used
% (configurable by \cs{hologoLogoSetup}).
%
% \cs{HologoVariant} is used if the logo is set in a context
% that needs an uppercase first letter (beginning of a sentence, \dots).
%
% \begin{declcs}{hologoList}\\
%   \cs{hologoEntry} \M{logo} \M{variant} \M{since}
% \end{declcs}
% Macro \cs{hologoList} contains all logos that are provided
% by the package including variants. The list consists of calls
% of \cs{hologoEntry} with three arguments starting with the
% logo name \meta{logo} and its variant \meta{variant}. An empty
% variant means the current default. Argument \meta{since} specifies
% with version of the package \xpackage{hologo} is needed to get
% the logo. If the logo is fixed, then the date gets updated.
% Therefore the date \meta{since} is not exactly the date of
% the first introduction, but rather the date of the latest fix.
%
% Before \cs{hologoList} can be used, macro \cs{hologoEntry} needs
% a definition. The example file in section \ref{sec:example}
% shows applications of \cs{hologoList}.
%
% \subsection{Supported contexts}
%
% Macros \cs{hologo} and friends support special contexts:
% \begin{itemize}
% \item \hologo{LaTeX}'s protection mechanism.
% \item Bookmarks of package \xpackage{hyperref}.
% \item Package \xpackage{tex4ht}.
% \item The macros can be used inside \cs{csname} constructs,
%   if \cs{ifincsname} is available (\hologo{pdfTeX}, \hologo{XeTeX},
%   \hologo{LuaTeX}).
% \end{itemize}
%
% \subsection{Example}
% \label{sec:example}
%
% The following example prints the logos in different fonts.
%    \begin{macrocode}
%<*example>
%<<verbatim
\NeedsTeXFormat{LaTeX2e}
\documentclass[a4paper]{article}
\usepackage[
  hmargin=20mm,
  vmargin=20mm,
]{geometry}
\pagestyle{empty}
\usepackage{hologo}[2016/05/12]
\usepackage{longtable}
\usepackage{array}
\setlength{\extrarowheight}{2pt}
\usepackage[T1]{fontenc}
\usepackage{lmodern}
\usepackage{pdflscape}
\usepackage[
  pdfencoding=auto,
]{hyperref}
\hypersetup{
  pdfauthor={Heiko Oberdiek},
  pdftitle={Example for package `hologo'},
  pdfsubject={Logos with fonts lmr, lmss, qtm, qpl, qhv},
}
\usepackage{bookmark}

% Print the logo list on the console

\begingroup
  \typeout{}%
  \typeout{*** Begin of logo list ***}%
  \newcommand*{\hologoEntry}[3]{%
    \typeout{#1 \ifx\\#2\\\else(#2) \fi[#3]}%
  }%
  \hologoList
  \typeout{*** End of logo list ***}%
  \typeout{}%
\endgroup

\begin{document}
\begin{landscape}

  \section{Example file for package `hologo'}

  % Table for font names

  \begin{longtable}{>{\bfseries}ll}
    \textbf{font} & \textbf{Font name}\\
    \hline
    lmr & Latin Modern Roman\\
    lmss & Latin Modern Sans\\
    qtm & \TeX\ Gyre Termes\\
    qhv & \TeX\ Gyre Heros\\
    qpl & \TeX\ Gyre Pagella\\
  \end{longtable}

  % Logo list with logos in different fonts

  \begingroup
    \newcommand*{\SetVariant}[2]{%
      \ifx\\#2\\%
      \else
        \hologoLogoSetup{#1}{variant=#2}%
      \fi
    }%
    \newcommand*{\hologoEntry}[3]{%
      \SetVariant{#1}{#2}%
      \raisebox{1em}[0pt][0pt]{\hypertarget{#1@#2}{}}%
      \bookmark[%
        dest={#1@#2},%
      ]{%
        #1\ifx\\#2\\\else\space(#2)\fi: \Hologo{#1}, \hologo{#1} %
        [Unicode]%
      }%
      \hypersetup{unicode=false}%
      \bookmark[%
        dest={#1@#2},%
      ]{%
        #1\ifx\\#2\\\else\space(#2)\fi: \Hologo{#1}, \hologo{#1} %
        [PDFDocEncoding]%
      }%
      \texttt{#1}%
      &%
      \texttt{#2}%
      &%
      \Hologo{#1}%
      &%
      \SetVariant{#1}{#2}%
      \hologo{#1}%
      &%
      \SetVariant{#1}{#2}%
      \fontfamily{qtm}\selectfont
      \hologo{#1}%
      &%
      \SetVariant{#1}{#2}%
      \fontfamily{qpl}\selectfont
      \hologo{#1}%
      &%
      \SetVariant{#1}{#2}%
      \textsf{\hologo{#1}}%
      &%
      \SetVariant{#1}{#2}%
      \fontfamily{qhv}\selectfont
      \hologo{#1}%
      \tabularnewline
    }%
    \begin{longtable}{llllllll}%
      \textbf{\textit{logo}} & \textbf{\textit{variant}} &
      \texttt{\string\Hologo} &
      \textbf{lmr} & \textbf{qtm} & \textbf{qpl} &
      \textbf{lmss} & \textbf{qhv}
      \tabularnewline
      \hline
      \endhead
      \hologoList
    \end{longtable}%
  \endgroup

\end{landscape}
\end{document}
%verbatim
%</example>
%    \end{macrocode}
%
% \StopEventually{
% }
%
% \section{Implementation}
%    \begin{macrocode}
%<*package>
%    \end{macrocode}
%    Reload check, especially if the package is not used with \LaTeX.
%    \begin{macrocode}
\begingroup\catcode61\catcode48\catcode32=10\relax%
  \catcode13=5 % ^^M
  \endlinechar=13 %
  \catcode35=6 % #
  \catcode39=12 % '
  \catcode44=12 % ,
  \catcode45=12 % -
  \catcode46=12 % .
  \catcode58=12 % :
  \catcode64=11 % @
  \catcode123=1 % {
  \catcode125=2 % }
  \expandafter\let\expandafter\x\csname ver@hologo.sty\endcsname
  \ifx\x\relax % plain-TeX, first loading
  \else
    \def\empty{}%
    \ifx\x\empty % LaTeX, first loading,
      % variable is initialized, but \ProvidesPackage not yet seen
    \else
      \expandafter\ifx\csname PackageInfo\endcsname\relax
        \def\x#1#2{%
          \immediate\write-1{Package #1 Info: #2.}%
        }%
      \else
        \def\x#1#2{\PackageInfo{#1}{#2, stopped}}%
      \fi
      \x{hologo}{The package is already loaded}%
      \aftergroup\endinput
    \fi
  \fi
\endgroup%
%    \end{macrocode}
%    Package identification:
%    \begin{macrocode}
\begingroup\catcode61\catcode48\catcode32=10\relax%
  \catcode13=5 % ^^M
  \endlinechar=13 %
  \catcode35=6 % #
  \catcode39=12 % '
  \catcode40=12 % (
  \catcode41=12 % )
  \catcode44=12 % ,
  \catcode45=12 % -
  \catcode46=12 % .
  \catcode47=12 % /
  \catcode58=12 % :
  \catcode64=11 % @
  \catcode91=12 % [
  \catcode93=12 % ]
  \catcode123=1 % {
  \catcode125=2 % }
  \expandafter\ifx\csname ProvidesPackage\endcsname\relax
    \def\x#1#2#3[#4]{\endgroup
      \immediate\write-1{Package: #3 #4}%
      \xdef#1{#4}%
    }%
  \else
    \def\x#1#2[#3]{\endgroup
      #2[{#3}]%
      \ifx#1\@undefined
        \xdef#1{#3}%
      \fi
      \ifx#1\relax
        \xdef#1{#3}%
      \fi
    }%
  \fi
\expandafter\x\csname ver@hologo.sty\endcsname
\ProvidesPackage{hologo}%
  [2016/05/12 v1.11 A logo collection with bookmark support (HO)]%
%    \end{macrocode}
%
%    \begin{macrocode}
\begingroup\catcode61\catcode48\catcode32=10\relax%
  \catcode13=5 % ^^M
  \endlinechar=13 %
  \catcode123=1 % {
  \catcode125=2 % }
  \catcode64=11 % @
  \def\x{\endgroup
    \expandafter\edef\csname HOLOGO@AtEnd\endcsname{%
      \endlinechar=\the\endlinechar\relax
      \catcode13=\the\catcode13\relax
      \catcode32=\the\catcode32\relax
      \catcode35=\the\catcode35\relax
      \catcode61=\the\catcode61\relax
      \catcode64=\the\catcode64\relax
      \catcode123=\the\catcode123\relax
      \catcode125=\the\catcode125\relax
    }%
  }%
\x\catcode61\catcode48\catcode32=10\relax%
\catcode13=5 % ^^M
\endlinechar=13 %
\catcode35=6 % #
\catcode64=11 % @
\catcode123=1 % {
\catcode125=2 % }
\def\TMP@EnsureCode#1#2{%
  \edef\HOLOGO@AtEnd{%
    \HOLOGO@AtEnd
    \catcode#1=\the\catcode#1\relax
  }%
  \catcode#1=#2\relax
}
\TMP@EnsureCode{10}{12}% ^^J
\TMP@EnsureCode{33}{12}% !
\TMP@EnsureCode{34}{12}% "
\TMP@EnsureCode{36}{3}% $
\TMP@EnsureCode{38}{4}% &
\TMP@EnsureCode{39}{12}% '
\TMP@EnsureCode{40}{12}% (
\TMP@EnsureCode{41}{12}% )
\TMP@EnsureCode{42}{12}% *
\TMP@EnsureCode{43}{12}% +
\TMP@EnsureCode{44}{12}% ,
\TMP@EnsureCode{45}{12}% -
\TMP@EnsureCode{46}{12}% .
\TMP@EnsureCode{47}{12}% /
\TMP@EnsureCode{58}{12}% :
\TMP@EnsureCode{59}{12}% ;
\TMP@EnsureCode{60}{12}% <
\TMP@EnsureCode{62}{12}% >
\TMP@EnsureCode{63}{12}% ?
\TMP@EnsureCode{91}{12}% [
\TMP@EnsureCode{93}{12}% ]
\TMP@EnsureCode{94}{7}% ^ (superscript)
\TMP@EnsureCode{95}{8}% _ (subscript)
\TMP@EnsureCode{96}{12}% `
\TMP@EnsureCode{124}{12}% |
\edef\HOLOGO@AtEnd{%
  \HOLOGO@AtEnd
  \escapechar\the\escapechar\relax
  \noexpand\endinput
}
\escapechar=92 %
%    \end{macrocode}
%
% \subsection{Logo list}
%
%    \begin{macro}{\hologoList}
%    \begin{macrocode}
\def\hologoList{%
  \hologoEntry{(La)TeX}{}{2011/10/01}%
  \hologoEntry{AmSLaTeX}{}{2010/04/16}%
  \hologoEntry{AmSTeX}{}{2010/04/16}%
  \hologoEntry{biber}{}{2011/10/01}%
  \hologoEntry{BibTeX}{}{2011/10/01}%
  \hologoEntry{BibTeX}{sf}{2011/10/01}%
  \hologoEntry{BibTeX}{sc}{2011/10/01}%
  \hologoEntry{BibTeX8}{}{2011/11/22}%
  \hologoEntry{ConTeXt}{}{2011/03/25}%
  \hologoEntry{ConTeXt}{narrow}{2011/03/25}%
  \hologoEntry{ConTeXt}{simple}{2011/03/25}%
  \hologoEntry{emTeX}{}{2010/04/26}%
  \hologoEntry{eTeX}{}{2010/04/08}%
  \hologoEntry{ExTeX}{}{2011/10/01}%
  \hologoEntry{HanTheThanh}{}{2011/11/29}%
  \hologoEntry{iniTeX}{}{2011/10/01}%
  \hologoEntry{KOMAScript}{}{2011/10/01}%
  \hologoEntry{La}{}{2010/05/08}%
  \hologoEntry{LaTeX}{}{2010/04/08}%
  \hologoEntry{LaTeX2e}{}{2010/04/08}%
  \hologoEntry{LaTeX3}{}{2010/04/24}%
  \hologoEntry{LaTeXe}{}{2010/04/08}%
  \hologoEntry{LaTeXML}{}{2011/11/22}%
  \hologoEntry{LaTeXTeX}{}{2011/10/01}%
  \hologoEntry{LuaLaTeX}{}{2010/04/08}%
  \hologoEntry{LuaTeX}{}{2010/04/08}%
  \hologoEntry{LyX}{}{2011/10/01}%
  \hologoEntry{METAFONT}{}{2011/10/01}%
  \hologoEntry{MetaFun}{}{2011/10/01}%
  \hologoEntry{METAPOST}{}{2011/10/01}%
  \hologoEntry{MetaPost}{}{2011/10/01}%
  \hologoEntry{MiKTeX}{}{2011/10/01}%
  \hologoEntry{NTS}{}{2011/10/01}%
  \hologoEntry{OzMF}{}{2011/10/01}%
  \hologoEntry{OzMP}{}{2011/10/01}%
  \hologoEntry{OzTeX}{}{2011/10/01}%
  \hologoEntry{OzTtH}{}{2011/10/01}%
  \hologoEntry{PCTeX}{}{2011/10/01}%
  \hologoEntry{pdfTeX}{}{2011/10/01}%
  \hologoEntry{pdfLaTeX}{}{2011/10/01}%
  \hologoEntry{PiC}{}{2011/10/01}%
  \hologoEntry{PiCTeX}{}{2011/10/01}%
  \hologoEntry{plainTeX}{}{2010/04/08}%
  \hologoEntry{plainTeX}{space}{2010/04/16}%
  \hologoEntry{plainTeX}{hyphen}{2010/04/16}%
  \hologoEntry{plainTeX}{runtogether}{2010/04/16}%
  \hologoEntry{SageTeX}{}{2011/11/22}%
  \hologoEntry{SLiTeX}{}{2011/10/01}%
  \hologoEntry{SLiTeX}{lift}{2011/10/01}%
  \hologoEntry{SLiTeX}{narrow}{2011/10/01}%
  \hologoEntry{SLiTeX}{simple}{2011/10/01}%
  \hologoEntry{SliTeX}{}{2011/10/01}%
  \hologoEntry{SliTeX}{narrow}{2011/10/01}%
  \hologoEntry{SliTeX}{simple}{2011/10/01}%
  \hologoEntry{SliTeX}{lift}{2011/10/01}%
  \hologoEntry{teTeX}{}{2011/10/01}%
  \hologoEntry{TeX}{}{2010/04/08}%
  \hologoEntry{TeX4ht}{}{2011/11/22}%
  \hologoEntry{TTH}{}{2011/11/22}%
  \hologoEntry{virTeX}{}{2011/10/01}%
  \hologoEntry{VTeX}{}{2010/04/24}%
  \hologoEntry{Xe}{}{2010/04/08}%
  \hologoEntry{XeLaTeX}{}{2010/04/08}%
  \hologoEntry{XeTeX}{}{2010/04/08}%
}
%    \end{macrocode}
%    \end{macro}
%
% \subsection{Load resources}
%
%    \begin{macrocode}
\begingroup\expandafter\expandafter\expandafter\endgroup
\expandafter\ifx\csname RequirePackage\endcsname\relax
  \def\TMP@RequirePackage#1[#2]{%
    \begingroup\expandafter\expandafter\expandafter\endgroup
    \expandafter\ifx\csname ver@#1.sty\endcsname\relax
      \input #1.sty\relax
    \fi
  }%
  \TMP@RequirePackage{ltxcmds}[2011/02/04]%
  \TMP@RequirePackage{infwarerr}[2010/04/08]%
  \TMP@RequirePackage{kvsetkeys}[2010/03/01]%
  \TMP@RequirePackage{kvdefinekeys}[2010/03/01]%
  \TMP@RequirePackage{pdftexcmds}[2010/04/01]%
  \TMP@RequirePackage{ifpdf}[2010/01/28]%
  \TMP@RequirePackage{ifluatex}[2010/03/01]%
  \ltx@IfUndefined{newif}{%
    \expandafter\let\csname newif\endcsname\ltx@newif
  }{}%
  \TMP@RequirePackage{ifxetex}[2009/01/23]%
  \TMP@RequirePackage{ifvtex}[2010/03/01]%
\else
  \RequirePackage{ltxcmds}[2011/02/04]%
  \RequirePackage{infwarerr}[2010/04/08]%
  \RequirePackage{kvsetkeys}[2010/03/01]%
  \RequirePackage{kvdefinekeys}[2010/03/01]%
  \RequirePackage{pdftexcmds}[2010/04/01]%
  \RequirePackage{ifpdf}[2010/01/28]%
  \RequirePackage{ifluatex}[2010/03/01]%
  \RequirePackage{ifxetex}[2009/01/23]%
  \RequirePackage{ifvtex}[2010/03/01]%
\fi
%    \end{macrocode}
%
%    \begin{macro}{\HOLOGO@IfDefined}
%    \begin{macrocode}
\def\HOLOGO@IfExists#1{%
  \ifx\@undefined#1%
    \expandafter\ltx@secondoftwo
  \else
    \ifx\relax#1%
      \expandafter\ltx@secondoftwo
    \else
      \expandafter\expandafter\expandafter\ltx@firstoftwo
    \fi
  \fi
}
%    \end{macrocode}
%    \end{macro}
%
% \subsection{Setup macros}
%
%    \begin{macro}{\hologoSetup}
%    \begin{macrocode}
\def\hologoSetup{%
  \let\HOLOGO@name\relax
  \HOLOGO@Setup
}
%    \end{macrocode}
%    \end{macro}
%
%    \begin{macro}{\hologoLogoSetup}
%    \begin{macrocode}
\def\hologoLogoSetup#1{%
  \edef\HOLOGO@name{#1}%
  \ltx@IfUndefined{HoLogo@\HOLOGO@name}{%
    \@PackageError{hologo}{%
      Unknown logo `\HOLOGO@name'%
    }\@ehc
    \ltx@gobble
  }{%
    \HOLOGO@Setup
  }%
}
%    \end{macrocode}
%    \end{macro}
%
%    \begin{macro}{\HOLOGO@Setup}
%    \begin{macrocode}
\def\HOLOGO@Setup{%
  \kvsetkeys{HoLogo}%
}
%    \end{macrocode}
%    \end{macro}
%
% \subsection{Options}
%
%    \begin{macro}{\HOLOGO@DeclareBoolOption}
%    \begin{macrocode}
\def\HOLOGO@DeclareBoolOption#1{%
  \expandafter\chardef\csname HOLOGOOPT@#1\endcsname\ltx@zero
  \kv@define@key{HoLogo}{#1}[true]{%
    \def\HOLOGO@temp{##1}%
    \ifx\HOLOGO@temp\HOLOGO@true
      \ifx\HOLOGO@name\relax
        \expandafter\chardef\csname HOLOGOOPT@#1\endcsname=\ltx@one
      \else
        \expandafter\chardef\csname
        HoLogoOpt@#1@\HOLOGO@name\endcsname\ltx@one
      \fi
      \HOLOGO@SetBreakAll{#1}%
    \else
      \ifx\HOLOGO@temp\HOLOGO@false
        \ifx\HOLOGO@name\relax
          \expandafter\chardef\csname HOLOGOOPT@#1\endcsname=\ltx@zero
        \else
          \expandafter\chardef\csname
          HoLogoOpt@#1@\HOLOGO@name\endcsname=\ltx@zero
        \fi
        \HOLOGO@SetBreakAll{#1}%
      \else
        \@PackageError{hologo}{%
          Unknown value `##1' for boolean option `#1'.\MessageBreak
          Known values are `true' and `false'%
        }\@ehc
      \fi
    \fi
  }%
}
%    \end{macrocode}
%    \end{macro}
%
%    \begin{macro}{\HOLOGO@SetBreakAll}
%    \begin{macrocode}
\def\HOLOGO@SetBreakAll#1{%
  \def\HOLOGO@temp{#1}%
  \ifx\HOLOGO@temp\HOLOGO@break
    \ifx\HOLOGO@name\relax
      \chardef\HOLOGOOPT@hyphenbreak=\HOLOGOOPT@break
      \chardef\HOLOGOOPT@spacebreak=\HOLOGOOPT@break
      \chardef\HOLOGOOPT@discretionarybreak=\HOLOGOOPT@break
    \else
      \expandafter\chardef
         \csname HoLogoOpt@hyphenbreak@\HOLOGO@name\endcsname=%
         \csname HoLogoOpt@break@\HOLOGO@name\endcsname
      \expandafter\chardef
         \csname HoLogoOpt@spacebreak@\HOLOGO@name\endcsname=%
         \csname HoLogoOpt@break@\HOLOGO@name\endcsname
      \expandafter\chardef
         \csname HoLogoOpt@discretionarybreak@\HOLOGO@name
             \endcsname=%
         \csname HoLogoOpt@break@\HOLOGO@name\endcsname
    \fi
  \fi
}
%    \end{macrocode}
%    \end{macro}
%
%    \begin{macro}{\HOLOGO@true}
%    \begin{macrocode}
\def\HOLOGO@true{true}
%    \end{macrocode}
%    \end{macro}
%    \begin{macro}{\HOLOGO@false}
%    \begin{macrocode}
\def\HOLOGO@false{false}
%    \end{macrocode}
%    \end{macro}
%    \begin{macro}{\HOLOGO@break}
%    \begin{macrocode}
\def\HOLOGO@break{break}
%    \end{macrocode}
%    \end{macro}
%
%    \begin{macrocode}
\HOLOGO@DeclareBoolOption{break}
\HOLOGO@DeclareBoolOption{hyphenbreak}
\HOLOGO@DeclareBoolOption{spacebreak}
\HOLOGO@DeclareBoolOption{discretionarybreak}
%    \end{macrocode}
%
%    \begin{macrocode}
\kv@define@key{HoLogo}{variant}{%
  \ifx\HOLOGO@name\relax
    \@PackageError{hologo}{%
      Option `variant' is not available in \string\hologoSetup,%
      \MessageBreak
      Use \string\hologoLogoSetup\space instead%
    }\@ehc
  \else
    \edef\HOLOGO@temp{#1}%
    \ifx\HOLOGO@temp\ltx@empty
      \expandafter
      \let\csname HoLogoOpt@variant@\HOLOGO@name\endcsname\@undefined
    \else
      \ltx@IfUndefined{HoLogo@\HOLOGO@name @\HOLOGO@temp}{%
        \@PackageError{hologo}{%
          Unknown variant `\HOLOGO@temp' of logo `\HOLOGO@name'%
        }\@ehc
      }{%
        \expandafter
        \let\csname HoLogoOpt@variant@\HOLOGO@name\endcsname
            \HOLOGO@temp
      }%
    \fi
  \fi
}
%    \end{macrocode}
%
%    \begin{macro}{\HOLOGO@Variant}
%    \begin{macrocode}
\def\HOLOGO@Variant#1{%
  #1%
  \ltx@ifundefined{HoLogoOpt@variant@#1}{%
  }{%
    @\csname HoLogoOpt@variant@#1\endcsname
  }%
}
%    \end{macrocode}
%    \end{macro}
%
% \subsection{Break/no-break support}
%
%    \begin{macro}{\HOLOGO@space}
%    \begin{macrocode}
\def\HOLOGO@space{%
  \ltx@ifundefined{HoLogoOpt@spacebreak@\HOLOGO@name}{%
    \ltx@ifundefined{HoLogoOpt@break@\HOLOGO@name}{%
      \chardef\HOLOGO@temp=\HOLOGOOPT@spacebreak
    }{%
      \chardef\HOLOGO@temp=%
        \csname HoLogoOpt@break@\HOLOGO@name\endcsname
    }%
  }{%
    \chardef\HOLOGO@temp=%
      \csname HoLogoOpt@spacebreak@\HOLOGO@name\endcsname
  }%
  \ifcase\HOLOGO@temp
    \penalty10000 %
  \fi
  \ltx@space
}
%    \end{macrocode}
%    \end{macro}
%
%    \begin{macro}{\HOLOGO@hyphen}
%    \begin{macrocode}
\def\HOLOGO@hyphen{%
  \ltx@ifundefined{HoLogoOpt@hyphenbreak@\HOLOGO@name}{%
    \ltx@ifundefined{HoLogoOpt@break@\HOLOGO@name}{%
      \chardef\HOLOGO@temp=\HOLOGOOPT@hyphenbreak
    }{%
      \chardef\HOLOGO@temp=%
        \csname HoLogoOpt@break@\HOLOGO@name\endcsname
    }%
  }{%
    \chardef\HOLOGO@temp=%
      \csname HoLogoOpt@hyphenbreak@\HOLOGO@name\endcsname
  }%
  \ifcase\HOLOGO@temp
    \ltx@mbox{-}%
  \else
    -%
  \fi
}
%    \end{macrocode}
%    \end{macro}
%
%    \begin{macro}{\HOLOGO@discretionary}
%    \begin{macrocode}
\def\HOLOGO@discretionary{%
  \ltx@ifundefined{HoLogoOpt@discretionarybreak@\HOLOGO@name}{%
    \ltx@ifundefined{HoLogoOpt@break@\HOLOGO@name}{%
      \chardef\HOLOGO@temp=\HOLOGOOPT@discretionarybreak
    }{%
      \chardef\HOLOGO@temp=%
        \csname HoLogoOpt@break@\HOLOGO@name\endcsname
    }%
  }{%
    \chardef\HOLOGO@temp=%
      \csname HoLogoOpt@discretionarybreak@\HOLOGO@name\endcsname
  }%
  \ifcase\HOLOGO@temp
  \else
    \-%
  \fi
}
%    \end{macrocode}
%    \end{macro}
%
%    \begin{macro}{\HOLOGO@mbox}
%    \begin{macrocode}
\def\HOLOGO@mbox#1{%
  \ltx@ifundefined{HoLogoOpt@break@\HOLOGO@name}{%
    \chardef\HOLOGO@temp=\HOLOGOOPT@hyphenbreak
  }{%
    \chardef\HOLOGO@temp=%
      \csname HoLogoOpt@break@\HOLOGO@name\endcsname
  }%
  \ifcase\HOLOGO@temp
    \ltx@mbox{#1}%
  \else
    #1%
  \fi
}
%    \end{macrocode}
%    \end{macro}
%
% \subsection{Font support}
%
%    \begin{macro}{\HoLogoFont@font}
%    \begin{tabular}{@{}ll@{}}
%    |#1|:& logo name\\
%    |#2|:& font short name\\
%    |#3|:& text
%    \end{tabular}
%    \begin{macrocode}
\def\HoLogoFont@font#1#2#3{%
  \begingroup
    \ltx@IfUndefined{HoLogoFont@logo@#1.#2}{%
      \ltx@IfUndefined{HoLogoFont@font@#2}{%
        \@PackageWarning{hologo}{%
          Missing font `#2' for logo `#1'%
        }%
        #3%
      }{%
        \csname HoLogoFont@font@#2\endcsname{#3}%
      }%
    }{%
      \csname HoLogoFont@logo@#1.#2\endcsname{#3}%
    }%
  \endgroup
}
%    \end{macrocode}
%    \end{macro}
%
%    \begin{macro}{\HoLogoFont@Def}
%    \begin{macrocode}
\def\HoLogoFont@Def#1{%
  \expandafter\def\csname HoLogoFont@font@#1\endcsname
}
%    \end{macrocode}
%    \end{macro}
%    \begin{macro}{\HoLogoFont@LogoDef}
%    \begin{macrocode}
\def\HoLogoFont@LogoDef#1#2{%
  \expandafter\def\csname HoLogoFont@logo@#1.#2\endcsname
}
%    \end{macrocode}
%    \end{macro}
%
% \subsubsection{Font defaults}
%
%    \begin{macro}{\HoLogoFont@font@general}
%    \begin{macrocode}
\HoLogoFont@Def{general}{}%
%    \end{macrocode}
%    \end{macro}
%
%    \begin{macro}{\HoLogoFont@font@rm}
%    \begin{macrocode}
\ltx@IfUndefined{rmfamily}{%
  \ltx@IfUndefined{rm}{%
  }{%
    \HoLogoFont@Def{rm}{\rm}%
  }%
}{%
  \HoLogoFont@Def{rm}{\rmfamily}%
}
%    \end{macrocode}
%    \end{macro}
%
%    \begin{macro}{\HoLogoFont@font@sf}
%    \begin{macrocode}
\ltx@IfUndefined{sffamily}{%
  \ltx@IfUndefined{sf}{%
  }{%
    \HoLogoFont@Def{sf}{\sf}%
  }%
}{%
  \HoLogoFont@Def{sf}{\sffamily}%
}
%    \end{macrocode}
%    \end{macro}
%
%    \begin{macro}{\HoLogoFont@font@bibsf}
%    In case of \hologo{plainTeX} the original small caps
%    variant is used as default. In \hologo{LaTeX}
%    the definition of package \xpackage{dtklogos} \cite{dtklogos}
%    is used.
%\begin{quote}
%\begin{verbatim}
%\DeclareRobustCommand{\BibTeX}{%
%  B%
%  \kern-.05em%
%  \hbox{%
%    $\m@th$% %% force math size calculations
%    \csname S@\f@size\endcsname
%    \fontsize\sf@size\z@
%    \math@fontsfalse
%    \selectfont
%    I%
%    \kern-.025em%
%    B
%  }%
%  \kern-.08em%
%  \-%
%  \TeX
%}
%\end{verbatim}
%\end{quote}
%    \begin{macrocode}
\ltx@IfUndefined{selectfont}{%
  \ltx@IfUndefined{tensc}{%
    \font\tensc=cmcsc10\relax
  }{}%
  \HoLogoFont@Def{bibsf}{\tensc}%
}{%
  \HoLogoFont@Def{bibsf}{%
    $\mathsurround=0pt$%
    \csname S@\f@size\endcsname
    \fontsize\sf@size{0pt}%
    \math@fontsfalse
    \selectfont
  }%
}
%    \end{macrocode}
%    \end{macro}
%
%    \begin{macro}{\HoLogoFont@font@sc}
%    \begin{macrocode}
\ltx@IfUndefined{scshape}{%
  \ltx@IfUndefined{tensc}{%
    \font\tensc=cmcsc10\relax
  }{}%
  \HoLogoFont@Def{sc}{\tensc}%
}{%
  \HoLogoFont@Def{sc}{\scshape}%
}
%    \end{macrocode}
%    \end{macro}
%
%    \begin{macro}{\HoLogoFont@font@sy}
%    \begin{macrocode}
\ltx@IfUndefined{usefont}{%
  \ltx@IfUndefined{tensy}{%
  }{%
    \HoLogoFont@Def{sy}{\tensy}%
  }%
}{%
  \HoLogoFont@Def{sy}{%
    \usefont{OMS}{cmsy}{m}{n}%
  }%
}
%    \end{macrocode}
%    \end{macro}
%
%    \begin{macro}{\HoLogoFont@font@logo}
%    \begin{macrocode}
\begingroup
  \def\x{LaTeX2e}%
\expandafter\endgroup
\ifx\fmtname\x
  \ltx@IfUndefined{logofamily}{%
    \DeclareRobustCommand\logofamily{%
      \not@math@alphabet\logofamily\relax
      \fontencoding{U}%
      \fontfamily{logo}%
      \selectfont
    }%
  }{}%
  \ltx@IfUndefined{logofamily}{%
  }{%
    \HoLogoFont@Def{logo}{\logofamily}%
  }%
\else
  \ltx@IfUndefined{tenlogo}{%
    \font\tenlogo=logo10\relax
  }{}%
  \HoLogoFont@Def{logo}{\tenlogo}%
\fi
%    \end{macrocode}
%    \end{macro}
%
% \subsubsection{Font setup}
%
%    \begin{macro}{\hologoFontSetup}
%    \begin{macrocode}
\def\hologoFontSetup{%
  \let\HOLOGO@name\relax
  \HOLOGO@FontSetup
}
%    \end{macrocode}
%    \end{macro}
%
%    \begin{macro}{\hologoLogoFontSetup}
%    \begin{macrocode}
\def\hologoLogoFontSetup#1{%
  \edef\HOLOGO@name{#1}%
  \ltx@IfUndefined{HoLogo@\HOLOGO@name}{%
    \@PackageError{hologo}{%
      Unknown logo `\HOLOGO@name'%
    }\@ehc
    \ltx@gobble
  }{%
    \HOLOGO@FontSetup
  }%
}
%    \end{macrocode}
%    \end{macro}
%
%    \begin{macro}{\HOLOGO@FontSetup}
%    \begin{macrocode}
\def\HOLOGO@FontSetup{%
  \kvsetkeys{HoLogoFont}%
}
%    \end{macrocode}
%    \end{macro}
%
%    \begin{macrocode}
\def\HOLOGO@temp#1{%
  \kv@define@key{HoLogoFont}{#1}{%
    \ifx\HOLOGO@name\relax
      \HoLogoFont@Def{#1}{##1}%
    \else
      \HoLogoFont@LogoDef\HOLOGO@name{#1}{##1}%
    \fi
  }%
}
\HOLOGO@temp{general}
\HOLOGO@temp{sf}
%    \end{macrocode}
%
% \subsection{Generic logo commands}
%
%    \begin{macrocode}
\HOLOGO@IfExists\hologo{%
  \@PackageError{hologo}{%
    \string\hologo\ltx@space is already defined.\MessageBreak
    Package loading is aborted%
  }\@ehc
  \HOLOGO@AtEnd
}%
\HOLOGO@IfExists\hologoRobust{%
  \@PackageError{hologo}{%
    \string\hologoRobust\ltx@space is already defined.\MessageBreak
    Package loading is aborted%
  }\@ehc
  \HOLOGO@AtEnd
}%
%    \end{macrocode}
%
% \subsubsection{\cs{hologo} and friends}
%
%    \begin{macrocode}
\ifluatex
  \expandafter\ltx@firstofone
\else
  \expandafter\ltx@gobble
\fi
{%
  \ltx@IfUndefined{ifincsname}{%
    \ifnum\luatexversion<36 %
      \expandafter\ltx@gobble
    \else
      \expandafter\ltx@firstofone
    \fi
    {%
      \begingroup
        \ifcase0%
            \directlua{%
              if tex.enableprimitives then %
                tex.enableprimitives('HOLOGO@', {'ifincsname'})%
              else %
                tex.print('1')%
              end%
            }%
            \ifx\HOLOGO@ifincsname\@undefined 1\fi%
            \relax
          \expandafter\ltx@firstofone
        \else
          \endgroup
          \expandafter\ltx@gobble
        \fi
        {%
          \global\let\ifincsname\HOLOGO@ifincsname
        }%
      \HOLOGO@temp
    }%
  }{}%
}
%    \end{macrocode}
%    \begin{macrocode}
\ltx@IfUndefined{ifincsname}{%
  \catcode`$=14 %
}{%
  \catcode`$=9 %
}
%    \end{macrocode}
%
%    \begin{macro}{\hologo}
%    \begin{macrocode}
\def\hologo#1{%
$ \ifincsname
$   \ltx@ifundefined{HoLogoCs@\HOLOGO@Variant{#1}}{%
$     #1%
$   }{%
$     \csname HoLogoCs@\HOLOGO@Variant{#1}\endcsname\ltx@firstoftwo
$   }%
$ \else
    \HOLOGO@IfExists\texorpdfstring\texorpdfstring\ltx@firstoftwo
    {%
      \hologoRobust{#1}%
    }{%
      \ltx@ifundefined{HoLogoBkm@\HOLOGO@Variant{#1}}{%
        \ltx@ifundefined{HoLogo@#1}{?#1?}{#1}%
      }{%
        \csname HoLogoBkm@\HOLOGO@Variant{#1}\endcsname
        \ltx@firstoftwo
      }%
    }%
$ \fi
}
%    \end{macrocode}
%    \end{macro}
%    \begin{macro}{\Hologo}
%    \begin{macrocode}
\def\Hologo#1{%
$ \ifincsname
$   \ltx@ifundefined{HoLogoCs@\HOLOGO@Variant{#1}}{%
$     #1%
$   }{%
$     \csname HoLogoCs@\HOLOGO@Variant{#1}\endcsname\ltx@secondoftwo
$   }%
$ \else
    \HOLOGO@IfExists\texorpdfstring\texorpdfstring\ltx@firstoftwo
    {%
      \HologoRobust{#1}%
    }{%
      \ltx@ifundefined{HoLogoBkm@\HOLOGO@Variant{#1}}{%
        \ltx@ifundefined{HoLogo@#1}{?#1?}{#1}%
      }{%
        \csname HoLogoBkm@\HOLOGO@Variant{#1}\endcsname
        \ltx@secondoftwo
      }%
    }%
$ \fi
}
%    \end{macrocode}
%    \end{macro}
%
%    \begin{macro}{\hologoVariant}
%    \begin{macrocode}
\def\hologoVariant#1#2{%
  \ifx\relax#2\relax
    \hologo{#1}%
  \else
$   \ifincsname
$     \ltx@ifundefined{HoLogoCs@#1@#2}{%
$       #1%
$     }{%
$       \csname HoLogoCs@#1@#2\endcsname\ltx@firstoftwo
$     }%
$   \else
      \HOLOGO@IfExists\texorpdfstring\texorpdfstring\ltx@firstoftwo
      {%
        \hologoVariantRobust{#1}{#2}%
      }{%
        \ltx@ifundefined{HoLogoBkm@#1@#2}{%
          \ltx@ifundefined{HoLogo@#1}{?#1?}{#1}%
        }{%
          \csname HoLogoBkm@#1@#2\endcsname
          \ltx@firstoftwo
        }%
      }%
$   \fi
  \fi
}
%    \end{macrocode}
%    \end{macro}
%    \begin{macro}{\HologoVariant}
%    \begin{macrocode}
\def\HologoVariant#1#2{%
  \ifx\relax#2\relax
    \Hologo{#1}%
  \else
$   \ifincsname
$     \ltx@ifundefined{HoLogoCs@#1@#2}{%
$       #1%
$     }{%
$       \csname HoLogoCs@#1@#2\endcsname\ltx@secondoftwo
$     }%
$   \else
      \HOLOGO@IfExists\texorpdfstring\texorpdfstring\ltx@firstoftwo
      {%
        \HologoVariantRobust{#1}{#2}%
      }{%
        \ltx@ifundefined{HoLogoBkm@#1@#2}{%
          \ltx@ifundefined{HoLogo@#1}{?#1?}{#1}%
        }{%
          \csname HoLogoBkm@#1@#2\endcsname
          \ltx@secondoftwo
        }%
      }%
$   \fi
  \fi
}
%    \end{macrocode}
%    \end{macro}
%
%    \begin{macrocode}
\catcode`\$=3 %
%    \end{macrocode}
%
% \subsubsection{\cs{hologoRobust} and friends}
%
%    \begin{macro}{\hologoRobust}
%    \begin{macrocode}
\ltx@IfUndefined{protected}{%
  \ltx@IfUndefined{DeclareRobustCommand}{%
    \def\hologoRobust#1%
  }{%
    \DeclareRobustCommand*\hologoRobust[1]%
  }%
}{%
  \protected\def\hologoRobust#1%
}%
{%
  \edef\HOLOGO@name{#1}%
  \ltx@IfUndefined{HoLogo@\HOLOGO@Variant\HOLOGO@name}{%
    \@PackageError{hologo}{%
      Unknown logo `\HOLOGO@name'%
    }\@ehc
    ?\HOLOGO@name?%
  }{%
    \ltx@IfUndefined{ver@tex4ht.sty}{%
      \HoLogoFont@font\HOLOGO@name{general}{%
        \csname HoLogo@\HOLOGO@Variant\HOLOGO@name\endcsname
        \ltx@firstoftwo
      }%
    }{%
      \ltx@IfUndefined{HoLogoHtml@\HOLOGO@Variant\HOLOGO@name}{%
        \HOLOGO@name
      }{%
        \csname HoLogoHtml@\HOLOGO@Variant\HOLOGO@name\endcsname
        \ltx@firstoftwo
      }%
    }%
  }%
}
%    \end{macrocode}
%    \end{macro}
%    \begin{macro}{\HologoRobust}
%    \begin{macrocode}
\ltx@IfUndefined{protected}{%
  \ltx@IfUndefined{DeclareRobustCommand}{%
    \def\HologoRobust#1%
  }{%
    \DeclareRobustCommand*\HologoRobust[1]%
  }%
}{%
  \protected\def\HologoRobust#1%
}%
{%
  \edef\HOLOGO@name{#1}%
  \ltx@IfUndefined{HoLogo@\HOLOGO@Variant\HOLOGO@name}{%
    \@PackageError{hologo}{%
      Unknown logo `\HOLOGO@name'%
    }\@ehc
    ?\HOLOGO@name?%
  }{%
    \ltx@IfUndefined{ver@tex4ht.sty}{%
      \HoLogoFont@font\HOLOGO@name{general}{%
        \csname HoLogo@\HOLOGO@Variant\HOLOGO@name\endcsname
        \ltx@secondoftwo
      }%
    }{%
      \ltx@IfUndefined{HoLogoHtml@\HOLOGO@Variant\HOLOGO@name}{%
        \expandafter\HOLOGO@Uppercase\HOLOGO@name
      }{%
        \csname HoLogoHtml@\HOLOGO@Variant\HOLOGO@name\endcsname
        \ltx@secondoftwo
      }%
    }%
  }%
}
%    \end{macrocode}
%    \end{macro}
%    \begin{macro}{\hologoVariantRobust}
%    \begin{macrocode}
\ltx@IfUndefined{protected}{%
  \ltx@IfUndefined{DeclareRobustCommand}{%
    \def\hologoVariantRobust#1#2%
  }{%
    \DeclareRobustCommand*\hologoVariantRobust[2]%
  }%
}{%
  \protected\def\hologoVariantRobust#1#2%
}%
{%
  \begingroup
    \hologoLogoSetup{#1}{variant={#2}}%
    \hologoRobust{#1}%
  \endgroup
}
%    \end{macrocode}
%    \end{macro}
%    \begin{macro}{\HologoVariantRobust}
%    \begin{macrocode}
\ltx@IfUndefined{protected}{%
  \ltx@IfUndefined{DeclareRobustCommand}{%
    \def\HologoVariantRobust#1#2%
  }{%
    \DeclareRobustCommand*\HologoVariantRobust[2]%
  }%
}{%
  \protected\def\HologoVariantRobust#1#2%
}%
{%
  \begingroup
    \hologoLogoSetup{#1}{variant={#2}}%
    \HologoRobust{#1}%
  \endgroup
}
%    \end{macrocode}
%    \end{macro}
%
%    \begin{macro}{\hologorobust}
%    Macro \cs{hologorobust} is only defined for compatibility.
%    Its use is deprecated.
%    \begin{macrocode}
\def\hologorobust{\hologoRobust}
%    \end{macrocode}
%    \end{macro}
%
% \subsection{Helpers}
%
%    \begin{macro}{\HOLOGO@Uppercase}
%    Macro \cs{HOLOGO@Uppercase} is restricted to \cs{uppercase},
%    because \hologo{plainTeX} or \hologo{iniTeX} do not provide
%    \cs{MakeUppercase}.
%    \begin{macrocode}
\def\HOLOGO@Uppercase#1{\uppercase{#1}}
%    \end{macrocode}
%    \end{macro}
%
%    \begin{macro}{\HOLOGO@PdfdocUnicode}
%    \begin{macrocode}
\def\HOLOGO@PdfdocUnicode{%
  \ifx\ifHy@unicode\iftrue
    \expandafter\ltx@secondoftwo
  \else
    \expandafter\ltx@firstoftwo
  \fi
}
%    \end{macrocode}
%    \end{macro}
%
%    \begin{macro}{\HOLOGO@Math}
%    \begin{macrocode}
\def\HOLOGO@MathSetup{%
  \mathsurround0pt\relax
  \HOLOGO@IfExists\f@series{%
    \if b\expandafter\ltx@car\f@series x\@nil
      \csname boldmath\endcsname
   \fi
  }{}%
}
%    \end{macrocode}
%    \end{macro}
%
%    \begin{macro}{\HOLOGO@TempDimen}
%    \begin{macrocode}
\dimendef\HOLOGO@TempDimen=\ltx@zero
%    \end{macrocode}
%    \end{macro}
%    \begin{macro}{\HOLOGO@NegativeKerning}
%    \begin{macrocode}
\def\HOLOGO@NegativeKerning#1{%
  \begingroup
    \HOLOGO@TempDimen=0pt\relax
    \comma@parse@normalized{#1}{%
      \ifdim\HOLOGO@TempDimen=0pt %
        \expandafter\HOLOGO@@NegativeKerning\comma@entry
      \fi
      \ltx@gobble
    }%
    \ifdim\HOLOGO@TempDimen<0pt %
      \kern\HOLOGO@TempDimen
    \fi
  \endgroup
}
%    \end{macrocode}
%    \end{macro}
%    \begin{macro}{\HOLOGO@@NegativeKerning}
%    \begin{macrocode}
\def\HOLOGO@@NegativeKerning#1#2{%
  \setbox\ltx@zero\hbox{#1#2}%
  \HOLOGO@TempDimen=\wd\ltx@zero
  \setbox\ltx@zero\hbox{#1\kern0pt#2}%
  \advance\HOLOGO@TempDimen by -\wd\ltx@zero
}
%    \end{macrocode}
%    \end{macro}
%
%    \begin{macro}{\HOLOGO@SpaceFactor}
%    \begin{macrocode}
\def\HOLOGO@SpaceFactor{%
  \spacefactor1000 %
}
%    \end{macrocode}
%    \end{macro}
%
%    \begin{macro}{\HOLOGO@Span}
%    \begin{macrocode}
\def\HOLOGO@Span#1#2{%
  \HCode{<span class="HoLogo-#1">}%
  #2%
  \HCode{</span>}%
}
%    \end{macrocode}
%    \end{macro}
%
% \subsubsection{Text subscript}
%
%    \begin{macro}{\HOLOGO@SubScript}%
%    \begin{macrocode}
\def\HOLOGO@SubScript#1{%
  \ltx@IfUndefined{textsubscript}{%
    \ltx@IfUndefined{text}{%
      \ltx@mbox{%
        \mathsurround=0pt\relax
        $%
          _{%
            \ltx@IfUndefined{sf@size}{%
              \mathrm{#1}%
            }{%
              \mbox{%
                \fontsize\sf@size{0pt}\selectfont
                #1%
              }%
            }%
          }%
        $%
      }%
    }{%
      \ltx@mbox{%
        \mathsurround=0pt\relax
        $_{\text{#1}}$%
      }%
    }%
  }{%
    \textsubscript{#1}%
  }%
}
%    \end{macrocode}
%    \end{macro}
%
% \subsection{\hologo{TeX} and friends}
%
% \subsubsection{\hologo{TeX}}
%
%    \begin{macro}{\HoLogo@TeX}
%    Source: \hologo{LaTeX} kernel.
%    \begin{macrocode}
\def\HoLogo@TeX#1{%
  T\kern-.1667em\lower.5ex\hbox{E}\kern-.125emX\HOLOGO@SpaceFactor
}
%    \end{macrocode}
%    \end{macro}
%    \begin{macro}{\HoLogoHtml@TeX}
%    \begin{macrocode}
\def\HoLogoHtml@TeX#1{%
  \HoLogoCss@TeX
  \HOLOGO@Span{TeX}{%
    T%
    \HOLOGO@Span{e}{%
      E%
    }%
    X%
  }%
}
%    \end{macrocode}
%    \end{macro}
%    \begin{macro}{\HoLogoCss@TeX}
%    \begin{macrocode}
\def\HoLogoCss@TeX{%
  \Css{%
    span.HoLogo-TeX span.HoLogo-e{%
      position:relative;%
      top:.5ex;%
      margin-left:-.1667em;%
      margin-right:-.125em;%
    }%
  }%
  \Css{%
    a span.HoLogo-TeX span.HoLogo-e{%
      text-decoration:none;%
    }%
  }%
  \global\let\HoLogoCss@TeX\relax
}
%    \end{macrocode}
%    \end{macro}
%
% \subsubsection{\hologo{plainTeX}}
%
%    \begin{macro}{\HoLogo@plainTeX@space}
%    Source: ``The \hologo{TeX}book''
%    \begin{macrocode}
\def\HoLogo@plainTeX@space#1{%
  \HOLOGO@mbox{#1{p}{P}lain}\HOLOGO@space\hologo{TeX}%
}
%    \end{macrocode}
%    \end{macro}
%    \begin{macro}{\HoLogoCs@plainTeX@space}
%    \begin{macrocode}
\def\HoLogoCs@plainTeX@space#1{#1{p}{P}lain TeX}%
%    \end{macrocode}
%    \end{macro}
%    \begin{macro}{\HoLogoBkm@plainTeX@space}
%    \begin{macrocode}
\def\HoLogoBkm@plainTeX@space#1{%
  #1{p}{P}lain \hologo{TeX}%
}
%    \end{macrocode}
%    \end{macro}
%    \begin{macro}{\HoLogoHtml@plainTeX@space}
%    \begin{macrocode}
\def\HoLogoHtml@plainTeX@space#1{%
  #1{p}{P}lain \hologo{TeX}%
}
%    \end{macrocode}
%    \end{macro}
%
%    \begin{macro}{\HoLogo@plainTeX@hyphen}
%    \begin{macrocode}
\def\HoLogo@plainTeX@hyphen#1{%
  \HOLOGO@mbox{#1{p}{P}lain}\HOLOGO@hyphen\hologo{TeX}%
}
%    \end{macrocode}
%    \end{macro}
%    \begin{macro}{\HoLogoCs@plainTeX@hyphen}
%    \begin{macrocode}
\def\HoLogoCs@plainTeX@hyphen#1{#1{p}{P}lain-TeX}
%    \end{macrocode}
%    \end{macro}
%    \begin{macro}{\HoLogoBkm@plainTeX@hyphen}
%    \begin{macrocode}
\def\HoLogoBkm@plainTeX@hyphen#1{%
  #1{p}{P}lain-\hologo{TeX}%
}
%    \end{macrocode}
%    \end{macro}
%    \begin{macro}{\HoLogoHtml@plainTeX@hyphen}
%    \begin{macrocode}
\def\HoLogoHtml@plainTeX@hyphen#1{%
  #1{p}{P}lain-\hologo{TeX}%
}
%    \end{macrocode}
%    \end{macro}
%
%    \begin{macro}{\HoLogo@plainTeX@runtogether}
%    \begin{macrocode}
\def\HoLogo@plainTeX@runtogether#1{%
  \HOLOGO@mbox{#1{p}{P}lain\hologo{TeX}}%
}
%    \end{macrocode}
%    \end{macro}
%    \begin{macro}{\HoLogoCs@plainTeX@runtogether}
%    \begin{macrocode}
\def\HoLogoCs@plainTeX@runtogether#1{#1{p}{P}lainTeX}
%    \end{macrocode}
%    \end{macro}
%    \begin{macro}{\HoLogoBkm@plainTeX@runtogether}
%    \begin{macrocode}
\def\HoLogoBkm@plainTeX@runtogether#1{%
  #1{p}{P}lain\hologo{TeX}%
}
%    \end{macrocode}
%    \end{macro}
%    \begin{macro}{\HoLogoHtml@plainTeX@runtogether}
%    \begin{macrocode}
\def\HoLogoHtml@plainTeX@runtogether#1{%
  #1{p}{P}lain\hologo{TeX}%
}
%    \end{macrocode}
%    \end{macro}
%
%    \begin{macro}{\HoLogo@plainTeX}
%    \begin{macrocode}
\def\HoLogo@plainTeX{\HoLogo@plainTeX@space}
%    \end{macrocode}
%    \end{macro}
%    \begin{macro}{\HoLogoCs@plainTeX}
%    \begin{macrocode}
\def\HoLogoCs@plainTeX{\HoLogoCs@plainTeX@space}
%    \end{macrocode}
%    \end{macro}
%    \begin{macro}{\HoLogoBkm@plainTeX}
%    \begin{macrocode}
\def\HoLogoBkm@plainTeX{\HoLogoBkm@plainTeX@space}
%    \end{macrocode}
%    \end{macro}
%    \begin{macro}{\HoLogoHtml@plainTeX}
%    \begin{macrocode}
\def\HoLogoHtml@plainTeX{\HoLogoHtml@plainTeX@space}
%    \end{macrocode}
%    \end{macro}
%
% \subsubsection{\hologo{LaTeX}}
%
%    Source: \hologo{LaTeX} kernel.
%\begin{quote}
%\begin{verbatim}
%\DeclareRobustCommand{\LaTeX}{%
%  L%
%  \kern-.36em%
%  {%
%    \sbox\z@ T%
%    \vbox to\ht\z@{%
%      \hbox{%
%        \check@mathfonts
%        \fontsize\sf@size\z@
%        \math@fontsfalse
%        \selectfont
%        A%
%      }%
%      \vss
%    }%
%  }%
%  \kern-.15em%
%  \TeX
%}
%\end{verbatim}
%\end{quote}
%
%    \begin{macro}{\HoLogo@La}
%    \begin{macrocode}
\def\HoLogo@La#1{%
  L%
  \kern-.36em%
  \begingroup
    \setbox\ltx@zero\hbox{T}%
    \vbox to\ht\ltx@zero{%
      \hbox{%
        \ltx@ifundefined{check@mathfonts}{%
          \csname sevenrm\endcsname
        }{%
          \check@mathfonts
          \fontsize\sf@size{0pt}%
          \math@fontsfalse\selectfont
        }%
        A%
      }%
      \vss
    }%
  \endgroup
}
%    \end{macrocode}
%    \end{macro}
%
%    \begin{macro}{\HoLogo@LaTeX}
%    Source: \hologo{LaTeX} kernel.
%    \begin{macrocode}
\def\HoLogo@LaTeX#1{%
  \hologo{La}%
  \kern-.15em%
  \hologo{TeX}%
}
%    \end{macrocode}
%    \end{macro}
%    \begin{macro}{\HoLogoHtml@LaTeX}
%    \begin{macrocode}
\def\HoLogoHtml@LaTeX#1{%
  \HoLogoCss@LaTeX
  \HOLOGO@Span{LaTeX}{%
    L%
    \HOLOGO@Span{a}{%
      A%
    }%
    \hologo{TeX}%
  }%
}
%    \end{macrocode}
%    \end{macro}
%    \begin{macro}{\HoLogoCss@LaTeX}
%    \begin{macrocode}
\def\HoLogoCss@LaTeX{%
  \Css{%
    span.HoLogo-LaTeX span.HoLogo-a{%
      position:relative;%
      top:-.5ex;%
      margin-left:-.36em;%
      margin-right:-.15em;%
      font-size:85\%;%
    }%
  }%
  \global\let\HoLogoCss@LaTeX\relax
}
%    \end{macrocode}
%    \end{macro}
%
% \subsubsection{\hologo{(La)TeX}}
%
%    \begin{macro}{\HoLogo@LaTeXTeX}
%    The kerning around the parentheses is taken
%    from package \xpackage{dtklogos} \cite{dtklogos}.
%\begin{quote}
%\begin{verbatim}
%\DeclareRobustCommand{\LaTeXTeX}{%
%  (%
%  \kern-.15em%
%  L%
%  \kern-.36em%
%  {%
%    \sbox\z@ T%
%    \vbox to\ht0{%
%      \hbox{%
%        $\m@th$%
%        \csname S@\f@size\endcsname
%        \fontsize\sf@size\z@
%        \math@fontsfalse
%        \selectfont
%        A%
%      }%
%      \vss
%    }%
%  }%
%  \kern-.2em%
%  )%
%  \kern-.15em%
%  \TeX
%}
%\end{verbatim}
%\end{quote}
%    \begin{macrocode}
\def\HoLogo@LaTeXTeX#1{%
  (%
  \kern-.15em%
  \hologo{La}%
  \kern-.2em%
  )%
  \kern-.15em%
  \hologo{TeX}%
}
%    \end{macrocode}
%    \end{macro}
%    \begin{macro}{\HoLogoBkm@LaTeXTeX}
%    \begin{macrocode}
\def\HoLogoBkm@LaTeXTeX#1{(La)TeX}
%    \end{macrocode}
%    \end{macro}
%
%    \begin{macro}{\HoLogo@(La)TeX}
%    \begin{macrocode}
\expandafter
\let\csname HoLogo@(La)TeX\endcsname\HoLogo@LaTeXTeX
%    \end{macrocode}
%    \end{macro}
%    \begin{macro}{\HoLogoBkm@(La)TeX}
%    \begin{macrocode}
\expandafter
\let\csname HoLogoBkm@(La)TeX\endcsname\HoLogoBkm@LaTeXTeX
%    \end{macrocode}
%    \end{macro}
%    \begin{macro}{\HoLogoHtml@LaTeXTeX}
%    \begin{macrocode}
\def\HoLogoHtml@LaTeXTeX#1{%
  \HoLogoCss@LaTeXTeX
  \HOLOGO@Span{LaTeXTeX}{%
    (%
    \HOLOGO@Span{L}{L}%
    \HOLOGO@Span{a}{A}%
    \HOLOGO@Span{ParenRight}{)}%
    \hologo{TeX}%
  }%
}
%    \end{macrocode}
%    \end{macro}
%    \begin{macro}{\HoLogoHtml@(La)TeX}
%    Kerning after opening parentheses and before closing parentheses
%    is $-0.1$\,em. The original values $-0.15$\,em
%    looked too ugly for a serif font.
%    \begin{macrocode}
\expandafter
\let\csname HoLogoHtml@(La)TeX\endcsname\HoLogoHtml@LaTeXTeX
%    \end{macrocode}
%    \end{macro}
%    \begin{macro}{\HoLogoCss@LaTeXTeX}
%    \begin{macrocode}
\def\HoLogoCss@LaTeXTeX{%
  \Css{%
    span.HoLogo-LaTeXTeX span.HoLogo-L{%
      margin-left:-.1em;%
    }%
  }%
  \Css{%
    span.HoLogo-LaTeXTeX span.HoLogo-a{%
      position:relative;%
      top:-.5ex;%
      margin-left:-.36em;%
      margin-right:-.1em;%
      font-size:85\%;%
    }%
  }%
  \Css{%
    span.HoLogo-LaTeXTeX span.HoLogo-ParenRight{%
      margin-right:-.15em;%
    }%
  }%
  \global\let\HoLogoCss@LaTeXTeX\relax
}
%    \end{macrocode}
%    \end{macro}
%
% \subsubsection{\hologo{LaTeXe}}
%
%    \begin{macro}{\HoLogo@LaTeXe}
%    Source: \hologo{LaTeX} kernel
%    \begin{macrocode}
\def\HoLogo@LaTeXe#1{%
  \hologo{LaTeX}%
  \kern.15em%
  \hbox{%
    \HOLOGO@MathSetup
    2%
    $_{\textstyle\varepsilon}$%
  }%
}
%    \end{macrocode}
%    \end{macro}
%
%    \begin{macro}{\HoLogoCs@LaTeXe}
%    \begin{macrocode}
\ifnum64=`\^^^^0040\relax % test for big chars of LuaTeX/XeTeX
  \catcode`\$=9 %
  \catcode`\&=14 %
\else
  \catcode`\$=14 %
  \catcode`\&=9 %
\fi
\def\HoLogoCs@LaTeXe#1{%
  LaTeX2%
$ \string ^^^^0395%
& e%
}%
\catcode`\$=3 %
\catcode`\&=4 %
%    \end{macrocode}
%    \end{macro}
%
%    \begin{macro}{\HoLogoBkm@LaTeXe}
%    \begin{macrocode}
\def\HoLogoBkm@LaTeXe#1{%
  \hologo{LaTeX}%
  2%
  \HOLOGO@PdfdocUnicode{e}{\textepsilon}%
}
%    \end{macrocode}
%    \end{macro}
%
%    \begin{macro}{\HoLogoHtml@LaTeXe}
%    \begin{macrocode}
\def\HoLogoHtml@LaTeXe#1{%
  \HoLogoCss@LaTeXe
  \HOLOGO@Span{LaTeX2e}{%
    \hologo{LaTeX}%
    \HOLOGO@Span{2}{2}%
    \HOLOGO@Span{e}{%
      \HOLOGO@MathSetup
      \ensuremath{\textstyle\varepsilon}%
    }%
  }%
}
%    \end{macrocode}
%    \end{macro}
%    \begin{macro}{\HoLogoCss@LaTeXe}
%    \begin{macrocode}
\def\HoLogoCss@LaTeXe{%
  \Css{%
    span.HoLogo-LaTeX2e span.HoLogo-2{%
      padding-left:.15em;%
    }%
  }%
  \Css{%
    span.HoLogo-LaTeX2e span.HoLogo-e{%
      position:relative;%
      top:.35ex;%
      text-decoration:none;%
    }%
  }%
  \global\let\HoLogoCss@LaTeXe\relax
}
%    \end{macrocode}
%    \end{macro}
%
%    \begin{macro}{\HoLogo@LaTeX2e}
%    \begin{macrocode}
\expandafter
\let\csname HoLogo@LaTeX2e\endcsname\HoLogo@LaTeXe
%    \end{macrocode}
%    \end{macro}
%    \begin{macro}{\HoLogoCs@LaTeX2e}
%    \begin{macrocode}
\expandafter
\let\csname HoLogoCs@LaTeX2e\endcsname\HoLogoCs@LaTeXe
%    \end{macrocode}
%    \end{macro}
%    \begin{macro}{\HoLogoBkm@LaTeX2e}
%    \begin{macrocode}
\expandafter
\let\csname HoLogoBkm@LaTeX2e\endcsname\HoLogoBkm@LaTeXe
%    \end{macrocode}
%    \end{macro}
%    \begin{macro}{\HoLogoHtml@LaTeX2e}
%    \begin{macrocode}
\expandafter
\let\csname HoLogoHtml@LaTeX2e\endcsname\HoLogoHtml@LaTeXe
%    \end{macrocode}
%    \end{macro}
%
% \subsubsection{\hologo{LaTeX3}}
%
%    \begin{macro}{\HoLogo@LaTeX3}
%    Source: \hologo{LaTeX} kernel
%    \begin{macrocode}
\expandafter\def\csname HoLogo@LaTeX3\endcsname#1{%
  \hologo{LaTeX}%
  3%
}
%    \end{macrocode}
%    \end{macro}
%
%    \begin{macro}{\HoLogoBkm@LaTeX3}
%    \begin{macrocode}
\expandafter\def\csname HoLogoBkm@LaTeX3\endcsname#1{%
  \hologo{LaTeX}%
  3%
}
%    \end{macrocode}
%    \end{macro}
%    \begin{macro}{\HoLogoHtml@LaTeX3}
%    \begin{macrocode}
\expandafter
\let\csname HoLogoHtml@LaTeX3\expandafter\endcsname
\csname HoLogo@LaTeX3\endcsname
%    \end{macrocode}
%    \end{macro}
%
% \subsubsection{\hologo{LaTeXML}}
%
%    \begin{macro}{\HoLogo@LaTeXML}
%    \begin{macrocode}
\def\HoLogo@LaTeXML#1{%
  \HOLOGO@mbox{%
    \hologo{La}%
    \kern-.15em%
    T%
    \kern-.1667em%
    \lower.5ex\hbox{E}%
    \kern-.125em%
    \HoLogoFont@font{LaTeXML}{sc}{xml}%
  }%
}
%    \end{macrocode}
%    \end{macro}
%    \begin{macro}{\HoLogoHtml@pdfLaTeX}
%    \begin{macrocode}
\def\HoLogoHtml@LaTeXML#1{%
  \HOLOGO@Span{LaTeXML}{%
    \HoLogoCss@LaTeX
    \HoLogoCss@TeX
    \HOLOGO@Span{LaTeX}{%
      L%
      \HOLOGO@Span{a}{%
        A%
      }%
    }%
    \HOLOGO@Span{TeX}{%
      T%
      \HOLOGO@Span{e}{%
        E%
      }%
    }%
    \HCode{<span style="font-variant: small-caps;">}%
    xml%
    \HCode{</span>}%
  }%
}
%    \end{macrocode}
%    \end{macro}
%
% \subsubsection{\hologo{eTeX}}
%
%    \begin{macro}{\HoLogo@eTeX}
%    Source: package \xpackage{etex}
%    \begin{macrocode}
\def\HoLogo@eTeX#1{%
  \ltx@mbox{%
    \HOLOGO@MathSetup
    $\varepsilon$%
    -%
    \HOLOGO@NegativeKerning{-T,T-,To}%
    \hologo{TeX}%
  }%
}
%    \end{macrocode}
%    \end{macro}
%    \begin{macro}{\HoLogoCs@eTeX}
%    \begin{macrocode}
\ifnum64=`\^^^^0040\relax % test for big chars of LuaTeX/XeTeX
  \catcode`\$=9 %
  \catcode`\&=14 %
\else
  \catcode`\$=14 %
  \catcode`\&=9 %
\fi
\def\HoLogoCs@eTeX#1{%
$ #1{\string ^^^^0395}{\string ^^^^03b5}%
& #1{e}{E}%
  TeX%
}%
\catcode`\$=3 %
\catcode`\&=4 %
%    \end{macrocode}
%    \end{macro}
%    \begin{macro}{\HoLogoBkm@eTeX}
%    \begin{macrocode}
\def\HoLogoBkm@eTeX#1{%
  \HOLOGO@PdfdocUnicode{#1{e}{E}}{\textepsilon}%
  -%
  \hologo{TeX}%
}
%    \end{macrocode}
%    \end{macro}
%    \begin{macro}{\HoLogoHtml@eTeX}
%    \begin{macrocode}
\def\HoLogoHtml@eTeX#1{%
  \ltx@mbox{%
    \HOLOGO@MathSetup
    $\varepsilon$%
    -%
    \hologo{TeX}%
  }%
}
%    \end{macrocode}
%    \end{macro}
%
% \subsubsection{\hologo{iniTeX}}
%
%    \begin{macro}{\HoLogo@iniTeX}
%    \begin{macrocode}
\def\HoLogo@iniTeX#1{%
  \HOLOGO@mbox{%
    #1{i}{I}ni\hologo{TeX}%
  }%
}
%    \end{macrocode}
%    \end{macro}
%    \begin{macro}{\HoLogoCs@iniTeX}
%    \begin{macrocode}
\def\HoLogoCs@iniTeX#1{#1{i}{I}niTeX}
%    \end{macrocode}
%    \end{macro}
%    \begin{macro}{\HoLogoBkm@iniTeX}
%    \begin{macrocode}
\def\HoLogoBkm@iniTeX#1{%
  #1{i}{I}ni\hologo{TeX}%
}
%    \end{macrocode}
%    \end{macro}
%    \begin{macro}{\HoLogoHtml@iniTeX}
%    \begin{macrocode}
\let\HoLogoHtml@iniTeX\HoLogo@iniTeX
%    \end{macrocode}
%    \end{macro}
%
% \subsubsection{\hologo{virTeX}}
%
%    \begin{macro}{\HoLogo@virTeX}
%    \begin{macrocode}
\def\HoLogo@virTeX#1{%
  \HOLOGO@mbox{%
    #1{v}{V}ir\hologo{TeX}%
  }%
}
%    \end{macrocode}
%    \end{macro}
%    \begin{macro}{\HoLogoCs@virTeX}
%    \begin{macrocode}
\def\HoLogoCs@virTeX#1{#1{v}{V}irTeX}
%    \end{macrocode}
%    \end{macro}
%    \begin{macro}{\HoLogoBkm@virTeX}
%    \begin{macrocode}
\def\HoLogoBkm@virTeX#1{%
  #1{v}{V}ir\hologo{TeX}%
}
%    \end{macrocode}
%    \end{macro}
%    \begin{macro}{\HoLogoHtml@virTeX}
%    \begin{macrocode}
\let\HoLogoHtml@virTeX\HoLogo@virTeX
%    \end{macrocode}
%    \end{macro}
%
% \subsubsection{\hologo{SliTeX}}
%
% \paragraph{Definitions of the three variants.}
%
%    \begin{macro}{\HoLogo@SLiTeX@lift}
%    \begin{macrocode}
\def\HoLogo@SLiTeX@lift#1{%
  \HoLogoFont@font{SliTeX}{rm}{%
    S%
    \kern-.06em%
    L%
    \kern-.18em%
    \raise.32ex\hbox{\HoLogoFont@font{SliTeX}{sc}{i}}%
    \HOLOGO@discretionary
    \kern-.06em%
    \hologo{TeX}%
  }%
}
%    \end{macrocode}
%    \end{macro}
%    \begin{macro}{\HoLogoBkm@SLiTeX@lift}
%    \begin{macrocode}
\def\HoLogoBkm@SLiTeX@lift#1{SLiTeX}
%    \end{macrocode}
%    \end{macro}
%    \begin{macro}{\HoLogoHtml@SLiTeX@lift}
%    \begin{macrocode}
\def\HoLogoHtml@SLiTeX@lift#1{%
  \HoLogoCss@SLiTeX@lift
  \HOLOGO@Span{SLiTeX-lift}{%
    \HoLogoFont@font{SliTeX}{rm}{%
      S%
      \HOLOGO@Span{L}{L}%
      \HOLOGO@Span{i}{i}%
      \hologo{TeX}%
    }%
  }%
}
%    \end{macrocode}
%    \end{macro}
%    \begin{macro}{\HoLogoCss@SLiTeX@lift}
%    \begin{macrocode}
\def\HoLogoCss@SLiTeX@lift{%
  \Css{%
    span.HoLogo-SLiTeX-lift span.HoLogo-L{%
      margin-left:-.06em;%
      margin-right:-.18em;%
    }%
  }%
  \Css{%
    span.HoLogo-SLiTeX-lift span.HoLogo-i{%
      position:relative;%
      top:-.32ex;%
      margin-right:-.06em;%
      font-variant:small-caps;%
    }%
  }%
  \global\let\HoLogoCss@SLiTeX@lift\relax
}
%    \end{macrocode}
%    \end{macro}
%
%    \begin{macro}{\HoLogo@SliTeX@simple}
%    \begin{macrocode}
\def\HoLogo@SliTeX@simple#1{%
  \HoLogoFont@font{SliTeX}{rm}{%
    \ltx@mbox{%
      \HoLogoFont@font{SliTeX}{sc}{Sli}%
    }%
    \HOLOGO@discretionary
    \hologo{TeX}%
  }%
}
%    \end{macrocode}
%    \end{macro}
%    \begin{macro}{\HoLogoBkm@SliTeX@simple}
%    \begin{macrocode}
\def\HoLogoBkm@SliTeX@simple#1{SliTeX}
%    \end{macrocode}
%    \end{macro}
%    \begin{macro}{\HoLogoHtml@SliTeX@simple}
%    \begin{macrocode}
\let\HoLogoHtml@SliTeX@simple\HoLogo@SliTeX@simple
%    \end{macrocode}
%    \end{macro}
%
%    \begin{macro}{\HoLogo@SliTeX@narrow}
%    \begin{macrocode}
\def\HoLogo@SliTeX@narrow#1{%
  \HoLogoFont@font{SliTeX}{rm}{%
    \ltx@mbox{%
      S%
      \kern-.06em%
      \HoLogoFont@font{SliTeX}{sc}{%
        l%
        \kern-.035em%
        i%
      }%
    }%
    \HOLOGO@discretionary
    \kern-.06em%
    \hologo{TeX}%
  }%
}
%    \end{macrocode}
%    \end{macro}
%    \begin{macro}{\HoLogoBkm@SliTeX@narrow}
%    \begin{macrocode}
\def\HoLogoBkm@SliTeX@narrow#1{SliTeX}
%    \end{macrocode}
%    \end{macro}
%    \begin{macro}{\HoLogoHtml@SliTeX@narrow}
%    \begin{macrocode}
\def\HoLogoHtml@SliTeX@narrow#1{%
  \HoLogoCss@SliTeX@narrow
  \HOLOGO@Span{SliTeX-narrow}{%
    \HoLogoFont@font{SliTeX}{rm}{%
      S%
        \HOLOGO@Span{l}{l}%
        \HOLOGO@Span{i}{i}%
      \hologo{TeX}%
    }%
  }%
}
%    \end{macrocode}
%    \end{macro}
%    \begin{macro}{\HoLogoCss@SliTeX@narrow}
%    \begin{macrocode}
\def\HoLogoCss@SliTeX@narrow{%
  \Css{%
    span.HoLogo-SliTeX-narrow span.HoLogo-l{%
      margin-left:-.06em;%
      margin-right:-.035em;%
      font-variant:small-caps;%
    }%
  }%
  \Css{%
    span.HoLogo-SliTeX-narrow span.HoLogo-i{%
      margin-right:-.06em;%
      font-variant:small-caps;%
    }%
  }%
  \global\let\HoLogoCss@SliTeX@narrow\relax
}
%    \end{macrocode}
%    \end{macro}
%
% \paragraph{Macro set completion.}
%
%    \begin{macro}{\HoLogo@SLiTeX@simple}
%    \begin{macrocode}
\def\HoLogo@SLiTeX@simple{\HoLogo@SliTeX@simple}
%    \end{macrocode}
%    \end{macro}
%    \begin{macro}{\HoLogoBkm@SLiTeX@simple}
%    \begin{macrocode}
\def\HoLogoBkm@SLiTeX@simple{\HoLogoBkm@SliTeX@simple}
%    \end{macrocode}
%    \end{macro}
%    \begin{macro}{\HoLogoHtml@SLiTeX@simple}
%    \begin{macrocode}
\def\HoLogoHtml@SLiTeX@simple{\HoLogoHtml@SliTeX@simple}
%    \end{macrocode}
%    \end{macro}
%
%    \begin{macro}{\HoLogo@SLiTeX@narrow}
%    \begin{macrocode}
\def\HoLogo@SLiTeX@narrow{\HoLogo@SliTeX@narrow}
%    \end{macrocode}
%    \end{macro}
%    \begin{macro}{\HoLogoBkm@SLiTeX@narrow}
%    \begin{macrocode}
\def\HoLogoBkm@SLiTeX@narrow{\HoLogoBkm@SliTeX@narrow}
%    \end{macrocode}
%    \end{macro}
%    \begin{macro}{\HoLogoHtml@SLiTeX@narrow}
%    \begin{macrocode}
\def\HoLogoHtml@SLiTeX@narrow{\HoLogoHtml@SliTeX@narrow}
%    \end{macrocode}
%    \end{macro}
%
%    \begin{macro}{\HoLogo@SliTeX@lift}
%    \begin{macrocode}
\def\HoLogo@SliTeX@lift{\HoLogo@SLiTeX@lift}
%    \end{macrocode}
%    \end{macro}
%    \begin{macro}{\HoLogoBkm@SliTeX@lift}
%    \begin{macrocode}
\def\HoLogoBkm@SliTeX@lift{\HoLogoBkm@SLiTeX@lift}
%    \end{macrocode}
%    \end{macro}
%    \begin{macro}{\HoLogoHtml@SliTeX@lift}
%    \begin{macrocode}
\def\HoLogoHtml@SliTeX@lift{\HoLogoHtml@SLiTeX@lift}
%    \end{macrocode}
%    \end{macro}
%
% \paragraph{Defaults.}
%
%    \begin{macro}{\HoLogo@SLiTeX}
%    \begin{macrocode}
\def\HoLogo@SLiTeX{\HoLogo@SLiTeX@lift}
%    \end{macrocode}
%    \end{macro}
%    \begin{macro}{\HoLogoBkm@SLiTeX}
%    \begin{macrocode}
\def\HoLogoBkm@SLiTeX{\HoLogoBkm@SLiTeX@lift}
%    \end{macrocode}
%    \end{macro}
%    \begin{macro}{\HoLogoHtml@SLiTeX}
%    \begin{macrocode}
\def\HoLogoHtml@SLiTeX{\HoLogoHtml@SLiTeX@lift}
%    \end{macrocode}
%    \end{macro}
%
%    \begin{macro}{\HoLogo@SliTeX}
%    \begin{macrocode}
\def\HoLogo@SliTeX{\HoLogo@SliTeX@narrow}
%    \end{macrocode}
%    \end{macro}
%    \begin{macro}{\HoLogoBkm@SliTeX}
%    \begin{macrocode}
\def\HoLogoBkm@SliTeX{\HoLogoBkm@SliTeX@narrow}
%    \end{macrocode}
%    \end{macro}
%    \begin{macro}{\HoLogoHtml@SliTeX}
%    \begin{macrocode}
\def\HoLogoHtml@SliTeX{\HoLogoHtml@SliTeX@narrow}
%    \end{macrocode}
%    \end{macro}
%
% \subsubsection{\hologo{LuaTeX}}
%
%    \begin{macro}{\HoLogo@LuaTeX}
%    The kerning is an idea of Hans Hagen, see mailing list
%    `luatex at tug dot org' in March 2010.
%    \begin{macrocode}
\def\HoLogo@LuaTeX#1{%
  \HOLOGO@mbox{%
    Lua%
    \HOLOGO@NegativeKerning{aT,oT,To}%
    \hologo{TeX}%
  }%
}
%    \end{macrocode}
%    \end{macro}
%    \begin{macro}{\HoLogoHtml@LuaTeX}
%    \begin{macrocode}
\let\HoLogoHtml@LuaTeX\HoLogo@LuaTeX
%    \end{macrocode}
%    \end{macro}
%
% \subsubsection{\hologo{LuaLaTeX}}
%
%    \begin{macro}{\HoLogo@LuaLaTeX}
%    \begin{macrocode}
\def\HoLogo@LuaLaTeX#1{%
  \HOLOGO@mbox{%
    Lua%
    \hologo{LaTeX}%
  }%
}
%    \end{macrocode}
%    \end{macro}
%    \begin{macro}{\HoLogoHtml@LuaLaTeX}
%    \begin{macrocode}
\let\HoLogoHtml@LuaLaTeX\HoLogo@LuaLaTeX
%    \end{macrocode}
%    \end{macro}
%
% \subsubsection{\hologo{XeTeX}, \hologo{XeLaTeX}}
%
%    \begin{macro}{\HOLOGO@IfCharExists}
%    \begin{macrocode}
\ifluatex
  \ifnum\luatexversion<36 %
  \else
    \def\HOLOGO@IfCharExists#1{%
      \ifnum
        \directlua{%
           if luaotfload and luaotfload.aux then
             if luaotfload.aux.font_has_glyph(%
                    font.current(), \number#1) then % 	 
	       tex.print("1") % 	 
	     end % 	 
	   elseif font and font.fonts and font.current then %
            local f = font.fonts[font.current()]%
            if f.characters and f.characters[\number#1] then %
              tex.print("1")%
            end %
          end%
        }0=\ltx@zero
        \expandafter\ltx@secondoftwo
      \else
        \expandafter\ltx@firstoftwo
      \fi
    }%
  \fi
\fi
\ltx@IfUndefined{HOLOGO@IfCharExists}{%
  \def\HOLOGO@@IfCharExists#1{%
    \begingroup
      \tracinglostchars=\ltx@zero
      \setbox\ltx@zero=\hbox{%
        \kern7sp\char#1\relax
        \ifnum\lastkern>\ltx@zero
          \expandafter\aftergroup\csname iffalse\endcsname
        \else
          \expandafter\aftergroup\csname iftrue\endcsname
        \fi
      }%
      % \if{true|false} from \aftergroup
      \endgroup
      \expandafter\ltx@firstoftwo
    \else
      \endgroup
      \expandafter\ltx@secondoftwo
    \fi
  }%
  \ifxetex
    \ltx@IfUndefined{XeTeXfonttype}{}{%
      \ltx@IfUndefined{XeTeXcharglyph}{}{%
        \def\HOLOGO@IfCharExists#1{%
          \ifnum\XeTeXfonttype\font>\ltx@zero
            \expandafter\ltx@firstofthree
          \else
            \expandafter\ltx@gobble
          \fi
          {%
            \ifnum\XeTeXcharglyph#1>\ltx@zero
              \expandafter\ltx@firstoftwo
            \else
              \expandafter\ltx@secondoftwo
            \fi
          }%
          \HOLOGO@@IfCharExists{#1}%
        }%
      }%
    }%
  \fi
}{}
\ltx@ifundefined{HOLOGO@IfCharExists}{%
  \ifnum64=`\^^^^0040\relax % test for big chars of LuaTeX/XeTeX
    \let\HOLOGO@IfCharExists\HOLOGO@@IfCharExists
  \else
    \def\HOLOGO@IfCharExists#1{%
      \ifnum#1>255 %
        \expandafter\ltx@fourthoffour
      \fi
      \HOLOGO@@IfCharExists{#1}%
    }%
  \fi
}{}
%    \end{macrocode}
%    \end{macro}
%
%    \begin{macro}{\HoLogo@Xe}
%    Source: package \xpackage{dtklogos}
%    \begin{macrocode}
\def\HoLogo@Xe#1{%
  X%
  \kern-.1em\relax
  \HOLOGO@IfCharExists{"018E}{%
    \lower.5ex\hbox{\char"018E}%
  }{%
    \chardef\HOLOGO@choice=\ltx@zero
    \ifdim\fontdimen\ltx@one\font>0pt %
      \ltx@IfUndefined{rotatebox}{%
        \ltx@IfUndefined{pgftext}{%
          \ltx@IfUndefined{psscalebox}{%
            \ltx@IfUndefined{HOLOGO@ScaleBox@\hologoDriver}{%
            }{%
              \chardef\HOLOGO@choice=4 %
            }%
          }{%
            \chardef\HOLOGO@choice=3 %
          }%
        }{%
          \chardef\HOLOGO@choice=2 %
        }%
      }{%
        \chardef\HOLOGO@choice=1 %
      }%
      \ifcase\HOLOGO@choice
        \HOLOGO@WarningUnsupportedDriver{Xe}%
        e%
      \or % 1: \rotatebox
        \begingroup
          \setbox\ltx@zero\hbox{\rotatebox{180}{E}}%
          \ltx@LocDimenA=\dp\ltx@zero
          \advance\ltx@LocDimenA by -.5ex\relax
          \raise\ltx@LocDimenA\box\ltx@zero
        \endgroup
      \or % 2: \pgftext
        \lower.5ex\hbox{%
          \pgfpicture
            \pgftext[rotate=180]{E}%
          \endpgfpicture
        }%
      \or % 3: \psscalebox
        \begingroup
          \setbox\ltx@zero\hbox{\psscalebox{-1 -1}{E}}%
          \ltx@LocDimenA=\dp\ltx@zero
          \advance\ltx@LocDimenA by -.5ex\relax
          \raise\ltx@LocDimenA\box\ltx@zero
        \endgroup
      \or % 4: \HOLOGO@PointReflectBox
        \lower.5ex\hbox{\HOLOGO@PointReflectBox{E}}%
      \else
        \@PackageError{hologo}{Internal error (choice/it}\@ehc
      \fi
    \else
      \ltx@IfUndefined{reflectbox}{%
        \ltx@IfUndefined{pgftext}{%
          \ltx@IfUndefined{psscalebox}{%
            \ltx@IfUndefined{HOLOGO@ScaleBox@\hologoDriver}{%
            }{%
              \chardef\HOLOGO@choice=4 %
            }%
          }{%
            \chardef\HOLOGO@choice=3 %
          }%
        }{%
          \chardef\HOLOGO@choice=2 %
        }%
      }{%
        \chardef\HOLOGO@choice=1 %
      }%
      \ifcase\HOLOGO@choice
        \HOLOGO@WarningUnsupportedDriver{Xe}%
        e%
      \or % 1: reflectbox
        \lower.5ex\hbox{%
          \reflectbox{E}%
        }%
      \or % 2: \pgftext
        \lower.5ex\hbox{%
          \pgfpicture
            \pgftransformxscale{-1}%
            \pgftext{E}%
          \endpgfpicture
        }%
      \or % 3: \psscalebox
        \lower.5ex\hbox{%
          \psscalebox{-1 1}{E}%
        }%
      \or % 4: \HOLOGO@Reflectbox
        \lower.5ex\hbox{%
          \HOLOGO@ReflectBox{E}%
        }%
      \else
        \@PackageError{hologo}{Internal error (choice/up)}\@ehc
      \fi
    \fi
  }%
}
%    \end{macrocode}
%    \end{macro}
%    \begin{macro}{\HoLogoHtml@Xe}
%    \begin{macrocode}
\def\HoLogoHtml@Xe#1{%
  \HoLogoCss@Xe
  \HOLOGO@Span{Xe}{%
    X%
    \HOLOGO@Span{e}{%
      \HCode{&\ltx@hashchar x018e;}%
    }%
  }%
}
%    \end{macrocode}
%    \end{macro}
%    \begin{macro}{\HoLogoCss@Xe}
%    \begin{macrocode}
\def\HoLogoCss@Xe{%
  \Css{%
    span.HoLogo-Xe span.HoLogo-e{%
      position:relative;%
      top:.5ex;%
      left-margin:-.1em;%
    }%
  }%
  \global\let\HoLogoCss@Xe\relax
}
%    \end{macrocode}
%    \end{macro}
%
%    \begin{macro}{\HoLogo@XeTeX}
%    \begin{macrocode}
\def\HoLogo@XeTeX#1{%
  \hologo{Xe}%
  \kern-.15em\relax
  \hologo{TeX}%
}
%    \end{macrocode}
%    \end{macro}
%
%    \begin{macro}{\HoLogoHtml@XeTeX}
%    \begin{macrocode}
\def\HoLogoHtml@XeTeX#1{%
  \HoLogoCss@XeTeX
  \HOLOGO@Span{XeTeX}{%
    \hologo{Xe}%
    \hologo{TeX}%
  }%
}
%    \end{macrocode}
%    \end{macro}
%    \begin{macro}{\HoLogoCss@XeTeX}
%    \begin{macrocode}
\def\HoLogoCss@XeTeX{%
  \Css{%
    span.HoLogo-XeTeX span.HoLogo-TeX{%
      margin-left:-.15em;%
    }%
  }%
  \global\let\HoLogoCss@XeTeX\relax
}
%    \end{macrocode}
%    \end{macro}
%
%    \begin{macro}{\HoLogo@XeLaTeX}
%    \begin{macrocode}
\def\HoLogo@XeLaTeX#1{%
  \hologo{Xe}%
  \kern-.13em%
  \hologo{LaTeX}%
}
%    \end{macrocode}
%    \end{macro}
%    \begin{macro}{\HoLogoHtml@XeLaTeX}
%    \begin{macrocode}
\def\HoLogoHtml@XeLaTeX#1{%
  \HoLogoCss@XeLaTeX
  \HOLOGO@Span{XeLaTeX}{%
    \hologo{Xe}%
    \hologo{LaTeX}%
  }%
}
%    \end{macrocode}
%    \end{macro}
%    \begin{macro}{\HoLogoCss@XeLaTeX}
%    \begin{macrocode}
\def\HoLogoCss@XeLaTeX{%
  \Css{%
    span.HoLogo-XeLaTeX span.HoLogo-Xe{%
      margin-right:-.13em;%
    }%
  }%
  \global\let\HoLogoCss@XeLaTeX\relax
}
%    \end{macrocode}
%    \end{macro}
%
% \subsubsection{\hologo{pdfTeX}, \hologo{pdfLaTeX}}
%
%    \begin{macro}{\HoLogo@pdfTeX}
%    \begin{macrocode}
\def\HoLogo@pdfTeX#1{%
  \HOLOGO@mbox{%
    #1{p}{P}df\hologo{TeX}%
  }%
}
%    \end{macrocode}
%    \end{macro}
%    \begin{macro}{\HoLogoCs@pdfTeX}
%    \begin{macrocode}
\def\HoLogoCs@pdfTeX#1{#1{p}{P}dfTeX}
%    \end{macrocode}
%    \end{macro}
%    \begin{macro}{\HoLogoBkm@pdfTeX}
%    \begin{macrocode}
\def\HoLogoBkm@pdfTeX#1{%
  #1{p}{P}df\hologo{TeX}%
}
%    \end{macrocode}
%    \end{macro}
%    \begin{macro}{\HoLogoHtml@pdfTeX}
%    \begin{macrocode}
\let\HoLogoHtml@pdfTeX\HoLogo@pdfTeX
%    \end{macrocode}
%    \end{macro}
%
%    \begin{macro}{\HoLogo@pdfLaTeX}
%    \begin{macrocode}
\def\HoLogo@pdfLaTeX#1{%
  \HOLOGO@mbox{%
    #1{p}{P}df\hologo{LaTeX}%
  }%
}
%    \end{macrocode}
%    \end{macro}
%    \begin{macro}{\HoLogoCs@pdfLaTeX}
%    \begin{macrocode}
\def\HoLogoCs@pdfLaTeX#1{#1{p}{P}dfLaTeX}
%    \end{macrocode}
%    \end{macro}
%    \begin{macro}{\HoLogoBkm@pdfLaTeX}
%    \begin{macrocode}
\def\HoLogoBkm@pdfLaTeX#1{%
  #1{p}{P}df\hologo{LaTeX}%
}
%    \end{macrocode}
%    \end{macro}
%    \begin{macro}{\HoLogoHtml@pdfLaTeX}
%    \begin{macrocode}
\let\HoLogoHtml@pdfLaTeX\HoLogo@pdfLaTeX
%    \end{macrocode}
%    \end{macro}
%
% \subsubsection{\hologo{VTeX}}
%
%    \begin{macro}{\HoLogo@VTeX}
%    \begin{macrocode}
\def\HoLogo@VTeX#1{%
  \HOLOGO@mbox{%
    V\hologo{TeX}%
  }%
}
%    \end{macrocode}
%    \end{macro}
%    \begin{macro}{\HoLogoHtml@VTeX}
%    \begin{macrocode}
\let\HoLogoHtml@VTeX\HoLogo@VTeX
%    \end{macrocode}
%    \end{macro}
%
% \subsubsection{\hologo{AmS}, \dots}
%
%    Source: class \xclass{amsdtx}
%
%    \begin{macro}{\HoLogo@AmS}
%    \begin{macrocode}
\def\HoLogo@AmS#1{%
  \HoLogoFont@font{AmS}{sy}{%
    A%
    \kern-.1667em%
    \lower.5ex\hbox{M}%
    \kern-.125em%
    S%
  }%
}
%    \end{macrocode}
%    \end{macro}
%    \begin{macro}{\HoLogoBkm@AmS}
%    \begin{macrocode}
\def\HoLogoBkm@AmS#1{AmS}
%    \end{macrocode}
%    \end{macro}
%    \begin{macro}{\HoLogoHtml@AmS}
%    \begin{macrocode}
\def\HoLogoHtml@AmS#1{%
  \HoLogoCss@AmS
%  \HoLogoFont@font{AmS}{sy}{%
    \HOLOGO@Span{AmS}{%
      A%
      \HOLOGO@Span{M}{M}%
      S%
    }%
%   }%
}
%    \end{macrocode}
%    \end{macro}
%    \begin{macro}{\HoLogoCss@AmS}
%    \begin{macrocode}
\def\HoLogoCss@AmS{%
  \Css{%
    span.HoLogo-AmS span.HoLogo-M{%
      position:relative;%
      top:.5ex;%
      margin-left:-.1667em;%
      margin-right:-.125em;%
      text-decoration:none;%
    }%
  }%
  \global\let\HoLogoCss@AmS\relax
}
%    \end{macrocode}
%    \end{macro}
%
%    \begin{macro}{\HoLogo@AmSTeX}
%    \begin{macrocode}
\def\HoLogo@AmSTeX#1{%
  \hologo{AmS}%
  \HOLOGO@hyphen
  \hologo{TeX}%
}
%    \end{macrocode}
%    \end{macro}
%    \begin{macro}{\HoLogoBkm@AmSTeX}
%    \begin{macrocode}
\def\HoLogoBkm@AmSTeX#1{AmS-TeX}%
%    \end{macrocode}
%    \end{macro}
%    \begin{macro}{\HoLogoHtml@AmSTeX}
%    \begin{macrocode}
\let\HoLogoHtml@AmSTeX\HoLogo@AmSTeX
%    \end{macrocode}
%    \end{macro}
%
%    \begin{macro}{\HoLogo@AmSLaTeX}
%    \begin{macrocode}
\def\HoLogo@AmSLaTeX#1{%
  \hologo{AmS}%
  \HOLOGO@hyphen
  \hologo{LaTeX}%
}
%    \end{macrocode}
%    \end{macro}
%    \begin{macro}{\HoLogoBkm@AmSLaTeX}
%    \begin{macrocode}
\def\HoLogoBkm@AmSLaTeX#1{AmS-LaTeX}%
%    \end{macrocode}
%    \end{macro}
%    \begin{macro}{\HoLogoHtml@AmSLaTeX}
%    \begin{macrocode}
\let\HoLogoHtml@AmSLaTeX\HoLogo@AmSLaTeX
%    \end{macrocode}
%    \end{macro}
%
% \subsubsection{\hologo{BibTeX}}
%
%    \begin{macro}{\HoLogo@BibTeX@sc}
%    A definition of \hologo{BibTeX} is provided in
%    the documentation source for the manual of \hologo{BibTeX}
%    \cite{btxdoc}.
%\begin{quote}
%\begin{verbatim}
%\def\BibTeX{%
%  {%
%    \rm
%    B%
%    \kern-.05em%
%    {%
%      \sc
%      i%
%      \kern-.025em %
%      b%
%    }%
%    \kern-.08em
%    T%
%    \kern-.1667em%
%    \lower.7ex\hbox{E}%
%    \kern-.125em%
%    X%
%  }%
%}
%\end{verbatim}
%\end{quote}
%    \begin{macrocode}
\def\HoLogo@BibTeX@sc#1{%
  B%
  \kern-.05em%
  \HoLogoFont@font{BibTeX}{sc}{%
    i%
    \kern-.025em%
    b%
  }%
  \HOLOGO@discretionary
  \kern-.08em%
  \hologo{TeX}%
}
%    \end{macrocode}
%    \end{macro}
%    \begin{macro}{\HoLogoHtml@BibTeX@sc}
%    \begin{macrocode}
\def\HoLogoHtml@BibTeX@sc#1{%
  \HoLogoCss@BibTeX@sc
  \HOLOGO@Span{BibTeX-sc}{%
    B%
    \HOLOGO@Span{i}{i}%
    \HOLOGO@Span{b}{b}%
    \hologo{TeX}%
  }%
}
%    \end{macrocode}
%    \end{macro}
%    \begin{macro}{\HoLogoCss@BibTeX@sc}
%    \begin{macrocode}
\def\HoLogoCss@BibTeX@sc{%
  \Css{%
    span.HoLogo-BibTeX-sc span.HoLogo-i{%
      margin-left:-.05em;%
      margin-right:-.025em;%
      font-variant:small-caps;%
    }%
  }%
  \Css{%
    span.HoLogo-BibTeX-sc span.HoLogo-b{%
      margin-right:-.08em;%
      font-variant:small-caps;%
    }%
  }%
  \global\let\HoLogoCss@BibTeX@sc\relax
}
%    \end{macrocode}
%    \end{macro}
%
%    \begin{macro}{\HoLogo@BibTeX@sf}
%    Variant \xoption{sf} avoids trouble with unavailable
%    small caps fonts (e.g., bold versions of Computer Modern or
%    Latin Modern). The definition is taken from
%    package \xpackage{dtklogos} \cite{dtklogos}.
%\begin{quote}
%\begin{verbatim}
%\DeclareRobustCommand{\BibTeX}{%
%  B%
%  \kern-.05em%
%  \hbox{%
%    $\m@th$% %% force math size calculations
%    \csname S@\f@size\endcsname
%    \fontsize\sf@size\z@
%    \math@fontsfalse
%    \selectfont
%    I%
%    \kern-.025em%
%    B
%  }%
%  \kern-.08em%
%  \-%
%  \TeX
%}
%\end{verbatim}
%\end{quote}
%    \begin{macrocode}
\def\HoLogo@BibTeX@sf#1{%
  B%
  \kern-.05em%
  \HoLogoFont@font{BibTeX}{bibsf}{%
    I%
    \kern-.025em%
    B%
  }%
  \HOLOGO@discretionary
  \kern-.08em%
  \hologo{TeX}%
}
%    \end{macrocode}
%    \end{macro}
%    \begin{macro}{\HoLogoHtml@BibTeX@sf}
%    \begin{macrocode}
\def\HoLogoHtml@BibTeX@sf#1{%
  \HoLogoCss@BibTeX@sf
  \HOLOGO@Span{BibTeX-sf}{%
    B%
    \HoLogoFont@font{BibTeX}{bibsf}{%
      \HOLOGO@Span{i}{I}%
      B%
    }%
    \hologo{TeX}%
  }%
}
%    \end{macrocode}
%    \end{macro}
%    \begin{macro}{\HoLogoCss@BibTeX@sf}
%    \begin{macrocode}
\def\HoLogoCss@BibTeX@sf{%
  \Css{%
    span.HoLogo-BibTeX-sf span.HoLogo-i{%
      margin-left:-.05em;%
      margin-right:-.025em;%
    }%
  }%
  \Css{%
    span.HoLogo-BibTeX-sf span.HoLogo-TeX{%
      margin-left:-.08em;%
    }%
  }%
  \global\let\HoLogoCss@BibTeX@sf\relax
}
%    \end{macrocode}
%    \end{macro}
%
%    \begin{macro}{\HoLogo@BibTeX}
%    \begin{macrocode}
\def\HoLogo@BibTeX{\HoLogo@BibTeX@sf}
%    \end{macrocode}
%    \end{macro}
%    \begin{macro}{\HoLogoHtml@BibTeX}
%    \begin{macrocode}
\def\HoLogoHtml@BibTeX{\HoLogoHtml@BibTeX@sf}
%    \end{macrocode}
%    \end{macro}
%
% \subsubsection{\hologo{BibTeX8}}
%
%    \begin{macro}{\HoLogo@BibTeX8}
%    \begin{macrocode}
\expandafter\def\csname HoLogo@BibTeX8\endcsname#1{%
  \hologo{BibTeX}%
  8%
}
%    \end{macrocode}
%    \end{macro}
%
%    \begin{macro}{\HoLogoBkm@BibTeX8}
%    \begin{macrocode}
\expandafter\def\csname HoLogoBkm@BibTeX8\endcsname#1{%
  \hologo{BibTeX}%
  8%
}
%    \end{macrocode}
%    \end{macro}
%    \begin{macro}{\HoLogoHtml@BibTeX8}
%    \begin{macrocode}
\expandafter
\let\csname HoLogoHtml@BibTeX8\expandafter\endcsname
\csname HoLogo@BibTeX8\endcsname
%    \end{macrocode}
%    \end{macro}
%
% \subsubsection{\hologo{ConTeXt}}
%
%    \begin{macro}{\HoLogo@ConTeXt@simple}
%    \begin{macrocode}
\def\HoLogo@ConTeXt@simple#1{%
  \HOLOGO@mbox{Con}%
  \HOLOGO@discretionary
  \HOLOGO@mbox{\hologo{TeX}t}%
}
%    \end{macrocode}
%    \end{macro}
%    \begin{macro}{\HoLogoHtml@ConTeXt@simple}
%    \begin{macrocode}
\let\HoLogoHtml@ConTeXt@simple\HoLogo@ConTeXt@simple
%    \end{macrocode}
%    \end{macro}
%
%    \begin{macro}{\HoLogo@ConTeXt@narrow}
%    This definition of logo \hologo{ConTeXt} with variant \xoption{narrow}
%    comes from TUGboat's class \xclass{ltugboat} (version 2010/11/15 v2.8).
%    \begin{macrocode}
\def\HoLogo@ConTeXt@narrow#1{%
  \HOLOGO@mbox{C\kern-.0333emon}%
  \HOLOGO@discretionary
  \kern-.0667em%
  \HOLOGO@mbox{\hologo{TeX}\kern-.0333emt}%
}
%    \end{macrocode}
%    \end{macro}
%    \begin{macro}{\HoLogoHtml@ConTeXt@narrow}
%    \begin{macrocode}
\def\HoLogoHtml@ConTeXt@narrow#1{%
  \HoLogoCss@ConTeXt@narrow
  \HOLOGO@Span{ConTeXt-narrow}{%
    \HOLOGO@Span{C}{C}%
    on%
    \hologo{TeX}%
    t%
  }%
}
%    \end{macrocode}
%    \end{macro}
%    \begin{macro}{\HoLogoCss@ConTeXt@narrow}
%    \begin{macrocode}
\def\HoLogoCss@ConTeXt@narrow{%
  \Css{%
    span.HoLogo-ConTeXt-narrow span.HoLogo-C{%
      margin-left:-.0333em;%
    }%
  }%
  \Css{%
    span.HoLogo-ConTeXt-narrow span.HoLogo-TeX{%
      margin-left:-.0667em;%
      margin-right:-.0333em;%
    }%
  }%
  \global\let\HoLogoCss@ConTeXt@narrow\relax
}
%    \end{macrocode}
%    \end{macro}
%
%    \begin{macro}{\HoLogo@ConTeXt}
%    \begin{macrocode}
\def\HoLogo@ConTeXt{\HoLogo@ConTeXt@narrow}
%    \end{macrocode}
%    \end{macro}
%    \begin{macro}{\HoLogoHtml@ConTeXt}
%    \begin{macrocode}
\def\HoLogoHtml@ConTeXt{\HoLogoHtml@ConTeXt@narrow}
%    \end{macrocode}
%    \end{macro}
%
% \subsubsection{\hologo{emTeX}}
%
%    \begin{macro}{\HoLogo@emTeX}
%    \begin{macrocode}
\def\HoLogo@emTeX#1{%
  \HOLOGO@mbox{#1{e}{E}m}%
  \HOLOGO@discretionary
  \hologo{TeX}%
}
%    \end{macrocode}
%    \end{macro}
%    \begin{macro}{\HoLogoCs@emTeX}
%    \begin{macrocode}
\def\HoLogoCs@emTeX#1{#1{e}{E}mTeX}%
%    \end{macrocode}
%    \end{macro}
%    \begin{macro}{\HoLogoBkm@emTeX}
%    \begin{macrocode}
\def\HoLogoBkm@emTeX#1{%
  #1{e}{E}m\hologo{TeX}%
}
%    \end{macrocode}
%    \end{macro}
%    \begin{macro}{\HoLogoHtml@emTeX}
%    \begin{macrocode}
\let\HoLogoHtml@emTeX\HoLogo@emTeX
%    \end{macrocode}
%    \end{macro}
%
% \subsubsection{\hologo{ExTeX}}
%
%    \begin{macro}{\HoLogo@ExTeX}
%    The definition is taken from the FAQ of the
%    project \hologo{ExTeX}
%    \cite{ExTeX-FAQ}.
%\begin{quote}
%\begin{verbatim}
%\def\ExTeX{%
%  \textrm{% Logo always with serifs
%    \ensuremath{%
%      \textstyle
%      \varepsilon_{%
%        \kern-0.15em%
%        \mathcal{X}%
%      }%
%    }%
%    \kern-.15em%
%    \TeX
%  }%
%}
%\end{verbatim}
%\end{quote}
%    \begin{macrocode}
\def\HoLogo@ExTeX#1{%
  \HoLogoFont@font{ExTeX}{rm}{%
    \ltx@mbox{%
      \HOLOGO@MathSetup
      $%
        \textstyle
        \varepsilon_{%
          \kern-0.15em%
          \HoLogoFont@font{ExTeX}{sy}{X}%
        }%
      $%
    }%
    \HOLOGO@discretionary
    \kern-.15em%
    \hologo{TeX}%
  }%
}
%    \end{macrocode}
%    \end{macro}
%    \begin{macro}{\HoLogoHtml@ExTeX}
%    \begin{macrocode}
\def\HoLogoHtml@ExTeX#1{%
  \HoLogoCss@ExTeX
  \HoLogoFont@font{ExTeX}{rm}{%
    \HOLOGO@Span{ExTeX}{%
      \ltx@mbox{%
        \HOLOGO@MathSetup
        $\textstyle\varepsilon$%
        \HOLOGO@Span{X}{$\textstyle\chi$}%
        \hologo{TeX}%
      }%
    }%
  }%
}
%    \end{macrocode}
%    \end{macro}
%    \begin{macro}{\HoLogoBkm@ExTeX}
%    \begin{macrocode}
\def\HoLogoBkm@ExTeX#1{%
  \HOLOGO@PdfdocUnicode{#1{e}{E}x}{\textepsilon\textchi}%
  \hologo{TeX}%
}
%    \end{macrocode}
%    \end{macro}
%    \begin{macro}{\HoLogoCss@ExTeX}
%    \begin{macrocode}
\def\HoLogoCss@ExTeX{%
  \Css{%
    span.HoLogo-ExTeX{%
      font-family:serif;%
    }%
  }%
  \Css{%
    span.HoLogo-ExTeX span.HoLogo-TeX{%
      margin-left:-.15em;%
    }%
  }%
  \global\let\HoLogoCss@ExTeX\relax
}
%    \end{macrocode}
%    \end{macro}
%
% \subsubsection{\hologo{MiKTeX}}
%
%    \begin{macro}{\HoLogo@MiKTeX}
%    \begin{macrocode}
\def\HoLogo@MiKTeX#1{%
  \HOLOGO@mbox{MiK}%
  \HOLOGO@discretionary
  \hologo{TeX}%
}
%    \end{macrocode}
%    \end{macro}
%    \begin{macro}{\HoLogoHtml@MiKTeX}
%    \begin{macrocode}
\let\HoLogoHtml@MiKTeX\HoLogo@MiKTeX
%    \end{macrocode}
%    \end{macro}
%
% \subsubsection{\hologo{OzTeX} and friends}
%
%    Source: \hologo{OzTeX} FAQ \cite{OzTeX}:
%    \begin{quote}
%      |\def\OzTeX{O\kern-.03em z\kern-.15em\TeX}|\\
%      (There is no kerning in OzMF, OzMP and OzTtH.)
%    \end{quote}
%
%    \begin{macro}{\HoLogo@OzTeX}
%    \begin{macrocode}
\def\HoLogo@OzTeX#1{%
  O%
  \kern-.03em %
  z%
  \kern-.15em %
  \hologo{TeX}%
}
%    \end{macrocode}
%    \end{macro}
%    \begin{macro}{\HoLogoHtml@OzTeX}
%    \begin{macrocode}
\def\HoLogoHtml@OzTeX#1{%
  \HoLogoCss@OzTeX
  \HOLOGO@Span{OzTeX}{%
    O%
    \HOLOGO@Span{z}{z}%
    \hologo{TeX}%
  }%
}
%    \end{macrocode}
%    \end{macro}
%    \begin{macro}{\HoLogoCss@OzTeX}
%    \begin{macrocode}
\def\HoLogoCss@OzTeX{%
  \Css{%
    span.HoLogo-OzTeX span.HoLogo-z{%
      margin-left:-.03em;%
      margin-right:-.15em;%
    }%
  }%
  \global\let\HoLogoCss@OzTeX\relax
}
%    \end{macrocode}
%    \end{macro}
%
%    \begin{macro}{\HoLogo@OzMF}
%    \begin{macrocode}
\def\HoLogo@OzMF#1{%
  \HOLOGO@mbox{OzMF}%
}
%    \end{macrocode}
%    \end{macro}
%    \begin{macro}{\HoLogo@OzMP}
%    \begin{macrocode}
\def\HoLogo@OzMP#1{%
  \HOLOGO@mbox{OzMP}%
}
%    \end{macrocode}
%    \end{macro}
%    \begin{macro}{\HoLogo@OzTtH}
%    \begin{macrocode}
\def\HoLogo@OzTtH#1{%
  \HOLOGO@mbox{OzTtH}%
}
%    \end{macrocode}
%    \end{macro}
%
% \subsubsection{\hologo{PCTeX}}
%
%    \begin{macro}{\HoLogo@PCTeX}
%    \begin{macrocode}
\def\HoLogo@PCTeX#1{%
  \HOLOGO@mbox{PC}%
  \hologo{TeX}%
}
%    \end{macrocode}
%    \end{macro}
%    \begin{macro}{\HoLogoHtml@PCTeX}
%    \begin{macrocode}
\let\HoLogoHtml@PCTeX\HoLogo@PCTeX
%    \end{macrocode}
%    \end{macro}
%
% \subsubsection{\hologo{PiCTeX}}
%
%    The original definitions from \xfile{pictex.tex} \cite{PiCTeX}:
%\begin{quote}
%\begin{verbatim}
%\def\PiC{%
%  P%
%  \kern-.12em%
%  \lower.5ex\hbox{I}%
%  \kern-.075em%
%  C%
%}
%\def\PiCTeX{%
%  \PiC
%  \kern-.11em%
%  \TeX
%}
%\end{verbatim}
%\end{quote}
%
%    \begin{macro}{\HoLogo@PiC}
%    \begin{macrocode}
\def\HoLogo@PiC#1{%
  P%
  \kern-.12em%
  \lower.5ex\hbox{I}%
  \kern-.075em%
  C%
  \HOLOGO@SpaceFactor
}
%    \end{macrocode}
%    \end{macro}
%    \begin{macro}{\HoLogoHtml@PiC}
%    \begin{macrocode}
\def\HoLogoHtml@PiC#1{%
  \HoLogoCss@PiC
  \HOLOGO@Span{PiC}{%
    P%
    \HOLOGO@Span{i}{I}%
    C%
  }%
}
%    \end{macrocode}
%    \end{macro}
%    \begin{macro}{\HoLogoCss@PiC}
%    \begin{macrocode}
\def\HoLogoCss@PiC{%
  \Css{%
    span.HoLogo-PiC span.HoLogo-i{%
      position:relative;%
      top:.5ex;%
      margin-left:-.12em;%
      margin-right:-.075em;%
      text-decoration:none;%
    }%
  }%
  \global\let\HoLogoCss@PiC\relax
}
%    \end{macrocode}
%    \end{macro}
%
%    \begin{macro}{\HoLogo@PiCTeX}
%    \begin{macrocode}
\def\HoLogo@PiCTeX#1{%
  \hologo{PiC}%
  \HOLOGO@discretionary
  \kern-.11em%
  \hologo{TeX}%
}
%    \end{macrocode}
%    \end{macro}
%    \begin{macro}{\HoLogoHtml@PiCTeX}
%    \begin{macrocode}
\def\HoLogoHtml@PiCTeX#1{%
  \HoLogoCss@PiCTeX
  \HOLOGO@Span{PiCTeX}{%
    \hologo{PiC}%
    \hologo{TeX}%
  }%
}
%    \end{macrocode}
%    \end{macro}
%    \begin{macro}{\HoLogoCss@PiCTeX}
%    \begin{macrocode}
\def\HoLogoCss@PiCTeX{%
  \Css{%
    span.HoLogo-PiCTeX span.HoLogo-PiC{%
      margin-right:-.11em;%
    }%
  }%
  \global\let\HoLogoCss@PiCTeX\relax
}
%    \end{macrocode}
%    \end{macro}
%
% \subsubsection{\hologo{teTeX}}
%
%    \begin{macro}{\HoLogo@teTeX}
%    \begin{macrocode}
\def\HoLogo@teTeX#1{%
  \HOLOGO@mbox{#1{t}{T}e}%
  \HOLOGO@discretionary
  \hologo{TeX}%
}
%    \end{macrocode}
%    \end{macro}
%    \begin{macro}{\HoLogoCs@teTeX}
%    \begin{macrocode}
\def\HoLogoCs@teTeX#1{#1{t}{T}dfTeX}
%    \end{macrocode}
%    \end{macro}
%    \begin{macro}{\HoLogoBkm@teTeX}
%    \begin{macrocode}
\def\HoLogoBkm@teTeX#1{%
  #1{t}{T}e\hologo{TeX}%
}
%    \end{macrocode}
%    \end{macro}
%    \begin{macro}{\HoLogoHtml@teTeX}
%    \begin{macrocode}
\let\HoLogoHtml@teTeX\HoLogo@teTeX
%    \end{macrocode}
%    \end{macro}
%
% \subsubsection{\hologo{TeX4ht}}
%
%    \begin{macro}{\HoLogo@TeX4ht}
%    \begin{macrocode}
\expandafter\def\csname HoLogo@TeX4ht\endcsname#1{%
  \HOLOGO@mbox{\hologo{TeX}4ht}%
}
%    \end{macrocode}
%    \end{macro}
%    \begin{macro}{\HoLogoHtml@TeX4ht}
%    \begin{macrocode}
\expandafter
\let\csname HoLogoHtml@TeX4ht\expandafter\endcsname
\csname HoLogo@TeX4ht\endcsname
%    \end{macrocode}
%    \end{macro}
%
%
% \subsubsection{\hologo{SageTeX}}
%
%    \begin{macro}{\HoLogo@SageTeX}
%    \begin{macrocode}
\def\HoLogo@SageTeX#1{%
  \HOLOGO@mbox{Sage}%
  \HOLOGO@discretionary
  \HOLOGO@NegativeKerning{eT,oT,To}%
  \hologo{TeX}%
}
%    \end{macrocode}
%    \end{macro}
%    \begin{macro}{\HoLogoHtml@SageTeX}
%    \begin{macrocode}
\let\HoLogoHtml@SageTeX\HoLogo@SageTeX
%    \end{macrocode}
%    \end{macro}
%
% \subsection{\hologo{METAFONT} and friends}
%
%    \begin{macro}{\HoLogo@METAFONT}
%    \begin{macrocode}
\def\HoLogo@METAFONT#1{%
  \HoLogoFont@font{METAFONT}{logo}{%
    \HOLOGO@mbox{META}%
    \HOLOGO@discretionary
    \HOLOGO@mbox{FONT}%
  }%
}
%    \end{macrocode}
%    \end{macro}
%
%    \begin{macro}{\HoLogo@METAPOST}
%    \begin{macrocode}
\def\HoLogo@METAPOST#1{%
  \HoLogoFont@font{METAPOST}{logo}{%
    \HOLOGO@mbox{META}%
    \HOLOGO@discretionary
    \HOLOGO@mbox{POST}%
  }%
}
%    \end{macrocode}
%    \end{macro}
%
%    \begin{macro}{\HoLogo@MetaFun}
%    \begin{macrocode}
\def\HoLogo@MetaFun#1{%
  \HOLOGO@mbox{Meta}%
  \HOLOGO@discretionary
  \HOLOGO@mbox{Fun}%
}
%    \end{macrocode}
%    \end{macro}
%
%    \begin{macro}{\HoLogo@MetaPost}
%    \begin{macrocode}
\def\HoLogo@MetaPost#1{%
  \HOLOGO@mbox{Meta}%
  \HOLOGO@discretionary
  \HOLOGO@mbox{Post}%
}
%    \end{macrocode}
%    \end{macro}
%
% \subsection{Others}
%
% \subsubsection{\hologo{biber}}
%
%    \begin{macro}{\HoLogo@biber}
%    \begin{macrocode}
\def\HoLogo@biber#1{%
  \HOLOGO@mbox{#1{b}{B}i}%
  \HOLOGO@discretionary
  \HOLOGO@mbox{ber}%
}
%    \end{macrocode}
%    \end{macro}
%    \begin{macro}{\HoLogoCs@biber}
%    \begin{macrocode}
\def\HoLogoCs@biber#1{#1{b}{B}iber}
%    \end{macrocode}
%    \end{macro}
%    \begin{macro}{\HoLogoBkm@biber}
%    \begin{macrocode}
\def\HoLogoBkm@biber#1{%
  #1{b}{B}iber%
}
%    \end{macrocode}
%    \end{macro}
%    \begin{macro}{\HoLogoHtml@biber}
%    \begin{macrocode}
\let\HoLogoHtml@biber\HoLogo@biber
%    \end{macrocode}
%    \end{macro}
%
% \subsubsection{\hologo{KOMAScript}}
%
%    \begin{macro}{\HoLogo@KOMAScript}
%    The definition for \hologo{KOMAScript} is taken
%    from \hologo{KOMAScript} (\xfile{scrlogo.dtx}, reformatted) \cite{scrlogo}:
%\begin{quote}
%\begin{verbatim}
%\@ifundefined{KOMAScript}{%
%  \DeclareRobustCommand{\KOMAScript}{%
%    \textsf{%
%      K\kern.05em O\kern.05emM\kern.05em A%
%      \kern.1em-\kern.1em %
%      Script%
%    }%
%  }%
%}{}
%\end{verbatim}
%\end{quote}
%    \begin{macrocode}
\def\HoLogo@KOMAScript#1{%
  \HoLogoFont@font{KOMAScript}{sf}{%
    \HOLOGO@mbox{%
      K\kern.05em%
      O\kern.05em%
      M\kern.05em%
      A%
    }%
    \kern.1em%
    \HOLOGO@hyphen
    \kern.1em%
    \HOLOGO@mbox{Script}%
  }%
}
%    \end{macrocode}
%    \end{macro}
%    \begin{macro}{\HoLogoBkm@KOMAScript}
%    \begin{macrocode}
\def\HoLogoBkm@KOMAScript#1{%
  KOMA-Script%
}
%    \end{macrocode}
%    \end{macro}
%    \begin{macro}{\HoLogoHtml@KOMAScript}
%    \begin{macrocode}
\def\HoLogoHtml@KOMAScript#1{%
  \HoLogoCss@KOMAScript
  \HoLogoFont@font{KOMAScript}{sf}{%
    \HOLOGO@Span{KOMAScript}{%
      K%
      \HOLOGO@Span{O}{O}%
      M%
      \HOLOGO@Span{A}{A}%
      \HOLOGO@Span{hyphen}{-}%
      Script%
    }%
  }%
}
%    \end{macrocode}
%    \end{macro}
%    \begin{macro}{\HoLogoCss@KOMAScript}
%    \begin{macrocode}
\def\HoLogoCss@KOMAScript{%
  \Css{%
    span.HoLogo-KOMAScript{%
      font-family:sans-serif;%
    }%
  }%
  \Css{%
    span.HoLogo-KOMAScript span.HoLogo-O{%
      padding-left:.05em;%
      padding-right:.05em;%
    }%
  }%
  \Css{%
    span.HoLogo-KOMAScript span.HoLogo-A{%
      padding-left:.05em;%
    }%
  }%
  \Css{%
    span.HoLogo-KOMAScript span.HoLogo-hyphen{%
      padding-left:.1em;%
      padding-right:.1em;%
    }%
  }%
  \global\let\HoLogoCss@KOMAScript\relax
}
%    \end{macrocode}
%    \end{macro}
%
% \subsubsection{\hologo{LyX}}
%
%    \begin{macro}{\HoLogo@LyX}
%    The definition is taken from the documentation source files
%    of \hologo{LyX}, \xfile{Intro.lyx} \cite{LyX}:
%\begin{quote}
%\begin{verbatim}
%\def\LyX{%
%  \texorpdfstring{%
%    L\kern-.1667em\lower.25em\hbox{Y}\kern-.125emX\@%
%  }{%
%    LyX%
%  }%
%}
%\end{verbatim}
%\end{quote}
%    \begin{macrocode}
\def\HoLogo@LyX#1{%
  L%
  \kern-.1667em%
  \lower.25em\hbox{Y}%
  \kern-.125em%
  X%
  \HOLOGO@SpaceFactor
}
%    \end{macrocode}
%    \end{macro}
%    \begin{macro}{\HoLogoHtml@LyX}
%    \begin{macrocode}
\def\HoLogoHtml@LyX#1{%
  \HoLogoCss@LyX
  \HOLOGO@Span{LyX}{%
    L%
    \HOLOGO@Span{y}{Y}%
    X%
  }%
}
%    \end{macrocode}
%    \end{macro}
%    \begin{macro}{\HoLogoCss@LyX}
%    \begin{macrocode}
\def\HoLogoCss@LyX{%
  \Css{%
    span.HoLogo-LyX span.HoLogo-y{%
      position:relative;%
      top:.25em;%
      margin-left:-.1667em;%
      margin-right:-.125em;%
      text-decoration:none;%
    }%
  }%
  \global\let\HoLogoCss@LyX\relax
}
%    \end{macrocode}
%    \end{macro}
%
% \subsubsection{\hologo{NTS}}
%
%    \begin{macro}{\HoLogo@NTS}
%    Definition for \hologo{NTS} can be found in
%    package \xpackage{etex\textunderscore man} for the \hologo{eTeX} manual \cite{etexman}
%    and in package \xpackage{dtklogos} \cite{dtklogos}:
%\begin{quote}
%\begin{verbatim}
%\def\NTS{%
%  \leavevmode
%  \hbox{%
%    $%
%      \cal N%
%      \kern-0.35em%
%      \lower0.5ex\hbox{$\cal T$}%
%      \kern-0.2em%
%      S%
%    $%
%  }%
%}
%\end{verbatim}
%\end{quote}
%    \begin{macrocode}
\def\HoLogo@NTS#1{%
  \HoLogoFont@font{NTS}{sy}{%
    N\/%
    \kern-.35em%
    \lower.5ex\hbox{T\/}%
    \kern-.2em%
    S\/%
  }%
  \HOLOGO@SpaceFactor
}
%    \end{macrocode}
%    \end{macro}
%
% \subsubsection{\Hologo{TTH} (\hologo{TeX} to HTML translator)}
%
%    Source: \url{http://hutchinson.belmont.ma.us/tth/}
%    In the HTML source the second `T' is printed as subscript.
%\begin{quote}
%\begin{verbatim}
%T<sub>T</sub>H
%\end{verbatim}
%\end{quote}
%    \begin{macro}{\HoLogo@TTH}
%    \begin{macrocode}
\def\HoLogo@TTH#1{%
  \ltx@mbox{%
    T\HOLOGO@SubScript{T}H%
  }%
  \HOLOGO@SpaceFactor
}
%    \end{macrocode}
%    \end{macro}
%
%    \begin{macro}{\HoLogoHtml@TTH}
%    \begin{macrocode}
\def\HoLogoHtml@TTH#1{%
  T\HCode{<sub>}T\HCode{</sub>}H%
}
%    \end{macrocode}
%    \end{macro}
%
% \subsubsection{\Hologo{HanTheThanh}}
%
%    Partial source: Package \xpackage{dtklogos}.
%    The double accent is U+1EBF (latin small letter e with circumflex
%    and acute).
%    \begin{macro}{\HoLogo@HanTheThanh}
%    \begin{macrocode}
\def\HoLogo@HanTheThanh#1{%
  \ltx@mbox{H\`an}%
  \HOLOGO@space
  \ltx@mbox{%
    Th%
    \HOLOGO@IfCharExists{"1EBF}{%
      \char"1EBF\relax
    }{%
      \^e\hbox to 0pt{\hss\raise .5ex\hbox{\'{}}}%
    }%
  }%
  \HOLOGO@space
  \ltx@mbox{Th\`anh}%
}
%    \end{macrocode}
%    \end{macro}
%    \begin{macro}{\HoLogoBkm@HanTheThanh}
%    \begin{macrocode}
\def\HoLogoBkm@HanTheThanh#1{%
  H\`an %
  Th\HOLOGO@PdfdocUnicode{\^e}{\9036\277} %
  Th\`anh%
}
%    \end{macrocode}
%    \end{macro}
%    \begin{macro}{\HoLogoHtml@HanTheThanh}
%    \begin{macrocode}
\def\HoLogoHtml@HanTheThanh#1{%
  H\`an %
  Th\HCode{&\ltx@hashchar x1ebf;} %
  Th\`anh%
}
%    \end{macrocode}
%    \end{macro}
%
% \subsection{Driver detection}
%
%    \begin{macrocode}
\HOLOGO@IfExists\InputIfFileExists{%
  \InputIfFileExists{hologo.cfg}{}{}%
}{%
  \ltx@IfUndefined{pdf@filesize}{%
    \def\HOLOGO@InputIfExists{%
      \openin\HOLOGO@temp=hologo.cfg\relax
      \ifeof\HOLOGO@temp
        \closein\HOLOGO@temp
      \else
        \closein\HOLOGO@temp
        \begingroup
          \def\x{LaTeX2e}%
        \expandafter\endgroup
        \ifx\fmtname\x
          \input{hologo.cfg}%
        \else
          \input hologo.cfg\relax
        \fi
      \fi
    }%
    \ltx@IfUndefined{newread}{%
      \chardef\HOLOGO@temp=15 %
      \def\HOLOGO@CheckRead{%
        \ifeof\HOLOGO@temp
          \HOLOGO@InputIfExists
        \else
          \ifcase\HOLOGO@temp
            \@PackageWarningNoLine{hologo}{%
              Configuration file ignored, because\MessageBreak
              a free read register could not be found%
            }%
          \else
            \begingroup
              \count\ltx@cclv=\HOLOGO@temp
              \advance\ltx@cclv by \ltx@minusone
              \edef\x{\endgroup
                \chardef\noexpand\HOLOGO@temp=\the\count\ltx@cclv
                \relax
              }%
            \x
          \fi
        \fi
      }%
    }{%
      \csname newread\endcsname\HOLOGO@temp
      \HOLOGO@InputIfExists
    }%
  }{%
    \edef\HOLOGO@temp{\pdf@filesize{hologo.cfg}}%
    \ifx\HOLOGO@temp\ltx@empty
    \else
      \ifnum\HOLOGO@temp>0 %
        \begingroup
          \def\x{LaTeX2e}%
        \expandafter\endgroup
        \ifx\fmtname\x
          \input{hologo.cfg}%
        \else
          \input hologo.cfg\relax
        \fi
      \else
        \@PackageInfoNoLine{hologo}{%
          Empty configuration file `hologo.cfg' ignored%
        }%
      \fi
    \fi
  }%
}
%    \end{macrocode}
%
%    \begin{macrocode}
\def\HOLOGO@temp#1#2{%
  \kv@define@key{HoLogoDriver}{#1}[]{%
    \begingroup
      \def\HOLOGO@temp{##1}%
      \ltx@onelevel@sanitize\HOLOGO@temp
      \ifx\HOLOGO@temp\ltx@empty
      \else
        \@PackageError{hologo}{%
          Value (\HOLOGO@temp) not permitted for option `#1'%
        }%
        \@ehc
      \fi
    \endgroup
    \def\hologoDriver{#2}%
  }%
}%
\def\HOLOGO@@temp#1#2{%
  \ifx\kv@value\relax
    \HOLOGO@temp{#1}{#1}%
  \else
    \HOLOGO@temp{#1}{#2}%
  \fi
}%
\kv@parse@normalized{%
  pdftex,%
  luatex=pdftex,%
  dvipdfm,%
  dvipdfmx=dvipdfm,%
  dvips,%
  dvipsone=dvips,%
  xdvi=dvips,%
  xetex,%
  vtex,%
}\HOLOGO@@temp
%    \end{macrocode}
%
%    \begin{macrocode}
\kv@define@key{HoLogoDriver}{driverfallback}{%
  \def\HOLOGO@DriverFallback{#1}%
}
%    \end{macrocode}
%
%    \begin{macro}{\HOLOGO@DriverFallback}
%    \begin{macrocode}
\def\HOLOGO@DriverFallback{dvips}
%    \end{macrocode}
%    \end{macro}
%
%    \begin{macro}{\hologoDriverSetup}
%    \begin{macrocode}
\def\hologoDriverSetup{%
  \let\hologoDriver\ltx@undefined
  \HOLOGO@DriverSetup
}
%    \end{macrocode}
%    \end{macro}
%
%    \begin{macro}{\HOLOGO@DriverSetup}
%    \begin{macrocode}
\def\HOLOGO@DriverSetup#1{%
  \kvsetkeys{HoLogoDriver}{#1}%
  \HOLOGO@CheckDriver
  \ltx@ifundefined{hologoDriver}{%
    \begingroup
    \edef\x{\endgroup
      \noexpand\kvsetkeys{HoLogoDriver}{\HOLOGO@DriverFallback}%
    }\x
  }{}%
  \@PackageInfoNoLine{hologo}{Using driver `\hologoDriver'}%
}
%    \end{macrocode}
%    \end{macro}
%
%    \begin{macro}{\HOLOGO@CheckDriver}
%    \begin{macrocode}
\def\HOLOGO@CheckDriver{%
  \ifpdf
    \def\hologoDriver{pdftex}%
    \let\HOLOGO@pdfliteral\pdfliteral
    \ifluatex
      \ifx\pdfextension\@undefined\else
        \protected\def\pdfliteral{\pdfextension literal}%
        \let\HOLOGO@pdfliteral\pdfliteral
      \fi
      \ltx@IfUndefined{HOLOGO@pdfliteral}{%
        \ifnum\luatexversion<36 %
        \else
          \begingroup
            \let\HOLOGO@temp\endgroup
            \ifcase0%
                \directlua{%
                  if tex.enableprimitives then %
                    tex.enableprimitives('HOLOGO@', {'pdfliteral'})%
                  else %
                    tex.print('1')%
                  end%
                }%
                \ifx\HOLOGO@pdfliteral\@undefined 1\fi%
                \relax%
              \endgroup
              \let\HOLOGO@temp\relax
              \global\let\HOLOGO@pdfliteral\HOLOGO@pdfliteral
            \fi%
          \HOLOGO@temp
        \fi
      }{}%
    \fi
    \ltx@IfUndefined{HOLOGO@pdfliteral}{%
      \@PackageWarningNoLine{hologo}{%
        Cannot find \string\pdfliteral
      }%
    }{}%
  \else
    \ifxetex
      \def\hologoDriver{xetex}%
    \else
      \ifvtex
        \def\hologoDriver{vtex}%
      \fi
    \fi
  \fi
}
%    \end{macrocode}
%    \end{macro}
%
%    \begin{macro}{\HOLOGO@WarningUnsupportedDriver}
%    \begin{macrocode}
\def\HOLOGO@WarningUnsupportedDriver#1{%
  \@PackageWarningNoLine{hologo}{%
    Logo `#1' needs driver specific macros,\MessageBreak
    but driver `\hologoDriver' is not supported.\MessageBreak
    Use a different driver or\MessageBreak
    load package `graphics' or `pgf'%
  }%
}
%    \end{macrocode}
%    \end{macro}
%
% \subsubsection{Reflect box macros}
%
%    Skip driver part if not needed.
%    \begin{macrocode}
\ltx@IfUndefined{reflectbox}{}{%
  \ltx@IfUndefined{rotatebox}{}{%
    \HOLOGO@AtEnd
  }%
}
\ltx@IfUndefined{pgftext}{}{%
  \HOLOGO@AtEnd
}
\ltx@IfUndefined{psscalebox}{}{%
  \HOLOGO@AtEnd
}
%    \end{macrocode}
%
%    \begin{macrocode}
\def\HOLOGO@temp{LaTeX2e}
\ifx\fmtname\HOLOGO@temp
  \RequirePackage{kvoptions}[2011/06/30]%
  \ProcessKeyvalOptions{HoLogoDriver}%
\fi
\HOLOGO@DriverSetup{}
%    \end{macrocode}
%
%    \begin{macro}{\HOLOGO@ReflectBox}
%    \begin{macrocode}
\def\HOLOGO@ReflectBox#1{%
  \begingroup
    \setbox\ltx@zero\hbox{\begingroup#1\endgroup}%
    \setbox\ltx@two\hbox{%
      \kern\wd\ltx@zero
      \csname HOLOGO@ScaleBox@\hologoDriver\endcsname{-1}{1}{%
        \hbox to 0pt{\copy\ltx@zero\hss}%
      }%
    }%
    \wd\ltx@two=\wd\ltx@zero
    \box\ltx@two
  \endgroup
}
%    \end{macrocode}
%    \end{macro}
%
%    \begin{macro}{\HOLOGO@PointReflectBox}
%    \begin{macrocode}
\def\HOLOGO@PointReflectBox#1{%
  \begingroup
    \setbox\ltx@zero\hbox{\begingroup#1\endgroup}%
    \setbox\ltx@two\hbox{%
      \kern\wd\ltx@zero
      \raise\ht\ltx@zero\hbox{%
        \csname HOLOGO@ScaleBox@\hologoDriver\endcsname{-1}{-1}{%
          \hbox to 0pt{\copy\ltx@zero\hss}%
        }%
      }%
    }%
    \wd\ltx@two=\wd\ltx@zero
    \box\ltx@two
  \endgroup
}
%    \end{macrocode}
%    \end{macro}
%
%    We must define all variants because of dynamic driver setup.
%    \begin{macrocode}
\def\HOLOGO@temp#1#2{#2}
%    \end{macrocode}
%
%    \begin{macro}{\HOLOGO@ScaleBox@pdftex}
%    \begin{macrocode}
\HOLOGO@temp{pdftex}{%
  \def\HOLOGO@ScaleBox@pdftex#1#2#3{%
    \HOLOGO@pdfliteral{%
      q #1 0 0 #2 0 0 cm%
    }%
    #3%
    \HOLOGO@pdfliteral{%
      Q%
    }%
  }%
}
%    \end{macrocode}
%    \end{macro}
%    \begin{macro}{\HOLOGO@ScaleBox@dvips}
%    \begin{macrocode}
\HOLOGO@temp{dvips}{%
  \def\HOLOGO@ScaleBox@dvips#1#2#3{%
    \special{ps:%
      gsave %
      currentpoint %
      currentpoint translate %
      #1 #2 scale %
      neg exch neg exch translate%
    }%
    #3%
    \special{ps:%
      currentpoint %
      grestore %
      moveto%
    }%
  }%
}
%    \end{macrocode}
%    \end{macro}
%    \begin{macro}{\HOLOGO@ScaleBox@dvipdfm}
%    \begin{macrocode}
\HOLOGO@temp{dvipdfm}{%
  \let\HOLOGO@ScaleBox@dvipdfm\HOLOGO@ScaleBox@dvips
}
%    \end{macrocode}
%    \end{macro}
%    Since \hologo{XeTeX} v0.6.
%    \begin{macro}{\HOLOGO@ScaleBox@xetex}
%    \begin{macrocode}
\HOLOGO@temp{xetex}{%
  \def\HOLOGO@ScaleBox@xetex#1#2#3{%
    \special{x:gsave}%
    \special{x:scale #1 #2}%
    #3%
    \special{x:grestore}%
  }%
}
%    \end{macrocode}
%    \end{macro}
%    \begin{macro}{\HOLOGO@ScaleBox@vtex}
%    \begin{macrocode}
\HOLOGO@temp{vtex}{%
  \def\HOLOGO@ScaleBox@vtex#1#2#3{%
    \special{r(#1,0,0,#2,0,0}%
    #3%
    \special{r)}%
  }%
}
%    \end{macrocode}
%    \end{macro}
%
%    \begin{macrocode}
\HOLOGO@AtEnd%
%</package>
%    \end{macrocode}
%
% \section{Test}
%
% \subsection{Catcode checks for loading}
%
%    \begin{macrocode}
%<*test1>
%    \end{macrocode}
%    \begin{macrocode}
\catcode`\{=1 %
\catcode`\}=2 %
\catcode`\#=6 %
\catcode`\@=11 %
\expandafter\ifx\csname count@\endcsname\relax
  \countdef\count@=255 %
\fi
\expandafter\ifx\csname @gobble\endcsname\relax
  \long\def\@gobble#1{}%
\fi
\expandafter\ifx\csname @firstofone\endcsname\relax
  \long\def\@firstofone#1{#1}%
\fi
\expandafter\ifx\csname loop\endcsname\relax
  \expandafter\@firstofone
\else
  \expandafter\@gobble
\fi
{%
  \def\loop#1\repeat{%
    \def\body{#1}%
    \iterate
  }%
  \def\iterate{%
    \body
      \let\next\iterate
    \else
      \let\next\relax
    \fi
    \next
  }%
  \let\repeat=\fi
}%
\def\RestoreCatcodes{}
\count@=0 %
\loop
  \edef\RestoreCatcodes{%
    \RestoreCatcodes
    \catcode\the\count@=\the\catcode\count@\relax
  }%
\ifnum\count@<255 %
  \advance\count@ 1 %
\repeat

\def\RangeCatcodeInvalid#1#2{%
  \count@=#1\relax
  \loop
    \catcode\count@=15 %
  \ifnum\count@<#2\relax
    \advance\count@ 1 %
  \repeat
}
\def\RangeCatcodeCheck#1#2#3{%
  \count@=#1\relax
  \loop
    \ifnum#3=\catcode\count@
    \else
      \errmessage{%
        Character \the\count@\space
        with wrong catcode \the\catcode\count@\space
        instead of \number#3%
      }%
    \fi
  \ifnum\count@<#2\relax
    \advance\count@ 1 %
  \repeat
}
\def\space{ }
\expandafter\ifx\csname LoadCommand\endcsname\relax
  \def\LoadCommand{\input hologo.sty\relax}%
\fi
\def\Test{%
  \RangeCatcodeInvalid{0}{47}%
  \RangeCatcodeInvalid{58}{64}%
  \RangeCatcodeInvalid{91}{96}%
  \RangeCatcodeInvalid{123}{255}%
  \catcode`\@=12 %
  \catcode`\\=0 %
  \catcode`\%=14 %
  \LoadCommand
  \RangeCatcodeCheck{0}{36}{15}%
  \RangeCatcodeCheck{37}{37}{14}%
  \RangeCatcodeCheck{38}{47}{15}%
  \RangeCatcodeCheck{48}{57}{12}%
  \RangeCatcodeCheck{58}{63}{15}%
  \RangeCatcodeCheck{64}{64}{12}%
  \RangeCatcodeCheck{65}{90}{11}%
  \RangeCatcodeCheck{91}{91}{15}%
  \RangeCatcodeCheck{92}{92}{0}%
  \RangeCatcodeCheck{93}{96}{15}%
  \RangeCatcodeCheck{97}{122}{11}%
  \RangeCatcodeCheck{123}{255}{15}%
  \RestoreCatcodes
}
\Test
\csname @@end\endcsname
\end
%    \end{macrocode}
%    \begin{macrocode}
%</test1>
%    \end{macrocode}
%
% \subsection{Spacefactor}
%
%    The space factor must be 1000 after a logo. If it is greater 1000
%    then the following space is a space after a sentence closing point.
%    If the space factor is smaller 1000 then an immediate following
%    dot is interpreted as abbreviation, not sentence closing point.
%
%    \begin{macrocode}
%<*test-spacefactor>
\NeedsTeXFormat{LaTeX2e}
\documentclass{article}
\usepackage{hologo}[2016/05/12]
\usepackage{kvsetkeys}
\usepackage{qstest}
\IncludeTests{*}
\LogTests{log}{*}{*}
\begin{document}
\begin{qstest}{spacefactor}{spacefactor}
\newcommand*{\Test}[1]{%
  \sbox0{%
    \hologo{#1}%
    \Expect*{1000 (#1)}*{\the\spacefactor\space(#1)}%
  }%
}%
\makeatletter
\def\TestList{}
\def\hologoEntry#1#2#3{%
  \edef\TestList{%
    \ifx\TestList\@empty
    \else
      \TestList,%
    \fi
    #1%
    \ifx\\#2\\%
    \else
      ={variant=#2}%
    \fi
  }%
}
\hologoList
\expandafter\kv@parse@normalized\expandafter{%
  \TestList
}{%
  \begingroup
    \let\@logo=\kv@key
    \ifx\kv@value\relax
    \else
      \expandafter\hologoLogoSetup\expandafter\@logo\expandafter{%
        \kv@value
      }%
    \fi
    \Test\@logo
  \endgroup
  \@gobbletwo
}
\end{qstest}
\end{document}
%</test-spacefactor>
%    \end{macrocode}
%
% \subsection{Complete list}
%
%    \begin{macrocode}
%<*test-list>
\NeedsTeXFormat{LaTeX2e}
\documentclass[12pt,a4paper]{article}
\usepackage{hologo}[2016/05/12]
\usepackage[T1]{fontenc}
\usepackage{lmodern}
\usepackage{parskip}
\usepackage[unicode]{hyperref}[2011/09/28]
\usepackage{bookmark}[2011/09/19]
\bookmarksetup{%
  numbered,%
  open,%
  openlevel=2,%
}
\renewcommand*{\contentsname}{List of logos}
\begin{document}
\tableofcontents
\def\TestFont#1#2#3#4#5#6{%
  \begingroup
    \usefont{#3}{#4}{#5}{#6}%
    \HologoVariant{#1}{#2}/\hologoVariant{#1}{#2}%
    \quad
    \begingroup\scriptsize\hologoVariant{#1}{#2}\endgroup
    \quad
  \endgroup
  (#3/#4/#5/#6)%
  \par
}
\makeatletter
\def\hologoEntry#1#2#3{%
  \section{%
    \HologoVariant{#1}{#2}/\hologoVariant{#1}{#2} %
    {[#1\ifx\\#2\\\else\space(#2)\fi]}% hash-ok
  }% braces around [] because of bug in tex4ht
  \begingroup
    \hypersetup{unicode=false}%
    \bookmark[%
      dest=\@currentHref,%
      rellevel=1,%
      keeplevel,%
    ]{%
      \HologoVariant{#1}{#2}/\hologoVariant{#1}{#2} %
      (PDFDocEncoding)%
    }%
  \endgroup
  \TestFont{#1}{#2}{OT1}{cmr}{m}{n}%
  \TestFont{#1}{#2}{OT1}{cmss}{m}{n}%
  \TestFont{#1}{#2}{OT1}{cmr}{b}{n}%
  \TestFont{#1}{#2}{OT1}{cmr}{m}{it}%
  \TestFont{#1}{#2}{OT1}{cmtt}{m}{n}%
  \TestFont{#1}{#2}{T1}{lmr}{m}{n}%
  \TestFont{#1}{#2}{T1}{lmss}{m}{n}%
  \TestFont{#1}{#2}{T1}{lmr}{b}{n}%
  \TestFont{#1}{#2}{T1}{lmr}{m}{it}%
  \TestFont{#1}{#2}{T1}{lmtt}{m}{n}%
  \TestFont{#1}{#2}{T1}{lmvtt}{m}{n}%
  \TestFont{#1}{#2}{T1}{qtm}{m}{n}%
  \TestFont{#1}{#2}{T1}{qhv}{m}{n}%
  \TestFont{#1}{#2}{T1}{qtm}{b}{n}%
  \TestFont{#1}{#2}{T1}{qtm}{m}{it}%
  \TestFont{#1}{#2}{T1}{qcr}{m}{n}%
  \newpage
}
\makeatother
\hologoList
\end{document}
%</test-list>
%    \end{macrocode}
%
% \section{Installation}
%
% \subsection{Download}
%
% \paragraph{Package.} This package is available on
% CTAN\footnote{\url{ftp://ftp.ctan.org/tex-archive/}}:
% \begin{description}
% \item[\CTAN{macros/latex/contrib/oberdiek/hologo.dtx}] The source file.
% \item[\CTAN{macros/latex/contrib/oberdiek/hologo.pdf}] Documentation.
% \end{description}
%
%
% \paragraph{Bundle.} All the packages of the bundle `oberdiek'
% are also available in a TDS compliant ZIP archive. There
% the packages are already unpacked and the documentation files
% are generated. The files and directories obey the TDS standard.
% \begin{description}
% \item[\CTAN{install/macros/latex/contrib/oberdiek.tds.zip}]
% \end{description}
% \emph{TDS} refers to the standard ``A Directory Structure
% for \TeX\ Files'' (\CTAN{tds/tds.pdf}). Directories
% with \xfile{texmf} in their name are usually organized this way.
%
% \subsection{Bundle installation}
%
% \paragraph{Unpacking.} Unpack the \xfile{oberdiek.tds.zip} in the
% TDS tree (also known as \xfile{texmf} tree) of your choice.
% Example (linux):
% \begin{quote}
%   |unzip oberdiek.tds.zip -d ~/texmf|
% \end{quote}
%
% \paragraph{Script installation.}
% Check the directory \xfile{TDS:scripts/oberdiek/} for
% scripts that need further installation steps.
% Package \xpackage{attachfile2} comes with the Perl script
% \xfile{pdfatfi.pl} that should be installed in such a way
% that it can be called as \texttt{pdfatfi}.
% Example (linux):
% \begin{quote}
%   |chmod +x scripts/oberdiek/pdfatfi.pl|\\
%   |cp scripts/oberdiek/pdfatfi.pl /usr/local/bin/|
% \end{quote}
%
% \subsection{Package installation}
%
% \paragraph{Unpacking.} The \xfile{.dtx} file is a self-extracting
% \docstrip\ archive. The files are extracted by running the
% \xfile{.dtx} through \plainTeX:
% \begin{quote}
%   \verb|tex hologo.dtx|
% \end{quote}
%
% \paragraph{TDS.} Now the different files must be moved into
% the different directories in your installation TDS tree
% (also known as \xfile{texmf} tree):
% \begin{quote}
% \def\t{^^A
% \begin{tabular}{@{}>{\ttfamily}l@{ $\rightarrow$ }>{\ttfamily}l@{}}
%   hologo.sty & tex/generic/oberdiek/hologo.sty\\
%   hologo.pdf & doc/latex/oberdiek/hologo.pdf\\
%   example/hologo-example.tex & doc/latex/oberdiek/example/hologo-example.tex\\
%   test/hologo-test1.tex & doc/latex/oberdiek/test/hologo-test1.tex\\
%   test/hologo-test-spacefactor.tex & doc/latex/oberdiek/test/hologo-test-spacefactor.tex\\
%   test/hologo-test-list.tex & doc/latex/oberdiek/test/hologo-test-list.tex\\
%   hologo.dtx & source/latex/oberdiek/hologo.dtx\\
% \end{tabular}^^A
% }^^A
% \sbox0{\t}^^A
% \ifdim\wd0>\linewidth
%   \begingroup
%     \advance\linewidth by\leftmargin
%     \advance\linewidth by\rightmargin
%   \edef\x{\endgroup
%     \def\noexpand\lw{\the\linewidth}^^A
%   }\x
%   \def\lwbox{^^A
%     \leavevmode
%     \hbox to \linewidth{^^A
%       \kern-\leftmargin\relax
%       \hss
%       \usebox0
%       \hss
%       \kern-\rightmargin\relax
%     }^^A
%   }^^A
%   \ifdim\wd0>\lw
%     \sbox0{\small\t}^^A
%     \ifdim\wd0>\linewidth
%       \ifdim\wd0>\lw
%         \sbox0{\footnotesize\t}^^A
%         \ifdim\wd0>\linewidth
%           \ifdim\wd0>\lw
%             \sbox0{\scriptsize\t}^^A
%             \ifdim\wd0>\linewidth
%               \ifdim\wd0>\lw
%                 \sbox0{\tiny\t}^^A
%                 \ifdim\wd0>\linewidth
%                   \lwbox
%                 \else
%                   \usebox0
%                 \fi
%               \else
%                 \lwbox
%               \fi
%             \else
%               \usebox0
%             \fi
%           \else
%             \lwbox
%           \fi
%         \else
%           \usebox0
%         \fi
%       \else
%         \lwbox
%       \fi
%     \else
%       \usebox0
%     \fi
%   \else
%     \lwbox
%   \fi
% \else
%   \usebox0
% \fi
% \end{quote}
% If you have a \xfile{docstrip.cfg} that configures and enables \docstrip's
% TDS installing feature, then some files can already be in the right
% place, see the documentation of \docstrip.
%
% \subsection{Refresh file name databases}
%
% If your \TeX~distribution
% (\teTeX, \mikTeX, \dots) relies on file name databases, you must refresh
% these. For example, \teTeX\ users run \verb|texhash| or
% \verb|mktexlsr|.
%
% \subsection{Some details for the interested}
%
% \paragraph{Attached source.}
%
% The PDF documentation on CTAN also includes the
% \xfile{.dtx} source file. It can be extracted by
% AcrobatReader 6 or higher. Another option is \textsf{pdftk},
% e.g. unpack the file into the current directory:
% \begin{quote}
%   \verb|pdftk hologo.pdf unpack_files output .|
% \end{quote}
%
% \paragraph{Unpacking with \LaTeX.}
% The \xfile{.dtx} chooses its action depending on the format:
% \begin{description}
% \item[\plainTeX:] Run \docstrip\ and extract the files.
% \item[\LaTeX:] Generate the documentation.
% \end{description}
% If you insist on using \LaTeX\ for \docstrip\ (really,
% \docstrip\ does not need \LaTeX), then inform the autodetect routine
% about your intention:
% \begin{quote}
%   \verb|latex \let\install=y\input{hologo.dtx}|
% \end{quote}
% Do not forget to quote the argument according to the demands
% of your shell.
%
% \paragraph{Generating the documentation.}
% You can use both the \xfile{.dtx} or the \xfile{.drv} to generate
% the documentation. The process can be configured by the
% configuration file \xfile{ltxdoc.cfg}. For instance, put this
% line into this file, if you want to have A4 as paper format:
% \begin{quote}
%   \verb|\PassOptionsToClass{a4paper}{article}|
% \end{quote}
% An example follows how to generate the
% documentation with pdf\LaTeX:
% \begin{quote}
%\begin{verbatim}
%pdflatex hologo.dtx
%makeindex -s gind.ist hologo.idx
%pdflatex hologo.dtx
%makeindex -s gind.ist hologo.idx
%pdflatex hologo.dtx
%\end{verbatim}
% \end{quote}
%
% \section{Catalogue}
%
% The following XML file can be used as source for the
% \href{http://mirror.ctan.org/help/Catalogue/catalogue.html}{\TeX\ Catalogue}.
% The elements \texttt{caption} and \texttt{description} are imported
% from the original XML file from the Catalogue.
% The name of the XML file in the Catalogue is \xfile{hologo.xml}.
%    \begin{macrocode}
%<*catalogue>
<?xml version='1.0' encoding='us-ascii'?>
<!DOCTYPE entry SYSTEM 'catalogue.dtd'>
<entry datestamp='$Date$' modifier='$Author$' id='hologo'>
  <name>hologo</name>
  <caption>A collection of logos with bookmark support.</caption>
  <authorref id='auth:oberdiek'/>
  <copyright owner='Heiko Oberdiek' year='2010-2012'/>
  <license type='lppl1.3'/>
  <version number='1.10'/>
  <description>
    The package defines a single command <tt>\hologo</tt>, whose
    argument is the usual case-confused ASCII version of the logo.
    The command is bookmark-enabled, so that every logo becomes
    available in bookmarks without further work.
    <p/>
    The package is part of the <xref refid='oberdiek'>oberdiek</xref>
    bundle.
  </description>
  <documentation details='Package documentation'
      href='ctan:/macros/latex/contrib/oberdiek/hologo.pdf'/>
  <ctan file='true' path='/macros/latex/contrib/oberdiek/hologo.dtx'/>
  <miktex location='oberdiek'/>
  <texlive location='oberdiek'/>
  <install path='/macros/latex/contrib/oberdiek/oberdiek.tds.zip'/>
</entry>
%</catalogue>
%    \end{macrocode}
%
% \begin{thebibliography}{9}
% \raggedright
%
% \bibitem{btxdoc}
% Oren Patashnik,
% \textit{\hologo{BibTeX}ing},
% 1988-02-08.\\
% \CTAN{biblio/bibtex/base/}
%
% \bibitem{dtklogos}
% Gerd Neugebauer, DANTE,
% \textit{Package \xpackage{dtklogos}},
% 2011-04-25.\\
% \CTAN{usergrps/dante/dtk/dtklogos.sty}
%
% \bibitem{etexman}
% The \hologo{NTS} Team,
% \textit{The \hologo{eTeX} manual},
% 1998-02.\\
% \CTAN{systems/e-tex/v2/doc/}
%
% \bibitem{ExTeX-FAQ}
% The \hologo{ExTeX} group,
% \textit{\hologo{ExTeX}: FAQ -- How is \hologo{ExTeX} typeset?},
% 2007-04-14.\\
% \url{http://www.extex.org/documentation/faq.html}
%
% \bibitem{LyX}
% %@MISC{ LyX,
% %  title = {{LyX 2.0.0 -- The Document Processor [Computer software and manual]}},
% %  author = {{The LyX Team}},
% %  howpublished = {Internet: http://www.lyx.org},
% %  year = {2011-05-08},
% %  note = {Retrieved May 10, 2011, from http://www.lyx.org},
% %  url = {http://www.lyx.org/}
% %}
% The \hologo{LyX} Team,
% \textit{\hologo{LyX} -- The Document Processor},
% 2011-05-08.\\
% \url{http://www.lyx.org/}
%
% \bibitem{OzTeX}
% Andrew Trevorrow,
% \hologo{OzTeX} FAQ: What is the correct way to typeset ``\hologo{OzTeX}''?,
% 2011-09-15 (visited).
% \url{http://www.trevorrow.com/oztex/ozfaq.html#oztex-logo}
%
% \bibitem{PiCTeX}
% Michael Wichura,
% \textit{The \hologo{PiCTeX} macro package},
% 1987-09-21.
% \CTAN{graphics/pictex/}
%
% \bibitem{scrlogo}
% Markus Kohm,
% \textit{\hologo{KOMAScript} Datei \xfile{scrlogo.dtx}},
% 2009-01-30.\\
% \CTAN{install/macros/latex/contrib/komascript.tds.zip}
%
% \end{thebibliography}
%
% \begin{History}
%   \begin{Version}{2010/04/08 v1.0}
%   \item
%     The first version.
%   \end{Version}
%   \begin{Version}{2010/04/16 v1.1}
%   \item
%     \cs{Hologo} added for support of logos at start of a sentence.
%   \item
%     \cs{hologoSetup} and \cs{hologoLogoSetup} added.
%   \item
%     Options \xoption{break}, \xoption{hyphenbreak}, \xoption{spacebreak}
%     added.
%   \item
%     Variant support added by option \xoption{variant}.
%   \end{Version}
%   \begin{Version}{2010/04/24 v1.2}
%   \item
%     \hologo{LaTeX3} added.
%   \item
%     \hologo{VTeX} added.
%   \end{Version}
%   \begin{Version}{2010/11/21 v1.3}
%   \item
%     \hologo{iniTeX}, \hologo{virTeX} added.
%   \end{Version}
%   \begin{Version}{2011/03/25 v1.4}
%   \item
%     \hologo{ConTeXt} with variants added.
%   \item
%     Option \xoption{discretionarybreak} added as refinement for
%     option \xoption{break}.
%   \end{Version}
%   \begin{Version}{2011/04/21 v1.5}
%   \item
%     Wrong TDS directory for test files fixed.
%   \end{Version}
%   \begin{Version}{2011/10/01 v1.6}
%   \item
%     Support for package \xpackage{tex4ht} added.
%   \item
%     Support for \cs{csname} added if \cs{ifincsname} is available.
%   \item
%     New logos:
%     \hologo{(La)TeX},
%     \hologo{biber},
%     \hologo{BibTeX} (\xoption{sc}, \xoption{sf}),
%     \hologo{emTeX},
%     \hologo{ExTeX},
%     \hologo{KOMAScript},
%     \hologo{La},
%     \hologo{LyX},
%     \hologo{MiKTeX},
%     \hologo{NTS},
%     \hologo{OzMF},
%     \hologo{OzMP},
%     \hologo{OzTeX},
%     \hologo{OzTtH},
%     \hologo{PCTeX},
%     \hologo{PiC},
%     \hologo{PiCTeX},
%     \hologo{METAFONT},
%     \hologo{MetaFun},
%     \hologo{METAPOST},
%     \hologo{MetaPost},
%     \hologo{SLiTeX} (\xoption{lift}, \xoption{narrow}, \xoption{simple}),
%     \hologo{SliTeX} (\xoption{narrow}, \xoption{simple}, \xoption{lift}),
%     \hologo{teTeX}.
%   \item
%     Fixes:
%     \hologo{iniTeX},
%     \hologo{pdfLaTeX},
%     \hologo{pdfTeX},
%     \hologo{virTeX}.
%   \item
%     \cs{hologoFontSetup} and \cs{hologoLogoFontSetup} added.
%   \item
%     \cs{hologoVariant} and \cs{HologoVariant} added.
%   \end{Version}
%   \begin{Version}{2011/11/22 v1.7}
%   \item
%     New logos:
%     \hologo{BibTeX8},
%     \hologo{LaTeXML},
%     \hologo{SageTeX},
%     \hologo{TeX4ht},
%     \hologo{TTH}.
%   \item
%     \hologo{Xe} and friends: Driver stuff fixed.
%   \item
%     \hologo{Xe} and friends: Support for italic added.
%   \item
%     \hologo{Xe} and friends: Package support for \xpackage{pgf}
%     and \xpackage{pstricks} added.
%   \end{Version}
%   \begin{Version}{2011/11/29 v1.8}
%   \item
%     New logos:
%     \hologo{HanTheThanh}.
%   \end{Version}
%   \begin{Version}{2011/12/21 v1.9}
%   \item
%     Patch for package \xpackage{ifxetex} added for the case that
%     \cs{newif} is undefined in \hologo{iniTeX}.
%   \item
%     Some fixes for \hologo{iniTeX}.
%   \end{Version}
%   \begin{Version}{2012/04/26 v1.10}
%   \item
%     Fix in bookmark version of logo ``\hologo{HanTheThanh}''.
%   \end{Version}
%   \begin{Version}{2016/05/12 v1.11}
%   \item
%     Update HOLOGO@IfCharExists (previously in texlive)
%   \item define pdfliteral in current luatex.
%   \end{Version}
% \end{History}
%
% \PrintIndex
%
% \Finale
\endinput
|
% \end{quote}
% Do not forget to quote the argument according to the demands
% of your shell.
%
% \paragraph{Generating the documentation.}
% You can use both the \xfile{.dtx} or the \xfile{.drv} to generate
% the documentation. The process can be configured by the
% configuration file \xfile{ltxdoc.cfg}. For instance, put this
% line into this file, if you want to have A4 as paper format:
% \begin{quote}
%   \verb|\PassOptionsToClass{a4paper}{article}|
% \end{quote}
% An example follows how to generate the
% documentation with pdf\LaTeX:
% \begin{quote}
%\begin{verbatim}
%pdflatex hologo.dtx
%makeindex -s gind.ist hologo.idx
%pdflatex hologo.dtx
%makeindex -s gind.ist hologo.idx
%pdflatex hologo.dtx
%\end{verbatim}
% \end{quote}
%
% \section{Catalogue}
%
% The following XML file can be used as source for the
% \href{http://mirror.ctan.org/help/Catalogue/catalogue.html}{\TeX\ Catalogue}.
% The elements \texttt{caption} and \texttt{description} are imported
% from the original XML file from the Catalogue.
% The name of the XML file in the Catalogue is \xfile{hologo.xml}.
%    \begin{macrocode}
%<*catalogue>
<?xml version='1.0' encoding='us-ascii'?>
<!DOCTYPE entry SYSTEM 'catalogue.dtd'>
<entry datestamp='$Date$' modifier='$Author$' id='hologo'>
  <name>hologo</name>
  <caption>A collection of logos with bookmark support.</caption>
  <authorref id='auth:oberdiek'/>
  <copyright owner='Heiko Oberdiek' year='2010-2012'/>
  <license type='lppl1.3'/>
  <version number='1.10'/>
  <description>
    The package defines a single command <tt>\hologo</tt>, whose
    argument is the usual case-confused ASCII version of the logo.
    The command is bookmark-enabled, so that every logo becomes
    available in bookmarks without further work.
    <p/>
    The package is part of the <xref refid='oberdiek'>oberdiek</xref>
    bundle.
  </description>
  <documentation details='Package documentation'
      href='ctan:/macros/latex/contrib/oberdiek/hologo.pdf'/>
  <ctan file='true' path='/macros/latex/contrib/oberdiek/hologo.dtx'/>
  <miktex location='oberdiek'/>
  <texlive location='oberdiek'/>
  <install path='/macros/latex/contrib/oberdiek/oberdiek.tds.zip'/>
</entry>
%</catalogue>
%    \end{macrocode}
%
% \begin{thebibliography}{9}
% \raggedright
%
% \bibitem{btxdoc}
% Oren Patashnik,
% \textit{\hologo{BibTeX}ing},
% 1988-02-08.\\
% \CTAN{biblio/bibtex/base/}
%
% \bibitem{dtklogos}
% Gerd Neugebauer, DANTE,
% \textit{Package \xpackage{dtklogos}},
% 2011-04-25.\\
% \CTAN{usergrps/dante/dtk/dtklogos.sty}
%
% \bibitem{etexman}
% The \hologo{NTS} Team,
% \textit{The \hologo{eTeX} manual},
% 1998-02.\\
% \CTAN{systems/e-tex/v2/doc/}
%
% \bibitem{ExTeX-FAQ}
% The \hologo{ExTeX} group,
% \textit{\hologo{ExTeX}: FAQ -- How is \hologo{ExTeX} typeset?},
% 2007-04-14.\\
% \url{http://www.extex.org/documentation/faq.html}
%
% \bibitem{LyX}
% %@MISC{ LyX,
% %  title = {{LyX 2.0.0 -- The Document Processor [Computer software and manual]}},
% %  author = {{The LyX Team}},
% %  howpublished = {Internet: http://www.lyx.org},
% %  year = {2011-05-08},
% %  note = {Retrieved May 10, 2011, from http://www.lyx.org},
% %  url = {http://www.lyx.org/}
% %}
% The \hologo{LyX} Team,
% \textit{\hologo{LyX} -- The Document Processor},
% 2011-05-08.\\
% \url{http://www.lyx.org/}
%
% \bibitem{OzTeX}
% Andrew Trevorrow,
% \hologo{OzTeX} FAQ: What is the correct way to typeset ``\hologo{OzTeX}''?,
% 2011-09-15 (visited).
% \url{http://www.trevorrow.com/oztex/ozfaq.html#oztex-logo}
%
% \bibitem{PiCTeX}
% Michael Wichura,
% \textit{The \hologo{PiCTeX} macro package},
% 1987-09-21.
% \CTAN{graphics/pictex/}
%
% \bibitem{scrlogo}
% Markus Kohm,
% \textit{\hologo{KOMAScript} Datei \xfile{scrlogo.dtx}},
% 2009-01-30.\\
% \CTAN{install/macros/latex/contrib/komascript.tds.zip}
%
% \end{thebibliography}
%
% \begin{History}
%   \begin{Version}{2010/04/08 v1.0}
%   \item
%     The first version.
%   \end{Version}
%   \begin{Version}{2010/04/16 v1.1}
%   \item
%     \cs{Hologo} added for support of logos at start of a sentence.
%   \item
%     \cs{hologoSetup} and \cs{hologoLogoSetup} added.
%   \item
%     Options \xoption{break}, \xoption{hyphenbreak}, \xoption{spacebreak}
%     added.
%   \item
%     Variant support added by option \xoption{variant}.
%   \end{Version}
%   \begin{Version}{2010/04/24 v1.2}
%   \item
%     \hologo{LaTeX3} added.
%   \item
%     \hologo{VTeX} added.
%   \end{Version}
%   \begin{Version}{2010/11/21 v1.3}
%   \item
%     \hologo{iniTeX}, \hologo{virTeX} added.
%   \end{Version}
%   \begin{Version}{2011/03/25 v1.4}
%   \item
%     \hologo{ConTeXt} with variants added.
%   \item
%     Option \xoption{discretionarybreak} added as refinement for
%     option \xoption{break}.
%   \end{Version}
%   \begin{Version}{2011/04/21 v1.5}
%   \item
%     Wrong TDS directory for test files fixed.
%   \end{Version}
%   \begin{Version}{2011/10/01 v1.6}
%   \item
%     Support for package \xpackage{tex4ht} added.
%   \item
%     Support for \cs{csname} added if \cs{ifincsname} is available.
%   \item
%     New logos:
%     \hologo{(La)TeX},
%     \hologo{biber},
%     \hologo{BibTeX} (\xoption{sc}, \xoption{sf}),
%     \hologo{emTeX},
%     \hologo{ExTeX},
%     \hologo{KOMAScript},
%     \hologo{La},
%     \hologo{LyX},
%     \hologo{MiKTeX},
%     \hologo{NTS},
%     \hologo{OzMF},
%     \hologo{OzMP},
%     \hologo{OzTeX},
%     \hologo{OzTtH},
%     \hologo{PCTeX},
%     \hologo{PiC},
%     \hologo{PiCTeX},
%     \hologo{METAFONT},
%     \hologo{MetaFun},
%     \hologo{METAPOST},
%     \hologo{MetaPost},
%     \hologo{SLiTeX} (\xoption{lift}, \xoption{narrow}, \xoption{simple}),
%     \hologo{SliTeX} (\xoption{narrow}, \xoption{simple}, \xoption{lift}),
%     \hologo{teTeX}.
%   \item
%     Fixes:
%     \hologo{iniTeX},
%     \hologo{pdfLaTeX},
%     \hologo{pdfTeX},
%     \hologo{virTeX}.
%   \item
%     \cs{hologoFontSetup} and \cs{hologoLogoFontSetup} added.
%   \item
%     \cs{hologoVariant} and \cs{HologoVariant} added.
%   \end{Version}
%   \begin{Version}{2011/11/22 v1.7}
%   \item
%     New logos:
%     \hologo{BibTeX8},
%     \hologo{LaTeXML},
%     \hologo{SageTeX},
%     \hologo{TeX4ht},
%     \hologo{TTH}.
%   \item
%     \hologo{Xe} and friends: Driver stuff fixed.
%   \item
%     \hologo{Xe} and friends: Support for italic added.
%   \item
%     \hologo{Xe} and friends: Package support for \xpackage{pgf}
%     and \xpackage{pstricks} added.
%   \end{Version}
%   \begin{Version}{2011/11/29 v1.8}
%   \item
%     New logos:
%     \hologo{HanTheThanh}.
%   \end{Version}
%   \begin{Version}{2011/12/21 v1.9}
%   \item
%     Patch for package \xpackage{ifxetex} added for the case that
%     \cs{newif} is undefined in \hologo{iniTeX}.
%   \item
%     Some fixes for \hologo{iniTeX}.
%   \end{Version}
%   \begin{Version}{2012/04/26 v1.10}
%   \item
%     Fix in bookmark version of logo ``\hologo{HanTheThanh}''.
%   \end{Version}
%   \begin{Version}{2016/05/12 v1.11}
%   \item
%     Update HOLOGO@IfCharExists (previously in texlive)
%   \item define pdfliteral in current luatex.
%   \end{Version}
% \end{History}
%
% \PrintIndex
%
% \Finale
\endinput

%        (quote the arguments according to the demands of your shell)
%
% Documentation:
%    (a) If hologo.drv is present:
%           latex hologo.drv
%    (b) Without hologo.drv:
%           latex hologo.dtx; ...
%    The class ltxdoc loads the configuration file ltxdoc.cfg
%    if available. Here you can specify further options, e.g.
%    use A4 as paper format:
%       \PassOptionsToClass{a4paper}{article}
%
%    Programm calls to get the documentation (example):
%       pdflatex hologo.dtx
%       makeindex -s gind.ist hologo.idx
%       pdflatex hologo.dtx
%       makeindex -s gind.ist hologo.idx
%       pdflatex hologo.dtx
%
% Installation:
%    TDS:tex/generic/oberdiek/hologo.sty
%    TDS:doc/latex/oberdiek/hologo.pdf
%    TDS:doc/latex/oberdiek/example/hologo-example.tex
%    TDS:doc/latex/oberdiek/test/hologo-test1.tex
%    TDS:doc/latex/oberdiek/test/hologo-test-spacefactor.tex
%    TDS:doc/latex/oberdiek/test/hologo-test-list.tex
%    TDS:source/latex/oberdiek/hologo.dtx
%
%<*ignore>
\begingroup
  \catcode123=1 %
  \catcode125=2 %
  \def\x{LaTeX2e}%
\expandafter\endgroup
\ifcase 0\ifx\install y1\fi\expandafter
         \ifx\csname processbatchFile\endcsname\relax\else1\fi
         \ifx\fmtname\x\else 1\fi\relax
\else\csname fi\endcsname
%</ignore>
%<*install>
\input docstrip.tex
\Msg{************************************************************************}
\Msg{* Installation}
\Msg{* Package: hologo 2016/05/12 v1.11 A logo collection with bookmark support (HO)}
\Msg{************************************************************************}

\keepsilent
\askforoverwritefalse

\let\MetaPrefix\relax
\preamble

This is a generated file.

Project: hologo
Version: 2016/05/12 v1.11

Copyright (C) 2010-2012 by
   Heiko Oberdiek <heiko.oberdiek at googlemail.com>

This work may be distributed and/or modified under the
conditions of the LaTeX Project Public License, either
version 1.3c of this license or (at your option) any later
version. This version of this license is in
   http://www.latex-project.org/lppl/lppl-1-3c.txt
and the latest version of this license is in
   http://www.latex-project.org/lppl.txt
and version 1.3 or later is part of all distributions of
LaTeX version 2005/12/01 or later.

This work has the LPPL maintenance status "maintained".

This Current Maintainer of this work is Heiko Oberdiek.

The Base Interpreter refers to any `TeX-Format',
because some files are installed in TDS:tex/generic//.

This work consists of the main source file hologo.dtx
and the derived files
   hologo.sty, hologo.pdf, hologo.ins, hologo.drv, hologo-example.tex,
   hologo-test1.tex, hologo-test-spacefactor.tex,
   hologo-test-list.tex.

\endpreamble
\let\MetaPrefix\DoubleperCent

\generate{%
  \file{hologo.ins}{\from{hologo.dtx}{install}}%
  \file{hologo.drv}{\from{hologo.dtx}{driver}}%
  \usedir{tex/generic/oberdiek}%
  \file{hologo.sty}{\from{hologo.dtx}{package}}%
  \usedir{doc/latex/oberdiek/example}%
  \file{hologo-example.tex}{\from{hologo.dtx}{example}}%
  \usedir{doc/latex/oberdiek/test}%
  \file{hologo-test1.tex}{\from{hologo.dtx}{test1}}%
  \file{hologo-test-spacefactor.tex}{\from{hologo.dtx}{test-spacefactor}}%
  \file{hologo-test-list.tex}{\from{hologo.dtx}{test-list}}%
  \nopreamble
  \nopostamble
  \usedir{source/latex/oberdiek/catalogue}%
  \file{hologo.xml}{\from{hologo.dtx}{catalogue}}%
}

\catcode32=13\relax% active space
\let =\space%
\Msg{************************************************************************}
\Msg{*}
\Msg{* To finish the installation you have to move the following}
\Msg{* file into a directory searched by TeX:}
\Msg{*}
\Msg{*     hologo.sty}
\Msg{*}
\Msg{* To produce the documentation run the file `hologo.drv'}
\Msg{* through LaTeX.}
\Msg{*}
\Msg{* Happy TeXing!}
\Msg{*}
\Msg{************************************************************************}

\endbatchfile
%</install>
%<*ignore>
\fi
%</ignore>
%<*driver>
\NeedsTeXFormat{LaTeX2e}
\ProvidesFile{hologo.drv}%
  [2016/05/12 v1.11 A logo collection with bookmark support (HO)]%
\documentclass{ltxdoc}
\usepackage{holtxdoc}[2011/11/22]
\usepackage{hologo}[2016/05/12]
\usepackage{longtable}
\usepackage{array}
\usepackage{paralist}
%\usepackage[T1]{fontenc}
%\usepackage{lmodern}
\begin{document}
  \DocInput{hologo.dtx}%
\end{document}
%</driver>
% \fi
%
%
% \CharacterTable
%  {Upper-case    \A\B\C\D\E\F\G\H\I\J\K\L\M\N\O\P\Q\R\S\T\U\V\W\X\Y\Z
%   Lower-case    \a\b\c\d\e\f\g\h\i\j\k\l\m\n\o\p\q\r\s\t\u\v\w\x\y\z
%   Digits        \0\1\2\3\4\5\6\7\8\9
%   Exclamation   \!     Double quote  \"     Hash (number) \#
%   Dollar        \$     Percent       \%     Ampersand     \&
%   Acute accent  \'     Left paren    \(     Right paren   \)
%   Asterisk      \*     Plus          \+     Comma         \,
%   Minus         \-     Point         \.     Solidus       \/
%   Colon         \:     Semicolon     \;     Less than     \<
%   Equals        \=     Greater than  \>     Question mark \?
%   Commercial at \@     Left bracket  \[     Backslash     \\
%   Right bracket \]     Circumflex    \^     Underscore    \_
%   Grave accent  \`     Left brace    \{     Vertical bar  \|
%   Right brace   \}     Tilde         \~}
%
% \GetFileInfo{hologo.drv}
%
% \title{The \xpackage{hologo} package}
% \date{2016/05/12 v1.11}
% \author{Heiko Oberdiek\\\xemail{heiko.oberdiek at googlemail.com}}
%
% \maketitle
%
% \begin{abstract}
% This package starts a collection of logos with support for bookmarks
% strings.
% \end{abstract}
%
% \tableofcontents
%
% \section{Documentation}
%
% \subsection{Logo macros}
%
% \begin{declcs}{hologo} \M{name}
% \end{declcs}
% Macro \cs{hologo} sets the logo with name \meta{name}.
% The following table shows the supported names.
%
% \begingroup
%   \def\hologoEntry#1#2#3{^^A
%     #1&#2&\hologoLogoSetup{#1}{variant=#2}\hologo{#1}&#3\tabularnewline
%   }
%   \begin{longtable}{>{\ttfamily}l>{\ttfamily}lll}
%     \rmfamily\bfseries{name} & \rmfamily\bfseries variant
%     & \bfseries logo & \bfseries since\\
%     \hline
%     \endhead
%     \hologoList
%   \end{longtable}
% \endgroup
%
% \begin{declcs}{Hologo} \M{name}
% \end{declcs}
% Macro \cs{Hologo} starts the logo \meta{name} with an uppercase
% letter. As an exception small greek letters are not converted
% to uppercase. Examples, see \hologo{eTeX} and \hologo{ExTeX}.
%
% \subsection{Setup macros}
%
% The package does not support package options, but the following
% setup macros can be used to set options.
%
% \begin{declcs}{hologoSetup} \M{key value list}
% \end{declcs}
% Macro \cs{hologoSetup} sets global options.
%
% \begin{declcs}{hologoLogoSetup} \M{logo} \M{key value list}
% \end{declcs}
% Some options can also be used to configure a logo.
% These settings take precedence over global option settings.
%
% \subsection{Options}\label{sec:options}
%
% There are boolean and string options:
% \begin{description}
% \item[Boolean option:]
% It takes |true| or |false|
% as value. If the value is omitted, then |true| is used.
% \item[String option:]
% A value must be given as string. (But the string might be empty.)
% \end{description}
% The following options can be used both in \cs{hologoSetup}
% and \cs{hologoLogoSetup}:
% \begin{description}
% \def\entry#1{\item[\xoption{#1}:]}
% \entry{break}
%   enables or disables line breaks inside the logo. This setting is
%   refined by options \xoption{hyphenbreak}, \xoption{spacebreak}
%   or \xoption{discretionarybreak}.
%   Default is |false|.
% \entry{hyphenbreak}
%   enables or disables the line break right after the hyphen character.
% \entry{spacebreak}
%   enables or disables line breaks at space characters.
% \entry{discretionarybreak}
%   enables or disables line breaks at hyphenation points
%   (inserted by \cs{-}).
% \end{description}
% Macro \cs{hologoLogoSetup} also knows:
% \begin{description}
% \item[\xoption{variant}:]
%   This is a string option. It specifies a variant of a logo that
%   must exist. An empty string selects the package default variant.
% \end{description}
% Example:
% \begin{quote}
%   |\hologoSetup{break=false}|\\
%   |\hologoLogoSetup{plainTeX}{variant=hyphen,hyphenbreak}|\\
%   Then ``plain-\TeX'' contains one break point after the hyphen.
% \end{quote}
%
% \subsection{Driver options}
%
% Sometimes graphical operations are needed to construct some
% glyphs (e.g.\ \hologo{XeTeX}). If package \xpackage{graphics}
% or package \xpackage{pgf} are found, then the macros are taken
% from there. Otherwise the packge defines its own operations
% and therefore needs the driver information. Many drivers are
% detected automatically (\hologo{pdfTeX}/\hologo{LuaTeX}
% in PDF mode, \hologo{XeTeX}, \hologo{VTeX}). These have precedence
% over a driver option. The driver can be given as package option
% or using \cs{hologoDriverSetup}.
% The following list contains the recognized driver options:
% \begin{itemize}
% \item \xoption{pdftex}, \xoption{luatex}
% \item \xoption{dvipdfm}, \xoption{dvipdfmx}
% \item \xoption{dvips}, \xoption{dvipsone}, \xoption{xdvi}
% \item \xoption{xetex}
% \item \xoption{vtex}
% \end{itemize}
% The left driver of a line is the driver name that is used internally.
% The following names are aliases for drivers that use the
% same method. Therefore the entry in the \xext{log} file for
% the used driver prints the internally used driver name.
% \begin{description}
% \item[\xoption{driverfallback}:]
%   This option expects a driver that is used,
%   if the driver could not be detected automatically.
% \end{description}
%
% \begin{declcs}{hologoDriverSetup} \M{driver option}
% \end{declcs}
% The driver can also be configured after package loading
% using \cs{hologoDriverSetup}, also the way for \hologo{plainTeX}
% to setup the driver.
%
% \subsection{Font setup}
%
% Some logos require a special font, but should also be usable by
% \hologo{plainTeX}. Therefore the package provides some ways
% to influence the font settings. The options below
% take font settings as values. Both font commands
% such as \cs{sffamily} and macros that take one argument
% like \cs{textsf} can be used.
%
% \begin{declcs}{hologoFontSetup} \M{key value list}
% \end{declcs}
% Macro \cs{hologoFontSetup} sets the fonts for all logos.
% Supported keys:
% \begin{description}
% \def\entry#1{\item[\xoption{#1}:]}
% \entry{general}
%   This font is used for all logos. The default is empty.
%   That means no special font is used.
% \entry{bibsf}
%   This font is used for
%   {\hologoLogoSetup{BibTeX}{variant=sf}\hologo{BibTeX}}
%   with variant \xoption{sf}.
% \entry{rm}
%   This font is a serif font. It is used for \hologo{ExTeX}.
% \entry{sc}
%   This font specifies a small caps font. It is used for
%   {\hologoLogoSetup{BibTeX}{variant=sc}\hologo{BibTeX}}
%   with variant \xoption{sc}.
% \entry{sf}
%   This font specifies a sans serif font. The default
%   is \cs{sffamily}, then \cs{sf} is tried. Otherwise
%   a warning is given. It is used by \hologo{KOMAScript}.
% \entry{sy}
%   This is the font for math symbols (e.g. cmsy).
%   It is used by \hologo{AmS}, \hologo{NTS}, \hologo{ExTeX}.
% \entry{logo}
%   \hologo{METAFONT} and \hologo{METAPOST} are using that font.
%   In \hologo{LaTeX} \cs{logofamily} is used and
%   the definitions of package \xpackage{mflogo} are used
%   if the package is not loaded.
%   Otherwise the \cs{tenlogo} is used and defined
%   if it does not already exists.
% \end{description}
%
% \begin{declcs}{hologoLogoFontSetup} \M{logo} \M{key value list}
% \end{declcs}
% Fonts can also be set for a logo or logo component separately,
% see the following list.
% The keys are the same as for \cs{hologoFontSetup}.
%
% \begin{longtable}{>{\ttfamily}l>{\sffamily}ll}
%   \meta{logo} & keys & result\\
%   \hline
%   \endhead
%   BibTeX & bibsf & {\hologoLogoSetup{BibTeX}{variant=sf}\hologo{BibTeX}}\\[.5ex]
%   BibTeX & sc & {\hologoLogoSetup{BibTeX}{variant=sc}\hologo{BibTeX}}\\[.5ex]
%   ExTeX & rm & \hologo{ExTeX}\\
%   SliTeX & rm & \hologo{SliTeX}\\[.5ex]
%   AmS & sy & \hologo{AmS}\\
%   ExTeX & sy & \hologo{ExTeX}\\
%   NTS & sy & \hologo{NTS}\\[.5ex]
%   KOMAScript & sf & \hologo{KOMAScript}\\[.5ex]
%   METAFONT & logo & \hologo{METAFONT}\\
%   METAPOST & logo & \hologo{METAPOST}\\[.5ex]
%   SliTeX & sc \hologo{SliTeX}
% \end{longtable}
%
% \subsubsection{Font order}
%
% For all logos the font \xoption{general} is applied first.
% Example:
%\begin{quote}
%|\hologoFontSetup{general=\color{red}}|
%\end{quote}
% will print red logos.
% Then if the font uses a special font \xoption{sf}, for example,
% the font is applied that is setup by \cs{hologoLogoFontSetup}.
% If this font is not setup, then the common font setup
% by \cs{hologoFontSetup} is used. Otherwise a warning is given,
% that there is no font configured.
%
% \subsection{Additional user macros}
%
% Usually a variant of a logo is configured by using
% \cs{hologoLogoSetup}, because it is bad style to mix
% different variants of the same logo in the same text.
% There the following macros are a convenience for testing.
%
% \begin{declcs}{hologoVariant} \M{name} \M{variant}\\
%   \cs{HologoVariant} \M{name} \M{variant}
% \end{declcs}
% Logo \meta{name} is set using \meta{variant} that specifies
% explicitely which variant of the macro is used. If the argument
% is empty, then the default form of the logo is used
% (configurable by \cs{hologoLogoSetup}).
%
% \cs{HologoVariant} is used if the logo is set in a context
% that needs an uppercase first letter (beginning of a sentence, \dots).
%
% \begin{declcs}{hologoList}\\
%   \cs{hologoEntry} \M{logo} \M{variant} \M{since}
% \end{declcs}
% Macro \cs{hologoList} contains all logos that are provided
% by the package including variants. The list consists of calls
% of \cs{hologoEntry} with three arguments starting with the
% logo name \meta{logo} and its variant \meta{variant}. An empty
% variant means the current default. Argument \meta{since} specifies
% with version of the package \xpackage{hologo} is needed to get
% the logo. If the logo is fixed, then the date gets updated.
% Therefore the date \meta{since} is not exactly the date of
% the first introduction, but rather the date of the latest fix.
%
% Before \cs{hologoList} can be used, macro \cs{hologoEntry} needs
% a definition. The example file in section \ref{sec:example}
% shows applications of \cs{hologoList}.
%
% \subsection{Supported contexts}
%
% Macros \cs{hologo} and friends support special contexts:
% \begin{itemize}
% \item \hologo{LaTeX}'s protection mechanism.
% \item Bookmarks of package \xpackage{hyperref}.
% \item Package \xpackage{tex4ht}.
% \item The macros can be used inside \cs{csname} constructs,
%   if \cs{ifincsname} is available (\hologo{pdfTeX}, \hologo{XeTeX},
%   \hologo{LuaTeX}).
% \end{itemize}
%
% \subsection{Example}
% \label{sec:example}
%
% The following example prints the logos in different fonts.
%    \begin{macrocode}
%<*example>
%<<verbatim
\NeedsTeXFormat{LaTeX2e}
\documentclass[a4paper]{article}
\usepackage[
  hmargin=20mm,
  vmargin=20mm,
]{geometry}
\pagestyle{empty}
\usepackage{hologo}[2016/05/12]
\usepackage{longtable}
\usepackage{array}
\setlength{\extrarowheight}{2pt}
\usepackage[T1]{fontenc}
\usepackage{lmodern}
\usepackage{pdflscape}
\usepackage[
  pdfencoding=auto,
]{hyperref}
\hypersetup{
  pdfauthor={Heiko Oberdiek},
  pdftitle={Example for package `hologo'},
  pdfsubject={Logos with fonts lmr, lmss, qtm, qpl, qhv},
}
\usepackage{bookmark}

% Print the logo list on the console

\begingroup
  \typeout{}%
  \typeout{*** Begin of logo list ***}%
  \newcommand*{\hologoEntry}[3]{%
    \typeout{#1 \ifx\\#2\\\else(#2) \fi[#3]}%
  }%
  \hologoList
  \typeout{*** End of logo list ***}%
  \typeout{}%
\endgroup

\begin{document}
\begin{landscape}

  \section{Example file for package `hologo'}

  % Table for font names

  \begin{longtable}{>{\bfseries}ll}
    \textbf{font} & \textbf{Font name}\\
    \hline
    lmr & Latin Modern Roman\\
    lmss & Latin Modern Sans\\
    qtm & \TeX\ Gyre Termes\\
    qhv & \TeX\ Gyre Heros\\
    qpl & \TeX\ Gyre Pagella\\
  \end{longtable}

  % Logo list with logos in different fonts

  \begingroup
    \newcommand*{\SetVariant}[2]{%
      \ifx\\#2\\%
      \else
        \hologoLogoSetup{#1}{variant=#2}%
      \fi
    }%
    \newcommand*{\hologoEntry}[3]{%
      \SetVariant{#1}{#2}%
      \raisebox{1em}[0pt][0pt]{\hypertarget{#1@#2}{}}%
      \bookmark[%
        dest={#1@#2},%
      ]{%
        #1\ifx\\#2\\\else\space(#2)\fi: \Hologo{#1}, \hologo{#1} %
        [Unicode]%
      }%
      \hypersetup{unicode=false}%
      \bookmark[%
        dest={#1@#2},%
      ]{%
        #1\ifx\\#2\\\else\space(#2)\fi: \Hologo{#1}, \hologo{#1} %
        [PDFDocEncoding]%
      }%
      \texttt{#1}%
      &%
      \texttt{#2}%
      &%
      \Hologo{#1}%
      &%
      \SetVariant{#1}{#2}%
      \hologo{#1}%
      &%
      \SetVariant{#1}{#2}%
      \fontfamily{qtm}\selectfont
      \hologo{#1}%
      &%
      \SetVariant{#1}{#2}%
      \fontfamily{qpl}\selectfont
      \hologo{#1}%
      &%
      \SetVariant{#1}{#2}%
      \textsf{\hologo{#1}}%
      &%
      \SetVariant{#1}{#2}%
      \fontfamily{qhv}\selectfont
      \hologo{#1}%
      \tabularnewline
    }%
    \begin{longtable}{llllllll}%
      \textbf{\textit{logo}} & \textbf{\textit{variant}} &
      \texttt{\string\Hologo} &
      \textbf{lmr} & \textbf{qtm} & \textbf{qpl} &
      \textbf{lmss} & \textbf{qhv}
      \tabularnewline
      \hline
      \endhead
      \hologoList
    \end{longtable}%
  \endgroup

\end{landscape}
\end{document}
%verbatim
%</example>
%    \end{macrocode}
%
% \StopEventually{
% }
%
% \section{Implementation}
%    \begin{macrocode}
%<*package>
%    \end{macrocode}
%    Reload check, especially if the package is not used with \LaTeX.
%    \begin{macrocode}
\begingroup\catcode61\catcode48\catcode32=10\relax%
  \catcode13=5 % ^^M
  \endlinechar=13 %
  \catcode35=6 % #
  \catcode39=12 % '
  \catcode44=12 % ,
  \catcode45=12 % -
  \catcode46=12 % .
  \catcode58=12 % :
  \catcode64=11 % @
  \catcode123=1 % {
  \catcode125=2 % }
  \expandafter\let\expandafter\x\csname ver@hologo.sty\endcsname
  \ifx\x\relax % plain-TeX, first loading
  \else
    \def\empty{}%
    \ifx\x\empty % LaTeX, first loading,
      % variable is initialized, but \ProvidesPackage not yet seen
    \else
      \expandafter\ifx\csname PackageInfo\endcsname\relax
        \def\x#1#2{%
          \immediate\write-1{Package #1 Info: #2.}%
        }%
      \else
        \def\x#1#2{\PackageInfo{#1}{#2, stopped}}%
      \fi
      \x{hologo}{The package is already loaded}%
      \aftergroup\endinput
    \fi
  \fi
\endgroup%
%    \end{macrocode}
%    Package identification:
%    \begin{macrocode}
\begingroup\catcode61\catcode48\catcode32=10\relax%
  \catcode13=5 % ^^M
  \endlinechar=13 %
  \catcode35=6 % #
  \catcode39=12 % '
  \catcode40=12 % (
  \catcode41=12 % )
  \catcode44=12 % ,
  \catcode45=12 % -
  \catcode46=12 % .
  \catcode47=12 % /
  \catcode58=12 % :
  \catcode64=11 % @
  \catcode91=12 % [
  \catcode93=12 % ]
  \catcode123=1 % {
  \catcode125=2 % }
  \expandafter\ifx\csname ProvidesPackage\endcsname\relax
    \def\x#1#2#3[#4]{\endgroup
      \immediate\write-1{Package: #3 #4}%
      \xdef#1{#4}%
    }%
  \else
    \def\x#1#2[#3]{\endgroup
      #2[{#3}]%
      \ifx#1\@undefined
        \xdef#1{#3}%
      \fi
      \ifx#1\relax
        \xdef#1{#3}%
      \fi
    }%
  \fi
\expandafter\x\csname ver@hologo.sty\endcsname
\ProvidesPackage{hologo}%
  [2016/05/12 v1.11 A logo collection with bookmark support (HO)]%
%    \end{macrocode}
%
%    \begin{macrocode}
\begingroup\catcode61\catcode48\catcode32=10\relax%
  \catcode13=5 % ^^M
  \endlinechar=13 %
  \catcode123=1 % {
  \catcode125=2 % }
  \catcode64=11 % @
  \def\x{\endgroup
    \expandafter\edef\csname HOLOGO@AtEnd\endcsname{%
      \endlinechar=\the\endlinechar\relax
      \catcode13=\the\catcode13\relax
      \catcode32=\the\catcode32\relax
      \catcode35=\the\catcode35\relax
      \catcode61=\the\catcode61\relax
      \catcode64=\the\catcode64\relax
      \catcode123=\the\catcode123\relax
      \catcode125=\the\catcode125\relax
    }%
  }%
\x\catcode61\catcode48\catcode32=10\relax%
\catcode13=5 % ^^M
\endlinechar=13 %
\catcode35=6 % #
\catcode64=11 % @
\catcode123=1 % {
\catcode125=2 % }
\def\TMP@EnsureCode#1#2{%
  \edef\HOLOGO@AtEnd{%
    \HOLOGO@AtEnd
    \catcode#1=\the\catcode#1\relax
  }%
  \catcode#1=#2\relax
}
\TMP@EnsureCode{10}{12}% ^^J
\TMP@EnsureCode{33}{12}% !
\TMP@EnsureCode{34}{12}% "
\TMP@EnsureCode{36}{3}% $
\TMP@EnsureCode{38}{4}% &
\TMP@EnsureCode{39}{12}% '
\TMP@EnsureCode{40}{12}% (
\TMP@EnsureCode{41}{12}% )
\TMP@EnsureCode{42}{12}% *
\TMP@EnsureCode{43}{12}% +
\TMP@EnsureCode{44}{12}% ,
\TMP@EnsureCode{45}{12}% -
\TMP@EnsureCode{46}{12}% .
\TMP@EnsureCode{47}{12}% /
\TMP@EnsureCode{58}{12}% :
\TMP@EnsureCode{59}{12}% ;
\TMP@EnsureCode{60}{12}% <
\TMP@EnsureCode{62}{12}% >
\TMP@EnsureCode{63}{12}% ?
\TMP@EnsureCode{91}{12}% [
\TMP@EnsureCode{93}{12}% ]
\TMP@EnsureCode{94}{7}% ^ (superscript)
\TMP@EnsureCode{95}{8}% _ (subscript)
\TMP@EnsureCode{96}{12}% `
\TMP@EnsureCode{124}{12}% |
\edef\HOLOGO@AtEnd{%
  \HOLOGO@AtEnd
  \escapechar\the\escapechar\relax
  \noexpand\endinput
}
\escapechar=92 %
%    \end{macrocode}
%
% \subsection{Logo list}
%
%    \begin{macro}{\hologoList}
%    \begin{macrocode}
\def\hologoList{%
  \hologoEntry{(La)TeX}{}{2011/10/01}%
  \hologoEntry{AmSLaTeX}{}{2010/04/16}%
  \hologoEntry{AmSTeX}{}{2010/04/16}%
  \hologoEntry{biber}{}{2011/10/01}%
  \hologoEntry{BibTeX}{}{2011/10/01}%
  \hologoEntry{BibTeX}{sf}{2011/10/01}%
  \hologoEntry{BibTeX}{sc}{2011/10/01}%
  \hologoEntry{BibTeX8}{}{2011/11/22}%
  \hologoEntry{ConTeXt}{}{2011/03/25}%
  \hologoEntry{ConTeXt}{narrow}{2011/03/25}%
  \hologoEntry{ConTeXt}{simple}{2011/03/25}%
  \hologoEntry{emTeX}{}{2010/04/26}%
  \hologoEntry{eTeX}{}{2010/04/08}%
  \hologoEntry{ExTeX}{}{2011/10/01}%
  \hologoEntry{HanTheThanh}{}{2011/11/29}%
  \hologoEntry{iniTeX}{}{2011/10/01}%
  \hologoEntry{KOMAScript}{}{2011/10/01}%
  \hologoEntry{La}{}{2010/05/08}%
  \hologoEntry{LaTeX}{}{2010/04/08}%
  \hologoEntry{LaTeX2e}{}{2010/04/08}%
  \hologoEntry{LaTeX3}{}{2010/04/24}%
  \hologoEntry{LaTeXe}{}{2010/04/08}%
  \hologoEntry{LaTeXML}{}{2011/11/22}%
  \hologoEntry{LaTeXTeX}{}{2011/10/01}%
  \hologoEntry{LuaLaTeX}{}{2010/04/08}%
  \hologoEntry{LuaTeX}{}{2010/04/08}%
  \hologoEntry{LyX}{}{2011/10/01}%
  \hologoEntry{METAFONT}{}{2011/10/01}%
  \hologoEntry{MetaFun}{}{2011/10/01}%
  \hologoEntry{METAPOST}{}{2011/10/01}%
  \hologoEntry{MetaPost}{}{2011/10/01}%
  \hologoEntry{MiKTeX}{}{2011/10/01}%
  \hologoEntry{NTS}{}{2011/10/01}%
  \hologoEntry{OzMF}{}{2011/10/01}%
  \hologoEntry{OzMP}{}{2011/10/01}%
  \hologoEntry{OzTeX}{}{2011/10/01}%
  \hologoEntry{OzTtH}{}{2011/10/01}%
  \hologoEntry{PCTeX}{}{2011/10/01}%
  \hologoEntry{pdfTeX}{}{2011/10/01}%
  \hologoEntry{pdfLaTeX}{}{2011/10/01}%
  \hologoEntry{PiC}{}{2011/10/01}%
  \hologoEntry{PiCTeX}{}{2011/10/01}%
  \hologoEntry{plainTeX}{}{2010/04/08}%
  \hologoEntry{plainTeX}{space}{2010/04/16}%
  \hologoEntry{plainTeX}{hyphen}{2010/04/16}%
  \hologoEntry{plainTeX}{runtogether}{2010/04/16}%
  \hologoEntry{SageTeX}{}{2011/11/22}%
  \hologoEntry{SLiTeX}{}{2011/10/01}%
  \hologoEntry{SLiTeX}{lift}{2011/10/01}%
  \hologoEntry{SLiTeX}{narrow}{2011/10/01}%
  \hologoEntry{SLiTeX}{simple}{2011/10/01}%
  \hologoEntry{SliTeX}{}{2011/10/01}%
  \hologoEntry{SliTeX}{narrow}{2011/10/01}%
  \hologoEntry{SliTeX}{simple}{2011/10/01}%
  \hologoEntry{SliTeX}{lift}{2011/10/01}%
  \hologoEntry{teTeX}{}{2011/10/01}%
  \hologoEntry{TeX}{}{2010/04/08}%
  \hologoEntry{TeX4ht}{}{2011/11/22}%
  \hologoEntry{TTH}{}{2011/11/22}%
  \hologoEntry{virTeX}{}{2011/10/01}%
  \hologoEntry{VTeX}{}{2010/04/24}%
  \hologoEntry{Xe}{}{2010/04/08}%
  \hologoEntry{XeLaTeX}{}{2010/04/08}%
  \hologoEntry{XeTeX}{}{2010/04/08}%
}
%    \end{macrocode}
%    \end{macro}
%
% \subsection{Load resources}
%
%    \begin{macrocode}
\begingroup\expandafter\expandafter\expandafter\endgroup
\expandafter\ifx\csname RequirePackage\endcsname\relax
  \def\TMP@RequirePackage#1[#2]{%
    \begingroup\expandafter\expandafter\expandafter\endgroup
    \expandafter\ifx\csname ver@#1.sty\endcsname\relax
      \input #1.sty\relax
    \fi
  }%
  \TMP@RequirePackage{ltxcmds}[2011/02/04]%
  \TMP@RequirePackage{infwarerr}[2010/04/08]%
  \TMP@RequirePackage{kvsetkeys}[2010/03/01]%
  \TMP@RequirePackage{kvdefinekeys}[2010/03/01]%
  \TMP@RequirePackage{pdftexcmds}[2010/04/01]%
  \TMP@RequirePackage{ifpdf}[2010/01/28]%
  \TMP@RequirePackage{ifluatex}[2010/03/01]%
  \ltx@IfUndefined{newif}{%
    \expandafter\let\csname newif\endcsname\ltx@newif
  }{}%
  \TMP@RequirePackage{ifxetex}[2009/01/23]%
  \TMP@RequirePackage{ifvtex}[2010/03/01]%
\else
  \RequirePackage{ltxcmds}[2011/02/04]%
  \RequirePackage{infwarerr}[2010/04/08]%
  \RequirePackage{kvsetkeys}[2010/03/01]%
  \RequirePackage{kvdefinekeys}[2010/03/01]%
  \RequirePackage{pdftexcmds}[2010/04/01]%
  \RequirePackage{ifpdf}[2010/01/28]%
  \RequirePackage{ifluatex}[2010/03/01]%
  \RequirePackage{ifxetex}[2009/01/23]%
  \RequirePackage{ifvtex}[2010/03/01]%
\fi
%    \end{macrocode}
%
%    \begin{macro}{\HOLOGO@IfDefined}
%    \begin{macrocode}
\def\HOLOGO@IfExists#1{%
  \ifx\@undefined#1%
    \expandafter\ltx@secondoftwo
  \else
    \ifx\relax#1%
      \expandafter\ltx@secondoftwo
    \else
      \expandafter\expandafter\expandafter\ltx@firstoftwo
    \fi
  \fi
}
%    \end{macrocode}
%    \end{macro}
%
% \subsection{Setup macros}
%
%    \begin{macro}{\hologoSetup}
%    \begin{macrocode}
\def\hologoSetup{%
  \let\HOLOGO@name\relax
  \HOLOGO@Setup
}
%    \end{macrocode}
%    \end{macro}
%
%    \begin{macro}{\hologoLogoSetup}
%    \begin{macrocode}
\def\hologoLogoSetup#1{%
  \edef\HOLOGO@name{#1}%
  \ltx@IfUndefined{HoLogo@\HOLOGO@name}{%
    \@PackageError{hologo}{%
      Unknown logo `\HOLOGO@name'%
    }\@ehc
    \ltx@gobble
  }{%
    \HOLOGO@Setup
  }%
}
%    \end{macrocode}
%    \end{macro}
%
%    \begin{macro}{\HOLOGO@Setup}
%    \begin{macrocode}
\def\HOLOGO@Setup{%
  \kvsetkeys{HoLogo}%
}
%    \end{macrocode}
%    \end{macro}
%
% \subsection{Options}
%
%    \begin{macro}{\HOLOGO@DeclareBoolOption}
%    \begin{macrocode}
\def\HOLOGO@DeclareBoolOption#1{%
  \expandafter\chardef\csname HOLOGOOPT@#1\endcsname\ltx@zero
  \kv@define@key{HoLogo}{#1}[true]{%
    \def\HOLOGO@temp{##1}%
    \ifx\HOLOGO@temp\HOLOGO@true
      \ifx\HOLOGO@name\relax
        \expandafter\chardef\csname HOLOGOOPT@#1\endcsname=\ltx@one
      \else
        \expandafter\chardef\csname
        HoLogoOpt@#1@\HOLOGO@name\endcsname\ltx@one
      \fi
      \HOLOGO@SetBreakAll{#1}%
    \else
      \ifx\HOLOGO@temp\HOLOGO@false
        \ifx\HOLOGO@name\relax
          \expandafter\chardef\csname HOLOGOOPT@#1\endcsname=\ltx@zero
        \else
          \expandafter\chardef\csname
          HoLogoOpt@#1@\HOLOGO@name\endcsname=\ltx@zero
        \fi
        \HOLOGO@SetBreakAll{#1}%
      \else
        \@PackageError{hologo}{%
          Unknown value `##1' for boolean option `#1'.\MessageBreak
          Known values are `true' and `false'%
        }\@ehc
      \fi
    \fi
  }%
}
%    \end{macrocode}
%    \end{macro}
%
%    \begin{macro}{\HOLOGO@SetBreakAll}
%    \begin{macrocode}
\def\HOLOGO@SetBreakAll#1{%
  \def\HOLOGO@temp{#1}%
  \ifx\HOLOGO@temp\HOLOGO@break
    \ifx\HOLOGO@name\relax
      \chardef\HOLOGOOPT@hyphenbreak=\HOLOGOOPT@break
      \chardef\HOLOGOOPT@spacebreak=\HOLOGOOPT@break
      \chardef\HOLOGOOPT@discretionarybreak=\HOLOGOOPT@break
    \else
      \expandafter\chardef
         \csname HoLogoOpt@hyphenbreak@\HOLOGO@name\endcsname=%
         \csname HoLogoOpt@break@\HOLOGO@name\endcsname
      \expandafter\chardef
         \csname HoLogoOpt@spacebreak@\HOLOGO@name\endcsname=%
         \csname HoLogoOpt@break@\HOLOGO@name\endcsname
      \expandafter\chardef
         \csname HoLogoOpt@discretionarybreak@\HOLOGO@name
             \endcsname=%
         \csname HoLogoOpt@break@\HOLOGO@name\endcsname
    \fi
  \fi
}
%    \end{macrocode}
%    \end{macro}
%
%    \begin{macro}{\HOLOGO@true}
%    \begin{macrocode}
\def\HOLOGO@true{true}
%    \end{macrocode}
%    \end{macro}
%    \begin{macro}{\HOLOGO@false}
%    \begin{macrocode}
\def\HOLOGO@false{false}
%    \end{macrocode}
%    \end{macro}
%    \begin{macro}{\HOLOGO@break}
%    \begin{macrocode}
\def\HOLOGO@break{break}
%    \end{macrocode}
%    \end{macro}
%
%    \begin{macrocode}
\HOLOGO@DeclareBoolOption{break}
\HOLOGO@DeclareBoolOption{hyphenbreak}
\HOLOGO@DeclareBoolOption{spacebreak}
\HOLOGO@DeclareBoolOption{discretionarybreak}
%    \end{macrocode}
%
%    \begin{macrocode}
\kv@define@key{HoLogo}{variant}{%
  \ifx\HOLOGO@name\relax
    \@PackageError{hologo}{%
      Option `variant' is not available in \string\hologoSetup,%
      \MessageBreak
      Use \string\hologoLogoSetup\space instead%
    }\@ehc
  \else
    \edef\HOLOGO@temp{#1}%
    \ifx\HOLOGO@temp\ltx@empty
      \expandafter
      \let\csname HoLogoOpt@variant@\HOLOGO@name\endcsname\@undefined
    \else
      \ltx@IfUndefined{HoLogo@\HOLOGO@name @\HOLOGO@temp}{%
        \@PackageError{hologo}{%
          Unknown variant `\HOLOGO@temp' of logo `\HOLOGO@name'%
        }\@ehc
      }{%
        \expandafter
        \let\csname HoLogoOpt@variant@\HOLOGO@name\endcsname
            \HOLOGO@temp
      }%
    \fi
  \fi
}
%    \end{macrocode}
%
%    \begin{macro}{\HOLOGO@Variant}
%    \begin{macrocode}
\def\HOLOGO@Variant#1{%
  #1%
  \ltx@ifundefined{HoLogoOpt@variant@#1}{%
  }{%
    @\csname HoLogoOpt@variant@#1\endcsname
  }%
}
%    \end{macrocode}
%    \end{macro}
%
% \subsection{Break/no-break support}
%
%    \begin{macro}{\HOLOGO@space}
%    \begin{macrocode}
\def\HOLOGO@space{%
  \ltx@ifundefined{HoLogoOpt@spacebreak@\HOLOGO@name}{%
    \ltx@ifundefined{HoLogoOpt@break@\HOLOGO@name}{%
      \chardef\HOLOGO@temp=\HOLOGOOPT@spacebreak
    }{%
      \chardef\HOLOGO@temp=%
        \csname HoLogoOpt@break@\HOLOGO@name\endcsname
    }%
  }{%
    \chardef\HOLOGO@temp=%
      \csname HoLogoOpt@spacebreak@\HOLOGO@name\endcsname
  }%
  \ifcase\HOLOGO@temp
    \penalty10000 %
  \fi
  \ltx@space
}
%    \end{macrocode}
%    \end{macro}
%
%    \begin{macro}{\HOLOGO@hyphen}
%    \begin{macrocode}
\def\HOLOGO@hyphen{%
  \ltx@ifundefined{HoLogoOpt@hyphenbreak@\HOLOGO@name}{%
    \ltx@ifundefined{HoLogoOpt@break@\HOLOGO@name}{%
      \chardef\HOLOGO@temp=\HOLOGOOPT@hyphenbreak
    }{%
      \chardef\HOLOGO@temp=%
        \csname HoLogoOpt@break@\HOLOGO@name\endcsname
    }%
  }{%
    \chardef\HOLOGO@temp=%
      \csname HoLogoOpt@hyphenbreak@\HOLOGO@name\endcsname
  }%
  \ifcase\HOLOGO@temp
    \ltx@mbox{-}%
  \else
    -%
  \fi
}
%    \end{macrocode}
%    \end{macro}
%
%    \begin{macro}{\HOLOGO@discretionary}
%    \begin{macrocode}
\def\HOLOGO@discretionary{%
  \ltx@ifundefined{HoLogoOpt@discretionarybreak@\HOLOGO@name}{%
    \ltx@ifundefined{HoLogoOpt@break@\HOLOGO@name}{%
      \chardef\HOLOGO@temp=\HOLOGOOPT@discretionarybreak
    }{%
      \chardef\HOLOGO@temp=%
        \csname HoLogoOpt@break@\HOLOGO@name\endcsname
    }%
  }{%
    \chardef\HOLOGO@temp=%
      \csname HoLogoOpt@discretionarybreak@\HOLOGO@name\endcsname
  }%
  \ifcase\HOLOGO@temp
  \else
    \-%
  \fi
}
%    \end{macrocode}
%    \end{macro}
%
%    \begin{macro}{\HOLOGO@mbox}
%    \begin{macrocode}
\def\HOLOGO@mbox#1{%
  \ltx@ifundefined{HoLogoOpt@break@\HOLOGO@name}{%
    \chardef\HOLOGO@temp=\HOLOGOOPT@hyphenbreak
  }{%
    \chardef\HOLOGO@temp=%
      \csname HoLogoOpt@break@\HOLOGO@name\endcsname
  }%
  \ifcase\HOLOGO@temp
    \ltx@mbox{#1}%
  \else
    #1%
  \fi
}
%    \end{macrocode}
%    \end{macro}
%
% \subsection{Font support}
%
%    \begin{macro}{\HoLogoFont@font}
%    \begin{tabular}{@{}ll@{}}
%    |#1|:& logo name\\
%    |#2|:& font short name\\
%    |#3|:& text
%    \end{tabular}
%    \begin{macrocode}
\def\HoLogoFont@font#1#2#3{%
  \begingroup
    \ltx@IfUndefined{HoLogoFont@logo@#1.#2}{%
      \ltx@IfUndefined{HoLogoFont@font@#2}{%
        \@PackageWarning{hologo}{%
          Missing font `#2' for logo `#1'%
        }%
        #3%
      }{%
        \csname HoLogoFont@font@#2\endcsname{#3}%
      }%
    }{%
      \csname HoLogoFont@logo@#1.#2\endcsname{#3}%
    }%
  \endgroup
}
%    \end{macrocode}
%    \end{macro}
%
%    \begin{macro}{\HoLogoFont@Def}
%    \begin{macrocode}
\def\HoLogoFont@Def#1{%
  \expandafter\def\csname HoLogoFont@font@#1\endcsname
}
%    \end{macrocode}
%    \end{macro}
%    \begin{macro}{\HoLogoFont@LogoDef}
%    \begin{macrocode}
\def\HoLogoFont@LogoDef#1#2{%
  \expandafter\def\csname HoLogoFont@logo@#1.#2\endcsname
}
%    \end{macrocode}
%    \end{macro}
%
% \subsubsection{Font defaults}
%
%    \begin{macro}{\HoLogoFont@font@general}
%    \begin{macrocode}
\HoLogoFont@Def{general}{}%
%    \end{macrocode}
%    \end{macro}
%
%    \begin{macro}{\HoLogoFont@font@rm}
%    \begin{macrocode}
\ltx@IfUndefined{rmfamily}{%
  \ltx@IfUndefined{rm}{%
  }{%
    \HoLogoFont@Def{rm}{\rm}%
  }%
}{%
  \HoLogoFont@Def{rm}{\rmfamily}%
}
%    \end{macrocode}
%    \end{macro}
%
%    \begin{macro}{\HoLogoFont@font@sf}
%    \begin{macrocode}
\ltx@IfUndefined{sffamily}{%
  \ltx@IfUndefined{sf}{%
  }{%
    \HoLogoFont@Def{sf}{\sf}%
  }%
}{%
  \HoLogoFont@Def{sf}{\sffamily}%
}
%    \end{macrocode}
%    \end{macro}
%
%    \begin{macro}{\HoLogoFont@font@bibsf}
%    In case of \hologo{plainTeX} the original small caps
%    variant is used as default. In \hologo{LaTeX}
%    the definition of package \xpackage{dtklogos} \cite{dtklogos}
%    is used.
%\begin{quote}
%\begin{verbatim}
%\DeclareRobustCommand{\BibTeX}{%
%  B%
%  \kern-.05em%
%  \hbox{%
%    $\m@th$% %% force math size calculations
%    \csname S@\f@size\endcsname
%    \fontsize\sf@size\z@
%    \math@fontsfalse
%    \selectfont
%    I%
%    \kern-.025em%
%    B
%  }%
%  \kern-.08em%
%  \-%
%  \TeX
%}
%\end{verbatim}
%\end{quote}
%    \begin{macrocode}
\ltx@IfUndefined{selectfont}{%
  \ltx@IfUndefined{tensc}{%
    \font\tensc=cmcsc10\relax
  }{}%
  \HoLogoFont@Def{bibsf}{\tensc}%
}{%
  \HoLogoFont@Def{bibsf}{%
    $\mathsurround=0pt$%
    \csname S@\f@size\endcsname
    \fontsize\sf@size{0pt}%
    \math@fontsfalse
    \selectfont
  }%
}
%    \end{macrocode}
%    \end{macro}
%
%    \begin{macro}{\HoLogoFont@font@sc}
%    \begin{macrocode}
\ltx@IfUndefined{scshape}{%
  \ltx@IfUndefined{tensc}{%
    \font\tensc=cmcsc10\relax
  }{}%
  \HoLogoFont@Def{sc}{\tensc}%
}{%
  \HoLogoFont@Def{sc}{\scshape}%
}
%    \end{macrocode}
%    \end{macro}
%
%    \begin{macro}{\HoLogoFont@font@sy}
%    \begin{macrocode}
\ltx@IfUndefined{usefont}{%
  \ltx@IfUndefined{tensy}{%
  }{%
    \HoLogoFont@Def{sy}{\tensy}%
  }%
}{%
  \HoLogoFont@Def{sy}{%
    \usefont{OMS}{cmsy}{m}{n}%
  }%
}
%    \end{macrocode}
%    \end{macro}
%
%    \begin{macro}{\HoLogoFont@font@logo}
%    \begin{macrocode}
\begingroup
  \def\x{LaTeX2e}%
\expandafter\endgroup
\ifx\fmtname\x
  \ltx@IfUndefined{logofamily}{%
    \DeclareRobustCommand\logofamily{%
      \not@math@alphabet\logofamily\relax
      \fontencoding{U}%
      \fontfamily{logo}%
      \selectfont
    }%
  }{}%
  \ltx@IfUndefined{logofamily}{%
  }{%
    \HoLogoFont@Def{logo}{\logofamily}%
  }%
\else
  \ltx@IfUndefined{tenlogo}{%
    \font\tenlogo=logo10\relax
  }{}%
  \HoLogoFont@Def{logo}{\tenlogo}%
\fi
%    \end{macrocode}
%    \end{macro}
%
% \subsubsection{Font setup}
%
%    \begin{macro}{\hologoFontSetup}
%    \begin{macrocode}
\def\hologoFontSetup{%
  \let\HOLOGO@name\relax
  \HOLOGO@FontSetup
}
%    \end{macrocode}
%    \end{macro}
%
%    \begin{macro}{\hologoLogoFontSetup}
%    \begin{macrocode}
\def\hologoLogoFontSetup#1{%
  \edef\HOLOGO@name{#1}%
  \ltx@IfUndefined{HoLogo@\HOLOGO@name}{%
    \@PackageError{hologo}{%
      Unknown logo `\HOLOGO@name'%
    }\@ehc
    \ltx@gobble
  }{%
    \HOLOGO@FontSetup
  }%
}
%    \end{macrocode}
%    \end{macro}
%
%    \begin{macro}{\HOLOGO@FontSetup}
%    \begin{macrocode}
\def\HOLOGO@FontSetup{%
  \kvsetkeys{HoLogoFont}%
}
%    \end{macrocode}
%    \end{macro}
%
%    \begin{macrocode}
\def\HOLOGO@temp#1{%
  \kv@define@key{HoLogoFont}{#1}{%
    \ifx\HOLOGO@name\relax
      \HoLogoFont@Def{#1}{##1}%
    \else
      \HoLogoFont@LogoDef\HOLOGO@name{#1}{##1}%
    \fi
  }%
}
\HOLOGO@temp{general}
\HOLOGO@temp{sf}
%    \end{macrocode}
%
% \subsection{Generic logo commands}
%
%    \begin{macrocode}
\HOLOGO@IfExists\hologo{%
  \@PackageError{hologo}{%
    \string\hologo\ltx@space is already defined.\MessageBreak
    Package loading is aborted%
  }\@ehc
  \HOLOGO@AtEnd
}%
\HOLOGO@IfExists\hologoRobust{%
  \@PackageError{hologo}{%
    \string\hologoRobust\ltx@space is already defined.\MessageBreak
    Package loading is aborted%
  }\@ehc
  \HOLOGO@AtEnd
}%
%    \end{macrocode}
%
% \subsubsection{\cs{hologo} and friends}
%
%    \begin{macrocode}
\ifluatex
  \expandafter\ltx@firstofone
\else
  \expandafter\ltx@gobble
\fi
{%
  \ltx@IfUndefined{ifincsname}{%
    \ifnum\luatexversion<36 %
      \expandafter\ltx@gobble
    \else
      \expandafter\ltx@firstofone
    \fi
    {%
      \begingroup
        \ifcase0%
            \directlua{%
              if tex.enableprimitives then %
                tex.enableprimitives('HOLOGO@', {'ifincsname'})%
              else %
                tex.print('1')%
              end%
            }%
            \ifx\HOLOGO@ifincsname\@undefined 1\fi%
            \relax
          \expandafter\ltx@firstofone
        \else
          \endgroup
          \expandafter\ltx@gobble
        \fi
        {%
          \global\let\ifincsname\HOLOGO@ifincsname
        }%
      \HOLOGO@temp
    }%
  }{}%
}
%    \end{macrocode}
%    \begin{macrocode}
\ltx@IfUndefined{ifincsname}{%
  \catcode`$=14 %
}{%
  \catcode`$=9 %
}
%    \end{macrocode}
%
%    \begin{macro}{\hologo}
%    \begin{macrocode}
\def\hologo#1{%
$ \ifincsname
$   \ltx@ifundefined{HoLogoCs@\HOLOGO@Variant{#1}}{%
$     #1%
$   }{%
$     \csname HoLogoCs@\HOLOGO@Variant{#1}\endcsname\ltx@firstoftwo
$   }%
$ \else
    \HOLOGO@IfExists\texorpdfstring\texorpdfstring\ltx@firstoftwo
    {%
      \hologoRobust{#1}%
    }{%
      \ltx@ifundefined{HoLogoBkm@\HOLOGO@Variant{#1}}{%
        \ltx@ifundefined{HoLogo@#1}{?#1?}{#1}%
      }{%
        \csname HoLogoBkm@\HOLOGO@Variant{#1}\endcsname
        \ltx@firstoftwo
      }%
    }%
$ \fi
}
%    \end{macrocode}
%    \end{macro}
%    \begin{macro}{\Hologo}
%    \begin{macrocode}
\def\Hologo#1{%
$ \ifincsname
$   \ltx@ifundefined{HoLogoCs@\HOLOGO@Variant{#1}}{%
$     #1%
$   }{%
$     \csname HoLogoCs@\HOLOGO@Variant{#1}\endcsname\ltx@secondoftwo
$   }%
$ \else
    \HOLOGO@IfExists\texorpdfstring\texorpdfstring\ltx@firstoftwo
    {%
      \HologoRobust{#1}%
    }{%
      \ltx@ifundefined{HoLogoBkm@\HOLOGO@Variant{#1}}{%
        \ltx@ifundefined{HoLogo@#1}{?#1?}{#1}%
      }{%
        \csname HoLogoBkm@\HOLOGO@Variant{#1}\endcsname
        \ltx@secondoftwo
      }%
    }%
$ \fi
}
%    \end{macrocode}
%    \end{macro}
%
%    \begin{macro}{\hologoVariant}
%    \begin{macrocode}
\def\hologoVariant#1#2{%
  \ifx\relax#2\relax
    \hologo{#1}%
  \else
$   \ifincsname
$     \ltx@ifundefined{HoLogoCs@#1@#2}{%
$       #1%
$     }{%
$       \csname HoLogoCs@#1@#2\endcsname\ltx@firstoftwo
$     }%
$   \else
      \HOLOGO@IfExists\texorpdfstring\texorpdfstring\ltx@firstoftwo
      {%
        \hologoVariantRobust{#1}{#2}%
      }{%
        \ltx@ifundefined{HoLogoBkm@#1@#2}{%
          \ltx@ifundefined{HoLogo@#1}{?#1?}{#1}%
        }{%
          \csname HoLogoBkm@#1@#2\endcsname
          \ltx@firstoftwo
        }%
      }%
$   \fi
  \fi
}
%    \end{macrocode}
%    \end{macro}
%    \begin{macro}{\HologoVariant}
%    \begin{macrocode}
\def\HologoVariant#1#2{%
  \ifx\relax#2\relax
    \Hologo{#1}%
  \else
$   \ifincsname
$     \ltx@ifundefined{HoLogoCs@#1@#2}{%
$       #1%
$     }{%
$       \csname HoLogoCs@#1@#2\endcsname\ltx@secondoftwo
$     }%
$   \else
      \HOLOGO@IfExists\texorpdfstring\texorpdfstring\ltx@firstoftwo
      {%
        \HologoVariantRobust{#1}{#2}%
      }{%
        \ltx@ifundefined{HoLogoBkm@#1@#2}{%
          \ltx@ifundefined{HoLogo@#1}{?#1?}{#1}%
        }{%
          \csname HoLogoBkm@#1@#2\endcsname
          \ltx@secondoftwo
        }%
      }%
$   \fi
  \fi
}
%    \end{macrocode}
%    \end{macro}
%
%    \begin{macrocode}
\catcode`\$=3 %
%    \end{macrocode}
%
% \subsubsection{\cs{hologoRobust} and friends}
%
%    \begin{macro}{\hologoRobust}
%    \begin{macrocode}
\ltx@IfUndefined{protected}{%
  \ltx@IfUndefined{DeclareRobustCommand}{%
    \def\hologoRobust#1%
  }{%
    \DeclareRobustCommand*\hologoRobust[1]%
  }%
}{%
  \protected\def\hologoRobust#1%
}%
{%
  \edef\HOLOGO@name{#1}%
  \ltx@IfUndefined{HoLogo@\HOLOGO@Variant\HOLOGO@name}{%
    \@PackageError{hologo}{%
      Unknown logo `\HOLOGO@name'%
    }\@ehc
    ?\HOLOGO@name?%
  }{%
    \ltx@IfUndefined{ver@tex4ht.sty}{%
      \HoLogoFont@font\HOLOGO@name{general}{%
        \csname HoLogo@\HOLOGO@Variant\HOLOGO@name\endcsname
        \ltx@firstoftwo
      }%
    }{%
      \ltx@IfUndefined{HoLogoHtml@\HOLOGO@Variant\HOLOGO@name}{%
        \HOLOGO@name
      }{%
        \csname HoLogoHtml@\HOLOGO@Variant\HOLOGO@name\endcsname
        \ltx@firstoftwo
      }%
    }%
  }%
}
%    \end{macrocode}
%    \end{macro}
%    \begin{macro}{\HologoRobust}
%    \begin{macrocode}
\ltx@IfUndefined{protected}{%
  \ltx@IfUndefined{DeclareRobustCommand}{%
    \def\HologoRobust#1%
  }{%
    \DeclareRobustCommand*\HologoRobust[1]%
  }%
}{%
  \protected\def\HologoRobust#1%
}%
{%
  \edef\HOLOGO@name{#1}%
  \ltx@IfUndefined{HoLogo@\HOLOGO@Variant\HOLOGO@name}{%
    \@PackageError{hologo}{%
      Unknown logo `\HOLOGO@name'%
    }\@ehc
    ?\HOLOGO@name?%
  }{%
    \ltx@IfUndefined{ver@tex4ht.sty}{%
      \HoLogoFont@font\HOLOGO@name{general}{%
        \csname HoLogo@\HOLOGO@Variant\HOLOGO@name\endcsname
        \ltx@secondoftwo
      }%
    }{%
      \ltx@IfUndefined{HoLogoHtml@\HOLOGO@Variant\HOLOGO@name}{%
        \expandafter\HOLOGO@Uppercase\HOLOGO@name
      }{%
        \csname HoLogoHtml@\HOLOGO@Variant\HOLOGO@name\endcsname
        \ltx@secondoftwo
      }%
    }%
  }%
}
%    \end{macrocode}
%    \end{macro}
%    \begin{macro}{\hologoVariantRobust}
%    \begin{macrocode}
\ltx@IfUndefined{protected}{%
  \ltx@IfUndefined{DeclareRobustCommand}{%
    \def\hologoVariantRobust#1#2%
  }{%
    \DeclareRobustCommand*\hologoVariantRobust[2]%
  }%
}{%
  \protected\def\hologoVariantRobust#1#2%
}%
{%
  \begingroup
    \hologoLogoSetup{#1}{variant={#2}}%
    \hologoRobust{#1}%
  \endgroup
}
%    \end{macrocode}
%    \end{macro}
%    \begin{macro}{\HologoVariantRobust}
%    \begin{macrocode}
\ltx@IfUndefined{protected}{%
  \ltx@IfUndefined{DeclareRobustCommand}{%
    \def\HologoVariantRobust#1#2%
  }{%
    \DeclareRobustCommand*\HologoVariantRobust[2]%
  }%
}{%
  \protected\def\HologoVariantRobust#1#2%
}%
{%
  \begingroup
    \hologoLogoSetup{#1}{variant={#2}}%
    \HologoRobust{#1}%
  \endgroup
}
%    \end{macrocode}
%    \end{macro}
%
%    \begin{macro}{\hologorobust}
%    Macro \cs{hologorobust} is only defined for compatibility.
%    Its use is deprecated.
%    \begin{macrocode}
\def\hologorobust{\hologoRobust}
%    \end{macrocode}
%    \end{macro}
%
% \subsection{Helpers}
%
%    \begin{macro}{\HOLOGO@Uppercase}
%    Macro \cs{HOLOGO@Uppercase} is restricted to \cs{uppercase},
%    because \hologo{plainTeX} or \hologo{iniTeX} do not provide
%    \cs{MakeUppercase}.
%    \begin{macrocode}
\def\HOLOGO@Uppercase#1{\uppercase{#1}}
%    \end{macrocode}
%    \end{macro}
%
%    \begin{macro}{\HOLOGO@PdfdocUnicode}
%    \begin{macrocode}
\def\HOLOGO@PdfdocUnicode{%
  \ifx\ifHy@unicode\iftrue
    \expandafter\ltx@secondoftwo
  \else
    \expandafter\ltx@firstoftwo
  \fi
}
%    \end{macrocode}
%    \end{macro}
%
%    \begin{macro}{\HOLOGO@Math}
%    \begin{macrocode}
\def\HOLOGO@MathSetup{%
  \mathsurround0pt\relax
  \HOLOGO@IfExists\f@series{%
    \if b\expandafter\ltx@car\f@series x\@nil
      \csname boldmath\endcsname
   \fi
  }{}%
}
%    \end{macrocode}
%    \end{macro}
%
%    \begin{macro}{\HOLOGO@TempDimen}
%    \begin{macrocode}
\dimendef\HOLOGO@TempDimen=\ltx@zero
%    \end{macrocode}
%    \end{macro}
%    \begin{macro}{\HOLOGO@NegativeKerning}
%    \begin{macrocode}
\def\HOLOGO@NegativeKerning#1{%
  \begingroup
    \HOLOGO@TempDimen=0pt\relax
    \comma@parse@normalized{#1}{%
      \ifdim\HOLOGO@TempDimen=0pt %
        \expandafter\HOLOGO@@NegativeKerning\comma@entry
      \fi
      \ltx@gobble
    }%
    \ifdim\HOLOGO@TempDimen<0pt %
      \kern\HOLOGO@TempDimen
    \fi
  \endgroup
}
%    \end{macrocode}
%    \end{macro}
%    \begin{macro}{\HOLOGO@@NegativeKerning}
%    \begin{macrocode}
\def\HOLOGO@@NegativeKerning#1#2{%
  \setbox\ltx@zero\hbox{#1#2}%
  \HOLOGO@TempDimen=\wd\ltx@zero
  \setbox\ltx@zero\hbox{#1\kern0pt#2}%
  \advance\HOLOGO@TempDimen by -\wd\ltx@zero
}
%    \end{macrocode}
%    \end{macro}
%
%    \begin{macro}{\HOLOGO@SpaceFactor}
%    \begin{macrocode}
\def\HOLOGO@SpaceFactor{%
  \spacefactor1000 %
}
%    \end{macrocode}
%    \end{macro}
%
%    \begin{macro}{\HOLOGO@Span}
%    \begin{macrocode}
\def\HOLOGO@Span#1#2{%
  \HCode{<span class="HoLogo-#1">}%
  #2%
  \HCode{</span>}%
}
%    \end{macrocode}
%    \end{macro}
%
% \subsubsection{Text subscript}
%
%    \begin{macro}{\HOLOGO@SubScript}%
%    \begin{macrocode}
\def\HOLOGO@SubScript#1{%
  \ltx@IfUndefined{textsubscript}{%
    \ltx@IfUndefined{text}{%
      \ltx@mbox{%
        \mathsurround=0pt\relax
        $%
          _{%
            \ltx@IfUndefined{sf@size}{%
              \mathrm{#1}%
            }{%
              \mbox{%
                \fontsize\sf@size{0pt}\selectfont
                #1%
              }%
            }%
          }%
        $%
      }%
    }{%
      \ltx@mbox{%
        \mathsurround=0pt\relax
        $_{\text{#1}}$%
      }%
    }%
  }{%
    \textsubscript{#1}%
  }%
}
%    \end{macrocode}
%    \end{macro}
%
% \subsection{\hologo{TeX} and friends}
%
% \subsubsection{\hologo{TeX}}
%
%    \begin{macro}{\HoLogo@TeX}
%    Source: \hologo{LaTeX} kernel.
%    \begin{macrocode}
\def\HoLogo@TeX#1{%
  T\kern-.1667em\lower.5ex\hbox{E}\kern-.125emX\HOLOGO@SpaceFactor
}
%    \end{macrocode}
%    \end{macro}
%    \begin{macro}{\HoLogoHtml@TeX}
%    \begin{macrocode}
\def\HoLogoHtml@TeX#1{%
  \HoLogoCss@TeX
  \HOLOGO@Span{TeX}{%
    T%
    \HOLOGO@Span{e}{%
      E%
    }%
    X%
  }%
}
%    \end{macrocode}
%    \end{macro}
%    \begin{macro}{\HoLogoCss@TeX}
%    \begin{macrocode}
\def\HoLogoCss@TeX{%
  \Css{%
    span.HoLogo-TeX span.HoLogo-e{%
      position:relative;%
      top:.5ex;%
      margin-left:-.1667em;%
      margin-right:-.125em;%
    }%
  }%
  \Css{%
    a span.HoLogo-TeX span.HoLogo-e{%
      text-decoration:none;%
    }%
  }%
  \global\let\HoLogoCss@TeX\relax
}
%    \end{macrocode}
%    \end{macro}
%
% \subsubsection{\hologo{plainTeX}}
%
%    \begin{macro}{\HoLogo@plainTeX@space}
%    Source: ``The \hologo{TeX}book''
%    \begin{macrocode}
\def\HoLogo@plainTeX@space#1{%
  \HOLOGO@mbox{#1{p}{P}lain}\HOLOGO@space\hologo{TeX}%
}
%    \end{macrocode}
%    \end{macro}
%    \begin{macro}{\HoLogoCs@plainTeX@space}
%    \begin{macrocode}
\def\HoLogoCs@plainTeX@space#1{#1{p}{P}lain TeX}%
%    \end{macrocode}
%    \end{macro}
%    \begin{macro}{\HoLogoBkm@plainTeX@space}
%    \begin{macrocode}
\def\HoLogoBkm@plainTeX@space#1{%
  #1{p}{P}lain \hologo{TeX}%
}
%    \end{macrocode}
%    \end{macro}
%    \begin{macro}{\HoLogoHtml@plainTeX@space}
%    \begin{macrocode}
\def\HoLogoHtml@plainTeX@space#1{%
  #1{p}{P}lain \hologo{TeX}%
}
%    \end{macrocode}
%    \end{macro}
%
%    \begin{macro}{\HoLogo@plainTeX@hyphen}
%    \begin{macrocode}
\def\HoLogo@plainTeX@hyphen#1{%
  \HOLOGO@mbox{#1{p}{P}lain}\HOLOGO@hyphen\hologo{TeX}%
}
%    \end{macrocode}
%    \end{macro}
%    \begin{macro}{\HoLogoCs@plainTeX@hyphen}
%    \begin{macrocode}
\def\HoLogoCs@plainTeX@hyphen#1{#1{p}{P}lain-TeX}
%    \end{macrocode}
%    \end{macro}
%    \begin{macro}{\HoLogoBkm@plainTeX@hyphen}
%    \begin{macrocode}
\def\HoLogoBkm@plainTeX@hyphen#1{%
  #1{p}{P}lain-\hologo{TeX}%
}
%    \end{macrocode}
%    \end{macro}
%    \begin{macro}{\HoLogoHtml@plainTeX@hyphen}
%    \begin{macrocode}
\def\HoLogoHtml@plainTeX@hyphen#1{%
  #1{p}{P}lain-\hologo{TeX}%
}
%    \end{macrocode}
%    \end{macro}
%
%    \begin{macro}{\HoLogo@plainTeX@runtogether}
%    \begin{macrocode}
\def\HoLogo@plainTeX@runtogether#1{%
  \HOLOGO@mbox{#1{p}{P}lain\hologo{TeX}}%
}
%    \end{macrocode}
%    \end{macro}
%    \begin{macro}{\HoLogoCs@plainTeX@runtogether}
%    \begin{macrocode}
\def\HoLogoCs@plainTeX@runtogether#1{#1{p}{P}lainTeX}
%    \end{macrocode}
%    \end{macro}
%    \begin{macro}{\HoLogoBkm@plainTeX@runtogether}
%    \begin{macrocode}
\def\HoLogoBkm@plainTeX@runtogether#1{%
  #1{p}{P}lain\hologo{TeX}%
}
%    \end{macrocode}
%    \end{macro}
%    \begin{macro}{\HoLogoHtml@plainTeX@runtogether}
%    \begin{macrocode}
\def\HoLogoHtml@plainTeX@runtogether#1{%
  #1{p}{P}lain\hologo{TeX}%
}
%    \end{macrocode}
%    \end{macro}
%
%    \begin{macro}{\HoLogo@plainTeX}
%    \begin{macrocode}
\def\HoLogo@plainTeX{\HoLogo@plainTeX@space}
%    \end{macrocode}
%    \end{macro}
%    \begin{macro}{\HoLogoCs@plainTeX}
%    \begin{macrocode}
\def\HoLogoCs@plainTeX{\HoLogoCs@plainTeX@space}
%    \end{macrocode}
%    \end{macro}
%    \begin{macro}{\HoLogoBkm@plainTeX}
%    \begin{macrocode}
\def\HoLogoBkm@plainTeX{\HoLogoBkm@plainTeX@space}
%    \end{macrocode}
%    \end{macro}
%    \begin{macro}{\HoLogoHtml@plainTeX}
%    \begin{macrocode}
\def\HoLogoHtml@plainTeX{\HoLogoHtml@plainTeX@space}
%    \end{macrocode}
%    \end{macro}
%
% \subsubsection{\hologo{LaTeX}}
%
%    Source: \hologo{LaTeX} kernel.
%\begin{quote}
%\begin{verbatim}
%\DeclareRobustCommand{\LaTeX}{%
%  L%
%  \kern-.36em%
%  {%
%    \sbox\z@ T%
%    \vbox to\ht\z@{%
%      \hbox{%
%        \check@mathfonts
%        \fontsize\sf@size\z@
%        \math@fontsfalse
%        \selectfont
%        A%
%      }%
%      \vss
%    }%
%  }%
%  \kern-.15em%
%  \TeX
%}
%\end{verbatim}
%\end{quote}
%
%    \begin{macro}{\HoLogo@La}
%    \begin{macrocode}
\def\HoLogo@La#1{%
  L%
  \kern-.36em%
  \begingroup
    \setbox\ltx@zero\hbox{T}%
    \vbox to\ht\ltx@zero{%
      \hbox{%
        \ltx@ifundefined{check@mathfonts}{%
          \csname sevenrm\endcsname
        }{%
          \check@mathfonts
          \fontsize\sf@size{0pt}%
          \math@fontsfalse\selectfont
        }%
        A%
      }%
      \vss
    }%
  \endgroup
}
%    \end{macrocode}
%    \end{macro}
%
%    \begin{macro}{\HoLogo@LaTeX}
%    Source: \hologo{LaTeX} kernel.
%    \begin{macrocode}
\def\HoLogo@LaTeX#1{%
  \hologo{La}%
  \kern-.15em%
  \hologo{TeX}%
}
%    \end{macrocode}
%    \end{macro}
%    \begin{macro}{\HoLogoHtml@LaTeX}
%    \begin{macrocode}
\def\HoLogoHtml@LaTeX#1{%
  \HoLogoCss@LaTeX
  \HOLOGO@Span{LaTeX}{%
    L%
    \HOLOGO@Span{a}{%
      A%
    }%
    \hologo{TeX}%
  }%
}
%    \end{macrocode}
%    \end{macro}
%    \begin{macro}{\HoLogoCss@LaTeX}
%    \begin{macrocode}
\def\HoLogoCss@LaTeX{%
  \Css{%
    span.HoLogo-LaTeX span.HoLogo-a{%
      position:relative;%
      top:-.5ex;%
      margin-left:-.36em;%
      margin-right:-.15em;%
      font-size:85\%;%
    }%
  }%
  \global\let\HoLogoCss@LaTeX\relax
}
%    \end{macrocode}
%    \end{macro}
%
% \subsubsection{\hologo{(La)TeX}}
%
%    \begin{macro}{\HoLogo@LaTeXTeX}
%    The kerning around the parentheses is taken
%    from package \xpackage{dtklogos} \cite{dtklogos}.
%\begin{quote}
%\begin{verbatim}
%\DeclareRobustCommand{\LaTeXTeX}{%
%  (%
%  \kern-.15em%
%  L%
%  \kern-.36em%
%  {%
%    \sbox\z@ T%
%    \vbox to\ht0{%
%      \hbox{%
%        $\m@th$%
%        \csname S@\f@size\endcsname
%        \fontsize\sf@size\z@
%        \math@fontsfalse
%        \selectfont
%        A%
%      }%
%      \vss
%    }%
%  }%
%  \kern-.2em%
%  )%
%  \kern-.15em%
%  \TeX
%}
%\end{verbatim}
%\end{quote}
%    \begin{macrocode}
\def\HoLogo@LaTeXTeX#1{%
  (%
  \kern-.15em%
  \hologo{La}%
  \kern-.2em%
  )%
  \kern-.15em%
  \hologo{TeX}%
}
%    \end{macrocode}
%    \end{macro}
%    \begin{macro}{\HoLogoBkm@LaTeXTeX}
%    \begin{macrocode}
\def\HoLogoBkm@LaTeXTeX#1{(La)TeX}
%    \end{macrocode}
%    \end{macro}
%
%    \begin{macro}{\HoLogo@(La)TeX}
%    \begin{macrocode}
\expandafter
\let\csname HoLogo@(La)TeX\endcsname\HoLogo@LaTeXTeX
%    \end{macrocode}
%    \end{macro}
%    \begin{macro}{\HoLogoBkm@(La)TeX}
%    \begin{macrocode}
\expandafter
\let\csname HoLogoBkm@(La)TeX\endcsname\HoLogoBkm@LaTeXTeX
%    \end{macrocode}
%    \end{macro}
%    \begin{macro}{\HoLogoHtml@LaTeXTeX}
%    \begin{macrocode}
\def\HoLogoHtml@LaTeXTeX#1{%
  \HoLogoCss@LaTeXTeX
  \HOLOGO@Span{LaTeXTeX}{%
    (%
    \HOLOGO@Span{L}{L}%
    \HOLOGO@Span{a}{A}%
    \HOLOGO@Span{ParenRight}{)}%
    \hologo{TeX}%
  }%
}
%    \end{macrocode}
%    \end{macro}
%    \begin{macro}{\HoLogoHtml@(La)TeX}
%    Kerning after opening parentheses and before closing parentheses
%    is $-0.1$\,em. The original values $-0.15$\,em
%    looked too ugly for a serif font.
%    \begin{macrocode}
\expandafter
\let\csname HoLogoHtml@(La)TeX\endcsname\HoLogoHtml@LaTeXTeX
%    \end{macrocode}
%    \end{macro}
%    \begin{macro}{\HoLogoCss@LaTeXTeX}
%    \begin{macrocode}
\def\HoLogoCss@LaTeXTeX{%
  \Css{%
    span.HoLogo-LaTeXTeX span.HoLogo-L{%
      margin-left:-.1em;%
    }%
  }%
  \Css{%
    span.HoLogo-LaTeXTeX span.HoLogo-a{%
      position:relative;%
      top:-.5ex;%
      margin-left:-.36em;%
      margin-right:-.1em;%
      font-size:85\%;%
    }%
  }%
  \Css{%
    span.HoLogo-LaTeXTeX span.HoLogo-ParenRight{%
      margin-right:-.15em;%
    }%
  }%
  \global\let\HoLogoCss@LaTeXTeX\relax
}
%    \end{macrocode}
%    \end{macro}
%
% \subsubsection{\hologo{LaTeXe}}
%
%    \begin{macro}{\HoLogo@LaTeXe}
%    Source: \hologo{LaTeX} kernel
%    \begin{macrocode}
\def\HoLogo@LaTeXe#1{%
  \hologo{LaTeX}%
  \kern.15em%
  \hbox{%
    \HOLOGO@MathSetup
    2%
    $_{\textstyle\varepsilon}$%
  }%
}
%    \end{macrocode}
%    \end{macro}
%
%    \begin{macro}{\HoLogoCs@LaTeXe}
%    \begin{macrocode}
\ifnum64=`\^^^^0040\relax % test for big chars of LuaTeX/XeTeX
  \catcode`\$=9 %
  \catcode`\&=14 %
\else
  \catcode`\$=14 %
  \catcode`\&=9 %
\fi
\def\HoLogoCs@LaTeXe#1{%
  LaTeX2%
$ \string ^^^^0395%
& e%
}%
\catcode`\$=3 %
\catcode`\&=4 %
%    \end{macrocode}
%    \end{macro}
%
%    \begin{macro}{\HoLogoBkm@LaTeXe}
%    \begin{macrocode}
\def\HoLogoBkm@LaTeXe#1{%
  \hologo{LaTeX}%
  2%
  \HOLOGO@PdfdocUnicode{e}{\textepsilon}%
}
%    \end{macrocode}
%    \end{macro}
%
%    \begin{macro}{\HoLogoHtml@LaTeXe}
%    \begin{macrocode}
\def\HoLogoHtml@LaTeXe#1{%
  \HoLogoCss@LaTeXe
  \HOLOGO@Span{LaTeX2e}{%
    \hologo{LaTeX}%
    \HOLOGO@Span{2}{2}%
    \HOLOGO@Span{e}{%
      \HOLOGO@MathSetup
      \ensuremath{\textstyle\varepsilon}%
    }%
  }%
}
%    \end{macrocode}
%    \end{macro}
%    \begin{macro}{\HoLogoCss@LaTeXe}
%    \begin{macrocode}
\def\HoLogoCss@LaTeXe{%
  \Css{%
    span.HoLogo-LaTeX2e span.HoLogo-2{%
      padding-left:.15em;%
    }%
  }%
  \Css{%
    span.HoLogo-LaTeX2e span.HoLogo-e{%
      position:relative;%
      top:.35ex;%
      text-decoration:none;%
    }%
  }%
  \global\let\HoLogoCss@LaTeXe\relax
}
%    \end{macrocode}
%    \end{macro}
%
%    \begin{macro}{\HoLogo@LaTeX2e}
%    \begin{macrocode}
\expandafter
\let\csname HoLogo@LaTeX2e\endcsname\HoLogo@LaTeXe
%    \end{macrocode}
%    \end{macro}
%    \begin{macro}{\HoLogoCs@LaTeX2e}
%    \begin{macrocode}
\expandafter
\let\csname HoLogoCs@LaTeX2e\endcsname\HoLogoCs@LaTeXe
%    \end{macrocode}
%    \end{macro}
%    \begin{macro}{\HoLogoBkm@LaTeX2e}
%    \begin{macrocode}
\expandafter
\let\csname HoLogoBkm@LaTeX2e\endcsname\HoLogoBkm@LaTeXe
%    \end{macrocode}
%    \end{macro}
%    \begin{macro}{\HoLogoHtml@LaTeX2e}
%    \begin{macrocode}
\expandafter
\let\csname HoLogoHtml@LaTeX2e\endcsname\HoLogoHtml@LaTeXe
%    \end{macrocode}
%    \end{macro}
%
% \subsubsection{\hologo{LaTeX3}}
%
%    \begin{macro}{\HoLogo@LaTeX3}
%    Source: \hologo{LaTeX} kernel
%    \begin{macrocode}
\expandafter\def\csname HoLogo@LaTeX3\endcsname#1{%
  \hologo{LaTeX}%
  3%
}
%    \end{macrocode}
%    \end{macro}
%
%    \begin{macro}{\HoLogoBkm@LaTeX3}
%    \begin{macrocode}
\expandafter\def\csname HoLogoBkm@LaTeX3\endcsname#1{%
  \hologo{LaTeX}%
  3%
}
%    \end{macrocode}
%    \end{macro}
%    \begin{macro}{\HoLogoHtml@LaTeX3}
%    \begin{macrocode}
\expandafter
\let\csname HoLogoHtml@LaTeX3\expandafter\endcsname
\csname HoLogo@LaTeX3\endcsname
%    \end{macrocode}
%    \end{macro}
%
% \subsubsection{\hologo{LaTeXML}}
%
%    \begin{macro}{\HoLogo@LaTeXML}
%    \begin{macrocode}
\def\HoLogo@LaTeXML#1{%
  \HOLOGO@mbox{%
    \hologo{La}%
    \kern-.15em%
    T%
    \kern-.1667em%
    \lower.5ex\hbox{E}%
    \kern-.125em%
    \HoLogoFont@font{LaTeXML}{sc}{xml}%
  }%
}
%    \end{macrocode}
%    \end{macro}
%    \begin{macro}{\HoLogoHtml@pdfLaTeX}
%    \begin{macrocode}
\def\HoLogoHtml@LaTeXML#1{%
  \HOLOGO@Span{LaTeXML}{%
    \HoLogoCss@LaTeX
    \HoLogoCss@TeX
    \HOLOGO@Span{LaTeX}{%
      L%
      \HOLOGO@Span{a}{%
        A%
      }%
    }%
    \HOLOGO@Span{TeX}{%
      T%
      \HOLOGO@Span{e}{%
        E%
      }%
    }%
    \HCode{<span style="font-variant: small-caps;">}%
    xml%
    \HCode{</span>}%
  }%
}
%    \end{macrocode}
%    \end{macro}
%
% \subsubsection{\hologo{eTeX}}
%
%    \begin{macro}{\HoLogo@eTeX}
%    Source: package \xpackage{etex}
%    \begin{macrocode}
\def\HoLogo@eTeX#1{%
  \ltx@mbox{%
    \HOLOGO@MathSetup
    $\varepsilon$%
    -%
    \HOLOGO@NegativeKerning{-T,T-,To}%
    \hologo{TeX}%
  }%
}
%    \end{macrocode}
%    \end{macro}
%    \begin{macro}{\HoLogoCs@eTeX}
%    \begin{macrocode}
\ifnum64=`\^^^^0040\relax % test for big chars of LuaTeX/XeTeX
  \catcode`\$=9 %
  \catcode`\&=14 %
\else
  \catcode`\$=14 %
  \catcode`\&=9 %
\fi
\def\HoLogoCs@eTeX#1{%
$ #1{\string ^^^^0395}{\string ^^^^03b5}%
& #1{e}{E}%
  TeX%
}%
\catcode`\$=3 %
\catcode`\&=4 %
%    \end{macrocode}
%    \end{macro}
%    \begin{macro}{\HoLogoBkm@eTeX}
%    \begin{macrocode}
\def\HoLogoBkm@eTeX#1{%
  \HOLOGO@PdfdocUnicode{#1{e}{E}}{\textepsilon}%
  -%
  \hologo{TeX}%
}
%    \end{macrocode}
%    \end{macro}
%    \begin{macro}{\HoLogoHtml@eTeX}
%    \begin{macrocode}
\def\HoLogoHtml@eTeX#1{%
  \ltx@mbox{%
    \HOLOGO@MathSetup
    $\varepsilon$%
    -%
    \hologo{TeX}%
  }%
}
%    \end{macrocode}
%    \end{macro}
%
% \subsubsection{\hologo{iniTeX}}
%
%    \begin{macro}{\HoLogo@iniTeX}
%    \begin{macrocode}
\def\HoLogo@iniTeX#1{%
  \HOLOGO@mbox{%
    #1{i}{I}ni\hologo{TeX}%
  }%
}
%    \end{macrocode}
%    \end{macro}
%    \begin{macro}{\HoLogoCs@iniTeX}
%    \begin{macrocode}
\def\HoLogoCs@iniTeX#1{#1{i}{I}niTeX}
%    \end{macrocode}
%    \end{macro}
%    \begin{macro}{\HoLogoBkm@iniTeX}
%    \begin{macrocode}
\def\HoLogoBkm@iniTeX#1{%
  #1{i}{I}ni\hologo{TeX}%
}
%    \end{macrocode}
%    \end{macro}
%    \begin{macro}{\HoLogoHtml@iniTeX}
%    \begin{macrocode}
\let\HoLogoHtml@iniTeX\HoLogo@iniTeX
%    \end{macrocode}
%    \end{macro}
%
% \subsubsection{\hologo{virTeX}}
%
%    \begin{macro}{\HoLogo@virTeX}
%    \begin{macrocode}
\def\HoLogo@virTeX#1{%
  \HOLOGO@mbox{%
    #1{v}{V}ir\hologo{TeX}%
  }%
}
%    \end{macrocode}
%    \end{macro}
%    \begin{macro}{\HoLogoCs@virTeX}
%    \begin{macrocode}
\def\HoLogoCs@virTeX#1{#1{v}{V}irTeX}
%    \end{macrocode}
%    \end{macro}
%    \begin{macro}{\HoLogoBkm@virTeX}
%    \begin{macrocode}
\def\HoLogoBkm@virTeX#1{%
  #1{v}{V}ir\hologo{TeX}%
}
%    \end{macrocode}
%    \end{macro}
%    \begin{macro}{\HoLogoHtml@virTeX}
%    \begin{macrocode}
\let\HoLogoHtml@virTeX\HoLogo@virTeX
%    \end{macrocode}
%    \end{macro}
%
% \subsubsection{\hologo{SliTeX}}
%
% \paragraph{Definitions of the three variants.}
%
%    \begin{macro}{\HoLogo@SLiTeX@lift}
%    \begin{macrocode}
\def\HoLogo@SLiTeX@lift#1{%
  \HoLogoFont@font{SliTeX}{rm}{%
    S%
    \kern-.06em%
    L%
    \kern-.18em%
    \raise.32ex\hbox{\HoLogoFont@font{SliTeX}{sc}{i}}%
    \HOLOGO@discretionary
    \kern-.06em%
    \hologo{TeX}%
  }%
}
%    \end{macrocode}
%    \end{macro}
%    \begin{macro}{\HoLogoBkm@SLiTeX@lift}
%    \begin{macrocode}
\def\HoLogoBkm@SLiTeX@lift#1{SLiTeX}
%    \end{macrocode}
%    \end{macro}
%    \begin{macro}{\HoLogoHtml@SLiTeX@lift}
%    \begin{macrocode}
\def\HoLogoHtml@SLiTeX@lift#1{%
  \HoLogoCss@SLiTeX@lift
  \HOLOGO@Span{SLiTeX-lift}{%
    \HoLogoFont@font{SliTeX}{rm}{%
      S%
      \HOLOGO@Span{L}{L}%
      \HOLOGO@Span{i}{i}%
      \hologo{TeX}%
    }%
  }%
}
%    \end{macrocode}
%    \end{macro}
%    \begin{macro}{\HoLogoCss@SLiTeX@lift}
%    \begin{macrocode}
\def\HoLogoCss@SLiTeX@lift{%
  \Css{%
    span.HoLogo-SLiTeX-lift span.HoLogo-L{%
      margin-left:-.06em;%
      margin-right:-.18em;%
    }%
  }%
  \Css{%
    span.HoLogo-SLiTeX-lift span.HoLogo-i{%
      position:relative;%
      top:-.32ex;%
      margin-right:-.06em;%
      font-variant:small-caps;%
    }%
  }%
  \global\let\HoLogoCss@SLiTeX@lift\relax
}
%    \end{macrocode}
%    \end{macro}
%
%    \begin{macro}{\HoLogo@SliTeX@simple}
%    \begin{macrocode}
\def\HoLogo@SliTeX@simple#1{%
  \HoLogoFont@font{SliTeX}{rm}{%
    \ltx@mbox{%
      \HoLogoFont@font{SliTeX}{sc}{Sli}%
    }%
    \HOLOGO@discretionary
    \hologo{TeX}%
  }%
}
%    \end{macrocode}
%    \end{macro}
%    \begin{macro}{\HoLogoBkm@SliTeX@simple}
%    \begin{macrocode}
\def\HoLogoBkm@SliTeX@simple#1{SliTeX}
%    \end{macrocode}
%    \end{macro}
%    \begin{macro}{\HoLogoHtml@SliTeX@simple}
%    \begin{macrocode}
\let\HoLogoHtml@SliTeX@simple\HoLogo@SliTeX@simple
%    \end{macrocode}
%    \end{macro}
%
%    \begin{macro}{\HoLogo@SliTeX@narrow}
%    \begin{macrocode}
\def\HoLogo@SliTeX@narrow#1{%
  \HoLogoFont@font{SliTeX}{rm}{%
    \ltx@mbox{%
      S%
      \kern-.06em%
      \HoLogoFont@font{SliTeX}{sc}{%
        l%
        \kern-.035em%
        i%
      }%
    }%
    \HOLOGO@discretionary
    \kern-.06em%
    \hologo{TeX}%
  }%
}
%    \end{macrocode}
%    \end{macro}
%    \begin{macro}{\HoLogoBkm@SliTeX@narrow}
%    \begin{macrocode}
\def\HoLogoBkm@SliTeX@narrow#1{SliTeX}
%    \end{macrocode}
%    \end{macro}
%    \begin{macro}{\HoLogoHtml@SliTeX@narrow}
%    \begin{macrocode}
\def\HoLogoHtml@SliTeX@narrow#1{%
  \HoLogoCss@SliTeX@narrow
  \HOLOGO@Span{SliTeX-narrow}{%
    \HoLogoFont@font{SliTeX}{rm}{%
      S%
        \HOLOGO@Span{l}{l}%
        \HOLOGO@Span{i}{i}%
      \hologo{TeX}%
    }%
  }%
}
%    \end{macrocode}
%    \end{macro}
%    \begin{macro}{\HoLogoCss@SliTeX@narrow}
%    \begin{macrocode}
\def\HoLogoCss@SliTeX@narrow{%
  \Css{%
    span.HoLogo-SliTeX-narrow span.HoLogo-l{%
      margin-left:-.06em;%
      margin-right:-.035em;%
      font-variant:small-caps;%
    }%
  }%
  \Css{%
    span.HoLogo-SliTeX-narrow span.HoLogo-i{%
      margin-right:-.06em;%
      font-variant:small-caps;%
    }%
  }%
  \global\let\HoLogoCss@SliTeX@narrow\relax
}
%    \end{macrocode}
%    \end{macro}
%
% \paragraph{Macro set completion.}
%
%    \begin{macro}{\HoLogo@SLiTeX@simple}
%    \begin{macrocode}
\def\HoLogo@SLiTeX@simple{\HoLogo@SliTeX@simple}
%    \end{macrocode}
%    \end{macro}
%    \begin{macro}{\HoLogoBkm@SLiTeX@simple}
%    \begin{macrocode}
\def\HoLogoBkm@SLiTeX@simple{\HoLogoBkm@SliTeX@simple}
%    \end{macrocode}
%    \end{macro}
%    \begin{macro}{\HoLogoHtml@SLiTeX@simple}
%    \begin{macrocode}
\def\HoLogoHtml@SLiTeX@simple{\HoLogoHtml@SliTeX@simple}
%    \end{macrocode}
%    \end{macro}
%
%    \begin{macro}{\HoLogo@SLiTeX@narrow}
%    \begin{macrocode}
\def\HoLogo@SLiTeX@narrow{\HoLogo@SliTeX@narrow}
%    \end{macrocode}
%    \end{macro}
%    \begin{macro}{\HoLogoBkm@SLiTeX@narrow}
%    \begin{macrocode}
\def\HoLogoBkm@SLiTeX@narrow{\HoLogoBkm@SliTeX@narrow}
%    \end{macrocode}
%    \end{macro}
%    \begin{macro}{\HoLogoHtml@SLiTeX@narrow}
%    \begin{macrocode}
\def\HoLogoHtml@SLiTeX@narrow{\HoLogoHtml@SliTeX@narrow}
%    \end{macrocode}
%    \end{macro}
%
%    \begin{macro}{\HoLogo@SliTeX@lift}
%    \begin{macrocode}
\def\HoLogo@SliTeX@lift{\HoLogo@SLiTeX@lift}
%    \end{macrocode}
%    \end{macro}
%    \begin{macro}{\HoLogoBkm@SliTeX@lift}
%    \begin{macrocode}
\def\HoLogoBkm@SliTeX@lift{\HoLogoBkm@SLiTeX@lift}
%    \end{macrocode}
%    \end{macro}
%    \begin{macro}{\HoLogoHtml@SliTeX@lift}
%    \begin{macrocode}
\def\HoLogoHtml@SliTeX@lift{\HoLogoHtml@SLiTeX@lift}
%    \end{macrocode}
%    \end{macro}
%
% \paragraph{Defaults.}
%
%    \begin{macro}{\HoLogo@SLiTeX}
%    \begin{macrocode}
\def\HoLogo@SLiTeX{\HoLogo@SLiTeX@lift}
%    \end{macrocode}
%    \end{macro}
%    \begin{macro}{\HoLogoBkm@SLiTeX}
%    \begin{macrocode}
\def\HoLogoBkm@SLiTeX{\HoLogoBkm@SLiTeX@lift}
%    \end{macrocode}
%    \end{macro}
%    \begin{macro}{\HoLogoHtml@SLiTeX}
%    \begin{macrocode}
\def\HoLogoHtml@SLiTeX{\HoLogoHtml@SLiTeX@lift}
%    \end{macrocode}
%    \end{macro}
%
%    \begin{macro}{\HoLogo@SliTeX}
%    \begin{macrocode}
\def\HoLogo@SliTeX{\HoLogo@SliTeX@narrow}
%    \end{macrocode}
%    \end{macro}
%    \begin{macro}{\HoLogoBkm@SliTeX}
%    \begin{macrocode}
\def\HoLogoBkm@SliTeX{\HoLogoBkm@SliTeX@narrow}
%    \end{macrocode}
%    \end{macro}
%    \begin{macro}{\HoLogoHtml@SliTeX}
%    \begin{macrocode}
\def\HoLogoHtml@SliTeX{\HoLogoHtml@SliTeX@narrow}
%    \end{macrocode}
%    \end{macro}
%
% \subsubsection{\hologo{LuaTeX}}
%
%    \begin{macro}{\HoLogo@LuaTeX}
%    The kerning is an idea of Hans Hagen, see mailing list
%    `luatex at tug dot org' in March 2010.
%    \begin{macrocode}
\def\HoLogo@LuaTeX#1{%
  \HOLOGO@mbox{%
    Lua%
    \HOLOGO@NegativeKerning{aT,oT,To}%
    \hologo{TeX}%
  }%
}
%    \end{macrocode}
%    \end{macro}
%    \begin{macro}{\HoLogoHtml@LuaTeX}
%    \begin{macrocode}
\let\HoLogoHtml@LuaTeX\HoLogo@LuaTeX
%    \end{macrocode}
%    \end{macro}
%
% \subsubsection{\hologo{LuaLaTeX}}
%
%    \begin{macro}{\HoLogo@LuaLaTeX}
%    \begin{macrocode}
\def\HoLogo@LuaLaTeX#1{%
  \HOLOGO@mbox{%
    Lua%
    \hologo{LaTeX}%
  }%
}
%    \end{macrocode}
%    \end{macro}
%    \begin{macro}{\HoLogoHtml@LuaLaTeX}
%    \begin{macrocode}
\let\HoLogoHtml@LuaLaTeX\HoLogo@LuaLaTeX
%    \end{macrocode}
%    \end{macro}
%
% \subsubsection{\hologo{XeTeX}, \hologo{XeLaTeX}}
%
%    \begin{macro}{\HOLOGO@IfCharExists}
%    \begin{macrocode}
\ifluatex
  \ifnum\luatexversion<36 %
  \else
    \def\HOLOGO@IfCharExists#1{%
      \ifnum
        \directlua{%
           if luaotfload and luaotfload.aux then
             if luaotfload.aux.font_has_glyph(%
                    font.current(), \number#1) then % 	 
	       tex.print("1") % 	 
	     end % 	 
	   elseif font and font.fonts and font.current then %
            local f = font.fonts[font.current()]%
            if f.characters and f.characters[\number#1] then %
              tex.print("1")%
            end %
          end%
        }0=\ltx@zero
        \expandafter\ltx@secondoftwo
      \else
        \expandafter\ltx@firstoftwo
      \fi
    }%
  \fi
\fi
\ltx@IfUndefined{HOLOGO@IfCharExists}{%
  \def\HOLOGO@@IfCharExists#1{%
    \begingroup
      \tracinglostchars=\ltx@zero
      \setbox\ltx@zero=\hbox{%
        \kern7sp\char#1\relax
        \ifnum\lastkern>\ltx@zero
          \expandafter\aftergroup\csname iffalse\endcsname
        \else
          \expandafter\aftergroup\csname iftrue\endcsname
        \fi
      }%
      % \if{true|false} from \aftergroup
      \endgroup
      \expandafter\ltx@firstoftwo
    \else
      \endgroup
      \expandafter\ltx@secondoftwo
    \fi
  }%
  \ifxetex
    \ltx@IfUndefined{XeTeXfonttype}{}{%
      \ltx@IfUndefined{XeTeXcharglyph}{}{%
        \def\HOLOGO@IfCharExists#1{%
          \ifnum\XeTeXfonttype\font>\ltx@zero
            \expandafter\ltx@firstofthree
          \else
            \expandafter\ltx@gobble
          \fi
          {%
            \ifnum\XeTeXcharglyph#1>\ltx@zero
              \expandafter\ltx@firstoftwo
            \else
              \expandafter\ltx@secondoftwo
            \fi
          }%
          \HOLOGO@@IfCharExists{#1}%
        }%
      }%
    }%
  \fi
}{}
\ltx@ifundefined{HOLOGO@IfCharExists}{%
  \ifnum64=`\^^^^0040\relax % test for big chars of LuaTeX/XeTeX
    \let\HOLOGO@IfCharExists\HOLOGO@@IfCharExists
  \else
    \def\HOLOGO@IfCharExists#1{%
      \ifnum#1>255 %
        \expandafter\ltx@fourthoffour
      \fi
      \HOLOGO@@IfCharExists{#1}%
    }%
  \fi
}{}
%    \end{macrocode}
%    \end{macro}
%
%    \begin{macro}{\HoLogo@Xe}
%    Source: package \xpackage{dtklogos}
%    \begin{macrocode}
\def\HoLogo@Xe#1{%
  X%
  \kern-.1em\relax
  \HOLOGO@IfCharExists{"018E}{%
    \lower.5ex\hbox{\char"018E}%
  }{%
    \chardef\HOLOGO@choice=\ltx@zero
    \ifdim\fontdimen\ltx@one\font>0pt %
      \ltx@IfUndefined{rotatebox}{%
        \ltx@IfUndefined{pgftext}{%
          \ltx@IfUndefined{psscalebox}{%
            \ltx@IfUndefined{HOLOGO@ScaleBox@\hologoDriver}{%
            }{%
              \chardef\HOLOGO@choice=4 %
            }%
          }{%
            \chardef\HOLOGO@choice=3 %
          }%
        }{%
          \chardef\HOLOGO@choice=2 %
        }%
      }{%
        \chardef\HOLOGO@choice=1 %
      }%
      \ifcase\HOLOGO@choice
        \HOLOGO@WarningUnsupportedDriver{Xe}%
        e%
      \or % 1: \rotatebox
        \begingroup
          \setbox\ltx@zero\hbox{\rotatebox{180}{E}}%
          \ltx@LocDimenA=\dp\ltx@zero
          \advance\ltx@LocDimenA by -.5ex\relax
          \raise\ltx@LocDimenA\box\ltx@zero
        \endgroup
      \or % 2: \pgftext
        \lower.5ex\hbox{%
          \pgfpicture
            \pgftext[rotate=180]{E}%
          \endpgfpicture
        }%
      \or % 3: \psscalebox
        \begingroup
          \setbox\ltx@zero\hbox{\psscalebox{-1 -1}{E}}%
          \ltx@LocDimenA=\dp\ltx@zero
          \advance\ltx@LocDimenA by -.5ex\relax
          \raise\ltx@LocDimenA\box\ltx@zero
        \endgroup
      \or % 4: \HOLOGO@PointReflectBox
        \lower.5ex\hbox{\HOLOGO@PointReflectBox{E}}%
      \else
        \@PackageError{hologo}{Internal error (choice/it}\@ehc
      \fi
    \else
      \ltx@IfUndefined{reflectbox}{%
        \ltx@IfUndefined{pgftext}{%
          \ltx@IfUndefined{psscalebox}{%
            \ltx@IfUndefined{HOLOGO@ScaleBox@\hologoDriver}{%
            }{%
              \chardef\HOLOGO@choice=4 %
            }%
          }{%
            \chardef\HOLOGO@choice=3 %
          }%
        }{%
          \chardef\HOLOGO@choice=2 %
        }%
      }{%
        \chardef\HOLOGO@choice=1 %
      }%
      \ifcase\HOLOGO@choice
        \HOLOGO@WarningUnsupportedDriver{Xe}%
        e%
      \or % 1: reflectbox
        \lower.5ex\hbox{%
          \reflectbox{E}%
        }%
      \or % 2: \pgftext
        \lower.5ex\hbox{%
          \pgfpicture
            \pgftransformxscale{-1}%
            \pgftext{E}%
          \endpgfpicture
        }%
      \or % 3: \psscalebox
        \lower.5ex\hbox{%
          \psscalebox{-1 1}{E}%
        }%
      \or % 4: \HOLOGO@Reflectbox
        \lower.5ex\hbox{%
          \HOLOGO@ReflectBox{E}%
        }%
      \else
        \@PackageError{hologo}{Internal error (choice/up)}\@ehc
      \fi
    \fi
  }%
}
%    \end{macrocode}
%    \end{macro}
%    \begin{macro}{\HoLogoHtml@Xe}
%    \begin{macrocode}
\def\HoLogoHtml@Xe#1{%
  \HoLogoCss@Xe
  \HOLOGO@Span{Xe}{%
    X%
    \HOLOGO@Span{e}{%
      \HCode{&\ltx@hashchar x018e;}%
    }%
  }%
}
%    \end{macrocode}
%    \end{macro}
%    \begin{macro}{\HoLogoCss@Xe}
%    \begin{macrocode}
\def\HoLogoCss@Xe{%
  \Css{%
    span.HoLogo-Xe span.HoLogo-e{%
      position:relative;%
      top:.5ex;%
      left-margin:-.1em;%
    }%
  }%
  \global\let\HoLogoCss@Xe\relax
}
%    \end{macrocode}
%    \end{macro}
%
%    \begin{macro}{\HoLogo@XeTeX}
%    \begin{macrocode}
\def\HoLogo@XeTeX#1{%
  \hologo{Xe}%
  \kern-.15em\relax
  \hologo{TeX}%
}
%    \end{macrocode}
%    \end{macro}
%
%    \begin{macro}{\HoLogoHtml@XeTeX}
%    \begin{macrocode}
\def\HoLogoHtml@XeTeX#1{%
  \HoLogoCss@XeTeX
  \HOLOGO@Span{XeTeX}{%
    \hologo{Xe}%
    \hologo{TeX}%
  }%
}
%    \end{macrocode}
%    \end{macro}
%    \begin{macro}{\HoLogoCss@XeTeX}
%    \begin{macrocode}
\def\HoLogoCss@XeTeX{%
  \Css{%
    span.HoLogo-XeTeX span.HoLogo-TeX{%
      margin-left:-.15em;%
    }%
  }%
  \global\let\HoLogoCss@XeTeX\relax
}
%    \end{macrocode}
%    \end{macro}
%
%    \begin{macro}{\HoLogo@XeLaTeX}
%    \begin{macrocode}
\def\HoLogo@XeLaTeX#1{%
  \hologo{Xe}%
  \kern-.13em%
  \hologo{LaTeX}%
}
%    \end{macrocode}
%    \end{macro}
%    \begin{macro}{\HoLogoHtml@XeLaTeX}
%    \begin{macrocode}
\def\HoLogoHtml@XeLaTeX#1{%
  \HoLogoCss@XeLaTeX
  \HOLOGO@Span{XeLaTeX}{%
    \hologo{Xe}%
    \hologo{LaTeX}%
  }%
}
%    \end{macrocode}
%    \end{macro}
%    \begin{macro}{\HoLogoCss@XeLaTeX}
%    \begin{macrocode}
\def\HoLogoCss@XeLaTeX{%
  \Css{%
    span.HoLogo-XeLaTeX span.HoLogo-Xe{%
      margin-right:-.13em;%
    }%
  }%
  \global\let\HoLogoCss@XeLaTeX\relax
}
%    \end{macrocode}
%    \end{macro}
%
% \subsubsection{\hologo{pdfTeX}, \hologo{pdfLaTeX}}
%
%    \begin{macro}{\HoLogo@pdfTeX}
%    \begin{macrocode}
\def\HoLogo@pdfTeX#1{%
  \HOLOGO@mbox{%
    #1{p}{P}df\hologo{TeX}%
  }%
}
%    \end{macrocode}
%    \end{macro}
%    \begin{macro}{\HoLogoCs@pdfTeX}
%    \begin{macrocode}
\def\HoLogoCs@pdfTeX#1{#1{p}{P}dfTeX}
%    \end{macrocode}
%    \end{macro}
%    \begin{macro}{\HoLogoBkm@pdfTeX}
%    \begin{macrocode}
\def\HoLogoBkm@pdfTeX#1{%
  #1{p}{P}df\hologo{TeX}%
}
%    \end{macrocode}
%    \end{macro}
%    \begin{macro}{\HoLogoHtml@pdfTeX}
%    \begin{macrocode}
\let\HoLogoHtml@pdfTeX\HoLogo@pdfTeX
%    \end{macrocode}
%    \end{macro}
%
%    \begin{macro}{\HoLogo@pdfLaTeX}
%    \begin{macrocode}
\def\HoLogo@pdfLaTeX#1{%
  \HOLOGO@mbox{%
    #1{p}{P}df\hologo{LaTeX}%
  }%
}
%    \end{macrocode}
%    \end{macro}
%    \begin{macro}{\HoLogoCs@pdfLaTeX}
%    \begin{macrocode}
\def\HoLogoCs@pdfLaTeX#1{#1{p}{P}dfLaTeX}
%    \end{macrocode}
%    \end{macro}
%    \begin{macro}{\HoLogoBkm@pdfLaTeX}
%    \begin{macrocode}
\def\HoLogoBkm@pdfLaTeX#1{%
  #1{p}{P}df\hologo{LaTeX}%
}
%    \end{macrocode}
%    \end{macro}
%    \begin{macro}{\HoLogoHtml@pdfLaTeX}
%    \begin{macrocode}
\let\HoLogoHtml@pdfLaTeX\HoLogo@pdfLaTeX
%    \end{macrocode}
%    \end{macro}
%
% \subsubsection{\hologo{VTeX}}
%
%    \begin{macro}{\HoLogo@VTeX}
%    \begin{macrocode}
\def\HoLogo@VTeX#1{%
  \HOLOGO@mbox{%
    V\hologo{TeX}%
  }%
}
%    \end{macrocode}
%    \end{macro}
%    \begin{macro}{\HoLogoHtml@VTeX}
%    \begin{macrocode}
\let\HoLogoHtml@VTeX\HoLogo@VTeX
%    \end{macrocode}
%    \end{macro}
%
% \subsubsection{\hologo{AmS}, \dots}
%
%    Source: class \xclass{amsdtx}
%
%    \begin{macro}{\HoLogo@AmS}
%    \begin{macrocode}
\def\HoLogo@AmS#1{%
  \HoLogoFont@font{AmS}{sy}{%
    A%
    \kern-.1667em%
    \lower.5ex\hbox{M}%
    \kern-.125em%
    S%
  }%
}
%    \end{macrocode}
%    \end{macro}
%    \begin{macro}{\HoLogoBkm@AmS}
%    \begin{macrocode}
\def\HoLogoBkm@AmS#1{AmS}
%    \end{macrocode}
%    \end{macro}
%    \begin{macro}{\HoLogoHtml@AmS}
%    \begin{macrocode}
\def\HoLogoHtml@AmS#1{%
  \HoLogoCss@AmS
%  \HoLogoFont@font{AmS}{sy}{%
    \HOLOGO@Span{AmS}{%
      A%
      \HOLOGO@Span{M}{M}%
      S%
    }%
%   }%
}
%    \end{macrocode}
%    \end{macro}
%    \begin{macro}{\HoLogoCss@AmS}
%    \begin{macrocode}
\def\HoLogoCss@AmS{%
  \Css{%
    span.HoLogo-AmS span.HoLogo-M{%
      position:relative;%
      top:.5ex;%
      margin-left:-.1667em;%
      margin-right:-.125em;%
      text-decoration:none;%
    }%
  }%
  \global\let\HoLogoCss@AmS\relax
}
%    \end{macrocode}
%    \end{macro}
%
%    \begin{macro}{\HoLogo@AmSTeX}
%    \begin{macrocode}
\def\HoLogo@AmSTeX#1{%
  \hologo{AmS}%
  \HOLOGO@hyphen
  \hologo{TeX}%
}
%    \end{macrocode}
%    \end{macro}
%    \begin{macro}{\HoLogoBkm@AmSTeX}
%    \begin{macrocode}
\def\HoLogoBkm@AmSTeX#1{AmS-TeX}%
%    \end{macrocode}
%    \end{macro}
%    \begin{macro}{\HoLogoHtml@AmSTeX}
%    \begin{macrocode}
\let\HoLogoHtml@AmSTeX\HoLogo@AmSTeX
%    \end{macrocode}
%    \end{macro}
%
%    \begin{macro}{\HoLogo@AmSLaTeX}
%    \begin{macrocode}
\def\HoLogo@AmSLaTeX#1{%
  \hologo{AmS}%
  \HOLOGO@hyphen
  \hologo{LaTeX}%
}
%    \end{macrocode}
%    \end{macro}
%    \begin{macro}{\HoLogoBkm@AmSLaTeX}
%    \begin{macrocode}
\def\HoLogoBkm@AmSLaTeX#1{AmS-LaTeX}%
%    \end{macrocode}
%    \end{macro}
%    \begin{macro}{\HoLogoHtml@AmSLaTeX}
%    \begin{macrocode}
\let\HoLogoHtml@AmSLaTeX\HoLogo@AmSLaTeX
%    \end{macrocode}
%    \end{macro}
%
% \subsubsection{\hologo{BibTeX}}
%
%    \begin{macro}{\HoLogo@BibTeX@sc}
%    A definition of \hologo{BibTeX} is provided in
%    the documentation source for the manual of \hologo{BibTeX}
%    \cite{btxdoc}.
%\begin{quote}
%\begin{verbatim}
%\def\BibTeX{%
%  {%
%    \rm
%    B%
%    \kern-.05em%
%    {%
%      \sc
%      i%
%      \kern-.025em %
%      b%
%    }%
%    \kern-.08em
%    T%
%    \kern-.1667em%
%    \lower.7ex\hbox{E}%
%    \kern-.125em%
%    X%
%  }%
%}
%\end{verbatim}
%\end{quote}
%    \begin{macrocode}
\def\HoLogo@BibTeX@sc#1{%
  B%
  \kern-.05em%
  \HoLogoFont@font{BibTeX}{sc}{%
    i%
    \kern-.025em%
    b%
  }%
  \HOLOGO@discretionary
  \kern-.08em%
  \hologo{TeX}%
}
%    \end{macrocode}
%    \end{macro}
%    \begin{macro}{\HoLogoHtml@BibTeX@sc}
%    \begin{macrocode}
\def\HoLogoHtml@BibTeX@sc#1{%
  \HoLogoCss@BibTeX@sc
  \HOLOGO@Span{BibTeX-sc}{%
    B%
    \HOLOGO@Span{i}{i}%
    \HOLOGO@Span{b}{b}%
    \hologo{TeX}%
  }%
}
%    \end{macrocode}
%    \end{macro}
%    \begin{macro}{\HoLogoCss@BibTeX@sc}
%    \begin{macrocode}
\def\HoLogoCss@BibTeX@sc{%
  \Css{%
    span.HoLogo-BibTeX-sc span.HoLogo-i{%
      margin-left:-.05em;%
      margin-right:-.025em;%
      font-variant:small-caps;%
    }%
  }%
  \Css{%
    span.HoLogo-BibTeX-sc span.HoLogo-b{%
      margin-right:-.08em;%
      font-variant:small-caps;%
    }%
  }%
  \global\let\HoLogoCss@BibTeX@sc\relax
}
%    \end{macrocode}
%    \end{macro}
%
%    \begin{macro}{\HoLogo@BibTeX@sf}
%    Variant \xoption{sf} avoids trouble with unavailable
%    small caps fonts (e.g., bold versions of Computer Modern or
%    Latin Modern). The definition is taken from
%    package \xpackage{dtklogos} \cite{dtklogos}.
%\begin{quote}
%\begin{verbatim}
%\DeclareRobustCommand{\BibTeX}{%
%  B%
%  \kern-.05em%
%  \hbox{%
%    $\m@th$% %% force math size calculations
%    \csname S@\f@size\endcsname
%    \fontsize\sf@size\z@
%    \math@fontsfalse
%    \selectfont
%    I%
%    \kern-.025em%
%    B
%  }%
%  \kern-.08em%
%  \-%
%  \TeX
%}
%\end{verbatim}
%\end{quote}
%    \begin{macrocode}
\def\HoLogo@BibTeX@sf#1{%
  B%
  \kern-.05em%
  \HoLogoFont@font{BibTeX}{bibsf}{%
    I%
    \kern-.025em%
    B%
  }%
  \HOLOGO@discretionary
  \kern-.08em%
  \hologo{TeX}%
}
%    \end{macrocode}
%    \end{macro}
%    \begin{macro}{\HoLogoHtml@BibTeX@sf}
%    \begin{macrocode}
\def\HoLogoHtml@BibTeX@sf#1{%
  \HoLogoCss@BibTeX@sf
  \HOLOGO@Span{BibTeX-sf}{%
    B%
    \HoLogoFont@font{BibTeX}{bibsf}{%
      \HOLOGO@Span{i}{I}%
      B%
    }%
    \hologo{TeX}%
  }%
}
%    \end{macrocode}
%    \end{macro}
%    \begin{macro}{\HoLogoCss@BibTeX@sf}
%    \begin{macrocode}
\def\HoLogoCss@BibTeX@sf{%
  \Css{%
    span.HoLogo-BibTeX-sf span.HoLogo-i{%
      margin-left:-.05em;%
      margin-right:-.025em;%
    }%
  }%
  \Css{%
    span.HoLogo-BibTeX-sf span.HoLogo-TeX{%
      margin-left:-.08em;%
    }%
  }%
  \global\let\HoLogoCss@BibTeX@sf\relax
}
%    \end{macrocode}
%    \end{macro}
%
%    \begin{macro}{\HoLogo@BibTeX}
%    \begin{macrocode}
\def\HoLogo@BibTeX{\HoLogo@BibTeX@sf}
%    \end{macrocode}
%    \end{macro}
%    \begin{macro}{\HoLogoHtml@BibTeX}
%    \begin{macrocode}
\def\HoLogoHtml@BibTeX{\HoLogoHtml@BibTeX@sf}
%    \end{macrocode}
%    \end{macro}
%
% \subsubsection{\hologo{BibTeX8}}
%
%    \begin{macro}{\HoLogo@BibTeX8}
%    \begin{macrocode}
\expandafter\def\csname HoLogo@BibTeX8\endcsname#1{%
  \hologo{BibTeX}%
  8%
}
%    \end{macrocode}
%    \end{macro}
%
%    \begin{macro}{\HoLogoBkm@BibTeX8}
%    \begin{macrocode}
\expandafter\def\csname HoLogoBkm@BibTeX8\endcsname#1{%
  \hologo{BibTeX}%
  8%
}
%    \end{macrocode}
%    \end{macro}
%    \begin{macro}{\HoLogoHtml@BibTeX8}
%    \begin{macrocode}
\expandafter
\let\csname HoLogoHtml@BibTeX8\expandafter\endcsname
\csname HoLogo@BibTeX8\endcsname
%    \end{macrocode}
%    \end{macro}
%
% \subsubsection{\hologo{ConTeXt}}
%
%    \begin{macro}{\HoLogo@ConTeXt@simple}
%    \begin{macrocode}
\def\HoLogo@ConTeXt@simple#1{%
  \HOLOGO@mbox{Con}%
  \HOLOGO@discretionary
  \HOLOGO@mbox{\hologo{TeX}t}%
}
%    \end{macrocode}
%    \end{macro}
%    \begin{macro}{\HoLogoHtml@ConTeXt@simple}
%    \begin{macrocode}
\let\HoLogoHtml@ConTeXt@simple\HoLogo@ConTeXt@simple
%    \end{macrocode}
%    \end{macro}
%
%    \begin{macro}{\HoLogo@ConTeXt@narrow}
%    This definition of logo \hologo{ConTeXt} with variant \xoption{narrow}
%    comes from TUGboat's class \xclass{ltugboat} (version 2010/11/15 v2.8).
%    \begin{macrocode}
\def\HoLogo@ConTeXt@narrow#1{%
  \HOLOGO@mbox{C\kern-.0333emon}%
  \HOLOGO@discretionary
  \kern-.0667em%
  \HOLOGO@mbox{\hologo{TeX}\kern-.0333emt}%
}
%    \end{macrocode}
%    \end{macro}
%    \begin{macro}{\HoLogoHtml@ConTeXt@narrow}
%    \begin{macrocode}
\def\HoLogoHtml@ConTeXt@narrow#1{%
  \HoLogoCss@ConTeXt@narrow
  \HOLOGO@Span{ConTeXt-narrow}{%
    \HOLOGO@Span{C}{C}%
    on%
    \hologo{TeX}%
    t%
  }%
}
%    \end{macrocode}
%    \end{macro}
%    \begin{macro}{\HoLogoCss@ConTeXt@narrow}
%    \begin{macrocode}
\def\HoLogoCss@ConTeXt@narrow{%
  \Css{%
    span.HoLogo-ConTeXt-narrow span.HoLogo-C{%
      margin-left:-.0333em;%
    }%
  }%
  \Css{%
    span.HoLogo-ConTeXt-narrow span.HoLogo-TeX{%
      margin-left:-.0667em;%
      margin-right:-.0333em;%
    }%
  }%
  \global\let\HoLogoCss@ConTeXt@narrow\relax
}
%    \end{macrocode}
%    \end{macro}
%
%    \begin{macro}{\HoLogo@ConTeXt}
%    \begin{macrocode}
\def\HoLogo@ConTeXt{\HoLogo@ConTeXt@narrow}
%    \end{macrocode}
%    \end{macro}
%    \begin{macro}{\HoLogoHtml@ConTeXt}
%    \begin{macrocode}
\def\HoLogoHtml@ConTeXt{\HoLogoHtml@ConTeXt@narrow}
%    \end{macrocode}
%    \end{macro}
%
% \subsubsection{\hologo{emTeX}}
%
%    \begin{macro}{\HoLogo@emTeX}
%    \begin{macrocode}
\def\HoLogo@emTeX#1{%
  \HOLOGO@mbox{#1{e}{E}m}%
  \HOLOGO@discretionary
  \hologo{TeX}%
}
%    \end{macrocode}
%    \end{macro}
%    \begin{macro}{\HoLogoCs@emTeX}
%    \begin{macrocode}
\def\HoLogoCs@emTeX#1{#1{e}{E}mTeX}%
%    \end{macrocode}
%    \end{macro}
%    \begin{macro}{\HoLogoBkm@emTeX}
%    \begin{macrocode}
\def\HoLogoBkm@emTeX#1{%
  #1{e}{E}m\hologo{TeX}%
}
%    \end{macrocode}
%    \end{macro}
%    \begin{macro}{\HoLogoHtml@emTeX}
%    \begin{macrocode}
\let\HoLogoHtml@emTeX\HoLogo@emTeX
%    \end{macrocode}
%    \end{macro}
%
% \subsubsection{\hologo{ExTeX}}
%
%    \begin{macro}{\HoLogo@ExTeX}
%    The definition is taken from the FAQ of the
%    project \hologo{ExTeX}
%    \cite{ExTeX-FAQ}.
%\begin{quote}
%\begin{verbatim}
%\def\ExTeX{%
%  \textrm{% Logo always with serifs
%    \ensuremath{%
%      \textstyle
%      \varepsilon_{%
%        \kern-0.15em%
%        \mathcal{X}%
%      }%
%    }%
%    \kern-.15em%
%    \TeX
%  }%
%}
%\end{verbatim}
%\end{quote}
%    \begin{macrocode}
\def\HoLogo@ExTeX#1{%
  \HoLogoFont@font{ExTeX}{rm}{%
    \ltx@mbox{%
      \HOLOGO@MathSetup
      $%
        \textstyle
        \varepsilon_{%
          \kern-0.15em%
          \HoLogoFont@font{ExTeX}{sy}{X}%
        }%
      $%
    }%
    \HOLOGO@discretionary
    \kern-.15em%
    \hologo{TeX}%
  }%
}
%    \end{macrocode}
%    \end{macro}
%    \begin{macro}{\HoLogoHtml@ExTeX}
%    \begin{macrocode}
\def\HoLogoHtml@ExTeX#1{%
  \HoLogoCss@ExTeX
  \HoLogoFont@font{ExTeX}{rm}{%
    \HOLOGO@Span{ExTeX}{%
      \ltx@mbox{%
        \HOLOGO@MathSetup
        $\textstyle\varepsilon$%
        \HOLOGO@Span{X}{$\textstyle\chi$}%
        \hologo{TeX}%
      }%
    }%
  }%
}
%    \end{macrocode}
%    \end{macro}
%    \begin{macro}{\HoLogoBkm@ExTeX}
%    \begin{macrocode}
\def\HoLogoBkm@ExTeX#1{%
  \HOLOGO@PdfdocUnicode{#1{e}{E}x}{\textepsilon\textchi}%
  \hologo{TeX}%
}
%    \end{macrocode}
%    \end{macro}
%    \begin{macro}{\HoLogoCss@ExTeX}
%    \begin{macrocode}
\def\HoLogoCss@ExTeX{%
  \Css{%
    span.HoLogo-ExTeX{%
      font-family:serif;%
    }%
  }%
  \Css{%
    span.HoLogo-ExTeX span.HoLogo-TeX{%
      margin-left:-.15em;%
    }%
  }%
  \global\let\HoLogoCss@ExTeX\relax
}
%    \end{macrocode}
%    \end{macro}
%
% \subsubsection{\hologo{MiKTeX}}
%
%    \begin{macro}{\HoLogo@MiKTeX}
%    \begin{macrocode}
\def\HoLogo@MiKTeX#1{%
  \HOLOGO@mbox{MiK}%
  \HOLOGO@discretionary
  \hologo{TeX}%
}
%    \end{macrocode}
%    \end{macro}
%    \begin{macro}{\HoLogoHtml@MiKTeX}
%    \begin{macrocode}
\let\HoLogoHtml@MiKTeX\HoLogo@MiKTeX
%    \end{macrocode}
%    \end{macro}
%
% \subsubsection{\hologo{OzTeX} and friends}
%
%    Source: \hologo{OzTeX} FAQ \cite{OzTeX}:
%    \begin{quote}
%      |\def\OzTeX{O\kern-.03em z\kern-.15em\TeX}|\\
%      (There is no kerning in OzMF, OzMP and OzTtH.)
%    \end{quote}
%
%    \begin{macro}{\HoLogo@OzTeX}
%    \begin{macrocode}
\def\HoLogo@OzTeX#1{%
  O%
  \kern-.03em %
  z%
  \kern-.15em %
  \hologo{TeX}%
}
%    \end{macrocode}
%    \end{macro}
%    \begin{macro}{\HoLogoHtml@OzTeX}
%    \begin{macrocode}
\def\HoLogoHtml@OzTeX#1{%
  \HoLogoCss@OzTeX
  \HOLOGO@Span{OzTeX}{%
    O%
    \HOLOGO@Span{z}{z}%
    \hologo{TeX}%
  }%
}
%    \end{macrocode}
%    \end{macro}
%    \begin{macro}{\HoLogoCss@OzTeX}
%    \begin{macrocode}
\def\HoLogoCss@OzTeX{%
  \Css{%
    span.HoLogo-OzTeX span.HoLogo-z{%
      margin-left:-.03em;%
      margin-right:-.15em;%
    }%
  }%
  \global\let\HoLogoCss@OzTeX\relax
}
%    \end{macrocode}
%    \end{macro}
%
%    \begin{macro}{\HoLogo@OzMF}
%    \begin{macrocode}
\def\HoLogo@OzMF#1{%
  \HOLOGO@mbox{OzMF}%
}
%    \end{macrocode}
%    \end{macro}
%    \begin{macro}{\HoLogo@OzMP}
%    \begin{macrocode}
\def\HoLogo@OzMP#1{%
  \HOLOGO@mbox{OzMP}%
}
%    \end{macrocode}
%    \end{macro}
%    \begin{macro}{\HoLogo@OzTtH}
%    \begin{macrocode}
\def\HoLogo@OzTtH#1{%
  \HOLOGO@mbox{OzTtH}%
}
%    \end{macrocode}
%    \end{macro}
%
% \subsubsection{\hologo{PCTeX}}
%
%    \begin{macro}{\HoLogo@PCTeX}
%    \begin{macrocode}
\def\HoLogo@PCTeX#1{%
  \HOLOGO@mbox{PC}%
  \hologo{TeX}%
}
%    \end{macrocode}
%    \end{macro}
%    \begin{macro}{\HoLogoHtml@PCTeX}
%    \begin{macrocode}
\let\HoLogoHtml@PCTeX\HoLogo@PCTeX
%    \end{macrocode}
%    \end{macro}
%
% \subsubsection{\hologo{PiCTeX}}
%
%    The original definitions from \xfile{pictex.tex} \cite{PiCTeX}:
%\begin{quote}
%\begin{verbatim}
%\def\PiC{%
%  P%
%  \kern-.12em%
%  \lower.5ex\hbox{I}%
%  \kern-.075em%
%  C%
%}
%\def\PiCTeX{%
%  \PiC
%  \kern-.11em%
%  \TeX
%}
%\end{verbatim}
%\end{quote}
%
%    \begin{macro}{\HoLogo@PiC}
%    \begin{macrocode}
\def\HoLogo@PiC#1{%
  P%
  \kern-.12em%
  \lower.5ex\hbox{I}%
  \kern-.075em%
  C%
  \HOLOGO@SpaceFactor
}
%    \end{macrocode}
%    \end{macro}
%    \begin{macro}{\HoLogoHtml@PiC}
%    \begin{macrocode}
\def\HoLogoHtml@PiC#1{%
  \HoLogoCss@PiC
  \HOLOGO@Span{PiC}{%
    P%
    \HOLOGO@Span{i}{I}%
    C%
  }%
}
%    \end{macrocode}
%    \end{macro}
%    \begin{macro}{\HoLogoCss@PiC}
%    \begin{macrocode}
\def\HoLogoCss@PiC{%
  \Css{%
    span.HoLogo-PiC span.HoLogo-i{%
      position:relative;%
      top:.5ex;%
      margin-left:-.12em;%
      margin-right:-.075em;%
      text-decoration:none;%
    }%
  }%
  \global\let\HoLogoCss@PiC\relax
}
%    \end{macrocode}
%    \end{macro}
%
%    \begin{macro}{\HoLogo@PiCTeX}
%    \begin{macrocode}
\def\HoLogo@PiCTeX#1{%
  \hologo{PiC}%
  \HOLOGO@discretionary
  \kern-.11em%
  \hologo{TeX}%
}
%    \end{macrocode}
%    \end{macro}
%    \begin{macro}{\HoLogoHtml@PiCTeX}
%    \begin{macrocode}
\def\HoLogoHtml@PiCTeX#1{%
  \HoLogoCss@PiCTeX
  \HOLOGO@Span{PiCTeX}{%
    \hologo{PiC}%
    \hologo{TeX}%
  }%
}
%    \end{macrocode}
%    \end{macro}
%    \begin{macro}{\HoLogoCss@PiCTeX}
%    \begin{macrocode}
\def\HoLogoCss@PiCTeX{%
  \Css{%
    span.HoLogo-PiCTeX span.HoLogo-PiC{%
      margin-right:-.11em;%
    }%
  }%
  \global\let\HoLogoCss@PiCTeX\relax
}
%    \end{macrocode}
%    \end{macro}
%
% \subsubsection{\hologo{teTeX}}
%
%    \begin{macro}{\HoLogo@teTeX}
%    \begin{macrocode}
\def\HoLogo@teTeX#1{%
  \HOLOGO@mbox{#1{t}{T}e}%
  \HOLOGO@discretionary
  \hologo{TeX}%
}
%    \end{macrocode}
%    \end{macro}
%    \begin{macro}{\HoLogoCs@teTeX}
%    \begin{macrocode}
\def\HoLogoCs@teTeX#1{#1{t}{T}dfTeX}
%    \end{macrocode}
%    \end{macro}
%    \begin{macro}{\HoLogoBkm@teTeX}
%    \begin{macrocode}
\def\HoLogoBkm@teTeX#1{%
  #1{t}{T}e\hologo{TeX}%
}
%    \end{macrocode}
%    \end{macro}
%    \begin{macro}{\HoLogoHtml@teTeX}
%    \begin{macrocode}
\let\HoLogoHtml@teTeX\HoLogo@teTeX
%    \end{macrocode}
%    \end{macro}
%
% \subsubsection{\hologo{TeX4ht}}
%
%    \begin{macro}{\HoLogo@TeX4ht}
%    \begin{macrocode}
\expandafter\def\csname HoLogo@TeX4ht\endcsname#1{%
  \HOLOGO@mbox{\hologo{TeX}4ht}%
}
%    \end{macrocode}
%    \end{macro}
%    \begin{macro}{\HoLogoHtml@TeX4ht}
%    \begin{macrocode}
\expandafter
\let\csname HoLogoHtml@TeX4ht\expandafter\endcsname
\csname HoLogo@TeX4ht\endcsname
%    \end{macrocode}
%    \end{macro}
%
%
% \subsubsection{\hologo{SageTeX}}
%
%    \begin{macro}{\HoLogo@SageTeX}
%    \begin{macrocode}
\def\HoLogo@SageTeX#1{%
  \HOLOGO@mbox{Sage}%
  \HOLOGO@discretionary
  \HOLOGO@NegativeKerning{eT,oT,To}%
  \hologo{TeX}%
}
%    \end{macrocode}
%    \end{macro}
%    \begin{macro}{\HoLogoHtml@SageTeX}
%    \begin{macrocode}
\let\HoLogoHtml@SageTeX\HoLogo@SageTeX
%    \end{macrocode}
%    \end{macro}
%
% \subsection{\hologo{METAFONT} and friends}
%
%    \begin{macro}{\HoLogo@METAFONT}
%    \begin{macrocode}
\def\HoLogo@METAFONT#1{%
  \HoLogoFont@font{METAFONT}{logo}{%
    \HOLOGO@mbox{META}%
    \HOLOGO@discretionary
    \HOLOGO@mbox{FONT}%
  }%
}
%    \end{macrocode}
%    \end{macro}
%
%    \begin{macro}{\HoLogo@METAPOST}
%    \begin{macrocode}
\def\HoLogo@METAPOST#1{%
  \HoLogoFont@font{METAPOST}{logo}{%
    \HOLOGO@mbox{META}%
    \HOLOGO@discretionary
    \HOLOGO@mbox{POST}%
  }%
}
%    \end{macrocode}
%    \end{macro}
%
%    \begin{macro}{\HoLogo@MetaFun}
%    \begin{macrocode}
\def\HoLogo@MetaFun#1{%
  \HOLOGO@mbox{Meta}%
  \HOLOGO@discretionary
  \HOLOGO@mbox{Fun}%
}
%    \end{macrocode}
%    \end{macro}
%
%    \begin{macro}{\HoLogo@MetaPost}
%    \begin{macrocode}
\def\HoLogo@MetaPost#1{%
  \HOLOGO@mbox{Meta}%
  \HOLOGO@discretionary
  \HOLOGO@mbox{Post}%
}
%    \end{macrocode}
%    \end{macro}
%
% \subsection{Others}
%
% \subsubsection{\hologo{biber}}
%
%    \begin{macro}{\HoLogo@biber}
%    \begin{macrocode}
\def\HoLogo@biber#1{%
  \HOLOGO@mbox{#1{b}{B}i}%
  \HOLOGO@discretionary
  \HOLOGO@mbox{ber}%
}
%    \end{macrocode}
%    \end{macro}
%    \begin{macro}{\HoLogoCs@biber}
%    \begin{macrocode}
\def\HoLogoCs@biber#1{#1{b}{B}iber}
%    \end{macrocode}
%    \end{macro}
%    \begin{macro}{\HoLogoBkm@biber}
%    \begin{macrocode}
\def\HoLogoBkm@biber#1{%
  #1{b}{B}iber%
}
%    \end{macrocode}
%    \end{macro}
%    \begin{macro}{\HoLogoHtml@biber}
%    \begin{macrocode}
\let\HoLogoHtml@biber\HoLogo@biber
%    \end{macrocode}
%    \end{macro}
%
% \subsubsection{\hologo{KOMAScript}}
%
%    \begin{macro}{\HoLogo@KOMAScript}
%    The definition for \hologo{KOMAScript} is taken
%    from \hologo{KOMAScript} (\xfile{scrlogo.dtx}, reformatted) \cite{scrlogo}:
%\begin{quote}
%\begin{verbatim}
%\@ifundefined{KOMAScript}{%
%  \DeclareRobustCommand{\KOMAScript}{%
%    \textsf{%
%      K\kern.05em O\kern.05emM\kern.05em A%
%      \kern.1em-\kern.1em %
%      Script%
%    }%
%  }%
%}{}
%\end{verbatim}
%\end{quote}
%    \begin{macrocode}
\def\HoLogo@KOMAScript#1{%
  \HoLogoFont@font{KOMAScript}{sf}{%
    \HOLOGO@mbox{%
      K\kern.05em%
      O\kern.05em%
      M\kern.05em%
      A%
    }%
    \kern.1em%
    \HOLOGO@hyphen
    \kern.1em%
    \HOLOGO@mbox{Script}%
  }%
}
%    \end{macrocode}
%    \end{macro}
%    \begin{macro}{\HoLogoBkm@KOMAScript}
%    \begin{macrocode}
\def\HoLogoBkm@KOMAScript#1{%
  KOMA-Script%
}
%    \end{macrocode}
%    \end{macro}
%    \begin{macro}{\HoLogoHtml@KOMAScript}
%    \begin{macrocode}
\def\HoLogoHtml@KOMAScript#1{%
  \HoLogoCss@KOMAScript
  \HoLogoFont@font{KOMAScript}{sf}{%
    \HOLOGO@Span{KOMAScript}{%
      K%
      \HOLOGO@Span{O}{O}%
      M%
      \HOLOGO@Span{A}{A}%
      \HOLOGO@Span{hyphen}{-}%
      Script%
    }%
  }%
}
%    \end{macrocode}
%    \end{macro}
%    \begin{macro}{\HoLogoCss@KOMAScript}
%    \begin{macrocode}
\def\HoLogoCss@KOMAScript{%
  \Css{%
    span.HoLogo-KOMAScript{%
      font-family:sans-serif;%
    }%
  }%
  \Css{%
    span.HoLogo-KOMAScript span.HoLogo-O{%
      padding-left:.05em;%
      padding-right:.05em;%
    }%
  }%
  \Css{%
    span.HoLogo-KOMAScript span.HoLogo-A{%
      padding-left:.05em;%
    }%
  }%
  \Css{%
    span.HoLogo-KOMAScript span.HoLogo-hyphen{%
      padding-left:.1em;%
      padding-right:.1em;%
    }%
  }%
  \global\let\HoLogoCss@KOMAScript\relax
}
%    \end{macrocode}
%    \end{macro}
%
% \subsubsection{\hologo{LyX}}
%
%    \begin{macro}{\HoLogo@LyX}
%    The definition is taken from the documentation source files
%    of \hologo{LyX}, \xfile{Intro.lyx} \cite{LyX}:
%\begin{quote}
%\begin{verbatim}
%\def\LyX{%
%  \texorpdfstring{%
%    L\kern-.1667em\lower.25em\hbox{Y}\kern-.125emX\@%
%  }{%
%    LyX%
%  }%
%}
%\end{verbatim}
%\end{quote}
%    \begin{macrocode}
\def\HoLogo@LyX#1{%
  L%
  \kern-.1667em%
  \lower.25em\hbox{Y}%
  \kern-.125em%
  X%
  \HOLOGO@SpaceFactor
}
%    \end{macrocode}
%    \end{macro}
%    \begin{macro}{\HoLogoHtml@LyX}
%    \begin{macrocode}
\def\HoLogoHtml@LyX#1{%
  \HoLogoCss@LyX
  \HOLOGO@Span{LyX}{%
    L%
    \HOLOGO@Span{y}{Y}%
    X%
  }%
}
%    \end{macrocode}
%    \end{macro}
%    \begin{macro}{\HoLogoCss@LyX}
%    \begin{macrocode}
\def\HoLogoCss@LyX{%
  \Css{%
    span.HoLogo-LyX span.HoLogo-y{%
      position:relative;%
      top:.25em;%
      margin-left:-.1667em;%
      margin-right:-.125em;%
      text-decoration:none;%
    }%
  }%
  \global\let\HoLogoCss@LyX\relax
}
%    \end{macrocode}
%    \end{macro}
%
% \subsubsection{\hologo{NTS}}
%
%    \begin{macro}{\HoLogo@NTS}
%    Definition for \hologo{NTS} can be found in
%    package \xpackage{etex\textunderscore man} for the \hologo{eTeX} manual \cite{etexman}
%    and in package \xpackage{dtklogos} \cite{dtklogos}:
%\begin{quote}
%\begin{verbatim}
%\def\NTS{%
%  \leavevmode
%  \hbox{%
%    $%
%      \cal N%
%      \kern-0.35em%
%      \lower0.5ex\hbox{$\cal T$}%
%      \kern-0.2em%
%      S%
%    $%
%  }%
%}
%\end{verbatim}
%\end{quote}
%    \begin{macrocode}
\def\HoLogo@NTS#1{%
  \HoLogoFont@font{NTS}{sy}{%
    N\/%
    \kern-.35em%
    \lower.5ex\hbox{T\/}%
    \kern-.2em%
    S\/%
  }%
  \HOLOGO@SpaceFactor
}
%    \end{macrocode}
%    \end{macro}
%
% \subsubsection{\Hologo{TTH} (\hologo{TeX} to HTML translator)}
%
%    Source: \url{http://hutchinson.belmont.ma.us/tth/}
%    In the HTML source the second `T' is printed as subscript.
%\begin{quote}
%\begin{verbatim}
%T<sub>T</sub>H
%\end{verbatim}
%\end{quote}
%    \begin{macro}{\HoLogo@TTH}
%    \begin{macrocode}
\def\HoLogo@TTH#1{%
  \ltx@mbox{%
    T\HOLOGO@SubScript{T}H%
  }%
  \HOLOGO@SpaceFactor
}
%    \end{macrocode}
%    \end{macro}
%
%    \begin{macro}{\HoLogoHtml@TTH}
%    \begin{macrocode}
\def\HoLogoHtml@TTH#1{%
  T\HCode{<sub>}T\HCode{</sub>}H%
}
%    \end{macrocode}
%    \end{macro}
%
% \subsubsection{\Hologo{HanTheThanh}}
%
%    Partial source: Package \xpackage{dtklogos}.
%    The double accent is U+1EBF (latin small letter e with circumflex
%    and acute).
%    \begin{macro}{\HoLogo@HanTheThanh}
%    \begin{macrocode}
\def\HoLogo@HanTheThanh#1{%
  \ltx@mbox{H\`an}%
  \HOLOGO@space
  \ltx@mbox{%
    Th%
    \HOLOGO@IfCharExists{"1EBF}{%
      \char"1EBF\relax
    }{%
      \^e\hbox to 0pt{\hss\raise .5ex\hbox{\'{}}}%
    }%
  }%
  \HOLOGO@space
  \ltx@mbox{Th\`anh}%
}
%    \end{macrocode}
%    \end{macro}
%    \begin{macro}{\HoLogoBkm@HanTheThanh}
%    \begin{macrocode}
\def\HoLogoBkm@HanTheThanh#1{%
  H\`an %
  Th\HOLOGO@PdfdocUnicode{\^e}{\9036\277} %
  Th\`anh%
}
%    \end{macrocode}
%    \end{macro}
%    \begin{macro}{\HoLogoHtml@HanTheThanh}
%    \begin{macrocode}
\def\HoLogoHtml@HanTheThanh#1{%
  H\`an %
  Th\HCode{&\ltx@hashchar x1ebf;} %
  Th\`anh%
}
%    \end{macrocode}
%    \end{macro}
%
% \subsection{Driver detection}
%
%    \begin{macrocode}
\HOLOGO@IfExists\InputIfFileExists{%
  \InputIfFileExists{hologo.cfg}{}{}%
}{%
  \ltx@IfUndefined{pdf@filesize}{%
    \def\HOLOGO@InputIfExists{%
      \openin\HOLOGO@temp=hologo.cfg\relax
      \ifeof\HOLOGO@temp
        \closein\HOLOGO@temp
      \else
        \closein\HOLOGO@temp
        \begingroup
          \def\x{LaTeX2e}%
        \expandafter\endgroup
        \ifx\fmtname\x
          % \iffalse meta-comment
%
% File: hologo.dtx
% Version: 2016/05/12 v1.11
% Info: A logo collection with bookmark support
%
% Copyright (C) 2010-2012 by
%    Heiko Oberdiek <heiko.oberdiek at googlemail.com>
%
% This work may be distributed and/or modified under the
% conditions of the LaTeX Project Public License, either
% version 1.3c of this license or (at your option) any later
% version. This version of this license is in
%    http://www.latex-project.org/lppl/lppl-1-3c.txt
% and the latest version of this license is in
%    http://www.latex-project.org/lppl.txt
% and version 1.3 or later is part of all distributions of
% LaTeX version 2005/12/01 or later.
%
% This work has the LPPL maintenance status "maintained".
%
% This Current Maintainer of this work is Heiko Oberdiek.
%
% The Base Interpreter refers to any `TeX-Format',
% because some files are installed in TDS:tex/generic//.
%
% This work consists of the main source file hologo.dtx
% and the derived files
%    hologo.sty, hologo.pdf, hologo.ins, hologo.drv, hologo-example.tex,
%    hologo-test1.tex, hologo-test-spacefactor.tex,
%    hologo-test-list.tex.
%
% Distribution:
%    CTAN:macros/latex/contrib/oberdiek/hologo.dtx
%    CTAN:macros/latex/contrib/oberdiek/hologo.pdf
%
% Unpacking:
%    (a) If hologo.ins is present:
%           tex hologo.ins
%    (b) Without hologo.ins:
%           tex hologo.dtx
%    (c) If you insist on using LaTeX
%           latex \let\install=y% \iffalse meta-comment
%
% File: hologo.dtx
% Version: 2016/05/12 v1.11
% Info: A logo collection with bookmark support
%
% Copyright (C) 2010-2012 by
%    Heiko Oberdiek <heiko.oberdiek at googlemail.com>
%
% This work may be distributed and/or modified under the
% conditions of the LaTeX Project Public License, either
% version 1.3c of this license or (at your option) any later
% version. This version of this license is in
%    http://www.latex-project.org/lppl/lppl-1-3c.txt
% and the latest version of this license is in
%    http://www.latex-project.org/lppl.txt
% and version 1.3 or later is part of all distributions of
% LaTeX version 2005/12/01 or later.
%
% This work has the LPPL maintenance status "maintained".
%
% This Current Maintainer of this work is Heiko Oberdiek.
%
% The Base Interpreter refers to any `TeX-Format',
% because some files are installed in TDS:tex/generic//.
%
% This work consists of the main source file hologo.dtx
% and the derived files
%    hologo.sty, hologo.pdf, hologo.ins, hologo.drv, hologo-example.tex,
%    hologo-test1.tex, hologo-test-spacefactor.tex,
%    hologo-test-list.tex.
%
% Distribution:
%    CTAN:macros/latex/contrib/oberdiek/hologo.dtx
%    CTAN:macros/latex/contrib/oberdiek/hologo.pdf
%
% Unpacking:
%    (a) If hologo.ins is present:
%           tex hologo.ins
%    (b) Without hologo.ins:
%           tex hologo.dtx
%    (c) If you insist on using LaTeX
%           latex \let\install=y\input{hologo.dtx}
%        (quote the arguments according to the demands of your shell)
%
% Documentation:
%    (a) If hologo.drv is present:
%           latex hologo.drv
%    (b) Without hologo.drv:
%           latex hologo.dtx; ...
%    The class ltxdoc loads the configuration file ltxdoc.cfg
%    if available. Here you can specify further options, e.g.
%    use A4 as paper format:
%       \PassOptionsToClass{a4paper}{article}
%
%    Programm calls to get the documentation (example):
%       pdflatex hologo.dtx
%       makeindex -s gind.ist hologo.idx
%       pdflatex hologo.dtx
%       makeindex -s gind.ist hologo.idx
%       pdflatex hologo.dtx
%
% Installation:
%    TDS:tex/generic/oberdiek/hologo.sty
%    TDS:doc/latex/oberdiek/hologo.pdf
%    TDS:doc/latex/oberdiek/example/hologo-example.tex
%    TDS:doc/latex/oberdiek/test/hologo-test1.tex
%    TDS:doc/latex/oberdiek/test/hologo-test-spacefactor.tex
%    TDS:doc/latex/oberdiek/test/hologo-test-list.tex
%    TDS:source/latex/oberdiek/hologo.dtx
%
%<*ignore>
\begingroup
  \catcode123=1 %
  \catcode125=2 %
  \def\x{LaTeX2e}%
\expandafter\endgroup
\ifcase 0\ifx\install y1\fi\expandafter
         \ifx\csname processbatchFile\endcsname\relax\else1\fi
         \ifx\fmtname\x\else 1\fi\relax
\else\csname fi\endcsname
%</ignore>
%<*install>
\input docstrip.tex
\Msg{************************************************************************}
\Msg{* Installation}
\Msg{* Package: hologo 2016/05/12 v1.11 A logo collection with bookmark support (HO)}
\Msg{************************************************************************}

\keepsilent
\askforoverwritefalse

\let\MetaPrefix\relax
\preamble

This is a generated file.

Project: hologo
Version: 2016/05/12 v1.11

Copyright (C) 2010-2012 by
   Heiko Oberdiek <heiko.oberdiek at googlemail.com>

This work may be distributed and/or modified under the
conditions of the LaTeX Project Public License, either
version 1.3c of this license or (at your option) any later
version. This version of this license is in
   http://www.latex-project.org/lppl/lppl-1-3c.txt
and the latest version of this license is in
   http://www.latex-project.org/lppl.txt
and version 1.3 or later is part of all distributions of
LaTeX version 2005/12/01 or later.

This work has the LPPL maintenance status "maintained".

This Current Maintainer of this work is Heiko Oberdiek.

The Base Interpreter refers to any `TeX-Format',
because some files are installed in TDS:tex/generic//.

This work consists of the main source file hologo.dtx
and the derived files
   hologo.sty, hologo.pdf, hologo.ins, hologo.drv, hologo-example.tex,
   hologo-test1.tex, hologo-test-spacefactor.tex,
   hologo-test-list.tex.

\endpreamble
\let\MetaPrefix\DoubleperCent

\generate{%
  \file{hologo.ins}{\from{hologo.dtx}{install}}%
  \file{hologo.drv}{\from{hologo.dtx}{driver}}%
  \usedir{tex/generic/oberdiek}%
  \file{hologo.sty}{\from{hologo.dtx}{package}}%
  \usedir{doc/latex/oberdiek/example}%
  \file{hologo-example.tex}{\from{hologo.dtx}{example}}%
  \usedir{doc/latex/oberdiek/test}%
  \file{hologo-test1.tex}{\from{hologo.dtx}{test1}}%
  \file{hologo-test-spacefactor.tex}{\from{hologo.dtx}{test-spacefactor}}%
  \file{hologo-test-list.tex}{\from{hologo.dtx}{test-list}}%
  \nopreamble
  \nopostamble
  \usedir{source/latex/oberdiek/catalogue}%
  \file{hologo.xml}{\from{hologo.dtx}{catalogue}}%
}

\catcode32=13\relax% active space
\let =\space%
\Msg{************************************************************************}
\Msg{*}
\Msg{* To finish the installation you have to move the following}
\Msg{* file into a directory searched by TeX:}
\Msg{*}
\Msg{*     hologo.sty}
\Msg{*}
\Msg{* To produce the documentation run the file `hologo.drv'}
\Msg{* through LaTeX.}
\Msg{*}
\Msg{* Happy TeXing!}
\Msg{*}
\Msg{************************************************************************}

\endbatchfile
%</install>
%<*ignore>
\fi
%</ignore>
%<*driver>
\NeedsTeXFormat{LaTeX2e}
\ProvidesFile{hologo.drv}%
  [2016/05/12 v1.11 A logo collection with bookmark support (HO)]%
\documentclass{ltxdoc}
\usepackage{holtxdoc}[2011/11/22]
\usepackage{hologo}[2016/05/12]
\usepackage{longtable}
\usepackage{array}
\usepackage{paralist}
%\usepackage[T1]{fontenc}
%\usepackage{lmodern}
\begin{document}
  \DocInput{hologo.dtx}%
\end{document}
%</driver>
% \fi
%
%
% \CharacterTable
%  {Upper-case    \A\B\C\D\E\F\G\H\I\J\K\L\M\N\O\P\Q\R\S\T\U\V\W\X\Y\Z
%   Lower-case    \a\b\c\d\e\f\g\h\i\j\k\l\m\n\o\p\q\r\s\t\u\v\w\x\y\z
%   Digits        \0\1\2\3\4\5\6\7\8\9
%   Exclamation   \!     Double quote  \"     Hash (number) \#
%   Dollar        \$     Percent       \%     Ampersand     \&
%   Acute accent  \'     Left paren    \(     Right paren   \)
%   Asterisk      \*     Plus          \+     Comma         \,
%   Minus         \-     Point         \.     Solidus       \/
%   Colon         \:     Semicolon     \;     Less than     \<
%   Equals        \=     Greater than  \>     Question mark \?
%   Commercial at \@     Left bracket  \[     Backslash     \\
%   Right bracket \]     Circumflex    \^     Underscore    \_
%   Grave accent  \`     Left brace    \{     Vertical bar  \|
%   Right brace   \}     Tilde         \~}
%
% \GetFileInfo{hologo.drv}
%
% \title{The \xpackage{hologo} package}
% \date{2016/05/12 v1.11}
% \author{Heiko Oberdiek\\\xemail{heiko.oberdiek at googlemail.com}}
%
% \maketitle
%
% \begin{abstract}
% This package starts a collection of logos with support for bookmarks
% strings.
% \end{abstract}
%
% \tableofcontents
%
% \section{Documentation}
%
% \subsection{Logo macros}
%
% \begin{declcs}{hologo} \M{name}
% \end{declcs}
% Macro \cs{hologo} sets the logo with name \meta{name}.
% The following table shows the supported names.
%
% \begingroup
%   \def\hologoEntry#1#2#3{^^A
%     #1&#2&\hologoLogoSetup{#1}{variant=#2}\hologo{#1}&#3\tabularnewline
%   }
%   \begin{longtable}{>{\ttfamily}l>{\ttfamily}lll}
%     \rmfamily\bfseries{name} & \rmfamily\bfseries variant
%     & \bfseries logo & \bfseries since\\
%     \hline
%     \endhead
%     \hologoList
%   \end{longtable}
% \endgroup
%
% \begin{declcs}{Hologo} \M{name}
% \end{declcs}
% Macro \cs{Hologo} starts the logo \meta{name} with an uppercase
% letter. As an exception small greek letters are not converted
% to uppercase. Examples, see \hologo{eTeX} and \hologo{ExTeX}.
%
% \subsection{Setup macros}
%
% The package does not support package options, but the following
% setup macros can be used to set options.
%
% \begin{declcs}{hologoSetup} \M{key value list}
% \end{declcs}
% Macro \cs{hologoSetup} sets global options.
%
% \begin{declcs}{hologoLogoSetup} \M{logo} \M{key value list}
% \end{declcs}
% Some options can also be used to configure a logo.
% These settings take precedence over global option settings.
%
% \subsection{Options}\label{sec:options}
%
% There are boolean and string options:
% \begin{description}
% \item[Boolean option:]
% It takes |true| or |false|
% as value. If the value is omitted, then |true| is used.
% \item[String option:]
% A value must be given as string. (But the string might be empty.)
% \end{description}
% The following options can be used both in \cs{hologoSetup}
% and \cs{hologoLogoSetup}:
% \begin{description}
% \def\entry#1{\item[\xoption{#1}:]}
% \entry{break}
%   enables or disables line breaks inside the logo. This setting is
%   refined by options \xoption{hyphenbreak}, \xoption{spacebreak}
%   or \xoption{discretionarybreak}.
%   Default is |false|.
% \entry{hyphenbreak}
%   enables or disables the line break right after the hyphen character.
% \entry{spacebreak}
%   enables or disables line breaks at space characters.
% \entry{discretionarybreak}
%   enables or disables line breaks at hyphenation points
%   (inserted by \cs{-}).
% \end{description}
% Macro \cs{hologoLogoSetup} also knows:
% \begin{description}
% \item[\xoption{variant}:]
%   This is a string option. It specifies a variant of a logo that
%   must exist. An empty string selects the package default variant.
% \end{description}
% Example:
% \begin{quote}
%   |\hologoSetup{break=false}|\\
%   |\hologoLogoSetup{plainTeX}{variant=hyphen,hyphenbreak}|\\
%   Then ``plain-\TeX'' contains one break point after the hyphen.
% \end{quote}
%
% \subsection{Driver options}
%
% Sometimes graphical operations are needed to construct some
% glyphs (e.g.\ \hologo{XeTeX}). If package \xpackage{graphics}
% or package \xpackage{pgf} are found, then the macros are taken
% from there. Otherwise the packge defines its own operations
% and therefore needs the driver information. Many drivers are
% detected automatically (\hologo{pdfTeX}/\hologo{LuaTeX}
% in PDF mode, \hologo{XeTeX}, \hologo{VTeX}). These have precedence
% over a driver option. The driver can be given as package option
% or using \cs{hologoDriverSetup}.
% The following list contains the recognized driver options:
% \begin{itemize}
% \item \xoption{pdftex}, \xoption{luatex}
% \item \xoption{dvipdfm}, \xoption{dvipdfmx}
% \item \xoption{dvips}, \xoption{dvipsone}, \xoption{xdvi}
% \item \xoption{xetex}
% \item \xoption{vtex}
% \end{itemize}
% The left driver of a line is the driver name that is used internally.
% The following names are aliases for drivers that use the
% same method. Therefore the entry in the \xext{log} file for
% the used driver prints the internally used driver name.
% \begin{description}
% \item[\xoption{driverfallback}:]
%   This option expects a driver that is used,
%   if the driver could not be detected automatically.
% \end{description}
%
% \begin{declcs}{hologoDriverSetup} \M{driver option}
% \end{declcs}
% The driver can also be configured after package loading
% using \cs{hologoDriverSetup}, also the way for \hologo{plainTeX}
% to setup the driver.
%
% \subsection{Font setup}
%
% Some logos require a special font, but should also be usable by
% \hologo{plainTeX}. Therefore the package provides some ways
% to influence the font settings. The options below
% take font settings as values. Both font commands
% such as \cs{sffamily} and macros that take one argument
% like \cs{textsf} can be used.
%
% \begin{declcs}{hologoFontSetup} \M{key value list}
% \end{declcs}
% Macro \cs{hologoFontSetup} sets the fonts for all logos.
% Supported keys:
% \begin{description}
% \def\entry#1{\item[\xoption{#1}:]}
% \entry{general}
%   This font is used for all logos. The default is empty.
%   That means no special font is used.
% \entry{bibsf}
%   This font is used for
%   {\hologoLogoSetup{BibTeX}{variant=sf}\hologo{BibTeX}}
%   with variant \xoption{sf}.
% \entry{rm}
%   This font is a serif font. It is used for \hologo{ExTeX}.
% \entry{sc}
%   This font specifies a small caps font. It is used for
%   {\hologoLogoSetup{BibTeX}{variant=sc}\hologo{BibTeX}}
%   with variant \xoption{sc}.
% \entry{sf}
%   This font specifies a sans serif font. The default
%   is \cs{sffamily}, then \cs{sf} is tried. Otherwise
%   a warning is given. It is used by \hologo{KOMAScript}.
% \entry{sy}
%   This is the font for math symbols (e.g. cmsy).
%   It is used by \hologo{AmS}, \hologo{NTS}, \hologo{ExTeX}.
% \entry{logo}
%   \hologo{METAFONT} and \hologo{METAPOST} are using that font.
%   In \hologo{LaTeX} \cs{logofamily} is used and
%   the definitions of package \xpackage{mflogo} are used
%   if the package is not loaded.
%   Otherwise the \cs{tenlogo} is used and defined
%   if it does not already exists.
% \end{description}
%
% \begin{declcs}{hologoLogoFontSetup} \M{logo} \M{key value list}
% \end{declcs}
% Fonts can also be set for a logo or logo component separately,
% see the following list.
% The keys are the same as for \cs{hologoFontSetup}.
%
% \begin{longtable}{>{\ttfamily}l>{\sffamily}ll}
%   \meta{logo} & keys & result\\
%   \hline
%   \endhead
%   BibTeX & bibsf & {\hologoLogoSetup{BibTeX}{variant=sf}\hologo{BibTeX}}\\[.5ex]
%   BibTeX & sc & {\hologoLogoSetup{BibTeX}{variant=sc}\hologo{BibTeX}}\\[.5ex]
%   ExTeX & rm & \hologo{ExTeX}\\
%   SliTeX & rm & \hologo{SliTeX}\\[.5ex]
%   AmS & sy & \hologo{AmS}\\
%   ExTeX & sy & \hologo{ExTeX}\\
%   NTS & sy & \hologo{NTS}\\[.5ex]
%   KOMAScript & sf & \hologo{KOMAScript}\\[.5ex]
%   METAFONT & logo & \hologo{METAFONT}\\
%   METAPOST & logo & \hologo{METAPOST}\\[.5ex]
%   SliTeX & sc \hologo{SliTeX}
% \end{longtable}
%
% \subsubsection{Font order}
%
% For all logos the font \xoption{general} is applied first.
% Example:
%\begin{quote}
%|\hologoFontSetup{general=\color{red}}|
%\end{quote}
% will print red logos.
% Then if the font uses a special font \xoption{sf}, for example,
% the font is applied that is setup by \cs{hologoLogoFontSetup}.
% If this font is not setup, then the common font setup
% by \cs{hologoFontSetup} is used. Otherwise a warning is given,
% that there is no font configured.
%
% \subsection{Additional user macros}
%
% Usually a variant of a logo is configured by using
% \cs{hologoLogoSetup}, because it is bad style to mix
% different variants of the same logo in the same text.
% There the following macros are a convenience for testing.
%
% \begin{declcs}{hologoVariant} \M{name} \M{variant}\\
%   \cs{HologoVariant} \M{name} \M{variant}
% \end{declcs}
% Logo \meta{name} is set using \meta{variant} that specifies
% explicitely which variant of the macro is used. If the argument
% is empty, then the default form of the logo is used
% (configurable by \cs{hologoLogoSetup}).
%
% \cs{HologoVariant} is used if the logo is set in a context
% that needs an uppercase first letter (beginning of a sentence, \dots).
%
% \begin{declcs}{hologoList}\\
%   \cs{hologoEntry} \M{logo} \M{variant} \M{since}
% \end{declcs}
% Macro \cs{hologoList} contains all logos that are provided
% by the package including variants. The list consists of calls
% of \cs{hologoEntry} with three arguments starting with the
% logo name \meta{logo} and its variant \meta{variant}. An empty
% variant means the current default. Argument \meta{since} specifies
% with version of the package \xpackage{hologo} is needed to get
% the logo. If the logo is fixed, then the date gets updated.
% Therefore the date \meta{since} is not exactly the date of
% the first introduction, but rather the date of the latest fix.
%
% Before \cs{hologoList} can be used, macro \cs{hologoEntry} needs
% a definition. The example file in section \ref{sec:example}
% shows applications of \cs{hologoList}.
%
% \subsection{Supported contexts}
%
% Macros \cs{hologo} and friends support special contexts:
% \begin{itemize}
% \item \hologo{LaTeX}'s protection mechanism.
% \item Bookmarks of package \xpackage{hyperref}.
% \item Package \xpackage{tex4ht}.
% \item The macros can be used inside \cs{csname} constructs,
%   if \cs{ifincsname} is available (\hologo{pdfTeX}, \hologo{XeTeX},
%   \hologo{LuaTeX}).
% \end{itemize}
%
% \subsection{Example}
% \label{sec:example}
%
% The following example prints the logos in different fonts.
%    \begin{macrocode}
%<*example>
%<<verbatim
\NeedsTeXFormat{LaTeX2e}
\documentclass[a4paper]{article}
\usepackage[
  hmargin=20mm,
  vmargin=20mm,
]{geometry}
\pagestyle{empty}
\usepackage{hologo}[2016/05/12]
\usepackage{longtable}
\usepackage{array}
\setlength{\extrarowheight}{2pt}
\usepackage[T1]{fontenc}
\usepackage{lmodern}
\usepackage{pdflscape}
\usepackage[
  pdfencoding=auto,
]{hyperref}
\hypersetup{
  pdfauthor={Heiko Oberdiek},
  pdftitle={Example for package `hologo'},
  pdfsubject={Logos with fonts lmr, lmss, qtm, qpl, qhv},
}
\usepackage{bookmark}

% Print the logo list on the console

\begingroup
  \typeout{}%
  \typeout{*** Begin of logo list ***}%
  \newcommand*{\hologoEntry}[3]{%
    \typeout{#1 \ifx\\#2\\\else(#2) \fi[#3]}%
  }%
  \hologoList
  \typeout{*** End of logo list ***}%
  \typeout{}%
\endgroup

\begin{document}
\begin{landscape}

  \section{Example file for package `hologo'}

  % Table for font names

  \begin{longtable}{>{\bfseries}ll}
    \textbf{font} & \textbf{Font name}\\
    \hline
    lmr & Latin Modern Roman\\
    lmss & Latin Modern Sans\\
    qtm & \TeX\ Gyre Termes\\
    qhv & \TeX\ Gyre Heros\\
    qpl & \TeX\ Gyre Pagella\\
  \end{longtable}

  % Logo list with logos in different fonts

  \begingroup
    \newcommand*{\SetVariant}[2]{%
      \ifx\\#2\\%
      \else
        \hologoLogoSetup{#1}{variant=#2}%
      \fi
    }%
    \newcommand*{\hologoEntry}[3]{%
      \SetVariant{#1}{#2}%
      \raisebox{1em}[0pt][0pt]{\hypertarget{#1@#2}{}}%
      \bookmark[%
        dest={#1@#2},%
      ]{%
        #1\ifx\\#2\\\else\space(#2)\fi: \Hologo{#1}, \hologo{#1} %
        [Unicode]%
      }%
      \hypersetup{unicode=false}%
      \bookmark[%
        dest={#1@#2},%
      ]{%
        #1\ifx\\#2\\\else\space(#2)\fi: \Hologo{#1}, \hologo{#1} %
        [PDFDocEncoding]%
      }%
      \texttt{#1}%
      &%
      \texttt{#2}%
      &%
      \Hologo{#1}%
      &%
      \SetVariant{#1}{#2}%
      \hologo{#1}%
      &%
      \SetVariant{#1}{#2}%
      \fontfamily{qtm}\selectfont
      \hologo{#1}%
      &%
      \SetVariant{#1}{#2}%
      \fontfamily{qpl}\selectfont
      \hologo{#1}%
      &%
      \SetVariant{#1}{#2}%
      \textsf{\hologo{#1}}%
      &%
      \SetVariant{#1}{#2}%
      \fontfamily{qhv}\selectfont
      \hologo{#1}%
      \tabularnewline
    }%
    \begin{longtable}{llllllll}%
      \textbf{\textit{logo}} & \textbf{\textit{variant}} &
      \texttt{\string\Hologo} &
      \textbf{lmr} & \textbf{qtm} & \textbf{qpl} &
      \textbf{lmss} & \textbf{qhv}
      \tabularnewline
      \hline
      \endhead
      \hologoList
    \end{longtable}%
  \endgroup

\end{landscape}
\end{document}
%verbatim
%</example>
%    \end{macrocode}
%
% \StopEventually{
% }
%
% \section{Implementation}
%    \begin{macrocode}
%<*package>
%    \end{macrocode}
%    Reload check, especially if the package is not used with \LaTeX.
%    \begin{macrocode}
\begingroup\catcode61\catcode48\catcode32=10\relax%
  \catcode13=5 % ^^M
  \endlinechar=13 %
  \catcode35=6 % #
  \catcode39=12 % '
  \catcode44=12 % ,
  \catcode45=12 % -
  \catcode46=12 % .
  \catcode58=12 % :
  \catcode64=11 % @
  \catcode123=1 % {
  \catcode125=2 % }
  \expandafter\let\expandafter\x\csname ver@hologo.sty\endcsname
  \ifx\x\relax % plain-TeX, first loading
  \else
    \def\empty{}%
    \ifx\x\empty % LaTeX, first loading,
      % variable is initialized, but \ProvidesPackage not yet seen
    \else
      \expandafter\ifx\csname PackageInfo\endcsname\relax
        \def\x#1#2{%
          \immediate\write-1{Package #1 Info: #2.}%
        }%
      \else
        \def\x#1#2{\PackageInfo{#1}{#2, stopped}}%
      \fi
      \x{hologo}{The package is already loaded}%
      \aftergroup\endinput
    \fi
  \fi
\endgroup%
%    \end{macrocode}
%    Package identification:
%    \begin{macrocode}
\begingroup\catcode61\catcode48\catcode32=10\relax%
  \catcode13=5 % ^^M
  \endlinechar=13 %
  \catcode35=6 % #
  \catcode39=12 % '
  \catcode40=12 % (
  \catcode41=12 % )
  \catcode44=12 % ,
  \catcode45=12 % -
  \catcode46=12 % .
  \catcode47=12 % /
  \catcode58=12 % :
  \catcode64=11 % @
  \catcode91=12 % [
  \catcode93=12 % ]
  \catcode123=1 % {
  \catcode125=2 % }
  \expandafter\ifx\csname ProvidesPackage\endcsname\relax
    \def\x#1#2#3[#4]{\endgroup
      \immediate\write-1{Package: #3 #4}%
      \xdef#1{#4}%
    }%
  \else
    \def\x#1#2[#3]{\endgroup
      #2[{#3}]%
      \ifx#1\@undefined
        \xdef#1{#3}%
      \fi
      \ifx#1\relax
        \xdef#1{#3}%
      \fi
    }%
  \fi
\expandafter\x\csname ver@hologo.sty\endcsname
\ProvidesPackage{hologo}%
  [2016/05/12 v1.11 A logo collection with bookmark support (HO)]%
%    \end{macrocode}
%
%    \begin{macrocode}
\begingroup\catcode61\catcode48\catcode32=10\relax%
  \catcode13=5 % ^^M
  \endlinechar=13 %
  \catcode123=1 % {
  \catcode125=2 % }
  \catcode64=11 % @
  \def\x{\endgroup
    \expandafter\edef\csname HOLOGO@AtEnd\endcsname{%
      \endlinechar=\the\endlinechar\relax
      \catcode13=\the\catcode13\relax
      \catcode32=\the\catcode32\relax
      \catcode35=\the\catcode35\relax
      \catcode61=\the\catcode61\relax
      \catcode64=\the\catcode64\relax
      \catcode123=\the\catcode123\relax
      \catcode125=\the\catcode125\relax
    }%
  }%
\x\catcode61\catcode48\catcode32=10\relax%
\catcode13=5 % ^^M
\endlinechar=13 %
\catcode35=6 % #
\catcode64=11 % @
\catcode123=1 % {
\catcode125=2 % }
\def\TMP@EnsureCode#1#2{%
  \edef\HOLOGO@AtEnd{%
    \HOLOGO@AtEnd
    \catcode#1=\the\catcode#1\relax
  }%
  \catcode#1=#2\relax
}
\TMP@EnsureCode{10}{12}% ^^J
\TMP@EnsureCode{33}{12}% !
\TMP@EnsureCode{34}{12}% "
\TMP@EnsureCode{36}{3}% $
\TMP@EnsureCode{38}{4}% &
\TMP@EnsureCode{39}{12}% '
\TMP@EnsureCode{40}{12}% (
\TMP@EnsureCode{41}{12}% )
\TMP@EnsureCode{42}{12}% *
\TMP@EnsureCode{43}{12}% +
\TMP@EnsureCode{44}{12}% ,
\TMP@EnsureCode{45}{12}% -
\TMP@EnsureCode{46}{12}% .
\TMP@EnsureCode{47}{12}% /
\TMP@EnsureCode{58}{12}% :
\TMP@EnsureCode{59}{12}% ;
\TMP@EnsureCode{60}{12}% <
\TMP@EnsureCode{62}{12}% >
\TMP@EnsureCode{63}{12}% ?
\TMP@EnsureCode{91}{12}% [
\TMP@EnsureCode{93}{12}% ]
\TMP@EnsureCode{94}{7}% ^ (superscript)
\TMP@EnsureCode{95}{8}% _ (subscript)
\TMP@EnsureCode{96}{12}% `
\TMP@EnsureCode{124}{12}% |
\edef\HOLOGO@AtEnd{%
  \HOLOGO@AtEnd
  \escapechar\the\escapechar\relax
  \noexpand\endinput
}
\escapechar=92 %
%    \end{macrocode}
%
% \subsection{Logo list}
%
%    \begin{macro}{\hologoList}
%    \begin{macrocode}
\def\hologoList{%
  \hologoEntry{(La)TeX}{}{2011/10/01}%
  \hologoEntry{AmSLaTeX}{}{2010/04/16}%
  \hologoEntry{AmSTeX}{}{2010/04/16}%
  \hologoEntry{biber}{}{2011/10/01}%
  \hologoEntry{BibTeX}{}{2011/10/01}%
  \hologoEntry{BibTeX}{sf}{2011/10/01}%
  \hologoEntry{BibTeX}{sc}{2011/10/01}%
  \hologoEntry{BibTeX8}{}{2011/11/22}%
  \hologoEntry{ConTeXt}{}{2011/03/25}%
  \hologoEntry{ConTeXt}{narrow}{2011/03/25}%
  \hologoEntry{ConTeXt}{simple}{2011/03/25}%
  \hologoEntry{emTeX}{}{2010/04/26}%
  \hologoEntry{eTeX}{}{2010/04/08}%
  \hologoEntry{ExTeX}{}{2011/10/01}%
  \hologoEntry{HanTheThanh}{}{2011/11/29}%
  \hologoEntry{iniTeX}{}{2011/10/01}%
  \hologoEntry{KOMAScript}{}{2011/10/01}%
  \hologoEntry{La}{}{2010/05/08}%
  \hologoEntry{LaTeX}{}{2010/04/08}%
  \hologoEntry{LaTeX2e}{}{2010/04/08}%
  \hologoEntry{LaTeX3}{}{2010/04/24}%
  \hologoEntry{LaTeXe}{}{2010/04/08}%
  \hologoEntry{LaTeXML}{}{2011/11/22}%
  \hologoEntry{LaTeXTeX}{}{2011/10/01}%
  \hologoEntry{LuaLaTeX}{}{2010/04/08}%
  \hologoEntry{LuaTeX}{}{2010/04/08}%
  \hologoEntry{LyX}{}{2011/10/01}%
  \hologoEntry{METAFONT}{}{2011/10/01}%
  \hologoEntry{MetaFun}{}{2011/10/01}%
  \hologoEntry{METAPOST}{}{2011/10/01}%
  \hologoEntry{MetaPost}{}{2011/10/01}%
  \hologoEntry{MiKTeX}{}{2011/10/01}%
  \hologoEntry{NTS}{}{2011/10/01}%
  \hologoEntry{OzMF}{}{2011/10/01}%
  \hologoEntry{OzMP}{}{2011/10/01}%
  \hologoEntry{OzTeX}{}{2011/10/01}%
  \hologoEntry{OzTtH}{}{2011/10/01}%
  \hologoEntry{PCTeX}{}{2011/10/01}%
  \hologoEntry{pdfTeX}{}{2011/10/01}%
  \hologoEntry{pdfLaTeX}{}{2011/10/01}%
  \hologoEntry{PiC}{}{2011/10/01}%
  \hologoEntry{PiCTeX}{}{2011/10/01}%
  \hologoEntry{plainTeX}{}{2010/04/08}%
  \hologoEntry{plainTeX}{space}{2010/04/16}%
  \hologoEntry{plainTeX}{hyphen}{2010/04/16}%
  \hologoEntry{plainTeX}{runtogether}{2010/04/16}%
  \hologoEntry{SageTeX}{}{2011/11/22}%
  \hologoEntry{SLiTeX}{}{2011/10/01}%
  \hologoEntry{SLiTeX}{lift}{2011/10/01}%
  \hologoEntry{SLiTeX}{narrow}{2011/10/01}%
  \hologoEntry{SLiTeX}{simple}{2011/10/01}%
  \hologoEntry{SliTeX}{}{2011/10/01}%
  \hologoEntry{SliTeX}{narrow}{2011/10/01}%
  \hologoEntry{SliTeX}{simple}{2011/10/01}%
  \hologoEntry{SliTeX}{lift}{2011/10/01}%
  \hologoEntry{teTeX}{}{2011/10/01}%
  \hologoEntry{TeX}{}{2010/04/08}%
  \hologoEntry{TeX4ht}{}{2011/11/22}%
  \hologoEntry{TTH}{}{2011/11/22}%
  \hologoEntry{virTeX}{}{2011/10/01}%
  \hologoEntry{VTeX}{}{2010/04/24}%
  \hologoEntry{Xe}{}{2010/04/08}%
  \hologoEntry{XeLaTeX}{}{2010/04/08}%
  \hologoEntry{XeTeX}{}{2010/04/08}%
}
%    \end{macrocode}
%    \end{macro}
%
% \subsection{Load resources}
%
%    \begin{macrocode}
\begingroup\expandafter\expandafter\expandafter\endgroup
\expandafter\ifx\csname RequirePackage\endcsname\relax
  \def\TMP@RequirePackage#1[#2]{%
    \begingroup\expandafter\expandafter\expandafter\endgroup
    \expandafter\ifx\csname ver@#1.sty\endcsname\relax
      \input #1.sty\relax
    \fi
  }%
  \TMP@RequirePackage{ltxcmds}[2011/02/04]%
  \TMP@RequirePackage{infwarerr}[2010/04/08]%
  \TMP@RequirePackage{kvsetkeys}[2010/03/01]%
  \TMP@RequirePackage{kvdefinekeys}[2010/03/01]%
  \TMP@RequirePackage{pdftexcmds}[2010/04/01]%
  \TMP@RequirePackage{ifpdf}[2010/01/28]%
  \TMP@RequirePackage{ifluatex}[2010/03/01]%
  \ltx@IfUndefined{newif}{%
    \expandafter\let\csname newif\endcsname\ltx@newif
  }{}%
  \TMP@RequirePackage{ifxetex}[2009/01/23]%
  \TMP@RequirePackage{ifvtex}[2010/03/01]%
\else
  \RequirePackage{ltxcmds}[2011/02/04]%
  \RequirePackage{infwarerr}[2010/04/08]%
  \RequirePackage{kvsetkeys}[2010/03/01]%
  \RequirePackage{kvdefinekeys}[2010/03/01]%
  \RequirePackage{pdftexcmds}[2010/04/01]%
  \RequirePackage{ifpdf}[2010/01/28]%
  \RequirePackage{ifluatex}[2010/03/01]%
  \RequirePackage{ifxetex}[2009/01/23]%
  \RequirePackage{ifvtex}[2010/03/01]%
\fi
%    \end{macrocode}
%
%    \begin{macro}{\HOLOGO@IfDefined}
%    \begin{macrocode}
\def\HOLOGO@IfExists#1{%
  \ifx\@undefined#1%
    \expandafter\ltx@secondoftwo
  \else
    \ifx\relax#1%
      \expandafter\ltx@secondoftwo
    \else
      \expandafter\expandafter\expandafter\ltx@firstoftwo
    \fi
  \fi
}
%    \end{macrocode}
%    \end{macro}
%
% \subsection{Setup macros}
%
%    \begin{macro}{\hologoSetup}
%    \begin{macrocode}
\def\hologoSetup{%
  \let\HOLOGO@name\relax
  \HOLOGO@Setup
}
%    \end{macrocode}
%    \end{macro}
%
%    \begin{macro}{\hologoLogoSetup}
%    \begin{macrocode}
\def\hologoLogoSetup#1{%
  \edef\HOLOGO@name{#1}%
  \ltx@IfUndefined{HoLogo@\HOLOGO@name}{%
    \@PackageError{hologo}{%
      Unknown logo `\HOLOGO@name'%
    }\@ehc
    \ltx@gobble
  }{%
    \HOLOGO@Setup
  }%
}
%    \end{macrocode}
%    \end{macro}
%
%    \begin{macro}{\HOLOGO@Setup}
%    \begin{macrocode}
\def\HOLOGO@Setup{%
  \kvsetkeys{HoLogo}%
}
%    \end{macrocode}
%    \end{macro}
%
% \subsection{Options}
%
%    \begin{macro}{\HOLOGO@DeclareBoolOption}
%    \begin{macrocode}
\def\HOLOGO@DeclareBoolOption#1{%
  \expandafter\chardef\csname HOLOGOOPT@#1\endcsname\ltx@zero
  \kv@define@key{HoLogo}{#1}[true]{%
    \def\HOLOGO@temp{##1}%
    \ifx\HOLOGO@temp\HOLOGO@true
      \ifx\HOLOGO@name\relax
        \expandafter\chardef\csname HOLOGOOPT@#1\endcsname=\ltx@one
      \else
        \expandafter\chardef\csname
        HoLogoOpt@#1@\HOLOGO@name\endcsname\ltx@one
      \fi
      \HOLOGO@SetBreakAll{#1}%
    \else
      \ifx\HOLOGO@temp\HOLOGO@false
        \ifx\HOLOGO@name\relax
          \expandafter\chardef\csname HOLOGOOPT@#1\endcsname=\ltx@zero
        \else
          \expandafter\chardef\csname
          HoLogoOpt@#1@\HOLOGO@name\endcsname=\ltx@zero
        \fi
        \HOLOGO@SetBreakAll{#1}%
      \else
        \@PackageError{hologo}{%
          Unknown value `##1' for boolean option `#1'.\MessageBreak
          Known values are `true' and `false'%
        }\@ehc
      \fi
    \fi
  }%
}
%    \end{macrocode}
%    \end{macro}
%
%    \begin{macro}{\HOLOGO@SetBreakAll}
%    \begin{macrocode}
\def\HOLOGO@SetBreakAll#1{%
  \def\HOLOGO@temp{#1}%
  \ifx\HOLOGO@temp\HOLOGO@break
    \ifx\HOLOGO@name\relax
      \chardef\HOLOGOOPT@hyphenbreak=\HOLOGOOPT@break
      \chardef\HOLOGOOPT@spacebreak=\HOLOGOOPT@break
      \chardef\HOLOGOOPT@discretionarybreak=\HOLOGOOPT@break
    \else
      \expandafter\chardef
         \csname HoLogoOpt@hyphenbreak@\HOLOGO@name\endcsname=%
         \csname HoLogoOpt@break@\HOLOGO@name\endcsname
      \expandafter\chardef
         \csname HoLogoOpt@spacebreak@\HOLOGO@name\endcsname=%
         \csname HoLogoOpt@break@\HOLOGO@name\endcsname
      \expandafter\chardef
         \csname HoLogoOpt@discretionarybreak@\HOLOGO@name
             \endcsname=%
         \csname HoLogoOpt@break@\HOLOGO@name\endcsname
    \fi
  \fi
}
%    \end{macrocode}
%    \end{macro}
%
%    \begin{macro}{\HOLOGO@true}
%    \begin{macrocode}
\def\HOLOGO@true{true}
%    \end{macrocode}
%    \end{macro}
%    \begin{macro}{\HOLOGO@false}
%    \begin{macrocode}
\def\HOLOGO@false{false}
%    \end{macrocode}
%    \end{macro}
%    \begin{macro}{\HOLOGO@break}
%    \begin{macrocode}
\def\HOLOGO@break{break}
%    \end{macrocode}
%    \end{macro}
%
%    \begin{macrocode}
\HOLOGO@DeclareBoolOption{break}
\HOLOGO@DeclareBoolOption{hyphenbreak}
\HOLOGO@DeclareBoolOption{spacebreak}
\HOLOGO@DeclareBoolOption{discretionarybreak}
%    \end{macrocode}
%
%    \begin{macrocode}
\kv@define@key{HoLogo}{variant}{%
  \ifx\HOLOGO@name\relax
    \@PackageError{hologo}{%
      Option `variant' is not available in \string\hologoSetup,%
      \MessageBreak
      Use \string\hologoLogoSetup\space instead%
    }\@ehc
  \else
    \edef\HOLOGO@temp{#1}%
    \ifx\HOLOGO@temp\ltx@empty
      \expandafter
      \let\csname HoLogoOpt@variant@\HOLOGO@name\endcsname\@undefined
    \else
      \ltx@IfUndefined{HoLogo@\HOLOGO@name @\HOLOGO@temp}{%
        \@PackageError{hologo}{%
          Unknown variant `\HOLOGO@temp' of logo `\HOLOGO@name'%
        }\@ehc
      }{%
        \expandafter
        \let\csname HoLogoOpt@variant@\HOLOGO@name\endcsname
            \HOLOGO@temp
      }%
    \fi
  \fi
}
%    \end{macrocode}
%
%    \begin{macro}{\HOLOGO@Variant}
%    \begin{macrocode}
\def\HOLOGO@Variant#1{%
  #1%
  \ltx@ifundefined{HoLogoOpt@variant@#1}{%
  }{%
    @\csname HoLogoOpt@variant@#1\endcsname
  }%
}
%    \end{macrocode}
%    \end{macro}
%
% \subsection{Break/no-break support}
%
%    \begin{macro}{\HOLOGO@space}
%    \begin{macrocode}
\def\HOLOGO@space{%
  \ltx@ifundefined{HoLogoOpt@spacebreak@\HOLOGO@name}{%
    \ltx@ifundefined{HoLogoOpt@break@\HOLOGO@name}{%
      \chardef\HOLOGO@temp=\HOLOGOOPT@spacebreak
    }{%
      \chardef\HOLOGO@temp=%
        \csname HoLogoOpt@break@\HOLOGO@name\endcsname
    }%
  }{%
    \chardef\HOLOGO@temp=%
      \csname HoLogoOpt@spacebreak@\HOLOGO@name\endcsname
  }%
  \ifcase\HOLOGO@temp
    \penalty10000 %
  \fi
  \ltx@space
}
%    \end{macrocode}
%    \end{macro}
%
%    \begin{macro}{\HOLOGO@hyphen}
%    \begin{macrocode}
\def\HOLOGO@hyphen{%
  \ltx@ifundefined{HoLogoOpt@hyphenbreak@\HOLOGO@name}{%
    \ltx@ifundefined{HoLogoOpt@break@\HOLOGO@name}{%
      \chardef\HOLOGO@temp=\HOLOGOOPT@hyphenbreak
    }{%
      \chardef\HOLOGO@temp=%
        \csname HoLogoOpt@break@\HOLOGO@name\endcsname
    }%
  }{%
    \chardef\HOLOGO@temp=%
      \csname HoLogoOpt@hyphenbreak@\HOLOGO@name\endcsname
  }%
  \ifcase\HOLOGO@temp
    \ltx@mbox{-}%
  \else
    -%
  \fi
}
%    \end{macrocode}
%    \end{macro}
%
%    \begin{macro}{\HOLOGO@discretionary}
%    \begin{macrocode}
\def\HOLOGO@discretionary{%
  \ltx@ifundefined{HoLogoOpt@discretionarybreak@\HOLOGO@name}{%
    \ltx@ifundefined{HoLogoOpt@break@\HOLOGO@name}{%
      \chardef\HOLOGO@temp=\HOLOGOOPT@discretionarybreak
    }{%
      \chardef\HOLOGO@temp=%
        \csname HoLogoOpt@break@\HOLOGO@name\endcsname
    }%
  }{%
    \chardef\HOLOGO@temp=%
      \csname HoLogoOpt@discretionarybreak@\HOLOGO@name\endcsname
  }%
  \ifcase\HOLOGO@temp
  \else
    \-%
  \fi
}
%    \end{macrocode}
%    \end{macro}
%
%    \begin{macro}{\HOLOGO@mbox}
%    \begin{macrocode}
\def\HOLOGO@mbox#1{%
  \ltx@ifundefined{HoLogoOpt@break@\HOLOGO@name}{%
    \chardef\HOLOGO@temp=\HOLOGOOPT@hyphenbreak
  }{%
    \chardef\HOLOGO@temp=%
      \csname HoLogoOpt@break@\HOLOGO@name\endcsname
  }%
  \ifcase\HOLOGO@temp
    \ltx@mbox{#1}%
  \else
    #1%
  \fi
}
%    \end{macrocode}
%    \end{macro}
%
% \subsection{Font support}
%
%    \begin{macro}{\HoLogoFont@font}
%    \begin{tabular}{@{}ll@{}}
%    |#1|:& logo name\\
%    |#2|:& font short name\\
%    |#3|:& text
%    \end{tabular}
%    \begin{macrocode}
\def\HoLogoFont@font#1#2#3{%
  \begingroup
    \ltx@IfUndefined{HoLogoFont@logo@#1.#2}{%
      \ltx@IfUndefined{HoLogoFont@font@#2}{%
        \@PackageWarning{hologo}{%
          Missing font `#2' for logo `#1'%
        }%
        #3%
      }{%
        \csname HoLogoFont@font@#2\endcsname{#3}%
      }%
    }{%
      \csname HoLogoFont@logo@#1.#2\endcsname{#3}%
    }%
  \endgroup
}
%    \end{macrocode}
%    \end{macro}
%
%    \begin{macro}{\HoLogoFont@Def}
%    \begin{macrocode}
\def\HoLogoFont@Def#1{%
  \expandafter\def\csname HoLogoFont@font@#1\endcsname
}
%    \end{macrocode}
%    \end{macro}
%    \begin{macro}{\HoLogoFont@LogoDef}
%    \begin{macrocode}
\def\HoLogoFont@LogoDef#1#2{%
  \expandafter\def\csname HoLogoFont@logo@#1.#2\endcsname
}
%    \end{macrocode}
%    \end{macro}
%
% \subsubsection{Font defaults}
%
%    \begin{macro}{\HoLogoFont@font@general}
%    \begin{macrocode}
\HoLogoFont@Def{general}{}%
%    \end{macrocode}
%    \end{macro}
%
%    \begin{macro}{\HoLogoFont@font@rm}
%    \begin{macrocode}
\ltx@IfUndefined{rmfamily}{%
  \ltx@IfUndefined{rm}{%
  }{%
    \HoLogoFont@Def{rm}{\rm}%
  }%
}{%
  \HoLogoFont@Def{rm}{\rmfamily}%
}
%    \end{macrocode}
%    \end{macro}
%
%    \begin{macro}{\HoLogoFont@font@sf}
%    \begin{macrocode}
\ltx@IfUndefined{sffamily}{%
  \ltx@IfUndefined{sf}{%
  }{%
    \HoLogoFont@Def{sf}{\sf}%
  }%
}{%
  \HoLogoFont@Def{sf}{\sffamily}%
}
%    \end{macrocode}
%    \end{macro}
%
%    \begin{macro}{\HoLogoFont@font@bibsf}
%    In case of \hologo{plainTeX} the original small caps
%    variant is used as default. In \hologo{LaTeX}
%    the definition of package \xpackage{dtklogos} \cite{dtklogos}
%    is used.
%\begin{quote}
%\begin{verbatim}
%\DeclareRobustCommand{\BibTeX}{%
%  B%
%  \kern-.05em%
%  \hbox{%
%    $\m@th$% %% force math size calculations
%    \csname S@\f@size\endcsname
%    \fontsize\sf@size\z@
%    \math@fontsfalse
%    \selectfont
%    I%
%    \kern-.025em%
%    B
%  }%
%  \kern-.08em%
%  \-%
%  \TeX
%}
%\end{verbatim}
%\end{quote}
%    \begin{macrocode}
\ltx@IfUndefined{selectfont}{%
  \ltx@IfUndefined{tensc}{%
    \font\tensc=cmcsc10\relax
  }{}%
  \HoLogoFont@Def{bibsf}{\tensc}%
}{%
  \HoLogoFont@Def{bibsf}{%
    $\mathsurround=0pt$%
    \csname S@\f@size\endcsname
    \fontsize\sf@size{0pt}%
    \math@fontsfalse
    \selectfont
  }%
}
%    \end{macrocode}
%    \end{macro}
%
%    \begin{macro}{\HoLogoFont@font@sc}
%    \begin{macrocode}
\ltx@IfUndefined{scshape}{%
  \ltx@IfUndefined{tensc}{%
    \font\tensc=cmcsc10\relax
  }{}%
  \HoLogoFont@Def{sc}{\tensc}%
}{%
  \HoLogoFont@Def{sc}{\scshape}%
}
%    \end{macrocode}
%    \end{macro}
%
%    \begin{macro}{\HoLogoFont@font@sy}
%    \begin{macrocode}
\ltx@IfUndefined{usefont}{%
  \ltx@IfUndefined{tensy}{%
  }{%
    \HoLogoFont@Def{sy}{\tensy}%
  }%
}{%
  \HoLogoFont@Def{sy}{%
    \usefont{OMS}{cmsy}{m}{n}%
  }%
}
%    \end{macrocode}
%    \end{macro}
%
%    \begin{macro}{\HoLogoFont@font@logo}
%    \begin{macrocode}
\begingroup
  \def\x{LaTeX2e}%
\expandafter\endgroup
\ifx\fmtname\x
  \ltx@IfUndefined{logofamily}{%
    \DeclareRobustCommand\logofamily{%
      \not@math@alphabet\logofamily\relax
      \fontencoding{U}%
      \fontfamily{logo}%
      \selectfont
    }%
  }{}%
  \ltx@IfUndefined{logofamily}{%
  }{%
    \HoLogoFont@Def{logo}{\logofamily}%
  }%
\else
  \ltx@IfUndefined{tenlogo}{%
    \font\tenlogo=logo10\relax
  }{}%
  \HoLogoFont@Def{logo}{\tenlogo}%
\fi
%    \end{macrocode}
%    \end{macro}
%
% \subsubsection{Font setup}
%
%    \begin{macro}{\hologoFontSetup}
%    \begin{macrocode}
\def\hologoFontSetup{%
  \let\HOLOGO@name\relax
  \HOLOGO@FontSetup
}
%    \end{macrocode}
%    \end{macro}
%
%    \begin{macro}{\hologoLogoFontSetup}
%    \begin{macrocode}
\def\hologoLogoFontSetup#1{%
  \edef\HOLOGO@name{#1}%
  \ltx@IfUndefined{HoLogo@\HOLOGO@name}{%
    \@PackageError{hologo}{%
      Unknown logo `\HOLOGO@name'%
    }\@ehc
    \ltx@gobble
  }{%
    \HOLOGO@FontSetup
  }%
}
%    \end{macrocode}
%    \end{macro}
%
%    \begin{macro}{\HOLOGO@FontSetup}
%    \begin{macrocode}
\def\HOLOGO@FontSetup{%
  \kvsetkeys{HoLogoFont}%
}
%    \end{macrocode}
%    \end{macro}
%
%    \begin{macrocode}
\def\HOLOGO@temp#1{%
  \kv@define@key{HoLogoFont}{#1}{%
    \ifx\HOLOGO@name\relax
      \HoLogoFont@Def{#1}{##1}%
    \else
      \HoLogoFont@LogoDef\HOLOGO@name{#1}{##1}%
    \fi
  }%
}
\HOLOGO@temp{general}
\HOLOGO@temp{sf}
%    \end{macrocode}
%
% \subsection{Generic logo commands}
%
%    \begin{macrocode}
\HOLOGO@IfExists\hologo{%
  \@PackageError{hologo}{%
    \string\hologo\ltx@space is already defined.\MessageBreak
    Package loading is aborted%
  }\@ehc
  \HOLOGO@AtEnd
}%
\HOLOGO@IfExists\hologoRobust{%
  \@PackageError{hologo}{%
    \string\hologoRobust\ltx@space is already defined.\MessageBreak
    Package loading is aborted%
  }\@ehc
  \HOLOGO@AtEnd
}%
%    \end{macrocode}
%
% \subsubsection{\cs{hologo} and friends}
%
%    \begin{macrocode}
\ifluatex
  \expandafter\ltx@firstofone
\else
  \expandafter\ltx@gobble
\fi
{%
  \ltx@IfUndefined{ifincsname}{%
    \ifnum\luatexversion<36 %
      \expandafter\ltx@gobble
    \else
      \expandafter\ltx@firstofone
    \fi
    {%
      \begingroup
        \ifcase0%
            \directlua{%
              if tex.enableprimitives then %
                tex.enableprimitives('HOLOGO@', {'ifincsname'})%
              else %
                tex.print('1')%
              end%
            }%
            \ifx\HOLOGO@ifincsname\@undefined 1\fi%
            \relax
          \expandafter\ltx@firstofone
        \else
          \endgroup
          \expandafter\ltx@gobble
        \fi
        {%
          \global\let\ifincsname\HOLOGO@ifincsname
        }%
      \HOLOGO@temp
    }%
  }{}%
}
%    \end{macrocode}
%    \begin{macrocode}
\ltx@IfUndefined{ifincsname}{%
  \catcode`$=14 %
}{%
  \catcode`$=9 %
}
%    \end{macrocode}
%
%    \begin{macro}{\hologo}
%    \begin{macrocode}
\def\hologo#1{%
$ \ifincsname
$   \ltx@ifundefined{HoLogoCs@\HOLOGO@Variant{#1}}{%
$     #1%
$   }{%
$     \csname HoLogoCs@\HOLOGO@Variant{#1}\endcsname\ltx@firstoftwo
$   }%
$ \else
    \HOLOGO@IfExists\texorpdfstring\texorpdfstring\ltx@firstoftwo
    {%
      \hologoRobust{#1}%
    }{%
      \ltx@ifundefined{HoLogoBkm@\HOLOGO@Variant{#1}}{%
        \ltx@ifundefined{HoLogo@#1}{?#1?}{#1}%
      }{%
        \csname HoLogoBkm@\HOLOGO@Variant{#1}\endcsname
        \ltx@firstoftwo
      }%
    }%
$ \fi
}
%    \end{macrocode}
%    \end{macro}
%    \begin{macro}{\Hologo}
%    \begin{macrocode}
\def\Hologo#1{%
$ \ifincsname
$   \ltx@ifundefined{HoLogoCs@\HOLOGO@Variant{#1}}{%
$     #1%
$   }{%
$     \csname HoLogoCs@\HOLOGO@Variant{#1}\endcsname\ltx@secondoftwo
$   }%
$ \else
    \HOLOGO@IfExists\texorpdfstring\texorpdfstring\ltx@firstoftwo
    {%
      \HologoRobust{#1}%
    }{%
      \ltx@ifundefined{HoLogoBkm@\HOLOGO@Variant{#1}}{%
        \ltx@ifundefined{HoLogo@#1}{?#1?}{#1}%
      }{%
        \csname HoLogoBkm@\HOLOGO@Variant{#1}\endcsname
        \ltx@secondoftwo
      }%
    }%
$ \fi
}
%    \end{macrocode}
%    \end{macro}
%
%    \begin{macro}{\hologoVariant}
%    \begin{macrocode}
\def\hologoVariant#1#2{%
  \ifx\relax#2\relax
    \hologo{#1}%
  \else
$   \ifincsname
$     \ltx@ifundefined{HoLogoCs@#1@#2}{%
$       #1%
$     }{%
$       \csname HoLogoCs@#1@#2\endcsname\ltx@firstoftwo
$     }%
$   \else
      \HOLOGO@IfExists\texorpdfstring\texorpdfstring\ltx@firstoftwo
      {%
        \hologoVariantRobust{#1}{#2}%
      }{%
        \ltx@ifundefined{HoLogoBkm@#1@#2}{%
          \ltx@ifundefined{HoLogo@#1}{?#1?}{#1}%
        }{%
          \csname HoLogoBkm@#1@#2\endcsname
          \ltx@firstoftwo
        }%
      }%
$   \fi
  \fi
}
%    \end{macrocode}
%    \end{macro}
%    \begin{macro}{\HologoVariant}
%    \begin{macrocode}
\def\HologoVariant#1#2{%
  \ifx\relax#2\relax
    \Hologo{#1}%
  \else
$   \ifincsname
$     \ltx@ifundefined{HoLogoCs@#1@#2}{%
$       #1%
$     }{%
$       \csname HoLogoCs@#1@#2\endcsname\ltx@secondoftwo
$     }%
$   \else
      \HOLOGO@IfExists\texorpdfstring\texorpdfstring\ltx@firstoftwo
      {%
        \HologoVariantRobust{#1}{#2}%
      }{%
        \ltx@ifundefined{HoLogoBkm@#1@#2}{%
          \ltx@ifundefined{HoLogo@#1}{?#1?}{#1}%
        }{%
          \csname HoLogoBkm@#1@#2\endcsname
          \ltx@secondoftwo
        }%
      }%
$   \fi
  \fi
}
%    \end{macrocode}
%    \end{macro}
%
%    \begin{macrocode}
\catcode`\$=3 %
%    \end{macrocode}
%
% \subsubsection{\cs{hologoRobust} and friends}
%
%    \begin{macro}{\hologoRobust}
%    \begin{macrocode}
\ltx@IfUndefined{protected}{%
  \ltx@IfUndefined{DeclareRobustCommand}{%
    \def\hologoRobust#1%
  }{%
    \DeclareRobustCommand*\hologoRobust[1]%
  }%
}{%
  \protected\def\hologoRobust#1%
}%
{%
  \edef\HOLOGO@name{#1}%
  \ltx@IfUndefined{HoLogo@\HOLOGO@Variant\HOLOGO@name}{%
    \@PackageError{hologo}{%
      Unknown logo `\HOLOGO@name'%
    }\@ehc
    ?\HOLOGO@name?%
  }{%
    \ltx@IfUndefined{ver@tex4ht.sty}{%
      \HoLogoFont@font\HOLOGO@name{general}{%
        \csname HoLogo@\HOLOGO@Variant\HOLOGO@name\endcsname
        \ltx@firstoftwo
      }%
    }{%
      \ltx@IfUndefined{HoLogoHtml@\HOLOGO@Variant\HOLOGO@name}{%
        \HOLOGO@name
      }{%
        \csname HoLogoHtml@\HOLOGO@Variant\HOLOGO@name\endcsname
        \ltx@firstoftwo
      }%
    }%
  }%
}
%    \end{macrocode}
%    \end{macro}
%    \begin{macro}{\HologoRobust}
%    \begin{macrocode}
\ltx@IfUndefined{protected}{%
  \ltx@IfUndefined{DeclareRobustCommand}{%
    \def\HologoRobust#1%
  }{%
    \DeclareRobustCommand*\HologoRobust[1]%
  }%
}{%
  \protected\def\HologoRobust#1%
}%
{%
  \edef\HOLOGO@name{#1}%
  \ltx@IfUndefined{HoLogo@\HOLOGO@Variant\HOLOGO@name}{%
    \@PackageError{hologo}{%
      Unknown logo `\HOLOGO@name'%
    }\@ehc
    ?\HOLOGO@name?%
  }{%
    \ltx@IfUndefined{ver@tex4ht.sty}{%
      \HoLogoFont@font\HOLOGO@name{general}{%
        \csname HoLogo@\HOLOGO@Variant\HOLOGO@name\endcsname
        \ltx@secondoftwo
      }%
    }{%
      \ltx@IfUndefined{HoLogoHtml@\HOLOGO@Variant\HOLOGO@name}{%
        \expandafter\HOLOGO@Uppercase\HOLOGO@name
      }{%
        \csname HoLogoHtml@\HOLOGO@Variant\HOLOGO@name\endcsname
        \ltx@secondoftwo
      }%
    }%
  }%
}
%    \end{macrocode}
%    \end{macro}
%    \begin{macro}{\hologoVariantRobust}
%    \begin{macrocode}
\ltx@IfUndefined{protected}{%
  \ltx@IfUndefined{DeclareRobustCommand}{%
    \def\hologoVariantRobust#1#2%
  }{%
    \DeclareRobustCommand*\hologoVariantRobust[2]%
  }%
}{%
  \protected\def\hologoVariantRobust#1#2%
}%
{%
  \begingroup
    \hologoLogoSetup{#1}{variant={#2}}%
    \hologoRobust{#1}%
  \endgroup
}
%    \end{macrocode}
%    \end{macro}
%    \begin{macro}{\HologoVariantRobust}
%    \begin{macrocode}
\ltx@IfUndefined{protected}{%
  \ltx@IfUndefined{DeclareRobustCommand}{%
    \def\HologoVariantRobust#1#2%
  }{%
    \DeclareRobustCommand*\HologoVariantRobust[2]%
  }%
}{%
  \protected\def\HologoVariantRobust#1#2%
}%
{%
  \begingroup
    \hologoLogoSetup{#1}{variant={#2}}%
    \HologoRobust{#1}%
  \endgroup
}
%    \end{macrocode}
%    \end{macro}
%
%    \begin{macro}{\hologorobust}
%    Macro \cs{hologorobust} is only defined for compatibility.
%    Its use is deprecated.
%    \begin{macrocode}
\def\hologorobust{\hologoRobust}
%    \end{macrocode}
%    \end{macro}
%
% \subsection{Helpers}
%
%    \begin{macro}{\HOLOGO@Uppercase}
%    Macro \cs{HOLOGO@Uppercase} is restricted to \cs{uppercase},
%    because \hologo{plainTeX} or \hologo{iniTeX} do not provide
%    \cs{MakeUppercase}.
%    \begin{macrocode}
\def\HOLOGO@Uppercase#1{\uppercase{#1}}
%    \end{macrocode}
%    \end{macro}
%
%    \begin{macro}{\HOLOGO@PdfdocUnicode}
%    \begin{macrocode}
\def\HOLOGO@PdfdocUnicode{%
  \ifx\ifHy@unicode\iftrue
    \expandafter\ltx@secondoftwo
  \else
    \expandafter\ltx@firstoftwo
  \fi
}
%    \end{macrocode}
%    \end{macro}
%
%    \begin{macro}{\HOLOGO@Math}
%    \begin{macrocode}
\def\HOLOGO@MathSetup{%
  \mathsurround0pt\relax
  \HOLOGO@IfExists\f@series{%
    \if b\expandafter\ltx@car\f@series x\@nil
      \csname boldmath\endcsname
   \fi
  }{}%
}
%    \end{macrocode}
%    \end{macro}
%
%    \begin{macro}{\HOLOGO@TempDimen}
%    \begin{macrocode}
\dimendef\HOLOGO@TempDimen=\ltx@zero
%    \end{macrocode}
%    \end{macro}
%    \begin{macro}{\HOLOGO@NegativeKerning}
%    \begin{macrocode}
\def\HOLOGO@NegativeKerning#1{%
  \begingroup
    \HOLOGO@TempDimen=0pt\relax
    \comma@parse@normalized{#1}{%
      \ifdim\HOLOGO@TempDimen=0pt %
        \expandafter\HOLOGO@@NegativeKerning\comma@entry
      \fi
      \ltx@gobble
    }%
    \ifdim\HOLOGO@TempDimen<0pt %
      \kern\HOLOGO@TempDimen
    \fi
  \endgroup
}
%    \end{macrocode}
%    \end{macro}
%    \begin{macro}{\HOLOGO@@NegativeKerning}
%    \begin{macrocode}
\def\HOLOGO@@NegativeKerning#1#2{%
  \setbox\ltx@zero\hbox{#1#2}%
  \HOLOGO@TempDimen=\wd\ltx@zero
  \setbox\ltx@zero\hbox{#1\kern0pt#2}%
  \advance\HOLOGO@TempDimen by -\wd\ltx@zero
}
%    \end{macrocode}
%    \end{macro}
%
%    \begin{macro}{\HOLOGO@SpaceFactor}
%    \begin{macrocode}
\def\HOLOGO@SpaceFactor{%
  \spacefactor1000 %
}
%    \end{macrocode}
%    \end{macro}
%
%    \begin{macro}{\HOLOGO@Span}
%    \begin{macrocode}
\def\HOLOGO@Span#1#2{%
  \HCode{<span class="HoLogo-#1">}%
  #2%
  \HCode{</span>}%
}
%    \end{macrocode}
%    \end{macro}
%
% \subsubsection{Text subscript}
%
%    \begin{macro}{\HOLOGO@SubScript}%
%    \begin{macrocode}
\def\HOLOGO@SubScript#1{%
  \ltx@IfUndefined{textsubscript}{%
    \ltx@IfUndefined{text}{%
      \ltx@mbox{%
        \mathsurround=0pt\relax
        $%
          _{%
            \ltx@IfUndefined{sf@size}{%
              \mathrm{#1}%
            }{%
              \mbox{%
                \fontsize\sf@size{0pt}\selectfont
                #1%
              }%
            }%
          }%
        $%
      }%
    }{%
      \ltx@mbox{%
        \mathsurround=0pt\relax
        $_{\text{#1}}$%
      }%
    }%
  }{%
    \textsubscript{#1}%
  }%
}
%    \end{macrocode}
%    \end{macro}
%
% \subsection{\hologo{TeX} and friends}
%
% \subsubsection{\hologo{TeX}}
%
%    \begin{macro}{\HoLogo@TeX}
%    Source: \hologo{LaTeX} kernel.
%    \begin{macrocode}
\def\HoLogo@TeX#1{%
  T\kern-.1667em\lower.5ex\hbox{E}\kern-.125emX\HOLOGO@SpaceFactor
}
%    \end{macrocode}
%    \end{macro}
%    \begin{macro}{\HoLogoHtml@TeX}
%    \begin{macrocode}
\def\HoLogoHtml@TeX#1{%
  \HoLogoCss@TeX
  \HOLOGO@Span{TeX}{%
    T%
    \HOLOGO@Span{e}{%
      E%
    }%
    X%
  }%
}
%    \end{macrocode}
%    \end{macro}
%    \begin{macro}{\HoLogoCss@TeX}
%    \begin{macrocode}
\def\HoLogoCss@TeX{%
  \Css{%
    span.HoLogo-TeX span.HoLogo-e{%
      position:relative;%
      top:.5ex;%
      margin-left:-.1667em;%
      margin-right:-.125em;%
    }%
  }%
  \Css{%
    a span.HoLogo-TeX span.HoLogo-e{%
      text-decoration:none;%
    }%
  }%
  \global\let\HoLogoCss@TeX\relax
}
%    \end{macrocode}
%    \end{macro}
%
% \subsubsection{\hologo{plainTeX}}
%
%    \begin{macro}{\HoLogo@plainTeX@space}
%    Source: ``The \hologo{TeX}book''
%    \begin{macrocode}
\def\HoLogo@plainTeX@space#1{%
  \HOLOGO@mbox{#1{p}{P}lain}\HOLOGO@space\hologo{TeX}%
}
%    \end{macrocode}
%    \end{macro}
%    \begin{macro}{\HoLogoCs@plainTeX@space}
%    \begin{macrocode}
\def\HoLogoCs@plainTeX@space#1{#1{p}{P}lain TeX}%
%    \end{macrocode}
%    \end{macro}
%    \begin{macro}{\HoLogoBkm@plainTeX@space}
%    \begin{macrocode}
\def\HoLogoBkm@plainTeX@space#1{%
  #1{p}{P}lain \hologo{TeX}%
}
%    \end{macrocode}
%    \end{macro}
%    \begin{macro}{\HoLogoHtml@plainTeX@space}
%    \begin{macrocode}
\def\HoLogoHtml@plainTeX@space#1{%
  #1{p}{P}lain \hologo{TeX}%
}
%    \end{macrocode}
%    \end{macro}
%
%    \begin{macro}{\HoLogo@plainTeX@hyphen}
%    \begin{macrocode}
\def\HoLogo@plainTeX@hyphen#1{%
  \HOLOGO@mbox{#1{p}{P}lain}\HOLOGO@hyphen\hologo{TeX}%
}
%    \end{macrocode}
%    \end{macro}
%    \begin{macro}{\HoLogoCs@plainTeX@hyphen}
%    \begin{macrocode}
\def\HoLogoCs@plainTeX@hyphen#1{#1{p}{P}lain-TeX}
%    \end{macrocode}
%    \end{macro}
%    \begin{macro}{\HoLogoBkm@plainTeX@hyphen}
%    \begin{macrocode}
\def\HoLogoBkm@plainTeX@hyphen#1{%
  #1{p}{P}lain-\hologo{TeX}%
}
%    \end{macrocode}
%    \end{macro}
%    \begin{macro}{\HoLogoHtml@plainTeX@hyphen}
%    \begin{macrocode}
\def\HoLogoHtml@plainTeX@hyphen#1{%
  #1{p}{P}lain-\hologo{TeX}%
}
%    \end{macrocode}
%    \end{macro}
%
%    \begin{macro}{\HoLogo@plainTeX@runtogether}
%    \begin{macrocode}
\def\HoLogo@plainTeX@runtogether#1{%
  \HOLOGO@mbox{#1{p}{P}lain\hologo{TeX}}%
}
%    \end{macrocode}
%    \end{macro}
%    \begin{macro}{\HoLogoCs@plainTeX@runtogether}
%    \begin{macrocode}
\def\HoLogoCs@plainTeX@runtogether#1{#1{p}{P}lainTeX}
%    \end{macrocode}
%    \end{macro}
%    \begin{macro}{\HoLogoBkm@plainTeX@runtogether}
%    \begin{macrocode}
\def\HoLogoBkm@plainTeX@runtogether#1{%
  #1{p}{P}lain\hologo{TeX}%
}
%    \end{macrocode}
%    \end{macro}
%    \begin{macro}{\HoLogoHtml@plainTeX@runtogether}
%    \begin{macrocode}
\def\HoLogoHtml@plainTeX@runtogether#1{%
  #1{p}{P}lain\hologo{TeX}%
}
%    \end{macrocode}
%    \end{macro}
%
%    \begin{macro}{\HoLogo@plainTeX}
%    \begin{macrocode}
\def\HoLogo@plainTeX{\HoLogo@plainTeX@space}
%    \end{macrocode}
%    \end{macro}
%    \begin{macro}{\HoLogoCs@plainTeX}
%    \begin{macrocode}
\def\HoLogoCs@plainTeX{\HoLogoCs@plainTeX@space}
%    \end{macrocode}
%    \end{macro}
%    \begin{macro}{\HoLogoBkm@plainTeX}
%    \begin{macrocode}
\def\HoLogoBkm@plainTeX{\HoLogoBkm@plainTeX@space}
%    \end{macrocode}
%    \end{macro}
%    \begin{macro}{\HoLogoHtml@plainTeX}
%    \begin{macrocode}
\def\HoLogoHtml@plainTeX{\HoLogoHtml@plainTeX@space}
%    \end{macrocode}
%    \end{macro}
%
% \subsubsection{\hologo{LaTeX}}
%
%    Source: \hologo{LaTeX} kernel.
%\begin{quote}
%\begin{verbatim}
%\DeclareRobustCommand{\LaTeX}{%
%  L%
%  \kern-.36em%
%  {%
%    \sbox\z@ T%
%    \vbox to\ht\z@{%
%      \hbox{%
%        \check@mathfonts
%        \fontsize\sf@size\z@
%        \math@fontsfalse
%        \selectfont
%        A%
%      }%
%      \vss
%    }%
%  }%
%  \kern-.15em%
%  \TeX
%}
%\end{verbatim}
%\end{quote}
%
%    \begin{macro}{\HoLogo@La}
%    \begin{macrocode}
\def\HoLogo@La#1{%
  L%
  \kern-.36em%
  \begingroup
    \setbox\ltx@zero\hbox{T}%
    \vbox to\ht\ltx@zero{%
      \hbox{%
        \ltx@ifundefined{check@mathfonts}{%
          \csname sevenrm\endcsname
        }{%
          \check@mathfonts
          \fontsize\sf@size{0pt}%
          \math@fontsfalse\selectfont
        }%
        A%
      }%
      \vss
    }%
  \endgroup
}
%    \end{macrocode}
%    \end{macro}
%
%    \begin{macro}{\HoLogo@LaTeX}
%    Source: \hologo{LaTeX} kernel.
%    \begin{macrocode}
\def\HoLogo@LaTeX#1{%
  \hologo{La}%
  \kern-.15em%
  \hologo{TeX}%
}
%    \end{macrocode}
%    \end{macro}
%    \begin{macro}{\HoLogoHtml@LaTeX}
%    \begin{macrocode}
\def\HoLogoHtml@LaTeX#1{%
  \HoLogoCss@LaTeX
  \HOLOGO@Span{LaTeX}{%
    L%
    \HOLOGO@Span{a}{%
      A%
    }%
    \hologo{TeX}%
  }%
}
%    \end{macrocode}
%    \end{macro}
%    \begin{macro}{\HoLogoCss@LaTeX}
%    \begin{macrocode}
\def\HoLogoCss@LaTeX{%
  \Css{%
    span.HoLogo-LaTeX span.HoLogo-a{%
      position:relative;%
      top:-.5ex;%
      margin-left:-.36em;%
      margin-right:-.15em;%
      font-size:85\%;%
    }%
  }%
  \global\let\HoLogoCss@LaTeX\relax
}
%    \end{macrocode}
%    \end{macro}
%
% \subsubsection{\hologo{(La)TeX}}
%
%    \begin{macro}{\HoLogo@LaTeXTeX}
%    The kerning around the parentheses is taken
%    from package \xpackage{dtklogos} \cite{dtklogos}.
%\begin{quote}
%\begin{verbatim}
%\DeclareRobustCommand{\LaTeXTeX}{%
%  (%
%  \kern-.15em%
%  L%
%  \kern-.36em%
%  {%
%    \sbox\z@ T%
%    \vbox to\ht0{%
%      \hbox{%
%        $\m@th$%
%        \csname S@\f@size\endcsname
%        \fontsize\sf@size\z@
%        \math@fontsfalse
%        \selectfont
%        A%
%      }%
%      \vss
%    }%
%  }%
%  \kern-.2em%
%  )%
%  \kern-.15em%
%  \TeX
%}
%\end{verbatim}
%\end{quote}
%    \begin{macrocode}
\def\HoLogo@LaTeXTeX#1{%
  (%
  \kern-.15em%
  \hologo{La}%
  \kern-.2em%
  )%
  \kern-.15em%
  \hologo{TeX}%
}
%    \end{macrocode}
%    \end{macro}
%    \begin{macro}{\HoLogoBkm@LaTeXTeX}
%    \begin{macrocode}
\def\HoLogoBkm@LaTeXTeX#1{(La)TeX}
%    \end{macrocode}
%    \end{macro}
%
%    \begin{macro}{\HoLogo@(La)TeX}
%    \begin{macrocode}
\expandafter
\let\csname HoLogo@(La)TeX\endcsname\HoLogo@LaTeXTeX
%    \end{macrocode}
%    \end{macro}
%    \begin{macro}{\HoLogoBkm@(La)TeX}
%    \begin{macrocode}
\expandafter
\let\csname HoLogoBkm@(La)TeX\endcsname\HoLogoBkm@LaTeXTeX
%    \end{macrocode}
%    \end{macro}
%    \begin{macro}{\HoLogoHtml@LaTeXTeX}
%    \begin{macrocode}
\def\HoLogoHtml@LaTeXTeX#1{%
  \HoLogoCss@LaTeXTeX
  \HOLOGO@Span{LaTeXTeX}{%
    (%
    \HOLOGO@Span{L}{L}%
    \HOLOGO@Span{a}{A}%
    \HOLOGO@Span{ParenRight}{)}%
    \hologo{TeX}%
  }%
}
%    \end{macrocode}
%    \end{macro}
%    \begin{macro}{\HoLogoHtml@(La)TeX}
%    Kerning after opening parentheses and before closing parentheses
%    is $-0.1$\,em. The original values $-0.15$\,em
%    looked too ugly for a serif font.
%    \begin{macrocode}
\expandafter
\let\csname HoLogoHtml@(La)TeX\endcsname\HoLogoHtml@LaTeXTeX
%    \end{macrocode}
%    \end{macro}
%    \begin{macro}{\HoLogoCss@LaTeXTeX}
%    \begin{macrocode}
\def\HoLogoCss@LaTeXTeX{%
  \Css{%
    span.HoLogo-LaTeXTeX span.HoLogo-L{%
      margin-left:-.1em;%
    }%
  }%
  \Css{%
    span.HoLogo-LaTeXTeX span.HoLogo-a{%
      position:relative;%
      top:-.5ex;%
      margin-left:-.36em;%
      margin-right:-.1em;%
      font-size:85\%;%
    }%
  }%
  \Css{%
    span.HoLogo-LaTeXTeX span.HoLogo-ParenRight{%
      margin-right:-.15em;%
    }%
  }%
  \global\let\HoLogoCss@LaTeXTeX\relax
}
%    \end{macrocode}
%    \end{macro}
%
% \subsubsection{\hologo{LaTeXe}}
%
%    \begin{macro}{\HoLogo@LaTeXe}
%    Source: \hologo{LaTeX} kernel
%    \begin{macrocode}
\def\HoLogo@LaTeXe#1{%
  \hologo{LaTeX}%
  \kern.15em%
  \hbox{%
    \HOLOGO@MathSetup
    2%
    $_{\textstyle\varepsilon}$%
  }%
}
%    \end{macrocode}
%    \end{macro}
%
%    \begin{macro}{\HoLogoCs@LaTeXe}
%    \begin{macrocode}
\ifnum64=`\^^^^0040\relax % test for big chars of LuaTeX/XeTeX
  \catcode`\$=9 %
  \catcode`\&=14 %
\else
  \catcode`\$=14 %
  \catcode`\&=9 %
\fi
\def\HoLogoCs@LaTeXe#1{%
  LaTeX2%
$ \string ^^^^0395%
& e%
}%
\catcode`\$=3 %
\catcode`\&=4 %
%    \end{macrocode}
%    \end{macro}
%
%    \begin{macro}{\HoLogoBkm@LaTeXe}
%    \begin{macrocode}
\def\HoLogoBkm@LaTeXe#1{%
  \hologo{LaTeX}%
  2%
  \HOLOGO@PdfdocUnicode{e}{\textepsilon}%
}
%    \end{macrocode}
%    \end{macro}
%
%    \begin{macro}{\HoLogoHtml@LaTeXe}
%    \begin{macrocode}
\def\HoLogoHtml@LaTeXe#1{%
  \HoLogoCss@LaTeXe
  \HOLOGO@Span{LaTeX2e}{%
    \hologo{LaTeX}%
    \HOLOGO@Span{2}{2}%
    \HOLOGO@Span{e}{%
      \HOLOGO@MathSetup
      \ensuremath{\textstyle\varepsilon}%
    }%
  }%
}
%    \end{macrocode}
%    \end{macro}
%    \begin{macro}{\HoLogoCss@LaTeXe}
%    \begin{macrocode}
\def\HoLogoCss@LaTeXe{%
  \Css{%
    span.HoLogo-LaTeX2e span.HoLogo-2{%
      padding-left:.15em;%
    }%
  }%
  \Css{%
    span.HoLogo-LaTeX2e span.HoLogo-e{%
      position:relative;%
      top:.35ex;%
      text-decoration:none;%
    }%
  }%
  \global\let\HoLogoCss@LaTeXe\relax
}
%    \end{macrocode}
%    \end{macro}
%
%    \begin{macro}{\HoLogo@LaTeX2e}
%    \begin{macrocode}
\expandafter
\let\csname HoLogo@LaTeX2e\endcsname\HoLogo@LaTeXe
%    \end{macrocode}
%    \end{macro}
%    \begin{macro}{\HoLogoCs@LaTeX2e}
%    \begin{macrocode}
\expandafter
\let\csname HoLogoCs@LaTeX2e\endcsname\HoLogoCs@LaTeXe
%    \end{macrocode}
%    \end{macro}
%    \begin{macro}{\HoLogoBkm@LaTeX2e}
%    \begin{macrocode}
\expandafter
\let\csname HoLogoBkm@LaTeX2e\endcsname\HoLogoBkm@LaTeXe
%    \end{macrocode}
%    \end{macro}
%    \begin{macro}{\HoLogoHtml@LaTeX2e}
%    \begin{macrocode}
\expandafter
\let\csname HoLogoHtml@LaTeX2e\endcsname\HoLogoHtml@LaTeXe
%    \end{macrocode}
%    \end{macro}
%
% \subsubsection{\hologo{LaTeX3}}
%
%    \begin{macro}{\HoLogo@LaTeX3}
%    Source: \hologo{LaTeX} kernel
%    \begin{macrocode}
\expandafter\def\csname HoLogo@LaTeX3\endcsname#1{%
  \hologo{LaTeX}%
  3%
}
%    \end{macrocode}
%    \end{macro}
%
%    \begin{macro}{\HoLogoBkm@LaTeX3}
%    \begin{macrocode}
\expandafter\def\csname HoLogoBkm@LaTeX3\endcsname#1{%
  \hologo{LaTeX}%
  3%
}
%    \end{macrocode}
%    \end{macro}
%    \begin{macro}{\HoLogoHtml@LaTeX3}
%    \begin{macrocode}
\expandafter
\let\csname HoLogoHtml@LaTeX3\expandafter\endcsname
\csname HoLogo@LaTeX3\endcsname
%    \end{macrocode}
%    \end{macro}
%
% \subsubsection{\hologo{LaTeXML}}
%
%    \begin{macro}{\HoLogo@LaTeXML}
%    \begin{macrocode}
\def\HoLogo@LaTeXML#1{%
  \HOLOGO@mbox{%
    \hologo{La}%
    \kern-.15em%
    T%
    \kern-.1667em%
    \lower.5ex\hbox{E}%
    \kern-.125em%
    \HoLogoFont@font{LaTeXML}{sc}{xml}%
  }%
}
%    \end{macrocode}
%    \end{macro}
%    \begin{macro}{\HoLogoHtml@pdfLaTeX}
%    \begin{macrocode}
\def\HoLogoHtml@LaTeXML#1{%
  \HOLOGO@Span{LaTeXML}{%
    \HoLogoCss@LaTeX
    \HoLogoCss@TeX
    \HOLOGO@Span{LaTeX}{%
      L%
      \HOLOGO@Span{a}{%
        A%
      }%
    }%
    \HOLOGO@Span{TeX}{%
      T%
      \HOLOGO@Span{e}{%
        E%
      }%
    }%
    \HCode{<span style="font-variant: small-caps;">}%
    xml%
    \HCode{</span>}%
  }%
}
%    \end{macrocode}
%    \end{macro}
%
% \subsubsection{\hologo{eTeX}}
%
%    \begin{macro}{\HoLogo@eTeX}
%    Source: package \xpackage{etex}
%    \begin{macrocode}
\def\HoLogo@eTeX#1{%
  \ltx@mbox{%
    \HOLOGO@MathSetup
    $\varepsilon$%
    -%
    \HOLOGO@NegativeKerning{-T,T-,To}%
    \hologo{TeX}%
  }%
}
%    \end{macrocode}
%    \end{macro}
%    \begin{macro}{\HoLogoCs@eTeX}
%    \begin{macrocode}
\ifnum64=`\^^^^0040\relax % test for big chars of LuaTeX/XeTeX
  \catcode`\$=9 %
  \catcode`\&=14 %
\else
  \catcode`\$=14 %
  \catcode`\&=9 %
\fi
\def\HoLogoCs@eTeX#1{%
$ #1{\string ^^^^0395}{\string ^^^^03b5}%
& #1{e}{E}%
  TeX%
}%
\catcode`\$=3 %
\catcode`\&=4 %
%    \end{macrocode}
%    \end{macro}
%    \begin{macro}{\HoLogoBkm@eTeX}
%    \begin{macrocode}
\def\HoLogoBkm@eTeX#1{%
  \HOLOGO@PdfdocUnicode{#1{e}{E}}{\textepsilon}%
  -%
  \hologo{TeX}%
}
%    \end{macrocode}
%    \end{macro}
%    \begin{macro}{\HoLogoHtml@eTeX}
%    \begin{macrocode}
\def\HoLogoHtml@eTeX#1{%
  \ltx@mbox{%
    \HOLOGO@MathSetup
    $\varepsilon$%
    -%
    \hologo{TeX}%
  }%
}
%    \end{macrocode}
%    \end{macro}
%
% \subsubsection{\hologo{iniTeX}}
%
%    \begin{macro}{\HoLogo@iniTeX}
%    \begin{macrocode}
\def\HoLogo@iniTeX#1{%
  \HOLOGO@mbox{%
    #1{i}{I}ni\hologo{TeX}%
  }%
}
%    \end{macrocode}
%    \end{macro}
%    \begin{macro}{\HoLogoCs@iniTeX}
%    \begin{macrocode}
\def\HoLogoCs@iniTeX#1{#1{i}{I}niTeX}
%    \end{macrocode}
%    \end{macro}
%    \begin{macro}{\HoLogoBkm@iniTeX}
%    \begin{macrocode}
\def\HoLogoBkm@iniTeX#1{%
  #1{i}{I}ni\hologo{TeX}%
}
%    \end{macrocode}
%    \end{macro}
%    \begin{macro}{\HoLogoHtml@iniTeX}
%    \begin{macrocode}
\let\HoLogoHtml@iniTeX\HoLogo@iniTeX
%    \end{macrocode}
%    \end{macro}
%
% \subsubsection{\hologo{virTeX}}
%
%    \begin{macro}{\HoLogo@virTeX}
%    \begin{macrocode}
\def\HoLogo@virTeX#1{%
  \HOLOGO@mbox{%
    #1{v}{V}ir\hologo{TeX}%
  }%
}
%    \end{macrocode}
%    \end{macro}
%    \begin{macro}{\HoLogoCs@virTeX}
%    \begin{macrocode}
\def\HoLogoCs@virTeX#1{#1{v}{V}irTeX}
%    \end{macrocode}
%    \end{macro}
%    \begin{macro}{\HoLogoBkm@virTeX}
%    \begin{macrocode}
\def\HoLogoBkm@virTeX#1{%
  #1{v}{V}ir\hologo{TeX}%
}
%    \end{macrocode}
%    \end{macro}
%    \begin{macro}{\HoLogoHtml@virTeX}
%    \begin{macrocode}
\let\HoLogoHtml@virTeX\HoLogo@virTeX
%    \end{macrocode}
%    \end{macro}
%
% \subsubsection{\hologo{SliTeX}}
%
% \paragraph{Definitions of the three variants.}
%
%    \begin{macro}{\HoLogo@SLiTeX@lift}
%    \begin{macrocode}
\def\HoLogo@SLiTeX@lift#1{%
  \HoLogoFont@font{SliTeX}{rm}{%
    S%
    \kern-.06em%
    L%
    \kern-.18em%
    \raise.32ex\hbox{\HoLogoFont@font{SliTeX}{sc}{i}}%
    \HOLOGO@discretionary
    \kern-.06em%
    \hologo{TeX}%
  }%
}
%    \end{macrocode}
%    \end{macro}
%    \begin{macro}{\HoLogoBkm@SLiTeX@lift}
%    \begin{macrocode}
\def\HoLogoBkm@SLiTeX@lift#1{SLiTeX}
%    \end{macrocode}
%    \end{macro}
%    \begin{macro}{\HoLogoHtml@SLiTeX@lift}
%    \begin{macrocode}
\def\HoLogoHtml@SLiTeX@lift#1{%
  \HoLogoCss@SLiTeX@lift
  \HOLOGO@Span{SLiTeX-lift}{%
    \HoLogoFont@font{SliTeX}{rm}{%
      S%
      \HOLOGO@Span{L}{L}%
      \HOLOGO@Span{i}{i}%
      \hologo{TeX}%
    }%
  }%
}
%    \end{macrocode}
%    \end{macro}
%    \begin{macro}{\HoLogoCss@SLiTeX@lift}
%    \begin{macrocode}
\def\HoLogoCss@SLiTeX@lift{%
  \Css{%
    span.HoLogo-SLiTeX-lift span.HoLogo-L{%
      margin-left:-.06em;%
      margin-right:-.18em;%
    }%
  }%
  \Css{%
    span.HoLogo-SLiTeX-lift span.HoLogo-i{%
      position:relative;%
      top:-.32ex;%
      margin-right:-.06em;%
      font-variant:small-caps;%
    }%
  }%
  \global\let\HoLogoCss@SLiTeX@lift\relax
}
%    \end{macrocode}
%    \end{macro}
%
%    \begin{macro}{\HoLogo@SliTeX@simple}
%    \begin{macrocode}
\def\HoLogo@SliTeX@simple#1{%
  \HoLogoFont@font{SliTeX}{rm}{%
    \ltx@mbox{%
      \HoLogoFont@font{SliTeX}{sc}{Sli}%
    }%
    \HOLOGO@discretionary
    \hologo{TeX}%
  }%
}
%    \end{macrocode}
%    \end{macro}
%    \begin{macro}{\HoLogoBkm@SliTeX@simple}
%    \begin{macrocode}
\def\HoLogoBkm@SliTeX@simple#1{SliTeX}
%    \end{macrocode}
%    \end{macro}
%    \begin{macro}{\HoLogoHtml@SliTeX@simple}
%    \begin{macrocode}
\let\HoLogoHtml@SliTeX@simple\HoLogo@SliTeX@simple
%    \end{macrocode}
%    \end{macro}
%
%    \begin{macro}{\HoLogo@SliTeX@narrow}
%    \begin{macrocode}
\def\HoLogo@SliTeX@narrow#1{%
  \HoLogoFont@font{SliTeX}{rm}{%
    \ltx@mbox{%
      S%
      \kern-.06em%
      \HoLogoFont@font{SliTeX}{sc}{%
        l%
        \kern-.035em%
        i%
      }%
    }%
    \HOLOGO@discretionary
    \kern-.06em%
    \hologo{TeX}%
  }%
}
%    \end{macrocode}
%    \end{macro}
%    \begin{macro}{\HoLogoBkm@SliTeX@narrow}
%    \begin{macrocode}
\def\HoLogoBkm@SliTeX@narrow#1{SliTeX}
%    \end{macrocode}
%    \end{macro}
%    \begin{macro}{\HoLogoHtml@SliTeX@narrow}
%    \begin{macrocode}
\def\HoLogoHtml@SliTeX@narrow#1{%
  \HoLogoCss@SliTeX@narrow
  \HOLOGO@Span{SliTeX-narrow}{%
    \HoLogoFont@font{SliTeX}{rm}{%
      S%
        \HOLOGO@Span{l}{l}%
        \HOLOGO@Span{i}{i}%
      \hologo{TeX}%
    }%
  }%
}
%    \end{macrocode}
%    \end{macro}
%    \begin{macro}{\HoLogoCss@SliTeX@narrow}
%    \begin{macrocode}
\def\HoLogoCss@SliTeX@narrow{%
  \Css{%
    span.HoLogo-SliTeX-narrow span.HoLogo-l{%
      margin-left:-.06em;%
      margin-right:-.035em;%
      font-variant:small-caps;%
    }%
  }%
  \Css{%
    span.HoLogo-SliTeX-narrow span.HoLogo-i{%
      margin-right:-.06em;%
      font-variant:small-caps;%
    }%
  }%
  \global\let\HoLogoCss@SliTeX@narrow\relax
}
%    \end{macrocode}
%    \end{macro}
%
% \paragraph{Macro set completion.}
%
%    \begin{macro}{\HoLogo@SLiTeX@simple}
%    \begin{macrocode}
\def\HoLogo@SLiTeX@simple{\HoLogo@SliTeX@simple}
%    \end{macrocode}
%    \end{macro}
%    \begin{macro}{\HoLogoBkm@SLiTeX@simple}
%    \begin{macrocode}
\def\HoLogoBkm@SLiTeX@simple{\HoLogoBkm@SliTeX@simple}
%    \end{macrocode}
%    \end{macro}
%    \begin{macro}{\HoLogoHtml@SLiTeX@simple}
%    \begin{macrocode}
\def\HoLogoHtml@SLiTeX@simple{\HoLogoHtml@SliTeX@simple}
%    \end{macrocode}
%    \end{macro}
%
%    \begin{macro}{\HoLogo@SLiTeX@narrow}
%    \begin{macrocode}
\def\HoLogo@SLiTeX@narrow{\HoLogo@SliTeX@narrow}
%    \end{macrocode}
%    \end{macro}
%    \begin{macro}{\HoLogoBkm@SLiTeX@narrow}
%    \begin{macrocode}
\def\HoLogoBkm@SLiTeX@narrow{\HoLogoBkm@SliTeX@narrow}
%    \end{macrocode}
%    \end{macro}
%    \begin{macro}{\HoLogoHtml@SLiTeX@narrow}
%    \begin{macrocode}
\def\HoLogoHtml@SLiTeX@narrow{\HoLogoHtml@SliTeX@narrow}
%    \end{macrocode}
%    \end{macro}
%
%    \begin{macro}{\HoLogo@SliTeX@lift}
%    \begin{macrocode}
\def\HoLogo@SliTeX@lift{\HoLogo@SLiTeX@lift}
%    \end{macrocode}
%    \end{macro}
%    \begin{macro}{\HoLogoBkm@SliTeX@lift}
%    \begin{macrocode}
\def\HoLogoBkm@SliTeX@lift{\HoLogoBkm@SLiTeX@lift}
%    \end{macrocode}
%    \end{macro}
%    \begin{macro}{\HoLogoHtml@SliTeX@lift}
%    \begin{macrocode}
\def\HoLogoHtml@SliTeX@lift{\HoLogoHtml@SLiTeX@lift}
%    \end{macrocode}
%    \end{macro}
%
% \paragraph{Defaults.}
%
%    \begin{macro}{\HoLogo@SLiTeX}
%    \begin{macrocode}
\def\HoLogo@SLiTeX{\HoLogo@SLiTeX@lift}
%    \end{macrocode}
%    \end{macro}
%    \begin{macro}{\HoLogoBkm@SLiTeX}
%    \begin{macrocode}
\def\HoLogoBkm@SLiTeX{\HoLogoBkm@SLiTeX@lift}
%    \end{macrocode}
%    \end{macro}
%    \begin{macro}{\HoLogoHtml@SLiTeX}
%    \begin{macrocode}
\def\HoLogoHtml@SLiTeX{\HoLogoHtml@SLiTeX@lift}
%    \end{macrocode}
%    \end{macro}
%
%    \begin{macro}{\HoLogo@SliTeX}
%    \begin{macrocode}
\def\HoLogo@SliTeX{\HoLogo@SliTeX@narrow}
%    \end{macrocode}
%    \end{macro}
%    \begin{macro}{\HoLogoBkm@SliTeX}
%    \begin{macrocode}
\def\HoLogoBkm@SliTeX{\HoLogoBkm@SliTeX@narrow}
%    \end{macrocode}
%    \end{macro}
%    \begin{macro}{\HoLogoHtml@SliTeX}
%    \begin{macrocode}
\def\HoLogoHtml@SliTeX{\HoLogoHtml@SliTeX@narrow}
%    \end{macrocode}
%    \end{macro}
%
% \subsubsection{\hologo{LuaTeX}}
%
%    \begin{macro}{\HoLogo@LuaTeX}
%    The kerning is an idea of Hans Hagen, see mailing list
%    `luatex at tug dot org' in March 2010.
%    \begin{macrocode}
\def\HoLogo@LuaTeX#1{%
  \HOLOGO@mbox{%
    Lua%
    \HOLOGO@NegativeKerning{aT,oT,To}%
    \hologo{TeX}%
  }%
}
%    \end{macrocode}
%    \end{macro}
%    \begin{macro}{\HoLogoHtml@LuaTeX}
%    \begin{macrocode}
\let\HoLogoHtml@LuaTeX\HoLogo@LuaTeX
%    \end{macrocode}
%    \end{macro}
%
% \subsubsection{\hologo{LuaLaTeX}}
%
%    \begin{macro}{\HoLogo@LuaLaTeX}
%    \begin{macrocode}
\def\HoLogo@LuaLaTeX#1{%
  \HOLOGO@mbox{%
    Lua%
    \hologo{LaTeX}%
  }%
}
%    \end{macrocode}
%    \end{macro}
%    \begin{macro}{\HoLogoHtml@LuaLaTeX}
%    \begin{macrocode}
\let\HoLogoHtml@LuaLaTeX\HoLogo@LuaLaTeX
%    \end{macrocode}
%    \end{macro}
%
% \subsubsection{\hologo{XeTeX}, \hologo{XeLaTeX}}
%
%    \begin{macro}{\HOLOGO@IfCharExists}
%    \begin{macrocode}
\ifluatex
  \ifnum\luatexversion<36 %
  \else
    \def\HOLOGO@IfCharExists#1{%
      \ifnum
        \directlua{%
           if luaotfload and luaotfload.aux then
             if luaotfload.aux.font_has_glyph(%
                    font.current(), \number#1) then % 	 
	       tex.print("1") % 	 
	     end % 	 
	   elseif font and font.fonts and font.current then %
            local f = font.fonts[font.current()]%
            if f.characters and f.characters[\number#1] then %
              tex.print("1")%
            end %
          end%
        }0=\ltx@zero
        \expandafter\ltx@secondoftwo
      \else
        \expandafter\ltx@firstoftwo
      \fi
    }%
  \fi
\fi
\ltx@IfUndefined{HOLOGO@IfCharExists}{%
  \def\HOLOGO@@IfCharExists#1{%
    \begingroup
      \tracinglostchars=\ltx@zero
      \setbox\ltx@zero=\hbox{%
        \kern7sp\char#1\relax
        \ifnum\lastkern>\ltx@zero
          \expandafter\aftergroup\csname iffalse\endcsname
        \else
          \expandafter\aftergroup\csname iftrue\endcsname
        \fi
      }%
      % \if{true|false} from \aftergroup
      \endgroup
      \expandafter\ltx@firstoftwo
    \else
      \endgroup
      \expandafter\ltx@secondoftwo
    \fi
  }%
  \ifxetex
    \ltx@IfUndefined{XeTeXfonttype}{}{%
      \ltx@IfUndefined{XeTeXcharglyph}{}{%
        \def\HOLOGO@IfCharExists#1{%
          \ifnum\XeTeXfonttype\font>\ltx@zero
            \expandafter\ltx@firstofthree
          \else
            \expandafter\ltx@gobble
          \fi
          {%
            \ifnum\XeTeXcharglyph#1>\ltx@zero
              \expandafter\ltx@firstoftwo
            \else
              \expandafter\ltx@secondoftwo
            \fi
          }%
          \HOLOGO@@IfCharExists{#1}%
        }%
      }%
    }%
  \fi
}{}
\ltx@ifundefined{HOLOGO@IfCharExists}{%
  \ifnum64=`\^^^^0040\relax % test for big chars of LuaTeX/XeTeX
    \let\HOLOGO@IfCharExists\HOLOGO@@IfCharExists
  \else
    \def\HOLOGO@IfCharExists#1{%
      \ifnum#1>255 %
        \expandafter\ltx@fourthoffour
      \fi
      \HOLOGO@@IfCharExists{#1}%
    }%
  \fi
}{}
%    \end{macrocode}
%    \end{macro}
%
%    \begin{macro}{\HoLogo@Xe}
%    Source: package \xpackage{dtklogos}
%    \begin{macrocode}
\def\HoLogo@Xe#1{%
  X%
  \kern-.1em\relax
  \HOLOGO@IfCharExists{"018E}{%
    \lower.5ex\hbox{\char"018E}%
  }{%
    \chardef\HOLOGO@choice=\ltx@zero
    \ifdim\fontdimen\ltx@one\font>0pt %
      \ltx@IfUndefined{rotatebox}{%
        \ltx@IfUndefined{pgftext}{%
          \ltx@IfUndefined{psscalebox}{%
            \ltx@IfUndefined{HOLOGO@ScaleBox@\hologoDriver}{%
            }{%
              \chardef\HOLOGO@choice=4 %
            }%
          }{%
            \chardef\HOLOGO@choice=3 %
          }%
        }{%
          \chardef\HOLOGO@choice=2 %
        }%
      }{%
        \chardef\HOLOGO@choice=1 %
      }%
      \ifcase\HOLOGO@choice
        \HOLOGO@WarningUnsupportedDriver{Xe}%
        e%
      \or % 1: \rotatebox
        \begingroup
          \setbox\ltx@zero\hbox{\rotatebox{180}{E}}%
          \ltx@LocDimenA=\dp\ltx@zero
          \advance\ltx@LocDimenA by -.5ex\relax
          \raise\ltx@LocDimenA\box\ltx@zero
        \endgroup
      \or % 2: \pgftext
        \lower.5ex\hbox{%
          \pgfpicture
            \pgftext[rotate=180]{E}%
          \endpgfpicture
        }%
      \or % 3: \psscalebox
        \begingroup
          \setbox\ltx@zero\hbox{\psscalebox{-1 -1}{E}}%
          \ltx@LocDimenA=\dp\ltx@zero
          \advance\ltx@LocDimenA by -.5ex\relax
          \raise\ltx@LocDimenA\box\ltx@zero
        \endgroup
      \or % 4: \HOLOGO@PointReflectBox
        \lower.5ex\hbox{\HOLOGO@PointReflectBox{E}}%
      \else
        \@PackageError{hologo}{Internal error (choice/it}\@ehc
      \fi
    \else
      \ltx@IfUndefined{reflectbox}{%
        \ltx@IfUndefined{pgftext}{%
          \ltx@IfUndefined{psscalebox}{%
            \ltx@IfUndefined{HOLOGO@ScaleBox@\hologoDriver}{%
            }{%
              \chardef\HOLOGO@choice=4 %
            }%
          }{%
            \chardef\HOLOGO@choice=3 %
          }%
        }{%
          \chardef\HOLOGO@choice=2 %
        }%
      }{%
        \chardef\HOLOGO@choice=1 %
      }%
      \ifcase\HOLOGO@choice
        \HOLOGO@WarningUnsupportedDriver{Xe}%
        e%
      \or % 1: reflectbox
        \lower.5ex\hbox{%
          \reflectbox{E}%
        }%
      \or % 2: \pgftext
        \lower.5ex\hbox{%
          \pgfpicture
            \pgftransformxscale{-1}%
            \pgftext{E}%
          \endpgfpicture
        }%
      \or % 3: \psscalebox
        \lower.5ex\hbox{%
          \psscalebox{-1 1}{E}%
        }%
      \or % 4: \HOLOGO@Reflectbox
        \lower.5ex\hbox{%
          \HOLOGO@ReflectBox{E}%
        }%
      \else
        \@PackageError{hologo}{Internal error (choice/up)}\@ehc
      \fi
    \fi
  }%
}
%    \end{macrocode}
%    \end{macro}
%    \begin{macro}{\HoLogoHtml@Xe}
%    \begin{macrocode}
\def\HoLogoHtml@Xe#1{%
  \HoLogoCss@Xe
  \HOLOGO@Span{Xe}{%
    X%
    \HOLOGO@Span{e}{%
      \HCode{&\ltx@hashchar x018e;}%
    }%
  }%
}
%    \end{macrocode}
%    \end{macro}
%    \begin{macro}{\HoLogoCss@Xe}
%    \begin{macrocode}
\def\HoLogoCss@Xe{%
  \Css{%
    span.HoLogo-Xe span.HoLogo-e{%
      position:relative;%
      top:.5ex;%
      left-margin:-.1em;%
    }%
  }%
  \global\let\HoLogoCss@Xe\relax
}
%    \end{macrocode}
%    \end{macro}
%
%    \begin{macro}{\HoLogo@XeTeX}
%    \begin{macrocode}
\def\HoLogo@XeTeX#1{%
  \hologo{Xe}%
  \kern-.15em\relax
  \hologo{TeX}%
}
%    \end{macrocode}
%    \end{macro}
%
%    \begin{macro}{\HoLogoHtml@XeTeX}
%    \begin{macrocode}
\def\HoLogoHtml@XeTeX#1{%
  \HoLogoCss@XeTeX
  \HOLOGO@Span{XeTeX}{%
    \hologo{Xe}%
    \hologo{TeX}%
  }%
}
%    \end{macrocode}
%    \end{macro}
%    \begin{macro}{\HoLogoCss@XeTeX}
%    \begin{macrocode}
\def\HoLogoCss@XeTeX{%
  \Css{%
    span.HoLogo-XeTeX span.HoLogo-TeX{%
      margin-left:-.15em;%
    }%
  }%
  \global\let\HoLogoCss@XeTeX\relax
}
%    \end{macrocode}
%    \end{macro}
%
%    \begin{macro}{\HoLogo@XeLaTeX}
%    \begin{macrocode}
\def\HoLogo@XeLaTeX#1{%
  \hologo{Xe}%
  \kern-.13em%
  \hologo{LaTeX}%
}
%    \end{macrocode}
%    \end{macro}
%    \begin{macro}{\HoLogoHtml@XeLaTeX}
%    \begin{macrocode}
\def\HoLogoHtml@XeLaTeX#1{%
  \HoLogoCss@XeLaTeX
  \HOLOGO@Span{XeLaTeX}{%
    \hologo{Xe}%
    \hologo{LaTeX}%
  }%
}
%    \end{macrocode}
%    \end{macro}
%    \begin{macro}{\HoLogoCss@XeLaTeX}
%    \begin{macrocode}
\def\HoLogoCss@XeLaTeX{%
  \Css{%
    span.HoLogo-XeLaTeX span.HoLogo-Xe{%
      margin-right:-.13em;%
    }%
  }%
  \global\let\HoLogoCss@XeLaTeX\relax
}
%    \end{macrocode}
%    \end{macro}
%
% \subsubsection{\hologo{pdfTeX}, \hologo{pdfLaTeX}}
%
%    \begin{macro}{\HoLogo@pdfTeX}
%    \begin{macrocode}
\def\HoLogo@pdfTeX#1{%
  \HOLOGO@mbox{%
    #1{p}{P}df\hologo{TeX}%
  }%
}
%    \end{macrocode}
%    \end{macro}
%    \begin{macro}{\HoLogoCs@pdfTeX}
%    \begin{macrocode}
\def\HoLogoCs@pdfTeX#1{#1{p}{P}dfTeX}
%    \end{macrocode}
%    \end{macro}
%    \begin{macro}{\HoLogoBkm@pdfTeX}
%    \begin{macrocode}
\def\HoLogoBkm@pdfTeX#1{%
  #1{p}{P}df\hologo{TeX}%
}
%    \end{macrocode}
%    \end{macro}
%    \begin{macro}{\HoLogoHtml@pdfTeX}
%    \begin{macrocode}
\let\HoLogoHtml@pdfTeX\HoLogo@pdfTeX
%    \end{macrocode}
%    \end{macro}
%
%    \begin{macro}{\HoLogo@pdfLaTeX}
%    \begin{macrocode}
\def\HoLogo@pdfLaTeX#1{%
  \HOLOGO@mbox{%
    #1{p}{P}df\hologo{LaTeX}%
  }%
}
%    \end{macrocode}
%    \end{macro}
%    \begin{macro}{\HoLogoCs@pdfLaTeX}
%    \begin{macrocode}
\def\HoLogoCs@pdfLaTeX#1{#1{p}{P}dfLaTeX}
%    \end{macrocode}
%    \end{macro}
%    \begin{macro}{\HoLogoBkm@pdfLaTeX}
%    \begin{macrocode}
\def\HoLogoBkm@pdfLaTeX#1{%
  #1{p}{P}df\hologo{LaTeX}%
}
%    \end{macrocode}
%    \end{macro}
%    \begin{macro}{\HoLogoHtml@pdfLaTeX}
%    \begin{macrocode}
\let\HoLogoHtml@pdfLaTeX\HoLogo@pdfLaTeX
%    \end{macrocode}
%    \end{macro}
%
% \subsubsection{\hologo{VTeX}}
%
%    \begin{macro}{\HoLogo@VTeX}
%    \begin{macrocode}
\def\HoLogo@VTeX#1{%
  \HOLOGO@mbox{%
    V\hologo{TeX}%
  }%
}
%    \end{macrocode}
%    \end{macro}
%    \begin{macro}{\HoLogoHtml@VTeX}
%    \begin{macrocode}
\let\HoLogoHtml@VTeX\HoLogo@VTeX
%    \end{macrocode}
%    \end{macro}
%
% \subsubsection{\hologo{AmS}, \dots}
%
%    Source: class \xclass{amsdtx}
%
%    \begin{macro}{\HoLogo@AmS}
%    \begin{macrocode}
\def\HoLogo@AmS#1{%
  \HoLogoFont@font{AmS}{sy}{%
    A%
    \kern-.1667em%
    \lower.5ex\hbox{M}%
    \kern-.125em%
    S%
  }%
}
%    \end{macrocode}
%    \end{macro}
%    \begin{macro}{\HoLogoBkm@AmS}
%    \begin{macrocode}
\def\HoLogoBkm@AmS#1{AmS}
%    \end{macrocode}
%    \end{macro}
%    \begin{macro}{\HoLogoHtml@AmS}
%    \begin{macrocode}
\def\HoLogoHtml@AmS#1{%
  \HoLogoCss@AmS
%  \HoLogoFont@font{AmS}{sy}{%
    \HOLOGO@Span{AmS}{%
      A%
      \HOLOGO@Span{M}{M}%
      S%
    }%
%   }%
}
%    \end{macrocode}
%    \end{macro}
%    \begin{macro}{\HoLogoCss@AmS}
%    \begin{macrocode}
\def\HoLogoCss@AmS{%
  \Css{%
    span.HoLogo-AmS span.HoLogo-M{%
      position:relative;%
      top:.5ex;%
      margin-left:-.1667em;%
      margin-right:-.125em;%
      text-decoration:none;%
    }%
  }%
  \global\let\HoLogoCss@AmS\relax
}
%    \end{macrocode}
%    \end{macro}
%
%    \begin{macro}{\HoLogo@AmSTeX}
%    \begin{macrocode}
\def\HoLogo@AmSTeX#1{%
  \hologo{AmS}%
  \HOLOGO@hyphen
  \hologo{TeX}%
}
%    \end{macrocode}
%    \end{macro}
%    \begin{macro}{\HoLogoBkm@AmSTeX}
%    \begin{macrocode}
\def\HoLogoBkm@AmSTeX#1{AmS-TeX}%
%    \end{macrocode}
%    \end{macro}
%    \begin{macro}{\HoLogoHtml@AmSTeX}
%    \begin{macrocode}
\let\HoLogoHtml@AmSTeX\HoLogo@AmSTeX
%    \end{macrocode}
%    \end{macro}
%
%    \begin{macro}{\HoLogo@AmSLaTeX}
%    \begin{macrocode}
\def\HoLogo@AmSLaTeX#1{%
  \hologo{AmS}%
  \HOLOGO@hyphen
  \hologo{LaTeX}%
}
%    \end{macrocode}
%    \end{macro}
%    \begin{macro}{\HoLogoBkm@AmSLaTeX}
%    \begin{macrocode}
\def\HoLogoBkm@AmSLaTeX#1{AmS-LaTeX}%
%    \end{macrocode}
%    \end{macro}
%    \begin{macro}{\HoLogoHtml@AmSLaTeX}
%    \begin{macrocode}
\let\HoLogoHtml@AmSLaTeX\HoLogo@AmSLaTeX
%    \end{macrocode}
%    \end{macro}
%
% \subsubsection{\hologo{BibTeX}}
%
%    \begin{macro}{\HoLogo@BibTeX@sc}
%    A definition of \hologo{BibTeX} is provided in
%    the documentation source for the manual of \hologo{BibTeX}
%    \cite{btxdoc}.
%\begin{quote}
%\begin{verbatim}
%\def\BibTeX{%
%  {%
%    \rm
%    B%
%    \kern-.05em%
%    {%
%      \sc
%      i%
%      \kern-.025em %
%      b%
%    }%
%    \kern-.08em
%    T%
%    \kern-.1667em%
%    \lower.7ex\hbox{E}%
%    \kern-.125em%
%    X%
%  }%
%}
%\end{verbatim}
%\end{quote}
%    \begin{macrocode}
\def\HoLogo@BibTeX@sc#1{%
  B%
  \kern-.05em%
  \HoLogoFont@font{BibTeX}{sc}{%
    i%
    \kern-.025em%
    b%
  }%
  \HOLOGO@discretionary
  \kern-.08em%
  \hologo{TeX}%
}
%    \end{macrocode}
%    \end{macro}
%    \begin{macro}{\HoLogoHtml@BibTeX@sc}
%    \begin{macrocode}
\def\HoLogoHtml@BibTeX@sc#1{%
  \HoLogoCss@BibTeX@sc
  \HOLOGO@Span{BibTeX-sc}{%
    B%
    \HOLOGO@Span{i}{i}%
    \HOLOGO@Span{b}{b}%
    \hologo{TeX}%
  }%
}
%    \end{macrocode}
%    \end{macro}
%    \begin{macro}{\HoLogoCss@BibTeX@sc}
%    \begin{macrocode}
\def\HoLogoCss@BibTeX@sc{%
  \Css{%
    span.HoLogo-BibTeX-sc span.HoLogo-i{%
      margin-left:-.05em;%
      margin-right:-.025em;%
      font-variant:small-caps;%
    }%
  }%
  \Css{%
    span.HoLogo-BibTeX-sc span.HoLogo-b{%
      margin-right:-.08em;%
      font-variant:small-caps;%
    }%
  }%
  \global\let\HoLogoCss@BibTeX@sc\relax
}
%    \end{macrocode}
%    \end{macro}
%
%    \begin{macro}{\HoLogo@BibTeX@sf}
%    Variant \xoption{sf} avoids trouble with unavailable
%    small caps fonts (e.g., bold versions of Computer Modern or
%    Latin Modern). The definition is taken from
%    package \xpackage{dtklogos} \cite{dtklogos}.
%\begin{quote}
%\begin{verbatim}
%\DeclareRobustCommand{\BibTeX}{%
%  B%
%  \kern-.05em%
%  \hbox{%
%    $\m@th$% %% force math size calculations
%    \csname S@\f@size\endcsname
%    \fontsize\sf@size\z@
%    \math@fontsfalse
%    \selectfont
%    I%
%    \kern-.025em%
%    B
%  }%
%  \kern-.08em%
%  \-%
%  \TeX
%}
%\end{verbatim}
%\end{quote}
%    \begin{macrocode}
\def\HoLogo@BibTeX@sf#1{%
  B%
  \kern-.05em%
  \HoLogoFont@font{BibTeX}{bibsf}{%
    I%
    \kern-.025em%
    B%
  }%
  \HOLOGO@discretionary
  \kern-.08em%
  \hologo{TeX}%
}
%    \end{macrocode}
%    \end{macro}
%    \begin{macro}{\HoLogoHtml@BibTeX@sf}
%    \begin{macrocode}
\def\HoLogoHtml@BibTeX@sf#1{%
  \HoLogoCss@BibTeX@sf
  \HOLOGO@Span{BibTeX-sf}{%
    B%
    \HoLogoFont@font{BibTeX}{bibsf}{%
      \HOLOGO@Span{i}{I}%
      B%
    }%
    \hologo{TeX}%
  }%
}
%    \end{macrocode}
%    \end{macro}
%    \begin{macro}{\HoLogoCss@BibTeX@sf}
%    \begin{macrocode}
\def\HoLogoCss@BibTeX@sf{%
  \Css{%
    span.HoLogo-BibTeX-sf span.HoLogo-i{%
      margin-left:-.05em;%
      margin-right:-.025em;%
    }%
  }%
  \Css{%
    span.HoLogo-BibTeX-sf span.HoLogo-TeX{%
      margin-left:-.08em;%
    }%
  }%
  \global\let\HoLogoCss@BibTeX@sf\relax
}
%    \end{macrocode}
%    \end{macro}
%
%    \begin{macro}{\HoLogo@BibTeX}
%    \begin{macrocode}
\def\HoLogo@BibTeX{\HoLogo@BibTeX@sf}
%    \end{macrocode}
%    \end{macro}
%    \begin{macro}{\HoLogoHtml@BibTeX}
%    \begin{macrocode}
\def\HoLogoHtml@BibTeX{\HoLogoHtml@BibTeX@sf}
%    \end{macrocode}
%    \end{macro}
%
% \subsubsection{\hologo{BibTeX8}}
%
%    \begin{macro}{\HoLogo@BibTeX8}
%    \begin{macrocode}
\expandafter\def\csname HoLogo@BibTeX8\endcsname#1{%
  \hologo{BibTeX}%
  8%
}
%    \end{macrocode}
%    \end{macro}
%
%    \begin{macro}{\HoLogoBkm@BibTeX8}
%    \begin{macrocode}
\expandafter\def\csname HoLogoBkm@BibTeX8\endcsname#1{%
  \hologo{BibTeX}%
  8%
}
%    \end{macrocode}
%    \end{macro}
%    \begin{macro}{\HoLogoHtml@BibTeX8}
%    \begin{macrocode}
\expandafter
\let\csname HoLogoHtml@BibTeX8\expandafter\endcsname
\csname HoLogo@BibTeX8\endcsname
%    \end{macrocode}
%    \end{macro}
%
% \subsubsection{\hologo{ConTeXt}}
%
%    \begin{macro}{\HoLogo@ConTeXt@simple}
%    \begin{macrocode}
\def\HoLogo@ConTeXt@simple#1{%
  \HOLOGO@mbox{Con}%
  \HOLOGO@discretionary
  \HOLOGO@mbox{\hologo{TeX}t}%
}
%    \end{macrocode}
%    \end{macro}
%    \begin{macro}{\HoLogoHtml@ConTeXt@simple}
%    \begin{macrocode}
\let\HoLogoHtml@ConTeXt@simple\HoLogo@ConTeXt@simple
%    \end{macrocode}
%    \end{macro}
%
%    \begin{macro}{\HoLogo@ConTeXt@narrow}
%    This definition of logo \hologo{ConTeXt} with variant \xoption{narrow}
%    comes from TUGboat's class \xclass{ltugboat} (version 2010/11/15 v2.8).
%    \begin{macrocode}
\def\HoLogo@ConTeXt@narrow#1{%
  \HOLOGO@mbox{C\kern-.0333emon}%
  \HOLOGO@discretionary
  \kern-.0667em%
  \HOLOGO@mbox{\hologo{TeX}\kern-.0333emt}%
}
%    \end{macrocode}
%    \end{macro}
%    \begin{macro}{\HoLogoHtml@ConTeXt@narrow}
%    \begin{macrocode}
\def\HoLogoHtml@ConTeXt@narrow#1{%
  \HoLogoCss@ConTeXt@narrow
  \HOLOGO@Span{ConTeXt-narrow}{%
    \HOLOGO@Span{C}{C}%
    on%
    \hologo{TeX}%
    t%
  }%
}
%    \end{macrocode}
%    \end{macro}
%    \begin{macro}{\HoLogoCss@ConTeXt@narrow}
%    \begin{macrocode}
\def\HoLogoCss@ConTeXt@narrow{%
  \Css{%
    span.HoLogo-ConTeXt-narrow span.HoLogo-C{%
      margin-left:-.0333em;%
    }%
  }%
  \Css{%
    span.HoLogo-ConTeXt-narrow span.HoLogo-TeX{%
      margin-left:-.0667em;%
      margin-right:-.0333em;%
    }%
  }%
  \global\let\HoLogoCss@ConTeXt@narrow\relax
}
%    \end{macrocode}
%    \end{macro}
%
%    \begin{macro}{\HoLogo@ConTeXt}
%    \begin{macrocode}
\def\HoLogo@ConTeXt{\HoLogo@ConTeXt@narrow}
%    \end{macrocode}
%    \end{macro}
%    \begin{macro}{\HoLogoHtml@ConTeXt}
%    \begin{macrocode}
\def\HoLogoHtml@ConTeXt{\HoLogoHtml@ConTeXt@narrow}
%    \end{macrocode}
%    \end{macro}
%
% \subsubsection{\hologo{emTeX}}
%
%    \begin{macro}{\HoLogo@emTeX}
%    \begin{macrocode}
\def\HoLogo@emTeX#1{%
  \HOLOGO@mbox{#1{e}{E}m}%
  \HOLOGO@discretionary
  \hologo{TeX}%
}
%    \end{macrocode}
%    \end{macro}
%    \begin{macro}{\HoLogoCs@emTeX}
%    \begin{macrocode}
\def\HoLogoCs@emTeX#1{#1{e}{E}mTeX}%
%    \end{macrocode}
%    \end{macro}
%    \begin{macro}{\HoLogoBkm@emTeX}
%    \begin{macrocode}
\def\HoLogoBkm@emTeX#1{%
  #1{e}{E}m\hologo{TeX}%
}
%    \end{macrocode}
%    \end{macro}
%    \begin{macro}{\HoLogoHtml@emTeX}
%    \begin{macrocode}
\let\HoLogoHtml@emTeX\HoLogo@emTeX
%    \end{macrocode}
%    \end{macro}
%
% \subsubsection{\hologo{ExTeX}}
%
%    \begin{macro}{\HoLogo@ExTeX}
%    The definition is taken from the FAQ of the
%    project \hologo{ExTeX}
%    \cite{ExTeX-FAQ}.
%\begin{quote}
%\begin{verbatim}
%\def\ExTeX{%
%  \textrm{% Logo always with serifs
%    \ensuremath{%
%      \textstyle
%      \varepsilon_{%
%        \kern-0.15em%
%        \mathcal{X}%
%      }%
%    }%
%    \kern-.15em%
%    \TeX
%  }%
%}
%\end{verbatim}
%\end{quote}
%    \begin{macrocode}
\def\HoLogo@ExTeX#1{%
  \HoLogoFont@font{ExTeX}{rm}{%
    \ltx@mbox{%
      \HOLOGO@MathSetup
      $%
        \textstyle
        \varepsilon_{%
          \kern-0.15em%
          \HoLogoFont@font{ExTeX}{sy}{X}%
        }%
      $%
    }%
    \HOLOGO@discretionary
    \kern-.15em%
    \hologo{TeX}%
  }%
}
%    \end{macrocode}
%    \end{macro}
%    \begin{macro}{\HoLogoHtml@ExTeX}
%    \begin{macrocode}
\def\HoLogoHtml@ExTeX#1{%
  \HoLogoCss@ExTeX
  \HoLogoFont@font{ExTeX}{rm}{%
    \HOLOGO@Span{ExTeX}{%
      \ltx@mbox{%
        \HOLOGO@MathSetup
        $\textstyle\varepsilon$%
        \HOLOGO@Span{X}{$\textstyle\chi$}%
        \hologo{TeX}%
      }%
    }%
  }%
}
%    \end{macrocode}
%    \end{macro}
%    \begin{macro}{\HoLogoBkm@ExTeX}
%    \begin{macrocode}
\def\HoLogoBkm@ExTeX#1{%
  \HOLOGO@PdfdocUnicode{#1{e}{E}x}{\textepsilon\textchi}%
  \hologo{TeX}%
}
%    \end{macrocode}
%    \end{macro}
%    \begin{macro}{\HoLogoCss@ExTeX}
%    \begin{macrocode}
\def\HoLogoCss@ExTeX{%
  \Css{%
    span.HoLogo-ExTeX{%
      font-family:serif;%
    }%
  }%
  \Css{%
    span.HoLogo-ExTeX span.HoLogo-TeX{%
      margin-left:-.15em;%
    }%
  }%
  \global\let\HoLogoCss@ExTeX\relax
}
%    \end{macrocode}
%    \end{macro}
%
% \subsubsection{\hologo{MiKTeX}}
%
%    \begin{macro}{\HoLogo@MiKTeX}
%    \begin{macrocode}
\def\HoLogo@MiKTeX#1{%
  \HOLOGO@mbox{MiK}%
  \HOLOGO@discretionary
  \hologo{TeX}%
}
%    \end{macrocode}
%    \end{macro}
%    \begin{macro}{\HoLogoHtml@MiKTeX}
%    \begin{macrocode}
\let\HoLogoHtml@MiKTeX\HoLogo@MiKTeX
%    \end{macrocode}
%    \end{macro}
%
% \subsubsection{\hologo{OzTeX} and friends}
%
%    Source: \hologo{OzTeX} FAQ \cite{OzTeX}:
%    \begin{quote}
%      |\def\OzTeX{O\kern-.03em z\kern-.15em\TeX}|\\
%      (There is no kerning in OzMF, OzMP and OzTtH.)
%    \end{quote}
%
%    \begin{macro}{\HoLogo@OzTeX}
%    \begin{macrocode}
\def\HoLogo@OzTeX#1{%
  O%
  \kern-.03em %
  z%
  \kern-.15em %
  \hologo{TeX}%
}
%    \end{macrocode}
%    \end{macro}
%    \begin{macro}{\HoLogoHtml@OzTeX}
%    \begin{macrocode}
\def\HoLogoHtml@OzTeX#1{%
  \HoLogoCss@OzTeX
  \HOLOGO@Span{OzTeX}{%
    O%
    \HOLOGO@Span{z}{z}%
    \hologo{TeX}%
  }%
}
%    \end{macrocode}
%    \end{macro}
%    \begin{macro}{\HoLogoCss@OzTeX}
%    \begin{macrocode}
\def\HoLogoCss@OzTeX{%
  \Css{%
    span.HoLogo-OzTeX span.HoLogo-z{%
      margin-left:-.03em;%
      margin-right:-.15em;%
    }%
  }%
  \global\let\HoLogoCss@OzTeX\relax
}
%    \end{macrocode}
%    \end{macro}
%
%    \begin{macro}{\HoLogo@OzMF}
%    \begin{macrocode}
\def\HoLogo@OzMF#1{%
  \HOLOGO@mbox{OzMF}%
}
%    \end{macrocode}
%    \end{macro}
%    \begin{macro}{\HoLogo@OzMP}
%    \begin{macrocode}
\def\HoLogo@OzMP#1{%
  \HOLOGO@mbox{OzMP}%
}
%    \end{macrocode}
%    \end{macro}
%    \begin{macro}{\HoLogo@OzTtH}
%    \begin{macrocode}
\def\HoLogo@OzTtH#1{%
  \HOLOGO@mbox{OzTtH}%
}
%    \end{macrocode}
%    \end{macro}
%
% \subsubsection{\hologo{PCTeX}}
%
%    \begin{macro}{\HoLogo@PCTeX}
%    \begin{macrocode}
\def\HoLogo@PCTeX#1{%
  \HOLOGO@mbox{PC}%
  \hologo{TeX}%
}
%    \end{macrocode}
%    \end{macro}
%    \begin{macro}{\HoLogoHtml@PCTeX}
%    \begin{macrocode}
\let\HoLogoHtml@PCTeX\HoLogo@PCTeX
%    \end{macrocode}
%    \end{macro}
%
% \subsubsection{\hologo{PiCTeX}}
%
%    The original definitions from \xfile{pictex.tex} \cite{PiCTeX}:
%\begin{quote}
%\begin{verbatim}
%\def\PiC{%
%  P%
%  \kern-.12em%
%  \lower.5ex\hbox{I}%
%  \kern-.075em%
%  C%
%}
%\def\PiCTeX{%
%  \PiC
%  \kern-.11em%
%  \TeX
%}
%\end{verbatim}
%\end{quote}
%
%    \begin{macro}{\HoLogo@PiC}
%    \begin{macrocode}
\def\HoLogo@PiC#1{%
  P%
  \kern-.12em%
  \lower.5ex\hbox{I}%
  \kern-.075em%
  C%
  \HOLOGO@SpaceFactor
}
%    \end{macrocode}
%    \end{macro}
%    \begin{macro}{\HoLogoHtml@PiC}
%    \begin{macrocode}
\def\HoLogoHtml@PiC#1{%
  \HoLogoCss@PiC
  \HOLOGO@Span{PiC}{%
    P%
    \HOLOGO@Span{i}{I}%
    C%
  }%
}
%    \end{macrocode}
%    \end{macro}
%    \begin{macro}{\HoLogoCss@PiC}
%    \begin{macrocode}
\def\HoLogoCss@PiC{%
  \Css{%
    span.HoLogo-PiC span.HoLogo-i{%
      position:relative;%
      top:.5ex;%
      margin-left:-.12em;%
      margin-right:-.075em;%
      text-decoration:none;%
    }%
  }%
  \global\let\HoLogoCss@PiC\relax
}
%    \end{macrocode}
%    \end{macro}
%
%    \begin{macro}{\HoLogo@PiCTeX}
%    \begin{macrocode}
\def\HoLogo@PiCTeX#1{%
  \hologo{PiC}%
  \HOLOGO@discretionary
  \kern-.11em%
  \hologo{TeX}%
}
%    \end{macrocode}
%    \end{macro}
%    \begin{macro}{\HoLogoHtml@PiCTeX}
%    \begin{macrocode}
\def\HoLogoHtml@PiCTeX#1{%
  \HoLogoCss@PiCTeX
  \HOLOGO@Span{PiCTeX}{%
    \hologo{PiC}%
    \hologo{TeX}%
  }%
}
%    \end{macrocode}
%    \end{macro}
%    \begin{macro}{\HoLogoCss@PiCTeX}
%    \begin{macrocode}
\def\HoLogoCss@PiCTeX{%
  \Css{%
    span.HoLogo-PiCTeX span.HoLogo-PiC{%
      margin-right:-.11em;%
    }%
  }%
  \global\let\HoLogoCss@PiCTeX\relax
}
%    \end{macrocode}
%    \end{macro}
%
% \subsubsection{\hologo{teTeX}}
%
%    \begin{macro}{\HoLogo@teTeX}
%    \begin{macrocode}
\def\HoLogo@teTeX#1{%
  \HOLOGO@mbox{#1{t}{T}e}%
  \HOLOGO@discretionary
  \hologo{TeX}%
}
%    \end{macrocode}
%    \end{macro}
%    \begin{macro}{\HoLogoCs@teTeX}
%    \begin{macrocode}
\def\HoLogoCs@teTeX#1{#1{t}{T}dfTeX}
%    \end{macrocode}
%    \end{macro}
%    \begin{macro}{\HoLogoBkm@teTeX}
%    \begin{macrocode}
\def\HoLogoBkm@teTeX#1{%
  #1{t}{T}e\hologo{TeX}%
}
%    \end{macrocode}
%    \end{macro}
%    \begin{macro}{\HoLogoHtml@teTeX}
%    \begin{macrocode}
\let\HoLogoHtml@teTeX\HoLogo@teTeX
%    \end{macrocode}
%    \end{macro}
%
% \subsubsection{\hologo{TeX4ht}}
%
%    \begin{macro}{\HoLogo@TeX4ht}
%    \begin{macrocode}
\expandafter\def\csname HoLogo@TeX4ht\endcsname#1{%
  \HOLOGO@mbox{\hologo{TeX}4ht}%
}
%    \end{macrocode}
%    \end{macro}
%    \begin{macro}{\HoLogoHtml@TeX4ht}
%    \begin{macrocode}
\expandafter
\let\csname HoLogoHtml@TeX4ht\expandafter\endcsname
\csname HoLogo@TeX4ht\endcsname
%    \end{macrocode}
%    \end{macro}
%
%
% \subsubsection{\hologo{SageTeX}}
%
%    \begin{macro}{\HoLogo@SageTeX}
%    \begin{macrocode}
\def\HoLogo@SageTeX#1{%
  \HOLOGO@mbox{Sage}%
  \HOLOGO@discretionary
  \HOLOGO@NegativeKerning{eT,oT,To}%
  \hologo{TeX}%
}
%    \end{macrocode}
%    \end{macro}
%    \begin{macro}{\HoLogoHtml@SageTeX}
%    \begin{macrocode}
\let\HoLogoHtml@SageTeX\HoLogo@SageTeX
%    \end{macrocode}
%    \end{macro}
%
% \subsection{\hologo{METAFONT} and friends}
%
%    \begin{macro}{\HoLogo@METAFONT}
%    \begin{macrocode}
\def\HoLogo@METAFONT#1{%
  \HoLogoFont@font{METAFONT}{logo}{%
    \HOLOGO@mbox{META}%
    \HOLOGO@discretionary
    \HOLOGO@mbox{FONT}%
  }%
}
%    \end{macrocode}
%    \end{macro}
%
%    \begin{macro}{\HoLogo@METAPOST}
%    \begin{macrocode}
\def\HoLogo@METAPOST#1{%
  \HoLogoFont@font{METAPOST}{logo}{%
    \HOLOGO@mbox{META}%
    \HOLOGO@discretionary
    \HOLOGO@mbox{POST}%
  }%
}
%    \end{macrocode}
%    \end{macro}
%
%    \begin{macro}{\HoLogo@MetaFun}
%    \begin{macrocode}
\def\HoLogo@MetaFun#1{%
  \HOLOGO@mbox{Meta}%
  \HOLOGO@discretionary
  \HOLOGO@mbox{Fun}%
}
%    \end{macrocode}
%    \end{macro}
%
%    \begin{macro}{\HoLogo@MetaPost}
%    \begin{macrocode}
\def\HoLogo@MetaPost#1{%
  \HOLOGO@mbox{Meta}%
  \HOLOGO@discretionary
  \HOLOGO@mbox{Post}%
}
%    \end{macrocode}
%    \end{macro}
%
% \subsection{Others}
%
% \subsubsection{\hologo{biber}}
%
%    \begin{macro}{\HoLogo@biber}
%    \begin{macrocode}
\def\HoLogo@biber#1{%
  \HOLOGO@mbox{#1{b}{B}i}%
  \HOLOGO@discretionary
  \HOLOGO@mbox{ber}%
}
%    \end{macrocode}
%    \end{macro}
%    \begin{macro}{\HoLogoCs@biber}
%    \begin{macrocode}
\def\HoLogoCs@biber#1{#1{b}{B}iber}
%    \end{macrocode}
%    \end{macro}
%    \begin{macro}{\HoLogoBkm@biber}
%    \begin{macrocode}
\def\HoLogoBkm@biber#1{%
  #1{b}{B}iber%
}
%    \end{macrocode}
%    \end{macro}
%    \begin{macro}{\HoLogoHtml@biber}
%    \begin{macrocode}
\let\HoLogoHtml@biber\HoLogo@biber
%    \end{macrocode}
%    \end{macro}
%
% \subsubsection{\hologo{KOMAScript}}
%
%    \begin{macro}{\HoLogo@KOMAScript}
%    The definition for \hologo{KOMAScript} is taken
%    from \hologo{KOMAScript} (\xfile{scrlogo.dtx}, reformatted) \cite{scrlogo}:
%\begin{quote}
%\begin{verbatim}
%\@ifundefined{KOMAScript}{%
%  \DeclareRobustCommand{\KOMAScript}{%
%    \textsf{%
%      K\kern.05em O\kern.05emM\kern.05em A%
%      \kern.1em-\kern.1em %
%      Script%
%    }%
%  }%
%}{}
%\end{verbatim}
%\end{quote}
%    \begin{macrocode}
\def\HoLogo@KOMAScript#1{%
  \HoLogoFont@font{KOMAScript}{sf}{%
    \HOLOGO@mbox{%
      K\kern.05em%
      O\kern.05em%
      M\kern.05em%
      A%
    }%
    \kern.1em%
    \HOLOGO@hyphen
    \kern.1em%
    \HOLOGO@mbox{Script}%
  }%
}
%    \end{macrocode}
%    \end{macro}
%    \begin{macro}{\HoLogoBkm@KOMAScript}
%    \begin{macrocode}
\def\HoLogoBkm@KOMAScript#1{%
  KOMA-Script%
}
%    \end{macrocode}
%    \end{macro}
%    \begin{macro}{\HoLogoHtml@KOMAScript}
%    \begin{macrocode}
\def\HoLogoHtml@KOMAScript#1{%
  \HoLogoCss@KOMAScript
  \HoLogoFont@font{KOMAScript}{sf}{%
    \HOLOGO@Span{KOMAScript}{%
      K%
      \HOLOGO@Span{O}{O}%
      M%
      \HOLOGO@Span{A}{A}%
      \HOLOGO@Span{hyphen}{-}%
      Script%
    }%
  }%
}
%    \end{macrocode}
%    \end{macro}
%    \begin{macro}{\HoLogoCss@KOMAScript}
%    \begin{macrocode}
\def\HoLogoCss@KOMAScript{%
  \Css{%
    span.HoLogo-KOMAScript{%
      font-family:sans-serif;%
    }%
  }%
  \Css{%
    span.HoLogo-KOMAScript span.HoLogo-O{%
      padding-left:.05em;%
      padding-right:.05em;%
    }%
  }%
  \Css{%
    span.HoLogo-KOMAScript span.HoLogo-A{%
      padding-left:.05em;%
    }%
  }%
  \Css{%
    span.HoLogo-KOMAScript span.HoLogo-hyphen{%
      padding-left:.1em;%
      padding-right:.1em;%
    }%
  }%
  \global\let\HoLogoCss@KOMAScript\relax
}
%    \end{macrocode}
%    \end{macro}
%
% \subsubsection{\hologo{LyX}}
%
%    \begin{macro}{\HoLogo@LyX}
%    The definition is taken from the documentation source files
%    of \hologo{LyX}, \xfile{Intro.lyx} \cite{LyX}:
%\begin{quote}
%\begin{verbatim}
%\def\LyX{%
%  \texorpdfstring{%
%    L\kern-.1667em\lower.25em\hbox{Y}\kern-.125emX\@%
%  }{%
%    LyX%
%  }%
%}
%\end{verbatim}
%\end{quote}
%    \begin{macrocode}
\def\HoLogo@LyX#1{%
  L%
  \kern-.1667em%
  \lower.25em\hbox{Y}%
  \kern-.125em%
  X%
  \HOLOGO@SpaceFactor
}
%    \end{macrocode}
%    \end{macro}
%    \begin{macro}{\HoLogoHtml@LyX}
%    \begin{macrocode}
\def\HoLogoHtml@LyX#1{%
  \HoLogoCss@LyX
  \HOLOGO@Span{LyX}{%
    L%
    \HOLOGO@Span{y}{Y}%
    X%
  }%
}
%    \end{macrocode}
%    \end{macro}
%    \begin{macro}{\HoLogoCss@LyX}
%    \begin{macrocode}
\def\HoLogoCss@LyX{%
  \Css{%
    span.HoLogo-LyX span.HoLogo-y{%
      position:relative;%
      top:.25em;%
      margin-left:-.1667em;%
      margin-right:-.125em;%
      text-decoration:none;%
    }%
  }%
  \global\let\HoLogoCss@LyX\relax
}
%    \end{macrocode}
%    \end{macro}
%
% \subsubsection{\hologo{NTS}}
%
%    \begin{macro}{\HoLogo@NTS}
%    Definition for \hologo{NTS} can be found in
%    package \xpackage{etex\textunderscore man} for the \hologo{eTeX} manual \cite{etexman}
%    and in package \xpackage{dtklogos} \cite{dtklogos}:
%\begin{quote}
%\begin{verbatim}
%\def\NTS{%
%  \leavevmode
%  \hbox{%
%    $%
%      \cal N%
%      \kern-0.35em%
%      \lower0.5ex\hbox{$\cal T$}%
%      \kern-0.2em%
%      S%
%    $%
%  }%
%}
%\end{verbatim}
%\end{quote}
%    \begin{macrocode}
\def\HoLogo@NTS#1{%
  \HoLogoFont@font{NTS}{sy}{%
    N\/%
    \kern-.35em%
    \lower.5ex\hbox{T\/}%
    \kern-.2em%
    S\/%
  }%
  \HOLOGO@SpaceFactor
}
%    \end{macrocode}
%    \end{macro}
%
% \subsubsection{\Hologo{TTH} (\hologo{TeX} to HTML translator)}
%
%    Source: \url{http://hutchinson.belmont.ma.us/tth/}
%    In the HTML source the second `T' is printed as subscript.
%\begin{quote}
%\begin{verbatim}
%T<sub>T</sub>H
%\end{verbatim}
%\end{quote}
%    \begin{macro}{\HoLogo@TTH}
%    \begin{macrocode}
\def\HoLogo@TTH#1{%
  \ltx@mbox{%
    T\HOLOGO@SubScript{T}H%
  }%
  \HOLOGO@SpaceFactor
}
%    \end{macrocode}
%    \end{macro}
%
%    \begin{macro}{\HoLogoHtml@TTH}
%    \begin{macrocode}
\def\HoLogoHtml@TTH#1{%
  T\HCode{<sub>}T\HCode{</sub>}H%
}
%    \end{macrocode}
%    \end{macro}
%
% \subsubsection{\Hologo{HanTheThanh}}
%
%    Partial source: Package \xpackage{dtklogos}.
%    The double accent is U+1EBF (latin small letter e with circumflex
%    and acute).
%    \begin{macro}{\HoLogo@HanTheThanh}
%    \begin{macrocode}
\def\HoLogo@HanTheThanh#1{%
  \ltx@mbox{H\`an}%
  \HOLOGO@space
  \ltx@mbox{%
    Th%
    \HOLOGO@IfCharExists{"1EBF}{%
      \char"1EBF\relax
    }{%
      \^e\hbox to 0pt{\hss\raise .5ex\hbox{\'{}}}%
    }%
  }%
  \HOLOGO@space
  \ltx@mbox{Th\`anh}%
}
%    \end{macrocode}
%    \end{macro}
%    \begin{macro}{\HoLogoBkm@HanTheThanh}
%    \begin{macrocode}
\def\HoLogoBkm@HanTheThanh#1{%
  H\`an %
  Th\HOLOGO@PdfdocUnicode{\^e}{\9036\277} %
  Th\`anh%
}
%    \end{macrocode}
%    \end{macro}
%    \begin{macro}{\HoLogoHtml@HanTheThanh}
%    \begin{macrocode}
\def\HoLogoHtml@HanTheThanh#1{%
  H\`an %
  Th\HCode{&\ltx@hashchar x1ebf;} %
  Th\`anh%
}
%    \end{macrocode}
%    \end{macro}
%
% \subsection{Driver detection}
%
%    \begin{macrocode}
\HOLOGO@IfExists\InputIfFileExists{%
  \InputIfFileExists{hologo.cfg}{}{}%
}{%
  \ltx@IfUndefined{pdf@filesize}{%
    \def\HOLOGO@InputIfExists{%
      \openin\HOLOGO@temp=hologo.cfg\relax
      \ifeof\HOLOGO@temp
        \closein\HOLOGO@temp
      \else
        \closein\HOLOGO@temp
        \begingroup
          \def\x{LaTeX2e}%
        \expandafter\endgroup
        \ifx\fmtname\x
          \input{hologo.cfg}%
        \else
          \input hologo.cfg\relax
        \fi
      \fi
    }%
    \ltx@IfUndefined{newread}{%
      \chardef\HOLOGO@temp=15 %
      \def\HOLOGO@CheckRead{%
        \ifeof\HOLOGO@temp
          \HOLOGO@InputIfExists
        \else
          \ifcase\HOLOGO@temp
            \@PackageWarningNoLine{hologo}{%
              Configuration file ignored, because\MessageBreak
              a free read register could not be found%
            }%
          \else
            \begingroup
              \count\ltx@cclv=\HOLOGO@temp
              \advance\ltx@cclv by \ltx@minusone
              \edef\x{\endgroup
                \chardef\noexpand\HOLOGO@temp=\the\count\ltx@cclv
                \relax
              }%
            \x
          \fi
        \fi
      }%
    }{%
      \csname newread\endcsname\HOLOGO@temp
      \HOLOGO@InputIfExists
    }%
  }{%
    \edef\HOLOGO@temp{\pdf@filesize{hologo.cfg}}%
    \ifx\HOLOGO@temp\ltx@empty
    \else
      \ifnum\HOLOGO@temp>0 %
        \begingroup
          \def\x{LaTeX2e}%
        \expandafter\endgroup
        \ifx\fmtname\x
          \input{hologo.cfg}%
        \else
          \input hologo.cfg\relax
        \fi
      \else
        \@PackageInfoNoLine{hologo}{%
          Empty configuration file `hologo.cfg' ignored%
        }%
      \fi
    \fi
  }%
}
%    \end{macrocode}
%
%    \begin{macrocode}
\def\HOLOGO@temp#1#2{%
  \kv@define@key{HoLogoDriver}{#1}[]{%
    \begingroup
      \def\HOLOGO@temp{##1}%
      \ltx@onelevel@sanitize\HOLOGO@temp
      \ifx\HOLOGO@temp\ltx@empty
      \else
        \@PackageError{hologo}{%
          Value (\HOLOGO@temp) not permitted for option `#1'%
        }%
        \@ehc
      \fi
    \endgroup
    \def\hologoDriver{#2}%
  }%
}%
\def\HOLOGO@@temp#1#2{%
  \ifx\kv@value\relax
    \HOLOGO@temp{#1}{#1}%
  \else
    \HOLOGO@temp{#1}{#2}%
  \fi
}%
\kv@parse@normalized{%
  pdftex,%
  luatex=pdftex,%
  dvipdfm,%
  dvipdfmx=dvipdfm,%
  dvips,%
  dvipsone=dvips,%
  xdvi=dvips,%
  xetex,%
  vtex,%
}\HOLOGO@@temp
%    \end{macrocode}
%
%    \begin{macrocode}
\kv@define@key{HoLogoDriver}{driverfallback}{%
  \def\HOLOGO@DriverFallback{#1}%
}
%    \end{macrocode}
%
%    \begin{macro}{\HOLOGO@DriverFallback}
%    \begin{macrocode}
\def\HOLOGO@DriverFallback{dvips}
%    \end{macrocode}
%    \end{macro}
%
%    \begin{macro}{\hologoDriverSetup}
%    \begin{macrocode}
\def\hologoDriverSetup{%
  \let\hologoDriver\ltx@undefined
  \HOLOGO@DriverSetup
}
%    \end{macrocode}
%    \end{macro}
%
%    \begin{macro}{\HOLOGO@DriverSetup}
%    \begin{macrocode}
\def\HOLOGO@DriverSetup#1{%
  \kvsetkeys{HoLogoDriver}{#1}%
  \HOLOGO@CheckDriver
  \ltx@ifundefined{hologoDriver}{%
    \begingroup
    \edef\x{\endgroup
      \noexpand\kvsetkeys{HoLogoDriver}{\HOLOGO@DriverFallback}%
    }\x
  }{}%
  \@PackageInfoNoLine{hologo}{Using driver `\hologoDriver'}%
}
%    \end{macrocode}
%    \end{macro}
%
%    \begin{macro}{\HOLOGO@CheckDriver}
%    \begin{macrocode}
\def\HOLOGO@CheckDriver{%
  \ifpdf
    \def\hologoDriver{pdftex}%
    \let\HOLOGO@pdfliteral\pdfliteral
    \ifluatex
      \ifx\pdfextension\@undefined\else
        \protected\def\pdfliteral{\pdfextension literal}%
        \let\HOLOGO@pdfliteral\pdfliteral
      \fi
      \ltx@IfUndefined{HOLOGO@pdfliteral}{%
        \ifnum\luatexversion<36 %
        \else
          \begingroup
            \let\HOLOGO@temp\endgroup
            \ifcase0%
                \directlua{%
                  if tex.enableprimitives then %
                    tex.enableprimitives('HOLOGO@', {'pdfliteral'})%
                  else %
                    tex.print('1')%
                  end%
                }%
                \ifx\HOLOGO@pdfliteral\@undefined 1\fi%
                \relax%
              \endgroup
              \let\HOLOGO@temp\relax
              \global\let\HOLOGO@pdfliteral\HOLOGO@pdfliteral
            \fi%
          \HOLOGO@temp
        \fi
      }{}%
    \fi
    \ltx@IfUndefined{HOLOGO@pdfliteral}{%
      \@PackageWarningNoLine{hologo}{%
        Cannot find \string\pdfliteral
      }%
    }{}%
  \else
    \ifxetex
      \def\hologoDriver{xetex}%
    \else
      \ifvtex
        \def\hologoDriver{vtex}%
      \fi
    \fi
  \fi
}
%    \end{macrocode}
%    \end{macro}
%
%    \begin{macro}{\HOLOGO@WarningUnsupportedDriver}
%    \begin{macrocode}
\def\HOLOGO@WarningUnsupportedDriver#1{%
  \@PackageWarningNoLine{hologo}{%
    Logo `#1' needs driver specific macros,\MessageBreak
    but driver `\hologoDriver' is not supported.\MessageBreak
    Use a different driver or\MessageBreak
    load package `graphics' or `pgf'%
  }%
}
%    \end{macrocode}
%    \end{macro}
%
% \subsubsection{Reflect box macros}
%
%    Skip driver part if not needed.
%    \begin{macrocode}
\ltx@IfUndefined{reflectbox}{}{%
  \ltx@IfUndefined{rotatebox}{}{%
    \HOLOGO@AtEnd
  }%
}
\ltx@IfUndefined{pgftext}{}{%
  \HOLOGO@AtEnd
}
\ltx@IfUndefined{psscalebox}{}{%
  \HOLOGO@AtEnd
}
%    \end{macrocode}
%
%    \begin{macrocode}
\def\HOLOGO@temp{LaTeX2e}
\ifx\fmtname\HOLOGO@temp
  \RequirePackage{kvoptions}[2011/06/30]%
  \ProcessKeyvalOptions{HoLogoDriver}%
\fi
\HOLOGO@DriverSetup{}
%    \end{macrocode}
%
%    \begin{macro}{\HOLOGO@ReflectBox}
%    \begin{macrocode}
\def\HOLOGO@ReflectBox#1{%
  \begingroup
    \setbox\ltx@zero\hbox{\begingroup#1\endgroup}%
    \setbox\ltx@two\hbox{%
      \kern\wd\ltx@zero
      \csname HOLOGO@ScaleBox@\hologoDriver\endcsname{-1}{1}{%
        \hbox to 0pt{\copy\ltx@zero\hss}%
      }%
    }%
    \wd\ltx@two=\wd\ltx@zero
    \box\ltx@two
  \endgroup
}
%    \end{macrocode}
%    \end{macro}
%
%    \begin{macro}{\HOLOGO@PointReflectBox}
%    \begin{macrocode}
\def\HOLOGO@PointReflectBox#1{%
  \begingroup
    \setbox\ltx@zero\hbox{\begingroup#1\endgroup}%
    \setbox\ltx@two\hbox{%
      \kern\wd\ltx@zero
      \raise\ht\ltx@zero\hbox{%
        \csname HOLOGO@ScaleBox@\hologoDriver\endcsname{-1}{-1}{%
          \hbox to 0pt{\copy\ltx@zero\hss}%
        }%
      }%
    }%
    \wd\ltx@two=\wd\ltx@zero
    \box\ltx@two
  \endgroup
}
%    \end{macrocode}
%    \end{macro}
%
%    We must define all variants because of dynamic driver setup.
%    \begin{macrocode}
\def\HOLOGO@temp#1#2{#2}
%    \end{macrocode}
%
%    \begin{macro}{\HOLOGO@ScaleBox@pdftex}
%    \begin{macrocode}
\HOLOGO@temp{pdftex}{%
  \def\HOLOGO@ScaleBox@pdftex#1#2#3{%
    \HOLOGO@pdfliteral{%
      q #1 0 0 #2 0 0 cm%
    }%
    #3%
    \HOLOGO@pdfliteral{%
      Q%
    }%
  }%
}
%    \end{macrocode}
%    \end{macro}
%    \begin{macro}{\HOLOGO@ScaleBox@dvips}
%    \begin{macrocode}
\HOLOGO@temp{dvips}{%
  \def\HOLOGO@ScaleBox@dvips#1#2#3{%
    \special{ps:%
      gsave %
      currentpoint %
      currentpoint translate %
      #1 #2 scale %
      neg exch neg exch translate%
    }%
    #3%
    \special{ps:%
      currentpoint %
      grestore %
      moveto%
    }%
  }%
}
%    \end{macrocode}
%    \end{macro}
%    \begin{macro}{\HOLOGO@ScaleBox@dvipdfm}
%    \begin{macrocode}
\HOLOGO@temp{dvipdfm}{%
  \let\HOLOGO@ScaleBox@dvipdfm\HOLOGO@ScaleBox@dvips
}
%    \end{macrocode}
%    \end{macro}
%    Since \hologo{XeTeX} v0.6.
%    \begin{macro}{\HOLOGO@ScaleBox@xetex}
%    \begin{macrocode}
\HOLOGO@temp{xetex}{%
  \def\HOLOGO@ScaleBox@xetex#1#2#3{%
    \special{x:gsave}%
    \special{x:scale #1 #2}%
    #3%
    \special{x:grestore}%
  }%
}
%    \end{macrocode}
%    \end{macro}
%    \begin{macro}{\HOLOGO@ScaleBox@vtex}
%    \begin{macrocode}
\HOLOGO@temp{vtex}{%
  \def\HOLOGO@ScaleBox@vtex#1#2#3{%
    \special{r(#1,0,0,#2,0,0}%
    #3%
    \special{r)}%
  }%
}
%    \end{macrocode}
%    \end{macro}
%
%    \begin{macrocode}
\HOLOGO@AtEnd%
%</package>
%    \end{macrocode}
%
% \section{Test}
%
% \subsection{Catcode checks for loading}
%
%    \begin{macrocode}
%<*test1>
%    \end{macrocode}
%    \begin{macrocode}
\catcode`\{=1 %
\catcode`\}=2 %
\catcode`\#=6 %
\catcode`\@=11 %
\expandafter\ifx\csname count@\endcsname\relax
  \countdef\count@=255 %
\fi
\expandafter\ifx\csname @gobble\endcsname\relax
  \long\def\@gobble#1{}%
\fi
\expandafter\ifx\csname @firstofone\endcsname\relax
  \long\def\@firstofone#1{#1}%
\fi
\expandafter\ifx\csname loop\endcsname\relax
  \expandafter\@firstofone
\else
  \expandafter\@gobble
\fi
{%
  \def\loop#1\repeat{%
    \def\body{#1}%
    \iterate
  }%
  \def\iterate{%
    \body
      \let\next\iterate
    \else
      \let\next\relax
    \fi
    \next
  }%
  \let\repeat=\fi
}%
\def\RestoreCatcodes{}
\count@=0 %
\loop
  \edef\RestoreCatcodes{%
    \RestoreCatcodes
    \catcode\the\count@=\the\catcode\count@\relax
  }%
\ifnum\count@<255 %
  \advance\count@ 1 %
\repeat

\def\RangeCatcodeInvalid#1#2{%
  \count@=#1\relax
  \loop
    \catcode\count@=15 %
  \ifnum\count@<#2\relax
    \advance\count@ 1 %
  \repeat
}
\def\RangeCatcodeCheck#1#2#3{%
  \count@=#1\relax
  \loop
    \ifnum#3=\catcode\count@
    \else
      \errmessage{%
        Character \the\count@\space
        with wrong catcode \the\catcode\count@\space
        instead of \number#3%
      }%
    \fi
  \ifnum\count@<#2\relax
    \advance\count@ 1 %
  \repeat
}
\def\space{ }
\expandafter\ifx\csname LoadCommand\endcsname\relax
  \def\LoadCommand{\input hologo.sty\relax}%
\fi
\def\Test{%
  \RangeCatcodeInvalid{0}{47}%
  \RangeCatcodeInvalid{58}{64}%
  \RangeCatcodeInvalid{91}{96}%
  \RangeCatcodeInvalid{123}{255}%
  \catcode`\@=12 %
  \catcode`\\=0 %
  \catcode`\%=14 %
  \LoadCommand
  \RangeCatcodeCheck{0}{36}{15}%
  \RangeCatcodeCheck{37}{37}{14}%
  \RangeCatcodeCheck{38}{47}{15}%
  \RangeCatcodeCheck{48}{57}{12}%
  \RangeCatcodeCheck{58}{63}{15}%
  \RangeCatcodeCheck{64}{64}{12}%
  \RangeCatcodeCheck{65}{90}{11}%
  \RangeCatcodeCheck{91}{91}{15}%
  \RangeCatcodeCheck{92}{92}{0}%
  \RangeCatcodeCheck{93}{96}{15}%
  \RangeCatcodeCheck{97}{122}{11}%
  \RangeCatcodeCheck{123}{255}{15}%
  \RestoreCatcodes
}
\Test
\csname @@end\endcsname
\end
%    \end{macrocode}
%    \begin{macrocode}
%</test1>
%    \end{macrocode}
%
% \subsection{Spacefactor}
%
%    The space factor must be 1000 after a logo. If it is greater 1000
%    then the following space is a space after a sentence closing point.
%    If the space factor is smaller 1000 then an immediate following
%    dot is interpreted as abbreviation, not sentence closing point.
%
%    \begin{macrocode}
%<*test-spacefactor>
\NeedsTeXFormat{LaTeX2e}
\documentclass{article}
\usepackage{hologo}[2016/05/12]
\usepackage{kvsetkeys}
\usepackage{qstest}
\IncludeTests{*}
\LogTests{log}{*}{*}
\begin{document}
\begin{qstest}{spacefactor}{spacefactor}
\newcommand*{\Test}[1]{%
  \sbox0{%
    \hologo{#1}%
    \Expect*{1000 (#1)}*{\the\spacefactor\space(#1)}%
  }%
}%
\makeatletter
\def\TestList{}
\def\hologoEntry#1#2#3{%
  \edef\TestList{%
    \ifx\TestList\@empty
    \else
      \TestList,%
    \fi
    #1%
    \ifx\\#2\\%
    \else
      ={variant=#2}%
    \fi
  }%
}
\hologoList
\expandafter\kv@parse@normalized\expandafter{%
  \TestList
}{%
  \begingroup
    \let\@logo=\kv@key
    \ifx\kv@value\relax
    \else
      \expandafter\hologoLogoSetup\expandafter\@logo\expandafter{%
        \kv@value
      }%
    \fi
    \Test\@logo
  \endgroup
  \@gobbletwo
}
\end{qstest}
\end{document}
%</test-spacefactor>
%    \end{macrocode}
%
% \subsection{Complete list}
%
%    \begin{macrocode}
%<*test-list>
\NeedsTeXFormat{LaTeX2e}
\documentclass[12pt,a4paper]{article}
\usepackage{hologo}[2016/05/12]
\usepackage[T1]{fontenc}
\usepackage{lmodern}
\usepackage{parskip}
\usepackage[unicode]{hyperref}[2011/09/28]
\usepackage{bookmark}[2011/09/19]
\bookmarksetup{%
  numbered,%
  open,%
  openlevel=2,%
}
\renewcommand*{\contentsname}{List of logos}
\begin{document}
\tableofcontents
\def\TestFont#1#2#3#4#5#6{%
  \begingroup
    \usefont{#3}{#4}{#5}{#6}%
    \HologoVariant{#1}{#2}/\hologoVariant{#1}{#2}%
    \quad
    \begingroup\scriptsize\hologoVariant{#1}{#2}\endgroup
    \quad
  \endgroup
  (#3/#4/#5/#6)%
  \par
}
\makeatletter
\def\hologoEntry#1#2#3{%
  \section{%
    \HologoVariant{#1}{#2}/\hologoVariant{#1}{#2} %
    {[#1\ifx\\#2\\\else\space(#2)\fi]}% hash-ok
  }% braces around [] because of bug in tex4ht
  \begingroup
    \hypersetup{unicode=false}%
    \bookmark[%
      dest=\@currentHref,%
      rellevel=1,%
      keeplevel,%
    ]{%
      \HologoVariant{#1}{#2}/\hologoVariant{#1}{#2} %
      (PDFDocEncoding)%
    }%
  \endgroup
  \TestFont{#1}{#2}{OT1}{cmr}{m}{n}%
  \TestFont{#1}{#2}{OT1}{cmss}{m}{n}%
  \TestFont{#1}{#2}{OT1}{cmr}{b}{n}%
  \TestFont{#1}{#2}{OT1}{cmr}{m}{it}%
  \TestFont{#1}{#2}{OT1}{cmtt}{m}{n}%
  \TestFont{#1}{#2}{T1}{lmr}{m}{n}%
  \TestFont{#1}{#2}{T1}{lmss}{m}{n}%
  \TestFont{#1}{#2}{T1}{lmr}{b}{n}%
  \TestFont{#1}{#2}{T1}{lmr}{m}{it}%
  \TestFont{#1}{#2}{T1}{lmtt}{m}{n}%
  \TestFont{#1}{#2}{T1}{lmvtt}{m}{n}%
  \TestFont{#1}{#2}{T1}{qtm}{m}{n}%
  \TestFont{#1}{#2}{T1}{qhv}{m}{n}%
  \TestFont{#1}{#2}{T1}{qtm}{b}{n}%
  \TestFont{#1}{#2}{T1}{qtm}{m}{it}%
  \TestFont{#1}{#2}{T1}{qcr}{m}{n}%
  \newpage
}
\makeatother
\hologoList
\end{document}
%</test-list>
%    \end{macrocode}
%
% \section{Installation}
%
% \subsection{Download}
%
% \paragraph{Package.} This package is available on
% CTAN\footnote{\url{ftp://ftp.ctan.org/tex-archive/}}:
% \begin{description}
% \item[\CTAN{macros/latex/contrib/oberdiek/hologo.dtx}] The source file.
% \item[\CTAN{macros/latex/contrib/oberdiek/hologo.pdf}] Documentation.
% \end{description}
%
%
% \paragraph{Bundle.} All the packages of the bundle `oberdiek'
% are also available in a TDS compliant ZIP archive. There
% the packages are already unpacked and the documentation files
% are generated. The files and directories obey the TDS standard.
% \begin{description}
% \item[\CTAN{install/macros/latex/contrib/oberdiek.tds.zip}]
% \end{description}
% \emph{TDS} refers to the standard ``A Directory Structure
% for \TeX\ Files'' (\CTAN{tds/tds.pdf}). Directories
% with \xfile{texmf} in their name are usually organized this way.
%
% \subsection{Bundle installation}
%
% \paragraph{Unpacking.} Unpack the \xfile{oberdiek.tds.zip} in the
% TDS tree (also known as \xfile{texmf} tree) of your choice.
% Example (linux):
% \begin{quote}
%   |unzip oberdiek.tds.zip -d ~/texmf|
% \end{quote}
%
% \paragraph{Script installation.}
% Check the directory \xfile{TDS:scripts/oberdiek/} for
% scripts that need further installation steps.
% Package \xpackage{attachfile2} comes with the Perl script
% \xfile{pdfatfi.pl} that should be installed in such a way
% that it can be called as \texttt{pdfatfi}.
% Example (linux):
% \begin{quote}
%   |chmod +x scripts/oberdiek/pdfatfi.pl|\\
%   |cp scripts/oberdiek/pdfatfi.pl /usr/local/bin/|
% \end{quote}
%
% \subsection{Package installation}
%
% \paragraph{Unpacking.} The \xfile{.dtx} file is a self-extracting
% \docstrip\ archive. The files are extracted by running the
% \xfile{.dtx} through \plainTeX:
% \begin{quote}
%   \verb|tex hologo.dtx|
% \end{quote}
%
% \paragraph{TDS.} Now the different files must be moved into
% the different directories in your installation TDS tree
% (also known as \xfile{texmf} tree):
% \begin{quote}
% \def\t{^^A
% \begin{tabular}{@{}>{\ttfamily}l@{ $\rightarrow$ }>{\ttfamily}l@{}}
%   hologo.sty & tex/generic/oberdiek/hologo.sty\\
%   hologo.pdf & doc/latex/oberdiek/hologo.pdf\\
%   example/hologo-example.tex & doc/latex/oberdiek/example/hologo-example.tex\\
%   test/hologo-test1.tex & doc/latex/oberdiek/test/hologo-test1.tex\\
%   test/hologo-test-spacefactor.tex & doc/latex/oberdiek/test/hologo-test-spacefactor.tex\\
%   test/hologo-test-list.tex & doc/latex/oberdiek/test/hologo-test-list.tex\\
%   hologo.dtx & source/latex/oberdiek/hologo.dtx\\
% \end{tabular}^^A
% }^^A
% \sbox0{\t}^^A
% \ifdim\wd0>\linewidth
%   \begingroup
%     \advance\linewidth by\leftmargin
%     \advance\linewidth by\rightmargin
%   \edef\x{\endgroup
%     \def\noexpand\lw{\the\linewidth}^^A
%   }\x
%   \def\lwbox{^^A
%     \leavevmode
%     \hbox to \linewidth{^^A
%       \kern-\leftmargin\relax
%       \hss
%       \usebox0
%       \hss
%       \kern-\rightmargin\relax
%     }^^A
%   }^^A
%   \ifdim\wd0>\lw
%     \sbox0{\small\t}^^A
%     \ifdim\wd0>\linewidth
%       \ifdim\wd0>\lw
%         \sbox0{\footnotesize\t}^^A
%         \ifdim\wd0>\linewidth
%           \ifdim\wd0>\lw
%             \sbox0{\scriptsize\t}^^A
%             \ifdim\wd0>\linewidth
%               \ifdim\wd0>\lw
%                 \sbox0{\tiny\t}^^A
%                 \ifdim\wd0>\linewidth
%                   \lwbox
%                 \else
%                   \usebox0
%                 \fi
%               \else
%                 \lwbox
%               \fi
%             \else
%               \usebox0
%             \fi
%           \else
%             \lwbox
%           \fi
%         \else
%           \usebox0
%         \fi
%       \else
%         \lwbox
%       \fi
%     \else
%       \usebox0
%     \fi
%   \else
%     \lwbox
%   \fi
% \else
%   \usebox0
% \fi
% \end{quote}
% If you have a \xfile{docstrip.cfg} that configures and enables \docstrip's
% TDS installing feature, then some files can already be in the right
% place, see the documentation of \docstrip.
%
% \subsection{Refresh file name databases}
%
% If your \TeX~distribution
% (\teTeX, \mikTeX, \dots) relies on file name databases, you must refresh
% these. For example, \teTeX\ users run \verb|texhash| or
% \verb|mktexlsr|.
%
% \subsection{Some details for the interested}
%
% \paragraph{Attached source.}
%
% The PDF documentation on CTAN also includes the
% \xfile{.dtx} source file. It can be extracted by
% AcrobatReader 6 or higher. Another option is \textsf{pdftk},
% e.g. unpack the file into the current directory:
% \begin{quote}
%   \verb|pdftk hologo.pdf unpack_files output .|
% \end{quote}
%
% \paragraph{Unpacking with \LaTeX.}
% The \xfile{.dtx} chooses its action depending on the format:
% \begin{description}
% \item[\plainTeX:] Run \docstrip\ and extract the files.
% \item[\LaTeX:] Generate the documentation.
% \end{description}
% If you insist on using \LaTeX\ for \docstrip\ (really,
% \docstrip\ does not need \LaTeX), then inform the autodetect routine
% about your intention:
% \begin{quote}
%   \verb|latex \let\install=y\input{hologo.dtx}|
% \end{quote}
% Do not forget to quote the argument according to the demands
% of your shell.
%
% \paragraph{Generating the documentation.}
% You can use both the \xfile{.dtx} or the \xfile{.drv} to generate
% the documentation. The process can be configured by the
% configuration file \xfile{ltxdoc.cfg}. For instance, put this
% line into this file, if you want to have A4 as paper format:
% \begin{quote}
%   \verb|\PassOptionsToClass{a4paper}{article}|
% \end{quote}
% An example follows how to generate the
% documentation with pdf\LaTeX:
% \begin{quote}
%\begin{verbatim}
%pdflatex hologo.dtx
%makeindex -s gind.ist hologo.idx
%pdflatex hologo.dtx
%makeindex -s gind.ist hologo.idx
%pdflatex hologo.dtx
%\end{verbatim}
% \end{quote}
%
% \section{Catalogue}
%
% The following XML file can be used as source for the
% \href{http://mirror.ctan.org/help/Catalogue/catalogue.html}{\TeX\ Catalogue}.
% The elements \texttt{caption} and \texttt{description} are imported
% from the original XML file from the Catalogue.
% The name of the XML file in the Catalogue is \xfile{hologo.xml}.
%    \begin{macrocode}
%<*catalogue>
<?xml version='1.0' encoding='us-ascii'?>
<!DOCTYPE entry SYSTEM 'catalogue.dtd'>
<entry datestamp='$Date$' modifier='$Author$' id='hologo'>
  <name>hologo</name>
  <caption>A collection of logos with bookmark support.</caption>
  <authorref id='auth:oberdiek'/>
  <copyright owner='Heiko Oberdiek' year='2010-2012'/>
  <license type='lppl1.3'/>
  <version number='1.10'/>
  <description>
    The package defines a single command <tt>\hologo</tt>, whose
    argument is the usual case-confused ASCII version of the logo.
    The command is bookmark-enabled, so that every logo becomes
    available in bookmarks without further work.
    <p/>
    The package is part of the <xref refid='oberdiek'>oberdiek</xref>
    bundle.
  </description>
  <documentation details='Package documentation'
      href='ctan:/macros/latex/contrib/oberdiek/hologo.pdf'/>
  <ctan file='true' path='/macros/latex/contrib/oberdiek/hologo.dtx'/>
  <miktex location='oberdiek'/>
  <texlive location='oberdiek'/>
  <install path='/macros/latex/contrib/oberdiek/oberdiek.tds.zip'/>
</entry>
%</catalogue>
%    \end{macrocode}
%
% \begin{thebibliography}{9}
% \raggedright
%
% \bibitem{btxdoc}
% Oren Patashnik,
% \textit{\hologo{BibTeX}ing},
% 1988-02-08.\\
% \CTAN{biblio/bibtex/base/}
%
% \bibitem{dtklogos}
% Gerd Neugebauer, DANTE,
% \textit{Package \xpackage{dtklogos}},
% 2011-04-25.\\
% \CTAN{usergrps/dante/dtk/dtklogos.sty}
%
% \bibitem{etexman}
% The \hologo{NTS} Team,
% \textit{The \hologo{eTeX} manual},
% 1998-02.\\
% \CTAN{systems/e-tex/v2/doc/}
%
% \bibitem{ExTeX-FAQ}
% The \hologo{ExTeX} group,
% \textit{\hologo{ExTeX}: FAQ -- How is \hologo{ExTeX} typeset?},
% 2007-04-14.\\
% \url{http://www.extex.org/documentation/faq.html}
%
% \bibitem{LyX}
% %@MISC{ LyX,
% %  title = {{LyX 2.0.0 -- The Document Processor [Computer software and manual]}},
% %  author = {{The LyX Team}},
% %  howpublished = {Internet: http://www.lyx.org},
% %  year = {2011-05-08},
% %  note = {Retrieved May 10, 2011, from http://www.lyx.org},
% %  url = {http://www.lyx.org/}
% %}
% The \hologo{LyX} Team,
% \textit{\hologo{LyX} -- The Document Processor},
% 2011-05-08.\\
% \url{http://www.lyx.org/}
%
% \bibitem{OzTeX}
% Andrew Trevorrow,
% \hologo{OzTeX} FAQ: What is the correct way to typeset ``\hologo{OzTeX}''?,
% 2011-09-15 (visited).
% \url{http://www.trevorrow.com/oztex/ozfaq.html#oztex-logo}
%
% \bibitem{PiCTeX}
% Michael Wichura,
% \textit{The \hologo{PiCTeX} macro package},
% 1987-09-21.
% \CTAN{graphics/pictex/}
%
% \bibitem{scrlogo}
% Markus Kohm,
% \textit{\hologo{KOMAScript} Datei \xfile{scrlogo.dtx}},
% 2009-01-30.\\
% \CTAN{install/macros/latex/contrib/komascript.tds.zip}
%
% \end{thebibliography}
%
% \begin{History}
%   \begin{Version}{2010/04/08 v1.0}
%   \item
%     The first version.
%   \end{Version}
%   \begin{Version}{2010/04/16 v1.1}
%   \item
%     \cs{Hologo} added for support of logos at start of a sentence.
%   \item
%     \cs{hologoSetup} and \cs{hologoLogoSetup} added.
%   \item
%     Options \xoption{break}, \xoption{hyphenbreak}, \xoption{spacebreak}
%     added.
%   \item
%     Variant support added by option \xoption{variant}.
%   \end{Version}
%   \begin{Version}{2010/04/24 v1.2}
%   \item
%     \hologo{LaTeX3} added.
%   \item
%     \hologo{VTeX} added.
%   \end{Version}
%   \begin{Version}{2010/11/21 v1.3}
%   \item
%     \hologo{iniTeX}, \hologo{virTeX} added.
%   \end{Version}
%   \begin{Version}{2011/03/25 v1.4}
%   \item
%     \hologo{ConTeXt} with variants added.
%   \item
%     Option \xoption{discretionarybreak} added as refinement for
%     option \xoption{break}.
%   \end{Version}
%   \begin{Version}{2011/04/21 v1.5}
%   \item
%     Wrong TDS directory for test files fixed.
%   \end{Version}
%   \begin{Version}{2011/10/01 v1.6}
%   \item
%     Support for package \xpackage{tex4ht} added.
%   \item
%     Support for \cs{csname} added if \cs{ifincsname} is available.
%   \item
%     New logos:
%     \hologo{(La)TeX},
%     \hologo{biber},
%     \hologo{BibTeX} (\xoption{sc}, \xoption{sf}),
%     \hologo{emTeX},
%     \hologo{ExTeX},
%     \hologo{KOMAScript},
%     \hologo{La},
%     \hologo{LyX},
%     \hologo{MiKTeX},
%     \hologo{NTS},
%     \hologo{OzMF},
%     \hologo{OzMP},
%     \hologo{OzTeX},
%     \hologo{OzTtH},
%     \hologo{PCTeX},
%     \hologo{PiC},
%     \hologo{PiCTeX},
%     \hologo{METAFONT},
%     \hologo{MetaFun},
%     \hologo{METAPOST},
%     \hologo{MetaPost},
%     \hologo{SLiTeX} (\xoption{lift}, \xoption{narrow}, \xoption{simple}),
%     \hologo{SliTeX} (\xoption{narrow}, \xoption{simple}, \xoption{lift}),
%     \hologo{teTeX}.
%   \item
%     Fixes:
%     \hologo{iniTeX},
%     \hologo{pdfLaTeX},
%     \hologo{pdfTeX},
%     \hologo{virTeX}.
%   \item
%     \cs{hologoFontSetup} and \cs{hologoLogoFontSetup} added.
%   \item
%     \cs{hologoVariant} and \cs{HologoVariant} added.
%   \end{Version}
%   \begin{Version}{2011/11/22 v1.7}
%   \item
%     New logos:
%     \hologo{BibTeX8},
%     \hologo{LaTeXML},
%     \hologo{SageTeX},
%     \hologo{TeX4ht},
%     \hologo{TTH}.
%   \item
%     \hologo{Xe} and friends: Driver stuff fixed.
%   \item
%     \hologo{Xe} and friends: Support for italic added.
%   \item
%     \hologo{Xe} and friends: Package support for \xpackage{pgf}
%     and \xpackage{pstricks} added.
%   \end{Version}
%   \begin{Version}{2011/11/29 v1.8}
%   \item
%     New logos:
%     \hologo{HanTheThanh}.
%   \end{Version}
%   \begin{Version}{2011/12/21 v1.9}
%   \item
%     Patch for package \xpackage{ifxetex} added for the case that
%     \cs{newif} is undefined in \hologo{iniTeX}.
%   \item
%     Some fixes for \hologo{iniTeX}.
%   \end{Version}
%   \begin{Version}{2012/04/26 v1.10}
%   \item
%     Fix in bookmark version of logo ``\hologo{HanTheThanh}''.
%   \end{Version}
%   \begin{Version}{2016/05/12 v1.11}
%   \item
%     Update HOLOGO@IfCharExists (previously in texlive)
%   \item define pdfliteral in current luatex.
%   \end{Version}
% \end{History}
%
% \PrintIndex
%
% \Finale
\endinput

%        (quote the arguments according to the demands of your shell)
%
% Documentation:
%    (a) If hologo.drv is present:
%           latex hologo.drv
%    (b) Without hologo.drv:
%           latex hologo.dtx; ...
%    The class ltxdoc loads the configuration file ltxdoc.cfg
%    if available. Here you can specify further options, e.g.
%    use A4 as paper format:
%       \PassOptionsToClass{a4paper}{article}
%
%    Programm calls to get the documentation (example):
%       pdflatex hologo.dtx
%       makeindex -s gind.ist hologo.idx
%       pdflatex hologo.dtx
%       makeindex -s gind.ist hologo.idx
%       pdflatex hologo.dtx
%
% Installation:
%    TDS:tex/generic/oberdiek/hologo.sty
%    TDS:doc/latex/oberdiek/hologo.pdf
%    TDS:doc/latex/oberdiek/example/hologo-example.tex
%    TDS:doc/latex/oberdiek/test/hologo-test1.tex
%    TDS:doc/latex/oberdiek/test/hologo-test-spacefactor.tex
%    TDS:doc/latex/oberdiek/test/hologo-test-list.tex
%    TDS:source/latex/oberdiek/hologo.dtx
%
%<*ignore>
\begingroup
  \catcode123=1 %
  \catcode125=2 %
  \def\x{LaTeX2e}%
\expandafter\endgroup
\ifcase 0\ifx\install y1\fi\expandafter
         \ifx\csname processbatchFile\endcsname\relax\else1\fi
         \ifx\fmtname\x\else 1\fi\relax
\else\csname fi\endcsname
%</ignore>
%<*install>
\input docstrip.tex
\Msg{************************************************************************}
\Msg{* Installation}
\Msg{* Package: hologo 2016/05/12 v1.11 A logo collection with bookmark support (HO)}
\Msg{************************************************************************}

\keepsilent
\askforoverwritefalse

\let\MetaPrefix\relax
\preamble

This is a generated file.

Project: hologo
Version: 2016/05/12 v1.11

Copyright (C) 2010-2012 by
   Heiko Oberdiek <heiko.oberdiek at googlemail.com>

This work may be distributed and/or modified under the
conditions of the LaTeX Project Public License, either
version 1.3c of this license or (at your option) any later
version. This version of this license is in
   http://www.latex-project.org/lppl/lppl-1-3c.txt
and the latest version of this license is in
   http://www.latex-project.org/lppl.txt
and version 1.3 or later is part of all distributions of
LaTeX version 2005/12/01 or later.

This work has the LPPL maintenance status "maintained".

This Current Maintainer of this work is Heiko Oberdiek.

The Base Interpreter refers to any `TeX-Format',
because some files are installed in TDS:tex/generic//.

This work consists of the main source file hologo.dtx
and the derived files
   hologo.sty, hologo.pdf, hologo.ins, hologo.drv, hologo-example.tex,
   hologo-test1.tex, hologo-test-spacefactor.tex,
   hologo-test-list.tex.

\endpreamble
\let\MetaPrefix\DoubleperCent

\generate{%
  \file{hologo.ins}{\from{hologo.dtx}{install}}%
  \file{hologo.drv}{\from{hologo.dtx}{driver}}%
  \usedir{tex/generic/oberdiek}%
  \file{hologo.sty}{\from{hologo.dtx}{package}}%
  \usedir{doc/latex/oberdiek/example}%
  \file{hologo-example.tex}{\from{hologo.dtx}{example}}%
  \usedir{doc/latex/oberdiek/test}%
  \file{hologo-test1.tex}{\from{hologo.dtx}{test1}}%
  \file{hologo-test-spacefactor.tex}{\from{hologo.dtx}{test-spacefactor}}%
  \file{hologo-test-list.tex}{\from{hologo.dtx}{test-list}}%
  \nopreamble
  \nopostamble
  \usedir{source/latex/oberdiek/catalogue}%
  \file{hologo.xml}{\from{hologo.dtx}{catalogue}}%
}

\catcode32=13\relax% active space
\let =\space%
\Msg{************************************************************************}
\Msg{*}
\Msg{* To finish the installation you have to move the following}
\Msg{* file into a directory searched by TeX:}
\Msg{*}
\Msg{*     hologo.sty}
\Msg{*}
\Msg{* To produce the documentation run the file `hologo.drv'}
\Msg{* through LaTeX.}
\Msg{*}
\Msg{* Happy TeXing!}
\Msg{*}
\Msg{************************************************************************}

\endbatchfile
%</install>
%<*ignore>
\fi
%</ignore>
%<*driver>
\NeedsTeXFormat{LaTeX2e}
\ProvidesFile{hologo.drv}%
  [2016/05/12 v1.11 A logo collection with bookmark support (HO)]%
\documentclass{ltxdoc}
\usepackage{holtxdoc}[2011/11/22]
\usepackage{hologo}[2016/05/12]
\usepackage{longtable}
\usepackage{array}
\usepackage{paralist}
%\usepackage[T1]{fontenc}
%\usepackage{lmodern}
\begin{document}
  \DocInput{hologo.dtx}%
\end{document}
%</driver>
% \fi
%
%
% \CharacterTable
%  {Upper-case    \A\B\C\D\E\F\G\H\I\J\K\L\M\N\O\P\Q\R\S\T\U\V\W\X\Y\Z
%   Lower-case    \a\b\c\d\e\f\g\h\i\j\k\l\m\n\o\p\q\r\s\t\u\v\w\x\y\z
%   Digits        \0\1\2\3\4\5\6\7\8\9
%   Exclamation   \!     Double quote  \"     Hash (number) \#
%   Dollar        \$     Percent       \%     Ampersand     \&
%   Acute accent  \'     Left paren    \(     Right paren   \)
%   Asterisk      \*     Plus          \+     Comma         \,
%   Minus         \-     Point         \.     Solidus       \/
%   Colon         \:     Semicolon     \;     Less than     \<
%   Equals        \=     Greater than  \>     Question mark \?
%   Commercial at \@     Left bracket  \[     Backslash     \\
%   Right bracket \]     Circumflex    \^     Underscore    \_
%   Grave accent  \`     Left brace    \{     Vertical bar  \|
%   Right brace   \}     Tilde         \~}
%
% \GetFileInfo{hologo.drv}
%
% \title{The \xpackage{hologo} package}
% \date{2016/05/12 v1.11}
% \author{Heiko Oberdiek\\\xemail{heiko.oberdiek at googlemail.com}}
%
% \maketitle
%
% \begin{abstract}
% This package starts a collection of logos with support for bookmarks
% strings.
% \end{abstract}
%
% \tableofcontents
%
% \section{Documentation}
%
% \subsection{Logo macros}
%
% \begin{declcs}{hologo} \M{name}
% \end{declcs}
% Macro \cs{hologo} sets the logo with name \meta{name}.
% The following table shows the supported names.
%
% \begingroup
%   \def\hologoEntry#1#2#3{^^A
%     #1&#2&\hologoLogoSetup{#1}{variant=#2}\hologo{#1}&#3\tabularnewline
%   }
%   \begin{longtable}{>{\ttfamily}l>{\ttfamily}lll}
%     \rmfamily\bfseries{name} & \rmfamily\bfseries variant
%     & \bfseries logo & \bfseries since\\
%     \hline
%     \endhead
%     \hologoList
%   \end{longtable}
% \endgroup
%
% \begin{declcs}{Hologo} \M{name}
% \end{declcs}
% Macro \cs{Hologo} starts the logo \meta{name} with an uppercase
% letter. As an exception small greek letters are not converted
% to uppercase. Examples, see \hologo{eTeX} and \hologo{ExTeX}.
%
% \subsection{Setup macros}
%
% The package does not support package options, but the following
% setup macros can be used to set options.
%
% \begin{declcs}{hologoSetup} \M{key value list}
% \end{declcs}
% Macro \cs{hologoSetup} sets global options.
%
% \begin{declcs}{hologoLogoSetup} \M{logo} \M{key value list}
% \end{declcs}
% Some options can also be used to configure a logo.
% These settings take precedence over global option settings.
%
% \subsection{Options}\label{sec:options}
%
% There are boolean and string options:
% \begin{description}
% \item[Boolean option:]
% It takes |true| or |false|
% as value. If the value is omitted, then |true| is used.
% \item[String option:]
% A value must be given as string. (But the string might be empty.)
% \end{description}
% The following options can be used both in \cs{hologoSetup}
% and \cs{hologoLogoSetup}:
% \begin{description}
% \def\entry#1{\item[\xoption{#1}:]}
% \entry{break}
%   enables or disables line breaks inside the logo. This setting is
%   refined by options \xoption{hyphenbreak}, \xoption{spacebreak}
%   or \xoption{discretionarybreak}.
%   Default is |false|.
% \entry{hyphenbreak}
%   enables or disables the line break right after the hyphen character.
% \entry{spacebreak}
%   enables or disables line breaks at space characters.
% \entry{discretionarybreak}
%   enables or disables line breaks at hyphenation points
%   (inserted by \cs{-}).
% \end{description}
% Macro \cs{hologoLogoSetup} also knows:
% \begin{description}
% \item[\xoption{variant}:]
%   This is a string option. It specifies a variant of a logo that
%   must exist. An empty string selects the package default variant.
% \end{description}
% Example:
% \begin{quote}
%   |\hologoSetup{break=false}|\\
%   |\hologoLogoSetup{plainTeX}{variant=hyphen,hyphenbreak}|\\
%   Then ``plain-\TeX'' contains one break point after the hyphen.
% \end{quote}
%
% \subsection{Driver options}
%
% Sometimes graphical operations are needed to construct some
% glyphs (e.g.\ \hologo{XeTeX}). If package \xpackage{graphics}
% or package \xpackage{pgf} are found, then the macros are taken
% from there. Otherwise the packge defines its own operations
% and therefore needs the driver information. Many drivers are
% detected automatically (\hologo{pdfTeX}/\hologo{LuaTeX}
% in PDF mode, \hologo{XeTeX}, \hologo{VTeX}). These have precedence
% over a driver option. The driver can be given as package option
% or using \cs{hologoDriverSetup}.
% The following list contains the recognized driver options:
% \begin{itemize}
% \item \xoption{pdftex}, \xoption{luatex}
% \item \xoption{dvipdfm}, \xoption{dvipdfmx}
% \item \xoption{dvips}, \xoption{dvipsone}, \xoption{xdvi}
% \item \xoption{xetex}
% \item \xoption{vtex}
% \end{itemize}
% The left driver of a line is the driver name that is used internally.
% The following names are aliases for drivers that use the
% same method. Therefore the entry in the \xext{log} file for
% the used driver prints the internally used driver name.
% \begin{description}
% \item[\xoption{driverfallback}:]
%   This option expects a driver that is used,
%   if the driver could not be detected automatically.
% \end{description}
%
% \begin{declcs}{hologoDriverSetup} \M{driver option}
% \end{declcs}
% The driver can also be configured after package loading
% using \cs{hologoDriverSetup}, also the way for \hologo{plainTeX}
% to setup the driver.
%
% \subsection{Font setup}
%
% Some logos require a special font, but should also be usable by
% \hologo{plainTeX}. Therefore the package provides some ways
% to influence the font settings. The options below
% take font settings as values. Both font commands
% such as \cs{sffamily} and macros that take one argument
% like \cs{textsf} can be used.
%
% \begin{declcs}{hologoFontSetup} \M{key value list}
% \end{declcs}
% Macro \cs{hologoFontSetup} sets the fonts for all logos.
% Supported keys:
% \begin{description}
% \def\entry#1{\item[\xoption{#1}:]}
% \entry{general}
%   This font is used for all logos. The default is empty.
%   That means no special font is used.
% \entry{bibsf}
%   This font is used for
%   {\hologoLogoSetup{BibTeX}{variant=sf}\hologo{BibTeX}}
%   with variant \xoption{sf}.
% \entry{rm}
%   This font is a serif font. It is used for \hologo{ExTeX}.
% \entry{sc}
%   This font specifies a small caps font. It is used for
%   {\hologoLogoSetup{BibTeX}{variant=sc}\hologo{BibTeX}}
%   with variant \xoption{sc}.
% \entry{sf}
%   This font specifies a sans serif font. The default
%   is \cs{sffamily}, then \cs{sf} is tried. Otherwise
%   a warning is given. It is used by \hologo{KOMAScript}.
% \entry{sy}
%   This is the font for math symbols (e.g. cmsy).
%   It is used by \hologo{AmS}, \hologo{NTS}, \hologo{ExTeX}.
% \entry{logo}
%   \hologo{METAFONT} and \hologo{METAPOST} are using that font.
%   In \hologo{LaTeX} \cs{logofamily} is used and
%   the definitions of package \xpackage{mflogo} are used
%   if the package is not loaded.
%   Otherwise the \cs{tenlogo} is used and defined
%   if it does not already exists.
% \end{description}
%
% \begin{declcs}{hologoLogoFontSetup} \M{logo} \M{key value list}
% \end{declcs}
% Fonts can also be set for a logo or logo component separately,
% see the following list.
% The keys are the same as for \cs{hologoFontSetup}.
%
% \begin{longtable}{>{\ttfamily}l>{\sffamily}ll}
%   \meta{logo} & keys & result\\
%   \hline
%   \endhead
%   BibTeX & bibsf & {\hologoLogoSetup{BibTeX}{variant=sf}\hologo{BibTeX}}\\[.5ex]
%   BibTeX & sc & {\hologoLogoSetup{BibTeX}{variant=sc}\hologo{BibTeX}}\\[.5ex]
%   ExTeX & rm & \hologo{ExTeX}\\
%   SliTeX & rm & \hologo{SliTeX}\\[.5ex]
%   AmS & sy & \hologo{AmS}\\
%   ExTeX & sy & \hologo{ExTeX}\\
%   NTS & sy & \hologo{NTS}\\[.5ex]
%   KOMAScript & sf & \hologo{KOMAScript}\\[.5ex]
%   METAFONT & logo & \hologo{METAFONT}\\
%   METAPOST & logo & \hologo{METAPOST}\\[.5ex]
%   SliTeX & sc \hologo{SliTeX}
% \end{longtable}
%
% \subsubsection{Font order}
%
% For all logos the font \xoption{general} is applied first.
% Example:
%\begin{quote}
%|\hologoFontSetup{general=\color{red}}|
%\end{quote}
% will print red logos.
% Then if the font uses a special font \xoption{sf}, for example,
% the font is applied that is setup by \cs{hologoLogoFontSetup}.
% If this font is not setup, then the common font setup
% by \cs{hologoFontSetup} is used. Otherwise a warning is given,
% that there is no font configured.
%
% \subsection{Additional user macros}
%
% Usually a variant of a logo is configured by using
% \cs{hologoLogoSetup}, because it is bad style to mix
% different variants of the same logo in the same text.
% There the following macros are a convenience for testing.
%
% \begin{declcs}{hologoVariant} \M{name} \M{variant}\\
%   \cs{HologoVariant} \M{name} \M{variant}
% \end{declcs}
% Logo \meta{name} is set using \meta{variant} that specifies
% explicitely which variant of the macro is used. If the argument
% is empty, then the default form of the logo is used
% (configurable by \cs{hologoLogoSetup}).
%
% \cs{HologoVariant} is used if the logo is set in a context
% that needs an uppercase first letter (beginning of a sentence, \dots).
%
% \begin{declcs}{hologoList}\\
%   \cs{hologoEntry} \M{logo} \M{variant} \M{since}
% \end{declcs}
% Macro \cs{hologoList} contains all logos that are provided
% by the package including variants. The list consists of calls
% of \cs{hologoEntry} with three arguments starting with the
% logo name \meta{logo} and its variant \meta{variant}. An empty
% variant means the current default. Argument \meta{since} specifies
% with version of the package \xpackage{hologo} is needed to get
% the logo. If the logo is fixed, then the date gets updated.
% Therefore the date \meta{since} is not exactly the date of
% the first introduction, but rather the date of the latest fix.
%
% Before \cs{hologoList} can be used, macro \cs{hologoEntry} needs
% a definition. The example file in section \ref{sec:example}
% shows applications of \cs{hologoList}.
%
% \subsection{Supported contexts}
%
% Macros \cs{hologo} and friends support special contexts:
% \begin{itemize}
% \item \hologo{LaTeX}'s protection mechanism.
% \item Bookmarks of package \xpackage{hyperref}.
% \item Package \xpackage{tex4ht}.
% \item The macros can be used inside \cs{csname} constructs,
%   if \cs{ifincsname} is available (\hologo{pdfTeX}, \hologo{XeTeX},
%   \hologo{LuaTeX}).
% \end{itemize}
%
% \subsection{Example}
% \label{sec:example}
%
% The following example prints the logos in different fonts.
%    \begin{macrocode}
%<*example>
%<<verbatim
\NeedsTeXFormat{LaTeX2e}
\documentclass[a4paper]{article}
\usepackage[
  hmargin=20mm,
  vmargin=20mm,
]{geometry}
\pagestyle{empty}
\usepackage{hologo}[2016/05/12]
\usepackage{longtable}
\usepackage{array}
\setlength{\extrarowheight}{2pt}
\usepackage[T1]{fontenc}
\usepackage{lmodern}
\usepackage{pdflscape}
\usepackage[
  pdfencoding=auto,
]{hyperref}
\hypersetup{
  pdfauthor={Heiko Oberdiek},
  pdftitle={Example for package `hologo'},
  pdfsubject={Logos with fonts lmr, lmss, qtm, qpl, qhv},
}
\usepackage{bookmark}

% Print the logo list on the console

\begingroup
  \typeout{}%
  \typeout{*** Begin of logo list ***}%
  \newcommand*{\hologoEntry}[3]{%
    \typeout{#1 \ifx\\#2\\\else(#2) \fi[#3]}%
  }%
  \hologoList
  \typeout{*** End of logo list ***}%
  \typeout{}%
\endgroup

\begin{document}
\begin{landscape}

  \section{Example file for package `hologo'}

  % Table for font names

  \begin{longtable}{>{\bfseries}ll}
    \textbf{font} & \textbf{Font name}\\
    \hline
    lmr & Latin Modern Roman\\
    lmss & Latin Modern Sans\\
    qtm & \TeX\ Gyre Termes\\
    qhv & \TeX\ Gyre Heros\\
    qpl & \TeX\ Gyre Pagella\\
  \end{longtable}

  % Logo list with logos in different fonts

  \begingroup
    \newcommand*{\SetVariant}[2]{%
      \ifx\\#2\\%
      \else
        \hologoLogoSetup{#1}{variant=#2}%
      \fi
    }%
    \newcommand*{\hologoEntry}[3]{%
      \SetVariant{#1}{#2}%
      \raisebox{1em}[0pt][0pt]{\hypertarget{#1@#2}{}}%
      \bookmark[%
        dest={#1@#2},%
      ]{%
        #1\ifx\\#2\\\else\space(#2)\fi: \Hologo{#1}, \hologo{#1} %
        [Unicode]%
      }%
      \hypersetup{unicode=false}%
      \bookmark[%
        dest={#1@#2},%
      ]{%
        #1\ifx\\#2\\\else\space(#2)\fi: \Hologo{#1}, \hologo{#1} %
        [PDFDocEncoding]%
      }%
      \texttt{#1}%
      &%
      \texttt{#2}%
      &%
      \Hologo{#1}%
      &%
      \SetVariant{#1}{#2}%
      \hologo{#1}%
      &%
      \SetVariant{#1}{#2}%
      \fontfamily{qtm}\selectfont
      \hologo{#1}%
      &%
      \SetVariant{#1}{#2}%
      \fontfamily{qpl}\selectfont
      \hologo{#1}%
      &%
      \SetVariant{#1}{#2}%
      \textsf{\hologo{#1}}%
      &%
      \SetVariant{#1}{#2}%
      \fontfamily{qhv}\selectfont
      \hologo{#1}%
      \tabularnewline
    }%
    \begin{longtable}{llllllll}%
      \textbf{\textit{logo}} & \textbf{\textit{variant}} &
      \texttt{\string\Hologo} &
      \textbf{lmr} & \textbf{qtm} & \textbf{qpl} &
      \textbf{lmss} & \textbf{qhv}
      \tabularnewline
      \hline
      \endhead
      \hologoList
    \end{longtable}%
  \endgroup

\end{landscape}
\end{document}
%verbatim
%</example>
%    \end{macrocode}
%
% \StopEventually{
% }
%
% \section{Implementation}
%    \begin{macrocode}
%<*package>
%    \end{macrocode}
%    Reload check, especially if the package is not used with \LaTeX.
%    \begin{macrocode}
\begingroup\catcode61\catcode48\catcode32=10\relax%
  \catcode13=5 % ^^M
  \endlinechar=13 %
  \catcode35=6 % #
  \catcode39=12 % '
  \catcode44=12 % ,
  \catcode45=12 % -
  \catcode46=12 % .
  \catcode58=12 % :
  \catcode64=11 % @
  \catcode123=1 % {
  \catcode125=2 % }
  \expandafter\let\expandafter\x\csname ver@hologo.sty\endcsname
  \ifx\x\relax % plain-TeX, first loading
  \else
    \def\empty{}%
    \ifx\x\empty % LaTeX, first loading,
      % variable is initialized, but \ProvidesPackage not yet seen
    \else
      \expandafter\ifx\csname PackageInfo\endcsname\relax
        \def\x#1#2{%
          \immediate\write-1{Package #1 Info: #2.}%
        }%
      \else
        \def\x#1#2{\PackageInfo{#1}{#2, stopped}}%
      \fi
      \x{hologo}{The package is already loaded}%
      \aftergroup\endinput
    \fi
  \fi
\endgroup%
%    \end{macrocode}
%    Package identification:
%    \begin{macrocode}
\begingroup\catcode61\catcode48\catcode32=10\relax%
  \catcode13=5 % ^^M
  \endlinechar=13 %
  \catcode35=6 % #
  \catcode39=12 % '
  \catcode40=12 % (
  \catcode41=12 % )
  \catcode44=12 % ,
  \catcode45=12 % -
  \catcode46=12 % .
  \catcode47=12 % /
  \catcode58=12 % :
  \catcode64=11 % @
  \catcode91=12 % [
  \catcode93=12 % ]
  \catcode123=1 % {
  \catcode125=2 % }
  \expandafter\ifx\csname ProvidesPackage\endcsname\relax
    \def\x#1#2#3[#4]{\endgroup
      \immediate\write-1{Package: #3 #4}%
      \xdef#1{#4}%
    }%
  \else
    \def\x#1#2[#3]{\endgroup
      #2[{#3}]%
      \ifx#1\@undefined
        \xdef#1{#3}%
      \fi
      \ifx#1\relax
        \xdef#1{#3}%
      \fi
    }%
  \fi
\expandafter\x\csname ver@hologo.sty\endcsname
\ProvidesPackage{hologo}%
  [2016/05/12 v1.11 A logo collection with bookmark support (HO)]%
%    \end{macrocode}
%
%    \begin{macrocode}
\begingroup\catcode61\catcode48\catcode32=10\relax%
  \catcode13=5 % ^^M
  \endlinechar=13 %
  \catcode123=1 % {
  \catcode125=2 % }
  \catcode64=11 % @
  \def\x{\endgroup
    \expandafter\edef\csname HOLOGO@AtEnd\endcsname{%
      \endlinechar=\the\endlinechar\relax
      \catcode13=\the\catcode13\relax
      \catcode32=\the\catcode32\relax
      \catcode35=\the\catcode35\relax
      \catcode61=\the\catcode61\relax
      \catcode64=\the\catcode64\relax
      \catcode123=\the\catcode123\relax
      \catcode125=\the\catcode125\relax
    }%
  }%
\x\catcode61\catcode48\catcode32=10\relax%
\catcode13=5 % ^^M
\endlinechar=13 %
\catcode35=6 % #
\catcode64=11 % @
\catcode123=1 % {
\catcode125=2 % }
\def\TMP@EnsureCode#1#2{%
  \edef\HOLOGO@AtEnd{%
    \HOLOGO@AtEnd
    \catcode#1=\the\catcode#1\relax
  }%
  \catcode#1=#2\relax
}
\TMP@EnsureCode{10}{12}% ^^J
\TMP@EnsureCode{33}{12}% !
\TMP@EnsureCode{34}{12}% "
\TMP@EnsureCode{36}{3}% $
\TMP@EnsureCode{38}{4}% &
\TMP@EnsureCode{39}{12}% '
\TMP@EnsureCode{40}{12}% (
\TMP@EnsureCode{41}{12}% )
\TMP@EnsureCode{42}{12}% *
\TMP@EnsureCode{43}{12}% +
\TMP@EnsureCode{44}{12}% ,
\TMP@EnsureCode{45}{12}% -
\TMP@EnsureCode{46}{12}% .
\TMP@EnsureCode{47}{12}% /
\TMP@EnsureCode{58}{12}% :
\TMP@EnsureCode{59}{12}% ;
\TMP@EnsureCode{60}{12}% <
\TMP@EnsureCode{62}{12}% >
\TMP@EnsureCode{63}{12}% ?
\TMP@EnsureCode{91}{12}% [
\TMP@EnsureCode{93}{12}% ]
\TMP@EnsureCode{94}{7}% ^ (superscript)
\TMP@EnsureCode{95}{8}% _ (subscript)
\TMP@EnsureCode{96}{12}% `
\TMP@EnsureCode{124}{12}% |
\edef\HOLOGO@AtEnd{%
  \HOLOGO@AtEnd
  \escapechar\the\escapechar\relax
  \noexpand\endinput
}
\escapechar=92 %
%    \end{macrocode}
%
% \subsection{Logo list}
%
%    \begin{macro}{\hologoList}
%    \begin{macrocode}
\def\hologoList{%
  \hologoEntry{(La)TeX}{}{2011/10/01}%
  \hologoEntry{AmSLaTeX}{}{2010/04/16}%
  \hologoEntry{AmSTeX}{}{2010/04/16}%
  \hologoEntry{biber}{}{2011/10/01}%
  \hologoEntry{BibTeX}{}{2011/10/01}%
  \hologoEntry{BibTeX}{sf}{2011/10/01}%
  \hologoEntry{BibTeX}{sc}{2011/10/01}%
  \hologoEntry{BibTeX8}{}{2011/11/22}%
  \hologoEntry{ConTeXt}{}{2011/03/25}%
  \hologoEntry{ConTeXt}{narrow}{2011/03/25}%
  \hologoEntry{ConTeXt}{simple}{2011/03/25}%
  \hologoEntry{emTeX}{}{2010/04/26}%
  \hologoEntry{eTeX}{}{2010/04/08}%
  \hologoEntry{ExTeX}{}{2011/10/01}%
  \hologoEntry{HanTheThanh}{}{2011/11/29}%
  \hologoEntry{iniTeX}{}{2011/10/01}%
  \hologoEntry{KOMAScript}{}{2011/10/01}%
  \hologoEntry{La}{}{2010/05/08}%
  \hologoEntry{LaTeX}{}{2010/04/08}%
  \hologoEntry{LaTeX2e}{}{2010/04/08}%
  \hologoEntry{LaTeX3}{}{2010/04/24}%
  \hologoEntry{LaTeXe}{}{2010/04/08}%
  \hologoEntry{LaTeXML}{}{2011/11/22}%
  \hologoEntry{LaTeXTeX}{}{2011/10/01}%
  \hologoEntry{LuaLaTeX}{}{2010/04/08}%
  \hologoEntry{LuaTeX}{}{2010/04/08}%
  \hologoEntry{LyX}{}{2011/10/01}%
  \hologoEntry{METAFONT}{}{2011/10/01}%
  \hologoEntry{MetaFun}{}{2011/10/01}%
  \hologoEntry{METAPOST}{}{2011/10/01}%
  \hologoEntry{MetaPost}{}{2011/10/01}%
  \hologoEntry{MiKTeX}{}{2011/10/01}%
  \hologoEntry{NTS}{}{2011/10/01}%
  \hologoEntry{OzMF}{}{2011/10/01}%
  \hologoEntry{OzMP}{}{2011/10/01}%
  \hologoEntry{OzTeX}{}{2011/10/01}%
  \hologoEntry{OzTtH}{}{2011/10/01}%
  \hologoEntry{PCTeX}{}{2011/10/01}%
  \hologoEntry{pdfTeX}{}{2011/10/01}%
  \hologoEntry{pdfLaTeX}{}{2011/10/01}%
  \hologoEntry{PiC}{}{2011/10/01}%
  \hologoEntry{PiCTeX}{}{2011/10/01}%
  \hologoEntry{plainTeX}{}{2010/04/08}%
  \hologoEntry{plainTeX}{space}{2010/04/16}%
  \hologoEntry{plainTeX}{hyphen}{2010/04/16}%
  \hologoEntry{plainTeX}{runtogether}{2010/04/16}%
  \hologoEntry{SageTeX}{}{2011/11/22}%
  \hologoEntry{SLiTeX}{}{2011/10/01}%
  \hologoEntry{SLiTeX}{lift}{2011/10/01}%
  \hologoEntry{SLiTeX}{narrow}{2011/10/01}%
  \hologoEntry{SLiTeX}{simple}{2011/10/01}%
  \hologoEntry{SliTeX}{}{2011/10/01}%
  \hologoEntry{SliTeX}{narrow}{2011/10/01}%
  \hologoEntry{SliTeX}{simple}{2011/10/01}%
  \hologoEntry{SliTeX}{lift}{2011/10/01}%
  \hologoEntry{teTeX}{}{2011/10/01}%
  \hologoEntry{TeX}{}{2010/04/08}%
  \hologoEntry{TeX4ht}{}{2011/11/22}%
  \hologoEntry{TTH}{}{2011/11/22}%
  \hologoEntry{virTeX}{}{2011/10/01}%
  \hologoEntry{VTeX}{}{2010/04/24}%
  \hologoEntry{Xe}{}{2010/04/08}%
  \hologoEntry{XeLaTeX}{}{2010/04/08}%
  \hologoEntry{XeTeX}{}{2010/04/08}%
}
%    \end{macrocode}
%    \end{macro}
%
% \subsection{Load resources}
%
%    \begin{macrocode}
\begingroup\expandafter\expandafter\expandafter\endgroup
\expandafter\ifx\csname RequirePackage\endcsname\relax
  \def\TMP@RequirePackage#1[#2]{%
    \begingroup\expandafter\expandafter\expandafter\endgroup
    \expandafter\ifx\csname ver@#1.sty\endcsname\relax
      \input #1.sty\relax
    \fi
  }%
  \TMP@RequirePackage{ltxcmds}[2011/02/04]%
  \TMP@RequirePackage{infwarerr}[2010/04/08]%
  \TMP@RequirePackage{kvsetkeys}[2010/03/01]%
  \TMP@RequirePackage{kvdefinekeys}[2010/03/01]%
  \TMP@RequirePackage{pdftexcmds}[2010/04/01]%
  \TMP@RequirePackage{ifpdf}[2010/01/28]%
  \TMP@RequirePackage{ifluatex}[2010/03/01]%
  \ltx@IfUndefined{newif}{%
    \expandafter\let\csname newif\endcsname\ltx@newif
  }{}%
  \TMP@RequirePackage{ifxetex}[2009/01/23]%
  \TMP@RequirePackage{ifvtex}[2010/03/01]%
\else
  \RequirePackage{ltxcmds}[2011/02/04]%
  \RequirePackage{infwarerr}[2010/04/08]%
  \RequirePackage{kvsetkeys}[2010/03/01]%
  \RequirePackage{kvdefinekeys}[2010/03/01]%
  \RequirePackage{pdftexcmds}[2010/04/01]%
  \RequirePackage{ifpdf}[2010/01/28]%
  \RequirePackage{ifluatex}[2010/03/01]%
  \RequirePackage{ifxetex}[2009/01/23]%
  \RequirePackage{ifvtex}[2010/03/01]%
\fi
%    \end{macrocode}
%
%    \begin{macro}{\HOLOGO@IfDefined}
%    \begin{macrocode}
\def\HOLOGO@IfExists#1{%
  \ifx\@undefined#1%
    \expandafter\ltx@secondoftwo
  \else
    \ifx\relax#1%
      \expandafter\ltx@secondoftwo
    \else
      \expandafter\expandafter\expandafter\ltx@firstoftwo
    \fi
  \fi
}
%    \end{macrocode}
%    \end{macro}
%
% \subsection{Setup macros}
%
%    \begin{macro}{\hologoSetup}
%    \begin{macrocode}
\def\hologoSetup{%
  \let\HOLOGO@name\relax
  \HOLOGO@Setup
}
%    \end{macrocode}
%    \end{macro}
%
%    \begin{macro}{\hologoLogoSetup}
%    \begin{macrocode}
\def\hologoLogoSetup#1{%
  \edef\HOLOGO@name{#1}%
  \ltx@IfUndefined{HoLogo@\HOLOGO@name}{%
    \@PackageError{hologo}{%
      Unknown logo `\HOLOGO@name'%
    }\@ehc
    \ltx@gobble
  }{%
    \HOLOGO@Setup
  }%
}
%    \end{macrocode}
%    \end{macro}
%
%    \begin{macro}{\HOLOGO@Setup}
%    \begin{macrocode}
\def\HOLOGO@Setup{%
  \kvsetkeys{HoLogo}%
}
%    \end{macrocode}
%    \end{macro}
%
% \subsection{Options}
%
%    \begin{macro}{\HOLOGO@DeclareBoolOption}
%    \begin{macrocode}
\def\HOLOGO@DeclareBoolOption#1{%
  \expandafter\chardef\csname HOLOGOOPT@#1\endcsname\ltx@zero
  \kv@define@key{HoLogo}{#1}[true]{%
    \def\HOLOGO@temp{##1}%
    \ifx\HOLOGO@temp\HOLOGO@true
      \ifx\HOLOGO@name\relax
        \expandafter\chardef\csname HOLOGOOPT@#1\endcsname=\ltx@one
      \else
        \expandafter\chardef\csname
        HoLogoOpt@#1@\HOLOGO@name\endcsname\ltx@one
      \fi
      \HOLOGO@SetBreakAll{#1}%
    \else
      \ifx\HOLOGO@temp\HOLOGO@false
        \ifx\HOLOGO@name\relax
          \expandafter\chardef\csname HOLOGOOPT@#1\endcsname=\ltx@zero
        \else
          \expandafter\chardef\csname
          HoLogoOpt@#1@\HOLOGO@name\endcsname=\ltx@zero
        \fi
        \HOLOGO@SetBreakAll{#1}%
      \else
        \@PackageError{hologo}{%
          Unknown value `##1' for boolean option `#1'.\MessageBreak
          Known values are `true' and `false'%
        }\@ehc
      \fi
    \fi
  }%
}
%    \end{macrocode}
%    \end{macro}
%
%    \begin{macro}{\HOLOGO@SetBreakAll}
%    \begin{macrocode}
\def\HOLOGO@SetBreakAll#1{%
  \def\HOLOGO@temp{#1}%
  \ifx\HOLOGO@temp\HOLOGO@break
    \ifx\HOLOGO@name\relax
      \chardef\HOLOGOOPT@hyphenbreak=\HOLOGOOPT@break
      \chardef\HOLOGOOPT@spacebreak=\HOLOGOOPT@break
      \chardef\HOLOGOOPT@discretionarybreak=\HOLOGOOPT@break
    \else
      \expandafter\chardef
         \csname HoLogoOpt@hyphenbreak@\HOLOGO@name\endcsname=%
         \csname HoLogoOpt@break@\HOLOGO@name\endcsname
      \expandafter\chardef
         \csname HoLogoOpt@spacebreak@\HOLOGO@name\endcsname=%
         \csname HoLogoOpt@break@\HOLOGO@name\endcsname
      \expandafter\chardef
         \csname HoLogoOpt@discretionarybreak@\HOLOGO@name
             \endcsname=%
         \csname HoLogoOpt@break@\HOLOGO@name\endcsname
    \fi
  \fi
}
%    \end{macrocode}
%    \end{macro}
%
%    \begin{macro}{\HOLOGO@true}
%    \begin{macrocode}
\def\HOLOGO@true{true}
%    \end{macrocode}
%    \end{macro}
%    \begin{macro}{\HOLOGO@false}
%    \begin{macrocode}
\def\HOLOGO@false{false}
%    \end{macrocode}
%    \end{macro}
%    \begin{macro}{\HOLOGO@break}
%    \begin{macrocode}
\def\HOLOGO@break{break}
%    \end{macrocode}
%    \end{macro}
%
%    \begin{macrocode}
\HOLOGO@DeclareBoolOption{break}
\HOLOGO@DeclareBoolOption{hyphenbreak}
\HOLOGO@DeclareBoolOption{spacebreak}
\HOLOGO@DeclareBoolOption{discretionarybreak}
%    \end{macrocode}
%
%    \begin{macrocode}
\kv@define@key{HoLogo}{variant}{%
  \ifx\HOLOGO@name\relax
    \@PackageError{hologo}{%
      Option `variant' is not available in \string\hologoSetup,%
      \MessageBreak
      Use \string\hologoLogoSetup\space instead%
    }\@ehc
  \else
    \edef\HOLOGO@temp{#1}%
    \ifx\HOLOGO@temp\ltx@empty
      \expandafter
      \let\csname HoLogoOpt@variant@\HOLOGO@name\endcsname\@undefined
    \else
      \ltx@IfUndefined{HoLogo@\HOLOGO@name @\HOLOGO@temp}{%
        \@PackageError{hologo}{%
          Unknown variant `\HOLOGO@temp' of logo `\HOLOGO@name'%
        }\@ehc
      }{%
        \expandafter
        \let\csname HoLogoOpt@variant@\HOLOGO@name\endcsname
            \HOLOGO@temp
      }%
    \fi
  \fi
}
%    \end{macrocode}
%
%    \begin{macro}{\HOLOGO@Variant}
%    \begin{macrocode}
\def\HOLOGO@Variant#1{%
  #1%
  \ltx@ifundefined{HoLogoOpt@variant@#1}{%
  }{%
    @\csname HoLogoOpt@variant@#1\endcsname
  }%
}
%    \end{macrocode}
%    \end{macro}
%
% \subsection{Break/no-break support}
%
%    \begin{macro}{\HOLOGO@space}
%    \begin{macrocode}
\def\HOLOGO@space{%
  \ltx@ifundefined{HoLogoOpt@spacebreak@\HOLOGO@name}{%
    \ltx@ifundefined{HoLogoOpt@break@\HOLOGO@name}{%
      \chardef\HOLOGO@temp=\HOLOGOOPT@spacebreak
    }{%
      \chardef\HOLOGO@temp=%
        \csname HoLogoOpt@break@\HOLOGO@name\endcsname
    }%
  }{%
    \chardef\HOLOGO@temp=%
      \csname HoLogoOpt@spacebreak@\HOLOGO@name\endcsname
  }%
  \ifcase\HOLOGO@temp
    \penalty10000 %
  \fi
  \ltx@space
}
%    \end{macrocode}
%    \end{macro}
%
%    \begin{macro}{\HOLOGO@hyphen}
%    \begin{macrocode}
\def\HOLOGO@hyphen{%
  \ltx@ifundefined{HoLogoOpt@hyphenbreak@\HOLOGO@name}{%
    \ltx@ifundefined{HoLogoOpt@break@\HOLOGO@name}{%
      \chardef\HOLOGO@temp=\HOLOGOOPT@hyphenbreak
    }{%
      \chardef\HOLOGO@temp=%
        \csname HoLogoOpt@break@\HOLOGO@name\endcsname
    }%
  }{%
    \chardef\HOLOGO@temp=%
      \csname HoLogoOpt@hyphenbreak@\HOLOGO@name\endcsname
  }%
  \ifcase\HOLOGO@temp
    \ltx@mbox{-}%
  \else
    -%
  \fi
}
%    \end{macrocode}
%    \end{macro}
%
%    \begin{macro}{\HOLOGO@discretionary}
%    \begin{macrocode}
\def\HOLOGO@discretionary{%
  \ltx@ifundefined{HoLogoOpt@discretionarybreak@\HOLOGO@name}{%
    \ltx@ifundefined{HoLogoOpt@break@\HOLOGO@name}{%
      \chardef\HOLOGO@temp=\HOLOGOOPT@discretionarybreak
    }{%
      \chardef\HOLOGO@temp=%
        \csname HoLogoOpt@break@\HOLOGO@name\endcsname
    }%
  }{%
    \chardef\HOLOGO@temp=%
      \csname HoLogoOpt@discretionarybreak@\HOLOGO@name\endcsname
  }%
  \ifcase\HOLOGO@temp
  \else
    \-%
  \fi
}
%    \end{macrocode}
%    \end{macro}
%
%    \begin{macro}{\HOLOGO@mbox}
%    \begin{macrocode}
\def\HOLOGO@mbox#1{%
  \ltx@ifundefined{HoLogoOpt@break@\HOLOGO@name}{%
    \chardef\HOLOGO@temp=\HOLOGOOPT@hyphenbreak
  }{%
    \chardef\HOLOGO@temp=%
      \csname HoLogoOpt@break@\HOLOGO@name\endcsname
  }%
  \ifcase\HOLOGO@temp
    \ltx@mbox{#1}%
  \else
    #1%
  \fi
}
%    \end{macrocode}
%    \end{macro}
%
% \subsection{Font support}
%
%    \begin{macro}{\HoLogoFont@font}
%    \begin{tabular}{@{}ll@{}}
%    |#1|:& logo name\\
%    |#2|:& font short name\\
%    |#3|:& text
%    \end{tabular}
%    \begin{macrocode}
\def\HoLogoFont@font#1#2#3{%
  \begingroup
    \ltx@IfUndefined{HoLogoFont@logo@#1.#2}{%
      \ltx@IfUndefined{HoLogoFont@font@#2}{%
        \@PackageWarning{hologo}{%
          Missing font `#2' for logo `#1'%
        }%
        #3%
      }{%
        \csname HoLogoFont@font@#2\endcsname{#3}%
      }%
    }{%
      \csname HoLogoFont@logo@#1.#2\endcsname{#3}%
    }%
  \endgroup
}
%    \end{macrocode}
%    \end{macro}
%
%    \begin{macro}{\HoLogoFont@Def}
%    \begin{macrocode}
\def\HoLogoFont@Def#1{%
  \expandafter\def\csname HoLogoFont@font@#1\endcsname
}
%    \end{macrocode}
%    \end{macro}
%    \begin{macro}{\HoLogoFont@LogoDef}
%    \begin{macrocode}
\def\HoLogoFont@LogoDef#1#2{%
  \expandafter\def\csname HoLogoFont@logo@#1.#2\endcsname
}
%    \end{macrocode}
%    \end{macro}
%
% \subsubsection{Font defaults}
%
%    \begin{macro}{\HoLogoFont@font@general}
%    \begin{macrocode}
\HoLogoFont@Def{general}{}%
%    \end{macrocode}
%    \end{macro}
%
%    \begin{macro}{\HoLogoFont@font@rm}
%    \begin{macrocode}
\ltx@IfUndefined{rmfamily}{%
  \ltx@IfUndefined{rm}{%
  }{%
    \HoLogoFont@Def{rm}{\rm}%
  }%
}{%
  \HoLogoFont@Def{rm}{\rmfamily}%
}
%    \end{macrocode}
%    \end{macro}
%
%    \begin{macro}{\HoLogoFont@font@sf}
%    \begin{macrocode}
\ltx@IfUndefined{sffamily}{%
  \ltx@IfUndefined{sf}{%
  }{%
    \HoLogoFont@Def{sf}{\sf}%
  }%
}{%
  \HoLogoFont@Def{sf}{\sffamily}%
}
%    \end{macrocode}
%    \end{macro}
%
%    \begin{macro}{\HoLogoFont@font@bibsf}
%    In case of \hologo{plainTeX} the original small caps
%    variant is used as default. In \hologo{LaTeX}
%    the definition of package \xpackage{dtklogos} \cite{dtklogos}
%    is used.
%\begin{quote}
%\begin{verbatim}
%\DeclareRobustCommand{\BibTeX}{%
%  B%
%  \kern-.05em%
%  \hbox{%
%    $\m@th$% %% force math size calculations
%    \csname S@\f@size\endcsname
%    \fontsize\sf@size\z@
%    \math@fontsfalse
%    \selectfont
%    I%
%    \kern-.025em%
%    B
%  }%
%  \kern-.08em%
%  \-%
%  \TeX
%}
%\end{verbatim}
%\end{quote}
%    \begin{macrocode}
\ltx@IfUndefined{selectfont}{%
  \ltx@IfUndefined{tensc}{%
    \font\tensc=cmcsc10\relax
  }{}%
  \HoLogoFont@Def{bibsf}{\tensc}%
}{%
  \HoLogoFont@Def{bibsf}{%
    $\mathsurround=0pt$%
    \csname S@\f@size\endcsname
    \fontsize\sf@size{0pt}%
    \math@fontsfalse
    \selectfont
  }%
}
%    \end{macrocode}
%    \end{macro}
%
%    \begin{macro}{\HoLogoFont@font@sc}
%    \begin{macrocode}
\ltx@IfUndefined{scshape}{%
  \ltx@IfUndefined{tensc}{%
    \font\tensc=cmcsc10\relax
  }{}%
  \HoLogoFont@Def{sc}{\tensc}%
}{%
  \HoLogoFont@Def{sc}{\scshape}%
}
%    \end{macrocode}
%    \end{macro}
%
%    \begin{macro}{\HoLogoFont@font@sy}
%    \begin{macrocode}
\ltx@IfUndefined{usefont}{%
  \ltx@IfUndefined{tensy}{%
  }{%
    \HoLogoFont@Def{sy}{\tensy}%
  }%
}{%
  \HoLogoFont@Def{sy}{%
    \usefont{OMS}{cmsy}{m}{n}%
  }%
}
%    \end{macrocode}
%    \end{macro}
%
%    \begin{macro}{\HoLogoFont@font@logo}
%    \begin{macrocode}
\begingroup
  \def\x{LaTeX2e}%
\expandafter\endgroup
\ifx\fmtname\x
  \ltx@IfUndefined{logofamily}{%
    \DeclareRobustCommand\logofamily{%
      \not@math@alphabet\logofamily\relax
      \fontencoding{U}%
      \fontfamily{logo}%
      \selectfont
    }%
  }{}%
  \ltx@IfUndefined{logofamily}{%
  }{%
    \HoLogoFont@Def{logo}{\logofamily}%
  }%
\else
  \ltx@IfUndefined{tenlogo}{%
    \font\tenlogo=logo10\relax
  }{}%
  \HoLogoFont@Def{logo}{\tenlogo}%
\fi
%    \end{macrocode}
%    \end{macro}
%
% \subsubsection{Font setup}
%
%    \begin{macro}{\hologoFontSetup}
%    \begin{macrocode}
\def\hologoFontSetup{%
  \let\HOLOGO@name\relax
  \HOLOGO@FontSetup
}
%    \end{macrocode}
%    \end{macro}
%
%    \begin{macro}{\hologoLogoFontSetup}
%    \begin{macrocode}
\def\hologoLogoFontSetup#1{%
  \edef\HOLOGO@name{#1}%
  \ltx@IfUndefined{HoLogo@\HOLOGO@name}{%
    \@PackageError{hologo}{%
      Unknown logo `\HOLOGO@name'%
    }\@ehc
    \ltx@gobble
  }{%
    \HOLOGO@FontSetup
  }%
}
%    \end{macrocode}
%    \end{macro}
%
%    \begin{macro}{\HOLOGO@FontSetup}
%    \begin{macrocode}
\def\HOLOGO@FontSetup{%
  \kvsetkeys{HoLogoFont}%
}
%    \end{macrocode}
%    \end{macro}
%
%    \begin{macrocode}
\def\HOLOGO@temp#1{%
  \kv@define@key{HoLogoFont}{#1}{%
    \ifx\HOLOGO@name\relax
      \HoLogoFont@Def{#1}{##1}%
    \else
      \HoLogoFont@LogoDef\HOLOGO@name{#1}{##1}%
    \fi
  }%
}
\HOLOGO@temp{general}
\HOLOGO@temp{sf}
%    \end{macrocode}
%
% \subsection{Generic logo commands}
%
%    \begin{macrocode}
\HOLOGO@IfExists\hologo{%
  \@PackageError{hologo}{%
    \string\hologo\ltx@space is already defined.\MessageBreak
    Package loading is aborted%
  }\@ehc
  \HOLOGO@AtEnd
}%
\HOLOGO@IfExists\hologoRobust{%
  \@PackageError{hologo}{%
    \string\hologoRobust\ltx@space is already defined.\MessageBreak
    Package loading is aborted%
  }\@ehc
  \HOLOGO@AtEnd
}%
%    \end{macrocode}
%
% \subsubsection{\cs{hologo} and friends}
%
%    \begin{macrocode}
\ifluatex
  \expandafter\ltx@firstofone
\else
  \expandafter\ltx@gobble
\fi
{%
  \ltx@IfUndefined{ifincsname}{%
    \ifnum\luatexversion<36 %
      \expandafter\ltx@gobble
    \else
      \expandafter\ltx@firstofone
    \fi
    {%
      \begingroup
        \ifcase0%
            \directlua{%
              if tex.enableprimitives then %
                tex.enableprimitives('HOLOGO@', {'ifincsname'})%
              else %
                tex.print('1')%
              end%
            }%
            \ifx\HOLOGO@ifincsname\@undefined 1\fi%
            \relax
          \expandafter\ltx@firstofone
        \else
          \endgroup
          \expandafter\ltx@gobble
        \fi
        {%
          \global\let\ifincsname\HOLOGO@ifincsname
        }%
      \HOLOGO@temp
    }%
  }{}%
}
%    \end{macrocode}
%    \begin{macrocode}
\ltx@IfUndefined{ifincsname}{%
  \catcode`$=14 %
}{%
  \catcode`$=9 %
}
%    \end{macrocode}
%
%    \begin{macro}{\hologo}
%    \begin{macrocode}
\def\hologo#1{%
$ \ifincsname
$   \ltx@ifundefined{HoLogoCs@\HOLOGO@Variant{#1}}{%
$     #1%
$   }{%
$     \csname HoLogoCs@\HOLOGO@Variant{#1}\endcsname\ltx@firstoftwo
$   }%
$ \else
    \HOLOGO@IfExists\texorpdfstring\texorpdfstring\ltx@firstoftwo
    {%
      \hologoRobust{#1}%
    }{%
      \ltx@ifundefined{HoLogoBkm@\HOLOGO@Variant{#1}}{%
        \ltx@ifundefined{HoLogo@#1}{?#1?}{#1}%
      }{%
        \csname HoLogoBkm@\HOLOGO@Variant{#1}\endcsname
        \ltx@firstoftwo
      }%
    }%
$ \fi
}
%    \end{macrocode}
%    \end{macro}
%    \begin{macro}{\Hologo}
%    \begin{macrocode}
\def\Hologo#1{%
$ \ifincsname
$   \ltx@ifundefined{HoLogoCs@\HOLOGO@Variant{#1}}{%
$     #1%
$   }{%
$     \csname HoLogoCs@\HOLOGO@Variant{#1}\endcsname\ltx@secondoftwo
$   }%
$ \else
    \HOLOGO@IfExists\texorpdfstring\texorpdfstring\ltx@firstoftwo
    {%
      \HologoRobust{#1}%
    }{%
      \ltx@ifundefined{HoLogoBkm@\HOLOGO@Variant{#1}}{%
        \ltx@ifundefined{HoLogo@#1}{?#1?}{#1}%
      }{%
        \csname HoLogoBkm@\HOLOGO@Variant{#1}\endcsname
        \ltx@secondoftwo
      }%
    }%
$ \fi
}
%    \end{macrocode}
%    \end{macro}
%
%    \begin{macro}{\hologoVariant}
%    \begin{macrocode}
\def\hologoVariant#1#2{%
  \ifx\relax#2\relax
    \hologo{#1}%
  \else
$   \ifincsname
$     \ltx@ifundefined{HoLogoCs@#1@#2}{%
$       #1%
$     }{%
$       \csname HoLogoCs@#1@#2\endcsname\ltx@firstoftwo
$     }%
$   \else
      \HOLOGO@IfExists\texorpdfstring\texorpdfstring\ltx@firstoftwo
      {%
        \hologoVariantRobust{#1}{#2}%
      }{%
        \ltx@ifundefined{HoLogoBkm@#1@#2}{%
          \ltx@ifundefined{HoLogo@#1}{?#1?}{#1}%
        }{%
          \csname HoLogoBkm@#1@#2\endcsname
          \ltx@firstoftwo
        }%
      }%
$   \fi
  \fi
}
%    \end{macrocode}
%    \end{macro}
%    \begin{macro}{\HologoVariant}
%    \begin{macrocode}
\def\HologoVariant#1#2{%
  \ifx\relax#2\relax
    \Hologo{#1}%
  \else
$   \ifincsname
$     \ltx@ifundefined{HoLogoCs@#1@#2}{%
$       #1%
$     }{%
$       \csname HoLogoCs@#1@#2\endcsname\ltx@secondoftwo
$     }%
$   \else
      \HOLOGO@IfExists\texorpdfstring\texorpdfstring\ltx@firstoftwo
      {%
        \HologoVariantRobust{#1}{#2}%
      }{%
        \ltx@ifundefined{HoLogoBkm@#1@#2}{%
          \ltx@ifundefined{HoLogo@#1}{?#1?}{#1}%
        }{%
          \csname HoLogoBkm@#1@#2\endcsname
          \ltx@secondoftwo
        }%
      }%
$   \fi
  \fi
}
%    \end{macrocode}
%    \end{macro}
%
%    \begin{macrocode}
\catcode`\$=3 %
%    \end{macrocode}
%
% \subsubsection{\cs{hologoRobust} and friends}
%
%    \begin{macro}{\hologoRobust}
%    \begin{macrocode}
\ltx@IfUndefined{protected}{%
  \ltx@IfUndefined{DeclareRobustCommand}{%
    \def\hologoRobust#1%
  }{%
    \DeclareRobustCommand*\hologoRobust[1]%
  }%
}{%
  \protected\def\hologoRobust#1%
}%
{%
  \edef\HOLOGO@name{#1}%
  \ltx@IfUndefined{HoLogo@\HOLOGO@Variant\HOLOGO@name}{%
    \@PackageError{hologo}{%
      Unknown logo `\HOLOGO@name'%
    }\@ehc
    ?\HOLOGO@name?%
  }{%
    \ltx@IfUndefined{ver@tex4ht.sty}{%
      \HoLogoFont@font\HOLOGO@name{general}{%
        \csname HoLogo@\HOLOGO@Variant\HOLOGO@name\endcsname
        \ltx@firstoftwo
      }%
    }{%
      \ltx@IfUndefined{HoLogoHtml@\HOLOGO@Variant\HOLOGO@name}{%
        \HOLOGO@name
      }{%
        \csname HoLogoHtml@\HOLOGO@Variant\HOLOGO@name\endcsname
        \ltx@firstoftwo
      }%
    }%
  }%
}
%    \end{macrocode}
%    \end{macro}
%    \begin{macro}{\HologoRobust}
%    \begin{macrocode}
\ltx@IfUndefined{protected}{%
  \ltx@IfUndefined{DeclareRobustCommand}{%
    \def\HologoRobust#1%
  }{%
    \DeclareRobustCommand*\HologoRobust[1]%
  }%
}{%
  \protected\def\HologoRobust#1%
}%
{%
  \edef\HOLOGO@name{#1}%
  \ltx@IfUndefined{HoLogo@\HOLOGO@Variant\HOLOGO@name}{%
    \@PackageError{hologo}{%
      Unknown logo `\HOLOGO@name'%
    }\@ehc
    ?\HOLOGO@name?%
  }{%
    \ltx@IfUndefined{ver@tex4ht.sty}{%
      \HoLogoFont@font\HOLOGO@name{general}{%
        \csname HoLogo@\HOLOGO@Variant\HOLOGO@name\endcsname
        \ltx@secondoftwo
      }%
    }{%
      \ltx@IfUndefined{HoLogoHtml@\HOLOGO@Variant\HOLOGO@name}{%
        \expandafter\HOLOGO@Uppercase\HOLOGO@name
      }{%
        \csname HoLogoHtml@\HOLOGO@Variant\HOLOGO@name\endcsname
        \ltx@secondoftwo
      }%
    }%
  }%
}
%    \end{macrocode}
%    \end{macro}
%    \begin{macro}{\hologoVariantRobust}
%    \begin{macrocode}
\ltx@IfUndefined{protected}{%
  \ltx@IfUndefined{DeclareRobustCommand}{%
    \def\hologoVariantRobust#1#2%
  }{%
    \DeclareRobustCommand*\hologoVariantRobust[2]%
  }%
}{%
  \protected\def\hologoVariantRobust#1#2%
}%
{%
  \begingroup
    \hologoLogoSetup{#1}{variant={#2}}%
    \hologoRobust{#1}%
  \endgroup
}
%    \end{macrocode}
%    \end{macro}
%    \begin{macro}{\HologoVariantRobust}
%    \begin{macrocode}
\ltx@IfUndefined{protected}{%
  \ltx@IfUndefined{DeclareRobustCommand}{%
    \def\HologoVariantRobust#1#2%
  }{%
    \DeclareRobustCommand*\HologoVariantRobust[2]%
  }%
}{%
  \protected\def\HologoVariantRobust#1#2%
}%
{%
  \begingroup
    \hologoLogoSetup{#1}{variant={#2}}%
    \HologoRobust{#1}%
  \endgroup
}
%    \end{macrocode}
%    \end{macro}
%
%    \begin{macro}{\hologorobust}
%    Macro \cs{hologorobust} is only defined for compatibility.
%    Its use is deprecated.
%    \begin{macrocode}
\def\hologorobust{\hologoRobust}
%    \end{macrocode}
%    \end{macro}
%
% \subsection{Helpers}
%
%    \begin{macro}{\HOLOGO@Uppercase}
%    Macro \cs{HOLOGO@Uppercase} is restricted to \cs{uppercase},
%    because \hologo{plainTeX} or \hologo{iniTeX} do not provide
%    \cs{MakeUppercase}.
%    \begin{macrocode}
\def\HOLOGO@Uppercase#1{\uppercase{#1}}
%    \end{macrocode}
%    \end{macro}
%
%    \begin{macro}{\HOLOGO@PdfdocUnicode}
%    \begin{macrocode}
\def\HOLOGO@PdfdocUnicode{%
  \ifx\ifHy@unicode\iftrue
    \expandafter\ltx@secondoftwo
  \else
    \expandafter\ltx@firstoftwo
  \fi
}
%    \end{macrocode}
%    \end{macro}
%
%    \begin{macro}{\HOLOGO@Math}
%    \begin{macrocode}
\def\HOLOGO@MathSetup{%
  \mathsurround0pt\relax
  \HOLOGO@IfExists\f@series{%
    \if b\expandafter\ltx@car\f@series x\@nil
      \csname boldmath\endcsname
   \fi
  }{}%
}
%    \end{macrocode}
%    \end{macro}
%
%    \begin{macro}{\HOLOGO@TempDimen}
%    \begin{macrocode}
\dimendef\HOLOGO@TempDimen=\ltx@zero
%    \end{macrocode}
%    \end{macro}
%    \begin{macro}{\HOLOGO@NegativeKerning}
%    \begin{macrocode}
\def\HOLOGO@NegativeKerning#1{%
  \begingroup
    \HOLOGO@TempDimen=0pt\relax
    \comma@parse@normalized{#1}{%
      \ifdim\HOLOGO@TempDimen=0pt %
        \expandafter\HOLOGO@@NegativeKerning\comma@entry
      \fi
      \ltx@gobble
    }%
    \ifdim\HOLOGO@TempDimen<0pt %
      \kern\HOLOGO@TempDimen
    \fi
  \endgroup
}
%    \end{macrocode}
%    \end{macro}
%    \begin{macro}{\HOLOGO@@NegativeKerning}
%    \begin{macrocode}
\def\HOLOGO@@NegativeKerning#1#2{%
  \setbox\ltx@zero\hbox{#1#2}%
  \HOLOGO@TempDimen=\wd\ltx@zero
  \setbox\ltx@zero\hbox{#1\kern0pt#2}%
  \advance\HOLOGO@TempDimen by -\wd\ltx@zero
}
%    \end{macrocode}
%    \end{macro}
%
%    \begin{macro}{\HOLOGO@SpaceFactor}
%    \begin{macrocode}
\def\HOLOGO@SpaceFactor{%
  \spacefactor1000 %
}
%    \end{macrocode}
%    \end{macro}
%
%    \begin{macro}{\HOLOGO@Span}
%    \begin{macrocode}
\def\HOLOGO@Span#1#2{%
  \HCode{<span class="HoLogo-#1">}%
  #2%
  \HCode{</span>}%
}
%    \end{macrocode}
%    \end{macro}
%
% \subsubsection{Text subscript}
%
%    \begin{macro}{\HOLOGO@SubScript}%
%    \begin{macrocode}
\def\HOLOGO@SubScript#1{%
  \ltx@IfUndefined{textsubscript}{%
    \ltx@IfUndefined{text}{%
      \ltx@mbox{%
        \mathsurround=0pt\relax
        $%
          _{%
            \ltx@IfUndefined{sf@size}{%
              \mathrm{#1}%
            }{%
              \mbox{%
                \fontsize\sf@size{0pt}\selectfont
                #1%
              }%
            }%
          }%
        $%
      }%
    }{%
      \ltx@mbox{%
        \mathsurround=0pt\relax
        $_{\text{#1}}$%
      }%
    }%
  }{%
    \textsubscript{#1}%
  }%
}
%    \end{macrocode}
%    \end{macro}
%
% \subsection{\hologo{TeX} and friends}
%
% \subsubsection{\hologo{TeX}}
%
%    \begin{macro}{\HoLogo@TeX}
%    Source: \hologo{LaTeX} kernel.
%    \begin{macrocode}
\def\HoLogo@TeX#1{%
  T\kern-.1667em\lower.5ex\hbox{E}\kern-.125emX\HOLOGO@SpaceFactor
}
%    \end{macrocode}
%    \end{macro}
%    \begin{macro}{\HoLogoHtml@TeX}
%    \begin{macrocode}
\def\HoLogoHtml@TeX#1{%
  \HoLogoCss@TeX
  \HOLOGO@Span{TeX}{%
    T%
    \HOLOGO@Span{e}{%
      E%
    }%
    X%
  }%
}
%    \end{macrocode}
%    \end{macro}
%    \begin{macro}{\HoLogoCss@TeX}
%    \begin{macrocode}
\def\HoLogoCss@TeX{%
  \Css{%
    span.HoLogo-TeX span.HoLogo-e{%
      position:relative;%
      top:.5ex;%
      margin-left:-.1667em;%
      margin-right:-.125em;%
    }%
  }%
  \Css{%
    a span.HoLogo-TeX span.HoLogo-e{%
      text-decoration:none;%
    }%
  }%
  \global\let\HoLogoCss@TeX\relax
}
%    \end{macrocode}
%    \end{macro}
%
% \subsubsection{\hologo{plainTeX}}
%
%    \begin{macro}{\HoLogo@plainTeX@space}
%    Source: ``The \hologo{TeX}book''
%    \begin{macrocode}
\def\HoLogo@plainTeX@space#1{%
  \HOLOGO@mbox{#1{p}{P}lain}\HOLOGO@space\hologo{TeX}%
}
%    \end{macrocode}
%    \end{macro}
%    \begin{macro}{\HoLogoCs@plainTeX@space}
%    \begin{macrocode}
\def\HoLogoCs@plainTeX@space#1{#1{p}{P}lain TeX}%
%    \end{macrocode}
%    \end{macro}
%    \begin{macro}{\HoLogoBkm@plainTeX@space}
%    \begin{macrocode}
\def\HoLogoBkm@plainTeX@space#1{%
  #1{p}{P}lain \hologo{TeX}%
}
%    \end{macrocode}
%    \end{macro}
%    \begin{macro}{\HoLogoHtml@plainTeX@space}
%    \begin{macrocode}
\def\HoLogoHtml@plainTeX@space#1{%
  #1{p}{P}lain \hologo{TeX}%
}
%    \end{macrocode}
%    \end{macro}
%
%    \begin{macro}{\HoLogo@plainTeX@hyphen}
%    \begin{macrocode}
\def\HoLogo@plainTeX@hyphen#1{%
  \HOLOGO@mbox{#1{p}{P}lain}\HOLOGO@hyphen\hologo{TeX}%
}
%    \end{macrocode}
%    \end{macro}
%    \begin{macro}{\HoLogoCs@plainTeX@hyphen}
%    \begin{macrocode}
\def\HoLogoCs@plainTeX@hyphen#1{#1{p}{P}lain-TeX}
%    \end{macrocode}
%    \end{macro}
%    \begin{macro}{\HoLogoBkm@plainTeX@hyphen}
%    \begin{macrocode}
\def\HoLogoBkm@plainTeX@hyphen#1{%
  #1{p}{P}lain-\hologo{TeX}%
}
%    \end{macrocode}
%    \end{macro}
%    \begin{macro}{\HoLogoHtml@plainTeX@hyphen}
%    \begin{macrocode}
\def\HoLogoHtml@plainTeX@hyphen#1{%
  #1{p}{P}lain-\hologo{TeX}%
}
%    \end{macrocode}
%    \end{macro}
%
%    \begin{macro}{\HoLogo@plainTeX@runtogether}
%    \begin{macrocode}
\def\HoLogo@plainTeX@runtogether#1{%
  \HOLOGO@mbox{#1{p}{P}lain\hologo{TeX}}%
}
%    \end{macrocode}
%    \end{macro}
%    \begin{macro}{\HoLogoCs@plainTeX@runtogether}
%    \begin{macrocode}
\def\HoLogoCs@plainTeX@runtogether#1{#1{p}{P}lainTeX}
%    \end{macrocode}
%    \end{macro}
%    \begin{macro}{\HoLogoBkm@plainTeX@runtogether}
%    \begin{macrocode}
\def\HoLogoBkm@plainTeX@runtogether#1{%
  #1{p}{P}lain\hologo{TeX}%
}
%    \end{macrocode}
%    \end{macro}
%    \begin{macro}{\HoLogoHtml@plainTeX@runtogether}
%    \begin{macrocode}
\def\HoLogoHtml@plainTeX@runtogether#1{%
  #1{p}{P}lain\hologo{TeX}%
}
%    \end{macrocode}
%    \end{macro}
%
%    \begin{macro}{\HoLogo@plainTeX}
%    \begin{macrocode}
\def\HoLogo@plainTeX{\HoLogo@plainTeX@space}
%    \end{macrocode}
%    \end{macro}
%    \begin{macro}{\HoLogoCs@plainTeX}
%    \begin{macrocode}
\def\HoLogoCs@plainTeX{\HoLogoCs@plainTeX@space}
%    \end{macrocode}
%    \end{macro}
%    \begin{macro}{\HoLogoBkm@plainTeX}
%    \begin{macrocode}
\def\HoLogoBkm@plainTeX{\HoLogoBkm@plainTeX@space}
%    \end{macrocode}
%    \end{macro}
%    \begin{macro}{\HoLogoHtml@plainTeX}
%    \begin{macrocode}
\def\HoLogoHtml@plainTeX{\HoLogoHtml@plainTeX@space}
%    \end{macrocode}
%    \end{macro}
%
% \subsubsection{\hologo{LaTeX}}
%
%    Source: \hologo{LaTeX} kernel.
%\begin{quote}
%\begin{verbatim}
%\DeclareRobustCommand{\LaTeX}{%
%  L%
%  \kern-.36em%
%  {%
%    \sbox\z@ T%
%    \vbox to\ht\z@{%
%      \hbox{%
%        \check@mathfonts
%        \fontsize\sf@size\z@
%        \math@fontsfalse
%        \selectfont
%        A%
%      }%
%      \vss
%    }%
%  }%
%  \kern-.15em%
%  \TeX
%}
%\end{verbatim}
%\end{quote}
%
%    \begin{macro}{\HoLogo@La}
%    \begin{macrocode}
\def\HoLogo@La#1{%
  L%
  \kern-.36em%
  \begingroup
    \setbox\ltx@zero\hbox{T}%
    \vbox to\ht\ltx@zero{%
      \hbox{%
        \ltx@ifundefined{check@mathfonts}{%
          \csname sevenrm\endcsname
        }{%
          \check@mathfonts
          \fontsize\sf@size{0pt}%
          \math@fontsfalse\selectfont
        }%
        A%
      }%
      \vss
    }%
  \endgroup
}
%    \end{macrocode}
%    \end{macro}
%
%    \begin{macro}{\HoLogo@LaTeX}
%    Source: \hologo{LaTeX} kernel.
%    \begin{macrocode}
\def\HoLogo@LaTeX#1{%
  \hologo{La}%
  \kern-.15em%
  \hologo{TeX}%
}
%    \end{macrocode}
%    \end{macro}
%    \begin{macro}{\HoLogoHtml@LaTeX}
%    \begin{macrocode}
\def\HoLogoHtml@LaTeX#1{%
  \HoLogoCss@LaTeX
  \HOLOGO@Span{LaTeX}{%
    L%
    \HOLOGO@Span{a}{%
      A%
    }%
    \hologo{TeX}%
  }%
}
%    \end{macrocode}
%    \end{macro}
%    \begin{macro}{\HoLogoCss@LaTeX}
%    \begin{macrocode}
\def\HoLogoCss@LaTeX{%
  \Css{%
    span.HoLogo-LaTeX span.HoLogo-a{%
      position:relative;%
      top:-.5ex;%
      margin-left:-.36em;%
      margin-right:-.15em;%
      font-size:85\%;%
    }%
  }%
  \global\let\HoLogoCss@LaTeX\relax
}
%    \end{macrocode}
%    \end{macro}
%
% \subsubsection{\hologo{(La)TeX}}
%
%    \begin{macro}{\HoLogo@LaTeXTeX}
%    The kerning around the parentheses is taken
%    from package \xpackage{dtklogos} \cite{dtklogos}.
%\begin{quote}
%\begin{verbatim}
%\DeclareRobustCommand{\LaTeXTeX}{%
%  (%
%  \kern-.15em%
%  L%
%  \kern-.36em%
%  {%
%    \sbox\z@ T%
%    \vbox to\ht0{%
%      \hbox{%
%        $\m@th$%
%        \csname S@\f@size\endcsname
%        \fontsize\sf@size\z@
%        \math@fontsfalse
%        \selectfont
%        A%
%      }%
%      \vss
%    }%
%  }%
%  \kern-.2em%
%  )%
%  \kern-.15em%
%  \TeX
%}
%\end{verbatim}
%\end{quote}
%    \begin{macrocode}
\def\HoLogo@LaTeXTeX#1{%
  (%
  \kern-.15em%
  \hologo{La}%
  \kern-.2em%
  )%
  \kern-.15em%
  \hologo{TeX}%
}
%    \end{macrocode}
%    \end{macro}
%    \begin{macro}{\HoLogoBkm@LaTeXTeX}
%    \begin{macrocode}
\def\HoLogoBkm@LaTeXTeX#1{(La)TeX}
%    \end{macrocode}
%    \end{macro}
%
%    \begin{macro}{\HoLogo@(La)TeX}
%    \begin{macrocode}
\expandafter
\let\csname HoLogo@(La)TeX\endcsname\HoLogo@LaTeXTeX
%    \end{macrocode}
%    \end{macro}
%    \begin{macro}{\HoLogoBkm@(La)TeX}
%    \begin{macrocode}
\expandafter
\let\csname HoLogoBkm@(La)TeX\endcsname\HoLogoBkm@LaTeXTeX
%    \end{macrocode}
%    \end{macro}
%    \begin{macro}{\HoLogoHtml@LaTeXTeX}
%    \begin{macrocode}
\def\HoLogoHtml@LaTeXTeX#1{%
  \HoLogoCss@LaTeXTeX
  \HOLOGO@Span{LaTeXTeX}{%
    (%
    \HOLOGO@Span{L}{L}%
    \HOLOGO@Span{a}{A}%
    \HOLOGO@Span{ParenRight}{)}%
    \hologo{TeX}%
  }%
}
%    \end{macrocode}
%    \end{macro}
%    \begin{macro}{\HoLogoHtml@(La)TeX}
%    Kerning after opening parentheses and before closing parentheses
%    is $-0.1$\,em. The original values $-0.15$\,em
%    looked too ugly for a serif font.
%    \begin{macrocode}
\expandafter
\let\csname HoLogoHtml@(La)TeX\endcsname\HoLogoHtml@LaTeXTeX
%    \end{macrocode}
%    \end{macro}
%    \begin{macro}{\HoLogoCss@LaTeXTeX}
%    \begin{macrocode}
\def\HoLogoCss@LaTeXTeX{%
  \Css{%
    span.HoLogo-LaTeXTeX span.HoLogo-L{%
      margin-left:-.1em;%
    }%
  }%
  \Css{%
    span.HoLogo-LaTeXTeX span.HoLogo-a{%
      position:relative;%
      top:-.5ex;%
      margin-left:-.36em;%
      margin-right:-.1em;%
      font-size:85\%;%
    }%
  }%
  \Css{%
    span.HoLogo-LaTeXTeX span.HoLogo-ParenRight{%
      margin-right:-.15em;%
    }%
  }%
  \global\let\HoLogoCss@LaTeXTeX\relax
}
%    \end{macrocode}
%    \end{macro}
%
% \subsubsection{\hologo{LaTeXe}}
%
%    \begin{macro}{\HoLogo@LaTeXe}
%    Source: \hologo{LaTeX} kernel
%    \begin{macrocode}
\def\HoLogo@LaTeXe#1{%
  \hologo{LaTeX}%
  \kern.15em%
  \hbox{%
    \HOLOGO@MathSetup
    2%
    $_{\textstyle\varepsilon}$%
  }%
}
%    \end{macrocode}
%    \end{macro}
%
%    \begin{macro}{\HoLogoCs@LaTeXe}
%    \begin{macrocode}
\ifnum64=`\^^^^0040\relax % test for big chars of LuaTeX/XeTeX
  \catcode`\$=9 %
  \catcode`\&=14 %
\else
  \catcode`\$=14 %
  \catcode`\&=9 %
\fi
\def\HoLogoCs@LaTeXe#1{%
  LaTeX2%
$ \string ^^^^0395%
& e%
}%
\catcode`\$=3 %
\catcode`\&=4 %
%    \end{macrocode}
%    \end{macro}
%
%    \begin{macro}{\HoLogoBkm@LaTeXe}
%    \begin{macrocode}
\def\HoLogoBkm@LaTeXe#1{%
  \hologo{LaTeX}%
  2%
  \HOLOGO@PdfdocUnicode{e}{\textepsilon}%
}
%    \end{macrocode}
%    \end{macro}
%
%    \begin{macro}{\HoLogoHtml@LaTeXe}
%    \begin{macrocode}
\def\HoLogoHtml@LaTeXe#1{%
  \HoLogoCss@LaTeXe
  \HOLOGO@Span{LaTeX2e}{%
    \hologo{LaTeX}%
    \HOLOGO@Span{2}{2}%
    \HOLOGO@Span{e}{%
      \HOLOGO@MathSetup
      \ensuremath{\textstyle\varepsilon}%
    }%
  }%
}
%    \end{macrocode}
%    \end{macro}
%    \begin{macro}{\HoLogoCss@LaTeXe}
%    \begin{macrocode}
\def\HoLogoCss@LaTeXe{%
  \Css{%
    span.HoLogo-LaTeX2e span.HoLogo-2{%
      padding-left:.15em;%
    }%
  }%
  \Css{%
    span.HoLogo-LaTeX2e span.HoLogo-e{%
      position:relative;%
      top:.35ex;%
      text-decoration:none;%
    }%
  }%
  \global\let\HoLogoCss@LaTeXe\relax
}
%    \end{macrocode}
%    \end{macro}
%
%    \begin{macro}{\HoLogo@LaTeX2e}
%    \begin{macrocode}
\expandafter
\let\csname HoLogo@LaTeX2e\endcsname\HoLogo@LaTeXe
%    \end{macrocode}
%    \end{macro}
%    \begin{macro}{\HoLogoCs@LaTeX2e}
%    \begin{macrocode}
\expandafter
\let\csname HoLogoCs@LaTeX2e\endcsname\HoLogoCs@LaTeXe
%    \end{macrocode}
%    \end{macro}
%    \begin{macro}{\HoLogoBkm@LaTeX2e}
%    \begin{macrocode}
\expandafter
\let\csname HoLogoBkm@LaTeX2e\endcsname\HoLogoBkm@LaTeXe
%    \end{macrocode}
%    \end{macro}
%    \begin{macro}{\HoLogoHtml@LaTeX2e}
%    \begin{macrocode}
\expandafter
\let\csname HoLogoHtml@LaTeX2e\endcsname\HoLogoHtml@LaTeXe
%    \end{macrocode}
%    \end{macro}
%
% \subsubsection{\hologo{LaTeX3}}
%
%    \begin{macro}{\HoLogo@LaTeX3}
%    Source: \hologo{LaTeX} kernel
%    \begin{macrocode}
\expandafter\def\csname HoLogo@LaTeX3\endcsname#1{%
  \hologo{LaTeX}%
  3%
}
%    \end{macrocode}
%    \end{macro}
%
%    \begin{macro}{\HoLogoBkm@LaTeX3}
%    \begin{macrocode}
\expandafter\def\csname HoLogoBkm@LaTeX3\endcsname#1{%
  \hologo{LaTeX}%
  3%
}
%    \end{macrocode}
%    \end{macro}
%    \begin{macro}{\HoLogoHtml@LaTeX3}
%    \begin{macrocode}
\expandafter
\let\csname HoLogoHtml@LaTeX3\expandafter\endcsname
\csname HoLogo@LaTeX3\endcsname
%    \end{macrocode}
%    \end{macro}
%
% \subsubsection{\hologo{LaTeXML}}
%
%    \begin{macro}{\HoLogo@LaTeXML}
%    \begin{macrocode}
\def\HoLogo@LaTeXML#1{%
  \HOLOGO@mbox{%
    \hologo{La}%
    \kern-.15em%
    T%
    \kern-.1667em%
    \lower.5ex\hbox{E}%
    \kern-.125em%
    \HoLogoFont@font{LaTeXML}{sc}{xml}%
  }%
}
%    \end{macrocode}
%    \end{macro}
%    \begin{macro}{\HoLogoHtml@pdfLaTeX}
%    \begin{macrocode}
\def\HoLogoHtml@LaTeXML#1{%
  \HOLOGO@Span{LaTeXML}{%
    \HoLogoCss@LaTeX
    \HoLogoCss@TeX
    \HOLOGO@Span{LaTeX}{%
      L%
      \HOLOGO@Span{a}{%
        A%
      }%
    }%
    \HOLOGO@Span{TeX}{%
      T%
      \HOLOGO@Span{e}{%
        E%
      }%
    }%
    \HCode{<span style="font-variant: small-caps;">}%
    xml%
    \HCode{</span>}%
  }%
}
%    \end{macrocode}
%    \end{macro}
%
% \subsubsection{\hologo{eTeX}}
%
%    \begin{macro}{\HoLogo@eTeX}
%    Source: package \xpackage{etex}
%    \begin{macrocode}
\def\HoLogo@eTeX#1{%
  \ltx@mbox{%
    \HOLOGO@MathSetup
    $\varepsilon$%
    -%
    \HOLOGO@NegativeKerning{-T,T-,To}%
    \hologo{TeX}%
  }%
}
%    \end{macrocode}
%    \end{macro}
%    \begin{macro}{\HoLogoCs@eTeX}
%    \begin{macrocode}
\ifnum64=`\^^^^0040\relax % test for big chars of LuaTeX/XeTeX
  \catcode`\$=9 %
  \catcode`\&=14 %
\else
  \catcode`\$=14 %
  \catcode`\&=9 %
\fi
\def\HoLogoCs@eTeX#1{%
$ #1{\string ^^^^0395}{\string ^^^^03b5}%
& #1{e}{E}%
  TeX%
}%
\catcode`\$=3 %
\catcode`\&=4 %
%    \end{macrocode}
%    \end{macro}
%    \begin{macro}{\HoLogoBkm@eTeX}
%    \begin{macrocode}
\def\HoLogoBkm@eTeX#1{%
  \HOLOGO@PdfdocUnicode{#1{e}{E}}{\textepsilon}%
  -%
  \hologo{TeX}%
}
%    \end{macrocode}
%    \end{macro}
%    \begin{macro}{\HoLogoHtml@eTeX}
%    \begin{macrocode}
\def\HoLogoHtml@eTeX#1{%
  \ltx@mbox{%
    \HOLOGO@MathSetup
    $\varepsilon$%
    -%
    \hologo{TeX}%
  }%
}
%    \end{macrocode}
%    \end{macro}
%
% \subsubsection{\hologo{iniTeX}}
%
%    \begin{macro}{\HoLogo@iniTeX}
%    \begin{macrocode}
\def\HoLogo@iniTeX#1{%
  \HOLOGO@mbox{%
    #1{i}{I}ni\hologo{TeX}%
  }%
}
%    \end{macrocode}
%    \end{macro}
%    \begin{macro}{\HoLogoCs@iniTeX}
%    \begin{macrocode}
\def\HoLogoCs@iniTeX#1{#1{i}{I}niTeX}
%    \end{macrocode}
%    \end{macro}
%    \begin{macro}{\HoLogoBkm@iniTeX}
%    \begin{macrocode}
\def\HoLogoBkm@iniTeX#1{%
  #1{i}{I}ni\hologo{TeX}%
}
%    \end{macrocode}
%    \end{macro}
%    \begin{macro}{\HoLogoHtml@iniTeX}
%    \begin{macrocode}
\let\HoLogoHtml@iniTeX\HoLogo@iniTeX
%    \end{macrocode}
%    \end{macro}
%
% \subsubsection{\hologo{virTeX}}
%
%    \begin{macro}{\HoLogo@virTeX}
%    \begin{macrocode}
\def\HoLogo@virTeX#1{%
  \HOLOGO@mbox{%
    #1{v}{V}ir\hologo{TeX}%
  }%
}
%    \end{macrocode}
%    \end{macro}
%    \begin{macro}{\HoLogoCs@virTeX}
%    \begin{macrocode}
\def\HoLogoCs@virTeX#1{#1{v}{V}irTeX}
%    \end{macrocode}
%    \end{macro}
%    \begin{macro}{\HoLogoBkm@virTeX}
%    \begin{macrocode}
\def\HoLogoBkm@virTeX#1{%
  #1{v}{V}ir\hologo{TeX}%
}
%    \end{macrocode}
%    \end{macro}
%    \begin{macro}{\HoLogoHtml@virTeX}
%    \begin{macrocode}
\let\HoLogoHtml@virTeX\HoLogo@virTeX
%    \end{macrocode}
%    \end{macro}
%
% \subsubsection{\hologo{SliTeX}}
%
% \paragraph{Definitions of the three variants.}
%
%    \begin{macro}{\HoLogo@SLiTeX@lift}
%    \begin{macrocode}
\def\HoLogo@SLiTeX@lift#1{%
  \HoLogoFont@font{SliTeX}{rm}{%
    S%
    \kern-.06em%
    L%
    \kern-.18em%
    \raise.32ex\hbox{\HoLogoFont@font{SliTeX}{sc}{i}}%
    \HOLOGO@discretionary
    \kern-.06em%
    \hologo{TeX}%
  }%
}
%    \end{macrocode}
%    \end{macro}
%    \begin{macro}{\HoLogoBkm@SLiTeX@lift}
%    \begin{macrocode}
\def\HoLogoBkm@SLiTeX@lift#1{SLiTeX}
%    \end{macrocode}
%    \end{macro}
%    \begin{macro}{\HoLogoHtml@SLiTeX@lift}
%    \begin{macrocode}
\def\HoLogoHtml@SLiTeX@lift#1{%
  \HoLogoCss@SLiTeX@lift
  \HOLOGO@Span{SLiTeX-lift}{%
    \HoLogoFont@font{SliTeX}{rm}{%
      S%
      \HOLOGO@Span{L}{L}%
      \HOLOGO@Span{i}{i}%
      \hologo{TeX}%
    }%
  }%
}
%    \end{macrocode}
%    \end{macro}
%    \begin{macro}{\HoLogoCss@SLiTeX@lift}
%    \begin{macrocode}
\def\HoLogoCss@SLiTeX@lift{%
  \Css{%
    span.HoLogo-SLiTeX-lift span.HoLogo-L{%
      margin-left:-.06em;%
      margin-right:-.18em;%
    }%
  }%
  \Css{%
    span.HoLogo-SLiTeX-lift span.HoLogo-i{%
      position:relative;%
      top:-.32ex;%
      margin-right:-.06em;%
      font-variant:small-caps;%
    }%
  }%
  \global\let\HoLogoCss@SLiTeX@lift\relax
}
%    \end{macrocode}
%    \end{macro}
%
%    \begin{macro}{\HoLogo@SliTeX@simple}
%    \begin{macrocode}
\def\HoLogo@SliTeX@simple#1{%
  \HoLogoFont@font{SliTeX}{rm}{%
    \ltx@mbox{%
      \HoLogoFont@font{SliTeX}{sc}{Sli}%
    }%
    \HOLOGO@discretionary
    \hologo{TeX}%
  }%
}
%    \end{macrocode}
%    \end{macro}
%    \begin{macro}{\HoLogoBkm@SliTeX@simple}
%    \begin{macrocode}
\def\HoLogoBkm@SliTeX@simple#1{SliTeX}
%    \end{macrocode}
%    \end{macro}
%    \begin{macro}{\HoLogoHtml@SliTeX@simple}
%    \begin{macrocode}
\let\HoLogoHtml@SliTeX@simple\HoLogo@SliTeX@simple
%    \end{macrocode}
%    \end{macro}
%
%    \begin{macro}{\HoLogo@SliTeX@narrow}
%    \begin{macrocode}
\def\HoLogo@SliTeX@narrow#1{%
  \HoLogoFont@font{SliTeX}{rm}{%
    \ltx@mbox{%
      S%
      \kern-.06em%
      \HoLogoFont@font{SliTeX}{sc}{%
        l%
        \kern-.035em%
        i%
      }%
    }%
    \HOLOGO@discretionary
    \kern-.06em%
    \hologo{TeX}%
  }%
}
%    \end{macrocode}
%    \end{macro}
%    \begin{macro}{\HoLogoBkm@SliTeX@narrow}
%    \begin{macrocode}
\def\HoLogoBkm@SliTeX@narrow#1{SliTeX}
%    \end{macrocode}
%    \end{macro}
%    \begin{macro}{\HoLogoHtml@SliTeX@narrow}
%    \begin{macrocode}
\def\HoLogoHtml@SliTeX@narrow#1{%
  \HoLogoCss@SliTeX@narrow
  \HOLOGO@Span{SliTeX-narrow}{%
    \HoLogoFont@font{SliTeX}{rm}{%
      S%
        \HOLOGO@Span{l}{l}%
        \HOLOGO@Span{i}{i}%
      \hologo{TeX}%
    }%
  }%
}
%    \end{macrocode}
%    \end{macro}
%    \begin{macro}{\HoLogoCss@SliTeX@narrow}
%    \begin{macrocode}
\def\HoLogoCss@SliTeX@narrow{%
  \Css{%
    span.HoLogo-SliTeX-narrow span.HoLogo-l{%
      margin-left:-.06em;%
      margin-right:-.035em;%
      font-variant:small-caps;%
    }%
  }%
  \Css{%
    span.HoLogo-SliTeX-narrow span.HoLogo-i{%
      margin-right:-.06em;%
      font-variant:small-caps;%
    }%
  }%
  \global\let\HoLogoCss@SliTeX@narrow\relax
}
%    \end{macrocode}
%    \end{macro}
%
% \paragraph{Macro set completion.}
%
%    \begin{macro}{\HoLogo@SLiTeX@simple}
%    \begin{macrocode}
\def\HoLogo@SLiTeX@simple{\HoLogo@SliTeX@simple}
%    \end{macrocode}
%    \end{macro}
%    \begin{macro}{\HoLogoBkm@SLiTeX@simple}
%    \begin{macrocode}
\def\HoLogoBkm@SLiTeX@simple{\HoLogoBkm@SliTeX@simple}
%    \end{macrocode}
%    \end{macro}
%    \begin{macro}{\HoLogoHtml@SLiTeX@simple}
%    \begin{macrocode}
\def\HoLogoHtml@SLiTeX@simple{\HoLogoHtml@SliTeX@simple}
%    \end{macrocode}
%    \end{macro}
%
%    \begin{macro}{\HoLogo@SLiTeX@narrow}
%    \begin{macrocode}
\def\HoLogo@SLiTeX@narrow{\HoLogo@SliTeX@narrow}
%    \end{macrocode}
%    \end{macro}
%    \begin{macro}{\HoLogoBkm@SLiTeX@narrow}
%    \begin{macrocode}
\def\HoLogoBkm@SLiTeX@narrow{\HoLogoBkm@SliTeX@narrow}
%    \end{macrocode}
%    \end{macro}
%    \begin{macro}{\HoLogoHtml@SLiTeX@narrow}
%    \begin{macrocode}
\def\HoLogoHtml@SLiTeX@narrow{\HoLogoHtml@SliTeX@narrow}
%    \end{macrocode}
%    \end{macro}
%
%    \begin{macro}{\HoLogo@SliTeX@lift}
%    \begin{macrocode}
\def\HoLogo@SliTeX@lift{\HoLogo@SLiTeX@lift}
%    \end{macrocode}
%    \end{macro}
%    \begin{macro}{\HoLogoBkm@SliTeX@lift}
%    \begin{macrocode}
\def\HoLogoBkm@SliTeX@lift{\HoLogoBkm@SLiTeX@lift}
%    \end{macrocode}
%    \end{macro}
%    \begin{macro}{\HoLogoHtml@SliTeX@lift}
%    \begin{macrocode}
\def\HoLogoHtml@SliTeX@lift{\HoLogoHtml@SLiTeX@lift}
%    \end{macrocode}
%    \end{macro}
%
% \paragraph{Defaults.}
%
%    \begin{macro}{\HoLogo@SLiTeX}
%    \begin{macrocode}
\def\HoLogo@SLiTeX{\HoLogo@SLiTeX@lift}
%    \end{macrocode}
%    \end{macro}
%    \begin{macro}{\HoLogoBkm@SLiTeX}
%    \begin{macrocode}
\def\HoLogoBkm@SLiTeX{\HoLogoBkm@SLiTeX@lift}
%    \end{macrocode}
%    \end{macro}
%    \begin{macro}{\HoLogoHtml@SLiTeX}
%    \begin{macrocode}
\def\HoLogoHtml@SLiTeX{\HoLogoHtml@SLiTeX@lift}
%    \end{macrocode}
%    \end{macro}
%
%    \begin{macro}{\HoLogo@SliTeX}
%    \begin{macrocode}
\def\HoLogo@SliTeX{\HoLogo@SliTeX@narrow}
%    \end{macrocode}
%    \end{macro}
%    \begin{macro}{\HoLogoBkm@SliTeX}
%    \begin{macrocode}
\def\HoLogoBkm@SliTeX{\HoLogoBkm@SliTeX@narrow}
%    \end{macrocode}
%    \end{macro}
%    \begin{macro}{\HoLogoHtml@SliTeX}
%    \begin{macrocode}
\def\HoLogoHtml@SliTeX{\HoLogoHtml@SliTeX@narrow}
%    \end{macrocode}
%    \end{macro}
%
% \subsubsection{\hologo{LuaTeX}}
%
%    \begin{macro}{\HoLogo@LuaTeX}
%    The kerning is an idea of Hans Hagen, see mailing list
%    `luatex at tug dot org' in March 2010.
%    \begin{macrocode}
\def\HoLogo@LuaTeX#1{%
  \HOLOGO@mbox{%
    Lua%
    \HOLOGO@NegativeKerning{aT,oT,To}%
    \hologo{TeX}%
  }%
}
%    \end{macrocode}
%    \end{macro}
%    \begin{macro}{\HoLogoHtml@LuaTeX}
%    \begin{macrocode}
\let\HoLogoHtml@LuaTeX\HoLogo@LuaTeX
%    \end{macrocode}
%    \end{macro}
%
% \subsubsection{\hologo{LuaLaTeX}}
%
%    \begin{macro}{\HoLogo@LuaLaTeX}
%    \begin{macrocode}
\def\HoLogo@LuaLaTeX#1{%
  \HOLOGO@mbox{%
    Lua%
    \hologo{LaTeX}%
  }%
}
%    \end{macrocode}
%    \end{macro}
%    \begin{macro}{\HoLogoHtml@LuaLaTeX}
%    \begin{macrocode}
\let\HoLogoHtml@LuaLaTeX\HoLogo@LuaLaTeX
%    \end{macrocode}
%    \end{macro}
%
% \subsubsection{\hologo{XeTeX}, \hologo{XeLaTeX}}
%
%    \begin{macro}{\HOLOGO@IfCharExists}
%    \begin{macrocode}
\ifluatex
  \ifnum\luatexversion<36 %
  \else
    \def\HOLOGO@IfCharExists#1{%
      \ifnum
        \directlua{%
           if luaotfload and luaotfload.aux then
             if luaotfload.aux.font_has_glyph(%
                    font.current(), \number#1) then % 	 
	       tex.print("1") % 	 
	     end % 	 
	   elseif font and font.fonts and font.current then %
            local f = font.fonts[font.current()]%
            if f.characters and f.characters[\number#1] then %
              tex.print("1")%
            end %
          end%
        }0=\ltx@zero
        \expandafter\ltx@secondoftwo
      \else
        \expandafter\ltx@firstoftwo
      \fi
    }%
  \fi
\fi
\ltx@IfUndefined{HOLOGO@IfCharExists}{%
  \def\HOLOGO@@IfCharExists#1{%
    \begingroup
      \tracinglostchars=\ltx@zero
      \setbox\ltx@zero=\hbox{%
        \kern7sp\char#1\relax
        \ifnum\lastkern>\ltx@zero
          \expandafter\aftergroup\csname iffalse\endcsname
        \else
          \expandafter\aftergroup\csname iftrue\endcsname
        \fi
      }%
      % \if{true|false} from \aftergroup
      \endgroup
      \expandafter\ltx@firstoftwo
    \else
      \endgroup
      \expandafter\ltx@secondoftwo
    \fi
  }%
  \ifxetex
    \ltx@IfUndefined{XeTeXfonttype}{}{%
      \ltx@IfUndefined{XeTeXcharglyph}{}{%
        \def\HOLOGO@IfCharExists#1{%
          \ifnum\XeTeXfonttype\font>\ltx@zero
            \expandafter\ltx@firstofthree
          \else
            \expandafter\ltx@gobble
          \fi
          {%
            \ifnum\XeTeXcharglyph#1>\ltx@zero
              \expandafter\ltx@firstoftwo
            \else
              \expandafter\ltx@secondoftwo
            \fi
          }%
          \HOLOGO@@IfCharExists{#1}%
        }%
      }%
    }%
  \fi
}{}
\ltx@ifundefined{HOLOGO@IfCharExists}{%
  \ifnum64=`\^^^^0040\relax % test for big chars of LuaTeX/XeTeX
    \let\HOLOGO@IfCharExists\HOLOGO@@IfCharExists
  \else
    \def\HOLOGO@IfCharExists#1{%
      \ifnum#1>255 %
        \expandafter\ltx@fourthoffour
      \fi
      \HOLOGO@@IfCharExists{#1}%
    }%
  \fi
}{}
%    \end{macrocode}
%    \end{macro}
%
%    \begin{macro}{\HoLogo@Xe}
%    Source: package \xpackage{dtklogos}
%    \begin{macrocode}
\def\HoLogo@Xe#1{%
  X%
  \kern-.1em\relax
  \HOLOGO@IfCharExists{"018E}{%
    \lower.5ex\hbox{\char"018E}%
  }{%
    \chardef\HOLOGO@choice=\ltx@zero
    \ifdim\fontdimen\ltx@one\font>0pt %
      \ltx@IfUndefined{rotatebox}{%
        \ltx@IfUndefined{pgftext}{%
          \ltx@IfUndefined{psscalebox}{%
            \ltx@IfUndefined{HOLOGO@ScaleBox@\hologoDriver}{%
            }{%
              \chardef\HOLOGO@choice=4 %
            }%
          }{%
            \chardef\HOLOGO@choice=3 %
          }%
        }{%
          \chardef\HOLOGO@choice=2 %
        }%
      }{%
        \chardef\HOLOGO@choice=1 %
      }%
      \ifcase\HOLOGO@choice
        \HOLOGO@WarningUnsupportedDriver{Xe}%
        e%
      \or % 1: \rotatebox
        \begingroup
          \setbox\ltx@zero\hbox{\rotatebox{180}{E}}%
          \ltx@LocDimenA=\dp\ltx@zero
          \advance\ltx@LocDimenA by -.5ex\relax
          \raise\ltx@LocDimenA\box\ltx@zero
        \endgroup
      \or % 2: \pgftext
        \lower.5ex\hbox{%
          \pgfpicture
            \pgftext[rotate=180]{E}%
          \endpgfpicture
        }%
      \or % 3: \psscalebox
        \begingroup
          \setbox\ltx@zero\hbox{\psscalebox{-1 -1}{E}}%
          \ltx@LocDimenA=\dp\ltx@zero
          \advance\ltx@LocDimenA by -.5ex\relax
          \raise\ltx@LocDimenA\box\ltx@zero
        \endgroup
      \or % 4: \HOLOGO@PointReflectBox
        \lower.5ex\hbox{\HOLOGO@PointReflectBox{E}}%
      \else
        \@PackageError{hologo}{Internal error (choice/it}\@ehc
      \fi
    \else
      \ltx@IfUndefined{reflectbox}{%
        \ltx@IfUndefined{pgftext}{%
          \ltx@IfUndefined{psscalebox}{%
            \ltx@IfUndefined{HOLOGO@ScaleBox@\hologoDriver}{%
            }{%
              \chardef\HOLOGO@choice=4 %
            }%
          }{%
            \chardef\HOLOGO@choice=3 %
          }%
        }{%
          \chardef\HOLOGO@choice=2 %
        }%
      }{%
        \chardef\HOLOGO@choice=1 %
      }%
      \ifcase\HOLOGO@choice
        \HOLOGO@WarningUnsupportedDriver{Xe}%
        e%
      \or % 1: reflectbox
        \lower.5ex\hbox{%
          \reflectbox{E}%
        }%
      \or % 2: \pgftext
        \lower.5ex\hbox{%
          \pgfpicture
            \pgftransformxscale{-1}%
            \pgftext{E}%
          \endpgfpicture
        }%
      \or % 3: \psscalebox
        \lower.5ex\hbox{%
          \psscalebox{-1 1}{E}%
        }%
      \or % 4: \HOLOGO@Reflectbox
        \lower.5ex\hbox{%
          \HOLOGO@ReflectBox{E}%
        }%
      \else
        \@PackageError{hologo}{Internal error (choice/up)}\@ehc
      \fi
    \fi
  }%
}
%    \end{macrocode}
%    \end{macro}
%    \begin{macro}{\HoLogoHtml@Xe}
%    \begin{macrocode}
\def\HoLogoHtml@Xe#1{%
  \HoLogoCss@Xe
  \HOLOGO@Span{Xe}{%
    X%
    \HOLOGO@Span{e}{%
      \HCode{&\ltx@hashchar x018e;}%
    }%
  }%
}
%    \end{macrocode}
%    \end{macro}
%    \begin{macro}{\HoLogoCss@Xe}
%    \begin{macrocode}
\def\HoLogoCss@Xe{%
  \Css{%
    span.HoLogo-Xe span.HoLogo-e{%
      position:relative;%
      top:.5ex;%
      left-margin:-.1em;%
    }%
  }%
  \global\let\HoLogoCss@Xe\relax
}
%    \end{macrocode}
%    \end{macro}
%
%    \begin{macro}{\HoLogo@XeTeX}
%    \begin{macrocode}
\def\HoLogo@XeTeX#1{%
  \hologo{Xe}%
  \kern-.15em\relax
  \hologo{TeX}%
}
%    \end{macrocode}
%    \end{macro}
%
%    \begin{macro}{\HoLogoHtml@XeTeX}
%    \begin{macrocode}
\def\HoLogoHtml@XeTeX#1{%
  \HoLogoCss@XeTeX
  \HOLOGO@Span{XeTeX}{%
    \hologo{Xe}%
    \hologo{TeX}%
  }%
}
%    \end{macrocode}
%    \end{macro}
%    \begin{macro}{\HoLogoCss@XeTeX}
%    \begin{macrocode}
\def\HoLogoCss@XeTeX{%
  \Css{%
    span.HoLogo-XeTeX span.HoLogo-TeX{%
      margin-left:-.15em;%
    }%
  }%
  \global\let\HoLogoCss@XeTeX\relax
}
%    \end{macrocode}
%    \end{macro}
%
%    \begin{macro}{\HoLogo@XeLaTeX}
%    \begin{macrocode}
\def\HoLogo@XeLaTeX#1{%
  \hologo{Xe}%
  \kern-.13em%
  \hologo{LaTeX}%
}
%    \end{macrocode}
%    \end{macro}
%    \begin{macro}{\HoLogoHtml@XeLaTeX}
%    \begin{macrocode}
\def\HoLogoHtml@XeLaTeX#1{%
  \HoLogoCss@XeLaTeX
  \HOLOGO@Span{XeLaTeX}{%
    \hologo{Xe}%
    \hologo{LaTeX}%
  }%
}
%    \end{macrocode}
%    \end{macro}
%    \begin{macro}{\HoLogoCss@XeLaTeX}
%    \begin{macrocode}
\def\HoLogoCss@XeLaTeX{%
  \Css{%
    span.HoLogo-XeLaTeX span.HoLogo-Xe{%
      margin-right:-.13em;%
    }%
  }%
  \global\let\HoLogoCss@XeLaTeX\relax
}
%    \end{macrocode}
%    \end{macro}
%
% \subsubsection{\hologo{pdfTeX}, \hologo{pdfLaTeX}}
%
%    \begin{macro}{\HoLogo@pdfTeX}
%    \begin{macrocode}
\def\HoLogo@pdfTeX#1{%
  \HOLOGO@mbox{%
    #1{p}{P}df\hologo{TeX}%
  }%
}
%    \end{macrocode}
%    \end{macro}
%    \begin{macro}{\HoLogoCs@pdfTeX}
%    \begin{macrocode}
\def\HoLogoCs@pdfTeX#1{#1{p}{P}dfTeX}
%    \end{macrocode}
%    \end{macro}
%    \begin{macro}{\HoLogoBkm@pdfTeX}
%    \begin{macrocode}
\def\HoLogoBkm@pdfTeX#1{%
  #1{p}{P}df\hologo{TeX}%
}
%    \end{macrocode}
%    \end{macro}
%    \begin{macro}{\HoLogoHtml@pdfTeX}
%    \begin{macrocode}
\let\HoLogoHtml@pdfTeX\HoLogo@pdfTeX
%    \end{macrocode}
%    \end{macro}
%
%    \begin{macro}{\HoLogo@pdfLaTeX}
%    \begin{macrocode}
\def\HoLogo@pdfLaTeX#1{%
  \HOLOGO@mbox{%
    #1{p}{P}df\hologo{LaTeX}%
  }%
}
%    \end{macrocode}
%    \end{macro}
%    \begin{macro}{\HoLogoCs@pdfLaTeX}
%    \begin{macrocode}
\def\HoLogoCs@pdfLaTeX#1{#1{p}{P}dfLaTeX}
%    \end{macrocode}
%    \end{macro}
%    \begin{macro}{\HoLogoBkm@pdfLaTeX}
%    \begin{macrocode}
\def\HoLogoBkm@pdfLaTeX#1{%
  #1{p}{P}df\hologo{LaTeX}%
}
%    \end{macrocode}
%    \end{macro}
%    \begin{macro}{\HoLogoHtml@pdfLaTeX}
%    \begin{macrocode}
\let\HoLogoHtml@pdfLaTeX\HoLogo@pdfLaTeX
%    \end{macrocode}
%    \end{macro}
%
% \subsubsection{\hologo{VTeX}}
%
%    \begin{macro}{\HoLogo@VTeX}
%    \begin{macrocode}
\def\HoLogo@VTeX#1{%
  \HOLOGO@mbox{%
    V\hologo{TeX}%
  }%
}
%    \end{macrocode}
%    \end{macro}
%    \begin{macro}{\HoLogoHtml@VTeX}
%    \begin{macrocode}
\let\HoLogoHtml@VTeX\HoLogo@VTeX
%    \end{macrocode}
%    \end{macro}
%
% \subsubsection{\hologo{AmS}, \dots}
%
%    Source: class \xclass{amsdtx}
%
%    \begin{macro}{\HoLogo@AmS}
%    \begin{macrocode}
\def\HoLogo@AmS#1{%
  \HoLogoFont@font{AmS}{sy}{%
    A%
    \kern-.1667em%
    \lower.5ex\hbox{M}%
    \kern-.125em%
    S%
  }%
}
%    \end{macrocode}
%    \end{macro}
%    \begin{macro}{\HoLogoBkm@AmS}
%    \begin{macrocode}
\def\HoLogoBkm@AmS#1{AmS}
%    \end{macrocode}
%    \end{macro}
%    \begin{macro}{\HoLogoHtml@AmS}
%    \begin{macrocode}
\def\HoLogoHtml@AmS#1{%
  \HoLogoCss@AmS
%  \HoLogoFont@font{AmS}{sy}{%
    \HOLOGO@Span{AmS}{%
      A%
      \HOLOGO@Span{M}{M}%
      S%
    }%
%   }%
}
%    \end{macrocode}
%    \end{macro}
%    \begin{macro}{\HoLogoCss@AmS}
%    \begin{macrocode}
\def\HoLogoCss@AmS{%
  \Css{%
    span.HoLogo-AmS span.HoLogo-M{%
      position:relative;%
      top:.5ex;%
      margin-left:-.1667em;%
      margin-right:-.125em;%
      text-decoration:none;%
    }%
  }%
  \global\let\HoLogoCss@AmS\relax
}
%    \end{macrocode}
%    \end{macro}
%
%    \begin{macro}{\HoLogo@AmSTeX}
%    \begin{macrocode}
\def\HoLogo@AmSTeX#1{%
  \hologo{AmS}%
  \HOLOGO@hyphen
  \hologo{TeX}%
}
%    \end{macrocode}
%    \end{macro}
%    \begin{macro}{\HoLogoBkm@AmSTeX}
%    \begin{macrocode}
\def\HoLogoBkm@AmSTeX#1{AmS-TeX}%
%    \end{macrocode}
%    \end{macro}
%    \begin{macro}{\HoLogoHtml@AmSTeX}
%    \begin{macrocode}
\let\HoLogoHtml@AmSTeX\HoLogo@AmSTeX
%    \end{macrocode}
%    \end{macro}
%
%    \begin{macro}{\HoLogo@AmSLaTeX}
%    \begin{macrocode}
\def\HoLogo@AmSLaTeX#1{%
  \hologo{AmS}%
  \HOLOGO@hyphen
  \hologo{LaTeX}%
}
%    \end{macrocode}
%    \end{macro}
%    \begin{macro}{\HoLogoBkm@AmSLaTeX}
%    \begin{macrocode}
\def\HoLogoBkm@AmSLaTeX#1{AmS-LaTeX}%
%    \end{macrocode}
%    \end{macro}
%    \begin{macro}{\HoLogoHtml@AmSLaTeX}
%    \begin{macrocode}
\let\HoLogoHtml@AmSLaTeX\HoLogo@AmSLaTeX
%    \end{macrocode}
%    \end{macro}
%
% \subsubsection{\hologo{BibTeX}}
%
%    \begin{macro}{\HoLogo@BibTeX@sc}
%    A definition of \hologo{BibTeX} is provided in
%    the documentation source for the manual of \hologo{BibTeX}
%    \cite{btxdoc}.
%\begin{quote}
%\begin{verbatim}
%\def\BibTeX{%
%  {%
%    \rm
%    B%
%    \kern-.05em%
%    {%
%      \sc
%      i%
%      \kern-.025em %
%      b%
%    }%
%    \kern-.08em
%    T%
%    \kern-.1667em%
%    \lower.7ex\hbox{E}%
%    \kern-.125em%
%    X%
%  }%
%}
%\end{verbatim}
%\end{quote}
%    \begin{macrocode}
\def\HoLogo@BibTeX@sc#1{%
  B%
  \kern-.05em%
  \HoLogoFont@font{BibTeX}{sc}{%
    i%
    \kern-.025em%
    b%
  }%
  \HOLOGO@discretionary
  \kern-.08em%
  \hologo{TeX}%
}
%    \end{macrocode}
%    \end{macro}
%    \begin{macro}{\HoLogoHtml@BibTeX@sc}
%    \begin{macrocode}
\def\HoLogoHtml@BibTeX@sc#1{%
  \HoLogoCss@BibTeX@sc
  \HOLOGO@Span{BibTeX-sc}{%
    B%
    \HOLOGO@Span{i}{i}%
    \HOLOGO@Span{b}{b}%
    \hologo{TeX}%
  }%
}
%    \end{macrocode}
%    \end{macro}
%    \begin{macro}{\HoLogoCss@BibTeX@sc}
%    \begin{macrocode}
\def\HoLogoCss@BibTeX@sc{%
  \Css{%
    span.HoLogo-BibTeX-sc span.HoLogo-i{%
      margin-left:-.05em;%
      margin-right:-.025em;%
      font-variant:small-caps;%
    }%
  }%
  \Css{%
    span.HoLogo-BibTeX-sc span.HoLogo-b{%
      margin-right:-.08em;%
      font-variant:small-caps;%
    }%
  }%
  \global\let\HoLogoCss@BibTeX@sc\relax
}
%    \end{macrocode}
%    \end{macro}
%
%    \begin{macro}{\HoLogo@BibTeX@sf}
%    Variant \xoption{sf} avoids trouble with unavailable
%    small caps fonts (e.g., bold versions of Computer Modern or
%    Latin Modern). The definition is taken from
%    package \xpackage{dtklogos} \cite{dtklogos}.
%\begin{quote}
%\begin{verbatim}
%\DeclareRobustCommand{\BibTeX}{%
%  B%
%  \kern-.05em%
%  \hbox{%
%    $\m@th$% %% force math size calculations
%    \csname S@\f@size\endcsname
%    \fontsize\sf@size\z@
%    \math@fontsfalse
%    \selectfont
%    I%
%    \kern-.025em%
%    B
%  }%
%  \kern-.08em%
%  \-%
%  \TeX
%}
%\end{verbatim}
%\end{quote}
%    \begin{macrocode}
\def\HoLogo@BibTeX@sf#1{%
  B%
  \kern-.05em%
  \HoLogoFont@font{BibTeX}{bibsf}{%
    I%
    \kern-.025em%
    B%
  }%
  \HOLOGO@discretionary
  \kern-.08em%
  \hologo{TeX}%
}
%    \end{macrocode}
%    \end{macro}
%    \begin{macro}{\HoLogoHtml@BibTeX@sf}
%    \begin{macrocode}
\def\HoLogoHtml@BibTeX@sf#1{%
  \HoLogoCss@BibTeX@sf
  \HOLOGO@Span{BibTeX-sf}{%
    B%
    \HoLogoFont@font{BibTeX}{bibsf}{%
      \HOLOGO@Span{i}{I}%
      B%
    }%
    \hologo{TeX}%
  }%
}
%    \end{macrocode}
%    \end{macro}
%    \begin{macro}{\HoLogoCss@BibTeX@sf}
%    \begin{macrocode}
\def\HoLogoCss@BibTeX@sf{%
  \Css{%
    span.HoLogo-BibTeX-sf span.HoLogo-i{%
      margin-left:-.05em;%
      margin-right:-.025em;%
    }%
  }%
  \Css{%
    span.HoLogo-BibTeX-sf span.HoLogo-TeX{%
      margin-left:-.08em;%
    }%
  }%
  \global\let\HoLogoCss@BibTeX@sf\relax
}
%    \end{macrocode}
%    \end{macro}
%
%    \begin{macro}{\HoLogo@BibTeX}
%    \begin{macrocode}
\def\HoLogo@BibTeX{\HoLogo@BibTeX@sf}
%    \end{macrocode}
%    \end{macro}
%    \begin{macro}{\HoLogoHtml@BibTeX}
%    \begin{macrocode}
\def\HoLogoHtml@BibTeX{\HoLogoHtml@BibTeX@sf}
%    \end{macrocode}
%    \end{macro}
%
% \subsubsection{\hologo{BibTeX8}}
%
%    \begin{macro}{\HoLogo@BibTeX8}
%    \begin{macrocode}
\expandafter\def\csname HoLogo@BibTeX8\endcsname#1{%
  \hologo{BibTeX}%
  8%
}
%    \end{macrocode}
%    \end{macro}
%
%    \begin{macro}{\HoLogoBkm@BibTeX8}
%    \begin{macrocode}
\expandafter\def\csname HoLogoBkm@BibTeX8\endcsname#1{%
  \hologo{BibTeX}%
  8%
}
%    \end{macrocode}
%    \end{macro}
%    \begin{macro}{\HoLogoHtml@BibTeX8}
%    \begin{macrocode}
\expandafter
\let\csname HoLogoHtml@BibTeX8\expandafter\endcsname
\csname HoLogo@BibTeX8\endcsname
%    \end{macrocode}
%    \end{macro}
%
% \subsubsection{\hologo{ConTeXt}}
%
%    \begin{macro}{\HoLogo@ConTeXt@simple}
%    \begin{macrocode}
\def\HoLogo@ConTeXt@simple#1{%
  \HOLOGO@mbox{Con}%
  \HOLOGO@discretionary
  \HOLOGO@mbox{\hologo{TeX}t}%
}
%    \end{macrocode}
%    \end{macro}
%    \begin{macro}{\HoLogoHtml@ConTeXt@simple}
%    \begin{macrocode}
\let\HoLogoHtml@ConTeXt@simple\HoLogo@ConTeXt@simple
%    \end{macrocode}
%    \end{macro}
%
%    \begin{macro}{\HoLogo@ConTeXt@narrow}
%    This definition of logo \hologo{ConTeXt} with variant \xoption{narrow}
%    comes from TUGboat's class \xclass{ltugboat} (version 2010/11/15 v2.8).
%    \begin{macrocode}
\def\HoLogo@ConTeXt@narrow#1{%
  \HOLOGO@mbox{C\kern-.0333emon}%
  \HOLOGO@discretionary
  \kern-.0667em%
  \HOLOGO@mbox{\hologo{TeX}\kern-.0333emt}%
}
%    \end{macrocode}
%    \end{macro}
%    \begin{macro}{\HoLogoHtml@ConTeXt@narrow}
%    \begin{macrocode}
\def\HoLogoHtml@ConTeXt@narrow#1{%
  \HoLogoCss@ConTeXt@narrow
  \HOLOGO@Span{ConTeXt-narrow}{%
    \HOLOGO@Span{C}{C}%
    on%
    \hologo{TeX}%
    t%
  }%
}
%    \end{macrocode}
%    \end{macro}
%    \begin{macro}{\HoLogoCss@ConTeXt@narrow}
%    \begin{macrocode}
\def\HoLogoCss@ConTeXt@narrow{%
  \Css{%
    span.HoLogo-ConTeXt-narrow span.HoLogo-C{%
      margin-left:-.0333em;%
    }%
  }%
  \Css{%
    span.HoLogo-ConTeXt-narrow span.HoLogo-TeX{%
      margin-left:-.0667em;%
      margin-right:-.0333em;%
    }%
  }%
  \global\let\HoLogoCss@ConTeXt@narrow\relax
}
%    \end{macrocode}
%    \end{macro}
%
%    \begin{macro}{\HoLogo@ConTeXt}
%    \begin{macrocode}
\def\HoLogo@ConTeXt{\HoLogo@ConTeXt@narrow}
%    \end{macrocode}
%    \end{macro}
%    \begin{macro}{\HoLogoHtml@ConTeXt}
%    \begin{macrocode}
\def\HoLogoHtml@ConTeXt{\HoLogoHtml@ConTeXt@narrow}
%    \end{macrocode}
%    \end{macro}
%
% \subsubsection{\hologo{emTeX}}
%
%    \begin{macro}{\HoLogo@emTeX}
%    \begin{macrocode}
\def\HoLogo@emTeX#1{%
  \HOLOGO@mbox{#1{e}{E}m}%
  \HOLOGO@discretionary
  \hologo{TeX}%
}
%    \end{macrocode}
%    \end{macro}
%    \begin{macro}{\HoLogoCs@emTeX}
%    \begin{macrocode}
\def\HoLogoCs@emTeX#1{#1{e}{E}mTeX}%
%    \end{macrocode}
%    \end{macro}
%    \begin{macro}{\HoLogoBkm@emTeX}
%    \begin{macrocode}
\def\HoLogoBkm@emTeX#1{%
  #1{e}{E}m\hologo{TeX}%
}
%    \end{macrocode}
%    \end{macro}
%    \begin{macro}{\HoLogoHtml@emTeX}
%    \begin{macrocode}
\let\HoLogoHtml@emTeX\HoLogo@emTeX
%    \end{macrocode}
%    \end{macro}
%
% \subsubsection{\hologo{ExTeX}}
%
%    \begin{macro}{\HoLogo@ExTeX}
%    The definition is taken from the FAQ of the
%    project \hologo{ExTeX}
%    \cite{ExTeX-FAQ}.
%\begin{quote}
%\begin{verbatim}
%\def\ExTeX{%
%  \textrm{% Logo always with serifs
%    \ensuremath{%
%      \textstyle
%      \varepsilon_{%
%        \kern-0.15em%
%        \mathcal{X}%
%      }%
%    }%
%    \kern-.15em%
%    \TeX
%  }%
%}
%\end{verbatim}
%\end{quote}
%    \begin{macrocode}
\def\HoLogo@ExTeX#1{%
  \HoLogoFont@font{ExTeX}{rm}{%
    \ltx@mbox{%
      \HOLOGO@MathSetup
      $%
        \textstyle
        \varepsilon_{%
          \kern-0.15em%
          \HoLogoFont@font{ExTeX}{sy}{X}%
        }%
      $%
    }%
    \HOLOGO@discretionary
    \kern-.15em%
    \hologo{TeX}%
  }%
}
%    \end{macrocode}
%    \end{macro}
%    \begin{macro}{\HoLogoHtml@ExTeX}
%    \begin{macrocode}
\def\HoLogoHtml@ExTeX#1{%
  \HoLogoCss@ExTeX
  \HoLogoFont@font{ExTeX}{rm}{%
    \HOLOGO@Span{ExTeX}{%
      \ltx@mbox{%
        \HOLOGO@MathSetup
        $\textstyle\varepsilon$%
        \HOLOGO@Span{X}{$\textstyle\chi$}%
        \hologo{TeX}%
      }%
    }%
  }%
}
%    \end{macrocode}
%    \end{macro}
%    \begin{macro}{\HoLogoBkm@ExTeX}
%    \begin{macrocode}
\def\HoLogoBkm@ExTeX#1{%
  \HOLOGO@PdfdocUnicode{#1{e}{E}x}{\textepsilon\textchi}%
  \hologo{TeX}%
}
%    \end{macrocode}
%    \end{macro}
%    \begin{macro}{\HoLogoCss@ExTeX}
%    \begin{macrocode}
\def\HoLogoCss@ExTeX{%
  \Css{%
    span.HoLogo-ExTeX{%
      font-family:serif;%
    }%
  }%
  \Css{%
    span.HoLogo-ExTeX span.HoLogo-TeX{%
      margin-left:-.15em;%
    }%
  }%
  \global\let\HoLogoCss@ExTeX\relax
}
%    \end{macrocode}
%    \end{macro}
%
% \subsubsection{\hologo{MiKTeX}}
%
%    \begin{macro}{\HoLogo@MiKTeX}
%    \begin{macrocode}
\def\HoLogo@MiKTeX#1{%
  \HOLOGO@mbox{MiK}%
  \HOLOGO@discretionary
  \hologo{TeX}%
}
%    \end{macrocode}
%    \end{macro}
%    \begin{macro}{\HoLogoHtml@MiKTeX}
%    \begin{macrocode}
\let\HoLogoHtml@MiKTeX\HoLogo@MiKTeX
%    \end{macrocode}
%    \end{macro}
%
% \subsubsection{\hologo{OzTeX} and friends}
%
%    Source: \hologo{OzTeX} FAQ \cite{OzTeX}:
%    \begin{quote}
%      |\def\OzTeX{O\kern-.03em z\kern-.15em\TeX}|\\
%      (There is no kerning in OzMF, OzMP and OzTtH.)
%    \end{quote}
%
%    \begin{macro}{\HoLogo@OzTeX}
%    \begin{macrocode}
\def\HoLogo@OzTeX#1{%
  O%
  \kern-.03em %
  z%
  \kern-.15em %
  \hologo{TeX}%
}
%    \end{macrocode}
%    \end{macro}
%    \begin{macro}{\HoLogoHtml@OzTeX}
%    \begin{macrocode}
\def\HoLogoHtml@OzTeX#1{%
  \HoLogoCss@OzTeX
  \HOLOGO@Span{OzTeX}{%
    O%
    \HOLOGO@Span{z}{z}%
    \hologo{TeX}%
  }%
}
%    \end{macrocode}
%    \end{macro}
%    \begin{macro}{\HoLogoCss@OzTeX}
%    \begin{macrocode}
\def\HoLogoCss@OzTeX{%
  \Css{%
    span.HoLogo-OzTeX span.HoLogo-z{%
      margin-left:-.03em;%
      margin-right:-.15em;%
    }%
  }%
  \global\let\HoLogoCss@OzTeX\relax
}
%    \end{macrocode}
%    \end{macro}
%
%    \begin{macro}{\HoLogo@OzMF}
%    \begin{macrocode}
\def\HoLogo@OzMF#1{%
  \HOLOGO@mbox{OzMF}%
}
%    \end{macrocode}
%    \end{macro}
%    \begin{macro}{\HoLogo@OzMP}
%    \begin{macrocode}
\def\HoLogo@OzMP#1{%
  \HOLOGO@mbox{OzMP}%
}
%    \end{macrocode}
%    \end{macro}
%    \begin{macro}{\HoLogo@OzTtH}
%    \begin{macrocode}
\def\HoLogo@OzTtH#1{%
  \HOLOGO@mbox{OzTtH}%
}
%    \end{macrocode}
%    \end{macro}
%
% \subsubsection{\hologo{PCTeX}}
%
%    \begin{macro}{\HoLogo@PCTeX}
%    \begin{macrocode}
\def\HoLogo@PCTeX#1{%
  \HOLOGO@mbox{PC}%
  \hologo{TeX}%
}
%    \end{macrocode}
%    \end{macro}
%    \begin{macro}{\HoLogoHtml@PCTeX}
%    \begin{macrocode}
\let\HoLogoHtml@PCTeX\HoLogo@PCTeX
%    \end{macrocode}
%    \end{macro}
%
% \subsubsection{\hologo{PiCTeX}}
%
%    The original definitions from \xfile{pictex.tex} \cite{PiCTeX}:
%\begin{quote}
%\begin{verbatim}
%\def\PiC{%
%  P%
%  \kern-.12em%
%  \lower.5ex\hbox{I}%
%  \kern-.075em%
%  C%
%}
%\def\PiCTeX{%
%  \PiC
%  \kern-.11em%
%  \TeX
%}
%\end{verbatim}
%\end{quote}
%
%    \begin{macro}{\HoLogo@PiC}
%    \begin{macrocode}
\def\HoLogo@PiC#1{%
  P%
  \kern-.12em%
  \lower.5ex\hbox{I}%
  \kern-.075em%
  C%
  \HOLOGO@SpaceFactor
}
%    \end{macrocode}
%    \end{macro}
%    \begin{macro}{\HoLogoHtml@PiC}
%    \begin{macrocode}
\def\HoLogoHtml@PiC#1{%
  \HoLogoCss@PiC
  \HOLOGO@Span{PiC}{%
    P%
    \HOLOGO@Span{i}{I}%
    C%
  }%
}
%    \end{macrocode}
%    \end{macro}
%    \begin{macro}{\HoLogoCss@PiC}
%    \begin{macrocode}
\def\HoLogoCss@PiC{%
  \Css{%
    span.HoLogo-PiC span.HoLogo-i{%
      position:relative;%
      top:.5ex;%
      margin-left:-.12em;%
      margin-right:-.075em;%
      text-decoration:none;%
    }%
  }%
  \global\let\HoLogoCss@PiC\relax
}
%    \end{macrocode}
%    \end{macro}
%
%    \begin{macro}{\HoLogo@PiCTeX}
%    \begin{macrocode}
\def\HoLogo@PiCTeX#1{%
  \hologo{PiC}%
  \HOLOGO@discretionary
  \kern-.11em%
  \hologo{TeX}%
}
%    \end{macrocode}
%    \end{macro}
%    \begin{macro}{\HoLogoHtml@PiCTeX}
%    \begin{macrocode}
\def\HoLogoHtml@PiCTeX#1{%
  \HoLogoCss@PiCTeX
  \HOLOGO@Span{PiCTeX}{%
    \hologo{PiC}%
    \hologo{TeX}%
  }%
}
%    \end{macrocode}
%    \end{macro}
%    \begin{macro}{\HoLogoCss@PiCTeX}
%    \begin{macrocode}
\def\HoLogoCss@PiCTeX{%
  \Css{%
    span.HoLogo-PiCTeX span.HoLogo-PiC{%
      margin-right:-.11em;%
    }%
  }%
  \global\let\HoLogoCss@PiCTeX\relax
}
%    \end{macrocode}
%    \end{macro}
%
% \subsubsection{\hologo{teTeX}}
%
%    \begin{macro}{\HoLogo@teTeX}
%    \begin{macrocode}
\def\HoLogo@teTeX#1{%
  \HOLOGO@mbox{#1{t}{T}e}%
  \HOLOGO@discretionary
  \hologo{TeX}%
}
%    \end{macrocode}
%    \end{macro}
%    \begin{macro}{\HoLogoCs@teTeX}
%    \begin{macrocode}
\def\HoLogoCs@teTeX#1{#1{t}{T}dfTeX}
%    \end{macrocode}
%    \end{macro}
%    \begin{macro}{\HoLogoBkm@teTeX}
%    \begin{macrocode}
\def\HoLogoBkm@teTeX#1{%
  #1{t}{T}e\hologo{TeX}%
}
%    \end{macrocode}
%    \end{macro}
%    \begin{macro}{\HoLogoHtml@teTeX}
%    \begin{macrocode}
\let\HoLogoHtml@teTeX\HoLogo@teTeX
%    \end{macrocode}
%    \end{macro}
%
% \subsubsection{\hologo{TeX4ht}}
%
%    \begin{macro}{\HoLogo@TeX4ht}
%    \begin{macrocode}
\expandafter\def\csname HoLogo@TeX4ht\endcsname#1{%
  \HOLOGO@mbox{\hologo{TeX}4ht}%
}
%    \end{macrocode}
%    \end{macro}
%    \begin{macro}{\HoLogoHtml@TeX4ht}
%    \begin{macrocode}
\expandafter
\let\csname HoLogoHtml@TeX4ht\expandafter\endcsname
\csname HoLogo@TeX4ht\endcsname
%    \end{macrocode}
%    \end{macro}
%
%
% \subsubsection{\hologo{SageTeX}}
%
%    \begin{macro}{\HoLogo@SageTeX}
%    \begin{macrocode}
\def\HoLogo@SageTeX#1{%
  \HOLOGO@mbox{Sage}%
  \HOLOGO@discretionary
  \HOLOGO@NegativeKerning{eT,oT,To}%
  \hologo{TeX}%
}
%    \end{macrocode}
%    \end{macro}
%    \begin{macro}{\HoLogoHtml@SageTeX}
%    \begin{macrocode}
\let\HoLogoHtml@SageTeX\HoLogo@SageTeX
%    \end{macrocode}
%    \end{macro}
%
% \subsection{\hologo{METAFONT} and friends}
%
%    \begin{macro}{\HoLogo@METAFONT}
%    \begin{macrocode}
\def\HoLogo@METAFONT#1{%
  \HoLogoFont@font{METAFONT}{logo}{%
    \HOLOGO@mbox{META}%
    \HOLOGO@discretionary
    \HOLOGO@mbox{FONT}%
  }%
}
%    \end{macrocode}
%    \end{macro}
%
%    \begin{macro}{\HoLogo@METAPOST}
%    \begin{macrocode}
\def\HoLogo@METAPOST#1{%
  \HoLogoFont@font{METAPOST}{logo}{%
    \HOLOGO@mbox{META}%
    \HOLOGO@discretionary
    \HOLOGO@mbox{POST}%
  }%
}
%    \end{macrocode}
%    \end{macro}
%
%    \begin{macro}{\HoLogo@MetaFun}
%    \begin{macrocode}
\def\HoLogo@MetaFun#1{%
  \HOLOGO@mbox{Meta}%
  \HOLOGO@discretionary
  \HOLOGO@mbox{Fun}%
}
%    \end{macrocode}
%    \end{macro}
%
%    \begin{macro}{\HoLogo@MetaPost}
%    \begin{macrocode}
\def\HoLogo@MetaPost#1{%
  \HOLOGO@mbox{Meta}%
  \HOLOGO@discretionary
  \HOLOGO@mbox{Post}%
}
%    \end{macrocode}
%    \end{macro}
%
% \subsection{Others}
%
% \subsubsection{\hologo{biber}}
%
%    \begin{macro}{\HoLogo@biber}
%    \begin{macrocode}
\def\HoLogo@biber#1{%
  \HOLOGO@mbox{#1{b}{B}i}%
  \HOLOGO@discretionary
  \HOLOGO@mbox{ber}%
}
%    \end{macrocode}
%    \end{macro}
%    \begin{macro}{\HoLogoCs@biber}
%    \begin{macrocode}
\def\HoLogoCs@biber#1{#1{b}{B}iber}
%    \end{macrocode}
%    \end{macro}
%    \begin{macro}{\HoLogoBkm@biber}
%    \begin{macrocode}
\def\HoLogoBkm@biber#1{%
  #1{b}{B}iber%
}
%    \end{macrocode}
%    \end{macro}
%    \begin{macro}{\HoLogoHtml@biber}
%    \begin{macrocode}
\let\HoLogoHtml@biber\HoLogo@biber
%    \end{macrocode}
%    \end{macro}
%
% \subsubsection{\hologo{KOMAScript}}
%
%    \begin{macro}{\HoLogo@KOMAScript}
%    The definition for \hologo{KOMAScript} is taken
%    from \hologo{KOMAScript} (\xfile{scrlogo.dtx}, reformatted) \cite{scrlogo}:
%\begin{quote}
%\begin{verbatim}
%\@ifundefined{KOMAScript}{%
%  \DeclareRobustCommand{\KOMAScript}{%
%    \textsf{%
%      K\kern.05em O\kern.05emM\kern.05em A%
%      \kern.1em-\kern.1em %
%      Script%
%    }%
%  }%
%}{}
%\end{verbatim}
%\end{quote}
%    \begin{macrocode}
\def\HoLogo@KOMAScript#1{%
  \HoLogoFont@font{KOMAScript}{sf}{%
    \HOLOGO@mbox{%
      K\kern.05em%
      O\kern.05em%
      M\kern.05em%
      A%
    }%
    \kern.1em%
    \HOLOGO@hyphen
    \kern.1em%
    \HOLOGO@mbox{Script}%
  }%
}
%    \end{macrocode}
%    \end{macro}
%    \begin{macro}{\HoLogoBkm@KOMAScript}
%    \begin{macrocode}
\def\HoLogoBkm@KOMAScript#1{%
  KOMA-Script%
}
%    \end{macrocode}
%    \end{macro}
%    \begin{macro}{\HoLogoHtml@KOMAScript}
%    \begin{macrocode}
\def\HoLogoHtml@KOMAScript#1{%
  \HoLogoCss@KOMAScript
  \HoLogoFont@font{KOMAScript}{sf}{%
    \HOLOGO@Span{KOMAScript}{%
      K%
      \HOLOGO@Span{O}{O}%
      M%
      \HOLOGO@Span{A}{A}%
      \HOLOGO@Span{hyphen}{-}%
      Script%
    }%
  }%
}
%    \end{macrocode}
%    \end{macro}
%    \begin{macro}{\HoLogoCss@KOMAScript}
%    \begin{macrocode}
\def\HoLogoCss@KOMAScript{%
  \Css{%
    span.HoLogo-KOMAScript{%
      font-family:sans-serif;%
    }%
  }%
  \Css{%
    span.HoLogo-KOMAScript span.HoLogo-O{%
      padding-left:.05em;%
      padding-right:.05em;%
    }%
  }%
  \Css{%
    span.HoLogo-KOMAScript span.HoLogo-A{%
      padding-left:.05em;%
    }%
  }%
  \Css{%
    span.HoLogo-KOMAScript span.HoLogo-hyphen{%
      padding-left:.1em;%
      padding-right:.1em;%
    }%
  }%
  \global\let\HoLogoCss@KOMAScript\relax
}
%    \end{macrocode}
%    \end{macro}
%
% \subsubsection{\hologo{LyX}}
%
%    \begin{macro}{\HoLogo@LyX}
%    The definition is taken from the documentation source files
%    of \hologo{LyX}, \xfile{Intro.lyx} \cite{LyX}:
%\begin{quote}
%\begin{verbatim}
%\def\LyX{%
%  \texorpdfstring{%
%    L\kern-.1667em\lower.25em\hbox{Y}\kern-.125emX\@%
%  }{%
%    LyX%
%  }%
%}
%\end{verbatim}
%\end{quote}
%    \begin{macrocode}
\def\HoLogo@LyX#1{%
  L%
  \kern-.1667em%
  \lower.25em\hbox{Y}%
  \kern-.125em%
  X%
  \HOLOGO@SpaceFactor
}
%    \end{macrocode}
%    \end{macro}
%    \begin{macro}{\HoLogoHtml@LyX}
%    \begin{macrocode}
\def\HoLogoHtml@LyX#1{%
  \HoLogoCss@LyX
  \HOLOGO@Span{LyX}{%
    L%
    \HOLOGO@Span{y}{Y}%
    X%
  }%
}
%    \end{macrocode}
%    \end{macro}
%    \begin{macro}{\HoLogoCss@LyX}
%    \begin{macrocode}
\def\HoLogoCss@LyX{%
  \Css{%
    span.HoLogo-LyX span.HoLogo-y{%
      position:relative;%
      top:.25em;%
      margin-left:-.1667em;%
      margin-right:-.125em;%
      text-decoration:none;%
    }%
  }%
  \global\let\HoLogoCss@LyX\relax
}
%    \end{macrocode}
%    \end{macro}
%
% \subsubsection{\hologo{NTS}}
%
%    \begin{macro}{\HoLogo@NTS}
%    Definition for \hologo{NTS} can be found in
%    package \xpackage{etex\textunderscore man} for the \hologo{eTeX} manual \cite{etexman}
%    and in package \xpackage{dtklogos} \cite{dtklogos}:
%\begin{quote}
%\begin{verbatim}
%\def\NTS{%
%  \leavevmode
%  \hbox{%
%    $%
%      \cal N%
%      \kern-0.35em%
%      \lower0.5ex\hbox{$\cal T$}%
%      \kern-0.2em%
%      S%
%    $%
%  }%
%}
%\end{verbatim}
%\end{quote}
%    \begin{macrocode}
\def\HoLogo@NTS#1{%
  \HoLogoFont@font{NTS}{sy}{%
    N\/%
    \kern-.35em%
    \lower.5ex\hbox{T\/}%
    \kern-.2em%
    S\/%
  }%
  \HOLOGO@SpaceFactor
}
%    \end{macrocode}
%    \end{macro}
%
% \subsubsection{\Hologo{TTH} (\hologo{TeX} to HTML translator)}
%
%    Source: \url{http://hutchinson.belmont.ma.us/tth/}
%    In the HTML source the second `T' is printed as subscript.
%\begin{quote}
%\begin{verbatim}
%T<sub>T</sub>H
%\end{verbatim}
%\end{quote}
%    \begin{macro}{\HoLogo@TTH}
%    \begin{macrocode}
\def\HoLogo@TTH#1{%
  \ltx@mbox{%
    T\HOLOGO@SubScript{T}H%
  }%
  \HOLOGO@SpaceFactor
}
%    \end{macrocode}
%    \end{macro}
%
%    \begin{macro}{\HoLogoHtml@TTH}
%    \begin{macrocode}
\def\HoLogoHtml@TTH#1{%
  T\HCode{<sub>}T\HCode{</sub>}H%
}
%    \end{macrocode}
%    \end{macro}
%
% \subsubsection{\Hologo{HanTheThanh}}
%
%    Partial source: Package \xpackage{dtklogos}.
%    The double accent is U+1EBF (latin small letter e with circumflex
%    and acute).
%    \begin{macro}{\HoLogo@HanTheThanh}
%    \begin{macrocode}
\def\HoLogo@HanTheThanh#1{%
  \ltx@mbox{H\`an}%
  \HOLOGO@space
  \ltx@mbox{%
    Th%
    \HOLOGO@IfCharExists{"1EBF}{%
      \char"1EBF\relax
    }{%
      \^e\hbox to 0pt{\hss\raise .5ex\hbox{\'{}}}%
    }%
  }%
  \HOLOGO@space
  \ltx@mbox{Th\`anh}%
}
%    \end{macrocode}
%    \end{macro}
%    \begin{macro}{\HoLogoBkm@HanTheThanh}
%    \begin{macrocode}
\def\HoLogoBkm@HanTheThanh#1{%
  H\`an %
  Th\HOLOGO@PdfdocUnicode{\^e}{\9036\277} %
  Th\`anh%
}
%    \end{macrocode}
%    \end{macro}
%    \begin{macro}{\HoLogoHtml@HanTheThanh}
%    \begin{macrocode}
\def\HoLogoHtml@HanTheThanh#1{%
  H\`an %
  Th\HCode{&\ltx@hashchar x1ebf;} %
  Th\`anh%
}
%    \end{macrocode}
%    \end{macro}
%
% \subsection{Driver detection}
%
%    \begin{macrocode}
\HOLOGO@IfExists\InputIfFileExists{%
  \InputIfFileExists{hologo.cfg}{}{}%
}{%
  \ltx@IfUndefined{pdf@filesize}{%
    \def\HOLOGO@InputIfExists{%
      \openin\HOLOGO@temp=hologo.cfg\relax
      \ifeof\HOLOGO@temp
        \closein\HOLOGO@temp
      \else
        \closein\HOLOGO@temp
        \begingroup
          \def\x{LaTeX2e}%
        \expandafter\endgroup
        \ifx\fmtname\x
          % \iffalse meta-comment
%
% File: hologo.dtx
% Version: 2016/05/12 v1.11
% Info: A logo collection with bookmark support
%
% Copyright (C) 2010-2012 by
%    Heiko Oberdiek <heiko.oberdiek at googlemail.com>
%
% This work may be distributed and/or modified under the
% conditions of the LaTeX Project Public License, either
% version 1.3c of this license or (at your option) any later
% version. This version of this license is in
%    http://www.latex-project.org/lppl/lppl-1-3c.txt
% and the latest version of this license is in
%    http://www.latex-project.org/lppl.txt
% and version 1.3 or later is part of all distributions of
% LaTeX version 2005/12/01 or later.
%
% This work has the LPPL maintenance status "maintained".
%
% This Current Maintainer of this work is Heiko Oberdiek.
%
% The Base Interpreter refers to any `TeX-Format',
% because some files are installed in TDS:tex/generic//.
%
% This work consists of the main source file hologo.dtx
% and the derived files
%    hologo.sty, hologo.pdf, hologo.ins, hologo.drv, hologo-example.tex,
%    hologo-test1.tex, hologo-test-spacefactor.tex,
%    hologo-test-list.tex.
%
% Distribution:
%    CTAN:macros/latex/contrib/oberdiek/hologo.dtx
%    CTAN:macros/latex/contrib/oberdiek/hologo.pdf
%
% Unpacking:
%    (a) If hologo.ins is present:
%           tex hologo.ins
%    (b) Without hologo.ins:
%           tex hologo.dtx
%    (c) If you insist on using LaTeX
%           latex \let\install=y\input{hologo.dtx}
%        (quote the arguments according to the demands of your shell)
%
% Documentation:
%    (a) If hologo.drv is present:
%           latex hologo.drv
%    (b) Without hologo.drv:
%           latex hologo.dtx; ...
%    The class ltxdoc loads the configuration file ltxdoc.cfg
%    if available. Here you can specify further options, e.g.
%    use A4 as paper format:
%       \PassOptionsToClass{a4paper}{article}
%
%    Programm calls to get the documentation (example):
%       pdflatex hologo.dtx
%       makeindex -s gind.ist hologo.idx
%       pdflatex hologo.dtx
%       makeindex -s gind.ist hologo.idx
%       pdflatex hologo.dtx
%
% Installation:
%    TDS:tex/generic/oberdiek/hologo.sty
%    TDS:doc/latex/oberdiek/hologo.pdf
%    TDS:doc/latex/oberdiek/example/hologo-example.tex
%    TDS:doc/latex/oberdiek/test/hologo-test1.tex
%    TDS:doc/latex/oberdiek/test/hologo-test-spacefactor.tex
%    TDS:doc/latex/oberdiek/test/hologo-test-list.tex
%    TDS:source/latex/oberdiek/hologo.dtx
%
%<*ignore>
\begingroup
  \catcode123=1 %
  \catcode125=2 %
  \def\x{LaTeX2e}%
\expandafter\endgroup
\ifcase 0\ifx\install y1\fi\expandafter
         \ifx\csname processbatchFile\endcsname\relax\else1\fi
         \ifx\fmtname\x\else 1\fi\relax
\else\csname fi\endcsname
%</ignore>
%<*install>
\input docstrip.tex
\Msg{************************************************************************}
\Msg{* Installation}
\Msg{* Package: hologo 2016/05/12 v1.11 A logo collection with bookmark support (HO)}
\Msg{************************************************************************}

\keepsilent
\askforoverwritefalse

\let\MetaPrefix\relax
\preamble

This is a generated file.

Project: hologo
Version: 2016/05/12 v1.11

Copyright (C) 2010-2012 by
   Heiko Oberdiek <heiko.oberdiek at googlemail.com>

This work may be distributed and/or modified under the
conditions of the LaTeX Project Public License, either
version 1.3c of this license or (at your option) any later
version. This version of this license is in
   http://www.latex-project.org/lppl/lppl-1-3c.txt
and the latest version of this license is in
   http://www.latex-project.org/lppl.txt
and version 1.3 or later is part of all distributions of
LaTeX version 2005/12/01 or later.

This work has the LPPL maintenance status "maintained".

This Current Maintainer of this work is Heiko Oberdiek.

The Base Interpreter refers to any `TeX-Format',
because some files are installed in TDS:tex/generic//.

This work consists of the main source file hologo.dtx
and the derived files
   hologo.sty, hologo.pdf, hologo.ins, hologo.drv, hologo-example.tex,
   hologo-test1.tex, hologo-test-spacefactor.tex,
   hologo-test-list.tex.

\endpreamble
\let\MetaPrefix\DoubleperCent

\generate{%
  \file{hologo.ins}{\from{hologo.dtx}{install}}%
  \file{hologo.drv}{\from{hologo.dtx}{driver}}%
  \usedir{tex/generic/oberdiek}%
  \file{hologo.sty}{\from{hologo.dtx}{package}}%
  \usedir{doc/latex/oberdiek/example}%
  \file{hologo-example.tex}{\from{hologo.dtx}{example}}%
  \usedir{doc/latex/oberdiek/test}%
  \file{hologo-test1.tex}{\from{hologo.dtx}{test1}}%
  \file{hologo-test-spacefactor.tex}{\from{hologo.dtx}{test-spacefactor}}%
  \file{hologo-test-list.tex}{\from{hologo.dtx}{test-list}}%
  \nopreamble
  \nopostamble
  \usedir{source/latex/oberdiek/catalogue}%
  \file{hologo.xml}{\from{hologo.dtx}{catalogue}}%
}

\catcode32=13\relax% active space
\let =\space%
\Msg{************************************************************************}
\Msg{*}
\Msg{* To finish the installation you have to move the following}
\Msg{* file into a directory searched by TeX:}
\Msg{*}
\Msg{*     hologo.sty}
\Msg{*}
\Msg{* To produce the documentation run the file `hologo.drv'}
\Msg{* through LaTeX.}
\Msg{*}
\Msg{* Happy TeXing!}
\Msg{*}
\Msg{************************************************************************}

\endbatchfile
%</install>
%<*ignore>
\fi
%</ignore>
%<*driver>
\NeedsTeXFormat{LaTeX2e}
\ProvidesFile{hologo.drv}%
  [2016/05/12 v1.11 A logo collection with bookmark support (HO)]%
\documentclass{ltxdoc}
\usepackage{holtxdoc}[2011/11/22]
\usepackage{hologo}[2016/05/12]
\usepackage{longtable}
\usepackage{array}
\usepackage{paralist}
%\usepackage[T1]{fontenc}
%\usepackage{lmodern}
\begin{document}
  \DocInput{hologo.dtx}%
\end{document}
%</driver>
% \fi
%
%
% \CharacterTable
%  {Upper-case    \A\B\C\D\E\F\G\H\I\J\K\L\M\N\O\P\Q\R\S\T\U\V\W\X\Y\Z
%   Lower-case    \a\b\c\d\e\f\g\h\i\j\k\l\m\n\o\p\q\r\s\t\u\v\w\x\y\z
%   Digits        \0\1\2\3\4\5\6\7\8\9
%   Exclamation   \!     Double quote  \"     Hash (number) \#
%   Dollar        \$     Percent       \%     Ampersand     \&
%   Acute accent  \'     Left paren    \(     Right paren   \)
%   Asterisk      \*     Plus          \+     Comma         \,
%   Minus         \-     Point         \.     Solidus       \/
%   Colon         \:     Semicolon     \;     Less than     \<
%   Equals        \=     Greater than  \>     Question mark \?
%   Commercial at \@     Left bracket  \[     Backslash     \\
%   Right bracket \]     Circumflex    \^     Underscore    \_
%   Grave accent  \`     Left brace    \{     Vertical bar  \|
%   Right brace   \}     Tilde         \~}
%
% \GetFileInfo{hologo.drv}
%
% \title{The \xpackage{hologo} package}
% \date{2016/05/12 v1.11}
% \author{Heiko Oberdiek\\\xemail{heiko.oberdiek at googlemail.com}}
%
% \maketitle
%
% \begin{abstract}
% This package starts a collection of logos with support for bookmarks
% strings.
% \end{abstract}
%
% \tableofcontents
%
% \section{Documentation}
%
% \subsection{Logo macros}
%
% \begin{declcs}{hologo} \M{name}
% \end{declcs}
% Macro \cs{hologo} sets the logo with name \meta{name}.
% The following table shows the supported names.
%
% \begingroup
%   \def\hologoEntry#1#2#3{^^A
%     #1&#2&\hologoLogoSetup{#1}{variant=#2}\hologo{#1}&#3\tabularnewline
%   }
%   \begin{longtable}{>{\ttfamily}l>{\ttfamily}lll}
%     \rmfamily\bfseries{name} & \rmfamily\bfseries variant
%     & \bfseries logo & \bfseries since\\
%     \hline
%     \endhead
%     \hologoList
%   \end{longtable}
% \endgroup
%
% \begin{declcs}{Hologo} \M{name}
% \end{declcs}
% Macro \cs{Hologo} starts the logo \meta{name} with an uppercase
% letter. As an exception small greek letters are not converted
% to uppercase. Examples, see \hologo{eTeX} and \hologo{ExTeX}.
%
% \subsection{Setup macros}
%
% The package does not support package options, but the following
% setup macros can be used to set options.
%
% \begin{declcs}{hologoSetup} \M{key value list}
% \end{declcs}
% Macro \cs{hologoSetup} sets global options.
%
% \begin{declcs}{hologoLogoSetup} \M{logo} \M{key value list}
% \end{declcs}
% Some options can also be used to configure a logo.
% These settings take precedence over global option settings.
%
% \subsection{Options}\label{sec:options}
%
% There are boolean and string options:
% \begin{description}
% \item[Boolean option:]
% It takes |true| or |false|
% as value. If the value is omitted, then |true| is used.
% \item[String option:]
% A value must be given as string. (But the string might be empty.)
% \end{description}
% The following options can be used both in \cs{hologoSetup}
% and \cs{hologoLogoSetup}:
% \begin{description}
% \def\entry#1{\item[\xoption{#1}:]}
% \entry{break}
%   enables or disables line breaks inside the logo. This setting is
%   refined by options \xoption{hyphenbreak}, \xoption{spacebreak}
%   or \xoption{discretionarybreak}.
%   Default is |false|.
% \entry{hyphenbreak}
%   enables or disables the line break right after the hyphen character.
% \entry{spacebreak}
%   enables or disables line breaks at space characters.
% \entry{discretionarybreak}
%   enables or disables line breaks at hyphenation points
%   (inserted by \cs{-}).
% \end{description}
% Macro \cs{hologoLogoSetup} also knows:
% \begin{description}
% \item[\xoption{variant}:]
%   This is a string option. It specifies a variant of a logo that
%   must exist. An empty string selects the package default variant.
% \end{description}
% Example:
% \begin{quote}
%   |\hologoSetup{break=false}|\\
%   |\hologoLogoSetup{plainTeX}{variant=hyphen,hyphenbreak}|\\
%   Then ``plain-\TeX'' contains one break point after the hyphen.
% \end{quote}
%
% \subsection{Driver options}
%
% Sometimes graphical operations are needed to construct some
% glyphs (e.g.\ \hologo{XeTeX}). If package \xpackage{graphics}
% or package \xpackage{pgf} are found, then the macros are taken
% from there. Otherwise the packge defines its own operations
% and therefore needs the driver information. Many drivers are
% detected automatically (\hologo{pdfTeX}/\hologo{LuaTeX}
% in PDF mode, \hologo{XeTeX}, \hologo{VTeX}). These have precedence
% over a driver option. The driver can be given as package option
% or using \cs{hologoDriverSetup}.
% The following list contains the recognized driver options:
% \begin{itemize}
% \item \xoption{pdftex}, \xoption{luatex}
% \item \xoption{dvipdfm}, \xoption{dvipdfmx}
% \item \xoption{dvips}, \xoption{dvipsone}, \xoption{xdvi}
% \item \xoption{xetex}
% \item \xoption{vtex}
% \end{itemize}
% The left driver of a line is the driver name that is used internally.
% The following names are aliases for drivers that use the
% same method. Therefore the entry in the \xext{log} file for
% the used driver prints the internally used driver name.
% \begin{description}
% \item[\xoption{driverfallback}:]
%   This option expects a driver that is used,
%   if the driver could not be detected automatically.
% \end{description}
%
% \begin{declcs}{hologoDriverSetup} \M{driver option}
% \end{declcs}
% The driver can also be configured after package loading
% using \cs{hologoDriverSetup}, also the way for \hologo{plainTeX}
% to setup the driver.
%
% \subsection{Font setup}
%
% Some logos require a special font, but should also be usable by
% \hologo{plainTeX}. Therefore the package provides some ways
% to influence the font settings. The options below
% take font settings as values. Both font commands
% such as \cs{sffamily} and macros that take one argument
% like \cs{textsf} can be used.
%
% \begin{declcs}{hologoFontSetup} \M{key value list}
% \end{declcs}
% Macro \cs{hologoFontSetup} sets the fonts for all logos.
% Supported keys:
% \begin{description}
% \def\entry#1{\item[\xoption{#1}:]}
% \entry{general}
%   This font is used for all logos. The default is empty.
%   That means no special font is used.
% \entry{bibsf}
%   This font is used for
%   {\hologoLogoSetup{BibTeX}{variant=sf}\hologo{BibTeX}}
%   with variant \xoption{sf}.
% \entry{rm}
%   This font is a serif font. It is used for \hologo{ExTeX}.
% \entry{sc}
%   This font specifies a small caps font. It is used for
%   {\hologoLogoSetup{BibTeX}{variant=sc}\hologo{BibTeX}}
%   with variant \xoption{sc}.
% \entry{sf}
%   This font specifies a sans serif font. The default
%   is \cs{sffamily}, then \cs{sf} is tried. Otherwise
%   a warning is given. It is used by \hologo{KOMAScript}.
% \entry{sy}
%   This is the font for math symbols (e.g. cmsy).
%   It is used by \hologo{AmS}, \hologo{NTS}, \hologo{ExTeX}.
% \entry{logo}
%   \hologo{METAFONT} and \hologo{METAPOST} are using that font.
%   In \hologo{LaTeX} \cs{logofamily} is used and
%   the definitions of package \xpackage{mflogo} are used
%   if the package is not loaded.
%   Otherwise the \cs{tenlogo} is used and defined
%   if it does not already exists.
% \end{description}
%
% \begin{declcs}{hologoLogoFontSetup} \M{logo} \M{key value list}
% \end{declcs}
% Fonts can also be set for a logo or logo component separately,
% see the following list.
% The keys are the same as for \cs{hologoFontSetup}.
%
% \begin{longtable}{>{\ttfamily}l>{\sffamily}ll}
%   \meta{logo} & keys & result\\
%   \hline
%   \endhead
%   BibTeX & bibsf & {\hologoLogoSetup{BibTeX}{variant=sf}\hologo{BibTeX}}\\[.5ex]
%   BibTeX & sc & {\hologoLogoSetup{BibTeX}{variant=sc}\hologo{BibTeX}}\\[.5ex]
%   ExTeX & rm & \hologo{ExTeX}\\
%   SliTeX & rm & \hologo{SliTeX}\\[.5ex]
%   AmS & sy & \hologo{AmS}\\
%   ExTeX & sy & \hologo{ExTeX}\\
%   NTS & sy & \hologo{NTS}\\[.5ex]
%   KOMAScript & sf & \hologo{KOMAScript}\\[.5ex]
%   METAFONT & logo & \hologo{METAFONT}\\
%   METAPOST & logo & \hologo{METAPOST}\\[.5ex]
%   SliTeX & sc \hologo{SliTeX}
% \end{longtable}
%
% \subsubsection{Font order}
%
% For all logos the font \xoption{general} is applied first.
% Example:
%\begin{quote}
%|\hologoFontSetup{general=\color{red}}|
%\end{quote}
% will print red logos.
% Then if the font uses a special font \xoption{sf}, for example,
% the font is applied that is setup by \cs{hologoLogoFontSetup}.
% If this font is not setup, then the common font setup
% by \cs{hologoFontSetup} is used. Otherwise a warning is given,
% that there is no font configured.
%
% \subsection{Additional user macros}
%
% Usually a variant of a logo is configured by using
% \cs{hologoLogoSetup}, because it is bad style to mix
% different variants of the same logo in the same text.
% There the following macros are a convenience for testing.
%
% \begin{declcs}{hologoVariant} \M{name} \M{variant}\\
%   \cs{HologoVariant} \M{name} \M{variant}
% \end{declcs}
% Logo \meta{name} is set using \meta{variant} that specifies
% explicitely which variant of the macro is used. If the argument
% is empty, then the default form of the logo is used
% (configurable by \cs{hologoLogoSetup}).
%
% \cs{HologoVariant} is used if the logo is set in a context
% that needs an uppercase first letter (beginning of a sentence, \dots).
%
% \begin{declcs}{hologoList}\\
%   \cs{hologoEntry} \M{logo} \M{variant} \M{since}
% \end{declcs}
% Macro \cs{hologoList} contains all logos that are provided
% by the package including variants. The list consists of calls
% of \cs{hologoEntry} with three arguments starting with the
% logo name \meta{logo} and its variant \meta{variant}. An empty
% variant means the current default. Argument \meta{since} specifies
% with version of the package \xpackage{hologo} is needed to get
% the logo. If the logo is fixed, then the date gets updated.
% Therefore the date \meta{since} is not exactly the date of
% the first introduction, but rather the date of the latest fix.
%
% Before \cs{hologoList} can be used, macro \cs{hologoEntry} needs
% a definition. The example file in section \ref{sec:example}
% shows applications of \cs{hologoList}.
%
% \subsection{Supported contexts}
%
% Macros \cs{hologo} and friends support special contexts:
% \begin{itemize}
% \item \hologo{LaTeX}'s protection mechanism.
% \item Bookmarks of package \xpackage{hyperref}.
% \item Package \xpackage{tex4ht}.
% \item The macros can be used inside \cs{csname} constructs,
%   if \cs{ifincsname} is available (\hologo{pdfTeX}, \hologo{XeTeX},
%   \hologo{LuaTeX}).
% \end{itemize}
%
% \subsection{Example}
% \label{sec:example}
%
% The following example prints the logos in different fonts.
%    \begin{macrocode}
%<*example>
%<<verbatim
\NeedsTeXFormat{LaTeX2e}
\documentclass[a4paper]{article}
\usepackage[
  hmargin=20mm,
  vmargin=20mm,
]{geometry}
\pagestyle{empty}
\usepackage{hologo}[2016/05/12]
\usepackage{longtable}
\usepackage{array}
\setlength{\extrarowheight}{2pt}
\usepackage[T1]{fontenc}
\usepackage{lmodern}
\usepackage{pdflscape}
\usepackage[
  pdfencoding=auto,
]{hyperref}
\hypersetup{
  pdfauthor={Heiko Oberdiek},
  pdftitle={Example for package `hologo'},
  pdfsubject={Logos with fonts lmr, lmss, qtm, qpl, qhv},
}
\usepackage{bookmark}

% Print the logo list on the console

\begingroup
  \typeout{}%
  \typeout{*** Begin of logo list ***}%
  \newcommand*{\hologoEntry}[3]{%
    \typeout{#1 \ifx\\#2\\\else(#2) \fi[#3]}%
  }%
  \hologoList
  \typeout{*** End of logo list ***}%
  \typeout{}%
\endgroup

\begin{document}
\begin{landscape}

  \section{Example file for package `hologo'}

  % Table for font names

  \begin{longtable}{>{\bfseries}ll}
    \textbf{font} & \textbf{Font name}\\
    \hline
    lmr & Latin Modern Roman\\
    lmss & Latin Modern Sans\\
    qtm & \TeX\ Gyre Termes\\
    qhv & \TeX\ Gyre Heros\\
    qpl & \TeX\ Gyre Pagella\\
  \end{longtable}

  % Logo list with logos in different fonts

  \begingroup
    \newcommand*{\SetVariant}[2]{%
      \ifx\\#2\\%
      \else
        \hologoLogoSetup{#1}{variant=#2}%
      \fi
    }%
    \newcommand*{\hologoEntry}[3]{%
      \SetVariant{#1}{#2}%
      \raisebox{1em}[0pt][0pt]{\hypertarget{#1@#2}{}}%
      \bookmark[%
        dest={#1@#2},%
      ]{%
        #1\ifx\\#2\\\else\space(#2)\fi: \Hologo{#1}, \hologo{#1} %
        [Unicode]%
      }%
      \hypersetup{unicode=false}%
      \bookmark[%
        dest={#1@#2},%
      ]{%
        #1\ifx\\#2\\\else\space(#2)\fi: \Hologo{#1}, \hologo{#1} %
        [PDFDocEncoding]%
      }%
      \texttt{#1}%
      &%
      \texttt{#2}%
      &%
      \Hologo{#1}%
      &%
      \SetVariant{#1}{#2}%
      \hologo{#1}%
      &%
      \SetVariant{#1}{#2}%
      \fontfamily{qtm}\selectfont
      \hologo{#1}%
      &%
      \SetVariant{#1}{#2}%
      \fontfamily{qpl}\selectfont
      \hologo{#1}%
      &%
      \SetVariant{#1}{#2}%
      \textsf{\hologo{#1}}%
      &%
      \SetVariant{#1}{#2}%
      \fontfamily{qhv}\selectfont
      \hologo{#1}%
      \tabularnewline
    }%
    \begin{longtable}{llllllll}%
      \textbf{\textit{logo}} & \textbf{\textit{variant}} &
      \texttt{\string\Hologo} &
      \textbf{lmr} & \textbf{qtm} & \textbf{qpl} &
      \textbf{lmss} & \textbf{qhv}
      \tabularnewline
      \hline
      \endhead
      \hologoList
    \end{longtable}%
  \endgroup

\end{landscape}
\end{document}
%verbatim
%</example>
%    \end{macrocode}
%
% \StopEventually{
% }
%
% \section{Implementation}
%    \begin{macrocode}
%<*package>
%    \end{macrocode}
%    Reload check, especially if the package is not used with \LaTeX.
%    \begin{macrocode}
\begingroup\catcode61\catcode48\catcode32=10\relax%
  \catcode13=5 % ^^M
  \endlinechar=13 %
  \catcode35=6 % #
  \catcode39=12 % '
  \catcode44=12 % ,
  \catcode45=12 % -
  \catcode46=12 % .
  \catcode58=12 % :
  \catcode64=11 % @
  \catcode123=1 % {
  \catcode125=2 % }
  \expandafter\let\expandafter\x\csname ver@hologo.sty\endcsname
  \ifx\x\relax % plain-TeX, first loading
  \else
    \def\empty{}%
    \ifx\x\empty % LaTeX, first loading,
      % variable is initialized, but \ProvidesPackage not yet seen
    \else
      \expandafter\ifx\csname PackageInfo\endcsname\relax
        \def\x#1#2{%
          \immediate\write-1{Package #1 Info: #2.}%
        }%
      \else
        \def\x#1#2{\PackageInfo{#1}{#2, stopped}}%
      \fi
      \x{hologo}{The package is already loaded}%
      \aftergroup\endinput
    \fi
  \fi
\endgroup%
%    \end{macrocode}
%    Package identification:
%    \begin{macrocode}
\begingroup\catcode61\catcode48\catcode32=10\relax%
  \catcode13=5 % ^^M
  \endlinechar=13 %
  \catcode35=6 % #
  \catcode39=12 % '
  \catcode40=12 % (
  \catcode41=12 % )
  \catcode44=12 % ,
  \catcode45=12 % -
  \catcode46=12 % .
  \catcode47=12 % /
  \catcode58=12 % :
  \catcode64=11 % @
  \catcode91=12 % [
  \catcode93=12 % ]
  \catcode123=1 % {
  \catcode125=2 % }
  \expandafter\ifx\csname ProvidesPackage\endcsname\relax
    \def\x#1#2#3[#4]{\endgroup
      \immediate\write-1{Package: #3 #4}%
      \xdef#1{#4}%
    }%
  \else
    \def\x#1#2[#3]{\endgroup
      #2[{#3}]%
      \ifx#1\@undefined
        \xdef#1{#3}%
      \fi
      \ifx#1\relax
        \xdef#1{#3}%
      \fi
    }%
  \fi
\expandafter\x\csname ver@hologo.sty\endcsname
\ProvidesPackage{hologo}%
  [2016/05/12 v1.11 A logo collection with bookmark support (HO)]%
%    \end{macrocode}
%
%    \begin{macrocode}
\begingroup\catcode61\catcode48\catcode32=10\relax%
  \catcode13=5 % ^^M
  \endlinechar=13 %
  \catcode123=1 % {
  \catcode125=2 % }
  \catcode64=11 % @
  \def\x{\endgroup
    \expandafter\edef\csname HOLOGO@AtEnd\endcsname{%
      \endlinechar=\the\endlinechar\relax
      \catcode13=\the\catcode13\relax
      \catcode32=\the\catcode32\relax
      \catcode35=\the\catcode35\relax
      \catcode61=\the\catcode61\relax
      \catcode64=\the\catcode64\relax
      \catcode123=\the\catcode123\relax
      \catcode125=\the\catcode125\relax
    }%
  }%
\x\catcode61\catcode48\catcode32=10\relax%
\catcode13=5 % ^^M
\endlinechar=13 %
\catcode35=6 % #
\catcode64=11 % @
\catcode123=1 % {
\catcode125=2 % }
\def\TMP@EnsureCode#1#2{%
  \edef\HOLOGO@AtEnd{%
    \HOLOGO@AtEnd
    \catcode#1=\the\catcode#1\relax
  }%
  \catcode#1=#2\relax
}
\TMP@EnsureCode{10}{12}% ^^J
\TMP@EnsureCode{33}{12}% !
\TMP@EnsureCode{34}{12}% "
\TMP@EnsureCode{36}{3}% $
\TMP@EnsureCode{38}{4}% &
\TMP@EnsureCode{39}{12}% '
\TMP@EnsureCode{40}{12}% (
\TMP@EnsureCode{41}{12}% )
\TMP@EnsureCode{42}{12}% *
\TMP@EnsureCode{43}{12}% +
\TMP@EnsureCode{44}{12}% ,
\TMP@EnsureCode{45}{12}% -
\TMP@EnsureCode{46}{12}% .
\TMP@EnsureCode{47}{12}% /
\TMP@EnsureCode{58}{12}% :
\TMP@EnsureCode{59}{12}% ;
\TMP@EnsureCode{60}{12}% <
\TMP@EnsureCode{62}{12}% >
\TMP@EnsureCode{63}{12}% ?
\TMP@EnsureCode{91}{12}% [
\TMP@EnsureCode{93}{12}% ]
\TMP@EnsureCode{94}{7}% ^ (superscript)
\TMP@EnsureCode{95}{8}% _ (subscript)
\TMP@EnsureCode{96}{12}% `
\TMP@EnsureCode{124}{12}% |
\edef\HOLOGO@AtEnd{%
  \HOLOGO@AtEnd
  \escapechar\the\escapechar\relax
  \noexpand\endinput
}
\escapechar=92 %
%    \end{macrocode}
%
% \subsection{Logo list}
%
%    \begin{macro}{\hologoList}
%    \begin{macrocode}
\def\hologoList{%
  \hologoEntry{(La)TeX}{}{2011/10/01}%
  \hologoEntry{AmSLaTeX}{}{2010/04/16}%
  \hologoEntry{AmSTeX}{}{2010/04/16}%
  \hologoEntry{biber}{}{2011/10/01}%
  \hologoEntry{BibTeX}{}{2011/10/01}%
  \hologoEntry{BibTeX}{sf}{2011/10/01}%
  \hologoEntry{BibTeX}{sc}{2011/10/01}%
  \hologoEntry{BibTeX8}{}{2011/11/22}%
  \hologoEntry{ConTeXt}{}{2011/03/25}%
  \hologoEntry{ConTeXt}{narrow}{2011/03/25}%
  \hologoEntry{ConTeXt}{simple}{2011/03/25}%
  \hologoEntry{emTeX}{}{2010/04/26}%
  \hologoEntry{eTeX}{}{2010/04/08}%
  \hologoEntry{ExTeX}{}{2011/10/01}%
  \hologoEntry{HanTheThanh}{}{2011/11/29}%
  \hologoEntry{iniTeX}{}{2011/10/01}%
  \hologoEntry{KOMAScript}{}{2011/10/01}%
  \hologoEntry{La}{}{2010/05/08}%
  \hologoEntry{LaTeX}{}{2010/04/08}%
  \hologoEntry{LaTeX2e}{}{2010/04/08}%
  \hologoEntry{LaTeX3}{}{2010/04/24}%
  \hologoEntry{LaTeXe}{}{2010/04/08}%
  \hologoEntry{LaTeXML}{}{2011/11/22}%
  \hologoEntry{LaTeXTeX}{}{2011/10/01}%
  \hologoEntry{LuaLaTeX}{}{2010/04/08}%
  \hologoEntry{LuaTeX}{}{2010/04/08}%
  \hologoEntry{LyX}{}{2011/10/01}%
  \hologoEntry{METAFONT}{}{2011/10/01}%
  \hologoEntry{MetaFun}{}{2011/10/01}%
  \hologoEntry{METAPOST}{}{2011/10/01}%
  \hologoEntry{MetaPost}{}{2011/10/01}%
  \hologoEntry{MiKTeX}{}{2011/10/01}%
  \hologoEntry{NTS}{}{2011/10/01}%
  \hologoEntry{OzMF}{}{2011/10/01}%
  \hologoEntry{OzMP}{}{2011/10/01}%
  \hologoEntry{OzTeX}{}{2011/10/01}%
  \hologoEntry{OzTtH}{}{2011/10/01}%
  \hologoEntry{PCTeX}{}{2011/10/01}%
  \hologoEntry{pdfTeX}{}{2011/10/01}%
  \hologoEntry{pdfLaTeX}{}{2011/10/01}%
  \hologoEntry{PiC}{}{2011/10/01}%
  \hologoEntry{PiCTeX}{}{2011/10/01}%
  \hologoEntry{plainTeX}{}{2010/04/08}%
  \hologoEntry{plainTeX}{space}{2010/04/16}%
  \hologoEntry{plainTeX}{hyphen}{2010/04/16}%
  \hologoEntry{plainTeX}{runtogether}{2010/04/16}%
  \hologoEntry{SageTeX}{}{2011/11/22}%
  \hologoEntry{SLiTeX}{}{2011/10/01}%
  \hologoEntry{SLiTeX}{lift}{2011/10/01}%
  \hologoEntry{SLiTeX}{narrow}{2011/10/01}%
  \hologoEntry{SLiTeX}{simple}{2011/10/01}%
  \hologoEntry{SliTeX}{}{2011/10/01}%
  \hologoEntry{SliTeX}{narrow}{2011/10/01}%
  \hologoEntry{SliTeX}{simple}{2011/10/01}%
  \hologoEntry{SliTeX}{lift}{2011/10/01}%
  \hologoEntry{teTeX}{}{2011/10/01}%
  \hologoEntry{TeX}{}{2010/04/08}%
  \hologoEntry{TeX4ht}{}{2011/11/22}%
  \hologoEntry{TTH}{}{2011/11/22}%
  \hologoEntry{virTeX}{}{2011/10/01}%
  \hologoEntry{VTeX}{}{2010/04/24}%
  \hologoEntry{Xe}{}{2010/04/08}%
  \hologoEntry{XeLaTeX}{}{2010/04/08}%
  \hologoEntry{XeTeX}{}{2010/04/08}%
}
%    \end{macrocode}
%    \end{macro}
%
% \subsection{Load resources}
%
%    \begin{macrocode}
\begingroup\expandafter\expandafter\expandafter\endgroup
\expandafter\ifx\csname RequirePackage\endcsname\relax
  \def\TMP@RequirePackage#1[#2]{%
    \begingroup\expandafter\expandafter\expandafter\endgroup
    \expandafter\ifx\csname ver@#1.sty\endcsname\relax
      \input #1.sty\relax
    \fi
  }%
  \TMP@RequirePackage{ltxcmds}[2011/02/04]%
  \TMP@RequirePackage{infwarerr}[2010/04/08]%
  \TMP@RequirePackage{kvsetkeys}[2010/03/01]%
  \TMP@RequirePackage{kvdefinekeys}[2010/03/01]%
  \TMP@RequirePackage{pdftexcmds}[2010/04/01]%
  \TMP@RequirePackage{ifpdf}[2010/01/28]%
  \TMP@RequirePackage{ifluatex}[2010/03/01]%
  \ltx@IfUndefined{newif}{%
    \expandafter\let\csname newif\endcsname\ltx@newif
  }{}%
  \TMP@RequirePackage{ifxetex}[2009/01/23]%
  \TMP@RequirePackage{ifvtex}[2010/03/01]%
\else
  \RequirePackage{ltxcmds}[2011/02/04]%
  \RequirePackage{infwarerr}[2010/04/08]%
  \RequirePackage{kvsetkeys}[2010/03/01]%
  \RequirePackage{kvdefinekeys}[2010/03/01]%
  \RequirePackage{pdftexcmds}[2010/04/01]%
  \RequirePackage{ifpdf}[2010/01/28]%
  \RequirePackage{ifluatex}[2010/03/01]%
  \RequirePackage{ifxetex}[2009/01/23]%
  \RequirePackage{ifvtex}[2010/03/01]%
\fi
%    \end{macrocode}
%
%    \begin{macro}{\HOLOGO@IfDefined}
%    \begin{macrocode}
\def\HOLOGO@IfExists#1{%
  \ifx\@undefined#1%
    \expandafter\ltx@secondoftwo
  \else
    \ifx\relax#1%
      \expandafter\ltx@secondoftwo
    \else
      \expandafter\expandafter\expandafter\ltx@firstoftwo
    \fi
  \fi
}
%    \end{macrocode}
%    \end{macro}
%
% \subsection{Setup macros}
%
%    \begin{macro}{\hologoSetup}
%    \begin{macrocode}
\def\hologoSetup{%
  \let\HOLOGO@name\relax
  \HOLOGO@Setup
}
%    \end{macrocode}
%    \end{macro}
%
%    \begin{macro}{\hologoLogoSetup}
%    \begin{macrocode}
\def\hologoLogoSetup#1{%
  \edef\HOLOGO@name{#1}%
  \ltx@IfUndefined{HoLogo@\HOLOGO@name}{%
    \@PackageError{hologo}{%
      Unknown logo `\HOLOGO@name'%
    }\@ehc
    \ltx@gobble
  }{%
    \HOLOGO@Setup
  }%
}
%    \end{macrocode}
%    \end{macro}
%
%    \begin{macro}{\HOLOGO@Setup}
%    \begin{macrocode}
\def\HOLOGO@Setup{%
  \kvsetkeys{HoLogo}%
}
%    \end{macrocode}
%    \end{macro}
%
% \subsection{Options}
%
%    \begin{macro}{\HOLOGO@DeclareBoolOption}
%    \begin{macrocode}
\def\HOLOGO@DeclareBoolOption#1{%
  \expandafter\chardef\csname HOLOGOOPT@#1\endcsname\ltx@zero
  \kv@define@key{HoLogo}{#1}[true]{%
    \def\HOLOGO@temp{##1}%
    \ifx\HOLOGO@temp\HOLOGO@true
      \ifx\HOLOGO@name\relax
        \expandafter\chardef\csname HOLOGOOPT@#1\endcsname=\ltx@one
      \else
        \expandafter\chardef\csname
        HoLogoOpt@#1@\HOLOGO@name\endcsname\ltx@one
      \fi
      \HOLOGO@SetBreakAll{#1}%
    \else
      \ifx\HOLOGO@temp\HOLOGO@false
        \ifx\HOLOGO@name\relax
          \expandafter\chardef\csname HOLOGOOPT@#1\endcsname=\ltx@zero
        \else
          \expandafter\chardef\csname
          HoLogoOpt@#1@\HOLOGO@name\endcsname=\ltx@zero
        \fi
        \HOLOGO@SetBreakAll{#1}%
      \else
        \@PackageError{hologo}{%
          Unknown value `##1' for boolean option `#1'.\MessageBreak
          Known values are `true' and `false'%
        }\@ehc
      \fi
    \fi
  }%
}
%    \end{macrocode}
%    \end{macro}
%
%    \begin{macro}{\HOLOGO@SetBreakAll}
%    \begin{macrocode}
\def\HOLOGO@SetBreakAll#1{%
  \def\HOLOGO@temp{#1}%
  \ifx\HOLOGO@temp\HOLOGO@break
    \ifx\HOLOGO@name\relax
      \chardef\HOLOGOOPT@hyphenbreak=\HOLOGOOPT@break
      \chardef\HOLOGOOPT@spacebreak=\HOLOGOOPT@break
      \chardef\HOLOGOOPT@discretionarybreak=\HOLOGOOPT@break
    \else
      \expandafter\chardef
         \csname HoLogoOpt@hyphenbreak@\HOLOGO@name\endcsname=%
         \csname HoLogoOpt@break@\HOLOGO@name\endcsname
      \expandafter\chardef
         \csname HoLogoOpt@spacebreak@\HOLOGO@name\endcsname=%
         \csname HoLogoOpt@break@\HOLOGO@name\endcsname
      \expandafter\chardef
         \csname HoLogoOpt@discretionarybreak@\HOLOGO@name
             \endcsname=%
         \csname HoLogoOpt@break@\HOLOGO@name\endcsname
    \fi
  \fi
}
%    \end{macrocode}
%    \end{macro}
%
%    \begin{macro}{\HOLOGO@true}
%    \begin{macrocode}
\def\HOLOGO@true{true}
%    \end{macrocode}
%    \end{macro}
%    \begin{macro}{\HOLOGO@false}
%    \begin{macrocode}
\def\HOLOGO@false{false}
%    \end{macrocode}
%    \end{macro}
%    \begin{macro}{\HOLOGO@break}
%    \begin{macrocode}
\def\HOLOGO@break{break}
%    \end{macrocode}
%    \end{macro}
%
%    \begin{macrocode}
\HOLOGO@DeclareBoolOption{break}
\HOLOGO@DeclareBoolOption{hyphenbreak}
\HOLOGO@DeclareBoolOption{spacebreak}
\HOLOGO@DeclareBoolOption{discretionarybreak}
%    \end{macrocode}
%
%    \begin{macrocode}
\kv@define@key{HoLogo}{variant}{%
  \ifx\HOLOGO@name\relax
    \@PackageError{hologo}{%
      Option `variant' is not available in \string\hologoSetup,%
      \MessageBreak
      Use \string\hologoLogoSetup\space instead%
    }\@ehc
  \else
    \edef\HOLOGO@temp{#1}%
    \ifx\HOLOGO@temp\ltx@empty
      \expandafter
      \let\csname HoLogoOpt@variant@\HOLOGO@name\endcsname\@undefined
    \else
      \ltx@IfUndefined{HoLogo@\HOLOGO@name @\HOLOGO@temp}{%
        \@PackageError{hologo}{%
          Unknown variant `\HOLOGO@temp' of logo `\HOLOGO@name'%
        }\@ehc
      }{%
        \expandafter
        \let\csname HoLogoOpt@variant@\HOLOGO@name\endcsname
            \HOLOGO@temp
      }%
    \fi
  \fi
}
%    \end{macrocode}
%
%    \begin{macro}{\HOLOGO@Variant}
%    \begin{macrocode}
\def\HOLOGO@Variant#1{%
  #1%
  \ltx@ifundefined{HoLogoOpt@variant@#1}{%
  }{%
    @\csname HoLogoOpt@variant@#1\endcsname
  }%
}
%    \end{macrocode}
%    \end{macro}
%
% \subsection{Break/no-break support}
%
%    \begin{macro}{\HOLOGO@space}
%    \begin{macrocode}
\def\HOLOGO@space{%
  \ltx@ifundefined{HoLogoOpt@spacebreak@\HOLOGO@name}{%
    \ltx@ifundefined{HoLogoOpt@break@\HOLOGO@name}{%
      \chardef\HOLOGO@temp=\HOLOGOOPT@spacebreak
    }{%
      \chardef\HOLOGO@temp=%
        \csname HoLogoOpt@break@\HOLOGO@name\endcsname
    }%
  }{%
    \chardef\HOLOGO@temp=%
      \csname HoLogoOpt@spacebreak@\HOLOGO@name\endcsname
  }%
  \ifcase\HOLOGO@temp
    \penalty10000 %
  \fi
  \ltx@space
}
%    \end{macrocode}
%    \end{macro}
%
%    \begin{macro}{\HOLOGO@hyphen}
%    \begin{macrocode}
\def\HOLOGO@hyphen{%
  \ltx@ifundefined{HoLogoOpt@hyphenbreak@\HOLOGO@name}{%
    \ltx@ifundefined{HoLogoOpt@break@\HOLOGO@name}{%
      \chardef\HOLOGO@temp=\HOLOGOOPT@hyphenbreak
    }{%
      \chardef\HOLOGO@temp=%
        \csname HoLogoOpt@break@\HOLOGO@name\endcsname
    }%
  }{%
    \chardef\HOLOGO@temp=%
      \csname HoLogoOpt@hyphenbreak@\HOLOGO@name\endcsname
  }%
  \ifcase\HOLOGO@temp
    \ltx@mbox{-}%
  \else
    -%
  \fi
}
%    \end{macrocode}
%    \end{macro}
%
%    \begin{macro}{\HOLOGO@discretionary}
%    \begin{macrocode}
\def\HOLOGO@discretionary{%
  \ltx@ifundefined{HoLogoOpt@discretionarybreak@\HOLOGO@name}{%
    \ltx@ifundefined{HoLogoOpt@break@\HOLOGO@name}{%
      \chardef\HOLOGO@temp=\HOLOGOOPT@discretionarybreak
    }{%
      \chardef\HOLOGO@temp=%
        \csname HoLogoOpt@break@\HOLOGO@name\endcsname
    }%
  }{%
    \chardef\HOLOGO@temp=%
      \csname HoLogoOpt@discretionarybreak@\HOLOGO@name\endcsname
  }%
  \ifcase\HOLOGO@temp
  \else
    \-%
  \fi
}
%    \end{macrocode}
%    \end{macro}
%
%    \begin{macro}{\HOLOGO@mbox}
%    \begin{macrocode}
\def\HOLOGO@mbox#1{%
  \ltx@ifundefined{HoLogoOpt@break@\HOLOGO@name}{%
    \chardef\HOLOGO@temp=\HOLOGOOPT@hyphenbreak
  }{%
    \chardef\HOLOGO@temp=%
      \csname HoLogoOpt@break@\HOLOGO@name\endcsname
  }%
  \ifcase\HOLOGO@temp
    \ltx@mbox{#1}%
  \else
    #1%
  \fi
}
%    \end{macrocode}
%    \end{macro}
%
% \subsection{Font support}
%
%    \begin{macro}{\HoLogoFont@font}
%    \begin{tabular}{@{}ll@{}}
%    |#1|:& logo name\\
%    |#2|:& font short name\\
%    |#3|:& text
%    \end{tabular}
%    \begin{macrocode}
\def\HoLogoFont@font#1#2#3{%
  \begingroup
    \ltx@IfUndefined{HoLogoFont@logo@#1.#2}{%
      \ltx@IfUndefined{HoLogoFont@font@#2}{%
        \@PackageWarning{hologo}{%
          Missing font `#2' for logo `#1'%
        }%
        #3%
      }{%
        \csname HoLogoFont@font@#2\endcsname{#3}%
      }%
    }{%
      \csname HoLogoFont@logo@#1.#2\endcsname{#3}%
    }%
  \endgroup
}
%    \end{macrocode}
%    \end{macro}
%
%    \begin{macro}{\HoLogoFont@Def}
%    \begin{macrocode}
\def\HoLogoFont@Def#1{%
  \expandafter\def\csname HoLogoFont@font@#1\endcsname
}
%    \end{macrocode}
%    \end{macro}
%    \begin{macro}{\HoLogoFont@LogoDef}
%    \begin{macrocode}
\def\HoLogoFont@LogoDef#1#2{%
  \expandafter\def\csname HoLogoFont@logo@#1.#2\endcsname
}
%    \end{macrocode}
%    \end{macro}
%
% \subsubsection{Font defaults}
%
%    \begin{macro}{\HoLogoFont@font@general}
%    \begin{macrocode}
\HoLogoFont@Def{general}{}%
%    \end{macrocode}
%    \end{macro}
%
%    \begin{macro}{\HoLogoFont@font@rm}
%    \begin{macrocode}
\ltx@IfUndefined{rmfamily}{%
  \ltx@IfUndefined{rm}{%
  }{%
    \HoLogoFont@Def{rm}{\rm}%
  }%
}{%
  \HoLogoFont@Def{rm}{\rmfamily}%
}
%    \end{macrocode}
%    \end{macro}
%
%    \begin{macro}{\HoLogoFont@font@sf}
%    \begin{macrocode}
\ltx@IfUndefined{sffamily}{%
  \ltx@IfUndefined{sf}{%
  }{%
    \HoLogoFont@Def{sf}{\sf}%
  }%
}{%
  \HoLogoFont@Def{sf}{\sffamily}%
}
%    \end{macrocode}
%    \end{macro}
%
%    \begin{macro}{\HoLogoFont@font@bibsf}
%    In case of \hologo{plainTeX} the original small caps
%    variant is used as default. In \hologo{LaTeX}
%    the definition of package \xpackage{dtklogos} \cite{dtklogos}
%    is used.
%\begin{quote}
%\begin{verbatim}
%\DeclareRobustCommand{\BibTeX}{%
%  B%
%  \kern-.05em%
%  \hbox{%
%    $\m@th$% %% force math size calculations
%    \csname S@\f@size\endcsname
%    \fontsize\sf@size\z@
%    \math@fontsfalse
%    \selectfont
%    I%
%    \kern-.025em%
%    B
%  }%
%  \kern-.08em%
%  \-%
%  \TeX
%}
%\end{verbatim}
%\end{quote}
%    \begin{macrocode}
\ltx@IfUndefined{selectfont}{%
  \ltx@IfUndefined{tensc}{%
    \font\tensc=cmcsc10\relax
  }{}%
  \HoLogoFont@Def{bibsf}{\tensc}%
}{%
  \HoLogoFont@Def{bibsf}{%
    $\mathsurround=0pt$%
    \csname S@\f@size\endcsname
    \fontsize\sf@size{0pt}%
    \math@fontsfalse
    \selectfont
  }%
}
%    \end{macrocode}
%    \end{macro}
%
%    \begin{macro}{\HoLogoFont@font@sc}
%    \begin{macrocode}
\ltx@IfUndefined{scshape}{%
  \ltx@IfUndefined{tensc}{%
    \font\tensc=cmcsc10\relax
  }{}%
  \HoLogoFont@Def{sc}{\tensc}%
}{%
  \HoLogoFont@Def{sc}{\scshape}%
}
%    \end{macrocode}
%    \end{macro}
%
%    \begin{macro}{\HoLogoFont@font@sy}
%    \begin{macrocode}
\ltx@IfUndefined{usefont}{%
  \ltx@IfUndefined{tensy}{%
  }{%
    \HoLogoFont@Def{sy}{\tensy}%
  }%
}{%
  \HoLogoFont@Def{sy}{%
    \usefont{OMS}{cmsy}{m}{n}%
  }%
}
%    \end{macrocode}
%    \end{macro}
%
%    \begin{macro}{\HoLogoFont@font@logo}
%    \begin{macrocode}
\begingroup
  \def\x{LaTeX2e}%
\expandafter\endgroup
\ifx\fmtname\x
  \ltx@IfUndefined{logofamily}{%
    \DeclareRobustCommand\logofamily{%
      \not@math@alphabet\logofamily\relax
      \fontencoding{U}%
      \fontfamily{logo}%
      \selectfont
    }%
  }{}%
  \ltx@IfUndefined{logofamily}{%
  }{%
    \HoLogoFont@Def{logo}{\logofamily}%
  }%
\else
  \ltx@IfUndefined{tenlogo}{%
    \font\tenlogo=logo10\relax
  }{}%
  \HoLogoFont@Def{logo}{\tenlogo}%
\fi
%    \end{macrocode}
%    \end{macro}
%
% \subsubsection{Font setup}
%
%    \begin{macro}{\hologoFontSetup}
%    \begin{macrocode}
\def\hologoFontSetup{%
  \let\HOLOGO@name\relax
  \HOLOGO@FontSetup
}
%    \end{macrocode}
%    \end{macro}
%
%    \begin{macro}{\hologoLogoFontSetup}
%    \begin{macrocode}
\def\hologoLogoFontSetup#1{%
  \edef\HOLOGO@name{#1}%
  \ltx@IfUndefined{HoLogo@\HOLOGO@name}{%
    \@PackageError{hologo}{%
      Unknown logo `\HOLOGO@name'%
    }\@ehc
    \ltx@gobble
  }{%
    \HOLOGO@FontSetup
  }%
}
%    \end{macrocode}
%    \end{macro}
%
%    \begin{macro}{\HOLOGO@FontSetup}
%    \begin{macrocode}
\def\HOLOGO@FontSetup{%
  \kvsetkeys{HoLogoFont}%
}
%    \end{macrocode}
%    \end{macro}
%
%    \begin{macrocode}
\def\HOLOGO@temp#1{%
  \kv@define@key{HoLogoFont}{#1}{%
    \ifx\HOLOGO@name\relax
      \HoLogoFont@Def{#1}{##1}%
    \else
      \HoLogoFont@LogoDef\HOLOGO@name{#1}{##1}%
    \fi
  }%
}
\HOLOGO@temp{general}
\HOLOGO@temp{sf}
%    \end{macrocode}
%
% \subsection{Generic logo commands}
%
%    \begin{macrocode}
\HOLOGO@IfExists\hologo{%
  \@PackageError{hologo}{%
    \string\hologo\ltx@space is already defined.\MessageBreak
    Package loading is aborted%
  }\@ehc
  \HOLOGO@AtEnd
}%
\HOLOGO@IfExists\hologoRobust{%
  \@PackageError{hologo}{%
    \string\hologoRobust\ltx@space is already defined.\MessageBreak
    Package loading is aborted%
  }\@ehc
  \HOLOGO@AtEnd
}%
%    \end{macrocode}
%
% \subsubsection{\cs{hologo} and friends}
%
%    \begin{macrocode}
\ifluatex
  \expandafter\ltx@firstofone
\else
  \expandafter\ltx@gobble
\fi
{%
  \ltx@IfUndefined{ifincsname}{%
    \ifnum\luatexversion<36 %
      \expandafter\ltx@gobble
    \else
      \expandafter\ltx@firstofone
    \fi
    {%
      \begingroup
        \ifcase0%
            \directlua{%
              if tex.enableprimitives then %
                tex.enableprimitives('HOLOGO@', {'ifincsname'})%
              else %
                tex.print('1')%
              end%
            }%
            \ifx\HOLOGO@ifincsname\@undefined 1\fi%
            \relax
          \expandafter\ltx@firstofone
        \else
          \endgroup
          \expandafter\ltx@gobble
        \fi
        {%
          \global\let\ifincsname\HOLOGO@ifincsname
        }%
      \HOLOGO@temp
    }%
  }{}%
}
%    \end{macrocode}
%    \begin{macrocode}
\ltx@IfUndefined{ifincsname}{%
  \catcode`$=14 %
}{%
  \catcode`$=9 %
}
%    \end{macrocode}
%
%    \begin{macro}{\hologo}
%    \begin{macrocode}
\def\hologo#1{%
$ \ifincsname
$   \ltx@ifundefined{HoLogoCs@\HOLOGO@Variant{#1}}{%
$     #1%
$   }{%
$     \csname HoLogoCs@\HOLOGO@Variant{#1}\endcsname\ltx@firstoftwo
$   }%
$ \else
    \HOLOGO@IfExists\texorpdfstring\texorpdfstring\ltx@firstoftwo
    {%
      \hologoRobust{#1}%
    }{%
      \ltx@ifundefined{HoLogoBkm@\HOLOGO@Variant{#1}}{%
        \ltx@ifundefined{HoLogo@#1}{?#1?}{#1}%
      }{%
        \csname HoLogoBkm@\HOLOGO@Variant{#1}\endcsname
        \ltx@firstoftwo
      }%
    }%
$ \fi
}
%    \end{macrocode}
%    \end{macro}
%    \begin{macro}{\Hologo}
%    \begin{macrocode}
\def\Hologo#1{%
$ \ifincsname
$   \ltx@ifundefined{HoLogoCs@\HOLOGO@Variant{#1}}{%
$     #1%
$   }{%
$     \csname HoLogoCs@\HOLOGO@Variant{#1}\endcsname\ltx@secondoftwo
$   }%
$ \else
    \HOLOGO@IfExists\texorpdfstring\texorpdfstring\ltx@firstoftwo
    {%
      \HologoRobust{#1}%
    }{%
      \ltx@ifundefined{HoLogoBkm@\HOLOGO@Variant{#1}}{%
        \ltx@ifundefined{HoLogo@#1}{?#1?}{#1}%
      }{%
        \csname HoLogoBkm@\HOLOGO@Variant{#1}\endcsname
        \ltx@secondoftwo
      }%
    }%
$ \fi
}
%    \end{macrocode}
%    \end{macro}
%
%    \begin{macro}{\hologoVariant}
%    \begin{macrocode}
\def\hologoVariant#1#2{%
  \ifx\relax#2\relax
    \hologo{#1}%
  \else
$   \ifincsname
$     \ltx@ifundefined{HoLogoCs@#1@#2}{%
$       #1%
$     }{%
$       \csname HoLogoCs@#1@#2\endcsname\ltx@firstoftwo
$     }%
$   \else
      \HOLOGO@IfExists\texorpdfstring\texorpdfstring\ltx@firstoftwo
      {%
        \hologoVariantRobust{#1}{#2}%
      }{%
        \ltx@ifundefined{HoLogoBkm@#1@#2}{%
          \ltx@ifundefined{HoLogo@#1}{?#1?}{#1}%
        }{%
          \csname HoLogoBkm@#1@#2\endcsname
          \ltx@firstoftwo
        }%
      }%
$   \fi
  \fi
}
%    \end{macrocode}
%    \end{macro}
%    \begin{macro}{\HologoVariant}
%    \begin{macrocode}
\def\HologoVariant#1#2{%
  \ifx\relax#2\relax
    \Hologo{#1}%
  \else
$   \ifincsname
$     \ltx@ifundefined{HoLogoCs@#1@#2}{%
$       #1%
$     }{%
$       \csname HoLogoCs@#1@#2\endcsname\ltx@secondoftwo
$     }%
$   \else
      \HOLOGO@IfExists\texorpdfstring\texorpdfstring\ltx@firstoftwo
      {%
        \HologoVariantRobust{#1}{#2}%
      }{%
        \ltx@ifundefined{HoLogoBkm@#1@#2}{%
          \ltx@ifundefined{HoLogo@#1}{?#1?}{#1}%
        }{%
          \csname HoLogoBkm@#1@#2\endcsname
          \ltx@secondoftwo
        }%
      }%
$   \fi
  \fi
}
%    \end{macrocode}
%    \end{macro}
%
%    \begin{macrocode}
\catcode`\$=3 %
%    \end{macrocode}
%
% \subsubsection{\cs{hologoRobust} and friends}
%
%    \begin{macro}{\hologoRobust}
%    \begin{macrocode}
\ltx@IfUndefined{protected}{%
  \ltx@IfUndefined{DeclareRobustCommand}{%
    \def\hologoRobust#1%
  }{%
    \DeclareRobustCommand*\hologoRobust[1]%
  }%
}{%
  \protected\def\hologoRobust#1%
}%
{%
  \edef\HOLOGO@name{#1}%
  \ltx@IfUndefined{HoLogo@\HOLOGO@Variant\HOLOGO@name}{%
    \@PackageError{hologo}{%
      Unknown logo `\HOLOGO@name'%
    }\@ehc
    ?\HOLOGO@name?%
  }{%
    \ltx@IfUndefined{ver@tex4ht.sty}{%
      \HoLogoFont@font\HOLOGO@name{general}{%
        \csname HoLogo@\HOLOGO@Variant\HOLOGO@name\endcsname
        \ltx@firstoftwo
      }%
    }{%
      \ltx@IfUndefined{HoLogoHtml@\HOLOGO@Variant\HOLOGO@name}{%
        \HOLOGO@name
      }{%
        \csname HoLogoHtml@\HOLOGO@Variant\HOLOGO@name\endcsname
        \ltx@firstoftwo
      }%
    }%
  }%
}
%    \end{macrocode}
%    \end{macro}
%    \begin{macro}{\HologoRobust}
%    \begin{macrocode}
\ltx@IfUndefined{protected}{%
  \ltx@IfUndefined{DeclareRobustCommand}{%
    \def\HologoRobust#1%
  }{%
    \DeclareRobustCommand*\HologoRobust[1]%
  }%
}{%
  \protected\def\HologoRobust#1%
}%
{%
  \edef\HOLOGO@name{#1}%
  \ltx@IfUndefined{HoLogo@\HOLOGO@Variant\HOLOGO@name}{%
    \@PackageError{hologo}{%
      Unknown logo `\HOLOGO@name'%
    }\@ehc
    ?\HOLOGO@name?%
  }{%
    \ltx@IfUndefined{ver@tex4ht.sty}{%
      \HoLogoFont@font\HOLOGO@name{general}{%
        \csname HoLogo@\HOLOGO@Variant\HOLOGO@name\endcsname
        \ltx@secondoftwo
      }%
    }{%
      \ltx@IfUndefined{HoLogoHtml@\HOLOGO@Variant\HOLOGO@name}{%
        \expandafter\HOLOGO@Uppercase\HOLOGO@name
      }{%
        \csname HoLogoHtml@\HOLOGO@Variant\HOLOGO@name\endcsname
        \ltx@secondoftwo
      }%
    }%
  }%
}
%    \end{macrocode}
%    \end{macro}
%    \begin{macro}{\hologoVariantRobust}
%    \begin{macrocode}
\ltx@IfUndefined{protected}{%
  \ltx@IfUndefined{DeclareRobustCommand}{%
    \def\hologoVariantRobust#1#2%
  }{%
    \DeclareRobustCommand*\hologoVariantRobust[2]%
  }%
}{%
  \protected\def\hologoVariantRobust#1#2%
}%
{%
  \begingroup
    \hologoLogoSetup{#1}{variant={#2}}%
    \hologoRobust{#1}%
  \endgroup
}
%    \end{macrocode}
%    \end{macro}
%    \begin{macro}{\HologoVariantRobust}
%    \begin{macrocode}
\ltx@IfUndefined{protected}{%
  \ltx@IfUndefined{DeclareRobustCommand}{%
    \def\HologoVariantRobust#1#2%
  }{%
    \DeclareRobustCommand*\HologoVariantRobust[2]%
  }%
}{%
  \protected\def\HologoVariantRobust#1#2%
}%
{%
  \begingroup
    \hologoLogoSetup{#1}{variant={#2}}%
    \HologoRobust{#1}%
  \endgroup
}
%    \end{macrocode}
%    \end{macro}
%
%    \begin{macro}{\hologorobust}
%    Macro \cs{hologorobust} is only defined for compatibility.
%    Its use is deprecated.
%    \begin{macrocode}
\def\hologorobust{\hologoRobust}
%    \end{macrocode}
%    \end{macro}
%
% \subsection{Helpers}
%
%    \begin{macro}{\HOLOGO@Uppercase}
%    Macro \cs{HOLOGO@Uppercase} is restricted to \cs{uppercase},
%    because \hologo{plainTeX} or \hologo{iniTeX} do not provide
%    \cs{MakeUppercase}.
%    \begin{macrocode}
\def\HOLOGO@Uppercase#1{\uppercase{#1}}
%    \end{macrocode}
%    \end{macro}
%
%    \begin{macro}{\HOLOGO@PdfdocUnicode}
%    \begin{macrocode}
\def\HOLOGO@PdfdocUnicode{%
  \ifx\ifHy@unicode\iftrue
    \expandafter\ltx@secondoftwo
  \else
    \expandafter\ltx@firstoftwo
  \fi
}
%    \end{macrocode}
%    \end{macro}
%
%    \begin{macro}{\HOLOGO@Math}
%    \begin{macrocode}
\def\HOLOGO@MathSetup{%
  \mathsurround0pt\relax
  \HOLOGO@IfExists\f@series{%
    \if b\expandafter\ltx@car\f@series x\@nil
      \csname boldmath\endcsname
   \fi
  }{}%
}
%    \end{macrocode}
%    \end{macro}
%
%    \begin{macro}{\HOLOGO@TempDimen}
%    \begin{macrocode}
\dimendef\HOLOGO@TempDimen=\ltx@zero
%    \end{macrocode}
%    \end{macro}
%    \begin{macro}{\HOLOGO@NegativeKerning}
%    \begin{macrocode}
\def\HOLOGO@NegativeKerning#1{%
  \begingroup
    \HOLOGO@TempDimen=0pt\relax
    \comma@parse@normalized{#1}{%
      \ifdim\HOLOGO@TempDimen=0pt %
        \expandafter\HOLOGO@@NegativeKerning\comma@entry
      \fi
      \ltx@gobble
    }%
    \ifdim\HOLOGO@TempDimen<0pt %
      \kern\HOLOGO@TempDimen
    \fi
  \endgroup
}
%    \end{macrocode}
%    \end{macro}
%    \begin{macro}{\HOLOGO@@NegativeKerning}
%    \begin{macrocode}
\def\HOLOGO@@NegativeKerning#1#2{%
  \setbox\ltx@zero\hbox{#1#2}%
  \HOLOGO@TempDimen=\wd\ltx@zero
  \setbox\ltx@zero\hbox{#1\kern0pt#2}%
  \advance\HOLOGO@TempDimen by -\wd\ltx@zero
}
%    \end{macrocode}
%    \end{macro}
%
%    \begin{macro}{\HOLOGO@SpaceFactor}
%    \begin{macrocode}
\def\HOLOGO@SpaceFactor{%
  \spacefactor1000 %
}
%    \end{macrocode}
%    \end{macro}
%
%    \begin{macro}{\HOLOGO@Span}
%    \begin{macrocode}
\def\HOLOGO@Span#1#2{%
  \HCode{<span class="HoLogo-#1">}%
  #2%
  \HCode{</span>}%
}
%    \end{macrocode}
%    \end{macro}
%
% \subsubsection{Text subscript}
%
%    \begin{macro}{\HOLOGO@SubScript}%
%    \begin{macrocode}
\def\HOLOGO@SubScript#1{%
  \ltx@IfUndefined{textsubscript}{%
    \ltx@IfUndefined{text}{%
      \ltx@mbox{%
        \mathsurround=0pt\relax
        $%
          _{%
            \ltx@IfUndefined{sf@size}{%
              \mathrm{#1}%
            }{%
              \mbox{%
                \fontsize\sf@size{0pt}\selectfont
                #1%
              }%
            }%
          }%
        $%
      }%
    }{%
      \ltx@mbox{%
        \mathsurround=0pt\relax
        $_{\text{#1}}$%
      }%
    }%
  }{%
    \textsubscript{#1}%
  }%
}
%    \end{macrocode}
%    \end{macro}
%
% \subsection{\hologo{TeX} and friends}
%
% \subsubsection{\hologo{TeX}}
%
%    \begin{macro}{\HoLogo@TeX}
%    Source: \hologo{LaTeX} kernel.
%    \begin{macrocode}
\def\HoLogo@TeX#1{%
  T\kern-.1667em\lower.5ex\hbox{E}\kern-.125emX\HOLOGO@SpaceFactor
}
%    \end{macrocode}
%    \end{macro}
%    \begin{macro}{\HoLogoHtml@TeX}
%    \begin{macrocode}
\def\HoLogoHtml@TeX#1{%
  \HoLogoCss@TeX
  \HOLOGO@Span{TeX}{%
    T%
    \HOLOGO@Span{e}{%
      E%
    }%
    X%
  }%
}
%    \end{macrocode}
%    \end{macro}
%    \begin{macro}{\HoLogoCss@TeX}
%    \begin{macrocode}
\def\HoLogoCss@TeX{%
  \Css{%
    span.HoLogo-TeX span.HoLogo-e{%
      position:relative;%
      top:.5ex;%
      margin-left:-.1667em;%
      margin-right:-.125em;%
    }%
  }%
  \Css{%
    a span.HoLogo-TeX span.HoLogo-e{%
      text-decoration:none;%
    }%
  }%
  \global\let\HoLogoCss@TeX\relax
}
%    \end{macrocode}
%    \end{macro}
%
% \subsubsection{\hologo{plainTeX}}
%
%    \begin{macro}{\HoLogo@plainTeX@space}
%    Source: ``The \hologo{TeX}book''
%    \begin{macrocode}
\def\HoLogo@plainTeX@space#1{%
  \HOLOGO@mbox{#1{p}{P}lain}\HOLOGO@space\hologo{TeX}%
}
%    \end{macrocode}
%    \end{macro}
%    \begin{macro}{\HoLogoCs@plainTeX@space}
%    \begin{macrocode}
\def\HoLogoCs@plainTeX@space#1{#1{p}{P}lain TeX}%
%    \end{macrocode}
%    \end{macro}
%    \begin{macro}{\HoLogoBkm@plainTeX@space}
%    \begin{macrocode}
\def\HoLogoBkm@plainTeX@space#1{%
  #1{p}{P}lain \hologo{TeX}%
}
%    \end{macrocode}
%    \end{macro}
%    \begin{macro}{\HoLogoHtml@plainTeX@space}
%    \begin{macrocode}
\def\HoLogoHtml@plainTeX@space#1{%
  #1{p}{P}lain \hologo{TeX}%
}
%    \end{macrocode}
%    \end{macro}
%
%    \begin{macro}{\HoLogo@plainTeX@hyphen}
%    \begin{macrocode}
\def\HoLogo@plainTeX@hyphen#1{%
  \HOLOGO@mbox{#1{p}{P}lain}\HOLOGO@hyphen\hologo{TeX}%
}
%    \end{macrocode}
%    \end{macro}
%    \begin{macro}{\HoLogoCs@plainTeX@hyphen}
%    \begin{macrocode}
\def\HoLogoCs@plainTeX@hyphen#1{#1{p}{P}lain-TeX}
%    \end{macrocode}
%    \end{macro}
%    \begin{macro}{\HoLogoBkm@plainTeX@hyphen}
%    \begin{macrocode}
\def\HoLogoBkm@plainTeX@hyphen#1{%
  #1{p}{P}lain-\hologo{TeX}%
}
%    \end{macrocode}
%    \end{macro}
%    \begin{macro}{\HoLogoHtml@plainTeX@hyphen}
%    \begin{macrocode}
\def\HoLogoHtml@plainTeX@hyphen#1{%
  #1{p}{P}lain-\hologo{TeX}%
}
%    \end{macrocode}
%    \end{macro}
%
%    \begin{macro}{\HoLogo@plainTeX@runtogether}
%    \begin{macrocode}
\def\HoLogo@plainTeX@runtogether#1{%
  \HOLOGO@mbox{#1{p}{P}lain\hologo{TeX}}%
}
%    \end{macrocode}
%    \end{macro}
%    \begin{macro}{\HoLogoCs@plainTeX@runtogether}
%    \begin{macrocode}
\def\HoLogoCs@plainTeX@runtogether#1{#1{p}{P}lainTeX}
%    \end{macrocode}
%    \end{macro}
%    \begin{macro}{\HoLogoBkm@plainTeX@runtogether}
%    \begin{macrocode}
\def\HoLogoBkm@plainTeX@runtogether#1{%
  #1{p}{P}lain\hologo{TeX}%
}
%    \end{macrocode}
%    \end{macro}
%    \begin{macro}{\HoLogoHtml@plainTeX@runtogether}
%    \begin{macrocode}
\def\HoLogoHtml@plainTeX@runtogether#1{%
  #1{p}{P}lain\hologo{TeX}%
}
%    \end{macrocode}
%    \end{macro}
%
%    \begin{macro}{\HoLogo@plainTeX}
%    \begin{macrocode}
\def\HoLogo@plainTeX{\HoLogo@plainTeX@space}
%    \end{macrocode}
%    \end{macro}
%    \begin{macro}{\HoLogoCs@plainTeX}
%    \begin{macrocode}
\def\HoLogoCs@plainTeX{\HoLogoCs@plainTeX@space}
%    \end{macrocode}
%    \end{macro}
%    \begin{macro}{\HoLogoBkm@plainTeX}
%    \begin{macrocode}
\def\HoLogoBkm@plainTeX{\HoLogoBkm@plainTeX@space}
%    \end{macrocode}
%    \end{macro}
%    \begin{macro}{\HoLogoHtml@plainTeX}
%    \begin{macrocode}
\def\HoLogoHtml@plainTeX{\HoLogoHtml@plainTeX@space}
%    \end{macrocode}
%    \end{macro}
%
% \subsubsection{\hologo{LaTeX}}
%
%    Source: \hologo{LaTeX} kernel.
%\begin{quote}
%\begin{verbatim}
%\DeclareRobustCommand{\LaTeX}{%
%  L%
%  \kern-.36em%
%  {%
%    \sbox\z@ T%
%    \vbox to\ht\z@{%
%      \hbox{%
%        \check@mathfonts
%        \fontsize\sf@size\z@
%        \math@fontsfalse
%        \selectfont
%        A%
%      }%
%      \vss
%    }%
%  }%
%  \kern-.15em%
%  \TeX
%}
%\end{verbatim}
%\end{quote}
%
%    \begin{macro}{\HoLogo@La}
%    \begin{macrocode}
\def\HoLogo@La#1{%
  L%
  \kern-.36em%
  \begingroup
    \setbox\ltx@zero\hbox{T}%
    \vbox to\ht\ltx@zero{%
      \hbox{%
        \ltx@ifundefined{check@mathfonts}{%
          \csname sevenrm\endcsname
        }{%
          \check@mathfonts
          \fontsize\sf@size{0pt}%
          \math@fontsfalse\selectfont
        }%
        A%
      }%
      \vss
    }%
  \endgroup
}
%    \end{macrocode}
%    \end{macro}
%
%    \begin{macro}{\HoLogo@LaTeX}
%    Source: \hologo{LaTeX} kernel.
%    \begin{macrocode}
\def\HoLogo@LaTeX#1{%
  \hologo{La}%
  \kern-.15em%
  \hologo{TeX}%
}
%    \end{macrocode}
%    \end{macro}
%    \begin{macro}{\HoLogoHtml@LaTeX}
%    \begin{macrocode}
\def\HoLogoHtml@LaTeX#1{%
  \HoLogoCss@LaTeX
  \HOLOGO@Span{LaTeX}{%
    L%
    \HOLOGO@Span{a}{%
      A%
    }%
    \hologo{TeX}%
  }%
}
%    \end{macrocode}
%    \end{macro}
%    \begin{macro}{\HoLogoCss@LaTeX}
%    \begin{macrocode}
\def\HoLogoCss@LaTeX{%
  \Css{%
    span.HoLogo-LaTeX span.HoLogo-a{%
      position:relative;%
      top:-.5ex;%
      margin-left:-.36em;%
      margin-right:-.15em;%
      font-size:85\%;%
    }%
  }%
  \global\let\HoLogoCss@LaTeX\relax
}
%    \end{macrocode}
%    \end{macro}
%
% \subsubsection{\hologo{(La)TeX}}
%
%    \begin{macro}{\HoLogo@LaTeXTeX}
%    The kerning around the parentheses is taken
%    from package \xpackage{dtklogos} \cite{dtklogos}.
%\begin{quote}
%\begin{verbatim}
%\DeclareRobustCommand{\LaTeXTeX}{%
%  (%
%  \kern-.15em%
%  L%
%  \kern-.36em%
%  {%
%    \sbox\z@ T%
%    \vbox to\ht0{%
%      \hbox{%
%        $\m@th$%
%        \csname S@\f@size\endcsname
%        \fontsize\sf@size\z@
%        \math@fontsfalse
%        \selectfont
%        A%
%      }%
%      \vss
%    }%
%  }%
%  \kern-.2em%
%  )%
%  \kern-.15em%
%  \TeX
%}
%\end{verbatim}
%\end{quote}
%    \begin{macrocode}
\def\HoLogo@LaTeXTeX#1{%
  (%
  \kern-.15em%
  \hologo{La}%
  \kern-.2em%
  )%
  \kern-.15em%
  \hologo{TeX}%
}
%    \end{macrocode}
%    \end{macro}
%    \begin{macro}{\HoLogoBkm@LaTeXTeX}
%    \begin{macrocode}
\def\HoLogoBkm@LaTeXTeX#1{(La)TeX}
%    \end{macrocode}
%    \end{macro}
%
%    \begin{macro}{\HoLogo@(La)TeX}
%    \begin{macrocode}
\expandafter
\let\csname HoLogo@(La)TeX\endcsname\HoLogo@LaTeXTeX
%    \end{macrocode}
%    \end{macro}
%    \begin{macro}{\HoLogoBkm@(La)TeX}
%    \begin{macrocode}
\expandafter
\let\csname HoLogoBkm@(La)TeX\endcsname\HoLogoBkm@LaTeXTeX
%    \end{macrocode}
%    \end{macro}
%    \begin{macro}{\HoLogoHtml@LaTeXTeX}
%    \begin{macrocode}
\def\HoLogoHtml@LaTeXTeX#1{%
  \HoLogoCss@LaTeXTeX
  \HOLOGO@Span{LaTeXTeX}{%
    (%
    \HOLOGO@Span{L}{L}%
    \HOLOGO@Span{a}{A}%
    \HOLOGO@Span{ParenRight}{)}%
    \hologo{TeX}%
  }%
}
%    \end{macrocode}
%    \end{macro}
%    \begin{macro}{\HoLogoHtml@(La)TeX}
%    Kerning after opening parentheses and before closing parentheses
%    is $-0.1$\,em. The original values $-0.15$\,em
%    looked too ugly for a serif font.
%    \begin{macrocode}
\expandafter
\let\csname HoLogoHtml@(La)TeX\endcsname\HoLogoHtml@LaTeXTeX
%    \end{macrocode}
%    \end{macro}
%    \begin{macro}{\HoLogoCss@LaTeXTeX}
%    \begin{macrocode}
\def\HoLogoCss@LaTeXTeX{%
  \Css{%
    span.HoLogo-LaTeXTeX span.HoLogo-L{%
      margin-left:-.1em;%
    }%
  }%
  \Css{%
    span.HoLogo-LaTeXTeX span.HoLogo-a{%
      position:relative;%
      top:-.5ex;%
      margin-left:-.36em;%
      margin-right:-.1em;%
      font-size:85\%;%
    }%
  }%
  \Css{%
    span.HoLogo-LaTeXTeX span.HoLogo-ParenRight{%
      margin-right:-.15em;%
    }%
  }%
  \global\let\HoLogoCss@LaTeXTeX\relax
}
%    \end{macrocode}
%    \end{macro}
%
% \subsubsection{\hologo{LaTeXe}}
%
%    \begin{macro}{\HoLogo@LaTeXe}
%    Source: \hologo{LaTeX} kernel
%    \begin{macrocode}
\def\HoLogo@LaTeXe#1{%
  \hologo{LaTeX}%
  \kern.15em%
  \hbox{%
    \HOLOGO@MathSetup
    2%
    $_{\textstyle\varepsilon}$%
  }%
}
%    \end{macrocode}
%    \end{macro}
%
%    \begin{macro}{\HoLogoCs@LaTeXe}
%    \begin{macrocode}
\ifnum64=`\^^^^0040\relax % test for big chars of LuaTeX/XeTeX
  \catcode`\$=9 %
  \catcode`\&=14 %
\else
  \catcode`\$=14 %
  \catcode`\&=9 %
\fi
\def\HoLogoCs@LaTeXe#1{%
  LaTeX2%
$ \string ^^^^0395%
& e%
}%
\catcode`\$=3 %
\catcode`\&=4 %
%    \end{macrocode}
%    \end{macro}
%
%    \begin{macro}{\HoLogoBkm@LaTeXe}
%    \begin{macrocode}
\def\HoLogoBkm@LaTeXe#1{%
  \hologo{LaTeX}%
  2%
  \HOLOGO@PdfdocUnicode{e}{\textepsilon}%
}
%    \end{macrocode}
%    \end{macro}
%
%    \begin{macro}{\HoLogoHtml@LaTeXe}
%    \begin{macrocode}
\def\HoLogoHtml@LaTeXe#1{%
  \HoLogoCss@LaTeXe
  \HOLOGO@Span{LaTeX2e}{%
    \hologo{LaTeX}%
    \HOLOGO@Span{2}{2}%
    \HOLOGO@Span{e}{%
      \HOLOGO@MathSetup
      \ensuremath{\textstyle\varepsilon}%
    }%
  }%
}
%    \end{macrocode}
%    \end{macro}
%    \begin{macro}{\HoLogoCss@LaTeXe}
%    \begin{macrocode}
\def\HoLogoCss@LaTeXe{%
  \Css{%
    span.HoLogo-LaTeX2e span.HoLogo-2{%
      padding-left:.15em;%
    }%
  }%
  \Css{%
    span.HoLogo-LaTeX2e span.HoLogo-e{%
      position:relative;%
      top:.35ex;%
      text-decoration:none;%
    }%
  }%
  \global\let\HoLogoCss@LaTeXe\relax
}
%    \end{macrocode}
%    \end{macro}
%
%    \begin{macro}{\HoLogo@LaTeX2e}
%    \begin{macrocode}
\expandafter
\let\csname HoLogo@LaTeX2e\endcsname\HoLogo@LaTeXe
%    \end{macrocode}
%    \end{macro}
%    \begin{macro}{\HoLogoCs@LaTeX2e}
%    \begin{macrocode}
\expandafter
\let\csname HoLogoCs@LaTeX2e\endcsname\HoLogoCs@LaTeXe
%    \end{macrocode}
%    \end{macro}
%    \begin{macro}{\HoLogoBkm@LaTeX2e}
%    \begin{macrocode}
\expandafter
\let\csname HoLogoBkm@LaTeX2e\endcsname\HoLogoBkm@LaTeXe
%    \end{macrocode}
%    \end{macro}
%    \begin{macro}{\HoLogoHtml@LaTeX2e}
%    \begin{macrocode}
\expandafter
\let\csname HoLogoHtml@LaTeX2e\endcsname\HoLogoHtml@LaTeXe
%    \end{macrocode}
%    \end{macro}
%
% \subsubsection{\hologo{LaTeX3}}
%
%    \begin{macro}{\HoLogo@LaTeX3}
%    Source: \hologo{LaTeX} kernel
%    \begin{macrocode}
\expandafter\def\csname HoLogo@LaTeX3\endcsname#1{%
  \hologo{LaTeX}%
  3%
}
%    \end{macrocode}
%    \end{macro}
%
%    \begin{macro}{\HoLogoBkm@LaTeX3}
%    \begin{macrocode}
\expandafter\def\csname HoLogoBkm@LaTeX3\endcsname#1{%
  \hologo{LaTeX}%
  3%
}
%    \end{macrocode}
%    \end{macro}
%    \begin{macro}{\HoLogoHtml@LaTeX3}
%    \begin{macrocode}
\expandafter
\let\csname HoLogoHtml@LaTeX3\expandafter\endcsname
\csname HoLogo@LaTeX3\endcsname
%    \end{macrocode}
%    \end{macro}
%
% \subsubsection{\hologo{LaTeXML}}
%
%    \begin{macro}{\HoLogo@LaTeXML}
%    \begin{macrocode}
\def\HoLogo@LaTeXML#1{%
  \HOLOGO@mbox{%
    \hologo{La}%
    \kern-.15em%
    T%
    \kern-.1667em%
    \lower.5ex\hbox{E}%
    \kern-.125em%
    \HoLogoFont@font{LaTeXML}{sc}{xml}%
  }%
}
%    \end{macrocode}
%    \end{macro}
%    \begin{macro}{\HoLogoHtml@pdfLaTeX}
%    \begin{macrocode}
\def\HoLogoHtml@LaTeXML#1{%
  \HOLOGO@Span{LaTeXML}{%
    \HoLogoCss@LaTeX
    \HoLogoCss@TeX
    \HOLOGO@Span{LaTeX}{%
      L%
      \HOLOGO@Span{a}{%
        A%
      }%
    }%
    \HOLOGO@Span{TeX}{%
      T%
      \HOLOGO@Span{e}{%
        E%
      }%
    }%
    \HCode{<span style="font-variant: small-caps;">}%
    xml%
    \HCode{</span>}%
  }%
}
%    \end{macrocode}
%    \end{macro}
%
% \subsubsection{\hologo{eTeX}}
%
%    \begin{macro}{\HoLogo@eTeX}
%    Source: package \xpackage{etex}
%    \begin{macrocode}
\def\HoLogo@eTeX#1{%
  \ltx@mbox{%
    \HOLOGO@MathSetup
    $\varepsilon$%
    -%
    \HOLOGO@NegativeKerning{-T,T-,To}%
    \hologo{TeX}%
  }%
}
%    \end{macrocode}
%    \end{macro}
%    \begin{macro}{\HoLogoCs@eTeX}
%    \begin{macrocode}
\ifnum64=`\^^^^0040\relax % test for big chars of LuaTeX/XeTeX
  \catcode`\$=9 %
  \catcode`\&=14 %
\else
  \catcode`\$=14 %
  \catcode`\&=9 %
\fi
\def\HoLogoCs@eTeX#1{%
$ #1{\string ^^^^0395}{\string ^^^^03b5}%
& #1{e}{E}%
  TeX%
}%
\catcode`\$=3 %
\catcode`\&=4 %
%    \end{macrocode}
%    \end{macro}
%    \begin{macro}{\HoLogoBkm@eTeX}
%    \begin{macrocode}
\def\HoLogoBkm@eTeX#1{%
  \HOLOGO@PdfdocUnicode{#1{e}{E}}{\textepsilon}%
  -%
  \hologo{TeX}%
}
%    \end{macrocode}
%    \end{macro}
%    \begin{macro}{\HoLogoHtml@eTeX}
%    \begin{macrocode}
\def\HoLogoHtml@eTeX#1{%
  \ltx@mbox{%
    \HOLOGO@MathSetup
    $\varepsilon$%
    -%
    \hologo{TeX}%
  }%
}
%    \end{macrocode}
%    \end{macro}
%
% \subsubsection{\hologo{iniTeX}}
%
%    \begin{macro}{\HoLogo@iniTeX}
%    \begin{macrocode}
\def\HoLogo@iniTeX#1{%
  \HOLOGO@mbox{%
    #1{i}{I}ni\hologo{TeX}%
  }%
}
%    \end{macrocode}
%    \end{macro}
%    \begin{macro}{\HoLogoCs@iniTeX}
%    \begin{macrocode}
\def\HoLogoCs@iniTeX#1{#1{i}{I}niTeX}
%    \end{macrocode}
%    \end{macro}
%    \begin{macro}{\HoLogoBkm@iniTeX}
%    \begin{macrocode}
\def\HoLogoBkm@iniTeX#1{%
  #1{i}{I}ni\hologo{TeX}%
}
%    \end{macrocode}
%    \end{macro}
%    \begin{macro}{\HoLogoHtml@iniTeX}
%    \begin{macrocode}
\let\HoLogoHtml@iniTeX\HoLogo@iniTeX
%    \end{macrocode}
%    \end{macro}
%
% \subsubsection{\hologo{virTeX}}
%
%    \begin{macro}{\HoLogo@virTeX}
%    \begin{macrocode}
\def\HoLogo@virTeX#1{%
  \HOLOGO@mbox{%
    #1{v}{V}ir\hologo{TeX}%
  }%
}
%    \end{macrocode}
%    \end{macro}
%    \begin{macro}{\HoLogoCs@virTeX}
%    \begin{macrocode}
\def\HoLogoCs@virTeX#1{#1{v}{V}irTeX}
%    \end{macrocode}
%    \end{macro}
%    \begin{macro}{\HoLogoBkm@virTeX}
%    \begin{macrocode}
\def\HoLogoBkm@virTeX#1{%
  #1{v}{V}ir\hologo{TeX}%
}
%    \end{macrocode}
%    \end{macro}
%    \begin{macro}{\HoLogoHtml@virTeX}
%    \begin{macrocode}
\let\HoLogoHtml@virTeX\HoLogo@virTeX
%    \end{macrocode}
%    \end{macro}
%
% \subsubsection{\hologo{SliTeX}}
%
% \paragraph{Definitions of the three variants.}
%
%    \begin{macro}{\HoLogo@SLiTeX@lift}
%    \begin{macrocode}
\def\HoLogo@SLiTeX@lift#1{%
  \HoLogoFont@font{SliTeX}{rm}{%
    S%
    \kern-.06em%
    L%
    \kern-.18em%
    \raise.32ex\hbox{\HoLogoFont@font{SliTeX}{sc}{i}}%
    \HOLOGO@discretionary
    \kern-.06em%
    \hologo{TeX}%
  }%
}
%    \end{macrocode}
%    \end{macro}
%    \begin{macro}{\HoLogoBkm@SLiTeX@lift}
%    \begin{macrocode}
\def\HoLogoBkm@SLiTeX@lift#1{SLiTeX}
%    \end{macrocode}
%    \end{macro}
%    \begin{macro}{\HoLogoHtml@SLiTeX@lift}
%    \begin{macrocode}
\def\HoLogoHtml@SLiTeX@lift#1{%
  \HoLogoCss@SLiTeX@lift
  \HOLOGO@Span{SLiTeX-lift}{%
    \HoLogoFont@font{SliTeX}{rm}{%
      S%
      \HOLOGO@Span{L}{L}%
      \HOLOGO@Span{i}{i}%
      \hologo{TeX}%
    }%
  }%
}
%    \end{macrocode}
%    \end{macro}
%    \begin{macro}{\HoLogoCss@SLiTeX@lift}
%    \begin{macrocode}
\def\HoLogoCss@SLiTeX@lift{%
  \Css{%
    span.HoLogo-SLiTeX-lift span.HoLogo-L{%
      margin-left:-.06em;%
      margin-right:-.18em;%
    }%
  }%
  \Css{%
    span.HoLogo-SLiTeX-lift span.HoLogo-i{%
      position:relative;%
      top:-.32ex;%
      margin-right:-.06em;%
      font-variant:small-caps;%
    }%
  }%
  \global\let\HoLogoCss@SLiTeX@lift\relax
}
%    \end{macrocode}
%    \end{macro}
%
%    \begin{macro}{\HoLogo@SliTeX@simple}
%    \begin{macrocode}
\def\HoLogo@SliTeX@simple#1{%
  \HoLogoFont@font{SliTeX}{rm}{%
    \ltx@mbox{%
      \HoLogoFont@font{SliTeX}{sc}{Sli}%
    }%
    \HOLOGO@discretionary
    \hologo{TeX}%
  }%
}
%    \end{macrocode}
%    \end{macro}
%    \begin{macro}{\HoLogoBkm@SliTeX@simple}
%    \begin{macrocode}
\def\HoLogoBkm@SliTeX@simple#1{SliTeX}
%    \end{macrocode}
%    \end{macro}
%    \begin{macro}{\HoLogoHtml@SliTeX@simple}
%    \begin{macrocode}
\let\HoLogoHtml@SliTeX@simple\HoLogo@SliTeX@simple
%    \end{macrocode}
%    \end{macro}
%
%    \begin{macro}{\HoLogo@SliTeX@narrow}
%    \begin{macrocode}
\def\HoLogo@SliTeX@narrow#1{%
  \HoLogoFont@font{SliTeX}{rm}{%
    \ltx@mbox{%
      S%
      \kern-.06em%
      \HoLogoFont@font{SliTeX}{sc}{%
        l%
        \kern-.035em%
        i%
      }%
    }%
    \HOLOGO@discretionary
    \kern-.06em%
    \hologo{TeX}%
  }%
}
%    \end{macrocode}
%    \end{macro}
%    \begin{macro}{\HoLogoBkm@SliTeX@narrow}
%    \begin{macrocode}
\def\HoLogoBkm@SliTeX@narrow#1{SliTeX}
%    \end{macrocode}
%    \end{macro}
%    \begin{macro}{\HoLogoHtml@SliTeX@narrow}
%    \begin{macrocode}
\def\HoLogoHtml@SliTeX@narrow#1{%
  \HoLogoCss@SliTeX@narrow
  \HOLOGO@Span{SliTeX-narrow}{%
    \HoLogoFont@font{SliTeX}{rm}{%
      S%
        \HOLOGO@Span{l}{l}%
        \HOLOGO@Span{i}{i}%
      \hologo{TeX}%
    }%
  }%
}
%    \end{macrocode}
%    \end{macro}
%    \begin{macro}{\HoLogoCss@SliTeX@narrow}
%    \begin{macrocode}
\def\HoLogoCss@SliTeX@narrow{%
  \Css{%
    span.HoLogo-SliTeX-narrow span.HoLogo-l{%
      margin-left:-.06em;%
      margin-right:-.035em;%
      font-variant:small-caps;%
    }%
  }%
  \Css{%
    span.HoLogo-SliTeX-narrow span.HoLogo-i{%
      margin-right:-.06em;%
      font-variant:small-caps;%
    }%
  }%
  \global\let\HoLogoCss@SliTeX@narrow\relax
}
%    \end{macrocode}
%    \end{macro}
%
% \paragraph{Macro set completion.}
%
%    \begin{macro}{\HoLogo@SLiTeX@simple}
%    \begin{macrocode}
\def\HoLogo@SLiTeX@simple{\HoLogo@SliTeX@simple}
%    \end{macrocode}
%    \end{macro}
%    \begin{macro}{\HoLogoBkm@SLiTeX@simple}
%    \begin{macrocode}
\def\HoLogoBkm@SLiTeX@simple{\HoLogoBkm@SliTeX@simple}
%    \end{macrocode}
%    \end{macro}
%    \begin{macro}{\HoLogoHtml@SLiTeX@simple}
%    \begin{macrocode}
\def\HoLogoHtml@SLiTeX@simple{\HoLogoHtml@SliTeX@simple}
%    \end{macrocode}
%    \end{macro}
%
%    \begin{macro}{\HoLogo@SLiTeX@narrow}
%    \begin{macrocode}
\def\HoLogo@SLiTeX@narrow{\HoLogo@SliTeX@narrow}
%    \end{macrocode}
%    \end{macro}
%    \begin{macro}{\HoLogoBkm@SLiTeX@narrow}
%    \begin{macrocode}
\def\HoLogoBkm@SLiTeX@narrow{\HoLogoBkm@SliTeX@narrow}
%    \end{macrocode}
%    \end{macro}
%    \begin{macro}{\HoLogoHtml@SLiTeX@narrow}
%    \begin{macrocode}
\def\HoLogoHtml@SLiTeX@narrow{\HoLogoHtml@SliTeX@narrow}
%    \end{macrocode}
%    \end{macro}
%
%    \begin{macro}{\HoLogo@SliTeX@lift}
%    \begin{macrocode}
\def\HoLogo@SliTeX@lift{\HoLogo@SLiTeX@lift}
%    \end{macrocode}
%    \end{macro}
%    \begin{macro}{\HoLogoBkm@SliTeX@lift}
%    \begin{macrocode}
\def\HoLogoBkm@SliTeX@lift{\HoLogoBkm@SLiTeX@lift}
%    \end{macrocode}
%    \end{macro}
%    \begin{macro}{\HoLogoHtml@SliTeX@lift}
%    \begin{macrocode}
\def\HoLogoHtml@SliTeX@lift{\HoLogoHtml@SLiTeX@lift}
%    \end{macrocode}
%    \end{macro}
%
% \paragraph{Defaults.}
%
%    \begin{macro}{\HoLogo@SLiTeX}
%    \begin{macrocode}
\def\HoLogo@SLiTeX{\HoLogo@SLiTeX@lift}
%    \end{macrocode}
%    \end{macro}
%    \begin{macro}{\HoLogoBkm@SLiTeX}
%    \begin{macrocode}
\def\HoLogoBkm@SLiTeX{\HoLogoBkm@SLiTeX@lift}
%    \end{macrocode}
%    \end{macro}
%    \begin{macro}{\HoLogoHtml@SLiTeX}
%    \begin{macrocode}
\def\HoLogoHtml@SLiTeX{\HoLogoHtml@SLiTeX@lift}
%    \end{macrocode}
%    \end{macro}
%
%    \begin{macro}{\HoLogo@SliTeX}
%    \begin{macrocode}
\def\HoLogo@SliTeX{\HoLogo@SliTeX@narrow}
%    \end{macrocode}
%    \end{macro}
%    \begin{macro}{\HoLogoBkm@SliTeX}
%    \begin{macrocode}
\def\HoLogoBkm@SliTeX{\HoLogoBkm@SliTeX@narrow}
%    \end{macrocode}
%    \end{macro}
%    \begin{macro}{\HoLogoHtml@SliTeX}
%    \begin{macrocode}
\def\HoLogoHtml@SliTeX{\HoLogoHtml@SliTeX@narrow}
%    \end{macrocode}
%    \end{macro}
%
% \subsubsection{\hologo{LuaTeX}}
%
%    \begin{macro}{\HoLogo@LuaTeX}
%    The kerning is an idea of Hans Hagen, see mailing list
%    `luatex at tug dot org' in March 2010.
%    \begin{macrocode}
\def\HoLogo@LuaTeX#1{%
  \HOLOGO@mbox{%
    Lua%
    \HOLOGO@NegativeKerning{aT,oT,To}%
    \hologo{TeX}%
  }%
}
%    \end{macrocode}
%    \end{macro}
%    \begin{macro}{\HoLogoHtml@LuaTeX}
%    \begin{macrocode}
\let\HoLogoHtml@LuaTeX\HoLogo@LuaTeX
%    \end{macrocode}
%    \end{macro}
%
% \subsubsection{\hologo{LuaLaTeX}}
%
%    \begin{macro}{\HoLogo@LuaLaTeX}
%    \begin{macrocode}
\def\HoLogo@LuaLaTeX#1{%
  \HOLOGO@mbox{%
    Lua%
    \hologo{LaTeX}%
  }%
}
%    \end{macrocode}
%    \end{macro}
%    \begin{macro}{\HoLogoHtml@LuaLaTeX}
%    \begin{macrocode}
\let\HoLogoHtml@LuaLaTeX\HoLogo@LuaLaTeX
%    \end{macrocode}
%    \end{macro}
%
% \subsubsection{\hologo{XeTeX}, \hologo{XeLaTeX}}
%
%    \begin{macro}{\HOLOGO@IfCharExists}
%    \begin{macrocode}
\ifluatex
  \ifnum\luatexversion<36 %
  \else
    \def\HOLOGO@IfCharExists#1{%
      \ifnum
        \directlua{%
           if luaotfload and luaotfload.aux then
             if luaotfload.aux.font_has_glyph(%
                    font.current(), \number#1) then % 	 
	       tex.print("1") % 	 
	     end % 	 
	   elseif font and font.fonts and font.current then %
            local f = font.fonts[font.current()]%
            if f.characters and f.characters[\number#1] then %
              tex.print("1")%
            end %
          end%
        }0=\ltx@zero
        \expandafter\ltx@secondoftwo
      \else
        \expandafter\ltx@firstoftwo
      \fi
    }%
  \fi
\fi
\ltx@IfUndefined{HOLOGO@IfCharExists}{%
  \def\HOLOGO@@IfCharExists#1{%
    \begingroup
      \tracinglostchars=\ltx@zero
      \setbox\ltx@zero=\hbox{%
        \kern7sp\char#1\relax
        \ifnum\lastkern>\ltx@zero
          \expandafter\aftergroup\csname iffalse\endcsname
        \else
          \expandafter\aftergroup\csname iftrue\endcsname
        \fi
      }%
      % \if{true|false} from \aftergroup
      \endgroup
      \expandafter\ltx@firstoftwo
    \else
      \endgroup
      \expandafter\ltx@secondoftwo
    \fi
  }%
  \ifxetex
    \ltx@IfUndefined{XeTeXfonttype}{}{%
      \ltx@IfUndefined{XeTeXcharglyph}{}{%
        \def\HOLOGO@IfCharExists#1{%
          \ifnum\XeTeXfonttype\font>\ltx@zero
            \expandafter\ltx@firstofthree
          \else
            \expandafter\ltx@gobble
          \fi
          {%
            \ifnum\XeTeXcharglyph#1>\ltx@zero
              \expandafter\ltx@firstoftwo
            \else
              \expandafter\ltx@secondoftwo
            \fi
          }%
          \HOLOGO@@IfCharExists{#1}%
        }%
      }%
    }%
  \fi
}{}
\ltx@ifundefined{HOLOGO@IfCharExists}{%
  \ifnum64=`\^^^^0040\relax % test for big chars of LuaTeX/XeTeX
    \let\HOLOGO@IfCharExists\HOLOGO@@IfCharExists
  \else
    \def\HOLOGO@IfCharExists#1{%
      \ifnum#1>255 %
        \expandafter\ltx@fourthoffour
      \fi
      \HOLOGO@@IfCharExists{#1}%
    }%
  \fi
}{}
%    \end{macrocode}
%    \end{macro}
%
%    \begin{macro}{\HoLogo@Xe}
%    Source: package \xpackage{dtklogos}
%    \begin{macrocode}
\def\HoLogo@Xe#1{%
  X%
  \kern-.1em\relax
  \HOLOGO@IfCharExists{"018E}{%
    \lower.5ex\hbox{\char"018E}%
  }{%
    \chardef\HOLOGO@choice=\ltx@zero
    \ifdim\fontdimen\ltx@one\font>0pt %
      \ltx@IfUndefined{rotatebox}{%
        \ltx@IfUndefined{pgftext}{%
          \ltx@IfUndefined{psscalebox}{%
            \ltx@IfUndefined{HOLOGO@ScaleBox@\hologoDriver}{%
            }{%
              \chardef\HOLOGO@choice=4 %
            }%
          }{%
            \chardef\HOLOGO@choice=3 %
          }%
        }{%
          \chardef\HOLOGO@choice=2 %
        }%
      }{%
        \chardef\HOLOGO@choice=1 %
      }%
      \ifcase\HOLOGO@choice
        \HOLOGO@WarningUnsupportedDriver{Xe}%
        e%
      \or % 1: \rotatebox
        \begingroup
          \setbox\ltx@zero\hbox{\rotatebox{180}{E}}%
          \ltx@LocDimenA=\dp\ltx@zero
          \advance\ltx@LocDimenA by -.5ex\relax
          \raise\ltx@LocDimenA\box\ltx@zero
        \endgroup
      \or % 2: \pgftext
        \lower.5ex\hbox{%
          \pgfpicture
            \pgftext[rotate=180]{E}%
          \endpgfpicture
        }%
      \or % 3: \psscalebox
        \begingroup
          \setbox\ltx@zero\hbox{\psscalebox{-1 -1}{E}}%
          \ltx@LocDimenA=\dp\ltx@zero
          \advance\ltx@LocDimenA by -.5ex\relax
          \raise\ltx@LocDimenA\box\ltx@zero
        \endgroup
      \or % 4: \HOLOGO@PointReflectBox
        \lower.5ex\hbox{\HOLOGO@PointReflectBox{E}}%
      \else
        \@PackageError{hologo}{Internal error (choice/it}\@ehc
      \fi
    \else
      \ltx@IfUndefined{reflectbox}{%
        \ltx@IfUndefined{pgftext}{%
          \ltx@IfUndefined{psscalebox}{%
            \ltx@IfUndefined{HOLOGO@ScaleBox@\hologoDriver}{%
            }{%
              \chardef\HOLOGO@choice=4 %
            }%
          }{%
            \chardef\HOLOGO@choice=3 %
          }%
        }{%
          \chardef\HOLOGO@choice=2 %
        }%
      }{%
        \chardef\HOLOGO@choice=1 %
      }%
      \ifcase\HOLOGO@choice
        \HOLOGO@WarningUnsupportedDriver{Xe}%
        e%
      \or % 1: reflectbox
        \lower.5ex\hbox{%
          \reflectbox{E}%
        }%
      \or % 2: \pgftext
        \lower.5ex\hbox{%
          \pgfpicture
            \pgftransformxscale{-1}%
            \pgftext{E}%
          \endpgfpicture
        }%
      \or % 3: \psscalebox
        \lower.5ex\hbox{%
          \psscalebox{-1 1}{E}%
        }%
      \or % 4: \HOLOGO@Reflectbox
        \lower.5ex\hbox{%
          \HOLOGO@ReflectBox{E}%
        }%
      \else
        \@PackageError{hologo}{Internal error (choice/up)}\@ehc
      \fi
    \fi
  }%
}
%    \end{macrocode}
%    \end{macro}
%    \begin{macro}{\HoLogoHtml@Xe}
%    \begin{macrocode}
\def\HoLogoHtml@Xe#1{%
  \HoLogoCss@Xe
  \HOLOGO@Span{Xe}{%
    X%
    \HOLOGO@Span{e}{%
      \HCode{&\ltx@hashchar x018e;}%
    }%
  }%
}
%    \end{macrocode}
%    \end{macro}
%    \begin{macro}{\HoLogoCss@Xe}
%    \begin{macrocode}
\def\HoLogoCss@Xe{%
  \Css{%
    span.HoLogo-Xe span.HoLogo-e{%
      position:relative;%
      top:.5ex;%
      left-margin:-.1em;%
    }%
  }%
  \global\let\HoLogoCss@Xe\relax
}
%    \end{macrocode}
%    \end{macro}
%
%    \begin{macro}{\HoLogo@XeTeX}
%    \begin{macrocode}
\def\HoLogo@XeTeX#1{%
  \hologo{Xe}%
  \kern-.15em\relax
  \hologo{TeX}%
}
%    \end{macrocode}
%    \end{macro}
%
%    \begin{macro}{\HoLogoHtml@XeTeX}
%    \begin{macrocode}
\def\HoLogoHtml@XeTeX#1{%
  \HoLogoCss@XeTeX
  \HOLOGO@Span{XeTeX}{%
    \hologo{Xe}%
    \hologo{TeX}%
  }%
}
%    \end{macrocode}
%    \end{macro}
%    \begin{macro}{\HoLogoCss@XeTeX}
%    \begin{macrocode}
\def\HoLogoCss@XeTeX{%
  \Css{%
    span.HoLogo-XeTeX span.HoLogo-TeX{%
      margin-left:-.15em;%
    }%
  }%
  \global\let\HoLogoCss@XeTeX\relax
}
%    \end{macrocode}
%    \end{macro}
%
%    \begin{macro}{\HoLogo@XeLaTeX}
%    \begin{macrocode}
\def\HoLogo@XeLaTeX#1{%
  \hologo{Xe}%
  \kern-.13em%
  \hologo{LaTeX}%
}
%    \end{macrocode}
%    \end{macro}
%    \begin{macro}{\HoLogoHtml@XeLaTeX}
%    \begin{macrocode}
\def\HoLogoHtml@XeLaTeX#1{%
  \HoLogoCss@XeLaTeX
  \HOLOGO@Span{XeLaTeX}{%
    \hologo{Xe}%
    \hologo{LaTeX}%
  }%
}
%    \end{macrocode}
%    \end{macro}
%    \begin{macro}{\HoLogoCss@XeLaTeX}
%    \begin{macrocode}
\def\HoLogoCss@XeLaTeX{%
  \Css{%
    span.HoLogo-XeLaTeX span.HoLogo-Xe{%
      margin-right:-.13em;%
    }%
  }%
  \global\let\HoLogoCss@XeLaTeX\relax
}
%    \end{macrocode}
%    \end{macro}
%
% \subsubsection{\hologo{pdfTeX}, \hologo{pdfLaTeX}}
%
%    \begin{macro}{\HoLogo@pdfTeX}
%    \begin{macrocode}
\def\HoLogo@pdfTeX#1{%
  \HOLOGO@mbox{%
    #1{p}{P}df\hologo{TeX}%
  }%
}
%    \end{macrocode}
%    \end{macro}
%    \begin{macro}{\HoLogoCs@pdfTeX}
%    \begin{macrocode}
\def\HoLogoCs@pdfTeX#1{#1{p}{P}dfTeX}
%    \end{macrocode}
%    \end{macro}
%    \begin{macro}{\HoLogoBkm@pdfTeX}
%    \begin{macrocode}
\def\HoLogoBkm@pdfTeX#1{%
  #1{p}{P}df\hologo{TeX}%
}
%    \end{macrocode}
%    \end{macro}
%    \begin{macro}{\HoLogoHtml@pdfTeX}
%    \begin{macrocode}
\let\HoLogoHtml@pdfTeX\HoLogo@pdfTeX
%    \end{macrocode}
%    \end{macro}
%
%    \begin{macro}{\HoLogo@pdfLaTeX}
%    \begin{macrocode}
\def\HoLogo@pdfLaTeX#1{%
  \HOLOGO@mbox{%
    #1{p}{P}df\hologo{LaTeX}%
  }%
}
%    \end{macrocode}
%    \end{macro}
%    \begin{macro}{\HoLogoCs@pdfLaTeX}
%    \begin{macrocode}
\def\HoLogoCs@pdfLaTeX#1{#1{p}{P}dfLaTeX}
%    \end{macrocode}
%    \end{macro}
%    \begin{macro}{\HoLogoBkm@pdfLaTeX}
%    \begin{macrocode}
\def\HoLogoBkm@pdfLaTeX#1{%
  #1{p}{P}df\hologo{LaTeX}%
}
%    \end{macrocode}
%    \end{macro}
%    \begin{macro}{\HoLogoHtml@pdfLaTeX}
%    \begin{macrocode}
\let\HoLogoHtml@pdfLaTeX\HoLogo@pdfLaTeX
%    \end{macrocode}
%    \end{macro}
%
% \subsubsection{\hologo{VTeX}}
%
%    \begin{macro}{\HoLogo@VTeX}
%    \begin{macrocode}
\def\HoLogo@VTeX#1{%
  \HOLOGO@mbox{%
    V\hologo{TeX}%
  }%
}
%    \end{macrocode}
%    \end{macro}
%    \begin{macro}{\HoLogoHtml@VTeX}
%    \begin{macrocode}
\let\HoLogoHtml@VTeX\HoLogo@VTeX
%    \end{macrocode}
%    \end{macro}
%
% \subsubsection{\hologo{AmS}, \dots}
%
%    Source: class \xclass{amsdtx}
%
%    \begin{macro}{\HoLogo@AmS}
%    \begin{macrocode}
\def\HoLogo@AmS#1{%
  \HoLogoFont@font{AmS}{sy}{%
    A%
    \kern-.1667em%
    \lower.5ex\hbox{M}%
    \kern-.125em%
    S%
  }%
}
%    \end{macrocode}
%    \end{macro}
%    \begin{macro}{\HoLogoBkm@AmS}
%    \begin{macrocode}
\def\HoLogoBkm@AmS#1{AmS}
%    \end{macrocode}
%    \end{macro}
%    \begin{macro}{\HoLogoHtml@AmS}
%    \begin{macrocode}
\def\HoLogoHtml@AmS#1{%
  \HoLogoCss@AmS
%  \HoLogoFont@font{AmS}{sy}{%
    \HOLOGO@Span{AmS}{%
      A%
      \HOLOGO@Span{M}{M}%
      S%
    }%
%   }%
}
%    \end{macrocode}
%    \end{macro}
%    \begin{macro}{\HoLogoCss@AmS}
%    \begin{macrocode}
\def\HoLogoCss@AmS{%
  \Css{%
    span.HoLogo-AmS span.HoLogo-M{%
      position:relative;%
      top:.5ex;%
      margin-left:-.1667em;%
      margin-right:-.125em;%
      text-decoration:none;%
    }%
  }%
  \global\let\HoLogoCss@AmS\relax
}
%    \end{macrocode}
%    \end{macro}
%
%    \begin{macro}{\HoLogo@AmSTeX}
%    \begin{macrocode}
\def\HoLogo@AmSTeX#1{%
  \hologo{AmS}%
  \HOLOGO@hyphen
  \hologo{TeX}%
}
%    \end{macrocode}
%    \end{macro}
%    \begin{macro}{\HoLogoBkm@AmSTeX}
%    \begin{macrocode}
\def\HoLogoBkm@AmSTeX#1{AmS-TeX}%
%    \end{macrocode}
%    \end{macro}
%    \begin{macro}{\HoLogoHtml@AmSTeX}
%    \begin{macrocode}
\let\HoLogoHtml@AmSTeX\HoLogo@AmSTeX
%    \end{macrocode}
%    \end{macro}
%
%    \begin{macro}{\HoLogo@AmSLaTeX}
%    \begin{macrocode}
\def\HoLogo@AmSLaTeX#1{%
  \hologo{AmS}%
  \HOLOGO@hyphen
  \hologo{LaTeX}%
}
%    \end{macrocode}
%    \end{macro}
%    \begin{macro}{\HoLogoBkm@AmSLaTeX}
%    \begin{macrocode}
\def\HoLogoBkm@AmSLaTeX#1{AmS-LaTeX}%
%    \end{macrocode}
%    \end{macro}
%    \begin{macro}{\HoLogoHtml@AmSLaTeX}
%    \begin{macrocode}
\let\HoLogoHtml@AmSLaTeX\HoLogo@AmSLaTeX
%    \end{macrocode}
%    \end{macro}
%
% \subsubsection{\hologo{BibTeX}}
%
%    \begin{macro}{\HoLogo@BibTeX@sc}
%    A definition of \hologo{BibTeX} is provided in
%    the documentation source for the manual of \hologo{BibTeX}
%    \cite{btxdoc}.
%\begin{quote}
%\begin{verbatim}
%\def\BibTeX{%
%  {%
%    \rm
%    B%
%    \kern-.05em%
%    {%
%      \sc
%      i%
%      \kern-.025em %
%      b%
%    }%
%    \kern-.08em
%    T%
%    \kern-.1667em%
%    \lower.7ex\hbox{E}%
%    \kern-.125em%
%    X%
%  }%
%}
%\end{verbatim}
%\end{quote}
%    \begin{macrocode}
\def\HoLogo@BibTeX@sc#1{%
  B%
  \kern-.05em%
  \HoLogoFont@font{BibTeX}{sc}{%
    i%
    \kern-.025em%
    b%
  }%
  \HOLOGO@discretionary
  \kern-.08em%
  \hologo{TeX}%
}
%    \end{macrocode}
%    \end{macro}
%    \begin{macro}{\HoLogoHtml@BibTeX@sc}
%    \begin{macrocode}
\def\HoLogoHtml@BibTeX@sc#1{%
  \HoLogoCss@BibTeX@sc
  \HOLOGO@Span{BibTeX-sc}{%
    B%
    \HOLOGO@Span{i}{i}%
    \HOLOGO@Span{b}{b}%
    \hologo{TeX}%
  }%
}
%    \end{macrocode}
%    \end{macro}
%    \begin{macro}{\HoLogoCss@BibTeX@sc}
%    \begin{macrocode}
\def\HoLogoCss@BibTeX@sc{%
  \Css{%
    span.HoLogo-BibTeX-sc span.HoLogo-i{%
      margin-left:-.05em;%
      margin-right:-.025em;%
      font-variant:small-caps;%
    }%
  }%
  \Css{%
    span.HoLogo-BibTeX-sc span.HoLogo-b{%
      margin-right:-.08em;%
      font-variant:small-caps;%
    }%
  }%
  \global\let\HoLogoCss@BibTeX@sc\relax
}
%    \end{macrocode}
%    \end{macro}
%
%    \begin{macro}{\HoLogo@BibTeX@sf}
%    Variant \xoption{sf} avoids trouble with unavailable
%    small caps fonts (e.g., bold versions of Computer Modern or
%    Latin Modern). The definition is taken from
%    package \xpackage{dtklogos} \cite{dtklogos}.
%\begin{quote}
%\begin{verbatim}
%\DeclareRobustCommand{\BibTeX}{%
%  B%
%  \kern-.05em%
%  \hbox{%
%    $\m@th$% %% force math size calculations
%    \csname S@\f@size\endcsname
%    \fontsize\sf@size\z@
%    \math@fontsfalse
%    \selectfont
%    I%
%    \kern-.025em%
%    B
%  }%
%  \kern-.08em%
%  \-%
%  \TeX
%}
%\end{verbatim}
%\end{quote}
%    \begin{macrocode}
\def\HoLogo@BibTeX@sf#1{%
  B%
  \kern-.05em%
  \HoLogoFont@font{BibTeX}{bibsf}{%
    I%
    \kern-.025em%
    B%
  }%
  \HOLOGO@discretionary
  \kern-.08em%
  \hologo{TeX}%
}
%    \end{macrocode}
%    \end{macro}
%    \begin{macro}{\HoLogoHtml@BibTeX@sf}
%    \begin{macrocode}
\def\HoLogoHtml@BibTeX@sf#1{%
  \HoLogoCss@BibTeX@sf
  \HOLOGO@Span{BibTeX-sf}{%
    B%
    \HoLogoFont@font{BibTeX}{bibsf}{%
      \HOLOGO@Span{i}{I}%
      B%
    }%
    \hologo{TeX}%
  }%
}
%    \end{macrocode}
%    \end{macro}
%    \begin{macro}{\HoLogoCss@BibTeX@sf}
%    \begin{macrocode}
\def\HoLogoCss@BibTeX@sf{%
  \Css{%
    span.HoLogo-BibTeX-sf span.HoLogo-i{%
      margin-left:-.05em;%
      margin-right:-.025em;%
    }%
  }%
  \Css{%
    span.HoLogo-BibTeX-sf span.HoLogo-TeX{%
      margin-left:-.08em;%
    }%
  }%
  \global\let\HoLogoCss@BibTeX@sf\relax
}
%    \end{macrocode}
%    \end{macro}
%
%    \begin{macro}{\HoLogo@BibTeX}
%    \begin{macrocode}
\def\HoLogo@BibTeX{\HoLogo@BibTeX@sf}
%    \end{macrocode}
%    \end{macro}
%    \begin{macro}{\HoLogoHtml@BibTeX}
%    \begin{macrocode}
\def\HoLogoHtml@BibTeX{\HoLogoHtml@BibTeX@sf}
%    \end{macrocode}
%    \end{macro}
%
% \subsubsection{\hologo{BibTeX8}}
%
%    \begin{macro}{\HoLogo@BibTeX8}
%    \begin{macrocode}
\expandafter\def\csname HoLogo@BibTeX8\endcsname#1{%
  \hologo{BibTeX}%
  8%
}
%    \end{macrocode}
%    \end{macro}
%
%    \begin{macro}{\HoLogoBkm@BibTeX8}
%    \begin{macrocode}
\expandafter\def\csname HoLogoBkm@BibTeX8\endcsname#1{%
  \hologo{BibTeX}%
  8%
}
%    \end{macrocode}
%    \end{macro}
%    \begin{macro}{\HoLogoHtml@BibTeX8}
%    \begin{macrocode}
\expandafter
\let\csname HoLogoHtml@BibTeX8\expandafter\endcsname
\csname HoLogo@BibTeX8\endcsname
%    \end{macrocode}
%    \end{macro}
%
% \subsubsection{\hologo{ConTeXt}}
%
%    \begin{macro}{\HoLogo@ConTeXt@simple}
%    \begin{macrocode}
\def\HoLogo@ConTeXt@simple#1{%
  \HOLOGO@mbox{Con}%
  \HOLOGO@discretionary
  \HOLOGO@mbox{\hologo{TeX}t}%
}
%    \end{macrocode}
%    \end{macro}
%    \begin{macro}{\HoLogoHtml@ConTeXt@simple}
%    \begin{macrocode}
\let\HoLogoHtml@ConTeXt@simple\HoLogo@ConTeXt@simple
%    \end{macrocode}
%    \end{macro}
%
%    \begin{macro}{\HoLogo@ConTeXt@narrow}
%    This definition of logo \hologo{ConTeXt} with variant \xoption{narrow}
%    comes from TUGboat's class \xclass{ltugboat} (version 2010/11/15 v2.8).
%    \begin{macrocode}
\def\HoLogo@ConTeXt@narrow#1{%
  \HOLOGO@mbox{C\kern-.0333emon}%
  \HOLOGO@discretionary
  \kern-.0667em%
  \HOLOGO@mbox{\hologo{TeX}\kern-.0333emt}%
}
%    \end{macrocode}
%    \end{macro}
%    \begin{macro}{\HoLogoHtml@ConTeXt@narrow}
%    \begin{macrocode}
\def\HoLogoHtml@ConTeXt@narrow#1{%
  \HoLogoCss@ConTeXt@narrow
  \HOLOGO@Span{ConTeXt-narrow}{%
    \HOLOGO@Span{C}{C}%
    on%
    \hologo{TeX}%
    t%
  }%
}
%    \end{macrocode}
%    \end{macro}
%    \begin{macro}{\HoLogoCss@ConTeXt@narrow}
%    \begin{macrocode}
\def\HoLogoCss@ConTeXt@narrow{%
  \Css{%
    span.HoLogo-ConTeXt-narrow span.HoLogo-C{%
      margin-left:-.0333em;%
    }%
  }%
  \Css{%
    span.HoLogo-ConTeXt-narrow span.HoLogo-TeX{%
      margin-left:-.0667em;%
      margin-right:-.0333em;%
    }%
  }%
  \global\let\HoLogoCss@ConTeXt@narrow\relax
}
%    \end{macrocode}
%    \end{macro}
%
%    \begin{macro}{\HoLogo@ConTeXt}
%    \begin{macrocode}
\def\HoLogo@ConTeXt{\HoLogo@ConTeXt@narrow}
%    \end{macrocode}
%    \end{macro}
%    \begin{macro}{\HoLogoHtml@ConTeXt}
%    \begin{macrocode}
\def\HoLogoHtml@ConTeXt{\HoLogoHtml@ConTeXt@narrow}
%    \end{macrocode}
%    \end{macro}
%
% \subsubsection{\hologo{emTeX}}
%
%    \begin{macro}{\HoLogo@emTeX}
%    \begin{macrocode}
\def\HoLogo@emTeX#1{%
  \HOLOGO@mbox{#1{e}{E}m}%
  \HOLOGO@discretionary
  \hologo{TeX}%
}
%    \end{macrocode}
%    \end{macro}
%    \begin{macro}{\HoLogoCs@emTeX}
%    \begin{macrocode}
\def\HoLogoCs@emTeX#1{#1{e}{E}mTeX}%
%    \end{macrocode}
%    \end{macro}
%    \begin{macro}{\HoLogoBkm@emTeX}
%    \begin{macrocode}
\def\HoLogoBkm@emTeX#1{%
  #1{e}{E}m\hologo{TeX}%
}
%    \end{macrocode}
%    \end{macro}
%    \begin{macro}{\HoLogoHtml@emTeX}
%    \begin{macrocode}
\let\HoLogoHtml@emTeX\HoLogo@emTeX
%    \end{macrocode}
%    \end{macro}
%
% \subsubsection{\hologo{ExTeX}}
%
%    \begin{macro}{\HoLogo@ExTeX}
%    The definition is taken from the FAQ of the
%    project \hologo{ExTeX}
%    \cite{ExTeX-FAQ}.
%\begin{quote}
%\begin{verbatim}
%\def\ExTeX{%
%  \textrm{% Logo always with serifs
%    \ensuremath{%
%      \textstyle
%      \varepsilon_{%
%        \kern-0.15em%
%        \mathcal{X}%
%      }%
%    }%
%    \kern-.15em%
%    \TeX
%  }%
%}
%\end{verbatim}
%\end{quote}
%    \begin{macrocode}
\def\HoLogo@ExTeX#1{%
  \HoLogoFont@font{ExTeX}{rm}{%
    \ltx@mbox{%
      \HOLOGO@MathSetup
      $%
        \textstyle
        \varepsilon_{%
          \kern-0.15em%
          \HoLogoFont@font{ExTeX}{sy}{X}%
        }%
      $%
    }%
    \HOLOGO@discretionary
    \kern-.15em%
    \hologo{TeX}%
  }%
}
%    \end{macrocode}
%    \end{macro}
%    \begin{macro}{\HoLogoHtml@ExTeX}
%    \begin{macrocode}
\def\HoLogoHtml@ExTeX#1{%
  \HoLogoCss@ExTeX
  \HoLogoFont@font{ExTeX}{rm}{%
    \HOLOGO@Span{ExTeX}{%
      \ltx@mbox{%
        \HOLOGO@MathSetup
        $\textstyle\varepsilon$%
        \HOLOGO@Span{X}{$\textstyle\chi$}%
        \hologo{TeX}%
      }%
    }%
  }%
}
%    \end{macrocode}
%    \end{macro}
%    \begin{macro}{\HoLogoBkm@ExTeX}
%    \begin{macrocode}
\def\HoLogoBkm@ExTeX#1{%
  \HOLOGO@PdfdocUnicode{#1{e}{E}x}{\textepsilon\textchi}%
  \hologo{TeX}%
}
%    \end{macrocode}
%    \end{macro}
%    \begin{macro}{\HoLogoCss@ExTeX}
%    \begin{macrocode}
\def\HoLogoCss@ExTeX{%
  \Css{%
    span.HoLogo-ExTeX{%
      font-family:serif;%
    }%
  }%
  \Css{%
    span.HoLogo-ExTeX span.HoLogo-TeX{%
      margin-left:-.15em;%
    }%
  }%
  \global\let\HoLogoCss@ExTeX\relax
}
%    \end{macrocode}
%    \end{macro}
%
% \subsubsection{\hologo{MiKTeX}}
%
%    \begin{macro}{\HoLogo@MiKTeX}
%    \begin{macrocode}
\def\HoLogo@MiKTeX#1{%
  \HOLOGO@mbox{MiK}%
  \HOLOGO@discretionary
  \hologo{TeX}%
}
%    \end{macrocode}
%    \end{macro}
%    \begin{macro}{\HoLogoHtml@MiKTeX}
%    \begin{macrocode}
\let\HoLogoHtml@MiKTeX\HoLogo@MiKTeX
%    \end{macrocode}
%    \end{macro}
%
% \subsubsection{\hologo{OzTeX} and friends}
%
%    Source: \hologo{OzTeX} FAQ \cite{OzTeX}:
%    \begin{quote}
%      |\def\OzTeX{O\kern-.03em z\kern-.15em\TeX}|\\
%      (There is no kerning in OzMF, OzMP and OzTtH.)
%    \end{quote}
%
%    \begin{macro}{\HoLogo@OzTeX}
%    \begin{macrocode}
\def\HoLogo@OzTeX#1{%
  O%
  \kern-.03em %
  z%
  \kern-.15em %
  \hologo{TeX}%
}
%    \end{macrocode}
%    \end{macro}
%    \begin{macro}{\HoLogoHtml@OzTeX}
%    \begin{macrocode}
\def\HoLogoHtml@OzTeX#1{%
  \HoLogoCss@OzTeX
  \HOLOGO@Span{OzTeX}{%
    O%
    \HOLOGO@Span{z}{z}%
    \hologo{TeX}%
  }%
}
%    \end{macrocode}
%    \end{macro}
%    \begin{macro}{\HoLogoCss@OzTeX}
%    \begin{macrocode}
\def\HoLogoCss@OzTeX{%
  \Css{%
    span.HoLogo-OzTeX span.HoLogo-z{%
      margin-left:-.03em;%
      margin-right:-.15em;%
    }%
  }%
  \global\let\HoLogoCss@OzTeX\relax
}
%    \end{macrocode}
%    \end{macro}
%
%    \begin{macro}{\HoLogo@OzMF}
%    \begin{macrocode}
\def\HoLogo@OzMF#1{%
  \HOLOGO@mbox{OzMF}%
}
%    \end{macrocode}
%    \end{macro}
%    \begin{macro}{\HoLogo@OzMP}
%    \begin{macrocode}
\def\HoLogo@OzMP#1{%
  \HOLOGO@mbox{OzMP}%
}
%    \end{macrocode}
%    \end{macro}
%    \begin{macro}{\HoLogo@OzTtH}
%    \begin{macrocode}
\def\HoLogo@OzTtH#1{%
  \HOLOGO@mbox{OzTtH}%
}
%    \end{macrocode}
%    \end{macro}
%
% \subsubsection{\hologo{PCTeX}}
%
%    \begin{macro}{\HoLogo@PCTeX}
%    \begin{macrocode}
\def\HoLogo@PCTeX#1{%
  \HOLOGO@mbox{PC}%
  \hologo{TeX}%
}
%    \end{macrocode}
%    \end{macro}
%    \begin{macro}{\HoLogoHtml@PCTeX}
%    \begin{macrocode}
\let\HoLogoHtml@PCTeX\HoLogo@PCTeX
%    \end{macrocode}
%    \end{macro}
%
% \subsubsection{\hologo{PiCTeX}}
%
%    The original definitions from \xfile{pictex.tex} \cite{PiCTeX}:
%\begin{quote}
%\begin{verbatim}
%\def\PiC{%
%  P%
%  \kern-.12em%
%  \lower.5ex\hbox{I}%
%  \kern-.075em%
%  C%
%}
%\def\PiCTeX{%
%  \PiC
%  \kern-.11em%
%  \TeX
%}
%\end{verbatim}
%\end{quote}
%
%    \begin{macro}{\HoLogo@PiC}
%    \begin{macrocode}
\def\HoLogo@PiC#1{%
  P%
  \kern-.12em%
  \lower.5ex\hbox{I}%
  \kern-.075em%
  C%
  \HOLOGO@SpaceFactor
}
%    \end{macrocode}
%    \end{macro}
%    \begin{macro}{\HoLogoHtml@PiC}
%    \begin{macrocode}
\def\HoLogoHtml@PiC#1{%
  \HoLogoCss@PiC
  \HOLOGO@Span{PiC}{%
    P%
    \HOLOGO@Span{i}{I}%
    C%
  }%
}
%    \end{macrocode}
%    \end{macro}
%    \begin{macro}{\HoLogoCss@PiC}
%    \begin{macrocode}
\def\HoLogoCss@PiC{%
  \Css{%
    span.HoLogo-PiC span.HoLogo-i{%
      position:relative;%
      top:.5ex;%
      margin-left:-.12em;%
      margin-right:-.075em;%
      text-decoration:none;%
    }%
  }%
  \global\let\HoLogoCss@PiC\relax
}
%    \end{macrocode}
%    \end{macro}
%
%    \begin{macro}{\HoLogo@PiCTeX}
%    \begin{macrocode}
\def\HoLogo@PiCTeX#1{%
  \hologo{PiC}%
  \HOLOGO@discretionary
  \kern-.11em%
  \hologo{TeX}%
}
%    \end{macrocode}
%    \end{macro}
%    \begin{macro}{\HoLogoHtml@PiCTeX}
%    \begin{macrocode}
\def\HoLogoHtml@PiCTeX#1{%
  \HoLogoCss@PiCTeX
  \HOLOGO@Span{PiCTeX}{%
    \hologo{PiC}%
    \hologo{TeX}%
  }%
}
%    \end{macrocode}
%    \end{macro}
%    \begin{macro}{\HoLogoCss@PiCTeX}
%    \begin{macrocode}
\def\HoLogoCss@PiCTeX{%
  \Css{%
    span.HoLogo-PiCTeX span.HoLogo-PiC{%
      margin-right:-.11em;%
    }%
  }%
  \global\let\HoLogoCss@PiCTeX\relax
}
%    \end{macrocode}
%    \end{macro}
%
% \subsubsection{\hologo{teTeX}}
%
%    \begin{macro}{\HoLogo@teTeX}
%    \begin{macrocode}
\def\HoLogo@teTeX#1{%
  \HOLOGO@mbox{#1{t}{T}e}%
  \HOLOGO@discretionary
  \hologo{TeX}%
}
%    \end{macrocode}
%    \end{macro}
%    \begin{macro}{\HoLogoCs@teTeX}
%    \begin{macrocode}
\def\HoLogoCs@teTeX#1{#1{t}{T}dfTeX}
%    \end{macrocode}
%    \end{macro}
%    \begin{macro}{\HoLogoBkm@teTeX}
%    \begin{macrocode}
\def\HoLogoBkm@teTeX#1{%
  #1{t}{T}e\hologo{TeX}%
}
%    \end{macrocode}
%    \end{macro}
%    \begin{macro}{\HoLogoHtml@teTeX}
%    \begin{macrocode}
\let\HoLogoHtml@teTeX\HoLogo@teTeX
%    \end{macrocode}
%    \end{macro}
%
% \subsubsection{\hologo{TeX4ht}}
%
%    \begin{macro}{\HoLogo@TeX4ht}
%    \begin{macrocode}
\expandafter\def\csname HoLogo@TeX4ht\endcsname#1{%
  \HOLOGO@mbox{\hologo{TeX}4ht}%
}
%    \end{macrocode}
%    \end{macro}
%    \begin{macro}{\HoLogoHtml@TeX4ht}
%    \begin{macrocode}
\expandafter
\let\csname HoLogoHtml@TeX4ht\expandafter\endcsname
\csname HoLogo@TeX4ht\endcsname
%    \end{macrocode}
%    \end{macro}
%
%
% \subsubsection{\hologo{SageTeX}}
%
%    \begin{macro}{\HoLogo@SageTeX}
%    \begin{macrocode}
\def\HoLogo@SageTeX#1{%
  \HOLOGO@mbox{Sage}%
  \HOLOGO@discretionary
  \HOLOGO@NegativeKerning{eT,oT,To}%
  \hologo{TeX}%
}
%    \end{macrocode}
%    \end{macro}
%    \begin{macro}{\HoLogoHtml@SageTeX}
%    \begin{macrocode}
\let\HoLogoHtml@SageTeX\HoLogo@SageTeX
%    \end{macrocode}
%    \end{macro}
%
% \subsection{\hologo{METAFONT} and friends}
%
%    \begin{macro}{\HoLogo@METAFONT}
%    \begin{macrocode}
\def\HoLogo@METAFONT#1{%
  \HoLogoFont@font{METAFONT}{logo}{%
    \HOLOGO@mbox{META}%
    \HOLOGO@discretionary
    \HOLOGO@mbox{FONT}%
  }%
}
%    \end{macrocode}
%    \end{macro}
%
%    \begin{macro}{\HoLogo@METAPOST}
%    \begin{macrocode}
\def\HoLogo@METAPOST#1{%
  \HoLogoFont@font{METAPOST}{logo}{%
    \HOLOGO@mbox{META}%
    \HOLOGO@discretionary
    \HOLOGO@mbox{POST}%
  }%
}
%    \end{macrocode}
%    \end{macro}
%
%    \begin{macro}{\HoLogo@MetaFun}
%    \begin{macrocode}
\def\HoLogo@MetaFun#1{%
  \HOLOGO@mbox{Meta}%
  \HOLOGO@discretionary
  \HOLOGO@mbox{Fun}%
}
%    \end{macrocode}
%    \end{macro}
%
%    \begin{macro}{\HoLogo@MetaPost}
%    \begin{macrocode}
\def\HoLogo@MetaPost#1{%
  \HOLOGO@mbox{Meta}%
  \HOLOGO@discretionary
  \HOLOGO@mbox{Post}%
}
%    \end{macrocode}
%    \end{macro}
%
% \subsection{Others}
%
% \subsubsection{\hologo{biber}}
%
%    \begin{macro}{\HoLogo@biber}
%    \begin{macrocode}
\def\HoLogo@biber#1{%
  \HOLOGO@mbox{#1{b}{B}i}%
  \HOLOGO@discretionary
  \HOLOGO@mbox{ber}%
}
%    \end{macrocode}
%    \end{macro}
%    \begin{macro}{\HoLogoCs@biber}
%    \begin{macrocode}
\def\HoLogoCs@biber#1{#1{b}{B}iber}
%    \end{macrocode}
%    \end{macro}
%    \begin{macro}{\HoLogoBkm@biber}
%    \begin{macrocode}
\def\HoLogoBkm@biber#1{%
  #1{b}{B}iber%
}
%    \end{macrocode}
%    \end{macro}
%    \begin{macro}{\HoLogoHtml@biber}
%    \begin{macrocode}
\let\HoLogoHtml@biber\HoLogo@biber
%    \end{macrocode}
%    \end{macro}
%
% \subsubsection{\hologo{KOMAScript}}
%
%    \begin{macro}{\HoLogo@KOMAScript}
%    The definition for \hologo{KOMAScript} is taken
%    from \hologo{KOMAScript} (\xfile{scrlogo.dtx}, reformatted) \cite{scrlogo}:
%\begin{quote}
%\begin{verbatim}
%\@ifundefined{KOMAScript}{%
%  \DeclareRobustCommand{\KOMAScript}{%
%    \textsf{%
%      K\kern.05em O\kern.05emM\kern.05em A%
%      \kern.1em-\kern.1em %
%      Script%
%    }%
%  }%
%}{}
%\end{verbatim}
%\end{quote}
%    \begin{macrocode}
\def\HoLogo@KOMAScript#1{%
  \HoLogoFont@font{KOMAScript}{sf}{%
    \HOLOGO@mbox{%
      K\kern.05em%
      O\kern.05em%
      M\kern.05em%
      A%
    }%
    \kern.1em%
    \HOLOGO@hyphen
    \kern.1em%
    \HOLOGO@mbox{Script}%
  }%
}
%    \end{macrocode}
%    \end{macro}
%    \begin{macro}{\HoLogoBkm@KOMAScript}
%    \begin{macrocode}
\def\HoLogoBkm@KOMAScript#1{%
  KOMA-Script%
}
%    \end{macrocode}
%    \end{macro}
%    \begin{macro}{\HoLogoHtml@KOMAScript}
%    \begin{macrocode}
\def\HoLogoHtml@KOMAScript#1{%
  \HoLogoCss@KOMAScript
  \HoLogoFont@font{KOMAScript}{sf}{%
    \HOLOGO@Span{KOMAScript}{%
      K%
      \HOLOGO@Span{O}{O}%
      M%
      \HOLOGO@Span{A}{A}%
      \HOLOGO@Span{hyphen}{-}%
      Script%
    }%
  }%
}
%    \end{macrocode}
%    \end{macro}
%    \begin{macro}{\HoLogoCss@KOMAScript}
%    \begin{macrocode}
\def\HoLogoCss@KOMAScript{%
  \Css{%
    span.HoLogo-KOMAScript{%
      font-family:sans-serif;%
    }%
  }%
  \Css{%
    span.HoLogo-KOMAScript span.HoLogo-O{%
      padding-left:.05em;%
      padding-right:.05em;%
    }%
  }%
  \Css{%
    span.HoLogo-KOMAScript span.HoLogo-A{%
      padding-left:.05em;%
    }%
  }%
  \Css{%
    span.HoLogo-KOMAScript span.HoLogo-hyphen{%
      padding-left:.1em;%
      padding-right:.1em;%
    }%
  }%
  \global\let\HoLogoCss@KOMAScript\relax
}
%    \end{macrocode}
%    \end{macro}
%
% \subsubsection{\hologo{LyX}}
%
%    \begin{macro}{\HoLogo@LyX}
%    The definition is taken from the documentation source files
%    of \hologo{LyX}, \xfile{Intro.lyx} \cite{LyX}:
%\begin{quote}
%\begin{verbatim}
%\def\LyX{%
%  \texorpdfstring{%
%    L\kern-.1667em\lower.25em\hbox{Y}\kern-.125emX\@%
%  }{%
%    LyX%
%  }%
%}
%\end{verbatim}
%\end{quote}
%    \begin{macrocode}
\def\HoLogo@LyX#1{%
  L%
  \kern-.1667em%
  \lower.25em\hbox{Y}%
  \kern-.125em%
  X%
  \HOLOGO@SpaceFactor
}
%    \end{macrocode}
%    \end{macro}
%    \begin{macro}{\HoLogoHtml@LyX}
%    \begin{macrocode}
\def\HoLogoHtml@LyX#1{%
  \HoLogoCss@LyX
  \HOLOGO@Span{LyX}{%
    L%
    \HOLOGO@Span{y}{Y}%
    X%
  }%
}
%    \end{macrocode}
%    \end{macro}
%    \begin{macro}{\HoLogoCss@LyX}
%    \begin{macrocode}
\def\HoLogoCss@LyX{%
  \Css{%
    span.HoLogo-LyX span.HoLogo-y{%
      position:relative;%
      top:.25em;%
      margin-left:-.1667em;%
      margin-right:-.125em;%
      text-decoration:none;%
    }%
  }%
  \global\let\HoLogoCss@LyX\relax
}
%    \end{macrocode}
%    \end{macro}
%
% \subsubsection{\hologo{NTS}}
%
%    \begin{macro}{\HoLogo@NTS}
%    Definition for \hologo{NTS} can be found in
%    package \xpackage{etex\textunderscore man} for the \hologo{eTeX} manual \cite{etexman}
%    and in package \xpackage{dtklogos} \cite{dtklogos}:
%\begin{quote}
%\begin{verbatim}
%\def\NTS{%
%  \leavevmode
%  \hbox{%
%    $%
%      \cal N%
%      \kern-0.35em%
%      \lower0.5ex\hbox{$\cal T$}%
%      \kern-0.2em%
%      S%
%    $%
%  }%
%}
%\end{verbatim}
%\end{quote}
%    \begin{macrocode}
\def\HoLogo@NTS#1{%
  \HoLogoFont@font{NTS}{sy}{%
    N\/%
    \kern-.35em%
    \lower.5ex\hbox{T\/}%
    \kern-.2em%
    S\/%
  }%
  \HOLOGO@SpaceFactor
}
%    \end{macrocode}
%    \end{macro}
%
% \subsubsection{\Hologo{TTH} (\hologo{TeX} to HTML translator)}
%
%    Source: \url{http://hutchinson.belmont.ma.us/tth/}
%    In the HTML source the second `T' is printed as subscript.
%\begin{quote}
%\begin{verbatim}
%T<sub>T</sub>H
%\end{verbatim}
%\end{quote}
%    \begin{macro}{\HoLogo@TTH}
%    \begin{macrocode}
\def\HoLogo@TTH#1{%
  \ltx@mbox{%
    T\HOLOGO@SubScript{T}H%
  }%
  \HOLOGO@SpaceFactor
}
%    \end{macrocode}
%    \end{macro}
%
%    \begin{macro}{\HoLogoHtml@TTH}
%    \begin{macrocode}
\def\HoLogoHtml@TTH#1{%
  T\HCode{<sub>}T\HCode{</sub>}H%
}
%    \end{macrocode}
%    \end{macro}
%
% \subsubsection{\Hologo{HanTheThanh}}
%
%    Partial source: Package \xpackage{dtklogos}.
%    The double accent is U+1EBF (latin small letter e with circumflex
%    and acute).
%    \begin{macro}{\HoLogo@HanTheThanh}
%    \begin{macrocode}
\def\HoLogo@HanTheThanh#1{%
  \ltx@mbox{H\`an}%
  \HOLOGO@space
  \ltx@mbox{%
    Th%
    \HOLOGO@IfCharExists{"1EBF}{%
      \char"1EBF\relax
    }{%
      \^e\hbox to 0pt{\hss\raise .5ex\hbox{\'{}}}%
    }%
  }%
  \HOLOGO@space
  \ltx@mbox{Th\`anh}%
}
%    \end{macrocode}
%    \end{macro}
%    \begin{macro}{\HoLogoBkm@HanTheThanh}
%    \begin{macrocode}
\def\HoLogoBkm@HanTheThanh#1{%
  H\`an %
  Th\HOLOGO@PdfdocUnicode{\^e}{\9036\277} %
  Th\`anh%
}
%    \end{macrocode}
%    \end{macro}
%    \begin{macro}{\HoLogoHtml@HanTheThanh}
%    \begin{macrocode}
\def\HoLogoHtml@HanTheThanh#1{%
  H\`an %
  Th\HCode{&\ltx@hashchar x1ebf;} %
  Th\`anh%
}
%    \end{macrocode}
%    \end{macro}
%
% \subsection{Driver detection}
%
%    \begin{macrocode}
\HOLOGO@IfExists\InputIfFileExists{%
  \InputIfFileExists{hologo.cfg}{}{}%
}{%
  \ltx@IfUndefined{pdf@filesize}{%
    \def\HOLOGO@InputIfExists{%
      \openin\HOLOGO@temp=hologo.cfg\relax
      \ifeof\HOLOGO@temp
        \closein\HOLOGO@temp
      \else
        \closein\HOLOGO@temp
        \begingroup
          \def\x{LaTeX2e}%
        \expandafter\endgroup
        \ifx\fmtname\x
          \input{hologo.cfg}%
        \else
          \input hologo.cfg\relax
        \fi
      \fi
    }%
    \ltx@IfUndefined{newread}{%
      \chardef\HOLOGO@temp=15 %
      \def\HOLOGO@CheckRead{%
        \ifeof\HOLOGO@temp
          \HOLOGO@InputIfExists
        \else
          \ifcase\HOLOGO@temp
            \@PackageWarningNoLine{hologo}{%
              Configuration file ignored, because\MessageBreak
              a free read register could not be found%
            }%
          \else
            \begingroup
              \count\ltx@cclv=\HOLOGO@temp
              \advance\ltx@cclv by \ltx@minusone
              \edef\x{\endgroup
                \chardef\noexpand\HOLOGO@temp=\the\count\ltx@cclv
                \relax
              }%
            \x
          \fi
        \fi
      }%
    }{%
      \csname newread\endcsname\HOLOGO@temp
      \HOLOGO@InputIfExists
    }%
  }{%
    \edef\HOLOGO@temp{\pdf@filesize{hologo.cfg}}%
    \ifx\HOLOGO@temp\ltx@empty
    \else
      \ifnum\HOLOGO@temp>0 %
        \begingroup
          \def\x{LaTeX2e}%
        \expandafter\endgroup
        \ifx\fmtname\x
          \input{hologo.cfg}%
        \else
          \input hologo.cfg\relax
        \fi
      \else
        \@PackageInfoNoLine{hologo}{%
          Empty configuration file `hologo.cfg' ignored%
        }%
      \fi
    \fi
  }%
}
%    \end{macrocode}
%
%    \begin{macrocode}
\def\HOLOGO@temp#1#2{%
  \kv@define@key{HoLogoDriver}{#1}[]{%
    \begingroup
      \def\HOLOGO@temp{##1}%
      \ltx@onelevel@sanitize\HOLOGO@temp
      \ifx\HOLOGO@temp\ltx@empty
      \else
        \@PackageError{hologo}{%
          Value (\HOLOGO@temp) not permitted for option `#1'%
        }%
        \@ehc
      \fi
    \endgroup
    \def\hologoDriver{#2}%
  }%
}%
\def\HOLOGO@@temp#1#2{%
  \ifx\kv@value\relax
    \HOLOGO@temp{#1}{#1}%
  \else
    \HOLOGO@temp{#1}{#2}%
  \fi
}%
\kv@parse@normalized{%
  pdftex,%
  luatex=pdftex,%
  dvipdfm,%
  dvipdfmx=dvipdfm,%
  dvips,%
  dvipsone=dvips,%
  xdvi=dvips,%
  xetex,%
  vtex,%
}\HOLOGO@@temp
%    \end{macrocode}
%
%    \begin{macrocode}
\kv@define@key{HoLogoDriver}{driverfallback}{%
  \def\HOLOGO@DriverFallback{#1}%
}
%    \end{macrocode}
%
%    \begin{macro}{\HOLOGO@DriverFallback}
%    \begin{macrocode}
\def\HOLOGO@DriverFallback{dvips}
%    \end{macrocode}
%    \end{macro}
%
%    \begin{macro}{\hologoDriverSetup}
%    \begin{macrocode}
\def\hologoDriverSetup{%
  \let\hologoDriver\ltx@undefined
  \HOLOGO@DriverSetup
}
%    \end{macrocode}
%    \end{macro}
%
%    \begin{macro}{\HOLOGO@DriverSetup}
%    \begin{macrocode}
\def\HOLOGO@DriverSetup#1{%
  \kvsetkeys{HoLogoDriver}{#1}%
  \HOLOGO@CheckDriver
  \ltx@ifundefined{hologoDriver}{%
    \begingroup
    \edef\x{\endgroup
      \noexpand\kvsetkeys{HoLogoDriver}{\HOLOGO@DriverFallback}%
    }\x
  }{}%
  \@PackageInfoNoLine{hologo}{Using driver `\hologoDriver'}%
}
%    \end{macrocode}
%    \end{macro}
%
%    \begin{macro}{\HOLOGO@CheckDriver}
%    \begin{macrocode}
\def\HOLOGO@CheckDriver{%
  \ifpdf
    \def\hologoDriver{pdftex}%
    \let\HOLOGO@pdfliteral\pdfliteral
    \ifluatex
      \ifx\pdfextension\@undefined\else
        \protected\def\pdfliteral{\pdfextension literal}%
        \let\HOLOGO@pdfliteral\pdfliteral
      \fi
      \ltx@IfUndefined{HOLOGO@pdfliteral}{%
        \ifnum\luatexversion<36 %
        \else
          \begingroup
            \let\HOLOGO@temp\endgroup
            \ifcase0%
                \directlua{%
                  if tex.enableprimitives then %
                    tex.enableprimitives('HOLOGO@', {'pdfliteral'})%
                  else %
                    tex.print('1')%
                  end%
                }%
                \ifx\HOLOGO@pdfliteral\@undefined 1\fi%
                \relax%
              \endgroup
              \let\HOLOGO@temp\relax
              \global\let\HOLOGO@pdfliteral\HOLOGO@pdfliteral
            \fi%
          \HOLOGO@temp
        \fi
      }{}%
    \fi
    \ltx@IfUndefined{HOLOGO@pdfliteral}{%
      \@PackageWarningNoLine{hologo}{%
        Cannot find \string\pdfliteral
      }%
    }{}%
  \else
    \ifxetex
      \def\hologoDriver{xetex}%
    \else
      \ifvtex
        \def\hologoDriver{vtex}%
      \fi
    \fi
  \fi
}
%    \end{macrocode}
%    \end{macro}
%
%    \begin{macro}{\HOLOGO@WarningUnsupportedDriver}
%    \begin{macrocode}
\def\HOLOGO@WarningUnsupportedDriver#1{%
  \@PackageWarningNoLine{hologo}{%
    Logo `#1' needs driver specific macros,\MessageBreak
    but driver `\hologoDriver' is not supported.\MessageBreak
    Use a different driver or\MessageBreak
    load package `graphics' or `pgf'%
  }%
}
%    \end{macrocode}
%    \end{macro}
%
% \subsubsection{Reflect box macros}
%
%    Skip driver part if not needed.
%    \begin{macrocode}
\ltx@IfUndefined{reflectbox}{}{%
  \ltx@IfUndefined{rotatebox}{}{%
    \HOLOGO@AtEnd
  }%
}
\ltx@IfUndefined{pgftext}{}{%
  \HOLOGO@AtEnd
}
\ltx@IfUndefined{psscalebox}{}{%
  \HOLOGO@AtEnd
}
%    \end{macrocode}
%
%    \begin{macrocode}
\def\HOLOGO@temp{LaTeX2e}
\ifx\fmtname\HOLOGO@temp
  \RequirePackage{kvoptions}[2011/06/30]%
  \ProcessKeyvalOptions{HoLogoDriver}%
\fi
\HOLOGO@DriverSetup{}
%    \end{macrocode}
%
%    \begin{macro}{\HOLOGO@ReflectBox}
%    \begin{macrocode}
\def\HOLOGO@ReflectBox#1{%
  \begingroup
    \setbox\ltx@zero\hbox{\begingroup#1\endgroup}%
    \setbox\ltx@two\hbox{%
      \kern\wd\ltx@zero
      \csname HOLOGO@ScaleBox@\hologoDriver\endcsname{-1}{1}{%
        \hbox to 0pt{\copy\ltx@zero\hss}%
      }%
    }%
    \wd\ltx@two=\wd\ltx@zero
    \box\ltx@two
  \endgroup
}
%    \end{macrocode}
%    \end{macro}
%
%    \begin{macro}{\HOLOGO@PointReflectBox}
%    \begin{macrocode}
\def\HOLOGO@PointReflectBox#1{%
  \begingroup
    \setbox\ltx@zero\hbox{\begingroup#1\endgroup}%
    \setbox\ltx@two\hbox{%
      \kern\wd\ltx@zero
      \raise\ht\ltx@zero\hbox{%
        \csname HOLOGO@ScaleBox@\hologoDriver\endcsname{-1}{-1}{%
          \hbox to 0pt{\copy\ltx@zero\hss}%
        }%
      }%
    }%
    \wd\ltx@two=\wd\ltx@zero
    \box\ltx@two
  \endgroup
}
%    \end{macrocode}
%    \end{macro}
%
%    We must define all variants because of dynamic driver setup.
%    \begin{macrocode}
\def\HOLOGO@temp#1#2{#2}
%    \end{macrocode}
%
%    \begin{macro}{\HOLOGO@ScaleBox@pdftex}
%    \begin{macrocode}
\HOLOGO@temp{pdftex}{%
  \def\HOLOGO@ScaleBox@pdftex#1#2#3{%
    \HOLOGO@pdfliteral{%
      q #1 0 0 #2 0 0 cm%
    }%
    #3%
    \HOLOGO@pdfliteral{%
      Q%
    }%
  }%
}
%    \end{macrocode}
%    \end{macro}
%    \begin{macro}{\HOLOGO@ScaleBox@dvips}
%    \begin{macrocode}
\HOLOGO@temp{dvips}{%
  \def\HOLOGO@ScaleBox@dvips#1#2#3{%
    \special{ps:%
      gsave %
      currentpoint %
      currentpoint translate %
      #1 #2 scale %
      neg exch neg exch translate%
    }%
    #3%
    \special{ps:%
      currentpoint %
      grestore %
      moveto%
    }%
  }%
}
%    \end{macrocode}
%    \end{macro}
%    \begin{macro}{\HOLOGO@ScaleBox@dvipdfm}
%    \begin{macrocode}
\HOLOGO@temp{dvipdfm}{%
  \let\HOLOGO@ScaleBox@dvipdfm\HOLOGO@ScaleBox@dvips
}
%    \end{macrocode}
%    \end{macro}
%    Since \hologo{XeTeX} v0.6.
%    \begin{macro}{\HOLOGO@ScaleBox@xetex}
%    \begin{macrocode}
\HOLOGO@temp{xetex}{%
  \def\HOLOGO@ScaleBox@xetex#1#2#3{%
    \special{x:gsave}%
    \special{x:scale #1 #2}%
    #3%
    \special{x:grestore}%
  }%
}
%    \end{macrocode}
%    \end{macro}
%    \begin{macro}{\HOLOGO@ScaleBox@vtex}
%    \begin{macrocode}
\HOLOGO@temp{vtex}{%
  \def\HOLOGO@ScaleBox@vtex#1#2#3{%
    \special{r(#1,0,0,#2,0,0}%
    #3%
    \special{r)}%
  }%
}
%    \end{macrocode}
%    \end{macro}
%
%    \begin{macrocode}
\HOLOGO@AtEnd%
%</package>
%    \end{macrocode}
%
% \section{Test}
%
% \subsection{Catcode checks for loading}
%
%    \begin{macrocode}
%<*test1>
%    \end{macrocode}
%    \begin{macrocode}
\catcode`\{=1 %
\catcode`\}=2 %
\catcode`\#=6 %
\catcode`\@=11 %
\expandafter\ifx\csname count@\endcsname\relax
  \countdef\count@=255 %
\fi
\expandafter\ifx\csname @gobble\endcsname\relax
  \long\def\@gobble#1{}%
\fi
\expandafter\ifx\csname @firstofone\endcsname\relax
  \long\def\@firstofone#1{#1}%
\fi
\expandafter\ifx\csname loop\endcsname\relax
  \expandafter\@firstofone
\else
  \expandafter\@gobble
\fi
{%
  \def\loop#1\repeat{%
    \def\body{#1}%
    \iterate
  }%
  \def\iterate{%
    \body
      \let\next\iterate
    \else
      \let\next\relax
    \fi
    \next
  }%
  \let\repeat=\fi
}%
\def\RestoreCatcodes{}
\count@=0 %
\loop
  \edef\RestoreCatcodes{%
    \RestoreCatcodes
    \catcode\the\count@=\the\catcode\count@\relax
  }%
\ifnum\count@<255 %
  \advance\count@ 1 %
\repeat

\def\RangeCatcodeInvalid#1#2{%
  \count@=#1\relax
  \loop
    \catcode\count@=15 %
  \ifnum\count@<#2\relax
    \advance\count@ 1 %
  \repeat
}
\def\RangeCatcodeCheck#1#2#3{%
  \count@=#1\relax
  \loop
    \ifnum#3=\catcode\count@
    \else
      \errmessage{%
        Character \the\count@\space
        with wrong catcode \the\catcode\count@\space
        instead of \number#3%
      }%
    \fi
  \ifnum\count@<#2\relax
    \advance\count@ 1 %
  \repeat
}
\def\space{ }
\expandafter\ifx\csname LoadCommand\endcsname\relax
  \def\LoadCommand{\input hologo.sty\relax}%
\fi
\def\Test{%
  \RangeCatcodeInvalid{0}{47}%
  \RangeCatcodeInvalid{58}{64}%
  \RangeCatcodeInvalid{91}{96}%
  \RangeCatcodeInvalid{123}{255}%
  \catcode`\@=12 %
  \catcode`\\=0 %
  \catcode`\%=14 %
  \LoadCommand
  \RangeCatcodeCheck{0}{36}{15}%
  \RangeCatcodeCheck{37}{37}{14}%
  \RangeCatcodeCheck{38}{47}{15}%
  \RangeCatcodeCheck{48}{57}{12}%
  \RangeCatcodeCheck{58}{63}{15}%
  \RangeCatcodeCheck{64}{64}{12}%
  \RangeCatcodeCheck{65}{90}{11}%
  \RangeCatcodeCheck{91}{91}{15}%
  \RangeCatcodeCheck{92}{92}{0}%
  \RangeCatcodeCheck{93}{96}{15}%
  \RangeCatcodeCheck{97}{122}{11}%
  \RangeCatcodeCheck{123}{255}{15}%
  \RestoreCatcodes
}
\Test
\csname @@end\endcsname
\end
%    \end{macrocode}
%    \begin{macrocode}
%</test1>
%    \end{macrocode}
%
% \subsection{Spacefactor}
%
%    The space factor must be 1000 after a logo. If it is greater 1000
%    then the following space is a space after a sentence closing point.
%    If the space factor is smaller 1000 then an immediate following
%    dot is interpreted as abbreviation, not sentence closing point.
%
%    \begin{macrocode}
%<*test-spacefactor>
\NeedsTeXFormat{LaTeX2e}
\documentclass{article}
\usepackage{hologo}[2016/05/12]
\usepackage{kvsetkeys}
\usepackage{qstest}
\IncludeTests{*}
\LogTests{log}{*}{*}
\begin{document}
\begin{qstest}{spacefactor}{spacefactor}
\newcommand*{\Test}[1]{%
  \sbox0{%
    \hologo{#1}%
    \Expect*{1000 (#1)}*{\the\spacefactor\space(#1)}%
  }%
}%
\makeatletter
\def\TestList{}
\def\hologoEntry#1#2#3{%
  \edef\TestList{%
    \ifx\TestList\@empty
    \else
      \TestList,%
    \fi
    #1%
    \ifx\\#2\\%
    \else
      ={variant=#2}%
    \fi
  }%
}
\hologoList
\expandafter\kv@parse@normalized\expandafter{%
  \TestList
}{%
  \begingroup
    \let\@logo=\kv@key
    \ifx\kv@value\relax
    \else
      \expandafter\hologoLogoSetup\expandafter\@logo\expandafter{%
        \kv@value
      }%
    \fi
    \Test\@logo
  \endgroup
  \@gobbletwo
}
\end{qstest}
\end{document}
%</test-spacefactor>
%    \end{macrocode}
%
% \subsection{Complete list}
%
%    \begin{macrocode}
%<*test-list>
\NeedsTeXFormat{LaTeX2e}
\documentclass[12pt,a4paper]{article}
\usepackage{hologo}[2016/05/12]
\usepackage[T1]{fontenc}
\usepackage{lmodern}
\usepackage{parskip}
\usepackage[unicode]{hyperref}[2011/09/28]
\usepackage{bookmark}[2011/09/19]
\bookmarksetup{%
  numbered,%
  open,%
  openlevel=2,%
}
\renewcommand*{\contentsname}{List of logos}
\begin{document}
\tableofcontents
\def\TestFont#1#2#3#4#5#6{%
  \begingroup
    \usefont{#3}{#4}{#5}{#6}%
    \HologoVariant{#1}{#2}/\hologoVariant{#1}{#2}%
    \quad
    \begingroup\scriptsize\hologoVariant{#1}{#2}\endgroup
    \quad
  \endgroup
  (#3/#4/#5/#6)%
  \par
}
\makeatletter
\def\hologoEntry#1#2#3{%
  \section{%
    \HologoVariant{#1}{#2}/\hologoVariant{#1}{#2} %
    {[#1\ifx\\#2\\\else\space(#2)\fi]}% hash-ok
  }% braces around [] because of bug in tex4ht
  \begingroup
    \hypersetup{unicode=false}%
    \bookmark[%
      dest=\@currentHref,%
      rellevel=1,%
      keeplevel,%
    ]{%
      \HologoVariant{#1}{#2}/\hologoVariant{#1}{#2} %
      (PDFDocEncoding)%
    }%
  \endgroup
  \TestFont{#1}{#2}{OT1}{cmr}{m}{n}%
  \TestFont{#1}{#2}{OT1}{cmss}{m}{n}%
  \TestFont{#1}{#2}{OT1}{cmr}{b}{n}%
  \TestFont{#1}{#2}{OT1}{cmr}{m}{it}%
  \TestFont{#1}{#2}{OT1}{cmtt}{m}{n}%
  \TestFont{#1}{#2}{T1}{lmr}{m}{n}%
  \TestFont{#1}{#2}{T1}{lmss}{m}{n}%
  \TestFont{#1}{#2}{T1}{lmr}{b}{n}%
  \TestFont{#1}{#2}{T1}{lmr}{m}{it}%
  \TestFont{#1}{#2}{T1}{lmtt}{m}{n}%
  \TestFont{#1}{#2}{T1}{lmvtt}{m}{n}%
  \TestFont{#1}{#2}{T1}{qtm}{m}{n}%
  \TestFont{#1}{#2}{T1}{qhv}{m}{n}%
  \TestFont{#1}{#2}{T1}{qtm}{b}{n}%
  \TestFont{#1}{#2}{T1}{qtm}{m}{it}%
  \TestFont{#1}{#2}{T1}{qcr}{m}{n}%
  \newpage
}
\makeatother
\hologoList
\end{document}
%</test-list>
%    \end{macrocode}
%
% \section{Installation}
%
% \subsection{Download}
%
% \paragraph{Package.} This package is available on
% CTAN\footnote{\url{ftp://ftp.ctan.org/tex-archive/}}:
% \begin{description}
% \item[\CTAN{macros/latex/contrib/oberdiek/hologo.dtx}] The source file.
% \item[\CTAN{macros/latex/contrib/oberdiek/hologo.pdf}] Documentation.
% \end{description}
%
%
% \paragraph{Bundle.} All the packages of the bundle `oberdiek'
% are also available in a TDS compliant ZIP archive. There
% the packages are already unpacked and the documentation files
% are generated. The files and directories obey the TDS standard.
% \begin{description}
% \item[\CTAN{install/macros/latex/contrib/oberdiek.tds.zip}]
% \end{description}
% \emph{TDS} refers to the standard ``A Directory Structure
% for \TeX\ Files'' (\CTAN{tds/tds.pdf}). Directories
% with \xfile{texmf} in their name are usually organized this way.
%
% \subsection{Bundle installation}
%
% \paragraph{Unpacking.} Unpack the \xfile{oberdiek.tds.zip} in the
% TDS tree (also known as \xfile{texmf} tree) of your choice.
% Example (linux):
% \begin{quote}
%   |unzip oberdiek.tds.zip -d ~/texmf|
% \end{quote}
%
% \paragraph{Script installation.}
% Check the directory \xfile{TDS:scripts/oberdiek/} for
% scripts that need further installation steps.
% Package \xpackage{attachfile2} comes with the Perl script
% \xfile{pdfatfi.pl} that should be installed in such a way
% that it can be called as \texttt{pdfatfi}.
% Example (linux):
% \begin{quote}
%   |chmod +x scripts/oberdiek/pdfatfi.pl|\\
%   |cp scripts/oberdiek/pdfatfi.pl /usr/local/bin/|
% \end{quote}
%
% \subsection{Package installation}
%
% \paragraph{Unpacking.} The \xfile{.dtx} file is a self-extracting
% \docstrip\ archive. The files are extracted by running the
% \xfile{.dtx} through \plainTeX:
% \begin{quote}
%   \verb|tex hologo.dtx|
% \end{quote}
%
% \paragraph{TDS.} Now the different files must be moved into
% the different directories in your installation TDS tree
% (also known as \xfile{texmf} tree):
% \begin{quote}
% \def\t{^^A
% \begin{tabular}{@{}>{\ttfamily}l@{ $\rightarrow$ }>{\ttfamily}l@{}}
%   hologo.sty & tex/generic/oberdiek/hologo.sty\\
%   hologo.pdf & doc/latex/oberdiek/hologo.pdf\\
%   example/hologo-example.tex & doc/latex/oberdiek/example/hologo-example.tex\\
%   test/hologo-test1.tex & doc/latex/oberdiek/test/hologo-test1.tex\\
%   test/hologo-test-spacefactor.tex & doc/latex/oberdiek/test/hologo-test-spacefactor.tex\\
%   test/hologo-test-list.tex & doc/latex/oberdiek/test/hologo-test-list.tex\\
%   hologo.dtx & source/latex/oberdiek/hologo.dtx\\
% \end{tabular}^^A
% }^^A
% \sbox0{\t}^^A
% \ifdim\wd0>\linewidth
%   \begingroup
%     \advance\linewidth by\leftmargin
%     \advance\linewidth by\rightmargin
%   \edef\x{\endgroup
%     \def\noexpand\lw{\the\linewidth}^^A
%   }\x
%   \def\lwbox{^^A
%     \leavevmode
%     \hbox to \linewidth{^^A
%       \kern-\leftmargin\relax
%       \hss
%       \usebox0
%       \hss
%       \kern-\rightmargin\relax
%     }^^A
%   }^^A
%   \ifdim\wd0>\lw
%     \sbox0{\small\t}^^A
%     \ifdim\wd0>\linewidth
%       \ifdim\wd0>\lw
%         \sbox0{\footnotesize\t}^^A
%         \ifdim\wd0>\linewidth
%           \ifdim\wd0>\lw
%             \sbox0{\scriptsize\t}^^A
%             \ifdim\wd0>\linewidth
%               \ifdim\wd0>\lw
%                 \sbox0{\tiny\t}^^A
%                 \ifdim\wd0>\linewidth
%                   \lwbox
%                 \else
%                   \usebox0
%                 \fi
%               \else
%                 \lwbox
%               \fi
%             \else
%               \usebox0
%             \fi
%           \else
%             \lwbox
%           \fi
%         \else
%           \usebox0
%         \fi
%       \else
%         \lwbox
%       \fi
%     \else
%       \usebox0
%     \fi
%   \else
%     \lwbox
%   \fi
% \else
%   \usebox0
% \fi
% \end{quote}
% If you have a \xfile{docstrip.cfg} that configures and enables \docstrip's
% TDS installing feature, then some files can already be in the right
% place, see the documentation of \docstrip.
%
% \subsection{Refresh file name databases}
%
% If your \TeX~distribution
% (\teTeX, \mikTeX, \dots) relies on file name databases, you must refresh
% these. For example, \teTeX\ users run \verb|texhash| or
% \verb|mktexlsr|.
%
% \subsection{Some details for the interested}
%
% \paragraph{Attached source.}
%
% The PDF documentation on CTAN also includes the
% \xfile{.dtx} source file. It can be extracted by
% AcrobatReader 6 or higher. Another option is \textsf{pdftk},
% e.g. unpack the file into the current directory:
% \begin{quote}
%   \verb|pdftk hologo.pdf unpack_files output .|
% \end{quote}
%
% \paragraph{Unpacking with \LaTeX.}
% The \xfile{.dtx} chooses its action depending on the format:
% \begin{description}
% \item[\plainTeX:] Run \docstrip\ and extract the files.
% \item[\LaTeX:] Generate the documentation.
% \end{description}
% If you insist on using \LaTeX\ for \docstrip\ (really,
% \docstrip\ does not need \LaTeX), then inform the autodetect routine
% about your intention:
% \begin{quote}
%   \verb|latex \let\install=y\input{hologo.dtx}|
% \end{quote}
% Do not forget to quote the argument according to the demands
% of your shell.
%
% \paragraph{Generating the documentation.}
% You can use both the \xfile{.dtx} or the \xfile{.drv} to generate
% the documentation. The process can be configured by the
% configuration file \xfile{ltxdoc.cfg}. For instance, put this
% line into this file, if you want to have A4 as paper format:
% \begin{quote}
%   \verb|\PassOptionsToClass{a4paper}{article}|
% \end{quote}
% An example follows how to generate the
% documentation with pdf\LaTeX:
% \begin{quote}
%\begin{verbatim}
%pdflatex hologo.dtx
%makeindex -s gind.ist hologo.idx
%pdflatex hologo.dtx
%makeindex -s gind.ist hologo.idx
%pdflatex hologo.dtx
%\end{verbatim}
% \end{quote}
%
% \section{Catalogue}
%
% The following XML file can be used as source for the
% \href{http://mirror.ctan.org/help/Catalogue/catalogue.html}{\TeX\ Catalogue}.
% The elements \texttt{caption} and \texttt{description} are imported
% from the original XML file from the Catalogue.
% The name of the XML file in the Catalogue is \xfile{hologo.xml}.
%    \begin{macrocode}
%<*catalogue>
<?xml version='1.0' encoding='us-ascii'?>
<!DOCTYPE entry SYSTEM 'catalogue.dtd'>
<entry datestamp='$Date$' modifier='$Author$' id='hologo'>
  <name>hologo</name>
  <caption>A collection of logos with bookmark support.</caption>
  <authorref id='auth:oberdiek'/>
  <copyright owner='Heiko Oberdiek' year='2010-2012'/>
  <license type='lppl1.3'/>
  <version number='1.10'/>
  <description>
    The package defines a single command <tt>\hologo</tt>, whose
    argument is the usual case-confused ASCII version of the logo.
    The command is bookmark-enabled, so that every logo becomes
    available in bookmarks without further work.
    <p/>
    The package is part of the <xref refid='oberdiek'>oberdiek</xref>
    bundle.
  </description>
  <documentation details='Package documentation'
      href='ctan:/macros/latex/contrib/oberdiek/hologo.pdf'/>
  <ctan file='true' path='/macros/latex/contrib/oberdiek/hologo.dtx'/>
  <miktex location='oberdiek'/>
  <texlive location='oberdiek'/>
  <install path='/macros/latex/contrib/oberdiek/oberdiek.tds.zip'/>
</entry>
%</catalogue>
%    \end{macrocode}
%
% \begin{thebibliography}{9}
% \raggedright
%
% \bibitem{btxdoc}
% Oren Patashnik,
% \textit{\hologo{BibTeX}ing},
% 1988-02-08.\\
% \CTAN{biblio/bibtex/base/}
%
% \bibitem{dtklogos}
% Gerd Neugebauer, DANTE,
% \textit{Package \xpackage{dtklogos}},
% 2011-04-25.\\
% \CTAN{usergrps/dante/dtk/dtklogos.sty}
%
% \bibitem{etexman}
% The \hologo{NTS} Team,
% \textit{The \hologo{eTeX} manual},
% 1998-02.\\
% \CTAN{systems/e-tex/v2/doc/}
%
% \bibitem{ExTeX-FAQ}
% The \hologo{ExTeX} group,
% \textit{\hologo{ExTeX}: FAQ -- How is \hologo{ExTeX} typeset?},
% 2007-04-14.\\
% \url{http://www.extex.org/documentation/faq.html}
%
% \bibitem{LyX}
% %@MISC{ LyX,
% %  title = {{LyX 2.0.0 -- The Document Processor [Computer software and manual]}},
% %  author = {{The LyX Team}},
% %  howpublished = {Internet: http://www.lyx.org},
% %  year = {2011-05-08},
% %  note = {Retrieved May 10, 2011, from http://www.lyx.org},
% %  url = {http://www.lyx.org/}
% %}
% The \hologo{LyX} Team,
% \textit{\hologo{LyX} -- The Document Processor},
% 2011-05-08.\\
% \url{http://www.lyx.org/}
%
% \bibitem{OzTeX}
% Andrew Trevorrow,
% \hologo{OzTeX} FAQ: What is the correct way to typeset ``\hologo{OzTeX}''?,
% 2011-09-15 (visited).
% \url{http://www.trevorrow.com/oztex/ozfaq.html#oztex-logo}
%
% \bibitem{PiCTeX}
% Michael Wichura,
% \textit{The \hologo{PiCTeX} macro package},
% 1987-09-21.
% \CTAN{graphics/pictex/}
%
% \bibitem{scrlogo}
% Markus Kohm,
% \textit{\hologo{KOMAScript} Datei \xfile{scrlogo.dtx}},
% 2009-01-30.\\
% \CTAN{install/macros/latex/contrib/komascript.tds.zip}
%
% \end{thebibliography}
%
% \begin{History}
%   \begin{Version}{2010/04/08 v1.0}
%   \item
%     The first version.
%   \end{Version}
%   \begin{Version}{2010/04/16 v1.1}
%   \item
%     \cs{Hologo} added for support of logos at start of a sentence.
%   \item
%     \cs{hologoSetup} and \cs{hologoLogoSetup} added.
%   \item
%     Options \xoption{break}, \xoption{hyphenbreak}, \xoption{spacebreak}
%     added.
%   \item
%     Variant support added by option \xoption{variant}.
%   \end{Version}
%   \begin{Version}{2010/04/24 v1.2}
%   \item
%     \hologo{LaTeX3} added.
%   \item
%     \hologo{VTeX} added.
%   \end{Version}
%   \begin{Version}{2010/11/21 v1.3}
%   \item
%     \hologo{iniTeX}, \hologo{virTeX} added.
%   \end{Version}
%   \begin{Version}{2011/03/25 v1.4}
%   \item
%     \hologo{ConTeXt} with variants added.
%   \item
%     Option \xoption{discretionarybreak} added as refinement for
%     option \xoption{break}.
%   \end{Version}
%   \begin{Version}{2011/04/21 v1.5}
%   \item
%     Wrong TDS directory for test files fixed.
%   \end{Version}
%   \begin{Version}{2011/10/01 v1.6}
%   \item
%     Support for package \xpackage{tex4ht} added.
%   \item
%     Support for \cs{csname} added if \cs{ifincsname} is available.
%   \item
%     New logos:
%     \hologo{(La)TeX},
%     \hologo{biber},
%     \hologo{BibTeX} (\xoption{sc}, \xoption{sf}),
%     \hologo{emTeX},
%     \hologo{ExTeX},
%     \hologo{KOMAScript},
%     \hologo{La},
%     \hologo{LyX},
%     \hologo{MiKTeX},
%     \hologo{NTS},
%     \hologo{OzMF},
%     \hologo{OzMP},
%     \hologo{OzTeX},
%     \hologo{OzTtH},
%     \hologo{PCTeX},
%     \hologo{PiC},
%     \hologo{PiCTeX},
%     \hologo{METAFONT},
%     \hologo{MetaFun},
%     \hologo{METAPOST},
%     \hologo{MetaPost},
%     \hologo{SLiTeX} (\xoption{lift}, \xoption{narrow}, \xoption{simple}),
%     \hologo{SliTeX} (\xoption{narrow}, \xoption{simple}, \xoption{lift}),
%     \hologo{teTeX}.
%   \item
%     Fixes:
%     \hologo{iniTeX},
%     \hologo{pdfLaTeX},
%     \hologo{pdfTeX},
%     \hologo{virTeX}.
%   \item
%     \cs{hologoFontSetup} and \cs{hologoLogoFontSetup} added.
%   \item
%     \cs{hologoVariant} and \cs{HologoVariant} added.
%   \end{Version}
%   \begin{Version}{2011/11/22 v1.7}
%   \item
%     New logos:
%     \hologo{BibTeX8},
%     \hologo{LaTeXML},
%     \hologo{SageTeX},
%     \hologo{TeX4ht},
%     \hologo{TTH}.
%   \item
%     \hologo{Xe} and friends: Driver stuff fixed.
%   \item
%     \hologo{Xe} and friends: Support for italic added.
%   \item
%     \hologo{Xe} and friends: Package support for \xpackage{pgf}
%     and \xpackage{pstricks} added.
%   \end{Version}
%   \begin{Version}{2011/11/29 v1.8}
%   \item
%     New logos:
%     \hologo{HanTheThanh}.
%   \end{Version}
%   \begin{Version}{2011/12/21 v1.9}
%   \item
%     Patch for package \xpackage{ifxetex} added for the case that
%     \cs{newif} is undefined in \hologo{iniTeX}.
%   \item
%     Some fixes for \hologo{iniTeX}.
%   \end{Version}
%   \begin{Version}{2012/04/26 v1.10}
%   \item
%     Fix in bookmark version of logo ``\hologo{HanTheThanh}''.
%   \end{Version}
%   \begin{Version}{2016/05/12 v1.11}
%   \item
%     Update HOLOGO@IfCharExists (previously in texlive)
%   \item define pdfliteral in current luatex.
%   \end{Version}
% \end{History}
%
% \PrintIndex
%
% \Finale
\endinput
%
        \else
          \input hologo.cfg\relax
        \fi
      \fi
    }%
    \ltx@IfUndefined{newread}{%
      \chardef\HOLOGO@temp=15 %
      \def\HOLOGO@CheckRead{%
        \ifeof\HOLOGO@temp
          \HOLOGO@InputIfExists
        \else
          \ifcase\HOLOGO@temp
            \@PackageWarningNoLine{hologo}{%
              Configuration file ignored, because\MessageBreak
              a free read register could not be found%
            }%
          \else
            \begingroup
              \count\ltx@cclv=\HOLOGO@temp
              \advance\ltx@cclv by \ltx@minusone
              \edef\x{\endgroup
                \chardef\noexpand\HOLOGO@temp=\the\count\ltx@cclv
                \relax
              }%
            \x
          \fi
        \fi
      }%
    }{%
      \csname newread\endcsname\HOLOGO@temp
      \HOLOGO@InputIfExists
    }%
  }{%
    \edef\HOLOGO@temp{\pdf@filesize{hologo.cfg}}%
    \ifx\HOLOGO@temp\ltx@empty
    \else
      \ifnum\HOLOGO@temp>0 %
        \begingroup
          \def\x{LaTeX2e}%
        \expandafter\endgroup
        \ifx\fmtname\x
          % \iffalse meta-comment
%
% File: hologo.dtx
% Version: 2016/05/12 v1.11
% Info: A logo collection with bookmark support
%
% Copyright (C) 2010-2012 by
%    Heiko Oberdiek <heiko.oberdiek at googlemail.com>
%
% This work may be distributed and/or modified under the
% conditions of the LaTeX Project Public License, either
% version 1.3c of this license or (at your option) any later
% version. This version of this license is in
%    http://www.latex-project.org/lppl/lppl-1-3c.txt
% and the latest version of this license is in
%    http://www.latex-project.org/lppl.txt
% and version 1.3 or later is part of all distributions of
% LaTeX version 2005/12/01 or later.
%
% This work has the LPPL maintenance status "maintained".
%
% This Current Maintainer of this work is Heiko Oberdiek.
%
% The Base Interpreter refers to any `TeX-Format',
% because some files are installed in TDS:tex/generic//.
%
% This work consists of the main source file hologo.dtx
% and the derived files
%    hologo.sty, hologo.pdf, hologo.ins, hologo.drv, hologo-example.tex,
%    hologo-test1.tex, hologo-test-spacefactor.tex,
%    hologo-test-list.tex.
%
% Distribution:
%    CTAN:macros/latex/contrib/oberdiek/hologo.dtx
%    CTAN:macros/latex/contrib/oberdiek/hologo.pdf
%
% Unpacking:
%    (a) If hologo.ins is present:
%           tex hologo.ins
%    (b) Without hologo.ins:
%           tex hologo.dtx
%    (c) If you insist on using LaTeX
%           latex \let\install=y\input{hologo.dtx}
%        (quote the arguments according to the demands of your shell)
%
% Documentation:
%    (a) If hologo.drv is present:
%           latex hologo.drv
%    (b) Without hologo.drv:
%           latex hologo.dtx; ...
%    The class ltxdoc loads the configuration file ltxdoc.cfg
%    if available. Here you can specify further options, e.g.
%    use A4 as paper format:
%       \PassOptionsToClass{a4paper}{article}
%
%    Programm calls to get the documentation (example):
%       pdflatex hologo.dtx
%       makeindex -s gind.ist hologo.idx
%       pdflatex hologo.dtx
%       makeindex -s gind.ist hologo.idx
%       pdflatex hologo.dtx
%
% Installation:
%    TDS:tex/generic/oberdiek/hologo.sty
%    TDS:doc/latex/oberdiek/hologo.pdf
%    TDS:doc/latex/oberdiek/example/hologo-example.tex
%    TDS:doc/latex/oberdiek/test/hologo-test1.tex
%    TDS:doc/latex/oberdiek/test/hologo-test-spacefactor.tex
%    TDS:doc/latex/oberdiek/test/hologo-test-list.tex
%    TDS:source/latex/oberdiek/hologo.dtx
%
%<*ignore>
\begingroup
  \catcode123=1 %
  \catcode125=2 %
  \def\x{LaTeX2e}%
\expandafter\endgroup
\ifcase 0\ifx\install y1\fi\expandafter
         \ifx\csname processbatchFile\endcsname\relax\else1\fi
         \ifx\fmtname\x\else 1\fi\relax
\else\csname fi\endcsname
%</ignore>
%<*install>
\input docstrip.tex
\Msg{************************************************************************}
\Msg{* Installation}
\Msg{* Package: hologo 2016/05/12 v1.11 A logo collection with bookmark support (HO)}
\Msg{************************************************************************}

\keepsilent
\askforoverwritefalse

\let\MetaPrefix\relax
\preamble

This is a generated file.

Project: hologo
Version: 2016/05/12 v1.11

Copyright (C) 2010-2012 by
   Heiko Oberdiek <heiko.oberdiek at googlemail.com>

This work may be distributed and/or modified under the
conditions of the LaTeX Project Public License, either
version 1.3c of this license or (at your option) any later
version. This version of this license is in
   http://www.latex-project.org/lppl/lppl-1-3c.txt
and the latest version of this license is in
   http://www.latex-project.org/lppl.txt
and version 1.3 or later is part of all distributions of
LaTeX version 2005/12/01 or later.

This work has the LPPL maintenance status "maintained".

This Current Maintainer of this work is Heiko Oberdiek.

The Base Interpreter refers to any `TeX-Format',
because some files are installed in TDS:tex/generic//.

This work consists of the main source file hologo.dtx
and the derived files
   hologo.sty, hologo.pdf, hologo.ins, hologo.drv, hologo-example.tex,
   hologo-test1.tex, hologo-test-spacefactor.tex,
   hologo-test-list.tex.

\endpreamble
\let\MetaPrefix\DoubleperCent

\generate{%
  \file{hologo.ins}{\from{hologo.dtx}{install}}%
  \file{hologo.drv}{\from{hologo.dtx}{driver}}%
  \usedir{tex/generic/oberdiek}%
  \file{hologo.sty}{\from{hologo.dtx}{package}}%
  \usedir{doc/latex/oberdiek/example}%
  \file{hologo-example.tex}{\from{hologo.dtx}{example}}%
  \usedir{doc/latex/oberdiek/test}%
  \file{hologo-test1.tex}{\from{hologo.dtx}{test1}}%
  \file{hologo-test-spacefactor.tex}{\from{hologo.dtx}{test-spacefactor}}%
  \file{hologo-test-list.tex}{\from{hologo.dtx}{test-list}}%
  \nopreamble
  \nopostamble
  \usedir{source/latex/oberdiek/catalogue}%
  \file{hologo.xml}{\from{hologo.dtx}{catalogue}}%
}

\catcode32=13\relax% active space
\let =\space%
\Msg{************************************************************************}
\Msg{*}
\Msg{* To finish the installation you have to move the following}
\Msg{* file into a directory searched by TeX:}
\Msg{*}
\Msg{*     hologo.sty}
\Msg{*}
\Msg{* To produce the documentation run the file `hologo.drv'}
\Msg{* through LaTeX.}
\Msg{*}
\Msg{* Happy TeXing!}
\Msg{*}
\Msg{************************************************************************}

\endbatchfile
%</install>
%<*ignore>
\fi
%</ignore>
%<*driver>
\NeedsTeXFormat{LaTeX2e}
\ProvidesFile{hologo.drv}%
  [2016/05/12 v1.11 A logo collection with bookmark support (HO)]%
\documentclass{ltxdoc}
\usepackage{holtxdoc}[2011/11/22]
\usepackage{hologo}[2016/05/12]
\usepackage{longtable}
\usepackage{array}
\usepackage{paralist}
%\usepackage[T1]{fontenc}
%\usepackage{lmodern}
\begin{document}
  \DocInput{hologo.dtx}%
\end{document}
%</driver>
% \fi
%
%
% \CharacterTable
%  {Upper-case    \A\B\C\D\E\F\G\H\I\J\K\L\M\N\O\P\Q\R\S\T\U\V\W\X\Y\Z
%   Lower-case    \a\b\c\d\e\f\g\h\i\j\k\l\m\n\o\p\q\r\s\t\u\v\w\x\y\z
%   Digits        \0\1\2\3\4\5\6\7\8\9
%   Exclamation   \!     Double quote  \"     Hash (number) \#
%   Dollar        \$     Percent       \%     Ampersand     \&
%   Acute accent  \'     Left paren    \(     Right paren   \)
%   Asterisk      \*     Plus          \+     Comma         \,
%   Minus         \-     Point         \.     Solidus       \/
%   Colon         \:     Semicolon     \;     Less than     \<
%   Equals        \=     Greater than  \>     Question mark \?
%   Commercial at \@     Left bracket  \[     Backslash     \\
%   Right bracket \]     Circumflex    \^     Underscore    \_
%   Grave accent  \`     Left brace    \{     Vertical bar  \|
%   Right brace   \}     Tilde         \~}
%
% \GetFileInfo{hologo.drv}
%
% \title{The \xpackage{hologo} package}
% \date{2016/05/12 v1.11}
% \author{Heiko Oberdiek\\\xemail{heiko.oberdiek at googlemail.com}}
%
% \maketitle
%
% \begin{abstract}
% This package starts a collection of logos with support for bookmarks
% strings.
% \end{abstract}
%
% \tableofcontents
%
% \section{Documentation}
%
% \subsection{Logo macros}
%
% \begin{declcs}{hologo} \M{name}
% \end{declcs}
% Macro \cs{hologo} sets the logo with name \meta{name}.
% The following table shows the supported names.
%
% \begingroup
%   \def\hologoEntry#1#2#3{^^A
%     #1&#2&\hologoLogoSetup{#1}{variant=#2}\hologo{#1}&#3\tabularnewline
%   }
%   \begin{longtable}{>{\ttfamily}l>{\ttfamily}lll}
%     \rmfamily\bfseries{name} & \rmfamily\bfseries variant
%     & \bfseries logo & \bfseries since\\
%     \hline
%     \endhead
%     \hologoList
%   \end{longtable}
% \endgroup
%
% \begin{declcs}{Hologo} \M{name}
% \end{declcs}
% Macro \cs{Hologo} starts the logo \meta{name} with an uppercase
% letter. As an exception small greek letters are not converted
% to uppercase. Examples, see \hologo{eTeX} and \hologo{ExTeX}.
%
% \subsection{Setup macros}
%
% The package does not support package options, but the following
% setup macros can be used to set options.
%
% \begin{declcs}{hologoSetup} \M{key value list}
% \end{declcs}
% Macro \cs{hologoSetup} sets global options.
%
% \begin{declcs}{hologoLogoSetup} \M{logo} \M{key value list}
% \end{declcs}
% Some options can also be used to configure a logo.
% These settings take precedence over global option settings.
%
% \subsection{Options}\label{sec:options}
%
% There are boolean and string options:
% \begin{description}
% \item[Boolean option:]
% It takes |true| or |false|
% as value. If the value is omitted, then |true| is used.
% \item[String option:]
% A value must be given as string. (But the string might be empty.)
% \end{description}
% The following options can be used both in \cs{hologoSetup}
% and \cs{hologoLogoSetup}:
% \begin{description}
% \def\entry#1{\item[\xoption{#1}:]}
% \entry{break}
%   enables or disables line breaks inside the logo. This setting is
%   refined by options \xoption{hyphenbreak}, \xoption{spacebreak}
%   or \xoption{discretionarybreak}.
%   Default is |false|.
% \entry{hyphenbreak}
%   enables or disables the line break right after the hyphen character.
% \entry{spacebreak}
%   enables or disables line breaks at space characters.
% \entry{discretionarybreak}
%   enables or disables line breaks at hyphenation points
%   (inserted by \cs{-}).
% \end{description}
% Macro \cs{hologoLogoSetup} also knows:
% \begin{description}
% \item[\xoption{variant}:]
%   This is a string option. It specifies a variant of a logo that
%   must exist. An empty string selects the package default variant.
% \end{description}
% Example:
% \begin{quote}
%   |\hologoSetup{break=false}|\\
%   |\hologoLogoSetup{plainTeX}{variant=hyphen,hyphenbreak}|\\
%   Then ``plain-\TeX'' contains one break point after the hyphen.
% \end{quote}
%
% \subsection{Driver options}
%
% Sometimes graphical operations are needed to construct some
% glyphs (e.g.\ \hologo{XeTeX}). If package \xpackage{graphics}
% or package \xpackage{pgf} are found, then the macros are taken
% from there. Otherwise the packge defines its own operations
% and therefore needs the driver information. Many drivers are
% detected automatically (\hologo{pdfTeX}/\hologo{LuaTeX}
% in PDF mode, \hologo{XeTeX}, \hologo{VTeX}). These have precedence
% over a driver option. The driver can be given as package option
% or using \cs{hologoDriverSetup}.
% The following list contains the recognized driver options:
% \begin{itemize}
% \item \xoption{pdftex}, \xoption{luatex}
% \item \xoption{dvipdfm}, \xoption{dvipdfmx}
% \item \xoption{dvips}, \xoption{dvipsone}, \xoption{xdvi}
% \item \xoption{xetex}
% \item \xoption{vtex}
% \end{itemize}
% The left driver of a line is the driver name that is used internally.
% The following names are aliases for drivers that use the
% same method. Therefore the entry in the \xext{log} file for
% the used driver prints the internally used driver name.
% \begin{description}
% \item[\xoption{driverfallback}:]
%   This option expects a driver that is used,
%   if the driver could not be detected automatically.
% \end{description}
%
% \begin{declcs}{hologoDriverSetup} \M{driver option}
% \end{declcs}
% The driver can also be configured after package loading
% using \cs{hologoDriverSetup}, also the way for \hologo{plainTeX}
% to setup the driver.
%
% \subsection{Font setup}
%
% Some logos require a special font, but should also be usable by
% \hologo{plainTeX}. Therefore the package provides some ways
% to influence the font settings. The options below
% take font settings as values. Both font commands
% such as \cs{sffamily} and macros that take one argument
% like \cs{textsf} can be used.
%
% \begin{declcs}{hologoFontSetup} \M{key value list}
% \end{declcs}
% Macro \cs{hologoFontSetup} sets the fonts for all logos.
% Supported keys:
% \begin{description}
% \def\entry#1{\item[\xoption{#1}:]}
% \entry{general}
%   This font is used for all logos. The default is empty.
%   That means no special font is used.
% \entry{bibsf}
%   This font is used for
%   {\hologoLogoSetup{BibTeX}{variant=sf}\hologo{BibTeX}}
%   with variant \xoption{sf}.
% \entry{rm}
%   This font is a serif font. It is used for \hologo{ExTeX}.
% \entry{sc}
%   This font specifies a small caps font. It is used for
%   {\hologoLogoSetup{BibTeX}{variant=sc}\hologo{BibTeX}}
%   with variant \xoption{sc}.
% \entry{sf}
%   This font specifies a sans serif font. The default
%   is \cs{sffamily}, then \cs{sf} is tried. Otherwise
%   a warning is given. It is used by \hologo{KOMAScript}.
% \entry{sy}
%   This is the font for math symbols (e.g. cmsy).
%   It is used by \hologo{AmS}, \hologo{NTS}, \hologo{ExTeX}.
% \entry{logo}
%   \hologo{METAFONT} and \hologo{METAPOST} are using that font.
%   In \hologo{LaTeX} \cs{logofamily} is used and
%   the definitions of package \xpackage{mflogo} are used
%   if the package is not loaded.
%   Otherwise the \cs{tenlogo} is used and defined
%   if it does not already exists.
% \end{description}
%
% \begin{declcs}{hologoLogoFontSetup} \M{logo} \M{key value list}
% \end{declcs}
% Fonts can also be set for a logo or logo component separately,
% see the following list.
% The keys are the same as for \cs{hologoFontSetup}.
%
% \begin{longtable}{>{\ttfamily}l>{\sffamily}ll}
%   \meta{logo} & keys & result\\
%   \hline
%   \endhead
%   BibTeX & bibsf & {\hologoLogoSetup{BibTeX}{variant=sf}\hologo{BibTeX}}\\[.5ex]
%   BibTeX & sc & {\hologoLogoSetup{BibTeX}{variant=sc}\hologo{BibTeX}}\\[.5ex]
%   ExTeX & rm & \hologo{ExTeX}\\
%   SliTeX & rm & \hologo{SliTeX}\\[.5ex]
%   AmS & sy & \hologo{AmS}\\
%   ExTeX & sy & \hologo{ExTeX}\\
%   NTS & sy & \hologo{NTS}\\[.5ex]
%   KOMAScript & sf & \hologo{KOMAScript}\\[.5ex]
%   METAFONT & logo & \hologo{METAFONT}\\
%   METAPOST & logo & \hologo{METAPOST}\\[.5ex]
%   SliTeX & sc \hologo{SliTeX}
% \end{longtable}
%
% \subsubsection{Font order}
%
% For all logos the font \xoption{general} is applied first.
% Example:
%\begin{quote}
%|\hologoFontSetup{general=\color{red}}|
%\end{quote}
% will print red logos.
% Then if the font uses a special font \xoption{sf}, for example,
% the font is applied that is setup by \cs{hologoLogoFontSetup}.
% If this font is not setup, then the common font setup
% by \cs{hologoFontSetup} is used. Otherwise a warning is given,
% that there is no font configured.
%
% \subsection{Additional user macros}
%
% Usually a variant of a logo is configured by using
% \cs{hologoLogoSetup}, because it is bad style to mix
% different variants of the same logo in the same text.
% There the following macros are a convenience for testing.
%
% \begin{declcs}{hologoVariant} \M{name} \M{variant}\\
%   \cs{HologoVariant} \M{name} \M{variant}
% \end{declcs}
% Logo \meta{name} is set using \meta{variant} that specifies
% explicitely which variant of the macro is used. If the argument
% is empty, then the default form of the logo is used
% (configurable by \cs{hologoLogoSetup}).
%
% \cs{HologoVariant} is used if the logo is set in a context
% that needs an uppercase first letter (beginning of a sentence, \dots).
%
% \begin{declcs}{hologoList}\\
%   \cs{hologoEntry} \M{logo} \M{variant} \M{since}
% \end{declcs}
% Macro \cs{hologoList} contains all logos that are provided
% by the package including variants. The list consists of calls
% of \cs{hologoEntry} with three arguments starting with the
% logo name \meta{logo} and its variant \meta{variant}. An empty
% variant means the current default. Argument \meta{since} specifies
% with version of the package \xpackage{hologo} is needed to get
% the logo. If the logo is fixed, then the date gets updated.
% Therefore the date \meta{since} is not exactly the date of
% the first introduction, but rather the date of the latest fix.
%
% Before \cs{hologoList} can be used, macro \cs{hologoEntry} needs
% a definition. The example file in section \ref{sec:example}
% shows applications of \cs{hologoList}.
%
% \subsection{Supported contexts}
%
% Macros \cs{hologo} and friends support special contexts:
% \begin{itemize}
% \item \hologo{LaTeX}'s protection mechanism.
% \item Bookmarks of package \xpackage{hyperref}.
% \item Package \xpackage{tex4ht}.
% \item The macros can be used inside \cs{csname} constructs,
%   if \cs{ifincsname} is available (\hologo{pdfTeX}, \hologo{XeTeX},
%   \hologo{LuaTeX}).
% \end{itemize}
%
% \subsection{Example}
% \label{sec:example}
%
% The following example prints the logos in different fonts.
%    \begin{macrocode}
%<*example>
%<<verbatim
\NeedsTeXFormat{LaTeX2e}
\documentclass[a4paper]{article}
\usepackage[
  hmargin=20mm,
  vmargin=20mm,
]{geometry}
\pagestyle{empty}
\usepackage{hologo}[2016/05/12]
\usepackage{longtable}
\usepackage{array}
\setlength{\extrarowheight}{2pt}
\usepackage[T1]{fontenc}
\usepackage{lmodern}
\usepackage{pdflscape}
\usepackage[
  pdfencoding=auto,
]{hyperref}
\hypersetup{
  pdfauthor={Heiko Oberdiek},
  pdftitle={Example for package `hologo'},
  pdfsubject={Logos with fonts lmr, lmss, qtm, qpl, qhv},
}
\usepackage{bookmark}

% Print the logo list on the console

\begingroup
  \typeout{}%
  \typeout{*** Begin of logo list ***}%
  \newcommand*{\hologoEntry}[3]{%
    \typeout{#1 \ifx\\#2\\\else(#2) \fi[#3]}%
  }%
  \hologoList
  \typeout{*** End of logo list ***}%
  \typeout{}%
\endgroup

\begin{document}
\begin{landscape}

  \section{Example file for package `hologo'}

  % Table for font names

  \begin{longtable}{>{\bfseries}ll}
    \textbf{font} & \textbf{Font name}\\
    \hline
    lmr & Latin Modern Roman\\
    lmss & Latin Modern Sans\\
    qtm & \TeX\ Gyre Termes\\
    qhv & \TeX\ Gyre Heros\\
    qpl & \TeX\ Gyre Pagella\\
  \end{longtable}

  % Logo list with logos in different fonts

  \begingroup
    \newcommand*{\SetVariant}[2]{%
      \ifx\\#2\\%
      \else
        \hologoLogoSetup{#1}{variant=#2}%
      \fi
    }%
    \newcommand*{\hologoEntry}[3]{%
      \SetVariant{#1}{#2}%
      \raisebox{1em}[0pt][0pt]{\hypertarget{#1@#2}{}}%
      \bookmark[%
        dest={#1@#2},%
      ]{%
        #1\ifx\\#2\\\else\space(#2)\fi: \Hologo{#1}, \hologo{#1} %
        [Unicode]%
      }%
      \hypersetup{unicode=false}%
      \bookmark[%
        dest={#1@#2},%
      ]{%
        #1\ifx\\#2\\\else\space(#2)\fi: \Hologo{#1}, \hologo{#1} %
        [PDFDocEncoding]%
      }%
      \texttt{#1}%
      &%
      \texttt{#2}%
      &%
      \Hologo{#1}%
      &%
      \SetVariant{#1}{#2}%
      \hologo{#1}%
      &%
      \SetVariant{#1}{#2}%
      \fontfamily{qtm}\selectfont
      \hologo{#1}%
      &%
      \SetVariant{#1}{#2}%
      \fontfamily{qpl}\selectfont
      \hologo{#1}%
      &%
      \SetVariant{#1}{#2}%
      \textsf{\hologo{#1}}%
      &%
      \SetVariant{#1}{#2}%
      \fontfamily{qhv}\selectfont
      \hologo{#1}%
      \tabularnewline
    }%
    \begin{longtable}{llllllll}%
      \textbf{\textit{logo}} & \textbf{\textit{variant}} &
      \texttt{\string\Hologo} &
      \textbf{lmr} & \textbf{qtm} & \textbf{qpl} &
      \textbf{lmss} & \textbf{qhv}
      \tabularnewline
      \hline
      \endhead
      \hologoList
    \end{longtable}%
  \endgroup

\end{landscape}
\end{document}
%verbatim
%</example>
%    \end{macrocode}
%
% \StopEventually{
% }
%
% \section{Implementation}
%    \begin{macrocode}
%<*package>
%    \end{macrocode}
%    Reload check, especially if the package is not used with \LaTeX.
%    \begin{macrocode}
\begingroup\catcode61\catcode48\catcode32=10\relax%
  \catcode13=5 % ^^M
  \endlinechar=13 %
  \catcode35=6 % #
  \catcode39=12 % '
  \catcode44=12 % ,
  \catcode45=12 % -
  \catcode46=12 % .
  \catcode58=12 % :
  \catcode64=11 % @
  \catcode123=1 % {
  \catcode125=2 % }
  \expandafter\let\expandafter\x\csname ver@hologo.sty\endcsname
  \ifx\x\relax % plain-TeX, first loading
  \else
    \def\empty{}%
    \ifx\x\empty % LaTeX, first loading,
      % variable is initialized, but \ProvidesPackage not yet seen
    \else
      \expandafter\ifx\csname PackageInfo\endcsname\relax
        \def\x#1#2{%
          \immediate\write-1{Package #1 Info: #2.}%
        }%
      \else
        \def\x#1#2{\PackageInfo{#1}{#2, stopped}}%
      \fi
      \x{hologo}{The package is already loaded}%
      \aftergroup\endinput
    \fi
  \fi
\endgroup%
%    \end{macrocode}
%    Package identification:
%    \begin{macrocode}
\begingroup\catcode61\catcode48\catcode32=10\relax%
  \catcode13=5 % ^^M
  \endlinechar=13 %
  \catcode35=6 % #
  \catcode39=12 % '
  \catcode40=12 % (
  \catcode41=12 % )
  \catcode44=12 % ,
  \catcode45=12 % -
  \catcode46=12 % .
  \catcode47=12 % /
  \catcode58=12 % :
  \catcode64=11 % @
  \catcode91=12 % [
  \catcode93=12 % ]
  \catcode123=1 % {
  \catcode125=2 % }
  \expandafter\ifx\csname ProvidesPackage\endcsname\relax
    \def\x#1#2#3[#4]{\endgroup
      \immediate\write-1{Package: #3 #4}%
      \xdef#1{#4}%
    }%
  \else
    \def\x#1#2[#3]{\endgroup
      #2[{#3}]%
      \ifx#1\@undefined
        \xdef#1{#3}%
      \fi
      \ifx#1\relax
        \xdef#1{#3}%
      \fi
    }%
  \fi
\expandafter\x\csname ver@hologo.sty\endcsname
\ProvidesPackage{hologo}%
  [2016/05/12 v1.11 A logo collection with bookmark support (HO)]%
%    \end{macrocode}
%
%    \begin{macrocode}
\begingroup\catcode61\catcode48\catcode32=10\relax%
  \catcode13=5 % ^^M
  \endlinechar=13 %
  \catcode123=1 % {
  \catcode125=2 % }
  \catcode64=11 % @
  \def\x{\endgroup
    \expandafter\edef\csname HOLOGO@AtEnd\endcsname{%
      \endlinechar=\the\endlinechar\relax
      \catcode13=\the\catcode13\relax
      \catcode32=\the\catcode32\relax
      \catcode35=\the\catcode35\relax
      \catcode61=\the\catcode61\relax
      \catcode64=\the\catcode64\relax
      \catcode123=\the\catcode123\relax
      \catcode125=\the\catcode125\relax
    }%
  }%
\x\catcode61\catcode48\catcode32=10\relax%
\catcode13=5 % ^^M
\endlinechar=13 %
\catcode35=6 % #
\catcode64=11 % @
\catcode123=1 % {
\catcode125=2 % }
\def\TMP@EnsureCode#1#2{%
  \edef\HOLOGO@AtEnd{%
    \HOLOGO@AtEnd
    \catcode#1=\the\catcode#1\relax
  }%
  \catcode#1=#2\relax
}
\TMP@EnsureCode{10}{12}% ^^J
\TMP@EnsureCode{33}{12}% !
\TMP@EnsureCode{34}{12}% "
\TMP@EnsureCode{36}{3}% $
\TMP@EnsureCode{38}{4}% &
\TMP@EnsureCode{39}{12}% '
\TMP@EnsureCode{40}{12}% (
\TMP@EnsureCode{41}{12}% )
\TMP@EnsureCode{42}{12}% *
\TMP@EnsureCode{43}{12}% +
\TMP@EnsureCode{44}{12}% ,
\TMP@EnsureCode{45}{12}% -
\TMP@EnsureCode{46}{12}% .
\TMP@EnsureCode{47}{12}% /
\TMP@EnsureCode{58}{12}% :
\TMP@EnsureCode{59}{12}% ;
\TMP@EnsureCode{60}{12}% <
\TMP@EnsureCode{62}{12}% >
\TMP@EnsureCode{63}{12}% ?
\TMP@EnsureCode{91}{12}% [
\TMP@EnsureCode{93}{12}% ]
\TMP@EnsureCode{94}{7}% ^ (superscript)
\TMP@EnsureCode{95}{8}% _ (subscript)
\TMP@EnsureCode{96}{12}% `
\TMP@EnsureCode{124}{12}% |
\edef\HOLOGO@AtEnd{%
  \HOLOGO@AtEnd
  \escapechar\the\escapechar\relax
  \noexpand\endinput
}
\escapechar=92 %
%    \end{macrocode}
%
% \subsection{Logo list}
%
%    \begin{macro}{\hologoList}
%    \begin{macrocode}
\def\hologoList{%
  \hologoEntry{(La)TeX}{}{2011/10/01}%
  \hologoEntry{AmSLaTeX}{}{2010/04/16}%
  \hologoEntry{AmSTeX}{}{2010/04/16}%
  \hologoEntry{biber}{}{2011/10/01}%
  \hologoEntry{BibTeX}{}{2011/10/01}%
  \hologoEntry{BibTeX}{sf}{2011/10/01}%
  \hologoEntry{BibTeX}{sc}{2011/10/01}%
  \hologoEntry{BibTeX8}{}{2011/11/22}%
  \hologoEntry{ConTeXt}{}{2011/03/25}%
  \hologoEntry{ConTeXt}{narrow}{2011/03/25}%
  \hologoEntry{ConTeXt}{simple}{2011/03/25}%
  \hologoEntry{emTeX}{}{2010/04/26}%
  \hologoEntry{eTeX}{}{2010/04/08}%
  \hologoEntry{ExTeX}{}{2011/10/01}%
  \hologoEntry{HanTheThanh}{}{2011/11/29}%
  \hologoEntry{iniTeX}{}{2011/10/01}%
  \hologoEntry{KOMAScript}{}{2011/10/01}%
  \hologoEntry{La}{}{2010/05/08}%
  \hologoEntry{LaTeX}{}{2010/04/08}%
  \hologoEntry{LaTeX2e}{}{2010/04/08}%
  \hologoEntry{LaTeX3}{}{2010/04/24}%
  \hologoEntry{LaTeXe}{}{2010/04/08}%
  \hologoEntry{LaTeXML}{}{2011/11/22}%
  \hologoEntry{LaTeXTeX}{}{2011/10/01}%
  \hologoEntry{LuaLaTeX}{}{2010/04/08}%
  \hologoEntry{LuaTeX}{}{2010/04/08}%
  \hologoEntry{LyX}{}{2011/10/01}%
  \hologoEntry{METAFONT}{}{2011/10/01}%
  \hologoEntry{MetaFun}{}{2011/10/01}%
  \hologoEntry{METAPOST}{}{2011/10/01}%
  \hologoEntry{MetaPost}{}{2011/10/01}%
  \hologoEntry{MiKTeX}{}{2011/10/01}%
  \hologoEntry{NTS}{}{2011/10/01}%
  \hologoEntry{OzMF}{}{2011/10/01}%
  \hologoEntry{OzMP}{}{2011/10/01}%
  \hologoEntry{OzTeX}{}{2011/10/01}%
  \hologoEntry{OzTtH}{}{2011/10/01}%
  \hologoEntry{PCTeX}{}{2011/10/01}%
  \hologoEntry{pdfTeX}{}{2011/10/01}%
  \hologoEntry{pdfLaTeX}{}{2011/10/01}%
  \hologoEntry{PiC}{}{2011/10/01}%
  \hologoEntry{PiCTeX}{}{2011/10/01}%
  \hologoEntry{plainTeX}{}{2010/04/08}%
  \hologoEntry{plainTeX}{space}{2010/04/16}%
  \hologoEntry{plainTeX}{hyphen}{2010/04/16}%
  \hologoEntry{plainTeX}{runtogether}{2010/04/16}%
  \hologoEntry{SageTeX}{}{2011/11/22}%
  \hologoEntry{SLiTeX}{}{2011/10/01}%
  \hologoEntry{SLiTeX}{lift}{2011/10/01}%
  \hologoEntry{SLiTeX}{narrow}{2011/10/01}%
  \hologoEntry{SLiTeX}{simple}{2011/10/01}%
  \hologoEntry{SliTeX}{}{2011/10/01}%
  \hologoEntry{SliTeX}{narrow}{2011/10/01}%
  \hologoEntry{SliTeX}{simple}{2011/10/01}%
  \hologoEntry{SliTeX}{lift}{2011/10/01}%
  \hologoEntry{teTeX}{}{2011/10/01}%
  \hologoEntry{TeX}{}{2010/04/08}%
  \hologoEntry{TeX4ht}{}{2011/11/22}%
  \hologoEntry{TTH}{}{2011/11/22}%
  \hologoEntry{virTeX}{}{2011/10/01}%
  \hologoEntry{VTeX}{}{2010/04/24}%
  \hologoEntry{Xe}{}{2010/04/08}%
  \hologoEntry{XeLaTeX}{}{2010/04/08}%
  \hologoEntry{XeTeX}{}{2010/04/08}%
}
%    \end{macrocode}
%    \end{macro}
%
% \subsection{Load resources}
%
%    \begin{macrocode}
\begingroup\expandafter\expandafter\expandafter\endgroup
\expandafter\ifx\csname RequirePackage\endcsname\relax
  \def\TMP@RequirePackage#1[#2]{%
    \begingroup\expandafter\expandafter\expandafter\endgroup
    \expandafter\ifx\csname ver@#1.sty\endcsname\relax
      \input #1.sty\relax
    \fi
  }%
  \TMP@RequirePackage{ltxcmds}[2011/02/04]%
  \TMP@RequirePackage{infwarerr}[2010/04/08]%
  \TMP@RequirePackage{kvsetkeys}[2010/03/01]%
  \TMP@RequirePackage{kvdefinekeys}[2010/03/01]%
  \TMP@RequirePackage{pdftexcmds}[2010/04/01]%
  \TMP@RequirePackage{ifpdf}[2010/01/28]%
  \TMP@RequirePackage{ifluatex}[2010/03/01]%
  \ltx@IfUndefined{newif}{%
    \expandafter\let\csname newif\endcsname\ltx@newif
  }{}%
  \TMP@RequirePackage{ifxetex}[2009/01/23]%
  \TMP@RequirePackage{ifvtex}[2010/03/01]%
\else
  \RequirePackage{ltxcmds}[2011/02/04]%
  \RequirePackage{infwarerr}[2010/04/08]%
  \RequirePackage{kvsetkeys}[2010/03/01]%
  \RequirePackage{kvdefinekeys}[2010/03/01]%
  \RequirePackage{pdftexcmds}[2010/04/01]%
  \RequirePackage{ifpdf}[2010/01/28]%
  \RequirePackage{ifluatex}[2010/03/01]%
  \RequirePackage{ifxetex}[2009/01/23]%
  \RequirePackage{ifvtex}[2010/03/01]%
\fi
%    \end{macrocode}
%
%    \begin{macro}{\HOLOGO@IfDefined}
%    \begin{macrocode}
\def\HOLOGO@IfExists#1{%
  \ifx\@undefined#1%
    \expandafter\ltx@secondoftwo
  \else
    \ifx\relax#1%
      \expandafter\ltx@secondoftwo
    \else
      \expandafter\expandafter\expandafter\ltx@firstoftwo
    \fi
  \fi
}
%    \end{macrocode}
%    \end{macro}
%
% \subsection{Setup macros}
%
%    \begin{macro}{\hologoSetup}
%    \begin{macrocode}
\def\hologoSetup{%
  \let\HOLOGO@name\relax
  \HOLOGO@Setup
}
%    \end{macrocode}
%    \end{macro}
%
%    \begin{macro}{\hologoLogoSetup}
%    \begin{macrocode}
\def\hologoLogoSetup#1{%
  \edef\HOLOGO@name{#1}%
  \ltx@IfUndefined{HoLogo@\HOLOGO@name}{%
    \@PackageError{hologo}{%
      Unknown logo `\HOLOGO@name'%
    }\@ehc
    \ltx@gobble
  }{%
    \HOLOGO@Setup
  }%
}
%    \end{macrocode}
%    \end{macro}
%
%    \begin{macro}{\HOLOGO@Setup}
%    \begin{macrocode}
\def\HOLOGO@Setup{%
  \kvsetkeys{HoLogo}%
}
%    \end{macrocode}
%    \end{macro}
%
% \subsection{Options}
%
%    \begin{macro}{\HOLOGO@DeclareBoolOption}
%    \begin{macrocode}
\def\HOLOGO@DeclareBoolOption#1{%
  \expandafter\chardef\csname HOLOGOOPT@#1\endcsname\ltx@zero
  \kv@define@key{HoLogo}{#1}[true]{%
    \def\HOLOGO@temp{##1}%
    \ifx\HOLOGO@temp\HOLOGO@true
      \ifx\HOLOGO@name\relax
        \expandafter\chardef\csname HOLOGOOPT@#1\endcsname=\ltx@one
      \else
        \expandafter\chardef\csname
        HoLogoOpt@#1@\HOLOGO@name\endcsname\ltx@one
      \fi
      \HOLOGO@SetBreakAll{#1}%
    \else
      \ifx\HOLOGO@temp\HOLOGO@false
        \ifx\HOLOGO@name\relax
          \expandafter\chardef\csname HOLOGOOPT@#1\endcsname=\ltx@zero
        \else
          \expandafter\chardef\csname
          HoLogoOpt@#1@\HOLOGO@name\endcsname=\ltx@zero
        \fi
        \HOLOGO@SetBreakAll{#1}%
      \else
        \@PackageError{hologo}{%
          Unknown value `##1' for boolean option `#1'.\MessageBreak
          Known values are `true' and `false'%
        }\@ehc
      \fi
    \fi
  }%
}
%    \end{macrocode}
%    \end{macro}
%
%    \begin{macro}{\HOLOGO@SetBreakAll}
%    \begin{macrocode}
\def\HOLOGO@SetBreakAll#1{%
  \def\HOLOGO@temp{#1}%
  \ifx\HOLOGO@temp\HOLOGO@break
    \ifx\HOLOGO@name\relax
      \chardef\HOLOGOOPT@hyphenbreak=\HOLOGOOPT@break
      \chardef\HOLOGOOPT@spacebreak=\HOLOGOOPT@break
      \chardef\HOLOGOOPT@discretionarybreak=\HOLOGOOPT@break
    \else
      \expandafter\chardef
         \csname HoLogoOpt@hyphenbreak@\HOLOGO@name\endcsname=%
         \csname HoLogoOpt@break@\HOLOGO@name\endcsname
      \expandafter\chardef
         \csname HoLogoOpt@spacebreak@\HOLOGO@name\endcsname=%
         \csname HoLogoOpt@break@\HOLOGO@name\endcsname
      \expandafter\chardef
         \csname HoLogoOpt@discretionarybreak@\HOLOGO@name
             \endcsname=%
         \csname HoLogoOpt@break@\HOLOGO@name\endcsname
    \fi
  \fi
}
%    \end{macrocode}
%    \end{macro}
%
%    \begin{macro}{\HOLOGO@true}
%    \begin{macrocode}
\def\HOLOGO@true{true}
%    \end{macrocode}
%    \end{macro}
%    \begin{macro}{\HOLOGO@false}
%    \begin{macrocode}
\def\HOLOGO@false{false}
%    \end{macrocode}
%    \end{macro}
%    \begin{macro}{\HOLOGO@break}
%    \begin{macrocode}
\def\HOLOGO@break{break}
%    \end{macrocode}
%    \end{macro}
%
%    \begin{macrocode}
\HOLOGO@DeclareBoolOption{break}
\HOLOGO@DeclareBoolOption{hyphenbreak}
\HOLOGO@DeclareBoolOption{spacebreak}
\HOLOGO@DeclareBoolOption{discretionarybreak}
%    \end{macrocode}
%
%    \begin{macrocode}
\kv@define@key{HoLogo}{variant}{%
  \ifx\HOLOGO@name\relax
    \@PackageError{hologo}{%
      Option `variant' is not available in \string\hologoSetup,%
      \MessageBreak
      Use \string\hologoLogoSetup\space instead%
    }\@ehc
  \else
    \edef\HOLOGO@temp{#1}%
    \ifx\HOLOGO@temp\ltx@empty
      \expandafter
      \let\csname HoLogoOpt@variant@\HOLOGO@name\endcsname\@undefined
    \else
      \ltx@IfUndefined{HoLogo@\HOLOGO@name @\HOLOGO@temp}{%
        \@PackageError{hologo}{%
          Unknown variant `\HOLOGO@temp' of logo `\HOLOGO@name'%
        }\@ehc
      }{%
        \expandafter
        \let\csname HoLogoOpt@variant@\HOLOGO@name\endcsname
            \HOLOGO@temp
      }%
    \fi
  \fi
}
%    \end{macrocode}
%
%    \begin{macro}{\HOLOGO@Variant}
%    \begin{macrocode}
\def\HOLOGO@Variant#1{%
  #1%
  \ltx@ifundefined{HoLogoOpt@variant@#1}{%
  }{%
    @\csname HoLogoOpt@variant@#1\endcsname
  }%
}
%    \end{macrocode}
%    \end{macro}
%
% \subsection{Break/no-break support}
%
%    \begin{macro}{\HOLOGO@space}
%    \begin{macrocode}
\def\HOLOGO@space{%
  \ltx@ifundefined{HoLogoOpt@spacebreak@\HOLOGO@name}{%
    \ltx@ifundefined{HoLogoOpt@break@\HOLOGO@name}{%
      \chardef\HOLOGO@temp=\HOLOGOOPT@spacebreak
    }{%
      \chardef\HOLOGO@temp=%
        \csname HoLogoOpt@break@\HOLOGO@name\endcsname
    }%
  }{%
    \chardef\HOLOGO@temp=%
      \csname HoLogoOpt@spacebreak@\HOLOGO@name\endcsname
  }%
  \ifcase\HOLOGO@temp
    \penalty10000 %
  \fi
  \ltx@space
}
%    \end{macrocode}
%    \end{macro}
%
%    \begin{macro}{\HOLOGO@hyphen}
%    \begin{macrocode}
\def\HOLOGO@hyphen{%
  \ltx@ifundefined{HoLogoOpt@hyphenbreak@\HOLOGO@name}{%
    \ltx@ifundefined{HoLogoOpt@break@\HOLOGO@name}{%
      \chardef\HOLOGO@temp=\HOLOGOOPT@hyphenbreak
    }{%
      \chardef\HOLOGO@temp=%
        \csname HoLogoOpt@break@\HOLOGO@name\endcsname
    }%
  }{%
    \chardef\HOLOGO@temp=%
      \csname HoLogoOpt@hyphenbreak@\HOLOGO@name\endcsname
  }%
  \ifcase\HOLOGO@temp
    \ltx@mbox{-}%
  \else
    -%
  \fi
}
%    \end{macrocode}
%    \end{macro}
%
%    \begin{macro}{\HOLOGO@discretionary}
%    \begin{macrocode}
\def\HOLOGO@discretionary{%
  \ltx@ifundefined{HoLogoOpt@discretionarybreak@\HOLOGO@name}{%
    \ltx@ifundefined{HoLogoOpt@break@\HOLOGO@name}{%
      \chardef\HOLOGO@temp=\HOLOGOOPT@discretionarybreak
    }{%
      \chardef\HOLOGO@temp=%
        \csname HoLogoOpt@break@\HOLOGO@name\endcsname
    }%
  }{%
    \chardef\HOLOGO@temp=%
      \csname HoLogoOpt@discretionarybreak@\HOLOGO@name\endcsname
  }%
  \ifcase\HOLOGO@temp
  \else
    \-%
  \fi
}
%    \end{macrocode}
%    \end{macro}
%
%    \begin{macro}{\HOLOGO@mbox}
%    \begin{macrocode}
\def\HOLOGO@mbox#1{%
  \ltx@ifundefined{HoLogoOpt@break@\HOLOGO@name}{%
    \chardef\HOLOGO@temp=\HOLOGOOPT@hyphenbreak
  }{%
    \chardef\HOLOGO@temp=%
      \csname HoLogoOpt@break@\HOLOGO@name\endcsname
  }%
  \ifcase\HOLOGO@temp
    \ltx@mbox{#1}%
  \else
    #1%
  \fi
}
%    \end{macrocode}
%    \end{macro}
%
% \subsection{Font support}
%
%    \begin{macro}{\HoLogoFont@font}
%    \begin{tabular}{@{}ll@{}}
%    |#1|:& logo name\\
%    |#2|:& font short name\\
%    |#3|:& text
%    \end{tabular}
%    \begin{macrocode}
\def\HoLogoFont@font#1#2#3{%
  \begingroup
    \ltx@IfUndefined{HoLogoFont@logo@#1.#2}{%
      \ltx@IfUndefined{HoLogoFont@font@#2}{%
        \@PackageWarning{hologo}{%
          Missing font `#2' for logo `#1'%
        }%
        #3%
      }{%
        \csname HoLogoFont@font@#2\endcsname{#3}%
      }%
    }{%
      \csname HoLogoFont@logo@#1.#2\endcsname{#3}%
    }%
  \endgroup
}
%    \end{macrocode}
%    \end{macro}
%
%    \begin{macro}{\HoLogoFont@Def}
%    \begin{macrocode}
\def\HoLogoFont@Def#1{%
  \expandafter\def\csname HoLogoFont@font@#1\endcsname
}
%    \end{macrocode}
%    \end{macro}
%    \begin{macro}{\HoLogoFont@LogoDef}
%    \begin{macrocode}
\def\HoLogoFont@LogoDef#1#2{%
  \expandafter\def\csname HoLogoFont@logo@#1.#2\endcsname
}
%    \end{macrocode}
%    \end{macro}
%
% \subsubsection{Font defaults}
%
%    \begin{macro}{\HoLogoFont@font@general}
%    \begin{macrocode}
\HoLogoFont@Def{general}{}%
%    \end{macrocode}
%    \end{macro}
%
%    \begin{macro}{\HoLogoFont@font@rm}
%    \begin{macrocode}
\ltx@IfUndefined{rmfamily}{%
  \ltx@IfUndefined{rm}{%
  }{%
    \HoLogoFont@Def{rm}{\rm}%
  }%
}{%
  \HoLogoFont@Def{rm}{\rmfamily}%
}
%    \end{macrocode}
%    \end{macro}
%
%    \begin{macro}{\HoLogoFont@font@sf}
%    \begin{macrocode}
\ltx@IfUndefined{sffamily}{%
  \ltx@IfUndefined{sf}{%
  }{%
    \HoLogoFont@Def{sf}{\sf}%
  }%
}{%
  \HoLogoFont@Def{sf}{\sffamily}%
}
%    \end{macrocode}
%    \end{macro}
%
%    \begin{macro}{\HoLogoFont@font@bibsf}
%    In case of \hologo{plainTeX} the original small caps
%    variant is used as default. In \hologo{LaTeX}
%    the definition of package \xpackage{dtklogos} \cite{dtklogos}
%    is used.
%\begin{quote}
%\begin{verbatim}
%\DeclareRobustCommand{\BibTeX}{%
%  B%
%  \kern-.05em%
%  \hbox{%
%    $\m@th$% %% force math size calculations
%    \csname S@\f@size\endcsname
%    \fontsize\sf@size\z@
%    \math@fontsfalse
%    \selectfont
%    I%
%    \kern-.025em%
%    B
%  }%
%  \kern-.08em%
%  \-%
%  \TeX
%}
%\end{verbatim}
%\end{quote}
%    \begin{macrocode}
\ltx@IfUndefined{selectfont}{%
  \ltx@IfUndefined{tensc}{%
    \font\tensc=cmcsc10\relax
  }{}%
  \HoLogoFont@Def{bibsf}{\tensc}%
}{%
  \HoLogoFont@Def{bibsf}{%
    $\mathsurround=0pt$%
    \csname S@\f@size\endcsname
    \fontsize\sf@size{0pt}%
    \math@fontsfalse
    \selectfont
  }%
}
%    \end{macrocode}
%    \end{macro}
%
%    \begin{macro}{\HoLogoFont@font@sc}
%    \begin{macrocode}
\ltx@IfUndefined{scshape}{%
  \ltx@IfUndefined{tensc}{%
    \font\tensc=cmcsc10\relax
  }{}%
  \HoLogoFont@Def{sc}{\tensc}%
}{%
  \HoLogoFont@Def{sc}{\scshape}%
}
%    \end{macrocode}
%    \end{macro}
%
%    \begin{macro}{\HoLogoFont@font@sy}
%    \begin{macrocode}
\ltx@IfUndefined{usefont}{%
  \ltx@IfUndefined{tensy}{%
  }{%
    \HoLogoFont@Def{sy}{\tensy}%
  }%
}{%
  \HoLogoFont@Def{sy}{%
    \usefont{OMS}{cmsy}{m}{n}%
  }%
}
%    \end{macrocode}
%    \end{macro}
%
%    \begin{macro}{\HoLogoFont@font@logo}
%    \begin{macrocode}
\begingroup
  \def\x{LaTeX2e}%
\expandafter\endgroup
\ifx\fmtname\x
  \ltx@IfUndefined{logofamily}{%
    \DeclareRobustCommand\logofamily{%
      \not@math@alphabet\logofamily\relax
      \fontencoding{U}%
      \fontfamily{logo}%
      \selectfont
    }%
  }{}%
  \ltx@IfUndefined{logofamily}{%
  }{%
    \HoLogoFont@Def{logo}{\logofamily}%
  }%
\else
  \ltx@IfUndefined{tenlogo}{%
    \font\tenlogo=logo10\relax
  }{}%
  \HoLogoFont@Def{logo}{\tenlogo}%
\fi
%    \end{macrocode}
%    \end{macro}
%
% \subsubsection{Font setup}
%
%    \begin{macro}{\hologoFontSetup}
%    \begin{macrocode}
\def\hologoFontSetup{%
  \let\HOLOGO@name\relax
  \HOLOGO@FontSetup
}
%    \end{macrocode}
%    \end{macro}
%
%    \begin{macro}{\hologoLogoFontSetup}
%    \begin{macrocode}
\def\hologoLogoFontSetup#1{%
  \edef\HOLOGO@name{#1}%
  \ltx@IfUndefined{HoLogo@\HOLOGO@name}{%
    \@PackageError{hologo}{%
      Unknown logo `\HOLOGO@name'%
    }\@ehc
    \ltx@gobble
  }{%
    \HOLOGO@FontSetup
  }%
}
%    \end{macrocode}
%    \end{macro}
%
%    \begin{macro}{\HOLOGO@FontSetup}
%    \begin{macrocode}
\def\HOLOGO@FontSetup{%
  \kvsetkeys{HoLogoFont}%
}
%    \end{macrocode}
%    \end{macro}
%
%    \begin{macrocode}
\def\HOLOGO@temp#1{%
  \kv@define@key{HoLogoFont}{#1}{%
    \ifx\HOLOGO@name\relax
      \HoLogoFont@Def{#1}{##1}%
    \else
      \HoLogoFont@LogoDef\HOLOGO@name{#1}{##1}%
    \fi
  }%
}
\HOLOGO@temp{general}
\HOLOGO@temp{sf}
%    \end{macrocode}
%
% \subsection{Generic logo commands}
%
%    \begin{macrocode}
\HOLOGO@IfExists\hologo{%
  \@PackageError{hologo}{%
    \string\hologo\ltx@space is already defined.\MessageBreak
    Package loading is aborted%
  }\@ehc
  \HOLOGO@AtEnd
}%
\HOLOGO@IfExists\hologoRobust{%
  \@PackageError{hologo}{%
    \string\hologoRobust\ltx@space is already defined.\MessageBreak
    Package loading is aborted%
  }\@ehc
  \HOLOGO@AtEnd
}%
%    \end{macrocode}
%
% \subsubsection{\cs{hologo} and friends}
%
%    \begin{macrocode}
\ifluatex
  \expandafter\ltx@firstofone
\else
  \expandafter\ltx@gobble
\fi
{%
  \ltx@IfUndefined{ifincsname}{%
    \ifnum\luatexversion<36 %
      \expandafter\ltx@gobble
    \else
      \expandafter\ltx@firstofone
    \fi
    {%
      \begingroup
        \ifcase0%
            \directlua{%
              if tex.enableprimitives then %
                tex.enableprimitives('HOLOGO@', {'ifincsname'})%
              else %
                tex.print('1')%
              end%
            }%
            \ifx\HOLOGO@ifincsname\@undefined 1\fi%
            \relax
          \expandafter\ltx@firstofone
        \else
          \endgroup
          \expandafter\ltx@gobble
        \fi
        {%
          \global\let\ifincsname\HOLOGO@ifincsname
        }%
      \HOLOGO@temp
    }%
  }{}%
}
%    \end{macrocode}
%    \begin{macrocode}
\ltx@IfUndefined{ifincsname}{%
  \catcode`$=14 %
}{%
  \catcode`$=9 %
}
%    \end{macrocode}
%
%    \begin{macro}{\hologo}
%    \begin{macrocode}
\def\hologo#1{%
$ \ifincsname
$   \ltx@ifundefined{HoLogoCs@\HOLOGO@Variant{#1}}{%
$     #1%
$   }{%
$     \csname HoLogoCs@\HOLOGO@Variant{#1}\endcsname\ltx@firstoftwo
$   }%
$ \else
    \HOLOGO@IfExists\texorpdfstring\texorpdfstring\ltx@firstoftwo
    {%
      \hologoRobust{#1}%
    }{%
      \ltx@ifundefined{HoLogoBkm@\HOLOGO@Variant{#1}}{%
        \ltx@ifundefined{HoLogo@#1}{?#1?}{#1}%
      }{%
        \csname HoLogoBkm@\HOLOGO@Variant{#1}\endcsname
        \ltx@firstoftwo
      }%
    }%
$ \fi
}
%    \end{macrocode}
%    \end{macro}
%    \begin{macro}{\Hologo}
%    \begin{macrocode}
\def\Hologo#1{%
$ \ifincsname
$   \ltx@ifundefined{HoLogoCs@\HOLOGO@Variant{#1}}{%
$     #1%
$   }{%
$     \csname HoLogoCs@\HOLOGO@Variant{#1}\endcsname\ltx@secondoftwo
$   }%
$ \else
    \HOLOGO@IfExists\texorpdfstring\texorpdfstring\ltx@firstoftwo
    {%
      \HologoRobust{#1}%
    }{%
      \ltx@ifundefined{HoLogoBkm@\HOLOGO@Variant{#1}}{%
        \ltx@ifundefined{HoLogo@#1}{?#1?}{#1}%
      }{%
        \csname HoLogoBkm@\HOLOGO@Variant{#1}\endcsname
        \ltx@secondoftwo
      }%
    }%
$ \fi
}
%    \end{macrocode}
%    \end{macro}
%
%    \begin{macro}{\hologoVariant}
%    \begin{macrocode}
\def\hologoVariant#1#2{%
  \ifx\relax#2\relax
    \hologo{#1}%
  \else
$   \ifincsname
$     \ltx@ifundefined{HoLogoCs@#1@#2}{%
$       #1%
$     }{%
$       \csname HoLogoCs@#1@#2\endcsname\ltx@firstoftwo
$     }%
$   \else
      \HOLOGO@IfExists\texorpdfstring\texorpdfstring\ltx@firstoftwo
      {%
        \hologoVariantRobust{#1}{#2}%
      }{%
        \ltx@ifundefined{HoLogoBkm@#1@#2}{%
          \ltx@ifundefined{HoLogo@#1}{?#1?}{#1}%
        }{%
          \csname HoLogoBkm@#1@#2\endcsname
          \ltx@firstoftwo
        }%
      }%
$   \fi
  \fi
}
%    \end{macrocode}
%    \end{macro}
%    \begin{macro}{\HologoVariant}
%    \begin{macrocode}
\def\HologoVariant#1#2{%
  \ifx\relax#2\relax
    \Hologo{#1}%
  \else
$   \ifincsname
$     \ltx@ifundefined{HoLogoCs@#1@#2}{%
$       #1%
$     }{%
$       \csname HoLogoCs@#1@#2\endcsname\ltx@secondoftwo
$     }%
$   \else
      \HOLOGO@IfExists\texorpdfstring\texorpdfstring\ltx@firstoftwo
      {%
        \HologoVariantRobust{#1}{#2}%
      }{%
        \ltx@ifundefined{HoLogoBkm@#1@#2}{%
          \ltx@ifundefined{HoLogo@#1}{?#1?}{#1}%
        }{%
          \csname HoLogoBkm@#1@#2\endcsname
          \ltx@secondoftwo
        }%
      }%
$   \fi
  \fi
}
%    \end{macrocode}
%    \end{macro}
%
%    \begin{macrocode}
\catcode`\$=3 %
%    \end{macrocode}
%
% \subsubsection{\cs{hologoRobust} and friends}
%
%    \begin{macro}{\hologoRobust}
%    \begin{macrocode}
\ltx@IfUndefined{protected}{%
  \ltx@IfUndefined{DeclareRobustCommand}{%
    \def\hologoRobust#1%
  }{%
    \DeclareRobustCommand*\hologoRobust[1]%
  }%
}{%
  \protected\def\hologoRobust#1%
}%
{%
  \edef\HOLOGO@name{#1}%
  \ltx@IfUndefined{HoLogo@\HOLOGO@Variant\HOLOGO@name}{%
    \@PackageError{hologo}{%
      Unknown logo `\HOLOGO@name'%
    }\@ehc
    ?\HOLOGO@name?%
  }{%
    \ltx@IfUndefined{ver@tex4ht.sty}{%
      \HoLogoFont@font\HOLOGO@name{general}{%
        \csname HoLogo@\HOLOGO@Variant\HOLOGO@name\endcsname
        \ltx@firstoftwo
      }%
    }{%
      \ltx@IfUndefined{HoLogoHtml@\HOLOGO@Variant\HOLOGO@name}{%
        \HOLOGO@name
      }{%
        \csname HoLogoHtml@\HOLOGO@Variant\HOLOGO@name\endcsname
        \ltx@firstoftwo
      }%
    }%
  }%
}
%    \end{macrocode}
%    \end{macro}
%    \begin{macro}{\HologoRobust}
%    \begin{macrocode}
\ltx@IfUndefined{protected}{%
  \ltx@IfUndefined{DeclareRobustCommand}{%
    \def\HologoRobust#1%
  }{%
    \DeclareRobustCommand*\HologoRobust[1]%
  }%
}{%
  \protected\def\HologoRobust#1%
}%
{%
  \edef\HOLOGO@name{#1}%
  \ltx@IfUndefined{HoLogo@\HOLOGO@Variant\HOLOGO@name}{%
    \@PackageError{hologo}{%
      Unknown logo `\HOLOGO@name'%
    }\@ehc
    ?\HOLOGO@name?%
  }{%
    \ltx@IfUndefined{ver@tex4ht.sty}{%
      \HoLogoFont@font\HOLOGO@name{general}{%
        \csname HoLogo@\HOLOGO@Variant\HOLOGO@name\endcsname
        \ltx@secondoftwo
      }%
    }{%
      \ltx@IfUndefined{HoLogoHtml@\HOLOGO@Variant\HOLOGO@name}{%
        \expandafter\HOLOGO@Uppercase\HOLOGO@name
      }{%
        \csname HoLogoHtml@\HOLOGO@Variant\HOLOGO@name\endcsname
        \ltx@secondoftwo
      }%
    }%
  }%
}
%    \end{macrocode}
%    \end{macro}
%    \begin{macro}{\hologoVariantRobust}
%    \begin{macrocode}
\ltx@IfUndefined{protected}{%
  \ltx@IfUndefined{DeclareRobustCommand}{%
    \def\hologoVariantRobust#1#2%
  }{%
    \DeclareRobustCommand*\hologoVariantRobust[2]%
  }%
}{%
  \protected\def\hologoVariantRobust#1#2%
}%
{%
  \begingroup
    \hologoLogoSetup{#1}{variant={#2}}%
    \hologoRobust{#1}%
  \endgroup
}
%    \end{macrocode}
%    \end{macro}
%    \begin{macro}{\HologoVariantRobust}
%    \begin{macrocode}
\ltx@IfUndefined{protected}{%
  \ltx@IfUndefined{DeclareRobustCommand}{%
    \def\HologoVariantRobust#1#2%
  }{%
    \DeclareRobustCommand*\HologoVariantRobust[2]%
  }%
}{%
  \protected\def\HologoVariantRobust#1#2%
}%
{%
  \begingroup
    \hologoLogoSetup{#1}{variant={#2}}%
    \HologoRobust{#1}%
  \endgroup
}
%    \end{macrocode}
%    \end{macro}
%
%    \begin{macro}{\hologorobust}
%    Macro \cs{hologorobust} is only defined for compatibility.
%    Its use is deprecated.
%    \begin{macrocode}
\def\hologorobust{\hologoRobust}
%    \end{macrocode}
%    \end{macro}
%
% \subsection{Helpers}
%
%    \begin{macro}{\HOLOGO@Uppercase}
%    Macro \cs{HOLOGO@Uppercase} is restricted to \cs{uppercase},
%    because \hologo{plainTeX} or \hologo{iniTeX} do not provide
%    \cs{MakeUppercase}.
%    \begin{macrocode}
\def\HOLOGO@Uppercase#1{\uppercase{#1}}
%    \end{macrocode}
%    \end{macro}
%
%    \begin{macro}{\HOLOGO@PdfdocUnicode}
%    \begin{macrocode}
\def\HOLOGO@PdfdocUnicode{%
  \ifx\ifHy@unicode\iftrue
    \expandafter\ltx@secondoftwo
  \else
    \expandafter\ltx@firstoftwo
  \fi
}
%    \end{macrocode}
%    \end{macro}
%
%    \begin{macro}{\HOLOGO@Math}
%    \begin{macrocode}
\def\HOLOGO@MathSetup{%
  \mathsurround0pt\relax
  \HOLOGO@IfExists\f@series{%
    \if b\expandafter\ltx@car\f@series x\@nil
      \csname boldmath\endcsname
   \fi
  }{}%
}
%    \end{macrocode}
%    \end{macro}
%
%    \begin{macro}{\HOLOGO@TempDimen}
%    \begin{macrocode}
\dimendef\HOLOGO@TempDimen=\ltx@zero
%    \end{macrocode}
%    \end{macro}
%    \begin{macro}{\HOLOGO@NegativeKerning}
%    \begin{macrocode}
\def\HOLOGO@NegativeKerning#1{%
  \begingroup
    \HOLOGO@TempDimen=0pt\relax
    \comma@parse@normalized{#1}{%
      \ifdim\HOLOGO@TempDimen=0pt %
        \expandafter\HOLOGO@@NegativeKerning\comma@entry
      \fi
      \ltx@gobble
    }%
    \ifdim\HOLOGO@TempDimen<0pt %
      \kern\HOLOGO@TempDimen
    \fi
  \endgroup
}
%    \end{macrocode}
%    \end{macro}
%    \begin{macro}{\HOLOGO@@NegativeKerning}
%    \begin{macrocode}
\def\HOLOGO@@NegativeKerning#1#2{%
  \setbox\ltx@zero\hbox{#1#2}%
  \HOLOGO@TempDimen=\wd\ltx@zero
  \setbox\ltx@zero\hbox{#1\kern0pt#2}%
  \advance\HOLOGO@TempDimen by -\wd\ltx@zero
}
%    \end{macrocode}
%    \end{macro}
%
%    \begin{macro}{\HOLOGO@SpaceFactor}
%    \begin{macrocode}
\def\HOLOGO@SpaceFactor{%
  \spacefactor1000 %
}
%    \end{macrocode}
%    \end{macro}
%
%    \begin{macro}{\HOLOGO@Span}
%    \begin{macrocode}
\def\HOLOGO@Span#1#2{%
  \HCode{<span class="HoLogo-#1">}%
  #2%
  \HCode{</span>}%
}
%    \end{macrocode}
%    \end{macro}
%
% \subsubsection{Text subscript}
%
%    \begin{macro}{\HOLOGO@SubScript}%
%    \begin{macrocode}
\def\HOLOGO@SubScript#1{%
  \ltx@IfUndefined{textsubscript}{%
    \ltx@IfUndefined{text}{%
      \ltx@mbox{%
        \mathsurround=0pt\relax
        $%
          _{%
            \ltx@IfUndefined{sf@size}{%
              \mathrm{#1}%
            }{%
              \mbox{%
                \fontsize\sf@size{0pt}\selectfont
                #1%
              }%
            }%
          }%
        $%
      }%
    }{%
      \ltx@mbox{%
        \mathsurround=0pt\relax
        $_{\text{#1}}$%
      }%
    }%
  }{%
    \textsubscript{#1}%
  }%
}
%    \end{macrocode}
%    \end{macro}
%
% \subsection{\hologo{TeX} and friends}
%
% \subsubsection{\hologo{TeX}}
%
%    \begin{macro}{\HoLogo@TeX}
%    Source: \hologo{LaTeX} kernel.
%    \begin{macrocode}
\def\HoLogo@TeX#1{%
  T\kern-.1667em\lower.5ex\hbox{E}\kern-.125emX\HOLOGO@SpaceFactor
}
%    \end{macrocode}
%    \end{macro}
%    \begin{macro}{\HoLogoHtml@TeX}
%    \begin{macrocode}
\def\HoLogoHtml@TeX#1{%
  \HoLogoCss@TeX
  \HOLOGO@Span{TeX}{%
    T%
    \HOLOGO@Span{e}{%
      E%
    }%
    X%
  }%
}
%    \end{macrocode}
%    \end{macro}
%    \begin{macro}{\HoLogoCss@TeX}
%    \begin{macrocode}
\def\HoLogoCss@TeX{%
  \Css{%
    span.HoLogo-TeX span.HoLogo-e{%
      position:relative;%
      top:.5ex;%
      margin-left:-.1667em;%
      margin-right:-.125em;%
    }%
  }%
  \Css{%
    a span.HoLogo-TeX span.HoLogo-e{%
      text-decoration:none;%
    }%
  }%
  \global\let\HoLogoCss@TeX\relax
}
%    \end{macrocode}
%    \end{macro}
%
% \subsubsection{\hologo{plainTeX}}
%
%    \begin{macro}{\HoLogo@plainTeX@space}
%    Source: ``The \hologo{TeX}book''
%    \begin{macrocode}
\def\HoLogo@plainTeX@space#1{%
  \HOLOGO@mbox{#1{p}{P}lain}\HOLOGO@space\hologo{TeX}%
}
%    \end{macrocode}
%    \end{macro}
%    \begin{macro}{\HoLogoCs@plainTeX@space}
%    \begin{macrocode}
\def\HoLogoCs@plainTeX@space#1{#1{p}{P}lain TeX}%
%    \end{macrocode}
%    \end{macro}
%    \begin{macro}{\HoLogoBkm@plainTeX@space}
%    \begin{macrocode}
\def\HoLogoBkm@plainTeX@space#1{%
  #1{p}{P}lain \hologo{TeX}%
}
%    \end{macrocode}
%    \end{macro}
%    \begin{macro}{\HoLogoHtml@plainTeX@space}
%    \begin{macrocode}
\def\HoLogoHtml@plainTeX@space#1{%
  #1{p}{P}lain \hologo{TeX}%
}
%    \end{macrocode}
%    \end{macro}
%
%    \begin{macro}{\HoLogo@plainTeX@hyphen}
%    \begin{macrocode}
\def\HoLogo@plainTeX@hyphen#1{%
  \HOLOGO@mbox{#1{p}{P}lain}\HOLOGO@hyphen\hologo{TeX}%
}
%    \end{macrocode}
%    \end{macro}
%    \begin{macro}{\HoLogoCs@plainTeX@hyphen}
%    \begin{macrocode}
\def\HoLogoCs@plainTeX@hyphen#1{#1{p}{P}lain-TeX}
%    \end{macrocode}
%    \end{macro}
%    \begin{macro}{\HoLogoBkm@plainTeX@hyphen}
%    \begin{macrocode}
\def\HoLogoBkm@plainTeX@hyphen#1{%
  #1{p}{P}lain-\hologo{TeX}%
}
%    \end{macrocode}
%    \end{macro}
%    \begin{macro}{\HoLogoHtml@plainTeX@hyphen}
%    \begin{macrocode}
\def\HoLogoHtml@plainTeX@hyphen#1{%
  #1{p}{P}lain-\hologo{TeX}%
}
%    \end{macrocode}
%    \end{macro}
%
%    \begin{macro}{\HoLogo@plainTeX@runtogether}
%    \begin{macrocode}
\def\HoLogo@plainTeX@runtogether#1{%
  \HOLOGO@mbox{#1{p}{P}lain\hologo{TeX}}%
}
%    \end{macrocode}
%    \end{macro}
%    \begin{macro}{\HoLogoCs@plainTeX@runtogether}
%    \begin{macrocode}
\def\HoLogoCs@plainTeX@runtogether#1{#1{p}{P}lainTeX}
%    \end{macrocode}
%    \end{macro}
%    \begin{macro}{\HoLogoBkm@plainTeX@runtogether}
%    \begin{macrocode}
\def\HoLogoBkm@plainTeX@runtogether#1{%
  #1{p}{P}lain\hologo{TeX}%
}
%    \end{macrocode}
%    \end{macro}
%    \begin{macro}{\HoLogoHtml@plainTeX@runtogether}
%    \begin{macrocode}
\def\HoLogoHtml@plainTeX@runtogether#1{%
  #1{p}{P}lain\hologo{TeX}%
}
%    \end{macrocode}
%    \end{macro}
%
%    \begin{macro}{\HoLogo@plainTeX}
%    \begin{macrocode}
\def\HoLogo@plainTeX{\HoLogo@plainTeX@space}
%    \end{macrocode}
%    \end{macro}
%    \begin{macro}{\HoLogoCs@plainTeX}
%    \begin{macrocode}
\def\HoLogoCs@plainTeX{\HoLogoCs@plainTeX@space}
%    \end{macrocode}
%    \end{macro}
%    \begin{macro}{\HoLogoBkm@plainTeX}
%    \begin{macrocode}
\def\HoLogoBkm@plainTeX{\HoLogoBkm@plainTeX@space}
%    \end{macrocode}
%    \end{macro}
%    \begin{macro}{\HoLogoHtml@plainTeX}
%    \begin{macrocode}
\def\HoLogoHtml@plainTeX{\HoLogoHtml@plainTeX@space}
%    \end{macrocode}
%    \end{macro}
%
% \subsubsection{\hologo{LaTeX}}
%
%    Source: \hologo{LaTeX} kernel.
%\begin{quote}
%\begin{verbatim}
%\DeclareRobustCommand{\LaTeX}{%
%  L%
%  \kern-.36em%
%  {%
%    \sbox\z@ T%
%    \vbox to\ht\z@{%
%      \hbox{%
%        \check@mathfonts
%        \fontsize\sf@size\z@
%        \math@fontsfalse
%        \selectfont
%        A%
%      }%
%      \vss
%    }%
%  }%
%  \kern-.15em%
%  \TeX
%}
%\end{verbatim}
%\end{quote}
%
%    \begin{macro}{\HoLogo@La}
%    \begin{macrocode}
\def\HoLogo@La#1{%
  L%
  \kern-.36em%
  \begingroup
    \setbox\ltx@zero\hbox{T}%
    \vbox to\ht\ltx@zero{%
      \hbox{%
        \ltx@ifundefined{check@mathfonts}{%
          \csname sevenrm\endcsname
        }{%
          \check@mathfonts
          \fontsize\sf@size{0pt}%
          \math@fontsfalse\selectfont
        }%
        A%
      }%
      \vss
    }%
  \endgroup
}
%    \end{macrocode}
%    \end{macro}
%
%    \begin{macro}{\HoLogo@LaTeX}
%    Source: \hologo{LaTeX} kernel.
%    \begin{macrocode}
\def\HoLogo@LaTeX#1{%
  \hologo{La}%
  \kern-.15em%
  \hologo{TeX}%
}
%    \end{macrocode}
%    \end{macro}
%    \begin{macro}{\HoLogoHtml@LaTeX}
%    \begin{macrocode}
\def\HoLogoHtml@LaTeX#1{%
  \HoLogoCss@LaTeX
  \HOLOGO@Span{LaTeX}{%
    L%
    \HOLOGO@Span{a}{%
      A%
    }%
    \hologo{TeX}%
  }%
}
%    \end{macrocode}
%    \end{macro}
%    \begin{macro}{\HoLogoCss@LaTeX}
%    \begin{macrocode}
\def\HoLogoCss@LaTeX{%
  \Css{%
    span.HoLogo-LaTeX span.HoLogo-a{%
      position:relative;%
      top:-.5ex;%
      margin-left:-.36em;%
      margin-right:-.15em;%
      font-size:85\%;%
    }%
  }%
  \global\let\HoLogoCss@LaTeX\relax
}
%    \end{macrocode}
%    \end{macro}
%
% \subsubsection{\hologo{(La)TeX}}
%
%    \begin{macro}{\HoLogo@LaTeXTeX}
%    The kerning around the parentheses is taken
%    from package \xpackage{dtklogos} \cite{dtklogos}.
%\begin{quote}
%\begin{verbatim}
%\DeclareRobustCommand{\LaTeXTeX}{%
%  (%
%  \kern-.15em%
%  L%
%  \kern-.36em%
%  {%
%    \sbox\z@ T%
%    \vbox to\ht0{%
%      \hbox{%
%        $\m@th$%
%        \csname S@\f@size\endcsname
%        \fontsize\sf@size\z@
%        \math@fontsfalse
%        \selectfont
%        A%
%      }%
%      \vss
%    }%
%  }%
%  \kern-.2em%
%  )%
%  \kern-.15em%
%  \TeX
%}
%\end{verbatim}
%\end{quote}
%    \begin{macrocode}
\def\HoLogo@LaTeXTeX#1{%
  (%
  \kern-.15em%
  \hologo{La}%
  \kern-.2em%
  )%
  \kern-.15em%
  \hologo{TeX}%
}
%    \end{macrocode}
%    \end{macro}
%    \begin{macro}{\HoLogoBkm@LaTeXTeX}
%    \begin{macrocode}
\def\HoLogoBkm@LaTeXTeX#1{(La)TeX}
%    \end{macrocode}
%    \end{macro}
%
%    \begin{macro}{\HoLogo@(La)TeX}
%    \begin{macrocode}
\expandafter
\let\csname HoLogo@(La)TeX\endcsname\HoLogo@LaTeXTeX
%    \end{macrocode}
%    \end{macro}
%    \begin{macro}{\HoLogoBkm@(La)TeX}
%    \begin{macrocode}
\expandafter
\let\csname HoLogoBkm@(La)TeX\endcsname\HoLogoBkm@LaTeXTeX
%    \end{macrocode}
%    \end{macro}
%    \begin{macro}{\HoLogoHtml@LaTeXTeX}
%    \begin{macrocode}
\def\HoLogoHtml@LaTeXTeX#1{%
  \HoLogoCss@LaTeXTeX
  \HOLOGO@Span{LaTeXTeX}{%
    (%
    \HOLOGO@Span{L}{L}%
    \HOLOGO@Span{a}{A}%
    \HOLOGO@Span{ParenRight}{)}%
    \hologo{TeX}%
  }%
}
%    \end{macrocode}
%    \end{macro}
%    \begin{macro}{\HoLogoHtml@(La)TeX}
%    Kerning after opening parentheses and before closing parentheses
%    is $-0.1$\,em. The original values $-0.15$\,em
%    looked too ugly for a serif font.
%    \begin{macrocode}
\expandafter
\let\csname HoLogoHtml@(La)TeX\endcsname\HoLogoHtml@LaTeXTeX
%    \end{macrocode}
%    \end{macro}
%    \begin{macro}{\HoLogoCss@LaTeXTeX}
%    \begin{macrocode}
\def\HoLogoCss@LaTeXTeX{%
  \Css{%
    span.HoLogo-LaTeXTeX span.HoLogo-L{%
      margin-left:-.1em;%
    }%
  }%
  \Css{%
    span.HoLogo-LaTeXTeX span.HoLogo-a{%
      position:relative;%
      top:-.5ex;%
      margin-left:-.36em;%
      margin-right:-.1em;%
      font-size:85\%;%
    }%
  }%
  \Css{%
    span.HoLogo-LaTeXTeX span.HoLogo-ParenRight{%
      margin-right:-.15em;%
    }%
  }%
  \global\let\HoLogoCss@LaTeXTeX\relax
}
%    \end{macrocode}
%    \end{macro}
%
% \subsubsection{\hologo{LaTeXe}}
%
%    \begin{macro}{\HoLogo@LaTeXe}
%    Source: \hologo{LaTeX} kernel
%    \begin{macrocode}
\def\HoLogo@LaTeXe#1{%
  \hologo{LaTeX}%
  \kern.15em%
  \hbox{%
    \HOLOGO@MathSetup
    2%
    $_{\textstyle\varepsilon}$%
  }%
}
%    \end{macrocode}
%    \end{macro}
%
%    \begin{macro}{\HoLogoCs@LaTeXe}
%    \begin{macrocode}
\ifnum64=`\^^^^0040\relax % test for big chars of LuaTeX/XeTeX
  \catcode`\$=9 %
  \catcode`\&=14 %
\else
  \catcode`\$=14 %
  \catcode`\&=9 %
\fi
\def\HoLogoCs@LaTeXe#1{%
  LaTeX2%
$ \string ^^^^0395%
& e%
}%
\catcode`\$=3 %
\catcode`\&=4 %
%    \end{macrocode}
%    \end{macro}
%
%    \begin{macro}{\HoLogoBkm@LaTeXe}
%    \begin{macrocode}
\def\HoLogoBkm@LaTeXe#1{%
  \hologo{LaTeX}%
  2%
  \HOLOGO@PdfdocUnicode{e}{\textepsilon}%
}
%    \end{macrocode}
%    \end{macro}
%
%    \begin{macro}{\HoLogoHtml@LaTeXe}
%    \begin{macrocode}
\def\HoLogoHtml@LaTeXe#1{%
  \HoLogoCss@LaTeXe
  \HOLOGO@Span{LaTeX2e}{%
    \hologo{LaTeX}%
    \HOLOGO@Span{2}{2}%
    \HOLOGO@Span{e}{%
      \HOLOGO@MathSetup
      \ensuremath{\textstyle\varepsilon}%
    }%
  }%
}
%    \end{macrocode}
%    \end{macro}
%    \begin{macro}{\HoLogoCss@LaTeXe}
%    \begin{macrocode}
\def\HoLogoCss@LaTeXe{%
  \Css{%
    span.HoLogo-LaTeX2e span.HoLogo-2{%
      padding-left:.15em;%
    }%
  }%
  \Css{%
    span.HoLogo-LaTeX2e span.HoLogo-e{%
      position:relative;%
      top:.35ex;%
      text-decoration:none;%
    }%
  }%
  \global\let\HoLogoCss@LaTeXe\relax
}
%    \end{macrocode}
%    \end{macro}
%
%    \begin{macro}{\HoLogo@LaTeX2e}
%    \begin{macrocode}
\expandafter
\let\csname HoLogo@LaTeX2e\endcsname\HoLogo@LaTeXe
%    \end{macrocode}
%    \end{macro}
%    \begin{macro}{\HoLogoCs@LaTeX2e}
%    \begin{macrocode}
\expandafter
\let\csname HoLogoCs@LaTeX2e\endcsname\HoLogoCs@LaTeXe
%    \end{macrocode}
%    \end{macro}
%    \begin{macro}{\HoLogoBkm@LaTeX2e}
%    \begin{macrocode}
\expandafter
\let\csname HoLogoBkm@LaTeX2e\endcsname\HoLogoBkm@LaTeXe
%    \end{macrocode}
%    \end{macro}
%    \begin{macro}{\HoLogoHtml@LaTeX2e}
%    \begin{macrocode}
\expandafter
\let\csname HoLogoHtml@LaTeX2e\endcsname\HoLogoHtml@LaTeXe
%    \end{macrocode}
%    \end{macro}
%
% \subsubsection{\hologo{LaTeX3}}
%
%    \begin{macro}{\HoLogo@LaTeX3}
%    Source: \hologo{LaTeX} kernel
%    \begin{macrocode}
\expandafter\def\csname HoLogo@LaTeX3\endcsname#1{%
  \hologo{LaTeX}%
  3%
}
%    \end{macrocode}
%    \end{macro}
%
%    \begin{macro}{\HoLogoBkm@LaTeX3}
%    \begin{macrocode}
\expandafter\def\csname HoLogoBkm@LaTeX3\endcsname#1{%
  \hologo{LaTeX}%
  3%
}
%    \end{macrocode}
%    \end{macro}
%    \begin{macro}{\HoLogoHtml@LaTeX3}
%    \begin{macrocode}
\expandafter
\let\csname HoLogoHtml@LaTeX3\expandafter\endcsname
\csname HoLogo@LaTeX3\endcsname
%    \end{macrocode}
%    \end{macro}
%
% \subsubsection{\hologo{LaTeXML}}
%
%    \begin{macro}{\HoLogo@LaTeXML}
%    \begin{macrocode}
\def\HoLogo@LaTeXML#1{%
  \HOLOGO@mbox{%
    \hologo{La}%
    \kern-.15em%
    T%
    \kern-.1667em%
    \lower.5ex\hbox{E}%
    \kern-.125em%
    \HoLogoFont@font{LaTeXML}{sc}{xml}%
  }%
}
%    \end{macrocode}
%    \end{macro}
%    \begin{macro}{\HoLogoHtml@pdfLaTeX}
%    \begin{macrocode}
\def\HoLogoHtml@LaTeXML#1{%
  \HOLOGO@Span{LaTeXML}{%
    \HoLogoCss@LaTeX
    \HoLogoCss@TeX
    \HOLOGO@Span{LaTeX}{%
      L%
      \HOLOGO@Span{a}{%
        A%
      }%
    }%
    \HOLOGO@Span{TeX}{%
      T%
      \HOLOGO@Span{e}{%
        E%
      }%
    }%
    \HCode{<span style="font-variant: small-caps;">}%
    xml%
    \HCode{</span>}%
  }%
}
%    \end{macrocode}
%    \end{macro}
%
% \subsubsection{\hologo{eTeX}}
%
%    \begin{macro}{\HoLogo@eTeX}
%    Source: package \xpackage{etex}
%    \begin{macrocode}
\def\HoLogo@eTeX#1{%
  \ltx@mbox{%
    \HOLOGO@MathSetup
    $\varepsilon$%
    -%
    \HOLOGO@NegativeKerning{-T,T-,To}%
    \hologo{TeX}%
  }%
}
%    \end{macrocode}
%    \end{macro}
%    \begin{macro}{\HoLogoCs@eTeX}
%    \begin{macrocode}
\ifnum64=`\^^^^0040\relax % test for big chars of LuaTeX/XeTeX
  \catcode`\$=9 %
  \catcode`\&=14 %
\else
  \catcode`\$=14 %
  \catcode`\&=9 %
\fi
\def\HoLogoCs@eTeX#1{%
$ #1{\string ^^^^0395}{\string ^^^^03b5}%
& #1{e}{E}%
  TeX%
}%
\catcode`\$=3 %
\catcode`\&=4 %
%    \end{macrocode}
%    \end{macro}
%    \begin{macro}{\HoLogoBkm@eTeX}
%    \begin{macrocode}
\def\HoLogoBkm@eTeX#1{%
  \HOLOGO@PdfdocUnicode{#1{e}{E}}{\textepsilon}%
  -%
  \hologo{TeX}%
}
%    \end{macrocode}
%    \end{macro}
%    \begin{macro}{\HoLogoHtml@eTeX}
%    \begin{macrocode}
\def\HoLogoHtml@eTeX#1{%
  \ltx@mbox{%
    \HOLOGO@MathSetup
    $\varepsilon$%
    -%
    \hologo{TeX}%
  }%
}
%    \end{macrocode}
%    \end{macro}
%
% \subsubsection{\hologo{iniTeX}}
%
%    \begin{macro}{\HoLogo@iniTeX}
%    \begin{macrocode}
\def\HoLogo@iniTeX#1{%
  \HOLOGO@mbox{%
    #1{i}{I}ni\hologo{TeX}%
  }%
}
%    \end{macrocode}
%    \end{macro}
%    \begin{macro}{\HoLogoCs@iniTeX}
%    \begin{macrocode}
\def\HoLogoCs@iniTeX#1{#1{i}{I}niTeX}
%    \end{macrocode}
%    \end{macro}
%    \begin{macro}{\HoLogoBkm@iniTeX}
%    \begin{macrocode}
\def\HoLogoBkm@iniTeX#1{%
  #1{i}{I}ni\hologo{TeX}%
}
%    \end{macrocode}
%    \end{macro}
%    \begin{macro}{\HoLogoHtml@iniTeX}
%    \begin{macrocode}
\let\HoLogoHtml@iniTeX\HoLogo@iniTeX
%    \end{macrocode}
%    \end{macro}
%
% \subsubsection{\hologo{virTeX}}
%
%    \begin{macro}{\HoLogo@virTeX}
%    \begin{macrocode}
\def\HoLogo@virTeX#1{%
  \HOLOGO@mbox{%
    #1{v}{V}ir\hologo{TeX}%
  }%
}
%    \end{macrocode}
%    \end{macro}
%    \begin{macro}{\HoLogoCs@virTeX}
%    \begin{macrocode}
\def\HoLogoCs@virTeX#1{#1{v}{V}irTeX}
%    \end{macrocode}
%    \end{macro}
%    \begin{macro}{\HoLogoBkm@virTeX}
%    \begin{macrocode}
\def\HoLogoBkm@virTeX#1{%
  #1{v}{V}ir\hologo{TeX}%
}
%    \end{macrocode}
%    \end{macro}
%    \begin{macro}{\HoLogoHtml@virTeX}
%    \begin{macrocode}
\let\HoLogoHtml@virTeX\HoLogo@virTeX
%    \end{macrocode}
%    \end{macro}
%
% \subsubsection{\hologo{SliTeX}}
%
% \paragraph{Definitions of the three variants.}
%
%    \begin{macro}{\HoLogo@SLiTeX@lift}
%    \begin{macrocode}
\def\HoLogo@SLiTeX@lift#1{%
  \HoLogoFont@font{SliTeX}{rm}{%
    S%
    \kern-.06em%
    L%
    \kern-.18em%
    \raise.32ex\hbox{\HoLogoFont@font{SliTeX}{sc}{i}}%
    \HOLOGO@discretionary
    \kern-.06em%
    \hologo{TeX}%
  }%
}
%    \end{macrocode}
%    \end{macro}
%    \begin{macro}{\HoLogoBkm@SLiTeX@lift}
%    \begin{macrocode}
\def\HoLogoBkm@SLiTeX@lift#1{SLiTeX}
%    \end{macrocode}
%    \end{macro}
%    \begin{macro}{\HoLogoHtml@SLiTeX@lift}
%    \begin{macrocode}
\def\HoLogoHtml@SLiTeX@lift#1{%
  \HoLogoCss@SLiTeX@lift
  \HOLOGO@Span{SLiTeX-lift}{%
    \HoLogoFont@font{SliTeX}{rm}{%
      S%
      \HOLOGO@Span{L}{L}%
      \HOLOGO@Span{i}{i}%
      \hologo{TeX}%
    }%
  }%
}
%    \end{macrocode}
%    \end{macro}
%    \begin{macro}{\HoLogoCss@SLiTeX@lift}
%    \begin{macrocode}
\def\HoLogoCss@SLiTeX@lift{%
  \Css{%
    span.HoLogo-SLiTeX-lift span.HoLogo-L{%
      margin-left:-.06em;%
      margin-right:-.18em;%
    }%
  }%
  \Css{%
    span.HoLogo-SLiTeX-lift span.HoLogo-i{%
      position:relative;%
      top:-.32ex;%
      margin-right:-.06em;%
      font-variant:small-caps;%
    }%
  }%
  \global\let\HoLogoCss@SLiTeX@lift\relax
}
%    \end{macrocode}
%    \end{macro}
%
%    \begin{macro}{\HoLogo@SliTeX@simple}
%    \begin{macrocode}
\def\HoLogo@SliTeX@simple#1{%
  \HoLogoFont@font{SliTeX}{rm}{%
    \ltx@mbox{%
      \HoLogoFont@font{SliTeX}{sc}{Sli}%
    }%
    \HOLOGO@discretionary
    \hologo{TeX}%
  }%
}
%    \end{macrocode}
%    \end{macro}
%    \begin{macro}{\HoLogoBkm@SliTeX@simple}
%    \begin{macrocode}
\def\HoLogoBkm@SliTeX@simple#1{SliTeX}
%    \end{macrocode}
%    \end{macro}
%    \begin{macro}{\HoLogoHtml@SliTeX@simple}
%    \begin{macrocode}
\let\HoLogoHtml@SliTeX@simple\HoLogo@SliTeX@simple
%    \end{macrocode}
%    \end{macro}
%
%    \begin{macro}{\HoLogo@SliTeX@narrow}
%    \begin{macrocode}
\def\HoLogo@SliTeX@narrow#1{%
  \HoLogoFont@font{SliTeX}{rm}{%
    \ltx@mbox{%
      S%
      \kern-.06em%
      \HoLogoFont@font{SliTeX}{sc}{%
        l%
        \kern-.035em%
        i%
      }%
    }%
    \HOLOGO@discretionary
    \kern-.06em%
    \hologo{TeX}%
  }%
}
%    \end{macrocode}
%    \end{macro}
%    \begin{macro}{\HoLogoBkm@SliTeX@narrow}
%    \begin{macrocode}
\def\HoLogoBkm@SliTeX@narrow#1{SliTeX}
%    \end{macrocode}
%    \end{macro}
%    \begin{macro}{\HoLogoHtml@SliTeX@narrow}
%    \begin{macrocode}
\def\HoLogoHtml@SliTeX@narrow#1{%
  \HoLogoCss@SliTeX@narrow
  \HOLOGO@Span{SliTeX-narrow}{%
    \HoLogoFont@font{SliTeX}{rm}{%
      S%
        \HOLOGO@Span{l}{l}%
        \HOLOGO@Span{i}{i}%
      \hologo{TeX}%
    }%
  }%
}
%    \end{macrocode}
%    \end{macro}
%    \begin{macro}{\HoLogoCss@SliTeX@narrow}
%    \begin{macrocode}
\def\HoLogoCss@SliTeX@narrow{%
  \Css{%
    span.HoLogo-SliTeX-narrow span.HoLogo-l{%
      margin-left:-.06em;%
      margin-right:-.035em;%
      font-variant:small-caps;%
    }%
  }%
  \Css{%
    span.HoLogo-SliTeX-narrow span.HoLogo-i{%
      margin-right:-.06em;%
      font-variant:small-caps;%
    }%
  }%
  \global\let\HoLogoCss@SliTeX@narrow\relax
}
%    \end{macrocode}
%    \end{macro}
%
% \paragraph{Macro set completion.}
%
%    \begin{macro}{\HoLogo@SLiTeX@simple}
%    \begin{macrocode}
\def\HoLogo@SLiTeX@simple{\HoLogo@SliTeX@simple}
%    \end{macrocode}
%    \end{macro}
%    \begin{macro}{\HoLogoBkm@SLiTeX@simple}
%    \begin{macrocode}
\def\HoLogoBkm@SLiTeX@simple{\HoLogoBkm@SliTeX@simple}
%    \end{macrocode}
%    \end{macro}
%    \begin{macro}{\HoLogoHtml@SLiTeX@simple}
%    \begin{macrocode}
\def\HoLogoHtml@SLiTeX@simple{\HoLogoHtml@SliTeX@simple}
%    \end{macrocode}
%    \end{macro}
%
%    \begin{macro}{\HoLogo@SLiTeX@narrow}
%    \begin{macrocode}
\def\HoLogo@SLiTeX@narrow{\HoLogo@SliTeX@narrow}
%    \end{macrocode}
%    \end{macro}
%    \begin{macro}{\HoLogoBkm@SLiTeX@narrow}
%    \begin{macrocode}
\def\HoLogoBkm@SLiTeX@narrow{\HoLogoBkm@SliTeX@narrow}
%    \end{macrocode}
%    \end{macro}
%    \begin{macro}{\HoLogoHtml@SLiTeX@narrow}
%    \begin{macrocode}
\def\HoLogoHtml@SLiTeX@narrow{\HoLogoHtml@SliTeX@narrow}
%    \end{macrocode}
%    \end{macro}
%
%    \begin{macro}{\HoLogo@SliTeX@lift}
%    \begin{macrocode}
\def\HoLogo@SliTeX@lift{\HoLogo@SLiTeX@lift}
%    \end{macrocode}
%    \end{macro}
%    \begin{macro}{\HoLogoBkm@SliTeX@lift}
%    \begin{macrocode}
\def\HoLogoBkm@SliTeX@lift{\HoLogoBkm@SLiTeX@lift}
%    \end{macrocode}
%    \end{macro}
%    \begin{macro}{\HoLogoHtml@SliTeX@lift}
%    \begin{macrocode}
\def\HoLogoHtml@SliTeX@lift{\HoLogoHtml@SLiTeX@lift}
%    \end{macrocode}
%    \end{macro}
%
% \paragraph{Defaults.}
%
%    \begin{macro}{\HoLogo@SLiTeX}
%    \begin{macrocode}
\def\HoLogo@SLiTeX{\HoLogo@SLiTeX@lift}
%    \end{macrocode}
%    \end{macro}
%    \begin{macro}{\HoLogoBkm@SLiTeX}
%    \begin{macrocode}
\def\HoLogoBkm@SLiTeX{\HoLogoBkm@SLiTeX@lift}
%    \end{macrocode}
%    \end{macro}
%    \begin{macro}{\HoLogoHtml@SLiTeX}
%    \begin{macrocode}
\def\HoLogoHtml@SLiTeX{\HoLogoHtml@SLiTeX@lift}
%    \end{macrocode}
%    \end{macro}
%
%    \begin{macro}{\HoLogo@SliTeX}
%    \begin{macrocode}
\def\HoLogo@SliTeX{\HoLogo@SliTeX@narrow}
%    \end{macrocode}
%    \end{macro}
%    \begin{macro}{\HoLogoBkm@SliTeX}
%    \begin{macrocode}
\def\HoLogoBkm@SliTeX{\HoLogoBkm@SliTeX@narrow}
%    \end{macrocode}
%    \end{macro}
%    \begin{macro}{\HoLogoHtml@SliTeX}
%    \begin{macrocode}
\def\HoLogoHtml@SliTeX{\HoLogoHtml@SliTeX@narrow}
%    \end{macrocode}
%    \end{macro}
%
% \subsubsection{\hologo{LuaTeX}}
%
%    \begin{macro}{\HoLogo@LuaTeX}
%    The kerning is an idea of Hans Hagen, see mailing list
%    `luatex at tug dot org' in March 2010.
%    \begin{macrocode}
\def\HoLogo@LuaTeX#1{%
  \HOLOGO@mbox{%
    Lua%
    \HOLOGO@NegativeKerning{aT,oT,To}%
    \hologo{TeX}%
  }%
}
%    \end{macrocode}
%    \end{macro}
%    \begin{macro}{\HoLogoHtml@LuaTeX}
%    \begin{macrocode}
\let\HoLogoHtml@LuaTeX\HoLogo@LuaTeX
%    \end{macrocode}
%    \end{macro}
%
% \subsubsection{\hologo{LuaLaTeX}}
%
%    \begin{macro}{\HoLogo@LuaLaTeX}
%    \begin{macrocode}
\def\HoLogo@LuaLaTeX#1{%
  \HOLOGO@mbox{%
    Lua%
    \hologo{LaTeX}%
  }%
}
%    \end{macrocode}
%    \end{macro}
%    \begin{macro}{\HoLogoHtml@LuaLaTeX}
%    \begin{macrocode}
\let\HoLogoHtml@LuaLaTeX\HoLogo@LuaLaTeX
%    \end{macrocode}
%    \end{macro}
%
% \subsubsection{\hologo{XeTeX}, \hologo{XeLaTeX}}
%
%    \begin{macro}{\HOLOGO@IfCharExists}
%    \begin{macrocode}
\ifluatex
  \ifnum\luatexversion<36 %
  \else
    \def\HOLOGO@IfCharExists#1{%
      \ifnum
        \directlua{%
           if luaotfload and luaotfload.aux then
             if luaotfload.aux.font_has_glyph(%
                    font.current(), \number#1) then % 	 
	       tex.print("1") % 	 
	     end % 	 
	   elseif font and font.fonts and font.current then %
            local f = font.fonts[font.current()]%
            if f.characters and f.characters[\number#1] then %
              tex.print("1")%
            end %
          end%
        }0=\ltx@zero
        \expandafter\ltx@secondoftwo
      \else
        \expandafter\ltx@firstoftwo
      \fi
    }%
  \fi
\fi
\ltx@IfUndefined{HOLOGO@IfCharExists}{%
  \def\HOLOGO@@IfCharExists#1{%
    \begingroup
      \tracinglostchars=\ltx@zero
      \setbox\ltx@zero=\hbox{%
        \kern7sp\char#1\relax
        \ifnum\lastkern>\ltx@zero
          \expandafter\aftergroup\csname iffalse\endcsname
        \else
          \expandafter\aftergroup\csname iftrue\endcsname
        \fi
      }%
      % \if{true|false} from \aftergroup
      \endgroup
      \expandafter\ltx@firstoftwo
    \else
      \endgroup
      \expandafter\ltx@secondoftwo
    \fi
  }%
  \ifxetex
    \ltx@IfUndefined{XeTeXfonttype}{}{%
      \ltx@IfUndefined{XeTeXcharglyph}{}{%
        \def\HOLOGO@IfCharExists#1{%
          \ifnum\XeTeXfonttype\font>\ltx@zero
            \expandafter\ltx@firstofthree
          \else
            \expandafter\ltx@gobble
          \fi
          {%
            \ifnum\XeTeXcharglyph#1>\ltx@zero
              \expandafter\ltx@firstoftwo
            \else
              \expandafter\ltx@secondoftwo
            \fi
          }%
          \HOLOGO@@IfCharExists{#1}%
        }%
      }%
    }%
  \fi
}{}
\ltx@ifundefined{HOLOGO@IfCharExists}{%
  \ifnum64=`\^^^^0040\relax % test for big chars of LuaTeX/XeTeX
    \let\HOLOGO@IfCharExists\HOLOGO@@IfCharExists
  \else
    \def\HOLOGO@IfCharExists#1{%
      \ifnum#1>255 %
        \expandafter\ltx@fourthoffour
      \fi
      \HOLOGO@@IfCharExists{#1}%
    }%
  \fi
}{}
%    \end{macrocode}
%    \end{macro}
%
%    \begin{macro}{\HoLogo@Xe}
%    Source: package \xpackage{dtklogos}
%    \begin{macrocode}
\def\HoLogo@Xe#1{%
  X%
  \kern-.1em\relax
  \HOLOGO@IfCharExists{"018E}{%
    \lower.5ex\hbox{\char"018E}%
  }{%
    \chardef\HOLOGO@choice=\ltx@zero
    \ifdim\fontdimen\ltx@one\font>0pt %
      \ltx@IfUndefined{rotatebox}{%
        \ltx@IfUndefined{pgftext}{%
          \ltx@IfUndefined{psscalebox}{%
            \ltx@IfUndefined{HOLOGO@ScaleBox@\hologoDriver}{%
            }{%
              \chardef\HOLOGO@choice=4 %
            }%
          }{%
            \chardef\HOLOGO@choice=3 %
          }%
        }{%
          \chardef\HOLOGO@choice=2 %
        }%
      }{%
        \chardef\HOLOGO@choice=1 %
      }%
      \ifcase\HOLOGO@choice
        \HOLOGO@WarningUnsupportedDriver{Xe}%
        e%
      \or % 1: \rotatebox
        \begingroup
          \setbox\ltx@zero\hbox{\rotatebox{180}{E}}%
          \ltx@LocDimenA=\dp\ltx@zero
          \advance\ltx@LocDimenA by -.5ex\relax
          \raise\ltx@LocDimenA\box\ltx@zero
        \endgroup
      \or % 2: \pgftext
        \lower.5ex\hbox{%
          \pgfpicture
            \pgftext[rotate=180]{E}%
          \endpgfpicture
        }%
      \or % 3: \psscalebox
        \begingroup
          \setbox\ltx@zero\hbox{\psscalebox{-1 -1}{E}}%
          \ltx@LocDimenA=\dp\ltx@zero
          \advance\ltx@LocDimenA by -.5ex\relax
          \raise\ltx@LocDimenA\box\ltx@zero
        \endgroup
      \or % 4: \HOLOGO@PointReflectBox
        \lower.5ex\hbox{\HOLOGO@PointReflectBox{E}}%
      \else
        \@PackageError{hologo}{Internal error (choice/it}\@ehc
      \fi
    \else
      \ltx@IfUndefined{reflectbox}{%
        \ltx@IfUndefined{pgftext}{%
          \ltx@IfUndefined{psscalebox}{%
            \ltx@IfUndefined{HOLOGO@ScaleBox@\hologoDriver}{%
            }{%
              \chardef\HOLOGO@choice=4 %
            }%
          }{%
            \chardef\HOLOGO@choice=3 %
          }%
        }{%
          \chardef\HOLOGO@choice=2 %
        }%
      }{%
        \chardef\HOLOGO@choice=1 %
      }%
      \ifcase\HOLOGO@choice
        \HOLOGO@WarningUnsupportedDriver{Xe}%
        e%
      \or % 1: reflectbox
        \lower.5ex\hbox{%
          \reflectbox{E}%
        }%
      \or % 2: \pgftext
        \lower.5ex\hbox{%
          \pgfpicture
            \pgftransformxscale{-1}%
            \pgftext{E}%
          \endpgfpicture
        }%
      \or % 3: \psscalebox
        \lower.5ex\hbox{%
          \psscalebox{-1 1}{E}%
        }%
      \or % 4: \HOLOGO@Reflectbox
        \lower.5ex\hbox{%
          \HOLOGO@ReflectBox{E}%
        }%
      \else
        \@PackageError{hologo}{Internal error (choice/up)}\@ehc
      \fi
    \fi
  }%
}
%    \end{macrocode}
%    \end{macro}
%    \begin{macro}{\HoLogoHtml@Xe}
%    \begin{macrocode}
\def\HoLogoHtml@Xe#1{%
  \HoLogoCss@Xe
  \HOLOGO@Span{Xe}{%
    X%
    \HOLOGO@Span{e}{%
      \HCode{&\ltx@hashchar x018e;}%
    }%
  }%
}
%    \end{macrocode}
%    \end{macro}
%    \begin{macro}{\HoLogoCss@Xe}
%    \begin{macrocode}
\def\HoLogoCss@Xe{%
  \Css{%
    span.HoLogo-Xe span.HoLogo-e{%
      position:relative;%
      top:.5ex;%
      left-margin:-.1em;%
    }%
  }%
  \global\let\HoLogoCss@Xe\relax
}
%    \end{macrocode}
%    \end{macro}
%
%    \begin{macro}{\HoLogo@XeTeX}
%    \begin{macrocode}
\def\HoLogo@XeTeX#1{%
  \hologo{Xe}%
  \kern-.15em\relax
  \hologo{TeX}%
}
%    \end{macrocode}
%    \end{macro}
%
%    \begin{macro}{\HoLogoHtml@XeTeX}
%    \begin{macrocode}
\def\HoLogoHtml@XeTeX#1{%
  \HoLogoCss@XeTeX
  \HOLOGO@Span{XeTeX}{%
    \hologo{Xe}%
    \hologo{TeX}%
  }%
}
%    \end{macrocode}
%    \end{macro}
%    \begin{macro}{\HoLogoCss@XeTeX}
%    \begin{macrocode}
\def\HoLogoCss@XeTeX{%
  \Css{%
    span.HoLogo-XeTeX span.HoLogo-TeX{%
      margin-left:-.15em;%
    }%
  }%
  \global\let\HoLogoCss@XeTeX\relax
}
%    \end{macrocode}
%    \end{macro}
%
%    \begin{macro}{\HoLogo@XeLaTeX}
%    \begin{macrocode}
\def\HoLogo@XeLaTeX#1{%
  \hologo{Xe}%
  \kern-.13em%
  \hologo{LaTeX}%
}
%    \end{macrocode}
%    \end{macro}
%    \begin{macro}{\HoLogoHtml@XeLaTeX}
%    \begin{macrocode}
\def\HoLogoHtml@XeLaTeX#1{%
  \HoLogoCss@XeLaTeX
  \HOLOGO@Span{XeLaTeX}{%
    \hologo{Xe}%
    \hologo{LaTeX}%
  }%
}
%    \end{macrocode}
%    \end{macro}
%    \begin{macro}{\HoLogoCss@XeLaTeX}
%    \begin{macrocode}
\def\HoLogoCss@XeLaTeX{%
  \Css{%
    span.HoLogo-XeLaTeX span.HoLogo-Xe{%
      margin-right:-.13em;%
    }%
  }%
  \global\let\HoLogoCss@XeLaTeX\relax
}
%    \end{macrocode}
%    \end{macro}
%
% \subsubsection{\hologo{pdfTeX}, \hologo{pdfLaTeX}}
%
%    \begin{macro}{\HoLogo@pdfTeX}
%    \begin{macrocode}
\def\HoLogo@pdfTeX#1{%
  \HOLOGO@mbox{%
    #1{p}{P}df\hologo{TeX}%
  }%
}
%    \end{macrocode}
%    \end{macro}
%    \begin{macro}{\HoLogoCs@pdfTeX}
%    \begin{macrocode}
\def\HoLogoCs@pdfTeX#1{#1{p}{P}dfTeX}
%    \end{macrocode}
%    \end{macro}
%    \begin{macro}{\HoLogoBkm@pdfTeX}
%    \begin{macrocode}
\def\HoLogoBkm@pdfTeX#1{%
  #1{p}{P}df\hologo{TeX}%
}
%    \end{macrocode}
%    \end{macro}
%    \begin{macro}{\HoLogoHtml@pdfTeX}
%    \begin{macrocode}
\let\HoLogoHtml@pdfTeX\HoLogo@pdfTeX
%    \end{macrocode}
%    \end{macro}
%
%    \begin{macro}{\HoLogo@pdfLaTeX}
%    \begin{macrocode}
\def\HoLogo@pdfLaTeX#1{%
  \HOLOGO@mbox{%
    #1{p}{P}df\hologo{LaTeX}%
  }%
}
%    \end{macrocode}
%    \end{macro}
%    \begin{macro}{\HoLogoCs@pdfLaTeX}
%    \begin{macrocode}
\def\HoLogoCs@pdfLaTeX#1{#1{p}{P}dfLaTeX}
%    \end{macrocode}
%    \end{macro}
%    \begin{macro}{\HoLogoBkm@pdfLaTeX}
%    \begin{macrocode}
\def\HoLogoBkm@pdfLaTeX#1{%
  #1{p}{P}df\hologo{LaTeX}%
}
%    \end{macrocode}
%    \end{macro}
%    \begin{macro}{\HoLogoHtml@pdfLaTeX}
%    \begin{macrocode}
\let\HoLogoHtml@pdfLaTeX\HoLogo@pdfLaTeX
%    \end{macrocode}
%    \end{macro}
%
% \subsubsection{\hologo{VTeX}}
%
%    \begin{macro}{\HoLogo@VTeX}
%    \begin{macrocode}
\def\HoLogo@VTeX#1{%
  \HOLOGO@mbox{%
    V\hologo{TeX}%
  }%
}
%    \end{macrocode}
%    \end{macro}
%    \begin{macro}{\HoLogoHtml@VTeX}
%    \begin{macrocode}
\let\HoLogoHtml@VTeX\HoLogo@VTeX
%    \end{macrocode}
%    \end{macro}
%
% \subsubsection{\hologo{AmS}, \dots}
%
%    Source: class \xclass{amsdtx}
%
%    \begin{macro}{\HoLogo@AmS}
%    \begin{macrocode}
\def\HoLogo@AmS#1{%
  \HoLogoFont@font{AmS}{sy}{%
    A%
    \kern-.1667em%
    \lower.5ex\hbox{M}%
    \kern-.125em%
    S%
  }%
}
%    \end{macrocode}
%    \end{macro}
%    \begin{macro}{\HoLogoBkm@AmS}
%    \begin{macrocode}
\def\HoLogoBkm@AmS#1{AmS}
%    \end{macrocode}
%    \end{macro}
%    \begin{macro}{\HoLogoHtml@AmS}
%    \begin{macrocode}
\def\HoLogoHtml@AmS#1{%
  \HoLogoCss@AmS
%  \HoLogoFont@font{AmS}{sy}{%
    \HOLOGO@Span{AmS}{%
      A%
      \HOLOGO@Span{M}{M}%
      S%
    }%
%   }%
}
%    \end{macrocode}
%    \end{macro}
%    \begin{macro}{\HoLogoCss@AmS}
%    \begin{macrocode}
\def\HoLogoCss@AmS{%
  \Css{%
    span.HoLogo-AmS span.HoLogo-M{%
      position:relative;%
      top:.5ex;%
      margin-left:-.1667em;%
      margin-right:-.125em;%
      text-decoration:none;%
    }%
  }%
  \global\let\HoLogoCss@AmS\relax
}
%    \end{macrocode}
%    \end{macro}
%
%    \begin{macro}{\HoLogo@AmSTeX}
%    \begin{macrocode}
\def\HoLogo@AmSTeX#1{%
  \hologo{AmS}%
  \HOLOGO@hyphen
  \hologo{TeX}%
}
%    \end{macrocode}
%    \end{macro}
%    \begin{macro}{\HoLogoBkm@AmSTeX}
%    \begin{macrocode}
\def\HoLogoBkm@AmSTeX#1{AmS-TeX}%
%    \end{macrocode}
%    \end{macro}
%    \begin{macro}{\HoLogoHtml@AmSTeX}
%    \begin{macrocode}
\let\HoLogoHtml@AmSTeX\HoLogo@AmSTeX
%    \end{macrocode}
%    \end{macro}
%
%    \begin{macro}{\HoLogo@AmSLaTeX}
%    \begin{macrocode}
\def\HoLogo@AmSLaTeX#1{%
  \hologo{AmS}%
  \HOLOGO@hyphen
  \hologo{LaTeX}%
}
%    \end{macrocode}
%    \end{macro}
%    \begin{macro}{\HoLogoBkm@AmSLaTeX}
%    \begin{macrocode}
\def\HoLogoBkm@AmSLaTeX#1{AmS-LaTeX}%
%    \end{macrocode}
%    \end{macro}
%    \begin{macro}{\HoLogoHtml@AmSLaTeX}
%    \begin{macrocode}
\let\HoLogoHtml@AmSLaTeX\HoLogo@AmSLaTeX
%    \end{macrocode}
%    \end{macro}
%
% \subsubsection{\hologo{BibTeX}}
%
%    \begin{macro}{\HoLogo@BibTeX@sc}
%    A definition of \hologo{BibTeX} is provided in
%    the documentation source for the manual of \hologo{BibTeX}
%    \cite{btxdoc}.
%\begin{quote}
%\begin{verbatim}
%\def\BibTeX{%
%  {%
%    \rm
%    B%
%    \kern-.05em%
%    {%
%      \sc
%      i%
%      \kern-.025em %
%      b%
%    }%
%    \kern-.08em
%    T%
%    \kern-.1667em%
%    \lower.7ex\hbox{E}%
%    \kern-.125em%
%    X%
%  }%
%}
%\end{verbatim}
%\end{quote}
%    \begin{macrocode}
\def\HoLogo@BibTeX@sc#1{%
  B%
  \kern-.05em%
  \HoLogoFont@font{BibTeX}{sc}{%
    i%
    \kern-.025em%
    b%
  }%
  \HOLOGO@discretionary
  \kern-.08em%
  \hologo{TeX}%
}
%    \end{macrocode}
%    \end{macro}
%    \begin{macro}{\HoLogoHtml@BibTeX@sc}
%    \begin{macrocode}
\def\HoLogoHtml@BibTeX@sc#1{%
  \HoLogoCss@BibTeX@sc
  \HOLOGO@Span{BibTeX-sc}{%
    B%
    \HOLOGO@Span{i}{i}%
    \HOLOGO@Span{b}{b}%
    \hologo{TeX}%
  }%
}
%    \end{macrocode}
%    \end{macro}
%    \begin{macro}{\HoLogoCss@BibTeX@sc}
%    \begin{macrocode}
\def\HoLogoCss@BibTeX@sc{%
  \Css{%
    span.HoLogo-BibTeX-sc span.HoLogo-i{%
      margin-left:-.05em;%
      margin-right:-.025em;%
      font-variant:small-caps;%
    }%
  }%
  \Css{%
    span.HoLogo-BibTeX-sc span.HoLogo-b{%
      margin-right:-.08em;%
      font-variant:small-caps;%
    }%
  }%
  \global\let\HoLogoCss@BibTeX@sc\relax
}
%    \end{macrocode}
%    \end{macro}
%
%    \begin{macro}{\HoLogo@BibTeX@sf}
%    Variant \xoption{sf} avoids trouble with unavailable
%    small caps fonts (e.g., bold versions of Computer Modern or
%    Latin Modern). The definition is taken from
%    package \xpackage{dtklogos} \cite{dtklogos}.
%\begin{quote}
%\begin{verbatim}
%\DeclareRobustCommand{\BibTeX}{%
%  B%
%  \kern-.05em%
%  \hbox{%
%    $\m@th$% %% force math size calculations
%    \csname S@\f@size\endcsname
%    \fontsize\sf@size\z@
%    \math@fontsfalse
%    \selectfont
%    I%
%    \kern-.025em%
%    B
%  }%
%  \kern-.08em%
%  \-%
%  \TeX
%}
%\end{verbatim}
%\end{quote}
%    \begin{macrocode}
\def\HoLogo@BibTeX@sf#1{%
  B%
  \kern-.05em%
  \HoLogoFont@font{BibTeX}{bibsf}{%
    I%
    \kern-.025em%
    B%
  }%
  \HOLOGO@discretionary
  \kern-.08em%
  \hologo{TeX}%
}
%    \end{macrocode}
%    \end{macro}
%    \begin{macro}{\HoLogoHtml@BibTeX@sf}
%    \begin{macrocode}
\def\HoLogoHtml@BibTeX@sf#1{%
  \HoLogoCss@BibTeX@sf
  \HOLOGO@Span{BibTeX-sf}{%
    B%
    \HoLogoFont@font{BibTeX}{bibsf}{%
      \HOLOGO@Span{i}{I}%
      B%
    }%
    \hologo{TeX}%
  }%
}
%    \end{macrocode}
%    \end{macro}
%    \begin{macro}{\HoLogoCss@BibTeX@sf}
%    \begin{macrocode}
\def\HoLogoCss@BibTeX@sf{%
  \Css{%
    span.HoLogo-BibTeX-sf span.HoLogo-i{%
      margin-left:-.05em;%
      margin-right:-.025em;%
    }%
  }%
  \Css{%
    span.HoLogo-BibTeX-sf span.HoLogo-TeX{%
      margin-left:-.08em;%
    }%
  }%
  \global\let\HoLogoCss@BibTeX@sf\relax
}
%    \end{macrocode}
%    \end{macro}
%
%    \begin{macro}{\HoLogo@BibTeX}
%    \begin{macrocode}
\def\HoLogo@BibTeX{\HoLogo@BibTeX@sf}
%    \end{macrocode}
%    \end{macro}
%    \begin{macro}{\HoLogoHtml@BibTeX}
%    \begin{macrocode}
\def\HoLogoHtml@BibTeX{\HoLogoHtml@BibTeX@sf}
%    \end{macrocode}
%    \end{macro}
%
% \subsubsection{\hologo{BibTeX8}}
%
%    \begin{macro}{\HoLogo@BibTeX8}
%    \begin{macrocode}
\expandafter\def\csname HoLogo@BibTeX8\endcsname#1{%
  \hologo{BibTeX}%
  8%
}
%    \end{macrocode}
%    \end{macro}
%
%    \begin{macro}{\HoLogoBkm@BibTeX8}
%    \begin{macrocode}
\expandafter\def\csname HoLogoBkm@BibTeX8\endcsname#1{%
  \hologo{BibTeX}%
  8%
}
%    \end{macrocode}
%    \end{macro}
%    \begin{macro}{\HoLogoHtml@BibTeX8}
%    \begin{macrocode}
\expandafter
\let\csname HoLogoHtml@BibTeX8\expandafter\endcsname
\csname HoLogo@BibTeX8\endcsname
%    \end{macrocode}
%    \end{macro}
%
% \subsubsection{\hologo{ConTeXt}}
%
%    \begin{macro}{\HoLogo@ConTeXt@simple}
%    \begin{macrocode}
\def\HoLogo@ConTeXt@simple#1{%
  \HOLOGO@mbox{Con}%
  \HOLOGO@discretionary
  \HOLOGO@mbox{\hologo{TeX}t}%
}
%    \end{macrocode}
%    \end{macro}
%    \begin{macro}{\HoLogoHtml@ConTeXt@simple}
%    \begin{macrocode}
\let\HoLogoHtml@ConTeXt@simple\HoLogo@ConTeXt@simple
%    \end{macrocode}
%    \end{macro}
%
%    \begin{macro}{\HoLogo@ConTeXt@narrow}
%    This definition of logo \hologo{ConTeXt} with variant \xoption{narrow}
%    comes from TUGboat's class \xclass{ltugboat} (version 2010/11/15 v2.8).
%    \begin{macrocode}
\def\HoLogo@ConTeXt@narrow#1{%
  \HOLOGO@mbox{C\kern-.0333emon}%
  \HOLOGO@discretionary
  \kern-.0667em%
  \HOLOGO@mbox{\hologo{TeX}\kern-.0333emt}%
}
%    \end{macrocode}
%    \end{macro}
%    \begin{macro}{\HoLogoHtml@ConTeXt@narrow}
%    \begin{macrocode}
\def\HoLogoHtml@ConTeXt@narrow#1{%
  \HoLogoCss@ConTeXt@narrow
  \HOLOGO@Span{ConTeXt-narrow}{%
    \HOLOGO@Span{C}{C}%
    on%
    \hologo{TeX}%
    t%
  }%
}
%    \end{macrocode}
%    \end{macro}
%    \begin{macro}{\HoLogoCss@ConTeXt@narrow}
%    \begin{macrocode}
\def\HoLogoCss@ConTeXt@narrow{%
  \Css{%
    span.HoLogo-ConTeXt-narrow span.HoLogo-C{%
      margin-left:-.0333em;%
    }%
  }%
  \Css{%
    span.HoLogo-ConTeXt-narrow span.HoLogo-TeX{%
      margin-left:-.0667em;%
      margin-right:-.0333em;%
    }%
  }%
  \global\let\HoLogoCss@ConTeXt@narrow\relax
}
%    \end{macrocode}
%    \end{macro}
%
%    \begin{macro}{\HoLogo@ConTeXt}
%    \begin{macrocode}
\def\HoLogo@ConTeXt{\HoLogo@ConTeXt@narrow}
%    \end{macrocode}
%    \end{macro}
%    \begin{macro}{\HoLogoHtml@ConTeXt}
%    \begin{macrocode}
\def\HoLogoHtml@ConTeXt{\HoLogoHtml@ConTeXt@narrow}
%    \end{macrocode}
%    \end{macro}
%
% \subsubsection{\hologo{emTeX}}
%
%    \begin{macro}{\HoLogo@emTeX}
%    \begin{macrocode}
\def\HoLogo@emTeX#1{%
  \HOLOGO@mbox{#1{e}{E}m}%
  \HOLOGO@discretionary
  \hologo{TeX}%
}
%    \end{macrocode}
%    \end{macro}
%    \begin{macro}{\HoLogoCs@emTeX}
%    \begin{macrocode}
\def\HoLogoCs@emTeX#1{#1{e}{E}mTeX}%
%    \end{macrocode}
%    \end{macro}
%    \begin{macro}{\HoLogoBkm@emTeX}
%    \begin{macrocode}
\def\HoLogoBkm@emTeX#1{%
  #1{e}{E}m\hologo{TeX}%
}
%    \end{macrocode}
%    \end{macro}
%    \begin{macro}{\HoLogoHtml@emTeX}
%    \begin{macrocode}
\let\HoLogoHtml@emTeX\HoLogo@emTeX
%    \end{macrocode}
%    \end{macro}
%
% \subsubsection{\hologo{ExTeX}}
%
%    \begin{macro}{\HoLogo@ExTeX}
%    The definition is taken from the FAQ of the
%    project \hologo{ExTeX}
%    \cite{ExTeX-FAQ}.
%\begin{quote}
%\begin{verbatim}
%\def\ExTeX{%
%  \textrm{% Logo always with serifs
%    \ensuremath{%
%      \textstyle
%      \varepsilon_{%
%        \kern-0.15em%
%        \mathcal{X}%
%      }%
%    }%
%    \kern-.15em%
%    \TeX
%  }%
%}
%\end{verbatim}
%\end{quote}
%    \begin{macrocode}
\def\HoLogo@ExTeX#1{%
  \HoLogoFont@font{ExTeX}{rm}{%
    \ltx@mbox{%
      \HOLOGO@MathSetup
      $%
        \textstyle
        \varepsilon_{%
          \kern-0.15em%
          \HoLogoFont@font{ExTeX}{sy}{X}%
        }%
      $%
    }%
    \HOLOGO@discretionary
    \kern-.15em%
    \hologo{TeX}%
  }%
}
%    \end{macrocode}
%    \end{macro}
%    \begin{macro}{\HoLogoHtml@ExTeX}
%    \begin{macrocode}
\def\HoLogoHtml@ExTeX#1{%
  \HoLogoCss@ExTeX
  \HoLogoFont@font{ExTeX}{rm}{%
    \HOLOGO@Span{ExTeX}{%
      \ltx@mbox{%
        \HOLOGO@MathSetup
        $\textstyle\varepsilon$%
        \HOLOGO@Span{X}{$\textstyle\chi$}%
        \hologo{TeX}%
      }%
    }%
  }%
}
%    \end{macrocode}
%    \end{macro}
%    \begin{macro}{\HoLogoBkm@ExTeX}
%    \begin{macrocode}
\def\HoLogoBkm@ExTeX#1{%
  \HOLOGO@PdfdocUnicode{#1{e}{E}x}{\textepsilon\textchi}%
  \hologo{TeX}%
}
%    \end{macrocode}
%    \end{macro}
%    \begin{macro}{\HoLogoCss@ExTeX}
%    \begin{macrocode}
\def\HoLogoCss@ExTeX{%
  \Css{%
    span.HoLogo-ExTeX{%
      font-family:serif;%
    }%
  }%
  \Css{%
    span.HoLogo-ExTeX span.HoLogo-TeX{%
      margin-left:-.15em;%
    }%
  }%
  \global\let\HoLogoCss@ExTeX\relax
}
%    \end{macrocode}
%    \end{macro}
%
% \subsubsection{\hologo{MiKTeX}}
%
%    \begin{macro}{\HoLogo@MiKTeX}
%    \begin{macrocode}
\def\HoLogo@MiKTeX#1{%
  \HOLOGO@mbox{MiK}%
  \HOLOGO@discretionary
  \hologo{TeX}%
}
%    \end{macrocode}
%    \end{macro}
%    \begin{macro}{\HoLogoHtml@MiKTeX}
%    \begin{macrocode}
\let\HoLogoHtml@MiKTeX\HoLogo@MiKTeX
%    \end{macrocode}
%    \end{macro}
%
% \subsubsection{\hologo{OzTeX} and friends}
%
%    Source: \hologo{OzTeX} FAQ \cite{OzTeX}:
%    \begin{quote}
%      |\def\OzTeX{O\kern-.03em z\kern-.15em\TeX}|\\
%      (There is no kerning in OzMF, OzMP and OzTtH.)
%    \end{quote}
%
%    \begin{macro}{\HoLogo@OzTeX}
%    \begin{macrocode}
\def\HoLogo@OzTeX#1{%
  O%
  \kern-.03em %
  z%
  \kern-.15em %
  \hologo{TeX}%
}
%    \end{macrocode}
%    \end{macro}
%    \begin{macro}{\HoLogoHtml@OzTeX}
%    \begin{macrocode}
\def\HoLogoHtml@OzTeX#1{%
  \HoLogoCss@OzTeX
  \HOLOGO@Span{OzTeX}{%
    O%
    \HOLOGO@Span{z}{z}%
    \hologo{TeX}%
  }%
}
%    \end{macrocode}
%    \end{macro}
%    \begin{macro}{\HoLogoCss@OzTeX}
%    \begin{macrocode}
\def\HoLogoCss@OzTeX{%
  \Css{%
    span.HoLogo-OzTeX span.HoLogo-z{%
      margin-left:-.03em;%
      margin-right:-.15em;%
    }%
  }%
  \global\let\HoLogoCss@OzTeX\relax
}
%    \end{macrocode}
%    \end{macro}
%
%    \begin{macro}{\HoLogo@OzMF}
%    \begin{macrocode}
\def\HoLogo@OzMF#1{%
  \HOLOGO@mbox{OzMF}%
}
%    \end{macrocode}
%    \end{macro}
%    \begin{macro}{\HoLogo@OzMP}
%    \begin{macrocode}
\def\HoLogo@OzMP#1{%
  \HOLOGO@mbox{OzMP}%
}
%    \end{macrocode}
%    \end{macro}
%    \begin{macro}{\HoLogo@OzTtH}
%    \begin{macrocode}
\def\HoLogo@OzTtH#1{%
  \HOLOGO@mbox{OzTtH}%
}
%    \end{macrocode}
%    \end{macro}
%
% \subsubsection{\hologo{PCTeX}}
%
%    \begin{macro}{\HoLogo@PCTeX}
%    \begin{macrocode}
\def\HoLogo@PCTeX#1{%
  \HOLOGO@mbox{PC}%
  \hologo{TeX}%
}
%    \end{macrocode}
%    \end{macro}
%    \begin{macro}{\HoLogoHtml@PCTeX}
%    \begin{macrocode}
\let\HoLogoHtml@PCTeX\HoLogo@PCTeX
%    \end{macrocode}
%    \end{macro}
%
% \subsubsection{\hologo{PiCTeX}}
%
%    The original definitions from \xfile{pictex.tex} \cite{PiCTeX}:
%\begin{quote}
%\begin{verbatim}
%\def\PiC{%
%  P%
%  \kern-.12em%
%  \lower.5ex\hbox{I}%
%  \kern-.075em%
%  C%
%}
%\def\PiCTeX{%
%  \PiC
%  \kern-.11em%
%  \TeX
%}
%\end{verbatim}
%\end{quote}
%
%    \begin{macro}{\HoLogo@PiC}
%    \begin{macrocode}
\def\HoLogo@PiC#1{%
  P%
  \kern-.12em%
  \lower.5ex\hbox{I}%
  \kern-.075em%
  C%
  \HOLOGO@SpaceFactor
}
%    \end{macrocode}
%    \end{macro}
%    \begin{macro}{\HoLogoHtml@PiC}
%    \begin{macrocode}
\def\HoLogoHtml@PiC#1{%
  \HoLogoCss@PiC
  \HOLOGO@Span{PiC}{%
    P%
    \HOLOGO@Span{i}{I}%
    C%
  }%
}
%    \end{macrocode}
%    \end{macro}
%    \begin{macro}{\HoLogoCss@PiC}
%    \begin{macrocode}
\def\HoLogoCss@PiC{%
  \Css{%
    span.HoLogo-PiC span.HoLogo-i{%
      position:relative;%
      top:.5ex;%
      margin-left:-.12em;%
      margin-right:-.075em;%
      text-decoration:none;%
    }%
  }%
  \global\let\HoLogoCss@PiC\relax
}
%    \end{macrocode}
%    \end{macro}
%
%    \begin{macro}{\HoLogo@PiCTeX}
%    \begin{macrocode}
\def\HoLogo@PiCTeX#1{%
  \hologo{PiC}%
  \HOLOGO@discretionary
  \kern-.11em%
  \hologo{TeX}%
}
%    \end{macrocode}
%    \end{macro}
%    \begin{macro}{\HoLogoHtml@PiCTeX}
%    \begin{macrocode}
\def\HoLogoHtml@PiCTeX#1{%
  \HoLogoCss@PiCTeX
  \HOLOGO@Span{PiCTeX}{%
    \hologo{PiC}%
    \hologo{TeX}%
  }%
}
%    \end{macrocode}
%    \end{macro}
%    \begin{macro}{\HoLogoCss@PiCTeX}
%    \begin{macrocode}
\def\HoLogoCss@PiCTeX{%
  \Css{%
    span.HoLogo-PiCTeX span.HoLogo-PiC{%
      margin-right:-.11em;%
    }%
  }%
  \global\let\HoLogoCss@PiCTeX\relax
}
%    \end{macrocode}
%    \end{macro}
%
% \subsubsection{\hologo{teTeX}}
%
%    \begin{macro}{\HoLogo@teTeX}
%    \begin{macrocode}
\def\HoLogo@teTeX#1{%
  \HOLOGO@mbox{#1{t}{T}e}%
  \HOLOGO@discretionary
  \hologo{TeX}%
}
%    \end{macrocode}
%    \end{macro}
%    \begin{macro}{\HoLogoCs@teTeX}
%    \begin{macrocode}
\def\HoLogoCs@teTeX#1{#1{t}{T}dfTeX}
%    \end{macrocode}
%    \end{macro}
%    \begin{macro}{\HoLogoBkm@teTeX}
%    \begin{macrocode}
\def\HoLogoBkm@teTeX#1{%
  #1{t}{T}e\hologo{TeX}%
}
%    \end{macrocode}
%    \end{macro}
%    \begin{macro}{\HoLogoHtml@teTeX}
%    \begin{macrocode}
\let\HoLogoHtml@teTeX\HoLogo@teTeX
%    \end{macrocode}
%    \end{macro}
%
% \subsubsection{\hologo{TeX4ht}}
%
%    \begin{macro}{\HoLogo@TeX4ht}
%    \begin{macrocode}
\expandafter\def\csname HoLogo@TeX4ht\endcsname#1{%
  \HOLOGO@mbox{\hologo{TeX}4ht}%
}
%    \end{macrocode}
%    \end{macro}
%    \begin{macro}{\HoLogoHtml@TeX4ht}
%    \begin{macrocode}
\expandafter
\let\csname HoLogoHtml@TeX4ht\expandafter\endcsname
\csname HoLogo@TeX4ht\endcsname
%    \end{macrocode}
%    \end{macro}
%
%
% \subsubsection{\hologo{SageTeX}}
%
%    \begin{macro}{\HoLogo@SageTeX}
%    \begin{macrocode}
\def\HoLogo@SageTeX#1{%
  \HOLOGO@mbox{Sage}%
  \HOLOGO@discretionary
  \HOLOGO@NegativeKerning{eT,oT,To}%
  \hologo{TeX}%
}
%    \end{macrocode}
%    \end{macro}
%    \begin{macro}{\HoLogoHtml@SageTeX}
%    \begin{macrocode}
\let\HoLogoHtml@SageTeX\HoLogo@SageTeX
%    \end{macrocode}
%    \end{macro}
%
% \subsection{\hologo{METAFONT} and friends}
%
%    \begin{macro}{\HoLogo@METAFONT}
%    \begin{macrocode}
\def\HoLogo@METAFONT#1{%
  \HoLogoFont@font{METAFONT}{logo}{%
    \HOLOGO@mbox{META}%
    \HOLOGO@discretionary
    \HOLOGO@mbox{FONT}%
  }%
}
%    \end{macrocode}
%    \end{macro}
%
%    \begin{macro}{\HoLogo@METAPOST}
%    \begin{macrocode}
\def\HoLogo@METAPOST#1{%
  \HoLogoFont@font{METAPOST}{logo}{%
    \HOLOGO@mbox{META}%
    \HOLOGO@discretionary
    \HOLOGO@mbox{POST}%
  }%
}
%    \end{macrocode}
%    \end{macro}
%
%    \begin{macro}{\HoLogo@MetaFun}
%    \begin{macrocode}
\def\HoLogo@MetaFun#1{%
  \HOLOGO@mbox{Meta}%
  \HOLOGO@discretionary
  \HOLOGO@mbox{Fun}%
}
%    \end{macrocode}
%    \end{macro}
%
%    \begin{macro}{\HoLogo@MetaPost}
%    \begin{macrocode}
\def\HoLogo@MetaPost#1{%
  \HOLOGO@mbox{Meta}%
  \HOLOGO@discretionary
  \HOLOGO@mbox{Post}%
}
%    \end{macrocode}
%    \end{macro}
%
% \subsection{Others}
%
% \subsubsection{\hologo{biber}}
%
%    \begin{macro}{\HoLogo@biber}
%    \begin{macrocode}
\def\HoLogo@biber#1{%
  \HOLOGO@mbox{#1{b}{B}i}%
  \HOLOGO@discretionary
  \HOLOGO@mbox{ber}%
}
%    \end{macrocode}
%    \end{macro}
%    \begin{macro}{\HoLogoCs@biber}
%    \begin{macrocode}
\def\HoLogoCs@biber#1{#1{b}{B}iber}
%    \end{macrocode}
%    \end{macro}
%    \begin{macro}{\HoLogoBkm@biber}
%    \begin{macrocode}
\def\HoLogoBkm@biber#1{%
  #1{b}{B}iber%
}
%    \end{macrocode}
%    \end{macro}
%    \begin{macro}{\HoLogoHtml@biber}
%    \begin{macrocode}
\let\HoLogoHtml@biber\HoLogo@biber
%    \end{macrocode}
%    \end{macro}
%
% \subsubsection{\hologo{KOMAScript}}
%
%    \begin{macro}{\HoLogo@KOMAScript}
%    The definition for \hologo{KOMAScript} is taken
%    from \hologo{KOMAScript} (\xfile{scrlogo.dtx}, reformatted) \cite{scrlogo}:
%\begin{quote}
%\begin{verbatim}
%\@ifundefined{KOMAScript}{%
%  \DeclareRobustCommand{\KOMAScript}{%
%    \textsf{%
%      K\kern.05em O\kern.05emM\kern.05em A%
%      \kern.1em-\kern.1em %
%      Script%
%    }%
%  }%
%}{}
%\end{verbatim}
%\end{quote}
%    \begin{macrocode}
\def\HoLogo@KOMAScript#1{%
  \HoLogoFont@font{KOMAScript}{sf}{%
    \HOLOGO@mbox{%
      K\kern.05em%
      O\kern.05em%
      M\kern.05em%
      A%
    }%
    \kern.1em%
    \HOLOGO@hyphen
    \kern.1em%
    \HOLOGO@mbox{Script}%
  }%
}
%    \end{macrocode}
%    \end{macro}
%    \begin{macro}{\HoLogoBkm@KOMAScript}
%    \begin{macrocode}
\def\HoLogoBkm@KOMAScript#1{%
  KOMA-Script%
}
%    \end{macrocode}
%    \end{macro}
%    \begin{macro}{\HoLogoHtml@KOMAScript}
%    \begin{macrocode}
\def\HoLogoHtml@KOMAScript#1{%
  \HoLogoCss@KOMAScript
  \HoLogoFont@font{KOMAScript}{sf}{%
    \HOLOGO@Span{KOMAScript}{%
      K%
      \HOLOGO@Span{O}{O}%
      M%
      \HOLOGO@Span{A}{A}%
      \HOLOGO@Span{hyphen}{-}%
      Script%
    }%
  }%
}
%    \end{macrocode}
%    \end{macro}
%    \begin{macro}{\HoLogoCss@KOMAScript}
%    \begin{macrocode}
\def\HoLogoCss@KOMAScript{%
  \Css{%
    span.HoLogo-KOMAScript{%
      font-family:sans-serif;%
    }%
  }%
  \Css{%
    span.HoLogo-KOMAScript span.HoLogo-O{%
      padding-left:.05em;%
      padding-right:.05em;%
    }%
  }%
  \Css{%
    span.HoLogo-KOMAScript span.HoLogo-A{%
      padding-left:.05em;%
    }%
  }%
  \Css{%
    span.HoLogo-KOMAScript span.HoLogo-hyphen{%
      padding-left:.1em;%
      padding-right:.1em;%
    }%
  }%
  \global\let\HoLogoCss@KOMAScript\relax
}
%    \end{macrocode}
%    \end{macro}
%
% \subsubsection{\hologo{LyX}}
%
%    \begin{macro}{\HoLogo@LyX}
%    The definition is taken from the documentation source files
%    of \hologo{LyX}, \xfile{Intro.lyx} \cite{LyX}:
%\begin{quote}
%\begin{verbatim}
%\def\LyX{%
%  \texorpdfstring{%
%    L\kern-.1667em\lower.25em\hbox{Y}\kern-.125emX\@%
%  }{%
%    LyX%
%  }%
%}
%\end{verbatim}
%\end{quote}
%    \begin{macrocode}
\def\HoLogo@LyX#1{%
  L%
  \kern-.1667em%
  \lower.25em\hbox{Y}%
  \kern-.125em%
  X%
  \HOLOGO@SpaceFactor
}
%    \end{macrocode}
%    \end{macro}
%    \begin{macro}{\HoLogoHtml@LyX}
%    \begin{macrocode}
\def\HoLogoHtml@LyX#1{%
  \HoLogoCss@LyX
  \HOLOGO@Span{LyX}{%
    L%
    \HOLOGO@Span{y}{Y}%
    X%
  }%
}
%    \end{macrocode}
%    \end{macro}
%    \begin{macro}{\HoLogoCss@LyX}
%    \begin{macrocode}
\def\HoLogoCss@LyX{%
  \Css{%
    span.HoLogo-LyX span.HoLogo-y{%
      position:relative;%
      top:.25em;%
      margin-left:-.1667em;%
      margin-right:-.125em;%
      text-decoration:none;%
    }%
  }%
  \global\let\HoLogoCss@LyX\relax
}
%    \end{macrocode}
%    \end{macro}
%
% \subsubsection{\hologo{NTS}}
%
%    \begin{macro}{\HoLogo@NTS}
%    Definition for \hologo{NTS} can be found in
%    package \xpackage{etex\textunderscore man} for the \hologo{eTeX} manual \cite{etexman}
%    and in package \xpackage{dtklogos} \cite{dtklogos}:
%\begin{quote}
%\begin{verbatim}
%\def\NTS{%
%  \leavevmode
%  \hbox{%
%    $%
%      \cal N%
%      \kern-0.35em%
%      \lower0.5ex\hbox{$\cal T$}%
%      \kern-0.2em%
%      S%
%    $%
%  }%
%}
%\end{verbatim}
%\end{quote}
%    \begin{macrocode}
\def\HoLogo@NTS#1{%
  \HoLogoFont@font{NTS}{sy}{%
    N\/%
    \kern-.35em%
    \lower.5ex\hbox{T\/}%
    \kern-.2em%
    S\/%
  }%
  \HOLOGO@SpaceFactor
}
%    \end{macrocode}
%    \end{macro}
%
% \subsubsection{\Hologo{TTH} (\hologo{TeX} to HTML translator)}
%
%    Source: \url{http://hutchinson.belmont.ma.us/tth/}
%    In the HTML source the second `T' is printed as subscript.
%\begin{quote}
%\begin{verbatim}
%T<sub>T</sub>H
%\end{verbatim}
%\end{quote}
%    \begin{macro}{\HoLogo@TTH}
%    \begin{macrocode}
\def\HoLogo@TTH#1{%
  \ltx@mbox{%
    T\HOLOGO@SubScript{T}H%
  }%
  \HOLOGO@SpaceFactor
}
%    \end{macrocode}
%    \end{macro}
%
%    \begin{macro}{\HoLogoHtml@TTH}
%    \begin{macrocode}
\def\HoLogoHtml@TTH#1{%
  T\HCode{<sub>}T\HCode{</sub>}H%
}
%    \end{macrocode}
%    \end{macro}
%
% \subsubsection{\Hologo{HanTheThanh}}
%
%    Partial source: Package \xpackage{dtklogos}.
%    The double accent is U+1EBF (latin small letter e with circumflex
%    and acute).
%    \begin{macro}{\HoLogo@HanTheThanh}
%    \begin{macrocode}
\def\HoLogo@HanTheThanh#1{%
  \ltx@mbox{H\`an}%
  \HOLOGO@space
  \ltx@mbox{%
    Th%
    \HOLOGO@IfCharExists{"1EBF}{%
      \char"1EBF\relax
    }{%
      \^e\hbox to 0pt{\hss\raise .5ex\hbox{\'{}}}%
    }%
  }%
  \HOLOGO@space
  \ltx@mbox{Th\`anh}%
}
%    \end{macrocode}
%    \end{macro}
%    \begin{macro}{\HoLogoBkm@HanTheThanh}
%    \begin{macrocode}
\def\HoLogoBkm@HanTheThanh#1{%
  H\`an %
  Th\HOLOGO@PdfdocUnicode{\^e}{\9036\277} %
  Th\`anh%
}
%    \end{macrocode}
%    \end{macro}
%    \begin{macro}{\HoLogoHtml@HanTheThanh}
%    \begin{macrocode}
\def\HoLogoHtml@HanTheThanh#1{%
  H\`an %
  Th\HCode{&\ltx@hashchar x1ebf;} %
  Th\`anh%
}
%    \end{macrocode}
%    \end{macro}
%
% \subsection{Driver detection}
%
%    \begin{macrocode}
\HOLOGO@IfExists\InputIfFileExists{%
  \InputIfFileExists{hologo.cfg}{}{}%
}{%
  \ltx@IfUndefined{pdf@filesize}{%
    \def\HOLOGO@InputIfExists{%
      \openin\HOLOGO@temp=hologo.cfg\relax
      \ifeof\HOLOGO@temp
        \closein\HOLOGO@temp
      \else
        \closein\HOLOGO@temp
        \begingroup
          \def\x{LaTeX2e}%
        \expandafter\endgroup
        \ifx\fmtname\x
          \input{hologo.cfg}%
        \else
          \input hologo.cfg\relax
        \fi
      \fi
    }%
    \ltx@IfUndefined{newread}{%
      \chardef\HOLOGO@temp=15 %
      \def\HOLOGO@CheckRead{%
        \ifeof\HOLOGO@temp
          \HOLOGO@InputIfExists
        \else
          \ifcase\HOLOGO@temp
            \@PackageWarningNoLine{hologo}{%
              Configuration file ignored, because\MessageBreak
              a free read register could not be found%
            }%
          \else
            \begingroup
              \count\ltx@cclv=\HOLOGO@temp
              \advance\ltx@cclv by \ltx@minusone
              \edef\x{\endgroup
                \chardef\noexpand\HOLOGO@temp=\the\count\ltx@cclv
                \relax
              }%
            \x
          \fi
        \fi
      }%
    }{%
      \csname newread\endcsname\HOLOGO@temp
      \HOLOGO@InputIfExists
    }%
  }{%
    \edef\HOLOGO@temp{\pdf@filesize{hologo.cfg}}%
    \ifx\HOLOGO@temp\ltx@empty
    \else
      \ifnum\HOLOGO@temp>0 %
        \begingroup
          \def\x{LaTeX2e}%
        \expandafter\endgroup
        \ifx\fmtname\x
          \input{hologo.cfg}%
        \else
          \input hologo.cfg\relax
        \fi
      \else
        \@PackageInfoNoLine{hologo}{%
          Empty configuration file `hologo.cfg' ignored%
        }%
      \fi
    \fi
  }%
}
%    \end{macrocode}
%
%    \begin{macrocode}
\def\HOLOGO@temp#1#2{%
  \kv@define@key{HoLogoDriver}{#1}[]{%
    \begingroup
      \def\HOLOGO@temp{##1}%
      \ltx@onelevel@sanitize\HOLOGO@temp
      \ifx\HOLOGO@temp\ltx@empty
      \else
        \@PackageError{hologo}{%
          Value (\HOLOGO@temp) not permitted for option `#1'%
        }%
        \@ehc
      \fi
    \endgroup
    \def\hologoDriver{#2}%
  }%
}%
\def\HOLOGO@@temp#1#2{%
  \ifx\kv@value\relax
    \HOLOGO@temp{#1}{#1}%
  \else
    \HOLOGO@temp{#1}{#2}%
  \fi
}%
\kv@parse@normalized{%
  pdftex,%
  luatex=pdftex,%
  dvipdfm,%
  dvipdfmx=dvipdfm,%
  dvips,%
  dvipsone=dvips,%
  xdvi=dvips,%
  xetex,%
  vtex,%
}\HOLOGO@@temp
%    \end{macrocode}
%
%    \begin{macrocode}
\kv@define@key{HoLogoDriver}{driverfallback}{%
  \def\HOLOGO@DriverFallback{#1}%
}
%    \end{macrocode}
%
%    \begin{macro}{\HOLOGO@DriverFallback}
%    \begin{macrocode}
\def\HOLOGO@DriverFallback{dvips}
%    \end{macrocode}
%    \end{macro}
%
%    \begin{macro}{\hologoDriverSetup}
%    \begin{macrocode}
\def\hologoDriverSetup{%
  \let\hologoDriver\ltx@undefined
  \HOLOGO@DriverSetup
}
%    \end{macrocode}
%    \end{macro}
%
%    \begin{macro}{\HOLOGO@DriverSetup}
%    \begin{macrocode}
\def\HOLOGO@DriverSetup#1{%
  \kvsetkeys{HoLogoDriver}{#1}%
  \HOLOGO@CheckDriver
  \ltx@ifundefined{hologoDriver}{%
    \begingroup
    \edef\x{\endgroup
      \noexpand\kvsetkeys{HoLogoDriver}{\HOLOGO@DriverFallback}%
    }\x
  }{}%
  \@PackageInfoNoLine{hologo}{Using driver `\hologoDriver'}%
}
%    \end{macrocode}
%    \end{macro}
%
%    \begin{macro}{\HOLOGO@CheckDriver}
%    \begin{macrocode}
\def\HOLOGO@CheckDriver{%
  \ifpdf
    \def\hologoDriver{pdftex}%
    \let\HOLOGO@pdfliteral\pdfliteral
    \ifluatex
      \ifx\pdfextension\@undefined\else
        \protected\def\pdfliteral{\pdfextension literal}%
        \let\HOLOGO@pdfliteral\pdfliteral
      \fi
      \ltx@IfUndefined{HOLOGO@pdfliteral}{%
        \ifnum\luatexversion<36 %
        \else
          \begingroup
            \let\HOLOGO@temp\endgroup
            \ifcase0%
                \directlua{%
                  if tex.enableprimitives then %
                    tex.enableprimitives('HOLOGO@', {'pdfliteral'})%
                  else %
                    tex.print('1')%
                  end%
                }%
                \ifx\HOLOGO@pdfliteral\@undefined 1\fi%
                \relax%
              \endgroup
              \let\HOLOGO@temp\relax
              \global\let\HOLOGO@pdfliteral\HOLOGO@pdfliteral
            \fi%
          \HOLOGO@temp
        \fi
      }{}%
    \fi
    \ltx@IfUndefined{HOLOGO@pdfliteral}{%
      \@PackageWarningNoLine{hologo}{%
        Cannot find \string\pdfliteral
      }%
    }{}%
  \else
    \ifxetex
      \def\hologoDriver{xetex}%
    \else
      \ifvtex
        \def\hologoDriver{vtex}%
      \fi
    \fi
  \fi
}
%    \end{macrocode}
%    \end{macro}
%
%    \begin{macro}{\HOLOGO@WarningUnsupportedDriver}
%    \begin{macrocode}
\def\HOLOGO@WarningUnsupportedDriver#1{%
  \@PackageWarningNoLine{hologo}{%
    Logo `#1' needs driver specific macros,\MessageBreak
    but driver `\hologoDriver' is not supported.\MessageBreak
    Use a different driver or\MessageBreak
    load package `graphics' or `pgf'%
  }%
}
%    \end{macrocode}
%    \end{macro}
%
% \subsubsection{Reflect box macros}
%
%    Skip driver part if not needed.
%    \begin{macrocode}
\ltx@IfUndefined{reflectbox}{}{%
  \ltx@IfUndefined{rotatebox}{}{%
    \HOLOGO@AtEnd
  }%
}
\ltx@IfUndefined{pgftext}{}{%
  \HOLOGO@AtEnd
}
\ltx@IfUndefined{psscalebox}{}{%
  \HOLOGO@AtEnd
}
%    \end{macrocode}
%
%    \begin{macrocode}
\def\HOLOGO@temp{LaTeX2e}
\ifx\fmtname\HOLOGO@temp
  \RequirePackage{kvoptions}[2011/06/30]%
  \ProcessKeyvalOptions{HoLogoDriver}%
\fi
\HOLOGO@DriverSetup{}
%    \end{macrocode}
%
%    \begin{macro}{\HOLOGO@ReflectBox}
%    \begin{macrocode}
\def\HOLOGO@ReflectBox#1{%
  \begingroup
    \setbox\ltx@zero\hbox{\begingroup#1\endgroup}%
    \setbox\ltx@two\hbox{%
      \kern\wd\ltx@zero
      \csname HOLOGO@ScaleBox@\hologoDriver\endcsname{-1}{1}{%
        \hbox to 0pt{\copy\ltx@zero\hss}%
      }%
    }%
    \wd\ltx@two=\wd\ltx@zero
    \box\ltx@two
  \endgroup
}
%    \end{macrocode}
%    \end{macro}
%
%    \begin{macro}{\HOLOGO@PointReflectBox}
%    \begin{macrocode}
\def\HOLOGO@PointReflectBox#1{%
  \begingroup
    \setbox\ltx@zero\hbox{\begingroup#1\endgroup}%
    \setbox\ltx@two\hbox{%
      \kern\wd\ltx@zero
      \raise\ht\ltx@zero\hbox{%
        \csname HOLOGO@ScaleBox@\hologoDriver\endcsname{-1}{-1}{%
          \hbox to 0pt{\copy\ltx@zero\hss}%
        }%
      }%
    }%
    \wd\ltx@two=\wd\ltx@zero
    \box\ltx@two
  \endgroup
}
%    \end{macrocode}
%    \end{macro}
%
%    We must define all variants because of dynamic driver setup.
%    \begin{macrocode}
\def\HOLOGO@temp#1#2{#2}
%    \end{macrocode}
%
%    \begin{macro}{\HOLOGO@ScaleBox@pdftex}
%    \begin{macrocode}
\HOLOGO@temp{pdftex}{%
  \def\HOLOGO@ScaleBox@pdftex#1#2#3{%
    \HOLOGO@pdfliteral{%
      q #1 0 0 #2 0 0 cm%
    }%
    #3%
    \HOLOGO@pdfliteral{%
      Q%
    }%
  }%
}
%    \end{macrocode}
%    \end{macro}
%    \begin{macro}{\HOLOGO@ScaleBox@dvips}
%    \begin{macrocode}
\HOLOGO@temp{dvips}{%
  \def\HOLOGO@ScaleBox@dvips#1#2#3{%
    \special{ps:%
      gsave %
      currentpoint %
      currentpoint translate %
      #1 #2 scale %
      neg exch neg exch translate%
    }%
    #3%
    \special{ps:%
      currentpoint %
      grestore %
      moveto%
    }%
  }%
}
%    \end{macrocode}
%    \end{macro}
%    \begin{macro}{\HOLOGO@ScaleBox@dvipdfm}
%    \begin{macrocode}
\HOLOGO@temp{dvipdfm}{%
  \let\HOLOGO@ScaleBox@dvipdfm\HOLOGO@ScaleBox@dvips
}
%    \end{macrocode}
%    \end{macro}
%    Since \hologo{XeTeX} v0.6.
%    \begin{macro}{\HOLOGO@ScaleBox@xetex}
%    \begin{macrocode}
\HOLOGO@temp{xetex}{%
  \def\HOLOGO@ScaleBox@xetex#1#2#3{%
    \special{x:gsave}%
    \special{x:scale #1 #2}%
    #3%
    \special{x:grestore}%
  }%
}
%    \end{macrocode}
%    \end{macro}
%    \begin{macro}{\HOLOGO@ScaleBox@vtex}
%    \begin{macrocode}
\HOLOGO@temp{vtex}{%
  \def\HOLOGO@ScaleBox@vtex#1#2#3{%
    \special{r(#1,0,0,#2,0,0}%
    #3%
    \special{r)}%
  }%
}
%    \end{macrocode}
%    \end{macro}
%
%    \begin{macrocode}
\HOLOGO@AtEnd%
%</package>
%    \end{macrocode}
%
% \section{Test}
%
% \subsection{Catcode checks for loading}
%
%    \begin{macrocode}
%<*test1>
%    \end{macrocode}
%    \begin{macrocode}
\catcode`\{=1 %
\catcode`\}=2 %
\catcode`\#=6 %
\catcode`\@=11 %
\expandafter\ifx\csname count@\endcsname\relax
  \countdef\count@=255 %
\fi
\expandafter\ifx\csname @gobble\endcsname\relax
  \long\def\@gobble#1{}%
\fi
\expandafter\ifx\csname @firstofone\endcsname\relax
  \long\def\@firstofone#1{#1}%
\fi
\expandafter\ifx\csname loop\endcsname\relax
  \expandafter\@firstofone
\else
  \expandafter\@gobble
\fi
{%
  \def\loop#1\repeat{%
    \def\body{#1}%
    \iterate
  }%
  \def\iterate{%
    \body
      \let\next\iterate
    \else
      \let\next\relax
    \fi
    \next
  }%
  \let\repeat=\fi
}%
\def\RestoreCatcodes{}
\count@=0 %
\loop
  \edef\RestoreCatcodes{%
    \RestoreCatcodes
    \catcode\the\count@=\the\catcode\count@\relax
  }%
\ifnum\count@<255 %
  \advance\count@ 1 %
\repeat

\def\RangeCatcodeInvalid#1#2{%
  \count@=#1\relax
  \loop
    \catcode\count@=15 %
  \ifnum\count@<#2\relax
    \advance\count@ 1 %
  \repeat
}
\def\RangeCatcodeCheck#1#2#3{%
  \count@=#1\relax
  \loop
    \ifnum#3=\catcode\count@
    \else
      \errmessage{%
        Character \the\count@\space
        with wrong catcode \the\catcode\count@\space
        instead of \number#3%
      }%
    \fi
  \ifnum\count@<#2\relax
    \advance\count@ 1 %
  \repeat
}
\def\space{ }
\expandafter\ifx\csname LoadCommand\endcsname\relax
  \def\LoadCommand{\input hologo.sty\relax}%
\fi
\def\Test{%
  \RangeCatcodeInvalid{0}{47}%
  \RangeCatcodeInvalid{58}{64}%
  \RangeCatcodeInvalid{91}{96}%
  \RangeCatcodeInvalid{123}{255}%
  \catcode`\@=12 %
  \catcode`\\=0 %
  \catcode`\%=14 %
  \LoadCommand
  \RangeCatcodeCheck{0}{36}{15}%
  \RangeCatcodeCheck{37}{37}{14}%
  \RangeCatcodeCheck{38}{47}{15}%
  \RangeCatcodeCheck{48}{57}{12}%
  \RangeCatcodeCheck{58}{63}{15}%
  \RangeCatcodeCheck{64}{64}{12}%
  \RangeCatcodeCheck{65}{90}{11}%
  \RangeCatcodeCheck{91}{91}{15}%
  \RangeCatcodeCheck{92}{92}{0}%
  \RangeCatcodeCheck{93}{96}{15}%
  \RangeCatcodeCheck{97}{122}{11}%
  \RangeCatcodeCheck{123}{255}{15}%
  \RestoreCatcodes
}
\Test
\csname @@end\endcsname
\end
%    \end{macrocode}
%    \begin{macrocode}
%</test1>
%    \end{macrocode}
%
% \subsection{Spacefactor}
%
%    The space factor must be 1000 after a logo. If it is greater 1000
%    then the following space is a space after a sentence closing point.
%    If the space factor is smaller 1000 then an immediate following
%    dot is interpreted as abbreviation, not sentence closing point.
%
%    \begin{macrocode}
%<*test-spacefactor>
\NeedsTeXFormat{LaTeX2e}
\documentclass{article}
\usepackage{hologo}[2016/05/12]
\usepackage{kvsetkeys}
\usepackage{qstest}
\IncludeTests{*}
\LogTests{log}{*}{*}
\begin{document}
\begin{qstest}{spacefactor}{spacefactor}
\newcommand*{\Test}[1]{%
  \sbox0{%
    \hologo{#1}%
    \Expect*{1000 (#1)}*{\the\spacefactor\space(#1)}%
  }%
}%
\makeatletter
\def\TestList{}
\def\hologoEntry#1#2#3{%
  \edef\TestList{%
    \ifx\TestList\@empty
    \else
      \TestList,%
    \fi
    #1%
    \ifx\\#2\\%
    \else
      ={variant=#2}%
    \fi
  }%
}
\hologoList
\expandafter\kv@parse@normalized\expandafter{%
  \TestList
}{%
  \begingroup
    \let\@logo=\kv@key
    \ifx\kv@value\relax
    \else
      \expandafter\hologoLogoSetup\expandafter\@logo\expandafter{%
        \kv@value
      }%
    \fi
    \Test\@logo
  \endgroup
  \@gobbletwo
}
\end{qstest}
\end{document}
%</test-spacefactor>
%    \end{macrocode}
%
% \subsection{Complete list}
%
%    \begin{macrocode}
%<*test-list>
\NeedsTeXFormat{LaTeX2e}
\documentclass[12pt,a4paper]{article}
\usepackage{hologo}[2016/05/12]
\usepackage[T1]{fontenc}
\usepackage{lmodern}
\usepackage{parskip}
\usepackage[unicode]{hyperref}[2011/09/28]
\usepackage{bookmark}[2011/09/19]
\bookmarksetup{%
  numbered,%
  open,%
  openlevel=2,%
}
\renewcommand*{\contentsname}{List of logos}
\begin{document}
\tableofcontents
\def\TestFont#1#2#3#4#5#6{%
  \begingroup
    \usefont{#3}{#4}{#5}{#6}%
    \HologoVariant{#1}{#2}/\hologoVariant{#1}{#2}%
    \quad
    \begingroup\scriptsize\hologoVariant{#1}{#2}\endgroup
    \quad
  \endgroup
  (#3/#4/#5/#6)%
  \par
}
\makeatletter
\def\hologoEntry#1#2#3{%
  \section{%
    \HologoVariant{#1}{#2}/\hologoVariant{#1}{#2} %
    {[#1\ifx\\#2\\\else\space(#2)\fi]}% hash-ok
  }% braces around [] because of bug in tex4ht
  \begingroup
    \hypersetup{unicode=false}%
    \bookmark[%
      dest=\@currentHref,%
      rellevel=1,%
      keeplevel,%
    ]{%
      \HologoVariant{#1}{#2}/\hologoVariant{#1}{#2} %
      (PDFDocEncoding)%
    }%
  \endgroup
  \TestFont{#1}{#2}{OT1}{cmr}{m}{n}%
  \TestFont{#1}{#2}{OT1}{cmss}{m}{n}%
  \TestFont{#1}{#2}{OT1}{cmr}{b}{n}%
  \TestFont{#1}{#2}{OT1}{cmr}{m}{it}%
  \TestFont{#1}{#2}{OT1}{cmtt}{m}{n}%
  \TestFont{#1}{#2}{T1}{lmr}{m}{n}%
  \TestFont{#1}{#2}{T1}{lmss}{m}{n}%
  \TestFont{#1}{#2}{T1}{lmr}{b}{n}%
  \TestFont{#1}{#2}{T1}{lmr}{m}{it}%
  \TestFont{#1}{#2}{T1}{lmtt}{m}{n}%
  \TestFont{#1}{#2}{T1}{lmvtt}{m}{n}%
  \TestFont{#1}{#2}{T1}{qtm}{m}{n}%
  \TestFont{#1}{#2}{T1}{qhv}{m}{n}%
  \TestFont{#1}{#2}{T1}{qtm}{b}{n}%
  \TestFont{#1}{#2}{T1}{qtm}{m}{it}%
  \TestFont{#1}{#2}{T1}{qcr}{m}{n}%
  \newpage
}
\makeatother
\hologoList
\end{document}
%</test-list>
%    \end{macrocode}
%
% \section{Installation}
%
% \subsection{Download}
%
% \paragraph{Package.} This package is available on
% CTAN\footnote{\url{ftp://ftp.ctan.org/tex-archive/}}:
% \begin{description}
% \item[\CTAN{macros/latex/contrib/oberdiek/hologo.dtx}] The source file.
% \item[\CTAN{macros/latex/contrib/oberdiek/hologo.pdf}] Documentation.
% \end{description}
%
%
% \paragraph{Bundle.} All the packages of the bundle `oberdiek'
% are also available in a TDS compliant ZIP archive. There
% the packages are already unpacked and the documentation files
% are generated. The files and directories obey the TDS standard.
% \begin{description}
% \item[\CTAN{install/macros/latex/contrib/oberdiek.tds.zip}]
% \end{description}
% \emph{TDS} refers to the standard ``A Directory Structure
% for \TeX\ Files'' (\CTAN{tds/tds.pdf}). Directories
% with \xfile{texmf} in their name are usually organized this way.
%
% \subsection{Bundle installation}
%
% \paragraph{Unpacking.} Unpack the \xfile{oberdiek.tds.zip} in the
% TDS tree (also known as \xfile{texmf} tree) of your choice.
% Example (linux):
% \begin{quote}
%   |unzip oberdiek.tds.zip -d ~/texmf|
% \end{quote}
%
% \paragraph{Script installation.}
% Check the directory \xfile{TDS:scripts/oberdiek/} for
% scripts that need further installation steps.
% Package \xpackage{attachfile2} comes with the Perl script
% \xfile{pdfatfi.pl} that should be installed in such a way
% that it can be called as \texttt{pdfatfi}.
% Example (linux):
% \begin{quote}
%   |chmod +x scripts/oberdiek/pdfatfi.pl|\\
%   |cp scripts/oberdiek/pdfatfi.pl /usr/local/bin/|
% \end{quote}
%
% \subsection{Package installation}
%
% \paragraph{Unpacking.} The \xfile{.dtx} file is a self-extracting
% \docstrip\ archive. The files are extracted by running the
% \xfile{.dtx} through \plainTeX:
% \begin{quote}
%   \verb|tex hologo.dtx|
% \end{quote}
%
% \paragraph{TDS.} Now the different files must be moved into
% the different directories in your installation TDS tree
% (also known as \xfile{texmf} tree):
% \begin{quote}
% \def\t{^^A
% \begin{tabular}{@{}>{\ttfamily}l@{ $\rightarrow$ }>{\ttfamily}l@{}}
%   hologo.sty & tex/generic/oberdiek/hologo.sty\\
%   hologo.pdf & doc/latex/oberdiek/hologo.pdf\\
%   example/hologo-example.tex & doc/latex/oberdiek/example/hologo-example.tex\\
%   test/hologo-test1.tex & doc/latex/oberdiek/test/hologo-test1.tex\\
%   test/hologo-test-spacefactor.tex & doc/latex/oberdiek/test/hologo-test-spacefactor.tex\\
%   test/hologo-test-list.tex & doc/latex/oberdiek/test/hologo-test-list.tex\\
%   hologo.dtx & source/latex/oberdiek/hologo.dtx\\
% \end{tabular}^^A
% }^^A
% \sbox0{\t}^^A
% \ifdim\wd0>\linewidth
%   \begingroup
%     \advance\linewidth by\leftmargin
%     \advance\linewidth by\rightmargin
%   \edef\x{\endgroup
%     \def\noexpand\lw{\the\linewidth}^^A
%   }\x
%   \def\lwbox{^^A
%     \leavevmode
%     \hbox to \linewidth{^^A
%       \kern-\leftmargin\relax
%       \hss
%       \usebox0
%       \hss
%       \kern-\rightmargin\relax
%     }^^A
%   }^^A
%   \ifdim\wd0>\lw
%     \sbox0{\small\t}^^A
%     \ifdim\wd0>\linewidth
%       \ifdim\wd0>\lw
%         \sbox0{\footnotesize\t}^^A
%         \ifdim\wd0>\linewidth
%           \ifdim\wd0>\lw
%             \sbox0{\scriptsize\t}^^A
%             \ifdim\wd0>\linewidth
%               \ifdim\wd0>\lw
%                 \sbox0{\tiny\t}^^A
%                 \ifdim\wd0>\linewidth
%                   \lwbox
%                 \else
%                   \usebox0
%                 \fi
%               \else
%                 \lwbox
%               \fi
%             \else
%               \usebox0
%             \fi
%           \else
%             \lwbox
%           \fi
%         \else
%           \usebox0
%         \fi
%       \else
%         \lwbox
%       \fi
%     \else
%       \usebox0
%     \fi
%   \else
%     \lwbox
%   \fi
% \else
%   \usebox0
% \fi
% \end{quote}
% If you have a \xfile{docstrip.cfg} that configures and enables \docstrip's
% TDS installing feature, then some files can already be in the right
% place, see the documentation of \docstrip.
%
% \subsection{Refresh file name databases}
%
% If your \TeX~distribution
% (\teTeX, \mikTeX, \dots) relies on file name databases, you must refresh
% these. For example, \teTeX\ users run \verb|texhash| or
% \verb|mktexlsr|.
%
% \subsection{Some details for the interested}
%
% \paragraph{Attached source.}
%
% The PDF documentation on CTAN also includes the
% \xfile{.dtx} source file. It can be extracted by
% AcrobatReader 6 or higher. Another option is \textsf{pdftk},
% e.g. unpack the file into the current directory:
% \begin{quote}
%   \verb|pdftk hologo.pdf unpack_files output .|
% \end{quote}
%
% \paragraph{Unpacking with \LaTeX.}
% The \xfile{.dtx} chooses its action depending on the format:
% \begin{description}
% \item[\plainTeX:] Run \docstrip\ and extract the files.
% \item[\LaTeX:] Generate the documentation.
% \end{description}
% If you insist on using \LaTeX\ for \docstrip\ (really,
% \docstrip\ does not need \LaTeX), then inform the autodetect routine
% about your intention:
% \begin{quote}
%   \verb|latex \let\install=y\input{hologo.dtx}|
% \end{quote}
% Do not forget to quote the argument according to the demands
% of your shell.
%
% \paragraph{Generating the documentation.}
% You can use both the \xfile{.dtx} or the \xfile{.drv} to generate
% the documentation. The process can be configured by the
% configuration file \xfile{ltxdoc.cfg}. For instance, put this
% line into this file, if you want to have A4 as paper format:
% \begin{quote}
%   \verb|\PassOptionsToClass{a4paper}{article}|
% \end{quote}
% An example follows how to generate the
% documentation with pdf\LaTeX:
% \begin{quote}
%\begin{verbatim}
%pdflatex hologo.dtx
%makeindex -s gind.ist hologo.idx
%pdflatex hologo.dtx
%makeindex -s gind.ist hologo.idx
%pdflatex hologo.dtx
%\end{verbatim}
% \end{quote}
%
% \section{Catalogue}
%
% The following XML file can be used as source for the
% \href{http://mirror.ctan.org/help/Catalogue/catalogue.html}{\TeX\ Catalogue}.
% The elements \texttt{caption} and \texttt{description} are imported
% from the original XML file from the Catalogue.
% The name of the XML file in the Catalogue is \xfile{hologo.xml}.
%    \begin{macrocode}
%<*catalogue>
<?xml version='1.0' encoding='us-ascii'?>
<!DOCTYPE entry SYSTEM 'catalogue.dtd'>
<entry datestamp='$Date$' modifier='$Author$' id='hologo'>
  <name>hologo</name>
  <caption>A collection of logos with bookmark support.</caption>
  <authorref id='auth:oberdiek'/>
  <copyright owner='Heiko Oberdiek' year='2010-2012'/>
  <license type='lppl1.3'/>
  <version number='1.10'/>
  <description>
    The package defines a single command <tt>\hologo</tt>, whose
    argument is the usual case-confused ASCII version of the logo.
    The command is bookmark-enabled, so that every logo becomes
    available in bookmarks without further work.
    <p/>
    The package is part of the <xref refid='oberdiek'>oberdiek</xref>
    bundle.
  </description>
  <documentation details='Package documentation'
      href='ctan:/macros/latex/contrib/oberdiek/hologo.pdf'/>
  <ctan file='true' path='/macros/latex/contrib/oberdiek/hologo.dtx'/>
  <miktex location='oberdiek'/>
  <texlive location='oberdiek'/>
  <install path='/macros/latex/contrib/oberdiek/oberdiek.tds.zip'/>
</entry>
%</catalogue>
%    \end{macrocode}
%
% \begin{thebibliography}{9}
% \raggedright
%
% \bibitem{btxdoc}
% Oren Patashnik,
% \textit{\hologo{BibTeX}ing},
% 1988-02-08.\\
% \CTAN{biblio/bibtex/base/}
%
% \bibitem{dtklogos}
% Gerd Neugebauer, DANTE,
% \textit{Package \xpackage{dtklogos}},
% 2011-04-25.\\
% \CTAN{usergrps/dante/dtk/dtklogos.sty}
%
% \bibitem{etexman}
% The \hologo{NTS} Team,
% \textit{The \hologo{eTeX} manual},
% 1998-02.\\
% \CTAN{systems/e-tex/v2/doc/}
%
% \bibitem{ExTeX-FAQ}
% The \hologo{ExTeX} group,
% \textit{\hologo{ExTeX}: FAQ -- How is \hologo{ExTeX} typeset?},
% 2007-04-14.\\
% \url{http://www.extex.org/documentation/faq.html}
%
% \bibitem{LyX}
% %@MISC{ LyX,
% %  title = {{LyX 2.0.0 -- The Document Processor [Computer software and manual]}},
% %  author = {{The LyX Team}},
% %  howpublished = {Internet: http://www.lyx.org},
% %  year = {2011-05-08},
% %  note = {Retrieved May 10, 2011, from http://www.lyx.org},
% %  url = {http://www.lyx.org/}
% %}
% The \hologo{LyX} Team,
% \textit{\hologo{LyX} -- The Document Processor},
% 2011-05-08.\\
% \url{http://www.lyx.org/}
%
% \bibitem{OzTeX}
% Andrew Trevorrow,
% \hologo{OzTeX} FAQ: What is the correct way to typeset ``\hologo{OzTeX}''?,
% 2011-09-15 (visited).
% \url{http://www.trevorrow.com/oztex/ozfaq.html#oztex-logo}
%
% \bibitem{PiCTeX}
% Michael Wichura,
% \textit{The \hologo{PiCTeX} macro package},
% 1987-09-21.
% \CTAN{graphics/pictex/}
%
% \bibitem{scrlogo}
% Markus Kohm,
% \textit{\hologo{KOMAScript} Datei \xfile{scrlogo.dtx}},
% 2009-01-30.\\
% \CTAN{install/macros/latex/contrib/komascript.tds.zip}
%
% \end{thebibliography}
%
% \begin{History}
%   \begin{Version}{2010/04/08 v1.0}
%   \item
%     The first version.
%   \end{Version}
%   \begin{Version}{2010/04/16 v1.1}
%   \item
%     \cs{Hologo} added for support of logos at start of a sentence.
%   \item
%     \cs{hologoSetup} and \cs{hologoLogoSetup} added.
%   \item
%     Options \xoption{break}, \xoption{hyphenbreak}, \xoption{spacebreak}
%     added.
%   \item
%     Variant support added by option \xoption{variant}.
%   \end{Version}
%   \begin{Version}{2010/04/24 v1.2}
%   \item
%     \hologo{LaTeX3} added.
%   \item
%     \hologo{VTeX} added.
%   \end{Version}
%   \begin{Version}{2010/11/21 v1.3}
%   \item
%     \hologo{iniTeX}, \hologo{virTeX} added.
%   \end{Version}
%   \begin{Version}{2011/03/25 v1.4}
%   \item
%     \hologo{ConTeXt} with variants added.
%   \item
%     Option \xoption{discretionarybreak} added as refinement for
%     option \xoption{break}.
%   \end{Version}
%   \begin{Version}{2011/04/21 v1.5}
%   \item
%     Wrong TDS directory for test files fixed.
%   \end{Version}
%   \begin{Version}{2011/10/01 v1.6}
%   \item
%     Support for package \xpackage{tex4ht} added.
%   \item
%     Support for \cs{csname} added if \cs{ifincsname} is available.
%   \item
%     New logos:
%     \hologo{(La)TeX},
%     \hologo{biber},
%     \hologo{BibTeX} (\xoption{sc}, \xoption{sf}),
%     \hologo{emTeX},
%     \hologo{ExTeX},
%     \hologo{KOMAScript},
%     \hologo{La},
%     \hologo{LyX},
%     \hologo{MiKTeX},
%     \hologo{NTS},
%     \hologo{OzMF},
%     \hologo{OzMP},
%     \hologo{OzTeX},
%     \hologo{OzTtH},
%     \hologo{PCTeX},
%     \hologo{PiC},
%     \hologo{PiCTeX},
%     \hologo{METAFONT},
%     \hologo{MetaFun},
%     \hologo{METAPOST},
%     \hologo{MetaPost},
%     \hologo{SLiTeX} (\xoption{lift}, \xoption{narrow}, \xoption{simple}),
%     \hologo{SliTeX} (\xoption{narrow}, \xoption{simple}, \xoption{lift}),
%     \hologo{teTeX}.
%   \item
%     Fixes:
%     \hologo{iniTeX},
%     \hologo{pdfLaTeX},
%     \hologo{pdfTeX},
%     \hologo{virTeX}.
%   \item
%     \cs{hologoFontSetup} and \cs{hologoLogoFontSetup} added.
%   \item
%     \cs{hologoVariant} and \cs{HologoVariant} added.
%   \end{Version}
%   \begin{Version}{2011/11/22 v1.7}
%   \item
%     New logos:
%     \hologo{BibTeX8},
%     \hologo{LaTeXML},
%     \hologo{SageTeX},
%     \hologo{TeX4ht},
%     \hologo{TTH}.
%   \item
%     \hologo{Xe} and friends: Driver stuff fixed.
%   \item
%     \hologo{Xe} and friends: Support for italic added.
%   \item
%     \hologo{Xe} and friends: Package support for \xpackage{pgf}
%     and \xpackage{pstricks} added.
%   \end{Version}
%   \begin{Version}{2011/11/29 v1.8}
%   \item
%     New logos:
%     \hologo{HanTheThanh}.
%   \end{Version}
%   \begin{Version}{2011/12/21 v1.9}
%   \item
%     Patch for package \xpackage{ifxetex} added for the case that
%     \cs{newif} is undefined in \hologo{iniTeX}.
%   \item
%     Some fixes for \hologo{iniTeX}.
%   \end{Version}
%   \begin{Version}{2012/04/26 v1.10}
%   \item
%     Fix in bookmark version of logo ``\hologo{HanTheThanh}''.
%   \end{Version}
%   \begin{Version}{2016/05/12 v1.11}
%   \item
%     Update HOLOGO@IfCharExists (previously in texlive)
%   \item define pdfliteral in current luatex.
%   \end{Version}
% \end{History}
%
% \PrintIndex
%
% \Finale
\endinput
%
        \else
          \input hologo.cfg\relax
        \fi
      \else
        \@PackageInfoNoLine{hologo}{%
          Empty configuration file `hologo.cfg' ignored%
        }%
      \fi
    \fi
  }%
}
%    \end{macrocode}
%
%    \begin{macrocode}
\def\HOLOGO@temp#1#2{%
  \kv@define@key{HoLogoDriver}{#1}[]{%
    \begingroup
      \def\HOLOGO@temp{##1}%
      \ltx@onelevel@sanitize\HOLOGO@temp
      \ifx\HOLOGO@temp\ltx@empty
      \else
        \@PackageError{hologo}{%
          Value (\HOLOGO@temp) not permitted for option `#1'%
        }%
        \@ehc
      \fi
    \endgroup
    \def\hologoDriver{#2}%
  }%
}%
\def\HOLOGO@@temp#1#2{%
  \ifx\kv@value\relax
    \HOLOGO@temp{#1}{#1}%
  \else
    \HOLOGO@temp{#1}{#2}%
  \fi
}%
\kv@parse@normalized{%
  pdftex,%
  luatex=pdftex,%
  dvipdfm,%
  dvipdfmx=dvipdfm,%
  dvips,%
  dvipsone=dvips,%
  xdvi=dvips,%
  xetex,%
  vtex,%
}\HOLOGO@@temp
%    \end{macrocode}
%
%    \begin{macrocode}
\kv@define@key{HoLogoDriver}{driverfallback}{%
  \def\HOLOGO@DriverFallback{#1}%
}
%    \end{macrocode}
%
%    \begin{macro}{\HOLOGO@DriverFallback}
%    \begin{macrocode}
\def\HOLOGO@DriverFallback{dvips}
%    \end{macrocode}
%    \end{macro}
%
%    \begin{macro}{\hologoDriverSetup}
%    \begin{macrocode}
\def\hologoDriverSetup{%
  \let\hologoDriver\ltx@undefined
  \HOLOGO@DriverSetup
}
%    \end{macrocode}
%    \end{macro}
%
%    \begin{macro}{\HOLOGO@DriverSetup}
%    \begin{macrocode}
\def\HOLOGO@DriverSetup#1{%
  \kvsetkeys{HoLogoDriver}{#1}%
  \HOLOGO@CheckDriver
  \ltx@ifundefined{hologoDriver}{%
    \begingroup
    \edef\x{\endgroup
      \noexpand\kvsetkeys{HoLogoDriver}{\HOLOGO@DriverFallback}%
    }\x
  }{}%
  \@PackageInfoNoLine{hologo}{Using driver `\hologoDriver'}%
}
%    \end{macrocode}
%    \end{macro}
%
%    \begin{macro}{\HOLOGO@CheckDriver}
%    \begin{macrocode}
\def\HOLOGO@CheckDriver{%
  \ifpdf
    \def\hologoDriver{pdftex}%
    \let\HOLOGO@pdfliteral\pdfliteral
    \ifluatex
      \ifx\pdfextension\@undefined\else
        \protected\def\pdfliteral{\pdfextension literal}%
        \let\HOLOGO@pdfliteral\pdfliteral
      \fi
      \ltx@IfUndefined{HOLOGO@pdfliteral}{%
        \ifnum\luatexversion<36 %
        \else
          \begingroup
            \let\HOLOGO@temp\endgroup
            \ifcase0%
                \directlua{%
                  if tex.enableprimitives then %
                    tex.enableprimitives('HOLOGO@', {'pdfliteral'})%
                  else %
                    tex.print('1')%
                  end%
                }%
                \ifx\HOLOGO@pdfliteral\@undefined 1\fi%
                \relax%
              \endgroup
              \let\HOLOGO@temp\relax
              \global\let\HOLOGO@pdfliteral\HOLOGO@pdfliteral
            \fi%
          \HOLOGO@temp
        \fi
      }{}%
    \fi
    \ltx@IfUndefined{HOLOGO@pdfliteral}{%
      \@PackageWarningNoLine{hologo}{%
        Cannot find \string\pdfliteral
      }%
    }{}%
  \else
    \ifxetex
      \def\hologoDriver{xetex}%
    \else
      \ifvtex
        \def\hologoDriver{vtex}%
      \fi
    \fi
  \fi
}
%    \end{macrocode}
%    \end{macro}
%
%    \begin{macro}{\HOLOGO@WarningUnsupportedDriver}
%    \begin{macrocode}
\def\HOLOGO@WarningUnsupportedDriver#1{%
  \@PackageWarningNoLine{hologo}{%
    Logo `#1' needs driver specific macros,\MessageBreak
    but driver `\hologoDriver' is not supported.\MessageBreak
    Use a different driver or\MessageBreak
    load package `graphics' or `pgf'%
  }%
}
%    \end{macrocode}
%    \end{macro}
%
% \subsubsection{Reflect box macros}
%
%    Skip driver part if not needed.
%    \begin{macrocode}
\ltx@IfUndefined{reflectbox}{}{%
  \ltx@IfUndefined{rotatebox}{}{%
    \HOLOGO@AtEnd
  }%
}
\ltx@IfUndefined{pgftext}{}{%
  \HOLOGO@AtEnd
}
\ltx@IfUndefined{psscalebox}{}{%
  \HOLOGO@AtEnd
}
%    \end{macrocode}
%
%    \begin{macrocode}
\def\HOLOGO@temp{LaTeX2e}
\ifx\fmtname\HOLOGO@temp
  \RequirePackage{kvoptions}[2011/06/30]%
  \ProcessKeyvalOptions{HoLogoDriver}%
\fi
\HOLOGO@DriverSetup{}
%    \end{macrocode}
%
%    \begin{macro}{\HOLOGO@ReflectBox}
%    \begin{macrocode}
\def\HOLOGO@ReflectBox#1{%
  \begingroup
    \setbox\ltx@zero\hbox{\begingroup#1\endgroup}%
    \setbox\ltx@two\hbox{%
      \kern\wd\ltx@zero
      \csname HOLOGO@ScaleBox@\hologoDriver\endcsname{-1}{1}{%
        \hbox to 0pt{\copy\ltx@zero\hss}%
      }%
    }%
    \wd\ltx@two=\wd\ltx@zero
    \box\ltx@two
  \endgroup
}
%    \end{macrocode}
%    \end{macro}
%
%    \begin{macro}{\HOLOGO@PointReflectBox}
%    \begin{macrocode}
\def\HOLOGO@PointReflectBox#1{%
  \begingroup
    \setbox\ltx@zero\hbox{\begingroup#1\endgroup}%
    \setbox\ltx@two\hbox{%
      \kern\wd\ltx@zero
      \raise\ht\ltx@zero\hbox{%
        \csname HOLOGO@ScaleBox@\hologoDriver\endcsname{-1}{-1}{%
          \hbox to 0pt{\copy\ltx@zero\hss}%
        }%
      }%
    }%
    \wd\ltx@two=\wd\ltx@zero
    \box\ltx@two
  \endgroup
}
%    \end{macrocode}
%    \end{macro}
%
%    We must define all variants because of dynamic driver setup.
%    \begin{macrocode}
\def\HOLOGO@temp#1#2{#2}
%    \end{macrocode}
%
%    \begin{macro}{\HOLOGO@ScaleBox@pdftex}
%    \begin{macrocode}
\HOLOGO@temp{pdftex}{%
  \def\HOLOGO@ScaleBox@pdftex#1#2#3{%
    \HOLOGO@pdfliteral{%
      q #1 0 0 #2 0 0 cm%
    }%
    #3%
    \HOLOGO@pdfliteral{%
      Q%
    }%
  }%
}
%    \end{macrocode}
%    \end{macro}
%    \begin{macro}{\HOLOGO@ScaleBox@dvips}
%    \begin{macrocode}
\HOLOGO@temp{dvips}{%
  \def\HOLOGO@ScaleBox@dvips#1#2#3{%
    \special{ps:%
      gsave %
      currentpoint %
      currentpoint translate %
      #1 #2 scale %
      neg exch neg exch translate%
    }%
    #3%
    \special{ps:%
      currentpoint %
      grestore %
      moveto%
    }%
  }%
}
%    \end{macrocode}
%    \end{macro}
%    \begin{macro}{\HOLOGO@ScaleBox@dvipdfm}
%    \begin{macrocode}
\HOLOGO@temp{dvipdfm}{%
  \let\HOLOGO@ScaleBox@dvipdfm\HOLOGO@ScaleBox@dvips
}
%    \end{macrocode}
%    \end{macro}
%    Since \hologo{XeTeX} v0.6.
%    \begin{macro}{\HOLOGO@ScaleBox@xetex}
%    \begin{macrocode}
\HOLOGO@temp{xetex}{%
  \def\HOLOGO@ScaleBox@xetex#1#2#3{%
    \special{x:gsave}%
    \special{x:scale #1 #2}%
    #3%
    \special{x:grestore}%
  }%
}
%    \end{macrocode}
%    \end{macro}
%    \begin{macro}{\HOLOGO@ScaleBox@vtex}
%    \begin{macrocode}
\HOLOGO@temp{vtex}{%
  \def\HOLOGO@ScaleBox@vtex#1#2#3{%
    \special{r(#1,0,0,#2,0,0}%
    #3%
    \special{r)}%
  }%
}
%    \end{macrocode}
%    \end{macro}
%
%    \begin{macrocode}
\HOLOGO@AtEnd%
%</package>
%    \end{macrocode}
%
% \section{Test}
%
% \subsection{Catcode checks for loading}
%
%    \begin{macrocode}
%<*test1>
%    \end{macrocode}
%    \begin{macrocode}
\catcode`\{=1 %
\catcode`\}=2 %
\catcode`\#=6 %
\catcode`\@=11 %
\expandafter\ifx\csname count@\endcsname\relax
  \countdef\count@=255 %
\fi
\expandafter\ifx\csname @gobble\endcsname\relax
  \long\def\@gobble#1{}%
\fi
\expandafter\ifx\csname @firstofone\endcsname\relax
  \long\def\@firstofone#1{#1}%
\fi
\expandafter\ifx\csname loop\endcsname\relax
  \expandafter\@firstofone
\else
  \expandafter\@gobble
\fi
{%
  \def\loop#1\repeat{%
    \def\body{#1}%
    \iterate
  }%
  \def\iterate{%
    \body
      \let\next\iterate
    \else
      \let\next\relax
    \fi
    \next
  }%
  \let\repeat=\fi
}%
\def\RestoreCatcodes{}
\count@=0 %
\loop
  \edef\RestoreCatcodes{%
    \RestoreCatcodes
    \catcode\the\count@=\the\catcode\count@\relax
  }%
\ifnum\count@<255 %
  \advance\count@ 1 %
\repeat

\def\RangeCatcodeInvalid#1#2{%
  \count@=#1\relax
  \loop
    \catcode\count@=15 %
  \ifnum\count@<#2\relax
    \advance\count@ 1 %
  \repeat
}
\def\RangeCatcodeCheck#1#2#3{%
  \count@=#1\relax
  \loop
    \ifnum#3=\catcode\count@
    \else
      \errmessage{%
        Character \the\count@\space
        with wrong catcode \the\catcode\count@\space
        instead of \number#3%
      }%
    \fi
  \ifnum\count@<#2\relax
    \advance\count@ 1 %
  \repeat
}
\def\space{ }
\expandafter\ifx\csname LoadCommand\endcsname\relax
  \def\LoadCommand{\input hologo.sty\relax}%
\fi
\def\Test{%
  \RangeCatcodeInvalid{0}{47}%
  \RangeCatcodeInvalid{58}{64}%
  \RangeCatcodeInvalid{91}{96}%
  \RangeCatcodeInvalid{123}{255}%
  \catcode`\@=12 %
  \catcode`\\=0 %
  \catcode`\%=14 %
  \LoadCommand
  \RangeCatcodeCheck{0}{36}{15}%
  \RangeCatcodeCheck{37}{37}{14}%
  \RangeCatcodeCheck{38}{47}{15}%
  \RangeCatcodeCheck{48}{57}{12}%
  \RangeCatcodeCheck{58}{63}{15}%
  \RangeCatcodeCheck{64}{64}{12}%
  \RangeCatcodeCheck{65}{90}{11}%
  \RangeCatcodeCheck{91}{91}{15}%
  \RangeCatcodeCheck{92}{92}{0}%
  \RangeCatcodeCheck{93}{96}{15}%
  \RangeCatcodeCheck{97}{122}{11}%
  \RangeCatcodeCheck{123}{255}{15}%
  \RestoreCatcodes
}
\Test
\csname @@end\endcsname
\end
%    \end{macrocode}
%    \begin{macrocode}
%</test1>
%    \end{macrocode}
%
% \subsection{Spacefactor}
%
%    The space factor must be 1000 after a logo. If it is greater 1000
%    then the following space is a space after a sentence closing point.
%    If the space factor is smaller 1000 then an immediate following
%    dot is interpreted as abbreviation, not sentence closing point.
%
%    \begin{macrocode}
%<*test-spacefactor>
\NeedsTeXFormat{LaTeX2e}
\documentclass{article}
\usepackage{hologo}[2016/05/12]
\usepackage{kvsetkeys}
\usepackage{qstest}
\IncludeTests{*}
\LogTests{log}{*}{*}
\begin{document}
\begin{qstest}{spacefactor}{spacefactor}
\newcommand*{\Test}[1]{%
  \sbox0{%
    \hologo{#1}%
    \Expect*{1000 (#1)}*{\the\spacefactor\space(#1)}%
  }%
}%
\makeatletter
\def\TestList{}
\def\hologoEntry#1#2#3{%
  \edef\TestList{%
    \ifx\TestList\@empty
    \else
      \TestList,%
    \fi
    #1%
    \ifx\\#2\\%
    \else
      ={variant=#2}%
    \fi
  }%
}
\hologoList
\expandafter\kv@parse@normalized\expandafter{%
  \TestList
}{%
  \begingroup
    \let\@logo=\kv@key
    \ifx\kv@value\relax
    \else
      \expandafter\hologoLogoSetup\expandafter\@logo\expandafter{%
        \kv@value
      }%
    \fi
    \Test\@logo
  \endgroup
  \@gobbletwo
}
\end{qstest}
\end{document}
%</test-spacefactor>
%    \end{macrocode}
%
% \subsection{Complete list}
%
%    \begin{macrocode}
%<*test-list>
\NeedsTeXFormat{LaTeX2e}
\documentclass[12pt,a4paper]{article}
\usepackage{hologo}[2016/05/12]
\usepackage[T1]{fontenc}
\usepackage{lmodern}
\usepackage{parskip}
\usepackage[unicode]{hyperref}[2011/09/28]
\usepackage{bookmark}[2011/09/19]
\bookmarksetup{%
  numbered,%
  open,%
  openlevel=2,%
}
\renewcommand*{\contentsname}{List of logos}
\begin{document}
\tableofcontents
\def\TestFont#1#2#3#4#5#6{%
  \begingroup
    \usefont{#3}{#4}{#5}{#6}%
    \HologoVariant{#1}{#2}/\hologoVariant{#1}{#2}%
    \quad
    \begingroup\scriptsize\hologoVariant{#1}{#2}\endgroup
    \quad
  \endgroup
  (#3/#4/#5/#6)%
  \par
}
\makeatletter
\def\hologoEntry#1#2#3{%
  \section{%
    \HologoVariant{#1}{#2}/\hologoVariant{#1}{#2} %
    {[#1\ifx\\#2\\\else\space(#2)\fi]}% hash-ok
  }% braces around [] because of bug in tex4ht
  \begingroup
    \hypersetup{unicode=false}%
    \bookmark[%
      dest=\@currentHref,%
      rellevel=1,%
      keeplevel,%
    ]{%
      \HologoVariant{#1}{#2}/\hologoVariant{#1}{#2} %
      (PDFDocEncoding)%
    }%
  \endgroup
  \TestFont{#1}{#2}{OT1}{cmr}{m}{n}%
  \TestFont{#1}{#2}{OT1}{cmss}{m}{n}%
  \TestFont{#1}{#2}{OT1}{cmr}{b}{n}%
  \TestFont{#1}{#2}{OT1}{cmr}{m}{it}%
  \TestFont{#1}{#2}{OT1}{cmtt}{m}{n}%
  \TestFont{#1}{#2}{T1}{lmr}{m}{n}%
  \TestFont{#1}{#2}{T1}{lmss}{m}{n}%
  \TestFont{#1}{#2}{T1}{lmr}{b}{n}%
  \TestFont{#1}{#2}{T1}{lmr}{m}{it}%
  \TestFont{#1}{#2}{T1}{lmtt}{m}{n}%
  \TestFont{#1}{#2}{T1}{lmvtt}{m}{n}%
  \TestFont{#1}{#2}{T1}{qtm}{m}{n}%
  \TestFont{#1}{#2}{T1}{qhv}{m}{n}%
  \TestFont{#1}{#2}{T1}{qtm}{b}{n}%
  \TestFont{#1}{#2}{T1}{qtm}{m}{it}%
  \TestFont{#1}{#2}{T1}{qcr}{m}{n}%
  \newpage
}
\makeatother
\hologoList
\end{document}
%</test-list>
%    \end{macrocode}
%
% \section{Installation}
%
% \subsection{Download}
%
% \paragraph{Package.} This package is available on
% CTAN\footnote{\url{ftp://ftp.ctan.org/tex-archive/}}:
% \begin{description}
% \item[\CTAN{macros/latex/contrib/oberdiek/hologo.dtx}] The source file.
% \item[\CTAN{macros/latex/contrib/oberdiek/hologo.pdf}] Documentation.
% \end{description}
%
%
% \paragraph{Bundle.} All the packages of the bundle `oberdiek'
% are also available in a TDS compliant ZIP archive. There
% the packages are already unpacked and the documentation files
% are generated. The files and directories obey the TDS standard.
% \begin{description}
% \item[\CTAN{install/macros/latex/contrib/oberdiek.tds.zip}]
% \end{description}
% \emph{TDS} refers to the standard ``A Directory Structure
% for \TeX\ Files'' (\CTAN{tds/tds.pdf}). Directories
% with \xfile{texmf} in their name are usually organized this way.
%
% \subsection{Bundle installation}
%
% \paragraph{Unpacking.} Unpack the \xfile{oberdiek.tds.zip} in the
% TDS tree (also known as \xfile{texmf} tree) of your choice.
% Example (linux):
% \begin{quote}
%   |unzip oberdiek.tds.zip -d ~/texmf|
% \end{quote}
%
% \paragraph{Script installation.}
% Check the directory \xfile{TDS:scripts/oberdiek/} for
% scripts that need further installation steps.
% Package \xpackage{attachfile2} comes with the Perl script
% \xfile{pdfatfi.pl} that should be installed in such a way
% that it can be called as \texttt{pdfatfi}.
% Example (linux):
% \begin{quote}
%   |chmod +x scripts/oberdiek/pdfatfi.pl|\\
%   |cp scripts/oberdiek/pdfatfi.pl /usr/local/bin/|
% \end{quote}
%
% \subsection{Package installation}
%
% \paragraph{Unpacking.} The \xfile{.dtx} file is a self-extracting
% \docstrip\ archive. The files are extracted by running the
% \xfile{.dtx} through \plainTeX:
% \begin{quote}
%   \verb|tex hologo.dtx|
% \end{quote}
%
% \paragraph{TDS.} Now the different files must be moved into
% the different directories in your installation TDS tree
% (also known as \xfile{texmf} tree):
% \begin{quote}
% \def\t{^^A
% \begin{tabular}{@{}>{\ttfamily}l@{ $\rightarrow$ }>{\ttfamily}l@{}}
%   hologo.sty & tex/generic/oberdiek/hologo.sty\\
%   hologo.pdf & doc/latex/oberdiek/hologo.pdf\\
%   example/hologo-example.tex & doc/latex/oberdiek/example/hologo-example.tex\\
%   test/hologo-test1.tex & doc/latex/oberdiek/test/hologo-test1.tex\\
%   test/hologo-test-spacefactor.tex & doc/latex/oberdiek/test/hologo-test-spacefactor.tex\\
%   test/hologo-test-list.tex & doc/latex/oberdiek/test/hologo-test-list.tex\\
%   hologo.dtx & source/latex/oberdiek/hologo.dtx\\
% \end{tabular}^^A
% }^^A
% \sbox0{\t}^^A
% \ifdim\wd0>\linewidth
%   \begingroup
%     \advance\linewidth by\leftmargin
%     \advance\linewidth by\rightmargin
%   \edef\x{\endgroup
%     \def\noexpand\lw{\the\linewidth}^^A
%   }\x
%   \def\lwbox{^^A
%     \leavevmode
%     \hbox to \linewidth{^^A
%       \kern-\leftmargin\relax
%       \hss
%       \usebox0
%       \hss
%       \kern-\rightmargin\relax
%     }^^A
%   }^^A
%   \ifdim\wd0>\lw
%     \sbox0{\small\t}^^A
%     \ifdim\wd0>\linewidth
%       \ifdim\wd0>\lw
%         \sbox0{\footnotesize\t}^^A
%         \ifdim\wd0>\linewidth
%           \ifdim\wd0>\lw
%             \sbox0{\scriptsize\t}^^A
%             \ifdim\wd0>\linewidth
%               \ifdim\wd0>\lw
%                 \sbox0{\tiny\t}^^A
%                 \ifdim\wd0>\linewidth
%                   \lwbox
%                 \else
%                   \usebox0
%                 \fi
%               \else
%                 \lwbox
%               \fi
%             \else
%               \usebox0
%             \fi
%           \else
%             \lwbox
%           \fi
%         \else
%           \usebox0
%         \fi
%       \else
%         \lwbox
%       \fi
%     \else
%       \usebox0
%     \fi
%   \else
%     \lwbox
%   \fi
% \else
%   \usebox0
% \fi
% \end{quote}
% If you have a \xfile{docstrip.cfg} that configures and enables \docstrip's
% TDS installing feature, then some files can already be in the right
% place, see the documentation of \docstrip.
%
% \subsection{Refresh file name databases}
%
% If your \TeX~distribution
% (\teTeX, \mikTeX, \dots) relies on file name databases, you must refresh
% these. For example, \teTeX\ users run \verb|texhash| or
% \verb|mktexlsr|.
%
% \subsection{Some details for the interested}
%
% \paragraph{Attached source.}
%
% The PDF documentation on CTAN also includes the
% \xfile{.dtx} source file. It can be extracted by
% AcrobatReader 6 or higher. Another option is \textsf{pdftk},
% e.g. unpack the file into the current directory:
% \begin{quote}
%   \verb|pdftk hologo.pdf unpack_files output .|
% \end{quote}
%
% \paragraph{Unpacking with \LaTeX.}
% The \xfile{.dtx} chooses its action depending on the format:
% \begin{description}
% \item[\plainTeX:] Run \docstrip\ and extract the files.
% \item[\LaTeX:] Generate the documentation.
% \end{description}
% If you insist on using \LaTeX\ for \docstrip\ (really,
% \docstrip\ does not need \LaTeX), then inform the autodetect routine
% about your intention:
% \begin{quote}
%   \verb|latex \let\install=y% \iffalse meta-comment
%
% File: hologo.dtx
% Version: 2016/05/12 v1.11
% Info: A logo collection with bookmark support
%
% Copyright (C) 2010-2012 by
%    Heiko Oberdiek <heiko.oberdiek at googlemail.com>
%
% This work may be distributed and/or modified under the
% conditions of the LaTeX Project Public License, either
% version 1.3c of this license or (at your option) any later
% version. This version of this license is in
%    http://www.latex-project.org/lppl/lppl-1-3c.txt
% and the latest version of this license is in
%    http://www.latex-project.org/lppl.txt
% and version 1.3 or later is part of all distributions of
% LaTeX version 2005/12/01 or later.
%
% This work has the LPPL maintenance status "maintained".
%
% This Current Maintainer of this work is Heiko Oberdiek.
%
% The Base Interpreter refers to any `TeX-Format',
% because some files are installed in TDS:tex/generic//.
%
% This work consists of the main source file hologo.dtx
% and the derived files
%    hologo.sty, hologo.pdf, hologo.ins, hologo.drv, hologo-example.tex,
%    hologo-test1.tex, hologo-test-spacefactor.tex,
%    hologo-test-list.tex.
%
% Distribution:
%    CTAN:macros/latex/contrib/oberdiek/hologo.dtx
%    CTAN:macros/latex/contrib/oberdiek/hologo.pdf
%
% Unpacking:
%    (a) If hologo.ins is present:
%           tex hologo.ins
%    (b) Without hologo.ins:
%           tex hologo.dtx
%    (c) If you insist on using LaTeX
%           latex \let\install=y\input{hologo.dtx}
%        (quote the arguments according to the demands of your shell)
%
% Documentation:
%    (a) If hologo.drv is present:
%           latex hologo.drv
%    (b) Without hologo.drv:
%           latex hologo.dtx; ...
%    The class ltxdoc loads the configuration file ltxdoc.cfg
%    if available. Here you can specify further options, e.g.
%    use A4 as paper format:
%       \PassOptionsToClass{a4paper}{article}
%
%    Programm calls to get the documentation (example):
%       pdflatex hologo.dtx
%       makeindex -s gind.ist hologo.idx
%       pdflatex hologo.dtx
%       makeindex -s gind.ist hologo.idx
%       pdflatex hologo.dtx
%
% Installation:
%    TDS:tex/generic/oberdiek/hologo.sty
%    TDS:doc/latex/oberdiek/hologo.pdf
%    TDS:doc/latex/oberdiek/example/hologo-example.tex
%    TDS:doc/latex/oberdiek/test/hologo-test1.tex
%    TDS:doc/latex/oberdiek/test/hologo-test-spacefactor.tex
%    TDS:doc/latex/oberdiek/test/hologo-test-list.tex
%    TDS:source/latex/oberdiek/hologo.dtx
%
%<*ignore>
\begingroup
  \catcode123=1 %
  \catcode125=2 %
  \def\x{LaTeX2e}%
\expandafter\endgroup
\ifcase 0\ifx\install y1\fi\expandafter
         \ifx\csname processbatchFile\endcsname\relax\else1\fi
         \ifx\fmtname\x\else 1\fi\relax
\else\csname fi\endcsname
%</ignore>
%<*install>
\input docstrip.tex
\Msg{************************************************************************}
\Msg{* Installation}
\Msg{* Package: hologo 2016/05/12 v1.11 A logo collection with bookmark support (HO)}
\Msg{************************************************************************}

\keepsilent
\askforoverwritefalse

\let\MetaPrefix\relax
\preamble

This is a generated file.

Project: hologo
Version: 2016/05/12 v1.11

Copyright (C) 2010-2012 by
   Heiko Oberdiek <heiko.oberdiek at googlemail.com>

This work may be distributed and/or modified under the
conditions of the LaTeX Project Public License, either
version 1.3c of this license or (at your option) any later
version. This version of this license is in
   http://www.latex-project.org/lppl/lppl-1-3c.txt
and the latest version of this license is in
   http://www.latex-project.org/lppl.txt
and version 1.3 or later is part of all distributions of
LaTeX version 2005/12/01 or later.

This work has the LPPL maintenance status "maintained".

This Current Maintainer of this work is Heiko Oberdiek.

The Base Interpreter refers to any `TeX-Format',
because some files are installed in TDS:tex/generic//.

This work consists of the main source file hologo.dtx
and the derived files
   hologo.sty, hologo.pdf, hologo.ins, hologo.drv, hologo-example.tex,
   hologo-test1.tex, hologo-test-spacefactor.tex,
   hologo-test-list.tex.

\endpreamble
\let\MetaPrefix\DoubleperCent

\generate{%
  \file{hologo.ins}{\from{hologo.dtx}{install}}%
  \file{hologo.drv}{\from{hologo.dtx}{driver}}%
  \usedir{tex/generic/oberdiek}%
  \file{hologo.sty}{\from{hologo.dtx}{package}}%
  \usedir{doc/latex/oberdiek/example}%
  \file{hologo-example.tex}{\from{hologo.dtx}{example}}%
  \usedir{doc/latex/oberdiek/test}%
  \file{hologo-test1.tex}{\from{hologo.dtx}{test1}}%
  \file{hologo-test-spacefactor.tex}{\from{hologo.dtx}{test-spacefactor}}%
  \file{hologo-test-list.tex}{\from{hologo.dtx}{test-list}}%
  \nopreamble
  \nopostamble
  \usedir{source/latex/oberdiek/catalogue}%
  \file{hologo.xml}{\from{hologo.dtx}{catalogue}}%
}

\catcode32=13\relax% active space
\let =\space%
\Msg{************************************************************************}
\Msg{*}
\Msg{* To finish the installation you have to move the following}
\Msg{* file into a directory searched by TeX:}
\Msg{*}
\Msg{*     hologo.sty}
\Msg{*}
\Msg{* To produce the documentation run the file `hologo.drv'}
\Msg{* through LaTeX.}
\Msg{*}
\Msg{* Happy TeXing!}
\Msg{*}
\Msg{************************************************************************}

\endbatchfile
%</install>
%<*ignore>
\fi
%</ignore>
%<*driver>
\NeedsTeXFormat{LaTeX2e}
\ProvidesFile{hologo.drv}%
  [2016/05/12 v1.11 A logo collection with bookmark support (HO)]%
\documentclass{ltxdoc}
\usepackage{holtxdoc}[2011/11/22]
\usepackage{hologo}[2016/05/12]
\usepackage{longtable}
\usepackage{array}
\usepackage{paralist}
%\usepackage[T1]{fontenc}
%\usepackage{lmodern}
\begin{document}
  \DocInput{hologo.dtx}%
\end{document}
%</driver>
% \fi
%
%
% \CharacterTable
%  {Upper-case    \A\B\C\D\E\F\G\H\I\J\K\L\M\N\O\P\Q\R\S\T\U\V\W\X\Y\Z
%   Lower-case    \a\b\c\d\e\f\g\h\i\j\k\l\m\n\o\p\q\r\s\t\u\v\w\x\y\z
%   Digits        \0\1\2\3\4\5\6\7\8\9
%   Exclamation   \!     Double quote  \"     Hash (number) \#
%   Dollar        \$     Percent       \%     Ampersand     \&
%   Acute accent  \'     Left paren    \(     Right paren   \)
%   Asterisk      \*     Plus          \+     Comma         \,
%   Minus         \-     Point         \.     Solidus       \/
%   Colon         \:     Semicolon     \;     Less than     \<
%   Equals        \=     Greater than  \>     Question mark \?
%   Commercial at \@     Left bracket  \[     Backslash     \\
%   Right bracket \]     Circumflex    \^     Underscore    \_
%   Grave accent  \`     Left brace    \{     Vertical bar  \|
%   Right brace   \}     Tilde         \~}
%
% \GetFileInfo{hologo.drv}
%
% \title{The \xpackage{hologo} package}
% \date{2016/05/12 v1.11}
% \author{Heiko Oberdiek\\\xemail{heiko.oberdiek at googlemail.com}}
%
% \maketitle
%
% \begin{abstract}
% This package starts a collection of logos with support for bookmarks
% strings.
% \end{abstract}
%
% \tableofcontents
%
% \section{Documentation}
%
% \subsection{Logo macros}
%
% \begin{declcs}{hologo} \M{name}
% \end{declcs}
% Macro \cs{hologo} sets the logo with name \meta{name}.
% The following table shows the supported names.
%
% \begingroup
%   \def\hologoEntry#1#2#3{^^A
%     #1&#2&\hologoLogoSetup{#1}{variant=#2}\hologo{#1}&#3\tabularnewline
%   }
%   \begin{longtable}{>{\ttfamily}l>{\ttfamily}lll}
%     \rmfamily\bfseries{name} & \rmfamily\bfseries variant
%     & \bfseries logo & \bfseries since\\
%     \hline
%     \endhead
%     \hologoList
%   \end{longtable}
% \endgroup
%
% \begin{declcs}{Hologo} \M{name}
% \end{declcs}
% Macro \cs{Hologo} starts the logo \meta{name} with an uppercase
% letter. As an exception small greek letters are not converted
% to uppercase. Examples, see \hologo{eTeX} and \hologo{ExTeX}.
%
% \subsection{Setup macros}
%
% The package does not support package options, but the following
% setup macros can be used to set options.
%
% \begin{declcs}{hologoSetup} \M{key value list}
% \end{declcs}
% Macro \cs{hologoSetup} sets global options.
%
% \begin{declcs}{hologoLogoSetup} \M{logo} \M{key value list}
% \end{declcs}
% Some options can also be used to configure a logo.
% These settings take precedence over global option settings.
%
% \subsection{Options}\label{sec:options}
%
% There are boolean and string options:
% \begin{description}
% \item[Boolean option:]
% It takes |true| or |false|
% as value. If the value is omitted, then |true| is used.
% \item[String option:]
% A value must be given as string. (But the string might be empty.)
% \end{description}
% The following options can be used both in \cs{hologoSetup}
% and \cs{hologoLogoSetup}:
% \begin{description}
% \def\entry#1{\item[\xoption{#1}:]}
% \entry{break}
%   enables or disables line breaks inside the logo. This setting is
%   refined by options \xoption{hyphenbreak}, \xoption{spacebreak}
%   or \xoption{discretionarybreak}.
%   Default is |false|.
% \entry{hyphenbreak}
%   enables or disables the line break right after the hyphen character.
% \entry{spacebreak}
%   enables or disables line breaks at space characters.
% \entry{discretionarybreak}
%   enables or disables line breaks at hyphenation points
%   (inserted by \cs{-}).
% \end{description}
% Macro \cs{hologoLogoSetup} also knows:
% \begin{description}
% \item[\xoption{variant}:]
%   This is a string option. It specifies a variant of a logo that
%   must exist. An empty string selects the package default variant.
% \end{description}
% Example:
% \begin{quote}
%   |\hologoSetup{break=false}|\\
%   |\hologoLogoSetup{plainTeX}{variant=hyphen,hyphenbreak}|\\
%   Then ``plain-\TeX'' contains one break point after the hyphen.
% \end{quote}
%
% \subsection{Driver options}
%
% Sometimes graphical operations are needed to construct some
% glyphs (e.g.\ \hologo{XeTeX}). If package \xpackage{graphics}
% or package \xpackage{pgf} are found, then the macros are taken
% from there. Otherwise the packge defines its own operations
% and therefore needs the driver information. Many drivers are
% detected automatically (\hologo{pdfTeX}/\hologo{LuaTeX}
% in PDF mode, \hologo{XeTeX}, \hologo{VTeX}). These have precedence
% over a driver option. The driver can be given as package option
% or using \cs{hologoDriverSetup}.
% The following list contains the recognized driver options:
% \begin{itemize}
% \item \xoption{pdftex}, \xoption{luatex}
% \item \xoption{dvipdfm}, \xoption{dvipdfmx}
% \item \xoption{dvips}, \xoption{dvipsone}, \xoption{xdvi}
% \item \xoption{xetex}
% \item \xoption{vtex}
% \end{itemize}
% The left driver of a line is the driver name that is used internally.
% The following names are aliases for drivers that use the
% same method. Therefore the entry in the \xext{log} file for
% the used driver prints the internally used driver name.
% \begin{description}
% \item[\xoption{driverfallback}:]
%   This option expects a driver that is used,
%   if the driver could not be detected automatically.
% \end{description}
%
% \begin{declcs}{hologoDriverSetup} \M{driver option}
% \end{declcs}
% The driver can also be configured after package loading
% using \cs{hologoDriverSetup}, also the way for \hologo{plainTeX}
% to setup the driver.
%
% \subsection{Font setup}
%
% Some logos require a special font, but should also be usable by
% \hologo{plainTeX}. Therefore the package provides some ways
% to influence the font settings. The options below
% take font settings as values. Both font commands
% such as \cs{sffamily} and macros that take one argument
% like \cs{textsf} can be used.
%
% \begin{declcs}{hologoFontSetup} \M{key value list}
% \end{declcs}
% Macro \cs{hologoFontSetup} sets the fonts for all logos.
% Supported keys:
% \begin{description}
% \def\entry#1{\item[\xoption{#1}:]}
% \entry{general}
%   This font is used for all logos. The default is empty.
%   That means no special font is used.
% \entry{bibsf}
%   This font is used for
%   {\hologoLogoSetup{BibTeX}{variant=sf}\hologo{BibTeX}}
%   with variant \xoption{sf}.
% \entry{rm}
%   This font is a serif font. It is used for \hologo{ExTeX}.
% \entry{sc}
%   This font specifies a small caps font. It is used for
%   {\hologoLogoSetup{BibTeX}{variant=sc}\hologo{BibTeX}}
%   with variant \xoption{sc}.
% \entry{sf}
%   This font specifies a sans serif font. The default
%   is \cs{sffamily}, then \cs{sf} is tried. Otherwise
%   a warning is given. It is used by \hologo{KOMAScript}.
% \entry{sy}
%   This is the font for math symbols (e.g. cmsy).
%   It is used by \hologo{AmS}, \hologo{NTS}, \hologo{ExTeX}.
% \entry{logo}
%   \hologo{METAFONT} and \hologo{METAPOST} are using that font.
%   In \hologo{LaTeX} \cs{logofamily} is used and
%   the definitions of package \xpackage{mflogo} are used
%   if the package is not loaded.
%   Otherwise the \cs{tenlogo} is used and defined
%   if it does not already exists.
% \end{description}
%
% \begin{declcs}{hologoLogoFontSetup} \M{logo} \M{key value list}
% \end{declcs}
% Fonts can also be set for a logo or logo component separately,
% see the following list.
% The keys are the same as for \cs{hologoFontSetup}.
%
% \begin{longtable}{>{\ttfamily}l>{\sffamily}ll}
%   \meta{logo} & keys & result\\
%   \hline
%   \endhead
%   BibTeX & bibsf & {\hologoLogoSetup{BibTeX}{variant=sf}\hologo{BibTeX}}\\[.5ex]
%   BibTeX & sc & {\hologoLogoSetup{BibTeX}{variant=sc}\hologo{BibTeX}}\\[.5ex]
%   ExTeX & rm & \hologo{ExTeX}\\
%   SliTeX & rm & \hologo{SliTeX}\\[.5ex]
%   AmS & sy & \hologo{AmS}\\
%   ExTeX & sy & \hologo{ExTeX}\\
%   NTS & sy & \hologo{NTS}\\[.5ex]
%   KOMAScript & sf & \hologo{KOMAScript}\\[.5ex]
%   METAFONT & logo & \hologo{METAFONT}\\
%   METAPOST & logo & \hologo{METAPOST}\\[.5ex]
%   SliTeX & sc \hologo{SliTeX}
% \end{longtable}
%
% \subsubsection{Font order}
%
% For all logos the font \xoption{general} is applied first.
% Example:
%\begin{quote}
%|\hologoFontSetup{general=\color{red}}|
%\end{quote}
% will print red logos.
% Then if the font uses a special font \xoption{sf}, for example,
% the font is applied that is setup by \cs{hologoLogoFontSetup}.
% If this font is not setup, then the common font setup
% by \cs{hologoFontSetup} is used. Otherwise a warning is given,
% that there is no font configured.
%
% \subsection{Additional user macros}
%
% Usually a variant of a logo is configured by using
% \cs{hologoLogoSetup}, because it is bad style to mix
% different variants of the same logo in the same text.
% There the following macros are a convenience for testing.
%
% \begin{declcs}{hologoVariant} \M{name} \M{variant}\\
%   \cs{HologoVariant} \M{name} \M{variant}
% \end{declcs}
% Logo \meta{name} is set using \meta{variant} that specifies
% explicitely which variant of the macro is used. If the argument
% is empty, then the default form of the logo is used
% (configurable by \cs{hologoLogoSetup}).
%
% \cs{HologoVariant} is used if the logo is set in a context
% that needs an uppercase first letter (beginning of a sentence, \dots).
%
% \begin{declcs}{hologoList}\\
%   \cs{hologoEntry} \M{logo} \M{variant} \M{since}
% \end{declcs}
% Macro \cs{hologoList} contains all logos that are provided
% by the package including variants. The list consists of calls
% of \cs{hologoEntry} with three arguments starting with the
% logo name \meta{logo} and its variant \meta{variant}. An empty
% variant means the current default. Argument \meta{since} specifies
% with version of the package \xpackage{hologo} is needed to get
% the logo. If the logo is fixed, then the date gets updated.
% Therefore the date \meta{since} is not exactly the date of
% the first introduction, but rather the date of the latest fix.
%
% Before \cs{hologoList} can be used, macro \cs{hologoEntry} needs
% a definition. The example file in section \ref{sec:example}
% shows applications of \cs{hologoList}.
%
% \subsection{Supported contexts}
%
% Macros \cs{hologo} and friends support special contexts:
% \begin{itemize}
% \item \hologo{LaTeX}'s protection mechanism.
% \item Bookmarks of package \xpackage{hyperref}.
% \item Package \xpackage{tex4ht}.
% \item The macros can be used inside \cs{csname} constructs,
%   if \cs{ifincsname} is available (\hologo{pdfTeX}, \hologo{XeTeX},
%   \hologo{LuaTeX}).
% \end{itemize}
%
% \subsection{Example}
% \label{sec:example}
%
% The following example prints the logos in different fonts.
%    \begin{macrocode}
%<*example>
%<<verbatim
\NeedsTeXFormat{LaTeX2e}
\documentclass[a4paper]{article}
\usepackage[
  hmargin=20mm,
  vmargin=20mm,
]{geometry}
\pagestyle{empty}
\usepackage{hologo}[2016/05/12]
\usepackage{longtable}
\usepackage{array}
\setlength{\extrarowheight}{2pt}
\usepackage[T1]{fontenc}
\usepackage{lmodern}
\usepackage{pdflscape}
\usepackage[
  pdfencoding=auto,
]{hyperref}
\hypersetup{
  pdfauthor={Heiko Oberdiek},
  pdftitle={Example for package `hologo'},
  pdfsubject={Logos with fonts lmr, lmss, qtm, qpl, qhv},
}
\usepackage{bookmark}

% Print the logo list on the console

\begingroup
  \typeout{}%
  \typeout{*** Begin of logo list ***}%
  \newcommand*{\hologoEntry}[3]{%
    \typeout{#1 \ifx\\#2\\\else(#2) \fi[#3]}%
  }%
  \hologoList
  \typeout{*** End of logo list ***}%
  \typeout{}%
\endgroup

\begin{document}
\begin{landscape}

  \section{Example file for package `hologo'}

  % Table for font names

  \begin{longtable}{>{\bfseries}ll}
    \textbf{font} & \textbf{Font name}\\
    \hline
    lmr & Latin Modern Roman\\
    lmss & Latin Modern Sans\\
    qtm & \TeX\ Gyre Termes\\
    qhv & \TeX\ Gyre Heros\\
    qpl & \TeX\ Gyre Pagella\\
  \end{longtable}

  % Logo list with logos in different fonts

  \begingroup
    \newcommand*{\SetVariant}[2]{%
      \ifx\\#2\\%
      \else
        \hologoLogoSetup{#1}{variant=#2}%
      \fi
    }%
    \newcommand*{\hologoEntry}[3]{%
      \SetVariant{#1}{#2}%
      \raisebox{1em}[0pt][0pt]{\hypertarget{#1@#2}{}}%
      \bookmark[%
        dest={#1@#2},%
      ]{%
        #1\ifx\\#2\\\else\space(#2)\fi: \Hologo{#1}, \hologo{#1} %
        [Unicode]%
      }%
      \hypersetup{unicode=false}%
      \bookmark[%
        dest={#1@#2},%
      ]{%
        #1\ifx\\#2\\\else\space(#2)\fi: \Hologo{#1}, \hologo{#1} %
        [PDFDocEncoding]%
      }%
      \texttt{#1}%
      &%
      \texttt{#2}%
      &%
      \Hologo{#1}%
      &%
      \SetVariant{#1}{#2}%
      \hologo{#1}%
      &%
      \SetVariant{#1}{#2}%
      \fontfamily{qtm}\selectfont
      \hologo{#1}%
      &%
      \SetVariant{#1}{#2}%
      \fontfamily{qpl}\selectfont
      \hologo{#1}%
      &%
      \SetVariant{#1}{#2}%
      \textsf{\hologo{#1}}%
      &%
      \SetVariant{#1}{#2}%
      \fontfamily{qhv}\selectfont
      \hologo{#1}%
      \tabularnewline
    }%
    \begin{longtable}{llllllll}%
      \textbf{\textit{logo}} & \textbf{\textit{variant}} &
      \texttt{\string\Hologo} &
      \textbf{lmr} & \textbf{qtm} & \textbf{qpl} &
      \textbf{lmss} & \textbf{qhv}
      \tabularnewline
      \hline
      \endhead
      \hologoList
    \end{longtable}%
  \endgroup

\end{landscape}
\end{document}
%verbatim
%</example>
%    \end{macrocode}
%
% \StopEventually{
% }
%
% \section{Implementation}
%    \begin{macrocode}
%<*package>
%    \end{macrocode}
%    Reload check, especially if the package is not used with \LaTeX.
%    \begin{macrocode}
\begingroup\catcode61\catcode48\catcode32=10\relax%
  \catcode13=5 % ^^M
  \endlinechar=13 %
  \catcode35=6 % #
  \catcode39=12 % '
  \catcode44=12 % ,
  \catcode45=12 % -
  \catcode46=12 % .
  \catcode58=12 % :
  \catcode64=11 % @
  \catcode123=1 % {
  \catcode125=2 % }
  \expandafter\let\expandafter\x\csname ver@hologo.sty\endcsname
  \ifx\x\relax % plain-TeX, first loading
  \else
    \def\empty{}%
    \ifx\x\empty % LaTeX, first loading,
      % variable is initialized, but \ProvidesPackage not yet seen
    \else
      \expandafter\ifx\csname PackageInfo\endcsname\relax
        \def\x#1#2{%
          \immediate\write-1{Package #1 Info: #2.}%
        }%
      \else
        \def\x#1#2{\PackageInfo{#1}{#2, stopped}}%
      \fi
      \x{hologo}{The package is already loaded}%
      \aftergroup\endinput
    \fi
  \fi
\endgroup%
%    \end{macrocode}
%    Package identification:
%    \begin{macrocode}
\begingroup\catcode61\catcode48\catcode32=10\relax%
  \catcode13=5 % ^^M
  \endlinechar=13 %
  \catcode35=6 % #
  \catcode39=12 % '
  \catcode40=12 % (
  \catcode41=12 % )
  \catcode44=12 % ,
  \catcode45=12 % -
  \catcode46=12 % .
  \catcode47=12 % /
  \catcode58=12 % :
  \catcode64=11 % @
  \catcode91=12 % [
  \catcode93=12 % ]
  \catcode123=1 % {
  \catcode125=2 % }
  \expandafter\ifx\csname ProvidesPackage\endcsname\relax
    \def\x#1#2#3[#4]{\endgroup
      \immediate\write-1{Package: #3 #4}%
      \xdef#1{#4}%
    }%
  \else
    \def\x#1#2[#3]{\endgroup
      #2[{#3}]%
      \ifx#1\@undefined
        \xdef#1{#3}%
      \fi
      \ifx#1\relax
        \xdef#1{#3}%
      \fi
    }%
  \fi
\expandafter\x\csname ver@hologo.sty\endcsname
\ProvidesPackage{hologo}%
  [2016/05/12 v1.11 A logo collection with bookmark support (HO)]%
%    \end{macrocode}
%
%    \begin{macrocode}
\begingroup\catcode61\catcode48\catcode32=10\relax%
  \catcode13=5 % ^^M
  \endlinechar=13 %
  \catcode123=1 % {
  \catcode125=2 % }
  \catcode64=11 % @
  \def\x{\endgroup
    \expandafter\edef\csname HOLOGO@AtEnd\endcsname{%
      \endlinechar=\the\endlinechar\relax
      \catcode13=\the\catcode13\relax
      \catcode32=\the\catcode32\relax
      \catcode35=\the\catcode35\relax
      \catcode61=\the\catcode61\relax
      \catcode64=\the\catcode64\relax
      \catcode123=\the\catcode123\relax
      \catcode125=\the\catcode125\relax
    }%
  }%
\x\catcode61\catcode48\catcode32=10\relax%
\catcode13=5 % ^^M
\endlinechar=13 %
\catcode35=6 % #
\catcode64=11 % @
\catcode123=1 % {
\catcode125=2 % }
\def\TMP@EnsureCode#1#2{%
  \edef\HOLOGO@AtEnd{%
    \HOLOGO@AtEnd
    \catcode#1=\the\catcode#1\relax
  }%
  \catcode#1=#2\relax
}
\TMP@EnsureCode{10}{12}% ^^J
\TMP@EnsureCode{33}{12}% !
\TMP@EnsureCode{34}{12}% "
\TMP@EnsureCode{36}{3}% $
\TMP@EnsureCode{38}{4}% &
\TMP@EnsureCode{39}{12}% '
\TMP@EnsureCode{40}{12}% (
\TMP@EnsureCode{41}{12}% )
\TMP@EnsureCode{42}{12}% *
\TMP@EnsureCode{43}{12}% +
\TMP@EnsureCode{44}{12}% ,
\TMP@EnsureCode{45}{12}% -
\TMP@EnsureCode{46}{12}% .
\TMP@EnsureCode{47}{12}% /
\TMP@EnsureCode{58}{12}% :
\TMP@EnsureCode{59}{12}% ;
\TMP@EnsureCode{60}{12}% <
\TMP@EnsureCode{62}{12}% >
\TMP@EnsureCode{63}{12}% ?
\TMP@EnsureCode{91}{12}% [
\TMP@EnsureCode{93}{12}% ]
\TMP@EnsureCode{94}{7}% ^ (superscript)
\TMP@EnsureCode{95}{8}% _ (subscript)
\TMP@EnsureCode{96}{12}% `
\TMP@EnsureCode{124}{12}% |
\edef\HOLOGO@AtEnd{%
  \HOLOGO@AtEnd
  \escapechar\the\escapechar\relax
  \noexpand\endinput
}
\escapechar=92 %
%    \end{macrocode}
%
% \subsection{Logo list}
%
%    \begin{macro}{\hologoList}
%    \begin{macrocode}
\def\hologoList{%
  \hologoEntry{(La)TeX}{}{2011/10/01}%
  \hologoEntry{AmSLaTeX}{}{2010/04/16}%
  \hologoEntry{AmSTeX}{}{2010/04/16}%
  \hologoEntry{biber}{}{2011/10/01}%
  \hologoEntry{BibTeX}{}{2011/10/01}%
  \hologoEntry{BibTeX}{sf}{2011/10/01}%
  \hologoEntry{BibTeX}{sc}{2011/10/01}%
  \hologoEntry{BibTeX8}{}{2011/11/22}%
  \hologoEntry{ConTeXt}{}{2011/03/25}%
  \hologoEntry{ConTeXt}{narrow}{2011/03/25}%
  \hologoEntry{ConTeXt}{simple}{2011/03/25}%
  \hologoEntry{emTeX}{}{2010/04/26}%
  \hologoEntry{eTeX}{}{2010/04/08}%
  \hologoEntry{ExTeX}{}{2011/10/01}%
  \hologoEntry{HanTheThanh}{}{2011/11/29}%
  \hologoEntry{iniTeX}{}{2011/10/01}%
  \hologoEntry{KOMAScript}{}{2011/10/01}%
  \hologoEntry{La}{}{2010/05/08}%
  \hologoEntry{LaTeX}{}{2010/04/08}%
  \hologoEntry{LaTeX2e}{}{2010/04/08}%
  \hologoEntry{LaTeX3}{}{2010/04/24}%
  \hologoEntry{LaTeXe}{}{2010/04/08}%
  \hologoEntry{LaTeXML}{}{2011/11/22}%
  \hologoEntry{LaTeXTeX}{}{2011/10/01}%
  \hologoEntry{LuaLaTeX}{}{2010/04/08}%
  \hologoEntry{LuaTeX}{}{2010/04/08}%
  \hologoEntry{LyX}{}{2011/10/01}%
  \hologoEntry{METAFONT}{}{2011/10/01}%
  \hologoEntry{MetaFun}{}{2011/10/01}%
  \hologoEntry{METAPOST}{}{2011/10/01}%
  \hologoEntry{MetaPost}{}{2011/10/01}%
  \hologoEntry{MiKTeX}{}{2011/10/01}%
  \hologoEntry{NTS}{}{2011/10/01}%
  \hologoEntry{OzMF}{}{2011/10/01}%
  \hologoEntry{OzMP}{}{2011/10/01}%
  \hologoEntry{OzTeX}{}{2011/10/01}%
  \hologoEntry{OzTtH}{}{2011/10/01}%
  \hologoEntry{PCTeX}{}{2011/10/01}%
  \hologoEntry{pdfTeX}{}{2011/10/01}%
  \hologoEntry{pdfLaTeX}{}{2011/10/01}%
  \hologoEntry{PiC}{}{2011/10/01}%
  \hologoEntry{PiCTeX}{}{2011/10/01}%
  \hologoEntry{plainTeX}{}{2010/04/08}%
  \hologoEntry{plainTeX}{space}{2010/04/16}%
  \hologoEntry{plainTeX}{hyphen}{2010/04/16}%
  \hologoEntry{plainTeX}{runtogether}{2010/04/16}%
  \hologoEntry{SageTeX}{}{2011/11/22}%
  \hologoEntry{SLiTeX}{}{2011/10/01}%
  \hologoEntry{SLiTeX}{lift}{2011/10/01}%
  \hologoEntry{SLiTeX}{narrow}{2011/10/01}%
  \hologoEntry{SLiTeX}{simple}{2011/10/01}%
  \hologoEntry{SliTeX}{}{2011/10/01}%
  \hologoEntry{SliTeX}{narrow}{2011/10/01}%
  \hologoEntry{SliTeX}{simple}{2011/10/01}%
  \hologoEntry{SliTeX}{lift}{2011/10/01}%
  \hologoEntry{teTeX}{}{2011/10/01}%
  \hologoEntry{TeX}{}{2010/04/08}%
  \hologoEntry{TeX4ht}{}{2011/11/22}%
  \hologoEntry{TTH}{}{2011/11/22}%
  \hologoEntry{virTeX}{}{2011/10/01}%
  \hologoEntry{VTeX}{}{2010/04/24}%
  \hologoEntry{Xe}{}{2010/04/08}%
  \hologoEntry{XeLaTeX}{}{2010/04/08}%
  \hologoEntry{XeTeX}{}{2010/04/08}%
}
%    \end{macrocode}
%    \end{macro}
%
% \subsection{Load resources}
%
%    \begin{macrocode}
\begingroup\expandafter\expandafter\expandafter\endgroup
\expandafter\ifx\csname RequirePackage\endcsname\relax
  \def\TMP@RequirePackage#1[#2]{%
    \begingroup\expandafter\expandafter\expandafter\endgroup
    \expandafter\ifx\csname ver@#1.sty\endcsname\relax
      \input #1.sty\relax
    \fi
  }%
  \TMP@RequirePackage{ltxcmds}[2011/02/04]%
  \TMP@RequirePackage{infwarerr}[2010/04/08]%
  \TMP@RequirePackage{kvsetkeys}[2010/03/01]%
  \TMP@RequirePackage{kvdefinekeys}[2010/03/01]%
  \TMP@RequirePackage{pdftexcmds}[2010/04/01]%
  \TMP@RequirePackage{ifpdf}[2010/01/28]%
  \TMP@RequirePackage{ifluatex}[2010/03/01]%
  \ltx@IfUndefined{newif}{%
    \expandafter\let\csname newif\endcsname\ltx@newif
  }{}%
  \TMP@RequirePackage{ifxetex}[2009/01/23]%
  \TMP@RequirePackage{ifvtex}[2010/03/01]%
\else
  \RequirePackage{ltxcmds}[2011/02/04]%
  \RequirePackage{infwarerr}[2010/04/08]%
  \RequirePackage{kvsetkeys}[2010/03/01]%
  \RequirePackage{kvdefinekeys}[2010/03/01]%
  \RequirePackage{pdftexcmds}[2010/04/01]%
  \RequirePackage{ifpdf}[2010/01/28]%
  \RequirePackage{ifluatex}[2010/03/01]%
  \RequirePackage{ifxetex}[2009/01/23]%
  \RequirePackage{ifvtex}[2010/03/01]%
\fi
%    \end{macrocode}
%
%    \begin{macro}{\HOLOGO@IfDefined}
%    \begin{macrocode}
\def\HOLOGO@IfExists#1{%
  \ifx\@undefined#1%
    \expandafter\ltx@secondoftwo
  \else
    \ifx\relax#1%
      \expandafter\ltx@secondoftwo
    \else
      \expandafter\expandafter\expandafter\ltx@firstoftwo
    \fi
  \fi
}
%    \end{macrocode}
%    \end{macro}
%
% \subsection{Setup macros}
%
%    \begin{macro}{\hologoSetup}
%    \begin{macrocode}
\def\hologoSetup{%
  \let\HOLOGO@name\relax
  \HOLOGO@Setup
}
%    \end{macrocode}
%    \end{macro}
%
%    \begin{macro}{\hologoLogoSetup}
%    \begin{macrocode}
\def\hologoLogoSetup#1{%
  \edef\HOLOGO@name{#1}%
  \ltx@IfUndefined{HoLogo@\HOLOGO@name}{%
    \@PackageError{hologo}{%
      Unknown logo `\HOLOGO@name'%
    }\@ehc
    \ltx@gobble
  }{%
    \HOLOGO@Setup
  }%
}
%    \end{macrocode}
%    \end{macro}
%
%    \begin{macro}{\HOLOGO@Setup}
%    \begin{macrocode}
\def\HOLOGO@Setup{%
  \kvsetkeys{HoLogo}%
}
%    \end{macrocode}
%    \end{macro}
%
% \subsection{Options}
%
%    \begin{macro}{\HOLOGO@DeclareBoolOption}
%    \begin{macrocode}
\def\HOLOGO@DeclareBoolOption#1{%
  \expandafter\chardef\csname HOLOGOOPT@#1\endcsname\ltx@zero
  \kv@define@key{HoLogo}{#1}[true]{%
    \def\HOLOGO@temp{##1}%
    \ifx\HOLOGO@temp\HOLOGO@true
      \ifx\HOLOGO@name\relax
        \expandafter\chardef\csname HOLOGOOPT@#1\endcsname=\ltx@one
      \else
        \expandafter\chardef\csname
        HoLogoOpt@#1@\HOLOGO@name\endcsname\ltx@one
      \fi
      \HOLOGO@SetBreakAll{#1}%
    \else
      \ifx\HOLOGO@temp\HOLOGO@false
        \ifx\HOLOGO@name\relax
          \expandafter\chardef\csname HOLOGOOPT@#1\endcsname=\ltx@zero
        \else
          \expandafter\chardef\csname
          HoLogoOpt@#1@\HOLOGO@name\endcsname=\ltx@zero
        \fi
        \HOLOGO@SetBreakAll{#1}%
      \else
        \@PackageError{hologo}{%
          Unknown value `##1' for boolean option `#1'.\MessageBreak
          Known values are `true' and `false'%
        }\@ehc
      \fi
    \fi
  }%
}
%    \end{macrocode}
%    \end{macro}
%
%    \begin{macro}{\HOLOGO@SetBreakAll}
%    \begin{macrocode}
\def\HOLOGO@SetBreakAll#1{%
  \def\HOLOGO@temp{#1}%
  \ifx\HOLOGO@temp\HOLOGO@break
    \ifx\HOLOGO@name\relax
      \chardef\HOLOGOOPT@hyphenbreak=\HOLOGOOPT@break
      \chardef\HOLOGOOPT@spacebreak=\HOLOGOOPT@break
      \chardef\HOLOGOOPT@discretionarybreak=\HOLOGOOPT@break
    \else
      \expandafter\chardef
         \csname HoLogoOpt@hyphenbreak@\HOLOGO@name\endcsname=%
         \csname HoLogoOpt@break@\HOLOGO@name\endcsname
      \expandafter\chardef
         \csname HoLogoOpt@spacebreak@\HOLOGO@name\endcsname=%
         \csname HoLogoOpt@break@\HOLOGO@name\endcsname
      \expandafter\chardef
         \csname HoLogoOpt@discretionarybreak@\HOLOGO@name
             \endcsname=%
         \csname HoLogoOpt@break@\HOLOGO@name\endcsname
    \fi
  \fi
}
%    \end{macrocode}
%    \end{macro}
%
%    \begin{macro}{\HOLOGO@true}
%    \begin{macrocode}
\def\HOLOGO@true{true}
%    \end{macrocode}
%    \end{macro}
%    \begin{macro}{\HOLOGO@false}
%    \begin{macrocode}
\def\HOLOGO@false{false}
%    \end{macrocode}
%    \end{macro}
%    \begin{macro}{\HOLOGO@break}
%    \begin{macrocode}
\def\HOLOGO@break{break}
%    \end{macrocode}
%    \end{macro}
%
%    \begin{macrocode}
\HOLOGO@DeclareBoolOption{break}
\HOLOGO@DeclareBoolOption{hyphenbreak}
\HOLOGO@DeclareBoolOption{spacebreak}
\HOLOGO@DeclareBoolOption{discretionarybreak}
%    \end{macrocode}
%
%    \begin{macrocode}
\kv@define@key{HoLogo}{variant}{%
  \ifx\HOLOGO@name\relax
    \@PackageError{hologo}{%
      Option `variant' is not available in \string\hologoSetup,%
      \MessageBreak
      Use \string\hologoLogoSetup\space instead%
    }\@ehc
  \else
    \edef\HOLOGO@temp{#1}%
    \ifx\HOLOGO@temp\ltx@empty
      \expandafter
      \let\csname HoLogoOpt@variant@\HOLOGO@name\endcsname\@undefined
    \else
      \ltx@IfUndefined{HoLogo@\HOLOGO@name @\HOLOGO@temp}{%
        \@PackageError{hologo}{%
          Unknown variant `\HOLOGO@temp' of logo `\HOLOGO@name'%
        }\@ehc
      }{%
        \expandafter
        \let\csname HoLogoOpt@variant@\HOLOGO@name\endcsname
            \HOLOGO@temp
      }%
    \fi
  \fi
}
%    \end{macrocode}
%
%    \begin{macro}{\HOLOGO@Variant}
%    \begin{macrocode}
\def\HOLOGO@Variant#1{%
  #1%
  \ltx@ifundefined{HoLogoOpt@variant@#1}{%
  }{%
    @\csname HoLogoOpt@variant@#1\endcsname
  }%
}
%    \end{macrocode}
%    \end{macro}
%
% \subsection{Break/no-break support}
%
%    \begin{macro}{\HOLOGO@space}
%    \begin{macrocode}
\def\HOLOGO@space{%
  \ltx@ifundefined{HoLogoOpt@spacebreak@\HOLOGO@name}{%
    \ltx@ifundefined{HoLogoOpt@break@\HOLOGO@name}{%
      \chardef\HOLOGO@temp=\HOLOGOOPT@spacebreak
    }{%
      \chardef\HOLOGO@temp=%
        \csname HoLogoOpt@break@\HOLOGO@name\endcsname
    }%
  }{%
    \chardef\HOLOGO@temp=%
      \csname HoLogoOpt@spacebreak@\HOLOGO@name\endcsname
  }%
  \ifcase\HOLOGO@temp
    \penalty10000 %
  \fi
  \ltx@space
}
%    \end{macrocode}
%    \end{macro}
%
%    \begin{macro}{\HOLOGO@hyphen}
%    \begin{macrocode}
\def\HOLOGO@hyphen{%
  \ltx@ifundefined{HoLogoOpt@hyphenbreak@\HOLOGO@name}{%
    \ltx@ifundefined{HoLogoOpt@break@\HOLOGO@name}{%
      \chardef\HOLOGO@temp=\HOLOGOOPT@hyphenbreak
    }{%
      \chardef\HOLOGO@temp=%
        \csname HoLogoOpt@break@\HOLOGO@name\endcsname
    }%
  }{%
    \chardef\HOLOGO@temp=%
      \csname HoLogoOpt@hyphenbreak@\HOLOGO@name\endcsname
  }%
  \ifcase\HOLOGO@temp
    \ltx@mbox{-}%
  \else
    -%
  \fi
}
%    \end{macrocode}
%    \end{macro}
%
%    \begin{macro}{\HOLOGO@discretionary}
%    \begin{macrocode}
\def\HOLOGO@discretionary{%
  \ltx@ifundefined{HoLogoOpt@discretionarybreak@\HOLOGO@name}{%
    \ltx@ifundefined{HoLogoOpt@break@\HOLOGO@name}{%
      \chardef\HOLOGO@temp=\HOLOGOOPT@discretionarybreak
    }{%
      \chardef\HOLOGO@temp=%
        \csname HoLogoOpt@break@\HOLOGO@name\endcsname
    }%
  }{%
    \chardef\HOLOGO@temp=%
      \csname HoLogoOpt@discretionarybreak@\HOLOGO@name\endcsname
  }%
  \ifcase\HOLOGO@temp
  \else
    \-%
  \fi
}
%    \end{macrocode}
%    \end{macro}
%
%    \begin{macro}{\HOLOGO@mbox}
%    \begin{macrocode}
\def\HOLOGO@mbox#1{%
  \ltx@ifundefined{HoLogoOpt@break@\HOLOGO@name}{%
    \chardef\HOLOGO@temp=\HOLOGOOPT@hyphenbreak
  }{%
    \chardef\HOLOGO@temp=%
      \csname HoLogoOpt@break@\HOLOGO@name\endcsname
  }%
  \ifcase\HOLOGO@temp
    \ltx@mbox{#1}%
  \else
    #1%
  \fi
}
%    \end{macrocode}
%    \end{macro}
%
% \subsection{Font support}
%
%    \begin{macro}{\HoLogoFont@font}
%    \begin{tabular}{@{}ll@{}}
%    |#1|:& logo name\\
%    |#2|:& font short name\\
%    |#3|:& text
%    \end{tabular}
%    \begin{macrocode}
\def\HoLogoFont@font#1#2#3{%
  \begingroup
    \ltx@IfUndefined{HoLogoFont@logo@#1.#2}{%
      \ltx@IfUndefined{HoLogoFont@font@#2}{%
        \@PackageWarning{hologo}{%
          Missing font `#2' for logo `#1'%
        }%
        #3%
      }{%
        \csname HoLogoFont@font@#2\endcsname{#3}%
      }%
    }{%
      \csname HoLogoFont@logo@#1.#2\endcsname{#3}%
    }%
  \endgroup
}
%    \end{macrocode}
%    \end{macro}
%
%    \begin{macro}{\HoLogoFont@Def}
%    \begin{macrocode}
\def\HoLogoFont@Def#1{%
  \expandafter\def\csname HoLogoFont@font@#1\endcsname
}
%    \end{macrocode}
%    \end{macro}
%    \begin{macro}{\HoLogoFont@LogoDef}
%    \begin{macrocode}
\def\HoLogoFont@LogoDef#1#2{%
  \expandafter\def\csname HoLogoFont@logo@#1.#2\endcsname
}
%    \end{macrocode}
%    \end{macro}
%
% \subsubsection{Font defaults}
%
%    \begin{macro}{\HoLogoFont@font@general}
%    \begin{macrocode}
\HoLogoFont@Def{general}{}%
%    \end{macrocode}
%    \end{macro}
%
%    \begin{macro}{\HoLogoFont@font@rm}
%    \begin{macrocode}
\ltx@IfUndefined{rmfamily}{%
  \ltx@IfUndefined{rm}{%
  }{%
    \HoLogoFont@Def{rm}{\rm}%
  }%
}{%
  \HoLogoFont@Def{rm}{\rmfamily}%
}
%    \end{macrocode}
%    \end{macro}
%
%    \begin{macro}{\HoLogoFont@font@sf}
%    \begin{macrocode}
\ltx@IfUndefined{sffamily}{%
  \ltx@IfUndefined{sf}{%
  }{%
    \HoLogoFont@Def{sf}{\sf}%
  }%
}{%
  \HoLogoFont@Def{sf}{\sffamily}%
}
%    \end{macrocode}
%    \end{macro}
%
%    \begin{macro}{\HoLogoFont@font@bibsf}
%    In case of \hologo{plainTeX} the original small caps
%    variant is used as default. In \hologo{LaTeX}
%    the definition of package \xpackage{dtklogos} \cite{dtklogos}
%    is used.
%\begin{quote}
%\begin{verbatim}
%\DeclareRobustCommand{\BibTeX}{%
%  B%
%  \kern-.05em%
%  \hbox{%
%    $\m@th$% %% force math size calculations
%    \csname S@\f@size\endcsname
%    \fontsize\sf@size\z@
%    \math@fontsfalse
%    \selectfont
%    I%
%    \kern-.025em%
%    B
%  }%
%  \kern-.08em%
%  \-%
%  \TeX
%}
%\end{verbatim}
%\end{quote}
%    \begin{macrocode}
\ltx@IfUndefined{selectfont}{%
  \ltx@IfUndefined{tensc}{%
    \font\tensc=cmcsc10\relax
  }{}%
  \HoLogoFont@Def{bibsf}{\tensc}%
}{%
  \HoLogoFont@Def{bibsf}{%
    $\mathsurround=0pt$%
    \csname S@\f@size\endcsname
    \fontsize\sf@size{0pt}%
    \math@fontsfalse
    \selectfont
  }%
}
%    \end{macrocode}
%    \end{macro}
%
%    \begin{macro}{\HoLogoFont@font@sc}
%    \begin{macrocode}
\ltx@IfUndefined{scshape}{%
  \ltx@IfUndefined{tensc}{%
    \font\tensc=cmcsc10\relax
  }{}%
  \HoLogoFont@Def{sc}{\tensc}%
}{%
  \HoLogoFont@Def{sc}{\scshape}%
}
%    \end{macrocode}
%    \end{macro}
%
%    \begin{macro}{\HoLogoFont@font@sy}
%    \begin{macrocode}
\ltx@IfUndefined{usefont}{%
  \ltx@IfUndefined{tensy}{%
  }{%
    \HoLogoFont@Def{sy}{\tensy}%
  }%
}{%
  \HoLogoFont@Def{sy}{%
    \usefont{OMS}{cmsy}{m}{n}%
  }%
}
%    \end{macrocode}
%    \end{macro}
%
%    \begin{macro}{\HoLogoFont@font@logo}
%    \begin{macrocode}
\begingroup
  \def\x{LaTeX2e}%
\expandafter\endgroup
\ifx\fmtname\x
  \ltx@IfUndefined{logofamily}{%
    \DeclareRobustCommand\logofamily{%
      \not@math@alphabet\logofamily\relax
      \fontencoding{U}%
      \fontfamily{logo}%
      \selectfont
    }%
  }{}%
  \ltx@IfUndefined{logofamily}{%
  }{%
    \HoLogoFont@Def{logo}{\logofamily}%
  }%
\else
  \ltx@IfUndefined{tenlogo}{%
    \font\tenlogo=logo10\relax
  }{}%
  \HoLogoFont@Def{logo}{\tenlogo}%
\fi
%    \end{macrocode}
%    \end{macro}
%
% \subsubsection{Font setup}
%
%    \begin{macro}{\hologoFontSetup}
%    \begin{macrocode}
\def\hologoFontSetup{%
  \let\HOLOGO@name\relax
  \HOLOGO@FontSetup
}
%    \end{macrocode}
%    \end{macro}
%
%    \begin{macro}{\hologoLogoFontSetup}
%    \begin{macrocode}
\def\hologoLogoFontSetup#1{%
  \edef\HOLOGO@name{#1}%
  \ltx@IfUndefined{HoLogo@\HOLOGO@name}{%
    \@PackageError{hologo}{%
      Unknown logo `\HOLOGO@name'%
    }\@ehc
    \ltx@gobble
  }{%
    \HOLOGO@FontSetup
  }%
}
%    \end{macrocode}
%    \end{macro}
%
%    \begin{macro}{\HOLOGO@FontSetup}
%    \begin{macrocode}
\def\HOLOGO@FontSetup{%
  \kvsetkeys{HoLogoFont}%
}
%    \end{macrocode}
%    \end{macro}
%
%    \begin{macrocode}
\def\HOLOGO@temp#1{%
  \kv@define@key{HoLogoFont}{#1}{%
    \ifx\HOLOGO@name\relax
      \HoLogoFont@Def{#1}{##1}%
    \else
      \HoLogoFont@LogoDef\HOLOGO@name{#1}{##1}%
    \fi
  }%
}
\HOLOGO@temp{general}
\HOLOGO@temp{sf}
%    \end{macrocode}
%
% \subsection{Generic logo commands}
%
%    \begin{macrocode}
\HOLOGO@IfExists\hologo{%
  \@PackageError{hologo}{%
    \string\hologo\ltx@space is already defined.\MessageBreak
    Package loading is aborted%
  }\@ehc
  \HOLOGO@AtEnd
}%
\HOLOGO@IfExists\hologoRobust{%
  \@PackageError{hologo}{%
    \string\hologoRobust\ltx@space is already defined.\MessageBreak
    Package loading is aborted%
  }\@ehc
  \HOLOGO@AtEnd
}%
%    \end{macrocode}
%
% \subsubsection{\cs{hologo} and friends}
%
%    \begin{macrocode}
\ifluatex
  \expandafter\ltx@firstofone
\else
  \expandafter\ltx@gobble
\fi
{%
  \ltx@IfUndefined{ifincsname}{%
    \ifnum\luatexversion<36 %
      \expandafter\ltx@gobble
    \else
      \expandafter\ltx@firstofone
    \fi
    {%
      \begingroup
        \ifcase0%
            \directlua{%
              if tex.enableprimitives then %
                tex.enableprimitives('HOLOGO@', {'ifincsname'})%
              else %
                tex.print('1')%
              end%
            }%
            \ifx\HOLOGO@ifincsname\@undefined 1\fi%
            \relax
          \expandafter\ltx@firstofone
        \else
          \endgroup
          \expandafter\ltx@gobble
        \fi
        {%
          \global\let\ifincsname\HOLOGO@ifincsname
        }%
      \HOLOGO@temp
    }%
  }{}%
}
%    \end{macrocode}
%    \begin{macrocode}
\ltx@IfUndefined{ifincsname}{%
  \catcode`$=14 %
}{%
  \catcode`$=9 %
}
%    \end{macrocode}
%
%    \begin{macro}{\hologo}
%    \begin{macrocode}
\def\hologo#1{%
$ \ifincsname
$   \ltx@ifundefined{HoLogoCs@\HOLOGO@Variant{#1}}{%
$     #1%
$   }{%
$     \csname HoLogoCs@\HOLOGO@Variant{#1}\endcsname\ltx@firstoftwo
$   }%
$ \else
    \HOLOGO@IfExists\texorpdfstring\texorpdfstring\ltx@firstoftwo
    {%
      \hologoRobust{#1}%
    }{%
      \ltx@ifundefined{HoLogoBkm@\HOLOGO@Variant{#1}}{%
        \ltx@ifundefined{HoLogo@#1}{?#1?}{#1}%
      }{%
        \csname HoLogoBkm@\HOLOGO@Variant{#1}\endcsname
        \ltx@firstoftwo
      }%
    }%
$ \fi
}
%    \end{macrocode}
%    \end{macro}
%    \begin{macro}{\Hologo}
%    \begin{macrocode}
\def\Hologo#1{%
$ \ifincsname
$   \ltx@ifundefined{HoLogoCs@\HOLOGO@Variant{#1}}{%
$     #1%
$   }{%
$     \csname HoLogoCs@\HOLOGO@Variant{#1}\endcsname\ltx@secondoftwo
$   }%
$ \else
    \HOLOGO@IfExists\texorpdfstring\texorpdfstring\ltx@firstoftwo
    {%
      \HologoRobust{#1}%
    }{%
      \ltx@ifundefined{HoLogoBkm@\HOLOGO@Variant{#1}}{%
        \ltx@ifundefined{HoLogo@#1}{?#1?}{#1}%
      }{%
        \csname HoLogoBkm@\HOLOGO@Variant{#1}\endcsname
        \ltx@secondoftwo
      }%
    }%
$ \fi
}
%    \end{macrocode}
%    \end{macro}
%
%    \begin{macro}{\hologoVariant}
%    \begin{macrocode}
\def\hologoVariant#1#2{%
  \ifx\relax#2\relax
    \hologo{#1}%
  \else
$   \ifincsname
$     \ltx@ifundefined{HoLogoCs@#1@#2}{%
$       #1%
$     }{%
$       \csname HoLogoCs@#1@#2\endcsname\ltx@firstoftwo
$     }%
$   \else
      \HOLOGO@IfExists\texorpdfstring\texorpdfstring\ltx@firstoftwo
      {%
        \hologoVariantRobust{#1}{#2}%
      }{%
        \ltx@ifundefined{HoLogoBkm@#1@#2}{%
          \ltx@ifundefined{HoLogo@#1}{?#1?}{#1}%
        }{%
          \csname HoLogoBkm@#1@#2\endcsname
          \ltx@firstoftwo
        }%
      }%
$   \fi
  \fi
}
%    \end{macrocode}
%    \end{macro}
%    \begin{macro}{\HologoVariant}
%    \begin{macrocode}
\def\HologoVariant#1#2{%
  \ifx\relax#2\relax
    \Hologo{#1}%
  \else
$   \ifincsname
$     \ltx@ifundefined{HoLogoCs@#1@#2}{%
$       #1%
$     }{%
$       \csname HoLogoCs@#1@#2\endcsname\ltx@secondoftwo
$     }%
$   \else
      \HOLOGO@IfExists\texorpdfstring\texorpdfstring\ltx@firstoftwo
      {%
        \HologoVariantRobust{#1}{#2}%
      }{%
        \ltx@ifundefined{HoLogoBkm@#1@#2}{%
          \ltx@ifundefined{HoLogo@#1}{?#1?}{#1}%
        }{%
          \csname HoLogoBkm@#1@#2\endcsname
          \ltx@secondoftwo
        }%
      }%
$   \fi
  \fi
}
%    \end{macrocode}
%    \end{macro}
%
%    \begin{macrocode}
\catcode`\$=3 %
%    \end{macrocode}
%
% \subsubsection{\cs{hologoRobust} and friends}
%
%    \begin{macro}{\hologoRobust}
%    \begin{macrocode}
\ltx@IfUndefined{protected}{%
  \ltx@IfUndefined{DeclareRobustCommand}{%
    \def\hologoRobust#1%
  }{%
    \DeclareRobustCommand*\hologoRobust[1]%
  }%
}{%
  \protected\def\hologoRobust#1%
}%
{%
  \edef\HOLOGO@name{#1}%
  \ltx@IfUndefined{HoLogo@\HOLOGO@Variant\HOLOGO@name}{%
    \@PackageError{hologo}{%
      Unknown logo `\HOLOGO@name'%
    }\@ehc
    ?\HOLOGO@name?%
  }{%
    \ltx@IfUndefined{ver@tex4ht.sty}{%
      \HoLogoFont@font\HOLOGO@name{general}{%
        \csname HoLogo@\HOLOGO@Variant\HOLOGO@name\endcsname
        \ltx@firstoftwo
      }%
    }{%
      \ltx@IfUndefined{HoLogoHtml@\HOLOGO@Variant\HOLOGO@name}{%
        \HOLOGO@name
      }{%
        \csname HoLogoHtml@\HOLOGO@Variant\HOLOGO@name\endcsname
        \ltx@firstoftwo
      }%
    }%
  }%
}
%    \end{macrocode}
%    \end{macro}
%    \begin{macro}{\HologoRobust}
%    \begin{macrocode}
\ltx@IfUndefined{protected}{%
  \ltx@IfUndefined{DeclareRobustCommand}{%
    \def\HologoRobust#1%
  }{%
    \DeclareRobustCommand*\HologoRobust[1]%
  }%
}{%
  \protected\def\HologoRobust#1%
}%
{%
  \edef\HOLOGO@name{#1}%
  \ltx@IfUndefined{HoLogo@\HOLOGO@Variant\HOLOGO@name}{%
    \@PackageError{hologo}{%
      Unknown logo `\HOLOGO@name'%
    }\@ehc
    ?\HOLOGO@name?%
  }{%
    \ltx@IfUndefined{ver@tex4ht.sty}{%
      \HoLogoFont@font\HOLOGO@name{general}{%
        \csname HoLogo@\HOLOGO@Variant\HOLOGO@name\endcsname
        \ltx@secondoftwo
      }%
    }{%
      \ltx@IfUndefined{HoLogoHtml@\HOLOGO@Variant\HOLOGO@name}{%
        \expandafter\HOLOGO@Uppercase\HOLOGO@name
      }{%
        \csname HoLogoHtml@\HOLOGO@Variant\HOLOGO@name\endcsname
        \ltx@secondoftwo
      }%
    }%
  }%
}
%    \end{macrocode}
%    \end{macro}
%    \begin{macro}{\hologoVariantRobust}
%    \begin{macrocode}
\ltx@IfUndefined{protected}{%
  \ltx@IfUndefined{DeclareRobustCommand}{%
    \def\hologoVariantRobust#1#2%
  }{%
    \DeclareRobustCommand*\hologoVariantRobust[2]%
  }%
}{%
  \protected\def\hologoVariantRobust#1#2%
}%
{%
  \begingroup
    \hologoLogoSetup{#1}{variant={#2}}%
    \hologoRobust{#1}%
  \endgroup
}
%    \end{macrocode}
%    \end{macro}
%    \begin{macro}{\HologoVariantRobust}
%    \begin{macrocode}
\ltx@IfUndefined{protected}{%
  \ltx@IfUndefined{DeclareRobustCommand}{%
    \def\HologoVariantRobust#1#2%
  }{%
    \DeclareRobustCommand*\HologoVariantRobust[2]%
  }%
}{%
  \protected\def\HologoVariantRobust#1#2%
}%
{%
  \begingroup
    \hologoLogoSetup{#1}{variant={#2}}%
    \HologoRobust{#1}%
  \endgroup
}
%    \end{macrocode}
%    \end{macro}
%
%    \begin{macro}{\hologorobust}
%    Macro \cs{hologorobust} is only defined for compatibility.
%    Its use is deprecated.
%    \begin{macrocode}
\def\hologorobust{\hologoRobust}
%    \end{macrocode}
%    \end{macro}
%
% \subsection{Helpers}
%
%    \begin{macro}{\HOLOGO@Uppercase}
%    Macro \cs{HOLOGO@Uppercase} is restricted to \cs{uppercase},
%    because \hologo{plainTeX} or \hologo{iniTeX} do not provide
%    \cs{MakeUppercase}.
%    \begin{macrocode}
\def\HOLOGO@Uppercase#1{\uppercase{#1}}
%    \end{macrocode}
%    \end{macro}
%
%    \begin{macro}{\HOLOGO@PdfdocUnicode}
%    \begin{macrocode}
\def\HOLOGO@PdfdocUnicode{%
  \ifx\ifHy@unicode\iftrue
    \expandafter\ltx@secondoftwo
  \else
    \expandafter\ltx@firstoftwo
  \fi
}
%    \end{macrocode}
%    \end{macro}
%
%    \begin{macro}{\HOLOGO@Math}
%    \begin{macrocode}
\def\HOLOGO@MathSetup{%
  \mathsurround0pt\relax
  \HOLOGO@IfExists\f@series{%
    \if b\expandafter\ltx@car\f@series x\@nil
      \csname boldmath\endcsname
   \fi
  }{}%
}
%    \end{macrocode}
%    \end{macro}
%
%    \begin{macro}{\HOLOGO@TempDimen}
%    \begin{macrocode}
\dimendef\HOLOGO@TempDimen=\ltx@zero
%    \end{macrocode}
%    \end{macro}
%    \begin{macro}{\HOLOGO@NegativeKerning}
%    \begin{macrocode}
\def\HOLOGO@NegativeKerning#1{%
  \begingroup
    \HOLOGO@TempDimen=0pt\relax
    \comma@parse@normalized{#1}{%
      \ifdim\HOLOGO@TempDimen=0pt %
        \expandafter\HOLOGO@@NegativeKerning\comma@entry
      \fi
      \ltx@gobble
    }%
    \ifdim\HOLOGO@TempDimen<0pt %
      \kern\HOLOGO@TempDimen
    \fi
  \endgroup
}
%    \end{macrocode}
%    \end{macro}
%    \begin{macro}{\HOLOGO@@NegativeKerning}
%    \begin{macrocode}
\def\HOLOGO@@NegativeKerning#1#2{%
  \setbox\ltx@zero\hbox{#1#2}%
  \HOLOGO@TempDimen=\wd\ltx@zero
  \setbox\ltx@zero\hbox{#1\kern0pt#2}%
  \advance\HOLOGO@TempDimen by -\wd\ltx@zero
}
%    \end{macrocode}
%    \end{macro}
%
%    \begin{macro}{\HOLOGO@SpaceFactor}
%    \begin{macrocode}
\def\HOLOGO@SpaceFactor{%
  \spacefactor1000 %
}
%    \end{macrocode}
%    \end{macro}
%
%    \begin{macro}{\HOLOGO@Span}
%    \begin{macrocode}
\def\HOLOGO@Span#1#2{%
  \HCode{<span class="HoLogo-#1">}%
  #2%
  \HCode{</span>}%
}
%    \end{macrocode}
%    \end{macro}
%
% \subsubsection{Text subscript}
%
%    \begin{macro}{\HOLOGO@SubScript}%
%    \begin{macrocode}
\def\HOLOGO@SubScript#1{%
  \ltx@IfUndefined{textsubscript}{%
    \ltx@IfUndefined{text}{%
      \ltx@mbox{%
        \mathsurround=0pt\relax
        $%
          _{%
            \ltx@IfUndefined{sf@size}{%
              \mathrm{#1}%
            }{%
              \mbox{%
                \fontsize\sf@size{0pt}\selectfont
                #1%
              }%
            }%
          }%
        $%
      }%
    }{%
      \ltx@mbox{%
        \mathsurround=0pt\relax
        $_{\text{#1}}$%
      }%
    }%
  }{%
    \textsubscript{#1}%
  }%
}
%    \end{macrocode}
%    \end{macro}
%
% \subsection{\hologo{TeX} and friends}
%
% \subsubsection{\hologo{TeX}}
%
%    \begin{macro}{\HoLogo@TeX}
%    Source: \hologo{LaTeX} kernel.
%    \begin{macrocode}
\def\HoLogo@TeX#1{%
  T\kern-.1667em\lower.5ex\hbox{E}\kern-.125emX\HOLOGO@SpaceFactor
}
%    \end{macrocode}
%    \end{macro}
%    \begin{macro}{\HoLogoHtml@TeX}
%    \begin{macrocode}
\def\HoLogoHtml@TeX#1{%
  \HoLogoCss@TeX
  \HOLOGO@Span{TeX}{%
    T%
    \HOLOGO@Span{e}{%
      E%
    }%
    X%
  }%
}
%    \end{macrocode}
%    \end{macro}
%    \begin{macro}{\HoLogoCss@TeX}
%    \begin{macrocode}
\def\HoLogoCss@TeX{%
  \Css{%
    span.HoLogo-TeX span.HoLogo-e{%
      position:relative;%
      top:.5ex;%
      margin-left:-.1667em;%
      margin-right:-.125em;%
    }%
  }%
  \Css{%
    a span.HoLogo-TeX span.HoLogo-e{%
      text-decoration:none;%
    }%
  }%
  \global\let\HoLogoCss@TeX\relax
}
%    \end{macrocode}
%    \end{macro}
%
% \subsubsection{\hologo{plainTeX}}
%
%    \begin{macro}{\HoLogo@plainTeX@space}
%    Source: ``The \hologo{TeX}book''
%    \begin{macrocode}
\def\HoLogo@plainTeX@space#1{%
  \HOLOGO@mbox{#1{p}{P}lain}\HOLOGO@space\hologo{TeX}%
}
%    \end{macrocode}
%    \end{macro}
%    \begin{macro}{\HoLogoCs@plainTeX@space}
%    \begin{macrocode}
\def\HoLogoCs@plainTeX@space#1{#1{p}{P}lain TeX}%
%    \end{macrocode}
%    \end{macro}
%    \begin{macro}{\HoLogoBkm@plainTeX@space}
%    \begin{macrocode}
\def\HoLogoBkm@plainTeX@space#1{%
  #1{p}{P}lain \hologo{TeX}%
}
%    \end{macrocode}
%    \end{macro}
%    \begin{macro}{\HoLogoHtml@plainTeX@space}
%    \begin{macrocode}
\def\HoLogoHtml@plainTeX@space#1{%
  #1{p}{P}lain \hologo{TeX}%
}
%    \end{macrocode}
%    \end{macro}
%
%    \begin{macro}{\HoLogo@plainTeX@hyphen}
%    \begin{macrocode}
\def\HoLogo@plainTeX@hyphen#1{%
  \HOLOGO@mbox{#1{p}{P}lain}\HOLOGO@hyphen\hologo{TeX}%
}
%    \end{macrocode}
%    \end{macro}
%    \begin{macro}{\HoLogoCs@plainTeX@hyphen}
%    \begin{macrocode}
\def\HoLogoCs@plainTeX@hyphen#1{#1{p}{P}lain-TeX}
%    \end{macrocode}
%    \end{macro}
%    \begin{macro}{\HoLogoBkm@plainTeX@hyphen}
%    \begin{macrocode}
\def\HoLogoBkm@plainTeX@hyphen#1{%
  #1{p}{P}lain-\hologo{TeX}%
}
%    \end{macrocode}
%    \end{macro}
%    \begin{macro}{\HoLogoHtml@plainTeX@hyphen}
%    \begin{macrocode}
\def\HoLogoHtml@plainTeX@hyphen#1{%
  #1{p}{P}lain-\hologo{TeX}%
}
%    \end{macrocode}
%    \end{macro}
%
%    \begin{macro}{\HoLogo@plainTeX@runtogether}
%    \begin{macrocode}
\def\HoLogo@plainTeX@runtogether#1{%
  \HOLOGO@mbox{#1{p}{P}lain\hologo{TeX}}%
}
%    \end{macrocode}
%    \end{macro}
%    \begin{macro}{\HoLogoCs@plainTeX@runtogether}
%    \begin{macrocode}
\def\HoLogoCs@plainTeX@runtogether#1{#1{p}{P}lainTeX}
%    \end{macrocode}
%    \end{macro}
%    \begin{macro}{\HoLogoBkm@plainTeX@runtogether}
%    \begin{macrocode}
\def\HoLogoBkm@plainTeX@runtogether#1{%
  #1{p}{P}lain\hologo{TeX}%
}
%    \end{macrocode}
%    \end{macro}
%    \begin{macro}{\HoLogoHtml@plainTeX@runtogether}
%    \begin{macrocode}
\def\HoLogoHtml@plainTeX@runtogether#1{%
  #1{p}{P}lain\hologo{TeX}%
}
%    \end{macrocode}
%    \end{macro}
%
%    \begin{macro}{\HoLogo@plainTeX}
%    \begin{macrocode}
\def\HoLogo@plainTeX{\HoLogo@plainTeX@space}
%    \end{macrocode}
%    \end{macro}
%    \begin{macro}{\HoLogoCs@plainTeX}
%    \begin{macrocode}
\def\HoLogoCs@plainTeX{\HoLogoCs@plainTeX@space}
%    \end{macrocode}
%    \end{macro}
%    \begin{macro}{\HoLogoBkm@plainTeX}
%    \begin{macrocode}
\def\HoLogoBkm@plainTeX{\HoLogoBkm@plainTeX@space}
%    \end{macrocode}
%    \end{macro}
%    \begin{macro}{\HoLogoHtml@plainTeX}
%    \begin{macrocode}
\def\HoLogoHtml@plainTeX{\HoLogoHtml@plainTeX@space}
%    \end{macrocode}
%    \end{macro}
%
% \subsubsection{\hologo{LaTeX}}
%
%    Source: \hologo{LaTeX} kernel.
%\begin{quote}
%\begin{verbatim}
%\DeclareRobustCommand{\LaTeX}{%
%  L%
%  \kern-.36em%
%  {%
%    \sbox\z@ T%
%    \vbox to\ht\z@{%
%      \hbox{%
%        \check@mathfonts
%        \fontsize\sf@size\z@
%        \math@fontsfalse
%        \selectfont
%        A%
%      }%
%      \vss
%    }%
%  }%
%  \kern-.15em%
%  \TeX
%}
%\end{verbatim}
%\end{quote}
%
%    \begin{macro}{\HoLogo@La}
%    \begin{macrocode}
\def\HoLogo@La#1{%
  L%
  \kern-.36em%
  \begingroup
    \setbox\ltx@zero\hbox{T}%
    \vbox to\ht\ltx@zero{%
      \hbox{%
        \ltx@ifundefined{check@mathfonts}{%
          \csname sevenrm\endcsname
        }{%
          \check@mathfonts
          \fontsize\sf@size{0pt}%
          \math@fontsfalse\selectfont
        }%
        A%
      }%
      \vss
    }%
  \endgroup
}
%    \end{macrocode}
%    \end{macro}
%
%    \begin{macro}{\HoLogo@LaTeX}
%    Source: \hologo{LaTeX} kernel.
%    \begin{macrocode}
\def\HoLogo@LaTeX#1{%
  \hologo{La}%
  \kern-.15em%
  \hologo{TeX}%
}
%    \end{macrocode}
%    \end{macro}
%    \begin{macro}{\HoLogoHtml@LaTeX}
%    \begin{macrocode}
\def\HoLogoHtml@LaTeX#1{%
  \HoLogoCss@LaTeX
  \HOLOGO@Span{LaTeX}{%
    L%
    \HOLOGO@Span{a}{%
      A%
    }%
    \hologo{TeX}%
  }%
}
%    \end{macrocode}
%    \end{macro}
%    \begin{macro}{\HoLogoCss@LaTeX}
%    \begin{macrocode}
\def\HoLogoCss@LaTeX{%
  \Css{%
    span.HoLogo-LaTeX span.HoLogo-a{%
      position:relative;%
      top:-.5ex;%
      margin-left:-.36em;%
      margin-right:-.15em;%
      font-size:85\%;%
    }%
  }%
  \global\let\HoLogoCss@LaTeX\relax
}
%    \end{macrocode}
%    \end{macro}
%
% \subsubsection{\hologo{(La)TeX}}
%
%    \begin{macro}{\HoLogo@LaTeXTeX}
%    The kerning around the parentheses is taken
%    from package \xpackage{dtklogos} \cite{dtklogos}.
%\begin{quote}
%\begin{verbatim}
%\DeclareRobustCommand{\LaTeXTeX}{%
%  (%
%  \kern-.15em%
%  L%
%  \kern-.36em%
%  {%
%    \sbox\z@ T%
%    \vbox to\ht0{%
%      \hbox{%
%        $\m@th$%
%        \csname S@\f@size\endcsname
%        \fontsize\sf@size\z@
%        \math@fontsfalse
%        \selectfont
%        A%
%      }%
%      \vss
%    }%
%  }%
%  \kern-.2em%
%  )%
%  \kern-.15em%
%  \TeX
%}
%\end{verbatim}
%\end{quote}
%    \begin{macrocode}
\def\HoLogo@LaTeXTeX#1{%
  (%
  \kern-.15em%
  \hologo{La}%
  \kern-.2em%
  )%
  \kern-.15em%
  \hologo{TeX}%
}
%    \end{macrocode}
%    \end{macro}
%    \begin{macro}{\HoLogoBkm@LaTeXTeX}
%    \begin{macrocode}
\def\HoLogoBkm@LaTeXTeX#1{(La)TeX}
%    \end{macrocode}
%    \end{macro}
%
%    \begin{macro}{\HoLogo@(La)TeX}
%    \begin{macrocode}
\expandafter
\let\csname HoLogo@(La)TeX\endcsname\HoLogo@LaTeXTeX
%    \end{macrocode}
%    \end{macro}
%    \begin{macro}{\HoLogoBkm@(La)TeX}
%    \begin{macrocode}
\expandafter
\let\csname HoLogoBkm@(La)TeX\endcsname\HoLogoBkm@LaTeXTeX
%    \end{macrocode}
%    \end{macro}
%    \begin{macro}{\HoLogoHtml@LaTeXTeX}
%    \begin{macrocode}
\def\HoLogoHtml@LaTeXTeX#1{%
  \HoLogoCss@LaTeXTeX
  \HOLOGO@Span{LaTeXTeX}{%
    (%
    \HOLOGO@Span{L}{L}%
    \HOLOGO@Span{a}{A}%
    \HOLOGO@Span{ParenRight}{)}%
    \hologo{TeX}%
  }%
}
%    \end{macrocode}
%    \end{macro}
%    \begin{macro}{\HoLogoHtml@(La)TeX}
%    Kerning after opening parentheses and before closing parentheses
%    is $-0.1$\,em. The original values $-0.15$\,em
%    looked too ugly for a serif font.
%    \begin{macrocode}
\expandafter
\let\csname HoLogoHtml@(La)TeX\endcsname\HoLogoHtml@LaTeXTeX
%    \end{macrocode}
%    \end{macro}
%    \begin{macro}{\HoLogoCss@LaTeXTeX}
%    \begin{macrocode}
\def\HoLogoCss@LaTeXTeX{%
  \Css{%
    span.HoLogo-LaTeXTeX span.HoLogo-L{%
      margin-left:-.1em;%
    }%
  }%
  \Css{%
    span.HoLogo-LaTeXTeX span.HoLogo-a{%
      position:relative;%
      top:-.5ex;%
      margin-left:-.36em;%
      margin-right:-.1em;%
      font-size:85\%;%
    }%
  }%
  \Css{%
    span.HoLogo-LaTeXTeX span.HoLogo-ParenRight{%
      margin-right:-.15em;%
    }%
  }%
  \global\let\HoLogoCss@LaTeXTeX\relax
}
%    \end{macrocode}
%    \end{macro}
%
% \subsubsection{\hologo{LaTeXe}}
%
%    \begin{macro}{\HoLogo@LaTeXe}
%    Source: \hologo{LaTeX} kernel
%    \begin{macrocode}
\def\HoLogo@LaTeXe#1{%
  \hologo{LaTeX}%
  \kern.15em%
  \hbox{%
    \HOLOGO@MathSetup
    2%
    $_{\textstyle\varepsilon}$%
  }%
}
%    \end{macrocode}
%    \end{macro}
%
%    \begin{macro}{\HoLogoCs@LaTeXe}
%    \begin{macrocode}
\ifnum64=`\^^^^0040\relax % test for big chars of LuaTeX/XeTeX
  \catcode`\$=9 %
  \catcode`\&=14 %
\else
  \catcode`\$=14 %
  \catcode`\&=9 %
\fi
\def\HoLogoCs@LaTeXe#1{%
  LaTeX2%
$ \string ^^^^0395%
& e%
}%
\catcode`\$=3 %
\catcode`\&=4 %
%    \end{macrocode}
%    \end{macro}
%
%    \begin{macro}{\HoLogoBkm@LaTeXe}
%    \begin{macrocode}
\def\HoLogoBkm@LaTeXe#1{%
  \hologo{LaTeX}%
  2%
  \HOLOGO@PdfdocUnicode{e}{\textepsilon}%
}
%    \end{macrocode}
%    \end{macro}
%
%    \begin{macro}{\HoLogoHtml@LaTeXe}
%    \begin{macrocode}
\def\HoLogoHtml@LaTeXe#1{%
  \HoLogoCss@LaTeXe
  \HOLOGO@Span{LaTeX2e}{%
    \hologo{LaTeX}%
    \HOLOGO@Span{2}{2}%
    \HOLOGO@Span{e}{%
      \HOLOGO@MathSetup
      \ensuremath{\textstyle\varepsilon}%
    }%
  }%
}
%    \end{macrocode}
%    \end{macro}
%    \begin{macro}{\HoLogoCss@LaTeXe}
%    \begin{macrocode}
\def\HoLogoCss@LaTeXe{%
  \Css{%
    span.HoLogo-LaTeX2e span.HoLogo-2{%
      padding-left:.15em;%
    }%
  }%
  \Css{%
    span.HoLogo-LaTeX2e span.HoLogo-e{%
      position:relative;%
      top:.35ex;%
      text-decoration:none;%
    }%
  }%
  \global\let\HoLogoCss@LaTeXe\relax
}
%    \end{macrocode}
%    \end{macro}
%
%    \begin{macro}{\HoLogo@LaTeX2e}
%    \begin{macrocode}
\expandafter
\let\csname HoLogo@LaTeX2e\endcsname\HoLogo@LaTeXe
%    \end{macrocode}
%    \end{macro}
%    \begin{macro}{\HoLogoCs@LaTeX2e}
%    \begin{macrocode}
\expandafter
\let\csname HoLogoCs@LaTeX2e\endcsname\HoLogoCs@LaTeXe
%    \end{macrocode}
%    \end{macro}
%    \begin{macro}{\HoLogoBkm@LaTeX2e}
%    \begin{macrocode}
\expandafter
\let\csname HoLogoBkm@LaTeX2e\endcsname\HoLogoBkm@LaTeXe
%    \end{macrocode}
%    \end{macro}
%    \begin{macro}{\HoLogoHtml@LaTeX2e}
%    \begin{macrocode}
\expandafter
\let\csname HoLogoHtml@LaTeX2e\endcsname\HoLogoHtml@LaTeXe
%    \end{macrocode}
%    \end{macro}
%
% \subsubsection{\hologo{LaTeX3}}
%
%    \begin{macro}{\HoLogo@LaTeX3}
%    Source: \hologo{LaTeX} kernel
%    \begin{macrocode}
\expandafter\def\csname HoLogo@LaTeX3\endcsname#1{%
  \hologo{LaTeX}%
  3%
}
%    \end{macrocode}
%    \end{macro}
%
%    \begin{macro}{\HoLogoBkm@LaTeX3}
%    \begin{macrocode}
\expandafter\def\csname HoLogoBkm@LaTeX3\endcsname#1{%
  \hologo{LaTeX}%
  3%
}
%    \end{macrocode}
%    \end{macro}
%    \begin{macro}{\HoLogoHtml@LaTeX3}
%    \begin{macrocode}
\expandafter
\let\csname HoLogoHtml@LaTeX3\expandafter\endcsname
\csname HoLogo@LaTeX3\endcsname
%    \end{macrocode}
%    \end{macro}
%
% \subsubsection{\hologo{LaTeXML}}
%
%    \begin{macro}{\HoLogo@LaTeXML}
%    \begin{macrocode}
\def\HoLogo@LaTeXML#1{%
  \HOLOGO@mbox{%
    \hologo{La}%
    \kern-.15em%
    T%
    \kern-.1667em%
    \lower.5ex\hbox{E}%
    \kern-.125em%
    \HoLogoFont@font{LaTeXML}{sc}{xml}%
  }%
}
%    \end{macrocode}
%    \end{macro}
%    \begin{macro}{\HoLogoHtml@pdfLaTeX}
%    \begin{macrocode}
\def\HoLogoHtml@LaTeXML#1{%
  \HOLOGO@Span{LaTeXML}{%
    \HoLogoCss@LaTeX
    \HoLogoCss@TeX
    \HOLOGO@Span{LaTeX}{%
      L%
      \HOLOGO@Span{a}{%
        A%
      }%
    }%
    \HOLOGO@Span{TeX}{%
      T%
      \HOLOGO@Span{e}{%
        E%
      }%
    }%
    \HCode{<span style="font-variant: small-caps;">}%
    xml%
    \HCode{</span>}%
  }%
}
%    \end{macrocode}
%    \end{macro}
%
% \subsubsection{\hologo{eTeX}}
%
%    \begin{macro}{\HoLogo@eTeX}
%    Source: package \xpackage{etex}
%    \begin{macrocode}
\def\HoLogo@eTeX#1{%
  \ltx@mbox{%
    \HOLOGO@MathSetup
    $\varepsilon$%
    -%
    \HOLOGO@NegativeKerning{-T,T-,To}%
    \hologo{TeX}%
  }%
}
%    \end{macrocode}
%    \end{macro}
%    \begin{macro}{\HoLogoCs@eTeX}
%    \begin{macrocode}
\ifnum64=`\^^^^0040\relax % test for big chars of LuaTeX/XeTeX
  \catcode`\$=9 %
  \catcode`\&=14 %
\else
  \catcode`\$=14 %
  \catcode`\&=9 %
\fi
\def\HoLogoCs@eTeX#1{%
$ #1{\string ^^^^0395}{\string ^^^^03b5}%
& #1{e}{E}%
  TeX%
}%
\catcode`\$=3 %
\catcode`\&=4 %
%    \end{macrocode}
%    \end{macro}
%    \begin{macro}{\HoLogoBkm@eTeX}
%    \begin{macrocode}
\def\HoLogoBkm@eTeX#1{%
  \HOLOGO@PdfdocUnicode{#1{e}{E}}{\textepsilon}%
  -%
  \hologo{TeX}%
}
%    \end{macrocode}
%    \end{macro}
%    \begin{macro}{\HoLogoHtml@eTeX}
%    \begin{macrocode}
\def\HoLogoHtml@eTeX#1{%
  \ltx@mbox{%
    \HOLOGO@MathSetup
    $\varepsilon$%
    -%
    \hologo{TeX}%
  }%
}
%    \end{macrocode}
%    \end{macro}
%
% \subsubsection{\hologo{iniTeX}}
%
%    \begin{macro}{\HoLogo@iniTeX}
%    \begin{macrocode}
\def\HoLogo@iniTeX#1{%
  \HOLOGO@mbox{%
    #1{i}{I}ni\hologo{TeX}%
  }%
}
%    \end{macrocode}
%    \end{macro}
%    \begin{macro}{\HoLogoCs@iniTeX}
%    \begin{macrocode}
\def\HoLogoCs@iniTeX#1{#1{i}{I}niTeX}
%    \end{macrocode}
%    \end{macro}
%    \begin{macro}{\HoLogoBkm@iniTeX}
%    \begin{macrocode}
\def\HoLogoBkm@iniTeX#1{%
  #1{i}{I}ni\hologo{TeX}%
}
%    \end{macrocode}
%    \end{macro}
%    \begin{macro}{\HoLogoHtml@iniTeX}
%    \begin{macrocode}
\let\HoLogoHtml@iniTeX\HoLogo@iniTeX
%    \end{macrocode}
%    \end{macro}
%
% \subsubsection{\hologo{virTeX}}
%
%    \begin{macro}{\HoLogo@virTeX}
%    \begin{macrocode}
\def\HoLogo@virTeX#1{%
  \HOLOGO@mbox{%
    #1{v}{V}ir\hologo{TeX}%
  }%
}
%    \end{macrocode}
%    \end{macro}
%    \begin{macro}{\HoLogoCs@virTeX}
%    \begin{macrocode}
\def\HoLogoCs@virTeX#1{#1{v}{V}irTeX}
%    \end{macrocode}
%    \end{macro}
%    \begin{macro}{\HoLogoBkm@virTeX}
%    \begin{macrocode}
\def\HoLogoBkm@virTeX#1{%
  #1{v}{V}ir\hologo{TeX}%
}
%    \end{macrocode}
%    \end{macro}
%    \begin{macro}{\HoLogoHtml@virTeX}
%    \begin{macrocode}
\let\HoLogoHtml@virTeX\HoLogo@virTeX
%    \end{macrocode}
%    \end{macro}
%
% \subsubsection{\hologo{SliTeX}}
%
% \paragraph{Definitions of the three variants.}
%
%    \begin{macro}{\HoLogo@SLiTeX@lift}
%    \begin{macrocode}
\def\HoLogo@SLiTeX@lift#1{%
  \HoLogoFont@font{SliTeX}{rm}{%
    S%
    \kern-.06em%
    L%
    \kern-.18em%
    \raise.32ex\hbox{\HoLogoFont@font{SliTeX}{sc}{i}}%
    \HOLOGO@discretionary
    \kern-.06em%
    \hologo{TeX}%
  }%
}
%    \end{macrocode}
%    \end{macro}
%    \begin{macro}{\HoLogoBkm@SLiTeX@lift}
%    \begin{macrocode}
\def\HoLogoBkm@SLiTeX@lift#1{SLiTeX}
%    \end{macrocode}
%    \end{macro}
%    \begin{macro}{\HoLogoHtml@SLiTeX@lift}
%    \begin{macrocode}
\def\HoLogoHtml@SLiTeX@lift#1{%
  \HoLogoCss@SLiTeX@lift
  \HOLOGO@Span{SLiTeX-lift}{%
    \HoLogoFont@font{SliTeX}{rm}{%
      S%
      \HOLOGO@Span{L}{L}%
      \HOLOGO@Span{i}{i}%
      \hologo{TeX}%
    }%
  }%
}
%    \end{macrocode}
%    \end{macro}
%    \begin{macro}{\HoLogoCss@SLiTeX@lift}
%    \begin{macrocode}
\def\HoLogoCss@SLiTeX@lift{%
  \Css{%
    span.HoLogo-SLiTeX-lift span.HoLogo-L{%
      margin-left:-.06em;%
      margin-right:-.18em;%
    }%
  }%
  \Css{%
    span.HoLogo-SLiTeX-lift span.HoLogo-i{%
      position:relative;%
      top:-.32ex;%
      margin-right:-.06em;%
      font-variant:small-caps;%
    }%
  }%
  \global\let\HoLogoCss@SLiTeX@lift\relax
}
%    \end{macrocode}
%    \end{macro}
%
%    \begin{macro}{\HoLogo@SliTeX@simple}
%    \begin{macrocode}
\def\HoLogo@SliTeX@simple#1{%
  \HoLogoFont@font{SliTeX}{rm}{%
    \ltx@mbox{%
      \HoLogoFont@font{SliTeX}{sc}{Sli}%
    }%
    \HOLOGO@discretionary
    \hologo{TeX}%
  }%
}
%    \end{macrocode}
%    \end{macro}
%    \begin{macro}{\HoLogoBkm@SliTeX@simple}
%    \begin{macrocode}
\def\HoLogoBkm@SliTeX@simple#1{SliTeX}
%    \end{macrocode}
%    \end{macro}
%    \begin{macro}{\HoLogoHtml@SliTeX@simple}
%    \begin{macrocode}
\let\HoLogoHtml@SliTeX@simple\HoLogo@SliTeX@simple
%    \end{macrocode}
%    \end{macro}
%
%    \begin{macro}{\HoLogo@SliTeX@narrow}
%    \begin{macrocode}
\def\HoLogo@SliTeX@narrow#1{%
  \HoLogoFont@font{SliTeX}{rm}{%
    \ltx@mbox{%
      S%
      \kern-.06em%
      \HoLogoFont@font{SliTeX}{sc}{%
        l%
        \kern-.035em%
        i%
      }%
    }%
    \HOLOGO@discretionary
    \kern-.06em%
    \hologo{TeX}%
  }%
}
%    \end{macrocode}
%    \end{macro}
%    \begin{macro}{\HoLogoBkm@SliTeX@narrow}
%    \begin{macrocode}
\def\HoLogoBkm@SliTeX@narrow#1{SliTeX}
%    \end{macrocode}
%    \end{macro}
%    \begin{macro}{\HoLogoHtml@SliTeX@narrow}
%    \begin{macrocode}
\def\HoLogoHtml@SliTeX@narrow#1{%
  \HoLogoCss@SliTeX@narrow
  \HOLOGO@Span{SliTeX-narrow}{%
    \HoLogoFont@font{SliTeX}{rm}{%
      S%
        \HOLOGO@Span{l}{l}%
        \HOLOGO@Span{i}{i}%
      \hologo{TeX}%
    }%
  }%
}
%    \end{macrocode}
%    \end{macro}
%    \begin{macro}{\HoLogoCss@SliTeX@narrow}
%    \begin{macrocode}
\def\HoLogoCss@SliTeX@narrow{%
  \Css{%
    span.HoLogo-SliTeX-narrow span.HoLogo-l{%
      margin-left:-.06em;%
      margin-right:-.035em;%
      font-variant:small-caps;%
    }%
  }%
  \Css{%
    span.HoLogo-SliTeX-narrow span.HoLogo-i{%
      margin-right:-.06em;%
      font-variant:small-caps;%
    }%
  }%
  \global\let\HoLogoCss@SliTeX@narrow\relax
}
%    \end{macrocode}
%    \end{macro}
%
% \paragraph{Macro set completion.}
%
%    \begin{macro}{\HoLogo@SLiTeX@simple}
%    \begin{macrocode}
\def\HoLogo@SLiTeX@simple{\HoLogo@SliTeX@simple}
%    \end{macrocode}
%    \end{macro}
%    \begin{macro}{\HoLogoBkm@SLiTeX@simple}
%    \begin{macrocode}
\def\HoLogoBkm@SLiTeX@simple{\HoLogoBkm@SliTeX@simple}
%    \end{macrocode}
%    \end{macro}
%    \begin{macro}{\HoLogoHtml@SLiTeX@simple}
%    \begin{macrocode}
\def\HoLogoHtml@SLiTeX@simple{\HoLogoHtml@SliTeX@simple}
%    \end{macrocode}
%    \end{macro}
%
%    \begin{macro}{\HoLogo@SLiTeX@narrow}
%    \begin{macrocode}
\def\HoLogo@SLiTeX@narrow{\HoLogo@SliTeX@narrow}
%    \end{macrocode}
%    \end{macro}
%    \begin{macro}{\HoLogoBkm@SLiTeX@narrow}
%    \begin{macrocode}
\def\HoLogoBkm@SLiTeX@narrow{\HoLogoBkm@SliTeX@narrow}
%    \end{macrocode}
%    \end{macro}
%    \begin{macro}{\HoLogoHtml@SLiTeX@narrow}
%    \begin{macrocode}
\def\HoLogoHtml@SLiTeX@narrow{\HoLogoHtml@SliTeX@narrow}
%    \end{macrocode}
%    \end{macro}
%
%    \begin{macro}{\HoLogo@SliTeX@lift}
%    \begin{macrocode}
\def\HoLogo@SliTeX@lift{\HoLogo@SLiTeX@lift}
%    \end{macrocode}
%    \end{macro}
%    \begin{macro}{\HoLogoBkm@SliTeX@lift}
%    \begin{macrocode}
\def\HoLogoBkm@SliTeX@lift{\HoLogoBkm@SLiTeX@lift}
%    \end{macrocode}
%    \end{macro}
%    \begin{macro}{\HoLogoHtml@SliTeX@lift}
%    \begin{macrocode}
\def\HoLogoHtml@SliTeX@lift{\HoLogoHtml@SLiTeX@lift}
%    \end{macrocode}
%    \end{macro}
%
% \paragraph{Defaults.}
%
%    \begin{macro}{\HoLogo@SLiTeX}
%    \begin{macrocode}
\def\HoLogo@SLiTeX{\HoLogo@SLiTeX@lift}
%    \end{macrocode}
%    \end{macro}
%    \begin{macro}{\HoLogoBkm@SLiTeX}
%    \begin{macrocode}
\def\HoLogoBkm@SLiTeX{\HoLogoBkm@SLiTeX@lift}
%    \end{macrocode}
%    \end{macro}
%    \begin{macro}{\HoLogoHtml@SLiTeX}
%    \begin{macrocode}
\def\HoLogoHtml@SLiTeX{\HoLogoHtml@SLiTeX@lift}
%    \end{macrocode}
%    \end{macro}
%
%    \begin{macro}{\HoLogo@SliTeX}
%    \begin{macrocode}
\def\HoLogo@SliTeX{\HoLogo@SliTeX@narrow}
%    \end{macrocode}
%    \end{macro}
%    \begin{macro}{\HoLogoBkm@SliTeX}
%    \begin{macrocode}
\def\HoLogoBkm@SliTeX{\HoLogoBkm@SliTeX@narrow}
%    \end{macrocode}
%    \end{macro}
%    \begin{macro}{\HoLogoHtml@SliTeX}
%    \begin{macrocode}
\def\HoLogoHtml@SliTeX{\HoLogoHtml@SliTeX@narrow}
%    \end{macrocode}
%    \end{macro}
%
% \subsubsection{\hologo{LuaTeX}}
%
%    \begin{macro}{\HoLogo@LuaTeX}
%    The kerning is an idea of Hans Hagen, see mailing list
%    `luatex at tug dot org' in March 2010.
%    \begin{macrocode}
\def\HoLogo@LuaTeX#1{%
  \HOLOGO@mbox{%
    Lua%
    \HOLOGO@NegativeKerning{aT,oT,To}%
    \hologo{TeX}%
  }%
}
%    \end{macrocode}
%    \end{macro}
%    \begin{macro}{\HoLogoHtml@LuaTeX}
%    \begin{macrocode}
\let\HoLogoHtml@LuaTeX\HoLogo@LuaTeX
%    \end{macrocode}
%    \end{macro}
%
% \subsubsection{\hologo{LuaLaTeX}}
%
%    \begin{macro}{\HoLogo@LuaLaTeX}
%    \begin{macrocode}
\def\HoLogo@LuaLaTeX#1{%
  \HOLOGO@mbox{%
    Lua%
    \hologo{LaTeX}%
  }%
}
%    \end{macrocode}
%    \end{macro}
%    \begin{macro}{\HoLogoHtml@LuaLaTeX}
%    \begin{macrocode}
\let\HoLogoHtml@LuaLaTeX\HoLogo@LuaLaTeX
%    \end{macrocode}
%    \end{macro}
%
% \subsubsection{\hologo{XeTeX}, \hologo{XeLaTeX}}
%
%    \begin{macro}{\HOLOGO@IfCharExists}
%    \begin{macrocode}
\ifluatex
  \ifnum\luatexversion<36 %
  \else
    \def\HOLOGO@IfCharExists#1{%
      \ifnum
        \directlua{%
           if luaotfload and luaotfload.aux then
             if luaotfload.aux.font_has_glyph(%
                    font.current(), \number#1) then % 	 
	       tex.print("1") % 	 
	     end % 	 
	   elseif font and font.fonts and font.current then %
            local f = font.fonts[font.current()]%
            if f.characters and f.characters[\number#1] then %
              tex.print("1")%
            end %
          end%
        }0=\ltx@zero
        \expandafter\ltx@secondoftwo
      \else
        \expandafter\ltx@firstoftwo
      \fi
    }%
  \fi
\fi
\ltx@IfUndefined{HOLOGO@IfCharExists}{%
  \def\HOLOGO@@IfCharExists#1{%
    \begingroup
      \tracinglostchars=\ltx@zero
      \setbox\ltx@zero=\hbox{%
        \kern7sp\char#1\relax
        \ifnum\lastkern>\ltx@zero
          \expandafter\aftergroup\csname iffalse\endcsname
        \else
          \expandafter\aftergroup\csname iftrue\endcsname
        \fi
      }%
      % \if{true|false} from \aftergroup
      \endgroup
      \expandafter\ltx@firstoftwo
    \else
      \endgroup
      \expandafter\ltx@secondoftwo
    \fi
  }%
  \ifxetex
    \ltx@IfUndefined{XeTeXfonttype}{}{%
      \ltx@IfUndefined{XeTeXcharglyph}{}{%
        \def\HOLOGO@IfCharExists#1{%
          \ifnum\XeTeXfonttype\font>\ltx@zero
            \expandafter\ltx@firstofthree
          \else
            \expandafter\ltx@gobble
          \fi
          {%
            \ifnum\XeTeXcharglyph#1>\ltx@zero
              \expandafter\ltx@firstoftwo
            \else
              \expandafter\ltx@secondoftwo
            \fi
          }%
          \HOLOGO@@IfCharExists{#1}%
        }%
      }%
    }%
  \fi
}{}
\ltx@ifundefined{HOLOGO@IfCharExists}{%
  \ifnum64=`\^^^^0040\relax % test for big chars of LuaTeX/XeTeX
    \let\HOLOGO@IfCharExists\HOLOGO@@IfCharExists
  \else
    \def\HOLOGO@IfCharExists#1{%
      \ifnum#1>255 %
        \expandafter\ltx@fourthoffour
      \fi
      \HOLOGO@@IfCharExists{#1}%
    }%
  \fi
}{}
%    \end{macrocode}
%    \end{macro}
%
%    \begin{macro}{\HoLogo@Xe}
%    Source: package \xpackage{dtklogos}
%    \begin{macrocode}
\def\HoLogo@Xe#1{%
  X%
  \kern-.1em\relax
  \HOLOGO@IfCharExists{"018E}{%
    \lower.5ex\hbox{\char"018E}%
  }{%
    \chardef\HOLOGO@choice=\ltx@zero
    \ifdim\fontdimen\ltx@one\font>0pt %
      \ltx@IfUndefined{rotatebox}{%
        \ltx@IfUndefined{pgftext}{%
          \ltx@IfUndefined{psscalebox}{%
            \ltx@IfUndefined{HOLOGO@ScaleBox@\hologoDriver}{%
            }{%
              \chardef\HOLOGO@choice=4 %
            }%
          }{%
            \chardef\HOLOGO@choice=3 %
          }%
        }{%
          \chardef\HOLOGO@choice=2 %
        }%
      }{%
        \chardef\HOLOGO@choice=1 %
      }%
      \ifcase\HOLOGO@choice
        \HOLOGO@WarningUnsupportedDriver{Xe}%
        e%
      \or % 1: \rotatebox
        \begingroup
          \setbox\ltx@zero\hbox{\rotatebox{180}{E}}%
          \ltx@LocDimenA=\dp\ltx@zero
          \advance\ltx@LocDimenA by -.5ex\relax
          \raise\ltx@LocDimenA\box\ltx@zero
        \endgroup
      \or % 2: \pgftext
        \lower.5ex\hbox{%
          \pgfpicture
            \pgftext[rotate=180]{E}%
          \endpgfpicture
        }%
      \or % 3: \psscalebox
        \begingroup
          \setbox\ltx@zero\hbox{\psscalebox{-1 -1}{E}}%
          \ltx@LocDimenA=\dp\ltx@zero
          \advance\ltx@LocDimenA by -.5ex\relax
          \raise\ltx@LocDimenA\box\ltx@zero
        \endgroup
      \or % 4: \HOLOGO@PointReflectBox
        \lower.5ex\hbox{\HOLOGO@PointReflectBox{E}}%
      \else
        \@PackageError{hologo}{Internal error (choice/it}\@ehc
      \fi
    \else
      \ltx@IfUndefined{reflectbox}{%
        \ltx@IfUndefined{pgftext}{%
          \ltx@IfUndefined{psscalebox}{%
            \ltx@IfUndefined{HOLOGO@ScaleBox@\hologoDriver}{%
            }{%
              \chardef\HOLOGO@choice=4 %
            }%
          }{%
            \chardef\HOLOGO@choice=3 %
          }%
        }{%
          \chardef\HOLOGO@choice=2 %
        }%
      }{%
        \chardef\HOLOGO@choice=1 %
      }%
      \ifcase\HOLOGO@choice
        \HOLOGO@WarningUnsupportedDriver{Xe}%
        e%
      \or % 1: reflectbox
        \lower.5ex\hbox{%
          \reflectbox{E}%
        }%
      \or % 2: \pgftext
        \lower.5ex\hbox{%
          \pgfpicture
            \pgftransformxscale{-1}%
            \pgftext{E}%
          \endpgfpicture
        }%
      \or % 3: \psscalebox
        \lower.5ex\hbox{%
          \psscalebox{-1 1}{E}%
        }%
      \or % 4: \HOLOGO@Reflectbox
        \lower.5ex\hbox{%
          \HOLOGO@ReflectBox{E}%
        }%
      \else
        \@PackageError{hologo}{Internal error (choice/up)}\@ehc
      \fi
    \fi
  }%
}
%    \end{macrocode}
%    \end{macro}
%    \begin{macro}{\HoLogoHtml@Xe}
%    \begin{macrocode}
\def\HoLogoHtml@Xe#1{%
  \HoLogoCss@Xe
  \HOLOGO@Span{Xe}{%
    X%
    \HOLOGO@Span{e}{%
      \HCode{&\ltx@hashchar x018e;}%
    }%
  }%
}
%    \end{macrocode}
%    \end{macro}
%    \begin{macro}{\HoLogoCss@Xe}
%    \begin{macrocode}
\def\HoLogoCss@Xe{%
  \Css{%
    span.HoLogo-Xe span.HoLogo-e{%
      position:relative;%
      top:.5ex;%
      left-margin:-.1em;%
    }%
  }%
  \global\let\HoLogoCss@Xe\relax
}
%    \end{macrocode}
%    \end{macro}
%
%    \begin{macro}{\HoLogo@XeTeX}
%    \begin{macrocode}
\def\HoLogo@XeTeX#1{%
  \hologo{Xe}%
  \kern-.15em\relax
  \hologo{TeX}%
}
%    \end{macrocode}
%    \end{macro}
%
%    \begin{macro}{\HoLogoHtml@XeTeX}
%    \begin{macrocode}
\def\HoLogoHtml@XeTeX#1{%
  \HoLogoCss@XeTeX
  \HOLOGO@Span{XeTeX}{%
    \hologo{Xe}%
    \hologo{TeX}%
  }%
}
%    \end{macrocode}
%    \end{macro}
%    \begin{macro}{\HoLogoCss@XeTeX}
%    \begin{macrocode}
\def\HoLogoCss@XeTeX{%
  \Css{%
    span.HoLogo-XeTeX span.HoLogo-TeX{%
      margin-left:-.15em;%
    }%
  }%
  \global\let\HoLogoCss@XeTeX\relax
}
%    \end{macrocode}
%    \end{macro}
%
%    \begin{macro}{\HoLogo@XeLaTeX}
%    \begin{macrocode}
\def\HoLogo@XeLaTeX#1{%
  \hologo{Xe}%
  \kern-.13em%
  \hologo{LaTeX}%
}
%    \end{macrocode}
%    \end{macro}
%    \begin{macro}{\HoLogoHtml@XeLaTeX}
%    \begin{macrocode}
\def\HoLogoHtml@XeLaTeX#1{%
  \HoLogoCss@XeLaTeX
  \HOLOGO@Span{XeLaTeX}{%
    \hologo{Xe}%
    \hologo{LaTeX}%
  }%
}
%    \end{macrocode}
%    \end{macro}
%    \begin{macro}{\HoLogoCss@XeLaTeX}
%    \begin{macrocode}
\def\HoLogoCss@XeLaTeX{%
  \Css{%
    span.HoLogo-XeLaTeX span.HoLogo-Xe{%
      margin-right:-.13em;%
    }%
  }%
  \global\let\HoLogoCss@XeLaTeX\relax
}
%    \end{macrocode}
%    \end{macro}
%
% \subsubsection{\hologo{pdfTeX}, \hologo{pdfLaTeX}}
%
%    \begin{macro}{\HoLogo@pdfTeX}
%    \begin{macrocode}
\def\HoLogo@pdfTeX#1{%
  \HOLOGO@mbox{%
    #1{p}{P}df\hologo{TeX}%
  }%
}
%    \end{macrocode}
%    \end{macro}
%    \begin{macro}{\HoLogoCs@pdfTeX}
%    \begin{macrocode}
\def\HoLogoCs@pdfTeX#1{#1{p}{P}dfTeX}
%    \end{macrocode}
%    \end{macro}
%    \begin{macro}{\HoLogoBkm@pdfTeX}
%    \begin{macrocode}
\def\HoLogoBkm@pdfTeX#1{%
  #1{p}{P}df\hologo{TeX}%
}
%    \end{macrocode}
%    \end{macro}
%    \begin{macro}{\HoLogoHtml@pdfTeX}
%    \begin{macrocode}
\let\HoLogoHtml@pdfTeX\HoLogo@pdfTeX
%    \end{macrocode}
%    \end{macro}
%
%    \begin{macro}{\HoLogo@pdfLaTeX}
%    \begin{macrocode}
\def\HoLogo@pdfLaTeX#1{%
  \HOLOGO@mbox{%
    #1{p}{P}df\hologo{LaTeX}%
  }%
}
%    \end{macrocode}
%    \end{macro}
%    \begin{macro}{\HoLogoCs@pdfLaTeX}
%    \begin{macrocode}
\def\HoLogoCs@pdfLaTeX#1{#1{p}{P}dfLaTeX}
%    \end{macrocode}
%    \end{macro}
%    \begin{macro}{\HoLogoBkm@pdfLaTeX}
%    \begin{macrocode}
\def\HoLogoBkm@pdfLaTeX#1{%
  #1{p}{P}df\hologo{LaTeX}%
}
%    \end{macrocode}
%    \end{macro}
%    \begin{macro}{\HoLogoHtml@pdfLaTeX}
%    \begin{macrocode}
\let\HoLogoHtml@pdfLaTeX\HoLogo@pdfLaTeX
%    \end{macrocode}
%    \end{macro}
%
% \subsubsection{\hologo{VTeX}}
%
%    \begin{macro}{\HoLogo@VTeX}
%    \begin{macrocode}
\def\HoLogo@VTeX#1{%
  \HOLOGO@mbox{%
    V\hologo{TeX}%
  }%
}
%    \end{macrocode}
%    \end{macro}
%    \begin{macro}{\HoLogoHtml@VTeX}
%    \begin{macrocode}
\let\HoLogoHtml@VTeX\HoLogo@VTeX
%    \end{macrocode}
%    \end{macro}
%
% \subsubsection{\hologo{AmS}, \dots}
%
%    Source: class \xclass{amsdtx}
%
%    \begin{macro}{\HoLogo@AmS}
%    \begin{macrocode}
\def\HoLogo@AmS#1{%
  \HoLogoFont@font{AmS}{sy}{%
    A%
    \kern-.1667em%
    \lower.5ex\hbox{M}%
    \kern-.125em%
    S%
  }%
}
%    \end{macrocode}
%    \end{macro}
%    \begin{macro}{\HoLogoBkm@AmS}
%    \begin{macrocode}
\def\HoLogoBkm@AmS#1{AmS}
%    \end{macrocode}
%    \end{macro}
%    \begin{macro}{\HoLogoHtml@AmS}
%    \begin{macrocode}
\def\HoLogoHtml@AmS#1{%
  \HoLogoCss@AmS
%  \HoLogoFont@font{AmS}{sy}{%
    \HOLOGO@Span{AmS}{%
      A%
      \HOLOGO@Span{M}{M}%
      S%
    }%
%   }%
}
%    \end{macrocode}
%    \end{macro}
%    \begin{macro}{\HoLogoCss@AmS}
%    \begin{macrocode}
\def\HoLogoCss@AmS{%
  \Css{%
    span.HoLogo-AmS span.HoLogo-M{%
      position:relative;%
      top:.5ex;%
      margin-left:-.1667em;%
      margin-right:-.125em;%
      text-decoration:none;%
    }%
  }%
  \global\let\HoLogoCss@AmS\relax
}
%    \end{macrocode}
%    \end{macro}
%
%    \begin{macro}{\HoLogo@AmSTeX}
%    \begin{macrocode}
\def\HoLogo@AmSTeX#1{%
  \hologo{AmS}%
  \HOLOGO@hyphen
  \hologo{TeX}%
}
%    \end{macrocode}
%    \end{macro}
%    \begin{macro}{\HoLogoBkm@AmSTeX}
%    \begin{macrocode}
\def\HoLogoBkm@AmSTeX#1{AmS-TeX}%
%    \end{macrocode}
%    \end{macro}
%    \begin{macro}{\HoLogoHtml@AmSTeX}
%    \begin{macrocode}
\let\HoLogoHtml@AmSTeX\HoLogo@AmSTeX
%    \end{macrocode}
%    \end{macro}
%
%    \begin{macro}{\HoLogo@AmSLaTeX}
%    \begin{macrocode}
\def\HoLogo@AmSLaTeX#1{%
  \hologo{AmS}%
  \HOLOGO@hyphen
  \hologo{LaTeX}%
}
%    \end{macrocode}
%    \end{macro}
%    \begin{macro}{\HoLogoBkm@AmSLaTeX}
%    \begin{macrocode}
\def\HoLogoBkm@AmSLaTeX#1{AmS-LaTeX}%
%    \end{macrocode}
%    \end{macro}
%    \begin{macro}{\HoLogoHtml@AmSLaTeX}
%    \begin{macrocode}
\let\HoLogoHtml@AmSLaTeX\HoLogo@AmSLaTeX
%    \end{macrocode}
%    \end{macro}
%
% \subsubsection{\hologo{BibTeX}}
%
%    \begin{macro}{\HoLogo@BibTeX@sc}
%    A definition of \hologo{BibTeX} is provided in
%    the documentation source for the manual of \hologo{BibTeX}
%    \cite{btxdoc}.
%\begin{quote}
%\begin{verbatim}
%\def\BibTeX{%
%  {%
%    \rm
%    B%
%    \kern-.05em%
%    {%
%      \sc
%      i%
%      \kern-.025em %
%      b%
%    }%
%    \kern-.08em
%    T%
%    \kern-.1667em%
%    \lower.7ex\hbox{E}%
%    \kern-.125em%
%    X%
%  }%
%}
%\end{verbatim}
%\end{quote}
%    \begin{macrocode}
\def\HoLogo@BibTeX@sc#1{%
  B%
  \kern-.05em%
  \HoLogoFont@font{BibTeX}{sc}{%
    i%
    \kern-.025em%
    b%
  }%
  \HOLOGO@discretionary
  \kern-.08em%
  \hologo{TeX}%
}
%    \end{macrocode}
%    \end{macro}
%    \begin{macro}{\HoLogoHtml@BibTeX@sc}
%    \begin{macrocode}
\def\HoLogoHtml@BibTeX@sc#1{%
  \HoLogoCss@BibTeX@sc
  \HOLOGO@Span{BibTeX-sc}{%
    B%
    \HOLOGO@Span{i}{i}%
    \HOLOGO@Span{b}{b}%
    \hologo{TeX}%
  }%
}
%    \end{macrocode}
%    \end{macro}
%    \begin{macro}{\HoLogoCss@BibTeX@sc}
%    \begin{macrocode}
\def\HoLogoCss@BibTeX@sc{%
  \Css{%
    span.HoLogo-BibTeX-sc span.HoLogo-i{%
      margin-left:-.05em;%
      margin-right:-.025em;%
      font-variant:small-caps;%
    }%
  }%
  \Css{%
    span.HoLogo-BibTeX-sc span.HoLogo-b{%
      margin-right:-.08em;%
      font-variant:small-caps;%
    }%
  }%
  \global\let\HoLogoCss@BibTeX@sc\relax
}
%    \end{macrocode}
%    \end{macro}
%
%    \begin{macro}{\HoLogo@BibTeX@sf}
%    Variant \xoption{sf} avoids trouble with unavailable
%    small caps fonts (e.g., bold versions of Computer Modern or
%    Latin Modern). The definition is taken from
%    package \xpackage{dtklogos} \cite{dtklogos}.
%\begin{quote}
%\begin{verbatim}
%\DeclareRobustCommand{\BibTeX}{%
%  B%
%  \kern-.05em%
%  \hbox{%
%    $\m@th$% %% force math size calculations
%    \csname S@\f@size\endcsname
%    \fontsize\sf@size\z@
%    \math@fontsfalse
%    \selectfont
%    I%
%    \kern-.025em%
%    B
%  }%
%  \kern-.08em%
%  \-%
%  \TeX
%}
%\end{verbatim}
%\end{quote}
%    \begin{macrocode}
\def\HoLogo@BibTeX@sf#1{%
  B%
  \kern-.05em%
  \HoLogoFont@font{BibTeX}{bibsf}{%
    I%
    \kern-.025em%
    B%
  }%
  \HOLOGO@discretionary
  \kern-.08em%
  \hologo{TeX}%
}
%    \end{macrocode}
%    \end{macro}
%    \begin{macro}{\HoLogoHtml@BibTeX@sf}
%    \begin{macrocode}
\def\HoLogoHtml@BibTeX@sf#1{%
  \HoLogoCss@BibTeX@sf
  \HOLOGO@Span{BibTeX-sf}{%
    B%
    \HoLogoFont@font{BibTeX}{bibsf}{%
      \HOLOGO@Span{i}{I}%
      B%
    }%
    \hologo{TeX}%
  }%
}
%    \end{macrocode}
%    \end{macro}
%    \begin{macro}{\HoLogoCss@BibTeX@sf}
%    \begin{macrocode}
\def\HoLogoCss@BibTeX@sf{%
  \Css{%
    span.HoLogo-BibTeX-sf span.HoLogo-i{%
      margin-left:-.05em;%
      margin-right:-.025em;%
    }%
  }%
  \Css{%
    span.HoLogo-BibTeX-sf span.HoLogo-TeX{%
      margin-left:-.08em;%
    }%
  }%
  \global\let\HoLogoCss@BibTeX@sf\relax
}
%    \end{macrocode}
%    \end{macro}
%
%    \begin{macro}{\HoLogo@BibTeX}
%    \begin{macrocode}
\def\HoLogo@BibTeX{\HoLogo@BibTeX@sf}
%    \end{macrocode}
%    \end{macro}
%    \begin{macro}{\HoLogoHtml@BibTeX}
%    \begin{macrocode}
\def\HoLogoHtml@BibTeX{\HoLogoHtml@BibTeX@sf}
%    \end{macrocode}
%    \end{macro}
%
% \subsubsection{\hologo{BibTeX8}}
%
%    \begin{macro}{\HoLogo@BibTeX8}
%    \begin{macrocode}
\expandafter\def\csname HoLogo@BibTeX8\endcsname#1{%
  \hologo{BibTeX}%
  8%
}
%    \end{macrocode}
%    \end{macro}
%
%    \begin{macro}{\HoLogoBkm@BibTeX8}
%    \begin{macrocode}
\expandafter\def\csname HoLogoBkm@BibTeX8\endcsname#1{%
  \hologo{BibTeX}%
  8%
}
%    \end{macrocode}
%    \end{macro}
%    \begin{macro}{\HoLogoHtml@BibTeX8}
%    \begin{macrocode}
\expandafter
\let\csname HoLogoHtml@BibTeX8\expandafter\endcsname
\csname HoLogo@BibTeX8\endcsname
%    \end{macrocode}
%    \end{macro}
%
% \subsubsection{\hologo{ConTeXt}}
%
%    \begin{macro}{\HoLogo@ConTeXt@simple}
%    \begin{macrocode}
\def\HoLogo@ConTeXt@simple#1{%
  \HOLOGO@mbox{Con}%
  \HOLOGO@discretionary
  \HOLOGO@mbox{\hologo{TeX}t}%
}
%    \end{macrocode}
%    \end{macro}
%    \begin{macro}{\HoLogoHtml@ConTeXt@simple}
%    \begin{macrocode}
\let\HoLogoHtml@ConTeXt@simple\HoLogo@ConTeXt@simple
%    \end{macrocode}
%    \end{macro}
%
%    \begin{macro}{\HoLogo@ConTeXt@narrow}
%    This definition of logo \hologo{ConTeXt} with variant \xoption{narrow}
%    comes from TUGboat's class \xclass{ltugboat} (version 2010/11/15 v2.8).
%    \begin{macrocode}
\def\HoLogo@ConTeXt@narrow#1{%
  \HOLOGO@mbox{C\kern-.0333emon}%
  \HOLOGO@discretionary
  \kern-.0667em%
  \HOLOGO@mbox{\hologo{TeX}\kern-.0333emt}%
}
%    \end{macrocode}
%    \end{macro}
%    \begin{macro}{\HoLogoHtml@ConTeXt@narrow}
%    \begin{macrocode}
\def\HoLogoHtml@ConTeXt@narrow#1{%
  \HoLogoCss@ConTeXt@narrow
  \HOLOGO@Span{ConTeXt-narrow}{%
    \HOLOGO@Span{C}{C}%
    on%
    \hologo{TeX}%
    t%
  }%
}
%    \end{macrocode}
%    \end{macro}
%    \begin{macro}{\HoLogoCss@ConTeXt@narrow}
%    \begin{macrocode}
\def\HoLogoCss@ConTeXt@narrow{%
  \Css{%
    span.HoLogo-ConTeXt-narrow span.HoLogo-C{%
      margin-left:-.0333em;%
    }%
  }%
  \Css{%
    span.HoLogo-ConTeXt-narrow span.HoLogo-TeX{%
      margin-left:-.0667em;%
      margin-right:-.0333em;%
    }%
  }%
  \global\let\HoLogoCss@ConTeXt@narrow\relax
}
%    \end{macrocode}
%    \end{macro}
%
%    \begin{macro}{\HoLogo@ConTeXt}
%    \begin{macrocode}
\def\HoLogo@ConTeXt{\HoLogo@ConTeXt@narrow}
%    \end{macrocode}
%    \end{macro}
%    \begin{macro}{\HoLogoHtml@ConTeXt}
%    \begin{macrocode}
\def\HoLogoHtml@ConTeXt{\HoLogoHtml@ConTeXt@narrow}
%    \end{macrocode}
%    \end{macro}
%
% \subsubsection{\hologo{emTeX}}
%
%    \begin{macro}{\HoLogo@emTeX}
%    \begin{macrocode}
\def\HoLogo@emTeX#1{%
  \HOLOGO@mbox{#1{e}{E}m}%
  \HOLOGO@discretionary
  \hologo{TeX}%
}
%    \end{macrocode}
%    \end{macro}
%    \begin{macro}{\HoLogoCs@emTeX}
%    \begin{macrocode}
\def\HoLogoCs@emTeX#1{#1{e}{E}mTeX}%
%    \end{macrocode}
%    \end{macro}
%    \begin{macro}{\HoLogoBkm@emTeX}
%    \begin{macrocode}
\def\HoLogoBkm@emTeX#1{%
  #1{e}{E}m\hologo{TeX}%
}
%    \end{macrocode}
%    \end{macro}
%    \begin{macro}{\HoLogoHtml@emTeX}
%    \begin{macrocode}
\let\HoLogoHtml@emTeX\HoLogo@emTeX
%    \end{macrocode}
%    \end{macro}
%
% \subsubsection{\hologo{ExTeX}}
%
%    \begin{macro}{\HoLogo@ExTeX}
%    The definition is taken from the FAQ of the
%    project \hologo{ExTeX}
%    \cite{ExTeX-FAQ}.
%\begin{quote}
%\begin{verbatim}
%\def\ExTeX{%
%  \textrm{% Logo always with serifs
%    \ensuremath{%
%      \textstyle
%      \varepsilon_{%
%        \kern-0.15em%
%        \mathcal{X}%
%      }%
%    }%
%    \kern-.15em%
%    \TeX
%  }%
%}
%\end{verbatim}
%\end{quote}
%    \begin{macrocode}
\def\HoLogo@ExTeX#1{%
  \HoLogoFont@font{ExTeX}{rm}{%
    \ltx@mbox{%
      \HOLOGO@MathSetup
      $%
        \textstyle
        \varepsilon_{%
          \kern-0.15em%
          \HoLogoFont@font{ExTeX}{sy}{X}%
        }%
      $%
    }%
    \HOLOGO@discretionary
    \kern-.15em%
    \hologo{TeX}%
  }%
}
%    \end{macrocode}
%    \end{macro}
%    \begin{macro}{\HoLogoHtml@ExTeX}
%    \begin{macrocode}
\def\HoLogoHtml@ExTeX#1{%
  \HoLogoCss@ExTeX
  \HoLogoFont@font{ExTeX}{rm}{%
    \HOLOGO@Span{ExTeX}{%
      \ltx@mbox{%
        \HOLOGO@MathSetup
        $\textstyle\varepsilon$%
        \HOLOGO@Span{X}{$\textstyle\chi$}%
        \hologo{TeX}%
      }%
    }%
  }%
}
%    \end{macrocode}
%    \end{macro}
%    \begin{macro}{\HoLogoBkm@ExTeX}
%    \begin{macrocode}
\def\HoLogoBkm@ExTeX#1{%
  \HOLOGO@PdfdocUnicode{#1{e}{E}x}{\textepsilon\textchi}%
  \hologo{TeX}%
}
%    \end{macrocode}
%    \end{macro}
%    \begin{macro}{\HoLogoCss@ExTeX}
%    \begin{macrocode}
\def\HoLogoCss@ExTeX{%
  \Css{%
    span.HoLogo-ExTeX{%
      font-family:serif;%
    }%
  }%
  \Css{%
    span.HoLogo-ExTeX span.HoLogo-TeX{%
      margin-left:-.15em;%
    }%
  }%
  \global\let\HoLogoCss@ExTeX\relax
}
%    \end{macrocode}
%    \end{macro}
%
% \subsubsection{\hologo{MiKTeX}}
%
%    \begin{macro}{\HoLogo@MiKTeX}
%    \begin{macrocode}
\def\HoLogo@MiKTeX#1{%
  \HOLOGO@mbox{MiK}%
  \HOLOGO@discretionary
  \hologo{TeX}%
}
%    \end{macrocode}
%    \end{macro}
%    \begin{macro}{\HoLogoHtml@MiKTeX}
%    \begin{macrocode}
\let\HoLogoHtml@MiKTeX\HoLogo@MiKTeX
%    \end{macrocode}
%    \end{macro}
%
% \subsubsection{\hologo{OzTeX} and friends}
%
%    Source: \hologo{OzTeX} FAQ \cite{OzTeX}:
%    \begin{quote}
%      |\def\OzTeX{O\kern-.03em z\kern-.15em\TeX}|\\
%      (There is no kerning in OzMF, OzMP and OzTtH.)
%    \end{quote}
%
%    \begin{macro}{\HoLogo@OzTeX}
%    \begin{macrocode}
\def\HoLogo@OzTeX#1{%
  O%
  \kern-.03em %
  z%
  \kern-.15em %
  \hologo{TeX}%
}
%    \end{macrocode}
%    \end{macro}
%    \begin{macro}{\HoLogoHtml@OzTeX}
%    \begin{macrocode}
\def\HoLogoHtml@OzTeX#1{%
  \HoLogoCss@OzTeX
  \HOLOGO@Span{OzTeX}{%
    O%
    \HOLOGO@Span{z}{z}%
    \hologo{TeX}%
  }%
}
%    \end{macrocode}
%    \end{macro}
%    \begin{macro}{\HoLogoCss@OzTeX}
%    \begin{macrocode}
\def\HoLogoCss@OzTeX{%
  \Css{%
    span.HoLogo-OzTeX span.HoLogo-z{%
      margin-left:-.03em;%
      margin-right:-.15em;%
    }%
  }%
  \global\let\HoLogoCss@OzTeX\relax
}
%    \end{macrocode}
%    \end{macro}
%
%    \begin{macro}{\HoLogo@OzMF}
%    \begin{macrocode}
\def\HoLogo@OzMF#1{%
  \HOLOGO@mbox{OzMF}%
}
%    \end{macrocode}
%    \end{macro}
%    \begin{macro}{\HoLogo@OzMP}
%    \begin{macrocode}
\def\HoLogo@OzMP#1{%
  \HOLOGO@mbox{OzMP}%
}
%    \end{macrocode}
%    \end{macro}
%    \begin{macro}{\HoLogo@OzTtH}
%    \begin{macrocode}
\def\HoLogo@OzTtH#1{%
  \HOLOGO@mbox{OzTtH}%
}
%    \end{macrocode}
%    \end{macro}
%
% \subsubsection{\hologo{PCTeX}}
%
%    \begin{macro}{\HoLogo@PCTeX}
%    \begin{macrocode}
\def\HoLogo@PCTeX#1{%
  \HOLOGO@mbox{PC}%
  \hologo{TeX}%
}
%    \end{macrocode}
%    \end{macro}
%    \begin{macro}{\HoLogoHtml@PCTeX}
%    \begin{macrocode}
\let\HoLogoHtml@PCTeX\HoLogo@PCTeX
%    \end{macrocode}
%    \end{macro}
%
% \subsubsection{\hologo{PiCTeX}}
%
%    The original definitions from \xfile{pictex.tex} \cite{PiCTeX}:
%\begin{quote}
%\begin{verbatim}
%\def\PiC{%
%  P%
%  \kern-.12em%
%  \lower.5ex\hbox{I}%
%  \kern-.075em%
%  C%
%}
%\def\PiCTeX{%
%  \PiC
%  \kern-.11em%
%  \TeX
%}
%\end{verbatim}
%\end{quote}
%
%    \begin{macro}{\HoLogo@PiC}
%    \begin{macrocode}
\def\HoLogo@PiC#1{%
  P%
  \kern-.12em%
  \lower.5ex\hbox{I}%
  \kern-.075em%
  C%
  \HOLOGO@SpaceFactor
}
%    \end{macrocode}
%    \end{macro}
%    \begin{macro}{\HoLogoHtml@PiC}
%    \begin{macrocode}
\def\HoLogoHtml@PiC#1{%
  \HoLogoCss@PiC
  \HOLOGO@Span{PiC}{%
    P%
    \HOLOGO@Span{i}{I}%
    C%
  }%
}
%    \end{macrocode}
%    \end{macro}
%    \begin{macro}{\HoLogoCss@PiC}
%    \begin{macrocode}
\def\HoLogoCss@PiC{%
  \Css{%
    span.HoLogo-PiC span.HoLogo-i{%
      position:relative;%
      top:.5ex;%
      margin-left:-.12em;%
      margin-right:-.075em;%
      text-decoration:none;%
    }%
  }%
  \global\let\HoLogoCss@PiC\relax
}
%    \end{macrocode}
%    \end{macro}
%
%    \begin{macro}{\HoLogo@PiCTeX}
%    \begin{macrocode}
\def\HoLogo@PiCTeX#1{%
  \hologo{PiC}%
  \HOLOGO@discretionary
  \kern-.11em%
  \hologo{TeX}%
}
%    \end{macrocode}
%    \end{macro}
%    \begin{macro}{\HoLogoHtml@PiCTeX}
%    \begin{macrocode}
\def\HoLogoHtml@PiCTeX#1{%
  \HoLogoCss@PiCTeX
  \HOLOGO@Span{PiCTeX}{%
    \hologo{PiC}%
    \hologo{TeX}%
  }%
}
%    \end{macrocode}
%    \end{macro}
%    \begin{macro}{\HoLogoCss@PiCTeX}
%    \begin{macrocode}
\def\HoLogoCss@PiCTeX{%
  \Css{%
    span.HoLogo-PiCTeX span.HoLogo-PiC{%
      margin-right:-.11em;%
    }%
  }%
  \global\let\HoLogoCss@PiCTeX\relax
}
%    \end{macrocode}
%    \end{macro}
%
% \subsubsection{\hologo{teTeX}}
%
%    \begin{macro}{\HoLogo@teTeX}
%    \begin{macrocode}
\def\HoLogo@teTeX#1{%
  \HOLOGO@mbox{#1{t}{T}e}%
  \HOLOGO@discretionary
  \hologo{TeX}%
}
%    \end{macrocode}
%    \end{macro}
%    \begin{macro}{\HoLogoCs@teTeX}
%    \begin{macrocode}
\def\HoLogoCs@teTeX#1{#1{t}{T}dfTeX}
%    \end{macrocode}
%    \end{macro}
%    \begin{macro}{\HoLogoBkm@teTeX}
%    \begin{macrocode}
\def\HoLogoBkm@teTeX#1{%
  #1{t}{T}e\hologo{TeX}%
}
%    \end{macrocode}
%    \end{macro}
%    \begin{macro}{\HoLogoHtml@teTeX}
%    \begin{macrocode}
\let\HoLogoHtml@teTeX\HoLogo@teTeX
%    \end{macrocode}
%    \end{macro}
%
% \subsubsection{\hologo{TeX4ht}}
%
%    \begin{macro}{\HoLogo@TeX4ht}
%    \begin{macrocode}
\expandafter\def\csname HoLogo@TeX4ht\endcsname#1{%
  \HOLOGO@mbox{\hologo{TeX}4ht}%
}
%    \end{macrocode}
%    \end{macro}
%    \begin{macro}{\HoLogoHtml@TeX4ht}
%    \begin{macrocode}
\expandafter
\let\csname HoLogoHtml@TeX4ht\expandafter\endcsname
\csname HoLogo@TeX4ht\endcsname
%    \end{macrocode}
%    \end{macro}
%
%
% \subsubsection{\hologo{SageTeX}}
%
%    \begin{macro}{\HoLogo@SageTeX}
%    \begin{macrocode}
\def\HoLogo@SageTeX#1{%
  \HOLOGO@mbox{Sage}%
  \HOLOGO@discretionary
  \HOLOGO@NegativeKerning{eT,oT,To}%
  \hologo{TeX}%
}
%    \end{macrocode}
%    \end{macro}
%    \begin{macro}{\HoLogoHtml@SageTeX}
%    \begin{macrocode}
\let\HoLogoHtml@SageTeX\HoLogo@SageTeX
%    \end{macrocode}
%    \end{macro}
%
% \subsection{\hologo{METAFONT} and friends}
%
%    \begin{macro}{\HoLogo@METAFONT}
%    \begin{macrocode}
\def\HoLogo@METAFONT#1{%
  \HoLogoFont@font{METAFONT}{logo}{%
    \HOLOGO@mbox{META}%
    \HOLOGO@discretionary
    \HOLOGO@mbox{FONT}%
  }%
}
%    \end{macrocode}
%    \end{macro}
%
%    \begin{macro}{\HoLogo@METAPOST}
%    \begin{macrocode}
\def\HoLogo@METAPOST#1{%
  \HoLogoFont@font{METAPOST}{logo}{%
    \HOLOGO@mbox{META}%
    \HOLOGO@discretionary
    \HOLOGO@mbox{POST}%
  }%
}
%    \end{macrocode}
%    \end{macro}
%
%    \begin{macro}{\HoLogo@MetaFun}
%    \begin{macrocode}
\def\HoLogo@MetaFun#1{%
  \HOLOGO@mbox{Meta}%
  \HOLOGO@discretionary
  \HOLOGO@mbox{Fun}%
}
%    \end{macrocode}
%    \end{macro}
%
%    \begin{macro}{\HoLogo@MetaPost}
%    \begin{macrocode}
\def\HoLogo@MetaPost#1{%
  \HOLOGO@mbox{Meta}%
  \HOLOGO@discretionary
  \HOLOGO@mbox{Post}%
}
%    \end{macrocode}
%    \end{macro}
%
% \subsection{Others}
%
% \subsubsection{\hologo{biber}}
%
%    \begin{macro}{\HoLogo@biber}
%    \begin{macrocode}
\def\HoLogo@biber#1{%
  \HOLOGO@mbox{#1{b}{B}i}%
  \HOLOGO@discretionary
  \HOLOGO@mbox{ber}%
}
%    \end{macrocode}
%    \end{macro}
%    \begin{macro}{\HoLogoCs@biber}
%    \begin{macrocode}
\def\HoLogoCs@biber#1{#1{b}{B}iber}
%    \end{macrocode}
%    \end{macro}
%    \begin{macro}{\HoLogoBkm@biber}
%    \begin{macrocode}
\def\HoLogoBkm@biber#1{%
  #1{b}{B}iber%
}
%    \end{macrocode}
%    \end{macro}
%    \begin{macro}{\HoLogoHtml@biber}
%    \begin{macrocode}
\let\HoLogoHtml@biber\HoLogo@biber
%    \end{macrocode}
%    \end{macro}
%
% \subsubsection{\hologo{KOMAScript}}
%
%    \begin{macro}{\HoLogo@KOMAScript}
%    The definition for \hologo{KOMAScript} is taken
%    from \hologo{KOMAScript} (\xfile{scrlogo.dtx}, reformatted) \cite{scrlogo}:
%\begin{quote}
%\begin{verbatim}
%\@ifundefined{KOMAScript}{%
%  \DeclareRobustCommand{\KOMAScript}{%
%    \textsf{%
%      K\kern.05em O\kern.05emM\kern.05em A%
%      \kern.1em-\kern.1em %
%      Script%
%    }%
%  }%
%}{}
%\end{verbatim}
%\end{quote}
%    \begin{macrocode}
\def\HoLogo@KOMAScript#1{%
  \HoLogoFont@font{KOMAScript}{sf}{%
    \HOLOGO@mbox{%
      K\kern.05em%
      O\kern.05em%
      M\kern.05em%
      A%
    }%
    \kern.1em%
    \HOLOGO@hyphen
    \kern.1em%
    \HOLOGO@mbox{Script}%
  }%
}
%    \end{macrocode}
%    \end{macro}
%    \begin{macro}{\HoLogoBkm@KOMAScript}
%    \begin{macrocode}
\def\HoLogoBkm@KOMAScript#1{%
  KOMA-Script%
}
%    \end{macrocode}
%    \end{macro}
%    \begin{macro}{\HoLogoHtml@KOMAScript}
%    \begin{macrocode}
\def\HoLogoHtml@KOMAScript#1{%
  \HoLogoCss@KOMAScript
  \HoLogoFont@font{KOMAScript}{sf}{%
    \HOLOGO@Span{KOMAScript}{%
      K%
      \HOLOGO@Span{O}{O}%
      M%
      \HOLOGO@Span{A}{A}%
      \HOLOGO@Span{hyphen}{-}%
      Script%
    }%
  }%
}
%    \end{macrocode}
%    \end{macro}
%    \begin{macro}{\HoLogoCss@KOMAScript}
%    \begin{macrocode}
\def\HoLogoCss@KOMAScript{%
  \Css{%
    span.HoLogo-KOMAScript{%
      font-family:sans-serif;%
    }%
  }%
  \Css{%
    span.HoLogo-KOMAScript span.HoLogo-O{%
      padding-left:.05em;%
      padding-right:.05em;%
    }%
  }%
  \Css{%
    span.HoLogo-KOMAScript span.HoLogo-A{%
      padding-left:.05em;%
    }%
  }%
  \Css{%
    span.HoLogo-KOMAScript span.HoLogo-hyphen{%
      padding-left:.1em;%
      padding-right:.1em;%
    }%
  }%
  \global\let\HoLogoCss@KOMAScript\relax
}
%    \end{macrocode}
%    \end{macro}
%
% \subsubsection{\hologo{LyX}}
%
%    \begin{macro}{\HoLogo@LyX}
%    The definition is taken from the documentation source files
%    of \hologo{LyX}, \xfile{Intro.lyx} \cite{LyX}:
%\begin{quote}
%\begin{verbatim}
%\def\LyX{%
%  \texorpdfstring{%
%    L\kern-.1667em\lower.25em\hbox{Y}\kern-.125emX\@%
%  }{%
%    LyX%
%  }%
%}
%\end{verbatim}
%\end{quote}
%    \begin{macrocode}
\def\HoLogo@LyX#1{%
  L%
  \kern-.1667em%
  \lower.25em\hbox{Y}%
  \kern-.125em%
  X%
  \HOLOGO@SpaceFactor
}
%    \end{macrocode}
%    \end{macro}
%    \begin{macro}{\HoLogoHtml@LyX}
%    \begin{macrocode}
\def\HoLogoHtml@LyX#1{%
  \HoLogoCss@LyX
  \HOLOGO@Span{LyX}{%
    L%
    \HOLOGO@Span{y}{Y}%
    X%
  }%
}
%    \end{macrocode}
%    \end{macro}
%    \begin{macro}{\HoLogoCss@LyX}
%    \begin{macrocode}
\def\HoLogoCss@LyX{%
  \Css{%
    span.HoLogo-LyX span.HoLogo-y{%
      position:relative;%
      top:.25em;%
      margin-left:-.1667em;%
      margin-right:-.125em;%
      text-decoration:none;%
    }%
  }%
  \global\let\HoLogoCss@LyX\relax
}
%    \end{macrocode}
%    \end{macro}
%
% \subsubsection{\hologo{NTS}}
%
%    \begin{macro}{\HoLogo@NTS}
%    Definition for \hologo{NTS} can be found in
%    package \xpackage{etex\textunderscore man} for the \hologo{eTeX} manual \cite{etexman}
%    and in package \xpackage{dtklogos} \cite{dtklogos}:
%\begin{quote}
%\begin{verbatim}
%\def\NTS{%
%  \leavevmode
%  \hbox{%
%    $%
%      \cal N%
%      \kern-0.35em%
%      \lower0.5ex\hbox{$\cal T$}%
%      \kern-0.2em%
%      S%
%    $%
%  }%
%}
%\end{verbatim}
%\end{quote}
%    \begin{macrocode}
\def\HoLogo@NTS#1{%
  \HoLogoFont@font{NTS}{sy}{%
    N\/%
    \kern-.35em%
    \lower.5ex\hbox{T\/}%
    \kern-.2em%
    S\/%
  }%
  \HOLOGO@SpaceFactor
}
%    \end{macrocode}
%    \end{macro}
%
% \subsubsection{\Hologo{TTH} (\hologo{TeX} to HTML translator)}
%
%    Source: \url{http://hutchinson.belmont.ma.us/tth/}
%    In the HTML source the second `T' is printed as subscript.
%\begin{quote}
%\begin{verbatim}
%T<sub>T</sub>H
%\end{verbatim}
%\end{quote}
%    \begin{macro}{\HoLogo@TTH}
%    \begin{macrocode}
\def\HoLogo@TTH#1{%
  \ltx@mbox{%
    T\HOLOGO@SubScript{T}H%
  }%
  \HOLOGO@SpaceFactor
}
%    \end{macrocode}
%    \end{macro}
%
%    \begin{macro}{\HoLogoHtml@TTH}
%    \begin{macrocode}
\def\HoLogoHtml@TTH#1{%
  T\HCode{<sub>}T\HCode{</sub>}H%
}
%    \end{macrocode}
%    \end{macro}
%
% \subsubsection{\Hologo{HanTheThanh}}
%
%    Partial source: Package \xpackage{dtklogos}.
%    The double accent is U+1EBF (latin small letter e with circumflex
%    and acute).
%    \begin{macro}{\HoLogo@HanTheThanh}
%    \begin{macrocode}
\def\HoLogo@HanTheThanh#1{%
  \ltx@mbox{H\`an}%
  \HOLOGO@space
  \ltx@mbox{%
    Th%
    \HOLOGO@IfCharExists{"1EBF}{%
      \char"1EBF\relax
    }{%
      \^e\hbox to 0pt{\hss\raise .5ex\hbox{\'{}}}%
    }%
  }%
  \HOLOGO@space
  \ltx@mbox{Th\`anh}%
}
%    \end{macrocode}
%    \end{macro}
%    \begin{macro}{\HoLogoBkm@HanTheThanh}
%    \begin{macrocode}
\def\HoLogoBkm@HanTheThanh#1{%
  H\`an %
  Th\HOLOGO@PdfdocUnicode{\^e}{\9036\277} %
  Th\`anh%
}
%    \end{macrocode}
%    \end{macro}
%    \begin{macro}{\HoLogoHtml@HanTheThanh}
%    \begin{macrocode}
\def\HoLogoHtml@HanTheThanh#1{%
  H\`an %
  Th\HCode{&\ltx@hashchar x1ebf;} %
  Th\`anh%
}
%    \end{macrocode}
%    \end{macro}
%
% \subsection{Driver detection}
%
%    \begin{macrocode}
\HOLOGO@IfExists\InputIfFileExists{%
  \InputIfFileExists{hologo.cfg}{}{}%
}{%
  \ltx@IfUndefined{pdf@filesize}{%
    \def\HOLOGO@InputIfExists{%
      \openin\HOLOGO@temp=hologo.cfg\relax
      \ifeof\HOLOGO@temp
        \closein\HOLOGO@temp
      \else
        \closein\HOLOGO@temp
        \begingroup
          \def\x{LaTeX2e}%
        \expandafter\endgroup
        \ifx\fmtname\x
          \input{hologo.cfg}%
        \else
          \input hologo.cfg\relax
        \fi
      \fi
    }%
    \ltx@IfUndefined{newread}{%
      \chardef\HOLOGO@temp=15 %
      \def\HOLOGO@CheckRead{%
        \ifeof\HOLOGO@temp
          \HOLOGO@InputIfExists
        \else
          \ifcase\HOLOGO@temp
            \@PackageWarningNoLine{hologo}{%
              Configuration file ignored, because\MessageBreak
              a free read register could not be found%
            }%
          \else
            \begingroup
              \count\ltx@cclv=\HOLOGO@temp
              \advance\ltx@cclv by \ltx@minusone
              \edef\x{\endgroup
                \chardef\noexpand\HOLOGO@temp=\the\count\ltx@cclv
                \relax
              }%
            \x
          \fi
        \fi
      }%
    }{%
      \csname newread\endcsname\HOLOGO@temp
      \HOLOGO@InputIfExists
    }%
  }{%
    \edef\HOLOGO@temp{\pdf@filesize{hologo.cfg}}%
    \ifx\HOLOGO@temp\ltx@empty
    \else
      \ifnum\HOLOGO@temp>0 %
        \begingroup
          \def\x{LaTeX2e}%
        \expandafter\endgroup
        \ifx\fmtname\x
          \input{hologo.cfg}%
        \else
          \input hologo.cfg\relax
        \fi
      \else
        \@PackageInfoNoLine{hologo}{%
          Empty configuration file `hologo.cfg' ignored%
        }%
      \fi
    \fi
  }%
}
%    \end{macrocode}
%
%    \begin{macrocode}
\def\HOLOGO@temp#1#2{%
  \kv@define@key{HoLogoDriver}{#1}[]{%
    \begingroup
      \def\HOLOGO@temp{##1}%
      \ltx@onelevel@sanitize\HOLOGO@temp
      \ifx\HOLOGO@temp\ltx@empty
      \else
        \@PackageError{hologo}{%
          Value (\HOLOGO@temp) not permitted for option `#1'%
        }%
        \@ehc
      \fi
    \endgroup
    \def\hologoDriver{#2}%
  }%
}%
\def\HOLOGO@@temp#1#2{%
  \ifx\kv@value\relax
    \HOLOGO@temp{#1}{#1}%
  \else
    \HOLOGO@temp{#1}{#2}%
  \fi
}%
\kv@parse@normalized{%
  pdftex,%
  luatex=pdftex,%
  dvipdfm,%
  dvipdfmx=dvipdfm,%
  dvips,%
  dvipsone=dvips,%
  xdvi=dvips,%
  xetex,%
  vtex,%
}\HOLOGO@@temp
%    \end{macrocode}
%
%    \begin{macrocode}
\kv@define@key{HoLogoDriver}{driverfallback}{%
  \def\HOLOGO@DriverFallback{#1}%
}
%    \end{macrocode}
%
%    \begin{macro}{\HOLOGO@DriverFallback}
%    \begin{macrocode}
\def\HOLOGO@DriverFallback{dvips}
%    \end{macrocode}
%    \end{macro}
%
%    \begin{macro}{\hologoDriverSetup}
%    \begin{macrocode}
\def\hologoDriverSetup{%
  \let\hologoDriver\ltx@undefined
  \HOLOGO@DriverSetup
}
%    \end{macrocode}
%    \end{macro}
%
%    \begin{macro}{\HOLOGO@DriverSetup}
%    \begin{macrocode}
\def\HOLOGO@DriverSetup#1{%
  \kvsetkeys{HoLogoDriver}{#1}%
  \HOLOGO@CheckDriver
  \ltx@ifundefined{hologoDriver}{%
    \begingroup
    \edef\x{\endgroup
      \noexpand\kvsetkeys{HoLogoDriver}{\HOLOGO@DriverFallback}%
    }\x
  }{}%
  \@PackageInfoNoLine{hologo}{Using driver `\hologoDriver'}%
}
%    \end{macrocode}
%    \end{macro}
%
%    \begin{macro}{\HOLOGO@CheckDriver}
%    \begin{macrocode}
\def\HOLOGO@CheckDriver{%
  \ifpdf
    \def\hologoDriver{pdftex}%
    \let\HOLOGO@pdfliteral\pdfliteral
    \ifluatex
      \ifx\pdfextension\@undefined\else
        \protected\def\pdfliteral{\pdfextension literal}%
        \let\HOLOGO@pdfliteral\pdfliteral
      \fi
      \ltx@IfUndefined{HOLOGO@pdfliteral}{%
        \ifnum\luatexversion<36 %
        \else
          \begingroup
            \let\HOLOGO@temp\endgroup
            \ifcase0%
                \directlua{%
                  if tex.enableprimitives then %
                    tex.enableprimitives('HOLOGO@', {'pdfliteral'})%
                  else %
                    tex.print('1')%
                  end%
                }%
                \ifx\HOLOGO@pdfliteral\@undefined 1\fi%
                \relax%
              \endgroup
              \let\HOLOGO@temp\relax
              \global\let\HOLOGO@pdfliteral\HOLOGO@pdfliteral
            \fi%
          \HOLOGO@temp
        \fi
      }{}%
    \fi
    \ltx@IfUndefined{HOLOGO@pdfliteral}{%
      \@PackageWarningNoLine{hologo}{%
        Cannot find \string\pdfliteral
      }%
    }{}%
  \else
    \ifxetex
      \def\hologoDriver{xetex}%
    \else
      \ifvtex
        \def\hologoDriver{vtex}%
      \fi
    \fi
  \fi
}
%    \end{macrocode}
%    \end{macro}
%
%    \begin{macro}{\HOLOGO@WarningUnsupportedDriver}
%    \begin{macrocode}
\def\HOLOGO@WarningUnsupportedDriver#1{%
  \@PackageWarningNoLine{hologo}{%
    Logo `#1' needs driver specific macros,\MessageBreak
    but driver `\hologoDriver' is not supported.\MessageBreak
    Use a different driver or\MessageBreak
    load package `graphics' or `pgf'%
  }%
}
%    \end{macrocode}
%    \end{macro}
%
% \subsubsection{Reflect box macros}
%
%    Skip driver part if not needed.
%    \begin{macrocode}
\ltx@IfUndefined{reflectbox}{}{%
  \ltx@IfUndefined{rotatebox}{}{%
    \HOLOGO@AtEnd
  }%
}
\ltx@IfUndefined{pgftext}{}{%
  \HOLOGO@AtEnd
}
\ltx@IfUndefined{psscalebox}{}{%
  \HOLOGO@AtEnd
}
%    \end{macrocode}
%
%    \begin{macrocode}
\def\HOLOGO@temp{LaTeX2e}
\ifx\fmtname\HOLOGO@temp
  \RequirePackage{kvoptions}[2011/06/30]%
  \ProcessKeyvalOptions{HoLogoDriver}%
\fi
\HOLOGO@DriverSetup{}
%    \end{macrocode}
%
%    \begin{macro}{\HOLOGO@ReflectBox}
%    \begin{macrocode}
\def\HOLOGO@ReflectBox#1{%
  \begingroup
    \setbox\ltx@zero\hbox{\begingroup#1\endgroup}%
    \setbox\ltx@two\hbox{%
      \kern\wd\ltx@zero
      \csname HOLOGO@ScaleBox@\hologoDriver\endcsname{-1}{1}{%
        \hbox to 0pt{\copy\ltx@zero\hss}%
      }%
    }%
    \wd\ltx@two=\wd\ltx@zero
    \box\ltx@two
  \endgroup
}
%    \end{macrocode}
%    \end{macro}
%
%    \begin{macro}{\HOLOGO@PointReflectBox}
%    \begin{macrocode}
\def\HOLOGO@PointReflectBox#1{%
  \begingroup
    \setbox\ltx@zero\hbox{\begingroup#1\endgroup}%
    \setbox\ltx@two\hbox{%
      \kern\wd\ltx@zero
      \raise\ht\ltx@zero\hbox{%
        \csname HOLOGO@ScaleBox@\hologoDriver\endcsname{-1}{-1}{%
          \hbox to 0pt{\copy\ltx@zero\hss}%
        }%
      }%
    }%
    \wd\ltx@two=\wd\ltx@zero
    \box\ltx@two
  \endgroup
}
%    \end{macrocode}
%    \end{macro}
%
%    We must define all variants because of dynamic driver setup.
%    \begin{macrocode}
\def\HOLOGO@temp#1#2{#2}
%    \end{macrocode}
%
%    \begin{macro}{\HOLOGO@ScaleBox@pdftex}
%    \begin{macrocode}
\HOLOGO@temp{pdftex}{%
  \def\HOLOGO@ScaleBox@pdftex#1#2#3{%
    \HOLOGO@pdfliteral{%
      q #1 0 0 #2 0 0 cm%
    }%
    #3%
    \HOLOGO@pdfliteral{%
      Q%
    }%
  }%
}
%    \end{macrocode}
%    \end{macro}
%    \begin{macro}{\HOLOGO@ScaleBox@dvips}
%    \begin{macrocode}
\HOLOGO@temp{dvips}{%
  \def\HOLOGO@ScaleBox@dvips#1#2#3{%
    \special{ps:%
      gsave %
      currentpoint %
      currentpoint translate %
      #1 #2 scale %
      neg exch neg exch translate%
    }%
    #3%
    \special{ps:%
      currentpoint %
      grestore %
      moveto%
    }%
  }%
}
%    \end{macrocode}
%    \end{macro}
%    \begin{macro}{\HOLOGO@ScaleBox@dvipdfm}
%    \begin{macrocode}
\HOLOGO@temp{dvipdfm}{%
  \let\HOLOGO@ScaleBox@dvipdfm\HOLOGO@ScaleBox@dvips
}
%    \end{macrocode}
%    \end{macro}
%    Since \hologo{XeTeX} v0.6.
%    \begin{macro}{\HOLOGO@ScaleBox@xetex}
%    \begin{macrocode}
\HOLOGO@temp{xetex}{%
  \def\HOLOGO@ScaleBox@xetex#1#2#3{%
    \special{x:gsave}%
    \special{x:scale #1 #2}%
    #3%
    \special{x:grestore}%
  }%
}
%    \end{macrocode}
%    \end{macro}
%    \begin{macro}{\HOLOGO@ScaleBox@vtex}
%    \begin{macrocode}
\HOLOGO@temp{vtex}{%
  \def\HOLOGO@ScaleBox@vtex#1#2#3{%
    \special{r(#1,0,0,#2,0,0}%
    #3%
    \special{r)}%
  }%
}
%    \end{macrocode}
%    \end{macro}
%
%    \begin{macrocode}
\HOLOGO@AtEnd%
%</package>
%    \end{macrocode}
%
% \section{Test}
%
% \subsection{Catcode checks for loading}
%
%    \begin{macrocode}
%<*test1>
%    \end{macrocode}
%    \begin{macrocode}
\catcode`\{=1 %
\catcode`\}=2 %
\catcode`\#=6 %
\catcode`\@=11 %
\expandafter\ifx\csname count@\endcsname\relax
  \countdef\count@=255 %
\fi
\expandafter\ifx\csname @gobble\endcsname\relax
  \long\def\@gobble#1{}%
\fi
\expandafter\ifx\csname @firstofone\endcsname\relax
  \long\def\@firstofone#1{#1}%
\fi
\expandafter\ifx\csname loop\endcsname\relax
  \expandafter\@firstofone
\else
  \expandafter\@gobble
\fi
{%
  \def\loop#1\repeat{%
    \def\body{#1}%
    \iterate
  }%
  \def\iterate{%
    \body
      \let\next\iterate
    \else
      \let\next\relax
    \fi
    \next
  }%
  \let\repeat=\fi
}%
\def\RestoreCatcodes{}
\count@=0 %
\loop
  \edef\RestoreCatcodes{%
    \RestoreCatcodes
    \catcode\the\count@=\the\catcode\count@\relax
  }%
\ifnum\count@<255 %
  \advance\count@ 1 %
\repeat

\def\RangeCatcodeInvalid#1#2{%
  \count@=#1\relax
  \loop
    \catcode\count@=15 %
  \ifnum\count@<#2\relax
    \advance\count@ 1 %
  \repeat
}
\def\RangeCatcodeCheck#1#2#3{%
  \count@=#1\relax
  \loop
    \ifnum#3=\catcode\count@
    \else
      \errmessage{%
        Character \the\count@\space
        with wrong catcode \the\catcode\count@\space
        instead of \number#3%
      }%
    \fi
  \ifnum\count@<#2\relax
    \advance\count@ 1 %
  \repeat
}
\def\space{ }
\expandafter\ifx\csname LoadCommand\endcsname\relax
  \def\LoadCommand{\input hologo.sty\relax}%
\fi
\def\Test{%
  \RangeCatcodeInvalid{0}{47}%
  \RangeCatcodeInvalid{58}{64}%
  \RangeCatcodeInvalid{91}{96}%
  \RangeCatcodeInvalid{123}{255}%
  \catcode`\@=12 %
  \catcode`\\=0 %
  \catcode`\%=14 %
  \LoadCommand
  \RangeCatcodeCheck{0}{36}{15}%
  \RangeCatcodeCheck{37}{37}{14}%
  \RangeCatcodeCheck{38}{47}{15}%
  \RangeCatcodeCheck{48}{57}{12}%
  \RangeCatcodeCheck{58}{63}{15}%
  \RangeCatcodeCheck{64}{64}{12}%
  \RangeCatcodeCheck{65}{90}{11}%
  \RangeCatcodeCheck{91}{91}{15}%
  \RangeCatcodeCheck{92}{92}{0}%
  \RangeCatcodeCheck{93}{96}{15}%
  \RangeCatcodeCheck{97}{122}{11}%
  \RangeCatcodeCheck{123}{255}{15}%
  \RestoreCatcodes
}
\Test
\csname @@end\endcsname
\end
%    \end{macrocode}
%    \begin{macrocode}
%</test1>
%    \end{macrocode}
%
% \subsection{Spacefactor}
%
%    The space factor must be 1000 after a logo. If it is greater 1000
%    then the following space is a space after a sentence closing point.
%    If the space factor is smaller 1000 then an immediate following
%    dot is interpreted as abbreviation, not sentence closing point.
%
%    \begin{macrocode}
%<*test-spacefactor>
\NeedsTeXFormat{LaTeX2e}
\documentclass{article}
\usepackage{hologo}[2016/05/12]
\usepackage{kvsetkeys}
\usepackage{qstest}
\IncludeTests{*}
\LogTests{log}{*}{*}
\begin{document}
\begin{qstest}{spacefactor}{spacefactor}
\newcommand*{\Test}[1]{%
  \sbox0{%
    \hologo{#1}%
    \Expect*{1000 (#1)}*{\the\spacefactor\space(#1)}%
  }%
}%
\makeatletter
\def\TestList{}
\def\hologoEntry#1#2#3{%
  \edef\TestList{%
    \ifx\TestList\@empty
    \else
      \TestList,%
    \fi
    #1%
    \ifx\\#2\\%
    \else
      ={variant=#2}%
    \fi
  }%
}
\hologoList
\expandafter\kv@parse@normalized\expandafter{%
  \TestList
}{%
  \begingroup
    \let\@logo=\kv@key
    \ifx\kv@value\relax
    \else
      \expandafter\hologoLogoSetup\expandafter\@logo\expandafter{%
        \kv@value
      }%
    \fi
    \Test\@logo
  \endgroup
  \@gobbletwo
}
\end{qstest}
\end{document}
%</test-spacefactor>
%    \end{macrocode}
%
% \subsection{Complete list}
%
%    \begin{macrocode}
%<*test-list>
\NeedsTeXFormat{LaTeX2e}
\documentclass[12pt,a4paper]{article}
\usepackage{hologo}[2016/05/12]
\usepackage[T1]{fontenc}
\usepackage{lmodern}
\usepackage{parskip}
\usepackage[unicode]{hyperref}[2011/09/28]
\usepackage{bookmark}[2011/09/19]
\bookmarksetup{%
  numbered,%
  open,%
  openlevel=2,%
}
\renewcommand*{\contentsname}{List of logos}
\begin{document}
\tableofcontents
\def\TestFont#1#2#3#4#5#6{%
  \begingroup
    \usefont{#3}{#4}{#5}{#6}%
    \HologoVariant{#1}{#2}/\hologoVariant{#1}{#2}%
    \quad
    \begingroup\scriptsize\hologoVariant{#1}{#2}\endgroup
    \quad
  \endgroup
  (#3/#4/#5/#6)%
  \par
}
\makeatletter
\def\hologoEntry#1#2#3{%
  \section{%
    \HologoVariant{#1}{#2}/\hologoVariant{#1}{#2} %
    {[#1\ifx\\#2\\\else\space(#2)\fi]}% hash-ok
  }% braces around [] because of bug in tex4ht
  \begingroup
    \hypersetup{unicode=false}%
    \bookmark[%
      dest=\@currentHref,%
      rellevel=1,%
      keeplevel,%
    ]{%
      \HologoVariant{#1}{#2}/\hologoVariant{#1}{#2} %
      (PDFDocEncoding)%
    }%
  \endgroup
  \TestFont{#1}{#2}{OT1}{cmr}{m}{n}%
  \TestFont{#1}{#2}{OT1}{cmss}{m}{n}%
  \TestFont{#1}{#2}{OT1}{cmr}{b}{n}%
  \TestFont{#1}{#2}{OT1}{cmr}{m}{it}%
  \TestFont{#1}{#2}{OT1}{cmtt}{m}{n}%
  \TestFont{#1}{#2}{T1}{lmr}{m}{n}%
  \TestFont{#1}{#2}{T1}{lmss}{m}{n}%
  \TestFont{#1}{#2}{T1}{lmr}{b}{n}%
  \TestFont{#1}{#2}{T1}{lmr}{m}{it}%
  \TestFont{#1}{#2}{T1}{lmtt}{m}{n}%
  \TestFont{#1}{#2}{T1}{lmvtt}{m}{n}%
  \TestFont{#1}{#2}{T1}{qtm}{m}{n}%
  \TestFont{#1}{#2}{T1}{qhv}{m}{n}%
  \TestFont{#1}{#2}{T1}{qtm}{b}{n}%
  \TestFont{#1}{#2}{T1}{qtm}{m}{it}%
  \TestFont{#1}{#2}{T1}{qcr}{m}{n}%
  \newpage
}
\makeatother
\hologoList
\end{document}
%</test-list>
%    \end{macrocode}
%
% \section{Installation}
%
% \subsection{Download}
%
% \paragraph{Package.} This package is available on
% CTAN\footnote{\url{ftp://ftp.ctan.org/tex-archive/}}:
% \begin{description}
% \item[\CTAN{macros/latex/contrib/oberdiek/hologo.dtx}] The source file.
% \item[\CTAN{macros/latex/contrib/oberdiek/hologo.pdf}] Documentation.
% \end{description}
%
%
% \paragraph{Bundle.} All the packages of the bundle `oberdiek'
% are also available in a TDS compliant ZIP archive. There
% the packages are already unpacked and the documentation files
% are generated. The files and directories obey the TDS standard.
% \begin{description}
% \item[\CTAN{install/macros/latex/contrib/oberdiek.tds.zip}]
% \end{description}
% \emph{TDS} refers to the standard ``A Directory Structure
% for \TeX\ Files'' (\CTAN{tds/tds.pdf}). Directories
% with \xfile{texmf} in their name are usually organized this way.
%
% \subsection{Bundle installation}
%
% \paragraph{Unpacking.} Unpack the \xfile{oberdiek.tds.zip} in the
% TDS tree (also known as \xfile{texmf} tree) of your choice.
% Example (linux):
% \begin{quote}
%   |unzip oberdiek.tds.zip -d ~/texmf|
% \end{quote}
%
% \paragraph{Script installation.}
% Check the directory \xfile{TDS:scripts/oberdiek/} for
% scripts that need further installation steps.
% Package \xpackage{attachfile2} comes with the Perl script
% \xfile{pdfatfi.pl} that should be installed in such a way
% that it can be called as \texttt{pdfatfi}.
% Example (linux):
% \begin{quote}
%   |chmod +x scripts/oberdiek/pdfatfi.pl|\\
%   |cp scripts/oberdiek/pdfatfi.pl /usr/local/bin/|
% \end{quote}
%
% \subsection{Package installation}
%
% \paragraph{Unpacking.} The \xfile{.dtx} file is a self-extracting
% \docstrip\ archive. The files are extracted by running the
% \xfile{.dtx} through \plainTeX:
% \begin{quote}
%   \verb|tex hologo.dtx|
% \end{quote}
%
% \paragraph{TDS.} Now the different files must be moved into
% the different directories in your installation TDS tree
% (also known as \xfile{texmf} tree):
% \begin{quote}
% \def\t{^^A
% \begin{tabular}{@{}>{\ttfamily}l@{ $\rightarrow$ }>{\ttfamily}l@{}}
%   hologo.sty & tex/generic/oberdiek/hologo.sty\\
%   hologo.pdf & doc/latex/oberdiek/hologo.pdf\\
%   example/hologo-example.tex & doc/latex/oberdiek/example/hologo-example.tex\\
%   test/hologo-test1.tex & doc/latex/oberdiek/test/hologo-test1.tex\\
%   test/hologo-test-spacefactor.tex & doc/latex/oberdiek/test/hologo-test-spacefactor.tex\\
%   test/hologo-test-list.tex & doc/latex/oberdiek/test/hologo-test-list.tex\\
%   hologo.dtx & source/latex/oberdiek/hologo.dtx\\
% \end{tabular}^^A
% }^^A
% \sbox0{\t}^^A
% \ifdim\wd0>\linewidth
%   \begingroup
%     \advance\linewidth by\leftmargin
%     \advance\linewidth by\rightmargin
%   \edef\x{\endgroup
%     \def\noexpand\lw{\the\linewidth}^^A
%   }\x
%   \def\lwbox{^^A
%     \leavevmode
%     \hbox to \linewidth{^^A
%       \kern-\leftmargin\relax
%       \hss
%       \usebox0
%       \hss
%       \kern-\rightmargin\relax
%     }^^A
%   }^^A
%   \ifdim\wd0>\lw
%     \sbox0{\small\t}^^A
%     \ifdim\wd0>\linewidth
%       \ifdim\wd0>\lw
%         \sbox0{\footnotesize\t}^^A
%         \ifdim\wd0>\linewidth
%           \ifdim\wd0>\lw
%             \sbox0{\scriptsize\t}^^A
%             \ifdim\wd0>\linewidth
%               \ifdim\wd0>\lw
%                 \sbox0{\tiny\t}^^A
%                 \ifdim\wd0>\linewidth
%                   \lwbox
%                 \else
%                   \usebox0
%                 \fi
%               \else
%                 \lwbox
%               \fi
%             \else
%               \usebox0
%             \fi
%           \else
%             \lwbox
%           \fi
%         \else
%           \usebox0
%         \fi
%       \else
%         \lwbox
%       \fi
%     \else
%       \usebox0
%     \fi
%   \else
%     \lwbox
%   \fi
% \else
%   \usebox0
% \fi
% \end{quote}
% If you have a \xfile{docstrip.cfg} that configures and enables \docstrip's
% TDS installing feature, then some files can already be in the right
% place, see the documentation of \docstrip.
%
% \subsection{Refresh file name databases}
%
% If your \TeX~distribution
% (\teTeX, \mikTeX, \dots) relies on file name databases, you must refresh
% these. For example, \teTeX\ users run \verb|texhash| or
% \verb|mktexlsr|.
%
% \subsection{Some details for the interested}
%
% \paragraph{Attached source.}
%
% The PDF documentation on CTAN also includes the
% \xfile{.dtx} source file. It can be extracted by
% AcrobatReader 6 or higher. Another option is \textsf{pdftk},
% e.g. unpack the file into the current directory:
% \begin{quote}
%   \verb|pdftk hologo.pdf unpack_files output .|
% \end{quote}
%
% \paragraph{Unpacking with \LaTeX.}
% The \xfile{.dtx} chooses its action depending on the format:
% \begin{description}
% \item[\plainTeX:] Run \docstrip\ and extract the files.
% \item[\LaTeX:] Generate the documentation.
% \end{description}
% If you insist on using \LaTeX\ for \docstrip\ (really,
% \docstrip\ does not need \LaTeX), then inform the autodetect routine
% about your intention:
% \begin{quote}
%   \verb|latex \let\install=y\input{hologo.dtx}|
% \end{quote}
% Do not forget to quote the argument according to the demands
% of your shell.
%
% \paragraph{Generating the documentation.}
% You can use both the \xfile{.dtx} or the \xfile{.drv} to generate
% the documentation. The process can be configured by the
% configuration file \xfile{ltxdoc.cfg}. For instance, put this
% line into this file, if you want to have A4 as paper format:
% \begin{quote}
%   \verb|\PassOptionsToClass{a4paper}{article}|
% \end{quote}
% An example follows how to generate the
% documentation with pdf\LaTeX:
% \begin{quote}
%\begin{verbatim}
%pdflatex hologo.dtx
%makeindex -s gind.ist hologo.idx
%pdflatex hologo.dtx
%makeindex -s gind.ist hologo.idx
%pdflatex hologo.dtx
%\end{verbatim}
% \end{quote}
%
% \section{Catalogue}
%
% The following XML file can be used as source for the
% \href{http://mirror.ctan.org/help/Catalogue/catalogue.html}{\TeX\ Catalogue}.
% The elements \texttt{caption} and \texttt{description} are imported
% from the original XML file from the Catalogue.
% The name of the XML file in the Catalogue is \xfile{hologo.xml}.
%    \begin{macrocode}
%<*catalogue>
<?xml version='1.0' encoding='us-ascii'?>
<!DOCTYPE entry SYSTEM 'catalogue.dtd'>
<entry datestamp='$Date$' modifier='$Author$' id='hologo'>
  <name>hologo</name>
  <caption>A collection of logos with bookmark support.</caption>
  <authorref id='auth:oberdiek'/>
  <copyright owner='Heiko Oberdiek' year='2010-2012'/>
  <license type='lppl1.3'/>
  <version number='1.10'/>
  <description>
    The package defines a single command <tt>\hologo</tt>, whose
    argument is the usual case-confused ASCII version of the logo.
    The command is bookmark-enabled, so that every logo becomes
    available in bookmarks without further work.
    <p/>
    The package is part of the <xref refid='oberdiek'>oberdiek</xref>
    bundle.
  </description>
  <documentation details='Package documentation'
      href='ctan:/macros/latex/contrib/oberdiek/hologo.pdf'/>
  <ctan file='true' path='/macros/latex/contrib/oberdiek/hologo.dtx'/>
  <miktex location='oberdiek'/>
  <texlive location='oberdiek'/>
  <install path='/macros/latex/contrib/oberdiek/oberdiek.tds.zip'/>
</entry>
%</catalogue>
%    \end{macrocode}
%
% \begin{thebibliography}{9}
% \raggedright
%
% \bibitem{btxdoc}
% Oren Patashnik,
% \textit{\hologo{BibTeX}ing},
% 1988-02-08.\\
% \CTAN{biblio/bibtex/base/}
%
% \bibitem{dtklogos}
% Gerd Neugebauer, DANTE,
% \textit{Package \xpackage{dtklogos}},
% 2011-04-25.\\
% \CTAN{usergrps/dante/dtk/dtklogos.sty}
%
% \bibitem{etexman}
% The \hologo{NTS} Team,
% \textit{The \hologo{eTeX} manual},
% 1998-02.\\
% \CTAN{systems/e-tex/v2/doc/}
%
% \bibitem{ExTeX-FAQ}
% The \hologo{ExTeX} group,
% \textit{\hologo{ExTeX}: FAQ -- How is \hologo{ExTeX} typeset?},
% 2007-04-14.\\
% \url{http://www.extex.org/documentation/faq.html}
%
% \bibitem{LyX}
% %@MISC{ LyX,
% %  title = {{LyX 2.0.0 -- The Document Processor [Computer software and manual]}},
% %  author = {{The LyX Team}},
% %  howpublished = {Internet: http://www.lyx.org},
% %  year = {2011-05-08},
% %  note = {Retrieved May 10, 2011, from http://www.lyx.org},
% %  url = {http://www.lyx.org/}
% %}
% The \hologo{LyX} Team,
% \textit{\hologo{LyX} -- The Document Processor},
% 2011-05-08.\\
% \url{http://www.lyx.org/}
%
% \bibitem{OzTeX}
% Andrew Trevorrow,
% \hologo{OzTeX} FAQ: What is the correct way to typeset ``\hologo{OzTeX}''?,
% 2011-09-15 (visited).
% \url{http://www.trevorrow.com/oztex/ozfaq.html#oztex-logo}
%
% \bibitem{PiCTeX}
% Michael Wichura,
% \textit{The \hologo{PiCTeX} macro package},
% 1987-09-21.
% \CTAN{graphics/pictex/}
%
% \bibitem{scrlogo}
% Markus Kohm,
% \textit{\hologo{KOMAScript} Datei \xfile{scrlogo.dtx}},
% 2009-01-30.\\
% \CTAN{install/macros/latex/contrib/komascript.tds.zip}
%
% \end{thebibliography}
%
% \begin{History}
%   \begin{Version}{2010/04/08 v1.0}
%   \item
%     The first version.
%   \end{Version}
%   \begin{Version}{2010/04/16 v1.1}
%   \item
%     \cs{Hologo} added for support of logos at start of a sentence.
%   \item
%     \cs{hologoSetup} and \cs{hologoLogoSetup} added.
%   \item
%     Options \xoption{break}, \xoption{hyphenbreak}, \xoption{spacebreak}
%     added.
%   \item
%     Variant support added by option \xoption{variant}.
%   \end{Version}
%   \begin{Version}{2010/04/24 v1.2}
%   \item
%     \hologo{LaTeX3} added.
%   \item
%     \hologo{VTeX} added.
%   \end{Version}
%   \begin{Version}{2010/11/21 v1.3}
%   \item
%     \hologo{iniTeX}, \hologo{virTeX} added.
%   \end{Version}
%   \begin{Version}{2011/03/25 v1.4}
%   \item
%     \hologo{ConTeXt} with variants added.
%   \item
%     Option \xoption{discretionarybreak} added as refinement for
%     option \xoption{break}.
%   \end{Version}
%   \begin{Version}{2011/04/21 v1.5}
%   \item
%     Wrong TDS directory for test files fixed.
%   \end{Version}
%   \begin{Version}{2011/10/01 v1.6}
%   \item
%     Support for package \xpackage{tex4ht} added.
%   \item
%     Support for \cs{csname} added if \cs{ifincsname} is available.
%   \item
%     New logos:
%     \hologo{(La)TeX},
%     \hologo{biber},
%     \hologo{BibTeX} (\xoption{sc}, \xoption{sf}),
%     \hologo{emTeX},
%     \hologo{ExTeX},
%     \hologo{KOMAScript},
%     \hologo{La},
%     \hologo{LyX},
%     \hologo{MiKTeX},
%     \hologo{NTS},
%     \hologo{OzMF},
%     \hologo{OzMP},
%     \hologo{OzTeX},
%     \hologo{OzTtH},
%     \hologo{PCTeX},
%     \hologo{PiC},
%     \hologo{PiCTeX},
%     \hologo{METAFONT},
%     \hologo{MetaFun},
%     \hologo{METAPOST},
%     \hologo{MetaPost},
%     \hologo{SLiTeX} (\xoption{lift}, \xoption{narrow}, \xoption{simple}),
%     \hologo{SliTeX} (\xoption{narrow}, \xoption{simple}, \xoption{lift}),
%     \hologo{teTeX}.
%   \item
%     Fixes:
%     \hologo{iniTeX},
%     \hologo{pdfLaTeX},
%     \hologo{pdfTeX},
%     \hologo{virTeX}.
%   \item
%     \cs{hologoFontSetup} and \cs{hologoLogoFontSetup} added.
%   \item
%     \cs{hologoVariant} and \cs{HologoVariant} added.
%   \end{Version}
%   \begin{Version}{2011/11/22 v1.7}
%   \item
%     New logos:
%     \hologo{BibTeX8},
%     \hologo{LaTeXML},
%     \hologo{SageTeX},
%     \hologo{TeX4ht},
%     \hologo{TTH}.
%   \item
%     \hologo{Xe} and friends: Driver stuff fixed.
%   \item
%     \hologo{Xe} and friends: Support for italic added.
%   \item
%     \hologo{Xe} and friends: Package support for \xpackage{pgf}
%     and \xpackage{pstricks} added.
%   \end{Version}
%   \begin{Version}{2011/11/29 v1.8}
%   \item
%     New logos:
%     \hologo{HanTheThanh}.
%   \end{Version}
%   \begin{Version}{2011/12/21 v1.9}
%   \item
%     Patch for package \xpackage{ifxetex} added for the case that
%     \cs{newif} is undefined in \hologo{iniTeX}.
%   \item
%     Some fixes for \hologo{iniTeX}.
%   \end{Version}
%   \begin{Version}{2012/04/26 v1.10}
%   \item
%     Fix in bookmark version of logo ``\hologo{HanTheThanh}''.
%   \end{Version}
%   \begin{Version}{2016/05/12 v1.11}
%   \item
%     Update HOLOGO@IfCharExists (previously in texlive)
%   \item define pdfliteral in current luatex.
%   \end{Version}
% \end{History}
%
% \PrintIndex
%
% \Finale
\endinput
|
% \end{quote}
% Do not forget to quote the argument according to the demands
% of your shell.
%
% \paragraph{Generating the documentation.}
% You can use both the \xfile{.dtx} or the \xfile{.drv} to generate
% the documentation. The process can be configured by the
% configuration file \xfile{ltxdoc.cfg}. For instance, put this
% line into this file, if you want to have A4 as paper format:
% \begin{quote}
%   \verb|\PassOptionsToClass{a4paper}{article}|
% \end{quote}
% An example follows how to generate the
% documentation with pdf\LaTeX:
% \begin{quote}
%\begin{verbatim}
%pdflatex hologo.dtx
%makeindex -s gind.ist hologo.idx
%pdflatex hologo.dtx
%makeindex -s gind.ist hologo.idx
%pdflatex hologo.dtx
%\end{verbatim}
% \end{quote}
%
% \section{Catalogue}
%
% The following XML file can be used as source for the
% \href{http://mirror.ctan.org/help/Catalogue/catalogue.html}{\TeX\ Catalogue}.
% The elements \texttt{caption} and \texttt{description} are imported
% from the original XML file from the Catalogue.
% The name of the XML file in the Catalogue is \xfile{hologo.xml}.
%    \begin{macrocode}
%<*catalogue>
<?xml version='1.0' encoding='us-ascii'?>
<!DOCTYPE entry SYSTEM 'catalogue.dtd'>
<entry datestamp='$Date$' modifier='$Author$' id='hologo'>
  <name>hologo</name>
  <caption>A collection of logos with bookmark support.</caption>
  <authorref id='auth:oberdiek'/>
  <copyright owner='Heiko Oberdiek' year='2010-2012'/>
  <license type='lppl1.3'/>
  <version number='1.10'/>
  <description>
    The package defines a single command <tt>\hologo</tt>, whose
    argument is the usual case-confused ASCII version of the logo.
    The command is bookmark-enabled, so that every logo becomes
    available in bookmarks without further work.
    <p/>
    The package is part of the <xref refid='oberdiek'>oberdiek</xref>
    bundle.
  </description>
  <documentation details='Package documentation'
      href='ctan:/macros/latex/contrib/oberdiek/hologo.pdf'/>
  <ctan file='true' path='/macros/latex/contrib/oberdiek/hologo.dtx'/>
  <miktex location='oberdiek'/>
  <texlive location='oberdiek'/>
  <install path='/macros/latex/contrib/oberdiek/oberdiek.tds.zip'/>
</entry>
%</catalogue>
%    \end{macrocode}
%
% \begin{thebibliography}{9}
% \raggedright
%
% \bibitem{btxdoc}
% Oren Patashnik,
% \textit{\hologo{BibTeX}ing},
% 1988-02-08.\\
% \CTAN{biblio/bibtex/base/}
%
% \bibitem{dtklogos}
% Gerd Neugebauer, DANTE,
% \textit{Package \xpackage{dtklogos}},
% 2011-04-25.\\
% \CTAN{usergrps/dante/dtk/dtklogos.sty}
%
% \bibitem{etexman}
% The \hologo{NTS} Team,
% \textit{The \hologo{eTeX} manual},
% 1998-02.\\
% \CTAN{systems/e-tex/v2/doc/}
%
% \bibitem{ExTeX-FAQ}
% The \hologo{ExTeX} group,
% \textit{\hologo{ExTeX}: FAQ -- How is \hologo{ExTeX} typeset?},
% 2007-04-14.\\
% \url{http://www.extex.org/documentation/faq.html}
%
% \bibitem{LyX}
% %@MISC{ LyX,
% %  title = {{LyX 2.0.0 -- The Document Processor [Computer software and manual]}},
% %  author = {{The LyX Team}},
% %  howpublished = {Internet: http://www.lyx.org},
% %  year = {2011-05-08},
% %  note = {Retrieved May 10, 2011, from http://www.lyx.org},
% %  url = {http://www.lyx.org/}
% %}
% The \hologo{LyX} Team,
% \textit{\hologo{LyX} -- The Document Processor},
% 2011-05-08.\\
% \url{http://www.lyx.org/}
%
% \bibitem{OzTeX}
% Andrew Trevorrow,
% \hologo{OzTeX} FAQ: What is the correct way to typeset ``\hologo{OzTeX}''?,
% 2011-09-15 (visited).
% \url{http://www.trevorrow.com/oztex/ozfaq.html#oztex-logo}
%
% \bibitem{PiCTeX}
% Michael Wichura,
% \textit{The \hologo{PiCTeX} macro package},
% 1987-09-21.
% \CTAN{graphics/pictex/}
%
% \bibitem{scrlogo}
% Markus Kohm,
% \textit{\hologo{KOMAScript} Datei \xfile{scrlogo.dtx}},
% 2009-01-30.\\
% \CTAN{install/macros/latex/contrib/komascript.tds.zip}
%
% \end{thebibliography}
%
% \begin{History}
%   \begin{Version}{2010/04/08 v1.0}
%   \item
%     The first version.
%   \end{Version}
%   \begin{Version}{2010/04/16 v1.1}
%   \item
%     \cs{Hologo} added for support of logos at start of a sentence.
%   \item
%     \cs{hologoSetup} and \cs{hologoLogoSetup} added.
%   \item
%     Options \xoption{break}, \xoption{hyphenbreak}, \xoption{spacebreak}
%     added.
%   \item
%     Variant support added by option \xoption{variant}.
%   \end{Version}
%   \begin{Version}{2010/04/24 v1.2}
%   \item
%     \hologo{LaTeX3} added.
%   \item
%     \hologo{VTeX} added.
%   \end{Version}
%   \begin{Version}{2010/11/21 v1.3}
%   \item
%     \hologo{iniTeX}, \hologo{virTeX} added.
%   \end{Version}
%   \begin{Version}{2011/03/25 v1.4}
%   \item
%     \hologo{ConTeXt} with variants added.
%   \item
%     Option \xoption{discretionarybreak} added as refinement for
%     option \xoption{break}.
%   \end{Version}
%   \begin{Version}{2011/04/21 v1.5}
%   \item
%     Wrong TDS directory for test files fixed.
%   \end{Version}
%   \begin{Version}{2011/10/01 v1.6}
%   \item
%     Support for package \xpackage{tex4ht} added.
%   \item
%     Support for \cs{csname} added if \cs{ifincsname} is available.
%   \item
%     New logos:
%     \hologo{(La)TeX},
%     \hologo{biber},
%     \hologo{BibTeX} (\xoption{sc}, \xoption{sf}),
%     \hologo{emTeX},
%     \hologo{ExTeX},
%     \hologo{KOMAScript},
%     \hologo{La},
%     \hologo{LyX},
%     \hologo{MiKTeX},
%     \hologo{NTS},
%     \hologo{OzMF},
%     \hologo{OzMP},
%     \hologo{OzTeX},
%     \hologo{OzTtH},
%     \hologo{PCTeX},
%     \hologo{PiC},
%     \hologo{PiCTeX},
%     \hologo{METAFONT},
%     \hologo{MetaFun},
%     \hologo{METAPOST},
%     \hologo{MetaPost},
%     \hologo{SLiTeX} (\xoption{lift}, \xoption{narrow}, \xoption{simple}),
%     \hologo{SliTeX} (\xoption{narrow}, \xoption{simple}, \xoption{lift}),
%     \hologo{teTeX}.
%   \item
%     Fixes:
%     \hologo{iniTeX},
%     \hologo{pdfLaTeX},
%     \hologo{pdfTeX},
%     \hologo{virTeX}.
%   \item
%     \cs{hologoFontSetup} and \cs{hologoLogoFontSetup} added.
%   \item
%     \cs{hologoVariant} and \cs{HologoVariant} added.
%   \end{Version}
%   \begin{Version}{2011/11/22 v1.7}
%   \item
%     New logos:
%     \hologo{BibTeX8},
%     \hologo{LaTeXML},
%     \hologo{SageTeX},
%     \hologo{TeX4ht},
%     \hologo{TTH}.
%   \item
%     \hologo{Xe} and friends: Driver stuff fixed.
%   \item
%     \hologo{Xe} and friends: Support for italic added.
%   \item
%     \hologo{Xe} and friends: Package support for \xpackage{pgf}
%     and \xpackage{pstricks} added.
%   \end{Version}
%   \begin{Version}{2011/11/29 v1.8}
%   \item
%     New logos:
%     \hologo{HanTheThanh}.
%   \end{Version}
%   \begin{Version}{2011/12/21 v1.9}
%   \item
%     Patch for package \xpackage{ifxetex} added for the case that
%     \cs{newif} is undefined in \hologo{iniTeX}.
%   \item
%     Some fixes for \hologo{iniTeX}.
%   \end{Version}
%   \begin{Version}{2012/04/26 v1.10}
%   \item
%     Fix in bookmark version of logo ``\hologo{HanTheThanh}''.
%   \end{Version}
%   \begin{Version}{2016/05/12 v1.11}
%   \item
%     Update HOLOGO@IfCharExists (previously in texlive)
%   \item define pdfliteral in current luatex.
%   \end{Version}
% \end{History}
%
% \PrintIndex
%
% \Finale
\endinput
%
        \else
          \input hologo.cfg\relax
        \fi
      \fi
    }%
    \ltx@IfUndefined{newread}{%
      \chardef\HOLOGO@temp=15 %
      \def\HOLOGO@CheckRead{%
        \ifeof\HOLOGO@temp
          \HOLOGO@InputIfExists
        \else
          \ifcase\HOLOGO@temp
            \@PackageWarningNoLine{hologo}{%
              Configuration file ignored, because\MessageBreak
              a free read register could not be found%
            }%
          \else
            \begingroup
              \count\ltx@cclv=\HOLOGO@temp
              \advance\ltx@cclv by \ltx@minusone
              \edef\x{\endgroup
                \chardef\noexpand\HOLOGO@temp=\the\count\ltx@cclv
                \relax
              }%
            \x
          \fi
        \fi
      }%
    }{%
      \csname newread\endcsname\HOLOGO@temp
      \HOLOGO@InputIfExists
    }%
  }{%
    \edef\HOLOGO@temp{\pdf@filesize{hologo.cfg}}%
    \ifx\HOLOGO@temp\ltx@empty
    \else
      \ifnum\HOLOGO@temp>0 %
        \begingroup
          \def\x{LaTeX2e}%
        \expandafter\endgroup
        \ifx\fmtname\x
          % \iffalse meta-comment
%
% File: hologo.dtx
% Version: 2016/05/12 v1.11
% Info: A logo collection with bookmark support
%
% Copyright (C) 2010-2012 by
%    Heiko Oberdiek <heiko.oberdiek at googlemail.com>
%
% This work may be distributed and/or modified under the
% conditions of the LaTeX Project Public License, either
% version 1.3c of this license or (at your option) any later
% version. This version of this license is in
%    http://www.latex-project.org/lppl/lppl-1-3c.txt
% and the latest version of this license is in
%    http://www.latex-project.org/lppl.txt
% and version 1.3 or later is part of all distributions of
% LaTeX version 2005/12/01 or later.
%
% This work has the LPPL maintenance status "maintained".
%
% This Current Maintainer of this work is Heiko Oberdiek.
%
% The Base Interpreter refers to any `TeX-Format',
% because some files are installed in TDS:tex/generic//.
%
% This work consists of the main source file hologo.dtx
% and the derived files
%    hologo.sty, hologo.pdf, hologo.ins, hologo.drv, hologo-example.tex,
%    hologo-test1.tex, hologo-test-spacefactor.tex,
%    hologo-test-list.tex.
%
% Distribution:
%    CTAN:macros/latex/contrib/oberdiek/hologo.dtx
%    CTAN:macros/latex/contrib/oberdiek/hologo.pdf
%
% Unpacking:
%    (a) If hologo.ins is present:
%           tex hologo.ins
%    (b) Without hologo.ins:
%           tex hologo.dtx
%    (c) If you insist on using LaTeX
%           latex \let\install=y% \iffalse meta-comment
%
% File: hologo.dtx
% Version: 2016/05/12 v1.11
% Info: A logo collection with bookmark support
%
% Copyright (C) 2010-2012 by
%    Heiko Oberdiek <heiko.oberdiek at googlemail.com>
%
% This work may be distributed and/or modified under the
% conditions of the LaTeX Project Public License, either
% version 1.3c of this license or (at your option) any later
% version. This version of this license is in
%    http://www.latex-project.org/lppl/lppl-1-3c.txt
% and the latest version of this license is in
%    http://www.latex-project.org/lppl.txt
% and version 1.3 or later is part of all distributions of
% LaTeX version 2005/12/01 or later.
%
% This work has the LPPL maintenance status "maintained".
%
% This Current Maintainer of this work is Heiko Oberdiek.
%
% The Base Interpreter refers to any `TeX-Format',
% because some files are installed in TDS:tex/generic//.
%
% This work consists of the main source file hologo.dtx
% and the derived files
%    hologo.sty, hologo.pdf, hologo.ins, hologo.drv, hologo-example.tex,
%    hologo-test1.tex, hologo-test-spacefactor.tex,
%    hologo-test-list.tex.
%
% Distribution:
%    CTAN:macros/latex/contrib/oberdiek/hologo.dtx
%    CTAN:macros/latex/contrib/oberdiek/hologo.pdf
%
% Unpacking:
%    (a) If hologo.ins is present:
%           tex hologo.ins
%    (b) Without hologo.ins:
%           tex hologo.dtx
%    (c) If you insist on using LaTeX
%           latex \let\install=y\input{hologo.dtx}
%        (quote the arguments according to the demands of your shell)
%
% Documentation:
%    (a) If hologo.drv is present:
%           latex hologo.drv
%    (b) Without hologo.drv:
%           latex hologo.dtx; ...
%    The class ltxdoc loads the configuration file ltxdoc.cfg
%    if available. Here you can specify further options, e.g.
%    use A4 as paper format:
%       \PassOptionsToClass{a4paper}{article}
%
%    Programm calls to get the documentation (example):
%       pdflatex hologo.dtx
%       makeindex -s gind.ist hologo.idx
%       pdflatex hologo.dtx
%       makeindex -s gind.ist hologo.idx
%       pdflatex hologo.dtx
%
% Installation:
%    TDS:tex/generic/oberdiek/hologo.sty
%    TDS:doc/latex/oberdiek/hologo.pdf
%    TDS:doc/latex/oberdiek/example/hologo-example.tex
%    TDS:doc/latex/oberdiek/test/hologo-test1.tex
%    TDS:doc/latex/oberdiek/test/hologo-test-spacefactor.tex
%    TDS:doc/latex/oberdiek/test/hologo-test-list.tex
%    TDS:source/latex/oberdiek/hologo.dtx
%
%<*ignore>
\begingroup
  \catcode123=1 %
  \catcode125=2 %
  \def\x{LaTeX2e}%
\expandafter\endgroup
\ifcase 0\ifx\install y1\fi\expandafter
         \ifx\csname processbatchFile\endcsname\relax\else1\fi
         \ifx\fmtname\x\else 1\fi\relax
\else\csname fi\endcsname
%</ignore>
%<*install>
\input docstrip.tex
\Msg{************************************************************************}
\Msg{* Installation}
\Msg{* Package: hologo 2016/05/12 v1.11 A logo collection with bookmark support (HO)}
\Msg{************************************************************************}

\keepsilent
\askforoverwritefalse

\let\MetaPrefix\relax
\preamble

This is a generated file.

Project: hologo
Version: 2016/05/12 v1.11

Copyright (C) 2010-2012 by
   Heiko Oberdiek <heiko.oberdiek at googlemail.com>

This work may be distributed and/or modified under the
conditions of the LaTeX Project Public License, either
version 1.3c of this license or (at your option) any later
version. This version of this license is in
   http://www.latex-project.org/lppl/lppl-1-3c.txt
and the latest version of this license is in
   http://www.latex-project.org/lppl.txt
and version 1.3 or later is part of all distributions of
LaTeX version 2005/12/01 or later.

This work has the LPPL maintenance status "maintained".

This Current Maintainer of this work is Heiko Oberdiek.

The Base Interpreter refers to any `TeX-Format',
because some files are installed in TDS:tex/generic//.

This work consists of the main source file hologo.dtx
and the derived files
   hologo.sty, hologo.pdf, hologo.ins, hologo.drv, hologo-example.tex,
   hologo-test1.tex, hologo-test-spacefactor.tex,
   hologo-test-list.tex.

\endpreamble
\let\MetaPrefix\DoubleperCent

\generate{%
  \file{hologo.ins}{\from{hologo.dtx}{install}}%
  \file{hologo.drv}{\from{hologo.dtx}{driver}}%
  \usedir{tex/generic/oberdiek}%
  \file{hologo.sty}{\from{hologo.dtx}{package}}%
  \usedir{doc/latex/oberdiek/example}%
  \file{hologo-example.tex}{\from{hologo.dtx}{example}}%
  \usedir{doc/latex/oberdiek/test}%
  \file{hologo-test1.tex}{\from{hologo.dtx}{test1}}%
  \file{hologo-test-spacefactor.tex}{\from{hologo.dtx}{test-spacefactor}}%
  \file{hologo-test-list.tex}{\from{hologo.dtx}{test-list}}%
  \nopreamble
  \nopostamble
  \usedir{source/latex/oberdiek/catalogue}%
  \file{hologo.xml}{\from{hologo.dtx}{catalogue}}%
}

\catcode32=13\relax% active space
\let =\space%
\Msg{************************************************************************}
\Msg{*}
\Msg{* To finish the installation you have to move the following}
\Msg{* file into a directory searched by TeX:}
\Msg{*}
\Msg{*     hologo.sty}
\Msg{*}
\Msg{* To produce the documentation run the file `hologo.drv'}
\Msg{* through LaTeX.}
\Msg{*}
\Msg{* Happy TeXing!}
\Msg{*}
\Msg{************************************************************************}

\endbatchfile
%</install>
%<*ignore>
\fi
%</ignore>
%<*driver>
\NeedsTeXFormat{LaTeX2e}
\ProvidesFile{hologo.drv}%
  [2016/05/12 v1.11 A logo collection with bookmark support (HO)]%
\documentclass{ltxdoc}
\usepackage{holtxdoc}[2011/11/22]
\usepackage{hologo}[2016/05/12]
\usepackage{longtable}
\usepackage{array}
\usepackage{paralist}
%\usepackage[T1]{fontenc}
%\usepackage{lmodern}
\begin{document}
  \DocInput{hologo.dtx}%
\end{document}
%</driver>
% \fi
%
%
% \CharacterTable
%  {Upper-case    \A\B\C\D\E\F\G\H\I\J\K\L\M\N\O\P\Q\R\S\T\U\V\W\X\Y\Z
%   Lower-case    \a\b\c\d\e\f\g\h\i\j\k\l\m\n\o\p\q\r\s\t\u\v\w\x\y\z
%   Digits        \0\1\2\3\4\5\6\7\8\9
%   Exclamation   \!     Double quote  \"     Hash (number) \#
%   Dollar        \$     Percent       \%     Ampersand     \&
%   Acute accent  \'     Left paren    \(     Right paren   \)
%   Asterisk      \*     Plus          \+     Comma         \,
%   Minus         \-     Point         \.     Solidus       \/
%   Colon         \:     Semicolon     \;     Less than     \<
%   Equals        \=     Greater than  \>     Question mark \?
%   Commercial at \@     Left bracket  \[     Backslash     \\
%   Right bracket \]     Circumflex    \^     Underscore    \_
%   Grave accent  \`     Left brace    \{     Vertical bar  \|
%   Right brace   \}     Tilde         \~}
%
% \GetFileInfo{hologo.drv}
%
% \title{The \xpackage{hologo} package}
% \date{2016/05/12 v1.11}
% \author{Heiko Oberdiek\\\xemail{heiko.oberdiek at googlemail.com}}
%
% \maketitle
%
% \begin{abstract}
% This package starts a collection of logos with support for bookmarks
% strings.
% \end{abstract}
%
% \tableofcontents
%
% \section{Documentation}
%
% \subsection{Logo macros}
%
% \begin{declcs}{hologo} \M{name}
% \end{declcs}
% Macro \cs{hologo} sets the logo with name \meta{name}.
% The following table shows the supported names.
%
% \begingroup
%   \def\hologoEntry#1#2#3{^^A
%     #1&#2&\hologoLogoSetup{#1}{variant=#2}\hologo{#1}&#3\tabularnewline
%   }
%   \begin{longtable}{>{\ttfamily}l>{\ttfamily}lll}
%     \rmfamily\bfseries{name} & \rmfamily\bfseries variant
%     & \bfseries logo & \bfseries since\\
%     \hline
%     \endhead
%     \hologoList
%   \end{longtable}
% \endgroup
%
% \begin{declcs}{Hologo} \M{name}
% \end{declcs}
% Macro \cs{Hologo} starts the logo \meta{name} with an uppercase
% letter. As an exception small greek letters are not converted
% to uppercase. Examples, see \hologo{eTeX} and \hologo{ExTeX}.
%
% \subsection{Setup macros}
%
% The package does not support package options, but the following
% setup macros can be used to set options.
%
% \begin{declcs}{hologoSetup} \M{key value list}
% \end{declcs}
% Macro \cs{hologoSetup} sets global options.
%
% \begin{declcs}{hologoLogoSetup} \M{logo} \M{key value list}
% \end{declcs}
% Some options can also be used to configure a logo.
% These settings take precedence over global option settings.
%
% \subsection{Options}\label{sec:options}
%
% There are boolean and string options:
% \begin{description}
% \item[Boolean option:]
% It takes |true| or |false|
% as value. If the value is omitted, then |true| is used.
% \item[String option:]
% A value must be given as string. (But the string might be empty.)
% \end{description}
% The following options can be used both in \cs{hologoSetup}
% and \cs{hologoLogoSetup}:
% \begin{description}
% \def\entry#1{\item[\xoption{#1}:]}
% \entry{break}
%   enables or disables line breaks inside the logo. This setting is
%   refined by options \xoption{hyphenbreak}, \xoption{spacebreak}
%   or \xoption{discretionarybreak}.
%   Default is |false|.
% \entry{hyphenbreak}
%   enables or disables the line break right after the hyphen character.
% \entry{spacebreak}
%   enables or disables line breaks at space characters.
% \entry{discretionarybreak}
%   enables or disables line breaks at hyphenation points
%   (inserted by \cs{-}).
% \end{description}
% Macro \cs{hologoLogoSetup} also knows:
% \begin{description}
% \item[\xoption{variant}:]
%   This is a string option. It specifies a variant of a logo that
%   must exist. An empty string selects the package default variant.
% \end{description}
% Example:
% \begin{quote}
%   |\hologoSetup{break=false}|\\
%   |\hologoLogoSetup{plainTeX}{variant=hyphen,hyphenbreak}|\\
%   Then ``plain-\TeX'' contains one break point after the hyphen.
% \end{quote}
%
% \subsection{Driver options}
%
% Sometimes graphical operations are needed to construct some
% glyphs (e.g.\ \hologo{XeTeX}). If package \xpackage{graphics}
% or package \xpackage{pgf} are found, then the macros are taken
% from there. Otherwise the packge defines its own operations
% and therefore needs the driver information. Many drivers are
% detected automatically (\hologo{pdfTeX}/\hologo{LuaTeX}
% in PDF mode, \hologo{XeTeX}, \hologo{VTeX}). These have precedence
% over a driver option. The driver can be given as package option
% or using \cs{hologoDriverSetup}.
% The following list contains the recognized driver options:
% \begin{itemize}
% \item \xoption{pdftex}, \xoption{luatex}
% \item \xoption{dvipdfm}, \xoption{dvipdfmx}
% \item \xoption{dvips}, \xoption{dvipsone}, \xoption{xdvi}
% \item \xoption{xetex}
% \item \xoption{vtex}
% \end{itemize}
% The left driver of a line is the driver name that is used internally.
% The following names are aliases for drivers that use the
% same method. Therefore the entry in the \xext{log} file for
% the used driver prints the internally used driver name.
% \begin{description}
% \item[\xoption{driverfallback}:]
%   This option expects a driver that is used,
%   if the driver could not be detected automatically.
% \end{description}
%
% \begin{declcs}{hologoDriverSetup} \M{driver option}
% \end{declcs}
% The driver can also be configured after package loading
% using \cs{hologoDriverSetup}, also the way for \hologo{plainTeX}
% to setup the driver.
%
% \subsection{Font setup}
%
% Some logos require a special font, but should also be usable by
% \hologo{plainTeX}. Therefore the package provides some ways
% to influence the font settings. The options below
% take font settings as values. Both font commands
% such as \cs{sffamily} and macros that take one argument
% like \cs{textsf} can be used.
%
% \begin{declcs}{hologoFontSetup} \M{key value list}
% \end{declcs}
% Macro \cs{hologoFontSetup} sets the fonts for all logos.
% Supported keys:
% \begin{description}
% \def\entry#1{\item[\xoption{#1}:]}
% \entry{general}
%   This font is used for all logos. The default is empty.
%   That means no special font is used.
% \entry{bibsf}
%   This font is used for
%   {\hologoLogoSetup{BibTeX}{variant=sf}\hologo{BibTeX}}
%   with variant \xoption{sf}.
% \entry{rm}
%   This font is a serif font. It is used for \hologo{ExTeX}.
% \entry{sc}
%   This font specifies a small caps font. It is used for
%   {\hologoLogoSetup{BibTeX}{variant=sc}\hologo{BibTeX}}
%   with variant \xoption{sc}.
% \entry{sf}
%   This font specifies a sans serif font. The default
%   is \cs{sffamily}, then \cs{sf} is tried. Otherwise
%   a warning is given. It is used by \hologo{KOMAScript}.
% \entry{sy}
%   This is the font for math symbols (e.g. cmsy).
%   It is used by \hologo{AmS}, \hologo{NTS}, \hologo{ExTeX}.
% \entry{logo}
%   \hologo{METAFONT} and \hologo{METAPOST} are using that font.
%   In \hologo{LaTeX} \cs{logofamily} is used and
%   the definitions of package \xpackage{mflogo} are used
%   if the package is not loaded.
%   Otherwise the \cs{tenlogo} is used and defined
%   if it does not already exists.
% \end{description}
%
% \begin{declcs}{hologoLogoFontSetup} \M{logo} \M{key value list}
% \end{declcs}
% Fonts can also be set for a logo or logo component separately,
% see the following list.
% The keys are the same as for \cs{hologoFontSetup}.
%
% \begin{longtable}{>{\ttfamily}l>{\sffamily}ll}
%   \meta{logo} & keys & result\\
%   \hline
%   \endhead
%   BibTeX & bibsf & {\hologoLogoSetup{BibTeX}{variant=sf}\hologo{BibTeX}}\\[.5ex]
%   BibTeX & sc & {\hologoLogoSetup{BibTeX}{variant=sc}\hologo{BibTeX}}\\[.5ex]
%   ExTeX & rm & \hologo{ExTeX}\\
%   SliTeX & rm & \hologo{SliTeX}\\[.5ex]
%   AmS & sy & \hologo{AmS}\\
%   ExTeX & sy & \hologo{ExTeX}\\
%   NTS & sy & \hologo{NTS}\\[.5ex]
%   KOMAScript & sf & \hologo{KOMAScript}\\[.5ex]
%   METAFONT & logo & \hologo{METAFONT}\\
%   METAPOST & logo & \hologo{METAPOST}\\[.5ex]
%   SliTeX & sc \hologo{SliTeX}
% \end{longtable}
%
% \subsubsection{Font order}
%
% For all logos the font \xoption{general} is applied first.
% Example:
%\begin{quote}
%|\hologoFontSetup{general=\color{red}}|
%\end{quote}
% will print red logos.
% Then if the font uses a special font \xoption{sf}, for example,
% the font is applied that is setup by \cs{hologoLogoFontSetup}.
% If this font is not setup, then the common font setup
% by \cs{hologoFontSetup} is used. Otherwise a warning is given,
% that there is no font configured.
%
% \subsection{Additional user macros}
%
% Usually a variant of a logo is configured by using
% \cs{hologoLogoSetup}, because it is bad style to mix
% different variants of the same logo in the same text.
% There the following macros are a convenience for testing.
%
% \begin{declcs}{hologoVariant} \M{name} \M{variant}\\
%   \cs{HologoVariant} \M{name} \M{variant}
% \end{declcs}
% Logo \meta{name} is set using \meta{variant} that specifies
% explicitely which variant of the macro is used. If the argument
% is empty, then the default form of the logo is used
% (configurable by \cs{hologoLogoSetup}).
%
% \cs{HologoVariant} is used if the logo is set in a context
% that needs an uppercase first letter (beginning of a sentence, \dots).
%
% \begin{declcs}{hologoList}\\
%   \cs{hologoEntry} \M{logo} \M{variant} \M{since}
% \end{declcs}
% Macro \cs{hologoList} contains all logos that are provided
% by the package including variants. The list consists of calls
% of \cs{hologoEntry} with three arguments starting with the
% logo name \meta{logo} and its variant \meta{variant}. An empty
% variant means the current default. Argument \meta{since} specifies
% with version of the package \xpackage{hologo} is needed to get
% the logo. If the logo is fixed, then the date gets updated.
% Therefore the date \meta{since} is not exactly the date of
% the first introduction, but rather the date of the latest fix.
%
% Before \cs{hologoList} can be used, macro \cs{hologoEntry} needs
% a definition. The example file in section \ref{sec:example}
% shows applications of \cs{hologoList}.
%
% \subsection{Supported contexts}
%
% Macros \cs{hologo} and friends support special contexts:
% \begin{itemize}
% \item \hologo{LaTeX}'s protection mechanism.
% \item Bookmarks of package \xpackage{hyperref}.
% \item Package \xpackage{tex4ht}.
% \item The macros can be used inside \cs{csname} constructs,
%   if \cs{ifincsname} is available (\hologo{pdfTeX}, \hologo{XeTeX},
%   \hologo{LuaTeX}).
% \end{itemize}
%
% \subsection{Example}
% \label{sec:example}
%
% The following example prints the logos in different fonts.
%    \begin{macrocode}
%<*example>
%<<verbatim
\NeedsTeXFormat{LaTeX2e}
\documentclass[a4paper]{article}
\usepackage[
  hmargin=20mm,
  vmargin=20mm,
]{geometry}
\pagestyle{empty}
\usepackage{hologo}[2016/05/12]
\usepackage{longtable}
\usepackage{array}
\setlength{\extrarowheight}{2pt}
\usepackage[T1]{fontenc}
\usepackage{lmodern}
\usepackage{pdflscape}
\usepackage[
  pdfencoding=auto,
]{hyperref}
\hypersetup{
  pdfauthor={Heiko Oberdiek},
  pdftitle={Example for package `hologo'},
  pdfsubject={Logos with fonts lmr, lmss, qtm, qpl, qhv},
}
\usepackage{bookmark}

% Print the logo list on the console

\begingroup
  \typeout{}%
  \typeout{*** Begin of logo list ***}%
  \newcommand*{\hologoEntry}[3]{%
    \typeout{#1 \ifx\\#2\\\else(#2) \fi[#3]}%
  }%
  \hologoList
  \typeout{*** End of logo list ***}%
  \typeout{}%
\endgroup

\begin{document}
\begin{landscape}

  \section{Example file for package `hologo'}

  % Table for font names

  \begin{longtable}{>{\bfseries}ll}
    \textbf{font} & \textbf{Font name}\\
    \hline
    lmr & Latin Modern Roman\\
    lmss & Latin Modern Sans\\
    qtm & \TeX\ Gyre Termes\\
    qhv & \TeX\ Gyre Heros\\
    qpl & \TeX\ Gyre Pagella\\
  \end{longtable}

  % Logo list with logos in different fonts

  \begingroup
    \newcommand*{\SetVariant}[2]{%
      \ifx\\#2\\%
      \else
        \hologoLogoSetup{#1}{variant=#2}%
      \fi
    }%
    \newcommand*{\hologoEntry}[3]{%
      \SetVariant{#1}{#2}%
      \raisebox{1em}[0pt][0pt]{\hypertarget{#1@#2}{}}%
      \bookmark[%
        dest={#1@#2},%
      ]{%
        #1\ifx\\#2\\\else\space(#2)\fi: \Hologo{#1}, \hologo{#1} %
        [Unicode]%
      }%
      \hypersetup{unicode=false}%
      \bookmark[%
        dest={#1@#2},%
      ]{%
        #1\ifx\\#2\\\else\space(#2)\fi: \Hologo{#1}, \hologo{#1} %
        [PDFDocEncoding]%
      }%
      \texttt{#1}%
      &%
      \texttt{#2}%
      &%
      \Hologo{#1}%
      &%
      \SetVariant{#1}{#2}%
      \hologo{#1}%
      &%
      \SetVariant{#1}{#2}%
      \fontfamily{qtm}\selectfont
      \hologo{#1}%
      &%
      \SetVariant{#1}{#2}%
      \fontfamily{qpl}\selectfont
      \hologo{#1}%
      &%
      \SetVariant{#1}{#2}%
      \textsf{\hologo{#1}}%
      &%
      \SetVariant{#1}{#2}%
      \fontfamily{qhv}\selectfont
      \hologo{#1}%
      \tabularnewline
    }%
    \begin{longtable}{llllllll}%
      \textbf{\textit{logo}} & \textbf{\textit{variant}} &
      \texttt{\string\Hologo} &
      \textbf{lmr} & \textbf{qtm} & \textbf{qpl} &
      \textbf{lmss} & \textbf{qhv}
      \tabularnewline
      \hline
      \endhead
      \hologoList
    \end{longtable}%
  \endgroup

\end{landscape}
\end{document}
%verbatim
%</example>
%    \end{macrocode}
%
% \StopEventually{
% }
%
% \section{Implementation}
%    \begin{macrocode}
%<*package>
%    \end{macrocode}
%    Reload check, especially if the package is not used with \LaTeX.
%    \begin{macrocode}
\begingroup\catcode61\catcode48\catcode32=10\relax%
  \catcode13=5 % ^^M
  \endlinechar=13 %
  \catcode35=6 % #
  \catcode39=12 % '
  \catcode44=12 % ,
  \catcode45=12 % -
  \catcode46=12 % .
  \catcode58=12 % :
  \catcode64=11 % @
  \catcode123=1 % {
  \catcode125=2 % }
  \expandafter\let\expandafter\x\csname ver@hologo.sty\endcsname
  \ifx\x\relax % plain-TeX, first loading
  \else
    \def\empty{}%
    \ifx\x\empty % LaTeX, first loading,
      % variable is initialized, but \ProvidesPackage not yet seen
    \else
      \expandafter\ifx\csname PackageInfo\endcsname\relax
        \def\x#1#2{%
          \immediate\write-1{Package #1 Info: #2.}%
        }%
      \else
        \def\x#1#2{\PackageInfo{#1}{#2, stopped}}%
      \fi
      \x{hologo}{The package is already loaded}%
      \aftergroup\endinput
    \fi
  \fi
\endgroup%
%    \end{macrocode}
%    Package identification:
%    \begin{macrocode}
\begingroup\catcode61\catcode48\catcode32=10\relax%
  \catcode13=5 % ^^M
  \endlinechar=13 %
  \catcode35=6 % #
  \catcode39=12 % '
  \catcode40=12 % (
  \catcode41=12 % )
  \catcode44=12 % ,
  \catcode45=12 % -
  \catcode46=12 % .
  \catcode47=12 % /
  \catcode58=12 % :
  \catcode64=11 % @
  \catcode91=12 % [
  \catcode93=12 % ]
  \catcode123=1 % {
  \catcode125=2 % }
  \expandafter\ifx\csname ProvidesPackage\endcsname\relax
    \def\x#1#2#3[#4]{\endgroup
      \immediate\write-1{Package: #3 #4}%
      \xdef#1{#4}%
    }%
  \else
    \def\x#1#2[#3]{\endgroup
      #2[{#3}]%
      \ifx#1\@undefined
        \xdef#1{#3}%
      \fi
      \ifx#1\relax
        \xdef#1{#3}%
      \fi
    }%
  \fi
\expandafter\x\csname ver@hologo.sty\endcsname
\ProvidesPackage{hologo}%
  [2016/05/12 v1.11 A logo collection with bookmark support (HO)]%
%    \end{macrocode}
%
%    \begin{macrocode}
\begingroup\catcode61\catcode48\catcode32=10\relax%
  \catcode13=5 % ^^M
  \endlinechar=13 %
  \catcode123=1 % {
  \catcode125=2 % }
  \catcode64=11 % @
  \def\x{\endgroup
    \expandafter\edef\csname HOLOGO@AtEnd\endcsname{%
      \endlinechar=\the\endlinechar\relax
      \catcode13=\the\catcode13\relax
      \catcode32=\the\catcode32\relax
      \catcode35=\the\catcode35\relax
      \catcode61=\the\catcode61\relax
      \catcode64=\the\catcode64\relax
      \catcode123=\the\catcode123\relax
      \catcode125=\the\catcode125\relax
    }%
  }%
\x\catcode61\catcode48\catcode32=10\relax%
\catcode13=5 % ^^M
\endlinechar=13 %
\catcode35=6 % #
\catcode64=11 % @
\catcode123=1 % {
\catcode125=2 % }
\def\TMP@EnsureCode#1#2{%
  \edef\HOLOGO@AtEnd{%
    \HOLOGO@AtEnd
    \catcode#1=\the\catcode#1\relax
  }%
  \catcode#1=#2\relax
}
\TMP@EnsureCode{10}{12}% ^^J
\TMP@EnsureCode{33}{12}% !
\TMP@EnsureCode{34}{12}% "
\TMP@EnsureCode{36}{3}% $
\TMP@EnsureCode{38}{4}% &
\TMP@EnsureCode{39}{12}% '
\TMP@EnsureCode{40}{12}% (
\TMP@EnsureCode{41}{12}% )
\TMP@EnsureCode{42}{12}% *
\TMP@EnsureCode{43}{12}% +
\TMP@EnsureCode{44}{12}% ,
\TMP@EnsureCode{45}{12}% -
\TMP@EnsureCode{46}{12}% .
\TMP@EnsureCode{47}{12}% /
\TMP@EnsureCode{58}{12}% :
\TMP@EnsureCode{59}{12}% ;
\TMP@EnsureCode{60}{12}% <
\TMP@EnsureCode{62}{12}% >
\TMP@EnsureCode{63}{12}% ?
\TMP@EnsureCode{91}{12}% [
\TMP@EnsureCode{93}{12}% ]
\TMP@EnsureCode{94}{7}% ^ (superscript)
\TMP@EnsureCode{95}{8}% _ (subscript)
\TMP@EnsureCode{96}{12}% `
\TMP@EnsureCode{124}{12}% |
\edef\HOLOGO@AtEnd{%
  \HOLOGO@AtEnd
  \escapechar\the\escapechar\relax
  \noexpand\endinput
}
\escapechar=92 %
%    \end{macrocode}
%
% \subsection{Logo list}
%
%    \begin{macro}{\hologoList}
%    \begin{macrocode}
\def\hologoList{%
  \hologoEntry{(La)TeX}{}{2011/10/01}%
  \hologoEntry{AmSLaTeX}{}{2010/04/16}%
  \hologoEntry{AmSTeX}{}{2010/04/16}%
  \hologoEntry{biber}{}{2011/10/01}%
  \hologoEntry{BibTeX}{}{2011/10/01}%
  \hologoEntry{BibTeX}{sf}{2011/10/01}%
  \hologoEntry{BibTeX}{sc}{2011/10/01}%
  \hologoEntry{BibTeX8}{}{2011/11/22}%
  \hologoEntry{ConTeXt}{}{2011/03/25}%
  \hologoEntry{ConTeXt}{narrow}{2011/03/25}%
  \hologoEntry{ConTeXt}{simple}{2011/03/25}%
  \hologoEntry{emTeX}{}{2010/04/26}%
  \hologoEntry{eTeX}{}{2010/04/08}%
  \hologoEntry{ExTeX}{}{2011/10/01}%
  \hologoEntry{HanTheThanh}{}{2011/11/29}%
  \hologoEntry{iniTeX}{}{2011/10/01}%
  \hologoEntry{KOMAScript}{}{2011/10/01}%
  \hologoEntry{La}{}{2010/05/08}%
  \hologoEntry{LaTeX}{}{2010/04/08}%
  \hologoEntry{LaTeX2e}{}{2010/04/08}%
  \hologoEntry{LaTeX3}{}{2010/04/24}%
  \hologoEntry{LaTeXe}{}{2010/04/08}%
  \hologoEntry{LaTeXML}{}{2011/11/22}%
  \hologoEntry{LaTeXTeX}{}{2011/10/01}%
  \hologoEntry{LuaLaTeX}{}{2010/04/08}%
  \hologoEntry{LuaTeX}{}{2010/04/08}%
  \hologoEntry{LyX}{}{2011/10/01}%
  \hologoEntry{METAFONT}{}{2011/10/01}%
  \hologoEntry{MetaFun}{}{2011/10/01}%
  \hologoEntry{METAPOST}{}{2011/10/01}%
  \hologoEntry{MetaPost}{}{2011/10/01}%
  \hologoEntry{MiKTeX}{}{2011/10/01}%
  \hologoEntry{NTS}{}{2011/10/01}%
  \hologoEntry{OzMF}{}{2011/10/01}%
  \hologoEntry{OzMP}{}{2011/10/01}%
  \hologoEntry{OzTeX}{}{2011/10/01}%
  \hologoEntry{OzTtH}{}{2011/10/01}%
  \hologoEntry{PCTeX}{}{2011/10/01}%
  \hologoEntry{pdfTeX}{}{2011/10/01}%
  \hologoEntry{pdfLaTeX}{}{2011/10/01}%
  \hologoEntry{PiC}{}{2011/10/01}%
  \hologoEntry{PiCTeX}{}{2011/10/01}%
  \hologoEntry{plainTeX}{}{2010/04/08}%
  \hologoEntry{plainTeX}{space}{2010/04/16}%
  \hologoEntry{plainTeX}{hyphen}{2010/04/16}%
  \hologoEntry{plainTeX}{runtogether}{2010/04/16}%
  \hologoEntry{SageTeX}{}{2011/11/22}%
  \hologoEntry{SLiTeX}{}{2011/10/01}%
  \hologoEntry{SLiTeX}{lift}{2011/10/01}%
  \hologoEntry{SLiTeX}{narrow}{2011/10/01}%
  \hologoEntry{SLiTeX}{simple}{2011/10/01}%
  \hologoEntry{SliTeX}{}{2011/10/01}%
  \hologoEntry{SliTeX}{narrow}{2011/10/01}%
  \hologoEntry{SliTeX}{simple}{2011/10/01}%
  \hologoEntry{SliTeX}{lift}{2011/10/01}%
  \hologoEntry{teTeX}{}{2011/10/01}%
  \hologoEntry{TeX}{}{2010/04/08}%
  \hologoEntry{TeX4ht}{}{2011/11/22}%
  \hologoEntry{TTH}{}{2011/11/22}%
  \hologoEntry{virTeX}{}{2011/10/01}%
  \hologoEntry{VTeX}{}{2010/04/24}%
  \hologoEntry{Xe}{}{2010/04/08}%
  \hologoEntry{XeLaTeX}{}{2010/04/08}%
  \hologoEntry{XeTeX}{}{2010/04/08}%
}
%    \end{macrocode}
%    \end{macro}
%
% \subsection{Load resources}
%
%    \begin{macrocode}
\begingroup\expandafter\expandafter\expandafter\endgroup
\expandafter\ifx\csname RequirePackage\endcsname\relax
  \def\TMP@RequirePackage#1[#2]{%
    \begingroup\expandafter\expandafter\expandafter\endgroup
    \expandafter\ifx\csname ver@#1.sty\endcsname\relax
      \input #1.sty\relax
    \fi
  }%
  \TMP@RequirePackage{ltxcmds}[2011/02/04]%
  \TMP@RequirePackage{infwarerr}[2010/04/08]%
  \TMP@RequirePackage{kvsetkeys}[2010/03/01]%
  \TMP@RequirePackage{kvdefinekeys}[2010/03/01]%
  \TMP@RequirePackage{pdftexcmds}[2010/04/01]%
  \TMP@RequirePackage{ifpdf}[2010/01/28]%
  \TMP@RequirePackage{ifluatex}[2010/03/01]%
  \ltx@IfUndefined{newif}{%
    \expandafter\let\csname newif\endcsname\ltx@newif
  }{}%
  \TMP@RequirePackage{ifxetex}[2009/01/23]%
  \TMP@RequirePackage{ifvtex}[2010/03/01]%
\else
  \RequirePackage{ltxcmds}[2011/02/04]%
  \RequirePackage{infwarerr}[2010/04/08]%
  \RequirePackage{kvsetkeys}[2010/03/01]%
  \RequirePackage{kvdefinekeys}[2010/03/01]%
  \RequirePackage{pdftexcmds}[2010/04/01]%
  \RequirePackage{ifpdf}[2010/01/28]%
  \RequirePackage{ifluatex}[2010/03/01]%
  \RequirePackage{ifxetex}[2009/01/23]%
  \RequirePackage{ifvtex}[2010/03/01]%
\fi
%    \end{macrocode}
%
%    \begin{macro}{\HOLOGO@IfDefined}
%    \begin{macrocode}
\def\HOLOGO@IfExists#1{%
  \ifx\@undefined#1%
    \expandafter\ltx@secondoftwo
  \else
    \ifx\relax#1%
      \expandafter\ltx@secondoftwo
    \else
      \expandafter\expandafter\expandafter\ltx@firstoftwo
    \fi
  \fi
}
%    \end{macrocode}
%    \end{macro}
%
% \subsection{Setup macros}
%
%    \begin{macro}{\hologoSetup}
%    \begin{macrocode}
\def\hologoSetup{%
  \let\HOLOGO@name\relax
  \HOLOGO@Setup
}
%    \end{macrocode}
%    \end{macro}
%
%    \begin{macro}{\hologoLogoSetup}
%    \begin{macrocode}
\def\hologoLogoSetup#1{%
  \edef\HOLOGO@name{#1}%
  \ltx@IfUndefined{HoLogo@\HOLOGO@name}{%
    \@PackageError{hologo}{%
      Unknown logo `\HOLOGO@name'%
    }\@ehc
    \ltx@gobble
  }{%
    \HOLOGO@Setup
  }%
}
%    \end{macrocode}
%    \end{macro}
%
%    \begin{macro}{\HOLOGO@Setup}
%    \begin{macrocode}
\def\HOLOGO@Setup{%
  \kvsetkeys{HoLogo}%
}
%    \end{macrocode}
%    \end{macro}
%
% \subsection{Options}
%
%    \begin{macro}{\HOLOGO@DeclareBoolOption}
%    \begin{macrocode}
\def\HOLOGO@DeclareBoolOption#1{%
  \expandafter\chardef\csname HOLOGOOPT@#1\endcsname\ltx@zero
  \kv@define@key{HoLogo}{#1}[true]{%
    \def\HOLOGO@temp{##1}%
    \ifx\HOLOGO@temp\HOLOGO@true
      \ifx\HOLOGO@name\relax
        \expandafter\chardef\csname HOLOGOOPT@#1\endcsname=\ltx@one
      \else
        \expandafter\chardef\csname
        HoLogoOpt@#1@\HOLOGO@name\endcsname\ltx@one
      \fi
      \HOLOGO@SetBreakAll{#1}%
    \else
      \ifx\HOLOGO@temp\HOLOGO@false
        \ifx\HOLOGO@name\relax
          \expandafter\chardef\csname HOLOGOOPT@#1\endcsname=\ltx@zero
        \else
          \expandafter\chardef\csname
          HoLogoOpt@#1@\HOLOGO@name\endcsname=\ltx@zero
        \fi
        \HOLOGO@SetBreakAll{#1}%
      \else
        \@PackageError{hologo}{%
          Unknown value `##1' for boolean option `#1'.\MessageBreak
          Known values are `true' and `false'%
        }\@ehc
      \fi
    \fi
  }%
}
%    \end{macrocode}
%    \end{macro}
%
%    \begin{macro}{\HOLOGO@SetBreakAll}
%    \begin{macrocode}
\def\HOLOGO@SetBreakAll#1{%
  \def\HOLOGO@temp{#1}%
  \ifx\HOLOGO@temp\HOLOGO@break
    \ifx\HOLOGO@name\relax
      \chardef\HOLOGOOPT@hyphenbreak=\HOLOGOOPT@break
      \chardef\HOLOGOOPT@spacebreak=\HOLOGOOPT@break
      \chardef\HOLOGOOPT@discretionarybreak=\HOLOGOOPT@break
    \else
      \expandafter\chardef
         \csname HoLogoOpt@hyphenbreak@\HOLOGO@name\endcsname=%
         \csname HoLogoOpt@break@\HOLOGO@name\endcsname
      \expandafter\chardef
         \csname HoLogoOpt@spacebreak@\HOLOGO@name\endcsname=%
         \csname HoLogoOpt@break@\HOLOGO@name\endcsname
      \expandafter\chardef
         \csname HoLogoOpt@discretionarybreak@\HOLOGO@name
             \endcsname=%
         \csname HoLogoOpt@break@\HOLOGO@name\endcsname
    \fi
  \fi
}
%    \end{macrocode}
%    \end{macro}
%
%    \begin{macro}{\HOLOGO@true}
%    \begin{macrocode}
\def\HOLOGO@true{true}
%    \end{macrocode}
%    \end{macro}
%    \begin{macro}{\HOLOGO@false}
%    \begin{macrocode}
\def\HOLOGO@false{false}
%    \end{macrocode}
%    \end{macro}
%    \begin{macro}{\HOLOGO@break}
%    \begin{macrocode}
\def\HOLOGO@break{break}
%    \end{macrocode}
%    \end{macro}
%
%    \begin{macrocode}
\HOLOGO@DeclareBoolOption{break}
\HOLOGO@DeclareBoolOption{hyphenbreak}
\HOLOGO@DeclareBoolOption{spacebreak}
\HOLOGO@DeclareBoolOption{discretionarybreak}
%    \end{macrocode}
%
%    \begin{macrocode}
\kv@define@key{HoLogo}{variant}{%
  \ifx\HOLOGO@name\relax
    \@PackageError{hologo}{%
      Option `variant' is not available in \string\hologoSetup,%
      \MessageBreak
      Use \string\hologoLogoSetup\space instead%
    }\@ehc
  \else
    \edef\HOLOGO@temp{#1}%
    \ifx\HOLOGO@temp\ltx@empty
      \expandafter
      \let\csname HoLogoOpt@variant@\HOLOGO@name\endcsname\@undefined
    \else
      \ltx@IfUndefined{HoLogo@\HOLOGO@name @\HOLOGO@temp}{%
        \@PackageError{hologo}{%
          Unknown variant `\HOLOGO@temp' of logo `\HOLOGO@name'%
        }\@ehc
      }{%
        \expandafter
        \let\csname HoLogoOpt@variant@\HOLOGO@name\endcsname
            \HOLOGO@temp
      }%
    \fi
  \fi
}
%    \end{macrocode}
%
%    \begin{macro}{\HOLOGO@Variant}
%    \begin{macrocode}
\def\HOLOGO@Variant#1{%
  #1%
  \ltx@ifundefined{HoLogoOpt@variant@#1}{%
  }{%
    @\csname HoLogoOpt@variant@#1\endcsname
  }%
}
%    \end{macrocode}
%    \end{macro}
%
% \subsection{Break/no-break support}
%
%    \begin{macro}{\HOLOGO@space}
%    \begin{macrocode}
\def\HOLOGO@space{%
  \ltx@ifundefined{HoLogoOpt@spacebreak@\HOLOGO@name}{%
    \ltx@ifundefined{HoLogoOpt@break@\HOLOGO@name}{%
      \chardef\HOLOGO@temp=\HOLOGOOPT@spacebreak
    }{%
      \chardef\HOLOGO@temp=%
        \csname HoLogoOpt@break@\HOLOGO@name\endcsname
    }%
  }{%
    \chardef\HOLOGO@temp=%
      \csname HoLogoOpt@spacebreak@\HOLOGO@name\endcsname
  }%
  \ifcase\HOLOGO@temp
    \penalty10000 %
  \fi
  \ltx@space
}
%    \end{macrocode}
%    \end{macro}
%
%    \begin{macro}{\HOLOGO@hyphen}
%    \begin{macrocode}
\def\HOLOGO@hyphen{%
  \ltx@ifundefined{HoLogoOpt@hyphenbreak@\HOLOGO@name}{%
    \ltx@ifundefined{HoLogoOpt@break@\HOLOGO@name}{%
      \chardef\HOLOGO@temp=\HOLOGOOPT@hyphenbreak
    }{%
      \chardef\HOLOGO@temp=%
        \csname HoLogoOpt@break@\HOLOGO@name\endcsname
    }%
  }{%
    \chardef\HOLOGO@temp=%
      \csname HoLogoOpt@hyphenbreak@\HOLOGO@name\endcsname
  }%
  \ifcase\HOLOGO@temp
    \ltx@mbox{-}%
  \else
    -%
  \fi
}
%    \end{macrocode}
%    \end{macro}
%
%    \begin{macro}{\HOLOGO@discretionary}
%    \begin{macrocode}
\def\HOLOGO@discretionary{%
  \ltx@ifundefined{HoLogoOpt@discretionarybreak@\HOLOGO@name}{%
    \ltx@ifundefined{HoLogoOpt@break@\HOLOGO@name}{%
      \chardef\HOLOGO@temp=\HOLOGOOPT@discretionarybreak
    }{%
      \chardef\HOLOGO@temp=%
        \csname HoLogoOpt@break@\HOLOGO@name\endcsname
    }%
  }{%
    \chardef\HOLOGO@temp=%
      \csname HoLogoOpt@discretionarybreak@\HOLOGO@name\endcsname
  }%
  \ifcase\HOLOGO@temp
  \else
    \-%
  \fi
}
%    \end{macrocode}
%    \end{macro}
%
%    \begin{macro}{\HOLOGO@mbox}
%    \begin{macrocode}
\def\HOLOGO@mbox#1{%
  \ltx@ifundefined{HoLogoOpt@break@\HOLOGO@name}{%
    \chardef\HOLOGO@temp=\HOLOGOOPT@hyphenbreak
  }{%
    \chardef\HOLOGO@temp=%
      \csname HoLogoOpt@break@\HOLOGO@name\endcsname
  }%
  \ifcase\HOLOGO@temp
    \ltx@mbox{#1}%
  \else
    #1%
  \fi
}
%    \end{macrocode}
%    \end{macro}
%
% \subsection{Font support}
%
%    \begin{macro}{\HoLogoFont@font}
%    \begin{tabular}{@{}ll@{}}
%    |#1|:& logo name\\
%    |#2|:& font short name\\
%    |#3|:& text
%    \end{tabular}
%    \begin{macrocode}
\def\HoLogoFont@font#1#2#3{%
  \begingroup
    \ltx@IfUndefined{HoLogoFont@logo@#1.#2}{%
      \ltx@IfUndefined{HoLogoFont@font@#2}{%
        \@PackageWarning{hologo}{%
          Missing font `#2' for logo `#1'%
        }%
        #3%
      }{%
        \csname HoLogoFont@font@#2\endcsname{#3}%
      }%
    }{%
      \csname HoLogoFont@logo@#1.#2\endcsname{#3}%
    }%
  \endgroup
}
%    \end{macrocode}
%    \end{macro}
%
%    \begin{macro}{\HoLogoFont@Def}
%    \begin{macrocode}
\def\HoLogoFont@Def#1{%
  \expandafter\def\csname HoLogoFont@font@#1\endcsname
}
%    \end{macrocode}
%    \end{macro}
%    \begin{macro}{\HoLogoFont@LogoDef}
%    \begin{macrocode}
\def\HoLogoFont@LogoDef#1#2{%
  \expandafter\def\csname HoLogoFont@logo@#1.#2\endcsname
}
%    \end{macrocode}
%    \end{macro}
%
% \subsubsection{Font defaults}
%
%    \begin{macro}{\HoLogoFont@font@general}
%    \begin{macrocode}
\HoLogoFont@Def{general}{}%
%    \end{macrocode}
%    \end{macro}
%
%    \begin{macro}{\HoLogoFont@font@rm}
%    \begin{macrocode}
\ltx@IfUndefined{rmfamily}{%
  \ltx@IfUndefined{rm}{%
  }{%
    \HoLogoFont@Def{rm}{\rm}%
  }%
}{%
  \HoLogoFont@Def{rm}{\rmfamily}%
}
%    \end{macrocode}
%    \end{macro}
%
%    \begin{macro}{\HoLogoFont@font@sf}
%    \begin{macrocode}
\ltx@IfUndefined{sffamily}{%
  \ltx@IfUndefined{sf}{%
  }{%
    \HoLogoFont@Def{sf}{\sf}%
  }%
}{%
  \HoLogoFont@Def{sf}{\sffamily}%
}
%    \end{macrocode}
%    \end{macro}
%
%    \begin{macro}{\HoLogoFont@font@bibsf}
%    In case of \hologo{plainTeX} the original small caps
%    variant is used as default. In \hologo{LaTeX}
%    the definition of package \xpackage{dtklogos} \cite{dtklogos}
%    is used.
%\begin{quote}
%\begin{verbatim}
%\DeclareRobustCommand{\BibTeX}{%
%  B%
%  \kern-.05em%
%  \hbox{%
%    $\m@th$% %% force math size calculations
%    \csname S@\f@size\endcsname
%    \fontsize\sf@size\z@
%    \math@fontsfalse
%    \selectfont
%    I%
%    \kern-.025em%
%    B
%  }%
%  \kern-.08em%
%  \-%
%  \TeX
%}
%\end{verbatim}
%\end{quote}
%    \begin{macrocode}
\ltx@IfUndefined{selectfont}{%
  \ltx@IfUndefined{tensc}{%
    \font\tensc=cmcsc10\relax
  }{}%
  \HoLogoFont@Def{bibsf}{\tensc}%
}{%
  \HoLogoFont@Def{bibsf}{%
    $\mathsurround=0pt$%
    \csname S@\f@size\endcsname
    \fontsize\sf@size{0pt}%
    \math@fontsfalse
    \selectfont
  }%
}
%    \end{macrocode}
%    \end{macro}
%
%    \begin{macro}{\HoLogoFont@font@sc}
%    \begin{macrocode}
\ltx@IfUndefined{scshape}{%
  \ltx@IfUndefined{tensc}{%
    \font\tensc=cmcsc10\relax
  }{}%
  \HoLogoFont@Def{sc}{\tensc}%
}{%
  \HoLogoFont@Def{sc}{\scshape}%
}
%    \end{macrocode}
%    \end{macro}
%
%    \begin{macro}{\HoLogoFont@font@sy}
%    \begin{macrocode}
\ltx@IfUndefined{usefont}{%
  \ltx@IfUndefined{tensy}{%
  }{%
    \HoLogoFont@Def{sy}{\tensy}%
  }%
}{%
  \HoLogoFont@Def{sy}{%
    \usefont{OMS}{cmsy}{m}{n}%
  }%
}
%    \end{macrocode}
%    \end{macro}
%
%    \begin{macro}{\HoLogoFont@font@logo}
%    \begin{macrocode}
\begingroup
  \def\x{LaTeX2e}%
\expandafter\endgroup
\ifx\fmtname\x
  \ltx@IfUndefined{logofamily}{%
    \DeclareRobustCommand\logofamily{%
      \not@math@alphabet\logofamily\relax
      \fontencoding{U}%
      \fontfamily{logo}%
      \selectfont
    }%
  }{}%
  \ltx@IfUndefined{logofamily}{%
  }{%
    \HoLogoFont@Def{logo}{\logofamily}%
  }%
\else
  \ltx@IfUndefined{tenlogo}{%
    \font\tenlogo=logo10\relax
  }{}%
  \HoLogoFont@Def{logo}{\tenlogo}%
\fi
%    \end{macrocode}
%    \end{macro}
%
% \subsubsection{Font setup}
%
%    \begin{macro}{\hologoFontSetup}
%    \begin{macrocode}
\def\hologoFontSetup{%
  \let\HOLOGO@name\relax
  \HOLOGO@FontSetup
}
%    \end{macrocode}
%    \end{macro}
%
%    \begin{macro}{\hologoLogoFontSetup}
%    \begin{macrocode}
\def\hologoLogoFontSetup#1{%
  \edef\HOLOGO@name{#1}%
  \ltx@IfUndefined{HoLogo@\HOLOGO@name}{%
    \@PackageError{hologo}{%
      Unknown logo `\HOLOGO@name'%
    }\@ehc
    \ltx@gobble
  }{%
    \HOLOGO@FontSetup
  }%
}
%    \end{macrocode}
%    \end{macro}
%
%    \begin{macro}{\HOLOGO@FontSetup}
%    \begin{macrocode}
\def\HOLOGO@FontSetup{%
  \kvsetkeys{HoLogoFont}%
}
%    \end{macrocode}
%    \end{macro}
%
%    \begin{macrocode}
\def\HOLOGO@temp#1{%
  \kv@define@key{HoLogoFont}{#1}{%
    \ifx\HOLOGO@name\relax
      \HoLogoFont@Def{#1}{##1}%
    \else
      \HoLogoFont@LogoDef\HOLOGO@name{#1}{##1}%
    \fi
  }%
}
\HOLOGO@temp{general}
\HOLOGO@temp{sf}
%    \end{macrocode}
%
% \subsection{Generic logo commands}
%
%    \begin{macrocode}
\HOLOGO@IfExists\hologo{%
  \@PackageError{hologo}{%
    \string\hologo\ltx@space is already defined.\MessageBreak
    Package loading is aborted%
  }\@ehc
  \HOLOGO@AtEnd
}%
\HOLOGO@IfExists\hologoRobust{%
  \@PackageError{hologo}{%
    \string\hologoRobust\ltx@space is already defined.\MessageBreak
    Package loading is aborted%
  }\@ehc
  \HOLOGO@AtEnd
}%
%    \end{macrocode}
%
% \subsubsection{\cs{hologo} and friends}
%
%    \begin{macrocode}
\ifluatex
  \expandafter\ltx@firstofone
\else
  \expandafter\ltx@gobble
\fi
{%
  \ltx@IfUndefined{ifincsname}{%
    \ifnum\luatexversion<36 %
      \expandafter\ltx@gobble
    \else
      \expandafter\ltx@firstofone
    \fi
    {%
      \begingroup
        \ifcase0%
            \directlua{%
              if tex.enableprimitives then %
                tex.enableprimitives('HOLOGO@', {'ifincsname'})%
              else %
                tex.print('1')%
              end%
            }%
            \ifx\HOLOGO@ifincsname\@undefined 1\fi%
            \relax
          \expandafter\ltx@firstofone
        \else
          \endgroup
          \expandafter\ltx@gobble
        \fi
        {%
          \global\let\ifincsname\HOLOGO@ifincsname
        }%
      \HOLOGO@temp
    }%
  }{}%
}
%    \end{macrocode}
%    \begin{macrocode}
\ltx@IfUndefined{ifincsname}{%
  \catcode`$=14 %
}{%
  \catcode`$=9 %
}
%    \end{macrocode}
%
%    \begin{macro}{\hologo}
%    \begin{macrocode}
\def\hologo#1{%
$ \ifincsname
$   \ltx@ifundefined{HoLogoCs@\HOLOGO@Variant{#1}}{%
$     #1%
$   }{%
$     \csname HoLogoCs@\HOLOGO@Variant{#1}\endcsname\ltx@firstoftwo
$   }%
$ \else
    \HOLOGO@IfExists\texorpdfstring\texorpdfstring\ltx@firstoftwo
    {%
      \hologoRobust{#1}%
    }{%
      \ltx@ifundefined{HoLogoBkm@\HOLOGO@Variant{#1}}{%
        \ltx@ifundefined{HoLogo@#1}{?#1?}{#1}%
      }{%
        \csname HoLogoBkm@\HOLOGO@Variant{#1}\endcsname
        \ltx@firstoftwo
      }%
    }%
$ \fi
}
%    \end{macrocode}
%    \end{macro}
%    \begin{macro}{\Hologo}
%    \begin{macrocode}
\def\Hologo#1{%
$ \ifincsname
$   \ltx@ifundefined{HoLogoCs@\HOLOGO@Variant{#1}}{%
$     #1%
$   }{%
$     \csname HoLogoCs@\HOLOGO@Variant{#1}\endcsname\ltx@secondoftwo
$   }%
$ \else
    \HOLOGO@IfExists\texorpdfstring\texorpdfstring\ltx@firstoftwo
    {%
      \HologoRobust{#1}%
    }{%
      \ltx@ifundefined{HoLogoBkm@\HOLOGO@Variant{#1}}{%
        \ltx@ifundefined{HoLogo@#1}{?#1?}{#1}%
      }{%
        \csname HoLogoBkm@\HOLOGO@Variant{#1}\endcsname
        \ltx@secondoftwo
      }%
    }%
$ \fi
}
%    \end{macrocode}
%    \end{macro}
%
%    \begin{macro}{\hologoVariant}
%    \begin{macrocode}
\def\hologoVariant#1#2{%
  \ifx\relax#2\relax
    \hologo{#1}%
  \else
$   \ifincsname
$     \ltx@ifundefined{HoLogoCs@#1@#2}{%
$       #1%
$     }{%
$       \csname HoLogoCs@#1@#2\endcsname\ltx@firstoftwo
$     }%
$   \else
      \HOLOGO@IfExists\texorpdfstring\texorpdfstring\ltx@firstoftwo
      {%
        \hologoVariantRobust{#1}{#2}%
      }{%
        \ltx@ifundefined{HoLogoBkm@#1@#2}{%
          \ltx@ifundefined{HoLogo@#1}{?#1?}{#1}%
        }{%
          \csname HoLogoBkm@#1@#2\endcsname
          \ltx@firstoftwo
        }%
      }%
$   \fi
  \fi
}
%    \end{macrocode}
%    \end{macro}
%    \begin{macro}{\HologoVariant}
%    \begin{macrocode}
\def\HologoVariant#1#2{%
  \ifx\relax#2\relax
    \Hologo{#1}%
  \else
$   \ifincsname
$     \ltx@ifundefined{HoLogoCs@#1@#2}{%
$       #1%
$     }{%
$       \csname HoLogoCs@#1@#2\endcsname\ltx@secondoftwo
$     }%
$   \else
      \HOLOGO@IfExists\texorpdfstring\texorpdfstring\ltx@firstoftwo
      {%
        \HologoVariantRobust{#1}{#2}%
      }{%
        \ltx@ifundefined{HoLogoBkm@#1@#2}{%
          \ltx@ifundefined{HoLogo@#1}{?#1?}{#1}%
        }{%
          \csname HoLogoBkm@#1@#2\endcsname
          \ltx@secondoftwo
        }%
      }%
$   \fi
  \fi
}
%    \end{macrocode}
%    \end{macro}
%
%    \begin{macrocode}
\catcode`\$=3 %
%    \end{macrocode}
%
% \subsubsection{\cs{hologoRobust} and friends}
%
%    \begin{macro}{\hologoRobust}
%    \begin{macrocode}
\ltx@IfUndefined{protected}{%
  \ltx@IfUndefined{DeclareRobustCommand}{%
    \def\hologoRobust#1%
  }{%
    \DeclareRobustCommand*\hologoRobust[1]%
  }%
}{%
  \protected\def\hologoRobust#1%
}%
{%
  \edef\HOLOGO@name{#1}%
  \ltx@IfUndefined{HoLogo@\HOLOGO@Variant\HOLOGO@name}{%
    \@PackageError{hologo}{%
      Unknown logo `\HOLOGO@name'%
    }\@ehc
    ?\HOLOGO@name?%
  }{%
    \ltx@IfUndefined{ver@tex4ht.sty}{%
      \HoLogoFont@font\HOLOGO@name{general}{%
        \csname HoLogo@\HOLOGO@Variant\HOLOGO@name\endcsname
        \ltx@firstoftwo
      }%
    }{%
      \ltx@IfUndefined{HoLogoHtml@\HOLOGO@Variant\HOLOGO@name}{%
        \HOLOGO@name
      }{%
        \csname HoLogoHtml@\HOLOGO@Variant\HOLOGO@name\endcsname
        \ltx@firstoftwo
      }%
    }%
  }%
}
%    \end{macrocode}
%    \end{macro}
%    \begin{macro}{\HologoRobust}
%    \begin{macrocode}
\ltx@IfUndefined{protected}{%
  \ltx@IfUndefined{DeclareRobustCommand}{%
    \def\HologoRobust#1%
  }{%
    \DeclareRobustCommand*\HologoRobust[1]%
  }%
}{%
  \protected\def\HologoRobust#1%
}%
{%
  \edef\HOLOGO@name{#1}%
  \ltx@IfUndefined{HoLogo@\HOLOGO@Variant\HOLOGO@name}{%
    \@PackageError{hologo}{%
      Unknown logo `\HOLOGO@name'%
    }\@ehc
    ?\HOLOGO@name?%
  }{%
    \ltx@IfUndefined{ver@tex4ht.sty}{%
      \HoLogoFont@font\HOLOGO@name{general}{%
        \csname HoLogo@\HOLOGO@Variant\HOLOGO@name\endcsname
        \ltx@secondoftwo
      }%
    }{%
      \ltx@IfUndefined{HoLogoHtml@\HOLOGO@Variant\HOLOGO@name}{%
        \expandafter\HOLOGO@Uppercase\HOLOGO@name
      }{%
        \csname HoLogoHtml@\HOLOGO@Variant\HOLOGO@name\endcsname
        \ltx@secondoftwo
      }%
    }%
  }%
}
%    \end{macrocode}
%    \end{macro}
%    \begin{macro}{\hologoVariantRobust}
%    \begin{macrocode}
\ltx@IfUndefined{protected}{%
  \ltx@IfUndefined{DeclareRobustCommand}{%
    \def\hologoVariantRobust#1#2%
  }{%
    \DeclareRobustCommand*\hologoVariantRobust[2]%
  }%
}{%
  \protected\def\hologoVariantRobust#1#2%
}%
{%
  \begingroup
    \hologoLogoSetup{#1}{variant={#2}}%
    \hologoRobust{#1}%
  \endgroup
}
%    \end{macrocode}
%    \end{macro}
%    \begin{macro}{\HologoVariantRobust}
%    \begin{macrocode}
\ltx@IfUndefined{protected}{%
  \ltx@IfUndefined{DeclareRobustCommand}{%
    \def\HologoVariantRobust#1#2%
  }{%
    \DeclareRobustCommand*\HologoVariantRobust[2]%
  }%
}{%
  \protected\def\HologoVariantRobust#1#2%
}%
{%
  \begingroup
    \hologoLogoSetup{#1}{variant={#2}}%
    \HologoRobust{#1}%
  \endgroup
}
%    \end{macrocode}
%    \end{macro}
%
%    \begin{macro}{\hologorobust}
%    Macro \cs{hologorobust} is only defined for compatibility.
%    Its use is deprecated.
%    \begin{macrocode}
\def\hologorobust{\hologoRobust}
%    \end{macrocode}
%    \end{macro}
%
% \subsection{Helpers}
%
%    \begin{macro}{\HOLOGO@Uppercase}
%    Macro \cs{HOLOGO@Uppercase} is restricted to \cs{uppercase},
%    because \hologo{plainTeX} or \hologo{iniTeX} do not provide
%    \cs{MakeUppercase}.
%    \begin{macrocode}
\def\HOLOGO@Uppercase#1{\uppercase{#1}}
%    \end{macrocode}
%    \end{macro}
%
%    \begin{macro}{\HOLOGO@PdfdocUnicode}
%    \begin{macrocode}
\def\HOLOGO@PdfdocUnicode{%
  \ifx\ifHy@unicode\iftrue
    \expandafter\ltx@secondoftwo
  \else
    \expandafter\ltx@firstoftwo
  \fi
}
%    \end{macrocode}
%    \end{macro}
%
%    \begin{macro}{\HOLOGO@Math}
%    \begin{macrocode}
\def\HOLOGO@MathSetup{%
  \mathsurround0pt\relax
  \HOLOGO@IfExists\f@series{%
    \if b\expandafter\ltx@car\f@series x\@nil
      \csname boldmath\endcsname
   \fi
  }{}%
}
%    \end{macrocode}
%    \end{macro}
%
%    \begin{macro}{\HOLOGO@TempDimen}
%    \begin{macrocode}
\dimendef\HOLOGO@TempDimen=\ltx@zero
%    \end{macrocode}
%    \end{macro}
%    \begin{macro}{\HOLOGO@NegativeKerning}
%    \begin{macrocode}
\def\HOLOGO@NegativeKerning#1{%
  \begingroup
    \HOLOGO@TempDimen=0pt\relax
    \comma@parse@normalized{#1}{%
      \ifdim\HOLOGO@TempDimen=0pt %
        \expandafter\HOLOGO@@NegativeKerning\comma@entry
      \fi
      \ltx@gobble
    }%
    \ifdim\HOLOGO@TempDimen<0pt %
      \kern\HOLOGO@TempDimen
    \fi
  \endgroup
}
%    \end{macrocode}
%    \end{macro}
%    \begin{macro}{\HOLOGO@@NegativeKerning}
%    \begin{macrocode}
\def\HOLOGO@@NegativeKerning#1#2{%
  \setbox\ltx@zero\hbox{#1#2}%
  \HOLOGO@TempDimen=\wd\ltx@zero
  \setbox\ltx@zero\hbox{#1\kern0pt#2}%
  \advance\HOLOGO@TempDimen by -\wd\ltx@zero
}
%    \end{macrocode}
%    \end{macro}
%
%    \begin{macro}{\HOLOGO@SpaceFactor}
%    \begin{macrocode}
\def\HOLOGO@SpaceFactor{%
  \spacefactor1000 %
}
%    \end{macrocode}
%    \end{macro}
%
%    \begin{macro}{\HOLOGO@Span}
%    \begin{macrocode}
\def\HOLOGO@Span#1#2{%
  \HCode{<span class="HoLogo-#1">}%
  #2%
  \HCode{</span>}%
}
%    \end{macrocode}
%    \end{macro}
%
% \subsubsection{Text subscript}
%
%    \begin{macro}{\HOLOGO@SubScript}%
%    \begin{macrocode}
\def\HOLOGO@SubScript#1{%
  \ltx@IfUndefined{textsubscript}{%
    \ltx@IfUndefined{text}{%
      \ltx@mbox{%
        \mathsurround=0pt\relax
        $%
          _{%
            \ltx@IfUndefined{sf@size}{%
              \mathrm{#1}%
            }{%
              \mbox{%
                \fontsize\sf@size{0pt}\selectfont
                #1%
              }%
            }%
          }%
        $%
      }%
    }{%
      \ltx@mbox{%
        \mathsurround=0pt\relax
        $_{\text{#1}}$%
      }%
    }%
  }{%
    \textsubscript{#1}%
  }%
}
%    \end{macrocode}
%    \end{macro}
%
% \subsection{\hologo{TeX} and friends}
%
% \subsubsection{\hologo{TeX}}
%
%    \begin{macro}{\HoLogo@TeX}
%    Source: \hologo{LaTeX} kernel.
%    \begin{macrocode}
\def\HoLogo@TeX#1{%
  T\kern-.1667em\lower.5ex\hbox{E}\kern-.125emX\HOLOGO@SpaceFactor
}
%    \end{macrocode}
%    \end{macro}
%    \begin{macro}{\HoLogoHtml@TeX}
%    \begin{macrocode}
\def\HoLogoHtml@TeX#1{%
  \HoLogoCss@TeX
  \HOLOGO@Span{TeX}{%
    T%
    \HOLOGO@Span{e}{%
      E%
    }%
    X%
  }%
}
%    \end{macrocode}
%    \end{macro}
%    \begin{macro}{\HoLogoCss@TeX}
%    \begin{macrocode}
\def\HoLogoCss@TeX{%
  \Css{%
    span.HoLogo-TeX span.HoLogo-e{%
      position:relative;%
      top:.5ex;%
      margin-left:-.1667em;%
      margin-right:-.125em;%
    }%
  }%
  \Css{%
    a span.HoLogo-TeX span.HoLogo-e{%
      text-decoration:none;%
    }%
  }%
  \global\let\HoLogoCss@TeX\relax
}
%    \end{macrocode}
%    \end{macro}
%
% \subsubsection{\hologo{plainTeX}}
%
%    \begin{macro}{\HoLogo@plainTeX@space}
%    Source: ``The \hologo{TeX}book''
%    \begin{macrocode}
\def\HoLogo@plainTeX@space#1{%
  \HOLOGO@mbox{#1{p}{P}lain}\HOLOGO@space\hologo{TeX}%
}
%    \end{macrocode}
%    \end{macro}
%    \begin{macro}{\HoLogoCs@plainTeX@space}
%    \begin{macrocode}
\def\HoLogoCs@plainTeX@space#1{#1{p}{P}lain TeX}%
%    \end{macrocode}
%    \end{macro}
%    \begin{macro}{\HoLogoBkm@plainTeX@space}
%    \begin{macrocode}
\def\HoLogoBkm@plainTeX@space#1{%
  #1{p}{P}lain \hologo{TeX}%
}
%    \end{macrocode}
%    \end{macro}
%    \begin{macro}{\HoLogoHtml@plainTeX@space}
%    \begin{macrocode}
\def\HoLogoHtml@plainTeX@space#1{%
  #1{p}{P}lain \hologo{TeX}%
}
%    \end{macrocode}
%    \end{macro}
%
%    \begin{macro}{\HoLogo@plainTeX@hyphen}
%    \begin{macrocode}
\def\HoLogo@plainTeX@hyphen#1{%
  \HOLOGO@mbox{#1{p}{P}lain}\HOLOGO@hyphen\hologo{TeX}%
}
%    \end{macrocode}
%    \end{macro}
%    \begin{macro}{\HoLogoCs@plainTeX@hyphen}
%    \begin{macrocode}
\def\HoLogoCs@plainTeX@hyphen#1{#1{p}{P}lain-TeX}
%    \end{macrocode}
%    \end{macro}
%    \begin{macro}{\HoLogoBkm@plainTeX@hyphen}
%    \begin{macrocode}
\def\HoLogoBkm@plainTeX@hyphen#1{%
  #1{p}{P}lain-\hologo{TeX}%
}
%    \end{macrocode}
%    \end{macro}
%    \begin{macro}{\HoLogoHtml@plainTeX@hyphen}
%    \begin{macrocode}
\def\HoLogoHtml@plainTeX@hyphen#1{%
  #1{p}{P}lain-\hologo{TeX}%
}
%    \end{macrocode}
%    \end{macro}
%
%    \begin{macro}{\HoLogo@plainTeX@runtogether}
%    \begin{macrocode}
\def\HoLogo@plainTeX@runtogether#1{%
  \HOLOGO@mbox{#1{p}{P}lain\hologo{TeX}}%
}
%    \end{macrocode}
%    \end{macro}
%    \begin{macro}{\HoLogoCs@plainTeX@runtogether}
%    \begin{macrocode}
\def\HoLogoCs@plainTeX@runtogether#1{#1{p}{P}lainTeX}
%    \end{macrocode}
%    \end{macro}
%    \begin{macro}{\HoLogoBkm@plainTeX@runtogether}
%    \begin{macrocode}
\def\HoLogoBkm@plainTeX@runtogether#1{%
  #1{p}{P}lain\hologo{TeX}%
}
%    \end{macrocode}
%    \end{macro}
%    \begin{macro}{\HoLogoHtml@plainTeX@runtogether}
%    \begin{macrocode}
\def\HoLogoHtml@plainTeX@runtogether#1{%
  #1{p}{P}lain\hologo{TeX}%
}
%    \end{macrocode}
%    \end{macro}
%
%    \begin{macro}{\HoLogo@plainTeX}
%    \begin{macrocode}
\def\HoLogo@plainTeX{\HoLogo@plainTeX@space}
%    \end{macrocode}
%    \end{macro}
%    \begin{macro}{\HoLogoCs@plainTeX}
%    \begin{macrocode}
\def\HoLogoCs@plainTeX{\HoLogoCs@plainTeX@space}
%    \end{macrocode}
%    \end{macro}
%    \begin{macro}{\HoLogoBkm@plainTeX}
%    \begin{macrocode}
\def\HoLogoBkm@plainTeX{\HoLogoBkm@plainTeX@space}
%    \end{macrocode}
%    \end{macro}
%    \begin{macro}{\HoLogoHtml@plainTeX}
%    \begin{macrocode}
\def\HoLogoHtml@plainTeX{\HoLogoHtml@plainTeX@space}
%    \end{macrocode}
%    \end{macro}
%
% \subsubsection{\hologo{LaTeX}}
%
%    Source: \hologo{LaTeX} kernel.
%\begin{quote}
%\begin{verbatim}
%\DeclareRobustCommand{\LaTeX}{%
%  L%
%  \kern-.36em%
%  {%
%    \sbox\z@ T%
%    \vbox to\ht\z@{%
%      \hbox{%
%        \check@mathfonts
%        \fontsize\sf@size\z@
%        \math@fontsfalse
%        \selectfont
%        A%
%      }%
%      \vss
%    }%
%  }%
%  \kern-.15em%
%  \TeX
%}
%\end{verbatim}
%\end{quote}
%
%    \begin{macro}{\HoLogo@La}
%    \begin{macrocode}
\def\HoLogo@La#1{%
  L%
  \kern-.36em%
  \begingroup
    \setbox\ltx@zero\hbox{T}%
    \vbox to\ht\ltx@zero{%
      \hbox{%
        \ltx@ifundefined{check@mathfonts}{%
          \csname sevenrm\endcsname
        }{%
          \check@mathfonts
          \fontsize\sf@size{0pt}%
          \math@fontsfalse\selectfont
        }%
        A%
      }%
      \vss
    }%
  \endgroup
}
%    \end{macrocode}
%    \end{macro}
%
%    \begin{macro}{\HoLogo@LaTeX}
%    Source: \hologo{LaTeX} kernel.
%    \begin{macrocode}
\def\HoLogo@LaTeX#1{%
  \hologo{La}%
  \kern-.15em%
  \hologo{TeX}%
}
%    \end{macrocode}
%    \end{macro}
%    \begin{macro}{\HoLogoHtml@LaTeX}
%    \begin{macrocode}
\def\HoLogoHtml@LaTeX#1{%
  \HoLogoCss@LaTeX
  \HOLOGO@Span{LaTeX}{%
    L%
    \HOLOGO@Span{a}{%
      A%
    }%
    \hologo{TeX}%
  }%
}
%    \end{macrocode}
%    \end{macro}
%    \begin{macro}{\HoLogoCss@LaTeX}
%    \begin{macrocode}
\def\HoLogoCss@LaTeX{%
  \Css{%
    span.HoLogo-LaTeX span.HoLogo-a{%
      position:relative;%
      top:-.5ex;%
      margin-left:-.36em;%
      margin-right:-.15em;%
      font-size:85\%;%
    }%
  }%
  \global\let\HoLogoCss@LaTeX\relax
}
%    \end{macrocode}
%    \end{macro}
%
% \subsubsection{\hologo{(La)TeX}}
%
%    \begin{macro}{\HoLogo@LaTeXTeX}
%    The kerning around the parentheses is taken
%    from package \xpackage{dtklogos} \cite{dtklogos}.
%\begin{quote}
%\begin{verbatim}
%\DeclareRobustCommand{\LaTeXTeX}{%
%  (%
%  \kern-.15em%
%  L%
%  \kern-.36em%
%  {%
%    \sbox\z@ T%
%    \vbox to\ht0{%
%      \hbox{%
%        $\m@th$%
%        \csname S@\f@size\endcsname
%        \fontsize\sf@size\z@
%        \math@fontsfalse
%        \selectfont
%        A%
%      }%
%      \vss
%    }%
%  }%
%  \kern-.2em%
%  )%
%  \kern-.15em%
%  \TeX
%}
%\end{verbatim}
%\end{quote}
%    \begin{macrocode}
\def\HoLogo@LaTeXTeX#1{%
  (%
  \kern-.15em%
  \hologo{La}%
  \kern-.2em%
  )%
  \kern-.15em%
  \hologo{TeX}%
}
%    \end{macrocode}
%    \end{macro}
%    \begin{macro}{\HoLogoBkm@LaTeXTeX}
%    \begin{macrocode}
\def\HoLogoBkm@LaTeXTeX#1{(La)TeX}
%    \end{macrocode}
%    \end{macro}
%
%    \begin{macro}{\HoLogo@(La)TeX}
%    \begin{macrocode}
\expandafter
\let\csname HoLogo@(La)TeX\endcsname\HoLogo@LaTeXTeX
%    \end{macrocode}
%    \end{macro}
%    \begin{macro}{\HoLogoBkm@(La)TeX}
%    \begin{macrocode}
\expandafter
\let\csname HoLogoBkm@(La)TeX\endcsname\HoLogoBkm@LaTeXTeX
%    \end{macrocode}
%    \end{macro}
%    \begin{macro}{\HoLogoHtml@LaTeXTeX}
%    \begin{macrocode}
\def\HoLogoHtml@LaTeXTeX#1{%
  \HoLogoCss@LaTeXTeX
  \HOLOGO@Span{LaTeXTeX}{%
    (%
    \HOLOGO@Span{L}{L}%
    \HOLOGO@Span{a}{A}%
    \HOLOGO@Span{ParenRight}{)}%
    \hologo{TeX}%
  }%
}
%    \end{macrocode}
%    \end{macro}
%    \begin{macro}{\HoLogoHtml@(La)TeX}
%    Kerning after opening parentheses and before closing parentheses
%    is $-0.1$\,em. The original values $-0.15$\,em
%    looked too ugly for a serif font.
%    \begin{macrocode}
\expandafter
\let\csname HoLogoHtml@(La)TeX\endcsname\HoLogoHtml@LaTeXTeX
%    \end{macrocode}
%    \end{macro}
%    \begin{macro}{\HoLogoCss@LaTeXTeX}
%    \begin{macrocode}
\def\HoLogoCss@LaTeXTeX{%
  \Css{%
    span.HoLogo-LaTeXTeX span.HoLogo-L{%
      margin-left:-.1em;%
    }%
  }%
  \Css{%
    span.HoLogo-LaTeXTeX span.HoLogo-a{%
      position:relative;%
      top:-.5ex;%
      margin-left:-.36em;%
      margin-right:-.1em;%
      font-size:85\%;%
    }%
  }%
  \Css{%
    span.HoLogo-LaTeXTeX span.HoLogo-ParenRight{%
      margin-right:-.15em;%
    }%
  }%
  \global\let\HoLogoCss@LaTeXTeX\relax
}
%    \end{macrocode}
%    \end{macro}
%
% \subsubsection{\hologo{LaTeXe}}
%
%    \begin{macro}{\HoLogo@LaTeXe}
%    Source: \hologo{LaTeX} kernel
%    \begin{macrocode}
\def\HoLogo@LaTeXe#1{%
  \hologo{LaTeX}%
  \kern.15em%
  \hbox{%
    \HOLOGO@MathSetup
    2%
    $_{\textstyle\varepsilon}$%
  }%
}
%    \end{macrocode}
%    \end{macro}
%
%    \begin{macro}{\HoLogoCs@LaTeXe}
%    \begin{macrocode}
\ifnum64=`\^^^^0040\relax % test for big chars of LuaTeX/XeTeX
  \catcode`\$=9 %
  \catcode`\&=14 %
\else
  \catcode`\$=14 %
  \catcode`\&=9 %
\fi
\def\HoLogoCs@LaTeXe#1{%
  LaTeX2%
$ \string ^^^^0395%
& e%
}%
\catcode`\$=3 %
\catcode`\&=4 %
%    \end{macrocode}
%    \end{macro}
%
%    \begin{macro}{\HoLogoBkm@LaTeXe}
%    \begin{macrocode}
\def\HoLogoBkm@LaTeXe#1{%
  \hologo{LaTeX}%
  2%
  \HOLOGO@PdfdocUnicode{e}{\textepsilon}%
}
%    \end{macrocode}
%    \end{macro}
%
%    \begin{macro}{\HoLogoHtml@LaTeXe}
%    \begin{macrocode}
\def\HoLogoHtml@LaTeXe#1{%
  \HoLogoCss@LaTeXe
  \HOLOGO@Span{LaTeX2e}{%
    \hologo{LaTeX}%
    \HOLOGO@Span{2}{2}%
    \HOLOGO@Span{e}{%
      \HOLOGO@MathSetup
      \ensuremath{\textstyle\varepsilon}%
    }%
  }%
}
%    \end{macrocode}
%    \end{macro}
%    \begin{macro}{\HoLogoCss@LaTeXe}
%    \begin{macrocode}
\def\HoLogoCss@LaTeXe{%
  \Css{%
    span.HoLogo-LaTeX2e span.HoLogo-2{%
      padding-left:.15em;%
    }%
  }%
  \Css{%
    span.HoLogo-LaTeX2e span.HoLogo-e{%
      position:relative;%
      top:.35ex;%
      text-decoration:none;%
    }%
  }%
  \global\let\HoLogoCss@LaTeXe\relax
}
%    \end{macrocode}
%    \end{macro}
%
%    \begin{macro}{\HoLogo@LaTeX2e}
%    \begin{macrocode}
\expandafter
\let\csname HoLogo@LaTeX2e\endcsname\HoLogo@LaTeXe
%    \end{macrocode}
%    \end{macro}
%    \begin{macro}{\HoLogoCs@LaTeX2e}
%    \begin{macrocode}
\expandafter
\let\csname HoLogoCs@LaTeX2e\endcsname\HoLogoCs@LaTeXe
%    \end{macrocode}
%    \end{macro}
%    \begin{macro}{\HoLogoBkm@LaTeX2e}
%    \begin{macrocode}
\expandafter
\let\csname HoLogoBkm@LaTeX2e\endcsname\HoLogoBkm@LaTeXe
%    \end{macrocode}
%    \end{macro}
%    \begin{macro}{\HoLogoHtml@LaTeX2e}
%    \begin{macrocode}
\expandafter
\let\csname HoLogoHtml@LaTeX2e\endcsname\HoLogoHtml@LaTeXe
%    \end{macrocode}
%    \end{macro}
%
% \subsubsection{\hologo{LaTeX3}}
%
%    \begin{macro}{\HoLogo@LaTeX3}
%    Source: \hologo{LaTeX} kernel
%    \begin{macrocode}
\expandafter\def\csname HoLogo@LaTeX3\endcsname#1{%
  \hologo{LaTeX}%
  3%
}
%    \end{macrocode}
%    \end{macro}
%
%    \begin{macro}{\HoLogoBkm@LaTeX3}
%    \begin{macrocode}
\expandafter\def\csname HoLogoBkm@LaTeX3\endcsname#1{%
  \hologo{LaTeX}%
  3%
}
%    \end{macrocode}
%    \end{macro}
%    \begin{macro}{\HoLogoHtml@LaTeX3}
%    \begin{macrocode}
\expandafter
\let\csname HoLogoHtml@LaTeX3\expandafter\endcsname
\csname HoLogo@LaTeX3\endcsname
%    \end{macrocode}
%    \end{macro}
%
% \subsubsection{\hologo{LaTeXML}}
%
%    \begin{macro}{\HoLogo@LaTeXML}
%    \begin{macrocode}
\def\HoLogo@LaTeXML#1{%
  \HOLOGO@mbox{%
    \hologo{La}%
    \kern-.15em%
    T%
    \kern-.1667em%
    \lower.5ex\hbox{E}%
    \kern-.125em%
    \HoLogoFont@font{LaTeXML}{sc}{xml}%
  }%
}
%    \end{macrocode}
%    \end{macro}
%    \begin{macro}{\HoLogoHtml@pdfLaTeX}
%    \begin{macrocode}
\def\HoLogoHtml@LaTeXML#1{%
  \HOLOGO@Span{LaTeXML}{%
    \HoLogoCss@LaTeX
    \HoLogoCss@TeX
    \HOLOGO@Span{LaTeX}{%
      L%
      \HOLOGO@Span{a}{%
        A%
      }%
    }%
    \HOLOGO@Span{TeX}{%
      T%
      \HOLOGO@Span{e}{%
        E%
      }%
    }%
    \HCode{<span style="font-variant: small-caps;">}%
    xml%
    \HCode{</span>}%
  }%
}
%    \end{macrocode}
%    \end{macro}
%
% \subsubsection{\hologo{eTeX}}
%
%    \begin{macro}{\HoLogo@eTeX}
%    Source: package \xpackage{etex}
%    \begin{macrocode}
\def\HoLogo@eTeX#1{%
  \ltx@mbox{%
    \HOLOGO@MathSetup
    $\varepsilon$%
    -%
    \HOLOGO@NegativeKerning{-T,T-,To}%
    \hologo{TeX}%
  }%
}
%    \end{macrocode}
%    \end{macro}
%    \begin{macro}{\HoLogoCs@eTeX}
%    \begin{macrocode}
\ifnum64=`\^^^^0040\relax % test for big chars of LuaTeX/XeTeX
  \catcode`\$=9 %
  \catcode`\&=14 %
\else
  \catcode`\$=14 %
  \catcode`\&=9 %
\fi
\def\HoLogoCs@eTeX#1{%
$ #1{\string ^^^^0395}{\string ^^^^03b5}%
& #1{e}{E}%
  TeX%
}%
\catcode`\$=3 %
\catcode`\&=4 %
%    \end{macrocode}
%    \end{macro}
%    \begin{macro}{\HoLogoBkm@eTeX}
%    \begin{macrocode}
\def\HoLogoBkm@eTeX#1{%
  \HOLOGO@PdfdocUnicode{#1{e}{E}}{\textepsilon}%
  -%
  \hologo{TeX}%
}
%    \end{macrocode}
%    \end{macro}
%    \begin{macro}{\HoLogoHtml@eTeX}
%    \begin{macrocode}
\def\HoLogoHtml@eTeX#1{%
  \ltx@mbox{%
    \HOLOGO@MathSetup
    $\varepsilon$%
    -%
    \hologo{TeX}%
  }%
}
%    \end{macrocode}
%    \end{macro}
%
% \subsubsection{\hologo{iniTeX}}
%
%    \begin{macro}{\HoLogo@iniTeX}
%    \begin{macrocode}
\def\HoLogo@iniTeX#1{%
  \HOLOGO@mbox{%
    #1{i}{I}ni\hologo{TeX}%
  }%
}
%    \end{macrocode}
%    \end{macro}
%    \begin{macro}{\HoLogoCs@iniTeX}
%    \begin{macrocode}
\def\HoLogoCs@iniTeX#1{#1{i}{I}niTeX}
%    \end{macrocode}
%    \end{macro}
%    \begin{macro}{\HoLogoBkm@iniTeX}
%    \begin{macrocode}
\def\HoLogoBkm@iniTeX#1{%
  #1{i}{I}ni\hologo{TeX}%
}
%    \end{macrocode}
%    \end{macro}
%    \begin{macro}{\HoLogoHtml@iniTeX}
%    \begin{macrocode}
\let\HoLogoHtml@iniTeX\HoLogo@iniTeX
%    \end{macrocode}
%    \end{macro}
%
% \subsubsection{\hologo{virTeX}}
%
%    \begin{macro}{\HoLogo@virTeX}
%    \begin{macrocode}
\def\HoLogo@virTeX#1{%
  \HOLOGO@mbox{%
    #1{v}{V}ir\hologo{TeX}%
  }%
}
%    \end{macrocode}
%    \end{macro}
%    \begin{macro}{\HoLogoCs@virTeX}
%    \begin{macrocode}
\def\HoLogoCs@virTeX#1{#1{v}{V}irTeX}
%    \end{macrocode}
%    \end{macro}
%    \begin{macro}{\HoLogoBkm@virTeX}
%    \begin{macrocode}
\def\HoLogoBkm@virTeX#1{%
  #1{v}{V}ir\hologo{TeX}%
}
%    \end{macrocode}
%    \end{macro}
%    \begin{macro}{\HoLogoHtml@virTeX}
%    \begin{macrocode}
\let\HoLogoHtml@virTeX\HoLogo@virTeX
%    \end{macrocode}
%    \end{macro}
%
% \subsubsection{\hologo{SliTeX}}
%
% \paragraph{Definitions of the three variants.}
%
%    \begin{macro}{\HoLogo@SLiTeX@lift}
%    \begin{macrocode}
\def\HoLogo@SLiTeX@lift#1{%
  \HoLogoFont@font{SliTeX}{rm}{%
    S%
    \kern-.06em%
    L%
    \kern-.18em%
    \raise.32ex\hbox{\HoLogoFont@font{SliTeX}{sc}{i}}%
    \HOLOGO@discretionary
    \kern-.06em%
    \hologo{TeX}%
  }%
}
%    \end{macrocode}
%    \end{macro}
%    \begin{macro}{\HoLogoBkm@SLiTeX@lift}
%    \begin{macrocode}
\def\HoLogoBkm@SLiTeX@lift#1{SLiTeX}
%    \end{macrocode}
%    \end{macro}
%    \begin{macro}{\HoLogoHtml@SLiTeX@lift}
%    \begin{macrocode}
\def\HoLogoHtml@SLiTeX@lift#1{%
  \HoLogoCss@SLiTeX@lift
  \HOLOGO@Span{SLiTeX-lift}{%
    \HoLogoFont@font{SliTeX}{rm}{%
      S%
      \HOLOGO@Span{L}{L}%
      \HOLOGO@Span{i}{i}%
      \hologo{TeX}%
    }%
  }%
}
%    \end{macrocode}
%    \end{macro}
%    \begin{macro}{\HoLogoCss@SLiTeX@lift}
%    \begin{macrocode}
\def\HoLogoCss@SLiTeX@lift{%
  \Css{%
    span.HoLogo-SLiTeX-lift span.HoLogo-L{%
      margin-left:-.06em;%
      margin-right:-.18em;%
    }%
  }%
  \Css{%
    span.HoLogo-SLiTeX-lift span.HoLogo-i{%
      position:relative;%
      top:-.32ex;%
      margin-right:-.06em;%
      font-variant:small-caps;%
    }%
  }%
  \global\let\HoLogoCss@SLiTeX@lift\relax
}
%    \end{macrocode}
%    \end{macro}
%
%    \begin{macro}{\HoLogo@SliTeX@simple}
%    \begin{macrocode}
\def\HoLogo@SliTeX@simple#1{%
  \HoLogoFont@font{SliTeX}{rm}{%
    \ltx@mbox{%
      \HoLogoFont@font{SliTeX}{sc}{Sli}%
    }%
    \HOLOGO@discretionary
    \hologo{TeX}%
  }%
}
%    \end{macrocode}
%    \end{macro}
%    \begin{macro}{\HoLogoBkm@SliTeX@simple}
%    \begin{macrocode}
\def\HoLogoBkm@SliTeX@simple#1{SliTeX}
%    \end{macrocode}
%    \end{macro}
%    \begin{macro}{\HoLogoHtml@SliTeX@simple}
%    \begin{macrocode}
\let\HoLogoHtml@SliTeX@simple\HoLogo@SliTeX@simple
%    \end{macrocode}
%    \end{macro}
%
%    \begin{macro}{\HoLogo@SliTeX@narrow}
%    \begin{macrocode}
\def\HoLogo@SliTeX@narrow#1{%
  \HoLogoFont@font{SliTeX}{rm}{%
    \ltx@mbox{%
      S%
      \kern-.06em%
      \HoLogoFont@font{SliTeX}{sc}{%
        l%
        \kern-.035em%
        i%
      }%
    }%
    \HOLOGO@discretionary
    \kern-.06em%
    \hologo{TeX}%
  }%
}
%    \end{macrocode}
%    \end{macro}
%    \begin{macro}{\HoLogoBkm@SliTeX@narrow}
%    \begin{macrocode}
\def\HoLogoBkm@SliTeX@narrow#1{SliTeX}
%    \end{macrocode}
%    \end{macro}
%    \begin{macro}{\HoLogoHtml@SliTeX@narrow}
%    \begin{macrocode}
\def\HoLogoHtml@SliTeX@narrow#1{%
  \HoLogoCss@SliTeX@narrow
  \HOLOGO@Span{SliTeX-narrow}{%
    \HoLogoFont@font{SliTeX}{rm}{%
      S%
        \HOLOGO@Span{l}{l}%
        \HOLOGO@Span{i}{i}%
      \hologo{TeX}%
    }%
  }%
}
%    \end{macrocode}
%    \end{macro}
%    \begin{macro}{\HoLogoCss@SliTeX@narrow}
%    \begin{macrocode}
\def\HoLogoCss@SliTeX@narrow{%
  \Css{%
    span.HoLogo-SliTeX-narrow span.HoLogo-l{%
      margin-left:-.06em;%
      margin-right:-.035em;%
      font-variant:small-caps;%
    }%
  }%
  \Css{%
    span.HoLogo-SliTeX-narrow span.HoLogo-i{%
      margin-right:-.06em;%
      font-variant:small-caps;%
    }%
  }%
  \global\let\HoLogoCss@SliTeX@narrow\relax
}
%    \end{macrocode}
%    \end{macro}
%
% \paragraph{Macro set completion.}
%
%    \begin{macro}{\HoLogo@SLiTeX@simple}
%    \begin{macrocode}
\def\HoLogo@SLiTeX@simple{\HoLogo@SliTeX@simple}
%    \end{macrocode}
%    \end{macro}
%    \begin{macro}{\HoLogoBkm@SLiTeX@simple}
%    \begin{macrocode}
\def\HoLogoBkm@SLiTeX@simple{\HoLogoBkm@SliTeX@simple}
%    \end{macrocode}
%    \end{macro}
%    \begin{macro}{\HoLogoHtml@SLiTeX@simple}
%    \begin{macrocode}
\def\HoLogoHtml@SLiTeX@simple{\HoLogoHtml@SliTeX@simple}
%    \end{macrocode}
%    \end{macro}
%
%    \begin{macro}{\HoLogo@SLiTeX@narrow}
%    \begin{macrocode}
\def\HoLogo@SLiTeX@narrow{\HoLogo@SliTeX@narrow}
%    \end{macrocode}
%    \end{macro}
%    \begin{macro}{\HoLogoBkm@SLiTeX@narrow}
%    \begin{macrocode}
\def\HoLogoBkm@SLiTeX@narrow{\HoLogoBkm@SliTeX@narrow}
%    \end{macrocode}
%    \end{macro}
%    \begin{macro}{\HoLogoHtml@SLiTeX@narrow}
%    \begin{macrocode}
\def\HoLogoHtml@SLiTeX@narrow{\HoLogoHtml@SliTeX@narrow}
%    \end{macrocode}
%    \end{macro}
%
%    \begin{macro}{\HoLogo@SliTeX@lift}
%    \begin{macrocode}
\def\HoLogo@SliTeX@lift{\HoLogo@SLiTeX@lift}
%    \end{macrocode}
%    \end{macro}
%    \begin{macro}{\HoLogoBkm@SliTeX@lift}
%    \begin{macrocode}
\def\HoLogoBkm@SliTeX@lift{\HoLogoBkm@SLiTeX@lift}
%    \end{macrocode}
%    \end{macro}
%    \begin{macro}{\HoLogoHtml@SliTeX@lift}
%    \begin{macrocode}
\def\HoLogoHtml@SliTeX@lift{\HoLogoHtml@SLiTeX@lift}
%    \end{macrocode}
%    \end{macro}
%
% \paragraph{Defaults.}
%
%    \begin{macro}{\HoLogo@SLiTeX}
%    \begin{macrocode}
\def\HoLogo@SLiTeX{\HoLogo@SLiTeX@lift}
%    \end{macrocode}
%    \end{macro}
%    \begin{macro}{\HoLogoBkm@SLiTeX}
%    \begin{macrocode}
\def\HoLogoBkm@SLiTeX{\HoLogoBkm@SLiTeX@lift}
%    \end{macrocode}
%    \end{macro}
%    \begin{macro}{\HoLogoHtml@SLiTeX}
%    \begin{macrocode}
\def\HoLogoHtml@SLiTeX{\HoLogoHtml@SLiTeX@lift}
%    \end{macrocode}
%    \end{macro}
%
%    \begin{macro}{\HoLogo@SliTeX}
%    \begin{macrocode}
\def\HoLogo@SliTeX{\HoLogo@SliTeX@narrow}
%    \end{macrocode}
%    \end{macro}
%    \begin{macro}{\HoLogoBkm@SliTeX}
%    \begin{macrocode}
\def\HoLogoBkm@SliTeX{\HoLogoBkm@SliTeX@narrow}
%    \end{macrocode}
%    \end{macro}
%    \begin{macro}{\HoLogoHtml@SliTeX}
%    \begin{macrocode}
\def\HoLogoHtml@SliTeX{\HoLogoHtml@SliTeX@narrow}
%    \end{macrocode}
%    \end{macro}
%
% \subsubsection{\hologo{LuaTeX}}
%
%    \begin{macro}{\HoLogo@LuaTeX}
%    The kerning is an idea of Hans Hagen, see mailing list
%    `luatex at tug dot org' in March 2010.
%    \begin{macrocode}
\def\HoLogo@LuaTeX#1{%
  \HOLOGO@mbox{%
    Lua%
    \HOLOGO@NegativeKerning{aT,oT,To}%
    \hologo{TeX}%
  }%
}
%    \end{macrocode}
%    \end{macro}
%    \begin{macro}{\HoLogoHtml@LuaTeX}
%    \begin{macrocode}
\let\HoLogoHtml@LuaTeX\HoLogo@LuaTeX
%    \end{macrocode}
%    \end{macro}
%
% \subsubsection{\hologo{LuaLaTeX}}
%
%    \begin{macro}{\HoLogo@LuaLaTeX}
%    \begin{macrocode}
\def\HoLogo@LuaLaTeX#1{%
  \HOLOGO@mbox{%
    Lua%
    \hologo{LaTeX}%
  }%
}
%    \end{macrocode}
%    \end{macro}
%    \begin{macro}{\HoLogoHtml@LuaLaTeX}
%    \begin{macrocode}
\let\HoLogoHtml@LuaLaTeX\HoLogo@LuaLaTeX
%    \end{macrocode}
%    \end{macro}
%
% \subsubsection{\hologo{XeTeX}, \hologo{XeLaTeX}}
%
%    \begin{macro}{\HOLOGO@IfCharExists}
%    \begin{macrocode}
\ifluatex
  \ifnum\luatexversion<36 %
  \else
    \def\HOLOGO@IfCharExists#1{%
      \ifnum
        \directlua{%
           if luaotfload and luaotfload.aux then
             if luaotfload.aux.font_has_glyph(%
                    font.current(), \number#1) then % 	 
	       tex.print("1") % 	 
	     end % 	 
	   elseif font and font.fonts and font.current then %
            local f = font.fonts[font.current()]%
            if f.characters and f.characters[\number#1] then %
              tex.print("1")%
            end %
          end%
        }0=\ltx@zero
        \expandafter\ltx@secondoftwo
      \else
        \expandafter\ltx@firstoftwo
      \fi
    }%
  \fi
\fi
\ltx@IfUndefined{HOLOGO@IfCharExists}{%
  \def\HOLOGO@@IfCharExists#1{%
    \begingroup
      \tracinglostchars=\ltx@zero
      \setbox\ltx@zero=\hbox{%
        \kern7sp\char#1\relax
        \ifnum\lastkern>\ltx@zero
          \expandafter\aftergroup\csname iffalse\endcsname
        \else
          \expandafter\aftergroup\csname iftrue\endcsname
        \fi
      }%
      % \if{true|false} from \aftergroup
      \endgroup
      \expandafter\ltx@firstoftwo
    \else
      \endgroup
      \expandafter\ltx@secondoftwo
    \fi
  }%
  \ifxetex
    \ltx@IfUndefined{XeTeXfonttype}{}{%
      \ltx@IfUndefined{XeTeXcharglyph}{}{%
        \def\HOLOGO@IfCharExists#1{%
          \ifnum\XeTeXfonttype\font>\ltx@zero
            \expandafter\ltx@firstofthree
          \else
            \expandafter\ltx@gobble
          \fi
          {%
            \ifnum\XeTeXcharglyph#1>\ltx@zero
              \expandafter\ltx@firstoftwo
            \else
              \expandafter\ltx@secondoftwo
            \fi
          }%
          \HOLOGO@@IfCharExists{#1}%
        }%
      }%
    }%
  \fi
}{}
\ltx@ifundefined{HOLOGO@IfCharExists}{%
  \ifnum64=`\^^^^0040\relax % test for big chars of LuaTeX/XeTeX
    \let\HOLOGO@IfCharExists\HOLOGO@@IfCharExists
  \else
    \def\HOLOGO@IfCharExists#1{%
      \ifnum#1>255 %
        \expandafter\ltx@fourthoffour
      \fi
      \HOLOGO@@IfCharExists{#1}%
    }%
  \fi
}{}
%    \end{macrocode}
%    \end{macro}
%
%    \begin{macro}{\HoLogo@Xe}
%    Source: package \xpackage{dtklogos}
%    \begin{macrocode}
\def\HoLogo@Xe#1{%
  X%
  \kern-.1em\relax
  \HOLOGO@IfCharExists{"018E}{%
    \lower.5ex\hbox{\char"018E}%
  }{%
    \chardef\HOLOGO@choice=\ltx@zero
    \ifdim\fontdimen\ltx@one\font>0pt %
      \ltx@IfUndefined{rotatebox}{%
        \ltx@IfUndefined{pgftext}{%
          \ltx@IfUndefined{psscalebox}{%
            \ltx@IfUndefined{HOLOGO@ScaleBox@\hologoDriver}{%
            }{%
              \chardef\HOLOGO@choice=4 %
            }%
          }{%
            \chardef\HOLOGO@choice=3 %
          }%
        }{%
          \chardef\HOLOGO@choice=2 %
        }%
      }{%
        \chardef\HOLOGO@choice=1 %
      }%
      \ifcase\HOLOGO@choice
        \HOLOGO@WarningUnsupportedDriver{Xe}%
        e%
      \or % 1: \rotatebox
        \begingroup
          \setbox\ltx@zero\hbox{\rotatebox{180}{E}}%
          \ltx@LocDimenA=\dp\ltx@zero
          \advance\ltx@LocDimenA by -.5ex\relax
          \raise\ltx@LocDimenA\box\ltx@zero
        \endgroup
      \or % 2: \pgftext
        \lower.5ex\hbox{%
          \pgfpicture
            \pgftext[rotate=180]{E}%
          \endpgfpicture
        }%
      \or % 3: \psscalebox
        \begingroup
          \setbox\ltx@zero\hbox{\psscalebox{-1 -1}{E}}%
          \ltx@LocDimenA=\dp\ltx@zero
          \advance\ltx@LocDimenA by -.5ex\relax
          \raise\ltx@LocDimenA\box\ltx@zero
        \endgroup
      \or % 4: \HOLOGO@PointReflectBox
        \lower.5ex\hbox{\HOLOGO@PointReflectBox{E}}%
      \else
        \@PackageError{hologo}{Internal error (choice/it}\@ehc
      \fi
    \else
      \ltx@IfUndefined{reflectbox}{%
        \ltx@IfUndefined{pgftext}{%
          \ltx@IfUndefined{psscalebox}{%
            \ltx@IfUndefined{HOLOGO@ScaleBox@\hologoDriver}{%
            }{%
              \chardef\HOLOGO@choice=4 %
            }%
          }{%
            \chardef\HOLOGO@choice=3 %
          }%
        }{%
          \chardef\HOLOGO@choice=2 %
        }%
      }{%
        \chardef\HOLOGO@choice=1 %
      }%
      \ifcase\HOLOGO@choice
        \HOLOGO@WarningUnsupportedDriver{Xe}%
        e%
      \or % 1: reflectbox
        \lower.5ex\hbox{%
          \reflectbox{E}%
        }%
      \or % 2: \pgftext
        \lower.5ex\hbox{%
          \pgfpicture
            \pgftransformxscale{-1}%
            \pgftext{E}%
          \endpgfpicture
        }%
      \or % 3: \psscalebox
        \lower.5ex\hbox{%
          \psscalebox{-1 1}{E}%
        }%
      \or % 4: \HOLOGO@Reflectbox
        \lower.5ex\hbox{%
          \HOLOGO@ReflectBox{E}%
        }%
      \else
        \@PackageError{hologo}{Internal error (choice/up)}\@ehc
      \fi
    \fi
  }%
}
%    \end{macrocode}
%    \end{macro}
%    \begin{macro}{\HoLogoHtml@Xe}
%    \begin{macrocode}
\def\HoLogoHtml@Xe#1{%
  \HoLogoCss@Xe
  \HOLOGO@Span{Xe}{%
    X%
    \HOLOGO@Span{e}{%
      \HCode{&\ltx@hashchar x018e;}%
    }%
  }%
}
%    \end{macrocode}
%    \end{macro}
%    \begin{macro}{\HoLogoCss@Xe}
%    \begin{macrocode}
\def\HoLogoCss@Xe{%
  \Css{%
    span.HoLogo-Xe span.HoLogo-e{%
      position:relative;%
      top:.5ex;%
      left-margin:-.1em;%
    }%
  }%
  \global\let\HoLogoCss@Xe\relax
}
%    \end{macrocode}
%    \end{macro}
%
%    \begin{macro}{\HoLogo@XeTeX}
%    \begin{macrocode}
\def\HoLogo@XeTeX#1{%
  \hologo{Xe}%
  \kern-.15em\relax
  \hologo{TeX}%
}
%    \end{macrocode}
%    \end{macro}
%
%    \begin{macro}{\HoLogoHtml@XeTeX}
%    \begin{macrocode}
\def\HoLogoHtml@XeTeX#1{%
  \HoLogoCss@XeTeX
  \HOLOGO@Span{XeTeX}{%
    \hologo{Xe}%
    \hologo{TeX}%
  }%
}
%    \end{macrocode}
%    \end{macro}
%    \begin{macro}{\HoLogoCss@XeTeX}
%    \begin{macrocode}
\def\HoLogoCss@XeTeX{%
  \Css{%
    span.HoLogo-XeTeX span.HoLogo-TeX{%
      margin-left:-.15em;%
    }%
  }%
  \global\let\HoLogoCss@XeTeX\relax
}
%    \end{macrocode}
%    \end{macro}
%
%    \begin{macro}{\HoLogo@XeLaTeX}
%    \begin{macrocode}
\def\HoLogo@XeLaTeX#1{%
  \hologo{Xe}%
  \kern-.13em%
  \hologo{LaTeX}%
}
%    \end{macrocode}
%    \end{macro}
%    \begin{macro}{\HoLogoHtml@XeLaTeX}
%    \begin{macrocode}
\def\HoLogoHtml@XeLaTeX#1{%
  \HoLogoCss@XeLaTeX
  \HOLOGO@Span{XeLaTeX}{%
    \hologo{Xe}%
    \hologo{LaTeX}%
  }%
}
%    \end{macrocode}
%    \end{macro}
%    \begin{macro}{\HoLogoCss@XeLaTeX}
%    \begin{macrocode}
\def\HoLogoCss@XeLaTeX{%
  \Css{%
    span.HoLogo-XeLaTeX span.HoLogo-Xe{%
      margin-right:-.13em;%
    }%
  }%
  \global\let\HoLogoCss@XeLaTeX\relax
}
%    \end{macrocode}
%    \end{macro}
%
% \subsubsection{\hologo{pdfTeX}, \hologo{pdfLaTeX}}
%
%    \begin{macro}{\HoLogo@pdfTeX}
%    \begin{macrocode}
\def\HoLogo@pdfTeX#1{%
  \HOLOGO@mbox{%
    #1{p}{P}df\hologo{TeX}%
  }%
}
%    \end{macrocode}
%    \end{macro}
%    \begin{macro}{\HoLogoCs@pdfTeX}
%    \begin{macrocode}
\def\HoLogoCs@pdfTeX#1{#1{p}{P}dfTeX}
%    \end{macrocode}
%    \end{macro}
%    \begin{macro}{\HoLogoBkm@pdfTeX}
%    \begin{macrocode}
\def\HoLogoBkm@pdfTeX#1{%
  #1{p}{P}df\hologo{TeX}%
}
%    \end{macrocode}
%    \end{macro}
%    \begin{macro}{\HoLogoHtml@pdfTeX}
%    \begin{macrocode}
\let\HoLogoHtml@pdfTeX\HoLogo@pdfTeX
%    \end{macrocode}
%    \end{macro}
%
%    \begin{macro}{\HoLogo@pdfLaTeX}
%    \begin{macrocode}
\def\HoLogo@pdfLaTeX#1{%
  \HOLOGO@mbox{%
    #1{p}{P}df\hologo{LaTeX}%
  }%
}
%    \end{macrocode}
%    \end{macro}
%    \begin{macro}{\HoLogoCs@pdfLaTeX}
%    \begin{macrocode}
\def\HoLogoCs@pdfLaTeX#1{#1{p}{P}dfLaTeX}
%    \end{macrocode}
%    \end{macro}
%    \begin{macro}{\HoLogoBkm@pdfLaTeX}
%    \begin{macrocode}
\def\HoLogoBkm@pdfLaTeX#1{%
  #1{p}{P}df\hologo{LaTeX}%
}
%    \end{macrocode}
%    \end{macro}
%    \begin{macro}{\HoLogoHtml@pdfLaTeX}
%    \begin{macrocode}
\let\HoLogoHtml@pdfLaTeX\HoLogo@pdfLaTeX
%    \end{macrocode}
%    \end{macro}
%
% \subsubsection{\hologo{VTeX}}
%
%    \begin{macro}{\HoLogo@VTeX}
%    \begin{macrocode}
\def\HoLogo@VTeX#1{%
  \HOLOGO@mbox{%
    V\hologo{TeX}%
  }%
}
%    \end{macrocode}
%    \end{macro}
%    \begin{macro}{\HoLogoHtml@VTeX}
%    \begin{macrocode}
\let\HoLogoHtml@VTeX\HoLogo@VTeX
%    \end{macrocode}
%    \end{macro}
%
% \subsubsection{\hologo{AmS}, \dots}
%
%    Source: class \xclass{amsdtx}
%
%    \begin{macro}{\HoLogo@AmS}
%    \begin{macrocode}
\def\HoLogo@AmS#1{%
  \HoLogoFont@font{AmS}{sy}{%
    A%
    \kern-.1667em%
    \lower.5ex\hbox{M}%
    \kern-.125em%
    S%
  }%
}
%    \end{macrocode}
%    \end{macro}
%    \begin{macro}{\HoLogoBkm@AmS}
%    \begin{macrocode}
\def\HoLogoBkm@AmS#1{AmS}
%    \end{macrocode}
%    \end{macro}
%    \begin{macro}{\HoLogoHtml@AmS}
%    \begin{macrocode}
\def\HoLogoHtml@AmS#1{%
  \HoLogoCss@AmS
%  \HoLogoFont@font{AmS}{sy}{%
    \HOLOGO@Span{AmS}{%
      A%
      \HOLOGO@Span{M}{M}%
      S%
    }%
%   }%
}
%    \end{macrocode}
%    \end{macro}
%    \begin{macro}{\HoLogoCss@AmS}
%    \begin{macrocode}
\def\HoLogoCss@AmS{%
  \Css{%
    span.HoLogo-AmS span.HoLogo-M{%
      position:relative;%
      top:.5ex;%
      margin-left:-.1667em;%
      margin-right:-.125em;%
      text-decoration:none;%
    }%
  }%
  \global\let\HoLogoCss@AmS\relax
}
%    \end{macrocode}
%    \end{macro}
%
%    \begin{macro}{\HoLogo@AmSTeX}
%    \begin{macrocode}
\def\HoLogo@AmSTeX#1{%
  \hologo{AmS}%
  \HOLOGO@hyphen
  \hologo{TeX}%
}
%    \end{macrocode}
%    \end{macro}
%    \begin{macro}{\HoLogoBkm@AmSTeX}
%    \begin{macrocode}
\def\HoLogoBkm@AmSTeX#1{AmS-TeX}%
%    \end{macrocode}
%    \end{macro}
%    \begin{macro}{\HoLogoHtml@AmSTeX}
%    \begin{macrocode}
\let\HoLogoHtml@AmSTeX\HoLogo@AmSTeX
%    \end{macrocode}
%    \end{macro}
%
%    \begin{macro}{\HoLogo@AmSLaTeX}
%    \begin{macrocode}
\def\HoLogo@AmSLaTeX#1{%
  \hologo{AmS}%
  \HOLOGO@hyphen
  \hologo{LaTeX}%
}
%    \end{macrocode}
%    \end{macro}
%    \begin{macro}{\HoLogoBkm@AmSLaTeX}
%    \begin{macrocode}
\def\HoLogoBkm@AmSLaTeX#1{AmS-LaTeX}%
%    \end{macrocode}
%    \end{macro}
%    \begin{macro}{\HoLogoHtml@AmSLaTeX}
%    \begin{macrocode}
\let\HoLogoHtml@AmSLaTeX\HoLogo@AmSLaTeX
%    \end{macrocode}
%    \end{macro}
%
% \subsubsection{\hologo{BibTeX}}
%
%    \begin{macro}{\HoLogo@BibTeX@sc}
%    A definition of \hologo{BibTeX} is provided in
%    the documentation source for the manual of \hologo{BibTeX}
%    \cite{btxdoc}.
%\begin{quote}
%\begin{verbatim}
%\def\BibTeX{%
%  {%
%    \rm
%    B%
%    \kern-.05em%
%    {%
%      \sc
%      i%
%      \kern-.025em %
%      b%
%    }%
%    \kern-.08em
%    T%
%    \kern-.1667em%
%    \lower.7ex\hbox{E}%
%    \kern-.125em%
%    X%
%  }%
%}
%\end{verbatim}
%\end{quote}
%    \begin{macrocode}
\def\HoLogo@BibTeX@sc#1{%
  B%
  \kern-.05em%
  \HoLogoFont@font{BibTeX}{sc}{%
    i%
    \kern-.025em%
    b%
  }%
  \HOLOGO@discretionary
  \kern-.08em%
  \hologo{TeX}%
}
%    \end{macrocode}
%    \end{macro}
%    \begin{macro}{\HoLogoHtml@BibTeX@sc}
%    \begin{macrocode}
\def\HoLogoHtml@BibTeX@sc#1{%
  \HoLogoCss@BibTeX@sc
  \HOLOGO@Span{BibTeX-sc}{%
    B%
    \HOLOGO@Span{i}{i}%
    \HOLOGO@Span{b}{b}%
    \hologo{TeX}%
  }%
}
%    \end{macrocode}
%    \end{macro}
%    \begin{macro}{\HoLogoCss@BibTeX@sc}
%    \begin{macrocode}
\def\HoLogoCss@BibTeX@sc{%
  \Css{%
    span.HoLogo-BibTeX-sc span.HoLogo-i{%
      margin-left:-.05em;%
      margin-right:-.025em;%
      font-variant:small-caps;%
    }%
  }%
  \Css{%
    span.HoLogo-BibTeX-sc span.HoLogo-b{%
      margin-right:-.08em;%
      font-variant:small-caps;%
    }%
  }%
  \global\let\HoLogoCss@BibTeX@sc\relax
}
%    \end{macrocode}
%    \end{macro}
%
%    \begin{macro}{\HoLogo@BibTeX@sf}
%    Variant \xoption{sf} avoids trouble with unavailable
%    small caps fonts (e.g., bold versions of Computer Modern or
%    Latin Modern). The definition is taken from
%    package \xpackage{dtklogos} \cite{dtklogos}.
%\begin{quote}
%\begin{verbatim}
%\DeclareRobustCommand{\BibTeX}{%
%  B%
%  \kern-.05em%
%  \hbox{%
%    $\m@th$% %% force math size calculations
%    \csname S@\f@size\endcsname
%    \fontsize\sf@size\z@
%    \math@fontsfalse
%    \selectfont
%    I%
%    \kern-.025em%
%    B
%  }%
%  \kern-.08em%
%  \-%
%  \TeX
%}
%\end{verbatim}
%\end{quote}
%    \begin{macrocode}
\def\HoLogo@BibTeX@sf#1{%
  B%
  \kern-.05em%
  \HoLogoFont@font{BibTeX}{bibsf}{%
    I%
    \kern-.025em%
    B%
  }%
  \HOLOGO@discretionary
  \kern-.08em%
  \hologo{TeX}%
}
%    \end{macrocode}
%    \end{macro}
%    \begin{macro}{\HoLogoHtml@BibTeX@sf}
%    \begin{macrocode}
\def\HoLogoHtml@BibTeX@sf#1{%
  \HoLogoCss@BibTeX@sf
  \HOLOGO@Span{BibTeX-sf}{%
    B%
    \HoLogoFont@font{BibTeX}{bibsf}{%
      \HOLOGO@Span{i}{I}%
      B%
    }%
    \hologo{TeX}%
  }%
}
%    \end{macrocode}
%    \end{macro}
%    \begin{macro}{\HoLogoCss@BibTeX@sf}
%    \begin{macrocode}
\def\HoLogoCss@BibTeX@sf{%
  \Css{%
    span.HoLogo-BibTeX-sf span.HoLogo-i{%
      margin-left:-.05em;%
      margin-right:-.025em;%
    }%
  }%
  \Css{%
    span.HoLogo-BibTeX-sf span.HoLogo-TeX{%
      margin-left:-.08em;%
    }%
  }%
  \global\let\HoLogoCss@BibTeX@sf\relax
}
%    \end{macrocode}
%    \end{macro}
%
%    \begin{macro}{\HoLogo@BibTeX}
%    \begin{macrocode}
\def\HoLogo@BibTeX{\HoLogo@BibTeX@sf}
%    \end{macrocode}
%    \end{macro}
%    \begin{macro}{\HoLogoHtml@BibTeX}
%    \begin{macrocode}
\def\HoLogoHtml@BibTeX{\HoLogoHtml@BibTeX@sf}
%    \end{macrocode}
%    \end{macro}
%
% \subsubsection{\hologo{BibTeX8}}
%
%    \begin{macro}{\HoLogo@BibTeX8}
%    \begin{macrocode}
\expandafter\def\csname HoLogo@BibTeX8\endcsname#1{%
  \hologo{BibTeX}%
  8%
}
%    \end{macrocode}
%    \end{macro}
%
%    \begin{macro}{\HoLogoBkm@BibTeX8}
%    \begin{macrocode}
\expandafter\def\csname HoLogoBkm@BibTeX8\endcsname#1{%
  \hologo{BibTeX}%
  8%
}
%    \end{macrocode}
%    \end{macro}
%    \begin{macro}{\HoLogoHtml@BibTeX8}
%    \begin{macrocode}
\expandafter
\let\csname HoLogoHtml@BibTeX8\expandafter\endcsname
\csname HoLogo@BibTeX8\endcsname
%    \end{macrocode}
%    \end{macro}
%
% \subsubsection{\hologo{ConTeXt}}
%
%    \begin{macro}{\HoLogo@ConTeXt@simple}
%    \begin{macrocode}
\def\HoLogo@ConTeXt@simple#1{%
  \HOLOGO@mbox{Con}%
  \HOLOGO@discretionary
  \HOLOGO@mbox{\hologo{TeX}t}%
}
%    \end{macrocode}
%    \end{macro}
%    \begin{macro}{\HoLogoHtml@ConTeXt@simple}
%    \begin{macrocode}
\let\HoLogoHtml@ConTeXt@simple\HoLogo@ConTeXt@simple
%    \end{macrocode}
%    \end{macro}
%
%    \begin{macro}{\HoLogo@ConTeXt@narrow}
%    This definition of logo \hologo{ConTeXt} with variant \xoption{narrow}
%    comes from TUGboat's class \xclass{ltugboat} (version 2010/11/15 v2.8).
%    \begin{macrocode}
\def\HoLogo@ConTeXt@narrow#1{%
  \HOLOGO@mbox{C\kern-.0333emon}%
  \HOLOGO@discretionary
  \kern-.0667em%
  \HOLOGO@mbox{\hologo{TeX}\kern-.0333emt}%
}
%    \end{macrocode}
%    \end{macro}
%    \begin{macro}{\HoLogoHtml@ConTeXt@narrow}
%    \begin{macrocode}
\def\HoLogoHtml@ConTeXt@narrow#1{%
  \HoLogoCss@ConTeXt@narrow
  \HOLOGO@Span{ConTeXt-narrow}{%
    \HOLOGO@Span{C}{C}%
    on%
    \hologo{TeX}%
    t%
  }%
}
%    \end{macrocode}
%    \end{macro}
%    \begin{macro}{\HoLogoCss@ConTeXt@narrow}
%    \begin{macrocode}
\def\HoLogoCss@ConTeXt@narrow{%
  \Css{%
    span.HoLogo-ConTeXt-narrow span.HoLogo-C{%
      margin-left:-.0333em;%
    }%
  }%
  \Css{%
    span.HoLogo-ConTeXt-narrow span.HoLogo-TeX{%
      margin-left:-.0667em;%
      margin-right:-.0333em;%
    }%
  }%
  \global\let\HoLogoCss@ConTeXt@narrow\relax
}
%    \end{macrocode}
%    \end{macro}
%
%    \begin{macro}{\HoLogo@ConTeXt}
%    \begin{macrocode}
\def\HoLogo@ConTeXt{\HoLogo@ConTeXt@narrow}
%    \end{macrocode}
%    \end{macro}
%    \begin{macro}{\HoLogoHtml@ConTeXt}
%    \begin{macrocode}
\def\HoLogoHtml@ConTeXt{\HoLogoHtml@ConTeXt@narrow}
%    \end{macrocode}
%    \end{macro}
%
% \subsubsection{\hologo{emTeX}}
%
%    \begin{macro}{\HoLogo@emTeX}
%    \begin{macrocode}
\def\HoLogo@emTeX#1{%
  \HOLOGO@mbox{#1{e}{E}m}%
  \HOLOGO@discretionary
  \hologo{TeX}%
}
%    \end{macrocode}
%    \end{macro}
%    \begin{macro}{\HoLogoCs@emTeX}
%    \begin{macrocode}
\def\HoLogoCs@emTeX#1{#1{e}{E}mTeX}%
%    \end{macrocode}
%    \end{macro}
%    \begin{macro}{\HoLogoBkm@emTeX}
%    \begin{macrocode}
\def\HoLogoBkm@emTeX#1{%
  #1{e}{E}m\hologo{TeX}%
}
%    \end{macrocode}
%    \end{macro}
%    \begin{macro}{\HoLogoHtml@emTeX}
%    \begin{macrocode}
\let\HoLogoHtml@emTeX\HoLogo@emTeX
%    \end{macrocode}
%    \end{macro}
%
% \subsubsection{\hologo{ExTeX}}
%
%    \begin{macro}{\HoLogo@ExTeX}
%    The definition is taken from the FAQ of the
%    project \hologo{ExTeX}
%    \cite{ExTeX-FAQ}.
%\begin{quote}
%\begin{verbatim}
%\def\ExTeX{%
%  \textrm{% Logo always with serifs
%    \ensuremath{%
%      \textstyle
%      \varepsilon_{%
%        \kern-0.15em%
%        \mathcal{X}%
%      }%
%    }%
%    \kern-.15em%
%    \TeX
%  }%
%}
%\end{verbatim}
%\end{quote}
%    \begin{macrocode}
\def\HoLogo@ExTeX#1{%
  \HoLogoFont@font{ExTeX}{rm}{%
    \ltx@mbox{%
      \HOLOGO@MathSetup
      $%
        \textstyle
        \varepsilon_{%
          \kern-0.15em%
          \HoLogoFont@font{ExTeX}{sy}{X}%
        }%
      $%
    }%
    \HOLOGO@discretionary
    \kern-.15em%
    \hologo{TeX}%
  }%
}
%    \end{macrocode}
%    \end{macro}
%    \begin{macro}{\HoLogoHtml@ExTeX}
%    \begin{macrocode}
\def\HoLogoHtml@ExTeX#1{%
  \HoLogoCss@ExTeX
  \HoLogoFont@font{ExTeX}{rm}{%
    \HOLOGO@Span{ExTeX}{%
      \ltx@mbox{%
        \HOLOGO@MathSetup
        $\textstyle\varepsilon$%
        \HOLOGO@Span{X}{$\textstyle\chi$}%
        \hologo{TeX}%
      }%
    }%
  }%
}
%    \end{macrocode}
%    \end{macro}
%    \begin{macro}{\HoLogoBkm@ExTeX}
%    \begin{macrocode}
\def\HoLogoBkm@ExTeX#1{%
  \HOLOGO@PdfdocUnicode{#1{e}{E}x}{\textepsilon\textchi}%
  \hologo{TeX}%
}
%    \end{macrocode}
%    \end{macro}
%    \begin{macro}{\HoLogoCss@ExTeX}
%    \begin{macrocode}
\def\HoLogoCss@ExTeX{%
  \Css{%
    span.HoLogo-ExTeX{%
      font-family:serif;%
    }%
  }%
  \Css{%
    span.HoLogo-ExTeX span.HoLogo-TeX{%
      margin-left:-.15em;%
    }%
  }%
  \global\let\HoLogoCss@ExTeX\relax
}
%    \end{macrocode}
%    \end{macro}
%
% \subsubsection{\hologo{MiKTeX}}
%
%    \begin{macro}{\HoLogo@MiKTeX}
%    \begin{macrocode}
\def\HoLogo@MiKTeX#1{%
  \HOLOGO@mbox{MiK}%
  \HOLOGO@discretionary
  \hologo{TeX}%
}
%    \end{macrocode}
%    \end{macro}
%    \begin{macro}{\HoLogoHtml@MiKTeX}
%    \begin{macrocode}
\let\HoLogoHtml@MiKTeX\HoLogo@MiKTeX
%    \end{macrocode}
%    \end{macro}
%
% \subsubsection{\hologo{OzTeX} and friends}
%
%    Source: \hologo{OzTeX} FAQ \cite{OzTeX}:
%    \begin{quote}
%      |\def\OzTeX{O\kern-.03em z\kern-.15em\TeX}|\\
%      (There is no kerning in OzMF, OzMP and OzTtH.)
%    \end{quote}
%
%    \begin{macro}{\HoLogo@OzTeX}
%    \begin{macrocode}
\def\HoLogo@OzTeX#1{%
  O%
  \kern-.03em %
  z%
  \kern-.15em %
  \hologo{TeX}%
}
%    \end{macrocode}
%    \end{macro}
%    \begin{macro}{\HoLogoHtml@OzTeX}
%    \begin{macrocode}
\def\HoLogoHtml@OzTeX#1{%
  \HoLogoCss@OzTeX
  \HOLOGO@Span{OzTeX}{%
    O%
    \HOLOGO@Span{z}{z}%
    \hologo{TeX}%
  }%
}
%    \end{macrocode}
%    \end{macro}
%    \begin{macro}{\HoLogoCss@OzTeX}
%    \begin{macrocode}
\def\HoLogoCss@OzTeX{%
  \Css{%
    span.HoLogo-OzTeX span.HoLogo-z{%
      margin-left:-.03em;%
      margin-right:-.15em;%
    }%
  }%
  \global\let\HoLogoCss@OzTeX\relax
}
%    \end{macrocode}
%    \end{macro}
%
%    \begin{macro}{\HoLogo@OzMF}
%    \begin{macrocode}
\def\HoLogo@OzMF#1{%
  \HOLOGO@mbox{OzMF}%
}
%    \end{macrocode}
%    \end{macro}
%    \begin{macro}{\HoLogo@OzMP}
%    \begin{macrocode}
\def\HoLogo@OzMP#1{%
  \HOLOGO@mbox{OzMP}%
}
%    \end{macrocode}
%    \end{macro}
%    \begin{macro}{\HoLogo@OzTtH}
%    \begin{macrocode}
\def\HoLogo@OzTtH#1{%
  \HOLOGO@mbox{OzTtH}%
}
%    \end{macrocode}
%    \end{macro}
%
% \subsubsection{\hologo{PCTeX}}
%
%    \begin{macro}{\HoLogo@PCTeX}
%    \begin{macrocode}
\def\HoLogo@PCTeX#1{%
  \HOLOGO@mbox{PC}%
  \hologo{TeX}%
}
%    \end{macrocode}
%    \end{macro}
%    \begin{macro}{\HoLogoHtml@PCTeX}
%    \begin{macrocode}
\let\HoLogoHtml@PCTeX\HoLogo@PCTeX
%    \end{macrocode}
%    \end{macro}
%
% \subsubsection{\hologo{PiCTeX}}
%
%    The original definitions from \xfile{pictex.tex} \cite{PiCTeX}:
%\begin{quote}
%\begin{verbatim}
%\def\PiC{%
%  P%
%  \kern-.12em%
%  \lower.5ex\hbox{I}%
%  \kern-.075em%
%  C%
%}
%\def\PiCTeX{%
%  \PiC
%  \kern-.11em%
%  \TeX
%}
%\end{verbatim}
%\end{quote}
%
%    \begin{macro}{\HoLogo@PiC}
%    \begin{macrocode}
\def\HoLogo@PiC#1{%
  P%
  \kern-.12em%
  \lower.5ex\hbox{I}%
  \kern-.075em%
  C%
  \HOLOGO@SpaceFactor
}
%    \end{macrocode}
%    \end{macro}
%    \begin{macro}{\HoLogoHtml@PiC}
%    \begin{macrocode}
\def\HoLogoHtml@PiC#1{%
  \HoLogoCss@PiC
  \HOLOGO@Span{PiC}{%
    P%
    \HOLOGO@Span{i}{I}%
    C%
  }%
}
%    \end{macrocode}
%    \end{macro}
%    \begin{macro}{\HoLogoCss@PiC}
%    \begin{macrocode}
\def\HoLogoCss@PiC{%
  \Css{%
    span.HoLogo-PiC span.HoLogo-i{%
      position:relative;%
      top:.5ex;%
      margin-left:-.12em;%
      margin-right:-.075em;%
      text-decoration:none;%
    }%
  }%
  \global\let\HoLogoCss@PiC\relax
}
%    \end{macrocode}
%    \end{macro}
%
%    \begin{macro}{\HoLogo@PiCTeX}
%    \begin{macrocode}
\def\HoLogo@PiCTeX#1{%
  \hologo{PiC}%
  \HOLOGO@discretionary
  \kern-.11em%
  \hologo{TeX}%
}
%    \end{macrocode}
%    \end{macro}
%    \begin{macro}{\HoLogoHtml@PiCTeX}
%    \begin{macrocode}
\def\HoLogoHtml@PiCTeX#1{%
  \HoLogoCss@PiCTeX
  \HOLOGO@Span{PiCTeX}{%
    \hologo{PiC}%
    \hologo{TeX}%
  }%
}
%    \end{macrocode}
%    \end{macro}
%    \begin{macro}{\HoLogoCss@PiCTeX}
%    \begin{macrocode}
\def\HoLogoCss@PiCTeX{%
  \Css{%
    span.HoLogo-PiCTeX span.HoLogo-PiC{%
      margin-right:-.11em;%
    }%
  }%
  \global\let\HoLogoCss@PiCTeX\relax
}
%    \end{macrocode}
%    \end{macro}
%
% \subsubsection{\hologo{teTeX}}
%
%    \begin{macro}{\HoLogo@teTeX}
%    \begin{macrocode}
\def\HoLogo@teTeX#1{%
  \HOLOGO@mbox{#1{t}{T}e}%
  \HOLOGO@discretionary
  \hologo{TeX}%
}
%    \end{macrocode}
%    \end{macro}
%    \begin{macro}{\HoLogoCs@teTeX}
%    \begin{macrocode}
\def\HoLogoCs@teTeX#1{#1{t}{T}dfTeX}
%    \end{macrocode}
%    \end{macro}
%    \begin{macro}{\HoLogoBkm@teTeX}
%    \begin{macrocode}
\def\HoLogoBkm@teTeX#1{%
  #1{t}{T}e\hologo{TeX}%
}
%    \end{macrocode}
%    \end{macro}
%    \begin{macro}{\HoLogoHtml@teTeX}
%    \begin{macrocode}
\let\HoLogoHtml@teTeX\HoLogo@teTeX
%    \end{macrocode}
%    \end{macro}
%
% \subsubsection{\hologo{TeX4ht}}
%
%    \begin{macro}{\HoLogo@TeX4ht}
%    \begin{macrocode}
\expandafter\def\csname HoLogo@TeX4ht\endcsname#1{%
  \HOLOGO@mbox{\hologo{TeX}4ht}%
}
%    \end{macrocode}
%    \end{macro}
%    \begin{macro}{\HoLogoHtml@TeX4ht}
%    \begin{macrocode}
\expandafter
\let\csname HoLogoHtml@TeX4ht\expandafter\endcsname
\csname HoLogo@TeX4ht\endcsname
%    \end{macrocode}
%    \end{macro}
%
%
% \subsubsection{\hologo{SageTeX}}
%
%    \begin{macro}{\HoLogo@SageTeX}
%    \begin{macrocode}
\def\HoLogo@SageTeX#1{%
  \HOLOGO@mbox{Sage}%
  \HOLOGO@discretionary
  \HOLOGO@NegativeKerning{eT,oT,To}%
  \hologo{TeX}%
}
%    \end{macrocode}
%    \end{macro}
%    \begin{macro}{\HoLogoHtml@SageTeX}
%    \begin{macrocode}
\let\HoLogoHtml@SageTeX\HoLogo@SageTeX
%    \end{macrocode}
%    \end{macro}
%
% \subsection{\hologo{METAFONT} and friends}
%
%    \begin{macro}{\HoLogo@METAFONT}
%    \begin{macrocode}
\def\HoLogo@METAFONT#1{%
  \HoLogoFont@font{METAFONT}{logo}{%
    \HOLOGO@mbox{META}%
    \HOLOGO@discretionary
    \HOLOGO@mbox{FONT}%
  }%
}
%    \end{macrocode}
%    \end{macro}
%
%    \begin{macro}{\HoLogo@METAPOST}
%    \begin{macrocode}
\def\HoLogo@METAPOST#1{%
  \HoLogoFont@font{METAPOST}{logo}{%
    \HOLOGO@mbox{META}%
    \HOLOGO@discretionary
    \HOLOGO@mbox{POST}%
  }%
}
%    \end{macrocode}
%    \end{macro}
%
%    \begin{macro}{\HoLogo@MetaFun}
%    \begin{macrocode}
\def\HoLogo@MetaFun#1{%
  \HOLOGO@mbox{Meta}%
  \HOLOGO@discretionary
  \HOLOGO@mbox{Fun}%
}
%    \end{macrocode}
%    \end{macro}
%
%    \begin{macro}{\HoLogo@MetaPost}
%    \begin{macrocode}
\def\HoLogo@MetaPost#1{%
  \HOLOGO@mbox{Meta}%
  \HOLOGO@discretionary
  \HOLOGO@mbox{Post}%
}
%    \end{macrocode}
%    \end{macro}
%
% \subsection{Others}
%
% \subsubsection{\hologo{biber}}
%
%    \begin{macro}{\HoLogo@biber}
%    \begin{macrocode}
\def\HoLogo@biber#1{%
  \HOLOGO@mbox{#1{b}{B}i}%
  \HOLOGO@discretionary
  \HOLOGO@mbox{ber}%
}
%    \end{macrocode}
%    \end{macro}
%    \begin{macro}{\HoLogoCs@biber}
%    \begin{macrocode}
\def\HoLogoCs@biber#1{#1{b}{B}iber}
%    \end{macrocode}
%    \end{macro}
%    \begin{macro}{\HoLogoBkm@biber}
%    \begin{macrocode}
\def\HoLogoBkm@biber#1{%
  #1{b}{B}iber%
}
%    \end{macrocode}
%    \end{macro}
%    \begin{macro}{\HoLogoHtml@biber}
%    \begin{macrocode}
\let\HoLogoHtml@biber\HoLogo@biber
%    \end{macrocode}
%    \end{macro}
%
% \subsubsection{\hologo{KOMAScript}}
%
%    \begin{macro}{\HoLogo@KOMAScript}
%    The definition for \hologo{KOMAScript} is taken
%    from \hologo{KOMAScript} (\xfile{scrlogo.dtx}, reformatted) \cite{scrlogo}:
%\begin{quote}
%\begin{verbatim}
%\@ifundefined{KOMAScript}{%
%  \DeclareRobustCommand{\KOMAScript}{%
%    \textsf{%
%      K\kern.05em O\kern.05emM\kern.05em A%
%      \kern.1em-\kern.1em %
%      Script%
%    }%
%  }%
%}{}
%\end{verbatim}
%\end{quote}
%    \begin{macrocode}
\def\HoLogo@KOMAScript#1{%
  \HoLogoFont@font{KOMAScript}{sf}{%
    \HOLOGO@mbox{%
      K\kern.05em%
      O\kern.05em%
      M\kern.05em%
      A%
    }%
    \kern.1em%
    \HOLOGO@hyphen
    \kern.1em%
    \HOLOGO@mbox{Script}%
  }%
}
%    \end{macrocode}
%    \end{macro}
%    \begin{macro}{\HoLogoBkm@KOMAScript}
%    \begin{macrocode}
\def\HoLogoBkm@KOMAScript#1{%
  KOMA-Script%
}
%    \end{macrocode}
%    \end{macro}
%    \begin{macro}{\HoLogoHtml@KOMAScript}
%    \begin{macrocode}
\def\HoLogoHtml@KOMAScript#1{%
  \HoLogoCss@KOMAScript
  \HoLogoFont@font{KOMAScript}{sf}{%
    \HOLOGO@Span{KOMAScript}{%
      K%
      \HOLOGO@Span{O}{O}%
      M%
      \HOLOGO@Span{A}{A}%
      \HOLOGO@Span{hyphen}{-}%
      Script%
    }%
  }%
}
%    \end{macrocode}
%    \end{macro}
%    \begin{macro}{\HoLogoCss@KOMAScript}
%    \begin{macrocode}
\def\HoLogoCss@KOMAScript{%
  \Css{%
    span.HoLogo-KOMAScript{%
      font-family:sans-serif;%
    }%
  }%
  \Css{%
    span.HoLogo-KOMAScript span.HoLogo-O{%
      padding-left:.05em;%
      padding-right:.05em;%
    }%
  }%
  \Css{%
    span.HoLogo-KOMAScript span.HoLogo-A{%
      padding-left:.05em;%
    }%
  }%
  \Css{%
    span.HoLogo-KOMAScript span.HoLogo-hyphen{%
      padding-left:.1em;%
      padding-right:.1em;%
    }%
  }%
  \global\let\HoLogoCss@KOMAScript\relax
}
%    \end{macrocode}
%    \end{macro}
%
% \subsubsection{\hologo{LyX}}
%
%    \begin{macro}{\HoLogo@LyX}
%    The definition is taken from the documentation source files
%    of \hologo{LyX}, \xfile{Intro.lyx} \cite{LyX}:
%\begin{quote}
%\begin{verbatim}
%\def\LyX{%
%  \texorpdfstring{%
%    L\kern-.1667em\lower.25em\hbox{Y}\kern-.125emX\@%
%  }{%
%    LyX%
%  }%
%}
%\end{verbatim}
%\end{quote}
%    \begin{macrocode}
\def\HoLogo@LyX#1{%
  L%
  \kern-.1667em%
  \lower.25em\hbox{Y}%
  \kern-.125em%
  X%
  \HOLOGO@SpaceFactor
}
%    \end{macrocode}
%    \end{macro}
%    \begin{macro}{\HoLogoHtml@LyX}
%    \begin{macrocode}
\def\HoLogoHtml@LyX#1{%
  \HoLogoCss@LyX
  \HOLOGO@Span{LyX}{%
    L%
    \HOLOGO@Span{y}{Y}%
    X%
  }%
}
%    \end{macrocode}
%    \end{macro}
%    \begin{macro}{\HoLogoCss@LyX}
%    \begin{macrocode}
\def\HoLogoCss@LyX{%
  \Css{%
    span.HoLogo-LyX span.HoLogo-y{%
      position:relative;%
      top:.25em;%
      margin-left:-.1667em;%
      margin-right:-.125em;%
      text-decoration:none;%
    }%
  }%
  \global\let\HoLogoCss@LyX\relax
}
%    \end{macrocode}
%    \end{macro}
%
% \subsubsection{\hologo{NTS}}
%
%    \begin{macro}{\HoLogo@NTS}
%    Definition for \hologo{NTS} can be found in
%    package \xpackage{etex\textunderscore man} for the \hologo{eTeX} manual \cite{etexman}
%    and in package \xpackage{dtklogos} \cite{dtklogos}:
%\begin{quote}
%\begin{verbatim}
%\def\NTS{%
%  \leavevmode
%  \hbox{%
%    $%
%      \cal N%
%      \kern-0.35em%
%      \lower0.5ex\hbox{$\cal T$}%
%      \kern-0.2em%
%      S%
%    $%
%  }%
%}
%\end{verbatim}
%\end{quote}
%    \begin{macrocode}
\def\HoLogo@NTS#1{%
  \HoLogoFont@font{NTS}{sy}{%
    N\/%
    \kern-.35em%
    \lower.5ex\hbox{T\/}%
    \kern-.2em%
    S\/%
  }%
  \HOLOGO@SpaceFactor
}
%    \end{macrocode}
%    \end{macro}
%
% \subsubsection{\Hologo{TTH} (\hologo{TeX} to HTML translator)}
%
%    Source: \url{http://hutchinson.belmont.ma.us/tth/}
%    In the HTML source the second `T' is printed as subscript.
%\begin{quote}
%\begin{verbatim}
%T<sub>T</sub>H
%\end{verbatim}
%\end{quote}
%    \begin{macro}{\HoLogo@TTH}
%    \begin{macrocode}
\def\HoLogo@TTH#1{%
  \ltx@mbox{%
    T\HOLOGO@SubScript{T}H%
  }%
  \HOLOGO@SpaceFactor
}
%    \end{macrocode}
%    \end{macro}
%
%    \begin{macro}{\HoLogoHtml@TTH}
%    \begin{macrocode}
\def\HoLogoHtml@TTH#1{%
  T\HCode{<sub>}T\HCode{</sub>}H%
}
%    \end{macrocode}
%    \end{macro}
%
% \subsubsection{\Hologo{HanTheThanh}}
%
%    Partial source: Package \xpackage{dtklogos}.
%    The double accent is U+1EBF (latin small letter e with circumflex
%    and acute).
%    \begin{macro}{\HoLogo@HanTheThanh}
%    \begin{macrocode}
\def\HoLogo@HanTheThanh#1{%
  \ltx@mbox{H\`an}%
  \HOLOGO@space
  \ltx@mbox{%
    Th%
    \HOLOGO@IfCharExists{"1EBF}{%
      \char"1EBF\relax
    }{%
      \^e\hbox to 0pt{\hss\raise .5ex\hbox{\'{}}}%
    }%
  }%
  \HOLOGO@space
  \ltx@mbox{Th\`anh}%
}
%    \end{macrocode}
%    \end{macro}
%    \begin{macro}{\HoLogoBkm@HanTheThanh}
%    \begin{macrocode}
\def\HoLogoBkm@HanTheThanh#1{%
  H\`an %
  Th\HOLOGO@PdfdocUnicode{\^e}{\9036\277} %
  Th\`anh%
}
%    \end{macrocode}
%    \end{macro}
%    \begin{macro}{\HoLogoHtml@HanTheThanh}
%    \begin{macrocode}
\def\HoLogoHtml@HanTheThanh#1{%
  H\`an %
  Th\HCode{&\ltx@hashchar x1ebf;} %
  Th\`anh%
}
%    \end{macrocode}
%    \end{macro}
%
% \subsection{Driver detection}
%
%    \begin{macrocode}
\HOLOGO@IfExists\InputIfFileExists{%
  \InputIfFileExists{hologo.cfg}{}{}%
}{%
  \ltx@IfUndefined{pdf@filesize}{%
    \def\HOLOGO@InputIfExists{%
      \openin\HOLOGO@temp=hologo.cfg\relax
      \ifeof\HOLOGO@temp
        \closein\HOLOGO@temp
      \else
        \closein\HOLOGO@temp
        \begingroup
          \def\x{LaTeX2e}%
        \expandafter\endgroup
        \ifx\fmtname\x
          \input{hologo.cfg}%
        \else
          \input hologo.cfg\relax
        \fi
      \fi
    }%
    \ltx@IfUndefined{newread}{%
      \chardef\HOLOGO@temp=15 %
      \def\HOLOGO@CheckRead{%
        \ifeof\HOLOGO@temp
          \HOLOGO@InputIfExists
        \else
          \ifcase\HOLOGO@temp
            \@PackageWarningNoLine{hologo}{%
              Configuration file ignored, because\MessageBreak
              a free read register could not be found%
            }%
          \else
            \begingroup
              \count\ltx@cclv=\HOLOGO@temp
              \advance\ltx@cclv by \ltx@minusone
              \edef\x{\endgroup
                \chardef\noexpand\HOLOGO@temp=\the\count\ltx@cclv
                \relax
              }%
            \x
          \fi
        \fi
      }%
    }{%
      \csname newread\endcsname\HOLOGO@temp
      \HOLOGO@InputIfExists
    }%
  }{%
    \edef\HOLOGO@temp{\pdf@filesize{hologo.cfg}}%
    \ifx\HOLOGO@temp\ltx@empty
    \else
      \ifnum\HOLOGO@temp>0 %
        \begingroup
          \def\x{LaTeX2e}%
        \expandafter\endgroup
        \ifx\fmtname\x
          \input{hologo.cfg}%
        \else
          \input hologo.cfg\relax
        \fi
      \else
        \@PackageInfoNoLine{hologo}{%
          Empty configuration file `hologo.cfg' ignored%
        }%
      \fi
    \fi
  }%
}
%    \end{macrocode}
%
%    \begin{macrocode}
\def\HOLOGO@temp#1#2{%
  \kv@define@key{HoLogoDriver}{#1}[]{%
    \begingroup
      \def\HOLOGO@temp{##1}%
      \ltx@onelevel@sanitize\HOLOGO@temp
      \ifx\HOLOGO@temp\ltx@empty
      \else
        \@PackageError{hologo}{%
          Value (\HOLOGO@temp) not permitted for option `#1'%
        }%
        \@ehc
      \fi
    \endgroup
    \def\hologoDriver{#2}%
  }%
}%
\def\HOLOGO@@temp#1#2{%
  \ifx\kv@value\relax
    \HOLOGO@temp{#1}{#1}%
  \else
    \HOLOGO@temp{#1}{#2}%
  \fi
}%
\kv@parse@normalized{%
  pdftex,%
  luatex=pdftex,%
  dvipdfm,%
  dvipdfmx=dvipdfm,%
  dvips,%
  dvipsone=dvips,%
  xdvi=dvips,%
  xetex,%
  vtex,%
}\HOLOGO@@temp
%    \end{macrocode}
%
%    \begin{macrocode}
\kv@define@key{HoLogoDriver}{driverfallback}{%
  \def\HOLOGO@DriverFallback{#1}%
}
%    \end{macrocode}
%
%    \begin{macro}{\HOLOGO@DriverFallback}
%    \begin{macrocode}
\def\HOLOGO@DriverFallback{dvips}
%    \end{macrocode}
%    \end{macro}
%
%    \begin{macro}{\hologoDriverSetup}
%    \begin{macrocode}
\def\hologoDriverSetup{%
  \let\hologoDriver\ltx@undefined
  \HOLOGO@DriverSetup
}
%    \end{macrocode}
%    \end{macro}
%
%    \begin{macro}{\HOLOGO@DriverSetup}
%    \begin{macrocode}
\def\HOLOGO@DriverSetup#1{%
  \kvsetkeys{HoLogoDriver}{#1}%
  \HOLOGO@CheckDriver
  \ltx@ifundefined{hologoDriver}{%
    \begingroup
    \edef\x{\endgroup
      \noexpand\kvsetkeys{HoLogoDriver}{\HOLOGO@DriverFallback}%
    }\x
  }{}%
  \@PackageInfoNoLine{hologo}{Using driver `\hologoDriver'}%
}
%    \end{macrocode}
%    \end{macro}
%
%    \begin{macro}{\HOLOGO@CheckDriver}
%    \begin{macrocode}
\def\HOLOGO@CheckDriver{%
  \ifpdf
    \def\hologoDriver{pdftex}%
    \let\HOLOGO@pdfliteral\pdfliteral
    \ifluatex
      \ifx\pdfextension\@undefined\else
        \protected\def\pdfliteral{\pdfextension literal}%
        \let\HOLOGO@pdfliteral\pdfliteral
      \fi
      \ltx@IfUndefined{HOLOGO@pdfliteral}{%
        \ifnum\luatexversion<36 %
        \else
          \begingroup
            \let\HOLOGO@temp\endgroup
            \ifcase0%
                \directlua{%
                  if tex.enableprimitives then %
                    tex.enableprimitives('HOLOGO@', {'pdfliteral'})%
                  else %
                    tex.print('1')%
                  end%
                }%
                \ifx\HOLOGO@pdfliteral\@undefined 1\fi%
                \relax%
              \endgroup
              \let\HOLOGO@temp\relax
              \global\let\HOLOGO@pdfliteral\HOLOGO@pdfliteral
            \fi%
          \HOLOGO@temp
        \fi
      }{}%
    \fi
    \ltx@IfUndefined{HOLOGO@pdfliteral}{%
      \@PackageWarningNoLine{hologo}{%
        Cannot find \string\pdfliteral
      }%
    }{}%
  \else
    \ifxetex
      \def\hologoDriver{xetex}%
    \else
      \ifvtex
        \def\hologoDriver{vtex}%
      \fi
    \fi
  \fi
}
%    \end{macrocode}
%    \end{macro}
%
%    \begin{macro}{\HOLOGO@WarningUnsupportedDriver}
%    \begin{macrocode}
\def\HOLOGO@WarningUnsupportedDriver#1{%
  \@PackageWarningNoLine{hologo}{%
    Logo `#1' needs driver specific macros,\MessageBreak
    but driver `\hologoDriver' is not supported.\MessageBreak
    Use a different driver or\MessageBreak
    load package `graphics' or `pgf'%
  }%
}
%    \end{macrocode}
%    \end{macro}
%
% \subsubsection{Reflect box macros}
%
%    Skip driver part if not needed.
%    \begin{macrocode}
\ltx@IfUndefined{reflectbox}{}{%
  \ltx@IfUndefined{rotatebox}{}{%
    \HOLOGO@AtEnd
  }%
}
\ltx@IfUndefined{pgftext}{}{%
  \HOLOGO@AtEnd
}
\ltx@IfUndefined{psscalebox}{}{%
  \HOLOGO@AtEnd
}
%    \end{macrocode}
%
%    \begin{macrocode}
\def\HOLOGO@temp{LaTeX2e}
\ifx\fmtname\HOLOGO@temp
  \RequirePackage{kvoptions}[2011/06/30]%
  \ProcessKeyvalOptions{HoLogoDriver}%
\fi
\HOLOGO@DriverSetup{}
%    \end{macrocode}
%
%    \begin{macro}{\HOLOGO@ReflectBox}
%    \begin{macrocode}
\def\HOLOGO@ReflectBox#1{%
  \begingroup
    \setbox\ltx@zero\hbox{\begingroup#1\endgroup}%
    \setbox\ltx@two\hbox{%
      \kern\wd\ltx@zero
      \csname HOLOGO@ScaleBox@\hologoDriver\endcsname{-1}{1}{%
        \hbox to 0pt{\copy\ltx@zero\hss}%
      }%
    }%
    \wd\ltx@two=\wd\ltx@zero
    \box\ltx@two
  \endgroup
}
%    \end{macrocode}
%    \end{macro}
%
%    \begin{macro}{\HOLOGO@PointReflectBox}
%    \begin{macrocode}
\def\HOLOGO@PointReflectBox#1{%
  \begingroup
    \setbox\ltx@zero\hbox{\begingroup#1\endgroup}%
    \setbox\ltx@two\hbox{%
      \kern\wd\ltx@zero
      \raise\ht\ltx@zero\hbox{%
        \csname HOLOGO@ScaleBox@\hologoDriver\endcsname{-1}{-1}{%
          \hbox to 0pt{\copy\ltx@zero\hss}%
        }%
      }%
    }%
    \wd\ltx@two=\wd\ltx@zero
    \box\ltx@two
  \endgroup
}
%    \end{macrocode}
%    \end{macro}
%
%    We must define all variants because of dynamic driver setup.
%    \begin{macrocode}
\def\HOLOGO@temp#1#2{#2}
%    \end{macrocode}
%
%    \begin{macro}{\HOLOGO@ScaleBox@pdftex}
%    \begin{macrocode}
\HOLOGO@temp{pdftex}{%
  \def\HOLOGO@ScaleBox@pdftex#1#2#3{%
    \HOLOGO@pdfliteral{%
      q #1 0 0 #2 0 0 cm%
    }%
    #3%
    \HOLOGO@pdfliteral{%
      Q%
    }%
  }%
}
%    \end{macrocode}
%    \end{macro}
%    \begin{macro}{\HOLOGO@ScaleBox@dvips}
%    \begin{macrocode}
\HOLOGO@temp{dvips}{%
  \def\HOLOGO@ScaleBox@dvips#1#2#3{%
    \special{ps:%
      gsave %
      currentpoint %
      currentpoint translate %
      #1 #2 scale %
      neg exch neg exch translate%
    }%
    #3%
    \special{ps:%
      currentpoint %
      grestore %
      moveto%
    }%
  }%
}
%    \end{macrocode}
%    \end{macro}
%    \begin{macro}{\HOLOGO@ScaleBox@dvipdfm}
%    \begin{macrocode}
\HOLOGO@temp{dvipdfm}{%
  \let\HOLOGO@ScaleBox@dvipdfm\HOLOGO@ScaleBox@dvips
}
%    \end{macrocode}
%    \end{macro}
%    Since \hologo{XeTeX} v0.6.
%    \begin{macro}{\HOLOGO@ScaleBox@xetex}
%    \begin{macrocode}
\HOLOGO@temp{xetex}{%
  \def\HOLOGO@ScaleBox@xetex#1#2#3{%
    \special{x:gsave}%
    \special{x:scale #1 #2}%
    #3%
    \special{x:grestore}%
  }%
}
%    \end{macrocode}
%    \end{macro}
%    \begin{macro}{\HOLOGO@ScaleBox@vtex}
%    \begin{macrocode}
\HOLOGO@temp{vtex}{%
  \def\HOLOGO@ScaleBox@vtex#1#2#3{%
    \special{r(#1,0,0,#2,0,0}%
    #3%
    \special{r)}%
  }%
}
%    \end{macrocode}
%    \end{macro}
%
%    \begin{macrocode}
\HOLOGO@AtEnd%
%</package>
%    \end{macrocode}
%
% \section{Test}
%
% \subsection{Catcode checks for loading}
%
%    \begin{macrocode}
%<*test1>
%    \end{macrocode}
%    \begin{macrocode}
\catcode`\{=1 %
\catcode`\}=2 %
\catcode`\#=6 %
\catcode`\@=11 %
\expandafter\ifx\csname count@\endcsname\relax
  \countdef\count@=255 %
\fi
\expandafter\ifx\csname @gobble\endcsname\relax
  \long\def\@gobble#1{}%
\fi
\expandafter\ifx\csname @firstofone\endcsname\relax
  \long\def\@firstofone#1{#1}%
\fi
\expandafter\ifx\csname loop\endcsname\relax
  \expandafter\@firstofone
\else
  \expandafter\@gobble
\fi
{%
  \def\loop#1\repeat{%
    \def\body{#1}%
    \iterate
  }%
  \def\iterate{%
    \body
      \let\next\iterate
    \else
      \let\next\relax
    \fi
    \next
  }%
  \let\repeat=\fi
}%
\def\RestoreCatcodes{}
\count@=0 %
\loop
  \edef\RestoreCatcodes{%
    \RestoreCatcodes
    \catcode\the\count@=\the\catcode\count@\relax
  }%
\ifnum\count@<255 %
  \advance\count@ 1 %
\repeat

\def\RangeCatcodeInvalid#1#2{%
  \count@=#1\relax
  \loop
    \catcode\count@=15 %
  \ifnum\count@<#2\relax
    \advance\count@ 1 %
  \repeat
}
\def\RangeCatcodeCheck#1#2#3{%
  \count@=#1\relax
  \loop
    \ifnum#3=\catcode\count@
    \else
      \errmessage{%
        Character \the\count@\space
        with wrong catcode \the\catcode\count@\space
        instead of \number#3%
      }%
    \fi
  \ifnum\count@<#2\relax
    \advance\count@ 1 %
  \repeat
}
\def\space{ }
\expandafter\ifx\csname LoadCommand\endcsname\relax
  \def\LoadCommand{\input hologo.sty\relax}%
\fi
\def\Test{%
  \RangeCatcodeInvalid{0}{47}%
  \RangeCatcodeInvalid{58}{64}%
  \RangeCatcodeInvalid{91}{96}%
  \RangeCatcodeInvalid{123}{255}%
  \catcode`\@=12 %
  \catcode`\\=0 %
  \catcode`\%=14 %
  \LoadCommand
  \RangeCatcodeCheck{0}{36}{15}%
  \RangeCatcodeCheck{37}{37}{14}%
  \RangeCatcodeCheck{38}{47}{15}%
  \RangeCatcodeCheck{48}{57}{12}%
  \RangeCatcodeCheck{58}{63}{15}%
  \RangeCatcodeCheck{64}{64}{12}%
  \RangeCatcodeCheck{65}{90}{11}%
  \RangeCatcodeCheck{91}{91}{15}%
  \RangeCatcodeCheck{92}{92}{0}%
  \RangeCatcodeCheck{93}{96}{15}%
  \RangeCatcodeCheck{97}{122}{11}%
  \RangeCatcodeCheck{123}{255}{15}%
  \RestoreCatcodes
}
\Test
\csname @@end\endcsname
\end
%    \end{macrocode}
%    \begin{macrocode}
%</test1>
%    \end{macrocode}
%
% \subsection{Spacefactor}
%
%    The space factor must be 1000 after a logo. If it is greater 1000
%    then the following space is a space after a sentence closing point.
%    If the space factor is smaller 1000 then an immediate following
%    dot is interpreted as abbreviation, not sentence closing point.
%
%    \begin{macrocode}
%<*test-spacefactor>
\NeedsTeXFormat{LaTeX2e}
\documentclass{article}
\usepackage{hologo}[2016/05/12]
\usepackage{kvsetkeys}
\usepackage{qstest}
\IncludeTests{*}
\LogTests{log}{*}{*}
\begin{document}
\begin{qstest}{spacefactor}{spacefactor}
\newcommand*{\Test}[1]{%
  \sbox0{%
    \hologo{#1}%
    \Expect*{1000 (#1)}*{\the\spacefactor\space(#1)}%
  }%
}%
\makeatletter
\def\TestList{}
\def\hologoEntry#1#2#3{%
  \edef\TestList{%
    \ifx\TestList\@empty
    \else
      \TestList,%
    \fi
    #1%
    \ifx\\#2\\%
    \else
      ={variant=#2}%
    \fi
  }%
}
\hologoList
\expandafter\kv@parse@normalized\expandafter{%
  \TestList
}{%
  \begingroup
    \let\@logo=\kv@key
    \ifx\kv@value\relax
    \else
      \expandafter\hologoLogoSetup\expandafter\@logo\expandafter{%
        \kv@value
      }%
    \fi
    \Test\@logo
  \endgroup
  \@gobbletwo
}
\end{qstest}
\end{document}
%</test-spacefactor>
%    \end{macrocode}
%
% \subsection{Complete list}
%
%    \begin{macrocode}
%<*test-list>
\NeedsTeXFormat{LaTeX2e}
\documentclass[12pt,a4paper]{article}
\usepackage{hologo}[2016/05/12]
\usepackage[T1]{fontenc}
\usepackage{lmodern}
\usepackage{parskip}
\usepackage[unicode]{hyperref}[2011/09/28]
\usepackage{bookmark}[2011/09/19]
\bookmarksetup{%
  numbered,%
  open,%
  openlevel=2,%
}
\renewcommand*{\contentsname}{List of logos}
\begin{document}
\tableofcontents
\def\TestFont#1#2#3#4#5#6{%
  \begingroup
    \usefont{#3}{#4}{#5}{#6}%
    \HologoVariant{#1}{#2}/\hologoVariant{#1}{#2}%
    \quad
    \begingroup\scriptsize\hologoVariant{#1}{#2}\endgroup
    \quad
  \endgroup
  (#3/#4/#5/#6)%
  \par
}
\makeatletter
\def\hologoEntry#1#2#3{%
  \section{%
    \HologoVariant{#1}{#2}/\hologoVariant{#1}{#2} %
    {[#1\ifx\\#2\\\else\space(#2)\fi]}% hash-ok
  }% braces around [] because of bug in tex4ht
  \begingroup
    \hypersetup{unicode=false}%
    \bookmark[%
      dest=\@currentHref,%
      rellevel=1,%
      keeplevel,%
    ]{%
      \HologoVariant{#1}{#2}/\hologoVariant{#1}{#2} %
      (PDFDocEncoding)%
    }%
  \endgroup
  \TestFont{#1}{#2}{OT1}{cmr}{m}{n}%
  \TestFont{#1}{#2}{OT1}{cmss}{m}{n}%
  \TestFont{#1}{#2}{OT1}{cmr}{b}{n}%
  \TestFont{#1}{#2}{OT1}{cmr}{m}{it}%
  \TestFont{#1}{#2}{OT1}{cmtt}{m}{n}%
  \TestFont{#1}{#2}{T1}{lmr}{m}{n}%
  \TestFont{#1}{#2}{T1}{lmss}{m}{n}%
  \TestFont{#1}{#2}{T1}{lmr}{b}{n}%
  \TestFont{#1}{#2}{T1}{lmr}{m}{it}%
  \TestFont{#1}{#2}{T1}{lmtt}{m}{n}%
  \TestFont{#1}{#2}{T1}{lmvtt}{m}{n}%
  \TestFont{#1}{#2}{T1}{qtm}{m}{n}%
  \TestFont{#1}{#2}{T1}{qhv}{m}{n}%
  \TestFont{#1}{#2}{T1}{qtm}{b}{n}%
  \TestFont{#1}{#2}{T1}{qtm}{m}{it}%
  \TestFont{#1}{#2}{T1}{qcr}{m}{n}%
  \newpage
}
\makeatother
\hologoList
\end{document}
%</test-list>
%    \end{macrocode}
%
% \section{Installation}
%
% \subsection{Download}
%
% \paragraph{Package.} This package is available on
% CTAN\footnote{\url{ftp://ftp.ctan.org/tex-archive/}}:
% \begin{description}
% \item[\CTAN{macros/latex/contrib/oberdiek/hologo.dtx}] The source file.
% \item[\CTAN{macros/latex/contrib/oberdiek/hologo.pdf}] Documentation.
% \end{description}
%
%
% \paragraph{Bundle.} All the packages of the bundle `oberdiek'
% are also available in a TDS compliant ZIP archive. There
% the packages are already unpacked and the documentation files
% are generated. The files and directories obey the TDS standard.
% \begin{description}
% \item[\CTAN{install/macros/latex/contrib/oberdiek.tds.zip}]
% \end{description}
% \emph{TDS} refers to the standard ``A Directory Structure
% for \TeX\ Files'' (\CTAN{tds/tds.pdf}). Directories
% with \xfile{texmf} in their name are usually organized this way.
%
% \subsection{Bundle installation}
%
% \paragraph{Unpacking.} Unpack the \xfile{oberdiek.tds.zip} in the
% TDS tree (also known as \xfile{texmf} tree) of your choice.
% Example (linux):
% \begin{quote}
%   |unzip oberdiek.tds.zip -d ~/texmf|
% \end{quote}
%
% \paragraph{Script installation.}
% Check the directory \xfile{TDS:scripts/oberdiek/} for
% scripts that need further installation steps.
% Package \xpackage{attachfile2} comes with the Perl script
% \xfile{pdfatfi.pl} that should be installed in such a way
% that it can be called as \texttt{pdfatfi}.
% Example (linux):
% \begin{quote}
%   |chmod +x scripts/oberdiek/pdfatfi.pl|\\
%   |cp scripts/oberdiek/pdfatfi.pl /usr/local/bin/|
% \end{quote}
%
% \subsection{Package installation}
%
% \paragraph{Unpacking.} The \xfile{.dtx} file is a self-extracting
% \docstrip\ archive. The files are extracted by running the
% \xfile{.dtx} through \plainTeX:
% \begin{quote}
%   \verb|tex hologo.dtx|
% \end{quote}
%
% \paragraph{TDS.} Now the different files must be moved into
% the different directories in your installation TDS tree
% (also known as \xfile{texmf} tree):
% \begin{quote}
% \def\t{^^A
% \begin{tabular}{@{}>{\ttfamily}l@{ $\rightarrow$ }>{\ttfamily}l@{}}
%   hologo.sty & tex/generic/oberdiek/hologo.sty\\
%   hologo.pdf & doc/latex/oberdiek/hologo.pdf\\
%   example/hologo-example.tex & doc/latex/oberdiek/example/hologo-example.tex\\
%   test/hologo-test1.tex & doc/latex/oberdiek/test/hologo-test1.tex\\
%   test/hologo-test-spacefactor.tex & doc/latex/oberdiek/test/hologo-test-spacefactor.tex\\
%   test/hologo-test-list.tex & doc/latex/oberdiek/test/hologo-test-list.tex\\
%   hologo.dtx & source/latex/oberdiek/hologo.dtx\\
% \end{tabular}^^A
% }^^A
% \sbox0{\t}^^A
% \ifdim\wd0>\linewidth
%   \begingroup
%     \advance\linewidth by\leftmargin
%     \advance\linewidth by\rightmargin
%   \edef\x{\endgroup
%     \def\noexpand\lw{\the\linewidth}^^A
%   }\x
%   \def\lwbox{^^A
%     \leavevmode
%     \hbox to \linewidth{^^A
%       \kern-\leftmargin\relax
%       \hss
%       \usebox0
%       \hss
%       \kern-\rightmargin\relax
%     }^^A
%   }^^A
%   \ifdim\wd0>\lw
%     \sbox0{\small\t}^^A
%     \ifdim\wd0>\linewidth
%       \ifdim\wd0>\lw
%         \sbox0{\footnotesize\t}^^A
%         \ifdim\wd0>\linewidth
%           \ifdim\wd0>\lw
%             \sbox0{\scriptsize\t}^^A
%             \ifdim\wd0>\linewidth
%               \ifdim\wd0>\lw
%                 \sbox0{\tiny\t}^^A
%                 \ifdim\wd0>\linewidth
%                   \lwbox
%                 \else
%                   \usebox0
%                 \fi
%               \else
%                 \lwbox
%               \fi
%             \else
%               \usebox0
%             \fi
%           \else
%             \lwbox
%           \fi
%         \else
%           \usebox0
%         \fi
%       \else
%         \lwbox
%       \fi
%     \else
%       \usebox0
%     \fi
%   \else
%     \lwbox
%   \fi
% \else
%   \usebox0
% \fi
% \end{quote}
% If you have a \xfile{docstrip.cfg} that configures and enables \docstrip's
% TDS installing feature, then some files can already be in the right
% place, see the documentation of \docstrip.
%
% \subsection{Refresh file name databases}
%
% If your \TeX~distribution
% (\teTeX, \mikTeX, \dots) relies on file name databases, you must refresh
% these. For example, \teTeX\ users run \verb|texhash| or
% \verb|mktexlsr|.
%
% \subsection{Some details for the interested}
%
% \paragraph{Attached source.}
%
% The PDF documentation on CTAN also includes the
% \xfile{.dtx} source file. It can be extracted by
% AcrobatReader 6 or higher. Another option is \textsf{pdftk},
% e.g. unpack the file into the current directory:
% \begin{quote}
%   \verb|pdftk hologo.pdf unpack_files output .|
% \end{quote}
%
% \paragraph{Unpacking with \LaTeX.}
% The \xfile{.dtx} chooses its action depending on the format:
% \begin{description}
% \item[\plainTeX:] Run \docstrip\ and extract the files.
% \item[\LaTeX:] Generate the documentation.
% \end{description}
% If you insist on using \LaTeX\ for \docstrip\ (really,
% \docstrip\ does not need \LaTeX), then inform the autodetect routine
% about your intention:
% \begin{quote}
%   \verb|latex \let\install=y\input{hologo.dtx}|
% \end{quote}
% Do not forget to quote the argument according to the demands
% of your shell.
%
% \paragraph{Generating the documentation.}
% You can use both the \xfile{.dtx} or the \xfile{.drv} to generate
% the documentation. The process can be configured by the
% configuration file \xfile{ltxdoc.cfg}. For instance, put this
% line into this file, if you want to have A4 as paper format:
% \begin{quote}
%   \verb|\PassOptionsToClass{a4paper}{article}|
% \end{quote}
% An example follows how to generate the
% documentation with pdf\LaTeX:
% \begin{quote}
%\begin{verbatim}
%pdflatex hologo.dtx
%makeindex -s gind.ist hologo.idx
%pdflatex hologo.dtx
%makeindex -s gind.ist hologo.idx
%pdflatex hologo.dtx
%\end{verbatim}
% \end{quote}
%
% \section{Catalogue}
%
% The following XML file can be used as source for the
% \href{http://mirror.ctan.org/help/Catalogue/catalogue.html}{\TeX\ Catalogue}.
% The elements \texttt{caption} and \texttt{description} are imported
% from the original XML file from the Catalogue.
% The name of the XML file in the Catalogue is \xfile{hologo.xml}.
%    \begin{macrocode}
%<*catalogue>
<?xml version='1.0' encoding='us-ascii'?>
<!DOCTYPE entry SYSTEM 'catalogue.dtd'>
<entry datestamp='$Date$' modifier='$Author$' id='hologo'>
  <name>hologo</name>
  <caption>A collection of logos with bookmark support.</caption>
  <authorref id='auth:oberdiek'/>
  <copyright owner='Heiko Oberdiek' year='2010-2012'/>
  <license type='lppl1.3'/>
  <version number='1.10'/>
  <description>
    The package defines a single command <tt>\hologo</tt>, whose
    argument is the usual case-confused ASCII version of the logo.
    The command is bookmark-enabled, so that every logo becomes
    available in bookmarks without further work.
    <p/>
    The package is part of the <xref refid='oberdiek'>oberdiek</xref>
    bundle.
  </description>
  <documentation details='Package documentation'
      href='ctan:/macros/latex/contrib/oberdiek/hologo.pdf'/>
  <ctan file='true' path='/macros/latex/contrib/oberdiek/hologo.dtx'/>
  <miktex location='oberdiek'/>
  <texlive location='oberdiek'/>
  <install path='/macros/latex/contrib/oberdiek/oberdiek.tds.zip'/>
</entry>
%</catalogue>
%    \end{macrocode}
%
% \begin{thebibliography}{9}
% \raggedright
%
% \bibitem{btxdoc}
% Oren Patashnik,
% \textit{\hologo{BibTeX}ing},
% 1988-02-08.\\
% \CTAN{biblio/bibtex/base/}
%
% \bibitem{dtklogos}
% Gerd Neugebauer, DANTE,
% \textit{Package \xpackage{dtklogos}},
% 2011-04-25.\\
% \CTAN{usergrps/dante/dtk/dtklogos.sty}
%
% \bibitem{etexman}
% The \hologo{NTS} Team,
% \textit{The \hologo{eTeX} manual},
% 1998-02.\\
% \CTAN{systems/e-tex/v2/doc/}
%
% \bibitem{ExTeX-FAQ}
% The \hologo{ExTeX} group,
% \textit{\hologo{ExTeX}: FAQ -- How is \hologo{ExTeX} typeset?},
% 2007-04-14.\\
% \url{http://www.extex.org/documentation/faq.html}
%
% \bibitem{LyX}
% %@MISC{ LyX,
% %  title = {{LyX 2.0.0 -- The Document Processor [Computer software and manual]}},
% %  author = {{The LyX Team}},
% %  howpublished = {Internet: http://www.lyx.org},
% %  year = {2011-05-08},
% %  note = {Retrieved May 10, 2011, from http://www.lyx.org},
% %  url = {http://www.lyx.org/}
% %}
% The \hologo{LyX} Team,
% \textit{\hologo{LyX} -- The Document Processor},
% 2011-05-08.\\
% \url{http://www.lyx.org/}
%
% \bibitem{OzTeX}
% Andrew Trevorrow,
% \hologo{OzTeX} FAQ: What is the correct way to typeset ``\hologo{OzTeX}''?,
% 2011-09-15 (visited).
% \url{http://www.trevorrow.com/oztex/ozfaq.html#oztex-logo}
%
% \bibitem{PiCTeX}
% Michael Wichura,
% \textit{The \hologo{PiCTeX} macro package},
% 1987-09-21.
% \CTAN{graphics/pictex/}
%
% \bibitem{scrlogo}
% Markus Kohm,
% \textit{\hologo{KOMAScript} Datei \xfile{scrlogo.dtx}},
% 2009-01-30.\\
% \CTAN{install/macros/latex/contrib/komascript.tds.zip}
%
% \end{thebibliography}
%
% \begin{History}
%   \begin{Version}{2010/04/08 v1.0}
%   \item
%     The first version.
%   \end{Version}
%   \begin{Version}{2010/04/16 v1.1}
%   \item
%     \cs{Hologo} added for support of logos at start of a sentence.
%   \item
%     \cs{hologoSetup} and \cs{hologoLogoSetup} added.
%   \item
%     Options \xoption{break}, \xoption{hyphenbreak}, \xoption{spacebreak}
%     added.
%   \item
%     Variant support added by option \xoption{variant}.
%   \end{Version}
%   \begin{Version}{2010/04/24 v1.2}
%   \item
%     \hologo{LaTeX3} added.
%   \item
%     \hologo{VTeX} added.
%   \end{Version}
%   \begin{Version}{2010/11/21 v1.3}
%   \item
%     \hologo{iniTeX}, \hologo{virTeX} added.
%   \end{Version}
%   \begin{Version}{2011/03/25 v1.4}
%   \item
%     \hologo{ConTeXt} with variants added.
%   \item
%     Option \xoption{discretionarybreak} added as refinement for
%     option \xoption{break}.
%   \end{Version}
%   \begin{Version}{2011/04/21 v1.5}
%   \item
%     Wrong TDS directory for test files fixed.
%   \end{Version}
%   \begin{Version}{2011/10/01 v1.6}
%   \item
%     Support for package \xpackage{tex4ht} added.
%   \item
%     Support for \cs{csname} added if \cs{ifincsname} is available.
%   \item
%     New logos:
%     \hologo{(La)TeX},
%     \hologo{biber},
%     \hologo{BibTeX} (\xoption{sc}, \xoption{sf}),
%     \hologo{emTeX},
%     \hologo{ExTeX},
%     \hologo{KOMAScript},
%     \hologo{La},
%     \hologo{LyX},
%     \hologo{MiKTeX},
%     \hologo{NTS},
%     \hologo{OzMF},
%     \hologo{OzMP},
%     \hologo{OzTeX},
%     \hologo{OzTtH},
%     \hologo{PCTeX},
%     \hologo{PiC},
%     \hologo{PiCTeX},
%     \hologo{METAFONT},
%     \hologo{MetaFun},
%     \hologo{METAPOST},
%     \hologo{MetaPost},
%     \hologo{SLiTeX} (\xoption{lift}, \xoption{narrow}, \xoption{simple}),
%     \hologo{SliTeX} (\xoption{narrow}, \xoption{simple}, \xoption{lift}),
%     \hologo{teTeX}.
%   \item
%     Fixes:
%     \hologo{iniTeX},
%     \hologo{pdfLaTeX},
%     \hologo{pdfTeX},
%     \hologo{virTeX}.
%   \item
%     \cs{hologoFontSetup} and \cs{hologoLogoFontSetup} added.
%   \item
%     \cs{hologoVariant} and \cs{HologoVariant} added.
%   \end{Version}
%   \begin{Version}{2011/11/22 v1.7}
%   \item
%     New logos:
%     \hologo{BibTeX8},
%     \hologo{LaTeXML},
%     \hologo{SageTeX},
%     \hologo{TeX4ht},
%     \hologo{TTH}.
%   \item
%     \hologo{Xe} and friends: Driver stuff fixed.
%   \item
%     \hologo{Xe} and friends: Support for italic added.
%   \item
%     \hologo{Xe} and friends: Package support for \xpackage{pgf}
%     and \xpackage{pstricks} added.
%   \end{Version}
%   \begin{Version}{2011/11/29 v1.8}
%   \item
%     New logos:
%     \hologo{HanTheThanh}.
%   \end{Version}
%   \begin{Version}{2011/12/21 v1.9}
%   \item
%     Patch for package \xpackage{ifxetex} added for the case that
%     \cs{newif} is undefined in \hologo{iniTeX}.
%   \item
%     Some fixes for \hologo{iniTeX}.
%   \end{Version}
%   \begin{Version}{2012/04/26 v1.10}
%   \item
%     Fix in bookmark version of logo ``\hologo{HanTheThanh}''.
%   \end{Version}
%   \begin{Version}{2016/05/12 v1.11}
%   \item
%     Update HOLOGO@IfCharExists (previously in texlive)
%   \item define pdfliteral in current luatex.
%   \end{Version}
% \end{History}
%
% \PrintIndex
%
% \Finale
\endinput

%        (quote the arguments according to the demands of your shell)
%
% Documentation:
%    (a) If hologo.drv is present:
%           latex hologo.drv
%    (b) Without hologo.drv:
%           latex hologo.dtx; ...
%    The class ltxdoc loads the configuration file ltxdoc.cfg
%    if available. Here you can specify further options, e.g.
%    use A4 as paper format:
%       \PassOptionsToClass{a4paper}{article}
%
%    Programm calls to get the documentation (example):
%       pdflatex hologo.dtx
%       makeindex -s gind.ist hologo.idx
%       pdflatex hologo.dtx
%       makeindex -s gind.ist hologo.idx
%       pdflatex hologo.dtx
%
% Installation:
%    TDS:tex/generic/oberdiek/hologo.sty
%    TDS:doc/latex/oberdiek/hologo.pdf
%    TDS:doc/latex/oberdiek/example/hologo-example.tex
%    TDS:doc/latex/oberdiek/test/hologo-test1.tex
%    TDS:doc/latex/oberdiek/test/hologo-test-spacefactor.tex
%    TDS:doc/latex/oberdiek/test/hologo-test-list.tex
%    TDS:source/latex/oberdiek/hologo.dtx
%
%<*ignore>
\begingroup
  \catcode123=1 %
  \catcode125=2 %
  \def\x{LaTeX2e}%
\expandafter\endgroup
\ifcase 0\ifx\install y1\fi\expandafter
         \ifx\csname processbatchFile\endcsname\relax\else1\fi
         \ifx\fmtname\x\else 1\fi\relax
\else\csname fi\endcsname
%</ignore>
%<*install>
\input docstrip.tex
\Msg{************************************************************************}
\Msg{* Installation}
\Msg{* Package: hologo 2016/05/12 v1.11 A logo collection with bookmark support (HO)}
\Msg{************************************************************************}

\keepsilent
\askforoverwritefalse

\let\MetaPrefix\relax
\preamble

This is a generated file.

Project: hologo
Version: 2016/05/12 v1.11

Copyright (C) 2010-2012 by
   Heiko Oberdiek <heiko.oberdiek at googlemail.com>

This work may be distributed and/or modified under the
conditions of the LaTeX Project Public License, either
version 1.3c of this license or (at your option) any later
version. This version of this license is in
   http://www.latex-project.org/lppl/lppl-1-3c.txt
and the latest version of this license is in
   http://www.latex-project.org/lppl.txt
and version 1.3 or later is part of all distributions of
LaTeX version 2005/12/01 or later.

This work has the LPPL maintenance status "maintained".

This Current Maintainer of this work is Heiko Oberdiek.

The Base Interpreter refers to any `TeX-Format',
because some files are installed in TDS:tex/generic//.

This work consists of the main source file hologo.dtx
and the derived files
   hologo.sty, hologo.pdf, hologo.ins, hologo.drv, hologo-example.tex,
   hologo-test1.tex, hologo-test-spacefactor.tex,
   hologo-test-list.tex.

\endpreamble
\let\MetaPrefix\DoubleperCent

\generate{%
  \file{hologo.ins}{\from{hologo.dtx}{install}}%
  \file{hologo.drv}{\from{hologo.dtx}{driver}}%
  \usedir{tex/generic/oberdiek}%
  \file{hologo.sty}{\from{hologo.dtx}{package}}%
  \usedir{doc/latex/oberdiek/example}%
  \file{hologo-example.tex}{\from{hologo.dtx}{example}}%
  \usedir{doc/latex/oberdiek/test}%
  \file{hologo-test1.tex}{\from{hologo.dtx}{test1}}%
  \file{hologo-test-spacefactor.tex}{\from{hologo.dtx}{test-spacefactor}}%
  \file{hologo-test-list.tex}{\from{hologo.dtx}{test-list}}%
  \nopreamble
  \nopostamble
  \usedir{source/latex/oberdiek/catalogue}%
  \file{hologo.xml}{\from{hologo.dtx}{catalogue}}%
}

\catcode32=13\relax% active space
\let =\space%
\Msg{************************************************************************}
\Msg{*}
\Msg{* To finish the installation you have to move the following}
\Msg{* file into a directory searched by TeX:}
\Msg{*}
\Msg{*     hologo.sty}
\Msg{*}
\Msg{* To produce the documentation run the file `hologo.drv'}
\Msg{* through LaTeX.}
\Msg{*}
\Msg{* Happy TeXing!}
\Msg{*}
\Msg{************************************************************************}

\endbatchfile
%</install>
%<*ignore>
\fi
%</ignore>
%<*driver>
\NeedsTeXFormat{LaTeX2e}
\ProvidesFile{hologo.drv}%
  [2016/05/12 v1.11 A logo collection with bookmark support (HO)]%
\documentclass{ltxdoc}
\usepackage{holtxdoc}[2011/11/22]
\usepackage{hologo}[2016/05/12]
\usepackage{longtable}
\usepackage{array}
\usepackage{paralist}
%\usepackage[T1]{fontenc}
%\usepackage{lmodern}
\begin{document}
  \DocInput{hologo.dtx}%
\end{document}
%</driver>
% \fi
%
%
% \CharacterTable
%  {Upper-case    \A\B\C\D\E\F\G\H\I\J\K\L\M\N\O\P\Q\R\S\T\U\V\W\X\Y\Z
%   Lower-case    \a\b\c\d\e\f\g\h\i\j\k\l\m\n\o\p\q\r\s\t\u\v\w\x\y\z
%   Digits        \0\1\2\3\4\5\6\7\8\9
%   Exclamation   \!     Double quote  \"     Hash (number) \#
%   Dollar        \$     Percent       \%     Ampersand     \&
%   Acute accent  \'     Left paren    \(     Right paren   \)
%   Asterisk      \*     Plus          \+     Comma         \,
%   Minus         \-     Point         \.     Solidus       \/
%   Colon         \:     Semicolon     \;     Less than     \<
%   Equals        \=     Greater than  \>     Question mark \?
%   Commercial at \@     Left bracket  \[     Backslash     \\
%   Right bracket \]     Circumflex    \^     Underscore    \_
%   Grave accent  \`     Left brace    \{     Vertical bar  \|
%   Right brace   \}     Tilde         \~}
%
% \GetFileInfo{hologo.drv}
%
% \title{The \xpackage{hologo} package}
% \date{2016/05/12 v1.11}
% \author{Heiko Oberdiek\\\xemail{heiko.oberdiek at googlemail.com}}
%
% \maketitle
%
% \begin{abstract}
% This package starts a collection of logos with support for bookmarks
% strings.
% \end{abstract}
%
% \tableofcontents
%
% \section{Documentation}
%
% \subsection{Logo macros}
%
% \begin{declcs}{hologo} \M{name}
% \end{declcs}
% Macro \cs{hologo} sets the logo with name \meta{name}.
% The following table shows the supported names.
%
% \begingroup
%   \def\hologoEntry#1#2#3{^^A
%     #1&#2&\hologoLogoSetup{#1}{variant=#2}\hologo{#1}&#3\tabularnewline
%   }
%   \begin{longtable}{>{\ttfamily}l>{\ttfamily}lll}
%     \rmfamily\bfseries{name} & \rmfamily\bfseries variant
%     & \bfseries logo & \bfseries since\\
%     \hline
%     \endhead
%     \hologoList
%   \end{longtable}
% \endgroup
%
% \begin{declcs}{Hologo} \M{name}
% \end{declcs}
% Macro \cs{Hologo} starts the logo \meta{name} with an uppercase
% letter. As an exception small greek letters are not converted
% to uppercase. Examples, see \hologo{eTeX} and \hologo{ExTeX}.
%
% \subsection{Setup macros}
%
% The package does not support package options, but the following
% setup macros can be used to set options.
%
% \begin{declcs}{hologoSetup} \M{key value list}
% \end{declcs}
% Macro \cs{hologoSetup} sets global options.
%
% \begin{declcs}{hologoLogoSetup} \M{logo} \M{key value list}
% \end{declcs}
% Some options can also be used to configure a logo.
% These settings take precedence over global option settings.
%
% \subsection{Options}\label{sec:options}
%
% There are boolean and string options:
% \begin{description}
% \item[Boolean option:]
% It takes |true| or |false|
% as value. If the value is omitted, then |true| is used.
% \item[String option:]
% A value must be given as string. (But the string might be empty.)
% \end{description}
% The following options can be used both in \cs{hologoSetup}
% and \cs{hologoLogoSetup}:
% \begin{description}
% \def\entry#1{\item[\xoption{#1}:]}
% \entry{break}
%   enables or disables line breaks inside the logo. This setting is
%   refined by options \xoption{hyphenbreak}, \xoption{spacebreak}
%   or \xoption{discretionarybreak}.
%   Default is |false|.
% \entry{hyphenbreak}
%   enables or disables the line break right after the hyphen character.
% \entry{spacebreak}
%   enables or disables line breaks at space characters.
% \entry{discretionarybreak}
%   enables or disables line breaks at hyphenation points
%   (inserted by \cs{-}).
% \end{description}
% Macro \cs{hologoLogoSetup} also knows:
% \begin{description}
% \item[\xoption{variant}:]
%   This is a string option. It specifies a variant of a logo that
%   must exist. An empty string selects the package default variant.
% \end{description}
% Example:
% \begin{quote}
%   |\hologoSetup{break=false}|\\
%   |\hologoLogoSetup{plainTeX}{variant=hyphen,hyphenbreak}|\\
%   Then ``plain-\TeX'' contains one break point after the hyphen.
% \end{quote}
%
% \subsection{Driver options}
%
% Sometimes graphical operations are needed to construct some
% glyphs (e.g.\ \hologo{XeTeX}). If package \xpackage{graphics}
% or package \xpackage{pgf} are found, then the macros are taken
% from there. Otherwise the packge defines its own operations
% and therefore needs the driver information. Many drivers are
% detected automatically (\hologo{pdfTeX}/\hologo{LuaTeX}
% in PDF mode, \hologo{XeTeX}, \hologo{VTeX}). These have precedence
% over a driver option. The driver can be given as package option
% or using \cs{hologoDriverSetup}.
% The following list contains the recognized driver options:
% \begin{itemize}
% \item \xoption{pdftex}, \xoption{luatex}
% \item \xoption{dvipdfm}, \xoption{dvipdfmx}
% \item \xoption{dvips}, \xoption{dvipsone}, \xoption{xdvi}
% \item \xoption{xetex}
% \item \xoption{vtex}
% \end{itemize}
% The left driver of a line is the driver name that is used internally.
% The following names are aliases for drivers that use the
% same method. Therefore the entry in the \xext{log} file for
% the used driver prints the internally used driver name.
% \begin{description}
% \item[\xoption{driverfallback}:]
%   This option expects a driver that is used,
%   if the driver could not be detected automatically.
% \end{description}
%
% \begin{declcs}{hologoDriverSetup} \M{driver option}
% \end{declcs}
% The driver can also be configured after package loading
% using \cs{hologoDriverSetup}, also the way for \hologo{plainTeX}
% to setup the driver.
%
% \subsection{Font setup}
%
% Some logos require a special font, but should also be usable by
% \hologo{plainTeX}. Therefore the package provides some ways
% to influence the font settings. The options below
% take font settings as values. Both font commands
% such as \cs{sffamily} and macros that take one argument
% like \cs{textsf} can be used.
%
% \begin{declcs}{hologoFontSetup} \M{key value list}
% \end{declcs}
% Macro \cs{hologoFontSetup} sets the fonts for all logos.
% Supported keys:
% \begin{description}
% \def\entry#1{\item[\xoption{#1}:]}
% \entry{general}
%   This font is used for all logos. The default is empty.
%   That means no special font is used.
% \entry{bibsf}
%   This font is used for
%   {\hologoLogoSetup{BibTeX}{variant=sf}\hologo{BibTeX}}
%   with variant \xoption{sf}.
% \entry{rm}
%   This font is a serif font. It is used for \hologo{ExTeX}.
% \entry{sc}
%   This font specifies a small caps font. It is used for
%   {\hologoLogoSetup{BibTeX}{variant=sc}\hologo{BibTeX}}
%   with variant \xoption{sc}.
% \entry{sf}
%   This font specifies a sans serif font. The default
%   is \cs{sffamily}, then \cs{sf} is tried. Otherwise
%   a warning is given. It is used by \hologo{KOMAScript}.
% \entry{sy}
%   This is the font for math symbols (e.g. cmsy).
%   It is used by \hologo{AmS}, \hologo{NTS}, \hologo{ExTeX}.
% \entry{logo}
%   \hologo{METAFONT} and \hologo{METAPOST} are using that font.
%   In \hologo{LaTeX} \cs{logofamily} is used and
%   the definitions of package \xpackage{mflogo} are used
%   if the package is not loaded.
%   Otherwise the \cs{tenlogo} is used and defined
%   if it does not already exists.
% \end{description}
%
% \begin{declcs}{hologoLogoFontSetup} \M{logo} \M{key value list}
% \end{declcs}
% Fonts can also be set for a logo or logo component separately,
% see the following list.
% The keys are the same as for \cs{hologoFontSetup}.
%
% \begin{longtable}{>{\ttfamily}l>{\sffamily}ll}
%   \meta{logo} & keys & result\\
%   \hline
%   \endhead
%   BibTeX & bibsf & {\hologoLogoSetup{BibTeX}{variant=sf}\hologo{BibTeX}}\\[.5ex]
%   BibTeX & sc & {\hologoLogoSetup{BibTeX}{variant=sc}\hologo{BibTeX}}\\[.5ex]
%   ExTeX & rm & \hologo{ExTeX}\\
%   SliTeX & rm & \hologo{SliTeX}\\[.5ex]
%   AmS & sy & \hologo{AmS}\\
%   ExTeX & sy & \hologo{ExTeX}\\
%   NTS & sy & \hologo{NTS}\\[.5ex]
%   KOMAScript & sf & \hologo{KOMAScript}\\[.5ex]
%   METAFONT & logo & \hologo{METAFONT}\\
%   METAPOST & logo & \hologo{METAPOST}\\[.5ex]
%   SliTeX & sc \hologo{SliTeX}
% \end{longtable}
%
% \subsubsection{Font order}
%
% For all logos the font \xoption{general} is applied first.
% Example:
%\begin{quote}
%|\hologoFontSetup{general=\color{red}}|
%\end{quote}
% will print red logos.
% Then if the font uses a special font \xoption{sf}, for example,
% the font is applied that is setup by \cs{hologoLogoFontSetup}.
% If this font is not setup, then the common font setup
% by \cs{hologoFontSetup} is used. Otherwise a warning is given,
% that there is no font configured.
%
% \subsection{Additional user macros}
%
% Usually a variant of a logo is configured by using
% \cs{hologoLogoSetup}, because it is bad style to mix
% different variants of the same logo in the same text.
% There the following macros are a convenience for testing.
%
% \begin{declcs}{hologoVariant} \M{name} \M{variant}\\
%   \cs{HologoVariant} \M{name} \M{variant}
% \end{declcs}
% Logo \meta{name} is set using \meta{variant} that specifies
% explicitely which variant of the macro is used. If the argument
% is empty, then the default form of the logo is used
% (configurable by \cs{hologoLogoSetup}).
%
% \cs{HologoVariant} is used if the logo is set in a context
% that needs an uppercase first letter (beginning of a sentence, \dots).
%
% \begin{declcs}{hologoList}\\
%   \cs{hologoEntry} \M{logo} \M{variant} \M{since}
% \end{declcs}
% Macro \cs{hologoList} contains all logos that are provided
% by the package including variants. The list consists of calls
% of \cs{hologoEntry} with three arguments starting with the
% logo name \meta{logo} and its variant \meta{variant}. An empty
% variant means the current default. Argument \meta{since} specifies
% with version of the package \xpackage{hologo} is needed to get
% the logo. If the logo is fixed, then the date gets updated.
% Therefore the date \meta{since} is not exactly the date of
% the first introduction, but rather the date of the latest fix.
%
% Before \cs{hologoList} can be used, macro \cs{hologoEntry} needs
% a definition. The example file in section \ref{sec:example}
% shows applications of \cs{hologoList}.
%
% \subsection{Supported contexts}
%
% Macros \cs{hologo} and friends support special contexts:
% \begin{itemize}
% \item \hologo{LaTeX}'s protection mechanism.
% \item Bookmarks of package \xpackage{hyperref}.
% \item Package \xpackage{tex4ht}.
% \item The macros can be used inside \cs{csname} constructs,
%   if \cs{ifincsname} is available (\hologo{pdfTeX}, \hologo{XeTeX},
%   \hologo{LuaTeX}).
% \end{itemize}
%
% \subsection{Example}
% \label{sec:example}
%
% The following example prints the logos in different fonts.
%    \begin{macrocode}
%<*example>
%<<verbatim
\NeedsTeXFormat{LaTeX2e}
\documentclass[a4paper]{article}
\usepackage[
  hmargin=20mm,
  vmargin=20mm,
]{geometry}
\pagestyle{empty}
\usepackage{hologo}[2016/05/12]
\usepackage{longtable}
\usepackage{array}
\setlength{\extrarowheight}{2pt}
\usepackage[T1]{fontenc}
\usepackage{lmodern}
\usepackage{pdflscape}
\usepackage[
  pdfencoding=auto,
]{hyperref}
\hypersetup{
  pdfauthor={Heiko Oberdiek},
  pdftitle={Example for package `hologo'},
  pdfsubject={Logos with fonts lmr, lmss, qtm, qpl, qhv},
}
\usepackage{bookmark}

% Print the logo list on the console

\begingroup
  \typeout{}%
  \typeout{*** Begin of logo list ***}%
  \newcommand*{\hologoEntry}[3]{%
    \typeout{#1 \ifx\\#2\\\else(#2) \fi[#3]}%
  }%
  \hologoList
  \typeout{*** End of logo list ***}%
  \typeout{}%
\endgroup

\begin{document}
\begin{landscape}

  \section{Example file for package `hologo'}

  % Table for font names

  \begin{longtable}{>{\bfseries}ll}
    \textbf{font} & \textbf{Font name}\\
    \hline
    lmr & Latin Modern Roman\\
    lmss & Latin Modern Sans\\
    qtm & \TeX\ Gyre Termes\\
    qhv & \TeX\ Gyre Heros\\
    qpl & \TeX\ Gyre Pagella\\
  \end{longtable}

  % Logo list with logos in different fonts

  \begingroup
    \newcommand*{\SetVariant}[2]{%
      \ifx\\#2\\%
      \else
        \hologoLogoSetup{#1}{variant=#2}%
      \fi
    }%
    \newcommand*{\hologoEntry}[3]{%
      \SetVariant{#1}{#2}%
      \raisebox{1em}[0pt][0pt]{\hypertarget{#1@#2}{}}%
      \bookmark[%
        dest={#1@#2},%
      ]{%
        #1\ifx\\#2\\\else\space(#2)\fi: \Hologo{#1}, \hologo{#1} %
        [Unicode]%
      }%
      \hypersetup{unicode=false}%
      \bookmark[%
        dest={#1@#2},%
      ]{%
        #1\ifx\\#2\\\else\space(#2)\fi: \Hologo{#1}, \hologo{#1} %
        [PDFDocEncoding]%
      }%
      \texttt{#1}%
      &%
      \texttt{#2}%
      &%
      \Hologo{#1}%
      &%
      \SetVariant{#1}{#2}%
      \hologo{#1}%
      &%
      \SetVariant{#1}{#2}%
      \fontfamily{qtm}\selectfont
      \hologo{#1}%
      &%
      \SetVariant{#1}{#2}%
      \fontfamily{qpl}\selectfont
      \hologo{#1}%
      &%
      \SetVariant{#1}{#2}%
      \textsf{\hologo{#1}}%
      &%
      \SetVariant{#1}{#2}%
      \fontfamily{qhv}\selectfont
      \hologo{#1}%
      \tabularnewline
    }%
    \begin{longtable}{llllllll}%
      \textbf{\textit{logo}} & \textbf{\textit{variant}} &
      \texttt{\string\Hologo} &
      \textbf{lmr} & \textbf{qtm} & \textbf{qpl} &
      \textbf{lmss} & \textbf{qhv}
      \tabularnewline
      \hline
      \endhead
      \hologoList
    \end{longtable}%
  \endgroup

\end{landscape}
\end{document}
%verbatim
%</example>
%    \end{macrocode}
%
% \StopEventually{
% }
%
% \section{Implementation}
%    \begin{macrocode}
%<*package>
%    \end{macrocode}
%    Reload check, especially if the package is not used with \LaTeX.
%    \begin{macrocode}
\begingroup\catcode61\catcode48\catcode32=10\relax%
  \catcode13=5 % ^^M
  \endlinechar=13 %
  \catcode35=6 % #
  \catcode39=12 % '
  \catcode44=12 % ,
  \catcode45=12 % -
  \catcode46=12 % .
  \catcode58=12 % :
  \catcode64=11 % @
  \catcode123=1 % {
  \catcode125=2 % }
  \expandafter\let\expandafter\x\csname ver@hologo.sty\endcsname
  \ifx\x\relax % plain-TeX, first loading
  \else
    \def\empty{}%
    \ifx\x\empty % LaTeX, first loading,
      % variable is initialized, but \ProvidesPackage not yet seen
    \else
      \expandafter\ifx\csname PackageInfo\endcsname\relax
        \def\x#1#2{%
          \immediate\write-1{Package #1 Info: #2.}%
        }%
      \else
        \def\x#1#2{\PackageInfo{#1}{#2, stopped}}%
      \fi
      \x{hologo}{The package is already loaded}%
      \aftergroup\endinput
    \fi
  \fi
\endgroup%
%    \end{macrocode}
%    Package identification:
%    \begin{macrocode}
\begingroup\catcode61\catcode48\catcode32=10\relax%
  \catcode13=5 % ^^M
  \endlinechar=13 %
  \catcode35=6 % #
  \catcode39=12 % '
  \catcode40=12 % (
  \catcode41=12 % )
  \catcode44=12 % ,
  \catcode45=12 % -
  \catcode46=12 % .
  \catcode47=12 % /
  \catcode58=12 % :
  \catcode64=11 % @
  \catcode91=12 % [
  \catcode93=12 % ]
  \catcode123=1 % {
  \catcode125=2 % }
  \expandafter\ifx\csname ProvidesPackage\endcsname\relax
    \def\x#1#2#3[#4]{\endgroup
      \immediate\write-1{Package: #3 #4}%
      \xdef#1{#4}%
    }%
  \else
    \def\x#1#2[#3]{\endgroup
      #2[{#3}]%
      \ifx#1\@undefined
        \xdef#1{#3}%
      \fi
      \ifx#1\relax
        \xdef#1{#3}%
      \fi
    }%
  \fi
\expandafter\x\csname ver@hologo.sty\endcsname
\ProvidesPackage{hologo}%
  [2016/05/12 v1.11 A logo collection with bookmark support (HO)]%
%    \end{macrocode}
%
%    \begin{macrocode}
\begingroup\catcode61\catcode48\catcode32=10\relax%
  \catcode13=5 % ^^M
  \endlinechar=13 %
  \catcode123=1 % {
  \catcode125=2 % }
  \catcode64=11 % @
  \def\x{\endgroup
    \expandafter\edef\csname HOLOGO@AtEnd\endcsname{%
      \endlinechar=\the\endlinechar\relax
      \catcode13=\the\catcode13\relax
      \catcode32=\the\catcode32\relax
      \catcode35=\the\catcode35\relax
      \catcode61=\the\catcode61\relax
      \catcode64=\the\catcode64\relax
      \catcode123=\the\catcode123\relax
      \catcode125=\the\catcode125\relax
    }%
  }%
\x\catcode61\catcode48\catcode32=10\relax%
\catcode13=5 % ^^M
\endlinechar=13 %
\catcode35=6 % #
\catcode64=11 % @
\catcode123=1 % {
\catcode125=2 % }
\def\TMP@EnsureCode#1#2{%
  \edef\HOLOGO@AtEnd{%
    \HOLOGO@AtEnd
    \catcode#1=\the\catcode#1\relax
  }%
  \catcode#1=#2\relax
}
\TMP@EnsureCode{10}{12}% ^^J
\TMP@EnsureCode{33}{12}% !
\TMP@EnsureCode{34}{12}% "
\TMP@EnsureCode{36}{3}% $
\TMP@EnsureCode{38}{4}% &
\TMP@EnsureCode{39}{12}% '
\TMP@EnsureCode{40}{12}% (
\TMP@EnsureCode{41}{12}% )
\TMP@EnsureCode{42}{12}% *
\TMP@EnsureCode{43}{12}% +
\TMP@EnsureCode{44}{12}% ,
\TMP@EnsureCode{45}{12}% -
\TMP@EnsureCode{46}{12}% .
\TMP@EnsureCode{47}{12}% /
\TMP@EnsureCode{58}{12}% :
\TMP@EnsureCode{59}{12}% ;
\TMP@EnsureCode{60}{12}% <
\TMP@EnsureCode{62}{12}% >
\TMP@EnsureCode{63}{12}% ?
\TMP@EnsureCode{91}{12}% [
\TMP@EnsureCode{93}{12}% ]
\TMP@EnsureCode{94}{7}% ^ (superscript)
\TMP@EnsureCode{95}{8}% _ (subscript)
\TMP@EnsureCode{96}{12}% `
\TMP@EnsureCode{124}{12}% |
\edef\HOLOGO@AtEnd{%
  \HOLOGO@AtEnd
  \escapechar\the\escapechar\relax
  \noexpand\endinput
}
\escapechar=92 %
%    \end{macrocode}
%
% \subsection{Logo list}
%
%    \begin{macro}{\hologoList}
%    \begin{macrocode}
\def\hologoList{%
  \hologoEntry{(La)TeX}{}{2011/10/01}%
  \hologoEntry{AmSLaTeX}{}{2010/04/16}%
  \hologoEntry{AmSTeX}{}{2010/04/16}%
  \hologoEntry{biber}{}{2011/10/01}%
  \hologoEntry{BibTeX}{}{2011/10/01}%
  \hologoEntry{BibTeX}{sf}{2011/10/01}%
  \hologoEntry{BibTeX}{sc}{2011/10/01}%
  \hologoEntry{BibTeX8}{}{2011/11/22}%
  \hologoEntry{ConTeXt}{}{2011/03/25}%
  \hologoEntry{ConTeXt}{narrow}{2011/03/25}%
  \hologoEntry{ConTeXt}{simple}{2011/03/25}%
  \hologoEntry{emTeX}{}{2010/04/26}%
  \hologoEntry{eTeX}{}{2010/04/08}%
  \hologoEntry{ExTeX}{}{2011/10/01}%
  \hologoEntry{HanTheThanh}{}{2011/11/29}%
  \hologoEntry{iniTeX}{}{2011/10/01}%
  \hologoEntry{KOMAScript}{}{2011/10/01}%
  \hologoEntry{La}{}{2010/05/08}%
  \hologoEntry{LaTeX}{}{2010/04/08}%
  \hologoEntry{LaTeX2e}{}{2010/04/08}%
  \hologoEntry{LaTeX3}{}{2010/04/24}%
  \hologoEntry{LaTeXe}{}{2010/04/08}%
  \hologoEntry{LaTeXML}{}{2011/11/22}%
  \hologoEntry{LaTeXTeX}{}{2011/10/01}%
  \hologoEntry{LuaLaTeX}{}{2010/04/08}%
  \hologoEntry{LuaTeX}{}{2010/04/08}%
  \hologoEntry{LyX}{}{2011/10/01}%
  \hologoEntry{METAFONT}{}{2011/10/01}%
  \hologoEntry{MetaFun}{}{2011/10/01}%
  \hologoEntry{METAPOST}{}{2011/10/01}%
  \hologoEntry{MetaPost}{}{2011/10/01}%
  \hologoEntry{MiKTeX}{}{2011/10/01}%
  \hologoEntry{NTS}{}{2011/10/01}%
  \hologoEntry{OzMF}{}{2011/10/01}%
  \hologoEntry{OzMP}{}{2011/10/01}%
  \hologoEntry{OzTeX}{}{2011/10/01}%
  \hologoEntry{OzTtH}{}{2011/10/01}%
  \hologoEntry{PCTeX}{}{2011/10/01}%
  \hologoEntry{pdfTeX}{}{2011/10/01}%
  \hologoEntry{pdfLaTeX}{}{2011/10/01}%
  \hologoEntry{PiC}{}{2011/10/01}%
  \hologoEntry{PiCTeX}{}{2011/10/01}%
  \hologoEntry{plainTeX}{}{2010/04/08}%
  \hologoEntry{plainTeX}{space}{2010/04/16}%
  \hologoEntry{plainTeX}{hyphen}{2010/04/16}%
  \hologoEntry{plainTeX}{runtogether}{2010/04/16}%
  \hologoEntry{SageTeX}{}{2011/11/22}%
  \hologoEntry{SLiTeX}{}{2011/10/01}%
  \hologoEntry{SLiTeX}{lift}{2011/10/01}%
  \hologoEntry{SLiTeX}{narrow}{2011/10/01}%
  \hologoEntry{SLiTeX}{simple}{2011/10/01}%
  \hologoEntry{SliTeX}{}{2011/10/01}%
  \hologoEntry{SliTeX}{narrow}{2011/10/01}%
  \hologoEntry{SliTeX}{simple}{2011/10/01}%
  \hologoEntry{SliTeX}{lift}{2011/10/01}%
  \hologoEntry{teTeX}{}{2011/10/01}%
  \hologoEntry{TeX}{}{2010/04/08}%
  \hologoEntry{TeX4ht}{}{2011/11/22}%
  \hologoEntry{TTH}{}{2011/11/22}%
  \hologoEntry{virTeX}{}{2011/10/01}%
  \hologoEntry{VTeX}{}{2010/04/24}%
  \hologoEntry{Xe}{}{2010/04/08}%
  \hologoEntry{XeLaTeX}{}{2010/04/08}%
  \hologoEntry{XeTeX}{}{2010/04/08}%
}
%    \end{macrocode}
%    \end{macro}
%
% \subsection{Load resources}
%
%    \begin{macrocode}
\begingroup\expandafter\expandafter\expandafter\endgroup
\expandafter\ifx\csname RequirePackage\endcsname\relax
  \def\TMP@RequirePackage#1[#2]{%
    \begingroup\expandafter\expandafter\expandafter\endgroup
    \expandafter\ifx\csname ver@#1.sty\endcsname\relax
      \input #1.sty\relax
    \fi
  }%
  \TMP@RequirePackage{ltxcmds}[2011/02/04]%
  \TMP@RequirePackage{infwarerr}[2010/04/08]%
  \TMP@RequirePackage{kvsetkeys}[2010/03/01]%
  \TMP@RequirePackage{kvdefinekeys}[2010/03/01]%
  \TMP@RequirePackage{pdftexcmds}[2010/04/01]%
  \TMP@RequirePackage{ifpdf}[2010/01/28]%
  \TMP@RequirePackage{ifluatex}[2010/03/01]%
  \ltx@IfUndefined{newif}{%
    \expandafter\let\csname newif\endcsname\ltx@newif
  }{}%
  \TMP@RequirePackage{ifxetex}[2009/01/23]%
  \TMP@RequirePackage{ifvtex}[2010/03/01]%
\else
  \RequirePackage{ltxcmds}[2011/02/04]%
  \RequirePackage{infwarerr}[2010/04/08]%
  \RequirePackage{kvsetkeys}[2010/03/01]%
  \RequirePackage{kvdefinekeys}[2010/03/01]%
  \RequirePackage{pdftexcmds}[2010/04/01]%
  \RequirePackage{ifpdf}[2010/01/28]%
  \RequirePackage{ifluatex}[2010/03/01]%
  \RequirePackage{ifxetex}[2009/01/23]%
  \RequirePackage{ifvtex}[2010/03/01]%
\fi
%    \end{macrocode}
%
%    \begin{macro}{\HOLOGO@IfDefined}
%    \begin{macrocode}
\def\HOLOGO@IfExists#1{%
  \ifx\@undefined#1%
    \expandafter\ltx@secondoftwo
  \else
    \ifx\relax#1%
      \expandafter\ltx@secondoftwo
    \else
      \expandafter\expandafter\expandafter\ltx@firstoftwo
    \fi
  \fi
}
%    \end{macrocode}
%    \end{macro}
%
% \subsection{Setup macros}
%
%    \begin{macro}{\hologoSetup}
%    \begin{macrocode}
\def\hologoSetup{%
  \let\HOLOGO@name\relax
  \HOLOGO@Setup
}
%    \end{macrocode}
%    \end{macro}
%
%    \begin{macro}{\hologoLogoSetup}
%    \begin{macrocode}
\def\hologoLogoSetup#1{%
  \edef\HOLOGO@name{#1}%
  \ltx@IfUndefined{HoLogo@\HOLOGO@name}{%
    \@PackageError{hologo}{%
      Unknown logo `\HOLOGO@name'%
    }\@ehc
    \ltx@gobble
  }{%
    \HOLOGO@Setup
  }%
}
%    \end{macrocode}
%    \end{macro}
%
%    \begin{macro}{\HOLOGO@Setup}
%    \begin{macrocode}
\def\HOLOGO@Setup{%
  \kvsetkeys{HoLogo}%
}
%    \end{macrocode}
%    \end{macro}
%
% \subsection{Options}
%
%    \begin{macro}{\HOLOGO@DeclareBoolOption}
%    \begin{macrocode}
\def\HOLOGO@DeclareBoolOption#1{%
  \expandafter\chardef\csname HOLOGOOPT@#1\endcsname\ltx@zero
  \kv@define@key{HoLogo}{#1}[true]{%
    \def\HOLOGO@temp{##1}%
    \ifx\HOLOGO@temp\HOLOGO@true
      \ifx\HOLOGO@name\relax
        \expandafter\chardef\csname HOLOGOOPT@#1\endcsname=\ltx@one
      \else
        \expandafter\chardef\csname
        HoLogoOpt@#1@\HOLOGO@name\endcsname\ltx@one
      \fi
      \HOLOGO@SetBreakAll{#1}%
    \else
      \ifx\HOLOGO@temp\HOLOGO@false
        \ifx\HOLOGO@name\relax
          \expandafter\chardef\csname HOLOGOOPT@#1\endcsname=\ltx@zero
        \else
          \expandafter\chardef\csname
          HoLogoOpt@#1@\HOLOGO@name\endcsname=\ltx@zero
        \fi
        \HOLOGO@SetBreakAll{#1}%
      \else
        \@PackageError{hologo}{%
          Unknown value `##1' for boolean option `#1'.\MessageBreak
          Known values are `true' and `false'%
        }\@ehc
      \fi
    \fi
  }%
}
%    \end{macrocode}
%    \end{macro}
%
%    \begin{macro}{\HOLOGO@SetBreakAll}
%    \begin{macrocode}
\def\HOLOGO@SetBreakAll#1{%
  \def\HOLOGO@temp{#1}%
  \ifx\HOLOGO@temp\HOLOGO@break
    \ifx\HOLOGO@name\relax
      \chardef\HOLOGOOPT@hyphenbreak=\HOLOGOOPT@break
      \chardef\HOLOGOOPT@spacebreak=\HOLOGOOPT@break
      \chardef\HOLOGOOPT@discretionarybreak=\HOLOGOOPT@break
    \else
      \expandafter\chardef
         \csname HoLogoOpt@hyphenbreak@\HOLOGO@name\endcsname=%
         \csname HoLogoOpt@break@\HOLOGO@name\endcsname
      \expandafter\chardef
         \csname HoLogoOpt@spacebreak@\HOLOGO@name\endcsname=%
         \csname HoLogoOpt@break@\HOLOGO@name\endcsname
      \expandafter\chardef
         \csname HoLogoOpt@discretionarybreak@\HOLOGO@name
             \endcsname=%
         \csname HoLogoOpt@break@\HOLOGO@name\endcsname
    \fi
  \fi
}
%    \end{macrocode}
%    \end{macro}
%
%    \begin{macro}{\HOLOGO@true}
%    \begin{macrocode}
\def\HOLOGO@true{true}
%    \end{macrocode}
%    \end{macro}
%    \begin{macro}{\HOLOGO@false}
%    \begin{macrocode}
\def\HOLOGO@false{false}
%    \end{macrocode}
%    \end{macro}
%    \begin{macro}{\HOLOGO@break}
%    \begin{macrocode}
\def\HOLOGO@break{break}
%    \end{macrocode}
%    \end{macro}
%
%    \begin{macrocode}
\HOLOGO@DeclareBoolOption{break}
\HOLOGO@DeclareBoolOption{hyphenbreak}
\HOLOGO@DeclareBoolOption{spacebreak}
\HOLOGO@DeclareBoolOption{discretionarybreak}
%    \end{macrocode}
%
%    \begin{macrocode}
\kv@define@key{HoLogo}{variant}{%
  \ifx\HOLOGO@name\relax
    \@PackageError{hologo}{%
      Option `variant' is not available in \string\hologoSetup,%
      \MessageBreak
      Use \string\hologoLogoSetup\space instead%
    }\@ehc
  \else
    \edef\HOLOGO@temp{#1}%
    \ifx\HOLOGO@temp\ltx@empty
      \expandafter
      \let\csname HoLogoOpt@variant@\HOLOGO@name\endcsname\@undefined
    \else
      \ltx@IfUndefined{HoLogo@\HOLOGO@name @\HOLOGO@temp}{%
        \@PackageError{hologo}{%
          Unknown variant `\HOLOGO@temp' of logo `\HOLOGO@name'%
        }\@ehc
      }{%
        \expandafter
        \let\csname HoLogoOpt@variant@\HOLOGO@name\endcsname
            \HOLOGO@temp
      }%
    \fi
  \fi
}
%    \end{macrocode}
%
%    \begin{macro}{\HOLOGO@Variant}
%    \begin{macrocode}
\def\HOLOGO@Variant#1{%
  #1%
  \ltx@ifundefined{HoLogoOpt@variant@#1}{%
  }{%
    @\csname HoLogoOpt@variant@#1\endcsname
  }%
}
%    \end{macrocode}
%    \end{macro}
%
% \subsection{Break/no-break support}
%
%    \begin{macro}{\HOLOGO@space}
%    \begin{macrocode}
\def\HOLOGO@space{%
  \ltx@ifundefined{HoLogoOpt@spacebreak@\HOLOGO@name}{%
    \ltx@ifundefined{HoLogoOpt@break@\HOLOGO@name}{%
      \chardef\HOLOGO@temp=\HOLOGOOPT@spacebreak
    }{%
      \chardef\HOLOGO@temp=%
        \csname HoLogoOpt@break@\HOLOGO@name\endcsname
    }%
  }{%
    \chardef\HOLOGO@temp=%
      \csname HoLogoOpt@spacebreak@\HOLOGO@name\endcsname
  }%
  \ifcase\HOLOGO@temp
    \penalty10000 %
  \fi
  \ltx@space
}
%    \end{macrocode}
%    \end{macro}
%
%    \begin{macro}{\HOLOGO@hyphen}
%    \begin{macrocode}
\def\HOLOGO@hyphen{%
  \ltx@ifundefined{HoLogoOpt@hyphenbreak@\HOLOGO@name}{%
    \ltx@ifundefined{HoLogoOpt@break@\HOLOGO@name}{%
      \chardef\HOLOGO@temp=\HOLOGOOPT@hyphenbreak
    }{%
      \chardef\HOLOGO@temp=%
        \csname HoLogoOpt@break@\HOLOGO@name\endcsname
    }%
  }{%
    \chardef\HOLOGO@temp=%
      \csname HoLogoOpt@hyphenbreak@\HOLOGO@name\endcsname
  }%
  \ifcase\HOLOGO@temp
    \ltx@mbox{-}%
  \else
    -%
  \fi
}
%    \end{macrocode}
%    \end{macro}
%
%    \begin{macro}{\HOLOGO@discretionary}
%    \begin{macrocode}
\def\HOLOGO@discretionary{%
  \ltx@ifundefined{HoLogoOpt@discretionarybreak@\HOLOGO@name}{%
    \ltx@ifundefined{HoLogoOpt@break@\HOLOGO@name}{%
      \chardef\HOLOGO@temp=\HOLOGOOPT@discretionarybreak
    }{%
      \chardef\HOLOGO@temp=%
        \csname HoLogoOpt@break@\HOLOGO@name\endcsname
    }%
  }{%
    \chardef\HOLOGO@temp=%
      \csname HoLogoOpt@discretionarybreak@\HOLOGO@name\endcsname
  }%
  \ifcase\HOLOGO@temp
  \else
    \-%
  \fi
}
%    \end{macrocode}
%    \end{macro}
%
%    \begin{macro}{\HOLOGO@mbox}
%    \begin{macrocode}
\def\HOLOGO@mbox#1{%
  \ltx@ifundefined{HoLogoOpt@break@\HOLOGO@name}{%
    \chardef\HOLOGO@temp=\HOLOGOOPT@hyphenbreak
  }{%
    \chardef\HOLOGO@temp=%
      \csname HoLogoOpt@break@\HOLOGO@name\endcsname
  }%
  \ifcase\HOLOGO@temp
    \ltx@mbox{#1}%
  \else
    #1%
  \fi
}
%    \end{macrocode}
%    \end{macro}
%
% \subsection{Font support}
%
%    \begin{macro}{\HoLogoFont@font}
%    \begin{tabular}{@{}ll@{}}
%    |#1|:& logo name\\
%    |#2|:& font short name\\
%    |#3|:& text
%    \end{tabular}
%    \begin{macrocode}
\def\HoLogoFont@font#1#2#3{%
  \begingroup
    \ltx@IfUndefined{HoLogoFont@logo@#1.#2}{%
      \ltx@IfUndefined{HoLogoFont@font@#2}{%
        \@PackageWarning{hologo}{%
          Missing font `#2' for logo `#1'%
        }%
        #3%
      }{%
        \csname HoLogoFont@font@#2\endcsname{#3}%
      }%
    }{%
      \csname HoLogoFont@logo@#1.#2\endcsname{#3}%
    }%
  \endgroup
}
%    \end{macrocode}
%    \end{macro}
%
%    \begin{macro}{\HoLogoFont@Def}
%    \begin{macrocode}
\def\HoLogoFont@Def#1{%
  \expandafter\def\csname HoLogoFont@font@#1\endcsname
}
%    \end{macrocode}
%    \end{macro}
%    \begin{macro}{\HoLogoFont@LogoDef}
%    \begin{macrocode}
\def\HoLogoFont@LogoDef#1#2{%
  \expandafter\def\csname HoLogoFont@logo@#1.#2\endcsname
}
%    \end{macrocode}
%    \end{macro}
%
% \subsubsection{Font defaults}
%
%    \begin{macro}{\HoLogoFont@font@general}
%    \begin{macrocode}
\HoLogoFont@Def{general}{}%
%    \end{macrocode}
%    \end{macro}
%
%    \begin{macro}{\HoLogoFont@font@rm}
%    \begin{macrocode}
\ltx@IfUndefined{rmfamily}{%
  \ltx@IfUndefined{rm}{%
  }{%
    \HoLogoFont@Def{rm}{\rm}%
  }%
}{%
  \HoLogoFont@Def{rm}{\rmfamily}%
}
%    \end{macrocode}
%    \end{macro}
%
%    \begin{macro}{\HoLogoFont@font@sf}
%    \begin{macrocode}
\ltx@IfUndefined{sffamily}{%
  \ltx@IfUndefined{sf}{%
  }{%
    \HoLogoFont@Def{sf}{\sf}%
  }%
}{%
  \HoLogoFont@Def{sf}{\sffamily}%
}
%    \end{macrocode}
%    \end{macro}
%
%    \begin{macro}{\HoLogoFont@font@bibsf}
%    In case of \hologo{plainTeX} the original small caps
%    variant is used as default. In \hologo{LaTeX}
%    the definition of package \xpackage{dtklogos} \cite{dtklogos}
%    is used.
%\begin{quote}
%\begin{verbatim}
%\DeclareRobustCommand{\BibTeX}{%
%  B%
%  \kern-.05em%
%  \hbox{%
%    $\m@th$% %% force math size calculations
%    \csname S@\f@size\endcsname
%    \fontsize\sf@size\z@
%    \math@fontsfalse
%    \selectfont
%    I%
%    \kern-.025em%
%    B
%  }%
%  \kern-.08em%
%  \-%
%  \TeX
%}
%\end{verbatim}
%\end{quote}
%    \begin{macrocode}
\ltx@IfUndefined{selectfont}{%
  \ltx@IfUndefined{tensc}{%
    \font\tensc=cmcsc10\relax
  }{}%
  \HoLogoFont@Def{bibsf}{\tensc}%
}{%
  \HoLogoFont@Def{bibsf}{%
    $\mathsurround=0pt$%
    \csname S@\f@size\endcsname
    \fontsize\sf@size{0pt}%
    \math@fontsfalse
    \selectfont
  }%
}
%    \end{macrocode}
%    \end{macro}
%
%    \begin{macro}{\HoLogoFont@font@sc}
%    \begin{macrocode}
\ltx@IfUndefined{scshape}{%
  \ltx@IfUndefined{tensc}{%
    \font\tensc=cmcsc10\relax
  }{}%
  \HoLogoFont@Def{sc}{\tensc}%
}{%
  \HoLogoFont@Def{sc}{\scshape}%
}
%    \end{macrocode}
%    \end{macro}
%
%    \begin{macro}{\HoLogoFont@font@sy}
%    \begin{macrocode}
\ltx@IfUndefined{usefont}{%
  \ltx@IfUndefined{tensy}{%
  }{%
    \HoLogoFont@Def{sy}{\tensy}%
  }%
}{%
  \HoLogoFont@Def{sy}{%
    \usefont{OMS}{cmsy}{m}{n}%
  }%
}
%    \end{macrocode}
%    \end{macro}
%
%    \begin{macro}{\HoLogoFont@font@logo}
%    \begin{macrocode}
\begingroup
  \def\x{LaTeX2e}%
\expandafter\endgroup
\ifx\fmtname\x
  \ltx@IfUndefined{logofamily}{%
    \DeclareRobustCommand\logofamily{%
      \not@math@alphabet\logofamily\relax
      \fontencoding{U}%
      \fontfamily{logo}%
      \selectfont
    }%
  }{}%
  \ltx@IfUndefined{logofamily}{%
  }{%
    \HoLogoFont@Def{logo}{\logofamily}%
  }%
\else
  \ltx@IfUndefined{tenlogo}{%
    \font\tenlogo=logo10\relax
  }{}%
  \HoLogoFont@Def{logo}{\tenlogo}%
\fi
%    \end{macrocode}
%    \end{macro}
%
% \subsubsection{Font setup}
%
%    \begin{macro}{\hologoFontSetup}
%    \begin{macrocode}
\def\hologoFontSetup{%
  \let\HOLOGO@name\relax
  \HOLOGO@FontSetup
}
%    \end{macrocode}
%    \end{macro}
%
%    \begin{macro}{\hologoLogoFontSetup}
%    \begin{macrocode}
\def\hologoLogoFontSetup#1{%
  \edef\HOLOGO@name{#1}%
  \ltx@IfUndefined{HoLogo@\HOLOGO@name}{%
    \@PackageError{hologo}{%
      Unknown logo `\HOLOGO@name'%
    }\@ehc
    \ltx@gobble
  }{%
    \HOLOGO@FontSetup
  }%
}
%    \end{macrocode}
%    \end{macro}
%
%    \begin{macro}{\HOLOGO@FontSetup}
%    \begin{macrocode}
\def\HOLOGO@FontSetup{%
  \kvsetkeys{HoLogoFont}%
}
%    \end{macrocode}
%    \end{macro}
%
%    \begin{macrocode}
\def\HOLOGO@temp#1{%
  \kv@define@key{HoLogoFont}{#1}{%
    \ifx\HOLOGO@name\relax
      \HoLogoFont@Def{#1}{##1}%
    \else
      \HoLogoFont@LogoDef\HOLOGO@name{#1}{##1}%
    \fi
  }%
}
\HOLOGO@temp{general}
\HOLOGO@temp{sf}
%    \end{macrocode}
%
% \subsection{Generic logo commands}
%
%    \begin{macrocode}
\HOLOGO@IfExists\hologo{%
  \@PackageError{hologo}{%
    \string\hologo\ltx@space is already defined.\MessageBreak
    Package loading is aborted%
  }\@ehc
  \HOLOGO@AtEnd
}%
\HOLOGO@IfExists\hologoRobust{%
  \@PackageError{hologo}{%
    \string\hologoRobust\ltx@space is already defined.\MessageBreak
    Package loading is aborted%
  }\@ehc
  \HOLOGO@AtEnd
}%
%    \end{macrocode}
%
% \subsubsection{\cs{hologo} and friends}
%
%    \begin{macrocode}
\ifluatex
  \expandafter\ltx@firstofone
\else
  \expandafter\ltx@gobble
\fi
{%
  \ltx@IfUndefined{ifincsname}{%
    \ifnum\luatexversion<36 %
      \expandafter\ltx@gobble
    \else
      \expandafter\ltx@firstofone
    \fi
    {%
      \begingroup
        \ifcase0%
            \directlua{%
              if tex.enableprimitives then %
                tex.enableprimitives('HOLOGO@', {'ifincsname'})%
              else %
                tex.print('1')%
              end%
            }%
            \ifx\HOLOGO@ifincsname\@undefined 1\fi%
            \relax
          \expandafter\ltx@firstofone
        \else
          \endgroup
          \expandafter\ltx@gobble
        \fi
        {%
          \global\let\ifincsname\HOLOGO@ifincsname
        }%
      \HOLOGO@temp
    }%
  }{}%
}
%    \end{macrocode}
%    \begin{macrocode}
\ltx@IfUndefined{ifincsname}{%
  \catcode`$=14 %
}{%
  \catcode`$=9 %
}
%    \end{macrocode}
%
%    \begin{macro}{\hologo}
%    \begin{macrocode}
\def\hologo#1{%
$ \ifincsname
$   \ltx@ifundefined{HoLogoCs@\HOLOGO@Variant{#1}}{%
$     #1%
$   }{%
$     \csname HoLogoCs@\HOLOGO@Variant{#1}\endcsname\ltx@firstoftwo
$   }%
$ \else
    \HOLOGO@IfExists\texorpdfstring\texorpdfstring\ltx@firstoftwo
    {%
      \hologoRobust{#1}%
    }{%
      \ltx@ifundefined{HoLogoBkm@\HOLOGO@Variant{#1}}{%
        \ltx@ifundefined{HoLogo@#1}{?#1?}{#1}%
      }{%
        \csname HoLogoBkm@\HOLOGO@Variant{#1}\endcsname
        \ltx@firstoftwo
      }%
    }%
$ \fi
}
%    \end{macrocode}
%    \end{macro}
%    \begin{macro}{\Hologo}
%    \begin{macrocode}
\def\Hologo#1{%
$ \ifincsname
$   \ltx@ifundefined{HoLogoCs@\HOLOGO@Variant{#1}}{%
$     #1%
$   }{%
$     \csname HoLogoCs@\HOLOGO@Variant{#1}\endcsname\ltx@secondoftwo
$   }%
$ \else
    \HOLOGO@IfExists\texorpdfstring\texorpdfstring\ltx@firstoftwo
    {%
      \HologoRobust{#1}%
    }{%
      \ltx@ifundefined{HoLogoBkm@\HOLOGO@Variant{#1}}{%
        \ltx@ifundefined{HoLogo@#1}{?#1?}{#1}%
      }{%
        \csname HoLogoBkm@\HOLOGO@Variant{#1}\endcsname
        \ltx@secondoftwo
      }%
    }%
$ \fi
}
%    \end{macrocode}
%    \end{macro}
%
%    \begin{macro}{\hologoVariant}
%    \begin{macrocode}
\def\hologoVariant#1#2{%
  \ifx\relax#2\relax
    \hologo{#1}%
  \else
$   \ifincsname
$     \ltx@ifundefined{HoLogoCs@#1@#2}{%
$       #1%
$     }{%
$       \csname HoLogoCs@#1@#2\endcsname\ltx@firstoftwo
$     }%
$   \else
      \HOLOGO@IfExists\texorpdfstring\texorpdfstring\ltx@firstoftwo
      {%
        \hologoVariantRobust{#1}{#2}%
      }{%
        \ltx@ifundefined{HoLogoBkm@#1@#2}{%
          \ltx@ifundefined{HoLogo@#1}{?#1?}{#1}%
        }{%
          \csname HoLogoBkm@#1@#2\endcsname
          \ltx@firstoftwo
        }%
      }%
$   \fi
  \fi
}
%    \end{macrocode}
%    \end{macro}
%    \begin{macro}{\HologoVariant}
%    \begin{macrocode}
\def\HologoVariant#1#2{%
  \ifx\relax#2\relax
    \Hologo{#1}%
  \else
$   \ifincsname
$     \ltx@ifundefined{HoLogoCs@#1@#2}{%
$       #1%
$     }{%
$       \csname HoLogoCs@#1@#2\endcsname\ltx@secondoftwo
$     }%
$   \else
      \HOLOGO@IfExists\texorpdfstring\texorpdfstring\ltx@firstoftwo
      {%
        \HologoVariantRobust{#1}{#2}%
      }{%
        \ltx@ifundefined{HoLogoBkm@#1@#2}{%
          \ltx@ifundefined{HoLogo@#1}{?#1?}{#1}%
        }{%
          \csname HoLogoBkm@#1@#2\endcsname
          \ltx@secondoftwo
        }%
      }%
$   \fi
  \fi
}
%    \end{macrocode}
%    \end{macro}
%
%    \begin{macrocode}
\catcode`\$=3 %
%    \end{macrocode}
%
% \subsubsection{\cs{hologoRobust} and friends}
%
%    \begin{macro}{\hologoRobust}
%    \begin{macrocode}
\ltx@IfUndefined{protected}{%
  \ltx@IfUndefined{DeclareRobustCommand}{%
    \def\hologoRobust#1%
  }{%
    \DeclareRobustCommand*\hologoRobust[1]%
  }%
}{%
  \protected\def\hologoRobust#1%
}%
{%
  \edef\HOLOGO@name{#1}%
  \ltx@IfUndefined{HoLogo@\HOLOGO@Variant\HOLOGO@name}{%
    \@PackageError{hologo}{%
      Unknown logo `\HOLOGO@name'%
    }\@ehc
    ?\HOLOGO@name?%
  }{%
    \ltx@IfUndefined{ver@tex4ht.sty}{%
      \HoLogoFont@font\HOLOGO@name{general}{%
        \csname HoLogo@\HOLOGO@Variant\HOLOGO@name\endcsname
        \ltx@firstoftwo
      }%
    }{%
      \ltx@IfUndefined{HoLogoHtml@\HOLOGO@Variant\HOLOGO@name}{%
        \HOLOGO@name
      }{%
        \csname HoLogoHtml@\HOLOGO@Variant\HOLOGO@name\endcsname
        \ltx@firstoftwo
      }%
    }%
  }%
}
%    \end{macrocode}
%    \end{macro}
%    \begin{macro}{\HologoRobust}
%    \begin{macrocode}
\ltx@IfUndefined{protected}{%
  \ltx@IfUndefined{DeclareRobustCommand}{%
    \def\HologoRobust#1%
  }{%
    \DeclareRobustCommand*\HologoRobust[1]%
  }%
}{%
  \protected\def\HologoRobust#1%
}%
{%
  \edef\HOLOGO@name{#1}%
  \ltx@IfUndefined{HoLogo@\HOLOGO@Variant\HOLOGO@name}{%
    \@PackageError{hologo}{%
      Unknown logo `\HOLOGO@name'%
    }\@ehc
    ?\HOLOGO@name?%
  }{%
    \ltx@IfUndefined{ver@tex4ht.sty}{%
      \HoLogoFont@font\HOLOGO@name{general}{%
        \csname HoLogo@\HOLOGO@Variant\HOLOGO@name\endcsname
        \ltx@secondoftwo
      }%
    }{%
      \ltx@IfUndefined{HoLogoHtml@\HOLOGO@Variant\HOLOGO@name}{%
        \expandafter\HOLOGO@Uppercase\HOLOGO@name
      }{%
        \csname HoLogoHtml@\HOLOGO@Variant\HOLOGO@name\endcsname
        \ltx@secondoftwo
      }%
    }%
  }%
}
%    \end{macrocode}
%    \end{macro}
%    \begin{macro}{\hologoVariantRobust}
%    \begin{macrocode}
\ltx@IfUndefined{protected}{%
  \ltx@IfUndefined{DeclareRobustCommand}{%
    \def\hologoVariantRobust#1#2%
  }{%
    \DeclareRobustCommand*\hologoVariantRobust[2]%
  }%
}{%
  \protected\def\hologoVariantRobust#1#2%
}%
{%
  \begingroup
    \hologoLogoSetup{#1}{variant={#2}}%
    \hologoRobust{#1}%
  \endgroup
}
%    \end{macrocode}
%    \end{macro}
%    \begin{macro}{\HologoVariantRobust}
%    \begin{macrocode}
\ltx@IfUndefined{protected}{%
  \ltx@IfUndefined{DeclareRobustCommand}{%
    \def\HologoVariantRobust#1#2%
  }{%
    \DeclareRobustCommand*\HologoVariantRobust[2]%
  }%
}{%
  \protected\def\HologoVariantRobust#1#2%
}%
{%
  \begingroup
    \hologoLogoSetup{#1}{variant={#2}}%
    \HologoRobust{#1}%
  \endgroup
}
%    \end{macrocode}
%    \end{macro}
%
%    \begin{macro}{\hologorobust}
%    Macro \cs{hologorobust} is only defined for compatibility.
%    Its use is deprecated.
%    \begin{macrocode}
\def\hologorobust{\hologoRobust}
%    \end{macrocode}
%    \end{macro}
%
% \subsection{Helpers}
%
%    \begin{macro}{\HOLOGO@Uppercase}
%    Macro \cs{HOLOGO@Uppercase} is restricted to \cs{uppercase},
%    because \hologo{plainTeX} or \hologo{iniTeX} do not provide
%    \cs{MakeUppercase}.
%    \begin{macrocode}
\def\HOLOGO@Uppercase#1{\uppercase{#1}}
%    \end{macrocode}
%    \end{macro}
%
%    \begin{macro}{\HOLOGO@PdfdocUnicode}
%    \begin{macrocode}
\def\HOLOGO@PdfdocUnicode{%
  \ifx\ifHy@unicode\iftrue
    \expandafter\ltx@secondoftwo
  \else
    \expandafter\ltx@firstoftwo
  \fi
}
%    \end{macrocode}
%    \end{macro}
%
%    \begin{macro}{\HOLOGO@Math}
%    \begin{macrocode}
\def\HOLOGO@MathSetup{%
  \mathsurround0pt\relax
  \HOLOGO@IfExists\f@series{%
    \if b\expandafter\ltx@car\f@series x\@nil
      \csname boldmath\endcsname
   \fi
  }{}%
}
%    \end{macrocode}
%    \end{macro}
%
%    \begin{macro}{\HOLOGO@TempDimen}
%    \begin{macrocode}
\dimendef\HOLOGO@TempDimen=\ltx@zero
%    \end{macrocode}
%    \end{macro}
%    \begin{macro}{\HOLOGO@NegativeKerning}
%    \begin{macrocode}
\def\HOLOGO@NegativeKerning#1{%
  \begingroup
    \HOLOGO@TempDimen=0pt\relax
    \comma@parse@normalized{#1}{%
      \ifdim\HOLOGO@TempDimen=0pt %
        \expandafter\HOLOGO@@NegativeKerning\comma@entry
      \fi
      \ltx@gobble
    }%
    \ifdim\HOLOGO@TempDimen<0pt %
      \kern\HOLOGO@TempDimen
    \fi
  \endgroup
}
%    \end{macrocode}
%    \end{macro}
%    \begin{macro}{\HOLOGO@@NegativeKerning}
%    \begin{macrocode}
\def\HOLOGO@@NegativeKerning#1#2{%
  \setbox\ltx@zero\hbox{#1#2}%
  \HOLOGO@TempDimen=\wd\ltx@zero
  \setbox\ltx@zero\hbox{#1\kern0pt#2}%
  \advance\HOLOGO@TempDimen by -\wd\ltx@zero
}
%    \end{macrocode}
%    \end{macro}
%
%    \begin{macro}{\HOLOGO@SpaceFactor}
%    \begin{macrocode}
\def\HOLOGO@SpaceFactor{%
  \spacefactor1000 %
}
%    \end{macrocode}
%    \end{macro}
%
%    \begin{macro}{\HOLOGO@Span}
%    \begin{macrocode}
\def\HOLOGO@Span#1#2{%
  \HCode{<span class="HoLogo-#1">}%
  #2%
  \HCode{</span>}%
}
%    \end{macrocode}
%    \end{macro}
%
% \subsubsection{Text subscript}
%
%    \begin{macro}{\HOLOGO@SubScript}%
%    \begin{macrocode}
\def\HOLOGO@SubScript#1{%
  \ltx@IfUndefined{textsubscript}{%
    \ltx@IfUndefined{text}{%
      \ltx@mbox{%
        \mathsurround=0pt\relax
        $%
          _{%
            \ltx@IfUndefined{sf@size}{%
              \mathrm{#1}%
            }{%
              \mbox{%
                \fontsize\sf@size{0pt}\selectfont
                #1%
              }%
            }%
          }%
        $%
      }%
    }{%
      \ltx@mbox{%
        \mathsurround=0pt\relax
        $_{\text{#1}}$%
      }%
    }%
  }{%
    \textsubscript{#1}%
  }%
}
%    \end{macrocode}
%    \end{macro}
%
% \subsection{\hologo{TeX} and friends}
%
% \subsubsection{\hologo{TeX}}
%
%    \begin{macro}{\HoLogo@TeX}
%    Source: \hologo{LaTeX} kernel.
%    \begin{macrocode}
\def\HoLogo@TeX#1{%
  T\kern-.1667em\lower.5ex\hbox{E}\kern-.125emX\HOLOGO@SpaceFactor
}
%    \end{macrocode}
%    \end{macro}
%    \begin{macro}{\HoLogoHtml@TeX}
%    \begin{macrocode}
\def\HoLogoHtml@TeX#1{%
  \HoLogoCss@TeX
  \HOLOGO@Span{TeX}{%
    T%
    \HOLOGO@Span{e}{%
      E%
    }%
    X%
  }%
}
%    \end{macrocode}
%    \end{macro}
%    \begin{macro}{\HoLogoCss@TeX}
%    \begin{macrocode}
\def\HoLogoCss@TeX{%
  \Css{%
    span.HoLogo-TeX span.HoLogo-e{%
      position:relative;%
      top:.5ex;%
      margin-left:-.1667em;%
      margin-right:-.125em;%
    }%
  }%
  \Css{%
    a span.HoLogo-TeX span.HoLogo-e{%
      text-decoration:none;%
    }%
  }%
  \global\let\HoLogoCss@TeX\relax
}
%    \end{macrocode}
%    \end{macro}
%
% \subsubsection{\hologo{plainTeX}}
%
%    \begin{macro}{\HoLogo@plainTeX@space}
%    Source: ``The \hologo{TeX}book''
%    \begin{macrocode}
\def\HoLogo@plainTeX@space#1{%
  \HOLOGO@mbox{#1{p}{P}lain}\HOLOGO@space\hologo{TeX}%
}
%    \end{macrocode}
%    \end{macro}
%    \begin{macro}{\HoLogoCs@plainTeX@space}
%    \begin{macrocode}
\def\HoLogoCs@plainTeX@space#1{#1{p}{P}lain TeX}%
%    \end{macrocode}
%    \end{macro}
%    \begin{macro}{\HoLogoBkm@plainTeX@space}
%    \begin{macrocode}
\def\HoLogoBkm@plainTeX@space#1{%
  #1{p}{P}lain \hologo{TeX}%
}
%    \end{macrocode}
%    \end{macro}
%    \begin{macro}{\HoLogoHtml@plainTeX@space}
%    \begin{macrocode}
\def\HoLogoHtml@plainTeX@space#1{%
  #1{p}{P}lain \hologo{TeX}%
}
%    \end{macrocode}
%    \end{macro}
%
%    \begin{macro}{\HoLogo@plainTeX@hyphen}
%    \begin{macrocode}
\def\HoLogo@plainTeX@hyphen#1{%
  \HOLOGO@mbox{#1{p}{P}lain}\HOLOGO@hyphen\hologo{TeX}%
}
%    \end{macrocode}
%    \end{macro}
%    \begin{macro}{\HoLogoCs@plainTeX@hyphen}
%    \begin{macrocode}
\def\HoLogoCs@plainTeX@hyphen#1{#1{p}{P}lain-TeX}
%    \end{macrocode}
%    \end{macro}
%    \begin{macro}{\HoLogoBkm@plainTeX@hyphen}
%    \begin{macrocode}
\def\HoLogoBkm@plainTeX@hyphen#1{%
  #1{p}{P}lain-\hologo{TeX}%
}
%    \end{macrocode}
%    \end{macro}
%    \begin{macro}{\HoLogoHtml@plainTeX@hyphen}
%    \begin{macrocode}
\def\HoLogoHtml@plainTeX@hyphen#1{%
  #1{p}{P}lain-\hologo{TeX}%
}
%    \end{macrocode}
%    \end{macro}
%
%    \begin{macro}{\HoLogo@plainTeX@runtogether}
%    \begin{macrocode}
\def\HoLogo@plainTeX@runtogether#1{%
  \HOLOGO@mbox{#1{p}{P}lain\hologo{TeX}}%
}
%    \end{macrocode}
%    \end{macro}
%    \begin{macro}{\HoLogoCs@plainTeX@runtogether}
%    \begin{macrocode}
\def\HoLogoCs@plainTeX@runtogether#1{#1{p}{P}lainTeX}
%    \end{macrocode}
%    \end{macro}
%    \begin{macro}{\HoLogoBkm@plainTeX@runtogether}
%    \begin{macrocode}
\def\HoLogoBkm@plainTeX@runtogether#1{%
  #1{p}{P}lain\hologo{TeX}%
}
%    \end{macrocode}
%    \end{macro}
%    \begin{macro}{\HoLogoHtml@plainTeX@runtogether}
%    \begin{macrocode}
\def\HoLogoHtml@plainTeX@runtogether#1{%
  #1{p}{P}lain\hologo{TeX}%
}
%    \end{macrocode}
%    \end{macro}
%
%    \begin{macro}{\HoLogo@plainTeX}
%    \begin{macrocode}
\def\HoLogo@plainTeX{\HoLogo@plainTeX@space}
%    \end{macrocode}
%    \end{macro}
%    \begin{macro}{\HoLogoCs@plainTeX}
%    \begin{macrocode}
\def\HoLogoCs@plainTeX{\HoLogoCs@plainTeX@space}
%    \end{macrocode}
%    \end{macro}
%    \begin{macro}{\HoLogoBkm@plainTeX}
%    \begin{macrocode}
\def\HoLogoBkm@plainTeX{\HoLogoBkm@plainTeX@space}
%    \end{macrocode}
%    \end{macro}
%    \begin{macro}{\HoLogoHtml@plainTeX}
%    \begin{macrocode}
\def\HoLogoHtml@plainTeX{\HoLogoHtml@plainTeX@space}
%    \end{macrocode}
%    \end{macro}
%
% \subsubsection{\hologo{LaTeX}}
%
%    Source: \hologo{LaTeX} kernel.
%\begin{quote}
%\begin{verbatim}
%\DeclareRobustCommand{\LaTeX}{%
%  L%
%  \kern-.36em%
%  {%
%    \sbox\z@ T%
%    \vbox to\ht\z@{%
%      \hbox{%
%        \check@mathfonts
%        \fontsize\sf@size\z@
%        \math@fontsfalse
%        \selectfont
%        A%
%      }%
%      \vss
%    }%
%  }%
%  \kern-.15em%
%  \TeX
%}
%\end{verbatim}
%\end{quote}
%
%    \begin{macro}{\HoLogo@La}
%    \begin{macrocode}
\def\HoLogo@La#1{%
  L%
  \kern-.36em%
  \begingroup
    \setbox\ltx@zero\hbox{T}%
    \vbox to\ht\ltx@zero{%
      \hbox{%
        \ltx@ifundefined{check@mathfonts}{%
          \csname sevenrm\endcsname
        }{%
          \check@mathfonts
          \fontsize\sf@size{0pt}%
          \math@fontsfalse\selectfont
        }%
        A%
      }%
      \vss
    }%
  \endgroup
}
%    \end{macrocode}
%    \end{macro}
%
%    \begin{macro}{\HoLogo@LaTeX}
%    Source: \hologo{LaTeX} kernel.
%    \begin{macrocode}
\def\HoLogo@LaTeX#1{%
  \hologo{La}%
  \kern-.15em%
  \hologo{TeX}%
}
%    \end{macrocode}
%    \end{macro}
%    \begin{macro}{\HoLogoHtml@LaTeX}
%    \begin{macrocode}
\def\HoLogoHtml@LaTeX#1{%
  \HoLogoCss@LaTeX
  \HOLOGO@Span{LaTeX}{%
    L%
    \HOLOGO@Span{a}{%
      A%
    }%
    \hologo{TeX}%
  }%
}
%    \end{macrocode}
%    \end{macro}
%    \begin{macro}{\HoLogoCss@LaTeX}
%    \begin{macrocode}
\def\HoLogoCss@LaTeX{%
  \Css{%
    span.HoLogo-LaTeX span.HoLogo-a{%
      position:relative;%
      top:-.5ex;%
      margin-left:-.36em;%
      margin-right:-.15em;%
      font-size:85\%;%
    }%
  }%
  \global\let\HoLogoCss@LaTeX\relax
}
%    \end{macrocode}
%    \end{macro}
%
% \subsubsection{\hologo{(La)TeX}}
%
%    \begin{macro}{\HoLogo@LaTeXTeX}
%    The kerning around the parentheses is taken
%    from package \xpackage{dtklogos} \cite{dtklogos}.
%\begin{quote}
%\begin{verbatim}
%\DeclareRobustCommand{\LaTeXTeX}{%
%  (%
%  \kern-.15em%
%  L%
%  \kern-.36em%
%  {%
%    \sbox\z@ T%
%    \vbox to\ht0{%
%      \hbox{%
%        $\m@th$%
%        \csname S@\f@size\endcsname
%        \fontsize\sf@size\z@
%        \math@fontsfalse
%        \selectfont
%        A%
%      }%
%      \vss
%    }%
%  }%
%  \kern-.2em%
%  )%
%  \kern-.15em%
%  \TeX
%}
%\end{verbatim}
%\end{quote}
%    \begin{macrocode}
\def\HoLogo@LaTeXTeX#1{%
  (%
  \kern-.15em%
  \hologo{La}%
  \kern-.2em%
  )%
  \kern-.15em%
  \hologo{TeX}%
}
%    \end{macrocode}
%    \end{macro}
%    \begin{macro}{\HoLogoBkm@LaTeXTeX}
%    \begin{macrocode}
\def\HoLogoBkm@LaTeXTeX#1{(La)TeX}
%    \end{macrocode}
%    \end{macro}
%
%    \begin{macro}{\HoLogo@(La)TeX}
%    \begin{macrocode}
\expandafter
\let\csname HoLogo@(La)TeX\endcsname\HoLogo@LaTeXTeX
%    \end{macrocode}
%    \end{macro}
%    \begin{macro}{\HoLogoBkm@(La)TeX}
%    \begin{macrocode}
\expandafter
\let\csname HoLogoBkm@(La)TeX\endcsname\HoLogoBkm@LaTeXTeX
%    \end{macrocode}
%    \end{macro}
%    \begin{macro}{\HoLogoHtml@LaTeXTeX}
%    \begin{macrocode}
\def\HoLogoHtml@LaTeXTeX#1{%
  \HoLogoCss@LaTeXTeX
  \HOLOGO@Span{LaTeXTeX}{%
    (%
    \HOLOGO@Span{L}{L}%
    \HOLOGO@Span{a}{A}%
    \HOLOGO@Span{ParenRight}{)}%
    \hologo{TeX}%
  }%
}
%    \end{macrocode}
%    \end{macro}
%    \begin{macro}{\HoLogoHtml@(La)TeX}
%    Kerning after opening parentheses and before closing parentheses
%    is $-0.1$\,em. The original values $-0.15$\,em
%    looked too ugly for a serif font.
%    \begin{macrocode}
\expandafter
\let\csname HoLogoHtml@(La)TeX\endcsname\HoLogoHtml@LaTeXTeX
%    \end{macrocode}
%    \end{macro}
%    \begin{macro}{\HoLogoCss@LaTeXTeX}
%    \begin{macrocode}
\def\HoLogoCss@LaTeXTeX{%
  \Css{%
    span.HoLogo-LaTeXTeX span.HoLogo-L{%
      margin-left:-.1em;%
    }%
  }%
  \Css{%
    span.HoLogo-LaTeXTeX span.HoLogo-a{%
      position:relative;%
      top:-.5ex;%
      margin-left:-.36em;%
      margin-right:-.1em;%
      font-size:85\%;%
    }%
  }%
  \Css{%
    span.HoLogo-LaTeXTeX span.HoLogo-ParenRight{%
      margin-right:-.15em;%
    }%
  }%
  \global\let\HoLogoCss@LaTeXTeX\relax
}
%    \end{macrocode}
%    \end{macro}
%
% \subsubsection{\hologo{LaTeXe}}
%
%    \begin{macro}{\HoLogo@LaTeXe}
%    Source: \hologo{LaTeX} kernel
%    \begin{macrocode}
\def\HoLogo@LaTeXe#1{%
  \hologo{LaTeX}%
  \kern.15em%
  \hbox{%
    \HOLOGO@MathSetup
    2%
    $_{\textstyle\varepsilon}$%
  }%
}
%    \end{macrocode}
%    \end{macro}
%
%    \begin{macro}{\HoLogoCs@LaTeXe}
%    \begin{macrocode}
\ifnum64=`\^^^^0040\relax % test for big chars of LuaTeX/XeTeX
  \catcode`\$=9 %
  \catcode`\&=14 %
\else
  \catcode`\$=14 %
  \catcode`\&=9 %
\fi
\def\HoLogoCs@LaTeXe#1{%
  LaTeX2%
$ \string ^^^^0395%
& e%
}%
\catcode`\$=3 %
\catcode`\&=4 %
%    \end{macrocode}
%    \end{macro}
%
%    \begin{macro}{\HoLogoBkm@LaTeXe}
%    \begin{macrocode}
\def\HoLogoBkm@LaTeXe#1{%
  \hologo{LaTeX}%
  2%
  \HOLOGO@PdfdocUnicode{e}{\textepsilon}%
}
%    \end{macrocode}
%    \end{macro}
%
%    \begin{macro}{\HoLogoHtml@LaTeXe}
%    \begin{macrocode}
\def\HoLogoHtml@LaTeXe#1{%
  \HoLogoCss@LaTeXe
  \HOLOGO@Span{LaTeX2e}{%
    \hologo{LaTeX}%
    \HOLOGO@Span{2}{2}%
    \HOLOGO@Span{e}{%
      \HOLOGO@MathSetup
      \ensuremath{\textstyle\varepsilon}%
    }%
  }%
}
%    \end{macrocode}
%    \end{macro}
%    \begin{macro}{\HoLogoCss@LaTeXe}
%    \begin{macrocode}
\def\HoLogoCss@LaTeXe{%
  \Css{%
    span.HoLogo-LaTeX2e span.HoLogo-2{%
      padding-left:.15em;%
    }%
  }%
  \Css{%
    span.HoLogo-LaTeX2e span.HoLogo-e{%
      position:relative;%
      top:.35ex;%
      text-decoration:none;%
    }%
  }%
  \global\let\HoLogoCss@LaTeXe\relax
}
%    \end{macrocode}
%    \end{macro}
%
%    \begin{macro}{\HoLogo@LaTeX2e}
%    \begin{macrocode}
\expandafter
\let\csname HoLogo@LaTeX2e\endcsname\HoLogo@LaTeXe
%    \end{macrocode}
%    \end{macro}
%    \begin{macro}{\HoLogoCs@LaTeX2e}
%    \begin{macrocode}
\expandafter
\let\csname HoLogoCs@LaTeX2e\endcsname\HoLogoCs@LaTeXe
%    \end{macrocode}
%    \end{macro}
%    \begin{macro}{\HoLogoBkm@LaTeX2e}
%    \begin{macrocode}
\expandafter
\let\csname HoLogoBkm@LaTeX2e\endcsname\HoLogoBkm@LaTeXe
%    \end{macrocode}
%    \end{macro}
%    \begin{macro}{\HoLogoHtml@LaTeX2e}
%    \begin{macrocode}
\expandafter
\let\csname HoLogoHtml@LaTeX2e\endcsname\HoLogoHtml@LaTeXe
%    \end{macrocode}
%    \end{macro}
%
% \subsubsection{\hologo{LaTeX3}}
%
%    \begin{macro}{\HoLogo@LaTeX3}
%    Source: \hologo{LaTeX} kernel
%    \begin{macrocode}
\expandafter\def\csname HoLogo@LaTeX3\endcsname#1{%
  \hologo{LaTeX}%
  3%
}
%    \end{macrocode}
%    \end{macro}
%
%    \begin{macro}{\HoLogoBkm@LaTeX3}
%    \begin{macrocode}
\expandafter\def\csname HoLogoBkm@LaTeX3\endcsname#1{%
  \hologo{LaTeX}%
  3%
}
%    \end{macrocode}
%    \end{macro}
%    \begin{macro}{\HoLogoHtml@LaTeX3}
%    \begin{macrocode}
\expandafter
\let\csname HoLogoHtml@LaTeX3\expandafter\endcsname
\csname HoLogo@LaTeX3\endcsname
%    \end{macrocode}
%    \end{macro}
%
% \subsubsection{\hologo{LaTeXML}}
%
%    \begin{macro}{\HoLogo@LaTeXML}
%    \begin{macrocode}
\def\HoLogo@LaTeXML#1{%
  \HOLOGO@mbox{%
    \hologo{La}%
    \kern-.15em%
    T%
    \kern-.1667em%
    \lower.5ex\hbox{E}%
    \kern-.125em%
    \HoLogoFont@font{LaTeXML}{sc}{xml}%
  }%
}
%    \end{macrocode}
%    \end{macro}
%    \begin{macro}{\HoLogoHtml@pdfLaTeX}
%    \begin{macrocode}
\def\HoLogoHtml@LaTeXML#1{%
  \HOLOGO@Span{LaTeXML}{%
    \HoLogoCss@LaTeX
    \HoLogoCss@TeX
    \HOLOGO@Span{LaTeX}{%
      L%
      \HOLOGO@Span{a}{%
        A%
      }%
    }%
    \HOLOGO@Span{TeX}{%
      T%
      \HOLOGO@Span{e}{%
        E%
      }%
    }%
    \HCode{<span style="font-variant: small-caps;">}%
    xml%
    \HCode{</span>}%
  }%
}
%    \end{macrocode}
%    \end{macro}
%
% \subsubsection{\hologo{eTeX}}
%
%    \begin{macro}{\HoLogo@eTeX}
%    Source: package \xpackage{etex}
%    \begin{macrocode}
\def\HoLogo@eTeX#1{%
  \ltx@mbox{%
    \HOLOGO@MathSetup
    $\varepsilon$%
    -%
    \HOLOGO@NegativeKerning{-T,T-,To}%
    \hologo{TeX}%
  }%
}
%    \end{macrocode}
%    \end{macro}
%    \begin{macro}{\HoLogoCs@eTeX}
%    \begin{macrocode}
\ifnum64=`\^^^^0040\relax % test for big chars of LuaTeX/XeTeX
  \catcode`\$=9 %
  \catcode`\&=14 %
\else
  \catcode`\$=14 %
  \catcode`\&=9 %
\fi
\def\HoLogoCs@eTeX#1{%
$ #1{\string ^^^^0395}{\string ^^^^03b5}%
& #1{e}{E}%
  TeX%
}%
\catcode`\$=3 %
\catcode`\&=4 %
%    \end{macrocode}
%    \end{macro}
%    \begin{macro}{\HoLogoBkm@eTeX}
%    \begin{macrocode}
\def\HoLogoBkm@eTeX#1{%
  \HOLOGO@PdfdocUnicode{#1{e}{E}}{\textepsilon}%
  -%
  \hologo{TeX}%
}
%    \end{macrocode}
%    \end{macro}
%    \begin{macro}{\HoLogoHtml@eTeX}
%    \begin{macrocode}
\def\HoLogoHtml@eTeX#1{%
  \ltx@mbox{%
    \HOLOGO@MathSetup
    $\varepsilon$%
    -%
    \hologo{TeX}%
  }%
}
%    \end{macrocode}
%    \end{macro}
%
% \subsubsection{\hologo{iniTeX}}
%
%    \begin{macro}{\HoLogo@iniTeX}
%    \begin{macrocode}
\def\HoLogo@iniTeX#1{%
  \HOLOGO@mbox{%
    #1{i}{I}ni\hologo{TeX}%
  }%
}
%    \end{macrocode}
%    \end{macro}
%    \begin{macro}{\HoLogoCs@iniTeX}
%    \begin{macrocode}
\def\HoLogoCs@iniTeX#1{#1{i}{I}niTeX}
%    \end{macrocode}
%    \end{macro}
%    \begin{macro}{\HoLogoBkm@iniTeX}
%    \begin{macrocode}
\def\HoLogoBkm@iniTeX#1{%
  #1{i}{I}ni\hologo{TeX}%
}
%    \end{macrocode}
%    \end{macro}
%    \begin{macro}{\HoLogoHtml@iniTeX}
%    \begin{macrocode}
\let\HoLogoHtml@iniTeX\HoLogo@iniTeX
%    \end{macrocode}
%    \end{macro}
%
% \subsubsection{\hologo{virTeX}}
%
%    \begin{macro}{\HoLogo@virTeX}
%    \begin{macrocode}
\def\HoLogo@virTeX#1{%
  \HOLOGO@mbox{%
    #1{v}{V}ir\hologo{TeX}%
  }%
}
%    \end{macrocode}
%    \end{macro}
%    \begin{macro}{\HoLogoCs@virTeX}
%    \begin{macrocode}
\def\HoLogoCs@virTeX#1{#1{v}{V}irTeX}
%    \end{macrocode}
%    \end{macro}
%    \begin{macro}{\HoLogoBkm@virTeX}
%    \begin{macrocode}
\def\HoLogoBkm@virTeX#1{%
  #1{v}{V}ir\hologo{TeX}%
}
%    \end{macrocode}
%    \end{macro}
%    \begin{macro}{\HoLogoHtml@virTeX}
%    \begin{macrocode}
\let\HoLogoHtml@virTeX\HoLogo@virTeX
%    \end{macrocode}
%    \end{macro}
%
% \subsubsection{\hologo{SliTeX}}
%
% \paragraph{Definitions of the three variants.}
%
%    \begin{macro}{\HoLogo@SLiTeX@lift}
%    \begin{macrocode}
\def\HoLogo@SLiTeX@lift#1{%
  \HoLogoFont@font{SliTeX}{rm}{%
    S%
    \kern-.06em%
    L%
    \kern-.18em%
    \raise.32ex\hbox{\HoLogoFont@font{SliTeX}{sc}{i}}%
    \HOLOGO@discretionary
    \kern-.06em%
    \hologo{TeX}%
  }%
}
%    \end{macrocode}
%    \end{macro}
%    \begin{macro}{\HoLogoBkm@SLiTeX@lift}
%    \begin{macrocode}
\def\HoLogoBkm@SLiTeX@lift#1{SLiTeX}
%    \end{macrocode}
%    \end{macro}
%    \begin{macro}{\HoLogoHtml@SLiTeX@lift}
%    \begin{macrocode}
\def\HoLogoHtml@SLiTeX@lift#1{%
  \HoLogoCss@SLiTeX@lift
  \HOLOGO@Span{SLiTeX-lift}{%
    \HoLogoFont@font{SliTeX}{rm}{%
      S%
      \HOLOGO@Span{L}{L}%
      \HOLOGO@Span{i}{i}%
      \hologo{TeX}%
    }%
  }%
}
%    \end{macrocode}
%    \end{macro}
%    \begin{macro}{\HoLogoCss@SLiTeX@lift}
%    \begin{macrocode}
\def\HoLogoCss@SLiTeX@lift{%
  \Css{%
    span.HoLogo-SLiTeX-lift span.HoLogo-L{%
      margin-left:-.06em;%
      margin-right:-.18em;%
    }%
  }%
  \Css{%
    span.HoLogo-SLiTeX-lift span.HoLogo-i{%
      position:relative;%
      top:-.32ex;%
      margin-right:-.06em;%
      font-variant:small-caps;%
    }%
  }%
  \global\let\HoLogoCss@SLiTeX@lift\relax
}
%    \end{macrocode}
%    \end{macro}
%
%    \begin{macro}{\HoLogo@SliTeX@simple}
%    \begin{macrocode}
\def\HoLogo@SliTeX@simple#1{%
  \HoLogoFont@font{SliTeX}{rm}{%
    \ltx@mbox{%
      \HoLogoFont@font{SliTeX}{sc}{Sli}%
    }%
    \HOLOGO@discretionary
    \hologo{TeX}%
  }%
}
%    \end{macrocode}
%    \end{macro}
%    \begin{macro}{\HoLogoBkm@SliTeX@simple}
%    \begin{macrocode}
\def\HoLogoBkm@SliTeX@simple#1{SliTeX}
%    \end{macrocode}
%    \end{macro}
%    \begin{macro}{\HoLogoHtml@SliTeX@simple}
%    \begin{macrocode}
\let\HoLogoHtml@SliTeX@simple\HoLogo@SliTeX@simple
%    \end{macrocode}
%    \end{macro}
%
%    \begin{macro}{\HoLogo@SliTeX@narrow}
%    \begin{macrocode}
\def\HoLogo@SliTeX@narrow#1{%
  \HoLogoFont@font{SliTeX}{rm}{%
    \ltx@mbox{%
      S%
      \kern-.06em%
      \HoLogoFont@font{SliTeX}{sc}{%
        l%
        \kern-.035em%
        i%
      }%
    }%
    \HOLOGO@discretionary
    \kern-.06em%
    \hologo{TeX}%
  }%
}
%    \end{macrocode}
%    \end{macro}
%    \begin{macro}{\HoLogoBkm@SliTeX@narrow}
%    \begin{macrocode}
\def\HoLogoBkm@SliTeX@narrow#1{SliTeX}
%    \end{macrocode}
%    \end{macro}
%    \begin{macro}{\HoLogoHtml@SliTeX@narrow}
%    \begin{macrocode}
\def\HoLogoHtml@SliTeX@narrow#1{%
  \HoLogoCss@SliTeX@narrow
  \HOLOGO@Span{SliTeX-narrow}{%
    \HoLogoFont@font{SliTeX}{rm}{%
      S%
        \HOLOGO@Span{l}{l}%
        \HOLOGO@Span{i}{i}%
      \hologo{TeX}%
    }%
  }%
}
%    \end{macrocode}
%    \end{macro}
%    \begin{macro}{\HoLogoCss@SliTeX@narrow}
%    \begin{macrocode}
\def\HoLogoCss@SliTeX@narrow{%
  \Css{%
    span.HoLogo-SliTeX-narrow span.HoLogo-l{%
      margin-left:-.06em;%
      margin-right:-.035em;%
      font-variant:small-caps;%
    }%
  }%
  \Css{%
    span.HoLogo-SliTeX-narrow span.HoLogo-i{%
      margin-right:-.06em;%
      font-variant:small-caps;%
    }%
  }%
  \global\let\HoLogoCss@SliTeX@narrow\relax
}
%    \end{macrocode}
%    \end{macro}
%
% \paragraph{Macro set completion.}
%
%    \begin{macro}{\HoLogo@SLiTeX@simple}
%    \begin{macrocode}
\def\HoLogo@SLiTeX@simple{\HoLogo@SliTeX@simple}
%    \end{macrocode}
%    \end{macro}
%    \begin{macro}{\HoLogoBkm@SLiTeX@simple}
%    \begin{macrocode}
\def\HoLogoBkm@SLiTeX@simple{\HoLogoBkm@SliTeX@simple}
%    \end{macrocode}
%    \end{macro}
%    \begin{macro}{\HoLogoHtml@SLiTeX@simple}
%    \begin{macrocode}
\def\HoLogoHtml@SLiTeX@simple{\HoLogoHtml@SliTeX@simple}
%    \end{macrocode}
%    \end{macro}
%
%    \begin{macro}{\HoLogo@SLiTeX@narrow}
%    \begin{macrocode}
\def\HoLogo@SLiTeX@narrow{\HoLogo@SliTeX@narrow}
%    \end{macrocode}
%    \end{macro}
%    \begin{macro}{\HoLogoBkm@SLiTeX@narrow}
%    \begin{macrocode}
\def\HoLogoBkm@SLiTeX@narrow{\HoLogoBkm@SliTeX@narrow}
%    \end{macrocode}
%    \end{macro}
%    \begin{macro}{\HoLogoHtml@SLiTeX@narrow}
%    \begin{macrocode}
\def\HoLogoHtml@SLiTeX@narrow{\HoLogoHtml@SliTeX@narrow}
%    \end{macrocode}
%    \end{macro}
%
%    \begin{macro}{\HoLogo@SliTeX@lift}
%    \begin{macrocode}
\def\HoLogo@SliTeX@lift{\HoLogo@SLiTeX@lift}
%    \end{macrocode}
%    \end{macro}
%    \begin{macro}{\HoLogoBkm@SliTeX@lift}
%    \begin{macrocode}
\def\HoLogoBkm@SliTeX@lift{\HoLogoBkm@SLiTeX@lift}
%    \end{macrocode}
%    \end{macro}
%    \begin{macro}{\HoLogoHtml@SliTeX@lift}
%    \begin{macrocode}
\def\HoLogoHtml@SliTeX@lift{\HoLogoHtml@SLiTeX@lift}
%    \end{macrocode}
%    \end{macro}
%
% \paragraph{Defaults.}
%
%    \begin{macro}{\HoLogo@SLiTeX}
%    \begin{macrocode}
\def\HoLogo@SLiTeX{\HoLogo@SLiTeX@lift}
%    \end{macrocode}
%    \end{macro}
%    \begin{macro}{\HoLogoBkm@SLiTeX}
%    \begin{macrocode}
\def\HoLogoBkm@SLiTeX{\HoLogoBkm@SLiTeX@lift}
%    \end{macrocode}
%    \end{macro}
%    \begin{macro}{\HoLogoHtml@SLiTeX}
%    \begin{macrocode}
\def\HoLogoHtml@SLiTeX{\HoLogoHtml@SLiTeX@lift}
%    \end{macrocode}
%    \end{macro}
%
%    \begin{macro}{\HoLogo@SliTeX}
%    \begin{macrocode}
\def\HoLogo@SliTeX{\HoLogo@SliTeX@narrow}
%    \end{macrocode}
%    \end{macro}
%    \begin{macro}{\HoLogoBkm@SliTeX}
%    \begin{macrocode}
\def\HoLogoBkm@SliTeX{\HoLogoBkm@SliTeX@narrow}
%    \end{macrocode}
%    \end{macro}
%    \begin{macro}{\HoLogoHtml@SliTeX}
%    \begin{macrocode}
\def\HoLogoHtml@SliTeX{\HoLogoHtml@SliTeX@narrow}
%    \end{macrocode}
%    \end{macro}
%
% \subsubsection{\hologo{LuaTeX}}
%
%    \begin{macro}{\HoLogo@LuaTeX}
%    The kerning is an idea of Hans Hagen, see mailing list
%    `luatex at tug dot org' in March 2010.
%    \begin{macrocode}
\def\HoLogo@LuaTeX#1{%
  \HOLOGO@mbox{%
    Lua%
    \HOLOGO@NegativeKerning{aT,oT,To}%
    \hologo{TeX}%
  }%
}
%    \end{macrocode}
%    \end{macro}
%    \begin{macro}{\HoLogoHtml@LuaTeX}
%    \begin{macrocode}
\let\HoLogoHtml@LuaTeX\HoLogo@LuaTeX
%    \end{macrocode}
%    \end{macro}
%
% \subsubsection{\hologo{LuaLaTeX}}
%
%    \begin{macro}{\HoLogo@LuaLaTeX}
%    \begin{macrocode}
\def\HoLogo@LuaLaTeX#1{%
  \HOLOGO@mbox{%
    Lua%
    \hologo{LaTeX}%
  }%
}
%    \end{macrocode}
%    \end{macro}
%    \begin{macro}{\HoLogoHtml@LuaLaTeX}
%    \begin{macrocode}
\let\HoLogoHtml@LuaLaTeX\HoLogo@LuaLaTeX
%    \end{macrocode}
%    \end{macro}
%
% \subsubsection{\hologo{XeTeX}, \hologo{XeLaTeX}}
%
%    \begin{macro}{\HOLOGO@IfCharExists}
%    \begin{macrocode}
\ifluatex
  \ifnum\luatexversion<36 %
  \else
    \def\HOLOGO@IfCharExists#1{%
      \ifnum
        \directlua{%
           if luaotfload and luaotfload.aux then
             if luaotfload.aux.font_has_glyph(%
                    font.current(), \number#1) then % 	 
	       tex.print("1") % 	 
	     end % 	 
	   elseif font and font.fonts and font.current then %
            local f = font.fonts[font.current()]%
            if f.characters and f.characters[\number#1] then %
              tex.print("1")%
            end %
          end%
        }0=\ltx@zero
        \expandafter\ltx@secondoftwo
      \else
        \expandafter\ltx@firstoftwo
      \fi
    }%
  \fi
\fi
\ltx@IfUndefined{HOLOGO@IfCharExists}{%
  \def\HOLOGO@@IfCharExists#1{%
    \begingroup
      \tracinglostchars=\ltx@zero
      \setbox\ltx@zero=\hbox{%
        \kern7sp\char#1\relax
        \ifnum\lastkern>\ltx@zero
          \expandafter\aftergroup\csname iffalse\endcsname
        \else
          \expandafter\aftergroup\csname iftrue\endcsname
        \fi
      }%
      % \if{true|false} from \aftergroup
      \endgroup
      \expandafter\ltx@firstoftwo
    \else
      \endgroup
      \expandafter\ltx@secondoftwo
    \fi
  }%
  \ifxetex
    \ltx@IfUndefined{XeTeXfonttype}{}{%
      \ltx@IfUndefined{XeTeXcharglyph}{}{%
        \def\HOLOGO@IfCharExists#1{%
          \ifnum\XeTeXfonttype\font>\ltx@zero
            \expandafter\ltx@firstofthree
          \else
            \expandafter\ltx@gobble
          \fi
          {%
            \ifnum\XeTeXcharglyph#1>\ltx@zero
              \expandafter\ltx@firstoftwo
            \else
              \expandafter\ltx@secondoftwo
            \fi
          }%
          \HOLOGO@@IfCharExists{#1}%
        }%
      }%
    }%
  \fi
}{}
\ltx@ifundefined{HOLOGO@IfCharExists}{%
  \ifnum64=`\^^^^0040\relax % test for big chars of LuaTeX/XeTeX
    \let\HOLOGO@IfCharExists\HOLOGO@@IfCharExists
  \else
    \def\HOLOGO@IfCharExists#1{%
      \ifnum#1>255 %
        \expandafter\ltx@fourthoffour
      \fi
      \HOLOGO@@IfCharExists{#1}%
    }%
  \fi
}{}
%    \end{macrocode}
%    \end{macro}
%
%    \begin{macro}{\HoLogo@Xe}
%    Source: package \xpackage{dtklogos}
%    \begin{macrocode}
\def\HoLogo@Xe#1{%
  X%
  \kern-.1em\relax
  \HOLOGO@IfCharExists{"018E}{%
    \lower.5ex\hbox{\char"018E}%
  }{%
    \chardef\HOLOGO@choice=\ltx@zero
    \ifdim\fontdimen\ltx@one\font>0pt %
      \ltx@IfUndefined{rotatebox}{%
        \ltx@IfUndefined{pgftext}{%
          \ltx@IfUndefined{psscalebox}{%
            \ltx@IfUndefined{HOLOGO@ScaleBox@\hologoDriver}{%
            }{%
              \chardef\HOLOGO@choice=4 %
            }%
          }{%
            \chardef\HOLOGO@choice=3 %
          }%
        }{%
          \chardef\HOLOGO@choice=2 %
        }%
      }{%
        \chardef\HOLOGO@choice=1 %
      }%
      \ifcase\HOLOGO@choice
        \HOLOGO@WarningUnsupportedDriver{Xe}%
        e%
      \or % 1: \rotatebox
        \begingroup
          \setbox\ltx@zero\hbox{\rotatebox{180}{E}}%
          \ltx@LocDimenA=\dp\ltx@zero
          \advance\ltx@LocDimenA by -.5ex\relax
          \raise\ltx@LocDimenA\box\ltx@zero
        \endgroup
      \or % 2: \pgftext
        \lower.5ex\hbox{%
          \pgfpicture
            \pgftext[rotate=180]{E}%
          \endpgfpicture
        }%
      \or % 3: \psscalebox
        \begingroup
          \setbox\ltx@zero\hbox{\psscalebox{-1 -1}{E}}%
          \ltx@LocDimenA=\dp\ltx@zero
          \advance\ltx@LocDimenA by -.5ex\relax
          \raise\ltx@LocDimenA\box\ltx@zero
        \endgroup
      \or % 4: \HOLOGO@PointReflectBox
        \lower.5ex\hbox{\HOLOGO@PointReflectBox{E}}%
      \else
        \@PackageError{hologo}{Internal error (choice/it}\@ehc
      \fi
    \else
      \ltx@IfUndefined{reflectbox}{%
        \ltx@IfUndefined{pgftext}{%
          \ltx@IfUndefined{psscalebox}{%
            \ltx@IfUndefined{HOLOGO@ScaleBox@\hologoDriver}{%
            }{%
              \chardef\HOLOGO@choice=4 %
            }%
          }{%
            \chardef\HOLOGO@choice=3 %
          }%
        }{%
          \chardef\HOLOGO@choice=2 %
        }%
      }{%
        \chardef\HOLOGO@choice=1 %
      }%
      \ifcase\HOLOGO@choice
        \HOLOGO@WarningUnsupportedDriver{Xe}%
        e%
      \or % 1: reflectbox
        \lower.5ex\hbox{%
          \reflectbox{E}%
        }%
      \or % 2: \pgftext
        \lower.5ex\hbox{%
          \pgfpicture
            \pgftransformxscale{-1}%
            \pgftext{E}%
          \endpgfpicture
        }%
      \or % 3: \psscalebox
        \lower.5ex\hbox{%
          \psscalebox{-1 1}{E}%
        }%
      \or % 4: \HOLOGO@Reflectbox
        \lower.5ex\hbox{%
          \HOLOGO@ReflectBox{E}%
        }%
      \else
        \@PackageError{hologo}{Internal error (choice/up)}\@ehc
      \fi
    \fi
  }%
}
%    \end{macrocode}
%    \end{macro}
%    \begin{macro}{\HoLogoHtml@Xe}
%    \begin{macrocode}
\def\HoLogoHtml@Xe#1{%
  \HoLogoCss@Xe
  \HOLOGO@Span{Xe}{%
    X%
    \HOLOGO@Span{e}{%
      \HCode{&\ltx@hashchar x018e;}%
    }%
  }%
}
%    \end{macrocode}
%    \end{macro}
%    \begin{macro}{\HoLogoCss@Xe}
%    \begin{macrocode}
\def\HoLogoCss@Xe{%
  \Css{%
    span.HoLogo-Xe span.HoLogo-e{%
      position:relative;%
      top:.5ex;%
      left-margin:-.1em;%
    }%
  }%
  \global\let\HoLogoCss@Xe\relax
}
%    \end{macrocode}
%    \end{macro}
%
%    \begin{macro}{\HoLogo@XeTeX}
%    \begin{macrocode}
\def\HoLogo@XeTeX#1{%
  \hologo{Xe}%
  \kern-.15em\relax
  \hologo{TeX}%
}
%    \end{macrocode}
%    \end{macro}
%
%    \begin{macro}{\HoLogoHtml@XeTeX}
%    \begin{macrocode}
\def\HoLogoHtml@XeTeX#1{%
  \HoLogoCss@XeTeX
  \HOLOGO@Span{XeTeX}{%
    \hologo{Xe}%
    \hologo{TeX}%
  }%
}
%    \end{macrocode}
%    \end{macro}
%    \begin{macro}{\HoLogoCss@XeTeX}
%    \begin{macrocode}
\def\HoLogoCss@XeTeX{%
  \Css{%
    span.HoLogo-XeTeX span.HoLogo-TeX{%
      margin-left:-.15em;%
    }%
  }%
  \global\let\HoLogoCss@XeTeX\relax
}
%    \end{macrocode}
%    \end{macro}
%
%    \begin{macro}{\HoLogo@XeLaTeX}
%    \begin{macrocode}
\def\HoLogo@XeLaTeX#1{%
  \hologo{Xe}%
  \kern-.13em%
  \hologo{LaTeX}%
}
%    \end{macrocode}
%    \end{macro}
%    \begin{macro}{\HoLogoHtml@XeLaTeX}
%    \begin{macrocode}
\def\HoLogoHtml@XeLaTeX#1{%
  \HoLogoCss@XeLaTeX
  \HOLOGO@Span{XeLaTeX}{%
    \hologo{Xe}%
    \hologo{LaTeX}%
  }%
}
%    \end{macrocode}
%    \end{macro}
%    \begin{macro}{\HoLogoCss@XeLaTeX}
%    \begin{macrocode}
\def\HoLogoCss@XeLaTeX{%
  \Css{%
    span.HoLogo-XeLaTeX span.HoLogo-Xe{%
      margin-right:-.13em;%
    }%
  }%
  \global\let\HoLogoCss@XeLaTeX\relax
}
%    \end{macrocode}
%    \end{macro}
%
% \subsubsection{\hologo{pdfTeX}, \hologo{pdfLaTeX}}
%
%    \begin{macro}{\HoLogo@pdfTeX}
%    \begin{macrocode}
\def\HoLogo@pdfTeX#1{%
  \HOLOGO@mbox{%
    #1{p}{P}df\hologo{TeX}%
  }%
}
%    \end{macrocode}
%    \end{macro}
%    \begin{macro}{\HoLogoCs@pdfTeX}
%    \begin{macrocode}
\def\HoLogoCs@pdfTeX#1{#1{p}{P}dfTeX}
%    \end{macrocode}
%    \end{macro}
%    \begin{macro}{\HoLogoBkm@pdfTeX}
%    \begin{macrocode}
\def\HoLogoBkm@pdfTeX#1{%
  #1{p}{P}df\hologo{TeX}%
}
%    \end{macrocode}
%    \end{macro}
%    \begin{macro}{\HoLogoHtml@pdfTeX}
%    \begin{macrocode}
\let\HoLogoHtml@pdfTeX\HoLogo@pdfTeX
%    \end{macrocode}
%    \end{macro}
%
%    \begin{macro}{\HoLogo@pdfLaTeX}
%    \begin{macrocode}
\def\HoLogo@pdfLaTeX#1{%
  \HOLOGO@mbox{%
    #1{p}{P}df\hologo{LaTeX}%
  }%
}
%    \end{macrocode}
%    \end{macro}
%    \begin{macro}{\HoLogoCs@pdfLaTeX}
%    \begin{macrocode}
\def\HoLogoCs@pdfLaTeX#1{#1{p}{P}dfLaTeX}
%    \end{macrocode}
%    \end{macro}
%    \begin{macro}{\HoLogoBkm@pdfLaTeX}
%    \begin{macrocode}
\def\HoLogoBkm@pdfLaTeX#1{%
  #1{p}{P}df\hologo{LaTeX}%
}
%    \end{macrocode}
%    \end{macro}
%    \begin{macro}{\HoLogoHtml@pdfLaTeX}
%    \begin{macrocode}
\let\HoLogoHtml@pdfLaTeX\HoLogo@pdfLaTeX
%    \end{macrocode}
%    \end{macro}
%
% \subsubsection{\hologo{VTeX}}
%
%    \begin{macro}{\HoLogo@VTeX}
%    \begin{macrocode}
\def\HoLogo@VTeX#1{%
  \HOLOGO@mbox{%
    V\hologo{TeX}%
  }%
}
%    \end{macrocode}
%    \end{macro}
%    \begin{macro}{\HoLogoHtml@VTeX}
%    \begin{macrocode}
\let\HoLogoHtml@VTeX\HoLogo@VTeX
%    \end{macrocode}
%    \end{macro}
%
% \subsubsection{\hologo{AmS}, \dots}
%
%    Source: class \xclass{amsdtx}
%
%    \begin{macro}{\HoLogo@AmS}
%    \begin{macrocode}
\def\HoLogo@AmS#1{%
  \HoLogoFont@font{AmS}{sy}{%
    A%
    \kern-.1667em%
    \lower.5ex\hbox{M}%
    \kern-.125em%
    S%
  }%
}
%    \end{macrocode}
%    \end{macro}
%    \begin{macro}{\HoLogoBkm@AmS}
%    \begin{macrocode}
\def\HoLogoBkm@AmS#1{AmS}
%    \end{macrocode}
%    \end{macro}
%    \begin{macro}{\HoLogoHtml@AmS}
%    \begin{macrocode}
\def\HoLogoHtml@AmS#1{%
  \HoLogoCss@AmS
%  \HoLogoFont@font{AmS}{sy}{%
    \HOLOGO@Span{AmS}{%
      A%
      \HOLOGO@Span{M}{M}%
      S%
    }%
%   }%
}
%    \end{macrocode}
%    \end{macro}
%    \begin{macro}{\HoLogoCss@AmS}
%    \begin{macrocode}
\def\HoLogoCss@AmS{%
  \Css{%
    span.HoLogo-AmS span.HoLogo-M{%
      position:relative;%
      top:.5ex;%
      margin-left:-.1667em;%
      margin-right:-.125em;%
      text-decoration:none;%
    }%
  }%
  \global\let\HoLogoCss@AmS\relax
}
%    \end{macrocode}
%    \end{macro}
%
%    \begin{macro}{\HoLogo@AmSTeX}
%    \begin{macrocode}
\def\HoLogo@AmSTeX#1{%
  \hologo{AmS}%
  \HOLOGO@hyphen
  \hologo{TeX}%
}
%    \end{macrocode}
%    \end{macro}
%    \begin{macro}{\HoLogoBkm@AmSTeX}
%    \begin{macrocode}
\def\HoLogoBkm@AmSTeX#1{AmS-TeX}%
%    \end{macrocode}
%    \end{macro}
%    \begin{macro}{\HoLogoHtml@AmSTeX}
%    \begin{macrocode}
\let\HoLogoHtml@AmSTeX\HoLogo@AmSTeX
%    \end{macrocode}
%    \end{macro}
%
%    \begin{macro}{\HoLogo@AmSLaTeX}
%    \begin{macrocode}
\def\HoLogo@AmSLaTeX#1{%
  \hologo{AmS}%
  \HOLOGO@hyphen
  \hologo{LaTeX}%
}
%    \end{macrocode}
%    \end{macro}
%    \begin{macro}{\HoLogoBkm@AmSLaTeX}
%    \begin{macrocode}
\def\HoLogoBkm@AmSLaTeX#1{AmS-LaTeX}%
%    \end{macrocode}
%    \end{macro}
%    \begin{macro}{\HoLogoHtml@AmSLaTeX}
%    \begin{macrocode}
\let\HoLogoHtml@AmSLaTeX\HoLogo@AmSLaTeX
%    \end{macrocode}
%    \end{macro}
%
% \subsubsection{\hologo{BibTeX}}
%
%    \begin{macro}{\HoLogo@BibTeX@sc}
%    A definition of \hologo{BibTeX} is provided in
%    the documentation source for the manual of \hologo{BibTeX}
%    \cite{btxdoc}.
%\begin{quote}
%\begin{verbatim}
%\def\BibTeX{%
%  {%
%    \rm
%    B%
%    \kern-.05em%
%    {%
%      \sc
%      i%
%      \kern-.025em %
%      b%
%    }%
%    \kern-.08em
%    T%
%    \kern-.1667em%
%    \lower.7ex\hbox{E}%
%    \kern-.125em%
%    X%
%  }%
%}
%\end{verbatim}
%\end{quote}
%    \begin{macrocode}
\def\HoLogo@BibTeX@sc#1{%
  B%
  \kern-.05em%
  \HoLogoFont@font{BibTeX}{sc}{%
    i%
    \kern-.025em%
    b%
  }%
  \HOLOGO@discretionary
  \kern-.08em%
  \hologo{TeX}%
}
%    \end{macrocode}
%    \end{macro}
%    \begin{macro}{\HoLogoHtml@BibTeX@sc}
%    \begin{macrocode}
\def\HoLogoHtml@BibTeX@sc#1{%
  \HoLogoCss@BibTeX@sc
  \HOLOGO@Span{BibTeX-sc}{%
    B%
    \HOLOGO@Span{i}{i}%
    \HOLOGO@Span{b}{b}%
    \hologo{TeX}%
  }%
}
%    \end{macrocode}
%    \end{macro}
%    \begin{macro}{\HoLogoCss@BibTeX@sc}
%    \begin{macrocode}
\def\HoLogoCss@BibTeX@sc{%
  \Css{%
    span.HoLogo-BibTeX-sc span.HoLogo-i{%
      margin-left:-.05em;%
      margin-right:-.025em;%
      font-variant:small-caps;%
    }%
  }%
  \Css{%
    span.HoLogo-BibTeX-sc span.HoLogo-b{%
      margin-right:-.08em;%
      font-variant:small-caps;%
    }%
  }%
  \global\let\HoLogoCss@BibTeX@sc\relax
}
%    \end{macrocode}
%    \end{macro}
%
%    \begin{macro}{\HoLogo@BibTeX@sf}
%    Variant \xoption{sf} avoids trouble with unavailable
%    small caps fonts (e.g., bold versions of Computer Modern or
%    Latin Modern). The definition is taken from
%    package \xpackage{dtklogos} \cite{dtklogos}.
%\begin{quote}
%\begin{verbatim}
%\DeclareRobustCommand{\BibTeX}{%
%  B%
%  \kern-.05em%
%  \hbox{%
%    $\m@th$% %% force math size calculations
%    \csname S@\f@size\endcsname
%    \fontsize\sf@size\z@
%    \math@fontsfalse
%    \selectfont
%    I%
%    \kern-.025em%
%    B
%  }%
%  \kern-.08em%
%  \-%
%  \TeX
%}
%\end{verbatim}
%\end{quote}
%    \begin{macrocode}
\def\HoLogo@BibTeX@sf#1{%
  B%
  \kern-.05em%
  \HoLogoFont@font{BibTeX}{bibsf}{%
    I%
    \kern-.025em%
    B%
  }%
  \HOLOGO@discretionary
  \kern-.08em%
  \hologo{TeX}%
}
%    \end{macrocode}
%    \end{macro}
%    \begin{macro}{\HoLogoHtml@BibTeX@sf}
%    \begin{macrocode}
\def\HoLogoHtml@BibTeX@sf#1{%
  \HoLogoCss@BibTeX@sf
  \HOLOGO@Span{BibTeX-sf}{%
    B%
    \HoLogoFont@font{BibTeX}{bibsf}{%
      \HOLOGO@Span{i}{I}%
      B%
    }%
    \hologo{TeX}%
  }%
}
%    \end{macrocode}
%    \end{macro}
%    \begin{macro}{\HoLogoCss@BibTeX@sf}
%    \begin{macrocode}
\def\HoLogoCss@BibTeX@sf{%
  \Css{%
    span.HoLogo-BibTeX-sf span.HoLogo-i{%
      margin-left:-.05em;%
      margin-right:-.025em;%
    }%
  }%
  \Css{%
    span.HoLogo-BibTeX-sf span.HoLogo-TeX{%
      margin-left:-.08em;%
    }%
  }%
  \global\let\HoLogoCss@BibTeX@sf\relax
}
%    \end{macrocode}
%    \end{macro}
%
%    \begin{macro}{\HoLogo@BibTeX}
%    \begin{macrocode}
\def\HoLogo@BibTeX{\HoLogo@BibTeX@sf}
%    \end{macrocode}
%    \end{macro}
%    \begin{macro}{\HoLogoHtml@BibTeX}
%    \begin{macrocode}
\def\HoLogoHtml@BibTeX{\HoLogoHtml@BibTeX@sf}
%    \end{macrocode}
%    \end{macro}
%
% \subsubsection{\hologo{BibTeX8}}
%
%    \begin{macro}{\HoLogo@BibTeX8}
%    \begin{macrocode}
\expandafter\def\csname HoLogo@BibTeX8\endcsname#1{%
  \hologo{BibTeX}%
  8%
}
%    \end{macrocode}
%    \end{macro}
%
%    \begin{macro}{\HoLogoBkm@BibTeX8}
%    \begin{macrocode}
\expandafter\def\csname HoLogoBkm@BibTeX8\endcsname#1{%
  \hologo{BibTeX}%
  8%
}
%    \end{macrocode}
%    \end{macro}
%    \begin{macro}{\HoLogoHtml@BibTeX8}
%    \begin{macrocode}
\expandafter
\let\csname HoLogoHtml@BibTeX8\expandafter\endcsname
\csname HoLogo@BibTeX8\endcsname
%    \end{macrocode}
%    \end{macro}
%
% \subsubsection{\hologo{ConTeXt}}
%
%    \begin{macro}{\HoLogo@ConTeXt@simple}
%    \begin{macrocode}
\def\HoLogo@ConTeXt@simple#1{%
  \HOLOGO@mbox{Con}%
  \HOLOGO@discretionary
  \HOLOGO@mbox{\hologo{TeX}t}%
}
%    \end{macrocode}
%    \end{macro}
%    \begin{macro}{\HoLogoHtml@ConTeXt@simple}
%    \begin{macrocode}
\let\HoLogoHtml@ConTeXt@simple\HoLogo@ConTeXt@simple
%    \end{macrocode}
%    \end{macro}
%
%    \begin{macro}{\HoLogo@ConTeXt@narrow}
%    This definition of logo \hologo{ConTeXt} with variant \xoption{narrow}
%    comes from TUGboat's class \xclass{ltugboat} (version 2010/11/15 v2.8).
%    \begin{macrocode}
\def\HoLogo@ConTeXt@narrow#1{%
  \HOLOGO@mbox{C\kern-.0333emon}%
  \HOLOGO@discretionary
  \kern-.0667em%
  \HOLOGO@mbox{\hologo{TeX}\kern-.0333emt}%
}
%    \end{macrocode}
%    \end{macro}
%    \begin{macro}{\HoLogoHtml@ConTeXt@narrow}
%    \begin{macrocode}
\def\HoLogoHtml@ConTeXt@narrow#1{%
  \HoLogoCss@ConTeXt@narrow
  \HOLOGO@Span{ConTeXt-narrow}{%
    \HOLOGO@Span{C}{C}%
    on%
    \hologo{TeX}%
    t%
  }%
}
%    \end{macrocode}
%    \end{macro}
%    \begin{macro}{\HoLogoCss@ConTeXt@narrow}
%    \begin{macrocode}
\def\HoLogoCss@ConTeXt@narrow{%
  \Css{%
    span.HoLogo-ConTeXt-narrow span.HoLogo-C{%
      margin-left:-.0333em;%
    }%
  }%
  \Css{%
    span.HoLogo-ConTeXt-narrow span.HoLogo-TeX{%
      margin-left:-.0667em;%
      margin-right:-.0333em;%
    }%
  }%
  \global\let\HoLogoCss@ConTeXt@narrow\relax
}
%    \end{macrocode}
%    \end{macro}
%
%    \begin{macro}{\HoLogo@ConTeXt}
%    \begin{macrocode}
\def\HoLogo@ConTeXt{\HoLogo@ConTeXt@narrow}
%    \end{macrocode}
%    \end{macro}
%    \begin{macro}{\HoLogoHtml@ConTeXt}
%    \begin{macrocode}
\def\HoLogoHtml@ConTeXt{\HoLogoHtml@ConTeXt@narrow}
%    \end{macrocode}
%    \end{macro}
%
% \subsubsection{\hologo{emTeX}}
%
%    \begin{macro}{\HoLogo@emTeX}
%    \begin{macrocode}
\def\HoLogo@emTeX#1{%
  \HOLOGO@mbox{#1{e}{E}m}%
  \HOLOGO@discretionary
  \hologo{TeX}%
}
%    \end{macrocode}
%    \end{macro}
%    \begin{macro}{\HoLogoCs@emTeX}
%    \begin{macrocode}
\def\HoLogoCs@emTeX#1{#1{e}{E}mTeX}%
%    \end{macrocode}
%    \end{macro}
%    \begin{macro}{\HoLogoBkm@emTeX}
%    \begin{macrocode}
\def\HoLogoBkm@emTeX#1{%
  #1{e}{E}m\hologo{TeX}%
}
%    \end{macrocode}
%    \end{macro}
%    \begin{macro}{\HoLogoHtml@emTeX}
%    \begin{macrocode}
\let\HoLogoHtml@emTeX\HoLogo@emTeX
%    \end{macrocode}
%    \end{macro}
%
% \subsubsection{\hologo{ExTeX}}
%
%    \begin{macro}{\HoLogo@ExTeX}
%    The definition is taken from the FAQ of the
%    project \hologo{ExTeX}
%    \cite{ExTeX-FAQ}.
%\begin{quote}
%\begin{verbatim}
%\def\ExTeX{%
%  \textrm{% Logo always with serifs
%    \ensuremath{%
%      \textstyle
%      \varepsilon_{%
%        \kern-0.15em%
%        \mathcal{X}%
%      }%
%    }%
%    \kern-.15em%
%    \TeX
%  }%
%}
%\end{verbatim}
%\end{quote}
%    \begin{macrocode}
\def\HoLogo@ExTeX#1{%
  \HoLogoFont@font{ExTeX}{rm}{%
    \ltx@mbox{%
      \HOLOGO@MathSetup
      $%
        \textstyle
        \varepsilon_{%
          \kern-0.15em%
          \HoLogoFont@font{ExTeX}{sy}{X}%
        }%
      $%
    }%
    \HOLOGO@discretionary
    \kern-.15em%
    \hologo{TeX}%
  }%
}
%    \end{macrocode}
%    \end{macro}
%    \begin{macro}{\HoLogoHtml@ExTeX}
%    \begin{macrocode}
\def\HoLogoHtml@ExTeX#1{%
  \HoLogoCss@ExTeX
  \HoLogoFont@font{ExTeX}{rm}{%
    \HOLOGO@Span{ExTeX}{%
      \ltx@mbox{%
        \HOLOGO@MathSetup
        $\textstyle\varepsilon$%
        \HOLOGO@Span{X}{$\textstyle\chi$}%
        \hologo{TeX}%
      }%
    }%
  }%
}
%    \end{macrocode}
%    \end{macro}
%    \begin{macro}{\HoLogoBkm@ExTeX}
%    \begin{macrocode}
\def\HoLogoBkm@ExTeX#1{%
  \HOLOGO@PdfdocUnicode{#1{e}{E}x}{\textepsilon\textchi}%
  \hologo{TeX}%
}
%    \end{macrocode}
%    \end{macro}
%    \begin{macro}{\HoLogoCss@ExTeX}
%    \begin{macrocode}
\def\HoLogoCss@ExTeX{%
  \Css{%
    span.HoLogo-ExTeX{%
      font-family:serif;%
    }%
  }%
  \Css{%
    span.HoLogo-ExTeX span.HoLogo-TeX{%
      margin-left:-.15em;%
    }%
  }%
  \global\let\HoLogoCss@ExTeX\relax
}
%    \end{macrocode}
%    \end{macro}
%
% \subsubsection{\hologo{MiKTeX}}
%
%    \begin{macro}{\HoLogo@MiKTeX}
%    \begin{macrocode}
\def\HoLogo@MiKTeX#1{%
  \HOLOGO@mbox{MiK}%
  \HOLOGO@discretionary
  \hologo{TeX}%
}
%    \end{macrocode}
%    \end{macro}
%    \begin{macro}{\HoLogoHtml@MiKTeX}
%    \begin{macrocode}
\let\HoLogoHtml@MiKTeX\HoLogo@MiKTeX
%    \end{macrocode}
%    \end{macro}
%
% \subsubsection{\hologo{OzTeX} and friends}
%
%    Source: \hologo{OzTeX} FAQ \cite{OzTeX}:
%    \begin{quote}
%      |\def\OzTeX{O\kern-.03em z\kern-.15em\TeX}|\\
%      (There is no kerning in OzMF, OzMP and OzTtH.)
%    \end{quote}
%
%    \begin{macro}{\HoLogo@OzTeX}
%    \begin{macrocode}
\def\HoLogo@OzTeX#1{%
  O%
  \kern-.03em %
  z%
  \kern-.15em %
  \hologo{TeX}%
}
%    \end{macrocode}
%    \end{macro}
%    \begin{macro}{\HoLogoHtml@OzTeX}
%    \begin{macrocode}
\def\HoLogoHtml@OzTeX#1{%
  \HoLogoCss@OzTeX
  \HOLOGO@Span{OzTeX}{%
    O%
    \HOLOGO@Span{z}{z}%
    \hologo{TeX}%
  }%
}
%    \end{macrocode}
%    \end{macro}
%    \begin{macro}{\HoLogoCss@OzTeX}
%    \begin{macrocode}
\def\HoLogoCss@OzTeX{%
  \Css{%
    span.HoLogo-OzTeX span.HoLogo-z{%
      margin-left:-.03em;%
      margin-right:-.15em;%
    }%
  }%
  \global\let\HoLogoCss@OzTeX\relax
}
%    \end{macrocode}
%    \end{macro}
%
%    \begin{macro}{\HoLogo@OzMF}
%    \begin{macrocode}
\def\HoLogo@OzMF#1{%
  \HOLOGO@mbox{OzMF}%
}
%    \end{macrocode}
%    \end{macro}
%    \begin{macro}{\HoLogo@OzMP}
%    \begin{macrocode}
\def\HoLogo@OzMP#1{%
  \HOLOGO@mbox{OzMP}%
}
%    \end{macrocode}
%    \end{macro}
%    \begin{macro}{\HoLogo@OzTtH}
%    \begin{macrocode}
\def\HoLogo@OzTtH#1{%
  \HOLOGO@mbox{OzTtH}%
}
%    \end{macrocode}
%    \end{macro}
%
% \subsubsection{\hologo{PCTeX}}
%
%    \begin{macro}{\HoLogo@PCTeX}
%    \begin{macrocode}
\def\HoLogo@PCTeX#1{%
  \HOLOGO@mbox{PC}%
  \hologo{TeX}%
}
%    \end{macrocode}
%    \end{macro}
%    \begin{macro}{\HoLogoHtml@PCTeX}
%    \begin{macrocode}
\let\HoLogoHtml@PCTeX\HoLogo@PCTeX
%    \end{macrocode}
%    \end{macro}
%
% \subsubsection{\hologo{PiCTeX}}
%
%    The original definitions from \xfile{pictex.tex} \cite{PiCTeX}:
%\begin{quote}
%\begin{verbatim}
%\def\PiC{%
%  P%
%  \kern-.12em%
%  \lower.5ex\hbox{I}%
%  \kern-.075em%
%  C%
%}
%\def\PiCTeX{%
%  \PiC
%  \kern-.11em%
%  \TeX
%}
%\end{verbatim}
%\end{quote}
%
%    \begin{macro}{\HoLogo@PiC}
%    \begin{macrocode}
\def\HoLogo@PiC#1{%
  P%
  \kern-.12em%
  \lower.5ex\hbox{I}%
  \kern-.075em%
  C%
  \HOLOGO@SpaceFactor
}
%    \end{macrocode}
%    \end{macro}
%    \begin{macro}{\HoLogoHtml@PiC}
%    \begin{macrocode}
\def\HoLogoHtml@PiC#1{%
  \HoLogoCss@PiC
  \HOLOGO@Span{PiC}{%
    P%
    \HOLOGO@Span{i}{I}%
    C%
  }%
}
%    \end{macrocode}
%    \end{macro}
%    \begin{macro}{\HoLogoCss@PiC}
%    \begin{macrocode}
\def\HoLogoCss@PiC{%
  \Css{%
    span.HoLogo-PiC span.HoLogo-i{%
      position:relative;%
      top:.5ex;%
      margin-left:-.12em;%
      margin-right:-.075em;%
      text-decoration:none;%
    }%
  }%
  \global\let\HoLogoCss@PiC\relax
}
%    \end{macrocode}
%    \end{macro}
%
%    \begin{macro}{\HoLogo@PiCTeX}
%    \begin{macrocode}
\def\HoLogo@PiCTeX#1{%
  \hologo{PiC}%
  \HOLOGO@discretionary
  \kern-.11em%
  \hologo{TeX}%
}
%    \end{macrocode}
%    \end{macro}
%    \begin{macro}{\HoLogoHtml@PiCTeX}
%    \begin{macrocode}
\def\HoLogoHtml@PiCTeX#1{%
  \HoLogoCss@PiCTeX
  \HOLOGO@Span{PiCTeX}{%
    \hologo{PiC}%
    \hologo{TeX}%
  }%
}
%    \end{macrocode}
%    \end{macro}
%    \begin{macro}{\HoLogoCss@PiCTeX}
%    \begin{macrocode}
\def\HoLogoCss@PiCTeX{%
  \Css{%
    span.HoLogo-PiCTeX span.HoLogo-PiC{%
      margin-right:-.11em;%
    }%
  }%
  \global\let\HoLogoCss@PiCTeX\relax
}
%    \end{macrocode}
%    \end{macro}
%
% \subsubsection{\hologo{teTeX}}
%
%    \begin{macro}{\HoLogo@teTeX}
%    \begin{macrocode}
\def\HoLogo@teTeX#1{%
  \HOLOGO@mbox{#1{t}{T}e}%
  \HOLOGO@discretionary
  \hologo{TeX}%
}
%    \end{macrocode}
%    \end{macro}
%    \begin{macro}{\HoLogoCs@teTeX}
%    \begin{macrocode}
\def\HoLogoCs@teTeX#1{#1{t}{T}dfTeX}
%    \end{macrocode}
%    \end{macro}
%    \begin{macro}{\HoLogoBkm@teTeX}
%    \begin{macrocode}
\def\HoLogoBkm@teTeX#1{%
  #1{t}{T}e\hologo{TeX}%
}
%    \end{macrocode}
%    \end{macro}
%    \begin{macro}{\HoLogoHtml@teTeX}
%    \begin{macrocode}
\let\HoLogoHtml@teTeX\HoLogo@teTeX
%    \end{macrocode}
%    \end{macro}
%
% \subsubsection{\hologo{TeX4ht}}
%
%    \begin{macro}{\HoLogo@TeX4ht}
%    \begin{macrocode}
\expandafter\def\csname HoLogo@TeX4ht\endcsname#1{%
  \HOLOGO@mbox{\hologo{TeX}4ht}%
}
%    \end{macrocode}
%    \end{macro}
%    \begin{macro}{\HoLogoHtml@TeX4ht}
%    \begin{macrocode}
\expandafter
\let\csname HoLogoHtml@TeX4ht\expandafter\endcsname
\csname HoLogo@TeX4ht\endcsname
%    \end{macrocode}
%    \end{macro}
%
%
% \subsubsection{\hologo{SageTeX}}
%
%    \begin{macro}{\HoLogo@SageTeX}
%    \begin{macrocode}
\def\HoLogo@SageTeX#1{%
  \HOLOGO@mbox{Sage}%
  \HOLOGO@discretionary
  \HOLOGO@NegativeKerning{eT,oT,To}%
  \hologo{TeX}%
}
%    \end{macrocode}
%    \end{macro}
%    \begin{macro}{\HoLogoHtml@SageTeX}
%    \begin{macrocode}
\let\HoLogoHtml@SageTeX\HoLogo@SageTeX
%    \end{macrocode}
%    \end{macro}
%
% \subsection{\hologo{METAFONT} and friends}
%
%    \begin{macro}{\HoLogo@METAFONT}
%    \begin{macrocode}
\def\HoLogo@METAFONT#1{%
  \HoLogoFont@font{METAFONT}{logo}{%
    \HOLOGO@mbox{META}%
    \HOLOGO@discretionary
    \HOLOGO@mbox{FONT}%
  }%
}
%    \end{macrocode}
%    \end{macro}
%
%    \begin{macro}{\HoLogo@METAPOST}
%    \begin{macrocode}
\def\HoLogo@METAPOST#1{%
  \HoLogoFont@font{METAPOST}{logo}{%
    \HOLOGO@mbox{META}%
    \HOLOGO@discretionary
    \HOLOGO@mbox{POST}%
  }%
}
%    \end{macrocode}
%    \end{macro}
%
%    \begin{macro}{\HoLogo@MetaFun}
%    \begin{macrocode}
\def\HoLogo@MetaFun#1{%
  \HOLOGO@mbox{Meta}%
  \HOLOGO@discretionary
  \HOLOGO@mbox{Fun}%
}
%    \end{macrocode}
%    \end{macro}
%
%    \begin{macro}{\HoLogo@MetaPost}
%    \begin{macrocode}
\def\HoLogo@MetaPost#1{%
  \HOLOGO@mbox{Meta}%
  \HOLOGO@discretionary
  \HOLOGO@mbox{Post}%
}
%    \end{macrocode}
%    \end{macro}
%
% \subsection{Others}
%
% \subsubsection{\hologo{biber}}
%
%    \begin{macro}{\HoLogo@biber}
%    \begin{macrocode}
\def\HoLogo@biber#1{%
  \HOLOGO@mbox{#1{b}{B}i}%
  \HOLOGO@discretionary
  \HOLOGO@mbox{ber}%
}
%    \end{macrocode}
%    \end{macro}
%    \begin{macro}{\HoLogoCs@biber}
%    \begin{macrocode}
\def\HoLogoCs@biber#1{#1{b}{B}iber}
%    \end{macrocode}
%    \end{macro}
%    \begin{macro}{\HoLogoBkm@biber}
%    \begin{macrocode}
\def\HoLogoBkm@biber#1{%
  #1{b}{B}iber%
}
%    \end{macrocode}
%    \end{macro}
%    \begin{macro}{\HoLogoHtml@biber}
%    \begin{macrocode}
\let\HoLogoHtml@biber\HoLogo@biber
%    \end{macrocode}
%    \end{macro}
%
% \subsubsection{\hologo{KOMAScript}}
%
%    \begin{macro}{\HoLogo@KOMAScript}
%    The definition for \hologo{KOMAScript} is taken
%    from \hologo{KOMAScript} (\xfile{scrlogo.dtx}, reformatted) \cite{scrlogo}:
%\begin{quote}
%\begin{verbatim}
%\@ifundefined{KOMAScript}{%
%  \DeclareRobustCommand{\KOMAScript}{%
%    \textsf{%
%      K\kern.05em O\kern.05emM\kern.05em A%
%      \kern.1em-\kern.1em %
%      Script%
%    }%
%  }%
%}{}
%\end{verbatim}
%\end{quote}
%    \begin{macrocode}
\def\HoLogo@KOMAScript#1{%
  \HoLogoFont@font{KOMAScript}{sf}{%
    \HOLOGO@mbox{%
      K\kern.05em%
      O\kern.05em%
      M\kern.05em%
      A%
    }%
    \kern.1em%
    \HOLOGO@hyphen
    \kern.1em%
    \HOLOGO@mbox{Script}%
  }%
}
%    \end{macrocode}
%    \end{macro}
%    \begin{macro}{\HoLogoBkm@KOMAScript}
%    \begin{macrocode}
\def\HoLogoBkm@KOMAScript#1{%
  KOMA-Script%
}
%    \end{macrocode}
%    \end{macro}
%    \begin{macro}{\HoLogoHtml@KOMAScript}
%    \begin{macrocode}
\def\HoLogoHtml@KOMAScript#1{%
  \HoLogoCss@KOMAScript
  \HoLogoFont@font{KOMAScript}{sf}{%
    \HOLOGO@Span{KOMAScript}{%
      K%
      \HOLOGO@Span{O}{O}%
      M%
      \HOLOGO@Span{A}{A}%
      \HOLOGO@Span{hyphen}{-}%
      Script%
    }%
  }%
}
%    \end{macrocode}
%    \end{macro}
%    \begin{macro}{\HoLogoCss@KOMAScript}
%    \begin{macrocode}
\def\HoLogoCss@KOMAScript{%
  \Css{%
    span.HoLogo-KOMAScript{%
      font-family:sans-serif;%
    }%
  }%
  \Css{%
    span.HoLogo-KOMAScript span.HoLogo-O{%
      padding-left:.05em;%
      padding-right:.05em;%
    }%
  }%
  \Css{%
    span.HoLogo-KOMAScript span.HoLogo-A{%
      padding-left:.05em;%
    }%
  }%
  \Css{%
    span.HoLogo-KOMAScript span.HoLogo-hyphen{%
      padding-left:.1em;%
      padding-right:.1em;%
    }%
  }%
  \global\let\HoLogoCss@KOMAScript\relax
}
%    \end{macrocode}
%    \end{macro}
%
% \subsubsection{\hologo{LyX}}
%
%    \begin{macro}{\HoLogo@LyX}
%    The definition is taken from the documentation source files
%    of \hologo{LyX}, \xfile{Intro.lyx} \cite{LyX}:
%\begin{quote}
%\begin{verbatim}
%\def\LyX{%
%  \texorpdfstring{%
%    L\kern-.1667em\lower.25em\hbox{Y}\kern-.125emX\@%
%  }{%
%    LyX%
%  }%
%}
%\end{verbatim}
%\end{quote}
%    \begin{macrocode}
\def\HoLogo@LyX#1{%
  L%
  \kern-.1667em%
  \lower.25em\hbox{Y}%
  \kern-.125em%
  X%
  \HOLOGO@SpaceFactor
}
%    \end{macrocode}
%    \end{macro}
%    \begin{macro}{\HoLogoHtml@LyX}
%    \begin{macrocode}
\def\HoLogoHtml@LyX#1{%
  \HoLogoCss@LyX
  \HOLOGO@Span{LyX}{%
    L%
    \HOLOGO@Span{y}{Y}%
    X%
  }%
}
%    \end{macrocode}
%    \end{macro}
%    \begin{macro}{\HoLogoCss@LyX}
%    \begin{macrocode}
\def\HoLogoCss@LyX{%
  \Css{%
    span.HoLogo-LyX span.HoLogo-y{%
      position:relative;%
      top:.25em;%
      margin-left:-.1667em;%
      margin-right:-.125em;%
      text-decoration:none;%
    }%
  }%
  \global\let\HoLogoCss@LyX\relax
}
%    \end{macrocode}
%    \end{macro}
%
% \subsubsection{\hologo{NTS}}
%
%    \begin{macro}{\HoLogo@NTS}
%    Definition for \hologo{NTS} can be found in
%    package \xpackage{etex\textunderscore man} for the \hologo{eTeX} manual \cite{etexman}
%    and in package \xpackage{dtklogos} \cite{dtklogos}:
%\begin{quote}
%\begin{verbatim}
%\def\NTS{%
%  \leavevmode
%  \hbox{%
%    $%
%      \cal N%
%      \kern-0.35em%
%      \lower0.5ex\hbox{$\cal T$}%
%      \kern-0.2em%
%      S%
%    $%
%  }%
%}
%\end{verbatim}
%\end{quote}
%    \begin{macrocode}
\def\HoLogo@NTS#1{%
  \HoLogoFont@font{NTS}{sy}{%
    N\/%
    \kern-.35em%
    \lower.5ex\hbox{T\/}%
    \kern-.2em%
    S\/%
  }%
  \HOLOGO@SpaceFactor
}
%    \end{macrocode}
%    \end{macro}
%
% \subsubsection{\Hologo{TTH} (\hologo{TeX} to HTML translator)}
%
%    Source: \url{http://hutchinson.belmont.ma.us/tth/}
%    In the HTML source the second `T' is printed as subscript.
%\begin{quote}
%\begin{verbatim}
%T<sub>T</sub>H
%\end{verbatim}
%\end{quote}
%    \begin{macro}{\HoLogo@TTH}
%    \begin{macrocode}
\def\HoLogo@TTH#1{%
  \ltx@mbox{%
    T\HOLOGO@SubScript{T}H%
  }%
  \HOLOGO@SpaceFactor
}
%    \end{macrocode}
%    \end{macro}
%
%    \begin{macro}{\HoLogoHtml@TTH}
%    \begin{macrocode}
\def\HoLogoHtml@TTH#1{%
  T\HCode{<sub>}T\HCode{</sub>}H%
}
%    \end{macrocode}
%    \end{macro}
%
% \subsubsection{\Hologo{HanTheThanh}}
%
%    Partial source: Package \xpackage{dtklogos}.
%    The double accent is U+1EBF (latin small letter e with circumflex
%    and acute).
%    \begin{macro}{\HoLogo@HanTheThanh}
%    \begin{macrocode}
\def\HoLogo@HanTheThanh#1{%
  \ltx@mbox{H\`an}%
  \HOLOGO@space
  \ltx@mbox{%
    Th%
    \HOLOGO@IfCharExists{"1EBF}{%
      \char"1EBF\relax
    }{%
      \^e\hbox to 0pt{\hss\raise .5ex\hbox{\'{}}}%
    }%
  }%
  \HOLOGO@space
  \ltx@mbox{Th\`anh}%
}
%    \end{macrocode}
%    \end{macro}
%    \begin{macro}{\HoLogoBkm@HanTheThanh}
%    \begin{macrocode}
\def\HoLogoBkm@HanTheThanh#1{%
  H\`an %
  Th\HOLOGO@PdfdocUnicode{\^e}{\9036\277} %
  Th\`anh%
}
%    \end{macrocode}
%    \end{macro}
%    \begin{macro}{\HoLogoHtml@HanTheThanh}
%    \begin{macrocode}
\def\HoLogoHtml@HanTheThanh#1{%
  H\`an %
  Th\HCode{&\ltx@hashchar x1ebf;} %
  Th\`anh%
}
%    \end{macrocode}
%    \end{macro}
%
% \subsection{Driver detection}
%
%    \begin{macrocode}
\HOLOGO@IfExists\InputIfFileExists{%
  \InputIfFileExists{hologo.cfg}{}{}%
}{%
  \ltx@IfUndefined{pdf@filesize}{%
    \def\HOLOGO@InputIfExists{%
      \openin\HOLOGO@temp=hologo.cfg\relax
      \ifeof\HOLOGO@temp
        \closein\HOLOGO@temp
      \else
        \closein\HOLOGO@temp
        \begingroup
          \def\x{LaTeX2e}%
        \expandafter\endgroup
        \ifx\fmtname\x
          % \iffalse meta-comment
%
% File: hologo.dtx
% Version: 2016/05/12 v1.11
% Info: A logo collection with bookmark support
%
% Copyright (C) 2010-2012 by
%    Heiko Oberdiek <heiko.oberdiek at googlemail.com>
%
% This work may be distributed and/or modified under the
% conditions of the LaTeX Project Public License, either
% version 1.3c of this license or (at your option) any later
% version. This version of this license is in
%    http://www.latex-project.org/lppl/lppl-1-3c.txt
% and the latest version of this license is in
%    http://www.latex-project.org/lppl.txt
% and version 1.3 or later is part of all distributions of
% LaTeX version 2005/12/01 or later.
%
% This work has the LPPL maintenance status "maintained".
%
% This Current Maintainer of this work is Heiko Oberdiek.
%
% The Base Interpreter refers to any `TeX-Format',
% because some files are installed in TDS:tex/generic//.
%
% This work consists of the main source file hologo.dtx
% and the derived files
%    hologo.sty, hologo.pdf, hologo.ins, hologo.drv, hologo-example.tex,
%    hologo-test1.tex, hologo-test-spacefactor.tex,
%    hologo-test-list.tex.
%
% Distribution:
%    CTAN:macros/latex/contrib/oberdiek/hologo.dtx
%    CTAN:macros/latex/contrib/oberdiek/hologo.pdf
%
% Unpacking:
%    (a) If hologo.ins is present:
%           tex hologo.ins
%    (b) Without hologo.ins:
%           tex hologo.dtx
%    (c) If you insist on using LaTeX
%           latex \let\install=y\input{hologo.dtx}
%        (quote the arguments according to the demands of your shell)
%
% Documentation:
%    (a) If hologo.drv is present:
%           latex hologo.drv
%    (b) Without hologo.drv:
%           latex hologo.dtx; ...
%    The class ltxdoc loads the configuration file ltxdoc.cfg
%    if available. Here you can specify further options, e.g.
%    use A4 as paper format:
%       \PassOptionsToClass{a4paper}{article}
%
%    Programm calls to get the documentation (example):
%       pdflatex hologo.dtx
%       makeindex -s gind.ist hologo.idx
%       pdflatex hologo.dtx
%       makeindex -s gind.ist hologo.idx
%       pdflatex hologo.dtx
%
% Installation:
%    TDS:tex/generic/oberdiek/hologo.sty
%    TDS:doc/latex/oberdiek/hologo.pdf
%    TDS:doc/latex/oberdiek/example/hologo-example.tex
%    TDS:doc/latex/oberdiek/test/hologo-test1.tex
%    TDS:doc/latex/oberdiek/test/hologo-test-spacefactor.tex
%    TDS:doc/latex/oberdiek/test/hologo-test-list.tex
%    TDS:source/latex/oberdiek/hologo.dtx
%
%<*ignore>
\begingroup
  \catcode123=1 %
  \catcode125=2 %
  \def\x{LaTeX2e}%
\expandafter\endgroup
\ifcase 0\ifx\install y1\fi\expandafter
         \ifx\csname processbatchFile\endcsname\relax\else1\fi
         \ifx\fmtname\x\else 1\fi\relax
\else\csname fi\endcsname
%</ignore>
%<*install>
\input docstrip.tex
\Msg{************************************************************************}
\Msg{* Installation}
\Msg{* Package: hologo 2016/05/12 v1.11 A logo collection with bookmark support (HO)}
\Msg{************************************************************************}

\keepsilent
\askforoverwritefalse

\let\MetaPrefix\relax
\preamble

This is a generated file.

Project: hologo
Version: 2016/05/12 v1.11

Copyright (C) 2010-2012 by
   Heiko Oberdiek <heiko.oberdiek at googlemail.com>

This work may be distributed and/or modified under the
conditions of the LaTeX Project Public License, either
version 1.3c of this license or (at your option) any later
version. This version of this license is in
   http://www.latex-project.org/lppl/lppl-1-3c.txt
and the latest version of this license is in
   http://www.latex-project.org/lppl.txt
and version 1.3 or later is part of all distributions of
LaTeX version 2005/12/01 or later.

This work has the LPPL maintenance status "maintained".

This Current Maintainer of this work is Heiko Oberdiek.

The Base Interpreter refers to any `TeX-Format',
because some files are installed in TDS:tex/generic//.

This work consists of the main source file hologo.dtx
and the derived files
   hologo.sty, hologo.pdf, hologo.ins, hologo.drv, hologo-example.tex,
   hologo-test1.tex, hologo-test-spacefactor.tex,
   hologo-test-list.tex.

\endpreamble
\let\MetaPrefix\DoubleperCent

\generate{%
  \file{hologo.ins}{\from{hologo.dtx}{install}}%
  \file{hologo.drv}{\from{hologo.dtx}{driver}}%
  \usedir{tex/generic/oberdiek}%
  \file{hologo.sty}{\from{hologo.dtx}{package}}%
  \usedir{doc/latex/oberdiek/example}%
  \file{hologo-example.tex}{\from{hologo.dtx}{example}}%
  \usedir{doc/latex/oberdiek/test}%
  \file{hologo-test1.tex}{\from{hologo.dtx}{test1}}%
  \file{hologo-test-spacefactor.tex}{\from{hologo.dtx}{test-spacefactor}}%
  \file{hologo-test-list.tex}{\from{hologo.dtx}{test-list}}%
  \nopreamble
  \nopostamble
  \usedir{source/latex/oberdiek/catalogue}%
  \file{hologo.xml}{\from{hologo.dtx}{catalogue}}%
}

\catcode32=13\relax% active space
\let =\space%
\Msg{************************************************************************}
\Msg{*}
\Msg{* To finish the installation you have to move the following}
\Msg{* file into a directory searched by TeX:}
\Msg{*}
\Msg{*     hologo.sty}
\Msg{*}
\Msg{* To produce the documentation run the file `hologo.drv'}
\Msg{* through LaTeX.}
\Msg{*}
\Msg{* Happy TeXing!}
\Msg{*}
\Msg{************************************************************************}

\endbatchfile
%</install>
%<*ignore>
\fi
%</ignore>
%<*driver>
\NeedsTeXFormat{LaTeX2e}
\ProvidesFile{hologo.drv}%
  [2016/05/12 v1.11 A logo collection with bookmark support (HO)]%
\documentclass{ltxdoc}
\usepackage{holtxdoc}[2011/11/22]
\usepackage{hologo}[2016/05/12]
\usepackage{longtable}
\usepackage{array}
\usepackage{paralist}
%\usepackage[T1]{fontenc}
%\usepackage{lmodern}
\begin{document}
  \DocInput{hologo.dtx}%
\end{document}
%</driver>
% \fi
%
%
% \CharacterTable
%  {Upper-case    \A\B\C\D\E\F\G\H\I\J\K\L\M\N\O\P\Q\R\S\T\U\V\W\X\Y\Z
%   Lower-case    \a\b\c\d\e\f\g\h\i\j\k\l\m\n\o\p\q\r\s\t\u\v\w\x\y\z
%   Digits        \0\1\2\3\4\5\6\7\8\9
%   Exclamation   \!     Double quote  \"     Hash (number) \#
%   Dollar        \$     Percent       \%     Ampersand     \&
%   Acute accent  \'     Left paren    \(     Right paren   \)
%   Asterisk      \*     Plus          \+     Comma         \,
%   Minus         \-     Point         \.     Solidus       \/
%   Colon         \:     Semicolon     \;     Less than     \<
%   Equals        \=     Greater than  \>     Question mark \?
%   Commercial at \@     Left bracket  \[     Backslash     \\
%   Right bracket \]     Circumflex    \^     Underscore    \_
%   Grave accent  \`     Left brace    \{     Vertical bar  \|
%   Right brace   \}     Tilde         \~}
%
% \GetFileInfo{hologo.drv}
%
% \title{The \xpackage{hologo} package}
% \date{2016/05/12 v1.11}
% \author{Heiko Oberdiek\\\xemail{heiko.oberdiek at googlemail.com}}
%
% \maketitle
%
% \begin{abstract}
% This package starts a collection of logos with support for bookmarks
% strings.
% \end{abstract}
%
% \tableofcontents
%
% \section{Documentation}
%
% \subsection{Logo macros}
%
% \begin{declcs}{hologo} \M{name}
% \end{declcs}
% Macro \cs{hologo} sets the logo with name \meta{name}.
% The following table shows the supported names.
%
% \begingroup
%   \def\hologoEntry#1#2#3{^^A
%     #1&#2&\hologoLogoSetup{#1}{variant=#2}\hologo{#1}&#3\tabularnewline
%   }
%   \begin{longtable}{>{\ttfamily}l>{\ttfamily}lll}
%     \rmfamily\bfseries{name} & \rmfamily\bfseries variant
%     & \bfseries logo & \bfseries since\\
%     \hline
%     \endhead
%     \hologoList
%   \end{longtable}
% \endgroup
%
% \begin{declcs}{Hologo} \M{name}
% \end{declcs}
% Macro \cs{Hologo} starts the logo \meta{name} with an uppercase
% letter. As an exception small greek letters are not converted
% to uppercase. Examples, see \hologo{eTeX} and \hologo{ExTeX}.
%
% \subsection{Setup macros}
%
% The package does not support package options, but the following
% setup macros can be used to set options.
%
% \begin{declcs}{hologoSetup} \M{key value list}
% \end{declcs}
% Macro \cs{hologoSetup} sets global options.
%
% \begin{declcs}{hologoLogoSetup} \M{logo} \M{key value list}
% \end{declcs}
% Some options can also be used to configure a logo.
% These settings take precedence over global option settings.
%
% \subsection{Options}\label{sec:options}
%
% There are boolean and string options:
% \begin{description}
% \item[Boolean option:]
% It takes |true| or |false|
% as value. If the value is omitted, then |true| is used.
% \item[String option:]
% A value must be given as string. (But the string might be empty.)
% \end{description}
% The following options can be used both in \cs{hologoSetup}
% and \cs{hologoLogoSetup}:
% \begin{description}
% \def\entry#1{\item[\xoption{#1}:]}
% \entry{break}
%   enables or disables line breaks inside the logo. This setting is
%   refined by options \xoption{hyphenbreak}, \xoption{spacebreak}
%   or \xoption{discretionarybreak}.
%   Default is |false|.
% \entry{hyphenbreak}
%   enables or disables the line break right after the hyphen character.
% \entry{spacebreak}
%   enables or disables line breaks at space characters.
% \entry{discretionarybreak}
%   enables or disables line breaks at hyphenation points
%   (inserted by \cs{-}).
% \end{description}
% Macro \cs{hologoLogoSetup} also knows:
% \begin{description}
% \item[\xoption{variant}:]
%   This is a string option. It specifies a variant of a logo that
%   must exist. An empty string selects the package default variant.
% \end{description}
% Example:
% \begin{quote}
%   |\hologoSetup{break=false}|\\
%   |\hologoLogoSetup{plainTeX}{variant=hyphen,hyphenbreak}|\\
%   Then ``plain-\TeX'' contains one break point after the hyphen.
% \end{quote}
%
% \subsection{Driver options}
%
% Sometimes graphical operations are needed to construct some
% glyphs (e.g.\ \hologo{XeTeX}). If package \xpackage{graphics}
% or package \xpackage{pgf} are found, then the macros are taken
% from there. Otherwise the packge defines its own operations
% and therefore needs the driver information. Many drivers are
% detected automatically (\hologo{pdfTeX}/\hologo{LuaTeX}
% in PDF mode, \hologo{XeTeX}, \hologo{VTeX}). These have precedence
% over a driver option. The driver can be given as package option
% or using \cs{hologoDriverSetup}.
% The following list contains the recognized driver options:
% \begin{itemize}
% \item \xoption{pdftex}, \xoption{luatex}
% \item \xoption{dvipdfm}, \xoption{dvipdfmx}
% \item \xoption{dvips}, \xoption{dvipsone}, \xoption{xdvi}
% \item \xoption{xetex}
% \item \xoption{vtex}
% \end{itemize}
% The left driver of a line is the driver name that is used internally.
% The following names are aliases for drivers that use the
% same method. Therefore the entry in the \xext{log} file for
% the used driver prints the internally used driver name.
% \begin{description}
% \item[\xoption{driverfallback}:]
%   This option expects a driver that is used,
%   if the driver could not be detected automatically.
% \end{description}
%
% \begin{declcs}{hologoDriverSetup} \M{driver option}
% \end{declcs}
% The driver can also be configured after package loading
% using \cs{hologoDriverSetup}, also the way for \hologo{plainTeX}
% to setup the driver.
%
% \subsection{Font setup}
%
% Some logos require a special font, but should also be usable by
% \hologo{plainTeX}. Therefore the package provides some ways
% to influence the font settings. The options below
% take font settings as values. Both font commands
% such as \cs{sffamily} and macros that take one argument
% like \cs{textsf} can be used.
%
% \begin{declcs}{hologoFontSetup} \M{key value list}
% \end{declcs}
% Macro \cs{hologoFontSetup} sets the fonts for all logos.
% Supported keys:
% \begin{description}
% \def\entry#1{\item[\xoption{#1}:]}
% \entry{general}
%   This font is used for all logos. The default is empty.
%   That means no special font is used.
% \entry{bibsf}
%   This font is used for
%   {\hologoLogoSetup{BibTeX}{variant=sf}\hologo{BibTeX}}
%   with variant \xoption{sf}.
% \entry{rm}
%   This font is a serif font. It is used for \hologo{ExTeX}.
% \entry{sc}
%   This font specifies a small caps font. It is used for
%   {\hologoLogoSetup{BibTeX}{variant=sc}\hologo{BibTeX}}
%   with variant \xoption{sc}.
% \entry{sf}
%   This font specifies a sans serif font. The default
%   is \cs{sffamily}, then \cs{sf} is tried. Otherwise
%   a warning is given. It is used by \hologo{KOMAScript}.
% \entry{sy}
%   This is the font for math symbols (e.g. cmsy).
%   It is used by \hologo{AmS}, \hologo{NTS}, \hologo{ExTeX}.
% \entry{logo}
%   \hologo{METAFONT} and \hologo{METAPOST} are using that font.
%   In \hologo{LaTeX} \cs{logofamily} is used and
%   the definitions of package \xpackage{mflogo} are used
%   if the package is not loaded.
%   Otherwise the \cs{tenlogo} is used and defined
%   if it does not already exists.
% \end{description}
%
% \begin{declcs}{hologoLogoFontSetup} \M{logo} \M{key value list}
% \end{declcs}
% Fonts can also be set for a logo or logo component separately,
% see the following list.
% The keys are the same as for \cs{hologoFontSetup}.
%
% \begin{longtable}{>{\ttfamily}l>{\sffamily}ll}
%   \meta{logo} & keys & result\\
%   \hline
%   \endhead
%   BibTeX & bibsf & {\hologoLogoSetup{BibTeX}{variant=sf}\hologo{BibTeX}}\\[.5ex]
%   BibTeX & sc & {\hologoLogoSetup{BibTeX}{variant=sc}\hologo{BibTeX}}\\[.5ex]
%   ExTeX & rm & \hologo{ExTeX}\\
%   SliTeX & rm & \hologo{SliTeX}\\[.5ex]
%   AmS & sy & \hologo{AmS}\\
%   ExTeX & sy & \hologo{ExTeX}\\
%   NTS & sy & \hologo{NTS}\\[.5ex]
%   KOMAScript & sf & \hologo{KOMAScript}\\[.5ex]
%   METAFONT & logo & \hologo{METAFONT}\\
%   METAPOST & logo & \hologo{METAPOST}\\[.5ex]
%   SliTeX & sc \hologo{SliTeX}
% \end{longtable}
%
% \subsubsection{Font order}
%
% For all logos the font \xoption{general} is applied first.
% Example:
%\begin{quote}
%|\hologoFontSetup{general=\color{red}}|
%\end{quote}
% will print red logos.
% Then if the font uses a special font \xoption{sf}, for example,
% the font is applied that is setup by \cs{hologoLogoFontSetup}.
% If this font is not setup, then the common font setup
% by \cs{hologoFontSetup} is used. Otherwise a warning is given,
% that there is no font configured.
%
% \subsection{Additional user macros}
%
% Usually a variant of a logo is configured by using
% \cs{hologoLogoSetup}, because it is bad style to mix
% different variants of the same logo in the same text.
% There the following macros are a convenience for testing.
%
% \begin{declcs}{hologoVariant} \M{name} \M{variant}\\
%   \cs{HologoVariant} \M{name} \M{variant}
% \end{declcs}
% Logo \meta{name} is set using \meta{variant} that specifies
% explicitely which variant of the macro is used. If the argument
% is empty, then the default form of the logo is used
% (configurable by \cs{hologoLogoSetup}).
%
% \cs{HologoVariant} is used if the logo is set in a context
% that needs an uppercase first letter (beginning of a sentence, \dots).
%
% \begin{declcs}{hologoList}\\
%   \cs{hologoEntry} \M{logo} \M{variant} \M{since}
% \end{declcs}
% Macro \cs{hologoList} contains all logos that are provided
% by the package including variants. The list consists of calls
% of \cs{hologoEntry} with three arguments starting with the
% logo name \meta{logo} and its variant \meta{variant}. An empty
% variant means the current default. Argument \meta{since} specifies
% with version of the package \xpackage{hologo} is needed to get
% the logo. If the logo is fixed, then the date gets updated.
% Therefore the date \meta{since} is not exactly the date of
% the first introduction, but rather the date of the latest fix.
%
% Before \cs{hologoList} can be used, macro \cs{hologoEntry} needs
% a definition. The example file in section \ref{sec:example}
% shows applications of \cs{hologoList}.
%
% \subsection{Supported contexts}
%
% Macros \cs{hologo} and friends support special contexts:
% \begin{itemize}
% \item \hologo{LaTeX}'s protection mechanism.
% \item Bookmarks of package \xpackage{hyperref}.
% \item Package \xpackage{tex4ht}.
% \item The macros can be used inside \cs{csname} constructs,
%   if \cs{ifincsname} is available (\hologo{pdfTeX}, \hologo{XeTeX},
%   \hologo{LuaTeX}).
% \end{itemize}
%
% \subsection{Example}
% \label{sec:example}
%
% The following example prints the logos in different fonts.
%    \begin{macrocode}
%<*example>
%<<verbatim
\NeedsTeXFormat{LaTeX2e}
\documentclass[a4paper]{article}
\usepackage[
  hmargin=20mm,
  vmargin=20mm,
]{geometry}
\pagestyle{empty}
\usepackage{hologo}[2016/05/12]
\usepackage{longtable}
\usepackage{array}
\setlength{\extrarowheight}{2pt}
\usepackage[T1]{fontenc}
\usepackage{lmodern}
\usepackage{pdflscape}
\usepackage[
  pdfencoding=auto,
]{hyperref}
\hypersetup{
  pdfauthor={Heiko Oberdiek},
  pdftitle={Example for package `hologo'},
  pdfsubject={Logos with fonts lmr, lmss, qtm, qpl, qhv},
}
\usepackage{bookmark}

% Print the logo list on the console

\begingroup
  \typeout{}%
  \typeout{*** Begin of logo list ***}%
  \newcommand*{\hologoEntry}[3]{%
    \typeout{#1 \ifx\\#2\\\else(#2) \fi[#3]}%
  }%
  \hologoList
  \typeout{*** End of logo list ***}%
  \typeout{}%
\endgroup

\begin{document}
\begin{landscape}

  \section{Example file for package `hologo'}

  % Table for font names

  \begin{longtable}{>{\bfseries}ll}
    \textbf{font} & \textbf{Font name}\\
    \hline
    lmr & Latin Modern Roman\\
    lmss & Latin Modern Sans\\
    qtm & \TeX\ Gyre Termes\\
    qhv & \TeX\ Gyre Heros\\
    qpl & \TeX\ Gyre Pagella\\
  \end{longtable}

  % Logo list with logos in different fonts

  \begingroup
    \newcommand*{\SetVariant}[2]{%
      \ifx\\#2\\%
      \else
        \hologoLogoSetup{#1}{variant=#2}%
      \fi
    }%
    \newcommand*{\hologoEntry}[3]{%
      \SetVariant{#1}{#2}%
      \raisebox{1em}[0pt][0pt]{\hypertarget{#1@#2}{}}%
      \bookmark[%
        dest={#1@#2},%
      ]{%
        #1\ifx\\#2\\\else\space(#2)\fi: \Hologo{#1}, \hologo{#1} %
        [Unicode]%
      }%
      \hypersetup{unicode=false}%
      \bookmark[%
        dest={#1@#2},%
      ]{%
        #1\ifx\\#2\\\else\space(#2)\fi: \Hologo{#1}, \hologo{#1} %
        [PDFDocEncoding]%
      }%
      \texttt{#1}%
      &%
      \texttt{#2}%
      &%
      \Hologo{#1}%
      &%
      \SetVariant{#1}{#2}%
      \hologo{#1}%
      &%
      \SetVariant{#1}{#2}%
      \fontfamily{qtm}\selectfont
      \hologo{#1}%
      &%
      \SetVariant{#1}{#2}%
      \fontfamily{qpl}\selectfont
      \hologo{#1}%
      &%
      \SetVariant{#1}{#2}%
      \textsf{\hologo{#1}}%
      &%
      \SetVariant{#1}{#2}%
      \fontfamily{qhv}\selectfont
      \hologo{#1}%
      \tabularnewline
    }%
    \begin{longtable}{llllllll}%
      \textbf{\textit{logo}} & \textbf{\textit{variant}} &
      \texttt{\string\Hologo} &
      \textbf{lmr} & \textbf{qtm} & \textbf{qpl} &
      \textbf{lmss} & \textbf{qhv}
      \tabularnewline
      \hline
      \endhead
      \hologoList
    \end{longtable}%
  \endgroup

\end{landscape}
\end{document}
%verbatim
%</example>
%    \end{macrocode}
%
% \StopEventually{
% }
%
% \section{Implementation}
%    \begin{macrocode}
%<*package>
%    \end{macrocode}
%    Reload check, especially if the package is not used with \LaTeX.
%    \begin{macrocode}
\begingroup\catcode61\catcode48\catcode32=10\relax%
  \catcode13=5 % ^^M
  \endlinechar=13 %
  \catcode35=6 % #
  \catcode39=12 % '
  \catcode44=12 % ,
  \catcode45=12 % -
  \catcode46=12 % .
  \catcode58=12 % :
  \catcode64=11 % @
  \catcode123=1 % {
  \catcode125=2 % }
  \expandafter\let\expandafter\x\csname ver@hologo.sty\endcsname
  \ifx\x\relax % plain-TeX, first loading
  \else
    \def\empty{}%
    \ifx\x\empty % LaTeX, first loading,
      % variable is initialized, but \ProvidesPackage not yet seen
    \else
      \expandafter\ifx\csname PackageInfo\endcsname\relax
        \def\x#1#2{%
          \immediate\write-1{Package #1 Info: #2.}%
        }%
      \else
        \def\x#1#2{\PackageInfo{#1}{#2, stopped}}%
      \fi
      \x{hologo}{The package is already loaded}%
      \aftergroup\endinput
    \fi
  \fi
\endgroup%
%    \end{macrocode}
%    Package identification:
%    \begin{macrocode}
\begingroup\catcode61\catcode48\catcode32=10\relax%
  \catcode13=5 % ^^M
  \endlinechar=13 %
  \catcode35=6 % #
  \catcode39=12 % '
  \catcode40=12 % (
  \catcode41=12 % )
  \catcode44=12 % ,
  \catcode45=12 % -
  \catcode46=12 % .
  \catcode47=12 % /
  \catcode58=12 % :
  \catcode64=11 % @
  \catcode91=12 % [
  \catcode93=12 % ]
  \catcode123=1 % {
  \catcode125=2 % }
  \expandafter\ifx\csname ProvidesPackage\endcsname\relax
    \def\x#1#2#3[#4]{\endgroup
      \immediate\write-1{Package: #3 #4}%
      \xdef#1{#4}%
    }%
  \else
    \def\x#1#2[#3]{\endgroup
      #2[{#3}]%
      \ifx#1\@undefined
        \xdef#1{#3}%
      \fi
      \ifx#1\relax
        \xdef#1{#3}%
      \fi
    }%
  \fi
\expandafter\x\csname ver@hologo.sty\endcsname
\ProvidesPackage{hologo}%
  [2016/05/12 v1.11 A logo collection with bookmark support (HO)]%
%    \end{macrocode}
%
%    \begin{macrocode}
\begingroup\catcode61\catcode48\catcode32=10\relax%
  \catcode13=5 % ^^M
  \endlinechar=13 %
  \catcode123=1 % {
  \catcode125=2 % }
  \catcode64=11 % @
  \def\x{\endgroup
    \expandafter\edef\csname HOLOGO@AtEnd\endcsname{%
      \endlinechar=\the\endlinechar\relax
      \catcode13=\the\catcode13\relax
      \catcode32=\the\catcode32\relax
      \catcode35=\the\catcode35\relax
      \catcode61=\the\catcode61\relax
      \catcode64=\the\catcode64\relax
      \catcode123=\the\catcode123\relax
      \catcode125=\the\catcode125\relax
    }%
  }%
\x\catcode61\catcode48\catcode32=10\relax%
\catcode13=5 % ^^M
\endlinechar=13 %
\catcode35=6 % #
\catcode64=11 % @
\catcode123=1 % {
\catcode125=2 % }
\def\TMP@EnsureCode#1#2{%
  \edef\HOLOGO@AtEnd{%
    \HOLOGO@AtEnd
    \catcode#1=\the\catcode#1\relax
  }%
  \catcode#1=#2\relax
}
\TMP@EnsureCode{10}{12}% ^^J
\TMP@EnsureCode{33}{12}% !
\TMP@EnsureCode{34}{12}% "
\TMP@EnsureCode{36}{3}% $
\TMP@EnsureCode{38}{4}% &
\TMP@EnsureCode{39}{12}% '
\TMP@EnsureCode{40}{12}% (
\TMP@EnsureCode{41}{12}% )
\TMP@EnsureCode{42}{12}% *
\TMP@EnsureCode{43}{12}% +
\TMP@EnsureCode{44}{12}% ,
\TMP@EnsureCode{45}{12}% -
\TMP@EnsureCode{46}{12}% .
\TMP@EnsureCode{47}{12}% /
\TMP@EnsureCode{58}{12}% :
\TMP@EnsureCode{59}{12}% ;
\TMP@EnsureCode{60}{12}% <
\TMP@EnsureCode{62}{12}% >
\TMP@EnsureCode{63}{12}% ?
\TMP@EnsureCode{91}{12}% [
\TMP@EnsureCode{93}{12}% ]
\TMP@EnsureCode{94}{7}% ^ (superscript)
\TMP@EnsureCode{95}{8}% _ (subscript)
\TMP@EnsureCode{96}{12}% `
\TMP@EnsureCode{124}{12}% |
\edef\HOLOGO@AtEnd{%
  \HOLOGO@AtEnd
  \escapechar\the\escapechar\relax
  \noexpand\endinput
}
\escapechar=92 %
%    \end{macrocode}
%
% \subsection{Logo list}
%
%    \begin{macro}{\hologoList}
%    \begin{macrocode}
\def\hologoList{%
  \hologoEntry{(La)TeX}{}{2011/10/01}%
  \hologoEntry{AmSLaTeX}{}{2010/04/16}%
  \hologoEntry{AmSTeX}{}{2010/04/16}%
  \hologoEntry{biber}{}{2011/10/01}%
  \hologoEntry{BibTeX}{}{2011/10/01}%
  \hologoEntry{BibTeX}{sf}{2011/10/01}%
  \hologoEntry{BibTeX}{sc}{2011/10/01}%
  \hologoEntry{BibTeX8}{}{2011/11/22}%
  \hologoEntry{ConTeXt}{}{2011/03/25}%
  \hologoEntry{ConTeXt}{narrow}{2011/03/25}%
  \hologoEntry{ConTeXt}{simple}{2011/03/25}%
  \hologoEntry{emTeX}{}{2010/04/26}%
  \hologoEntry{eTeX}{}{2010/04/08}%
  \hologoEntry{ExTeX}{}{2011/10/01}%
  \hologoEntry{HanTheThanh}{}{2011/11/29}%
  \hologoEntry{iniTeX}{}{2011/10/01}%
  \hologoEntry{KOMAScript}{}{2011/10/01}%
  \hologoEntry{La}{}{2010/05/08}%
  \hologoEntry{LaTeX}{}{2010/04/08}%
  \hologoEntry{LaTeX2e}{}{2010/04/08}%
  \hologoEntry{LaTeX3}{}{2010/04/24}%
  \hologoEntry{LaTeXe}{}{2010/04/08}%
  \hologoEntry{LaTeXML}{}{2011/11/22}%
  \hologoEntry{LaTeXTeX}{}{2011/10/01}%
  \hologoEntry{LuaLaTeX}{}{2010/04/08}%
  \hologoEntry{LuaTeX}{}{2010/04/08}%
  \hologoEntry{LyX}{}{2011/10/01}%
  \hologoEntry{METAFONT}{}{2011/10/01}%
  \hologoEntry{MetaFun}{}{2011/10/01}%
  \hologoEntry{METAPOST}{}{2011/10/01}%
  \hologoEntry{MetaPost}{}{2011/10/01}%
  \hologoEntry{MiKTeX}{}{2011/10/01}%
  \hologoEntry{NTS}{}{2011/10/01}%
  \hologoEntry{OzMF}{}{2011/10/01}%
  \hologoEntry{OzMP}{}{2011/10/01}%
  \hologoEntry{OzTeX}{}{2011/10/01}%
  \hologoEntry{OzTtH}{}{2011/10/01}%
  \hologoEntry{PCTeX}{}{2011/10/01}%
  \hologoEntry{pdfTeX}{}{2011/10/01}%
  \hologoEntry{pdfLaTeX}{}{2011/10/01}%
  \hologoEntry{PiC}{}{2011/10/01}%
  \hologoEntry{PiCTeX}{}{2011/10/01}%
  \hologoEntry{plainTeX}{}{2010/04/08}%
  \hologoEntry{plainTeX}{space}{2010/04/16}%
  \hologoEntry{plainTeX}{hyphen}{2010/04/16}%
  \hologoEntry{plainTeX}{runtogether}{2010/04/16}%
  \hologoEntry{SageTeX}{}{2011/11/22}%
  \hologoEntry{SLiTeX}{}{2011/10/01}%
  \hologoEntry{SLiTeX}{lift}{2011/10/01}%
  \hologoEntry{SLiTeX}{narrow}{2011/10/01}%
  \hologoEntry{SLiTeX}{simple}{2011/10/01}%
  \hologoEntry{SliTeX}{}{2011/10/01}%
  \hologoEntry{SliTeX}{narrow}{2011/10/01}%
  \hologoEntry{SliTeX}{simple}{2011/10/01}%
  \hologoEntry{SliTeX}{lift}{2011/10/01}%
  \hologoEntry{teTeX}{}{2011/10/01}%
  \hologoEntry{TeX}{}{2010/04/08}%
  \hologoEntry{TeX4ht}{}{2011/11/22}%
  \hologoEntry{TTH}{}{2011/11/22}%
  \hologoEntry{virTeX}{}{2011/10/01}%
  \hologoEntry{VTeX}{}{2010/04/24}%
  \hologoEntry{Xe}{}{2010/04/08}%
  \hologoEntry{XeLaTeX}{}{2010/04/08}%
  \hologoEntry{XeTeX}{}{2010/04/08}%
}
%    \end{macrocode}
%    \end{macro}
%
% \subsection{Load resources}
%
%    \begin{macrocode}
\begingroup\expandafter\expandafter\expandafter\endgroup
\expandafter\ifx\csname RequirePackage\endcsname\relax
  \def\TMP@RequirePackage#1[#2]{%
    \begingroup\expandafter\expandafter\expandafter\endgroup
    \expandafter\ifx\csname ver@#1.sty\endcsname\relax
      \input #1.sty\relax
    \fi
  }%
  \TMP@RequirePackage{ltxcmds}[2011/02/04]%
  \TMP@RequirePackage{infwarerr}[2010/04/08]%
  \TMP@RequirePackage{kvsetkeys}[2010/03/01]%
  \TMP@RequirePackage{kvdefinekeys}[2010/03/01]%
  \TMP@RequirePackage{pdftexcmds}[2010/04/01]%
  \TMP@RequirePackage{ifpdf}[2010/01/28]%
  \TMP@RequirePackage{ifluatex}[2010/03/01]%
  \ltx@IfUndefined{newif}{%
    \expandafter\let\csname newif\endcsname\ltx@newif
  }{}%
  \TMP@RequirePackage{ifxetex}[2009/01/23]%
  \TMP@RequirePackage{ifvtex}[2010/03/01]%
\else
  \RequirePackage{ltxcmds}[2011/02/04]%
  \RequirePackage{infwarerr}[2010/04/08]%
  \RequirePackage{kvsetkeys}[2010/03/01]%
  \RequirePackage{kvdefinekeys}[2010/03/01]%
  \RequirePackage{pdftexcmds}[2010/04/01]%
  \RequirePackage{ifpdf}[2010/01/28]%
  \RequirePackage{ifluatex}[2010/03/01]%
  \RequirePackage{ifxetex}[2009/01/23]%
  \RequirePackage{ifvtex}[2010/03/01]%
\fi
%    \end{macrocode}
%
%    \begin{macro}{\HOLOGO@IfDefined}
%    \begin{macrocode}
\def\HOLOGO@IfExists#1{%
  \ifx\@undefined#1%
    \expandafter\ltx@secondoftwo
  \else
    \ifx\relax#1%
      \expandafter\ltx@secondoftwo
    \else
      \expandafter\expandafter\expandafter\ltx@firstoftwo
    \fi
  \fi
}
%    \end{macrocode}
%    \end{macro}
%
% \subsection{Setup macros}
%
%    \begin{macro}{\hologoSetup}
%    \begin{macrocode}
\def\hologoSetup{%
  \let\HOLOGO@name\relax
  \HOLOGO@Setup
}
%    \end{macrocode}
%    \end{macro}
%
%    \begin{macro}{\hologoLogoSetup}
%    \begin{macrocode}
\def\hologoLogoSetup#1{%
  \edef\HOLOGO@name{#1}%
  \ltx@IfUndefined{HoLogo@\HOLOGO@name}{%
    \@PackageError{hologo}{%
      Unknown logo `\HOLOGO@name'%
    }\@ehc
    \ltx@gobble
  }{%
    \HOLOGO@Setup
  }%
}
%    \end{macrocode}
%    \end{macro}
%
%    \begin{macro}{\HOLOGO@Setup}
%    \begin{macrocode}
\def\HOLOGO@Setup{%
  \kvsetkeys{HoLogo}%
}
%    \end{macrocode}
%    \end{macro}
%
% \subsection{Options}
%
%    \begin{macro}{\HOLOGO@DeclareBoolOption}
%    \begin{macrocode}
\def\HOLOGO@DeclareBoolOption#1{%
  \expandafter\chardef\csname HOLOGOOPT@#1\endcsname\ltx@zero
  \kv@define@key{HoLogo}{#1}[true]{%
    \def\HOLOGO@temp{##1}%
    \ifx\HOLOGO@temp\HOLOGO@true
      \ifx\HOLOGO@name\relax
        \expandafter\chardef\csname HOLOGOOPT@#1\endcsname=\ltx@one
      \else
        \expandafter\chardef\csname
        HoLogoOpt@#1@\HOLOGO@name\endcsname\ltx@one
      \fi
      \HOLOGO@SetBreakAll{#1}%
    \else
      \ifx\HOLOGO@temp\HOLOGO@false
        \ifx\HOLOGO@name\relax
          \expandafter\chardef\csname HOLOGOOPT@#1\endcsname=\ltx@zero
        \else
          \expandafter\chardef\csname
          HoLogoOpt@#1@\HOLOGO@name\endcsname=\ltx@zero
        \fi
        \HOLOGO@SetBreakAll{#1}%
      \else
        \@PackageError{hologo}{%
          Unknown value `##1' for boolean option `#1'.\MessageBreak
          Known values are `true' and `false'%
        }\@ehc
      \fi
    \fi
  }%
}
%    \end{macrocode}
%    \end{macro}
%
%    \begin{macro}{\HOLOGO@SetBreakAll}
%    \begin{macrocode}
\def\HOLOGO@SetBreakAll#1{%
  \def\HOLOGO@temp{#1}%
  \ifx\HOLOGO@temp\HOLOGO@break
    \ifx\HOLOGO@name\relax
      \chardef\HOLOGOOPT@hyphenbreak=\HOLOGOOPT@break
      \chardef\HOLOGOOPT@spacebreak=\HOLOGOOPT@break
      \chardef\HOLOGOOPT@discretionarybreak=\HOLOGOOPT@break
    \else
      \expandafter\chardef
         \csname HoLogoOpt@hyphenbreak@\HOLOGO@name\endcsname=%
         \csname HoLogoOpt@break@\HOLOGO@name\endcsname
      \expandafter\chardef
         \csname HoLogoOpt@spacebreak@\HOLOGO@name\endcsname=%
         \csname HoLogoOpt@break@\HOLOGO@name\endcsname
      \expandafter\chardef
         \csname HoLogoOpt@discretionarybreak@\HOLOGO@name
             \endcsname=%
         \csname HoLogoOpt@break@\HOLOGO@name\endcsname
    \fi
  \fi
}
%    \end{macrocode}
%    \end{macro}
%
%    \begin{macro}{\HOLOGO@true}
%    \begin{macrocode}
\def\HOLOGO@true{true}
%    \end{macrocode}
%    \end{macro}
%    \begin{macro}{\HOLOGO@false}
%    \begin{macrocode}
\def\HOLOGO@false{false}
%    \end{macrocode}
%    \end{macro}
%    \begin{macro}{\HOLOGO@break}
%    \begin{macrocode}
\def\HOLOGO@break{break}
%    \end{macrocode}
%    \end{macro}
%
%    \begin{macrocode}
\HOLOGO@DeclareBoolOption{break}
\HOLOGO@DeclareBoolOption{hyphenbreak}
\HOLOGO@DeclareBoolOption{spacebreak}
\HOLOGO@DeclareBoolOption{discretionarybreak}
%    \end{macrocode}
%
%    \begin{macrocode}
\kv@define@key{HoLogo}{variant}{%
  \ifx\HOLOGO@name\relax
    \@PackageError{hologo}{%
      Option `variant' is not available in \string\hologoSetup,%
      \MessageBreak
      Use \string\hologoLogoSetup\space instead%
    }\@ehc
  \else
    \edef\HOLOGO@temp{#1}%
    \ifx\HOLOGO@temp\ltx@empty
      \expandafter
      \let\csname HoLogoOpt@variant@\HOLOGO@name\endcsname\@undefined
    \else
      \ltx@IfUndefined{HoLogo@\HOLOGO@name @\HOLOGO@temp}{%
        \@PackageError{hologo}{%
          Unknown variant `\HOLOGO@temp' of logo `\HOLOGO@name'%
        }\@ehc
      }{%
        \expandafter
        \let\csname HoLogoOpt@variant@\HOLOGO@name\endcsname
            \HOLOGO@temp
      }%
    \fi
  \fi
}
%    \end{macrocode}
%
%    \begin{macro}{\HOLOGO@Variant}
%    \begin{macrocode}
\def\HOLOGO@Variant#1{%
  #1%
  \ltx@ifundefined{HoLogoOpt@variant@#1}{%
  }{%
    @\csname HoLogoOpt@variant@#1\endcsname
  }%
}
%    \end{macrocode}
%    \end{macro}
%
% \subsection{Break/no-break support}
%
%    \begin{macro}{\HOLOGO@space}
%    \begin{macrocode}
\def\HOLOGO@space{%
  \ltx@ifundefined{HoLogoOpt@spacebreak@\HOLOGO@name}{%
    \ltx@ifundefined{HoLogoOpt@break@\HOLOGO@name}{%
      \chardef\HOLOGO@temp=\HOLOGOOPT@spacebreak
    }{%
      \chardef\HOLOGO@temp=%
        \csname HoLogoOpt@break@\HOLOGO@name\endcsname
    }%
  }{%
    \chardef\HOLOGO@temp=%
      \csname HoLogoOpt@spacebreak@\HOLOGO@name\endcsname
  }%
  \ifcase\HOLOGO@temp
    \penalty10000 %
  \fi
  \ltx@space
}
%    \end{macrocode}
%    \end{macro}
%
%    \begin{macro}{\HOLOGO@hyphen}
%    \begin{macrocode}
\def\HOLOGO@hyphen{%
  \ltx@ifundefined{HoLogoOpt@hyphenbreak@\HOLOGO@name}{%
    \ltx@ifundefined{HoLogoOpt@break@\HOLOGO@name}{%
      \chardef\HOLOGO@temp=\HOLOGOOPT@hyphenbreak
    }{%
      \chardef\HOLOGO@temp=%
        \csname HoLogoOpt@break@\HOLOGO@name\endcsname
    }%
  }{%
    \chardef\HOLOGO@temp=%
      \csname HoLogoOpt@hyphenbreak@\HOLOGO@name\endcsname
  }%
  \ifcase\HOLOGO@temp
    \ltx@mbox{-}%
  \else
    -%
  \fi
}
%    \end{macrocode}
%    \end{macro}
%
%    \begin{macro}{\HOLOGO@discretionary}
%    \begin{macrocode}
\def\HOLOGO@discretionary{%
  \ltx@ifundefined{HoLogoOpt@discretionarybreak@\HOLOGO@name}{%
    \ltx@ifundefined{HoLogoOpt@break@\HOLOGO@name}{%
      \chardef\HOLOGO@temp=\HOLOGOOPT@discretionarybreak
    }{%
      \chardef\HOLOGO@temp=%
        \csname HoLogoOpt@break@\HOLOGO@name\endcsname
    }%
  }{%
    \chardef\HOLOGO@temp=%
      \csname HoLogoOpt@discretionarybreak@\HOLOGO@name\endcsname
  }%
  \ifcase\HOLOGO@temp
  \else
    \-%
  \fi
}
%    \end{macrocode}
%    \end{macro}
%
%    \begin{macro}{\HOLOGO@mbox}
%    \begin{macrocode}
\def\HOLOGO@mbox#1{%
  \ltx@ifundefined{HoLogoOpt@break@\HOLOGO@name}{%
    \chardef\HOLOGO@temp=\HOLOGOOPT@hyphenbreak
  }{%
    \chardef\HOLOGO@temp=%
      \csname HoLogoOpt@break@\HOLOGO@name\endcsname
  }%
  \ifcase\HOLOGO@temp
    \ltx@mbox{#1}%
  \else
    #1%
  \fi
}
%    \end{macrocode}
%    \end{macro}
%
% \subsection{Font support}
%
%    \begin{macro}{\HoLogoFont@font}
%    \begin{tabular}{@{}ll@{}}
%    |#1|:& logo name\\
%    |#2|:& font short name\\
%    |#3|:& text
%    \end{tabular}
%    \begin{macrocode}
\def\HoLogoFont@font#1#2#3{%
  \begingroup
    \ltx@IfUndefined{HoLogoFont@logo@#1.#2}{%
      \ltx@IfUndefined{HoLogoFont@font@#2}{%
        \@PackageWarning{hologo}{%
          Missing font `#2' for logo `#1'%
        }%
        #3%
      }{%
        \csname HoLogoFont@font@#2\endcsname{#3}%
      }%
    }{%
      \csname HoLogoFont@logo@#1.#2\endcsname{#3}%
    }%
  \endgroup
}
%    \end{macrocode}
%    \end{macro}
%
%    \begin{macro}{\HoLogoFont@Def}
%    \begin{macrocode}
\def\HoLogoFont@Def#1{%
  \expandafter\def\csname HoLogoFont@font@#1\endcsname
}
%    \end{macrocode}
%    \end{macro}
%    \begin{macro}{\HoLogoFont@LogoDef}
%    \begin{macrocode}
\def\HoLogoFont@LogoDef#1#2{%
  \expandafter\def\csname HoLogoFont@logo@#1.#2\endcsname
}
%    \end{macrocode}
%    \end{macro}
%
% \subsubsection{Font defaults}
%
%    \begin{macro}{\HoLogoFont@font@general}
%    \begin{macrocode}
\HoLogoFont@Def{general}{}%
%    \end{macrocode}
%    \end{macro}
%
%    \begin{macro}{\HoLogoFont@font@rm}
%    \begin{macrocode}
\ltx@IfUndefined{rmfamily}{%
  \ltx@IfUndefined{rm}{%
  }{%
    \HoLogoFont@Def{rm}{\rm}%
  }%
}{%
  \HoLogoFont@Def{rm}{\rmfamily}%
}
%    \end{macrocode}
%    \end{macro}
%
%    \begin{macro}{\HoLogoFont@font@sf}
%    \begin{macrocode}
\ltx@IfUndefined{sffamily}{%
  \ltx@IfUndefined{sf}{%
  }{%
    \HoLogoFont@Def{sf}{\sf}%
  }%
}{%
  \HoLogoFont@Def{sf}{\sffamily}%
}
%    \end{macrocode}
%    \end{macro}
%
%    \begin{macro}{\HoLogoFont@font@bibsf}
%    In case of \hologo{plainTeX} the original small caps
%    variant is used as default. In \hologo{LaTeX}
%    the definition of package \xpackage{dtklogos} \cite{dtklogos}
%    is used.
%\begin{quote}
%\begin{verbatim}
%\DeclareRobustCommand{\BibTeX}{%
%  B%
%  \kern-.05em%
%  \hbox{%
%    $\m@th$% %% force math size calculations
%    \csname S@\f@size\endcsname
%    \fontsize\sf@size\z@
%    \math@fontsfalse
%    \selectfont
%    I%
%    \kern-.025em%
%    B
%  }%
%  \kern-.08em%
%  \-%
%  \TeX
%}
%\end{verbatim}
%\end{quote}
%    \begin{macrocode}
\ltx@IfUndefined{selectfont}{%
  \ltx@IfUndefined{tensc}{%
    \font\tensc=cmcsc10\relax
  }{}%
  \HoLogoFont@Def{bibsf}{\tensc}%
}{%
  \HoLogoFont@Def{bibsf}{%
    $\mathsurround=0pt$%
    \csname S@\f@size\endcsname
    \fontsize\sf@size{0pt}%
    \math@fontsfalse
    \selectfont
  }%
}
%    \end{macrocode}
%    \end{macro}
%
%    \begin{macro}{\HoLogoFont@font@sc}
%    \begin{macrocode}
\ltx@IfUndefined{scshape}{%
  \ltx@IfUndefined{tensc}{%
    \font\tensc=cmcsc10\relax
  }{}%
  \HoLogoFont@Def{sc}{\tensc}%
}{%
  \HoLogoFont@Def{sc}{\scshape}%
}
%    \end{macrocode}
%    \end{macro}
%
%    \begin{macro}{\HoLogoFont@font@sy}
%    \begin{macrocode}
\ltx@IfUndefined{usefont}{%
  \ltx@IfUndefined{tensy}{%
  }{%
    \HoLogoFont@Def{sy}{\tensy}%
  }%
}{%
  \HoLogoFont@Def{sy}{%
    \usefont{OMS}{cmsy}{m}{n}%
  }%
}
%    \end{macrocode}
%    \end{macro}
%
%    \begin{macro}{\HoLogoFont@font@logo}
%    \begin{macrocode}
\begingroup
  \def\x{LaTeX2e}%
\expandafter\endgroup
\ifx\fmtname\x
  \ltx@IfUndefined{logofamily}{%
    \DeclareRobustCommand\logofamily{%
      \not@math@alphabet\logofamily\relax
      \fontencoding{U}%
      \fontfamily{logo}%
      \selectfont
    }%
  }{}%
  \ltx@IfUndefined{logofamily}{%
  }{%
    \HoLogoFont@Def{logo}{\logofamily}%
  }%
\else
  \ltx@IfUndefined{tenlogo}{%
    \font\tenlogo=logo10\relax
  }{}%
  \HoLogoFont@Def{logo}{\tenlogo}%
\fi
%    \end{macrocode}
%    \end{macro}
%
% \subsubsection{Font setup}
%
%    \begin{macro}{\hologoFontSetup}
%    \begin{macrocode}
\def\hologoFontSetup{%
  \let\HOLOGO@name\relax
  \HOLOGO@FontSetup
}
%    \end{macrocode}
%    \end{macro}
%
%    \begin{macro}{\hologoLogoFontSetup}
%    \begin{macrocode}
\def\hologoLogoFontSetup#1{%
  \edef\HOLOGO@name{#1}%
  \ltx@IfUndefined{HoLogo@\HOLOGO@name}{%
    \@PackageError{hologo}{%
      Unknown logo `\HOLOGO@name'%
    }\@ehc
    \ltx@gobble
  }{%
    \HOLOGO@FontSetup
  }%
}
%    \end{macrocode}
%    \end{macro}
%
%    \begin{macro}{\HOLOGO@FontSetup}
%    \begin{macrocode}
\def\HOLOGO@FontSetup{%
  \kvsetkeys{HoLogoFont}%
}
%    \end{macrocode}
%    \end{macro}
%
%    \begin{macrocode}
\def\HOLOGO@temp#1{%
  \kv@define@key{HoLogoFont}{#1}{%
    \ifx\HOLOGO@name\relax
      \HoLogoFont@Def{#1}{##1}%
    \else
      \HoLogoFont@LogoDef\HOLOGO@name{#1}{##1}%
    \fi
  }%
}
\HOLOGO@temp{general}
\HOLOGO@temp{sf}
%    \end{macrocode}
%
% \subsection{Generic logo commands}
%
%    \begin{macrocode}
\HOLOGO@IfExists\hologo{%
  \@PackageError{hologo}{%
    \string\hologo\ltx@space is already defined.\MessageBreak
    Package loading is aborted%
  }\@ehc
  \HOLOGO@AtEnd
}%
\HOLOGO@IfExists\hologoRobust{%
  \@PackageError{hologo}{%
    \string\hologoRobust\ltx@space is already defined.\MessageBreak
    Package loading is aborted%
  }\@ehc
  \HOLOGO@AtEnd
}%
%    \end{macrocode}
%
% \subsubsection{\cs{hologo} and friends}
%
%    \begin{macrocode}
\ifluatex
  \expandafter\ltx@firstofone
\else
  \expandafter\ltx@gobble
\fi
{%
  \ltx@IfUndefined{ifincsname}{%
    \ifnum\luatexversion<36 %
      \expandafter\ltx@gobble
    \else
      \expandafter\ltx@firstofone
    \fi
    {%
      \begingroup
        \ifcase0%
            \directlua{%
              if tex.enableprimitives then %
                tex.enableprimitives('HOLOGO@', {'ifincsname'})%
              else %
                tex.print('1')%
              end%
            }%
            \ifx\HOLOGO@ifincsname\@undefined 1\fi%
            \relax
          \expandafter\ltx@firstofone
        \else
          \endgroup
          \expandafter\ltx@gobble
        \fi
        {%
          \global\let\ifincsname\HOLOGO@ifincsname
        }%
      \HOLOGO@temp
    }%
  }{}%
}
%    \end{macrocode}
%    \begin{macrocode}
\ltx@IfUndefined{ifincsname}{%
  \catcode`$=14 %
}{%
  \catcode`$=9 %
}
%    \end{macrocode}
%
%    \begin{macro}{\hologo}
%    \begin{macrocode}
\def\hologo#1{%
$ \ifincsname
$   \ltx@ifundefined{HoLogoCs@\HOLOGO@Variant{#1}}{%
$     #1%
$   }{%
$     \csname HoLogoCs@\HOLOGO@Variant{#1}\endcsname\ltx@firstoftwo
$   }%
$ \else
    \HOLOGO@IfExists\texorpdfstring\texorpdfstring\ltx@firstoftwo
    {%
      \hologoRobust{#1}%
    }{%
      \ltx@ifundefined{HoLogoBkm@\HOLOGO@Variant{#1}}{%
        \ltx@ifundefined{HoLogo@#1}{?#1?}{#1}%
      }{%
        \csname HoLogoBkm@\HOLOGO@Variant{#1}\endcsname
        \ltx@firstoftwo
      }%
    }%
$ \fi
}
%    \end{macrocode}
%    \end{macro}
%    \begin{macro}{\Hologo}
%    \begin{macrocode}
\def\Hologo#1{%
$ \ifincsname
$   \ltx@ifundefined{HoLogoCs@\HOLOGO@Variant{#1}}{%
$     #1%
$   }{%
$     \csname HoLogoCs@\HOLOGO@Variant{#1}\endcsname\ltx@secondoftwo
$   }%
$ \else
    \HOLOGO@IfExists\texorpdfstring\texorpdfstring\ltx@firstoftwo
    {%
      \HologoRobust{#1}%
    }{%
      \ltx@ifundefined{HoLogoBkm@\HOLOGO@Variant{#1}}{%
        \ltx@ifundefined{HoLogo@#1}{?#1?}{#1}%
      }{%
        \csname HoLogoBkm@\HOLOGO@Variant{#1}\endcsname
        \ltx@secondoftwo
      }%
    }%
$ \fi
}
%    \end{macrocode}
%    \end{macro}
%
%    \begin{macro}{\hologoVariant}
%    \begin{macrocode}
\def\hologoVariant#1#2{%
  \ifx\relax#2\relax
    \hologo{#1}%
  \else
$   \ifincsname
$     \ltx@ifundefined{HoLogoCs@#1@#2}{%
$       #1%
$     }{%
$       \csname HoLogoCs@#1@#2\endcsname\ltx@firstoftwo
$     }%
$   \else
      \HOLOGO@IfExists\texorpdfstring\texorpdfstring\ltx@firstoftwo
      {%
        \hologoVariantRobust{#1}{#2}%
      }{%
        \ltx@ifundefined{HoLogoBkm@#1@#2}{%
          \ltx@ifundefined{HoLogo@#1}{?#1?}{#1}%
        }{%
          \csname HoLogoBkm@#1@#2\endcsname
          \ltx@firstoftwo
        }%
      }%
$   \fi
  \fi
}
%    \end{macrocode}
%    \end{macro}
%    \begin{macro}{\HologoVariant}
%    \begin{macrocode}
\def\HologoVariant#1#2{%
  \ifx\relax#2\relax
    \Hologo{#1}%
  \else
$   \ifincsname
$     \ltx@ifundefined{HoLogoCs@#1@#2}{%
$       #1%
$     }{%
$       \csname HoLogoCs@#1@#2\endcsname\ltx@secondoftwo
$     }%
$   \else
      \HOLOGO@IfExists\texorpdfstring\texorpdfstring\ltx@firstoftwo
      {%
        \HologoVariantRobust{#1}{#2}%
      }{%
        \ltx@ifundefined{HoLogoBkm@#1@#2}{%
          \ltx@ifundefined{HoLogo@#1}{?#1?}{#1}%
        }{%
          \csname HoLogoBkm@#1@#2\endcsname
          \ltx@secondoftwo
        }%
      }%
$   \fi
  \fi
}
%    \end{macrocode}
%    \end{macro}
%
%    \begin{macrocode}
\catcode`\$=3 %
%    \end{macrocode}
%
% \subsubsection{\cs{hologoRobust} and friends}
%
%    \begin{macro}{\hologoRobust}
%    \begin{macrocode}
\ltx@IfUndefined{protected}{%
  \ltx@IfUndefined{DeclareRobustCommand}{%
    \def\hologoRobust#1%
  }{%
    \DeclareRobustCommand*\hologoRobust[1]%
  }%
}{%
  \protected\def\hologoRobust#1%
}%
{%
  \edef\HOLOGO@name{#1}%
  \ltx@IfUndefined{HoLogo@\HOLOGO@Variant\HOLOGO@name}{%
    \@PackageError{hologo}{%
      Unknown logo `\HOLOGO@name'%
    }\@ehc
    ?\HOLOGO@name?%
  }{%
    \ltx@IfUndefined{ver@tex4ht.sty}{%
      \HoLogoFont@font\HOLOGO@name{general}{%
        \csname HoLogo@\HOLOGO@Variant\HOLOGO@name\endcsname
        \ltx@firstoftwo
      }%
    }{%
      \ltx@IfUndefined{HoLogoHtml@\HOLOGO@Variant\HOLOGO@name}{%
        \HOLOGO@name
      }{%
        \csname HoLogoHtml@\HOLOGO@Variant\HOLOGO@name\endcsname
        \ltx@firstoftwo
      }%
    }%
  }%
}
%    \end{macrocode}
%    \end{macro}
%    \begin{macro}{\HologoRobust}
%    \begin{macrocode}
\ltx@IfUndefined{protected}{%
  \ltx@IfUndefined{DeclareRobustCommand}{%
    \def\HologoRobust#1%
  }{%
    \DeclareRobustCommand*\HologoRobust[1]%
  }%
}{%
  \protected\def\HologoRobust#1%
}%
{%
  \edef\HOLOGO@name{#1}%
  \ltx@IfUndefined{HoLogo@\HOLOGO@Variant\HOLOGO@name}{%
    \@PackageError{hologo}{%
      Unknown logo `\HOLOGO@name'%
    }\@ehc
    ?\HOLOGO@name?%
  }{%
    \ltx@IfUndefined{ver@tex4ht.sty}{%
      \HoLogoFont@font\HOLOGO@name{general}{%
        \csname HoLogo@\HOLOGO@Variant\HOLOGO@name\endcsname
        \ltx@secondoftwo
      }%
    }{%
      \ltx@IfUndefined{HoLogoHtml@\HOLOGO@Variant\HOLOGO@name}{%
        \expandafter\HOLOGO@Uppercase\HOLOGO@name
      }{%
        \csname HoLogoHtml@\HOLOGO@Variant\HOLOGO@name\endcsname
        \ltx@secondoftwo
      }%
    }%
  }%
}
%    \end{macrocode}
%    \end{macro}
%    \begin{macro}{\hologoVariantRobust}
%    \begin{macrocode}
\ltx@IfUndefined{protected}{%
  \ltx@IfUndefined{DeclareRobustCommand}{%
    \def\hologoVariantRobust#1#2%
  }{%
    \DeclareRobustCommand*\hologoVariantRobust[2]%
  }%
}{%
  \protected\def\hologoVariantRobust#1#2%
}%
{%
  \begingroup
    \hologoLogoSetup{#1}{variant={#2}}%
    \hologoRobust{#1}%
  \endgroup
}
%    \end{macrocode}
%    \end{macro}
%    \begin{macro}{\HologoVariantRobust}
%    \begin{macrocode}
\ltx@IfUndefined{protected}{%
  \ltx@IfUndefined{DeclareRobustCommand}{%
    \def\HologoVariantRobust#1#2%
  }{%
    \DeclareRobustCommand*\HologoVariantRobust[2]%
  }%
}{%
  \protected\def\HologoVariantRobust#1#2%
}%
{%
  \begingroup
    \hologoLogoSetup{#1}{variant={#2}}%
    \HologoRobust{#1}%
  \endgroup
}
%    \end{macrocode}
%    \end{macro}
%
%    \begin{macro}{\hologorobust}
%    Macro \cs{hologorobust} is only defined for compatibility.
%    Its use is deprecated.
%    \begin{macrocode}
\def\hologorobust{\hologoRobust}
%    \end{macrocode}
%    \end{macro}
%
% \subsection{Helpers}
%
%    \begin{macro}{\HOLOGO@Uppercase}
%    Macro \cs{HOLOGO@Uppercase} is restricted to \cs{uppercase},
%    because \hologo{plainTeX} or \hologo{iniTeX} do not provide
%    \cs{MakeUppercase}.
%    \begin{macrocode}
\def\HOLOGO@Uppercase#1{\uppercase{#1}}
%    \end{macrocode}
%    \end{macro}
%
%    \begin{macro}{\HOLOGO@PdfdocUnicode}
%    \begin{macrocode}
\def\HOLOGO@PdfdocUnicode{%
  \ifx\ifHy@unicode\iftrue
    \expandafter\ltx@secondoftwo
  \else
    \expandafter\ltx@firstoftwo
  \fi
}
%    \end{macrocode}
%    \end{macro}
%
%    \begin{macro}{\HOLOGO@Math}
%    \begin{macrocode}
\def\HOLOGO@MathSetup{%
  \mathsurround0pt\relax
  \HOLOGO@IfExists\f@series{%
    \if b\expandafter\ltx@car\f@series x\@nil
      \csname boldmath\endcsname
   \fi
  }{}%
}
%    \end{macrocode}
%    \end{macro}
%
%    \begin{macro}{\HOLOGO@TempDimen}
%    \begin{macrocode}
\dimendef\HOLOGO@TempDimen=\ltx@zero
%    \end{macrocode}
%    \end{macro}
%    \begin{macro}{\HOLOGO@NegativeKerning}
%    \begin{macrocode}
\def\HOLOGO@NegativeKerning#1{%
  \begingroup
    \HOLOGO@TempDimen=0pt\relax
    \comma@parse@normalized{#1}{%
      \ifdim\HOLOGO@TempDimen=0pt %
        \expandafter\HOLOGO@@NegativeKerning\comma@entry
      \fi
      \ltx@gobble
    }%
    \ifdim\HOLOGO@TempDimen<0pt %
      \kern\HOLOGO@TempDimen
    \fi
  \endgroup
}
%    \end{macrocode}
%    \end{macro}
%    \begin{macro}{\HOLOGO@@NegativeKerning}
%    \begin{macrocode}
\def\HOLOGO@@NegativeKerning#1#2{%
  \setbox\ltx@zero\hbox{#1#2}%
  \HOLOGO@TempDimen=\wd\ltx@zero
  \setbox\ltx@zero\hbox{#1\kern0pt#2}%
  \advance\HOLOGO@TempDimen by -\wd\ltx@zero
}
%    \end{macrocode}
%    \end{macro}
%
%    \begin{macro}{\HOLOGO@SpaceFactor}
%    \begin{macrocode}
\def\HOLOGO@SpaceFactor{%
  \spacefactor1000 %
}
%    \end{macrocode}
%    \end{macro}
%
%    \begin{macro}{\HOLOGO@Span}
%    \begin{macrocode}
\def\HOLOGO@Span#1#2{%
  \HCode{<span class="HoLogo-#1">}%
  #2%
  \HCode{</span>}%
}
%    \end{macrocode}
%    \end{macro}
%
% \subsubsection{Text subscript}
%
%    \begin{macro}{\HOLOGO@SubScript}%
%    \begin{macrocode}
\def\HOLOGO@SubScript#1{%
  \ltx@IfUndefined{textsubscript}{%
    \ltx@IfUndefined{text}{%
      \ltx@mbox{%
        \mathsurround=0pt\relax
        $%
          _{%
            \ltx@IfUndefined{sf@size}{%
              \mathrm{#1}%
            }{%
              \mbox{%
                \fontsize\sf@size{0pt}\selectfont
                #1%
              }%
            }%
          }%
        $%
      }%
    }{%
      \ltx@mbox{%
        \mathsurround=0pt\relax
        $_{\text{#1}}$%
      }%
    }%
  }{%
    \textsubscript{#1}%
  }%
}
%    \end{macrocode}
%    \end{macro}
%
% \subsection{\hologo{TeX} and friends}
%
% \subsubsection{\hologo{TeX}}
%
%    \begin{macro}{\HoLogo@TeX}
%    Source: \hologo{LaTeX} kernel.
%    \begin{macrocode}
\def\HoLogo@TeX#1{%
  T\kern-.1667em\lower.5ex\hbox{E}\kern-.125emX\HOLOGO@SpaceFactor
}
%    \end{macrocode}
%    \end{macro}
%    \begin{macro}{\HoLogoHtml@TeX}
%    \begin{macrocode}
\def\HoLogoHtml@TeX#1{%
  \HoLogoCss@TeX
  \HOLOGO@Span{TeX}{%
    T%
    \HOLOGO@Span{e}{%
      E%
    }%
    X%
  }%
}
%    \end{macrocode}
%    \end{macro}
%    \begin{macro}{\HoLogoCss@TeX}
%    \begin{macrocode}
\def\HoLogoCss@TeX{%
  \Css{%
    span.HoLogo-TeX span.HoLogo-e{%
      position:relative;%
      top:.5ex;%
      margin-left:-.1667em;%
      margin-right:-.125em;%
    }%
  }%
  \Css{%
    a span.HoLogo-TeX span.HoLogo-e{%
      text-decoration:none;%
    }%
  }%
  \global\let\HoLogoCss@TeX\relax
}
%    \end{macrocode}
%    \end{macro}
%
% \subsubsection{\hologo{plainTeX}}
%
%    \begin{macro}{\HoLogo@plainTeX@space}
%    Source: ``The \hologo{TeX}book''
%    \begin{macrocode}
\def\HoLogo@plainTeX@space#1{%
  \HOLOGO@mbox{#1{p}{P}lain}\HOLOGO@space\hologo{TeX}%
}
%    \end{macrocode}
%    \end{macro}
%    \begin{macro}{\HoLogoCs@plainTeX@space}
%    \begin{macrocode}
\def\HoLogoCs@plainTeX@space#1{#1{p}{P}lain TeX}%
%    \end{macrocode}
%    \end{macro}
%    \begin{macro}{\HoLogoBkm@plainTeX@space}
%    \begin{macrocode}
\def\HoLogoBkm@plainTeX@space#1{%
  #1{p}{P}lain \hologo{TeX}%
}
%    \end{macrocode}
%    \end{macro}
%    \begin{macro}{\HoLogoHtml@plainTeX@space}
%    \begin{macrocode}
\def\HoLogoHtml@plainTeX@space#1{%
  #1{p}{P}lain \hologo{TeX}%
}
%    \end{macrocode}
%    \end{macro}
%
%    \begin{macro}{\HoLogo@plainTeX@hyphen}
%    \begin{macrocode}
\def\HoLogo@plainTeX@hyphen#1{%
  \HOLOGO@mbox{#1{p}{P}lain}\HOLOGO@hyphen\hologo{TeX}%
}
%    \end{macrocode}
%    \end{macro}
%    \begin{macro}{\HoLogoCs@plainTeX@hyphen}
%    \begin{macrocode}
\def\HoLogoCs@plainTeX@hyphen#1{#1{p}{P}lain-TeX}
%    \end{macrocode}
%    \end{macro}
%    \begin{macro}{\HoLogoBkm@plainTeX@hyphen}
%    \begin{macrocode}
\def\HoLogoBkm@plainTeX@hyphen#1{%
  #1{p}{P}lain-\hologo{TeX}%
}
%    \end{macrocode}
%    \end{macro}
%    \begin{macro}{\HoLogoHtml@plainTeX@hyphen}
%    \begin{macrocode}
\def\HoLogoHtml@plainTeX@hyphen#1{%
  #1{p}{P}lain-\hologo{TeX}%
}
%    \end{macrocode}
%    \end{macro}
%
%    \begin{macro}{\HoLogo@plainTeX@runtogether}
%    \begin{macrocode}
\def\HoLogo@plainTeX@runtogether#1{%
  \HOLOGO@mbox{#1{p}{P}lain\hologo{TeX}}%
}
%    \end{macrocode}
%    \end{macro}
%    \begin{macro}{\HoLogoCs@plainTeX@runtogether}
%    \begin{macrocode}
\def\HoLogoCs@plainTeX@runtogether#1{#1{p}{P}lainTeX}
%    \end{macrocode}
%    \end{macro}
%    \begin{macro}{\HoLogoBkm@plainTeX@runtogether}
%    \begin{macrocode}
\def\HoLogoBkm@plainTeX@runtogether#1{%
  #1{p}{P}lain\hologo{TeX}%
}
%    \end{macrocode}
%    \end{macro}
%    \begin{macro}{\HoLogoHtml@plainTeX@runtogether}
%    \begin{macrocode}
\def\HoLogoHtml@plainTeX@runtogether#1{%
  #1{p}{P}lain\hologo{TeX}%
}
%    \end{macrocode}
%    \end{macro}
%
%    \begin{macro}{\HoLogo@plainTeX}
%    \begin{macrocode}
\def\HoLogo@plainTeX{\HoLogo@plainTeX@space}
%    \end{macrocode}
%    \end{macro}
%    \begin{macro}{\HoLogoCs@plainTeX}
%    \begin{macrocode}
\def\HoLogoCs@plainTeX{\HoLogoCs@plainTeX@space}
%    \end{macrocode}
%    \end{macro}
%    \begin{macro}{\HoLogoBkm@plainTeX}
%    \begin{macrocode}
\def\HoLogoBkm@plainTeX{\HoLogoBkm@plainTeX@space}
%    \end{macrocode}
%    \end{macro}
%    \begin{macro}{\HoLogoHtml@plainTeX}
%    \begin{macrocode}
\def\HoLogoHtml@plainTeX{\HoLogoHtml@plainTeX@space}
%    \end{macrocode}
%    \end{macro}
%
% \subsubsection{\hologo{LaTeX}}
%
%    Source: \hologo{LaTeX} kernel.
%\begin{quote}
%\begin{verbatim}
%\DeclareRobustCommand{\LaTeX}{%
%  L%
%  \kern-.36em%
%  {%
%    \sbox\z@ T%
%    \vbox to\ht\z@{%
%      \hbox{%
%        \check@mathfonts
%        \fontsize\sf@size\z@
%        \math@fontsfalse
%        \selectfont
%        A%
%      }%
%      \vss
%    }%
%  }%
%  \kern-.15em%
%  \TeX
%}
%\end{verbatim}
%\end{quote}
%
%    \begin{macro}{\HoLogo@La}
%    \begin{macrocode}
\def\HoLogo@La#1{%
  L%
  \kern-.36em%
  \begingroup
    \setbox\ltx@zero\hbox{T}%
    \vbox to\ht\ltx@zero{%
      \hbox{%
        \ltx@ifundefined{check@mathfonts}{%
          \csname sevenrm\endcsname
        }{%
          \check@mathfonts
          \fontsize\sf@size{0pt}%
          \math@fontsfalse\selectfont
        }%
        A%
      }%
      \vss
    }%
  \endgroup
}
%    \end{macrocode}
%    \end{macro}
%
%    \begin{macro}{\HoLogo@LaTeX}
%    Source: \hologo{LaTeX} kernel.
%    \begin{macrocode}
\def\HoLogo@LaTeX#1{%
  \hologo{La}%
  \kern-.15em%
  \hologo{TeX}%
}
%    \end{macrocode}
%    \end{macro}
%    \begin{macro}{\HoLogoHtml@LaTeX}
%    \begin{macrocode}
\def\HoLogoHtml@LaTeX#1{%
  \HoLogoCss@LaTeX
  \HOLOGO@Span{LaTeX}{%
    L%
    \HOLOGO@Span{a}{%
      A%
    }%
    \hologo{TeX}%
  }%
}
%    \end{macrocode}
%    \end{macro}
%    \begin{macro}{\HoLogoCss@LaTeX}
%    \begin{macrocode}
\def\HoLogoCss@LaTeX{%
  \Css{%
    span.HoLogo-LaTeX span.HoLogo-a{%
      position:relative;%
      top:-.5ex;%
      margin-left:-.36em;%
      margin-right:-.15em;%
      font-size:85\%;%
    }%
  }%
  \global\let\HoLogoCss@LaTeX\relax
}
%    \end{macrocode}
%    \end{macro}
%
% \subsubsection{\hologo{(La)TeX}}
%
%    \begin{macro}{\HoLogo@LaTeXTeX}
%    The kerning around the parentheses is taken
%    from package \xpackage{dtklogos} \cite{dtklogos}.
%\begin{quote}
%\begin{verbatim}
%\DeclareRobustCommand{\LaTeXTeX}{%
%  (%
%  \kern-.15em%
%  L%
%  \kern-.36em%
%  {%
%    \sbox\z@ T%
%    \vbox to\ht0{%
%      \hbox{%
%        $\m@th$%
%        \csname S@\f@size\endcsname
%        \fontsize\sf@size\z@
%        \math@fontsfalse
%        \selectfont
%        A%
%      }%
%      \vss
%    }%
%  }%
%  \kern-.2em%
%  )%
%  \kern-.15em%
%  \TeX
%}
%\end{verbatim}
%\end{quote}
%    \begin{macrocode}
\def\HoLogo@LaTeXTeX#1{%
  (%
  \kern-.15em%
  \hologo{La}%
  \kern-.2em%
  )%
  \kern-.15em%
  \hologo{TeX}%
}
%    \end{macrocode}
%    \end{macro}
%    \begin{macro}{\HoLogoBkm@LaTeXTeX}
%    \begin{macrocode}
\def\HoLogoBkm@LaTeXTeX#1{(La)TeX}
%    \end{macrocode}
%    \end{macro}
%
%    \begin{macro}{\HoLogo@(La)TeX}
%    \begin{macrocode}
\expandafter
\let\csname HoLogo@(La)TeX\endcsname\HoLogo@LaTeXTeX
%    \end{macrocode}
%    \end{macro}
%    \begin{macro}{\HoLogoBkm@(La)TeX}
%    \begin{macrocode}
\expandafter
\let\csname HoLogoBkm@(La)TeX\endcsname\HoLogoBkm@LaTeXTeX
%    \end{macrocode}
%    \end{macro}
%    \begin{macro}{\HoLogoHtml@LaTeXTeX}
%    \begin{macrocode}
\def\HoLogoHtml@LaTeXTeX#1{%
  \HoLogoCss@LaTeXTeX
  \HOLOGO@Span{LaTeXTeX}{%
    (%
    \HOLOGO@Span{L}{L}%
    \HOLOGO@Span{a}{A}%
    \HOLOGO@Span{ParenRight}{)}%
    \hologo{TeX}%
  }%
}
%    \end{macrocode}
%    \end{macro}
%    \begin{macro}{\HoLogoHtml@(La)TeX}
%    Kerning after opening parentheses and before closing parentheses
%    is $-0.1$\,em. The original values $-0.15$\,em
%    looked too ugly for a serif font.
%    \begin{macrocode}
\expandafter
\let\csname HoLogoHtml@(La)TeX\endcsname\HoLogoHtml@LaTeXTeX
%    \end{macrocode}
%    \end{macro}
%    \begin{macro}{\HoLogoCss@LaTeXTeX}
%    \begin{macrocode}
\def\HoLogoCss@LaTeXTeX{%
  \Css{%
    span.HoLogo-LaTeXTeX span.HoLogo-L{%
      margin-left:-.1em;%
    }%
  }%
  \Css{%
    span.HoLogo-LaTeXTeX span.HoLogo-a{%
      position:relative;%
      top:-.5ex;%
      margin-left:-.36em;%
      margin-right:-.1em;%
      font-size:85\%;%
    }%
  }%
  \Css{%
    span.HoLogo-LaTeXTeX span.HoLogo-ParenRight{%
      margin-right:-.15em;%
    }%
  }%
  \global\let\HoLogoCss@LaTeXTeX\relax
}
%    \end{macrocode}
%    \end{macro}
%
% \subsubsection{\hologo{LaTeXe}}
%
%    \begin{macro}{\HoLogo@LaTeXe}
%    Source: \hologo{LaTeX} kernel
%    \begin{macrocode}
\def\HoLogo@LaTeXe#1{%
  \hologo{LaTeX}%
  \kern.15em%
  \hbox{%
    \HOLOGO@MathSetup
    2%
    $_{\textstyle\varepsilon}$%
  }%
}
%    \end{macrocode}
%    \end{macro}
%
%    \begin{macro}{\HoLogoCs@LaTeXe}
%    \begin{macrocode}
\ifnum64=`\^^^^0040\relax % test for big chars of LuaTeX/XeTeX
  \catcode`\$=9 %
  \catcode`\&=14 %
\else
  \catcode`\$=14 %
  \catcode`\&=9 %
\fi
\def\HoLogoCs@LaTeXe#1{%
  LaTeX2%
$ \string ^^^^0395%
& e%
}%
\catcode`\$=3 %
\catcode`\&=4 %
%    \end{macrocode}
%    \end{macro}
%
%    \begin{macro}{\HoLogoBkm@LaTeXe}
%    \begin{macrocode}
\def\HoLogoBkm@LaTeXe#1{%
  \hologo{LaTeX}%
  2%
  \HOLOGO@PdfdocUnicode{e}{\textepsilon}%
}
%    \end{macrocode}
%    \end{macro}
%
%    \begin{macro}{\HoLogoHtml@LaTeXe}
%    \begin{macrocode}
\def\HoLogoHtml@LaTeXe#1{%
  \HoLogoCss@LaTeXe
  \HOLOGO@Span{LaTeX2e}{%
    \hologo{LaTeX}%
    \HOLOGO@Span{2}{2}%
    \HOLOGO@Span{e}{%
      \HOLOGO@MathSetup
      \ensuremath{\textstyle\varepsilon}%
    }%
  }%
}
%    \end{macrocode}
%    \end{macro}
%    \begin{macro}{\HoLogoCss@LaTeXe}
%    \begin{macrocode}
\def\HoLogoCss@LaTeXe{%
  \Css{%
    span.HoLogo-LaTeX2e span.HoLogo-2{%
      padding-left:.15em;%
    }%
  }%
  \Css{%
    span.HoLogo-LaTeX2e span.HoLogo-e{%
      position:relative;%
      top:.35ex;%
      text-decoration:none;%
    }%
  }%
  \global\let\HoLogoCss@LaTeXe\relax
}
%    \end{macrocode}
%    \end{macro}
%
%    \begin{macro}{\HoLogo@LaTeX2e}
%    \begin{macrocode}
\expandafter
\let\csname HoLogo@LaTeX2e\endcsname\HoLogo@LaTeXe
%    \end{macrocode}
%    \end{macro}
%    \begin{macro}{\HoLogoCs@LaTeX2e}
%    \begin{macrocode}
\expandafter
\let\csname HoLogoCs@LaTeX2e\endcsname\HoLogoCs@LaTeXe
%    \end{macrocode}
%    \end{macro}
%    \begin{macro}{\HoLogoBkm@LaTeX2e}
%    \begin{macrocode}
\expandafter
\let\csname HoLogoBkm@LaTeX2e\endcsname\HoLogoBkm@LaTeXe
%    \end{macrocode}
%    \end{macro}
%    \begin{macro}{\HoLogoHtml@LaTeX2e}
%    \begin{macrocode}
\expandafter
\let\csname HoLogoHtml@LaTeX2e\endcsname\HoLogoHtml@LaTeXe
%    \end{macrocode}
%    \end{macro}
%
% \subsubsection{\hologo{LaTeX3}}
%
%    \begin{macro}{\HoLogo@LaTeX3}
%    Source: \hologo{LaTeX} kernel
%    \begin{macrocode}
\expandafter\def\csname HoLogo@LaTeX3\endcsname#1{%
  \hologo{LaTeX}%
  3%
}
%    \end{macrocode}
%    \end{macro}
%
%    \begin{macro}{\HoLogoBkm@LaTeX3}
%    \begin{macrocode}
\expandafter\def\csname HoLogoBkm@LaTeX3\endcsname#1{%
  \hologo{LaTeX}%
  3%
}
%    \end{macrocode}
%    \end{macro}
%    \begin{macro}{\HoLogoHtml@LaTeX3}
%    \begin{macrocode}
\expandafter
\let\csname HoLogoHtml@LaTeX3\expandafter\endcsname
\csname HoLogo@LaTeX3\endcsname
%    \end{macrocode}
%    \end{macro}
%
% \subsubsection{\hologo{LaTeXML}}
%
%    \begin{macro}{\HoLogo@LaTeXML}
%    \begin{macrocode}
\def\HoLogo@LaTeXML#1{%
  \HOLOGO@mbox{%
    \hologo{La}%
    \kern-.15em%
    T%
    \kern-.1667em%
    \lower.5ex\hbox{E}%
    \kern-.125em%
    \HoLogoFont@font{LaTeXML}{sc}{xml}%
  }%
}
%    \end{macrocode}
%    \end{macro}
%    \begin{macro}{\HoLogoHtml@pdfLaTeX}
%    \begin{macrocode}
\def\HoLogoHtml@LaTeXML#1{%
  \HOLOGO@Span{LaTeXML}{%
    \HoLogoCss@LaTeX
    \HoLogoCss@TeX
    \HOLOGO@Span{LaTeX}{%
      L%
      \HOLOGO@Span{a}{%
        A%
      }%
    }%
    \HOLOGO@Span{TeX}{%
      T%
      \HOLOGO@Span{e}{%
        E%
      }%
    }%
    \HCode{<span style="font-variant: small-caps;">}%
    xml%
    \HCode{</span>}%
  }%
}
%    \end{macrocode}
%    \end{macro}
%
% \subsubsection{\hologo{eTeX}}
%
%    \begin{macro}{\HoLogo@eTeX}
%    Source: package \xpackage{etex}
%    \begin{macrocode}
\def\HoLogo@eTeX#1{%
  \ltx@mbox{%
    \HOLOGO@MathSetup
    $\varepsilon$%
    -%
    \HOLOGO@NegativeKerning{-T,T-,To}%
    \hologo{TeX}%
  }%
}
%    \end{macrocode}
%    \end{macro}
%    \begin{macro}{\HoLogoCs@eTeX}
%    \begin{macrocode}
\ifnum64=`\^^^^0040\relax % test for big chars of LuaTeX/XeTeX
  \catcode`\$=9 %
  \catcode`\&=14 %
\else
  \catcode`\$=14 %
  \catcode`\&=9 %
\fi
\def\HoLogoCs@eTeX#1{%
$ #1{\string ^^^^0395}{\string ^^^^03b5}%
& #1{e}{E}%
  TeX%
}%
\catcode`\$=3 %
\catcode`\&=4 %
%    \end{macrocode}
%    \end{macro}
%    \begin{macro}{\HoLogoBkm@eTeX}
%    \begin{macrocode}
\def\HoLogoBkm@eTeX#1{%
  \HOLOGO@PdfdocUnicode{#1{e}{E}}{\textepsilon}%
  -%
  \hologo{TeX}%
}
%    \end{macrocode}
%    \end{macro}
%    \begin{macro}{\HoLogoHtml@eTeX}
%    \begin{macrocode}
\def\HoLogoHtml@eTeX#1{%
  \ltx@mbox{%
    \HOLOGO@MathSetup
    $\varepsilon$%
    -%
    \hologo{TeX}%
  }%
}
%    \end{macrocode}
%    \end{macro}
%
% \subsubsection{\hologo{iniTeX}}
%
%    \begin{macro}{\HoLogo@iniTeX}
%    \begin{macrocode}
\def\HoLogo@iniTeX#1{%
  \HOLOGO@mbox{%
    #1{i}{I}ni\hologo{TeX}%
  }%
}
%    \end{macrocode}
%    \end{macro}
%    \begin{macro}{\HoLogoCs@iniTeX}
%    \begin{macrocode}
\def\HoLogoCs@iniTeX#1{#1{i}{I}niTeX}
%    \end{macrocode}
%    \end{macro}
%    \begin{macro}{\HoLogoBkm@iniTeX}
%    \begin{macrocode}
\def\HoLogoBkm@iniTeX#1{%
  #1{i}{I}ni\hologo{TeX}%
}
%    \end{macrocode}
%    \end{macro}
%    \begin{macro}{\HoLogoHtml@iniTeX}
%    \begin{macrocode}
\let\HoLogoHtml@iniTeX\HoLogo@iniTeX
%    \end{macrocode}
%    \end{macro}
%
% \subsubsection{\hologo{virTeX}}
%
%    \begin{macro}{\HoLogo@virTeX}
%    \begin{macrocode}
\def\HoLogo@virTeX#1{%
  \HOLOGO@mbox{%
    #1{v}{V}ir\hologo{TeX}%
  }%
}
%    \end{macrocode}
%    \end{macro}
%    \begin{macro}{\HoLogoCs@virTeX}
%    \begin{macrocode}
\def\HoLogoCs@virTeX#1{#1{v}{V}irTeX}
%    \end{macrocode}
%    \end{macro}
%    \begin{macro}{\HoLogoBkm@virTeX}
%    \begin{macrocode}
\def\HoLogoBkm@virTeX#1{%
  #1{v}{V}ir\hologo{TeX}%
}
%    \end{macrocode}
%    \end{macro}
%    \begin{macro}{\HoLogoHtml@virTeX}
%    \begin{macrocode}
\let\HoLogoHtml@virTeX\HoLogo@virTeX
%    \end{macrocode}
%    \end{macro}
%
% \subsubsection{\hologo{SliTeX}}
%
% \paragraph{Definitions of the three variants.}
%
%    \begin{macro}{\HoLogo@SLiTeX@lift}
%    \begin{macrocode}
\def\HoLogo@SLiTeX@lift#1{%
  \HoLogoFont@font{SliTeX}{rm}{%
    S%
    \kern-.06em%
    L%
    \kern-.18em%
    \raise.32ex\hbox{\HoLogoFont@font{SliTeX}{sc}{i}}%
    \HOLOGO@discretionary
    \kern-.06em%
    \hologo{TeX}%
  }%
}
%    \end{macrocode}
%    \end{macro}
%    \begin{macro}{\HoLogoBkm@SLiTeX@lift}
%    \begin{macrocode}
\def\HoLogoBkm@SLiTeX@lift#1{SLiTeX}
%    \end{macrocode}
%    \end{macro}
%    \begin{macro}{\HoLogoHtml@SLiTeX@lift}
%    \begin{macrocode}
\def\HoLogoHtml@SLiTeX@lift#1{%
  \HoLogoCss@SLiTeX@lift
  \HOLOGO@Span{SLiTeX-lift}{%
    \HoLogoFont@font{SliTeX}{rm}{%
      S%
      \HOLOGO@Span{L}{L}%
      \HOLOGO@Span{i}{i}%
      \hologo{TeX}%
    }%
  }%
}
%    \end{macrocode}
%    \end{macro}
%    \begin{macro}{\HoLogoCss@SLiTeX@lift}
%    \begin{macrocode}
\def\HoLogoCss@SLiTeX@lift{%
  \Css{%
    span.HoLogo-SLiTeX-lift span.HoLogo-L{%
      margin-left:-.06em;%
      margin-right:-.18em;%
    }%
  }%
  \Css{%
    span.HoLogo-SLiTeX-lift span.HoLogo-i{%
      position:relative;%
      top:-.32ex;%
      margin-right:-.06em;%
      font-variant:small-caps;%
    }%
  }%
  \global\let\HoLogoCss@SLiTeX@lift\relax
}
%    \end{macrocode}
%    \end{macro}
%
%    \begin{macro}{\HoLogo@SliTeX@simple}
%    \begin{macrocode}
\def\HoLogo@SliTeX@simple#1{%
  \HoLogoFont@font{SliTeX}{rm}{%
    \ltx@mbox{%
      \HoLogoFont@font{SliTeX}{sc}{Sli}%
    }%
    \HOLOGO@discretionary
    \hologo{TeX}%
  }%
}
%    \end{macrocode}
%    \end{macro}
%    \begin{macro}{\HoLogoBkm@SliTeX@simple}
%    \begin{macrocode}
\def\HoLogoBkm@SliTeX@simple#1{SliTeX}
%    \end{macrocode}
%    \end{macro}
%    \begin{macro}{\HoLogoHtml@SliTeX@simple}
%    \begin{macrocode}
\let\HoLogoHtml@SliTeX@simple\HoLogo@SliTeX@simple
%    \end{macrocode}
%    \end{macro}
%
%    \begin{macro}{\HoLogo@SliTeX@narrow}
%    \begin{macrocode}
\def\HoLogo@SliTeX@narrow#1{%
  \HoLogoFont@font{SliTeX}{rm}{%
    \ltx@mbox{%
      S%
      \kern-.06em%
      \HoLogoFont@font{SliTeX}{sc}{%
        l%
        \kern-.035em%
        i%
      }%
    }%
    \HOLOGO@discretionary
    \kern-.06em%
    \hologo{TeX}%
  }%
}
%    \end{macrocode}
%    \end{macro}
%    \begin{macro}{\HoLogoBkm@SliTeX@narrow}
%    \begin{macrocode}
\def\HoLogoBkm@SliTeX@narrow#1{SliTeX}
%    \end{macrocode}
%    \end{macro}
%    \begin{macro}{\HoLogoHtml@SliTeX@narrow}
%    \begin{macrocode}
\def\HoLogoHtml@SliTeX@narrow#1{%
  \HoLogoCss@SliTeX@narrow
  \HOLOGO@Span{SliTeX-narrow}{%
    \HoLogoFont@font{SliTeX}{rm}{%
      S%
        \HOLOGO@Span{l}{l}%
        \HOLOGO@Span{i}{i}%
      \hologo{TeX}%
    }%
  }%
}
%    \end{macrocode}
%    \end{macro}
%    \begin{macro}{\HoLogoCss@SliTeX@narrow}
%    \begin{macrocode}
\def\HoLogoCss@SliTeX@narrow{%
  \Css{%
    span.HoLogo-SliTeX-narrow span.HoLogo-l{%
      margin-left:-.06em;%
      margin-right:-.035em;%
      font-variant:small-caps;%
    }%
  }%
  \Css{%
    span.HoLogo-SliTeX-narrow span.HoLogo-i{%
      margin-right:-.06em;%
      font-variant:small-caps;%
    }%
  }%
  \global\let\HoLogoCss@SliTeX@narrow\relax
}
%    \end{macrocode}
%    \end{macro}
%
% \paragraph{Macro set completion.}
%
%    \begin{macro}{\HoLogo@SLiTeX@simple}
%    \begin{macrocode}
\def\HoLogo@SLiTeX@simple{\HoLogo@SliTeX@simple}
%    \end{macrocode}
%    \end{macro}
%    \begin{macro}{\HoLogoBkm@SLiTeX@simple}
%    \begin{macrocode}
\def\HoLogoBkm@SLiTeX@simple{\HoLogoBkm@SliTeX@simple}
%    \end{macrocode}
%    \end{macro}
%    \begin{macro}{\HoLogoHtml@SLiTeX@simple}
%    \begin{macrocode}
\def\HoLogoHtml@SLiTeX@simple{\HoLogoHtml@SliTeX@simple}
%    \end{macrocode}
%    \end{macro}
%
%    \begin{macro}{\HoLogo@SLiTeX@narrow}
%    \begin{macrocode}
\def\HoLogo@SLiTeX@narrow{\HoLogo@SliTeX@narrow}
%    \end{macrocode}
%    \end{macro}
%    \begin{macro}{\HoLogoBkm@SLiTeX@narrow}
%    \begin{macrocode}
\def\HoLogoBkm@SLiTeX@narrow{\HoLogoBkm@SliTeX@narrow}
%    \end{macrocode}
%    \end{macro}
%    \begin{macro}{\HoLogoHtml@SLiTeX@narrow}
%    \begin{macrocode}
\def\HoLogoHtml@SLiTeX@narrow{\HoLogoHtml@SliTeX@narrow}
%    \end{macrocode}
%    \end{macro}
%
%    \begin{macro}{\HoLogo@SliTeX@lift}
%    \begin{macrocode}
\def\HoLogo@SliTeX@lift{\HoLogo@SLiTeX@lift}
%    \end{macrocode}
%    \end{macro}
%    \begin{macro}{\HoLogoBkm@SliTeX@lift}
%    \begin{macrocode}
\def\HoLogoBkm@SliTeX@lift{\HoLogoBkm@SLiTeX@lift}
%    \end{macrocode}
%    \end{macro}
%    \begin{macro}{\HoLogoHtml@SliTeX@lift}
%    \begin{macrocode}
\def\HoLogoHtml@SliTeX@lift{\HoLogoHtml@SLiTeX@lift}
%    \end{macrocode}
%    \end{macro}
%
% \paragraph{Defaults.}
%
%    \begin{macro}{\HoLogo@SLiTeX}
%    \begin{macrocode}
\def\HoLogo@SLiTeX{\HoLogo@SLiTeX@lift}
%    \end{macrocode}
%    \end{macro}
%    \begin{macro}{\HoLogoBkm@SLiTeX}
%    \begin{macrocode}
\def\HoLogoBkm@SLiTeX{\HoLogoBkm@SLiTeX@lift}
%    \end{macrocode}
%    \end{macro}
%    \begin{macro}{\HoLogoHtml@SLiTeX}
%    \begin{macrocode}
\def\HoLogoHtml@SLiTeX{\HoLogoHtml@SLiTeX@lift}
%    \end{macrocode}
%    \end{macro}
%
%    \begin{macro}{\HoLogo@SliTeX}
%    \begin{macrocode}
\def\HoLogo@SliTeX{\HoLogo@SliTeX@narrow}
%    \end{macrocode}
%    \end{macro}
%    \begin{macro}{\HoLogoBkm@SliTeX}
%    \begin{macrocode}
\def\HoLogoBkm@SliTeX{\HoLogoBkm@SliTeX@narrow}
%    \end{macrocode}
%    \end{macro}
%    \begin{macro}{\HoLogoHtml@SliTeX}
%    \begin{macrocode}
\def\HoLogoHtml@SliTeX{\HoLogoHtml@SliTeX@narrow}
%    \end{macrocode}
%    \end{macro}
%
% \subsubsection{\hologo{LuaTeX}}
%
%    \begin{macro}{\HoLogo@LuaTeX}
%    The kerning is an idea of Hans Hagen, see mailing list
%    `luatex at tug dot org' in March 2010.
%    \begin{macrocode}
\def\HoLogo@LuaTeX#1{%
  \HOLOGO@mbox{%
    Lua%
    \HOLOGO@NegativeKerning{aT,oT,To}%
    \hologo{TeX}%
  }%
}
%    \end{macrocode}
%    \end{macro}
%    \begin{macro}{\HoLogoHtml@LuaTeX}
%    \begin{macrocode}
\let\HoLogoHtml@LuaTeX\HoLogo@LuaTeX
%    \end{macrocode}
%    \end{macro}
%
% \subsubsection{\hologo{LuaLaTeX}}
%
%    \begin{macro}{\HoLogo@LuaLaTeX}
%    \begin{macrocode}
\def\HoLogo@LuaLaTeX#1{%
  \HOLOGO@mbox{%
    Lua%
    \hologo{LaTeX}%
  }%
}
%    \end{macrocode}
%    \end{macro}
%    \begin{macro}{\HoLogoHtml@LuaLaTeX}
%    \begin{macrocode}
\let\HoLogoHtml@LuaLaTeX\HoLogo@LuaLaTeX
%    \end{macrocode}
%    \end{macro}
%
% \subsubsection{\hologo{XeTeX}, \hologo{XeLaTeX}}
%
%    \begin{macro}{\HOLOGO@IfCharExists}
%    \begin{macrocode}
\ifluatex
  \ifnum\luatexversion<36 %
  \else
    \def\HOLOGO@IfCharExists#1{%
      \ifnum
        \directlua{%
           if luaotfload and luaotfload.aux then
             if luaotfload.aux.font_has_glyph(%
                    font.current(), \number#1) then % 	 
	       tex.print("1") % 	 
	     end % 	 
	   elseif font and font.fonts and font.current then %
            local f = font.fonts[font.current()]%
            if f.characters and f.characters[\number#1] then %
              tex.print("1")%
            end %
          end%
        }0=\ltx@zero
        \expandafter\ltx@secondoftwo
      \else
        \expandafter\ltx@firstoftwo
      \fi
    }%
  \fi
\fi
\ltx@IfUndefined{HOLOGO@IfCharExists}{%
  \def\HOLOGO@@IfCharExists#1{%
    \begingroup
      \tracinglostchars=\ltx@zero
      \setbox\ltx@zero=\hbox{%
        \kern7sp\char#1\relax
        \ifnum\lastkern>\ltx@zero
          \expandafter\aftergroup\csname iffalse\endcsname
        \else
          \expandafter\aftergroup\csname iftrue\endcsname
        \fi
      }%
      % \if{true|false} from \aftergroup
      \endgroup
      \expandafter\ltx@firstoftwo
    \else
      \endgroup
      \expandafter\ltx@secondoftwo
    \fi
  }%
  \ifxetex
    \ltx@IfUndefined{XeTeXfonttype}{}{%
      \ltx@IfUndefined{XeTeXcharglyph}{}{%
        \def\HOLOGO@IfCharExists#1{%
          \ifnum\XeTeXfonttype\font>\ltx@zero
            \expandafter\ltx@firstofthree
          \else
            \expandafter\ltx@gobble
          \fi
          {%
            \ifnum\XeTeXcharglyph#1>\ltx@zero
              \expandafter\ltx@firstoftwo
            \else
              \expandafter\ltx@secondoftwo
            \fi
          }%
          \HOLOGO@@IfCharExists{#1}%
        }%
      }%
    }%
  \fi
}{}
\ltx@ifundefined{HOLOGO@IfCharExists}{%
  \ifnum64=`\^^^^0040\relax % test for big chars of LuaTeX/XeTeX
    \let\HOLOGO@IfCharExists\HOLOGO@@IfCharExists
  \else
    \def\HOLOGO@IfCharExists#1{%
      \ifnum#1>255 %
        \expandafter\ltx@fourthoffour
      \fi
      \HOLOGO@@IfCharExists{#1}%
    }%
  \fi
}{}
%    \end{macrocode}
%    \end{macro}
%
%    \begin{macro}{\HoLogo@Xe}
%    Source: package \xpackage{dtklogos}
%    \begin{macrocode}
\def\HoLogo@Xe#1{%
  X%
  \kern-.1em\relax
  \HOLOGO@IfCharExists{"018E}{%
    \lower.5ex\hbox{\char"018E}%
  }{%
    \chardef\HOLOGO@choice=\ltx@zero
    \ifdim\fontdimen\ltx@one\font>0pt %
      \ltx@IfUndefined{rotatebox}{%
        \ltx@IfUndefined{pgftext}{%
          \ltx@IfUndefined{psscalebox}{%
            \ltx@IfUndefined{HOLOGO@ScaleBox@\hologoDriver}{%
            }{%
              \chardef\HOLOGO@choice=4 %
            }%
          }{%
            \chardef\HOLOGO@choice=3 %
          }%
        }{%
          \chardef\HOLOGO@choice=2 %
        }%
      }{%
        \chardef\HOLOGO@choice=1 %
      }%
      \ifcase\HOLOGO@choice
        \HOLOGO@WarningUnsupportedDriver{Xe}%
        e%
      \or % 1: \rotatebox
        \begingroup
          \setbox\ltx@zero\hbox{\rotatebox{180}{E}}%
          \ltx@LocDimenA=\dp\ltx@zero
          \advance\ltx@LocDimenA by -.5ex\relax
          \raise\ltx@LocDimenA\box\ltx@zero
        \endgroup
      \or % 2: \pgftext
        \lower.5ex\hbox{%
          \pgfpicture
            \pgftext[rotate=180]{E}%
          \endpgfpicture
        }%
      \or % 3: \psscalebox
        \begingroup
          \setbox\ltx@zero\hbox{\psscalebox{-1 -1}{E}}%
          \ltx@LocDimenA=\dp\ltx@zero
          \advance\ltx@LocDimenA by -.5ex\relax
          \raise\ltx@LocDimenA\box\ltx@zero
        \endgroup
      \or % 4: \HOLOGO@PointReflectBox
        \lower.5ex\hbox{\HOLOGO@PointReflectBox{E}}%
      \else
        \@PackageError{hologo}{Internal error (choice/it}\@ehc
      \fi
    \else
      \ltx@IfUndefined{reflectbox}{%
        \ltx@IfUndefined{pgftext}{%
          \ltx@IfUndefined{psscalebox}{%
            \ltx@IfUndefined{HOLOGO@ScaleBox@\hologoDriver}{%
            }{%
              \chardef\HOLOGO@choice=4 %
            }%
          }{%
            \chardef\HOLOGO@choice=3 %
          }%
        }{%
          \chardef\HOLOGO@choice=2 %
        }%
      }{%
        \chardef\HOLOGO@choice=1 %
      }%
      \ifcase\HOLOGO@choice
        \HOLOGO@WarningUnsupportedDriver{Xe}%
        e%
      \or % 1: reflectbox
        \lower.5ex\hbox{%
          \reflectbox{E}%
        }%
      \or % 2: \pgftext
        \lower.5ex\hbox{%
          \pgfpicture
            \pgftransformxscale{-1}%
            \pgftext{E}%
          \endpgfpicture
        }%
      \or % 3: \psscalebox
        \lower.5ex\hbox{%
          \psscalebox{-1 1}{E}%
        }%
      \or % 4: \HOLOGO@Reflectbox
        \lower.5ex\hbox{%
          \HOLOGO@ReflectBox{E}%
        }%
      \else
        \@PackageError{hologo}{Internal error (choice/up)}\@ehc
      \fi
    \fi
  }%
}
%    \end{macrocode}
%    \end{macro}
%    \begin{macro}{\HoLogoHtml@Xe}
%    \begin{macrocode}
\def\HoLogoHtml@Xe#1{%
  \HoLogoCss@Xe
  \HOLOGO@Span{Xe}{%
    X%
    \HOLOGO@Span{e}{%
      \HCode{&\ltx@hashchar x018e;}%
    }%
  }%
}
%    \end{macrocode}
%    \end{macro}
%    \begin{macro}{\HoLogoCss@Xe}
%    \begin{macrocode}
\def\HoLogoCss@Xe{%
  \Css{%
    span.HoLogo-Xe span.HoLogo-e{%
      position:relative;%
      top:.5ex;%
      left-margin:-.1em;%
    }%
  }%
  \global\let\HoLogoCss@Xe\relax
}
%    \end{macrocode}
%    \end{macro}
%
%    \begin{macro}{\HoLogo@XeTeX}
%    \begin{macrocode}
\def\HoLogo@XeTeX#1{%
  \hologo{Xe}%
  \kern-.15em\relax
  \hologo{TeX}%
}
%    \end{macrocode}
%    \end{macro}
%
%    \begin{macro}{\HoLogoHtml@XeTeX}
%    \begin{macrocode}
\def\HoLogoHtml@XeTeX#1{%
  \HoLogoCss@XeTeX
  \HOLOGO@Span{XeTeX}{%
    \hologo{Xe}%
    \hologo{TeX}%
  }%
}
%    \end{macrocode}
%    \end{macro}
%    \begin{macro}{\HoLogoCss@XeTeX}
%    \begin{macrocode}
\def\HoLogoCss@XeTeX{%
  \Css{%
    span.HoLogo-XeTeX span.HoLogo-TeX{%
      margin-left:-.15em;%
    }%
  }%
  \global\let\HoLogoCss@XeTeX\relax
}
%    \end{macrocode}
%    \end{macro}
%
%    \begin{macro}{\HoLogo@XeLaTeX}
%    \begin{macrocode}
\def\HoLogo@XeLaTeX#1{%
  \hologo{Xe}%
  \kern-.13em%
  \hologo{LaTeX}%
}
%    \end{macrocode}
%    \end{macro}
%    \begin{macro}{\HoLogoHtml@XeLaTeX}
%    \begin{macrocode}
\def\HoLogoHtml@XeLaTeX#1{%
  \HoLogoCss@XeLaTeX
  \HOLOGO@Span{XeLaTeX}{%
    \hologo{Xe}%
    \hologo{LaTeX}%
  }%
}
%    \end{macrocode}
%    \end{macro}
%    \begin{macro}{\HoLogoCss@XeLaTeX}
%    \begin{macrocode}
\def\HoLogoCss@XeLaTeX{%
  \Css{%
    span.HoLogo-XeLaTeX span.HoLogo-Xe{%
      margin-right:-.13em;%
    }%
  }%
  \global\let\HoLogoCss@XeLaTeX\relax
}
%    \end{macrocode}
%    \end{macro}
%
% \subsubsection{\hologo{pdfTeX}, \hologo{pdfLaTeX}}
%
%    \begin{macro}{\HoLogo@pdfTeX}
%    \begin{macrocode}
\def\HoLogo@pdfTeX#1{%
  \HOLOGO@mbox{%
    #1{p}{P}df\hologo{TeX}%
  }%
}
%    \end{macrocode}
%    \end{macro}
%    \begin{macro}{\HoLogoCs@pdfTeX}
%    \begin{macrocode}
\def\HoLogoCs@pdfTeX#1{#1{p}{P}dfTeX}
%    \end{macrocode}
%    \end{macro}
%    \begin{macro}{\HoLogoBkm@pdfTeX}
%    \begin{macrocode}
\def\HoLogoBkm@pdfTeX#1{%
  #1{p}{P}df\hologo{TeX}%
}
%    \end{macrocode}
%    \end{macro}
%    \begin{macro}{\HoLogoHtml@pdfTeX}
%    \begin{macrocode}
\let\HoLogoHtml@pdfTeX\HoLogo@pdfTeX
%    \end{macrocode}
%    \end{macro}
%
%    \begin{macro}{\HoLogo@pdfLaTeX}
%    \begin{macrocode}
\def\HoLogo@pdfLaTeX#1{%
  \HOLOGO@mbox{%
    #1{p}{P}df\hologo{LaTeX}%
  }%
}
%    \end{macrocode}
%    \end{macro}
%    \begin{macro}{\HoLogoCs@pdfLaTeX}
%    \begin{macrocode}
\def\HoLogoCs@pdfLaTeX#1{#1{p}{P}dfLaTeX}
%    \end{macrocode}
%    \end{macro}
%    \begin{macro}{\HoLogoBkm@pdfLaTeX}
%    \begin{macrocode}
\def\HoLogoBkm@pdfLaTeX#1{%
  #1{p}{P}df\hologo{LaTeX}%
}
%    \end{macrocode}
%    \end{macro}
%    \begin{macro}{\HoLogoHtml@pdfLaTeX}
%    \begin{macrocode}
\let\HoLogoHtml@pdfLaTeX\HoLogo@pdfLaTeX
%    \end{macrocode}
%    \end{macro}
%
% \subsubsection{\hologo{VTeX}}
%
%    \begin{macro}{\HoLogo@VTeX}
%    \begin{macrocode}
\def\HoLogo@VTeX#1{%
  \HOLOGO@mbox{%
    V\hologo{TeX}%
  }%
}
%    \end{macrocode}
%    \end{macro}
%    \begin{macro}{\HoLogoHtml@VTeX}
%    \begin{macrocode}
\let\HoLogoHtml@VTeX\HoLogo@VTeX
%    \end{macrocode}
%    \end{macro}
%
% \subsubsection{\hologo{AmS}, \dots}
%
%    Source: class \xclass{amsdtx}
%
%    \begin{macro}{\HoLogo@AmS}
%    \begin{macrocode}
\def\HoLogo@AmS#1{%
  \HoLogoFont@font{AmS}{sy}{%
    A%
    \kern-.1667em%
    \lower.5ex\hbox{M}%
    \kern-.125em%
    S%
  }%
}
%    \end{macrocode}
%    \end{macro}
%    \begin{macro}{\HoLogoBkm@AmS}
%    \begin{macrocode}
\def\HoLogoBkm@AmS#1{AmS}
%    \end{macrocode}
%    \end{macro}
%    \begin{macro}{\HoLogoHtml@AmS}
%    \begin{macrocode}
\def\HoLogoHtml@AmS#1{%
  \HoLogoCss@AmS
%  \HoLogoFont@font{AmS}{sy}{%
    \HOLOGO@Span{AmS}{%
      A%
      \HOLOGO@Span{M}{M}%
      S%
    }%
%   }%
}
%    \end{macrocode}
%    \end{macro}
%    \begin{macro}{\HoLogoCss@AmS}
%    \begin{macrocode}
\def\HoLogoCss@AmS{%
  \Css{%
    span.HoLogo-AmS span.HoLogo-M{%
      position:relative;%
      top:.5ex;%
      margin-left:-.1667em;%
      margin-right:-.125em;%
      text-decoration:none;%
    }%
  }%
  \global\let\HoLogoCss@AmS\relax
}
%    \end{macrocode}
%    \end{macro}
%
%    \begin{macro}{\HoLogo@AmSTeX}
%    \begin{macrocode}
\def\HoLogo@AmSTeX#1{%
  \hologo{AmS}%
  \HOLOGO@hyphen
  \hologo{TeX}%
}
%    \end{macrocode}
%    \end{macro}
%    \begin{macro}{\HoLogoBkm@AmSTeX}
%    \begin{macrocode}
\def\HoLogoBkm@AmSTeX#1{AmS-TeX}%
%    \end{macrocode}
%    \end{macro}
%    \begin{macro}{\HoLogoHtml@AmSTeX}
%    \begin{macrocode}
\let\HoLogoHtml@AmSTeX\HoLogo@AmSTeX
%    \end{macrocode}
%    \end{macro}
%
%    \begin{macro}{\HoLogo@AmSLaTeX}
%    \begin{macrocode}
\def\HoLogo@AmSLaTeX#1{%
  \hologo{AmS}%
  \HOLOGO@hyphen
  \hologo{LaTeX}%
}
%    \end{macrocode}
%    \end{macro}
%    \begin{macro}{\HoLogoBkm@AmSLaTeX}
%    \begin{macrocode}
\def\HoLogoBkm@AmSLaTeX#1{AmS-LaTeX}%
%    \end{macrocode}
%    \end{macro}
%    \begin{macro}{\HoLogoHtml@AmSLaTeX}
%    \begin{macrocode}
\let\HoLogoHtml@AmSLaTeX\HoLogo@AmSLaTeX
%    \end{macrocode}
%    \end{macro}
%
% \subsubsection{\hologo{BibTeX}}
%
%    \begin{macro}{\HoLogo@BibTeX@sc}
%    A definition of \hologo{BibTeX} is provided in
%    the documentation source for the manual of \hologo{BibTeX}
%    \cite{btxdoc}.
%\begin{quote}
%\begin{verbatim}
%\def\BibTeX{%
%  {%
%    \rm
%    B%
%    \kern-.05em%
%    {%
%      \sc
%      i%
%      \kern-.025em %
%      b%
%    }%
%    \kern-.08em
%    T%
%    \kern-.1667em%
%    \lower.7ex\hbox{E}%
%    \kern-.125em%
%    X%
%  }%
%}
%\end{verbatim}
%\end{quote}
%    \begin{macrocode}
\def\HoLogo@BibTeX@sc#1{%
  B%
  \kern-.05em%
  \HoLogoFont@font{BibTeX}{sc}{%
    i%
    \kern-.025em%
    b%
  }%
  \HOLOGO@discretionary
  \kern-.08em%
  \hologo{TeX}%
}
%    \end{macrocode}
%    \end{macro}
%    \begin{macro}{\HoLogoHtml@BibTeX@sc}
%    \begin{macrocode}
\def\HoLogoHtml@BibTeX@sc#1{%
  \HoLogoCss@BibTeX@sc
  \HOLOGO@Span{BibTeX-sc}{%
    B%
    \HOLOGO@Span{i}{i}%
    \HOLOGO@Span{b}{b}%
    \hologo{TeX}%
  }%
}
%    \end{macrocode}
%    \end{macro}
%    \begin{macro}{\HoLogoCss@BibTeX@sc}
%    \begin{macrocode}
\def\HoLogoCss@BibTeX@sc{%
  \Css{%
    span.HoLogo-BibTeX-sc span.HoLogo-i{%
      margin-left:-.05em;%
      margin-right:-.025em;%
      font-variant:small-caps;%
    }%
  }%
  \Css{%
    span.HoLogo-BibTeX-sc span.HoLogo-b{%
      margin-right:-.08em;%
      font-variant:small-caps;%
    }%
  }%
  \global\let\HoLogoCss@BibTeX@sc\relax
}
%    \end{macrocode}
%    \end{macro}
%
%    \begin{macro}{\HoLogo@BibTeX@sf}
%    Variant \xoption{sf} avoids trouble with unavailable
%    small caps fonts (e.g., bold versions of Computer Modern or
%    Latin Modern). The definition is taken from
%    package \xpackage{dtklogos} \cite{dtklogos}.
%\begin{quote}
%\begin{verbatim}
%\DeclareRobustCommand{\BibTeX}{%
%  B%
%  \kern-.05em%
%  \hbox{%
%    $\m@th$% %% force math size calculations
%    \csname S@\f@size\endcsname
%    \fontsize\sf@size\z@
%    \math@fontsfalse
%    \selectfont
%    I%
%    \kern-.025em%
%    B
%  }%
%  \kern-.08em%
%  \-%
%  \TeX
%}
%\end{verbatim}
%\end{quote}
%    \begin{macrocode}
\def\HoLogo@BibTeX@sf#1{%
  B%
  \kern-.05em%
  \HoLogoFont@font{BibTeX}{bibsf}{%
    I%
    \kern-.025em%
    B%
  }%
  \HOLOGO@discretionary
  \kern-.08em%
  \hologo{TeX}%
}
%    \end{macrocode}
%    \end{macro}
%    \begin{macro}{\HoLogoHtml@BibTeX@sf}
%    \begin{macrocode}
\def\HoLogoHtml@BibTeX@sf#1{%
  \HoLogoCss@BibTeX@sf
  \HOLOGO@Span{BibTeX-sf}{%
    B%
    \HoLogoFont@font{BibTeX}{bibsf}{%
      \HOLOGO@Span{i}{I}%
      B%
    }%
    \hologo{TeX}%
  }%
}
%    \end{macrocode}
%    \end{macro}
%    \begin{macro}{\HoLogoCss@BibTeX@sf}
%    \begin{macrocode}
\def\HoLogoCss@BibTeX@sf{%
  \Css{%
    span.HoLogo-BibTeX-sf span.HoLogo-i{%
      margin-left:-.05em;%
      margin-right:-.025em;%
    }%
  }%
  \Css{%
    span.HoLogo-BibTeX-sf span.HoLogo-TeX{%
      margin-left:-.08em;%
    }%
  }%
  \global\let\HoLogoCss@BibTeX@sf\relax
}
%    \end{macrocode}
%    \end{macro}
%
%    \begin{macro}{\HoLogo@BibTeX}
%    \begin{macrocode}
\def\HoLogo@BibTeX{\HoLogo@BibTeX@sf}
%    \end{macrocode}
%    \end{macro}
%    \begin{macro}{\HoLogoHtml@BibTeX}
%    \begin{macrocode}
\def\HoLogoHtml@BibTeX{\HoLogoHtml@BibTeX@sf}
%    \end{macrocode}
%    \end{macro}
%
% \subsubsection{\hologo{BibTeX8}}
%
%    \begin{macro}{\HoLogo@BibTeX8}
%    \begin{macrocode}
\expandafter\def\csname HoLogo@BibTeX8\endcsname#1{%
  \hologo{BibTeX}%
  8%
}
%    \end{macrocode}
%    \end{macro}
%
%    \begin{macro}{\HoLogoBkm@BibTeX8}
%    \begin{macrocode}
\expandafter\def\csname HoLogoBkm@BibTeX8\endcsname#1{%
  \hologo{BibTeX}%
  8%
}
%    \end{macrocode}
%    \end{macro}
%    \begin{macro}{\HoLogoHtml@BibTeX8}
%    \begin{macrocode}
\expandafter
\let\csname HoLogoHtml@BibTeX8\expandafter\endcsname
\csname HoLogo@BibTeX8\endcsname
%    \end{macrocode}
%    \end{macro}
%
% \subsubsection{\hologo{ConTeXt}}
%
%    \begin{macro}{\HoLogo@ConTeXt@simple}
%    \begin{macrocode}
\def\HoLogo@ConTeXt@simple#1{%
  \HOLOGO@mbox{Con}%
  \HOLOGO@discretionary
  \HOLOGO@mbox{\hologo{TeX}t}%
}
%    \end{macrocode}
%    \end{macro}
%    \begin{macro}{\HoLogoHtml@ConTeXt@simple}
%    \begin{macrocode}
\let\HoLogoHtml@ConTeXt@simple\HoLogo@ConTeXt@simple
%    \end{macrocode}
%    \end{macro}
%
%    \begin{macro}{\HoLogo@ConTeXt@narrow}
%    This definition of logo \hologo{ConTeXt} with variant \xoption{narrow}
%    comes from TUGboat's class \xclass{ltugboat} (version 2010/11/15 v2.8).
%    \begin{macrocode}
\def\HoLogo@ConTeXt@narrow#1{%
  \HOLOGO@mbox{C\kern-.0333emon}%
  \HOLOGO@discretionary
  \kern-.0667em%
  \HOLOGO@mbox{\hologo{TeX}\kern-.0333emt}%
}
%    \end{macrocode}
%    \end{macro}
%    \begin{macro}{\HoLogoHtml@ConTeXt@narrow}
%    \begin{macrocode}
\def\HoLogoHtml@ConTeXt@narrow#1{%
  \HoLogoCss@ConTeXt@narrow
  \HOLOGO@Span{ConTeXt-narrow}{%
    \HOLOGO@Span{C}{C}%
    on%
    \hologo{TeX}%
    t%
  }%
}
%    \end{macrocode}
%    \end{macro}
%    \begin{macro}{\HoLogoCss@ConTeXt@narrow}
%    \begin{macrocode}
\def\HoLogoCss@ConTeXt@narrow{%
  \Css{%
    span.HoLogo-ConTeXt-narrow span.HoLogo-C{%
      margin-left:-.0333em;%
    }%
  }%
  \Css{%
    span.HoLogo-ConTeXt-narrow span.HoLogo-TeX{%
      margin-left:-.0667em;%
      margin-right:-.0333em;%
    }%
  }%
  \global\let\HoLogoCss@ConTeXt@narrow\relax
}
%    \end{macrocode}
%    \end{macro}
%
%    \begin{macro}{\HoLogo@ConTeXt}
%    \begin{macrocode}
\def\HoLogo@ConTeXt{\HoLogo@ConTeXt@narrow}
%    \end{macrocode}
%    \end{macro}
%    \begin{macro}{\HoLogoHtml@ConTeXt}
%    \begin{macrocode}
\def\HoLogoHtml@ConTeXt{\HoLogoHtml@ConTeXt@narrow}
%    \end{macrocode}
%    \end{macro}
%
% \subsubsection{\hologo{emTeX}}
%
%    \begin{macro}{\HoLogo@emTeX}
%    \begin{macrocode}
\def\HoLogo@emTeX#1{%
  \HOLOGO@mbox{#1{e}{E}m}%
  \HOLOGO@discretionary
  \hologo{TeX}%
}
%    \end{macrocode}
%    \end{macro}
%    \begin{macro}{\HoLogoCs@emTeX}
%    \begin{macrocode}
\def\HoLogoCs@emTeX#1{#1{e}{E}mTeX}%
%    \end{macrocode}
%    \end{macro}
%    \begin{macro}{\HoLogoBkm@emTeX}
%    \begin{macrocode}
\def\HoLogoBkm@emTeX#1{%
  #1{e}{E}m\hologo{TeX}%
}
%    \end{macrocode}
%    \end{macro}
%    \begin{macro}{\HoLogoHtml@emTeX}
%    \begin{macrocode}
\let\HoLogoHtml@emTeX\HoLogo@emTeX
%    \end{macrocode}
%    \end{macro}
%
% \subsubsection{\hologo{ExTeX}}
%
%    \begin{macro}{\HoLogo@ExTeX}
%    The definition is taken from the FAQ of the
%    project \hologo{ExTeX}
%    \cite{ExTeX-FAQ}.
%\begin{quote}
%\begin{verbatim}
%\def\ExTeX{%
%  \textrm{% Logo always with serifs
%    \ensuremath{%
%      \textstyle
%      \varepsilon_{%
%        \kern-0.15em%
%        \mathcal{X}%
%      }%
%    }%
%    \kern-.15em%
%    \TeX
%  }%
%}
%\end{verbatim}
%\end{quote}
%    \begin{macrocode}
\def\HoLogo@ExTeX#1{%
  \HoLogoFont@font{ExTeX}{rm}{%
    \ltx@mbox{%
      \HOLOGO@MathSetup
      $%
        \textstyle
        \varepsilon_{%
          \kern-0.15em%
          \HoLogoFont@font{ExTeX}{sy}{X}%
        }%
      $%
    }%
    \HOLOGO@discretionary
    \kern-.15em%
    \hologo{TeX}%
  }%
}
%    \end{macrocode}
%    \end{macro}
%    \begin{macro}{\HoLogoHtml@ExTeX}
%    \begin{macrocode}
\def\HoLogoHtml@ExTeX#1{%
  \HoLogoCss@ExTeX
  \HoLogoFont@font{ExTeX}{rm}{%
    \HOLOGO@Span{ExTeX}{%
      \ltx@mbox{%
        \HOLOGO@MathSetup
        $\textstyle\varepsilon$%
        \HOLOGO@Span{X}{$\textstyle\chi$}%
        \hologo{TeX}%
      }%
    }%
  }%
}
%    \end{macrocode}
%    \end{macro}
%    \begin{macro}{\HoLogoBkm@ExTeX}
%    \begin{macrocode}
\def\HoLogoBkm@ExTeX#1{%
  \HOLOGO@PdfdocUnicode{#1{e}{E}x}{\textepsilon\textchi}%
  \hologo{TeX}%
}
%    \end{macrocode}
%    \end{macro}
%    \begin{macro}{\HoLogoCss@ExTeX}
%    \begin{macrocode}
\def\HoLogoCss@ExTeX{%
  \Css{%
    span.HoLogo-ExTeX{%
      font-family:serif;%
    }%
  }%
  \Css{%
    span.HoLogo-ExTeX span.HoLogo-TeX{%
      margin-left:-.15em;%
    }%
  }%
  \global\let\HoLogoCss@ExTeX\relax
}
%    \end{macrocode}
%    \end{macro}
%
% \subsubsection{\hologo{MiKTeX}}
%
%    \begin{macro}{\HoLogo@MiKTeX}
%    \begin{macrocode}
\def\HoLogo@MiKTeX#1{%
  \HOLOGO@mbox{MiK}%
  \HOLOGO@discretionary
  \hologo{TeX}%
}
%    \end{macrocode}
%    \end{macro}
%    \begin{macro}{\HoLogoHtml@MiKTeX}
%    \begin{macrocode}
\let\HoLogoHtml@MiKTeX\HoLogo@MiKTeX
%    \end{macrocode}
%    \end{macro}
%
% \subsubsection{\hologo{OzTeX} and friends}
%
%    Source: \hologo{OzTeX} FAQ \cite{OzTeX}:
%    \begin{quote}
%      |\def\OzTeX{O\kern-.03em z\kern-.15em\TeX}|\\
%      (There is no kerning in OzMF, OzMP and OzTtH.)
%    \end{quote}
%
%    \begin{macro}{\HoLogo@OzTeX}
%    \begin{macrocode}
\def\HoLogo@OzTeX#1{%
  O%
  \kern-.03em %
  z%
  \kern-.15em %
  \hologo{TeX}%
}
%    \end{macrocode}
%    \end{macro}
%    \begin{macro}{\HoLogoHtml@OzTeX}
%    \begin{macrocode}
\def\HoLogoHtml@OzTeX#1{%
  \HoLogoCss@OzTeX
  \HOLOGO@Span{OzTeX}{%
    O%
    \HOLOGO@Span{z}{z}%
    \hologo{TeX}%
  }%
}
%    \end{macrocode}
%    \end{macro}
%    \begin{macro}{\HoLogoCss@OzTeX}
%    \begin{macrocode}
\def\HoLogoCss@OzTeX{%
  \Css{%
    span.HoLogo-OzTeX span.HoLogo-z{%
      margin-left:-.03em;%
      margin-right:-.15em;%
    }%
  }%
  \global\let\HoLogoCss@OzTeX\relax
}
%    \end{macrocode}
%    \end{macro}
%
%    \begin{macro}{\HoLogo@OzMF}
%    \begin{macrocode}
\def\HoLogo@OzMF#1{%
  \HOLOGO@mbox{OzMF}%
}
%    \end{macrocode}
%    \end{macro}
%    \begin{macro}{\HoLogo@OzMP}
%    \begin{macrocode}
\def\HoLogo@OzMP#1{%
  \HOLOGO@mbox{OzMP}%
}
%    \end{macrocode}
%    \end{macro}
%    \begin{macro}{\HoLogo@OzTtH}
%    \begin{macrocode}
\def\HoLogo@OzTtH#1{%
  \HOLOGO@mbox{OzTtH}%
}
%    \end{macrocode}
%    \end{macro}
%
% \subsubsection{\hologo{PCTeX}}
%
%    \begin{macro}{\HoLogo@PCTeX}
%    \begin{macrocode}
\def\HoLogo@PCTeX#1{%
  \HOLOGO@mbox{PC}%
  \hologo{TeX}%
}
%    \end{macrocode}
%    \end{macro}
%    \begin{macro}{\HoLogoHtml@PCTeX}
%    \begin{macrocode}
\let\HoLogoHtml@PCTeX\HoLogo@PCTeX
%    \end{macrocode}
%    \end{macro}
%
% \subsubsection{\hologo{PiCTeX}}
%
%    The original definitions from \xfile{pictex.tex} \cite{PiCTeX}:
%\begin{quote}
%\begin{verbatim}
%\def\PiC{%
%  P%
%  \kern-.12em%
%  \lower.5ex\hbox{I}%
%  \kern-.075em%
%  C%
%}
%\def\PiCTeX{%
%  \PiC
%  \kern-.11em%
%  \TeX
%}
%\end{verbatim}
%\end{quote}
%
%    \begin{macro}{\HoLogo@PiC}
%    \begin{macrocode}
\def\HoLogo@PiC#1{%
  P%
  \kern-.12em%
  \lower.5ex\hbox{I}%
  \kern-.075em%
  C%
  \HOLOGO@SpaceFactor
}
%    \end{macrocode}
%    \end{macro}
%    \begin{macro}{\HoLogoHtml@PiC}
%    \begin{macrocode}
\def\HoLogoHtml@PiC#1{%
  \HoLogoCss@PiC
  \HOLOGO@Span{PiC}{%
    P%
    \HOLOGO@Span{i}{I}%
    C%
  }%
}
%    \end{macrocode}
%    \end{macro}
%    \begin{macro}{\HoLogoCss@PiC}
%    \begin{macrocode}
\def\HoLogoCss@PiC{%
  \Css{%
    span.HoLogo-PiC span.HoLogo-i{%
      position:relative;%
      top:.5ex;%
      margin-left:-.12em;%
      margin-right:-.075em;%
      text-decoration:none;%
    }%
  }%
  \global\let\HoLogoCss@PiC\relax
}
%    \end{macrocode}
%    \end{macro}
%
%    \begin{macro}{\HoLogo@PiCTeX}
%    \begin{macrocode}
\def\HoLogo@PiCTeX#1{%
  \hologo{PiC}%
  \HOLOGO@discretionary
  \kern-.11em%
  \hologo{TeX}%
}
%    \end{macrocode}
%    \end{macro}
%    \begin{macro}{\HoLogoHtml@PiCTeX}
%    \begin{macrocode}
\def\HoLogoHtml@PiCTeX#1{%
  \HoLogoCss@PiCTeX
  \HOLOGO@Span{PiCTeX}{%
    \hologo{PiC}%
    \hologo{TeX}%
  }%
}
%    \end{macrocode}
%    \end{macro}
%    \begin{macro}{\HoLogoCss@PiCTeX}
%    \begin{macrocode}
\def\HoLogoCss@PiCTeX{%
  \Css{%
    span.HoLogo-PiCTeX span.HoLogo-PiC{%
      margin-right:-.11em;%
    }%
  }%
  \global\let\HoLogoCss@PiCTeX\relax
}
%    \end{macrocode}
%    \end{macro}
%
% \subsubsection{\hologo{teTeX}}
%
%    \begin{macro}{\HoLogo@teTeX}
%    \begin{macrocode}
\def\HoLogo@teTeX#1{%
  \HOLOGO@mbox{#1{t}{T}e}%
  \HOLOGO@discretionary
  \hologo{TeX}%
}
%    \end{macrocode}
%    \end{macro}
%    \begin{macro}{\HoLogoCs@teTeX}
%    \begin{macrocode}
\def\HoLogoCs@teTeX#1{#1{t}{T}dfTeX}
%    \end{macrocode}
%    \end{macro}
%    \begin{macro}{\HoLogoBkm@teTeX}
%    \begin{macrocode}
\def\HoLogoBkm@teTeX#1{%
  #1{t}{T}e\hologo{TeX}%
}
%    \end{macrocode}
%    \end{macro}
%    \begin{macro}{\HoLogoHtml@teTeX}
%    \begin{macrocode}
\let\HoLogoHtml@teTeX\HoLogo@teTeX
%    \end{macrocode}
%    \end{macro}
%
% \subsubsection{\hologo{TeX4ht}}
%
%    \begin{macro}{\HoLogo@TeX4ht}
%    \begin{macrocode}
\expandafter\def\csname HoLogo@TeX4ht\endcsname#1{%
  \HOLOGO@mbox{\hologo{TeX}4ht}%
}
%    \end{macrocode}
%    \end{macro}
%    \begin{macro}{\HoLogoHtml@TeX4ht}
%    \begin{macrocode}
\expandafter
\let\csname HoLogoHtml@TeX4ht\expandafter\endcsname
\csname HoLogo@TeX4ht\endcsname
%    \end{macrocode}
%    \end{macro}
%
%
% \subsubsection{\hologo{SageTeX}}
%
%    \begin{macro}{\HoLogo@SageTeX}
%    \begin{macrocode}
\def\HoLogo@SageTeX#1{%
  \HOLOGO@mbox{Sage}%
  \HOLOGO@discretionary
  \HOLOGO@NegativeKerning{eT,oT,To}%
  \hologo{TeX}%
}
%    \end{macrocode}
%    \end{macro}
%    \begin{macro}{\HoLogoHtml@SageTeX}
%    \begin{macrocode}
\let\HoLogoHtml@SageTeX\HoLogo@SageTeX
%    \end{macrocode}
%    \end{macro}
%
% \subsection{\hologo{METAFONT} and friends}
%
%    \begin{macro}{\HoLogo@METAFONT}
%    \begin{macrocode}
\def\HoLogo@METAFONT#1{%
  \HoLogoFont@font{METAFONT}{logo}{%
    \HOLOGO@mbox{META}%
    \HOLOGO@discretionary
    \HOLOGO@mbox{FONT}%
  }%
}
%    \end{macrocode}
%    \end{macro}
%
%    \begin{macro}{\HoLogo@METAPOST}
%    \begin{macrocode}
\def\HoLogo@METAPOST#1{%
  \HoLogoFont@font{METAPOST}{logo}{%
    \HOLOGO@mbox{META}%
    \HOLOGO@discretionary
    \HOLOGO@mbox{POST}%
  }%
}
%    \end{macrocode}
%    \end{macro}
%
%    \begin{macro}{\HoLogo@MetaFun}
%    \begin{macrocode}
\def\HoLogo@MetaFun#1{%
  \HOLOGO@mbox{Meta}%
  \HOLOGO@discretionary
  \HOLOGO@mbox{Fun}%
}
%    \end{macrocode}
%    \end{macro}
%
%    \begin{macro}{\HoLogo@MetaPost}
%    \begin{macrocode}
\def\HoLogo@MetaPost#1{%
  \HOLOGO@mbox{Meta}%
  \HOLOGO@discretionary
  \HOLOGO@mbox{Post}%
}
%    \end{macrocode}
%    \end{macro}
%
% \subsection{Others}
%
% \subsubsection{\hologo{biber}}
%
%    \begin{macro}{\HoLogo@biber}
%    \begin{macrocode}
\def\HoLogo@biber#1{%
  \HOLOGO@mbox{#1{b}{B}i}%
  \HOLOGO@discretionary
  \HOLOGO@mbox{ber}%
}
%    \end{macrocode}
%    \end{macro}
%    \begin{macro}{\HoLogoCs@biber}
%    \begin{macrocode}
\def\HoLogoCs@biber#1{#1{b}{B}iber}
%    \end{macrocode}
%    \end{macro}
%    \begin{macro}{\HoLogoBkm@biber}
%    \begin{macrocode}
\def\HoLogoBkm@biber#1{%
  #1{b}{B}iber%
}
%    \end{macrocode}
%    \end{macro}
%    \begin{macro}{\HoLogoHtml@biber}
%    \begin{macrocode}
\let\HoLogoHtml@biber\HoLogo@biber
%    \end{macrocode}
%    \end{macro}
%
% \subsubsection{\hologo{KOMAScript}}
%
%    \begin{macro}{\HoLogo@KOMAScript}
%    The definition for \hologo{KOMAScript} is taken
%    from \hologo{KOMAScript} (\xfile{scrlogo.dtx}, reformatted) \cite{scrlogo}:
%\begin{quote}
%\begin{verbatim}
%\@ifundefined{KOMAScript}{%
%  \DeclareRobustCommand{\KOMAScript}{%
%    \textsf{%
%      K\kern.05em O\kern.05emM\kern.05em A%
%      \kern.1em-\kern.1em %
%      Script%
%    }%
%  }%
%}{}
%\end{verbatim}
%\end{quote}
%    \begin{macrocode}
\def\HoLogo@KOMAScript#1{%
  \HoLogoFont@font{KOMAScript}{sf}{%
    \HOLOGO@mbox{%
      K\kern.05em%
      O\kern.05em%
      M\kern.05em%
      A%
    }%
    \kern.1em%
    \HOLOGO@hyphen
    \kern.1em%
    \HOLOGO@mbox{Script}%
  }%
}
%    \end{macrocode}
%    \end{macro}
%    \begin{macro}{\HoLogoBkm@KOMAScript}
%    \begin{macrocode}
\def\HoLogoBkm@KOMAScript#1{%
  KOMA-Script%
}
%    \end{macrocode}
%    \end{macro}
%    \begin{macro}{\HoLogoHtml@KOMAScript}
%    \begin{macrocode}
\def\HoLogoHtml@KOMAScript#1{%
  \HoLogoCss@KOMAScript
  \HoLogoFont@font{KOMAScript}{sf}{%
    \HOLOGO@Span{KOMAScript}{%
      K%
      \HOLOGO@Span{O}{O}%
      M%
      \HOLOGO@Span{A}{A}%
      \HOLOGO@Span{hyphen}{-}%
      Script%
    }%
  }%
}
%    \end{macrocode}
%    \end{macro}
%    \begin{macro}{\HoLogoCss@KOMAScript}
%    \begin{macrocode}
\def\HoLogoCss@KOMAScript{%
  \Css{%
    span.HoLogo-KOMAScript{%
      font-family:sans-serif;%
    }%
  }%
  \Css{%
    span.HoLogo-KOMAScript span.HoLogo-O{%
      padding-left:.05em;%
      padding-right:.05em;%
    }%
  }%
  \Css{%
    span.HoLogo-KOMAScript span.HoLogo-A{%
      padding-left:.05em;%
    }%
  }%
  \Css{%
    span.HoLogo-KOMAScript span.HoLogo-hyphen{%
      padding-left:.1em;%
      padding-right:.1em;%
    }%
  }%
  \global\let\HoLogoCss@KOMAScript\relax
}
%    \end{macrocode}
%    \end{macro}
%
% \subsubsection{\hologo{LyX}}
%
%    \begin{macro}{\HoLogo@LyX}
%    The definition is taken from the documentation source files
%    of \hologo{LyX}, \xfile{Intro.lyx} \cite{LyX}:
%\begin{quote}
%\begin{verbatim}
%\def\LyX{%
%  \texorpdfstring{%
%    L\kern-.1667em\lower.25em\hbox{Y}\kern-.125emX\@%
%  }{%
%    LyX%
%  }%
%}
%\end{verbatim}
%\end{quote}
%    \begin{macrocode}
\def\HoLogo@LyX#1{%
  L%
  \kern-.1667em%
  \lower.25em\hbox{Y}%
  \kern-.125em%
  X%
  \HOLOGO@SpaceFactor
}
%    \end{macrocode}
%    \end{macro}
%    \begin{macro}{\HoLogoHtml@LyX}
%    \begin{macrocode}
\def\HoLogoHtml@LyX#1{%
  \HoLogoCss@LyX
  \HOLOGO@Span{LyX}{%
    L%
    \HOLOGO@Span{y}{Y}%
    X%
  }%
}
%    \end{macrocode}
%    \end{macro}
%    \begin{macro}{\HoLogoCss@LyX}
%    \begin{macrocode}
\def\HoLogoCss@LyX{%
  \Css{%
    span.HoLogo-LyX span.HoLogo-y{%
      position:relative;%
      top:.25em;%
      margin-left:-.1667em;%
      margin-right:-.125em;%
      text-decoration:none;%
    }%
  }%
  \global\let\HoLogoCss@LyX\relax
}
%    \end{macrocode}
%    \end{macro}
%
% \subsubsection{\hologo{NTS}}
%
%    \begin{macro}{\HoLogo@NTS}
%    Definition for \hologo{NTS} can be found in
%    package \xpackage{etex\textunderscore man} for the \hologo{eTeX} manual \cite{etexman}
%    and in package \xpackage{dtklogos} \cite{dtklogos}:
%\begin{quote}
%\begin{verbatim}
%\def\NTS{%
%  \leavevmode
%  \hbox{%
%    $%
%      \cal N%
%      \kern-0.35em%
%      \lower0.5ex\hbox{$\cal T$}%
%      \kern-0.2em%
%      S%
%    $%
%  }%
%}
%\end{verbatim}
%\end{quote}
%    \begin{macrocode}
\def\HoLogo@NTS#1{%
  \HoLogoFont@font{NTS}{sy}{%
    N\/%
    \kern-.35em%
    \lower.5ex\hbox{T\/}%
    \kern-.2em%
    S\/%
  }%
  \HOLOGO@SpaceFactor
}
%    \end{macrocode}
%    \end{macro}
%
% \subsubsection{\Hologo{TTH} (\hologo{TeX} to HTML translator)}
%
%    Source: \url{http://hutchinson.belmont.ma.us/tth/}
%    In the HTML source the second `T' is printed as subscript.
%\begin{quote}
%\begin{verbatim}
%T<sub>T</sub>H
%\end{verbatim}
%\end{quote}
%    \begin{macro}{\HoLogo@TTH}
%    \begin{macrocode}
\def\HoLogo@TTH#1{%
  \ltx@mbox{%
    T\HOLOGO@SubScript{T}H%
  }%
  \HOLOGO@SpaceFactor
}
%    \end{macrocode}
%    \end{macro}
%
%    \begin{macro}{\HoLogoHtml@TTH}
%    \begin{macrocode}
\def\HoLogoHtml@TTH#1{%
  T\HCode{<sub>}T\HCode{</sub>}H%
}
%    \end{macrocode}
%    \end{macro}
%
% \subsubsection{\Hologo{HanTheThanh}}
%
%    Partial source: Package \xpackage{dtklogos}.
%    The double accent is U+1EBF (latin small letter e with circumflex
%    and acute).
%    \begin{macro}{\HoLogo@HanTheThanh}
%    \begin{macrocode}
\def\HoLogo@HanTheThanh#1{%
  \ltx@mbox{H\`an}%
  \HOLOGO@space
  \ltx@mbox{%
    Th%
    \HOLOGO@IfCharExists{"1EBF}{%
      \char"1EBF\relax
    }{%
      \^e\hbox to 0pt{\hss\raise .5ex\hbox{\'{}}}%
    }%
  }%
  \HOLOGO@space
  \ltx@mbox{Th\`anh}%
}
%    \end{macrocode}
%    \end{macro}
%    \begin{macro}{\HoLogoBkm@HanTheThanh}
%    \begin{macrocode}
\def\HoLogoBkm@HanTheThanh#1{%
  H\`an %
  Th\HOLOGO@PdfdocUnicode{\^e}{\9036\277} %
  Th\`anh%
}
%    \end{macrocode}
%    \end{macro}
%    \begin{macro}{\HoLogoHtml@HanTheThanh}
%    \begin{macrocode}
\def\HoLogoHtml@HanTheThanh#1{%
  H\`an %
  Th\HCode{&\ltx@hashchar x1ebf;} %
  Th\`anh%
}
%    \end{macrocode}
%    \end{macro}
%
% \subsection{Driver detection}
%
%    \begin{macrocode}
\HOLOGO@IfExists\InputIfFileExists{%
  \InputIfFileExists{hologo.cfg}{}{}%
}{%
  \ltx@IfUndefined{pdf@filesize}{%
    \def\HOLOGO@InputIfExists{%
      \openin\HOLOGO@temp=hologo.cfg\relax
      \ifeof\HOLOGO@temp
        \closein\HOLOGO@temp
      \else
        \closein\HOLOGO@temp
        \begingroup
          \def\x{LaTeX2e}%
        \expandafter\endgroup
        \ifx\fmtname\x
          \input{hologo.cfg}%
        \else
          \input hologo.cfg\relax
        \fi
      \fi
    }%
    \ltx@IfUndefined{newread}{%
      \chardef\HOLOGO@temp=15 %
      \def\HOLOGO@CheckRead{%
        \ifeof\HOLOGO@temp
          \HOLOGO@InputIfExists
        \else
          \ifcase\HOLOGO@temp
            \@PackageWarningNoLine{hologo}{%
              Configuration file ignored, because\MessageBreak
              a free read register could not be found%
            }%
          \else
            \begingroup
              \count\ltx@cclv=\HOLOGO@temp
              \advance\ltx@cclv by \ltx@minusone
              \edef\x{\endgroup
                \chardef\noexpand\HOLOGO@temp=\the\count\ltx@cclv
                \relax
              }%
            \x
          \fi
        \fi
      }%
    }{%
      \csname newread\endcsname\HOLOGO@temp
      \HOLOGO@InputIfExists
    }%
  }{%
    \edef\HOLOGO@temp{\pdf@filesize{hologo.cfg}}%
    \ifx\HOLOGO@temp\ltx@empty
    \else
      \ifnum\HOLOGO@temp>0 %
        \begingroup
          \def\x{LaTeX2e}%
        \expandafter\endgroup
        \ifx\fmtname\x
          \input{hologo.cfg}%
        \else
          \input hologo.cfg\relax
        \fi
      \else
        \@PackageInfoNoLine{hologo}{%
          Empty configuration file `hologo.cfg' ignored%
        }%
      \fi
    \fi
  }%
}
%    \end{macrocode}
%
%    \begin{macrocode}
\def\HOLOGO@temp#1#2{%
  \kv@define@key{HoLogoDriver}{#1}[]{%
    \begingroup
      \def\HOLOGO@temp{##1}%
      \ltx@onelevel@sanitize\HOLOGO@temp
      \ifx\HOLOGO@temp\ltx@empty
      \else
        \@PackageError{hologo}{%
          Value (\HOLOGO@temp) not permitted for option `#1'%
        }%
        \@ehc
      \fi
    \endgroup
    \def\hologoDriver{#2}%
  }%
}%
\def\HOLOGO@@temp#1#2{%
  \ifx\kv@value\relax
    \HOLOGO@temp{#1}{#1}%
  \else
    \HOLOGO@temp{#1}{#2}%
  \fi
}%
\kv@parse@normalized{%
  pdftex,%
  luatex=pdftex,%
  dvipdfm,%
  dvipdfmx=dvipdfm,%
  dvips,%
  dvipsone=dvips,%
  xdvi=dvips,%
  xetex,%
  vtex,%
}\HOLOGO@@temp
%    \end{macrocode}
%
%    \begin{macrocode}
\kv@define@key{HoLogoDriver}{driverfallback}{%
  \def\HOLOGO@DriverFallback{#1}%
}
%    \end{macrocode}
%
%    \begin{macro}{\HOLOGO@DriverFallback}
%    \begin{macrocode}
\def\HOLOGO@DriverFallback{dvips}
%    \end{macrocode}
%    \end{macro}
%
%    \begin{macro}{\hologoDriverSetup}
%    \begin{macrocode}
\def\hologoDriverSetup{%
  \let\hologoDriver\ltx@undefined
  \HOLOGO@DriverSetup
}
%    \end{macrocode}
%    \end{macro}
%
%    \begin{macro}{\HOLOGO@DriverSetup}
%    \begin{macrocode}
\def\HOLOGO@DriverSetup#1{%
  \kvsetkeys{HoLogoDriver}{#1}%
  \HOLOGO@CheckDriver
  \ltx@ifundefined{hologoDriver}{%
    \begingroup
    \edef\x{\endgroup
      \noexpand\kvsetkeys{HoLogoDriver}{\HOLOGO@DriverFallback}%
    }\x
  }{}%
  \@PackageInfoNoLine{hologo}{Using driver `\hologoDriver'}%
}
%    \end{macrocode}
%    \end{macro}
%
%    \begin{macro}{\HOLOGO@CheckDriver}
%    \begin{macrocode}
\def\HOLOGO@CheckDriver{%
  \ifpdf
    \def\hologoDriver{pdftex}%
    \let\HOLOGO@pdfliteral\pdfliteral
    \ifluatex
      \ifx\pdfextension\@undefined\else
        \protected\def\pdfliteral{\pdfextension literal}%
        \let\HOLOGO@pdfliteral\pdfliteral
      \fi
      \ltx@IfUndefined{HOLOGO@pdfliteral}{%
        \ifnum\luatexversion<36 %
        \else
          \begingroup
            \let\HOLOGO@temp\endgroup
            \ifcase0%
                \directlua{%
                  if tex.enableprimitives then %
                    tex.enableprimitives('HOLOGO@', {'pdfliteral'})%
                  else %
                    tex.print('1')%
                  end%
                }%
                \ifx\HOLOGO@pdfliteral\@undefined 1\fi%
                \relax%
              \endgroup
              \let\HOLOGO@temp\relax
              \global\let\HOLOGO@pdfliteral\HOLOGO@pdfliteral
            \fi%
          \HOLOGO@temp
        \fi
      }{}%
    \fi
    \ltx@IfUndefined{HOLOGO@pdfliteral}{%
      \@PackageWarningNoLine{hologo}{%
        Cannot find \string\pdfliteral
      }%
    }{}%
  \else
    \ifxetex
      \def\hologoDriver{xetex}%
    \else
      \ifvtex
        \def\hologoDriver{vtex}%
      \fi
    \fi
  \fi
}
%    \end{macrocode}
%    \end{macro}
%
%    \begin{macro}{\HOLOGO@WarningUnsupportedDriver}
%    \begin{macrocode}
\def\HOLOGO@WarningUnsupportedDriver#1{%
  \@PackageWarningNoLine{hologo}{%
    Logo `#1' needs driver specific macros,\MessageBreak
    but driver `\hologoDriver' is not supported.\MessageBreak
    Use a different driver or\MessageBreak
    load package `graphics' or `pgf'%
  }%
}
%    \end{macrocode}
%    \end{macro}
%
% \subsubsection{Reflect box macros}
%
%    Skip driver part if not needed.
%    \begin{macrocode}
\ltx@IfUndefined{reflectbox}{}{%
  \ltx@IfUndefined{rotatebox}{}{%
    \HOLOGO@AtEnd
  }%
}
\ltx@IfUndefined{pgftext}{}{%
  \HOLOGO@AtEnd
}
\ltx@IfUndefined{psscalebox}{}{%
  \HOLOGO@AtEnd
}
%    \end{macrocode}
%
%    \begin{macrocode}
\def\HOLOGO@temp{LaTeX2e}
\ifx\fmtname\HOLOGO@temp
  \RequirePackage{kvoptions}[2011/06/30]%
  \ProcessKeyvalOptions{HoLogoDriver}%
\fi
\HOLOGO@DriverSetup{}
%    \end{macrocode}
%
%    \begin{macro}{\HOLOGO@ReflectBox}
%    \begin{macrocode}
\def\HOLOGO@ReflectBox#1{%
  \begingroup
    \setbox\ltx@zero\hbox{\begingroup#1\endgroup}%
    \setbox\ltx@two\hbox{%
      \kern\wd\ltx@zero
      \csname HOLOGO@ScaleBox@\hologoDriver\endcsname{-1}{1}{%
        \hbox to 0pt{\copy\ltx@zero\hss}%
      }%
    }%
    \wd\ltx@two=\wd\ltx@zero
    \box\ltx@two
  \endgroup
}
%    \end{macrocode}
%    \end{macro}
%
%    \begin{macro}{\HOLOGO@PointReflectBox}
%    \begin{macrocode}
\def\HOLOGO@PointReflectBox#1{%
  \begingroup
    \setbox\ltx@zero\hbox{\begingroup#1\endgroup}%
    \setbox\ltx@two\hbox{%
      \kern\wd\ltx@zero
      \raise\ht\ltx@zero\hbox{%
        \csname HOLOGO@ScaleBox@\hologoDriver\endcsname{-1}{-1}{%
          \hbox to 0pt{\copy\ltx@zero\hss}%
        }%
      }%
    }%
    \wd\ltx@two=\wd\ltx@zero
    \box\ltx@two
  \endgroup
}
%    \end{macrocode}
%    \end{macro}
%
%    We must define all variants because of dynamic driver setup.
%    \begin{macrocode}
\def\HOLOGO@temp#1#2{#2}
%    \end{macrocode}
%
%    \begin{macro}{\HOLOGO@ScaleBox@pdftex}
%    \begin{macrocode}
\HOLOGO@temp{pdftex}{%
  \def\HOLOGO@ScaleBox@pdftex#1#2#3{%
    \HOLOGO@pdfliteral{%
      q #1 0 0 #2 0 0 cm%
    }%
    #3%
    \HOLOGO@pdfliteral{%
      Q%
    }%
  }%
}
%    \end{macrocode}
%    \end{macro}
%    \begin{macro}{\HOLOGO@ScaleBox@dvips}
%    \begin{macrocode}
\HOLOGO@temp{dvips}{%
  \def\HOLOGO@ScaleBox@dvips#1#2#3{%
    \special{ps:%
      gsave %
      currentpoint %
      currentpoint translate %
      #1 #2 scale %
      neg exch neg exch translate%
    }%
    #3%
    \special{ps:%
      currentpoint %
      grestore %
      moveto%
    }%
  }%
}
%    \end{macrocode}
%    \end{macro}
%    \begin{macro}{\HOLOGO@ScaleBox@dvipdfm}
%    \begin{macrocode}
\HOLOGO@temp{dvipdfm}{%
  \let\HOLOGO@ScaleBox@dvipdfm\HOLOGO@ScaleBox@dvips
}
%    \end{macrocode}
%    \end{macro}
%    Since \hologo{XeTeX} v0.6.
%    \begin{macro}{\HOLOGO@ScaleBox@xetex}
%    \begin{macrocode}
\HOLOGO@temp{xetex}{%
  \def\HOLOGO@ScaleBox@xetex#1#2#3{%
    \special{x:gsave}%
    \special{x:scale #1 #2}%
    #3%
    \special{x:grestore}%
  }%
}
%    \end{macrocode}
%    \end{macro}
%    \begin{macro}{\HOLOGO@ScaleBox@vtex}
%    \begin{macrocode}
\HOLOGO@temp{vtex}{%
  \def\HOLOGO@ScaleBox@vtex#1#2#3{%
    \special{r(#1,0,0,#2,0,0}%
    #3%
    \special{r)}%
  }%
}
%    \end{macrocode}
%    \end{macro}
%
%    \begin{macrocode}
\HOLOGO@AtEnd%
%</package>
%    \end{macrocode}
%
% \section{Test}
%
% \subsection{Catcode checks for loading}
%
%    \begin{macrocode}
%<*test1>
%    \end{macrocode}
%    \begin{macrocode}
\catcode`\{=1 %
\catcode`\}=2 %
\catcode`\#=6 %
\catcode`\@=11 %
\expandafter\ifx\csname count@\endcsname\relax
  \countdef\count@=255 %
\fi
\expandafter\ifx\csname @gobble\endcsname\relax
  \long\def\@gobble#1{}%
\fi
\expandafter\ifx\csname @firstofone\endcsname\relax
  \long\def\@firstofone#1{#1}%
\fi
\expandafter\ifx\csname loop\endcsname\relax
  \expandafter\@firstofone
\else
  \expandafter\@gobble
\fi
{%
  \def\loop#1\repeat{%
    \def\body{#1}%
    \iterate
  }%
  \def\iterate{%
    \body
      \let\next\iterate
    \else
      \let\next\relax
    \fi
    \next
  }%
  \let\repeat=\fi
}%
\def\RestoreCatcodes{}
\count@=0 %
\loop
  \edef\RestoreCatcodes{%
    \RestoreCatcodes
    \catcode\the\count@=\the\catcode\count@\relax
  }%
\ifnum\count@<255 %
  \advance\count@ 1 %
\repeat

\def\RangeCatcodeInvalid#1#2{%
  \count@=#1\relax
  \loop
    \catcode\count@=15 %
  \ifnum\count@<#2\relax
    \advance\count@ 1 %
  \repeat
}
\def\RangeCatcodeCheck#1#2#3{%
  \count@=#1\relax
  \loop
    \ifnum#3=\catcode\count@
    \else
      \errmessage{%
        Character \the\count@\space
        with wrong catcode \the\catcode\count@\space
        instead of \number#3%
      }%
    \fi
  \ifnum\count@<#2\relax
    \advance\count@ 1 %
  \repeat
}
\def\space{ }
\expandafter\ifx\csname LoadCommand\endcsname\relax
  \def\LoadCommand{\input hologo.sty\relax}%
\fi
\def\Test{%
  \RangeCatcodeInvalid{0}{47}%
  \RangeCatcodeInvalid{58}{64}%
  \RangeCatcodeInvalid{91}{96}%
  \RangeCatcodeInvalid{123}{255}%
  \catcode`\@=12 %
  \catcode`\\=0 %
  \catcode`\%=14 %
  \LoadCommand
  \RangeCatcodeCheck{0}{36}{15}%
  \RangeCatcodeCheck{37}{37}{14}%
  \RangeCatcodeCheck{38}{47}{15}%
  \RangeCatcodeCheck{48}{57}{12}%
  \RangeCatcodeCheck{58}{63}{15}%
  \RangeCatcodeCheck{64}{64}{12}%
  \RangeCatcodeCheck{65}{90}{11}%
  \RangeCatcodeCheck{91}{91}{15}%
  \RangeCatcodeCheck{92}{92}{0}%
  \RangeCatcodeCheck{93}{96}{15}%
  \RangeCatcodeCheck{97}{122}{11}%
  \RangeCatcodeCheck{123}{255}{15}%
  \RestoreCatcodes
}
\Test
\csname @@end\endcsname
\end
%    \end{macrocode}
%    \begin{macrocode}
%</test1>
%    \end{macrocode}
%
% \subsection{Spacefactor}
%
%    The space factor must be 1000 after a logo. If it is greater 1000
%    then the following space is a space after a sentence closing point.
%    If the space factor is smaller 1000 then an immediate following
%    dot is interpreted as abbreviation, not sentence closing point.
%
%    \begin{macrocode}
%<*test-spacefactor>
\NeedsTeXFormat{LaTeX2e}
\documentclass{article}
\usepackage{hologo}[2016/05/12]
\usepackage{kvsetkeys}
\usepackage{qstest}
\IncludeTests{*}
\LogTests{log}{*}{*}
\begin{document}
\begin{qstest}{spacefactor}{spacefactor}
\newcommand*{\Test}[1]{%
  \sbox0{%
    \hologo{#1}%
    \Expect*{1000 (#1)}*{\the\spacefactor\space(#1)}%
  }%
}%
\makeatletter
\def\TestList{}
\def\hologoEntry#1#2#3{%
  \edef\TestList{%
    \ifx\TestList\@empty
    \else
      \TestList,%
    \fi
    #1%
    \ifx\\#2\\%
    \else
      ={variant=#2}%
    \fi
  }%
}
\hologoList
\expandafter\kv@parse@normalized\expandafter{%
  \TestList
}{%
  \begingroup
    \let\@logo=\kv@key
    \ifx\kv@value\relax
    \else
      \expandafter\hologoLogoSetup\expandafter\@logo\expandafter{%
        \kv@value
      }%
    \fi
    \Test\@logo
  \endgroup
  \@gobbletwo
}
\end{qstest}
\end{document}
%</test-spacefactor>
%    \end{macrocode}
%
% \subsection{Complete list}
%
%    \begin{macrocode}
%<*test-list>
\NeedsTeXFormat{LaTeX2e}
\documentclass[12pt,a4paper]{article}
\usepackage{hologo}[2016/05/12]
\usepackage[T1]{fontenc}
\usepackage{lmodern}
\usepackage{parskip}
\usepackage[unicode]{hyperref}[2011/09/28]
\usepackage{bookmark}[2011/09/19]
\bookmarksetup{%
  numbered,%
  open,%
  openlevel=2,%
}
\renewcommand*{\contentsname}{List of logos}
\begin{document}
\tableofcontents
\def\TestFont#1#2#3#4#5#6{%
  \begingroup
    \usefont{#3}{#4}{#5}{#6}%
    \HologoVariant{#1}{#2}/\hologoVariant{#1}{#2}%
    \quad
    \begingroup\scriptsize\hologoVariant{#1}{#2}\endgroup
    \quad
  \endgroup
  (#3/#4/#5/#6)%
  \par
}
\makeatletter
\def\hologoEntry#1#2#3{%
  \section{%
    \HologoVariant{#1}{#2}/\hologoVariant{#1}{#2} %
    {[#1\ifx\\#2\\\else\space(#2)\fi]}% hash-ok
  }% braces around [] because of bug in tex4ht
  \begingroup
    \hypersetup{unicode=false}%
    \bookmark[%
      dest=\@currentHref,%
      rellevel=1,%
      keeplevel,%
    ]{%
      \HologoVariant{#1}{#2}/\hologoVariant{#1}{#2} %
      (PDFDocEncoding)%
    }%
  \endgroup
  \TestFont{#1}{#2}{OT1}{cmr}{m}{n}%
  \TestFont{#1}{#2}{OT1}{cmss}{m}{n}%
  \TestFont{#1}{#2}{OT1}{cmr}{b}{n}%
  \TestFont{#1}{#2}{OT1}{cmr}{m}{it}%
  \TestFont{#1}{#2}{OT1}{cmtt}{m}{n}%
  \TestFont{#1}{#2}{T1}{lmr}{m}{n}%
  \TestFont{#1}{#2}{T1}{lmss}{m}{n}%
  \TestFont{#1}{#2}{T1}{lmr}{b}{n}%
  \TestFont{#1}{#2}{T1}{lmr}{m}{it}%
  \TestFont{#1}{#2}{T1}{lmtt}{m}{n}%
  \TestFont{#1}{#2}{T1}{lmvtt}{m}{n}%
  \TestFont{#1}{#2}{T1}{qtm}{m}{n}%
  \TestFont{#1}{#2}{T1}{qhv}{m}{n}%
  \TestFont{#1}{#2}{T1}{qtm}{b}{n}%
  \TestFont{#1}{#2}{T1}{qtm}{m}{it}%
  \TestFont{#1}{#2}{T1}{qcr}{m}{n}%
  \newpage
}
\makeatother
\hologoList
\end{document}
%</test-list>
%    \end{macrocode}
%
% \section{Installation}
%
% \subsection{Download}
%
% \paragraph{Package.} This package is available on
% CTAN\footnote{\url{ftp://ftp.ctan.org/tex-archive/}}:
% \begin{description}
% \item[\CTAN{macros/latex/contrib/oberdiek/hologo.dtx}] The source file.
% \item[\CTAN{macros/latex/contrib/oberdiek/hologo.pdf}] Documentation.
% \end{description}
%
%
% \paragraph{Bundle.} All the packages of the bundle `oberdiek'
% are also available in a TDS compliant ZIP archive. There
% the packages are already unpacked and the documentation files
% are generated. The files and directories obey the TDS standard.
% \begin{description}
% \item[\CTAN{install/macros/latex/contrib/oberdiek.tds.zip}]
% \end{description}
% \emph{TDS} refers to the standard ``A Directory Structure
% for \TeX\ Files'' (\CTAN{tds/tds.pdf}). Directories
% with \xfile{texmf} in their name are usually organized this way.
%
% \subsection{Bundle installation}
%
% \paragraph{Unpacking.} Unpack the \xfile{oberdiek.tds.zip} in the
% TDS tree (also known as \xfile{texmf} tree) of your choice.
% Example (linux):
% \begin{quote}
%   |unzip oberdiek.tds.zip -d ~/texmf|
% \end{quote}
%
% \paragraph{Script installation.}
% Check the directory \xfile{TDS:scripts/oberdiek/} for
% scripts that need further installation steps.
% Package \xpackage{attachfile2} comes with the Perl script
% \xfile{pdfatfi.pl} that should be installed in such a way
% that it can be called as \texttt{pdfatfi}.
% Example (linux):
% \begin{quote}
%   |chmod +x scripts/oberdiek/pdfatfi.pl|\\
%   |cp scripts/oberdiek/pdfatfi.pl /usr/local/bin/|
% \end{quote}
%
% \subsection{Package installation}
%
% \paragraph{Unpacking.} The \xfile{.dtx} file is a self-extracting
% \docstrip\ archive. The files are extracted by running the
% \xfile{.dtx} through \plainTeX:
% \begin{quote}
%   \verb|tex hologo.dtx|
% \end{quote}
%
% \paragraph{TDS.} Now the different files must be moved into
% the different directories in your installation TDS tree
% (also known as \xfile{texmf} tree):
% \begin{quote}
% \def\t{^^A
% \begin{tabular}{@{}>{\ttfamily}l@{ $\rightarrow$ }>{\ttfamily}l@{}}
%   hologo.sty & tex/generic/oberdiek/hologo.sty\\
%   hologo.pdf & doc/latex/oberdiek/hologo.pdf\\
%   example/hologo-example.tex & doc/latex/oberdiek/example/hologo-example.tex\\
%   test/hologo-test1.tex & doc/latex/oberdiek/test/hologo-test1.tex\\
%   test/hologo-test-spacefactor.tex & doc/latex/oberdiek/test/hologo-test-spacefactor.tex\\
%   test/hologo-test-list.tex & doc/latex/oberdiek/test/hologo-test-list.tex\\
%   hologo.dtx & source/latex/oberdiek/hologo.dtx\\
% \end{tabular}^^A
% }^^A
% \sbox0{\t}^^A
% \ifdim\wd0>\linewidth
%   \begingroup
%     \advance\linewidth by\leftmargin
%     \advance\linewidth by\rightmargin
%   \edef\x{\endgroup
%     \def\noexpand\lw{\the\linewidth}^^A
%   }\x
%   \def\lwbox{^^A
%     \leavevmode
%     \hbox to \linewidth{^^A
%       \kern-\leftmargin\relax
%       \hss
%       \usebox0
%       \hss
%       \kern-\rightmargin\relax
%     }^^A
%   }^^A
%   \ifdim\wd0>\lw
%     \sbox0{\small\t}^^A
%     \ifdim\wd0>\linewidth
%       \ifdim\wd0>\lw
%         \sbox0{\footnotesize\t}^^A
%         \ifdim\wd0>\linewidth
%           \ifdim\wd0>\lw
%             \sbox0{\scriptsize\t}^^A
%             \ifdim\wd0>\linewidth
%               \ifdim\wd0>\lw
%                 \sbox0{\tiny\t}^^A
%                 \ifdim\wd0>\linewidth
%                   \lwbox
%                 \else
%                   \usebox0
%                 \fi
%               \else
%                 \lwbox
%               \fi
%             \else
%               \usebox0
%             \fi
%           \else
%             \lwbox
%           \fi
%         \else
%           \usebox0
%         \fi
%       \else
%         \lwbox
%       \fi
%     \else
%       \usebox0
%     \fi
%   \else
%     \lwbox
%   \fi
% \else
%   \usebox0
% \fi
% \end{quote}
% If you have a \xfile{docstrip.cfg} that configures and enables \docstrip's
% TDS installing feature, then some files can already be in the right
% place, see the documentation of \docstrip.
%
% \subsection{Refresh file name databases}
%
% If your \TeX~distribution
% (\teTeX, \mikTeX, \dots) relies on file name databases, you must refresh
% these. For example, \teTeX\ users run \verb|texhash| or
% \verb|mktexlsr|.
%
% \subsection{Some details for the interested}
%
% \paragraph{Attached source.}
%
% The PDF documentation on CTAN also includes the
% \xfile{.dtx} source file. It can be extracted by
% AcrobatReader 6 or higher. Another option is \textsf{pdftk},
% e.g. unpack the file into the current directory:
% \begin{quote}
%   \verb|pdftk hologo.pdf unpack_files output .|
% \end{quote}
%
% \paragraph{Unpacking with \LaTeX.}
% The \xfile{.dtx} chooses its action depending on the format:
% \begin{description}
% \item[\plainTeX:] Run \docstrip\ and extract the files.
% \item[\LaTeX:] Generate the documentation.
% \end{description}
% If you insist on using \LaTeX\ for \docstrip\ (really,
% \docstrip\ does not need \LaTeX), then inform the autodetect routine
% about your intention:
% \begin{quote}
%   \verb|latex \let\install=y\input{hologo.dtx}|
% \end{quote}
% Do not forget to quote the argument according to the demands
% of your shell.
%
% \paragraph{Generating the documentation.}
% You can use both the \xfile{.dtx} or the \xfile{.drv} to generate
% the documentation. The process can be configured by the
% configuration file \xfile{ltxdoc.cfg}. For instance, put this
% line into this file, if you want to have A4 as paper format:
% \begin{quote}
%   \verb|\PassOptionsToClass{a4paper}{article}|
% \end{quote}
% An example follows how to generate the
% documentation with pdf\LaTeX:
% \begin{quote}
%\begin{verbatim}
%pdflatex hologo.dtx
%makeindex -s gind.ist hologo.idx
%pdflatex hologo.dtx
%makeindex -s gind.ist hologo.idx
%pdflatex hologo.dtx
%\end{verbatim}
% \end{quote}
%
% \section{Catalogue}
%
% The following XML file can be used as source for the
% \href{http://mirror.ctan.org/help/Catalogue/catalogue.html}{\TeX\ Catalogue}.
% The elements \texttt{caption} and \texttt{description} are imported
% from the original XML file from the Catalogue.
% The name of the XML file in the Catalogue is \xfile{hologo.xml}.
%    \begin{macrocode}
%<*catalogue>
<?xml version='1.0' encoding='us-ascii'?>
<!DOCTYPE entry SYSTEM 'catalogue.dtd'>
<entry datestamp='$Date$' modifier='$Author$' id='hologo'>
  <name>hologo</name>
  <caption>A collection of logos with bookmark support.</caption>
  <authorref id='auth:oberdiek'/>
  <copyright owner='Heiko Oberdiek' year='2010-2012'/>
  <license type='lppl1.3'/>
  <version number='1.10'/>
  <description>
    The package defines a single command <tt>\hologo</tt>, whose
    argument is the usual case-confused ASCII version of the logo.
    The command is bookmark-enabled, so that every logo becomes
    available in bookmarks without further work.
    <p/>
    The package is part of the <xref refid='oberdiek'>oberdiek</xref>
    bundle.
  </description>
  <documentation details='Package documentation'
      href='ctan:/macros/latex/contrib/oberdiek/hologo.pdf'/>
  <ctan file='true' path='/macros/latex/contrib/oberdiek/hologo.dtx'/>
  <miktex location='oberdiek'/>
  <texlive location='oberdiek'/>
  <install path='/macros/latex/contrib/oberdiek/oberdiek.tds.zip'/>
</entry>
%</catalogue>
%    \end{macrocode}
%
% \begin{thebibliography}{9}
% \raggedright
%
% \bibitem{btxdoc}
% Oren Patashnik,
% \textit{\hologo{BibTeX}ing},
% 1988-02-08.\\
% \CTAN{biblio/bibtex/base/}
%
% \bibitem{dtklogos}
% Gerd Neugebauer, DANTE,
% \textit{Package \xpackage{dtklogos}},
% 2011-04-25.\\
% \CTAN{usergrps/dante/dtk/dtklogos.sty}
%
% \bibitem{etexman}
% The \hologo{NTS} Team,
% \textit{The \hologo{eTeX} manual},
% 1998-02.\\
% \CTAN{systems/e-tex/v2/doc/}
%
% \bibitem{ExTeX-FAQ}
% The \hologo{ExTeX} group,
% \textit{\hologo{ExTeX}: FAQ -- How is \hologo{ExTeX} typeset?},
% 2007-04-14.\\
% \url{http://www.extex.org/documentation/faq.html}
%
% \bibitem{LyX}
% %@MISC{ LyX,
% %  title = {{LyX 2.0.0 -- The Document Processor [Computer software and manual]}},
% %  author = {{The LyX Team}},
% %  howpublished = {Internet: http://www.lyx.org},
% %  year = {2011-05-08},
% %  note = {Retrieved May 10, 2011, from http://www.lyx.org},
% %  url = {http://www.lyx.org/}
% %}
% The \hologo{LyX} Team,
% \textit{\hologo{LyX} -- The Document Processor},
% 2011-05-08.\\
% \url{http://www.lyx.org/}
%
% \bibitem{OzTeX}
% Andrew Trevorrow,
% \hologo{OzTeX} FAQ: What is the correct way to typeset ``\hologo{OzTeX}''?,
% 2011-09-15 (visited).
% \url{http://www.trevorrow.com/oztex/ozfaq.html#oztex-logo}
%
% \bibitem{PiCTeX}
% Michael Wichura,
% \textit{The \hologo{PiCTeX} macro package},
% 1987-09-21.
% \CTAN{graphics/pictex/}
%
% \bibitem{scrlogo}
% Markus Kohm,
% \textit{\hologo{KOMAScript} Datei \xfile{scrlogo.dtx}},
% 2009-01-30.\\
% \CTAN{install/macros/latex/contrib/komascript.tds.zip}
%
% \end{thebibliography}
%
% \begin{History}
%   \begin{Version}{2010/04/08 v1.0}
%   \item
%     The first version.
%   \end{Version}
%   \begin{Version}{2010/04/16 v1.1}
%   \item
%     \cs{Hologo} added for support of logos at start of a sentence.
%   \item
%     \cs{hologoSetup} and \cs{hologoLogoSetup} added.
%   \item
%     Options \xoption{break}, \xoption{hyphenbreak}, \xoption{spacebreak}
%     added.
%   \item
%     Variant support added by option \xoption{variant}.
%   \end{Version}
%   \begin{Version}{2010/04/24 v1.2}
%   \item
%     \hologo{LaTeX3} added.
%   \item
%     \hologo{VTeX} added.
%   \end{Version}
%   \begin{Version}{2010/11/21 v1.3}
%   \item
%     \hologo{iniTeX}, \hologo{virTeX} added.
%   \end{Version}
%   \begin{Version}{2011/03/25 v1.4}
%   \item
%     \hologo{ConTeXt} with variants added.
%   \item
%     Option \xoption{discretionarybreak} added as refinement for
%     option \xoption{break}.
%   \end{Version}
%   \begin{Version}{2011/04/21 v1.5}
%   \item
%     Wrong TDS directory for test files fixed.
%   \end{Version}
%   \begin{Version}{2011/10/01 v1.6}
%   \item
%     Support for package \xpackage{tex4ht} added.
%   \item
%     Support for \cs{csname} added if \cs{ifincsname} is available.
%   \item
%     New logos:
%     \hologo{(La)TeX},
%     \hologo{biber},
%     \hologo{BibTeX} (\xoption{sc}, \xoption{sf}),
%     \hologo{emTeX},
%     \hologo{ExTeX},
%     \hologo{KOMAScript},
%     \hologo{La},
%     \hologo{LyX},
%     \hologo{MiKTeX},
%     \hologo{NTS},
%     \hologo{OzMF},
%     \hologo{OzMP},
%     \hologo{OzTeX},
%     \hologo{OzTtH},
%     \hologo{PCTeX},
%     \hologo{PiC},
%     \hologo{PiCTeX},
%     \hologo{METAFONT},
%     \hologo{MetaFun},
%     \hologo{METAPOST},
%     \hologo{MetaPost},
%     \hologo{SLiTeX} (\xoption{lift}, \xoption{narrow}, \xoption{simple}),
%     \hologo{SliTeX} (\xoption{narrow}, \xoption{simple}, \xoption{lift}),
%     \hologo{teTeX}.
%   \item
%     Fixes:
%     \hologo{iniTeX},
%     \hologo{pdfLaTeX},
%     \hologo{pdfTeX},
%     \hologo{virTeX}.
%   \item
%     \cs{hologoFontSetup} and \cs{hologoLogoFontSetup} added.
%   \item
%     \cs{hologoVariant} and \cs{HologoVariant} added.
%   \end{Version}
%   \begin{Version}{2011/11/22 v1.7}
%   \item
%     New logos:
%     \hologo{BibTeX8},
%     \hologo{LaTeXML},
%     \hologo{SageTeX},
%     \hologo{TeX4ht},
%     \hologo{TTH}.
%   \item
%     \hologo{Xe} and friends: Driver stuff fixed.
%   \item
%     \hologo{Xe} and friends: Support for italic added.
%   \item
%     \hologo{Xe} and friends: Package support for \xpackage{pgf}
%     and \xpackage{pstricks} added.
%   \end{Version}
%   \begin{Version}{2011/11/29 v1.8}
%   \item
%     New logos:
%     \hologo{HanTheThanh}.
%   \end{Version}
%   \begin{Version}{2011/12/21 v1.9}
%   \item
%     Patch for package \xpackage{ifxetex} added for the case that
%     \cs{newif} is undefined in \hologo{iniTeX}.
%   \item
%     Some fixes for \hologo{iniTeX}.
%   \end{Version}
%   \begin{Version}{2012/04/26 v1.10}
%   \item
%     Fix in bookmark version of logo ``\hologo{HanTheThanh}''.
%   \end{Version}
%   \begin{Version}{2016/05/12 v1.11}
%   \item
%     Update HOLOGO@IfCharExists (previously in texlive)
%   \item define pdfliteral in current luatex.
%   \end{Version}
% \end{History}
%
% \PrintIndex
%
% \Finale
\endinput
%
        \else
          \input hologo.cfg\relax
        \fi
      \fi
    }%
    \ltx@IfUndefined{newread}{%
      \chardef\HOLOGO@temp=15 %
      \def\HOLOGO@CheckRead{%
        \ifeof\HOLOGO@temp
          \HOLOGO@InputIfExists
        \else
          \ifcase\HOLOGO@temp
            \@PackageWarningNoLine{hologo}{%
              Configuration file ignored, because\MessageBreak
              a free read register could not be found%
            }%
          \else
            \begingroup
              \count\ltx@cclv=\HOLOGO@temp
              \advance\ltx@cclv by \ltx@minusone
              \edef\x{\endgroup
                \chardef\noexpand\HOLOGO@temp=\the\count\ltx@cclv
                \relax
              }%
            \x
          \fi
        \fi
      }%
    }{%
      \csname newread\endcsname\HOLOGO@temp
      \HOLOGO@InputIfExists
    }%
  }{%
    \edef\HOLOGO@temp{\pdf@filesize{hologo.cfg}}%
    \ifx\HOLOGO@temp\ltx@empty
    \else
      \ifnum\HOLOGO@temp>0 %
        \begingroup
          \def\x{LaTeX2e}%
        \expandafter\endgroup
        \ifx\fmtname\x
          % \iffalse meta-comment
%
% File: hologo.dtx
% Version: 2016/05/12 v1.11
% Info: A logo collection with bookmark support
%
% Copyright (C) 2010-2012 by
%    Heiko Oberdiek <heiko.oberdiek at googlemail.com>
%
% This work may be distributed and/or modified under the
% conditions of the LaTeX Project Public License, either
% version 1.3c of this license or (at your option) any later
% version. This version of this license is in
%    http://www.latex-project.org/lppl/lppl-1-3c.txt
% and the latest version of this license is in
%    http://www.latex-project.org/lppl.txt
% and version 1.3 or later is part of all distributions of
% LaTeX version 2005/12/01 or later.
%
% This work has the LPPL maintenance status "maintained".
%
% This Current Maintainer of this work is Heiko Oberdiek.
%
% The Base Interpreter refers to any `TeX-Format',
% because some files are installed in TDS:tex/generic//.
%
% This work consists of the main source file hologo.dtx
% and the derived files
%    hologo.sty, hologo.pdf, hologo.ins, hologo.drv, hologo-example.tex,
%    hologo-test1.tex, hologo-test-spacefactor.tex,
%    hologo-test-list.tex.
%
% Distribution:
%    CTAN:macros/latex/contrib/oberdiek/hologo.dtx
%    CTAN:macros/latex/contrib/oberdiek/hologo.pdf
%
% Unpacking:
%    (a) If hologo.ins is present:
%           tex hologo.ins
%    (b) Without hologo.ins:
%           tex hologo.dtx
%    (c) If you insist on using LaTeX
%           latex \let\install=y\input{hologo.dtx}
%        (quote the arguments according to the demands of your shell)
%
% Documentation:
%    (a) If hologo.drv is present:
%           latex hologo.drv
%    (b) Without hologo.drv:
%           latex hologo.dtx; ...
%    The class ltxdoc loads the configuration file ltxdoc.cfg
%    if available. Here you can specify further options, e.g.
%    use A4 as paper format:
%       \PassOptionsToClass{a4paper}{article}
%
%    Programm calls to get the documentation (example):
%       pdflatex hologo.dtx
%       makeindex -s gind.ist hologo.idx
%       pdflatex hologo.dtx
%       makeindex -s gind.ist hologo.idx
%       pdflatex hologo.dtx
%
% Installation:
%    TDS:tex/generic/oberdiek/hologo.sty
%    TDS:doc/latex/oberdiek/hologo.pdf
%    TDS:doc/latex/oberdiek/example/hologo-example.tex
%    TDS:doc/latex/oberdiek/test/hologo-test1.tex
%    TDS:doc/latex/oberdiek/test/hologo-test-spacefactor.tex
%    TDS:doc/latex/oberdiek/test/hologo-test-list.tex
%    TDS:source/latex/oberdiek/hologo.dtx
%
%<*ignore>
\begingroup
  \catcode123=1 %
  \catcode125=2 %
  \def\x{LaTeX2e}%
\expandafter\endgroup
\ifcase 0\ifx\install y1\fi\expandafter
         \ifx\csname processbatchFile\endcsname\relax\else1\fi
         \ifx\fmtname\x\else 1\fi\relax
\else\csname fi\endcsname
%</ignore>
%<*install>
\input docstrip.tex
\Msg{************************************************************************}
\Msg{* Installation}
\Msg{* Package: hologo 2016/05/12 v1.11 A logo collection with bookmark support (HO)}
\Msg{************************************************************************}

\keepsilent
\askforoverwritefalse

\let\MetaPrefix\relax
\preamble

This is a generated file.

Project: hologo
Version: 2016/05/12 v1.11

Copyright (C) 2010-2012 by
   Heiko Oberdiek <heiko.oberdiek at googlemail.com>

This work may be distributed and/or modified under the
conditions of the LaTeX Project Public License, either
version 1.3c of this license or (at your option) any later
version. This version of this license is in
   http://www.latex-project.org/lppl/lppl-1-3c.txt
and the latest version of this license is in
   http://www.latex-project.org/lppl.txt
and version 1.3 or later is part of all distributions of
LaTeX version 2005/12/01 or later.

This work has the LPPL maintenance status "maintained".

This Current Maintainer of this work is Heiko Oberdiek.

The Base Interpreter refers to any `TeX-Format',
because some files are installed in TDS:tex/generic//.

This work consists of the main source file hologo.dtx
and the derived files
   hologo.sty, hologo.pdf, hologo.ins, hologo.drv, hologo-example.tex,
   hologo-test1.tex, hologo-test-spacefactor.tex,
   hologo-test-list.tex.

\endpreamble
\let\MetaPrefix\DoubleperCent

\generate{%
  \file{hologo.ins}{\from{hologo.dtx}{install}}%
  \file{hologo.drv}{\from{hologo.dtx}{driver}}%
  \usedir{tex/generic/oberdiek}%
  \file{hologo.sty}{\from{hologo.dtx}{package}}%
  \usedir{doc/latex/oberdiek/example}%
  \file{hologo-example.tex}{\from{hologo.dtx}{example}}%
  \usedir{doc/latex/oberdiek/test}%
  \file{hologo-test1.tex}{\from{hologo.dtx}{test1}}%
  \file{hologo-test-spacefactor.tex}{\from{hologo.dtx}{test-spacefactor}}%
  \file{hologo-test-list.tex}{\from{hologo.dtx}{test-list}}%
  \nopreamble
  \nopostamble
  \usedir{source/latex/oberdiek/catalogue}%
  \file{hologo.xml}{\from{hologo.dtx}{catalogue}}%
}

\catcode32=13\relax% active space
\let =\space%
\Msg{************************************************************************}
\Msg{*}
\Msg{* To finish the installation you have to move the following}
\Msg{* file into a directory searched by TeX:}
\Msg{*}
\Msg{*     hologo.sty}
\Msg{*}
\Msg{* To produce the documentation run the file `hologo.drv'}
\Msg{* through LaTeX.}
\Msg{*}
\Msg{* Happy TeXing!}
\Msg{*}
\Msg{************************************************************************}

\endbatchfile
%</install>
%<*ignore>
\fi
%</ignore>
%<*driver>
\NeedsTeXFormat{LaTeX2e}
\ProvidesFile{hologo.drv}%
  [2016/05/12 v1.11 A logo collection with bookmark support (HO)]%
\documentclass{ltxdoc}
\usepackage{holtxdoc}[2011/11/22]
\usepackage{hologo}[2016/05/12]
\usepackage{longtable}
\usepackage{array}
\usepackage{paralist}
%\usepackage[T1]{fontenc}
%\usepackage{lmodern}
\begin{document}
  \DocInput{hologo.dtx}%
\end{document}
%</driver>
% \fi
%
%
% \CharacterTable
%  {Upper-case    \A\B\C\D\E\F\G\H\I\J\K\L\M\N\O\P\Q\R\S\T\U\V\W\X\Y\Z
%   Lower-case    \a\b\c\d\e\f\g\h\i\j\k\l\m\n\o\p\q\r\s\t\u\v\w\x\y\z
%   Digits        \0\1\2\3\4\5\6\7\8\9
%   Exclamation   \!     Double quote  \"     Hash (number) \#
%   Dollar        \$     Percent       \%     Ampersand     \&
%   Acute accent  \'     Left paren    \(     Right paren   \)
%   Asterisk      \*     Plus          \+     Comma         \,
%   Minus         \-     Point         \.     Solidus       \/
%   Colon         \:     Semicolon     \;     Less than     \<
%   Equals        \=     Greater than  \>     Question mark \?
%   Commercial at \@     Left bracket  \[     Backslash     \\
%   Right bracket \]     Circumflex    \^     Underscore    \_
%   Grave accent  \`     Left brace    \{     Vertical bar  \|
%   Right brace   \}     Tilde         \~}
%
% \GetFileInfo{hologo.drv}
%
% \title{The \xpackage{hologo} package}
% \date{2016/05/12 v1.11}
% \author{Heiko Oberdiek\\\xemail{heiko.oberdiek at googlemail.com}}
%
% \maketitle
%
% \begin{abstract}
% This package starts a collection of logos with support for bookmarks
% strings.
% \end{abstract}
%
% \tableofcontents
%
% \section{Documentation}
%
% \subsection{Logo macros}
%
% \begin{declcs}{hologo} \M{name}
% \end{declcs}
% Macro \cs{hologo} sets the logo with name \meta{name}.
% The following table shows the supported names.
%
% \begingroup
%   \def\hologoEntry#1#2#3{^^A
%     #1&#2&\hologoLogoSetup{#1}{variant=#2}\hologo{#1}&#3\tabularnewline
%   }
%   \begin{longtable}{>{\ttfamily}l>{\ttfamily}lll}
%     \rmfamily\bfseries{name} & \rmfamily\bfseries variant
%     & \bfseries logo & \bfseries since\\
%     \hline
%     \endhead
%     \hologoList
%   \end{longtable}
% \endgroup
%
% \begin{declcs}{Hologo} \M{name}
% \end{declcs}
% Macro \cs{Hologo} starts the logo \meta{name} with an uppercase
% letter. As an exception small greek letters are not converted
% to uppercase. Examples, see \hologo{eTeX} and \hologo{ExTeX}.
%
% \subsection{Setup macros}
%
% The package does not support package options, but the following
% setup macros can be used to set options.
%
% \begin{declcs}{hologoSetup} \M{key value list}
% \end{declcs}
% Macro \cs{hologoSetup} sets global options.
%
% \begin{declcs}{hologoLogoSetup} \M{logo} \M{key value list}
% \end{declcs}
% Some options can also be used to configure a logo.
% These settings take precedence over global option settings.
%
% \subsection{Options}\label{sec:options}
%
% There are boolean and string options:
% \begin{description}
% \item[Boolean option:]
% It takes |true| or |false|
% as value. If the value is omitted, then |true| is used.
% \item[String option:]
% A value must be given as string. (But the string might be empty.)
% \end{description}
% The following options can be used both in \cs{hologoSetup}
% and \cs{hologoLogoSetup}:
% \begin{description}
% \def\entry#1{\item[\xoption{#1}:]}
% \entry{break}
%   enables or disables line breaks inside the logo. This setting is
%   refined by options \xoption{hyphenbreak}, \xoption{spacebreak}
%   or \xoption{discretionarybreak}.
%   Default is |false|.
% \entry{hyphenbreak}
%   enables or disables the line break right after the hyphen character.
% \entry{spacebreak}
%   enables or disables line breaks at space characters.
% \entry{discretionarybreak}
%   enables or disables line breaks at hyphenation points
%   (inserted by \cs{-}).
% \end{description}
% Macro \cs{hologoLogoSetup} also knows:
% \begin{description}
% \item[\xoption{variant}:]
%   This is a string option. It specifies a variant of a logo that
%   must exist. An empty string selects the package default variant.
% \end{description}
% Example:
% \begin{quote}
%   |\hologoSetup{break=false}|\\
%   |\hologoLogoSetup{plainTeX}{variant=hyphen,hyphenbreak}|\\
%   Then ``plain-\TeX'' contains one break point after the hyphen.
% \end{quote}
%
% \subsection{Driver options}
%
% Sometimes graphical operations are needed to construct some
% glyphs (e.g.\ \hologo{XeTeX}). If package \xpackage{graphics}
% or package \xpackage{pgf} are found, then the macros are taken
% from there. Otherwise the packge defines its own operations
% and therefore needs the driver information. Many drivers are
% detected automatically (\hologo{pdfTeX}/\hologo{LuaTeX}
% in PDF mode, \hologo{XeTeX}, \hologo{VTeX}). These have precedence
% over a driver option. The driver can be given as package option
% or using \cs{hologoDriverSetup}.
% The following list contains the recognized driver options:
% \begin{itemize}
% \item \xoption{pdftex}, \xoption{luatex}
% \item \xoption{dvipdfm}, \xoption{dvipdfmx}
% \item \xoption{dvips}, \xoption{dvipsone}, \xoption{xdvi}
% \item \xoption{xetex}
% \item \xoption{vtex}
% \end{itemize}
% The left driver of a line is the driver name that is used internally.
% The following names are aliases for drivers that use the
% same method. Therefore the entry in the \xext{log} file for
% the used driver prints the internally used driver name.
% \begin{description}
% \item[\xoption{driverfallback}:]
%   This option expects a driver that is used,
%   if the driver could not be detected automatically.
% \end{description}
%
% \begin{declcs}{hologoDriverSetup} \M{driver option}
% \end{declcs}
% The driver can also be configured after package loading
% using \cs{hologoDriverSetup}, also the way for \hologo{plainTeX}
% to setup the driver.
%
% \subsection{Font setup}
%
% Some logos require a special font, but should also be usable by
% \hologo{plainTeX}. Therefore the package provides some ways
% to influence the font settings. The options below
% take font settings as values. Both font commands
% such as \cs{sffamily} and macros that take one argument
% like \cs{textsf} can be used.
%
% \begin{declcs}{hologoFontSetup} \M{key value list}
% \end{declcs}
% Macro \cs{hologoFontSetup} sets the fonts for all logos.
% Supported keys:
% \begin{description}
% \def\entry#1{\item[\xoption{#1}:]}
% \entry{general}
%   This font is used for all logos. The default is empty.
%   That means no special font is used.
% \entry{bibsf}
%   This font is used for
%   {\hologoLogoSetup{BibTeX}{variant=sf}\hologo{BibTeX}}
%   with variant \xoption{sf}.
% \entry{rm}
%   This font is a serif font. It is used for \hologo{ExTeX}.
% \entry{sc}
%   This font specifies a small caps font. It is used for
%   {\hologoLogoSetup{BibTeX}{variant=sc}\hologo{BibTeX}}
%   with variant \xoption{sc}.
% \entry{sf}
%   This font specifies a sans serif font. The default
%   is \cs{sffamily}, then \cs{sf} is tried. Otherwise
%   a warning is given. It is used by \hologo{KOMAScript}.
% \entry{sy}
%   This is the font for math symbols (e.g. cmsy).
%   It is used by \hologo{AmS}, \hologo{NTS}, \hologo{ExTeX}.
% \entry{logo}
%   \hologo{METAFONT} and \hologo{METAPOST} are using that font.
%   In \hologo{LaTeX} \cs{logofamily} is used and
%   the definitions of package \xpackage{mflogo} are used
%   if the package is not loaded.
%   Otherwise the \cs{tenlogo} is used and defined
%   if it does not already exists.
% \end{description}
%
% \begin{declcs}{hologoLogoFontSetup} \M{logo} \M{key value list}
% \end{declcs}
% Fonts can also be set for a logo or logo component separately,
% see the following list.
% The keys are the same as for \cs{hologoFontSetup}.
%
% \begin{longtable}{>{\ttfamily}l>{\sffamily}ll}
%   \meta{logo} & keys & result\\
%   \hline
%   \endhead
%   BibTeX & bibsf & {\hologoLogoSetup{BibTeX}{variant=sf}\hologo{BibTeX}}\\[.5ex]
%   BibTeX & sc & {\hologoLogoSetup{BibTeX}{variant=sc}\hologo{BibTeX}}\\[.5ex]
%   ExTeX & rm & \hologo{ExTeX}\\
%   SliTeX & rm & \hologo{SliTeX}\\[.5ex]
%   AmS & sy & \hologo{AmS}\\
%   ExTeX & sy & \hologo{ExTeX}\\
%   NTS & sy & \hologo{NTS}\\[.5ex]
%   KOMAScript & sf & \hologo{KOMAScript}\\[.5ex]
%   METAFONT & logo & \hologo{METAFONT}\\
%   METAPOST & logo & \hologo{METAPOST}\\[.5ex]
%   SliTeX & sc \hologo{SliTeX}
% \end{longtable}
%
% \subsubsection{Font order}
%
% For all logos the font \xoption{general} is applied first.
% Example:
%\begin{quote}
%|\hologoFontSetup{general=\color{red}}|
%\end{quote}
% will print red logos.
% Then if the font uses a special font \xoption{sf}, for example,
% the font is applied that is setup by \cs{hologoLogoFontSetup}.
% If this font is not setup, then the common font setup
% by \cs{hologoFontSetup} is used. Otherwise a warning is given,
% that there is no font configured.
%
% \subsection{Additional user macros}
%
% Usually a variant of a logo is configured by using
% \cs{hologoLogoSetup}, because it is bad style to mix
% different variants of the same logo in the same text.
% There the following macros are a convenience for testing.
%
% \begin{declcs}{hologoVariant} \M{name} \M{variant}\\
%   \cs{HologoVariant} \M{name} \M{variant}
% \end{declcs}
% Logo \meta{name} is set using \meta{variant} that specifies
% explicitely which variant of the macro is used. If the argument
% is empty, then the default form of the logo is used
% (configurable by \cs{hologoLogoSetup}).
%
% \cs{HologoVariant} is used if the logo is set in a context
% that needs an uppercase first letter (beginning of a sentence, \dots).
%
% \begin{declcs}{hologoList}\\
%   \cs{hologoEntry} \M{logo} \M{variant} \M{since}
% \end{declcs}
% Macro \cs{hologoList} contains all logos that are provided
% by the package including variants. The list consists of calls
% of \cs{hologoEntry} with three arguments starting with the
% logo name \meta{logo} and its variant \meta{variant}. An empty
% variant means the current default. Argument \meta{since} specifies
% with version of the package \xpackage{hologo} is needed to get
% the logo. If the logo is fixed, then the date gets updated.
% Therefore the date \meta{since} is not exactly the date of
% the first introduction, but rather the date of the latest fix.
%
% Before \cs{hologoList} can be used, macro \cs{hologoEntry} needs
% a definition. The example file in section \ref{sec:example}
% shows applications of \cs{hologoList}.
%
% \subsection{Supported contexts}
%
% Macros \cs{hologo} and friends support special contexts:
% \begin{itemize}
% \item \hologo{LaTeX}'s protection mechanism.
% \item Bookmarks of package \xpackage{hyperref}.
% \item Package \xpackage{tex4ht}.
% \item The macros can be used inside \cs{csname} constructs,
%   if \cs{ifincsname} is available (\hologo{pdfTeX}, \hologo{XeTeX},
%   \hologo{LuaTeX}).
% \end{itemize}
%
% \subsection{Example}
% \label{sec:example}
%
% The following example prints the logos in different fonts.
%    \begin{macrocode}
%<*example>
%<<verbatim
\NeedsTeXFormat{LaTeX2e}
\documentclass[a4paper]{article}
\usepackage[
  hmargin=20mm,
  vmargin=20mm,
]{geometry}
\pagestyle{empty}
\usepackage{hologo}[2016/05/12]
\usepackage{longtable}
\usepackage{array}
\setlength{\extrarowheight}{2pt}
\usepackage[T1]{fontenc}
\usepackage{lmodern}
\usepackage{pdflscape}
\usepackage[
  pdfencoding=auto,
]{hyperref}
\hypersetup{
  pdfauthor={Heiko Oberdiek},
  pdftitle={Example for package `hologo'},
  pdfsubject={Logos with fonts lmr, lmss, qtm, qpl, qhv},
}
\usepackage{bookmark}

% Print the logo list on the console

\begingroup
  \typeout{}%
  \typeout{*** Begin of logo list ***}%
  \newcommand*{\hologoEntry}[3]{%
    \typeout{#1 \ifx\\#2\\\else(#2) \fi[#3]}%
  }%
  \hologoList
  \typeout{*** End of logo list ***}%
  \typeout{}%
\endgroup

\begin{document}
\begin{landscape}

  \section{Example file for package `hologo'}

  % Table for font names

  \begin{longtable}{>{\bfseries}ll}
    \textbf{font} & \textbf{Font name}\\
    \hline
    lmr & Latin Modern Roman\\
    lmss & Latin Modern Sans\\
    qtm & \TeX\ Gyre Termes\\
    qhv & \TeX\ Gyre Heros\\
    qpl & \TeX\ Gyre Pagella\\
  \end{longtable}

  % Logo list with logos in different fonts

  \begingroup
    \newcommand*{\SetVariant}[2]{%
      \ifx\\#2\\%
      \else
        \hologoLogoSetup{#1}{variant=#2}%
      \fi
    }%
    \newcommand*{\hologoEntry}[3]{%
      \SetVariant{#1}{#2}%
      \raisebox{1em}[0pt][0pt]{\hypertarget{#1@#2}{}}%
      \bookmark[%
        dest={#1@#2},%
      ]{%
        #1\ifx\\#2\\\else\space(#2)\fi: \Hologo{#1}, \hologo{#1} %
        [Unicode]%
      }%
      \hypersetup{unicode=false}%
      \bookmark[%
        dest={#1@#2},%
      ]{%
        #1\ifx\\#2\\\else\space(#2)\fi: \Hologo{#1}, \hologo{#1} %
        [PDFDocEncoding]%
      }%
      \texttt{#1}%
      &%
      \texttt{#2}%
      &%
      \Hologo{#1}%
      &%
      \SetVariant{#1}{#2}%
      \hologo{#1}%
      &%
      \SetVariant{#1}{#2}%
      \fontfamily{qtm}\selectfont
      \hologo{#1}%
      &%
      \SetVariant{#1}{#2}%
      \fontfamily{qpl}\selectfont
      \hologo{#1}%
      &%
      \SetVariant{#1}{#2}%
      \textsf{\hologo{#1}}%
      &%
      \SetVariant{#1}{#2}%
      \fontfamily{qhv}\selectfont
      \hologo{#1}%
      \tabularnewline
    }%
    \begin{longtable}{llllllll}%
      \textbf{\textit{logo}} & \textbf{\textit{variant}} &
      \texttt{\string\Hologo} &
      \textbf{lmr} & \textbf{qtm} & \textbf{qpl} &
      \textbf{lmss} & \textbf{qhv}
      \tabularnewline
      \hline
      \endhead
      \hologoList
    \end{longtable}%
  \endgroup

\end{landscape}
\end{document}
%verbatim
%</example>
%    \end{macrocode}
%
% \StopEventually{
% }
%
% \section{Implementation}
%    \begin{macrocode}
%<*package>
%    \end{macrocode}
%    Reload check, especially if the package is not used with \LaTeX.
%    \begin{macrocode}
\begingroup\catcode61\catcode48\catcode32=10\relax%
  \catcode13=5 % ^^M
  \endlinechar=13 %
  \catcode35=6 % #
  \catcode39=12 % '
  \catcode44=12 % ,
  \catcode45=12 % -
  \catcode46=12 % .
  \catcode58=12 % :
  \catcode64=11 % @
  \catcode123=1 % {
  \catcode125=2 % }
  \expandafter\let\expandafter\x\csname ver@hologo.sty\endcsname
  \ifx\x\relax % plain-TeX, first loading
  \else
    \def\empty{}%
    \ifx\x\empty % LaTeX, first loading,
      % variable is initialized, but \ProvidesPackage not yet seen
    \else
      \expandafter\ifx\csname PackageInfo\endcsname\relax
        \def\x#1#2{%
          \immediate\write-1{Package #1 Info: #2.}%
        }%
      \else
        \def\x#1#2{\PackageInfo{#1}{#2, stopped}}%
      \fi
      \x{hologo}{The package is already loaded}%
      \aftergroup\endinput
    \fi
  \fi
\endgroup%
%    \end{macrocode}
%    Package identification:
%    \begin{macrocode}
\begingroup\catcode61\catcode48\catcode32=10\relax%
  \catcode13=5 % ^^M
  \endlinechar=13 %
  \catcode35=6 % #
  \catcode39=12 % '
  \catcode40=12 % (
  \catcode41=12 % )
  \catcode44=12 % ,
  \catcode45=12 % -
  \catcode46=12 % .
  \catcode47=12 % /
  \catcode58=12 % :
  \catcode64=11 % @
  \catcode91=12 % [
  \catcode93=12 % ]
  \catcode123=1 % {
  \catcode125=2 % }
  \expandafter\ifx\csname ProvidesPackage\endcsname\relax
    \def\x#1#2#3[#4]{\endgroup
      \immediate\write-1{Package: #3 #4}%
      \xdef#1{#4}%
    }%
  \else
    \def\x#1#2[#3]{\endgroup
      #2[{#3}]%
      \ifx#1\@undefined
        \xdef#1{#3}%
      \fi
      \ifx#1\relax
        \xdef#1{#3}%
      \fi
    }%
  \fi
\expandafter\x\csname ver@hologo.sty\endcsname
\ProvidesPackage{hologo}%
  [2016/05/12 v1.11 A logo collection with bookmark support (HO)]%
%    \end{macrocode}
%
%    \begin{macrocode}
\begingroup\catcode61\catcode48\catcode32=10\relax%
  \catcode13=5 % ^^M
  \endlinechar=13 %
  \catcode123=1 % {
  \catcode125=2 % }
  \catcode64=11 % @
  \def\x{\endgroup
    \expandafter\edef\csname HOLOGO@AtEnd\endcsname{%
      \endlinechar=\the\endlinechar\relax
      \catcode13=\the\catcode13\relax
      \catcode32=\the\catcode32\relax
      \catcode35=\the\catcode35\relax
      \catcode61=\the\catcode61\relax
      \catcode64=\the\catcode64\relax
      \catcode123=\the\catcode123\relax
      \catcode125=\the\catcode125\relax
    }%
  }%
\x\catcode61\catcode48\catcode32=10\relax%
\catcode13=5 % ^^M
\endlinechar=13 %
\catcode35=6 % #
\catcode64=11 % @
\catcode123=1 % {
\catcode125=2 % }
\def\TMP@EnsureCode#1#2{%
  \edef\HOLOGO@AtEnd{%
    \HOLOGO@AtEnd
    \catcode#1=\the\catcode#1\relax
  }%
  \catcode#1=#2\relax
}
\TMP@EnsureCode{10}{12}% ^^J
\TMP@EnsureCode{33}{12}% !
\TMP@EnsureCode{34}{12}% "
\TMP@EnsureCode{36}{3}% $
\TMP@EnsureCode{38}{4}% &
\TMP@EnsureCode{39}{12}% '
\TMP@EnsureCode{40}{12}% (
\TMP@EnsureCode{41}{12}% )
\TMP@EnsureCode{42}{12}% *
\TMP@EnsureCode{43}{12}% +
\TMP@EnsureCode{44}{12}% ,
\TMP@EnsureCode{45}{12}% -
\TMP@EnsureCode{46}{12}% .
\TMP@EnsureCode{47}{12}% /
\TMP@EnsureCode{58}{12}% :
\TMP@EnsureCode{59}{12}% ;
\TMP@EnsureCode{60}{12}% <
\TMP@EnsureCode{62}{12}% >
\TMP@EnsureCode{63}{12}% ?
\TMP@EnsureCode{91}{12}% [
\TMP@EnsureCode{93}{12}% ]
\TMP@EnsureCode{94}{7}% ^ (superscript)
\TMP@EnsureCode{95}{8}% _ (subscript)
\TMP@EnsureCode{96}{12}% `
\TMP@EnsureCode{124}{12}% |
\edef\HOLOGO@AtEnd{%
  \HOLOGO@AtEnd
  \escapechar\the\escapechar\relax
  \noexpand\endinput
}
\escapechar=92 %
%    \end{macrocode}
%
% \subsection{Logo list}
%
%    \begin{macro}{\hologoList}
%    \begin{macrocode}
\def\hologoList{%
  \hologoEntry{(La)TeX}{}{2011/10/01}%
  \hologoEntry{AmSLaTeX}{}{2010/04/16}%
  \hologoEntry{AmSTeX}{}{2010/04/16}%
  \hologoEntry{biber}{}{2011/10/01}%
  \hologoEntry{BibTeX}{}{2011/10/01}%
  \hologoEntry{BibTeX}{sf}{2011/10/01}%
  \hologoEntry{BibTeX}{sc}{2011/10/01}%
  \hologoEntry{BibTeX8}{}{2011/11/22}%
  \hologoEntry{ConTeXt}{}{2011/03/25}%
  \hologoEntry{ConTeXt}{narrow}{2011/03/25}%
  \hologoEntry{ConTeXt}{simple}{2011/03/25}%
  \hologoEntry{emTeX}{}{2010/04/26}%
  \hologoEntry{eTeX}{}{2010/04/08}%
  \hologoEntry{ExTeX}{}{2011/10/01}%
  \hologoEntry{HanTheThanh}{}{2011/11/29}%
  \hologoEntry{iniTeX}{}{2011/10/01}%
  \hologoEntry{KOMAScript}{}{2011/10/01}%
  \hologoEntry{La}{}{2010/05/08}%
  \hologoEntry{LaTeX}{}{2010/04/08}%
  \hologoEntry{LaTeX2e}{}{2010/04/08}%
  \hologoEntry{LaTeX3}{}{2010/04/24}%
  \hologoEntry{LaTeXe}{}{2010/04/08}%
  \hologoEntry{LaTeXML}{}{2011/11/22}%
  \hologoEntry{LaTeXTeX}{}{2011/10/01}%
  \hologoEntry{LuaLaTeX}{}{2010/04/08}%
  \hologoEntry{LuaTeX}{}{2010/04/08}%
  \hologoEntry{LyX}{}{2011/10/01}%
  \hologoEntry{METAFONT}{}{2011/10/01}%
  \hologoEntry{MetaFun}{}{2011/10/01}%
  \hologoEntry{METAPOST}{}{2011/10/01}%
  \hologoEntry{MetaPost}{}{2011/10/01}%
  \hologoEntry{MiKTeX}{}{2011/10/01}%
  \hologoEntry{NTS}{}{2011/10/01}%
  \hologoEntry{OzMF}{}{2011/10/01}%
  \hologoEntry{OzMP}{}{2011/10/01}%
  \hologoEntry{OzTeX}{}{2011/10/01}%
  \hologoEntry{OzTtH}{}{2011/10/01}%
  \hologoEntry{PCTeX}{}{2011/10/01}%
  \hologoEntry{pdfTeX}{}{2011/10/01}%
  \hologoEntry{pdfLaTeX}{}{2011/10/01}%
  \hologoEntry{PiC}{}{2011/10/01}%
  \hologoEntry{PiCTeX}{}{2011/10/01}%
  \hologoEntry{plainTeX}{}{2010/04/08}%
  \hologoEntry{plainTeX}{space}{2010/04/16}%
  \hologoEntry{plainTeX}{hyphen}{2010/04/16}%
  \hologoEntry{plainTeX}{runtogether}{2010/04/16}%
  \hologoEntry{SageTeX}{}{2011/11/22}%
  \hologoEntry{SLiTeX}{}{2011/10/01}%
  \hologoEntry{SLiTeX}{lift}{2011/10/01}%
  \hologoEntry{SLiTeX}{narrow}{2011/10/01}%
  \hologoEntry{SLiTeX}{simple}{2011/10/01}%
  \hologoEntry{SliTeX}{}{2011/10/01}%
  \hologoEntry{SliTeX}{narrow}{2011/10/01}%
  \hologoEntry{SliTeX}{simple}{2011/10/01}%
  \hologoEntry{SliTeX}{lift}{2011/10/01}%
  \hologoEntry{teTeX}{}{2011/10/01}%
  \hologoEntry{TeX}{}{2010/04/08}%
  \hologoEntry{TeX4ht}{}{2011/11/22}%
  \hologoEntry{TTH}{}{2011/11/22}%
  \hologoEntry{virTeX}{}{2011/10/01}%
  \hologoEntry{VTeX}{}{2010/04/24}%
  \hologoEntry{Xe}{}{2010/04/08}%
  \hologoEntry{XeLaTeX}{}{2010/04/08}%
  \hologoEntry{XeTeX}{}{2010/04/08}%
}
%    \end{macrocode}
%    \end{macro}
%
% \subsection{Load resources}
%
%    \begin{macrocode}
\begingroup\expandafter\expandafter\expandafter\endgroup
\expandafter\ifx\csname RequirePackage\endcsname\relax
  \def\TMP@RequirePackage#1[#2]{%
    \begingroup\expandafter\expandafter\expandafter\endgroup
    \expandafter\ifx\csname ver@#1.sty\endcsname\relax
      \input #1.sty\relax
    \fi
  }%
  \TMP@RequirePackage{ltxcmds}[2011/02/04]%
  \TMP@RequirePackage{infwarerr}[2010/04/08]%
  \TMP@RequirePackage{kvsetkeys}[2010/03/01]%
  \TMP@RequirePackage{kvdefinekeys}[2010/03/01]%
  \TMP@RequirePackage{pdftexcmds}[2010/04/01]%
  \TMP@RequirePackage{ifpdf}[2010/01/28]%
  \TMP@RequirePackage{ifluatex}[2010/03/01]%
  \ltx@IfUndefined{newif}{%
    \expandafter\let\csname newif\endcsname\ltx@newif
  }{}%
  \TMP@RequirePackage{ifxetex}[2009/01/23]%
  \TMP@RequirePackage{ifvtex}[2010/03/01]%
\else
  \RequirePackage{ltxcmds}[2011/02/04]%
  \RequirePackage{infwarerr}[2010/04/08]%
  \RequirePackage{kvsetkeys}[2010/03/01]%
  \RequirePackage{kvdefinekeys}[2010/03/01]%
  \RequirePackage{pdftexcmds}[2010/04/01]%
  \RequirePackage{ifpdf}[2010/01/28]%
  \RequirePackage{ifluatex}[2010/03/01]%
  \RequirePackage{ifxetex}[2009/01/23]%
  \RequirePackage{ifvtex}[2010/03/01]%
\fi
%    \end{macrocode}
%
%    \begin{macro}{\HOLOGO@IfDefined}
%    \begin{macrocode}
\def\HOLOGO@IfExists#1{%
  \ifx\@undefined#1%
    \expandafter\ltx@secondoftwo
  \else
    \ifx\relax#1%
      \expandafter\ltx@secondoftwo
    \else
      \expandafter\expandafter\expandafter\ltx@firstoftwo
    \fi
  \fi
}
%    \end{macrocode}
%    \end{macro}
%
% \subsection{Setup macros}
%
%    \begin{macro}{\hologoSetup}
%    \begin{macrocode}
\def\hologoSetup{%
  \let\HOLOGO@name\relax
  \HOLOGO@Setup
}
%    \end{macrocode}
%    \end{macro}
%
%    \begin{macro}{\hologoLogoSetup}
%    \begin{macrocode}
\def\hologoLogoSetup#1{%
  \edef\HOLOGO@name{#1}%
  \ltx@IfUndefined{HoLogo@\HOLOGO@name}{%
    \@PackageError{hologo}{%
      Unknown logo `\HOLOGO@name'%
    }\@ehc
    \ltx@gobble
  }{%
    \HOLOGO@Setup
  }%
}
%    \end{macrocode}
%    \end{macro}
%
%    \begin{macro}{\HOLOGO@Setup}
%    \begin{macrocode}
\def\HOLOGO@Setup{%
  \kvsetkeys{HoLogo}%
}
%    \end{macrocode}
%    \end{macro}
%
% \subsection{Options}
%
%    \begin{macro}{\HOLOGO@DeclareBoolOption}
%    \begin{macrocode}
\def\HOLOGO@DeclareBoolOption#1{%
  \expandafter\chardef\csname HOLOGOOPT@#1\endcsname\ltx@zero
  \kv@define@key{HoLogo}{#1}[true]{%
    \def\HOLOGO@temp{##1}%
    \ifx\HOLOGO@temp\HOLOGO@true
      \ifx\HOLOGO@name\relax
        \expandafter\chardef\csname HOLOGOOPT@#1\endcsname=\ltx@one
      \else
        \expandafter\chardef\csname
        HoLogoOpt@#1@\HOLOGO@name\endcsname\ltx@one
      \fi
      \HOLOGO@SetBreakAll{#1}%
    \else
      \ifx\HOLOGO@temp\HOLOGO@false
        \ifx\HOLOGO@name\relax
          \expandafter\chardef\csname HOLOGOOPT@#1\endcsname=\ltx@zero
        \else
          \expandafter\chardef\csname
          HoLogoOpt@#1@\HOLOGO@name\endcsname=\ltx@zero
        \fi
        \HOLOGO@SetBreakAll{#1}%
      \else
        \@PackageError{hologo}{%
          Unknown value `##1' for boolean option `#1'.\MessageBreak
          Known values are `true' and `false'%
        }\@ehc
      \fi
    \fi
  }%
}
%    \end{macrocode}
%    \end{macro}
%
%    \begin{macro}{\HOLOGO@SetBreakAll}
%    \begin{macrocode}
\def\HOLOGO@SetBreakAll#1{%
  \def\HOLOGO@temp{#1}%
  \ifx\HOLOGO@temp\HOLOGO@break
    \ifx\HOLOGO@name\relax
      \chardef\HOLOGOOPT@hyphenbreak=\HOLOGOOPT@break
      \chardef\HOLOGOOPT@spacebreak=\HOLOGOOPT@break
      \chardef\HOLOGOOPT@discretionarybreak=\HOLOGOOPT@break
    \else
      \expandafter\chardef
         \csname HoLogoOpt@hyphenbreak@\HOLOGO@name\endcsname=%
         \csname HoLogoOpt@break@\HOLOGO@name\endcsname
      \expandafter\chardef
         \csname HoLogoOpt@spacebreak@\HOLOGO@name\endcsname=%
         \csname HoLogoOpt@break@\HOLOGO@name\endcsname
      \expandafter\chardef
         \csname HoLogoOpt@discretionarybreak@\HOLOGO@name
             \endcsname=%
         \csname HoLogoOpt@break@\HOLOGO@name\endcsname
    \fi
  \fi
}
%    \end{macrocode}
%    \end{macro}
%
%    \begin{macro}{\HOLOGO@true}
%    \begin{macrocode}
\def\HOLOGO@true{true}
%    \end{macrocode}
%    \end{macro}
%    \begin{macro}{\HOLOGO@false}
%    \begin{macrocode}
\def\HOLOGO@false{false}
%    \end{macrocode}
%    \end{macro}
%    \begin{macro}{\HOLOGO@break}
%    \begin{macrocode}
\def\HOLOGO@break{break}
%    \end{macrocode}
%    \end{macro}
%
%    \begin{macrocode}
\HOLOGO@DeclareBoolOption{break}
\HOLOGO@DeclareBoolOption{hyphenbreak}
\HOLOGO@DeclareBoolOption{spacebreak}
\HOLOGO@DeclareBoolOption{discretionarybreak}
%    \end{macrocode}
%
%    \begin{macrocode}
\kv@define@key{HoLogo}{variant}{%
  \ifx\HOLOGO@name\relax
    \@PackageError{hologo}{%
      Option `variant' is not available in \string\hologoSetup,%
      \MessageBreak
      Use \string\hologoLogoSetup\space instead%
    }\@ehc
  \else
    \edef\HOLOGO@temp{#1}%
    \ifx\HOLOGO@temp\ltx@empty
      \expandafter
      \let\csname HoLogoOpt@variant@\HOLOGO@name\endcsname\@undefined
    \else
      \ltx@IfUndefined{HoLogo@\HOLOGO@name @\HOLOGO@temp}{%
        \@PackageError{hologo}{%
          Unknown variant `\HOLOGO@temp' of logo `\HOLOGO@name'%
        }\@ehc
      }{%
        \expandafter
        \let\csname HoLogoOpt@variant@\HOLOGO@name\endcsname
            \HOLOGO@temp
      }%
    \fi
  \fi
}
%    \end{macrocode}
%
%    \begin{macro}{\HOLOGO@Variant}
%    \begin{macrocode}
\def\HOLOGO@Variant#1{%
  #1%
  \ltx@ifundefined{HoLogoOpt@variant@#1}{%
  }{%
    @\csname HoLogoOpt@variant@#1\endcsname
  }%
}
%    \end{macrocode}
%    \end{macro}
%
% \subsection{Break/no-break support}
%
%    \begin{macro}{\HOLOGO@space}
%    \begin{macrocode}
\def\HOLOGO@space{%
  \ltx@ifundefined{HoLogoOpt@spacebreak@\HOLOGO@name}{%
    \ltx@ifundefined{HoLogoOpt@break@\HOLOGO@name}{%
      \chardef\HOLOGO@temp=\HOLOGOOPT@spacebreak
    }{%
      \chardef\HOLOGO@temp=%
        \csname HoLogoOpt@break@\HOLOGO@name\endcsname
    }%
  }{%
    \chardef\HOLOGO@temp=%
      \csname HoLogoOpt@spacebreak@\HOLOGO@name\endcsname
  }%
  \ifcase\HOLOGO@temp
    \penalty10000 %
  \fi
  \ltx@space
}
%    \end{macrocode}
%    \end{macro}
%
%    \begin{macro}{\HOLOGO@hyphen}
%    \begin{macrocode}
\def\HOLOGO@hyphen{%
  \ltx@ifundefined{HoLogoOpt@hyphenbreak@\HOLOGO@name}{%
    \ltx@ifundefined{HoLogoOpt@break@\HOLOGO@name}{%
      \chardef\HOLOGO@temp=\HOLOGOOPT@hyphenbreak
    }{%
      \chardef\HOLOGO@temp=%
        \csname HoLogoOpt@break@\HOLOGO@name\endcsname
    }%
  }{%
    \chardef\HOLOGO@temp=%
      \csname HoLogoOpt@hyphenbreak@\HOLOGO@name\endcsname
  }%
  \ifcase\HOLOGO@temp
    \ltx@mbox{-}%
  \else
    -%
  \fi
}
%    \end{macrocode}
%    \end{macro}
%
%    \begin{macro}{\HOLOGO@discretionary}
%    \begin{macrocode}
\def\HOLOGO@discretionary{%
  \ltx@ifundefined{HoLogoOpt@discretionarybreak@\HOLOGO@name}{%
    \ltx@ifundefined{HoLogoOpt@break@\HOLOGO@name}{%
      \chardef\HOLOGO@temp=\HOLOGOOPT@discretionarybreak
    }{%
      \chardef\HOLOGO@temp=%
        \csname HoLogoOpt@break@\HOLOGO@name\endcsname
    }%
  }{%
    \chardef\HOLOGO@temp=%
      \csname HoLogoOpt@discretionarybreak@\HOLOGO@name\endcsname
  }%
  \ifcase\HOLOGO@temp
  \else
    \-%
  \fi
}
%    \end{macrocode}
%    \end{macro}
%
%    \begin{macro}{\HOLOGO@mbox}
%    \begin{macrocode}
\def\HOLOGO@mbox#1{%
  \ltx@ifundefined{HoLogoOpt@break@\HOLOGO@name}{%
    \chardef\HOLOGO@temp=\HOLOGOOPT@hyphenbreak
  }{%
    \chardef\HOLOGO@temp=%
      \csname HoLogoOpt@break@\HOLOGO@name\endcsname
  }%
  \ifcase\HOLOGO@temp
    \ltx@mbox{#1}%
  \else
    #1%
  \fi
}
%    \end{macrocode}
%    \end{macro}
%
% \subsection{Font support}
%
%    \begin{macro}{\HoLogoFont@font}
%    \begin{tabular}{@{}ll@{}}
%    |#1|:& logo name\\
%    |#2|:& font short name\\
%    |#3|:& text
%    \end{tabular}
%    \begin{macrocode}
\def\HoLogoFont@font#1#2#3{%
  \begingroup
    \ltx@IfUndefined{HoLogoFont@logo@#1.#2}{%
      \ltx@IfUndefined{HoLogoFont@font@#2}{%
        \@PackageWarning{hologo}{%
          Missing font `#2' for logo `#1'%
        }%
        #3%
      }{%
        \csname HoLogoFont@font@#2\endcsname{#3}%
      }%
    }{%
      \csname HoLogoFont@logo@#1.#2\endcsname{#3}%
    }%
  \endgroup
}
%    \end{macrocode}
%    \end{macro}
%
%    \begin{macro}{\HoLogoFont@Def}
%    \begin{macrocode}
\def\HoLogoFont@Def#1{%
  \expandafter\def\csname HoLogoFont@font@#1\endcsname
}
%    \end{macrocode}
%    \end{macro}
%    \begin{macro}{\HoLogoFont@LogoDef}
%    \begin{macrocode}
\def\HoLogoFont@LogoDef#1#2{%
  \expandafter\def\csname HoLogoFont@logo@#1.#2\endcsname
}
%    \end{macrocode}
%    \end{macro}
%
% \subsubsection{Font defaults}
%
%    \begin{macro}{\HoLogoFont@font@general}
%    \begin{macrocode}
\HoLogoFont@Def{general}{}%
%    \end{macrocode}
%    \end{macro}
%
%    \begin{macro}{\HoLogoFont@font@rm}
%    \begin{macrocode}
\ltx@IfUndefined{rmfamily}{%
  \ltx@IfUndefined{rm}{%
  }{%
    \HoLogoFont@Def{rm}{\rm}%
  }%
}{%
  \HoLogoFont@Def{rm}{\rmfamily}%
}
%    \end{macrocode}
%    \end{macro}
%
%    \begin{macro}{\HoLogoFont@font@sf}
%    \begin{macrocode}
\ltx@IfUndefined{sffamily}{%
  \ltx@IfUndefined{sf}{%
  }{%
    \HoLogoFont@Def{sf}{\sf}%
  }%
}{%
  \HoLogoFont@Def{sf}{\sffamily}%
}
%    \end{macrocode}
%    \end{macro}
%
%    \begin{macro}{\HoLogoFont@font@bibsf}
%    In case of \hologo{plainTeX} the original small caps
%    variant is used as default. In \hologo{LaTeX}
%    the definition of package \xpackage{dtklogos} \cite{dtklogos}
%    is used.
%\begin{quote}
%\begin{verbatim}
%\DeclareRobustCommand{\BibTeX}{%
%  B%
%  \kern-.05em%
%  \hbox{%
%    $\m@th$% %% force math size calculations
%    \csname S@\f@size\endcsname
%    \fontsize\sf@size\z@
%    \math@fontsfalse
%    \selectfont
%    I%
%    \kern-.025em%
%    B
%  }%
%  \kern-.08em%
%  \-%
%  \TeX
%}
%\end{verbatim}
%\end{quote}
%    \begin{macrocode}
\ltx@IfUndefined{selectfont}{%
  \ltx@IfUndefined{tensc}{%
    \font\tensc=cmcsc10\relax
  }{}%
  \HoLogoFont@Def{bibsf}{\tensc}%
}{%
  \HoLogoFont@Def{bibsf}{%
    $\mathsurround=0pt$%
    \csname S@\f@size\endcsname
    \fontsize\sf@size{0pt}%
    \math@fontsfalse
    \selectfont
  }%
}
%    \end{macrocode}
%    \end{macro}
%
%    \begin{macro}{\HoLogoFont@font@sc}
%    \begin{macrocode}
\ltx@IfUndefined{scshape}{%
  \ltx@IfUndefined{tensc}{%
    \font\tensc=cmcsc10\relax
  }{}%
  \HoLogoFont@Def{sc}{\tensc}%
}{%
  \HoLogoFont@Def{sc}{\scshape}%
}
%    \end{macrocode}
%    \end{macro}
%
%    \begin{macro}{\HoLogoFont@font@sy}
%    \begin{macrocode}
\ltx@IfUndefined{usefont}{%
  \ltx@IfUndefined{tensy}{%
  }{%
    \HoLogoFont@Def{sy}{\tensy}%
  }%
}{%
  \HoLogoFont@Def{sy}{%
    \usefont{OMS}{cmsy}{m}{n}%
  }%
}
%    \end{macrocode}
%    \end{macro}
%
%    \begin{macro}{\HoLogoFont@font@logo}
%    \begin{macrocode}
\begingroup
  \def\x{LaTeX2e}%
\expandafter\endgroup
\ifx\fmtname\x
  \ltx@IfUndefined{logofamily}{%
    \DeclareRobustCommand\logofamily{%
      \not@math@alphabet\logofamily\relax
      \fontencoding{U}%
      \fontfamily{logo}%
      \selectfont
    }%
  }{}%
  \ltx@IfUndefined{logofamily}{%
  }{%
    \HoLogoFont@Def{logo}{\logofamily}%
  }%
\else
  \ltx@IfUndefined{tenlogo}{%
    \font\tenlogo=logo10\relax
  }{}%
  \HoLogoFont@Def{logo}{\tenlogo}%
\fi
%    \end{macrocode}
%    \end{macro}
%
% \subsubsection{Font setup}
%
%    \begin{macro}{\hologoFontSetup}
%    \begin{macrocode}
\def\hologoFontSetup{%
  \let\HOLOGO@name\relax
  \HOLOGO@FontSetup
}
%    \end{macrocode}
%    \end{macro}
%
%    \begin{macro}{\hologoLogoFontSetup}
%    \begin{macrocode}
\def\hologoLogoFontSetup#1{%
  \edef\HOLOGO@name{#1}%
  \ltx@IfUndefined{HoLogo@\HOLOGO@name}{%
    \@PackageError{hologo}{%
      Unknown logo `\HOLOGO@name'%
    }\@ehc
    \ltx@gobble
  }{%
    \HOLOGO@FontSetup
  }%
}
%    \end{macrocode}
%    \end{macro}
%
%    \begin{macro}{\HOLOGO@FontSetup}
%    \begin{macrocode}
\def\HOLOGO@FontSetup{%
  \kvsetkeys{HoLogoFont}%
}
%    \end{macrocode}
%    \end{macro}
%
%    \begin{macrocode}
\def\HOLOGO@temp#1{%
  \kv@define@key{HoLogoFont}{#1}{%
    \ifx\HOLOGO@name\relax
      \HoLogoFont@Def{#1}{##1}%
    \else
      \HoLogoFont@LogoDef\HOLOGO@name{#1}{##1}%
    \fi
  }%
}
\HOLOGO@temp{general}
\HOLOGO@temp{sf}
%    \end{macrocode}
%
% \subsection{Generic logo commands}
%
%    \begin{macrocode}
\HOLOGO@IfExists\hologo{%
  \@PackageError{hologo}{%
    \string\hologo\ltx@space is already defined.\MessageBreak
    Package loading is aborted%
  }\@ehc
  \HOLOGO@AtEnd
}%
\HOLOGO@IfExists\hologoRobust{%
  \@PackageError{hologo}{%
    \string\hologoRobust\ltx@space is already defined.\MessageBreak
    Package loading is aborted%
  }\@ehc
  \HOLOGO@AtEnd
}%
%    \end{macrocode}
%
% \subsubsection{\cs{hologo} and friends}
%
%    \begin{macrocode}
\ifluatex
  \expandafter\ltx@firstofone
\else
  \expandafter\ltx@gobble
\fi
{%
  \ltx@IfUndefined{ifincsname}{%
    \ifnum\luatexversion<36 %
      \expandafter\ltx@gobble
    \else
      \expandafter\ltx@firstofone
    \fi
    {%
      \begingroup
        \ifcase0%
            \directlua{%
              if tex.enableprimitives then %
                tex.enableprimitives('HOLOGO@', {'ifincsname'})%
              else %
                tex.print('1')%
              end%
            }%
            \ifx\HOLOGO@ifincsname\@undefined 1\fi%
            \relax
          \expandafter\ltx@firstofone
        \else
          \endgroup
          \expandafter\ltx@gobble
        \fi
        {%
          \global\let\ifincsname\HOLOGO@ifincsname
        }%
      \HOLOGO@temp
    }%
  }{}%
}
%    \end{macrocode}
%    \begin{macrocode}
\ltx@IfUndefined{ifincsname}{%
  \catcode`$=14 %
}{%
  \catcode`$=9 %
}
%    \end{macrocode}
%
%    \begin{macro}{\hologo}
%    \begin{macrocode}
\def\hologo#1{%
$ \ifincsname
$   \ltx@ifundefined{HoLogoCs@\HOLOGO@Variant{#1}}{%
$     #1%
$   }{%
$     \csname HoLogoCs@\HOLOGO@Variant{#1}\endcsname\ltx@firstoftwo
$   }%
$ \else
    \HOLOGO@IfExists\texorpdfstring\texorpdfstring\ltx@firstoftwo
    {%
      \hologoRobust{#1}%
    }{%
      \ltx@ifundefined{HoLogoBkm@\HOLOGO@Variant{#1}}{%
        \ltx@ifundefined{HoLogo@#1}{?#1?}{#1}%
      }{%
        \csname HoLogoBkm@\HOLOGO@Variant{#1}\endcsname
        \ltx@firstoftwo
      }%
    }%
$ \fi
}
%    \end{macrocode}
%    \end{macro}
%    \begin{macro}{\Hologo}
%    \begin{macrocode}
\def\Hologo#1{%
$ \ifincsname
$   \ltx@ifundefined{HoLogoCs@\HOLOGO@Variant{#1}}{%
$     #1%
$   }{%
$     \csname HoLogoCs@\HOLOGO@Variant{#1}\endcsname\ltx@secondoftwo
$   }%
$ \else
    \HOLOGO@IfExists\texorpdfstring\texorpdfstring\ltx@firstoftwo
    {%
      \HologoRobust{#1}%
    }{%
      \ltx@ifundefined{HoLogoBkm@\HOLOGO@Variant{#1}}{%
        \ltx@ifundefined{HoLogo@#1}{?#1?}{#1}%
      }{%
        \csname HoLogoBkm@\HOLOGO@Variant{#1}\endcsname
        \ltx@secondoftwo
      }%
    }%
$ \fi
}
%    \end{macrocode}
%    \end{macro}
%
%    \begin{macro}{\hologoVariant}
%    \begin{macrocode}
\def\hologoVariant#1#2{%
  \ifx\relax#2\relax
    \hologo{#1}%
  \else
$   \ifincsname
$     \ltx@ifundefined{HoLogoCs@#1@#2}{%
$       #1%
$     }{%
$       \csname HoLogoCs@#1@#2\endcsname\ltx@firstoftwo
$     }%
$   \else
      \HOLOGO@IfExists\texorpdfstring\texorpdfstring\ltx@firstoftwo
      {%
        \hologoVariantRobust{#1}{#2}%
      }{%
        \ltx@ifundefined{HoLogoBkm@#1@#2}{%
          \ltx@ifundefined{HoLogo@#1}{?#1?}{#1}%
        }{%
          \csname HoLogoBkm@#1@#2\endcsname
          \ltx@firstoftwo
        }%
      }%
$   \fi
  \fi
}
%    \end{macrocode}
%    \end{macro}
%    \begin{macro}{\HologoVariant}
%    \begin{macrocode}
\def\HologoVariant#1#2{%
  \ifx\relax#2\relax
    \Hologo{#1}%
  \else
$   \ifincsname
$     \ltx@ifundefined{HoLogoCs@#1@#2}{%
$       #1%
$     }{%
$       \csname HoLogoCs@#1@#2\endcsname\ltx@secondoftwo
$     }%
$   \else
      \HOLOGO@IfExists\texorpdfstring\texorpdfstring\ltx@firstoftwo
      {%
        \HologoVariantRobust{#1}{#2}%
      }{%
        \ltx@ifundefined{HoLogoBkm@#1@#2}{%
          \ltx@ifundefined{HoLogo@#1}{?#1?}{#1}%
        }{%
          \csname HoLogoBkm@#1@#2\endcsname
          \ltx@secondoftwo
        }%
      }%
$   \fi
  \fi
}
%    \end{macrocode}
%    \end{macro}
%
%    \begin{macrocode}
\catcode`\$=3 %
%    \end{macrocode}
%
% \subsubsection{\cs{hologoRobust} and friends}
%
%    \begin{macro}{\hologoRobust}
%    \begin{macrocode}
\ltx@IfUndefined{protected}{%
  \ltx@IfUndefined{DeclareRobustCommand}{%
    \def\hologoRobust#1%
  }{%
    \DeclareRobustCommand*\hologoRobust[1]%
  }%
}{%
  \protected\def\hologoRobust#1%
}%
{%
  \edef\HOLOGO@name{#1}%
  \ltx@IfUndefined{HoLogo@\HOLOGO@Variant\HOLOGO@name}{%
    \@PackageError{hologo}{%
      Unknown logo `\HOLOGO@name'%
    }\@ehc
    ?\HOLOGO@name?%
  }{%
    \ltx@IfUndefined{ver@tex4ht.sty}{%
      \HoLogoFont@font\HOLOGO@name{general}{%
        \csname HoLogo@\HOLOGO@Variant\HOLOGO@name\endcsname
        \ltx@firstoftwo
      }%
    }{%
      \ltx@IfUndefined{HoLogoHtml@\HOLOGO@Variant\HOLOGO@name}{%
        \HOLOGO@name
      }{%
        \csname HoLogoHtml@\HOLOGO@Variant\HOLOGO@name\endcsname
        \ltx@firstoftwo
      }%
    }%
  }%
}
%    \end{macrocode}
%    \end{macro}
%    \begin{macro}{\HologoRobust}
%    \begin{macrocode}
\ltx@IfUndefined{protected}{%
  \ltx@IfUndefined{DeclareRobustCommand}{%
    \def\HologoRobust#1%
  }{%
    \DeclareRobustCommand*\HologoRobust[1]%
  }%
}{%
  \protected\def\HologoRobust#1%
}%
{%
  \edef\HOLOGO@name{#1}%
  \ltx@IfUndefined{HoLogo@\HOLOGO@Variant\HOLOGO@name}{%
    \@PackageError{hologo}{%
      Unknown logo `\HOLOGO@name'%
    }\@ehc
    ?\HOLOGO@name?%
  }{%
    \ltx@IfUndefined{ver@tex4ht.sty}{%
      \HoLogoFont@font\HOLOGO@name{general}{%
        \csname HoLogo@\HOLOGO@Variant\HOLOGO@name\endcsname
        \ltx@secondoftwo
      }%
    }{%
      \ltx@IfUndefined{HoLogoHtml@\HOLOGO@Variant\HOLOGO@name}{%
        \expandafter\HOLOGO@Uppercase\HOLOGO@name
      }{%
        \csname HoLogoHtml@\HOLOGO@Variant\HOLOGO@name\endcsname
        \ltx@secondoftwo
      }%
    }%
  }%
}
%    \end{macrocode}
%    \end{macro}
%    \begin{macro}{\hologoVariantRobust}
%    \begin{macrocode}
\ltx@IfUndefined{protected}{%
  \ltx@IfUndefined{DeclareRobustCommand}{%
    \def\hologoVariantRobust#1#2%
  }{%
    \DeclareRobustCommand*\hologoVariantRobust[2]%
  }%
}{%
  \protected\def\hologoVariantRobust#1#2%
}%
{%
  \begingroup
    \hologoLogoSetup{#1}{variant={#2}}%
    \hologoRobust{#1}%
  \endgroup
}
%    \end{macrocode}
%    \end{macro}
%    \begin{macro}{\HologoVariantRobust}
%    \begin{macrocode}
\ltx@IfUndefined{protected}{%
  \ltx@IfUndefined{DeclareRobustCommand}{%
    \def\HologoVariantRobust#1#2%
  }{%
    \DeclareRobustCommand*\HologoVariantRobust[2]%
  }%
}{%
  \protected\def\HologoVariantRobust#1#2%
}%
{%
  \begingroup
    \hologoLogoSetup{#1}{variant={#2}}%
    \HologoRobust{#1}%
  \endgroup
}
%    \end{macrocode}
%    \end{macro}
%
%    \begin{macro}{\hologorobust}
%    Macro \cs{hologorobust} is only defined for compatibility.
%    Its use is deprecated.
%    \begin{macrocode}
\def\hologorobust{\hologoRobust}
%    \end{macrocode}
%    \end{macro}
%
% \subsection{Helpers}
%
%    \begin{macro}{\HOLOGO@Uppercase}
%    Macro \cs{HOLOGO@Uppercase} is restricted to \cs{uppercase},
%    because \hologo{plainTeX} or \hologo{iniTeX} do not provide
%    \cs{MakeUppercase}.
%    \begin{macrocode}
\def\HOLOGO@Uppercase#1{\uppercase{#1}}
%    \end{macrocode}
%    \end{macro}
%
%    \begin{macro}{\HOLOGO@PdfdocUnicode}
%    \begin{macrocode}
\def\HOLOGO@PdfdocUnicode{%
  \ifx\ifHy@unicode\iftrue
    \expandafter\ltx@secondoftwo
  \else
    \expandafter\ltx@firstoftwo
  \fi
}
%    \end{macrocode}
%    \end{macro}
%
%    \begin{macro}{\HOLOGO@Math}
%    \begin{macrocode}
\def\HOLOGO@MathSetup{%
  \mathsurround0pt\relax
  \HOLOGO@IfExists\f@series{%
    \if b\expandafter\ltx@car\f@series x\@nil
      \csname boldmath\endcsname
   \fi
  }{}%
}
%    \end{macrocode}
%    \end{macro}
%
%    \begin{macro}{\HOLOGO@TempDimen}
%    \begin{macrocode}
\dimendef\HOLOGO@TempDimen=\ltx@zero
%    \end{macrocode}
%    \end{macro}
%    \begin{macro}{\HOLOGO@NegativeKerning}
%    \begin{macrocode}
\def\HOLOGO@NegativeKerning#1{%
  \begingroup
    \HOLOGO@TempDimen=0pt\relax
    \comma@parse@normalized{#1}{%
      \ifdim\HOLOGO@TempDimen=0pt %
        \expandafter\HOLOGO@@NegativeKerning\comma@entry
      \fi
      \ltx@gobble
    }%
    \ifdim\HOLOGO@TempDimen<0pt %
      \kern\HOLOGO@TempDimen
    \fi
  \endgroup
}
%    \end{macrocode}
%    \end{macro}
%    \begin{macro}{\HOLOGO@@NegativeKerning}
%    \begin{macrocode}
\def\HOLOGO@@NegativeKerning#1#2{%
  \setbox\ltx@zero\hbox{#1#2}%
  \HOLOGO@TempDimen=\wd\ltx@zero
  \setbox\ltx@zero\hbox{#1\kern0pt#2}%
  \advance\HOLOGO@TempDimen by -\wd\ltx@zero
}
%    \end{macrocode}
%    \end{macro}
%
%    \begin{macro}{\HOLOGO@SpaceFactor}
%    \begin{macrocode}
\def\HOLOGO@SpaceFactor{%
  \spacefactor1000 %
}
%    \end{macrocode}
%    \end{macro}
%
%    \begin{macro}{\HOLOGO@Span}
%    \begin{macrocode}
\def\HOLOGO@Span#1#2{%
  \HCode{<span class="HoLogo-#1">}%
  #2%
  \HCode{</span>}%
}
%    \end{macrocode}
%    \end{macro}
%
% \subsubsection{Text subscript}
%
%    \begin{macro}{\HOLOGO@SubScript}%
%    \begin{macrocode}
\def\HOLOGO@SubScript#1{%
  \ltx@IfUndefined{textsubscript}{%
    \ltx@IfUndefined{text}{%
      \ltx@mbox{%
        \mathsurround=0pt\relax
        $%
          _{%
            \ltx@IfUndefined{sf@size}{%
              \mathrm{#1}%
            }{%
              \mbox{%
                \fontsize\sf@size{0pt}\selectfont
                #1%
              }%
            }%
          }%
        $%
      }%
    }{%
      \ltx@mbox{%
        \mathsurround=0pt\relax
        $_{\text{#1}}$%
      }%
    }%
  }{%
    \textsubscript{#1}%
  }%
}
%    \end{macrocode}
%    \end{macro}
%
% \subsection{\hologo{TeX} and friends}
%
% \subsubsection{\hologo{TeX}}
%
%    \begin{macro}{\HoLogo@TeX}
%    Source: \hologo{LaTeX} kernel.
%    \begin{macrocode}
\def\HoLogo@TeX#1{%
  T\kern-.1667em\lower.5ex\hbox{E}\kern-.125emX\HOLOGO@SpaceFactor
}
%    \end{macrocode}
%    \end{macro}
%    \begin{macro}{\HoLogoHtml@TeX}
%    \begin{macrocode}
\def\HoLogoHtml@TeX#1{%
  \HoLogoCss@TeX
  \HOLOGO@Span{TeX}{%
    T%
    \HOLOGO@Span{e}{%
      E%
    }%
    X%
  }%
}
%    \end{macrocode}
%    \end{macro}
%    \begin{macro}{\HoLogoCss@TeX}
%    \begin{macrocode}
\def\HoLogoCss@TeX{%
  \Css{%
    span.HoLogo-TeX span.HoLogo-e{%
      position:relative;%
      top:.5ex;%
      margin-left:-.1667em;%
      margin-right:-.125em;%
    }%
  }%
  \Css{%
    a span.HoLogo-TeX span.HoLogo-e{%
      text-decoration:none;%
    }%
  }%
  \global\let\HoLogoCss@TeX\relax
}
%    \end{macrocode}
%    \end{macro}
%
% \subsubsection{\hologo{plainTeX}}
%
%    \begin{macro}{\HoLogo@plainTeX@space}
%    Source: ``The \hologo{TeX}book''
%    \begin{macrocode}
\def\HoLogo@plainTeX@space#1{%
  \HOLOGO@mbox{#1{p}{P}lain}\HOLOGO@space\hologo{TeX}%
}
%    \end{macrocode}
%    \end{macro}
%    \begin{macro}{\HoLogoCs@plainTeX@space}
%    \begin{macrocode}
\def\HoLogoCs@plainTeX@space#1{#1{p}{P}lain TeX}%
%    \end{macrocode}
%    \end{macro}
%    \begin{macro}{\HoLogoBkm@plainTeX@space}
%    \begin{macrocode}
\def\HoLogoBkm@plainTeX@space#1{%
  #1{p}{P}lain \hologo{TeX}%
}
%    \end{macrocode}
%    \end{macro}
%    \begin{macro}{\HoLogoHtml@plainTeX@space}
%    \begin{macrocode}
\def\HoLogoHtml@plainTeX@space#1{%
  #1{p}{P}lain \hologo{TeX}%
}
%    \end{macrocode}
%    \end{macro}
%
%    \begin{macro}{\HoLogo@plainTeX@hyphen}
%    \begin{macrocode}
\def\HoLogo@plainTeX@hyphen#1{%
  \HOLOGO@mbox{#1{p}{P}lain}\HOLOGO@hyphen\hologo{TeX}%
}
%    \end{macrocode}
%    \end{macro}
%    \begin{macro}{\HoLogoCs@plainTeX@hyphen}
%    \begin{macrocode}
\def\HoLogoCs@plainTeX@hyphen#1{#1{p}{P}lain-TeX}
%    \end{macrocode}
%    \end{macro}
%    \begin{macro}{\HoLogoBkm@plainTeX@hyphen}
%    \begin{macrocode}
\def\HoLogoBkm@plainTeX@hyphen#1{%
  #1{p}{P}lain-\hologo{TeX}%
}
%    \end{macrocode}
%    \end{macro}
%    \begin{macro}{\HoLogoHtml@plainTeX@hyphen}
%    \begin{macrocode}
\def\HoLogoHtml@plainTeX@hyphen#1{%
  #1{p}{P}lain-\hologo{TeX}%
}
%    \end{macrocode}
%    \end{macro}
%
%    \begin{macro}{\HoLogo@plainTeX@runtogether}
%    \begin{macrocode}
\def\HoLogo@plainTeX@runtogether#1{%
  \HOLOGO@mbox{#1{p}{P}lain\hologo{TeX}}%
}
%    \end{macrocode}
%    \end{macro}
%    \begin{macro}{\HoLogoCs@plainTeX@runtogether}
%    \begin{macrocode}
\def\HoLogoCs@plainTeX@runtogether#1{#1{p}{P}lainTeX}
%    \end{macrocode}
%    \end{macro}
%    \begin{macro}{\HoLogoBkm@plainTeX@runtogether}
%    \begin{macrocode}
\def\HoLogoBkm@plainTeX@runtogether#1{%
  #1{p}{P}lain\hologo{TeX}%
}
%    \end{macrocode}
%    \end{macro}
%    \begin{macro}{\HoLogoHtml@plainTeX@runtogether}
%    \begin{macrocode}
\def\HoLogoHtml@plainTeX@runtogether#1{%
  #1{p}{P}lain\hologo{TeX}%
}
%    \end{macrocode}
%    \end{macro}
%
%    \begin{macro}{\HoLogo@plainTeX}
%    \begin{macrocode}
\def\HoLogo@plainTeX{\HoLogo@plainTeX@space}
%    \end{macrocode}
%    \end{macro}
%    \begin{macro}{\HoLogoCs@plainTeX}
%    \begin{macrocode}
\def\HoLogoCs@plainTeX{\HoLogoCs@plainTeX@space}
%    \end{macrocode}
%    \end{macro}
%    \begin{macro}{\HoLogoBkm@plainTeX}
%    \begin{macrocode}
\def\HoLogoBkm@plainTeX{\HoLogoBkm@plainTeX@space}
%    \end{macrocode}
%    \end{macro}
%    \begin{macro}{\HoLogoHtml@plainTeX}
%    \begin{macrocode}
\def\HoLogoHtml@plainTeX{\HoLogoHtml@plainTeX@space}
%    \end{macrocode}
%    \end{macro}
%
% \subsubsection{\hologo{LaTeX}}
%
%    Source: \hologo{LaTeX} kernel.
%\begin{quote}
%\begin{verbatim}
%\DeclareRobustCommand{\LaTeX}{%
%  L%
%  \kern-.36em%
%  {%
%    \sbox\z@ T%
%    \vbox to\ht\z@{%
%      \hbox{%
%        \check@mathfonts
%        \fontsize\sf@size\z@
%        \math@fontsfalse
%        \selectfont
%        A%
%      }%
%      \vss
%    }%
%  }%
%  \kern-.15em%
%  \TeX
%}
%\end{verbatim}
%\end{quote}
%
%    \begin{macro}{\HoLogo@La}
%    \begin{macrocode}
\def\HoLogo@La#1{%
  L%
  \kern-.36em%
  \begingroup
    \setbox\ltx@zero\hbox{T}%
    \vbox to\ht\ltx@zero{%
      \hbox{%
        \ltx@ifundefined{check@mathfonts}{%
          \csname sevenrm\endcsname
        }{%
          \check@mathfonts
          \fontsize\sf@size{0pt}%
          \math@fontsfalse\selectfont
        }%
        A%
      }%
      \vss
    }%
  \endgroup
}
%    \end{macrocode}
%    \end{macro}
%
%    \begin{macro}{\HoLogo@LaTeX}
%    Source: \hologo{LaTeX} kernel.
%    \begin{macrocode}
\def\HoLogo@LaTeX#1{%
  \hologo{La}%
  \kern-.15em%
  \hologo{TeX}%
}
%    \end{macrocode}
%    \end{macro}
%    \begin{macro}{\HoLogoHtml@LaTeX}
%    \begin{macrocode}
\def\HoLogoHtml@LaTeX#1{%
  \HoLogoCss@LaTeX
  \HOLOGO@Span{LaTeX}{%
    L%
    \HOLOGO@Span{a}{%
      A%
    }%
    \hologo{TeX}%
  }%
}
%    \end{macrocode}
%    \end{macro}
%    \begin{macro}{\HoLogoCss@LaTeX}
%    \begin{macrocode}
\def\HoLogoCss@LaTeX{%
  \Css{%
    span.HoLogo-LaTeX span.HoLogo-a{%
      position:relative;%
      top:-.5ex;%
      margin-left:-.36em;%
      margin-right:-.15em;%
      font-size:85\%;%
    }%
  }%
  \global\let\HoLogoCss@LaTeX\relax
}
%    \end{macrocode}
%    \end{macro}
%
% \subsubsection{\hologo{(La)TeX}}
%
%    \begin{macro}{\HoLogo@LaTeXTeX}
%    The kerning around the parentheses is taken
%    from package \xpackage{dtklogos} \cite{dtklogos}.
%\begin{quote}
%\begin{verbatim}
%\DeclareRobustCommand{\LaTeXTeX}{%
%  (%
%  \kern-.15em%
%  L%
%  \kern-.36em%
%  {%
%    \sbox\z@ T%
%    \vbox to\ht0{%
%      \hbox{%
%        $\m@th$%
%        \csname S@\f@size\endcsname
%        \fontsize\sf@size\z@
%        \math@fontsfalse
%        \selectfont
%        A%
%      }%
%      \vss
%    }%
%  }%
%  \kern-.2em%
%  )%
%  \kern-.15em%
%  \TeX
%}
%\end{verbatim}
%\end{quote}
%    \begin{macrocode}
\def\HoLogo@LaTeXTeX#1{%
  (%
  \kern-.15em%
  \hologo{La}%
  \kern-.2em%
  )%
  \kern-.15em%
  \hologo{TeX}%
}
%    \end{macrocode}
%    \end{macro}
%    \begin{macro}{\HoLogoBkm@LaTeXTeX}
%    \begin{macrocode}
\def\HoLogoBkm@LaTeXTeX#1{(La)TeX}
%    \end{macrocode}
%    \end{macro}
%
%    \begin{macro}{\HoLogo@(La)TeX}
%    \begin{macrocode}
\expandafter
\let\csname HoLogo@(La)TeX\endcsname\HoLogo@LaTeXTeX
%    \end{macrocode}
%    \end{macro}
%    \begin{macro}{\HoLogoBkm@(La)TeX}
%    \begin{macrocode}
\expandafter
\let\csname HoLogoBkm@(La)TeX\endcsname\HoLogoBkm@LaTeXTeX
%    \end{macrocode}
%    \end{macro}
%    \begin{macro}{\HoLogoHtml@LaTeXTeX}
%    \begin{macrocode}
\def\HoLogoHtml@LaTeXTeX#1{%
  \HoLogoCss@LaTeXTeX
  \HOLOGO@Span{LaTeXTeX}{%
    (%
    \HOLOGO@Span{L}{L}%
    \HOLOGO@Span{a}{A}%
    \HOLOGO@Span{ParenRight}{)}%
    \hologo{TeX}%
  }%
}
%    \end{macrocode}
%    \end{macro}
%    \begin{macro}{\HoLogoHtml@(La)TeX}
%    Kerning after opening parentheses and before closing parentheses
%    is $-0.1$\,em. The original values $-0.15$\,em
%    looked too ugly for a serif font.
%    \begin{macrocode}
\expandafter
\let\csname HoLogoHtml@(La)TeX\endcsname\HoLogoHtml@LaTeXTeX
%    \end{macrocode}
%    \end{macro}
%    \begin{macro}{\HoLogoCss@LaTeXTeX}
%    \begin{macrocode}
\def\HoLogoCss@LaTeXTeX{%
  \Css{%
    span.HoLogo-LaTeXTeX span.HoLogo-L{%
      margin-left:-.1em;%
    }%
  }%
  \Css{%
    span.HoLogo-LaTeXTeX span.HoLogo-a{%
      position:relative;%
      top:-.5ex;%
      margin-left:-.36em;%
      margin-right:-.1em;%
      font-size:85\%;%
    }%
  }%
  \Css{%
    span.HoLogo-LaTeXTeX span.HoLogo-ParenRight{%
      margin-right:-.15em;%
    }%
  }%
  \global\let\HoLogoCss@LaTeXTeX\relax
}
%    \end{macrocode}
%    \end{macro}
%
% \subsubsection{\hologo{LaTeXe}}
%
%    \begin{macro}{\HoLogo@LaTeXe}
%    Source: \hologo{LaTeX} kernel
%    \begin{macrocode}
\def\HoLogo@LaTeXe#1{%
  \hologo{LaTeX}%
  \kern.15em%
  \hbox{%
    \HOLOGO@MathSetup
    2%
    $_{\textstyle\varepsilon}$%
  }%
}
%    \end{macrocode}
%    \end{macro}
%
%    \begin{macro}{\HoLogoCs@LaTeXe}
%    \begin{macrocode}
\ifnum64=`\^^^^0040\relax % test for big chars of LuaTeX/XeTeX
  \catcode`\$=9 %
  \catcode`\&=14 %
\else
  \catcode`\$=14 %
  \catcode`\&=9 %
\fi
\def\HoLogoCs@LaTeXe#1{%
  LaTeX2%
$ \string ^^^^0395%
& e%
}%
\catcode`\$=3 %
\catcode`\&=4 %
%    \end{macrocode}
%    \end{macro}
%
%    \begin{macro}{\HoLogoBkm@LaTeXe}
%    \begin{macrocode}
\def\HoLogoBkm@LaTeXe#1{%
  \hologo{LaTeX}%
  2%
  \HOLOGO@PdfdocUnicode{e}{\textepsilon}%
}
%    \end{macrocode}
%    \end{macro}
%
%    \begin{macro}{\HoLogoHtml@LaTeXe}
%    \begin{macrocode}
\def\HoLogoHtml@LaTeXe#1{%
  \HoLogoCss@LaTeXe
  \HOLOGO@Span{LaTeX2e}{%
    \hologo{LaTeX}%
    \HOLOGO@Span{2}{2}%
    \HOLOGO@Span{e}{%
      \HOLOGO@MathSetup
      \ensuremath{\textstyle\varepsilon}%
    }%
  }%
}
%    \end{macrocode}
%    \end{macro}
%    \begin{macro}{\HoLogoCss@LaTeXe}
%    \begin{macrocode}
\def\HoLogoCss@LaTeXe{%
  \Css{%
    span.HoLogo-LaTeX2e span.HoLogo-2{%
      padding-left:.15em;%
    }%
  }%
  \Css{%
    span.HoLogo-LaTeX2e span.HoLogo-e{%
      position:relative;%
      top:.35ex;%
      text-decoration:none;%
    }%
  }%
  \global\let\HoLogoCss@LaTeXe\relax
}
%    \end{macrocode}
%    \end{macro}
%
%    \begin{macro}{\HoLogo@LaTeX2e}
%    \begin{macrocode}
\expandafter
\let\csname HoLogo@LaTeX2e\endcsname\HoLogo@LaTeXe
%    \end{macrocode}
%    \end{macro}
%    \begin{macro}{\HoLogoCs@LaTeX2e}
%    \begin{macrocode}
\expandafter
\let\csname HoLogoCs@LaTeX2e\endcsname\HoLogoCs@LaTeXe
%    \end{macrocode}
%    \end{macro}
%    \begin{macro}{\HoLogoBkm@LaTeX2e}
%    \begin{macrocode}
\expandafter
\let\csname HoLogoBkm@LaTeX2e\endcsname\HoLogoBkm@LaTeXe
%    \end{macrocode}
%    \end{macro}
%    \begin{macro}{\HoLogoHtml@LaTeX2e}
%    \begin{macrocode}
\expandafter
\let\csname HoLogoHtml@LaTeX2e\endcsname\HoLogoHtml@LaTeXe
%    \end{macrocode}
%    \end{macro}
%
% \subsubsection{\hologo{LaTeX3}}
%
%    \begin{macro}{\HoLogo@LaTeX3}
%    Source: \hologo{LaTeX} kernel
%    \begin{macrocode}
\expandafter\def\csname HoLogo@LaTeX3\endcsname#1{%
  \hologo{LaTeX}%
  3%
}
%    \end{macrocode}
%    \end{macro}
%
%    \begin{macro}{\HoLogoBkm@LaTeX3}
%    \begin{macrocode}
\expandafter\def\csname HoLogoBkm@LaTeX3\endcsname#1{%
  \hologo{LaTeX}%
  3%
}
%    \end{macrocode}
%    \end{macro}
%    \begin{macro}{\HoLogoHtml@LaTeX3}
%    \begin{macrocode}
\expandafter
\let\csname HoLogoHtml@LaTeX3\expandafter\endcsname
\csname HoLogo@LaTeX3\endcsname
%    \end{macrocode}
%    \end{macro}
%
% \subsubsection{\hologo{LaTeXML}}
%
%    \begin{macro}{\HoLogo@LaTeXML}
%    \begin{macrocode}
\def\HoLogo@LaTeXML#1{%
  \HOLOGO@mbox{%
    \hologo{La}%
    \kern-.15em%
    T%
    \kern-.1667em%
    \lower.5ex\hbox{E}%
    \kern-.125em%
    \HoLogoFont@font{LaTeXML}{sc}{xml}%
  }%
}
%    \end{macrocode}
%    \end{macro}
%    \begin{macro}{\HoLogoHtml@pdfLaTeX}
%    \begin{macrocode}
\def\HoLogoHtml@LaTeXML#1{%
  \HOLOGO@Span{LaTeXML}{%
    \HoLogoCss@LaTeX
    \HoLogoCss@TeX
    \HOLOGO@Span{LaTeX}{%
      L%
      \HOLOGO@Span{a}{%
        A%
      }%
    }%
    \HOLOGO@Span{TeX}{%
      T%
      \HOLOGO@Span{e}{%
        E%
      }%
    }%
    \HCode{<span style="font-variant: small-caps;">}%
    xml%
    \HCode{</span>}%
  }%
}
%    \end{macrocode}
%    \end{macro}
%
% \subsubsection{\hologo{eTeX}}
%
%    \begin{macro}{\HoLogo@eTeX}
%    Source: package \xpackage{etex}
%    \begin{macrocode}
\def\HoLogo@eTeX#1{%
  \ltx@mbox{%
    \HOLOGO@MathSetup
    $\varepsilon$%
    -%
    \HOLOGO@NegativeKerning{-T,T-,To}%
    \hologo{TeX}%
  }%
}
%    \end{macrocode}
%    \end{macro}
%    \begin{macro}{\HoLogoCs@eTeX}
%    \begin{macrocode}
\ifnum64=`\^^^^0040\relax % test for big chars of LuaTeX/XeTeX
  \catcode`\$=9 %
  \catcode`\&=14 %
\else
  \catcode`\$=14 %
  \catcode`\&=9 %
\fi
\def\HoLogoCs@eTeX#1{%
$ #1{\string ^^^^0395}{\string ^^^^03b5}%
& #1{e}{E}%
  TeX%
}%
\catcode`\$=3 %
\catcode`\&=4 %
%    \end{macrocode}
%    \end{macro}
%    \begin{macro}{\HoLogoBkm@eTeX}
%    \begin{macrocode}
\def\HoLogoBkm@eTeX#1{%
  \HOLOGO@PdfdocUnicode{#1{e}{E}}{\textepsilon}%
  -%
  \hologo{TeX}%
}
%    \end{macrocode}
%    \end{macro}
%    \begin{macro}{\HoLogoHtml@eTeX}
%    \begin{macrocode}
\def\HoLogoHtml@eTeX#1{%
  \ltx@mbox{%
    \HOLOGO@MathSetup
    $\varepsilon$%
    -%
    \hologo{TeX}%
  }%
}
%    \end{macrocode}
%    \end{macro}
%
% \subsubsection{\hologo{iniTeX}}
%
%    \begin{macro}{\HoLogo@iniTeX}
%    \begin{macrocode}
\def\HoLogo@iniTeX#1{%
  \HOLOGO@mbox{%
    #1{i}{I}ni\hologo{TeX}%
  }%
}
%    \end{macrocode}
%    \end{macro}
%    \begin{macro}{\HoLogoCs@iniTeX}
%    \begin{macrocode}
\def\HoLogoCs@iniTeX#1{#1{i}{I}niTeX}
%    \end{macrocode}
%    \end{macro}
%    \begin{macro}{\HoLogoBkm@iniTeX}
%    \begin{macrocode}
\def\HoLogoBkm@iniTeX#1{%
  #1{i}{I}ni\hologo{TeX}%
}
%    \end{macrocode}
%    \end{macro}
%    \begin{macro}{\HoLogoHtml@iniTeX}
%    \begin{macrocode}
\let\HoLogoHtml@iniTeX\HoLogo@iniTeX
%    \end{macrocode}
%    \end{macro}
%
% \subsubsection{\hologo{virTeX}}
%
%    \begin{macro}{\HoLogo@virTeX}
%    \begin{macrocode}
\def\HoLogo@virTeX#1{%
  \HOLOGO@mbox{%
    #1{v}{V}ir\hologo{TeX}%
  }%
}
%    \end{macrocode}
%    \end{macro}
%    \begin{macro}{\HoLogoCs@virTeX}
%    \begin{macrocode}
\def\HoLogoCs@virTeX#1{#1{v}{V}irTeX}
%    \end{macrocode}
%    \end{macro}
%    \begin{macro}{\HoLogoBkm@virTeX}
%    \begin{macrocode}
\def\HoLogoBkm@virTeX#1{%
  #1{v}{V}ir\hologo{TeX}%
}
%    \end{macrocode}
%    \end{macro}
%    \begin{macro}{\HoLogoHtml@virTeX}
%    \begin{macrocode}
\let\HoLogoHtml@virTeX\HoLogo@virTeX
%    \end{macrocode}
%    \end{macro}
%
% \subsubsection{\hologo{SliTeX}}
%
% \paragraph{Definitions of the three variants.}
%
%    \begin{macro}{\HoLogo@SLiTeX@lift}
%    \begin{macrocode}
\def\HoLogo@SLiTeX@lift#1{%
  \HoLogoFont@font{SliTeX}{rm}{%
    S%
    \kern-.06em%
    L%
    \kern-.18em%
    \raise.32ex\hbox{\HoLogoFont@font{SliTeX}{sc}{i}}%
    \HOLOGO@discretionary
    \kern-.06em%
    \hologo{TeX}%
  }%
}
%    \end{macrocode}
%    \end{macro}
%    \begin{macro}{\HoLogoBkm@SLiTeX@lift}
%    \begin{macrocode}
\def\HoLogoBkm@SLiTeX@lift#1{SLiTeX}
%    \end{macrocode}
%    \end{macro}
%    \begin{macro}{\HoLogoHtml@SLiTeX@lift}
%    \begin{macrocode}
\def\HoLogoHtml@SLiTeX@lift#1{%
  \HoLogoCss@SLiTeX@lift
  \HOLOGO@Span{SLiTeX-lift}{%
    \HoLogoFont@font{SliTeX}{rm}{%
      S%
      \HOLOGO@Span{L}{L}%
      \HOLOGO@Span{i}{i}%
      \hologo{TeX}%
    }%
  }%
}
%    \end{macrocode}
%    \end{macro}
%    \begin{macro}{\HoLogoCss@SLiTeX@lift}
%    \begin{macrocode}
\def\HoLogoCss@SLiTeX@lift{%
  \Css{%
    span.HoLogo-SLiTeX-lift span.HoLogo-L{%
      margin-left:-.06em;%
      margin-right:-.18em;%
    }%
  }%
  \Css{%
    span.HoLogo-SLiTeX-lift span.HoLogo-i{%
      position:relative;%
      top:-.32ex;%
      margin-right:-.06em;%
      font-variant:small-caps;%
    }%
  }%
  \global\let\HoLogoCss@SLiTeX@lift\relax
}
%    \end{macrocode}
%    \end{macro}
%
%    \begin{macro}{\HoLogo@SliTeX@simple}
%    \begin{macrocode}
\def\HoLogo@SliTeX@simple#1{%
  \HoLogoFont@font{SliTeX}{rm}{%
    \ltx@mbox{%
      \HoLogoFont@font{SliTeX}{sc}{Sli}%
    }%
    \HOLOGO@discretionary
    \hologo{TeX}%
  }%
}
%    \end{macrocode}
%    \end{macro}
%    \begin{macro}{\HoLogoBkm@SliTeX@simple}
%    \begin{macrocode}
\def\HoLogoBkm@SliTeX@simple#1{SliTeX}
%    \end{macrocode}
%    \end{macro}
%    \begin{macro}{\HoLogoHtml@SliTeX@simple}
%    \begin{macrocode}
\let\HoLogoHtml@SliTeX@simple\HoLogo@SliTeX@simple
%    \end{macrocode}
%    \end{macro}
%
%    \begin{macro}{\HoLogo@SliTeX@narrow}
%    \begin{macrocode}
\def\HoLogo@SliTeX@narrow#1{%
  \HoLogoFont@font{SliTeX}{rm}{%
    \ltx@mbox{%
      S%
      \kern-.06em%
      \HoLogoFont@font{SliTeX}{sc}{%
        l%
        \kern-.035em%
        i%
      }%
    }%
    \HOLOGO@discretionary
    \kern-.06em%
    \hologo{TeX}%
  }%
}
%    \end{macrocode}
%    \end{macro}
%    \begin{macro}{\HoLogoBkm@SliTeX@narrow}
%    \begin{macrocode}
\def\HoLogoBkm@SliTeX@narrow#1{SliTeX}
%    \end{macrocode}
%    \end{macro}
%    \begin{macro}{\HoLogoHtml@SliTeX@narrow}
%    \begin{macrocode}
\def\HoLogoHtml@SliTeX@narrow#1{%
  \HoLogoCss@SliTeX@narrow
  \HOLOGO@Span{SliTeX-narrow}{%
    \HoLogoFont@font{SliTeX}{rm}{%
      S%
        \HOLOGO@Span{l}{l}%
        \HOLOGO@Span{i}{i}%
      \hologo{TeX}%
    }%
  }%
}
%    \end{macrocode}
%    \end{macro}
%    \begin{macro}{\HoLogoCss@SliTeX@narrow}
%    \begin{macrocode}
\def\HoLogoCss@SliTeX@narrow{%
  \Css{%
    span.HoLogo-SliTeX-narrow span.HoLogo-l{%
      margin-left:-.06em;%
      margin-right:-.035em;%
      font-variant:small-caps;%
    }%
  }%
  \Css{%
    span.HoLogo-SliTeX-narrow span.HoLogo-i{%
      margin-right:-.06em;%
      font-variant:small-caps;%
    }%
  }%
  \global\let\HoLogoCss@SliTeX@narrow\relax
}
%    \end{macrocode}
%    \end{macro}
%
% \paragraph{Macro set completion.}
%
%    \begin{macro}{\HoLogo@SLiTeX@simple}
%    \begin{macrocode}
\def\HoLogo@SLiTeX@simple{\HoLogo@SliTeX@simple}
%    \end{macrocode}
%    \end{macro}
%    \begin{macro}{\HoLogoBkm@SLiTeX@simple}
%    \begin{macrocode}
\def\HoLogoBkm@SLiTeX@simple{\HoLogoBkm@SliTeX@simple}
%    \end{macrocode}
%    \end{macro}
%    \begin{macro}{\HoLogoHtml@SLiTeX@simple}
%    \begin{macrocode}
\def\HoLogoHtml@SLiTeX@simple{\HoLogoHtml@SliTeX@simple}
%    \end{macrocode}
%    \end{macro}
%
%    \begin{macro}{\HoLogo@SLiTeX@narrow}
%    \begin{macrocode}
\def\HoLogo@SLiTeX@narrow{\HoLogo@SliTeX@narrow}
%    \end{macrocode}
%    \end{macro}
%    \begin{macro}{\HoLogoBkm@SLiTeX@narrow}
%    \begin{macrocode}
\def\HoLogoBkm@SLiTeX@narrow{\HoLogoBkm@SliTeX@narrow}
%    \end{macrocode}
%    \end{macro}
%    \begin{macro}{\HoLogoHtml@SLiTeX@narrow}
%    \begin{macrocode}
\def\HoLogoHtml@SLiTeX@narrow{\HoLogoHtml@SliTeX@narrow}
%    \end{macrocode}
%    \end{macro}
%
%    \begin{macro}{\HoLogo@SliTeX@lift}
%    \begin{macrocode}
\def\HoLogo@SliTeX@lift{\HoLogo@SLiTeX@lift}
%    \end{macrocode}
%    \end{macro}
%    \begin{macro}{\HoLogoBkm@SliTeX@lift}
%    \begin{macrocode}
\def\HoLogoBkm@SliTeX@lift{\HoLogoBkm@SLiTeX@lift}
%    \end{macrocode}
%    \end{macro}
%    \begin{macro}{\HoLogoHtml@SliTeX@lift}
%    \begin{macrocode}
\def\HoLogoHtml@SliTeX@lift{\HoLogoHtml@SLiTeX@lift}
%    \end{macrocode}
%    \end{macro}
%
% \paragraph{Defaults.}
%
%    \begin{macro}{\HoLogo@SLiTeX}
%    \begin{macrocode}
\def\HoLogo@SLiTeX{\HoLogo@SLiTeX@lift}
%    \end{macrocode}
%    \end{macro}
%    \begin{macro}{\HoLogoBkm@SLiTeX}
%    \begin{macrocode}
\def\HoLogoBkm@SLiTeX{\HoLogoBkm@SLiTeX@lift}
%    \end{macrocode}
%    \end{macro}
%    \begin{macro}{\HoLogoHtml@SLiTeX}
%    \begin{macrocode}
\def\HoLogoHtml@SLiTeX{\HoLogoHtml@SLiTeX@lift}
%    \end{macrocode}
%    \end{macro}
%
%    \begin{macro}{\HoLogo@SliTeX}
%    \begin{macrocode}
\def\HoLogo@SliTeX{\HoLogo@SliTeX@narrow}
%    \end{macrocode}
%    \end{macro}
%    \begin{macro}{\HoLogoBkm@SliTeX}
%    \begin{macrocode}
\def\HoLogoBkm@SliTeX{\HoLogoBkm@SliTeX@narrow}
%    \end{macrocode}
%    \end{macro}
%    \begin{macro}{\HoLogoHtml@SliTeX}
%    \begin{macrocode}
\def\HoLogoHtml@SliTeX{\HoLogoHtml@SliTeX@narrow}
%    \end{macrocode}
%    \end{macro}
%
% \subsubsection{\hologo{LuaTeX}}
%
%    \begin{macro}{\HoLogo@LuaTeX}
%    The kerning is an idea of Hans Hagen, see mailing list
%    `luatex at tug dot org' in March 2010.
%    \begin{macrocode}
\def\HoLogo@LuaTeX#1{%
  \HOLOGO@mbox{%
    Lua%
    \HOLOGO@NegativeKerning{aT,oT,To}%
    \hologo{TeX}%
  }%
}
%    \end{macrocode}
%    \end{macro}
%    \begin{macro}{\HoLogoHtml@LuaTeX}
%    \begin{macrocode}
\let\HoLogoHtml@LuaTeX\HoLogo@LuaTeX
%    \end{macrocode}
%    \end{macro}
%
% \subsubsection{\hologo{LuaLaTeX}}
%
%    \begin{macro}{\HoLogo@LuaLaTeX}
%    \begin{macrocode}
\def\HoLogo@LuaLaTeX#1{%
  \HOLOGO@mbox{%
    Lua%
    \hologo{LaTeX}%
  }%
}
%    \end{macrocode}
%    \end{macro}
%    \begin{macro}{\HoLogoHtml@LuaLaTeX}
%    \begin{macrocode}
\let\HoLogoHtml@LuaLaTeX\HoLogo@LuaLaTeX
%    \end{macrocode}
%    \end{macro}
%
% \subsubsection{\hologo{XeTeX}, \hologo{XeLaTeX}}
%
%    \begin{macro}{\HOLOGO@IfCharExists}
%    \begin{macrocode}
\ifluatex
  \ifnum\luatexversion<36 %
  \else
    \def\HOLOGO@IfCharExists#1{%
      \ifnum
        \directlua{%
           if luaotfload and luaotfload.aux then
             if luaotfload.aux.font_has_glyph(%
                    font.current(), \number#1) then % 	 
	       tex.print("1") % 	 
	     end % 	 
	   elseif font and font.fonts and font.current then %
            local f = font.fonts[font.current()]%
            if f.characters and f.characters[\number#1] then %
              tex.print("1")%
            end %
          end%
        }0=\ltx@zero
        \expandafter\ltx@secondoftwo
      \else
        \expandafter\ltx@firstoftwo
      \fi
    }%
  \fi
\fi
\ltx@IfUndefined{HOLOGO@IfCharExists}{%
  \def\HOLOGO@@IfCharExists#1{%
    \begingroup
      \tracinglostchars=\ltx@zero
      \setbox\ltx@zero=\hbox{%
        \kern7sp\char#1\relax
        \ifnum\lastkern>\ltx@zero
          \expandafter\aftergroup\csname iffalse\endcsname
        \else
          \expandafter\aftergroup\csname iftrue\endcsname
        \fi
      }%
      % \if{true|false} from \aftergroup
      \endgroup
      \expandafter\ltx@firstoftwo
    \else
      \endgroup
      \expandafter\ltx@secondoftwo
    \fi
  }%
  \ifxetex
    \ltx@IfUndefined{XeTeXfonttype}{}{%
      \ltx@IfUndefined{XeTeXcharglyph}{}{%
        \def\HOLOGO@IfCharExists#1{%
          \ifnum\XeTeXfonttype\font>\ltx@zero
            \expandafter\ltx@firstofthree
          \else
            \expandafter\ltx@gobble
          \fi
          {%
            \ifnum\XeTeXcharglyph#1>\ltx@zero
              \expandafter\ltx@firstoftwo
            \else
              \expandafter\ltx@secondoftwo
            \fi
          }%
          \HOLOGO@@IfCharExists{#1}%
        }%
      }%
    }%
  \fi
}{}
\ltx@ifundefined{HOLOGO@IfCharExists}{%
  \ifnum64=`\^^^^0040\relax % test for big chars of LuaTeX/XeTeX
    \let\HOLOGO@IfCharExists\HOLOGO@@IfCharExists
  \else
    \def\HOLOGO@IfCharExists#1{%
      \ifnum#1>255 %
        \expandafter\ltx@fourthoffour
      \fi
      \HOLOGO@@IfCharExists{#1}%
    }%
  \fi
}{}
%    \end{macrocode}
%    \end{macro}
%
%    \begin{macro}{\HoLogo@Xe}
%    Source: package \xpackage{dtklogos}
%    \begin{macrocode}
\def\HoLogo@Xe#1{%
  X%
  \kern-.1em\relax
  \HOLOGO@IfCharExists{"018E}{%
    \lower.5ex\hbox{\char"018E}%
  }{%
    \chardef\HOLOGO@choice=\ltx@zero
    \ifdim\fontdimen\ltx@one\font>0pt %
      \ltx@IfUndefined{rotatebox}{%
        \ltx@IfUndefined{pgftext}{%
          \ltx@IfUndefined{psscalebox}{%
            \ltx@IfUndefined{HOLOGO@ScaleBox@\hologoDriver}{%
            }{%
              \chardef\HOLOGO@choice=4 %
            }%
          }{%
            \chardef\HOLOGO@choice=3 %
          }%
        }{%
          \chardef\HOLOGO@choice=2 %
        }%
      }{%
        \chardef\HOLOGO@choice=1 %
      }%
      \ifcase\HOLOGO@choice
        \HOLOGO@WarningUnsupportedDriver{Xe}%
        e%
      \or % 1: \rotatebox
        \begingroup
          \setbox\ltx@zero\hbox{\rotatebox{180}{E}}%
          \ltx@LocDimenA=\dp\ltx@zero
          \advance\ltx@LocDimenA by -.5ex\relax
          \raise\ltx@LocDimenA\box\ltx@zero
        \endgroup
      \or % 2: \pgftext
        \lower.5ex\hbox{%
          \pgfpicture
            \pgftext[rotate=180]{E}%
          \endpgfpicture
        }%
      \or % 3: \psscalebox
        \begingroup
          \setbox\ltx@zero\hbox{\psscalebox{-1 -1}{E}}%
          \ltx@LocDimenA=\dp\ltx@zero
          \advance\ltx@LocDimenA by -.5ex\relax
          \raise\ltx@LocDimenA\box\ltx@zero
        \endgroup
      \or % 4: \HOLOGO@PointReflectBox
        \lower.5ex\hbox{\HOLOGO@PointReflectBox{E}}%
      \else
        \@PackageError{hologo}{Internal error (choice/it}\@ehc
      \fi
    \else
      \ltx@IfUndefined{reflectbox}{%
        \ltx@IfUndefined{pgftext}{%
          \ltx@IfUndefined{psscalebox}{%
            \ltx@IfUndefined{HOLOGO@ScaleBox@\hologoDriver}{%
            }{%
              \chardef\HOLOGO@choice=4 %
            }%
          }{%
            \chardef\HOLOGO@choice=3 %
          }%
        }{%
          \chardef\HOLOGO@choice=2 %
        }%
      }{%
        \chardef\HOLOGO@choice=1 %
      }%
      \ifcase\HOLOGO@choice
        \HOLOGO@WarningUnsupportedDriver{Xe}%
        e%
      \or % 1: reflectbox
        \lower.5ex\hbox{%
          \reflectbox{E}%
        }%
      \or % 2: \pgftext
        \lower.5ex\hbox{%
          \pgfpicture
            \pgftransformxscale{-1}%
            \pgftext{E}%
          \endpgfpicture
        }%
      \or % 3: \psscalebox
        \lower.5ex\hbox{%
          \psscalebox{-1 1}{E}%
        }%
      \or % 4: \HOLOGO@Reflectbox
        \lower.5ex\hbox{%
          \HOLOGO@ReflectBox{E}%
        }%
      \else
        \@PackageError{hologo}{Internal error (choice/up)}\@ehc
      \fi
    \fi
  }%
}
%    \end{macrocode}
%    \end{macro}
%    \begin{macro}{\HoLogoHtml@Xe}
%    \begin{macrocode}
\def\HoLogoHtml@Xe#1{%
  \HoLogoCss@Xe
  \HOLOGO@Span{Xe}{%
    X%
    \HOLOGO@Span{e}{%
      \HCode{&\ltx@hashchar x018e;}%
    }%
  }%
}
%    \end{macrocode}
%    \end{macro}
%    \begin{macro}{\HoLogoCss@Xe}
%    \begin{macrocode}
\def\HoLogoCss@Xe{%
  \Css{%
    span.HoLogo-Xe span.HoLogo-e{%
      position:relative;%
      top:.5ex;%
      left-margin:-.1em;%
    }%
  }%
  \global\let\HoLogoCss@Xe\relax
}
%    \end{macrocode}
%    \end{macro}
%
%    \begin{macro}{\HoLogo@XeTeX}
%    \begin{macrocode}
\def\HoLogo@XeTeX#1{%
  \hologo{Xe}%
  \kern-.15em\relax
  \hologo{TeX}%
}
%    \end{macrocode}
%    \end{macro}
%
%    \begin{macro}{\HoLogoHtml@XeTeX}
%    \begin{macrocode}
\def\HoLogoHtml@XeTeX#1{%
  \HoLogoCss@XeTeX
  \HOLOGO@Span{XeTeX}{%
    \hologo{Xe}%
    \hologo{TeX}%
  }%
}
%    \end{macrocode}
%    \end{macro}
%    \begin{macro}{\HoLogoCss@XeTeX}
%    \begin{macrocode}
\def\HoLogoCss@XeTeX{%
  \Css{%
    span.HoLogo-XeTeX span.HoLogo-TeX{%
      margin-left:-.15em;%
    }%
  }%
  \global\let\HoLogoCss@XeTeX\relax
}
%    \end{macrocode}
%    \end{macro}
%
%    \begin{macro}{\HoLogo@XeLaTeX}
%    \begin{macrocode}
\def\HoLogo@XeLaTeX#1{%
  \hologo{Xe}%
  \kern-.13em%
  \hologo{LaTeX}%
}
%    \end{macrocode}
%    \end{macro}
%    \begin{macro}{\HoLogoHtml@XeLaTeX}
%    \begin{macrocode}
\def\HoLogoHtml@XeLaTeX#1{%
  \HoLogoCss@XeLaTeX
  \HOLOGO@Span{XeLaTeX}{%
    \hologo{Xe}%
    \hologo{LaTeX}%
  }%
}
%    \end{macrocode}
%    \end{macro}
%    \begin{macro}{\HoLogoCss@XeLaTeX}
%    \begin{macrocode}
\def\HoLogoCss@XeLaTeX{%
  \Css{%
    span.HoLogo-XeLaTeX span.HoLogo-Xe{%
      margin-right:-.13em;%
    }%
  }%
  \global\let\HoLogoCss@XeLaTeX\relax
}
%    \end{macrocode}
%    \end{macro}
%
% \subsubsection{\hologo{pdfTeX}, \hologo{pdfLaTeX}}
%
%    \begin{macro}{\HoLogo@pdfTeX}
%    \begin{macrocode}
\def\HoLogo@pdfTeX#1{%
  \HOLOGO@mbox{%
    #1{p}{P}df\hologo{TeX}%
  }%
}
%    \end{macrocode}
%    \end{macro}
%    \begin{macro}{\HoLogoCs@pdfTeX}
%    \begin{macrocode}
\def\HoLogoCs@pdfTeX#1{#1{p}{P}dfTeX}
%    \end{macrocode}
%    \end{macro}
%    \begin{macro}{\HoLogoBkm@pdfTeX}
%    \begin{macrocode}
\def\HoLogoBkm@pdfTeX#1{%
  #1{p}{P}df\hologo{TeX}%
}
%    \end{macrocode}
%    \end{macro}
%    \begin{macro}{\HoLogoHtml@pdfTeX}
%    \begin{macrocode}
\let\HoLogoHtml@pdfTeX\HoLogo@pdfTeX
%    \end{macrocode}
%    \end{macro}
%
%    \begin{macro}{\HoLogo@pdfLaTeX}
%    \begin{macrocode}
\def\HoLogo@pdfLaTeX#1{%
  \HOLOGO@mbox{%
    #1{p}{P}df\hologo{LaTeX}%
  }%
}
%    \end{macrocode}
%    \end{macro}
%    \begin{macro}{\HoLogoCs@pdfLaTeX}
%    \begin{macrocode}
\def\HoLogoCs@pdfLaTeX#1{#1{p}{P}dfLaTeX}
%    \end{macrocode}
%    \end{macro}
%    \begin{macro}{\HoLogoBkm@pdfLaTeX}
%    \begin{macrocode}
\def\HoLogoBkm@pdfLaTeX#1{%
  #1{p}{P}df\hologo{LaTeX}%
}
%    \end{macrocode}
%    \end{macro}
%    \begin{macro}{\HoLogoHtml@pdfLaTeX}
%    \begin{macrocode}
\let\HoLogoHtml@pdfLaTeX\HoLogo@pdfLaTeX
%    \end{macrocode}
%    \end{macro}
%
% \subsubsection{\hologo{VTeX}}
%
%    \begin{macro}{\HoLogo@VTeX}
%    \begin{macrocode}
\def\HoLogo@VTeX#1{%
  \HOLOGO@mbox{%
    V\hologo{TeX}%
  }%
}
%    \end{macrocode}
%    \end{macro}
%    \begin{macro}{\HoLogoHtml@VTeX}
%    \begin{macrocode}
\let\HoLogoHtml@VTeX\HoLogo@VTeX
%    \end{macrocode}
%    \end{macro}
%
% \subsubsection{\hologo{AmS}, \dots}
%
%    Source: class \xclass{amsdtx}
%
%    \begin{macro}{\HoLogo@AmS}
%    \begin{macrocode}
\def\HoLogo@AmS#1{%
  \HoLogoFont@font{AmS}{sy}{%
    A%
    \kern-.1667em%
    \lower.5ex\hbox{M}%
    \kern-.125em%
    S%
  }%
}
%    \end{macrocode}
%    \end{macro}
%    \begin{macro}{\HoLogoBkm@AmS}
%    \begin{macrocode}
\def\HoLogoBkm@AmS#1{AmS}
%    \end{macrocode}
%    \end{macro}
%    \begin{macro}{\HoLogoHtml@AmS}
%    \begin{macrocode}
\def\HoLogoHtml@AmS#1{%
  \HoLogoCss@AmS
%  \HoLogoFont@font{AmS}{sy}{%
    \HOLOGO@Span{AmS}{%
      A%
      \HOLOGO@Span{M}{M}%
      S%
    }%
%   }%
}
%    \end{macrocode}
%    \end{macro}
%    \begin{macro}{\HoLogoCss@AmS}
%    \begin{macrocode}
\def\HoLogoCss@AmS{%
  \Css{%
    span.HoLogo-AmS span.HoLogo-M{%
      position:relative;%
      top:.5ex;%
      margin-left:-.1667em;%
      margin-right:-.125em;%
      text-decoration:none;%
    }%
  }%
  \global\let\HoLogoCss@AmS\relax
}
%    \end{macrocode}
%    \end{macro}
%
%    \begin{macro}{\HoLogo@AmSTeX}
%    \begin{macrocode}
\def\HoLogo@AmSTeX#1{%
  \hologo{AmS}%
  \HOLOGO@hyphen
  \hologo{TeX}%
}
%    \end{macrocode}
%    \end{macro}
%    \begin{macro}{\HoLogoBkm@AmSTeX}
%    \begin{macrocode}
\def\HoLogoBkm@AmSTeX#1{AmS-TeX}%
%    \end{macrocode}
%    \end{macro}
%    \begin{macro}{\HoLogoHtml@AmSTeX}
%    \begin{macrocode}
\let\HoLogoHtml@AmSTeX\HoLogo@AmSTeX
%    \end{macrocode}
%    \end{macro}
%
%    \begin{macro}{\HoLogo@AmSLaTeX}
%    \begin{macrocode}
\def\HoLogo@AmSLaTeX#1{%
  \hologo{AmS}%
  \HOLOGO@hyphen
  \hologo{LaTeX}%
}
%    \end{macrocode}
%    \end{macro}
%    \begin{macro}{\HoLogoBkm@AmSLaTeX}
%    \begin{macrocode}
\def\HoLogoBkm@AmSLaTeX#1{AmS-LaTeX}%
%    \end{macrocode}
%    \end{macro}
%    \begin{macro}{\HoLogoHtml@AmSLaTeX}
%    \begin{macrocode}
\let\HoLogoHtml@AmSLaTeX\HoLogo@AmSLaTeX
%    \end{macrocode}
%    \end{macro}
%
% \subsubsection{\hologo{BibTeX}}
%
%    \begin{macro}{\HoLogo@BibTeX@sc}
%    A definition of \hologo{BibTeX} is provided in
%    the documentation source for the manual of \hologo{BibTeX}
%    \cite{btxdoc}.
%\begin{quote}
%\begin{verbatim}
%\def\BibTeX{%
%  {%
%    \rm
%    B%
%    \kern-.05em%
%    {%
%      \sc
%      i%
%      \kern-.025em %
%      b%
%    }%
%    \kern-.08em
%    T%
%    \kern-.1667em%
%    \lower.7ex\hbox{E}%
%    \kern-.125em%
%    X%
%  }%
%}
%\end{verbatim}
%\end{quote}
%    \begin{macrocode}
\def\HoLogo@BibTeX@sc#1{%
  B%
  \kern-.05em%
  \HoLogoFont@font{BibTeX}{sc}{%
    i%
    \kern-.025em%
    b%
  }%
  \HOLOGO@discretionary
  \kern-.08em%
  \hologo{TeX}%
}
%    \end{macrocode}
%    \end{macro}
%    \begin{macro}{\HoLogoHtml@BibTeX@sc}
%    \begin{macrocode}
\def\HoLogoHtml@BibTeX@sc#1{%
  \HoLogoCss@BibTeX@sc
  \HOLOGO@Span{BibTeX-sc}{%
    B%
    \HOLOGO@Span{i}{i}%
    \HOLOGO@Span{b}{b}%
    \hologo{TeX}%
  }%
}
%    \end{macrocode}
%    \end{macro}
%    \begin{macro}{\HoLogoCss@BibTeX@sc}
%    \begin{macrocode}
\def\HoLogoCss@BibTeX@sc{%
  \Css{%
    span.HoLogo-BibTeX-sc span.HoLogo-i{%
      margin-left:-.05em;%
      margin-right:-.025em;%
      font-variant:small-caps;%
    }%
  }%
  \Css{%
    span.HoLogo-BibTeX-sc span.HoLogo-b{%
      margin-right:-.08em;%
      font-variant:small-caps;%
    }%
  }%
  \global\let\HoLogoCss@BibTeX@sc\relax
}
%    \end{macrocode}
%    \end{macro}
%
%    \begin{macro}{\HoLogo@BibTeX@sf}
%    Variant \xoption{sf} avoids trouble with unavailable
%    small caps fonts (e.g., bold versions of Computer Modern or
%    Latin Modern). The definition is taken from
%    package \xpackage{dtklogos} \cite{dtklogos}.
%\begin{quote}
%\begin{verbatim}
%\DeclareRobustCommand{\BibTeX}{%
%  B%
%  \kern-.05em%
%  \hbox{%
%    $\m@th$% %% force math size calculations
%    \csname S@\f@size\endcsname
%    \fontsize\sf@size\z@
%    \math@fontsfalse
%    \selectfont
%    I%
%    \kern-.025em%
%    B
%  }%
%  \kern-.08em%
%  \-%
%  \TeX
%}
%\end{verbatim}
%\end{quote}
%    \begin{macrocode}
\def\HoLogo@BibTeX@sf#1{%
  B%
  \kern-.05em%
  \HoLogoFont@font{BibTeX}{bibsf}{%
    I%
    \kern-.025em%
    B%
  }%
  \HOLOGO@discretionary
  \kern-.08em%
  \hologo{TeX}%
}
%    \end{macrocode}
%    \end{macro}
%    \begin{macro}{\HoLogoHtml@BibTeX@sf}
%    \begin{macrocode}
\def\HoLogoHtml@BibTeX@sf#1{%
  \HoLogoCss@BibTeX@sf
  \HOLOGO@Span{BibTeX-sf}{%
    B%
    \HoLogoFont@font{BibTeX}{bibsf}{%
      \HOLOGO@Span{i}{I}%
      B%
    }%
    \hologo{TeX}%
  }%
}
%    \end{macrocode}
%    \end{macro}
%    \begin{macro}{\HoLogoCss@BibTeX@sf}
%    \begin{macrocode}
\def\HoLogoCss@BibTeX@sf{%
  \Css{%
    span.HoLogo-BibTeX-sf span.HoLogo-i{%
      margin-left:-.05em;%
      margin-right:-.025em;%
    }%
  }%
  \Css{%
    span.HoLogo-BibTeX-sf span.HoLogo-TeX{%
      margin-left:-.08em;%
    }%
  }%
  \global\let\HoLogoCss@BibTeX@sf\relax
}
%    \end{macrocode}
%    \end{macro}
%
%    \begin{macro}{\HoLogo@BibTeX}
%    \begin{macrocode}
\def\HoLogo@BibTeX{\HoLogo@BibTeX@sf}
%    \end{macrocode}
%    \end{macro}
%    \begin{macro}{\HoLogoHtml@BibTeX}
%    \begin{macrocode}
\def\HoLogoHtml@BibTeX{\HoLogoHtml@BibTeX@sf}
%    \end{macrocode}
%    \end{macro}
%
% \subsubsection{\hologo{BibTeX8}}
%
%    \begin{macro}{\HoLogo@BibTeX8}
%    \begin{macrocode}
\expandafter\def\csname HoLogo@BibTeX8\endcsname#1{%
  \hologo{BibTeX}%
  8%
}
%    \end{macrocode}
%    \end{macro}
%
%    \begin{macro}{\HoLogoBkm@BibTeX8}
%    \begin{macrocode}
\expandafter\def\csname HoLogoBkm@BibTeX8\endcsname#1{%
  \hologo{BibTeX}%
  8%
}
%    \end{macrocode}
%    \end{macro}
%    \begin{macro}{\HoLogoHtml@BibTeX8}
%    \begin{macrocode}
\expandafter
\let\csname HoLogoHtml@BibTeX8\expandafter\endcsname
\csname HoLogo@BibTeX8\endcsname
%    \end{macrocode}
%    \end{macro}
%
% \subsubsection{\hologo{ConTeXt}}
%
%    \begin{macro}{\HoLogo@ConTeXt@simple}
%    \begin{macrocode}
\def\HoLogo@ConTeXt@simple#1{%
  \HOLOGO@mbox{Con}%
  \HOLOGO@discretionary
  \HOLOGO@mbox{\hologo{TeX}t}%
}
%    \end{macrocode}
%    \end{macro}
%    \begin{macro}{\HoLogoHtml@ConTeXt@simple}
%    \begin{macrocode}
\let\HoLogoHtml@ConTeXt@simple\HoLogo@ConTeXt@simple
%    \end{macrocode}
%    \end{macro}
%
%    \begin{macro}{\HoLogo@ConTeXt@narrow}
%    This definition of logo \hologo{ConTeXt} with variant \xoption{narrow}
%    comes from TUGboat's class \xclass{ltugboat} (version 2010/11/15 v2.8).
%    \begin{macrocode}
\def\HoLogo@ConTeXt@narrow#1{%
  \HOLOGO@mbox{C\kern-.0333emon}%
  \HOLOGO@discretionary
  \kern-.0667em%
  \HOLOGO@mbox{\hologo{TeX}\kern-.0333emt}%
}
%    \end{macrocode}
%    \end{macro}
%    \begin{macro}{\HoLogoHtml@ConTeXt@narrow}
%    \begin{macrocode}
\def\HoLogoHtml@ConTeXt@narrow#1{%
  \HoLogoCss@ConTeXt@narrow
  \HOLOGO@Span{ConTeXt-narrow}{%
    \HOLOGO@Span{C}{C}%
    on%
    \hologo{TeX}%
    t%
  }%
}
%    \end{macrocode}
%    \end{macro}
%    \begin{macro}{\HoLogoCss@ConTeXt@narrow}
%    \begin{macrocode}
\def\HoLogoCss@ConTeXt@narrow{%
  \Css{%
    span.HoLogo-ConTeXt-narrow span.HoLogo-C{%
      margin-left:-.0333em;%
    }%
  }%
  \Css{%
    span.HoLogo-ConTeXt-narrow span.HoLogo-TeX{%
      margin-left:-.0667em;%
      margin-right:-.0333em;%
    }%
  }%
  \global\let\HoLogoCss@ConTeXt@narrow\relax
}
%    \end{macrocode}
%    \end{macro}
%
%    \begin{macro}{\HoLogo@ConTeXt}
%    \begin{macrocode}
\def\HoLogo@ConTeXt{\HoLogo@ConTeXt@narrow}
%    \end{macrocode}
%    \end{macro}
%    \begin{macro}{\HoLogoHtml@ConTeXt}
%    \begin{macrocode}
\def\HoLogoHtml@ConTeXt{\HoLogoHtml@ConTeXt@narrow}
%    \end{macrocode}
%    \end{macro}
%
% \subsubsection{\hologo{emTeX}}
%
%    \begin{macro}{\HoLogo@emTeX}
%    \begin{macrocode}
\def\HoLogo@emTeX#1{%
  \HOLOGO@mbox{#1{e}{E}m}%
  \HOLOGO@discretionary
  \hologo{TeX}%
}
%    \end{macrocode}
%    \end{macro}
%    \begin{macro}{\HoLogoCs@emTeX}
%    \begin{macrocode}
\def\HoLogoCs@emTeX#1{#1{e}{E}mTeX}%
%    \end{macrocode}
%    \end{macro}
%    \begin{macro}{\HoLogoBkm@emTeX}
%    \begin{macrocode}
\def\HoLogoBkm@emTeX#1{%
  #1{e}{E}m\hologo{TeX}%
}
%    \end{macrocode}
%    \end{macro}
%    \begin{macro}{\HoLogoHtml@emTeX}
%    \begin{macrocode}
\let\HoLogoHtml@emTeX\HoLogo@emTeX
%    \end{macrocode}
%    \end{macro}
%
% \subsubsection{\hologo{ExTeX}}
%
%    \begin{macro}{\HoLogo@ExTeX}
%    The definition is taken from the FAQ of the
%    project \hologo{ExTeX}
%    \cite{ExTeX-FAQ}.
%\begin{quote}
%\begin{verbatim}
%\def\ExTeX{%
%  \textrm{% Logo always with serifs
%    \ensuremath{%
%      \textstyle
%      \varepsilon_{%
%        \kern-0.15em%
%        \mathcal{X}%
%      }%
%    }%
%    \kern-.15em%
%    \TeX
%  }%
%}
%\end{verbatim}
%\end{quote}
%    \begin{macrocode}
\def\HoLogo@ExTeX#1{%
  \HoLogoFont@font{ExTeX}{rm}{%
    \ltx@mbox{%
      \HOLOGO@MathSetup
      $%
        \textstyle
        \varepsilon_{%
          \kern-0.15em%
          \HoLogoFont@font{ExTeX}{sy}{X}%
        }%
      $%
    }%
    \HOLOGO@discretionary
    \kern-.15em%
    \hologo{TeX}%
  }%
}
%    \end{macrocode}
%    \end{macro}
%    \begin{macro}{\HoLogoHtml@ExTeX}
%    \begin{macrocode}
\def\HoLogoHtml@ExTeX#1{%
  \HoLogoCss@ExTeX
  \HoLogoFont@font{ExTeX}{rm}{%
    \HOLOGO@Span{ExTeX}{%
      \ltx@mbox{%
        \HOLOGO@MathSetup
        $\textstyle\varepsilon$%
        \HOLOGO@Span{X}{$\textstyle\chi$}%
        \hologo{TeX}%
      }%
    }%
  }%
}
%    \end{macrocode}
%    \end{macro}
%    \begin{macro}{\HoLogoBkm@ExTeX}
%    \begin{macrocode}
\def\HoLogoBkm@ExTeX#1{%
  \HOLOGO@PdfdocUnicode{#1{e}{E}x}{\textepsilon\textchi}%
  \hologo{TeX}%
}
%    \end{macrocode}
%    \end{macro}
%    \begin{macro}{\HoLogoCss@ExTeX}
%    \begin{macrocode}
\def\HoLogoCss@ExTeX{%
  \Css{%
    span.HoLogo-ExTeX{%
      font-family:serif;%
    }%
  }%
  \Css{%
    span.HoLogo-ExTeX span.HoLogo-TeX{%
      margin-left:-.15em;%
    }%
  }%
  \global\let\HoLogoCss@ExTeX\relax
}
%    \end{macrocode}
%    \end{macro}
%
% \subsubsection{\hologo{MiKTeX}}
%
%    \begin{macro}{\HoLogo@MiKTeX}
%    \begin{macrocode}
\def\HoLogo@MiKTeX#1{%
  \HOLOGO@mbox{MiK}%
  \HOLOGO@discretionary
  \hologo{TeX}%
}
%    \end{macrocode}
%    \end{macro}
%    \begin{macro}{\HoLogoHtml@MiKTeX}
%    \begin{macrocode}
\let\HoLogoHtml@MiKTeX\HoLogo@MiKTeX
%    \end{macrocode}
%    \end{macro}
%
% \subsubsection{\hologo{OzTeX} and friends}
%
%    Source: \hologo{OzTeX} FAQ \cite{OzTeX}:
%    \begin{quote}
%      |\def\OzTeX{O\kern-.03em z\kern-.15em\TeX}|\\
%      (There is no kerning in OzMF, OzMP and OzTtH.)
%    \end{quote}
%
%    \begin{macro}{\HoLogo@OzTeX}
%    \begin{macrocode}
\def\HoLogo@OzTeX#1{%
  O%
  \kern-.03em %
  z%
  \kern-.15em %
  \hologo{TeX}%
}
%    \end{macrocode}
%    \end{macro}
%    \begin{macro}{\HoLogoHtml@OzTeX}
%    \begin{macrocode}
\def\HoLogoHtml@OzTeX#1{%
  \HoLogoCss@OzTeX
  \HOLOGO@Span{OzTeX}{%
    O%
    \HOLOGO@Span{z}{z}%
    \hologo{TeX}%
  }%
}
%    \end{macrocode}
%    \end{macro}
%    \begin{macro}{\HoLogoCss@OzTeX}
%    \begin{macrocode}
\def\HoLogoCss@OzTeX{%
  \Css{%
    span.HoLogo-OzTeX span.HoLogo-z{%
      margin-left:-.03em;%
      margin-right:-.15em;%
    }%
  }%
  \global\let\HoLogoCss@OzTeX\relax
}
%    \end{macrocode}
%    \end{macro}
%
%    \begin{macro}{\HoLogo@OzMF}
%    \begin{macrocode}
\def\HoLogo@OzMF#1{%
  \HOLOGO@mbox{OzMF}%
}
%    \end{macrocode}
%    \end{macro}
%    \begin{macro}{\HoLogo@OzMP}
%    \begin{macrocode}
\def\HoLogo@OzMP#1{%
  \HOLOGO@mbox{OzMP}%
}
%    \end{macrocode}
%    \end{macro}
%    \begin{macro}{\HoLogo@OzTtH}
%    \begin{macrocode}
\def\HoLogo@OzTtH#1{%
  \HOLOGO@mbox{OzTtH}%
}
%    \end{macrocode}
%    \end{macro}
%
% \subsubsection{\hologo{PCTeX}}
%
%    \begin{macro}{\HoLogo@PCTeX}
%    \begin{macrocode}
\def\HoLogo@PCTeX#1{%
  \HOLOGO@mbox{PC}%
  \hologo{TeX}%
}
%    \end{macrocode}
%    \end{macro}
%    \begin{macro}{\HoLogoHtml@PCTeX}
%    \begin{macrocode}
\let\HoLogoHtml@PCTeX\HoLogo@PCTeX
%    \end{macrocode}
%    \end{macro}
%
% \subsubsection{\hologo{PiCTeX}}
%
%    The original definitions from \xfile{pictex.tex} \cite{PiCTeX}:
%\begin{quote}
%\begin{verbatim}
%\def\PiC{%
%  P%
%  \kern-.12em%
%  \lower.5ex\hbox{I}%
%  \kern-.075em%
%  C%
%}
%\def\PiCTeX{%
%  \PiC
%  \kern-.11em%
%  \TeX
%}
%\end{verbatim}
%\end{quote}
%
%    \begin{macro}{\HoLogo@PiC}
%    \begin{macrocode}
\def\HoLogo@PiC#1{%
  P%
  \kern-.12em%
  \lower.5ex\hbox{I}%
  \kern-.075em%
  C%
  \HOLOGO@SpaceFactor
}
%    \end{macrocode}
%    \end{macro}
%    \begin{macro}{\HoLogoHtml@PiC}
%    \begin{macrocode}
\def\HoLogoHtml@PiC#1{%
  \HoLogoCss@PiC
  \HOLOGO@Span{PiC}{%
    P%
    \HOLOGO@Span{i}{I}%
    C%
  }%
}
%    \end{macrocode}
%    \end{macro}
%    \begin{macro}{\HoLogoCss@PiC}
%    \begin{macrocode}
\def\HoLogoCss@PiC{%
  \Css{%
    span.HoLogo-PiC span.HoLogo-i{%
      position:relative;%
      top:.5ex;%
      margin-left:-.12em;%
      margin-right:-.075em;%
      text-decoration:none;%
    }%
  }%
  \global\let\HoLogoCss@PiC\relax
}
%    \end{macrocode}
%    \end{macro}
%
%    \begin{macro}{\HoLogo@PiCTeX}
%    \begin{macrocode}
\def\HoLogo@PiCTeX#1{%
  \hologo{PiC}%
  \HOLOGO@discretionary
  \kern-.11em%
  \hologo{TeX}%
}
%    \end{macrocode}
%    \end{macro}
%    \begin{macro}{\HoLogoHtml@PiCTeX}
%    \begin{macrocode}
\def\HoLogoHtml@PiCTeX#1{%
  \HoLogoCss@PiCTeX
  \HOLOGO@Span{PiCTeX}{%
    \hologo{PiC}%
    \hologo{TeX}%
  }%
}
%    \end{macrocode}
%    \end{macro}
%    \begin{macro}{\HoLogoCss@PiCTeX}
%    \begin{macrocode}
\def\HoLogoCss@PiCTeX{%
  \Css{%
    span.HoLogo-PiCTeX span.HoLogo-PiC{%
      margin-right:-.11em;%
    }%
  }%
  \global\let\HoLogoCss@PiCTeX\relax
}
%    \end{macrocode}
%    \end{macro}
%
% \subsubsection{\hologo{teTeX}}
%
%    \begin{macro}{\HoLogo@teTeX}
%    \begin{macrocode}
\def\HoLogo@teTeX#1{%
  \HOLOGO@mbox{#1{t}{T}e}%
  \HOLOGO@discretionary
  \hologo{TeX}%
}
%    \end{macrocode}
%    \end{macro}
%    \begin{macro}{\HoLogoCs@teTeX}
%    \begin{macrocode}
\def\HoLogoCs@teTeX#1{#1{t}{T}dfTeX}
%    \end{macrocode}
%    \end{macro}
%    \begin{macro}{\HoLogoBkm@teTeX}
%    \begin{macrocode}
\def\HoLogoBkm@teTeX#1{%
  #1{t}{T}e\hologo{TeX}%
}
%    \end{macrocode}
%    \end{macro}
%    \begin{macro}{\HoLogoHtml@teTeX}
%    \begin{macrocode}
\let\HoLogoHtml@teTeX\HoLogo@teTeX
%    \end{macrocode}
%    \end{macro}
%
% \subsubsection{\hologo{TeX4ht}}
%
%    \begin{macro}{\HoLogo@TeX4ht}
%    \begin{macrocode}
\expandafter\def\csname HoLogo@TeX4ht\endcsname#1{%
  \HOLOGO@mbox{\hologo{TeX}4ht}%
}
%    \end{macrocode}
%    \end{macro}
%    \begin{macro}{\HoLogoHtml@TeX4ht}
%    \begin{macrocode}
\expandafter
\let\csname HoLogoHtml@TeX4ht\expandafter\endcsname
\csname HoLogo@TeX4ht\endcsname
%    \end{macrocode}
%    \end{macro}
%
%
% \subsubsection{\hologo{SageTeX}}
%
%    \begin{macro}{\HoLogo@SageTeX}
%    \begin{macrocode}
\def\HoLogo@SageTeX#1{%
  \HOLOGO@mbox{Sage}%
  \HOLOGO@discretionary
  \HOLOGO@NegativeKerning{eT,oT,To}%
  \hologo{TeX}%
}
%    \end{macrocode}
%    \end{macro}
%    \begin{macro}{\HoLogoHtml@SageTeX}
%    \begin{macrocode}
\let\HoLogoHtml@SageTeX\HoLogo@SageTeX
%    \end{macrocode}
%    \end{macro}
%
% \subsection{\hologo{METAFONT} and friends}
%
%    \begin{macro}{\HoLogo@METAFONT}
%    \begin{macrocode}
\def\HoLogo@METAFONT#1{%
  \HoLogoFont@font{METAFONT}{logo}{%
    \HOLOGO@mbox{META}%
    \HOLOGO@discretionary
    \HOLOGO@mbox{FONT}%
  }%
}
%    \end{macrocode}
%    \end{macro}
%
%    \begin{macro}{\HoLogo@METAPOST}
%    \begin{macrocode}
\def\HoLogo@METAPOST#1{%
  \HoLogoFont@font{METAPOST}{logo}{%
    \HOLOGO@mbox{META}%
    \HOLOGO@discretionary
    \HOLOGO@mbox{POST}%
  }%
}
%    \end{macrocode}
%    \end{macro}
%
%    \begin{macro}{\HoLogo@MetaFun}
%    \begin{macrocode}
\def\HoLogo@MetaFun#1{%
  \HOLOGO@mbox{Meta}%
  \HOLOGO@discretionary
  \HOLOGO@mbox{Fun}%
}
%    \end{macrocode}
%    \end{macro}
%
%    \begin{macro}{\HoLogo@MetaPost}
%    \begin{macrocode}
\def\HoLogo@MetaPost#1{%
  \HOLOGO@mbox{Meta}%
  \HOLOGO@discretionary
  \HOLOGO@mbox{Post}%
}
%    \end{macrocode}
%    \end{macro}
%
% \subsection{Others}
%
% \subsubsection{\hologo{biber}}
%
%    \begin{macro}{\HoLogo@biber}
%    \begin{macrocode}
\def\HoLogo@biber#1{%
  \HOLOGO@mbox{#1{b}{B}i}%
  \HOLOGO@discretionary
  \HOLOGO@mbox{ber}%
}
%    \end{macrocode}
%    \end{macro}
%    \begin{macro}{\HoLogoCs@biber}
%    \begin{macrocode}
\def\HoLogoCs@biber#1{#1{b}{B}iber}
%    \end{macrocode}
%    \end{macro}
%    \begin{macro}{\HoLogoBkm@biber}
%    \begin{macrocode}
\def\HoLogoBkm@biber#1{%
  #1{b}{B}iber%
}
%    \end{macrocode}
%    \end{macro}
%    \begin{macro}{\HoLogoHtml@biber}
%    \begin{macrocode}
\let\HoLogoHtml@biber\HoLogo@biber
%    \end{macrocode}
%    \end{macro}
%
% \subsubsection{\hologo{KOMAScript}}
%
%    \begin{macro}{\HoLogo@KOMAScript}
%    The definition for \hologo{KOMAScript} is taken
%    from \hologo{KOMAScript} (\xfile{scrlogo.dtx}, reformatted) \cite{scrlogo}:
%\begin{quote}
%\begin{verbatim}
%\@ifundefined{KOMAScript}{%
%  \DeclareRobustCommand{\KOMAScript}{%
%    \textsf{%
%      K\kern.05em O\kern.05emM\kern.05em A%
%      \kern.1em-\kern.1em %
%      Script%
%    }%
%  }%
%}{}
%\end{verbatim}
%\end{quote}
%    \begin{macrocode}
\def\HoLogo@KOMAScript#1{%
  \HoLogoFont@font{KOMAScript}{sf}{%
    \HOLOGO@mbox{%
      K\kern.05em%
      O\kern.05em%
      M\kern.05em%
      A%
    }%
    \kern.1em%
    \HOLOGO@hyphen
    \kern.1em%
    \HOLOGO@mbox{Script}%
  }%
}
%    \end{macrocode}
%    \end{macro}
%    \begin{macro}{\HoLogoBkm@KOMAScript}
%    \begin{macrocode}
\def\HoLogoBkm@KOMAScript#1{%
  KOMA-Script%
}
%    \end{macrocode}
%    \end{macro}
%    \begin{macro}{\HoLogoHtml@KOMAScript}
%    \begin{macrocode}
\def\HoLogoHtml@KOMAScript#1{%
  \HoLogoCss@KOMAScript
  \HoLogoFont@font{KOMAScript}{sf}{%
    \HOLOGO@Span{KOMAScript}{%
      K%
      \HOLOGO@Span{O}{O}%
      M%
      \HOLOGO@Span{A}{A}%
      \HOLOGO@Span{hyphen}{-}%
      Script%
    }%
  }%
}
%    \end{macrocode}
%    \end{macro}
%    \begin{macro}{\HoLogoCss@KOMAScript}
%    \begin{macrocode}
\def\HoLogoCss@KOMAScript{%
  \Css{%
    span.HoLogo-KOMAScript{%
      font-family:sans-serif;%
    }%
  }%
  \Css{%
    span.HoLogo-KOMAScript span.HoLogo-O{%
      padding-left:.05em;%
      padding-right:.05em;%
    }%
  }%
  \Css{%
    span.HoLogo-KOMAScript span.HoLogo-A{%
      padding-left:.05em;%
    }%
  }%
  \Css{%
    span.HoLogo-KOMAScript span.HoLogo-hyphen{%
      padding-left:.1em;%
      padding-right:.1em;%
    }%
  }%
  \global\let\HoLogoCss@KOMAScript\relax
}
%    \end{macrocode}
%    \end{macro}
%
% \subsubsection{\hologo{LyX}}
%
%    \begin{macro}{\HoLogo@LyX}
%    The definition is taken from the documentation source files
%    of \hologo{LyX}, \xfile{Intro.lyx} \cite{LyX}:
%\begin{quote}
%\begin{verbatim}
%\def\LyX{%
%  \texorpdfstring{%
%    L\kern-.1667em\lower.25em\hbox{Y}\kern-.125emX\@%
%  }{%
%    LyX%
%  }%
%}
%\end{verbatim}
%\end{quote}
%    \begin{macrocode}
\def\HoLogo@LyX#1{%
  L%
  \kern-.1667em%
  \lower.25em\hbox{Y}%
  \kern-.125em%
  X%
  \HOLOGO@SpaceFactor
}
%    \end{macrocode}
%    \end{macro}
%    \begin{macro}{\HoLogoHtml@LyX}
%    \begin{macrocode}
\def\HoLogoHtml@LyX#1{%
  \HoLogoCss@LyX
  \HOLOGO@Span{LyX}{%
    L%
    \HOLOGO@Span{y}{Y}%
    X%
  }%
}
%    \end{macrocode}
%    \end{macro}
%    \begin{macro}{\HoLogoCss@LyX}
%    \begin{macrocode}
\def\HoLogoCss@LyX{%
  \Css{%
    span.HoLogo-LyX span.HoLogo-y{%
      position:relative;%
      top:.25em;%
      margin-left:-.1667em;%
      margin-right:-.125em;%
      text-decoration:none;%
    }%
  }%
  \global\let\HoLogoCss@LyX\relax
}
%    \end{macrocode}
%    \end{macro}
%
% \subsubsection{\hologo{NTS}}
%
%    \begin{macro}{\HoLogo@NTS}
%    Definition for \hologo{NTS} can be found in
%    package \xpackage{etex\textunderscore man} for the \hologo{eTeX} manual \cite{etexman}
%    and in package \xpackage{dtklogos} \cite{dtklogos}:
%\begin{quote}
%\begin{verbatim}
%\def\NTS{%
%  \leavevmode
%  \hbox{%
%    $%
%      \cal N%
%      \kern-0.35em%
%      \lower0.5ex\hbox{$\cal T$}%
%      \kern-0.2em%
%      S%
%    $%
%  }%
%}
%\end{verbatim}
%\end{quote}
%    \begin{macrocode}
\def\HoLogo@NTS#1{%
  \HoLogoFont@font{NTS}{sy}{%
    N\/%
    \kern-.35em%
    \lower.5ex\hbox{T\/}%
    \kern-.2em%
    S\/%
  }%
  \HOLOGO@SpaceFactor
}
%    \end{macrocode}
%    \end{macro}
%
% \subsubsection{\Hologo{TTH} (\hologo{TeX} to HTML translator)}
%
%    Source: \url{http://hutchinson.belmont.ma.us/tth/}
%    In the HTML source the second `T' is printed as subscript.
%\begin{quote}
%\begin{verbatim}
%T<sub>T</sub>H
%\end{verbatim}
%\end{quote}
%    \begin{macro}{\HoLogo@TTH}
%    \begin{macrocode}
\def\HoLogo@TTH#1{%
  \ltx@mbox{%
    T\HOLOGO@SubScript{T}H%
  }%
  \HOLOGO@SpaceFactor
}
%    \end{macrocode}
%    \end{macro}
%
%    \begin{macro}{\HoLogoHtml@TTH}
%    \begin{macrocode}
\def\HoLogoHtml@TTH#1{%
  T\HCode{<sub>}T\HCode{</sub>}H%
}
%    \end{macrocode}
%    \end{macro}
%
% \subsubsection{\Hologo{HanTheThanh}}
%
%    Partial source: Package \xpackage{dtklogos}.
%    The double accent is U+1EBF (latin small letter e with circumflex
%    and acute).
%    \begin{macro}{\HoLogo@HanTheThanh}
%    \begin{macrocode}
\def\HoLogo@HanTheThanh#1{%
  \ltx@mbox{H\`an}%
  \HOLOGO@space
  \ltx@mbox{%
    Th%
    \HOLOGO@IfCharExists{"1EBF}{%
      \char"1EBF\relax
    }{%
      \^e\hbox to 0pt{\hss\raise .5ex\hbox{\'{}}}%
    }%
  }%
  \HOLOGO@space
  \ltx@mbox{Th\`anh}%
}
%    \end{macrocode}
%    \end{macro}
%    \begin{macro}{\HoLogoBkm@HanTheThanh}
%    \begin{macrocode}
\def\HoLogoBkm@HanTheThanh#1{%
  H\`an %
  Th\HOLOGO@PdfdocUnicode{\^e}{\9036\277} %
  Th\`anh%
}
%    \end{macrocode}
%    \end{macro}
%    \begin{macro}{\HoLogoHtml@HanTheThanh}
%    \begin{macrocode}
\def\HoLogoHtml@HanTheThanh#1{%
  H\`an %
  Th\HCode{&\ltx@hashchar x1ebf;} %
  Th\`anh%
}
%    \end{macrocode}
%    \end{macro}
%
% \subsection{Driver detection}
%
%    \begin{macrocode}
\HOLOGO@IfExists\InputIfFileExists{%
  \InputIfFileExists{hologo.cfg}{}{}%
}{%
  \ltx@IfUndefined{pdf@filesize}{%
    \def\HOLOGO@InputIfExists{%
      \openin\HOLOGO@temp=hologo.cfg\relax
      \ifeof\HOLOGO@temp
        \closein\HOLOGO@temp
      \else
        \closein\HOLOGO@temp
        \begingroup
          \def\x{LaTeX2e}%
        \expandafter\endgroup
        \ifx\fmtname\x
          \input{hologo.cfg}%
        \else
          \input hologo.cfg\relax
        \fi
      \fi
    }%
    \ltx@IfUndefined{newread}{%
      \chardef\HOLOGO@temp=15 %
      \def\HOLOGO@CheckRead{%
        \ifeof\HOLOGO@temp
          \HOLOGO@InputIfExists
        \else
          \ifcase\HOLOGO@temp
            \@PackageWarningNoLine{hologo}{%
              Configuration file ignored, because\MessageBreak
              a free read register could not be found%
            }%
          \else
            \begingroup
              \count\ltx@cclv=\HOLOGO@temp
              \advance\ltx@cclv by \ltx@minusone
              \edef\x{\endgroup
                \chardef\noexpand\HOLOGO@temp=\the\count\ltx@cclv
                \relax
              }%
            \x
          \fi
        \fi
      }%
    }{%
      \csname newread\endcsname\HOLOGO@temp
      \HOLOGO@InputIfExists
    }%
  }{%
    \edef\HOLOGO@temp{\pdf@filesize{hologo.cfg}}%
    \ifx\HOLOGO@temp\ltx@empty
    \else
      \ifnum\HOLOGO@temp>0 %
        \begingroup
          \def\x{LaTeX2e}%
        \expandafter\endgroup
        \ifx\fmtname\x
          \input{hologo.cfg}%
        \else
          \input hologo.cfg\relax
        \fi
      \else
        \@PackageInfoNoLine{hologo}{%
          Empty configuration file `hologo.cfg' ignored%
        }%
      \fi
    \fi
  }%
}
%    \end{macrocode}
%
%    \begin{macrocode}
\def\HOLOGO@temp#1#2{%
  \kv@define@key{HoLogoDriver}{#1}[]{%
    \begingroup
      \def\HOLOGO@temp{##1}%
      \ltx@onelevel@sanitize\HOLOGO@temp
      \ifx\HOLOGO@temp\ltx@empty
      \else
        \@PackageError{hologo}{%
          Value (\HOLOGO@temp) not permitted for option `#1'%
        }%
        \@ehc
      \fi
    \endgroup
    \def\hologoDriver{#2}%
  }%
}%
\def\HOLOGO@@temp#1#2{%
  \ifx\kv@value\relax
    \HOLOGO@temp{#1}{#1}%
  \else
    \HOLOGO@temp{#1}{#2}%
  \fi
}%
\kv@parse@normalized{%
  pdftex,%
  luatex=pdftex,%
  dvipdfm,%
  dvipdfmx=dvipdfm,%
  dvips,%
  dvipsone=dvips,%
  xdvi=dvips,%
  xetex,%
  vtex,%
}\HOLOGO@@temp
%    \end{macrocode}
%
%    \begin{macrocode}
\kv@define@key{HoLogoDriver}{driverfallback}{%
  \def\HOLOGO@DriverFallback{#1}%
}
%    \end{macrocode}
%
%    \begin{macro}{\HOLOGO@DriverFallback}
%    \begin{macrocode}
\def\HOLOGO@DriverFallback{dvips}
%    \end{macrocode}
%    \end{macro}
%
%    \begin{macro}{\hologoDriverSetup}
%    \begin{macrocode}
\def\hologoDriverSetup{%
  \let\hologoDriver\ltx@undefined
  \HOLOGO@DriverSetup
}
%    \end{macrocode}
%    \end{macro}
%
%    \begin{macro}{\HOLOGO@DriverSetup}
%    \begin{macrocode}
\def\HOLOGO@DriverSetup#1{%
  \kvsetkeys{HoLogoDriver}{#1}%
  \HOLOGO@CheckDriver
  \ltx@ifundefined{hologoDriver}{%
    \begingroup
    \edef\x{\endgroup
      \noexpand\kvsetkeys{HoLogoDriver}{\HOLOGO@DriverFallback}%
    }\x
  }{}%
  \@PackageInfoNoLine{hologo}{Using driver `\hologoDriver'}%
}
%    \end{macrocode}
%    \end{macro}
%
%    \begin{macro}{\HOLOGO@CheckDriver}
%    \begin{macrocode}
\def\HOLOGO@CheckDriver{%
  \ifpdf
    \def\hologoDriver{pdftex}%
    \let\HOLOGO@pdfliteral\pdfliteral
    \ifluatex
      \ifx\pdfextension\@undefined\else
        \protected\def\pdfliteral{\pdfextension literal}%
        \let\HOLOGO@pdfliteral\pdfliteral
      \fi
      \ltx@IfUndefined{HOLOGO@pdfliteral}{%
        \ifnum\luatexversion<36 %
        \else
          \begingroup
            \let\HOLOGO@temp\endgroup
            \ifcase0%
                \directlua{%
                  if tex.enableprimitives then %
                    tex.enableprimitives('HOLOGO@', {'pdfliteral'})%
                  else %
                    tex.print('1')%
                  end%
                }%
                \ifx\HOLOGO@pdfliteral\@undefined 1\fi%
                \relax%
              \endgroup
              \let\HOLOGO@temp\relax
              \global\let\HOLOGO@pdfliteral\HOLOGO@pdfliteral
            \fi%
          \HOLOGO@temp
        \fi
      }{}%
    \fi
    \ltx@IfUndefined{HOLOGO@pdfliteral}{%
      \@PackageWarningNoLine{hologo}{%
        Cannot find \string\pdfliteral
      }%
    }{}%
  \else
    \ifxetex
      \def\hologoDriver{xetex}%
    \else
      \ifvtex
        \def\hologoDriver{vtex}%
      \fi
    \fi
  \fi
}
%    \end{macrocode}
%    \end{macro}
%
%    \begin{macro}{\HOLOGO@WarningUnsupportedDriver}
%    \begin{macrocode}
\def\HOLOGO@WarningUnsupportedDriver#1{%
  \@PackageWarningNoLine{hologo}{%
    Logo `#1' needs driver specific macros,\MessageBreak
    but driver `\hologoDriver' is not supported.\MessageBreak
    Use a different driver or\MessageBreak
    load package `graphics' or `pgf'%
  }%
}
%    \end{macrocode}
%    \end{macro}
%
% \subsubsection{Reflect box macros}
%
%    Skip driver part if not needed.
%    \begin{macrocode}
\ltx@IfUndefined{reflectbox}{}{%
  \ltx@IfUndefined{rotatebox}{}{%
    \HOLOGO@AtEnd
  }%
}
\ltx@IfUndefined{pgftext}{}{%
  \HOLOGO@AtEnd
}
\ltx@IfUndefined{psscalebox}{}{%
  \HOLOGO@AtEnd
}
%    \end{macrocode}
%
%    \begin{macrocode}
\def\HOLOGO@temp{LaTeX2e}
\ifx\fmtname\HOLOGO@temp
  \RequirePackage{kvoptions}[2011/06/30]%
  \ProcessKeyvalOptions{HoLogoDriver}%
\fi
\HOLOGO@DriverSetup{}
%    \end{macrocode}
%
%    \begin{macro}{\HOLOGO@ReflectBox}
%    \begin{macrocode}
\def\HOLOGO@ReflectBox#1{%
  \begingroup
    \setbox\ltx@zero\hbox{\begingroup#1\endgroup}%
    \setbox\ltx@two\hbox{%
      \kern\wd\ltx@zero
      \csname HOLOGO@ScaleBox@\hologoDriver\endcsname{-1}{1}{%
        \hbox to 0pt{\copy\ltx@zero\hss}%
      }%
    }%
    \wd\ltx@two=\wd\ltx@zero
    \box\ltx@two
  \endgroup
}
%    \end{macrocode}
%    \end{macro}
%
%    \begin{macro}{\HOLOGO@PointReflectBox}
%    \begin{macrocode}
\def\HOLOGO@PointReflectBox#1{%
  \begingroup
    \setbox\ltx@zero\hbox{\begingroup#1\endgroup}%
    \setbox\ltx@two\hbox{%
      \kern\wd\ltx@zero
      \raise\ht\ltx@zero\hbox{%
        \csname HOLOGO@ScaleBox@\hologoDriver\endcsname{-1}{-1}{%
          \hbox to 0pt{\copy\ltx@zero\hss}%
        }%
      }%
    }%
    \wd\ltx@two=\wd\ltx@zero
    \box\ltx@two
  \endgroup
}
%    \end{macrocode}
%    \end{macro}
%
%    We must define all variants because of dynamic driver setup.
%    \begin{macrocode}
\def\HOLOGO@temp#1#2{#2}
%    \end{macrocode}
%
%    \begin{macro}{\HOLOGO@ScaleBox@pdftex}
%    \begin{macrocode}
\HOLOGO@temp{pdftex}{%
  \def\HOLOGO@ScaleBox@pdftex#1#2#3{%
    \HOLOGO@pdfliteral{%
      q #1 0 0 #2 0 0 cm%
    }%
    #3%
    \HOLOGO@pdfliteral{%
      Q%
    }%
  }%
}
%    \end{macrocode}
%    \end{macro}
%    \begin{macro}{\HOLOGO@ScaleBox@dvips}
%    \begin{macrocode}
\HOLOGO@temp{dvips}{%
  \def\HOLOGO@ScaleBox@dvips#1#2#3{%
    \special{ps:%
      gsave %
      currentpoint %
      currentpoint translate %
      #1 #2 scale %
      neg exch neg exch translate%
    }%
    #3%
    \special{ps:%
      currentpoint %
      grestore %
      moveto%
    }%
  }%
}
%    \end{macrocode}
%    \end{macro}
%    \begin{macro}{\HOLOGO@ScaleBox@dvipdfm}
%    \begin{macrocode}
\HOLOGO@temp{dvipdfm}{%
  \let\HOLOGO@ScaleBox@dvipdfm\HOLOGO@ScaleBox@dvips
}
%    \end{macrocode}
%    \end{macro}
%    Since \hologo{XeTeX} v0.6.
%    \begin{macro}{\HOLOGO@ScaleBox@xetex}
%    \begin{macrocode}
\HOLOGO@temp{xetex}{%
  \def\HOLOGO@ScaleBox@xetex#1#2#3{%
    \special{x:gsave}%
    \special{x:scale #1 #2}%
    #3%
    \special{x:grestore}%
  }%
}
%    \end{macrocode}
%    \end{macro}
%    \begin{macro}{\HOLOGO@ScaleBox@vtex}
%    \begin{macrocode}
\HOLOGO@temp{vtex}{%
  \def\HOLOGO@ScaleBox@vtex#1#2#3{%
    \special{r(#1,0,0,#2,0,0}%
    #3%
    \special{r)}%
  }%
}
%    \end{macrocode}
%    \end{macro}
%
%    \begin{macrocode}
\HOLOGO@AtEnd%
%</package>
%    \end{macrocode}
%
% \section{Test}
%
% \subsection{Catcode checks for loading}
%
%    \begin{macrocode}
%<*test1>
%    \end{macrocode}
%    \begin{macrocode}
\catcode`\{=1 %
\catcode`\}=2 %
\catcode`\#=6 %
\catcode`\@=11 %
\expandafter\ifx\csname count@\endcsname\relax
  \countdef\count@=255 %
\fi
\expandafter\ifx\csname @gobble\endcsname\relax
  \long\def\@gobble#1{}%
\fi
\expandafter\ifx\csname @firstofone\endcsname\relax
  \long\def\@firstofone#1{#1}%
\fi
\expandafter\ifx\csname loop\endcsname\relax
  \expandafter\@firstofone
\else
  \expandafter\@gobble
\fi
{%
  \def\loop#1\repeat{%
    \def\body{#1}%
    \iterate
  }%
  \def\iterate{%
    \body
      \let\next\iterate
    \else
      \let\next\relax
    \fi
    \next
  }%
  \let\repeat=\fi
}%
\def\RestoreCatcodes{}
\count@=0 %
\loop
  \edef\RestoreCatcodes{%
    \RestoreCatcodes
    \catcode\the\count@=\the\catcode\count@\relax
  }%
\ifnum\count@<255 %
  \advance\count@ 1 %
\repeat

\def\RangeCatcodeInvalid#1#2{%
  \count@=#1\relax
  \loop
    \catcode\count@=15 %
  \ifnum\count@<#2\relax
    \advance\count@ 1 %
  \repeat
}
\def\RangeCatcodeCheck#1#2#3{%
  \count@=#1\relax
  \loop
    \ifnum#3=\catcode\count@
    \else
      \errmessage{%
        Character \the\count@\space
        with wrong catcode \the\catcode\count@\space
        instead of \number#3%
      }%
    \fi
  \ifnum\count@<#2\relax
    \advance\count@ 1 %
  \repeat
}
\def\space{ }
\expandafter\ifx\csname LoadCommand\endcsname\relax
  \def\LoadCommand{\input hologo.sty\relax}%
\fi
\def\Test{%
  \RangeCatcodeInvalid{0}{47}%
  \RangeCatcodeInvalid{58}{64}%
  \RangeCatcodeInvalid{91}{96}%
  \RangeCatcodeInvalid{123}{255}%
  \catcode`\@=12 %
  \catcode`\\=0 %
  \catcode`\%=14 %
  \LoadCommand
  \RangeCatcodeCheck{0}{36}{15}%
  \RangeCatcodeCheck{37}{37}{14}%
  \RangeCatcodeCheck{38}{47}{15}%
  \RangeCatcodeCheck{48}{57}{12}%
  \RangeCatcodeCheck{58}{63}{15}%
  \RangeCatcodeCheck{64}{64}{12}%
  \RangeCatcodeCheck{65}{90}{11}%
  \RangeCatcodeCheck{91}{91}{15}%
  \RangeCatcodeCheck{92}{92}{0}%
  \RangeCatcodeCheck{93}{96}{15}%
  \RangeCatcodeCheck{97}{122}{11}%
  \RangeCatcodeCheck{123}{255}{15}%
  \RestoreCatcodes
}
\Test
\csname @@end\endcsname
\end
%    \end{macrocode}
%    \begin{macrocode}
%</test1>
%    \end{macrocode}
%
% \subsection{Spacefactor}
%
%    The space factor must be 1000 after a logo. If it is greater 1000
%    then the following space is a space after a sentence closing point.
%    If the space factor is smaller 1000 then an immediate following
%    dot is interpreted as abbreviation, not sentence closing point.
%
%    \begin{macrocode}
%<*test-spacefactor>
\NeedsTeXFormat{LaTeX2e}
\documentclass{article}
\usepackage{hologo}[2016/05/12]
\usepackage{kvsetkeys}
\usepackage{qstest}
\IncludeTests{*}
\LogTests{log}{*}{*}
\begin{document}
\begin{qstest}{spacefactor}{spacefactor}
\newcommand*{\Test}[1]{%
  \sbox0{%
    \hologo{#1}%
    \Expect*{1000 (#1)}*{\the\spacefactor\space(#1)}%
  }%
}%
\makeatletter
\def\TestList{}
\def\hologoEntry#1#2#3{%
  \edef\TestList{%
    \ifx\TestList\@empty
    \else
      \TestList,%
    \fi
    #1%
    \ifx\\#2\\%
    \else
      ={variant=#2}%
    \fi
  }%
}
\hologoList
\expandafter\kv@parse@normalized\expandafter{%
  \TestList
}{%
  \begingroup
    \let\@logo=\kv@key
    \ifx\kv@value\relax
    \else
      \expandafter\hologoLogoSetup\expandafter\@logo\expandafter{%
        \kv@value
      }%
    \fi
    \Test\@logo
  \endgroup
  \@gobbletwo
}
\end{qstest}
\end{document}
%</test-spacefactor>
%    \end{macrocode}
%
% \subsection{Complete list}
%
%    \begin{macrocode}
%<*test-list>
\NeedsTeXFormat{LaTeX2e}
\documentclass[12pt,a4paper]{article}
\usepackage{hologo}[2016/05/12]
\usepackage[T1]{fontenc}
\usepackage{lmodern}
\usepackage{parskip}
\usepackage[unicode]{hyperref}[2011/09/28]
\usepackage{bookmark}[2011/09/19]
\bookmarksetup{%
  numbered,%
  open,%
  openlevel=2,%
}
\renewcommand*{\contentsname}{List of logos}
\begin{document}
\tableofcontents
\def\TestFont#1#2#3#4#5#6{%
  \begingroup
    \usefont{#3}{#4}{#5}{#6}%
    \HologoVariant{#1}{#2}/\hologoVariant{#1}{#2}%
    \quad
    \begingroup\scriptsize\hologoVariant{#1}{#2}\endgroup
    \quad
  \endgroup
  (#3/#4/#5/#6)%
  \par
}
\makeatletter
\def\hologoEntry#1#2#3{%
  \section{%
    \HologoVariant{#1}{#2}/\hologoVariant{#1}{#2} %
    {[#1\ifx\\#2\\\else\space(#2)\fi]}% hash-ok
  }% braces around [] because of bug in tex4ht
  \begingroup
    \hypersetup{unicode=false}%
    \bookmark[%
      dest=\@currentHref,%
      rellevel=1,%
      keeplevel,%
    ]{%
      \HologoVariant{#1}{#2}/\hologoVariant{#1}{#2} %
      (PDFDocEncoding)%
    }%
  \endgroup
  \TestFont{#1}{#2}{OT1}{cmr}{m}{n}%
  \TestFont{#1}{#2}{OT1}{cmss}{m}{n}%
  \TestFont{#1}{#2}{OT1}{cmr}{b}{n}%
  \TestFont{#1}{#2}{OT1}{cmr}{m}{it}%
  \TestFont{#1}{#2}{OT1}{cmtt}{m}{n}%
  \TestFont{#1}{#2}{T1}{lmr}{m}{n}%
  \TestFont{#1}{#2}{T1}{lmss}{m}{n}%
  \TestFont{#1}{#2}{T1}{lmr}{b}{n}%
  \TestFont{#1}{#2}{T1}{lmr}{m}{it}%
  \TestFont{#1}{#2}{T1}{lmtt}{m}{n}%
  \TestFont{#1}{#2}{T1}{lmvtt}{m}{n}%
  \TestFont{#1}{#2}{T1}{qtm}{m}{n}%
  \TestFont{#1}{#2}{T1}{qhv}{m}{n}%
  \TestFont{#1}{#2}{T1}{qtm}{b}{n}%
  \TestFont{#1}{#2}{T1}{qtm}{m}{it}%
  \TestFont{#1}{#2}{T1}{qcr}{m}{n}%
  \newpage
}
\makeatother
\hologoList
\end{document}
%</test-list>
%    \end{macrocode}
%
% \section{Installation}
%
% \subsection{Download}
%
% \paragraph{Package.} This package is available on
% CTAN\footnote{\url{ftp://ftp.ctan.org/tex-archive/}}:
% \begin{description}
% \item[\CTAN{macros/latex/contrib/oberdiek/hologo.dtx}] The source file.
% \item[\CTAN{macros/latex/contrib/oberdiek/hologo.pdf}] Documentation.
% \end{description}
%
%
% \paragraph{Bundle.} All the packages of the bundle `oberdiek'
% are also available in a TDS compliant ZIP archive. There
% the packages are already unpacked and the documentation files
% are generated. The files and directories obey the TDS standard.
% \begin{description}
% \item[\CTAN{install/macros/latex/contrib/oberdiek.tds.zip}]
% \end{description}
% \emph{TDS} refers to the standard ``A Directory Structure
% for \TeX\ Files'' (\CTAN{tds/tds.pdf}). Directories
% with \xfile{texmf} in their name are usually organized this way.
%
% \subsection{Bundle installation}
%
% \paragraph{Unpacking.} Unpack the \xfile{oberdiek.tds.zip} in the
% TDS tree (also known as \xfile{texmf} tree) of your choice.
% Example (linux):
% \begin{quote}
%   |unzip oberdiek.tds.zip -d ~/texmf|
% \end{quote}
%
% \paragraph{Script installation.}
% Check the directory \xfile{TDS:scripts/oberdiek/} for
% scripts that need further installation steps.
% Package \xpackage{attachfile2} comes with the Perl script
% \xfile{pdfatfi.pl} that should be installed in such a way
% that it can be called as \texttt{pdfatfi}.
% Example (linux):
% \begin{quote}
%   |chmod +x scripts/oberdiek/pdfatfi.pl|\\
%   |cp scripts/oberdiek/pdfatfi.pl /usr/local/bin/|
% \end{quote}
%
% \subsection{Package installation}
%
% \paragraph{Unpacking.} The \xfile{.dtx} file is a self-extracting
% \docstrip\ archive. The files are extracted by running the
% \xfile{.dtx} through \plainTeX:
% \begin{quote}
%   \verb|tex hologo.dtx|
% \end{quote}
%
% \paragraph{TDS.} Now the different files must be moved into
% the different directories in your installation TDS tree
% (also known as \xfile{texmf} tree):
% \begin{quote}
% \def\t{^^A
% \begin{tabular}{@{}>{\ttfamily}l@{ $\rightarrow$ }>{\ttfamily}l@{}}
%   hologo.sty & tex/generic/oberdiek/hologo.sty\\
%   hologo.pdf & doc/latex/oberdiek/hologo.pdf\\
%   example/hologo-example.tex & doc/latex/oberdiek/example/hologo-example.tex\\
%   test/hologo-test1.tex & doc/latex/oberdiek/test/hologo-test1.tex\\
%   test/hologo-test-spacefactor.tex & doc/latex/oberdiek/test/hologo-test-spacefactor.tex\\
%   test/hologo-test-list.tex & doc/latex/oberdiek/test/hologo-test-list.tex\\
%   hologo.dtx & source/latex/oberdiek/hologo.dtx\\
% \end{tabular}^^A
% }^^A
% \sbox0{\t}^^A
% \ifdim\wd0>\linewidth
%   \begingroup
%     \advance\linewidth by\leftmargin
%     \advance\linewidth by\rightmargin
%   \edef\x{\endgroup
%     \def\noexpand\lw{\the\linewidth}^^A
%   }\x
%   \def\lwbox{^^A
%     \leavevmode
%     \hbox to \linewidth{^^A
%       \kern-\leftmargin\relax
%       \hss
%       \usebox0
%       \hss
%       \kern-\rightmargin\relax
%     }^^A
%   }^^A
%   \ifdim\wd0>\lw
%     \sbox0{\small\t}^^A
%     \ifdim\wd0>\linewidth
%       \ifdim\wd0>\lw
%         \sbox0{\footnotesize\t}^^A
%         \ifdim\wd0>\linewidth
%           \ifdim\wd0>\lw
%             \sbox0{\scriptsize\t}^^A
%             \ifdim\wd0>\linewidth
%               \ifdim\wd0>\lw
%                 \sbox0{\tiny\t}^^A
%                 \ifdim\wd0>\linewidth
%                   \lwbox
%                 \else
%                   \usebox0
%                 \fi
%               \else
%                 \lwbox
%               \fi
%             \else
%               \usebox0
%             \fi
%           \else
%             \lwbox
%           \fi
%         \else
%           \usebox0
%         \fi
%       \else
%         \lwbox
%       \fi
%     \else
%       \usebox0
%     \fi
%   \else
%     \lwbox
%   \fi
% \else
%   \usebox0
% \fi
% \end{quote}
% If you have a \xfile{docstrip.cfg} that configures and enables \docstrip's
% TDS installing feature, then some files can already be in the right
% place, see the documentation of \docstrip.
%
% \subsection{Refresh file name databases}
%
% If your \TeX~distribution
% (\teTeX, \mikTeX, \dots) relies on file name databases, you must refresh
% these. For example, \teTeX\ users run \verb|texhash| or
% \verb|mktexlsr|.
%
% \subsection{Some details for the interested}
%
% \paragraph{Attached source.}
%
% The PDF documentation on CTAN also includes the
% \xfile{.dtx} source file. It can be extracted by
% AcrobatReader 6 or higher. Another option is \textsf{pdftk},
% e.g. unpack the file into the current directory:
% \begin{quote}
%   \verb|pdftk hologo.pdf unpack_files output .|
% \end{quote}
%
% \paragraph{Unpacking with \LaTeX.}
% The \xfile{.dtx} chooses its action depending on the format:
% \begin{description}
% \item[\plainTeX:] Run \docstrip\ and extract the files.
% \item[\LaTeX:] Generate the documentation.
% \end{description}
% If you insist on using \LaTeX\ for \docstrip\ (really,
% \docstrip\ does not need \LaTeX), then inform the autodetect routine
% about your intention:
% \begin{quote}
%   \verb|latex \let\install=y\input{hologo.dtx}|
% \end{quote}
% Do not forget to quote the argument according to the demands
% of your shell.
%
% \paragraph{Generating the documentation.}
% You can use both the \xfile{.dtx} or the \xfile{.drv} to generate
% the documentation. The process can be configured by the
% configuration file \xfile{ltxdoc.cfg}. For instance, put this
% line into this file, if you want to have A4 as paper format:
% \begin{quote}
%   \verb|\PassOptionsToClass{a4paper}{article}|
% \end{quote}
% An example follows how to generate the
% documentation with pdf\LaTeX:
% \begin{quote}
%\begin{verbatim}
%pdflatex hologo.dtx
%makeindex -s gind.ist hologo.idx
%pdflatex hologo.dtx
%makeindex -s gind.ist hologo.idx
%pdflatex hologo.dtx
%\end{verbatim}
% \end{quote}
%
% \section{Catalogue}
%
% The following XML file can be used as source for the
% \href{http://mirror.ctan.org/help/Catalogue/catalogue.html}{\TeX\ Catalogue}.
% The elements \texttt{caption} and \texttt{description} are imported
% from the original XML file from the Catalogue.
% The name of the XML file in the Catalogue is \xfile{hologo.xml}.
%    \begin{macrocode}
%<*catalogue>
<?xml version='1.0' encoding='us-ascii'?>
<!DOCTYPE entry SYSTEM 'catalogue.dtd'>
<entry datestamp='$Date$' modifier='$Author$' id='hologo'>
  <name>hologo</name>
  <caption>A collection of logos with bookmark support.</caption>
  <authorref id='auth:oberdiek'/>
  <copyright owner='Heiko Oberdiek' year='2010-2012'/>
  <license type='lppl1.3'/>
  <version number='1.10'/>
  <description>
    The package defines a single command <tt>\hologo</tt>, whose
    argument is the usual case-confused ASCII version of the logo.
    The command is bookmark-enabled, so that every logo becomes
    available in bookmarks without further work.
    <p/>
    The package is part of the <xref refid='oberdiek'>oberdiek</xref>
    bundle.
  </description>
  <documentation details='Package documentation'
      href='ctan:/macros/latex/contrib/oberdiek/hologo.pdf'/>
  <ctan file='true' path='/macros/latex/contrib/oberdiek/hologo.dtx'/>
  <miktex location='oberdiek'/>
  <texlive location='oberdiek'/>
  <install path='/macros/latex/contrib/oberdiek/oberdiek.tds.zip'/>
</entry>
%</catalogue>
%    \end{macrocode}
%
% \begin{thebibliography}{9}
% \raggedright
%
% \bibitem{btxdoc}
% Oren Patashnik,
% \textit{\hologo{BibTeX}ing},
% 1988-02-08.\\
% \CTAN{biblio/bibtex/base/}
%
% \bibitem{dtklogos}
% Gerd Neugebauer, DANTE,
% \textit{Package \xpackage{dtklogos}},
% 2011-04-25.\\
% \CTAN{usergrps/dante/dtk/dtklogos.sty}
%
% \bibitem{etexman}
% The \hologo{NTS} Team,
% \textit{The \hologo{eTeX} manual},
% 1998-02.\\
% \CTAN{systems/e-tex/v2/doc/}
%
% \bibitem{ExTeX-FAQ}
% The \hologo{ExTeX} group,
% \textit{\hologo{ExTeX}: FAQ -- How is \hologo{ExTeX} typeset?},
% 2007-04-14.\\
% \url{http://www.extex.org/documentation/faq.html}
%
% \bibitem{LyX}
% %@MISC{ LyX,
% %  title = {{LyX 2.0.0 -- The Document Processor [Computer software and manual]}},
% %  author = {{The LyX Team}},
% %  howpublished = {Internet: http://www.lyx.org},
% %  year = {2011-05-08},
% %  note = {Retrieved May 10, 2011, from http://www.lyx.org},
% %  url = {http://www.lyx.org/}
% %}
% The \hologo{LyX} Team,
% \textit{\hologo{LyX} -- The Document Processor},
% 2011-05-08.\\
% \url{http://www.lyx.org/}
%
% \bibitem{OzTeX}
% Andrew Trevorrow,
% \hologo{OzTeX} FAQ: What is the correct way to typeset ``\hologo{OzTeX}''?,
% 2011-09-15 (visited).
% \url{http://www.trevorrow.com/oztex/ozfaq.html#oztex-logo}
%
% \bibitem{PiCTeX}
% Michael Wichura,
% \textit{The \hologo{PiCTeX} macro package},
% 1987-09-21.
% \CTAN{graphics/pictex/}
%
% \bibitem{scrlogo}
% Markus Kohm,
% \textit{\hologo{KOMAScript} Datei \xfile{scrlogo.dtx}},
% 2009-01-30.\\
% \CTAN{install/macros/latex/contrib/komascript.tds.zip}
%
% \end{thebibliography}
%
% \begin{History}
%   \begin{Version}{2010/04/08 v1.0}
%   \item
%     The first version.
%   \end{Version}
%   \begin{Version}{2010/04/16 v1.1}
%   \item
%     \cs{Hologo} added for support of logos at start of a sentence.
%   \item
%     \cs{hologoSetup} and \cs{hologoLogoSetup} added.
%   \item
%     Options \xoption{break}, \xoption{hyphenbreak}, \xoption{spacebreak}
%     added.
%   \item
%     Variant support added by option \xoption{variant}.
%   \end{Version}
%   \begin{Version}{2010/04/24 v1.2}
%   \item
%     \hologo{LaTeX3} added.
%   \item
%     \hologo{VTeX} added.
%   \end{Version}
%   \begin{Version}{2010/11/21 v1.3}
%   \item
%     \hologo{iniTeX}, \hologo{virTeX} added.
%   \end{Version}
%   \begin{Version}{2011/03/25 v1.4}
%   \item
%     \hologo{ConTeXt} with variants added.
%   \item
%     Option \xoption{discretionarybreak} added as refinement for
%     option \xoption{break}.
%   \end{Version}
%   \begin{Version}{2011/04/21 v1.5}
%   \item
%     Wrong TDS directory for test files fixed.
%   \end{Version}
%   \begin{Version}{2011/10/01 v1.6}
%   \item
%     Support for package \xpackage{tex4ht} added.
%   \item
%     Support for \cs{csname} added if \cs{ifincsname} is available.
%   \item
%     New logos:
%     \hologo{(La)TeX},
%     \hologo{biber},
%     \hologo{BibTeX} (\xoption{sc}, \xoption{sf}),
%     \hologo{emTeX},
%     \hologo{ExTeX},
%     \hologo{KOMAScript},
%     \hologo{La},
%     \hologo{LyX},
%     \hologo{MiKTeX},
%     \hologo{NTS},
%     \hologo{OzMF},
%     \hologo{OzMP},
%     \hologo{OzTeX},
%     \hologo{OzTtH},
%     \hologo{PCTeX},
%     \hologo{PiC},
%     \hologo{PiCTeX},
%     \hologo{METAFONT},
%     \hologo{MetaFun},
%     \hologo{METAPOST},
%     \hologo{MetaPost},
%     \hologo{SLiTeX} (\xoption{lift}, \xoption{narrow}, \xoption{simple}),
%     \hologo{SliTeX} (\xoption{narrow}, \xoption{simple}, \xoption{lift}),
%     \hologo{teTeX}.
%   \item
%     Fixes:
%     \hologo{iniTeX},
%     \hologo{pdfLaTeX},
%     \hologo{pdfTeX},
%     \hologo{virTeX}.
%   \item
%     \cs{hologoFontSetup} and \cs{hologoLogoFontSetup} added.
%   \item
%     \cs{hologoVariant} and \cs{HologoVariant} added.
%   \end{Version}
%   \begin{Version}{2011/11/22 v1.7}
%   \item
%     New logos:
%     \hologo{BibTeX8},
%     \hologo{LaTeXML},
%     \hologo{SageTeX},
%     \hologo{TeX4ht},
%     \hologo{TTH}.
%   \item
%     \hologo{Xe} and friends: Driver stuff fixed.
%   \item
%     \hologo{Xe} and friends: Support for italic added.
%   \item
%     \hologo{Xe} and friends: Package support for \xpackage{pgf}
%     and \xpackage{pstricks} added.
%   \end{Version}
%   \begin{Version}{2011/11/29 v1.8}
%   \item
%     New logos:
%     \hologo{HanTheThanh}.
%   \end{Version}
%   \begin{Version}{2011/12/21 v1.9}
%   \item
%     Patch for package \xpackage{ifxetex} added for the case that
%     \cs{newif} is undefined in \hologo{iniTeX}.
%   \item
%     Some fixes for \hologo{iniTeX}.
%   \end{Version}
%   \begin{Version}{2012/04/26 v1.10}
%   \item
%     Fix in bookmark version of logo ``\hologo{HanTheThanh}''.
%   \end{Version}
%   \begin{Version}{2016/05/12 v1.11}
%   \item
%     Update HOLOGO@IfCharExists (previously in texlive)
%   \item define pdfliteral in current luatex.
%   \end{Version}
% \end{History}
%
% \PrintIndex
%
% \Finale
\endinput
%
        \else
          \input hologo.cfg\relax
        \fi
      \else
        \@PackageInfoNoLine{hologo}{%
          Empty configuration file `hologo.cfg' ignored%
        }%
      \fi
    \fi
  }%
}
%    \end{macrocode}
%
%    \begin{macrocode}
\def\HOLOGO@temp#1#2{%
  \kv@define@key{HoLogoDriver}{#1}[]{%
    \begingroup
      \def\HOLOGO@temp{##1}%
      \ltx@onelevel@sanitize\HOLOGO@temp
      \ifx\HOLOGO@temp\ltx@empty
      \else
        \@PackageError{hologo}{%
          Value (\HOLOGO@temp) not permitted for option `#1'%
        }%
        \@ehc
      \fi
    \endgroup
    \def\hologoDriver{#2}%
  }%
}%
\def\HOLOGO@@temp#1#2{%
  \ifx\kv@value\relax
    \HOLOGO@temp{#1}{#1}%
  \else
    \HOLOGO@temp{#1}{#2}%
  \fi
}%
\kv@parse@normalized{%
  pdftex,%
  luatex=pdftex,%
  dvipdfm,%
  dvipdfmx=dvipdfm,%
  dvips,%
  dvipsone=dvips,%
  xdvi=dvips,%
  xetex,%
  vtex,%
}\HOLOGO@@temp
%    \end{macrocode}
%
%    \begin{macrocode}
\kv@define@key{HoLogoDriver}{driverfallback}{%
  \def\HOLOGO@DriverFallback{#1}%
}
%    \end{macrocode}
%
%    \begin{macro}{\HOLOGO@DriverFallback}
%    \begin{macrocode}
\def\HOLOGO@DriverFallback{dvips}
%    \end{macrocode}
%    \end{macro}
%
%    \begin{macro}{\hologoDriverSetup}
%    \begin{macrocode}
\def\hologoDriverSetup{%
  \let\hologoDriver\ltx@undefined
  \HOLOGO@DriverSetup
}
%    \end{macrocode}
%    \end{macro}
%
%    \begin{macro}{\HOLOGO@DriverSetup}
%    \begin{macrocode}
\def\HOLOGO@DriverSetup#1{%
  \kvsetkeys{HoLogoDriver}{#1}%
  \HOLOGO@CheckDriver
  \ltx@ifundefined{hologoDriver}{%
    \begingroup
    \edef\x{\endgroup
      \noexpand\kvsetkeys{HoLogoDriver}{\HOLOGO@DriverFallback}%
    }\x
  }{}%
  \@PackageInfoNoLine{hologo}{Using driver `\hologoDriver'}%
}
%    \end{macrocode}
%    \end{macro}
%
%    \begin{macro}{\HOLOGO@CheckDriver}
%    \begin{macrocode}
\def\HOLOGO@CheckDriver{%
  \ifpdf
    \def\hologoDriver{pdftex}%
    \let\HOLOGO@pdfliteral\pdfliteral
    \ifluatex
      \ifx\pdfextension\@undefined\else
        \protected\def\pdfliteral{\pdfextension literal}%
        \let\HOLOGO@pdfliteral\pdfliteral
      \fi
      \ltx@IfUndefined{HOLOGO@pdfliteral}{%
        \ifnum\luatexversion<36 %
        \else
          \begingroup
            \let\HOLOGO@temp\endgroup
            \ifcase0%
                \directlua{%
                  if tex.enableprimitives then %
                    tex.enableprimitives('HOLOGO@', {'pdfliteral'})%
                  else %
                    tex.print('1')%
                  end%
                }%
                \ifx\HOLOGO@pdfliteral\@undefined 1\fi%
                \relax%
              \endgroup
              \let\HOLOGO@temp\relax
              \global\let\HOLOGO@pdfliteral\HOLOGO@pdfliteral
            \fi%
          \HOLOGO@temp
        \fi
      }{}%
    \fi
    \ltx@IfUndefined{HOLOGO@pdfliteral}{%
      \@PackageWarningNoLine{hologo}{%
        Cannot find \string\pdfliteral
      }%
    }{}%
  \else
    \ifxetex
      \def\hologoDriver{xetex}%
    \else
      \ifvtex
        \def\hologoDriver{vtex}%
      \fi
    \fi
  \fi
}
%    \end{macrocode}
%    \end{macro}
%
%    \begin{macro}{\HOLOGO@WarningUnsupportedDriver}
%    \begin{macrocode}
\def\HOLOGO@WarningUnsupportedDriver#1{%
  \@PackageWarningNoLine{hologo}{%
    Logo `#1' needs driver specific macros,\MessageBreak
    but driver `\hologoDriver' is not supported.\MessageBreak
    Use a different driver or\MessageBreak
    load package `graphics' or `pgf'%
  }%
}
%    \end{macrocode}
%    \end{macro}
%
% \subsubsection{Reflect box macros}
%
%    Skip driver part if not needed.
%    \begin{macrocode}
\ltx@IfUndefined{reflectbox}{}{%
  \ltx@IfUndefined{rotatebox}{}{%
    \HOLOGO@AtEnd
  }%
}
\ltx@IfUndefined{pgftext}{}{%
  \HOLOGO@AtEnd
}
\ltx@IfUndefined{psscalebox}{}{%
  \HOLOGO@AtEnd
}
%    \end{macrocode}
%
%    \begin{macrocode}
\def\HOLOGO@temp{LaTeX2e}
\ifx\fmtname\HOLOGO@temp
  \RequirePackage{kvoptions}[2011/06/30]%
  \ProcessKeyvalOptions{HoLogoDriver}%
\fi
\HOLOGO@DriverSetup{}
%    \end{macrocode}
%
%    \begin{macro}{\HOLOGO@ReflectBox}
%    \begin{macrocode}
\def\HOLOGO@ReflectBox#1{%
  \begingroup
    \setbox\ltx@zero\hbox{\begingroup#1\endgroup}%
    \setbox\ltx@two\hbox{%
      \kern\wd\ltx@zero
      \csname HOLOGO@ScaleBox@\hologoDriver\endcsname{-1}{1}{%
        \hbox to 0pt{\copy\ltx@zero\hss}%
      }%
    }%
    \wd\ltx@two=\wd\ltx@zero
    \box\ltx@two
  \endgroup
}
%    \end{macrocode}
%    \end{macro}
%
%    \begin{macro}{\HOLOGO@PointReflectBox}
%    \begin{macrocode}
\def\HOLOGO@PointReflectBox#1{%
  \begingroup
    \setbox\ltx@zero\hbox{\begingroup#1\endgroup}%
    \setbox\ltx@two\hbox{%
      \kern\wd\ltx@zero
      \raise\ht\ltx@zero\hbox{%
        \csname HOLOGO@ScaleBox@\hologoDriver\endcsname{-1}{-1}{%
          \hbox to 0pt{\copy\ltx@zero\hss}%
        }%
      }%
    }%
    \wd\ltx@two=\wd\ltx@zero
    \box\ltx@two
  \endgroup
}
%    \end{macrocode}
%    \end{macro}
%
%    We must define all variants because of dynamic driver setup.
%    \begin{macrocode}
\def\HOLOGO@temp#1#2{#2}
%    \end{macrocode}
%
%    \begin{macro}{\HOLOGO@ScaleBox@pdftex}
%    \begin{macrocode}
\HOLOGO@temp{pdftex}{%
  \def\HOLOGO@ScaleBox@pdftex#1#2#3{%
    \HOLOGO@pdfliteral{%
      q #1 0 0 #2 0 0 cm%
    }%
    #3%
    \HOLOGO@pdfliteral{%
      Q%
    }%
  }%
}
%    \end{macrocode}
%    \end{macro}
%    \begin{macro}{\HOLOGO@ScaleBox@dvips}
%    \begin{macrocode}
\HOLOGO@temp{dvips}{%
  \def\HOLOGO@ScaleBox@dvips#1#2#3{%
    \special{ps:%
      gsave %
      currentpoint %
      currentpoint translate %
      #1 #2 scale %
      neg exch neg exch translate%
    }%
    #3%
    \special{ps:%
      currentpoint %
      grestore %
      moveto%
    }%
  }%
}
%    \end{macrocode}
%    \end{macro}
%    \begin{macro}{\HOLOGO@ScaleBox@dvipdfm}
%    \begin{macrocode}
\HOLOGO@temp{dvipdfm}{%
  \let\HOLOGO@ScaleBox@dvipdfm\HOLOGO@ScaleBox@dvips
}
%    \end{macrocode}
%    \end{macro}
%    Since \hologo{XeTeX} v0.6.
%    \begin{macro}{\HOLOGO@ScaleBox@xetex}
%    \begin{macrocode}
\HOLOGO@temp{xetex}{%
  \def\HOLOGO@ScaleBox@xetex#1#2#3{%
    \special{x:gsave}%
    \special{x:scale #1 #2}%
    #3%
    \special{x:grestore}%
  }%
}
%    \end{macrocode}
%    \end{macro}
%    \begin{macro}{\HOLOGO@ScaleBox@vtex}
%    \begin{macrocode}
\HOLOGO@temp{vtex}{%
  \def\HOLOGO@ScaleBox@vtex#1#2#3{%
    \special{r(#1,0,0,#2,0,0}%
    #3%
    \special{r)}%
  }%
}
%    \end{macrocode}
%    \end{macro}
%
%    \begin{macrocode}
\HOLOGO@AtEnd%
%</package>
%    \end{macrocode}
%
% \section{Test}
%
% \subsection{Catcode checks for loading}
%
%    \begin{macrocode}
%<*test1>
%    \end{macrocode}
%    \begin{macrocode}
\catcode`\{=1 %
\catcode`\}=2 %
\catcode`\#=6 %
\catcode`\@=11 %
\expandafter\ifx\csname count@\endcsname\relax
  \countdef\count@=255 %
\fi
\expandafter\ifx\csname @gobble\endcsname\relax
  \long\def\@gobble#1{}%
\fi
\expandafter\ifx\csname @firstofone\endcsname\relax
  \long\def\@firstofone#1{#1}%
\fi
\expandafter\ifx\csname loop\endcsname\relax
  \expandafter\@firstofone
\else
  \expandafter\@gobble
\fi
{%
  \def\loop#1\repeat{%
    \def\body{#1}%
    \iterate
  }%
  \def\iterate{%
    \body
      \let\next\iterate
    \else
      \let\next\relax
    \fi
    \next
  }%
  \let\repeat=\fi
}%
\def\RestoreCatcodes{}
\count@=0 %
\loop
  \edef\RestoreCatcodes{%
    \RestoreCatcodes
    \catcode\the\count@=\the\catcode\count@\relax
  }%
\ifnum\count@<255 %
  \advance\count@ 1 %
\repeat

\def\RangeCatcodeInvalid#1#2{%
  \count@=#1\relax
  \loop
    \catcode\count@=15 %
  \ifnum\count@<#2\relax
    \advance\count@ 1 %
  \repeat
}
\def\RangeCatcodeCheck#1#2#3{%
  \count@=#1\relax
  \loop
    \ifnum#3=\catcode\count@
    \else
      \errmessage{%
        Character \the\count@\space
        with wrong catcode \the\catcode\count@\space
        instead of \number#3%
      }%
    \fi
  \ifnum\count@<#2\relax
    \advance\count@ 1 %
  \repeat
}
\def\space{ }
\expandafter\ifx\csname LoadCommand\endcsname\relax
  \def\LoadCommand{\input hologo.sty\relax}%
\fi
\def\Test{%
  \RangeCatcodeInvalid{0}{47}%
  \RangeCatcodeInvalid{58}{64}%
  \RangeCatcodeInvalid{91}{96}%
  \RangeCatcodeInvalid{123}{255}%
  \catcode`\@=12 %
  \catcode`\\=0 %
  \catcode`\%=14 %
  \LoadCommand
  \RangeCatcodeCheck{0}{36}{15}%
  \RangeCatcodeCheck{37}{37}{14}%
  \RangeCatcodeCheck{38}{47}{15}%
  \RangeCatcodeCheck{48}{57}{12}%
  \RangeCatcodeCheck{58}{63}{15}%
  \RangeCatcodeCheck{64}{64}{12}%
  \RangeCatcodeCheck{65}{90}{11}%
  \RangeCatcodeCheck{91}{91}{15}%
  \RangeCatcodeCheck{92}{92}{0}%
  \RangeCatcodeCheck{93}{96}{15}%
  \RangeCatcodeCheck{97}{122}{11}%
  \RangeCatcodeCheck{123}{255}{15}%
  \RestoreCatcodes
}
\Test
\csname @@end\endcsname
\end
%    \end{macrocode}
%    \begin{macrocode}
%</test1>
%    \end{macrocode}
%
% \subsection{Spacefactor}
%
%    The space factor must be 1000 after a logo. If it is greater 1000
%    then the following space is a space after a sentence closing point.
%    If the space factor is smaller 1000 then an immediate following
%    dot is interpreted as abbreviation, not sentence closing point.
%
%    \begin{macrocode}
%<*test-spacefactor>
\NeedsTeXFormat{LaTeX2e}
\documentclass{article}
\usepackage{hologo}[2016/05/12]
\usepackage{kvsetkeys}
\usepackage{qstest}
\IncludeTests{*}
\LogTests{log}{*}{*}
\begin{document}
\begin{qstest}{spacefactor}{spacefactor}
\newcommand*{\Test}[1]{%
  \sbox0{%
    \hologo{#1}%
    \Expect*{1000 (#1)}*{\the\spacefactor\space(#1)}%
  }%
}%
\makeatletter
\def\TestList{}
\def\hologoEntry#1#2#3{%
  \edef\TestList{%
    \ifx\TestList\@empty
    \else
      \TestList,%
    \fi
    #1%
    \ifx\\#2\\%
    \else
      ={variant=#2}%
    \fi
  }%
}
\hologoList
\expandafter\kv@parse@normalized\expandafter{%
  \TestList
}{%
  \begingroup
    \let\@logo=\kv@key
    \ifx\kv@value\relax
    \else
      \expandafter\hologoLogoSetup\expandafter\@logo\expandafter{%
        \kv@value
      }%
    \fi
    \Test\@logo
  \endgroup
  \@gobbletwo
}
\end{qstest}
\end{document}
%</test-spacefactor>
%    \end{macrocode}
%
% \subsection{Complete list}
%
%    \begin{macrocode}
%<*test-list>
\NeedsTeXFormat{LaTeX2e}
\documentclass[12pt,a4paper]{article}
\usepackage{hologo}[2016/05/12]
\usepackage[T1]{fontenc}
\usepackage{lmodern}
\usepackage{parskip}
\usepackage[unicode]{hyperref}[2011/09/28]
\usepackage{bookmark}[2011/09/19]
\bookmarksetup{%
  numbered,%
  open,%
  openlevel=2,%
}
\renewcommand*{\contentsname}{List of logos}
\begin{document}
\tableofcontents
\def\TestFont#1#2#3#4#5#6{%
  \begingroup
    \usefont{#3}{#4}{#5}{#6}%
    \HologoVariant{#1}{#2}/\hologoVariant{#1}{#2}%
    \quad
    \begingroup\scriptsize\hologoVariant{#1}{#2}\endgroup
    \quad
  \endgroup
  (#3/#4/#5/#6)%
  \par
}
\makeatletter
\def\hologoEntry#1#2#3{%
  \section{%
    \HologoVariant{#1}{#2}/\hologoVariant{#1}{#2} %
    {[#1\ifx\\#2\\\else\space(#2)\fi]}% hash-ok
  }% braces around [] because of bug in tex4ht
  \begingroup
    \hypersetup{unicode=false}%
    \bookmark[%
      dest=\@currentHref,%
      rellevel=1,%
      keeplevel,%
    ]{%
      \HologoVariant{#1}{#2}/\hologoVariant{#1}{#2} %
      (PDFDocEncoding)%
    }%
  \endgroup
  \TestFont{#1}{#2}{OT1}{cmr}{m}{n}%
  \TestFont{#1}{#2}{OT1}{cmss}{m}{n}%
  \TestFont{#1}{#2}{OT1}{cmr}{b}{n}%
  \TestFont{#1}{#2}{OT1}{cmr}{m}{it}%
  \TestFont{#1}{#2}{OT1}{cmtt}{m}{n}%
  \TestFont{#1}{#2}{T1}{lmr}{m}{n}%
  \TestFont{#1}{#2}{T1}{lmss}{m}{n}%
  \TestFont{#1}{#2}{T1}{lmr}{b}{n}%
  \TestFont{#1}{#2}{T1}{lmr}{m}{it}%
  \TestFont{#1}{#2}{T1}{lmtt}{m}{n}%
  \TestFont{#1}{#2}{T1}{lmvtt}{m}{n}%
  \TestFont{#1}{#2}{T1}{qtm}{m}{n}%
  \TestFont{#1}{#2}{T1}{qhv}{m}{n}%
  \TestFont{#1}{#2}{T1}{qtm}{b}{n}%
  \TestFont{#1}{#2}{T1}{qtm}{m}{it}%
  \TestFont{#1}{#2}{T1}{qcr}{m}{n}%
  \newpage
}
\makeatother
\hologoList
\end{document}
%</test-list>
%    \end{macrocode}
%
% \section{Installation}
%
% \subsection{Download}
%
% \paragraph{Package.} This package is available on
% CTAN\footnote{\url{ftp://ftp.ctan.org/tex-archive/}}:
% \begin{description}
% \item[\CTAN{macros/latex/contrib/oberdiek/hologo.dtx}] The source file.
% \item[\CTAN{macros/latex/contrib/oberdiek/hologo.pdf}] Documentation.
% \end{description}
%
%
% \paragraph{Bundle.} All the packages of the bundle `oberdiek'
% are also available in a TDS compliant ZIP archive. There
% the packages are already unpacked and the documentation files
% are generated. The files and directories obey the TDS standard.
% \begin{description}
% \item[\CTAN{install/macros/latex/contrib/oberdiek.tds.zip}]
% \end{description}
% \emph{TDS} refers to the standard ``A Directory Structure
% for \TeX\ Files'' (\CTAN{tds/tds.pdf}). Directories
% with \xfile{texmf} in their name are usually organized this way.
%
% \subsection{Bundle installation}
%
% \paragraph{Unpacking.} Unpack the \xfile{oberdiek.tds.zip} in the
% TDS tree (also known as \xfile{texmf} tree) of your choice.
% Example (linux):
% \begin{quote}
%   |unzip oberdiek.tds.zip -d ~/texmf|
% \end{quote}
%
% \paragraph{Script installation.}
% Check the directory \xfile{TDS:scripts/oberdiek/} for
% scripts that need further installation steps.
% Package \xpackage{attachfile2} comes with the Perl script
% \xfile{pdfatfi.pl} that should be installed in such a way
% that it can be called as \texttt{pdfatfi}.
% Example (linux):
% \begin{quote}
%   |chmod +x scripts/oberdiek/pdfatfi.pl|\\
%   |cp scripts/oberdiek/pdfatfi.pl /usr/local/bin/|
% \end{quote}
%
% \subsection{Package installation}
%
% \paragraph{Unpacking.} The \xfile{.dtx} file is a self-extracting
% \docstrip\ archive. The files are extracted by running the
% \xfile{.dtx} through \plainTeX:
% \begin{quote}
%   \verb|tex hologo.dtx|
% \end{quote}
%
% \paragraph{TDS.} Now the different files must be moved into
% the different directories in your installation TDS tree
% (also known as \xfile{texmf} tree):
% \begin{quote}
% \def\t{^^A
% \begin{tabular}{@{}>{\ttfamily}l@{ $\rightarrow$ }>{\ttfamily}l@{}}
%   hologo.sty & tex/generic/oberdiek/hologo.sty\\
%   hologo.pdf & doc/latex/oberdiek/hologo.pdf\\
%   example/hologo-example.tex & doc/latex/oberdiek/example/hologo-example.tex\\
%   test/hologo-test1.tex & doc/latex/oberdiek/test/hologo-test1.tex\\
%   test/hologo-test-spacefactor.tex & doc/latex/oberdiek/test/hologo-test-spacefactor.tex\\
%   test/hologo-test-list.tex & doc/latex/oberdiek/test/hologo-test-list.tex\\
%   hologo.dtx & source/latex/oberdiek/hologo.dtx\\
% \end{tabular}^^A
% }^^A
% \sbox0{\t}^^A
% \ifdim\wd0>\linewidth
%   \begingroup
%     \advance\linewidth by\leftmargin
%     \advance\linewidth by\rightmargin
%   \edef\x{\endgroup
%     \def\noexpand\lw{\the\linewidth}^^A
%   }\x
%   \def\lwbox{^^A
%     \leavevmode
%     \hbox to \linewidth{^^A
%       \kern-\leftmargin\relax
%       \hss
%       \usebox0
%       \hss
%       \kern-\rightmargin\relax
%     }^^A
%   }^^A
%   \ifdim\wd0>\lw
%     \sbox0{\small\t}^^A
%     \ifdim\wd0>\linewidth
%       \ifdim\wd0>\lw
%         \sbox0{\footnotesize\t}^^A
%         \ifdim\wd0>\linewidth
%           \ifdim\wd0>\lw
%             \sbox0{\scriptsize\t}^^A
%             \ifdim\wd0>\linewidth
%               \ifdim\wd0>\lw
%                 \sbox0{\tiny\t}^^A
%                 \ifdim\wd0>\linewidth
%                   \lwbox
%                 \else
%                   \usebox0
%                 \fi
%               \else
%                 \lwbox
%               \fi
%             \else
%               \usebox0
%             \fi
%           \else
%             \lwbox
%           \fi
%         \else
%           \usebox0
%         \fi
%       \else
%         \lwbox
%       \fi
%     \else
%       \usebox0
%     \fi
%   \else
%     \lwbox
%   \fi
% \else
%   \usebox0
% \fi
% \end{quote}
% If you have a \xfile{docstrip.cfg} that configures and enables \docstrip's
% TDS installing feature, then some files can already be in the right
% place, see the documentation of \docstrip.
%
% \subsection{Refresh file name databases}
%
% If your \TeX~distribution
% (\teTeX, \mikTeX, \dots) relies on file name databases, you must refresh
% these. For example, \teTeX\ users run \verb|texhash| or
% \verb|mktexlsr|.
%
% \subsection{Some details for the interested}
%
% \paragraph{Attached source.}
%
% The PDF documentation on CTAN also includes the
% \xfile{.dtx} source file. It can be extracted by
% AcrobatReader 6 or higher. Another option is \textsf{pdftk},
% e.g. unpack the file into the current directory:
% \begin{quote}
%   \verb|pdftk hologo.pdf unpack_files output .|
% \end{quote}
%
% \paragraph{Unpacking with \LaTeX.}
% The \xfile{.dtx} chooses its action depending on the format:
% \begin{description}
% \item[\plainTeX:] Run \docstrip\ and extract the files.
% \item[\LaTeX:] Generate the documentation.
% \end{description}
% If you insist on using \LaTeX\ for \docstrip\ (really,
% \docstrip\ does not need \LaTeX), then inform the autodetect routine
% about your intention:
% \begin{quote}
%   \verb|latex \let\install=y% \iffalse meta-comment
%
% File: hologo.dtx
% Version: 2016/05/12 v1.11
% Info: A logo collection with bookmark support
%
% Copyright (C) 2010-2012 by
%    Heiko Oberdiek <heiko.oberdiek at googlemail.com>
%
% This work may be distributed and/or modified under the
% conditions of the LaTeX Project Public License, either
% version 1.3c of this license or (at your option) any later
% version. This version of this license is in
%    http://www.latex-project.org/lppl/lppl-1-3c.txt
% and the latest version of this license is in
%    http://www.latex-project.org/lppl.txt
% and version 1.3 or later is part of all distributions of
% LaTeX version 2005/12/01 or later.
%
% This work has the LPPL maintenance status "maintained".
%
% This Current Maintainer of this work is Heiko Oberdiek.
%
% The Base Interpreter refers to any `TeX-Format',
% because some files are installed in TDS:tex/generic//.
%
% This work consists of the main source file hologo.dtx
% and the derived files
%    hologo.sty, hologo.pdf, hologo.ins, hologo.drv, hologo-example.tex,
%    hologo-test1.tex, hologo-test-spacefactor.tex,
%    hologo-test-list.tex.
%
% Distribution:
%    CTAN:macros/latex/contrib/oberdiek/hologo.dtx
%    CTAN:macros/latex/contrib/oberdiek/hologo.pdf
%
% Unpacking:
%    (a) If hologo.ins is present:
%           tex hologo.ins
%    (b) Without hologo.ins:
%           tex hologo.dtx
%    (c) If you insist on using LaTeX
%           latex \let\install=y\input{hologo.dtx}
%        (quote the arguments according to the demands of your shell)
%
% Documentation:
%    (a) If hologo.drv is present:
%           latex hologo.drv
%    (b) Without hologo.drv:
%           latex hologo.dtx; ...
%    The class ltxdoc loads the configuration file ltxdoc.cfg
%    if available. Here you can specify further options, e.g.
%    use A4 as paper format:
%       \PassOptionsToClass{a4paper}{article}
%
%    Programm calls to get the documentation (example):
%       pdflatex hologo.dtx
%       makeindex -s gind.ist hologo.idx
%       pdflatex hologo.dtx
%       makeindex -s gind.ist hologo.idx
%       pdflatex hologo.dtx
%
% Installation:
%    TDS:tex/generic/oberdiek/hologo.sty
%    TDS:doc/latex/oberdiek/hologo.pdf
%    TDS:doc/latex/oberdiek/example/hologo-example.tex
%    TDS:doc/latex/oberdiek/test/hologo-test1.tex
%    TDS:doc/latex/oberdiek/test/hologo-test-spacefactor.tex
%    TDS:doc/latex/oberdiek/test/hologo-test-list.tex
%    TDS:source/latex/oberdiek/hologo.dtx
%
%<*ignore>
\begingroup
  \catcode123=1 %
  \catcode125=2 %
  \def\x{LaTeX2e}%
\expandafter\endgroup
\ifcase 0\ifx\install y1\fi\expandafter
         \ifx\csname processbatchFile\endcsname\relax\else1\fi
         \ifx\fmtname\x\else 1\fi\relax
\else\csname fi\endcsname
%</ignore>
%<*install>
\input docstrip.tex
\Msg{************************************************************************}
\Msg{* Installation}
\Msg{* Package: hologo 2016/05/12 v1.11 A logo collection with bookmark support (HO)}
\Msg{************************************************************************}

\keepsilent
\askforoverwritefalse

\let\MetaPrefix\relax
\preamble

This is a generated file.

Project: hologo
Version: 2016/05/12 v1.11

Copyright (C) 2010-2012 by
   Heiko Oberdiek <heiko.oberdiek at googlemail.com>

This work may be distributed and/or modified under the
conditions of the LaTeX Project Public License, either
version 1.3c of this license or (at your option) any later
version. This version of this license is in
   http://www.latex-project.org/lppl/lppl-1-3c.txt
and the latest version of this license is in
   http://www.latex-project.org/lppl.txt
and version 1.3 or later is part of all distributions of
LaTeX version 2005/12/01 or later.

This work has the LPPL maintenance status "maintained".

This Current Maintainer of this work is Heiko Oberdiek.

The Base Interpreter refers to any `TeX-Format',
because some files are installed in TDS:tex/generic//.

This work consists of the main source file hologo.dtx
and the derived files
   hologo.sty, hologo.pdf, hologo.ins, hologo.drv, hologo-example.tex,
   hologo-test1.tex, hologo-test-spacefactor.tex,
   hologo-test-list.tex.

\endpreamble
\let\MetaPrefix\DoubleperCent

\generate{%
  \file{hologo.ins}{\from{hologo.dtx}{install}}%
  \file{hologo.drv}{\from{hologo.dtx}{driver}}%
  \usedir{tex/generic/oberdiek}%
  \file{hologo.sty}{\from{hologo.dtx}{package}}%
  \usedir{doc/latex/oberdiek/example}%
  \file{hologo-example.tex}{\from{hologo.dtx}{example}}%
  \usedir{doc/latex/oberdiek/test}%
  \file{hologo-test1.tex}{\from{hologo.dtx}{test1}}%
  \file{hologo-test-spacefactor.tex}{\from{hologo.dtx}{test-spacefactor}}%
  \file{hologo-test-list.tex}{\from{hologo.dtx}{test-list}}%
  \nopreamble
  \nopostamble
  \usedir{source/latex/oberdiek/catalogue}%
  \file{hologo.xml}{\from{hologo.dtx}{catalogue}}%
}

\catcode32=13\relax% active space
\let =\space%
\Msg{************************************************************************}
\Msg{*}
\Msg{* To finish the installation you have to move the following}
\Msg{* file into a directory searched by TeX:}
\Msg{*}
\Msg{*     hologo.sty}
\Msg{*}
\Msg{* To produce the documentation run the file `hologo.drv'}
\Msg{* through LaTeX.}
\Msg{*}
\Msg{* Happy TeXing!}
\Msg{*}
\Msg{************************************************************************}

\endbatchfile
%</install>
%<*ignore>
\fi
%</ignore>
%<*driver>
\NeedsTeXFormat{LaTeX2e}
\ProvidesFile{hologo.drv}%
  [2016/05/12 v1.11 A logo collection with bookmark support (HO)]%
\documentclass{ltxdoc}
\usepackage{holtxdoc}[2011/11/22]
\usepackage{hologo}[2016/05/12]
\usepackage{longtable}
\usepackage{array}
\usepackage{paralist}
%\usepackage[T1]{fontenc}
%\usepackage{lmodern}
\begin{document}
  \DocInput{hologo.dtx}%
\end{document}
%</driver>
% \fi
%
%
% \CharacterTable
%  {Upper-case    \A\B\C\D\E\F\G\H\I\J\K\L\M\N\O\P\Q\R\S\T\U\V\W\X\Y\Z
%   Lower-case    \a\b\c\d\e\f\g\h\i\j\k\l\m\n\o\p\q\r\s\t\u\v\w\x\y\z
%   Digits        \0\1\2\3\4\5\6\7\8\9
%   Exclamation   \!     Double quote  \"     Hash (number) \#
%   Dollar        \$     Percent       \%     Ampersand     \&
%   Acute accent  \'     Left paren    \(     Right paren   \)
%   Asterisk      \*     Plus          \+     Comma         \,
%   Minus         \-     Point         \.     Solidus       \/
%   Colon         \:     Semicolon     \;     Less than     \<
%   Equals        \=     Greater than  \>     Question mark \?
%   Commercial at \@     Left bracket  \[     Backslash     \\
%   Right bracket \]     Circumflex    \^     Underscore    \_
%   Grave accent  \`     Left brace    \{     Vertical bar  \|
%   Right brace   \}     Tilde         \~}
%
% \GetFileInfo{hologo.drv}
%
% \title{The \xpackage{hologo} package}
% \date{2016/05/12 v1.11}
% \author{Heiko Oberdiek\\\xemail{heiko.oberdiek at googlemail.com}}
%
% \maketitle
%
% \begin{abstract}
% This package starts a collection of logos with support for bookmarks
% strings.
% \end{abstract}
%
% \tableofcontents
%
% \section{Documentation}
%
% \subsection{Logo macros}
%
% \begin{declcs}{hologo} \M{name}
% \end{declcs}
% Macro \cs{hologo} sets the logo with name \meta{name}.
% The following table shows the supported names.
%
% \begingroup
%   \def\hologoEntry#1#2#3{^^A
%     #1&#2&\hologoLogoSetup{#1}{variant=#2}\hologo{#1}&#3\tabularnewline
%   }
%   \begin{longtable}{>{\ttfamily}l>{\ttfamily}lll}
%     \rmfamily\bfseries{name} & \rmfamily\bfseries variant
%     & \bfseries logo & \bfseries since\\
%     \hline
%     \endhead
%     \hologoList
%   \end{longtable}
% \endgroup
%
% \begin{declcs}{Hologo} \M{name}
% \end{declcs}
% Macro \cs{Hologo} starts the logo \meta{name} with an uppercase
% letter. As an exception small greek letters are not converted
% to uppercase. Examples, see \hologo{eTeX} and \hologo{ExTeX}.
%
% \subsection{Setup macros}
%
% The package does not support package options, but the following
% setup macros can be used to set options.
%
% \begin{declcs}{hologoSetup} \M{key value list}
% \end{declcs}
% Macro \cs{hologoSetup} sets global options.
%
% \begin{declcs}{hologoLogoSetup} \M{logo} \M{key value list}
% \end{declcs}
% Some options can also be used to configure a logo.
% These settings take precedence over global option settings.
%
% \subsection{Options}\label{sec:options}
%
% There are boolean and string options:
% \begin{description}
% \item[Boolean option:]
% It takes |true| or |false|
% as value. If the value is omitted, then |true| is used.
% \item[String option:]
% A value must be given as string. (But the string might be empty.)
% \end{description}
% The following options can be used both in \cs{hologoSetup}
% and \cs{hologoLogoSetup}:
% \begin{description}
% \def\entry#1{\item[\xoption{#1}:]}
% \entry{break}
%   enables or disables line breaks inside the logo. This setting is
%   refined by options \xoption{hyphenbreak}, \xoption{spacebreak}
%   or \xoption{discretionarybreak}.
%   Default is |false|.
% \entry{hyphenbreak}
%   enables or disables the line break right after the hyphen character.
% \entry{spacebreak}
%   enables or disables line breaks at space characters.
% \entry{discretionarybreak}
%   enables or disables line breaks at hyphenation points
%   (inserted by \cs{-}).
% \end{description}
% Macro \cs{hologoLogoSetup} also knows:
% \begin{description}
% \item[\xoption{variant}:]
%   This is a string option. It specifies a variant of a logo that
%   must exist. An empty string selects the package default variant.
% \end{description}
% Example:
% \begin{quote}
%   |\hologoSetup{break=false}|\\
%   |\hologoLogoSetup{plainTeX}{variant=hyphen,hyphenbreak}|\\
%   Then ``plain-\TeX'' contains one break point after the hyphen.
% \end{quote}
%
% \subsection{Driver options}
%
% Sometimes graphical operations are needed to construct some
% glyphs (e.g.\ \hologo{XeTeX}). If package \xpackage{graphics}
% or package \xpackage{pgf} are found, then the macros are taken
% from there. Otherwise the packge defines its own operations
% and therefore needs the driver information. Many drivers are
% detected automatically (\hologo{pdfTeX}/\hologo{LuaTeX}
% in PDF mode, \hologo{XeTeX}, \hologo{VTeX}). These have precedence
% over a driver option. The driver can be given as package option
% or using \cs{hologoDriverSetup}.
% The following list contains the recognized driver options:
% \begin{itemize}
% \item \xoption{pdftex}, \xoption{luatex}
% \item \xoption{dvipdfm}, \xoption{dvipdfmx}
% \item \xoption{dvips}, \xoption{dvipsone}, \xoption{xdvi}
% \item \xoption{xetex}
% \item \xoption{vtex}
% \end{itemize}
% The left driver of a line is the driver name that is used internally.
% The following names are aliases for drivers that use the
% same method. Therefore the entry in the \xext{log} file for
% the used driver prints the internally used driver name.
% \begin{description}
% \item[\xoption{driverfallback}:]
%   This option expects a driver that is used,
%   if the driver could not be detected automatically.
% \end{description}
%
% \begin{declcs}{hologoDriverSetup} \M{driver option}
% \end{declcs}
% The driver can also be configured after package loading
% using \cs{hologoDriverSetup}, also the way for \hologo{plainTeX}
% to setup the driver.
%
% \subsection{Font setup}
%
% Some logos require a special font, but should also be usable by
% \hologo{plainTeX}. Therefore the package provides some ways
% to influence the font settings. The options below
% take font settings as values. Both font commands
% such as \cs{sffamily} and macros that take one argument
% like \cs{textsf} can be used.
%
% \begin{declcs}{hologoFontSetup} \M{key value list}
% \end{declcs}
% Macro \cs{hologoFontSetup} sets the fonts for all logos.
% Supported keys:
% \begin{description}
% \def\entry#1{\item[\xoption{#1}:]}
% \entry{general}
%   This font is used for all logos. The default is empty.
%   That means no special font is used.
% \entry{bibsf}
%   This font is used for
%   {\hologoLogoSetup{BibTeX}{variant=sf}\hologo{BibTeX}}
%   with variant \xoption{sf}.
% \entry{rm}
%   This font is a serif font. It is used for \hologo{ExTeX}.
% \entry{sc}
%   This font specifies a small caps font. It is used for
%   {\hologoLogoSetup{BibTeX}{variant=sc}\hologo{BibTeX}}
%   with variant \xoption{sc}.
% \entry{sf}
%   This font specifies a sans serif font. The default
%   is \cs{sffamily}, then \cs{sf} is tried. Otherwise
%   a warning is given. It is used by \hologo{KOMAScript}.
% \entry{sy}
%   This is the font for math symbols (e.g. cmsy).
%   It is used by \hologo{AmS}, \hologo{NTS}, \hologo{ExTeX}.
% \entry{logo}
%   \hologo{METAFONT} and \hologo{METAPOST} are using that font.
%   In \hologo{LaTeX} \cs{logofamily} is used and
%   the definitions of package \xpackage{mflogo} are used
%   if the package is not loaded.
%   Otherwise the \cs{tenlogo} is used and defined
%   if it does not already exists.
% \end{description}
%
% \begin{declcs}{hologoLogoFontSetup} \M{logo} \M{key value list}
% \end{declcs}
% Fonts can also be set for a logo or logo component separately,
% see the following list.
% The keys are the same as for \cs{hologoFontSetup}.
%
% \begin{longtable}{>{\ttfamily}l>{\sffamily}ll}
%   \meta{logo} & keys & result\\
%   \hline
%   \endhead
%   BibTeX & bibsf & {\hologoLogoSetup{BibTeX}{variant=sf}\hologo{BibTeX}}\\[.5ex]
%   BibTeX & sc & {\hologoLogoSetup{BibTeX}{variant=sc}\hologo{BibTeX}}\\[.5ex]
%   ExTeX & rm & \hologo{ExTeX}\\
%   SliTeX & rm & \hologo{SliTeX}\\[.5ex]
%   AmS & sy & \hologo{AmS}\\
%   ExTeX & sy & \hologo{ExTeX}\\
%   NTS & sy & \hologo{NTS}\\[.5ex]
%   KOMAScript & sf & \hologo{KOMAScript}\\[.5ex]
%   METAFONT & logo & \hologo{METAFONT}\\
%   METAPOST & logo & \hologo{METAPOST}\\[.5ex]
%   SliTeX & sc \hologo{SliTeX}
% \end{longtable}
%
% \subsubsection{Font order}
%
% For all logos the font \xoption{general} is applied first.
% Example:
%\begin{quote}
%|\hologoFontSetup{general=\color{red}}|
%\end{quote}
% will print red logos.
% Then if the font uses a special font \xoption{sf}, for example,
% the font is applied that is setup by \cs{hologoLogoFontSetup}.
% If this font is not setup, then the common font setup
% by \cs{hologoFontSetup} is used. Otherwise a warning is given,
% that there is no font configured.
%
% \subsection{Additional user macros}
%
% Usually a variant of a logo is configured by using
% \cs{hologoLogoSetup}, because it is bad style to mix
% different variants of the same logo in the same text.
% There the following macros are a convenience for testing.
%
% \begin{declcs}{hologoVariant} \M{name} \M{variant}\\
%   \cs{HologoVariant} \M{name} \M{variant}
% \end{declcs}
% Logo \meta{name} is set using \meta{variant} that specifies
% explicitely which variant of the macro is used. If the argument
% is empty, then the default form of the logo is used
% (configurable by \cs{hologoLogoSetup}).
%
% \cs{HologoVariant} is used if the logo is set in a context
% that needs an uppercase first letter (beginning of a sentence, \dots).
%
% \begin{declcs}{hologoList}\\
%   \cs{hologoEntry} \M{logo} \M{variant} \M{since}
% \end{declcs}
% Macro \cs{hologoList} contains all logos that are provided
% by the package including variants. The list consists of calls
% of \cs{hologoEntry} with three arguments starting with the
% logo name \meta{logo} and its variant \meta{variant}. An empty
% variant means the current default. Argument \meta{since} specifies
% with version of the package \xpackage{hologo} is needed to get
% the logo. If the logo is fixed, then the date gets updated.
% Therefore the date \meta{since} is not exactly the date of
% the first introduction, but rather the date of the latest fix.
%
% Before \cs{hologoList} can be used, macro \cs{hologoEntry} needs
% a definition. The example file in section \ref{sec:example}
% shows applications of \cs{hologoList}.
%
% \subsection{Supported contexts}
%
% Macros \cs{hologo} and friends support special contexts:
% \begin{itemize}
% \item \hologo{LaTeX}'s protection mechanism.
% \item Bookmarks of package \xpackage{hyperref}.
% \item Package \xpackage{tex4ht}.
% \item The macros can be used inside \cs{csname} constructs,
%   if \cs{ifincsname} is available (\hologo{pdfTeX}, \hologo{XeTeX},
%   \hologo{LuaTeX}).
% \end{itemize}
%
% \subsection{Example}
% \label{sec:example}
%
% The following example prints the logos in different fonts.
%    \begin{macrocode}
%<*example>
%<<verbatim
\NeedsTeXFormat{LaTeX2e}
\documentclass[a4paper]{article}
\usepackage[
  hmargin=20mm,
  vmargin=20mm,
]{geometry}
\pagestyle{empty}
\usepackage{hologo}[2016/05/12]
\usepackage{longtable}
\usepackage{array}
\setlength{\extrarowheight}{2pt}
\usepackage[T1]{fontenc}
\usepackage{lmodern}
\usepackage{pdflscape}
\usepackage[
  pdfencoding=auto,
]{hyperref}
\hypersetup{
  pdfauthor={Heiko Oberdiek},
  pdftitle={Example for package `hologo'},
  pdfsubject={Logos with fonts lmr, lmss, qtm, qpl, qhv},
}
\usepackage{bookmark}

% Print the logo list on the console

\begingroup
  \typeout{}%
  \typeout{*** Begin of logo list ***}%
  \newcommand*{\hologoEntry}[3]{%
    \typeout{#1 \ifx\\#2\\\else(#2) \fi[#3]}%
  }%
  \hologoList
  \typeout{*** End of logo list ***}%
  \typeout{}%
\endgroup

\begin{document}
\begin{landscape}

  \section{Example file for package `hologo'}

  % Table for font names

  \begin{longtable}{>{\bfseries}ll}
    \textbf{font} & \textbf{Font name}\\
    \hline
    lmr & Latin Modern Roman\\
    lmss & Latin Modern Sans\\
    qtm & \TeX\ Gyre Termes\\
    qhv & \TeX\ Gyre Heros\\
    qpl & \TeX\ Gyre Pagella\\
  \end{longtable}

  % Logo list with logos in different fonts

  \begingroup
    \newcommand*{\SetVariant}[2]{%
      \ifx\\#2\\%
      \else
        \hologoLogoSetup{#1}{variant=#2}%
      \fi
    }%
    \newcommand*{\hologoEntry}[3]{%
      \SetVariant{#1}{#2}%
      \raisebox{1em}[0pt][0pt]{\hypertarget{#1@#2}{}}%
      \bookmark[%
        dest={#1@#2},%
      ]{%
        #1\ifx\\#2\\\else\space(#2)\fi: \Hologo{#1}, \hologo{#1} %
        [Unicode]%
      }%
      \hypersetup{unicode=false}%
      \bookmark[%
        dest={#1@#2},%
      ]{%
        #1\ifx\\#2\\\else\space(#2)\fi: \Hologo{#1}, \hologo{#1} %
        [PDFDocEncoding]%
      }%
      \texttt{#1}%
      &%
      \texttt{#2}%
      &%
      \Hologo{#1}%
      &%
      \SetVariant{#1}{#2}%
      \hologo{#1}%
      &%
      \SetVariant{#1}{#2}%
      \fontfamily{qtm}\selectfont
      \hologo{#1}%
      &%
      \SetVariant{#1}{#2}%
      \fontfamily{qpl}\selectfont
      \hologo{#1}%
      &%
      \SetVariant{#1}{#2}%
      \textsf{\hologo{#1}}%
      &%
      \SetVariant{#1}{#2}%
      \fontfamily{qhv}\selectfont
      \hologo{#1}%
      \tabularnewline
    }%
    \begin{longtable}{llllllll}%
      \textbf{\textit{logo}} & \textbf{\textit{variant}} &
      \texttt{\string\Hologo} &
      \textbf{lmr} & \textbf{qtm} & \textbf{qpl} &
      \textbf{lmss} & \textbf{qhv}
      \tabularnewline
      \hline
      \endhead
      \hologoList
    \end{longtable}%
  \endgroup

\end{landscape}
\end{document}
%verbatim
%</example>
%    \end{macrocode}
%
% \StopEventually{
% }
%
% \section{Implementation}
%    \begin{macrocode}
%<*package>
%    \end{macrocode}
%    Reload check, especially if the package is not used with \LaTeX.
%    \begin{macrocode}
\begingroup\catcode61\catcode48\catcode32=10\relax%
  \catcode13=5 % ^^M
  \endlinechar=13 %
  \catcode35=6 % #
  \catcode39=12 % '
  \catcode44=12 % ,
  \catcode45=12 % -
  \catcode46=12 % .
  \catcode58=12 % :
  \catcode64=11 % @
  \catcode123=1 % {
  \catcode125=2 % }
  \expandafter\let\expandafter\x\csname ver@hologo.sty\endcsname
  \ifx\x\relax % plain-TeX, first loading
  \else
    \def\empty{}%
    \ifx\x\empty % LaTeX, first loading,
      % variable is initialized, but \ProvidesPackage not yet seen
    \else
      \expandafter\ifx\csname PackageInfo\endcsname\relax
        \def\x#1#2{%
          \immediate\write-1{Package #1 Info: #2.}%
        }%
      \else
        \def\x#1#2{\PackageInfo{#1}{#2, stopped}}%
      \fi
      \x{hologo}{The package is already loaded}%
      \aftergroup\endinput
    \fi
  \fi
\endgroup%
%    \end{macrocode}
%    Package identification:
%    \begin{macrocode}
\begingroup\catcode61\catcode48\catcode32=10\relax%
  \catcode13=5 % ^^M
  \endlinechar=13 %
  \catcode35=6 % #
  \catcode39=12 % '
  \catcode40=12 % (
  \catcode41=12 % )
  \catcode44=12 % ,
  \catcode45=12 % -
  \catcode46=12 % .
  \catcode47=12 % /
  \catcode58=12 % :
  \catcode64=11 % @
  \catcode91=12 % [
  \catcode93=12 % ]
  \catcode123=1 % {
  \catcode125=2 % }
  \expandafter\ifx\csname ProvidesPackage\endcsname\relax
    \def\x#1#2#3[#4]{\endgroup
      \immediate\write-1{Package: #3 #4}%
      \xdef#1{#4}%
    }%
  \else
    \def\x#1#2[#3]{\endgroup
      #2[{#3}]%
      \ifx#1\@undefined
        \xdef#1{#3}%
      \fi
      \ifx#1\relax
        \xdef#1{#3}%
      \fi
    }%
  \fi
\expandafter\x\csname ver@hologo.sty\endcsname
\ProvidesPackage{hologo}%
  [2016/05/12 v1.11 A logo collection with bookmark support (HO)]%
%    \end{macrocode}
%
%    \begin{macrocode}
\begingroup\catcode61\catcode48\catcode32=10\relax%
  \catcode13=5 % ^^M
  \endlinechar=13 %
  \catcode123=1 % {
  \catcode125=2 % }
  \catcode64=11 % @
  \def\x{\endgroup
    \expandafter\edef\csname HOLOGO@AtEnd\endcsname{%
      \endlinechar=\the\endlinechar\relax
      \catcode13=\the\catcode13\relax
      \catcode32=\the\catcode32\relax
      \catcode35=\the\catcode35\relax
      \catcode61=\the\catcode61\relax
      \catcode64=\the\catcode64\relax
      \catcode123=\the\catcode123\relax
      \catcode125=\the\catcode125\relax
    }%
  }%
\x\catcode61\catcode48\catcode32=10\relax%
\catcode13=5 % ^^M
\endlinechar=13 %
\catcode35=6 % #
\catcode64=11 % @
\catcode123=1 % {
\catcode125=2 % }
\def\TMP@EnsureCode#1#2{%
  \edef\HOLOGO@AtEnd{%
    \HOLOGO@AtEnd
    \catcode#1=\the\catcode#1\relax
  }%
  \catcode#1=#2\relax
}
\TMP@EnsureCode{10}{12}% ^^J
\TMP@EnsureCode{33}{12}% !
\TMP@EnsureCode{34}{12}% "
\TMP@EnsureCode{36}{3}% $
\TMP@EnsureCode{38}{4}% &
\TMP@EnsureCode{39}{12}% '
\TMP@EnsureCode{40}{12}% (
\TMP@EnsureCode{41}{12}% )
\TMP@EnsureCode{42}{12}% *
\TMP@EnsureCode{43}{12}% +
\TMP@EnsureCode{44}{12}% ,
\TMP@EnsureCode{45}{12}% -
\TMP@EnsureCode{46}{12}% .
\TMP@EnsureCode{47}{12}% /
\TMP@EnsureCode{58}{12}% :
\TMP@EnsureCode{59}{12}% ;
\TMP@EnsureCode{60}{12}% <
\TMP@EnsureCode{62}{12}% >
\TMP@EnsureCode{63}{12}% ?
\TMP@EnsureCode{91}{12}% [
\TMP@EnsureCode{93}{12}% ]
\TMP@EnsureCode{94}{7}% ^ (superscript)
\TMP@EnsureCode{95}{8}% _ (subscript)
\TMP@EnsureCode{96}{12}% `
\TMP@EnsureCode{124}{12}% |
\edef\HOLOGO@AtEnd{%
  \HOLOGO@AtEnd
  \escapechar\the\escapechar\relax
  \noexpand\endinput
}
\escapechar=92 %
%    \end{macrocode}
%
% \subsection{Logo list}
%
%    \begin{macro}{\hologoList}
%    \begin{macrocode}
\def\hologoList{%
  \hologoEntry{(La)TeX}{}{2011/10/01}%
  \hologoEntry{AmSLaTeX}{}{2010/04/16}%
  \hologoEntry{AmSTeX}{}{2010/04/16}%
  \hologoEntry{biber}{}{2011/10/01}%
  \hologoEntry{BibTeX}{}{2011/10/01}%
  \hologoEntry{BibTeX}{sf}{2011/10/01}%
  \hologoEntry{BibTeX}{sc}{2011/10/01}%
  \hologoEntry{BibTeX8}{}{2011/11/22}%
  \hologoEntry{ConTeXt}{}{2011/03/25}%
  \hologoEntry{ConTeXt}{narrow}{2011/03/25}%
  \hologoEntry{ConTeXt}{simple}{2011/03/25}%
  \hologoEntry{emTeX}{}{2010/04/26}%
  \hologoEntry{eTeX}{}{2010/04/08}%
  \hologoEntry{ExTeX}{}{2011/10/01}%
  \hologoEntry{HanTheThanh}{}{2011/11/29}%
  \hologoEntry{iniTeX}{}{2011/10/01}%
  \hologoEntry{KOMAScript}{}{2011/10/01}%
  \hologoEntry{La}{}{2010/05/08}%
  \hologoEntry{LaTeX}{}{2010/04/08}%
  \hologoEntry{LaTeX2e}{}{2010/04/08}%
  \hologoEntry{LaTeX3}{}{2010/04/24}%
  \hologoEntry{LaTeXe}{}{2010/04/08}%
  \hologoEntry{LaTeXML}{}{2011/11/22}%
  \hologoEntry{LaTeXTeX}{}{2011/10/01}%
  \hologoEntry{LuaLaTeX}{}{2010/04/08}%
  \hologoEntry{LuaTeX}{}{2010/04/08}%
  \hologoEntry{LyX}{}{2011/10/01}%
  \hologoEntry{METAFONT}{}{2011/10/01}%
  \hologoEntry{MetaFun}{}{2011/10/01}%
  \hologoEntry{METAPOST}{}{2011/10/01}%
  \hologoEntry{MetaPost}{}{2011/10/01}%
  \hologoEntry{MiKTeX}{}{2011/10/01}%
  \hologoEntry{NTS}{}{2011/10/01}%
  \hologoEntry{OzMF}{}{2011/10/01}%
  \hologoEntry{OzMP}{}{2011/10/01}%
  \hologoEntry{OzTeX}{}{2011/10/01}%
  \hologoEntry{OzTtH}{}{2011/10/01}%
  \hologoEntry{PCTeX}{}{2011/10/01}%
  \hologoEntry{pdfTeX}{}{2011/10/01}%
  \hologoEntry{pdfLaTeX}{}{2011/10/01}%
  \hologoEntry{PiC}{}{2011/10/01}%
  \hologoEntry{PiCTeX}{}{2011/10/01}%
  \hologoEntry{plainTeX}{}{2010/04/08}%
  \hologoEntry{plainTeX}{space}{2010/04/16}%
  \hologoEntry{plainTeX}{hyphen}{2010/04/16}%
  \hologoEntry{plainTeX}{runtogether}{2010/04/16}%
  \hologoEntry{SageTeX}{}{2011/11/22}%
  \hologoEntry{SLiTeX}{}{2011/10/01}%
  \hologoEntry{SLiTeX}{lift}{2011/10/01}%
  \hologoEntry{SLiTeX}{narrow}{2011/10/01}%
  \hologoEntry{SLiTeX}{simple}{2011/10/01}%
  \hologoEntry{SliTeX}{}{2011/10/01}%
  \hologoEntry{SliTeX}{narrow}{2011/10/01}%
  \hologoEntry{SliTeX}{simple}{2011/10/01}%
  \hologoEntry{SliTeX}{lift}{2011/10/01}%
  \hologoEntry{teTeX}{}{2011/10/01}%
  \hologoEntry{TeX}{}{2010/04/08}%
  \hologoEntry{TeX4ht}{}{2011/11/22}%
  \hologoEntry{TTH}{}{2011/11/22}%
  \hologoEntry{virTeX}{}{2011/10/01}%
  \hologoEntry{VTeX}{}{2010/04/24}%
  \hologoEntry{Xe}{}{2010/04/08}%
  \hologoEntry{XeLaTeX}{}{2010/04/08}%
  \hologoEntry{XeTeX}{}{2010/04/08}%
}
%    \end{macrocode}
%    \end{macro}
%
% \subsection{Load resources}
%
%    \begin{macrocode}
\begingroup\expandafter\expandafter\expandafter\endgroup
\expandafter\ifx\csname RequirePackage\endcsname\relax
  \def\TMP@RequirePackage#1[#2]{%
    \begingroup\expandafter\expandafter\expandafter\endgroup
    \expandafter\ifx\csname ver@#1.sty\endcsname\relax
      \input #1.sty\relax
    \fi
  }%
  \TMP@RequirePackage{ltxcmds}[2011/02/04]%
  \TMP@RequirePackage{infwarerr}[2010/04/08]%
  \TMP@RequirePackage{kvsetkeys}[2010/03/01]%
  \TMP@RequirePackage{kvdefinekeys}[2010/03/01]%
  \TMP@RequirePackage{pdftexcmds}[2010/04/01]%
  \TMP@RequirePackage{ifpdf}[2010/01/28]%
  \TMP@RequirePackage{ifluatex}[2010/03/01]%
  \ltx@IfUndefined{newif}{%
    \expandafter\let\csname newif\endcsname\ltx@newif
  }{}%
  \TMP@RequirePackage{ifxetex}[2009/01/23]%
  \TMP@RequirePackage{ifvtex}[2010/03/01]%
\else
  \RequirePackage{ltxcmds}[2011/02/04]%
  \RequirePackage{infwarerr}[2010/04/08]%
  \RequirePackage{kvsetkeys}[2010/03/01]%
  \RequirePackage{kvdefinekeys}[2010/03/01]%
  \RequirePackage{pdftexcmds}[2010/04/01]%
  \RequirePackage{ifpdf}[2010/01/28]%
  \RequirePackage{ifluatex}[2010/03/01]%
  \RequirePackage{ifxetex}[2009/01/23]%
  \RequirePackage{ifvtex}[2010/03/01]%
\fi
%    \end{macrocode}
%
%    \begin{macro}{\HOLOGO@IfDefined}
%    \begin{macrocode}
\def\HOLOGO@IfExists#1{%
  \ifx\@undefined#1%
    \expandafter\ltx@secondoftwo
  \else
    \ifx\relax#1%
      \expandafter\ltx@secondoftwo
    \else
      \expandafter\expandafter\expandafter\ltx@firstoftwo
    \fi
  \fi
}
%    \end{macrocode}
%    \end{macro}
%
% \subsection{Setup macros}
%
%    \begin{macro}{\hologoSetup}
%    \begin{macrocode}
\def\hologoSetup{%
  \let\HOLOGO@name\relax
  \HOLOGO@Setup
}
%    \end{macrocode}
%    \end{macro}
%
%    \begin{macro}{\hologoLogoSetup}
%    \begin{macrocode}
\def\hologoLogoSetup#1{%
  \edef\HOLOGO@name{#1}%
  \ltx@IfUndefined{HoLogo@\HOLOGO@name}{%
    \@PackageError{hologo}{%
      Unknown logo `\HOLOGO@name'%
    }\@ehc
    \ltx@gobble
  }{%
    \HOLOGO@Setup
  }%
}
%    \end{macrocode}
%    \end{macro}
%
%    \begin{macro}{\HOLOGO@Setup}
%    \begin{macrocode}
\def\HOLOGO@Setup{%
  \kvsetkeys{HoLogo}%
}
%    \end{macrocode}
%    \end{macro}
%
% \subsection{Options}
%
%    \begin{macro}{\HOLOGO@DeclareBoolOption}
%    \begin{macrocode}
\def\HOLOGO@DeclareBoolOption#1{%
  \expandafter\chardef\csname HOLOGOOPT@#1\endcsname\ltx@zero
  \kv@define@key{HoLogo}{#1}[true]{%
    \def\HOLOGO@temp{##1}%
    \ifx\HOLOGO@temp\HOLOGO@true
      \ifx\HOLOGO@name\relax
        \expandafter\chardef\csname HOLOGOOPT@#1\endcsname=\ltx@one
      \else
        \expandafter\chardef\csname
        HoLogoOpt@#1@\HOLOGO@name\endcsname\ltx@one
      \fi
      \HOLOGO@SetBreakAll{#1}%
    \else
      \ifx\HOLOGO@temp\HOLOGO@false
        \ifx\HOLOGO@name\relax
          \expandafter\chardef\csname HOLOGOOPT@#1\endcsname=\ltx@zero
        \else
          \expandafter\chardef\csname
          HoLogoOpt@#1@\HOLOGO@name\endcsname=\ltx@zero
        \fi
        \HOLOGO@SetBreakAll{#1}%
      \else
        \@PackageError{hologo}{%
          Unknown value `##1' for boolean option `#1'.\MessageBreak
          Known values are `true' and `false'%
        }\@ehc
      \fi
    \fi
  }%
}
%    \end{macrocode}
%    \end{macro}
%
%    \begin{macro}{\HOLOGO@SetBreakAll}
%    \begin{macrocode}
\def\HOLOGO@SetBreakAll#1{%
  \def\HOLOGO@temp{#1}%
  \ifx\HOLOGO@temp\HOLOGO@break
    \ifx\HOLOGO@name\relax
      \chardef\HOLOGOOPT@hyphenbreak=\HOLOGOOPT@break
      \chardef\HOLOGOOPT@spacebreak=\HOLOGOOPT@break
      \chardef\HOLOGOOPT@discretionarybreak=\HOLOGOOPT@break
    \else
      \expandafter\chardef
         \csname HoLogoOpt@hyphenbreak@\HOLOGO@name\endcsname=%
         \csname HoLogoOpt@break@\HOLOGO@name\endcsname
      \expandafter\chardef
         \csname HoLogoOpt@spacebreak@\HOLOGO@name\endcsname=%
         \csname HoLogoOpt@break@\HOLOGO@name\endcsname
      \expandafter\chardef
         \csname HoLogoOpt@discretionarybreak@\HOLOGO@name
             \endcsname=%
         \csname HoLogoOpt@break@\HOLOGO@name\endcsname
    \fi
  \fi
}
%    \end{macrocode}
%    \end{macro}
%
%    \begin{macro}{\HOLOGO@true}
%    \begin{macrocode}
\def\HOLOGO@true{true}
%    \end{macrocode}
%    \end{macro}
%    \begin{macro}{\HOLOGO@false}
%    \begin{macrocode}
\def\HOLOGO@false{false}
%    \end{macrocode}
%    \end{macro}
%    \begin{macro}{\HOLOGO@break}
%    \begin{macrocode}
\def\HOLOGO@break{break}
%    \end{macrocode}
%    \end{macro}
%
%    \begin{macrocode}
\HOLOGO@DeclareBoolOption{break}
\HOLOGO@DeclareBoolOption{hyphenbreak}
\HOLOGO@DeclareBoolOption{spacebreak}
\HOLOGO@DeclareBoolOption{discretionarybreak}
%    \end{macrocode}
%
%    \begin{macrocode}
\kv@define@key{HoLogo}{variant}{%
  \ifx\HOLOGO@name\relax
    \@PackageError{hologo}{%
      Option `variant' is not available in \string\hologoSetup,%
      \MessageBreak
      Use \string\hologoLogoSetup\space instead%
    }\@ehc
  \else
    \edef\HOLOGO@temp{#1}%
    \ifx\HOLOGO@temp\ltx@empty
      \expandafter
      \let\csname HoLogoOpt@variant@\HOLOGO@name\endcsname\@undefined
    \else
      \ltx@IfUndefined{HoLogo@\HOLOGO@name @\HOLOGO@temp}{%
        \@PackageError{hologo}{%
          Unknown variant `\HOLOGO@temp' of logo `\HOLOGO@name'%
        }\@ehc
      }{%
        \expandafter
        \let\csname HoLogoOpt@variant@\HOLOGO@name\endcsname
            \HOLOGO@temp
      }%
    \fi
  \fi
}
%    \end{macrocode}
%
%    \begin{macro}{\HOLOGO@Variant}
%    \begin{macrocode}
\def\HOLOGO@Variant#1{%
  #1%
  \ltx@ifundefined{HoLogoOpt@variant@#1}{%
  }{%
    @\csname HoLogoOpt@variant@#1\endcsname
  }%
}
%    \end{macrocode}
%    \end{macro}
%
% \subsection{Break/no-break support}
%
%    \begin{macro}{\HOLOGO@space}
%    \begin{macrocode}
\def\HOLOGO@space{%
  \ltx@ifundefined{HoLogoOpt@spacebreak@\HOLOGO@name}{%
    \ltx@ifundefined{HoLogoOpt@break@\HOLOGO@name}{%
      \chardef\HOLOGO@temp=\HOLOGOOPT@spacebreak
    }{%
      \chardef\HOLOGO@temp=%
        \csname HoLogoOpt@break@\HOLOGO@name\endcsname
    }%
  }{%
    \chardef\HOLOGO@temp=%
      \csname HoLogoOpt@spacebreak@\HOLOGO@name\endcsname
  }%
  \ifcase\HOLOGO@temp
    \penalty10000 %
  \fi
  \ltx@space
}
%    \end{macrocode}
%    \end{macro}
%
%    \begin{macro}{\HOLOGO@hyphen}
%    \begin{macrocode}
\def\HOLOGO@hyphen{%
  \ltx@ifundefined{HoLogoOpt@hyphenbreak@\HOLOGO@name}{%
    \ltx@ifundefined{HoLogoOpt@break@\HOLOGO@name}{%
      \chardef\HOLOGO@temp=\HOLOGOOPT@hyphenbreak
    }{%
      \chardef\HOLOGO@temp=%
        \csname HoLogoOpt@break@\HOLOGO@name\endcsname
    }%
  }{%
    \chardef\HOLOGO@temp=%
      \csname HoLogoOpt@hyphenbreak@\HOLOGO@name\endcsname
  }%
  \ifcase\HOLOGO@temp
    \ltx@mbox{-}%
  \else
    -%
  \fi
}
%    \end{macrocode}
%    \end{macro}
%
%    \begin{macro}{\HOLOGO@discretionary}
%    \begin{macrocode}
\def\HOLOGO@discretionary{%
  \ltx@ifundefined{HoLogoOpt@discretionarybreak@\HOLOGO@name}{%
    \ltx@ifundefined{HoLogoOpt@break@\HOLOGO@name}{%
      \chardef\HOLOGO@temp=\HOLOGOOPT@discretionarybreak
    }{%
      \chardef\HOLOGO@temp=%
        \csname HoLogoOpt@break@\HOLOGO@name\endcsname
    }%
  }{%
    \chardef\HOLOGO@temp=%
      \csname HoLogoOpt@discretionarybreak@\HOLOGO@name\endcsname
  }%
  \ifcase\HOLOGO@temp
  \else
    \-%
  \fi
}
%    \end{macrocode}
%    \end{macro}
%
%    \begin{macro}{\HOLOGO@mbox}
%    \begin{macrocode}
\def\HOLOGO@mbox#1{%
  \ltx@ifundefined{HoLogoOpt@break@\HOLOGO@name}{%
    \chardef\HOLOGO@temp=\HOLOGOOPT@hyphenbreak
  }{%
    \chardef\HOLOGO@temp=%
      \csname HoLogoOpt@break@\HOLOGO@name\endcsname
  }%
  \ifcase\HOLOGO@temp
    \ltx@mbox{#1}%
  \else
    #1%
  \fi
}
%    \end{macrocode}
%    \end{macro}
%
% \subsection{Font support}
%
%    \begin{macro}{\HoLogoFont@font}
%    \begin{tabular}{@{}ll@{}}
%    |#1|:& logo name\\
%    |#2|:& font short name\\
%    |#3|:& text
%    \end{tabular}
%    \begin{macrocode}
\def\HoLogoFont@font#1#2#3{%
  \begingroup
    \ltx@IfUndefined{HoLogoFont@logo@#1.#2}{%
      \ltx@IfUndefined{HoLogoFont@font@#2}{%
        \@PackageWarning{hologo}{%
          Missing font `#2' for logo `#1'%
        }%
        #3%
      }{%
        \csname HoLogoFont@font@#2\endcsname{#3}%
      }%
    }{%
      \csname HoLogoFont@logo@#1.#2\endcsname{#3}%
    }%
  \endgroup
}
%    \end{macrocode}
%    \end{macro}
%
%    \begin{macro}{\HoLogoFont@Def}
%    \begin{macrocode}
\def\HoLogoFont@Def#1{%
  \expandafter\def\csname HoLogoFont@font@#1\endcsname
}
%    \end{macrocode}
%    \end{macro}
%    \begin{macro}{\HoLogoFont@LogoDef}
%    \begin{macrocode}
\def\HoLogoFont@LogoDef#1#2{%
  \expandafter\def\csname HoLogoFont@logo@#1.#2\endcsname
}
%    \end{macrocode}
%    \end{macro}
%
% \subsubsection{Font defaults}
%
%    \begin{macro}{\HoLogoFont@font@general}
%    \begin{macrocode}
\HoLogoFont@Def{general}{}%
%    \end{macrocode}
%    \end{macro}
%
%    \begin{macro}{\HoLogoFont@font@rm}
%    \begin{macrocode}
\ltx@IfUndefined{rmfamily}{%
  \ltx@IfUndefined{rm}{%
  }{%
    \HoLogoFont@Def{rm}{\rm}%
  }%
}{%
  \HoLogoFont@Def{rm}{\rmfamily}%
}
%    \end{macrocode}
%    \end{macro}
%
%    \begin{macro}{\HoLogoFont@font@sf}
%    \begin{macrocode}
\ltx@IfUndefined{sffamily}{%
  \ltx@IfUndefined{sf}{%
  }{%
    \HoLogoFont@Def{sf}{\sf}%
  }%
}{%
  \HoLogoFont@Def{sf}{\sffamily}%
}
%    \end{macrocode}
%    \end{macro}
%
%    \begin{macro}{\HoLogoFont@font@bibsf}
%    In case of \hologo{plainTeX} the original small caps
%    variant is used as default. In \hologo{LaTeX}
%    the definition of package \xpackage{dtklogos} \cite{dtklogos}
%    is used.
%\begin{quote}
%\begin{verbatim}
%\DeclareRobustCommand{\BibTeX}{%
%  B%
%  \kern-.05em%
%  \hbox{%
%    $\m@th$% %% force math size calculations
%    \csname S@\f@size\endcsname
%    \fontsize\sf@size\z@
%    \math@fontsfalse
%    \selectfont
%    I%
%    \kern-.025em%
%    B
%  }%
%  \kern-.08em%
%  \-%
%  \TeX
%}
%\end{verbatim}
%\end{quote}
%    \begin{macrocode}
\ltx@IfUndefined{selectfont}{%
  \ltx@IfUndefined{tensc}{%
    \font\tensc=cmcsc10\relax
  }{}%
  \HoLogoFont@Def{bibsf}{\tensc}%
}{%
  \HoLogoFont@Def{bibsf}{%
    $\mathsurround=0pt$%
    \csname S@\f@size\endcsname
    \fontsize\sf@size{0pt}%
    \math@fontsfalse
    \selectfont
  }%
}
%    \end{macrocode}
%    \end{macro}
%
%    \begin{macro}{\HoLogoFont@font@sc}
%    \begin{macrocode}
\ltx@IfUndefined{scshape}{%
  \ltx@IfUndefined{tensc}{%
    \font\tensc=cmcsc10\relax
  }{}%
  \HoLogoFont@Def{sc}{\tensc}%
}{%
  \HoLogoFont@Def{sc}{\scshape}%
}
%    \end{macrocode}
%    \end{macro}
%
%    \begin{macro}{\HoLogoFont@font@sy}
%    \begin{macrocode}
\ltx@IfUndefined{usefont}{%
  \ltx@IfUndefined{tensy}{%
  }{%
    \HoLogoFont@Def{sy}{\tensy}%
  }%
}{%
  \HoLogoFont@Def{sy}{%
    \usefont{OMS}{cmsy}{m}{n}%
  }%
}
%    \end{macrocode}
%    \end{macro}
%
%    \begin{macro}{\HoLogoFont@font@logo}
%    \begin{macrocode}
\begingroup
  \def\x{LaTeX2e}%
\expandafter\endgroup
\ifx\fmtname\x
  \ltx@IfUndefined{logofamily}{%
    \DeclareRobustCommand\logofamily{%
      \not@math@alphabet\logofamily\relax
      \fontencoding{U}%
      \fontfamily{logo}%
      \selectfont
    }%
  }{}%
  \ltx@IfUndefined{logofamily}{%
  }{%
    \HoLogoFont@Def{logo}{\logofamily}%
  }%
\else
  \ltx@IfUndefined{tenlogo}{%
    \font\tenlogo=logo10\relax
  }{}%
  \HoLogoFont@Def{logo}{\tenlogo}%
\fi
%    \end{macrocode}
%    \end{macro}
%
% \subsubsection{Font setup}
%
%    \begin{macro}{\hologoFontSetup}
%    \begin{macrocode}
\def\hologoFontSetup{%
  \let\HOLOGO@name\relax
  \HOLOGO@FontSetup
}
%    \end{macrocode}
%    \end{macro}
%
%    \begin{macro}{\hologoLogoFontSetup}
%    \begin{macrocode}
\def\hologoLogoFontSetup#1{%
  \edef\HOLOGO@name{#1}%
  \ltx@IfUndefined{HoLogo@\HOLOGO@name}{%
    \@PackageError{hologo}{%
      Unknown logo `\HOLOGO@name'%
    }\@ehc
    \ltx@gobble
  }{%
    \HOLOGO@FontSetup
  }%
}
%    \end{macrocode}
%    \end{macro}
%
%    \begin{macro}{\HOLOGO@FontSetup}
%    \begin{macrocode}
\def\HOLOGO@FontSetup{%
  \kvsetkeys{HoLogoFont}%
}
%    \end{macrocode}
%    \end{macro}
%
%    \begin{macrocode}
\def\HOLOGO@temp#1{%
  \kv@define@key{HoLogoFont}{#1}{%
    \ifx\HOLOGO@name\relax
      \HoLogoFont@Def{#1}{##1}%
    \else
      \HoLogoFont@LogoDef\HOLOGO@name{#1}{##1}%
    \fi
  }%
}
\HOLOGO@temp{general}
\HOLOGO@temp{sf}
%    \end{macrocode}
%
% \subsection{Generic logo commands}
%
%    \begin{macrocode}
\HOLOGO@IfExists\hologo{%
  \@PackageError{hologo}{%
    \string\hologo\ltx@space is already defined.\MessageBreak
    Package loading is aborted%
  }\@ehc
  \HOLOGO@AtEnd
}%
\HOLOGO@IfExists\hologoRobust{%
  \@PackageError{hologo}{%
    \string\hologoRobust\ltx@space is already defined.\MessageBreak
    Package loading is aborted%
  }\@ehc
  \HOLOGO@AtEnd
}%
%    \end{macrocode}
%
% \subsubsection{\cs{hologo} and friends}
%
%    \begin{macrocode}
\ifluatex
  \expandafter\ltx@firstofone
\else
  \expandafter\ltx@gobble
\fi
{%
  \ltx@IfUndefined{ifincsname}{%
    \ifnum\luatexversion<36 %
      \expandafter\ltx@gobble
    \else
      \expandafter\ltx@firstofone
    \fi
    {%
      \begingroup
        \ifcase0%
            \directlua{%
              if tex.enableprimitives then %
                tex.enableprimitives('HOLOGO@', {'ifincsname'})%
              else %
                tex.print('1')%
              end%
            }%
            \ifx\HOLOGO@ifincsname\@undefined 1\fi%
            \relax
          \expandafter\ltx@firstofone
        \else
          \endgroup
          \expandafter\ltx@gobble
        \fi
        {%
          \global\let\ifincsname\HOLOGO@ifincsname
        }%
      \HOLOGO@temp
    }%
  }{}%
}
%    \end{macrocode}
%    \begin{macrocode}
\ltx@IfUndefined{ifincsname}{%
  \catcode`$=14 %
}{%
  \catcode`$=9 %
}
%    \end{macrocode}
%
%    \begin{macro}{\hologo}
%    \begin{macrocode}
\def\hologo#1{%
$ \ifincsname
$   \ltx@ifundefined{HoLogoCs@\HOLOGO@Variant{#1}}{%
$     #1%
$   }{%
$     \csname HoLogoCs@\HOLOGO@Variant{#1}\endcsname\ltx@firstoftwo
$   }%
$ \else
    \HOLOGO@IfExists\texorpdfstring\texorpdfstring\ltx@firstoftwo
    {%
      \hologoRobust{#1}%
    }{%
      \ltx@ifundefined{HoLogoBkm@\HOLOGO@Variant{#1}}{%
        \ltx@ifundefined{HoLogo@#1}{?#1?}{#1}%
      }{%
        \csname HoLogoBkm@\HOLOGO@Variant{#1}\endcsname
        \ltx@firstoftwo
      }%
    }%
$ \fi
}
%    \end{macrocode}
%    \end{macro}
%    \begin{macro}{\Hologo}
%    \begin{macrocode}
\def\Hologo#1{%
$ \ifincsname
$   \ltx@ifundefined{HoLogoCs@\HOLOGO@Variant{#1}}{%
$     #1%
$   }{%
$     \csname HoLogoCs@\HOLOGO@Variant{#1}\endcsname\ltx@secondoftwo
$   }%
$ \else
    \HOLOGO@IfExists\texorpdfstring\texorpdfstring\ltx@firstoftwo
    {%
      \HologoRobust{#1}%
    }{%
      \ltx@ifundefined{HoLogoBkm@\HOLOGO@Variant{#1}}{%
        \ltx@ifundefined{HoLogo@#1}{?#1?}{#1}%
      }{%
        \csname HoLogoBkm@\HOLOGO@Variant{#1}\endcsname
        \ltx@secondoftwo
      }%
    }%
$ \fi
}
%    \end{macrocode}
%    \end{macro}
%
%    \begin{macro}{\hologoVariant}
%    \begin{macrocode}
\def\hologoVariant#1#2{%
  \ifx\relax#2\relax
    \hologo{#1}%
  \else
$   \ifincsname
$     \ltx@ifundefined{HoLogoCs@#1@#2}{%
$       #1%
$     }{%
$       \csname HoLogoCs@#1@#2\endcsname\ltx@firstoftwo
$     }%
$   \else
      \HOLOGO@IfExists\texorpdfstring\texorpdfstring\ltx@firstoftwo
      {%
        \hologoVariantRobust{#1}{#2}%
      }{%
        \ltx@ifundefined{HoLogoBkm@#1@#2}{%
          \ltx@ifundefined{HoLogo@#1}{?#1?}{#1}%
        }{%
          \csname HoLogoBkm@#1@#2\endcsname
          \ltx@firstoftwo
        }%
      }%
$   \fi
  \fi
}
%    \end{macrocode}
%    \end{macro}
%    \begin{macro}{\HologoVariant}
%    \begin{macrocode}
\def\HologoVariant#1#2{%
  \ifx\relax#2\relax
    \Hologo{#1}%
  \else
$   \ifincsname
$     \ltx@ifundefined{HoLogoCs@#1@#2}{%
$       #1%
$     }{%
$       \csname HoLogoCs@#1@#2\endcsname\ltx@secondoftwo
$     }%
$   \else
      \HOLOGO@IfExists\texorpdfstring\texorpdfstring\ltx@firstoftwo
      {%
        \HologoVariantRobust{#1}{#2}%
      }{%
        \ltx@ifundefined{HoLogoBkm@#1@#2}{%
          \ltx@ifundefined{HoLogo@#1}{?#1?}{#1}%
        }{%
          \csname HoLogoBkm@#1@#2\endcsname
          \ltx@secondoftwo
        }%
      }%
$   \fi
  \fi
}
%    \end{macrocode}
%    \end{macro}
%
%    \begin{macrocode}
\catcode`\$=3 %
%    \end{macrocode}
%
% \subsubsection{\cs{hologoRobust} and friends}
%
%    \begin{macro}{\hologoRobust}
%    \begin{macrocode}
\ltx@IfUndefined{protected}{%
  \ltx@IfUndefined{DeclareRobustCommand}{%
    \def\hologoRobust#1%
  }{%
    \DeclareRobustCommand*\hologoRobust[1]%
  }%
}{%
  \protected\def\hologoRobust#1%
}%
{%
  \edef\HOLOGO@name{#1}%
  \ltx@IfUndefined{HoLogo@\HOLOGO@Variant\HOLOGO@name}{%
    \@PackageError{hologo}{%
      Unknown logo `\HOLOGO@name'%
    }\@ehc
    ?\HOLOGO@name?%
  }{%
    \ltx@IfUndefined{ver@tex4ht.sty}{%
      \HoLogoFont@font\HOLOGO@name{general}{%
        \csname HoLogo@\HOLOGO@Variant\HOLOGO@name\endcsname
        \ltx@firstoftwo
      }%
    }{%
      \ltx@IfUndefined{HoLogoHtml@\HOLOGO@Variant\HOLOGO@name}{%
        \HOLOGO@name
      }{%
        \csname HoLogoHtml@\HOLOGO@Variant\HOLOGO@name\endcsname
        \ltx@firstoftwo
      }%
    }%
  }%
}
%    \end{macrocode}
%    \end{macro}
%    \begin{macro}{\HologoRobust}
%    \begin{macrocode}
\ltx@IfUndefined{protected}{%
  \ltx@IfUndefined{DeclareRobustCommand}{%
    \def\HologoRobust#1%
  }{%
    \DeclareRobustCommand*\HologoRobust[1]%
  }%
}{%
  \protected\def\HologoRobust#1%
}%
{%
  \edef\HOLOGO@name{#1}%
  \ltx@IfUndefined{HoLogo@\HOLOGO@Variant\HOLOGO@name}{%
    \@PackageError{hologo}{%
      Unknown logo `\HOLOGO@name'%
    }\@ehc
    ?\HOLOGO@name?%
  }{%
    \ltx@IfUndefined{ver@tex4ht.sty}{%
      \HoLogoFont@font\HOLOGO@name{general}{%
        \csname HoLogo@\HOLOGO@Variant\HOLOGO@name\endcsname
        \ltx@secondoftwo
      }%
    }{%
      \ltx@IfUndefined{HoLogoHtml@\HOLOGO@Variant\HOLOGO@name}{%
        \expandafter\HOLOGO@Uppercase\HOLOGO@name
      }{%
        \csname HoLogoHtml@\HOLOGO@Variant\HOLOGO@name\endcsname
        \ltx@secondoftwo
      }%
    }%
  }%
}
%    \end{macrocode}
%    \end{macro}
%    \begin{macro}{\hologoVariantRobust}
%    \begin{macrocode}
\ltx@IfUndefined{protected}{%
  \ltx@IfUndefined{DeclareRobustCommand}{%
    \def\hologoVariantRobust#1#2%
  }{%
    \DeclareRobustCommand*\hologoVariantRobust[2]%
  }%
}{%
  \protected\def\hologoVariantRobust#1#2%
}%
{%
  \begingroup
    \hologoLogoSetup{#1}{variant={#2}}%
    \hologoRobust{#1}%
  \endgroup
}
%    \end{macrocode}
%    \end{macro}
%    \begin{macro}{\HologoVariantRobust}
%    \begin{macrocode}
\ltx@IfUndefined{protected}{%
  \ltx@IfUndefined{DeclareRobustCommand}{%
    \def\HologoVariantRobust#1#2%
  }{%
    \DeclareRobustCommand*\HologoVariantRobust[2]%
  }%
}{%
  \protected\def\HologoVariantRobust#1#2%
}%
{%
  \begingroup
    \hologoLogoSetup{#1}{variant={#2}}%
    \HologoRobust{#1}%
  \endgroup
}
%    \end{macrocode}
%    \end{macro}
%
%    \begin{macro}{\hologorobust}
%    Macro \cs{hologorobust} is only defined for compatibility.
%    Its use is deprecated.
%    \begin{macrocode}
\def\hologorobust{\hologoRobust}
%    \end{macrocode}
%    \end{macro}
%
% \subsection{Helpers}
%
%    \begin{macro}{\HOLOGO@Uppercase}
%    Macro \cs{HOLOGO@Uppercase} is restricted to \cs{uppercase},
%    because \hologo{plainTeX} or \hologo{iniTeX} do not provide
%    \cs{MakeUppercase}.
%    \begin{macrocode}
\def\HOLOGO@Uppercase#1{\uppercase{#1}}
%    \end{macrocode}
%    \end{macro}
%
%    \begin{macro}{\HOLOGO@PdfdocUnicode}
%    \begin{macrocode}
\def\HOLOGO@PdfdocUnicode{%
  \ifx\ifHy@unicode\iftrue
    \expandafter\ltx@secondoftwo
  \else
    \expandafter\ltx@firstoftwo
  \fi
}
%    \end{macrocode}
%    \end{macro}
%
%    \begin{macro}{\HOLOGO@Math}
%    \begin{macrocode}
\def\HOLOGO@MathSetup{%
  \mathsurround0pt\relax
  \HOLOGO@IfExists\f@series{%
    \if b\expandafter\ltx@car\f@series x\@nil
      \csname boldmath\endcsname
   \fi
  }{}%
}
%    \end{macrocode}
%    \end{macro}
%
%    \begin{macro}{\HOLOGO@TempDimen}
%    \begin{macrocode}
\dimendef\HOLOGO@TempDimen=\ltx@zero
%    \end{macrocode}
%    \end{macro}
%    \begin{macro}{\HOLOGO@NegativeKerning}
%    \begin{macrocode}
\def\HOLOGO@NegativeKerning#1{%
  \begingroup
    \HOLOGO@TempDimen=0pt\relax
    \comma@parse@normalized{#1}{%
      \ifdim\HOLOGO@TempDimen=0pt %
        \expandafter\HOLOGO@@NegativeKerning\comma@entry
      \fi
      \ltx@gobble
    }%
    \ifdim\HOLOGO@TempDimen<0pt %
      \kern\HOLOGO@TempDimen
    \fi
  \endgroup
}
%    \end{macrocode}
%    \end{macro}
%    \begin{macro}{\HOLOGO@@NegativeKerning}
%    \begin{macrocode}
\def\HOLOGO@@NegativeKerning#1#2{%
  \setbox\ltx@zero\hbox{#1#2}%
  \HOLOGO@TempDimen=\wd\ltx@zero
  \setbox\ltx@zero\hbox{#1\kern0pt#2}%
  \advance\HOLOGO@TempDimen by -\wd\ltx@zero
}
%    \end{macrocode}
%    \end{macro}
%
%    \begin{macro}{\HOLOGO@SpaceFactor}
%    \begin{macrocode}
\def\HOLOGO@SpaceFactor{%
  \spacefactor1000 %
}
%    \end{macrocode}
%    \end{macro}
%
%    \begin{macro}{\HOLOGO@Span}
%    \begin{macrocode}
\def\HOLOGO@Span#1#2{%
  \HCode{<span class="HoLogo-#1">}%
  #2%
  \HCode{</span>}%
}
%    \end{macrocode}
%    \end{macro}
%
% \subsubsection{Text subscript}
%
%    \begin{macro}{\HOLOGO@SubScript}%
%    \begin{macrocode}
\def\HOLOGO@SubScript#1{%
  \ltx@IfUndefined{textsubscript}{%
    \ltx@IfUndefined{text}{%
      \ltx@mbox{%
        \mathsurround=0pt\relax
        $%
          _{%
            \ltx@IfUndefined{sf@size}{%
              \mathrm{#1}%
            }{%
              \mbox{%
                \fontsize\sf@size{0pt}\selectfont
                #1%
              }%
            }%
          }%
        $%
      }%
    }{%
      \ltx@mbox{%
        \mathsurround=0pt\relax
        $_{\text{#1}}$%
      }%
    }%
  }{%
    \textsubscript{#1}%
  }%
}
%    \end{macrocode}
%    \end{macro}
%
% \subsection{\hologo{TeX} and friends}
%
% \subsubsection{\hologo{TeX}}
%
%    \begin{macro}{\HoLogo@TeX}
%    Source: \hologo{LaTeX} kernel.
%    \begin{macrocode}
\def\HoLogo@TeX#1{%
  T\kern-.1667em\lower.5ex\hbox{E}\kern-.125emX\HOLOGO@SpaceFactor
}
%    \end{macrocode}
%    \end{macro}
%    \begin{macro}{\HoLogoHtml@TeX}
%    \begin{macrocode}
\def\HoLogoHtml@TeX#1{%
  \HoLogoCss@TeX
  \HOLOGO@Span{TeX}{%
    T%
    \HOLOGO@Span{e}{%
      E%
    }%
    X%
  }%
}
%    \end{macrocode}
%    \end{macro}
%    \begin{macro}{\HoLogoCss@TeX}
%    \begin{macrocode}
\def\HoLogoCss@TeX{%
  \Css{%
    span.HoLogo-TeX span.HoLogo-e{%
      position:relative;%
      top:.5ex;%
      margin-left:-.1667em;%
      margin-right:-.125em;%
    }%
  }%
  \Css{%
    a span.HoLogo-TeX span.HoLogo-e{%
      text-decoration:none;%
    }%
  }%
  \global\let\HoLogoCss@TeX\relax
}
%    \end{macrocode}
%    \end{macro}
%
% \subsubsection{\hologo{plainTeX}}
%
%    \begin{macro}{\HoLogo@plainTeX@space}
%    Source: ``The \hologo{TeX}book''
%    \begin{macrocode}
\def\HoLogo@plainTeX@space#1{%
  \HOLOGO@mbox{#1{p}{P}lain}\HOLOGO@space\hologo{TeX}%
}
%    \end{macrocode}
%    \end{macro}
%    \begin{macro}{\HoLogoCs@plainTeX@space}
%    \begin{macrocode}
\def\HoLogoCs@plainTeX@space#1{#1{p}{P}lain TeX}%
%    \end{macrocode}
%    \end{macro}
%    \begin{macro}{\HoLogoBkm@plainTeX@space}
%    \begin{macrocode}
\def\HoLogoBkm@plainTeX@space#1{%
  #1{p}{P}lain \hologo{TeX}%
}
%    \end{macrocode}
%    \end{macro}
%    \begin{macro}{\HoLogoHtml@plainTeX@space}
%    \begin{macrocode}
\def\HoLogoHtml@plainTeX@space#1{%
  #1{p}{P}lain \hologo{TeX}%
}
%    \end{macrocode}
%    \end{macro}
%
%    \begin{macro}{\HoLogo@plainTeX@hyphen}
%    \begin{macrocode}
\def\HoLogo@plainTeX@hyphen#1{%
  \HOLOGO@mbox{#1{p}{P}lain}\HOLOGO@hyphen\hologo{TeX}%
}
%    \end{macrocode}
%    \end{macro}
%    \begin{macro}{\HoLogoCs@plainTeX@hyphen}
%    \begin{macrocode}
\def\HoLogoCs@plainTeX@hyphen#1{#1{p}{P}lain-TeX}
%    \end{macrocode}
%    \end{macro}
%    \begin{macro}{\HoLogoBkm@plainTeX@hyphen}
%    \begin{macrocode}
\def\HoLogoBkm@plainTeX@hyphen#1{%
  #1{p}{P}lain-\hologo{TeX}%
}
%    \end{macrocode}
%    \end{macro}
%    \begin{macro}{\HoLogoHtml@plainTeX@hyphen}
%    \begin{macrocode}
\def\HoLogoHtml@plainTeX@hyphen#1{%
  #1{p}{P}lain-\hologo{TeX}%
}
%    \end{macrocode}
%    \end{macro}
%
%    \begin{macro}{\HoLogo@plainTeX@runtogether}
%    \begin{macrocode}
\def\HoLogo@plainTeX@runtogether#1{%
  \HOLOGO@mbox{#1{p}{P}lain\hologo{TeX}}%
}
%    \end{macrocode}
%    \end{macro}
%    \begin{macro}{\HoLogoCs@plainTeX@runtogether}
%    \begin{macrocode}
\def\HoLogoCs@plainTeX@runtogether#1{#1{p}{P}lainTeX}
%    \end{macrocode}
%    \end{macro}
%    \begin{macro}{\HoLogoBkm@plainTeX@runtogether}
%    \begin{macrocode}
\def\HoLogoBkm@plainTeX@runtogether#1{%
  #1{p}{P}lain\hologo{TeX}%
}
%    \end{macrocode}
%    \end{macro}
%    \begin{macro}{\HoLogoHtml@plainTeX@runtogether}
%    \begin{macrocode}
\def\HoLogoHtml@plainTeX@runtogether#1{%
  #1{p}{P}lain\hologo{TeX}%
}
%    \end{macrocode}
%    \end{macro}
%
%    \begin{macro}{\HoLogo@plainTeX}
%    \begin{macrocode}
\def\HoLogo@plainTeX{\HoLogo@plainTeX@space}
%    \end{macrocode}
%    \end{macro}
%    \begin{macro}{\HoLogoCs@plainTeX}
%    \begin{macrocode}
\def\HoLogoCs@plainTeX{\HoLogoCs@plainTeX@space}
%    \end{macrocode}
%    \end{macro}
%    \begin{macro}{\HoLogoBkm@plainTeX}
%    \begin{macrocode}
\def\HoLogoBkm@plainTeX{\HoLogoBkm@plainTeX@space}
%    \end{macrocode}
%    \end{macro}
%    \begin{macro}{\HoLogoHtml@plainTeX}
%    \begin{macrocode}
\def\HoLogoHtml@plainTeX{\HoLogoHtml@plainTeX@space}
%    \end{macrocode}
%    \end{macro}
%
% \subsubsection{\hologo{LaTeX}}
%
%    Source: \hologo{LaTeX} kernel.
%\begin{quote}
%\begin{verbatim}
%\DeclareRobustCommand{\LaTeX}{%
%  L%
%  \kern-.36em%
%  {%
%    \sbox\z@ T%
%    \vbox to\ht\z@{%
%      \hbox{%
%        \check@mathfonts
%        \fontsize\sf@size\z@
%        \math@fontsfalse
%        \selectfont
%        A%
%      }%
%      \vss
%    }%
%  }%
%  \kern-.15em%
%  \TeX
%}
%\end{verbatim}
%\end{quote}
%
%    \begin{macro}{\HoLogo@La}
%    \begin{macrocode}
\def\HoLogo@La#1{%
  L%
  \kern-.36em%
  \begingroup
    \setbox\ltx@zero\hbox{T}%
    \vbox to\ht\ltx@zero{%
      \hbox{%
        \ltx@ifundefined{check@mathfonts}{%
          \csname sevenrm\endcsname
        }{%
          \check@mathfonts
          \fontsize\sf@size{0pt}%
          \math@fontsfalse\selectfont
        }%
        A%
      }%
      \vss
    }%
  \endgroup
}
%    \end{macrocode}
%    \end{macro}
%
%    \begin{macro}{\HoLogo@LaTeX}
%    Source: \hologo{LaTeX} kernel.
%    \begin{macrocode}
\def\HoLogo@LaTeX#1{%
  \hologo{La}%
  \kern-.15em%
  \hologo{TeX}%
}
%    \end{macrocode}
%    \end{macro}
%    \begin{macro}{\HoLogoHtml@LaTeX}
%    \begin{macrocode}
\def\HoLogoHtml@LaTeX#1{%
  \HoLogoCss@LaTeX
  \HOLOGO@Span{LaTeX}{%
    L%
    \HOLOGO@Span{a}{%
      A%
    }%
    \hologo{TeX}%
  }%
}
%    \end{macrocode}
%    \end{macro}
%    \begin{macro}{\HoLogoCss@LaTeX}
%    \begin{macrocode}
\def\HoLogoCss@LaTeX{%
  \Css{%
    span.HoLogo-LaTeX span.HoLogo-a{%
      position:relative;%
      top:-.5ex;%
      margin-left:-.36em;%
      margin-right:-.15em;%
      font-size:85\%;%
    }%
  }%
  \global\let\HoLogoCss@LaTeX\relax
}
%    \end{macrocode}
%    \end{macro}
%
% \subsubsection{\hologo{(La)TeX}}
%
%    \begin{macro}{\HoLogo@LaTeXTeX}
%    The kerning around the parentheses is taken
%    from package \xpackage{dtklogos} \cite{dtklogos}.
%\begin{quote}
%\begin{verbatim}
%\DeclareRobustCommand{\LaTeXTeX}{%
%  (%
%  \kern-.15em%
%  L%
%  \kern-.36em%
%  {%
%    \sbox\z@ T%
%    \vbox to\ht0{%
%      \hbox{%
%        $\m@th$%
%        \csname S@\f@size\endcsname
%        \fontsize\sf@size\z@
%        \math@fontsfalse
%        \selectfont
%        A%
%      }%
%      \vss
%    }%
%  }%
%  \kern-.2em%
%  )%
%  \kern-.15em%
%  \TeX
%}
%\end{verbatim}
%\end{quote}
%    \begin{macrocode}
\def\HoLogo@LaTeXTeX#1{%
  (%
  \kern-.15em%
  \hologo{La}%
  \kern-.2em%
  )%
  \kern-.15em%
  \hologo{TeX}%
}
%    \end{macrocode}
%    \end{macro}
%    \begin{macro}{\HoLogoBkm@LaTeXTeX}
%    \begin{macrocode}
\def\HoLogoBkm@LaTeXTeX#1{(La)TeX}
%    \end{macrocode}
%    \end{macro}
%
%    \begin{macro}{\HoLogo@(La)TeX}
%    \begin{macrocode}
\expandafter
\let\csname HoLogo@(La)TeX\endcsname\HoLogo@LaTeXTeX
%    \end{macrocode}
%    \end{macro}
%    \begin{macro}{\HoLogoBkm@(La)TeX}
%    \begin{macrocode}
\expandafter
\let\csname HoLogoBkm@(La)TeX\endcsname\HoLogoBkm@LaTeXTeX
%    \end{macrocode}
%    \end{macro}
%    \begin{macro}{\HoLogoHtml@LaTeXTeX}
%    \begin{macrocode}
\def\HoLogoHtml@LaTeXTeX#1{%
  \HoLogoCss@LaTeXTeX
  \HOLOGO@Span{LaTeXTeX}{%
    (%
    \HOLOGO@Span{L}{L}%
    \HOLOGO@Span{a}{A}%
    \HOLOGO@Span{ParenRight}{)}%
    \hologo{TeX}%
  }%
}
%    \end{macrocode}
%    \end{macro}
%    \begin{macro}{\HoLogoHtml@(La)TeX}
%    Kerning after opening parentheses and before closing parentheses
%    is $-0.1$\,em. The original values $-0.15$\,em
%    looked too ugly for a serif font.
%    \begin{macrocode}
\expandafter
\let\csname HoLogoHtml@(La)TeX\endcsname\HoLogoHtml@LaTeXTeX
%    \end{macrocode}
%    \end{macro}
%    \begin{macro}{\HoLogoCss@LaTeXTeX}
%    \begin{macrocode}
\def\HoLogoCss@LaTeXTeX{%
  \Css{%
    span.HoLogo-LaTeXTeX span.HoLogo-L{%
      margin-left:-.1em;%
    }%
  }%
  \Css{%
    span.HoLogo-LaTeXTeX span.HoLogo-a{%
      position:relative;%
      top:-.5ex;%
      margin-left:-.36em;%
      margin-right:-.1em;%
      font-size:85\%;%
    }%
  }%
  \Css{%
    span.HoLogo-LaTeXTeX span.HoLogo-ParenRight{%
      margin-right:-.15em;%
    }%
  }%
  \global\let\HoLogoCss@LaTeXTeX\relax
}
%    \end{macrocode}
%    \end{macro}
%
% \subsubsection{\hologo{LaTeXe}}
%
%    \begin{macro}{\HoLogo@LaTeXe}
%    Source: \hologo{LaTeX} kernel
%    \begin{macrocode}
\def\HoLogo@LaTeXe#1{%
  \hologo{LaTeX}%
  \kern.15em%
  \hbox{%
    \HOLOGO@MathSetup
    2%
    $_{\textstyle\varepsilon}$%
  }%
}
%    \end{macrocode}
%    \end{macro}
%
%    \begin{macro}{\HoLogoCs@LaTeXe}
%    \begin{macrocode}
\ifnum64=`\^^^^0040\relax % test for big chars of LuaTeX/XeTeX
  \catcode`\$=9 %
  \catcode`\&=14 %
\else
  \catcode`\$=14 %
  \catcode`\&=9 %
\fi
\def\HoLogoCs@LaTeXe#1{%
  LaTeX2%
$ \string ^^^^0395%
& e%
}%
\catcode`\$=3 %
\catcode`\&=4 %
%    \end{macrocode}
%    \end{macro}
%
%    \begin{macro}{\HoLogoBkm@LaTeXe}
%    \begin{macrocode}
\def\HoLogoBkm@LaTeXe#1{%
  \hologo{LaTeX}%
  2%
  \HOLOGO@PdfdocUnicode{e}{\textepsilon}%
}
%    \end{macrocode}
%    \end{macro}
%
%    \begin{macro}{\HoLogoHtml@LaTeXe}
%    \begin{macrocode}
\def\HoLogoHtml@LaTeXe#1{%
  \HoLogoCss@LaTeXe
  \HOLOGO@Span{LaTeX2e}{%
    \hologo{LaTeX}%
    \HOLOGO@Span{2}{2}%
    \HOLOGO@Span{e}{%
      \HOLOGO@MathSetup
      \ensuremath{\textstyle\varepsilon}%
    }%
  }%
}
%    \end{macrocode}
%    \end{macro}
%    \begin{macro}{\HoLogoCss@LaTeXe}
%    \begin{macrocode}
\def\HoLogoCss@LaTeXe{%
  \Css{%
    span.HoLogo-LaTeX2e span.HoLogo-2{%
      padding-left:.15em;%
    }%
  }%
  \Css{%
    span.HoLogo-LaTeX2e span.HoLogo-e{%
      position:relative;%
      top:.35ex;%
      text-decoration:none;%
    }%
  }%
  \global\let\HoLogoCss@LaTeXe\relax
}
%    \end{macrocode}
%    \end{macro}
%
%    \begin{macro}{\HoLogo@LaTeX2e}
%    \begin{macrocode}
\expandafter
\let\csname HoLogo@LaTeX2e\endcsname\HoLogo@LaTeXe
%    \end{macrocode}
%    \end{macro}
%    \begin{macro}{\HoLogoCs@LaTeX2e}
%    \begin{macrocode}
\expandafter
\let\csname HoLogoCs@LaTeX2e\endcsname\HoLogoCs@LaTeXe
%    \end{macrocode}
%    \end{macro}
%    \begin{macro}{\HoLogoBkm@LaTeX2e}
%    \begin{macrocode}
\expandafter
\let\csname HoLogoBkm@LaTeX2e\endcsname\HoLogoBkm@LaTeXe
%    \end{macrocode}
%    \end{macro}
%    \begin{macro}{\HoLogoHtml@LaTeX2e}
%    \begin{macrocode}
\expandafter
\let\csname HoLogoHtml@LaTeX2e\endcsname\HoLogoHtml@LaTeXe
%    \end{macrocode}
%    \end{macro}
%
% \subsubsection{\hologo{LaTeX3}}
%
%    \begin{macro}{\HoLogo@LaTeX3}
%    Source: \hologo{LaTeX} kernel
%    \begin{macrocode}
\expandafter\def\csname HoLogo@LaTeX3\endcsname#1{%
  \hologo{LaTeX}%
  3%
}
%    \end{macrocode}
%    \end{macro}
%
%    \begin{macro}{\HoLogoBkm@LaTeX3}
%    \begin{macrocode}
\expandafter\def\csname HoLogoBkm@LaTeX3\endcsname#1{%
  \hologo{LaTeX}%
  3%
}
%    \end{macrocode}
%    \end{macro}
%    \begin{macro}{\HoLogoHtml@LaTeX3}
%    \begin{macrocode}
\expandafter
\let\csname HoLogoHtml@LaTeX3\expandafter\endcsname
\csname HoLogo@LaTeX3\endcsname
%    \end{macrocode}
%    \end{macro}
%
% \subsubsection{\hologo{LaTeXML}}
%
%    \begin{macro}{\HoLogo@LaTeXML}
%    \begin{macrocode}
\def\HoLogo@LaTeXML#1{%
  \HOLOGO@mbox{%
    \hologo{La}%
    \kern-.15em%
    T%
    \kern-.1667em%
    \lower.5ex\hbox{E}%
    \kern-.125em%
    \HoLogoFont@font{LaTeXML}{sc}{xml}%
  }%
}
%    \end{macrocode}
%    \end{macro}
%    \begin{macro}{\HoLogoHtml@pdfLaTeX}
%    \begin{macrocode}
\def\HoLogoHtml@LaTeXML#1{%
  \HOLOGO@Span{LaTeXML}{%
    \HoLogoCss@LaTeX
    \HoLogoCss@TeX
    \HOLOGO@Span{LaTeX}{%
      L%
      \HOLOGO@Span{a}{%
        A%
      }%
    }%
    \HOLOGO@Span{TeX}{%
      T%
      \HOLOGO@Span{e}{%
        E%
      }%
    }%
    \HCode{<span style="font-variant: small-caps;">}%
    xml%
    \HCode{</span>}%
  }%
}
%    \end{macrocode}
%    \end{macro}
%
% \subsubsection{\hologo{eTeX}}
%
%    \begin{macro}{\HoLogo@eTeX}
%    Source: package \xpackage{etex}
%    \begin{macrocode}
\def\HoLogo@eTeX#1{%
  \ltx@mbox{%
    \HOLOGO@MathSetup
    $\varepsilon$%
    -%
    \HOLOGO@NegativeKerning{-T,T-,To}%
    \hologo{TeX}%
  }%
}
%    \end{macrocode}
%    \end{macro}
%    \begin{macro}{\HoLogoCs@eTeX}
%    \begin{macrocode}
\ifnum64=`\^^^^0040\relax % test for big chars of LuaTeX/XeTeX
  \catcode`\$=9 %
  \catcode`\&=14 %
\else
  \catcode`\$=14 %
  \catcode`\&=9 %
\fi
\def\HoLogoCs@eTeX#1{%
$ #1{\string ^^^^0395}{\string ^^^^03b5}%
& #1{e}{E}%
  TeX%
}%
\catcode`\$=3 %
\catcode`\&=4 %
%    \end{macrocode}
%    \end{macro}
%    \begin{macro}{\HoLogoBkm@eTeX}
%    \begin{macrocode}
\def\HoLogoBkm@eTeX#1{%
  \HOLOGO@PdfdocUnicode{#1{e}{E}}{\textepsilon}%
  -%
  \hologo{TeX}%
}
%    \end{macrocode}
%    \end{macro}
%    \begin{macro}{\HoLogoHtml@eTeX}
%    \begin{macrocode}
\def\HoLogoHtml@eTeX#1{%
  \ltx@mbox{%
    \HOLOGO@MathSetup
    $\varepsilon$%
    -%
    \hologo{TeX}%
  }%
}
%    \end{macrocode}
%    \end{macro}
%
% \subsubsection{\hologo{iniTeX}}
%
%    \begin{macro}{\HoLogo@iniTeX}
%    \begin{macrocode}
\def\HoLogo@iniTeX#1{%
  \HOLOGO@mbox{%
    #1{i}{I}ni\hologo{TeX}%
  }%
}
%    \end{macrocode}
%    \end{macro}
%    \begin{macro}{\HoLogoCs@iniTeX}
%    \begin{macrocode}
\def\HoLogoCs@iniTeX#1{#1{i}{I}niTeX}
%    \end{macrocode}
%    \end{macro}
%    \begin{macro}{\HoLogoBkm@iniTeX}
%    \begin{macrocode}
\def\HoLogoBkm@iniTeX#1{%
  #1{i}{I}ni\hologo{TeX}%
}
%    \end{macrocode}
%    \end{macro}
%    \begin{macro}{\HoLogoHtml@iniTeX}
%    \begin{macrocode}
\let\HoLogoHtml@iniTeX\HoLogo@iniTeX
%    \end{macrocode}
%    \end{macro}
%
% \subsubsection{\hologo{virTeX}}
%
%    \begin{macro}{\HoLogo@virTeX}
%    \begin{macrocode}
\def\HoLogo@virTeX#1{%
  \HOLOGO@mbox{%
    #1{v}{V}ir\hologo{TeX}%
  }%
}
%    \end{macrocode}
%    \end{macro}
%    \begin{macro}{\HoLogoCs@virTeX}
%    \begin{macrocode}
\def\HoLogoCs@virTeX#1{#1{v}{V}irTeX}
%    \end{macrocode}
%    \end{macro}
%    \begin{macro}{\HoLogoBkm@virTeX}
%    \begin{macrocode}
\def\HoLogoBkm@virTeX#1{%
  #1{v}{V}ir\hologo{TeX}%
}
%    \end{macrocode}
%    \end{macro}
%    \begin{macro}{\HoLogoHtml@virTeX}
%    \begin{macrocode}
\let\HoLogoHtml@virTeX\HoLogo@virTeX
%    \end{macrocode}
%    \end{macro}
%
% \subsubsection{\hologo{SliTeX}}
%
% \paragraph{Definitions of the three variants.}
%
%    \begin{macro}{\HoLogo@SLiTeX@lift}
%    \begin{macrocode}
\def\HoLogo@SLiTeX@lift#1{%
  \HoLogoFont@font{SliTeX}{rm}{%
    S%
    \kern-.06em%
    L%
    \kern-.18em%
    \raise.32ex\hbox{\HoLogoFont@font{SliTeX}{sc}{i}}%
    \HOLOGO@discretionary
    \kern-.06em%
    \hologo{TeX}%
  }%
}
%    \end{macrocode}
%    \end{macro}
%    \begin{macro}{\HoLogoBkm@SLiTeX@lift}
%    \begin{macrocode}
\def\HoLogoBkm@SLiTeX@lift#1{SLiTeX}
%    \end{macrocode}
%    \end{macro}
%    \begin{macro}{\HoLogoHtml@SLiTeX@lift}
%    \begin{macrocode}
\def\HoLogoHtml@SLiTeX@lift#1{%
  \HoLogoCss@SLiTeX@lift
  \HOLOGO@Span{SLiTeX-lift}{%
    \HoLogoFont@font{SliTeX}{rm}{%
      S%
      \HOLOGO@Span{L}{L}%
      \HOLOGO@Span{i}{i}%
      \hologo{TeX}%
    }%
  }%
}
%    \end{macrocode}
%    \end{macro}
%    \begin{macro}{\HoLogoCss@SLiTeX@lift}
%    \begin{macrocode}
\def\HoLogoCss@SLiTeX@lift{%
  \Css{%
    span.HoLogo-SLiTeX-lift span.HoLogo-L{%
      margin-left:-.06em;%
      margin-right:-.18em;%
    }%
  }%
  \Css{%
    span.HoLogo-SLiTeX-lift span.HoLogo-i{%
      position:relative;%
      top:-.32ex;%
      margin-right:-.06em;%
      font-variant:small-caps;%
    }%
  }%
  \global\let\HoLogoCss@SLiTeX@lift\relax
}
%    \end{macrocode}
%    \end{macro}
%
%    \begin{macro}{\HoLogo@SliTeX@simple}
%    \begin{macrocode}
\def\HoLogo@SliTeX@simple#1{%
  \HoLogoFont@font{SliTeX}{rm}{%
    \ltx@mbox{%
      \HoLogoFont@font{SliTeX}{sc}{Sli}%
    }%
    \HOLOGO@discretionary
    \hologo{TeX}%
  }%
}
%    \end{macrocode}
%    \end{macro}
%    \begin{macro}{\HoLogoBkm@SliTeX@simple}
%    \begin{macrocode}
\def\HoLogoBkm@SliTeX@simple#1{SliTeX}
%    \end{macrocode}
%    \end{macro}
%    \begin{macro}{\HoLogoHtml@SliTeX@simple}
%    \begin{macrocode}
\let\HoLogoHtml@SliTeX@simple\HoLogo@SliTeX@simple
%    \end{macrocode}
%    \end{macro}
%
%    \begin{macro}{\HoLogo@SliTeX@narrow}
%    \begin{macrocode}
\def\HoLogo@SliTeX@narrow#1{%
  \HoLogoFont@font{SliTeX}{rm}{%
    \ltx@mbox{%
      S%
      \kern-.06em%
      \HoLogoFont@font{SliTeX}{sc}{%
        l%
        \kern-.035em%
        i%
      }%
    }%
    \HOLOGO@discretionary
    \kern-.06em%
    \hologo{TeX}%
  }%
}
%    \end{macrocode}
%    \end{macro}
%    \begin{macro}{\HoLogoBkm@SliTeX@narrow}
%    \begin{macrocode}
\def\HoLogoBkm@SliTeX@narrow#1{SliTeX}
%    \end{macrocode}
%    \end{macro}
%    \begin{macro}{\HoLogoHtml@SliTeX@narrow}
%    \begin{macrocode}
\def\HoLogoHtml@SliTeX@narrow#1{%
  \HoLogoCss@SliTeX@narrow
  \HOLOGO@Span{SliTeX-narrow}{%
    \HoLogoFont@font{SliTeX}{rm}{%
      S%
        \HOLOGO@Span{l}{l}%
        \HOLOGO@Span{i}{i}%
      \hologo{TeX}%
    }%
  }%
}
%    \end{macrocode}
%    \end{macro}
%    \begin{macro}{\HoLogoCss@SliTeX@narrow}
%    \begin{macrocode}
\def\HoLogoCss@SliTeX@narrow{%
  \Css{%
    span.HoLogo-SliTeX-narrow span.HoLogo-l{%
      margin-left:-.06em;%
      margin-right:-.035em;%
      font-variant:small-caps;%
    }%
  }%
  \Css{%
    span.HoLogo-SliTeX-narrow span.HoLogo-i{%
      margin-right:-.06em;%
      font-variant:small-caps;%
    }%
  }%
  \global\let\HoLogoCss@SliTeX@narrow\relax
}
%    \end{macrocode}
%    \end{macro}
%
% \paragraph{Macro set completion.}
%
%    \begin{macro}{\HoLogo@SLiTeX@simple}
%    \begin{macrocode}
\def\HoLogo@SLiTeX@simple{\HoLogo@SliTeX@simple}
%    \end{macrocode}
%    \end{macro}
%    \begin{macro}{\HoLogoBkm@SLiTeX@simple}
%    \begin{macrocode}
\def\HoLogoBkm@SLiTeX@simple{\HoLogoBkm@SliTeX@simple}
%    \end{macrocode}
%    \end{macro}
%    \begin{macro}{\HoLogoHtml@SLiTeX@simple}
%    \begin{macrocode}
\def\HoLogoHtml@SLiTeX@simple{\HoLogoHtml@SliTeX@simple}
%    \end{macrocode}
%    \end{macro}
%
%    \begin{macro}{\HoLogo@SLiTeX@narrow}
%    \begin{macrocode}
\def\HoLogo@SLiTeX@narrow{\HoLogo@SliTeX@narrow}
%    \end{macrocode}
%    \end{macro}
%    \begin{macro}{\HoLogoBkm@SLiTeX@narrow}
%    \begin{macrocode}
\def\HoLogoBkm@SLiTeX@narrow{\HoLogoBkm@SliTeX@narrow}
%    \end{macrocode}
%    \end{macro}
%    \begin{macro}{\HoLogoHtml@SLiTeX@narrow}
%    \begin{macrocode}
\def\HoLogoHtml@SLiTeX@narrow{\HoLogoHtml@SliTeX@narrow}
%    \end{macrocode}
%    \end{macro}
%
%    \begin{macro}{\HoLogo@SliTeX@lift}
%    \begin{macrocode}
\def\HoLogo@SliTeX@lift{\HoLogo@SLiTeX@lift}
%    \end{macrocode}
%    \end{macro}
%    \begin{macro}{\HoLogoBkm@SliTeX@lift}
%    \begin{macrocode}
\def\HoLogoBkm@SliTeX@lift{\HoLogoBkm@SLiTeX@lift}
%    \end{macrocode}
%    \end{macro}
%    \begin{macro}{\HoLogoHtml@SliTeX@lift}
%    \begin{macrocode}
\def\HoLogoHtml@SliTeX@lift{\HoLogoHtml@SLiTeX@lift}
%    \end{macrocode}
%    \end{macro}
%
% \paragraph{Defaults.}
%
%    \begin{macro}{\HoLogo@SLiTeX}
%    \begin{macrocode}
\def\HoLogo@SLiTeX{\HoLogo@SLiTeX@lift}
%    \end{macrocode}
%    \end{macro}
%    \begin{macro}{\HoLogoBkm@SLiTeX}
%    \begin{macrocode}
\def\HoLogoBkm@SLiTeX{\HoLogoBkm@SLiTeX@lift}
%    \end{macrocode}
%    \end{macro}
%    \begin{macro}{\HoLogoHtml@SLiTeX}
%    \begin{macrocode}
\def\HoLogoHtml@SLiTeX{\HoLogoHtml@SLiTeX@lift}
%    \end{macrocode}
%    \end{macro}
%
%    \begin{macro}{\HoLogo@SliTeX}
%    \begin{macrocode}
\def\HoLogo@SliTeX{\HoLogo@SliTeX@narrow}
%    \end{macrocode}
%    \end{macro}
%    \begin{macro}{\HoLogoBkm@SliTeX}
%    \begin{macrocode}
\def\HoLogoBkm@SliTeX{\HoLogoBkm@SliTeX@narrow}
%    \end{macrocode}
%    \end{macro}
%    \begin{macro}{\HoLogoHtml@SliTeX}
%    \begin{macrocode}
\def\HoLogoHtml@SliTeX{\HoLogoHtml@SliTeX@narrow}
%    \end{macrocode}
%    \end{macro}
%
% \subsubsection{\hologo{LuaTeX}}
%
%    \begin{macro}{\HoLogo@LuaTeX}
%    The kerning is an idea of Hans Hagen, see mailing list
%    `luatex at tug dot org' in March 2010.
%    \begin{macrocode}
\def\HoLogo@LuaTeX#1{%
  \HOLOGO@mbox{%
    Lua%
    \HOLOGO@NegativeKerning{aT,oT,To}%
    \hologo{TeX}%
  }%
}
%    \end{macrocode}
%    \end{macro}
%    \begin{macro}{\HoLogoHtml@LuaTeX}
%    \begin{macrocode}
\let\HoLogoHtml@LuaTeX\HoLogo@LuaTeX
%    \end{macrocode}
%    \end{macro}
%
% \subsubsection{\hologo{LuaLaTeX}}
%
%    \begin{macro}{\HoLogo@LuaLaTeX}
%    \begin{macrocode}
\def\HoLogo@LuaLaTeX#1{%
  \HOLOGO@mbox{%
    Lua%
    \hologo{LaTeX}%
  }%
}
%    \end{macrocode}
%    \end{macro}
%    \begin{macro}{\HoLogoHtml@LuaLaTeX}
%    \begin{macrocode}
\let\HoLogoHtml@LuaLaTeX\HoLogo@LuaLaTeX
%    \end{macrocode}
%    \end{macro}
%
% \subsubsection{\hologo{XeTeX}, \hologo{XeLaTeX}}
%
%    \begin{macro}{\HOLOGO@IfCharExists}
%    \begin{macrocode}
\ifluatex
  \ifnum\luatexversion<36 %
  \else
    \def\HOLOGO@IfCharExists#1{%
      \ifnum
        \directlua{%
           if luaotfload and luaotfload.aux then
             if luaotfload.aux.font_has_glyph(%
                    font.current(), \number#1) then % 	 
	       tex.print("1") % 	 
	     end % 	 
	   elseif font and font.fonts and font.current then %
            local f = font.fonts[font.current()]%
            if f.characters and f.characters[\number#1] then %
              tex.print("1")%
            end %
          end%
        }0=\ltx@zero
        \expandafter\ltx@secondoftwo
      \else
        \expandafter\ltx@firstoftwo
      \fi
    }%
  \fi
\fi
\ltx@IfUndefined{HOLOGO@IfCharExists}{%
  \def\HOLOGO@@IfCharExists#1{%
    \begingroup
      \tracinglostchars=\ltx@zero
      \setbox\ltx@zero=\hbox{%
        \kern7sp\char#1\relax
        \ifnum\lastkern>\ltx@zero
          \expandafter\aftergroup\csname iffalse\endcsname
        \else
          \expandafter\aftergroup\csname iftrue\endcsname
        \fi
      }%
      % \if{true|false} from \aftergroup
      \endgroup
      \expandafter\ltx@firstoftwo
    \else
      \endgroup
      \expandafter\ltx@secondoftwo
    \fi
  }%
  \ifxetex
    \ltx@IfUndefined{XeTeXfonttype}{}{%
      \ltx@IfUndefined{XeTeXcharglyph}{}{%
        \def\HOLOGO@IfCharExists#1{%
          \ifnum\XeTeXfonttype\font>\ltx@zero
            \expandafter\ltx@firstofthree
          \else
            \expandafter\ltx@gobble
          \fi
          {%
            \ifnum\XeTeXcharglyph#1>\ltx@zero
              \expandafter\ltx@firstoftwo
            \else
              \expandafter\ltx@secondoftwo
            \fi
          }%
          \HOLOGO@@IfCharExists{#1}%
        }%
      }%
    }%
  \fi
}{}
\ltx@ifundefined{HOLOGO@IfCharExists}{%
  \ifnum64=`\^^^^0040\relax % test for big chars of LuaTeX/XeTeX
    \let\HOLOGO@IfCharExists\HOLOGO@@IfCharExists
  \else
    \def\HOLOGO@IfCharExists#1{%
      \ifnum#1>255 %
        \expandafter\ltx@fourthoffour
      \fi
      \HOLOGO@@IfCharExists{#1}%
    }%
  \fi
}{}
%    \end{macrocode}
%    \end{macro}
%
%    \begin{macro}{\HoLogo@Xe}
%    Source: package \xpackage{dtklogos}
%    \begin{macrocode}
\def\HoLogo@Xe#1{%
  X%
  \kern-.1em\relax
  \HOLOGO@IfCharExists{"018E}{%
    \lower.5ex\hbox{\char"018E}%
  }{%
    \chardef\HOLOGO@choice=\ltx@zero
    \ifdim\fontdimen\ltx@one\font>0pt %
      \ltx@IfUndefined{rotatebox}{%
        \ltx@IfUndefined{pgftext}{%
          \ltx@IfUndefined{psscalebox}{%
            \ltx@IfUndefined{HOLOGO@ScaleBox@\hologoDriver}{%
            }{%
              \chardef\HOLOGO@choice=4 %
            }%
          }{%
            \chardef\HOLOGO@choice=3 %
          }%
        }{%
          \chardef\HOLOGO@choice=2 %
        }%
      }{%
        \chardef\HOLOGO@choice=1 %
      }%
      \ifcase\HOLOGO@choice
        \HOLOGO@WarningUnsupportedDriver{Xe}%
        e%
      \or % 1: \rotatebox
        \begingroup
          \setbox\ltx@zero\hbox{\rotatebox{180}{E}}%
          \ltx@LocDimenA=\dp\ltx@zero
          \advance\ltx@LocDimenA by -.5ex\relax
          \raise\ltx@LocDimenA\box\ltx@zero
        \endgroup
      \or % 2: \pgftext
        \lower.5ex\hbox{%
          \pgfpicture
            \pgftext[rotate=180]{E}%
          \endpgfpicture
        }%
      \or % 3: \psscalebox
        \begingroup
          \setbox\ltx@zero\hbox{\psscalebox{-1 -1}{E}}%
          \ltx@LocDimenA=\dp\ltx@zero
          \advance\ltx@LocDimenA by -.5ex\relax
          \raise\ltx@LocDimenA\box\ltx@zero
        \endgroup
      \or % 4: \HOLOGO@PointReflectBox
        \lower.5ex\hbox{\HOLOGO@PointReflectBox{E}}%
      \else
        \@PackageError{hologo}{Internal error (choice/it}\@ehc
      \fi
    \else
      \ltx@IfUndefined{reflectbox}{%
        \ltx@IfUndefined{pgftext}{%
          \ltx@IfUndefined{psscalebox}{%
            \ltx@IfUndefined{HOLOGO@ScaleBox@\hologoDriver}{%
            }{%
              \chardef\HOLOGO@choice=4 %
            }%
          }{%
            \chardef\HOLOGO@choice=3 %
          }%
        }{%
          \chardef\HOLOGO@choice=2 %
        }%
      }{%
        \chardef\HOLOGO@choice=1 %
      }%
      \ifcase\HOLOGO@choice
        \HOLOGO@WarningUnsupportedDriver{Xe}%
        e%
      \or % 1: reflectbox
        \lower.5ex\hbox{%
          \reflectbox{E}%
        }%
      \or % 2: \pgftext
        \lower.5ex\hbox{%
          \pgfpicture
            \pgftransformxscale{-1}%
            \pgftext{E}%
          \endpgfpicture
        }%
      \or % 3: \psscalebox
        \lower.5ex\hbox{%
          \psscalebox{-1 1}{E}%
        }%
      \or % 4: \HOLOGO@Reflectbox
        \lower.5ex\hbox{%
          \HOLOGO@ReflectBox{E}%
        }%
      \else
        \@PackageError{hologo}{Internal error (choice/up)}\@ehc
      \fi
    \fi
  }%
}
%    \end{macrocode}
%    \end{macro}
%    \begin{macro}{\HoLogoHtml@Xe}
%    \begin{macrocode}
\def\HoLogoHtml@Xe#1{%
  \HoLogoCss@Xe
  \HOLOGO@Span{Xe}{%
    X%
    \HOLOGO@Span{e}{%
      \HCode{&\ltx@hashchar x018e;}%
    }%
  }%
}
%    \end{macrocode}
%    \end{macro}
%    \begin{macro}{\HoLogoCss@Xe}
%    \begin{macrocode}
\def\HoLogoCss@Xe{%
  \Css{%
    span.HoLogo-Xe span.HoLogo-e{%
      position:relative;%
      top:.5ex;%
      left-margin:-.1em;%
    }%
  }%
  \global\let\HoLogoCss@Xe\relax
}
%    \end{macrocode}
%    \end{macro}
%
%    \begin{macro}{\HoLogo@XeTeX}
%    \begin{macrocode}
\def\HoLogo@XeTeX#1{%
  \hologo{Xe}%
  \kern-.15em\relax
  \hologo{TeX}%
}
%    \end{macrocode}
%    \end{macro}
%
%    \begin{macro}{\HoLogoHtml@XeTeX}
%    \begin{macrocode}
\def\HoLogoHtml@XeTeX#1{%
  \HoLogoCss@XeTeX
  \HOLOGO@Span{XeTeX}{%
    \hologo{Xe}%
    \hologo{TeX}%
  }%
}
%    \end{macrocode}
%    \end{macro}
%    \begin{macro}{\HoLogoCss@XeTeX}
%    \begin{macrocode}
\def\HoLogoCss@XeTeX{%
  \Css{%
    span.HoLogo-XeTeX span.HoLogo-TeX{%
      margin-left:-.15em;%
    }%
  }%
  \global\let\HoLogoCss@XeTeX\relax
}
%    \end{macrocode}
%    \end{macro}
%
%    \begin{macro}{\HoLogo@XeLaTeX}
%    \begin{macrocode}
\def\HoLogo@XeLaTeX#1{%
  \hologo{Xe}%
  \kern-.13em%
  \hologo{LaTeX}%
}
%    \end{macrocode}
%    \end{macro}
%    \begin{macro}{\HoLogoHtml@XeLaTeX}
%    \begin{macrocode}
\def\HoLogoHtml@XeLaTeX#1{%
  \HoLogoCss@XeLaTeX
  \HOLOGO@Span{XeLaTeX}{%
    \hologo{Xe}%
    \hologo{LaTeX}%
  }%
}
%    \end{macrocode}
%    \end{macro}
%    \begin{macro}{\HoLogoCss@XeLaTeX}
%    \begin{macrocode}
\def\HoLogoCss@XeLaTeX{%
  \Css{%
    span.HoLogo-XeLaTeX span.HoLogo-Xe{%
      margin-right:-.13em;%
    }%
  }%
  \global\let\HoLogoCss@XeLaTeX\relax
}
%    \end{macrocode}
%    \end{macro}
%
% \subsubsection{\hologo{pdfTeX}, \hologo{pdfLaTeX}}
%
%    \begin{macro}{\HoLogo@pdfTeX}
%    \begin{macrocode}
\def\HoLogo@pdfTeX#1{%
  \HOLOGO@mbox{%
    #1{p}{P}df\hologo{TeX}%
  }%
}
%    \end{macrocode}
%    \end{macro}
%    \begin{macro}{\HoLogoCs@pdfTeX}
%    \begin{macrocode}
\def\HoLogoCs@pdfTeX#1{#1{p}{P}dfTeX}
%    \end{macrocode}
%    \end{macro}
%    \begin{macro}{\HoLogoBkm@pdfTeX}
%    \begin{macrocode}
\def\HoLogoBkm@pdfTeX#1{%
  #1{p}{P}df\hologo{TeX}%
}
%    \end{macrocode}
%    \end{macro}
%    \begin{macro}{\HoLogoHtml@pdfTeX}
%    \begin{macrocode}
\let\HoLogoHtml@pdfTeX\HoLogo@pdfTeX
%    \end{macrocode}
%    \end{macro}
%
%    \begin{macro}{\HoLogo@pdfLaTeX}
%    \begin{macrocode}
\def\HoLogo@pdfLaTeX#1{%
  \HOLOGO@mbox{%
    #1{p}{P}df\hologo{LaTeX}%
  }%
}
%    \end{macrocode}
%    \end{macro}
%    \begin{macro}{\HoLogoCs@pdfLaTeX}
%    \begin{macrocode}
\def\HoLogoCs@pdfLaTeX#1{#1{p}{P}dfLaTeX}
%    \end{macrocode}
%    \end{macro}
%    \begin{macro}{\HoLogoBkm@pdfLaTeX}
%    \begin{macrocode}
\def\HoLogoBkm@pdfLaTeX#1{%
  #1{p}{P}df\hologo{LaTeX}%
}
%    \end{macrocode}
%    \end{macro}
%    \begin{macro}{\HoLogoHtml@pdfLaTeX}
%    \begin{macrocode}
\let\HoLogoHtml@pdfLaTeX\HoLogo@pdfLaTeX
%    \end{macrocode}
%    \end{macro}
%
% \subsubsection{\hologo{VTeX}}
%
%    \begin{macro}{\HoLogo@VTeX}
%    \begin{macrocode}
\def\HoLogo@VTeX#1{%
  \HOLOGO@mbox{%
    V\hologo{TeX}%
  }%
}
%    \end{macrocode}
%    \end{macro}
%    \begin{macro}{\HoLogoHtml@VTeX}
%    \begin{macrocode}
\let\HoLogoHtml@VTeX\HoLogo@VTeX
%    \end{macrocode}
%    \end{macro}
%
% \subsubsection{\hologo{AmS}, \dots}
%
%    Source: class \xclass{amsdtx}
%
%    \begin{macro}{\HoLogo@AmS}
%    \begin{macrocode}
\def\HoLogo@AmS#1{%
  \HoLogoFont@font{AmS}{sy}{%
    A%
    \kern-.1667em%
    \lower.5ex\hbox{M}%
    \kern-.125em%
    S%
  }%
}
%    \end{macrocode}
%    \end{macro}
%    \begin{macro}{\HoLogoBkm@AmS}
%    \begin{macrocode}
\def\HoLogoBkm@AmS#1{AmS}
%    \end{macrocode}
%    \end{macro}
%    \begin{macro}{\HoLogoHtml@AmS}
%    \begin{macrocode}
\def\HoLogoHtml@AmS#1{%
  \HoLogoCss@AmS
%  \HoLogoFont@font{AmS}{sy}{%
    \HOLOGO@Span{AmS}{%
      A%
      \HOLOGO@Span{M}{M}%
      S%
    }%
%   }%
}
%    \end{macrocode}
%    \end{macro}
%    \begin{macro}{\HoLogoCss@AmS}
%    \begin{macrocode}
\def\HoLogoCss@AmS{%
  \Css{%
    span.HoLogo-AmS span.HoLogo-M{%
      position:relative;%
      top:.5ex;%
      margin-left:-.1667em;%
      margin-right:-.125em;%
      text-decoration:none;%
    }%
  }%
  \global\let\HoLogoCss@AmS\relax
}
%    \end{macrocode}
%    \end{macro}
%
%    \begin{macro}{\HoLogo@AmSTeX}
%    \begin{macrocode}
\def\HoLogo@AmSTeX#1{%
  \hologo{AmS}%
  \HOLOGO@hyphen
  \hologo{TeX}%
}
%    \end{macrocode}
%    \end{macro}
%    \begin{macro}{\HoLogoBkm@AmSTeX}
%    \begin{macrocode}
\def\HoLogoBkm@AmSTeX#1{AmS-TeX}%
%    \end{macrocode}
%    \end{macro}
%    \begin{macro}{\HoLogoHtml@AmSTeX}
%    \begin{macrocode}
\let\HoLogoHtml@AmSTeX\HoLogo@AmSTeX
%    \end{macrocode}
%    \end{macro}
%
%    \begin{macro}{\HoLogo@AmSLaTeX}
%    \begin{macrocode}
\def\HoLogo@AmSLaTeX#1{%
  \hologo{AmS}%
  \HOLOGO@hyphen
  \hologo{LaTeX}%
}
%    \end{macrocode}
%    \end{macro}
%    \begin{macro}{\HoLogoBkm@AmSLaTeX}
%    \begin{macrocode}
\def\HoLogoBkm@AmSLaTeX#1{AmS-LaTeX}%
%    \end{macrocode}
%    \end{macro}
%    \begin{macro}{\HoLogoHtml@AmSLaTeX}
%    \begin{macrocode}
\let\HoLogoHtml@AmSLaTeX\HoLogo@AmSLaTeX
%    \end{macrocode}
%    \end{macro}
%
% \subsubsection{\hologo{BibTeX}}
%
%    \begin{macro}{\HoLogo@BibTeX@sc}
%    A definition of \hologo{BibTeX} is provided in
%    the documentation source for the manual of \hologo{BibTeX}
%    \cite{btxdoc}.
%\begin{quote}
%\begin{verbatim}
%\def\BibTeX{%
%  {%
%    \rm
%    B%
%    \kern-.05em%
%    {%
%      \sc
%      i%
%      \kern-.025em %
%      b%
%    }%
%    \kern-.08em
%    T%
%    \kern-.1667em%
%    \lower.7ex\hbox{E}%
%    \kern-.125em%
%    X%
%  }%
%}
%\end{verbatim}
%\end{quote}
%    \begin{macrocode}
\def\HoLogo@BibTeX@sc#1{%
  B%
  \kern-.05em%
  \HoLogoFont@font{BibTeX}{sc}{%
    i%
    \kern-.025em%
    b%
  }%
  \HOLOGO@discretionary
  \kern-.08em%
  \hologo{TeX}%
}
%    \end{macrocode}
%    \end{macro}
%    \begin{macro}{\HoLogoHtml@BibTeX@sc}
%    \begin{macrocode}
\def\HoLogoHtml@BibTeX@sc#1{%
  \HoLogoCss@BibTeX@sc
  \HOLOGO@Span{BibTeX-sc}{%
    B%
    \HOLOGO@Span{i}{i}%
    \HOLOGO@Span{b}{b}%
    \hologo{TeX}%
  }%
}
%    \end{macrocode}
%    \end{macro}
%    \begin{macro}{\HoLogoCss@BibTeX@sc}
%    \begin{macrocode}
\def\HoLogoCss@BibTeX@sc{%
  \Css{%
    span.HoLogo-BibTeX-sc span.HoLogo-i{%
      margin-left:-.05em;%
      margin-right:-.025em;%
      font-variant:small-caps;%
    }%
  }%
  \Css{%
    span.HoLogo-BibTeX-sc span.HoLogo-b{%
      margin-right:-.08em;%
      font-variant:small-caps;%
    }%
  }%
  \global\let\HoLogoCss@BibTeX@sc\relax
}
%    \end{macrocode}
%    \end{macro}
%
%    \begin{macro}{\HoLogo@BibTeX@sf}
%    Variant \xoption{sf} avoids trouble with unavailable
%    small caps fonts (e.g., bold versions of Computer Modern or
%    Latin Modern). The definition is taken from
%    package \xpackage{dtklogos} \cite{dtklogos}.
%\begin{quote}
%\begin{verbatim}
%\DeclareRobustCommand{\BibTeX}{%
%  B%
%  \kern-.05em%
%  \hbox{%
%    $\m@th$% %% force math size calculations
%    \csname S@\f@size\endcsname
%    \fontsize\sf@size\z@
%    \math@fontsfalse
%    \selectfont
%    I%
%    \kern-.025em%
%    B
%  }%
%  \kern-.08em%
%  \-%
%  \TeX
%}
%\end{verbatim}
%\end{quote}
%    \begin{macrocode}
\def\HoLogo@BibTeX@sf#1{%
  B%
  \kern-.05em%
  \HoLogoFont@font{BibTeX}{bibsf}{%
    I%
    \kern-.025em%
    B%
  }%
  \HOLOGO@discretionary
  \kern-.08em%
  \hologo{TeX}%
}
%    \end{macrocode}
%    \end{macro}
%    \begin{macro}{\HoLogoHtml@BibTeX@sf}
%    \begin{macrocode}
\def\HoLogoHtml@BibTeX@sf#1{%
  \HoLogoCss@BibTeX@sf
  \HOLOGO@Span{BibTeX-sf}{%
    B%
    \HoLogoFont@font{BibTeX}{bibsf}{%
      \HOLOGO@Span{i}{I}%
      B%
    }%
    \hologo{TeX}%
  }%
}
%    \end{macrocode}
%    \end{macro}
%    \begin{macro}{\HoLogoCss@BibTeX@sf}
%    \begin{macrocode}
\def\HoLogoCss@BibTeX@sf{%
  \Css{%
    span.HoLogo-BibTeX-sf span.HoLogo-i{%
      margin-left:-.05em;%
      margin-right:-.025em;%
    }%
  }%
  \Css{%
    span.HoLogo-BibTeX-sf span.HoLogo-TeX{%
      margin-left:-.08em;%
    }%
  }%
  \global\let\HoLogoCss@BibTeX@sf\relax
}
%    \end{macrocode}
%    \end{macro}
%
%    \begin{macro}{\HoLogo@BibTeX}
%    \begin{macrocode}
\def\HoLogo@BibTeX{\HoLogo@BibTeX@sf}
%    \end{macrocode}
%    \end{macro}
%    \begin{macro}{\HoLogoHtml@BibTeX}
%    \begin{macrocode}
\def\HoLogoHtml@BibTeX{\HoLogoHtml@BibTeX@sf}
%    \end{macrocode}
%    \end{macro}
%
% \subsubsection{\hologo{BibTeX8}}
%
%    \begin{macro}{\HoLogo@BibTeX8}
%    \begin{macrocode}
\expandafter\def\csname HoLogo@BibTeX8\endcsname#1{%
  \hologo{BibTeX}%
  8%
}
%    \end{macrocode}
%    \end{macro}
%
%    \begin{macro}{\HoLogoBkm@BibTeX8}
%    \begin{macrocode}
\expandafter\def\csname HoLogoBkm@BibTeX8\endcsname#1{%
  \hologo{BibTeX}%
  8%
}
%    \end{macrocode}
%    \end{macro}
%    \begin{macro}{\HoLogoHtml@BibTeX8}
%    \begin{macrocode}
\expandafter
\let\csname HoLogoHtml@BibTeX8\expandafter\endcsname
\csname HoLogo@BibTeX8\endcsname
%    \end{macrocode}
%    \end{macro}
%
% \subsubsection{\hologo{ConTeXt}}
%
%    \begin{macro}{\HoLogo@ConTeXt@simple}
%    \begin{macrocode}
\def\HoLogo@ConTeXt@simple#1{%
  \HOLOGO@mbox{Con}%
  \HOLOGO@discretionary
  \HOLOGO@mbox{\hologo{TeX}t}%
}
%    \end{macrocode}
%    \end{macro}
%    \begin{macro}{\HoLogoHtml@ConTeXt@simple}
%    \begin{macrocode}
\let\HoLogoHtml@ConTeXt@simple\HoLogo@ConTeXt@simple
%    \end{macrocode}
%    \end{macro}
%
%    \begin{macro}{\HoLogo@ConTeXt@narrow}
%    This definition of logo \hologo{ConTeXt} with variant \xoption{narrow}
%    comes from TUGboat's class \xclass{ltugboat} (version 2010/11/15 v2.8).
%    \begin{macrocode}
\def\HoLogo@ConTeXt@narrow#1{%
  \HOLOGO@mbox{C\kern-.0333emon}%
  \HOLOGO@discretionary
  \kern-.0667em%
  \HOLOGO@mbox{\hologo{TeX}\kern-.0333emt}%
}
%    \end{macrocode}
%    \end{macro}
%    \begin{macro}{\HoLogoHtml@ConTeXt@narrow}
%    \begin{macrocode}
\def\HoLogoHtml@ConTeXt@narrow#1{%
  \HoLogoCss@ConTeXt@narrow
  \HOLOGO@Span{ConTeXt-narrow}{%
    \HOLOGO@Span{C}{C}%
    on%
    \hologo{TeX}%
    t%
  }%
}
%    \end{macrocode}
%    \end{macro}
%    \begin{macro}{\HoLogoCss@ConTeXt@narrow}
%    \begin{macrocode}
\def\HoLogoCss@ConTeXt@narrow{%
  \Css{%
    span.HoLogo-ConTeXt-narrow span.HoLogo-C{%
      margin-left:-.0333em;%
    }%
  }%
  \Css{%
    span.HoLogo-ConTeXt-narrow span.HoLogo-TeX{%
      margin-left:-.0667em;%
      margin-right:-.0333em;%
    }%
  }%
  \global\let\HoLogoCss@ConTeXt@narrow\relax
}
%    \end{macrocode}
%    \end{macro}
%
%    \begin{macro}{\HoLogo@ConTeXt}
%    \begin{macrocode}
\def\HoLogo@ConTeXt{\HoLogo@ConTeXt@narrow}
%    \end{macrocode}
%    \end{macro}
%    \begin{macro}{\HoLogoHtml@ConTeXt}
%    \begin{macrocode}
\def\HoLogoHtml@ConTeXt{\HoLogoHtml@ConTeXt@narrow}
%    \end{macrocode}
%    \end{macro}
%
% \subsubsection{\hologo{emTeX}}
%
%    \begin{macro}{\HoLogo@emTeX}
%    \begin{macrocode}
\def\HoLogo@emTeX#1{%
  \HOLOGO@mbox{#1{e}{E}m}%
  \HOLOGO@discretionary
  \hologo{TeX}%
}
%    \end{macrocode}
%    \end{macro}
%    \begin{macro}{\HoLogoCs@emTeX}
%    \begin{macrocode}
\def\HoLogoCs@emTeX#1{#1{e}{E}mTeX}%
%    \end{macrocode}
%    \end{macro}
%    \begin{macro}{\HoLogoBkm@emTeX}
%    \begin{macrocode}
\def\HoLogoBkm@emTeX#1{%
  #1{e}{E}m\hologo{TeX}%
}
%    \end{macrocode}
%    \end{macro}
%    \begin{macro}{\HoLogoHtml@emTeX}
%    \begin{macrocode}
\let\HoLogoHtml@emTeX\HoLogo@emTeX
%    \end{macrocode}
%    \end{macro}
%
% \subsubsection{\hologo{ExTeX}}
%
%    \begin{macro}{\HoLogo@ExTeX}
%    The definition is taken from the FAQ of the
%    project \hologo{ExTeX}
%    \cite{ExTeX-FAQ}.
%\begin{quote}
%\begin{verbatim}
%\def\ExTeX{%
%  \textrm{% Logo always with serifs
%    \ensuremath{%
%      \textstyle
%      \varepsilon_{%
%        \kern-0.15em%
%        \mathcal{X}%
%      }%
%    }%
%    \kern-.15em%
%    \TeX
%  }%
%}
%\end{verbatim}
%\end{quote}
%    \begin{macrocode}
\def\HoLogo@ExTeX#1{%
  \HoLogoFont@font{ExTeX}{rm}{%
    \ltx@mbox{%
      \HOLOGO@MathSetup
      $%
        \textstyle
        \varepsilon_{%
          \kern-0.15em%
          \HoLogoFont@font{ExTeX}{sy}{X}%
        }%
      $%
    }%
    \HOLOGO@discretionary
    \kern-.15em%
    \hologo{TeX}%
  }%
}
%    \end{macrocode}
%    \end{macro}
%    \begin{macro}{\HoLogoHtml@ExTeX}
%    \begin{macrocode}
\def\HoLogoHtml@ExTeX#1{%
  \HoLogoCss@ExTeX
  \HoLogoFont@font{ExTeX}{rm}{%
    \HOLOGO@Span{ExTeX}{%
      \ltx@mbox{%
        \HOLOGO@MathSetup
        $\textstyle\varepsilon$%
        \HOLOGO@Span{X}{$\textstyle\chi$}%
        \hologo{TeX}%
      }%
    }%
  }%
}
%    \end{macrocode}
%    \end{macro}
%    \begin{macro}{\HoLogoBkm@ExTeX}
%    \begin{macrocode}
\def\HoLogoBkm@ExTeX#1{%
  \HOLOGO@PdfdocUnicode{#1{e}{E}x}{\textepsilon\textchi}%
  \hologo{TeX}%
}
%    \end{macrocode}
%    \end{macro}
%    \begin{macro}{\HoLogoCss@ExTeX}
%    \begin{macrocode}
\def\HoLogoCss@ExTeX{%
  \Css{%
    span.HoLogo-ExTeX{%
      font-family:serif;%
    }%
  }%
  \Css{%
    span.HoLogo-ExTeX span.HoLogo-TeX{%
      margin-left:-.15em;%
    }%
  }%
  \global\let\HoLogoCss@ExTeX\relax
}
%    \end{macrocode}
%    \end{macro}
%
% \subsubsection{\hologo{MiKTeX}}
%
%    \begin{macro}{\HoLogo@MiKTeX}
%    \begin{macrocode}
\def\HoLogo@MiKTeX#1{%
  \HOLOGO@mbox{MiK}%
  \HOLOGO@discretionary
  \hologo{TeX}%
}
%    \end{macrocode}
%    \end{macro}
%    \begin{macro}{\HoLogoHtml@MiKTeX}
%    \begin{macrocode}
\let\HoLogoHtml@MiKTeX\HoLogo@MiKTeX
%    \end{macrocode}
%    \end{macro}
%
% \subsubsection{\hologo{OzTeX} and friends}
%
%    Source: \hologo{OzTeX} FAQ \cite{OzTeX}:
%    \begin{quote}
%      |\def\OzTeX{O\kern-.03em z\kern-.15em\TeX}|\\
%      (There is no kerning in OzMF, OzMP and OzTtH.)
%    \end{quote}
%
%    \begin{macro}{\HoLogo@OzTeX}
%    \begin{macrocode}
\def\HoLogo@OzTeX#1{%
  O%
  \kern-.03em %
  z%
  \kern-.15em %
  \hologo{TeX}%
}
%    \end{macrocode}
%    \end{macro}
%    \begin{macro}{\HoLogoHtml@OzTeX}
%    \begin{macrocode}
\def\HoLogoHtml@OzTeX#1{%
  \HoLogoCss@OzTeX
  \HOLOGO@Span{OzTeX}{%
    O%
    \HOLOGO@Span{z}{z}%
    \hologo{TeX}%
  }%
}
%    \end{macrocode}
%    \end{macro}
%    \begin{macro}{\HoLogoCss@OzTeX}
%    \begin{macrocode}
\def\HoLogoCss@OzTeX{%
  \Css{%
    span.HoLogo-OzTeX span.HoLogo-z{%
      margin-left:-.03em;%
      margin-right:-.15em;%
    }%
  }%
  \global\let\HoLogoCss@OzTeX\relax
}
%    \end{macrocode}
%    \end{macro}
%
%    \begin{macro}{\HoLogo@OzMF}
%    \begin{macrocode}
\def\HoLogo@OzMF#1{%
  \HOLOGO@mbox{OzMF}%
}
%    \end{macrocode}
%    \end{macro}
%    \begin{macro}{\HoLogo@OzMP}
%    \begin{macrocode}
\def\HoLogo@OzMP#1{%
  \HOLOGO@mbox{OzMP}%
}
%    \end{macrocode}
%    \end{macro}
%    \begin{macro}{\HoLogo@OzTtH}
%    \begin{macrocode}
\def\HoLogo@OzTtH#1{%
  \HOLOGO@mbox{OzTtH}%
}
%    \end{macrocode}
%    \end{macro}
%
% \subsubsection{\hologo{PCTeX}}
%
%    \begin{macro}{\HoLogo@PCTeX}
%    \begin{macrocode}
\def\HoLogo@PCTeX#1{%
  \HOLOGO@mbox{PC}%
  \hologo{TeX}%
}
%    \end{macrocode}
%    \end{macro}
%    \begin{macro}{\HoLogoHtml@PCTeX}
%    \begin{macrocode}
\let\HoLogoHtml@PCTeX\HoLogo@PCTeX
%    \end{macrocode}
%    \end{macro}
%
% \subsubsection{\hologo{PiCTeX}}
%
%    The original definitions from \xfile{pictex.tex} \cite{PiCTeX}:
%\begin{quote}
%\begin{verbatim}
%\def\PiC{%
%  P%
%  \kern-.12em%
%  \lower.5ex\hbox{I}%
%  \kern-.075em%
%  C%
%}
%\def\PiCTeX{%
%  \PiC
%  \kern-.11em%
%  \TeX
%}
%\end{verbatim}
%\end{quote}
%
%    \begin{macro}{\HoLogo@PiC}
%    \begin{macrocode}
\def\HoLogo@PiC#1{%
  P%
  \kern-.12em%
  \lower.5ex\hbox{I}%
  \kern-.075em%
  C%
  \HOLOGO@SpaceFactor
}
%    \end{macrocode}
%    \end{macro}
%    \begin{macro}{\HoLogoHtml@PiC}
%    \begin{macrocode}
\def\HoLogoHtml@PiC#1{%
  \HoLogoCss@PiC
  \HOLOGO@Span{PiC}{%
    P%
    \HOLOGO@Span{i}{I}%
    C%
  }%
}
%    \end{macrocode}
%    \end{macro}
%    \begin{macro}{\HoLogoCss@PiC}
%    \begin{macrocode}
\def\HoLogoCss@PiC{%
  \Css{%
    span.HoLogo-PiC span.HoLogo-i{%
      position:relative;%
      top:.5ex;%
      margin-left:-.12em;%
      margin-right:-.075em;%
      text-decoration:none;%
    }%
  }%
  \global\let\HoLogoCss@PiC\relax
}
%    \end{macrocode}
%    \end{macro}
%
%    \begin{macro}{\HoLogo@PiCTeX}
%    \begin{macrocode}
\def\HoLogo@PiCTeX#1{%
  \hologo{PiC}%
  \HOLOGO@discretionary
  \kern-.11em%
  \hologo{TeX}%
}
%    \end{macrocode}
%    \end{macro}
%    \begin{macro}{\HoLogoHtml@PiCTeX}
%    \begin{macrocode}
\def\HoLogoHtml@PiCTeX#1{%
  \HoLogoCss@PiCTeX
  \HOLOGO@Span{PiCTeX}{%
    \hologo{PiC}%
    \hologo{TeX}%
  }%
}
%    \end{macrocode}
%    \end{macro}
%    \begin{macro}{\HoLogoCss@PiCTeX}
%    \begin{macrocode}
\def\HoLogoCss@PiCTeX{%
  \Css{%
    span.HoLogo-PiCTeX span.HoLogo-PiC{%
      margin-right:-.11em;%
    }%
  }%
  \global\let\HoLogoCss@PiCTeX\relax
}
%    \end{macrocode}
%    \end{macro}
%
% \subsubsection{\hologo{teTeX}}
%
%    \begin{macro}{\HoLogo@teTeX}
%    \begin{macrocode}
\def\HoLogo@teTeX#1{%
  \HOLOGO@mbox{#1{t}{T}e}%
  \HOLOGO@discretionary
  \hologo{TeX}%
}
%    \end{macrocode}
%    \end{macro}
%    \begin{macro}{\HoLogoCs@teTeX}
%    \begin{macrocode}
\def\HoLogoCs@teTeX#1{#1{t}{T}dfTeX}
%    \end{macrocode}
%    \end{macro}
%    \begin{macro}{\HoLogoBkm@teTeX}
%    \begin{macrocode}
\def\HoLogoBkm@teTeX#1{%
  #1{t}{T}e\hologo{TeX}%
}
%    \end{macrocode}
%    \end{macro}
%    \begin{macro}{\HoLogoHtml@teTeX}
%    \begin{macrocode}
\let\HoLogoHtml@teTeX\HoLogo@teTeX
%    \end{macrocode}
%    \end{macro}
%
% \subsubsection{\hologo{TeX4ht}}
%
%    \begin{macro}{\HoLogo@TeX4ht}
%    \begin{macrocode}
\expandafter\def\csname HoLogo@TeX4ht\endcsname#1{%
  \HOLOGO@mbox{\hologo{TeX}4ht}%
}
%    \end{macrocode}
%    \end{macro}
%    \begin{macro}{\HoLogoHtml@TeX4ht}
%    \begin{macrocode}
\expandafter
\let\csname HoLogoHtml@TeX4ht\expandafter\endcsname
\csname HoLogo@TeX4ht\endcsname
%    \end{macrocode}
%    \end{macro}
%
%
% \subsubsection{\hologo{SageTeX}}
%
%    \begin{macro}{\HoLogo@SageTeX}
%    \begin{macrocode}
\def\HoLogo@SageTeX#1{%
  \HOLOGO@mbox{Sage}%
  \HOLOGO@discretionary
  \HOLOGO@NegativeKerning{eT,oT,To}%
  \hologo{TeX}%
}
%    \end{macrocode}
%    \end{macro}
%    \begin{macro}{\HoLogoHtml@SageTeX}
%    \begin{macrocode}
\let\HoLogoHtml@SageTeX\HoLogo@SageTeX
%    \end{macrocode}
%    \end{macro}
%
% \subsection{\hologo{METAFONT} and friends}
%
%    \begin{macro}{\HoLogo@METAFONT}
%    \begin{macrocode}
\def\HoLogo@METAFONT#1{%
  \HoLogoFont@font{METAFONT}{logo}{%
    \HOLOGO@mbox{META}%
    \HOLOGO@discretionary
    \HOLOGO@mbox{FONT}%
  }%
}
%    \end{macrocode}
%    \end{macro}
%
%    \begin{macro}{\HoLogo@METAPOST}
%    \begin{macrocode}
\def\HoLogo@METAPOST#1{%
  \HoLogoFont@font{METAPOST}{logo}{%
    \HOLOGO@mbox{META}%
    \HOLOGO@discretionary
    \HOLOGO@mbox{POST}%
  }%
}
%    \end{macrocode}
%    \end{macro}
%
%    \begin{macro}{\HoLogo@MetaFun}
%    \begin{macrocode}
\def\HoLogo@MetaFun#1{%
  \HOLOGO@mbox{Meta}%
  \HOLOGO@discretionary
  \HOLOGO@mbox{Fun}%
}
%    \end{macrocode}
%    \end{macro}
%
%    \begin{macro}{\HoLogo@MetaPost}
%    \begin{macrocode}
\def\HoLogo@MetaPost#1{%
  \HOLOGO@mbox{Meta}%
  \HOLOGO@discretionary
  \HOLOGO@mbox{Post}%
}
%    \end{macrocode}
%    \end{macro}
%
% \subsection{Others}
%
% \subsubsection{\hologo{biber}}
%
%    \begin{macro}{\HoLogo@biber}
%    \begin{macrocode}
\def\HoLogo@biber#1{%
  \HOLOGO@mbox{#1{b}{B}i}%
  \HOLOGO@discretionary
  \HOLOGO@mbox{ber}%
}
%    \end{macrocode}
%    \end{macro}
%    \begin{macro}{\HoLogoCs@biber}
%    \begin{macrocode}
\def\HoLogoCs@biber#1{#1{b}{B}iber}
%    \end{macrocode}
%    \end{macro}
%    \begin{macro}{\HoLogoBkm@biber}
%    \begin{macrocode}
\def\HoLogoBkm@biber#1{%
  #1{b}{B}iber%
}
%    \end{macrocode}
%    \end{macro}
%    \begin{macro}{\HoLogoHtml@biber}
%    \begin{macrocode}
\let\HoLogoHtml@biber\HoLogo@biber
%    \end{macrocode}
%    \end{macro}
%
% \subsubsection{\hologo{KOMAScript}}
%
%    \begin{macro}{\HoLogo@KOMAScript}
%    The definition for \hologo{KOMAScript} is taken
%    from \hologo{KOMAScript} (\xfile{scrlogo.dtx}, reformatted) \cite{scrlogo}:
%\begin{quote}
%\begin{verbatim}
%\@ifundefined{KOMAScript}{%
%  \DeclareRobustCommand{\KOMAScript}{%
%    \textsf{%
%      K\kern.05em O\kern.05emM\kern.05em A%
%      \kern.1em-\kern.1em %
%      Script%
%    }%
%  }%
%}{}
%\end{verbatim}
%\end{quote}
%    \begin{macrocode}
\def\HoLogo@KOMAScript#1{%
  \HoLogoFont@font{KOMAScript}{sf}{%
    \HOLOGO@mbox{%
      K\kern.05em%
      O\kern.05em%
      M\kern.05em%
      A%
    }%
    \kern.1em%
    \HOLOGO@hyphen
    \kern.1em%
    \HOLOGO@mbox{Script}%
  }%
}
%    \end{macrocode}
%    \end{macro}
%    \begin{macro}{\HoLogoBkm@KOMAScript}
%    \begin{macrocode}
\def\HoLogoBkm@KOMAScript#1{%
  KOMA-Script%
}
%    \end{macrocode}
%    \end{macro}
%    \begin{macro}{\HoLogoHtml@KOMAScript}
%    \begin{macrocode}
\def\HoLogoHtml@KOMAScript#1{%
  \HoLogoCss@KOMAScript
  \HoLogoFont@font{KOMAScript}{sf}{%
    \HOLOGO@Span{KOMAScript}{%
      K%
      \HOLOGO@Span{O}{O}%
      M%
      \HOLOGO@Span{A}{A}%
      \HOLOGO@Span{hyphen}{-}%
      Script%
    }%
  }%
}
%    \end{macrocode}
%    \end{macro}
%    \begin{macro}{\HoLogoCss@KOMAScript}
%    \begin{macrocode}
\def\HoLogoCss@KOMAScript{%
  \Css{%
    span.HoLogo-KOMAScript{%
      font-family:sans-serif;%
    }%
  }%
  \Css{%
    span.HoLogo-KOMAScript span.HoLogo-O{%
      padding-left:.05em;%
      padding-right:.05em;%
    }%
  }%
  \Css{%
    span.HoLogo-KOMAScript span.HoLogo-A{%
      padding-left:.05em;%
    }%
  }%
  \Css{%
    span.HoLogo-KOMAScript span.HoLogo-hyphen{%
      padding-left:.1em;%
      padding-right:.1em;%
    }%
  }%
  \global\let\HoLogoCss@KOMAScript\relax
}
%    \end{macrocode}
%    \end{macro}
%
% \subsubsection{\hologo{LyX}}
%
%    \begin{macro}{\HoLogo@LyX}
%    The definition is taken from the documentation source files
%    of \hologo{LyX}, \xfile{Intro.lyx} \cite{LyX}:
%\begin{quote}
%\begin{verbatim}
%\def\LyX{%
%  \texorpdfstring{%
%    L\kern-.1667em\lower.25em\hbox{Y}\kern-.125emX\@%
%  }{%
%    LyX%
%  }%
%}
%\end{verbatim}
%\end{quote}
%    \begin{macrocode}
\def\HoLogo@LyX#1{%
  L%
  \kern-.1667em%
  \lower.25em\hbox{Y}%
  \kern-.125em%
  X%
  \HOLOGO@SpaceFactor
}
%    \end{macrocode}
%    \end{macro}
%    \begin{macro}{\HoLogoHtml@LyX}
%    \begin{macrocode}
\def\HoLogoHtml@LyX#1{%
  \HoLogoCss@LyX
  \HOLOGO@Span{LyX}{%
    L%
    \HOLOGO@Span{y}{Y}%
    X%
  }%
}
%    \end{macrocode}
%    \end{macro}
%    \begin{macro}{\HoLogoCss@LyX}
%    \begin{macrocode}
\def\HoLogoCss@LyX{%
  \Css{%
    span.HoLogo-LyX span.HoLogo-y{%
      position:relative;%
      top:.25em;%
      margin-left:-.1667em;%
      margin-right:-.125em;%
      text-decoration:none;%
    }%
  }%
  \global\let\HoLogoCss@LyX\relax
}
%    \end{macrocode}
%    \end{macro}
%
% \subsubsection{\hologo{NTS}}
%
%    \begin{macro}{\HoLogo@NTS}
%    Definition for \hologo{NTS} can be found in
%    package \xpackage{etex\textunderscore man} for the \hologo{eTeX} manual \cite{etexman}
%    and in package \xpackage{dtklogos} \cite{dtklogos}:
%\begin{quote}
%\begin{verbatim}
%\def\NTS{%
%  \leavevmode
%  \hbox{%
%    $%
%      \cal N%
%      \kern-0.35em%
%      \lower0.5ex\hbox{$\cal T$}%
%      \kern-0.2em%
%      S%
%    $%
%  }%
%}
%\end{verbatim}
%\end{quote}
%    \begin{macrocode}
\def\HoLogo@NTS#1{%
  \HoLogoFont@font{NTS}{sy}{%
    N\/%
    \kern-.35em%
    \lower.5ex\hbox{T\/}%
    \kern-.2em%
    S\/%
  }%
  \HOLOGO@SpaceFactor
}
%    \end{macrocode}
%    \end{macro}
%
% \subsubsection{\Hologo{TTH} (\hologo{TeX} to HTML translator)}
%
%    Source: \url{http://hutchinson.belmont.ma.us/tth/}
%    In the HTML source the second `T' is printed as subscript.
%\begin{quote}
%\begin{verbatim}
%T<sub>T</sub>H
%\end{verbatim}
%\end{quote}
%    \begin{macro}{\HoLogo@TTH}
%    \begin{macrocode}
\def\HoLogo@TTH#1{%
  \ltx@mbox{%
    T\HOLOGO@SubScript{T}H%
  }%
  \HOLOGO@SpaceFactor
}
%    \end{macrocode}
%    \end{macro}
%
%    \begin{macro}{\HoLogoHtml@TTH}
%    \begin{macrocode}
\def\HoLogoHtml@TTH#1{%
  T\HCode{<sub>}T\HCode{</sub>}H%
}
%    \end{macrocode}
%    \end{macro}
%
% \subsubsection{\Hologo{HanTheThanh}}
%
%    Partial source: Package \xpackage{dtklogos}.
%    The double accent is U+1EBF (latin small letter e with circumflex
%    and acute).
%    \begin{macro}{\HoLogo@HanTheThanh}
%    \begin{macrocode}
\def\HoLogo@HanTheThanh#1{%
  \ltx@mbox{H\`an}%
  \HOLOGO@space
  \ltx@mbox{%
    Th%
    \HOLOGO@IfCharExists{"1EBF}{%
      \char"1EBF\relax
    }{%
      \^e\hbox to 0pt{\hss\raise .5ex\hbox{\'{}}}%
    }%
  }%
  \HOLOGO@space
  \ltx@mbox{Th\`anh}%
}
%    \end{macrocode}
%    \end{macro}
%    \begin{macro}{\HoLogoBkm@HanTheThanh}
%    \begin{macrocode}
\def\HoLogoBkm@HanTheThanh#1{%
  H\`an %
  Th\HOLOGO@PdfdocUnicode{\^e}{\9036\277} %
  Th\`anh%
}
%    \end{macrocode}
%    \end{macro}
%    \begin{macro}{\HoLogoHtml@HanTheThanh}
%    \begin{macrocode}
\def\HoLogoHtml@HanTheThanh#1{%
  H\`an %
  Th\HCode{&\ltx@hashchar x1ebf;} %
  Th\`anh%
}
%    \end{macrocode}
%    \end{macro}
%
% \subsection{Driver detection}
%
%    \begin{macrocode}
\HOLOGO@IfExists\InputIfFileExists{%
  \InputIfFileExists{hologo.cfg}{}{}%
}{%
  \ltx@IfUndefined{pdf@filesize}{%
    \def\HOLOGO@InputIfExists{%
      \openin\HOLOGO@temp=hologo.cfg\relax
      \ifeof\HOLOGO@temp
        \closein\HOLOGO@temp
      \else
        \closein\HOLOGO@temp
        \begingroup
          \def\x{LaTeX2e}%
        \expandafter\endgroup
        \ifx\fmtname\x
          \input{hologo.cfg}%
        \else
          \input hologo.cfg\relax
        \fi
      \fi
    }%
    \ltx@IfUndefined{newread}{%
      \chardef\HOLOGO@temp=15 %
      \def\HOLOGO@CheckRead{%
        \ifeof\HOLOGO@temp
          \HOLOGO@InputIfExists
        \else
          \ifcase\HOLOGO@temp
            \@PackageWarningNoLine{hologo}{%
              Configuration file ignored, because\MessageBreak
              a free read register could not be found%
            }%
          \else
            \begingroup
              \count\ltx@cclv=\HOLOGO@temp
              \advance\ltx@cclv by \ltx@minusone
              \edef\x{\endgroup
                \chardef\noexpand\HOLOGO@temp=\the\count\ltx@cclv
                \relax
              }%
            \x
          \fi
        \fi
      }%
    }{%
      \csname newread\endcsname\HOLOGO@temp
      \HOLOGO@InputIfExists
    }%
  }{%
    \edef\HOLOGO@temp{\pdf@filesize{hologo.cfg}}%
    \ifx\HOLOGO@temp\ltx@empty
    \else
      \ifnum\HOLOGO@temp>0 %
        \begingroup
          \def\x{LaTeX2e}%
        \expandafter\endgroup
        \ifx\fmtname\x
          \input{hologo.cfg}%
        \else
          \input hologo.cfg\relax
        \fi
      \else
        \@PackageInfoNoLine{hologo}{%
          Empty configuration file `hologo.cfg' ignored%
        }%
      \fi
    \fi
  }%
}
%    \end{macrocode}
%
%    \begin{macrocode}
\def\HOLOGO@temp#1#2{%
  \kv@define@key{HoLogoDriver}{#1}[]{%
    \begingroup
      \def\HOLOGO@temp{##1}%
      \ltx@onelevel@sanitize\HOLOGO@temp
      \ifx\HOLOGO@temp\ltx@empty
      \else
        \@PackageError{hologo}{%
          Value (\HOLOGO@temp) not permitted for option `#1'%
        }%
        \@ehc
      \fi
    \endgroup
    \def\hologoDriver{#2}%
  }%
}%
\def\HOLOGO@@temp#1#2{%
  \ifx\kv@value\relax
    \HOLOGO@temp{#1}{#1}%
  \else
    \HOLOGO@temp{#1}{#2}%
  \fi
}%
\kv@parse@normalized{%
  pdftex,%
  luatex=pdftex,%
  dvipdfm,%
  dvipdfmx=dvipdfm,%
  dvips,%
  dvipsone=dvips,%
  xdvi=dvips,%
  xetex,%
  vtex,%
}\HOLOGO@@temp
%    \end{macrocode}
%
%    \begin{macrocode}
\kv@define@key{HoLogoDriver}{driverfallback}{%
  \def\HOLOGO@DriverFallback{#1}%
}
%    \end{macrocode}
%
%    \begin{macro}{\HOLOGO@DriverFallback}
%    \begin{macrocode}
\def\HOLOGO@DriverFallback{dvips}
%    \end{macrocode}
%    \end{macro}
%
%    \begin{macro}{\hologoDriverSetup}
%    \begin{macrocode}
\def\hologoDriverSetup{%
  \let\hologoDriver\ltx@undefined
  \HOLOGO@DriverSetup
}
%    \end{macrocode}
%    \end{macro}
%
%    \begin{macro}{\HOLOGO@DriverSetup}
%    \begin{macrocode}
\def\HOLOGO@DriverSetup#1{%
  \kvsetkeys{HoLogoDriver}{#1}%
  \HOLOGO@CheckDriver
  \ltx@ifundefined{hologoDriver}{%
    \begingroup
    \edef\x{\endgroup
      \noexpand\kvsetkeys{HoLogoDriver}{\HOLOGO@DriverFallback}%
    }\x
  }{}%
  \@PackageInfoNoLine{hologo}{Using driver `\hologoDriver'}%
}
%    \end{macrocode}
%    \end{macro}
%
%    \begin{macro}{\HOLOGO@CheckDriver}
%    \begin{macrocode}
\def\HOLOGO@CheckDriver{%
  \ifpdf
    \def\hologoDriver{pdftex}%
    \let\HOLOGO@pdfliteral\pdfliteral
    \ifluatex
      \ifx\pdfextension\@undefined\else
        \protected\def\pdfliteral{\pdfextension literal}%
        \let\HOLOGO@pdfliteral\pdfliteral
      \fi
      \ltx@IfUndefined{HOLOGO@pdfliteral}{%
        \ifnum\luatexversion<36 %
        \else
          \begingroup
            \let\HOLOGO@temp\endgroup
            \ifcase0%
                \directlua{%
                  if tex.enableprimitives then %
                    tex.enableprimitives('HOLOGO@', {'pdfliteral'})%
                  else %
                    tex.print('1')%
                  end%
                }%
                \ifx\HOLOGO@pdfliteral\@undefined 1\fi%
                \relax%
              \endgroup
              \let\HOLOGO@temp\relax
              \global\let\HOLOGO@pdfliteral\HOLOGO@pdfliteral
            \fi%
          \HOLOGO@temp
        \fi
      }{}%
    \fi
    \ltx@IfUndefined{HOLOGO@pdfliteral}{%
      \@PackageWarningNoLine{hologo}{%
        Cannot find \string\pdfliteral
      }%
    }{}%
  \else
    \ifxetex
      \def\hologoDriver{xetex}%
    \else
      \ifvtex
        \def\hologoDriver{vtex}%
      \fi
    \fi
  \fi
}
%    \end{macrocode}
%    \end{macro}
%
%    \begin{macro}{\HOLOGO@WarningUnsupportedDriver}
%    \begin{macrocode}
\def\HOLOGO@WarningUnsupportedDriver#1{%
  \@PackageWarningNoLine{hologo}{%
    Logo `#1' needs driver specific macros,\MessageBreak
    but driver `\hologoDriver' is not supported.\MessageBreak
    Use a different driver or\MessageBreak
    load package `graphics' or `pgf'%
  }%
}
%    \end{macrocode}
%    \end{macro}
%
% \subsubsection{Reflect box macros}
%
%    Skip driver part if not needed.
%    \begin{macrocode}
\ltx@IfUndefined{reflectbox}{}{%
  \ltx@IfUndefined{rotatebox}{}{%
    \HOLOGO@AtEnd
  }%
}
\ltx@IfUndefined{pgftext}{}{%
  \HOLOGO@AtEnd
}
\ltx@IfUndefined{psscalebox}{}{%
  \HOLOGO@AtEnd
}
%    \end{macrocode}
%
%    \begin{macrocode}
\def\HOLOGO@temp{LaTeX2e}
\ifx\fmtname\HOLOGO@temp
  \RequirePackage{kvoptions}[2011/06/30]%
  \ProcessKeyvalOptions{HoLogoDriver}%
\fi
\HOLOGO@DriverSetup{}
%    \end{macrocode}
%
%    \begin{macro}{\HOLOGO@ReflectBox}
%    \begin{macrocode}
\def\HOLOGO@ReflectBox#1{%
  \begingroup
    \setbox\ltx@zero\hbox{\begingroup#1\endgroup}%
    \setbox\ltx@two\hbox{%
      \kern\wd\ltx@zero
      \csname HOLOGO@ScaleBox@\hologoDriver\endcsname{-1}{1}{%
        \hbox to 0pt{\copy\ltx@zero\hss}%
      }%
    }%
    \wd\ltx@two=\wd\ltx@zero
    \box\ltx@two
  \endgroup
}
%    \end{macrocode}
%    \end{macro}
%
%    \begin{macro}{\HOLOGO@PointReflectBox}
%    \begin{macrocode}
\def\HOLOGO@PointReflectBox#1{%
  \begingroup
    \setbox\ltx@zero\hbox{\begingroup#1\endgroup}%
    \setbox\ltx@two\hbox{%
      \kern\wd\ltx@zero
      \raise\ht\ltx@zero\hbox{%
        \csname HOLOGO@ScaleBox@\hologoDriver\endcsname{-1}{-1}{%
          \hbox to 0pt{\copy\ltx@zero\hss}%
        }%
      }%
    }%
    \wd\ltx@two=\wd\ltx@zero
    \box\ltx@two
  \endgroup
}
%    \end{macrocode}
%    \end{macro}
%
%    We must define all variants because of dynamic driver setup.
%    \begin{macrocode}
\def\HOLOGO@temp#1#2{#2}
%    \end{macrocode}
%
%    \begin{macro}{\HOLOGO@ScaleBox@pdftex}
%    \begin{macrocode}
\HOLOGO@temp{pdftex}{%
  \def\HOLOGO@ScaleBox@pdftex#1#2#3{%
    \HOLOGO@pdfliteral{%
      q #1 0 0 #2 0 0 cm%
    }%
    #3%
    \HOLOGO@pdfliteral{%
      Q%
    }%
  }%
}
%    \end{macrocode}
%    \end{macro}
%    \begin{macro}{\HOLOGO@ScaleBox@dvips}
%    \begin{macrocode}
\HOLOGO@temp{dvips}{%
  \def\HOLOGO@ScaleBox@dvips#1#2#3{%
    \special{ps:%
      gsave %
      currentpoint %
      currentpoint translate %
      #1 #2 scale %
      neg exch neg exch translate%
    }%
    #3%
    \special{ps:%
      currentpoint %
      grestore %
      moveto%
    }%
  }%
}
%    \end{macrocode}
%    \end{macro}
%    \begin{macro}{\HOLOGO@ScaleBox@dvipdfm}
%    \begin{macrocode}
\HOLOGO@temp{dvipdfm}{%
  \let\HOLOGO@ScaleBox@dvipdfm\HOLOGO@ScaleBox@dvips
}
%    \end{macrocode}
%    \end{macro}
%    Since \hologo{XeTeX} v0.6.
%    \begin{macro}{\HOLOGO@ScaleBox@xetex}
%    \begin{macrocode}
\HOLOGO@temp{xetex}{%
  \def\HOLOGO@ScaleBox@xetex#1#2#3{%
    \special{x:gsave}%
    \special{x:scale #1 #2}%
    #3%
    \special{x:grestore}%
  }%
}
%    \end{macrocode}
%    \end{macro}
%    \begin{macro}{\HOLOGO@ScaleBox@vtex}
%    \begin{macrocode}
\HOLOGO@temp{vtex}{%
  \def\HOLOGO@ScaleBox@vtex#1#2#3{%
    \special{r(#1,0,0,#2,0,0}%
    #3%
    \special{r)}%
  }%
}
%    \end{macrocode}
%    \end{macro}
%
%    \begin{macrocode}
\HOLOGO@AtEnd%
%</package>
%    \end{macrocode}
%
% \section{Test}
%
% \subsection{Catcode checks for loading}
%
%    \begin{macrocode}
%<*test1>
%    \end{macrocode}
%    \begin{macrocode}
\catcode`\{=1 %
\catcode`\}=2 %
\catcode`\#=6 %
\catcode`\@=11 %
\expandafter\ifx\csname count@\endcsname\relax
  \countdef\count@=255 %
\fi
\expandafter\ifx\csname @gobble\endcsname\relax
  \long\def\@gobble#1{}%
\fi
\expandafter\ifx\csname @firstofone\endcsname\relax
  \long\def\@firstofone#1{#1}%
\fi
\expandafter\ifx\csname loop\endcsname\relax
  \expandafter\@firstofone
\else
  \expandafter\@gobble
\fi
{%
  \def\loop#1\repeat{%
    \def\body{#1}%
    \iterate
  }%
  \def\iterate{%
    \body
      \let\next\iterate
    \else
      \let\next\relax
    \fi
    \next
  }%
  \let\repeat=\fi
}%
\def\RestoreCatcodes{}
\count@=0 %
\loop
  \edef\RestoreCatcodes{%
    \RestoreCatcodes
    \catcode\the\count@=\the\catcode\count@\relax
  }%
\ifnum\count@<255 %
  \advance\count@ 1 %
\repeat

\def\RangeCatcodeInvalid#1#2{%
  \count@=#1\relax
  \loop
    \catcode\count@=15 %
  \ifnum\count@<#2\relax
    \advance\count@ 1 %
  \repeat
}
\def\RangeCatcodeCheck#1#2#3{%
  \count@=#1\relax
  \loop
    \ifnum#3=\catcode\count@
    \else
      \errmessage{%
        Character \the\count@\space
        with wrong catcode \the\catcode\count@\space
        instead of \number#3%
      }%
    \fi
  \ifnum\count@<#2\relax
    \advance\count@ 1 %
  \repeat
}
\def\space{ }
\expandafter\ifx\csname LoadCommand\endcsname\relax
  \def\LoadCommand{\input hologo.sty\relax}%
\fi
\def\Test{%
  \RangeCatcodeInvalid{0}{47}%
  \RangeCatcodeInvalid{58}{64}%
  \RangeCatcodeInvalid{91}{96}%
  \RangeCatcodeInvalid{123}{255}%
  \catcode`\@=12 %
  \catcode`\\=0 %
  \catcode`\%=14 %
  \LoadCommand
  \RangeCatcodeCheck{0}{36}{15}%
  \RangeCatcodeCheck{37}{37}{14}%
  \RangeCatcodeCheck{38}{47}{15}%
  \RangeCatcodeCheck{48}{57}{12}%
  \RangeCatcodeCheck{58}{63}{15}%
  \RangeCatcodeCheck{64}{64}{12}%
  \RangeCatcodeCheck{65}{90}{11}%
  \RangeCatcodeCheck{91}{91}{15}%
  \RangeCatcodeCheck{92}{92}{0}%
  \RangeCatcodeCheck{93}{96}{15}%
  \RangeCatcodeCheck{97}{122}{11}%
  \RangeCatcodeCheck{123}{255}{15}%
  \RestoreCatcodes
}
\Test
\csname @@end\endcsname
\end
%    \end{macrocode}
%    \begin{macrocode}
%</test1>
%    \end{macrocode}
%
% \subsection{Spacefactor}
%
%    The space factor must be 1000 after a logo. If it is greater 1000
%    then the following space is a space after a sentence closing point.
%    If the space factor is smaller 1000 then an immediate following
%    dot is interpreted as abbreviation, not sentence closing point.
%
%    \begin{macrocode}
%<*test-spacefactor>
\NeedsTeXFormat{LaTeX2e}
\documentclass{article}
\usepackage{hologo}[2016/05/12]
\usepackage{kvsetkeys}
\usepackage{qstest}
\IncludeTests{*}
\LogTests{log}{*}{*}
\begin{document}
\begin{qstest}{spacefactor}{spacefactor}
\newcommand*{\Test}[1]{%
  \sbox0{%
    \hologo{#1}%
    \Expect*{1000 (#1)}*{\the\spacefactor\space(#1)}%
  }%
}%
\makeatletter
\def\TestList{}
\def\hologoEntry#1#2#3{%
  \edef\TestList{%
    \ifx\TestList\@empty
    \else
      \TestList,%
    \fi
    #1%
    \ifx\\#2\\%
    \else
      ={variant=#2}%
    \fi
  }%
}
\hologoList
\expandafter\kv@parse@normalized\expandafter{%
  \TestList
}{%
  \begingroup
    \let\@logo=\kv@key
    \ifx\kv@value\relax
    \else
      \expandafter\hologoLogoSetup\expandafter\@logo\expandafter{%
        \kv@value
      }%
    \fi
    \Test\@logo
  \endgroup
  \@gobbletwo
}
\end{qstest}
\end{document}
%</test-spacefactor>
%    \end{macrocode}
%
% \subsection{Complete list}
%
%    \begin{macrocode}
%<*test-list>
\NeedsTeXFormat{LaTeX2e}
\documentclass[12pt,a4paper]{article}
\usepackage{hologo}[2016/05/12]
\usepackage[T1]{fontenc}
\usepackage{lmodern}
\usepackage{parskip}
\usepackage[unicode]{hyperref}[2011/09/28]
\usepackage{bookmark}[2011/09/19]
\bookmarksetup{%
  numbered,%
  open,%
  openlevel=2,%
}
\renewcommand*{\contentsname}{List of logos}
\begin{document}
\tableofcontents
\def\TestFont#1#2#3#4#5#6{%
  \begingroup
    \usefont{#3}{#4}{#5}{#6}%
    \HologoVariant{#1}{#2}/\hologoVariant{#1}{#2}%
    \quad
    \begingroup\scriptsize\hologoVariant{#1}{#2}\endgroup
    \quad
  \endgroup
  (#3/#4/#5/#6)%
  \par
}
\makeatletter
\def\hologoEntry#1#2#3{%
  \section{%
    \HologoVariant{#1}{#2}/\hologoVariant{#1}{#2} %
    {[#1\ifx\\#2\\\else\space(#2)\fi]}% hash-ok
  }% braces around [] because of bug in tex4ht
  \begingroup
    \hypersetup{unicode=false}%
    \bookmark[%
      dest=\@currentHref,%
      rellevel=1,%
      keeplevel,%
    ]{%
      \HologoVariant{#1}{#2}/\hologoVariant{#1}{#2} %
      (PDFDocEncoding)%
    }%
  \endgroup
  \TestFont{#1}{#2}{OT1}{cmr}{m}{n}%
  \TestFont{#1}{#2}{OT1}{cmss}{m}{n}%
  \TestFont{#1}{#2}{OT1}{cmr}{b}{n}%
  \TestFont{#1}{#2}{OT1}{cmr}{m}{it}%
  \TestFont{#1}{#2}{OT1}{cmtt}{m}{n}%
  \TestFont{#1}{#2}{T1}{lmr}{m}{n}%
  \TestFont{#1}{#2}{T1}{lmss}{m}{n}%
  \TestFont{#1}{#2}{T1}{lmr}{b}{n}%
  \TestFont{#1}{#2}{T1}{lmr}{m}{it}%
  \TestFont{#1}{#2}{T1}{lmtt}{m}{n}%
  \TestFont{#1}{#2}{T1}{lmvtt}{m}{n}%
  \TestFont{#1}{#2}{T1}{qtm}{m}{n}%
  \TestFont{#1}{#2}{T1}{qhv}{m}{n}%
  \TestFont{#1}{#2}{T1}{qtm}{b}{n}%
  \TestFont{#1}{#2}{T1}{qtm}{m}{it}%
  \TestFont{#1}{#2}{T1}{qcr}{m}{n}%
  \newpage
}
\makeatother
\hologoList
\end{document}
%</test-list>
%    \end{macrocode}
%
% \section{Installation}
%
% \subsection{Download}
%
% \paragraph{Package.} This package is available on
% CTAN\footnote{\url{ftp://ftp.ctan.org/tex-archive/}}:
% \begin{description}
% \item[\CTAN{macros/latex/contrib/oberdiek/hologo.dtx}] The source file.
% \item[\CTAN{macros/latex/contrib/oberdiek/hologo.pdf}] Documentation.
% \end{description}
%
%
% \paragraph{Bundle.} All the packages of the bundle `oberdiek'
% are also available in a TDS compliant ZIP archive. There
% the packages are already unpacked and the documentation files
% are generated. The files and directories obey the TDS standard.
% \begin{description}
% \item[\CTAN{install/macros/latex/contrib/oberdiek.tds.zip}]
% \end{description}
% \emph{TDS} refers to the standard ``A Directory Structure
% for \TeX\ Files'' (\CTAN{tds/tds.pdf}). Directories
% with \xfile{texmf} in their name are usually organized this way.
%
% \subsection{Bundle installation}
%
% \paragraph{Unpacking.} Unpack the \xfile{oberdiek.tds.zip} in the
% TDS tree (also known as \xfile{texmf} tree) of your choice.
% Example (linux):
% \begin{quote}
%   |unzip oberdiek.tds.zip -d ~/texmf|
% \end{quote}
%
% \paragraph{Script installation.}
% Check the directory \xfile{TDS:scripts/oberdiek/} for
% scripts that need further installation steps.
% Package \xpackage{attachfile2} comes with the Perl script
% \xfile{pdfatfi.pl} that should be installed in such a way
% that it can be called as \texttt{pdfatfi}.
% Example (linux):
% \begin{quote}
%   |chmod +x scripts/oberdiek/pdfatfi.pl|\\
%   |cp scripts/oberdiek/pdfatfi.pl /usr/local/bin/|
% \end{quote}
%
% \subsection{Package installation}
%
% \paragraph{Unpacking.} The \xfile{.dtx} file is a self-extracting
% \docstrip\ archive. The files are extracted by running the
% \xfile{.dtx} through \plainTeX:
% \begin{quote}
%   \verb|tex hologo.dtx|
% \end{quote}
%
% \paragraph{TDS.} Now the different files must be moved into
% the different directories in your installation TDS tree
% (also known as \xfile{texmf} tree):
% \begin{quote}
% \def\t{^^A
% \begin{tabular}{@{}>{\ttfamily}l@{ $\rightarrow$ }>{\ttfamily}l@{}}
%   hologo.sty & tex/generic/oberdiek/hologo.sty\\
%   hologo.pdf & doc/latex/oberdiek/hologo.pdf\\
%   example/hologo-example.tex & doc/latex/oberdiek/example/hologo-example.tex\\
%   test/hologo-test1.tex & doc/latex/oberdiek/test/hologo-test1.tex\\
%   test/hologo-test-spacefactor.tex & doc/latex/oberdiek/test/hologo-test-spacefactor.tex\\
%   test/hologo-test-list.tex & doc/latex/oberdiek/test/hologo-test-list.tex\\
%   hologo.dtx & source/latex/oberdiek/hologo.dtx\\
% \end{tabular}^^A
% }^^A
% \sbox0{\t}^^A
% \ifdim\wd0>\linewidth
%   \begingroup
%     \advance\linewidth by\leftmargin
%     \advance\linewidth by\rightmargin
%   \edef\x{\endgroup
%     \def\noexpand\lw{\the\linewidth}^^A
%   }\x
%   \def\lwbox{^^A
%     \leavevmode
%     \hbox to \linewidth{^^A
%       \kern-\leftmargin\relax
%       \hss
%       \usebox0
%       \hss
%       \kern-\rightmargin\relax
%     }^^A
%   }^^A
%   \ifdim\wd0>\lw
%     \sbox0{\small\t}^^A
%     \ifdim\wd0>\linewidth
%       \ifdim\wd0>\lw
%         \sbox0{\footnotesize\t}^^A
%         \ifdim\wd0>\linewidth
%           \ifdim\wd0>\lw
%             \sbox0{\scriptsize\t}^^A
%             \ifdim\wd0>\linewidth
%               \ifdim\wd0>\lw
%                 \sbox0{\tiny\t}^^A
%                 \ifdim\wd0>\linewidth
%                   \lwbox
%                 \else
%                   \usebox0
%                 \fi
%               \else
%                 \lwbox
%               \fi
%             \else
%               \usebox0
%             \fi
%           \else
%             \lwbox
%           \fi
%         \else
%           \usebox0
%         \fi
%       \else
%         \lwbox
%       \fi
%     \else
%       \usebox0
%     \fi
%   \else
%     \lwbox
%   \fi
% \else
%   \usebox0
% \fi
% \end{quote}
% If you have a \xfile{docstrip.cfg} that configures and enables \docstrip's
% TDS installing feature, then some files can already be in the right
% place, see the documentation of \docstrip.
%
% \subsection{Refresh file name databases}
%
% If your \TeX~distribution
% (\teTeX, \mikTeX, \dots) relies on file name databases, you must refresh
% these. For example, \teTeX\ users run \verb|texhash| or
% \verb|mktexlsr|.
%
% \subsection{Some details for the interested}
%
% \paragraph{Attached source.}
%
% The PDF documentation on CTAN also includes the
% \xfile{.dtx} source file. It can be extracted by
% AcrobatReader 6 or higher. Another option is \textsf{pdftk},
% e.g. unpack the file into the current directory:
% \begin{quote}
%   \verb|pdftk hologo.pdf unpack_files output .|
% \end{quote}
%
% \paragraph{Unpacking with \LaTeX.}
% The \xfile{.dtx} chooses its action depending on the format:
% \begin{description}
% \item[\plainTeX:] Run \docstrip\ and extract the files.
% \item[\LaTeX:] Generate the documentation.
% \end{description}
% If you insist on using \LaTeX\ for \docstrip\ (really,
% \docstrip\ does not need \LaTeX), then inform the autodetect routine
% about your intention:
% \begin{quote}
%   \verb|latex \let\install=y\input{hologo.dtx}|
% \end{quote}
% Do not forget to quote the argument according to the demands
% of your shell.
%
% \paragraph{Generating the documentation.}
% You can use both the \xfile{.dtx} or the \xfile{.drv} to generate
% the documentation. The process can be configured by the
% configuration file \xfile{ltxdoc.cfg}. For instance, put this
% line into this file, if you want to have A4 as paper format:
% \begin{quote}
%   \verb|\PassOptionsToClass{a4paper}{article}|
% \end{quote}
% An example follows how to generate the
% documentation with pdf\LaTeX:
% \begin{quote}
%\begin{verbatim}
%pdflatex hologo.dtx
%makeindex -s gind.ist hologo.idx
%pdflatex hologo.dtx
%makeindex -s gind.ist hologo.idx
%pdflatex hologo.dtx
%\end{verbatim}
% \end{quote}
%
% \section{Catalogue}
%
% The following XML file can be used as source for the
% \href{http://mirror.ctan.org/help/Catalogue/catalogue.html}{\TeX\ Catalogue}.
% The elements \texttt{caption} and \texttt{description} are imported
% from the original XML file from the Catalogue.
% The name of the XML file in the Catalogue is \xfile{hologo.xml}.
%    \begin{macrocode}
%<*catalogue>
<?xml version='1.0' encoding='us-ascii'?>
<!DOCTYPE entry SYSTEM 'catalogue.dtd'>
<entry datestamp='$Date$' modifier='$Author$' id='hologo'>
  <name>hologo</name>
  <caption>A collection of logos with bookmark support.</caption>
  <authorref id='auth:oberdiek'/>
  <copyright owner='Heiko Oberdiek' year='2010-2012'/>
  <license type='lppl1.3'/>
  <version number='1.10'/>
  <description>
    The package defines a single command <tt>\hologo</tt>, whose
    argument is the usual case-confused ASCII version of the logo.
    The command is bookmark-enabled, so that every logo becomes
    available in bookmarks without further work.
    <p/>
    The package is part of the <xref refid='oberdiek'>oberdiek</xref>
    bundle.
  </description>
  <documentation details='Package documentation'
      href='ctan:/macros/latex/contrib/oberdiek/hologo.pdf'/>
  <ctan file='true' path='/macros/latex/contrib/oberdiek/hologo.dtx'/>
  <miktex location='oberdiek'/>
  <texlive location='oberdiek'/>
  <install path='/macros/latex/contrib/oberdiek/oberdiek.tds.zip'/>
</entry>
%</catalogue>
%    \end{macrocode}
%
% \begin{thebibliography}{9}
% \raggedright
%
% \bibitem{btxdoc}
% Oren Patashnik,
% \textit{\hologo{BibTeX}ing},
% 1988-02-08.\\
% \CTAN{biblio/bibtex/base/}
%
% \bibitem{dtklogos}
% Gerd Neugebauer, DANTE,
% \textit{Package \xpackage{dtklogos}},
% 2011-04-25.\\
% \CTAN{usergrps/dante/dtk/dtklogos.sty}
%
% \bibitem{etexman}
% The \hologo{NTS} Team,
% \textit{The \hologo{eTeX} manual},
% 1998-02.\\
% \CTAN{systems/e-tex/v2/doc/}
%
% \bibitem{ExTeX-FAQ}
% The \hologo{ExTeX} group,
% \textit{\hologo{ExTeX}: FAQ -- How is \hologo{ExTeX} typeset?},
% 2007-04-14.\\
% \url{http://www.extex.org/documentation/faq.html}
%
% \bibitem{LyX}
% %@MISC{ LyX,
% %  title = {{LyX 2.0.0 -- The Document Processor [Computer software and manual]}},
% %  author = {{The LyX Team}},
% %  howpublished = {Internet: http://www.lyx.org},
% %  year = {2011-05-08},
% %  note = {Retrieved May 10, 2011, from http://www.lyx.org},
% %  url = {http://www.lyx.org/}
% %}
% The \hologo{LyX} Team,
% \textit{\hologo{LyX} -- The Document Processor},
% 2011-05-08.\\
% \url{http://www.lyx.org/}
%
% \bibitem{OzTeX}
% Andrew Trevorrow,
% \hologo{OzTeX} FAQ: What is the correct way to typeset ``\hologo{OzTeX}''?,
% 2011-09-15 (visited).
% \url{http://www.trevorrow.com/oztex/ozfaq.html#oztex-logo}
%
% \bibitem{PiCTeX}
% Michael Wichura,
% \textit{The \hologo{PiCTeX} macro package},
% 1987-09-21.
% \CTAN{graphics/pictex/}
%
% \bibitem{scrlogo}
% Markus Kohm,
% \textit{\hologo{KOMAScript} Datei \xfile{scrlogo.dtx}},
% 2009-01-30.\\
% \CTAN{install/macros/latex/contrib/komascript.tds.zip}
%
% \end{thebibliography}
%
% \begin{History}
%   \begin{Version}{2010/04/08 v1.0}
%   \item
%     The first version.
%   \end{Version}
%   \begin{Version}{2010/04/16 v1.1}
%   \item
%     \cs{Hologo} added for support of logos at start of a sentence.
%   \item
%     \cs{hologoSetup} and \cs{hologoLogoSetup} added.
%   \item
%     Options \xoption{break}, \xoption{hyphenbreak}, \xoption{spacebreak}
%     added.
%   \item
%     Variant support added by option \xoption{variant}.
%   \end{Version}
%   \begin{Version}{2010/04/24 v1.2}
%   \item
%     \hologo{LaTeX3} added.
%   \item
%     \hologo{VTeX} added.
%   \end{Version}
%   \begin{Version}{2010/11/21 v1.3}
%   \item
%     \hologo{iniTeX}, \hologo{virTeX} added.
%   \end{Version}
%   \begin{Version}{2011/03/25 v1.4}
%   \item
%     \hologo{ConTeXt} with variants added.
%   \item
%     Option \xoption{discretionarybreak} added as refinement for
%     option \xoption{break}.
%   \end{Version}
%   \begin{Version}{2011/04/21 v1.5}
%   \item
%     Wrong TDS directory for test files fixed.
%   \end{Version}
%   \begin{Version}{2011/10/01 v1.6}
%   \item
%     Support for package \xpackage{tex4ht} added.
%   \item
%     Support for \cs{csname} added if \cs{ifincsname} is available.
%   \item
%     New logos:
%     \hologo{(La)TeX},
%     \hologo{biber},
%     \hologo{BibTeX} (\xoption{sc}, \xoption{sf}),
%     \hologo{emTeX},
%     \hologo{ExTeX},
%     \hologo{KOMAScript},
%     \hologo{La},
%     \hologo{LyX},
%     \hologo{MiKTeX},
%     \hologo{NTS},
%     \hologo{OzMF},
%     \hologo{OzMP},
%     \hologo{OzTeX},
%     \hologo{OzTtH},
%     \hologo{PCTeX},
%     \hologo{PiC},
%     \hologo{PiCTeX},
%     \hologo{METAFONT},
%     \hologo{MetaFun},
%     \hologo{METAPOST},
%     \hologo{MetaPost},
%     \hologo{SLiTeX} (\xoption{lift}, \xoption{narrow}, \xoption{simple}),
%     \hologo{SliTeX} (\xoption{narrow}, \xoption{simple}, \xoption{lift}),
%     \hologo{teTeX}.
%   \item
%     Fixes:
%     \hologo{iniTeX},
%     \hologo{pdfLaTeX},
%     \hologo{pdfTeX},
%     \hologo{virTeX}.
%   \item
%     \cs{hologoFontSetup} and \cs{hologoLogoFontSetup} added.
%   \item
%     \cs{hologoVariant} and \cs{HologoVariant} added.
%   \end{Version}
%   \begin{Version}{2011/11/22 v1.7}
%   \item
%     New logos:
%     \hologo{BibTeX8},
%     \hologo{LaTeXML},
%     \hologo{SageTeX},
%     \hologo{TeX4ht},
%     \hologo{TTH}.
%   \item
%     \hologo{Xe} and friends: Driver stuff fixed.
%   \item
%     \hologo{Xe} and friends: Support for italic added.
%   \item
%     \hologo{Xe} and friends: Package support for \xpackage{pgf}
%     and \xpackage{pstricks} added.
%   \end{Version}
%   \begin{Version}{2011/11/29 v1.8}
%   \item
%     New logos:
%     \hologo{HanTheThanh}.
%   \end{Version}
%   \begin{Version}{2011/12/21 v1.9}
%   \item
%     Patch for package \xpackage{ifxetex} added for the case that
%     \cs{newif} is undefined in \hologo{iniTeX}.
%   \item
%     Some fixes for \hologo{iniTeX}.
%   \end{Version}
%   \begin{Version}{2012/04/26 v1.10}
%   \item
%     Fix in bookmark version of logo ``\hologo{HanTheThanh}''.
%   \end{Version}
%   \begin{Version}{2016/05/12 v1.11}
%   \item
%     Update HOLOGO@IfCharExists (previously in texlive)
%   \item define pdfliteral in current luatex.
%   \end{Version}
% \end{History}
%
% \PrintIndex
%
% \Finale
\endinput
|
% \end{quote}
% Do not forget to quote the argument according to the demands
% of your shell.
%
% \paragraph{Generating the documentation.}
% You can use both the \xfile{.dtx} or the \xfile{.drv} to generate
% the documentation. The process can be configured by the
% configuration file \xfile{ltxdoc.cfg}. For instance, put this
% line into this file, if you want to have A4 as paper format:
% \begin{quote}
%   \verb|\PassOptionsToClass{a4paper}{article}|
% \end{quote}
% An example follows how to generate the
% documentation with pdf\LaTeX:
% \begin{quote}
%\begin{verbatim}
%pdflatex hologo.dtx
%makeindex -s gind.ist hologo.idx
%pdflatex hologo.dtx
%makeindex -s gind.ist hologo.idx
%pdflatex hologo.dtx
%\end{verbatim}
% \end{quote}
%
% \section{Catalogue}
%
% The following XML file can be used as source for the
% \href{http://mirror.ctan.org/help/Catalogue/catalogue.html}{\TeX\ Catalogue}.
% The elements \texttt{caption} and \texttt{description} are imported
% from the original XML file from the Catalogue.
% The name of the XML file in the Catalogue is \xfile{hologo.xml}.
%    \begin{macrocode}
%<*catalogue>
<?xml version='1.0' encoding='us-ascii'?>
<!DOCTYPE entry SYSTEM 'catalogue.dtd'>
<entry datestamp='$Date$' modifier='$Author$' id='hologo'>
  <name>hologo</name>
  <caption>A collection of logos with bookmark support.</caption>
  <authorref id='auth:oberdiek'/>
  <copyright owner='Heiko Oberdiek' year='2010-2012'/>
  <license type='lppl1.3'/>
  <version number='1.10'/>
  <description>
    The package defines a single command <tt>\hologo</tt>, whose
    argument is the usual case-confused ASCII version of the logo.
    The command is bookmark-enabled, so that every logo becomes
    available in bookmarks without further work.
    <p/>
    The package is part of the <xref refid='oberdiek'>oberdiek</xref>
    bundle.
  </description>
  <documentation details='Package documentation'
      href='ctan:/macros/latex/contrib/oberdiek/hologo.pdf'/>
  <ctan file='true' path='/macros/latex/contrib/oberdiek/hologo.dtx'/>
  <miktex location='oberdiek'/>
  <texlive location='oberdiek'/>
  <install path='/macros/latex/contrib/oberdiek/oberdiek.tds.zip'/>
</entry>
%</catalogue>
%    \end{macrocode}
%
% \begin{thebibliography}{9}
% \raggedright
%
% \bibitem{btxdoc}
% Oren Patashnik,
% \textit{\hologo{BibTeX}ing},
% 1988-02-08.\\
% \CTAN{biblio/bibtex/base/}
%
% \bibitem{dtklogos}
% Gerd Neugebauer, DANTE,
% \textit{Package \xpackage{dtklogos}},
% 2011-04-25.\\
% \CTAN{usergrps/dante/dtk/dtklogos.sty}
%
% \bibitem{etexman}
% The \hologo{NTS} Team,
% \textit{The \hologo{eTeX} manual},
% 1998-02.\\
% \CTAN{systems/e-tex/v2/doc/}
%
% \bibitem{ExTeX-FAQ}
% The \hologo{ExTeX} group,
% \textit{\hologo{ExTeX}: FAQ -- How is \hologo{ExTeX} typeset?},
% 2007-04-14.\\
% \url{http://www.extex.org/documentation/faq.html}
%
% \bibitem{LyX}
% %@MISC{ LyX,
% %  title = {{LyX 2.0.0 -- The Document Processor [Computer software and manual]}},
% %  author = {{The LyX Team}},
% %  howpublished = {Internet: http://www.lyx.org},
% %  year = {2011-05-08},
% %  note = {Retrieved May 10, 2011, from http://www.lyx.org},
% %  url = {http://www.lyx.org/}
% %}
% The \hologo{LyX} Team,
% \textit{\hologo{LyX} -- The Document Processor},
% 2011-05-08.\\
% \url{http://www.lyx.org/}
%
% \bibitem{OzTeX}
% Andrew Trevorrow,
% \hologo{OzTeX} FAQ: What is the correct way to typeset ``\hologo{OzTeX}''?,
% 2011-09-15 (visited).
% \url{http://www.trevorrow.com/oztex/ozfaq.html#oztex-logo}
%
% \bibitem{PiCTeX}
% Michael Wichura,
% \textit{The \hologo{PiCTeX} macro package},
% 1987-09-21.
% \CTAN{graphics/pictex/}
%
% \bibitem{scrlogo}
% Markus Kohm,
% \textit{\hologo{KOMAScript} Datei \xfile{scrlogo.dtx}},
% 2009-01-30.\\
% \CTAN{install/macros/latex/contrib/komascript.tds.zip}
%
% \end{thebibliography}
%
% \begin{History}
%   \begin{Version}{2010/04/08 v1.0}
%   \item
%     The first version.
%   \end{Version}
%   \begin{Version}{2010/04/16 v1.1}
%   \item
%     \cs{Hologo} added for support of logos at start of a sentence.
%   \item
%     \cs{hologoSetup} and \cs{hologoLogoSetup} added.
%   \item
%     Options \xoption{break}, \xoption{hyphenbreak}, \xoption{spacebreak}
%     added.
%   \item
%     Variant support added by option \xoption{variant}.
%   \end{Version}
%   \begin{Version}{2010/04/24 v1.2}
%   \item
%     \hologo{LaTeX3} added.
%   \item
%     \hologo{VTeX} added.
%   \end{Version}
%   \begin{Version}{2010/11/21 v1.3}
%   \item
%     \hologo{iniTeX}, \hologo{virTeX} added.
%   \end{Version}
%   \begin{Version}{2011/03/25 v1.4}
%   \item
%     \hologo{ConTeXt} with variants added.
%   \item
%     Option \xoption{discretionarybreak} added as refinement for
%     option \xoption{break}.
%   \end{Version}
%   \begin{Version}{2011/04/21 v1.5}
%   \item
%     Wrong TDS directory for test files fixed.
%   \end{Version}
%   \begin{Version}{2011/10/01 v1.6}
%   \item
%     Support for package \xpackage{tex4ht} added.
%   \item
%     Support for \cs{csname} added if \cs{ifincsname} is available.
%   \item
%     New logos:
%     \hologo{(La)TeX},
%     \hologo{biber},
%     \hologo{BibTeX} (\xoption{sc}, \xoption{sf}),
%     \hologo{emTeX},
%     \hologo{ExTeX},
%     \hologo{KOMAScript},
%     \hologo{La},
%     \hologo{LyX},
%     \hologo{MiKTeX},
%     \hologo{NTS},
%     \hologo{OzMF},
%     \hologo{OzMP},
%     \hologo{OzTeX},
%     \hologo{OzTtH},
%     \hologo{PCTeX},
%     \hologo{PiC},
%     \hologo{PiCTeX},
%     \hologo{METAFONT},
%     \hologo{MetaFun},
%     \hologo{METAPOST},
%     \hologo{MetaPost},
%     \hologo{SLiTeX} (\xoption{lift}, \xoption{narrow}, \xoption{simple}),
%     \hologo{SliTeX} (\xoption{narrow}, \xoption{simple}, \xoption{lift}),
%     \hologo{teTeX}.
%   \item
%     Fixes:
%     \hologo{iniTeX},
%     \hologo{pdfLaTeX},
%     \hologo{pdfTeX},
%     \hologo{virTeX}.
%   \item
%     \cs{hologoFontSetup} and \cs{hologoLogoFontSetup} added.
%   \item
%     \cs{hologoVariant} and \cs{HologoVariant} added.
%   \end{Version}
%   \begin{Version}{2011/11/22 v1.7}
%   \item
%     New logos:
%     \hologo{BibTeX8},
%     \hologo{LaTeXML},
%     \hologo{SageTeX},
%     \hologo{TeX4ht},
%     \hologo{TTH}.
%   \item
%     \hologo{Xe} and friends: Driver stuff fixed.
%   \item
%     \hologo{Xe} and friends: Support for italic added.
%   \item
%     \hologo{Xe} and friends: Package support for \xpackage{pgf}
%     and \xpackage{pstricks} added.
%   \end{Version}
%   \begin{Version}{2011/11/29 v1.8}
%   \item
%     New logos:
%     \hologo{HanTheThanh}.
%   \end{Version}
%   \begin{Version}{2011/12/21 v1.9}
%   \item
%     Patch for package \xpackage{ifxetex} added for the case that
%     \cs{newif} is undefined in \hologo{iniTeX}.
%   \item
%     Some fixes for \hologo{iniTeX}.
%   \end{Version}
%   \begin{Version}{2012/04/26 v1.10}
%   \item
%     Fix in bookmark version of logo ``\hologo{HanTheThanh}''.
%   \end{Version}
%   \begin{Version}{2016/05/12 v1.11}
%   \item
%     Update HOLOGO@IfCharExists (previously in texlive)
%   \item define pdfliteral in current luatex.
%   \end{Version}
% \end{History}
%
% \PrintIndex
%
% \Finale
\endinput
%
        \else
          \input hologo.cfg\relax
        \fi
      \else
        \@PackageInfoNoLine{hologo}{%
          Empty configuration file `hologo.cfg' ignored%
        }%
      \fi
    \fi
  }%
}
%    \end{macrocode}
%
%    \begin{macrocode}
\def\HOLOGO@temp#1#2{%
  \kv@define@key{HoLogoDriver}{#1}[]{%
    \begingroup
      \def\HOLOGO@temp{##1}%
      \ltx@onelevel@sanitize\HOLOGO@temp
      \ifx\HOLOGO@temp\ltx@empty
      \else
        \@PackageError{hologo}{%
          Value (\HOLOGO@temp) not permitted for option `#1'%
        }%
        \@ehc
      \fi
    \endgroup
    \def\hologoDriver{#2}%
  }%
}%
\def\HOLOGO@@temp#1#2{%
  \ifx\kv@value\relax
    \HOLOGO@temp{#1}{#1}%
  \else
    \HOLOGO@temp{#1}{#2}%
  \fi
}%
\kv@parse@normalized{%
  pdftex,%
  luatex=pdftex,%
  dvipdfm,%
  dvipdfmx=dvipdfm,%
  dvips,%
  dvipsone=dvips,%
  xdvi=dvips,%
  xetex,%
  vtex,%
}\HOLOGO@@temp
%    \end{macrocode}
%
%    \begin{macrocode}
\kv@define@key{HoLogoDriver}{driverfallback}{%
  \def\HOLOGO@DriverFallback{#1}%
}
%    \end{macrocode}
%
%    \begin{macro}{\HOLOGO@DriverFallback}
%    \begin{macrocode}
\def\HOLOGO@DriverFallback{dvips}
%    \end{macrocode}
%    \end{macro}
%
%    \begin{macro}{\hologoDriverSetup}
%    \begin{macrocode}
\def\hologoDriverSetup{%
  \let\hologoDriver\ltx@undefined
  \HOLOGO@DriverSetup
}
%    \end{macrocode}
%    \end{macro}
%
%    \begin{macro}{\HOLOGO@DriverSetup}
%    \begin{macrocode}
\def\HOLOGO@DriverSetup#1{%
  \kvsetkeys{HoLogoDriver}{#1}%
  \HOLOGO@CheckDriver
  \ltx@ifundefined{hologoDriver}{%
    \begingroup
    \edef\x{\endgroup
      \noexpand\kvsetkeys{HoLogoDriver}{\HOLOGO@DriverFallback}%
    }\x
  }{}%
  \@PackageInfoNoLine{hologo}{Using driver `\hologoDriver'}%
}
%    \end{macrocode}
%    \end{macro}
%
%    \begin{macro}{\HOLOGO@CheckDriver}
%    \begin{macrocode}
\def\HOLOGO@CheckDriver{%
  \ifpdf
    \def\hologoDriver{pdftex}%
    \let\HOLOGO@pdfliteral\pdfliteral
    \ifluatex
      \ifx\pdfextension\@undefined\else
        \protected\def\pdfliteral{\pdfextension literal}%
        \let\HOLOGO@pdfliteral\pdfliteral
      \fi
      \ltx@IfUndefined{HOLOGO@pdfliteral}{%
        \ifnum\luatexversion<36 %
        \else
          \begingroup
            \let\HOLOGO@temp\endgroup
            \ifcase0%
                \directlua{%
                  if tex.enableprimitives then %
                    tex.enableprimitives('HOLOGO@', {'pdfliteral'})%
                  else %
                    tex.print('1')%
                  end%
                }%
                \ifx\HOLOGO@pdfliteral\@undefined 1\fi%
                \relax%
              \endgroup
              \let\HOLOGO@temp\relax
              \global\let\HOLOGO@pdfliteral\HOLOGO@pdfliteral
            \fi%
          \HOLOGO@temp
        \fi
      }{}%
    \fi
    \ltx@IfUndefined{HOLOGO@pdfliteral}{%
      \@PackageWarningNoLine{hologo}{%
        Cannot find \string\pdfliteral
      }%
    }{}%
  \else
    \ifxetex
      \def\hologoDriver{xetex}%
    \else
      \ifvtex
        \def\hologoDriver{vtex}%
      \fi
    \fi
  \fi
}
%    \end{macrocode}
%    \end{macro}
%
%    \begin{macro}{\HOLOGO@WarningUnsupportedDriver}
%    \begin{macrocode}
\def\HOLOGO@WarningUnsupportedDriver#1{%
  \@PackageWarningNoLine{hologo}{%
    Logo `#1' needs driver specific macros,\MessageBreak
    but driver `\hologoDriver' is not supported.\MessageBreak
    Use a different driver or\MessageBreak
    load package `graphics' or `pgf'%
  }%
}
%    \end{macrocode}
%    \end{macro}
%
% \subsubsection{Reflect box macros}
%
%    Skip driver part if not needed.
%    \begin{macrocode}
\ltx@IfUndefined{reflectbox}{}{%
  \ltx@IfUndefined{rotatebox}{}{%
    \HOLOGO@AtEnd
  }%
}
\ltx@IfUndefined{pgftext}{}{%
  \HOLOGO@AtEnd
}
\ltx@IfUndefined{psscalebox}{}{%
  \HOLOGO@AtEnd
}
%    \end{macrocode}
%
%    \begin{macrocode}
\def\HOLOGO@temp{LaTeX2e}
\ifx\fmtname\HOLOGO@temp
  \RequirePackage{kvoptions}[2011/06/30]%
  \ProcessKeyvalOptions{HoLogoDriver}%
\fi
\HOLOGO@DriverSetup{}
%    \end{macrocode}
%
%    \begin{macro}{\HOLOGO@ReflectBox}
%    \begin{macrocode}
\def\HOLOGO@ReflectBox#1{%
  \begingroup
    \setbox\ltx@zero\hbox{\begingroup#1\endgroup}%
    \setbox\ltx@two\hbox{%
      \kern\wd\ltx@zero
      \csname HOLOGO@ScaleBox@\hologoDriver\endcsname{-1}{1}{%
        \hbox to 0pt{\copy\ltx@zero\hss}%
      }%
    }%
    \wd\ltx@two=\wd\ltx@zero
    \box\ltx@two
  \endgroup
}
%    \end{macrocode}
%    \end{macro}
%
%    \begin{macro}{\HOLOGO@PointReflectBox}
%    \begin{macrocode}
\def\HOLOGO@PointReflectBox#1{%
  \begingroup
    \setbox\ltx@zero\hbox{\begingroup#1\endgroup}%
    \setbox\ltx@two\hbox{%
      \kern\wd\ltx@zero
      \raise\ht\ltx@zero\hbox{%
        \csname HOLOGO@ScaleBox@\hologoDriver\endcsname{-1}{-1}{%
          \hbox to 0pt{\copy\ltx@zero\hss}%
        }%
      }%
    }%
    \wd\ltx@two=\wd\ltx@zero
    \box\ltx@two
  \endgroup
}
%    \end{macrocode}
%    \end{macro}
%
%    We must define all variants because of dynamic driver setup.
%    \begin{macrocode}
\def\HOLOGO@temp#1#2{#2}
%    \end{macrocode}
%
%    \begin{macro}{\HOLOGO@ScaleBox@pdftex}
%    \begin{macrocode}
\HOLOGO@temp{pdftex}{%
  \def\HOLOGO@ScaleBox@pdftex#1#2#3{%
    \HOLOGO@pdfliteral{%
      q #1 0 0 #2 0 0 cm%
    }%
    #3%
    \HOLOGO@pdfliteral{%
      Q%
    }%
  }%
}
%    \end{macrocode}
%    \end{macro}
%    \begin{macro}{\HOLOGO@ScaleBox@dvips}
%    \begin{macrocode}
\HOLOGO@temp{dvips}{%
  \def\HOLOGO@ScaleBox@dvips#1#2#3{%
    \special{ps:%
      gsave %
      currentpoint %
      currentpoint translate %
      #1 #2 scale %
      neg exch neg exch translate%
    }%
    #3%
    \special{ps:%
      currentpoint %
      grestore %
      moveto%
    }%
  }%
}
%    \end{macrocode}
%    \end{macro}
%    \begin{macro}{\HOLOGO@ScaleBox@dvipdfm}
%    \begin{macrocode}
\HOLOGO@temp{dvipdfm}{%
  \let\HOLOGO@ScaleBox@dvipdfm\HOLOGO@ScaleBox@dvips
}
%    \end{macrocode}
%    \end{macro}
%    Since \hologo{XeTeX} v0.6.
%    \begin{macro}{\HOLOGO@ScaleBox@xetex}
%    \begin{macrocode}
\HOLOGO@temp{xetex}{%
  \def\HOLOGO@ScaleBox@xetex#1#2#3{%
    \special{x:gsave}%
    \special{x:scale #1 #2}%
    #3%
    \special{x:grestore}%
  }%
}
%    \end{macrocode}
%    \end{macro}
%    \begin{macro}{\HOLOGO@ScaleBox@vtex}
%    \begin{macrocode}
\HOLOGO@temp{vtex}{%
  \def\HOLOGO@ScaleBox@vtex#1#2#3{%
    \special{r(#1,0,0,#2,0,0}%
    #3%
    \special{r)}%
  }%
}
%    \end{macrocode}
%    \end{macro}
%
%    \begin{macrocode}
\HOLOGO@AtEnd%
%</package>
%    \end{macrocode}
%
% \section{Test}
%
% \subsection{Catcode checks for loading}
%
%    \begin{macrocode}
%<*test1>
%    \end{macrocode}
%    \begin{macrocode}
\catcode`\{=1 %
\catcode`\}=2 %
\catcode`\#=6 %
\catcode`\@=11 %
\expandafter\ifx\csname count@\endcsname\relax
  \countdef\count@=255 %
\fi
\expandafter\ifx\csname @gobble\endcsname\relax
  \long\def\@gobble#1{}%
\fi
\expandafter\ifx\csname @firstofone\endcsname\relax
  \long\def\@firstofone#1{#1}%
\fi
\expandafter\ifx\csname loop\endcsname\relax
  \expandafter\@firstofone
\else
  \expandafter\@gobble
\fi
{%
  \def\loop#1\repeat{%
    \def\body{#1}%
    \iterate
  }%
  \def\iterate{%
    \body
      \let\next\iterate
    \else
      \let\next\relax
    \fi
    \next
  }%
  \let\repeat=\fi
}%
\def\RestoreCatcodes{}
\count@=0 %
\loop
  \edef\RestoreCatcodes{%
    \RestoreCatcodes
    \catcode\the\count@=\the\catcode\count@\relax
  }%
\ifnum\count@<255 %
  \advance\count@ 1 %
\repeat

\def\RangeCatcodeInvalid#1#2{%
  \count@=#1\relax
  \loop
    \catcode\count@=15 %
  \ifnum\count@<#2\relax
    \advance\count@ 1 %
  \repeat
}
\def\RangeCatcodeCheck#1#2#3{%
  \count@=#1\relax
  \loop
    \ifnum#3=\catcode\count@
    \else
      \errmessage{%
        Character \the\count@\space
        with wrong catcode \the\catcode\count@\space
        instead of \number#3%
      }%
    \fi
  \ifnum\count@<#2\relax
    \advance\count@ 1 %
  \repeat
}
\def\space{ }
\expandafter\ifx\csname LoadCommand\endcsname\relax
  \def\LoadCommand{\input hologo.sty\relax}%
\fi
\def\Test{%
  \RangeCatcodeInvalid{0}{47}%
  \RangeCatcodeInvalid{58}{64}%
  \RangeCatcodeInvalid{91}{96}%
  \RangeCatcodeInvalid{123}{255}%
  \catcode`\@=12 %
  \catcode`\\=0 %
  \catcode`\%=14 %
  \LoadCommand
  \RangeCatcodeCheck{0}{36}{15}%
  \RangeCatcodeCheck{37}{37}{14}%
  \RangeCatcodeCheck{38}{47}{15}%
  \RangeCatcodeCheck{48}{57}{12}%
  \RangeCatcodeCheck{58}{63}{15}%
  \RangeCatcodeCheck{64}{64}{12}%
  \RangeCatcodeCheck{65}{90}{11}%
  \RangeCatcodeCheck{91}{91}{15}%
  \RangeCatcodeCheck{92}{92}{0}%
  \RangeCatcodeCheck{93}{96}{15}%
  \RangeCatcodeCheck{97}{122}{11}%
  \RangeCatcodeCheck{123}{255}{15}%
  \RestoreCatcodes
}
\Test
\csname @@end\endcsname
\end
%    \end{macrocode}
%    \begin{macrocode}
%</test1>
%    \end{macrocode}
%
% \subsection{Spacefactor}
%
%    The space factor must be 1000 after a logo. If it is greater 1000
%    then the following space is a space after a sentence closing point.
%    If the space factor is smaller 1000 then an immediate following
%    dot is interpreted as abbreviation, not sentence closing point.
%
%    \begin{macrocode}
%<*test-spacefactor>
\NeedsTeXFormat{LaTeX2e}
\documentclass{article}
\usepackage{hologo}[2016/05/12]
\usepackage{kvsetkeys}
\usepackage{qstest}
\IncludeTests{*}
\LogTests{log}{*}{*}
\begin{document}
\begin{qstest}{spacefactor}{spacefactor}
\newcommand*{\Test}[1]{%
  \sbox0{%
    \hologo{#1}%
    \Expect*{1000 (#1)}*{\the\spacefactor\space(#1)}%
  }%
}%
\makeatletter
\def\TestList{}
\def\hologoEntry#1#2#3{%
  \edef\TestList{%
    \ifx\TestList\@empty
    \else
      \TestList,%
    \fi
    #1%
    \ifx\\#2\\%
    \else
      ={variant=#2}%
    \fi
  }%
}
\hologoList
\expandafter\kv@parse@normalized\expandafter{%
  \TestList
}{%
  \begingroup
    \let\@logo=\kv@key
    \ifx\kv@value\relax
    \else
      \expandafter\hologoLogoSetup\expandafter\@logo\expandafter{%
        \kv@value
      }%
    \fi
    \Test\@logo
  \endgroup
  \@gobbletwo
}
\end{qstest}
\end{document}
%</test-spacefactor>
%    \end{macrocode}
%
% \subsection{Complete list}
%
%    \begin{macrocode}
%<*test-list>
\NeedsTeXFormat{LaTeX2e}
\documentclass[12pt,a4paper]{article}
\usepackage{hologo}[2016/05/12]
\usepackage[T1]{fontenc}
\usepackage{lmodern}
\usepackage{parskip}
\usepackage[unicode]{hyperref}[2011/09/28]
\usepackage{bookmark}[2011/09/19]
\bookmarksetup{%
  numbered,%
  open,%
  openlevel=2,%
}
\renewcommand*{\contentsname}{List of logos}
\begin{document}
\tableofcontents
\def\TestFont#1#2#3#4#5#6{%
  \begingroup
    \usefont{#3}{#4}{#5}{#6}%
    \HologoVariant{#1}{#2}/\hologoVariant{#1}{#2}%
    \quad
    \begingroup\scriptsize\hologoVariant{#1}{#2}\endgroup
    \quad
  \endgroup
  (#3/#4/#5/#6)%
  \par
}
\makeatletter
\def\hologoEntry#1#2#3{%
  \section{%
    \HologoVariant{#1}{#2}/\hologoVariant{#1}{#2} %
    {[#1\ifx\\#2\\\else\space(#2)\fi]}% hash-ok
  }% braces around [] because of bug in tex4ht
  \begingroup
    \hypersetup{unicode=false}%
    \bookmark[%
      dest=\@currentHref,%
      rellevel=1,%
      keeplevel,%
    ]{%
      \HologoVariant{#1}{#2}/\hologoVariant{#1}{#2} %
      (PDFDocEncoding)%
    }%
  \endgroup
  \TestFont{#1}{#2}{OT1}{cmr}{m}{n}%
  \TestFont{#1}{#2}{OT1}{cmss}{m}{n}%
  \TestFont{#1}{#2}{OT1}{cmr}{b}{n}%
  \TestFont{#1}{#2}{OT1}{cmr}{m}{it}%
  \TestFont{#1}{#2}{OT1}{cmtt}{m}{n}%
  \TestFont{#1}{#2}{T1}{lmr}{m}{n}%
  \TestFont{#1}{#2}{T1}{lmss}{m}{n}%
  \TestFont{#1}{#2}{T1}{lmr}{b}{n}%
  \TestFont{#1}{#2}{T1}{lmr}{m}{it}%
  \TestFont{#1}{#2}{T1}{lmtt}{m}{n}%
  \TestFont{#1}{#2}{T1}{lmvtt}{m}{n}%
  \TestFont{#1}{#2}{T1}{qtm}{m}{n}%
  \TestFont{#1}{#2}{T1}{qhv}{m}{n}%
  \TestFont{#1}{#2}{T1}{qtm}{b}{n}%
  \TestFont{#1}{#2}{T1}{qtm}{m}{it}%
  \TestFont{#1}{#2}{T1}{qcr}{m}{n}%
  \newpage
}
\makeatother
\hologoList
\end{document}
%</test-list>
%    \end{macrocode}
%
% \section{Installation}
%
% \subsection{Download}
%
% \paragraph{Package.} This package is available on
% CTAN\footnote{\url{ftp://ftp.ctan.org/tex-archive/}}:
% \begin{description}
% \item[\CTAN{macros/latex/contrib/oberdiek/hologo.dtx}] The source file.
% \item[\CTAN{macros/latex/contrib/oberdiek/hologo.pdf}] Documentation.
% \end{description}
%
%
% \paragraph{Bundle.} All the packages of the bundle `oberdiek'
% are also available in a TDS compliant ZIP archive. There
% the packages are already unpacked and the documentation files
% are generated. The files and directories obey the TDS standard.
% \begin{description}
% \item[\CTAN{install/macros/latex/contrib/oberdiek.tds.zip}]
% \end{description}
% \emph{TDS} refers to the standard ``A Directory Structure
% for \TeX\ Files'' (\CTAN{tds/tds.pdf}). Directories
% with \xfile{texmf} in their name are usually organized this way.
%
% \subsection{Bundle installation}
%
% \paragraph{Unpacking.} Unpack the \xfile{oberdiek.tds.zip} in the
% TDS tree (also known as \xfile{texmf} tree) of your choice.
% Example (linux):
% \begin{quote}
%   |unzip oberdiek.tds.zip -d ~/texmf|
% \end{quote}
%
% \paragraph{Script installation.}
% Check the directory \xfile{TDS:scripts/oberdiek/} for
% scripts that need further installation steps.
% Package \xpackage{attachfile2} comes with the Perl script
% \xfile{pdfatfi.pl} that should be installed in such a way
% that it can be called as \texttt{pdfatfi}.
% Example (linux):
% \begin{quote}
%   |chmod +x scripts/oberdiek/pdfatfi.pl|\\
%   |cp scripts/oberdiek/pdfatfi.pl /usr/local/bin/|
% \end{quote}
%
% \subsection{Package installation}
%
% \paragraph{Unpacking.} The \xfile{.dtx} file is a self-extracting
% \docstrip\ archive. The files are extracted by running the
% \xfile{.dtx} through \plainTeX:
% \begin{quote}
%   \verb|tex hologo.dtx|
% \end{quote}
%
% \paragraph{TDS.} Now the different files must be moved into
% the different directories in your installation TDS tree
% (also known as \xfile{texmf} tree):
% \begin{quote}
% \def\t{^^A
% \begin{tabular}{@{}>{\ttfamily}l@{ $\rightarrow$ }>{\ttfamily}l@{}}
%   hologo.sty & tex/generic/oberdiek/hologo.sty\\
%   hologo.pdf & doc/latex/oberdiek/hologo.pdf\\
%   example/hologo-example.tex & doc/latex/oberdiek/example/hologo-example.tex\\
%   test/hologo-test1.tex & doc/latex/oberdiek/test/hologo-test1.tex\\
%   test/hologo-test-spacefactor.tex & doc/latex/oberdiek/test/hologo-test-spacefactor.tex\\
%   test/hologo-test-list.tex & doc/latex/oberdiek/test/hologo-test-list.tex\\
%   hologo.dtx & source/latex/oberdiek/hologo.dtx\\
% \end{tabular}^^A
% }^^A
% \sbox0{\t}^^A
% \ifdim\wd0>\linewidth
%   \begingroup
%     \advance\linewidth by\leftmargin
%     \advance\linewidth by\rightmargin
%   \edef\x{\endgroup
%     \def\noexpand\lw{\the\linewidth}^^A
%   }\x
%   \def\lwbox{^^A
%     \leavevmode
%     \hbox to \linewidth{^^A
%       \kern-\leftmargin\relax
%       \hss
%       \usebox0
%       \hss
%       \kern-\rightmargin\relax
%     }^^A
%   }^^A
%   \ifdim\wd0>\lw
%     \sbox0{\small\t}^^A
%     \ifdim\wd0>\linewidth
%       \ifdim\wd0>\lw
%         \sbox0{\footnotesize\t}^^A
%         \ifdim\wd0>\linewidth
%           \ifdim\wd0>\lw
%             \sbox0{\scriptsize\t}^^A
%             \ifdim\wd0>\linewidth
%               \ifdim\wd0>\lw
%                 \sbox0{\tiny\t}^^A
%                 \ifdim\wd0>\linewidth
%                   \lwbox
%                 \else
%                   \usebox0
%                 \fi
%               \else
%                 \lwbox
%               \fi
%             \else
%               \usebox0
%             \fi
%           \else
%             \lwbox
%           \fi
%         \else
%           \usebox0
%         \fi
%       \else
%         \lwbox
%       \fi
%     \else
%       \usebox0
%     \fi
%   \else
%     \lwbox
%   \fi
% \else
%   \usebox0
% \fi
% \end{quote}
% If you have a \xfile{docstrip.cfg} that configures and enables \docstrip's
% TDS installing feature, then some files can already be in the right
% place, see the documentation of \docstrip.
%
% \subsection{Refresh file name databases}
%
% If your \TeX~distribution
% (\teTeX, \mikTeX, \dots) relies on file name databases, you must refresh
% these. For example, \teTeX\ users run \verb|texhash| or
% \verb|mktexlsr|.
%
% \subsection{Some details for the interested}
%
% \paragraph{Attached source.}
%
% The PDF documentation on CTAN also includes the
% \xfile{.dtx} source file. It can be extracted by
% AcrobatReader 6 or higher. Another option is \textsf{pdftk},
% e.g. unpack the file into the current directory:
% \begin{quote}
%   \verb|pdftk hologo.pdf unpack_files output .|
% \end{quote}
%
% \paragraph{Unpacking with \LaTeX.}
% The \xfile{.dtx} chooses its action depending on the format:
% \begin{description}
% \item[\plainTeX:] Run \docstrip\ and extract the files.
% \item[\LaTeX:] Generate the documentation.
% \end{description}
% If you insist on using \LaTeX\ for \docstrip\ (really,
% \docstrip\ does not need \LaTeX), then inform the autodetect routine
% about your intention:
% \begin{quote}
%   \verb|latex \let\install=y% \iffalse meta-comment
%
% File: hologo.dtx
% Version: 2016/05/12 v1.11
% Info: A logo collection with bookmark support
%
% Copyright (C) 2010-2012 by
%    Heiko Oberdiek <heiko.oberdiek at googlemail.com>
%
% This work may be distributed and/or modified under the
% conditions of the LaTeX Project Public License, either
% version 1.3c of this license or (at your option) any later
% version. This version of this license is in
%    http://www.latex-project.org/lppl/lppl-1-3c.txt
% and the latest version of this license is in
%    http://www.latex-project.org/lppl.txt
% and version 1.3 or later is part of all distributions of
% LaTeX version 2005/12/01 or later.
%
% This work has the LPPL maintenance status "maintained".
%
% This Current Maintainer of this work is Heiko Oberdiek.
%
% The Base Interpreter refers to any `TeX-Format',
% because some files are installed in TDS:tex/generic//.
%
% This work consists of the main source file hologo.dtx
% and the derived files
%    hologo.sty, hologo.pdf, hologo.ins, hologo.drv, hologo-example.tex,
%    hologo-test1.tex, hologo-test-spacefactor.tex,
%    hologo-test-list.tex.
%
% Distribution:
%    CTAN:macros/latex/contrib/oberdiek/hologo.dtx
%    CTAN:macros/latex/contrib/oberdiek/hologo.pdf
%
% Unpacking:
%    (a) If hologo.ins is present:
%           tex hologo.ins
%    (b) Without hologo.ins:
%           tex hologo.dtx
%    (c) If you insist on using LaTeX
%           latex \let\install=y% \iffalse meta-comment
%
% File: hologo.dtx
% Version: 2016/05/12 v1.11
% Info: A logo collection with bookmark support
%
% Copyright (C) 2010-2012 by
%    Heiko Oberdiek <heiko.oberdiek at googlemail.com>
%
% This work may be distributed and/or modified under the
% conditions of the LaTeX Project Public License, either
% version 1.3c of this license or (at your option) any later
% version. This version of this license is in
%    http://www.latex-project.org/lppl/lppl-1-3c.txt
% and the latest version of this license is in
%    http://www.latex-project.org/lppl.txt
% and version 1.3 or later is part of all distributions of
% LaTeX version 2005/12/01 or later.
%
% This work has the LPPL maintenance status "maintained".
%
% This Current Maintainer of this work is Heiko Oberdiek.
%
% The Base Interpreter refers to any `TeX-Format',
% because some files are installed in TDS:tex/generic//.
%
% This work consists of the main source file hologo.dtx
% and the derived files
%    hologo.sty, hologo.pdf, hologo.ins, hologo.drv, hologo-example.tex,
%    hologo-test1.tex, hologo-test-spacefactor.tex,
%    hologo-test-list.tex.
%
% Distribution:
%    CTAN:macros/latex/contrib/oberdiek/hologo.dtx
%    CTAN:macros/latex/contrib/oberdiek/hologo.pdf
%
% Unpacking:
%    (a) If hologo.ins is present:
%           tex hologo.ins
%    (b) Without hologo.ins:
%           tex hologo.dtx
%    (c) If you insist on using LaTeX
%           latex \let\install=y\input{hologo.dtx}
%        (quote the arguments according to the demands of your shell)
%
% Documentation:
%    (a) If hologo.drv is present:
%           latex hologo.drv
%    (b) Without hologo.drv:
%           latex hologo.dtx; ...
%    The class ltxdoc loads the configuration file ltxdoc.cfg
%    if available. Here you can specify further options, e.g.
%    use A4 as paper format:
%       \PassOptionsToClass{a4paper}{article}
%
%    Programm calls to get the documentation (example):
%       pdflatex hologo.dtx
%       makeindex -s gind.ist hologo.idx
%       pdflatex hologo.dtx
%       makeindex -s gind.ist hologo.idx
%       pdflatex hologo.dtx
%
% Installation:
%    TDS:tex/generic/oberdiek/hologo.sty
%    TDS:doc/latex/oberdiek/hologo.pdf
%    TDS:doc/latex/oberdiek/example/hologo-example.tex
%    TDS:doc/latex/oberdiek/test/hologo-test1.tex
%    TDS:doc/latex/oberdiek/test/hologo-test-spacefactor.tex
%    TDS:doc/latex/oberdiek/test/hologo-test-list.tex
%    TDS:source/latex/oberdiek/hologo.dtx
%
%<*ignore>
\begingroup
  \catcode123=1 %
  \catcode125=2 %
  \def\x{LaTeX2e}%
\expandafter\endgroup
\ifcase 0\ifx\install y1\fi\expandafter
         \ifx\csname processbatchFile\endcsname\relax\else1\fi
         \ifx\fmtname\x\else 1\fi\relax
\else\csname fi\endcsname
%</ignore>
%<*install>
\input docstrip.tex
\Msg{************************************************************************}
\Msg{* Installation}
\Msg{* Package: hologo 2016/05/12 v1.11 A logo collection with bookmark support (HO)}
\Msg{************************************************************************}

\keepsilent
\askforoverwritefalse

\let\MetaPrefix\relax
\preamble

This is a generated file.

Project: hologo
Version: 2016/05/12 v1.11

Copyright (C) 2010-2012 by
   Heiko Oberdiek <heiko.oberdiek at googlemail.com>

This work may be distributed and/or modified under the
conditions of the LaTeX Project Public License, either
version 1.3c of this license or (at your option) any later
version. This version of this license is in
   http://www.latex-project.org/lppl/lppl-1-3c.txt
and the latest version of this license is in
   http://www.latex-project.org/lppl.txt
and version 1.3 or later is part of all distributions of
LaTeX version 2005/12/01 or later.

This work has the LPPL maintenance status "maintained".

This Current Maintainer of this work is Heiko Oberdiek.

The Base Interpreter refers to any `TeX-Format',
because some files are installed in TDS:tex/generic//.

This work consists of the main source file hologo.dtx
and the derived files
   hologo.sty, hologo.pdf, hologo.ins, hologo.drv, hologo-example.tex,
   hologo-test1.tex, hologo-test-spacefactor.tex,
   hologo-test-list.tex.

\endpreamble
\let\MetaPrefix\DoubleperCent

\generate{%
  \file{hologo.ins}{\from{hologo.dtx}{install}}%
  \file{hologo.drv}{\from{hologo.dtx}{driver}}%
  \usedir{tex/generic/oberdiek}%
  \file{hologo.sty}{\from{hologo.dtx}{package}}%
  \usedir{doc/latex/oberdiek/example}%
  \file{hologo-example.tex}{\from{hologo.dtx}{example}}%
  \usedir{doc/latex/oberdiek/test}%
  \file{hologo-test1.tex}{\from{hologo.dtx}{test1}}%
  \file{hologo-test-spacefactor.tex}{\from{hologo.dtx}{test-spacefactor}}%
  \file{hologo-test-list.tex}{\from{hologo.dtx}{test-list}}%
  \nopreamble
  \nopostamble
  \usedir{source/latex/oberdiek/catalogue}%
  \file{hologo.xml}{\from{hologo.dtx}{catalogue}}%
}

\catcode32=13\relax% active space
\let =\space%
\Msg{************************************************************************}
\Msg{*}
\Msg{* To finish the installation you have to move the following}
\Msg{* file into a directory searched by TeX:}
\Msg{*}
\Msg{*     hologo.sty}
\Msg{*}
\Msg{* To produce the documentation run the file `hologo.drv'}
\Msg{* through LaTeX.}
\Msg{*}
\Msg{* Happy TeXing!}
\Msg{*}
\Msg{************************************************************************}

\endbatchfile
%</install>
%<*ignore>
\fi
%</ignore>
%<*driver>
\NeedsTeXFormat{LaTeX2e}
\ProvidesFile{hologo.drv}%
  [2016/05/12 v1.11 A logo collection with bookmark support (HO)]%
\documentclass{ltxdoc}
\usepackage{holtxdoc}[2011/11/22]
\usepackage{hologo}[2016/05/12]
\usepackage{longtable}
\usepackage{array}
\usepackage{paralist}
%\usepackage[T1]{fontenc}
%\usepackage{lmodern}
\begin{document}
  \DocInput{hologo.dtx}%
\end{document}
%</driver>
% \fi
%
%
% \CharacterTable
%  {Upper-case    \A\B\C\D\E\F\G\H\I\J\K\L\M\N\O\P\Q\R\S\T\U\V\W\X\Y\Z
%   Lower-case    \a\b\c\d\e\f\g\h\i\j\k\l\m\n\o\p\q\r\s\t\u\v\w\x\y\z
%   Digits        \0\1\2\3\4\5\6\7\8\9
%   Exclamation   \!     Double quote  \"     Hash (number) \#
%   Dollar        \$     Percent       \%     Ampersand     \&
%   Acute accent  \'     Left paren    \(     Right paren   \)
%   Asterisk      \*     Plus          \+     Comma         \,
%   Minus         \-     Point         \.     Solidus       \/
%   Colon         \:     Semicolon     \;     Less than     \<
%   Equals        \=     Greater than  \>     Question mark \?
%   Commercial at \@     Left bracket  \[     Backslash     \\
%   Right bracket \]     Circumflex    \^     Underscore    \_
%   Grave accent  \`     Left brace    \{     Vertical bar  \|
%   Right brace   \}     Tilde         \~}
%
% \GetFileInfo{hologo.drv}
%
% \title{The \xpackage{hologo} package}
% \date{2016/05/12 v1.11}
% \author{Heiko Oberdiek\\\xemail{heiko.oberdiek at googlemail.com}}
%
% \maketitle
%
% \begin{abstract}
% This package starts a collection of logos with support for bookmarks
% strings.
% \end{abstract}
%
% \tableofcontents
%
% \section{Documentation}
%
% \subsection{Logo macros}
%
% \begin{declcs}{hologo} \M{name}
% \end{declcs}
% Macro \cs{hologo} sets the logo with name \meta{name}.
% The following table shows the supported names.
%
% \begingroup
%   \def\hologoEntry#1#2#3{^^A
%     #1&#2&\hologoLogoSetup{#1}{variant=#2}\hologo{#1}&#3\tabularnewline
%   }
%   \begin{longtable}{>{\ttfamily}l>{\ttfamily}lll}
%     \rmfamily\bfseries{name} & \rmfamily\bfseries variant
%     & \bfseries logo & \bfseries since\\
%     \hline
%     \endhead
%     \hologoList
%   \end{longtable}
% \endgroup
%
% \begin{declcs}{Hologo} \M{name}
% \end{declcs}
% Macro \cs{Hologo} starts the logo \meta{name} with an uppercase
% letter. As an exception small greek letters are not converted
% to uppercase. Examples, see \hologo{eTeX} and \hologo{ExTeX}.
%
% \subsection{Setup macros}
%
% The package does not support package options, but the following
% setup macros can be used to set options.
%
% \begin{declcs}{hologoSetup} \M{key value list}
% \end{declcs}
% Macro \cs{hologoSetup} sets global options.
%
% \begin{declcs}{hologoLogoSetup} \M{logo} \M{key value list}
% \end{declcs}
% Some options can also be used to configure a logo.
% These settings take precedence over global option settings.
%
% \subsection{Options}\label{sec:options}
%
% There are boolean and string options:
% \begin{description}
% \item[Boolean option:]
% It takes |true| or |false|
% as value. If the value is omitted, then |true| is used.
% \item[String option:]
% A value must be given as string. (But the string might be empty.)
% \end{description}
% The following options can be used both in \cs{hologoSetup}
% and \cs{hologoLogoSetup}:
% \begin{description}
% \def\entry#1{\item[\xoption{#1}:]}
% \entry{break}
%   enables or disables line breaks inside the logo. This setting is
%   refined by options \xoption{hyphenbreak}, \xoption{spacebreak}
%   or \xoption{discretionarybreak}.
%   Default is |false|.
% \entry{hyphenbreak}
%   enables or disables the line break right after the hyphen character.
% \entry{spacebreak}
%   enables or disables line breaks at space characters.
% \entry{discretionarybreak}
%   enables or disables line breaks at hyphenation points
%   (inserted by \cs{-}).
% \end{description}
% Macro \cs{hologoLogoSetup} also knows:
% \begin{description}
% \item[\xoption{variant}:]
%   This is a string option. It specifies a variant of a logo that
%   must exist. An empty string selects the package default variant.
% \end{description}
% Example:
% \begin{quote}
%   |\hologoSetup{break=false}|\\
%   |\hologoLogoSetup{plainTeX}{variant=hyphen,hyphenbreak}|\\
%   Then ``plain-\TeX'' contains one break point after the hyphen.
% \end{quote}
%
% \subsection{Driver options}
%
% Sometimes graphical operations are needed to construct some
% glyphs (e.g.\ \hologo{XeTeX}). If package \xpackage{graphics}
% or package \xpackage{pgf} are found, then the macros are taken
% from there. Otherwise the packge defines its own operations
% and therefore needs the driver information. Many drivers are
% detected automatically (\hologo{pdfTeX}/\hologo{LuaTeX}
% in PDF mode, \hologo{XeTeX}, \hologo{VTeX}). These have precedence
% over a driver option. The driver can be given as package option
% or using \cs{hologoDriverSetup}.
% The following list contains the recognized driver options:
% \begin{itemize}
% \item \xoption{pdftex}, \xoption{luatex}
% \item \xoption{dvipdfm}, \xoption{dvipdfmx}
% \item \xoption{dvips}, \xoption{dvipsone}, \xoption{xdvi}
% \item \xoption{xetex}
% \item \xoption{vtex}
% \end{itemize}
% The left driver of a line is the driver name that is used internally.
% The following names are aliases for drivers that use the
% same method. Therefore the entry in the \xext{log} file for
% the used driver prints the internally used driver name.
% \begin{description}
% \item[\xoption{driverfallback}:]
%   This option expects a driver that is used,
%   if the driver could not be detected automatically.
% \end{description}
%
% \begin{declcs}{hologoDriverSetup} \M{driver option}
% \end{declcs}
% The driver can also be configured after package loading
% using \cs{hologoDriverSetup}, also the way for \hologo{plainTeX}
% to setup the driver.
%
% \subsection{Font setup}
%
% Some logos require a special font, but should also be usable by
% \hologo{plainTeX}. Therefore the package provides some ways
% to influence the font settings. The options below
% take font settings as values. Both font commands
% such as \cs{sffamily} and macros that take one argument
% like \cs{textsf} can be used.
%
% \begin{declcs}{hologoFontSetup} \M{key value list}
% \end{declcs}
% Macro \cs{hologoFontSetup} sets the fonts for all logos.
% Supported keys:
% \begin{description}
% \def\entry#1{\item[\xoption{#1}:]}
% \entry{general}
%   This font is used for all logos. The default is empty.
%   That means no special font is used.
% \entry{bibsf}
%   This font is used for
%   {\hologoLogoSetup{BibTeX}{variant=sf}\hologo{BibTeX}}
%   with variant \xoption{sf}.
% \entry{rm}
%   This font is a serif font. It is used for \hologo{ExTeX}.
% \entry{sc}
%   This font specifies a small caps font. It is used for
%   {\hologoLogoSetup{BibTeX}{variant=sc}\hologo{BibTeX}}
%   with variant \xoption{sc}.
% \entry{sf}
%   This font specifies a sans serif font. The default
%   is \cs{sffamily}, then \cs{sf} is tried. Otherwise
%   a warning is given. It is used by \hologo{KOMAScript}.
% \entry{sy}
%   This is the font for math symbols (e.g. cmsy).
%   It is used by \hologo{AmS}, \hologo{NTS}, \hologo{ExTeX}.
% \entry{logo}
%   \hologo{METAFONT} and \hologo{METAPOST} are using that font.
%   In \hologo{LaTeX} \cs{logofamily} is used and
%   the definitions of package \xpackage{mflogo} are used
%   if the package is not loaded.
%   Otherwise the \cs{tenlogo} is used and defined
%   if it does not already exists.
% \end{description}
%
% \begin{declcs}{hologoLogoFontSetup} \M{logo} \M{key value list}
% \end{declcs}
% Fonts can also be set for a logo or logo component separately,
% see the following list.
% The keys are the same as for \cs{hologoFontSetup}.
%
% \begin{longtable}{>{\ttfamily}l>{\sffamily}ll}
%   \meta{logo} & keys & result\\
%   \hline
%   \endhead
%   BibTeX & bibsf & {\hologoLogoSetup{BibTeX}{variant=sf}\hologo{BibTeX}}\\[.5ex]
%   BibTeX & sc & {\hologoLogoSetup{BibTeX}{variant=sc}\hologo{BibTeX}}\\[.5ex]
%   ExTeX & rm & \hologo{ExTeX}\\
%   SliTeX & rm & \hologo{SliTeX}\\[.5ex]
%   AmS & sy & \hologo{AmS}\\
%   ExTeX & sy & \hologo{ExTeX}\\
%   NTS & sy & \hologo{NTS}\\[.5ex]
%   KOMAScript & sf & \hologo{KOMAScript}\\[.5ex]
%   METAFONT & logo & \hologo{METAFONT}\\
%   METAPOST & logo & \hologo{METAPOST}\\[.5ex]
%   SliTeX & sc \hologo{SliTeX}
% \end{longtable}
%
% \subsubsection{Font order}
%
% For all logos the font \xoption{general} is applied first.
% Example:
%\begin{quote}
%|\hologoFontSetup{general=\color{red}}|
%\end{quote}
% will print red logos.
% Then if the font uses a special font \xoption{sf}, for example,
% the font is applied that is setup by \cs{hologoLogoFontSetup}.
% If this font is not setup, then the common font setup
% by \cs{hologoFontSetup} is used. Otherwise a warning is given,
% that there is no font configured.
%
% \subsection{Additional user macros}
%
% Usually a variant of a logo is configured by using
% \cs{hologoLogoSetup}, because it is bad style to mix
% different variants of the same logo in the same text.
% There the following macros are a convenience for testing.
%
% \begin{declcs}{hologoVariant} \M{name} \M{variant}\\
%   \cs{HologoVariant} \M{name} \M{variant}
% \end{declcs}
% Logo \meta{name} is set using \meta{variant} that specifies
% explicitely which variant of the macro is used. If the argument
% is empty, then the default form of the logo is used
% (configurable by \cs{hologoLogoSetup}).
%
% \cs{HologoVariant} is used if the logo is set in a context
% that needs an uppercase first letter (beginning of a sentence, \dots).
%
% \begin{declcs}{hologoList}\\
%   \cs{hologoEntry} \M{logo} \M{variant} \M{since}
% \end{declcs}
% Macro \cs{hologoList} contains all logos that are provided
% by the package including variants. The list consists of calls
% of \cs{hologoEntry} with three arguments starting with the
% logo name \meta{logo} and its variant \meta{variant}. An empty
% variant means the current default. Argument \meta{since} specifies
% with version of the package \xpackage{hologo} is needed to get
% the logo. If the logo is fixed, then the date gets updated.
% Therefore the date \meta{since} is not exactly the date of
% the first introduction, but rather the date of the latest fix.
%
% Before \cs{hologoList} can be used, macro \cs{hologoEntry} needs
% a definition. The example file in section \ref{sec:example}
% shows applications of \cs{hologoList}.
%
% \subsection{Supported contexts}
%
% Macros \cs{hologo} and friends support special contexts:
% \begin{itemize}
% \item \hologo{LaTeX}'s protection mechanism.
% \item Bookmarks of package \xpackage{hyperref}.
% \item Package \xpackage{tex4ht}.
% \item The macros can be used inside \cs{csname} constructs,
%   if \cs{ifincsname} is available (\hologo{pdfTeX}, \hologo{XeTeX},
%   \hologo{LuaTeX}).
% \end{itemize}
%
% \subsection{Example}
% \label{sec:example}
%
% The following example prints the logos in different fonts.
%    \begin{macrocode}
%<*example>
%<<verbatim
\NeedsTeXFormat{LaTeX2e}
\documentclass[a4paper]{article}
\usepackage[
  hmargin=20mm,
  vmargin=20mm,
]{geometry}
\pagestyle{empty}
\usepackage{hologo}[2016/05/12]
\usepackage{longtable}
\usepackage{array}
\setlength{\extrarowheight}{2pt}
\usepackage[T1]{fontenc}
\usepackage{lmodern}
\usepackage{pdflscape}
\usepackage[
  pdfencoding=auto,
]{hyperref}
\hypersetup{
  pdfauthor={Heiko Oberdiek},
  pdftitle={Example for package `hologo'},
  pdfsubject={Logos with fonts lmr, lmss, qtm, qpl, qhv},
}
\usepackage{bookmark}

% Print the logo list on the console

\begingroup
  \typeout{}%
  \typeout{*** Begin of logo list ***}%
  \newcommand*{\hologoEntry}[3]{%
    \typeout{#1 \ifx\\#2\\\else(#2) \fi[#3]}%
  }%
  \hologoList
  \typeout{*** End of logo list ***}%
  \typeout{}%
\endgroup

\begin{document}
\begin{landscape}

  \section{Example file for package `hologo'}

  % Table for font names

  \begin{longtable}{>{\bfseries}ll}
    \textbf{font} & \textbf{Font name}\\
    \hline
    lmr & Latin Modern Roman\\
    lmss & Latin Modern Sans\\
    qtm & \TeX\ Gyre Termes\\
    qhv & \TeX\ Gyre Heros\\
    qpl & \TeX\ Gyre Pagella\\
  \end{longtable}

  % Logo list with logos in different fonts

  \begingroup
    \newcommand*{\SetVariant}[2]{%
      \ifx\\#2\\%
      \else
        \hologoLogoSetup{#1}{variant=#2}%
      \fi
    }%
    \newcommand*{\hologoEntry}[3]{%
      \SetVariant{#1}{#2}%
      \raisebox{1em}[0pt][0pt]{\hypertarget{#1@#2}{}}%
      \bookmark[%
        dest={#1@#2},%
      ]{%
        #1\ifx\\#2\\\else\space(#2)\fi: \Hologo{#1}, \hologo{#1} %
        [Unicode]%
      }%
      \hypersetup{unicode=false}%
      \bookmark[%
        dest={#1@#2},%
      ]{%
        #1\ifx\\#2\\\else\space(#2)\fi: \Hologo{#1}, \hologo{#1} %
        [PDFDocEncoding]%
      }%
      \texttt{#1}%
      &%
      \texttt{#2}%
      &%
      \Hologo{#1}%
      &%
      \SetVariant{#1}{#2}%
      \hologo{#1}%
      &%
      \SetVariant{#1}{#2}%
      \fontfamily{qtm}\selectfont
      \hologo{#1}%
      &%
      \SetVariant{#1}{#2}%
      \fontfamily{qpl}\selectfont
      \hologo{#1}%
      &%
      \SetVariant{#1}{#2}%
      \textsf{\hologo{#1}}%
      &%
      \SetVariant{#1}{#2}%
      \fontfamily{qhv}\selectfont
      \hologo{#1}%
      \tabularnewline
    }%
    \begin{longtable}{llllllll}%
      \textbf{\textit{logo}} & \textbf{\textit{variant}} &
      \texttt{\string\Hologo} &
      \textbf{lmr} & \textbf{qtm} & \textbf{qpl} &
      \textbf{lmss} & \textbf{qhv}
      \tabularnewline
      \hline
      \endhead
      \hologoList
    \end{longtable}%
  \endgroup

\end{landscape}
\end{document}
%verbatim
%</example>
%    \end{macrocode}
%
% \StopEventually{
% }
%
% \section{Implementation}
%    \begin{macrocode}
%<*package>
%    \end{macrocode}
%    Reload check, especially if the package is not used with \LaTeX.
%    \begin{macrocode}
\begingroup\catcode61\catcode48\catcode32=10\relax%
  \catcode13=5 % ^^M
  \endlinechar=13 %
  \catcode35=6 % #
  \catcode39=12 % '
  \catcode44=12 % ,
  \catcode45=12 % -
  \catcode46=12 % .
  \catcode58=12 % :
  \catcode64=11 % @
  \catcode123=1 % {
  \catcode125=2 % }
  \expandafter\let\expandafter\x\csname ver@hologo.sty\endcsname
  \ifx\x\relax % plain-TeX, first loading
  \else
    \def\empty{}%
    \ifx\x\empty % LaTeX, first loading,
      % variable is initialized, but \ProvidesPackage not yet seen
    \else
      \expandafter\ifx\csname PackageInfo\endcsname\relax
        \def\x#1#2{%
          \immediate\write-1{Package #1 Info: #2.}%
        }%
      \else
        \def\x#1#2{\PackageInfo{#1}{#2, stopped}}%
      \fi
      \x{hologo}{The package is already loaded}%
      \aftergroup\endinput
    \fi
  \fi
\endgroup%
%    \end{macrocode}
%    Package identification:
%    \begin{macrocode}
\begingroup\catcode61\catcode48\catcode32=10\relax%
  \catcode13=5 % ^^M
  \endlinechar=13 %
  \catcode35=6 % #
  \catcode39=12 % '
  \catcode40=12 % (
  \catcode41=12 % )
  \catcode44=12 % ,
  \catcode45=12 % -
  \catcode46=12 % .
  \catcode47=12 % /
  \catcode58=12 % :
  \catcode64=11 % @
  \catcode91=12 % [
  \catcode93=12 % ]
  \catcode123=1 % {
  \catcode125=2 % }
  \expandafter\ifx\csname ProvidesPackage\endcsname\relax
    \def\x#1#2#3[#4]{\endgroup
      \immediate\write-1{Package: #3 #4}%
      \xdef#1{#4}%
    }%
  \else
    \def\x#1#2[#3]{\endgroup
      #2[{#3}]%
      \ifx#1\@undefined
        \xdef#1{#3}%
      \fi
      \ifx#1\relax
        \xdef#1{#3}%
      \fi
    }%
  \fi
\expandafter\x\csname ver@hologo.sty\endcsname
\ProvidesPackage{hologo}%
  [2016/05/12 v1.11 A logo collection with bookmark support (HO)]%
%    \end{macrocode}
%
%    \begin{macrocode}
\begingroup\catcode61\catcode48\catcode32=10\relax%
  \catcode13=5 % ^^M
  \endlinechar=13 %
  \catcode123=1 % {
  \catcode125=2 % }
  \catcode64=11 % @
  \def\x{\endgroup
    \expandafter\edef\csname HOLOGO@AtEnd\endcsname{%
      \endlinechar=\the\endlinechar\relax
      \catcode13=\the\catcode13\relax
      \catcode32=\the\catcode32\relax
      \catcode35=\the\catcode35\relax
      \catcode61=\the\catcode61\relax
      \catcode64=\the\catcode64\relax
      \catcode123=\the\catcode123\relax
      \catcode125=\the\catcode125\relax
    }%
  }%
\x\catcode61\catcode48\catcode32=10\relax%
\catcode13=5 % ^^M
\endlinechar=13 %
\catcode35=6 % #
\catcode64=11 % @
\catcode123=1 % {
\catcode125=2 % }
\def\TMP@EnsureCode#1#2{%
  \edef\HOLOGO@AtEnd{%
    \HOLOGO@AtEnd
    \catcode#1=\the\catcode#1\relax
  }%
  \catcode#1=#2\relax
}
\TMP@EnsureCode{10}{12}% ^^J
\TMP@EnsureCode{33}{12}% !
\TMP@EnsureCode{34}{12}% "
\TMP@EnsureCode{36}{3}% $
\TMP@EnsureCode{38}{4}% &
\TMP@EnsureCode{39}{12}% '
\TMP@EnsureCode{40}{12}% (
\TMP@EnsureCode{41}{12}% )
\TMP@EnsureCode{42}{12}% *
\TMP@EnsureCode{43}{12}% +
\TMP@EnsureCode{44}{12}% ,
\TMP@EnsureCode{45}{12}% -
\TMP@EnsureCode{46}{12}% .
\TMP@EnsureCode{47}{12}% /
\TMP@EnsureCode{58}{12}% :
\TMP@EnsureCode{59}{12}% ;
\TMP@EnsureCode{60}{12}% <
\TMP@EnsureCode{62}{12}% >
\TMP@EnsureCode{63}{12}% ?
\TMP@EnsureCode{91}{12}% [
\TMP@EnsureCode{93}{12}% ]
\TMP@EnsureCode{94}{7}% ^ (superscript)
\TMP@EnsureCode{95}{8}% _ (subscript)
\TMP@EnsureCode{96}{12}% `
\TMP@EnsureCode{124}{12}% |
\edef\HOLOGO@AtEnd{%
  \HOLOGO@AtEnd
  \escapechar\the\escapechar\relax
  \noexpand\endinput
}
\escapechar=92 %
%    \end{macrocode}
%
% \subsection{Logo list}
%
%    \begin{macro}{\hologoList}
%    \begin{macrocode}
\def\hologoList{%
  \hologoEntry{(La)TeX}{}{2011/10/01}%
  \hologoEntry{AmSLaTeX}{}{2010/04/16}%
  \hologoEntry{AmSTeX}{}{2010/04/16}%
  \hologoEntry{biber}{}{2011/10/01}%
  \hologoEntry{BibTeX}{}{2011/10/01}%
  \hologoEntry{BibTeX}{sf}{2011/10/01}%
  \hologoEntry{BibTeX}{sc}{2011/10/01}%
  \hologoEntry{BibTeX8}{}{2011/11/22}%
  \hologoEntry{ConTeXt}{}{2011/03/25}%
  \hologoEntry{ConTeXt}{narrow}{2011/03/25}%
  \hologoEntry{ConTeXt}{simple}{2011/03/25}%
  \hologoEntry{emTeX}{}{2010/04/26}%
  \hologoEntry{eTeX}{}{2010/04/08}%
  \hologoEntry{ExTeX}{}{2011/10/01}%
  \hologoEntry{HanTheThanh}{}{2011/11/29}%
  \hologoEntry{iniTeX}{}{2011/10/01}%
  \hologoEntry{KOMAScript}{}{2011/10/01}%
  \hologoEntry{La}{}{2010/05/08}%
  \hologoEntry{LaTeX}{}{2010/04/08}%
  \hologoEntry{LaTeX2e}{}{2010/04/08}%
  \hologoEntry{LaTeX3}{}{2010/04/24}%
  \hologoEntry{LaTeXe}{}{2010/04/08}%
  \hologoEntry{LaTeXML}{}{2011/11/22}%
  \hologoEntry{LaTeXTeX}{}{2011/10/01}%
  \hologoEntry{LuaLaTeX}{}{2010/04/08}%
  \hologoEntry{LuaTeX}{}{2010/04/08}%
  \hologoEntry{LyX}{}{2011/10/01}%
  \hologoEntry{METAFONT}{}{2011/10/01}%
  \hologoEntry{MetaFun}{}{2011/10/01}%
  \hologoEntry{METAPOST}{}{2011/10/01}%
  \hologoEntry{MetaPost}{}{2011/10/01}%
  \hologoEntry{MiKTeX}{}{2011/10/01}%
  \hologoEntry{NTS}{}{2011/10/01}%
  \hologoEntry{OzMF}{}{2011/10/01}%
  \hologoEntry{OzMP}{}{2011/10/01}%
  \hologoEntry{OzTeX}{}{2011/10/01}%
  \hologoEntry{OzTtH}{}{2011/10/01}%
  \hologoEntry{PCTeX}{}{2011/10/01}%
  \hologoEntry{pdfTeX}{}{2011/10/01}%
  \hologoEntry{pdfLaTeX}{}{2011/10/01}%
  \hologoEntry{PiC}{}{2011/10/01}%
  \hologoEntry{PiCTeX}{}{2011/10/01}%
  \hologoEntry{plainTeX}{}{2010/04/08}%
  \hologoEntry{plainTeX}{space}{2010/04/16}%
  \hologoEntry{plainTeX}{hyphen}{2010/04/16}%
  \hologoEntry{plainTeX}{runtogether}{2010/04/16}%
  \hologoEntry{SageTeX}{}{2011/11/22}%
  \hologoEntry{SLiTeX}{}{2011/10/01}%
  \hologoEntry{SLiTeX}{lift}{2011/10/01}%
  \hologoEntry{SLiTeX}{narrow}{2011/10/01}%
  \hologoEntry{SLiTeX}{simple}{2011/10/01}%
  \hologoEntry{SliTeX}{}{2011/10/01}%
  \hologoEntry{SliTeX}{narrow}{2011/10/01}%
  \hologoEntry{SliTeX}{simple}{2011/10/01}%
  \hologoEntry{SliTeX}{lift}{2011/10/01}%
  \hologoEntry{teTeX}{}{2011/10/01}%
  \hologoEntry{TeX}{}{2010/04/08}%
  \hologoEntry{TeX4ht}{}{2011/11/22}%
  \hologoEntry{TTH}{}{2011/11/22}%
  \hologoEntry{virTeX}{}{2011/10/01}%
  \hologoEntry{VTeX}{}{2010/04/24}%
  \hologoEntry{Xe}{}{2010/04/08}%
  \hologoEntry{XeLaTeX}{}{2010/04/08}%
  \hologoEntry{XeTeX}{}{2010/04/08}%
}
%    \end{macrocode}
%    \end{macro}
%
% \subsection{Load resources}
%
%    \begin{macrocode}
\begingroup\expandafter\expandafter\expandafter\endgroup
\expandafter\ifx\csname RequirePackage\endcsname\relax
  \def\TMP@RequirePackage#1[#2]{%
    \begingroup\expandafter\expandafter\expandafter\endgroup
    \expandafter\ifx\csname ver@#1.sty\endcsname\relax
      \input #1.sty\relax
    \fi
  }%
  \TMP@RequirePackage{ltxcmds}[2011/02/04]%
  \TMP@RequirePackage{infwarerr}[2010/04/08]%
  \TMP@RequirePackage{kvsetkeys}[2010/03/01]%
  \TMP@RequirePackage{kvdefinekeys}[2010/03/01]%
  \TMP@RequirePackage{pdftexcmds}[2010/04/01]%
  \TMP@RequirePackage{ifpdf}[2010/01/28]%
  \TMP@RequirePackage{ifluatex}[2010/03/01]%
  \ltx@IfUndefined{newif}{%
    \expandafter\let\csname newif\endcsname\ltx@newif
  }{}%
  \TMP@RequirePackage{ifxetex}[2009/01/23]%
  \TMP@RequirePackage{ifvtex}[2010/03/01]%
\else
  \RequirePackage{ltxcmds}[2011/02/04]%
  \RequirePackage{infwarerr}[2010/04/08]%
  \RequirePackage{kvsetkeys}[2010/03/01]%
  \RequirePackage{kvdefinekeys}[2010/03/01]%
  \RequirePackage{pdftexcmds}[2010/04/01]%
  \RequirePackage{ifpdf}[2010/01/28]%
  \RequirePackage{ifluatex}[2010/03/01]%
  \RequirePackage{ifxetex}[2009/01/23]%
  \RequirePackage{ifvtex}[2010/03/01]%
\fi
%    \end{macrocode}
%
%    \begin{macro}{\HOLOGO@IfDefined}
%    \begin{macrocode}
\def\HOLOGO@IfExists#1{%
  \ifx\@undefined#1%
    \expandafter\ltx@secondoftwo
  \else
    \ifx\relax#1%
      \expandafter\ltx@secondoftwo
    \else
      \expandafter\expandafter\expandafter\ltx@firstoftwo
    \fi
  \fi
}
%    \end{macrocode}
%    \end{macro}
%
% \subsection{Setup macros}
%
%    \begin{macro}{\hologoSetup}
%    \begin{macrocode}
\def\hologoSetup{%
  \let\HOLOGO@name\relax
  \HOLOGO@Setup
}
%    \end{macrocode}
%    \end{macro}
%
%    \begin{macro}{\hologoLogoSetup}
%    \begin{macrocode}
\def\hologoLogoSetup#1{%
  \edef\HOLOGO@name{#1}%
  \ltx@IfUndefined{HoLogo@\HOLOGO@name}{%
    \@PackageError{hologo}{%
      Unknown logo `\HOLOGO@name'%
    }\@ehc
    \ltx@gobble
  }{%
    \HOLOGO@Setup
  }%
}
%    \end{macrocode}
%    \end{macro}
%
%    \begin{macro}{\HOLOGO@Setup}
%    \begin{macrocode}
\def\HOLOGO@Setup{%
  \kvsetkeys{HoLogo}%
}
%    \end{macrocode}
%    \end{macro}
%
% \subsection{Options}
%
%    \begin{macro}{\HOLOGO@DeclareBoolOption}
%    \begin{macrocode}
\def\HOLOGO@DeclareBoolOption#1{%
  \expandafter\chardef\csname HOLOGOOPT@#1\endcsname\ltx@zero
  \kv@define@key{HoLogo}{#1}[true]{%
    \def\HOLOGO@temp{##1}%
    \ifx\HOLOGO@temp\HOLOGO@true
      \ifx\HOLOGO@name\relax
        \expandafter\chardef\csname HOLOGOOPT@#1\endcsname=\ltx@one
      \else
        \expandafter\chardef\csname
        HoLogoOpt@#1@\HOLOGO@name\endcsname\ltx@one
      \fi
      \HOLOGO@SetBreakAll{#1}%
    \else
      \ifx\HOLOGO@temp\HOLOGO@false
        \ifx\HOLOGO@name\relax
          \expandafter\chardef\csname HOLOGOOPT@#1\endcsname=\ltx@zero
        \else
          \expandafter\chardef\csname
          HoLogoOpt@#1@\HOLOGO@name\endcsname=\ltx@zero
        \fi
        \HOLOGO@SetBreakAll{#1}%
      \else
        \@PackageError{hologo}{%
          Unknown value `##1' for boolean option `#1'.\MessageBreak
          Known values are `true' and `false'%
        }\@ehc
      \fi
    \fi
  }%
}
%    \end{macrocode}
%    \end{macro}
%
%    \begin{macro}{\HOLOGO@SetBreakAll}
%    \begin{macrocode}
\def\HOLOGO@SetBreakAll#1{%
  \def\HOLOGO@temp{#1}%
  \ifx\HOLOGO@temp\HOLOGO@break
    \ifx\HOLOGO@name\relax
      \chardef\HOLOGOOPT@hyphenbreak=\HOLOGOOPT@break
      \chardef\HOLOGOOPT@spacebreak=\HOLOGOOPT@break
      \chardef\HOLOGOOPT@discretionarybreak=\HOLOGOOPT@break
    \else
      \expandafter\chardef
         \csname HoLogoOpt@hyphenbreak@\HOLOGO@name\endcsname=%
         \csname HoLogoOpt@break@\HOLOGO@name\endcsname
      \expandafter\chardef
         \csname HoLogoOpt@spacebreak@\HOLOGO@name\endcsname=%
         \csname HoLogoOpt@break@\HOLOGO@name\endcsname
      \expandafter\chardef
         \csname HoLogoOpt@discretionarybreak@\HOLOGO@name
             \endcsname=%
         \csname HoLogoOpt@break@\HOLOGO@name\endcsname
    \fi
  \fi
}
%    \end{macrocode}
%    \end{macro}
%
%    \begin{macro}{\HOLOGO@true}
%    \begin{macrocode}
\def\HOLOGO@true{true}
%    \end{macrocode}
%    \end{macro}
%    \begin{macro}{\HOLOGO@false}
%    \begin{macrocode}
\def\HOLOGO@false{false}
%    \end{macrocode}
%    \end{macro}
%    \begin{macro}{\HOLOGO@break}
%    \begin{macrocode}
\def\HOLOGO@break{break}
%    \end{macrocode}
%    \end{macro}
%
%    \begin{macrocode}
\HOLOGO@DeclareBoolOption{break}
\HOLOGO@DeclareBoolOption{hyphenbreak}
\HOLOGO@DeclareBoolOption{spacebreak}
\HOLOGO@DeclareBoolOption{discretionarybreak}
%    \end{macrocode}
%
%    \begin{macrocode}
\kv@define@key{HoLogo}{variant}{%
  \ifx\HOLOGO@name\relax
    \@PackageError{hologo}{%
      Option `variant' is not available in \string\hologoSetup,%
      \MessageBreak
      Use \string\hologoLogoSetup\space instead%
    }\@ehc
  \else
    \edef\HOLOGO@temp{#1}%
    \ifx\HOLOGO@temp\ltx@empty
      \expandafter
      \let\csname HoLogoOpt@variant@\HOLOGO@name\endcsname\@undefined
    \else
      \ltx@IfUndefined{HoLogo@\HOLOGO@name @\HOLOGO@temp}{%
        \@PackageError{hologo}{%
          Unknown variant `\HOLOGO@temp' of logo `\HOLOGO@name'%
        }\@ehc
      }{%
        \expandafter
        \let\csname HoLogoOpt@variant@\HOLOGO@name\endcsname
            \HOLOGO@temp
      }%
    \fi
  \fi
}
%    \end{macrocode}
%
%    \begin{macro}{\HOLOGO@Variant}
%    \begin{macrocode}
\def\HOLOGO@Variant#1{%
  #1%
  \ltx@ifundefined{HoLogoOpt@variant@#1}{%
  }{%
    @\csname HoLogoOpt@variant@#1\endcsname
  }%
}
%    \end{macrocode}
%    \end{macro}
%
% \subsection{Break/no-break support}
%
%    \begin{macro}{\HOLOGO@space}
%    \begin{macrocode}
\def\HOLOGO@space{%
  \ltx@ifundefined{HoLogoOpt@spacebreak@\HOLOGO@name}{%
    \ltx@ifundefined{HoLogoOpt@break@\HOLOGO@name}{%
      \chardef\HOLOGO@temp=\HOLOGOOPT@spacebreak
    }{%
      \chardef\HOLOGO@temp=%
        \csname HoLogoOpt@break@\HOLOGO@name\endcsname
    }%
  }{%
    \chardef\HOLOGO@temp=%
      \csname HoLogoOpt@spacebreak@\HOLOGO@name\endcsname
  }%
  \ifcase\HOLOGO@temp
    \penalty10000 %
  \fi
  \ltx@space
}
%    \end{macrocode}
%    \end{macro}
%
%    \begin{macro}{\HOLOGO@hyphen}
%    \begin{macrocode}
\def\HOLOGO@hyphen{%
  \ltx@ifundefined{HoLogoOpt@hyphenbreak@\HOLOGO@name}{%
    \ltx@ifundefined{HoLogoOpt@break@\HOLOGO@name}{%
      \chardef\HOLOGO@temp=\HOLOGOOPT@hyphenbreak
    }{%
      \chardef\HOLOGO@temp=%
        \csname HoLogoOpt@break@\HOLOGO@name\endcsname
    }%
  }{%
    \chardef\HOLOGO@temp=%
      \csname HoLogoOpt@hyphenbreak@\HOLOGO@name\endcsname
  }%
  \ifcase\HOLOGO@temp
    \ltx@mbox{-}%
  \else
    -%
  \fi
}
%    \end{macrocode}
%    \end{macro}
%
%    \begin{macro}{\HOLOGO@discretionary}
%    \begin{macrocode}
\def\HOLOGO@discretionary{%
  \ltx@ifundefined{HoLogoOpt@discretionarybreak@\HOLOGO@name}{%
    \ltx@ifundefined{HoLogoOpt@break@\HOLOGO@name}{%
      \chardef\HOLOGO@temp=\HOLOGOOPT@discretionarybreak
    }{%
      \chardef\HOLOGO@temp=%
        \csname HoLogoOpt@break@\HOLOGO@name\endcsname
    }%
  }{%
    \chardef\HOLOGO@temp=%
      \csname HoLogoOpt@discretionarybreak@\HOLOGO@name\endcsname
  }%
  \ifcase\HOLOGO@temp
  \else
    \-%
  \fi
}
%    \end{macrocode}
%    \end{macro}
%
%    \begin{macro}{\HOLOGO@mbox}
%    \begin{macrocode}
\def\HOLOGO@mbox#1{%
  \ltx@ifundefined{HoLogoOpt@break@\HOLOGO@name}{%
    \chardef\HOLOGO@temp=\HOLOGOOPT@hyphenbreak
  }{%
    \chardef\HOLOGO@temp=%
      \csname HoLogoOpt@break@\HOLOGO@name\endcsname
  }%
  \ifcase\HOLOGO@temp
    \ltx@mbox{#1}%
  \else
    #1%
  \fi
}
%    \end{macrocode}
%    \end{macro}
%
% \subsection{Font support}
%
%    \begin{macro}{\HoLogoFont@font}
%    \begin{tabular}{@{}ll@{}}
%    |#1|:& logo name\\
%    |#2|:& font short name\\
%    |#3|:& text
%    \end{tabular}
%    \begin{macrocode}
\def\HoLogoFont@font#1#2#3{%
  \begingroup
    \ltx@IfUndefined{HoLogoFont@logo@#1.#2}{%
      \ltx@IfUndefined{HoLogoFont@font@#2}{%
        \@PackageWarning{hologo}{%
          Missing font `#2' for logo `#1'%
        }%
        #3%
      }{%
        \csname HoLogoFont@font@#2\endcsname{#3}%
      }%
    }{%
      \csname HoLogoFont@logo@#1.#2\endcsname{#3}%
    }%
  \endgroup
}
%    \end{macrocode}
%    \end{macro}
%
%    \begin{macro}{\HoLogoFont@Def}
%    \begin{macrocode}
\def\HoLogoFont@Def#1{%
  \expandafter\def\csname HoLogoFont@font@#1\endcsname
}
%    \end{macrocode}
%    \end{macro}
%    \begin{macro}{\HoLogoFont@LogoDef}
%    \begin{macrocode}
\def\HoLogoFont@LogoDef#1#2{%
  \expandafter\def\csname HoLogoFont@logo@#1.#2\endcsname
}
%    \end{macrocode}
%    \end{macro}
%
% \subsubsection{Font defaults}
%
%    \begin{macro}{\HoLogoFont@font@general}
%    \begin{macrocode}
\HoLogoFont@Def{general}{}%
%    \end{macrocode}
%    \end{macro}
%
%    \begin{macro}{\HoLogoFont@font@rm}
%    \begin{macrocode}
\ltx@IfUndefined{rmfamily}{%
  \ltx@IfUndefined{rm}{%
  }{%
    \HoLogoFont@Def{rm}{\rm}%
  }%
}{%
  \HoLogoFont@Def{rm}{\rmfamily}%
}
%    \end{macrocode}
%    \end{macro}
%
%    \begin{macro}{\HoLogoFont@font@sf}
%    \begin{macrocode}
\ltx@IfUndefined{sffamily}{%
  \ltx@IfUndefined{sf}{%
  }{%
    \HoLogoFont@Def{sf}{\sf}%
  }%
}{%
  \HoLogoFont@Def{sf}{\sffamily}%
}
%    \end{macrocode}
%    \end{macro}
%
%    \begin{macro}{\HoLogoFont@font@bibsf}
%    In case of \hologo{plainTeX} the original small caps
%    variant is used as default. In \hologo{LaTeX}
%    the definition of package \xpackage{dtklogos} \cite{dtklogos}
%    is used.
%\begin{quote}
%\begin{verbatim}
%\DeclareRobustCommand{\BibTeX}{%
%  B%
%  \kern-.05em%
%  \hbox{%
%    $\m@th$% %% force math size calculations
%    \csname S@\f@size\endcsname
%    \fontsize\sf@size\z@
%    \math@fontsfalse
%    \selectfont
%    I%
%    \kern-.025em%
%    B
%  }%
%  \kern-.08em%
%  \-%
%  \TeX
%}
%\end{verbatim}
%\end{quote}
%    \begin{macrocode}
\ltx@IfUndefined{selectfont}{%
  \ltx@IfUndefined{tensc}{%
    \font\tensc=cmcsc10\relax
  }{}%
  \HoLogoFont@Def{bibsf}{\tensc}%
}{%
  \HoLogoFont@Def{bibsf}{%
    $\mathsurround=0pt$%
    \csname S@\f@size\endcsname
    \fontsize\sf@size{0pt}%
    \math@fontsfalse
    \selectfont
  }%
}
%    \end{macrocode}
%    \end{macro}
%
%    \begin{macro}{\HoLogoFont@font@sc}
%    \begin{macrocode}
\ltx@IfUndefined{scshape}{%
  \ltx@IfUndefined{tensc}{%
    \font\tensc=cmcsc10\relax
  }{}%
  \HoLogoFont@Def{sc}{\tensc}%
}{%
  \HoLogoFont@Def{sc}{\scshape}%
}
%    \end{macrocode}
%    \end{macro}
%
%    \begin{macro}{\HoLogoFont@font@sy}
%    \begin{macrocode}
\ltx@IfUndefined{usefont}{%
  \ltx@IfUndefined{tensy}{%
  }{%
    \HoLogoFont@Def{sy}{\tensy}%
  }%
}{%
  \HoLogoFont@Def{sy}{%
    \usefont{OMS}{cmsy}{m}{n}%
  }%
}
%    \end{macrocode}
%    \end{macro}
%
%    \begin{macro}{\HoLogoFont@font@logo}
%    \begin{macrocode}
\begingroup
  \def\x{LaTeX2e}%
\expandafter\endgroup
\ifx\fmtname\x
  \ltx@IfUndefined{logofamily}{%
    \DeclareRobustCommand\logofamily{%
      \not@math@alphabet\logofamily\relax
      \fontencoding{U}%
      \fontfamily{logo}%
      \selectfont
    }%
  }{}%
  \ltx@IfUndefined{logofamily}{%
  }{%
    \HoLogoFont@Def{logo}{\logofamily}%
  }%
\else
  \ltx@IfUndefined{tenlogo}{%
    \font\tenlogo=logo10\relax
  }{}%
  \HoLogoFont@Def{logo}{\tenlogo}%
\fi
%    \end{macrocode}
%    \end{macro}
%
% \subsubsection{Font setup}
%
%    \begin{macro}{\hologoFontSetup}
%    \begin{macrocode}
\def\hologoFontSetup{%
  \let\HOLOGO@name\relax
  \HOLOGO@FontSetup
}
%    \end{macrocode}
%    \end{macro}
%
%    \begin{macro}{\hologoLogoFontSetup}
%    \begin{macrocode}
\def\hologoLogoFontSetup#1{%
  \edef\HOLOGO@name{#1}%
  \ltx@IfUndefined{HoLogo@\HOLOGO@name}{%
    \@PackageError{hologo}{%
      Unknown logo `\HOLOGO@name'%
    }\@ehc
    \ltx@gobble
  }{%
    \HOLOGO@FontSetup
  }%
}
%    \end{macrocode}
%    \end{macro}
%
%    \begin{macro}{\HOLOGO@FontSetup}
%    \begin{macrocode}
\def\HOLOGO@FontSetup{%
  \kvsetkeys{HoLogoFont}%
}
%    \end{macrocode}
%    \end{macro}
%
%    \begin{macrocode}
\def\HOLOGO@temp#1{%
  \kv@define@key{HoLogoFont}{#1}{%
    \ifx\HOLOGO@name\relax
      \HoLogoFont@Def{#1}{##1}%
    \else
      \HoLogoFont@LogoDef\HOLOGO@name{#1}{##1}%
    \fi
  }%
}
\HOLOGO@temp{general}
\HOLOGO@temp{sf}
%    \end{macrocode}
%
% \subsection{Generic logo commands}
%
%    \begin{macrocode}
\HOLOGO@IfExists\hologo{%
  \@PackageError{hologo}{%
    \string\hologo\ltx@space is already defined.\MessageBreak
    Package loading is aborted%
  }\@ehc
  \HOLOGO@AtEnd
}%
\HOLOGO@IfExists\hologoRobust{%
  \@PackageError{hologo}{%
    \string\hologoRobust\ltx@space is already defined.\MessageBreak
    Package loading is aborted%
  }\@ehc
  \HOLOGO@AtEnd
}%
%    \end{macrocode}
%
% \subsubsection{\cs{hologo} and friends}
%
%    \begin{macrocode}
\ifluatex
  \expandafter\ltx@firstofone
\else
  \expandafter\ltx@gobble
\fi
{%
  \ltx@IfUndefined{ifincsname}{%
    \ifnum\luatexversion<36 %
      \expandafter\ltx@gobble
    \else
      \expandafter\ltx@firstofone
    \fi
    {%
      \begingroup
        \ifcase0%
            \directlua{%
              if tex.enableprimitives then %
                tex.enableprimitives('HOLOGO@', {'ifincsname'})%
              else %
                tex.print('1')%
              end%
            }%
            \ifx\HOLOGO@ifincsname\@undefined 1\fi%
            \relax
          \expandafter\ltx@firstofone
        \else
          \endgroup
          \expandafter\ltx@gobble
        \fi
        {%
          \global\let\ifincsname\HOLOGO@ifincsname
        }%
      \HOLOGO@temp
    }%
  }{}%
}
%    \end{macrocode}
%    \begin{macrocode}
\ltx@IfUndefined{ifincsname}{%
  \catcode`$=14 %
}{%
  \catcode`$=9 %
}
%    \end{macrocode}
%
%    \begin{macro}{\hologo}
%    \begin{macrocode}
\def\hologo#1{%
$ \ifincsname
$   \ltx@ifundefined{HoLogoCs@\HOLOGO@Variant{#1}}{%
$     #1%
$   }{%
$     \csname HoLogoCs@\HOLOGO@Variant{#1}\endcsname\ltx@firstoftwo
$   }%
$ \else
    \HOLOGO@IfExists\texorpdfstring\texorpdfstring\ltx@firstoftwo
    {%
      \hologoRobust{#1}%
    }{%
      \ltx@ifundefined{HoLogoBkm@\HOLOGO@Variant{#1}}{%
        \ltx@ifundefined{HoLogo@#1}{?#1?}{#1}%
      }{%
        \csname HoLogoBkm@\HOLOGO@Variant{#1}\endcsname
        \ltx@firstoftwo
      }%
    }%
$ \fi
}
%    \end{macrocode}
%    \end{macro}
%    \begin{macro}{\Hologo}
%    \begin{macrocode}
\def\Hologo#1{%
$ \ifincsname
$   \ltx@ifundefined{HoLogoCs@\HOLOGO@Variant{#1}}{%
$     #1%
$   }{%
$     \csname HoLogoCs@\HOLOGO@Variant{#1}\endcsname\ltx@secondoftwo
$   }%
$ \else
    \HOLOGO@IfExists\texorpdfstring\texorpdfstring\ltx@firstoftwo
    {%
      \HologoRobust{#1}%
    }{%
      \ltx@ifundefined{HoLogoBkm@\HOLOGO@Variant{#1}}{%
        \ltx@ifundefined{HoLogo@#1}{?#1?}{#1}%
      }{%
        \csname HoLogoBkm@\HOLOGO@Variant{#1}\endcsname
        \ltx@secondoftwo
      }%
    }%
$ \fi
}
%    \end{macrocode}
%    \end{macro}
%
%    \begin{macro}{\hologoVariant}
%    \begin{macrocode}
\def\hologoVariant#1#2{%
  \ifx\relax#2\relax
    \hologo{#1}%
  \else
$   \ifincsname
$     \ltx@ifundefined{HoLogoCs@#1@#2}{%
$       #1%
$     }{%
$       \csname HoLogoCs@#1@#2\endcsname\ltx@firstoftwo
$     }%
$   \else
      \HOLOGO@IfExists\texorpdfstring\texorpdfstring\ltx@firstoftwo
      {%
        \hologoVariantRobust{#1}{#2}%
      }{%
        \ltx@ifundefined{HoLogoBkm@#1@#2}{%
          \ltx@ifundefined{HoLogo@#1}{?#1?}{#1}%
        }{%
          \csname HoLogoBkm@#1@#2\endcsname
          \ltx@firstoftwo
        }%
      }%
$   \fi
  \fi
}
%    \end{macrocode}
%    \end{macro}
%    \begin{macro}{\HologoVariant}
%    \begin{macrocode}
\def\HologoVariant#1#2{%
  \ifx\relax#2\relax
    \Hologo{#1}%
  \else
$   \ifincsname
$     \ltx@ifundefined{HoLogoCs@#1@#2}{%
$       #1%
$     }{%
$       \csname HoLogoCs@#1@#2\endcsname\ltx@secondoftwo
$     }%
$   \else
      \HOLOGO@IfExists\texorpdfstring\texorpdfstring\ltx@firstoftwo
      {%
        \HologoVariantRobust{#1}{#2}%
      }{%
        \ltx@ifundefined{HoLogoBkm@#1@#2}{%
          \ltx@ifundefined{HoLogo@#1}{?#1?}{#1}%
        }{%
          \csname HoLogoBkm@#1@#2\endcsname
          \ltx@secondoftwo
        }%
      }%
$   \fi
  \fi
}
%    \end{macrocode}
%    \end{macro}
%
%    \begin{macrocode}
\catcode`\$=3 %
%    \end{macrocode}
%
% \subsubsection{\cs{hologoRobust} and friends}
%
%    \begin{macro}{\hologoRobust}
%    \begin{macrocode}
\ltx@IfUndefined{protected}{%
  \ltx@IfUndefined{DeclareRobustCommand}{%
    \def\hologoRobust#1%
  }{%
    \DeclareRobustCommand*\hologoRobust[1]%
  }%
}{%
  \protected\def\hologoRobust#1%
}%
{%
  \edef\HOLOGO@name{#1}%
  \ltx@IfUndefined{HoLogo@\HOLOGO@Variant\HOLOGO@name}{%
    \@PackageError{hologo}{%
      Unknown logo `\HOLOGO@name'%
    }\@ehc
    ?\HOLOGO@name?%
  }{%
    \ltx@IfUndefined{ver@tex4ht.sty}{%
      \HoLogoFont@font\HOLOGO@name{general}{%
        \csname HoLogo@\HOLOGO@Variant\HOLOGO@name\endcsname
        \ltx@firstoftwo
      }%
    }{%
      \ltx@IfUndefined{HoLogoHtml@\HOLOGO@Variant\HOLOGO@name}{%
        \HOLOGO@name
      }{%
        \csname HoLogoHtml@\HOLOGO@Variant\HOLOGO@name\endcsname
        \ltx@firstoftwo
      }%
    }%
  }%
}
%    \end{macrocode}
%    \end{macro}
%    \begin{macro}{\HologoRobust}
%    \begin{macrocode}
\ltx@IfUndefined{protected}{%
  \ltx@IfUndefined{DeclareRobustCommand}{%
    \def\HologoRobust#1%
  }{%
    \DeclareRobustCommand*\HologoRobust[1]%
  }%
}{%
  \protected\def\HologoRobust#1%
}%
{%
  \edef\HOLOGO@name{#1}%
  \ltx@IfUndefined{HoLogo@\HOLOGO@Variant\HOLOGO@name}{%
    \@PackageError{hologo}{%
      Unknown logo `\HOLOGO@name'%
    }\@ehc
    ?\HOLOGO@name?%
  }{%
    \ltx@IfUndefined{ver@tex4ht.sty}{%
      \HoLogoFont@font\HOLOGO@name{general}{%
        \csname HoLogo@\HOLOGO@Variant\HOLOGO@name\endcsname
        \ltx@secondoftwo
      }%
    }{%
      \ltx@IfUndefined{HoLogoHtml@\HOLOGO@Variant\HOLOGO@name}{%
        \expandafter\HOLOGO@Uppercase\HOLOGO@name
      }{%
        \csname HoLogoHtml@\HOLOGO@Variant\HOLOGO@name\endcsname
        \ltx@secondoftwo
      }%
    }%
  }%
}
%    \end{macrocode}
%    \end{macro}
%    \begin{macro}{\hologoVariantRobust}
%    \begin{macrocode}
\ltx@IfUndefined{protected}{%
  \ltx@IfUndefined{DeclareRobustCommand}{%
    \def\hologoVariantRobust#1#2%
  }{%
    \DeclareRobustCommand*\hologoVariantRobust[2]%
  }%
}{%
  \protected\def\hologoVariantRobust#1#2%
}%
{%
  \begingroup
    \hologoLogoSetup{#1}{variant={#2}}%
    \hologoRobust{#1}%
  \endgroup
}
%    \end{macrocode}
%    \end{macro}
%    \begin{macro}{\HologoVariantRobust}
%    \begin{macrocode}
\ltx@IfUndefined{protected}{%
  \ltx@IfUndefined{DeclareRobustCommand}{%
    \def\HologoVariantRobust#1#2%
  }{%
    \DeclareRobustCommand*\HologoVariantRobust[2]%
  }%
}{%
  \protected\def\HologoVariantRobust#1#2%
}%
{%
  \begingroup
    \hologoLogoSetup{#1}{variant={#2}}%
    \HologoRobust{#1}%
  \endgroup
}
%    \end{macrocode}
%    \end{macro}
%
%    \begin{macro}{\hologorobust}
%    Macro \cs{hologorobust} is only defined for compatibility.
%    Its use is deprecated.
%    \begin{macrocode}
\def\hologorobust{\hologoRobust}
%    \end{macrocode}
%    \end{macro}
%
% \subsection{Helpers}
%
%    \begin{macro}{\HOLOGO@Uppercase}
%    Macro \cs{HOLOGO@Uppercase} is restricted to \cs{uppercase},
%    because \hologo{plainTeX} or \hologo{iniTeX} do not provide
%    \cs{MakeUppercase}.
%    \begin{macrocode}
\def\HOLOGO@Uppercase#1{\uppercase{#1}}
%    \end{macrocode}
%    \end{macro}
%
%    \begin{macro}{\HOLOGO@PdfdocUnicode}
%    \begin{macrocode}
\def\HOLOGO@PdfdocUnicode{%
  \ifx\ifHy@unicode\iftrue
    \expandafter\ltx@secondoftwo
  \else
    \expandafter\ltx@firstoftwo
  \fi
}
%    \end{macrocode}
%    \end{macro}
%
%    \begin{macro}{\HOLOGO@Math}
%    \begin{macrocode}
\def\HOLOGO@MathSetup{%
  \mathsurround0pt\relax
  \HOLOGO@IfExists\f@series{%
    \if b\expandafter\ltx@car\f@series x\@nil
      \csname boldmath\endcsname
   \fi
  }{}%
}
%    \end{macrocode}
%    \end{macro}
%
%    \begin{macro}{\HOLOGO@TempDimen}
%    \begin{macrocode}
\dimendef\HOLOGO@TempDimen=\ltx@zero
%    \end{macrocode}
%    \end{macro}
%    \begin{macro}{\HOLOGO@NegativeKerning}
%    \begin{macrocode}
\def\HOLOGO@NegativeKerning#1{%
  \begingroup
    \HOLOGO@TempDimen=0pt\relax
    \comma@parse@normalized{#1}{%
      \ifdim\HOLOGO@TempDimen=0pt %
        \expandafter\HOLOGO@@NegativeKerning\comma@entry
      \fi
      \ltx@gobble
    }%
    \ifdim\HOLOGO@TempDimen<0pt %
      \kern\HOLOGO@TempDimen
    \fi
  \endgroup
}
%    \end{macrocode}
%    \end{macro}
%    \begin{macro}{\HOLOGO@@NegativeKerning}
%    \begin{macrocode}
\def\HOLOGO@@NegativeKerning#1#2{%
  \setbox\ltx@zero\hbox{#1#2}%
  \HOLOGO@TempDimen=\wd\ltx@zero
  \setbox\ltx@zero\hbox{#1\kern0pt#2}%
  \advance\HOLOGO@TempDimen by -\wd\ltx@zero
}
%    \end{macrocode}
%    \end{macro}
%
%    \begin{macro}{\HOLOGO@SpaceFactor}
%    \begin{macrocode}
\def\HOLOGO@SpaceFactor{%
  \spacefactor1000 %
}
%    \end{macrocode}
%    \end{macro}
%
%    \begin{macro}{\HOLOGO@Span}
%    \begin{macrocode}
\def\HOLOGO@Span#1#2{%
  \HCode{<span class="HoLogo-#1">}%
  #2%
  \HCode{</span>}%
}
%    \end{macrocode}
%    \end{macro}
%
% \subsubsection{Text subscript}
%
%    \begin{macro}{\HOLOGO@SubScript}%
%    \begin{macrocode}
\def\HOLOGO@SubScript#1{%
  \ltx@IfUndefined{textsubscript}{%
    \ltx@IfUndefined{text}{%
      \ltx@mbox{%
        \mathsurround=0pt\relax
        $%
          _{%
            \ltx@IfUndefined{sf@size}{%
              \mathrm{#1}%
            }{%
              \mbox{%
                \fontsize\sf@size{0pt}\selectfont
                #1%
              }%
            }%
          }%
        $%
      }%
    }{%
      \ltx@mbox{%
        \mathsurround=0pt\relax
        $_{\text{#1}}$%
      }%
    }%
  }{%
    \textsubscript{#1}%
  }%
}
%    \end{macrocode}
%    \end{macro}
%
% \subsection{\hologo{TeX} and friends}
%
% \subsubsection{\hologo{TeX}}
%
%    \begin{macro}{\HoLogo@TeX}
%    Source: \hologo{LaTeX} kernel.
%    \begin{macrocode}
\def\HoLogo@TeX#1{%
  T\kern-.1667em\lower.5ex\hbox{E}\kern-.125emX\HOLOGO@SpaceFactor
}
%    \end{macrocode}
%    \end{macro}
%    \begin{macro}{\HoLogoHtml@TeX}
%    \begin{macrocode}
\def\HoLogoHtml@TeX#1{%
  \HoLogoCss@TeX
  \HOLOGO@Span{TeX}{%
    T%
    \HOLOGO@Span{e}{%
      E%
    }%
    X%
  }%
}
%    \end{macrocode}
%    \end{macro}
%    \begin{macro}{\HoLogoCss@TeX}
%    \begin{macrocode}
\def\HoLogoCss@TeX{%
  \Css{%
    span.HoLogo-TeX span.HoLogo-e{%
      position:relative;%
      top:.5ex;%
      margin-left:-.1667em;%
      margin-right:-.125em;%
    }%
  }%
  \Css{%
    a span.HoLogo-TeX span.HoLogo-e{%
      text-decoration:none;%
    }%
  }%
  \global\let\HoLogoCss@TeX\relax
}
%    \end{macrocode}
%    \end{macro}
%
% \subsubsection{\hologo{plainTeX}}
%
%    \begin{macro}{\HoLogo@plainTeX@space}
%    Source: ``The \hologo{TeX}book''
%    \begin{macrocode}
\def\HoLogo@plainTeX@space#1{%
  \HOLOGO@mbox{#1{p}{P}lain}\HOLOGO@space\hologo{TeX}%
}
%    \end{macrocode}
%    \end{macro}
%    \begin{macro}{\HoLogoCs@plainTeX@space}
%    \begin{macrocode}
\def\HoLogoCs@plainTeX@space#1{#1{p}{P}lain TeX}%
%    \end{macrocode}
%    \end{macro}
%    \begin{macro}{\HoLogoBkm@plainTeX@space}
%    \begin{macrocode}
\def\HoLogoBkm@plainTeX@space#1{%
  #1{p}{P}lain \hologo{TeX}%
}
%    \end{macrocode}
%    \end{macro}
%    \begin{macro}{\HoLogoHtml@plainTeX@space}
%    \begin{macrocode}
\def\HoLogoHtml@plainTeX@space#1{%
  #1{p}{P}lain \hologo{TeX}%
}
%    \end{macrocode}
%    \end{macro}
%
%    \begin{macro}{\HoLogo@plainTeX@hyphen}
%    \begin{macrocode}
\def\HoLogo@plainTeX@hyphen#1{%
  \HOLOGO@mbox{#1{p}{P}lain}\HOLOGO@hyphen\hologo{TeX}%
}
%    \end{macrocode}
%    \end{macro}
%    \begin{macro}{\HoLogoCs@plainTeX@hyphen}
%    \begin{macrocode}
\def\HoLogoCs@plainTeX@hyphen#1{#1{p}{P}lain-TeX}
%    \end{macrocode}
%    \end{macro}
%    \begin{macro}{\HoLogoBkm@plainTeX@hyphen}
%    \begin{macrocode}
\def\HoLogoBkm@plainTeX@hyphen#1{%
  #1{p}{P}lain-\hologo{TeX}%
}
%    \end{macrocode}
%    \end{macro}
%    \begin{macro}{\HoLogoHtml@plainTeX@hyphen}
%    \begin{macrocode}
\def\HoLogoHtml@plainTeX@hyphen#1{%
  #1{p}{P}lain-\hologo{TeX}%
}
%    \end{macrocode}
%    \end{macro}
%
%    \begin{macro}{\HoLogo@plainTeX@runtogether}
%    \begin{macrocode}
\def\HoLogo@plainTeX@runtogether#1{%
  \HOLOGO@mbox{#1{p}{P}lain\hologo{TeX}}%
}
%    \end{macrocode}
%    \end{macro}
%    \begin{macro}{\HoLogoCs@plainTeX@runtogether}
%    \begin{macrocode}
\def\HoLogoCs@plainTeX@runtogether#1{#1{p}{P}lainTeX}
%    \end{macrocode}
%    \end{macro}
%    \begin{macro}{\HoLogoBkm@plainTeX@runtogether}
%    \begin{macrocode}
\def\HoLogoBkm@plainTeX@runtogether#1{%
  #1{p}{P}lain\hologo{TeX}%
}
%    \end{macrocode}
%    \end{macro}
%    \begin{macro}{\HoLogoHtml@plainTeX@runtogether}
%    \begin{macrocode}
\def\HoLogoHtml@plainTeX@runtogether#1{%
  #1{p}{P}lain\hologo{TeX}%
}
%    \end{macrocode}
%    \end{macro}
%
%    \begin{macro}{\HoLogo@plainTeX}
%    \begin{macrocode}
\def\HoLogo@plainTeX{\HoLogo@plainTeX@space}
%    \end{macrocode}
%    \end{macro}
%    \begin{macro}{\HoLogoCs@plainTeX}
%    \begin{macrocode}
\def\HoLogoCs@plainTeX{\HoLogoCs@plainTeX@space}
%    \end{macrocode}
%    \end{macro}
%    \begin{macro}{\HoLogoBkm@plainTeX}
%    \begin{macrocode}
\def\HoLogoBkm@plainTeX{\HoLogoBkm@plainTeX@space}
%    \end{macrocode}
%    \end{macro}
%    \begin{macro}{\HoLogoHtml@plainTeX}
%    \begin{macrocode}
\def\HoLogoHtml@plainTeX{\HoLogoHtml@plainTeX@space}
%    \end{macrocode}
%    \end{macro}
%
% \subsubsection{\hologo{LaTeX}}
%
%    Source: \hologo{LaTeX} kernel.
%\begin{quote}
%\begin{verbatim}
%\DeclareRobustCommand{\LaTeX}{%
%  L%
%  \kern-.36em%
%  {%
%    \sbox\z@ T%
%    \vbox to\ht\z@{%
%      \hbox{%
%        \check@mathfonts
%        \fontsize\sf@size\z@
%        \math@fontsfalse
%        \selectfont
%        A%
%      }%
%      \vss
%    }%
%  }%
%  \kern-.15em%
%  \TeX
%}
%\end{verbatim}
%\end{quote}
%
%    \begin{macro}{\HoLogo@La}
%    \begin{macrocode}
\def\HoLogo@La#1{%
  L%
  \kern-.36em%
  \begingroup
    \setbox\ltx@zero\hbox{T}%
    \vbox to\ht\ltx@zero{%
      \hbox{%
        \ltx@ifundefined{check@mathfonts}{%
          \csname sevenrm\endcsname
        }{%
          \check@mathfonts
          \fontsize\sf@size{0pt}%
          \math@fontsfalse\selectfont
        }%
        A%
      }%
      \vss
    }%
  \endgroup
}
%    \end{macrocode}
%    \end{macro}
%
%    \begin{macro}{\HoLogo@LaTeX}
%    Source: \hologo{LaTeX} kernel.
%    \begin{macrocode}
\def\HoLogo@LaTeX#1{%
  \hologo{La}%
  \kern-.15em%
  \hologo{TeX}%
}
%    \end{macrocode}
%    \end{macro}
%    \begin{macro}{\HoLogoHtml@LaTeX}
%    \begin{macrocode}
\def\HoLogoHtml@LaTeX#1{%
  \HoLogoCss@LaTeX
  \HOLOGO@Span{LaTeX}{%
    L%
    \HOLOGO@Span{a}{%
      A%
    }%
    \hologo{TeX}%
  }%
}
%    \end{macrocode}
%    \end{macro}
%    \begin{macro}{\HoLogoCss@LaTeX}
%    \begin{macrocode}
\def\HoLogoCss@LaTeX{%
  \Css{%
    span.HoLogo-LaTeX span.HoLogo-a{%
      position:relative;%
      top:-.5ex;%
      margin-left:-.36em;%
      margin-right:-.15em;%
      font-size:85\%;%
    }%
  }%
  \global\let\HoLogoCss@LaTeX\relax
}
%    \end{macrocode}
%    \end{macro}
%
% \subsubsection{\hologo{(La)TeX}}
%
%    \begin{macro}{\HoLogo@LaTeXTeX}
%    The kerning around the parentheses is taken
%    from package \xpackage{dtklogos} \cite{dtklogos}.
%\begin{quote}
%\begin{verbatim}
%\DeclareRobustCommand{\LaTeXTeX}{%
%  (%
%  \kern-.15em%
%  L%
%  \kern-.36em%
%  {%
%    \sbox\z@ T%
%    \vbox to\ht0{%
%      \hbox{%
%        $\m@th$%
%        \csname S@\f@size\endcsname
%        \fontsize\sf@size\z@
%        \math@fontsfalse
%        \selectfont
%        A%
%      }%
%      \vss
%    }%
%  }%
%  \kern-.2em%
%  )%
%  \kern-.15em%
%  \TeX
%}
%\end{verbatim}
%\end{quote}
%    \begin{macrocode}
\def\HoLogo@LaTeXTeX#1{%
  (%
  \kern-.15em%
  \hologo{La}%
  \kern-.2em%
  )%
  \kern-.15em%
  \hologo{TeX}%
}
%    \end{macrocode}
%    \end{macro}
%    \begin{macro}{\HoLogoBkm@LaTeXTeX}
%    \begin{macrocode}
\def\HoLogoBkm@LaTeXTeX#1{(La)TeX}
%    \end{macrocode}
%    \end{macro}
%
%    \begin{macro}{\HoLogo@(La)TeX}
%    \begin{macrocode}
\expandafter
\let\csname HoLogo@(La)TeX\endcsname\HoLogo@LaTeXTeX
%    \end{macrocode}
%    \end{macro}
%    \begin{macro}{\HoLogoBkm@(La)TeX}
%    \begin{macrocode}
\expandafter
\let\csname HoLogoBkm@(La)TeX\endcsname\HoLogoBkm@LaTeXTeX
%    \end{macrocode}
%    \end{macro}
%    \begin{macro}{\HoLogoHtml@LaTeXTeX}
%    \begin{macrocode}
\def\HoLogoHtml@LaTeXTeX#1{%
  \HoLogoCss@LaTeXTeX
  \HOLOGO@Span{LaTeXTeX}{%
    (%
    \HOLOGO@Span{L}{L}%
    \HOLOGO@Span{a}{A}%
    \HOLOGO@Span{ParenRight}{)}%
    \hologo{TeX}%
  }%
}
%    \end{macrocode}
%    \end{macro}
%    \begin{macro}{\HoLogoHtml@(La)TeX}
%    Kerning after opening parentheses and before closing parentheses
%    is $-0.1$\,em. The original values $-0.15$\,em
%    looked too ugly for a serif font.
%    \begin{macrocode}
\expandafter
\let\csname HoLogoHtml@(La)TeX\endcsname\HoLogoHtml@LaTeXTeX
%    \end{macrocode}
%    \end{macro}
%    \begin{macro}{\HoLogoCss@LaTeXTeX}
%    \begin{macrocode}
\def\HoLogoCss@LaTeXTeX{%
  \Css{%
    span.HoLogo-LaTeXTeX span.HoLogo-L{%
      margin-left:-.1em;%
    }%
  }%
  \Css{%
    span.HoLogo-LaTeXTeX span.HoLogo-a{%
      position:relative;%
      top:-.5ex;%
      margin-left:-.36em;%
      margin-right:-.1em;%
      font-size:85\%;%
    }%
  }%
  \Css{%
    span.HoLogo-LaTeXTeX span.HoLogo-ParenRight{%
      margin-right:-.15em;%
    }%
  }%
  \global\let\HoLogoCss@LaTeXTeX\relax
}
%    \end{macrocode}
%    \end{macro}
%
% \subsubsection{\hologo{LaTeXe}}
%
%    \begin{macro}{\HoLogo@LaTeXe}
%    Source: \hologo{LaTeX} kernel
%    \begin{macrocode}
\def\HoLogo@LaTeXe#1{%
  \hologo{LaTeX}%
  \kern.15em%
  \hbox{%
    \HOLOGO@MathSetup
    2%
    $_{\textstyle\varepsilon}$%
  }%
}
%    \end{macrocode}
%    \end{macro}
%
%    \begin{macro}{\HoLogoCs@LaTeXe}
%    \begin{macrocode}
\ifnum64=`\^^^^0040\relax % test for big chars of LuaTeX/XeTeX
  \catcode`\$=9 %
  \catcode`\&=14 %
\else
  \catcode`\$=14 %
  \catcode`\&=9 %
\fi
\def\HoLogoCs@LaTeXe#1{%
  LaTeX2%
$ \string ^^^^0395%
& e%
}%
\catcode`\$=3 %
\catcode`\&=4 %
%    \end{macrocode}
%    \end{macro}
%
%    \begin{macro}{\HoLogoBkm@LaTeXe}
%    \begin{macrocode}
\def\HoLogoBkm@LaTeXe#1{%
  \hologo{LaTeX}%
  2%
  \HOLOGO@PdfdocUnicode{e}{\textepsilon}%
}
%    \end{macrocode}
%    \end{macro}
%
%    \begin{macro}{\HoLogoHtml@LaTeXe}
%    \begin{macrocode}
\def\HoLogoHtml@LaTeXe#1{%
  \HoLogoCss@LaTeXe
  \HOLOGO@Span{LaTeX2e}{%
    \hologo{LaTeX}%
    \HOLOGO@Span{2}{2}%
    \HOLOGO@Span{e}{%
      \HOLOGO@MathSetup
      \ensuremath{\textstyle\varepsilon}%
    }%
  }%
}
%    \end{macrocode}
%    \end{macro}
%    \begin{macro}{\HoLogoCss@LaTeXe}
%    \begin{macrocode}
\def\HoLogoCss@LaTeXe{%
  \Css{%
    span.HoLogo-LaTeX2e span.HoLogo-2{%
      padding-left:.15em;%
    }%
  }%
  \Css{%
    span.HoLogo-LaTeX2e span.HoLogo-e{%
      position:relative;%
      top:.35ex;%
      text-decoration:none;%
    }%
  }%
  \global\let\HoLogoCss@LaTeXe\relax
}
%    \end{macrocode}
%    \end{macro}
%
%    \begin{macro}{\HoLogo@LaTeX2e}
%    \begin{macrocode}
\expandafter
\let\csname HoLogo@LaTeX2e\endcsname\HoLogo@LaTeXe
%    \end{macrocode}
%    \end{macro}
%    \begin{macro}{\HoLogoCs@LaTeX2e}
%    \begin{macrocode}
\expandafter
\let\csname HoLogoCs@LaTeX2e\endcsname\HoLogoCs@LaTeXe
%    \end{macrocode}
%    \end{macro}
%    \begin{macro}{\HoLogoBkm@LaTeX2e}
%    \begin{macrocode}
\expandafter
\let\csname HoLogoBkm@LaTeX2e\endcsname\HoLogoBkm@LaTeXe
%    \end{macrocode}
%    \end{macro}
%    \begin{macro}{\HoLogoHtml@LaTeX2e}
%    \begin{macrocode}
\expandafter
\let\csname HoLogoHtml@LaTeX2e\endcsname\HoLogoHtml@LaTeXe
%    \end{macrocode}
%    \end{macro}
%
% \subsubsection{\hologo{LaTeX3}}
%
%    \begin{macro}{\HoLogo@LaTeX3}
%    Source: \hologo{LaTeX} kernel
%    \begin{macrocode}
\expandafter\def\csname HoLogo@LaTeX3\endcsname#1{%
  \hologo{LaTeX}%
  3%
}
%    \end{macrocode}
%    \end{macro}
%
%    \begin{macro}{\HoLogoBkm@LaTeX3}
%    \begin{macrocode}
\expandafter\def\csname HoLogoBkm@LaTeX3\endcsname#1{%
  \hologo{LaTeX}%
  3%
}
%    \end{macrocode}
%    \end{macro}
%    \begin{macro}{\HoLogoHtml@LaTeX3}
%    \begin{macrocode}
\expandafter
\let\csname HoLogoHtml@LaTeX3\expandafter\endcsname
\csname HoLogo@LaTeX3\endcsname
%    \end{macrocode}
%    \end{macro}
%
% \subsubsection{\hologo{LaTeXML}}
%
%    \begin{macro}{\HoLogo@LaTeXML}
%    \begin{macrocode}
\def\HoLogo@LaTeXML#1{%
  \HOLOGO@mbox{%
    \hologo{La}%
    \kern-.15em%
    T%
    \kern-.1667em%
    \lower.5ex\hbox{E}%
    \kern-.125em%
    \HoLogoFont@font{LaTeXML}{sc}{xml}%
  }%
}
%    \end{macrocode}
%    \end{macro}
%    \begin{macro}{\HoLogoHtml@pdfLaTeX}
%    \begin{macrocode}
\def\HoLogoHtml@LaTeXML#1{%
  \HOLOGO@Span{LaTeXML}{%
    \HoLogoCss@LaTeX
    \HoLogoCss@TeX
    \HOLOGO@Span{LaTeX}{%
      L%
      \HOLOGO@Span{a}{%
        A%
      }%
    }%
    \HOLOGO@Span{TeX}{%
      T%
      \HOLOGO@Span{e}{%
        E%
      }%
    }%
    \HCode{<span style="font-variant: small-caps;">}%
    xml%
    \HCode{</span>}%
  }%
}
%    \end{macrocode}
%    \end{macro}
%
% \subsubsection{\hologo{eTeX}}
%
%    \begin{macro}{\HoLogo@eTeX}
%    Source: package \xpackage{etex}
%    \begin{macrocode}
\def\HoLogo@eTeX#1{%
  \ltx@mbox{%
    \HOLOGO@MathSetup
    $\varepsilon$%
    -%
    \HOLOGO@NegativeKerning{-T,T-,To}%
    \hologo{TeX}%
  }%
}
%    \end{macrocode}
%    \end{macro}
%    \begin{macro}{\HoLogoCs@eTeX}
%    \begin{macrocode}
\ifnum64=`\^^^^0040\relax % test for big chars of LuaTeX/XeTeX
  \catcode`\$=9 %
  \catcode`\&=14 %
\else
  \catcode`\$=14 %
  \catcode`\&=9 %
\fi
\def\HoLogoCs@eTeX#1{%
$ #1{\string ^^^^0395}{\string ^^^^03b5}%
& #1{e}{E}%
  TeX%
}%
\catcode`\$=3 %
\catcode`\&=4 %
%    \end{macrocode}
%    \end{macro}
%    \begin{macro}{\HoLogoBkm@eTeX}
%    \begin{macrocode}
\def\HoLogoBkm@eTeX#1{%
  \HOLOGO@PdfdocUnicode{#1{e}{E}}{\textepsilon}%
  -%
  \hologo{TeX}%
}
%    \end{macrocode}
%    \end{macro}
%    \begin{macro}{\HoLogoHtml@eTeX}
%    \begin{macrocode}
\def\HoLogoHtml@eTeX#1{%
  \ltx@mbox{%
    \HOLOGO@MathSetup
    $\varepsilon$%
    -%
    \hologo{TeX}%
  }%
}
%    \end{macrocode}
%    \end{macro}
%
% \subsubsection{\hologo{iniTeX}}
%
%    \begin{macro}{\HoLogo@iniTeX}
%    \begin{macrocode}
\def\HoLogo@iniTeX#1{%
  \HOLOGO@mbox{%
    #1{i}{I}ni\hologo{TeX}%
  }%
}
%    \end{macrocode}
%    \end{macro}
%    \begin{macro}{\HoLogoCs@iniTeX}
%    \begin{macrocode}
\def\HoLogoCs@iniTeX#1{#1{i}{I}niTeX}
%    \end{macrocode}
%    \end{macro}
%    \begin{macro}{\HoLogoBkm@iniTeX}
%    \begin{macrocode}
\def\HoLogoBkm@iniTeX#1{%
  #1{i}{I}ni\hologo{TeX}%
}
%    \end{macrocode}
%    \end{macro}
%    \begin{macro}{\HoLogoHtml@iniTeX}
%    \begin{macrocode}
\let\HoLogoHtml@iniTeX\HoLogo@iniTeX
%    \end{macrocode}
%    \end{macro}
%
% \subsubsection{\hologo{virTeX}}
%
%    \begin{macro}{\HoLogo@virTeX}
%    \begin{macrocode}
\def\HoLogo@virTeX#1{%
  \HOLOGO@mbox{%
    #1{v}{V}ir\hologo{TeX}%
  }%
}
%    \end{macrocode}
%    \end{macro}
%    \begin{macro}{\HoLogoCs@virTeX}
%    \begin{macrocode}
\def\HoLogoCs@virTeX#1{#1{v}{V}irTeX}
%    \end{macrocode}
%    \end{macro}
%    \begin{macro}{\HoLogoBkm@virTeX}
%    \begin{macrocode}
\def\HoLogoBkm@virTeX#1{%
  #1{v}{V}ir\hologo{TeX}%
}
%    \end{macrocode}
%    \end{macro}
%    \begin{macro}{\HoLogoHtml@virTeX}
%    \begin{macrocode}
\let\HoLogoHtml@virTeX\HoLogo@virTeX
%    \end{macrocode}
%    \end{macro}
%
% \subsubsection{\hologo{SliTeX}}
%
% \paragraph{Definitions of the three variants.}
%
%    \begin{macro}{\HoLogo@SLiTeX@lift}
%    \begin{macrocode}
\def\HoLogo@SLiTeX@lift#1{%
  \HoLogoFont@font{SliTeX}{rm}{%
    S%
    \kern-.06em%
    L%
    \kern-.18em%
    \raise.32ex\hbox{\HoLogoFont@font{SliTeX}{sc}{i}}%
    \HOLOGO@discretionary
    \kern-.06em%
    \hologo{TeX}%
  }%
}
%    \end{macrocode}
%    \end{macro}
%    \begin{macro}{\HoLogoBkm@SLiTeX@lift}
%    \begin{macrocode}
\def\HoLogoBkm@SLiTeX@lift#1{SLiTeX}
%    \end{macrocode}
%    \end{macro}
%    \begin{macro}{\HoLogoHtml@SLiTeX@lift}
%    \begin{macrocode}
\def\HoLogoHtml@SLiTeX@lift#1{%
  \HoLogoCss@SLiTeX@lift
  \HOLOGO@Span{SLiTeX-lift}{%
    \HoLogoFont@font{SliTeX}{rm}{%
      S%
      \HOLOGO@Span{L}{L}%
      \HOLOGO@Span{i}{i}%
      \hologo{TeX}%
    }%
  }%
}
%    \end{macrocode}
%    \end{macro}
%    \begin{macro}{\HoLogoCss@SLiTeX@lift}
%    \begin{macrocode}
\def\HoLogoCss@SLiTeX@lift{%
  \Css{%
    span.HoLogo-SLiTeX-lift span.HoLogo-L{%
      margin-left:-.06em;%
      margin-right:-.18em;%
    }%
  }%
  \Css{%
    span.HoLogo-SLiTeX-lift span.HoLogo-i{%
      position:relative;%
      top:-.32ex;%
      margin-right:-.06em;%
      font-variant:small-caps;%
    }%
  }%
  \global\let\HoLogoCss@SLiTeX@lift\relax
}
%    \end{macrocode}
%    \end{macro}
%
%    \begin{macro}{\HoLogo@SliTeX@simple}
%    \begin{macrocode}
\def\HoLogo@SliTeX@simple#1{%
  \HoLogoFont@font{SliTeX}{rm}{%
    \ltx@mbox{%
      \HoLogoFont@font{SliTeX}{sc}{Sli}%
    }%
    \HOLOGO@discretionary
    \hologo{TeX}%
  }%
}
%    \end{macrocode}
%    \end{macro}
%    \begin{macro}{\HoLogoBkm@SliTeX@simple}
%    \begin{macrocode}
\def\HoLogoBkm@SliTeX@simple#1{SliTeX}
%    \end{macrocode}
%    \end{macro}
%    \begin{macro}{\HoLogoHtml@SliTeX@simple}
%    \begin{macrocode}
\let\HoLogoHtml@SliTeX@simple\HoLogo@SliTeX@simple
%    \end{macrocode}
%    \end{macro}
%
%    \begin{macro}{\HoLogo@SliTeX@narrow}
%    \begin{macrocode}
\def\HoLogo@SliTeX@narrow#1{%
  \HoLogoFont@font{SliTeX}{rm}{%
    \ltx@mbox{%
      S%
      \kern-.06em%
      \HoLogoFont@font{SliTeX}{sc}{%
        l%
        \kern-.035em%
        i%
      }%
    }%
    \HOLOGO@discretionary
    \kern-.06em%
    \hologo{TeX}%
  }%
}
%    \end{macrocode}
%    \end{macro}
%    \begin{macro}{\HoLogoBkm@SliTeX@narrow}
%    \begin{macrocode}
\def\HoLogoBkm@SliTeX@narrow#1{SliTeX}
%    \end{macrocode}
%    \end{macro}
%    \begin{macro}{\HoLogoHtml@SliTeX@narrow}
%    \begin{macrocode}
\def\HoLogoHtml@SliTeX@narrow#1{%
  \HoLogoCss@SliTeX@narrow
  \HOLOGO@Span{SliTeX-narrow}{%
    \HoLogoFont@font{SliTeX}{rm}{%
      S%
        \HOLOGO@Span{l}{l}%
        \HOLOGO@Span{i}{i}%
      \hologo{TeX}%
    }%
  }%
}
%    \end{macrocode}
%    \end{macro}
%    \begin{macro}{\HoLogoCss@SliTeX@narrow}
%    \begin{macrocode}
\def\HoLogoCss@SliTeX@narrow{%
  \Css{%
    span.HoLogo-SliTeX-narrow span.HoLogo-l{%
      margin-left:-.06em;%
      margin-right:-.035em;%
      font-variant:small-caps;%
    }%
  }%
  \Css{%
    span.HoLogo-SliTeX-narrow span.HoLogo-i{%
      margin-right:-.06em;%
      font-variant:small-caps;%
    }%
  }%
  \global\let\HoLogoCss@SliTeX@narrow\relax
}
%    \end{macrocode}
%    \end{macro}
%
% \paragraph{Macro set completion.}
%
%    \begin{macro}{\HoLogo@SLiTeX@simple}
%    \begin{macrocode}
\def\HoLogo@SLiTeX@simple{\HoLogo@SliTeX@simple}
%    \end{macrocode}
%    \end{macro}
%    \begin{macro}{\HoLogoBkm@SLiTeX@simple}
%    \begin{macrocode}
\def\HoLogoBkm@SLiTeX@simple{\HoLogoBkm@SliTeX@simple}
%    \end{macrocode}
%    \end{macro}
%    \begin{macro}{\HoLogoHtml@SLiTeX@simple}
%    \begin{macrocode}
\def\HoLogoHtml@SLiTeX@simple{\HoLogoHtml@SliTeX@simple}
%    \end{macrocode}
%    \end{macro}
%
%    \begin{macro}{\HoLogo@SLiTeX@narrow}
%    \begin{macrocode}
\def\HoLogo@SLiTeX@narrow{\HoLogo@SliTeX@narrow}
%    \end{macrocode}
%    \end{macro}
%    \begin{macro}{\HoLogoBkm@SLiTeX@narrow}
%    \begin{macrocode}
\def\HoLogoBkm@SLiTeX@narrow{\HoLogoBkm@SliTeX@narrow}
%    \end{macrocode}
%    \end{macro}
%    \begin{macro}{\HoLogoHtml@SLiTeX@narrow}
%    \begin{macrocode}
\def\HoLogoHtml@SLiTeX@narrow{\HoLogoHtml@SliTeX@narrow}
%    \end{macrocode}
%    \end{macro}
%
%    \begin{macro}{\HoLogo@SliTeX@lift}
%    \begin{macrocode}
\def\HoLogo@SliTeX@lift{\HoLogo@SLiTeX@lift}
%    \end{macrocode}
%    \end{macro}
%    \begin{macro}{\HoLogoBkm@SliTeX@lift}
%    \begin{macrocode}
\def\HoLogoBkm@SliTeX@lift{\HoLogoBkm@SLiTeX@lift}
%    \end{macrocode}
%    \end{macro}
%    \begin{macro}{\HoLogoHtml@SliTeX@lift}
%    \begin{macrocode}
\def\HoLogoHtml@SliTeX@lift{\HoLogoHtml@SLiTeX@lift}
%    \end{macrocode}
%    \end{macro}
%
% \paragraph{Defaults.}
%
%    \begin{macro}{\HoLogo@SLiTeX}
%    \begin{macrocode}
\def\HoLogo@SLiTeX{\HoLogo@SLiTeX@lift}
%    \end{macrocode}
%    \end{macro}
%    \begin{macro}{\HoLogoBkm@SLiTeX}
%    \begin{macrocode}
\def\HoLogoBkm@SLiTeX{\HoLogoBkm@SLiTeX@lift}
%    \end{macrocode}
%    \end{macro}
%    \begin{macro}{\HoLogoHtml@SLiTeX}
%    \begin{macrocode}
\def\HoLogoHtml@SLiTeX{\HoLogoHtml@SLiTeX@lift}
%    \end{macrocode}
%    \end{macro}
%
%    \begin{macro}{\HoLogo@SliTeX}
%    \begin{macrocode}
\def\HoLogo@SliTeX{\HoLogo@SliTeX@narrow}
%    \end{macrocode}
%    \end{macro}
%    \begin{macro}{\HoLogoBkm@SliTeX}
%    \begin{macrocode}
\def\HoLogoBkm@SliTeX{\HoLogoBkm@SliTeX@narrow}
%    \end{macrocode}
%    \end{macro}
%    \begin{macro}{\HoLogoHtml@SliTeX}
%    \begin{macrocode}
\def\HoLogoHtml@SliTeX{\HoLogoHtml@SliTeX@narrow}
%    \end{macrocode}
%    \end{macro}
%
% \subsubsection{\hologo{LuaTeX}}
%
%    \begin{macro}{\HoLogo@LuaTeX}
%    The kerning is an idea of Hans Hagen, see mailing list
%    `luatex at tug dot org' in March 2010.
%    \begin{macrocode}
\def\HoLogo@LuaTeX#1{%
  \HOLOGO@mbox{%
    Lua%
    \HOLOGO@NegativeKerning{aT,oT,To}%
    \hologo{TeX}%
  }%
}
%    \end{macrocode}
%    \end{macro}
%    \begin{macro}{\HoLogoHtml@LuaTeX}
%    \begin{macrocode}
\let\HoLogoHtml@LuaTeX\HoLogo@LuaTeX
%    \end{macrocode}
%    \end{macro}
%
% \subsubsection{\hologo{LuaLaTeX}}
%
%    \begin{macro}{\HoLogo@LuaLaTeX}
%    \begin{macrocode}
\def\HoLogo@LuaLaTeX#1{%
  \HOLOGO@mbox{%
    Lua%
    \hologo{LaTeX}%
  }%
}
%    \end{macrocode}
%    \end{macro}
%    \begin{macro}{\HoLogoHtml@LuaLaTeX}
%    \begin{macrocode}
\let\HoLogoHtml@LuaLaTeX\HoLogo@LuaLaTeX
%    \end{macrocode}
%    \end{macro}
%
% \subsubsection{\hologo{XeTeX}, \hologo{XeLaTeX}}
%
%    \begin{macro}{\HOLOGO@IfCharExists}
%    \begin{macrocode}
\ifluatex
  \ifnum\luatexversion<36 %
  \else
    \def\HOLOGO@IfCharExists#1{%
      \ifnum
        \directlua{%
           if luaotfload and luaotfload.aux then
             if luaotfload.aux.font_has_glyph(%
                    font.current(), \number#1) then % 	 
	       tex.print("1") % 	 
	     end % 	 
	   elseif font and font.fonts and font.current then %
            local f = font.fonts[font.current()]%
            if f.characters and f.characters[\number#1] then %
              tex.print("1")%
            end %
          end%
        }0=\ltx@zero
        \expandafter\ltx@secondoftwo
      \else
        \expandafter\ltx@firstoftwo
      \fi
    }%
  \fi
\fi
\ltx@IfUndefined{HOLOGO@IfCharExists}{%
  \def\HOLOGO@@IfCharExists#1{%
    \begingroup
      \tracinglostchars=\ltx@zero
      \setbox\ltx@zero=\hbox{%
        \kern7sp\char#1\relax
        \ifnum\lastkern>\ltx@zero
          \expandafter\aftergroup\csname iffalse\endcsname
        \else
          \expandafter\aftergroup\csname iftrue\endcsname
        \fi
      }%
      % \if{true|false} from \aftergroup
      \endgroup
      \expandafter\ltx@firstoftwo
    \else
      \endgroup
      \expandafter\ltx@secondoftwo
    \fi
  }%
  \ifxetex
    \ltx@IfUndefined{XeTeXfonttype}{}{%
      \ltx@IfUndefined{XeTeXcharglyph}{}{%
        \def\HOLOGO@IfCharExists#1{%
          \ifnum\XeTeXfonttype\font>\ltx@zero
            \expandafter\ltx@firstofthree
          \else
            \expandafter\ltx@gobble
          \fi
          {%
            \ifnum\XeTeXcharglyph#1>\ltx@zero
              \expandafter\ltx@firstoftwo
            \else
              \expandafter\ltx@secondoftwo
            \fi
          }%
          \HOLOGO@@IfCharExists{#1}%
        }%
      }%
    }%
  \fi
}{}
\ltx@ifundefined{HOLOGO@IfCharExists}{%
  \ifnum64=`\^^^^0040\relax % test for big chars of LuaTeX/XeTeX
    \let\HOLOGO@IfCharExists\HOLOGO@@IfCharExists
  \else
    \def\HOLOGO@IfCharExists#1{%
      \ifnum#1>255 %
        \expandafter\ltx@fourthoffour
      \fi
      \HOLOGO@@IfCharExists{#1}%
    }%
  \fi
}{}
%    \end{macrocode}
%    \end{macro}
%
%    \begin{macro}{\HoLogo@Xe}
%    Source: package \xpackage{dtklogos}
%    \begin{macrocode}
\def\HoLogo@Xe#1{%
  X%
  \kern-.1em\relax
  \HOLOGO@IfCharExists{"018E}{%
    \lower.5ex\hbox{\char"018E}%
  }{%
    \chardef\HOLOGO@choice=\ltx@zero
    \ifdim\fontdimen\ltx@one\font>0pt %
      \ltx@IfUndefined{rotatebox}{%
        \ltx@IfUndefined{pgftext}{%
          \ltx@IfUndefined{psscalebox}{%
            \ltx@IfUndefined{HOLOGO@ScaleBox@\hologoDriver}{%
            }{%
              \chardef\HOLOGO@choice=4 %
            }%
          }{%
            \chardef\HOLOGO@choice=3 %
          }%
        }{%
          \chardef\HOLOGO@choice=2 %
        }%
      }{%
        \chardef\HOLOGO@choice=1 %
      }%
      \ifcase\HOLOGO@choice
        \HOLOGO@WarningUnsupportedDriver{Xe}%
        e%
      \or % 1: \rotatebox
        \begingroup
          \setbox\ltx@zero\hbox{\rotatebox{180}{E}}%
          \ltx@LocDimenA=\dp\ltx@zero
          \advance\ltx@LocDimenA by -.5ex\relax
          \raise\ltx@LocDimenA\box\ltx@zero
        \endgroup
      \or % 2: \pgftext
        \lower.5ex\hbox{%
          \pgfpicture
            \pgftext[rotate=180]{E}%
          \endpgfpicture
        }%
      \or % 3: \psscalebox
        \begingroup
          \setbox\ltx@zero\hbox{\psscalebox{-1 -1}{E}}%
          \ltx@LocDimenA=\dp\ltx@zero
          \advance\ltx@LocDimenA by -.5ex\relax
          \raise\ltx@LocDimenA\box\ltx@zero
        \endgroup
      \or % 4: \HOLOGO@PointReflectBox
        \lower.5ex\hbox{\HOLOGO@PointReflectBox{E}}%
      \else
        \@PackageError{hologo}{Internal error (choice/it}\@ehc
      \fi
    \else
      \ltx@IfUndefined{reflectbox}{%
        \ltx@IfUndefined{pgftext}{%
          \ltx@IfUndefined{psscalebox}{%
            \ltx@IfUndefined{HOLOGO@ScaleBox@\hologoDriver}{%
            }{%
              \chardef\HOLOGO@choice=4 %
            }%
          }{%
            \chardef\HOLOGO@choice=3 %
          }%
        }{%
          \chardef\HOLOGO@choice=2 %
        }%
      }{%
        \chardef\HOLOGO@choice=1 %
      }%
      \ifcase\HOLOGO@choice
        \HOLOGO@WarningUnsupportedDriver{Xe}%
        e%
      \or % 1: reflectbox
        \lower.5ex\hbox{%
          \reflectbox{E}%
        }%
      \or % 2: \pgftext
        \lower.5ex\hbox{%
          \pgfpicture
            \pgftransformxscale{-1}%
            \pgftext{E}%
          \endpgfpicture
        }%
      \or % 3: \psscalebox
        \lower.5ex\hbox{%
          \psscalebox{-1 1}{E}%
        }%
      \or % 4: \HOLOGO@Reflectbox
        \lower.5ex\hbox{%
          \HOLOGO@ReflectBox{E}%
        }%
      \else
        \@PackageError{hologo}{Internal error (choice/up)}\@ehc
      \fi
    \fi
  }%
}
%    \end{macrocode}
%    \end{macro}
%    \begin{macro}{\HoLogoHtml@Xe}
%    \begin{macrocode}
\def\HoLogoHtml@Xe#1{%
  \HoLogoCss@Xe
  \HOLOGO@Span{Xe}{%
    X%
    \HOLOGO@Span{e}{%
      \HCode{&\ltx@hashchar x018e;}%
    }%
  }%
}
%    \end{macrocode}
%    \end{macro}
%    \begin{macro}{\HoLogoCss@Xe}
%    \begin{macrocode}
\def\HoLogoCss@Xe{%
  \Css{%
    span.HoLogo-Xe span.HoLogo-e{%
      position:relative;%
      top:.5ex;%
      left-margin:-.1em;%
    }%
  }%
  \global\let\HoLogoCss@Xe\relax
}
%    \end{macrocode}
%    \end{macro}
%
%    \begin{macro}{\HoLogo@XeTeX}
%    \begin{macrocode}
\def\HoLogo@XeTeX#1{%
  \hologo{Xe}%
  \kern-.15em\relax
  \hologo{TeX}%
}
%    \end{macrocode}
%    \end{macro}
%
%    \begin{macro}{\HoLogoHtml@XeTeX}
%    \begin{macrocode}
\def\HoLogoHtml@XeTeX#1{%
  \HoLogoCss@XeTeX
  \HOLOGO@Span{XeTeX}{%
    \hologo{Xe}%
    \hologo{TeX}%
  }%
}
%    \end{macrocode}
%    \end{macro}
%    \begin{macro}{\HoLogoCss@XeTeX}
%    \begin{macrocode}
\def\HoLogoCss@XeTeX{%
  \Css{%
    span.HoLogo-XeTeX span.HoLogo-TeX{%
      margin-left:-.15em;%
    }%
  }%
  \global\let\HoLogoCss@XeTeX\relax
}
%    \end{macrocode}
%    \end{macro}
%
%    \begin{macro}{\HoLogo@XeLaTeX}
%    \begin{macrocode}
\def\HoLogo@XeLaTeX#1{%
  \hologo{Xe}%
  \kern-.13em%
  \hologo{LaTeX}%
}
%    \end{macrocode}
%    \end{macro}
%    \begin{macro}{\HoLogoHtml@XeLaTeX}
%    \begin{macrocode}
\def\HoLogoHtml@XeLaTeX#1{%
  \HoLogoCss@XeLaTeX
  \HOLOGO@Span{XeLaTeX}{%
    \hologo{Xe}%
    \hologo{LaTeX}%
  }%
}
%    \end{macrocode}
%    \end{macro}
%    \begin{macro}{\HoLogoCss@XeLaTeX}
%    \begin{macrocode}
\def\HoLogoCss@XeLaTeX{%
  \Css{%
    span.HoLogo-XeLaTeX span.HoLogo-Xe{%
      margin-right:-.13em;%
    }%
  }%
  \global\let\HoLogoCss@XeLaTeX\relax
}
%    \end{macrocode}
%    \end{macro}
%
% \subsubsection{\hologo{pdfTeX}, \hologo{pdfLaTeX}}
%
%    \begin{macro}{\HoLogo@pdfTeX}
%    \begin{macrocode}
\def\HoLogo@pdfTeX#1{%
  \HOLOGO@mbox{%
    #1{p}{P}df\hologo{TeX}%
  }%
}
%    \end{macrocode}
%    \end{macro}
%    \begin{macro}{\HoLogoCs@pdfTeX}
%    \begin{macrocode}
\def\HoLogoCs@pdfTeX#1{#1{p}{P}dfTeX}
%    \end{macrocode}
%    \end{macro}
%    \begin{macro}{\HoLogoBkm@pdfTeX}
%    \begin{macrocode}
\def\HoLogoBkm@pdfTeX#1{%
  #1{p}{P}df\hologo{TeX}%
}
%    \end{macrocode}
%    \end{macro}
%    \begin{macro}{\HoLogoHtml@pdfTeX}
%    \begin{macrocode}
\let\HoLogoHtml@pdfTeX\HoLogo@pdfTeX
%    \end{macrocode}
%    \end{macro}
%
%    \begin{macro}{\HoLogo@pdfLaTeX}
%    \begin{macrocode}
\def\HoLogo@pdfLaTeX#1{%
  \HOLOGO@mbox{%
    #1{p}{P}df\hologo{LaTeX}%
  }%
}
%    \end{macrocode}
%    \end{macro}
%    \begin{macro}{\HoLogoCs@pdfLaTeX}
%    \begin{macrocode}
\def\HoLogoCs@pdfLaTeX#1{#1{p}{P}dfLaTeX}
%    \end{macrocode}
%    \end{macro}
%    \begin{macro}{\HoLogoBkm@pdfLaTeX}
%    \begin{macrocode}
\def\HoLogoBkm@pdfLaTeX#1{%
  #1{p}{P}df\hologo{LaTeX}%
}
%    \end{macrocode}
%    \end{macro}
%    \begin{macro}{\HoLogoHtml@pdfLaTeX}
%    \begin{macrocode}
\let\HoLogoHtml@pdfLaTeX\HoLogo@pdfLaTeX
%    \end{macrocode}
%    \end{macro}
%
% \subsubsection{\hologo{VTeX}}
%
%    \begin{macro}{\HoLogo@VTeX}
%    \begin{macrocode}
\def\HoLogo@VTeX#1{%
  \HOLOGO@mbox{%
    V\hologo{TeX}%
  }%
}
%    \end{macrocode}
%    \end{macro}
%    \begin{macro}{\HoLogoHtml@VTeX}
%    \begin{macrocode}
\let\HoLogoHtml@VTeX\HoLogo@VTeX
%    \end{macrocode}
%    \end{macro}
%
% \subsubsection{\hologo{AmS}, \dots}
%
%    Source: class \xclass{amsdtx}
%
%    \begin{macro}{\HoLogo@AmS}
%    \begin{macrocode}
\def\HoLogo@AmS#1{%
  \HoLogoFont@font{AmS}{sy}{%
    A%
    \kern-.1667em%
    \lower.5ex\hbox{M}%
    \kern-.125em%
    S%
  }%
}
%    \end{macrocode}
%    \end{macro}
%    \begin{macro}{\HoLogoBkm@AmS}
%    \begin{macrocode}
\def\HoLogoBkm@AmS#1{AmS}
%    \end{macrocode}
%    \end{macro}
%    \begin{macro}{\HoLogoHtml@AmS}
%    \begin{macrocode}
\def\HoLogoHtml@AmS#1{%
  \HoLogoCss@AmS
%  \HoLogoFont@font{AmS}{sy}{%
    \HOLOGO@Span{AmS}{%
      A%
      \HOLOGO@Span{M}{M}%
      S%
    }%
%   }%
}
%    \end{macrocode}
%    \end{macro}
%    \begin{macro}{\HoLogoCss@AmS}
%    \begin{macrocode}
\def\HoLogoCss@AmS{%
  \Css{%
    span.HoLogo-AmS span.HoLogo-M{%
      position:relative;%
      top:.5ex;%
      margin-left:-.1667em;%
      margin-right:-.125em;%
      text-decoration:none;%
    }%
  }%
  \global\let\HoLogoCss@AmS\relax
}
%    \end{macrocode}
%    \end{macro}
%
%    \begin{macro}{\HoLogo@AmSTeX}
%    \begin{macrocode}
\def\HoLogo@AmSTeX#1{%
  \hologo{AmS}%
  \HOLOGO@hyphen
  \hologo{TeX}%
}
%    \end{macrocode}
%    \end{macro}
%    \begin{macro}{\HoLogoBkm@AmSTeX}
%    \begin{macrocode}
\def\HoLogoBkm@AmSTeX#1{AmS-TeX}%
%    \end{macrocode}
%    \end{macro}
%    \begin{macro}{\HoLogoHtml@AmSTeX}
%    \begin{macrocode}
\let\HoLogoHtml@AmSTeX\HoLogo@AmSTeX
%    \end{macrocode}
%    \end{macro}
%
%    \begin{macro}{\HoLogo@AmSLaTeX}
%    \begin{macrocode}
\def\HoLogo@AmSLaTeX#1{%
  \hologo{AmS}%
  \HOLOGO@hyphen
  \hologo{LaTeX}%
}
%    \end{macrocode}
%    \end{macro}
%    \begin{macro}{\HoLogoBkm@AmSLaTeX}
%    \begin{macrocode}
\def\HoLogoBkm@AmSLaTeX#1{AmS-LaTeX}%
%    \end{macrocode}
%    \end{macro}
%    \begin{macro}{\HoLogoHtml@AmSLaTeX}
%    \begin{macrocode}
\let\HoLogoHtml@AmSLaTeX\HoLogo@AmSLaTeX
%    \end{macrocode}
%    \end{macro}
%
% \subsubsection{\hologo{BibTeX}}
%
%    \begin{macro}{\HoLogo@BibTeX@sc}
%    A definition of \hologo{BibTeX} is provided in
%    the documentation source for the manual of \hologo{BibTeX}
%    \cite{btxdoc}.
%\begin{quote}
%\begin{verbatim}
%\def\BibTeX{%
%  {%
%    \rm
%    B%
%    \kern-.05em%
%    {%
%      \sc
%      i%
%      \kern-.025em %
%      b%
%    }%
%    \kern-.08em
%    T%
%    \kern-.1667em%
%    \lower.7ex\hbox{E}%
%    \kern-.125em%
%    X%
%  }%
%}
%\end{verbatim}
%\end{quote}
%    \begin{macrocode}
\def\HoLogo@BibTeX@sc#1{%
  B%
  \kern-.05em%
  \HoLogoFont@font{BibTeX}{sc}{%
    i%
    \kern-.025em%
    b%
  }%
  \HOLOGO@discretionary
  \kern-.08em%
  \hologo{TeX}%
}
%    \end{macrocode}
%    \end{macro}
%    \begin{macro}{\HoLogoHtml@BibTeX@sc}
%    \begin{macrocode}
\def\HoLogoHtml@BibTeX@sc#1{%
  \HoLogoCss@BibTeX@sc
  \HOLOGO@Span{BibTeX-sc}{%
    B%
    \HOLOGO@Span{i}{i}%
    \HOLOGO@Span{b}{b}%
    \hologo{TeX}%
  }%
}
%    \end{macrocode}
%    \end{macro}
%    \begin{macro}{\HoLogoCss@BibTeX@sc}
%    \begin{macrocode}
\def\HoLogoCss@BibTeX@sc{%
  \Css{%
    span.HoLogo-BibTeX-sc span.HoLogo-i{%
      margin-left:-.05em;%
      margin-right:-.025em;%
      font-variant:small-caps;%
    }%
  }%
  \Css{%
    span.HoLogo-BibTeX-sc span.HoLogo-b{%
      margin-right:-.08em;%
      font-variant:small-caps;%
    }%
  }%
  \global\let\HoLogoCss@BibTeX@sc\relax
}
%    \end{macrocode}
%    \end{macro}
%
%    \begin{macro}{\HoLogo@BibTeX@sf}
%    Variant \xoption{sf} avoids trouble with unavailable
%    small caps fonts (e.g., bold versions of Computer Modern or
%    Latin Modern). The definition is taken from
%    package \xpackage{dtklogos} \cite{dtklogos}.
%\begin{quote}
%\begin{verbatim}
%\DeclareRobustCommand{\BibTeX}{%
%  B%
%  \kern-.05em%
%  \hbox{%
%    $\m@th$% %% force math size calculations
%    \csname S@\f@size\endcsname
%    \fontsize\sf@size\z@
%    \math@fontsfalse
%    \selectfont
%    I%
%    \kern-.025em%
%    B
%  }%
%  \kern-.08em%
%  \-%
%  \TeX
%}
%\end{verbatim}
%\end{quote}
%    \begin{macrocode}
\def\HoLogo@BibTeX@sf#1{%
  B%
  \kern-.05em%
  \HoLogoFont@font{BibTeX}{bibsf}{%
    I%
    \kern-.025em%
    B%
  }%
  \HOLOGO@discretionary
  \kern-.08em%
  \hologo{TeX}%
}
%    \end{macrocode}
%    \end{macro}
%    \begin{macro}{\HoLogoHtml@BibTeX@sf}
%    \begin{macrocode}
\def\HoLogoHtml@BibTeX@sf#1{%
  \HoLogoCss@BibTeX@sf
  \HOLOGO@Span{BibTeX-sf}{%
    B%
    \HoLogoFont@font{BibTeX}{bibsf}{%
      \HOLOGO@Span{i}{I}%
      B%
    }%
    \hologo{TeX}%
  }%
}
%    \end{macrocode}
%    \end{macro}
%    \begin{macro}{\HoLogoCss@BibTeX@sf}
%    \begin{macrocode}
\def\HoLogoCss@BibTeX@sf{%
  \Css{%
    span.HoLogo-BibTeX-sf span.HoLogo-i{%
      margin-left:-.05em;%
      margin-right:-.025em;%
    }%
  }%
  \Css{%
    span.HoLogo-BibTeX-sf span.HoLogo-TeX{%
      margin-left:-.08em;%
    }%
  }%
  \global\let\HoLogoCss@BibTeX@sf\relax
}
%    \end{macrocode}
%    \end{macro}
%
%    \begin{macro}{\HoLogo@BibTeX}
%    \begin{macrocode}
\def\HoLogo@BibTeX{\HoLogo@BibTeX@sf}
%    \end{macrocode}
%    \end{macro}
%    \begin{macro}{\HoLogoHtml@BibTeX}
%    \begin{macrocode}
\def\HoLogoHtml@BibTeX{\HoLogoHtml@BibTeX@sf}
%    \end{macrocode}
%    \end{macro}
%
% \subsubsection{\hologo{BibTeX8}}
%
%    \begin{macro}{\HoLogo@BibTeX8}
%    \begin{macrocode}
\expandafter\def\csname HoLogo@BibTeX8\endcsname#1{%
  \hologo{BibTeX}%
  8%
}
%    \end{macrocode}
%    \end{macro}
%
%    \begin{macro}{\HoLogoBkm@BibTeX8}
%    \begin{macrocode}
\expandafter\def\csname HoLogoBkm@BibTeX8\endcsname#1{%
  \hologo{BibTeX}%
  8%
}
%    \end{macrocode}
%    \end{macro}
%    \begin{macro}{\HoLogoHtml@BibTeX8}
%    \begin{macrocode}
\expandafter
\let\csname HoLogoHtml@BibTeX8\expandafter\endcsname
\csname HoLogo@BibTeX8\endcsname
%    \end{macrocode}
%    \end{macro}
%
% \subsubsection{\hologo{ConTeXt}}
%
%    \begin{macro}{\HoLogo@ConTeXt@simple}
%    \begin{macrocode}
\def\HoLogo@ConTeXt@simple#1{%
  \HOLOGO@mbox{Con}%
  \HOLOGO@discretionary
  \HOLOGO@mbox{\hologo{TeX}t}%
}
%    \end{macrocode}
%    \end{macro}
%    \begin{macro}{\HoLogoHtml@ConTeXt@simple}
%    \begin{macrocode}
\let\HoLogoHtml@ConTeXt@simple\HoLogo@ConTeXt@simple
%    \end{macrocode}
%    \end{macro}
%
%    \begin{macro}{\HoLogo@ConTeXt@narrow}
%    This definition of logo \hologo{ConTeXt} with variant \xoption{narrow}
%    comes from TUGboat's class \xclass{ltugboat} (version 2010/11/15 v2.8).
%    \begin{macrocode}
\def\HoLogo@ConTeXt@narrow#1{%
  \HOLOGO@mbox{C\kern-.0333emon}%
  \HOLOGO@discretionary
  \kern-.0667em%
  \HOLOGO@mbox{\hologo{TeX}\kern-.0333emt}%
}
%    \end{macrocode}
%    \end{macro}
%    \begin{macro}{\HoLogoHtml@ConTeXt@narrow}
%    \begin{macrocode}
\def\HoLogoHtml@ConTeXt@narrow#1{%
  \HoLogoCss@ConTeXt@narrow
  \HOLOGO@Span{ConTeXt-narrow}{%
    \HOLOGO@Span{C}{C}%
    on%
    \hologo{TeX}%
    t%
  }%
}
%    \end{macrocode}
%    \end{macro}
%    \begin{macro}{\HoLogoCss@ConTeXt@narrow}
%    \begin{macrocode}
\def\HoLogoCss@ConTeXt@narrow{%
  \Css{%
    span.HoLogo-ConTeXt-narrow span.HoLogo-C{%
      margin-left:-.0333em;%
    }%
  }%
  \Css{%
    span.HoLogo-ConTeXt-narrow span.HoLogo-TeX{%
      margin-left:-.0667em;%
      margin-right:-.0333em;%
    }%
  }%
  \global\let\HoLogoCss@ConTeXt@narrow\relax
}
%    \end{macrocode}
%    \end{macro}
%
%    \begin{macro}{\HoLogo@ConTeXt}
%    \begin{macrocode}
\def\HoLogo@ConTeXt{\HoLogo@ConTeXt@narrow}
%    \end{macrocode}
%    \end{macro}
%    \begin{macro}{\HoLogoHtml@ConTeXt}
%    \begin{macrocode}
\def\HoLogoHtml@ConTeXt{\HoLogoHtml@ConTeXt@narrow}
%    \end{macrocode}
%    \end{macro}
%
% \subsubsection{\hologo{emTeX}}
%
%    \begin{macro}{\HoLogo@emTeX}
%    \begin{macrocode}
\def\HoLogo@emTeX#1{%
  \HOLOGO@mbox{#1{e}{E}m}%
  \HOLOGO@discretionary
  \hologo{TeX}%
}
%    \end{macrocode}
%    \end{macro}
%    \begin{macro}{\HoLogoCs@emTeX}
%    \begin{macrocode}
\def\HoLogoCs@emTeX#1{#1{e}{E}mTeX}%
%    \end{macrocode}
%    \end{macro}
%    \begin{macro}{\HoLogoBkm@emTeX}
%    \begin{macrocode}
\def\HoLogoBkm@emTeX#1{%
  #1{e}{E}m\hologo{TeX}%
}
%    \end{macrocode}
%    \end{macro}
%    \begin{macro}{\HoLogoHtml@emTeX}
%    \begin{macrocode}
\let\HoLogoHtml@emTeX\HoLogo@emTeX
%    \end{macrocode}
%    \end{macro}
%
% \subsubsection{\hologo{ExTeX}}
%
%    \begin{macro}{\HoLogo@ExTeX}
%    The definition is taken from the FAQ of the
%    project \hologo{ExTeX}
%    \cite{ExTeX-FAQ}.
%\begin{quote}
%\begin{verbatim}
%\def\ExTeX{%
%  \textrm{% Logo always with serifs
%    \ensuremath{%
%      \textstyle
%      \varepsilon_{%
%        \kern-0.15em%
%        \mathcal{X}%
%      }%
%    }%
%    \kern-.15em%
%    \TeX
%  }%
%}
%\end{verbatim}
%\end{quote}
%    \begin{macrocode}
\def\HoLogo@ExTeX#1{%
  \HoLogoFont@font{ExTeX}{rm}{%
    \ltx@mbox{%
      \HOLOGO@MathSetup
      $%
        \textstyle
        \varepsilon_{%
          \kern-0.15em%
          \HoLogoFont@font{ExTeX}{sy}{X}%
        }%
      $%
    }%
    \HOLOGO@discretionary
    \kern-.15em%
    \hologo{TeX}%
  }%
}
%    \end{macrocode}
%    \end{macro}
%    \begin{macro}{\HoLogoHtml@ExTeX}
%    \begin{macrocode}
\def\HoLogoHtml@ExTeX#1{%
  \HoLogoCss@ExTeX
  \HoLogoFont@font{ExTeX}{rm}{%
    \HOLOGO@Span{ExTeX}{%
      \ltx@mbox{%
        \HOLOGO@MathSetup
        $\textstyle\varepsilon$%
        \HOLOGO@Span{X}{$\textstyle\chi$}%
        \hologo{TeX}%
      }%
    }%
  }%
}
%    \end{macrocode}
%    \end{macro}
%    \begin{macro}{\HoLogoBkm@ExTeX}
%    \begin{macrocode}
\def\HoLogoBkm@ExTeX#1{%
  \HOLOGO@PdfdocUnicode{#1{e}{E}x}{\textepsilon\textchi}%
  \hologo{TeX}%
}
%    \end{macrocode}
%    \end{macro}
%    \begin{macro}{\HoLogoCss@ExTeX}
%    \begin{macrocode}
\def\HoLogoCss@ExTeX{%
  \Css{%
    span.HoLogo-ExTeX{%
      font-family:serif;%
    }%
  }%
  \Css{%
    span.HoLogo-ExTeX span.HoLogo-TeX{%
      margin-left:-.15em;%
    }%
  }%
  \global\let\HoLogoCss@ExTeX\relax
}
%    \end{macrocode}
%    \end{macro}
%
% \subsubsection{\hologo{MiKTeX}}
%
%    \begin{macro}{\HoLogo@MiKTeX}
%    \begin{macrocode}
\def\HoLogo@MiKTeX#1{%
  \HOLOGO@mbox{MiK}%
  \HOLOGO@discretionary
  \hologo{TeX}%
}
%    \end{macrocode}
%    \end{macro}
%    \begin{macro}{\HoLogoHtml@MiKTeX}
%    \begin{macrocode}
\let\HoLogoHtml@MiKTeX\HoLogo@MiKTeX
%    \end{macrocode}
%    \end{macro}
%
% \subsubsection{\hologo{OzTeX} and friends}
%
%    Source: \hologo{OzTeX} FAQ \cite{OzTeX}:
%    \begin{quote}
%      |\def\OzTeX{O\kern-.03em z\kern-.15em\TeX}|\\
%      (There is no kerning in OzMF, OzMP and OzTtH.)
%    \end{quote}
%
%    \begin{macro}{\HoLogo@OzTeX}
%    \begin{macrocode}
\def\HoLogo@OzTeX#1{%
  O%
  \kern-.03em %
  z%
  \kern-.15em %
  \hologo{TeX}%
}
%    \end{macrocode}
%    \end{macro}
%    \begin{macro}{\HoLogoHtml@OzTeX}
%    \begin{macrocode}
\def\HoLogoHtml@OzTeX#1{%
  \HoLogoCss@OzTeX
  \HOLOGO@Span{OzTeX}{%
    O%
    \HOLOGO@Span{z}{z}%
    \hologo{TeX}%
  }%
}
%    \end{macrocode}
%    \end{macro}
%    \begin{macro}{\HoLogoCss@OzTeX}
%    \begin{macrocode}
\def\HoLogoCss@OzTeX{%
  \Css{%
    span.HoLogo-OzTeX span.HoLogo-z{%
      margin-left:-.03em;%
      margin-right:-.15em;%
    }%
  }%
  \global\let\HoLogoCss@OzTeX\relax
}
%    \end{macrocode}
%    \end{macro}
%
%    \begin{macro}{\HoLogo@OzMF}
%    \begin{macrocode}
\def\HoLogo@OzMF#1{%
  \HOLOGO@mbox{OzMF}%
}
%    \end{macrocode}
%    \end{macro}
%    \begin{macro}{\HoLogo@OzMP}
%    \begin{macrocode}
\def\HoLogo@OzMP#1{%
  \HOLOGO@mbox{OzMP}%
}
%    \end{macrocode}
%    \end{macro}
%    \begin{macro}{\HoLogo@OzTtH}
%    \begin{macrocode}
\def\HoLogo@OzTtH#1{%
  \HOLOGO@mbox{OzTtH}%
}
%    \end{macrocode}
%    \end{macro}
%
% \subsubsection{\hologo{PCTeX}}
%
%    \begin{macro}{\HoLogo@PCTeX}
%    \begin{macrocode}
\def\HoLogo@PCTeX#1{%
  \HOLOGO@mbox{PC}%
  \hologo{TeX}%
}
%    \end{macrocode}
%    \end{macro}
%    \begin{macro}{\HoLogoHtml@PCTeX}
%    \begin{macrocode}
\let\HoLogoHtml@PCTeX\HoLogo@PCTeX
%    \end{macrocode}
%    \end{macro}
%
% \subsubsection{\hologo{PiCTeX}}
%
%    The original definitions from \xfile{pictex.tex} \cite{PiCTeX}:
%\begin{quote}
%\begin{verbatim}
%\def\PiC{%
%  P%
%  \kern-.12em%
%  \lower.5ex\hbox{I}%
%  \kern-.075em%
%  C%
%}
%\def\PiCTeX{%
%  \PiC
%  \kern-.11em%
%  \TeX
%}
%\end{verbatim}
%\end{quote}
%
%    \begin{macro}{\HoLogo@PiC}
%    \begin{macrocode}
\def\HoLogo@PiC#1{%
  P%
  \kern-.12em%
  \lower.5ex\hbox{I}%
  \kern-.075em%
  C%
  \HOLOGO@SpaceFactor
}
%    \end{macrocode}
%    \end{macro}
%    \begin{macro}{\HoLogoHtml@PiC}
%    \begin{macrocode}
\def\HoLogoHtml@PiC#1{%
  \HoLogoCss@PiC
  \HOLOGO@Span{PiC}{%
    P%
    \HOLOGO@Span{i}{I}%
    C%
  }%
}
%    \end{macrocode}
%    \end{macro}
%    \begin{macro}{\HoLogoCss@PiC}
%    \begin{macrocode}
\def\HoLogoCss@PiC{%
  \Css{%
    span.HoLogo-PiC span.HoLogo-i{%
      position:relative;%
      top:.5ex;%
      margin-left:-.12em;%
      margin-right:-.075em;%
      text-decoration:none;%
    }%
  }%
  \global\let\HoLogoCss@PiC\relax
}
%    \end{macrocode}
%    \end{macro}
%
%    \begin{macro}{\HoLogo@PiCTeX}
%    \begin{macrocode}
\def\HoLogo@PiCTeX#1{%
  \hologo{PiC}%
  \HOLOGO@discretionary
  \kern-.11em%
  \hologo{TeX}%
}
%    \end{macrocode}
%    \end{macro}
%    \begin{macro}{\HoLogoHtml@PiCTeX}
%    \begin{macrocode}
\def\HoLogoHtml@PiCTeX#1{%
  \HoLogoCss@PiCTeX
  \HOLOGO@Span{PiCTeX}{%
    \hologo{PiC}%
    \hologo{TeX}%
  }%
}
%    \end{macrocode}
%    \end{macro}
%    \begin{macro}{\HoLogoCss@PiCTeX}
%    \begin{macrocode}
\def\HoLogoCss@PiCTeX{%
  \Css{%
    span.HoLogo-PiCTeX span.HoLogo-PiC{%
      margin-right:-.11em;%
    }%
  }%
  \global\let\HoLogoCss@PiCTeX\relax
}
%    \end{macrocode}
%    \end{macro}
%
% \subsubsection{\hologo{teTeX}}
%
%    \begin{macro}{\HoLogo@teTeX}
%    \begin{macrocode}
\def\HoLogo@teTeX#1{%
  \HOLOGO@mbox{#1{t}{T}e}%
  \HOLOGO@discretionary
  \hologo{TeX}%
}
%    \end{macrocode}
%    \end{macro}
%    \begin{macro}{\HoLogoCs@teTeX}
%    \begin{macrocode}
\def\HoLogoCs@teTeX#1{#1{t}{T}dfTeX}
%    \end{macrocode}
%    \end{macro}
%    \begin{macro}{\HoLogoBkm@teTeX}
%    \begin{macrocode}
\def\HoLogoBkm@teTeX#1{%
  #1{t}{T}e\hologo{TeX}%
}
%    \end{macrocode}
%    \end{macro}
%    \begin{macro}{\HoLogoHtml@teTeX}
%    \begin{macrocode}
\let\HoLogoHtml@teTeX\HoLogo@teTeX
%    \end{macrocode}
%    \end{macro}
%
% \subsubsection{\hologo{TeX4ht}}
%
%    \begin{macro}{\HoLogo@TeX4ht}
%    \begin{macrocode}
\expandafter\def\csname HoLogo@TeX4ht\endcsname#1{%
  \HOLOGO@mbox{\hologo{TeX}4ht}%
}
%    \end{macrocode}
%    \end{macro}
%    \begin{macro}{\HoLogoHtml@TeX4ht}
%    \begin{macrocode}
\expandafter
\let\csname HoLogoHtml@TeX4ht\expandafter\endcsname
\csname HoLogo@TeX4ht\endcsname
%    \end{macrocode}
%    \end{macro}
%
%
% \subsubsection{\hologo{SageTeX}}
%
%    \begin{macro}{\HoLogo@SageTeX}
%    \begin{macrocode}
\def\HoLogo@SageTeX#1{%
  \HOLOGO@mbox{Sage}%
  \HOLOGO@discretionary
  \HOLOGO@NegativeKerning{eT,oT,To}%
  \hologo{TeX}%
}
%    \end{macrocode}
%    \end{macro}
%    \begin{macro}{\HoLogoHtml@SageTeX}
%    \begin{macrocode}
\let\HoLogoHtml@SageTeX\HoLogo@SageTeX
%    \end{macrocode}
%    \end{macro}
%
% \subsection{\hologo{METAFONT} and friends}
%
%    \begin{macro}{\HoLogo@METAFONT}
%    \begin{macrocode}
\def\HoLogo@METAFONT#1{%
  \HoLogoFont@font{METAFONT}{logo}{%
    \HOLOGO@mbox{META}%
    \HOLOGO@discretionary
    \HOLOGO@mbox{FONT}%
  }%
}
%    \end{macrocode}
%    \end{macro}
%
%    \begin{macro}{\HoLogo@METAPOST}
%    \begin{macrocode}
\def\HoLogo@METAPOST#1{%
  \HoLogoFont@font{METAPOST}{logo}{%
    \HOLOGO@mbox{META}%
    \HOLOGO@discretionary
    \HOLOGO@mbox{POST}%
  }%
}
%    \end{macrocode}
%    \end{macro}
%
%    \begin{macro}{\HoLogo@MetaFun}
%    \begin{macrocode}
\def\HoLogo@MetaFun#1{%
  \HOLOGO@mbox{Meta}%
  \HOLOGO@discretionary
  \HOLOGO@mbox{Fun}%
}
%    \end{macrocode}
%    \end{macro}
%
%    \begin{macro}{\HoLogo@MetaPost}
%    \begin{macrocode}
\def\HoLogo@MetaPost#1{%
  \HOLOGO@mbox{Meta}%
  \HOLOGO@discretionary
  \HOLOGO@mbox{Post}%
}
%    \end{macrocode}
%    \end{macro}
%
% \subsection{Others}
%
% \subsubsection{\hologo{biber}}
%
%    \begin{macro}{\HoLogo@biber}
%    \begin{macrocode}
\def\HoLogo@biber#1{%
  \HOLOGO@mbox{#1{b}{B}i}%
  \HOLOGO@discretionary
  \HOLOGO@mbox{ber}%
}
%    \end{macrocode}
%    \end{macro}
%    \begin{macro}{\HoLogoCs@biber}
%    \begin{macrocode}
\def\HoLogoCs@biber#1{#1{b}{B}iber}
%    \end{macrocode}
%    \end{macro}
%    \begin{macro}{\HoLogoBkm@biber}
%    \begin{macrocode}
\def\HoLogoBkm@biber#1{%
  #1{b}{B}iber%
}
%    \end{macrocode}
%    \end{macro}
%    \begin{macro}{\HoLogoHtml@biber}
%    \begin{macrocode}
\let\HoLogoHtml@biber\HoLogo@biber
%    \end{macrocode}
%    \end{macro}
%
% \subsubsection{\hologo{KOMAScript}}
%
%    \begin{macro}{\HoLogo@KOMAScript}
%    The definition for \hologo{KOMAScript} is taken
%    from \hologo{KOMAScript} (\xfile{scrlogo.dtx}, reformatted) \cite{scrlogo}:
%\begin{quote}
%\begin{verbatim}
%\@ifundefined{KOMAScript}{%
%  \DeclareRobustCommand{\KOMAScript}{%
%    \textsf{%
%      K\kern.05em O\kern.05emM\kern.05em A%
%      \kern.1em-\kern.1em %
%      Script%
%    }%
%  }%
%}{}
%\end{verbatim}
%\end{quote}
%    \begin{macrocode}
\def\HoLogo@KOMAScript#1{%
  \HoLogoFont@font{KOMAScript}{sf}{%
    \HOLOGO@mbox{%
      K\kern.05em%
      O\kern.05em%
      M\kern.05em%
      A%
    }%
    \kern.1em%
    \HOLOGO@hyphen
    \kern.1em%
    \HOLOGO@mbox{Script}%
  }%
}
%    \end{macrocode}
%    \end{macro}
%    \begin{macro}{\HoLogoBkm@KOMAScript}
%    \begin{macrocode}
\def\HoLogoBkm@KOMAScript#1{%
  KOMA-Script%
}
%    \end{macrocode}
%    \end{macro}
%    \begin{macro}{\HoLogoHtml@KOMAScript}
%    \begin{macrocode}
\def\HoLogoHtml@KOMAScript#1{%
  \HoLogoCss@KOMAScript
  \HoLogoFont@font{KOMAScript}{sf}{%
    \HOLOGO@Span{KOMAScript}{%
      K%
      \HOLOGO@Span{O}{O}%
      M%
      \HOLOGO@Span{A}{A}%
      \HOLOGO@Span{hyphen}{-}%
      Script%
    }%
  }%
}
%    \end{macrocode}
%    \end{macro}
%    \begin{macro}{\HoLogoCss@KOMAScript}
%    \begin{macrocode}
\def\HoLogoCss@KOMAScript{%
  \Css{%
    span.HoLogo-KOMAScript{%
      font-family:sans-serif;%
    }%
  }%
  \Css{%
    span.HoLogo-KOMAScript span.HoLogo-O{%
      padding-left:.05em;%
      padding-right:.05em;%
    }%
  }%
  \Css{%
    span.HoLogo-KOMAScript span.HoLogo-A{%
      padding-left:.05em;%
    }%
  }%
  \Css{%
    span.HoLogo-KOMAScript span.HoLogo-hyphen{%
      padding-left:.1em;%
      padding-right:.1em;%
    }%
  }%
  \global\let\HoLogoCss@KOMAScript\relax
}
%    \end{macrocode}
%    \end{macro}
%
% \subsubsection{\hologo{LyX}}
%
%    \begin{macro}{\HoLogo@LyX}
%    The definition is taken from the documentation source files
%    of \hologo{LyX}, \xfile{Intro.lyx} \cite{LyX}:
%\begin{quote}
%\begin{verbatim}
%\def\LyX{%
%  \texorpdfstring{%
%    L\kern-.1667em\lower.25em\hbox{Y}\kern-.125emX\@%
%  }{%
%    LyX%
%  }%
%}
%\end{verbatim}
%\end{quote}
%    \begin{macrocode}
\def\HoLogo@LyX#1{%
  L%
  \kern-.1667em%
  \lower.25em\hbox{Y}%
  \kern-.125em%
  X%
  \HOLOGO@SpaceFactor
}
%    \end{macrocode}
%    \end{macro}
%    \begin{macro}{\HoLogoHtml@LyX}
%    \begin{macrocode}
\def\HoLogoHtml@LyX#1{%
  \HoLogoCss@LyX
  \HOLOGO@Span{LyX}{%
    L%
    \HOLOGO@Span{y}{Y}%
    X%
  }%
}
%    \end{macrocode}
%    \end{macro}
%    \begin{macro}{\HoLogoCss@LyX}
%    \begin{macrocode}
\def\HoLogoCss@LyX{%
  \Css{%
    span.HoLogo-LyX span.HoLogo-y{%
      position:relative;%
      top:.25em;%
      margin-left:-.1667em;%
      margin-right:-.125em;%
      text-decoration:none;%
    }%
  }%
  \global\let\HoLogoCss@LyX\relax
}
%    \end{macrocode}
%    \end{macro}
%
% \subsubsection{\hologo{NTS}}
%
%    \begin{macro}{\HoLogo@NTS}
%    Definition for \hologo{NTS} can be found in
%    package \xpackage{etex\textunderscore man} for the \hologo{eTeX} manual \cite{etexman}
%    and in package \xpackage{dtklogos} \cite{dtklogos}:
%\begin{quote}
%\begin{verbatim}
%\def\NTS{%
%  \leavevmode
%  \hbox{%
%    $%
%      \cal N%
%      \kern-0.35em%
%      \lower0.5ex\hbox{$\cal T$}%
%      \kern-0.2em%
%      S%
%    $%
%  }%
%}
%\end{verbatim}
%\end{quote}
%    \begin{macrocode}
\def\HoLogo@NTS#1{%
  \HoLogoFont@font{NTS}{sy}{%
    N\/%
    \kern-.35em%
    \lower.5ex\hbox{T\/}%
    \kern-.2em%
    S\/%
  }%
  \HOLOGO@SpaceFactor
}
%    \end{macrocode}
%    \end{macro}
%
% \subsubsection{\Hologo{TTH} (\hologo{TeX} to HTML translator)}
%
%    Source: \url{http://hutchinson.belmont.ma.us/tth/}
%    In the HTML source the second `T' is printed as subscript.
%\begin{quote}
%\begin{verbatim}
%T<sub>T</sub>H
%\end{verbatim}
%\end{quote}
%    \begin{macro}{\HoLogo@TTH}
%    \begin{macrocode}
\def\HoLogo@TTH#1{%
  \ltx@mbox{%
    T\HOLOGO@SubScript{T}H%
  }%
  \HOLOGO@SpaceFactor
}
%    \end{macrocode}
%    \end{macro}
%
%    \begin{macro}{\HoLogoHtml@TTH}
%    \begin{macrocode}
\def\HoLogoHtml@TTH#1{%
  T\HCode{<sub>}T\HCode{</sub>}H%
}
%    \end{macrocode}
%    \end{macro}
%
% \subsubsection{\Hologo{HanTheThanh}}
%
%    Partial source: Package \xpackage{dtklogos}.
%    The double accent is U+1EBF (latin small letter e with circumflex
%    and acute).
%    \begin{macro}{\HoLogo@HanTheThanh}
%    \begin{macrocode}
\def\HoLogo@HanTheThanh#1{%
  \ltx@mbox{H\`an}%
  \HOLOGO@space
  \ltx@mbox{%
    Th%
    \HOLOGO@IfCharExists{"1EBF}{%
      \char"1EBF\relax
    }{%
      \^e\hbox to 0pt{\hss\raise .5ex\hbox{\'{}}}%
    }%
  }%
  \HOLOGO@space
  \ltx@mbox{Th\`anh}%
}
%    \end{macrocode}
%    \end{macro}
%    \begin{macro}{\HoLogoBkm@HanTheThanh}
%    \begin{macrocode}
\def\HoLogoBkm@HanTheThanh#1{%
  H\`an %
  Th\HOLOGO@PdfdocUnicode{\^e}{\9036\277} %
  Th\`anh%
}
%    \end{macrocode}
%    \end{macro}
%    \begin{macro}{\HoLogoHtml@HanTheThanh}
%    \begin{macrocode}
\def\HoLogoHtml@HanTheThanh#1{%
  H\`an %
  Th\HCode{&\ltx@hashchar x1ebf;} %
  Th\`anh%
}
%    \end{macrocode}
%    \end{macro}
%
% \subsection{Driver detection}
%
%    \begin{macrocode}
\HOLOGO@IfExists\InputIfFileExists{%
  \InputIfFileExists{hologo.cfg}{}{}%
}{%
  \ltx@IfUndefined{pdf@filesize}{%
    \def\HOLOGO@InputIfExists{%
      \openin\HOLOGO@temp=hologo.cfg\relax
      \ifeof\HOLOGO@temp
        \closein\HOLOGO@temp
      \else
        \closein\HOLOGO@temp
        \begingroup
          \def\x{LaTeX2e}%
        \expandafter\endgroup
        \ifx\fmtname\x
          \input{hologo.cfg}%
        \else
          \input hologo.cfg\relax
        \fi
      \fi
    }%
    \ltx@IfUndefined{newread}{%
      \chardef\HOLOGO@temp=15 %
      \def\HOLOGO@CheckRead{%
        \ifeof\HOLOGO@temp
          \HOLOGO@InputIfExists
        \else
          \ifcase\HOLOGO@temp
            \@PackageWarningNoLine{hologo}{%
              Configuration file ignored, because\MessageBreak
              a free read register could not be found%
            }%
          \else
            \begingroup
              \count\ltx@cclv=\HOLOGO@temp
              \advance\ltx@cclv by \ltx@minusone
              \edef\x{\endgroup
                \chardef\noexpand\HOLOGO@temp=\the\count\ltx@cclv
                \relax
              }%
            \x
          \fi
        \fi
      }%
    }{%
      \csname newread\endcsname\HOLOGO@temp
      \HOLOGO@InputIfExists
    }%
  }{%
    \edef\HOLOGO@temp{\pdf@filesize{hologo.cfg}}%
    \ifx\HOLOGO@temp\ltx@empty
    \else
      \ifnum\HOLOGO@temp>0 %
        \begingroup
          \def\x{LaTeX2e}%
        \expandafter\endgroup
        \ifx\fmtname\x
          \input{hologo.cfg}%
        \else
          \input hologo.cfg\relax
        \fi
      \else
        \@PackageInfoNoLine{hologo}{%
          Empty configuration file `hologo.cfg' ignored%
        }%
      \fi
    \fi
  }%
}
%    \end{macrocode}
%
%    \begin{macrocode}
\def\HOLOGO@temp#1#2{%
  \kv@define@key{HoLogoDriver}{#1}[]{%
    \begingroup
      \def\HOLOGO@temp{##1}%
      \ltx@onelevel@sanitize\HOLOGO@temp
      \ifx\HOLOGO@temp\ltx@empty
      \else
        \@PackageError{hologo}{%
          Value (\HOLOGO@temp) not permitted for option `#1'%
        }%
        \@ehc
      \fi
    \endgroup
    \def\hologoDriver{#2}%
  }%
}%
\def\HOLOGO@@temp#1#2{%
  \ifx\kv@value\relax
    \HOLOGO@temp{#1}{#1}%
  \else
    \HOLOGO@temp{#1}{#2}%
  \fi
}%
\kv@parse@normalized{%
  pdftex,%
  luatex=pdftex,%
  dvipdfm,%
  dvipdfmx=dvipdfm,%
  dvips,%
  dvipsone=dvips,%
  xdvi=dvips,%
  xetex,%
  vtex,%
}\HOLOGO@@temp
%    \end{macrocode}
%
%    \begin{macrocode}
\kv@define@key{HoLogoDriver}{driverfallback}{%
  \def\HOLOGO@DriverFallback{#1}%
}
%    \end{macrocode}
%
%    \begin{macro}{\HOLOGO@DriverFallback}
%    \begin{macrocode}
\def\HOLOGO@DriverFallback{dvips}
%    \end{macrocode}
%    \end{macro}
%
%    \begin{macro}{\hologoDriverSetup}
%    \begin{macrocode}
\def\hologoDriverSetup{%
  \let\hologoDriver\ltx@undefined
  \HOLOGO@DriverSetup
}
%    \end{macrocode}
%    \end{macro}
%
%    \begin{macro}{\HOLOGO@DriverSetup}
%    \begin{macrocode}
\def\HOLOGO@DriverSetup#1{%
  \kvsetkeys{HoLogoDriver}{#1}%
  \HOLOGO@CheckDriver
  \ltx@ifundefined{hologoDriver}{%
    \begingroup
    \edef\x{\endgroup
      \noexpand\kvsetkeys{HoLogoDriver}{\HOLOGO@DriverFallback}%
    }\x
  }{}%
  \@PackageInfoNoLine{hologo}{Using driver `\hologoDriver'}%
}
%    \end{macrocode}
%    \end{macro}
%
%    \begin{macro}{\HOLOGO@CheckDriver}
%    \begin{macrocode}
\def\HOLOGO@CheckDriver{%
  \ifpdf
    \def\hologoDriver{pdftex}%
    \let\HOLOGO@pdfliteral\pdfliteral
    \ifluatex
      \ifx\pdfextension\@undefined\else
        \protected\def\pdfliteral{\pdfextension literal}%
        \let\HOLOGO@pdfliteral\pdfliteral
      \fi
      \ltx@IfUndefined{HOLOGO@pdfliteral}{%
        \ifnum\luatexversion<36 %
        \else
          \begingroup
            \let\HOLOGO@temp\endgroup
            \ifcase0%
                \directlua{%
                  if tex.enableprimitives then %
                    tex.enableprimitives('HOLOGO@', {'pdfliteral'})%
                  else %
                    tex.print('1')%
                  end%
                }%
                \ifx\HOLOGO@pdfliteral\@undefined 1\fi%
                \relax%
              \endgroup
              \let\HOLOGO@temp\relax
              \global\let\HOLOGO@pdfliteral\HOLOGO@pdfliteral
            \fi%
          \HOLOGO@temp
        \fi
      }{}%
    \fi
    \ltx@IfUndefined{HOLOGO@pdfliteral}{%
      \@PackageWarningNoLine{hologo}{%
        Cannot find \string\pdfliteral
      }%
    }{}%
  \else
    \ifxetex
      \def\hologoDriver{xetex}%
    \else
      \ifvtex
        \def\hologoDriver{vtex}%
      \fi
    \fi
  \fi
}
%    \end{macrocode}
%    \end{macro}
%
%    \begin{macro}{\HOLOGO@WarningUnsupportedDriver}
%    \begin{macrocode}
\def\HOLOGO@WarningUnsupportedDriver#1{%
  \@PackageWarningNoLine{hologo}{%
    Logo `#1' needs driver specific macros,\MessageBreak
    but driver `\hologoDriver' is not supported.\MessageBreak
    Use a different driver or\MessageBreak
    load package `graphics' or `pgf'%
  }%
}
%    \end{macrocode}
%    \end{macro}
%
% \subsubsection{Reflect box macros}
%
%    Skip driver part if not needed.
%    \begin{macrocode}
\ltx@IfUndefined{reflectbox}{}{%
  \ltx@IfUndefined{rotatebox}{}{%
    \HOLOGO@AtEnd
  }%
}
\ltx@IfUndefined{pgftext}{}{%
  \HOLOGO@AtEnd
}
\ltx@IfUndefined{psscalebox}{}{%
  \HOLOGO@AtEnd
}
%    \end{macrocode}
%
%    \begin{macrocode}
\def\HOLOGO@temp{LaTeX2e}
\ifx\fmtname\HOLOGO@temp
  \RequirePackage{kvoptions}[2011/06/30]%
  \ProcessKeyvalOptions{HoLogoDriver}%
\fi
\HOLOGO@DriverSetup{}
%    \end{macrocode}
%
%    \begin{macro}{\HOLOGO@ReflectBox}
%    \begin{macrocode}
\def\HOLOGO@ReflectBox#1{%
  \begingroup
    \setbox\ltx@zero\hbox{\begingroup#1\endgroup}%
    \setbox\ltx@two\hbox{%
      \kern\wd\ltx@zero
      \csname HOLOGO@ScaleBox@\hologoDriver\endcsname{-1}{1}{%
        \hbox to 0pt{\copy\ltx@zero\hss}%
      }%
    }%
    \wd\ltx@two=\wd\ltx@zero
    \box\ltx@two
  \endgroup
}
%    \end{macrocode}
%    \end{macro}
%
%    \begin{macro}{\HOLOGO@PointReflectBox}
%    \begin{macrocode}
\def\HOLOGO@PointReflectBox#1{%
  \begingroup
    \setbox\ltx@zero\hbox{\begingroup#1\endgroup}%
    \setbox\ltx@two\hbox{%
      \kern\wd\ltx@zero
      \raise\ht\ltx@zero\hbox{%
        \csname HOLOGO@ScaleBox@\hologoDriver\endcsname{-1}{-1}{%
          \hbox to 0pt{\copy\ltx@zero\hss}%
        }%
      }%
    }%
    \wd\ltx@two=\wd\ltx@zero
    \box\ltx@two
  \endgroup
}
%    \end{macrocode}
%    \end{macro}
%
%    We must define all variants because of dynamic driver setup.
%    \begin{macrocode}
\def\HOLOGO@temp#1#2{#2}
%    \end{macrocode}
%
%    \begin{macro}{\HOLOGO@ScaleBox@pdftex}
%    \begin{macrocode}
\HOLOGO@temp{pdftex}{%
  \def\HOLOGO@ScaleBox@pdftex#1#2#3{%
    \HOLOGO@pdfliteral{%
      q #1 0 0 #2 0 0 cm%
    }%
    #3%
    \HOLOGO@pdfliteral{%
      Q%
    }%
  }%
}
%    \end{macrocode}
%    \end{macro}
%    \begin{macro}{\HOLOGO@ScaleBox@dvips}
%    \begin{macrocode}
\HOLOGO@temp{dvips}{%
  \def\HOLOGO@ScaleBox@dvips#1#2#3{%
    \special{ps:%
      gsave %
      currentpoint %
      currentpoint translate %
      #1 #2 scale %
      neg exch neg exch translate%
    }%
    #3%
    \special{ps:%
      currentpoint %
      grestore %
      moveto%
    }%
  }%
}
%    \end{macrocode}
%    \end{macro}
%    \begin{macro}{\HOLOGO@ScaleBox@dvipdfm}
%    \begin{macrocode}
\HOLOGO@temp{dvipdfm}{%
  \let\HOLOGO@ScaleBox@dvipdfm\HOLOGO@ScaleBox@dvips
}
%    \end{macrocode}
%    \end{macro}
%    Since \hologo{XeTeX} v0.6.
%    \begin{macro}{\HOLOGO@ScaleBox@xetex}
%    \begin{macrocode}
\HOLOGO@temp{xetex}{%
  \def\HOLOGO@ScaleBox@xetex#1#2#3{%
    \special{x:gsave}%
    \special{x:scale #1 #2}%
    #3%
    \special{x:grestore}%
  }%
}
%    \end{macrocode}
%    \end{macro}
%    \begin{macro}{\HOLOGO@ScaleBox@vtex}
%    \begin{macrocode}
\HOLOGO@temp{vtex}{%
  \def\HOLOGO@ScaleBox@vtex#1#2#3{%
    \special{r(#1,0,0,#2,0,0}%
    #3%
    \special{r)}%
  }%
}
%    \end{macrocode}
%    \end{macro}
%
%    \begin{macrocode}
\HOLOGO@AtEnd%
%</package>
%    \end{macrocode}
%
% \section{Test}
%
% \subsection{Catcode checks for loading}
%
%    \begin{macrocode}
%<*test1>
%    \end{macrocode}
%    \begin{macrocode}
\catcode`\{=1 %
\catcode`\}=2 %
\catcode`\#=6 %
\catcode`\@=11 %
\expandafter\ifx\csname count@\endcsname\relax
  \countdef\count@=255 %
\fi
\expandafter\ifx\csname @gobble\endcsname\relax
  \long\def\@gobble#1{}%
\fi
\expandafter\ifx\csname @firstofone\endcsname\relax
  \long\def\@firstofone#1{#1}%
\fi
\expandafter\ifx\csname loop\endcsname\relax
  \expandafter\@firstofone
\else
  \expandafter\@gobble
\fi
{%
  \def\loop#1\repeat{%
    \def\body{#1}%
    \iterate
  }%
  \def\iterate{%
    \body
      \let\next\iterate
    \else
      \let\next\relax
    \fi
    \next
  }%
  \let\repeat=\fi
}%
\def\RestoreCatcodes{}
\count@=0 %
\loop
  \edef\RestoreCatcodes{%
    \RestoreCatcodes
    \catcode\the\count@=\the\catcode\count@\relax
  }%
\ifnum\count@<255 %
  \advance\count@ 1 %
\repeat

\def\RangeCatcodeInvalid#1#2{%
  \count@=#1\relax
  \loop
    \catcode\count@=15 %
  \ifnum\count@<#2\relax
    \advance\count@ 1 %
  \repeat
}
\def\RangeCatcodeCheck#1#2#3{%
  \count@=#1\relax
  \loop
    \ifnum#3=\catcode\count@
    \else
      \errmessage{%
        Character \the\count@\space
        with wrong catcode \the\catcode\count@\space
        instead of \number#3%
      }%
    \fi
  \ifnum\count@<#2\relax
    \advance\count@ 1 %
  \repeat
}
\def\space{ }
\expandafter\ifx\csname LoadCommand\endcsname\relax
  \def\LoadCommand{\input hologo.sty\relax}%
\fi
\def\Test{%
  \RangeCatcodeInvalid{0}{47}%
  \RangeCatcodeInvalid{58}{64}%
  \RangeCatcodeInvalid{91}{96}%
  \RangeCatcodeInvalid{123}{255}%
  \catcode`\@=12 %
  \catcode`\\=0 %
  \catcode`\%=14 %
  \LoadCommand
  \RangeCatcodeCheck{0}{36}{15}%
  \RangeCatcodeCheck{37}{37}{14}%
  \RangeCatcodeCheck{38}{47}{15}%
  \RangeCatcodeCheck{48}{57}{12}%
  \RangeCatcodeCheck{58}{63}{15}%
  \RangeCatcodeCheck{64}{64}{12}%
  \RangeCatcodeCheck{65}{90}{11}%
  \RangeCatcodeCheck{91}{91}{15}%
  \RangeCatcodeCheck{92}{92}{0}%
  \RangeCatcodeCheck{93}{96}{15}%
  \RangeCatcodeCheck{97}{122}{11}%
  \RangeCatcodeCheck{123}{255}{15}%
  \RestoreCatcodes
}
\Test
\csname @@end\endcsname
\end
%    \end{macrocode}
%    \begin{macrocode}
%</test1>
%    \end{macrocode}
%
% \subsection{Spacefactor}
%
%    The space factor must be 1000 after a logo. If it is greater 1000
%    then the following space is a space after a sentence closing point.
%    If the space factor is smaller 1000 then an immediate following
%    dot is interpreted as abbreviation, not sentence closing point.
%
%    \begin{macrocode}
%<*test-spacefactor>
\NeedsTeXFormat{LaTeX2e}
\documentclass{article}
\usepackage{hologo}[2016/05/12]
\usepackage{kvsetkeys}
\usepackage{qstest}
\IncludeTests{*}
\LogTests{log}{*}{*}
\begin{document}
\begin{qstest}{spacefactor}{spacefactor}
\newcommand*{\Test}[1]{%
  \sbox0{%
    \hologo{#1}%
    \Expect*{1000 (#1)}*{\the\spacefactor\space(#1)}%
  }%
}%
\makeatletter
\def\TestList{}
\def\hologoEntry#1#2#3{%
  \edef\TestList{%
    \ifx\TestList\@empty
    \else
      \TestList,%
    \fi
    #1%
    \ifx\\#2\\%
    \else
      ={variant=#2}%
    \fi
  }%
}
\hologoList
\expandafter\kv@parse@normalized\expandafter{%
  \TestList
}{%
  \begingroup
    \let\@logo=\kv@key
    \ifx\kv@value\relax
    \else
      \expandafter\hologoLogoSetup\expandafter\@logo\expandafter{%
        \kv@value
      }%
    \fi
    \Test\@logo
  \endgroup
  \@gobbletwo
}
\end{qstest}
\end{document}
%</test-spacefactor>
%    \end{macrocode}
%
% \subsection{Complete list}
%
%    \begin{macrocode}
%<*test-list>
\NeedsTeXFormat{LaTeX2e}
\documentclass[12pt,a4paper]{article}
\usepackage{hologo}[2016/05/12]
\usepackage[T1]{fontenc}
\usepackage{lmodern}
\usepackage{parskip}
\usepackage[unicode]{hyperref}[2011/09/28]
\usepackage{bookmark}[2011/09/19]
\bookmarksetup{%
  numbered,%
  open,%
  openlevel=2,%
}
\renewcommand*{\contentsname}{List of logos}
\begin{document}
\tableofcontents
\def\TestFont#1#2#3#4#5#6{%
  \begingroup
    \usefont{#3}{#4}{#5}{#6}%
    \HologoVariant{#1}{#2}/\hologoVariant{#1}{#2}%
    \quad
    \begingroup\scriptsize\hologoVariant{#1}{#2}\endgroup
    \quad
  \endgroup
  (#3/#4/#5/#6)%
  \par
}
\makeatletter
\def\hologoEntry#1#2#3{%
  \section{%
    \HologoVariant{#1}{#2}/\hologoVariant{#1}{#2} %
    {[#1\ifx\\#2\\\else\space(#2)\fi]}% hash-ok
  }% braces around [] because of bug in tex4ht
  \begingroup
    \hypersetup{unicode=false}%
    \bookmark[%
      dest=\@currentHref,%
      rellevel=1,%
      keeplevel,%
    ]{%
      \HologoVariant{#1}{#2}/\hologoVariant{#1}{#2} %
      (PDFDocEncoding)%
    }%
  \endgroup
  \TestFont{#1}{#2}{OT1}{cmr}{m}{n}%
  \TestFont{#1}{#2}{OT1}{cmss}{m}{n}%
  \TestFont{#1}{#2}{OT1}{cmr}{b}{n}%
  \TestFont{#1}{#2}{OT1}{cmr}{m}{it}%
  \TestFont{#1}{#2}{OT1}{cmtt}{m}{n}%
  \TestFont{#1}{#2}{T1}{lmr}{m}{n}%
  \TestFont{#1}{#2}{T1}{lmss}{m}{n}%
  \TestFont{#1}{#2}{T1}{lmr}{b}{n}%
  \TestFont{#1}{#2}{T1}{lmr}{m}{it}%
  \TestFont{#1}{#2}{T1}{lmtt}{m}{n}%
  \TestFont{#1}{#2}{T1}{lmvtt}{m}{n}%
  \TestFont{#1}{#2}{T1}{qtm}{m}{n}%
  \TestFont{#1}{#2}{T1}{qhv}{m}{n}%
  \TestFont{#1}{#2}{T1}{qtm}{b}{n}%
  \TestFont{#1}{#2}{T1}{qtm}{m}{it}%
  \TestFont{#1}{#2}{T1}{qcr}{m}{n}%
  \newpage
}
\makeatother
\hologoList
\end{document}
%</test-list>
%    \end{macrocode}
%
% \section{Installation}
%
% \subsection{Download}
%
% \paragraph{Package.} This package is available on
% CTAN\footnote{\url{ftp://ftp.ctan.org/tex-archive/}}:
% \begin{description}
% \item[\CTAN{macros/latex/contrib/oberdiek/hologo.dtx}] The source file.
% \item[\CTAN{macros/latex/contrib/oberdiek/hologo.pdf}] Documentation.
% \end{description}
%
%
% \paragraph{Bundle.} All the packages of the bundle `oberdiek'
% are also available in a TDS compliant ZIP archive. There
% the packages are already unpacked and the documentation files
% are generated. The files and directories obey the TDS standard.
% \begin{description}
% \item[\CTAN{install/macros/latex/contrib/oberdiek.tds.zip}]
% \end{description}
% \emph{TDS} refers to the standard ``A Directory Structure
% for \TeX\ Files'' (\CTAN{tds/tds.pdf}). Directories
% with \xfile{texmf} in their name are usually organized this way.
%
% \subsection{Bundle installation}
%
% \paragraph{Unpacking.} Unpack the \xfile{oberdiek.tds.zip} in the
% TDS tree (also known as \xfile{texmf} tree) of your choice.
% Example (linux):
% \begin{quote}
%   |unzip oberdiek.tds.zip -d ~/texmf|
% \end{quote}
%
% \paragraph{Script installation.}
% Check the directory \xfile{TDS:scripts/oberdiek/} for
% scripts that need further installation steps.
% Package \xpackage{attachfile2} comes with the Perl script
% \xfile{pdfatfi.pl} that should be installed in such a way
% that it can be called as \texttt{pdfatfi}.
% Example (linux):
% \begin{quote}
%   |chmod +x scripts/oberdiek/pdfatfi.pl|\\
%   |cp scripts/oberdiek/pdfatfi.pl /usr/local/bin/|
% \end{quote}
%
% \subsection{Package installation}
%
% \paragraph{Unpacking.} The \xfile{.dtx} file is a self-extracting
% \docstrip\ archive. The files are extracted by running the
% \xfile{.dtx} through \plainTeX:
% \begin{quote}
%   \verb|tex hologo.dtx|
% \end{quote}
%
% \paragraph{TDS.} Now the different files must be moved into
% the different directories in your installation TDS tree
% (also known as \xfile{texmf} tree):
% \begin{quote}
% \def\t{^^A
% \begin{tabular}{@{}>{\ttfamily}l@{ $\rightarrow$ }>{\ttfamily}l@{}}
%   hologo.sty & tex/generic/oberdiek/hologo.sty\\
%   hologo.pdf & doc/latex/oberdiek/hologo.pdf\\
%   example/hologo-example.tex & doc/latex/oberdiek/example/hologo-example.tex\\
%   test/hologo-test1.tex & doc/latex/oberdiek/test/hologo-test1.tex\\
%   test/hologo-test-spacefactor.tex & doc/latex/oberdiek/test/hologo-test-spacefactor.tex\\
%   test/hologo-test-list.tex & doc/latex/oberdiek/test/hologo-test-list.tex\\
%   hologo.dtx & source/latex/oberdiek/hologo.dtx\\
% \end{tabular}^^A
% }^^A
% \sbox0{\t}^^A
% \ifdim\wd0>\linewidth
%   \begingroup
%     \advance\linewidth by\leftmargin
%     \advance\linewidth by\rightmargin
%   \edef\x{\endgroup
%     \def\noexpand\lw{\the\linewidth}^^A
%   }\x
%   \def\lwbox{^^A
%     \leavevmode
%     \hbox to \linewidth{^^A
%       \kern-\leftmargin\relax
%       \hss
%       \usebox0
%       \hss
%       \kern-\rightmargin\relax
%     }^^A
%   }^^A
%   \ifdim\wd0>\lw
%     \sbox0{\small\t}^^A
%     \ifdim\wd0>\linewidth
%       \ifdim\wd0>\lw
%         \sbox0{\footnotesize\t}^^A
%         \ifdim\wd0>\linewidth
%           \ifdim\wd0>\lw
%             \sbox0{\scriptsize\t}^^A
%             \ifdim\wd0>\linewidth
%               \ifdim\wd0>\lw
%                 \sbox0{\tiny\t}^^A
%                 \ifdim\wd0>\linewidth
%                   \lwbox
%                 \else
%                   \usebox0
%                 \fi
%               \else
%                 \lwbox
%               \fi
%             \else
%               \usebox0
%             \fi
%           \else
%             \lwbox
%           \fi
%         \else
%           \usebox0
%         \fi
%       \else
%         \lwbox
%       \fi
%     \else
%       \usebox0
%     \fi
%   \else
%     \lwbox
%   \fi
% \else
%   \usebox0
% \fi
% \end{quote}
% If you have a \xfile{docstrip.cfg} that configures and enables \docstrip's
% TDS installing feature, then some files can already be in the right
% place, see the documentation of \docstrip.
%
% \subsection{Refresh file name databases}
%
% If your \TeX~distribution
% (\teTeX, \mikTeX, \dots) relies on file name databases, you must refresh
% these. For example, \teTeX\ users run \verb|texhash| or
% \verb|mktexlsr|.
%
% \subsection{Some details for the interested}
%
% \paragraph{Attached source.}
%
% The PDF documentation on CTAN also includes the
% \xfile{.dtx} source file. It can be extracted by
% AcrobatReader 6 or higher. Another option is \textsf{pdftk},
% e.g. unpack the file into the current directory:
% \begin{quote}
%   \verb|pdftk hologo.pdf unpack_files output .|
% \end{quote}
%
% \paragraph{Unpacking with \LaTeX.}
% The \xfile{.dtx} chooses its action depending on the format:
% \begin{description}
% \item[\plainTeX:] Run \docstrip\ and extract the files.
% \item[\LaTeX:] Generate the documentation.
% \end{description}
% If you insist on using \LaTeX\ for \docstrip\ (really,
% \docstrip\ does not need \LaTeX), then inform the autodetect routine
% about your intention:
% \begin{quote}
%   \verb|latex \let\install=y\input{hologo.dtx}|
% \end{quote}
% Do not forget to quote the argument according to the demands
% of your shell.
%
% \paragraph{Generating the documentation.}
% You can use both the \xfile{.dtx} or the \xfile{.drv} to generate
% the documentation. The process can be configured by the
% configuration file \xfile{ltxdoc.cfg}. For instance, put this
% line into this file, if you want to have A4 as paper format:
% \begin{quote}
%   \verb|\PassOptionsToClass{a4paper}{article}|
% \end{quote}
% An example follows how to generate the
% documentation with pdf\LaTeX:
% \begin{quote}
%\begin{verbatim}
%pdflatex hologo.dtx
%makeindex -s gind.ist hologo.idx
%pdflatex hologo.dtx
%makeindex -s gind.ist hologo.idx
%pdflatex hologo.dtx
%\end{verbatim}
% \end{quote}
%
% \section{Catalogue}
%
% The following XML file can be used as source for the
% \href{http://mirror.ctan.org/help/Catalogue/catalogue.html}{\TeX\ Catalogue}.
% The elements \texttt{caption} and \texttt{description} are imported
% from the original XML file from the Catalogue.
% The name of the XML file in the Catalogue is \xfile{hologo.xml}.
%    \begin{macrocode}
%<*catalogue>
<?xml version='1.0' encoding='us-ascii'?>
<!DOCTYPE entry SYSTEM 'catalogue.dtd'>
<entry datestamp='$Date$' modifier='$Author$' id='hologo'>
  <name>hologo</name>
  <caption>A collection of logos with bookmark support.</caption>
  <authorref id='auth:oberdiek'/>
  <copyright owner='Heiko Oberdiek' year='2010-2012'/>
  <license type='lppl1.3'/>
  <version number='1.10'/>
  <description>
    The package defines a single command <tt>\hologo</tt>, whose
    argument is the usual case-confused ASCII version of the logo.
    The command is bookmark-enabled, so that every logo becomes
    available in bookmarks without further work.
    <p/>
    The package is part of the <xref refid='oberdiek'>oberdiek</xref>
    bundle.
  </description>
  <documentation details='Package documentation'
      href='ctan:/macros/latex/contrib/oberdiek/hologo.pdf'/>
  <ctan file='true' path='/macros/latex/contrib/oberdiek/hologo.dtx'/>
  <miktex location='oberdiek'/>
  <texlive location='oberdiek'/>
  <install path='/macros/latex/contrib/oberdiek/oberdiek.tds.zip'/>
</entry>
%</catalogue>
%    \end{macrocode}
%
% \begin{thebibliography}{9}
% \raggedright
%
% \bibitem{btxdoc}
% Oren Patashnik,
% \textit{\hologo{BibTeX}ing},
% 1988-02-08.\\
% \CTAN{biblio/bibtex/base/}
%
% \bibitem{dtklogos}
% Gerd Neugebauer, DANTE,
% \textit{Package \xpackage{dtklogos}},
% 2011-04-25.\\
% \CTAN{usergrps/dante/dtk/dtklogos.sty}
%
% \bibitem{etexman}
% The \hologo{NTS} Team,
% \textit{The \hologo{eTeX} manual},
% 1998-02.\\
% \CTAN{systems/e-tex/v2/doc/}
%
% \bibitem{ExTeX-FAQ}
% The \hologo{ExTeX} group,
% \textit{\hologo{ExTeX}: FAQ -- How is \hologo{ExTeX} typeset?},
% 2007-04-14.\\
% \url{http://www.extex.org/documentation/faq.html}
%
% \bibitem{LyX}
% %@MISC{ LyX,
% %  title = {{LyX 2.0.0 -- The Document Processor [Computer software and manual]}},
% %  author = {{The LyX Team}},
% %  howpublished = {Internet: http://www.lyx.org},
% %  year = {2011-05-08},
% %  note = {Retrieved May 10, 2011, from http://www.lyx.org},
% %  url = {http://www.lyx.org/}
% %}
% The \hologo{LyX} Team,
% \textit{\hologo{LyX} -- The Document Processor},
% 2011-05-08.\\
% \url{http://www.lyx.org/}
%
% \bibitem{OzTeX}
% Andrew Trevorrow,
% \hologo{OzTeX} FAQ: What is the correct way to typeset ``\hologo{OzTeX}''?,
% 2011-09-15 (visited).
% \url{http://www.trevorrow.com/oztex/ozfaq.html#oztex-logo}
%
% \bibitem{PiCTeX}
% Michael Wichura,
% \textit{The \hologo{PiCTeX} macro package},
% 1987-09-21.
% \CTAN{graphics/pictex/}
%
% \bibitem{scrlogo}
% Markus Kohm,
% \textit{\hologo{KOMAScript} Datei \xfile{scrlogo.dtx}},
% 2009-01-30.\\
% \CTAN{install/macros/latex/contrib/komascript.tds.zip}
%
% \end{thebibliography}
%
% \begin{History}
%   \begin{Version}{2010/04/08 v1.0}
%   \item
%     The first version.
%   \end{Version}
%   \begin{Version}{2010/04/16 v1.1}
%   \item
%     \cs{Hologo} added for support of logos at start of a sentence.
%   \item
%     \cs{hologoSetup} and \cs{hologoLogoSetup} added.
%   \item
%     Options \xoption{break}, \xoption{hyphenbreak}, \xoption{spacebreak}
%     added.
%   \item
%     Variant support added by option \xoption{variant}.
%   \end{Version}
%   \begin{Version}{2010/04/24 v1.2}
%   \item
%     \hologo{LaTeX3} added.
%   \item
%     \hologo{VTeX} added.
%   \end{Version}
%   \begin{Version}{2010/11/21 v1.3}
%   \item
%     \hologo{iniTeX}, \hologo{virTeX} added.
%   \end{Version}
%   \begin{Version}{2011/03/25 v1.4}
%   \item
%     \hologo{ConTeXt} with variants added.
%   \item
%     Option \xoption{discretionarybreak} added as refinement for
%     option \xoption{break}.
%   \end{Version}
%   \begin{Version}{2011/04/21 v1.5}
%   \item
%     Wrong TDS directory for test files fixed.
%   \end{Version}
%   \begin{Version}{2011/10/01 v1.6}
%   \item
%     Support for package \xpackage{tex4ht} added.
%   \item
%     Support for \cs{csname} added if \cs{ifincsname} is available.
%   \item
%     New logos:
%     \hologo{(La)TeX},
%     \hologo{biber},
%     \hologo{BibTeX} (\xoption{sc}, \xoption{sf}),
%     \hologo{emTeX},
%     \hologo{ExTeX},
%     \hologo{KOMAScript},
%     \hologo{La},
%     \hologo{LyX},
%     \hologo{MiKTeX},
%     \hologo{NTS},
%     \hologo{OzMF},
%     \hologo{OzMP},
%     \hologo{OzTeX},
%     \hologo{OzTtH},
%     \hologo{PCTeX},
%     \hologo{PiC},
%     \hologo{PiCTeX},
%     \hologo{METAFONT},
%     \hologo{MetaFun},
%     \hologo{METAPOST},
%     \hologo{MetaPost},
%     \hologo{SLiTeX} (\xoption{lift}, \xoption{narrow}, \xoption{simple}),
%     \hologo{SliTeX} (\xoption{narrow}, \xoption{simple}, \xoption{lift}),
%     \hologo{teTeX}.
%   \item
%     Fixes:
%     \hologo{iniTeX},
%     \hologo{pdfLaTeX},
%     \hologo{pdfTeX},
%     \hologo{virTeX}.
%   \item
%     \cs{hologoFontSetup} and \cs{hologoLogoFontSetup} added.
%   \item
%     \cs{hologoVariant} and \cs{HologoVariant} added.
%   \end{Version}
%   \begin{Version}{2011/11/22 v1.7}
%   \item
%     New logos:
%     \hologo{BibTeX8},
%     \hologo{LaTeXML},
%     \hologo{SageTeX},
%     \hologo{TeX4ht},
%     \hologo{TTH}.
%   \item
%     \hologo{Xe} and friends: Driver stuff fixed.
%   \item
%     \hologo{Xe} and friends: Support for italic added.
%   \item
%     \hologo{Xe} and friends: Package support for \xpackage{pgf}
%     and \xpackage{pstricks} added.
%   \end{Version}
%   \begin{Version}{2011/11/29 v1.8}
%   \item
%     New logos:
%     \hologo{HanTheThanh}.
%   \end{Version}
%   \begin{Version}{2011/12/21 v1.9}
%   \item
%     Patch for package \xpackage{ifxetex} added for the case that
%     \cs{newif} is undefined in \hologo{iniTeX}.
%   \item
%     Some fixes for \hologo{iniTeX}.
%   \end{Version}
%   \begin{Version}{2012/04/26 v1.10}
%   \item
%     Fix in bookmark version of logo ``\hologo{HanTheThanh}''.
%   \end{Version}
%   \begin{Version}{2016/05/12 v1.11}
%   \item
%     Update HOLOGO@IfCharExists (previously in texlive)
%   \item define pdfliteral in current luatex.
%   \end{Version}
% \end{History}
%
% \PrintIndex
%
% \Finale
\endinput

%        (quote the arguments according to the demands of your shell)
%
% Documentation:
%    (a) If hologo.drv is present:
%           latex hologo.drv
%    (b) Without hologo.drv:
%           latex hologo.dtx; ...
%    The class ltxdoc loads the configuration file ltxdoc.cfg
%    if available. Here you can specify further options, e.g.
%    use A4 as paper format:
%       \PassOptionsToClass{a4paper}{article}
%
%    Programm calls to get the documentation (example):
%       pdflatex hologo.dtx
%       makeindex -s gind.ist hologo.idx
%       pdflatex hologo.dtx
%       makeindex -s gind.ist hologo.idx
%       pdflatex hologo.dtx
%
% Installation:
%    TDS:tex/generic/oberdiek/hologo.sty
%    TDS:doc/latex/oberdiek/hologo.pdf
%    TDS:doc/latex/oberdiek/example/hologo-example.tex
%    TDS:doc/latex/oberdiek/test/hologo-test1.tex
%    TDS:doc/latex/oberdiek/test/hologo-test-spacefactor.tex
%    TDS:doc/latex/oberdiek/test/hologo-test-list.tex
%    TDS:source/latex/oberdiek/hologo.dtx
%
%<*ignore>
\begingroup
  \catcode123=1 %
  \catcode125=2 %
  \def\x{LaTeX2e}%
\expandafter\endgroup
\ifcase 0\ifx\install y1\fi\expandafter
         \ifx\csname processbatchFile\endcsname\relax\else1\fi
         \ifx\fmtname\x\else 1\fi\relax
\else\csname fi\endcsname
%</ignore>
%<*install>
\input docstrip.tex
\Msg{************************************************************************}
\Msg{* Installation}
\Msg{* Package: hologo 2016/05/12 v1.11 A logo collection with bookmark support (HO)}
\Msg{************************************************************************}

\keepsilent
\askforoverwritefalse

\let\MetaPrefix\relax
\preamble

This is a generated file.

Project: hologo
Version: 2016/05/12 v1.11

Copyright (C) 2010-2012 by
   Heiko Oberdiek <heiko.oberdiek at googlemail.com>

This work may be distributed and/or modified under the
conditions of the LaTeX Project Public License, either
version 1.3c of this license or (at your option) any later
version. This version of this license is in
   http://www.latex-project.org/lppl/lppl-1-3c.txt
and the latest version of this license is in
   http://www.latex-project.org/lppl.txt
and version 1.3 or later is part of all distributions of
LaTeX version 2005/12/01 or later.

This work has the LPPL maintenance status "maintained".

This Current Maintainer of this work is Heiko Oberdiek.

The Base Interpreter refers to any `TeX-Format',
because some files are installed in TDS:tex/generic//.

This work consists of the main source file hologo.dtx
and the derived files
   hologo.sty, hologo.pdf, hologo.ins, hologo.drv, hologo-example.tex,
   hologo-test1.tex, hologo-test-spacefactor.tex,
   hologo-test-list.tex.

\endpreamble
\let\MetaPrefix\DoubleperCent

\generate{%
  \file{hologo.ins}{\from{hologo.dtx}{install}}%
  \file{hologo.drv}{\from{hologo.dtx}{driver}}%
  \usedir{tex/generic/oberdiek}%
  \file{hologo.sty}{\from{hologo.dtx}{package}}%
  \usedir{doc/latex/oberdiek/example}%
  \file{hologo-example.tex}{\from{hologo.dtx}{example}}%
  \usedir{doc/latex/oberdiek/test}%
  \file{hologo-test1.tex}{\from{hologo.dtx}{test1}}%
  \file{hologo-test-spacefactor.tex}{\from{hologo.dtx}{test-spacefactor}}%
  \file{hologo-test-list.tex}{\from{hologo.dtx}{test-list}}%
  \nopreamble
  \nopostamble
  \usedir{source/latex/oberdiek/catalogue}%
  \file{hologo.xml}{\from{hologo.dtx}{catalogue}}%
}

\catcode32=13\relax% active space
\let =\space%
\Msg{************************************************************************}
\Msg{*}
\Msg{* To finish the installation you have to move the following}
\Msg{* file into a directory searched by TeX:}
\Msg{*}
\Msg{*     hologo.sty}
\Msg{*}
\Msg{* To produce the documentation run the file `hologo.drv'}
\Msg{* through LaTeX.}
\Msg{*}
\Msg{* Happy TeXing!}
\Msg{*}
\Msg{************************************************************************}

\endbatchfile
%</install>
%<*ignore>
\fi
%</ignore>
%<*driver>
\NeedsTeXFormat{LaTeX2e}
\ProvidesFile{hologo.drv}%
  [2016/05/12 v1.11 A logo collection with bookmark support (HO)]%
\documentclass{ltxdoc}
\usepackage{holtxdoc}[2011/11/22]
\usepackage{hologo}[2016/05/12]
\usepackage{longtable}
\usepackage{array}
\usepackage{paralist}
%\usepackage[T1]{fontenc}
%\usepackage{lmodern}
\begin{document}
  \DocInput{hologo.dtx}%
\end{document}
%</driver>
% \fi
%
%
% \CharacterTable
%  {Upper-case    \A\B\C\D\E\F\G\H\I\J\K\L\M\N\O\P\Q\R\S\T\U\V\W\X\Y\Z
%   Lower-case    \a\b\c\d\e\f\g\h\i\j\k\l\m\n\o\p\q\r\s\t\u\v\w\x\y\z
%   Digits        \0\1\2\3\4\5\6\7\8\9
%   Exclamation   \!     Double quote  \"     Hash (number) \#
%   Dollar        \$     Percent       \%     Ampersand     \&
%   Acute accent  \'     Left paren    \(     Right paren   \)
%   Asterisk      \*     Plus          \+     Comma         \,
%   Minus         \-     Point         \.     Solidus       \/
%   Colon         \:     Semicolon     \;     Less than     \<
%   Equals        \=     Greater than  \>     Question mark \?
%   Commercial at \@     Left bracket  \[     Backslash     \\
%   Right bracket \]     Circumflex    \^     Underscore    \_
%   Grave accent  \`     Left brace    \{     Vertical bar  \|
%   Right brace   \}     Tilde         \~}
%
% \GetFileInfo{hologo.drv}
%
% \title{The \xpackage{hologo} package}
% \date{2016/05/12 v1.11}
% \author{Heiko Oberdiek\\\xemail{heiko.oberdiek at googlemail.com}}
%
% \maketitle
%
% \begin{abstract}
% This package starts a collection of logos with support for bookmarks
% strings.
% \end{abstract}
%
% \tableofcontents
%
% \section{Documentation}
%
% \subsection{Logo macros}
%
% \begin{declcs}{hologo} \M{name}
% \end{declcs}
% Macro \cs{hologo} sets the logo with name \meta{name}.
% The following table shows the supported names.
%
% \begingroup
%   \def\hologoEntry#1#2#3{^^A
%     #1&#2&\hologoLogoSetup{#1}{variant=#2}\hologo{#1}&#3\tabularnewline
%   }
%   \begin{longtable}{>{\ttfamily}l>{\ttfamily}lll}
%     \rmfamily\bfseries{name} & \rmfamily\bfseries variant
%     & \bfseries logo & \bfseries since\\
%     \hline
%     \endhead
%     \hologoList
%   \end{longtable}
% \endgroup
%
% \begin{declcs}{Hologo} \M{name}
% \end{declcs}
% Macro \cs{Hologo} starts the logo \meta{name} with an uppercase
% letter. As an exception small greek letters are not converted
% to uppercase. Examples, see \hologo{eTeX} and \hologo{ExTeX}.
%
% \subsection{Setup macros}
%
% The package does not support package options, but the following
% setup macros can be used to set options.
%
% \begin{declcs}{hologoSetup} \M{key value list}
% \end{declcs}
% Macro \cs{hologoSetup} sets global options.
%
% \begin{declcs}{hologoLogoSetup} \M{logo} \M{key value list}
% \end{declcs}
% Some options can also be used to configure a logo.
% These settings take precedence over global option settings.
%
% \subsection{Options}\label{sec:options}
%
% There are boolean and string options:
% \begin{description}
% \item[Boolean option:]
% It takes |true| or |false|
% as value. If the value is omitted, then |true| is used.
% \item[String option:]
% A value must be given as string. (But the string might be empty.)
% \end{description}
% The following options can be used both in \cs{hologoSetup}
% and \cs{hologoLogoSetup}:
% \begin{description}
% \def\entry#1{\item[\xoption{#1}:]}
% \entry{break}
%   enables or disables line breaks inside the logo. This setting is
%   refined by options \xoption{hyphenbreak}, \xoption{spacebreak}
%   or \xoption{discretionarybreak}.
%   Default is |false|.
% \entry{hyphenbreak}
%   enables or disables the line break right after the hyphen character.
% \entry{spacebreak}
%   enables or disables line breaks at space characters.
% \entry{discretionarybreak}
%   enables or disables line breaks at hyphenation points
%   (inserted by \cs{-}).
% \end{description}
% Macro \cs{hologoLogoSetup} also knows:
% \begin{description}
% \item[\xoption{variant}:]
%   This is a string option. It specifies a variant of a logo that
%   must exist. An empty string selects the package default variant.
% \end{description}
% Example:
% \begin{quote}
%   |\hologoSetup{break=false}|\\
%   |\hologoLogoSetup{plainTeX}{variant=hyphen,hyphenbreak}|\\
%   Then ``plain-\TeX'' contains one break point after the hyphen.
% \end{quote}
%
% \subsection{Driver options}
%
% Sometimes graphical operations are needed to construct some
% glyphs (e.g.\ \hologo{XeTeX}). If package \xpackage{graphics}
% or package \xpackage{pgf} are found, then the macros are taken
% from there. Otherwise the packge defines its own operations
% and therefore needs the driver information. Many drivers are
% detected automatically (\hologo{pdfTeX}/\hologo{LuaTeX}
% in PDF mode, \hologo{XeTeX}, \hologo{VTeX}). These have precedence
% over a driver option. The driver can be given as package option
% or using \cs{hologoDriverSetup}.
% The following list contains the recognized driver options:
% \begin{itemize}
% \item \xoption{pdftex}, \xoption{luatex}
% \item \xoption{dvipdfm}, \xoption{dvipdfmx}
% \item \xoption{dvips}, \xoption{dvipsone}, \xoption{xdvi}
% \item \xoption{xetex}
% \item \xoption{vtex}
% \end{itemize}
% The left driver of a line is the driver name that is used internally.
% The following names are aliases for drivers that use the
% same method. Therefore the entry in the \xext{log} file for
% the used driver prints the internally used driver name.
% \begin{description}
% \item[\xoption{driverfallback}:]
%   This option expects a driver that is used,
%   if the driver could not be detected automatically.
% \end{description}
%
% \begin{declcs}{hologoDriverSetup} \M{driver option}
% \end{declcs}
% The driver can also be configured after package loading
% using \cs{hologoDriverSetup}, also the way for \hologo{plainTeX}
% to setup the driver.
%
% \subsection{Font setup}
%
% Some logos require a special font, but should also be usable by
% \hologo{plainTeX}. Therefore the package provides some ways
% to influence the font settings. The options below
% take font settings as values. Both font commands
% such as \cs{sffamily} and macros that take one argument
% like \cs{textsf} can be used.
%
% \begin{declcs}{hologoFontSetup} \M{key value list}
% \end{declcs}
% Macro \cs{hologoFontSetup} sets the fonts for all logos.
% Supported keys:
% \begin{description}
% \def\entry#1{\item[\xoption{#1}:]}
% \entry{general}
%   This font is used for all logos. The default is empty.
%   That means no special font is used.
% \entry{bibsf}
%   This font is used for
%   {\hologoLogoSetup{BibTeX}{variant=sf}\hologo{BibTeX}}
%   with variant \xoption{sf}.
% \entry{rm}
%   This font is a serif font. It is used for \hologo{ExTeX}.
% \entry{sc}
%   This font specifies a small caps font. It is used for
%   {\hologoLogoSetup{BibTeX}{variant=sc}\hologo{BibTeX}}
%   with variant \xoption{sc}.
% \entry{sf}
%   This font specifies a sans serif font. The default
%   is \cs{sffamily}, then \cs{sf} is tried. Otherwise
%   a warning is given. It is used by \hologo{KOMAScript}.
% \entry{sy}
%   This is the font for math symbols (e.g. cmsy).
%   It is used by \hologo{AmS}, \hologo{NTS}, \hologo{ExTeX}.
% \entry{logo}
%   \hologo{METAFONT} and \hologo{METAPOST} are using that font.
%   In \hologo{LaTeX} \cs{logofamily} is used and
%   the definitions of package \xpackage{mflogo} are used
%   if the package is not loaded.
%   Otherwise the \cs{tenlogo} is used and defined
%   if it does not already exists.
% \end{description}
%
% \begin{declcs}{hologoLogoFontSetup} \M{logo} \M{key value list}
% \end{declcs}
% Fonts can also be set for a logo or logo component separately,
% see the following list.
% The keys are the same as for \cs{hologoFontSetup}.
%
% \begin{longtable}{>{\ttfamily}l>{\sffamily}ll}
%   \meta{logo} & keys & result\\
%   \hline
%   \endhead
%   BibTeX & bibsf & {\hologoLogoSetup{BibTeX}{variant=sf}\hologo{BibTeX}}\\[.5ex]
%   BibTeX & sc & {\hologoLogoSetup{BibTeX}{variant=sc}\hologo{BibTeX}}\\[.5ex]
%   ExTeX & rm & \hologo{ExTeX}\\
%   SliTeX & rm & \hologo{SliTeX}\\[.5ex]
%   AmS & sy & \hologo{AmS}\\
%   ExTeX & sy & \hologo{ExTeX}\\
%   NTS & sy & \hologo{NTS}\\[.5ex]
%   KOMAScript & sf & \hologo{KOMAScript}\\[.5ex]
%   METAFONT & logo & \hologo{METAFONT}\\
%   METAPOST & logo & \hologo{METAPOST}\\[.5ex]
%   SliTeX & sc \hologo{SliTeX}
% \end{longtable}
%
% \subsubsection{Font order}
%
% For all logos the font \xoption{general} is applied first.
% Example:
%\begin{quote}
%|\hologoFontSetup{general=\color{red}}|
%\end{quote}
% will print red logos.
% Then if the font uses a special font \xoption{sf}, for example,
% the font is applied that is setup by \cs{hologoLogoFontSetup}.
% If this font is not setup, then the common font setup
% by \cs{hologoFontSetup} is used. Otherwise a warning is given,
% that there is no font configured.
%
% \subsection{Additional user macros}
%
% Usually a variant of a logo is configured by using
% \cs{hologoLogoSetup}, because it is bad style to mix
% different variants of the same logo in the same text.
% There the following macros are a convenience for testing.
%
% \begin{declcs}{hologoVariant} \M{name} \M{variant}\\
%   \cs{HologoVariant} \M{name} \M{variant}
% \end{declcs}
% Logo \meta{name} is set using \meta{variant} that specifies
% explicitely which variant of the macro is used. If the argument
% is empty, then the default form of the logo is used
% (configurable by \cs{hologoLogoSetup}).
%
% \cs{HologoVariant} is used if the logo is set in a context
% that needs an uppercase first letter (beginning of a sentence, \dots).
%
% \begin{declcs}{hologoList}\\
%   \cs{hologoEntry} \M{logo} \M{variant} \M{since}
% \end{declcs}
% Macro \cs{hologoList} contains all logos that are provided
% by the package including variants. The list consists of calls
% of \cs{hologoEntry} with three arguments starting with the
% logo name \meta{logo} and its variant \meta{variant}. An empty
% variant means the current default. Argument \meta{since} specifies
% with version of the package \xpackage{hologo} is needed to get
% the logo. If the logo is fixed, then the date gets updated.
% Therefore the date \meta{since} is not exactly the date of
% the first introduction, but rather the date of the latest fix.
%
% Before \cs{hologoList} can be used, macro \cs{hologoEntry} needs
% a definition. The example file in section \ref{sec:example}
% shows applications of \cs{hologoList}.
%
% \subsection{Supported contexts}
%
% Macros \cs{hologo} and friends support special contexts:
% \begin{itemize}
% \item \hologo{LaTeX}'s protection mechanism.
% \item Bookmarks of package \xpackage{hyperref}.
% \item Package \xpackage{tex4ht}.
% \item The macros can be used inside \cs{csname} constructs,
%   if \cs{ifincsname} is available (\hologo{pdfTeX}, \hologo{XeTeX},
%   \hologo{LuaTeX}).
% \end{itemize}
%
% \subsection{Example}
% \label{sec:example}
%
% The following example prints the logos in different fonts.
%    \begin{macrocode}
%<*example>
%<<verbatim
\NeedsTeXFormat{LaTeX2e}
\documentclass[a4paper]{article}
\usepackage[
  hmargin=20mm,
  vmargin=20mm,
]{geometry}
\pagestyle{empty}
\usepackage{hologo}[2016/05/12]
\usepackage{longtable}
\usepackage{array}
\setlength{\extrarowheight}{2pt}
\usepackage[T1]{fontenc}
\usepackage{lmodern}
\usepackage{pdflscape}
\usepackage[
  pdfencoding=auto,
]{hyperref}
\hypersetup{
  pdfauthor={Heiko Oberdiek},
  pdftitle={Example for package `hologo'},
  pdfsubject={Logos with fonts lmr, lmss, qtm, qpl, qhv},
}
\usepackage{bookmark}

% Print the logo list on the console

\begingroup
  \typeout{}%
  \typeout{*** Begin of logo list ***}%
  \newcommand*{\hologoEntry}[3]{%
    \typeout{#1 \ifx\\#2\\\else(#2) \fi[#3]}%
  }%
  \hologoList
  \typeout{*** End of logo list ***}%
  \typeout{}%
\endgroup

\begin{document}
\begin{landscape}

  \section{Example file for package `hologo'}

  % Table for font names

  \begin{longtable}{>{\bfseries}ll}
    \textbf{font} & \textbf{Font name}\\
    \hline
    lmr & Latin Modern Roman\\
    lmss & Latin Modern Sans\\
    qtm & \TeX\ Gyre Termes\\
    qhv & \TeX\ Gyre Heros\\
    qpl & \TeX\ Gyre Pagella\\
  \end{longtable}

  % Logo list with logos in different fonts

  \begingroup
    \newcommand*{\SetVariant}[2]{%
      \ifx\\#2\\%
      \else
        \hologoLogoSetup{#1}{variant=#2}%
      \fi
    }%
    \newcommand*{\hologoEntry}[3]{%
      \SetVariant{#1}{#2}%
      \raisebox{1em}[0pt][0pt]{\hypertarget{#1@#2}{}}%
      \bookmark[%
        dest={#1@#2},%
      ]{%
        #1\ifx\\#2\\\else\space(#2)\fi: \Hologo{#1}, \hologo{#1} %
        [Unicode]%
      }%
      \hypersetup{unicode=false}%
      \bookmark[%
        dest={#1@#2},%
      ]{%
        #1\ifx\\#2\\\else\space(#2)\fi: \Hologo{#1}, \hologo{#1} %
        [PDFDocEncoding]%
      }%
      \texttt{#1}%
      &%
      \texttt{#2}%
      &%
      \Hologo{#1}%
      &%
      \SetVariant{#1}{#2}%
      \hologo{#1}%
      &%
      \SetVariant{#1}{#2}%
      \fontfamily{qtm}\selectfont
      \hologo{#1}%
      &%
      \SetVariant{#1}{#2}%
      \fontfamily{qpl}\selectfont
      \hologo{#1}%
      &%
      \SetVariant{#1}{#2}%
      \textsf{\hologo{#1}}%
      &%
      \SetVariant{#1}{#2}%
      \fontfamily{qhv}\selectfont
      \hologo{#1}%
      \tabularnewline
    }%
    \begin{longtable}{llllllll}%
      \textbf{\textit{logo}} & \textbf{\textit{variant}} &
      \texttt{\string\Hologo} &
      \textbf{lmr} & \textbf{qtm} & \textbf{qpl} &
      \textbf{lmss} & \textbf{qhv}
      \tabularnewline
      \hline
      \endhead
      \hologoList
    \end{longtable}%
  \endgroup

\end{landscape}
\end{document}
%verbatim
%</example>
%    \end{macrocode}
%
% \StopEventually{
% }
%
% \section{Implementation}
%    \begin{macrocode}
%<*package>
%    \end{macrocode}
%    Reload check, especially if the package is not used with \LaTeX.
%    \begin{macrocode}
\begingroup\catcode61\catcode48\catcode32=10\relax%
  \catcode13=5 % ^^M
  \endlinechar=13 %
  \catcode35=6 % #
  \catcode39=12 % '
  \catcode44=12 % ,
  \catcode45=12 % -
  \catcode46=12 % .
  \catcode58=12 % :
  \catcode64=11 % @
  \catcode123=1 % {
  \catcode125=2 % }
  \expandafter\let\expandafter\x\csname ver@hologo.sty\endcsname
  \ifx\x\relax % plain-TeX, first loading
  \else
    \def\empty{}%
    \ifx\x\empty % LaTeX, first loading,
      % variable is initialized, but \ProvidesPackage not yet seen
    \else
      \expandafter\ifx\csname PackageInfo\endcsname\relax
        \def\x#1#2{%
          \immediate\write-1{Package #1 Info: #2.}%
        }%
      \else
        \def\x#1#2{\PackageInfo{#1}{#2, stopped}}%
      \fi
      \x{hologo}{The package is already loaded}%
      \aftergroup\endinput
    \fi
  \fi
\endgroup%
%    \end{macrocode}
%    Package identification:
%    \begin{macrocode}
\begingroup\catcode61\catcode48\catcode32=10\relax%
  \catcode13=5 % ^^M
  \endlinechar=13 %
  \catcode35=6 % #
  \catcode39=12 % '
  \catcode40=12 % (
  \catcode41=12 % )
  \catcode44=12 % ,
  \catcode45=12 % -
  \catcode46=12 % .
  \catcode47=12 % /
  \catcode58=12 % :
  \catcode64=11 % @
  \catcode91=12 % [
  \catcode93=12 % ]
  \catcode123=1 % {
  \catcode125=2 % }
  \expandafter\ifx\csname ProvidesPackage\endcsname\relax
    \def\x#1#2#3[#4]{\endgroup
      \immediate\write-1{Package: #3 #4}%
      \xdef#1{#4}%
    }%
  \else
    \def\x#1#2[#3]{\endgroup
      #2[{#3}]%
      \ifx#1\@undefined
        \xdef#1{#3}%
      \fi
      \ifx#1\relax
        \xdef#1{#3}%
      \fi
    }%
  \fi
\expandafter\x\csname ver@hologo.sty\endcsname
\ProvidesPackage{hologo}%
  [2016/05/12 v1.11 A logo collection with bookmark support (HO)]%
%    \end{macrocode}
%
%    \begin{macrocode}
\begingroup\catcode61\catcode48\catcode32=10\relax%
  \catcode13=5 % ^^M
  \endlinechar=13 %
  \catcode123=1 % {
  \catcode125=2 % }
  \catcode64=11 % @
  \def\x{\endgroup
    \expandafter\edef\csname HOLOGO@AtEnd\endcsname{%
      \endlinechar=\the\endlinechar\relax
      \catcode13=\the\catcode13\relax
      \catcode32=\the\catcode32\relax
      \catcode35=\the\catcode35\relax
      \catcode61=\the\catcode61\relax
      \catcode64=\the\catcode64\relax
      \catcode123=\the\catcode123\relax
      \catcode125=\the\catcode125\relax
    }%
  }%
\x\catcode61\catcode48\catcode32=10\relax%
\catcode13=5 % ^^M
\endlinechar=13 %
\catcode35=6 % #
\catcode64=11 % @
\catcode123=1 % {
\catcode125=2 % }
\def\TMP@EnsureCode#1#2{%
  \edef\HOLOGO@AtEnd{%
    \HOLOGO@AtEnd
    \catcode#1=\the\catcode#1\relax
  }%
  \catcode#1=#2\relax
}
\TMP@EnsureCode{10}{12}% ^^J
\TMP@EnsureCode{33}{12}% !
\TMP@EnsureCode{34}{12}% "
\TMP@EnsureCode{36}{3}% $
\TMP@EnsureCode{38}{4}% &
\TMP@EnsureCode{39}{12}% '
\TMP@EnsureCode{40}{12}% (
\TMP@EnsureCode{41}{12}% )
\TMP@EnsureCode{42}{12}% *
\TMP@EnsureCode{43}{12}% +
\TMP@EnsureCode{44}{12}% ,
\TMP@EnsureCode{45}{12}% -
\TMP@EnsureCode{46}{12}% .
\TMP@EnsureCode{47}{12}% /
\TMP@EnsureCode{58}{12}% :
\TMP@EnsureCode{59}{12}% ;
\TMP@EnsureCode{60}{12}% <
\TMP@EnsureCode{62}{12}% >
\TMP@EnsureCode{63}{12}% ?
\TMP@EnsureCode{91}{12}% [
\TMP@EnsureCode{93}{12}% ]
\TMP@EnsureCode{94}{7}% ^ (superscript)
\TMP@EnsureCode{95}{8}% _ (subscript)
\TMP@EnsureCode{96}{12}% `
\TMP@EnsureCode{124}{12}% |
\edef\HOLOGO@AtEnd{%
  \HOLOGO@AtEnd
  \escapechar\the\escapechar\relax
  \noexpand\endinput
}
\escapechar=92 %
%    \end{macrocode}
%
% \subsection{Logo list}
%
%    \begin{macro}{\hologoList}
%    \begin{macrocode}
\def\hologoList{%
  \hologoEntry{(La)TeX}{}{2011/10/01}%
  \hologoEntry{AmSLaTeX}{}{2010/04/16}%
  \hologoEntry{AmSTeX}{}{2010/04/16}%
  \hologoEntry{biber}{}{2011/10/01}%
  \hologoEntry{BibTeX}{}{2011/10/01}%
  \hologoEntry{BibTeX}{sf}{2011/10/01}%
  \hologoEntry{BibTeX}{sc}{2011/10/01}%
  \hologoEntry{BibTeX8}{}{2011/11/22}%
  \hologoEntry{ConTeXt}{}{2011/03/25}%
  \hologoEntry{ConTeXt}{narrow}{2011/03/25}%
  \hologoEntry{ConTeXt}{simple}{2011/03/25}%
  \hologoEntry{emTeX}{}{2010/04/26}%
  \hologoEntry{eTeX}{}{2010/04/08}%
  \hologoEntry{ExTeX}{}{2011/10/01}%
  \hologoEntry{HanTheThanh}{}{2011/11/29}%
  \hologoEntry{iniTeX}{}{2011/10/01}%
  \hologoEntry{KOMAScript}{}{2011/10/01}%
  \hologoEntry{La}{}{2010/05/08}%
  \hologoEntry{LaTeX}{}{2010/04/08}%
  \hologoEntry{LaTeX2e}{}{2010/04/08}%
  \hologoEntry{LaTeX3}{}{2010/04/24}%
  \hologoEntry{LaTeXe}{}{2010/04/08}%
  \hologoEntry{LaTeXML}{}{2011/11/22}%
  \hologoEntry{LaTeXTeX}{}{2011/10/01}%
  \hologoEntry{LuaLaTeX}{}{2010/04/08}%
  \hologoEntry{LuaTeX}{}{2010/04/08}%
  \hologoEntry{LyX}{}{2011/10/01}%
  \hologoEntry{METAFONT}{}{2011/10/01}%
  \hologoEntry{MetaFun}{}{2011/10/01}%
  \hologoEntry{METAPOST}{}{2011/10/01}%
  \hologoEntry{MetaPost}{}{2011/10/01}%
  \hologoEntry{MiKTeX}{}{2011/10/01}%
  \hologoEntry{NTS}{}{2011/10/01}%
  \hologoEntry{OzMF}{}{2011/10/01}%
  \hologoEntry{OzMP}{}{2011/10/01}%
  \hologoEntry{OzTeX}{}{2011/10/01}%
  \hologoEntry{OzTtH}{}{2011/10/01}%
  \hologoEntry{PCTeX}{}{2011/10/01}%
  \hologoEntry{pdfTeX}{}{2011/10/01}%
  \hologoEntry{pdfLaTeX}{}{2011/10/01}%
  \hologoEntry{PiC}{}{2011/10/01}%
  \hologoEntry{PiCTeX}{}{2011/10/01}%
  \hologoEntry{plainTeX}{}{2010/04/08}%
  \hologoEntry{plainTeX}{space}{2010/04/16}%
  \hologoEntry{plainTeX}{hyphen}{2010/04/16}%
  \hologoEntry{plainTeX}{runtogether}{2010/04/16}%
  \hologoEntry{SageTeX}{}{2011/11/22}%
  \hologoEntry{SLiTeX}{}{2011/10/01}%
  \hologoEntry{SLiTeX}{lift}{2011/10/01}%
  \hologoEntry{SLiTeX}{narrow}{2011/10/01}%
  \hologoEntry{SLiTeX}{simple}{2011/10/01}%
  \hologoEntry{SliTeX}{}{2011/10/01}%
  \hologoEntry{SliTeX}{narrow}{2011/10/01}%
  \hologoEntry{SliTeX}{simple}{2011/10/01}%
  \hologoEntry{SliTeX}{lift}{2011/10/01}%
  \hologoEntry{teTeX}{}{2011/10/01}%
  \hologoEntry{TeX}{}{2010/04/08}%
  \hologoEntry{TeX4ht}{}{2011/11/22}%
  \hologoEntry{TTH}{}{2011/11/22}%
  \hologoEntry{virTeX}{}{2011/10/01}%
  \hologoEntry{VTeX}{}{2010/04/24}%
  \hologoEntry{Xe}{}{2010/04/08}%
  \hologoEntry{XeLaTeX}{}{2010/04/08}%
  \hologoEntry{XeTeX}{}{2010/04/08}%
}
%    \end{macrocode}
%    \end{macro}
%
% \subsection{Load resources}
%
%    \begin{macrocode}
\begingroup\expandafter\expandafter\expandafter\endgroup
\expandafter\ifx\csname RequirePackage\endcsname\relax
  \def\TMP@RequirePackage#1[#2]{%
    \begingroup\expandafter\expandafter\expandafter\endgroup
    \expandafter\ifx\csname ver@#1.sty\endcsname\relax
      \input #1.sty\relax
    \fi
  }%
  \TMP@RequirePackage{ltxcmds}[2011/02/04]%
  \TMP@RequirePackage{infwarerr}[2010/04/08]%
  \TMP@RequirePackage{kvsetkeys}[2010/03/01]%
  \TMP@RequirePackage{kvdefinekeys}[2010/03/01]%
  \TMP@RequirePackage{pdftexcmds}[2010/04/01]%
  \TMP@RequirePackage{ifpdf}[2010/01/28]%
  \TMP@RequirePackage{ifluatex}[2010/03/01]%
  \ltx@IfUndefined{newif}{%
    \expandafter\let\csname newif\endcsname\ltx@newif
  }{}%
  \TMP@RequirePackage{ifxetex}[2009/01/23]%
  \TMP@RequirePackage{ifvtex}[2010/03/01]%
\else
  \RequirePackage{ltxcmds}[2011/02/04]%
  \RequirePackage{infwarerr}[2010/04/08]%
  \RequirePackage{kvsetkeys}[2010/03/01]%
  \RequirePackage{kvdefinekeys}[2010/03/01]%
  \RequirePackage{pdftexcmds}[2010/04/01]%
  \RequirePackage{ifpdf}[2010/01/28]%
  \RequirePackage{ifluatex}[2010/03/01]%
  \RequirePackage{ifxetex}[2009/01/23]%
  \RequirePackage{ifvtex}[2010/03/01]%
\fi
%    \end{macrocode}
%
%    \begin{macro}{\HOLOGO@IfDefined}
%    \begin{macrocode}
\def\HOLOGO@IfExists#1{%
  \ifx\@undefined#1%
    \expandafter\ltx@secondoftwo
  \else
    \ifx\relax#1%
      \expandafter\ltx@secondoftwo
    \else
      \expandafter\expandafter\expandafter\ltx@firstoftwo
    \fi
  \fi
}
%    \end{macrocode}
%    \end{macro}
%
% \subsection{Setup macros}
%
%    \begin{macro}{\hologoSetup}
%    \begin{macrocode}
\def\hologoSetup{%
  \let\HOLOGO@name\relax
  \HOLOGO@Setup
}
%    \end{macrocode}
%    \end{macro}
%
%    \begin{macro}{\hologoLogoSetup}
%    \begin{macrocode}
\def\hologoLogoSetup#1{%
  \edef\HOLOGO@name{#1}%
  \ltx@IfUndefined{HoLogo@\HOLOGO@name}{%
    \@PackageError{hologo}{%
      Unknown logo `\HOLOGO@name'%
    }\@ehc
    \ltx@gobble
  }{%
    \HOLOGO@Setup
  }%
}
%    \end{macrocode}
%    \end{macro}
%
%    \begin{macro}{\HOLOGO@Setup}
%    \begin{macrocode}
\def\HOLOGO@Setup{%
  \kvsetkeys{HoLogo}%
}
%    \end{macrocode}
%    \end{macro}
%
% \subsection{Options}
%
%    \begin{macro}{\HOLOGO@DeclareBoolOption}
%    \begin{macrocode}
\def\HOLOGO@DeclareBoolOption#1{%
  \expandafter\chardef\csname HOLOGOOPT@#1\endcsname\ltx@zero
  \kv@define@key{HoLogo}{#1}[true]{%
    \def\HOLOGO@temp{##1}%
    \ifx\HOLOGO@temp\HOLOGO@true
      \ifx\HOLOGO@name\relax
        \expandafter\chardef\csname HOLOGOOPT@#1\endcsname=\ltx@one
      \else
        \expandafter\chardef\csname
        HoLogoOpt@#1@\HOLOGO@name\endcsname\ltx@one
      \fi
      \HOLOGO@SetBreakAll{#1}%
    \else
      \ifx\HOLOGO@temp\HOLOGO@false
        \ifx\HOLOGO@name\relax
          \expandafter\chardef\csname HOLOGOOPT@#1\endcsname=\ltx@zero
        \else
          \expandafter\chardef\csname
          HoLogoOpt@#1@\HOLOGO@name\endcsname=\ltx@zero
        \fi
        \HOLOGO@SetBreakAll{#1}%
      \else
        \@PackageError{hologo}{%
          Unknown value `##1' for boolean option `#1'.\MessageBreak
          Known values are `true' and `false'%
        }\@ehc
      \fi
    \fi
  }%
}
%    \end{macrocode}
%    \end{macro}
%
%    \begin{macro}{\HOLOGO@SetBreakAll}
%    \begin{macrocode}
\def\HOLOGO@SetBreakAll#1{%
  \def\HOLOGO@temp{#1}%
  \ifx\HOLOGO@temp\HOLOGO@break
    \ifx\HOLOGO@name\relax
      \chardef\HOLOGOOPT@hyphenbreak=\HOLOGOOPT@break
      \chardef\HOLOGOOPT@spacebreak=\HOLOGOOPT@break
      \chardef\HOLOGOOPT@discretionarybreak=\HOLOGOOPT@break
    \else
      \expandafter\chardef
         \csname HoLogoOpt@hyphenbreak@\HOLOGO@name\endcsname=%
         \csname HoLogoOpt@break@\HOLOGO@name\endcsname
      \expandafter\chardef
         \csname HoLogoOpt@spacebreak@\HOLOGO@name\endcsname=%
         \csname HoLogoOpt@break@\HOLOGO@name\endcsname
      \expandafter\chardef
         \csname HoLogoOpt@discretionarybreak@\HOLOGO@name
             \endcsname=%
         \csname HoLogoOpt@break@\HOLOGO@name\endcsname
    \fi
  \fi
}
%    \end{macrocode}
%    \end{macro}
%
%    \begin{macro}{\HOLOGO@true}
%    \begin{macrocode}
\def\HOLOGO@true{true}
%    \end{macrocode}
%    \end{macro}
%    \begin{macro}{\HOLOGO@false}
%    \begin{macrocode}
\def\HOLOGO@false{false}
%    \end{macrocode}
%    \end{macro}
%    \begin{macro}{\HOLOGO@break}
%    \begin{macrocode}
\def\HOLOGO@break{break}
%    \end{macrocode}
%    \end{macro}
%
%    \begin{macrocode}
\HOLOGO@DeclareBoolOption{break}
\HOLOGO@DeclareBoolOption{hyphenbreak}
\HOLOGO@DeclareBoolOption{spacebreak}
\HOLOGO@DeclareBoolOption{discretionarybreak}
%    \end{macrocode}
%
%    \begin{macrocode}
\kv@define@key{HoLogo}{variant}{%
  \ifx\HOLOGO@name\relax
    \@PackageError{hologo}{%
      Option `variant' is not available in \string\hologoSetup,%
      \MessageBreak
      Use \string\hologoLogoSetup\space instead%
    }\@ehc
  \else
    \edef\HOLOGO@temp{#1}%
    \ifx\HOLOGO@temp\ltx@empty
      \expandafter
      \let\csname HoLogoOpt@variant@\HOLOGO@name\endcsname\@undefined
    \else
      \ltx@IfUndefined{HoLogo@\HOLOGO@name @\HOLOGO@temp}{%
        \@PackageError{hologo}{%
          Unknown variant `\HOLOGO@temp' of logo `\HOLOGO@name'%
        }\@ehc
      }{%
        \expandafter
        \let\csname HoLogoOpt@variant@\HOLOGO@name\endcsname
            \HOLOGO@temp
      }%
    \fi
  \fi
}
%    \end{macrocode}
%
%    \begin{macro}{\HOLOGO@Variant}
%    \begin{macrocode}
\def\HOLOGO@Variant#1{%
  #1%
  \ltx@ifundefined{HoLogoOpt@variant@#1}{%
  }{%
    @\csname HoLogoOpt@variant@#1\endcsname
  }%
}
%    \end{macrocode}
%    \end{macro}
%
% \subsection{Break/no-break support}
%
%    \begin{macro}{\HOLOGO@space}
%    \begin{macrocode}
\def\HOLOGO@space{%
  \ltx@ifundefined{HoLogoOpt@spacebreak@\HOLOGO@name}{%
    \ltx@ifundefined{HoLogoOpt@break@\HOLOGO@name}{%
      \chardef\HOLOGO@temp=\HOLOGOOPT@spacebreak
    }{%
      \chardef\HOLOGO@temp=%
        \csname HoLogoOpt@break@\HOLOGO@name\endcsname
    }%
  }{%
    \chardef\HOLOGO@temp=%
      \csname HoLogoOpt@spacebreak@\HOLOGO@name\endcsname
  }%
  \ifcase\HOLOGO@temp
    \penalty10000 %
  \fi
  \ltx@space
}
%    \end{macrocode}
%    \end{macro}
%
%    \begin{macro}{\HOLOGO@hyphen}
%    \begin{macrocode}
\def\HOLOGO@hyphen{%
  \ltx@ifundefined{HoLogoOpt@hyphenbreak@\HOLOGO@name}{%
    \ltx@ifundefined{HoLogoOpt@break@\HOLOGO@name}{%
      \chardef\HOLOGO@temp=\HOLOGOOPT@hyphenbreak
    }{%
      \chardef\HOLOGO@temp=%
        \csname HoLogoOpt@break@\HOLOGO@name\endcsname
    }%
  }{%
    \chardef\HOLOGO@temp=%
      \csname HoLogoOpt@hyphenbreak@\HOLOGO@name\endcsname
  }%
  \ifcase\HOLOGO@temp
    \ltx@mbox{-}%
  \else
    -%
  \fi
}
%    \end{macrocode}
%    \end{macro}
%
%    \begin{macro}{\HOLOGO@discretionary}
%    \begin{macrocode}
\def\HOLOGO@discretionary{%
  \ltx@ifundefined{HoLogoOpt@discretionarybreak@\HOLOGO@name}{%
    \ltx@ifundefined{HoLogoOpt@break@\HOLOGO@name}{%
      \chardef\HOLOGO@temp=\HOLOGOOPT@discretionarybreak
    }{%
      \chardef\HOLOGO@temp=%
        \csname HoLogoOpt@break@\HOLOGO@name\endcsname
    }%
  }{%
    \chardef\HOLOGO@temp=%
      \csname HoLogoOpt@discretionarybreak@\HOLOGO@name\endcsname
  }%
  \ifcase\HOLOGO@temp
  \else
    \-%
  \fi
}
%    \end{macrocode}
%    \end{macro}
%
%    \begin{macro}{\HOLOGO@mbox}
%    \begin{macrocode}
\def\HOLOGO@mbox#1{%
  \ltx@ifundefined{HoLogoOpt@break@\HOLOGO@name}{%
    \chardef\HOLOGO@temp=\HOLOGOOPT@hyphenbreak
  }{%
    \chardef\HOLOGO@temp=%
      \csname HoLogoOpt@break@\HOLOGO@name\endcsname
  }%
  \ifcase\HOLOGO@temp
    \ltx@mbox{#1}%
  \else
    #1%
  \fi
}
%    \end{macrocode}
%    \end{macro}
%
% \subsection{Font support}
%
%    \begin{macro}{\HoLogoFont@font}
%    \begin{tabular}{@{}ll@{}}
%    |#1|:& logo name\\
%    |#2|:& font short name\\
%    |#3|:& text
%    \end{tabular}
%    \begin{macrocode}
\def\HoLogoFont@font#1#2#3{%
  \begingroup
    \ltx@IfUndefined{HoLogoFont@logo@#1.#2}{%
      \ltx@IfUndefined{HoLogoFont@font@#2}{%
        \@PackageWarning{hologo}{%
          Missing font `#2' for logo `#1'%
        }%
        #3%
      }{%
        \csname HoLogoFont@font@#2\endcsname{#3}%
      }%
    }{%
      \csname HoLogoFont@logo@#1.#2\endcsname{#3}%
    }%
  \endgroup
}
%    \end{macrocode}
%    \end{macro}
%
%    \begin{macro}{\HoLogoFont@Def}
%    \begin{macrocode}
\def\HoLogoFont@Def#1{%
  \expandafter\def\csname HoLogoFont@font@#1\endcsname
}
%    \end{macrocode}
%    \end{macro}
%    \begin{macro}{\HoLogoFont@LogoDef}
%    \begin{macrocode}
\def\HoLogoFont@LogoDef#1#2{%
  \expandafter\def\csname HoLogoFont@logo@#1.#2\endcsname
}
%    \end{macrocode}
%    \end{macro}
%
% \subsubsection{Font defaults}
%
%    \begin{macro}{\HoLogoFont@font@general}
%    \begin{macrocode}
\HoLogoFont@Def{general}{}%
%    \end{macrocode}
%    \end{macro}
%
%    \begin{macro}{\HoLogoFont@font@rm}
%    \begin{macrocode}
\ltx@IfUndefined{rmfamily}{%
  \ltx@IfUndefined{rm}{%
  }{%
    \HoLogoFont@Def{rm}{\rm}%
  }%
}{%
  \HoLogoFont@Def{rm}{\rmfamily}%
}
%    \end{macrocode}
%    \end{macro}
%
%    \begin{macro}{\HoLogoFont@font@sf}
%    \begin{macrocode}
\ltx@IfUndefined{sffamily}{%
  \ltx@IfUndefined{sf}{%
  }{%
    \HoLogoFont@Def{sf}{\sf}%
  }%
}{%
  \HoLogoFont@Def{sf}{\sffamily}%
}
%    \end{macrocode}
%    \end{macro}
%
%    \begin{macro}{\HoLogoFont@font@bibsf}
%    In case of \hologo{plainTeX} the original small caps
%    variant is used as default. In \hologo{LaTeX}
%    the definition of package \xpackage{dtklogos} \cite{dtklogos}
%    is used.
%\begin{quote}
%\begin{verbatim}
%\DeclareRobustCommand{\BibTeX}{%
%  B%
%  \kern-.05em%
%  \hbox{%
%    $\m@th$% %% force math size calculations
%    \csname S@\f@size\endcsname
%    \fontsize\sf@size\z@
%    \math@fontsfalse
%    \selectfont
%    I%
%    \kern-.025em%
%    B
%  }%
%  \kern-.08em%
%  \-%
%  \TeX
%}
%\end{verbatim}
%\end{quote}
%    \begin{macrocode}
\ltx@IfUndefined{selectfont}{%
  \ltx@IfUndefined{tensc}{%
    \font\tensc=cmcsc10\relax
  }{}%
  \HoLogoFont@Def{bibsf}{\tensc}%
}{%
  \HoLogoFont@Def{bibsf}{%
    $\mathsurround=0pt$%
    \csname S@\f@size\endcsname
    \fontsize\sf@size{0pt}%
    \math@fontsfalse
    \selectfont
  }%
}
%    \end{macrocode}
%    \end{macro}
%
%    \begin{macro}{\HoLogoFont@font@sc}
%    \begin{macrocode}
\ltx@IfUndefined{scshape}{%
  \ltx@IfUndefined{tensc}{%
    \font\tensc=cmcsc10\relax
  }{}%
  \HoLogoFont@Def{sc}{\tensc}%
}{%
  \HoLogoFont@Def{sc}{\scshape}%
}
%    \end{macrocode}
%    \end{macro}
%
%    \begin{macro}{\HoLogoFont@font@sy}
%    \begin{macrocode}
\ltx@IfUndefined{usefont}{%
  \ltx@IfUndefined{tensy}{%
  }{%
    \HoLogoFont@Def{sy}{\tensy}%
  }%
}{%
  \HoLogoFont@Def{sy}{%
    \usefont{OMS}{cmsy}{m}{n}%
  }%
}
%    \end{macrocode}
%    \end{macro}
%
%    \begin{macro}{\HoLogoFont@font@logo}
%    \begin{macrocode}
\begingroup
  \def\x{LaTeX2e}%
\expandafter\endgroup
\ifx\fmtname\x
  \ltx@IfUndefined{logofamily}{%
    \DeclareRobustCommand\logofamily{%
      \not@math@alphabet\logofamily\relax
      \fontencoding{U}%
      \fontfamily{logo}%
      \selectfont
    }%
  }{}%
  \ltx@IfUndefined{logofamily}{%
  }{%
    \HoLogoFont@Def{logo}{\logofamily}%
  }%
\else
  \ltx@IfUndefined{tenlogo}{%
    \font\tenlogo=logo10\relax
  }{}%
  \HoLogoFont@Def{logo}{\tenlogo}%
\fi
%    \end{macrocode}
%    \end{macro}
%
% \subsubsection{Font setup}
%
%    \begin{macro}{\hologoFontSetup}
%    \begin{macrocode}
\def\hologoFontSetup{%
  \let\HOLOGO@name\relax
  \HOLOGO@FontSetup
}
%    \end{macrocode}
%    \end{macro}
%
%    \begin{macro}{\hologoLogoFontSetup}
%    \begin{macrocode}
\def\hologoLogoFontSetup#1{%
  \edef\HOLOGO@name{#1}%
  \ltx@IfUndefined{HoLogo@\HOLOGO@name}{%
    \@PackageError{hologo}{%
      Unknown logo `\HOLOGO@name'%
    }\@ehc
    \ltx@gobble
  }{%
    \HOLOGO@FontSetup
  }%
}
%    \end{macrocode}
%    \end{macro}
%
%    \begin{macro}{\HOLOGO@FontSetup}
%    \begin{macrocode}
\def\HOLOGO@FontSetup{%
  \kvsetkeys{HoLogoFont}%
}
%    \end{macrocode}
%    \end{macro}
%
%    \begin{macrocode}
\def\HOLOGO@temp#1{%
  \kv@define@key{HoLogoFont}{#1}{%
    \ifx\HOLOGO@name\relax
      \HoLogoFont@Def{#1}{##1}%
    \else
      \HoLogoFont@LogoDef\HOLOGO@name{#1}{##1}%
    \fi
  }%
}
\HOLOGO@temp{general}
\HOLOGO@temp{sf}
%    \end{macrocode}
%
% \subsection{Generic logo commands}
%
%    \begin{macrocode}
\HOLOGO@IfExists\hologo{%
  \@PackageError{hologo}{%
    \string\hologo\ltx@space is already defined.\MessageBreak
    Package loading is aborted%
  }\@ehc
  \HOLOGO@AtEnd
}%
\HOLOGO@IfExists\hologoRobust{%
  \@PackageError{hologo}{%
    \string\hologoRobust\ltx@space is already defined.\MessageBreak
    Package loading is aborted%
  }\@ehc
  \HOLOGO@AtEnd
}%
%    \end{macrocode}
%
% \subsubsection{\cs{hologo} and friends}
%
%    \begin{macrocode}
\ifluatex
  \expandafter\ltx@firstofone
\else
  \expandafter\ltx@gobble
\fi
{%
  \ltx@IfUndefined{ifincsname}{%
    \ifnum\luatexversion<36 %
      \expandafter\ltx@gobble
    \else
      \expandafter\ltx@firstofone
    \fi
    {%
      \begingroup
        \ifcase0%
            \directlua{%
              if tex.enableprimitives then %
                tex.enableprimitives('HOLOGO@', {'ifincsname'})%
              else %
                tex.print('1')%
              end%
            }%
            \ifx\HOLOGO@ifincsname\@undefined 1\fi%
            \relax
          \expandafter\ltx@firstofone
        \else
          \endgroup
          \expandafter\ltx@gobble
        \fi
        {%
          \global\let\ifincsname\HOLOGO@ifincsname
        }%
      \HOLOGO@temp
    }%
  }{}%
}
%    \end{macrocode}
%    \begin{macrocode}
\ltx@IfUndefined{ifincsname}{%
  \catcode`$=14 %
}{%
  \catcode`$=9 %
}
%    \end{macrocode}
%
%    \begin{macro}{\hologo}
%    \begin{macrocode}
\def\hologo#1{%
$ \ifincsname
$   \ltx@ifundefined{HoLogoCs@\HOLOGO@Variant{#1}}{%
$     #1%
$   }{%
$     \csname HoLogoCs@\HOLOGO@Variant{#1}\endcsname\ltx@firstoftwo
$   }%
$ \else
    \HOLOGO@IfExists\texorpdfstring\texorpdfstring\ltx@firstoftwo
    {%
      \hologoRobust{#1}%
    }{%
      \ltx@ifundefined{HoLogoBkm@\HOLOGO@Variant{#1}}{%
        \ltx@ifundefined{HoLogo@#1}{?#1?}{#1}%
      }{%
        \csname HoLogoBkm@\HOLOGO@Variant{#1}\endcsname
        \ltx@firstoftwo
      }%
    }%
$ \fi
}
%    \end{macrocode}
%    \end{macro}
%    \begin{macro}{\Hologo}
%    \begin{macrocode}
\def\Hologo#1{%
$ \ifincsname
$   \ltx@ifundefined{HoLogoCs@\HOLOGO@Variant{#1}}{%
$     #1%
$   }{%
$     \csname HoLogoCs@\HOLOGO@Variant{#1}\endcsname\ltx@secondoftwo
$   }%
$ \else
    \HOLOGO@IfExists\texorpdfstring\texorpdfstring\ltx@firstoftwo
    {%
      \HologoRobust{#1}%
    }{%
      \ltx@ifundefined{HoLogoBkm@\HOLOGO@Variant{#1}}{%
        \ltx@ifundefined{HoLogo@#1}{?#1?}{#1}%
      }{%
        \csname HoLogoBkm@\HOLOGO@Variant{#1}\endcsname
        \ltx@secondoftwo
      }%
    }%
$ \fi
}
%    \end{macrocode}
%    \end{macro}
%
%    \begin{macro}{\hologoVariant}
%    \begin{macrocode}
\def\hologoVariant#1#2{%
  \ifx\relax#2\relax
    \hologo{#1}%
  \else
$   \ifincsname
$     \ltx@ifundefined{HoLogoCs@#1@#2}{%
$       #1%
$     }{%
$       \csname HoLogoCs@#1@#2\endcsname\ltx@firstoftwo
$     }%
$   \else
      \HOLOGO@IfExists\texorpdfstring\texorpdfstring\ltx@firstoftwo
      {%
        \hologoVariantRobust{#1}{#2}%
      }{%
        \ltx@ifundefined{HoLogoBkm@#1@#2}{%
          \ltx@ifundefined{HoLogo@#1}{?#1?}{#1}%
        }{%
          \csname HoLogoBkm@#1@#2\endcsname
          \ltx@firstoftwo
        }%
      }%
$   \fi
  \fi
}
%    \end{macrocode}
%    \end{macro}
%    \begin{macro}{\HologoVariant}
%    \begin{macrocode}
\def\HologoVariant#1#2{%
  \ifx\relax#2\relax
    \Hologo{#1}%
  \else
$   \ifincsname
$     \ltx@ifundefined{HoLogoCs@#1@#2}{%
$       #1%
$     }{%
$       \csname HoLogoCs@#1@#2\endcsname\ltx@secondoftwo
$     }%
$   \else
      \HOLOGO@IfExists\texorpdfstring\texorpdfstring\ltx@firstoftwo
      {%
        \HologoVariantRobust{#1}{#2}%
      }{%
        \ltx@ifundefined{HoLogoBkm@#1@#2}{%
          \ltx@ifundefined{HoLogo@#1}{?#1?}{#1}%
        }{%
          \csname HoLogoBkm@#1@#2\endcsname
          \ltx@secondoftwo
        }%
      }%
$   \fi
  \fi
}
%    \end{macrocode}
%    \end{macro}
%
%    \begin{macrocode}
\catcode`\$=3 %
%    \end{macrocode}
%
% \subsubsection{\cs{hologoRobust} and friends}
%
%    \begin{macro}{\hologoRobust}
%    \begin{macrocode}
\ltx@IfUndefined{protected}{%
  \ltx@IfUndefined{DeclareRobustCommand}{%
    \def\hologoRobust#1%
  }{%
    \DeclareRobustCommand*\hologoRobust[1]%
  }%
}{%
  \protected\def\hologoRobust#1%
}%
{%
  \edef\HOLOGO@name{#1}%
  \ltx@IfUndefined{HoLogo@\HOLOGO@Variant\HOLOGO@name}{%
    \@PackageError{hologo}{%
      Unknown logo `\HOLOGO@name'%
    }\@ehc
    ?\HOLOGO@name?%
  }{%
    \ltx@IfUndefined{ver@tex4ht.sty}{%
      \HoLogoFont@font\HOLOGO@name{general}{%
        \csname HoLogo@\HOLOGO@Variant\HOLOGO@name\endcsname
        \ltx@firstoftwo
      }%
    }{%
      \ltx@IfUndefined{HoLogoHtml@\HOLOGO@Variant\HOLOGO@name}{%
        \HOLOGO@name
      }{%
        \csname HoLogoHtml@\HOLOGO@Variant\HOLOGO@name\endcsname
        \ltx@firstoftwo
      }%
    }%
  }%
}
%    \end{macrocode}
%    \end{macro}
%    \begin{macro}{\HologoRobust}
%    \begin{macrocode}
\ltx@IfUndefined{protected}{%
  \ltx@IfUndefined{DeclareRobustCommand}{%
    \def\HologoRobust#1%
  }{%
    \DeclareRobustCommand*\HologoRobust[1]%
  }%
}{%
  \protected\def\HologoRobust#1%
}%
{%
  \edef\HOLOGO@name{#1}%
  \ltx@IfUndefined{HoLogo@\HOLOGO@Variant\HOLOGO@name}{%
    \@PackageError{hologo}{%
      Unknown logo `\HOLOGO@name'%
    }\@ehc
    ?\HOLOGO@name?%
  }{%
    \ltx@IfUndefined{ver@tex4ht.sty}{%
      \HoLogoFont@font\HOLOGO@name{general}{%
        \csname HoLogo@\HOLOGO@Variant\HOLOGO@name\endcsname
        \ltx@secondoftwo
      }%
    }{%
      \ltx@IfUndefined{HoLogoHtml@\HOLOGO@Variant\HOLOGO@name}{%
        \expandafter\HOLOGO@Uppercase\HOLOGO@name
      }{%
        \csname HoLogoHtml@\HOLOGO@Variant\HOLOGO@name\endcsname
        \ltx@secondoftwo
      }%
    }%
  }%
}
%    \end{macrocode}
%    \end{macro}
%    \begin{macro}{\hologoVariantRobust}
%    \begin{macrocode}
\ltx@IfUndefined{protected}{%
  \ltx@IfUndefined{DeclareRobustCommand}{%
    \def\hologoVariantRobust#1#2%
  }{%
    \DeclareRobustCommand*\hologoVariantRobust[2]%
  }%
}{%
  \protected\def\hologoVariantRobust#1#2%
}%
{%
  \begingroup
    \hologoLogoSetup{#1}{variant={#2}}%
    \hologoRobust{#1}%
  \endgroup
}
%    \end{macrocode}
%    \end{macro}
%    \begin{macro}{\HologoVariantRobust}
%    \begin{macrocode}
\ltx@IfUndefined{protected}{%
  \ltx@IfUndefined{DeclareRobustCommand}{%
    \def\HologoVariantRobust#1#2%
  }{%
    \DeclareRobustCommand*\HologoVariantRobust[2]%
  }%
}{%
  \protected\def\HologoVariantRobust#1#2%
}%
{%
  \begingroup
    \hologoLogoSetup{#1}{variant={#2}}%
    \HologoRobust{#1}%
  \endgroup
}
%    \end{macrocode}
%    \end{macro}
%
%    \begin{macro}{\hologorobust}
%    Macro \cs{hologorobust} is only defined for compatibility.
%    Its use is deprecated.
%    \begin{macrocode}
\def\hologorobust{\hologoRobust}
%    \end{macrocode}
%    \end{macro}
%
% \subsection{Helpers}
%
%    \begin{macro}{\HOLOGO@Uppercase}
%    Macro \cs{HOLOGO@Uppercase} is restricted to \cs{uppercase},
%    because \hologo{plainTeX} or \hologo{iniTeX} do not provide
%    \cs{MakeUppercase}.
%    \begin{macrocode}
\def\HOLOGO@Uppercase#1{\uppercase{#1}}
%    \end{macrocode}
%    \end{macro}
%
%    \begin{macro}{\HOLOGO@PdfdocUnicode}
%    \begin{macrocode}
\def\HOLOGO@PdfdocUnicode{%
  \ifx\ifHy@unicode\iftrue
    \expandafter\ltx@secondoftwo
  \else
    \expandafter\ltx@firstoftwo
  \fi
}
%    \end{macrocode}
%    \end{macro}
%
%    \begin{macro}{\HOLOGO@Math}
%    \begin{macrocode}
\def\HOLOGO@MathSetup{%
  \mathsurround0pt\relax
  \HOLOGO@IfExists\f@series{%
    \if b\expandafter\ltx@car\f@series x\@nil
      \csname boldmath\endcsname
   \fi
  }{}%
}
%    \end{macrocode}
%    \end{macro}
%
%    \begin{macro}{\HOLOGO@TempDimen}
%    \begin{macrocode}
\dimendef\HOLOGO@TempDimen=\ltx@zero
%    \end{macrocode}
%    \end{macro}
%    \begin{macro}{\HOLOGO@NegativeKerning}
%    \begin{macrocode}
\def\HOLOGO@NegativeKerning#1{%
  \begingroup
    \HOLOGO@TempDimen=0pt\relax
    \comma@parse@normalized{#1}{%
      \ifdim\HOLOGO@TempDimen=0pt %
        \expandafter\HOLOGO@@NegativeKerning\comma@entry
      \fi
      \ltx@gobble
    }%
    \ifdim\HOLOGO@TempDimen<0pt %
      \kern\HOLOGO@TempDimen
    \fi
  \endgroup
}
%    \end{macrocode}
%    \end{macro}
%    \begin{macro}{\HOLOGO@@NegativeKerning}
%    \begin{macrocode}
\def\HOLOGO@@NegativeKerning#1#2{%
  \setbox\ltx@zero\hbox{#1#2}%
  \HOLOGO@TempDimen=\wd\ltx@zero
  \setbox\ltx@zero\hbox{#1\kern0pt#2}%
  \advance\HOLOGO@TempDimen by -\wd\ltx@zero
}
%    \end{macrocode}
%    \end{macro}
%
%    \begin{macro}{\HOLOGO@SpaceFactor}
%    \begin{macrocode}
\def\HOLOGO@SpaceFactor{%
  \spacefactor1000 %
}
%    \end{macrocode}
%    \end{macro}
%
%    \begin{macro}{\HOLOGO@Span}
%    \begin{macrocode}
\def\HOLOGO@Span#1#2{%
  \HCode{<span class="HoLogo-#1">}%
  #2%
  \HCode{</span>}%
}
%    \end{macrocode}
%    \end{macro}
%
% \subsubsection{Text subscript}
%
%    \begin{macro}{\HOLOGO@SubScript}%
%    \begin{macrocode}
\def\HOLOGO@SubScript#1{%
  \ltx@IfUndefined{textsubscript}{%
    \ltx@IfUndefined{text}{%
      \ltx@mbox{%
        \mathsurround=0pt\relax
        $%
          _{%
            \ltx@IfUndefined{sf@size}{%
              \mathrm{#1}%
            }{%
              \mbox{%
                \fontsize\sf@size{0pt}\selectfont
                #1%
              }%
            }%
          }%
        $%
      }%
    }{%
      \ltx@mbox{%
        \mathsurround=0pt\relax
        $_{\text{#1}}$%
      }%
    }%
  }{%
    \textsubscript{#1}%
  }%
}
%    \end{macrocode}
%    \end{macro}
%
% \subsection{\hologo{TeX} and friends}
%
% \subsubsection{\hologo{TeX}}
%
%    \begin{macro}{\HoLogo@TeX}
%    Source: \hologo{LaTeX} kernel.
%    \begin{macrocode}
\def\HoLogo@TeX#1{%
  T\kern-.1667em\lower.5ex\hbox{E}\kern-.125emX\HOLOGO@SpaceFactor
}
%    \end{macrocode}
%    \end{macro}
%    \begin{macro}{\HoLogoHtml@TeX}
%    \begin{macrocode}
\def\HoLogoHtml@TeX#1{%
  \HoLogoCss@TeX
  \HOLOGO@Span{TeX}{%
    T%
    \HOLOGO@Span{e}{%
      E%
    }%
    X%
  }%
}
%    \end{macrocode}
%    \end{macro}
%    \begin{macro}{\HoLogoCss@TeX}
%    \begin{macrocode}
\def\HoLogoCss@TeX{%
  \Css{%
    span.HoLogo-TeX span.HoLogo-e{%
      position:relative;%
      top:.5ex;%
      margin-left:-.1667em;%
      margin-right:-.125em;%
    }%
  }%
  \Css{%
    a span.HoLogo-TeX span.HoLogo-e{%
      text-decoration:none;%
    }%
  }%
  \global\let\HoLogoCss@TeX\relax
}
%    \end{macrocode}
%    \end{macro}
%
% \subsubsection{\hologo{plainTeX}}
%
%    \begin{macro}{\HoLogo@plainTeX@space}
%    Source: ``The \hologo{TeX}book''
%    \begin{macrocode}
\def\HoLogo@plainTeX@space#1{%
  \HOLOGO@mbox{#1{p}{P}lain}\HOLOGO@space\hologo{TeX}%
}
%    \end{macrocode}
%    \end{macro}
%    \begin{macro}{\HoLogoCs@plainTeX@space}
%    \begin{macrocode}
\def\HoLogoCs@plainTeX@space#1{#1{p}{P}lain TeX}%
%    \end{macrocode}
%    \end{macro}
%    \begin{macro}{\HoLogoBkm@plainTeX@space}
%    \begin{macrocode}
\def\HoLogoBkm@plainTeX@space#1{%
  #1{p}{P}lain \hologo{TeX}%
}
%    \end{macrocode}
%    \end{macro}
%    \begin{macro}{\HoLogoHtml@plainTeX@space}
%    \begin{macrocode}
\def\HoLogoHtml@plainTeX@space#1{%
  #1{p}{P}lain \hologo{TeX}%
}
%    \end{macrocode}
%    \end{macro}
%
%    \begin{macro}{\HoLogo@plainTeX@hyphen}
%    \begin{macrocode}
\def\HoLogo@plainTeX@hyphen#1{%
  \HOLOGO@mbox{#1{p}{P}lain}\HOLOGO@hyphen\hologo{TeX}%
}
%    \end{macrocode}
%    \end{macro}
%    \begin{macro}{\HoLogoCs@plainTeX@hyphen}
%    \begin{macrocode}
\def\HoLogoCs@plainTeX@hyphen#1{#1{p}{P}lain-TeX}
%    \end{macrocode}
%    \end{macro}
%    \begin{macro}{\HoLogoBkm@plainTeX@hyphen}
%    \begin{macrocode}
\def\HoLogoBkm@plainTeX@hyphen#1{%
  #1{p}{P}lain-\hologo{TeX}%
}
%    \end{macrocode}
%    \end{macro}
%    \begin{macro}{\HoLogoHtml@plainTeX@hyphen}
%    \begin{macrocode}
\def\HoLogoHtml@plainTeX@hyphen#1{%
  #1{p}{P}lain-\hologo{TeX}%
}
%    \end{macrocode}
%    \end{macro}
%
%    \begin{macro}{\HoLogo@plainTeX@runtogether}
%    \begin{macrocode}
\def\HoLogo@plainTeX@runtogether#1{%
  \HOLOGO@mbox{#1{p}{P}lain\hologo{TeX}}%
}
%    \end{macrocode}
%    \end{macro}
%    \begin{macro}{\HoLogoCs@plainTeX@runtogether}
%    \begin{macrocode}
\def\HoLogoCs@plainTeX@runtogether#1{#1{p}{P}lainTeX}
%    \end{macrocode}
%    \end{macro}
%    \begin{macro}{\HoLogoBkm@plainTeX@runtogether}
%    \begin{macrocode}
\def\HoLogoBkm@plainTeX@runtogether#1{%
  #1{p}{P}lain\hologo{TeX}%
}
%    \end{macrocode}
%    \end{macro}
%    \begin{macro}{\HoLogoHtml@plainTeX@runtogether}
%    \begin{macrocode}
\def\HoLogoHtml@plainTeX@runtogether#1{%
  #1{p}{P}lain\hologo{TeX}%
}
%    \end{macrocode}
%    \end{macro}
%
%    \begin{macro}{\HoLogo@plainTeX}
%    \begin{macrocode}
\def\HoLogo@plainTeX{\HoLogo@plainTeX@space}
%    \end{macrocode}
%    \end{macro}
%    \begin{macro}{\HoLogoCs@plainTeX}
%    \begin{macrocode}
\def\HoLogoCs@plainTeX{\HoLogoCs@plainTeX@space}
%    \end{macrocode}
%    \end{macro}
%    \begin{macro}{\HoLogoBkm@plainTeX}
%    \begin{macrocode}
\def\HoLogoBkm@plainTeX{\HoLogoBkm@plainTeX@space}
%    \end{macrocode}
%    \end{macro}
%    \begin{macro}{\HoLogoHtml@plainTeX}
%    \begin{macrocode}
\def\HoLogoHtml@plainTeX{\HoLogoHtml@plainTeX@space}
%    \end{macrocode}
%    \end{macro}
%
% \subsubsection{\hologo{LaTeX}}
%
%    Source: \hologo{LaTeX} kernel.
%\begin{quote}
%\begin{verbatim}
%\DeclareRobustCommand{\LaTeX}{%
%  L%
%  \kern-.36em%
%  {%
%    \sbox\z@ T%
%    \vbox to\ht\z@{%
%      \hbox{%
%        \check@mathfonts
%        \fontsize\sf@size\z@
%        \math@fontsfalse
%        \selectfont
%        A%
%      }%
%      \vss
%    }%
%  }%
%  \kern-.15em%
%  \TeX
%}
%\end{verbatim}
%\end{quote}
%
%    \begin{macro}{\HoLogo@La}
%    \begin{macrocode}
\def\HoLogo@La#1{%
  L%
  \kern-.36em%
  \begingroup
    \setbox\ltx@zero\hbox{T}%
    \vbox to\ht\ltx@zero{%
      \hbox{%
        \ltx@ifundefined{check@mathfonts}{%
          \csname sevenrm\endcsname
        }{%
          \check@mathfonts
          \fontsize\sf@size{0pt}%
          \math@fontsfalse\selectfont
        }%
        A%
      }%
      \vss
    }%
  \endgroup
}
%    \end{macrocode}
%    \end{macro}
%
%    \begin{macro}{\HoLogo@LaTeX}
%    Source: \hologo{LaTeX} kernel.
%    \begin{macrocode}
\def\HoLogo@LaTeX#1{%
  \hologo{La}%
  \kern-.15em%
  \hologo{TeX}%
}
%    \end{macrocode}
%    \end{macro}
%    \begin{macro}{\HoLogoHtml@LaTeX}
%    \begin{macrocode}
\def\HoLogoHtml@LaTeX#1{%
  \HoLogoCss@LaTeX
  \HOLOGO@Span{LaTeX}{%
    L%
    \HOLOGO@Span{a}{%
      A%
    }%
    \hologo{TeX}%
  }%
}
%    \end{macrocode}
%    \end{macro}
%    \begin{macro}{\HoLogoCss@LaTeX}
%    \begin{macrocode}
\def\HoLogoCss@LaTeX{%
  \Css{%
    span.HoLogo-LaTeX span.HoLogo-a{%
      position:relative;%
      top:-.5ex;%
      margin-left:-.36em;%
      margin-right:-.15em;%
      font-size:85\%;%
    }%
  }%
  \global\let\HoLogoCss@LaTeX\relax
}
%    \end{macrocode}
%    \end{macro}
%
% \subsubsection{\hologo{(La)TeX}}
%
%    \begin{macro}{\HoLogo@LaTeXTeX}
%    The kerning around the parentheses is taken
%    from package \xpackage{dtklogos} \cite{dtklogos}.
%\begin{quote}
%\begin{verbatim}
%\DeclareRobustCommand{\LaTeXTeX}{%
%  (%
%  \kern-.15em%
%  L%
%  \kern-.36em%
%  {%
%    \sbox\z@ T%
%    \vbox to\ht0{%
%      \hbox{%
%        $\m@th$%
%        \csname S@\f@size\endcsname
%        \fontsize\sf@size\z@
%        \math@fontsfalse
%        \selectfont
%        A%
%      }%
%      \vss
%    }%
%  }%
%  \kern-.2em%
%  )%
%  \kern-.15em%
%  \TeX
%}
%\end{verbatim}
%\end{quote}
%    \begin{macrocode}
\def\HoLogo@LaTeXTeX#1{%
  (%
  \kern-.15em%
  \hologo{La}%
  \kern-.2em%
  )%
  \kern-.15em%
  \hologo{TeX}%
}
%    \end{macrocode}
%    \end{macro}
%    \begin{macro}{\HoLogoBkm@LaTeXTeX}
%    \begin{macrocode}
\def\HoLogoBkm@LaTeXTeX#1{(La)TeX}
%    \end{macrocode}
%    \end{macro}
%
%    \begin{macro}{\HoLogo@(La)TeX}
%    \begin{macrocode}
\expandafter
\let\csname HoLogo@(La)TeX\endcsname\HoLogo@LaTeXTeX
%    \end{macrocode}
%    \end{macro}
%    \begin{macro}{\HoLogoBkm@(La)TeX}
%    \begin{macrocode}
\expandafter
\let\csname HoLogoBkm@(La)TeX\endcsname\HoLogoBkm@LaTeXTeX
%    \end{macrocode}
%    \end{macro}
%    \begin{macro}{\HoLogoHtml@LaTeXTeX}
%    \begin{macrocode}
\def\HoLogoHtml@LaTeXTeX#1{%
  \HoLogoCss@LaTeXTeX
  \HOLOGO@Span{LaTeXTeX}{%
    (%
    \HOLOGO@Span{L}{L}%
    \HOLOGO@Span{a}{A}%
    \HOLOGO@Span{ParenRight}{)}%
    \hologo{TeX}%
  }%
}
%    \end{macrocode}
%    \end{macro}
%    \begin{macro}{\HoLogoHtml@(La)TeX}
%    Kerning after opening parentheses and before closing parentheses
%    is $-0.1$\,em. The original values $-0.15$\,em
%    looked too ugly for a serif font.
%    \begin{macrocode}
\expandafter
\let\csname HoLogoHtml@(La)TeX\endcsname\HoLogoHtml@LaTeXTeX
%    \end{macrocode}
%    \end{macro}
%    \begin{macro}{\HoLogoCss@LaTeXTeX}
%    \begin{macrocode}
\def\HoLogoCss@LaTeXTeX{%
  \Css{%
    span.HoLogo-LaTeXTeX span.HoLogo-L{%
      margin-left:-.1em;%
    }%
  }%
  \Css{%
    span.HoLogo-LaTeXTeX span.HoLogo-a{%
      position:relative;%
      top:-.5ex;%
      margin-left:-.36em;%
      margin-right:-.1em;%
      font-size:85\%;%
    }%
  }%
  \Css{%
    span.HoLogo-LaTeXTeX span.HoLogo-ParenRight{%
      margin-right:-.15em;%
    }%
  }%
  \global\let\HoLogoCss@LaTeXTeX\relax
}
%    \end{macrocode}
%    \end{macro}
%
% \subsubsection{\hologo{LaTeXe}}
%
%    \begin{macro}{\HoLogo@LaTeXe}
%    Source: \hologo{LaTeX} kernel
%    \begin{macrocode}
\def\HoLogo@LaTeXe#1{%
  \hologo{LaTeX}%
  \kern.15em%
  \hbox{%
    \HOLOGO@MathSetup
    2%
    $_{\textstyle\varepsilon}$%
  }%
}
%    \end{macrocode}
%    \end{macro}
%
%    \begin{macro}{\HoLogoCs@LaTeXe}
%    \begin{macrocode}
\ifnum64=`\^^^^0040\relax % test for big chars of LuaTeX/XeTeX
  \catcode`\$=9 %
  \catcode`\&=14 %
\else
  \catcode`\$=14 %
  \catcode`\&=9 %
\fi
\def\HoLogoCs@LaTeXe#1{%
  LaTeX2%
$ \string ^^^^0395%
& e%
}%
\catcode`\$=3 %
\catcode`\&=4 %
%    \end{macrocode}
%    \end{macro}
%
%    \begin{macro}{\HoLogoBkm@LaTeXe}
%    \begin{macrocode}
\def\HoLogoBkm@LaTeXe#1{%
  \hologo{LaTeX}%
  2%
  \HOLOGO@PdfdocUnicode{e}{\textepsilon}%
}
%    \end{macrocode}
%    \end{macro}
%
%    \begin{macro}{\HoLogoHtml@LaTeXe}
%    \begin{macrocode}
\def\HoLogoHtml@LaTeXe#1{%
  \HoLogoCss@LaTeXe
  \HOLOGO@Span{LaTeX2e}{%
    \hologo{LaTeX}%
    \HOLOGO@Span{2}{2}%
    \HOLOGO@Span{e}{%
      \HOLOGO@MathSetup
      \ensuremath{\textstyle\varepsilon}%
    }%
  }%
}
%    \end{macrocode}
%    \end{macro}
%    \begin{macro}{\HoLogoCss@LaTeXe}
%    \begin{macrocode}
\def\HoLogoCss@LaTeXe{%
  \Css{%
    span.HoLogo-LaTeX2e span.HoLogo-2{%
      padding-left:.15em;%
    }%
  }%
  \Css{%
    span.HoLogo-LaTeX2e span.HoLogo-e{%
      position:relative;%
      top:.35ex;%
      text-decoration:none;%
    }%
  }%
  \global\let\HoLogoCss@LaTeXe\relax
}
%    \end{macrocode}
%    \end{macro}
%
%    \begin{macro}{\HoLogo@LaTeX2e}
%    \begin{macrocode}
\expandafter
\let\csname HoLogo@LaTeX2e\endcsname\HoLogo@LaTeXe
%    \end{macrocode}
%    \end{macro}
%    \begin{macro}{\HoLogoCs@LaTeX2e}
%    \begin{macrocode}
\expandafter
\let\csname HoLogoCs@LaTeX2e\endcsname\HoLogoCs@LaTeXe
%    \end{macrocode}
%    \end{macro}
%    \begin{macro}{\HoLogoBkm@LaTeX2e}
%    \begin{macrocode}
\expandafter
\let\csname HoLogoBkm@LaTeX2e\endcsname\HoLogoBkm@LaTeXe
%    \end{macrocode}
%    \end{macro}
%    \begin{macro}{\HoLogoHtml@LaTeX2e}
%    \begin{macrocode}
\expandafter
\let\csname HoLogoHtml@LaTeX2e\endcsname\HoLogoHtml@LaTeXe
%    \end{macrocode}
%    \end{macro}
%
% \subsubsection{\hologo{LaTeX3}}
%
%    \begin{macro}{\HoLogo@LaTeX3}
%    Source: \hologo{LaTeX} kernel
%    \begin{macrocode}
\expandafter\def\csname HoLogo@LaTeX3\endcsname#1{%
  \hologo{LaTeX}%
  3%
}
%    \end{macrocode}
%    \end{macro}
%
%    \begin{macro}{\HoLogoBkm@LaTeX3}
%    \begin{macrocode}
\expandafter\def\csname HoLogoBkm@LaTeX3\endcsname#1{%
  \hologo{LaTeX}%
  3%
}
%    \end{macrocode}
%    \end{macro}
%    \begin{macro}{\HoLogoHtml@LaTeX3}
%    \begin{macrocode}
\expandafter
\let\csname HoLogoHtml@LaTeX3\expandafter\endcsname
\csname HoLogo@LaTeX3\endcsname
%    \end{macrocode}
%    \end{macro}
%
% \subsubsection{\hologo{LaTeXML}}
%
%    \begin{macro}{\HoLogo@LaTeXML}
%    \begin{macrocode}
\def\HoLogo@LaTeXML#1{%
  \HOLOGO@mbox{%
    \hologo{La}%
    \kern-.15em%
    T%
    \kern-.1667em%
    \lower.5ex\hbox{E}%
    \kern-.125em%
    \HoLogoFont@font{LaTeXML}{sc}{xml}%
  }%
}
%    \end{macrocode}
%    \end{macro}
%    \begin{macro}{\HoLogoHtml@pdfLaTeX}
%    \begin{macrocode}
\def\HoLogoHtml@LaTeXML#1{%
  \HOLOGO@Span{LaTeXML}{%
    \HoLogoCss@LaTeX
    \HoLogoCss@TeX
    \HOLOGO@Span{LaTeX}{%
      L%
      \HOLOGO@Span{a}{%
        A%
      }%
    }%
    \HOLOGO@Span{TeX}{%
      T%
      \HOLOGO@Span{e}{%
        E%
      }%
    }%
    \HCode{<span style="font-variant: small-caps;">}%
    xml%
    \HCode{</span>}%
  }%
}
%    \end{macrocode}
%    \end{macro}
%
% \subsubsection{\hologo{eTeX}}
%
%    \begin{macro}{\HoLogo@eTeX}
%    Source: package \xpackage{etex}
%    \begin{macrocode}
\def\HoLogo@eTeX#1{%
  \ltx@mbox{%
    \HOLOGO@MathSetup
    $\varepsilon$%
    -%
    \HOLOGO@NegativeKerning{-T,T-,To}%
    \hologo{TeX}%
  }%
}
%    \end{macrocode}
%    \end{macro}
%    \begin{macro}{\HoLogoCs@eTeX}
%    \begin{macrocode}
\ifnum64=`\^^^^0040\relax % test for big chars of LuaTeX/XeTeX
  \catcode`\$=9 %
  \catcode`\&=14 %
\else
  \catcode`\$=14 %
  \catcode`\&=9 %
\fi
\def\HoLogoCs@eTeX#1{%
$ #1{\string ^^^^0395}{\string ^^^^03b5}%
& #1{e}{E}%
  TeX%
}%
\catcode`\$=3 %
\catcode`\&=4 %
%    \end{macrocode}
%    \end{macro}
%    \begin{macro}{\HoLogoBkm@eTeX}
%    \begin{macrocode}
\def\HoLogoBkm@eTeX#1{%
  \HOLOGO@PdfdocUnicode{#1{e}{E}}{\textepsilon}%
  -%
  \hologo{TeX}%
}
%    \end{macrocode}
%    \end{macro}
%    \begin{macro}{\HoLogoHtml@eTeX}
%    \begin{macrocode}
\def\HoLogoHtml@eTeX#1{%
  \ltx@mbox{%
    \HOLOGO@MathSetup
    $\varepsilon$%
    -%
    \hologo{TeX}%
  }%
}
%    \end{macrocode}
%    \end{macro}
%
% \subsubsection{\hologo{iniTeX}}
%
%    \begin{macro}{\HoLogo@iniTeX}
%    \begin{macrocode}
\def\HoLogo@iniTeX#1{%
  \HOLOGO@mbox{%
    #1{i}{I}ni\hologo{TeX}%
  }%
}
%    \end{macrocode}
%    \end{macro}
%    \begin{macro}{\HoLogoCs@iniTeX}
%    \begin{macrocode}
\def\HoLogoCs@iniTeX#1{#1{i}{I}niTeX}
%    \end{macrocode}
%    \end{macro}
%    \begin{macro}{\HoLogoBkm@iniTeX}
%    \begin{macrocode}
\def\HoLogoBkm@iniTeX#1{%
  #1{i}{I}ni\hologo{TeX}%
}
%    \end{macrocode}
%    \end{macro}
%    \begin{macro}{\HoLogoHtml@iniTeX}
%    \begin{macrocode}
\let\HoLogoHtml@iniTeX\HoLogo@iniTeX
%    \end{macrocode}
%    \end{macro}
%
% \subsubsection{\hologo{virTeX}}
%
%    \begin{macro}{\HoLogo@virTeX}
%    \begin{macrocode}
\def\HoLogo@virTeX#1{%
  \HOLOGO@mbox{%
    #1{v}{V}ir\hologo{TeX}%
  }%
}
%    \end{macrocode}
%    \end{macro}
%    \begin{macro}{\HoLogoCs@virTeX}
%    \begin{macrocode}
\def\HoLogoCs@virTeX#1{#1{v}{V}irTeX}
%    \end{macrocode}
%    \end{macro}
%    \begin{macro}{\HoLogoBkm@virTeX}
%    \begin{macrocode}
\def\HoLogoBkm@virTeX#1{%
  #1{v}{V}ir\hologo{TeX}%
}
%    \end{macrocode}
%    \end{macro}
%    \begin{macro}{\HoLogoHtml@virTeX}
%    \begin{macrocode}
\let\HoLogoHtml@virTeX\HoLogo@virTeX
%    \end{macrocode}
%    \end{macro}
%
% \subsubsection{\hologo{SliTeX}}
%
% \paragraph{Definitions of the three variants.}
%
%    \begin{macro}{\HoLogo@SLiTeX@lift}
%    \begin{macrocode}
\def\HoLogo@SLiTeX@lift#1{%
  \HoLogoFont@font{SliTeX}{rm}{%
    S%
    \kern-.06em%
    L%
    \kern-.18em%
    \raise.32ex\hbox{\HoLogoFont@font{SliTeX}{sc}{i}}%
    \HOLOGO@discretionary
    \kern-.06em%
    \hologo{TeX}%
  }%
}
%    \end{macrocode}
%    \end{macro}
%    \begin{macro}{\HoLogoBkm@SLiTeX@lift}
%    \begin{macrocode}
\def\HoLogoBkm@SLiTeX@lift#1{SLiTeX}
%    \end{macrocode}
%    \end{macro}
%    \begin{macro}{\HoLogoHtml@SLiTeX@lift}
%    \begin{macrocode}
\def\HoLogoHtml@SLiTeX@lift#1{%
  \HoLogoCss@SLiTeX@lift
  \HOLOGO@Span{SLiTeX-lift}{%
    \HoLogoFont@font{SliTeX}{rm}{%
      S%
      \HOLOGO@Span{L}{L}%
      \HOLOGO@Span{i}{i}%
      \hologo{TeX}%
    }%
  }%
}
%    \end{macrocode}
%    \end{macro}
%    \begin{macro}{\HoLogoCss@SLiTeX@lift}
%    \begin{macrocode}
\def\HoLogoCss@SLiTeX@lift{%
  \Css{%
    span.HoLogo-SLiTeX-lift span.HoLogo-L{%
      margin-left:-.06em;%
      margin-right:-.18em;%
    }%
  }%
  \Css{%
    span.HoLogo-SLiTeX-lift span.HoLogo-i{%
      position:relative;%
      top:-.32ex;%
      margin-right:-.06em;%
      font-variant:small-caps;%
    }%
  }%
  \global\let\HoLogoCss@SLiTeX@lift\relax
}
%    \end{macrocode}
%    \end{macro}
%
%    \begin{macro}{\HoLogo@SliTeX@simple}
%    \begin{macrocode}
\def\HoLogo@SliTeX@simple#1{%
  \HoLogoFont@font{SliTeX}{rm}{%
    \ltx@mbox{%
      \HoLogoFont@font{SliTeX}{sc}{Sli}%
    }%
    \HOLOGO@discretionary
    \hologo{TeX}%
  }%
}
%    \end{macrocode}
%    \end{macro}
%    \begin{macro}{\HoLogoBkm@SliTeX@simple}
%    \begin{macrocode}
\def\HoLogoBkm@SliTeX@simple#1{SliTeX}
%    \end{macrocode}
%    \end{macro}
%    \begin{macro}{\HoLogoHtml@SliTeX@simple}
%    \begin{macrocode}
\let\HoLogoHtml@SliTeX@simple\HoLogo@SliTeX@simple
%    \end{macrocode}
%    \end{macro}
%
%    \begin{macro}{\HoLogo@SliTeX@narrow}
%    \begin{macrocode}
\def\HoLogo@SliTeX@narrow#1{%
  \HoLogoFont@font{SliTeX}{rm}{%
    \ltx@mbox{%
      S%
      \kern-.06em%
      \HoLogoFont@font{SliTeX}{sc}{%
        l%
        \kern-.035em%
        i%
      }%
    }%
    \HOLOGO@discretionary
    \kern-.06em%
    \hologo{TeX}%
  }%
}
%    \end{macrocode}
%    \end{macro}
%    \begin{macro}{\HoLogoBkm@SliTeX@narrow}
%    \begin{macrocode}
\def\HoLogoBkm@SliTeX@narrow#1{SliTeX}
%    \end{macrocode}
%    \end{macro}
%    \begin{macro}{\HoLogoHtml@SliTeX@narrow}
%    \begin{macrocode}
\def\HoLogoHtml@SliTeX@narrow#1{%
  \HoLogoCss@SliTeX@narrow
  \HOLOGO@Span{SliTeX-narrow}{%
    \HoLogoFont@font{SliTeX}{rm}{%
      S%
        \HOLOGO@Span{l}{l}%
        \HOLOGO@Span{i}{i}%
      \hologo{TeX}%
    }%
  }%
}
%    \end{macrocode}
%    \end{macro}
%    \begin{macro}{\HoLogoCss@SliTeX@narrow}
%    \begin{macrocode}
\def\HoLogoCss@SliTeX@narrow{%
  \Css{%
    span.HoLogo-SliTeX-narrow span.HoLogo-l{%
      margin-left:-.06em;%
      margin-right:-.035em;%
      font-variant:small-caps;%
    }%
  }%
  \Css{%
    span.HoLogo-SliTeX-narrow span.HoLogo-i{%
      margin-right:-.06em;%
      font-variant:small-caps;%
    }%
  }%
  \global\let\HoLogoCss@SliTeX@narrow\relax
}
%    \end{macrocode}
%    \end{macro}
%
% \paragraph{Macro set completion.}
%
%    \begin{macro}{\HoLogo@SLiTeX@simple}
%    \begin{macrocode}
\def\HoLogo@SLiTeX@simple{\HoLogo@SliTeX@simple}
%    \end{macrocode}
%    \end{macro}
%    \begin{macro}{\HoLogoBkm@SLiTeX@simple}
%    \begin{macrocode}
\def\HoLogoBkm@SLiTeX@simple{\HoLogoBkm@SliTeX@simple}
%    \end{macrocode}
%    \end{macro}
%    \begin{macro}{\HoLogoHtml@SLiTeX@simple}
%    \begin{macrocode}
\def\HoLogoHtml@SLiTeX@simple{\HoLogoHtml@SliTeX@simple}
%    \end{macrocode}
%    \end{macro}
%
%    \begin{macro}{\HoLogo@SLiTeX@narrow}
%    \begin{macrocode}
\def\HoLogo@SLiTeX@narrow{\HoLogo@SliTeX@narrow}
%    \end{macrocode}
%    \end{macro}
%    \begin{macro}{\HoLogoBkm@SLiTeX@narrow}
%    \begin{macrocode}
\def\HoLogoBkm@SLiTeX@narrow{\HoLogoBkm@SliTeX@narrow}
%    \end{macrocode}
%    \end{macro}
%    \begin{macro}{\HoLogoHtml@SLiTeX@narrow}
%    \begin{macrocode}
\def\HoLogoHtml@SLiTeX@narrow{\HoLogoHtml@SliTeX@narrow}
%    \end{macrocode}
%    \end{macro}
%
%    \begin{macro}{\HoLogo@SliTeX@lift}
%    \begin{macrocode}
\def\HoLogo@SliTeX@lift{\HoLogo@SLiTeX@lift}
%    \end{macrocode}
%    \end{macro}
%    \begin{macro}{\HoLogoBkm@SliTeX@lift}
%    \begin{macrocode}
\def\HoLogoBkm@SliTeX@lift{\HoLogoBkm@SLiTeX@lift}
%    \end{macrocode}
%    \end{macro}
%    \begin{macro}{\HoLogoHtml@SliTeX@lift}
%    \begin{macrocode}
\def\HoLogoHtml@SliTeX@lift{\HoLogoHtml@SLiTeX@lift}
%    \end{macrocode}
%    \end{macro}
%
% \paragraph{Defaults.}
%
%    \begin{macro}{\HoLogo@SLiTeX}
%    \begin{macrocode}
\def\HoLogo@SLiTeX{\HoLogo@SLiTeX@lift}
%    \end{macrocode}
%    \end{macro}
%    \begin{macro}{\HoLogoBkm@SLiTeX}
%    \begin{macrocode}
\def\HoLogoBkm@SLiTeX{\HoLogoBkm@SLiTeX@lift}
%    \end{macrocode}
%    \end{macro}
%    \begin{macro}{\HoLogoHtml@SLiTeX}
%    \begin{macrocode}
\def\HoLogoHtml@SLiTeX{\HoLogoHtml@SLiTeX@lift}
%    \end{macrocode}
%    \end{macro}
%
%    \begin{macro}{\HoLogo@SliTeX}
%    \begin{macrocode}
\def\HoLogo@SliTeX{\HoLogo@SliTeX@narrow}
%    \end{macrocode}
%    \end{macro}
%    \begin{macro}{\HoLogoBkm@SliTeX}
%    \begin{macrocode}
\def\HoLogoBkm@SliTeX{\HoLogoBkm@SliTeX@narrow}
%    \end{macrocode}
%    \end{macro}
%    \begin{macro}{\HoLogoHtml@SliTeX}
%    \begin{macrocode}
\def\HoLogoHtml@SliTeX{\HoLogoHtml@SliTeX@narrow}
%    \end{macrocode}
%    \end{macro}
%
% \subsubsection{\hologo{LuaTeX}}
%
%    \begin{macro}{\HoLogo@LuaTeX}
%    The kerning is an idea of Hans Hagen, see mailing list
%    `luatex at tug dot org' in March 2010.
%    \begin{macrocode}
\def\HoLogo@LuaTeX#1{%
  \HOLOGO@mbox{%
    Lua%
    \HOLOGO@NegativeKerning{aT,oT,To}%
    \hologo{TeX}%
  }%
}
%    \end{macrocode}
%    \end{macro}
%    \begin{macro}{\HoLogoHtml@LuaTeX}
%    \begin{macrocode}
\let\HoLogoHtml@LuaTeX\HoLogo@LuaTeX
%    \end{macrocode}
%    \end{macro}
%
% \subsubsection{\hologo{LuaLaTeX}}
%
%    \begin{macro}{\HoLogo@LuaLaTeX}
%    \begin{macrocode}
\def\HoLogo@LuaLaTeX#1{%
  \HOLOGO@mbox{%
    Lua%
    \hologo{LaTeX}%
  }%
}
%    \end{macrocode}
%    \end{macro}
%    \begin{macro}{\HoLogoHtml@LuaLaTeX}
%    \begin{macrocode}
\let\HoLogoHtml@LuaLaTeX\HoLogo@LuaLaTeX
%    \end{macrocode}
%    \end{macro}
%
% \subsubsection{\hologo{XeTeX}, \hologo{XeLaTeX}}
%
%    \begin{macro}{\HOLOGO@IfCharExists}
%    \begin{macrocode}
\ifluatex
  \ifnum\luatexversion<36 %
  \else
    \def\HOLOGO@IfCharExists#1{%
      \ifnum
        \directlua{%
           if luaotfload and luaotfload.aux then
             if luaotfload.aux.font_has_glyph(%
                    font.current(), \number#1) then % 	 
	       tex.print("1") % 	 
	     end % 	 
	   elseif font and font.fonts and font.current then %
            local f = font.fonts[font.current()]%
            if f.characters and f.characters[\number#1] then %
              tex.print("1")%
            end %
          end%
        }0=\ltx@zero
        \expandafter\ltx@secondoftwo
      \else
        \expandafter\ltx@firstoftwo
      \fi
    }%
  \fi
\fi
\ltx@IfUndefined{HOLOGO@IfCharExists}{%
  \def\HOLOGO@@IfCharExists#1{%
    \begingroup
      \tracinglostchars=\ltx@zero
      \setbox\ltx@zero=\hbox{%
        \kern7sp\char#1\relax
        \ifnum\lastkern>\ltx@zero
          \expandafter\aftergroup\csname iffalse\endcsname
        \else
          \expandafter\aftergroup\csname iftrue\endcsname
        \fi
      }%
      % \if{true|false} from \aftergroup
      \endgroup
      \expandafter\ltx@firstoftwo
    \else
      \endgroup
      \expandafter\ltx@secondoftwo
    \fi
  }%
  \ifxetex
    \ltx@IfUndefined{XeTeXfonttype}{}{%
      \ltx@IfUndefined{XeTeXcharglyph}{}{%
        \def\HOLOGO@IfCharExists#1{%
          \ifnum\XeTeXfonttype\font>\ltx@zero
            \expandafter\ltx@firstofthree
          \else
            \expandafter\ltx@gobble
          \fi
          {%
            \ifnum\XeTeXcharglyph#1>\ltx@zero
              \expandafter\ltx@firstoftwo
            \else
              \expandafter\ltx@secondoftwo
            \fi
          }%
          \HOLOGO@@IfCharExists{#1}%
        }%
      }%
    }%
  \fi
}{}
\ltx@ifundefined{HOLOGO@IfCharExists}{%
  \ifnum64=`\^^^^0040\relax % test for big chars of LuaTeX/XeTeX
    \let\HOLOGO@IfCharExists\HOLOGO@@IfCharExists
  \else
    \def\HOLOGO@IfCharExists#1{%
      \ifnum#1>255 %
        \expandafter\ltx@fourthoffour
      \fi
      \HOLOGO@@IfCharExists{#1}%
    }%
  \fi
}{}
%    \end{macrocode}
%    \end{macro}
%
%    \begin{macro}{\HoLogo@Xe}
%    Source: package \xpackage{dtklogos}
%    \begin{macrocode}
\def\HoLogo@Xe#1{%
  X%
  \kern-.1em\relax
  \HOLOGO@IfCharExists{"018E}{%
    \lower.5ex\hbox{\char"018E}%
  }{%
    \chardef\HOLOGO@choice=\ltx@zero
    \ifdim\fontdimen\ltx@one\font>0pt %
      \ltx@IfUndefined{rotatebox}{%
        \ltx@IfUndefined{pgftext}{%
          \ltx@IfUndefined{psscalebox}{%
            \ltx@IfUndefined{HOLOGO@ScaleBox@\hologoDriver}{%
            }{%
              \chardef\HOLOGO@choice=4 %
            }%
          }{%
            \chardef\HOLOGO@choice=3 %
          }%
        }{%
          \chardef\HOLOGO@choice=2 %
        }%
      }{%
        \chardef\HOLOGO@choice=1 %
      }%
      \ifcase\HOLOGO@choice
        \HOLOGO@WarningUnsupportedDriver{Xe}%
        e%
      \or % 1: \rotatebox
        \begingroup
          \setbox\ltx@zero\hbox{\rotatebox{180}{E}}%
          \ltx@LocDimenA=\dp\ltx@zero
          \advance\ltx@LocDimenA by -.5ex\relax
          \raise\ltx@LocDimenA\box\ltx@zero
        \endgroup
      \or % 2: \pgftext
        \lower.5ex\hbox{%
          \pgfpicture
            \pgftext[rotate=180]{E}%
          \endpgfpicture
        }%
      \or % 3: \psscalebox
        \begingroup
          \setbox\ltx@zero\hbox{\psscalebox{-1 -1}{E}}%
          \ltx@LocDimenA=\dp\ltx@zero
          \advance\ltx@LocDimenA by -.5ex\relax
          \raise\ltx@LocDimenA\box\ltx@zero
        \endgroup
      \or % 4: \HOLOGO@PointReflectBox
        \lower.5ex\hbox{\HOLOGO@PointReflectBox{E}}%
      \else
        \@PackageError{hologo}{Internal error (choice/it}\@ehc
      \fi
    \else
      \ltx@IfUndefined{reflectbox}{%
        \ltx@IfUndefined{pgftext}{%
          \ltx@IfUndefined{psscalebox}{%
            \ltx@IfUndefined{HOLOGO@ScaleBox@\hologoDriver}{%
            }{%
              \chardef\HOLOGO@choice=4 %
            }%
          }{%
            \chardef\HOLOGO@choice=3 %
          }%
        }{%
          \chardef\HOLOGO@choice=2 %
        }%
      }{%
        \chardef\HOLOGO@choice=1 %
      }%
      \ifcase\HOLOGO@choice
        \HOLOGO@WarningUnsupportedDriver{Xe}%
        e%
      \or % 1: reflectbox
        \lower.5ex\hbox{%
          \reflectbox{E}%
        }%
      \or % 2: \pgftext
        \lower.5ex\hbox{%
          \pgfpicture
            \pgftransformxscale{-1}%
            \pgftext{E}%
          \endpgfpicture
        }%
      \or % 3: \psscalebox
        \lower.5ex\hbox{%
          \psscalebox{-1 1}{E}%
        }%
      \or % 4: \HOLOGO@Reflectbox
        \lower.5ex\hbox{%
          \HOLOGO@ReflectBox{E}%
        }%
      \else
        \@PackageError{hologo}{Internal error (choice/up)}\@ehc
      \fi
    \fi
  }%
}
%    \end{macrocode}
%    \end{macro}
%    \begin{macro}{\HoLogoHtml@Xe}
%    \begin{macrocode}
\def\HoLogoHtml@Xe#1{%
  \HoLogoCss@Xe
  \HOLOGO@Span{Xe}{%
    X%
    \HOLOGO@Span{e}{%
      \HCode{&\ltx@hashchar x018e;}%
    }%
  }%
}
%    \end{macrocode}
%    \end{macro}
%    \begin{macro}{\HoLogoCss@Xe}
%    \begin{macrocode}
\def\HoLogoCss@Xe{%
  \Css{%
    span.HoLogo-Xe span.HoLogo-e{%
      position:relative;%
      top:.5ex;%
      left-margin:-.1em;%
    }%
  }%
  \global\let\HoLogoCss@Xe\relax
}
%    \end{macrocode}
%    \end{macro}
%
%    \begin{macro}{\HoLogo@XeTeX}
%    \begin{macrocode}
\def\HoLogo@XeTeX#1{%
  \hologo{Xe}%
  \kern-.15em\relax
  \hologo{TeX}%
}
%    \end{macrocode}
%    \end{macro}
%
%    \begin{macro}{\HoLogoHtml@XeTeX}
%    \begin{macrocode}
\def\HoLogoHtml@XeTeX#1{%
  \HoLogoCss@XeTeX
  \HOLOGO@Span{XeTeX}{%
    \hologo{Xe}%
    \hologo{TeX}%
  }%
}
%    \end{macrocode}
%    \end{macro}
%    \begin{macro}{\HoLogoCss@XeTeX}
%    \begin{macrocode}
\def\HoLogoCss@XeTeX{%
  \Css{%
    span.HoLogo-XeTeX span.HoLogo-TeX{%
      margin-left:-.15em;%
    }%
  }%
  \global\let\HoLogoCss@XeTeX\relax
}
%    \end{macrocode}
%    \end{macro}
%
%    \begin{macro}{\HoLogo@XeLaTeX}
%    \begin{macrocode}
\def\HoLogo@XeLaTeX#1{%
  \hologo{Xe}%
  \kern-.13em%
  \hologo{LaTeX}%
}
%    \end{macrocode}
%    \end{macro}
%    \begin{macro}{\HoLogoHtml@XeLaTeX}
%    \begin{macrocode}
\def\HoLogoHtml@XeLaTeX#1{%
  \HoLogoCss@XeLaTeX
  \HOLOGO@Span{XeLaTeX}{%
    \hologo{Xe}%
    \hologo{LaTeX}%
  }%
}
%    \end{macrocode}
%    \end{macro}
%    \begin{macro}{\HoLogoCss@XeLaTeX}
%    \begin{macrocode}
\def\HoLogoCss@XeLaTeX{%
  \Css{%
    span.HoLogo-XeLaTeX span.HoLogo-Xe{%
      margin-right:-.13em;%
    }%
  }%
  \global\let\HoLogoCss@XeLaTeX\relax
}
%    \end{macrocode}
%    \end{macro}
%
% \subsubsection{\hologo{pdfTeX}, \hologo{pdfLaTeX}}
%
%    \begin{macro}{\HoLogo@pdfTeX}
%    \begin{macrocode}
\def\HoLogo@pdfTeX#1{%
  \HOLOGO@mbox{%
    #1{p}{P}df\hologo{TeX}%
  }%
}
%    \end{macrocode}
%    \end{macro}
%    \begin{macro}{\HoLogoCs@pdfTeX}
%    \begin{macrocode}
\def\HoLogoCs@pdfTeX#1{#1{p}{P}dfTeX}
%    \end{macrocode}
%    \end{macro}
%    \begin{macro}{\HoLogoBkm@pdfTeX}
%    \begin{macrocode}
\def\HoLogoBkm@pdfTeX#1{%
  #1{p}{P}df\hologo{TeX}%
}
%    \end{macrocode}
%    \end{macro}
%    \begin{macro}{\HoLogoHtml@pdfTeX}
%    \begin{macrocode}
\let\HoLogoHtml@pdfTeX\HoLogo@pdfTeX
%    \end{macrocode}
%    \end{macro}
%
%    \begin{macro}{\HoLogo@pdfLaTeX}
%    \begin{macrocode}
\def\HoLogo@pdfLaTeX#1{%
  \HOLOGO@mbox{%
    #1{p}{P}df\hologo{LaTeX}%
  }%
}
%    \end{macrocode}
%    \end{macro}
%    \begin{macro}{\HoLogoCs@pdfLaTeX}
%    \begin{macrocode}
\def\HoLogoCs@pdfLaTeX#1{#1{p}{P}dfLaTeX}
%    \end{macrocode}
%    \end{macro}
%    \begin{macro}{\HoLogoBkm@pdfLaTeX}
%    \begin{macrocode}
\def\HoLogoBkm@pdfLaTeX#1{%
  #1{p}{P}df\hologo{LaTeX}%
}
%    \end{macrocode}
%    \end{macro}
%    \begin{macro}{\HoLogoHtml@pdfLaTeX}
%    \begin{macrocode}
\let\HoLogoHtml@pdfLaTeX\HoLogo@pdfLaTeX
%    \end{macrocode}
%    \end{macro}
%
% \subsubsection{\hologo{VTeX}}
%
%    \begin{macro}{\HoLogo@VTeX}
%    \begin{macrocode}
\def\HoLogo@VTeX#1{%
  \HOLOGO@mbox{%
    V\hologo{TeX}%
  }%
}
%    \end{macrocode}
%    \end{macro}
%    \begin{macro}{\HoLogoHtml@VTeX}
%    \begin{macrocode}
\let\HoLogoHtml@VTeX\HoLogo@VTeX
%    \end{macrocode}
%    \end{macro}
%
% \subsubsection{\hologo{AmS}, \dots}
%
%    Source: class \xclass{amsdtx}
%
%    \begin{macro}{\HoLogo@AmS}
%    \begin{macrocode}
\def\HoLogo@AmS#1{%
  \HoLogoFont@font{AmS}{sy}{%
    A%
    \kern-.1667em%
    \lower.5ex\hbox{M}%
    \kern-.125em%
    S%
  }%
}
%    \end{macrocode}
%    \end{macro}
%    \begin{macro}{\HoLogoBkm@AmS}
%    \begin{macrocode}
\def\HoLogoBkm@AmS#1{AmS}
%    \end{macrocode}
%    \end{macro}
%    \begin{macro}{\HoLogoHtml@AmS}
%    \begin{macrocode}
\def\HoLogoHtml@AmS#1{%
  \HoLogoCss@AmS
%  \HoLogoFont@font{AmS}{sy}{%
    \HOLOGO@Span{AmS}{%
      A%
      \HOLOGO@Span{M}{M}%
      S%
    }%
%   }%
}
%    \end{macrocode}
%    \end{macro}
%    \begin{macro}{\HoLogoCss@AmS}
%    \begin{macrocode}
\def\HoLogoCss@AmS{%
  \Css{%
    span.HoLogo-AmS span.HoLogo-M{%
      position:relative;%
      top:.5ex;%
      margin-left:-.1667em;%
      margin-right:-.125em;%
      text-decoration:none;%
    }%
  }%
  \global\let\HoLogoCss@AmS\relax
}
%    \end{macrocode}
%    \end{macro}
%
%    \begin{macro}{\HoLogo@AmSTeX}
%    \begin{macrocode}
\def\HoLogo@AmSTeX#1{%
  \hologo{AmS}%
  \HOLOGO@hyphen
  \hologo{TeX}%
}
%    \end{macrocode}
%    \end{macro}
%    \begin{macro}{\HoLogoBkm@AmSTeX}
%    \begin{macrocode}
\def\HoLogoBkm@AmSTeX#1{AmS-TeX}%
%    \end{macrocode}
%    \end{macro}
%    \begin{macro}{\HoLogoHtml@AmSTeX}
%    \begin{macrocode}
\let\HoLogoHtml@AmSTeX\HoLogo@AmSTeX
%    \end{macrocode}
%    \end{macro}
%
%    \begin{macro}{\HoLogo@AmSLaTeX}
%    \begin{macrocode}
\def\HoLogo@AmSLaTeX#1{%
  \hologo{AmS}%
  \HOLOGO@hyphen
  \hologo{LaTeX}%
}
%    \end{macrocode}
%    \end{macro}
%    \begin{macro}{\HoLogoBkm@AmSLaTeX}
%    \begin{macrocode}
\def\HoLogoBkm@AmSLaTeX#1{AmS-LaTeX}%
%    \end{macrocode}
%    \end{macro}
%    \begin{macro}{\HoLogoHtml@AmSLaTeX}
%    \begin{macrocode}
\let\HoLogoHtml@AmSLaTeX\HoLogo@AmSLaTeX
%    \end{macrocode}
%    \end{macro}
%
% \subsubsection{\hologo{BibTeX}}
%
%    \begin{macro}{\HoLogo@BibTeX@sc}
%    A definition of \hologo{BibTeX} is provided in
%    the documentation source for the manual of \hologo{BibTeX}
%    \cite{btxdoc}.
%\begin{quote}
%\begin{verbatim}
%\def\BibTeX{%
%  {%
%    \rm
%    B%
%    \kern-.05em%
%    {%
%      \sc
%      i%
%      \kern-.025em %
%      b%
%    }%
%    \kern-.08em
%    T%
%    \kern-.1667em%
%    \lower.7ex\hbox{E}%
%    \kern-.125em%
%    X%
%  }%
%}
%\end{verbatim}
%\end{quote}
%    \begin{macrocode}
\def\HoLogo@BibTeX@sc#1{%
  B%
  \kern-.05em%
  \HoLogoFont@font{BibTeX}{sc}{%
    i%
    \kern-.025em%
    b%
  }%
  \HOLOGO@discretionary
  \kern-.08em%
  \hologo{TeX}%
}
%    \end{macrocode}
%    \end{macro}
%    \begin{macro}{\HoLogoHtml@BibTeX@sc}
%    \begin{macrocode}
\def\HoLogoHtml@BibTeX@sc#1{%
  \HoLogoCss@BibTeX@sc
  \HOLOGO@Span{BibTeX-sc}{%
    B%
    \HOLOGO@Span{i}{i}%
    \HOLOGO@Span{b}{b}%
    \hologo{TeX}%
  }%
}
%    \end{macrocode}
%    \end{macro}
%    \begin{macro}{\HoLogoCss@BibTeX@sc}
%    \begin{macrocode}
\def\HoLogoCss@BibTeX@sc{%
  \Css{%
    span.HoLogo-BibTeX-sc span.HoLogo-i{%
      margin-left:-.05em;%
      margin-right:-.025em;%
      font-variant:small-caps;%
    }%
  }%
  \Css{%
    span.HoLogo-BibTeX-sc span.HoLogo-b{%
      margin-right:-.08em;%
      font-variant:small-caps;%
    }%
  }%
  \global\let\HoLogoCss@BibTeX@sc\relax
}
%    \end{macrocode}
%    \end{macro}
%
%    \begin{macro}{\HoLogo@BibTeX@sf}
%    Variant \xoption{sf} avoids trouble with unavailable
%    small caps fonts (e.g., bold versions of Computer Modern or
%    Latin Modern). The definition is taken from
%    package \xpackage{dtklogos} \cite{dtklogos}.
%\begin{quote}
%\begin{verbatim}
%\DeclareRobustCommand{\BibTeX}{%
%  B%
%  \kern-.05em%
%  \hbox{%
%    $\m@th$% %% force math size calculations
%    \csname S@\f@size\endcsname
%    \fontsize\sf@size\z@
%    \math@fontsfalse
%    \selectfont
%    I%
%    \kern-.025em%
%    B
%  }%
%  \kern-.08em%
%  \-%
%  \TeX
%}
%\end{verbatim}
%\end{quote}
%    \begin{macrocode}
\def\HoLogo@BibTeX@sf#1{%
  B%
  \kern-.05em%
  \HoLogoFont@font{BibTeX}{bibsf}{%
    I%
    \kern-.025em%
    B%
  }%
  \HOLOGO@discretionary
  \kern-.08em%
  \hologo{TeX}%
}
%    \end{macrocode}
%    \end{macro}
%    \begin{macro}{\HoLogoHtml@BibTeX@sf}
%    \begin{macrocode}
\def\HoLogoHtml@BibTeX@sf#1{%
  \HoLogoCss@BibTeX@sf
  \HOLOGO@Span{BibTeX-sf}{%
    B%
    \HoLogoFont@font{BibTeX}{bibsf}{%
      \HOLOGO@Span{i}{I}%
      B%
    }%
    \hologo{TeX}%
  }%
}
%    \end{macrocode}
%    \end{macro}
%    \begin{macro}{\HoLogoCss@BibTeX@sf}
%    \begin{macrocode}
\def\HoLogoCss@BibTeX@sf{%
  \Css{%
    span.HoLogo-BibTeX-sf span.HoLogo-i{%
      margin-left:-.05em;%
      margin-right:-.025em;%
    }%
  }%
  \Css{%
    span.HoLogo-BibTeX-sf span.HoLogo-TeX{%
      margin-left:-.08em;%
    }%
  }%
  \global\let\HoLogoCss@BibTeX@sf\relax
}
%    \end{macrocode}
%    \end{macro}
%
%    \begin{macro}{\HoLogo@BibTeX}
%    \begin{macrocode}
\def\HoLogo@BibTeX{\HoLogo@BibTeX@sf}
%    \end{macrocode}
%    \end{macro}
%    \begin{macro}{\HoLogoHtml@BibTeX}
%    \begin{macrocode}
\def\HoLogoHtml@BibTeX{\HoLogoHtml@BibTeX@sf}
%    \end{macrocode}
%    \end{macro}
%
% \subsubsection{\hologo{BibTeX8}}
%
%    \begin{macro}{\HoLogo@BibTeX8}
%    \begin{macrocode}
\expandafter\def\csname HoLogo@BibTeX8\endcsname#1{%
  \hologo{BibTeX}%
  8%
}
%    \end{macrocode}
%    \end{macro}
%
%    \begin{macro}{\HoLogoBkm@BibTeX8}
%    \begin{macrocode}
\expandafter\def\csname HoLogoBkm@BibTeX8\endcsname#1{%
  \hologo{BibTeX}%
  8%
}
%    \end{macrocode}
%    \end{macro}
%    \begin{macro}{\HoLogoHtml@BibTeX8}
%    \begin{macrocode}
\expandafter
\let\csname HoLogoHtml@BibTeX8\expandafter\endcsname
\csname HoLogo@BibTeX8\endcsname
%    \end{macrocode}
%    \end{macro}
%
% \subsubsection{\hologo{ConTeXt}}
%
%    \begin{macro}{\HoLogo@ConTeXt@simple}
%    \begin{macrocode}
\def\HoLogo@ConTeXt@simple#1{%
  \HOLOGO@mbox{Con}%
  \HOLOGO@discretionary
  \HOLOGO@mbox{\hologo{TeX}t}%
}
%    \end{macrocode}
%    \end{macro}
%    \begin{macro}{\HoLogoHtml@ConTeXt@simple}
%    \begin{macrocode}
\let\HoLogoHtml@ConTeXt@simple\HoLogo@ConTeXt@simple
%    \end{macrocode}
%    \end{macro}
%
%    \begin{macro}{\HoLogo@ConTeXt@narrow}
%    This definition of logo \hologo{ConTeXt} with variant \xoption{narrow}
%    comes from TUGboat's class \xclass{ltugboat} (version 2010/11/15 v2.8).
%    \begin{macrocode}
\def\HoLogo@ConTeXt@narrow#1{%
  \HOLOGO@mbox{C\kern-.0333emon}%
  \HOLOGO@discretionary
  \kern-.0667em%
  \HOLOGO@mbox{\hologo{TeX}\kern-.0333emt}%
}
%    \end{macrocode}
%    \end{macro}
%    \begin{macro}{\HoLogoHtml@ConTeXt@narrow}
%    \begin{macrocode}
\def\HoLogoHtml@ConTeXt@narrow#1{%
  \HoLogoCss@ConTeXt@narrow
  \HOLOGO@Span{ConTeXt-narrow}{%
    \HOLOGO@Span{C}{C}%
    on%
    \hologo{TeX}%
    t%
  }%
}
%    \end{macrocode}
%    \end{macro}
%    \begin{macro}{\HoLogoCss@ConTeXt@narrow}
%    \begin{macrocode}
\def\HoLogoCss@ConTeXt@narrow{%
  \Css{%
    span.HoLogo-ConTeXt-narrow span.HoLogo-C{%
      margin-left:-.0333em;%
    }%
  }%
  \Css{%
    span.HoLogo-ConTeXt-narrow span.HoLogo-TeX{%
      margin-left:-.0667em;%
      margin-right:-.0333em;%
    }%
  }%
  \global\let\HoLogoCss@ConTeXt@narrow\relax
}
%    \end{macrocode}
%    \end{macro}
%
%    \begin{macro}{\HoLogo@ConTeXt}
%    \begin{macrocode}
\def\HoLogo@ConTeXt{\HoLogo@ConTeXt@narrow}
%    \end{macrocode}
%    \end{macro}
%    \begin{macro}{\HoLogoHtml@ConTeXt}
%    \begin{macrocode}
\def\HoLogoHtml@ConTeXt{\HoLogoHtml@ConTeXt@narrow}
%    \end{macrocode}
%    \end{macro}
%
% \subsubsection{\hologo{emTeX}}
%
%    \begin{macro}{\HoLogo@emTeX}
%    \begin{macrocode}
\def\HoLogo@emTeX#1{%
  \HOLOGO@mbox{#1{e}{E}m}%
  \HOLOGO@discretionary
  \hologo{TeX}%
}
%    \end{macrocode}
%    \end{macro}
%    \begin{macro}{\HoLogoCs@emTeX}
%    \begin{macrocode}
\def\HoLogoCs@emTeX#1{#1{e}{E}mTeX}%
%    \end{macrocode}
%    \end{macro}
%    \begin{macro}{\HoLogoBkm@emTeX}
%    \begin{macrocode}
\def\HoLogoBkm@emTeX#1{%
  #1{e}{E}m\hologo{TeX}%
}
%    \end{macrocode}
%    \end{macro}
%    \begin{macro}{\HoLogoHtml@emTeX}
%    \begin{macrocode}
\let\HoLogoHtml@emTeX\HoLogo@emTeX
%    \end{macrocode}
%    \end{macro}
%
% \subsubsection{\hologo{ExTeX}}
%
%    \begin{macro}{\HoLogo@ExTeX}
%    The definition is taken from the FAQ of the
%    project \hologo{ExTeX}
%    \cite{ExTeX-FAQ}.
%\begin{quote}
%\begin{verbatim}
%\def\ExTeX{%
%  \textrm{% Logo always with serifs
%    \ensuremath{%
%      \textstyle
%      \varepsilon_{%
%        \kern-0.15em%
%        \mathcal{X}%
%      }%
%    }%
%    \kern-.15em%
%    \TeX
%  }%
%}
%\end{verbatim}
%\end{quote}
%    \begin{macrocode}
\def\HoLogo@ExTeX#1{%
  \HoLogoFont@font{ExTeX}{rm}{%
    \ltx@mbox{%
      \HOLOGO@MathSetup
      $%
        \textstyle
        \varepsilon_{%
          \kern-0.15em%
          \HoLogoFont@font{ExTeX}{sy}{X}%
        }%
      $%
    }%
    \HOLOGO@discretionary
    \kern-.15em%
    \hologo{TeX}%
  }%
}
%    \end{macrocode}
%    \end{macro}
%    \begin{macro}{\HoLogoHtml@ExTeX}
%    \begin{macrocode}
\def\HoLogoHtml@ExTeX#1{%
  \HoLogoCss@ExTeX
  \HoLogoFont@font{ExTeX}{rm}{%
    \HOLOGO@Span{ExTeX}{%
      \ltx@mbox{%
        \HOLOGO@MathSetup
        $\textstyle\varepsilon$%
        \HOLOGO@Span{X}{$\textstyle\chi$}%
        \hologo{TeX}%
      }%
    }%
  }%
}
%    \end{macrocode}
%    \end{macro}
%    \begin{macro}{\HoLogoBkm@ExTeX}
%    \begin{macrocode}
\def\HoLogoBkm@ExTeX#1{%
  \HOLOGO@PdfdocUnicode{#1{e}{E}x}{\textepsilon\textchi}%
  \hologo{TeX}%
}
%    \end{macrocode}
%    \end{macro}
%    \begin{macro}{\HoLogoCss@ExTeX}
%    \begin{macrocode}
\def\HoLogoCss@ExTeX{%
  \Css{%
    span.HoLogo-ExTeX{%
      font-family:serif;%
    }%
  }%
  \Css{%
    span.HoLogo-ExTeX span.HoLogo-TeX{%
      margin-left:-.15em;%
    }%
  }%
  \global\let\HoLogoCss@ExTeX\relax
}
%    \end{macrocode}
%    \end{macro}
%
% \subsubsection{\hologo{MiKTeX}}
%
%    \begin{macro}{\HoLogo@MiKTeX}
%    \begin{macrocode}
\def\HoLogo@MiKTeX#1{%
  \HOLOGO@mbox{MiK}%
  \HOLOGO@discretionary
  \hologo{TeX}%
}
%    \end{macrocode}
%    \end{macro}
%    \begin{macro}{\HoLogoHtml@MiKTeX}
%    \begin{macrocode}
\let\HoLogoHtml@MiKTeX\HoLogo@MiKTeX
%    \end{macrocode}
%    \end{macro}
%
% \subsubsection{\hologo{OzTeX} and friends}
%
%    Source: \hologo{OzTeX} FAQ \cite{OzTeX}:
%    \begin{quote}
%      |\def\OzTeX{O\kern-.03em z\kern-.15em\TeX}|\\
%      (There is no kerning in OzMF, OzMP and OzTtH.)
%    \end{quote}
%
%    \begin{macro}{\HoLogo@OzTeX}
%    \begin{macrocode}
\def\HoLogo@OzTeX#1{%
  O%
  \kern-.03em %
  z%
  \kern-.15em %
  \hologo{TeX}%
}
%    \end{macrocode}
%    \end{macro}
%    \begin{macro}{\HoLogoHtml@OzTeX}
%    \begin{macrocode}
\def\HoLogoHtml@OzTeX#1{%
  \HoLogoCss@OzTeX
  \HOLOGO@Span{OzTeX}{%
    O%
    \HOLOGO@Span{z}{z}%
    \hologo{TeX}%
  }%
}
%    \end{macrocode}
%    \end{macro}
%    \begin{macro}{\HoLogoCss@OzTeX}
%    \begin{macrocode}
\def\HoLogoCss@OzTeX{%
  \Css{%
    span.HoLogo-OzTeX span.HoLogo-z{%
      margin-left:-.03em;%
      margin-right:-.15em;%
    }%
  }%
  \global\let\HoLogoCss@OzTeX\relax
}
%    \end{macrocode}
%    \end{macro}
%
%    \begin{macro}{\HoLogo@OzMF}
%    \begin{macrocode}
\def\HoLogo@OzMF#1{%
  \HOLOGO@mbox{OzMF}%
}
%    \end{macrocode}
%    \end{macro}
%    \begin{macro}{\HoLogo@OzMP}
%    \begin{macrocode}
\def\HoLogo@OzMP#1{%
  \HOLOGO@mbox{OzMP}%
}
%    \end{macrocode}
%    \end{macro}
%    \begin{macro}{\HoLogo@OzTtH}
%    \begin{macrocode}
\def\HoLogo@OzTtH#1{%
  \HOLOGO@mbox{OzTtH}%
}
%    \end{macrocode}
%    \end{macro}
%
% \subsubsection{\hologo{PCTeX}}
%
%    \begin{macro}{\HoLogo@PCTeX}
%    \begin{macrocode}
\def\HoLogo@PCTeX#1{%
  \HOLOGO@mbox{PC}%
  \hologo{TeX}%
}
%    \end{macrocode}
%    \end{macro}
%    \begin{macro}{\HoLogoHtml@PCTeX}
%    \begin{macrocode}
\let\HoLogoHtml@PCTeX\HoLogo@PCTeX
%    \end{macrocode}
%    \end{macro}
%
% \subsubsection{\hologo{PiCTeX}}
%
%    The original definitions from \xfile{pictex.tex} \cite{PiCTeX}:
%\begin{quote}
%\begin{verbatim}
%\def\PiC{%
%  P%
%  \kern-.12em%
%  \lower.5ex\hbox{I}%
%  \kern-.075em%
%  C%
%}
%\def\PiCTeX{%
%  \PiC
%  \kern-.11em%
%  \TeX
%}
%\end{verbatim}
%\end{quote}
%
%    \begin{macro}{\HoLogo@PiC}
%    \begin{macrocode}
\def\HoLogo@PiC#1{%
  P%
  \kern-.12em%
  \lower.5ex\hbox{I}%
  \kern-.075em%
  C%
  \HOLOGO@SpaceFactor
}
%    \end{macrocode}
%    \end{macro}
%    \begin{macro}{\HoLogoHtml@PiC}
%    \begin{macrocode}
\def\HoLogoHtml@PiC#1{%
  \HoLogoCss@PiC
  \HOLOGO@Span{PiC}{%
    P%
    \HOLOGO@Span{i}{I}%
    C%
  }%
}
%    \end{macrocode}
%    \end{macro}
%    \begin{macro}{\HoLogoCss@PiC}
%    \begin{macrocode}
\def\HoLogoCss@PiC{%
  \Css{%
    span.HoLogo-PiC span.HoLogo-i{%
      position:relative;%
      top:.5ex;%
      margin-left:-.12em;%
      margin-right:-.075em;%
      text-decoration:none;%
    }%
  }%
  \global\let\HoLogoCss@PiC\relax
}
%    \end{macrocode}
%    \end{macro}
%
%    \begin{macro}{\HoLogo@PiCTeX}
%    \begin{macrocode}
\def\HoLogo@PiCTeX#1{%
  \hologo{PiC}%
  \HOLOGO@discretionary
  \kern-.11em%
  \hologo{TeX}%
}
%    \end{macrocode}
%    \end{macro}
%    \begin{macro}{\HoLogoHtml@PiCTeX}
%    \begin{macrocode}
\def\HoLogoHtml@PiCTeX#1{%
  \HoLogoCss@PiCTeX
  \HOLOGO@Span{PiCTeX}{%
    \hologo{PiC}%
    \hologo{TeX}%
  }%
}
%    \end{macrocode}
%    \end{macro}
%    \begin{macro}{\HoLogoCss@PiCTeX}
%    \begin{macrocode}
\def\HoLogoCss@PiCTeX{%
  \Css{%
    span.HoLogo-PiCTeX span.HoLogo-PiC{%
      margin-right:-.11em;%
    }%
  }%
  \global\let\HoLogoCss@PiCTeX\relax
}
%    \end{macrocode}
%    \end{macro}
%
% \subsubsection{\hologo{teTeX}}
%
%    \begin{macro}{\HoLogo@teTeX}
%    \begin{macrocode}
\def\HoLogo@teTeX#1{%
  \HOLOGO@mbox{#1{t}{T}e}%
  \HOLOGO@discretionary
  \hologo{TeX}%
}
%    \end{macrocode}
%    \end{macro}
%    \begin{macro}{\HoLogoCs@teTeX}
%    \begin{macrocode}
\def\HoLogoCs@teTeX#1{#1{t}{T}dfTeX}
%    \end{macrocode}
%    \end{macro}
%    \begin{macro}{\HoLogoBkm@teTeX}
%    \begin{macrocode}
\def\HoLogoBkm@teTeX#1{%
  #1{t}{T}e\hologo{TeX}%
}
%    \end{macrocode}
%    \end{macro}
%    \begin{macro}{\HoLogoHtml@teTeX}
%    \begin{macrocode}
\let\HoLogoHtml@teTeX\HoLogo@teTeX
%    \end{macrocode}
%    \end{macro}
%
% \subsubsection{\hologo{TeX4ht}}
%
%    \begin{macro}{\HoLogo@TeX4ht}
%    \begin{macrocode}
\expandafter\def\csname HoLogo@TeX4ht\endcsname#1{%
  \HOLOGO@mbox{\hologo{TeX}4ht}%
}
%    \end{macrocode}
%    \end{macro}
%    \begin{macro}{\HoLogoHtml@TeX4ht}
%    \begin{macrocode}
\expandafter
\let\csname HoLogoHtml@TeX4ht\expandafter\endcsname
\csname HoLogo@TeX4ht\endcsname
%    \end{macrocode}
%    \end{macro}
%
%
% \subsubsection{\hologo{SageTeX}}
%
%    \begin{macro}{\HoLogo@SageTeX}
%    \begin{macrocode}
\def\HoLogo@SageTeX#1{%
  \HOLOGO@mbox{Sage}%
  \HOLOGO@discretionary
  \HOLOGO@NegativeKerning{eT,oT,To}%
  \hologo{TeX}%
}
%    \end{macrocode}
%    \end{macro}
%    \begin{macro}{\HoLogoHtml@SageTeX}
%    \begin{macrocode}
\let\HoLogoHtml@SageTeX\HoLogo@SageTeX
%    \end{macrocode}
%    \end{macro}
%
% \subsection{\hologo{METAFONT} and friends}
%
%    \begin{macro}{\HoLogo@METAFONT}
%    \begin{macrocode}
\def\HoLogo@METAFONT#1{%
  \HoLogoFont@font{METAFONT}{logo}{%
    \HOLOGO@mbox{META}%
    \HOLOGO@discretionary
    \HOLOGO@mbox{FONT}%
  }%
}
%    \end{macrocode}
%    \end{macro}
%
%    \begin{macro}{\HoLogo@METAPOST}
%    \begin{macrocode}
\def\HoLogo@METAPOST#1{%
  \HoLogoFont@font{METAPOST}{logo}{%
    \HOLOGO@mbox{META}%
    \HOLOGO@discretionary
    \HOLOGO@mbox{POST}%
  }%
}
%    \end{macrocode}
%    \end{macro}
%
%    \begin{macro}{\HoLogo@MetaFun}
%    \begin{macrocode}
\def\HoLogo@MetaFun#1{%
  \HOLOGO@mbox{Meta}%
  \HOLOGO@discretionary
  \HOLOGO@mbox{Fun}%
}
%    \end{macrocode}
%    \end{macro}
%
%    \begin{macro}{\HoLogo@MetaPost}
%    \begin{macrocode}
\def\HoLogo@MetaPost#1{%
  \HOLOGO@mbox{Meta}%
  \HOLOGO@discretionary
  \HOLOGO@mbox{Post}%
}
%    \end{macrocode}
%    \end{macro}
%
% \subsection{Others}
%
% \subsubsection{\hologo{biber}}
%
%    \begin{macro}{\HoLogo@biber}
%    \begin{macrocode}
\def\HoLogo@biber#1{%
  \HOLOGO@mbox{#1{b}{B}i}%
  \HOLOGO@discretionary
  \HOLOGO@mbox{ber}%
}
%    \end{macrocode}
%    \end{macro}
%    \begin{macro}{\HoLogoCs@biber}
%    \begin{macrocode}
\def\HoLogoCs@biber#1{#1{b}{B}iber}
%    \end{macrocode}
%    \end{macro}
%    \begin{macro}{\HoLogoBkm@biber}
%    \begin{macrocode}
\def\HoLogoBkm@biber#1{%
  #1{b}{B}iber%
}
%    \end{macrocode}
%    \end{macro}
%    \begin{macro}{\HoLogoHtml@biber}
%    \begin{macrocode}
\let\HoLogoHtml@biber\HoLogo@biber
%    \end{macrocode}
%    \end{macro}
%
% \subsubsection{\hologo{KOMAScript}}
%
%    \begin{macro}{\HoLogo@KOMAScript}
%    The definition for \hologo{KOMAScript} is taken
%    from \hologo{KOMAScript} (\xfile{scrlogo.dtx}, reformatted) \cite{scrlogo}:
%\begin{quote}
%\begin{verbatim}
%\@ifundefined{KOMAScript}{%
%  \DeclareRobustCommand{\KOMAScript}{%
%    \textsf{%
%      K\kern.05em O\kern.05emM\kern.05em A%
%      \kern.1em-\kern.1em %
%      Script%
%    }%
%  }%
%}{}
%\end{verbatim}
%\end{quote}
%    \begin{macrocode}
\def\HoLogo@KOMAScript#1{%
  \HoLogoFont@font{KOMAScript}{sf}{%
    \HOLOGO@mbox{%
      K\kern.05em%
      O\kern.05em%
      M\kern.05em%
      A%
    }%
    \kern.1em%
    \HOLOGO@hyphen
    \kern.1em%
    \HOLOGO@mbox{Script}%
  }%
}
%    \end{macrocode}
%    \end{macro}
%    \begin{macro}{\HoLogoBkm@KOMAScript}
%    \begin{macrocode}
\def\HoLogoBkm@KOMAScript#1{%
  KOMA-Script%
}
%    \end{macrocode}
%    \end{macro}
%    \begin{macro}{\HoLogoHtml@KOMAScript}
%    \begin{macrocode}
\def\HoLogoHtml@KOMAScript#1{%
  \HoLogoCss@KOMAScript
  \HoLogoFont@font{KOMAScript}{sf}{%
    \HOLOGO@Span{KOMAScript}{%
      K%
      \HOLOGO@Span{O}{O}%
      M%
      \HOLOGO@Span{A}{A}%
      \HOLOGO@Span{hyphen}{-}%
      Script%
    }%
  }%
}
%    \end{macrocode}
%    \end{macro}
%    \begin{macro}{\HoLogoCss@KOMAScript}
%    \begin{macrocode}
\def\HoLogoCss@KOMAScript{%
  \Css{%
    span.HoLogo-KOMAScript{%
      font-family:sans-serif;%
    }%
  }%
  \Css{%
    span.HoLogo-KOMAScript span.HoLogo-O{%
      padding-left:.05em;%
      padding-right:.05em;%
    }%
  }%
  \Css{%
    span.HoLogo-KOMAScript span.HoLogo-A{%
      padding-left:.05em;%
    }%
  }%
  \Css{%
    span.HoLogo-KOMAScript span.HoLogo-hyphen{%
      padding-left:.1em;%
      padding-right:.1em;%
    }%
  }%
  \global\let\HoLogoCss@KOMAScript\relax
}
%    \end{macrocode}
%    \end{macro}
%
% \subsubsection{\hologo{LyX}}
%
%    \begin{macro}{\HoLogo@LyX}
%    The definition is taken from the documentation source files
%    of \hologo{LyX}, \xfile{Intro.lyx} \cite{LyX}:
%\begin{quote}
%\begin{verbatim}
%\def\LyX{%
%  \texorpdfstring{%
%    L\kern-.1667em\lower.25em\hbox{Y}\kern-.125emX\@%
%  }{%
%    LyX%
%  }%
%}
%\end{verbatim}
%\end{quote}
%    \begin{macrocode}
\def\HoLogo@LyX#1{%
  L%
  \kern-.1667em%
  \lower.25em\hbox{Y}%
  \kern-.125em%
  X%
  \HOLOGO@SpaceFactor
}
%    \end{macrocode}
%    \end{macro}
%    \begin{macro}{\HoLogoHtml@LyX}
%    \begin{macrocode}
\def\HoLogoHtml@LyX#1{%
  \HoLogoCss@LyX
  \HOLOGO@Span{LyX}{%
    L%
    \HOLOGO@Span{y}{Y}%
    X%
  }%
}
%    \end{macrocode}
%    \end{macro}
%    \begin{macro}{\HoLogoCss@LyX}
%    \begin{macrocode}
\def\HoLogoCss@LyX{%
  \Css{%
    span.HoLogo-LyX span.HoLogo-y{%
      position:relative;%
      top:.25em;%
      margin-left:-.1667em;%
      margin-right:-.125em;%
      text-decoration:none;%
    }%
  }%
  \global\let\HoLogoCss@LyX\relax
}
%    \end{macrocode}
%    \end{macro}
%
% \subsubsection{\hologo{NTS}}
%
%    \begin{macro}{\HoLogo@NTS}
%    Definition for \hologo{NTS} can be found in
%    package \xpackage{etex\textunderscore man} for the \hologo{eTeX} manual \cite{etexman}
%    and in package \xpackage{dtklogos} \cite{dtklogos}:
%\begin{quote}
%\begin{verbatim}
%\def\NTS{%
%  \leavevmode
%  \hbox{%
%    $%
%      \cal N%
%      \kern-0.35em%
%      \lower0.5ex\hbox{$\cal T$}%
%      \kern-0.2em%
%      S%
%    $%
%  }%
%}
%\end{verbatim}
%\end{quote}
%    \begin{macrocode}
\def\HoLogo@NTS#1{%
  \HoLogoFont@font{NTS}{sy}{%
    N\/%
    \kern-.35em%
    \lower.5ex\hbox{T\/}%
    \kern-.2em%
    S\/%
  }%
  \HOLOGO@SpaceFactor
}
%    \end{macrocode}
%    \end{macro}
%
% \subsubsection{\Hologo{TTH} (\hologo{TeX} to HTML translator)}
%
%    Source: \url{http://hutchinson.belmont.ma.us/tth/}
%    In the HTML source the second `T' is printed as subscript.
%\begin{quote}
%\begin{verbatim}
%T<sub>T</sub>H
%\end{verbatim}
%\end{quote}
%    \begin{macro}{\HoLogo@TTH}
%    \begin{macrocode}
\def\HoLogo@TTH#1{%
  \ltx@mbox{%
    T\HOLOGO@SubScript{T}H%
  }%
  \HOLOGO@SpaceFactor
}
%    \end{macrocode}
%    \end{macro}
%
%    \begin{macro}{\HoLogoHtml@TTH}
%    \begin{macrocode}
\def\HoLogoHtml@TTH#1{%
  T\HCode{<sub>}T\HCode{</sub>}H%
}
%    \end{macrocode}
%    \end{macro}
%
% \subsubsection{\Hologo{HanTheThanh}}
%
%    Partial source: Package \xpackage{dtklogos}.
%    The double accent is U+1EBF (latin small letter e with circumflex
%    and acute).
%    \begin{macro}{\HoLogo@HanTheThanh}
%    \begin{macrocode}
\def\HoLogo@HanTheThanh#1{%
  \ltx@mbox{H\`an}%
  \HOLOGO@space
  \ltx@mbox{%
    Th%
    \HOLOGO@IfCharExists{"1EBF}{%
      \char"1EBF\relax
    }{%
      \^e\hbox to 0pt{\hss\raise .5ex\hbox{\'{}}}%
    }%
  }%
  \HOLOGO@space
  \ltx@mbox{Th\`anh}%
}
%    \end{macrocode}
%    \end{macro}
%    \begin{macro}{\HoLogoBkm@HanTheThanh}
%    \begin{macrocode}
\def\HoLogoBkm@HanTheThanh#1{%
  H\`an %
  Th\HOLOGO@PdfdocUnicode{\^e}{\9036\277} %
  Th\`anh%
}
%    \end{macrocode}
%    \end{macro}
%    \begin{macro}{\HoLogoHtml@HanTheThanh}
%    \begin{macrocode}
\def\HoLogoHtml@HanTheThanh#1{%
  H\`an %
  Th\HCode{&\ltx@hashchar x1ebf;} %
  Th\`anh%
}
%    \end{macrocode}
%    \end{macro}
%
% \subsection{Driver detection}
%
%    \begin{macrocode}
\HOLOGO@IfExists\InputIfFileExists{%
  \InputIfFileExists{hologo.cfg}{}{}%
}{%
  \ltx@IfUndefined{pdf@filesize}{%
    \def\HOLOGO@InputIfExists{%
      \openin\HOLOGO@temp=hologo.cfg\relax
      \ifeof\HOLOGO@temp
        \closein\HOLOGO@temp
      \else
        \closein\HOLOGO@temp
        \begingroup
          \def\x{LaTeX2e}%
        \expandafter\endgroup
        \ifx\fmtname\x
          % \iffalse meta-comment
%
% File: hologo.dtx
% Version: 2016/05/12 v1.11
% Info: A logo collection with bookmark support
%
% Copyright (C) 2010-2012 by
%    Heiko Oberdiek <heiko.oberdiek at googlemail.com>
%
% This work may be distributed and/or modified under the
% conditions of the LaTeX Project Public License, either
% version 1.3c of this license or (at your option) any later
% version. This version of this license is in
%    http://www.latex-project.org/lppl/lppl-1-3c.txt
% and the latest version of this license is in
%    http://www.latex-project.org/lppl.txt
% and version 1.3 or later is part of all distributions of
% LaTeX version 2005/12/01 or later.
%
% This work has the LPPL maintenance status "maintained".
%
% This Current Maintainer of this work is Heiko Oberdiek.
%
% The Base Interpreter refers to any `TeX-Format',
% because some files are installed in TDS:tex/generic//.
%
% This work consists of the main source file hologo.dtx
% and the derived files
%    hologo.sty, hologo.pdf, hologo.ins, hologo.drv, hologo-example.tex,
%    hologo-test1.tex, hologo-test-spacefactor.tex,
%    hologo-test-list.tex.
%
% Distribution:
%    CTAN:macros/latex/contrib/oberdiek/hologo.dtx
%    CTAN:macros/latex/contrib/oberdiek/hologo.pdf
%
% Unpacking:
%    (a) If hologo.ins is present:
%           tex hologo.ins
%    (b) Without hologo.ins:
%           tex hologo.dtx
%    (c) If you insist on using LaTeX
%           latex \let\install=y\input{hologo.dtx}
%        (quote the arguments according to the demands of your shell)
%
% Documentation:
%    (a) If hologo.drv is present:
%           latex hologo.drv
%    (b) Without hologo.drv:
%           latex hologo.dtx; ...
%    The class ltxdoc loads the configuration file ltxdoc.cfg
%    if available. Here you can specify further options, e.g.
%    use A4 as paper format:
%       \PassOptionsToClass{a4paper}{article}
%
%    Programm calls to get the documentation (example):
%       pdflatex hologo.dtx
%       makeindex -s gind.ist hologo.idx
%       pdflatex hologo.dtx
%       makeindex -s gind.ist hologo.idx
%       pdflatex hologo.dtx
%
% Installation:
%    TDS:tex/generic/oberdiek/hologo.sty
%    TDS:doc/latex/oberdiek/hologo.pdf
%    TDS:doc/latex/oberdiek/example/hologo-example.tex
%    TDS:doc/latex/oberdiek/test/hologo-test1.tex
%    TDS:doc/latex/oberdiek/test/hologo-test-spacefactor.tex
%    TDS:doc/latex/oberdiek/test/hologo-test-list.tex
%    TDS:source/latex/oberdiek/hologo.dtx
%
%<*ignore>
\begingroup
  \catcode123=1 %
  \catcode125=2 %
  \def\x{LaTeX2e}%
\expandafter\endgroup
\ifcase 0\ifx\install y1\fi\expandafter
         \ifx\csname processbatchFile\endcsname\relax\else1\fi
         \ifx\fmtname\x\else 1\fi\relax
\else\csname fi\endcsname
%</ignore>
%<*install>
\input docstrip.tex
\Msg{************************************************************************}
\Msg{* Installation}
\Msg{* Package: hologo 2016/05/12 v1.11 A logo collection with bookmark support (HO)}
\Msg{************************************************************************}

\keepsilent
\askforoverwritefalse

\let\MetaPrefix\relax
\preamble

This is a generated file.

Project: hologo
Version: 2016/05/12 v1.11

Copyright (C) 2010-2012 by
   Heiko Oberdiek <heiko.oberdiek at googlemail.com>

This work may be distributed and/or modified under the
conditions of the LaTeX Project Public License, either
version 1.3c of this license or (at your option) any later
version. This version of this license is in
   http://www.latex-project.org/lppl/lppl-1-3c.txt
and the latest version of this license is in
   http://www.latex-project.org/lppl.txt
and version 1.3 or later is part of all distributions of
LaTeX version 2005/12/01 or later.

This work has the LPPL maintenance status "maintained".

This Current Maintainer of this work is Heiko Oberdiek.

The Base Interpreter refers to any `TeX-Format',
because some files are installed in TDS:tex/generic//.

This work consists of the main source file hologo.dtx
and the derived files
   hologo.sty, hologo.pdf, hologo.ins, hologo.drv, hologo-example.tex,
   hologo-test1.tex, hologo-test-spacefactor.tex,
   hologo-test-list.tex.

\endpreamble
\let\MetaPrefix\DoubleperCent

\generate{%
  \file{hologo.ins}{\from{hologo.dtx}{install}}%
  \file{hologo.drv}{\from{hologo.dtx}{driver}}%
  \usedir{tex/generic/oberdiek}%
  \file{hologo.sty}{\from{hologo.dtx}{package}}%
  \usedir{doc/latex/oberdiek/example}%
  \file{hologo-example.tex}{\from{hologo.dtx}{example}}%
  \usedir{doc/latex/oberdiek/test}%
  \file{hologo-test1.tex}{\from{hologo.dtx}{test1}}%
  \file{hologo-test-spacefactor.tex}{\from{hologo.dtx}{test-spacefactor}}%
  \file{hologo-test-list.tex}{\from{hologo.dtx}{test-list}}%
  \nopreamble
  \nopostamble
  \usedir{source/latex/oberdiek/catalogue}%
  \file{hologo.xml}{\from{hologo.dtx}{catalogue}}%
}

\catcode32=13\relax% active space
\let =\space%
\Msg{************************************************************************}
\Msg{*}
\Msg{* To finish the installation you have to move the following}
\Msg{* file into a directory searched by TeX:}
\Msg{*}
\Msg{*     hologo.sty}
\Msg{*}
\Msg{* To produce the documentation run the file `hologo.drv'}
\Msg{* through LaTeX.}
\Msg{*}
\Msg{* Happy TeXing!}
\Msg{*}
\Msg{************************************************************************}

\endbatchfile
%</install>
%<*ignore>
\fi
%</ignore>
%<*driver>
\NeedsTeXFormat{LaTeX2e}
\ProvidesFile{hologo.drv}%
  [2016/05/12 v1.11 A logo collection with bookmark support (HO)]%
\documentclass{ltxdoc}
\usepackage{holtxdoc}[2011/11/22]
\usepackage{hologo}[2016/05/12]
\usepackage{longtable}
\usepackage{array}
\usepackage{paralist}
%\usepackage[T1]{fontenc}
%\usepackage{lmodern}
\begin{document}
  \DocInput{hologo.dtx}%
\end{document}
%</driver>
% \fi
%
%
% \CharacterTable
%  {Upper-case    \A\B\C\D\E\F\G\H\I\J\K\L\M\N\O\P\Q\R\S\T\U\V\W\X\Y\Z
%   Lower-case    \a\b\c\d\e\f\g\h\i\j\k\l\m\n\o\p\q\r\s\t\u\v\w\x\y\z
%   Digits        \0\1\2\3\4\5\6\7\8\9
%   Exclamation   \!     Double quote  \"     Hash (number) \#
%   Dollar        \$     Percent       \%     Ampersand     \&
%   Acute accent  \'     Left paren    \(     Right paren   \)
%   Asterisk      \*     Plus          \+     Comma         \,
%   Minus         \-     Point         \.     Solidus       \/
%   Colon         \:     Semicolon     \;     Less than     \<
%   Equals        \=     Greater than  \>     Question mark \?
%   Commercial at \@     Left bracket  \[     Backslash     \\
%   Right bracket \]     Circumflex    \^     Underscore    \_
%   Grave accent  \`     Left brace    \{     Vertical bar  \|
%   Right brace   \}     Tilde         \~}
%
% \GetFileInfo{hologo.drv}
%
% \title{The \xpackage{hologo} package}
% \date{2016/05/12 v1.11}
% \author{Heiko Oberdiek\\\xemail{heiko.oberdiek at googlemail.com}}
%
% \maketitle
%
% \begin{abstract}
% This package starts a collection of logos with support for bookmarks
% strings.
% \end{abstract}
%
% \tableofcontents
%
% \section{Documentation}
%
% \subsection{Logo macros}
%
% \begin{declcs}{hologo} \M{name}
% \end{declcs}
% Macro \cs{hologo} sets the logo with name \meta{name}.
% The following table shows the supported names.
%
% \begingroup
%   \def\hologoEntry#1#2#3{^^A
%     #1&#2&\hologoLogoSetup{#1}{variant=#2}\hologo{#1}&#3\tabularnewline
%   }
%   \begin{longtable}{>{\ttfamily}l>{\ttfamily}lll}
%     \rmfamily\bfseries{name} & \rmfamily\bfseries variant
%     & \bfseries logo & \bfseries since\\
%     \hline
%     \endhead
%     \hologoList
%   \end{longtable}
% \endgroup
%
% \begin{declcs}{Hologo} \M{name}
% \end{declcs}
% Macro \cs{Hologo} starts the logo \meta{name} with an uppercase
% letter. As an exception small greek letters are not converted
% to uppercase. Examples, see \hologo{eTeX} and \hologo{ExTeX}.
%
% \subsection{Setup macros}
%
% The package does not support package options, but the following
% setup macros can be used to set options.
%
% \begin{declcs}{hologoSetup} \M{key value list}
% \end{declcs}
% Macro \cs{hologoSetup} sets global options.
%
% \begin{declcs}{hologoLogoSetup} \M{logo} \M{key value list}
% \end{declcs}
% Some options can also be used to configure a logo.
% These settings take precedence over global option settings.
%
% \subsection{Options}\label{sec:options}
%
% There are boolean and string options:
% \begin{description}
% \item[Boolean option:]
% It takes |true| or |false|
% as value. If the value is omitted, then |true| is used.
% \item[String option:]
% A value must be given as string. (But the string might be empty.)
% \end{description}
% The following options can be used both in \cs{hologoSetup}
% and \cs{hologoLogoSetup}:
% \begin{description}
% \def\entry#1{\item[\xoption{#1}:]}
% \entry{break}
%   enables or disables line breaks inside the logo. This setting is
%   refined by options \xoption{hyphenbreak}, \xoption{spacebreak}
%   or \xoption{discretionarybreak}.
%   Default is |false|.
% \entry{hyphenbreak}
%   enables or disables the line break right after the hyphen character.
% \entry{spacebreak}
%   enables or disables line breaks at space characters.
% \entry{discretionarybreak}
%   enables or disables line breaks at hyphenation points
%   (inserted by \cs{-}).
% \end{description}
% Macro \cs{hologoLogoSetup} also knows:
% \begin{description}
% \item[\xoption{variant}:]
%   This is a string option. It specifies a variant of a logo that
%   must exist. An empty string selects the package default variant.
% \end{description}
% Example:
% \begin{quote}
%   |\hologoSetup{break=false}|\\
%   |\hologoLogoSetup{plainTeX}{variant=hyphen,hyphenbreak}|\\
%   Then ``plain-\TeX'' contains one break point after the hyphen.
% \end{quote}
%
% \subsection{Driver options}
%
% Sometimes graphical operations are needed to construct some
% glyphs (e.g.\ \hologo{XeTeX}). If package \xpackage{graphics}
% or package \xpackage{pgf} are found, then the macros are taken
% from there. Otherwise the packge defines its own operations
% and therefore needs the driver information. Many drivers are
% detected automatically (\hologo{pdfTeX}/\hologo{LuaTeX}
% in PDF mode, \hologo{XeTeX}, \hologo{VTeX}). These have precedence
% over a driver option. The driver can be given as package option
% or using \cs{hologoDriverSetup}.
% The following list contains the recognized driver options:
% \begin{itemize}
% \item \xoption{pdftex}, \xoption{luatex}
% \item \xoption{dvipdfm}, \xoption{dvipdfmx}
% \item \xoption{dvips}, \xoption{dvipsone}, \xoption{xdvi}
% \item \xoption{xetex}
% \item \xoption{vtex}
% \end{itemize}
% The left driver of a line is the driver name that is used internally.
% The following names are aliases for drivers that use the
% same method. Therefore the entry in the \xext{log} file for
% the used driver prints the internally used driver name.
% \begin{description}
% \item[\xoption{driverfallback}:]
%   This option expects a driver that is used,
%   if the driver could not be detected automatically.
% \end{description}
%
% \begin{declcs}{hologoDriverSetup} \M{driver option}
% \end{declcs}
% The driver can also be configured after package loading
% using \cs{hologoDriverSetup}, also the way for \hologo{plainTeX}
% to setup the driver.
%
% \subsection{Font setup}
%
% Some logos require a special font, but should also be usable by
% \hologo{plainTeX}. Therefore the package provides some ways
% to influence the font settings. The options below
% take font settings as values. Both font commands
% such as \cs{sffamily} and macros that take one argument
% like \cs{textsf} can be used.
%
% \begin{declcs}{hologoFontSetup} \M{key value list}
% \end{declcs}
% Macro \cs{hologoFontSetup} sets the fonts for all logos.
% Supported keys:
% \begin{description}
% \def\entry#1{\item[\xoption{#1}:]}
% \entry{general}
%   This font is used for all logos. The default is empty.
%   That means no special font is used.
% \entry{bibsf}
%   This font is used for
%   {\hologoLogoSetup{BibTeX}{variant=sf}\hologo{BibTeX}}
%   with variant \xoption{sf}.
% \entry{rm}
%   This font is a serif font. It is used for \hologo{ExTeX}.
% \entry{sc}
%   This font specifies a small caps font. It is used for
%   {\hologoLogoSetup{BibTeX}{variant=sc}\hologo{BibTeX}}
%   with variant \xoption{sc}.
% \entry{sf}
%   This font specifies a sans serif font. The default
%   is \cs{sffamily}, then \cs{sf} is tried. Otherwise
%   a warning is given. It is used by \hologo{KOMAScript}.
% \entry{sy}
%   This is the font for math symbols (e.g. cmsy).
%   It is used by \hologo{AmS}, \hologo{NTS}, \hologo{ExTeX}.
% \entry{logo}
%   \hologo{METAFONT} and \hologo{METAPOST} are using that font.
%   In \hologo{LaTeX} \cs{logofamily} is used and
%   the definitions of package \xpackage{mflogo} are used
%   if the package is not loaded.
%   Otherwise the \cs{tenlogo} is used and defined
%   if it does not already exists.
% \end{description}
%
% \begin{declcs}{hologoLogoFontSetup} \M{logo} \M{key value list}
% \end{declcs}
% Fonts can also be set for a logo or logo component separately,
% see the following list.
% The keys are the same as for \cs{hologoFontSetup}.
%
% \begin{longtable}{>{\ttfamily}l>{\sffamily}ll}
%   \meta{logo} & keys & result\\
%   \hline
%   \endhead
%   BibTeX & bibsf & {\hologoLogoSetup{BibTeX}{variant=sf}\hologo{BibTeX}}\\[.5ex]
%   BibTeX & sc & {\hologoLogoSetup{BibTeX}{variant=sc}\hologo{BibTeX}}\\[.5ex]
%   ExTeX & rm & \hologo{ExTeX}\\
%   SliTeX & rm & \hologo{SliTeX}\\[.5ex]
%   AmS & sy & \hologo{AmS}\\
%   ExTeX & sy & \hologo{ExTeX}\\
%   NTS & sy & \hologo{NTS}\\[.5ex]
%   KOMAScript & sf & \hologo{KOMAScript}\\[.5ex]
%   METAFONT & logo & \hologo{METAFONT}\\
%   METAPOST & logo & \hologo{METAPOST}\\[.5ex]
%   SliTeX & sc \hologo{SliTeX}
% \end{longtable}
%
% \subsubsection{Font order}
%
% For all logos the font \xoption{general} is applied first.
% Example:
%\begin{quote}
%|\hologoFontSetup{general=\color{red}}|
%\end{quote}
% will print red logos.
% Then if the font uses a special font \xoption{sf}, for example,
% the font is applied that is setup by \cs{hologoLogoFontSetup}.
% If this font is not setup, then the common font setup
% by \cs{hologoFontSetup} is used. Otherwise a warning is given,
% that there is no font configured.
%
% \subsection{Additional user macros}
%
% Usually a variant of a logo is configured by using
% \cs{hologoLogoSetup}, because it is bad style to mix
% different variants of the same logo in the same text.
% There the following macros are a convenience for testing.
%
% \begin{declcs}{hologoVariant} \M{name} \M{variant}\\
%   \cs{HologoVariant} \M{name} \M{variant}
% \end{declcs}
% Logo \meta{name} is set using \meta{variant} that specifies
% explicitely which variant of the macro is used. If the argument
% is empty, then the default form of the logo is used
% (configurable by \cs{hologoLogoSetup}).
%
% \cs{HologoVariant} is used if the logo is set in a context
% that needs an uppercase first letter (beginning of a sentence, \dots).
%
% \begin{declcs}{hologoList}\\
%   \cs{hologoEntry} \M{logo} \M{variant} \M{since}
% \end{declcs}
% Macro \cs{hologoList} contains all logos that are provided
% by the package including variants. The list consists of calls
% of \cs{hologoEntry} with three arguments starting with the
% logo name \meta{logo} and its variant \meta{variant}. An empty
% variant means the current default. Argument \meta{since} specifies
% with version of the package \xpackage{hologo} is needed to get
% the logo. If the logo is fixed, then the date gets updated.
% Therefore the date \meta{since} is not exactly the date of
% the first introduction, but rather the date of the latest fix.
%
% Before \cs{hologoList} can be used, macro \cs{hologoEntry} needs
% a definition. The example file in section \ref{sec:example}
% shows applications of \cs{hologoList}.
%
% \subsection{Supported contexts}
%
% Macros \cs{hologo} and friends support special contexts:
% \begin{itemize}
% \item \hologo{LaTeX}'s protection mechanism.
% \item Bookmarks of package \xpackage{hyperref}.
% \item Package \xpackage{tex4ht}.
% \item The macros can be used inside \cs{csname} constructs,
%   if \cs{ifincsname} is available (\hologo{pdfTeX}, \hologo{XeTeX},
%   \hologo{LuaTeX}).
% \end{itemize}
%
% \subsection{Example}
% \label{sec:example}
%
% The following example prints the logos in different fonts.
%    \begin{macrocode}
%<*example>
%<<verbatim
\NeedsTeXFormat{LaTeX2e}
\documentclass[a4paper]{article}
\usepackage[
  hmargin=20mm,
  vmargin=20mm,
]{geometry}
\pagestyle{empty}
\usepackage{hologo}[2016/05/12]
\usepackage{longtable}
\usepackage{array}
\setlength{\extrarowheight}{2pt}
\usepackage[T1]{fontenc}
\usepackage{lmodern}
\usepackage{pdflscape}
\usepackage[
  pdfencoding=auto,
]{hyperref}
\hypersetup{
  pdfauthor={Heiko Oberdiek},
  pdftitle={Example for package `hologo'},
  pdfsubject={Logos with fonts lmr, lmss, qtm, qpl, qhv},
}
\usepackage{bookmark}

% Print the logo list on the console

\begingroup
  \typeout{}%
  \typeout{*** Begin of logo list ***}%
  \newcommand*{\hologoEntry}[3]{%
    \typeout{#1 \ifx\\#2\\\else(#2) \fi[#3]}%
  }%
  \hologoList
  \typeout{*** End of logo list ***}%
  \typeout{}%
\endgroup

\begin{document}
\begin{landscape}

  \section{Example file for package `hologo'}

  % Table for font names

  \begin{longtable}{>{\bfseries}ll}
    \textbf{font} & \textbf{Font name}\\
    \hline
    lmr & Latin Modern Roman\\
    lmss & Latin Modern Sans\\
    qtm & \TeX\ Gyre Termes\\
    qhv & \TeX\ Gyre Heros\\
    qpl & \TeX\ Gyre Pagella\\
  \end{longtable}

  % Logo list with logos in different fonts

  \begingroup
    \newcommand*{\SetVariant}[2]{%
      \ifx\\#2\\%
      \else
        \hologoLogoSetup{#1}{variant=#2}%
      \fi
    }%
    \newcommand*{\hologoEntry}[3]{%
      \SetVariant{#1}{#2}%
      \raisebox{1em}[0pt][0pt]{\hypertarget{#1@#2}{}}%
      \bookmark[%
        dest={#1@#2},%
      ]{%
        #1\ifx\\#2\\\else\space(#2)\fi: \Hologo{#1}, \hologo{#1} %
        [Unicode]%
      }%
      \hypersetup{unicode=false}%
      \bookmark[%
        dest={#1@#2},%
      ]{%
        #1\ifx\\#2\\\else\space(#2)\fi: \Hologo{#1}, \hologo{#1} %
        [PDFDocEncoding]%
      }%
      \texttt{#1}%
      &%
      \texttt{#2}%
      &%
      \Hologo{#1}%
      &%
      \SetVariant{#1}{#2}%
      \hologo{#1}%
      &%
      \SetVariant{#1}{#2}%
      \fontfamily{qtm}\selectfont
      \hologo{#1}%
      &%
      \SetVariant{#1}{#2}%
      \fontfamily{qpl}\selectfont
      \hologo{#1}%
      &%
      \SetVariant{#1}{#2}%
      \textsf{\hologo{#1}}%
      &%
      \SetVariant{#1}{#2}%
      \fontfamily{qhv}\selectfont
      \hologo{#1}%
      \tabularnewline
    }%
    \begin{longtable}{llllllll}%
      \textbf{\textit{logo}} & \textbf{\textit{variant}} &
      \texttt{\string\Hologo} &
      \textbf{lmr} & \textbf{qtm} & \textbf{qpl} &
      \textbf{lmss} & \textbf{qhv}
      \tabularnewline
      \hline
      \endhead
      \hologoList
    \end{longtable}%
  \endgroup

\end{landscape}
\end{document}
%verbatim
%</example>
%    \end{macrocode}
%
% \StopEventually{
% }
%
% \section{Implementation}
%    \begin{macrocode}
%<*package>
%    \end{macrocode}
%    Reload check, especially if the package is not used with \LaTeX.
%    \begin{macrocode}
\begingroup\catcode61\catcode48\catcode32=10\relax%
  \catcode13=5 % ^^M
  \endlinechar=13 %
  \catcode35=6 % #
  \catcode39=12 % '
  \catcode44=12 % ,
  \catcode45=12 % -
  \catcode46=12 % .
  \catcode58=12 % :
  \catcode64=11 % @
  \catcode123=1 % {
  \catcode125=2 % }
  \expandafter\let\expandafter\x\csname ver@hologo.sty\endcsname
  \ifx\x\relax % plain-TeX, first loading
  \else
    \def\empty{}%
    \ifx\x\empty % LaTeX, first loading,
      % variable is initialized, but \ProvidesPackage not yet seen
    \else
      \expandafter\ifx\csname PackageInfo\endcsname\relax
        \def\x#1#2{%
          \immediate\write-1{Package #1 Info: #2.}%
        }%
      \else
        \def\x#1#2{\PackageInfo{#1}{#2, stopped}}%
      \fi
      \x{hologo}{The package is already loaded}%
      \aftergroup\endinput
    \fi
  \fi
\endgroup%
%    \end{macrocode}
%    Package identification:
%    \begin{macrocode}
\begingroup\catcode61\catcode48\catcode32=10\relax%
  \catcode13=5 % ^^M
  \endlinechar=13 %
  \catcode35=6 % #
  \catcode39=12 % '
  \catcode40=12 % (
  \catcode41=12 % )
  \catcode44=12 % ,
  \catcode45=12 % -
  \catcode46=12 % .
  \catcode47=12 % /
  \catcode58=12 % :
  \catcode64=11 % @
  \catcode91=12 % [
  \catcode93=12 % ]
  \catcode123=1 % {
  \catcode125=2 % }
  \expandafter\ifx\csname ProvidesPackage\endcsname\relax
    \def\x#1#2#3[#4]{\endgroup
      \immediate\write-1{Package: #3 #4}%
      \xdef#1{#4}%
    }%
  \else
    \def\x#1#2[#3]{\endgroup
      #2[{#3}]%
      \ifx#1\@undefined
        \xdef#1{#3}%
      \fi
      \ifx#1\relax
        \xdef#1{#3}%
      \fi
    }%
  \fi
\expandafter\x\csname ver@hologo.sty\endcsname
\ProvidesPackage{hologo}%
  [2016/05/12 v1.11 A logo collection with bookmark support (HO)]%
%    \end{macrocode}
%
%    \begin{macrocode}
\begingroup\catcode61\catcode48\catcode32=10\relax%
  \catcode13=5 % ^^M
  \endlinechar=13 %
  \catcode123=1 % {
  \catcode125=2 % }
  \catcode64=11 % @
  \def\x{\endgroup
    \expandafter\edef\csname HOLOGO@AtEnd\endcsname{%
      \endlinechar=\the\endlinechar\relax
      \catcode13=\the\catcode13\relax
      \catcode32=\the\catcode32\relax
      \catcode35=\the\catcode35\relax
      \catcode61=\the\catcode61\relax
      \catcode64=\the\catcode64\relax
      \catcode123=\the\catcode123\relax
      \catcode125=\the\catcode125\relax
    }%
  }%
\x\catcode61\catcode48\catcode32=10\relax%
\catcode13=5 % ^^M
\endlinechar=13 %
\catcode35=6 % #
\catcode64=11 % @
\catcode123=1 % {
\catcode125=2 % }
\def\TMP@EnsureCode#1#2{%
  \edef\HOLOGO@AtEnd{%
    \HOLOGO@AtEnd
    \catcode#1=\the\catcode#1\relax
  }%
  \catcode#1=#2\relax
}
\TMP@EnsureCode{10}{12}% ^^J
\TMP@EnsureCode{33}{12}% !
\TMP@EnsureCode{34}{12}% "
\TMP@EnsureCode{36}{3}% $
\TMP@EnsureCode{38}{4}% &
\TMP@EnsureCode{39}{12}% '
\TMP@EnsureCode{40}{12}% (
\TMP@EnsureCode{41}{12}% )
\TMP@EnsureCode{42}{12}% *
\TMP@EnsureCode{43}{12}% +
\TMP@EnsureCode{44}{12}% ,
\TMP@EnsureCode{45}{12}% -
\TMP@EnsureCode{46}{12}% .
\TMP@EnsureCode{47}{12}% /
\TMP@EnsureCode{58}{12}% :
\TMP@EnsureCode{59}{12}% ;
\TMP@EnsureCode{60}{12}% <
\TMP@EnsureCode{62}{12}% >
\TMP@EnsureCode{63}{12}% ?
\TMP@EnsureCode{91}{12}% [
\TMP@EnsureCode{93}{12}% ]
\TMP@EnsureCode{94}{7}% ^ (superscript)
\TMP@EnsureCode{95}{8}% _ (subscript)
\TMP@EnsureCode{96}{12}% `
\TMP@EnsureCode{124}{12}% |
\edef\HOLOGO@AtEnd{%
  \HOLOGO@AtEnd
  \escapechar\the\escapechar\relax
  \noexpand\endinput
}
\escapechar=92 %
%    \end{macrocode}
%
% \subsection{Logo list}
%
%    \begin{macro}{\hologoList}
%    \begin{macrocode}
\def\hologoList{%
  \hologoEntry{(La)TeX}{}{2011/10/01}%
  \hologoEntry{AmSLaTeX}{}{2010/04/16}%
  \hologoEntry{AmSTeX}{}{2010/04/16}%
  \hologoEntry{biber}{}{2011/10/01}%
  \hologoEntry{BibTeX}{}{2011/10/01}%
  \hologoEntry{BibTeX}{sf}{2011/10/01}%
  \hologoEntry{BibTeX}{sc}{2011/10/01}%
  \hologoEntry{BibTeX8}{}{2011/11/22}%
  \hologoEntry{ConTeXt}{}{2011/03/25}%
  \hologoEntry{ConTeXt}{narrow}{2011/03/25}%
  \hologoEntry{ConTeXt}{simple}{2011/03/25}%
  \hologoEntry{emTeX}{}{2010/04/26}%
  \hologoEntry{eTeX}{}{2010/04/08}%
  \hologoEntry{ExTeX}{}{2011/10/01}%
  \hologoEntry{HanTheThanh}{}{2011/11/29}%
  \hologoEntry{iniTeX}{}{2011/10/01}%
  \hologoEntry{KOMAScript}{}{2011/10/01}%
  \hologoEntry{La}{}{2010/05/08}%
  \hologoEntry{LaTeX}{}{2010/04/08}%
  \hologoEntry{LaTeX2e}{}{2010/04/08}%
  \hologoEntry{LaTeX3}{}{2010/04/24}%
  \hologoEntry{LaTeXe}{}{2010/04/08}%
  \hologoEntry{LaTeXML}{}{2011/11/22}%
  \hologoEntry{LaTeXTeX}{}{2011/10/01}%
  \hologoEntry{LuaLaTeX}{}{2010/04/08}%
  \hologoEntry{LuaTeX}{}{2010/04/08}%
  \hologoEntry{LyX}{}{2011/10/01}%
  \hologoEntry{METAFONT}{}{2011/10/01}%
  \hologoEntry{MetaFun}{}{2011/10/01}%
  \hologoEntry{METAPOST}{}{2011/10/01}%
  \hologoEntry{MetaPost}{}{2011/10/01}%
  \hologoEntry{MiKTeX}{}{2011/10/01}%
  \hologoEntry{NTS}{}{2011/10/01}%
  \hologoEntry{OzMF}{}{2011/10/01}%
  \hologoEntry{OzMP}{}{2011/10/01}%
  \hologoEntry{OzTeX}{}{2011/10/01}%
  \hologoEntry{OzTtH}{}{2011/10/01}%
  \hologoEntry{PCTeX}{}{2011/10/01}%
  \hologoEntry{pdfTeX}{}{2011/10/01}%
  \hologoEntry{pdfLaTeX}{}{2011/10/01}%
  \hologoEntry{PiC}{}{2011/10/01}%
  \hologoEntry{PiCTeX}{}{2011/10/01}%
  \hologoEntry{plainTeX}{}{2010/04/08}%
  \hologoEntry{plainTeX}{space}{2010/04/16}%
  \hologoEntry{plainTeX}{hyphen}{2010/04/16}%
  \hologoEntry{plainTeX}{runtogether}{2010/04/16}%
  \hologoEntry{SageTeX}{}{2011/11/22}%
  \hologoEntry{SLiTeX}{}{2011/10/01}%
  \hologoEntry{SLiTeX}{lift}{2011/10/01}%
  \hologoEntry{SLiTeX}{narrow}{2011/10/01}%
  \hologoEntry{SLiTeX}{simple}{2011/10/01}%
  \hologoEntry{SliTeX}{}{2011/10/01}%
  \hologoEntry{SliTeX}{narrow}{2011/10/01}%
  \hologoEntry{SliTeX}{simple}{2011/10/01}%
  \hologoEntry{SliTeX}{lift}{2011/10/01}%
  \hologoEntry{teTeX}{}{2011/10/01}%
  \hologoEntry{TeX}{}{2010/04/08}%
  \hologoEntry{TeX4ht}{}{2011/11/22}%
  \hologoEntry{TTH}{}{2011/11/22}%
  \hologoEntry{virTeX}{}{2011/10/01}%
  \hologoEntry{VTeX}{}{2010/04/24}%
  \hologoEntry{Xe}{}{2010/04/08}%
  \hologoEntry{XeLaTeX}{}{2010/04/08}%
  \hologoEntry{XeTeX}{}{2010/04/08}%
}
%    \end{macrocode}
%    \end{macro}
%
% \subsection{Load resources}
%
%    \begin{macrocode}
\begingroup\expandafter\expandafter\expandafter\endgroup
\expandafter\ifx\csname RequirePackage\endcsname\relax
  \def\TMP@RequirePackage#1[#2]{%
    \begingroup\expandafter\expandafter\expandafter\endgroup
    \expandafter\ifx\csname ver@#1.sty\endcsname\relax
      \input #1.sty\relax
    \fi
  }%
  \TMP@RequirePackage{ltxcmds}[2011/02/04]%
  \TMP@RequirePackage{infwarerr}[2010/04/08]%
  \TMP@RequirePackage{kvsetkeys}[2010/03/01]%
  \TMP@RequirePackage{kvdefinekeys}[2010/03/01]%
  \TMP@RequirePackage{pdftexcmds}[2010/04/01]%
  \TMP@RequirePackage{ifpdf}[2010/01/28]%
  \TMP@RequirePackage{ifluatex}[2010/03/01]%
  \ltx@IfUndefined{newif}{%
    \expandafter\let\csname newif\endcsname\ltx@newif
  }{}%
  \TMP@RequirePackage{ifxetex}[2009/01/23]%
  \TMP@RequirePackage{ifvtex}[2010/03/01]%
\else
  \RequirePackage{ltxcmds}[2011/02/04]%
  \RequirePackage{infwarerr}[2010/04/08]%
  \RequirePackage{kvsetkeys}[2010/03/01]%
  \RequirePackage{kvdefinekeys}[2010/03/01]%
  \RequirePackage{pdftexcmds}[2010/04/01]%
  \RequirePackage{ifpdf}[2010/01/28]%
  \RequirePackage{ifluatex}[2010/03/01]%
  \RequirePackage{ifxetex}[2009/01/23]%
  \RequirePackage{ifvtex}[2010/03/01]%
\fi
%    \end{macrocode}
%
%    \begin{macro}{\HOLOGO@IfDefined}
%    \begin{macrocode}
\def\HOLOGO@IfExists#1{%
  \ifx\@undefined#1%
    \expandafter\ltx@secondoftwo
  \else
    \ifx\relax#1%
      \expandafter\ltx@secondoftwo
    \else
      \expandafter\expandafter\expandafter\ltx@firstoftwo
    \fi
  \fi
}
%    \end{macrocode}
%    \end{macro}
%
% \subsection{Setup macros}
%
%    \begin{macro}{\hologoSetup}
%    \begin{macrocode}
\def\hologoSetup{%
  \let\HOLOGO@name\relax
  \HOLOGO@Setup
}
%    \end{macrocode}
%    \end{macro}
%
%    \begin{macro}{\hologoLogoSetup}
%    \begin{macrocode}
\def\hologoLogoSetup#1{%
  \edef\HOLOGO@name{#1}%
  \ltx@IfUndefined{HoLogo@\HOLOGO@name}{%
    \@PackageError{hologo}{%
      Unknown logo `\HOLOGO@name'%
    }\@ehc
    \ltx@gobble
  }{%
    \HOLOGO@Setup
  }%
}
%    \end{macrocode}
%    \end{macro}
%
%    \begin{macro}{\HOLOGO@Setup}
%    \begin{macrocode}
\def\HOLOGO@Setup{%
  \kvsetkeys{HoLogo}%
}
%    \end{macrocode}
%    \end{macro}
%
% \subsection{Options}
%
%    \begin{macro}{\HOLOGO@DeclareBoolOption}
%    \begin{macrocode}
\def\HOLOGO@DeclareBoolOption#1{%
  \expandafter\chardef\csname HOLOGOOPT@#1\endcsname\ltx@zero
  \kv@define@key{HoLogo}{#1}[true]{%
    \def\HOLOGO@temp{##1}%
    \ifx\HOLOGO@temp\HOLOGO@true
      \ifx\HOLOGO@name\relax
        \expandafter\chardef\csname HOLOGOOPT@#1\endcsname=\ltx@one
      \else
        \expandafter\chardef\csname
        HoLogoOpt@#1@\HOLOGO@name\endcsname\ltx@one
      \fi
      \HOLOGO@SetBreakAll{#1}%
    \else
      \ifx\HOLOGO@temp\HOLOGO@false
        \ifx\HOLOGO@name\relax
          \expandafter\chardef\csname HOLOGOOPT@#1\endcsname=\ltx@zero
        \else
          \expandafter\chardef\csname
          HoLogoOpt@#1@\HOLOGO@name\endcsname=\ltx@zero
        \fi
        \HOLOGO@SetBreakAll{#1}%
      \else
        \@PackageError{hologo}{%
          Unknown value `##1' for boolean option `#1'.\MessageBreak
          Known values are `true' and `false'%
        }\@ehc
      \fi
    \fi
  }%
}
%    \end{macrocode}
%    \end{macro}
%
%    \begin{macro}{\HOLOGO@SetBreakAll}
%    \begin{macrocode}
\def\HOLOGO@SetBreakAll#1{%
  \def\HOLOGO@temp{#1}%
  \ifx\HOLOGO@temp\HOLOGO@break
    \ifx\HOLOGO@name\relax
      \chardef\HOLOGOOPT@hyphenbreak=\HOLOGOOPT@break
      \chardef\HOLOGOOPT@spacebreak=\HOLOGOOPT@break
      \chardef\HOLOGOOPT@discretionarybreak=\HOLOGOOPT@break
    \else
      \expandafter\chardef
         \csname HoLogoOpt@hyphenbreak@\HOLOGO@name\endcsname=%
         \csname HoLogoOpt@break@\HOLOGO@name\endcsname
      \expandafter\chardef
         \csname HoLogoOpt@spacebreak@\HOLOGO@name\endcsname=%
         \csname HoLogoOpt@break@\HOLOGO@name\endcsname
      \expandafter\chardef
         \csname HoLogoOpt@discretionarybreak@\HOLOGO@name
             \endcsname=%
         \csname HoLogoOpt@break@\HOLOGO@name\endcsname
    \fi
  \fi
}
%    \end{macrocode}
%    \end{macro}
%
%    \begin{macro}{\HOLOGO@true}
%    \begin{macrocode}
\def\HOLOGO@true{true}
%    \end{macrocode}
%    \end{macro}
%    \begin{macro}{\HOLOGO@false}
%    \begin{macrocode}
\def\HOLOGO@false{false}
%    \end{macrocode}
%    \end{macro}
%    \begin{macro}{\HOLOGO@break}
%    \begin{macrocode}
\def\HOLOGO@break{break}
%    \end{macrocode}
%    \end{macro}
%
%    \begin{macrocode}
\HOLOGO@DeclareBoolOption{break}
\HOLOGO@DeclareBoolOption{hyphenbreak}
\HOLOGO@DeclareBoolOption{spacebreak}
\HOLOGO@DeclareBoolOption{discretionarybreak}
%    \end{macrocode}
%
%    \begin{macrocode}
\kv@define@key{HoLogo}{variant}{%
  \ifx\HOLOGO@name\relax
    \@PackageError{hologo}{%
      Option `variant' is not available in \string\hologoSetup,%
      \MessageBreak
      Use \string\hologoLogoSetup\space instead%
    }\@ehc
  \else
    \edef\HOLOGO@temp{#1}%
    \ifx\HOLOGO@temp\ltx@empty
      \expandafter
      \let\csname HoLogoOpt@variant@\HOLOGO@name\endcsname\@undefined
    \else
      \ltx@IfUndefined{HoLogo@\HOLOGO@name @\HOLOGO@temp}{%
        \@PackageError{hologo}{%
          Unknown variant `\HOLOGO@temp' of logo `\HOLOGO@name'%
        }\@ehc
      }{%
        \expandafter
        \let\csname HoLogoOpt@variant@\HOLOGO@name\endcsname
            \HOLOGO@temp
      }%
    \fi
  \fi
}
%    \end{macrocode}
%
%    \begin{macro}{\HOLOGO@Variant}
%    \begin{macrocode}
\def\HOLOGO@Variant#1{%
  #1%
  \ltx@ifundefined{HoLogoOpt@variant@#1}{%
  }{%
    @\csname HoLogoOpt@variant@#1\endcsname
  }%
}
%    \end{macrocode}
%    \end{macro}
%
% \subsection{Break/no-break support}
%
%    \begin{macro}{\HOLOGO@space}
%    \begin{macrocode}
\def\HOLOGO@space{%
  \ltx@ifundefined{HoLogoOpt@spacebreak@\HOLOGO@name}{%
    \ltx@ifundefined{HoLogoOpt@break@\HOLOGO@name}{%
      \chardef\HOLOGO@temp=\HOLOGOOPT@spacebreak
    }{%
      \chardef\HOLOGO@temp=%
        \csname HoLogoOpt@break@\HOLOGO@name\endcsname
    }%
  }{%
    \chardef\HOLOGO@temp=%
      \csname HoLogoOpt@spacebreak@\HOLOGO@name\endcsname
  }%
  \ifcase\HOLOGO@temp
    \penalty10000 %
  \fi
  \ltx@space
}
%    \end{macrocode}
%    \end{macro}
%
%    \begin{macro}{\HOLOGO@hyphen}
%    \begin{macrocode}
\def\HOLOGO@hyphen{%
  \ltx@ifundefined{HoLogoOpt@hyphenbreak@\HOLOGO@name}{%
    \ltx@ifundefined{HoLogoOpt@break@\HOLOGO@name}{%
      \chardef\HOLOGO@temp=\HOLOGOOPT@hyphenbreak
    }{%
      \chardef\HOLOGO@temp=%
        \csname HoLogoOpt@break@\HOLOGO@name\endcsname
    }%
  }{%
    \chardef\HOLOGO@temp=%
      \csname HoLogoOpt@hyphenbreak@\HOLOGO@name\endcsname
  }%
  \ifcase\HOLOGO@temp
    \ltx@mbox{-}%
  \else
    -%
  \fi
}
%    \end{macrocode}
%    \end{macro}
%
%    \begin{macro}{\HOLOGO@discretionary}
%    \begin{macrocode}
\def\HOLOGO@discretionary{%
  \ltx@ifundefined{HoLogoOpt@discretionarybreak@\HOLOGO@name}{%
    \ltx@ifundefined{HoLogoOpt@break@\HOLOGO@name}{%
      \chardef\HOLOGO@temp=\HOLOGOOPT@discretionarybreak
    }{%
      \chardef\HOLOGO@temp=%
        \csname HoLogoOpt@break@\HOLOGO@name\endcsname
    }%
  }{%
    \chardef\HOLOGO@temp=%
      \csname HoLogoOpt@discretionarybreak@\HOLOGO@name\endcsname
  }%
  \ifcase\HOLOGO@temp
  \else
    \-%
  \fi
}
%    \end{macrocode}
%    \end{macro}
%
%    \begin{macro}{\HOLOGO@mbox}
%    \begin{macrocode}
\def\HOLOGO@mbox#1{%
  \ltx@ifundefined{HoLogoOpt@break@\HOLOGO@name}{%
    \chardef\HOLOGO@temp=\HOLOGOOPT@hyphenbreak
  }{%
    \chardef\HOLOGO@temp=%
      \csname HoLogoOpt@break@\HOLOGO@name\endcsname
  }%
  \ifcase\HOLOGO@temp
    \ltx@mbox{#1}%
  \else
    #1%
  \fi
}
%    \end{macrocode}
%    \end{macro}
%
% \subsection{Font support}
%
%    \begin{macro}{\HoLogoFont@font}
%    \begin{tabular}{@{}ll@{}}
%    |#1|:& logo name\\
%    |#2|:& font short name\\
%    |#3|:& text
%    \end{tabular}
%    \begin{macrocode}
\def\HoLogoFont@font#1#2#3{%
  \begingroup
    \ltx@IfUndefined{HoLogoFont@logo@#1.#2}{%
      \ltx@IfUndefined{HoLogoFont@font@#2}{%
        \@PackageWarning{hologo}{%
          Missing font `#2' for logo `#1'%
        }%
        #3%
      }{%
        \csname HoLogoFont@font@#2\endcsname{#3}%
      }%
    }{%
      \csname HoLogoFont@logo@#1.#2\endcsname{#3}%
    }%
  \endgroup
}
%    \end{macrocode}
%    \end{macro}
%
%    \begin{macro}{\HoLogoFont@Def}
%    \begin{macrocode}
\def\HoLogoFont@Def#1{%
  \expandafter\def\csname HoLogoFont@font@#1\endcsname
}
%    \end{macrocode}
%    \end{macro}
%    \begin{macro}{\HoLogoFont@LogoDef}
%    \begin{macrocode}
\def\HoLogoFont@LogoDef#1#2{%
  \expandafter\def\csname HoLogoFont@logo@#1.#2\endcsname
}
%    \end{macrocode}
%    \end{macro}
%
% \subsubsection{Font defaults}
%
%    \begin{macro}{\HoLogoFont@font@general}
%    \begin{macrocode}
\HoLogoFont@Def{general}{}%
%    \end{macrocode}
%    \end{macro}
%
%    \begin{macro}{\HoLogoFont@font@rm}
%    \begin{macrocode}
\ltx@IfUndefined{rmfamily}{%
  \ltx@IfUndefined{rm}{%
  }{%
    \HoLogoFont@Def{rm}{\rm}%
  }%
}{%
  \HoLogoFont@Def{rm}{\rmfamily}%
}
%    \end{macrocode}
%    \end{macro}
%
%    \begin{macro}{\HoLogoFont@font@sf}
%    \begin{macrocode}
\ltx@IfUndefined{sffamily}{%
  \ltx@IfUndefined{sf}{%
  }{%
    \HoLogoFont@Def{sf}{\sf}%
  }%
}{%
  \HoLogoFont@Def{sf}{\sffamily}%
}
%    \end{macrocode}
%    \end{macro}
%
%    \begin{macro}{\HoLogoFont@font@bibsf}
%    In case of \hologo{plainTeX} the original small caps
%    variant is used as default. In \hologo{LaTeX}
%    the definition of package \xpackage{dtklogos} \cite{dtklogos}
%    is used.
%\begin{quote}
%\begin{verbatim}
%\DeclareRobustCommand{\BibTeX}{%
%  B%
%  \kern-.05em%
%  \hbox{%
%    $\m@th$% %% force math size calculations
%    \csname S@\f@size\endcsname
%    \fontsize\sf@size\z@
%    \math@fontsfalse
%    \selectfont
%    I%
%    \kern-.025em%
%    B
%  }%
%  \kern-.08em%
%  \-%
%  \TeX
%}
%\end{verbatim}
%\end{quote}
%    \begin{macrocode}
\ltx@IfUndefined{selectfont}{%
  \ltx@IfUndefined{tensc}{%
    \font\tensc=cmcsc10\relax
  }{}%
  \HoLogoFont@Def{bibsf}{\tensc}%
}{%
  \HoLogoFont@Def{bibsf}{%
    $\mathsurround=0pt$%
    \csname S@\f@size\endcsname
    \fontsize\sf@size{0pt}%
    \math@fontsfalse
    \selectfont
  }%
}
%    \end{macrocode}
%    \end{macro}
%
%    \begin{macro}{\HoLogoFont@font@sc}
%    \begin{macrocode}
\ltx@IfUndefined{scshape}{%
  \ltx@IfUndefined{tensc}{%
    \font\tensc=cmcsc10\relax
  }{}%
  \HoLogoFont@Def{sc}{\tensc}%
}{%
  \HoLogoFont@Def{sc}{\scshape}%
}
%    \end{macrocode}
%    \end{macro}
%
%    \begin{macro}{\HoLogoFont@font@sy}
%    \begin{macrocode}
\ltx@IfUndefined{usefont}{%
  \ltx@IfUndefined{tensy}{%
  }{%
    \HoLogoFont@Def{sy}{\tensy}%
  }%
}{%
  \HoLogoFont@Def{sy}{%
    \usefont{OMS}{cmsy}{m}{n}%
  }%
}
%    \end{macrocode}
%    \end{macro}
%
%    \begin{macro}{\HoLogoFont@font@logo}
%    \begin{macrocode}
\begingroup
  \def\x{LaTeX2e}%
\expandafter\endgroup
\ifx\fmtname\x
  \ltx@IfUndefined{logofamily}{%
    \DeclareRobustCommand\logofamily{%
      \not@math@alphabet\logofamily\relax
      \fontencoding{U}%
      \fontfamily{logo}%
      \selectfont
    }%
  }{}%
  \ltx@IfUndefined{logofamily}{%
  }{%
    \HoLogoFont@Def{logo}{\logofamily}%
  }%
\else
  \ltx@IfUndefined{tenlogo}{%
    \font\tenlogo=logo10\relax
  }{}%
  \HoLogoFont@Def{logo}{\tenlogo}%
\fi
%    \end{macrocode}
%    \end{macro}
%
% \subsubsection{Font setup}
%
%    \begin{macro}{\hologoFontSetup}
%    \begin{macrocode}
\def\hologoFontSetup{%
  \let\HOLOGO@name\relax
  \HOLOGO@FontSetup
}
%    \end{macrocode}
%    \end{macro}
%
%    \begin{macro}{\hologoLogoFontSetup}
%    \begin{macrocode}
\def\hologoLogoFontSetup#1{%
  \edef\HOLOGO@name{#1}%
  \ltx@IfUndefined{HoLogo@\HOLOGO@name}{%
    \@PackageError{hologo}{%
      Unknown logo `\HOLOGO@name'%
    }\@ehc
    \ltx@gobble
  }{%
    \HOLOGO@FontSetup
  }%
}
%    \end{macrocode}
%    \end{macro}
%
%    \begin{macro}{\HOLOGO@FontSetup}
%    \begin{macrocode}
\def\HOLOGO@FontSetup{%
  \kvsetkeys{HoLogoFont}%
}
%    \end{macrocode}
%    \end{macro}
%
%    \begin{macrocode}
\def\HOLOGO@temp#1{%
  \kv@define@key{HoLogoFont}{#1}{%
    \ifx\HOLOGO@name\relax
      \HoLogoFont@Def{#1}{##1}%
    \else
      \HoLogoFont@LogoDef\HOLOGO@name{#1}{##1}%
    \fi
  }%
}
\HOLOGO@temp{general}
\HOLOGO@temp{sf}
%    \end{macrocode}
%
% \subsection{Generic logo commands}
%
%    \begin{macrocode}
\HOLOGO@IfExists\hologo{%
  \@PackageError{hologo}{%
    \string\hologo\ltx@space is already defined.\MessageBreak
    Package loading is aborted%
  }\@ehc
  \HOLOGO@AtEnd
}%
\HOLOGO@IfExists\hologoRobust{%
  \@PackageError{hologo}{%
    \string\hologoRobust\ltx@space is already defined.\MessageBreak
    Package loading is aborted%
  }\@ehc
  \HOLOGO@AtEnd
}%
%    \end{macrocode}
%
% \subsubsection{\cs{hologo} and friends}
%
%    \begin{macrocode}
\ifluatex
  \expandafter\ltx@firstofone
\else
  \expandafter\ltx@gobble
\fi
{%
  \ltx@IfUndefined{ifincsname}{%
    \ifnum\luatexversion<36 %
      \expandafter\ltx@gobble
    \else
      \expandafter\ltx@firstofone
    \fi
    {%
      \begingroup
        \ifcase0%
            \directlua{%
              if tex.enableprimitives then %
                tex.enableprimitives('HOLOGO@', {'ifincsname'})%
              else %
                tex.print('1')%
              end%
            }%
            \ifx\HOLOGO@ifincsname\@undefined 1\fi%
            \relax
          \expandafter\ltx@firstofone
        \else
          \endgroup
          \expandafter\ltx@gobble
        \fi
        {%
          \global\let\ifincsname\HOLOGO@ifincsname
        }%
      \HOLOGO@temp
    }%
  }{}%
}
%    \end{macrocode}
%    \begin{macrocode}
\ltx@IfUndefined{ifincsname}{%
  \catcode`$=14 %
}{%
  \catcode`$=9 %
}
%    \end{macrocode}
%
%    \begin{macro}{\hologo}
%    \begin{macrocode}
\def\hologo#1{%
$ \ifincsname
$   \ltx@ifundefined{HoLogoCs@\HOLOGO@Variant{#1}}{%
$     #1%
$   }{%
$     \csname HoLogoCs@\HOLOGO@Variant{#1}\endcsname\ltx@firstoftwo
$   }%
$ \else
    \HOLOGO@IfExists\texorpdfstring\texorpdfstring\ltx@firstoftwo
    {%
      \hologoRobust{#1}%
    }{%
      \ltx@ifundefined{HoLogoBkm@\HOLOGO@Variant{#1}}{%
        \ltx@ifundefined{HoLogo@#1}{?#1?}{#1}%
      }{%
        \csname HoLogoBkm@\HOLOGO@Variant{#1}\endcsname
        \ltx@firstoftwo
      }%
    }%
$ \fi
}
%    \end{macrocode}
%    \end{macro}
%    \begin{macro}{\Hologo}
%    \begin{macrocode}
\def\Hologo#1{%
$ \ifincsname
$   \ltx@ifundefined{HoLogoCs@\HOLOGO@Variant{#1}}{%
$     #1%
$   }{%
$     \csname HoLogoCs@\HOLOGO@Variant{#1}\endcsname\ltx@secondoftwo
$   }%
$ \else
    \HOLOGO@IfExists\texorpdfstring\texorpdfstring\ltx@firstoftwo
    {%
      \HologoRobust{#1}%
    }{%
      \ltx@ifundefined{HoLogoBkm@\HOLOGO@Variant{#1}}{%
        \ltx@ifundefined{HoLogo@#1}{?#1?}{#1}%
      }{%
        \csname HoLogoBkm@\HOLOGO@Variant{#1}\endcsname
        \ltx@secondoftwo
      }%
    }%
$ \fi
}
%    \end{macrocode}
%    \end{macro}
%
%    \begin{macro}{\hologoVariant}
%    \begin{macrocode}
\def\hologoVariant#1#2{%
  \ifx\relax#2\relax
    \hologo{#1}%
  \else
$   \ifincsname
$     \ltx@ifundefined{HoLogoCs@#1@#2}{%
$       #1%
$     }{%
$       \csname HoLogoCs@#1@#2\endcsname\ltx@firstoftwo
$     }%
$   \else
      \HOLOGO@IfExists\texorpdfstring\texorpdfstring\ltx@firstoftwo
      {%
        \hologoVariantRobust{#1}{#2}%
      }{%
        \ltx@ifundefined{HoLogoBkm@#1@#2}{%
          \ltx@ifundefined{HoLogo@#1}{?#1?}{#1}%
        }{%
          \csname HoLogoBkm@#1@#2\endcsname
          \ltx@firstoftwo
        }%
      }%
$   \fi
  \fi
}
%    \end{macrocode}
%    \end{macro}
%    \begin{macro}{\HologoVariant}
%    \begin{macrocode}
\def\HologoVariant#1#2{%
  \ifx\relax#2\relax
    \Hologo{#1}%
  \else
$   \ifincsname
$     \ltx@ifundefined{HoLogoCs@#1@#2}{%
$       #1%
$     }{%
$       \csname HoLogoCs@#1@#2\endcsname\ltx@secondoftwo
$     }%
$   \else
      \HOLOGO@IfExists\texorpdfstring\texorpdfstring\ltx@firstoftwo
      {%
        \HologoVariantRobust{#1}{#2}%
      }{%
        \ltx@ifundefined{HoLogoBkm@#1@#2}{%
          \ltx@ifundefined{HoLogo@#1}{?#1?}{#1}%
        }{%
          \csname HoLogoBkm@#1@#2\endcsname
          \ltx@secondoftwo
        }%
      }%
$   \fi
  \fi
}
%    \end{macrocode}
%    \end{macro}
%
%    \begin{macrocode}
\catcode`\$=3 %
%    \end{macrocode}
%
% \subsubsection{\cs{hologoRobust} and friends}
%
%    \begin{macro}{\hologoRobust}
%    \begin{macrocode}
\ltx@IfUndefined{protected}{%
  \ltx@IfUndefined{DeclareRobustCommand}{%
    \def\hologoRobust#1%
  }{%
    \DeclareRobustCommand*\hologoRobust[1]%
  }%
}{%
  \protected\def\hologoRobust#1%
}%
{%
  \edef\HOLOGO@name{#1}%
  \ltx@IfUndefined{HoLogo@\HOLOGO@Variant\HOLOGO@name}{%
    \@PackageError{hologo}{%
      Unknown logo `\HOLOGO@name'%
    }\@ehc
    ?\HOLOGO@name?%
  }{%
    \ltx@IfUndefined{ver@tex4ht.sty}{%
      \HoLogoFont@font\HOLOGO@name{general}{%
        \csname HoLogo@\HOLOGO@Variant\HOLOGO@name\endcsname
        \ltx@firstoftwo
      }%
    }{%
      \ltx@IfUndefined{HoLogoHtml@\HOLOGO@Variant\HOLOGO@name}{%
        \HOLOGO@name
      }{%
        \csname HoLogoHtml@\HOLOGO@Variant\HOLOGO@name\endcsname
        \ltx@firstoftwo
      }%
    }%
  }%
}
%    \end{macrocode}
%    \end{macro}
%    \begin{macro}{\HologoRobust}
%    \begin{macrocode}
\ltx@IfUndefined{protected}{%
  \ltx@IfUndefined{DeclareRobustCommand}{%
    \def\HologoRobust#1%
  }{%
    \DeclareRobustCommand*\HologoRobust[1]%
  }%
}{%
  \protected\def\HologoRobust#1%
}%
{%
  \edef\HOLOGO@name{#1}%
  \ltx@IfUndefined{HoLogo@\HOLOGO@Variant\HOLOGO@name}{%
    \@PackageError{hologo}{%
      Unknown logo `\HOLOGO@name'%
    }\@ehc
    ?\HOLOGO@name?%
  }{%
    \ltx@IfUndefined{ver@tex4ht.sty}{%
      \HoLogoFont@font\HOLOGO@name{general}{%
        \csname HoLogo@\HOLOGO@Variant\HOLOGO@name\endcsname
        \ltx@secondoftwo
      }%
    }{%
      \ltx@IfUndefined{HoLogoHtml@\HOLOGO@Variant\HOLOGO@name}{%
        \expandafter\HOLOGO@Uppercase\HOLOGO@name
      }{%
        \csname HoLogoHtml@\HOLOGO@Variant\HOLOGO@name\endcsname
        \ltx@secondoftwo
      }%
    }%
  }%
}
%    \end{macrocode}
%    \end{macro}
%    \begin{macro}{\hologoVariantRobust}
%    \begin{macrocode}
\ltx@IfUndefined{protected}{%
  \ltx@IfUndefined{DeclareRobustCommand}{%
    \def\hologoVariantRobust#1#2%
  }{%
    \DeclareRobustCommand*\hologoVariantRobust[2]%
  }%
}{%
  \protected\def\hologoVariantRobust#1#2%
}%
{%
  \begingroup
    \hologoLogoSetup{#1}{variant={#2}}%
    \hologoRobust{#1}%
  \endgroup
}
%    \end{macrocode}
%    \end{macro}
%    \begin{macro}{\HologoVariantRobust}
%    \begin{macrocode}
\ltx@IfUndefined{protected}{%
  \ltx@IfUndefined{DeclareRobustCommand}{%
    \def\HologoVariantRobust#1#2%
  }{%
    \DeclareRobustCommand*\HologoVariantRobust[2]%
  }%
}{%
  \protected\def\HologoVariantRobust#1#2%
}%
{%
  \begingroup
    \hologoLogoSetup{#1}{variant={#2}}%
    \HologoRobust{#1}%
  \endgroup
}
%    \end{macrocode}
%    \end{macro}
%
%    \begin{macro}{\hologorobust}
%    Macro \cs{hologorobust} is only defined for compatibility.
%    Its use is deprecated.
%    \begin{macrocode}
\def\hologorobust{\hologoRobust}
%    \end{macrocode}
%    \end{macro}
%
% \subsection{Helpers}
%
%    \begin{macro}{\HOLOGO@Uppercase}
%    Macro \cs{HOLOGO@Uppercase} is restricted to \cs{uppercase},
%    because \hologo{plainTeX} or \hologo{iniTeX} do not provide
%    \cs{MakeUppercase}.
%    \begin{macrocode}
\def\HOLOGO@Uppercase#1{\uppercase{#1}}
%    \end{macrocode}
%    \end{macro}
%
%    \begin{macro}{\HOLOGO@PdfdocUnicode}
%    \begin{macrocode}
\def\HOLOGO@PdfdocUnicode{%
  \ifx\ifHy@unicode\iftrue
    \expandafter\ltx@secondoftwo
  \else
    \expandafter\ltx@firstoftwo
  \fi
}
%    \end{macrocode}
%    \end{macro}
%
%    \begin{macro}{\HOLOGO@Math}
%    \begin{macrocode}
\def\HOLOGO@MathSetup{%
  \mathsurround0pt\relax
  \HOLOGO@IfExists\f@series{%
    \if b\expandafter\ltx@car\f@series x\@nil
      \csname boldmath\endcsname
   \fi
  }{}%
}
%    \end{macrocode}
%    \end{macro}
%
%    \begin{macro}{\HOLOGO@TempDimen}
%    \begin{macrocode}
\dimendef\HOLOGO@TempDimen=\ltx@zero
%    \end{macrocode}
%    \end{macro}
%    \begin{macro}{\HOLOGO@NegativeKerning}
%    \begin{macrocode}
\def\HOLOGO@NegativeKerning#1{%
  \begingroup
    \HOLOGO@TempDimen=0pt\relax
    \comma@parse@normalized{#1}{%
      \ifdim\HOLOGO@TempDimen=0pt %
        \expandafter\HOLOGO@@NegativeKerning\comma@entry
      \fi
      \ltx@gobble
    }%
    \ifdim\HOLOGO@TempDimen<0pt %
      \kern\HOLOGO@TempDimen
    \fi
  \endgroup
}
%    \end{macrocode}
%    \end{macro}
%    \begin{macro}{\HOLOGO@@NegativeKerning}
%    \begin{macrocode}
\def\HOLOGO@@NegativeKerning#1#2{%
  \setbox\ltx@zero\hbox{#1#2}%
  \HOLOGO@TempDimen=\wd\ltx@zero
  \setbox\ltx@zero\hbox{#1\kern0pt#2}%
  \advance\HOLOGO@TempDimen by -\wd\ltx@zero
}
%    \end{macrocode}
%    \end{macro}
%
%    \begin{macro}{\HOLOGO@SpaceFactor}
%    \begin{macrocode}
\def\HOLOGO@SpaceFactor{%
  \spacefactor1000 %
}
%    \end{macrocode}
%    \end{macro}
%
%    \begin{macro}{\HOLOGO@Span}
%    \begin{macrocode}
\def\HOLOGO@Span#1#2{%
  \HCode{<span class="HoLogo-#1">}%
  #2%
  \HCode{</span>}%
}
%    \end{macrocode}
%    \end{macro}
%
% \subsubsection{Text subscript}
%
%    \begin{macro}{\HOLOGO@SubScript}%
%    \begin{macrocode}
\def\HOLOGO@SubScript#1{%
  \ltx@IfUndefined{textsubscript}{%
    \ltx@IfUndefined{text}{%
      \ltx@mbox{%
        \mathsurround=0pt\relax
        $%
          _{%
            \ltx@IfUndefined{sf@size}{%
              \mathrm{#1}%
            }{%
              \mbox{%
                \fontsize\sf@size{0pt}\selectfont
                #1%
              }%
            }%
          }%
        $%
      }%
    }{%
      \ltx@mbox{%
        \mathsurround=0pt\relax
        $_{\text{#1}}$%
      }%
    }%
  }{%
    \textsubscript{#1}%
  }%
}
%    \end{macrocode}
%    \end{macro}
%
% \subsection{\hologo{TeX} and friends}
%
% \subsubsection{\hologo{TeX}}
%
%    \begin{macro}{\HoLogo@TeX}
%    Source: \hologo{LaTeX} kernel.
%    \begin{macrocode}
\def\HoLogo@TeX#1{%
  T\kern-.1667em\lower.5ex\hbox{E}\kern-.125emX\HOLOGO@SpaceFactor
}
%    \end{macrocode}
%    \end{macro}
%    \begin{macro}{\HoLogoHtml@TeX}
%    \begin{macrocode}
\def\HoLogoHtml@TeX#1{%
  \HoLogoCss@TeX
  \HOLOGO@Span{TeX}{%
    T%
    \HOLOGO@Span{e}{%
      E%
    }%
    X%
  }%
}
%    \end{macrocode}
%    \end{macro}
%    \begin{macro}{\HoLogoCss@TeX}
%    \begin{macrocode}
\def\HoLogoCss@TeX{%
  \Css{%
    span.HoLogo-TeX span.HoLogo-e{%
      position:relative;%
      top:.5ex;%
      margin-left:-.1667em;%
      margin-right:-.125em;%
    }%
  }%
  \Css{%
    a span.HoLogo-TeX span.HoLogo-e{%
      text-decoration:none;%
    }%
  }%
  \global\let\HoLogoCss@TeX\relax
}
%    \end{macrocode}
%    \end{macro}
%
% \subsubsection{\hologo{plainTeX}}
%
%    \begin{macro}{\HoLogo@plainTeX@space}
%    Source: ``The \hologo{TeX}book''
%    \begin{macrocode}
\def\HoLogo@plainTeX@space#1{%
  \HOLOGO@mbox{#1{p}{P}lain}\HOLOGO@space\hologo{TeX}%
}
%    \end{macrocode}
%    \end{macro}
%    \begin{macro}{\HoLogoCs@plainTeX@space}
%    \begin{macrocode}
\def\HoLogoCs@plainTeX@space#1{#1{p}{P}lain TeX}%
%    \end{macrocode}
%    \end{macro}
%    \begin{macro}{\HoLogoBkm@plainTeX@space}
%    \begin{macrocode}
\def\HoLogoBkm@plainTeX@space#1{%
  #1{p}{P}lain \hologo{TeX}%
}
%    \end{macrocode}
%    \end{macro}
%    \begin{macro}{\HoLogoHtml@plainTeX@space}
%    \begin{macrocode}
\def\HoLogoHtml@plainTeX@space#1{%
  #1{p}{P}lain \hologo{TeX}%
}
%    \end{macrocode}
%    \end{macro}
%
%    \begin{macro}{\HoLogo@plainTeX@hyphen}
%    \begin{macrocode}
\def\HoLogo@plainTeX@hyphen#1{%
  \HOLOGO@mbox{#1{p}{P}lain}\HOLOGO@hyphen\hologo{TeX}%
}
%    \end{macrocode}
%    \end{macro}
%    \begin{macro}{\HoLogoCs@plainTeX@hyphen}
%    \begin{macrocode}
\def\HoLogoCs@plainTeX@hyphen#1{#1{p}{P}lain-TeX}
%    \end{macrocode}
%    \end{macro}
%    \begin{macro}{\HoLogoBkm@plainTeX@hyphen}
%    \begin{macrocode}
\def\HoLogoBkm@plainTeX@hyphen#1{%
  #1{p}{P}lain-\hologo{TeX}%
}
%    \end{macrocode}
%    \end{macro}
%    \begin{macro}{\HoLogoHtml@plainTeX@hyphen}
%    \begin{macrocode}
\def\HoLogoHtml@plainTeX@hyphen#1{%
  #1{p}{P}lain-\hologo{TeX}%
}
%    \end{macrocode}
%    \end{macro}
%
%    \begin{macro}{\HoLogo@plainTeX@runtogether}
%    \begin{macrocode}
\def\HoLogo@plainTeX@runtogether#1{%
  \HOLOGO@mbox{#1{p}{P}lain\hologo{TeX}}%
}
%    \end{macrocode}
%    \end{macro}
%    \begin{macro}{\HoLogoCs@plainTeX@runtogether}
%    \begin{macrocode}
\def\HoLogoCs@plainTeX@runtogether#1{#1{p}{P}lainTeX}
%    \end{macrocode}
%    \end{macro}
%    \begin{macro}{\HoLogoBkm@plainTeX@runtogether}
%    \begin{macrocode}
\def\HoLogoBkm@plainTeX@runtogether#1{%
  #1{p}{P}lain\hologo{TeX}%
}
%    \end{macrocode}
%    \end{macro}
%    \begin{macro}{\HoLogoHtml@plainTeX@runtogether}
%    \begin{macrocode}
\def\HoLogoHtml@plainTeX@runtogether#1{%
  #1{p}{P}lain\hologo{TeX}%
}
%    \end{macrocode}
%    \end{macro}
%
%    \begin{macro}{\HoLogo@plainTeX}
%    \begin{macrocode}
\def\HoLogo@plainTeX{\HoLogo@plainTeX@space}
%    \end{macrocode}
%    \end{macro}
%    \begin{macro}{\HoLogoCs@plainTeX}
%    \begin{macrocode}
\def\HoLogoCs@plainTeX{\HoLogoCs@plainTeX@space}
%    \end{macrocode}
%    \end{macro}
%    \begin{macro}{\HoLogoBkm@plainTeX}
%    \begin{macrocode}
\def\HoLogoBkm@plainTeX{\HoLogoBkm@plainTeX@space}
%    \end{macrocode}
%    \end{macro}
%    \begin{macro}{\HoLogoHtml@plainTeX}
%    \begin{macrocode}
\def\HoLogoHtml@plainTeX{\HoLogoHtml@plainTeX@space}
%    \end{macrocode}
%    \end{macro}
%
% \subsubsection{\hologo{LaTeX}}
%
%    Source: \hologo{LaTeX} kernel.
%\begin{quote}
%\begin{verbatim}
%\DeclareRobustCommand{\LaTeX}{%
%  L%
%  \kern-.36em%
%  {%
%    \sbox\z@ T%
%    \vbox to\ht\z@{%
%      \hbox{%
%        \check@mathfonts
%        \fontsize\sf@size\z@
%        \math@fontsfalse
%        \selectfont
%        A%
%      }%
%      \vss
%    }%
%  }%
%  \kern-.15em%
%  \TeX
%}
%\end{verbatim}
%\end{quote}
%
%    \begin{macro}{\HoLogo@La}
%    \begin{macrocode}
\def\HoLogo@La#1{%
  L%
  \kern-.36em%
  \begingroup
    \setbox\ltx@zero\hbox{T}%
    \vbox to\ht\ltx@zero{%
      \hbox{%
        \ltx@ifundefined{check@mathfonts}{%
          \csname sevenrm\endcsname
        }{%
          \check@mathfonts
          \fontsize\sf@size{0pt}%
          \math@fontsfalse\selectfont
        }%
        A%
      }%
      \vss
    }%
  \endgroup
}
%    \end{macrocode}
%    \end{macro}
%
%    \begin{macro}{\HoLogo@LaTeX}
%    Source: \hologo{LaTeX} kernel.
%    \begin{macrocode}
\def\HoLogo@LaTeX#1{%
  \hologo{La}%
  \kern-.15em%
  \hologo{TeX}%
}
%    \end{macrocode}
%    \end{macro}
%    \begin{macro}{\HoLogoHtml@LaTeX}
%    \begin{macrocode}
\def\HoLogoHtml@LaTeX#1{%
  \HoLogoCss@LaTeX
  \HOLOGO@Span{LaTeX}{%
    L%
    \HOLOGO@Span{a}{%
      A%
    }%
    \hologo{TeX}%
  }%
}
%    \end{macrocode}
%    \end{macro}
%    \begin{macro}{\HoLogoCss@LaTeX}
%    \begin{macrocode}
\def\HoLogoCss@LaTeX{%
  \Css{%
    span.HoLogo-LaTeX span.HoLogo-a{%
      position:relative;%
      top:-.5ex;%
      margin-left:-.36em;%
      margin-right:-.15em;%
      font-size:85\%;%
    }%
  }%
  \global\let\HoLogoCss@LaTeX\relax
}
%    \end{macrocode}
%    \end{macro}
%
% \subsubsection{\hologo{(La)TeX}}
%
%    \begin{macro}{\HoLogo@LaTeXTeX}
%    The kerning around the parentheses is taken
%    from package \xpackage{dtklogos} \cite{dtklogos}.
%\begin{quote}
%\begin{verbatim}
%\DeclareRobustCommand{\LaTeXTeX}{%
%  (%
%  \kern-.15em%
%  L%
%  \kern-.36em%
%  {%
%    \sbox\z@ T%
%    \vbox to\ht0{%
%      \hbox{%
%        $\m@th$%
%        \csname S@\f@size\endcsname
%        \fontsize\sf@size\z@
%        \math@fontsfalse
%        \selectfont
%        A%
%      }%
%      \vss
%    }%
%  }%
%  \kern-.2em%
%  )%
%  \kern-.15em%
%  \TeX
%}
%\end{verbatim}
%\end{quote}
%    \begin{macrocode}
\def\HoLogo@LaTeXTeX#1{%
  (%
  \kern-.15em%
  \hologo{La}%
  \kern-.2em%
  )%
  \kern-.15em%
  \hologo{TeX}%
}
%    \end{macrocode}
%    \end{macro}
%    \begin{macro}{\HoLogoBkm@LaTeXTeX}
%    \begin{macrocode}
\def\HoLogoBkm@LaTeXTeX#1{(La)TeX}
%    \end{macrocode}
%    \end{macro}
%
%    \begin{macro}{\HoLogo@(La)TeX}
%    \begin{macrocode}
\expandafter
\let\csname HoLogo@(La)TeX\endcsname\HoLogo@LaTeXTeX
%    \end{macrocode}
%    \end{macro}
%    \begin{macro}{\HoLogoBkm@(La)TeX}
%    \begin{macrocode}
\expandafter
\let\csname HoLogoBkm@(La)TeX\endcsname\HoLogoBkm@LaTeXTeX
%    \end{macrocode}
%    \end{macro}
%    \begin{macro}{\HoLogoHtml@LaTeXTeX}
%    \begin{macrocode}
\def\HoLogoHtml@LaTeXTeX#1{%
  \HoLogoCss@LaTeXTeX
  \HOLOGO@Span{LaTeXTeX}{%
    (%
    \HOLOGO@Span{L}{L}%
    \HOLOGO@Span{a}{A}%
    \HOLOGO@Span{ParenRight}{)}%
    \hologo{TeX}%
  }%
}
%    \end{macrocode}
%    \end{macro}
%    \begin{macro}{\HoLogoHtml@(La)TeX}
%    Kerning after opening parentheses and before closing parentheses
%    is $-0.1$\,em. The original values $-0.15$\,em
%    looked too ugly for a serif font.
%    \begin{macrocode}
\expandafter
\let\csname HoLogoHtml@(La)TeX\endcsname\HoLogoHtml@LaTeXTeX
%    \end{macrocode}
%    \end{macro}
%    \begin{macro}{\HoLogoCss@LaTeXTeX}
%    \begin{macrocode}
\def\HoLogoCss@LaTeXTeX{%
  \Css{%
    span.HoLogo-LaTeXTeX span.HoLogo-L{%
      margin-left:-.1em;%
    }%
  }%
  \Css{%
    span.HoLogo-LaTeXTeX span.HoLogo-a{%
      position:relative;%
      top:-.5ex;%
      margin-left:-.36em;%
      margin-right:-.1em;%
      font-size:85\%;%
    }%
  }%
  \Css{%
    span.HoLogo-LaTeXTeX span.HoLogo-ParenRight{%
      margin-right:-.15em;%
    }%
  }%
  \global\let\HoLogoCss@LaTeXTeX\relax
}
%    \end{macrocode}
%    \end{macro}
%
% \subsubsection{\hologo{LaTeXe}}
%
%    \begin{macro}{\HoLogo@LaTeXe}
%    Source: \hologo{LaTeX} kernel
%    \begin{macrocode}
\def\HoLogo@LaTeXe#1{%
  \hologo{LaTeX}%
  \kern.15em%
  \hbox{%
    \HOLOGO@MathSetup
    2%
    $_{\textstyle\varepsilon}$%
  }%
}
%    \end{macrocode}
%    \end{macro}
%
%    \begin{macro}{\HoLogoCs@LaTeXe}
%    \begin{macrocode}
\ifnum64=`\^^^^0040\relax % test for big chars of LuaTeX/XeTeX
  \catcode`\$=9 %
  \catcode`\&=14 %
\else
  \catcode`\$=14 %
  \catcode`\&=9 %
\fi
\def\HoLogoCs@LaTeXe#1{%
  LaTeX2%
$ \string ^^^^0395%
& e%
}%
\catcode`\$=3 %
\catcode`\&=4 %
%    \end{macrocode}
%    \end{macro}
%
%    \begin{macro}{\HoLogoBkm@LaTeXe}
%    \begin{macrocode}
\def\HoLogoBkm@LaTeXe#1{%
  \hologo{LaTeX}%
  2%
  \HOLOGO@PdfdocUnicode{e}{\textepsilon}%
}
%    \end{macrocode}
%    \end{macro}
%
%    \begin{macro}{\HoLogoHtml@LaTeXe}
%    \begin{macrocode}
\def\HoLogoHtml@LaTeXe#1{%
  \HoLogoCss@LaTeXe
  \HOLOGO@Span{LaTeX2e}{%
    \hologo{LaTeX}%
    \HOLOGO@Span{2}{2}%
    \HOLOGO@Span{e}{%
      \HOLOGO@MathSetup
      \ensuremath{\textstyle\varepsilon}%
    }%
  }%
}
%    \end{macrocode}
%    \end{macro}
%    \begin{macro}{\HoLogoCss@LaTeXe}
%    \begin{macrocode}
\def\HoLogoCss@LaTeXe{%
  \Css{%
    span.HoLogo-LaTeX2e span.HoLogo-2{%
      padding-left:.15em;%
    }%
  }%
  \Css{%
    span.HoLogo-LaTeX2e span.HoLogo-e{%
      position:relative;%
      top:.35ex;%
      text-decoration:none;%
    }%
  }%
  \global\let\HoLogoCss@LaTeXe\relax
}
%    \end{macrocode}
%    \end{macro}
%
%    \begin{macro}{\HoLogo@LaTeX2e}
%    \begin{macrocode}
\expandafter
\let\csname HoLogo@LaTeX2e\endcsname\HoLogo@LaTeXe
%    \end{macrocode}
%    \end{macro}
%    \begin{macro}{\HoLogoCs@LaTeX2e}
%    \begin{macrocode}
\expandafter
\let\csname HoLogoCs@LaTeX2e\endcsname\HoLogoCs@LaTeXe
%    \end{macrocode}
%    \end{macro}
%    \begin{macro}{\HoLogoBkm@LaTeX2e}
%    \begin{macrocode}
\expandafter
\let\csname HoLogoBkm@LaTeX2e\endcsname\HoLogoBkm@LaTeXe
%    \end{macrocode}
%    \end{macro}
%    \begin{macro}{\HoLogoHtml@LaTeX2e}
%    \begin{macrocode}
\expandafter
\let\csname HoLogoHtml@LaTeX2e\endcsname\HoLogoHtml@LaTeXe
%    \end{macrocode}
%    \end{macro}
%
% \subsubsection{\hologo{LaTeX3}}
%
%    \begin{macro}{\HoLogo@LaTeX3}
%    Source: \hologo{LaTeX} kernel
%    \begin{macrocode}
\expandafter\def\csname HoLogo@LaTeX3\endcsname#1{%
  \hologo{LaTeX}%
  3%
}
%    \end{macrocode}
%    \end{macro}
%
%    \begin{macro}{\HoLogoBkm@LaTeX3}
%    \begin{macrocode}
\expandafter\def\csname HoLogoBkm@LaTeX3\endcsname#1{%
  \hologo{LaTeX}%
  3%
}
%    \end{macrocode}
%    \end{macro}
%    \begin{macro}{\HoLogoHtml@LaTeX3}
%    \begin{macrocode}
\expandafter
\let\csname HoLogoHtml@LaTeX3\expandafter\endcsname
\csname HoLogo@LaTeX3\endcsname
%    \end{macrocode}
%    \end{macro}
%
% \subsubsection{\hologo{LaTeXML}}
%
%    \begin{macro}{\HoLogo@LaTeXML}
%    \begin{macrocode}
\def\HoLogo@LaTeXML#1{%
  \HOLOGO@mbox{%
    \hologo{La}%
    \kern-.15em%
    T%
    \kern-.1667em%
    \lower.5ex\hbox{E}%
    \kern-.125em%
    \HoLogoFont@font{LaTeXML}{sc}{xml}%
  }%
}
%    \end{macrocode}
%    \end{macro}
%    \begin{macro}{\HoLogoHtml@pdfLaTeX}
%    \begin{macrocode}
\def\HoLogoHtml@LaTeXML#1{%
  \HOLOGO@Span{LaTeXML}{%
    \HoLogoCss@LaTeX
    \HoLogoCss@TeX
    \HOLOGO@Span{LaTeX}{%
      L%
      \HOLOGO@Span{a}{%
        A%
      }%
    }%
    \HOLOGO@Span{TeX}{%
      T%
      \HOLOGO@Span{e}{%
        E%
      }%
    }%
    \HCode{<span style="font-variant: small-caps;">}%
    xml%
    \HCode{</span>}%
  }%
}
%    \end{macrocode}
%    \end{macro}
%
% \subsubsection{\hologo{eTeX}}
%
%    \begin{macro}{\HoLogo@eTeX}
%    Source: package \xpackage{etex}
%    \begin{macrocode}
\def\HoLogo@eTeX#1{%
  \ltx@mbox{%
    \HOLOGO@MathSetup
    $\varepsilon$%
    -%
    \HOLOGO@NegativeKerning{-T,T-,To}%
    \hologo{TeX}%
  }%
}
%    \end{macrocode}
%    \end{macro}
%    \begin{macro}{\HoLogoCs@eTeX}
%    \begin{macrocode}
\ifnum64=`\^^^^0040\relax % test for big chars of LuaTeX/XeTeX
  \catcode`\$=9 %
  \catcode`\&=14 %
\else
  \catcode`\$=14 %
  \catcode`\&=9 %
\fi
\def\HoLogoCs@eTeX#1{%
$ #1{\string ^^^^0395}{\string ^^^^03b5}%
& #1{e}{E}%
  TeX%
}%
\catcode`\$=3 %
\catcode`\&=4 %
%    \end{macrocode}
%    \end{macro}
%    \begin{macro}{\HoLogoBkm@eTeX}
%    \begin{macrocode}
\def\HoLogoBkm@eTeX#1{%
  \HOLOGO@PdfdocUnicode{#1{e}{E}}{\textepsilon}%
  -%
  \hologo{TeX}%
}
%    \end{macrocode}
%    \end{macro}
%    \begin{macro}{\HoLogoHtml@eTeX}
%    \begin{macrocode}
\def\HoLogoHtml@eTeX#1{%
  \ltx@mbox{%
    \HOLOGO@MathSetup
    $\varepsilon$%
    -%
    \hologo{TeX}%
  }%
}
%    \end{macrocode}
%    \end{macro}
%
% \subsubsection{\hologo{iniTeX}}
%
%    \begin{macro}{\HoLogo@iniTeX}
%    \begin{macrocode}
\def\HoLogo@iniTeX#1{%
  \HOLOGO@mbox{%
    #1{i}{I}ni\hologo{TeX}%
  }%
}
%    \end{macrocode}
%    \end{macro}
%    \begin{macro}{\HoLogoCs@iniTeX}
%    \begin{macrocode}
\def\HoLogoCs@iniTeX#1{#1{i}{I}niTeX}
%    \end{macrocode}
%    \end{macro}
%    \begin{macro}{\HoLogoBkm@iniTeX}
%    \begin{macrocode}
\def\HoLogoBkm@iniTeX#1{%
  #1{i}{I}ni\hologo{TeX}%
}
%    \end{macrocode}
%    \end{macro}
%    \begin{macro}{\HoLogoHtml@iniTeX}
%    \begin{macrocode}
\let\HoLogoHtml@iniTeX\HoLogo@iniTeX
%    \end{macrocode}
%    \end{macro}
%
% \subsubsection{\hologo{virTeX}}
%
%    \begin{macro}{\HoLogo@virTeX}
%    \begin{macrocode}
\def\HoLogo@virTeX#1{%
  \HOLOGO@mbox{%
    #1{v}{V}ir\hologo{TeX}%
  }%
}
%    \end{macrocode}
%    \end{macro}
%    \begin{macro}{\HoLogoCs@virTeX}
%    \begin{macrocode}
\def\HoLogoCs@virTeX#1{#1{v}{V}irTeX}
%    \end{macrocode}
%    \end{macro}
%    \begin{macro}{\HoLogoBkm@virTeX}
%    \begin{macrocode}
\def\HoLogoBkm@virTeX#1{%
  #1{v}{V}ir\hologo{TeX}%
}
%    \end{macrocode}
%    \end{macro}
%    \begin{macro}{\HoLogoHtml@virTeX}
%    \begin{macrocode}
\let\HoLogoHtml@virTeX\HoLogo@virTeX
%    \end{macrocode}
%    \end{macro}
%
% \subsubsection{\hologo{SliTeX}}
%
% \paragraph{Definitions of the three variants.}
%
%    \begin{macro}{\HoLogo@SLiTeX@lift}
%    \begin{macrocode}
\def\HoLogo@SLiTeX@lift#1{%
  \HoLogoFont@font{SliTeX}{rm}{%
    S%
    \kern-.06em%
    L%
    \kern-.18em%
    \raise.32ex\hbox{\HoLogoFont@font{SliTeX}{sc}{i}}%
    \HOLOGO@discretionary
    \kern-.06em%
    \hologo{TeX}%
  }%
}
%    \end{macrocode}
%    \end{macro}
%    \begin{macro}{\HoLogoBkm@SLiTeX@lift}
%    \begin{macrocode}
\def\HoLogoBkm@SLiTeX@lift#1{SLiTeX}
%    \end{macrocode}
%    \end{macro}
%    \begin{macro}{\HoLogoHtml@SLiTeX@lift}
%    \begin{macrocode}
\def\HoLogoHtml@SLiTeX@lift#1{%
  \HoLogoCss@SLiTeX@lift
  \HOLOGO@Span{SLiTeX-lift}{%
    \HoLogoFont@font{SliTeX}{rm}{%
      S%
      \HOLOGO@Span{L}{L}%
      \HOLOGO@Span{i}{i}%
      \hologo{TeX}%
    }%
  }%
}
%    \end{macrocode}
%    \end{macro}
%    \begin{macro}{\HoLogoCss@SLiTeX@lift}
%    \begin{macrocode}
\def\HoLogoCss@SLiTeX@lift{%
  \Css{%
    span.HoLogo-SLiTeX-lift span.HoLogo-L{%
      margin-left:-.06em;%
      margin-right:-.18em;%
    }%
  }%
  \Css{%
    span.HoLogo-SLiTeX-lift span.HoLogo-i{%
      position:relative;%
      top:-.32ex;%
      margin-right:-.06em;%
      font-variant:small-caps;%
    }%
  }%
  \global\let\HoLogoCss@SLiTeX@lift\relax
}
%    \end{macrocode}
%    \end{macro}
%
%    \begin{macro}{\HoLogo@SliTeX@simple}
%    \begin{macrocode}
\def\HoLogo@SliTeX@simple#1{%
  \HoLogoFont@font{SliTeX}{rm}{%
    \ltx@mbox{%
      \HoLogoFont@font{SliTeX}{sc}{Sli}%
    }%
    \HOLOGO@discretionary
    \hologo{TeX}%
  }%
}
%    \end{macrocode}
%    \end{macro}
%    \begin{macro}{\HoLogoBkm@SliTeX@simple}
%    \begin{macrocode}
\def\HoLogoBkm@SliTeX@simple#1{SliTeX}
%    \end{macrocode}
%    \end{macro}
%    \begin{macro}{\HoLogoHtml@SliTeX@simple}
%    \begin{macrocode}
\let\HoLogoHtml@SliTeX@simple\HoLogo@SliTeX@simple
%    \end{macrocode}
%    \end{macro}
%
%    \begin{macro}{\HoLogo@SliTeX@narrow}
%    \begin{macrocode}
\def\HoLogo@SliTeX@narrow#1{%
  \HoLogoFont@font{SliTeX}{rm}{%
    \ltx@mbox{%
      S%
      \kern-.06em%
      \HoLogoFont@font{SliTeX}{sc}{%
        l%
        \kern-.035em%
        i%
      }%
    }%
    \HOLOGO@discretionary
    \kern-.06em%
    \hologo{TeX}%
  }%
}
%    \end{macrocode}
%    \end{macro}
%    \begin{macro}{\HoLogoBkm@SliTeX@narrow}
%    \begin{macrocode}
\def\HoLogoBkm@SliTeX@narrow#1{SliTeX}
%    \end{macrocode}
%    \end{macro}
%    \begin{macro}{\HoLogoHtml@SliTeX@narrow}
%    \begin{macrocode}
\def\HoLogoHtml@SliTeX@narrow#1{%
  \HoLogoCss@SliTeX@narrow
  \HOLOGO@Span{SliTeX-narrow}{%
    \HoLogoFont@font{SliTeX}{rm}{%
      S%
        \HOLOGO@Span{l}{l}%
        \HOLOGO@Span{i}{i}%
      \hologo{TeX}%
    }%
  }%
}
%    \end{macrocode}
%    \end{macro}
%    \begin{macro}{\HoLogoCss@SliTeX@narrow}
%    \begin{macrocode}
\def\HoLogoCss@SliTeX@narrow{%
  \Css{%
    span.HoLogo-SliTeX-narrow span.HoLogo-l{%
      margin-left:-.06em;%
      margin-right:-.035em;%
      font-variant:small-caps;%
    }%
  }%
  \Css{%
    span.HoLogo-SliTeX-narrow span.HoLogo-i{%
      margin-right:-.06em;%
      font-variant:small-caps;%
    }%
  }%
  \global\let\HoLogoCss@SliTeX@narrow\relax
}
%    \end{macrocode}
%    \end{macro}
%
% \paragraph{Macro set completion.}
%
%    \begin{macro}{\HoLogo@SLiTeX@simple}
%    \begin{macrocode}
\def\HoLogo@SLiTeX@simple{\HoLogo@SliTeX@simple}
%    \end{macrocode}
%    \end{macro}
%    \begin{macro}{\HoLogoBkm@SLiTeX@simple}
%    \begin{macrocode}
\def\HoLogoBkm@SLiTeX@simple{\HoLogoBkm@SliTeX@simple}
%    \end{macrocode}
%    \end{macro}
%    \begin{macro}{\HoLogoHtml@SLiTeX@simple}
%    \begin{macrocode}
\def\HoLogoHtml@SLiTeX@simple{\HoLogoHtml@SliTeX@simple}
%    \end{macrocode}
%    \end{macro}
%
%    \begin{macro}{\HoLogo@SLiTeX@narrow}
%    \begin{macrocode}
\def\HoLogo@SLiTeX@narrow{\HoLogo@SliTeX@narrow}
%    \end{macrocode}
%    \end{macro}
%    \begin{macro}{\HoLogoBkm@SLiTeX@narrow}
%    \begin{macrocode}
\def\HoLogoBkm@SLiTeX@narrow{\HoLogoBkm@SliTeX@narrow}
%    \end{macrocode}
%    \end{macro}
%    \begin{macro}{\HoLogoHtml@SLiTeX@narrow}
%    \begin{macrocode}
\def\HoLogoHtml@SLiTeX@narrow{\HoLogoHtml@SliTeX@narrow}
%    \end{macrocode}
%    \end{macro}
%
%    \begin{macro}{\HoLogo@SliTeX@lift}
%    \begin{macrocode}
\def\HoLogo@SliTeX@lift{\HoLogo@SLiTeX@lift}
%    \end{macrocode}
%    \end{macro}
%    \begin{macro}{\HoLogoBkm@SliTeX@lift}
%    \begin{macrocode}
\def\HoLogoBkm@SliTeX@lift{\HoLogoBkm@SLiTeX@lift}
%    \end{macrocode}
%    \end{macro}
%    \begin{macro}{\HoLogoHtml@SliTeX@lift}
%    \begin{macrocode}
\def\HoLogoHtml@SliTeX@lift{\HoLogoHtml@SLiTeX@lift}
%    \end{macrocode}
%    \end{macro}
%
% \paragraph{Defaults.}
%
%    \begin{macro}{\HoLogo@SLiTeX}
%    \begin{macrocode}
\def\HoLogo@SLiTeX{\HoLogo@SLiTeX@lift}
%    \end{macrocode}
%    \end{macro}
%    \begin{macro}{\HoLogoBkm@SLiTeX}
%    \begin{macrocode}
\def\HoLogoBkm@SLiTeX{\HoLogoBkm@SLiTeX@lift}
%    \end{macrocode}
%    \end{macro}
%    \begin{macro}{\HoLogoHtml@SLiTeX}
%    \begin{macrocode}
\def\HoLogoHtml@SLiTeX{\HoLogoHtml@SLiTeX@lift}
%    \end{macrocode}
%    \end{macro}
%
%    \begin{macro}{\HoLogo@SliTeX}
%    \begin{macrocode}
\def\HoLogo@SliTeX{\HoLogo@SliTeX@narrow}
%    \end{macrocode}
%    \end{macro}
%    \begin{macro}{\HoLogoBkm@SliTeX}
%    \begin{macrocode}
\def\HoLogoBkm@SliTeX{\HoLogoBkm@SliTeX@narrow}
%    \end{macrocode}
%    \end{macro}
%    \begin{macro}{\HoLogoHtml@SliTeX}
%    \begin{macrocode}
\def\HoLogoHtml@SliTeX{\HoLogoHtml@SliTeX@narrow}
%    \end{macrocode}
%    \end{macro}
%
% \subsubsection{\hologo{LuaTeX}}
%
%    \begin{macro}{\HoLogo@LuaTeX}
%    The kerning is an idea of Hans Hagen, see mailing list
%    `luatex at tug dot org' in March 2010.
%    \begin{macrocode}
\def\HoLogo@LuaTeX#1{%
  \HOLOGO@mbox{%
    Lua%
    \HOLOGO@NegativeKerning{aT,oT,To}%
    \hologo{TeX}%
  }%
}
%    \end{macrocode}
%    \end{macro}
%    \begin{macro}{\HoLogoHtml@LuaTeX}
%    \begin{macrocode}
\let\HoLogoHtml@LuaTeX\HoLogo@LuaTeX
%    \end{macrocode}
%    \end{macro}
%
% \subsubsection{\hologo{LuaLaTeX}}
%
%    \begin{macro}{\HoLogo@LuaLaTeX}
%    \begin{macrocode}
\def\HoLogo@LuaLaTeX#1{%
  \HOLOGO@mbox{%
    Lua%
    \hologo{LaTeX}%
  }%
}
%    \end{macrocode}
%    \end{macro}
%    \begin{macro}{\HoLogoHtml@LuaLaTeX}
%    \begin{macrocode}
\let\HoLogoHtml@LuaLaTeX\HoLogo@LuaLaTeX
%    \end{macrocode}
%    \end{macro}
%
% \subsubsection{\hologo{XeTeX}, \hologo{XeLaTeX}}
%
%    \begin{macro}{\HOLOGO@IfCharExists}
%    \begin{macrocode}
\ifluatex
  \ifnum\luatexversion<36 %
  \else
    \def\HOLOGO@IfCharExists#1{%
      \ifnum
        \directlua{%
           if luaotfload and luaotfload.aux then
             if luaotfload.aux.font_has_glyph(%
                    font.current(), \number#1) then % 	 
	       tex.print("1") % 	 
	     end % 	 
	   elseif font and font.fonts and font.current then %
            local f = font.fonts[font.current()]%
            if f.characters and f.characters[\number#1] then %
              tex.print("1")%
            end %
          end%
        }0=\ltx@zero
        \expandafter\ltx@secondoftwo
      \else
        \expandafter\ltx@firstoftwo
      \fi
    }%
  \fi
\fi
\ltx@IfUndefined{HOLOGO@IfCharExists}{%
  \def\HOLOGO@@IfCharExists#1{%
    \begingroup
      \tracinglostchars=\ltx@zero
      \setbox\ltx@zero=\hbox{%
        \kern7sp\char#1\relax
        \ifnum\lastkern>\ltx@zero
          \expandafter\aftergroup\csname iffalse\endcsname
        \else
          \expandafter\aftergroup\csname iftrue\endcsname
        \fi
      }%
      % \if{true|false} from \aftergroup
      \endgroup
      \expandafter\ltx@firstoftwo
    \else
      \endgroup
      \expandafter\ltx@secondoftwo
    \fi
  }%
  \ifxetex
    \ltx@IfUndefined{XeTeXfonttype}{}{%
      \ltx@IfUndefined{XeTeXcharglyph}{}{%
        \def\HOLOGO@IfCharExists#1{%
          \ifnum\XeTeXfonttype\font>\ltx@zero
            \expandafter\ltx@firstofthree
          \else
            \expandafter\ltx@gobble
          \fi
          {%
            \ifnum\XeTeXcharglyph#1>\ltx@zero
              \expandafter\ltx@firstoftwo
            \else
              \expandafter\ltx@secondoftwo
            \fi
          }%
          \HOLOGO@@IfCharExists{#1}%
        }%
      }%
    }%
  \fi
}{}
\ltx@ifundefined{HOLOGO@IfCharExists}{%
  \ifnum64=`\^^^^0040\relax % test for big chars of LuaTeX/XeTeX
    \let\HOLOGO@IfCharExists\HOLOGO@@IfCharExists
  \else
    \def\HOLOGO@IfCharExists#1{%
      \ifnum#1>255 %
        \expandafter\ltx@fourthoffour
      \fi
      \HOLOGO@@IfCharExists{#1}%
    }%
  \fi
}{}
%    \end{macrocode}
%    \end{macro}
%
%    \begin{macro}{\HoLogo@Xe}
%    Source: package \xpackage{dtklogos}
%    \begin{macrocode}
\def\HoLogo@Xe#1{%
  X%
  \kern-.1em\relax
  \HOLOGO@IfCharExists{"018E}{%
    \lower.5ex\hbox{\char"018E}%
  }{%
    \chardef\HOLOGO@choice=\ltx@zero
    \ifdim\fontdimen\ltx@one\font>0pt %
      \ltx@IfUndefined{rotatebox}{%
        \ltx@IfUndefined{pgftext}{%
          \ltx@IfUndefined{psscalebox}{%
            \ltx@IfUndefined{HOLOGO@ScaleBox@\hologoDriver}{%
            }{%
              \chardef\HOLOGO@choice=4 %
            }%
          }{%
            \chardef\HOLOGO@choice=3 %
          }%
        }{%
          \chardef\HOLOGO@choice=2 %
        }%
      }{%
        \chardef\HOLOGO@choice=1 %
      }%
      \ifcase\HOLOGO@choice
        \HOLOGO@WarningUnsupportedDriver{Xe}%
        e%
      \or % 1: \rotatebox
        \begingroup
          \setbox\ltx@zero\hbox{\rotatebox{180}{E}}%
          \ltx@LocDimenA=\dp\ltx@zero
          \advance\ltx@LocDimenA by -.5ex\relax
          \raise\ltx@LocDimenA\box\ltx@zero
        \endgroup
      \or % 2: \pgftext
        \lower.5ex\hbox{%
          \pgfpicture
            \pgftext[rotate=180]{E}%
          \endpgfpicture
        }%
      \or % 3: \psscalebox
        \begingroup
          \setbox\ltx@zero\hbox{\psscalebox{-1 -1}{E}}%
          \ltx@LocDimenA=\dp\ltx@zero
          \advance\ltx@LocDimenA by -.5ex\relax
          \raise\ltx@LocDimenA\box\ltx@zero
        \endgroup
      \or % 4: \HOLOGO@PointReflectBox
        \lower.5ex\hbox{\HOLOGO@PointReflectBox{E}}%
      \else
        \@PackageError{hologo}{Internal error (choice/it}\@ehc
      \fi
    \else
      \ltx@IfUndefined{reflectbox}{%
        \ltx@IfUndefined{pgftext}{%
          \ltx@IfUndefined{psscalebox}{%
            \ltx@IfUndefined{HOLOGO@ScaleBox@\hologoDriver}{%
            }{%
              \chardef\HOLOGO@choice=4 %
            }%
          }{%
            \chardef\HOLOGO@choice=3 %
          }%
        }{%
          \chardef\HOLOGO@choice=2 %
        }%
      }{%
        \chardef\HOLOGO@choice=1 %
      }%
      \ifcase\HOLOGO@choice
        \HOLOGO@WarningUnsupportedDriver{Xe}%
        e%
      \or % 1: reflectbox
        \lower.5ex\hbox{%
          \reflectbox{E}%
        }%
      \or % 2: \pgftext
        \lower.5ex\hbox{%
          \pgfpicture
            \pgftransformxscale{-1}%
            \pgftext{E}%
          \endpgfpicture
        }%
      \or % 3: \psscalebox
        \lower.5ex\hbox{%
          \psscalebox{-1 1}{E}%
        }%
      \or % 4: \HOLOGO@Reflectbox
        \lower.5ex\hbox{%
          \HOLOGO@ReflectBox{E}%
        }%
      \else
        \@PackageError{hologo}{Internal error (choice/up)}\@ehc
      \fi
    \fi
  }%
}
%    \end{macrocode}
%    \end{macro}
%    \begin{macro}{\HoLogoHtml@Xe}
%    \begin{macrocode}
\def\HoLogoHtml@Xe#1{%
  \HoLogoCss@Xe
  \HOLOGO@Span{Xe}{%
    X%
    \HOLOGO@Span{e}{%
      \HCode{&\ltx@hashchar x018e;}%
    }%
  }%
}
%    \end{macrocode}
%    \end{macro}
%    \begin{macro}{\HoLogoCss@Xe}
%    \begin{macrocode}
\def\HoLogoCss@Xe{%
  \Css{%
    span.HoLogo-Xe span.HoLogo-e{%
      position:relative;%
      top:.5ex;%
      left-margin:-.1em;%
    }%
  }%
  \global\let\HoLogoCss@Xe\relax
}
%    \end{macrocode}
%    \end{macro}
%
%    \begin{macro}{\HoLogo@XeTeX}
%    \begin{macrocode}
\def\HoLogo@XeTeX#1{%
  \hologo{Xe}%
  \kern-.15em\relax
  \hologo{TeX}%
}
%    \end{macrocode}
%    \end{macro}
%
%    \begin{macro}{\HoLogoHtml@XeTeX}
%    \begin{macrocode}
\def\HoLogoHtml@XeTeX#1{%
  \HoLogoCss@XeTeX
  \HOLOGO@Span{XeTeX}{%
    \hologo{Xe}%
    \hologo{TeX}%
  }%
}
%    \end{macrocode}
%    \end{macro}
%    \begin{macro}{\HoLogoCss@XeTeX}
%    \begin{macrocode}
\def\HoLogoCss@XeTeX{%
  \Css{%
    span.HoLogo-XeTeX span.HoLogo-TeX{%
      margin-left:-.15em;%
    }%
  }%
  \global\let\HoLogoCss@XeTeX\relax
}
%    \end{macrocode}
%    \end{macro}
%
%    \begin{macro}{\HoLogo@XeLaTeX}
%    \begin{macrocode}
\def\HoLogo@XeLaTeX#1{%
  \hologo{Xe}%
  \kern-.13em%
  \hologo{LaTeX}%
}
%    \end{macrocode}
%    \end{macro}
%    \begin{macro}{\HoLogoHtml@XeLaTeX}
%    \begin{macrocode}
\def\HoLogoHtml@XeLaTeX#1{%
  \HoLogoCss@XeLaTeX
  \HOLOGO@Span{XeLaTeX}{%
    \hologo{Xe}%
    \hologo{LaTeX}%
  }%
}
%    \end{macrocode}
%    \end{macro}
%    \begin{macro}{\HoLogoCss@XeLaTeX}
%    \begin{macrocode}
\def\HoLogoCss@XeLaTeX{%
  \Css{%
    span.HoLogo-XeLaTeX span.HoLogo-Xe{%
      margin-right:-.13em;%
    }%
  }%
  \global\let\HoLogoCss@XeLaTeX\relax
}
%    \end{macrocode}
%    \end{macro}
%
% \subsubsection{\hologo{pdfTeX}, \hologo{pdfLaTeX}}
%
%    \begin{macro}{\HoLogo@pdfTeX}
%    \begin{macrocode}
\def\HoLogo@pdfTeX#1{%
  \HOLOGO@mbox{%
    #1{p}{P}df\hologo{TeX}%
  }%
}
%    \end{macrocode}
%    \end{macro}
%    \begin{macro}{\HoLogoCs@pdfTeX}
%    \begin{macrocode}
\def\HoLogoCs@pdfTeX#1{#1{p}{P}dfTeX}
%    \end{macrocode}
%    \end{macro}
%    \begin{macro}{\HoLogoBkm@pdfTeX}
%    \begin{macrocode}
\def\HoLogoBkm@pdfTeX#1{%
  #1{p}{P}df\hologo{TeX}%
}
%    \end{macrocode}
%    \end{macro}
%    \begin{macro}{\HoLogoHtml@pdfTeX}
%    \begin{macrocode}
\let\HoLogoHtml@pdfTeX\HoLogo@pdfTeX
%    \end{macrocode}
%    \end{macro}
%
%    \begin{macro}{\HoLogo@pdfLaTeX}
%    \begin{macrocode}
\def\HoLogo@pdfLaTeX#1{%
  \HOLOGO@mbox{%
    #1{p}{P}df\hologo{LaTeX}%
  }%
}
%    \end{macrocode}
%    \end{macro}
%    \begin{macro}{\HoLogoCs@pdfLaTeX}
%    \begin{macrocode}
\def\HoLogoCs@pdfLaTeX#1{#1{p}{P}dfLaTeX}
%    \end{macrocode}
%    \end{macro}
%    \begin{macro}{\HoLogoBkm@pdfLaTeX}
%    \begin{macrocode}
\def\HoLogoBkm@pdfLaTeX#1{%
  #1{p}{P}df\hologo{LaTeX}%
}
%    \end{macrocode}
%    \end{macro}
%    \begin{macro}{\HoLogoHtml@pdfLaTeX}
%    \begin{macrocode}
\let\HoLogoHtml@pdfLaTeX\HoLogo@pdfLaTeX
%    \end{macrocode}
%    \end{macro}
%
% \subsubsection{\hologo{VTeX}}
%
%    \begin{macro}{\HoLogo@VTeX}
%    \begin{macrocode}
\def\HoLogo@VTeX#1{%
  \HOLOGO@mbox{%
    V\hologo{TeX}%
  }%
}
%    \end{macrocode}
%    \end{macro}
%    \begin{macro}{\HoLogoHtml@VTeX}
%    \begin{macrocode}
\let\HoLogoHtml@VTeX\HoLogo@VTeX
%    \end{macrocode}
%    \end{macro}
%
% \subsubsection{\hologo{AmS}, \dots}
%
%    Source: class \xclass{amsdtx}
%
%    \begin{macro}{\HoLogo@AmS}
%    \begin{macrocode}
\def\HoLogo@AmS#1{%
  \HoLogoFont@font{AmS}{sy}{%
    A%
    \kern-.1667em%
    \lower.5ex\hbox{M}%
    \kern-.125em%
    S%
  }%
}
%    \end{macrocode}
%    \end{macro}
%    \begin{macro}{\HoLogoBkm@AmS}
%    \begin{macrocode}
\def\HoLogoBkm@AmS#1{AmS}
%    \end{macrocode}
%    \end{macro}
%    \begin{macro}{\HoLogoHtml@AmS}
%    \begin{macrocode}
\def\HoLogoHtml@AmS#1{%
  \HoLogoCss@AmS
%  \HoLogoFont@font{AmS}{sy}{%
    \HOLOGO@Span{AmS}{%
      A%
      \HOLOGO@Span{M}{M}%
      S%
    }%
%   }%
}
%    \end{macrocode}
%    \end{macro}
%    \begin{macro}{\HoLogoCss@AmS}
%    \begin{macrocode}
\def\HoLogoCss@AmS{%
  \Css{%
    span.HoLogo-AmS span.HoLogo-M{%
      position:relative;%
      top:.5ex;%
      margin-left:-.1667em;%
      margin-right:-.125em;%
      text-decoration:none;%
    }%
  }%
  \global\let\HoLogoCss@AmS\relax
}
%    \end{macrocode}
%    \end{macro}
%
%    \begin{macro}{\HoLogo@AmSTeX}
%    \begin{macrocode}
\def\HoLogo@AmSTeX#1{%
  \hologo{AmS}%
  \HOLOGO@hyphen
  \hologo{TeX}%
}
%    \end{macrocode}
%    \end{macro}
%    \begin{macro}{\HoLogoBkm@AmSTeX}
%    \begin{macrocode}
\def\HoLogoBkm@AmSTeX#1{AmS-TeX}%
%    \end{macrocode}
%    \end{macro}
%    \begin{macro}{\HoLogoHtml@AmSTeX}
%    \begin{macrocode}
\let\HoLogoHtml@AmSTeX\HoLogo@AmSTeX
%    \end{macrocode}
%    \end{macro}
%
%    \begin{macro}{\HoLogo@AmSLaTeX}
%    \begin{macrocode}
\def\HoLogo@AmSLaTeX#1{%
  \hologo{AmS}%
  \HOLOGO@hyphen
  \hologo{LaTeX}%
}
%    \end{macrocode}
%    \end{macro}
%    \begin{macro}{\HoLogoBkm@AmSLaTeX}
%    \begin{macrocode}
\def\HoLogoBkm@AmSLaTeX#1{AmS-LaTeX}%
%    \end{macrocode}
%    \end{macro}
%    \begin{macro}{\HoLogoHtml@AmSLaTeX}
%    \begin{macrocode}
\let\HoLogoHtml@AmSLaTeX\HoLogo@AmSLaTeX
%    \end{macrocode}
%    \end{macro}
%
% \subsubsection{\hologo{BibTeX}}
%
%    \begin{macro}{\HoLogo@BibTeX@sc}
%    A definition of \hologo{BibTeX} is provided in
%    the documentation source for the manual of \hologo{BibTeX}
%    \cite{btxdoc}.
%\begin{quote}
%\begin{verbatim}
%\def\BibTeX{%
%  {%
%    \rm
%    B%
%    \kern-.05em%
%    {%
%      \sc
%      i%
%      \kern-.025em %
%      b%
%    }%
%    \kern-.08em
%    T%
%    \kern-.1667em%
%    \lower.7ex\hbox{E}%
%    \kern-.125em%
%    X%
%  }%
%}
%\end{verbatim}
%\end{quote}
%    \begin{macrocode}
\def\HoLogo@BibTeX@sc#1{%
  B%
  \kern-.05em%
  \HoLogoFont@font{BibTeX}{sc}{%
    i%
    \kern-.025em%
    b%
  }%
  \HOLOGO@discretionary
  \kern-.08em%
  \hologo{TeX}%
}
%    \end{macrocode}
%    \end{macro}
%    \begin{macro}{\HoLogoHtml@BibTeX@sc}
%    \begin{macrocode}
\def\HoLogoHtml@BibTeX@sc#1{%
  \HoLogoCss@BibTeX@sc
  \HOLOGO@Span{BibTeX-sc}{%
    B%
    \HOLOGO@Span{i}{i}%
    \HOLOGO@Span{b}{b}%
    \hologo{TeX}%
  }%
}
%    \end{macrocode}
%    \end{macro}
%    \begin{macro}{\HoLogoCss@BibTeX@sc}
%    \begin{macrocode}
\def\HoLogoCss@BibTeX@sc{%
  \Css{%
    span.HoLogo-BibTeX-sc span.HoLogo-i{%
      margin-left:-.05em;%
      margin-right:-.025em;%
      font-variant:small-caps;%
    }%
  }%
  \Css{%
    span.HoLogo-BibTeX-sc span.HoLogo-b{%
      margin-right:-.08em;%
      font-variant:small-caps;%
    }%
  }%
  \global\let\HoLogoCss@BibTeX@sc\relax
}
%    \end{macrocode}
%    \end{macro}
%
%    \begin{macro}{\HoLogo@BibTeX@sf}
%    Variant \xoption{sf} avoids trouble with unavailable
%    small caps fonts (e.g., bold versions of Computer Modern or
%    Latin Modern). The definition is taken from
%    package \xpackage{dtklogos} \cite{dtklogos}.
%\begin{quote}
%\begin{verbatim}
%\DeclareRobustCommand{\BibTeX}{%
%  B%
%  \kern-.05em%
%  \hbox{%
%    $\m@th$% %% force math size calculations
%    \csname S@\f@size\endcsname
%    \fontsize\sf@size\z@
%    \math@fontsfalse
%    \selectfont
%    I%
%    \kern-.025em%
%    B
%  }%
%  \kern-.08em%
%  \-%
%  \TeX
%}
%\end{verbatim}
%\end{quote}
%    \begin{macrocode}
\def\HoLogo@BibTeX@sf#1{%
  B%
  \kern-.05em%
  \HoLogoFont@font{BibTeX}{bibsf}{%
    I%
    \kern-.025em%
    B%
  }%
  \HOLOGO@discretionary
  \kern-.08em%
  \hologo{TeX}%
}
%    \end{macrocode}
%    \end{macro}
%    \begin{macro}{\HoLogoHtml@BibTeX@sf}
%    \begin{macrocode}
\def\HoLogoHtml@BibTeX@sf#1{%
  \HoLogoCss@BibTeX@sf
  \HOLOGO@Span{BibTeX-sf}{%
    B%
    \HoLogoFont@font{BibTeX}{bibsf}{%
      \HOLOGO@Span{i}{I}%
      B%
    }%
    \hologo{TeX}%
  }%
}
%    \end{macrocode}
%    \end{macro}
%    \begin{macro}{\HoLogoCss@BibTeX@sf}
%    \begin{macrocode}
\def\HoLogoCss@BibTeX@sf{%
  \Css{%
    span.HoLogo-BibTeX-sf span.HoLogo-i{%
      margin-left:-.05em;%
      margin-right:-.025em;%
    }%
  }%
  \Css{%
    span.HoLogo-BibTeX-sf span.HoLogo-TeX{%
      margin-left:-.08em;%
    }%
  }%
  \global\let\HoLogoCss@BibTeX@sf\relax
}
%    \end{macrocode}
%    \end{macro}
%
%    \begin{macro}{\HoLogo@BibTeX}
%    \begin{macrocode}
\def\HoLogo@BibTeX{\HoLogo@BibTeX@sf}
%    \end{macrocode}
%    \end{macro}
%    \begin{macro}{\HoLogoHtml@BibTeX}
%    \begin{macrocode}
\def\HoLogoHtml@BibTeX{\HoLogoHtml@BibTeX@sf}
%    \end{macrocode}
%    \end{macro}
%
% \subsubsection{\hologo{BibTeX8}}
%
%    \begin{macro}{\HoLogo@BibTeX8}
%    \begin{macrocode}
\expandafter\def\csname HoLogo@BibTeX8\endcsname#1{%
  \hologo{BibTeX}%
  8%
}
%    \end{macrocode}
%    \end{macro}
%
%    \begin{macro}{\HoLogoBkm@BibTeX8}
%    \begin{macrocode}
\expandafter\def\csname HoLogoBkm@BibTeX8\endcsname#1{%
  \hologo{BibTeX}%
  8%
}
%    \end{macrocode}
%    \end{macro}
%    \begin{macro}{\HoLogoHtml@BibTeX8}
%    \begin{macrocode}
\expandafter
\let\csname HoLogoHtml@BibTeX8\expandafter\endcsname
\csname HoLogo@BibTeX8\endcsname
%    \end{macrocode}
%    \end{macro}
%
% \subsubsection{\hologo{ConTeXt}}
%
%    \begin{macro}{\HoLogo@ConTeXt@simple}
%    \begin{macrocode}
\def\HoLogo@ConTeXt@simple#1{%
  \HOLOGO@mbox{Con}%
  \HOLOGO@discretionary
  \HOLOGO@mbox{\hologo{TeX}t}%
}
%    \end{macrocode}
%    \end{macro}
%    \begin{macro}{\HoLogoHtml@ConTeXt@simple}
%    \begin{macrocode}
\let\HoLogoHtml@ConTeXt@simple\HoLogo@ConTeXt@simple
%    \end{macrocode}
%    \end{macro}
%
%    \begin{macro}{\HoLogo@ConTeXt@narrow}
%    This definition of logo \hologo{ConTeXt} with variant \xoption{narrow}
%    comes from TUGboat's class \xclass{ltugboat} (version 2010/11/15 v2.8).
%    \begin{macrocode}
\def\HoLogo@ConTeXt@narrow#1{%
  \HOLOGO@mbox{C\kern-.0333emon}%
  \HOLOGO@discretionary
  \kern-.0667em%
  \HOLOGO@mbox{\hologo{TeX}\kern-.0333emt}%
}
%    \end{macrocode}
%    \end{macro}
%    \begin{macro}{\HoLogoHtml@ConTeXt@narrow}
%    \begin{macrocode}
\def\HoLogoHtml@ConTeXt@narrow#1{%
  \HoLogoCss@ConTeXt@narrow
  \HOLOGO@Span{ConTeXt-narrow}{%
    \HOLOGO@Span{C}{C}%
    on%
    \hologo{TeX}%
    t%
  }%
}
%    \end{macrocode}
%    \end{macro}
%    \begin{macro}{\HoLogoCss@ConTeXt@narrow}
%    \begin{macrocode}
\def\HoLogoCss@ConTeXt@narrow{%
  \Css{%
    span.HoLogo-ConTeXt-narrow span.HoLogo-C{%
      margin-left:-.0333em;%
    }%
  }%
  \Css{%
    span.HoLogo-ConTeXt-narrow span.HoLogo-TeX{%
      margin-left:-.0667em;%
      margin-right:-.0333em;%
    }%
  }%
  \global\let\HoLogoCss@ConTeXt@narrow\relax
}
%    \end{macrocode}
%    \end{macro}
%
%    \begin{macro}{\HoLogo@ConTeXt}
%    \begin{macrocode}
\def\HoLogo@ConTeXt{\HoLogo@ConTeXt@narrow}
%    \end{macrocode}
%    \end{macro}
%    \begin{macro}{\HoLogoHtml@ConTeXt}
%    \begin{macrocode}
\def\HoLogoHtml@ConTeXt{\HoLogoHtml@ConTeXt@narrow}
%    \end{macrocode}
%    \end{macro}
%
% \subsubsection{\hologo{emTeX}}
%
%    \begin{macro}{\HoLogo@emTeX}
%    \begin{macrocode}
\def\HoLogo@emTeX#1{%
  \HOLOGO@mbox{#1{e}{E}m}%
  \HOLOGO@discretionary
  \hologo{TeX}%
}
%    \end{macrocode}
%    \end{macro}
%    \begin{macro}{\HoLogoCs@emTeX}
%    \begin{macrocode}
\def\HoLogoCs@emTeX#1{#1{e}{E}mTeX}%
%    \end{macrocode}
%    \end{macro}
%    \begin{macro}{\HoLogoBkm@emTeX}
%    \begin{macrocode}
\def\HoLogoBkm@emTeX#1{%
  #1{e}{E}m\hologo{TeX}%
}
%    \end{macrocode}
%    \end{macro}
%    \begin{macro}{\HoLogoHtml@emTeX}
%    \begin{macrocode}
\let\HoLogoHtml@emTeX\HoLogo@emTeX
%    \end{macrocode}
%    \end{macro}
%
% \subsubsection{\hologo{ExTeX}}
%
%    \begin{macro}{\HoLogo@ExTeX}
%    The definition is taken from the FAQ of the
%    project \hologo{ExTeX}
%    \cite{ExTeX-FAQ}.
%\begin{quote}
%\begin{verbatim}
%\def\ExTeX{%
%  \textrm{% Logo always with serifs
%    \ensuremath{%
%      \textstyle
%      \varepsilon_{%
%        \kern-0.15em%
%        \mathcal{X}%
%      }%
%    }%
%    \kern-.15em%
%    \TeX
%  }%
%}
%\end{verbatim}
%\end{quote}
%    \begin{macrocode}
\def\HoLogo@ExTeX#1{%
  \HoLogoFont@font{ExTeX}{rm}{%
    \ltx@mbox{%
      \HOLOGO@MathSetup
      $%
        \textstyle
        \varepsilon_{%
          \kern-0.15em%
          \HoLogoFont@font{ExTeX}{sy}{X}%
        }%
      $%
    }%
    \HOLOGO@discretionary
    \kern-.15em%
    \hologo{TeX}%
  }%
}
%    \end{macrocode}
%    \end{macro}
%    \begin{macro}{\HoLogoHtml@ExTeX}
%    \begin{macrocode}
\def\HoLogoHtml@ExTeX#1{%
  \HoLogoCss@ExTeX
  \HoLogoFont@font{ExTeX}{rm}{%
    \HOLOGO@Span{ExTeX}{%
      \ltx@mbox{%
        \HOLOGO@MathSetup
        $\textstyle\varepsilon$%
        \HOLOGO@Span{X}{$\textstyle\chi$}%
        \hologo{TeX}%
      }%
    }%
  }%
}
%    \end{macrocode}
%    \end{macro}
%    \begin{macro}{\HoLogoBkm@ExTeX}
%    \begin{macrocode}
\def\HoLogoBkm@ExTeX#1{%
  \HOLOGO@PdfdocUnicode{#1{e}{E}x}{\textepsilon\textchi}%
  \hologo{TeX}%
}
%    \end{macrocode}
%    \end{macro}
%    \begin{macro}{\HoLogoCss@ExTeX}
%    \begin{macrocode}
\def\HoLogoCss@ExTeX{%
  \Css{%
    span.HoLogo-ExTeX{%
      font-family:serif;%
    }%
  }%
  \Css{%
    span.HoLogo-ExTeX span.HoLogo-TeX{%
      margin-left:-.15em;%
    }%
  }%
  \global\let\HoLogoCss@ExTeX\relax
}
%    \end{macrocode}
%    \end{macro}
%
% \subsubsection{\hologo{MiKTeX}}
%
%    \begin{macro}{\HoLogo@MiKTeX}
%    \begin{macrocode}
\def\HoLogo@MiKTeX#1{%
  \HOLOGO@mbox{MiK}%
  \HOLOGO@discretionary
  \hologo{TeX}%
}
%    \end{macrocode}
%    \end{macro}
%    \begin{macro}{\HoLogoHtml@MiKTeX}
%    \begin{macrocode}
\let\HoLogoHtml@MiKTeX\HoLogo@MiKTeX
%    \end{macrocode}
%    \end{macro}
%
% \subsubsection{\hologo{OzTeX} and friends}
%
%    Source: \hologo{OzTeX} FAQ \cite{OzTeX}:
%    \begin{quote}
%      |\def\OzTeX{O\kern-.03em z\kern-.15em\TeX}|\\
%      (There is no kerning in OzMF, OzMP and OzTtH.)
%    \end{quote}
%
%    \begin{macro}{\HoLogo@OzTeX}
%    \begin{macrocode}
\def\HoLogo@OzTeX#1{%
  O%
  \kern-.03em %
  z%
  \kern-.15em %
  \hologo{TeX}%
}
%    \end{macrocode}
%    \end{macro}
%    \begin{macro}{\HoLogoHtml@OzTeX}
%    \begin{macrocode}
\def\HoLogoHtml@OzTeX#1{%
  \HoLogoCss@OzTeX
  \HOLOGO@Span{OzTeX}{%
    O%
    \HOLOGO@Span{z}{z}%
    \hologo{TeX}%
  }%
}
%    \end{macrocode}
%    \end{macro}
%    \begin{macro}{\HoLogoCss@OzTeX}
%    \begin{macrocode}
\def\HoLogoCss@OzTeX{%
  \Css{%
    span.HoLogo-OzTeX span.HoLogo-z{%
      margin-left:-.03em;%
      margin-right:-.15em;%
    }%
  }%
  \global\let\HoLogoCss@OzTeX\relax
}
%    \end{macrocode}
%    \end{macro}
%
%    \begin{macro}{\HoLogo@OzMF}
%    \begin{macrocode}
\def\HoLogo@OzMF#1{%
  \HOLOGO@mbox{OzMF}%
}
%    \end{macrocode}
%    \end{macro}
%    \begin{macro}{\HoLogo@OzMP}
%    \begin{macrocode}
\def\HoLogo@OzMP#1{%
  \HOLOGO@mbox{OzMP}%
}
%    \end{macrocode}
%    \end{macro}
%    \begin{macro}{\HoLogo@OzTtH}
%    \begin{macrocode}
\def\HoLogo@OzTtH#1{%
  \HOLOGO@mbox{OzTtH}%
}
%    \end{macrocode}
%    \end{macro}
%
% \subsubsection{\hologo{PCTeX}}
%
%    \begin{macro}{\HoLogo@PCTeX}
%    \begin{macrocode}
\def\HoLogo@PCTeX#1{%
  \HOLOGO@mbox{PC}%
  \hologo{TeX}%
}
%    \end{macrocode}
%    \end{macro}
%    \begin{macro}{\HoLogoHtml@PCTeX}
%    \begin{macrocode}
\let\HoLogoHtml@PCTeX\HoLogo@PCTeX
%    \end{macrocode}
%    \end{macro}
%
% \subsubsection{\hologo{PiCTeX}}
%
%    The original definitions from \xfile{pictex.tex} \cite{PiCTeX}:
%\begin{quote}
%\begin{verbatim}
%\def\PiC{%
%  P%
%  \kern-.12em%
%  \lower.5ex\hbox{I}%
%  \kern-.075em%
%  C%
%}
%\def\PiCTeX{%
%  \PiC
%  \kern-.11em%
%  \TeX
%}
%\end{verbatim}
%\end{quote}
%
%    \begin{macro}{\HoLogo@PiC}
%    \begin{macrocode}
\def\HoLogo@PiC#1{%
  P%
  \kern-.12em%
  \lower.5ex\hbox{I}%
  \kern-.075em%
  C%
  \HOLOGO@SpaceFactor
}
%    \end{macrocode}
%    \end{macro}
%    \begin{macro}{\HoLogoHtml@PiC}
%    \begin{macrocode}
\def\HoLogoHtml@PiC#1{%
  \HoLogoCss@PiC
  \HOLOGO@Span{PiC}{%
    P%
    \HOLOGO@Span{i}{I}%
    C%
  }%
}
%    \end{macrocode}
%    \end{macro}
%    \begin{macro}{\HoLogoCss@PiC}
%    \begin{macrocode}
\def\HoLogoCss@PiC{%
  \Css{%
    span.HoLogo-PiC span.HoLogo-i{%
      position:relative;%
      top:.5ex;%
      margin-left:-.12em;%
      margin-right:-.075em;%
      text-decoration:none;%
    }%
  }%
  \global\let\HoLogoCss@PiC\relax
}
%    \end{macrocode}
%    \end{macro}
%
%    \begin{macro}{\HoLogo@PiCTeX}
%    \begin{macrocode}
\def\HoLogo@PiCTeX#1{%
  \hologo{PiC}%
  \HOLOGO@discretionary
  \kern-.11em%
  \hologo{TeX}%
}
%    \end{macrocode}
%    \end{macro}
%    \begin{macro}{\HoLogoHtml@PiCTeX}
%    \begin{macrocode}
\def\HoLogoHtml@PiCTeX#1{%
  \HoLogoCss@PiCTeX
  \HOLOGO@Span{PiCTeX}{%
    \hologo{PiC}%
    \hologo{TeX}%
  }%
}
%    \end{macrocode}
%    \end{macro}
%    \begin{macro}{\HoLogoCss@PiCTeX}
%    \begin{macrocode}
\def\HoLogoCss@PiCTeX{%
  \Css{%
    span.HoLogo-PiCTeX span.HoLogo-PiC{%
      margin-right:-.11em;%
    }%
  }%
  \global\let\HoLogoCss@PiCTeX\relax
}
%    \end{macrocode}
%    \end{macro}
%
% \subsubsection{\hologo{teTeX}}
%
%    \begin{macro}{\HoLogo@teTeX}
%    \begin{macrocode}
\def\HoLogo@teTeX#1{%
  \HOLOGO@mbox{#1{t}{T}e}%
  \HOLOGO@discretionary
  \hologo{TeX}%
}
%    \end{macrocode}
%    \end{macro}
%    \begin{macro}{\HoLogoCs@teTeX}
%    \begin{macrocode}
\def\HoLogoCs@teTeX#1{#1{t}{T}dfTeX}
%    \end{macrocode}
%    \end{macro}
%    \begin{macro}{\HoLogoBkm@teTeX}
%    \begin{macrocode}
\def\HoLogoBkm@teTeX#1{%
  #1{t}{T}e\hologo{TeX}%
}
%    \end{macrocode}
%    \end{macro}
%    \begin{macro}{\HoLogoHtml@teTeX}
%    \begin{macrocode}
\let\HoLogoHtml@teTeX\HoLogo@teTeX
%    \end{macrocode}
%    \end{macro}
%
% \subsubsection{\hologo{TeX4ht}}
%
%    \begin{macro}{\HoLogo@TeX4ht}
%    \begin{macrocode}
\expandafter\def\csname HoLogo@TeX4ht\endcsname#1{%
  \HOLOGO@mbox{\hologo{TeX}4ht}%
}
%    \end{macrocode}
%    \end{macro}
%    \begin{macro}{\HoLogoHtml@TeX4ht}
%    \begin{macrocode}
\expandafter
\let\csname HoLogoHtml@TeX4ht\expandafter\endcsname
\csname HoLogo@TeX4ht\endcsname
%    \end{macrocode}
%    \end{macro}
%
%
% \subsubsection{\hologo{SageTeX}}
%
%    \begin{macro}{\HoLogo@SageTeX}
%    \begin{macrocode}
\def\HoLogo@SageTeX#1{%
  \HOLOGO@mbox{Sage}%
  \HOLOGO@discretionary
  \HOLOGO@NegativeKerning{eT,oT,To}%
  \hologo{TeX}%
}
%    \end{macrocode}
%    \end{macro}
%    \begin{macro}{\HoLogoHtml@SageTeX}
%    \begin{macrocode}
\let\HoLogoHtml@SageTeX\HoLogo@SageTeX
%    \end{macrocode}
%    \end{macro}
%
% \subsection{\hologo{METAFONT} and friends}
%
%    \begin{macro}{\HoLogo@METAFONT}
%    \begin{macrocode}
\def\HoLogo@METAFONT#1{%
  \HoLogoFont@font{METAFONT}{logo}{%
    \HOLOGO@mbox{META}%
    \HOLOGO@discretionary
    \HOLOGO@mbox{FONT}%
  }%
}
%    \end{macrocode}
%    \end{macro}
%
%    \begin{macro}{\HoLogo@METAPOST}
%    \begin{macrocode}
\def\HoLogo@METAPOST#1{%
  \HoLogoFont@font{METAPOST}{logo}{%
    \HOLOGO@mbox{META}%
    \HOLOGO@discretionary
    \HOLOGO@mbox{POST}%
  }%
}
%    \end{macrocode}
%    \end{macro}
%
%    \begin{macro}{\HoLogo@MetaFun}
%    \begin{macrocode}
\def\HoLogo@MetaFun#1{%
  \HOLOGO@mbox{Meta}%
  \HOLOGO@discretionary
  \HOLOGO@mbox{Fun}%
}
%    \end{macrocode}
%    \end{macro}
%
%    \begin{macro}{\HoLogo@MetaPost}
%    \begin{macrocode}
\def\HoLogo@MetaPost#1{%
  \HOLOGO@mbox{Meta}%
  \HOLOGO@discretionary
  \HOLOGO@mbox{Post}%
}
%    \end{macrocode}
%    \end{macro}
%
% \subsection{Others}
%
% \subsubsection{\hologo{biber}}
%
%    \begin{macro}{\HoLogo@biber}
%    \begin{macrocode}
\def\HoLogo@biber#1{%
  \HOLOGO@mbox{#1{b}{B}i}%
  \HOLOGO@discretionary
  \HOLOGO@mbox{ber}%
}
%    \end{macrocode}
%    \end{macro}
%    \begin{macro}{\HoLogoCs@biber}
%    \begin{macrocode}
\def\HoLogoCs@biber#1{#1{b}{B}iber}
%    \end{macrocode}
%    \end{macro}
%    \begin{macro}{\HoLogoBkm@biber}
%    \begin{macrocode}
\def\HoLogoBkm@biber#1{%
  #1{b}{B}iber%
}
%    \end{macrocode}
%    \end{macro}
%    \begin{macro}{\HoLogoHtml@biber}
%    \begin{macrocode}
\let\HoLogoHtml@biber\HoLogo@biber
%    \end{macrocode}
%    \end{macro}
%
% \subsubsection{\hologo{KOMAScript}}
%
%    \begin{macro}{\HoLogo@KOMAScript}
%    The definition for \hologo{KOMAScript} is taken
%    from \hologo{KOMAScript} (\xfile{scrlogo.dtx}, reformatted) \cite{scrlogo}:
%\begin{quote}
%\begin{verbatim}
%\@ifundefined{KOMAScript}{%
%  \DeclareRobustCommand{\KOMAScript}{%
%    \textsf{%
%      K\kern.05em O\kern.05emM\kern.05em A%
%      \kern.1em-\kern.1em %
%      Script%
%    }%
%  }%
%}{}
%\end{verbatim}
%\end{quote}
%    \begin{macrocode}
\def\HoLogo@KOMAScript#1{%
  \HoLogoFont@font{KOMAScript}{sf}{%
    \HOLOGO@mbox{%
      K\kern.05em%
      O\kern.05em%
      M\kern.05em%
      A%
    }%
    \kern.1em%
    \HOLOGO@hyphen
    \kern.1em%
    \HOLOGO@mbox{Script}%
  }%
}
%    \end{macrocode}
%    \end{macro}
%    \begin{macro}{\HoLogoBkm@KOMAScript}
%    \begin{macrocode}
\def\HoLogoBkm@KOMAScript#1{%
  KOMA-Script%
}
%    \end{macrocode}
%    \end{macro}
%    \begin{macro}{\HoLogoHtml@KOMAScript}
%    \begin{macrocode}
\def\HoLogoHtml@KOMAScript#1{%
  \HoLogoCss@KOMAScript
  \HoLogoFont@font{KOMAScript}{sf}{%
    \HOLOGO@Span{KOMAScript}{%
      K%
      \HOLOGO@Span{O}{O}%
      M%
      \HOLOGO@Span{A}{A}%
      \HOLOGO@Span{hyphen}{-}%
      Script%
    }%
  }%
}
%    \end{macrocode}
%    \end{macro}
%    \begin{macro}{\HoLogoCss@KOMAScript}
%    \begin{macrocode}
\def\HoLogoCss@KOMAScript{%
  \Css{%
    span.HoLogo-KOMAScript{%
      font-family:sans-serif;%
    }%
  }%
  \Css{%
    span.HoLogo-KOMAScript span.HoLogo-O{%
      padding-left:.05em;%
      padding-right:.05em;%
    }%
  }%
  \Css{%
    span.HoLogo-KOMAScript span.HoLogo-A{%
      padding-left:.05em;%
    }%
  }%
  \Css{%
    span.HoLogo-KOMAScript span.HoLogo-hyphen{%
      padding-left:.1em;%
      padding-right:.1em;%
    }%
  }%
  \global\let\HoLogoCss@KOMAScript\relax
}
%    \end{macrocode}
%    \end{macro}
%
% \subsubsection{\hologo{LyX}}
%
%    \begin{macro}{\HoLogo@LyX}
%    The definition is taken from the documentation source files
%    of \hologo{LyX}, \xfile{Intro.lyx} \cite{LyX}:
%\begin{quote}
%\begin{verbatim}
%\def\LyX{%
%  \texorpdfstring{%
%    L\kern-.1667em\lower.25em\hbox{Y}\kern-.125emX\@%
%  }{%
%    LyX%
%  }%
%}
%\end{verbatim}
%\end{quote}
%    \begin{macrocode}
\def\HoLogo@LyX#1{%
  L%
  \kern-.1667em%
  \lower.25em\hbox{Y}%
  \kern-.125em%
  X%
  \HOLOGO@SpaceFactor
}
%    \end{macrocode}
%    \end{macro}
%    \begin{macro}{\HoLogoHtml@LyX}
%    \begin{macrocode}
\def\HoLogoHtml@LyX#1{%
  \HoLogoCss@LyX
  \HOLOGO@Span{LyX}{%
    L%
    \HOLOGO@Span{y}{Y}%
    X%
  }%
}
%    \end{macrocode}
%    \end{macro}
%    \begin{macro}{\HoLogoCss@LyX}
%    \begin{macrocode}
\def\HoLogoCss@LyX{%
  \Css{%
    span.HoLogo-LyX span.HoLogo-y{%
      position:relative;%
      top:.25em;%
      margin-left:-.1667em;%
      margin-right:-.125em;%
      text-decoration:none;%
    }%
  }%
  \global\let\HoLogoCss@LyX\relax
}
%    \end{macrocode}
%    \end{macro}
%
% \subsubsection{\hologo{NTS}}
%
%    \begin{macro}{\HoLogo@NTS}
%    Definition for \hologo{NTS} can be found in
%    package \xpackage{etex\textunderscore man} for the \hologo{eTeX} manual \cite{etexman}
%    and in package \xpackage{dtklogos} \cite{dtklogos}:
%\begin{quote}
%\begin{verbatim}
%\def\NTS{%
%  \leavevmode
%  \hbox{%
%    $%
%      \cal N%
%      \kern-0.35em%
%      \lower0.5ex\hbox{$\cal T$}%
%      \kern-0.2em%
%      S%
%    $%
%  }%
%}
%\end{verbatim}
%\end{quote}
%    \begin{macrocode}
\def\HoLogo@NTS#1{%
  \HoLogoFont@font{NTS}{sy}{%
    N\/%
    \kern-.35em%
    \lower.5ex\hbox{T\/}%
    \kern-.2em%
    S\/%
  }%
  \HOLOGO@SpaceFactor
}
%    \end{macrocode}
%    \end{macro}
%
% \subsubsection{\Hologo{TTH} (\hologo{TeX} to HTML translator)}
%
%    Source: \url{http://hutchinson.belmont.ma.us/tth/}
%    In the HTML source the second `T' is printed as subscript.
%\begin{quote}
%\begin{verbatim}
%T<sub>T</sub>H
%\end{verbatim}
%\end{quote}
%    \begin{macro}{\HoLogo@TTH}
%    \begin{macrocode}
\def\HoLogo@TTH#1{%
  \ltx@mbox{%
    T\HOLOGO@SubScript{T}H%
  }%
  \HOLOGO@SpaceFactor
}
%    \end{macrocode}
%    \end{macro}
%
%    \begin{macro}{\HoLogoHtml@TTH}
%    \begin{macrocode}
\def\HoLogoHtml@TTH#1{%
  T\HCode{<sub>}T\HCode{</sub>}H%
}
%    \end{macrocode}
%    \end{macro}
%
% \subsubsection{\Hologo{HanTheThanh}}
%
%    Partial source: Package \xpackage{dtklogos}.
%    The double accent is U+1EBF (latin small letter e with circumflex
%    and acute).
%    \begin{macro}{\HoLogo@HanTheThanh}
%    \begin{macrocode}
\def\HoLogo@HanTheThanh#1{%
  \ltx@mbox{H\`an}%
  \HOLOGO@space
  \ltx@mbox{%
    Th%
    \HOLOGO@IfCharExists{"1EBF}{%
      \char"1EBF\relax
    }{%
      \^e\hbox to 0pt{\hss\raise .5ex\hbox{\'{}}}%
    }%
  }%
  \HOLOGO@space
  \ltx@mbox{Th\`anh}%
}
%    \end{macrocode}
%    \end{macro}
%    \begin{macro}{\HoLogoBkm@HanTheThanh}
%    \begin{macrocode}
\def\HoLogoBkm@HanTheThanh#1{%
  H\`an %
  Th\HOLOGO@PdfdocUnicode{\^e}{\9036\277} %
  Th\`anh%
}
%    \end{macrocode}
%    \end{macro}
%    \begin{macro}{\HoLogoHtml@HanTheThanh}
%    \begin{macrocode}
\def\HoLogoHtml@HanTheThanh#1{%
  H\`an %
  Th\HCode{&\ltx@hashchar x1ebf;} %
  Th\`anh%
}
%    \end{macrocode}
%    \end{macro}
%
% \subsection{Driver detection}
%
%    \begin{macrocode}
\HOLOGO@IfExists\InputIfFileExists{%
  \InputIfFileExists{hologo.cfg}{}{}%
}{%
  \ltx@IfUndefined{pdf@filesize}{%
    \def\HOLOGO@InputIfExists{%
      \openin\HOLOGO@temp=hologo.cfg\relax
      \ifeof\HOLOGO@temp
        \closein\HOLOGO@temp
      \else
        \closein\HOLOGO@temp
        \begingroup
          \def\x{LaTeX2e}%
        \expandafter\endgroup
        \ifx\fmtname\x
          \input{hologo.cfg}%
        \else
          \input hologo.cfg\relax
        \fi
      \fi
    }%
    \ltx@IfUndefined{newread}{%
      \chardef\HOLOGO@temp=15 %
      \def\HOLOGO@CheckRead{%
        \ifeof\HOLOGO@temp
          \HOLOGO@InputIfExists
        \else
          \ifcase\HOLOGO@temp
            \@PackageWarningNoLine{hologo}{%
              Configuration file ignored, because\MessageBreak
              a free read register could not be found%
            }%
          \else
            \begingroup
              \count\ltx@cclv=\HOLOGO@temp
              \advance\ltx@cclv by \ltx@minusone
              \edef\x{\endgroup
                \chardef\noexpand\HOLOGO@temp=\the\count\ltx@cclv
                \relax
              }%
            \x
          \fi
        \fi
      }%
    }{%
      \csname newread\endcsname\HOLOGO@temp
      \HOLOGO@InputIfExists
    }%
  }{%
    \edef\HOLOGO@temp{\pdf@filesize{hologo.cfg}}%
    \ifx\HOLOGO@temp\ltx@empty
    \else
      \ifnum\HOLOGO@temp>0 %
        \begingroup
          \def\x{LaTeX2e}%
        \expandafter\endgroup
        \ifx\fmtname\x
          \input{hologo.cfg}%
        \else
          \input hologo.cfg\relax
        \fi
      \else
        \@PackageInfoNoLine{hologo}{%
          Empty configuration file `hologo.cfg' ignored%
        }%
      \fi
    \fi
  }%
}
%    \end{macrocode}
%
%    \begin{macrocode}
\def\HOLOGO@temp#1#2{%
  \kv@define@key{HoLogoDriver}{#1}[]{%
    \begingroup
      \def\HOLOGO@temp{##1}%
      \ltx@onelevel@sanitize\HOLOGO@temp
      \ifx\HOLOGO@temp\ltx@empty
      \else
        \@PackageError{hologo}{%
          Value (\HOLOGO@temp) not permitted for option `#1'%
        }%
        \@ehc
      \fi
    \endgroup
    \def\hologoDriver{#2}%
  }%
}%
\def\HOLOGO@@temp#1#2{%
  \ifx\kv@value\relax
    \HOLOGO@temp{#1}{#1}%
  \else
    \HOLOGO@temp{#1}{#2}%
  \fi
}%
\kv@parse@normalized{%
  pdftex,%
  luatex=pdftex,%
  dvipdfm,%
  dvipdfmx=dvipdfm,%
  dvips,%
  dvipsone=dvips,%
  xdvi=dvips,%
  xetex,%
  vtex,%
}\HOLOGO@@temp
%    \end{macrocode}
%
%    \begin{macrocode}
\kv@define@key{HoLogoDriver}{driverfallback}{%
  \def\HOLOGO@DriverFallback{#1}%
}
%    \end{macrocode}
%
%    \begin{macro}{\HOLOGO@DriverFallback}
%    \begin{macrocode}
\def\HOLOGO@DriverFallback{dvips}
%    \end{macrocode}
%    \end{macro}
%
%    \begin{macro}{\hologoDriverSetup}
%    \begin{macrocode}
\def\hologoDriverSetup{%
  \let\hologoDriver\ltx@undefined
  \HOLOGO@DriverSetup
}
%    \end{macrocode}
%    \end{macro}
%
%    \begin{macro}{\HOLOGO@DriverSetup}
%    \begin{macrocode}
\def\HOLOGO@DriverSetup#1{%
  \kvsetkeys{HoLogoDriver}{#1}%
  \HOLOGO@CheckDriver
  \ltx@ifundefined{hologoDriver}{%
    \begingroup
    \edef\x{\endgroup
      \noexpand\kvsetkeys{HoLogoDriver}{\HOLOGO@DriverFallback}%
    }\x
  }{}%
  \@PackageInfoNoLine{hologo}{Using driver `\hologoDriver'}%
}
%    \end{macrocode}
%    \end{macro}
%
%    \begin{macro}{\HOLOGO@CheckDriver}
%    \begin{macrocode}
\def\HOLOGO@CheckDriver{%
  \ifpdf
    \def\hologoDriver{pdftex}%
    \let\HOLOGO@pdfliteral\pdfliteral
    \ifluatex
      \ifx\pdfextension\@undefined\else
        \protected\def\pdfliteral{\pdfextension literal}%
        \let\HOLOGO@pdfliteral\pdfliteral
      \fi
      \ltx@IfUndefined{HOLOGO@pdfliteral}{%
        \ifnum\luatexversion<36 %
        \else
          \begingroup
            \let\HOLOGO@temp\endgroup
            \ifcase0%
                \directlua{%
                  if tex.enableprimitives then %
                    tex.enableprimitives('HOLOGO@', {'pdfliteral'})%
                  else %
                    tex.print('1')%
                  end%
                }%
                \ifx\HOLOGO@pdfliteral\@undefined 1\fi%
                \relax%
              \endgroup
              \let\HOLOGO@temp\relax
              \global\let\HOLOGO@pdfliteral\HOLOGO@pdfliteral
            \fi%
          \HOLOGO@temp
        \fi
      }{}%
    \fi
    \ltx@IfUndefined{HOLOGO@pdfliteral}{%
      \@PackageWarningNoLine{hologo}{%
        Cannot find \string\pdfliteral
      }%
    }{}%
  \else
    \ifxetex
      \def\hologoDriver{xetex}%
    \else
      \ifvtex
        \def\hologoDriver{vtex}%
      \fi
    \fi
  \fi
}
%    \end{macrocode}
%    \end{macro}
%
%    \begin{macro}{\HOLOGO@WarningUnsupportedDriver}
%    \begin{macrocode}
\def\HOLOGO@WarningUnsupportedDriver#1{%
  \@PackageWarningNoLine{hologo}{%
    Logo `#1' needs driver specific macros,\MessageBreak
    but driver `\hologoDriver' is not supported.\MessageBreak
    Use a different driver or\MessageBreak
    load package `graphics' or `pgf'%
  }%
}
%    \end{macrocode}
%    \end{macro}
%
% \subsubsection{Reflect box macros}
%
%    Skip driver part if not needed.
%    \begin{macrocode}
\ltx@IfUndefined{reflectbox}{}{%
  \ltx@IfUndefined{rotatebox}{}{%
    \HOLOGO@AtEnd
  }%
}
\ltx@IfUndefined{pgftext}{}{%
  \HOLOGO@AtEnd
}
\ltx@IfUndefined{psscalebox}{}{%
  \HOLOGO@AtEnd
}
%    \end{macrocode}
%
%    \begin{macrocode}
\def\HOLOGO@temp{LaTeX2e}
\ifx\fmtname\HOLOGO@temp
  \RequirePackage{kvoptions}[2011/06/30]%
  \ProcessKeyvalOptions{HoLogoDriver}%
\fi
\HOLOGO@DriverSetup{}
%    \end{macrocode}
%
%    \begin{macro}{\HOLOGO@ReflectBox}
%    \begin{macrocode}
\def\HOLOGO@ReflectBox#1{%
  \begingroup
    \setbox\ltx@zero\hbox{\begingroup#1\endgroup}%
    \setbox\ltx@two\hbox{%
      \kern\wd\ltx@zero
      \csname HOLOGO@ScaleBox@\hologoDriver\endcsname{-1}{1}{%
        \hbox to 0pt{\copy\ltx@zero\hss}%
      }%
    }%
    \wd\ltx@two=\wd\ltx@zero
    \box\ltx@two
  \endgroup
}
%    \end{macrocode}
%    \end{macro}
%
%    \begin{macro}{\HOLOGO@PointReflectBox}
%    \begin{macrocode}
\def\HOLOGO@PointReflectBox#1{%
  \begingroup
    \setbox\ltx@zero\hbox{\begingroup#1\endgroup}%
    \setbox\ltx@two\hbox{%
      \kern\wd\ltx@zero
      \raise\ht\ltx@zero\hbox{%
        \csname HOLOGO@ScaleBox@\hologoDriver\endcsname{-1}{-1}{%
          \hbox to 0pt{\copy\ltx@zero\hss}%
        }%
      }%
    }%
    \wd\ltx@two=\wd\ltx@zero
    \box\ltx@two
  \endgroup
}
%    \end{macrocode}
%    \end{macro}
%
%    We must define all variants because of dynamic driver setup.
%    \begin{macrocode}
\def\HOLOGO@temp#1#2{#2}
%    \end{macrocode}
%
%    \begin{macro}{\HOLOGO@ScaleBox@pdftex}
%    \begin{macrocode}
\HOLOGO@temp{pdftex}{%
  \def\HOLOGO@ScaleBox@pdftex#1#2#3{%
    \HOLOGO@pdfliteral{%
      q #1 0 0 #2 0 0 cm%
    }%
    #3%
    \HOLOGO@pdfliteral{%
      Q%
    }%
  }%
}
%    \end{macrocode}
%    \end{macro}
%    \begin{macro}{\HOLOGO@ScaleBox@dvips}
%    \begin{macrocode}
\HOLOGO@temp{dvips}{%
  \def\HOLOGO@ScaleBox@dvips#1#2#3{%
    \special{ps:%
      gsave %
      currentpoint %
      currentpoint translate %
      #1 #2 scale %
      neg exch neg exch translate%
    }%
    #3%
    \special{ps:%
      currentpoint %
      grestore %
      moveto%
    }%
  }%
}
%    \end{macrocode}
%    \end{macro}
%    \begin{macro}{\HOLOGO@ScaleBox@dvipdfm}
%    \begin{macrocode}
\HOLOGO@temp{dvipdfm}{%
  \let\HOLOGO@ScaleBox@dvipdfm\HOLOGO@ScaleBox@dvips
}
%    \end{macrocode}
%    \end{macro}
%    Since \hologo{XeTeX} v0.6.
%    \begin{macro}{\HOLOGO@ScaleBox@xetex}
%    \begin{macrocode}
\HOLOGO@temp{xetex}{%
  \def\HOLOGO@ScaleBox@xetex#1#2#3{%
    \special{x:gsave}%
    \special{x:scale #1 #2}%
    #3%
    \special{x:grestore}%
  }%
}
%    \end{macrocode}
%    \end{macro}
%    \begin{macro}{\HOLOGO@ScaleBox@vtex}
%    \begin{macrocode}
\HOLOGO@temp{vtex}{%
  \def\HOLOGO@ScaleBox@vtex#1#2#3{%
    \special{r(#1,0,0,#2,0,0}%
    #3%
    \special{r)}%
  }%
}
%    \end{macrocode}
%    \end{macro}
%
%    \begin{macrocode}
\HOLOGO@AtEnd%
%</package>
%    \end{macrocode}
%
% \section{Test}
%
% \subsection{Catcode checks for loading}
%
%    \begin{macrocode}
%<*test1>
%    \end{macrocode}
%    \begin{macrocode}
\catcode`\{=1 %
\catcode`\}=2 %
\catcode`\#=6 %
\catcode`\@=11 %
\expandafter\ifx\csname count@\endcsname\relax
  \countdef\count@=255 %
\fi
\expandafter\ifx\csname @gobble\endcsname\relax
  \long\def\@gobble#1{}%
\fi
\expandafter\ifx\csname @firstofone\endcsname\relax
  \long\def\@firstofone#1{#1}%
\fi
\expandafter\ifx\csname loop\endcsname\relax
  \expandafter\@firstofone
\else
  \expandafter\@gobble
\fi
{%
  \def\loop#1\repeat{%
    \def\body{#1}%
    \iterate
  }%
  \def\iterate{%
    \body
      \let\next\iterate
    \else
      \let\next\relax
    \fi
    \next
  }%
  \let\repeat=\fi
}%
\def\RestoreCatcodes{}
\count@=0 %
\loop
  \edef\RestoreCatcodes{%
    \RestoreCatcodes
    \catcode\the\count@=\the\catcode\count@\relax
  }%
\ifnum\count@<255 %
  \advance\count@ 1 %
\repeat

\def\RangeCatcodeInvalid#1#2{%
  \count@=#1\relax
  \loop
    \catcode\count@=15 %
  \ifnum\count@<#2\relax
    \advance\count@ 1 %
  \repeat
}
\def\RangeCatcodeCheck#1#2#3{%
  \count@=#1\relax
  \loop
    \ifnum#3=\catcode\count@
    \else
      \errmessage{%
        Character \the\count@\space
        with wrong catcode \the\catcode\count@\space
        instead of \number#3%
      }%
    \fi
  \ifnum\count@<#2\relax
    \advance\count@ 1 %
  \repeat
}
\def\space{ }
\expandafter\ifx\csname LoadCommand\endcsname\relax
  \def\LoadCommand{\input hologo.sty\relax}%
\fi
\def\Test{%
  \RangeCatcodeInvalid{0}{47}%
  \RangeCatcodeInvalid{58}{64}%
  \RangeCatcodeInvalid{91}{96}%
  \RangeCatcodeInvalid{123}{255}%
  \catcode`\@=12 %
  \catcode`\\=0 %
  \catcode`\%=14 %
  \LoadCommand
  \RangeCatcodeCheck{0}{36}{15}%
  \RangeCatcodeCheck{37}{37}{14}%
  \RangeCatcodeCheck{38}{47}{15}%
  \RangeCatcodeCheck{48}{57}{12}%
  \RangeCatcodeCheck{58}{63}{15}%
  \RangeCatcodeCheck{64}{64}{12}%
  \RangeCatcodeCheck{65}{90}{11}%
  \RangeCatcodeCheck{91}{91}{15}%
  \RangeCatcodeCheck{92}{92}{0}%
  \RangeCatcodeCheck{93}{96}{15}%
  \RangeCatcodeCheck{97}{122}{11}%
  \RangeCatcodeCheck{123}{255}{15}%
  \RestoreCatcodes
}
\Test
\csname @@end\endcsname
\end
%    \end{macrocode}
%    \begin{macrocode}
%</test1>
%    \end{macrocode}
%
% \subsection{Spacefactor}
%
%    The space factor must be 1000 after a logo. If it is greater 1000
%    then the following space is a space after a sentence closing point.
%    If the space factor is smaller 1000 then an immediate following
%    dot is interpreted as abbreviation, not sentence closing point.
%
%    \begin{macrocode}
%<*test-spacefactor>
\NeedsTeXFormat{LaTeX2e}
\documentclass{article}
\usepackage{hologo}[2016/05/12]
\usepackage{kvsetkeys}
\usepackage{qstest}
\IncludeTests{*}
\LogTests{log}{*}{*}
\begin{document}
\begin{qstest}{spacefactor}{spacefactor}
\newcommand*{\Test}[1]{%
  \sbox0{%
    \hologo{#1}%
    \Expect*{1000 (#1)}*{\the\spacefactor\space(#1)}%
  }%
}%
\makeatletter
\def\TestList{}
\def\hologoEntry#1#2#3{%
  \edef\TestList{%
    \ifx\TestList\@empty
    \else
      \TestList,%
    \fi
    #1%
    \ifx\\#2\\%
    \else
      ={variant=#2}%
    \fi
  }%
}
\hologoList
\expandafter\kv@parse@normalized\expandafter{%
  \TestList
}{%
  \begingroup
    \let\@logo=\kv@key
    \ifx\kv@value\relax
    \else
      \expandafter\hologoLogoSetup\expandafter\@logo\expandafter{%
        \kv@value
      }%
    \fi
    \Test\@logo
  \endgroup
  \@gobbletwo
}
\end{qstest}
\end{document}
%</test-spacefactor>
%    \end{macrocode}
%
% \subsection{Complete list}
%
%    \begin{macrocode}
%<*test-list>
\NeedsTeXFormat{LaTeX2e}
\documentclass[12pt,a4paper]{article}
\usepackage{hologo}[2016/05/12]
\usepackage[T1]{fontenc}
\usepackage{lmodern}
\usepackage{parskip}
\usepackage[unicode]{hyperref}[2011/09/28]
\usepackage{bookmark}[2011/09/19]
\bookmarksetup{%
  numbered,%
  open,%
  openlevel=2,%
}
\renewcommand*{\contentsname}{List of logos}
\begin{document}
\tableofcontents
\def\TestFont#1#2#3#4#5#6{%
  \begingroup
    \usefont{#3}{#4}{#5}{#6}%
    \HologoVariant{#1}{#2}/\hologoVariant{#1}{#2}%
    \quad
    \begingroup\scriptsize\hologoVariant{#1}{#2}\endgroup
    \quad
  \endgroup
  (#3/#4/#5/#6)%
  \par
}
\makeatletter
\def\hologoEntry#1#2#3{%
  \section{%
    \HologoVariant{#1}{#2}/\hologoVariant{#1}{#2} %
    {[#1\ifx\\#2\\\else\space(#2)\fi]}% hash-ok
  }% braces around [] because of bug in tex4ht
  \begingroup
    \hypersetup{unicode=false}%
    \bookmark[%
      dest=\@currentHref,%
      rellevel=1,%
      keeplevel,%
    ]{%
      \HologoVariant{#1}{#2}/\hologoVariant{#1}{#2} %
      (PDFDocEncoding)%
    }%
  \endgroup
  \TestFont{#1}{#2}{OT1}{cmr}{m}{n}%
  \TestFont{#1}{#2}{OT1}{cmss}{m}{n}%
  \TestFont{#1}{#2}{OT1}{cmr}{b}{n}%
  \TestFont{#1}{#2}{OT1}{cmr}{m}{it}%
  \TestFont{#1}{#2}{OT1}{cmtt}{m}{n}%
  \TestFont{#1}{#2}{T1}{lmr}{m}{n}%
  \TestFont{#1}{#2}{T1}{lmss}{m}{n}%
  \TestFont{#1}{#2}{T1}{lmr}{b}{n}%
  \TestFont{#1}{#2}{T1}{lmr}{m}{it}%
  \TestFont{#1}{#2}{T1}{lmtt}{m}{n}%
  \TestFont{#1}{#2}{T1}{lmvtt}{m}{n}%
  \TestFont{#1}{#2}{T1}{qtm}{m}{n}%
  \TestFont{#1}{#2}{T1}{qhv}{m}{n}%
  \TestFont{#1}{#2}{T1}{qtm}{b}{n}%
  \TestFont{#1}{#2}{T1}{qtm}{m}{it}%
  \TestFont{#1}{#2}{T1}{qcr}{m}{n}%
  \newpage
}
\makeatother
\hologoList
\end{document}
%</test-list>
%    \end{macrocode}
%
% \section{Installation}
%
% \subsection{Download}
%
% \paragraph{Package.} This package is available on
% CTAN\footnote{\url{ftp://ftp.ctan.org/tex-archive/}}:
% \begin{description}
% \item[\CTAN{macros/latex/contrib/oberdiek/hologo.dtx}] The source file.
% \item[\CTAN{macros/latex/contrib/oberdiek/hologo.pdf}] Documentation.
% \end{description}
%
%
% \paragraph{Bundle.} All the packages of the bundle `oberdiek'
% are also available in a TDS compliant ZIP archive. There
% the packages are already unpacked and the documentation files
% are generated. The files and directories obey the TDS standard.
% \begin{description}
% \item[\CTAN{install/macros/latex/contrib/oberdiek.tds.zip}]
% \end{description}
% \emph{TDS} refers to the standard ``A Directory Structure
% for \TeX\ Files'' (\CTAN{tds/tds.pdf}). Directories
% with \xfile{texmf} in their name are usually organized this way.
%
% \subsection{Bundle installation}
%
% \paragraph{Unpacking.} Unpack the \xfile{oberdiek.tds.zip} in the
% TDS tree (also known as \xfile{texmf} tree) of your choice.
% Example (linux):
% \begin{quote}
%   |unzip oberdiek.tds.zip -d ~/texmf|
% \end{quote}
%
% \paragraph{Script installation.}
% Check the directory \xfile{TDS:scripts/oberdiek/} for
% scripts that need further installation steps.
% Package \xpackage{attachfile2} comes with the Perl script
% \xfile{pdfatfi.pl} that should be installed in such a way
% that it can be called as \texttt{pdfatfi}.
% Example (linux):
% \begin{quote}
%   |chmod +x scripts/oberdiek/pdfatfi.pl|\\
%   |cp scripts/oberdiek/pdfatfi.pl /usr/local/bin/|
% \end{quote}
%
% \subsection{Package installation}
%
% \paragraph{Unpacking.} The \xfile{.dtx} file is a self-extracting
% \docstrip\ archive. The files are extracted by running the
% \xfile{.dtx} through \plainTeX:
% \begin{quote}
%   \verb|tex hologo.dtx|
% \end{quote}
%
% \paragraph{TDS.} Now the different files must be moved into
% the different directories in your installation TDS tree
% (also known as \xfile{texmf} tree):
% \begin{quote}
% \def\t{^^A
% \begin{tabular}{@{}>{\ttfamily}l@{ $\rightarrow$ }>{\ttfamily}l@{}}
%   hologo.sty & tex/generic/oberdiek/hologo.sty\\
%   hologo.pdf & doc/latex/oberdiek/hologo.pdf\\
%   example/hologo-example.tex & doc/latex/oberdiek/example/hologo-example.tex\\
%   test/hologo-test1.tex & doc/latex/oberdiek/test/hologo-test1.tex\\
%   test/hologo-test-spacefactor.tex & doc/latex/oberdiek/test/hologo-test-spacefactor.tex\\
%   test/hologo-test-list.tex & doc/latex/oberdiek/test/hologo-test-list.tex\\
%   hologo.dtx & source/latex/oberdiek/hologo.dtx\\
% \end{tabular}^^A
% }^^A
% \sbox0{\t}^^A
% \ifdim\wd0>\linewidth
%   \begingroup
%     \advance\linewidth by\leftmargin
%     \advance\linewidth by\rightmargin
%   \edef\x{\endgroup
%     \def\noexpand\lw{\the\linewidth}^^A
%   }\x
%   \def\lwbox{^^A
%     \leavevmode
%     \hbox to \linewidth{^^A
%       \kern-\leftmargin\relax
%       \hss
%       \usebox0
%       \hss
%       \kern-\rightmargin\relax
%     }^^A
%   }^^A
%   \ifdim\wd0>\lw
%     \sbox0{\small\t}^^A
%     \ifdim\wd0>\linewidth
%       \ifdim\wd0>\lw
%         \sbox0{\footnotesize\t}^^A
%         \ifdim\wd0>\linewidth
%           \ifdim\wd0>\lw
%             \sbox0{\scriptsize\t}^^A
%             \ifdim\wd0>\linewidth
%               \ifdim\wd0>\lw
%                 \sbox0{\tiny\t}^^A
%                 \ifdim\wd0>\linewidth
%                   \lwbox
%                 \else
%                   \usebox0
%                 \fi
%               \else
%                 \lwbox
%               \fi
%             \else
%               \usebox0
%             \fi
%           \else
%             \lwbox
%           \fi
%         \else
%           \usebox0
%         \fi
%       \else
%         \lwbox
%       \fi
%     \else
%       \usebox0
%     \fi
%   \else
%     \lwbox
%   \fi
% \else
%   \usebox0
% \fi
% \end{quote}
% If you have a \xfile{docstrip.cfg} that configures and enables \docstrip's
% TDS installing feature, then some files can already be in the right
% place, see the documentation of \docstrip.
%
% \subsection{Refresh file name databases}
%
% If your \TeX~distribution
% (\teTeX, \mikTeX, \dots) relies on file name databases, you must refresh
% these. For example, \teTeX\ users run \verb|texhash| or
% \verb|mktexlsr|.
%
% \subsection{Some details for the interested}
%
% \paragraph{Attached source.}
%
% The PDF documentation on CTAN also includes the
% \xfile{.dtx} source file. It can be extracted by
% AcrobatReader 6 or higher. Another option is \textsf{pdftk},
% e.g. unpack the file into the current directory:
% \begin{quote}
%   \verb|pdftk hologo.pdf unpack_files output .|
% \end{quote}
%
% \paragraph{Unpacking with \LaTeX.}
% The \xfile{.dtx} chooses its action depending on the format:
% \begin{description}
% \item[\plainTeX:] Run \docstrip\ and extract the files.
% \item[\LaTeX:] Generate the documentation.
% \end{description}
% If you insist on using \LaTeX\ for \docstrip\ (really,
% \docstrip\ does not need \LaTeX), then inform the autodetect routine
% about your intention:
% \begin{quote}
%   \verb|latex \let\install=y\input{hologo.dtx}|
% \end{quote}
% Do not forget to quote the argument according to the demands
% of your shell.
%
% \paragraph{Generating the documentation.}
% You can use both the \xfile{.dtx} or the \xfile{.drv} to generate
% the documentation. The process can be configured by the
% configuration file \xfile{ltxdoc.cfg}. For instance, put this
% line into this file, if you want to have A4 as paper format:
% \begin{quote}
%   \verb|\PassOptionsToClass{a4paper}{article}|
% \end{quote}
% An example follows how to generate the
% documentation with pdf\LaTeX:
% \begin{quote}
%\begin{verbatim}
%pdflatex hologo.dtx
%makeindex -s gind.ist hologo.idx
%pdflatex hologo.dtx
%makeindex -s gind.ist hologo.idx
%pdflatex hologo.dtx
%\end{verbatim}
% \end{quote}
%
% \section{Catalogue}
%
% The following XML file can be used as source for the
% \href{http://mirror.ctan.org/help/Catalogue/catalogue.html}{\TeX\ Catalogue}.
% The elements \texttt{caption} and \texttt{description} are imported
% from the original XML file from the Catalogue.
% The name of the XML file in the Catalogue is \xfile{hologo.xml}.
%    \begin{macrocode}
%<*catalogue>
<?xml version='1.0' encoding='us-ascii'?>
<!DOCTYPE entry SYSTEM 'catalogue.dtd'>
<entry datestamp='$Date$' modifier='$Author$' id='hologo'>
  <name>hologo</name>
  <caption>A collection of logos with bookmark support.</caption>
  <authorref id='auth:oberdiek'/>
  <copyright owner='Heiko Oberdiek' year='2010-2012'/>
  <license type='lppl1.3'/>
  <version number='1.10'/>
  <description>
    The package defines a single command <tt>\hologo</tt>, whose
    argument is the usual case-confused ASCII version of the logo.
    The command is bookmark-enabled, so that every logo becomes
    available in bookmarks without further work.
    <p/>
    The package is part of the <xref refid='oberdiek'>oberdiek</xref>
    bundle.
  </description>
  <documentation details='Package documentation'
      href='ctan:/macros/latex/contrib/oberdiek/hologo.pdf'/>
  <ctan file='true' path='/macros/latex/contrib/oberdiek/hologo.dtx'/>
  <miktex location='oberdiek'/>
  <texlive location='oberdiek'/>
  <install path='/macros/latex/contrib/oberdiek/oberdiek.tds.zip'/>
</entry>
%</catalogue>
%    \end{macrocode}
%
% \begin{thebibliography}{9}
% \raggedright
%
% \bibitem{btxdoc}
% Oren Patashnik,
% \textit{\hologo{BibTeX}ing},
% 1988-02-08.\\
% \CTAN{biblio/bibtex/base/}
%
% \bibitem{dtklogos}
% Gerd Neugebauer, DANTE,
% \textit{Package \xpackage{dtklogos}},
% 2011-04-25.\\
% \CTAN{usergrps/dante/dtk/dtklogos.sty}
%
% \bibitem{etexman}
% The \hologo{NTS} Team,
% \textit{The \hologo{eTeX} manual},
% 1998-02.\\
% \CTAN{systems/e-tex/v2/doc/}
%
% \bibitem{ExTeX-FAQ}
% The \hologo{ExTeX} group,
% \textit{\hologo{ExTeX}: FAQ -- How is \hologo{ExTeX} typeset?},
% 2007-04-14.\\
% \url{http://www.extex.org/documentation/faq.html}
%
% \bibitem{LyX}
% %@MISC{ LyX,
% %  title = {{LyX 2.0.0 -- The Document Processor [Computer software and manual]}},
% %  author = {{The LyX Team}},
% %  howpublished = {Internet: http://www.lyx.org},
% %  year = {2011-05-08},
% %  note = {Retrieved May 10, 2011, from http://www.lyx.org},
% %  url = {http://www.lyx.org/}
% %}
% The \hologo{LyX} Team,
% \textit{\hologo{LyX} -- The Document Processor},
% 2011-05-08.\\
% \url{http://www.lyx.org/}
%
% \bibitem{OzTeX}
% Andrew Trevorrow,
% \hologo{OzTeX} FAQ: What is the correct way to typeset ``\hologo{OzTeX}''?,
% 2011-09-15 (visited).
% \url{http://www.trevorrow.com/oztex/ozfaq.html#oztex-logo}
%
% \bibitem{PiCTeX}
% Michael Wichura,
% \textit{The \hologo{PiCTeX} macro package},
% 1987-09-21.
% \CTAN{graphics/pictex/}
%
% \bibitem{scrlogo}
% Markus Kohm,
% \textit{\hologo{KOMAScript} Datei \xfile{scrlogo.dtx}},
% 2009-01-30.\\
% \CTAN{install/macros/latex/contrib/komascript.tds.zip}
%
% \end{thebibliography}
%
% \begin{History}
%   \begin{Version}{2010/04/08 v1.0}
%   \item
%     The first version.
%   \end{Version}
%   \begin{Version}{2010/04/16 v1.1}
%   \item
%     \cs{Hologo} added for support of logos at start of a sentence.
%   \item
%     \cs{hologoSetup} and \cs{hologoLogoSetup} added.
%   \item
%     Options \xoption{break}, \xoption{hyphenbreak}, \xoption{spacebreak}
%     added.
%   \item
%     Variant support added by option \xoption{variant}.
%   \end{Version}
%   \begin{Version}{2010/04/24 v1.2}
%   \item
%     \hologo{LaTeX3} added.
%   \item
%     \hologo{VTeX} added.
%   \end{Version}
%   \begin{Version}{2010/11/21 v1.3}
%   \item
%     \hologo{iniTeX}, \hologo{virTeX} added.
%   \end{Version}
%   \begin{Version}{2011/03/25 v1.4}
%   \item
%     \hologo{ConTeXt} with variants added.
%   \item
%     Option \xoption{discretionarybreak} added as refinement for
%     option \xoption{break}.
%   \end{Version}
%   \begin{Version}{2011/04/21 v1.5}
%   \item
%     Wrong TDS directory for test files fixed.
%   \end{Version}
%   \begin{Version}{2011/10/01 v1.6}
%   \item
%     Support for package \xpackage{tex4ht} added.
%   \item
%     Support for \cs{csname} added if \cs{ifincsname} is available.
%   \item
%     New logos:
%     \hologo{(La)TeX},
%     \hologo{biber},
%     \hologo{BibTeX} (\xoption{sc}, \xoption{sf}),
%     \hologo{emTeX},
%     \hologo{ExTeX},
%     \hologo{KOMAScript},
%     \hologo{La},
%     \hologo{LyX},
%     \hologo{MiKTeX},
%     \hologo{NTS},
%     \hologo{OzMF},
%     \hologo{OzMP},
%     \hologo{OzTeX},
%     \hologo{OzTtH},
%     \hologo{PCTeX},
%     \hologo{PiC},
%     \hologo{PiCTeX},
%     \hologo{METAFONT},
%     \hologo{MetaFun},
%     \hologo{METAPOST},
%     \hologo{MetaPost},
%     \hologo{SLiTeX} (\xoption{lift}, \xoption{narrow}, \xoption{simple}),
%     \hologo{SliTeX} (\xoption{narrow}, \xoption{simple}, \xoption{lift}),
%     \hologo{teTeX}.
%   \item
%     Fixes:
%     \hologo{iniTeX},
%     \hologo{pdfLaTeX},
%     \hologo{pdfTeX},
%     \hologo{virTeX}.
%   \item
%     \cs{hologoFontSetup} and \cs{hologoLogoFontSetup} added.
%   \item
%     \cs{hologoVariant} and \cs{HologoVariant} added.
%   \end{Version}
%   \begin{Version}{2011/11/22 v1.7}
%   \item
%     New logos:
%     \hologo{BibTeX8},
%     \hologo{LaTeXML},
%     \hologo{SageTeX},
%     \hologo{TeX4ht},
%     \hologo{TTH}.
%   \item
%     \hologo{Xe} and friends: Driver stuff fixed.
%   \item
%     \hologo{Xe} and friends: Support for italic added.
%   \item
%     \hologo{Xe} and friends: Package support for \xpackage{pgf}
%     and \xpackage{pstricks} added.
%   \end{Version}
%   \begin{Version}{2011/11/29 v1.8}
%   \item
%     New logos:
%     \hologo{HanTheThanh}.
%   \end{Version}
%   \begin{Version}{2011/12/21 v1.9}
%   \item
%     Patch for package \xpackage{ifxetex} added for the case that
%     \cs{newif} is undefined in \hologo{iniTeX}.
%   \item
%     Some fixes for \hologo{iniTeX}.
%   \end{Version}
%   \begin{Version}{2012/04/26 v1.10}
%   \item
%     Fix in bookmark version of logo ``\hologo{HanTheThanh}''.
%   \end{Version}
%   \begin{Version}{2016/05/12 v1.11}
%   \item
%     Update HOLOGO@IfCharExists (previously in texlive)
%   \item define pdfliteral in current luatex.
%   \end{Version}
% \end{History}
%
% \PrintIndex
%
% \Finale
\endinput
%
        \else
          \input hologo.cfg\relax
        \fi
      \fi
    }%
    \ltx@IfUndefined{newread}{%
      \chardef\HOLOGO@temp=15 %
      \def\HOLOGO@CheckRead{%
        \ifeof\HOLOGO@temp
          \HOLOGO@InputIfExists
        \else
          \ifcase\HOLOGO@temp
            \@PackageWarningNoLine{hologo}{%
              Configuration file ignored, because\MessageBreak
              a free read register could not be found%
            }%
          \else
            \begingroup
              \count\ltx@cclv=\HOLOGO@temp
              \advance\ltx@cclv by \ltx@minusone
              \edef\x{\endgroup
                \chardef\noexpand\HOLOGO@temp=\the\count\ltx@cclv
                \relax
              }%
            \x
          \fi
        \fi
      }%
    }{%
      \csname newread\endcsname\HOLOGO@temp
      \HOLOGO@InputIfExists
    }%
  }{%
    \edef\HOLOGO@temp{\pdf@filesize{hologo.cfg}}%
    \ifx\HOLOGO@temp\ltx@empty
    \else
      \ifnum\HOLOGO@temp>0 %
        \begingroup
          \def\x{LaTeX2e}%
        \expandafter\endgroup
        \ifx\fmtname\x
          % \iffalse meta-comment
%
% File: hologo.dtx
% Version: 2016/05/12 v1.11
% Info: A logo collection with bookmark support
%
% Copyright (C) 2010-2012 by
%    Heiko Oberdiek <heiko.oberdiek at googlemail.com>
%
% This work may be distributed and/or modified under the
% conditions of the LaTeX Project Public License, either
% version 1.3c of this license or (at your option) any later
% version. This version of this license is in
%    http://www.latex-project.org/lppl/lppl-1-3c.txt
% and the latest version of this license is in
%    http://www.latex-project.org/lppl.txt
% and version 1.3 or later is part of all distributions of
% LaTeX version 2005/12/01 or later.
%
% This work has the LPPL maintenance status "maintained".
%
% This Current Maintainer of this work is Heiko Oberdiek.
%
% The Base Interpreter refers to any `TeX-Format',
% because some files are installed in TDS:tex/generic//.
%
% This work consists of the main source file hologo.dtx
% and the derived files
%    hologo.sty, hologo.pdf, hologo.ins, hologo.drv, hologo-example.tex,
%    hologo-test1.tex, hologo-test-spacefactor.tex,
%    hologo-test-list.tex.
%
% Distribution:
%    CTAN:macros/latex/contrib/oberdiek/hologo.dtx
%    CTAN:macros/latex/contrib/oberdiek/hologo.pdf
%
% Unpacking:
%    (a) If hologo.ins is present:
%           tex hologo.ins
%    (b) Without hologo.ins:
%           tex hologo.dtx
%    (c) If you insist on using LaTeX
%           latex \let\install=y\input{hologo.dtx}
%        (quote the arguments according to the demands of your shell)
%
% Documentation:
%    (a) If hologo.drv is present:
%           latex hologo.drv
%    (b) Without hologo.drv:
%           latex hologo.dtx; ...
%    The class ltxdoc loads the configuration file ltxdoc.cfg
%    if available. Here you can specify further options, e.g.
%    use A4 as paper format:
%       \PassOptionsToClass{a4paper}{article}
%
%    Programm calls to get the documentation (example):
%       pdflatex hologo.dtx
%       makeindex -s gind.ist hologo.idx
%       pdflatex hologo.dtx
%       makeindex -s gind.ist hologo.idx
%       pdflatex hologo.dtx
%
% Installation:
%    TDS:tex/generic/oberdiek/hologo.sty
%    TDS:doc/latex/oberdiek/hologo.pdf
%    TDS:doc/latex/oberdiek/example/hologo-example.tex
%    TDS:doc/latex/oberdiek/test/hologo-test1.tex
%    TDS:doc/latex/oberdiek/test/hologo-test-spacefactor.tex
%    TDS:doc/latex/oberdiek/test/hologo-test-list.tex
%    TDS:source/latex/oberdiek/hologo.dtx
%
%<*ignore>
\begingroup
  \catcode123=1 %
  \catcode125=2 %
  \def\x{LaTeX2e}%
\expandafter\endgroup
\ifcase 0\ifx\install y1\fi\expandafter
         \ifx\csname processbatchFile\endcsname\relax\else1\fi
         \ifx\fmtname\x\else 1\fi\relax
\else\csname fi\endcsname
%</ignore>
%<*install>
\input docstrip.tex
\Msg{************************************************************************}
\Msg{* Installation}
\Msg{* Package: hologo 2016/05/12 v1.11 A logo collection with bookmark support (HO)}
\Msg{************************************************************************}

\keepsilent
\askforoverwritefalse

\let\MetaPrefix\relax
\preamble

This is a generated file.

Project: hologo
Version: 2016/05/12 v1.11

Copyright (C) 2010-2012 by
   Heiko Oberdiek <heiko.oberdiek at googlemail.com>

This work may be distributed and/or modified under the
conditions of the LaTeX Project Public License, either
version 1.3c of this license or (at your option) any later
version. This version of this license is in
   http://www.latex-project.org/lppl/lppl-1-3c.txt
and the latest version of this license is in
   http://www.latex-project.org/lppl.txt
and version 1.3 or later is part of all distributions of
LaTeX version 2005/12/01 or later.

This work has the LPPL maintenance status "maintained".

This Current Maintainer of this work is Heiko Oberdiek.

The Base Interpreter refers to any `TeX-Format',
because some files are installed in TDS:tex/generic//.

This work consists of the main source file hologo.dtx
and the derived files
   hologo.sty, hologo.pdf, hologo.ins, hologo.drv, hologo-example.tex,
   hologo-test1.tex, hologo-test-spacefactor.tex,
   hologo-test-list.tex.

\endpreamble
\let\MetaPrefix\DoubleperCent

\generate{%
  \file{hologo.ins}{\from{hologo.dtx}{install}}%
  \file{hologo.drv}{\from{hologo.dtx}{driver}}%
  \usedir{tex/generic/oberdiek}%
  \file{hologo.sty}{\from{hologo.dtx}{package}}%
  \usedir{doc/latex/oberdiek/example}%
  \file{hologo-example.tex}{\from{hologo.dtx}{example}}%
  \usedir{doc/latex/oberdiek/test}%
  \file{hologo-test1.tex}{\from{hologo.dtx}{test1}}%
  \file{hologo-test-spacefactor.tex}{\from{hologo.dtx}{test-spacefactor}}%
  \file{hologo-test-list.tex}{\from{hologo.dtx}{test-list}}%
  \nopreamble
  \nopostamble
  \usedir{source/latex/oberdiek/catalogue}%
  \file{hologo.xml}{\from{hologo.dtx}{catalogue}}%
}

\catcode32=13\relax% active space
\let =\space%
\Msg{************************************************************************}
\Msg{*}
\Msg{* To finish the installation you have to move the following}
\Msg{* file into a directory searched by TeX:}
\Msg{*}
\Msg{*     hologo.sty}
\Msg{*}
\Msg{* To produce the documentation run the file `hologo.drv'}
\Msg{* through LaTeX.}
\Msg{*}
\Msg{* Happy TeXing!}
\Msg{*}
\Msg{************************************************************************}

\endbatchfile
%</install>
%<*ignore>
\fi
%</ignore>
%<*driver>
\NeedsTeXFormat{LaTeX2e}
\ProvidesFile{hologo.drv}%
  [2016/05/12 v1.11 A logo collection with bookmark support (HO)]%
\documentclass{ltxdoc}
\usepackage{holtxdoc}[2011/11/22]
\usepackage{hologo}[2016/05/12]
\usepackage{longtable}
\usepackage{array}
\usepackage{paralist}
%\usepackage[T1]{fontenc}
%\usepackage{lmodern}
\begin{document}
  \DocInput{hologo.dtx}%
\end{document}
%</driver>
% \fi
%
%
% \CharacterTable
%  {Upper-case    \A\B\C\D\E\F\G\H\I\J\K\L\M\N\O\P\Q\R\S\T\U\V\W\X\Y\Z
%   Lower-case    \a\b\c\d\e\f\g\h\i\j\k\l\m\n\o\p\q\r\s\t\u\v\w\x\y\z
%   Digits        \0\1\2\3\4\5\6\7\8\9
%   Exclamation   \!     Double quote  \"     Hash (number) \#
%   Dollar        \$     Percent       \%     Ampersand     \&
%   Acute accent  \'     Left paren    \(     Right paren   \)
%   Asterisk      \*     Plus          \+     Comma         \,
%   Minus         \-     Point         \.     Solidus       \/
%   Colon         \:     Semicolon     \;     Less than     \<
%   Equals        \=     Greater than  \>     Question mark \?
%   Commercial at \@     Left bracket  \[     Backslash     \\
%   Right bracket \]     Circumflex    \^     Underscore    \_
%   Grave accent  \`     Left brace    \{     Vertical bar  \|
%   Right brace   \}     Tilde         \~}
%
% \GetFileInfo{hologo.drv}
%
% \title{The \xpackage{hologo} package}
% \date{2016/05/12 v1.11}
% \author{Heiko Oberdiek\\\xemail{heiko.oberdiek at googlemail.com}}
%
% \maketitle
%
% \begin{abstract}
% This package starts a collection of logos with support for bookmarks
% strings.
% \end{abstract}
%
% \tableofcontents
%
% \section{Documentation}
%
% \subsection{Logo macros}
%
% \begin{declcs}{hologo} \M{name}
% \end{declcs}
% Macro \cs{hologo} sets the logo with name \meta{name}.
% The following table shows the supported names.
%
% \begingroup
%   \def\hologoEntry#1#2#3{^^A
%     #1&#2&\hologoLogoSetup{#1}{variant=#2}\hologo{#1}&#3\tabularnewline
%   }
%   \begin{longtable}{>{\ttfamily}l>{\ttfamily}lll}
%     \rmfamily\bfseries{name} & \rmfamily\bfseries variant
%     & \bfseries logo & \bfseries since\\
%     \hline
%     \endhead
%     \hologoList
%   \end{longtable}
% \endgroup
%
% \begin{declcs}{Hologo} \M{name}
% \end{declcs}
% Macro \cs{Hologo} starts the logo \meta{name} with an uppercase
% letter. As an exception small greek letters are not converted
% to uppercase. Examples, see \hologo{eTeX} and \hologo{ExTeX}.
%
% \subsection{Setup macros}
%
% The package does not support package options, but the following
% setup macros can be used to set options.
%
% \begin{declcs}{hologoSetup} \M{key value list}
% \end{declcs}
% Macro \cs{hologoSetup} sets global options.
%
% \begin{declcs}{hologoLogoSetup} \M{logo} \M{key value list}
% \end{declcs}
% Some options can also be used to configure a logo.
% These settings take precedence over global option settings.
%
% \subsection{Options}\label{sec:options}
%
% There are boolean and string options:
% \begin{description}
% \item[Boolean option:]
% It takes |true| or |false|
% as value. If the value is omitted, then |true| is used.
% \item[String option:]
% A value must be given as string. (But the string might be empty.)
% \end{description}
% The following options can be used both in \cs{hologoSetup}
% and \cs{hologoLogoSetup}:
% \begin{description}
% \def\entry#1{\item[\xoption{#1}:]}
% \entry{break}
%   enables or disables line breaks inside the logo. This setting is
%   refined by options \xoption{hyphenbreak}, \xoption{spacebreak}
%   or \xoption{discretionarybreak}.
%   Default is |false|.
% \entry{hyphenbreak}
%   enables or disables the line break right after the hyphen character.
% \entry{spacebreak}
%   enables or disables line breaks at space characters.
% \entry{discretionarybreak}
%   enables or disables line breaks at hyphenation points
%   (inserted by \cs{-}).
% \end{description}
% Macro \cs{hologoLogoSetup} also knows:
% \begin{description}
% \item[\xoption{variant}:]
%   This is a string option. It specifies a variant of a logo that
%   must exist. An empty string selects the package default variant.
% \end{description}
% Example:
% \begin{quote}
%   |\hologoSetup{break=false}|\\
%   |\hologoLogoSetup{plainTeX}{variant=hyphen,hyphenbreak}|\\
%   Then ``plain-\TeX'' contains one break point after the hyphen.
% \end{quote}
%
% \subsection{Driver options}
%
% Sometimes graphical operations are needed to construct some
% glyphs (e.g.\ \hologo{XeTeX}). If package \xpackage{graphics}
% or package \xpackage{pgf} are found, then the macros are taken
% from there. Otherwise the packge defines its own operations
% and therefore needs the driver information. Many drivers are
% detected automatically (\hologo{pdfTeX}/\hologo{LuaTeX}
% in PDF mode, \hologo{XeTeX}, \hologo{VTeX}). These have precedence
% over a driver option. The driver can be given as package option
% or using \cs{hologoDriverSetup}.
% The following list contains the recognized driver options:
% \begin{itemize}
% \item \xoption{pdftex}, \xoption{luatex}
% \item \xoption{dvipdfm}, \xoption{dvipdfmx}
% \item \xoption{dvips}, \xoption{dvipsone}, \xoption{xdvi}
% \item \xoption{xetex}
% \item \xoption{vtex}
% \end{itemize}
% The left driver of a line is the driver name that is used internally.
% The following names are aliases for drivers that use the
% same method. Therefore the entry in the \xext{log} file for
% the used driver prints the internally used driver name.
% \begin{description}
% \item[\xoption{driverfallback}:]
%   This option expects a driver that is used,
%   if the driver could not be detected automatically.
% \end{description}
%
% \begin{declcs}{hologoDriverSetup} \M{driver option}
% \end{declcs}
% The driver can also be configured after package loading
% using \cs{hologoDriverSetup}, also the way for \hologo{plainTeX}
% to setup the driver.
%
% \subsection{Font setup}
%
% Some logos require a special font, but should also be usable by
% \hologo{plainTeX}. Therefore the package provides some ways
% to influence the font settings. The options below
% take font settings as values. Both font commands
% such as \cs{sffamily} and macros that take one argument
% like \cs{textsf} can be used.
%
% \begin{declcs}{hologoFontSetup} \M{key value list}
% \end{declcs}
% Macro \cs{hologoFontSetup} sets the fonts for all logos.
% Supported keys:
% \begin{description}
% \def\entry#1{\item[\xoption{#1}:]}
% \entry{general}
%   This font is used for all logos. The default is empty.
%   That means no special font is used.
% \entry{bibsf}
%   This font is used for
%   {\hologoLogoSetup{BibTeX}{variant=sf}\hologo{BibTeX}}
%   with variant \xoption{sf}.
% \entry{rm}
%   This font is a serif font. It is used for \hologo{ExTeX}.
% \entry{sc}
%   This font specifies a small caps font. It is used for
%   {\hologoLogoSetup{BibTeX}{variant=sc}\hologo{BibTeX}}
%   with variant \xoption{sc}.
% \entry{sf}
%   This font specifies a sans serif font. The default
%   is \cs{sffamily}, then \cs{sf} is tried. Otherwise
%   a warning is given. It is used by \hologo{KOMAScript}.
% \entry{sy}
%   This is the font for math symbols (e.g. cmsy).
%   It is used by \hologo{AmS}, \hologo{NTS}, \hologo{ExTeX}.
% \entry{logo}
%   \hologo{METAFONT} and \hologo{METAPOST} are using that font.
%   In \hologo{LaTeX} \cs{logofamily} is used and
%   the definitions of package \xpackage{mflogo} are used
%   if the package is not loaded.
%   Otherwise the \cs{tenlogo} is used and defined
%   if it does not already exists.
% \end{description}
%
% \begin{declcs}{hologoLogoFontSetup} \M{logo} \M{key value list}
% \end{declcs}
% Fonts can also be set for a logo or logo component separately,
% see the following list.
% The keys are the same as for \cs{hologoFontSetup}.
%
% \begin{longtable}{>{\ttfamily}l>{\sffamily}ll}
%   \meta{logo} & keys & result\\
%   \hline
%   \endhead
%   BibTeX & bibsf & {\hologoLogoSetup{BibTeX}{variant=sf}\hologo{BibTeX}}\\[.5ex]
%   BibTeX & sc & {\hologoLogoSetup{BibTeX}{variant=sc}\hologo{BibTeX}}\\[.5ex]
%   ExTeX & rm & \hologo{ExTeX}\\
%   SliTeX & rm & \hologo{SliTeX}\\[.5ex]
%   AmS & sy & \hologo{AmS}\\
%   ExTeX & sy & \hologo{ExTeX}\\
%   NTS & sy & \hologo{NTS}\\[.5ex]
%   KOMAScript & sf & \hologo{KOMAScript}\\[.5ex]
%   METAFONT & logo & \hologo{METAFONT}\\
%   METAPOST & logo & \hologo{METAPOST}\\[.5ex]
%   SliTeX & sc \hologo{SliTeX}
% \end{longtable}
%
% \subsubsection{Font order}
%
% For all logos the font \xoption{general} is applied first.
% Example:
%\begin{quote}
%|\hologoFontSetup{general=\color{red}}|
%\end{quote}
% will print red logos.
% Then if the font uses a special font \xoption{sf}, for example,
% the font is applied that is setup by \cs{hologoLogoFontSetup}.
% If this font is not setup, then the common font setup
% by \cs{hologoFontSetup} is used. Otherwise a warning is given,
% that there is no font configured.
%
% \subsection{Additional user macros}
%
% Usually a variant of a logo is configured by using
% \cs{hologoLogoSetup}, because it is bad style to mix
% different variants of the same logo in the same text.
% There the following macros are a convenience for testing.
%
% \begin{declcs}{hologoVariant} \M{name} \M{variant}\\
%   \cs{HologoVariant} \M{name} \M{variant}
% \end{declcs}
% Logo \meta{name} is set using \meta{variant} that specifies
% explicitely which variant of the macro is used. If the argument
% is empty, then the default form of the logo is used
% (configurable by \cs{hologoLogoSetup}).
%
% \cs{HologoVariant} is used if the logo is set in a context
% that needs an uppercase first letter (beginning of a sentence, \dots).
%
% \begin{declcs}{hologoList}\\
%   \cs{hologoEntry} \M{logo} \M{variant} \M{since}
% \end{declcs}
% Macro \cs{hologoList} contains all logos that are provided
% by the package including variants. The list consists of calls
% of \cs{hologoEntry} with three arguments starting with the
% logo name \meta{logo} and its variant \meta{variant}. An empty
% variant means the current default. Argument \meta{since} specifies
% with version of the package \xpackage{hologo} is needed to get
% the logo. If the logo is fixed, then the date gets updated.
% Therefore the date \meta{since} is not exactly the date of
% the first introduction, but rather the date of the latest fix.
%
% Before \cs{hologoList} can be used, macro \cs{hologoEntry} needs
% a definition. The example file in section \ref{sec:example}
% shows applications of \cs{hologoList}.
%
% \subsection{Supported contexts}
%
% Macros \cs{hologo} and friends support special contexts:
% \begin{itemize}
% \item \hologo{LaTeX}'s protection mechanism.
% \item Bookmarks of package \xpackage{hyperref}.
% \item Package \xpackage{tex4ht}.
% \item The macros can be used inside \cs{csname} constructs,
%   if \cs{ifincsname} is available (\hologo{pdfTeX}, \hologo{XeTeX},
%   \hologo{LuaTeX}).
% \end{itemize}
%
% \subsection{Example}
% \label{sec:example}
%
% The following example prints the logos in different fonts.
%    \begin{macrocode}
%<*example>
%<<verbatim
\NeedsTeXFormat{LaTeX2e}
\documentclass[a4paper]{article}
\usepackage[
  hmargin=20mm,
  vmargin=20mm,
]{geometry}
\pagestyle{empty}
\usepackage{hologo}[2016/05/12]
\usepackage{longtable}
\usepackage{array}
\setlength{\extrarowheight}{2pt}
\usepackage[T1]{fontenc}
\usepackage{lmodern}
\usepackage{pdflscape}
\usepackage[
  pdfencoding=auto,
]{hyperref}
\hypersetup{
  pdfauthor={Heiko Oberdiek},
  pdftitle={Example for package `hologo'},
  pdfsubject={Logos with fonts lmr, lmss, qtm, qpl, qhv},
}
\usepackage{bookmark}

% Print the logo list on the console

\begingroup
  \typeout{}%
  \typeout{*** Begin of logo list ***}%
  \newcommand*{\hologoEntry}[3]{%
    \typeout{#1 \ifx\\#2\\\else(#2) \fi[#3]}%
  }%
  \hologoList
  \typeout{*** End of logo list ***}%
  \typeout{}%
\endgroup

\begin{document}
\begin{landscape}

  \section{Example file for package `hologo'}

  % Table for font names

  \begin{longtable}{>{\bfseries}ll}
    \textbf{font} & \textbf{Font name}\\
    \hline
    lmr & Latin Modern Roman\\
    lmss & Latin Modern Sans\\
    qtm & \TeX\ Gyre Termes\\
    qhv & \TeX\ Gyre Heros\\
    qpl & \TeX\ Gyre Pagella\\
  \end{longtable}

  % Logo list with logos in different fonts

  \begingroup
    \newcommand*{\SetVariant}[2]{%
      \ifx\\#2\\%
      \else
        \hologoLogoSetup{#1}{variant=#2}%
      \fi
    }%
    \newcommand*{\hologoEntry}[3]{%
      \SetVariant{#1}{#2}%
      \raisebox{1em}[0pt][0pt]{\hypertarget{#1@#2}{}}%
      \bookmark[%
        dest={#1@#2},%
      ]{%
        #1\ifx\\#2\\\else\space(#2)\fi: \Hologo{#1}, \hologo{#1} %
        [Unicode]%
      }%
      \hypersetup{unicode=false}%
      \bookmark[%
        dest={#1@#2},%
      ]{%
        #1\ifx\\#2\\\else\space(#2)\fi: \Hologo{#1}, \hologo{#1} %
        [PDFDocEncoding]%
      }%
      \texttt{#1}%
      &%
      \texttt{#2}%
      &%
      \Hologo{#1}%
      &%
      \SetVariant{#1}{#2}%
      \hologo{#1}%
      &%
      \SetVariant{#1}{#2}%
      \fontfamily{qtm}\selectfont
      \hologo{#1}%
      &%
      \SetVariant{#1}{#2}%
      \fontfamily{qpl}\selectfont
      \hologo{#1}%
      &%
      \SetVariant{#1}{#2}%
      \textsf{\hologo{#1}}%
      &%
      \SetVariant{#1}{#2}%
      \fontfamily{qhv}\selectfont
      \hologo{#1}%
      \tabularnewline
    }%
    \begin{longtable}{llllllll}%
      \textbf{\textit{logo}} & \textbf{\textit{variant}} &
      \texttt{\string\Hologo} &
      \textbf{lmr} & \textbf{qtm} & \textbf{qpl} &
      \textbf{lmss} & \textbf{qhv}
      \tabularnewline
      \hline
      \endhead
      \hologoList
    \end{longtable}%
  \endgroup

\end{landscape}
\end{document}
%verbatim
%</example>
%    \end{macrocode}
%
% \StopEventually{
% }
%
% \section{Implementation}
%    \begin{macrocode}
%<*package>
%    \end{macrocode}
%    Reload check, especially if the package is not used with \LaTeX.
%    \begin{macrocode}
\begingroup\catcode61\catcode48\catcode32=10\relax%
  \catcode13=5 % ^^M
  \endlinechar=13 %
  \catcode35=6 % #
  \catcode39=12 % '
  \catcode44=12 % ,
  \catcode45=12 % -
  \catcode46=12 % .
  \catcode58=12 % :
  \catcode64=11 % @
  \catcode123=1 % {
  \catcode125=2 % }
  \expandafter\let\expandafter\x\csname ver@hologo.sty\endcsname
  \ifx\x\relax % plain-TeX, first loading
  \else
    \def\empty{}%
    \ifx\x\empty % LaTeX, first loading,
      % variable is initialized, but \ProvidesPackage not yet seen
    \else
      \expandafter\ifx\csname PackageInfo\endcsname\relax
        \def\x#1#2{%
          \immediate\write-1{Package #1 Info: #2.}%
        }%
      \else
        \def\x#1#2{\PackageInfo{#1}{#2, stopped}}%
      \fi
      \x{hologo}{The package is already loaded}%
      \aftergroup\endinput
    \fi
  \fi
\endgroup%
%    \end{macrocode}
%    Package identification:
%    \begin{macrocode}
\begingroup\catcode61\catcode48\catcode32=10\relax%
  \catcode13=5 % ^^M
  \endlinechar=13 %
  \catcode35=6 % #
  \catcode39=12 % '
  \catcode40=12 % (
  \catcode41=12 % )
  \catcode44=12 % ,
  \catcode45=12 % -
  \catcode46=12 % .
  \catcode47=12 % /
  \catcode58=12 % :
  \catcode64=11 % @
  \catcode91=12 % [
  \catcode93=12 % ]
  \catcode123=1 % {
  \catcode125=2 % }
  \expandafter\ifx\csname ProvidesPackage\endcsname\relax
    \def\x#1#2#3[#4]{\endgroup
      \immediate\write-1{Package: #3 #4}%
      \xdef#1{#4}%
    }%
  \else
    \def\x#1#2[#3]{\endgroup
      #2[{#3}]%
      \ifx#1\@undefined
        \xdef#1{#3}%
      \fi
      \ifx#1\relax
        \xdef#1{#3}%
      \fi
    }%
  \fi
\expandafter\x\csname ver@hologo.sty\endcsname
\ProvidesPackage{hologo}%
  [2016/05/12 v1.11 A logo collection with bookmark support (HO)]%
%    \end{macrocode}
%
%    \begin{macrocode}
\begingroup\catcode61\catcode48\catcode32=10\relax%
  \catcode13=5 % ^^M
  \endlinechar=13 %
  \catcode123=1 % {
  \catcode125=2 % }
  \catcode64=11 % @
  \def\x{\endgroup
    \expandafter\edef\csname HOLOGO@AtEnd\endcsname{%
      \endlinechar=\the\endlinechar\relax
      \catcode13=\the\catcode13\relax
      \catcode32=\the\catcode32\relax
      \catcode35=\the\catcode35\relax
      \catcode61=\the\catcode61\relax
      \catcode64=\the\catcode64\relax
      \catcode123=\the\catcode123\relax
      \catcode125=\the\catcode125\relax
    }%
  }%
\x\catcode61\catcode48\catcode32=10\relax%
\catcode13=5 % ^^M
\endlinechar=13 %
\catcode35=6 % #
\catcode64=11 % @
\catcode123=1 % {
\catcode125=2 % }
\def\TMP@EnsureCode#1#2{%
  \edef\HOLOGO@AtEnd{%
    \HOLOGO@AtEnd
    \catcode#1=\the\catcode#1\relax
  }%
  \catcode#1=#2\relax
}
\TMP@EnsureCode{10}{12}% ^^J
\TMP@EnsureCode{33}{12}% !
\TMP@EnsureCode{34}{12}% "
\TMP@EnsureCode{36}{3}% $
\TMP@EnsureCode{38}{4}% &
\TMP@EnsureCode{39}{12}% '
\TMP@EnsureCode{40}{12}% (
\TMP@EnsureCode{41}{12}% )
\TMP@EnsureCode{42}{12}% *
\TMP@EnsureCode{43}{12}% +
\TMP@EnsureCode{44}{12}% ,
\TMP@EnsureCode{45}{12}% -
\TMP@EnsureCode{46}{12}% .
\TMP@EnsureCode{47}{12}% /
\TMP@EnsureCode{58}{12}% :
\TMP@EnsureCode{59}{12}% ;
\TMP@EnsureCode{60}{12}% <
\TMP@EnsureCode{62}{12}% >
\TMP@EnsureCode{63}{12}% ?
\TMP@EnsureCode{91}{12}% [
\TMP@EnsureCode{93}{12}% ]
\TMP@EnsureCode{94}{7}% ^ (superscript)
\TMP@EnsureCode{95}{8}% _ (subscript)
\TMP@EnsureCode{96}{12}% `
\TMP@EnsureCode{124}{12}% |
\edef\HOLOGO@AtEnd{%
  \HOLOGO@AtEnd
  \escapechar\the\escapechar\relax
  \noexpand\endinput
}
\escapechar=92 %
%    \end{macrocode}
%
% \subsection{Logo list}
%
%    \begin{macro}{\hologoList}
%    \begin{macrocode}
\def\hologoList{%
  \hologoEntry{(La)TeX}{}{2011/10/01}%
  \hologoEntry{AmSLaTeX}{}{2010/04/16}%
  \hologoEntry{AmSTeX}{}{2010/04/16}%
  \hologoEntry{biber}{}{2011/10/01}%
  \hologoEntry{BibTeX}{}{2011/10/01}%
  \hologoEntry{BibTeX}{sf}{2011/10/01}%
  \hologoEntry{BibTeX}{sc}{2011/10/01}%
  \hologoEntry{BibTeX8}{}{2011/11/22}%
  \hologoEntry{ConTeXt}{}{2011/03/25}%
  \hologoEntry{ConTeXt}{narrow}{2011/03/25}%
  \hologoEntry{ConTeXt}{simple}{2011/03/25}%
  \hologoEntry{emTeX}{}{2010/04/26}%
  \hologoEntry{eTeX}{}{2010/04/08}%
  \hologoEntry{ExTeX}{}{2011/10/01}%
  \hologoEntry{HanTheThanh}{}{2011/11/29}%
  \hologoEntry{iniTeX}{}{2011/10/01}%
  \hologoEntry{KOMAScript}{}{2011/10/01}%
  \hologoEntry{La}{}{2010/05/08}%
  \hologoEntry{LaTeX}{}{2010/04/08}%
  \hologoEntry{LaTeX2e}{}{2010/04/08}%
  \hologoEntry{LaTeX3}{}{2010/04/24}%
  \hologoEntry{LaTeXe}{}{2010/04/08}%
  \hologoEntry{LaTeXML}{}{2011/11/22}%
  \hologoEntry{LaTeXTeX}{}{2011/10/01}%
  \hologoEntry{LuaLaTeX}{}{2010/04/08}%
  \hologoEntry{LuaTeX}{}{2010/04/08}%
  \hologoEntry{LyX}{}{2011/10/01}%
  \hologoEntry{METAFONT}{}{2011/10/01}%
  \hologoEntry{MetaFun}{}{2011/10/01}%
  \hologoEntry{METAPOST}{}{2011/10/01}%
  \hologoEntry{MetaPost}{}{2011/10/01}%
  \hologoEntry{MiKTeX}{}{2011/10/01}%
  \hologoEntry{NTS}{}{2011/10/01}%
  \hologoEntry{OzMF}{}{2011/10/01}%
  \hologoEntry{OzMP}{}{2011/10/01}%
  \hologoEntry{OzTeX}{}{2011/10/01}%
  \hologoEntry{OzTtH}{}{2011/10/01}%
  \hologoEntry{PCTeX}{}{2011/10/01}%
  \hologoEntry{pdfTeX}{}{2011/10/01}%
  \hologoEntry{pdfLaTeX}{}{2011/10/01}%
  \hologoEntry{PiC}{}{2011/10/01}%
  \hologoEntry{PiCTeX}{}{2011/10/01}%
  \hologoEntry{plainTeX}{}{2010/04/08}%
  \hologoEntry{plainTeX}{space}{2010/04/16}%
  \hologoEntry{plainTeX}{hyphen}{2010/04/16}%
  \hologoEntry{plainTeX}{runtogether}{2010/04/16}%
  \hologoEntry{SageTeX}{}{2011/11/22}%
  \hologoEntry{SLiTeX}{}{2011/10/01}%
  \hologoEntry{SLiTeX}{lift}{2011/10/01}%
  \hologoEntry{SLiTeX}{narrow}{2011/10/01}%
  \hologoEntry{SLiTeX}{simple}{2011/10/01}%
  \hologoEntry{SliTeX}{}{2011/10/01}%
  \hologoEntry{SliTeX}{narrow}{2011/10/01}%
  \hologoEntry{SliTeX}{simple}{2011/10/01}%
  \hologoEntry{SliTeX}{lift}{2011/10/01}%
  \hologoEntry{teTeX}{}{2011/10/01}%
  \hologoEntry{TeX}{}{2010/04/08}%
  \hologoEntry{TeX4ht}{}{2011/11/22}%
  \hologoEntry{TTH}{}{2011/11/22}%
  \hologoEntry{virTeX}{}{2011/10/01}%
  \hologoEntry{VTeX}{}{2010/04/24}%
  \hologoEntry{Xe}{}{2010/04/08}%
  \hologoEntry{XeLaTeX}{}{2010/04/08}%
  \hologoEntry{XeTeX}{}{2010/04/08}%
}
%    \end{macrocode}
%    \end{macro}
%
% \subsection{Load resources}
%
%    \begin{macrocode}
\begingroup\expandafter\expandafter\expandafter\endgroup
\expandafter\ifx\csname RequirePackage\endcsname\relax
  \def\TMP@RequirePackage#1[#2]{%
    \begingroup\expandafter\expandafter\expandafter\endgroup
    \expandafter\ifx\csname ver@#1.sty\endcsname\relax
      \input #1.sty\relax
    \fi
  }%
  \TMP@RequirePackage{ltxcmds}[2011/02/04]%
  \TMP@RequirePackage{infwarerr}[2010/04/08]%
  \TMP@RequirePackage{kvsetkeys}[2010/03/01]%
  \TMP@RequirePackage{kvdefinekeys}[2010/03/01]%
  \TMP@RequirePackage{pdftexcmds}[2010/04/01]%
  \TMP@RequirePackage{ifpdf}[2010/01/28]%
  \TMP@RequirePackage{ifluatex}[2010/03/01]%
  \ltx@IfUndefined{newif}{%
    \expandafter\let\csname newif\endcsname\ltx@newif
  }{}%
  \TMP@RequirePackage{ifxetex}[2009/01/23]%
  \TMP@RequirePackage{ifvtex}[2010/03/01]%
\else
  \RequirePackage{ltxcmds}[2011/02/04]%
  \RequirePackage{infwarerr}[2010/04/08]%
  \RequirePackage{kvsetkeys}[2010/03/01]%
  \RequirePackage{kvdefinekeys}[2010/03/01]%
  \RequirePackage{pdftexcmds}[2010/04/01]%
  \RequirePackage{ifpdf}[2010/01/28]%
  \RequirePackage{ifluatex}[2010/03/01]%
  \RequirePackage{ifxetex}[2009/01/23]%
  \RequirePackage{ifvtex}[2010/03/01]%
\fi
%    \end{macrocode}
%
%    \begin{macro}{\HOLOGO@IfDefined}
%    \begin{macrocode}
\def\HOLOGO@IfExists#1{%
  \ifx\@undefined#1%
    \expandafter\ltx@secondoftwo
  \else
    \ifx\relax#1%
      \expandafter\ltx@secondoftwo
    \else
      \expandafter\expandafter\expandafter\ltx@firstoftwo
    \fi
  \fi
}
%    \end{macrocode}
%    \end{macro}
%
% \subsection{Setup macros}
%
%    \begin{macro}{\hologoSetup}
%    \begin{macrocode}
\def\hologoSetup{%
  \let\HOLOGO@name\relax
  \HOLOGO@Setup
}
%    \end{macrocode}
%    \end{macro}
%
%    \begin{macro}{\hologoLogoSetup}
%    \begin{macrocode}
\def\hologoLogoSetup#1{%
  \edef\HOLOGO@name{#1}%
  \ltx@IfUndefined{HoLogo@\HOLOGO@name}{%
    \@PackageError{hologo}{%
      Unknown logo `\HOLOGO@name'%
    }\@ehc
    \ltx@gobble
  }{%
    \HOLOGO@Setup
  }%
}
%    \end{macrocode}
%    \end{macro}
%
%    \begin{macro}{\HOLOGO@Setup}
%    \begin{macrocode}
\def\HOLOGO@Setup{%
  \kvsetkeys{HoLogo}%
}
%    \end{macrocode}
%    \end{macro}
%
% \subsection{Options}
%
%    \begin{macro}{\HOLOGO@DeclareBoolOption}
%    \begin{macrocode}
\def\HOLOGO@DeclareBoolOption#1{%
  \expandafter\chardef\csname HOLOGOOPT@#1\endcsname\ltx@zero
  \kv@define@key{HoLogo}{#1}[true]{%
    \def\HOLOGO@temp{##1}%
    \ifx\HOLOGO@temp\HOLOGO@true
      \ifx\HOLOGO@name\relax
        \expandafter\chardef\csname HOLOGOOPT@#1\endcsname=\ltx@one
      \else
        \expandafter\chardef\csname
        HoLogoOpt@#1@\HOLOGO@name\endcsname\ltx@one
      \fi
      \HOLOGO@SetBreakAll{#1}%
    \else
      \ifx\HOLOGO@temp\HOLOGO@false
        \ifx\HOLOGO@name\relax
          \expandafter\chardef\csname HOLOGOOPT@#1\endcsname=\ltx@zero
        \else
          \expandafter\chardef\csname
          HoLogoOpt@#1@\HOLOGO@name\endcsname=\ltx@zero
        \fi
        \HOLOGO@SetBreakAll{#1}%
      \else
        \@PackageError{hologo}{%
          Unknown value `##1' for boolean option `#1'.\MessageBreak
          Known values are `true' and `false'%
        }\@ehc
      \fi
    \fi
  }%
}
%    \end{macrocode}
%    \end{macro}
%
%    \begin{macro}{\HOLOGO@SetBreakAll}
%    \begin{macrocode}
\def\HOLOGO@SetBreakAll#1{%
  \def\HOLOGO@temp{#1}%
  \ifx\HOLOGO@temp\HOLOGO@break
    \ifx\HOLOGO@name\relax
      \chardef\HOLOGOOPT@hyphenbreak=\HOLOGOOPT@break
      \chardef\HOLOGOOPT@spacebreak=\HOLOGOOPT@break
      \chardef\HOLOGOOPT@discretionarybreak=\HOLOGOOPT@break
    \else
      \expandafter\chardef
         \csname HoLogoOpt@hyphenbreak@\HOLOGO@name\endcsname=%
         \csname HoLogoOpt@break@\HOLOGO@name\endcsname
      \expandafter\chardef
         \csname HoLogoOpt@spacebreak@\HOLOGO@name\endcsname=%
         \csname HoLogoOpt@break@\HOLOGO@name\endcsname
      \expandafter\chardef
         \csname HoLogoOpt@discretionarybreak@\HOLOGO@name
             \endcsname=%
         \csname HoLogoOpt@break@\HOLOGO@name\endcsname
    \fi
  \fi
}
%    \end{macrocode}
%    \end{macro}
%
%    \begin{macro}{\HOLOGO@true}
%    \begin{macrocode}
\def\HOLOGO@true{true}
%    \end{macrocode}
%    \end{macro}
%    \begin{macro}{\HOLOGO@false}
%    \begin{macrocode}
\def\HOLOGO@false{false}
%    \end{macrocode}
%    \end{macro}
%    \begin{macro}{\HOLOGO@break}
%    \begin{macrocode}
\def\HOLOGO@break{break}
%    \end{macrocode}
%    \end{macro}
%
%    \begin{macrocode}
\HOLOGO@DeclareBoolOption{break}
\HOLOGO@DeclareBoolOption{hyphenbreak}
\HOLOGO@DeclareBoolOption{spacebreak}
\HOLOGO@DeclareBoolOption{discretionarybreak}
%    \end{macrocode}
%
%    \begin{macrocode}
\kv@define@key{HoLogo}{variant}{%
  \ifx\HOLOGO@name\relax
    \@PackageError{hologo}{%
      Option `variant' is not available in \string\hologoSetup,%
      \MessageBreak
      Use \string\hologoLogoSetup\space instead%
    }\@ehc
  \else
    \edef\HOLOGO@temp{#1}%
    \ifx\HOLOGO@temp\ltx@empty
      \expandafter
      \let\csname HoLogoOpt@variant@\HOLOGO@name\endcsname\@undefined
    \else
      \ltx@IfUndefined{HoLogo@\HOLOGO@name @\HOLOGO@temp}{%
        \@PackageError{hologo}{%
          Unknown variant `\HOLOGO@temp' of logo `\HOLOGO@name'%
        }\@ehc
      }{%
        \expandafter
        \let\csname HoLogoOpt@variant@\HOLOGO@name\endcsname
            \HOLOGO@temp
      }%
    \fi
  \fi
}
%    \end{macrocode}
%
%    \begin{macro}{\HOLOGO@Variant}
%    \begin{macrocode}
\def\HOLOGO@Variant#1{%
  #1%
  \ltx@ifundefined{HoLogoOpt@variant@#1}{%
  }{%
    @\csname HoLogoOpt@variant@#1\endcsname
  }%
}
%    \end{macrocode}
%    \end{macro}
%
% \subsection{Break/no-break support}
%
%    \begin{macro}{\HOLOGO@space}
%    \begin{macrocode}
\def\HOLOGO@space{%
  \ltx@ifundefined{HoLogoOpt@spacebreak@\HOLOGO@name}{%
    \ltx@ifundefined{HoLogoOpt@break@\HOLOGO@name}{%
      \chardef\HOLOGO@temp=\HOLOGOOPT@spacebreak
    }{%
      \chardef\HOLOGO@temp=%
        \csname HoLogoOpt@break@\HOLOGO@name\endcsname
    }%
  }{%
    \chardef\HOLOGO@temp=%
      \csname HoLogoOpt@spacebreak@\HOLOGO@name\endcsname
  }%
  \ifcase\HOLOGO@temp
    \penalty10000 %
  \fi
  \ltx@space
}
%    \end{macrocode}
%    \end{macro}
%
%    \begin{macro}{\HOLOGO@hyphen}
%    \begin{macrocode}
\def\HOLOGO@hyphen{%
  \ltx@ifundefined{HoLogoOpt@hyphenbreak@\HOLOGO@name}{%
    \ltx@ifundefined{HoLogoOpt@break@\HOLOGO@name}{%
      \chardef\HOLOGO@temp=\HOLOGOOPT@hyphenbreak
    }{%
      \chardef\HOLOGO@temp=%
        \csname HoLogoOpt@break@\HOLOGO@name\endcsname
    }%
  }{%
    \chardef\HOLOGO@temp=%
      \csname HoLogoOpt@hyphenbreak@\HOLOGO@name\endcsname
  }%
  \ifcase\HOLOGO@temp
    \ltx@mbox{-}%
  \else
    -%
  \fi
}
%    \end{macrocode}
%    \end{macro}
%
%    \begin{macro}{\HOLOGO@discretionary}
%    \begin{macrocode}
\def\HOLOGO@discretionary{%
  \ltx@ifundefined{HoLogoOpt@discretionarybreak@\HOLOGO@name}{%
    \ltx@ifundefined{HoLogoOpt@break@\HOLOGO@name}{%
      \chardef\HOLOGO@temp=\HOLOGOOPT@discretionarybreak
    }{%
      \chardef\HOLOGO@temp=%
        \csname HoLogoOpt@break@\HOLOGO@name\endcsname
    }%
  }{%
    \chardef\HOLOGO@temp=%
      \csname HoLogoOpt@discretionarybreak@\HOLOGO@name\endcsname
  }%
  \ifcase\HOLOGO@temp
  \else
    \-%
  \fi
}
%    \end{macrocode}
%    \end{macro}
%
%    \begin{macro}{\HOLOGO@mbox}
%    \begin{macrocode}
\def\HOLOGO@mbox#1{%
  \ltx@ifundefined{HoLogoOpt@break@\HOLOGO@name}{%
    \chardef\HOLOGO@temp=\HOLOGOOPT@hyphenbreak
  }{%
    \chardef\HOLOGO@temp=%
      \csname HoLogoOpt@break@\HOLOGO@name\endcsname
  }%
  \ifcase\HOLOGO@temp
    \ltx@mbox{#1}%
  \else
    #1%
  \fi
}
%    \end{macrocode}
%    \end{macro}
%
% \subsection{Font support}
%
%    \begin{macro}{\HoLogoFont@font}
%    \begin{tabular}{@{}ll@{}}
%    |#1|:& logo name\\
%    |#2|:& font short name\\
%    |#3|:& text
%    \end{tabular}
%    \begin{macrocode}
\def\HoLogoFont@font#1#2#3{%
  \begingroup
    \ltx@IfUndefined{HoLogoFont@logo@#1.#2}{%
      \ltx@IfUndefined{HoLogoFont@font@#2}{%
        \@PackageWarning{hologo}{%
          Missing font `#2' for logo `#1'%
        }%
        #3%
      }{%
        \csname HoLogoFont@font@#2\endcsname{#3}%
      }%
    }{%
      \csname HoLogoFont@logo@#1.#2\endcsname{#3}%
    }%
  \endgroup
}
%    \end{macrocode}
%    \end{macro}
%
%    \begin{macro}{\HoLogoFont@Def}
%    \begin{macrocode}
\def\HoLogoFont@Def#1{%
  \expandafter\def\csname HoLogoFont@font@#1\endcsname
}
%    \end{macrocode}
%    \end{macro}
%    \begin{macro}{\HoLogoFont@LogoDef}
%    \begin{macrocode}
\def\HoLogoFont@LogoDef#1#2{%
  \expandafter\def\csname HoLogoFont@logo@#1.#2\endcsname
}
%    \end{macrocode}
%    \end{macro}
%
% \subsubsection{Font defaults}
%
%    \begin{macro}{\HoLogoFont@font@general}
%    \begin{macrocode}
\HoLogoFont@Def{general}{}%
%    \end{macrocode}
%    \end{macro}
%
%    \begin{macro}{\HoLogoFont@font@rm}
%    \begin{macrocode}
\ltx@IfUndefined{rmfamily}{%
  \ltx@IfUndefined{rm}{%
  }{%
    \HoLogoFont@Def{rm}{\rm}%
  }%
}{%
  \HoLogoFont@Def{rm}{\rmfamily}%
}
%    \end{macrocode}
%    \end{macro}
%
%    \begin{macro}{\HoLogoFont@font@sf}
%    \begin{macrocode}
\ltx@IfUndefined{sffamily}{%
  \ltx@IfUndefined{sf}{%
  }{%
    \HoLogoFont@Def{sf}{\sf}%
  }%
}{%
  \HoLogoFont@Def{sf}{\sffamily}%
}
%    \end{macrocode}
%    \end{macro}
%
%    \begin{macro}{\HoLogoFont@font@bibsf}
%    In case of \hologo{plainTeX} the original small caps
%    variant is used as default. In \hologo{LaTeX}
%    the definition of package \xpackage{dtklogos} \cite{dtklogos}
%    is used.
%\begin{quote}
%\begin{verbatim}
%\DeclareRobustCommand{\BibTeX}{%
%  B%
%  \kern-.05em%
%  \hbox{%
%    $\m@th$% %% force math size calculations
%    \csname S@\f@size\endcsname
%    \fontsize\sf@size\z@
%    \math@fontsfalse
%    \selectfont
%    I%
%    \kern-.025em%
%    B
%  }%
%  \kern-.08em%
%  \-%
%  \TeX
%}
%\end{verbatim}
%\end{quote}
%    \begin{macrocode}
\ltx@IfUndefined{selectfont}{%
  \ltx@IfUndefined{tensc}{%
    \font\tensc=cmcsc10\relax
  }{}%
  \HoLogoFont@Def{bibsf}{\tensc}%
}{%
  \HoLogoFont@Def{bibsf}{%
    $\mathsurround=0pt$%
    \csname S@\f@size\endcsname
    \fontsize\sf@size{0pt}%
    \math@fontsfalse
    \selectfont
  }%
}
%    \end{macrocode}
%    \end{macro}
%
%    \begin{macro}{\HoLogoFont@font@sc}
%    \begin{macrocode}
\ltx@IfUndefined{scshape}{%
  \ltx@IfUndefined{tensc}{%
    \font\tensc=cmcsc10\relax
  }{}%
  \HoLogoFont@Def{sc}{\tensc}%
}{%
  \HoLogoFont@Def{sc}{\scshape}%
}
%    \end{macrocode}
%    \end{macro}
%
%    \begin{macro}{\HoLogoFont@font@sy}
%    \begin{macrocode}
\ltx@IfUndefined{usefont}{%
  \ltx@IfUndefined{tensy}{%
  }{%
    \HoLogoFont@Def{sy}{\tensy}%
  }%
}{%
  \HoLogoFont@Def{sy}{%
    \usefont{OMS}{cmsy}{m}{n}%
  }%
}
%    \end{macrocode}
%    \end{macro}
%
%    \begin{macro}{\HoLogoFont@font@logo}
%    \begin{macrocode}
\begingroup
  \def\x{LaTeX2e}%
\expandafter\endgroup
\ifx\fmtname\x
  \ltx@IfUndefined{logofamily}{%
    \DeclareRobustCommand\logofamily{%
      \not@math@alphabet\logofamily\relax
      \fontencoding{U}%
      \fontfamily{logo}%
      \selectfont
    }%
  }{}%
  \ltx@IfUndefined{logofamily}{%
  }{%
    \HoLogoFont@Def{logo}{\logofamily}%
  }%
\else
  \ltx@IfUndefined{tenlogo}{%
    \font\tenlogo=logo10\relax
  }{}%
  \HoLogoFont@Def{logo}{\tenlogo}%
\fi
%    \end{macrocode}
%    \end{macro}
%
% \subsubsection{Font setup}
%
%    \begin{macro}{\hologoFontSetup}
%    \begin{macrocode}
\def\hologoFontSetup{%
  \let\HOLOGO@name\relax
  \HOLOGO@FontSetup
}
%    \end{macrocode}
%    \end{macro}
%
%    \begin{macro}{\hologoLogoFontSetup}
%    \begin{macrocode}
\def\hologoLogoFontSetup#1{%
  \edef\HOLOGO@name{#1}%
  \ltx@IfUndefined{HoLogo@\HOLOGO@name}{%
    \@PackageError{hologo}{%
      Unknown logo `\HOLOGO@name'%
    }\@ehc
    \ltx@gobble
  }{%
    \HOLOGO@FontSetup
  }%
}
%    \end{macrocode}
%    \end{macro}
%
%    \begin{macro}{\HOLOGO@FontSetup}
%    \begin{macrocode}
\def\HOLOGO@FontSetup{%
  \kvsetkeys{HoLogoFont}%
}
%    \end{macrocode}
%    \end{macro}
%
%    \begin{macrocode}
\def\HOLOGO@temp#1{%
  \kv@define@key{HoLogoFont}{#1}{%
    \ifx\HOLOGO@name\relax
      \HoLogoFont@Def{#1}{##1}%
    \else
      \HoLogoFont@LogoDef\HOLOGO@name{#1}{##1}%
    \fi
  }%
}
\HOLOGO@temp{general}
\HOLOGO@temp{sf}
%    \end{macrocode}
%
% \subsection{Generic logo commands}
%
%    \begin{macrocode}
\HOLOGO@IfExists\hologo{%
  \@PackageError{hologo}{%
    \string\hologo\ltx@space is already defined.\MessageBreak
    Package loading is aborted%
  }\@ehc
  \HOLOGO@AtEnd
}%
\HOLOGO@IfExists\hologoRobust{%
  \@PackageError{hologo}{%
    \string\hologoRobust\ltx@space is already defined.\MessageBreak
    Package loading is aborted%
  }\@ehc
  \HOLOGO@AtEnd
}%
%    \end{macrocode}
%
% \subsubsection{\cs{hologo} and friends}
%
%    \begin{macrocode}
\ifluatex
  \expandafter\ltx@firstofone
\else
  \expandafter\ltx@gobble
\fi
{%
  \ltx@IfUndefined{ifincsname}{%
    \ifnum\luatexversion<36 %
      \expandafter\ltx@gobble
    \else
      \expandafter\ltx@firstofone
    \fi
    {%
      \begingroup
        \ifcase0%
            \directlua{%
              if tex.enableprimitives then %
                tex.enableprimitives('HOLOGO@', {'ifincsname'})%
              else %
                tex.print('1')%
              end%
            }%
            \ifx\HOLOGO@ifincsname\@undefined 1\fi%
            \relax
          \expandafter\ltx@firstofone
        \else
          \endgroup
          \expandafter\ltx@gobble
        \fi
        {%
          \global\let\ifincsname\HOLOGO@ifincsname
        }%
      \HOLOGO@temp
    }%
  }{}%
}
%    \end{macrocode}
%    \begin{macrocode}
\ltx@IfUndefined{ifincsname}{%
  \catcode`$=14 %
}{%
  \catcode`$=9 %
}
%    \end{macrocode}
%
%    \begin{macro}{\hologo}
%    \begin{macrocode}
\def\hologo#1{%
$ \ifincsname
$   \ltx@ifundefined{HoLogoCs@\HOLOGO@Variant{#1}}{%
$     #1%
$   }{%
$     \csname HoLogoCs@\HOLOGO@Variant{#1}\endcsname\ltx@firstoftwo
$   }%
$ \else
    \HOLOGO@IfExists\texorpdfstring\texorpdfstring\ltx@firstoftwo
    {%
      \hologoRobust{#1}%
    }{%
      \ltx@ifundefined{HoLogoBkm@\HOLOGO@Variant{#1}}{%
        \ltx@ifundefined{HoLogo@#1}{?#1?}{#1}%
      }{%
        \csname HoLogoBkm@\HOLOGO@Variant{#1}\endcsname
        \ltx@firstoftwo
      }%
    }%
$ \fi
}
%    \end{macrocode}
%    \end{macro}
%    \begin{macro}{\Hologo}
%    \begin{macrocode}
\def\Hologo#1{%
$ \ifincsname
$   \ltx@ifundefined{HoLogoCs@\HOLOGO@Variant{#1}}{%
$     #1%
$   }{%
$     \csname HoLogoCs@\HOLOGO@Variant{#1}\endcsname\ltx@secondoftwo
$   }%
$ \else
    \HOLOGO@IfExists\texorpdfstring\texorpdfstring\ltx@firstoftwo
    {%
      \HologoRobust{#1}%
    }{%
      \ltx@ifundefined{HoLogoBkm@\HOLOGO@Variant{#1}}{%
        \ltx@ifundefined{HoLogo@#1}{?#1?}{#1}%
      }{%
        \csname HoLogoBkm@\HOLOGO@Variant{#1}\endcsname
        \ltx@secondoftwo
      }%
    }%
$ \fi
}
%    \end{macrocode}
%    \end{macro}
%
%    \begin{macro}{\hologoVariant}
%    \begin{macrocode}
\def\hologoVariant#1#2{%
  \ifx\relax#2\relax
    \hologo{#1}%
  \else
$   \ifincsname
$     \ltx@ifundefined{HoLogoCs@#1@#2}{%
$       #1%
$     }{%
$       \csname HoLogoCs@#1@#2\endcsname\ltx@firstoftwo
$     }%
$   \else
      \HOLOGO@IfExists\texorpdfstring\texorpdfstring\ltx@firstoftwo
      {%
        \hologoVariantRobust{#1}{#2}%
      }{%
        \ltx@ifundefined{HoLogoBkm@#1@#2}{%
          \ltx@ifundefined{HoLogo@#1}{?#1?}{#1}%
        }{%
          \csname HoLogoBkm@#1@#2\endcsname
          \ltx@firstoftwo
        }%
      }%
$   \fi
  \fi
}
%    \end{macrocode}
%    \end{macro}
%    \begin{macro}{\HologoVariant}
%    \begin{macrocode}
\def\HologoVariant#1#2{%
  \ifx\relax#2\relax
    \Hologo{#1}%
  \else
$   \ifincsname
$     \ltx@ifundefined{HoLogoCs@#1@#2}{%
$       #1%
$     }{%
$       \csname HoLogoCs@#1@#2\endcsname\ltx@secondoftwo
$     }%
$   \else
      \HOLOGO@IfExists\texorpdfstring\texorpdfstring\ltx@firstoftwo
      {%
        \HologoVariantRobust{#1}{#2}%
      }{%
        \ltx@ifundefined{HoLogoBkm@#1@#2}{%
          \ltx@ifundefined{HoLogo@#1}{?#1?}{#1}%
        }{%
          \csname HoLogoBkm@#1@#2\endcsname
          \ltx@secondoftwo
        }%
      }%
$   \fi
  \fi
}
%    \end{macrocode}
%    \end{macro}
%
%    \begin{macrocode}
\catcode`\$=3 %
%    \end{macrocode}
%
% \subsubsection{\cs{hologoRobust} and friends}
%
%    \begin{macro}{\hologoRobust}
%    \begin{macrocode}
\ltx@IfUndefined{protected}{%
  \ltx@IfUndefined{DeclareRobustCommand}{%
    \def\hologoRobust#1%
  }{%
    \DeclareRobustCommand*\hologoRobust[1]%
  }%
}{%
  \protected\def\hologoRobust#1%
}%
{%
  \edef\HOLOGO@name{#1}%
  \ltx@IfUndefined{HoLogo@\HOLOGO@Variant\HOLOGO@name}{%
    \@PackageError{hologo}{%
      Unknown logo `\HOLOGO@name'%
    }\@ehc
    ?\HOLOGO@name?%
  }{%
    \ltx@IfUndefined{ver@tex4ht.sty}{%
      \HoLogoFont@font\HOLOGO@name{general}{%
        \csname HoLogo@\HOLOGO@Variant\HOLOGO@name\endcsname
        \ltx@firstoftwo
      }%
    }{%
      \ltx@IfUndefined{HoLogoHtml@\HOLOGO@Variant\HOLOGO@name}{%
        \HOLOGO@name
      }{%
        \csname HoLogoHtml@\HOLOGO@Variant\HOLOGO@name\endcsname
        \ltx@firstoftwo
      }%
    }%
  }%
}
%    \end{macrocode}
%    \end{macro}
%    \begin{macro}{\HologoRobust}
%    \begin{macrocode}
\ltx@IfUndefined{protected}{%
  \ltx@IfUndefined{DeclareRobustCommand}{%
    \def\HologoRobust#1%
  }{%
    \DeclareRobustCommand*\HologoRobust[1]%
  }%
}{%
  \protected\def\HologoRobust#1%
}%
{%
  \edef\HOLOGO@name{#1}%
  \ltx@IfUndefined{HoLogo@\HOLOGO@Variant\HOLOGO@name}{%
    \@PackageError{hologo}{%
      Unknown logo `\HOLOGO@name'%
    }\@ehc
    ?\HOLOGO@name?%
  }{%
    \ltx@IfUndefined{ver@tex4ht.sty}{%
      \HoLogoFont@font\HOLOGO@name{general}{%
        \csname HoLogo@\HOLOGO@Variant\HOLOGO@name\endcsname
        \ltx@secondoftwo
      }%
    }{%
      \ltx@IfUndefined{HoLogoHtml@\HOLOGO@Variant\HOLOGO@name}{%
        \expandafter\HOLOGO@Uppercase\HOLOGO@name
      }{%
        \csname HoLogoHtml@\HOLOGO@Variant\HOLOGO@name\endcsname
        \ltx@secondoftwo
      }%
    }%
  }%
}
%    \end{macrocode}
%    \end{macro}
%    \begin{macro}{\hologoVariantRobust}
%    \begin{macrocode}
\ltx@IfUndefined{protected}{%
  \ltx@IfUndefined{DeclareRobustCommand}{%
    \def\hologoVariantRobust#1#2%
  }{%
    \DeclareRobustCommand*\hologoVariantRobust[2]%
  }%
}{%
  \protected\def\hologoVariantRobust#1#2%
}%
{%
  \begingroup
    \hologoLogoSetup{#1}{variant={#2}}%
    \hologoRobust{#1}%
  \endgroup
}
%    \end{macrocode}
%    \end{macro}
%    \begin{macro}{\HologoVariantRobust}
%    \begin{macrocode}
\ltx@IfUndefined{protected}{%
  \ltx@IfUndefined{DeclareRobustCommand}{%
    \def\HologoVariantRobust#1#2%
  }{%
    \DeclareRobustCommand*\HologoVariantRobust[2]%
  }%
}{%
  \protected\def\HologoVariantRobust#1#2%
}%
{%
  \begingroup
    \hologoLogoSetup{#1}{variant={#2}}%
    \HologoRobust{#1}%
  \endgroup
}
%    \end{macrocode}
%    \end{macro}
%
%    \begin{macro}{\hologorobust}
%    Macro \cs{hologorobust} is only defined for compatibility.
%    Its use is deprecated.
%    \begin{macrocode}
\def\hologorobust{\hologoRobust}
%    \end{macrocode}
%    \end{macro}
%
% \subsection{Helpers}
%
%    \begin{macro}{\HOLOGO@Uppercase}
%    Macro \cs{HOLOGO@Uppercase} is restricted to \cs{uppercase},
%    because \hologo{plainTeX} or \hologo{iniTeX} do not provide
%    \cs{MakeUppercase}.
%    \begin{macrocode}
\def\HOLOGO@Uppercase#1{\uppercase{#1}}
%    \end{macrocode}
%    \end{macro}
%
%    \begin{macro}{\HOLOGO@PdfdocUnicode}
%    \begin{macrocode}
\def\HOLOGO@PdfdocUnicode{%
  \ifx\ifHy@unicode\iftrue
    \expandafter\ltx@secondoftwo
  \else
    \expandafter\ltx@firstoftwo
  \fi
}
%    \end{macrocode}
%    \end{macro}
%
%    \begin{macro}{\HOLOGO@Math}
%    \begin{macrocode}
\def\HOLOGO@MathSetup{%
  \mathsurround0pt\relax
  \HOLOGO@IfExists\f@series{%
    \if b\expandafter\ltx@car\f@series x\@nil
      \csname boldmath\endcsname
   \fi
  }{}%
}
%    \end{macrocode}
%    \end{macro}
%
%    \begin{macro}{\HOLOGO@TempDimen}
%    \begin{macrocode}
\dimendef\HOLOGO@TempDimen=\ltx@zero
%    \end{macrocode}
%    \end{macro}
%    \begin{macro}{\HOLOGO@NegativeKerning}
%    \begin{macrocode}
\def\HOLOGO@NegativeKerning#1{%
  \begingroup
    \HOLOGO@TempDimen=0pt\relax
    \comma@parse@normalized{#1}{%
      \ifdim\HOLOGO@TempDimen=0pt %
        \expandafter\HOLOGO@@NegativeKerning\comma@entry
      \fi
      \ltx@gobble
    }%
    \ifdim\HOLOGO@TempDimen<0pt %
      \kern\HOLOGO@TempDimen
    \fi
  \endgroup
}
%    \end{macrocode}
%    \end{macro}
%    \begin{macro}{\HOLOGO@@NegativeKerning}
%    \begin{macrocode}
\def\HOLOGO@@NegativeKerning#1#2{%
  \setbox\ltx@zero\hbox{#1#2}%
  \HOLOGO@TempDimen=\wd\ltx@zero
  \setbox\ltx@zero\hbox{#1\kern0pt#2}%
  \advance\HOLOGO@TempDimen by -\wd\ltx@zero
}
%    \end{macrocode}
%    \end{macro}
%
%    \begin{macro}{\HOLOGO@SpaceFactor}
%    \begin{macrocode}
\def\HOLOGO@SpaceFactor{%
  \spacefactor1000 %
}
%    \end{macrocode}
%    \end{macro}
%
%    \begin{macro}{\HOLOGO@Span}
%    \begin{macrocode}
\def\HOLOGO@Span#1#2{%
  \HCode{<span class="HoLogo-#1">}%
  #2%
  \HCode{</span>}%
}
%    \end{macrocode}
%    \end{macro}
%
% \subsubsection{Text subscript}
%
%    \begin{macro}{\HOLOGO@SubScript}%
%    \begin{macrocode}
\def\HOLOGO@SubScript#1{%
  \ltx@IfUndefined{textsubscript}{%
    \ltx@IfUndefined{text}{%
      \ltx@mbox{%
        \mathsurround=0pt\relax
        $%
          _{%
            \ltx@IfUndefined{sf@size}{%
              \mathrm{#1}%
            }{%
              \mbox{%
                \fontsize\sf@size{0pt}\selectfont
                #1%
              }%
            }%
          }%
        $%
      }%
    }{%
      \ltx@mbox{%
        \mathsurround=0pt\relax
        $_{\text{#1}}$%
      }%
    }%
  }{%
    \textsubscript{#1}%
  }%
}
%    \end{macrocode}
%    \end{macro}
%
% \subsection{\hologo{TeX} and friends}
%
% \subsubsection{\hologo{TeX}}
%
%    \begin{macro}{\HoLogo@TeX}
%    Source: \hologo{LaTeX} kernel.
%    \begin{macrocode}
\def\HoLogo@TeX#1{%
  T\kern-.1667em\lower.5ex\hbox{E}\kern-.125emX\HOLOGO@SpaceFactor
}
%    \end{macrocode}
%    \end{macro}
%    \begin{macro}{\HoLogoHtml@TeX}
%    \begin{macrocode}
\def\HoLogoHtml@TeX#1{%
  \HoLogoCss@TeX
  \HOLOGO@Span{TeX}{%
    T%
    \HOLOGO@Span{e}{%
      E%
    }%
    X%
  }%
}
%    \end{macrocode}
%    \end{macro}
%    \begin{macro}{\HoLogoCss@TeX}
%    \begin{macrocode}
\def\HoLogoCss@TeX{%
  \Css{%
    span.HoLogo-TeX span.HoLogo-e{%
      position:relative;%
      top:.5ex;%
      margin-left:-.1667em;%
      margin-right:-.125em;%
    }%
  }%
  \Css{%
    a span.HoLogo-TeX span.HoLogo-e{%
      text-decoration:none;%
    }%
  }%
  \global\let\HoLogoCss@TeX\relax
}
%    \end{macrocode}
%    \end{macro}
%
% \subsubsection{\hologo{plainTeX}}
%
%    \begin{macro}{\HoLogo@plainTeX@space}
%    Source: ``The \hologo{TeX}book''
%    \begin{macrocode}
\def\HoLogo@plainTeX@space#1{%
  \HOLOGO@mbox{#1{p}{P}lain}\HOLOGO@space\hologo{TeX}%
}
%    \end{macrocode}
%    \end{macro}
%    \begin{macro}{\HoLogoCs@plainTeX@space}
%    \begin{macrocode}
\def\HoLogoCs@plainTeX@space#1{#1{p}{P}lain TeX}%
%    \end{macrocode}
%    \end{macro}
%    \begin{macro}{\HoLogoBkm@plainTeX@space}
%    \begin{macrocode}
\def\HoLogoBkm@plainTeX@space#1{%
  #1{p}{P}lain \hologo{TeX}%
}
%    \end{macrocode}
%    \end{macro}
%    \begin{macro}{\HoLogoHtml@plainTeX@space}
%    \begin{macrocode}
\def\HoLogoHtml@plainTeX@space#1{%
  #1{p}{P}lain \hologo{TeX}%
}
%    \end{macrocode}
%    \end{macro}
%
%    \begin{macro}{\HoLogo@plainTeX@hyphen}
%    \begin{macrocode}
\def\HoLogo@plainTeX@hyphen#1{%
  \HOLOGO@mbox{#1{p}{P}lain}\HOLOGO@hyphen\hologo{TeX}%
}
%    \end{macrocode}
%    \end{macro}
%    \begin{macro}{\HoLogoCs@plainTeX@hyphen}
%    \begin{macrocode}
\def\HoLogoCs@plainTeX@hyphen#1{#1{p}{P}lain-TeX}
%    \end{macrocode}
%    \end{macro}
%    \begin{macro}{\HoLogoBkm@plainTeX@hyphen}
%    \begin{macrocode}
\def\HoLogoBkm@plainTeX@hyphen#1{%
  #1{p}{P}lain-\hologo{TeX}%
}
%    \end{macrocode}
%    \end{macro}
%    \begin{macro}{\HoLogoHtml@plainTeX@hyphen}
%    \begin{macrocode}
\def\HoLogoHtml@plainTeX@hyphen#1{%
  #1{p}{P}lain-\hologo{TeX}%
}
%    \end{macrocode}
%    \end{macro}
%
%    \begin{macro}{\HoLogo@plainTeX@runtogether}
%    \begin{macrocode}
\def\HoLogo@plainTeX@runtogether#1{%
  \HOLOGO@mbox{#1{p}{P}lain\hologo{TeX}}%
}
%    \end{macrocode}
%    \end{macro}
%    \begin{macro}{\HoLogoCs@plainTeX@runtogether}
%    \begin{macrocode}
\def\HoLogoCs@plainTeX@runtogether#1{#1{p}{P}lainTeX}
%    \end{macrocode}
%    \end{macro}
%    \begin{macro}{\HoLogoBkm@plainTeX@runtogether}
%    \begin{macrocode}
\def\HoLogoBkm@plainTeX@runtogether#1{%
  #1{p}{P}lain\hologo{TeX}%
}
%    \end{macrocode}
%    \end{macro}
%    \begin{macro}{\HoLogoHtml@plainTeX@runtogether}
%    \begin{macrocode}
\def\HoLogoHtml@plainTeX@runtogether#1{%
  #1{p}{P}lain\hologo{TeX}%
}
%    \end{macrocode}
%    \end{macro}
%
%    \begin{macro}{\HoLogo@plainTeX}
%    \begin{macrocode}
\def\HoLogo@plainTeX{\HoLogo@plainTeX@space}
%    \end{macrocode}
%    \end{macro}
%    \begin{macro}{\HoLogoCs@plainTeX}
%    \begin{macrocode}
\def\HoLogoCs@plainTeX{\HoLogoCs@plainTeX@space}
%    \end{macrocode}
%    \end{macro}
%    \begin{macro}{\HoLogoBkm@plainTeX}
%    \begin{macrocode}
\def\HoLogoBkm@plainTeX{\HoLogoBkm@plainTeX@space}
%    \end{macrocode}
%    \end{macro}
%    \begin{macro}{\HoLogoHtml@plainTeX}
%    \begin{macrocode}
\def\HoLogoHtml@plainTeX{\HoLogoHtml@plainTeX@space}
%    \end{macrocode}
%    \end{macro}
%
% \subsubsection{\hologo{LaTeX}}
%
%    Source: \hologo{LaTeX} kernel.
%\begin{quote}
%\begin{verbatim}
%\DeclareRobustCommand{\LaTeX}{%
%  L%
%  \kern-.36em%
%  {%
%    \sbox\z@ T%
%    \vbox to\ht\z@{%
%      \hbox{%
%        \check@mathfonts
%        \fontsize\sf@size\z@
%        \math@fontsfalse
%        \selectfont
%        A%
%      }%
%      \vss
%    }%
%  }%
%  \kern-.15em%
%  \TeX
%}
%\end{verbatim}
%\end{quote}
%
%    \begin{macro}{\HoLogo@La}
%    \begin{macrocode}
\def\HoLogo@La#1{%
  L%
  \kern-.36em%
  \begingroup
    \setbox\ltx@zero\hbox{T}%
    \vbox to\ht\ltx@zero{%
      \hbox{%
        \ltx@ifundefined{check@mathfonts}{%
          \csname sevenrm\endcsname
        }{%
          \check@mathfonts
          \fontsize\sf@size{0pt}%
          \math@fontsfalse\selectfont
        }%
        A%
      }%
      \vss
    }%
  \endgroup
}
%    \end{macrocode}
%    \end{macro}
%
%    \begin{macro}{\HoLogo@LaTeX}
%    Source: \hologo{LaTeX} kernel.
%    \begin{macrocode}
\def\HoLogo@LaTeX#1{%
  \hologo{La}%
  \kern-.15em%
  \hologo{TeX}%
}
%    \end{macrocode}
%    \end{macro}
%    \begin{macro}{\HoLogoHtml@LaTeX}
%    \begin{macrocode}
\def\HoLogoHtml@LaTeX#1{%
  \HoLogoCss@LaTeX
  \HOLOGO@Span{LaTeX}{%
    L%
    \HOLOGO@Span{a}{%
      A%
    }%
    \hologo{TeX}%
  }%
}
%    \end{macrocode}
%    \end{macro}
%    \begin{macro}{\HoLogoCss@LaTeX}
%    \begin{macrocode}
\def\HoLogoCss@LaTeX{%
  \Css{%
    span.HoLogo-LaTeX span.HoLogo-a{%
      position:relative;%
      top:-.5ex;%
      margin-left:-.36em;%
      margin-right:-.15em;%
      font-size:85\%;%
    }%
  }%
  \global\let\HoLogoCss@LaTeX\relax
}
%    \end{macrocode}
%    \end{macro}
%
% \subsubsection{\hologo{(La)TeX}}
%
%    \begin{macro}{\HoLogo@LaTeXTeX}
%    The kerning around the parentheses is taken
%    from package \xpackage{dtklogos} \cite{dtklogos}.
%\begin{quote}
%\begin{verbatim}
%\DeclareRobustCommand{\LaTeXTeX}{%
%  (%
%  \kern-.15em%
%  L%
%  \kern-.36em%
%  {%
%    \sbox\z@ T%
%    \vbox to\ht0{%
%      \hbox{%
%        $\m@th$%
%        \csname S@\f@size\endcsname
%        \fontsize\sf@size\z@
%        \math@fontsfalse
%        \selectfont
%        A%
%      }%
%      \vss
%    }%
%  }%
%  \kern-.2em%
%  )%
%  \kern-.15em%
%  \TeX
%}
%\end{verbatim}
%\end{quote}
%    \begin{macrocode}
\def\HoLogo@LaTeXTeX#1{%
  (%
  \kern-.15em%
  \hologo{La}%
  \kern-.2em%
  )%
  \kern-.15em%
  \hologo{TeX}%
}
%    \end{macrocode}
%    \end{macro}
%    \begin{macro}{\HoLogoBkm@LaTeXTeX}
%    \begin{macrocode}
\def\HoLogoBkm@LaTeXTeX#1{(La)TeX}
%    \end{macrocode}
%    \end{macro}
%
%    \begin{macro}{\HoLogo@(La)TeX}
%    \begin{macrocode}
\expandafter
\let\csname HoLogo@(La)TeX\endcsname\HoLogo@LaTeXTeX
%    \end{macrocode}
%    \end{macro}
%    \begin{macro}{\HoLogoBkm@(La)TeX}
%    \begin{macrocode}
\expandafter
\let\csname HoLogoBkm@(La)TeX\endcsname\HoLogoBkm@LaTeXTeX
%    \end{macrocode}
%    \end{macro}
%    \begin{macro}{\HoLogoHtml@LaTeXTeX}
%    \begin{macrocode}
\def\HoLogoHtml@LaTeXTeX#1{%
  \HoLogoCss@LaTeXTeX
  \HOLOGO@Span{LaTeXTeX}{%
    (%
    \HOLOGO@Span{L}{L}%
    \HOLOGO@Span{a}{A}%
    \HOLOGO@Span{ParenRight}{)}%
    \hologo{TeX}%
  }%
}
%    \end{macrocode}
%    \end{macro}
%    \begin{macro}{\HoLogoHtml@(La)TeX}
%    Kerning after opening parentheses and before closing parentheses
%    is $-0.1$\,em. The original values $-0.15$\,em
%    looked too ugly for a serif font.
%    \begin{macrocode}
\expandafter
\let\csname HoLogoHtml@(La)TeX\endcsname\HoLogoHtml@LaTeXTeX
%    \end{macrocode}
%    \end{macro}
%    \begin{macro}{\HoLogoCss@LaTeXTeX}
%    \begin{macrocode}
\def\HoLogoCss@LaTeXTeX{%
  \Css{%
    span.HoLogo-LaTeXTeX span.HoLogo-L{%
      margin-left:-.1em;%
    }%
  }%
  \Css{%
    span.HoLogo-LaTeXTeX span.HoLogo-a{%
      position:relative;%
      top:-.5ex;%
      margin-left:-.36em;%
      margin-right:-.1em;%
      font-size:85\%;%
    }%
  }%
  \Css{%
    span.HoLogo-LaTeXTeX span.HoLogo-ParenRight{%
      margin-right:-.15em;%
    }%
  }%
  \global\let\HoLogoCss@LaTeXTeX\relax
}
%    \end{macrocode}
%    \end{macro}
%
% \subsubsection{\hologo{LaTeXe}}
%
%    \begin{macro}{\HoLogo@LaTeXe}
%    Source: \hologo{LaTeX} kernel
%    \begin{macrocode}
\def\HoLogo@LaTeXe#1{%
  \hologo{LaTeX}%
  \kern.15em%
  \hbox{%
    \HOLOGO@MathSetup
    2%
    $_{\textstyle\varepsilon}$%
  }%
}
%    \end{macrocode}
%    \end{macro}
%
%    \begin{macro}{\HoLogoCs@LaTeXe}
%    \begin{macrocode}
\ifnum64=`\^^^^0040\relax % test for big chars of LuaTeX/XeTeX
  \catcode`\$=9 %
  \catcode`\&=14 %
\else
  \catcode`\$=14 %
  \catcode`\&=9 %
\fi
\def\HoLogoCs@LaTeXe#1{%
  LaTeX2%
$ \string ^^^^0395%
& e%
}%
\catcode`\$=3 %
\catcode`\&=4 %
%    \end{macrocode}
%    \end{macro}
%
%    \begin{macro}{\HoLogoBkm@LaTeXe}
%    \begin{macrocode}
\def\HoLogoBkm@LaTeXe#1{%
  \hologo{LaTeX}%
  2%
  \HOLOGO@PdfdocUnicode{e}{\textepsilon}%
}
%    \end{macrocode}
%    \end{macro}
%
%    \begin{macro}{\HoLogoHtml@LaTeXe}
%    \begin{macrocode}
\def\HoLogoHtml@LaTeXe#1{%
  \HoLogoCss@LaTeXe
  \HOLOGO@Span{LaTeX2e}{%
    \hologo{LaTeX}%
    \HOLOGO@Span{2}{2}%
    \HOLOGO@Span{e}{%
      \HOLOGO@MathSetup
      \ensuremath{\textstyle\varepsilon}%
    }%
  }%
}
%    \end{macrocode}
%    \end{macro}
%    \begin{macro}{\HoLogoCss@LaTeXe}
%    \begin{macrocode}
\def\HoLogoCss@LaTeXe{%
  \Css{%
    span.HoLogo-LaTeX2e span.HoLogo-2{%
      padding-left:.15em;%
    }%
  }%
  \Css{%
    span.HoLogo-LaTeX2e span.HoLogo-e{%
      position:relative;%
      top:.35ex;%
      text-decoration:none;%
    }%
  }%
  \global\let\HoLogoCss@LaTeXe\relax
}
%    \end{macrocode}
%    \end{macro}
%
%    \begin{macro}{\HoLogo@LaTeX2e}
%    \begin{macrocode}
\expandafter
\let\csname HoLogo@LaTeX2e\endcsname\HoLogo@LaTeXe
%    \end{macrocode}
%    \end{macro}
%    \begin{macro}{\HoLogoCs@LaTeX2e}
%    \begin{macrocode}
\expandafter
\let\csname HoLogoCs@LaTeX2e\endcsname\HoLogoCs@LaTeXe
%    \end{macrocode}
%    \end{macro}
%    \begin{macro}{\HoLogoBkm@LaTeX2e}
%    \begin{macrocode}
\expandafter
\let\csname HoLogoBkm@LaTeX2e\endcsname\HoLogoBkm@LaTeXe
%    \end{macrocode}
%    \end{macro}
%    \begin{macro}{\HoLogoHtml@LaTeX2e}
%    \begin{macrocode}
\expandafter
\let\csname HoLogoHtml@LaTeX2e\endcsname\HoLogoHtml@LaTeXe
%    \end{macrocode}
%    \end{macro}
%
% \subsubsection{\hologo{LaTeX3}}
%
%    \begin{macro}{\HoLogo@LaTeX3}
%    Source: \hologo{LaTeX} kernel
%    \begin{macrocode}
\expandafter\def\csname HoLogo@LaTeX3\endcsname#1{%
  \hologo{LaTeX}%
  3%
}
%    \end{macrocode}
%    \end{macro}
%
%    \begin{macro}{\HoLogoBkm@LaTeX3}
%    \begin{macrocode}
\expandafter\def\csname HoLogoBkm@LaTeX3\endcsname#1{%
  \hologo{LaTeX}%
  3%
}
%    \end{macrocode}
%    \end{macro}
%    \begin{macro}{\HoLogoHtml@LaTeX3}
%    \begin{macrocode}
\expandafter
\let\csname HoLogoHtml@LaTeX3\expandafter\endcsname
\csname HoLogo@LaTeX3\endcsname
%    \end{macrocode}
%    \end{macro}
%
% \subsubsection{\hologo{LaTeXML}}
%
%    \begin{macro}{\HoLogo@LaTeXML}
%    \begin{macrocode}
\def\HoLogo@LaTeXML#1{%
  \HOLOGO@mbox{%
    \hologo{La}%
    \kern-.15em%
    T%
    \kern-.1667em%
    \lower.5ex\hbox{E}%
    \kern-.125em%
    \HoLogoFont@font{LaTeXML}{sc}{xml}%
  }%
}
%    \end{macrocode}
%    \end{macro}
%    \begin{macro}{\HoLogoHtml@pdfLaTeX}
%    \begin{macrocode}
\def\HoLogoHtml@LaTeXML#1{%
  \HOLOGO@Span{LaTeXML}{%
    \HoLogoCss@LaTeX
    \HoLogoCss@TeX
    \HOLOGO@Span{LaTeX}{%
      L%
      \HOLOGO@Span{a}{%
        A%
      }%
    }%
    \HOLOGO@Span{TeX}{%
      T%
      \HOLOGO@Span{e}{%
        E%
      }%
    }%
    \HCode{<span style="font-variant: small-caps;">}%
    xml%
    \HCode{</span>}%
  }%
}
%    \end{macrocode}
%    \end{macro}
%
% \subsubsection{\hologo{eTeX}}
%
%    \begin{macro}{\HoLogo@eTeX}
%    Source: package \xpackage{etex}
%    \begin{macrocode}
\def\HoLogo@eTeX#1{%
  \ltx@mbox{%
    \HOLOGO@MathSetup
    $\varepsilon$%
    -%
    \HOLOGO@NegativeKerning{-T,T-,To}%
    \hologo{TeX}%
  }%
}
%    \end{macrocode}
%    \end{macro}
%    \begin{macro}{\HoLogoCs@eTeX}
%    \begin{macrocode}
\ifnum64=`\^^^^0040\relax % test for big chars of LuaTeX/XeTeX
  \catcode`\$=9 %
  \catcode`\&=14 %
\else
  \catcode`\$=14 %
  \catcode`\&=9 %
\fi
\def\HoLogoCs@eTeX#1{%
$ #1{\string ^^^^0395}{\string ^^^^03b5}%
& #1{e}{E}%
  TeX%
}%
\catcode`\$=3 %
\catcode`\&=4 %
%    \end{macrocode}
%    \end{macro}
%    \begin{macro}{\HoLogoBkm@eTeX}
%    \begin{macrocode}
\def\HoLogoBkm@eTeX#1{%
  \HOLOGO@PdfdocUnicode{#1{e}{E}}{\textepsilon}%
  -%
  \hologo{TeX}%
}
%    \end{macrocode}
%    \end{macro}
%    \begin{macro}{\HoLogoHtml@eTeX}
%    \begin{macrocode}
\def\HoLogoHtml@eTeX#1{%
  \ltx@mbox{%
    \HOLOGO@MathSetup
    $\varepsilon$%
    -%
    \hologo{TeX}%
  }%
}
%    \end{macrocode}
%    \end{macro}
%
% \subsubsection{\hologo{iniTeX}}
%
%    \begin{macro}{\HoLogo@iniTeX}
%    \begin{macrocode}
\def\HoLogo@iniTeX#1{%
  \HOLOGO@mbox{%
    #1{i}{I}ni\hologo{TeX}%
  }%
}
%    \end{macrocode}
%    \end{macro}
%    \begin{macro}{\HoLogoCs@iniTeX}
%    \begin{macrocode}
\def\HoLogoCs@iniTeX#1{#1{i}{I}niTeX}
%    \end{macrocode}
%    \end{macro}
%    \begin{macro}{\HoLogoBkm@iniTeX}
%    \begin{macrocode}
\def\HoLogoBkm@iniTeX#1{%
  #1{i}{I}ni\hologo{TeX}%
}
%    \end{macrocode}
%    \end{macro}
%    \begin{macro}{\HoLogoHtml@iniTeX}
%    \begin{macrocode}
\let\HoLogoHtml@iniTeX\HoLogo@iniTeX
%    \end{macrocode}
%    \end{macro}
%
% \subsubsection{\hologo{virTeX}}
%
%    \begin{macro}{\HoLogo@virTeX}
%    \begin{macrocode}
\def\HoLogo@virTeX#1{%
  \HOLOGO@mbox{%
    #1{v}{V}ir\hologo{TeX}%
  }%
}
%    \end{macrocode}
%    \end{macro}
%    \begin{macro}{\HoLogoCs@virTeX}
%    \begin{macrocode}
\def\HoLogoCs@virTeX#1{#1{v}{V}irTeX}
%    \end{macrocode}
%    \end{macro}
%    \begin{macro}{\HoLogoBkm@virTeX}
%    \begin{macrocode}
\def\HoLogoBkm@virTeX#1{%
  #1{v}{V}ir\hologo{TeX}%
}
%    \end{macrocode}
%    \end{macro}
%    \begin{macro}{\HoLogoHtml@virTeX}
%    \begin{macrocode}
\let\HoLogoHtml@virTeX\HoLogo@virTeX
%    \end{macrocode}
%    \end{macro}
%
% \subsubsection{\hologo{SliTeX}}
%
% \paragraph{Definitions of the three variants.}
%
%    \begin{macro}{\HoLogo@SLiTeX@lift}
%    \begin{macrocode}
\def\HoLogo@SLiTeX@lift#1{%
  \HoLogoFont@font{SliTeX}{rm}{%
    S%
    \kern-.06em%
    L%
    \kern-.18em%
    \raise.32ex\hbox{\HoLogoFont@font{SliTeX}{sc}{i}}%
    \HOLOGO@discretionary
    \kern-.06em%
    \hologo{TeX}%
  }%
}
%    \end{macrocode}
%    \end{macro}
%    \begin{macro}{\HoLogoBkm@SLiTeX@lift}
%    \begin{macrocode}
\def\HoLogoBkm@SLiTeX@lift#1{SLiTeX}
%    \end{macrocode}
%    \end{macro}
%    \begin{macro}{\HoLogoHtml@SLiTeX@lift}
%    \begin{macrocode}
\def\HoLogoHtml@SLiTeX@lift#1{%
  \HoLogoCss@SLiTeX@lift
  \HOLOGO@Span{SLiTeX-lift}{%
    \HoLogoFont@font{SliTeX}{rm}{%
      S%
      \HOLOGO@Span{L}{L}%
      \HOLOGO@Span{i}{i}%
      \hologo{TeX}%
    }%
  }%
}
%    \end{macrocode}
%    \end{macro}
%    \begin{macro}{\HoLogoCss@SLiTeX@lift}
%    \begin{macrocode}
\def\HoLogoCss@SLiTeX@lift{%
  \Css{%
    span.HoLogo-SLiTeX-lift span.HoLogo-L{%
      margin-left:-.06em;%
      margin-right:-.18em;%
    }%
  }%
  \Css{%
    span.HoLogo-SLiTeX-lift span.HoLogo-i{%
      position:relative;%
      top:-.32ex;%
      margin-right:-.06em;%
      font-variant:small-caps;%
    }%
  }%
  \global\let\HoLogoCss@SLiTeX@lift\relax
}
%    \end{macrocode}
%    \end{macro}
%
%    \begin{macro}{\HoLogo@SliTeX@simple}
%    \begin{macrocode}
\def\HoLogo@SliTeX@simple#1{%
  \HoLogoFont@font{SliTeX}{rm}{%
    \ltx@mbox{%
      \HoLogoFont@font{SliTeX}{sc}{Sli}%
    }%
    \HOLOGO@discretionary
    \hologo{TeX}%
  }%
}
%    \end{macrocode}
%    \end{macro}
%    \begin{macro}{\HoLogoBkm@SliTeX@simple}
%    \begin{macrocode}
\def\HoLogoBkm@SliTeX@simple#1{SliTeX}
%    \end{macrocode}
%    \end{macro}
%    \begin{macro}{\HoLogoHtml@SliTeX@simple}
%    \begin{macrocode}
\let\HoLogoHtml@SliTeX@simple\HoLogo@SliTeX@simple
%    \end{macrocode}
%    \end{macro}
%
%    \begin{macro}{\HoLogo@SliTeX@narrow}
%    \begin{macrocode}
\def\HoLogo@SliTeX@narrow#1{%
  \HoLogoFont@font{SliTeX}{rm}{%
    \ltx@mbox{%
      S%
      \kern-.06em%
      \HoLogoFont@font{SliTeX}{sc}{%
        l%
        \kern-.035em%
        i%
      }%
    }%
    \HOLOGO@discretionary
    \kern-.06em%
    \hologo{TeX}%
  }%
}
%    \end{macrocode}
%    \end{macro}
%    \begin{macro}{\HoLogoBkm@SliTeX@narrow}
%    \begin{macrocode}
\def\HoLogoBkm@SliTeX@narrow#1{SliTeX}
%    \end{macrocode}
%    \end{macro}
%    \begin{macro}{\HoLogoHtml@SliTeX@narrow}
%    \begin{macrocode}
\def\HoLogoHtml@SliTeX@narrow#1{%
  \HoLogoCss@SliTeX@narrow
  \HOLOGO@Span{SliTeX-narrow}{%
    \HoLogoFont@font{SliTeX}{rm}{%
      S%
        \HOLOGO@Span{l}{l}%
        \HOLOGO@Span{i}{i}%
      \hologo{TeX}%
    }%
  }%
}
%    \end{macrocode}
%    \end{macro}
%    \begin{macro}{\HoLogoCss@SliTeX@narrow}
%    \begin{macrocode}
\def\HoLogoCss@SliTeX@narrow{%
  \Css{%
    span.HoLogo-SliTeX-narrow span.HoLogo-l{%
      margin-left:-.06em;%
      margin-right:-.035em;%
      font-variant:small-caps;%
    }%
  }%
  \Css{%
    span.HoLogo-SliTeX-narrow span.HoLogo-i{%
      margin-right:-.06em;%
      font-variant:small-caps;%
    }%
  }%
  \global\let\HoLogoCss@SliTeX@narrow\relax
}
%    \end{macrocode}
%    \end{macro}
%
% \paragraph{Macro set completion.}
%
%    \begin{macro}{\HoLogo@SLiTeX@simple}
%    \begin{macrocode}
\def\HoLogo@SLiTeX@simple{\HoLogo@SliTeX@simple}
%    \end{macrocode}
%    \end{macro}
%    \begin{macro}{\HoLogoBkm@SLiTeX@simple}
%    \begin{macrocode}
\def\HoLogoBkm@SLiTeX@simple{\HoLogoBkm@SliTeX@simple}
%    \end{macrocode}
%    \end{macro}
%    \begin{macro}{\HoLogoHtml@SLiTeX@simple}
%    \begin{macrocode}
\def\HoLogoHtml@SLiTeX@simple{\HoLogoHtml@SliTeX@simple}
%    \end{macrocode}
%    \end{macro}
%
%    \begin{macro}{\HoLogo@SLiTeX@narrow}
%    \begin{macrocode}
\def\HoLogo@SLiTeX@narrow{\HoLogo@SliTeX@narrow}
%    \end{macrocode}
%    \end{macro}
%    \begin{macro}{\HoLogoBkm@SLiTeX@narrow}
%    \begin{macrocode}
\def\HoLogoBkm@SLiTeX@narrow{\HoLogoBkm@SliTeX@narrow}
%    \end{macrocode}
%    \end{macro}
%    \begin{macro}{\HoLogoHtml@SLiTeX@narrow}
%    \begin{macrocode}
\def\HoLogoHtml@SLiTeX@narrow{\HoLogoHtml@SliTeX@narrow}
%    \end{macrocode}
%    \end{macro}
%
%    \begin{macro}{\HoLogo@SliTeX@lift}
%    \begin{macrocode}
\def\HoLogo@SliTeX@lift{\HoLogo@SLiTeX@lift}
%    \end{macrocode}
%    \end{macro}
%    \begin{macro}{\HoLogoBkm@SliTeX@lift}
%    \begin{macrocode}
\def\HoLogoBkm@SliTeX@lift{\HoLogoBkm@SLiTeX@lift}
%    \end{macrocode}
%    \end{macro}
%    \begin{macro}{\HoLogoHtml@SliTeX@lift}
%    \begin{macrocode}
\def\HoLogoHtml@SliTeX@lift{\HoLogoHtml@SLiTeX@lift}
%    \end{macrocode}
%    \end{macro}
%
% \paragraph{Defaults.}
%
%    \begin{macro}{\HoLogo@SLiTeX}
%    \begin{macrocode}
\def\HoLogo@SLiTeX{\HoLogo@SLiTeX@lift}
%    \end{macrocode}
%    \end{macro}
%    \begin{macro}{\HoLogoBkm@SLiTeX}
%    \begin{macrocode}
\def\HoLogoBkm@SLiTeX{\HoLogoBkm@SLiTeX@lift}
%    \end{macrocode}
%    \end{macro}
%    \begin{macro}{\HoLogoHtml@SLiTeX}
%    \begin{macrocode}
\def\HoLogoHtml@SLiTeX{\HoLogoHtml@SLiTeX@lift}
%    \end{macrocode}
%    \end{macro}
%
%    \begin{macro}{\HoLogo@SliTeX}
%    \begin{macrocode}
\def\HoLogo@SliTeX{\HoLogo@SliTeX@narrow}
%    \end{macrocode}
%    \end{macro}
%    \begin{macro}{\HoLogoBkm@SliTeX}
%    \begin{macrocode}
\def\HoLogoBkm@SliTeX{\HoLogoBkm@SliTeX@narrow}
%    \end{macrocode}
%    \end{macro}
%    \begin{macro}{\HoLogoHtml@SliTeX}
%    \begin{macrocode}
\def\HoLogoHtml@SliTeX{\HoLogoHtml@SliTeX@narrow}
%    \end{macrocode}
%    \end{macro}
%
% \subsubsection{\hologo{LuaTeX}}
%
%    \begin{macro}{\HoLogo@LuaTeX}
%    The kerning is an idea of Hans Hagen, see mailing list
%    `luatex at tug dot org' in March 2010.
%    \begin{macrocode}
\def\HoLogo@LuaTeX#1{%
  \HOLOGO@mbox{%
    Lua%
    \HOLOGO@NegativeKerning{aT,oT,To}%
    \hologo{TeX}%
  }%
}
%    \end{macrocode}
%    \end{macro}
%    \begin{macro}{\HoLogoHtml@LuaTeX}
%    \begin{macrocode}
\let\HoLogoHtml@LuaTeX\HoLogo@LuaTeX
%    \end{macrocode}
%    \end{macro}
%
% \subsubsection{\hologo{LuaLaTeX}}
%
%    \begin{macro}{\HoLogo@LuaLaTeX}
%    \begin{macrocode}
\def\HoLogo@LuaLaTeX#1{%
  \HOLOGO@mbox{%
    Lua%
    \hologo{LaTeX}%
  }%
}
%    \end{macrocode}
%    \end{macro}
%    \begin{macro}{\HoLogoHtml@LuaLaTeX}
%    \begin{macrocode}
\let\HoLogoHtml@LuaLaTeX\HoLogo@LuaLaTeX
%    \end{macrocode}
%    \end{macro}
%
% \subsubsection{\hologo{XeTeX}, \hologo{XeLaTeX}}
%
%    \begin{macro}{\HOLOGO@IfCharExists}
%    \begin{macrocode}
\ifluatex
  \ifnum\luatexversion<36 %
  \else
    \def\HOLOGO@IfCharExists#1{%
      \ifnum
        \directlua{%
           if luaotfload and luaotfload.aux then
             if luaotfload.aux.font_has_glyph(%
                    font.current(), \number#1) then % 	 
	       tex.print("1") % 	 
	     end % 	 
	   elseif font and font.fonts and font.current then %
            local f = font.fonts[font.current()]%
            if f.characters and f.characters[\number#1] then %
              tex.print("1")%
            end %
          end%
        }0=\ltx@zero
        \expandafter\ltx@secondoftwo
      \else
        \expandafter\ltx@firstoftwo
      \fi
    }%
  \fi
\fi
\ltx@IfUndefined{HOLOGO@IfCharExists}{%
  \def\HOLOGO@@IfCharExists#1{%
    \begingroup
      \tracinglostchars=\ltx@zero
      \setbox\ltx@zero=\hbox{%
        \kern7sp\char#1\relax
        \ifnum\lastkern>\ltx@zero
          \expandafter\aftergroup\csname iffalse\endcsname
        \else
          \expandafter\aftergroup\csname iftrue\endcsname
        \fi
      }%
      % \if{true|false} from \aftergroup
      \endgroup
      \expandafter\ltx@firstoftwo
    \else
      \endgroup
      \expandafter\ltx@secondoftwo
    \fi
  }%
  \ifxetex
    \ltx@IfUndefined{XeTeXfonttype}{}{%
      \ltx@IfUndefined{XeTeXcharglyph}{}{%
        \def\HOLOGO@IfCharExists#1{%
          \ifnum\XeTeXfonttype\font>\ltx@zero
            \expandafter\ltx@firstofthree
          \else
            \expandafter\ltx@gobble
          \fi
          {%
            \ifnum\XeTeXcharglyph#1>\ltx@zero
              \expandafter\ltx@firstoftwo
            \else
              \expandafter\ltx@secondoftwo
            \fi
          }%
          \HOLOGO@@IfCharExists{#1}%
        }%
      }%
    }%
  \fi
}{}
\ltx@ifundefined{HOLOGO@IfCharExists}{%
  \ifnum64=`\^^^^0040\relax % test for big chars of LuaTeX/XeTeX
    \let\HOLOGO@IfCharExists\HOLOGO@@IfCharExists
  \else
    \def\HOLOGO@IfCharExists#1{%
      \ifnum#1>255 %
        \expandafter\ltx@fourthoffour
      \fi
      \HOLOGO@@IfCharExists{#1}%
    }%
  \fi
}{}
%    \end{macrocode}
%    \end{macro}
%
%    \begin{macro}{\HoLogo@Xe}
%    Source: package \xpackage{dtklogos}
%    \begin{macrocode}
\def\HoLogo@Xe#1{%
  X%
  \kern-.1em\relax
  \HOLOGO@IfCharExists{"018E}{%
    \lower.5ex\hbox{\char"018E}%
  }{%
    \chardef\HOLOGO@choice=\ltx@zero
    \ifdim\fontdimen\ltx@one\font>0pt %
      \ltx@IfUndefined{rotatebox}{%
        \ltx@IfUndefined{pgftext}{%
          \ltx@IfUndefined{psscalebox}{%
            \ltx@IfUndefined{HOLOGO@ScaleBox@\hologoDriver}{%
            }{%
              \chardef\HOLOGO@choice=4 %
            }%
          }{%
            \chardef\HOLOGO@choice=3 %
          }%
        }{%
          \chardef\HOLOGO@choice=2 %
        }%
      }{%
        \chardef\HOLOGO@choice=1 %
      }%
      \ifcase\HOLOGO@choice
        \HOLOGO@WarningUnsupportedDriver{Xe}%
        e%
      \or % 1: \rotatebox
        \begingroup
          \setbox\ltx@zero\hbox{\rotatebox{180}{E}}%
          \ltx@LocDimenA=\dp\ltx@zero
          \advance\ltx@LocDimenA by -.5ex\relax
          \raise\ltx@LocDimenA\box\ltx@zero
        \endgroup
      \or % 2: \pgftext
        \lower.5ex\hbox{%
          \pgfpicture
            \pgftext[rotate=180]{E}%
          \endpgfpicture
        }%
      \or % 3: \psscalebox
        \begingroup
          \setbox\ltx@zero\hbox{\psscalebox{-1 -1}{E}}%
          \ltx@LocDimenA=\dp\ltx@zero
          \advance\ltx@LocDimenA by -.5ex\relax
          \raise\ltx@LocDimenA\box\ltx@zero
        \endgroup
      \or % 4: \HOLOGO@PointReflectBox
        \lower.5ex\hbox{\HOLOGO@PointReflectBox{E}}%
      \else
        \@PackageError{hologo}{Internal error (choice/it}\@ehc
      \fi
    \else
      \ltx@IfUndefined{reflectbox}{%
        \ltx@IfUndefined{pgftext}{%
          \ltx@IfUndefined{psscalebox}{%
            \ltx@IfUndefined{HOLOGO@ScaleBox@\hologoDriver}{%
            }{%
              \chardef\HOLOGO@choice=4 %
            }%
          }{%
            \chardef\HOLOGO@choice=3 %
          }%
        }{%
          \chardef\HOLOGO@choice=2 %
        }%
      }{%
        \chardef\HOLOGO@choice=1 %
      }%
      \ifcase\HOLOGO@choice
        \HOLOGO@WarningUnsupportedDriver{Xe}%
        e%
      \or % 1: reflectbox
        \lower.5ex\hbox{%
          \reflectbox{E}%
        }%
      \or % 2: \pgftext
        \lower.5ex\hbox{%
          \pgfpicture
            \pgftransformxscale{-1}%
            \pgftext{E}%
          \endpgfpicture
        }%
      \or % 3: \psscalebox
        \lower.5ex\hbox{%
          \psscalebox{-1 1}{E}%
        }%
      \or % 4: \HOLOGO@Reflectbox
        \lower.5ex\hbox{%
          \HOLOGO@ReflectBox{E}%
        }%
      \else
        \@PackageError{hologo}{Internal error (choice/up)}\@ehc
      \fi
    \fi
  }%
}
%    \end{macrocode}
%    \end{macro}
%    \begin{macro}{\HoLogoHtml@Xe}
%    \begin{macrocode}
\def\HoLogoHtml@Xe#1{%
  \HoLogoCss@Xe
  \HOLOGO@Span{Xe}{%
    X%
    \HOLOGO@Span{e}{%
      \HCode{&\ltx@hashchar x018e;}%
    }%
  }%
}
%    \end{macrocode}
%    \end{macro}
%    \begin{macro}{\HoLogoCss@Xe}
%    \begin{macrocode}
\def\HoLogoCss@Xe{%
  \Css{%
    span.HoLogo-Xe span.HoLogo-e{%
      position:relative;%
      top:.5ex;%
      left-margin:-.1em;%
    }%
  }%
  \global\let\HoLogoCss@Xe\relax
}
%    \end{macrocode}
%    \end{macro}
%
%    \begin{macro}{\HoLogo@XeTeX}
%    \begin{macrocode}
\def\HoLogo@XeTeX#1{%
  \hologo{Xe}%
  \kern-.15em\relax
  \hologo{TeX}%
}
%    \end{macrocode}
%    \end{macro}
%
%    \begin{macro}{\HoLogoHtml@XeTeX}
%    \begin{macrocode}
\def\HoLogoHtml@XeTeX#1{%
  \HoLogoCss@XeTeX
  \HOLOGO@Span{XeTeX}{%
    \hologo{Xe}%
    \hologo{TeX}%
  }%
}
%    \end{macrocode}
%    \end{macro}
%    \begin{macro}{\HoLogoCss@XeTeX}
%    \begin{macrocode}
\def\HoLogoCss@XeTeX{%
  \Css{%
    span.HoLogo-XeTeX span.HoLogo-TeX{%
      margin-left:-.15em;%
    }%
  }%
  \global\let\HoLogoCss@XeTeX\relax
}
%    \end{macrocode}
%    \end{macro}
%
%    \begin{macro}{\HoLogo@XeLaTeX}
%    \begin{macrocode}
\def\HoLogo@XeLaTeX#1{%
  \hologo{Xe}%
  \kern-.13em%
  \hologo{LaTeX}%
}
%    \end{macrocode}
%    \end{macro}
%    \begin{macro}{\HoLogoHtml@XeLaTeX}
%    \begin{macrocode}
\def\HoLogoHtml@XeLaTeX#1{%
  \HoLogoCss@XeLaTeX
  \HOLOGO@Span{XeLaTeX}{%
    \hologo{Xe}%
    \hologo{LaTeX}%
  }%
}
%    \end{macrocode}
%    \end{macro}
%    \begin{macro}{\HoLogoCss@XeLaTeX}
%    \begin{macrocode}
\def\HoLogoCss@XeLaTeX{%
  \Css{%
    span.HoLogo-XeLaTeX span.HoLogo-Xe{%
      margin-right:-.13em;%
    }%
  }%
  \global\let\HoLogoCss@XeLaTeX\relax
}
%    \end{macrocode}
%    \end{macro}
%
% \subsubsection{\hologo{pdfTeX}, \hologo{pdfLaTeX}}
%
%    \begin{macro}{\HoLogo@pdfTeX}
%    \begin{macrocode}
\def\HoLogo@pdfTeX#1{%
  \HOLOGO@mbox{%
    #1{p}{P}df\hologo{TeX}%
  }%
}
%    \end{macrocode}
%    \end{macro}
%    \begin{macro}{\HoLogoCs@pdfTeX}
%    \begin{macrocode}
\def\HoLogoCs@pdfTeX#1{#1{p}{P}dfTeX}
%    \end{macrocode}
%    \end{macro}
%    \begin{macro}{\HoLogoBkm@pdfTeX}
%    \begin{macrocode}
\def\HoLogoBkm@pdfTeX#1{%
  #1{p}{P}df\hologo{TeX}%
}
%    \end{macrocode}
%    \end{macro}
%    \begin{macro}{\HoLogoHtml@pdfTeX}
%    \begin{macrocode}
\let\HoLogoHtml@pdfTeX\HoLogo@pdfTeX
%    \end{macrocode}
%    \end{macro}
%
%    \begin{macro}{\HoLogo@pdfLaTeX}
%    \begin{macrocode}
\def\HoLogo@pdfLaTeX#1{%
  \HOLOGO@mbox{%
    #1{p}{P}df\hologo{LaTeX}%
  }%
}
%    \end{macrocode}
%    \end{macro}
%    \begin{macro}{\HoLogoCs@pdfLaTeX}
%    \begin{macrocode}
\def\HoLogoCs@pdfLaTeX#1{#1{p}{P}dfLaTeX}
%    \end{macrocode}
%    \end{macro}
%    \begin{macro}{\HoLogoBkm@pdfLaTeX}
%    \begin{macrocode}
\def\HoLogoBkm@pdfLaTeX#1{%
  #1{p}{P}df\hologo{LaTeX}%
}
%    \end{macrocode}
%    \end{macro}
%    \begin{macro}{\HoLogoHtml@pdfLaTeX}
%    \begin{macrocode}
\let\HoLogoHtml@pdfLaTeX\HoLogo@pdfLaTeX
%    \end{macrocode}
%    \end{macro}
%
% \subsubsection{\hologo{VTeX}}
%
%    \begin{macro}{\HoLogo@VTeX}
%    \begin{macrocode}
\def\HoLogo@VTeX#1{%
  \HOLOGO@mbox{%
    V\hologo{TeX}%
  }%
}
%    \end{macrocode}
%    \end{macro}
%    \begin{macro}{\HoLogoHtml@VTeX}
%    \begin{macrocode}
\let\HoLogoHtml@VTeX\HoLogo@VTeX
%    \end{macrocode}
%    \end{macro}
%
% \subsubsection{\hologo{AmS}, \dots}
%
%    Source: class \xclass{amsdtx}
%
%    \begin{macro}{\HoLogo@AmS}
%    \begin{macrocode}
\def\HoLogo@AmS#1{%
  \HoLogoFont@font{AmS}{sy}{%
    A%
    \kern-.1667em%
    \lower.5ex\hbox{M}%
    \kern-.125em%
    S%
  }%
}
%    \end{macrocode}
%    \end{macro}
%    \begin{macro}{\HoLogoBkm@AmS}
%    \begin{macrocode}
\def\HoLogoBkm@AmS#1{AmS}
%    \end{macrocode}
%    \end{macro}
%    \begin{macro}{\HoLogoHtml@AmS}
%    \begin{macrocode}
\def\HoLogoHtml@AmS#1{%
  \HoLogoCss@AmS
%  \HoLogoFont@font{AmS}{sy}{%
    \HOLOGO@Span{AmS}{%
      A%
      \HOLOGO@Span{M}{M}%
      S%
    }%
%   }%
}
%    \end{macrocode}
%    \end{macro}
%    \begin{macro}{\HoLogoCss@AmS}
%    \begin{macrocode}
\def\HoLogoCss@AmS{%
  \Css{%
    span.HoLogo-AmS span.HoLogo-M{%
      position:relative;%
      top:.5ex;%
      margin-left:-.1667em;%
      margin-right:-.125em;%
      text-decoration:none;%
    }%
  }%
  \global\let\HoLogoCss@AmS\relax
}
%    \end{macrocode}
%    \end{macro}
%
%    \begin{macro}{\HoLogo@AmSTeX}
%    \begin{macrocode}
\def\HoLogo@AmSTeX#1{%
  \hologo{AmS}%
  \HOLOGO@hyphen
  \hologo{TeX}%
}
%    \end{macrocode}
%    \end{macro}
%    \begin{macro}{\HoLogoBkm@AmSTeX}
%    \begin{macrocode}
\def\HoLogoBkm@AmSTeX#1{AmS-TeX}%
%    \end{macrocode}
%    \end{macro}
%    \begin{macro}{\HoLogoHtml@AmSTeX}
%    \begin{macrocode}
\let\HoLogoHtml@AmSTeX\HoLogo@AmSTeX
%    \end{macrocode}
%    \end{macro}
%
%    \begin{macro}{\HoLogo@AmSLaTeX}
%    \begin{macrocode}
\def\HoLogo@AmSLaTeX#1{%
  \hologo{AmS}%
  \HOLOGO@hyphen
  \hologo{LaTeX}%
}
%    \end{macrocode}
%    \end{macro}
%    \begin{macro}{\HoLogoBkm@AmSLaTeX}
%    \begin{macrocode}
\def\HoLogoBkm@AmSLaTeX#1{AmS-LaTeX}%
%    \end{macrocode}
%    \end{macro}
%    \begin{macro}{\HoLogoHtml@AmSLaTeX}
%    \begin{macrocode}
\let\HoLogoHtml@AmSLaTeX\HoLogo@AmSLaTeX
%    \end{macrocode}
%    \end{macro}
%
% \subsubsection{\hologo{BibTeX}}
%
%    \begin{macro}{\HoLogo@BibTeX@sc}
%    A definition of \hologo{BibTeX} is provided in
%    the documentation source for the manual of \hologo{BibTeX}
%    \cite{btxdoc}.
%\begin{quote}
%\begin{verbatim}
%\def\BibTeX{%
%  {%
%    \rm
%    B%
%    \kern-.05em%
%    {%
%      \sc
%      i%
%      \kern-.025em %
%      b%
%    }%
%    \kern-.08em
%    T%
%    \kern-.1667em%
%    \lower.7ex\hbox{E}%
%    \kern-.125em%
%    X%
%  }%
%}
%\end{verbatim}
%\end{quote}
%    \begin{macrocode}
\def\HoLogo@BibTeX@sc#1{%
  B%
  \kern-.05em%
  \HoLogoFont@font{BibTeX}{sc}{%
    i%
    \kern-.025em%
    b%
  }%
  \HOLOGO@discretionary
  \kern-.08em%
  \hologo{TeX}%
}
%    \end{macrocode}
%    \end{macro}
%    \begin{macro}{\HoLogoHtml@BibTeX@sc}
%    \begin{macrocode}
\def\HoLogoHtml@BibTeX@sc#1{%
  \HoLogoCss@BibTeX@sc
  \HOLOGO@Span{BibTeX-sc}{%
    B%
    \HOLOGO@Span{i}{i}%
    \HOLOGO@Span{b}{b}%
    \hologo{TeX}%
  }%
}
%    \end{macrocode}
%    \end{macro}
%    \begin{macro}{\HoLogoCss@BibTeX@sc}
%    \begin{macrocode}
\def\HoLogoCss@BibTeX@sc{%
  \Css{%
    span.HoLogo-BibTeX-sc span.HoLogo-i{%
      margin-left:-.05em;%
      margin-right:-.025em;%
      font-variant:small-caps;%
    }%
  }%
  \Css{%
    span.HoLogo-BibTeX-sc span.HoLogo-b{%
      margin-right:-.08em;%
      font-variant:small-caps;%
    }%
  }%
  \global\let\HoLogoCss@BibTeX@sc\relax
}
%    \end{macrocode}
%    \end{macro}
%
%    \begin{macro}{\HoLogo@BibTeX@sf}
%    Variant \xoption{sf} avoids trouble with unavailable
%    small caps fonts (e.g., bold versions of Computer Modern or
%    Latin Modern). The definition is taken from
%    package \xpackage{dtklogos} \cite{dtklogos}.
%\begin{quote}
%\begin{verbatim}
%\DeclareRobustCommand{\BibTeX}{%
%  B%
%  \kern-.05em%
%  \hbox{%
%    $\m@th$% %% force math size calculations
%    \csname S@\f@size\endcsname
%    \fontsize\sf@size\z@
%    \math@fontsfalse
%    \selectfont
%    I%
%    \kern-.025em%
%    B
%  }%
%  \kern-.08em%
%  \-%
%  \TeX
%}
%\end{verbatim}
%\end{quote}
%    \begin{macrocode}
\def\HoLogo@BibTeX@sf#1{%
  B%
  \kern-.05em%
  \HoLogoFont@font{BibTeX}{bibsf}{%
    I%
    \kern-.025em%
    B%
  }%
  \HOLOGO@discretionary
  \kern-.08em%
  \hologo{TeX}%
}
%    \end{macrocode}
%    \end{macro}
%    \begin{macro}{\HoLogoHtml@BibTeX@sf}
%    \begin{macrocode}
\def\HoLogoHtml@BibTeX@sf#1{%
  \HoLogoCss@BibTeX@sf
  \HOLOGO@Span{BibTeX-sf}{%
    B%
    \HoLogoFont@font{BibTeX}{bibsf}{%
      \HOLOGO@Span{i}{I}%
      B%
    }%
    \hologo{TeX}%
  }%
}
%    \end{macrocode}
%    \end{macro}
%    \begin{macro}{\HoLogoCss@BibTeX@sf}
%    \begin{macrocode}
\def\HoLogoCss@BibTeX@sf{%
  \Css{%
    span.HoLogo-BibTeX-sf span.HoLogo-i{%
      margin-left:-.05em;%
      margin-right:-.025em;%
    }%
  }%
  \Css{%
    span.HoLogo-BibTeX-sf span.HoLogo-TeX{%
      margin-left:-.08em;%
    }%
  }%
  \global\let\HoLogoCss@BibTeX@sf\relax
}
%    \end{macrocode}
%    \end{macro}
%
%    \begin{macro}{\HoLogo@BibTeX}
%    \begin{macrocode}
\def\HoLogo@BibTeX{\HoLogo@BibTeX@sf}
%    \end{macrocode}
%    \end{macro}
%    \begin{macro}{\HoLogoHtml@BibTeX}
%    \begin{macrocode}
\def\HoLogoHtml@BibTeX{\HoLogoHtml@BibTeX@sf}
%    \end{macrocode}
%    \end{macro}
%
% \subsubsection{\hologo{BibTeX8}}
%
%    \begin{macro}{\HoLogo@BibTeX8}
%    \begin{macrocode}
\expandafter\def\csname HoLogo@BibTeX8\endcsname#1{%
  \hologo{BibTeX}%
  8%
}
%    \end{macrocode}
%    \end{macro}
%
%    \begin{macro}{\HoLogoBkm@BibTeX8}
%    \begin{macrocode}
\expandafter\def\csname HoLogoBkm@BibTeX8\endcsname#1{%
  \hologo{BibTeX}%
  8%
}
%    \end{macrocode}
%    \end{macro}
%    \begin{macro}{\HoLogoHtml@BibTeX8}
%    \begin{macrocode}
\expandafter
\let\csname HoLogoHtml@BibTeX8\expandafter\endcsname
\csname HoLogo@BibTeX8\endcsname
%    \end{macrocode}
%    \end{macro}
%
% \subsubsection{\hologo{ConTeXt}}
%
%    \begin{macro}{\HoLogo@ConTeXt@simple}
%    \begin{macrocode}
\def\HoLogo@ConTeXt@simple#1{%
  \HOLOGO@mbox{Con}%
  \HOLOGO@discretionary
  \HOLOGO@mbox{\hologo{TeX}t}%
}
%    \end{macrocode}
%    \end{macro}
%    \begin{macro}{\HoLogoHtml@ConTeXt@simple}
%    \begin{macrocode}
\let\HoLogoHtml@ConTeXt@simple\HoLogo@ConTeXt@simple
%    \end{macrocode}
%    \end{macro}
%
%    \begin{macro}{\HoLogo@ConTeXt@narrow}
%    This definition of logo \hologo{ConTeXt} with variant \xoption{narrow}
%    comes from TUGboat's class \xclass{ltugboat} (version 2010/11/15 v2.8).
%    \begin{macrocode}
\def\HoLogo@ConTeXt@narrow#1{%
  \HOLOGO@mbox{C\kern-.0333emon}%
  \HOLOGO@discretionary
  \kern-.0667em%
  \HOLOGO@mbox{\hologo{TeX}\kern-.0333emt}%
}
%    \end{macrocode}
%    \end{macro}
%    \begin{macro}{\HoLogoHtml@ConTeXt@narrow}
%    \begin{macrocode}
\def\HoLogoHtml@ConTeXt@narrow#1{%
  \HoLogoCss@ConTeXt@narrow
  \HOLOGO@Span{ConTeXt-narrow}{%
    \HOLOGO@Span{C}{C}%
    on%
    \hologo{TeX}%
    t%
  }%
}
%    \end{macrocode}
%    \end{macro}
%    \begin{macro}{\HoLogoCss@ConTeXt@narrow}
%    \begin{macrocode}
\def\HoLogoCss@ConTeXt@narrow{%
  \Css{%
    span.HoLogo-ConTeXt-narrow span.HoLogo-C{%
      margin-left:-.0333em;%
    }%
  }%
  \Css{%
    span.HoLogo-ConTeXt-narrow span.HoLogo-TeX{%
      margin-left:-.0667em;%
      margin-right:-.0333em;%
    }%
  }%
  \global\let\HoLogoCss@ConTeXt@narrow\relax
}
%    \end{macrocode}
%    \end{macro}
%
%    \begin{macro}{\HoLogo@ConTeXt}
%    \begin{macrocode}
\def\HoLogo@ConTeXt{\HoLogo@ConTeXt@narrow}
%    \end{macrocode}
%    \end{macro}
%    \begin{macro}{\HoLogoHtml@ConTeXt}
%    \begin{macrocode}
\def\HoLogoHtml@ConTeXt{\HoLogoHtml@ConTeXt@narrow}
%    \end{macrocode}
%    \end{macro}
%
% \subsubsection{\hologo{emTeX}}
%
%    \begin{macro}{\HoLogo@emTeX}
%    \begin{macrocode}
\def\HoLogo@emTeX#1{%
  \HOLOGO@mbox{#1{e}{E}m}%
  \HOLOGO@discretionary
  \hologo{TeX}%
}
%    \end{macrocode}
%    \end{macro}
%    \begin{macro}{\HoLogoCs@emTeX}
%    \begin{macrocode}
\def\HoLogoCs@emTeX#1{#1{e}{E}mTeX}%
%    \end{macrocode}
%    \end{macro}
%    \begin{macro}{\HoLogoBkm@emTeX}
%    \begin{macrocode}
\def\HoLogoBkm@emTeX#1{%
  #1{e}{E}m\hologo{TeX}%
}
%    \end{macrocode}
%    \end{macro}
%    \begin{macro}{\HoLogoHtml@emTeX}
%    \begin{macrocode}
\let\HoLogoHtml@emTeX\HoLogo@emTeX
%    \end{macrocode}
%    \end{macro}
%
% \subsubsection{\hologo{ExTeX}}
%
%    \begin{macro}{\HoLogo@ExTeX}
%    The definition is taken from the FAQ of the
%    project \hologo{ExTeX}
%    \cite{ExTeX-FAQ}.
%\begin{quote}
%\begin{verbatim}
%\def\ExTeX{%
%  \textrm{% Logo always with serifs
%    \ensuremath{%
%      \textstyle
%      \varepsilon_{%
%        \kern-0.15em%
%        \mathcal{X}%
%      }%
%    }%
%    \kern-.15em%
%    \TeX
%  }%
%}
%\end{verbatim}
%\end{quote}
%    \begin{macrocode}
\def\HoLogo@ExTeX#1{%
  \HoLogoFont@font{ExTeX}{rm}{%
    \ltx@mbox{%
      \HOLOGO@MathSetup
      $%
        \textstyle
        \varepsilon_{%
          \kern-0.15em%
          \HoLogoFont@font{ExTeX}{sy}{X}%
        }%
      $%
    }%
    \HOLOGO@discretionary
    \kern-.15em%
    \hologo{TeX}%
  }%
}
%    \end{macrocode}
%    \end{macro}
%    \begin{macro}{\HoLogoHtml@ExTeX}
%    \begin{macrocode}
\def\HoLogoHtml@ExTeX#1{%
  \HoLogoCss@ExTeX
  \HoLogoFont@font{ExTeX}{rm}{%
    \HOLOGO@Span{ExTeX}{%
      \ltx@mbox{%
        \HOLOGO@MathSetup
        $\textstyle\varepsilon$%
        \HOLOGO@Span{X}{$\textstyle\chi$}%
        \hologo{TeX}%
      }%
    }%
  }%
}
%    \end{macrocode}
%    \end{macro}
%    \begin{macro}{\HoLogoBkm@ExTeX}
%    \begin{macrocode}
\def\HoLogoBkm@ExTeX#1{%
  \HOLOGO@PdfdocUnicode{#1{e}{E}x}{\textepsilon\textchi}%
  \hologo{TeX}%
}
%    \end{macrocode}
%    \end{macro}
%    \begin{macro}{\HoLogoCss@ExTeX}
%    \begin{macrocode}
\def\HoLogoCss@ExTeX{%
  \Css{%
    span.HoLogo-ExTeX{%
      font-family:serif;%
    }%
  }%
  \Css{%
    span.HoLogo-ExTeX span.HoLogo-TeX{%
      margin-left:-.15em;%
    }%
  }%
  \global\let\HoLogoCss@ExTeX\relax
}
%    \end{macrocode}
%    \end{macro}
%
% \subsubsection{\hologo{MiKTeX}}
%
%    \begin{macro}{\HoLogo@MiKTeX}
%    \begin{macrocode}
\def\HoLogo@MiKTeX#1{%
  \HOLOGO@mbox{MiK}%
  \HOLOGO@discretionary
  \hologo{TeX}%
}
%    \end{macrocode}
%    \end{macro}
%    \begin{macro}{\HoLogoHtml@MiKTeX}
%    \begin{macrocode}
\let\HoLogoHtml@MiKTeX\HoLogo@MiKTeX
%    \end{macrocode}
%    \end{macro}
%
% \subsubsection{\hologo{OzTeX} and friends}
%
%    Source: \hologo{OzTeX} FAQ \cite{OzTeX}:
%    \begin{quote}
%      |\def\OzTeX{O\kern-.03em z\kern-.15em\TeX}|\\
%      (There is no kerning in OzMF, OzMP and OzTtH.)
%    \end{quote}
%
%    \begin{macro}{\HoLogo@OzTeX}
%    \begin{macrocode}
\def\HoLogo@OzTeX#1{%
  O%
  \kern-.03em %
  z%
  \kern-.15em %
  \hologo{TeX}%
}
%    \end{macrocode}
%    \end{macro}
%    \begin{macro}{\HoLogoHtml@OzTeX}
%    \begin{macrocode}
\def\HoLogoHtml@OzTeX#1{%
  \HoLogoCss@OzTeX
  \HOLOGO@Span{OzTeX}{%
    O%
    \HOLOGO@Span{z}{z}%
    \hologo{TeX}%
  }%
}
%    \end{macrocode}
%    \end{macro}
%    \begin{macro}{\HoLogoCss@OzTeX}
%    \begin{macrocode}
\def\HoLogoCss@OzTeX{%
  \Css{%
    span.HoLogo-OzTeX span.HoLogo-z{%
      margin-left:-.03em;%
      margin-right:-.15em;%
    }%
  }%
  \global\let\HoLogoCss@OzTeX\relax
}
%    \end{macrocode}
%    \end{macro}
%
%    \begin{macro}{\HoLogo@OzMF}
%    \begin{macrocode}
\def\HoLogo@OzMF#1{%
  \HOLOGO@mbox{OzMF}%
}
%    \end{macrocode}
%    \end{macro}
%    \begin{macro}{\HoLogo@OzMP}
%    \begin{macrocode}
\def\HoLogo@OzMP#1{%
  \HOLOGO@mbox{OzMP}%
}
%    \end{macrocode}
%    \end{macro}
%    \begin{macro}{\HoLogo@OzTtH}
%    \begin{macrocode}
\def\HoLogo@OzTtH#1{%
  \HOLOGO@mbox{OzTtH}%
}
%    \end{macrocode}
%    \end{macro}
%
% \subsubsection{\hologo{PCTeX}}
%
%    \begin{macro}{\HoLogo@PCTeX}
%    \begin{macrocode}
\def\HoLogo@PCTeX#1{%
  \HOLOGO@mbox{PC}%
  \hologo{TeX}%
}
%    \end{macrocode}
%    \end{macro}
%    \begin{macro}{\HoLogoHtml@PCTeX}
%    \begin{macrocode}
\let\HoLogoHtml@PCTeX\HoLogo@PCTeX
%    \end{macrocode}
%    \end{macro}
%
% \subsubsection{\hologo{PiCTeX}}
%
%    The original definitions from \xfile{pictex.tex} \cite{PiCTeX}:
%\begin{quote}
%\begin{verbatim}
%\def\PiC{%
%  P%
%  \kern-.12em%
%  \lower.5ex\hbox{I}%
%  \kern-.075em%
%  C%
%}
%\def\PiCTeX{%
%  \PiC
%  \kern-.11em%
%  \TeX
%}
%\end{verbatim}
%\end{quote}
%
%    \begin{macro}{\HoLogo@PiC}
%    \begin{macrocode}
\def\HoLogo@PiC#1{%
  P%
  \kern-.12em%
  \lower.5ex\hbox{I}%
  \kern-.075em%
  C%
  \HOLOGO@SpaceFactor
}
%    \end{macrocode}
%    \end{macro}
%    \begin{macro}{\HoLogoHtml@PiC}
%    \begin{macrocode}
\def\HoLogoHtml@PiC#1{%
  \HoLogoCss@PiC
  \HOLOGO@Span{PiC}{%
    P%
    \HOLOGO@Span{i}{I}%
    C%
  }%
}
%    \end{macrocode}
%    \end{macro}
%    \begin{macro}{\HoLogoCss@PiC}
%    \begin{macrocode}
\def\HoLogoCss@PiC{%
  \Css{%
    span.HoLogo-PiC span.HoLogo-i{%
      position:relative;%
      top:.5ex;%
      margin-left:-.12em;%
      margin-right:-.075em;%
      text-decoration:none;%
    }%
  }%
  \global\let\HoLogoCss@PiC\relax
}
%    \end{macrocode}
%    \end{macro}
%
%    \begin{macro}{\HoLogo@PiCTeX}
%    \begin{macrocode}
\def\HoLogo@PiCTeX#1{%
  \hologo{PiC}%
  \HOLOGO@discretionary
  \kern-.11em%
  \hologo{TeX}%
}
%    \end{macrocode}
%    \end{macro}
%    \begin{macro}{\HoLogoHtml@PiCTeX}
%    \begin{macrocode}
\def\HoLogoHtml@PiCTeX#1{%
  \HoLogoCss@PiCTeX
  \HOLOGO@Span{PiCTeX}{%
    \hologo{PiC}%
    \hologo{TeX}%
  }%
}
%    \end{macrocode}
%    \end{macro}
%    \begin{macro}{\HoLogoCss@PiCTeX}
%    \begin{macrocode}
\def\HoLogoCss@PiCTeX{%
  \Css{%
    span.HoLogo-PiCTeX span.HoLogo-PiC{%
      margin-right:-.11em;%
    }%
  }%
  \global\let\HoLogoCss@PiCTeX\relax
}
%    \end{macrocode}
%    \end{macro}
%
% \subsubsection{\hologo{teTeX}}
%
%    \begin{macro}{\HoLogo@teTeX}
%    \begin{macrocode}
\def\HoLogo@teTeX#1{%
  \HOLOGO@mbox{#1{t}{T}e}%
  \HOLOGO@discretionary
  \hologo{TeX}%
}
%    \end{macrocode}
%    \end{macro}
%    \begin{macro}{\HoLogoCs@teTeX}
%    \begin{macrocode}
\def\HoLogoCs@teTeX#1{#1{t}{T}dfTeX}
%    \end{macrocode}
%    \end{macro}
%    \begin{macro}{\HoLogoBkm@teTeX}
%    \begin{macrocode}
\def\HoLogoBkm@teTeX#1{%
  #1{t}{T}e\hologo{TeX}%
}
%    \end{macrocode}
%    \end{macro}
%    \begin{macro}{\HoLogoHtml@teTeX}
%    \begin{macrocode}
\let\HoLogoHtml@teTeX\HoLogo@teTeX
%    \end{macrocode}
%    \end{macro}
%
% \subsubsection{\hologo{TeX4ht}}
%
%    \begin{macro}{\HoLogo@TeX4ht}
%    \begin{macrocode}
\expandafter\def\csname HoLogo@TeX4ht\endcsname#1{%
  \HOLOGO@mbox{\hologo{TeX}4ht}%
}
%    \end{macrocode}
%    \end{macro}
%    \begin{macro}{\HoLogoHtml@TeX4ht}
%    \begin{macrocode}
\expandafter
\let\csname HoLogoHtml@TeX4ht\expandafter\endcsname
\csname HoLogo@TeX4ht\endcsname
%    \end{macrocode}
%    \end{macro}
%
%
% \subsubsection{\hologo{SageTeX}}
%
%    \begin{macro}{\HoLogo@SageTeX}
%    \begin{macrocode}
\def\HoLogo@SageTeX#1{%
  \HOLOGO@mbox{Sage}%
  \HOLOGO@discretionary
  \HOLOGO@NegativeKerning{eT,oT,To}%
  \hologo{TeX}%
}
%    \end{macrocode}
%    \end{macro}
%    \begin{macro}{\HoLogoHtml@SageTeX}
%    \begin{macrocode}
\let\HoLogoHtml@SageTeX\HoLogo@SageTeX
%    \end{macrocode}
%    \end{macro}
%
% \subsection{\hologo{METAFONT} and friends}
%
%    \begin{macro}{\HoLogo@METAFONT}
%    \begin{macrocode}
\def\HoLogo@METAFONT#1{%
  \HoLogoFont@font{METAFONT}{logo}{%
    \HOLOGO@mbox{META}%
    \HOLOGO@discretionary
    \HOLOGO@mbox{FONT}%
  }%
}
%    \end{macrocode}
%    \end{macro}
%
%    \begin{macro}{\HoLogo@METAPOST}
%    \begin{macrocode}
\def\HoLogo@METAPOST#1{%
  \HoLogoFont@font{METAPOST}{logo}{%
    \HOLOGO@mbox{META}%
    \HOLOGO@discretionary
    \HOLOGO@mbox{POST}%
  }%
}
%    \end{macrocode}
%    \end{macro}
%
%    \begin{macro}{\HoLogo@MetaFun}
%    \begin{macrocode}
\def\HoLogo@MetaFun#1{%
  \HOLOGO@mbox{Meta}%
  \HOLOGO@discretionary
  \HOLOGO@mbox{Fun}%
}
%    \end{macrocode}
%    \end{macro}
%
%    \begin{macro}{\HoLogo@MetaPost}
%    \begin{macrocode}
\def\HoLogo@MetaPost#1{%
  \HOLOGO@mbox{Meta}%
  \HOLOGO@discretionary
  \HOLOGO@mbox{Post}%
}
%    \end{macrocode}
%    \end{macro}
%
% \subsection{Others}
%
% \subsubsection{\hologo{biber}}
%
%    \begin{macro}{\HoLogo@biber}
%    \begin{macrocode}
\def\HoLogo@biber#1{%
  \HOLOGO@mbox{#1{b}{B}i}%
  \HOLOGO@discretionary
  \HOLOGO@mbox{ber}%
}
%    \end{macrocode}
%    \end{macro}
%    \begin{macro}{\HoLogoCs@biber}
%    \begin{macrocode}
\def\HoLogoCs@biber#1{#1{b}{B}iber}
%    \end{macrocode}
%    \end{macro}
%    \begin{macro}{\HoLogoBkm@biber}
%    \begin{macrocode}
\def\HoLogoBkm@biber#1{%
  #1{b}{B}iber%
}
%    \end{macrocode}
%    \end{macro}
%    \begin{macro}{\HoLogoHtml@biber}
%    \begin{macrocode}
\let\HoLogoHtml@biber\HoLogo@biber
%    \end{macrocode}
%    \end{macro}
%
% \subsubsection{\hologo{KOMAScript}}
%
%    \begin{macro}{\HoLogo@KOMAScript}
%    The definition for \hologo{KOMAScript} is taken
%    from \hologo{KOMAScript} (\xfile{scrlogo.dtx}, reformatted) \cite{scrlogo}:
%\begin{quote}
%\begin{verbatim}
%\@ifundefined{KOMAScript}{%
%  \DeclareRobustCommand{\KOMAScript}{%
%    \textsf{%
%      K\kern.05em O\kern.05emM\kern.05em A%
%      \kern.1em-\kern.1em %
%      Script%
%    }%
%  }%
%}{}
%\end{verbatim}
%\end{quote}
%    \begin{macrocode}
\def\HoLogo@KOMAScript#1{%
  \HoLogoFont@font{KOMAScript}{sf}{%
    \HOLOGO@mbox{%
      K\kern.05em%
      O\kern.05em%
      M\kern.05em%
      A%
    }%
    \kern.1em%
    \HOLOGO@hyphen
    \kern.1em%
    \HOLOGO@mbox{Script}%
  }%
}
%    \end{macrocode}
%    \end{macro}
%    \begin{macro}{\HoLogoBkm@KOMAScript}
%    \begin{macrocode}
\def\HoLogoBkm@KOMAScript#1{%
  KOMA-Script%
}
%    \end{macrocode}
%    \end{macro}
%    \begin{macro}{\HoLogoHtml@KOMAScript}
%    \begin{macrocode}
\def\HoLogoHtml@KOMAScript#1{%
  \HoLogoCss@KOMAScript
  \HoLogoFont@font{KOMAScript}{sf}{%
    \HOLOGO@Span{KOMAScript}{%
      K%
      \HOLOGO@Span{O}{O}%
      M%
      \HOLOGO@Span{A}{A}%
      \HOLOGO@Span{hyphen}{-}%
      Script%
    }%
  }%
}
%    \end{macrocode}
%    \end{macro}
%    \begin{macro}{\HoLogoCss@KOMAScript}
%    \begin{macrocode}
\def\HoLogoCss@KOMAScript{%
  \Css{%
    span.HoLogo-KOMAScript{%
      font-family:sans-serif;%
    }%
  }%
  \Css{%
    span.HoLogo-KOMAScript span.HoLogo-O{%
      padding-left:.05em;%
      padding-right:.05em;%
    }%
  }%
  \Css{%
    span.HoLogo-KOMAScript span.HoLogo-A{%
      padding-left:.05em;%
    }%
  }%
  \Css{%
    span.HoLogo-KOMAScript span.HoLogo-hyphen{%
      padding-left:.1em;%
      padding-right:.1em;%
    }%
  }%
  \global\let\HoLogoCss@KOMAScript\relax
}
%    \end{macrocode}
%    \end{macro}
%
% \subsubsection{\hologo{LyX}}
%
%    \begin{macro}{\HoLogo@LyX}
%    The definition is taken from the documentation source files
%    of \hologo{LyX}, \xfile{Intro.lyx} \cite{LyX}:
%\begin{quote}
%\begin{verbatim}
%\def\LyX{%
%  \texorpdfstring{%
%    L\kern-.1667em\lower.25em\hbox{Y}\kern-.125emX\@%
%  }{%
%    LyX%
%  }%
%}
%\end{verbatim}
%\end{quote}
%    \begin{macrocode}
\def\HoLogo@LyX#1{%
  L%
  \kern-.1667em%
  \lower.25em\hbox{Y}%
  \kern-.125em%
  X%
  \HOLOGO@SpaceFactor
}
%    \end{macrocode}
%    \end{macro}
%    \begin{macro}{\HoLogoHtml@LyX}
%    \begin{macrocode}
\def\HoLogoHtml@LyX#1{%
  \HoLogoCss@LyX
  \HOLOGO@Span{LyX}{%
    L%
    \HOLOGO@Span{y}{Y}%
    X%
  }%
}
%    \end{macrocode}
%    \end{macro}
%    \begin{macro}{\HoLogoCss@LyX}
%    \begin{macrocode}
\def\HoLogoCss@LyX{%
  \Css{%
    span.HoLogo-LyX span.HoLogo-y{%
      position:relative;%
      top:.25em;%
      margin-left:-.1667em;%
      margin-right:-.125em;%
      text-decoration:none;%
    }%
  }%
  \global\let\HoLogoCss@LyX\relax
}
%    \end{macrocode}
%    \end{macro}
%
% \subsubsection{\hologo{NTS}}
%
%    \begin{macro}{\HoLogo@NTS}
%    Definition for \hologo{NTS} can be found in
%    package \xpackage{etex\textunderscore man} for the \hologo{eTeX} manual \cite{etexman}
%    and in package \xpackage{dtklogos} \cite{dtklogos}:
%\begin{quote}
%\begin{verbatim}
%\def\NTS{%
%  \leavevmode
%  \hbox{%
%    $%
%      \cal N%
%      \kern-0.35em%
%      \lower0.5ex\hbox{$\cal T$}%
%      \kern-0.2em%
%      S%
%    $%
%  }%
%}
%\end{verbatim}
%\end{quote}
%    \begin{macrocode}
\def\HoLogo@NTS#1{%
  \HoLogoFont@font{NTS}{sy}{%
    N\/%
    \kern-.35em%
    \lower.5ex\hbox{T\/}%
    \kern-.2em%
    S\/%
  }%
  \HOLOGO@SpaceFactor
}
%    \end{macrocode}
%    \end{macro}
%
% \subsubsection{\Hologo{TTH} (\hologo{TeX} to HTML translator)}
%
%    Source: \url{http://hutchinson.belmont.ma.us/tth/}
%    In the HTML source the second `T' is printed as subscript.
%\begin{quote}
%\begin{verbatim}
%T<sub>T</sub>H
%\end{verbatim}
%\end{quote}
%    \begin{macro}{\HoLogo@TTH}
%    \begin{macrocode}
\def\HoLogo@TTH#1{%
  \ltx@mbox{%
    T\HOLOGO@SubScript{T}H%
  }%
  \HOLOGO@SpaceFactor
}
%    \end{macrocode}
%    \end{macro}
%
%    \begin{macro}{\HoLogoHtml@TTH}
%    \begin{macrocode}
\def\HoLogoHtml@TTH#1{%
  T\HCode{<sub>}T\HCode{</sub>}H%
}
%    \end{macrocode}
%    \end{macro}
%
% \subsubsection{\Hologo{HanTheThanh}}
%
%    Partial source: Package \xpackage{dtklogos}.
%    The double accent is U+1EBF (latin small letter e with circumflex
%    and acute).
%    \begin{macro}{\HoLogo@HanTheThanh}
%    \begin{macrocode}
\def\HoLogo@HanTheThanh#1{%
  \ltx@mbox{H\`an}%
  \HOLOGO@space
  \ltx@mbox{%
    Th%
    \HOLOGO@IfCharExists{"1EBF}{%
      \char"1EBF\relax
    }{%
      \^e\hbox to 0pt{\hss\raise .5ex\hbox{\'{}}}%
    }%
  }%
  \HOLOGO@space
  \ltx@mbox{Th\`anh}%
}
%    \end{macrocode}
%    \end{macro}
%    \begin{macro}{\HoLogoBkm@HanTheThanh}
%    \begin{macrocode}
\def\HoLogoBkm@HanTheThanh#1{%
  H\`an %
  Th\HOLOGO@PdfdocUnicode{\^e}{\9036\277} %
  Th\`anh%
}
%    \end{macrocode}
%    \end{macro}
%    \begin{macro}{\HoLogoHtml@HanTheThanh}
%    \begin{macrocode}
\def\HoLogoHtml@HanTheThanh#1{%
  H\`an %
  Th\HCode{&\ltx@hashchar x1ebf;} %
  Th\`anh%
}
%    \end{macrocode}
%    \end{macro}
%
% \subsection{Driver detection}
%
%    \begin{macrocode}
\HOLOGO@IfExists\InputIfFileExists{%
  \InputIfFileExists{hologo.cfg}{}{}%
}{%
  \ltx@IfUndefined{pdf@filesize}{%
    \def\HOLOGO@InputIfExists{%
      \openin\HOLOGO@temp=hologo.cfg\relax
      \ifeof\HOLOGO@temp
        \closein\HOLOGO@temp
      \else
        \closein\HOLOGO@temp
        \begingroup
          \def\x{LaTeX2e}%
        \expandafter\endgroup
        \ifx\fmtname\x
          \input{hologo.cfg}%
        \else
          \input hologo.cfg\relax
        \fi
      \fi
    }%
    \ltx@IfUndefined{newread}{%
      \chardef\HOLOGO@temp=15 %
      \def\HOLOGO@CheckRead{%
        \ifeof\HOLOGO@temp
          \HOLOGO@InputIfExists
        \else
          \ifcase\HOLOGO@temp
            \@PackageWarningNoLine{hologo}{%
              Configuration file ignored, because\MessageBreak
              a free read register could not be found%
            }%
          \else
            \begingroup
              \count\ltx@cclv=\HOLOGO@temp
              \advance\ltx@cclv by \ltx@minusone
              \edef\x{\endgroup
                \chardef\noexpand\HOLOGO@temp=\the\count\ltx@cclv
                \relax
              }%
            \x
          \fi
        \fi
      }%
    }{%
      \csname newread\endcsname\HOLOGO@temp
      \HOLOGO@InputIfExists
    }%
  }{%
    \edef\HOLOGO@temp{\pdf@filesize{hologo.cfg}}%
    \ifx\HOLOGO@temp\ltx@empty
    \else
      \ifnum\HOLOGO@temp>0 %
        \begingroup
          \def\x{LaTeX2e}%
        \expandafter\endgroup
        \ifx\fmtname\x
          \input{hologo.cfg}%
        \else
          \input hologo.cfg\relax
        \fi
      \else
        \@PackageInfoNoLine{hologo}{%
          Empty configuration file `hologo.cfg' ignored%
        }%
      \fi
    \fi
  }%
}
%    \end{macrocode}
%
%    \begin{macrocode}
\def\HOLOGO@temp#1#2{%
  \kv@define@key{HoLogoDriver}{#1}[]{%
    \begingroup
      \def\HOLOGO@temp{##1}%
      \ltx@onelevel@sanitize\HOLOGO@temp
      \ifx\HOLOGO@temp\ltx@empty
      \else
        \@PackageError{hologo}{%
          Value (\HOLOGO@temp) not permitted for option `#1'%
        }%
        \@ehc
      \fi
    \endgroup
    \def\hologoDriver{#2}%
  }%
}%
\def\HOLOGO@@temp#1#2{%
  \ifx\kv@value\relax
    \HOLOGO@temp{#1}{#1}%
  \else
    \HOLOGO@temp{#1}{#2}%
  \fi
}%
\kv@parse@normalized{%
  pdftex,%
  luatex=pdftex,%
  dvipdfm,%
  dvipdfmx=dvipdfm,%
  dvips,%
  dvipsone=dvips,%
  xdvi=dvips,%
  xetex,%
  vtex,%
}\HOLOGO@@temp
%    \end{macrocode}
%
%    \begin{macrocode}
\kv@define@key{HoLogoDriver}{driverfallback}{%
  \def\HOLOGO@DriverFallback{#1}%
}
%    \end{macrocode}
%
%    \begin{macro}{\HOLOGO@DriverFallback}
%    \begin{macrocode}
\def\HOLOGO@DriverFallback{dvips}
%    \end{macrocode}
%    \end{macro}
%
%    \begin{macro}{\hologoDriverSetup}
%    \begin{macrocode}
\def\hologoDriverSetup{%
  \let\hologoDriver\ltx@undefined
  \HOLOGO@DriverSetup
}
%    \end{macrocode}
%    \end{macro}
%
%    \begin{macro}{\HOLOGO@DriverSetup}
%    \begin{macrocode}
\def\HOLOGO@DriverSetup#1{%
  \kvsetkeys{HoLogoDriver}{#1}%
  \HOLOGO@CheckDriver
  \ltx@ifundefined{hologoDriver}{%
    \begingroup
    \edef\x{\endgroup
      \noexpand\kvsetkeys{HoLogoDriver}{\HOLOGO@DriverFallback}%
    }\x
  }{}%
  \@PackageInfoNoLine{hologo}{Using driver `\hologoDriver'}%
}
%    \end{macrocode}
%    \end{macro}
%
%    \begin{macro}{\HOLOGO@CheckDriver}
%    \begin{macrocode}
\def\HOLOGO@CheckDriver{%
  \ifpdf
    \def\hologoDriver{pdftex}%
    \let\HOLOGO@pdfliteral\pdfliteral
    \ifluatex
      \ifx\pdfextension\@undefined\else
        \protected\def\pdfliteral{\pdfextension literal}%
        \let\HOLOGO@pdfliteral\pdfliteral
      \fi
      \ltx@IfUndefined{HOLOGO@pdfliteral}{%
        \ifnum\luatexversion<36 %
        \else
          \begingroup
            \let\HOLOGO@temp\endgroup
            \ifcase0%
                \directlua{%
                  if tex.enableprimitives then %
                    tex.enableprimitives('HOLOGO@', {'pdfliteral'})%
                  else %
                    tex.print('1')%
                  end%
                }%
                \ifx\HOLOGO@pdfliteral\@undefined 1\fi%
                \relax%
              \endgroup
              \let\HOLOGO@temp\relax
              \global\let\HOLOGO@pdfliteral\HOLOGO@pdfliteral
            \fi%
          \HOLOGO@temp
        \fi
      }{}%
    \fi
    \ltx@IfUndefined{HOLOGO@pdfliteral}{%
      \@PackageWarningNoLine{hologo}{%
        Cannot find \string\pdfliteral
      }%
    }{}%
  \else
    \ifxetex
      \def\hologoDriver{xetex}%
    \else
      \ifvtex
        \def\hologoDriver{vtex}%
      \fi
    \fi
  \fi
}
%    \end{macrocode}
%    \end{macro}
%
%    \begin{macro}{\HOLOGO@WarningUnsupportedDriver}
%    \begin{macrocode}
\def\HOLOGO@WarningUnsupportedDriver#1{%
  \@PackageWarningNoLine{hologo}{%
    Logo `#1' needs driver specific macros,\MessageBreak
    but driver `\hologoDriver' is not supported.\MessageBreak
    Use a different driver or\MessageBreak
    load package `graphics' or `pgf'%
  }%
}
%    \end{macrocode}
%    \end{macro}
%
% \subsubsection{Reflect box macros}
%
%    Skip driver part if not needed.
%    \begin{macrocode}
\ltx@IfUndefined{reflectbox}{}{%
  \ltx@IfUndefined{rotatebox}{}{%
    \HOLOGO@AtEnd
  }%
}
\ltx@IfUndefined{pgftext}{}{%
  \HOLOGO@AtEnd
}
\ltx@IfUndefined{psscalebox}{}{%
  \HOLOGO@AtEnd
}
%    \end{macrocode}
%
%    \begin{macrocode}
\def\HOLOGO@temp{LaTeX2e}
\ifx\fmtname\HOLOGO@temp
  \RequirePackage{kvoptions}[2011/06/30]%
  \ProcessKeyvalOptions{HoLogoDriver}%
\fi
\HOLOGO@DriverSetup{}
%    \end{macrocode}
%
%    \begin{macro}{\HOLOGO@ReflectBox}
%    \begin{macrocode}
\def\HOLOGO@ReflectBox#1{%
  \begingroup
    \setbox\ltx@zero\hbox{\begingroup#1\endgroup}%
    \setbox\ltx@two\hbox{%
      \kern\wd\ltx@zero
      \csname HOLOGO@ScaleBox@\hologoDriver\endcsname{-1}{1}{%
        \hbox to 0pt{\copy\ltx@zero\hss}%
      }%
    }%
    \wd\ltx@two=\wd\ltx@zero
    \box\ltx@two
  \endgroup
}
%    \end{macrocode}
%    \end{macro}
%
%    \begin{macro}{\HOLOGO@PointReflectBox}
%    \begin{macrocode}
\def\HOLOGO@PointReflectBox#1{%
  \begingroup
    \setbox\ltx@zero\hbox{\begingroup#1\endgroup}%
    \setbox\ltx@two\hbox{%
      \kern\wd\ltx@zero
      \raise\ht\ltx@zero\hbox{%
        \csname HOLOGO@ScaleBox@\hologoDriver\endcsname{-1}{-1}{%
          \hbox to 0pt{\copy\ltx@zero\hss}%
        }%
      }%
    }%
    \wd\ltx@two=\wd\ltx@zero
    \box\ltx@two
  \endgroup
}
%    \end{macrocode}
%    \end{macro}
%
%    We must define all variants because of dynamic driver setup.
%    \begin{macrocode}
\def\HOLOGO@temp#1#2{#2}
%    \end{macrocode}
%
%    \begin{macro}{\HOLOGO@ScaleBox@pdftex}
%    \begin{macrocode}
\HOLOGO@temp{pdftex}{%
  \def\HOLOGO@ScaleBox@pdftex#1#2#3{%
    \HOLOGO@pdfliteral{%
      q #1 0 0 #2 0 0 cm%
    }%
    #3%
    \HOLOGO@pdfliteral{%
      Q%
    }%
  }%
}
%    \end{macrocode}
%    \end{macro}
%    \begin{macro}{\HOLOGO@ScaleBox@dvips}
%    \begin{macrocode}
\HOLOGO@temp{dvips}{%
  \def\HOLOGO@ScaleBox@dvips#1#2#3{%
    \special{ps:%
      gsave %
      currentpoint %
      currentpoint translate %
      #1 #2 scale %
      neg exch neg exch translate%
    }%
    #3%
    \special{ps:%
      currentpoint %
      grestore %
      moveto%
    }%
  }%
}
%    \end{macrocode}
%    \end{macro}
%    \begin{macro}{\HOLOGO@ScaleBox@dvipdfm}
%    \begin{macrocode}
\HOLOGO@temp{dvipdfm}{%
  \let\HOLOGO@ScaleBox@dvipdfm\HOLOGO@ScaleBox@dvips
}
%    \end{macrocode}
%    \end{macro}
%    Since \hologo{XeTeX} v0.6.
%    \begin{macro}{\HOLOGO@ScaleBox@xetex}
%    \begin{macrocode}
\HOLOGO@temp{xetex}{%
  \def\HOLOGO@ScaleBox@xetex#1#2#3{%
    \special{x:gsave}%
    \special{x:scale #1 #2}%
    #3%
    \special{x:grestore}%
  }%
}
%    \end{macrocode}
%    \end{macro}
%    \begin{macro}{\HOLOGO@ScaleBox@vtex}
%    \begin{macrocode}
\HOLOGO@temp{vtex}{%
  \def\HOLOGO@ScaleBox@vtex#1#2#3{%
    \special{r(#1,0,0,#2,0,0}%
    #3%
    \special{r)}%
  }%
}
%    \end{macrocode}
%    \end{macro}
%
%    \begin{macrocode}
\HOLOGO@AtEnd%
%</package>
%    \end{macrocode}
%
% \section{Test}
%
% \subsection{Catcode checks for loading}
%
%    \begin{macrocode}
%<*test1>
%    \end{macrocode}
%    \begin{macrocode}
\catcode`\{=1 %
\catcode`\}=2 %
\catcode`\#=6 %
\catcode`\@=11 %
\expandafter\ifx\csname count@\endcsname\relax
  \countdef\count@=255 %
\fi
\expandafter\ifx\csname @gobble\endcsname\relax
  \long\def\@gobble#1{}%
\fi
\expandafter\ifx\csname @firstofone\endcsname\relax
  \long\def\@firstofone#1{#1}%
\fi
\expandafter\ifx\csname loop\endcsname\relax
  \expandafter\@firstofone
\else
  \expandafter\@gobble
\fi
{%
  \def\loop#1\repeat{%
    \def\body{#1}%
    \iterate
  }%
  \def\iterate{%
    \body
      \let\next\iterate
    \else
      \let\next\relax
    \fi
    \next
  }%
  \let\repeat=\fi
}%
\def\RestoreCatcodes{}
\count@=0 %
\loop
  \edef\RestoreCatcodes{%
    \RestoreCatcodes
    \catcode\the\count@=\the\catcode\count@\relax
  }%
\ifnum\count@<255 %
  \advance\count@ 1 %
\repeat

\def\RangeCatcodeInvalid#1#2{%
  \count@=#1\relax
  \loop
    \catcode\count@=15 %
  \ifnum\count@<#2\relax
    \advance\count@ 1 %
  \repeat
}
\def\RangeCatcodeCheck#1#2#3{%
  \count@=#1\relax
  \loop
    \ifnum#3=\catcode\count@
    \else
      \errmessage{%
        Character \the\count@\space
        with wrong catcode \the\catcode\count@\space
        instead of \number#3%
      }%
    \fi
  \ifnum\count@<#2\relax
    \advance\count@ 1 %
  \repeat
}
\def\space{ }
\expandafter\ifx\csname LoadCommand\endcsname\relax
  \def\LoadCommand{\input hologo.sty\relax}%
\fi
\def\Test{%
  \RangeCatcodeInvalid{0}{47}%
  \RangeCatcodeInvalid{58}{64}%
  \RangeCatcodeInvalid{91}{96}%
  \RangeCatcodeInvalid{123}{255}%
  \catcode`\@=12 %
  \catcode`\\=0 %
  \catcode`\%=14 %
  \LoadCommand
  \RangeCatcodeCheck{0}{36}{15}%
  \RangeCatcodeCheck{37}{37}{14}%
  \RangeCatcodeCheck{38}{47}{15}%
  \RangeCatcodeCheck{48}{57}{12}%
  \RangeCatcodeCheck{58}{63}{15}%
  \RangeCatcodeCheck{64}{64}{12}%
  \RangeCatcodeCheck{65}{90}{11}%
  \RangeCatcodeCheck{91}{91}{15}%
  \RangeCatcodeCheck{92}{92}{0}%
  \RangeCatcodeCheck{93}{96}{15}%
  \RangeCatcodeCheck{97}{122}{11}%
  \RangeCatcodeCheck{123}{255}{15}%
  \RestoreCatcodes
}
\Test
\csname @@end\endcsname
\end
%    \end{macrocode}
%    \begin{macrocode}
%</test1>
%    \end{macrocode}
%
% \subsection{Spacefactor}
%
%    The space factor must be 1000 after a logo. If it is greater 1000
%    then the following space is a space after a sentence closing point.
%    If the space factor is smaller 1000 then an immediate following
%    dot is interpreted as abbreviation, not sentence closing point.
%
%    \begin{macrocode}
%<*test-spacefactor>
\NeedsTeXFormat{LaTeX2e}
\documentclass{article}
\usepackage{hologo}[2016/05/12]
\usepackage{kvsetkeys}
\usepackage{qstest}
\IncludeTests{*}
\LogTests{log}{*}{*}
\begin{document}
\begin{qstest}{spacefactor}{spacefactor}
\newcommand*{\Test}[1]{%
  \sbox0{%
    \hologo{#1}%
    \Expect*{1000 (#1)}*{\the\spacefactor\space(#1)}%
  }%
}%
\makeatletter
\def\TestList{}
\def\hologoEntry#1#2#3{%
  \edef\TestList{%
    \ifx\TestList\@empty
    \else
      \TestList,%
    \fi
    #1%
    \ifx\\#2\\%
    \else
      ={variant=#2}%
    \fi
  }%
}
\hologoList
\expandafter\kv@parse@normalized\expandafter{%
  \TestList
}{%
  \begingroup
    \let\@logo=\kv@key
    \ifx\kv@value\relax
    \else
      \expandafter\hologoLogoSetup\expandafter\@logo\expandafter{%
        \kv@value
      }%
    \fi
    \Test\@logo
  \endgroup
  \@gobbletwo
}
\end{qstest}
\end{document}
%</test-spacefactor>
%    \end{macrocode}
%
% \subsection{Complete list}
%
%    \begin{macrocode}
%<*test-list>
\NeedsTeXFormat{LaTeX2e}
\documentclass[12pt,a4paper]{article}
\usepackage{hologo}[2016/05/12]
\usepackage[T1]{fontenc}
\usepackage{lmodern}
\usepackage{parskip}
\usepackage[unicode]{hyperref}[2011/09/28]
\usepackage{bookmark}[2011/09/19]
\bookmarksetup{%
  numbered,%
  open,%
  openlevel=2,%
}
\renewcommand*{\contentsname}{List of logos}
\begin{document}
\tableofcontents
\def\TestFont#1#2#3#4#5#6{%
  \begingroup
    \usefont{#3}{#4}{#5}{#6}%
    \HologoVariant{#1}{#2}/\hologoVariant{#1}{#2}%
    \quad
    \begingroup\scriptsize\hologoVariant{#1}{#2}\endgroup
    \quad
  \endgroup
  (#3/#4/#5/#6)%
  \par
}
\makeatletter
\def\hologoEntry#1#2#3{%
  \section{%
    \HologoVariant{#1}{#2}/\hologoVariant{#1}{#2} %
    {[#1\ifx\\#2\\\else\space(#2)\fi]}% hash-ok
  }% braces around [] because of bug in tex4ht
  \begingroup
    \hypersetup{unicode=false}%
    \bookmark[%
      dest=\@currentHref,%
      rellevel=1,%
      keeplevel,%
    ]{%
      \HologoVariant{#1}{#2}/\hologoVariant{#1}{#2} %
      (PDFDocEncoding)%
    }%
  \endgroup
  \TestFont{#1}{#2}{OT1}{cmr}{m}{n}%
  \TestFont{#1}{#2}{OT1}{cmss}{m}{n}%
  \TestFont{#1}{#2}{OT1}{cmr}{b}{n}%
  \TestFont{#1}{#2}{OT1}{cmr}{m}{it}%
  \TestFont{#1}{#2}{OT1}{cmtt}{m}{n}%
  \TestFont{#1}{#2}{T1}{lmr}{m}{n}%
  \TestFont{#1}{#2}{T1}{lmss}{m}{n}%
  \TestFont{#1}{#2}{T1}{lmr}{b}{n}%
  \TestFont{#1}{#2}{T1}{lmr}{m}{it}%
  \TestFont{#1}{#2}{T1}{lmtt}{m}{n}%
  \TestFont{#1}{#2}{T1}{lmvtt}{m}{n}%
  \TestFont{#1}{#2}{T1}{qtm}{m}{n}%
  \TestFont{#1}{#2}{T1}{qhv}{m}{n}%
  \TestFont{#1}{#2}{T1}{qtm}{b}{n}%
  \TestFont{#1}{#2}{T1}{qtm}{m}{it}%
  \TestFont{#1}{#2}{T1}{qcr}{m}{n}%
  \newpage
}
\makeatother
\hologoList
\end{document}
%</test-list>
%    \end{macrocode}
%
% \section{Installation}
%
% \subsection{Download}
%
% \paragraph{Package.} This package is available on
% CTAN\footnote{\url{ftp://ftp.ctan.org/tex-archive/}}:
% \begin{description}
% \item[\CTAN{macros/latex/contrib/oberdiek/hologo.dtx}] The source file.
% \item[\CTAN{macros/latex/contrib/oberdiek/hologo.pdf}] Documentation.
% \end{description}
%
%
% \paragraph{Bundle.} All the packages of the bundle `oberdiek'
% are also available in a TDS compliant ZIP archive. There
% the packages are already unpacked and the documentation files
% are generated. The files and directories obey the TDS standard.
% \begin{description}
% \item[\CTAN{install/macros/latex/contrib/oberdiek.tds.zip}]
% \end{description}
% \emph{TDS} refers to the standard ``A Directory Structure
% for \TeX\ Files'' (\CTAN{tds/tds.pdf}). Directories
% with \xfile{texmf} in their name are usually organized this way.
%
% \subsection{Bundle installation}
%
% \paragraph{Unpacking.} Unpack the \xfile{oberdiek.tds.zip} in the
% TDS tree (also known as \xfile{texmf} tree) of your choice.
% Example (linux):
% \begin{quote}
%   |unzip oberdiek.tds.zip -d ~/texmf|
% \end{quote}
%
% \paragraph{Script installation.}
% Check the directory \xfile{TDS:scripts/oberdiek/} for
% scripts that need further installation steps.
% Package \xpackage{attachfile2} comes with the Perl script
% \xfile{pdfatfi.pl} that should be installed in such a way
% that it can be called as \texttt{pdfatfi}.
% Example (linux):
% \begin{quote}
%   |chmod +x scripts/oberdiek/pdfatfi.pl|\\
%   |cp scripts/oberdiek/pdfatfi.pl /usr/local/bin/|
% \end{quote}
%
% \subsection{Package installation}
%
% \paragraph{Unpacking.} The \xfile{.dtx} file is a self-extracting
% \docstrip\ archive. The files are extracted by running the
% \xfile{.dtx} through \plainTeX:
% \begin{quote}
%   \verb|tex hologo.dtx|
% \end{quote}
%
% \paragraph{TDS.} Now the different files must be moved into
% the different directories in your installation TDS tree
% (also known as \xfile{texmf} tree):
% \begin{quote}
% \def\t{^^A
% \begin{tabular}{@{}>{\ttfamily}l@{ $\rightarrow$ }>{\ttfamily}l@{}}
%   hologo.sty & tex/generic/oberdiek/hologo.sty\\
%   hologo.pdf & doc/latex/oberdiek/hologo.pdf\\
%   example/hologo-example.tex & doc/latex/oberdiek/example/hologo-example.tex\\
%   test/hologo-test1.tex & doc/latex/oberdiek/test/hologo-test1.tex\\
%   test/hologo-test-spacefactor.tex & doc/latex/oberdiek/test/hologo-test-spacefactor.tex\\
%   test/hologo-test-list.tex & doc/latex/oberdiek/test/hologo-test-list.tex\\
%   hologo.dtx & source/latex/oberdiek/hologo.dtx\\
% \end{tabular}^^A
% }^^A
% \sbox0{\t}^^A
% \ifdim\wd0>\linewidth
%   \begingroup
%     \advance\linewidth by\leftmargin
%     \advance\linewidth by\rightmargin
%   \edef\x{\endgroup
%     \def\noexpand\lw{\the\linewidth}^^A
%   }\x
%   \def\lwbox{^^A
%     \leavevmode
%     \hbox to \linewidth{^^A
%       \kern-\leftmargin\relax
%       \hss
%       \usebox0
%       \hss
%       \kern-\rightmargin\relax
%     }^^A
%   }^^A
%   \ifdim\wd0>\lw
%     \sbox0{\small\t}^^A
%     \ifdim\wd0>\linewidth
%       \ifdim\wd0>\lw
%         \sbox0{\footnotesize\t}^^A
%         \ifdim\wd0>\linewidth
%           \ifdim\wd0>\lw
%             \sbox0{\scriptsize\t}^^A
%             \ifdim\wd0>\linewidth
%               \ifdim\wd0>\lw
%                 \sbox0{\tiny\t}^^A
%                 \ifdim\wd0>\linewidth
%                   \lwbox
%                 \else
%                   \usebox0
%                 \fi
%               \else
%                 \lwbox
%               \fi
%             \else
%               \usebox0
%             \fi
%           \else
%             \lwbox
%           \fi
%         \else
%           \usebox0
%         \fi
%       \else
%         \lwbox
%       \fi
%     \else
%       \usebox0
%     \fi
%   \else
%     \lwbox
%   \fi
% \else
%   \usebox0
% \fi
% \end{quote}
% If you have a \xfile{docstrip.cfg} that configures and enables \docstrip's
% TDS installing feature, then some files can already be in the right
% place, see the documentation of \docstrip.
%
% \subsection{Refresh file name databases}
%
% If your \TeX~distribution
% (\teTeX, \mikTeX, \dots) relies on file name databases, you must refresh
% these. For example, \teTeX\ users run \verb|texhash| or
% \verb|mktexlsr|.
%
% \subsection{Some details for the interested}
%
% \paragraph{Attached source.}
%
% The PDF documentation on CTAN also includes the
% \xfile{.dtx} source file. It can be extracted by
% AcrobatReader 6 or higher. Another option is \textsf{pdftk},
% e.g. unpack the file into the current directory:
% \begin{quote}
%   \verb|pdftk hologo.pdf unpack_files output .|
% \end{quote}
%
% \paragraph{Unpacking with \LaTeX.}
% The \xfile{.dtx} chooses its action depending on the format:
% \begin{description}
% \item[\plainTeX:] Run \docstrip\ and extract the files.
% \item[\LaTeX:] Generate the documentation.
% \end{description}
% If you insist on using \LaTeX\ for \docstrip\ (really,
% \docstrip\ does not need \LaTeX), then inform the autodetect routine
% about your intention:
% \begin{quote}
%   \verb|latex \let\install=y\input{hologo.dtx}|
% \end{quote}
% Do not forget to quote the argument according to the demands
% of your shell.
%
% \paragraph{Generating the documentation.}
% You can use both the \xfile{.dtx} or the \xfile{.drv} to generate
% the documentation. The process can be configured by the
% configuration file \xfile{ltxdoc.cfg}. For instance, put this
% line into this file, if you want to have A4 as paper format:
% \begin{quote}
%   \verb|\PassOptionsToClass{a4paper}{article}|
% \end{quote}
% An example follows how to generate the
% documentation with pdf\LaTeX:
% \begin{quote}
%\begin{verbatim}
%pdflatex hologo.dtx
%makeindex -s gind.ist hologo.idx
%pdflatex hologo.dtx
%makeindex -s gind.ist hologo.idx
%pdflatex hologo.dtx
%\end{verbatim}
% \end{quote}
%
% \section{Catalogue}
%
% The following XML file can be used as source for the
% \href{http://mirror.ctan.org/help/Catalogue/catalogue.html}{\TeX\ Catalogue}.
% The elements \texttt{caption} and \texttt{description} are imported
% from the original XML file from the Catalogue.
% The name of the XML file in the Catalogue is \xfile{hologo.xml}.
%    \begin{macrocode}
%<*catalogue>
<?xml version='1.0' encoding='us-ascii'?>
<!DOCTYPE entry SYSTEM 'catalogue.dtd'>
<entry datestamp='$Date$' modifier='$Author$' id='hologo'>
  <name>hologo</name>
  <caption>A collection of logos with bookmark support.</caption>
  <authorref id='auth:oberdiek'/>
  <copyright owner='Heiko Oberdiek' year='2010-2012'/>
  <license type='lppl1.3'/>
  <version number='1.10'/>
  <description>
    The package defines a single command <tt>\hologo</tt>, whose
    argument is the usual case-confused ASCII version of the logo.
    The command is bookmark-enabled, so that every logo becomes
    available in bookmarks without further work.
    <p/>
    The package is part of the <xref refid='oberdiek'>oberdiek</xref>
    bundle.
  </description>
  <documentation details='Package documentation'
      href='ctan:/macros/latex/contrib/oberdiek/hologo.pdf'/>
  <ctan file='true' path='/macros/latex/contrib/oberdiek/hologo.dtx'/>
  <miktex location='oberdiek'/>
  <texlive location='oberdiek'/>
  <install path='/macros/latex/contrib/oberdiek/oberdiek.tds.zip'/>
</entry>
%</catalogue>
%    \end{macrocode}
%
% \begin{thebibliography}{9}
% \raggedright
%
% \bibitem{btxdoc}
% Oren Patashnik,
% \textit{\hologo{BibTeX}ing},
% 1988-02-08.\\
% \CTAN{biblio/bibtex/base/}
%
% \bibitem{dtklogos}
% Gerd Neugebauer, DANTE,
% \textit{Package \xpackage{dtklogos}},
% 2011-04-25.\\
% \CTAN{usergrps/dante/dtk/dtklogos.sty}
%
% \bibitem{etexman}
% The \hologo{NTS} Team,
% \textit{The \hologo{eTeX} manual},
% 1998-02.\\
% \CTAN{systems/e-tex/v2/doc/}
%
% \bibitem{ExTeX-FAQ}
% The \hologo{ExTeX} group,
% \textit{\hologo{ExTeX}: FAQ -- How is \hologo{ExTeX} typeset?},
% 2007-04-14.\\
% \url{http://www.extex.org/documentation/faq.html}
%
% \bibitem{LyX}
% %@MISC{ LyX,
% %  title = {{LyX 2.0.0 -- The Document Processor [Computer software and manual]}},
% %  author = {{The LyX Team}},
% %  howpublished = {Internet: http://www.lyx.org},
% %  year = {2011-05-08},
% %  note = {Retrieved May 10, 2011, from http://www.lyx.org},
% %  url = {http://www.lyx.org/}
% %}
% The \hologo{LyX} Team,
% \textit{\hologo{LyX} -- The Document Processor},
% 2011-05-08.\\
% \url{http://www.lyx.org/}
%
% \bibitem{OzTeX}
% Andrew Trevorrow,
% \hologo{OzTeX} FAQ: What is the correct way to typeset ``\hologo{OzTeX}''?,
% 2011-09-15 (visited).
% \url{http://www.trevorrow.com/oztex/ozfaq.html#oztex-logo}
%
% \bibitem{PiCTeX}
% Michael Wichura,
% \textit{The \hologo{PiCTeX} macro package},
% 1987-09-21.
% \CTAN{graphics/pictex/}
%
% \bibitem{scrlogo}
% Markus Kohm,
% \textit{\hologo{KOMAScript} Datei \xfile{scrlogo.dtx}},
% 2009-01-30.\\
% \CTAN{install/macros/latex/contrib/komascript.tds.zip}
%
% \end{thebibliography}
%
% \begin{History}
%   \begin{Version}{2010/04/08 v1.0}
%   \item
%     The first version.
%   \end{Version}
%   \begin{Version}{2010/04/16 v1.1}
%   \item
%     \cs{Hologo} added for support of logos at start of a sentence.
%   \item
%     \cs{hologoSetup} and \cs{hologoLogoSetup} added.
%   \item
%     Options \xoption{break}, \xoption{hyphenbreak}, \xoption{spacebreak}
%     added.
%   \item
%     Variant support added by option \xoption{variant}.
%   \end{Version}
%   \begin{Version}{2010/04/24 v1.2}
%   \item
%     \hologo{LaTeX3} added.
%   \item
%     \hologo{VTeX} added.
%   \end{Version}
%   \begin{Version}{2010/11/21 v1.3}
%   \item
%     \hologo{iniTeX}, \hologo{virTeX} added.
%   \end{Version}
%   \begin{Version}{2011/03/25 v1.4}
%   \item
%     \hologo{ConTeXt} with variants added.
%   \item
%     Option \xoption{discretionarybreak} added as refinement for
%     option \xoption{break}.
%   \end{Version}
%   \begin{Version}{2011/04/21 v1.5}
%   \item
%     Wrong TDS directory for test files fixed.
%   \end{Version}
%   \begin{Version}{2011/10/01 v1.6}
%   \item
%     Support for package \xpackage{tex4ht} added.
%   \item
%     Support for \cs{csname} added if \cs{ifincsname} is available.
%   \item
%     New logos:
%     \hologo{(La)TeX},
%     \hologo{biber},
%     \hologo{BibTeX} (\xoption{sc}, \xoption{sf}),
%     \hologo{emTeX},
%     \hologo{ExTeX},
%     \hologo{KOMAScript},
%     \hologo{La},
%     \hologo{LyX},
%     \hologo{MiKTeX},
%     \hologo{NTS},
%     \hologo{OzMF},
%     \hologo{OzMP},
%     \hologo{OzTeX},
%     \hologo{OzTtH},
%     \hologo{PCTeX},
%     \hologo{PiC},
%     \hologo{PiCTeX},
%     \hologo{METAFONT},
%     \hologo{MetaFun},
%     \hologo{METAPOST},
%     \hologo{MetaPost},
%     \hologo{SLiTeX} (\xoption{lift}, \xoption{narrow}, \xoption{simple}),
%     \hologo{SliTeX} (\xoption{narrow}, \xoption{simple}, \xoption{lift}),
%     \hologo{teTeX}.
%   \item
%     Fixes:
%     \hologo{iniTeX},
%     \hologo{pdfLaTeX},
%     \hologo{pdfTeX},
%     \hologo{virTeX}.
%   \item
%     \cs{hologoFontSetup} and \cs{hologoLogoFontSetup} added.
%   \item
%     \cs{hologoVariant} and \cs{HologoVariant} added.
%   \end{Version}
%   \begin{Version}{2011/11/22 v1.7}
%   \item
%     New logos:
%     \hologo{BibTeX8},
%     \hologo{LaTeXML},
%     \hologo{SageTeX},
%     \hologo{TeX4ht},
%     \hologo{TTH}.
%   \item
%     \hologo{Xe} and friends: Driver stuff fixed.
%   \item
%     \hologo{Xe} and friends: Support for italic added.
%   \item
%     \hologo{Xe} and friends: Package support for \xpackage{pgf}
%     and \xpackage{pstricks} added.
%   \end{Version}
%   \begin{Version}{2011/11/29 v1.8}
%   \item
%     New logos:
%     \hologo{HanTheThanh}.
%   \end{Version}
%   \begin{Version}{2011/12/21 v1.9}
%   \item
%     Patch for package \xpackage{ifxetex} added for the case that
%     \cs{newif} is undefined in \hologo{iniTeX}.
%   \item
%     Some fixes for \hologo{iniTeX}.
%   \end{Version}
%   \begin{Version}{2012/04/26 v1.10}
%   \item
%     Fix in bookmark version of logo ``\hologo{HanTheThanh}''.
%   \end{Version}
%   \begin{Version}{2016/05/12 v1.11}
%   \item
%     Update HOLOGO@IfCharExists (previously in texlive)
%   \item define pdfliteral in current luatex.
%   \end{Version}
% \end{History}
%
% \PrintIndex
%
% \Finale
\endinput
%
        \else
          \input hologo.cfg\relax
        \fi
      \else
        \@PackageInfoNoLine{hologo}{%
          Empty configuration file `hologo.cfg' ignored%
        }%
      \fi
    \fi
  }%
}
%    \end{macrocode}
%
%    \begin{macrocode}
\def\HOLOGO@temp#1#2{%
  \kv@define@key{HoLogoDriver}{#1}[]{%
    \begingroup
      \def\HOLOGO@temp{##1}%
      \ltx@onelevel@sanitize\HOLOGO@temp
      \ifx\HOLOGO@temp\ltx@empty
      \else
        \@PackageError{hologo}{%
          Value (\HOLOGO@temp) not permitted for option `#1'%
        }%
        \@ehc
      \fi
    \endgroup
    \def\hologoDriver{#2}%
  }%
}%
\def\HOLOGO@@temp#1#2{%
  \ifx\kv@value\relax
    \HOLOGO@temp{#1}{#1}%
  \else
    \HOLOGO@temp{#1}{#2}%
  \fi
}%
\kv@parse@normalized{%
  pdftex,%
  luatex=pdftex,%
  dvipdfm,%
  dvipdfmx=dvipdfm,%
  dvips,%
  dvipsone=dvips,%
  xdvi=dvips,%
  xetex,%
  vtex,%
}\HOLOGO@@temp
%    \end{macrocode}
%
%    \begin{macrocode}
\kv@define@key{HoLogoDriver}{driverfallback}{%
  \def\HOLOGO@DriverFallback{#1}%
}
%    \end{macrocode}
%
%    \begin{macro}{\HOLOGO@DriverFallback}
%    \begin{macrocode}
\def\HOLOGO@DriverFallback{dvips}
%    \end{macrocode}
%    \end{macro}
%
%    \begin{macro}{\hologoDriverSetup}
%    \begin{macrocode}
\def\hologoDriverSetup{%
  \let\hologoDriver\ltx@undefined
  \HOLOGO@DriverSetup
}
%    \end{macrocode}
%    \end{macro}
%
%    \begin{macro}{\HOLOGO@DriverSetup}
%    \begin{macrocode}
\def\HOLOGO@DriverSetup#1{%
  \kvsetkeys{HoLogoDriver}{#1}%
  \HOLOGO@CheckDriver
  \ltx@ifundefined{hologoDriver}{%
    \begingroup
    \edef\x{\endgroup
      \noexpand\kvsetkeys{HoLogoDriver}{\HOLOGO@DriverFallback}%
    }\x
  }{}%
  \@PackageInfoNoLine{hologo}{Using driver `\hologoDriver'}%
}
%    \end{macrocode}
%    \end{macro}
%
%    \begin{macro}{\HOLOGO@CheckDriver}
%    \begin{macrocode}
\def\HOLOGO@CheckDriver{%
  \ifpdf
    \def\hologoDriver{pdftex}%
    \let\HOLOGO@pdfliteral\pdfliteral
    \ifluatex
      \ifx\pdfextension\@undefined\else
        \protected\def\pdfliteral{\pdfextension literal}%
        \let\HOLOGO@pdfliteral\pdfliteral
      \fi
      \ltx@IfUndefined{HOLOGO@pdfliteral}{%
        \ifnum\luatexversion<36 %
        \else
          \begingroup
            \let\HOLOGO@temp\endgroup
            \ifcase0%
                \directlua{%
                  if tex.enableprimitives then %
                    tex.enableprimitives('HOLOGO@', {'pdfliteral'})%
                  else %
                    tex.print('1')%
                  end%
                }%
                \ifx\HOLOGO@pdfliteral\@undefined 1\fi%
                \relax%
              \endgroup
              \let\HOLOGO@temp\relax
              \global\let\HOLOGO@pdfliteral\HOLOGO@pdfliteral
            \fi%
          \HOLOGO@temp
        \fi
      }{}%
    \fi
    \ltx@IfUndefined{HOLOGO@pdfliteral}{%
      \@PackageWarningNoLine{hologo}{%
        Cannot find \string\pdfliteral
      }%
    }{}%
  \else
    \ifxetex
      \def\hologoDriver{xetex}%
    \else
      \ifvtex
        \def\hologoDriver{vtex}%
      \fi
    \fi
  \fi
}
%    \end{macrocode}
%    \end{macro}
%
%    \begin{macro}{\HOLOGO@WarningUnsupportedDriver}
%    \begin{macrocode}
\def\HOLOGO@WarningUnsupportedDriver#1{%
  \@PackageWarningNoLine{hologo}{%
    Logo `#1' needs driver specific macros,\MessageBreak
    but driver `\hologoDriver' is not supported.\MessageBreak
    Use a different driver or\MessageBreak
    load package `graphics' or `pgf'%
  }%
}
%    \end{macrocode}
%    \end{macro}
%
% \subsubsection{Reflect box macros}
%
%    Skip driver part if not needed.
%    \begin{macrocode}
\ltx@IfUndefined{reflectbox}{}{%
  \ltx@IfUndefined{rotatebox}{}{%
    \HOLOGO@AtEnd
  }%
}
\ltx@IfUndefined{pgftext}{}{%
  \HOLOGO@AtEnd
}
\ltx@IfUndefined{psscalebox}{}{%
  \HOLOGO@AtEnd
}
%    \end{macrocode}
%
%    \begin{macrocode}
\def\HOLOGO@temp{LaTeX2e}
\ifx\fmtname\HOLOGO@temp
  \RequirePackage{kvoptions}[2011/06/30]%
  \ProcessKeyvalOptions{HoLogoDriver}%
\fi
\HOLOGO@DriverSetup{}
%    \end{macrocode}
%
%    \begin{macro}{\HOLOGO@ReflectBox}
%    \begin{macrocode}
\def\HOLOGO@ReflectBox#1{%
  \begingroup
    \setbox\ltx@zero\hbox{\begingroup#1\endgroup}%
    \setbox\ltx@two\hbox{%
      \kern\wd\ltx@zero
      \csname HOLOGO@ScaleBox@\hologoDriver\endcsname{-1}{1}{%
        \hbox to 0pt{\copy\ltx@zero\hss}%
      }%
    }%
    \wd\ltx@two=\wd\ltx@zero
    \box\ltx@two
  \endgroup
}
%    \end{macrocode}
%    \end{macro}
%
%    \begin{macro}{\HOLOGO@PointReflectBox}
%    \begin{macrocode}
\def\HOLOGO@PointReflectBox#1{%
  \begingroup
    \setbox\ltx@zero\hbox{\begingroup#1\endgroup}%
    \setbox\ltx@two\hbox{%
      \kern\wd\ltx@zero
      \raise\ht\ltx@zero\hbox{%
        \csname HOLOGO@ScaleBox@\hologoDriver\endcsname{-1}{-1}{%
          \hbox to 0pt{\copy\ltx@zero\hss}%
        }%
      }%
    }%
    \wd\ltx@two=\wd\ltx@zero
    \box\ltx@two
  \endgroup
}
%    \end{macrocode}
%    \end{macro}
%
%    We must define all variants because of dynamic driver setup.
%    \begin{macrocode}
\def\HOLOGO@temp#1#2{#2}
%    \end{macrocode}
%
%    \begin{macro}{\HOLOGO@ScaleBox@pdftex}
%    \begin{macrocode}
\HOLOGO@temp{pdftex}{%
  \def\HOLOGO@ScaleBox@pdftex#1#2#3{%
    \HOLOGO@pdfliteral{%
      q #1 0 0 #2 0 0 cm%
    }%
    #3%
    \HOLOGO@pdfliteral{%
      Q%
    }%
  }%
}
%    \end{macrocode}
%    \end{macro}
%    \begin{macro}{\HOLOGO@ScaleBox@dvips}
%    \begin{macrocode}
\HOLOGO@temp{dvips}{%
  \def\HOLOGO@ScaleBox@dvips#1#2#3{%
    \special{ps:%
      gsave %
      currentpoint %
      currentpoint translate %
      #1 #2 scale %
      neg exch neg exch translate%
    }%
    #3%
    \special{ps:%
      currentpoint %
      grestore %
      moveto%
    }%
  }%
}
%    \end{macrocode}
%    \end{macro}
%    \begin{macro}{\HOLOGO@ScaleBox@dvipdfm}
%    \begin{macrocode}
\HOLOGO@temp{dvipdfm}{%
  \let\HOLOGO@ScaleBox@dvipdfm\HOLOGO@ScaleBox@dvips
}
%    \end{macrocode}
%    \end{macro}
%    Since \hologo{XeTeX} v0.6.
%    \begin{macro}{\HOLOGO@ScaleBox@xetex}
%    \begin{macrocode}
\HOLOGO@temp{xetex}{%
  \def\HOLOGO@ScaleBox@xetex#1#2#3{%
    \special{x:gsave}%
    \special{x:scale #1 #2}%
    #3%
    \special{x:grestore}%
  }%
}
%    \end{macrocode}
%    \end{macro}
%    \begin{macro}{\HOLOGO@ScaleBox@vtex}
%    \begin{macrocode}
\HOLOGO@temp{vtex}{%
  \def\HOLOGO@ScaleBox@vtex#1#2#3{%
    \special{r(#1,0,0,#2,0,0}%
    #3%
    \special{r)}%
  }%
}
%    \end{macrocode}
%    \end{macro}
%
%    \begin{macrocode}
\HOLOGO@AtEnd%
%</package>
%    \end{macrocode}
%
% \section{Test}
%
% \subsection{Catcode checks for loading}
%
%    \begin{macrocode}
%<*test1>
%    \end{macrocode}
%    \begin{macrocode}
\catcode`\{=1 %
\catcode`\}=2 %
\catcode`\#=6 %
\catcode`\@=11 %
\expandafter\ifx\csname count@\endcsname\relax
  \countdef\count@=255 %
\fi
\expandafter\ifx\csname @gobble\endcsname\relax
  \long\def\@gobble#1{}%
\fi
\expandafter\ifx\csname @firstofone\endcsname\relax
  \long\def\@firstofone#1{#1}%
\fi
\expandafter\ifx\csname loop\endcsname\relax
  \expandafter\@firstofone
\else
  \expandafter\@gobble
\fi
{%
  \def\loop#1\repeat{%
    \def\body{#1}%
    \iterate
  }%
  \def\iterate{%
    \body
      \let\next\iterate
    \else
      \let\next\relax
    \fi
    \next
  }%
  \let\repeat=\fi
}%
\def\RestoreCatcodes{}
\count@=0 %
\loop
  \edef\RestoreCatcodes{%
    \RestoreCatcodes
    \catcode\the\count@=\the\catcode\count@\relax
  }%
\ifnum\count@<255 %
  \advance\count@ 1 %
\repeat

\def\RangeCatcodeInvalid#1#2{%
  \count@=#1\relax
  \loop
    \catcode\count@=15 %
  \ifnum\count@<#2\relax
    \advance\count@ 1 %
  \repeat
}
\def\RangeCatcodeCheck#1#2#3{%
  \count@=#1\relax
  \loop
    \ifnum#3=\catcode\count@
    \else
      \errmessage{%
        Character \the\count@\space
        with wrong catcode \the\catcode\count@\space
        instead of \number#3%
      }%
    \fi
  \ifnum\count@<#2\relax
    \advance\count@ 1 %
  \repeat
}
\def\space{ }
\expandafter\ifx\csname LoadCommand\endcsname\relax
  \def\LoadCommand{\input hologo.sty\relax}%
\fi
\def\Test{%
  \RangeCatcodeInvalid{0}{47}%
  \RangeCatcodeInvalid{58}{64}%
  \RangeCatcodeInvalid{91}{96}%
  \RangeCatcodeInvalid{123}{255}%
  \catcode`\@=12 %
  \catcode`\\=0 %
  \catcode`\%=14 %
  \LoadCommand
  \RangeCatcodeCheck{0}{36}{15}%
  \RangeCatcodeCheck{37}{37}{14}%
  \RangeCatcodeCheck{38}{47}{15}%
  \RangeCatcodeCheck{48}{57}{12}%
  \RangeCatcodeCheck{58}{63}{15}%
  \RangeCatcodeCheck{64}{64}{12}%
  \RangeCatcodeCheck{65}{90}{11}%
  \RangeCatcodeCheck{91}{91}{15}%
  \RangeCatcodeCheck{92}{92}{0}%
  \RangeCatcodeCheck{93}{96}{15}%
  \RangeCatcodeCheck{97}{122}{11}%
  \RangeCatcodeCheck{123}{255}{15}%
  \RestoreCatcodes
}
\Test
\csname @@end\endcsname
\end
%    \end{macrocode}
%    \begin{macrocode}
%</test1>
%    \end{macrocode}
%
% \subsection{Spacefactor}
%
%    The space factor must be 1000 after a logo. If it is greater 1000
%    then the following space is a space after a sentence closing point.
%    If the space factor is smaller 1000 then an immediate following
%    dot is interpreted as abbreviation, not sentence closing point.
%
%    \begin{macrocode}
%<*test-spacefactor>
\NeedsTeXFormat{LaTeX2e}
\documentclass{article}
\usepackage{hologo}[2016/05/12]
\usepackage{kvsetkeys}
\usepackage{qstest}
\IncludeTests{*}
\LogTests{log}{*}{*}
\begin{document}
\begin{qstest}{spacefactor}{spacefactor}
\newcommand*{\Test}[1]{%
  \sbox0{%
    \hologo{#1}%
    \Expect*{1000 (#1)}*{\the\spacefactor\space(#1)}%
  }%
}%
\makeatletter
\def\TestList{}
\def\hologoEntry#1#2#3{%
  \edef\TestList{%
    \ifx\TestList\@empty
    \else
      \TestList,%
    \fi
    #1%
    \ifx\\#2\\%
    \else
      ={variant=#2}%
    \fi
  }%
}
\hologoList
\expandafter\kv@parse@normalized\expandafter{%
  \TestList
}{%
  \begingroup
    \let\@logo=\kv@key
    \ifx\kv@value\relax
    \else
      \expandafter\hologoLogoSetup\expandafter\@logo\expandafter{%
        \kv@value
      }%
    \fi
    \Test\@logo
  \endgroup
  \@gobbletwo
}
\end{qstest}
\end{document}
%</test-spacefactor>
%    \end{macrocode}
%
% \subsection{Complete list}
%
%    \begin{macrocode}
%<*test-list>
\NeedsTeXFormat{LaTeX2e}
\documentclass[12pt,a4paper]{article}
\usepackage{hologo}[2016/05/12]
\usepackage[T1]{fontenc}
\usepackage{lmodern}
\usepackage{parskip}
\usepackage[unicode]{hyperref}[2011/09/28]
\usepackage{bookmark}[2011/09/19]
\bookmarksetup{%
  numbered,%
  open,%
  openlevel=2,%
}
\renewcommand*{\contentsname}{List of logos}
\begin{document}
\tableofcontents
\def\TestFont#1#2#3#4#5#6{%
  \begingroup
    \usefont{#3}{#4}{#5}{#6}%
    \HologoVariant{#1}{#2}/\hologoVariant{#1}{#2}%
    \quad
    \begingroup\scriptsize\hologoVariant{#1}{#2}\endgroup
    \quad
  \endgroup
  (#3/#4/#5/#6)%
  \par
}
\makeatletter
\def\hologoEntry#1#2#3{%
  \section{%
    \HologoVariant{#1}{#2}/\hologoVariant{#1}{#2} %
    {[#1\ifx\\#2\\\else\space(#2)\fi]}% hash-ok
  }% braces around [] because of bug in tex4ht
  \begingroup
    \hypersetup{unicode=false}%
    \bookmark[%
      dest=\@currentHref,%
      rellevel=1,%
      keeplevel,%
    ]{%
      \HologoVariant{#1}{#2}/\hologoVariant{#1}{#2} %
      (PDFDocEncoding)%
    }%
  \endgroup
  \TestFont{#1}{#2}{OT1}{cmr}{m}{n}%
  \TestFont{#1}{#2}{OT1}{cmss}{m}{n}%
  \TestFont{#1}{#2}{OT1}{cmr}{b}{n}%
  \TestFont{#1}{#2}{OT1}{cmr}{m}{it}%
  \TestFont{#1}{#2}{OT1}{cmtt}{m}{n}%
  \TestFont{#1}{#2}{T1}{lmr}{m}{n}%
  \TestFont{#1}{#2}{T1}{lmss}{m}{n}%
  \TestFont{#1}{#2}{T1}{lmr}{b}{n}%
  \TestFont{#1}{#2}{T1}{lmr}{m}{it}%
  \TestFont{#1}{#2}{T1}{lmtt}{m}{n}%
  \TestFont{#1}{#2}{T1}{lmvtt}{m}{n}%
  \TestFont{#1}{#2}{T1}{qtm}{m}{n}%
  \TestFont{#1}{#2}{T1}{qhv}{m}{n}%
  \TestFont{#1}{#2}{T1}{qtm}{b}{n}%
  \TestFont{#1}{#2}{T1}{qtm}{m}{it}%
  \TestFont{#1}{#2}{T1}{qcr}{m}{n}%
  \newpage
}
\makeatother
\hologoList
\end{document}
%</test-list>
%    \end{macrocode}
%
% \section{Installation}
%
% \subsection{Download}
%
% \paragraph{Package.} This package is available on
% CTAN\footnote{\url{ftp://ftp.ctan.org/tex-archive/}}:
% \begin{description}
% \item[\CTAN{macros/latex/contrib/oberdiek/hologo.dtx}] The source file.
% \item[\CTAN{macros/latex/contrib/oberdiek/hologo.pdf}] Documentation.
% \end{description}
%
%
% \paragraph{Bundle.} All the packages of the bundle `oberdiek'
% are also available in a TDS compliant ZIP archive. There
% the packages are already unpacked and the documentation files
% are generated. The files and directories obey the TDS standard.
% \begin{description}
% \item[\CTAN{install/macros/latex/contrib/oberdiek.tds.zip}]
% \end{description}
% \emph{TDS} refers to the standard ``A Directory Structure
% for \TeX\ Files'' (\CTAN{tds/tds.pdf}). Directories
% with \xfile{texmf} in their name are usually organized this way.
%
% \subsection{Bundle installation}
%
% \paragraph{Unpacking.} Unpack the \xfile{oberdiek.tds.zip} in the
% TDS tree (also known as \xfile{texmf} tree) of your choice.
% Example (linux):
% \begin{quote}
%   |unzip oberdiek.tds.zip -d ~/texmf|
% \end{quote}
%
% \paragraph{Script installation.}
% Check the directory \xfile{TDS:scripts/oberdiek/} for
% scripts that need further installation steps.
% Package \xpackage{attachfile2} comes with the Perl script
% \xfile{pdfatfi.pl} that should be installed in such a way
% that it can be called as \texttt{pdfatfi}.
% Example (linux):
% \begin{quote}
%   |chmod +x scripts/oberdiek/pdfatfi.pl|\\
%   |cp scripts/oberdiek/pdfatfi.pl /usr/local/bin/|
% \end{quote}
%
% \subsection{Package installation}
%
% \paragraph{Unpacking.} The \xfile{.dtx} file is a self-extracting
% \docstrip\ archive. The files are extracted by running the
% \xfile{.dtx} through \plainTeX:
% \begin{quote}
%   \verb|tex hologo.dtx|
% \end{quote}
%
% \paragraph{TDS.} Now the different files must be moved into
% the different directories in your installation TDS tree
% (also known as \xfile{texmf} tree):
% \begin{quote}
% \def\t{^^A
% \begin{tabular}{@{}>{\ttfamily}l@{ $\rightarrow$ }>{\ttfamily}l@{}}
%   hologo.sty & tex/generic/oberdiek/hologo.sty\\
%   hologo.pdf & doc/latex/oberdiek/hologo.pdf\\
%   example/hologo-example.tex & doc/latex/oberdiek/example/hologo-example.tex\\
%   test/hologo-test1.tex & doc/latex/oberdiek/test/hologo-test1.tex\\
%   test/hologo-test-spacefactor.tex & doc/latex/oberdiek/test/hologo-test-spacefactor.tex\\
%   test/hologo-test-list.tex & doc/latex/oberdiek/test/hologo-test-list.tex\\
%   hologo.dtx & source/latex/oberdiek/hologo.dtx\\
% \end{tabular}^^A
% }^^A
% \sbox0{\t}^^A
% \ifdim\wd0>\linewidth
%   \begingroup
%     \advance\linewidth by\leftmargin
%     \advance\linewidth by\rightmargin
%   \edef\x{\endgroup
%     \def\noexpand\lw{\the\linewidth}^^A
%   }\x
%   \def\lwbox{^^A
%     \leavevmode
%     \hbox to \linewidth{^^A
%       \kern-\leftmargin\relax
%       \hss
%       \usebox0
%       \hss
%       \kern-\rightmargin\relax
%     }^^A
%   }^^A
%   \ifdim\wd0>\lw
%     \sbox0{\small\t}^^A
%     \ifdim\wd0>\linewidth
%       \ifdim\wd0>\lw
%         \sbox0{\footnotesize\t}^^A
%         \ifdim\wd0>\linewidth
%           \ifdim\wd0>\lw
%             \sbox0{\scriptsize\t}^^A
%             \ifdim\wd0>\linewidth
%               \ifdim\wd0>\lw
%                 \sbox0{\tiny\t}^^A
%                 \ifdim\wd0>\linewidth
%                   \lwbox
%                 \else
%                   \usebox0
%                 \fi
%               \else
%                 \lwbox
%               \fi
%             \else
%               \usebox0
%             \fi
%           \else
%             \lwbox
%           \fi
%         \else
%           \usebox0
%         \fi
%       \else
%         \lwbox
%       \fi
%     \else
%       \usebox0
%     \fi
%   \else
%     \lwbox
%   \fi
% \else
%   \usebox0
% \fi
% \end{quote}
% If you have a \xfile{docstrip.cfg} that configures and enables \docstrip's
% TDS installing feature, then some files can already be in the right
% place, see the documentation of \docstrip.
%
% \subsection{Refresh file name databases}
%
% If your \TeX~distribution
% (\teTeX, \mikTeX, \dots) relies on file name databases, you must refresh
% these. For example, \teTeX\ users run \verb|texhash| or
% \verb|mktexlsr|.
%
% \subsection{Some details for the interested}
%
% \paragraph{Attached source.}
%
% The PDF documentation on CTAN also includes the
% \xfile{.dtx} source file. It can be extracted by
% AcrobatReader 6 or higher. Another option is \textsf{pdftk},
% e.g. unpack the file into the current directory:
% \begin{quote}
%   \verb|pdftk hologo.pdf unpack_files output .|
% \end{quote}
%
% \paragraph{Unpacking with \LaTeX.}
% The \xfile{.dtx} chooses its action depending on the format:
% \begin{description}
% \item[\plainTeX:] Run \docstrip\ and extract the files.
% \item[\LaTeX:] Generate the documentation.
% \end{description}
% If you insist on using \LaTeX\ for \docstrip\ (really,
% \docstrip\ does not need \LaTeX), then inform the autodetect routine
% about your intention:
% \begin{quote}
%   \verb|latex \let\install=y% \iffalse meta-comment
%
% File: hologo.dtx
% Version: 2016/05/12 v1.11
% Info: A logo collection with bookmark support
%
% Copyright (C) 2010-2012 by
%    Heiko Oberdiek <heiko.oberdiek at googlemail.com>
%
% This work may be distributed and/or modified under the
% conditions of the LaTeX Project Public License, either
% version 1.3c of this license or (at your option) any later
% version. This version of this license is in
%    http://www.latex-project.org/lppl/lppl-1-3c.txt
% and the latest version of this license is in
%    http://www.latex-project.org/lppl.txt
% and version 1.3 or later is part of all distributions of
% LaTeX version 2005/12/01 or later.
%
% This work has the LPPL maintenance status "maintained".
%
% This Current Maintainer of this work is Heiko Oberdiek.
%
% The Base Interpreter refers to any `TeX-Format',
% because some files are installed in TDS:tex/generic//.
%
% This work consists of the main source file hologo.dtx
% and the derived files
%    hologo.sty, hologo.pdf, hologo.ins, hologo.drv, hologo-example.tex,
%    hologo-test1.tex, hologo-test-spacefactor.tex,
%    hologo-test-list.tex.
%
% Distribution:
%    CTAN:macros/latex/contrib/oberdiek/hologo.dtx
%    CTAN:macros/latex/contrib/oberdiek/hologo.pdf
%
% Unpacking:
%    (a) If hologo.ins is present:
%           tex hologo.ins
%    (b) Without hologo.ins:
%           tex hologo.dtx
%    (c) If you insist on using LaTeX
%           latex \let\install=y\input{hologo.dtx}
%        (quote the arguments according to the demands of your shell)
%
% Documentation:
%    (a) If hologo.drv is present:
%           latex hologo.drv
%    (b) Without hologo.drv:
%           latex hologo.dtx; ...
%    The class ltxdoc loads the configuration file ltxdoc.cfg
%    if available. Here you can specify further options, e.g.
%    use A4 as paper format:
%       \PassOptionsToClass{a4paper}{article}
%
%    Programm calls to get the documentation (example):
%       pdflatex hologo.dtx
%       makeindex -s gind.ist hologo.idx
%       pdflatex hologo.dtx
%       makeindex -s gind.ist hologo.idx
%       pdflatex hologo.dtx
%
% Installation:
%    TDS:tex/generic/oberdiek/hologo.sty
%    TDS:doc/latex/oberdiek/hologo.pdf
%    TDS:doc/latex/oberdiek/example/hologo-example.tex
%    TDS:doc/latex/oberdiek/test/hologo-test1.tex
%    TDS:doc/latex/oberdiek/test/hologo-test-spacefactor.tex
%    TDS:doc/latex/oberdiek/test/hologo-test-list.tex
%    TDS:source/latex/oberdiek/hologo.dtx
%
%<*ignore>
\begingroup
  \catcode123=1 %
  \catcode125=2 %
  \def\x{LaTeX2e}%
\expandafter\endgroup
\ifcase 0\ifx\install y1\fi\expandafter
         \ifx\csname processbatchFile\endcsname\relax\else1\fi
         \ifx\fmtname\x\else 1\fi\relax
\else\csname fi\endcsname
%</ignore>
%<*install>
\input docstrip.tex
\Msg{************************************************************************}
\Msg{* Installation}
\Msg{* Package: hologo 2016/05/12 v1.11 A logo collection with bookmark support (HO)}
\Msg{************************************************************************}

\keepsilent
\askforoverwritefalse

\let\MetaPrefix\relax
\preamble

This is a generated file.

Project: hologo
Version: 2016/05/12 v1.11

Copyright (C) 2010-2012 by
   Heiko Oberdiek <heiko.oberdiek at googlemail.com>

This work may be distributed and/or modified under the
conditions of the LaTeX Project Public License, either
version 1.3c of this license or (at your option) any later
version. This version of this license is in
   http://www.latex-project.org/lppl/lppl-1-3c.txt
and the latest version of this license is in
   http://www.latex-project.org/lppl.txt
and version 1.3 or later is part of all distributions of
LaTeX version 2005/12/01 or later.

This work has the LPPL maintenance status "maintained".

This Current Maintainer of this work is Heiko Oberdiek.

The Base Interpreter refers to any `TeX-Format',
because some files are installed in TDS:tex/generic//.

This work consists of the main source file hologo.dtx
and the derived files
   hologo.sty, hologo.pdf, hologo.ins, hologo.drv, hologo-example.tex,
   hologo-test1.tex, hologo-test-spacefactor.tex,
   hologo-test-list.tex.

\endpreamble
\let\MetaPrefix\DoubleperCent

\generate{%
  \file{hologo.ins}{\from{hologo.dtx}{install}}%
  \file{hologo.drv}{\from{hologo.dtx}{driver}}%
  \usedir{tex/generic/oberdiek}%
  \file{hologo.sty}{\from{hologo.dtx}{package}}%
  \usedir{doc/latex/oberdiek/example}%
  \file{hologo-example.tex}{\from{hologo.dtx}{example}}%
  \usedir{doc/latex/oberdiek/test}%
  \file{hologo-test1.tex}{\from{hologo.dtx}{test1}}%
  \file{hologo-test-spacefactor.tex}{\from{hologo.dtx}{test-spacefactor}}%
  \file{hologo-test-list.tex}{\from{hologo.dtx}{test-list}}%
  \nopreamble
  \nopostamble
  \usedir{source/latex/oberdiek/catalogue}%
  \file{hologo.xml}{\from{hologo.dtx}{catalogue}}%
}

\catcode32=13\relax% active space
\let =\space%
\Msg{************************************************************************}
\Msg{*}
\Msg{* To finish the installation you have to move the following}
\Msg{* file into a directory searched by TeX:}
\Msg{*}
\Msg{*     hologo.sty}
\Msg{*}
\Msg{* To produce the documentation run the file `hologo.drv'}
\Msg{* through LaTeX.}
\Msg{*}
\Msg{* Happy TeXing!}
\Msg{*}
\Msg{************************************************************************}

\endbatchfile
%</install>
%<*ignore>
\fi
%</ignore>
%<*driver>
\NeedsTeXFormat{LaTeX2e}
\ProvidesFile{hologo.drv}%
  [2016/05/12 v1.11 A logo collection with bookmark support (HO)]%
\documentclass{ltxdoc}
\usepackage{holtxdoc}[2011/11/22]
\usepackage{hologo}[2016/05/12]
\usepackage{longtable}
\usepackage{array}
\usepackage{paralist}
%\usepackage[T1]{fontenc}
%\usepackage{lmodern}
\begin{document}
  \DocInput{hologo.dtx}%
\end{document}
%</driver>
% \fi
%
%
% \CharacterTable
%  {Upper-case    \A\B\C\D\E\F\G\H\I\J\K\L\M\N\O\P\Q\R\S\T\U\V\W\X\Y\Z
%   Lower-case    \a\b\c\d\e\f\g\h\i\j\k\l\m\n\o\p\q\r\s\t\u\v\w\x\y\z
%   Digits        \0\1\2\3\4\5\6\7\8\9
%   Exclamation   \!     Double quote  \"     Hash (number) \#
%   Dollar        \$     Percent       \%     Ampersand     \&
%   Acute accent  \'     Left paren    \(     Right paren   \)
%   Asterisk      \*     Plus          \+     Comma         \,
%   Minus         \-     Point         \.     Solidus       \/
%   Colon         \:     Semicolon     \;     Less than     \<
%   Equals        \=     Greater than  \>     Question mark \?
%   Commercial at \@     Left bracket  \[     Backslash     \\
%   Right bracket \]     Circumflex    \^     Underscore    \_
%   Grave accent  \`     Left brace    \{     Vertical bar  \|
%   Right brace   \}     Tilde         \~}
%
% \GetFileInfo{hologo.drv}
%
% \title{The \xpackage{hologo} package}
% \date{2016/05/12 v1.11}
% \author{Heiko Oberdiek\\\xemail{heiko.oberdiek at googlemail.com}}
%
% \maketitle
%
% \begin{abstract}
% This package starts a collection of logos with support for bookmarks
% strings.
% \end{abstract}
%
% \tableofcontents
%
% \section{Documentation}
%
% \subsection{Logo macros}
%
% \begin{declcs}{hologo} \M{name}
% \end{declcs}
% Macro \cs{hologo} sets the logo with name \meta{name}.
% The following table shows the supported names.
%
% \begingroup
%   \def\hologoEntry#1#2#3{^^A
%     #1&#2&\hologoLogoSetup{#1}{variant=#2}\hologo{#1}&#3\tabularnewline
%   }
%   \begin{longtable}{>{\ttfamily}l>{\ttfamily}lll}
%     \rmfamily\bfseries{name} & \rmfamily\bfseries variant
%     & \bfseries logo & \bfseries since\\
%     \hline
%     \endhead
%     \hologoList
%   \end{longtable}
% \endgroup
%
% \begin{declcs}{Hologo} \M{name}
% \end{declcs}
% Macro \cs{Hologo} starts the logo \meta{name} with an uppercase
% letter. As an exception small greek letters are not converted
% to uppercase. Examples, see \hologo{eTeX} and \hologo{ExTeX}.
%
% \subsection{Setup macros}
%
% The package does not support package options, but the following
% setup macros can be used to set options.
%
% \begin{declcs}{hologoSetup} \M{key value list}
% \end{declcs}
% Macro \cs{hologoSetup} sets global options.
%
% \begin{declcs}{hologoLogoSetup} \M{logo} \M{key value list}
% \end{declcs}
% Some options can also be used to configure a logo.
% These settings take precedence over global option settings.
%
% \subsection{Options}\label{sec:options}
%
% There are boolean and string options:
% \begin{description}
% \item[Boolean option:]
% It takes |true| or |false|
% as value. If the value is omitted, then |true| is used.
% \item[String option:]
% A value must be given as string. (But the string might be empty.)
% \end{description}
% The following options can be used both in \cs{hologoSetup}
% and \cs{hologoLogoSetup}:
% \begin{description}
% \def\entry#1{\item[\xoption{#1}:]}
% \entry{break}
%   enables or disables line breaks inside the logo. This setting is
%   refined by options \xoption{hyphenbreak}, \xoption{spacebreak}
%   or \xoption{discretionarybreak}.
%   Default is |false|.
% \entry{hyphenbreak}
%   enables or disables the line break right after the hyphen character.
% \entry{spacebreak}
%   enables or disables line breaks at space characters.
% \entry{discretionarybreak}
%   enables or disables line breaks at hyphenation points
%   (inserted by \cs{-}).
% \end{description}
% Macro \cs{hologoLogoSetup} also knows:
% \begin{description}
% \item[\xoption{variant}:]
%   This is a string option. It specifies a variant of a logo that
%   must exist. An empty string selects the package default variant.
% \end{description}
% Example:
% \begin{quote}
%   |\hologoSetup{break=false}|\\
%   |\hologoLogoSetup{plainTeX}{variant=hyphen,hyphenbreak}|\\
%   Then ``plain-\TeX'' contains one break point after the hyphen.
% \end{quote}
%
% \subsection{Driver options}
%
% Sometimes graphical operations are needed to construct some
% glyphs (e.g.\ \hologo{XeTeX}). If package \xpackage{graphics}
% or package \xpackage{pgf} are found, then the macros are taken
% from there. Otherwise the packge defines its own operations
% and therefore needs the driver information. Many drivers are
% detected automatically (\hologo{pdfTeX}/\hologo{LuaTeX}
% in PDF mode, \hologo{XeTeX}, \hologo{VTeX}). These have precedence
% over a driver option. The driver can be given as package option
% or using \cs{hologoDriverSetup}.
% The following list contains the recognized driver options:
% \begin{itemize}
% \item \xoption{pdftex}, \xoption{luatex}
% \item \xoption{dvipdfm}, \xoption{dvipdfmx}
% \item \xoption{dvips}, \xoption{dvipsone}, \xoption{xdvi}
% \item \xoption{xetex}
% \item \xoption{vtex}
% \end{itemize}
% The left driver of a line is the driver name that is used internally.
% The following names are aliases for drivers that use the
% same method. Therefore the entry in the \xext{log} file for
% the used driver prints the internally used driver name.
% \begin{description}
% \item[\xoption{driverfallback}:]
%   This option expects a driver that is used,
%   if the driver could not be detected automatically.
% \end{description}
%
% \begin{declcs}{hologoDriverSetup} \M{driver option}
% \end{declcs}
% The driver can also be configured after package loading
% using \cs{hologoDriverSetup}, also the way for \hologo{plainTeX}
% to setup the driver.
%
% \subsection{Font setup}
%
% Some logos require a special font, but should also be usable by
% \hologo{plainTeX}. Therefore the package provides some ways
% to influence the font settings. The options below
% take font settings as values. Both font commands
% such as \cs{sffamily} and macros that take one argument
% like \cs{textsf} can be used.
%
% \begin{declcs}{hologoFontSetup} \M{key value list}
% \end{declcs}
% Macro \cs{hologoFontSetup} sets the fonts for all logos.
% Supported keys:
% \begin{description}
% \def\entry#1{\item[\xoption{#1}:]}
% \entry{general}
%   This font is used for all logos. The default is empty.
%   That means no special font is used.
% \entry{bibsf}
%   This font is used for
%   {\hologoLogoSetup{BibTeX}{variant=sf}\hologo{BibTeX}}
%   with variant \xoption{sf}.
% \entry{rm}
%   This font is a serif font. It is used for \hologo{ExTeX}.
% \entry{sc}
%   This font specifies a small caps font. It is used for
%   {\hologoLogoSetup{BibTeX}{variant=sc}\hologo{BibTeX}}
%   with variant \xoption{sc}.
% \entry{sf}
%   This font specifies a sans serif font. The default
%   is \cs{sffamily}, then \cs{sf} is tried. Otherwise
%   a warning is given. It is used by \hologo{KOMAScript}.
% \entry{sy}
%   This is the font for math symbols (e.g. cmsy).
%   It is used by \hologo{AmS}, \hologo{NTS}, \hologo{ExTeX}.
% \entry{logo}
%   \hologo{METAFONT} and \hologo{METAPOST} are using that font.
%   In \hologo{LaTeX} \cs{logofamily} is used and
%   the definitions of package \xpackage{mflogo} are used
%   if the package is not loaded.
%   Otherwise the \cs{tenlogo} is used and defined
%   if it does not already exists.
% \end{description}
%
% \begin{declcs}{hologoLogoFontSetup} \M{logo} \M{key value list}
% \end{declcs}
% Fonts can also be set for a logo or logo component separately,
% see the following list.
% The keys are the same as for \cs{hologoFontSetup}.
%
% \begin{longtable}{>{\ttfamily}l>{\sffamily}ll}
%   \meta{logo} & keys & result\\
%   \hline
%   \endhead
%   BibTeX & bibsf & {\hologoLogoSetup{BibTeX}{variant=sf}\hologo{BibTeX}}\\[.5ex]
%   BibTeX & sc & {\hologoLogoSetup{BibTeX}{variant=sc}\hologo{BibTeX}}\\[.5ex]
%   ExTeX & rm & \hologo{ExTeX}\\
%   SliTeX & rm & \hologo{SliTeX}\\[.5ex]
%   AmS & sy & \hologo{AmS}\\
%   ExTeX & sy & \hologo{ExTeX}\\
%   NTS & sy & \hologo{NTS}\\[.5ex]
%   KOMAScript & sf & \hologo{KOMAScript}\\[.5ex]
%   METAFONT & logo & \hologo{METAFONT}\\
%   METAPOST & logo & \hologo{METAPOST}\\[.5ex]
%   SliTeX & sc \hologo{SliTeX}
% \end{longtable}
%
% \subsubsection{Font order}
%
% For all logos the font \xoption{general} is applied first.
% Example:
%\begin{quote}
%|\hologoFontSetup{general=\color{red}}|
%\end{quote}
% will print red logos.
% Then if the font uses a special font \xoption{sf}, for example,
% the font is applied that is setup by \cs{hologoLogoFontSetup}.
% If this font is not setup, then the common font setup
% by \cs{hologoFontSetup} is used. Otherwise a warning is given,
% that there is no font configured.
%
% \subsection{Additional user macros}
%
% Usually a variant of a logo is configured by using
% \cs{hologoLogoSetup}, because it is bad style to mix
% different variants of the same logo in the same text.
% There the following macros are a convenience for testing.
%
% \begin{declcs}{hologoVariant} \M{name} \M{variant}\\
%   \cs{HologoVariant} \M{name} \M{variant}
% \end{declcs}
% Logo \meta{name} is set using \meta{variant} that specifies
% explicitely which variant of the macro is used. If the argument
% is empty, then the default form of the logo is used
% (configurable by \cs{hologoLogoSetup}).
%
% \cs{HologoVariant} is used if the logo is set in a context
% that needs an uppercase first letter (beginning of a sentence, \dots).
%
% \begin{declcs}{hologoList}\\
%   \cs{hologoEntry} \M{logo} \M{variant} \M{since}
% \end{declcs}
% Macro \cs{hologoList} contains all logos that are provided
% by the package including variants. The list consists of calls
% of \cs{hologoEntry} with three arguments starting with the
% logo name \meta{logo} and its variant \meta{variant}. An empty
% variant means the current default. Argument \meta{since} specifies
% with version of the package \xpackage{hologo} is needed to get
% the logo. If the logo is fixed, then the date gets updated.
% Therefore the date \meta{since} is not exactly the date of
% the first introduction, but rather the date of the latest fix.
%
% Before \cs{hologoList} can be used, macro \cs{hologoEntry} needs
% a definition. The example file in section \ref{sec:example}
% shows applications of \cs{hologoList}.
%
% \subsection{Supported contexts}
%
% Macros \cs{hologo} and friends support special contexts:
% \begin{itemize}
% \item \hologo{LaTeX}'s protection mechanism.
% \item Bookmarks of package \xpackage{hyperref}.
% \item Package \xpackage{tex4ht}.
% \item The macros can be used inside \cs{csname} constructs,
%   if \cs{ifincsname} is available (\hologo{pdfTeX}, \hologo{XeTeX},
%   \hologo{LuaTeX}).
% \end{itemize}
%
% \subsection{Example}
% \label{sec:example}
%
% The following example prints the logos in different fonts.
%    \begin{macrocode}
%<*example>
%<<verbatim
\NeedsTeXFormat{LaTeX2e}
\documentclass[a4paper]{article}
\usepackage[
  hmargin=20mm,
  vmargin=20mm,
]{geometry}
\pagestyle{empty}
\usepackage{hologo}[2016/05/12]
\usepackage{longtable}
\usepackage{array}
\setlength{\extrarowheight}{2pt}
\usepackage[T1]{fontenc}
\usepackage{lmodern}
\usepackage{pdflscape}
\usepackage[
  pdfencoding=auto,
]{hyperref}
\hypersetup{
  pdfauthor={Heiko Oberdiek},
  pdftitle={Example for package `hologo'},
  pdfsubject={Logos with fonts lmr, lmss, qtm, qpl, qhv},
}
\usepackage{bookmark}

% Print the logo list on the console

\begingroup
  \typeout{}%
  \typeout{*** Begin of logo list ***}%
  \newcommand*{\hologoEntry}[3]{%
    \typeout{#1 \ifx\\#2\\\else(#2) \fi[#3]}%
  }%
  \hologoList
  \typeout{*** End of logo list ***}%
  \typeout{}%
\endgroup

\begin{document}
\begin{landscape}

  \section{Example file for package `hologo'}

  % Table for font names

  \begin{longtable}{>{\bfseries}ll}
    \textbf{font} & \textbf{Font name}\\
    \hline
    lmr & Latin Modern Roman\\
    lmss & Latin Modern Sans\\
    qtm & \TeX\ Gyre Termes\\
    qhv & \TeX\ Gyre Heros\\
    qpl & \TeX\ Gyre Pagella\\
  \end{longtable}

  % Logo list with logos in different fonts

  \begingroup
    \newcommand*{\SetVariant}[2]{%
      \ifx\\#2\\%
      \else
        \hologoLogoSetup{#1}{variant=#2}%
      \fi
    }%
    \newcommand*{\hologoEntry}[3]{%
      \SetVariant{#1}{#2}%
      \raisebox{1em}[0pt][0pt]{\hypertarget{#1@#2}{}}%
      \bookmark[%
        dest={#1@#2},%
      ]{%
        #1\ifx\\#2\\\else\space(#2)\fi: \Hologo{#1}, \hologo{#1} %
        [Unicode]%
      }%
      \hypersetup{unicode=false}%
      \bookmark[%
        dest={#1@#2},%
      ]{%
        #1\ifx\\#2\\\else\space(#2)\fi: \Hologo{#1}, \hologo{#1} %
        [PDFDocEncoding]%
      }%
      \texttt{#1}%
      &%
      \texttt{#2}%
      &%
      \Hologo{#1}%
      &%
      \SetVariant{#1}{#2}%
      \hologo{#1}%
      &%
      \SetVariant{#1}{#2}%
      \fontfamily{qtm}\selectfont
      \hologo{#1}%
      &%
      \SetVariant{#1}{#2}%
      \fontfamily{qpl}\selectfont
      \hologo{#1}%
      &%
      \SetVariant{#1}{#2}%
      \textsf{\hologo{#1}}%
      &%
      \SetVariant{#1}{#2}%
      \fontfamily{qhv}\selectfont
      \hologo{#1}%
      \tabularnewline
    }%
    \begin{longtable}{llllllll}%
      \textbf{\textit{logo}} & \textbf{\textit{variant}} &
      \texttt{\string\Hologo} &
      \textbf{lmr} & \textbf{qtm} & \textbf{qpl} &
      \textbf{lmss} & \textbf{qhv}
      \tabularnewline
      \hline
      \endhead
      \hologoList
    \end{longtable}%
  \endgroup

\end{landscape}
\end{document}
%verbatim
%</example>
%    \end{macrocode}
%
% \StopEventually{
% }
%
% \section{Implementation}
%    \begin{macrocode}
%<*package>
%    \end{macrocode}
%    Reload check, especially if the package is not used with \LaTeX.
%    \begin{macrocode}
\begingroup\catcode61\catcode48\catcode32=10\relax%
  \catcode13=5 % ^^M
  \endlinechar=13 %
  \catcode35=6 % #
  \catcode39=12 % '
  \catcode44=12 % ,
  \catcode45=12 % -
  \catcode46=12 % .
  \catcode58=12 % :
  \catcode64=11 % @
  \catcode123=1 % {
  \catcode125=2 % }
  \expandafter\let\expandafter\x\csname ver@hologo.sty\endcsname
  \ifx\x\relax % plain-TeX, first loading
  \else
    \def\empty{}%
    \ifx\x\empty % LaTeX, first loading,
      % variable is initialized, but \ProvidesPackage not yet seen
    \else
      \expandafter\ifx\csname PackageInfo\endcsname\relax
        \def\x#1#2{%
          \immediate\write-1{Package #1 Info: #2.}%
        }%
      \else
        \def\x#1#2{\PackageInfo{#1}{#2, stopped}}%
      \fi
      \x{hologo}{The package is already loaded}%
      \aftergroup\endinput
    \fi
  \fi
\endgroup%
%    \end{macrocode}
%    Package identification:
%    \begin{macrocode}
\begingroup\catcode61\catcode48\catcode32=10\relax%
  \catcode13=5 % ^^M
  \endlinechar=13 %
  \catcode35=6 % #
  \catcode39=12 % '
  \catcode40=12 % (
  \catcode41=12 % )
  \catcode44=12 % ,
  \catcode45=12 % -
  \catcode46=12 % .
  \catcode47=12 % /
  \catcode58=12 % :
  \catcode64=11 % @
  \catcode91=12 % [
  \catcode93=12 % ]
  \catcode123=1 % {
  \catcode125=2 % }
  \expandafter\ifx\csname ProvidesPackage\endcsname\relax
    \def\x#1#2#3[#4]{\endgroup
      \immediate\write-1{Package: #3 #4}%
      \xdef#1{#4}%
    }%
  \else
    \def\x#1#2[#3]{\endgroup
      #2[{#3}]%
      \ifx#1\@undefined
        \xdef#1{#3}%
      \fi
      \ifx#1\relax
        \xdef#1{#3}%
      \fi
    }%
  \fi
\expandafter\x\csname ver@hologo.sty\endcsname
\ProvidesPackage{hologo}%
  [2016/05/12 v1.11 A logo collection with bookmark support (HO)]%
%    \end{macrocode}
%
%    \begin{macrocode}
\begingroup\catcode61\catcode48\catcode32=10\relax%
  \catcode13=5 % ^^M
  \endlinechar=13 %
  \catcode123=1 % {
  \catcode125=2 % }
  \catcode64=11 % @
  \def\x{\endgroup
    \expandafter\edef\csname HOLOGO@AtEnd\endcsname{%
      \endlinechar=\the\endlinechar\relax
      \catcode13=\the\catcode13\relax
      \catcode32=\the\catcode32\relax
      \catcode35=\the\catcode35\relax
      \catcode61=\the\catcode61\relax
      \catcode64=\the\catcode64\relax
      \catcode123=\the\catcode123\relax
      \catcode125=\the\catcode125\relax
    }%
  }%
\x\catcode61\catcode48\catcode32=10\relax%
\catcode13=5 % ^^M
\endlinechar=13 %
\catcode35=6 % #
\catcode64=11 % @
\catcode123=1 % {
\catcode125=2 % }
\def\TMP@EnsureCode#1#2{%
  \edef\HOLOGO@AtEnd{%
    \HOLOGO@AtEnd
    \catcode#1=\the\catcode#1\relax
  }%
  \catcode#1=#2\relax
}
\TMP@EnsureCode{10}{12}% ^^J
\TMP@EnsureCode{33}{12}% !
\TMP@EnsureCode{34}{12}% "
\TMP@EnsureCode{36}{3}% $
\TMP@EnsureCode{38}{4}% &
\TMP@EnsureCode{39}{12}% '
\TMP@EnsureCode{40}{12}% (
\TMP@EnsureCode{41}{12}% )
\TMP@EnsureCode{42}{12}% *
\TMP@EnsureCode{43}{12}% +
\TMP@EnsureCode{44}{12}% ,
\TMP@EnsureCode{45}{12}% -
\TMP@EnsureCode{46}{12}% .
\TMP@EnsureCode{47}{12}% /
\TMP@EnsureCode{58}{12}% :
\TMP@EnsureCode{59}{12}% ;
\TMP@EnsureCode{60}{12}% <
\TMP@EnsureCode{62}{12}% >
\TMP@EnsureCode{63}{12}% ?
\TMP@EnsureCode{91}{12}% [
\TMP@EnsureCode{93}{12}% ]
\TMP@EnsureCode{94}{7}% ^ (superscript)
\TMP@EnsureCode{95}{8}% _ (subscript)
\TMP@EnsureCode{96}{12}% `
\TMP@EnsureCode{124}{12}% |
\edef\HOLOGO@AtEnd{%
  \HOLOGO@AtEnd
  \escapechar\the\escapechar\relax
  \noexpand\endinput
}
\escapechar=92 %
%    \end{macrocode}
%
% \subsection{Logo list}
%
%    \begin{macro}{\hologoList}
%    \begin{macrocode}
\def\hologoList{%
  \hologoEntry{(La)TeX}{}{2011/10/01}%
  \hologoEntry{AmSLaTeX}{}{2010/04/16}%
  \hologoEntry{AmSTeX}{}{2010/04/16}%
  \hologoEntry{biber}{}{2011/10/01}%
  \hologoEntry{BibTeX}{}{2011/10/01}%
  \hologoEntry{BibTeX}{sf}{2011/10/01}%
  \hologoEntry{BibTeX}{sc}{2011/10/01}%
  \hologoEntry{BibTeX8}{}{2011/11/22}%
  \hologoEntry{ConTeXt}{}{2011/03/25}%
  \hologoEntry{ConTeXt}{narrow}{2011/03/25}%
  \hologoEntry{ConTeXt}{simple}{2011/03/25}%
  \hologoEntry{emTeX}{}{2010/04/26}%
  \hologoEntry{eTeX}{}{2010/04/08}%
  \hologoEntry{ExTeX}{}{2011/10/01}%
  \hologoEntry{HanTheThanh}{}{2011/11/29}%
  \hologoEntry{iniTeX}{}{2011/10/01}%
  \hologoEntry{KOMAScript}{}{2011/10/01}%
  \hologoEntry{La}{}{2010/05/08}%
  \hologoEntry{LaTeX}{}{2010/04/08}%
  \hologoEntry{LaTeX2e}{}{2010/04/08}%
  \hologoEntry{LaTeX3}{}{2010/04/24}%
  \hologoEntry{LaTeXe}{}{2010/04/08}%
  \hologoEntry{LaTeXML}{}{2011/11/22}%
  \hologoEntry{LaTeXTeX}{}{2011/10/01}%
  \hologoEntry{LuaLaTeX}{}{2010/04/08}%
  \hologoEntry{LuaTeX}{}{2010/04/08}%
  \hologoEntry{LyX}{}{2011/10/01}%
  \hologoEntry{METAFONT}{}{2011/10/01}%
  \hologoEntry{MetaFun}{}{2011/10/01}%
  \hologoEntry{METAPOST}{}{2011/10/01}%
  \hologoEntry{MetaPost}{}{2011/10/01}%
  \hologoEntry{MiKTeX}{}{2011/10/01}%
  \hologoEntry{NTS}{}{2011/10/01}%
  \hologoEntry{OzMF}{}{2011/10/01}%
  \hologoEntry{OzMP}{}{2011/10/01}%
  \hologoEntry{OzTeX}{}{2011/10/01}%
  \hologoEntry{OzTtH}{}{2011/10/01}%
  \hologoEntry{PCTeX}{}{2011/10/01}%
  \hologoEntry{pdfTeX}{}{2011/10/01}%
  \hologoEntry{pdfLaTeX}{}{2011/10/01}%
  \hologoEntry{PiC}{}{2011/10/01}%
  \hologoEntry{PiCTeX}{}{2011/10/01}%
  \hologoEntry{plainTeX}{}{2010/04/08}%
  \hologoEntry{plainTeX}{space}{2010/04/16}%
  \hologoEntry{plainTeX}{hyphen}{2010/04/16}%
  \hologoEntry{plainTeX}{runtogether}{2010/04/16}%
  \hologoEntry{SageTeX}{}{2011/11/22}%
  \hologoEntry{SLiTeX}{}{2011/10/01}%
  \hologoEntry{SLiTeX}{lift}{2011/10/01}%
  \hologoEntry{SLiTeX}{narrow}{2011/10/01}%
  \hologoEntry{SLiTeX}{simple}{2011/10/01}%
  \hologoEntry{SliTeX}{}{2011/10/01}%
  \hologoEntry{SliTeX}{narrow}{2011/10/01}%
  \hologoEntry{SliTeX}{simple}{2011/10/01}%
  \hologoEntry{SliTeX}{lift}{2011/10/01}%
  \hologoEntry{teTeX}{}{2011/10/01}%
  \hologoEntry{TeX}{}{2010/04/08}%
  \hologoEntry{TeX4ht}{}{2011/11/22}%
  \hologoEntry{TTH}{}{2011/11/22}%
  \hologoEntry{virTeX}{}{2011/10/01}%
  \hologoEntry{VTeX}{}{2010/04/24}%
  \hologoEntry{Xe}{}{2010/04/08}%
  \hologoEntry{XeLaTeX}{}{2010/04/08}%
  \hologoEntry{XeTeX}{}{2010/04/08}%
}
%    \end{macrocode}
%    \end{macro}
%
% \subsection{Load resources}
%
%    \begin{macrocode}
\begingroup\expandafter\expandafter\expandafter\endgroup
\expandafter\ifx\csname RequirePackage\endcsname\relax
  \def\TMP@RequirePackage#1[#2]{%
    \begingroup\expandafter\expandafter\expandafter\endgroup
    \expandafter\ifx\csname ver@#1.sty\endcsname\relax
      \input #1.sty\relax
    \fi
  }%
  \TMP@RequirePackage{ltxcmds}[2011/02/04]%
  \TMP@RequirePackage{infwarerr}[2010/04/08]%
  \TMP@RequirePackage{kvsetkeys}[2010/03/01]%
  \TMP@RequirePackage{kvdefinekeys}[2010/03/01]%
  \TMP@RequirePackage{pdftexcmds}[2010/04/01]%
  \TMP@RequirePackage{ifpdf}[2010/01/28]%
  \TMP@RequirePackage{ifluatex}[2010/03/01]%
  \ltx@IfUndefined{newif}{%
    \expandafter\let\csname newif\endcsname\ltx@newif
  }{}%
  \TMP@RequirePackage{ifxetex}[2009/01/23]%
  \TMP@RequirePackage{ifvtex}[2010/03/01]%
\else
  \RequirePackage{ltxcmds}[2011/02/04]%
  \RequirePackage{infwarerr}[2010/04/08]%
  \RequirePackage{kvsetkeys}[2010/03/01]%
  \RequirePackage{kvdefinekeys}[2010/03/01]%
  \RequirePackage{pdftexcmds}[2010/04/01]%
  \RequirePackage{ifpdf}[2010/01/28]%
  \RequirePackage{ifluatex}[2010/03/01]%
  \RequirePackage{ifxetex}[2009/01/23]%
  \RequirePackage{ifvtex}[2010/03/01]%
\fi
%    \end{macrocode}
%
%    \begin{macro}{\HOLOGO@IfDefined}
%    \begin{macrocode}
\def\HOLOGO@IfExists#1{%
  \ifx\@undefined#1%
    \expandafter\ltx@secondoftwo
  \else
    \ifx\relax#1%
      \expandafter\ltx@secondoftwo
    \else
      \expandafter\expandafter\expandafter\ltx@firstoftwo
    \fi
  \fi
}
%    \end{macrocode}
%    \end{macro}
%
% \subsection{Setup macros}
%
%    \begin{macro}{\hologoSetup}
%    \begin{macrocode}
\def\hologoSetup{%
  \let\HOLOGO@name\relax
  \HOLOGO@Setup
}
%    \end{macrocode}
%    \end{macro}
%
%    \begin{macro}{\hologoLogoSetup}
%    \begin{macrocode}
\def\hologoLogoSetup#1{%
  \edef\HOLOGO@name{#1}%
  \ltx@IfUndefined{HoLogo@\HOLOGO@name}{%
    \@PackageError{hologo}{%
      Unknown logo `\HOLOGO@name'%
    }\@ehc
    \ltx@gobble
  }{%
    \HOLOGO@Setup
  }%
}
%    \end{macrocode}
%    \end{macro}
%
%    \begin{macro}{\HOLOGO@Setup}
%    \begin{macrocode}
\def\HOLOGO@Setup{%
  \kvsetkeys{HoLogo}%
}
%    \end{macrocode}
%    \end{macro}
%
% \subsection{Options}
%
%    \begin{macro}{\HOLOGO@DeclareBoolOption}
%    \begin{macrocode}
\def\HOLOGO@DeclareBoolOption#1{%
  \expandafter\chardef\csname HOLOGOOPT@#1\endcsname\ltx@zero
  \kv@define@key{HoLogo}{#1}[true]{%
    \def\HOLOGO@temp{##1}%
    \ifx\HOLOGO@temp\HOLOGO@true
      \ifx\HOLOGO@name\relax
        \expandafter\chardef\csname HOLOGOOPT@#1\endcsname=\ltx@one
      \else
        \expandafter\chardef\csname
        HoLogoOpt@#1@\HOLOGO@name\endcsname\ltx@one
      \fi
      \HOLOGO@SetBreakAll{#1}%
    \else
      \ifx\HOLOGO@temp\HOLOGO@false
        \ifx\HOLOGO@name\relax
          \expandafter\chardef\csname HOLOGOOPT@#1\endcsname=\ltx@zero
        \else
          \expandafter\chardef\csname
          HoLogoOpt@#1@\HOLOGO@name\endcsname=\ltx@zero
        \fi
        \HOLOGO@SetBreakAll{#1}%
      \else
        \@PackageError{hologo}{%
          Unknown value `##1' for boolean option `#1'.\MessageBreak
          Known values are `true' and `false'%
        }\@ehc
      \fi
    \fi
  }%
}
%    \end{macrocode}
%    \end{macro}
%
%    \begin{macro}{\HOLOGO@SetBreakAll}
%    \begin{macrocode}
\def\HOLOGO@SetBreakAll#1{%
  \def\HOLOGO@temp{#1}%
  \ifx\HOLOGO@temp\HOLOGO@break
    \ifx\HOLOGO@name\relax
      \chardef\HOLOGOOPT@hyphenbreak=\HOLOGOOPT@break
      \chardef\HOLOGOOPT@spacebreak=\HOLOGOOPT@break
      \chardef\HOLOGOOPT@discretionarybreak=\HOLOGOOPT@break
    \else
      \expandafter\chardef
         \csname HoLogoOpt@hyphenbreak@\HOLOGO@name\endcsname=%
         \csname HoLogoOpt@break@\HOLOGO@name\endcsname
      \expandafter\chardef
         \csname HoLogoOpt@spacebreak@\HOLOGO@name\endcsname=%
         \csname HoLogoOpt@break@\HOLOGO@name\endcsname
      \expandafter\chardef
         \csname HoLogoOpt@discretionarybreak@\HOLOGO@name
             \endcsname=%
         \csname HoLogoOpt@break@\HOLOGO@name\endcsname
    \fi
  \fi
}
%    \end{macrocode}
%    \end{macro}
%
%    \begin{macro}{\HOLOGO@true}
%    \begin{macrocode}
\def\HOLOGO@true{true}
%    \end{macrocode}
%    \end{macro}
%    \begin{macro}{\HOLOGO@false}
%    \begin{macrocode}
\def\HOLOGO@false{false}
%    \end{macrocode}
%    \end{macro}
%    \begin{macro}{\HOLOGO@break}
%    \begin{macrocode}
\def\HOLOGO@break{break}
%    \end{macrocode}
%    \end{macro}
%
%    \begin{macrocode}
\HOLOGO@DeclareBoolOption{break}
\HOLOGO@DeclareBoolOption{hyphenbreak}
\HOLOGO@DeclareBoolOption{spacebreak}
\HOLOGO@DeclareBoolOption{discretionarybreak}
%    \end{macrocode}
%
%    \begin{macrocode}
\kv@define@key{HoLogo}{variant}{%
  \ifx\HOLOGO@name\relax
    \@PackageError{hologo}{%
      Option `variant' is not available in \string\hologoSetup,%
      \MessageBreak
      Use \string\hologoLogoSetup\space instead%
    }\@ehc
  \else
    \edef\HOLOGO@temp{#1}%
    \ifx\HOLOGO@temp\ltx@empty
      \expandafter
      \let\csname HoLogoOpt@variant@\HOLOGO@name\endcsname\@undefined
    \else
      \ltx@IfUndefined{HoLogo@\HOLOGO@name @\HOLOGO@temp}{%
        \@PackageError{hologo}{%
          Unknown variant `\HOLOGO@temp' of logo `\HOLOGO@name'%
        }\@ehc
      }{%
        \expandafter
        \let\csname HoLogoOpt@variant@\HOLOGO@name\endcsname
            \HOLOGO@temp
      }%
    \fi
  \fi
}
%    \end{macrocode}
%
%    \begin{macro}{\HOLOGO@Variant}
%    \begin{macrocode}
\def\HOLOGO@Variant#1{%
  #1%
  \ltx@ifundefined{HoLogoOpt@variant@#1}{%
  }{%
    @\csname HoLogoOpt@variant@#1\endcsname
  }%
}
%    \end{macrocode}
%    \end{macro}
%
% \subsection{Break/no-break support}
%
%    \begin{macro}{\HOLOGO@space}
%    \begin{macrocode}
\def\HOLOGO@space{%
  \ltx@ifundefined{HoLogoOpt@spacebreak@\HOLOGO@name}{%
    \ltx@ifundefined{HoLogoOpt@break@\HOLOGO@name}{%
      \chardef\HOLOGO@temp=\HOLOGOOPT@spacebreak
    }{%
      \chardef\HOLOGO@temp=%
        \csname HoLogoOpt@break@\HOLOGO@name\endcsname
    }%
  }{%
    \chardef\HOLOGO@temp=%
      \csname HoLogoOpt@spacebreak@\HOLOGO@name\endcsname
  }%
  \ifcase\HOLOGO@temp
    \penalty10000 %
  \fi
  \ltx@space
}
%    \end{macrocode}
%    \end{macro}
%
%    \begin{macro}{\HOLOGO@hyphen}
%    \begin{macrocode}
\def\HOLOGO@hyphen{%
  \ltx@ifundefined{HoLogoOpt@hyphenbreak@\HOLOGO@name}{%
    \ltx@ifundefined{HoLogoOpt@break@\HOLOGO@name}{%
      \chardef\HOLOGO@temp=\HOLOGOOPT@hyphenbreak
    }{%
      \chardef\HOLOGO@temp=%
        \csname HoLogoOpt@break@\HOLOGO@name\endcsname
    }%
  }{%
    \chardef\HOLOGO@temp=%
      \csname HoLogoOpt@hyphenbreak@\HOLOGO@name\endcsname
  }%
  \ifcase\HOLOGO@temp
    \ltx@mbox{-}%
  \else
    -%
  \fi
}
%    \end{macrocode}
%    \end{macro}
%
%    \begin{macro}{\HOLOGO@discretionary}
%    \begin{macrocode}
\def\HOLOGO@discretionary{%
  \ltx@ifundefined{HoLogoOpt@discretionarybreak@\HOLOGO@name}{%
    \ltx@ifundefined{HoLogoOpt@break@\HOLOGO@name}{%
      \chardef\HOLOGO@temp=\HOLOGOOPT@discretionarybreak
    }{%
      \chardef\HOLOGO@temp=%
        \csname HoLogoOpt@break@\HOLOGO@name\endcsname
    }%
  }{%
    \chardef\HOLOGO@temp=%
      \csname HoLogoOpt@discretionarybreak@\HOLOGO@name\endcsname
  }%
  \ifcase\HOLOGO@temp
  \else
    \-%
  \fi
}
%    \end{macrocode}
%    \end{macro}
%
%    \begin{macro}{\HOLOGO@mbox}
%    \begin{macrocode}
\def\HOLOGO@mbox#1{%
  \ltx@ifundefined{HoLogoOpt@break@\HOLOGO@name}{%
    \chardef\HOLOGO@temp=\HOLOGOOPT@hyphenbreak
  }{%
    \chardef\HOLOGO@temp=%
      \csname HoLogoOpt@break@\HOLOGO@name\endcsname
  }%
  \ifcase\HOLOGO@temp
    \ltx@mbox{#1}%
  \else
    #1%
  \fi
}
%    \end{macrocode}
%    \end{macro}
%
% \subsection{Font support}
%
%    \begin{macro}{\HoLogoFont@font}
%    \begin{tabular}{@{}ll@{}}
%    |#1|:& logo name\\
%    |#2|:& font short name\\
%    |#3|:& text
%    \end{tabular}
%    \begin{macrocode}
\def\HoLogoFont@font#1#2#3{%
  \begingroup
    \ltx@IfUndefined{HoLogoFont@logo@#1.#2}{%
      \ltx@IfUndefined{HoLogoFont@font@#2}{%
        \@PackageWarning{hologo}{%
          Missing font `#2' for logo `#1'%
        }%
        #3%
      }{%
        \csname HoLogoFont@font@#2\endcsname{#3}%
      }%
    }{%
      \csname HoLogoFont@logo@#1.#2\endcsname{#3}%
    }%
  \endgroup
}
%    \end{macrocode}
%    \end{macro}
%
%    \begin{macro}{\HoLogoFont@Def}
%    \begin{macrocode}
\def\HoLogoFont@Def#1{%
  \expandafter\def\csname HoLogoFont@font@#1\endcsname
}
%    \end{macrocode}
%    \end{macro}
%    \begin{macro}{\HoLogoFont@LogoDef}
%    \begin{macrocode}
\def\HoLogoFont@LogoDef#1#2{%
  \expandafter\def\csname HoLogoFont@logo@#1.#2\endcsname
}
%    \end{macrocode}
%    \end{macro}
%
% \subsubsection{Font defaults}
%
%    \begin{macro}{\HoLogoFont@font@general}
%    \begin{macrocode}
\HoLogoFont@Def{general}{}%
%    \end{macrocode}
%    \end{macro}
%
%    \begin{macro}{\HoLogoFont@font@rm}
%    \begin{macrocode}
\ltx@IfUndefined{rmfamily}{%
  \ltx@IfUndefined{rm}{%
  }{%
    \HoLogoFont@Def{rm}{\rm}%
  }%
}{%
  \HoLogoFont@Def{rm}{\rmfamily}%
}
%    \end{macrocode}
%    \end{macro}
%
%    \begin{macro}{\HoLogoFont@font@sf}
%    \begin{macrocode}
\ltx@IfUndefined{sffamily}{%
  \ltx@IfUndefined{sf}{%
  }{%
    \HoLogoFont@Def{sf}{\sf}%
  }%
}{%
  \HoLogoFont@Def{sf}{\sffamily}%
}
%    \end{macrocode}
%    \end{macro}
%
%    \begin{macro}{\HoLogoFont@font@bibsf}
%    In case of \hologo{plainTeX} the original small caps
%    variant is used as default. In \hologo{LaTeX}
%    the definition of package \xpackage{dtklogos} \cite{dtklogos}
%    is used.
%\begin{quote}
%\begin{verbatim}
%\DeclareRobustCommand{\BibTeX}{%
%  B%
%  \kern-.05em%
%  \hbox{%
%    $\m@th$% %% force math size calculations
%    \csname S@\f@size\endcsname
%    \fontsize\sf@size\z@
%    \math@fontsfalse
%    \selectfont
%    I%
%    \kern-.025em%
%    B
%  }%
%  \kern-.08em%
%  \-%
%  \TeX
%}
%\end{verbatim}
%\end{quote}
%    \begin{macrocode}
\ltx@IfUndefined{selectfont}{%
  \ltx@IfUndefined{tensc}{%
    \font\tensc=cmcsc10\relax
  }{}%
  \HoLogoFont@Def{bibsf}{\tensc}%
}{%
  \HoLogoFont@Def{bibsf}{%
    $\mathsurround=0pt$%
    \csname S@\f@size\endcsname
    \fontsize\sf@size{0pt}%
    \math@fontsfalse
    \selectfont
  }%
}
%    \end{macrocode}
%    \end{macro}
%
%    \begin{macro}{\HoLogoFont@font@sc}
%    \begin{macrocode}
\ltx@IfUndefined{scshape}{%
  \ltx@IfUndefined{tensc}{%
    \font\tensc=cmcsc10\relax
  }{}%
  \HoLogoFont@Def{sc}{\tensc}%
}{%
  \HoLogoFont@Def{sc}{\scshape}%
}
%    \end{macrocode}
%    \end{macro}
%
%    \begin{macro}{\HoLogoFont@font@sy}
%    \begin{macrocode}
\ltx@IfUndefined{usefont}{%
  \ltx@IfUndefined{tensy}{%
  }{%
    \HoLogoFont@Def{sy}{\tensy}%
  }%
}{%
  \HoLogoFont@Def{sy}{%
    \usefont{OMS}{cmsy}{m}{n}%
  }%
}
%    \end{macrocode}
%    \end{macro}
%
%    \begin{macro}{\HoLogoFont@font@logo}
%    \begin{macrocode}
\begingroup
  \def\x{LaTeX2e}%
\expandafter\endgroup
\ifx\fmtname\x
  \ltx@IfUndefined{logofamily}{%
    \DeclareRobustCommand\logofamily{%
      \not@math@alphabet\logofamily\relax
      \fontencoding{U}%
      \fontfamily{logo}%
      \selectfont
    }%
  }{}%
  \ltx@IfUndefined{logofamily}{%
  }{%
    \HoLogoFont@Def{logo}{\logofamily}%
  }%
\else
  \ltx@IfUndefined{tenlogo}{%
    \font\tenlogo=logo10\relax
  }{}%
  \HoLogoFont@Def{logo}{\tenlogo}%
\fi
%    \end{macrocode}
%    \end{macro}
%
% \subsubsection{Font setup}
%
%    \begin{macro}{\hologoFontSetup}
%    \begin{macrocode}
\def\hologoFontSetup{%
  \let\HOLOGO@name\relax
  \HOLOGO@FontSetup
}
%    \end{macrocode}
%    \end{macro}
%
%    \begin{macro}{\hologoLogoFontSetup}
%    \begin{macrocode}
\def\hologoLogoFontSetup#1{%
  \edef\HOLOGO@name{#1}%
  \ltx@IfUndefined{HoLogo@\HOLOGO@name}{%
    \@PackageError{hologo}{%
      Unknown logo `\HOLOGO@name'%
    }\@ehc
    \ltx@gobble
  }{%
    \HOLOGO@FontSetup
  }%
}
%    \end{macrocode}
%    \end{macro}
%
%    \begin{macro}{\HOLOGO@FontSetup}
%    \begin{macrocode}
\def\HOLOGO@FontSetup{%
  \kvsetkeys{HoLogoFont}%
}
%    \end{macrocode}
%    \end{macro}
%
%    \begin{macrocode}
\def\HOLOGO@temp#1{%
  \kv@define@key{HoLogoFont}{#1}{%
    \ifx\HOLOGO@name\relax
      \HoLogoFont@Def{#1}{##1}%
    \else
      \HoLogoFont@LogoDef\HOLOGO@name{#1}{##1}%
    \fi
  }%
}
\HOLOGO@temp{general}
\HOLOGO@temp{sf}
%    \end{macrocode}
%
% \subsection{Generic logo commands}
%
%    \begin{macrocode}
\HOLOGO@IfExists\hologo{%
  \@PackageError{hologo}{%
    \string\hologo\ltx@space is already defined.\MessageBreak
    Package loading is aborted%
  }\@ehc
  \HOLOGO@AtEnd
}%
\HOLOGO@IfExists\hologoRobust{%
  \@PackageError{hologo}{%
    \string\hologoRobust\ltx@space is already defined.\MessageBreak
    Package loading is aborted%
  }\@ehc
  \HOLOGO@AtEnd
}%
%    \end{macrocode}
%
% \subsubsection{\cs{hologo} and friends}
%
%    \begin{macrocode}
\ifluatex
  \expandafter\ltx@firstofone
\else
  \expandafter\ltx@gobble
\fi
{%
  \ltx@IfUndefined{ifincsname}{%
    \ifnum\luatexversion<36 %
      \expandafter\ltx@gobble
    \else
      \expandafter\ltx@firstofone
    \fi
    {%
      \begingroup
        \ifcase0%
            \directlua{%
              if tex.enableprimitives then %
                tex.enableprimitives('HOLOGO@', {'ifincsname'})%
              else %
                tex.print('1')%
              end%
            }%
            \ifx\HOLOGO@ifincsname\@undefined 1\fi%
            \relax
          \expandafter\ltx@firstofone
        \else
          \endgroup
          \expandafter\ltx@gobble
        \fi
        {%
          \global\let\ifincsname\HOLOGO@ifincsname
        }%
      \HOLOGO@temp
    }%
  }{}%
}
%    \end{macrocode}
%    \begin{macrocode}
\ltx@IfUndefined{ifincsname}{%
  \catcode`$=14 %
}{%
  \catcode`$=9 %
}
%    \end{macrocode}
%
%    \begin{macro}{\hologo}
%    \begin{macrocode}
\def\hologo#1{%
$ \ifincsname
$   \ltx@ifundefined{HoLogoCs@\HOLOGO@Variant{#1}}{%
$     #1%
$   }{%
$     \csname HoLogoCs@\HOLOGO@Variant{#1}\endcsname\ltx@firstoftwo
$   }%
$ \else
    \HOLOGO@IfExists\texorpdfstring\texorpdfstring\ltx@firstoftwo
    {%
      \hologoRobust{#1}%
    }{%
      \ltx@ifundefined{HoLogoBkm@\HOLOGO@Variant{#1}}{%
        \ltx@ifundefined{HoLogo@#1}{?#1?}{#1}%
      }{%
        \csname HoLogoBkm@\HOLOGO@Variant{#1}\endcsname
        \ltx@firstoftwo
      }%
    }%
$ \fi
}
%    \end{macrocode}
%    \end{macro}
%    \begin{macro}{\Hologo}
%    \begin{macrocode}
\def\Hologo#1{%
$ \ifincsname
$   \ltx@ifundefined{HoLogoCs@\HOLOGO@Variant{#1}}{%
$     #1%
$   }{%
$     \csname HoLogoCs@\HOLOGO@Variant{#1}\endcsname\ltx@secondoftwo
$   }%
$ \else
    \HOLOGO@IfExists\texorpdfstring\texorpdfstring\ltx@firstoftwo
    {%
      \HologoRobust{#1}%
    }{%
      \ltx@ifundefined{HoLogoBkm@\HOLOGO@Variant{#1}}{%
        \ltx@ifundefined{HoLogo@#1}{?#1?}{#1}%
      }{%
        \csname HoLogoBkm@\HOLOGO@Variant{#1}\endcsname
        \ltx@secondoftwo
      }%
    }%
$ \fi
}
%    \end{macrocode}
%    \end{macro}
%
%    \begin{macro}{\hologoVariant}
%    \begin{macrocode}
\def\hologoVariant#1#2{%
  \ifx\relax#2\relax
    \hologo{#1}%
  \else
$   \ifincsname
$     \ltx@ifundefined{HoLogoCs@#1@#2}{%
$       #1%
$     }{%
$       \csname HoLogoCs@#1@#2\endcsname\ltx@firstoftwo
$     }%
$   \else
      \HOLOGO@IfExists\texorpdfstring\texorpdfstring\ltx@firstoftwo
      {%
        \hologoVariantRobust{#1}{#2}%
      }{%
        \ltx@ifundefined{HoLogoBkm@#1@#2}{%
          \ltx@ifundefined{HoLogo@#1}{?#1?}{#1}%
        }{%
          \csname HoLogoBkm@#1@#2\endcsname
          \ltx@firstoftwo
        }%
      }%
$   \fi
  \fi
}
%    \end{macrocode}
%    \end{macro}
%    \begin{macro}{\HologoVariant}
%    \begin{macrocode}
\def\HologoVariant#1#2{%
  \ifx\relax#2\relax
    \Hologo{#1}%
  \else
$   \ifincsname
$     \ltx@ifundefined{HoLogoCs@#1@#2}{%
$       #1%
$     }{%
$       \csname HoLogoCs@#1@#2\endcsname\ltx@secondoftwo
$     }%
$   \else
      \HOLOGO@IfExists\texorpdfstring\texorpdfstring\ltx@firstoftwo
      {%
        \HologoVariantRobust{#1}{#2}%
      }{%
        \ltx@ifundefined{HoLogoBkm@#1@#2}{%
          \ltx@ifundefined{HoLogo@#1}{?#1?}{#1}%
        }{%
          \csname HoLogoBkm@#1@#2\endcsname
          \ltx@secondoftwo
        }%
      }%
$   \fi
  \fi
}
%    \end{macrocode}
%    \end{macro}
%
%    \begin{macrocode}
\catcode`\$=3 %
%    \end{macrocode}
%
% \subsubsection{\cs{hologoRobust} and friends}
%
%    \begin{macro}{\hologoRobust}
%    \begin{macrocode}
\ltx@IfUndefined{protected}{%
  \ltx@IfUndefined{DeclareRobustCommand}{%
    \def\hologoRobust#1%
  }{%
    \DeclareRobustCommand*\hologoRobust[1]%
  }%
}{%
  \protected\def\hologoRobust#1%
}%
{%
  \edef\HOLOGO@name{#1}%
  \ltx@IfUndefined{HoLogo@\HOLOGO@Variant\HOLOGO@name}{%
    \@PackageError{hologo}{%
      Unknown logo `\HOLOGO@name'%
    }\@ehc
    ?\HOLOGO@name?%
  }{%
    \ltx@IfUndefined{ver@tex4ht.sty}{%
      \HoLogoFont@font\HOLOGO@name{general}{%
        \csname HoLogo@\HOLOGO@Variant\HOLOGO@name\endcsname
        \ltx@firstoftwo
      }%
    }{%
      \ltx@IfUndefined{HoLogoHtml@\HOLOGO@Variant\HOLOGO@name}{%
        \HOLOGO@name
      }{%
        \csname HoLogoHtml@\HOLOGO@Variant\HOLOGO@name\endcsname
        \ltx@firstoftwo
      }%
    }%
  }%
}
%    \end{macrocode}
%    \end{macro}
%    \begin{macro}{\HologoRobust}
%    \begin{macrocode}
\ltx@IfUndefined{protected}{%
  \ltx@IfUndefined{DeclareRobustCommand}{%
    \def\HologoRobust#1%
  }{%
    \DeclareRobustCommand*\HologoRobust[1]%
  }%
}{%
  \protected\def\HologoRobust#1%
}%
{%
  \edef\HOLOGO@name{#1}%
  \ltx@IfUndefined{HoLogo@\HOLOGO@Variant\HOLOGO@name}{%
    \@PackageError{hologo}{%
      Unknown logo `\HOLOGO@name'%
    }\@ehc
    ?\HOLOGO@name?%
  }{%
    \ltx@IfUndefined{ver@tex4ht.sty}{%
      \HoLogoFont@font\HOLOGO@name{general}{%
        \csname HoLogo@\HOLOGO@Variant\HOLOGO@name\endcsname
        \ltx@secondoftwo
      }%
    }{%
      \ltx@IfUndefined{HoLogoHtml@\HOLOGO@Variant\HOLOGO@name}{%
        \expandafter\HOLOGO@Uppercase\HOLOGO@name
      }{%
        \csname HoLogoHtml@\HOLOGO@Variant\HOLOGO@name\endcsname
        \ltx@secondoftwo
      }%
    }%
  }%
}
%    \end{macrocode}
%    \end{macro}
%    \begin{macro}{\hologoVariantRobust}
%    \begin{macrocode}
\ltx@IfUndefined{protected}{%
  \ltx@IfUndefined{DeclareRobustCommand}{%
    \def\hologoVariantRobust#1#2%
  }{%
    \DeclareRobustCommand*\hologoVariantRobust[2]%
  }%
}{%
  \protected\def\hologoVariantRobust#1#2%
}%
{%
  \begingroup
    \hologoLogoSetup{#1}{variant={#2}}%
    \hologoRobust{#1}%
  \endgroup
}
%    \end{macrocode}
%    \end{macro}
%    \begin{macro}{\HologoVariantRobust}
%    \begin{macrocode}
\ltx@IfUndefined{protected}{%
  \ltx@IfUndefined{DeclareRobustCommand}{%
    \def\HologoVariantRobust#1#2%
  }{%
    \DeclareRobustCommand*\HologoVariantRobust[2]%
  }%
}{%
  \protected\def\HologoVariantRobust#1#2%
}%
{%
  \begingroup
    \hologoLogoSetup{#1}{variant={#2}}%
    \HologoRobust{#1}%
  \endgroup
}
%    \end{macrocode}
%    \end{macro}
%
%    \begin{macro}{\hologorobust}
%    Macro \cs{hologorobust} is only defined for compatibility.
%    Its use is deprecated.
%    \begin{macrocode}
\def\hologorobust{\hologoRobust}
%    \end{macrocode}
%    \end{macro}
%
% \subsection{Helpers}
%
%    \begin{macro}{\HOLOGO@Uppercase}
%    Macro \cs{HOLOGO@Uppercase} is restricted to \cs{uppercase},
%    because \hologo{plainTeX} or \hologo{iniTeX} do not provide
%    \cs{MakeUppercase}.
%    \begin{macrocode}
\def\HOLOGO@Uppercase#1{\uppercase{#1}}
%    \end{macrocode}
%    \end{macro}
%
%    \begin{macro}{\HOLOGO@PdfdocUnicode}
%    \begin{macrocode}
\def\HOLOGO@PdfdocUnicode{%
  \ifx\ifHy@unicode\iftrue
    \expandafter\ltx@secondoftwo
  \else
    \expandafter\ltx@firstoftwo
  \fi
}
%    \end{macrocode}
%    \end{macro}
%
%    \begin{macro}{\HOLOGO@Math}
%    \begin{macrocode}
\def\HOLOGO@MathSetup{%
  \mathsurround0pt\relax
  \HOLOGO@IfExists\f@series{%
    \if b\expandafter\ltx@car\f@series x\@nil
      \csname boldmath\endcsname
   \fi
  }{}%
}
%    \end{macrocode}
%    \end{macro}
%
%    \begin{macro}{\HOLOGO@TempDimen}
%    \begin{macrocode}
\dimendef\HOLOGO@TempDimen=\ltx@zero
%    \end{macrocode}
%    \end{macro}
%    \begin{macro}{\HOLOGO@NegativeKerning}
%    \begin{macrocode}
\def\HOLOGO@NegativeKerning#1{%
  \begingroup
    \HOLOGO@TempDimen=0pt\relax
    \comma@parse@normalized{#1}{%
      \ifdim\HOLOGO@TempDimen=0pt %
        \expandafter\HOLOGO@@NegativeKerning\comma@entry
      \fi
      \ltx@gobble
    }%
    \ifdim\HOLOGO@TempDimen<0pt %
      \kern\HOLOGO@TempDimen
    \fi
  \endgroup
}
%    \end{macrocode}
%    \end{macro}
%    \begin{macro}{\HOLOGO@@NegativeKerning}
%    \begin{macrocode}
\def\HOLOGO@@NegativeKerning#1#2{%
  \setbox\ltx@zero\hbox{#1#2}%
  \HOLOGO@TempDimen=\wd\ltx@zero
  \setbox\ltx@zero\hbox{#1\kern0pt#2}%
  \advance\HOLOGO@TempDimen by -\wd\ltx@zero
}
%    \end{macrocode}
%    \end{macro}
%
%    \begin{macro}{\HOLOGO@SpaceFactor}
%    \begin{macrocode}
\def\HOLOGO@SpaceFactor{%
  \spacefactor1000 %
}
%    \end{macrocode}
%    \end{macro}
%
%    \begin{macro}{\HOLOGO@Span}
%    \begin{macrocode}
\def\HOLOGO@Span#1#2{%
  \HCode{<span class="HoLogo-#1">}%
  #2%
  \HCode{</span>}%
}
%    \end{macrocode}
%    \end{macro}
%
% \subsubsection{Text subscript}
%
%    \begin{macro}{\HOLOGO@SubScript}%
%    \begin{macrocode}
\def\HOLOGO@SubScript#1{%
  \ltx@IfUndefined{textsubscript}{%
    \ltx@IfUndefined{text}{%
      \ltx@mbox{%
        \mathsurround=0pt\relax
        $%
          _{%
            \ltx@IfUndefined{sf@size}{%
              \mathrm{#1}%
            }{%
              \mbox{%
                \fontsize\sf@size{0pt}\selectfont
                #1%
              }%
            }%
          }%
        $%
      }%
    }{%
      \ltx@mbox{%
        \mathsurround=0pt\relax
        $_{\text{#1}}$%
      }%
    }%
  }{%
    \textsubscript{#1}%
  }%
}
%    \end{macrocode}
%    \end{macro}
%
% \subsection{\hologo{TeX} and friends}
%
% \subsubsection{\hologo{TeX}}
%
%    \begin{macro}{\HoLogo@TeX}
%    Source: \hologo{LaTeX} kernel.
%    \begin{macrocode}
\def\HoLogo@TeX#1{%
  T\kern-.1667em\lower.5ex\hbox{E}\kern-.125emX\HOLOGO@SpaceFactor
}
%    \end{macrocode}
%    \end{macro}
%    \begin{macro}{\HoLogoHtml@TeX}
%    \begin{macrocode}
\def\HoLogoHtml@TeX#1{%
  \HoLogoCss@TeX
  \HOLOGO@Span{TeX}{%
    T%
    \HOLOGO@Span{e}{%
      E%
    }%
    X%
  }%
}
%    \end{macrocode}
%    \end{macro}
%    \begin{macro}{\HoLogoCss@TeX}
%    \begin{macrocode}
\def\HoLogoCss@TeX{%
  \Css{%
    span.HoLogo-TeX span.HoLogo-e{%
      position:relative;%
      top:.5ex;%
      margin-left:-.1667em;%
      margin-right:-.125em;%
    }%
  }%
  \Css{%
    a span.HoLogo-TeX span.HoLogo-e{%
      text-decoration:none;%
    }%
  }%
  \global\let\HoLogoCss@TeX\relax
}
%    \end{macrocode}
%    \end{macro}
%
% \subsubsection{\hologo{plainTeX}}
%
%    \begin{macro}{\HoLogo@plainTeX@space}
%    Source: ``The \hologo{TeX}book''
%    \begin{macrocode}
\def\HoLogo@plainTeX@space#1{%
  \HOLOGO@mbox{#1{p}{P}lain}\HOLOGO@space\hologo{TeX}%
}
%    \end{macrocode}
%    \end{macro}
%    \begin{macro}{\HoLogoCs@plainTeX@space}
%    \begin{macrocode}
\def\HoLogoCs@plainTeX@space#1{#1{p}{P}lain TeX}%
%    \end{macrocode}
%    \end{macro}
%    \begin{macro}{\HoLogoBkm@plainTeX@space}
%    \begin{macrocode}
\def\HoLogoBkm@plainTeX@space#1{%
  #1{p}{P}lain \hologo{TeX}%
}
%    \end{macrocode}
%    \end{macro}
%    \begin{macro}{\HoLogoHtml@plainTeX@space}
%    \begin{macrocode}
\def\HoLogoHtml@plainTeX@space#1{%
  #1{p}{P}lain \hologo{TeX}%
}
%    \end{macrocode}
%    \end{macro}
%
%    \begin{macro}{\HoLogo@plainTeX@hyphen}
%    \begin{macrocode}
\def\HoLogo@plainTeX@hyphen#1{%
  \HOLOGO@mbox{#1{p}{P}lain}\HOLOGO@hyphen\hologo{TeX}%
}
%    \end{macrocode}
%    \end{macro}
%    \begin{macro}{\HoLogoCs@plainTeX@hyphen}
%    \begin{macrocode}
\def\HoLogoCs@plainTeX@hyphen#1{#1{p}{P}lain-TeX}
%    \end{macrocode}
%    \end{macro}
%    \begin{macro}{\HoLogoBkm@plainTeX@hyphen}
%    \begin{macrocode}
\def\HoLogoBkm@plainTeX@hyphen#1{%
  #1{p}{P}lain-\hologo{TeX}%
}
%    \end{macrocode}
%    \end{macro}
%    \begin{macro}{\HoLogoHtml@plainTeX@hyphen}
%    \begin{macrocode}
\def\HoLogoHtml@plainTeX@hyphen#1{%
  #1{p}{P}lain-\hologo{TeX}%
}
%    \end{macrocode}
%    \end{macro}
%
%    \begin{macro}{\HoLogo@plainTeX@runtogether}
%    \begin{macrocode}
\def\HoLogo@plainTeX@runtogether#1{%
  \HOLOGO@mbox{#1{p}{P}lain\hologo{TeX}}%
}
%    \end{macrocode}
%    \end{macro}
%    \begin{macro}{\HoLogoCs@plainTeX@runtogether}
%    \begin{macrocode}
\def\HoLogoCs@plainTeX@runtogether#1{#1{p}{P}lainTeX}
%    \end{macrocode}
%    \end{macro}
%    \begin{macro}{\HoLogoBkm@plainTeX@runtogether}
%    \begin{macrocode}
\def\HoLogoBkm@plainTeX@runtogether#1{%
  #1{p}{P}lain\hologo{TeX}%
}
%    \end{macrocode}
%    \end{macro}
%    \begin{macro}{\HoLogoHtml@plainTeX@runtogether}
%    \begin{macrocode}
\def\HoLogoHtml@plainTeX@runtogether#1{%
  #1{p}{P}lain\hologo{TeX}%
}
%    \end{macrocode}
%    \end{macro}
%
%    \begin{macro}{\HoLogo@plainTeX}
%    \begin{macrocode}
\def\HoLogo@plainTeX{\HoLogo@plainTeX@space}
%    \end{macrocode}
%    \end{macro}
%    \begin{macro}{\HoLogoCs@plainTeX}
%    \begin{macrocode}
\def\HoLogoCs@plainTeX{\HoLogoCs@plainTeX@space}
%    \end{macrocode}
%    \end{macro}
%    \begin{macro}{\HoLogoBkm@plainTeX}
%    \begin{macrocode}
\def\HoLogoBkm@plainTeX{\HoLogoBkm@plainTeX@space}
%    \end{macrocode}
%    \end{macro}
%    \begin{macro}{\HoLogoHtml@plainTeX}
%    \begin{macrocode}
\def\HoLogoHtml@plainTeX{\HoLogoHtml@plainTeX@space}
%    \end{macrocode}
%    \end{macro}
%
% \subsubsection{\hologo{LaTeX}}
%
%    Source: \hologo{LaTeX} kernel.
%\begin{quote}
%\begin{verbatim}
%\DeclareRobustCommand{\LaTeX}{%
%  L%
%  \kern-.36em%
%  {%
%    \sbox\z@ T%
%    \vbox to\ht\z@{%
%      \hbox{%
%        \check@mathfonts
%        \fontsize\sf@size\z@
%        \math@fontsfalse
%        \selectfont
%        A%
%      }%
%      \vss
%    }%
%  }%
%  \kern-.15em%
%  \TeX
%}
%\end{verbatim}
%\end{quote}
%
%    \begin{macro}{\HoLogo@La}
%    \begin{macrocode}
\def\HoLogo@La#1{%
  L%
  \kern-.36em%
  \begingroup
    \setbox\ltx@zero\hbox{T}%
    \vbox to\ht\ltx@zero{%
      \hbox{%
        \ltx@ifundefined{check@mathfonts}{%
          \csname sevenrm\endcsname
        }{%
          \check@mathfonts
          \fontsize\sf@size{0pt}%
          \math@fontsfalse\selectfont
        }%
        A%
      }%
      \vss
    }%
  \endgroup
}
%    \end{macrocode}
%    \end{macro}
%
%    \begin{macro}{\HoLogo@LaTeX}
%    Source: \hologo{LaTeX} kernel.
%    \begin{macrocode}
\def\HoLogo@LaTeX#1{%
  \hologo{La}%
  \kern-.15em%
  \hologo{TeX}%
}
%    \end{macrocode}
%    \end{macro}
%    \begin{macro}{\HoLogoHtml@LaTeX}
%    \begin{macrocode}
\def\HoLogoHtml@LaTeX#1{%
  \HoLogoCss@LaTeX
  \HOLOGO@Span{LaTeX}{%
    L%
    \HOLOGO@Span{a}{%
      A%
    }%
    \hologo{TeX}%
  }%
}
%    \end{macrocode}
%    \end{macro}
%    \begin{macro}{\HoLogoCss@LaTeX}
%    \begin{macrocode}
\def\HoLogoCss@LaTeX{%
  \Css{%
    span.HoLogo-LaTeX span.HoLogo-a{%
      position:relative;%
      top:-.5ex;%
      margin-left:-.36em;%
      margin-right:-.15em;%
      font-size:85\%;%
    }%
  }%
  \global\let\HoLogoCss@LaTeX\relax
}
%    \end{macrocode}
%    \end{macro}
%
% \subsubsection{\hologo{(La)TeX}}
%
%    \begin{macro}{\HoLogo@LaTeXTeX}
%    The kerning around the parentheses is taken
%    from package \xpackage{dtklogos} \cite{dtklogos}.
%\begin{quote}
%\begin{verbatim}
%\DeclareRobustCommand{\LaTeXTeX}{%
%  (%
%  \kern-.15em%
%  L%
%  \kern-.36em%
%  {%
%    \sbox\z@ T%
%    \vbox to\ht0{%
%      \hbox{%
%        $\m@th$%
%        \csname S@\f@size\endcsname
%        \fontsize\sf@size\z@
%        \math@fontsfalse
%        \selectfont
%        A%
%      }%
%      \vss
%    }%
%  }%
%  \kern-.2em%
%  )%
%  \kern-.15em%
%  \TeX
%}
%\end{verbatim}
%\end{quote}
%    \begin{macrocode}
\def\HoLogo@LaTeXTeX#1{%
  (%
  \kern-.15em%
  \hologo{La}%
  \kern-.2em%
  )%
  \kern-.15em%
  \hologo{TeX}%
}
%    \end{macrocode}
%    \end{macro}
%    \begin{macro}{\HoLogoBkm@LaTeXTeX}
%    \begin{macrocode}
\def\HoLogoBkm@LaTeXTeX#1{(La)TeX}
%    \end{macrocode}
%    \end{macro}
%
%    \begin{macro}{\HoLogo@(La)TeX}
%    \begin{macrocode}
\expandafter
\let\csname HoLogo@(La)TeX\endcsname\HoLogo@LaTeXTeX
%    \end{macrocode}
%    \end{macro}
%    \begin{macro}{\HoLogoBkm@(La)TeX}
%    \begin{macrocode}
\expandafter
\let\csname HoLogoBkm@(La)TeX\endcsname\HoLogoBkm@LaTeXTeX
%    \end{macrocode}
%    \end{macro}
%    \begin{macro}{\HoLogoHtml@LaTeXTeX}
%    \begin{macrocode}
\def\HoLogoHtml@LaTeXTeX#1{%
  \HoLogoCss@LaTeXTeX
  \HOLOGO@Span{LaTeXTeX}{%
    (%
    \HOLOGO@Span{L}{L}%
    \HOLOGO@Span{a}{A}%
    \HOLOGO@Span{ParenRight}{)}%
    \hologo{TeX}%
  }%
}
%    \end{macrocode}
%    \end{macro}
%    \begin{macro}{\HoLogoHtml@(La)TeX}
%    Kerning after opening parentheses and before closing parentheses
%    is $-0.1$\,em. The original values $-0.15$\,em
%    looked too ugly for a serif font.
%    \begin{macrocode}
\expandafter
\let\csname HoLogoHtml@(La)TeX\endcsname\HoLogoHtml@LaTeXTeX
%    \end{macrocode}
%    \end{macro}
%    \begin{macro}{\HoLogoCss@LaTeXTeX}
%    \begin{macrocode}
\def\HoLogoCss@LaTeXTeX{%
  \Css{%
    span.HoLogo-LaTeXTeX span.HoLogo-L{%
      margin-left:-.1em;%
    }%
  }%
  \Css{%
    span.HoLogo-LaTeXTeX span.HoLogo-a{%
      position:relative;%
      top:-.5ex;%
      margin-left:-.36em;%
      margin-right:-.1em;%
      font-size:85\%;%
    }%
  }%
  \Css{%
    span.HoLogo-LaTeXTeX span.HoLogo-ParenRight{%
      margin-right:-.15em;%
    }%
  }%
  \global\let\HoLogoCss@LaTeXTeX\relax
}
%    \end{macrocode}
%    \end{macro}
%
% \subsubsection{\hologo{LaTeXe}}
%
%    \begin{macro}{\HoLogo@LaTeXe}
%    Source: \hologo{LaTeX} kernel
%    \begin{macrocode}
\def\HoLogo@LaTeXe#1{%
  \hologo{LaTeX}%
  \kern.15em%
  \hbox{%
    \HOLOGO@MathSetup
    2%
    $_{\textstyle\varepsilon}$%
  }%
}
%    \end{macrocode}
%    \end{macro}
%
%    \begin{macro}{\HoLogoCs@LaTeXe}
%    \begin{macrocode}
\ifnum64=`\^^^^0040\relax % test for big chars of LuaTeX/XeTeX
  \catcode`\$=9 %
  \catcode`\&=14 %
\else
  \catcode`\$=14 %
  \catcode`\&=9 %
\fi
\def\HoLogoCs@LaTeXe#1{%
  LaTeX2%
$ \string ^^^^0395%
& e%
}%
\catcode`\$=3 %
\catcode`\&=4 %
%    \end{macrocode}
%    \end{macro}
%
%    \begin{macro}{\HoLogoBkm@LaTeXe}
%    \begin{macrocode}
\def\HoLogoBkm@LaTeXe#1{%
  \hologo{LaTeX}%
  2%
  \HOLOGO@PdfdocUnicode{e}{\textepsilon}%
}
%    \end{macrocode}
%    \end{macro}
%
%    \begin{macro}{\HoLogoHtml@LaTeXe}
%    \begin{macrocode}
\def\HoLogoHtml@LaTeXe#1{%
  \HoLogoCss@LaTeXe
  \HOLOGO@Span{LaTeX2e}{%
    \hologo{LaTeX}%
    \HOLOGO@Span{2}{2}%
    \HOLOGO@Span{e}{%
      \HOLOGO@MathSetup
      \ensuremath{\textstyle\varepsilon}%
    }%
  }%
}
%    \end{macrocode}
%    \end{macro}
%    \begin{macro}{\HoLogoCss@LaTeXe}
%    \begin{macrocode}
\def\HoLogoCss@LaTeXe{%
  \Css{%
    span.HoLogo-LaTeX2e span.HoLogo-2{%
      padding-left:.15em;%
    }%
  }%
  \Css{%
    span.HoLogo-LaTeX2e span.HoLogo-e{%
      position:relative;%
      top:.35ex;%
      text-decoration:none;%
    }%
  }%
  \global\let\HoLogoCss@LaTeXe\relax
}
%    \end{macrocode}
%    \end{macro}
%
%    \begin{macro}{\HoLogo@LaTeX2e}
%    \begin{macrocode}
\expandafter
\let\csname HoLogo@LaTeX2e\endcsname\HoLogo@LaTeXe
%    \end{macrocode}
%    \end{macro}
%    \begin{macro}{\HoLogoCs@LaTeX2e}
%    \begin{macrocode}
\expandafter
\let\csname HoLogoCs@LaTeX2e\endcsname\HoLogoCs@LaTeXe
%    \end{macrocode}
%    \end{macro}
%    \begin{macro}{\HoLogoBkm@LaTeX2e}
%    \begin{macrocode}
\expandafter
\let\csname HoLogoBkm@LaTeX2e\endcsname\HoLogoBkm@LaTeXe
%    \end{macrocode}
%    \end{macro}
%    \begin{macro}{\HoLogoHtml@LaTeX2e}
%    \begin{macrocode}
\expandafter
\let\csname HoLogoHtml@LaTeX2e\endcsname\HoLogoHtml@LaTeXe
%    \end{macrocode}
%    \end{macro}
%
% \subsubsection{\hologo{LaTeX3}}
%
%    \begin{macro}{\HoLogo@LaTeX3}
%    Source: \hologo{LaTeX} kernel
%    \begin{macrocode}
\expandafter\def\csname HoLogo@LaTeX3\endcsname#1{%
  \hologo{LaTeX}%
  3%
}
%    \end{macrocode}
%    \end{macro}
%
%    \begin{macro}{\HoLogoBkm@LaTeX3}
%    \begin{macrocode}
\expandafter\def\csname HoLogoBkm@LaTeX3\endcsname#1{%
  \hologo{LaTeX}%
  3%
}
%    \end{macrocode}
%    \end{macro}
%    \begin{macro}{\HoLogoHtml@LaTeX3}
%    \begin{macrocode}
\expandafter
\let\csname HoLogoHtml@LaTeX3\expandafter\endcsname
\csname HoLogo@LaTeX3\endcsname
%    \end{macrocode}
%    \end{macro}
%
% \subsubsection{\hologo{LaTeXML}}
%
%    \begin{macro}{\HoLogo@LaTeXML}
%    \begin{macrocode}
\def\HoLogo@LaTeXML#1{%
  \HOLOGO@mbox{%
    \hologo{La}%
    \kern-.15em%
    T%
    \kern-.1667em%
    \lower.5ex\hbox{E}%
    \kern-.125em%
    \HoLogoFont@font{LaTeXML}{sc}{xml}%
  }%
}
%    \end{macrocode}
%    \end{macro}
%    \begin{macro}{\HoLogoHtml@pdfLaTeX}
%    \begin{macrocode}
\def\HoLogoHtml@LaTeXML#1{%
  \HOLOGO@Span{LaTeXML}{%
    \HoLogoCss@LaTeX
    \HoLogoCss@TeX
    \HOLOGO@Span{LaTeX}{%
      L%
      \HOLOGO@Span{a}{%
        A%
      }%
    }%
    \HOLOGO@Span{TeX}{%
      T%
      \HOLOGO@Span{e}{%
        E%
      }%
    }%
    \HCode{<span style="font-variant: small-caps;">}%
    xml%
    \HCode{</span>}%
  }%
}
%    \end{macrocode}
%    \end{macro}
%
% \subsubsection{\hologo{eTeX}}
%
%    \begin{macro}{\HoLogo@eTeX}
%    Source: package \xpackage{etex}
%    \begin{macrocode}
\def\HoLogo@eTeX#1{%
  \ltx@mbox{%
    \HOLOGO@MathSetup
    $\varepsilon$%
    -%
    \HOLOGO@NegativeKerning{-T,T-,To}%
    \hologo{TeX}%
  }%
}
%    \end{macrocode}
%    \end{macro}
%    \begin{macro}{\HoLogoCs@eTeX}
%    \begin{macrocode}
\ifnum64=`\^^^^0040\relax % test for big chars of LuaTeX/XeTeX
  \catcode`\$=9 %
  \catcode`\&=14 %
\else
  \catcode`\$=14 %
  \catcode`\&=9 %
\fi
\def\HoLogoCs@eTeX#1{%
$ #1{\string ^^^^0395}{\string ^^^^03b5}%
& #1{e}{E}%
  TeX%
}%
\catcode`\$=3 %
\catcode`\&=4 %
%    \end{macrocode}
%    \end{macro}
%    \begin{macro}{\HoLogoBkm@eTeX}
%    \begin{macrocode}
\def\HoLogoBkm@eTeX#1{%
  \HOLOGO@PdfdocUnicode{#1{e}{E}}{\textepsilon}%
  -%
  \hologo{TeX}%
}
%    \end{macrocode}
%    \end{macro}
%    \begin{macro}{\HoLogoHtml@eTeX}
%    \begin{macrocode}
\def\HoLogoHtml@eTeX#1{%
  \ltx@mbox{%
    \HOLOGO@MathSetup
    $\varepsilon$%
    -%
    \hologo{TeX}%
  }%
}
%    \end{macrocode}
%    \end{macro}
%
% \subsubsection{\hologo{iniTeX}}
%
%    \begin{macro}{\HoLogo@iniTeX}
%    \begin{macrocode}
\def\HoLogo@iniTeX#1{%
  \HOLOGO@mbox{%
    #1{i}{I}ni\hologo{TeX}%
  }%
}
%    \end{macrocode}
%    \end{macro}
%    \begin{macro}{\HoLogoCs@iniTeX}
%    \begin{macrocode}
\def\HoLogoCs@iniTeX#1{#1{i}{I}niTeX}
%    \end{macrocode}
%    \end{macro}
%    \begin{macro}{\HoLogoBkm@iniTeX}
%    \begin{macrocode}
\def\HoLogoBkm@iniTeX#1{%
  #1{i}{I}ni\hologo{TeX}%
}
%    \end{macrocode}
%    \end{macro}
%    \begin{macro}{\HoLogoHtml@iniTeX}
%    \begin{macrocode}
\let\HoLogoHtml@iniTeX\HoLogo@iniTeX
%    \end{macrocode}
%    \end{macro}
%
% \subsubsection{\hologo{virTeX}}
%
%    \begin{macro}{\HoLogo@virTeX}
%    \begin{macrocode}
\def\HoLogo@virTeX#1{%
  \HOLOGO@mbox{%
    #1{v}{V}ir\hologo{TeX}%
  }%
}
%    \end{macrocode}
%    \end{macro}
%    \begin{macro}{\HoLogoCs@virTeX}
%    \begin{macrocode}
\def\HoLogoCs@virTeX#1{#1{v}{V}irTeX}
%    \end{macrocode}
%    \end{macro}
%    \begin{macro}{\HoLogoBkm@virTeX}
%    \begin{macrocode}
\def\HoLogoBkm@virTeX#1{%
  #1{v}{V}ir\hologo{TeX}%
}
%    \end{macrocode}
%    \end{macro}
%    \begin{macro}{\HoLogoHtml@virTeX}
%    \begin{macrocode}
\let\HoLogoHtml@virTeX\HoLogo@virTeX
%    \end{macrocode}
%    \end{macro}
%
% \subsubsection{\hologo{SliTeX}}
%
% \paragraph{Definitions of the three variants.}
%
%    \begin{macro}{\HoLogo@SLiTeX@lift}
%    \begin{macrocode}
\def\HoLogo@SLiTeX@lift#1{%
  \HoLogoFont@font{SliTeX}{rm}{%
    S%
    \kern-.06em%
    L%
    \kern-.18em%
    \raise.32ex\hbox{\HoLogoFont@font{SliTeX}{sc}{i}}%
    \HOLOGO@discretionary
    \kern-.06em%
    \hologo{TeX}%
  }%
}
%    \end{macrocode}
%    \end{macro}
%    \begin{macro}{\HoLogoBkm@SLiTeX@lift}
%    \begin{macrocode}
\def\HoLogoBkm@SLiTeX@lift#1{SLiTeX}
%    \end{macrocode}
%    \end{macro}
%    \begin{macro}{\HoLogoHtml@SLiTeX@lift}
%    \begin{macrocode}
\def\HoLogoHtml@SLiTeX@lift#1{%
  \HoLogoCss@SLiTeX@lift
  \HOLOGO@Span{SLiTeX-lift}{%
    \HoLogoFont@font{SliTeX}{rm}{%
      S%
      \HOLOGO@Span{L}{L}%
      \HOLOGO@Span{i}{i}%
      \hologo{TeX}%
    }%
  }%
}
%    \end{macrocode}
%    \end{macro}
%    \begin{macro}{\HoLogoCss@SLiTeX@lift}
%    \begin{macrocode}
\def\HoLogoCss@SLiTeX@lift{%
  \Css{%
    span.HoLogo-SLiTeX-lift span.HoLogo-L{%
      margin-left:-.06em;%
      margin-right:-.18em;%
    }%
  }%
  \Css{%
    span.HoLogo-SLiTeX-lift span.HoLogo-i{%
      position:relative;%
      top:-.32ex;%
      margin-right:-.06em;%
      font-variant:small-caps;%
    }%
  }%
  \global\let\HoLogoCss@SLiTeX@lift\relax
}
%    \end{macrocode}
%    \end{macro}
%
%    \begin{macro}{\HoLogo@SliTeX@simple}
%    \begin{macrocode}
\def\HoLogo@SliTeX@simple#1{%
  \HoLogoFont@font{SliTeX}{rm}{%
    \ltx@mbox{%
      \HoLogoFont@font{SliTeX}{sc}{Sli}%
    }%
    \HOLOGO@discretionary
    \hologo{TeX}%
  }%
}
%    \end{macrocode}
%    \end{macro}
%    \begin{macro}{\HoLogoBkm@SliTeX@simple}
%    \begin{macrocode}
\def\HoLogoBkm@SliTeX@simple#1{SliTeX}
%    \end{macrocode}
%    \end{macro}
%    \begin{macro}{\HoLogoHtml@SliTeX@simple}
%    \begin{macrocode}
\let\HoLogoHtml@SliTeX@simple\HoLogo@SliTeX@simple
%    \end{macrocode}
%    \end{macro}
%
%    \begin{macro}{\HoLogo@SliTeX@narrow}
%    \begin{macrocode}
\def\HoLogo@SliTeX@narrow#1{%
  \HoLogoFont@font{SliTeX}{rm}{%
    \ltx@mbox{%
      S%
      \kern-.06em%
      \HoLogoFont@font{SliTeX}{sc}{%
        l%
        \kern-.035em%
        i%
      }%
    }%
    \HOLOGO@discretionary
    \kern-.06em%
    \hologo{TeX}%
  }%
}
%    \end{macrocode}
%    \end{macro}
%    \begin{macro}{\HoLogoBkm@SliTeX@narrow}
%    \begin{macrocode}
\def\HoLogoBkm@SliTeX@narrow#1{SliTeX}
%    \end{macrocode}
%    \end{macro}
%    \begin{macro}{\HoLogoHtml@SliTeX@narrow}
%    \begin{macrocode}
\def\HoLogoHtml@SliTeX@narrow#1{%
  \HoLogoCss@SliTeX@narrow
  \HOLOGO@Span{SliTeX-narrow}{%
    \HoLogoFont@font{SliTeX}{rm}{%
      S%
        \HOLOGO@Span{l}{l}%
        \HOLOGO@Span{i}{i}%
      \hologo{TeX}%
    }%
  }%
}
%    \end{macrocode}
%    \end{macro}
%    \begin{macro}{\HoLogoCss@SliTeX@narrow}
%    \begin{macrocode}
\def\HoLogoCss@SliTeX@narrow{%
  \Css{%
    span.HoLogo-SliTeX-narrow span.HoLogo-l{%
      margin-left:-.06em;%
      margin-right:-.035em;%
      font-variant:small-caps;%
    }%
  }%
  \Css{%
    span.HoLogo-SliTeX-narrow span.HoLogo-i{%
      margin-right:-.06em;%
      font-variant:small-caps;%
    }%
  }%
  \global\let\HoLogoCss@SliTeX@narrow\relax
}
%    \end{macrocode}
%    \end{macro}
%
% \paragraph{Macro set completion.}
%
%    \begin{macro}{\HoLogo@SLiTeX@simple}
%    \begin{macrocode}
\def\HoLogo@SLiTeX@simple{\HoLogo@SliTeX@simple}
%    \end{macrocode}
%    \end{macro}
%    \begin{macro}{\HoLogoBkm@SLiTeX@simple}
%    \begin{macrocode}
\def\HoLogoBkm@SLiTeX@simple{\HoLogoBkm@SliTeX@simple}
%    \end{macrocode}
%    \end{macro}
%    \begin{macro}{\HoLogoHtml@SLiTeX@simple}
%    \begin{macrocode}
\def\HoLogoHtml@SLiTeX@simple{\HoLogoHtml@SliTeX@simple}
%    \end{macrocode}
%    \end{macro}
%
%    \begin{macro}{\HoLogo@SLiTeX@narrow}
%    \begin{macrocode}
\def\HoLogo@SLiTeX@narrow{\HoLogo@SliTeX@narrow}
%    \end{macrocode}
%    \end{macro}
%    \begin{macro}{\HoLogoBkm@SLiTeX@narrow}
%    \begin{macrocode}
\def\HoLogoBkm@SLiTeX@narrow{\HoLogoBkm@SliTeX@narrow}
%    \end{macrocode}
%    \end{macro}
%    \begin{macro}{\HoLogoHtml@SLiTeX@narrow}
%    \begin{macrocode}
\def\HoLogoHtml@SLiTeX@narrow{\HoLogoHtml@SliTeX@narrow}
%    \end{macrocode}
%    \end{macro}
%
%    \begin{macro}{\HoLogo@SliTeX@lift}
%    \begin{macrocode}
\def\HoLogo@SliTeX@lift{\HoLogo@SLiTeX@lift}
%    \end{macrocode}
%    \end{macro}
%    \begin{macro}{\HoLogoBkm@SliTeX@lift}
%    \begin{macrocode}
\def\HoLogoBkm@SliTeX@lift{\HoLogoBkm@SLiTeX@lift}
%    \end{macrocode}
%    \end{macro}
%    \begin{macro}{\HoLogoHtml@SliTeX@lift}
%    \begin{macrocode}
\def\HoLogoHtml@SliTeX@lift{\HoLogoHtml@SLiTeX@lift}
%    \end{macrocode}
%    \end{macro}
%
% \paragraph{Defaults.}
%
%    \begin{macro}{\HoLogo@SLiTeX}
%    \begin{macrocode}
\def\HoLogo@SLiTeX{\HoLogo@SLiTeX@lift}
%    \end{macrocode}
%    \end{macro}
%    \begin{macro}{\HoLogoBkm@SLiTeX}
%    \begin{macrocode}
\def\HoLogoBkm@SLiTeX{\HoLogoBkm@SLiTeX@lift}
%    \end{macrocode}
%    \end{macro}
%    \begin{macro}{\HoLogoHtml@SLiTeX}
%    \begin{macrocode}
\def\HoLogoHtml@SLiTeX{\HoLogoHtml@SLiTeX@lift}
%    \end{macrocode}
%    \end{macro}
%
%    \begin{macro}{\HoLogo@SliTeX}
%    \begin{macrocode}
\def\HoLogo@SliTeX{\HoLogo@SliTeX@narrow}
%    \end{macrocode}
%    \end{macro}
%    \begin{macro}{\HoLogoBkm@SliTeX}
%    \begin{macrocode}
\def\HoLogoBkm@SliTeX{\HoLogoBkm@SliTeX@narrow}
%    \end{macrocode}
%    \end{macro}
%    \begin{macro}{\HoLogoHtml@SliTeX}
%    \begin{macrocode}
\def\HoLogoHtml@SliTeX{\HoLogoHtml@SliTeX@narrow}
%    \end{macrocode}
%    \end{macro}
%
% \subsubsection{\hologo{LuaTeX}}
%
%    \begin{macro}{\HoLogo@LuaTeX}
%    The kerning is an idea of Hans Hagen, see mailing list
%    `luatex at tug dot org' in March 2010.
%    \begin{macrocode}
\def\HoLogo@LuaTeX#1{%
  \HOLOGO@mbox{%
    Lua%
    \HOLOGO@NegativeKerning{aT,oT,To}%
    \hologo{TeX}%
  }%
}
%    \end{macrocode}
%    \end{macro}
%    \begin{macro}{\HoLogoHtml@LuaTeX}
%    \begin{macrocode}
\let\HoLogoHtml@LuaTeX\HoLogo@LuaTeX
%    \end{macrocode}
%    \end{macro}
%
% \subsubsection{\hologo{LuaLaTeX}}
%
%    \begin{macro}{\HoLogo@LuaLaTeX}
%    \begin{macrocode}
\def\HoLogo@LuaLaTeX#1{%
  \HOLOGO@mbox{%
    Lua%
    \hologo{LaTeX}%
  }%
}
%    \end{macrocode}
%    \end{macro}
%    \begin{macro}{\HoLogoHtml@LuaLaTeX}
%    \begin{macrocode}
\let\HoLogoHtml@LuaLaTeX\HoLogo@LuaLaTeX
%    \end{macrocode}
%    \end{macro}
%
% \subsubsection{\hologo{XeTeX}, \hologo{XeLaTeX}}
%
%    \begin{macro}{\HOLOGO@IfCharExists}
%    \begin{macrocode}
\ifluatex
  \ifnum\luatexversion<36 %
  \else
    \def\HOLOGO@IfCharExists#1{%
      \ifnum
        \directlua{%
           if luaotfload and luaotfload.aux then
             if luaotfload.aux.font_has_glyph(%
                    font.current(), \number#1) then % 	 
	       tex.print("1") % 	 
	     end % 	 
	   elseif font and font.fonts and font.current then %
            local f = font.fonts[font.current()]%
            if f.characters and f.characters[\number#1] then %
              tex.print("1")%
            end %
          end%
        }0=\ltx@zero
        \expandafter\ltx@secondoftwo
      \else
        \expandafter\ltx@firstoftwo
      \fi
    }%
  \fi
\fi
\ltx@IfUndefined{HOLOGO@IfCharExists}{%
  \def\HOLOGO@@IfCharExists#1{%
    \begingroup
      \tracinglostchars=\ltx@zero
      \setbox\ltx@zero=\hbox{%
        \kern7sp\char#1\relax
        \ifnum\lastkern>\ltx@zero
          \expandafter\aftergroup\csname iffalse\endcsname
        \else
          \expandafter\aftergroup\csname iftrue\endcsname
        \fi
      }%
      % \if{true|false} from \aftergroup
      \endgroup
      \expandafter\ltx@firstoftwo
    \else
      \endgroup
      \expandafter\ltx@secondoftwo
    \fi
  }%
  \ifxetex
    \ltx@IfUndefined{XeTeXfonttype}{}{%
      \ltx@IfUndefined{XeTeXcharglyph}{}{%
        \def\HOLOGO@IfCharExists#1{%
          \ifnum\XeTeXfonttype\font>\ltx@zero
            \expandafter\ltx@firstofthree
          \else
            \expandafter\ltx@gobble
          \fi
          {%
            \ifnum\XeTeXcharglyph#1>\ltx@zero
              \expandafter\ltx@firstoftwo
            \else
              \expandafter\ltx@secondoftwo
            \fi
          }%
          \HOLOGO@@IfCharExists{#1}%
        }%
      }%
    }%
  \fi
}{}
\ltx@ifundefined{HOLOGO@IfCharExists}{%
  \ifnum64=`\^^^^0040\relax % test for big chars of LuaTeX/XeTeX
    \let\HOLOGO@IfCharExists\HOLOGO@@IfCharExists
  \else
    \def\HOLOGO@IfCharExists#1{%
      \ifnum#1>255 %
        \expandafter\ltx@fourthoffour
      \fi
      \HOLOGO@@IfCharExists{#1}%
    }%
  \fi
}{}
%    \end{macrocode}
%    \end{macro}
%
%    \begin{macro}{\HoLogo@Xe}
%    Source: package \xpackage{dtklogos}
%    \begin{macrocode}
\def\HoLogo@Xe#1{%
  X%
  \kern-.1em\relax
  \HOLOGO@IfCharExists{"018E}{%
    \lower.5ex\hbox{\char"018E}%
  }{%
    \chardef\HOLOGO@choice=\ltx@zero
    \ifdim\fontdimen\ltx@one\font>0pt %
      \ltx@IfUndefined{rotatebox}{%
        \ltx@IfUndefined{pgftext}{%
          \ltx@IfUndefined{psscalebox}{%
            \ltx@IfUndefined{HOLOGO@ScaleBox@\hologoDriver}{%
            }{%
              \chardef\HOLOGO@choice=4 %
            }%
          }{%
            \chardef\HOLOGO@choice=3 %
          }%
        }{%
          \chardef\HOLOGO@choice=2 %
        }%
      }{%
        \chardef\HOLOGO@choice=1 %
      }%
      \ifcase\HOLOGO@choice
        \HOLOGO@WarningUnsupportedDriver{Xe}%
        e%
      \or % 1: \rotatebox
        \begingroup
          \setbox\ltx@zero\hbox{\rotatebox{180}{E}}%
          \ltx@LocDimenA=\dp\ltx@zero
          \advance\ltx@LocDimenA by -.5ex\relax
          \raise\ltx@LocDimenA\box\ltx@zero
        \endgroup
      \or % 2: \pgftext
        \lower.5ex\hbox{%
          \pgfpicture
            \pgftext[rotate=180]{E}%
          \endpgfpicture
        }%
      \or % 3: \psscalebox
        \begingroup
          \setbox\ltx@zero\hbox{\psscalebox{-1 -1}{E}}%
          \ltx@LocDimenA=\dp\ltx@zero
          \advance\ltx@LocDimenA by -.5ex\relax
          \raise\ltx@LocDimenA\box\ltx@zero
        \endgroup
      \or % 4: \HOLOGO@PointReflectBox
        \lower.5ex\hbox{\HOLOGO@PointReflectBox{E}}%
      \else
        \@PackageError{hologo}{Internal error (choice/it}\@ehc
      \fi
    \else
      \ltx@IfUndefined{reflectbox}{%
        \ltx@IfUndefined{pgftext}{%
          \ltx@IfUndefined{psscalebox}{%
            \ltx@IfUndefined{HOLOGO@ScaleBox@\hologoDriver}{%
            }{%
              \chardef\HOLOGO@choice=4 %
            }%
          }{%
            \chardef\HOLOGO@choice=3 %
          }%
        }{%
          \chardef\HOLOGO@choice=2 %
        }%
      }{%
        \chardef\HOLOGO@choice=1 %
      }%
      \ifcase\HOLOGO@choice
        \HOLOGO@WarningUnsupportedDriver{Xe}%
        e%
      \or % 1: reflectbox
        \lower.5ex\hbox{%
          \reflectbox{E}%
        }%
      \or % 2: \pgftext
        \lower.5ex\hbox{%
          \pgfpicture
            \pgftransformxscale{-1}%
            \pgftext{E}%
          \endpgfpicture
        }%
      \or % 3: \psscalebox
        \lower.5ex\hbox{%
          \psscalebox{-1 1}{E}%
        }%
      \or % 4: \HOLOGO@Reflectbox
        \lower.5ex\hbox{%
          \HOLOGO@ReflectBox{E}%
        }%
      \else
        \@PackageError{hologo}{Internal error (choice/up)}\@ehc
      \fi
    \fi
  }%
}
%    \end{macrocode}
%    \end{macro}
%    \begin{macro}{\HoLogoHtml@Xe}
%    \begin{macrocode}
\def\HoLogoHtml@Xe#1{%
  \HoLogoCss@Xe
  \HOLOGO@Span{Xe}{%
    X%
    \HOLOGO@Span{e}{%
      \HCode{&\ltx@hashchar x018e;}%
    }%
  }%
}
%    \end{macrocode}
%    \end{macro}
%    \begin{macro}{\HoLogoCss@Xe}
%    \begin{macrocode}
\def\HoLogoCss@Xe{%
  \Css{%
    span.HoLogo-Xe span.HoLogo-e{%
      position:relative;%
      top:.5ex;%
      left-margin:-.1em;%
    }%
  }%
  \global\let\HoLogoCss@Xe\relax
}
%    \end{macrocode}
%    \end{macro}
%
%    \begin{macro}{\HoLogo@XeTeX}
%    \begin{macrocode}
\def\HoLogo@XeTeX#1{%
  \hologo{Xe}%
  \kern-.15em\relax
  \hologo{TeX}%
}
%    \end{macrocode}
%    \end{macro}
%
%    \begin{macro}{\HoLogoHtml@XeTeX}
%    \begin{macrocode}
\def\HoLogoHtml@XeTeX#1{%
  \HoLogoCss@XeTeX
  \HOLOGO@Span{XeTeX}{%
    \hologo{Xe}%
    \hologo{TeX}%
  }%
}
%    \end{macrocode}
%    \end{macro}
%    \begin{macro}{\HoLogoCss@XeTeX}
%    \begin{macrocode}
\def\HoLogoCss@XeTeX{%
  \Css{%
    span.HoLogo-XeTeX span.HoLogo-TeX{%
      margin-left:-.15em;%
    }%
  }%
  \global\let\HoLogoCss@XeTeX\relax
}
%    \end{macrocode}
%    \end{macro}
%
%    \begin{macro}{\HoLogo@XeLaTeX}
%    \begin{macrocode}
\def\HoLogo@XeLaTeX#1{%
  \hologo{Xe}%
  \kern-.13em%
  \hologo{LaTeX}%
}
%    \end{macrocode}
%    \end{macro}
%    \begin{macro}{\HoLogoHtml@XeLaTeX}
%    \begin{macrocode}
\def\HoLogoHtml@XeLaTeX#1{%
  \HoLogoCss@XeLaTeX
  \HOLOGO@Span{XeLaTeX}{%
    \hologo{Xe}%
    \hologo{LaTeX}%
  }%
}
%    \end{macrocode}
%    \end{macro}
%    \begin{macro}{\HoLogoCss@XeLaTeX}
%    \begin{macrocode}
\def\HoLogoCss@XeLaTeX{%
  \Css{%
    span.HoLogo-XeLaTeX span.HoLogo-Xe{%
      margin-right:-.13em;%
    }%
  }%
  \global\let\HoLogoCss@XeLaTeX\relax
}
%    \end{macrocode}
%    \end{macro}
%
% \subsubsection{\hologo{pdfTeX}, \hologo{pdfLaTeX}}
%
%    \begin{macro}{\HoLogo@pdfTeX}
%    \begin{macrocode}
\def\HoLogo@pdfTeX#1{%
  \HOLOGO@mbox{%
    #1{p}{P}df\hologo{TeX}%
  }%
}
%    \end{macrocode}
%    \end{macro}
%    \begin{macro}{\HoLogoCs@pdfTeX}
%    \begin{macrocode}
\def\HoLogoCs@pdfTeX#1{#1{p}{P}dfTeX}
%    \end{macrocode}
%    \end{macro}
%    \begin{macro}{\HoLogoBkm@pdfTeX}
%    \begin{macrocode}
\def\HoLogoBkm@pdfTeX#1{%
  #1{p}{P}df\hologo{TeX}%
}
%    \end{macrocode}
%    \end{macro}
%    \begin{macro}{\HoLogoHtml@pdfTeX}
%    \begin{macrocode}
\let\HoLogoHtml@pdfTeX\HoLogo@pdfTeX
%    \end{macrocode}
%    \end{macro}
%
%    \begin{macro}{\HoLogo@pdfLaTeX}
%    \begin{macrocode}
\def\HoLogo@pdfLaTeX#1{%
  \HOLOGO@mbox{%
    #1{p}{P}df\hologo{LaTeX}%
  }%
}
%    \end{macrocode}
%    \end{macro}
%    \begin{macro}{\HoLogoCs@pdfLaTeX}
%    \begin{macrocode}
\def\HoLogoCs@pdfLaTeX#1{#1{p}{P}dfLaTeX}
%    \end{macrocode}
%    \end{macro}
%    \begin{macro}{\HoLogoBkm@pdfLaTeX}
%    \begin{macrocode}
\def\HoLogoBkm@pdfLaTeX#1{%
  #1{p}{P}df\hologo{LaTeX}%
}
%    \end{macrocode}
%    \end{macro}
%    \begin{macro}{\HoLogoHtml@pdfLaTeX}
%    \begin{macrocode}
\let\HoLogoHtml@pdfLaTeX\HoLogo@pdfLaTeX
%    \end{macrocode}
%    \end{macro}
%
% \subsubsection{\hologo{VTeX}}
%
%    \begin{macro}{\HoLogo@VTeX}
%    \begin{macrocode}
\def\HoLogo@VTeX#1{%
  \HOLOGO@mbox{%
    V\hologo{TeX}%
  }%
}
%    \end{macrocode}
%    \end{macro}
%    \begin{macro}{\HoLogoHtml@VTeX}
%    \begin{macrocode}
\let\HoLogoHtml@VTeX\HoLogo@VTeX
%    \end{macrocode}
%    \end{macro}
%
% \subsubsection{\hologo{AmS}, \dots}
%
%    Source: class \xclass{amsdtx}
%
%    \begin{macro}{\HoLogo@AmS}
%    \begin{macrocode}
\def\HoLogo@AmS#1{%
  \HoLogoFont@font{AmS}{sy}{%
    A%
    \kern-.1667em%
    \lower.5ex\hbox{M}%
    \kern-.125em%
    S%
  }%
}
%    \end{macrocode}
%    \end{macro}
%    \begin{macro}{\HoLogoBkm@AmS}
%    \begin{macrocode}
\def\HoLogoBkm@AmS#1{AmS}
%    \end{macrocode}
%    \end{macro}
%    \begin{macro}{\HoLogoHtml@AmS}
%    \begin{macrocode}
\def\HoLogoHtml@AmS#1{%
  \HoLogoCss@AmS
%  \HoLogoFont@font{AmS}{sy}{%
    \HOLOGO@Span{AmS}{%
      A%
      \HOLOGO@Span{M}{M}%
      S%
    }%
%   }%
}
%    \end{macrocode}
%    \end{macro}
%    \begin{macro}{\HoLogoCss@AmS}
%    \begin{macrocode}
\def\HoLogoCss@AmS{%
  \Css{%
    span.HoLogo-AmS span.HoLogo-M{%
      position:relative;%
      top:.5ex;%
      margin-left:-.1667em;%
      margin-right:-.125em;%
      text-decoration:none;%
    }%
  }%
  \global\let\HoLogoCss@AmS\relax
}
%    \end{macrocode}
%    \end{macro}
%
%    \begin{macro}{\HoLogo@AmSTeX}
%    \begin{macrocode}
\def\HoLogo@AmSTeX#1{%
  \hologo{AmS}%
  \HOLOGO@hyphen
  \hologo{TeX}%
}
%    \end{macrocode}
%    \end{macro}
%    \begin{macro}{\HoLogoBkm@AmSTeX}
%    \begin{macrocode}
\def\HoLogoBkm@AmSTeX#1{AmS-TeX}%
%    \end{macrocode}
%    \end{macro}
%    \begin{macro}{\HoLogoHtml@AmSTeX}
%    \begin{macrocode}
\let\HoLogoHtml@AmSTeX\HoLogo@AmSTeX
%    \end{macrocode}
%    \end{macro}
%
%    \begin{macro}{\HoLogo@AmSLaTeX}
%    \begin{macrocode}
\def\HoLogo@AmSLaTeX#1{%
  \hologo{AmS}%
  \HOLOGO@hyphen
  \hologo{LaTeX}%
}
%    \end{macrocode}
%    \end{macro}
%    \begin{macro}{\HoLogoBkm@AmSLaTeX}
%    \begin{macrocode}
\def\HoLogoBkm@AmSLaTeX#1{AmS-LaTeX}%
%    \end{macrocode}
%    \end{macro}
%    \begin{macro}{\HoLogoHtml@AmSLaTeX}
%    \begin{macrocode}
\let\HoLogoHtml@AmSLaTeX\HoLogo@AmSLaTeX
%    \end{macrocode}
%    \end{macro}
%
% \subsubsection{\hologo{BibTeX}}
%
%    \begin{macro}{\HoLogo@BibTeX@sc}
%    A definition of \hologo{BibTeX} is provided in
%    the documentation source for the manual of \hologo{BibTeX}
%    \cite{btxdoc}.
%\begin{quote}
%\begin{verbatim}
%\def\BibTeX{%
%  {%
%    \rm
%    B%
%    \kern-.05em%
%    {%
%      \sc
%      i%
%      \kern-.025em %
%      b%
%    }%
%    \kern-.08em
%    T%
%    \kern-.1667em%
%    \lower.7ex\hbox{E}%
%    \kern-.125em%
%    X%
%  }%
%}
%\end{verbatim}
%\end{quote}
%    \begin{macrocode}
\def\HoLogo@BibTeX@sc#1{%
  B%
  \kern-.05em%
  \HoLogoFont@font{BibTeX}{sc}{%
    i%
    \kern-.025em%
    b%
  }%
  \HOLOGO@discretionary
  \kern-.08em%
  \hologo{TeX}%
}
%    \end{macrocode}
%    \end{macro}
%    \begin{macro}{\HoLogoHtml@BibTeX@sc}
%    \begin{macrocode}
\def\HoLogoHtml@BibTeX@sc#1{%
  \HoLogoCss@BibTeX@sc
  \HOLOGO@Span{BibTeX-sc}{%
    B%
    \HOLOGO@Span{i}{i}%
    \HOLOGO@Span{b}{b}%
    \hologo{TeX}%
  }%
}
%    \end{macrocode}
%    \end{macro}
%    \begin{macro}{\HoLogoCss@BibTeX@sc}
%    \begin{macrocode}
\def\HoLogoCss@BibTeX@sc{%
  \Css{%
    span.HoLogo-BibTeX-sc span.HoLogo-i{%
      margin-left:-.05em;%
      margin-right:-.025em;%
      font-variant:small-caps;%
    }%
  }%
  \Css{%
    span.HoLogo-BibTeX-sc span.HoLogo-b{%
      margin-right:-.08em;%
      font-variant:small-caps;%
    }%
  }%
  \global\let\HoLogoCss@BibTeX@sc\relax
}
%    \end{macrocode}
%    \end{macro}
%
%    \begin{macro}{\HoLogo@BibTeX@sf}
%    Variant \xoption{sf} avoids trouble with unavailable
%    small caps fonts (e.g., bold versions of Computer Modern or
%    Latin Modern). The definition is taken from
%    package \xpackage{dtklogos} \cite{dtklogos}.
%\begin{quote}
%\begin{verbatim}
%\DeclareRobustCommand{\BibTeX}{%
%  B%
%  \kern-.05em%
%  \hbox{%
%    $\m@th$% %% force math size calculations
%    \csname S@\f@size\endcsname
%    \fontsize\sf@size\z@
%    \math@fontsfalse
%    \selectfont
%    I%
%    \kern-.025em%
%    B
%  }%
%  \kern-.08em%
%  \-%
%  \TeX
%}
%\end{verbatim}
%\end{quote}
%    \begin{macrocode}
\def\HoLogo@BibTeX@sf#1{%
  B%
  \kern-.05em%
  \HoLogoFont@font{BibTeX}{bibsf}{%
    I%
    \kern-.025em%
    B%
  }%
  \HOLOGO@discretionary
  \kern-.08em%
  \hologo{TeX}%
}
%    \end{macrocode}
%    \end{macro}
%    \begin{macro}{\HoLogoHtml@BibTeX@sf}
%    \begin{macrocode}
\def\HoLogoHtml@BibTeX@sf#1{%
  \HoLogoCss@BibTeX@sf
  \HOLOGO@Span{BibTeX-sf}{%
    B%
    \HoLogoFont@font{BibTeX}{bibsf}{%
      \HOLOGO@Span{i}{I}%
      B%
    }%
    \hologo{TeX}%
  }%
}
%    \end{macrocode}
%    \end{macro}
%    \begin{macro}{\HoLogoCss@BibTeX@sf}
%    \begin{macrocode}
\def\HoLogoCss@BibTeX@sf{%
  \Css{%
    span.HoLogo-BibTeX-sf span.HoLogo-i{%
      margin-left:-.05em;%
      margin-right:-.025em;%
    }%
  }%
  \Css{%
    span.HoLogo-BibTeX-sf span.HoLogo-TeX{%
      margin-left:-.08em;%
    }%
  }%
  \global\let\HoLogoCss@BibTeX@sf\relax
}
%    \end{macrocode}
%    \end{macro}
%
%    \begin{macro}{\HoLogo@BibTeX}
%    \begin{macrocode}
\def\HoLogo@BibTeX{\HoLogo@BibTeX@sf}
%    \end{macrocode}
%    \end{macro}
%    \begin{macro}{\HoLogoHtml@BibTeX}
%    \begin{macrocode}
\def\HoLogoHtml@BibTeX{\HoLogoHtml@BibTeX@sf}
%    \end{macrocode}
%    \end{macro}
%
% \subsubsection{\hologo{BibTeX8}}
%
%    \begin{macro}{\HoLogo@BibTeX8}
%    \begin{macrocode}
\expandafter\def\csname HoLogo@BibTeX8\endcsname#1{%
  \hologo{BibTeX}%
  8%
}
%    \end{macrocode}
%    \end{macro}
%
%    \begin{macro}{\HoLogoBkm@BibTeX8}
%    \begin{macrocode}
\expandafter\def\csname HoLogoBkm@BibTeX8\endcsname#1{%
  \hologo{BibTeX}%
  8%
}
%    \end{macrocode}
%    \end{macro}
%    \begin{macro}{\HoLogoHtml@BibTeX8}
%    \begin{macrocode}
\expandafter
\let\csname HoLogoHtml@BibTeX8\expandafter\endcsname
\csname HoLogo@BibTeX8\endcsname
%    \end{macrocode}
%    \end{macro}
%
% \subsubsection{\hologo{ConTeXt}}
%
%    \begin{macro}{\HoLogo@ConTeXt@simple}
%    \begin{macrocode}
\def\HoLogo@ConTeXt@simple#1{%
  \HOLOGO@mbox{Con}%
  \HOLOGO@discretionary
  \HOLOGO@mbox{\hologo{TeX}t}%
}
%    \end{macrocode}
%    \end{macro}
%    \begin{macro}{\HoLogoHtml@ConTeXt@simple}
%    \begin{macrocode}
\let\HoLogoHtml@ConTeXt@simple\HoLogo@ConTeXt@simple
%    \end{macrocode}
%    \end{macro}
%
%    \begin{macro}{\HoLogo@ConTeXt@narrow}
%    This definition of logo \hologo{ConTeXt} with variant \xoption{narrow}
%    comes from TUGboat's class \xclass{ltugboat} (version 2010/11/15 v2.8).
%    \begin{macrocode}
\def\HoLogo@ConTeXt@narrow#1{%
  \HOLOGO@mbox{C\kern-.0333emon}%
  \HOLOGO@discretionary
  \kern-.0667em%
  \HOLOGO@mbox{\hologo{TeX}\kern-.0333emt}%
}
%    \end{macrocode}
%    \end{macro}
%    \begin{macro}{\HoLogoHtml@ConTeXt@narrow}
%    \begin{macrocode}
\def\HoLogoHtml@ConTeXt@narrow#1{%
  \HoLogoCss@ConTeXt@narrow
  \HOLOGO@Span{ConTeXt-narrow}{%
    \HOLOGO@Span{C}{C}%
    on%
    \hologo{TeX}%
    t%
  }%
}
%    \end{macrocode}
%    \end{macro}
%    \begin{macro}{\HoLogoCss@ConTeXt@narrow}
%    \begin{macrocode}
\def\HoLogoCss@ConTeXt@narrow{%
  \Css{%
    span.HoLogo-ConTeXt-narrow span.HoLogo-C{%
      margin-left:-.0333em;%
    }%
  }%
  \Css{%
    span.HoLogo-ConTeXt-narrow span.HoLogo-TeX{%
      margin-left:-.0667em;%
      margin-right:-.0333em;%
    }%
  }%
  \global\let\HoLogoCss@ConTeXt@narrow\relax
}
%    \end{macrocode}
%    \end{macro}
%
%    \begin{macro}{\HoLogo@ConTeXt}
%    \begin{macrocode}
\def\HoLogo@ConTeXt{\HoLogo@ConTeXt@narrow}
%    \end{macrocode}
%    \end{macro}
%    \begin{macro}{\HoLogoHtml@ConTeXt}
%    \begin{macrocode}
\def\HoLogoHtml@ConTeXt{\HoLogoHtml@ConTeXt@narrow}
%    \end{macrocode}
%    \end{macro}
%
% \subsubsection{\hologo{emTeX}}
%
%    \begin{macro}{\HoLogo@emTeX}
%    \begin{macrocode}
\def\HoLogo@emTeX#1{%
  \HOLOGO@mbox{#1{e}{E}m}%
  \HOLOGO@discretionary
  \hologo{TeX}%
}
%    \end{macrocode}
%    \end{macro}
%    \begin{macro}{\HoLogoCs@emTeX}
%    \begin{macrocode}
\def\HoLogoCs@emTeX#1{#1{e}{E}mTeX}%
%    \end{macrocode}
%    \end{macro}
%    \begin{macro}{\HoLogoBkm@emTeX}
%    \begin{macrocode}
\def\HoLogoBkm@emTeX#1{%
  #1{e}{E}m\hologo{TeX}%
}
%    \end{macrocode}
%    \end{macro}
%    \begin{macro}{\HoLogoHtml@emTeX}
%    \begin{macrocode}
\let\HoLogoHtml@emTeX\HoLogo@emTeX
%    \end{macrocode}
%    \end{macro}
%
% \subsubsection{\hologo{ExTeX}}
%
%    \begin{macro}{\HoLogo@ExTeX}
%    The definition is taken from the FAQ of the
%    project \hologo{ExTeX}
%    \cite{ExTeX-FAQ}.
%\begin{quote}
%\begin{verbatim}
%\def\ExTeX{%
%  \textrm{% Logo always with serifs
%    \ensuremath{%
%      \textstyle
%      \varepsilon_{%
%        \kern-0.15em%
%        \mathcal{X}%
%      }%
%    }%
%    \kern-.15em%
%    \TeX
%  }%
%}
%\end{verbatim}
%\end{quote}
%    \begin{macrocode}
\def\HoLogo@ExTeX#1{%
  \HoLogoFont@font{ExTeX}{rm}{%
    \ltx@mbox{%
      \HOLOGO@MathSetup
      $%
        \textstyle
        \varepsilon_{%
          \kern-0.15em%
          \HoLogoFont@font{ExTeX}{sy}{X}%
        }%
      $%
    }%
    \HOLOGO@discretionary
    \kern-.15em%
    \hologo{TeX}%
  }%
}
%    \end{macrocode}
%    \end{macro}
%    \begin{macro}{\HoLogoHtml@ExTeX}
%    \begin{macrocode}
\def\HoLogoHtml@ExTeX#1{%
  \HoLogoCss@ExTeX
  \HoLogoFont@font{ExTeX}{rm}{%
    \HOLOGO@Span{ExTeX}{%
      \ltx@mbox{%
        \HOLOGO@MathSetup
        $\textstyle\varepsilon$%
        \HOLOGO@Span{X}{$\textstyle\chi$}%
        \hologo{TeX}%
      }%
    }%
  }%
}
%    \end{macrocode}
%    \end{macro}
%    \begin{macro}{\HoLogoBkm@ExTeX}
%    \begin{macrocode}
\def\HoLogoBkm@ExTeX#1{%
  \HOLOGO@PdfdocUnicode{#1{e}{E}x}{\textepsilon\textchi}%
  \hologo{TeX}%
}
%    \end{macrocode}
%    \end{macro}
%    \begin{macro}{\HoLogoCss@ExTeX}
%    \begin{macrocode}
\def\HoLogoCss@ExTeX{%
  \Css{%
    span.HoLogo-ExTeX{%
      font-family:serif;%
    }%
  }%
  \Css{%
    span.HoLogo-ExTeX span.HoLogo-TeX{%
      margin-left:-.15em;%
    }%
  }%
  \global\let\HoLogoCss@ExTeX\relax
}
%    \end{macrocode}
%    \end{macro}
%
% \subsubsection{\hologo{MiKTeX}}
%
%    \begin{macro}{\HoLogo@MiKTeX}
%    \begin{macrocode}
\def\HoLogo@MiKTeX#1{%
  \HOLOGO@mbox{MiK}%
  \HOLOGO@discretionary
  \hologo{TeX}%
}
%    \end{macrocode}
%    \end{macro}
%    \begin{macro}{\HoLogoHtml@MiKTeX}
%    \begin{macrocode}
\let\HoLogoHtml@MiKTeX\HoLogo@MiKTeX
%    \end{macrocode}
%    \end{macro}
%
% \subsubsection{\hologo{OzTeX} and friends}
%
%    Source: \hologo{OzTeX} FAQ \cite{OzTeX}:
%    \begin{quote}
%      |\def\OzTeX{O\kern-.03em z\kern-.15em\TeX}|\\
%      (There is no kerning in OzMF, OzMP and OzTtH.)
%    \end{quote}
%
%    \begin{macro}{\HoLogo@OzTeX}
%    \begin{macrocode}
\def\HoLogo@OzTeX#1{%
  O%
  \kern-.03em %
  z%
  \kern-.15em %
  \hologo{TeX}%
}
%    \end{macrocode}
%    \end{macro}
%    \begin{macro}{\HoLogoHtml@OzTeX}
%    \begin{macrocode}
\def\HoLogoHtml@OzTeX#1{%
  \HoLogoCss@OzTeX
  \HOLOGO@Span{OzTeX}{%
    O%
    \HOLOGO@Span{z}{z}%
    \hologo{TeX}%
  }%
}
%    \end{macrocode}
%    \end{macro}
%    \begin{macro}{\HoLogoCss@OzTeX}
%    \begin{macrocode}
\def\HoLogoCss@OzTeX{%
  \Css{%
    span.HoLogo-OzTeX span.HoLogo-z{%
      margin-left:-.03em;%
      margin-right:-.15em;%
    }%
  }%
  \global\let\HoLogoCss@OzTeX\relax
}
%    \end{macrocode}
%    \end{macro}
%
%    \begin{macro}{\HoLogo@OzMF}
%    \begin{macrocode}
\def\HoLogo@OzMF#1{%
  \HOLOGO@mbox{OzMF}%
}
%    \end{macrocode}
%    \end{macro}
%    \begin{macro}{\HoLogo@OzMP}
%    \begin{macrocode}
\def\HoLogo@OzMP#1{%
  \HOLOGO@mbox{OzMP}%
}
%    \end{macrocode}
%    \end{macro}
%    \begin{macro}{\HoLogo@OzTtH}
%    \begin{macrocode}
\def\HoLogo@OzTtH#1{%
  \HOLOGO@mbox{OzTtH}%
}
%    \end{macrocode}
%    \end{macro}
%
% \subsubsection{\hologo{PCTeX}}
%
%    \begin{macro}{\HoLogo@PCTeX}
%    \begin{macrocode}
\def\HoLogo@PCTeX#1{%
  \HOLOGO@mbox{PC}%
  \hologo{TeX}%
}
%    \end{macrocode}
%    \end{macro}
%    \begin{macro}{\HoLogoHtml@PCTeX}
%    \begin{macrocode}
\let\HoLogoHtml@PCTeX\HoLogo@PCTeX
%    \end{macrocode}
%    \end{macro}
%
% \subsubsection{\hologo{PiCTeX}}
%
%    The original definitions from \xfile{pictex.tex} \cite{PiCTeX}:
%\begin{quote}
%\begin{verbatim}
%\def\PiC{%
%  P%
%  \kern-.12em%
%  \lower.5ex\hbox{I}%
%  \kern-.075em%
%  C%
%}
%\def\PiCTeX{%
%  \PiC
%  \kern-.11em%
%  \TeX
%}
%\end{verbatim}
%\end{quote}
%
%    \begin{macro}{\HoLogo@PiC}
%    \begin{macrocode}
\def\HoLogo@PiC#1{%
  P%
  \kern-.12em%
  \lower.5ex\hbox{I}%
  \kern-.075em%
  C%
  \HOLOGO@SpaceFactor
}
%    \end{macrocode}
%    \end{macro}
%    \begin{macro}{\HoLogoHtml@PiC}
%    \begin{macrocode}
\def\HoLogoHtml@PiC#1{%
  \HoLogoCss@PiC
  \HOLOGO@Span{PiC}{%
    P%
    \HOLOGO@Span{i}{I}%
    C%
  }%
}
%    \end{macrocode}
%    \end{macro}
%    \begin{macro}{\HoLogoCss@PiC}
%    \begin{macrocode}
\def\HoLogoCss@PiC{%
  \Css{%
    span.HoLogo-PiC span.HoLogo-i{%
      position:relative;%
      top:.5ex;%
      margin-left:-.12em;%
      margin-right:-.075em;%
      text-decoration:none;%
    }%
  }%
  \global\let\HoLogoCss@PiC\relax
}
%    \end{macrocode}
%    \end{macro}
%
%    \begin{macro}{\HoLogo@PiCTeX}
%    \begin{macrocode}
\def\HoLogo@PiCTeX#1{%
  \hologo{PiC}%
  \HOLOGO@discretionary
  \kern-.11em%
  \hologo{TeX}%
}
%    \end{macrocode}
%    \end{macro}
%    \begin{macro}{\HoLogoHtml@PiCTeX}
%    \begin{macrocode}
\def\HoLogoHtml@PiCTeX#1{%
  \HoLogoCss@PiCTeX
  \HOLOGO@Span{PiCTeX}{%
    \hologo{PiC}%
    \hologo{TeX}%
  }%
}
%    \end{macrocode}
%    \end{macro}
%    \begin{macro}{\HoLogoCss@PiCTeX}
%    \begin{macrocode}
\def\HoLogoCss@PiCTeX{%
  \Css{%
    span.HoLogo-PiCTeX span.HoLogo-PiC{%
      margin-right:-.11em;%
    }%
  }%
  \global\let\HoLogoCss@PiCTeX\relax
}
%    \end{macrocode}
%    \end{macro}
%
% \subsubsection{\hologo{teTeX}}
%
%    \begin{macro}{\HoLogo@teTeX}
%    \begin{macrocode}
\def\HoLogo@teTeX#1{%
  \HOLOGO@mbox{#1{t}{T}e}%
  \HOLOGO@discretionary
  \hologo{TeX}%
}
%    \end{macrocode}
%    \end{macro}
%    \begin{macro}{\HoLogoCs@teTeX}
%    \begin{macrocode}
\def\HoLogoCs@teTeX#1{#1{t}{T}dfTeX}
%    \end{macrocode}
%    \end{macro}
%    \begin{macro}{\HoLogoBkm@teTeX}
%    \begin{macrocode}
\def\HoLogoBkm@teTeX#1{%
  #1{t}{T}e\hologo{TeX}%
}
%    \end{macrocode}
%    \end{macro}
%    \begin{macro}{\HoLogoHtml@teTeX}
%    \begin{macrocode}
\let\HoLogoHtml@teTeX\HoLogo@teTeX
%    \end{macrocode}
%    \end{macro}
%
% \subsubsection{\hologo{TeX4ht}}
%
%    \begin{macro}{\HoLogo@TeX4ht}
%    \begin{macrocode}
\expandafter\def\csname HoLogo@TeX4ht\endcsname#1{%
  \HOLOGO@mbox{\hologo{TeX}4ht}%
}
%    \end{macrocode}
%    \end{macro}
%    \begin{macro}{\HoLogoHtml@TeX4ht}
%    \begin{macrocode}
\expandafter
\let\csname HoLogoHtml@TeX4ht\expandafter\endcsname
\csname HoLogo@TeX4ht\endcsname
%    \end{macrocode}
%    \end{macro}
%
%
% \subsubsection{\hologo{SageTeX}}
%
%    \begin{macro}{\HoLogo@SageTeX}
%    \begin{macrocode}
\def\HoLogo@SageTeX#1{%
  \HOLOGO@mbox{Sage}%
  \HOLOGO@discretionary
  \HOLOGO@NegativeKerning{eT,oT,To}%
  \hologo{TeX}%
}
%    \end{macrocode}
%    \end{macro}
%    \begin{macro}{\HoLogoHtml@SageTeX}
%    \begin{macrocode}
\let\HoLogoHtml@SageTeX\HoLogo@SageTeX
%    \end{macrocode}
%    \end{macro}
%
% \subsection{\hologo{METAFONT} and friends}
%
%    \begin{macro}{\HoLogo@METAFONT}
%    \begin{macrocode}
\def\HoLogo@METAFONT#1{%
  \HoLogoFont@font{METAFONT}{logo}{%
    \HOLOGO@mbox{META}%
    \HOLOGO@discretionary
    \HOLOGO@mbox{FONT}%
  }%
}
%    \end{macrocode}
%    \end{macro}
%
%    \begin{macro}{\HoLogo@METAPOST}
%    \begin{macrocode}
\def\HoLogo@METAPOST#1{%
  \HoLogoFont@font{METAPOST}{logo}{%
    \HOLOGO@mbox{META}%
    \HOLOGO@discretionary
    \HOLOGO@mbox{POST}%
  }%
}
%    \end{macrocode}
%    \end{macro}
%
%    \begin{macro}{\HoLogo@MetaFun}
%    \begin{macrocode}
\def\HoLogo@MetaFun#1{%
  \HOLOGO@mbox{Meta}%
  \HOLOGO@discretionary
  \HOLOGO@mbox{Fun}%
}
%    \end{macrocode}
%    \end{macro}
%
%    \begin{macro}{\HoLogo@MetaPost}
%    \begin{macrocode}
\def\HoLogo@MetaPost#1{%
  \HOLOGO@mbox{Meta}%
  \HOLOGO@discretionary
  \HOLOGO@mbox{Post}%
}
%    \end{macrocode}
%    \end{macro}
%
% \subsection{Others}
%
% \subsubsection{\hologo{biber}}
%
%    \begin{macro}{\HoLogo@biber}
%    \begin{macrocode}
\def\HoLogo@biber#1{%
  \HOLOGO@mbox{#1{b}{B}i}%
  \HOLOGO@discretionary
  \HOLOGO@mbox{ber}%
}
%    \end{macrocode}
%    \end{macro}
%    \begin{macro}{\HoLogoCs@biber}
%    \begin{macrocode}
\def\HoLogoCs@biber#1{#1{b}{B}iber}
%    \end{macrocode}
%    \end{macro}
%    \begin{macro}{\HoLogoBkm@biber}
%    \begin{macrocode}
\def\HoLogoBkm@biber#1{%
  #1{b}{B}iber%
}
%    \end{macrocode}
%    \end{macro}
%    \begin{macro}{\HoLogoHtml@biber}
%    \begin{macrocode}
\let\HoLogoHtml@biber\HoLogo@biber
%    \end{macrocode}
%    \end{macro}
%
% \subsubsection{\hologo{KOMAScript}}
%
%    \begin{macro}{\HoLogo@KOMAScript}
%    The definition for \hologo{KOMAScript} is taken
%    from \hologo{KOMAScript} (\xfile{scrlogo.dtx}, reformatted) \cite{scrlogo}:
%\begin{quote}
%\begin{verbatim}
%\@ifundefined{KOMAScript}{%
%  \DeclareRobustCommand{\KOMAScript}{%
%    \textsf{%
%      K\kern.05em O\kern.05emM\kern.05em A%
%      \kern.1em-\kern.1em %
%      Script%
%    }%
%  }%
%}{}
%\end{verbatim}
%\end{quote}
%    \begin{macrocode}
\def\HoLogo@KOMAScript#1{%
  \HoLogoFont@font{KOMAScript}{sf}{%
    \HOLOGO@mbox{%
      K\kern.05em%
      O\kern.05em%
      M\kern.05em%
      A%
    }%
    \kern.1em%
    \HOLOGO@hyphen
    \kern.1em%
    \HOLOGO@mbox{Script}%
  }%
}
%    \end{macrocode}
%    \end{macro}
%    \begin{macro}{\HoLogoBkm@KOMAScript}
%    \begin{macrocode}
\def\HoLogoBkm@KOMAScript#1{%
  KOMA-Script%
}
%    \end{macrocode}
%    \end{macro}
%    \begin{macro}{\HoLogoHtml@KOMAScript}
%    \begin{macrocode}
\def\HoLogoHtml@KOMAScript#1{%
  \HoLogoCss@KOMAScript
  \HoLogoFont@font{KOMAScript}{sf}{%
    \HOLOGO@Span{KOMAScript}{%
      K%
      \HOLOGO@Span{O}{O}%
      M%
      \HOLOGO@Span{A}{A}%
      \HOLOGO@Span{hyphen}{-}%
      Script%
    }%
  }%
}
%    \end{macrocode}
%    \end{macro}
%    \begin{macro}{\HoLogoCss@KOMAScript}
%    \begin{macrocode}
\def\HoLogoCss@KOMAScript{%
  \Css{%
    span.HoLogo-KOMAScript{%
      font-family:sans-serif;%
    }%
  }%
  \Css{%
    span.HoLogo-KOMAScript span.HoLogo-O{%
      padding-left:.05em;%
      padding-right:.05em;%
    }%
  }%
  \Css{%
    span.HoLogo-KOMAScript span.HoLogo-A{%
      padding-left:.05em;%
    }%
  }%
  \Css{%
    span.HoLogo-KOMAScript span.HoLogo-hyphen{%
      padding-left:.1em;%
      padding-right:.1em;%
    }%
  }%
  \global\let\HoLogoCss@KOMAScript\relax
}
%    \end{macrocode}
%    \end{macro}
%
% \subsubsection{\hologo{LyX}}
%
%    \begin{macro}{\HoLogo@LyX}
%    The definition is taken from the documentation source files
%    of \hologo{LyX}, \xfile{Intro.lyx} \cite{LyX}:
%\begin{quote}
%\begin{verbatim}
%\def\LyX{%
%  \texorpdfstring{%
%    L\kern-.1667em\lower.25em\hbox{Y}\kern-.125emX\@%
%  }{%
%    LyX%
%  }%
%}
%\end{verbatim}
%\end{quote}
%    \begin{macrocode}
\def\HoLogo@LyX#1{%
  L%
  \kern-.1667em%
  \lower.25em\hbox{Y}%
  \kern-.125em%
  X%
  \HOLOGO@SpaceFactor
}
%    \end{macrocode}
%    \end{macro}
%    \begin{macro}{\HoLogoHtml@LyX}
%    \begin{macrocode}
\def\HoLogoHtml@LyX#1{%
  \HoLogoCss@LyX
  \HOLOGO@Span{LyX}{%
    L%
    \HOLOGO@Span{y}{Y}%
    X%
  }%
}
%    \end{macrocode}
%    \end{macro}
%    \begin{macro}{\HoLogoCss@LyX}
%    \begin{macrocode}
\def\HoLogoCss@LyX{%
  \Css{%
    span.HoLogo-LyX span.HoLogo-y{%
      position:relative;%
      top:.25em;%
      margin-left:-.1667em;%
      margin-right:-.125em;%
      text-decoration:none;%
    }%
  }%
  \global\let\HoLogoCss@LyX\relax
}
%    \end{macrocode}
%    \end{macro}
%
% \subsubsection{\hologo{NTS}}
%
%    \begin{macro}{\HoLogo@NTS}
%    Definition for \hologo{NTS} can be found in
%    package \xpackage{etex\textunderscore man} for the \hologo{eTeX} manual \cite{etexman}
%    and in package \xpackage{dtklogos} \cite{dtklogos}:
%\begin{quote}
%\begin{verbatim}
%\def\NTS{%
%  \leavevmode
%  \hbox{%
%    $%
%      \cal N%
%      \kern-0.35em%
%      \lower0.5ex\hbox{$\cal T$}%
%      \kern-0.2em%
%      S%
%    $%
%  }%
%}
%\end{verbatim}
%\end{quote}
%    \begin{macrocode}
\def\HoLogo@NTS#1{%
  \HoLogoFont@font{NTS}{sy}{%
    N\/%
    \kern-.35em%
    \lower.5ex\hbox{T\/}%
    \kern-.2em%
    S\/%
  }%
  \HOLOGO@SpaceFactor
}
%    \end{macrocode}
%    \end{macro}
%
% \subsubsection{\Hologo{TTH} (\hologo{TeX} to HTML translator)}
%
%    Source: \url{http://hutchinson.belmont.ma.us/tth/}
%    In the HTML source the second `T' is printed as subscript.
%\begin{quote}
%\begin{verbatim}
%T<sub>T</sub>H
%\end{verbatim}
%\end{quote}
%    \begin{macro}{\HoLogo@TTH}
%    \begin{macrocode}
\def\HoLogo@TTH#1{%
  \ltx@mbox{%
    T\HOLOGO@SubScript{T}H%
  }%
  \HOLOGO@SpaceFactor
}
%    \end{macrocode}
%    \end{macro}
%
%    \begin{macro}{\HoLogoHtml@TTH}
%    \begin{macrocode}
\def\HoLogoHtml@TTH#1{%
  T\HCode{<sub>}T\HCode{</sub>}H%
}
%    \end{macrocode}
%    \end{macro}
%
% \subsubsection{\Hologo{HanTheThanh}}
%
%    Partial source: Package \xpackage{dtklogos}.
%    The double accent is U+1EBF (latin small letter e with circumflex
%    and acute).
%    \begin{macro}{\HoLogo@HanTheThanh}
%    \begin{macrocode}
\def\HoLogo@HanTheThanh#1{%
  \ltx@mbox{H\`an}%
  \HOLOGO@space
  \ltx@mbox{%
    Th%
    \HOLOGO@IfCharExists{"1EBF}{%
      \char"1EBF\relax
    }{%
      \^e\hbox to 0pt{\hss\raise .5ex\hbox{\'{}}}%
    }%
  }%
  \HOLOGO@space
  \ltx@mbox{Th\`anh}%
}
%    \end{macrocode}
%    \end{macro}
%    \begin{macro}{\HoLogoBkm@HanTheThanh}
%    \begin{macrocode}
\def\HoLogoBkm@HanTheThanh#1{%
  H\`an %
  Th\HOLOGO@PdfdocUnicode{\^e}{\9036\277} %
  Th\`anh%
}
%    \end{macrocode}
%    \end{macro}
%    \begin{macro}{\HoLogoHtml@HanTheThanh}
%    \begin{macrocode}
\def\HoLogoHtml@HanTheThanh#1{%
  H\`an %
  Th\HCode{&\ltx@hashchar x1ebf;} %
  Th\`anh%
}
%    \end{macrocode}
%    \end{macro}
%
% \subsection{Driver detection}
%
%    \begin{macrocode}
\HOLOGO@IfExists\InputIfFileExists{%
  \InputIfFileExists{hologo.cfg}{}{}%
}{%
  \ltx@IfUndefined{pdf@filesize}{%
    \def\HOLOGO@InputIfExists{%
      \openin\HOLOGO@temp=hologo.cfg\relax
      \ifeof\HOLOGO@temp
        \closein\HOLOGO@temp
      \else
        \closein\HOLOGO@temp
        \begingroup
          \def\x{LaTeX2e}%
        \expandafter\endgroup
        \ifx\fmtname\x
          \input{hologo.cfg}%
        \else
          \input hologo.cfg\relax
        \fi
      \fi
    }%
    \ltx@IfUndefined{newread}{%
      \chardef\HOLOGO@temp=15 %
      \def\HOLOGO@CheckRead{%
        \ifeof\HOLOGO@temp
          \HOLOGO@InputIfExists
        \else
          \ifcase\HOLOGO@temp
            \@PackageWarningNoLine{hologo}{%
              Configuration file ignored, because\MessageBreak
              a free read register could not be found%
            }%
          \else
            \begingroup
              \count\ltx@cclv=\HOLOGO@temp
              \advance\ltx@cclv by \ltx@minusone
              \edef\x{\endgroup
                \chardef\noexpand\HOLOGO@temp=\the\count\ltx@cclv
                \relax
              }%
            \x
          \fi
        \fi
      }%
    }{%
      \csname newread\endcsname\HOLOGO@temp
      \HOLOGO@InputIfExists
    }%
  }{%
    \edef\HOLOGO@temp{\pdf@filesize{hologo.cfg}}%
    \ifx\HOLOGO@temp\ltx@empty
    \else
      \ifnum\HOLOGO@temp>0 %
        \begingroup
          \def\x{LaTeX2e}%
        \expandafter\endgroup
        \ifx\fmtname\x
          \input{hologo.cfg}%
        \else
          \input hologo.cfg\relax
        \fi
      \else
        \@PackageInfoNoLine{hologo}{%
          Empty configuration file `hologo.cfg' ignored%
        }%
      \fi
    \fi
  }%
}
%    \end{macrocode}
%
%    \begin{macrocode}
\def\HOLOGO@temp#1#2{%
  \kv@define@key{HoLogoDriver}{#1}[]{%
    \begingroup
      \def\HOLOGO@temp{##1}%
      \ltx@onelevel@sanitize\HOLOGO@temp
      \ifx\HOLOGO@temp\ltx@empty
      \else
        \@PackageError{hologo}{%
          Value (\HOLOGO@temp) not permitted for option `#1'%
        }%
        \@ehc
      \fi
    \endgroup
    \def\hologoDriver{#2}%
  }%
}%
\def\HOLOGO@@temp#1#2{%
  \ifx\kv@value\relax
    \HOLOGO@temp{#1}{#1}%
  \else
    \HOLOGO@temp{#1}{#2}%
  \fi
}%
\kv@parse@normalized{%
  pdftex,%
  luatex=pdftex,%
  dvipdfm,%
  dvipdfmx=dvipdfm,%
  dvips,%
  dvipsone=dvips,%
  xdvi=dvips,%
  xetex,%
  vtex,%
}\HOLOGO@@temp
%    \end{macrocode}
%
%    \begin{macrocode}
\kv@define@key{HoLogoDriver}{driverfallback}{%
  \def\HOLOGO@DriverFallback{#1}%
}
%    \end{macrocode}
%
%    \begin{macro}{\HOLOGO@DriverFallback}
%    \begin{macrocode}
\def\HOLOGO@DriverFallback{dvips}
%    \end{macrocode}
%    \end{macro}
%
%    \begin{macro}{\hologoDriverSetup}
%    \begin{macrocode}
\def\hologoDriverSetup{%
  \let\hologoDriver\ltx@undefined
  \HOLOGO@DriverSetup
}
%    \end{macrocode}
%    \end{macro}
%
%    \begin{macro}{\HOLOGO@DriverSetup}
%    \begin{macrocode}
\def\HOLOGO@DriverSetup#1{%
  \kvsetkeys{HoLogoDriver}{#1}%
  \HOLOGO@CheckDriver
  \ltx@ifundefined{hologoDriver}{%
    \begingroup
    \edef\x{\endgroup
      \noexpand\kvsetkeys{HoLogoDriver}{\HOLOGO@DriverFallback}%
    }\x
  }{}%
  \@PackageInfoNoLine{hologo}{Using driver `\hologoDriver'}%
}
%    \end{macrocode}
%    \end{macro}
%
%    \begin{macro}{\HOLOGO@CheckDriver}
%    \begin{macrocode}
\def\HOLOGO@CheckDriver{%
  \ifpdf
    \def\hologoDriver{pdftex}%
    \let\HOLOGO@pdfliteral\pdfliteral
    \ifluatex
      \ifx\pdfextension\@undefined\else
        \protected\def\pdfliteral{\pdfextension literal}%
        \let\HOLOGO@pdfliteral\pdfliteral
      \fi
      \ltx@IfUndefined{HOLOGO@pdfliteral}{%
        \ifnum\luatexversion<36 %
        \else
          \begingroup
            \let\HOLOGO@temp\endgroup
            \ifcase0%
                \directlua{%
                  if tex.enableprimitives then %
                    tex.enableprimitives('HOLOGO@', {'pdfliteral'})%
                  else %
                    tex.print('1')%
                  end%
                }%
                \ifx\HOLOGO@pdfliteral\@undefined 1\fi%
                \relax%
              \endgroup
              \let\HOLOGO@temp\relax
              \global\let\HOLOGO@pdfliteral\HOLOGO@pdfliteral
            \fi%
          \HOLOGO@temp
        \fi
      }{}%
    \fi
    \ltx@IfUndefined{HOLOGO@pdfliteral}{%
      \@PackageWarningNoLine{hologo}{%
        Cannot find \string\pdfliteral
      }%
    }{}%
  \else
    \ifxetex
      \def\hologoDriver{xetex}%
    \else
      \ifvtex
        \def\hologoDriver{vtex}%
      \fi
    \fi
  \fi
}
%    \end{macrocode}
%    \end{macro}
%
%    \begin{macro}{\HOLOGO@WarningUnsupportedDriver}
%    \begin{macrocode}
\def\HOLOGO@WarningUnsupportedDriver#1{%
  \@PackageWarningNoLine{hologo}{%
    Logo `#1' needs driver specific macros,\MessageBreak
    but driver `\hologoDriver' is not supported.\MessageBreak
    Use a different driver or\MessageBreak
    load package `graphics' or `pgf'%
  }%
}
%    \end{macrocode}
%    \end{macro}
%
% \subsubsection{Reflect box macros}
%
%    Skip driver part if not needed.
%    \begin{macrocode}
\ltx@IfUndefined{reflectbox}{}{%
  \ltx@IfUndefined{rotatebox}{}{%
    \HOLOGO@AtEnd
  }%
}
\ltx@IfUndefined{pgftext}{}{%
  \HOLOGO@AtEnd
}
\ltx@IfUndefined{psscalebox}{}{%
  \HOLOGO@AtEnd
}
%    \end{macrocode}
%
%    \begin{macrocode}
\def\HOLOGO@temp{LaTeX2e}
\ifx\fmtname\HOLOGO@temp
  \RequirePackage{kvoptions}[2011/06/30]%
  \ProcessKeyvalOptions{HoLogoDriver}%
\fi
\HOLOGO@DriverSetup{}
%    \end{macrocode}
%
%    \begin{macro}{\HOLOGO@ReflectBox}
%    \begin{macrocode}
\def\HOLOGO@ReflectBox#1{%
  \begingroup
    \setbox\ltx@zero\hbox{\begingroup#1\endgroup}%
    \setbox\ltx@two\hbox{%
      \kern\wd\ltx@zero
      \csname HOLOGO@ScaleBox@\hologoDriver\endcsname{-1}{1}{%
        \hbox to 0pt{\copy\ltx@zero\hss}%
      }%
    }%
    \wd\ltx@two=\wd\ltx@zero
    \box\ltx@two
  \endgroup
}
%    \end{macrocode}
%    \end{macro}
%
%    \begin{macro}{\HOLOGO@PointReflectBox}
%    \begin{macrocode}
\def\HOLOGO@PointReflectBox#1{%
  \begingroup
    \setbox\ltx@zero\hbox{\begingroup#1\endgroup}%
    \setbox\ltx@two\hbox{%
      \kern\wd\ltx@zero
      \raise\ht\ltx@zero\hbox{%
        \csname HOLOGO@ScaleBox@\hologoDriver\endcsname{-1}{-1}{%
          \hbox to 0pt{\copy\ltx@zero\hss}%
        }%
      }%
    }%
    \wd\ltx@two=\wd\ltx@zero
    \box\ltx@two
  \endgroup
}
%    \end{macrocode}
%    \end{macro}
%
%    We must define all variants because of dynamic driver setup.
%    \begin{macrocode}
\def\HOLOGO@temp#1#2{#2}
%    \end{macrocode}
%
%    \begin{macro}{\HOLOGO@ScaleBox@pdftex}
%    \begin{macrocode}
\HOLOGO@temp{pdftex}{%
  \def\HOLOGO@ScaleBox@pdftex#1#2#3{%
    \HOLOGO@pdfliteral{%
      q #1 0 0 #2 0 0 cm%
    }%
    #3%
    \HOLOGO@pdfliteral{%
      Q%
    }%
  }%
}
%    \end{macrocode}
%    \end{macro}
%    \begin{macro}{\HOLOGO@ScaleBox@dvips}
%    \begin{macrocode}
\HOLOGO@temp{dvips}{%
  \def\HOLOGO@ScaleBox@dvips#1#2#3{%
    \special{ps:%
      gsave %
      currentpoint %
      currentpoint translate %
      #1 #2 scale %
      neg exch neg exch translate%
    }%
    #3%
    \special{ps:%
      currentpoint %
      grestore %
      moveto%
    }%
  }%
}
%    \end{macrocode}
%    \end{macro}
%    \begin{macro}{\HOLOGO@ScaleBox@dvipdfm}
%    \begin{macrocode}
\HOLOGO@temp{dvipdfm}{%
  \let\HOLOGO@ScaleBox@dvipdfm\HOLOGO@ScaleBox@dvips
}
%    \end{macrocode}
%    \end{macro}
%    Since \hologo{XeTeX} v0.6.
%    \begin{macro}{\HOLOGO@ScaleBox@xetex}
%    \begin{macrocode}
\HOLOGO@temp{xetex}{%
  \def\HOLOGO@ScaleBox@xetex#1#2#3{%
    \special{x:gsave}%
    \special{x:scale #1 #2}%
    #3%
    \special{x:grestore}%
  }%
}
%    \end{macrocode}
%    \end{macro}
%    \begin{macro}{\HOLOGO@ScaleBox@vtex}
%    \begin{macrocode}
\HOLOGO@temp{vtex}{%
  \def\HOLOGO@ScaleBox@vtex#1#2#3{%
    \special{r(#1,0,0,#2,0,0}%
    #3%
    \special{r)}%
  }%
}
%    \end{macrocode}
%    \end{macro}
%
%    \begin{macrocode}
\HOLOGO@AtEnd%
%</package>
%    \end{macrocode}
%
% \section{Test}
%
% \subsection{Catcode checks for loading}
%
%    \begin{macrocode}
%<*test1>
%    \end{macrocode}
%    \begin{macrocode}
\catcode`\{=1 %
\catcode`\}=2 %
\catcode`\#=6 %
\catcode`\@=11 %
\expandafter\ifx\csname count@\endcsname\relax
  \countdef\count@=255 %
\fi
\expandafter\ifx\csname @gobble\endcsname\relax
  \long\def\@gobble#1{}%
\fi
\expandafter\ifx\csname @firstofone\endcsname\relax
  \long\def\@firstofone#1{#1}%
\fi
\expandafter\ifx\csname loop\endcsname\relax
  \expandafter\@firstofone
\else
  \expandafter\@gobble
\fi
{%
  \def\loop#1\repeat{%
    \def\body{#1}%
    \iterate
  }%
  \def\iterate{%
    \body
      \let\next\iterate
    \else
      \let\next\relax
    \fi
    \next
  }%
  \let\repeat=\fi
}%
\def\RestoreCatcodes{}
\count@=0 %
\loop
  \edef\RestoreCatcodes{%
    \RestoreCatcodes
    \catcode\the\count@=\the\catcode\count@\relax
  }%
\ifnum\count@<255 %
  \advance\count@ 1 %
\repeat

\def\RangeCatcodeInvalid#1#2{%
  \count@=#1\relax
  \loop
    \catcode\count@=15 %
  \ifnum\count@<#2\relax
    \advance\count@ 1 %
  \repeat
}
\def\RangeCatcodeCheck#1#2#3{%
  \count@=#1\relax
  \loop
    \ifnum#3=\catcode\count@
    \else
      \errmessage{%
        Character \the\count@\space
        with wrong catcode \the\catcode\count@\space
        instead of \number#3%
      }%
    \fi
  \ifnum\count@<#2\relax
    \advance\count@ 1 %
  \repeat
}
\def\space{ }
\expandafter\ifx\csname LoadCommand\endcsname\relax
  \def\LoadCommand{\input hologo.sty\relax}%
\fi
\def\Test{%
  \RangeCatcodeInvalid{0}{47}%
  \RangeCatcodeInvalid{58}{64}%
  \RangeCatcodeInvalid{91}{96}%
  \RangeCatcodeInvalid{123}{255}%
  \catcode`\@=12 %
  \catcode`\\=0 %
  \catcode`\%=14 %
  \LoadCommand
  \RangeCatcodeCheck{0}{36}{15}%
  \RangeCatcodeCheck{37}{37}{14}%
  \RangeCatcodeCheck{38}{47}{15}%
  \RangeCatcodeCheck{48}{57}{12}%
  \RangeCatcodeCheck{58}{63}{15}%
  \RangeCatcodeCheck{64}{64}{12}%
  \RangeCatcodeCheck{65}{90}{11}%
  \RangeCatcodeCheck{91}{91}{15}%
  \RangeCatcodeCheck{92}{92}{0}%
  \RangeCatcodeCheck{93}{96}{15}%
  \RangeCatcodeCheck{97}{122}{11}%
  \RangeCatcodeCheck{123}{255}{15}%
  \RestoreCatcodes
}
\Test
\csname @@end\endcsname
\end
%    \end{macrocode}
%    \begin{macrocode}
%</test1>
%    \end{macrocode}
%
% \subsection{Spacefactor}
%
%    The space factor must be 1000 after a logo. If it is greater 1000
%    then the following space is a space after a sentence closing point.
%    If the space factor is smaller 1000 then an immediate following
%    dot is interpreted as abbreviation, not sentence closing point.
%
%    \begin{macrocode}
%<*test-spacefactor>
\NeedsTeXFormat{LaTeX2e}
\documentclass{article}
\usepackage{hologo}[2016/05/12]
\usepackage{kvsetkeys}
\usepackage{qstest}
\IncludeTests{*}
\LogTests{log}{*}{*}
\begin{document}
\begin{qstest}{spacefactor}{spacefactor}
\newcommand*{\Test}[1]{%
  \sbox0{%
    \hologo{#1}%
    \Expect*{1000 (#1)}*{\the\spacefactor\space(#1)}%
  }%
}%
\makeatletter
\def\TestList{}
\def\hologoEntry#1#2#3{%
  \edef\TestList{%
    \ifx\TestList\@empty
    \else
      \TestList,%
    \fi
    #1%
    \ifx\\#2\\%
    \else
      ={variant=#2}%
    \fi
  }%
}
\hologoList
\expandafter\kv@parse@normalized\expandafter{%
  \TestList
}{%
  \begingroup
    \let\@logo=\kv@key
    \ifx\kv@value\relax
    \else
      \expandafter\hologoLogoSetup\expandafter\@logo\expandafter{%
        \kv@value
      }%
    \fi
    \Test\@logo
  \endgroup
  \@gobbletwo
}
\end{qstest}
\end{document}
%</test-spacefactor>
%    \end{macrocode}
%
% \subsection{Complete list}
%
%    \begin{macrocode}
%<*test-list>
\NeedsTeXFormat{LaTeX2e}
\documentclass[12pt,a4paper]{article}
\usepackage{hologo}[2016/05/12]
\usepackage[T1]{fontenc}
\usepackage{lmodern}
\usepackage{parskip}
\usepackage[unicode]{hyperref}[2011/09/28]
\usepackage{bookmark}[2011/09/19]
\bookmarksetup{%
  numbered,%
  open,%
  openlevel=2,%
}
\renewcommand*{\contentsname}{List of logos}
\begin{document}
\tableofcontents
\def\TestFont#1#2#3#4#5#6{%
  \begingroup
    \usefont{#3}{#4}{#5}{#6}%
    \HologoVariant{#1}{#2}/\hologoVariant{#1}{#2}%
    \quad
    \begingroup\scriptsize\hologoVariant{#1}{#2}\endgroup
    \quad
  \endgroup
  (#3/#4/#5/#6)%
  \par
}
\makeatletter
\def\hologoEntry#1#2#3{%
  \section{%
    \HologoVariant{#1}{#2}/\hologoVariant{#1}{#2} %
    {[#1\ifx\\#2\\\else\space(#2)\fi]}% hash-ok
  }% braces around [] because of bug in tex4ht
  \begingroup
    \hypersetup{unicode=false}%
    \bookmark[%
      dest=\@currentHref,%
      rellevel=1,%
      keeplevel,%
    ]{%
      \HologoVariant{#1}{#2}/\hologoVariant{#1}{#2} %
      (PDFDocEncoding)%
    }%
  \endgroup
  \TestFont{#1}{#2}{OT1}{cmr}{m}{n}%
  \TestFont{#1}{#2}{OT1}{cmss}{m}{n}%
  \TestFont{#1}{#2}{OT1}{cmr}{b}{n}%
  \TestFont{#1}{#2}{OT1}{cmr}{m}{it}%
  \TestFont{#1}{#2}{OT1}{cmtt}{m}{n}%
  \TestFont{#1}{#2}{T1}{lmr}{m}{n}%
  \TestFont{#1}{#2}{T1}{lmss}{m}{n}%
  \TestFont{#1}{#2}{T1}{lmr}{b}{n}%
  \TestFont{#1}{#2}{T1}{lmr}{m}{it}%
  \TestFont{#1}{#2}{T1}{lmtt}{m}{n}%
  \TestFont{#1}{#2}{T1}{lmvtt}{m}{n}%
  \TestFont{#1}{#2}{T1}{qtm}{m}{n}%
  \TestFont{#1}{#2}{T1}{qhv}{m}{n}%
  \TestFont{#1}{#2}{T1}{qtm}{b}{n}%
  \TestFont{#1}{#2}{T1}{qtm}{m}{it}%
  \TestFont{#1}{#2}{T1}{qcr}{m}{n}%
  \newpage
}
\makeatother
\hologoList
\end{document}
%</test-list>
%    \end{macrocode}
%
% \section{Installation}
%
% \subsection{Download}
%
% \paragraph{Package.} This package is available on
% CTAN\footnote{\url{ftp://ftp.ctan.org/tex-archive/}}:
% \begin{description}
% \item[\CTAN{macros/latex/contrib/oberdiek/hologo.dtx}] The source file.
% \item[\CTAN{macros/latex/contrib/oberdiek/hologo.pdf}] Documentation.
% \end{description}
%
%
% \paragraph{Bundle.} All the packages of the bundle `oberdiek'
% are also available in a TDS compliant ZIP archive. There
% the packages are already unpacked and the documentation files
% are generated. The files and directories obey the TDS standard.
% \begin{description}
% \item[\CTAN{install/macros/latex/contrib/oberdiek.tds.zip}]
% \end{description}
% \emph{TDS} refers to the standard ``A Directory Structure
% for \TeX\ Files'' (\CTAN{tds/tds.pdf}). Directories
% with \xfile{texmf} in their name are usually organized this way.
%
% \subsection{Bundle installation}
%
% \paragraph{Unpacking.} Unpack the \xfile{oberdiek.tds.zip} in the
% TDS tree (also known as \xfile{texmf} tree) of your choice.
% Example (linux):
% \begin{quote}
%   |unzip oberdiek.tds.zip -d ~/texmf|
% \end{quote}
%
% \paragraph{Script installation.}
% Check the directory \xfile{TDS:scripts/oberdiek/} for
% scripts that need further installation steps.
% Package \xpackage{attachfile2} comes with the Perl script
% \xfile{pdfatfi.pl} that should be installed in such a way
% that it can be called as \texttt{pdfatfi}.
% Example (linux):
% \begin{quote}
%   |chmod +x scripts/oberdiek/pdfatfi.pl|\\
%   |cp scripts/oberdiek/pdfatfi.pl /usr/local/bin/|
% \end{quote}
%
% \subsection{Package installation}
%
% \paragraph{Unpacking.} The \xfile{.dtx} file is a self-extracting
% \docstrip\ archive. The files are extracted by running the
% \xfile{.dtx} through \plainTeX:
% \begin{quote}
%   \verb|tex hologo.dtx|
% \end{quote}
%
% \paragraph{TDS.} Now the different files must be moved into
% the different directories in your installation TDS tree
% (also known as \xfile{texmf} tree):
% \begin{quote}
% \def\t{^^A
% \begin{tabular}{@{}>{\ttfamily}l@{ $\rightarrow$ }>{\ttfamily}l@{}}
%   hologo.sty & tex/generic/oberdiek/hologo.sty\\
%   hologo.pdf & doc/latex/oberdiek/hologo.pdf\\
%   example/hologo-example.tex & doc/latex/oberdiek/example/hologo-example.tex\\
%   test/hologo-test1.tex & doc/latex/oberdiek/test/hologo-test1.tex\\
%   test/hologo-test-spacefactor.tex & doc/latex/oberdiek/test/hologo-test-spacefactor.tex\\
%   test/hologo-test-list.tex & doc/latex/oberdiek/test/hologo-test-list.tex\\
%   hologo.dtx & source/latex/oberdiek/hologo.dtx\\
% \end{tabular}^^A
% }^^A
% \sbox0{\t}^^A
% \ifdim\wd0>\linewidth
%   \begingroup
%     \advance\linewidth by\leftmargin
%     \advance\linewidth by\rightmargin
%   \edef\x{\endgroup
%     \def\noexpand\lw{\the\linewidth}^^A
%   }\x
%   \def\lwbox{^^A
%     \leavevmode
%     \hbox to \linewidth{^^A
%       \kern-\leftmargin\relax
%       \hss
%       \usebox0
%       \hss
%       \kern-\rightmargin\relax
%     }^^A
%   }^^A
%   \ifdim\wd0>\lw
%     \sbox0{\small\t}^^A
%     \ifdim\wd0>\linewidth
%       \ifdim\wd0>\lw
%         \sbox0{\footnotesize\t}^^A
%         \ifdim\wd0>\linewidth
%           \ifdim\wd0>\lw
%             \sbox0{\scriptsize\t}^^A
%             \ifdim\wd0>\linewidth
%               \ifdim\wd0>\lw
%                 \sbox0{\tiny\t}^^A
%                 \ifdim\wd0>\linewidth
%                   \lwbox
%                 \else
%                   \usebox0
%                 \fi
%               \else
%                 \lwbox
%               \fi
%             \else
%               \usebox0
%             \fi
%           \else
%             \lwbox
%           \fi
%         \else
%           \usebox0
%         \fi
%       \else
%         \lwbox
%       \fi
%     \else
%       \usebox0
%     \fi
%   \else
%     \lwbox
%   \fi
% \else
%   \usebox0
% \fi
% \end{quote}
% If you have a \xfile{docstrip.cfg} that configures and enables \docstrip's
% TDS installing feature, then some files can already be in the right
% place, see the documentation of \docstrip.
%
% \subsection{Refresh file name databases}
%
% If your \TeX~distribution
% (\teTeX, \mikTeX, \dots) relies on file name databases, you must refresh
% these. For example, \teTeX\ users run \verb|texhash| or
% \verb|mktexlsr|.
%
% \subsection{Some details for the interested}
%
% \paragraph{Attached source.}
%
% The PDF documentation on CTAN also includes the
% \xfile{.dtx} source file. It can be extracted by
% AcrobatReader 6 or higher. Another option is \textsf{pdftk},
% e.g. unpack the file into the current directory:
% \begin{quote}
%   \verb|pdftk hologo.pdf unpack_files output .|
% \end{quote}
%
% \paragraph{Unpacking with \LaTeX.}
% The \xfile{.dtx} chooses its action depending on the format:
% \begin{description}
% \item[\plainTeX:] Run \docstrip\ and extract the files.
% \item[\LaTeX:] Generate the documentation.
% \end{description}
% If you insist on using \LaTeX\ for \docstrip\ (really,
% \docstrip\ does not need \LaTeX), then inform the autodetect routine
% about your intention:
% \begin{quote}
%   \verb|latex \let\install=y\input{hologo.dtx}|
% \end{quote}
% Do not forget to quote the argument according to the demands
% of your shell.
%
% \paragraph{Generating the documentation.}
% You can use both the \xfile{.dtx} or the \xfile{.drv} to generate
% the documentation. The process can be configured by the
% configuration file \xfile{ltxdoc.cfg}. For instance, put this
% line into this file, if you want to have A4 as paper format:
% \begin{quote}
%   \verb|\PassOptionsToClass{a4paper}{article}|
% \end{quote}
% An example follows how to generate the
% documentation with pdf\LaTeX:
% \begin{quote}
%\begin{verbatim}
%pdflatex hologo.dtx
%makeindex -s gind.ist hologo.idx
%pdflatex hologo.dtx
%makeindex -s gind.ist hologo.idx
%pdflatex hologo.dtx
%\end{verbatim}
% \end{quote}
%
% \section{Catalogue}
%
% The following XML file can be used as source for the
% \href{http://mirror.ctan.org/help/Catalogue/catalogue.html}{\TeX\ Catalogue}.
% The elements \texttt{caption} and \texttt{description} are imported
% from the original XML file from the Catalogue.
% The name of the XML file in the Catalogue is \xfile{hologo.xml}.
%    \begin{macrocode}
%<*catalogue>
<?xml version='1.0' encoding='us-ascii'?>
<!DOCTYPE entry SYSTEM 'catalogue.dtd'>
<entry datestamp='$Date$' modifier='$Author$' id='hologo'>
  <name>hologo</name>
  <caption>A collection of logos with bookmark support.</caption>
  <authorref id='auth:oberdiek'/>
  <copyright owner='Heiko Oberdiek' year='2010-2012'/>
  <license type='lppl1.3'/>
  <version number='1.10'/>
  <description>
    The package defines a single command <tt>\hologo</tt>, whose
    argument is the usual case-confused ASCII version of the logo.
    The command is bookmark-enabled, so that every logo becomes
    available in bookmarks without further work.
    <p/>
    The package is part of the <xref refid='oberdiek'>oberdiek</xref>
    bundle.
  </description>
  <documentation details='Package documentation'
      href='ctan:/macros/latex/contrib/oberdiek/hologo.pdf'/>
  <ctan file='true' path='/macros/latex/contrib/oberdiek/hologo.dtx'/>
  <miktex location='oberdiek'/>
  <texlive location='oberdiek'/>
  <install path='/macros/latex/contrib/oberdiek/oberdiek.tds.zip'/>
</entry>
%</catalogue>
%    \end{macrocode}
%
% \begin{thebibliography}{9}
% \raggedright
%
% \bibitem{btxdoc}
% Oren Patashnik,
% \textit{\hologo{BibTeX}ing},
% 1988-02-08.\\
% \CTAN{biblio/bibtex/base/}
%
% \bibitem{dtklogos}
% Gerd Neugebauer, DANTE,
% \textit{Package \xpackage{dtklogos}},
% 2011-04-25.\\
% \CTAN{usergrps/dante/dtk/dtklogos.sty}
%
% \bibitem{etexman}
% The \hologo{NTS} Team,
% \textit{The \hologo{eTeX} manual},
% 1998-02.\\
% \CTAN{systems/e-tex/v2/doc/}
%
% \bibitem{ExTeX-FAQ}
% The \hologo{ExTeX} group,
% \textit{\hologo{ExTeX}: FAQ -- How is \hologo{ExTeX} typeset?},
% 2007-04-14.\\
% \url{http://www.extex.org/documentation/faq.html}
%
% \bibitem{LyX}
% %@MISC{ LyX,
% %  title = {{LyX 2.0.0 -- The Document Processor [Computer software and manual]}},
% %  author = {{The LyX Team}},
% %  howpublished = {Internet: http://www.lyx.org},
% %  year = {2011-05-08},
% %  note = {Retrieved May 10, 2011, from http://www.lyx.org},
% %  url = {http://www.lyx.org/}
% %}
% The \hologo{LyX} Team,
% \textit{\hologo{LyX} -- The Document Processor},
% 2011-05-08.\\
% \url{http://www.lyx.org/}
%
% \bibitem{OzTeX}
% Andrew Trevorrow,
% \hologo{OzTeX} FAQ: What is the correct way to typeset ``\hologo{OzTeX}''?,
% 2011-09-15 (visited).
% \url{http://www.trevorrow.com/oztex/ozfaq.html#oztex-logo}
%
% \bibitem{PiCTeX}
% Michael Wichura,
% \textit{The \hologo{PiCTeX} macro package},
% 1987-09-21.
% \CTAN{graphics/pictex/}
%
% \bibitem{scrlogo}
% Markus Kohm,
% \textit{\hologo{KOMAScript} Datei \xfile{scrlogo.dtx}},
% 2009-01-30.\\
% \CTAN{install/macros/latex/contrib/komascript.tds.zip}
%
% \end{thebibliography}
%
% \begin{History}
%   \begin{Version}{2010/04/08 v1.0}
%   \item
%     The first version.
%   \end{Version}
%   \begin{Version}{2010/04/16 v1.1}
%   \item
%     \cs{Hologo} added for support of logos at start of a sentence.
%   \item
%     \cs{hologoSetup} and \cs{hologoLogoSetup} added.
%   \item
%     Options \xoption{break}, \xoption{hyphenbreak}, \xoption{spacebreak}
%     added.
%   \item
%     Variant support added by option \xoption{variant}.
%   \end{Version}
%   \begin{Version}{2010/04/24 v1.2}
%   \item
%     \hologo{LaTeX3} added.
%   \item
%     \hologo{VTeX} added.
%   \end{Version}
%   \begin{Version}{2010/11/21 v1.3}
%   \item
%     \hologo{iniTeX}, \hologo{virTeX} added.
%   \end{Version}
%   \begin{Version}{2011/03/25 v1.4}
%   \item
%     \hologo{ConTeXt} with variants added.
%   \item
%     Option \xoption{discretionarybreak} added as refinement for
%     option \xoption{break}.
%   \end{Version}
%   \begin{Version}{2011/04/21 v1.5}
%   \item
%     Wrong TDS directory for test files fixed.
%   \end{Version}
%   \begin{Version}{2011/10/01 v1.6}
%   \item
%     Support for package \xpackage{tex4ht} added.
%   \item
%     Support for \cs{csname} added if \cs{ifincsname} is available.
%   \item
%     New logos:
%     \hologo{(La)TeX},
%     \hologo{biber},
%     \hologo{BibTeX} (\xoption{sc}, \xoption{sf}),
%     \hologo{emTeX},
%     \hologo{ExTeX},
%     \hologo{KOMAScript},
%     \hologo{La},
%     \hologo{LyX},
%     \hologo{MiKTeX},
%     \hologo{NTS},
%     \hologo{OzMF},
%     \hologo{OzMP},
%     \hologo{OzTeX},
%     \hologo{OzTtH},
%     \hologo{PCTeX},
%     \hologo{PiC},
%     \hologo{PiCTeX},
%     \hologo{METAFONT},
%     \hologo{MetaFun},
%     \hologo{METAPOST},
%     \hologo{MetaPost},
%     \hologo{SLiTeX} (\xoption{lift}, \xoption{narrow}, \xoption{simple}),
%     \hologo{SliTeX} (\xoption{narrow}, \xoption{simple}, \xoption{lift}),
%     \hologo{teTeX}.
%   \item
%     Fixes:
%     \hologo{iniTeX},
%     \hologo{pdfLaTeX},
%     \hologo{pdfTeX},
%     \hologo{virTeX}.
%   \item
%     \cs{hologoFontSetup} and \cs{hologoLogoFontSetup} added.
%   \item
%     \cs{hologoVariant} and \cs{HologoVariant} added.
%   \end{Version}
%   \begin{Version}{2011/11/22 v1.7}
%   \item
%     New logos:
%     \hologo{BibTeX8},
%     \hologo{LaTeXML},
%     \hologo{SageTeX},
%     \hologo{TeX4ht},
%     \hologo{TTH}.
%   \item
%     \hologo{Xe} and friends: Driver stuff fixed.
%   \item
%     \hologo{Xe} and friends: Support for italic added.
%   \item
%     \hologo{Xe} and friends: Package support for \xpackage{pgf}
%     and \xpackage{pstricks} added.
%   \end{Version}
%   \begin{Version}{2011/11/29 v1.8}
%   \item
%     New logos:
%     \hologo{HanTheThanh}.
%   \end{Version}
%   \begin{Version}{2011/12/21 v1.9}
%   \item
%     Patch for package \xpackage{ifxetex} added for the case that
%     \cs{newif} is undefined in \hologo{iniTeX}.
%   \item
%     Some fixes for \hologo{iniTeX}.
%   \end{Version}
%   \begin{Version}{2012/04/26 v1.10}
%   \item
%     Fix in bookmark version of logo ``\hologo{HanTheThanh}''.
%   \end{Version}
%   \begin{Version}{2016/05/12 v1.11}
%   \item
%     Update HOLOGO@IfCharExists (previously in texlive)
%   \item define pdfliteral in current luatex.
%   \end{Version}
% \end{History}
%
% \PrintIndex
%
% \Finale
\endinput
|
% \end{quote}
% Do not forget to quote the argument according to the demands
% of your shell.
%
% \paragraph{Generating the documentation.}
% You can use both the \xfile{.dtx} or the \xfile{.drv} to generate
% the documentation. The process can be configured by the
% configuration file \xfile{ltxdoc.cfg}. For instance, put this
% line into this file, if you want to have A4 as paper format:
% \begin{quote}
%   \verb|\PassOptionsToClass{a4paper}{article}|
% \end{quote}
% An example follows how to generate the
% documentation with pdf\LaTeX:
% \begin{quote}
%\begin{verbatim}
%pdflatex hologo.dtx
%makeindex -s gind.ist hologo.idx
%pdflatex hologo.dtx
%makeindex -s gind.ist hologo.idx
%pdflatex hologo.dtx
%\end{verbatim}
% \end{quote}
%
% \section{Catalogue}
%
% The following XML file can be used as source for the
% \href{http://mirror.ctan.org/help/Catalogue/catalogue.html}{\TeX\ Catalogue}.
% The elements \texttt{caption} and \texttt{description} are imported
% from the original XML file from the Catalogue.
% The name of the XML file in the Catalogue is \xfile{hologo.xml}.
%    \begin{macrocode}
%<*catalogue>
<?xml version='1.0' encoding='us-ascii'?>
<!DOCTYPE entry SYSTEM 'catalogue.dtd'>
<entry datestamp='$Date$' modifier='$Author$' id='hologo'>
  <name>hologo</name>
  <caption>A collection of logos with bookmark support.</caption>
  <authorref id='auth:oberdiek'/>
  <copyright owner='Heiko Oberdiek' year='2010-2012'/>
  <license type='lppl1.3'/>
  <version number='1.10'/>
  <description>
    The package defines a single command <tt>\hologo</tt>, whose
    argument is the usual case-confused ASCII version of the logo.
    The command is bookmark-enabled, so that every logo becomes
    available in bookmarks without further work.
    <p/>
    The package is part of the <xref refid='oberdiek'>oberdiek</xref>
    bundle.
  </description>
  <documentation details='Package documentation'
      href='ctan:/macros/latex/contrib/oberdiek/hologo.pdf'/>
  <ctan file='true' path='/macros/latex/contrib/oberdiek/hologo.dtx'/>
  <miktex location='oberdiek'/>
  <texlive location='oberdiek'/>
  <install path='/macros/latex/contrib/oberdiek/oberdiek.tds.zip'/>
</entry>
%</catalogue>
%    \end{macrocode}
%
% \begin{thebibliography}{9}
% \raggedright
%
% \bibitem{btxdoc}
% Oren Patashnik,
% \textit{\hologo{BibTeX}ing},
% 1988-02-08.\\
% \CTAN{biblio/bibtex/base/}
%
% \bibitem{dtklogos}
% Gerd Neugebauer, DANTE,
% \textit{Package \xpackage{dtklogos}},
% 2011-04-25.\\
% \CTAN{usergrps/dante/dtk/dtklogos.sty}
%
% \bibitem{etexman}
% The \hologo{NTS} Team,
% \textit{The \hologo{eTeX} manual},
% 1998-02.\\
% \CTAN{systems/e-tex/v2/doc/}
%
% \bibitem{ExTeX-FAQ}
% The \hologo{ExTeX} group,
% \textit{\hologo{ExTeX}: FAQ -- How is \hologo{ExTeX} typeset?},
% 2007-04-14.\\
% \url{http://www.extex.org/documentation/faq.html}
%
% \bibitem{LyX}
% %@MISC{ LyX,
% %  title = {{LyX 2.0.0 -- The Document Processor [Computer software and manual]}},
% %  author = {{The LyX Team}},
% %  howpublished = {Internet: http://www.lyx.org},
% %  year = {2011-05-08},
% %  note = {Retrieved May 10, 2011, from http://www.lyx.org},
% %  url = {http://www.lyx.org/}
% %}
% The \hologo{LyX} Team,
% \textit{\hologo{LyX} -- The Document Processor},
% 2011-05-08.\\
% \url{http://www.lyx.org/}
%
% \bibitem{OzTeX}
% Andrew Trevorrow,
% \hologo{OzTeX} FAQ: What is the correct way to typeset ``\hologo{OzTeX}''?,
% 2011-09-15 (visited).
% \url{http://www.trevorrow.com/oztex/ozfaq.html#oztex-logo}
%
% \bibitem{PiCTeX}
% Michael Wichura,
% \textit{The \hologo{PiCTeX} macro package},
% 1987-09-21.
% \CTAN{graphics/pictex/}
%
% \bibitem{scrlogo}
% Markus Kohm,
% \textit{\hologo{KOMAScript} Datei \xfile{scrlogo.dtx}},
% 2009-01-30.\\
% \CTAN{install/macros/latex/contrib/komascript.tds.zip}
%
% \end{thebibliography}
%
% \begin{History}
%   \begin{Version}{2010/04/08 v1.0}
%   \item
%     The first version.
%   \end{Version}
%   \begin{Version}{2010/04/16 v1.1}
%   \item
%     \cs{Hologo} added for support of logos at start of a sentence.
%   \item
%     \cs{hologoSetup} and \cs{hologoLogoSetup} added.
%   \item
%     Options \xoption{break}, \xoption{hyphenbreak}, \xoption{spacebreak}
%     added.
%   \item
%     Variant support added by option \xoption{variant}.
%   \end{Version}
%   \begin{Version}{2010/04/24 v1.2}
%   \item
%     \hologo{LaTeX3} added.
%   \item
%     \hologo{VTeX} added.
%   \end{Version}
%   \begin{Version}{2010/11/21 v1.3}
%   \item
%     \hologo{iniTeX}, \hologo{virTeX} added.
%   \end{Version}
%   \begin{Version}{2011/03/25 v1.4}
%   \item
%     \hologo{ConTeXt} with variants added.
%   \item
%     Option \xoption{discretionarybreak} added as refinement for
%     option \xoption{break}.
%   \end{Version}
%   \begin{Version}{2011/04/21 v1.5}
%   \item
%     Wrong TDS directory for test files fixed.
%   \end{Version}
%   \begin{Version}{2011/10/01 v1.6}
%   \item
%     Support for package \xpackage{tex4ht} added.
%   \item
%     Support for \cs{csname} added if \cs{ifincsname} is available.
%   \item
%     New logos:
%     \hologo{(La)TeX},
%     \hologo{biber},
%     \hologo{BibTeX} (\xoption{sc}, \xoption{sf}),
%     \hologo{emTeX},
%     \hologo{ExTeX},
%     \hologo{KOMAScript},
%     \hologo{La},
%     \hologo{LyX},
%     \hologo{MiKTeX},
%     \hologo{NTS},
%     \hologo{OzMF},
%     \hologo{OzMP},
%     \hologo{OzTeX},
%     \hologo{OzTtH},
%     \hologo{PCTeX},
%     \hologo{PiC},
%     \hologo{PiCTeX},
%     \hologo{METAFONT},
%     \hologo{MetaFun},
%     \hologo{METAPOST},
%     \hologo{MetaPost},
%     \hologo{SLiTeX} (\xoption{lift}, \xoption{narrow}, \xoption{simple}),
%     \hologo{SliTeX} (\xoption{narrow}, \xoption{simple}, \xoption{lift}),
%     \hologo{teTeX}.
%   \item
%     Fixes:
%     \hologo{iniTeX},
%     \hologo{pdfLaTeX},
%     \hologo{pdfTeX},
%     \hologo{virTeX}.
%   \item
%     \cs{hologoFontSetup} and \cs{hologoLogoFontSetup} added.
%   \item
%     \cs{hologoVariant} and \cs{HologoVariant} added.
%   \end{Version}
%   \begin{Version}{2011/11/22 v1.7}
%   \item
%     New logos:
%     \hologo{BibTeX8},
%     \hologo{LaTeXML},
%     \hologo{SageTeX},
%     \hologo{TeX4ht},
%     \hologo{TTH}.
%   \item
%     \hologo{Xe} and friends: Driver stuff fixed.
%   \item
%     \hologo{Xe} and friends: Support for italic added.
%   \item
%     \hologo{Xe} and friends: Package support for \xpackage{pgf}
%     and \xpackage{pstricks} added.
%   \end{Version}
%   \begin{Version}{2011/11/29 v1.8}
%   \item
%     New logos:
%     \hologo{HanTheThanh}.
%   \end{Version}
%   \begin{Version}{2011/12/21 v1.9}
%   \item
%     Patch for package \xpackage{ifxetex} added for the case that
%     \cs{newif} is undefined in \hologo{iniTeX}.
%   \item
%     Some fixes for \hologo{iniTeX}.
%   \end{Version}
%   \begin{Version}{2012/04/26 v1.10}
%   \item
%     Fix in bookmark version of logo ``\hologo{HanTheThanh}''.
%   \end{Version}
%   \begin{Version}{2016/05/12 v1.11}
%   \item
%     Update HOLOGO@IfCharExists (previously in texlive)
%   \item define pdfliteral in current luatex.
%   \end{Version}
% \end{History}
%
% \PrintIndex
%
% \Finale
\endinput
|
% \end{quote}
% Do not forget to quote the argument according to the demands
% of your shell.
%
% \paragraph{Generating the documentation.}
% You can use both the \xfile{.dtx} or the \xfile{.drv} to generate
% the documentation. The process can be configured by the
% configuration file \xfile{ltxdoc.cfg}. For instance, put this
% line into this file, if you want to have A4 as paper format:
% \begin{quote}
%   \verb|\PassOptionsToClass{a4paper}{article}|
% \end{quote}
% An example follows how to generate the
% documentation with pdf\LaTeX:
% \begin{quote}
%\begin{verbatim}
%pdflatex hologo.dtx
%makeindex -s gind.ist hologo.idx
%pdflatex hologo.dtx
%makeindex -s gind.ist hologo.idx
%pdflatex hologo.dtx
%\end{verbatim}
% \end{quote}
%
% \section{Catalogue}
%
% The following XML file can be used as source for the
% \href{http://mirror.ctan.org/help/Catalogue/catalogue.html}{\TeX\ Catalogue}.
% The elements \texttt{caption} and \texttt{description} are imported
% from the original XML file from the Catalogue.
% The name of the XML file in the Catalogue is \xfile{hologo.xml}.
%    \begin{macrocode}
%<*catalogue>
<?xml version='1.0' encoding='us-ascii'?>
<!DOCTYPE entry SYSTEM 'catalogue.dtd'>
<entry datestamp='$Date$' modifier='$Author$' id='hologo'>
  <name>hologo</name>
  <caption>A collection of logos with bookmark support.</caption>
  <authorref id='auth:oberdiek'/>
  <copyright owner='Heiko Oberdiek' year='2010-2012'/>
  <license type='lppl1.3'/>
  <version number='1.10'/>
  <description>
    The package defines a single command <tt>\hologo</tt>, whose
    argument is the usual case-confused ASCII version of the logo.
    The command is bookmark-enabled, so that every logo becomes
    available in bookmarks without further work.
    <p/>
    The package is part of the <xref refid='oberdiek'>oberdiek</xref>
    bundle.
  </description>
  <documentation details='Package documentation'
      href='ctan:/macros/latex/contrib/oberdiek/hologo.pdf'/>
  <ctan file='true' path='/macros/latex/contrib/oberdiek/hologo.dtx'/>
  <miktex location='oberdiek'/>
  <texlive location='oberdiek'/>
  <install path='/macros/latex/contrib/oberdiek/oberdiek.tds.zip'/>
</entry>
%</catalogue>
%    \end{macrocode}
%
% \begin{thebibliography}{9}
% \raggedright
%
% \bibitem{btxdoc}
% Oren Patashnik,
% \textit{\hologo{BibTeX}ing},
% 1988-02-08.\\
% \CTAN{biblio/bibtex/base/}
%
% \bibitem{dtklogos}
% Gerd Neugebauer, DANTE,
% \textit{Package \xpackage{dtklogos}},
% 2011-04-25.\\
% \CTAN{usergrps/dante/dtk/dtklogos.sty}
%
% \bibitem{etexman}
% The \hologo{NTS} Team,
% \textit{The \hologo{eTeX} manual},
% 1998-02.\\
% \CTAN{systems/e-tex/v2/doc/}
%
% \bibitem{ExTeX-FAQ}
% The \hologo{ExTeX} group,
% \textit{\hologo{ExTeX}: FAQ -- How is \hologo{ExTeX} typeset?},
% 2007-04-14.\\
% \url{http://www.extex.org/documentation/faq.html}
%
% \bibitem{LyX}
% %@MISC{ LyX,
% %  title = {{LyX 2.0.0 -- The Document Processor [Computer software and manual]}},
% %  author = {{The LyX Team}},
% %  howpublished = {Internet: http://www.lyx.org},
% %  year = {2011-05-08},
% %  note = {Retrieved May 10, 2011, from http://www.lyx.org},
% %  url = {http://www.lyx.org/}
% %}
% The \hologo{LyX} Team,
% \textit{\hologo{LyX} -- The Document Processor},
% 2011-05-08.\\
% \url{http://www.lyx.org/}
%
% \bibitem{OzTeX}
% Andrew Trevorrow,
% \hologo{OzTeX} FAQ: What is the correct way to typeset ``\hologo{OzTeX}''?,
% 2011-09-15 (visited).
% \url{http://www.trevorrow.com/oztex/ozfaq.html#oztex-logo}
%
% \bibitem{PiCTeX}
% Michael Wichura,
% \textit{The \hologo{PiCTeX} macro package},
% 1987-09-21.
% \CTAN{graphics/pictex/}
%
% \bibitem{scrlogo}
% Markus Kohm,
% \textit{\hologo{KOMAScript} Datei \xfile{scrlogo.dtx}},
% 2009-01-30.\\
% \CTAN{install/macros/latex/contrib/komascript.tds.zip}
%
% \end{thebibliography}
%
% \begin{History}
%   \begin{Version}{2010/04/08 v1.0}
%   \item
%     The first version.
%   \end{Version}
%   \begin{Version}{2010/04/16 v1.1}
%   \item
%     \cs{Hologo} added for support of logos at start of a sentence.
%   \item
%     \cs{hologoSetup} and \cs{hologoLogoSetup} added.
%   \item
%     Options \xoption{break}, \xoption{hyphenbreak}, \xoption{spacebreak}
%     added.
%   \item
%     Variant support added by option \xoption{variant}.
%   \end{Version}
%   \begin{Version}{2010/04/24 v1.2}
%   \item
%     \hologo{LaTeX3} added.
%   \item
%     \hologo{VTeX} added.
%   \end{Version}
%   \begin{Version}{2010/11/21 v1.3}
%   \item
%     \hologo{iniTeX}, \hologo{virTeX} added.
%   \end{Version}
%   \begin{Version}{2011/03/25 v1.4}
%   \item
%     \hologo{ConTeXt} with variants added.
%   \item
%     Option \xoption{discretionarybreak} added as refinement for
%     option \xoption{break}.
%   \end{Version}
%   \begin{Version}{2011/04/21 v1.5}
%   \item
%     Wrong TDS directory for test files fixed.
%   \end{Version}
%   \begin{Version}{2011/10/01 v1.6}
%   \item
%     Support for package \xpackage{tex4ht} added.
%   \item
%     Support for \cs{csname} added if \cs{ifincsname} is available.
%   \item
%     New logos:
%     \hologo{(La)TeX},
%     \hologo{biber},
%     \hologo{BibTeX} (\xoption{sc}, \xoption{sf}),
%     \hologo{emTeX},
%     \hologo{ExTeX},
%     \hologo{KOMAScript},
%     \hologo{La},
%     \hologo{LyX},
%     \hologo{MiKTeX},
%     \hologo{NTS},
%     \hologo{OzMF},
%     \hologo{OzMP},
%     \hologo{OzTeX},
%     \hologo{OzTtH},
%     \hologo{PCTeX},
%     \hologo{PiC},
%     \hologo{PiCTeX},
%     \hologo{METAFONT},
%     \hologo{MetaFun},
%     \hologo{METAPOST},
%     \hologo{MetaPost},
%     \hologo{SLiTeX} (\xoption{lift}, \xoption{narrow}, \xoption{simple}),
%     \hologo{SliTeX} (\xoption{narrow}, \xoption{simple}, \xoption{lift}),
%     \hologo{teTeX}.
%   \item
%     Fixes:
%     \hologo{iniTeX},
%     \hologo{pdfLaTeX},
%     \hologo{pdfTeX},
%     \hologo{virTeX}.
%   \item
%     \cs{hologoFontSetup} and \cs{hologoLogoFontSetup} added.
%   \item
%     \cs{hologoVariant} and \cs{HologoVariant} added.
%   \end{Version}
%   \begin{Version}{2011/11/22 v1.7}
%   \item
%     New logos:
%     \hologo{BibTeX8},
%     \hologo{LaTeXML},
%     \hologo{SageTeX},
%     \hologo{TeX4ht},
%     \hologo{TTH}.
%   \item
%     \hologo{Xe} and friends: Driver stuff fixed.
%   \item
%     \hologo{Xe} and friends: Support for italic added.
%   \item
%     \hologo{Xe} and friends: Package support for \xpackage{pgf}
%     and \xpackage{pstricks} added.
%   \end{Version}
%   \begin{Version}{2011/11/29 v1.8}
%   \item
%     New logos:
%     \hologo{HanTheThanh}.
%   \end{Version}
%   \begin{Version}{2011/12/21 v1.9}
%   \item
%     Patch for package \xpackage{ifxetex} added for the case that
%     \cs{newif} is undefined in \hologo{iniTeX}.
%   \item
%     Some fixes for \hologo{iniTeX}.
%   \end{Version}
%   \begin{Version}{2012/04/26 v1.10}
%   \item
%     Fix in bookmark version of logo ``\hologo{HanTheThanh}''.
%   \end{Version}
%   \begin{Version}{2016/05/12 v1.11}
%   \item
%     Update HOLOGO@IfCharExists (previously in texlive)
%   \item define pdfliteral in current luatex.
%   \end{Version}
% \end{History}
%
% \PrintIndex
%
% \Finale
\endinput

%        (quote the arguments according to the demands of your shell)
%
% Documentation:
%    (a) If hologo.drv is present:
%           latex hologo.drv
%    (b) Without hologo.drv:
%           latex hologo.dtx; ...
%    The class ltxdoc loads the configuration file ltxdoc.cfg
%    if available. Here you can specify further options, e.g.
%    use A4 as paper format:
%       \PassOptionsToClass{a4paper}{article}
%
%    Programm calls to get the documentation (example):
%       pdflatex hologo.dtx
%       makeindex -s gind.ist hologo.idx
%       pdflatex hologo.dtx
%       makeindex -s gind.ist hologo.idx
%       pdflatex hologo.dtx
%
% Installation:
%    TDS:tex/generic/oberdiek/hologo.sty
%    TDS:doc/latex/oberdiek/hologo.pdf
%    TDS:doc/latex/oberdiek/example/hologo-example.tex
%    TDS:doc/latex/oberdiek/test/hologo-test1.tex
%    TDS:doc/latex/oberdiek/test/hologo-test-spacefactor.tex
%    TDS:doc/latex/oberdiek/test/hologo-test-list.tex
%    TDS:source/latex/oberdiek/hologo.dtx
%
%<*ignore>
\begingroup
  \catcode123=1 %
  \catcode125=2 %
  \def\x{LaTeX2e}%
\expandafter\endgroup
\ifcase 0\ifx\install y1\fi\expandafter
         \ifx\csname processbatchFile\endcsname\relax\else1\fi
         \ifx\fmtname\x\else 1\fi\relax
\else\csname fi\endcsname
%</ignore>
%<*install>
\input docstrip.tex
\Msg{************************************************************************}
\Msg{* Installation}
\Msg{* Package: hologo 2016/05/12 v1.11 A logo collection with bookmark support (HO)}
\Msg{************************************************************************}

\keepsilent
\askforoverwritefalse

\let\MetaPrefix\relax
\preamble

This is a generated file.

Project: hologo
Version: 2016/05/12 v1.11

Copyright (C) 2010-2012 by
   Heiko Oberdiek <heiko.oberdiek at googlemail.com>

This work may be distributed and/or modified under the
conditions of the LaTeX Project Public License, either
version 1.3c of this license or (at your option) any later
version. This version of this license is in
   http://www.latex-project.org/lppl/lppl-1-3c.txt
and the latest version of this license is in
   http://www.latex-project.org/lppl.txt
and version 1.3 or later is part of all distributions of
LaTeX version 2005/12/01 or later.

This work has the LPPL maintenance status "maintained".

This Current Maintainer of this work is Heiko Oberdiek.

The Base Interpreter refers to any `TeX-Format',
because some files are installed in TDS:tex/generic//.

This work consists of the main source file hologo.dtx
and the derived files
   hologo.sty, hologo.pdf, hologo.ins, hologo.drv, hologo-example.tex,
   hologo-test1.tex, hologo-test-spacefactor.tex,
   hologo-test-list.tex.

\endpreamble
\let\MetaPrefix\DoubleperCent

\generate{%
  \file{hologo.ins}{\from{hologo.dtx}{install}}%
  \file{hologo.drv}{\from{hologo.dtx}{driver}}%
  \usedir{tex/generic/oberdiek}%
  \file{hologo.sty}{\from{hologo.dtx}{package}}%
  \usedir{doc/latex/oberdiek/example}%
  \file{hologo-example.tex}{\from{hologo.dtx}{example}}%
  \usedir{doc/latex/oberdiek/test}%
  \file{hologo-test1.tex}{\from{hologo.dtx}{test1}}%
  \file{hologo-test-spacefactor.tex}{\from{hologo.dtx}{test-spacefactor}}%
  \file{hologo-test-list.tex}{\from{hologo.dtx}{test-list}}%
  \nopreamble
  \nopostamble
  \usedir{source/latex/oberdiek/catalogue}%
  \file{hologo.xml}{\from{hologo.dtx}{catalogue}}%
}

\catcode32=13\relax% active space
\let =\space%
\Msg{************************************************************************}
\Msg{*}
\Msg{* To finish the installation you have to move the following}
\Msg{* file into a directory searched by TeX:}
\Msg{*}
\Msg{*     hologo.sty}
\Msg{*}
\Msg{* To produce the documentation run the file `hologo.drv'}
\Msg{* through LaTeX.}
\Msg{*}
\Msg{* Happy TeXing!}
\Msg{*}
\Msg{************************************************************************}

\endbatchfile
%</install>
%<*ignore>
\fi
%</ignore>
%<*driver>
\NeedsTeXFormat{LaTeX2e}
\ProvidesFile{hologo.drv}%
  [2016/05/12 v1.11 A logo collection with bookmark support (HO)]%
\documentclass{ltxdoc}
\usepackage{holtxdoc}[2011/11/22]
\usepackage{hologo}[2016/05/12]
\usepackage{longtable}
\usepackage{array}
\usepackage{paralist}
%\usepackage[T1]{fontenc}
%\usepackage{lmodern}
\begin{document}
  \DocInput{hologo.dtx}%
\end{document}
%</driver>
% \fi
%
%
% \CharacterTable
%  {Upper-case    \A\B\C\D\E\F\G\H\I\J\K\L\M\N\O\P\Q\R\S\T\U\V\W\X\Y\Z
%   Lower-case    \a\b\c\d\e\f\g\h\i\j\k\l\m\n\o\p\q\r\s\t\u\v\w\x\y\z
%   Digits        \0\1\2\3\4\5\6\7\8\9
%   Exclamation   \!     Double quote  \"     Hash (number) \#
%   Dollar        \$     Percent       \%     Ampersand     \&
%   Acute accent  \'     Left paren    \(     Right paren   \)
%   Asterisk      \*     Plus          \+     Comma         \,
%   Minus         \-     Point         \.     Solidus       \/
%   Colon         \:     Semicolon     \;     Less than     \<
%   Equals        \=     Greater than  \>     Question mark \?
%   Commercial at \@     Left bracket  \[     Backslash     \\
%   Right bracket \]     Circumflex    \^     Underscore    \_
%   Grave accent  \`     Left brace    \{     Vertical bar  \|
%   Right brace   \}     Tilde         \~}
%
% \GetFileInfo{hologo.drv}
%
% \title{The \xpackage{hologo} package}
% \date{2016/05/12 v1.11}
% \author{Heiko Oberdiek\\\xemail{heiko.oberdiek at googlemail.com}}
%
% \maketitle
%
% \begin{abstract}
% This package starts a collection of logos with support for bookmarks
% strings.
% \end{abstract}
%
% \tableofcontents
%
% \section{Documentation}
%
% \subsection{Logo macros}
%
% \begin{declcs}{hologo} \M{name}
% \end{declcs}
% Macro \cs{hologo} sets the logo with name \meta{name}.
% The following table shows the supported names.
%
% \begingroup
%   \def\hologoEntry#1#2#3{^^A
%     #1&#2&\hologoLogoSetup{#1}{variant=#2}\hologo{#1}&#3\tabularnewline
%   }
%   \begin{longtable}{>{\ttfamily}l>{\ttfamily}lll}
%     \rmfamily\bfseries{name} & \rmfamily\bfseries variant
%     & \bfseries logo & \bfseries since\\
%     \hline
%     \endhead
%     \hologoList
%   \end{longtable}
% \endgroup
%
% \begin{declcs}{Hologo} \M{name}
% \end{declcs}
% Macro \cs{Hologo} starts the logo \meta{name} with an uppercase
% letter. As an exception small greek letters are not converted
% to uppercase. Examples, see \hologo{eTeX} and \hologo{ExTeX}.
%
% \subsection{Setup macros}
%
% The package does not support package options, but the following
% setup macros can be used to set options.
%
% \begin{declcs}{hologoSetup} \M{key value list}
% \end{declcs}
% Macro \cs{hologoSetup} sets global options.
%
% \begin{declcs}{hologoLogoSetup} \M{logo} \M{key value list}
% \end{declcs}
% Some options can also be used to configure a logo.
% These settings take precedence over global option settings.
%
% \subsection{Options}\label{sec:options}
%
% There are boolean and string options:
% \begin{description}
% \item[Boolean option:]
% It takes |true| or |false|
% as value. If the value is omitted, then |true| is used.
% \item[String option:]
% A value must be given as string. (But the string might be empty.)
% \end{description}
% The following options can be used both in \cs{hologoSetup}
% and \cs{hologoLogoSetup}:
% \begin{description}
% \def\entry#1{\item[\xoption{#1}:]}
% \entry{break}
%   enables or disables line breaks inside the logo. This setting is
%   refined by options \xoption{hyphenbreak}, \xoption{spacebreak}
%   or \xoption{discretionarybreak}.
%   Default is |false|.
% \entry{hyphenbreak}
%   enables or disables the line break right after the hyphen character.
% \entry{spacebreak}
%   enables or disables line breaks at space characters.
% \entry{discretionarybreak}
%   enables or disables line breaks at hyphenation points
%   (inserted by \cs{-}).
% \end{description}
% Macro \cs{hologoLogoSetup} also knows:
% \begin{description}
% \item[\xoption{variant}:]
%   This is a string option. It specifies a variant of a logo that
%   must exist. An empty string selects the package default variant.
% \end{description}
% Example:
% \begin{quote}
%   |\hologoSetup{break=false}|\\
%   |\hologoLogoSetup{plainTeX}{variant=hyphen,hyphenbreak}|\\
%   Then ``plain-\TeX'' contains one break point after the hyphen.
% \end{quote}
%
% \subsection{Driver options}
%
% Sometimes graphical operations are needed to construct some
% glyphs (e.g.\ \hologo{XeTeX}). If package \xpackage{graphics}
% or package \xpackage{pgf} are found, then the macros are taken
% from there. Otherwise the packge defines its own operations
% and therefore needs the driver information. Many drivers are
% detected automatically (\hologo{pdfTeX}/\hologo{LuaTeX}
% in PDF mode, \hologo{XeTeX}, \hologo{VTeX}). These have precedence
% over a driver option. The driver can be given as package option
% or using \cs{hologoDriverSetup}.
% The following list contains the recognized driver options:
% \begin{itemize}
% \item \xoption{pdftex}, \xoption{luatex}
% \item \xoption{dvipdfm}, \xoption{dvipdfmx}
% \item \xoption{dvips}, \xoption{dvipsone}, \xoption{xdvi}
% \item \xoption{xetex}
% \item \xoption{vtex}
% \end{itemize}
% The left driver of a line is the driver name that is used internally.
% The following names are aliases for drivers that use the
% same method. Therefore the entry in the \xext{log} file for
% the used driver prints the internally used driver name.
% \begin{description}
% \item[\xoption{driverfallback}:]
%   This option expects a driver that is used,
%   if the driver could not be detected automatically.
% \end{description}
%
% \begin{declcs}{hologoDriverSetup} \M{driver option}
% \end{declcs}
% The driver can also be configured after package loading
% using \cs{hologoDriverSetup}, also the way for \hologo{plainTeX}
% to setup the driver.
%
% \subsection{Font setup}
%
% Some logos require a special font, but should also be usable by
% \hologo{plainTeX}. Therefore the package provides some ways
% to influence the font settings. The options below
% take font settings as values. Both font commands
% such as \cs{sffamily} and macros that take one argument
% like \cs{textsf} can be used.
%
% \begin{declcs}{hologoFontSetup} \M{key value list}
% \end{declcs}
% Macro \cs{hologoFontSetup} sets the fonts for all logos.
% Supported keys:
% \begin{description}
% \def\entry#1{\item[\xoption{#1}:]}
% \entry{general}
%   This font is used for all logos. The default is empty.
%   That means no special font is used.
% \entry{bibsf}
%   This font is used for
%   {\hologoLogoSetup{BibTeX}{variant=sf}\hologo{BibTeX}}
%   with variant \xoption{sf}.
% \entry{rm}
%   This font is a serif font. It is used for \hologo{ExTeX}.
% \entry{sc}
%   This font specifies a small caps font. It is used for
%   {\hologoLogoSetup{BibTeX}{variant=sc}\hologo{BibTeX}}
%   with variant \xoption{sc}.
% \entry{sf}
%   This font specifies a sans serif font. The default
%   is \cs{sffamily}, then \cs{sf} is tried. Otherwise
%   a warning is given. It is used by \hologo{KOMAScript}.
% \entry{sy}
%   This is the font for math symbols (e.g. cmsy).
%   It is used by \hologo{AmS}, \hologo{NTS}, \hologo{ExTeX}.
% \entry{logo}
%   \hologo{METAFONT} and \hologo{METAPOST} are using that font.
%   In \hologo{LaTeX} \cs{logofamily} is used and
%   the definitions of package \xpackage{mflogo} are used
%   if the package is not loaded.
%   Otherwise the \cs{tenlogo} is used and defined
%   if it does not already exists.
% \end{description}
%
% \begin{declcs}{hologoLogoFontSetup} \M{logo} \M{key value list}
% \end{declcs}
% Fonts can also be set for a logo or logo component separately,
% see the following list.
% The keys are the same as for \cs{hologoFontSetup}.
%
% \begin{longtable}{>{\ttfamily}l>{\sffamily}ll}
%   \meta{logo} & keys & result\\
%   \hline
%   \endhead
%   BibTeX & bibsf & {\hologoLogoSetup{BibTeX}{variant=sf}\hologo{BibTeX}}\\[.5ex]
%   BibTeX & sc & {\hologoLogoSetup{BibTeX}{variant=sc}\hologo{BibTeX}}\\[.5ex]
%   ExTeX & rm & \hologo{ExTeX}\\
%   SliTeX & rm & \hologo{SliTeX}\\[.5ex]
%   AmS & sy & \hologo{AmS}\\
%   ExTeX & sy & \hologo{ExTeX}\\
%   NTS & sy & \hologo{NTS}\\[.5ex]
%   KOMAScript & sf & \hologo{KOMAScript}\\[.5ex]
%   METAFONT & logo & \hologo{METAFONT}\\
%   METAPOST & logo & \hologo{METAPOST}\\[.5ex]
%   SliTeX & sc \hologo{SliTeX}
% \end{longtable}
%
% \subsubsection{Font order}
%
% For all logos the font \xoption{general} is applied first.
% Example:
%\begin{quote}
%|\hologoFontSetup{general=\color{red}}|
%\end{quote}
% will print red logos.
% Then if the font uses a special font \xoption{sf}, for example,
% the font is applied that is setup by \cs{hologoLogoFontSetup}.
% If this font is not setup, then the common font setup
% by \cs{hologoFontSetup} is used. Otherwise a warning is given,
% that there is no font configured.
%
% \subsection{Additional user macros}
%
% Usually a variant of a logo is configured by using
% \cs{hologoLogoSetup}, because it is bad style to mix
% different variants of the same logo in the same text.
% There the following macros are a convenience for testing.
%
% \begin{declcs}{hologoVariant} \M{name} \M{variant}\\
%   \cs{HologoVariant} \M{name} \M{variant}
% \end{declcs}
% Logo \meta{name} is set using \meta{variant} that specifies
% explicitely which variant of the macro is used. If the argument
% is empty, then the default form of the logo is used
% (configurable by \cs{hologoLogoSetup}).
%
% \cs{HologoVariant} is used if the logo is set in a context
% that needs an uppercase first letter (beginning of a sentence, \dots).
%
% \begin{declcs}{hologoList}\\
%   \cs{hologoEntry} \M{logo} \M{variant} \M{since}
% \end{declcs}
% Macro \cs{hologoList} contains all logos that are provided
% by the package including variants. The list consists of calls
% of \cs{hologoEntry} with three arguments starting with the
% logo name \meta{logo} and its variant \meta{variant}. An empty
% variant means the current default. Argument \meta{since} specifies
% with version of the package \xpackage{hologo} is needed to get
% the logo. If the logo is fixed, then the date gets updated.
% Therefore the date \meta{since} is not exactly the date of
% the first introduction, but rather the date of the latest fix.
%
% Before \cs{hologoList} can be used, macro \cs{hologoEntry} needs
% a definition. The example file in section \ref{sec:example}
% shows applications of \cs{hologoList}.
%
% \subsection{Supported contexts}
%
% Macros \cs{hologo} and friends support special contexts:
% \begin{itemize}
% \item \hologo{LaTeX}'s protection mechanism.
% \item Bookmarks of package \xpackage{hyperref}.
% \item Package \xpackage{tex4ht}.
% \item The macros can be used inside \cs{csname} constructs,
%   if \cs{ifincsname} is available (\hologo{pdfTeX}, \hologo{XeTeX},
%   \hologo{LuaTeX}).
% \end{itemize}
%
% \subsection{Example}
% \label{sec:example}
%
% The following example prints the logos in different fonts.
%    \begin{macrocode}
%<*example>
%<<verbatim
\NeedsTeXFormat{LaTeX2e}
\documentclass[a4paper]{article}
\usepackage[
  hmargin=20mm,
  vmargin=20mm,
]{geometry}
\pagestyle{empty}
\usepackage{hologo}[2016/05/12]
\usepackage{longtable}
\usepackage{array}
\setlength{\extrarowheight}{2pt}
\usepackage[T1]{fontenc}
\usepackage{lmodern}
\usepackage{pdflscape}
\usepackage[
  pdfencoding=auto,
]{hyperref}
\hypersetup{
  pdfauthor={Heiko Oberdiek},
  pdftitle={Example for package `hologo'},
  pdfsubject={Logos with fonts lmr, lmss, qtm, qpl, qhv},
}
\usepackage{bookmark}

% Print the logo list on the console

\begingroup
  \typeout{}%
  \typeout{*** Begin of logo list ***}%
  \newcommand*{\hologoEntry}[3]{%
    \typeout{#1 \ifx\\#2\\\else(#2) \fi[#3]}%
  }%
  \hologoList
  \typeout{*** End of logo list ***}%
  \typeout{}%
\endgroup

\begin{document}
\begin{landscape}

  \section{Example file for package `hologo'}

  % Table for font names

  \begin{longtable}{>{\bfseries}ll}
    \textbf{font} & \textbf{Font name}\\
    \hline
    lmr & Latin Modern Roman\\
    lmss & Latin Modern Sans\\
    qtm & \TeX\ Gyre Termes\\
    qhv & \TeX\ Gyre Heros\\
    qpl & \TeX\ Gyre Pagella\\
  \end{longtable}

  % Logo list with logos in different fonts

  \begingroup
    \newcommand*{\SetVariant}[2]{%
      \ifx\\#2\\%
      \else
        \hologoLogoSetup{#1}{variant=#2}%
      \fi
    }%
    \newcommand*{\hologoEntry}[3]{%
      \SetVariant{#1}{#2}%
      \raisebox{1em}[0pt][0pt]{\hypertarget{#1@#2}{}}%
      \bookmark[%
        dest={#1@#2},%
      ]{%
        #1\ifx\\#2\\\else\space(#2)\fi: \Hologo{#1}, \hologo{#1} %
        [Unicode]%
      }%
      \hypersetup{unicode=false}%
      \bookmark[%
        dest={#1@#2},%
      ]{%
        #1\ifx\\#2\\\else\space(#2)\fi: \Hologo{#1}, \hologo{#1} %
        [PDFDocEncoding]%
      }%
      \texttt{#1}%
      &%
      \texttt{#2}%
      &%
      \Hologo{#1}%
      &%
      \SetVariant{#1}{#2}%
      \hologo{#1}%
      &%
      \SetVariant{#1}{#2}%
      \fontfamily{qtm}\selectfont
      \hologo{#1}%
      &%
      \SetVariant{#1}{#2}%
      \fontfamily{qpl}\selectfont
      \hologo{#1}%
      &%
      \SetVariant{#1}{#2}%
      \textsf{\hologo{#1}}%
      &%
      \SetVariant{#1}{#2}%
      \fontfamily{qhv}\selectfont
      \hologo{#1}%
      \tabularnewline
    }%
    \begin{longtable}{llllllll}%
      \textbf{\textit{logo}} & \textbf{\textit{variant}} &
      \texttt{\string\Hologo} &
      \textbf{lmr} & \textbf{qtm} & \textbf{qpl} &
      \textbf{lmss} & \textbf{qhv}
      \tabularnewline
      \hline
      \endhead
      \hologoList
    \end{longtable}%
  \endgroup

\end{landscape}
\end{document}
%verbatim
%</example>
%    \end{macrocode}
%
% \StopEventually{
% }
%
% \section{Implementation}
%    \begin{macrocode}
%<*package>
%    \end{macrocode}
%    Reload check, especially if the package is not used with \LaTeX.
%    \begin{macrocode}
\begingroup\catcode61\catcode48\catcode32=10\relax%
  \catcode13=5 % ^^M
  \endlinechar=13 %
  \catcode35=6 % #
  \catcode39=12 % '
  \catcode44=12 % ,
  \catcode45=12 % -
  \catcode46=12 % .
  \catcode58=12 % :
  \catcode64=11 % @
  \catcode123=1 % {
  \catcode125=2 % }
  \expandafter\let\expandafter\x\csname ver@hologo.sty\endcsname
  \ifx\x\relax % plain-TeX, first loading
  \else
    \def\empty{}%
    \ifx\x\empty % LaTeX, first loading,
      % variable is initialized, but \ProvidesPackage not yet seen
    \else
      \expandafter\ifx\csname PackageInfo\endcsname\relax
        \def\x#1#2{%
          \immediate\write-1{Package #1 Info: #2.}%
        }%
      \else
        \def\x#1#2{\PackageInfo{#1}{#2, stopped}}%
      \fi
      \x{hologo}{The package is already loaded}%
      \aftergroup\endinput
    \fi
  \fi
\endgroup%
%    \end{macrocode}
%    Package identification:
%    \begin{macrocode}
\begingroup\catcode61\catcode48\catcode32=10\relax%
  \catcode13=5 % ^^M
  \endlinechar=13 %
  \catcode35=6 % #
  \catcode39=12 % '
  \catcode40=12 % (
  \catcode41=12 % )
  \catcode44=12 % ,
  \catcode45=12 % -
  \catcode46=12 % .
  \catcode47=12 % /
  \catcode58=12 % :
  \catcode64=11 % @
  \catcode91=12 % [
  \catcode93=12 % ]
  \catcode123=1 % {
  \catcode125=2 % }
  \expandafter\ifx\csname ProvidesPackage\endcsname\relax
    \def\x#1#2#3[#4]{\endgroup
      \immediate\write-1{Package: #3 #4}%
      \xdef#1{#4}%
    }%
  \else
    \def\x#1#2[#3]{\endgroup
      #2[{#3}]%
      \ifx#1\@undefined
        \xdef#1{#3}%
      \fi
      \ifx#1\relax
        \xdef#1{#3}%
      \fi
    }%
  \fi
\expandafter\x\csname ver@hologo.sty\endcsname
\ProvidesPackage{hologo}%
  [2016/05/12 v1.11 A logo collection with bookmark support (HO)]%
%    \end{macrocode}
%
%    \begin{macrocode}
\begingroup\catcode61\catcode48\catcode32=10\relax%
  \catcode13=5 % ^^M
  \endlinechar=13 %
  \catcode123=1 % {
  \catcode125=2 % }
  \catcode64=11 % @
  \def\x{\endgroup
    \expandafter\edef\csname HOLOGO@AtEnd\endcsname{%
      \endlinechar=\the\endlinechar\relax
      \catcode13=\the\catcode13\relax
      \catcode32=\the\catcode32\relax
      \catcode35=\the\catcode35\relax
      \catcode61=\the\catcode61\relax
      \catcode64=\the\catcode64\relax
      \catcode123=\the\catcode123\relax
      \catcode125=\the\catcode125\relax
    }%
  }%
\x\catcode61\catcode48\catcode32=10\relax%
\catcode13=5 % ^^M
\endlinechar=13 %
\catcode35=6 % #
\catcode64=11 % @
\catcode123=1 % {
\catcode125=2 % }
\def\TMP@EnsureCode#1#2{%
  \edef\HOLOGO@AtEnd{%
    \HOLOGO@AtEnd
    \catcode#1=\the\catcode#1\relax
  }%
  \catcode#1=#2\relax
}
\TMP@EnsureCode{10}{12}% ^^J
\TMP@EnsureCode{33}{12}% !
\TMP@EnsureCode{34}{12}% "
\TMP@EnsureCode{36}{3}% $
\TMP@EnsureCode{38}{4}% &
\TMP@EnsureCode{39}{12}% '
\TMP@EnsureCode{40}{12}% (
\TMP@EnsureCode{41}{12}% )
\TMP@EnsureCode{42}{12}% *
\TMP@EnsureCode{43}{12}% +
\TMP@EnsureCode{44}{12}% ,
\TMP@EnsureCode{45}{12}% -
\TMP@EnsureCode{46}{12}% .
\TMP@EnsureCode{47}{12}% /
\TMP@EnsureCode{58}{12}% :
\TMP@EnsureCode{59}{12}% ;
\TMP@EnsureCode{60}{12}% <
\TMP@EnsureCode{62}{12}% >
\TMP@EnsureCode{63}{12}% ?
\TMP@EnsureCode{91}{12}% [
\TMP@EnsureCode{93}{12}% ]
\TMP@EnsureCode{94}{7}% ^ (superscript)
\TMP@EnsureCode{95}{8}% _ (subscript)
\TMP@EnsureCode{96}{12}% `
\TMP@EnsureCode{124}{12}% |
\edef\HOLOGO@AtEnd{%
  \HOLOGO@AtEnd
  \escapechar\the\escapechar\relax
  \noexpand\endinput
}
\escapechar=92 %
%    \end{macrocode}
%
% \subsection{Logo list}
%
%    \begin{macro}{\hologoList}
%    \begin{macrocode}
\def\hologoList{%
  \hologoEntry{(La)TeX}{}{2011/10/01}%
  \hologoEntry{AmSLaTeX}{}{2010/04/16}%
  \hologoEntry{AmSTeX}{}{2010/04/16}%
  \hologoEntry{biber}{}{2011/10/01}%
  \hologoEntry{BibTeX}{}{2011/10/01}%
  \hologoEntry{BibTeX}{sf}{2011/10/01}%
  \hologoEntry{BibTeX}{sc}{2011/10/01}%
  \hologoEntry{BibTeX8}{}{2011/11/22}%
  \hologoEntry{ConTeXt}{}{2011/03/25}%
  \hologoEntry{ConTeXt}{narrow}{2011/03/25}%
  \hologoEntry{ConTeXt}{simple}{2011/03/25}%
  \hologoEntry{emTeX}{}{2010/04/26}%
  \hologoEntry{eTeX}{}{2010/04/08}%
  \hologoEntry{ExTeX}{}{2011/10/01}%
  \hologoEntry{HanTheThanh}{}{2011/11/29}%
  \hologoEntry{iniTeX}{}{2011/10/01}%
  \hologoEntry{KOMAScript}{}{2011/10/01}%
  \hologoEntry{La}{}{2010/05/08}%
  \hologoEntry{LaTeX}{}{2010/04/08}%
  \hologoEntry{LaTeX2e}{}{2010/04/08}%
  \hologoEntry{LaTeX3}{}{2010/04/24}%
  \hologoEntry{LaTeXe}{}{2010/04/08}%
  \hologoEntry{LaTeXML}{}{2011/11/22}%
  \hologoEntry{LaTeXTeX}{}{2011/10/01}%
  \hologoEntry{LuaLaTeX}{}{2010/04/08}%
  \hologoEntry{LuaTeX}{}{2010/04/08}%
  \hologoEntry{LyX}{}{2011/10/01}%
  \hologoEntry{METAFONT}{}{2011/10/01}%
  \hologoEntry{MetaFun}{}{2011/10/01}%
  \hologoEntry{METAPOST}{}{2011/10/01}%
  \hologoEntry{MetaPost}{}{2011/10/01}%
  \hologoEntry{MiKTeX}{}{2011/10/01}%
  \hologoEntry{NTS}{}{2011/10/01}%
  \hologoEntry{OzMF}{}{2011/10/01}%
  \hologoEntry{OzMP}{}{2011/10/01}%
  \hologoEntry{OzTeX}{}{2011/10/01}%
  \hologoEntry{OzTtH}{}{2011/10/01}%
  \hologoEntry{PCTeX}{}{2011/10/01}%
  \hologoEntry{pdfTeX}{}{2011/10/01}%
  \hologoEntry{pdfLaTeX}{}{2011/10/01}%
  \hologoEntry{PiC}{}{2011/10/01}%
  \hologoEntry{PiCTeX}{}{2011/10/01}%
  \hologoEntry{plainTeX}{}{2010/04/08}%
  \hologoEntry{plainTeX}{space}{2010/04/16}%
  \hologoEntry{plainTeX}{hyphen}{2010/04/16}%
  \hologoEntry{plainTeX}{runtogether}{2010/04/16}%
  \hologoEntry{SageTeX}{}{2011/11/22}%
  \hologoEntry{SLiTeX}{}{2011/10/01}%
  \hologoEntry{SLiTeX}{lift}{2011/10/01}%
  \hologoEntry{SLiTeX}{narrow}{2011/10/01}%
  \hologoEntry{SLiTeX}{simple}{2011/10/01}%
  \hologoEntry{SliTeX}{}{2011/10/01}%
  \hologoEntry{SliTeX}{narrow}{2011/10/01}%
  \hologoEntry{SliTeX}{simple}{2011/10/01}%
  \hologoEntry{SliTeX}{lift}{2011/10/01}%
  \hologoEntry{teTeX}{}{2011/10/01}%
  \hologoEntry{TeX}{}{2010/04/08}%
  \hologoEntry{TeX4ht}{}{2011/11/22}%
  \hologoEntry{TTH}{}{2011/11/22}%
  \hologoEntry{virTeX}{}{2011/10/01}%
  \hologoEntry{VTeX}{}{2010/04/24}%
  \hologoEntry{Xe}{}{2010/04/08}%
  \hologoEntry{XeLaTeX}{}{2010/04/08}%
  \hologoEntry{XeTeX}{}{2010/04/08}%
}
%    \end{macrocode}
%    \end{macro}
%
% \subsection{Load resources}
%
%    \begin{macrocode}
\begingroup\expandafter\expandafter\expandafter\endgroup
\expandafter\ifx\csname RequirePackage\endcsname\relax
  \def\TMP@RequirePackage#1[#2]{%
    \begingroup\expandafter\expandafter\expandafter\endgroup
    \expandafter\ifx\csname ver@#1.sty\endcsname\relax
      \input #1.sty\relax
    \fi
  }%
  \TMP@RequirePackage{ltxcmds}[2011/02/04]%
  \TMP@RequirePackage{infwarerr}[2010/04/08]%
  \TMP@RequirePackage{kvsetkeys}[2010/03/01]%
  \TMP@RequirePackage{kvdefinekeys}[2010/03/01]%
  \TMP@RequirePackage{pdftexcmds}[2010/04/01]%
  \TMP@RequirePackage{ifpdf}[2010/01/28]%
  \TMP@RequirePackage{ifluatex}[2010/03/01]%
  \ltx@IfUndefined{newif}{%
    \expandafter\let\csname newif\endcsname\ltx@newif
  }{}%
  \TMP@RequirePackage{ifxetex}[2009/01/23]%
  \TMP@RequirePackage{ifvtex}[2010/03/01]%
\else
  \RequirePackage{ltxcmds}[2011/02/04]%
  \RequirePackage{infwarerr}[2010/04/08]%
  \RequirePackage{kvsetkeys}[2010/03/01]%
  \RequirePackage{kvdefinekeys}[2010/03/01]%
  \RequirePackage{pdftexcmds}[2010/04/01]%
  \RequirePackage{ifpdf}[2010/01/28]%
  \RequirePackage{ifluatex}[2010/03/01]%
  \RequirePackage{ifxetex}[2009/01/23]%
  \RequirePackage{ifvtex}[2010/03/01]%
\fi
%    \end{macrocode}
%
%    \begin{macro}{\HOLOGO@IfDefined}
%    \begin{macrocode}
\def\HOLOGO@IfExists#1{%
  \ifx\@undefined#1%
    \expandafter\ltx@secondoftwo
  \else
    \ifx\relax#1%
      \expandafter\ltx@secondoftwo
    \else
      \expandafter\expandafter\expandafter\ltx@firstoftwo
    \fi
  \fi
}
%    \end{macrocode}
%    \end{macro}
%
% \subsection{Setup macros}
%
%    \begin{macro}{\hologoSetup}
%    \begin{macrocode}
\def\hologoSetup{%
  \let\HOLOGO@name\relax
  \HOLOGO@Setup
}
%    \end{macrocode}
%    \end{macro}
%
%    \begin{macro}{\hologoLogoSetup}
%    \begin{macrocode}
\def\hologoLogoSetup#1{%
  \edef\HOLOGO@name{#1}%
  \ltx@IfUndefined{HoLogo@\HOLOGO@name}{%
    \@PackageError{hologo}{%
      Unknown logo `\HOLOGO@name'%
    }\@ehc
    \ltx@gobble
  }{%
    \HOLOGO@Setup
  }%
}
%    \end{macrocode}
%    \end{macro}
%
%    \begin{macro}{\HOLOGO@Setup}
%    \begin{macrocode}
\def\HOLOGO@Setup{%
  \kvsetkeys{HoLogo}%
}
%    \end{macrocode}
%    \end{macro}
%
% \subsection{Options}
%
%    \begin{macro}{\HOLOGO@DeclareBoolOption}
%    \begin{macrocode}
\def\HOLOGO@DeclareBoolOption#1{%
  \expandafter\chardef\csname HOLOGOOPT@#1\endcsname\ltx@zero
  \kv@define@key{HoLogo}{#1}[true]{%
    \def\HOLOGO@temp{##1}%
    \ifx\HOLOGO@temp\HOLOGO@true
      \ifx\HOLOGO@name\relax
        \expandafter\chardef\csname HOLOGOOPT@#1\endcsname=\ltx@one
      \else
        \expandafter\chardef\csname
        HoLogoOpt@#1@\HOLOGO@name\endcsname\ltx@one
      \fi
      \HOLOGO@SetBreakAll{#1}%
    \else
      \ifx\HOLOGO@temp\HOLOGO@false
        \ifx\HOLOGO@name\relax
          \expandafter\chardef\csname HOLOGOOPT@#1\endcsname=\ltx@zero
        \else
          \expandafter\chardef\csname
          HoLogoOpt@#1@\HOLOGO@name\endcsname=\ltx@zero
        \fi
        \HOLOGO@SetBreakAll{#1}%
      \else
        \@PackageError{hologo}{%
          Unknown value `##1' for boolean option `#1'.\MessageBreak
          Known values are `true' and `false'%
        }\@ehc
      \fi
    \fi
  }%
}
%    \end{macrocode}
%    \end{macro}
%
%    \begin{macro}{\HOLOGO@SetBreakAll}
%    \begin{macrocode}
\def\HOLOGO@SetBreakAll#1{%
  \def\HOLOGO@temp{#1}%
  \ifx\HOLOGO@temp\HOLOGO@break
    \ifx\HOLOGO@name\relax
      \chardef\HOLOGOOPT@hyphenbreak=\HOLOGOOPT@break
      \chardef\HOLOGOOPT@spacebreak=\HOLOGOOPT@break
      \chardef\HOLOGOOPT@discretionarybreak=\HOLOGOOPT@break
    \else
      \expandafter\chardef
         \csname HoLogoOpt@hyphenbreak@\HOLOGO@name\endcsname=%
         \csname HoLogoOpt@break@\HOLOGO@name\endcsname
      \expandafter\chardef
         \csname HoLogoOpt@spacebreak@\HOLOGO@name\endcsname=%
         \csname HoLogoOpt@break@\HOLOGO@name\endcsname
      \expandafter\chardef
         \csname HoLogoOpt@discretionarybreak@\HOLOGO@name
             \endcsname=%
         \csname HoLogoOpt@break@\HOLOGO@name\endcsname
    \fi
  \fi
}
%    \end{macrocode}
%    \end{macro}
%
%    \begin{macro}{\HOLOGO@true}
%    \begin{macrocode}
\def\HOLOGO@true{true}
%    \end{macrocode}
%    \end{macro}
%    \begin{macro}{\HOLOGO@false}
%    \begin{macrocode}
\def\HOLOGO@false{false}
%    \end{macrocode}
%    \end{macro}
%    \begin{macro}{\HOLOGO@break}
%    \begin{macrocode}
\def\HOLOGO@break{break}
%    \end{macrocode}
%    \end{macro}
%
%    \begin{macrocode}
\HOLOGO@DeclareBoolOption{break}
\HOLOGO@DeclareBoolOption{hyphenbreak}
\HOLOGO@DeclareBoolOption{spacebreak}
\HOLOGO@DeclareBoolOption{discretionarybreak}
%    \end{macrocode}
%
%    \begin{macrocode}
\kv@define@key{HoLogo}{variant}{%
  \ifx\HOLOGO@name\relax
    \@PackageError{hologo}{%
      Option `variant' is not available in \string\hologoSetup,%
      \MessageBreak
      Use \string\hologoLogoSetup\space instead%
    }\@ehc
  \else
    \edef\HOLOGO@temp{#1}%
    \ifx\HOLOGO@temp\ltx@empty
      \expandafter
      \let\csname HoLogoOpt@variant@\HOLOGO@name\endcsname\@undefined
    \else
      \ltx@IfUndefined{HoLogo@\HOLOGO@name @\HOLOGO@temp}{%
        \@PackageError{hologo}{%
          Unknown variant `\HOLOGO@temp' of logo `\HOLOGO@name'%
        }\@ehc
      }{%
        \expandafter
        \let\csname HoLogoOpt@variant@\HOLOGO@name\endcsname
            \HOLOGO@temp
      }%
    \fi
  \fi
}
%    \end{macrocode}
%
%    \begin{macro}{\HOLOGO@Variant}
%    \begin{macrocode}
\def\HOLOGO@Variant#1{%
  #1%
  \ltx@ifundefined{HoLogoOpt@variant@#1}{%
  }{%
    @\csname HoLogoOpt@variant@#1\endcsname
  }%
}
%    \end{macrocode}
%    \end{macro}
%
% \subsection{Break/no-break support}
%
%    \begin{macro}{\HOLOGO@space}
%    \begin{macrocode}
\def\HOLOGO@space{%
  \ltx@ifundefined{HoLogoOpt@spacebreak@\HOLOGO@name}{%
    \ltx@ifundefined{HoLogoOpt@break@\HOLOGO@name}{%
      \chardef\HOLOGO@temp=\HOLOGOOPT@spacebreak
    }{%
      \chardef\HOLOGO@temp=%
        \csname HoLogoOpt@break@\HOLOGO@name\endcsname
    }%
  }{%
    \chardef\HOLOGO@temp=%
      \csname HoLogoOpt@spacebreak@\HOLOGO@name\endcsname
  }%
  \ifcase\HOLOGO@temp
    \penalty10000 %
  \fi
  \ltx@space
}
%    \end{macrocode}
%    \end{macro}
%
%    \begin{macro}{\HOLOGO@hyphen}
%    \begin{macrocode}
\def\HOLOGO@hyphen{%
  \ltx@ifundefined{HoLogoOpt@hyphenbreak@\HOLOGO@name}{%
    \ltx@ifundefined{HoLogoOpt@break@\HOLOGO@name}{%
      \chardef\HOLOGO@temp=\HOLOGOOPT@hyphenbreak
    }{%
      \chardef\HOLOGO@temp=%
        \csname HoLogoOpt@break@\HOLOGO@name\endcsname
    }%
  }{%
    \chardef\HOLOGO@temp=%
      \csname HoLogoOpt@hyphenbreak@\HOLOGO@name\endcsname
  }%
  \ifcase\HOLOGO@temp
    \ltx@mbox{-}%
  \else
    -%
  \fi
}
%    \end{macrocode}
%    \end{macro}
%
%    \begin{macro}{\HOLOGO@discretionary}
%    \begin{macrocode}
\def\HOLOGO@discretionary{%
  \ltx@ifundefined{HoLogoOpt@discretionarybreak@\HOLOGO@name}{%
    \ltx@ifundefined{HoLogoOpt@break@\HOLOGO@name}{%
      \chardef\HOLOGO@temp=\HOLOGOOPT@discretionarybreak
    }{%
      \chardef\HOLOGO@temp=%
        \csname HoLogoOpt@break@\HOLOGO@name\endcsname
    }%
  }{%
    \chardef\HOLOGO@temp=%
      \csname HoLogoOpt@discretionarybreak@\HOLOGO@name\endcsname
  }%
  \ifcase\HOLOGO@temp
  \else
    \-%
  \fi
}
%    \end{macrocode}
%    \end{macro}
%
%    \begin{macro}{\HOLOGO@mbox}
%    \begin{macrocode}
\def\HOLOGO@mbox#1{%
  \ltx@ifundefined{HoLogoOpt@break@\HOLOGO@name}{%
    \chardef\HOLOGO@temp=\HOLOGOOPT@hyphenbreak
  }{%
    \chardef\HOLOGO@temp=%
      \csname HoLogoOpt@break@\HOLOGO@name\endcsname
  }%
  \ifcase\HOLOGO@temp
    \ltx@mbox{#1}%
  \else
    #1%
  \fi
}
%    \end{macrocode}
%    \end{macro}
%
% \subsection{Font support}
%
%    \begin{macro}{\HoLogoFont@font}
%    \begin{tabular}{@{}ll@{}}
%    |#1|:& logo name\\
%    |#2|:& font short name\\
%    |#3|:& text
%    \end{tabular}
%    \begin{macrocode}
\def\HoLogoFont@font#1#2#3{%
  \begingroup
    \ltx@IfUndefined{HoLogoFont@logo@#1.#2}{%
      \ltx@IfUndefined{HoLogoFont@font@#2}{%
        \@PackageWarning{hologo}{%
          Missing font `#2' for logo `#1'%
        }%
        #3%
      }{%
        \csname HoLogoFont@font@#2\endcsname{#3}%
      }%
    }{%
      \csname HoLogoFont@logo@#1.#2\endcsname{#3}%
    }%
  \endgroup
}
%    \end{macrocode}
%    \end{macro}
%
%    \begin{macro}{\HoLogoFont@Def}
%    \begin{macrocode}
\def\HoLogoFont@Def#1{%
  \expandafter\def\csname HoLogoFont@font@#1\endcsname
}
%    \end{macrocode}
%    \end{macro}
%    \begin{macro}{\HoLogoFont@LogoDef}
%    \begin{macrocode}
\def\HoLogoFont@LogoDef#1#2{%
  \expandafter\def\csname HoLogoFont@logo@#1.#2\endcsname
}
%    \end{macrocode}
%    \end{macro}
%
% \subsubsection{Font defaults}
%
%    \begin{macro}{\HoLogoFont@font@general}
%    \begin{macrocode}
\HoLogoFont@Def{general}{}%
%    \end{macrocode}
%    \end{macro}
%
%    \begin{macro}{\HoLogoFont@font@rm}
%    \begin{macrocode}
\ltx@IfUndefined{rmfamily}{%
  \ltx@IfUndefined{rm}{%
  }{%
    \HoLogoFont@Def{rm}{\rm}%
  }%
}{%
  \HoLogoFont@Def{rm}{\rmfamily}%
}
%    \end{macrocode}
%    \end{macro}
%
%    \begin{macro}{\HoLogoFont@font@sf}
%    \begin{macrocode}
\ltx@IfUndefined{sffamily}{%
  \ltx@IfUndefined{sf}{%
  }{%
    \HoLogoFont@Def{sf}{\sf}%
  }%
}{%
  \HoLogoFont@Def{sf}{\sffamily}%
}
%    \end{macrocode}
%    \end{macro}
%
%    \begin{macro}{\HoLogoFont@font@bibsf}
%    In case of \hologo{plainTeX} the original small caps
%    variant is used as default. In \hologo{LaTeX}
%    the definition of package \xpackage{dtklogos} \cite{dtklogos}
%    is used.
%\begin{quote}
%\begin{verbatim}
%\DeclareRobustCommand{\BibTeX}{%
%  B%
%  \kern-.05em%
%  \hbox{%
%    $\m@th$% %% force math size calculations
%    \csname S@\f@size\endcsname
%    \fontsize\sf@size\z@
%    \math@fontsfalse
%    \selectfont
%    I%
%    \kern-.025em%
%    B
%  }%
%  \kern-.08em%
%  \-%
%  \TeX
%}
%\end{verbatim}
%\end{quote}
%    \begin{macrocode}
\ltx@IfUndefined{selectfont}{%
  \ltx@IfUndefined{tensc}{%
    \font\tensc=cmcsc10\relax
  }{}%
  \HoLogoFont@Def{bibsf}{\tensc}%
}{%
  \HoLogoFont@Def{bibsf}{%
    $\mathsurround=0pt$%
    \csname S@\f@size\endcsname
    \fontsize\sf@size{0pt}%
    \math@fontsfalse
    \selectfont
  }%
}
%    \end{macrocode}
%    \end{macro}
%
%    \begin{macro}{\HoLogoFont@font@sc}
%    \begin{macrocode}
\ltx@IfUndefined{scshape}{%
  \ltx@IfUndefined{tensc}{%
    \font\tensc=cmcsc10\relax
  }{}%
  \HoLogoFont@Def{sc}{\tensc}%
}{%
  \HoLogoFont@Def{sc}{\scshape}%
}
%    \end{macrocode}
%    \end{macro}
%
%    \begin{macro}{\HoLogoFont@font@sy}
%    \begin{macrocode}
\ltx@IfUndefined{usefont}{%
  \ltx@IfUndefined{tensy}{%
  }{%
    \HoLogoFont@Def{sy}{\tensy}%
  }%
}{%
  \HoLogoFont@Def{sy}{%
    \usefont{OMS}{cmsy}{m}{n}%
  }%
}
%    \end{macrocode}
%    \end{macro}
%
%    \begin{macro}{\HoLogoFont@font@logo}
%    \begin{macrocode}
\begingroup
  \def\x{LaTeX2e}%
\expandafter\endgroup
\ifx\fmtname\x
  \ltx@IfUndefined{logofamily}{%
    \DeclareRobustCommand\logofamily{%
      \not@math@alphabet\logofamily\relax
      \fontencoding{U}%
      \fontfamily{logo}%
      \selectfont
    }%
  }{}%
  \ltx@IfUndefined{logofamily}{%
  }{%
    \HoLogoFont@Def{logo}{\logofamily}%
  }%
\else
  \ltx@IfUndefined{tenlogo}{%
    \font\tenlogo=logo10\relax
  }{}%
  \HoLogoFont@Def{logo}{\tenlogo}%
\fi
%    \end{macrocode}
%    \end{macro}
%
% \subsubsection{Font setup}
%
%    \begin{macro}{\hologoFontSetup}
%    \begin{macrocode}
\def\hologoFontSetup{%
  \let\HOLOGO@name\relax
  \HOLOGO@FontSetup
}
%    \end{macrocode}
%    \end{macro}
%
%    \begin{macro}{\hologoLogoFontSetup}
%    \begin{macrocode}
\def\hologoLogoFontSetup#1{%
  \edef\HOLOGO@name{#1}%
  \ltx@IfUndefined{HoLogo@\HOLOGO@name}{%
    \@PackageError{hologo}{%
      Unknown logo `\HOLOGO@name'%
    }\@ehc
    \ltx@gobble
  }{%
    \HOLOGO@FontSetup
  }%
}
%    \end{macrocode}
%    \end{macro}
%
%    \begin{macro}{\HOLOGO@FontSetup}
%    \begin{macrocode}
\def\HOLOGO@FontSetup{%
  \kvsetkeys{HoLogoFont}%
}
%    \end{macrocode}
%    \end{macro}
%
%    \begin{macrocode}
\def\HOLOGO@temp#1{%
  \kv@define@key{HoLogoFont}{#1}{%
    \ifx\HOLOGO@name\relax
      \HoLogoFont@Def{#1}{##1}%
    \else
      \HoLogoFont@LogoDef\HOLOGO@name{#1}{##1}%
    \fi
  }%
}
\HOLOGO@temp{general}
\HOLOGO@temp{sf}
%    \end{macrocode}
%
% \subsection{Generic logo commands}
%
%    \begin{macrocode}
\HOLOGO@IfExists\hologo{%
  \@PackageError{hologo}{%
    \string\hologo\ltx@space is already defined.\MessageBreak
    Package loading is aborted%
  }\@ehc
  \HOLOGO@AtEnd
}%
\HOLOGO@IfExists\hologoRobust{%
  \@PackageError{hologo}{%
    \string\hologoRobust\ltx@space is already defined.\MessageBreak
    Package loading is aborted%
  }\@ehc
  \HOLOGO@AtEnd
}%
%    \end{macrocode}
%
% \subsubsection{\cs{hologo} and friends}
%
%    \begin{macrocode}
\ifluatex
  \expandafter\ltx@firstofone
\else
  \expandafter\ltx@gobble
\fi
{%
  \ltx@IfUndefined{ifincsname}{%
    \ifnum\luatexversion<36 %
      \expandafter\ltx@gobble
    \else
      \expandafter\ltx@firstofone
    \fi
    {%
      \begingroup
        \ifcase0%
            \directlua{%
              if tex.enableprimitives then %
                tex.enableprimitives('HOLOGO@', {'ifincsname'})%
              else %
                tex.print('1')%
              end%
            }%
            \ifx\HOLOGO@ifincsname\@undefined 1\fi%
            \relax
          \expandafter\ltx@firstofone
        \else
          \endgroup
          \expandafter\ltx@gobble
        \fi
        {%
          \global\let\ifincsname\HOLOGO@ifincsname
        }%
      \HOLOGO@temp
    }%
  }{}%
}
%    \end{macrocode}
%    \begin{macrocode}
\ltx@IfUndefined{ifincsname}{%
  \catcode`$=14 %
}{%
  \catcode`$=9 %
}
%    \end{macrocode}
%
%    \begin{macro}{\hologo}
%    \begin{macrocode}
\def\hologo#1{%
$ \ifincsname
$   \ltx@ifundefined{HoLogoCs@\HOLOGO@Variant{#1}}{%
$     #1%
$   }{%
$     \csname HoLogoCs@\HOLOGO@Variant{#1}\endcsname\ltx@firstoftwo
$   }%
$ \else
    \HOLOGO@IfExists\texorpdfstring\texorpdfstring\ltx@firstoftwo
    {%
      \hologoRobust{#1}%
    }{%
      \ltx@ifundefined{HoLogoBkm@\HOLOGO@Variant{#1}}{%
        \ltx@ifundefined{HoLogo@#1}{?#1?}{#1}%
      }{%
        \csname HoLogoBkm@\HOLOGO@Variant{#1}\endcsname
        \ltx@firstoftwo
      }%
    }%
$ \fi
}
%    \end{macrocode}
%    \end{macro}
%    \begin{macro}{\Hologo}
%    \begin{macrocode}
\def\Hologo#1{%
$ \ifincsname
$   \ltx@ifundefined{HoLogoCs@\HOLOGO@Variant{#1}}{%
$     #1%
$   }{%
$     \csname HoLogoCs@\HOLOGO@Variant{#1}\endcsname\ltx@secondoftwo
$   }%
$ \else
    \HOLOGO@IfExists\texorpdfstring\texorpdfstring\ltx@firstoftwo
    {%
      \HologoRobust{#1}%
    }{%
      \ltx@ifundefined{HoLogoBkm@\HOLOGO@Variant{#1}}{%
        \ltx@ifundefined{HoLogo@#1}{?#1?}{#1}%
      }{%
        \csname HoLogoBkm@\HOLOGO@Variant{#1}\endcsname
        \ltx@secondoftwo
      }%
    }%
$ \fi
}
%    \end{macrocode}
%    \end{macro}
%
%    \begin{macro}{\hologoVariant}
%    \begin{macrocode}
\def\hologoVariant#1#2{%
  \ifx\relax#2\relax
    \hologo{#1}%
  \else
$   \ifincsname
$     \ltx@ifundefined{HoLogoCs@#1@#2}{%
$       #1%
$     }{%
$       \csname HoLogoCs@#1@#2\endcsname\ltx@firstoftwo
$     }%
$   \else
      \HOLOGO@IfExists\texorpdfstring\texorpdfstring\ltx@firstoftwo
      {%
        \hologoVariantRobust{#1}{#2}%
      }{%
        \ltx@ifundefined{HoLogoBkm@#1@#2}{%
          \ltx@ifundefined{HoLogo@#1}{?#1?}{#1}%
        }{%
          \csname HoLogoBkm@#1@#2\endcsname
          \ltx@firstoftwo
        }%
      }%
$   \fi
  \fi
}
%    \end{macrocode}
%    \end{macro}
%    \begin{macro}{\HologoVariant}
%    \begin{macrocode}
\def\HologoVariant#1#2{%
  \ifx\relax#2\relax
    \Hologo{#1}%
  \else
$   \ifincsname
$     \ltx@ifundefined{HoLogoCs@#1@#2}{%
$       #1%
$     }{%
$       \csname HoLogoCs@#1@#2\endcsname\ltx@secondoftwo
$     }%
$   \else
      \HOLOGO@IfExists\texorpdfstring\texorpdfstring\ltx@firstoftwo
      {%
        \HologoVariantRobust{#1}{#2}%
      }{%
        \ltx@ifundefined{HoLogoBkm@#1@#2}{%
          \ltx@ifundefined{HoLogo@#1}{?#1?}{#1}%
        }{%
          \csname HoLogoBkm@#1@#2\endcsname
          \ltx@secondoftwo
        }%
      }%
$   \fi
  \fi
}
%    \end{macrocode}
%    \end{macro}
%
%    \begin{macrocode}
\catcode`\$=3 %
%    \end{macrocode}
%
% \subsubsection{\cs{hologoRobust} and friends}
%
%    \begin{macro}{\hologoRobust}
%    \begin{macrocode}
\ltx@IfUndefined{protected}{%
  \ltx@IfUndefined{DeclareRobustCommand}{%
    \def\hologoRobust#1%
  }{%
    \DeclareRobustCommand*\hologoRobust[1]%
  }%
}{%
  \protected\def\hologoRobust#1%
}%
{%
  \edef\HOLOGO@name{#1}%
  \ltx@IfUndefined{HoLogo@\HOLOGO@Variant\HOLOGO@name}{%
    \@PackageError{hologo}{%
      Unknown logo `\HOLOGO@name'%
    }\@ehc
    ?\HOLOGO@name?%
  }{%
    \ltx@IfUndefined{ver@tex4ht.sty}{%
      \HoLogoFont@font\HOLOGO@name{general}{%
        \csname HoLogo@\HOLOGO@Variant\HOLOGO@name\endcsname
        \ltx@firstoftwo
      }%
    }{%
      \ltx@IfUndefined{HoLogoHtml@\HOLOGO@Variant\HOLOGO@name}{%
        \HOLOGO@name
      }{%
        \csname HoLogoHtml@\HOLOGO@Variant\HOLOGO@name\endcsname
        \ltx@firstoftwo
      }%
    }%
  }%
}
%    \end{macrocode}
%    \end{macro}
%    \begin{macro}{\HologoRobust}
%    \begin{macrocode}
\ltx@IfUndefined{protected}{%
  \ltx@IfUndefined{DeclareRobustCommand}{%
    \def\HologoRobust#1%
  }{%
    \DeclareRobustCommand*\HologoRobust[1]%
  }%
}{%
  \protected\def\HologoRobust#1%
}%
{%
  \edef\HOLOGO@name{#1}%
  \ltx@IfUndefined{HoLogo@\HOLOGO@Variant\HOLOGO@name}{%
    \@PackageError{hologo}{%
      Unknown logo `\HOLOGO@name'%
    }\@ehc
    ?\HOLOGO@name?%
  }{%
    \ltx@IfUndefined{ver@tex4ht.sty}{%
      \HoLogoFont@font\HOLOGO@name{general}{%
        \csname HoLogo@\HOLOGO@Variant\HOLOGO@name\endcsname
        \ltx@secondoftwo
      }%
    }{%
      \ltx@IfUndefined{HoLogoHtml@\HOLOGO@Variant\HOLOGO@name}{%
        \expandafter\HOLOGO@Uppercase\HOLOGO@name
      }{%
        \csname HoLogoHtml@\HOLOGO@Variant\HOLOGO@name\endcsname
        \ltx@secondoftwo
      }%
    }%
  }%
}
%    \end{macrocode}
%    \end{macro}
%    \begin{macro}{\hologoVariantRobust}
%    \begin{macrocode}
\ltx@IfUndefined{protected}{%
  \ltx@IfUndefined{DeclareRobustCommand}{%
    \def\hologoVariantRobust#1#2%
  }{%
    \DeclareRobustCommand*\hologoVariantRobust[2]%
  }%
}{%
  \protected\def\hologoVariantRobust#1#2%
}%
{%
  \begingroup
    \hologoLogoSetup{#1}{variant={#2}}%
    \hologoRobust{#1}%
  \endgroup
}
%    \end{macrocode}
%    \end{macro}
%    \begin{macro}{\HologoVariantRobust}
%    \begin{macrocode}
\ltx@IfUndefined{protected}{%
  \ltx@IfUndefined{DeclareRobustCommand}{%
    \def\HologoVariantRobust#1#2%
  }{%
    \DeclareRobustCommand*\HologoVariantRobust[2]%
  }%
}{%
  \protected\def\HologoVariantRobust#1#2%
}%
{%
  \begingroup
    \hologoLogoSetup{#1}{variant={#2}}%
    \HologoRobust{#1}%
  \endgroup
}
%    \end{macrocode}
%    \end{macro}
%
%    \begin{macro}{\hologorobust}
%    Macro \cs{hologorobust} is only defined for compatibility.
%    Its use is deprecated.
%    \begin{macrocode}
\def\hologorobust{\hologoRobust}
%    \end{macrocode}
%    \end{macro}
%
% \subsection{Helpers}
%
%    \begin{macro}{\HOLOGO@Uppercase}
%    Macro \cs{HOLOGO@Uppercase} is restricted to \cs{uppercase},
%    because \hologo{plainTeX} or \hologo{iniTeX} do not provide
%    \cs{MakeUppercase}.
%    \begin{macrocode}
\def\HOLOGO@Uppercase#1{\uppercase{#1}}
%    \end{macrocode}
%    \end{macro}
%
%    \begin{macro}{\HOLOGO@PdfdocUnicode}
%    \begin{macrocode}
\def\HOLOGO@PdfdocUnicode{%
  \ifx\ifHy@unicode\iftrue
    \expandafter\ltx@secondoftwo
  \else
    \expandafter\ltx@firstoftwo
  \fi
}
%    \end{macrocode}
%    \end{macro}
%
%    \begin{macro}{\HOLOGO@Math}
%    \begin{macrocode}
\def\HOLOGO@MathSetup{%
  \mathsurround0pt\relax
  \HOLOGO@IfExists\f@series{%
    \if b\expandafter\ltx@car\f@series x\@nil
      \csname boldmath\endcsname
   \fi
  }{}%
}
%    \end{macrocode}
%    \end{macro}
%
%    \begin{macro}{\HOLOGO@TempDimen}
%    \begin{macrocode}
\dimendef\HOLOGO@TempDimen=\ltx@zero
%    \end{macrocode}
%    \end{macro}
%    \begin{macro}{\HOLOGO@NegativeKerning}
%    \begin{macrocode}
\def\HOLOGO@NegativeKerning#1{%
  \begingroup
    \HOLOGO@TempDimen=0pt\relax
    \comma@parse@normalized{#1}{%
      \ifdim\HOLOGO@TempDimen=0pt %
        \expandafter\HOLOGO@@NegativeKerning\comma@entry
      \fi
      \ltx@gobble
    }%
    \ifdim\HOLOGO@TempDimen<0pt %
      \kern\HOLOGO@TempDimen
    \fi
  \endgroup
}
%    \end{macrocode}
%    \end{macro}
%    \begin{macro}{\HOLOGO@@NegativeKerning}
%    \begin{macrocode}
\def\HOLOGO@@NegativeKerning#1#2{%
  \setbox\ltx@zero\hbox{#1#2}%
  \HOLOGO@TempDimen=\wd\ltx@zero
  \setbox\ltx@zero\hbox{#1\kern0pt#2}%
  \advance\HOLOGO@TempDimen by -\wd\ltx@zero
}
%    \end{macrocode}
%    \end{macro}
%
%    \begin{macro}{\HOLOGO@SpaceFactor}
%    \begin{macrocode}
\def\HOLOGO@SpaceFactor{%
  \spacefactor1000 %
}
%    \end{macrocode}
%    \end{macro}
%
%    \begin{macro}{\HOLOGO@Span}
%    \begin{macrocode}
\def\HOLOGO@Span#1#2{%
  \HCode{<span class="HoLogo-#1">}%
  #2%
  \HCode{</span>}%
}
%    \end{macrocode}
%    \end{macro}
%
% \subsubsection{Text subscript}
%
%    \begin{macro}{\HOLOGO@SubScript}%
%    \begin{macrocode}
\def\HOLOGO@SubScript#1{%
  \ltx@IfUndefined{textsubscript}{%
    \ltx@IfUndefined{text}{%
      \ltx@mbox{%
        \mathsurround=0pt\relax
        $%
          _{%
            \ltx@IfUndefined{sf@size}{%
              \mathrm{#1}%
            }{%
              \mbox{%
                \fontsize\sf@size{0pt}\selectfont
                #1%
              }%
            }%
          }%
        $%
      }%
    }{%
      \ltx@mbox{%
        \mathsurround=0pt\relax
        $_{\text{#1}}$%
      }%
    }%
  }{%
    \textsubscript{#1}%
  }%
}
%    \end{macrocode}
%    \end{macro}
%
% \subsection{\hologo{TeX} and friends}
%
% \subsubsection{\hologo{TeX}}
%
%    \begin{macro}{\HoLogo@TeX}
%    Source: \hologo{LaTeX} kernel.
%    \begin{macrocode}
\def\HoLogo@TeX#1{%
  T\kern-.1667em\lower.5ex\hbox{E}\kern-.125emX\HOLOGO@SpaceFactor
}
%    \end{macrocode}
%    \end{macro}
%    \begin{macro}{\HoLogoHtml@TeX}
%    \begin{macrocode}
\def\HoLogoHtml@TeX#1{%
  \HoLogoCss@TeX
  \HOLOGO@Span{TeX}{%
    T%
    \HOLOGO@Span{e}{%
      E%
    }%
    X%
  }%
}
%    \end{macrocode}
%    \end{macro}
%    \begin{macro}{\HoLogoCss@TeX}
%    \begin{macrocode}
\def\HoLogoCss@TeX{%
  \Css{%
    span.HoLogo-TeX span.HoLogo-e{%
      position:relative;%
      top:.5ex;%
      margin-left:-.1667em;%
      margin-right:-.125em;%
    }%
  }%
  \Css{%
    a span.HoLogo-TeX span.HoLogo-e{%
      text-decoration:none;%
    }%
  }%
  \global\let\HoLogoCss@TeX\relax
}
%    \end{macrocode}
%    \end{macro}
%
% \subsubsection{\hologo{plainTeX}}
%
%    \begin{macro}{\HoLogo@plainTeX@space}
%    Source: ``The \hologo{TeX}book''
%    \begin{macrocode}
\def\HoLogo@plainTeX@space#1{%
  \HOLOGO@mbox{#1{p}{P}lain}\HOLOGO@space\hologo{TeX}%
}
%    \end{macrocode}
%    \end{macro}
%    \begin{macro}{\HoLogoCs@plainTeX@space}
%    \begin{macrocode}
\def\HoLogoCs@plainTeX@space#1{#1{p}{P}lain TeX}%
%    \end{macrocode}
%    \end{macro}
%    \begin{macro}{\HoLogoBkm@plainTeX@space}
%    \begin{macrocode}
\def\HoLogoBkm@plainTeX@space#1{%
  #1{p}{P}lain \hologo{TeX}%
}
%    \end{macrocode}
%    \end{macro}
%    \begin{macro}{\HoLogoHtml@plainTeX@space}
%    \begin{macrocode}
\def\HoLogoHtml@plainTeX@space#1{%
  #1{p}{P}lain \hologo{TeX}%
}
%    \end{macrocode}
%    \end{macro}
%
%    \begin{macro}{\HoLogo@plainTeX@hyphen}
%    \begin{macrocode}
\def\HoLogo@plainTeX@hyphen#1{%
  \HOLOGO@mbox{#1{p}{P}lain}\HOLOGO@hyphen\hologo{TeX}%
}
%    \end{macrocode}
%    \end{macro}
%    \begin{macro}{\HoLogoCs@plainTeX@hyphen}
%    \begin{macrocode}
\def\HoLogoCs@plainTeX@hyphen#1{#1{p}{P}lain-TeX}
%    \end{macrocode}
%    \end{macro}
%    \begin{macro}{\HoLogoBkm@plainTeX@hyphen}
%    \begin{macrocode}
\def\HoLogoBkm@plainTeX@hyphen#1{%
  #1{p}{P}lain-\hologo{TeX}%
}
%    \end{macrocode}
%    \end{macro}
%    \begin{macro}{\HoLogoHtml@plainTeX@hyphen}
%    \begin{macrocode}
\def\HoLogoHtml@plainTeX@hyphen#1{%
  #1{p}{P}lain-\hologo{TeX}%
}
%    \end{macrocode}
%    \end{macro}
%
%    \begin{macro}{\HoLogo@plainTeX@runtogether}
%    \begin{macrocode}
\def\HoLogo@plainTeX@runtogether#1{%
  \HOLOGO@mbox{#1{p}{P}lain\hologo{TeX}}%
}
%    \end{macrocode}
%    \end{macro}
%    \begin{macro}{\HoLogoCs@plainTeX@runtogether}
%    \begin{macrocode}
\def\HoLogoCs@plainTeX@runtogether#1{#1{p}{P}lainTeX}
%    \end{macrocode}
%    \end{macro}
%    \begin{macro}{\HoLogoBkm@plainTeX@runtogether}
%    \begin{macrocode}
\def\HoLogoBkm@plainTeX@runtogether#1{%
  #1{p}{P}lain\hologo{TeX}%
}
%    \end{macrocode}
%    \end{macro}
%    \begin{macro}{\HoLogoHtml@plainTeX@runtogether}
%    \begin{macrocode}
\def\HoLogoHtml@plainTeX@runtogether#1{%
  #1{p}{P}lain\hologo{TeX}%
}
%    \end{macrocode}
%    \end{macro}
%
%    \begin{macro}{\HoLogo@plainTeX}
%    \begin{macrocode}
\def\HoLogo@plainTeX{\HoLogo@plainTeX@space}
%    \end{macrocode}
%    \end{macro}
%    \begin{macro}{\HoLogoCs@plainTeX}
%    \begin{macrocode}
\def\HoLogoCs@plainTeX{\HoLogoCs@plainTeX@space}
%    \end{macrocode}
%    \end{macro}
%    \begin{macro}{\HoLogoBkm@plainTeX}
%    \begin{macrocode}
\def\HoLogoBkm@plainTeX{\HoLogoBkm@plainTeX@space}
%    \end{macrocode}
%    \end{macro}
%    \begin{macro}{\HoLogoHtml@plainTeX}
%    \begin{macrocode}
\def\HoLogoHtml@plainTeX{\HoLogoHtml@plainTeX@space}
%    \end{macrocode}
%    \end{macro}
%
% \subsubsection{\hologo{LaTeX}}
%
%    Source: \hologo{LaTeX} kernel.
%\begin{quote}
%\begin{verbatim}
%\DeclareRobustCommand{\LaTeX}{%
%  L%
%  \kern-.36em%
%  {%
%    \sbox\z@ T%
%    \vbox to\ht\z@{%
%      \hbox{%
%        \check@mathfonts
%        \fontsize\sf@size\z@
%        \math@fontsfalse
%        \selectfont
%        A%
%      }%
%      \vss
%    }%
%  }%
%  \kern-.15em%
%  \TeX
%}
%\end{verbatim}
%\end{quote}
%
%    \begin{macro}{\HoLogo@La}
%    \begin{macrocode}
\def\HoLogo@La#1{%
  L%
  \kern-.36em%
  \begingroup
    \setbox\ltx@zero\hbox{T}%
    \vbox to\ht\ltx@zero{%
      \hbox{%
        \ltx@ifundefined{check@mathfonts}{%
          \csname sevenrm\endcsname
        }{%
          \check@mathfonts
          \fontsize\sf@size{0pt}%
          \math@fontsfalse\selectfont
        }%
        A%
      }%
      \vss
    }%
  \endgroup
}
%    \end{macrocode}
%    \end{macro}
%
%    \begin{macro}{\HoLogo@LaTeX}
%    Source: \hologo{LaTeX} kernel.
%    \begin{macrocode}
\def\HoLogo@LaTeX#1{%
  \hologo{La}%
  \kern-.15em%
  \hologo{TeX}%
}
%    \end{macrocode}
%    \end{macro}
%    \begin{macro}{\HoLogoHtml@LaTeX}
%    \begin{macrocode}
\def\HoLogoHtml@LaTeX#1{%
  \HoLogoCss@LaTeX
  \HOLOGO@Span{LaTeX}{%
    L%
    \HOLOGO@Span{a}{%
      A%
    }%
    \hologo{TeX}%
  }%
}
%    \end{macrocode}
%    \end{macro}
%    \begin{macro}{\HoLogoCss@LaTeX}
%    \begin{macrocode}
\def\HoLogoCss@LaTeX{%
  \Css{%
    span.HoLogo-LaTeX span.HoLogo-a{%
      position:relative;%
      top:-.5ex;%
      margin-left:-.36em;%
      margin-right:-.15em;%
      font-size:85\%;%
    }%
  }%
  \global\let\HoLogoCss@LaTeX\relax
}
%    \end{macrocode}
%    \end{macro}
%
% \subsubsection{\hologo{(La)TeX}}
%
%    \begin{macro}{\HoLogo@LaTeXTeX}
%    The kerning around the parentheses is taken
%    from package \xpackage{dtklogos} \cite{dtklogos}.
%\begin{quote}
%\begin{verbatim}
%\DeclareRobustCommand{\LaTeXTeX}{%
%  (%
%  \kern-.15em%
%  L%
%  \kern-.36em%
%  {%
%    \sbox\z@ T%
%    \vbox to\ht0{%
%      \hbox{%
%        $\m@th$%
%        \csname S@\f@size\endcsname
%        \fontsize\sf@size\z@
%        \math@fontsfalse
%        \selectfont
%        A%
%      }%
%      \vss
%    }%
%  }%
%  \kern-.2em%
%  )%
%  \kern-.15em%
%  \TeX
%}
%\end{verbatim}
%\end{quote}
%    \begin{macrocode}
\def\HoLogo@LaTeXTeX#1{%
  (%
  \kern-.15em%
  \hologo{La}%
  \kern-.2em%
  )%
  \kern-.15em%
  \hologo{TeX}%
}
%    \end{macrocode}
%    \end{macro}
%    \begin{macro}{\HoLogoBkm@LaTeXTeX}
%    \begin{macrocode}
\def\HoLogoBkm@LaTeXTeX#1{(La)TeX}
%    \end{macrocode}
%    \end{macro}
%
%    \begin{macro}{\HoLogo@(La)TeX}
%    \begin{macrocode}
\expandafter
\let\csname HoLogo@(La)TeX\endcsname\HoLogo@LaTeXTeX
%    \end{macrocode}
%    \end{macro}
%    \begin{macro}{\HoLogoBkm@(La)TeX}
%    \begin{macrocode}
\expandafter
\let\csname HoLogoBkm@(La)TeX\endcsname\HoLogoBkm@LaTeXTeX
%    \end{macrocode}
%    \end{macro}
%    \begin{macro}{\HoLogoHtml@LaTeXTeX}
%    \begin{macrocode}
\def\HoLogoHtml@LaTeXTeX#1{%
  \HoLogoCss@LaTeXTeX
  \HOLOGO@Span{LaTeXTeX}{%
    (%
    \HOLOGO@Span{L}{L}%
    \HOLOGO@Span{a}{A}%
    \HOLOGO@Span{ParenRight}{)}%
    \hologo{TeX}%
  }%
}
%    \end{macrocode}
%    \end{macro}
%    \begin{macro}{\HoLogoHtml@(La)TeX}
%    Kerning after opening parentheses and before closing parentheses
%    is $-0.1$\,em. The original values $-0.15$\,em
%    looked too ugly for a serif font.
%    \begin{macrocode}
\expandafter
\let\csname HoLogoHtml@(La)TeX\endcsname\HoLogoHtml@LaTeXTeX
%    \end{macrocode}
%    \end{macro}
%    \begin{macro}{\HoLogoCss@LaTeXTeX}
%    \begin{macrocode}
\def\HoLogoCss@LaTeXTeX{%
  \Css{%
    span.HoLogo-LaTeXTeX span.HoLogo-L{%
      margin-left:-.1em;%
    }%
  }%
  \Css{%
    span.HoLogo-LaTeXTeX span.HoLogo-a{%
      position:relative;%
      top:-.5ex;%
      margin-left:-.36em;%
      margin-right:-.1em;%
      font-size:85\%;%
    }%
  }%
  \Css{%
    span.HoLogo-LaTeXTeX span.HoLogo-ParenRight{%
      margin-right:-.15em;%
    }%
  }%
  \global\let\HoLogoCss@LaTeXTeX\relax
}
%    \end{macrocode}
%    \end{macro}
%
% \subsubsection{\hologo{LaTeXe}}
%
%    \begin{macro}{\HoLogo@LaTeXe}
%    Source: \hologo{LaTeX} kernel
%    \begin{macrocode}
\def\HoLogo@LaTeXe#1{%
  \hologo{LaTeX}%
  \kern.15em%
  \hbox{%
    \HOLOGO@MathSetup
    2%
    $_{\textstyle\varepsilon}$%
  }%
}
%    \end{macrocode}
%    \end{macro}
%
%    \begin{macro}{\HoLogoCs@LaTeXe}
%    \begin{macrocode}
\ifnum64=`\^^^^0040\relax % test for big chars of LuaTeX/XeTeX
  \catcode`\$=9 %
  \catcode`\&=14 %
\else
  \catcode`\$=14 %
  \catcode`\&=9 %
\fi
\def\HoLogoCs@LaTeXe#1{%
  LaTeX2%
$ \string ^^^^0395%
& e%
}%
\catcode`\$=3 %
\catcode`\&=4 %
%    \end{macrocode}
%    \end{macro}
%
%    \begin{macro}{\HoLogoBkm@LaTeXe}
%    \begin{macrocode}
\def\HoLogoBkm@LaTeXe#1{%
  \hologo{LaTeX}%
  2%
  \HOLOGO@PdfdocUnicode{e}{\textepsilon}%
}
%    \end{macrocode}
%    \end{macro}
%
%    \begin{macro}{\HoLogoHtml@LaTeXe}
%    \begin{macrocode}
\def\HoLogoHtml@LaTeXe#1{%
  \HoLogoCss@LaTeXe
  \HOLOGO@Span{LaTeX2e}{%
    \hologo{LaTeX}%
    \HOLOGO@Span{2}{2}%
    \HOLOGO@Span{e}{%
      \HOLOGO@MathSetup
      \ensuremath{\textstyle\varepsilon}%
    }%
  }%
}
%    \end{macrocode}
%    \end{macro}
%    \begin{macro}{\HoLogoCss@LaTeXe}
%    \begin{macrocode}
\def\HoLogoCss@LaTeXe{%
  \Css{%
    span.HoLogo-LaTeX2e span.HoLogo-2{%
      padding-left:.15em;%
    }%
  }%
  \Css{%
    span.HoLogo-LaTeX2e span.HoLogo-e{%
      position:relative;%
      top:.35ex;%
      text-decoration:none;%
    }%
  }%
  \global\let\HoLogoCss@LaTeXe\relax
}
%    \end{macrocode}
%    \end{macro}
%
%    \begin{macro}{\HoLogo@LaTeX2e}
%    \begin{macrocode}
\expandafter
\let\csname HoLogo@LaTeX2e\endcsname\HoLogo@LaTeXe
%    \end{macrocode}
%    \end{macro}
%    \begin{macro}{\HoLogoCs@LaTeX2e}
%    \begin{macrocode}
\expandafter
\let\csname HoLogoCs@LaTeX2e\endcsname\HoLogoCs@LaTeXe
%    \end{macrocode}
%    \end{macro}
%    \begin{macro}{\HoLogoBkm@LaTeX2e}
%    \begin{macrocode}
\expandafter
\let\csname HoLogoBkm@LaTeX2e\endcsname\HoLogoBkm@LaTeXe
%    \end{macrocode}
%    \end{macro}
%    \begin{macro}{\HoLogoHtml@LaTeX2e}
%    \begin{macrocode}
\expandafter
\let\csname HoLogoHtml@LaTeX2e\endcsname\HoLogoHtml@LaTeXe
%    \end{macrocode}
%    \end{macro}
%
% \subsubsection{\hologo{LaTeX3}}
%
%    \begin{macro}{\HoLogo@LaTeX3}
%    Source: \hologo{LaTeX} kernel
%    \begin{macrocode}
\expandafter\def\csname HoLogo@LaTeX3\endcsname#1{%
  \hologo{LaTeX}%
  3%
}
%    \end{macrocode}
%    \end{macro}
%
%    \begin{macro}{\HoLogoBkm@LaTeX3}
%    \begin{macrocode}
\expandafter\def\csname HoLogoBkm@LaTeX3\endcsname#1{%
  \hologo{LaTeX}%
  3%
}
%    \end{macrocode}
%    \end{macro}
%    \begin{macro}{\HoLogoHtml@LaTeX3}
%    \begin{macrocode}
\expandafter
\let\csname HoLogoHtml@LaTeX3\expandafter\endcsname
\csname HoLogo@LaTeX3\endcsname
%    \end{macrocode}
%    \end{macro}
%
% \subsubsection{\hologo{LaTeXML}}
%
%    \begin{macro}{\HoLogo@LaTeXML}
%    \begin{macrocode}
\def\HoLogo@LaTeXML#1{%
  \HOLOGO@mbox{%
    \hologo{La}%
    \kern-.15em%
    T%
    \kern-.1667em%
    \lower.5ex\hbox{E}%
    \kern-.125em%
    \HoLogoFont@font{LaTeXML}{sc}{xml}%
  }%
}
%    \end{macrocode}
%    \end{macro}
%    \begin{macro}{\HoLogoHtml@pdfLaTeX}
%    \begin{macrocode}
\def\HoLogoHtml@LaTeXML#1{%
  \HOLOGO@Span{LaTeXML}{%
    \HoLogoCss@LaTeX
    \HoLogoCss@TeX
    \HOLOGO@Span{LaTeX}{%
      L%
      \HOLOGO@Span{a}{%
        A%
      }%
    }%
    \HOLOGO@Span{TeX}{%
      T%
      \HOLOGO@Span{e}{%
        E%
      }%
    }%
    \HCode{<span style="font-variant: small-caps;">}%
    xml%
    \HCode{</span>}%
  }%
}
%    \end{macrocode}
%    \end{macro}
%
% \subsubsection{\hologo{eTeX}}
%
%    \begin{macro}{\HoLogo@eTeX}
%    Source: package \xpackage{etex}
%    \begin{macrocode}
\def\HoLogo@eTeX#1{%
  \ltx@mbox{%
    \HOLOGO@MathSetup
    $\varepsilon$%
    -%
    \HOLOGO@NegativeKerning{-T,T-,To}%
    \hologo{TeX}%
  }%
}
%    \end{macrocode}
%    \end{macro}
%    \begin{macro}{\HoLogoCs@eTeX}
%    \begin{macrocode}
\ifnum64=`\^^^^0040\relax % test for big chars of LuaTeX/XeTeX
  \catcode`\$=9 %
  \catcode`\&=14 %
\else
  \catcode`\$=14 %
  \catcode`\&=9 %
\fi
\def\HoLogoCs@eTeX#1{%
$ #1{\string ^^^^0395}{\string ^^^^03b5}%
& #1{e}{E}%
  TeX%
}%
\catcode`\$=3 %
\catcode`\&=4 %
%    \end{macrocode}
%    \end{macro}
%    \begin{macro}{\HoLogoBkm@eTeX}
%    \begin{macrocode}
\def\HoLogoBkm@eTeX#1{%
  \HOLOGO@PdfdocUnicode{#1{e}{E}}{\textepsilon}%
  -%
  \hologo{TeX}%
}
%    \end{macrocode}
%    \end{macro}
%    \begin{macro}{\HoLogoHtml@eTeX}
%    \begin{macrocode}
\def\HoLogoHtml@eTeX#1{%
  \ltx@mbox{%
    \HOLOGO@MathSetup
    $\varepsilon$%
    -%
    \hologo{TeX}%
  }%
}
%    \end{macrocode}
%    \end{macro}
%
% \subsubsection{\hologo{iniTeX}}
%
%    \begin{macro}{\HoLogo@iniTeX}
%    \begin{macrocode}
\def\HoLogo@iniTeX#1{%
  \HOLOGO@mbox{%
    #1{i}{I}ni\hologo{TeX}%
  }%
}
%    \end{macrocode}
%    \end{macro}
%    \begin{macro}{\HoLogoCs@iniTeX}
%    \begin{macrocode}
\def\HoLogoCs@iniTeX#1{#1{i}{I}niTeX}
%    \end{macrocode}
%    \end{macro}
%    \begin{macro}{\HoLogoBkm@iniTeX}
%    \begin{macrocode}
\def\HoLogoBkm@iniTeX#1{%
  #1{i}{I}ni\hologo{TeX}%
}
%    \end{macrocode}
%    \end{macro}
%    \begin{macro}{\HoLogoHtml@iniTeX}
%    \begin{macrocode}
\let\HoLogoHtml@iniTeX\HoLogo@iniTeX
%    \end{macrocode}
%    \end{macro}
%
% \subsubsection{\hologo{virTeX}}
%
%    \begin{macro}{\HoLogo@virTeX}
%    \begin{macrocode}
\def\HoLogo@virTeX#1{%
  \HOLOGO@mbox{%
    #1{v}{V}ir\hologo{TeX}%
  }%
}
%    \end{macrocode}
%    \end{macro}
%    \begin{macro}{\HoLogoCs@virTeX}
%    \begin{macrocode}
\def\HoLogoCs@virTeX#1{#1{v}{V}irTeX}
%    \end{macrocode}
%    \end{macro}
%    \begin{macro}{\HoLogoBkm@virTeX}
%    \begin{macrocode}
\def\HoLogoBkm@virTeX#1{%
  #1{v}{V}ir\hologo{TeX}%
}
%    \end{macrocode}
%    \end{macro}
%    \begin{macro}{\HoLogoHtml@virTeX}
%    \begin{macrocode}
\let\HoLogoHtml@virTeX\HoLogo@virTeX
%    \end{macrocode}
%    \end{macro}
%
% \subsubsection{\hologo{SliTeX}}
%
% \paragraph{Definitions of the three variants.}
%
%    \begin{macro}{\HoLogo@SLiTeX@lift}
%    \begin{macrocode}
\def\HoLogo@SLiTeX@lift#1{%
  \HoLogoFont@font{SliTeX}{rm}{%
    S%
    \kern-.06em%
    L%
    \kern-.18em%
    \raise.32ex\hbox{\HoLogoFont@font{SliTeX}{sc}{i}}%
    \HOLOGO@discretionary
    \kern-.06em%
    \hologo{TeX}%
  }%
}
%    \end{macrocode}
%    \end{macro}
%    \begin{macro}{\HoLogoBkm@SLiTeX@lift}
%    \begin{macrocode}
\def\HoLogoBkm@SLiTeX@lift#1{SLiTeX}
%    \end{macrocode}
%    \end{macro}
%    \begin{macro}{\HoLogoHtml@SLiTeX@lift}
%    \begin{macrocode}
\def\HoLogoHtml@SLiTeX@lift#1{%
  \HoLogoCss@SLiTeX@lift
  \HOLOGO@Span{SLiTeX-lift}{%
    \HoLogoFont@font{SliTeX}{rm}{%
      S%
      \HOLOGO@Span{L}{L}%
      \HOLOGO@Span{i}{i}%
      \hologo{TeX}%
    }%
  }%
}
%    \end{macrocode}
%    \end{macro}
%    \begin{macro}{\HoLogoCss@SLiTeX@lift}
%    \begin{macrocode}
\def\HoLogoCss@SLiTeX@lift{%
  \Css{%
    span.HoLogo-SLiTeX-lift span.HoLogo-L{%
      margin-left:-.06em;%
      margin-right:-.18em;%
    }%
  }%
  \Css{%
    span.HoLogo-SLiTeX-lift span.HoLogo-i{%
      position:relative;%
      top:-.32ex;%
      margin-right:-.06em;%
      font-variant:small-caps;%
    }%
  }%
  \global\let\HoLogoCss@SLiTeX@lift\relax
}
%    \end{macrocode}
%    \end{macro}
%
%    \begin{macro}{\HoLogo@SliTeX@simple}
%    \begin{macrocode}
\def\HoLogo@SliTeX@simple#1{%
  \HoLogoFont@font{SliTeX}{rm}{%
    \ltx@mbox{%
      \HoLogoFont@font{SliTeX}{sc}{Sli}%
    }%
    \HOLOGO@discretionary
    \hologo{TeX}%
  }%
}
%    \end{macrocode}
%    \end{macro}
%    \begin{macro}{\HoLogoBkm@SliTeX@simple}
%    \begin{macrocode}
\def\HoLogoBkm@SliTeX@simple#1{SliTeX}
%    \end{macrocode}
%    \end{macro}
%    \begin{macro}{\HoLogoHtml@SliTeX@simple}
%    \begin{macrocode}
\let\HoLogoHtml@SliTeX@simple\HoLogo@SliTeX@simple
%    \end{macrocode}
%    \end{macro}
%
%    \begin{macro}{\HoLogo@SliTeX@narrow}
%    \begin{macrocode}
\def\HoLogo@SliTeX@narrow#1{%
  \HoLogoFont@font{SliTeX}{rm}{%
    \ltx@mbox{%
      S%
      \kern-.06em%
      \HoLogoFont@font{SliTeX}{sc}{%
        l%
        \kern-.035em%
        i%
      }%
    }%
    \HOLOGO@discretionary
    \kern-.06em%
    \hologo{TeX}%
  }%
}
%    \end{macrocode}
%    \end{macro}
%    \begin{macro}{\HoLogoBkm@SliTeX@narrow}
%    \begin{macrocode}
\def\HoLogoBkm@SliTeX@narrow#1{SliTeX}
%    \end{macrocode}
%    \end{macro}
%    \begin{macro}{\HoLogoHtml@SliTeX@narrow}
%    \begin{macrocode}
\def\HoLogoHtml@SliTeX@narrow#1{%
  \HoLogoCss@SliTeX@narrow
  \HOLOGO@Span{SliTeX-narrow}{%
    \HoLogoFont@font{SliTeX}{rm}{%
      S%
        \HOLOGO@Span{l}{l}%
        \HOLOGO@Span{i}{i}%
      \hologo{TeX}%
    }%
  }%
}
%    \end{macrocode}
%    \end{macro}
%    \begin{macro}{\HoLogoCss@SliTeX@narrow}
%    \begin{macrocode}
\def\HoLogoCss@SliTeX@narrow{%
  \Css{%
    span.HoLogo-SliTeX-narrow span.HoLogo-l{%
      margin-left:-.06em;%
      margin-right:-.035em;%
      font-variant:small-caps;%
    }%
  }%
  \Css{%
    span.HoLogo-SliTeX-narrow span.HoLogo-i{%
      margin-right:-.06em;%
      font-variant:small-caps;%
    }%
  }%
  \global\let\HoLogoCss@SliTeX@narrow\relax
}
%    \end{macrocode}
%    \end{macro}
%
% \paragraph{Macro set completion.}
%
%    \begin{macro}{\HoLogo@SLiTeX@simple}
%    \begin{macrocode}
\def\HoLogo@SLiTeX@simple{\HoLogo@SliTeX@simple}
%    \end{macrocode}
%    \end{macro}
%    \begin{macro}{\HoLogoBkm@SLiTeX@simple}
%    \begin{macrocode}
\def\HoLogoBkm@SLiTeX@simple{\HoLogoBkm@SliTeX@simple}
%    \end{macrocode}
%    \end{macro}
%    \begin{macro}{\HoLogoHtml@SLiTeX@simple}
%    \begin{macrocode}
\def\HoLogoHtml@SLiTeX@simple{\HoLogoHtml@SliTeX@simple}
%    \end{macrocode}
%    \end{macro}
%
%    \begin{macro}{\HoLogo@SLiTeX@narrow}
%    \begin{macrocode}
\def\HoLogo@SLiTeX@narrow{\HoLogo@SliTeX@narrow}
%    \end{macrocode}
%    \end{macro}
%    \begin{macro}{\HoLogoBkm@SLiTeX@narrow}
%    \begin{macrocode}
\def\HoLogoBkm@SLiTeX@narrow{\HoLogoBkm@SliTeX@narrow}
%    \end{macrocode}
%    \end{macro}
%    \begin{macro}{\HoLogoHtml@SLiTeX@narrow}
%    \begin{macrocode}
\def\HoLogoHtml@SLiTeX@narrow{\HoLogoHtml@SliTeX@narrow}
%    \end{macrocode}
%    \end{macro}
%
%    \begin{macro}{\HoLogo@SliTeX@lift}
%    \begin{macrocode}
\def\HoLogo@SliTeX@lift{\HoLogo@SLiTeX@lift}
%    \end{macrocode}
%    \end{macro}
%    \begin{macro}{\HoLogoBkm@SliTeX@lift}
%    \begin{macrocode}
\def\HoLogoBkm@SliTeX@lift{\HoLogoBkm@SLiTeX@lift}
%    \end{macrocode}
%    \end{macro}
%    \begin{macro}{\HoLogoHtml@SliTeX@lift}
%    \begin{macrocode}
\def\HoLogoHtml@SliTeX@lift{\HoLogoHtml@SLiTeX@lift}
%    \end{macrocode}
%    \end{macro}
%
% \paragraph{Defaults.}
%
%    \begin{macro}{\HoLogo@SLiTeX}
%    \begin{macrocode}
\def\HoLogo@SLiTeX{\HoLogo@SLiTeX@lift}
%    \end{macrocode}
%    \end{macro}
%    \begin{macro}{\HoLogoBkm@SLiTeX}
%    \begin{macrocode}
\def\HoLogoBkm@SLiTeX{\HoLogoBkm@SLiTeX@lift}
%    \end{macrocode}
%    \end{macro}
%    \begin{macro}{\HoLogoHtml@SLiTeX}
%    \begin{macrocode}
\def\HoLogoHtml@SLiTeX{\HoLogoHtml@SLiTeX@lift}
%    \end{macrocode}
%    \end{macro}
%
%    \begin{macro}{\HoLogo@SliTeX}
%    \begin{macrocode}
\def\HoLogo@SliTeX{\HoLogo@SliTeX@narrow}
%    \end{macrocode}
%    \end{macro}
%    \begin{macro}{\HoLogoBkm@SliTeX}
%    \begin{macrocode}
\def\HoLogoBkm@SliTeX{\HoLogoBkm@SliTeX@narrow}
%    \end{macrocode}
%    \end{macro}
%    \begin{macro}{\HoLogoHtml@SliTeX}
%    \begin{macrocode}
\def\HoLogoHtml@SliTeX{\HoLogoHtml@SliTeX@narrow}
%    \end{macrocode}
%    \end{macro}
%
% \subsubsection{\hologo{LuaTeX}}
%
%    \begin{macro}{\HoLogo@LuaTeX}
%    The kerning is an idea of Hans Hagen, see mailing list
%    `luatex at tug dot org' in March 2010.
%    \begin{macrocode}
\def\HoLogo@LuaTeX#1{%
  \HOLOGO@mbox{%
    Lua%
    \HOLOGO@NegativeKerning{aT,oT,To}%
    \hologo{TeX}%
  }%
}
%    \end{macrocode}
%    \end{macro}
%    \begin{macro}{\HoLogoHtml@LuaTeX}
%    \begin{macrocode}
\let\HoLogoHtml@LuaTeX\HoLogo@LuaTeX
%    \end{macrocode}
%    \end{macro}
%
% \subsubsection{\hologo{LuaLaTeX}}
%
%    \begin{macro}{\HoLogo@LuaLaTeX}
%    \begin{macrocode}
\def\HoLogo@LuaLaTeX#1{%
  \HOLOGO@mbox{%
    Lua%
    \hologo{LaTeX}%
  }%
}
%    \end{macrocode}
%    \end{macro}
%    \begin{macro}{\HoLogoHtml@LuaLaTeX}
%    \begin{macrocode}
\let\HoLogoHtml@LuaLaTeX\HoLogo@LuaLaTeX
%    \end{macrocode}
%    \end{macro}
%
% \subsubsection{\hologo{XeTeX}, \hologo{XeLaTeX}}
%
%    \begin{macro}{\HOLOGO@IfCharExists}
%    \begin{macrocode}
\ifluatex
  \ifnum\luatexversion<36 %
  \else
    \def\HOLOGO@IfCharExists#1{%
      \ifnum
        \directlua{%
           if luaotfload and luaotfload.aux then
             if luaotfload.aux.font_has_glyph(%
                    font.current(), \number#1) then % 	 
	       tex.print("1") % 	 
	     end % 	 
	   elseif font and font.fonts and font.current then %
            local f = font.fonts[font.current()]%
            if f.characters and f.characters[\number#1] then %
              tex.print("1")%
            end %
          end%
        }0=\ltx@zero
        \expandafter\ltx@secondoftwo
      \else
        \expandafter\ltx@firstoftwo
      \fi
    }%
  \fi
\fi
\ltx@IfUndefined{HOLOGO@IfCharExists}{%
  \def\HOLOGO@@IfCharExists#1{%
    \begingroup
      \tracinglostchars=\ltx@zero
      \setbox\ltx@zero=\hbox{%
        \kern7sp\char#1\relax
        \ifnum\lastkern>\ltx@zero
          \expandafter\aftergroup\csname iffalse\endcsname
        \else
          \expandafter\aftergroup\csname iftrue\endcsname
        \fi
      }%
      % \if{true|false} from \aftergroup
      \endgroup
      \expandafter\ltx@firstoftwo
    \else
      \endgroup
      \expandafter\ltx@secondoftwo
    \fi
  }%
  \ifxetex
    \ltx@IfUndefined{XeTeXfonttype}{}{%
      \ltx@IfUndefined{XeTeXcharglyph}{}{%
        \def\HOLOGO@IfCharExists#1{%
          \ifnum\XeTeXfonttype\font>\ltx@zero
            \expandafter\ltx@firstofthree
          \else
            \expandafter\ltx@gobble
          \fi
          {%
            \ifnum\XeTeXcharglyph#1>\ltx@zero
              \expandafter\ltx@firstoftwo
            \else
              \expandafter\ltx@secondoftwo
            \fi
          }%
          \HOLOGO@@IfCharExists{#1}%
        }%
      }%
    }%
  \fi
}{}
\ltx@ifundefined{HOLOGO@IfCharExists}{%
  \ifnum64=`\^^^^0040\relax % test for big chars of LuaTeX/XeTeX
    \let\HOLOGO@IfCharExists\HOLOGO@@IfCharExists
  \else
    \def\HOLOGO@IfCharExists#1{%
      \ifnum#1>255 %
        \expandafter\ltx@fourthoffour
      \fi
      \HOLOGO@@IfCharExists{#1}%
    }%
  \fi
}{}
%    \end{macrocode}
%    \end{macro}
%
%    \begin{macro}{\HoLogo@Xe}
%    Source: package \xpackage{dtklogos}
%    \begin{macrocode}
\def\HoLogo@Xe#1{%
  X%
  \kern-.1em\relax
  \HOLOGO@IfCharExists{"018E}{%
    \lower.5ex\hbox{\char"018E}%
  }{%
    \chardef\HOLOGO@choice=\ltx@zero
    \ifdim\fontdimen\ltx@one\font>0pt %
      \ltx@IfUndefined{rotatebox}{%
        \ltx@IfUndefined{pgftext}{%
          \ltx@IfUndefined{psscalebox}{%
            \ltx@IfUndefined{HOLOGO@ScaleBox@\hologoDriver}{%
            }{%
              \chardef\HOLOGO@choice=4 %
            }%
          }{%
            \chardef\HOLOGO@choice=3 %
          }%
        }{%
          \chardef\HOLOGO@choice=2 %
        }%
      }{%
        \chardef\HOLOGO@choice=1 %
      }%
      \ifcase\HOLOGO@choice
        \HOLOGO@WarningUnsupportedDriver{Xe}%
        e%
      \or % 1: \rotatebox
        \begingroup
          \setbox\ltx@zero\hbox{\rotatebox{180}{E}}%
          \ltx@LocDimenA=\dp\ltx@zero
          \advance\ltx@LocDimenA by -.5ex\relax
          \raise\ltx@LocDimenA\box\ltx@zero
        \endgroup
      \or % 2: \pgftext
        \lower.5ex\hbox{%
          \pgfpicture
            \pgftext[rotate=180]{E}%
          \endpgfpicture
        }%
      \or % 3: \psscalebox
        \begingroup
          \setbox\ltx@zero\hbox{\psscalebox{-1 -1}{E}}%
          \ltx@LocDimenA=\dp\ltx@zero
          \advance\ltx@LocDimenA by -.5ex\relax
          \raise\ltx@LocDimenA\box\ltx@zero
        \endgroup
      \or % 4: \HOLOGO@PointReflectBox
        \lower.5ex\hbox{\HOLOGO@PointReflectBox{E}}%
      \else
        \@PackageError{hologo}{Internal error (choice/it}\@ehc
      \fi
    \else
      \ltx@IfUndefined{reflectbox}{%
        \ltx@IfUndefined{pgftext}{%
          \ltx@IfUndefined{psscalebox}{%
            \ltx@IfUndefined{HOLOGO@ScaleBox@\hologoDriver}{%
            }{%
              \chardef\HOLOGO@choice=4 %
            }%
          }{%
            \chardef\HOLOGO@choice=3 %
          }%
        }{%
          \chardef\HOLOGO@choice=2 %
        }%
      }{%
        \chardef\HOLOGO@choice=1 %
      }%
      \ifcase\HOLOGO@choice
        \HOLOGO@WarningUnsupportedDriver{Xe}%
        e%
      \or % 1: reflectbox
        \lower.5ex\hbox{%
          \reflectbox{E}%
        }%
      \or % 2: \pgftext
        \lower.5ex\hbox{%
          \pgfpicture
            \pgftransformxscale{-1}%
            \pgftext{E}%
          \endpgfpicture
        }%
      \or % 3: \psscalebox
        \lower.5ex\hbox{%
          \psscalebox{-1 1}{E}%
        }%
      \or % 4: \HOLOGO@Reflectbox
        \lower.5ex\hbox{%
          \HOLOGO@ReflectBox{E}%
        }%
      \else
        \@PackageError{hologo}{Internal error (choice/up)}\@ehc
      \fi
    \fi
  }%
}
%    \end{macrocode}
%    \end{macro}
%    \begin{macro}{\HoLogoHtml@Xe}
%    \begin{macrocode}
\def\HoLogoHtml@Xe#1{%
  \HoLogoCss@Xe
  \HOLOGO@Span{Xe}{%
    X%
    \HOLOGO@Span{e}{%
      \HCode{&\ltx@hashchar x018e;}%
    }%
  }%
}
%    \end{macrocode}
%    \end{macro}
%    \begin{macro}{\HoLogoCss@Xe}
%    \begin{macrocode}
\def\HoLogoCss@Xe{%
  \Css{%
    span.HoLogo-Xe span.HoLogo-e{%
      position:relative;%
      top:.5ex;%
      left-margin:-.1em;%
    }%
  }%
  \global\let\HoLogoCss@Xe\relax
}
%    \end{macrocode}
%    \end{macro}
%
%    \begin{macro}{\HoLogo@XeTeX}
%    \begin{macrocode}
\def\HoLogo@XeTeX#1{%
  \hologo{Xe}%
  \kern-.15em\relax
  \hologo{TeX}%
}
%    \end{macrocode}
%    \end{macro}
%
%    \begin{macro}{\HoLogoHtml@XeTeX}
%    \begin{macrocode}
\def\HoLogoHtml@XeTeX#1{%
  \HoLogoCss@XeTeX
  \HOLOGO@Span{XeTeX}{%
    \hologo{Xe}%
    \hologo{TeX}%
  }%
}
%    \end{macrocode}
%    \end{macro}
%    \begin{macro}{\HoLogoCss@XeTeX}
%    \begin{macrocode}
\def\HoLogoCss@XeTeX{%
  \Css{%
    span.HoLogo-XeTeX span.HoLogo-TeX{%
      margin-left:-.15em;%
    }%
  }%
  \global\let\HoLogoCss@XeTeX\relax
}
%    \end{macrocode}
%    \end{macro}
%
%    \begin{macro}{\HoLogo@XeLaTeX}
%    \begin{macrocode}
\def\HoLogo@XeLaTeX#1{%
  \hologo{Xe}%
  \kern-.13em%
  \hologo{LaTeX}%
}
%    \end{macrocode}
%    \end{macro}
%    \begin{macro}{\HoLogoHtml@XeLaTeX}
%    \begin{macrocode}
\def\HoLogoHtml@XeLaTeX#1{%
  \HoLogoCss@XeLaTeX
  \HOLOGO@Span{XeLaTeX}{%
    \hologo{Xe}%
    \hologo{LaTeX}%
  }%
}
%    \end{macrocode}
%    \end{macro}
%    \begin{macro}{\HoLogoCss@XeLaTeX}
%    \begin{macrocode}
\def\HoLogoCss@XeLaTeX{%
  \Css{%
    span.HoLogo-XeLaTeX span.HoLogo-Xe{%
      margin-right:-.13em;%
    }%
  }%
  \global\let\HoLogoCss@XeLaTeX\relax
}
%    \end{macrocode}
%    \end{macro}
%
% \subsubsection{\hologo{pdfTeX}, \hologo{pdfLaTeX}}
%
%    \begin{macro}{\HoLogo@pdfTeX}
%    \begin{macrocode}
\def\HoLogo@pdfTeX#1{%
  \HOLOGO@mbox{%
    #1{p}{P}df\hologo{TeX}%
  }%
}
%    \end{macrocode}
%    \end{macro}
%    \begin{macro}{\HoLogoCs@pdfTeX}
%    \begin{macrocode}
\def\HoLogoCs@pdfTeX#1{#1{p}{P}dfTeX}
%    \end{macrocode}
%    \end{macro}
%    \begin{macro}{\HoLogoBkm@pdfTeX}
%    \begin{macrocode}
\def\HoLogoBkm@pdfTeX#1{%
  #1{p}{P}df\hologo{TeX}%
}
%    \end{macrocode}
%    \end{macro}
%    \begin{macro}{\HoLogoHtml@pdfTeX}
%    \begin{macrocode}
\let\HoLogoHtml@pdfTeX\HoLogo@pdfTeX
%    \end{macrocode}
%    \end{macro}
%
%    \begin{macro}{\HoLogo@pdfLaTeX}
%    \begin{macrocode}
\def\HoLogo@pdfLaTeX#1{%
  \HOLOGO@mbox{%
    #1{p}{P}df\hologo{LaTeX}%
  }%
}
%    \end{macrocode}
%    \end{macro}
%    \begin{macro}{\HoLogoCs@pdfLaTeX}
%    \begin{macrocode}
\def\HoLogoCs@pdfLaTeX#1{#1{p}{P}dfLaTeX}
%    \end{macrocode}
%    \end{macro}
%    \begin{macro}{\HoLogoBkm@pdfLaTeX}
%    \begin{macrocode}
\def\HoLogoBkm@pdfLaTeX#1{%
  #1{p}{P}df\hologo{LaTeX}%
}
%    \end{macrocode}
%    \end{macro}
%    \begin{macro}{\HoLogoHtml@pdfLaTeX}
%    \begin{macrocode}
\let\HoLogoHtml@pdfLaTeX\HoLogo@pdfLaTeX
%    \end{macrocode}
%    \end{macro}
%
% \subsubsection{\hologo{VTeX}}
%
%    \begin{macro}{\HoLogo@VTeX}
%    \begin{macrocode}
\def\HoLogo@VTeX#1{%
  \HOLOGO@mbox{%
    V\hologo{TeX}%
  }%
}
%    \end{macrocode}
%    \end{macro}
%    \begin{macro}{\HoLogoHtml@VTeX}
%    \begin{macrocode}
\let\HoLogoHtml@VTeX\HoLogo@VTeX
%    \end{macrocode}
%    \end{macro}
%
% \subsubsection{\hologo{AmS}, \dots}
%
%    Source: class \xclass{amsdtx}
%
%    \begin{macro}{\HoLogo@AmS}
%    \begin{macrocode}
\def\HoLogo@AmS#1{%
  \HoLogoFont@font{AmS}{sy}{%
    A%
    \kern-.1667em%
    \lower.5ex\hbox{M}%
    \kern-.125em%
    S%
  }%
}
%    \end{macrocode}
%    \end{macro}
%    \begin{macro}{\HoLogoBkm@AmS}
%    \begin{macrocode}
\def\HoLogoBkm@AmS#1{AmS}
%    \end{macrocode}
%    \end{macro}
%    \begin{macro}{\HoLogoHtml@AmS}
%    \begin{macrocode}
\def\HoLogoHtml@AmS#1{%
  \HoLogoCss@AmS
%  \HoLogoFont@font{AmS}{sy}{%
    \HOLOGO@Span{AmS}{%
      A%
      \HOLOGO@Span{M}{M}%
      S%
    }%
%   }%
}
%    \end{macrocode}
%    \end{macro}
%    \begin{macro}{\HoLogoCss@AmS}
%    \begin{macrocode}
\def\HoLogoCss@AmS{%
  \Css{%
    span.HoLogo-AmS span.HoLogo-M{%
      position:relative;%
      top:.5ex;%
      margin-left:-.1667em;%
      margin-right:-.125em;%
      text-decoration:none;%
    }%
  }%
  \global\let\HoLogoCss@AmS\relax
}
%    \end{macrocode}
%    \end{macro}
%
%    \begin{macro}{\HoLogo@AmSTeX}
%    \begin{macrocode}
\def\HoLogo@AmSTeX#1{%
  \hologo{AmS}%
  \HOLOGO@hyphen
  \hologo{TeX}%
}
%    \end{macrocode}
%    \end{macro}
%    \begin{macro}{\HoLogoBkm@AmSTeX}
%    \begin{macrocode}
\def\HoLogoBkm@AmSTeX#1{AmS-TeX}%
%    \end{macrocode}
%    \end{macro}
%    \begin{macro}{\HoLogoHtml@AmSTeX}
%    \begin{macrocode}
\let\HoLogoHtml@AmSTeX\HoLogo@AmSTeX
%    \end{macrocode}
%    \end{macro}
%
%    \begin{macro}{\HoLogo@AmSLaTeX}
%    \begin{macrocode}
\def\HoLogo@AmSLaTeX#1{%
  \hologo{AmS}%
  \HOLOGO@hyphen
  \hologo{LaTeX}%
}
%    \end{macrocode}
%    \end{macro}
%    \begin{macro}{\HoLogoBkm@AmSLaTeX}
%    \begin{macrocode}
\def\HoLogoBkm@AmSLaTeX#1{AmS-LaTeX}%
%    \end{macrocode}
%    \end{macro}
%    \begin{macro}{\HoLogoHtml@AmSLaTeX}
%    \begin{macrocode}
\let\HoLogoHtml@AmSLaTeX\HoLogo@AmSLaTeX
%    \end{macrocode}
%    \end{macro}
%
% \subsubsection{\hologo{BibTeX}}
%
%    \begin{macro}{\HoLogo@BibTeX@sc}
%    A definition of \hologo{BibTeX} is provided in
%    the documentation source for the manual of \hologo{BibTeX}
%    \cite{btxdoc}.
%\begin{quote}
%\begin{verbatim}
%\def\BibTeX{%
%  {%
%    \rm
%    B%
%    \kern-.05em%
%    {%
%      \sc
%      i%
%      \kern-.025em %
%      b%
%    }%
%    \kern-.08em
%    T%
%    \kern-.1667em%
%    \lower.7ex\hbox{E}%
%    \kern-.125em%
%    X%
%  }%
%}
%\end{verbatim}
%\end{quote}
%    \begin{macrocode}
\def\HoLogo@BibTeX@sc#1{%
  B%
  \kern-.05em%
  \HoLogoFont@font{BibTeX}{sc}{%
    i%
    \kern-.025em%
    b%
  }%
  \HOLOGO@discretionary
  \kern-.08em%
  \hologo{TeX}%
}
%    \end{macrocode}
%    \end{macro}
%    \begin{macro}{\HoLogoHtml@BibTeX@sc}
%    \begin{macrocode}
\def\HoLogoHtml@BibTeX@sc#1{%
  \HoLogoCss@BibTeX@sc
  \HOLOGO@Span{BibTeX-sc}{%
    B%
    \HOLOGO@Span{i}{i}%
    \HOLOGO@Span{b}{b}%
    \hologo{TeX}%
  }%
}
%    \end{macrocode}
%    \end{macro}
%    \begin{macro}{\HoLogoCss@BibTeX@sc}
%    \begin{macrocode}
\def\HoLogoCss@BibTeX@sc{%
  \Css{%
    span.HoLogo-BibTeX-sc span.HoLogo-i{%
      margin-left:-.05em;%
      margin-right:-.025em;%
      font-variant:small-caps;%
    }%
  }%
  \Css{%
    span.HoLogo-BibTeX-sc span.HoLogo-b{%
      margin-right:-.08em;%
      font-variant:small-caps;%
    }%
  }%
  \global\let\HoLogoCss@BibTeX@sc\relax
}
%    \end{macrocode}
%    \end{macro}
%
%    \begin{macro}{\HoLogo@BibTeX@sf}
%    Variant \xoption{sf} avoids trouble with unavailable
%    small caps fonts (e.g., bold versions of Computer Modern or
%    Latin Modern). The definition is taken from
%    package \xpackage{dtklogos} \cite{dtklogos}.
%\begin{quote}
%\begin{verbatim}
%\DeclareRobustCommand{\BibTeX}{%
%  B%
%  \kern-.05em%
%  \hbox{%
%    $\m@th$% %% force math size calculations
%    \csname S@\f@size\endcsname
%    \fontsize\sf@size\z@
%    \math@fontsfalse
%    \selectfont
%    I%
%    \kern-.025em%
%    B
%  }%
%  \kern-.08em%
%  \-%
%  \TeX
%}
%\end{verbatim}
%\end{quote}
%    \begin{macrocode}
\def\HoLogo@BibTeX@sf#1{%
  B%
  \kern-.05em%
  \HoLogoFont@font{BibTeX}{bibsf}{%
    I%
    \kern-.025em%
    B%
  }%
  \HOLOGO@discretionary
  \kern-.08em%
  \hologo{TeX}%
}
%    \end{macrocode}
%    \end{macro}
%    \begin{macro}{\HoLogoHtml@BibTeX@sf}
%    \begin{macrocode}
\def\HoLogoHtml@BibTeX@sf#1{%
  \HoLogoCss@BibTeX@sf
  \HOLOGO@Span{BibTeX-sf}{%
    B%
    \HoLogoFont@font{BibTeX}{bibsf}{%
      \HOLOGO@Span{i}{I}%
      B%
    }%
    \hologo{TeX}%
  }%
}
%    \end{macrocode}
%    \end{macro}
%    \begin{macro}{\HoLogoCss@BibTeX@sf}
%    \begin{macrocode}
\def\HoLogoCss@BibTeX@sf{%
  \Css{%
    span.HoLogo-BibTeX-sf span.HoLogo-i{%
      margin-left:-.05em;%
      margin-right:-.025em;%
    }%
  }%
  \Css{%
    span.HoLogo-BibTeX-sf span.HoLogo-TeX{%
      margin-left:-.08em;%
    }%
  }%
  \global\let\HoLogoCss@BibTeX@sf\relax
}
%    \end{macrocode}
%    \end{macro}
%
%    \begin{macro}{\HoLogo@BibTeX}
%    \begin{macrocode}
\def\HoLogo@BibTeX{\HoLogo@BibTeX@sf}
%    \end{macrocode}
%    \end{macro}
%    \begin{macro}{\HoLogoHtml@BibTeX}
%    \begin{macrocode}
\def\HoLogoHtml@BibTeX{\HoLogoHtml@BibTeX@sf}
%    \end{macrocode}
%    \end{macro}
%
% \subsubsection{\hologo{BibTeX8}}
%
%    \begin{macro}{\HoLogo@BibTeX8}
%    \begin{macrocode}
\expandafter\def\csname HoLogo@BibTeX8\endcsname#1{%
  \hologo{BibTeX}%
  8%
}
%    \end{macrocode}
%    \end{macro}
%
%    \begin{macro}{\HoLogoBkm@BibTeX8}
%    \begin{macrocode}
\expandafter\def\csname HoLogoBkm@BibTeX8\endcsname#1{%
  \hologo{BibTeX}%
  8%
}
%    \end{macrocode}
%    \end{macro}
%    \begin{macro}{\HoLogoHtml@BibTeX8}
%    \begin{macrocode}
\expandafter
\let\csname HoLogoHtml@BibTeX8\expandafter\endcsname
\csname HoLogo@BibTeX8\endcsname
%    \end{macrocode}
%    \end{macro}
%
% \subsubsection{\hologo{ConTeXt}}
%
%    \begin{macro}{\HoLogo@ConTeXt@simple}
%    \begin{macrocode}
\def\HoLogo@ConTeXt@simple#1{%
  \HOLOGO@mbox{Con}%
  \HOLOGO@discretionary
  \HOLOGO@mbox{\hologo{TeX}t}%
}
%    \end{macrocode}
%    \end{macro}
%    \begin{macro}{\HoLogoHtml@ConTeXt@simple}
%    \begin{macrocode}
\let\HoLogoHtml@ConTeXt@simple\HoLogo@ConTeXt@simple
%    \end{macrocode}
%    \end{macro}
%
%    \begin{macro}{\HoLogo@ConTeXt@narrow}
%    This definition of logo \hologo{ConTeXt} with variant \xoption{narrow}
%    comes from TUGboat's class \xclass{ltugboat} (version 2010/11/15 v2.8).
%    \begin{macrocode}
\def\HoLogo@ConTeXt@narrow#1{%
  \HOLOGO@mbox{C\kern-.0333emon}%
  \HOLOGO@discretionary
  \kern-.0667em%
  \HOLOGO@mbox{\hologo{TeX}\kern-.0333emt}%
}
%    \end{macrocode}
%    \end{macro}
%    \begin{macro}{\HoLogoHtml@ConTeXt@narrow}
%    \begin{macrocode}
\def\HoLogoHtml@ConTeXt@narrow#1{%
  \HoLogoCss@ConTeXt@narrow
  \HOLOGO@Span{ConTeXt-narrow}{%
    \HOLOGO@Span{C}{C}%
    on%
    \hologo{TeX}%
    t%
  }%
}
%    \end{macrocode}
%    \end{macro}
%    \begin{macro}{\HoLogoCss@ConTeXt@narrow}
%    \begin{macrocode}
\def\HoLogoCss@ConTeXt@narrow{%
  \Css{%
    span.HoLogo-ConTeXt-narrow span.HoLogo-C{%
      margin-left:-.0333em;%
    }%
  }%
  \Css{%
    span.HoLogo-ConTeXt-narrow span.HoLogo-TeX{%
      margin-left:-.0667em;%
      margin-right:-.0333em;%
    }%
  }%
  \global\let\HoLogoCss@ConTeXt@narrow\relax
}
%    \end{macrocode}
%    \end{macro}
%
%    \begin{macro}{\HoLogo@ConTeXt}
%    \begin{macrocode}
\def\HoLogo@ConTeXt{\HoLogo@ConTeXt@narrow}
%    \end{macrocode}
%    \end{macro}
%    \begin{macro}{\HoLogoHtml@ConTeXt}
%    \begin{macrocode}
\def\HoLogoHtml@ConTeXt{\HoLogoHtml@ConTeXt@narrow}
%    \end{macrocode}
%    \end{macro}
%
% \subsubsection{\hologo{emTeX}}
%
%    \begin{macro}{\HoLogo@emTeX}
%    \begin{macrocode}
\def\HoLogo@emTeX#1{%
  \HOLOGO@mbox{#1{e}{E}m}%
  \HOLOGO@discretionary
  \hologo{TeX}%
}
%    \end{macrocode}
%    \end{macro}
%    \begin{macro}{\HoLogoCs@emTeX}
%    \begin{macrocode}
\def\HoLogoCs@emTeX#1{#1{e}{E}mTeX}%
%    \end{macrocode}
%    \end{macro}
%    \begin{macro}{\HoLogoBkm@emTeX}
%    \begin{macrocode}
\def\HoLogoBkm@emTeX#1{%
  #1{e}{E}m\hologo{TeX}%
}
%    \end{macrocode}
%    \end{macro}
%    \begin{macro}{\HoLogoHtml@emTeX}
%    \begin{macrocode}
\let\HoLogoHtml@emTeX\HoLogo@emTeX
%    \end{macrocode}
%    \end{macro}
%
% \subsubsection{\hologo{ExTeX}}
%
%    \begin{macro}{\HoLogo@ExTeX}
%    The definition is taken from the FAQ of the
%    project \hologo{ExTeX}
%    \cite{ExTeX-FAQ}.
%\begin{quote}
%\begin{verbatim}
%\def\ExTeX{%
%  \textrm{% Logo always with serifs
%    \ensuremath{%
%      \textstyle
%      \varepsilon_{%
%        \kern-0.15em%
%        \mathcal{X}%
%      }%
%    }%
%    \kern-.15em%
%    \TeX
%  }%
%}
%\end{verbatim}
%\end{quote}
%    \begin{macrocode}
\def\HoLogo@ExTeX#1{%
  \HoLogoFont@font{ExTeX}{rm}{%
    \ltx@mbox{%
      \HOLOGO@MathSetup
      $%
        \textstyle
        \varepsilon_{%
          \kern-0.15em%
          \HoLogoFont@font{ExTeX}{sy}{X}%
        }%
      $%
    }%
    \HOLOGO@discretionary
    \kern-.15em%
    \hologo{TeX}%
  }%
}
%    \end{macrocode}
%    \end{macro}
%    \begin{macro}{\HoLogoHtml@ExTeX}
%    \begin{macrocode}
\def\HoLogoHtml@ExTeX#1{%
  \HoLogoCss@ExTeX
  \HoLogoFont@font{ExTeX}{rm}{%
    \HOLOGO@Span{ExTeX}{%
      \ltx@mbox{%
        \HOLOGO@MathSetup
        $\textstyle\varepsilon$%
        \HOLOGO@Span{X}{$\textstyle\chi$}%
        \hologo{TeX}%
      }%
    }%
  }%
}
%    \end{macrocode}
%    \end{macro}
%    \begin{macro}{\HoLogoBkm@ExTeX}
%    \begin{macrocode}
\def\HoLogoBkm@ExTeX#1{%
  \HOLOGO@PdfdocUnicode{#1{e}{E}x}{\textepsilon\textchi}%
  \hologo{TeX}%
}
%    \end{macrocode}
%    \end{macro}
%    \begin{macro}{\HoLogoCss@ExTeX}
%    \begin{macrocode}
\def\HoLogoCss@ExTeX{%
  \Css{%
    span.HoLogo-ExTeX{%
      font-family:serif;%
    }%
  }%
  \Css{%
    span.HoLogo-ExTeX span.HoLogo-TeX{%
      margin-left:-.15em;%
    }%
  }%
  \global\let\HoLogoCss@ExTeX\relax
}
%    \end{macrocode}
%    \end{macro}
%
% \subsubsection{\hologo{MiKTeX}}
%
%    \begin{macro}{\HoLogo@MiKTeX}
%    \begin{macrocode}
\def\HoLogo@MiKTeX#1{%
  \HOLOGO@mbox{MiK}%
  \HOLOGO@discretionary
  \hologo{TeX}%
}
%    \end{macrocode}
%    \end{macro}
%    \begin{macro}{\HoLogoHtml@MiKTeX}
%    \begin{macrocode}
\let\HoLogoHtml@MiKTeX\HoLogo@MiKTeX
%    \end{macrocode}
%    \end{macro}
%
% \subsubsection{\hologo{OzTeX} and friends}
%
%    Source: \hologo{OzTeX} FAQ \cite{OzTeX}:
%    \begin{quote}
%      |\def\OzTeX{O\kern-.03em z\kern-.15em\TeX}|\\
%      (There is no kerning in OzMF, OzMP and OzTtH.)
%    \end{quote}
%
%    \begin{macro}{\HoLogo@OzTeX}
%    \begin{macrocode}
\def\HoLogo@OzTeX#1{%
  O%
  \kern-.03em %
  z%
  \kern-.15em %
  \hologo{TeX}%
}
%    \end{macrocode}
%    \end{macro}
%    \begin{macro}{\HoLogoHtml@OzTeX}
%    \begin{macrocode}
\def\HoLogoHtml@OzTeX#1{%
  \HoLogoCss@OzTeX
  \HOLOGO@Span{OzTeX}{%
    O%
    \HOLOGO@Span{z}{z}%
    \hologo{TeX}%
  }%
}
%    \end{macrocode}
%    \end{macro}
%    \begin{macro}{\HoLogoCss@OzTeX}
%    \begin{macrocode}
\def\HoLogoCss@OzTeX{%
  \Css{%
    span.HoLogo-OzTeX span.HoLogo-z{%
      margin-left:-.03em;%
      margin-right:-.15em;%
    }%
  }%
  \global\let\HoLogoCss@OzTeX\relax
}
%    \end{macrocode}
%    \end{macro}
%
%    \begin{macro}{\HoLogo@OzMF}
%    \begin{macrocode}
\def\HoLogo@OzMF#1{%
  \HOLOGO@mbox{OzMF}%
}
%    \end{macrocode}
%    \end{macro}
%    \begin{macro}{\HoLogo@OzMP}
%    \begin{macrocode}
\def\HoLogo@OzMP#1{%
  \HOLOGO@mbox{OzMP}%
}
%    \end{macrocode}
%    \end{macro}
%    \begin{macro}{\HoLogo@OzTtH}
%    \begin{macrocode}
\def\HoLogo@OzTtH#1{%
  \HOLOGO@mbox{OzTtH}%
}
%    \end{macrocode}
%    \end{macro}
%
% \subsubsection{\hologo{PCTeX}}
%
%    \begin{macro}{\HoLogo@PCTeX}
%    \begin{macrocode}
\def\HoLogo@PCTeX#1{%
  \HOLOGO@mbox{PC}%
  \hologo{TeX}%
}
%    \end{macrocode}
%    \end{macro}
%    \begin{macro}{\HoLogoHtml@PCTeX}
%    \begin{macrocode}
\let\HoLogoHtml@PCTeX\HoLogo@PCTeX
%    \end{macrocode}
%    \end{macro}
%
% \subsubsection{\hologo{PiCTeX}}
%
%    The original definitions from \xfile{pictex.tex} \cite{PiCTeX}:
%\begin{quote}
%\begin{verbatim}
%\def\PiC{%
%  P%
%  \kern-.12em%
%  \lower.5ex\hbox{I}%
%  \kern-.075em%
%  C%
%}
%\def\PiCTeX{%
%  \PiC
%  \kern-.11em%
%  \TeX
%}
%\end{verbatim}
%\end{quote}
%
%    \begin{macro}{\HoLogo@PiC}
%    \begin{macrocode}
\def\HoLogo@PiC#1{%
  P%
  \kern-.12em%
  \lower.5ex\hbox{I}%
  \kern-.075em%
  C%
  \HOLOGO@SpaceFactor
}
%    \end{macrocode}
%    \end{macro}
%    \begin{macro}{\HoLogoHtml@PiC}
%    \begin{macrocode}
\def\HoLogoHtml@PiC#1{%
  \HoLogoCss@PiC
  \HOLOGO@Span{PiC}{%
    P%
    \HOLOGO@Span{i}{I}%
    C%
  }%
}
%    \end{macrocode}
%    \end{macro}
%    \begin{macro}{\HoLogoCss@PiC}
%    \begin{macrocode}
\def\HoLogoCss@PiC{%
  \Css{%
    span.HoLogo-PiC span.HoLogo-i{%
      position:relative;%
      top:.5ex;%
      margin-left:-.12em;%
      margin-right:-.075em;%
      text-decoration:none;%
    }%
  }%
  \global\let\HoLogoCss@PiC\relax
}
%    \end{macrocode}
%    \end{macro}
%
%    \begin{macro}{\HoLogo@PiCTeX}
%    \begin{macrocode}
\def\HoLogo@PiCTeX#1{%
  \hologo{PiC}%
  \HOLOGO@discretionary
  \kern-.11em%
  \hologo{TeX}%
}
%    \end{macrocode}
%    \end{macro}
%    \begin{macro}{\HoLogoHtml@PiCTeX}
%    \begin{macrocode}
\def\HoLogoHtml@PiCTeX#1{%
  \HoLogoCss@PiCTeX
  \HOLOGO@Span{PiCTeX}{%
    \hologo{PiC}%
    \hologo{TeX}%
  }%
}
%    \end{macrocode}
%    \end{macro}
%    \begin{macro}{\HoLogoCss@PiCTeX}
%    \begin{macrocode}
\def\HoLogoCss@PiCTeX{%
  \Css{%
    span.HoLogo-PiCTeX span.HoLogo-PiC{%
      margin-right:-.11em;%
    }%
  }%
  \global\let\HoLogoCss@PiCTeX\relax
}
%    \end{macrocode}
%    \end{macro}
%
% \subsubsection{\hologo{teTeX}}
%
%    \begin{macro}{\HoLogo@teTeX}
%    \begin{macrocode}
\def\HoLogo@teTeX#1{%
  \HOLOGO@mbox{#1{t}{T}e}%
  \HOLOGO@discretionary
  \hologo{TeX}%
}
%    \end{macrocode}
%    \end{macro}
%    \begin{macro}{\HoLogoCs@teTeX}
%    \begin{macrocode}
\def\HoLogoCs@teTeX#1{#1{t}{T}dfTeX}
%    \end{macrocode}
%    \end{macro}
%    \begin{macro}{\HoLogoBkm@teTeX}
%    \begin{macrocode}
\def\HoLogoBkm@teTeX#1{%
  #1{t}{T}e\hologo{TeX}%
}
%    \end{macrocode}
%    \end{macro}
%    \begin{macro}{\HoLogoHtml@teTeX}
%    \begin{macrocode}
\let\HoLogoHtml@teTeX\HoLogo@teTeX
%    \end{macrocode}
%    \end{macro}
%
% \subsubsection{\hologo{TeX4ht}}
%
%    \begin{macro}{\HoLogo@TeX4ht}
%    \begin{macrocode}
\expandafter\def\csname HoLogo@TeX4ht\endcsname#1{%
  \HOLOGO@mbox{\hologo{TeX}4ht}%
}
%    \end{macrocode}
%    \end{macro}
%    \begin{macro}{\HoLogoHtml@TeX4ht}
%    \begin{macrocode}
\expandafter
\let\csname HoLogoHtml@TeX4ht\expandafter\endcsname
\csname HoLogo@TeX4ht\endcsname
%    \end{macrocode}
%    \end{macro}
%
%
% \subsubsection{\hologo{SageTeX}}
%
%    \begin{macro}{\HoLogo@SageTeX}
%    \begin{macrocode}
\def\HoLogo@SageTeX#1{%
  \HOLOGO@mbox{Sage}%
  \HOLOGO@discretionary
  \HOLOGO@NegativeKerning{eT,oT,To}%
  \hologo{TeX}%
}
%    \end{macrocode}
%    \end{macro}
%    \begin{macro}{\HoLogoHtml@SageTeX}
%    \begin{macrocode}
\let\HoLogoHtml@SageTeX\HoLogo@SageTeX
%    \end{macrocode}
%    \end{macro}
%
% \subsection{\hologo{METAFONT} and friends}
%
%    \begin{macro}{\HoLogo@METAFONT}
%    \begin{macrocode}
\def\HoLogo@METAFONT#1{%
  \HoLogoFont@font{METAFONT}{logo}{%
    \HOLOGO@mbox{META}%
    \HOLOGO@discretionary
    \HOLOGO@mbox{FONT}%
  }%
}
%    \end{macrocode}
%    \end{macro}
%
%    \begin{macro}{\HoLogo@METAPOST}
%    \begin{macrocode}
\def\HoLogo@METAPOST#1{%
  \HoLogoFont@font{METAPOST}{logo}{%
    \HOLOGO@mbox{META}%
    \HOLOGO@discretionary
    \HOLOGO@mbox{POST}%
  }%
}
%    \end{macrocode}
%    \end{macro}
%
%    \begin{macro}{\HoLogo@MetaFun}
%    \begin{macrocode}
\def\HoLogo@MetaFun#1{%
  \HOLOGO@mbox{Meta}%
  \HOLOGO@discretionary
  \HOLOGO@mbox{Fun}%
}
%    \end{macrocode}
%    \end{macro}
%
%    \begin{macro}{\HoLogo@MetaPost}
%    \begin{macrocode}
\def\HoLogo@MetaPost#1{%
  \HOLOGO@mbox{Meta}%
  \HOLOGO@discretionary
  \HOLOGO@mbox{Post}%
}
%    \end{macrocode}
%    \end{macro}
%
% \subsection{Others}
%
% \subsubsection{\hologo{biber}}
%
%    \begin{macro}{\HoLogo@biber}
%    \begin{macrocode}
\def\HoLogo@biber#1{%
  \HOLOGO@mbox{#1{b}{B}i}%
  \HOLOGO@discretionary
  \HOLOGO@mbox{ber}%
}
%    \end{macrocode}
%    \end{macro}
%    \begin{macro}{\HoLogoCs@biber}
%    \begin{macrocode}
\def\HoLogoCs@biber#1{#1{b}{B}iber}
%    \end{macrocode}
%    \end{macro}
%    \begin{macro}{\HoLogoBkm@biber}
%    \begin{macrocode}
\def\HoLogoBkm@biber#1{%
  #1{b}{B}iber%
}
%    \end{macrocode}
%    \end{macro}
%    \begin{macro}{\HoLogoHtml@biber}
%    \begin{macrocode}
\let\HoLogoHtml@biber\HoLogo@biber
%    \end{macrocode}
%    \end{macro}
%
% \subsubsection{\hologo{KOMAScript}}
%
%    \begin{macro}{\HoLogo@KOMAScript}
%    The definition for \hologo{KOMAScript} is taken
%    from \hologo{KOMAScript} (\xfile{scrlogo.dtx}, reformatted) \cite{scrlogo}:
%\begin{quote}
%\begin{verbatim}
%\@ifundefined{KOMAScript}{%
%  \DeclareRobustCommand{\KOMAScript}{%
%    \textsf{%
%      K\kern.05em O\kern.05emM\kern.05em A%
%      \kern.1em-\kern.1em %
%      Script%
%    }%
%  }%
%}{}
%\end{verbatim}
%\end{quote}
%    \begin{macrocode}
\def\HoLogo@KOMAScript#1{%
  \HoLogoFont@font{KOMAScript}{sf}{%
    \HOLOGO@mbox{%
      K\kern.05em%
      O\kern.05em%
      M\kern.05em%
      A%
    }%
    \kern.1em%
    \HOLOGO@hyphen
    \kern.1em%
    \HOLOGO@mbox{Script}%
  }%
}
%    \end{macrocode}
%    \end{macro}
%    \begin{macro}{\HoLogoBkm@KOMAScript}
%    \begin{macrocode}
\def\HoLogoBkm@KOMAScript#1{%
  KOMA-Script%
}
%    \end{macrocode}
%    \end{macro}
%    \begin{macro}{\HoLogoHtml@KOMAScript}
%    \begin{macrocode}
\def\HoLogoHtml@KOMAScript#1{%
  \HoLogoCss@KOMAScript
  \HoLogoFont@font{KOMAScript}{sf}{%
    \HOLOGO@Span{KOMAScript}{%
      K%
      \HOLOGO@Span{O}{O}%
      M%
      \HOLOGO@Span{A}{A}%
      \HOLOGO@Span{hyphen}{-}%
      Script%
    }%
  }%
}
%    \end{macrocode}
%    \end{macro}
%    \begin{macro}{\HoLogoCss@KOMAScript}
%    \begin{macrocode}
\def\HoLogoCss@KOMAScript{%
  \Css{%
    span.HoLogo-KOMAScript{%
      font-family:sans-serif;%
    }%
  }%
  \Css{%
    span.HoLogo-KOMAScript span.HoLogo-O{%
      padding-left:.05em;%
      padding-right:.05em;%
    }%
  }%
  \Css{%
    span.HoLogo-KOMAScript span.HoLogo-A{%
      padding-left:.05em;%
    }%
  }%
  \Css{%
    span.HoLogo-KOMAScript span.HoLogo-hyphen{%
      padding-left:.1em;%
      padding-right:.1em;%
    }%
  }%
  \global\let\HoLogoCss@KOMAScript\relax
}
%    \end{macrocode}
%    \end{macro}
%
% \subsubsection{\hologo{LyX}}
%
%    \begin{macro}{\HoLogo@LyX}
%    The definition is taken from the documentation source files
%    of \hologo{LyX}, \xfile{Intro.lyx} \cite{LyX}:
%\begin{quote}
%\begin{verbatim}
%\def\LyX{%
%  \texorpdfstring{%
%    L\kern-.1667em\lower.25em\hbox{Y}\kern-.125emX\@%
%  }{%
%    LyX%
%  }%
%}
%\end{verbatim}
%\end{quote}
%    \begin{macrocode}
\def\HoLogo@LyX#1{%
  L%
  \kern-.1667em%
  \lower.25em\hbox{Y}%
  \kern-.125em%
  X%
  \HOLOGO@SpaceFactor
}
%    \end{macrocode}
%    \end{macro}
%    \begin{macro}{\HoLogoHtml@LyX}
%    \begin{macrocode}
\def\HoLogoHtml@LyX#1{%
  \HoLogoCss@LyX
  \HOLOGO@Span{LyX}{%
    L%
    \HOLOGO@Span{y}{Y}%
    X%
  }%
}
%    \end{macrocode}
%    \end{macro}
%    \begin{macro}{\HoLogoCss@LyX}
%    \begin{macrocode}
\def\HoLogoCss@LyX{%
  \Css{%
    span.HoLogo-LyX span.HoLogo-y{%
      position:relative;%
      top:.25em;%
      margin-left:-.1667em;%
      margin-right:-.125em;%
      text-decoration:none;%
    }%
  }%
  \global\let\HoLogoCss@LyX\relax
}
%    \end{macrocode}
%    \end{macro}
%
% \subsubsection{\hologo{NTS}}
%
%    \begin{macro}{\HoLogo@NTS}
%    Definition for \hologo{NTS} can be found in
%    package \xpackage{etex\textunderscore man} for the \hologo{eTeX} manual \cite{etexman}
%    and in package \xpackage{dtklogos} \cite{dtklogos}:
%\begin{quote}
%\begin{verbatim}
%\def\NTS{%
%  \leavevmode
%  \hbox{%
%    $%
%      \cal N%
%      \kern-0.35em%
%      \lower0.5ex\hbox{$\cal T$}%
%      \kern-0.2em%
%      S%
%    $%
%  }%
%}
%\end{verbatim}
%\end{quote}
%    \begin{macrocode}
\def\HoLogo@NTS#1{%
  \HoLogoFont@font{NTS}{sy}{%
    N\/%
    \kern-.35em%
    \lower.5ex\hbox{T\/}%
    \kern-.2em%
    S\/%
  }%
  \HOLOGO@SpaceFactor
}
%    \end{macrocode}
%    \end{macro}
%
% \subsubsection{\Hologo{TTH} (\hologo{TeX} to HTML translator)}
%
%    Source: \url{http://hutchinson.belmont.ma.us/tth/}
%    In the HTML source the second `T' is printed as subscript.
%\begin{quote}
%\begin{verbatim}
%T<sub>T</sub>H
%\end{verbatim}
%\end{quote}
%    \begin{macro}{\HoLogo@TTH}
%    \begin{macrocode}
\def\HoLogo@TTH#1{%
  \ltx@mbox{%
    T\HOLOGO@SubScript{T}H%
  }%
  \HOLOGO@SpaceFactor
}
%    \end{macrocode}
%    \end{macro}
%
%    \begin{macro}{\HoLogoHtml@TTH}
%    \begin{macrocode}
\def\HoLogoHtml@TTH#1{%
  T\HCode{<sub>}T\HCode{</sub>}H%
}
%    \end{macrocode}
%    \end{macro}
%
% \subsubsection{\Hologo{HanTheThanh}}
%
%    Partial source: Package \xpackage{dtklogos}.
%    The double accent is U+1EBF (latin small letter e with circumflex
%    and acute).
%    \begin{macro}{\HoLogo@HanTheThanh}
%    \begin{macrocode}
\def\HoLogo@HanTheThanh#1{%
  \ltx@mbox{H\`an}%
  \HOLOGO@space
  \ltx@mbox{%
    Th%
    \HOLOGO@IfCharExists{"1EBF}{%
      \char"1EBF\relax
    }{%
      \^e\hbox to 0pt{\hss\raise .5ex\hbox{\'{}}}%
    }%
  }%
  \HOLOGO@space
  \ltx@mbox{Th\`anh}%
}
%    \end{macrocode}
%    \end{macro}
%    \begin{macro}{\HoLogoBkm@HanTheThanh}
%    \begin{macrocode}
\def\HoLogoBkm@HanTheThanh#1{%
  H\`an %
  Th\HOLOGO@PdfdocUnicode{\^e}{\9036\277} %
  Th\`anh%
}
%    \end{macrocode}
%    \end{macro}
%    \begin{macro}{\HoLogoHtml@HanTheThanh}
%    \begin{macrocode}
\def\HoLogoHtml@HanTheThanh#1{%
  H\`an %
  Th\HCode{&\ltx@hashchar x1ebf;} %
  Th\`anh%
}
%    \end{macrocode}
%    \end{macro}
%
% \subsection{Driver detection}
%
%    \begin{macrocode}
\HOLOGO@IfExists\InputIfFileExists{%
  \InputIfFileExists{hologo.cfg}{}{}%
}{%
  \ltx@IfUndefined{pdf@filesize}{%
    \def\HOLOGO@InputIfExists{%
      \openin\HOLOGO@temp=hologo.cfg\relax
      \ifeof\HOLOGO@temp
        \closein\HOLOGO@temp
      \else
        \closein\HOLOGO@temp
        \begingroup
          \def\x{LaTeX2e}%
        \expandafter\endgroup
        \ifx\fmtname\x
          % \iffalse meta-comment
%
% File: hologo.dtx
% Version: 2016/05/12 v1.11
% Info: A logo collection with bookmark support
%
% Copyright (C) 2010-2012 by
%    Heiko Oberdiek <heiko.oberdiek at googlemail.com>
%
% This work may be distributed and/or modified under the
% conditions of the LaTeX Project Public License, either
% version 1.3c of this license or (at your option) any later
% version. This version of this license is in
%    http://www.latex-project.org/lppl/lppl-1-3c.txt
% and the latest version of this license is in
%    http://www.latex-project.org/lppl.txt
% and version 1.3 or later is part of all distributions of
% LaTeX version 2005/12/01 or later.
%
% This work has the LPPL maintenance status "maintained".
%
% This Current Maintainer of this work is Heiko Oberdiek.
%
% The Base Interpreter refers to any `TeX-Format',
% because some files are installed in TDS:tex/generic//.
%
% This work consists of the main source file hologo.dtx
% and the derived files
%    hologo.sty, hologo.pdf, hologo.ins, hologo.drv, hologo-example.tex,
%    hologo-test1.tex, hologo-test-spacefactor.tex,
%    hologo-test-list.tex.
%
% Distribution:
%    CTAN:macros/latex/contrib/oberdiek/hologo.dtx
%    CTAN:macros/latex/contrib/oberdiek/hologo.pdf
%
% Unpacking:
%    (a) If hologo.ins is present:
%           tex hologo.ins
%    (b) Without hologo.ins:
%           tex hologo.dtx
%    (c) If you insist on using LaTeX
%           latex \let\install=y% \iffalse meta-comment
%
% File: hologo.dtx
% Version: 2016/05/12 v1.11
% Info: A logo collection with bookmark support
%
% Copyright (C) 2010-2012 by
%    Heiko Oberdiek <heiko.oberdiek at googlemail.com>
%
% This work may be distributed and/or modified under the
% conditions of the LaTeX Project Public License, either
% version 1.3c of this license or (at your option) any later
% version. This version of this license is in
%    http://www.latex-project.org/lppl/lppl-1-3c.txt
% and the latest version of this license is in
%    http://www.latex-project.org/lppl.txt
% and version 1.3 or later is part of all distributions of
% LaTeX version 2005/12/01 or later.
%
% This work has the LPPL maintenance status "maintained".
%
% This Current Maintainer of this work is Heiko Oberdiek.
%
% The Base Interpreter refers to any `TeX-Format',
% because some files are installed in TDS:tex/generic//.
%
% This work consists of the main source file hologo.dtx
% and the derived files
%    hologo.sty, hologo.pdf, hologo.ins, hologo.drv, hologo-example.tex,
%    hologo-test1.tex, hologo-test-spacefactor.tex,
%    hologo-test-list.tex.
%
% Distribution:
%    CTAN:macros/latex/contrib/oberdiek/hologo.dtx
%    CTAN:macros/latex/contrib/oberdiek/hologo.pdf
%
% Unpacking:
%    (a) If hologo.ins is present:
%           tex hologo.ins
%    (b) Without hologo.ins:
%           tex hologo.dtx
%    (c) If you insist on using LaTeX
%           latex \let\install=y% \iffalse meta-comment
%
% File: hologo.dtx
% Version: 2016/05/12 v1.11
% Info: A logo collection with bookmark support
%
% Copyright (C) 2010-2012 by
%    Heiko Oberdiek <heiko.oberdiek at googlemail.com>
%
% This work may be distributed and/or modified under the
% conditions of the LaTeX Project Public License, either
% version 1.3c of this license or (at your option) any later
% version. This version of this license is in
%    http://www.latex-project.org/lppl/lppl-1-3c.txt
% and the latest version of this license is in
%    http://www.latex-project.org/lppl.txt
% and version 1.3 or later is part of all distributions of
% LaTeX version 2005/12/01 or later.
%
% This work has the LPPL maintenance status "maintained".
%
% This Current Maintainer of this work is Heiko Oberdiek.
%
% The Base Interpreter refers to any `TeX-Format',
% because some files are installed in TDS:tex/generic//.
%
% This work consists of the main source file hologo.dtx
% and the derived files
%    hologo.sty, hologo.pdf, hologo.ins, hologo.drv, hologo-example.tex,
%    hologo-test1.tex, hologo-test-spacefactor.tex,
%    hologo-test-list.tex.
%
% Distribution:
%    CTAN:macros/latex/contrib/oberdiek/hologo.dtx
%    CTAN:macros/latex/contrib/oberdiek/hologo.pdf
%
% Unpacking:
%    (a) If hologo.ins is present:
%           tex hologo.ins
%    (b) Without hologo.ins:
%           tex hologo.dtx
%    (c) If you insist on using LaTeX
%           latex \let\install=y\input{hologo.dtx}
%        (quote the arguments according to the demands of your shell)
%
% Documentation:
%    (a) If hologo.drv is present:
%           latex hologo.drv
%    (b) Without hologo.drv:
%           latex hologo.dtx; ...
%    The class ltxdoc loads the configuration file ltxdoc.cfg
%    if available. Here you can specify further options, e.g.
%    use A4 as paper format:
%       \PassOptionsToClass{a4paper}{article}
%
%    Programm calls to get the documentation (example):
%       pdflatex hologo.dtx
%       makeindex -s gind.ist hologo.idx
%       pdflatex hologo.dtx
%       makeindex -s gind.ist hologo.idx
%       pdflatex hologo.dtx
%
% Installation:
%    TDS:tex/generic/oberdiek/hologo.sty
%    TDS:doc/latex/oberdiek/hologo.pdf
%    TDS:doc/latex/oberdiek/example/hologo-example.tex
%    TDS:doc/latex/oberdiek/test/hologo-test1.tex
%    TDS:doc/latex/oberdiek/test/hologo-test-spacefactor.tex
%    TDS:doc/latex/oberdiek/test/hologo-test-list.tex
%    TDS:source/latex/oberdiek/hologo.dtx
%
%<*ignore>
\begingroup
  \catcode123=1 %
  \catcode125=2 %
  \def\x{LaTeX2e}%
\expandafter\endgroup
\ifcase 0\ifx\install y1\fi\expandafter
         \ifx\csname processbatchFile\endcsname\relax\else1\fi
         \ifx\fmtname\x\else 1\fi\relax
\else\csname fi\endcsname
%</ignore>
%<*install>
\input docstrip.tex
\Msg{************************************************************************}
\Msg{* Installation}
\Msg{* Package: hologo 2016/05/12 v1.11 A logo collection with bookmark support (HO)}
\Msg{************************************************************************}

\keepsilent
\askforoverwritefalse

\let\MetaPrefix\relax
\preamble

This is a generated file.

Project: hologo
Version: 2016/05/12 v1.11

Copyright (C) 2010-2012 by
   Heiko Oberdiek <heiko.oberdiek at googlemail.com>

This work may be distributed and/or modified under the
conditions of the LaTeX Project Public License, either
version 1.3c of this license or (at your option) any later
version. This version of this license is in
   http://www.latex-project.org/lppl/lppl-1-3c.txt
and the latest version of this license is in
   http://www.latex-project.org/lppl.txt
and version 1.3 or later is part of all distributions of
LaTeX version 2005/12/01 or later.

This work has the LPPL maintenance status "maintained".

This Current Maintainer of this work is Heiko Oberdiek.

The Base Interpreter refers to any `TeX-Format',
because some files are installed in TDS:tex/generic//.

This work consists of the main source file hologo.dtx
and the derived files
   hologo.sty, hologo.pdf, hologo.ins, hologo.drv, hologo-example.tex,
   hologo-test1.tex, hologo-test-spacefactor.tex,
   hologo-test-list.tex.

\endpreamble
\let\MetaPrefix\DoubleperCent

\generate{%
  \file{hologo.ins}{\from{hologo.dtx}{install}}%
  \file{hologo.drv}{\from{hologo.dtx}{driver}}%
  \usedir{tex/generic/oberdiek}%
  \file{hologo.sty}{\from{hologo.dtx}{package}}%
  \usedir{doc/latex/oberdiek/example}%
  \file{hologo-example.tex}{\from{hologo.dtx}{example}}%
  \usedir{doc/latex/oberdiek/test}%
  \file{hologo-test1.tex}{\from{hologo.dtx}{test1}}%
  \file{hologo-test-spacefactor.tex}{\from{hologo.dtx}{test-spacefactor}}%
  \file{hologo-test-list.tex}{\from{hologo.dtx}{test-list}}%
  \nopreamble
  \nopostamble
  \usedir{source/latex/oberdiek/catalogue}%
  \file{hologo.xml}{\from{hologo.dtx}{catalogue}}%
}

\catcode32=13\relax% active space
\let =\space%
\Msg{************************************************************************}
\Msg{*}
\Msg{* To finish the installation you have to move the following}
\Msg{* file into a directory searched by TeX:}
\Msg{*}
\Msg{*     hologo.sty}
\Msg{*}
\Msg{* To produce the documentation run the file `hologo.drv'}
\Msg{* through LaTeX.}
\Msg{*}
\Msg{* Happy TeXing!}
\Msg{*}
\Msg{************************************************************************}

\endbatchfile
%</install>
%<*ignore>
\fi
%</ignore>
%<*driver>
\NeedsTeXFormat{LaTeX2e}
\ProvidesFile{hologo.drv}%
  [2016/05/12 v1.11 A logo collection with bookmark support (HO)]%
\documentclass{ltxdoc}
\usepackage{holtxdoc}[2011/11/22]
\usepackage{hologo}[2016/05/12]
\usepackage{longtable}
\usepackage{array}
\usepackage{paralist}
%\usepackage[T1]{fontenc}
%\usepackage{lmodern}
\begin{document}
  \DocInput{hologo.dtx}%
\end{document}
%</driver>
% \fi
%
%
% \CharacterTable
%  {Upper-case    \A\B\C\D\E\F\G\H\I\J\K\L\M\N\O\P\Q\R\S\T\U\V\W\X\Y\Z
%   Lower-case    \a\b\c\d\e\f\g\h\i\j\k\l\m\n\o\p\q\r\s\t\u\v\w\x\y\z
%   Digits        \0\1\2\3\4\5\6\7\8\9
%   Exclamation   \!     Double quote  \"     Hash (number) \#
%   Dollar        \$     Percent       \%     Ampersand     \&
%   Acute accent  \'     Left paren    \(     Right paren   \)
%   Asterisk      \*     Plus          \+     Comma         \,
%   Minus         \-     Point         \.     Solidus       \/
%   Colon         \:     Semicolon     \;     Less than     \<
%   Equals        \=     Greater than  \>     Question mark \?
%   Commercial at \@     Left bracket  \[     Backslash     \\
%   Right bracket \]     Circumflex    \^     Underscore    \_
%   Grave accent  \`     Left brace    \{     Vertical bar  \|
%   Right brace   \}     Tilde         \~}
%
% \GetFileInfo{hologo.drv}
%
% \title{The \xpackage{hologo} package}
% \date{2016/05/12 v1.11}
% \author{Heiko Oberdiek\\\xemail{heiko.oberdiek at googlemail.com}}
%
% \maketitle
%
% \begin{abstract}
% This package starts a collection of logos with support for bookmarks
% strings.
% \end{abstract}
%
% \tableofcontents
%
% \section{Documentation}
%
% \subsection{Logo macros}
%
% \begin{declcs}{hologo} \M{name}
% \end{declcs}
% Macro \cs{hologo} sets the logo with name \meta{name}.
% The following table shows the supported names.
%
% \begingroup
%   \def\hologoEntry#1#2#3{^^A
%     #1&#2&\hologoLogoSetup{#1}{variant=#2}\hologo{#1}&#3\tabularnewline
%   }
%   \begin{longtable}{>{\ttfamily}l>{\ttfamily}lll}
%     \rmfamily\bfseries{name} & \rmfamily\bfseries variant
%     & \bfseries logo & \bfseries since\\
%     \hline
%     \endhead
%     \hologoList
%   \end{longtable}
% \endgroup
%
% \begin{declcs}{Hologo} \M{name}
% \end{declcs}
% Macro \cs{Hologo} starts the logo \meta{name} with an uppercase
% letter. As an exception small greek letters are not converted
% to uppercase. Examples, see \hologo{eTeX} and \hologo{ExTeX}.
%
% \subsection{Setup macros}
%
% The package does not support package options, but the following
% setup macros can be used to set options.
%
% \begin{declcs}{hologoSetup} \M{key value list}
% \end{declcs}
% Macro \cs{hologoSetup} sets global options.
%
% \begin{declcs}{hologoLogoSetup} \M{logo} \M{key value list}
% \end{declcs}
% Some options can also be used to configure a logo.
% These settings take precedence over global option settings.
%
% \subsection{Options}\label{sec:options}
%
% There are boolean and string options:
% \begin{description}
% \item[Boolean option:]
% It takes |true| or |false|
% as value. If the value is omitted, then |true| is used.
% \item[String option:]
% A value must be given as string. (But the string might be empty.)
% \end{description}
% The following options can be used both in \cs{hologoSetup}
% and \cs{hologoLogoSetup}:
% \begin{description}
% \def\entry#1{\item[\xoption{#1}:]}
% \entry{break}
%   enables or disables line breaks inside the logo. This setting is
%   refined by options \xoption{hyphenbreak}, \xoption{spacebreak}
%   or \xoption{discretionarybreak}.
%   Default is |false|.
% \entry{hyphenbreak}
%   enables or disables the line break right after the hyphen character.
% \entry{spacebreak}
%   enables or disables line breaks at space characters.
% \entry{discretionarybreak}
%   enables or disables line breaks at hyphenation points
%   (inserted by \cs{-}).
% \end{description}
% Macro \cs{hologoLogoSetup} also knows:
% \begin{description}
% \item[\xoption{variant}:]
%   This is a string option. It specifies a variant of a logo that
%   must exist. An empty string selects the package default variant.
% \end{description}
% Example:
% \begin{quote}
%   |\hologoSetup{break=false}|\\
%   |\hologoLogoSetup{plainTeX}{variant=hyphen,hyphenbreak}|\\
%   Then ``plain-\TeX'' contains one break point after the hyphen.
% \end{quote}
%
% \subsection{Driver options}
%
% Sometimes graphical operations are needed to construct some
% glyphs (e.g.\ \hologo{XeTeX}). If package \xpackage{graphics}
% or package \xpackage{pgf} are found, then the macros are taken
% from there. Otherwise the packge defines its own operations
% and therefore needs the driver information. Many drivers are
% detected automatically (\hologo{pdfTeX}/\hologo{LuaTeX}
% in PDF mode, \hologo{XeTeX}, \hologo{VTeX}). These have precedence
% over a driver option. The driver can be given as package option
% or using \cs{hologoDriverSetup}.
% The following list contains the recognized driver options:
% \begin{itemize}
% \item \xoption{pdftex}, \xoption{luatex}
% \item \xoption{dvipdfm}, \xoption{dvipdfmx}
% \item \xoption{dvips}, \xoption{dvipsone}, \xoption{xdvi}
% \item \xoption{xetex}
% \item \xoption{vtex}
% \end{itemize}
% The left driver of a line is the driver name that is used internally.
% The following names are aliases for drivers that use the
% same method. Therefore the entry in the \xext{log} file for
% the used driver prints the internally used driver name.
% \begin{description}
% \item[\xoption{driverfallback}:]
%   This option expects a driver that is used,
%   if the driver could not be detected automatically.
% \end{description}
%
% \begin{declcs}{hologoDriverSetup} \M{driver option}
% \end{declcs}
% The driver can also be configured after package loading
% using \cs{hologoDriverSetup}, also the way for \hologo{plainTeX}
% to setup the driver.
%
% \subsection{Font setup}
%
% Some logos require a special font, but should also be usable by
% \hologo{plainTeX}. Therefore the package provides some ways
% to influence the font settings. The options below
% take font settings as values. Both font commands
% such as \cs{sffamily} and macros that take one argument
% like \cs{textsf} can be used.
%
% \begin{declcs}{hologoFontSetup} \M{key value list}
% \end{declcs}
% Macro \cs{hologoFontSetup} sets the fonts for all logos.
% Supported keys:
% \begin{description}
% \def\entry#1{\item[\xoption{#1}:]}
% \entry{general}
%   This font is used for all logos. The default is empty.
%   That means no special font is used.
% \entry{bibsf}
%   This font is used for
%   {\hologoLogoSetup{BibTeX}{variant=sf}\hologo{BibTeX}}
%   with variant \xoption{sf}.
% \entry{rm}
%   This font is a serif font. It is used for \hologo{ExTeX}.
% \entry{sc}
%   This font specifies a small caps font. It is used for
%   {\hologoLogoSetup{BibTeX}{variant=sc}\hologo{BibTeX}}
%   with variant \xoption{sc}.
% \entry{sf}
%   This font specifies a sans serif font. The default
%   is \cs{sffamily}, then \cs{sf} is tried. Otherwise
%   a warning is given. It is used by \hologo{KOMAScript}.
% \entry{sy}
%   This is the font for math symbols (e.g. cmsy).
%   It is used by \hologo{AmS}, \hologo{NTS}, \hologo{ExTeX}.
% \entry{logo}
%   \hologo{METAFONT} and \hologo{METAPOST} are using that font.
%   In \hologo{LaTeX} \cs{logofamily} is used and
%   the definitions of package \xpackage{mflogo} are used
%   if the package is not loaded.
%   Otherwise the \cs{tenlogo} is used and defined
%   if it does not already exists.
% \end{description}
%
% \begin{declcs}{hologoLogoFontSetup} \M{logo} \M{key value list}
% \end{declcs}
% Fonts can also be set for a logo or logo component separately,
% see the following list.
% The keys are the same as for \cs{hologoFontSetup}.
%
% \begin{longtable}{>{\ttfamily}l>{\sffamily}ll}
%   \meta{logo} & keys & result\\
%   \hline
%   \endhead
%   BibTeX & bibsf & {\hologoLogoSetup{BibTeX}{variant=sf}\hologo{BibTeX}}\\[.5ex]
%   BibTeX & sc & {\hologoLogoSetup{BibTeX}{variant=sc}\hologo{BibTeX}}\\[.5ex]
%   ExTeX & rm & \hologo{ExTeX}\\
%   SliTeX & rm & \hologo{SliTeX}\\[.5ex]
%   AmS & sy & \hologo{AmS}\\
%   ExTeX & sy & \hologo{ExTeX}\\
%   NTS & sy & \hologo{NTS}\\[.5ex]
%   KOMAScript & sf & \hologo{KOMAScript}\\[.5ex]
%   METAFONT & logo & \hologo{METAFONT}\\
%   METAPOST & logo & \hologo{METAPOST}\\[.5ex]
%   SliTeX & sc \hologo{SliTeX}
% \end{longtable}
%
% \subsubsection{Font order}
%
% For all logos the font \xoption{general} is applied first.
% Example:
%\begin{quote}
%|\hologoFontSetup{general=\color{red}}|
%\end{quote}
% will print red logos.
% Then if the font uses a special font \xoption{sf}, for example,
% the font is applied that is setup by \cs{hologoLogoFontSetup}.
% If this font is not setup, then the common font setup
% by \cs{hologoFontSetup} is used. Otherwise a warning is given,
% that there is no font configured.
%
% \subsection{Additional user macros}
%
% Usually a variant of a logo is configured by using
% \cs{hologoLogoSetup}, because it is bad style to mix
% different variants of the same logo in the same text.
% There the following macros are a convenience for testing.
%
% \begin{declcs}{hologoVariant} \M{name} \M{variant}\\
%   \cs{HologoVariant} \M{name} \M{variant}
% \end{declcs}
% Logo \meta{name} is set using \meta{variant} that specifies
% explicitely which variant of the macro is used. If the argument
% is empty, then the default form of the logo is used
% (configurable by \cs{hologoLogoSetup}).
%
% \cs{HologoVariant} is used if the logo is set in a context
% that needs an uppercase first letter (beginning of a sentence, \dots).
%
% \begin{declcs}{hologoList}\\
%   \cs{hologoEntry} \M{logo} \M{variant} \M{since}
% \end{declcs}
% Macro \cs{hologoList} contains all logos that are provided
% by the package including variants. The list consists of calls
% of \cs{hologoEntry} with three arguments starting with the
% logo name \meta{logo} and its variant \meta{variant}. An empty
% variant means the current default. Argument \meta{since} specifies
% with version of the package \xpackage{hologo} is needed to get
% the logo. If the logo is fixed, then the date gets updated.
% Therefore the date \meta{since} is not exactly the date of
% the first introduction, but rather the date of the latest fix.
%
% Before \cs{hologoList} can be used, macro \cs{hologoEntry} needs
% a definition. The example file in section \ref{sec:example}
% shows applications of \cs{hologoList}.
%
% \subsection{Supported contexts}
%
% Macros \cs{hologo} and friends support special contexts:
% \begin{itemize}
% \item \hologo{LaTeX}'s protection mechanism.
% \item Bookmarks of package \xpackage{hyperref}.
% \item Package \xpackage{tex4ht}.
% \item The macros can be used inside \cs{csname} constructs,
%   if \cs{ifincsname} is available (\hologo{pdfTeX}, \hologo{XeTeX},
%   \hologo{LuaTeX}).
% \end{itemize}
%
% \subsection{Example}
% \label{sec:example}
%
% The following example prints the logos in different fonts.
%    \begin{macrocode}
%<*example>
%<<verbatim
\NeedsTeXFormat{LaTeX2e}
\documentclass[a4paper]{article}
\usepackage[
  hmargin=20mm,
  vmargin=20mm,
]{geometry}
\pagestyle{empty}
\usepackage{hologo}[2016/05/12]
\usepackage{longtable}
\usepackage{array}
\setlength{\extrarowheight}{2pt}
\usepackage[T1]{fontenc}
\usepackage{lmodern}
\usepackage{pdflscape}
\usepackage[
  pdfencoding=auto,
]{hyperref}
\hypersetup{
  pdfauthor={Heiko Oberdiek},
  pdftitle={Example for package `hologo'},
  pdfsubject={Logos with fonts lmr, lmss, qtm, qpl, qhv},
}
\usepackage{bookmark}

% Print the logo list on the console

\begingroup
  \typeout{}%
  \typeout{*** Begin of logo list ***}%
  \newcommand*{\hologoEntry}[3]{%
    \typeout{#1 \ifx\\#2\\\else(#2) \fi[#3]}%
  }%
  \hologoList
  \typeout{*** End of logo list ***}%
  \typeout{}%
\endgroup

\begin{document}
\begin{landscape}

  \section{Example file for package `hologo'}

  % Table for font names

  \begin{longtable}{>{\bfseries}ll}
    \textbf{font} & \textbf{Font name}\\
    \hline
    lmr & Latin Modern Roman\\
    lmss & Latin Modern Sans\\
    qtm & \TeX\ Gyre Termes\\
    qhv & \TeX\ Gyre Heros\\
    qpl & \TeX\ Gyre Pagella\\
  \end{longtable}

  % Logo list with logos in different fonts

  \begingroup
    \newcommand*{\SetVariant}[2]{%
      \ifx\\#2\\%
      \else
        \hologoLogoSetup{#1}{variant=#2}%
      \fi
    }%
    \newcommand*{\hologoEntry}[3]{%
      \SetVariant{#1}{#2}%
      \raisebox{1em}[0pt][0pt]{\hypertarget{#1@#2}{}}%
      \bookmark[%
        dest={#1@#2},%
      ]{%
        #1\ifx\\#2\\\else\space(#2)\fi: \Hologo{#1}, \hologo{#1} %
        [Unicode]%
      }%
      \hypersetup{unicode=false}%
      \bookmark[%
        dest={#1@#2},%
      ]{%
        #1\ifx\\#2\\\else\space(#2)\fi: \Hologo{#1}, \hologo{#1} %
        [PDFDocEncoding]%
      }%
      \texttt{#1}%
      &%
      \texttt{#2}%
      &%
      \Hologo{#1}%
      &%
      \SetVariant{#1}{#2}%
      \hologo{#1}%
      &%
      \SetVariant{#1}{#2}%
      \fontfamily{qtm}\selectfont
      \hologo{#1}%
      &%
      \SetVariant{#1}{#2}%
      \fontfamily{qpl}\selectfont
      \hologo{#1}%
      &%
      \SetVariant{#1}{#2}%
      \textsf{\hologo{#1}}%
      &%
      \SetVariant{#1}{#2}%
      \fontfamily{qhv}\selectfont
      \hologo{#1}%
      \tabularnewline
    }%
    \begin{longtable}{llllllll}%
      \textbf{\textit{logo}} & \textbf{\textit{variant}} &
      \texttt{\string\Hologo} &
      \textbf{lmr} & \textbf{qtm} & \textbf{qpl} &
      \textbf{lmss} & \textbf{qhv}
      \tabularnewline
      \hline
      \endhead
      \hologoList
    \end{longtable}%
  \endgroup

\end{landscape}
\end{document}
%verbatim
%</example>
%    \end{macrocode}
%
% \StopEventually{
% }
%
% \section{Implementation}
%    \begin{macrocode}
%<*package>
%    \end{macrocode}
%    Reload check, especially if the package is not used with \LaTeX.
%    \begin{macrocode}
\begingroup\catcode61\catcode48\catcode32=10\relax%
  \catcode13=5 % ^^M
  \endlinechar=13 %
  \catcode35=6 % #
  \catcode39=12 % '
  \catcode44=12 % ,
  \catcode45=12 % -
  \catcode46=12 % .
  \catcode58=12 % :
  \catcode64=11 % @
  \catcode123=1 % {
  \catcode125=2 % }
  \expandafter\let\expandafter\x\csname ver@hologo.sty\endcsname
  \ifx\x\relax % plain-TeX, first loading
  \else
    \def\empty{}%
    \ifx\x\empty % LaTeX, first loading,
      % variable is initialized, but \ProvidesPackage not yet seen
    \else
      \expandafter\ifx\csname PackageInfo\endcsname\relax
        \def\x#1#2{%
          \immediate\write-1{Package #1 Info: #2.}%
        }%
      \else
        \def\x#1#2{\PackageInfo{#1}{#2, stopped}}%
      \fi
      \x{hologo}{The package is already loaded}%
      \aftergroup\endinput
    \fi
  \fi
\endgroup%
%    \end{macrocode}
%    Package identification:
%    \begin{macrocode}
\begingroup\catcode61\catcode48\catcode32=10\relax%
  \catcode13=5 % ^^M
  \endlinechar=13 %
  \catcode35=6 % #
  \catcode39=12 % '
  \catcode40=12 % (
  \catcode41=12 % )
  \catcode44=12 % ,
  \catcode45=12 % -
  \catcode46=12 % .
  \catcode47=12 % /
  \catcode58=12 % :
  \catcode64=11 % @
  \catcode91=12 % [
  \catcode93=12 % ]
  \catcode123=1 % {
  \catcode125=2 % }
  \expandafter\ifx\csname ProvidesPackage\endcsname\relax
    \def\x#1#2#3[#4]{\endgroup
      \immediate\write-1{Package: #3 #4}%
      \xdef#1{#4}%
    }%
  \else
    \def\x#1#2[#3]{\endgroup
      #2[{#3}]%
      \ifx#1\@undefined
        \xdef#1{#3}%
      \fi
      \ifx#1\relax
        \xdef#1{#3}%
      \fi
    }%
  \fi
\expandafter\x\csname ver@hologo.sty\endcsname
\ProvidesPackage{hologo}%
  [2016/05/12 v1.11 A logo collection with bookmark support (HO)]%
%    \end{macrocode}
%
%    \begin{macrocode}
\begingroup\catcode61\catcode48\catcode32=10\relax%
  \catcode13=5 % ^^M
  \endlinechar=13 %
  \catcode123=1 % {
  \catcode125=2 % }
  \catcode64=11 % @
  \def\x{\endgroup
    \expandafter\edef\csname HOLOGO@AtEnd\endcsname{%
      \endlinechar=\the\endlinechar\relax
      \catcode13=\the\catcode13\relax
      \catcode32=\the\catcode32\relax
      \catcode35=\the\catcode35\relax
      \catcode61=\the\catcode61\relax
      \catcode64=\the\catcode64\relax
      \catcode123=\the\catcode123\relax
      \catcode125=\the\catcode125\relax
    }%
  }%
\x\catcode61\catcode48\catcode32=10\relax%
\catcode13=5 % ^^M
\endlinechar=13 %
\catcode35=6 % #
\catcode64=11 % @
\catcode123=1 % {
\catcode125=2 % }
\def\TMP@EnsureCode#1#2{%
  \edef\HOLOGO@AtEnd{%
    \HOLOGO@AtEnd
    \catcode#1=\the\catcode#1\relax
  }%
  \catcode#1=#2\relax
}
\TMP@EnsureCode{10}{12}% ^^J
\TMP@EnsureCode{33}{12}% !
\TMP@EnsureCode{34}{12}% "
\TMP@EnsureCode{36}{3}% $
\TMP@EnsureCode{38}{4}% &
\TMP@EnsureCode{39}{12}% '
\TMP@EnsureCode{40}{12}% (
\TMP@EnsureCode{41}{12}% )
\TMP@EnsureCode{42}{12}% *
\TMP@EnsureCode{43}{12}% +
\TMP@EnsureCode{44}{12}% ,
\TMP@EnsureCode{45}{12}% -
\TMP@EnsureCode{46}{12}% .
\TMP@EnsureCode{47}{12}% /
\TMP@EnsureCode{58}{12}% :
\TMP@EnsureCode{59}{12}% ;
\TMP@EnsureCode{60}{12}% <
\TMP@EnsureCode{62}{12}% >
\TMP@EnsureCode{63}{12}% ?
\TMP@EnsureCode{91}{12}% [
\TMP@EnsureCode{93}{12}% ]
\TMP@EnsureCode{94}{7}% ^ (superscript)
\TMP@EnsureCode{95}{8}% _ (subscript)
\TMP@EnsureCode{96}{12}% `
\TMP@EnsureCode{124}{12}% |
\edef\HOLOGO@AtEnd{%
  \HOLOGO@AtEnd
  \escapechar\the\escapechar\relax
  \noexpand\endinput
}
\escapechar=92 %
%    \end{macrocode}
%
% \subsection{Logo list}
%
%    \begin{macro}{\hologoList}
%    \begin{macrocode}
\def\hologoList{%
  \hologoEntry{(La)TeX}{}{2011/10/01}%
  \hologoEntry{AmSLaTeX}{}{2010/04/16}%
  \hologoEntry{AmSTeX}{}{2010/04/16}%
  \hologoEntry{biber}{}{2011/10/01}%
  \hologoEntry{BibTeX}{}{2011/10/01}%
  \hologoEntry{BibTeX}{sf}{2011/10/01}%
  \hologoEntry{BibTeX}{sc}{2011/10/01}%
  \hologoEntry{BibTeX8}{}{2011/11/22}%
  \hologoEntry{ConTeXt}{}{2011/03/25}%
  \hologoEntry{ConTeXt}{narrow}{2011/03/25}%
  \hologoEntry{ConTeXt}{simple}{2011/03/25}%
  \hologoEntry{emTeX}{}{2010/04/26}%
  \hologoEntry{eTeX}{}{2010/04/08}%
  \hologoEntry{ExTeX}{}{2011/10/01}%
  \hologoEntry{HanTheThanh}{}{2011/11/29}%
  \hologoEntry{iniTeX}{}{2011/10/01}%
  \hologoEntry{KOMAScript}{}{2011/10/01}%
  \hologoEntry{La}{}{2010/05/08}%
  \hologoEntry{LaTeX}{}{2010/04/08}%
  \hologoEntry{LaTeX2e}{}{2010/04/08}%
  \hologoEntry{LaTeX3}{}{2010/04/24}%
  \hologoEntry{LaTeXe}{}{2010/04/08}%
  \hologoEntry{LaTeXML}{}{2011/11/22}%
  \hologoEntry{LaTeXTeX}{}{2011/10/01}%
  \hologoEntry{LuaLaTeX}{}{2010/04/08}%
  \hologoEntry{LuaTeX}{}{2010/04/08}%
  \hologoEntry{LyX}{}{2011/10/01}%
  \hologoEntry{METAFONT}{}{2011/10/01}%
  \hologoEntry{MetaFun}{}{2011/10/01}%
  \hologoEntry{METAPOST}{}{2011/10/01}%
  \hologoEntry{MetaPost}{}{2011/10/01}%
  \hologoEntry{MiKTeX}{}{2011/10/01}%
  \hologoEntry{NTS}{}{2011/10/01}%
  \hologoEntry{OzMF}{}{2011/10/01}%
  \hologoEntry{OzMP}{}{2011/10/01}%
  \hologoEntry{OzTeX}{}{2011/10/01}%
  \hologoEntry{OzTtH}{}{2011/10/01}%
  \hologoEntry{PCTeX}{}{2011/10/01}%
  \hologoEntry{pdfTeX}{}{2011/10/01}%
  \hologoEntry{pdfLaTeX}{}{2011/10/01}%
  \hologoEntry{PiC}{}{2011/10/01}%
  \hologoEntry{PiCTeX}{}{2011/10/01}%
  \hologoEntry{plainTeX}{}{2010/04/08}%
  \hologoEntry{plainTeX}{space}{2010/04/16}%
  \hologoEntry{plainTeX}{hyphen}{2010/04/16}%
  \hologoEntry{plainTeX}{runtogether}{2010/04/16}%
  \hologoEntry{SageTeX}{}{2011/11/22}%
  \hologoEntry{SLiTeX}{}{2011/10/01}%
  \hologoEntry{SLiTeX}{lift}{2011/10/01}%
  \hologoEntry{SLiTeX}{narrow}{2011/10/01}%
  \hologoEntry{SLiTeX}{simple}{2011/10/01}%
  \hologoEntry{SliTeX}{}{2011/10/01}%
  \hologoEntry{SliTeX}{narrow}{2011/10/01}%
  \hologoEntry{SliTeX}{simple}{2011/10/01}%
  \hologoEntry{SliTeX}{lift}{2011/10/01}%
  \hologoEntry{teTeX}{}{2011/10/01}%
  \hologoEntry{TeX}{}{2010/04/08}%
  \hologoEntry{TeX4ht}{}{2011/11/22}%
  \hologoEntry{TTH}{}{2011/11/22}%
  \hologoEntry{virTeX}{}{2011/10/01}%
  \hologoEntry{VTeX}{}{2010/04/24}%
  \hologoEntry{Xe}{}{2010/04/08}%
  \hologoEntry{XeLaTeX}{}{2010/04/08}%
  \hologoEntry{XeTeX}{}{2010/04/08}%
}
%    \end{macrocode}
%    \end{macro}
%
% \subsection{Load resources}
%
%    \begin{macrocode}
\begingroup\expandafter\expandafter\expandafter\endgroup
\expandafter\ifx\csname RequirePackage\endcsname\relax
  \def\TMP@RequirePackage#1[#2]{%
    \begingroup\expandafter\expandafter\expandafter\endgroup
    \expandafter\ifx\csname ver@#1.sty\endcsname\relax
      \input #1.sty\relax
    \fi
  }%
  \TMP@RequirePackage{ltxcmds}[2011/02/04]%
  \TMP@RequirePackage{infwarerr}[2010/04/08]%
  \TMP@RequirePackage{kvsetkeys}[2010/03/01]%
  \TMP@RequirePackage{kvdefinekeys}[2010/03/01]%
  \TMP@RequirePackage{pdftexcmds}[2010/04/01]%
  \TMP@RequirePackage{ifpdf}[2010/01/28]%
  \TMP@RequirePackage{ifluatex}[2010/03/01]%
  \ltx@IfUndefined{newif}{%
    \expandafter\let\csname newif\endcsname\ltx@newif
  }{}%
  \TMP@RequirePackage{ifxetex}[2009/01/23]%
  \TMP@RequirePackage{ifvtex}[2010/03/01]%
\else
  \RequirePackage{ltxcmds}[2011/02/04]%
  \RequirePackage{infwarerr}[2010/04/08]%
  \RequirePackage{kvsetkeys}[2010/03/01]%
  \RequirePackage{kvdefinekeys}[2010/03/01]%
  \RequirePackage{pdftexcmds}[2010/04/01]%
  \RequirePackage{ifpdf}[2010/01/28]%
  \RequirePackage{ifluatex}[2010/03/01]%
  \RequirePackage{ifxetex}[2009/01/23]%
  \RequirePackage{ifvtex}[2010/03/01]%
\fi
%    \end{macrocode}
%
%    \begin{macro}{\HOLOGO@IfDefined}
%    \begin{macrocode}
\def\HOLOGO@IfExists#1{%
  \ifx\@undefined#1%
    \expandafter\ltx@secondoftwo
  \else
    \ifx\relax#1%
      \expandafter\ltx@secondoftwo
    \else
      \expandafter\expandafter\expandafter\ltx@firstoftwo
    \fi
  \fi
}
%    \end{macrocode}
%    \end{macro}
%
% \subsection{Setup macros}
%
%    \begin{macro}{\hologoSetup}
%    \begin{macrocode}
\def\hologoSetup{%
  \let\HOLOGO@name\relax
  \HOLOGO@Setup
}
%    \end{macrocode}
%    \end{macro}
%
%    \begin{macro}{\hologoLogoSetup}
%    \begin{macrocode}
\def\hologoLogoSetup#1{%
  \edef\HOLOGO@name{#1}%
  \ltx@IfUndefined{HoLogo@\HOLOGO@name}{%
    \@PackageError{hologo}{%
      Unknown logo `\HOLOGO@name'%
    }\@ehc
    \ltx@gobble
  }{%
    \HOLOGO@Setup
  }%
}
%    \end{macrocode}
%    \end{macro}
%
%    \begin{macro}{\HOLOGO@Setup}
%    \begin{macrocode}
\def\HOLOGO@Setup{%
  \kvsetkeys{HoLogo}%
}
%    \end{macrocode}
%    \end{macro}
%
% \subsection{Options}
%
%    \begin{macro}{\HOLOGO@DeclareBoolOption}
%    \begin{macrocode}
\def\HOLOGO@DeclareBoolOption#1{%
  \expandafter\chardef\csname HOLOGOOPT@#1\endcsname\ltx@zero
  \kv@define@key{HoLogo}{#1}[true]{%
    \def\HOLOGO@temp{##1}%
    \ifx\HOLOGO@temp\HOLOGO@true
      \ifx\HOLOGO@name\relax
        \expandafter\chardef\csname HOLOGOOPT@#1\endcsname=\ltx@one
      \else
        \expandafter\chardef\csname
        HoLogoOpt@#1@\HOLOGO@name\endcsname\ltx@one
      \fi
      \HOLOGO@SetBreakAll{#1}%
    \else
      \ifx\HOLOGO@temp\HOLOGO@false
        \ifx\HOLOGO@name\relax
          \expandafter\chardef\csname HOLOGOOPT@#1\endcsname=\ltx@zero
        \else
          \expandafter\chardef\csname
          HoLogoOpt@#1@\HOLOGO@name\endcsname=\ltx@zero
        \fi
        \HOLOGO@SetBreakAll{#1}%
      \else
        \@PackageError{hologo}{%
          Unknown value `##1' for boolean option `#1'.\MessageBreak
          Known values are `true' and `false'%
        }\@ehc
      \fi
    \fi
  }%
}
%    \end{macrocode}
%    \end{macro}
%
%    \begin{macro}{\HOLOGO@SetBreakAll}
%    \begin{macrocode}
\def\HOLOGO@SetBreakAll#1{%
  \def\HOLOGO@temp{#1}%
  \ifx\HOLOGO@temp\HOLOGO@break
    \ifx\HOLOGO@name\relax
      \chardef\HOLOGOOPT@hyphenbreak=\HOLOGOOPT@break
      \chardef\HOLOGOOPT@spacebreak=\HOLOGOOPT@break
      \chardef\HOLOGOOPT@discretionarybreak=\HOLOGOOPT@break
    \else
      \expandafter\chardef
         \csname HoLogoOpt@hyphenbreak@\HOLOGO@name\endcsname=%
         \csname HoLogoOpt@break@\HOLOGO@name\endcsname
      \expandafter\chardef
         \csname HoLogoOpt@spacebreak@\HOLOGO@name\endcsname=%
         \csname HoLogoOpt@break@\HOLOGO@name\endcsname
      \expandafter\chardef
         \csname HoLogoOpt@discretionarybreak@\HOLOGO@name
             \endcsname=%
         \csname HoLogoOpt@break@\HOLOGO@name\endcsname
    \fi
  \fi
}
%    \end{macrocode}
%    \end{macro}
%
%    \begin{macro}{\HOLOGO@true}
%    \begin{macrocode}
\def\HOLOGO@true{true}
%    \end{macrocode}
%    \end{macro}
%    \begin{macro}{\HOLOGO@false}
%    \begin{macrocode}
\def\HOLOGO@false{false}
%    \end{macrocode}
%    \end{macro}
%    \begin{macro}{\HOLOGO@break}
%    \begin{macrocode}
\def\HOLOGO@break{break}
%    \end{macrocode}
%    \end{macro}
%
%    \begin{macrocode}
\HOLOGO@DeclareBoolOption{break}
\HOLOGO@DeclareBoolOption{hyphenbreak}
\HOLOGO@DeclareBoolOption{spacebreak}
\HOLOGO@DeclareBoolOption{discretionarybreak}
%    \end{macrocode}
%
%    \begin{macrocode}
\kv@define@key{HoLogo}{variant}{%
  \ifx\HOLOGO@name\relax
    \@PackageError{hologo}{%
      Option `variant' is not available in \string\hologoSetup,%
      \MessageBreak
      Use \string\hologoLogoSetup\space instead%
    }\@ehc
  \else
    \edef\HOLOGO@temp{#1}%
    \ifx\HOLOGO@temp\ltx@empty
      \expandafter
      \let\csname HoLogoOpt@variant@\HOLOGO@name\endcsname\@undefined
    \else
      \ltx@IfUndefined{HoLogo@\HOLOGO@name @\HOLOGO@temp}{%
        \@PackageError{hologo}{%
          Unknown variant `\HOLOGO@temp' of logo `\HOLOGO@name'%
        }\@ehc
      }{%
        \expandafter
        \let\csname HoLogoOpt@variant@\HOLOGO@name\endcsname
            \HOLOGO@temp
      }%
    \fi
  \fi
}
%    \end{macrocode}
%
%    \begin{macro}{\HOLOGO@Variant}
%    \begin{macrocode}
\def\HOLOGO@Variant#1{%
  #1%
  \ltx@ifundefined{HoLogoOpt@variant@#1}{%
  }{%
    @\csname HoLogoOpt@variant@#1\endcsname
  }%
}
%    \end{macrocode}
%    \end{macro}
%
% \subsection{Break/no-break support}
%
%    \begin{macro}{\HOLOGO@space}
%    \begin{macrocode}
\def\HOLOGO@space{%
  \ltx@ifundefined{HoLogoOpt@spacebreak@\HOLOGO@name}{%
    \ltx@ifundefined{HoLogoOpt@break@\HOLOGO@name}{%
      \chardef\HOLOGO@temp=\HOLOGOOPT@spacebreak
    }{%
      \chardef\HOLOGO@temp=%
        \csname HoLogoOpt@break@\HOLOGO@name\endcsname
    }%
  }{%
    \chardef\HOLOGO@temp=%
      \csname HoLogoOpt@spacebreak@\HOLOGO@name\endcsname
  }%
  \ifcase\HOLOGO@temp
    \penalty10000 %
  \fi
  \ltx@space
}
%    \end{macrocode}
%    \end{macro}
%
%    \begin{macro}{\HOLOGO@hyphen}
%    \begin{macrocode}
\def\HOLOGO@hyphen{%
  \ltx@ifundefined{HoLogoOpt@hyphenbreak@\HOLOGO@name}{%
    \ltx@ifundefined{HoLogoOpt@break@\HOLOGO@name}{%
      \chardef\HOLOGO@temp=\HOLOGOOPT@hyphenbreak
    }{%
      \chardef\HOLOGO@temp=%
        \csname HoLogoOpt@break@\HOLOGO@name\endcsname
    }%
  }{%
    \chardef\HOLOGO@temp=%
      \csname HoLogoOpt@hyphenbreak@\HOLOGO@name\endcsname
  }%
  \ifcase\HOLOGO@temp
    \ltx@mbox{-}%
  \else
    -%
  \fi
}
%    \end{macrocode}
%    \end{macro}
%
%    \begin{macro}{\HOLOGO@discretionary}
%    \begin{macrocode}
\def\HOLOGO@discretionary{%
  \ltx@ifundefined{HoLogoOpt@discretionarybreak@\HOLOGO@name}{%
    \ltx@ifundefined{HoLogoOpt@break@\HOLOGO@name}{%
      \chardef\HOLOGO@temp=\HOLOGOOPT@discretionarybreak
    }{%
      \chardef\HOLOGO@temp=%
        \csname HoLogoOpt@break@\HOLOGO@name\endcsname
    }%
  }{%
    \chardef\HOLOGO@temp=%
      \csname HoLogoOpt@discretionarybreak@\HOLOGO@name\endcsname
  }%
  \ifcase\HOLOGO@temp
  \else
    \-%
  \fi
}
%    \end{macrocode}
%    \end{macro}
%
%    \begin{macro}{\HOLOGO@mbox}
%    \begin{macrocode}
\def\HOLOGO@mbox#1{%
  \ltx@ifundefined{HoLogoOpt@break@\HOLOGO@name}{%
    \chardef\HOLOGO@temp=\HOLOGOOPT@hyphenbreak
  }{%
    \chardef\HOLOGO@temp=%
      \csname HoLogoOpt@break@\HOLOGO@name\endcsname
  }%
  \ifcase\HOLOGO@temp
    \ltx@mbox{#1}%
  \else
    #1%
  \fi
}
%    \end{macrocode}
%    \end{macro}
%
% \subsection{Font support}
%
%    \begin{macro}{\HoLogoFont@font}
%    \begin{tabular}{@{}ll@{}}
%    |#1|:& logo name\\
%    |#2|:& font short name\\
%    |#3|:& text
%    \end{tabular}
%    \begin{macrocode}
\def\HoLogoFont@font#1#2#3{%
  \begingroup
    \ltx@IfUndefined{HoLogoFont@logo@#1.#2}{%
      \ltx@IfUndefined{HoLogoFont@font@#2}{%
        \@PackageWarning{hologo}{%
          Missing font `#2' for logo `#1'%
        }%
        #3%
      }{%
        \csname HoLogoFont@font@#2\endcsname{#3}%
      }%
    }{%
      \csname HoLogoFont@logo@#1.#2\endcsname{#3}%
    }%
  \endgroup
}
%    \end{macrocode}
%    \end{macro}
%
%    \begin{macro}{\HoLogoFont@Def}
%    \begin{macrocode}
\def\HoLogoFont@Def#1{%
  \expandafter\def\csname HoLogoFont@font@#1\endcsname
}
%    \end{macrocode}
%    \end{macro}
%    \begin{macro}{\HoLogoFont@LogoDef}
%    \begin{macrocode}
\def\HoLogoFont@LogoDef#1#2{%
  \expandafter\def\csname HoLogoFont@logo@#1.#2\endcsname
}
%    \end{macrocode}
%    \end{macro}
%
% \subsubsection{Font defaults}
%
%    \begin{macro}{\HoLogoFont@font@general}
%    \begin{macrocode}
\HoLogoFont@Def{general}{}%
%    \end{macrocode}
%    \end{macro}
%
%    \begin{macro}{\HoLogoFont@font@rm}
%    \begin{macrocode}
\ltx@IfUndefined{rmfamily}{%
  \ltx@IfUndefined{rm}{%
  }{%
    \HoLogoFont@Def{rm}{\rm}%
  }%
}{%
  \HoLogoFont@Def{rm}{\rmfamily}%
}
%    \end{macrocode}
%    \end{macro}
%
%    \begin{macro}{\HoLogoFont@font@sf}
%    \begin{macrocode}
\ltx@IfUndefined{sffamily}{%
  \ltx@IfUndefined{sf}{%
  }{%
    \HoLogoFont@Def{sf}{\sf}%
  }%
}{%
  \HoLogoFont@Def{sf}{\sffamily}%
}
%    \end{macrocode}
%    \end{macro}
%
%    \begin{macro}{\HoLogoFont@font@bibsf}
%    In case of \hologo{plainTeX} the original small caps
%    variant is used as default. In \hologo{LaTeX}
%    the definition of package \xpackage{dtklogos} \cite{dtklogos}
%    is used.
%\begin{quote}
%\begin{verbatim}
%\DeclareRobustCommand{\BibTeX}{%
%  B%
%  \kern-.05em%
%  \hbox{%
%    $\m@th$% %% force math size calculations
%    \csname S@\f@size\endcsname
%    \fontsize\sf@size\z@
%    \math@fontsfalse
%    \selectfont
%    I%
%    \kern-.025em%
%    B
%  }%
%  \kern-.08em%
%  \-%
%  \TeX
%}
%\end{verbatim}
%\end{quote}
%    \begin{macrocode}
\ltx@IfUndefined{selectfont}{%
  \ltx@IfUndefined{tensc}{%
    \font\tensc=cmcsc10\relax
  }{}%
  \HoLogoFont@Def{bibsf}{\tensc}%
}{%
  \HoLogoFont@Def{bibsf}{%
    $\mathsurround=0pt$%
    \csname S@\f@size\endcsname
    \fontsize\sf@size{0pt}%
    \math@fontsfalse
    \selectfont
  }%
}
%    \end{macrocode}
%    \end{macro}
%
%    \begin{macro}{\HoLogoFont@font@sc}
%    \begin{macrocode}
\ltx@IfUndefined{scshape}{%
  \ltx@IfUndefined{tensc}{%
    \font\tensc=cmcsc10\relax
  }{}%
  \HoLogoFont@Def{sc}{\tensc}%
}{%
  \HoLogoFont@Def{sc}{\scshape}%
}
%    \end{macrocode}
%    \end{macro}
%
%    \begin{macro}{\HoLogoFont@font@sy}
%    \begin{macrocode}
\ltx@IfUndefined{usefont}{%
  \ltx@IfUndefined{tensy}{%
  }{%
    \HoLogoFont@Def{sy}{\tensy}%
  }%
}{%
  \HoLogoFont@Def{sy}{%
    \usefont{OMS}{cmsy}{m}{n}%
  }%
}
%    \end{macrocode}
%    \end{macro}
%
%    \begin{macro}{\HoLogoFont@font@logo}
%    \begin{macrocode}
\begingroup
  \def\x{LaTeX2e}%
\expandafter\endgroup
\ifx\fmtname\x
  \ltx@IfUndefined{logofamily}{%
    \DeclareRobustCommand\logofamily{%
      \not@math@alphabet\logofamily\relax
      \fontencoding{U}%
      \fontfamily{logo}%
      \selectfont
    }%
  }{}%
  \ltx@IfUndefined{logofamily}{%
  }{%
    \HoLogoFont@Def{logo}{\logofamily}%
  }%
\else
  \ltx@IfUndefined{tenlogo}{%
    \font\tenlogo=logo10\relax
  }{}%
  \HoLogoFont@Def{logo}{\tenlogo}%
\fi
%    \end{macrocode}
%    \end{macro}
%
% \subsubsection{Font setup}
%
%    \begin{macro}{\hologoFontSetup}
%    \begin{macrocode}
\def\hologoFontSetup{%
  \let\HOLOGO@name\relax
  \HOLOGO@FontSetup
}
%    \end{macrocode}
%    \end{macro}
%
%    \begin{macro}{\hologoLogoFontSetup}
%    \begin{macrocode}
\def\hologoLogoFontSetup#1{%
  \edef\HOLOGO@name{#1}%
  \ltx@IfUndefined{HoLogo@\HOLOGO@name}{%
    \@PackageError{hologo}{%
      Unknown logo `\HOLOGO@name'%
    }\@ehc
    \ltx@gobble
  }{%
    \HOLOGO@FontSetup
  }%
}
%    \end{macrocode}
%    \end{macro}
%
%    \begin{macro}{\HOLOGO@FontSetup}
%    \begin{macrocode}
\def\HOLOGO@FontSetup{%
  \kvsetkeys{HoLogoFont}%
}
%    \end{macrocode}
%    \end{macro}
%
%    \begin{macrocode}
\def\HOLOGO@temp#1{%
  \kv@define@key{HoLogoFont}{#1}{%
    \ifx\HOLOGO@name\relax
      \HoLogoFont@Def{#1}{##1}%
    \else
      \HoLogoFont@LogoDef\HOLOGO@name{#1}{##1}%
    \fi
  }%
}
\HOLOGO@temp{general}
\HOLOGO@temp{sf}
%    \end{macrocode}
%
% \subsection{Generic logo commands}
%
%    \begin{macrocode}
\HOLOGO@IfExists\hologo{%
  \@PackageError{hologo}{%
    \string\hologo\ltx@space is already defined.\MessageBreak
    Package loading is aborted%
  }\@ehc
  \HOLOGO@AtEnd
}%
\HOLOGO@IfExists\hologoRobust{%
  \@PackageError{hologo}{%
    \string\hologoRobust\ltx@space is already defined.\MessageBreak
    Package loading is aborted%
  }\@ehc
  \HOLOGO@AtEnd
}%
%    \end{macrocode}
%
% \subsubsection{\cs{hologo} and friends}
%
%    \begin{macrocode}
\ifluatex
  \expandafter\ltx@firstofone
\else
  \expandafter\ltx@gobble
\fi
{%
  \ltx@IfUndefined{ifincsname}{%
    \ifnum\luatexversion<36 %
      \expandafter\ltx@gobble
    \else
      \expandafter\ltx@firstofone
    \fi
    {%
      \begingroup
        \ifcase0%
            \directlua{%
              if tex.enableprimitives then %
                tex.enableprimitives('HOLOGO@', {'ifincsname'})%
              else %
                tex.print('1')%
              end%
            }%
            \ifx\HOLOGO@ifincsname\@undefined 1\fi%
            \relax
          \expandafter\ltx@firstofone
        \else
          \endgroup
          \expandafter\ltx@gobble
        \fi
        {%
          \global\let\ifincsname\HOLOGO@ifincsname
        }%
      \HOLOGO@temp
    }%
  }{}%
}
%    \end{macrocode}
%    \begin{macrocode}
\ltx@IfUndefined{ifincsname}{%
  \catcode`$=14 %
}{%
  \catcode`$=9 %
}
%    \end{macrocode}
%
%    \begin{macro}{\hologo}
%    \begin{macrocode}
\def\hologo#1{%
$ \ifincsname
$   \ltx@ifundefined{HoLogoCs@\HOLOGO@Variant{#1}}{%
$     #1%
$   }{%
$     \csname HoLogoCs@\HOLOGO@Variant{#1}\endcsname\ltx@firstoftwo
$   }%
$ \else
    \HOLOGO@IfExists\texorpdfstring\texorpdfstring\ltx@firstoftwo
    {%
      \hologoRobust{#1}%
    }{%
      \ltx@ifundefined{HoLogoBkm@\HOLOGO@Variant{#1}}{%
        \ltx@ifundefined{HoLogo@#1}{?#1?}{#1}%
      }{%
        \csname HoLogoBkm@\HOLOGO@Variant{#1}\endcsname
        \ltx@firstoftwo
      }%
    }%
$ \fi
}
%    \end{macrocode}
%    \end{macro}
%    \begin{macro}{\Hologo}
%    \begin{macrocode}
\def\Hologo#1{%
$ \ifincsname
$   \ltx@ifundefined{HoLogoCs@\HOLOGO@Variant{#1}}{%
$     #1%
$   }{%
$     \csname HoLogoCs@\HOLOGO@Variant{#1}\endcsname\ltx@secondoftwo
$   }%
$ \else
    \HOLOGO@IfExists\texorpdfstring\texorpdfstring\ltx@firstoftwo
    {%
      \HologoRobust{#1}%
    }{%
      \ltx@ifundefined{HoLogoBkm@\HOLOGO@Variant{#1}}{%
        \ltx@ifundefined{HoLogo@#1}{?#1?}{#1}%
      }{%
        \csname HoLogoBkm@\HOLOGO@Variant{#1}\endcsname
        \ltx@secondoftwo
      }%
    }%
$ \fi
}
%    \end{macrocode}
%    \end{macro}
%
%    \begin{macro}{\hologoVariant}
%    \begin{macrocode}
\def\hologoVariant#1#2{%
  \ifx\relax#2\relax
    \hologo{#1}%
  \else
$   \ifincsname
$     \ltx@ifundefined{HoLogoCs@#1@#2}{%
$       #1%
$     }{%
$       \csname HoLogoCs@#1@#2\endcsname\ltx@firstoftwo
$     }%
$   \else
      \HOLOGO@IfExists\texorpdfstring\texorpdfstring\ltx@firstoftwo
      {%
        \hologoVariantRobust{#1}{#2}%
      }{%
        \ltx@ifundefined{HoLogoBkm@#1@#2}{%
          \ltx@ifundefined{HoLogo@#1}{?#1?}{#1}%
        }{%
          \csname HoLogoBkm@#1@#2\endcsname
          \ltx@firstoftwo
        }%
      }%
$   \fi
  \fi
}
%    \end{macrocode}
%    \end{macro}
%    \begin{macro}{\HologoVariant}
%    \begin{macrocode}
\def\HologoVariant#1#2{%
  \ifx\relax#2\relax
    \Hologo{#1}%
  \else
$   \ifincsname
$     \ltx@ifundefined{HoLogoCs@#1@#2}{%
$       #1%
$     }{%
$       \csname HoLogoCs@#1@#2\endcsname\ltx@secondoftwo
$     }%
$   \else
      \HOLOGO@IfExists\texorpdfstring\texorpdfstring\ltx@firstoftwo
      {%
        \HologoVariantRobust{#1}{#2}%
      }{%
        \ltx@ifundefined{HoLogoBkm@#1@#2}{%
          \ltx@ifundefined{HoLogo@#1}{?#1?}{#1}%
        }{%
          \csname HoLogoBkm@#1@#2\endcsname
          \ltx@secondoftwo
        }%
      }%
$   \fi
  \fi
}
%    \end{macrocode}
%    \end{macro}
%
%    \begin{macrocode}
\catcode`\$=3 %
%    \end{macrocode}
%
% \subsubsection{\cs{hologoRobust} and friends}
%
%    \begin{macro}{\hologoRobust}
%    \begin{macrocode}
\ltx@IfUndefined{protected}{%
  \ltx@IfUndefined{DeclareRobustCommand}{%
    \def\hologoRobust#1%
  }{%
    \DeclareRobustCommand*\hologoRobust[1]%
  }%
}{%
  \protected\def\hologoRobust#1%
}%
{%
  \edef\HOLOGO@name{#1}%
  \ltx@IfUndefined{HoLogo@\HOLOGO@Variant\HOLOGO@name}{%
    \@PackageError{hologo}{%
      Unknown logo `\HOLOGO@name'%
    }\@ehc
    ?\HOLOGO@name?%
  }{%
    \ltx@IfUndefined{ver@tex4ht.sty}{%
      \HoLogoFont@font\HOLOGO@name{general}{%
        \csname HoLogo@\HOLOGO@Variant\HOLOGO@name\endcsname
        \ltx@firstoftwo
      }%
    }{%
      \ltx@IfUndefined{HoLogoHtml@\HOLOGO@Variant\HOLOGO@name}{%
        \HOLOGO@name
      }{%
        \csname HoLogoHtml@\HOLOGO@Variant\HOLOGO@name\endcsname
        \ltx@firstoftwo
      }%
    }%
  }%
}
%    \end{macrocode}
%    \end{macro}
%    \begin{macro}{\HologoRobust}
%    \begin{macrocode}
\ltx@IfUndefined{protected}{%
  \ltx@IfUndefined{DeclareRobustCommand}{%
    \def\HologoRobust#1%
  }{%
    \DeclareRobustCommand*\HologoRobust[1]%
  }%
}{%
  \protected\def\HologoRobust#1%
}%
{%
  \edef\HOLOGO@name{#1}%
  \ltx@IfUndefined{HoLogo@\HOLOGO@Variant\HOLOGO@name}{%
    \@PackageError{hologo}{%
      Unknown logo `\HOLOGO@name'%
    }\@ehc
    ?\HOLOGO@name?%
  }{%
    \ltx@IfUndefined{ver@tex4ht.sty}{%
      \HoLogoFont@font\HOLOGO@name{general}{%
        \csname HoLogo@\HOLOGO@Variant\HOLOGO@name\endcsname
        \ltx@secondoftwo
      }%
    }{%
      \ltx@IfUndefined{HoLogoHtml@\HOLOGO@Variant\HOLOGO@name}{%
        \expandafter\HOLOGO@Uppercase\HOLOGO@name
      }{%
        \csname HoLogoHtml@\HOLOGO@Variant\HOLOGO@name\endcsname
        \ltx@secondoftwo
      }%
    }%
  }%
}
%    \end{macrocode}
%    \end{macro}
%    \begin{macro}{\hologoVariantRobust}
%    \begin{macrocode}
\ltx@IfUndefined{protected}{%
  \ltx@IfUndefined{DeclareRobustCommand}{%
    \def\hologoVariantRobust#1#2%
  }{%
    \DeclareRobustCommand*\hologoVariantRobust[2]%
  }%
}{%
  \protected\def\hologoVariantRobust#1#2%
}%
{%
  \begingroup
    \hologoLogoSetup{#1}{variant={#2}}%
    \hologoRobust{#1}%
  \endgroup
}
%    \end{macrocode}
%    \end{macro}
%    \begin{macro}{\HologoVariantRobust}
%    \begin{macrocode}
\ltx@IfUndefined{protected}{%
  \ltx@IfUndefined{DeclareRobustCommand}{%
    \def\HologoVariantRobust#1#2%
  }{%
    \DeclareRobustCommand*\HologoVariantRobust[2]%
  }%
}{%
  \protected\def\HologoVariantRobust#1#2%
}%
{%
  \begingroup
    \hologoLogoSetup{#1}{variant={#2}}%
    \HologoRobust{#1}%
  \endgroup
}
%    \end{macrocode}
%    \end{macro}
%
%    \begin{macro}{\hologorobust}
%    Macro \cs{hologorobust} is only defined for compatibility.
%    Its use is deprecated.
%    \begin{macrocode}
\def\hologorobust{\hologoRobust}
%    \end{macrocode}
%    \end{macro}
%
% \subsection{Helpers}
%
%    \begin{macro}{\HOLOGO@Uppercase}
%    Macro \cs{HOLOGO@Uppercase} is restricted to \cs{uppercase},
%    because \hologo{plainTeX} or \hologo{iniTeX} do not provide
%    \cs{MakeUppercase}.
%    \begin{macrocode}
\def\HOLOGO@Uppercase#1{\uppercase{#1}}
%    \end{macrocode}
%    \end{macro}
%
%    \begin{macro}{\HOLOGO@PdfdocUnicode}
%    \begin{macrocode}
\def\HOLOGO@PdfdocUnicode{%
  \ifx\ifHy@unicode\iftrue
    \expandafter\ltx@secondoftwo
  \else
    \expandafter\ltx@firstoftwo
  \fi
}
%    \end{macrocode}
%    \end{macro}
%
%    \begin{macro}{\HOLOGO@Math}
%    \begin{macrocode}
\def\HOLOGO@MathSetup{%
  \mathsurround0pt\relax
  \HOLOGO@IfExists\f@series{%
    \if b\expandafter\ltx@car\f@series x\@nil
      \csname boldmath\endcsname
   \fi
  }{}%
}
%    \end{macrocode}
%    \end{macro}
%
%    \begin{macro}{\HOLOGO@TempDimen}
%    \begin{macrocode}
\dimendef\HOLOGO@TempDimen=\ltx@zero
%    \end{macrocode}
%    \end{macro}
%    \begin{macro}{\HOLOGO@NegativeKerning}
%    \begin{macrocode}
\def\HOLOGO@NegativeKerning#1{%
  \begingroup
    \HOLOGO@TempDimen=0pt\relax
    \comma@parse@normalized{#1}{%
      \ifdim\HOLOGO@TempDimen=0pt %
        \expandafter\HOLOGO@@NegativeKerning\comma@entry
      \fi
      \ltx@gobble
    }%
    \ifdim\HOLOGO@TempDimen<0pt %
      \kern\HOLOGO@TempDimen
    \fi
  \endgroup
}
%    \end{macrocode}
%    \end{macro}
%    \begin{macro}{\HOLOGO@@NegativeKerning}
%    \begin{macrocode}
\def\HOLOGO@@NegativeKerning#1#2{%
  \setbox\ltx@zero\hbox{#1#2}%
  \HOLOGO@TempDimen=\wd\ltx@zero
  \setbox\ltx@zero\hbox{#1\kern0pt#2}%
  \advance\HOLOGO@TempDimen by -\wd\ltx@zero
}
%    \end{macrocode}
%    \end{macro}
%
%    \begin{macro}{\HOLOGO@SpaceFactor}
%    \begin{macrocode}
\def\HOLOGO@SpaceFactor{%
  \spacefactor1000 %
}
%    \end{macrocode}
%    \end{macro}
%
%    \begin{macro}{\HOLOGO@Span}
%    \begin{macrocode}
\def\HOLOGO@Span#1#2{%
  \HCode{<span class="HoLogo-#1">}%
  #2%
  \HCode{</span>}%
}
%    \end{macrocode}
%    \end{macro}
%
% \subsubsection{Text subscript}
%
%    \begin{macro}{\HOLOGO@SubScript}%
%    \begin{macrocode}
\def\HOLOGO@SubScript#1{%
  \ltx@IfUndefined{textsubscript}{%
    \ltx@IfUndefined{text}{%
      \ltx@mbox{%
        \mathsurround=0pt\relax
        $%
          _{%
            \ltx@IfUndefined{sf@size}{%
              \mathrm{#1}%
            }{%
              \mbox{%
                \fontsize\sf@size{0pt}\selectfont
                #1%
              }%
            }%
          }%
        $%
      }%
    }{%
      \ltx@mbox{%
        \mathsurround=0pt\relax
        $_{\text{#1}}$%
      }%
    }%
  }{%
    \textsubscript{#1}%
  }%
}
%    \end{macrocode}
%    \end{macro}
%
% \subsection{\hologo{TeX} and friends}
%
% \subsubsection{\hologo{TeX}}
%
%    \begin{macro}{\HoLogo@TeX}
%    Source: \hologo{LaTeX} kernel.
%    \begin{macrocode}
\def\HoLogo@TeX#1{%
  T\kern-.1667em\lower.5ex\hbox{E}\kern-.125emX\HOLOGO@SpaceFactor
}
%    \end{macrocode}
%    \end{macro}
%    \begin{macro}{\HoLogoHtml@TeX}
%    \begin{macrocode}
\def\HoLogoHtml@TeX#1{%
  \HoLogoCss@TeX
  \HOLOGO@Span{TeX}{%
    T%
    \HOLOGO@Span{e}{%
      E%
    }%
    X%
  }%
}
%    \end{macrocode}
%    \end{macro}
%    \begin{macro}{\HoLogoCss@TeX}
%    \begin{macrocode}
\def\HoLogoCss@TeX{%
  \Css{%
    span.HoLogo-TeX span.HoLogo-e{%
      position:relative;%
      top:.5ex;%
      margin-left:-.1667em;%
      margin-right:-.125em;%
    }%
  }%
  \Css{%
    a span.HoLogo-TeX span.HoLogo-e{%
      text-decoration:none;%
    }%
  }%
  \global\let\HoLogoCss@TeX\relax
}
%    \end{macrocode}
%    \end{macro}
%
% \subsubsection{\hologo{plainTeX}}
%
%    \begin{macro}{\HoLogo@plainTeX@space}
%    Source: ``The \hologo{TeX}book''
%    \begin{macrocode}
\def\HoLogo@plainTeX@space#1{%
  \HOLOGO@mbox{#1{p}{P}lain}\HOLOGO@space\hologo{TeX}%
}
%    \end{macrocode}
%    \end{macro}
%    \begin{macro}{\HoLogoCs@plainTeX@space}
%    \begin{macrocode}
\def\HoLogoCs@plainTeX@space#1{#1{p}{P}lain TeX}%
%    \end{macrocode}
%    \end{macro}
%    \begin{macro}{\HoLogoBkm@plainTeX@space}
%    \begin{macrocode}
\def\HoLogoBkm@plainTeX@space#1{%
  #1{p}{P}lain \hologo{TeX}%
}
%    \end{macrocode}
%    \end{macro}
%    \begin{macro}{\HoLogoHtml@plainTeX@space}
%    \begin{macrocode}
\def\HoLogoHtml@plainTeX@space#1{%
  #1{p}{P}lain \hologo{TeX}%
}
%    \end{macrocode}
%    \end{macro}
%
%    \begin{macro}{\HoLogo@plainTeX@hyphen}
%    \begin{macrocode}
\def\HoLogo@plainTeX@hyphen#1{%
  \HOLOGO@mbox{#1{p}{P}lain}\HOLOGO@hyphen\hologo{TeX}%
}
%    \end{macrocode}
%    \end{macro}
%    \begin{macro}{\HoLogoCs@plainTeX@hyphen}
%    \begin{macrocode}
\def\HoLogoCs@plainTeX@hyphen#1{#1{p}{P}lain-TeX}
%    \end{macrocode}
%    \end{macro}
%    \begin{macro}{\HoLogoBkm@plainTeX@hyphen}
%    \begin{macrocode}
\def\HoLogoBkm@plainTeX@hyphen#1{%
  #1{p}{P}lain-\hologo{TeX}%
}
%    \end{macrocode}
%    \end{macro}
%    \begin{macro}{\HoLogoHtml@plainTeX@hyphen}
%    \begin{macrocode}
\def\HoLogoHtml@plainTeX@hyphen#1{%
  #1{p}{P}lain-\hologo{TeX}%
}
%    \end{macrocode}
%    \end{macro}
%
%    \begin{macro}{\HoLogo@plainTeX@runtogether}
%    \begin{macrocode}
\def\HoLogo@plainTeX@runtogether#1{%
  \HOLOGO@mbox{#1{p}{P}lain\hologo{TeX}}%
}
%    \end{macrocode}
%    \end{macro}
%    \begin{macro}{\HoLogoCs@plainTeX@runtogether}
%    \begin{macrocode}
\def\HoLogoCs@plainTeX@runtogether#1{#1{p}{P}lainTeX}
%    \end{macrocode}
%    \end{macro}
%    \begin{macro}{\HoLogoBkm@plainTeX@runtogether}
%    \begin{macrocode}
\def\HoLogoBkm@plainTeX@runtogether#1{%
  #1{p}{P}lain\hologo{TeX}%
}
%    \end{macrocode}
%    \end{macro}
%    \begin{macro}{\HoLogoHtml@plainTeX@runtogether}
%    \begin{macrocode}
\def\HoLogoHtml@plainTeX@runtogether#1{%
  #1{p}{P}lain\hologo{TeX}%
}
%    \end{macrocode}
%    \end{macro}
%
%    \begin{macro}{\HoLogo@plainTeX}
%    \begin{macrocode}
\def\HoLogo@plainTeX{\HoLogo@plainTeX@space}
%    \end{macrocode}
%    \end{macro}
%    \begin{macro}{\HoLogoCs@plainTeX}
%    \begin{macrocode}
\def\HoLogoCs@plainTeX{\HoLogoCs@plainTeX@space}
%    \end{macrocode}
%    \end{macro}
%    \begin{macro}{\HoLogoBkm@plainTeX}
%    \begin{macrocode}
\def\HoLogoBkm@plainTeX{\HoLogoBkm@plainTeX@space}
%    \end{macrocode}
%    \end{macro}
%    \begin{macro}{\HoLogoHtml@plainTeX}
%    \begin{macrocode}
\def\HoLogoHtml@plainTeX{\HoLogoHtml@plainTeX@space}
%    \end{macrocode}
%    \end{macro}
%
% \subsubsection{\hologo{LaTeX}}
%
%    Source: \hologo{LaTeX} kernel.
%\begin{quote}
%\begin{verbatim}
%\DeclareRobustCommand{\LaTeX}{%
%  L%
%  \kern-.36em%
%  {%
%    \sbox\z@ T%
%    \vbox to\ht\z@{%
%      \hbox{%
%        \check@mathfonts
%        \fontsize\sf@size\z@
%        \math@fontsfalse
%        \selectfont
%        A%
%      }%
%      \vss
%    }%
%  }%
%  \kern-.15em%
%  \TeX
%}
%\end{verbatim}
%\end{quote}
%
%    \begin{macro}{\HoLogo@La}
%    \begin{macrocode}
\def\HoLogo@La#1{%
  L%
  \kern-.36em%
  \begingroup
    \setbox\ltx@zero\hbox{T}%
    \vbox to\ht\ltx@zero{%
      \hbox{%
        \ltx@ifundefined{check@mathfonts}{%
          \csname sevenrm\endcsname
        }{%
          \check@mathfonts
          \fontsize\sf@size{0pt}%
          \math@fontsfalse\selectfont
        }%
        A%
      }%
      \vss
    }%
  \endgroup
}
%    \end{macrocode}
%    \end{macro}
%
%    \begin{macro}{\HoLogo@LaTeX}
%    Source: \hologo{LaTeX} kernel.
%    \begin{macrocode}
\def\HoLogo@LaTeX#1{%
  \hologo{La}%
  \kern-.15em%
  \hologo{TeX}%
}
%    \end{macrocode}
%    \end{macro}
%    \begin{macro}{\HoLogoHtml@LaTeX}
%    \begin{macrocode}
\def\HoLogoHtml@LaTeX#1{%
  \HoLogoCss@LaTeX
  \HOLOGO@Span{LaTeX}{%
    L%
    \HOLOGO@Span{a}{%
      A%
    }%
    \hologo{TeX}%
  }%
}
%    \end{macrocode}
%    \end{macro}
%    \begin{macro}{\HoLogoCss@LaTeX}
%    \begin{macrocode}
\def\HoLogoCss@LaTeX{%
  \Css{%
    span.HoLogo-LaTeX span.HoLogo-a{%
      position:relative;%
      top:-.5ex;%
      margin-left:-.36em;%
      margin-right:-.15em;%
      font-size:85\%;%
    }%
  }%
  \global\let\HoLogoCss@LaTeX\relax
}
%    \end{macrocode}
%    \end{macro}
%
% \subsubsection{\hologo{(La)TeX}}
%
%    \begin{macro}{\HoLogo@LaTeXTeX}
%    The kerning around the parentheses is taken
%    from package \xpackage{dtklogos} \cite{dtklogos}.
%\begin{quote}
%\begin{verbatim}
%\DeclareRobustCommand{\LaTeXTeX}{%
%  (%
%  \kern-.15em%
%  L%
%  \kern-.36em%
%  {%
%    \sbox\z@ T%
%    \vbox to\ht0{%
%      \hbox{%
%        $\m@th$%
%        \csname S@\f@size\endcsname
%        \fontsize\sf@size\z@
%        \math@fontsfalse
%        \selectfont
%        A%
%      }%
%      \vss
%    }%
%  }%
%  \kern-.2em%
%  )%
%  \kern-.15em%
%  \TeX
%}
%\end{verbatim}
%\end{quote}
%    \begin{macrocode}
\def\HoLogo@LaTeXTeX#1{%
  (%
  \kern-.15em%
  \hologo{La}%
  \kern-.2em%
  )%
  \kern-.15em%
  \hologo{TeX}%
}
%    \end{macrocode}
%    \end{macro}
%    \begin{macro}{\HoLogoBkm@LaTeXTeX}
%    \begin{macrocode}
\def\HoLogoBkm@LaTeXTeX#1{(La)TeX}
%    \end{macrocode}
%    \end{macro}
%
%    \begin{macro}{\HoLogo@(La)TeX}
%    \begin{macrocode}
\expandafter
\let\csname HoLogo@(La)TeX\endcsname\HoLogo@LaTeXTeX
%    \end{macrocode}
%    \end{macro}
%    \begin{macro}{\HoLogoBkm@(La)TeX}
%    \begin{macrocode}
\expandafter
\let\csname HoLogoBkm@(La)TeX\endcsname\HoLogoBkm@LaTeXTeX
%    \end{macrocode}
%    \end{macro}
%    \begin{macro}{\HoLogoHtml@LaTeXTeX}
%    \begin{macrocode}
\def\HoLogoHtml@LaTeXTeX#1{%
  \HoLogoCss@LaTeXTeX
  \HOLOGO@Span{LaTeXTeX}{%
    (%
    \HOLOGO@Span{L}{L}%
    \HOLOGO@Span{a}{A}%
    \HOLOGO@Span{ParenRight}{)}%
    \hologo{TeX}%
  }%
}
%    \end{macrocode}
%    \end{macro}
%    \begin{macro}{\HoLogoHtml@(La)TeX}
%    Kerning after opening parentheses and before closing parentheses
%    is $-0.1$\,em. The original values $-0.15$\,em
%    looked too ugly for a serif font.
%    \begin{macrocode}
\expandafter
\let\csname HoLogoHtml@(La)TeX\endcsname\HoLogoHtml@LaTeXTeX
%    \end{macrocode}
%    \end{macro}
%    \begin{macro}{\HoLogoCss@LaTeXTeX}
%    \begin{macrocode}
\def\HoLogoCss@LaTeXTeX{%
  \Css{%
    span.HoLogo-LaTeXTeX span.HoLogo-L{%
      margin-left:-.1em;%
    }%
  }%
  \Css{%
    span.HoLogo-LaTeXTeX span.HoLogo-a{%
      position:relative;%
      top:-.5ex;%
      margin-left:-.36em;%
      margin-right:-.1em;%
      font-size:85\%;%
    }%
  }%
  \Css{%
    span.HoLogo-LaTeXTeX span.HoLogo-ParenRight{%
      margin-right:-.15em;%
    }%
  }%
  \global\let\HoLogoCss@LaTeXTeX\relax
}
%    \end{macrocode}
%    \end{macro}
%
% \subsubsection{\hologo{LaTeXe}}
%
%    \begin{macro}{\HoLogo@LaTeXe}
%    Source: \hologo{LaTeX} kernel
%    \begin{macrocode}
\def\HoLogo@LaTeXe#1{%
  \hologo{LaTeX}%
  \kern.15em%
  \hbox{%
    \HOLOGO@MathSetup
    2%
    $_{\textstyle\varepsilon}$%
  }%
}
%    \end{macrocode}
%    \end{macro}
%
%    \begin{macro}{\HoLogoCs@LaTeXe}
%    \begin{macrocode}
\ifnum64=`\^^^^0040\relax % test for big chars of LuaTeX/XeTeX
  \catcode`\$=9 %
  \catcode`\&=14 %
\else
  \catcode`\$=14 %
  \catcode`\&=9 %
\fi
\def\HoLogoCs@LaTeXe#1{%
  LaTeX2%
$ \string ^^^^0395%
& e%
}%
\catcode`\$=3 %
\catcode`\&=4 %
%    \end{macrocode}
%    \end{macro}
%
%    \begin{macro}{\HoLogoBkm@LaTeXe}
%    \begin{macrocode}
\def\HoLogoBkm@LaTeXe#1{%
  \hologo{LaTeX}%
  2%
  \HOLOGO@PdfdocUnicode{e}{\textepsilon}%
}
%    \end{macrocode}
%    \end{macro}
%
%    \begin{macro}{\HoLogoHtml@LaTeXe}
%    \begin{macrocode}
\def\HoLogoHtml@LaTeXe#1{%
  \HoLogoCss@LaTeXe
  \HOLOGO@Span{LaTeX2e}{%
    \hologo{LaTeX}%
    \HOLOGO@Span{2}{2}%
    \HOLOGO@Span{e}{%
      \HOLOGO@MathSetup
      \ensuremath{\textstyle\varepsilon}%
    }%
  }%
}
%    \end{macrocode}
%    \end{macro}
%    \begin{macro}{\HoLogoCss@LaTeXe}
%    \begin{macrocode}
\def\HoLogoCss@LaTeXe{%
  \Css{%
    span.HoLogo-LaTeX2e span.HoLogo-2{%
      padding-left:.15em;%
    }%
  }%
  \Css{%
    span.HoLogo-LaTeX2e span.HoLogo-e{%
      position:relative;%
      top:.35ex;%
      text-decoration:none;%
    }%
  }%
  \global\let\HoLogoCss@LaTeXe\relax
}
%    \end{macrocode}
%    \end{macro}
%
%    \begin{macro}{\HoLogo@LaTeX2e}
%    \begin{macrocode}
\expandafter
\let\csname HoLogo@LaTeX2e\endcsname\HoLogo@LaTeXe
%    \end{macrocode}
%    \end{macro}
%    \begin{macro}{\HoLogoCs@LaTeX2e}
%    \begin{macrocode}
\expandafter
\let\csname HoLogoCs@LaTeX2e\endcsname\HoLogoCs@LaTeXe
%    \end{macrocode}
%    \end{macro}
%    \begin{macro}{\HoLogoBkm@LaTeX2e}
%    \begin{macrocode}
\expandafter
\let\csname HoLogoBkm@LaTeX2e\endcsname\HoLogoBkm@LaTeXe
%    \end{macrocode}
%    \end{macro}
%    \begin{macro}{\HoLogoHtml@LaTeX2e}
%    \begin{macrocode}
\expandafter
\let\csname HoLogoHtml@LaTeX2e\endcsname\HoLogoHtml@LaTeXe
%    \end{macrocode}
%    \end{macro}
%
% \subsubsection{\hologo{LaTeX3}}
%
%    \begin{macro}{\HoLogo@LaTeX3}
%    Source: \hologo{LaTeX} kernel
%    \begin{macrocode}
\expandafter\def\csname HoLogo@LaTeX3\endcsname#1{%
  \hologo{LaTeX}%
  3%
}
%    \end{macrocode}
%    \end{macro}
%
%    \begin{macro}{\HoLogoBkm@LaTeX3}
%    \begin{macrocode}
\expandafter\def\csname HoLogoBkm@LaTeX3\endcsname#1{%
  \hologo{LaTeX}%
  3%
}
%    \end{macrocode}
%    \end{macro}
%    \begin{macro}{\HoLogoHtml@LaTeX3}
%    \begin{macrocode}
\expandafter
\let\csname HoLogoHtml@LaTeX3\expandafter\endcsname
\csname HoLogo@LaTeX3\endcsname
%    \end{macrocode}
%    \end{macro}
%
% \subsubsection{\hologo{LaTeXML}}
%
%    \begin{macro}{\HoLogo@LaTeXML}
%    \begin{macrocode}
\def\HoLogo@LaTeXML#1{%
  \HOLOGO@mbox{%
    \hologo{La}%
    \kern-.15em%
    T%
    \kern-.1667em%
    \lower.5ex\hbox{E}%
    \kern-.125em%
    \HoLogoFont@font{LaTeXML}{sc}{xml}%
  }%
}
%    \end{macrocode}
%    \end{macro}
%    \begin{macro}{\HoLogoHtml@pdfLaTeX}
%    \begin{macrocode}
\def\HoLogoHtml@LaTeXML#1{%
  \HOLOGO@Span{LaTeXML}{%
    \HoLogoCss@LaTeX
    \HoLogoCss@TeX
    \HOLOGO@Span{LaTeX}{%
      L%
      \HOLOGO@Span{a}{%
        A%
      }%
    }%
    \HOLOGO@Span{TeX}{%
      T%
      \HOLOGO@Span{e}{%
        E%
      }%
    }%
    \HCode{<span style="font-variant: small-caps;">}%
    xml%
    \HCode{</span>}%
  }%
}
%    \end{macrocode}
%    \end{macro}
%
% \subsubsection{\hologo{eTeX}}
%
%    \begin{macro}{\HoLogo@eTeX}
%    Source: package \xpackage{etex}
%    \begin{macrocode}
\def\HoLogo@eTeX#1{%
  \ltx@mbox{%
    \HOLOGO@MathSetup
    $\varepsilon$%
    -%
    \HOLOGO@NegativeKerning{-T,T-,To}%
    \hologo{TeX}%
  }%
}
%    \end{macrocode}
%    \end{macro}
%    \begin{macro}{\HoLogoCs@eTeX}
%    \begin{macrocode}
\ifnum64=`\^^^^0040\relax % test for big chars of LuaTeX/XeTeX
  \catcode`\$=9 %
  \catcode`\&=14 %
\else
  \catcode`\$=14 %
  \catcode`\&=9 %
\fi
\def\HoLogoCs@eTeX#1{%
$ #1{\string ^^^^0395}{\string ^^^^03b5}%
& #1{e}{E}%
  TeX%
}%
\catcode`\$=3 %
\catcode`\&=4 %
%    \end{macrocode}
%    \end{macro}
%    \begin{macro}{\HoLogoBkm@eTeX}
%    \begin{macrocode}
\def\HoLogoBkm@eTeX#1{%
  \HOLOGO@PdfdocUnicode{#1{e}{E}}{\textepsilon}%
  -%
  \hologo{TeX}%
}
%    \end{macrocode}
%    \end{macro}
%    \begin{macro}{\HoLogoHtml@eTeX}
%    \begin{macrocode}
\def\HoLogoHtml@eTeX#1{%
  \ltx@mbox{%
    \HOLOGO@MathSetup
    $\varepsilon$%
    -%
    \hologo{TeX}%
  }%
}
%    \end{macrocode}
%    \end{macro}
%
% \subsubsection{\hologo{iniTeX}}
%
%    \begin{macro}{\HoLogo@iniTeX}
%    \begin{macrocode}
\def\HoLogo@iniTeX#1{%
  \HOLOGO@mbox{%
    #1{i}{I}ni\hologo{TeX}%
  }%
}
%    \end{macrocode}
%    \end{macro}
%    \begin{macro}{\HoLogoCs@iniTeX}
%    \begin{macrocode}
\def\HoLogoCs@iniTeX#1{#1{i}{I}niTeX}
%    \end{macrocode}
%    \end{macro}
%    \begin{macro}{\HoLogoBkm@iniTeX}
%    \begin{macrocode}
\def\HoLogoBkm@iniTeX#1{%
  #1{i}{I}ni\hologo{TeX}%
}
%    \end{macrocode}
%    \end{macro}
%    \begin{macro}{\HoLogoHtml@iniTeX}
%    \begin{macrocode}
\let\HoLogoHtml@iniTeX\HoLogo@iniTeX
%    \end{macrocode}
%    \end{macro}
%
% \subsubsection{\hologo{virTeX}}
%
%    \begin{macro}{\HoLogo@virTeX}
%    \begin{macrocode}
\def\HoLogo@virTeX#1{%
  \HOLOGO@mbox{%
    #1{v}{V}ir\hologo{TeX}%
  }%
}
%    \end{macrocode}
%    \end{macro}
%    \begin{macro}{\HoLogoCs@virTeX}
%    \begin{macrocode}
\def\HoLogoCs@virTeX#1{#1{v}{V}irTeX}
%    \end{macrocode}
%    \end{macro}
%    \begin{macro}{\HoLogoBkm@virTeX}
%    \begin{macrocode}
\def\HoLogoBkm@virTeX#1{%
  #1{v}{V}ir\hologo{TeX}%
}
%    \end{macrocode}
%    \end{macro}
%    \begin{macro}{\HoLogoHtml@virTeX}
%    \begin{macrocode}
\let\HoLogoHtml@virTeX\HoLogo@virTeX
%    \end{macrocode}
%    \end{macro}
%
% \subsubsection{\hologo{SliTeX}}
%
% \paragraph{Definitions of the three variants.}
%
%    \begin{macro}{\HoLogo@SLiTeX@lift}
%    \begin{macrocode}
\def\HoLogo@SLiTeX@lift#1{%
  \HoLogoFont@font{SliTeX}{rm}{%
    S%
    \kern-.06em%
    L%
    \kern-.18em%
    \raise.32ex\hbox{\HoLogoFont@font{SliTeX}{sc}{i}}%
    \HOLOGO@discretionary
    \kern-.06em%
    \hologo{TeX}%
  }%
}
%    \end{macrocode}
%    \end{macro}
%    \begin{macro}{\HoLogoBkm@SLiTeX@lift}
%    \begin{macrocode}
\def\HoLogoBkm@SLiTeX@lift#1{SLiTeX}
%    \end{macrocode}
%    \end{macro}
%    \begin{macro}{\HoLogoHtml@SLiTeX@lift}
%    \begin{macrocode}
\def\HoLogoHtml@SLiTeX@lift#1{%
  \HoLogoCss@SLiTeX@lift
  \HOLOGO@Span{SLiTeX-lift}{%
    \HoLogoFont@font{SliTeX}{rm}{%
      S%
      \HOLOGO@Span{L}{L}%
      \HOLOGO@Span{i}{i}%
      \hologo{TeX}%
    }%
  }%
}
%    \end{macrocode}
%    \end{macro}
%    \begin{macro}{\HoLogoCss@SLiTeX@lift}
%    \begin{macrocode}
\def\HoLogoCss@SLiTeX@lift{%
  \Css{%
    span.HoLogo-SLiTeX-lift span.HoLogo-L{%
      margin-left:-.06em;%
      margin-right:-.18em;%
    }%
  }%
  \Css{%
    span.HoLogo-SLiTeX-lift span.HoLogo-i{%
      position:relative;%
      top:-.32ex;%
      margin-right:-.06em;%
      font-variant:small-caps;%
    }%
  }%
  \global\let\HoLogoCss@SLiTeX@lift\relax
}
%    \end{macrocode}
%    \end{macro}
%
%    \begin{macro}{\HoLogo@SliTeX@simple}
%    \begin{macrocode}
\def\HoLogo@SliTeX@simple#1{%
  \HoLogoFont@font{SliTeX}{rm}{%
    \ltx@mbox{%
      \HoLogoFont@font{SliTeX}{sc}{Sli}%
    }%
    \HOLOGO@discretionary
    \hologo{TeX}%
  }%
}
%    \end{macrocode}
%    \end{macro}
%    \begin{macro}{\HoLogoBkm@SliTeX@simple}
%    \begin{macrocode}
\def\HoLogoBkm@SliTeX@simple#1{SliTeX}
%    \end{macrocode}
%    \end{macro}
%    \begin{macro}{\HoLogoHtml@SliTeX@simple}
%    \begin{macrocode}
\let\HoLogoHtml@SliTeX@simple\HoLogo@SliTeX@simple
%    \end{macrocode}
%    \end{macro}
%
%    \begin{macro}{\HoLogo@SliTeX@narrow}
%    \begin{macrocode}
\def\HoLogo@SliTeX@narrow#1{%
  \HoLogoFont@font{SliTeX}{rm}{%
    \ltx@mbox{%
      S%
      \kern-.06em%
      \HoLogoFont@font{SliTeX}{sc}{%
        l%
        \kern-.035em%
        i%
      }%
    }%
    \HOLOGO@discretionary
    \kern-.06em%
    \hologo{TeX}%
  }%
}
%    \end{macrocode}
%    \end{macro}
%    \begin{macro}{\HoLogoBkm@SliTeX@narrow}
%    \begin{macrocode}
\def\HoLogoBkm@SliTeX@narrow#1{SliTeX}
%    \end{macrocode}
%    \end{macro}
%    \begin{macro}{\HoLogoHtml@SliTeX@narrow}
%    \begin{macrocode}
\def\HoLogoHtml@SliTeX@narrow#1{%
  \HoLogoCss@SliTeX@narrow
  \HOLOGO@Span{SliTeX-narrow}{%
    \HoLogoFont@font{SliTeX}{rm}{%
      S%
        \HOLOGO@Span{l}{l}%
        \HOLOGO@Span{i}{i}%
      \hologo{TeX}%
    }%
  }%
}
%    \end{macrocode}
%    \end{macro}
%    \begin{macro}{\HoLogoCss@SliTeX@narrow}
%    \begin{macrocode}
\def\HoLogoCss@SliTeX@narrow{%
  \Css{%
    span.HoLogo-SliTeX-narrow span.HoLogo-l{%
      margin-left:-.06em;%
      margin-right:-.035em;%
      font-variant:small-caps;%
    }%
  }%
  \Css{%
    span.HoLogo-SliTeX-narrow span.HoLogo-i{%
      margin-right:-.06em;%
      font-variant:small-caps;%
    }%
  }%
  \global\let\HoLogoCss@SliTeX@narrow\relax
}
%    \end{macrocode}
%    \end{macro}
%
% \paragraph{Macro set completion.}
%
%    \begin{macro}{\HoLogo@SLiTeX@simple}
%    \begin{macrocode}
\def\HoLogo@SLiTeX@simple{\HoLogo@SliTeX@simple}
%    \end{macrocode}
%    \end{macro}
%    \begin{macro}{\HoLogoBkm@SLiTeX@simple}
%    \begin{macrocode}
\def\HoLogoBkm@SLiTeX@simple{\HoLogoBkm@SliTeX@simple}
%    \end{macrocode}
%    \end{macro}
%    \begin{macro}{\HoLogoHtml@SLiTeX@simple}
%    \begin{macrocode}
\def\HoLogoHtml@SLiTeX@simple{\HoLogoHtml@SliTeX@simple}
%    \end{macrocode}
%    \end{macro}
%
%    \begin{macro}{\HoLogo@SLiTeX@narrow}
%    \begin{macrocode}
\def\HoLogo@SLiTeX@narrow{\HoLogo@SliTeX@narrow}
%    \end{macrocode}
%    \end{macro}
%    \begin{macro}{\HoLogoBkm@SLiTeX@narrow}
%    \begin{macrocode}
\def\HoLogoBkm@SLiTeX@narrow{\HoLogoBkm@SliTeX@narrow}
%    \end{macrocode}
%    \end{macro}
%    \begin{macro}{\HoLogoHtml@SLiTeX@narrow}
%    \begin{macrocode}
\def\HoLogoHtml@SLiTeX@narrow{\HoLogoHtml@SliTeX@narrow}
%    \end{macrocode}
%    \end{macro}
%
%    \begin{macro}{\HoLogo@SliTeX@lift}
%    \begin{macrocode}
\def\HoLogo@SliTeX@lift{\HoLogo@SLiTeX@lift}
%    \end{macrocode}
%    \end{macro}
%    \begin{macro}{\HoLogoBkm@SliTeX@lift}
%    \begin{macrocode}
\def\HoLogoBkm@SliTeX@lift{\HoLogoBkm@SLiTeX@lift}
%    \end{macrocode}
%    \end{macro}
%    \begin{macro}{\HoLogoHtml@SliTeX@lift}
%    \begin{macrocode}
\def\HoLogoHtml@SliTeX@lift{\HoLogoHtml@SLiTeX@lift}
%    \end{macrocode}
%    \end{macro}
%
% \paragraph{Defaults.}
%
%    \begin{macro}{\HoLogo@SLiTeX}
%    \begin{macrocode}
\def\HoLogo@SLiTeX{\HoLogo@SLiTeX@lift}
%    \end{macrocode}
%    \end{macro}
%    \begin{macro}{\HoLogoBkm@SLiTeX}
%    \begin{macrocode}
\def\HoLogoBkm@SLiTeX{\HoLogoBkm@SLiTeX@lift}
%    \end{macrocode}
%    \end{macro}
%    \begin{macro}{\HoLogoHtml@SLiTeX}
%    \begin{macrocode}
\def\HoLogoHtml@SLiTeX{\HoLogoHtml@SLiTeX@lift}
%    \end{macrocode}
%    \end{macro}
%
%    \begin{macro}{\HoLogo@SliTeX}
%    \begin{macrocode}
\def\HoLogo@SliTeX{\HoLogo@SliTeX@narrow}
%    \end{macrocode}
%    \end{macro}
%    \begin{macro}{\HoLogoBkm@SliTeX}
%    \begin{macrocode}
\def\HoLogoBkm@SliTeX{\HoLogoBkm@SliTeX@narrow}
%    \end{macrocode}
%    \end{macro}
%    \begin{macro}{\HoLogoHtml@SliTeX}
%    \begin{macrocode}
\def\HoLogoHtml@SliTeX{\HoLogoHtml@SliTeX@narrow}
%    \end{macrocode}
%    \end{macro}
%
% \subsubsection{\hologo{LuaTeX}}
%
%    \begin{macro}{\HoLogo@LuaTeX}
%    The kerning is an idea of Hans Hagen, see mailing list
%    `luatex at tug dot org' in March 2010.
%    \begin{macrocode}
\def\HoLogo@LuaTeX#1{%
  \HOLOGO@mbox{%
    Lua%
    \HOLOGO@NegativeKerning{aT,oT,To}%
    \hologo{TeX}%
  }%
}
%    \end{macrocode}
%    \end{macro}
%    \begin{macro}{\HoLogoHtml@LuaTeX}
%    \begin{macrocode}
\let\HoLogoHtml@LuaTeX\HoLogo@LuaTeX
%    \end{macrocode}
%    \end{macro}
%
% \subsubsection{\hologo{LuaLaTeX}}
%
%    \begin{macro}{\HoLogo@LuaLaTeX}
%    \begin{macrocode}
\def\HoLogo@LuaLaTeX#1{%
  \HOLOGO@mbox{%
    Lua%
    \hologo{LaTeX}%
  }%
}
%    \end{macrocode}
%    \end{macro}
%    \begin{macro}{\HoLogoHtml@LuaLaTeX}
%    \begin{macrocode}
\let\HoLogoHtml@LuaLaTeX\HoLogo@LuaLaTeX
%    \end{macrocode}
%    \end{macro}
%
% \subsubsection{\hologo{XeTeX}, \hologo{XeLaTeX}}
%
%    \begin{macro}{\HOLOGO@IfCharExists}
%    \begin{macrocode}
\ifluatex
  \ifnum\luatexversion<36 %
  \else
    \def\HOLOGO@IfCharExists#1{%
      \ifnum
        \directlua{%
           if luaotfload and luaotfload.aux then
             if luaotfload.aux.font_has_glyph(%
                    font.current(), \number#1) then % 	 
	       tex.print("1") % 	 
	     end % 	 
	   elseif font and font.fonts and font.current then %
            local f = font.fonts[font.current()]%
            if f.characters and f.characters[\number#1] then %
              tex.print("1")%
            end %
          end%
        }0=\ltx@zero
        \expandafter\ltx@secondoftwo
      \else
        \expandafter\ltx@firstoftwo
      \fi
    }%
  \fi
\fi
\ltx@IfUndefined{HOLOGO@IfCharExists}{%
  \def\HOLOGO@@IfCharExists#1{%
    \begingroup
      \tracinglostchars=\ltx@zero
      \setbox\ltx@zero=\hbox{%
        \kern7sp\char#1\relax
        \ifnum\lastkern>\ltx@zero
          \expandafter\aftergroup\csname iffalse\endcsname
        \else
          \expandafter\aftergroup\csname iftrue\endcsname
        \fi
      }%
      % \if{true|false} from \aftergroup
      \endgroup
      \expandafter\ltx@firstoftwo
    \else
      \endgroup
      \expandafter\ltx@secondoftwo
    \fi
  }%
  \ifxetex
    \ltx@IfUndefined{XeTeXfonttype}{}{%
      \ltx@IfUndefined{XeTeXcharglyph}{}{%
        \def\HOLOGO@IfCharExists#1{%
          \ifnum\XeTeXfonttype\font>\ltx@zero
            \expandafter\ltx@firstofthree
          \else
            \expandafter\ltx@gobble
          \fi
          {%
            \ifnum\XeTeXcharglyph#1>\ltx@zero
              \expandafter\ltx@firstoftwo
            \else
              \expandafter\ltx@secondoftwo
            \fi
          }%
          \HOLOGO@@IfCharExists{#1}%
        }%
      }%
    }%
  \fi
}{}
\ltx@ifundefined{HOLOGO@IfCharExists}{%
  \ifnum64=`\^^^^0040\relax % test for big chars of LuaTeX/XeTeX
    \let\HOLOGO@IfCharExists\HOLOGO@@IfCharExists
  \else
    \def\HOLOGO@IfCharExists#1{%
      \ifnum#1>255 %
        \expandafter\ltx@fourthoffour
      \fi
      \HOLOGO@@IfCharExists{#1}%
    }%
  \fi
}{}
%    \end{macrocode}
%    \end{macro}
%
%    \begin{macro}{\HoLogo@Xe}
%    Source: package \xpackage{dtklogos}
%    \begin{macrocode}
\def\HoLogo@Xe#1{%
  X%
  \kern-.1em\relax
  \HOLOGO@IfCharExists{"018E}{%
    \lower.5ex\hbox{\char"018E}%
  }{%
    \chardef\HOLOGO@choice=\ltx@zero
    \ifdim\fontdimen\ltx@one\font>0pt %
      \ltx@IfUndefined{rotatebox}{%
        \ltx@IfUndefined{pgftext}{%
          \ltx@IfUndefined{psscalebox}{%
            \ltx@IfUndefined{HOLOGO@ScaleBox@\hologoDriver}{%
            }{%
              \chardef\HOLOGO@choice=4 %
            }%
          }{%
            \chardef\HOLOGO@choice=3 %
          }%
        }{%
          \chardef\HOLOGO@choice=2 %
        }%
      }{%
        \chardef\HOLOGO@choice=1 %
      }%
      \ifcase\HOLOGO@choice
        \HOLOGO@WarningUnsupportedDriver{Xe}%
        e%
      \or % 1: \rotatebox
        \begingroup
          \setbox\ltx@zero\hbox{\rotatebox{180}{E}}%
          \ltx@LocDimenA=\dp\ltx@zero
          \advance\ltx@LocDimenA by -.5ex\relax
          \raise\ltx@LocDimenA\box\ltx@zero
        \endgroup
      \or % 2: \pgftext
        \lower.5ex\hbox{%
          \pgfpicture
            \pgftext[rotate=180]{E}%
          \endpgfpicture
        }%
      \or % 3: \psscalebox
        \begingroup
          \setbox\ltx@zero\hbox{\psscalebox{-1 -1}{E}}%
          \ltx@LocDimenA=\dp\ltx@zero
          \advance\ltx@LocDimenA by -.5ex\relax
          \raise\ltx@LocDimenA\box\ltx@zero
        \endgroup
      \or % 4: \HOLOGO@PointReflectBox
        \lower.5ex\hbox{\HOLOGO@PointReflectBox{E}}%
      \else
        \@PackageError{hologo}{Internal error (choice/it}\@ehc
      \fi
    \else
      \ltx@IfUndefined{reflectbox}{%
        \ltx@IfUndefined{pgftext}{%
          \ltx@IfUndefined{psscalebox}{%
            \ltx@IfUndefined{HOLOGO@ScaleBox@\hologoDriver}{%
            }{%
              \chardef\HOLOGO@choice=4 %
            }%
          }{%
            \chardef\HOLOGO@choice=3 %
          }%
        }{%
          \chardef\HOLOGO@choice=2 %
        }%
      }{%
        \chardef\HOLOGO@choice=1 %
      }%
      \ifcase\HOLOGO@choice
        \HOLOGO@WarningUnsupportedDriver{Xe}%
        e%
      \or % 1: reflectbox
        \lower.5ex\hbox{%
          \reflectbox{E}%
        }%
      \or % 2: \pgftext
        \lower.5ex\hbox{%
          \pgfpicture
            \pgftransformxscale{-1}%
            \pgftext{E}%
          \endpgfpicture
        }%
      \or % 3: \psscalebox
        \lower.5ex\hbox{%
          \psscalebox{-1 1}{E}%
        }%
      \or % 4: \HOLOGO@Reflectbox
        \lower.5ex\hbox{%
          \HOLOGO@ReflectBox{E}%
        }%
      \else
        \@PackageError{hologo}{Internal error (choice/up)}\@ehc
      \fi
    \fi
  }%
}
%    \end{macrocode}
%    \end{macro}
%    \begin{macro}{\HoLogoHtml@Xe}
%    \begin{macrocode}
\def\HoLogoHtml@Xe#1{%
  \HoLogoCss@Xe
  \HOLOGO@Span{Xe}{%
    X%
    \HOLOGO@Span{e}{%
      \HCode{&\ltx@hashchar x018e;}%
    }%
  }%
}
%    \end{macrocode}
%    \end{macro}
%    \begin{macro}{\HoLogoCss@Xe}
%    \begin{macrocode}
\def\HoLogoCss@Xe{%
  \Css{%
    span.HoLogo-Xe span.HoLogo-e{%
      position:relative;%
      top:.5ex;%
      left-margin:-.1em;%
    }%
  }%
  \global\let\HoLogoCss@Xe\relax
}
%    \end{macrocode}
%    \end{macro}
%
%    \begin{macro}{\HoLogo@XeTeX}
%    \begin{macrocode}
\def\HoLogo@XeTeX#1{%
  \hologo{Xe}%
  \kern-.15em\relax
  \hologo{TeX}%
}
%    \end{macrocode}
%    \end{macro}
%
%    \begin{macro}{\HoLogoHtml@XeTeX}
%    \begin{macrocode}
\def\HoLogoHtml@XeTeX#1{%
  \HoLogoCss@XeTeX
  \HOLOGO@Span{XeTeX}{%
    \hologo{Xe}%
    \hologo{TeX}%
  }%
}
%    \end{macrocode}
%    \end{macro}
%    \begin{macro}{\HoLogoCss@XeTeX}
%    \begin{macrocode}
\def\HoLogoCss@XeTeX{%
  \Css{%
    span.HoLogo-XeTeX span.HoLogo-TeX{%
      margin-left:-.15em;%
    }%
  }%
  \global\let\HoLogoCss@XeTeX\relax
}
%    \end{macrocode}
%    \end{macro}
%
%    \begin{macro}{\HoLogo@XeLaTeX}
%    \begin{macrocode}
\def\HoLogo@XeLaTeX#1{%
  \hologo{Xe}%
  \kern-.13em%
  \hologo{LaTeX}%
}
%    \end{macrocode}
%    \end{macro}
%    \begin{macro}{\HoLogoHtml@XeLaTeX}
%    \begin{macrocode}
\def\HoLogoHtml@XeLaTeX#1{%
  \HoLogoCss@XeLaTeX
  \HOLOGO@Span{XeLaTeX}{%
    \hologo{Xe}%
    \hologo{LaTeX}%
  }%
}
%    \end{macrocode}
%    \end{macro}
%    \begin{macro}{\HoLogoCss@XeLaTeX}
%    \begin{macrocode}
\def\HoLogoCss@XeLaTeX{%
  \Css{%
    span.HoLogo-XeLaTeX span.HoLogo-Xe{%
      margin-right:-.13em;%
    }%
  }%
  \global\let\HoLogoCss@XeLaTeX\relax
}
%    \end{macrocode}
%    \end{macro}
%
% \subsubsection{\hologo{pdfTeX}, \hologo{pdfLaTeX}}
%
%    \begin{macro}{\HoLogo@pdfTeX}
%    \begin{macrocode}
\def\HoLogo@pdfTeX#1{%
  \HOLOGO@mbox{%
    #1{p}{P}df\hologo{TeX}%
  }%
}
%    \end{macrocode}
%    \end{macro}
%    \begin{macro}{\HoLogoCs@pdfTeX}
%    \begin{macrocode}
\def\HoLogoCs@pdfTeX#1{#1{p}{P}dfTeX}
%    \end{macrocode}
%    \end{macro}
%    \begin{macro}{\HoLogoBkm@pdfTeX}
%    \begin{macrocode}
\def\HoLogoBkm@pdfTeX#1{%
  #1{p}{P}df\hologo{TeX}%
}
%    \end{macrocode}
%    \end{macro}
%    \begin{macro}{\HoLogoHtml@pdfTeX}
%    \begin{macrocode}
\let\HoLogoHtml@pdfTeX\HoLogo@pdfTeX
%    \end{macrocode}
%    \end{macro}
%
%    \begin{macro}{\HoLogo@pdfLaTeX}
%    \begin{macrocode}
\def\HoLogo@pdfLaTeX#1{%
  \HOLOGO@mbox{%
    #1{p}{P}df\hologo{LaTeX}%
  }%
}
%    \end{macrocode}
%    \end{macro}
%    \begin{macro}{\HoLogoCs@pdfLaTeX}
%    \begin{macrocode}
\def\HoLogoCs@pdfLaTeX#1{#1{p}{P}dfLaTeX}
%    \end{macrocode}
%    \end{macro}
%    \begin{macro}{\HoLogoBkm@pdfLaTeX}
%    \begin{macrocode}
\def\HoLogoBkm@pdfLaTeX#1{%
  #1{p}{P}df\hologo{LaTeX}%
}
%    \end{macrocode}
%    \end{macro}
%    \begin{macro}{\HoLogoHtml@pdfLaTeX}
%    \begin{macrocode}
\let\HoLogoHtml@pdfLaTeX\HoLogo@pdfLaTeX
%    \end{macrocode}
%    \end{macro}
%
% \subsubsection{\hologo{VTeX}}
%
%    \begin{macro}{\HoLogo@VTeX}
%    \begin{macrocode}
\def\HoLogo@VTeX#1{%
  \HOLOGO@mbox{%
    V\hologo{TeX}%
  }%
}
%    \end{macrocode}
%    \end{macro}
%    \begin{macro}{\HoLogoHtml@VTeX}
%    \begin{macrocode}
\let\HoLogoHtml@VTeX\HoLogo@VTeX
%    \end{macrocode}
%    \end{macro}
%
% \subsubsection{\hologo{AmS}, \dots}
%
%    Source: class \xclass{amsdtx}
%
%    \begin{macro}{\HoLogo@AmS}
%    \begin{macrocode}
\def\HoLogo@AmS#1{%
  \HoLogoFont@font{AmS}{sy}{%
    A%
    \kern-.1667em%
    \lower.5ex\hbox{M}%
    \kern-.125em%
    S%
  }%
}
%    \end{macrocode}
%    \end{macro}
%    \begin{macro}{\HoLogoBkm@AmS}
%    \begin{macrocode}
\def\HoLogoBkm@AmS#1{AmS}
%    \end{macrocode}
%    \end{macro}
%    \begin{macro}{\HoLogoHtml@AmS}
%    \begin{macrocode}
\def\HoLogoHtml@AmS#1{%
  \HoLogoCss@AmS
%  \HoLogoFont@font{AmS}{sy}{%
    \HOLOGO@Span{AmS}{%
      A%
      \HOLOGO@Span{M}{M}%
      S%
    }%
%   }%
}
%    \end{macrocode}
%    \end{macro}
%    \begin{macro}{\HoLogoCss@AmS}
%    \begin{macrocode}
\def\HoLogoCss@AmS{%
  \Css{%
    span.HoLogo-AmS span.HoLogo-M{%
      position:relative;%
      top:.5ex;%
      margin-left:-.1667em;%
      margin-right:-.125em;%
      text-decoration:none;%
    }%
  }%
  \global\let\HoLogoCss@AmS\relax
}
%    \end{macrocode}
%    \end{macro}
%
%    \begin{macro}{\HoLogo@AmSTeX}
%    \begin{macrocode}
\def\HoLogo@AmSTeX#1{%
  \hologo{AmS}%
  \HOLOGO@hyphen
  \hologo{TeX}%
}
%    \end{macrocode}
%    \end{macro}
%    \begin{macro}{\HoLogoBkm@AmSTeX}
%    \begin{macrocode}
\def\HoLogoBkm@AmSTeX#1{AmS-TeX}%
%    \end{macrocode}
%    \end{macro}
%    \begin{macro}{\HoLogoHtml@AmSTeX}
%    \begin{macrocode}
\let\HoLogoHtml@AmSTeX\HoLogo@AmSTeX
%    \end{macrocode}
%    \end{macro}
%
%    \begin{macro}{\HoLogo@AmSLaTeX}
%    \begin{macrocode}
\def\HoLogo@AmSLaTeX#1{%
  \hologo{AmS}%
  \HOLOGO@hyphen
  \hologo{LaTeX}%
}
%    \end{macrocode}
%    \end{macro}
%    \begin{macro}{\HoLogoBkm@AmSLaTeX}
%    \begin{macrocode}
\def\HoLogoBkm@AmSLaTeX#1{AmS-LaTeX}%
%    \end{macrocode}
%    \end{macro}
%    \begin{macro}{\HoLogoHtml@AmSLaTeX}
%    \begin{macrocode}
\let\HoLogoHtml@AmSLaTeX\HoLogo@AmSLaTeX
%    \end{macrocode}
%    \end{macro}
%
% \subsubsection{\hologo{BibTeX}}
%
%    \begin{macro}{\HoLogo@BibTeX@sc}
%    A definition of \hologo{BibTeX} is provided in
%    the documentation source for the manual of \hologo{BibTeX}
%    \cite{btxdoc}.
%\begin{quote}
%\begin{verbatim}
%\def\BibTeX{%
%  {%
%    \rm
%    B%
%    \kern-.05em%
%    {%
%      \sc
%      i%
%      \kern-.025em %
%      b%
%    }%
%    \kern-.08em
%    T%
%    \kern-.1667em%
%    \lower.7ex\hbox{E}%
%    \kern-.125em%
%    X%
%  }%
%}
%\end{verbatim}
%\end{quote}
%    \begin{macrocode}
\def\HoLogo@BibTeX@sc#1{%
  B%
  \kern-.05em%
  \HoLogoFont@font{BibTeX}{sc}{%
    i%
    \kern-.025em%
    b%
  }%
  \HOLOGO@discretionary
  \kern-.08em%
  \hologo{TeX}%
}
%    \end{macrocode}
%    \end{macro}
%    \begin{macro}{\HoLogoHtml@BibTeX@sc}
%    \begin{macrocode}
\def\HoLogoHtml@BibTeX@sc#1{%
  \HoLogoCss@BibTeX@sc
  \HOLOGO@Span{BibTeX-sc}{%
    B%
    \HOLOGO@Span{i}{i}%
    \HOLOGO@Span{b}{b}%
    \hologo{TeX}%
  }%
}
%    \end{macrocode}
%    \end{macro}
%    \begin{macro}{\HoLogoCss@BibTeX@sc}
%    \begin{macrocode}
\def\HoLogoCss@BibTeX@sc{%
  \Css{%
    span.HoLogo-BibTeX-sc span.HoLogo-i{%
      margin-left:-.05em;%
      margin-right:-.025em;%
      font-variant:small-caps;%
    }%
  }%
  \Css{%
    span.HoLogo-BibTeX-sc span.HoLogo-b{%
      margin-right:-.08em;%
      font-variant:small-caps;%
    }%
  }%
  \global\let\HoLogoCss@BibTeX@sc\relax
}
%    \end{macrocode}
%    \end{macro}
%
%    \begin{macro}{\HoLogo@BibTeX@sf}
%    Variant \xoption{sf} avoids trouble with unavailable
%    small caps fonts (e.g., bold versions of Computer Modern or
%    Latin Modern). The definition is taken from
%    package \xpackage{dtklogos} \cite{dtklogos}.
%\begin{quote}
%\begin{verbatim}
%\DeclareRobustCommand{\BibTeX}{%
%  B%
%  \kern-.05em%
%  \hbox{%
%    $\m@th$% %% force math size calculations
%    \csname S@\f@size\endcsname
%    \fontsize\sf@size\z@
%    \math@fontsfalse
%    \selectfont
%    I%
%    \kern-.025em%
%    B
%  }%
%  \kern-.08em%
%  \-%
%  \TeX
%}
%\end{verbatim}
%\end{quote}
%    \begin{macrocode}
\def\HoLogo@BibTeX@sf#1{%
  B%
  \kern-.05em%
  \HoLogoFont@font{BibTeX}{bibsf}{%
    I%
    \kern-.025em%
    B%
  }%
  \HOLOGO@discretionary
  \kern-.08em%
  \hologo{TeX}%
}
%    \end{macrocode}
%    \end{macro}
%    \begin{macro}{\HoLogoHtml@BibTeX@sf}
%    \begin{macrocode}
\def\HoLogoHtml@BibTeX@sf#1{%
  \HoLogoCss@BibTeX@sf
  \HOLOGO@Span{BibTeX-sf}{%
    B%
    \HoLogoFont@font{BibTeX}{bibsf}{%
      \HOLOGO@Span{i}{I}%
      B%
    }%
    \hologo{TeX}%
  }%
}
%    \end{macrocode}
%    \end{macro}
%    \begin{macro}{\HoLogoCss@BibTeX@sf}
%    \begin{macrocode}
\def\HoLogoCss@BibTeX@sf{%
  \Css{%
    span.HoLogo-BibTeX-sf span.HoLogo-i{%
      margin-left:-.05em;%
      margin-right:-.025em;%
    }%
  }%
  \Css{%
    span.HoLogo-BibTeX-sf span.HoLogo-TeX{%
      margin-left:-.08em;%
    }%
  }%
  \global\let\HoLogoCss@BibTeX@sf\relax
}
%    \end{macrocode}
%    \end{macro}
%
%    \begin{macro}{\HoLogo@BibTeX}
%    \begin{macrocode}
\def\HoLogo@BibTeX{\HoLogo@BibTeX@sf}
%    \end{macrocode}
%    \end{macro}
%    \begin{macro}{\HoLogoHtml@BibTeX}
%    \begin{macrocode}
\def\HoLogoHtml@BibTeX{\HoLogoHtml@BibTeX@sf}
%    \end{macrocode}
%    \end{macro}
%
% \subsubsection{\hologo{BibTeX8}}
%
%    \begin{macro}{\HoLogo@BibTeX8}
%    \begin{macrocode}
\expandafter\def\csname HoLogo@BibTeX8\endcsname#1{%
  \hologo{BibTeX}%
  8%
}
%    \end{macrocode}
%    \end{macro}
%
%    \begin{macro}{\HoLogoBkm@BibTeX8}
%    \begin{macrocode}
\expandafter\def\csname HoLogoBkm@BibTeX8\endcsname#1{%
  \hologo{BibTeX}%
  8%
}
%    \end{macrocode}
%    \end{macro}
%    \begin{macro}{\HoLogoHtml@BibTeX8}
%    \begin{macrocode}
\expandafter
\let\csname HoLogoHtml@BibTeX8\expandafter\endcsname
\csname HoLogo@BibTeX8\endcsname
%    \end{macrocode}
%    \end{macro}
%
% \subsubsection{\hologo{ConTeXt}}
%
%    \begin{macro}{\HoLogo@ConTeXt@simple}
%    \begin{macrocode}
\def\HoLogo@ConTeXt@simple#1{%
  \HOLOGO@mbox{Con}%
  \HOLOGO@discretionary
  \HOLOGO@mbox{\hologo{TeX}t}%
}
%    \end{macrocode}
%    \end{macro}
%    \begin{macro}{\HoLogoHtml@ConTeXt@simple}
%    \begin{macrocode}
\let\HoLogoHtml@ConTeXt@simple\HoLogo@ConTeXt@simple
%    \end{macrocode}
%    \end{macro}
%
%    \begin{macro}{\HoLogo@ConTeXt@narrow}
%    This definition of logo \hologo{ConTeXt} with variant \xoption{narrow}
%    comes from TUGboat's class \xclass{ltugboat} (version 2010/11/15 v2.8).
%    \begin{macrocode}
\def\HoLogo@ConTeXt@narrow#1{%
  \HOLOGO@mbox{C\kern-.0333emon}%
  \HOLOGO@discretionary
  \kern-.0667em%
  \HOLOGO@mbox{\hologo{TeX}\kern-.0333emt}%
}
%    \end{macrocode}
%    \end{macro}
%    \begin{macro}{\HoLogoHtml@ConTeXt@narrow}
%    \begin{macrocode}
\def\HoLogoHtml@ConTeXt@narrow#1{%
  \HoLogoCss@ConTeXt@narrow
  \HOLOGO@Span{ConTeXt-narrow}{%
    \HOLOGO@Span{C}{C}%
    on%
    \hologo{TeX}%
    t%
  }%
}
%    \end{macrocode}
%    \end{macro}
%    \begin{macro}{\HoLogoCss@ConTeXt@narrow}
%    \begin{macrocode}
\def\HoLogoCss@ConTeXt@narrow{%
  \Css{%
    span.HoLogo-ConTeXt-narrow span.HoLogo-C{%
      margin-left:-.0333em;%
    }%
  }%
  \Css{%
    span.HoLogo-ConTeXt-narrow span.HoLogo-TeX{%
      margin-left:-.0667em;%
      margin-right:-.0333em;%
    }%
  }%
  \global\let\HoLogoCss@ConTeXt@narrow\relax
}
%    \end{macrocode}
%    \end{macro}
%
%    \begin{macro}{\HoLogo@ConTeXt}
%    \begin{macrocode}
\def\HoLogo@ConTeXt{\HoLogo@ConTeXt@narrow}
%    \end{macrocode}
%    \end{macro}
%    \begin{macro}{\HoLogoHtml@ConTeXt}
%    \begin{macrocode}
\def\HoLogoHtml@ConTeXt{\HoLogoHtml@ConTeXt@narrow}
%    \end{macrocode}
%    \end{macro}
%
% \subsubsection{\hologo{emTeX}}
%
%    \begin{macro}{\HoLogo@emTeX}
%    \begin{macrocode}
\def\HoLogo@emTeX#1{%
  \HOLOGO@mbox{#1{e}{E}m}%
  \HOLOGO@discretionary
  \hologo{TeX}%
}
%    \end{macrocode}
%    \end{macro}
%    \begin{macro}{\HoLogoCs@emTeX}
%    \begin{macrocode}
\def\HoLogoCs@emTeX#1{#1{e}{E}mTeX}%
%    \end{macrocode}
%    \end{macro}
%    \begin{macro}{\HoLogoBkm@emTeX}
%    \begin{macrocode}
\def\HoLogoBkm@emTeX#1{%
  #1{e}{E}m\hologo{TeX}%
}
%    \end{macrocode}
%    \end{macro}
%    \begin{macro}{\HoLogoHtml@emTeX}
%    \begin{macrocode}
\let\HoLogoHtml@emTeX\HoLogo@emTeX
%    \end{macrocode}
%    \end{macro}
%
% \subsubsection{\hologo{ExTeX}}
%
%    \begin{macro}{\HoLogo@ExTeX}
%    The definition is taken from the FAQ of the
%    project \hologo{ExTeX}
%    \cite{ExTeX-FAQ}.
%\begin{quote}
%\begin{verbatim}
%\def\ExTeX{%
%  \textrm{% Logo always with serifs
%    \ensuremath{%
%      \textstyle
%      \varepsilon_{%
%        \kern-0.15em%
%        \mathcal{X}%
%      }%
%    }%
%    \kern-.15em%
%    \TeX
%  }%
%}
%\end{verbatim}
%\end{quote}
%    \begin{macrocode}
\def\HoLogo@ExTeX#1{%
  \HoLogoFont@font{ExTeX}{rm}{%
    \ltx@mbox{%
      \HOLOGO@MathSetup
      $%
        \textstyle
        \varepsilon_{%
          \kern-0.15em%
          \HoLogoFont@font{ExTeX}{sy}{X}%
        }%
      $%
    }%
    \HOLOGO@discretionary
    \kern-.15em%
    \hologo{TeX}%
  }%
}
%    \end{macrocode}
%    \end{macro}
%    \begin{macro}{\HoLogoHtml@ExTeX}
%    \begin{macrocode}
\def\HoLogoHtml@ExTeX#1{%
  \HoLogoCss@ExTeX
  \HoLogoFont@font{ExTeX}{rm}{%
    \HOLOGO@Span{ExTeX}{%
      \ltx@mbox{%
        \HOLOGO@MathSetup
        $\textstyle\varepsilon$%
        \HOLOGO@Span{X}{$\textstyle\chi$}%
        \hologo{TeX}%
      }%
    }%
  }%
}
%    \end{macrocode}
%    \end{macro}
%    \begin{macro}{\HoLogoBkm@ExTeX}
%    \begin{macrocode}
\def\HoLogoBkm@ExTeX#1{%
  \HOLOGO@PdfdocUnicode{#1{e}{E}x}{\textepsilon\textchi}%
  \hologo{TeX}%
}
%    \end{macrocode}
%    \end{macro}
%    \begin{macro}{\HoLogoCss@ExTeX}
%    \begin{macrocode}
\def\HoLogoCss@ExTeX{%
  \Css{%
    span.HoLogo-ExTeX{%
      font-family:serif;%
    }%
  }%
  \Css{%
    span.HoLogo-ExTeX span.HoLogo-TeX{%
      margin-left:-.15em;%
    }%
  }%
  \global\let\HoLogoCss@ExTeX\relax
}
%    \end{macrocode}
%    \end{macro}
%
% \subsubsection{\hologo{MiKTeX}}
%
%    \begin{macro}{\HoLogo@MiKTeX}
%    \begin{macrocode}
\def\HoLogo@MiKTeX#1{%
  \HOLOGO@mbox{MiK}%
  \HOLOGO@discretionary
  \hologo{TeX}%
}
%    \end{macrocode}
%    \end{macro}
%    \begin{macro}{\HoLogoHtml@MiKTeX}
%    \begin{macrocode}
\let\HoLogoHtml@MiKTeX\HoLogo@MiKTeX
%    \end{macrocode}
%    \end{macro}
%
% \subsubsection{\hologo{OzTeX} and friends}
%
%    Source: \hologo{OzTeX} FAQ \cite{OzTeX}:
%    \begin{quote}
%      |\def\OzTeX{O\kern-.03em z\kern-.15em\TeX}|\\
%      (There is no kerning in OzMF, OzMP and OzTtH.)
%    \end{quote}
%
%    \begin{macro}{\HoLogo@OzTeX}
%    \begin{macrocode}
\def\HoLogo@OzTeX#1{%
  O%
  \kern-.03em %
  z%
  \kern-.15em %
  \hologo{TeX}%
}
%    \end{macrocode}
%    \end{macro}
%    \begin{macro}{\HoLogoHtml@OzTeX}
%    \begin{macrocode}
\def\HoLogoHtml@OzTeX#1{%
  \HoLogoCss@OzTeX
  \HOLOGO@Span{OzTeX}{%
    O%
    \HOLOGO@Span{z}{z}%
    \hologo{TeX}%
  }%
}
%    \end{macrocode}
%    \end{macro}
%    \begin{macro}{\HoLogoCss@OzTeX}
%    \begin{macrocode}
\def\HoLogoCss@OzTeX{%
  \Css{%
    span.HoLogo-OzTeX span.HoLogo-z{%
      margin-left:-.03em;%
      margin-right:-.15em;%
    }%
  }%
  \global\let\HoLogoCss@OzTeX\relax
}
%    \end{macrocode}
%    \end{macro}
%
%    \begin{macro}{\HoLogo@OzMF}
%    \begin{macrocode}
\def\HoLogo@OzMF#1{%
  \HOLOGO@mbox{OzMF}%
}
%    \end{macrocode}
%    \end{macro}
%    \begin{macro}{\HoLogo@OzMP}
%    \begin{macrocode}
\def\HoLogo@OzMP#1{%
  \HOLOGO@mbox{OzMP}%
}
%    \end{macrocode}
%    \end{macro}
%    \begin{macro}{\HoLogo@OzTtH}
%    \begin{macrocode}
\def\HoLogo@OzTtH#1{%
  \HOLOGO@mbox{OzTtH}%
}
%    \end{macrocode}
%    \end{macro}
%
% \subsubsection{\hologo{PCTeX}}
%
%    \begin{macro}{\HoLogo@PCTeX}
%    \begin{macrocode}
\def\HoLogo@PCTeX#1{%
  \HOLOGO@mbox{PC}%
  \hologo{TeX}%
}
%    \end{macrocode}
%    \end{macro}
%    \begin{macro}{\HoLogoHtml@PCTeX}
%    \begin{macrocode}
\let\HoLogoHtml@PCTeX\HoLogo@PCTeX
%    \end{macrocode}
%    \end{macro}
%
% \subsubsection{\hologo{PiCTeX}}
%
%    The original definitions from \xfile{pictex.tex} \cite{PiCTeX}:
%\begin{quote}
%\begin{verbatim}
%\def\PiC{%
%  P%
%  \kern-.12em%
%  \lower.5ex\hbox{I}%
%  \kern-.075em%
%  C%
%}
%\def\PiCTeX{%
%  \PiC
%  \kern-.11em%
%  \TeX
%}
%\end{verbatim}
%\end{quote}
%
%    \begin{macro}{\HoLogo@PiC}
%    \begin{macrocode}
\def\HoLogo@PiC#1{%
  P%
  \kern-.12em%
  \lower.5ex\hbox{I}%
  \kern-.075em%
  C%
  \HOLOGO@SpaceFactor
}
%    \end{macrocode}
%    \end{macro}
%    \begin{macro}{\HoLogoHtml@PiC}
%    \begin{macrocode}
\def\HoLogoHtml@PiC#1{%
  \HoLogoCss@PiC
  \HOLOGO@Span{PiC}{%
    P%
    \HOLOGO@Span{i}{I}%
    C%
  }%
}
%    \end{macrocode}
%    \end{macro}
%    \begin{macro}{\HoLogoCss@PiC}
%    \begin{macrocode}
\def\HoLogoCss@PiC{%
  \Css{%
    span.HoLogo-PiC span.HoLogo-i{%
      position:relative;%
      top:.5ex;%
      margin-left:-.12em;%
      margin-right:-.075em;%
      text-decoration:none;%
    }%
  }%
  \global\let\HoLogoCss@PiC\relax
}
%    \end{macrocode}
%    \end{macro}
%
%    \begin{macro}{\HoLogo@PiCTeX}
%    \begin{macrocode}
\def\HoLogo@PiCTeX#1{%
  \hologo{PiC}%
  \HOLOGO@discretionary
  \kern-.11em%
  \hologo{TeX}%
}
%    \end{macrocode}
%    \end{macro}
%    \begin{macro}{\HoLogoHtml@PiCTeX}
%    \begin{macrocode}
\def\HoLogoHtml@PiCTeX#1{%
  \HoLogoCss@PiCTeX
  \HOLOGO@Span{PiCTeX}{%
    \hologo{PiC}%
    \hologo{TeX}%
  }%
}
%    \end{macrocode}
%    \end{macro}
%    \begin{macro}{\HoLogoCss@PiCTeX}
%    \begin{macrocode}
\def\HoLogoCss@PiCTeX{%
  \Css{%
    span.HoLogo-PiCTeX span.HoLogo-PiC{%
      margin-right:-.11em;%
    }%
  }%
  \global\let\HoLogoCss@PiCTeX\relax
}
%    \end{macrocode}
%    \end{macro}
%
% \subsubsection{\hologo{teTeX}}
%
%    \begin{macro}{\HoLogo@teTeX}
%    \begin{macrocode}
\def\HoLogo@teTeX#1{%
  \HOLOGO@mbox{#1{t}{T}e}%
  \HOLOGO@discretionary
  \hologo{TeX}%
}
%    \end{macrocode}
%    \end{macro}
%    \begin{macro}{\HoLogoCs@teTeX}
%    \begin{macrocode}
\def\HoLogoCs@teTeX#1{#1{t}{T}dfTeX}
%    \end{macrocode}
%    \end{macro}
%    \begin{macro}{\HoLogoBkm@teTeX}
%    \begin{macrocode}
\def\HoLogoBkm@teTeX#1{%
  #1{t}{T}e\hologo{TeX}%
}
%    \end{macrocode}
%    \end{macro}
%    \begin{macro}{\HoLogoHtml@teTeX}
%    \begin{macrocode}
\let\HoLogoHtml@teTeX\HoLogo@teTeX
%    \end{macrocode}
%    \end{macro}
%
% \subsubsection{\hologo{TeX4ht}}
%
%    \begin{macro}{\HoLogo@TeX4ht}
%    \begin{macrocode}
\expandafter\def\csname HoLogo@TeX4ht\endcsname#1{%
  \HOLOGO@mbox{\hologo{TeX}4ht}%
}
%    \end{macrocode}
%    \end{macro}
%    \begin{macro}{\HoLogoHtml@TeX4ht}
%    \begin{macrocode}
\expandafter
\let\csname HoLogoHtml@TeX4ht\expandafter\endcsname
\csname HoLogo@TeX4ht\endcsname
%    \end{macrocode}
%    \end{macro}
%
%
% \subsubsection{\hologo{SageTeX}}
%
%    \begin{macro}{\HoLogo@SageTeX}
%    \begin{macrocode}
\def\HoLogo@SageTeX#1{%
  \HOLOGO@mbox{Sage}%
  \HOLOGO@discretionary
  \HOLOGO@NegativeKerning{eT,oT,To}%
  \hologo{TeX}%
}
%    \end{macrocode}
%    \end{macro}
%    \begin{macro}{\HoLogoHtml@SageTeX}
%    \begin{macrocode}
\let\HoLogoHtml@SageTeX\HoLogo@SageTeX
%    \end{macrocode}
%    \end{macro}
%
% \subsection{\hologo{METAFONT} and friends}
%
%    \begin{macro}{\HoLogo@METAFONT}
%    \begin{macrocode}
\def\HoLogo@METAFONT#1{%
  \HoLogoFont@font{METAFONT}{logo}{%
    \HOLOGO@mbox{META}%
    \HOLOGO@discretionary
    \HOLOGO@mbox{FONT}%
  }%
}
%    \end{macrocode}
%    \end{macro}
%
%    \begin{macro}{\HoLogo@METAPOST}
%    \begin{macrocode}
\def\HoLogo@METAPOST#1{%
  \HoLogoFont@font{METAPOST}{logo}{%
    \HOLOGO@mbox{META}%
    \HOLOGO@discretionary
    \HOLOGO@mbox{POST}%
  }%
}
%    \end{macrocode}
%    \end{macro}
%
%    \begin{macro}{\HoLogo@MetaFun}
%    \begin{macrocode}
\def\HoLogo@MetaFun#1{%
  \HOLOGO@mbox{Meta}%
  \HOLOGO@discretionary
  \HOLOGO@mbox{Fun}%
}
%    \end{macrocode}
%    \end{macro}
%
%    \begin{macro}{\HoLogo@MetaPost}
%    \begin{macrocode}
\def\HoLogo@MetaPost#1{%
  \HOLOGO@mbox{Meta}%
  \HOLOGO@discretionary
  \HOLOGO@mbox{Post}%
}
%    \end{macrocode}
%    \end{macro}
%
% \subsection{Others}
%
% \subsubsection{\hologo{biber}}
%
%    \begin{macro}{\HoLogo@biber}
%    \begin{macrocode}
\def\HoLogo@biber#1{%
  \HOLOGO@mbox{#1{b}{B}i}%
  \HOLOGO@discretionary
  \HOLOGO@mbox{ber}%
}
%    \end{macrocode}
%    \end{macro}
%    \begin{macro}{\HoLogoCs@biber}
%    \begin{macrocode}
\def\HoLogoCs@biber#1{#1{b}{B}iber}
%    \end{macrocode}
%    \end{macro}
%    \begin{macro}{\HoLogoBkm@biber}
%    \begin{macrocode}
\def\HoLogoBkm@biber#1{%
  #1{b}{B}iber%
}
%    \end{macrocode}
%    \end{macro}
%    \begin{macro}{\HoLogoHtml@biber}
%    \begin{macrocode}
\let\HoLogoHtml@biber\HoLogo@biber
%    \end{macrocode}
%    \end{macro}
%
% \subsubsection{\hologo{KOMAScript}}
%
%    \begin{macro}{\HoLogo@KOMAScript}
%    The definition for \hologo{KOMAScript} is taken
%    from \hologo{KOMAScript} (\xfile{scrlogo.dtx}, reformatted) \cite{scrlogo}:
%\begin{quote}
%\begin{verbatim}
%\@ifundefined{KOMAScript}{%
%  \DeclareRobustCommand{\KOMAScript}{%
%    \textsf{%
%      K\kern.05em O\kern.05emM\kern.05em A%
%      \kern.1em-\kern.1em %
%      Script%
%    }%
%  }%
%}{}
%\end{verbatim}
%\end{quote}
%    \begin{macrocode}
\def\HoLogo@KOMAScript#1{%
  \HoLogoFont@font{KOMAScript}{sf}{%
    \HOLOGO@mbox{%
      K\kern.05em%
      O\kern.05em%
      M\kern.05em%
      A%
    }%
    \kern.1em%
    \HOLOGO@hyphen
    \kern.1em%
    \HOLOGO@mbox{Script}%
  }%
}
%    \end{macrocode}
%    \end{macro}
%    \begin{macro}{\HoLogoBkm@KOMAScript}
%    \begin{macrocode}
\def\HoLogoBkm@KOMAScript#1{%
  KOMA-Script%
}
%    \end{macrocode}
%    \end{macro}
%    \begin{macro}{\HoLogoHtml@KOMAScript}
%    \begin{macrocode}
\def\HoLogoHtml@KOMAScript#1{%
  \HoLogoCss@KOMAScript
  \HoLogoFont@font{KOMAScript}{sf}{%
    \HOLOGO@Span{KOMAScript}{%
      K%
      \HOLOGO@Span{O}{O}%
      M%
      \HOLOGO@Span{A}{A}%
      \HOLOGO@Span{hyphen}{-}%
      Script%
    }%
  }%
}
%    \end{macrocode}
%    \end{macro}
%    \begin{macro}{\HoLogoCss@KOMAScript}
%    \begin{macrocode}
\def\HoLogoCss@KOMAScript{%
  \Css{%
    span.HoLogo-KOMAScript{%
      font-family:sans-serif;%
    }%
  }%
  \Css{%
    span.HoLogo-KOMAScript span.HoLogo-O{%
      padding-left:.05em;%
      padding-right:.05em;%
    }%
  }%
  \Css{%
    span.HoLogo-KOMAScript span.HoLogo-A{%
      padding-left:.05em;%
    }%
  }%
  \Css{%
    span.HoLogo-KOMAScript span.HoLogo-hyphen{%
      padding-left:.1em;%
      padding-right:.1em;%
    }%
  }%
  \global\let\HoLogoCss@KOMAScript\relax
}
%    \end{macrocode}
%    \end{macro}
%
% \subsubsection{\hologo{LyX}}
%
%    \begin{macro}{\HoLogo@LyX}
%    The definition is taken from the documentation source files
%    of \hologo{LyX}, \xfile{Intro.lyx} \cite{LyX}:
%\begin{quote}
%\begin{verbatim}
%\def\LyX{%
%  \texorpdfstring{%
%    L\kern-.1667em\lower.25em\hbox{Y}\kern-.125emX\@%
%  }{%
%    LyX%
%  }%
%}
%\end{verbatim}
%\end{quote}
%    \begin{macrocode}
\def\HoLogo@LyX#1{%
  L%
  \kern-.1667em%
  \lower.25em\hbox{Y}%
  \kern-.125em%
  X%
  \HOLOGO@SpaceFactor
}
%    \end{macrocode}
%    \end{macro}
%    \begin{macro}{\HoLogoHtml@LyX}
%    \begin{macrocode}
\def\HoLogoHtml@LyX#1{%
  \HoLogoCss@LyX
  \HOLOGO@Span{LyX}{%
    L%
    \HOLOGO@Span{y}{Y}%
    X%
  }%
}
%    \end{macrocode}
%    \end{macro}
%    \begin{macro}{\HoLogoCss@LyX}
%    \begin{macrocode}
\def\HoLogoCss@LyX{%
  \Css{%
    span.HoLogo-LyX span.HoLogo-y{%
      position:relative;%
      top:.25em;%
      margin-left:-.1667em;%
      margin-right:-.125em;%
      text-decoration:none;%
    }%
  }%
  \global\let\HoLogoCss@LyX\relax
}
%    \end{macrocode}
%    \end{macro}
%
% \subsubsection{\hologo{NTS}}
%
%    \begin{macro}{\HoLogo@NTS}
%    Definition for \hologo{NTS} can be found in
%    package \xpackage{etex\textunderscore man} for the \hologo{eTeX} manual \cite{etexman}
%    and in package \xpackage{dtklogos} \cite{dtklogos}:
%\begin{quote}
%\begin{verbatim}
%\def\NTS{%
%  \leavevmode
%  \hbox{%
%    $%
%      \cal N%
%      \kern-0.35em%
%      \lower0.5ex\hbox{$\cal T$}%
%      \kern-0.2em%
%      S%
%    $%
%  }%
%}
%\end{verbatim}
%\end{quote}
%    \begin{macrocode}
\def\HoLogo@NTS#1{%
  \HoLogoFont@font{NTS}{sy}{%
    N\/%
    \kern-.35em%
    \lower.5ex\hbox{T\/}%
    \kern-.2em%
    S\/%
  }%
  \HOLOGO@SpaceFactor
}
%    \end{macrocode}
%    \end{macro}
%
% \subsubsection{\Hologo{TTH} (\hologo{TeX} to HTML translator)}
%
%    Source: \url{http://hutchinson.belmont.ma.us/tth/}
%    In the HTML source the second `T' is printed as subscript.
%\begin{quote}
%\begin{verbatim}
%T<sub>T</sub>H
%\end{verbatim}
%\end{quote}
%    \begin{macro}{\HoLogo@TTH}
%    \begin{macrocode}
\def\HoLogo@TTH#1{%
  \ltx@mbox{%
    T\HOLOGO@SubScript{T}H%
  }%
  \HOLOGO@SpaceFactor
}
%    \end{macrocode}
%    \end{macro}
%
%    \begin{macro}{\HoLogoHtml@TTH}
%    \begin{macrocode}
\def\HoLogoHtml@TTH#1{%
  T\HCode{<sub>}T\HCode{</sub>}H%
}
%    \end{macrocode}
%    \end{macro}
%
% \subsubsection{\Hologo{HanTheThanh}}
%
%    Partial source: Package \xpackage{dtklogos}.
%    The double accent is U+1EBF (latin small letter e with circumflex
%    and acute).
%    \begin{macro}{\HoLogo@HanTheThanh}
%    \begin{macrocode}
\def\HoLogo@HanTheThanh#1{%
  \ltx@mbox{H\`an}%
  \HOLOGO@space
  \ltx@mbox{%
    Th%
    \HOLOGO@IfCharExists{"1EBF}{%
      \char"1EBF\relax
    }{%
      \^e\hbox to 0pt{\hss\raise .5ex\hbox{\'{}}}%
    }%
  }%
  \HOLOGO@space
  \ltx@mbox{Th\`anh}%
}
%    \end{macrocode}
%    \end{macro}
%    \begin{macro}{\HoLogoBkm@HanTheThanh}
%    \begin{macrocode}
\def\HoLogoBkm@HanTheThanh#1{%
  H\`an %
  Th\HOLOGO@PdfdocUnicode{\^e}{\9036\277} %
  Th\`anh%
}
%    \end{macrocode}
%    \end{macro}
%    \begin{macro}{\HoLogoHtml@HanTheThanh}
%    \begin{macrocode}
\def\HoLogoHtml@HanTheThanh#1{%
  H\`an %
  Th\HCode{&\ltx@hashchar x1ebf;} %
  Th\`anh%
}
%    \end{macrocode}
%    \end{macro}
%
% \subsection{Driver detection}
%
%    \begin{macrocode}
\HOLOGO@IfExists\InputIfFileExists{%
  \InputIfFileExists{hologo.cfg}{}{}%
}{%
  \ltx@IfUndefined{pdf@filesize}{%
    \def\HOLOGO@InputIfExists{%
      \openin\HOLOGO@temp=hologo.cfg\relax
      \ifeof\HOLOGO@temp
        \closein\HOLOGO@temp
      \else
        \closein\HOLOGO@temp
        \begingroup
          \def\x{LaTeX2e}%
        \expandafter\endgroup
        \ifx\fmtname\x
          \input{hologo.cfg}%
        \else
          \input hologo.cfg\relax
        \fi
      \fi
    }%
    \ltx@IfUndefined{newread}{%
      \chardef\HOLOGO@temp=15 %
      \def\HOLOGO@CheckRead{%
        \ifeof\HOLOGO@temp
          \HOLOGO@InputIfExists
        \else
          \ifcase\HOLOGO@temp
            \@PackageWarningNoLine{hologo}{%
              Configuration file ignored, because\MessageBreak
              a free read register could not be found%
            }%
          \else
            \begingroup
              \count\ltx@cclv=\HOLOGO@temp
              \advance\ltx@cclv by \ltx@minusone
              \edef\x{\endgroup
                \chardef\noexpand\HOLOGO@temp=\the\count\ltx@cclv
                \relax
              }%
            \x
          \fi
        \fi
      }%
    }{%
      \csname newread\endcsname\HOLOGO@temp
      \HOLOGO@InputIfExists
    }%
  }{%
    \edef\HOLOGO@temp{\pdf@filesize{hologo.cfg}}%
    \ifx\HOLOGO@temp\ltx@empty
    \else
      \ifnum\HOLOGO@temp>0 %
        \begingroup
          \def\x{LaTeX2e}%
        \expandafter\endgroup
        \ifx\fmtname\x
          \input{hologo.cfg}%
        \else
          \input hologo.cfg\relax
        \fi
      \else
        \@PackageInfoNoLine{hologo}{%
          Empty configuration file `hologo.cfg' ignored%
        }%
      \fi
    \fi
  }%
}
%    \end{macrocode}
%
%    \begin{macrocode}
\def\HOLOGO@temp#1#2{%
  \kv@define@key{HoLogoDriver}{#1}[]{%
    \begingroup
      \def\HOLOGO@temp{##1}%
      \ltx@onelevel@sanitize\HOLOGO@temp
      \ifx\HOLOGO@temp\ltx@empty
      \else
        \@PackageError{hologo}{%
          Value (\HOLOGO@temp) not permitted for option `#1'%
        }%
        \@ehc
      \fi
    \endgroup
    \def\hologoDriver{#2}%
  }%
}%
\def\HOLOGO@@temp#1#2{%
  \ifx\kv@value\relax
    \HOLOGO@temp{#1}{#1}%
  \else
    \HOLOGO@temp{#1}{#2}%
  \fi
}%
\kv@parse@normalized{%
  pdftex,%
  luatex=pdftex,%
  dvipdfm,%
  dvipdfmx=dvipdfm,%
  dvips,%
  dvipsone=dvips,%
  xdvi=dvips,%
  xetex,%
  vtex,%
}\HOLOGO@@temp
%    \end{macrocode}
%
%    \begin{macrocode}
\kv@define@key{HoLogoDriver}{driverfallback}{%
  \def\HOLOGO@DriverFallback{#1}%
}
%    \end{macrocode}
%
%    \begin{macro}{\HOLOGO@DriverFallback}
%    \begin{macrocode}
\def\HOLOGO@DriverFallback{dvips}
%    \end{macrocode}
%    \end{macro}
%
%    \begin{macro}{\hologoDriverSetup}
%    \begin{macrocode}
\def\hologoDriverSetup{%
  \let\hologoDriver\ltx@undefined
  \HOLOGO@DriverSetup
}
%    \end{macrocode}
%    \end{macro}
%
%    \begin{macro}{\HOLOGO@DriverSetup}
%    \begin{macrocode}
\def\HOLOGO@DriverSetup#1{%
  \kvsetkeys{HoLogoDriver}{#1}%
  \HOLOGO@CheckDriver
  \ltx@ifundefined{hologoDriver}{%
    \begingroup
    \edef\x{\endgroup
      \noexpand\kvsetkeys{HoLogoDriver}{\HOLOGO@DriverFallback}%
    }\x
  }{}%
  \@PackageInfoNoLine{hologo}{Using driver `\hologoDriver'}%
}
%    \end{macrocode}
%    \end{macro}
%
%    \begin{macro}{\HOLOGO@CheckDriver}
%    \begin{macrocode}
\def\HOLOGO@CheckDriver{%
  \ifpdf
    \def\hologoDriver{pdftex}%
    \let\HOLOGO@pdfliteral\pdfliteral
    \ifluatex
      \ifx\pdfextension\@undefined\else
        \protected\def\pdfliteral{\pdfextension literal}%
        \let\HOLOGO@pdfliteral\pdfliteral
      \fi
      \ltx@IfUndefined{HOLOGO@pdfliteral}{%
        \ifnum\luatexversion<36 %
        \else
          \begingroup
            \let\HOLOGO@temp\endgroup
            \ifcase0%
                \directlua{%
                  if tex.enableprimitives then %
                    tex.enableprimitives('HOLOGO@', {'pdfliteral'})%
                  else %
                    tex.print('1')%
                  end%
                }%
                \ifx\HOLOGO@pdfliteral\@undefined 1\fi%
                \relax%
              \endgroup
              \let\HOLOGO@temp\relax
              \global\let\HOLOGO@pdfliteral\HOLOGO@pdfliteral
            \fi%
          \HOLOGO@temp
        \fi
      }{}%
    \fi
    \ltx@IfUndefined{HOLOGO@pdfliteral}{%
      \@PackageWarningNoLine{hologo}{%
        Cannot find \string\pdfliteral
      }%
    }{}%
  \else
    \ifxetex
      \def\hologoDriver{xetex}%
    \else
      \ifvtex
        \def\hologoDriver{vtex}%
      \fi
    \fi
  \fi
}
%    \end{macrocode}
%    \end{macro}
%
%    \begin{macro}{\HOLOGO@WarningUnsupportedDriver}
%    \begin{macrocode}
\def\HOLOGO@WarningUnsupportedDriver#1{%
  \@PackageWarningNoLine{hologo}{%
    Logo `#1' needs driver specific macros,\MessageBreak
    but driver `\hologoDriver' is not supported.\MessageBreak
    Use a different driver or\MessageBreak
    load package `graphics' or `pgf'%
  }%
}
%    \end{macrocode}
%    \end{macro}
%
% \subsubsection{Reflect box macros}
%
%    Skip driver part if not needed.
%    \begin{macrocode}
\ltx@IfUndefined{reflectbox}{}{%
  \ltx@IfUndefined{rotatebox}{}{%
    \HOLOGO@AtEnd
  }%
}
\ltx@IfUndefined{pgftext}{}{%
  \HOLOGO@AtEnd
}
\ltx@IfUndefined{psscalebox}{}{%
  \HOLOGO@AtEnd
}
%    \end{macrocode}
%
%    \begin{macrocode}
\def\HOLOGO@temp{LaTeX2e}
\ifx\fmtname\HOLOGO@temp
  \RequirePackage{kvoptions}[2011/06/30]%
  \ProcessKeyvalOptions{HoLogoDriver}%
\fi
\HOLOGO@DriverSetup{}
%    \end{macrocode}
%
%    \begin{macro}{\HOLOGO@ReflectBox}
%    \begin{macrocode}
\def\HOLOGO@ReflectBox#1{%
  \begingroup
    \setbox\ltx@zero\hbox{\begingroup#1\endgroup}%
    \setbox\ltx@two\hbox{%
      \kern\wd\ltx@zero
      \csname HOLOGO@ScaleBox@\hologoDriver\endcsname{-1}{1}{%
        \hbox to 0pt{\copy\ltx@zero\hss}%
      }%
    }%
    \wd\ltx@two=\wd\ltx@zero
    \box\ltx@two
  \endgroup
}
%    \end{macrocode}
%    \end{macro}
%
%    \begin{macro}{\HOLOGO@PointReflectBox}
%    \begin{macrocode}
\def\HOLOGO@PointReflectBox#1{%
  \begingroup
    \setbox\ltx@zero\hbox{\begingroup#1\endgroup}%
    \setbox\ltx@two\hbox{%
      \kern\wd\ltx@zero
      \raise\ht\ltx@zero\hbox{%
        \csname HOLOGO@ScaleBox@\hologoDriver\endcsname{-1}{-1}{%
          \hbox to 0pt{\copy\ltx@zero\hss}%
        }%
      }%
    }%
    \wd\ltx@two=\wd\ltx@zero
    \box\ltx@two
  \endgroup
}
%    \end{macrocode}
%    \end{macro}
%
%    We must define all variants because of dynamic driver setup.
%    \begin{macrocode}
\def\HOLOGO@temp#1#2{#2}
%    \end{macrocode}
%
%    \begin{macro}{\HOLOGO@ScaleBox@pdftex}
%    \begin{macrocode}
\HOLOGO@temp{pdftex}{%
  \def\HOLOGO@ScaleBox@pdftex#1#2#3{%
    \HOLOGO@pdfliteral{%
      q #1 0 0 #2 0 0 cm%
    }%
    #3%
    \HOLOGO@pdfliteral{%
      Q%
    }%
  }%
}
%    \end{macrocode}
%    \end{macro}
%    \begin{macro}{\HOLOGO@ScaleBox@dvips}
%    \begin{macrocode}
\HOLOGO@temp{dvips}{%
  \def\HOLOGO@ScaleBox@dvips#1#2#3{%
    \special{ps:%
      gsave %
      currentpoint %
      currentpoint translate %
      #1 #2 scale %
      neg exch neg exch translate%
    }%
    #3%
    \special{ps:%
      currentpoint %
      grestore %
      moveto%
    }%
  }%
}
%    \end{macrocode}
%    \end{macro}
%    \begin{macro}{\HOLOGO@ScaleBox@dvipdfm}
%    \begin{macrocode}
\HOLOGO@temp{dvipdfm}{%
  \let\HOLOGO@ScaleBox@dvipdfm\HOLOGO@ScaleBox@dvips
}
%    \end{macrocode}
%    \end{macro}
%    Since \hologo{XeTeX} v0.6.
%    \begin{macro}{\HOLOGO@ScaleBox@xetex}
%    \begin{macrocode}
\HOLOGO@temp{xetex}{%
  \def\HOLOGO@ScaleBox@xetex#1#2#3{%
    \special{x:gsave}%
    \special{x:scale #1 #2}%
    #3%
    \special{x:grestore}%
  }%
}
%    \end{macrocode}
%    \end{macro}
%    \begin{macro}{\HOLOGO@ScaleBox@vtex}
%    \begin{macrocode}
\HOLOGO@temp{vtex}{%
  \def\HOLOGO@ScaleBox@vtex#1#2#3{%
    \special{r(#1,0,0,#2,0,0}%
    #3%
    \special{r)}%
  }%
}
%    \end{macrocode}
%    \end{macro}
%
%    \begin{macrocode}
\HOLOGO@AtEnd%
%</package>
%    \end{macrocode}
%
% \section{Test}
%
% \subsection{Catcode checks for loading}
%
%    \begin{macrocode}
%<*test1>
%    \end{macrocode}
%    \begin{macrocode}
\catcode`\{=1 %
\catcode`\}=2 %
\catcode`\#=6 %
\catcode`\@=11 %
\expandafter\ifx\csname count@\endcsname\relax
  \countdef\count@=255 %
\fi
\expandafter\ifx\csname @gobble\endcsname\relax
  \long\def\@gobble#1{}%
\fi
\expandafter\ifx\csname @firstofone\endcsname\relax
  \long\def\@firstofone#1{#1}%
\fi
\expandafter\ifx\csname loop\endcsname\relax
  \expandafter\@firstofone
\else
  \expandafter\@gobble
\fi
{%
  \def\loop#1\repeat{%
    \def\body{#1}%
    \iterate
  }%
  \def\iterate{%
    \body
      \let\next\iterate
    \else
      \let\next\relax
    \fi
    \next
  }%
  \let\repeat=\fi
}%
\def\RestoreCatcodes{}
\count@=0 %
\loop
  \edef\RestoreCatcodes{%
    \RestoreCatcodes
    \catcode\the\count@=\the\catcode\count@\relax
  }%
\ifnum\count@<255 %
  \advance\count@ 1 %
\repeat

\def\RangeCatcodeInvalid#1#2{%
  \count@=#1\relax
  \loop
    \catcode\count@=15 %
  \ifnum\count@<#2\relax
    \advance\count@ 1 %
  \repeat
}
\def\RangeCatcodeCheck#1#2#3{%
  \count@=#1\relax
  \loop
    \ifnum#3=\catcode\count@
    \else
      \errmessage{%
        Character \the\count@\space
        with wrong catcode \the\catcode\count@\space
        instead of \number#3%
      }%
    \fi
  \ifnum\count@<#2\relax
    \advance\count@ 1 %
  \repeat
}
\def\space{ }
\expandafter\ifx\csname LoadCommand\endcsname\relax
  \def\LoadCommand{\input hologo.sty\relax}%
\fi
\def\Test{%
  \RangeCatcodeInvalid{0}{47}%
  \RangeCatcodeInvalid{58}{64}%
  \RangeCatcodeInvalid{91}{96}%
  \RangeCatcodeInvalid{123}{255}%
  \catcode`\@=12 %
  \catcode`\\=0 %
  \catcode`\%=14 %
  \LoadCommand
  \RangeCatcodeCheck{0}{36}{15}%
  \RangeCatcodeCheck{37}{37}{14}%
  \RangeCatcodeCheck{38}{47}{15}%
  \RangeCatcodeCheck{48}{57}{12}%
  \RangeCatcodeCheck{58}{63}{15}%
  \RangeCatcodeCheck{64}{64}{12}%
  \RangeCatcodeCheck{65}{90}{11}%
  \RangeCatcodeCheck{91}{91}{15}%
  \RangeCatcodeCheck{92}{92}{0}%
  \RangeCatcodeCheck{93}{96}{15}%
  \RangeCatcodeCheck{97}{122}{11}%
  \RangeCatcodeCheck{123}{255}{15}%
  \RestoreCatcodes
}
\Test
\csname @@end\endcsname
\end
%    \end{macrocode}
%    \begin{macrocode}
%</test1>
%    \end{macrocode}
%
% \subsection{Spacefactor}
%
%    The space factor must be 1000 after a logo. If it is greater 1000
%    then the following space is a space after a sentence closing point.
%    If the space factor is smaller 1000 then an immediate following
%    dot is interpreted as abbreviation, not sentence closing point.
%
%    \begin{macrocode}
%<*test-spacefactor>
\NeedsTeXFormat{LaTeX2e}
\documentclass{article}
\usepackage{hologo}[2016/05/12]
\usepackage{kvsetkeys}
\usepackage{qstest}
\IncludeTests{*}
\LogTests{log}{*}{*}
\begin{document}
\begin{qstest}{spacefactor}{spacefactor}
\newcommand*{\Test}[1]{%
  \sbox0{%
    \hologo{#1}%
    \Expect*{1000 (#1)}*{\the\spacefactor\space(#1)}%
  }%
}%
\makeatletter
\def\TestList{}
\def\hologoEntry#1#2#3{%
  \edef\TestList{%
    \ifx\TestList\@empty
    \else
      \TestList,%
    \fi
    #1%
    \ifx\\#2\\%
    \else
      ={variant=#2}%
    \fi
  }%
}
\hologoList
\expandafter\kv@parse@normalized\expandafter{%
  \TestList
}{%
  \begingroup
    \let\@logo=\kv@key
    \ifx\kv@value\relax
    \else
      \expandafter\hologoLogoSetup\expandafter\@logo\expandafter{%
        \kv@value
      }%
    \fi
    \Test\@logo
  \endgroup
  \@gobbletwo
}
\end{qstest}
\end{document}
%</test-spacefactor>
%    \end{macrocode}
%
% \subsection{Complete list}
%
%    \begin{macrocode}
%<*test-list>
\NeedsTeXFormat{LaTeX2e}
\documentclass[12pt,a4paper]{article}
\usepackage{hologo}[2016/05/12]
\usepackage[T1]{fontenc}
\usepackage{lmodern}
\usepackage{parskip}
\usepackage[unicode]{hyperref}[2011/09/28]
\usepackage{bookmark}[2011/09/19]
\bookmarksetup{%
  numbered,%
  open,%
  openlevel=2,%
}
\renewcommand*{\contentsname}{List of logos}
\begin{document}
\tableofcontents
\def\TestFont#1#2#3#4#5#6{%
  \begingroup
    \usefont{#3}{#4}{#5}{#6}%
    \HologoVariant{#1}{#2}/\hologoVariant{#1}{#2}%
    \quad
    \begingroup\scriptsize\hologoVariant{#1}{#2}\endgroup
    \quad
  \endgroup
  (#3/#4/#5/#6)%
  \par
}
\makeatletter
\def\hologoEntry#1#2#3{%
  \section{%
    \HologoVariant{#1}{#2}/\hologoVariant{#1}{#2} %
    {[#1\ifx\\#2\\\else\space(#2)\fi]}% hash-ok
  }% braces around [] because of bug in tex4ht
  \begingroup
    \hypersetup{unicode=false}%
    \bookmark[%
      dest=\@currentHref,%
      rellevel=1,%
      keeplevel,%
    ]{%
      \HologoVariant{#1}{#2}/\hologoVariant{#1}{#2} %
      (PDFDocEncoding)%
    }%
  \endgroup
  \TestFont{#1}{#2}{OT1}{cmr}{m}{n}%
  \TestFont{#1}{#2}{OT1}{cmss}{m}{n}%
  \TestFont{#1}{#2}{OT1}{cmr}{b}{n}%
  \TestFont{#1}{#2}{OT1}{cmr}{m}{it}%
  \TestFont{#1}{#2}{OT1}{cmtt}{m}{n}%
  \TestFont{#1}{#2}{T1}{lmr}{m}{n}%
  \TestFont{#1}{#2}{T1}{lmss}{m}{n}%
  \TestFont{#1}{#2}{T1}{lmr}{b}{n}%
  \TestFont{#1}{#2}{T1}{lmr}{m}{it}%
  \TestFont{#1}{#2}{T1}{lmtt}{m}{n}%
  \TestFont{#1}{#2}{T1}{lmvtt}{m}{n}%
  \TestFont{#1}{#2}{T1}{qtm}{m}{n}%
  \TestFont{#1}{#2}{T1}{qhv}{m}{n}%
  \TestFont{#1}{#2}{T1}{qtm}{b}{n}%
  \TestFont{#1}{#2}{T1}{qtm}{m}{it}%
  \TestFont{#1}{#2}{T1}{qcr}{m}{n}%
  \newpage
}
\makeatother
\hologoList
\end{document}
%</test-list>
%    \end{macrocode}
%
% \section{Installation}
%
% \subsection{Download}
%
% \paragraph{Package.} This package is available on
% CTAN\footnote{\url{ftp://ftp.ctan.org/tex-archive/}}:
% \begin{description}
% \item[\CTAN{macros/latex/contrib/oberdiek/hologo.dtx}] The source file.
% \item[\CTAN{macros/latex/contrib/oberdiek/hologo.pdf}] Documentation.
% \end{description}
%
%
% \paragraph{Bundle.} All the packages of the bundle `oberdiek'
% are also available in a TDS compliant ZIP archive. There
% the packages are already unpacked and the documentation files
% are generated. The files and directories obey the TDS standard.
% \begin{description}
% \item[\CTAN{install/macros/latex/contrib/oberdiek.tds.zip}]
% \end{description}
% \emph{TDS} refers to the standard ``A Directory Structure
% for \TeX\ Files'' (\CTAN{tds/tds.pdf}). Directories
% with \xfile{texmf} in their name are usually organized this way.
%
% \subsection{Bundle installation}
%
% \paragraph{Unpacking.} Unpack the \xfile{oberdiek.tds.zip} in the
% TDS tree (also known as \xfile{texmf} tree) of your choice.
% Example (linux):
% \begin{quote}
%   |unzip oberdiek.tds.zip -d ~/texmf|
% \end{quote}
%
% \paragraph{Script installation.}
% Check the directory \xfile{TDS:scripts/oberdiek/} for
% scripts that need further installation steps.
% Package \xpackage{attachfile2} comes with the Perl script
% \xfile{pdfatfi.pl} that should be installed in such a way
% that it can be called as \texttt{pdfatfi}.
% Example (linux):
% \begin{quote}
%   |chmod +x scripts/oberdiek/pdfatfi.pl|\\
%   |cp scripts/oberdiek/pdfatfi.pl /usr/local/bin/|
% \end{quote}
%
% \subsection{Package installation}
%
% \paragraph{Unpacking.} The \xfile{.dtx} file is a self-extracting
% \docstrip\ archive. The files are extracted by running the
% \xfile{.dtx} through \plainTeX:
% \begin{quote}
%   \verb|tex hologo.dtx|
% \end{quote}
%
% \paragraph{TDS.} Now the different files must be moved into
% the different directories in your installation TDS tree
% (also known as \xfile{texmf} tree):
% \begin{quote}
% \def\t{^^A
% \begin{tabular}{@{}>{\ttfamily}l@{ $\rightarrow$ }>{\ttfamily}l@{}}
%   hologo.sty & tex/generic/oberdiek/hologo.sty\\
%   hologo.pdf & doc/latex/oberdiek/hologo.pdf\\
%   example/hologo-example.tex & doc/latex/oberdiek/example/hologo-example.tex\\
%   test/hologo-test1.tex & doc/latex/oberdiek/test/hologo-test1.tex\\
%   test/hologo-test-spacefactor.tex & doc/latex/oberdiek/test/hologo-test-spacefactor.tex\\
%   test/hologo-test-list.tex & doc/latex/oberdiek/test/hologo-test-list.tex\\
%   hologo.dtx & source/latex/oberdiek/hologo.dtx\\
% \end{tabular}^^A
% }^^A
% \sbox0{\t}^^A
% \ifdim\wd0>\linewidth
%   \begingroup
%     \advance\linewidth by\leftmargin
%     \advance\linewidth by\rightmargin
%   \edef\x{\endgroup
%     \def\noexpand\lw{\the\linewidth}^^A
%   }\x
%   \def\lwbox{^^A
%     \leavevmode
%     \hbox to \linewidth{^^A
%       \kern-\leftmargin\relax
%       \hss
%       \usebox0
%       \hss
%       \kern-\rightmargin\relax
%     }^^A
%   }^^A
%   \ifdim\wd0>\lw
%     \sbox0{\small\t}^^A
%     \ifdim\wd0>\linewidth
%       \ifdim\wd0>\lw
%         \sbox0{\footnotesize\t}^^A
%         \ifdim\wd0>\linewidth
%           \ifdim\wd0>\lw
%             \sbox0{\scriptsize\t}^^A
%             \ifdim\wd0>\linewidth
%               \ifdim\wd0>\lw
%                 \sbox0{\tiny\t}^^A
%                 \ifdim\wd0>\linewidth
%                   \lwbox
%                 \else
%                   \usebox0
%                 \fi
%               \else
%                 \lwbox
%               \fi
%             \else
%               \usebox0
%             \fi
%           \else
%             \lwbox
%           \fi
%         \else
%           \usebox0
%         \fi
%       \else
%         \lwbox
%       \fi
%     \else
%       \usebox0
%     \fi
%   \else
%     \lwbox
%   \fi
% \else
%   \usebox0
% \fi
% \end{quote}
% If you have a \xfile{docstrip.cfg} that configures and enables \docstrip's
% TDS installing feature, then some files can already be in the right
% place, see the documentation of \docstrip.
%
% \subsection{Refresh file name databases}
%
% If your \TeX~distribution
% (\teTeX, \mikTeX, \dots) relies on file name databases, you must refresh
% these. For example, \teTeX\ users run \verb|texhash| or
% \verb|mktexlsr|.
%
% \subsection{Some details for the interested}
%
% \paragraph{Attached source.}
%
% The PDF documentation on CTAN also includes the
% \xfile{.dtx} source file. It can be extracted by
% AcrobatReader 6 or higher. Another option is \textsf{pdftk},
% e.g. unpack the file into the current directory:
% \begin{quote}
%   \verb|pdftk hologo.pdf unpack_files output .|
% \end{quote}
%
% \paragraph{Unpacking with \LaTeX.}
% The \xfile{.dtx} chooses its action depending on the format:
% \begin{description}
% \item[\plainTeX:] Run \docstrip\ and extract the files.
% \item[\LaTeX:] Generate the documentation.
% \end{description}
% If you insist on using \LaTeX\ for \docstrip\ (really,
% \docstrip\ does not need \LaTeX), then inform the autodetect routine
% about your intention:
% \begin{quote}
%   \verb|latex \let\install=y\input{hologo.dtx}|
% \end{quote}
% Do not forget to quote the argument according to the demands
% of your shell.
%
% \paragraph{Generating the documentation.}
% You can use both the \xfile{.dtx} or the \xfile{.drv} to generate
% the documentation. The process can be configured by the
% configuration file \xfile{ltxdoc.cfg}. For instance, put this
% line into this file, if you want to have A4 as paper format:
% \begin{quote}
%   \verb|\PassOptionsToClass{a4paper}{article}|
% \end{quote}
% An example follows how to generate the
% documentation with pdf\LaTeX:
% \begin{quote}
%\begin{verbatim}
%pdflatex hologo.dtx
%makeindex -s gind.ist hologo.idx
%pdflatex hologo.dtx
%makeindex -s gind.ist hologo.idx
%pdflatex hologo.dtx
%\end{verbatim}
% \end{quote}
%
% \section{Catalogue}
%
% The following XML file can be used as source for the
% \href{http://mirror.ctan.org/help/Catalogue/catalogue.html}{\TeX\ Catalogue}.
% The elements \texttt{caption} and \texttt{description} are imported
% from the original XML file from the Catalogue.
% The name of the XML file in the Catalogue is \xfile{hologo.xml}.
%    \begin{macrocode}
%<*catalogue>
<?xml version='1.0' encoding='us-ascii'?>
<!DOCTYPE entry SYSTEM 'catalogue.dtd'>
<entry datestamp='$Date$' modifier='$Author$' id='hologo'>
  <name>hologo</name>
  <caption>A collection of logos with bookmark support.</caption>
  <authorref id='auth:oberdiek'/>
  <copyright owner='Heiko Oberdiek' year='2010-2012'/>
  <license type='lppl1.3'/>
  <version number='1.10'/>
  <description>
    The package defines a single command <tt>\hologo</tt>, whose
    argument is the usual case-confused ASCII version of the logo.
    The command is bookmark-enabled, so that every logo becomes
    available in bookmarks without further work.
    <p/>
    The package is part of the <xref refid='oberdiek'>oberdiek</xref>
    bundle.
  </description>
  <documentation details='Package documentation'
      href='ctan:/macros/latex/contrib/oberdiek/hologo.pdf'/>
  <ctan file='true' path='/macros/latex/contrib/oberdiek/hologo.dtx'/>
  <miktex location='oberdiek'/>
  <texlive location='oberdiek'/>
  <install path='/macros/latex/contrib/oberdiek/oberdiek.tds.zip'/>
</entry>
%</catalogue>
%    \end{macrocode}
%
% \begin{thebibliography}{9}
% \raggedright
%
% \bibitem{btxdoc}
% Oren Patashnik,
% \textit{\hologo{BibTeX}ing},
% 1988-02-08.\\
% \CTAN{biblio/bibtex/base/}
%
% \bibitem{dtklogos}
% Gerd Neugebauer, DANTE,
% \textit{Package \xpackage{dtklogos}},
% 2011-04-25.\\
% \CTAN{usergrps/dante/dtk/dtklogos.sty}
%
% \bibitem{etexman}
% The \hologo{NTS} Team,
% \textit{The \hologo{eTeX} manual},
% 1998-02.\\
% \CTAN{systems/e-tex/v2/doc/}
%
% \bibitem{ExTeX-FAQ}
% The \hologo{ExTeX} group,
% \textit{\hologo{ExTeX}: FAQ -- How is \hologo{ExTeX} typeset?},
% 2007-04-14.\\
% \url{http://www.extex.org/documentation/faq.html}
%
% \bibitem{LyX}
% %@MISC{ LyX,
% %  title = {{LyX 2.0.0 -- The Document Processor [Computer software and manual]}},
% %  author = {{The LyX Team}},
% %  howpublished = {Internet: http://www.lyx.org},
% %  year = {2011-05-08},
% %  note = {Retrieved May 10, 2011, from http://www.lyx.org},
% %  url = {http://www.lyx.org/}
% %}
% The \hologo{LyX} Team,
% \textit{\hologo{LyX} -- The Document Processor},
% 2011-05-08.\\
% \url{http://www.lyx.org/}
%
% \bibitem{OzTeX}
% Andrew Trevorrow,
% \hologo{OzTeX} FAQ: What is the correct way to typeset ``\hologo{OzTeX}''?,
% 2011-09-15 (visited).
% \url{http://www.trevorrow.com/oztex/ozfaq.html#oztex-logo}
%
% \bibitem{PiCTeX}
% Michael Wichura,
% \textit{The \hologo{PiCTeX} macro package},
% 1987-09-21.
% \CTAN{graphics/pictex/}
%
% \bibitem{scrlogo}
% Markus Kohm,
% \textit{\hologo{KOMAScript} Datei \xfile{scrlogo.dtx}},
% 2009-01-30.\\
% \CTAN{install/macros/latex/contrib/komascript.tds.zip}
%
% \end{thebibliography}
%
% \begin{History}
%   \begin{Version}{2010/04/08 v1.0}
%   \item
%     The first version.
%   \end{Version}
%   \begin{Version}{2010/04/16 v1.1}
%   \item
%     \cs{Hologo} added for support of logos at start of a sentence.
%   \item
%     \cs{hologoSetup} and \cs{hologoLogoSetup} added.
%   \item
%     Options \xoption{break}, \xoption{hyphenbreak}, \xoption{spacebreak}
%     added.
%   \item
%     Variant support added by option \xoption{variant}.
%   \end{Version}
%   \begin{Version}{2010/04/24 v1.2}
%   \item
%     \hologo{LaTeX3} added.
%   \item
%     \hologo{VTeX} added.
%   \end{Version}
%   \begin{Version}{2010/11/21 v1.3}
%   \item
%     \hologo{iniTeX}, \hologo{virTeX} added.
%   \end{Version}
%   \begin{Version}{2011/03/25 v1.4}
%   \item
%     \hologo{ConTeXt} with variants added.
%   \item
%     Option \xoption{discretionarybreak} added as refinement for
%     option \xoption{break}.
%   \end{Version}
%   \begin{Version}{2011/04/21 v1.5}
%   \item
%     Wrong TDS directory for test files fixed.
%   \end{Version}
%   \begin{Version}{2011/10/01 v1.6}
%   \item
%     Support for package \xpackage{tex4ht} added.
%   \item
%     Support for \cs{csname} added if \cs{ifincsname} is available.
%   \item
%     New logos:
%     \hologo{(La)TeX},
%     \hologo{biber},
%     \hologo{BibTeX} (\xoption{sc}, \xoption{sf}),
%     \hologo{emTeX},
%     \hologo{ExTeX},
%     \hologo{KOMAScript},
%     \hologo{La},
%     \hologo{LyX},
%     \hologo{MiKTeX},
%     \hologo{NTS},
%     \hologo{OzMF},
%     \hologo{OzMP},
%     \hologo{OzTeX},
%     \hologo{OzTtH},
%     \hologo{PCTeX},
%     \hologo{PiC},
%     \hologo{PiCTeX},
%     \hologo{METAFONT},
%     \hologo{MetaFun},
%     \hologo{METAPOST},
%     \hologo{MetaPost},
%     \hologo{SLiTeX} (\xoption{lift}, \xoption{narrow}, \xoption{simple}),
%     \hologo{SliTeX} (\xoption{narrow}, \xoption{simple}, \xoption{lift}),
%     \hologo{teTeX}.
%   \item
%     Fixes:
%     \hologo{iniTeX},
%     \hologo{pdfLaTeX},
%     \hologo{pdfTeX},
%     \hologo{virTeX}.
%   \item
%     \cs{hologoFontSetup} and \cs{hologoLogoFontSetup} added.
%   \item
%     \cs{hologoVariant} and \cs{HologoVariant} added.
%   \end{Version}
%   \begin{Version}{2011/11/22 v1.7}
%   \item
%     New logos:
%     \hologo{BibTeX8},
%     \hologo{LaTeXML},
%     \hologo{SageTeX},
%     \hologo{TeX4ht},
%     \hologo{TTH}.
%   \item
%     \hologo{Xe} and friends: Driver stuff fixed.
%   \item
%     \hologo{Xe} and friends: Support for italic added.
%   \item
%     \hologo{Xe} and friends: Package support for \xpackage{pgf}
%     and \xpackage{pstricks} added.
%   \end{Version}
%   \begin{Version}{2011/11/29 v1.8}
%   \item
%     New logos:
%     \hologo{HanTheThanh}.
%   \end{Version}
%   \begin{Version}{2011/12/21 v1.9}
%   \item
%     Patch for package \xpackage{ifxetex} added for the case that
%     \cs{newif} is undefined in \hologo{iniTeX}.
%   \item
%     Some fixes for \hologo{iniTeX}.
%   \end{Version}
%   \begin{Version}{2012/04/26 v1.10}
%   \item
%     Fix in bookmark version of logo ``\hologo{HanTheThanh}''.
%   \end{Version}
%   \begin{Version}{2016/05/12 v1.11}
%   \item
%     Update HOLOGO@IfCharExists (previously in texlive)
%   \item define pdfliteral in current luatex.
%   \end{Version}
% \end{History}
%
% \PrintIndex
%
% \Finale
\endinput

%        (quote the arguments according to the demands of your shell)
%
% Documentation:
%    (a) If hologo.drv is present:
%           latex hologo.drv
%    (b) Without hologo.drv:
%           latex hologo.dtx; ...
%    The class ltxdoc loads the configuration file ltxdoc.cfg
%    if available. Here you can specify further options, e.g.
%    use A4 as paper format:
%       \PassOptionsToClass{a4paper}{article}
%
%    Programm calls to get the documentation (example):
%       pdflatex hologo.dtx
%       makeindex -s gind.ist hologo.idx
%       pdflatex hologo.dtx
%       makeindex -s gind.ist hologo.idx
%       pdflatex hologo.dtx
%
% Installation:
%    TDS:tex/generic/oberdiek/hologo.sty
%    TDS:doc/latex/oberdiek/hologo.pdf
%    TDS:doc/latex/oberdiek/example/hologo-example.tex
%    TDS:doc/latex/oberdiek/test/hologo-test1.tex
%    TDS:doc/latex/oberdiek/test/hologo-test-spacefactor.tex
%    TDS:doc/latex/oberdiek/test/hologo-test-list.tex
%    TDS:source/latex/oberdiek/hologo.dtx
%
%<*ignore>
\begingroup
  \catcode123=1 %
  \catcode125=2 %
  \def\x{LaTeX2e}%
\expandafter\endgroup
\ifcase 0\ifx\install y1\fi\expandafter
         \ifx\csname processbatchFile\endcsname\relax\else1\fi
         \ifx\fmtname\x\else 1\fi\relax
\else\csname fi\endcsname
%</ignore>
%<*install>
\input docstrip.tex
\Msg{************************************************************************}
\Msg{* Installation}
\Msg{* Package: hologo 2016/05/12 v1.11 A logo collection with bookmark support (HO)}
\Msg{************************************************************************}

\keepsilent
\askforoverwritefalse

\let\MetaPrefix\relax
\preamble

This is a generated file.

Project: hologo
Version: 2016/05/12 v1.11

Copyright (C) 2010-2012 by
   Heiko Oberdiek <heiko.oberdiek at googlemail.com>

This work may be distributed and/or modified under the
conditions of the LaTeX Project Public License, either
version 1.3c of this license or (at your option) any later
version. This version of this license is in
   http://www.latex-project.org/lppl/lppl-1-3c.txt
and the latest version of this license is in
   http://www.latex-project.org/lppl.txt
and version 1.3 or later is part of all distributions of
LaTeX version 2005/12/01 or later.

This work has the LPPL maintenance status "maintained".

This Current Maintainer of this work is Heiko Oberdiek.

The Base Interpreter refers to any `TeX-Format',
because some files are installed in TDS:tex/generic//.

This work consists of the main source file hologo.dtx
and the derived files
   hologo.sty, hologo.pdf, hologo.ins, hologo.drv, hologo-example.tex,
   hologo-test1.tex, hologo-test-spacefactor.tex,
   hologo-test-list.tex.

\endpreamble
\let\MetaPrefix\DoubleperCent

\generate{%
  \file{hologo.ins}{\from{hologo.dtx}{install}}%
  \file{hologo.drv}{\from{hologo.dtx}{driver}}%
  \usedir{tex/generic/oberdiek}%
  \file{hologo.sty}{\from{hologo.dtx}{package}}%
  \usedir{doc/latex/oberdiek/example}%
  \file{hologo-example.tex}{\from{hologo.dtx}{example}}%
  \usedir{doc/latex/oberdiek/test}%
  \file{hologo-test1.tex}{\from{hologo.dtx}{test1}}%
  \file{hologo-test-spacefactor.tex}{\from{hologo.dtx}{test-spacefactor}}%
  \file{hologo-test-list.tex}{\from{hologo.dtx}{test-list}}%
  \nopreamble
  \nopostamble
  \usedir{source/latex/oberdiek/catalogue}%
  \file{hologo.xml}{\from{hologo.dtx}{catalogue}}%
}

\catcode32=13\relax% active space
\let =\space%
\Msg{************************************************************************}
\Msg{*}
\Msg{* To finish the installation you have to move the following}
\Msg{* file into a directory searched by TeX:}
\Msg{*}
\Msg{*     hologo.sty}
\Msg{*}
\Msg{* To produce the documentation run the file `hologo.drv'}
\Msg{* through LaTeX.}
\Msg{*}
\Msg{* Happy TeXing!}
\Msg{*}
\Msg{************************************************************************}

\endbatchfile
%</install>
%<*ignore>
\fi
%</ignore>
%<*driver>
\NeedsTeXFormat{LaTeX2e}
\ProvidesFile{hologo.drv}%
  [2016/05/12 v1.11 A logo collection with bookmark support (HO)]%
\documentclass{ltxdoc}
\usepackage{holtxdoc}[2011/11/22]
\usepackage{hologo}[2016/05/12]
\usepackage{longtable}
\usepackage{array}
\usepackage{paralist}
%\usepackage[T1]{fontenc}
%\usepackage{lmodern}
\begin{document}
  \DocInput{hologo.dtx}%
\end{document}
%</driver>
% \fi
%
%
% \CharacterTable
%  {Upper-case    \A\B\C\D\E\F\G\H\I\J\K\L\M\N\O\P\Q\R\S\T\U\V\W\X\Y\Z
%   Lower-case    \a\b\c\d\e\f\g\h\i\j\k\l\m\n\o\p\q\r\s\t\u\v\w\x\y\z
%   Digits        \0\1\2\3\4\5\6\7\8\9
%   Exclamation   \!     Double quote  \"     Hash (number) \#
%   Dollar        \$     Percent       \%     Ampersand     \&
%   Acute accent  \'     Left paren    \(     Right paren   \)
%   Asterisk      \*     Plus          \+     Comma         \,
%   Minus         \-     Point         \.     Solidus       \/
%   Colon         \:     Semicolon     \;     Less than     \<
%   Equals        \=     Greater than  \>     Question mark \?
%   Commercial at \@     Left bracket  \[     Backslash     \\
%   Right bracket \]     Circumflex    \^     Underscore    \_
%   Grave accent  \`     Left brace    \{     Vertical bar  \|
%   Right brace   \}     Tilde         \~}
%
% \GetFileInfo{hologo.drv}
%
% \title{The \xpackage{hologo} package}
% \date{2016/05/12 v1.11}
% \author{Heiko Oberdiek\\\xemail{heiko.oberdiek at googlemail.com}}
%
% \maketitle
%
% \begin{abstract}
% This package starts a collection of logos with support for bookmarks
% strings.
% \end{abstract}
%
% \tableofcontents
%
% \section{Documentation}
%
% \subsection{Logo macros}
%
% \begin{declcs}{hologo} \M{name}
% \end{declcs}
% Macro \cs{hologo} sets the logo with name \meta{name}.
% The following table shows the supported names.
%
% \begingroup
%   \def\hologoEntry#1#2#3{^^A
%     #1&#2&\hologoLogoSetup{#1}{variant=#2}\hologo{#1}&#3\tabularnewline
%   }
%   \begin{longtable}{>{\ttfamily}l>{\ttfamily}lll}
%     \rmfamily\bfseries{name} & \rmfamily\bfseries variant
%     & \bfseries logo & \bfseries since\\
%     \hline
%     \endhead
%     \hologoList
%   \end{longtable}
% \endgroup
%
% \begin{declcs}{Hologo} \M{name}
% \end{declcs}
% Macro \cs{Hologo} starts the logo \meta{name} with an uppercase
% letter. As an exception small greek letters are not converted
% to uppercase. Examples, see \hologo{eTeX} and \hologo{ExTeX}.
%
% \subsection{Setup macros}
%
% The package does not support package options, but the following
% setup macros can be used to set options.
%
% \begin{declcs}{hologoSetup} \M{key value list}
% \end{declcs}
% Macro \cs{hologoSetup} sets global options.
%
% \begin{declcs}{hologoLogoSetup} \M{logo} \M{key value list}
% \end{declcs}
% Some options can also be used to configure a logo.
% These settings take precedence over global option settings.
%
% \subsection{Options}\label{sec:options}
%
% There are boolean and string options:
% \begin{description}
% \item[Boolean option:]
% It takes |true| or |false|
% as value. If the value is omitted, then |true| is used.
% \item[String option:]
% A value must be given as string. (But the string might be empty.)
% \end{description}
% The following options can be used both in \cs{hologoSetup}
% and \cs{hologoLogoSetup}:
% \begin{description}
% \def\entry#1{\item[\xoption{#1}:]}
% \entry{break}
%   enables or disables line breaks inside the logo. This setting is
%   refined by options \xoption{hyphenbreak}, \xoption{spacebreak}
%   or \xoption{discretionarybreak}.
%   Default is |false|.
% \entry{hyphenbreak}
%   enables or disables the line break right after the hyphen character.
% \entry{spacebreak}
%   enables or disables line breaks at space characters.
% \entry{discretionarybreak}
%   enables or disables line breaks at hyphenation points
%   (inserted by \cs{-}).
% \end{description}
% Macro \cs{hologoLogoSetup} also knows:
% \begin{description}
% \item[\xoption{variant}:]
%   This is a string option. It specifies a variant of a logo that
%   must exist. An empty string selects the package default variant.
% \end{description}
% Example:
% \begin{quote}
%   |\hologoSetup{break=false}|\\
%   |\hologoLogoSetup{plainTeX}{variant=hyphen,hyphenbreak}|\\
%   Then ``plain-\TeX'' contains one break point after the hyphen.
% \end{quote}
%
% \subsection{Driver options}
%
% Sometimes graphical operations are needed to construct some
% glyphs (e.g.\ \hologo{XeTeX}). If package \xpackage{graphics}
% or package \xpackage{pgf} are found, then the macros are taken
% from there. Otherwise the packge defines its own operations
% and therefore needs the driver information. Many drivers are
% detected automatically (\hologo{pdfTeX}/\hologo{LuaTeX}
% in PDF mode, \hologo{XeTeX}, \hologo{VTeX}). These have precedence
% over a driver option. The driver can be given as package option
% or using \cs{hologoDriverSetup}.
% The following list contains the recognized driver options:
% \begin{itemize}
% \item \xoption{pdftex}, \xoption{luatex}
% \item \xoption{dvipdfm}, \xoption{dvipdfmx}
% \item \xoption{dvips}, \xoption{dvipsone}, \xoption{xdvi}
% \item \xoption{xetex}
% \item \xoption{vtex}
% \end{itemize}
% The left driver of a line is the driver name that is used internally.
% The following names are aliases for drivers that use the
% same method. Therefore the entry in the \xext{log} file for
% the used driver prints the internally used driver name.
% \begin{description}
% \item[\xoption{driverfallback}:]
%   This option expects a driver that is used,
%   if the driver could not be detected automatically.
% \end{description}
%
% \begin{declcs}{hologoDriverSetup} \M{driver option}
% \end{declcs}
% The driver can also be configured after package loading
% using \cs{hologoDriverSetup}, also the way for \hologo{plainTeX}
% to setup the driver.
%
% \subsection{Font setup}
%
% Some logos require a special font, but should also be usable by
% \hologo{plainTeX}. Therefore the package provides some ways
% to influence the font settings. The options below
% take font settings as values. Both font commands
% such as \cs{sffamily} and macros that take one argument
% like \cs{textsf} can be used.
%
% \begin{declcs}{hologoFontSetup} \M{key value list}
% \end{declcs}
% Macro \cs{hologoFontSetup} sets the fonts for all logos.
% Supported keys:
% \begin{description}
% \def\entry#1{\item[\xoption{#1}:]}
% \entry{general}
%   This font is used for all logos. The default is empty.
%   That means no special font is used.
% \entry{bibsf}
%   This font is used for
%   {\hologoLogoSetup{BibTeX}{variant=sf}\hologo{BibTeX}}
%   with variant \xoption{sf}.
% \entry{rm}
%   This font is a serif font. It is used for \hologo{ExTeX}.
% \entry{sc}
%   This font specifies a small caps font. It is used for
%   {\hologoLogoSetup{BibTeX}{variant=sc}\hologo{BibTeX}}
%   with variant \xoption{sc}.
% \entry{sf}
%   This font specifies a sans serif font. The default
%   is \cs{sffamily}, then \cs{sf} is tried. Otherwise
%   a warning is given. It is used by \hologo{KOMAScript}.
% \entry{sy}
%   This is the font for math symbols (e.g. cmsy).
%   It is used by \hologo{AmS}, \hologo{NTS}, \hologo{ExTeX}.
% \entry{logo}
%   \hologo{METAFONT} and \hologo{METAPOST} are using that font.
%   In \hologo{LaTeX} \cs{logofamily} is used and
%   the definitions of package \xpackage{mflogo} are used
%   if the package is not loaded.
%   Otherwise the \cs{tenlogo} is used and defined
%   if it does not already exists.
% \end{description}
%
% \begin{declcs}{hologoLogoFontSetup} \M{logo} \M{key value list}
% \end{declcs}
% Fonts can also be set for a logo or logo component separately,
% see the following list.
% The keys are the same as for \cs{hologoFontSetup}.
%
% \begin{longtable}{>{\ttfamily}l>{\sffamily}ll}
%   \meta{logo} & keys & result\\
%   \hline
%   \endhead
%   BibTeX & bibsf & {\hologoLogoSetup{BibTeX}{variant=sf}\hologo{BibTeX}}\\[.5ex]
%   BibTeX & sc & {\hologoLogoSetup{BibTeX}{variant=sc}\hologo{BibTeX}}\\[.5ex]
%   ExTeX & rm & \hologo{ExTeX}\\
%   SliTeX & rm & \hologo{SliTeX}\\[.5ex]
%   AmS & sy & \hologo{AmS}\\
%   ExTeX & sy & \hologo{ExTeX}\\
%   NTS & sy & \hologo{NTS}\\[.5ex]
%   KOMAScript & sf & \hologo{KOMAScript}\\[.5ex]
%   METAFONT & logo & \hologo{METAFONT}\\
%   METAPOST & logo & \hologo{METAPOST}\\[.5ex]
%   SliTeX & sc \hologo{SliTeX}
% \end{longtable}
%
% \subsubsection{Font order}
%
% For all logos the font \xoption{general} is applied first.
% Example:
%\begin{quote}
%|\hologoFontSetup{general=\color{red}}|
%\end{quote}
% will print red logos.
% Then if the font uses a special font \xoption{sf}, for example,
% the font is applied that is setup by \cs{hologoLogoFontSetup}.
% If this font is not setup, then the common font setup
% by \cs{hologoFontSetup} is used. Otherwise a warning is given,
% that there is no font configured.
%
% \subsection{Additional user macros}
%
% Usually a variant of a logo is configured by using
% \cs{hologoLogoSetup}, because it is bad style to mix
% different variants of the same logo in the same text.
% There the following macros are a convenience for testing.
%
% \begin{declcs}{hologoVariant} \M{name} \M{variant}\\
%   \cs{HologoVariant} \M{name} \M{variant}
% \end{declcs}
% Logo \meta{name} is set using \meta{variant} that specifies
% explicitely which variant of the macro is used. If the argument
% is empty, then the default form of the logo is used
% (configurable by \cs{hologoLogoSetup}).
%
% \cs{HologoVariant} is used if the logo is set in a context
% that needs an uppercase first letter (beginning of a sentence, \dots).
%
% \begin{declcs}{hologoList}\\
%   \cs{hologoEntry} \M{logo} \M{variant} \M{since}
% \end{declcs}
% Macro \cs{hologoList} contains all logos that are provided
% by the package including variants. The list consists of calls
% of \cs{hologoEntry} with three arguments starting with the
% logo name \meta{logo} and its variant \meta{variant}. An empty
% variant means the current default. Argument \meta{since} specifies
% with version of the package \xpackage{hologo} is needed to get
% the logo. If the logo is fixed, then the date gets updated.
% Therefore the date \meta{since} is not exactly the date of
% the first introduction, but rather the date of the latest fix.
%
% Before \cs{hologoList} can be used, macro \cs{hologoEntry} needs
% a definition. The example file in section \ref{sec:example}
% shows applications of \cs{hologoList}.
%
% \subsection{Supported contexts}
%
% Macros \cs{hologo} and friends support special contexts:
% \begin{itemize}
% \item \hologo{LaTeX}'s protection mechanism.
% \item Bookmarks of package \xpackage{hyperref}.
% \item Package \xpackage{tex4ht}.
% \item The macros can be used inside \cs{csname} constructs,
%   if \cs{ifincsname} is available (\hologo{pdfTeX}, \hologo{XeTeX},
%   \hologo{LuaTeX}).
% \end{itemize}
%
% \subsection{Example}
% \label{sec:example}
%
% The following example prints the logos in different fonts.
%    \begin{macrocode}
%<*example>
%<<verbatim
\NeedsTeXFormat{LaTeX2e}
\documentclass[a4paper]{article}
\usepackage[
  hmargin=20mm,
  vmargin=20mm,
]{geometry}
\pagestyle{empty}
\usepackage{hologo}[2016/05/12]
\usepackage{longtable}
\usepackage{array}
\setlength{\extrarowheight}{2pt}
\usepackage[T1]{fontenc}
\usepackage{lmodern}
\usepackage{pdflscape}
\usepackage[
  pdfencoding=auto,
]{hyperref}
\hypersetup{
  pdfauthor={Heiko Oberdiek},
  pdftitle={Example for package `hologo'},
  pdfsubject={Logos with fonts lmr, lmss, qtm, qpl, qhv},
}
\usepackage{bookmark}

% Print the logo list on the console

\begingroup
  \typeout{}%
  \typeout{*** Begin of logo list ***}%
  \newcommand*{\hologoEntry}[3]{%
    \typeout{#1 \ifx\\#2\\\else(#2) \fi[#3]}%
  }%
  \hologoList
  \typeout{*** End of logo list ***}%
  \typeout{}%
\endgroup

\begin{document}
\begin{landscape}

  \section{Example file for package `hologo'}

  % Table for font names

  \begin{longtable}{>{\bfseries}ll}
    \textbf{font} & \textbf{Font name}\\
    \hline
    lmr & Latin Modern Roman\\
    lmss & Latin Modern Sans\\
    qtm & \TeX\ Gyre Termes\\
    qhv & \TeX\ Gyre Heros\\
    qpl & \TeX\ Gyre Pagella\\
  \end{longtable}

  % Logo list with logos in different fonts

  \begingroup
    \newcommand*{\SetVariant}[2]{%
      \ifx\\#2\\%
      \else
        \hologoLogoSetup{#1}{variant=#2}%
      \fi
    }%
    \newcommand*{\hologoEntry}[3]{%
      \SetVariant{#1}{#2}%
      \raisebox{1em}[0pt][0pt]{\hypertarget{#1@#2}{}}%
      \bookmark[%
        dest={#1@#2},%
      ]{%
        #1\ifx\\#2\\\else\space(#2)\fi: \Hologo{#1}, \hologo{#1} %
        [Unicode]%
      }%
      \hypersetup{unicode=false}%
      \bookmark[%
        dest={#1@#2},%
      ]{%
        #1\ifx\\#2\\\else\space(#2)\fi: \Hologo{#1}, \hologo{#1} %
        [PDFDocEncoding]%
      }%
      \texttt{#1}%
      &%
      \texttt{#2}%
      &%
      \Hologo{#1}%
      &%
      \SetVariant{#1}{#2}%
      \hologo{#1}%
      &%
      \SetVariant{#1}{#2}%
      \fontfamily{qtm}\selectfont
      \hologo{#1}%
      &%
      \SetVariant{#1}{#2}%
      \fontfamily{qpl}\selectfont
      \hologo{#1}%
      &%
      \SetVariant{#1}{#2}%
      \textsf{\hologo{#1}}%
      &%
      \SetVariant{#1}{#2}%
      \fontfamily{qhv}\selectfont
      \hologo{#1}%
      \tabularnewline
    }%
    \begin{longtable}{llllllll}%
      \textbf{\textit{logo}} & \textbf{\textit{variant}} &
      \texttt{\string\Hologo} &
      \textbf{lmr} & \textbf{qtm} & \textbf{qpl} &
      \textbf{lmss} & \textbf{qhv}
      \tabularnewline
      \hline
      \endhead
      \hologoList
    \end{longtable}%
  \endgroup

\end{landscape}
\end{document}
%verbatim
%</example>
%    \end{macrocode}
%
% \StopEventually{
% }
%
% \section{Implementation}
%    \begin{macrocode}
%<*package>
%    \end{macrocode}
%    Reload check, especially if the package is not used with \LaTeX.
%    \begin{macrocode}
\begingroup\catcode61\catcode48\catcode32=10\relax%
  \catcode13=5 % ^^M
  \endlinechar=13 %
  \catcode35=6 % #
  \catcode39=12 % '
  \catcode44=12 % ,
  \catcode45=12 % -
  \catcode46=12 % .
  \catcode58=12 % :
  \catcode64=11 % @
  \catcode123=1 % {
  \catcode125=2 % }
  \expandafter\let\expandafter\x\csname ver@hologo.sty\endcsname
  \ifx\x\relax % plain-TeX, first loading
  \else
    \def\empty{}%
    \ifx\x\empty % LaTeX, first loading,
      % variable is initialized, but \ProvidesPackage not yet seen
    \else
      \expandafter\ifx\csname PackageInfo\endcsname\relax
        \def\x#1#2{%
          \immediate\write-1{Package #1 Info: #2.}%
        }%
      \else
        \def\x#1#2{\PackageInfo{#1}{#2, stopped}}%
      \fi
      \x{hologo}{The package is already loaded}%
      \aftergroup\endinput
    \fi
  \fi
\endgroup%
%    \end{macrocode}
%    Package identification:
%    \begin{macrocode}
\begingroup\catcode61\catcode48\catcode32=10\relax%
  \catcode13=5 % ^^M
  \endlinechar=13 %
  \catcode35=6 % #
  \catcode39=12 % '
  \catcode40=12 % (
  \catcode41=12 % )
  \catcode44=12 % ,
  \catcode45=12 % -
  \catcode46=12 % .
  \catcode47=12 % /
  \catcode58=12 % :
  \catcode64=11 % @
  \catcode91=12 % [
  \catcode93=12 % ]
  \catcode123=1 % {
  \catcode125=2 % }
  \expandafter\ifx\csname ProvidesPackage\endcsname\relax
    \def\x#1#2#3[#4]{\endgroup
      \immediate\write-1{Package: #3 #4}%
      \xdef#1{#4}%
    }%
  \else
    \def\x#1#2[#3]{\endgroup
      #2[{#3}]%
      \ifx#1\@undefined
        \xdef#1{#3}%
      \fi
      \ifx#1\relax
        \xdef#1{#3}%
      \fi
    }%
  \fi
\expandafter\x\csname ver@hologo.sty\endcsname
\ProvidesPackage{hologo}%
  [2016/05/12 v1.11 A logo collection with bookmark support (HO)]%
%    \end{macrocode}
%
%    \begin{macrocode}
\begingroup\catcode61\catcode48\catcode32=10\relax%
  \catcode13=5 % ^^M
  \endlinechar=13 %
  \catcode123=1 % {
  \catcode125=2 % }
  \catcode64=11 % @
  \def\x{\endgroup
    \expandafter\edef\csname HOLOGO@AtEnd\endcsname{%
      \endlinechar=\the\endlinechar\relax
      \catcode13=\the\catcode13\relax
      \catcode32=\the\catcode32\relax
      \catcode35=\the\catcode35\relax
      \catcode61=\the\catcode61\relax
      \catcode64=\the\catcode64\relax
      \catcode123=\the\catcode123\relax
      \catcode125=\the\catcode125\relax
    }%
  }%
\x\catcode61\catcode48\catcode32=10\relax%
\catcode13=5 % ^^M
\endlinechar=13 %
\catcode35=6 % #
\catcode64=11 % @
\catcode123=1 % {
\catcode125=2 % }
\def\TMP@EnsureCode#1#2{%
  \edef\HOLOGO@AtEnd{%
    \HOLOGO@AtEnd
    \catcode#1=\the\catcode#1\relax
  }%
  \catcode#1=#2\relax
}
\TMP@EnsureCode{10}{12}% ^^J
\TMP@EnsureCode{33}{12}% !
\TMP@EnsureCode{34}{12}% "
\TMP@EnsureCode{36}{3}% $
\TMP@EnsureCode{38}{4}% &
\TMP@EnsureCode{39}{12}% '
\TMP@EnsureCode{40}{12}% (
\TMP@EnsureCode{41}{12}% )
\TMP@EnsureCode{42}{12}% *
\TMP@EnsureCode{43}{12}% +
\TMP@EnsureCode{44}{12}% ,
\TMP@EnsureCode{45}{12}% -
\TMP@EnsureCode{46}{12}% .
\TMP@EnsureCode{47}{12}% /
\TMP@EnsureCode{58}{12}% :
\TMP@EnsureCode{59}{12}% ;
\TMP@EnsureCode{60}{12}% <
\TMP@EnsureCode{62}{12}% >
\TMP@EnsureCode{63}{12}% ?
\TMP@EnsureCode{91}{12}% [
\TMP@EnsureCode{93}{12}% ]
\TMP@EnsureCode{94}{7}% ^ (superscript)
\TMP@EnsureCode{95}{8}% _ (subscript)
\TMP@EnsureCode{96}{12}% `
\TMP@EnsureCode{124}{12}% |
\edef\HOLOGO@AtEnd{%
  \HOLOGO@AtEnd
  \escapechar\the\escapechar\relax
  \noexpand\endinput
}
\escapechar=92 %
%    \end{macrocode}
%
% \subsection{Logo list}
%
%    \begin{macro}{\hologoList}
%    \begin{macrocode}
\def\hologoList{%
  \hologoEntry{(La)TeX}{}{2011/10/01}%
  \hologoEntry{AmSLaTeX}{}{2010/04/16}%
  \hologoEntry{AmSTeX}{}{2010/04/16}%
  \hologoEntry{biber}{}{2011/10/01}%
  \hologoEntry{BibTeX}{}{2011/10/01}%
  \hologoEntry{BibTeX}{sf}{2011/10/01}%
  \hologoEntry{BibTeX}{sc}{2011/10/01}%
  \hologoEntry{BibTeX8}{}{2011/11/22}%
  \hologoEntry{ConTeXt}{}{2011/03/25}%
  \hologoEntry{ConTeXt}{narrow}{2011/03/25}%
  \hologoEntry{ConTeXt}{simple}{2011/03/25}%
  \hologoEntry{emTeX}{}{2010/04/26}%
  \hologoEntry{eTeX}{}{2010/04/08}%
  \hologoEntry{ExTeX}{}{2011/10/01}%
  \hologoEntry{HanTheThanh}{}{2011/11/29}%
  \hologoEntry{iniTeX}{}{2011/10/01}%
  \hologoEntry{KOMAScript}{}{2011/10/01}%
  \hologoEntry{La}{}{2010/05/08}%
  \hologoEntry{LaTeX}{}{2010/04/08}%
  \hologoEntry{LaTeX2e}{}{2010/04/08}%
  \hologoEntry{LaTeX3}{}{2010/04/24}%
  \hologoEntry{LaTeXe}{}{2010/04/08}%
  \hologoEntry{LaTeXML}{}{2011/11/22}%
  \hologoEntry{LaTeXTeX}{}{2011/10/01}%
  \hologoEntry{LuaLaTeX}{}{2010/04/08}%
  \hologoEntry{LuaTeX}{}{2010/04/08}%
  \hologoEntry{LyX}{}{2011/10/01}%
  \hologoEntry{METAFONT}{}{2011/10/01}%
  \hologoEntry{MetaFun}{}{2011/10/01}%
  \hologoEntry{METAPOST}{}{2011/10/01}%
  \hologoEntry{MetaPost}{}{2011/10/01}%
  \hologoEntry{MiKTeX}{}{2011/10/01}%
  \hologoEntry{NTS}{}{2011/10/01}%
  \hologoEntry{OzMF}{}{2011/10/01}%
  \hologoEntry{OzMP}{}{2011/10/01}%
  \hologoEntry{OzTeX}{}{2011/10/01}%
  \hologoEntry{OzTtH}{}{2011/10/01}%
  \hologoEntry{PCTeX}{}{2011/10/01}%
  \hologoEntry{pdfTeX}{}{2011/10/01}%
  \hologoEntry{pdfLaTeX}{}{2011/10/01}%
  \hologoEntry{PiC}{}{2011/10/01}%
  \hologoEntry{PiCTeX}{}{2011/10/01}%
  \hologoEntry{plainTeX}{}{2010/04/08}%
  \hologoEntry{plainTeX}{space}{2010/04/16}%
  \hologoEntry{plainTeX}{hyphen}{2010/04/16}%
  \hologoEntry{plainTeX}{runtogether}{2010/04/16}%
  \hologoEntry{SageTeX}{}{2011/11/22}%
  \hologoEntry{SLiTeX}{}{2011/10/01}%
  \hologoEntry{SLiTeX}{lift}{2011/10/01}%
  \hologoEntry{SLiTeX}{narrow}{2011/10/01}%
  \hologoEntry{SLiTeX}{simple}{2011/10/01}%
  \hologoEntry{SliTeX}{}{2011/10/01}%
  \hologoEntry{SliTeX}{narrow}{2011/10/01}%
  \hologoEntry{SliTeX}{simple}{2011/10/01}%
  \hologoEntry{SliTeX}{lift}{2011/10/01}%
  \hologoEntry{teTeX}{}{2011/10/01}%
  \hologoEntry{TeX}{}{2010/04/08}%
  \hologoEntry{TeX4ht}{}{2011/11/22}%
  \hologoEntry{TTH}{}{2011/11/22}%
  \hologoEntry{virTeX}{}{2011/10/01}%
  \hologoEntry{VTeX}{}{2010/04/24}%
  \hologoEntry{Xe}{}{2010/04/08}%
  \hologoEntry{XeLaTeX}{}{2010/04/08}%
  \hologoEntry{XeTeX}{}{2010/04/08}%
}
%    \end{macrocode}
%    \end{macro}
%
% \subsection{Load resources}
%
%    \begin{macrocode}
\begingroup\expandafter\expandafter\expandafter\endgroup
\expandafter\ifx\csname RequirePackage\endcsname\relax
  \def\TMP@RequirePackage#1[#2]{%
    \begingroup\expandafter\expandafter\expandafter\endgroup
    \expandafter\ifx\csname ver@#1.sty\endcsname\relax
      \input #1.sty\relax
    \fi
  }%
  \TMP@RequirePackage{ltxcmds}[2011/02/04]%
  \TMP@RequirePackage{infwarerr}[2010/04/08]%
  \TMP@RequirePackage{kvsetkeys}[2010/03/01]%
  \TMP@RequirePackage{kvdefinekeys}[2010/03/01]%
  \TMP@RequirePackage{pdftexcmds}[2010/04/01]%
  \TMP@RequirePackage{ifpdf}[2010/01/28]%
  \TMP@RequirePackage{ifluatex}[2010/03/01]%
  \ltx@IfUndefined{newif}{%
    \expandafter\let\csname newif\endcsname\ltx@newif
  }{}%
  \TMP@RequirePackage{ifxetex}[2009/01/23]%
  \TMP@RequirePackage{ifvtex}[2010/03/01]%
\else
  \RequirePackage{ltxcmds}[2011/02/04]%
  \RequirePackage{infwarerr}[2010/04/08]%
  \RequirePackage{kvsetkeys}[2010/03/01]%
  \RequirePackage{kvdefinekeys}[2010/03/01]%
  \RequirePackage{pdftexcmds}[2010/04/01]%
  \RequirePackage{ifpdf}[2010/01/28]%
  \RequirePackage{ifluatex}[2010/03/01]%
  \RequirePackage{ifxetex}[2009/01/23]%
  \RequirePackage{ifvtex}[2010/03/01]%
\fi
%    \end{macrocode}
%
%    \begin{macro}{\HOLOGO@IfDefined}
%    \begin{macrocode}
\def\HOLOGO@IfExists#1{%
  \ifx\@undefined#1%
    \expandafter\ltx@secondoftwo
  \else
    \ifx\relax#1%
      \expandafter\ltx@secondoftwo
    \else
      \expandafter\expandafter\expandafter\ltx@firstoftwo
    \fi
  \fi
}
%    \end{macrocode}
%    \end{macro}
%
% \subsection{Setup macros}
%
%    \begin{macro}{\hologoSetup}
%    \begin{macrocode}
\def\hologoSetup{%
  \let\HOLOGO@name\relax
  \HOLOGO@Setup
}
%    \end{macrocode}
%    \end{macro}
%
%    \begin{macro}{\hologoLogoSetup}
%    \begin{macrocode}
\def\hologoLogoSetup#1{%
  \edef\HOLOGO@name{#1}%
  \ltx@IfUndefined{HoLogo@\HOLOGO@name}{%
    \@PackageError{hologo}{%
      Unknown logo `\HOLOGO@name'%
    }\@ehc
    \ltx@gobble
  }{%
    \HOLOGO@Setup
  }%
}
%    \end{macrocode}
%    \end{macro}
%
%    \begin{macro}{\HOLOGO@Setup}
%    \begin{macrocode}
\def\HOLOGO@Setup{%
  \kvsetkeys{HoLogo}%
}
%    \end{macrocode}
%    \end{macro}
%
% \subsection{Options}
%
%    \begin{macro}{\HOLOGO@DeclareBoolOption}
%    \begin{macrocode}
\def\HOLOGO@DeclareBoolOption#1{%
  \expandafter\chardef\csname HOLOGOOPT@#1\endcsname\ltx@zero
  \kv@define@key{HoLogo}{#1}[true]{%
    \def\HOLOGO@temp{##1}%
    \ifx\HOLOGO@temp\HOLOGO@true
      \ifx\HOLOGO@name\relax
        \expandafter\chardef\csname HOLOGOOPT@#1\endcsname=\ltx@one
      \else
        \expandafter\chardef\csname
        HoLogoOpt@#1@\HOLOGO@name\endcsname\ltx@one
      \fi
      \HOLOGO@SetBreakAll{#1}%
    \else
      \ifx\HOLOGO@temp\HOLOGO@false
        \ifx\HOLOGO@name\relax
          \expandafter\chardef\csname HOLOGOOPT@#1\endcsname=\ltx@zero
        \else
          \expandafter\chardef\csname
          HoLogoOpt@#1@\HOLOGO@name\endcsname=\ltx@zero
        \fi
        \HOLOGO@SetBreakAll{#1}%
      \else
        \@PackageError{hologo}{%
          Unknown value `##1' for boolean option `#1'.\MessageBreak
          Known values are `true' and `false'%
        }\@ehc
      \fi
    \fi
  }%
}
%    \end{macrocode}
%    \end{macro}
%
%    \begin{macro}{\HOLOGO@SetBreakAll}
%    \begin{macrocode}
\def\HOLOGO@SetBreakAll#1{%
  \def\HOLOGO@temp{#1}%
  \ifx\HOLOGO@temp\HOLOGO@break
    \ifx\HOLOGO@name\relax
      \chardef\HOLOGOOPT@hyphenbreak=\HOLOGOOPT@break
      \chardef\HOLOGOOPT@spacebreak=\HOLOGOOPT@break
      \chardef\HOLOGOOPT@discretionarybreak=\HOLOGOOPT@break
    \else
      \expandafter\chardef
         \csname HoLogoOpt@hyphenbreak@\HOLOGO@name\endcsname=%
         \csname HoLogoOpt@break@\HOLOGO@name\endcsname
      \expandafter\chardef
         \csname HoLogoOpt@spacebreak@\HOLOGO@name\endcsname=%
         \csname HoLogoOpt@break@\HOLOGO@name\endcsname
      \expandafter\chardef
         \csname HoLogoOpt@discretionarybreak@\HOLOGO@name
             \endcsname=%
         \csname HoLogoOpt@break@\HOLOGO@name\endcsname
    \fi
  \fi
}
%    \end{macrocode}
%    \end{macro}
%
%    \begin{macro}{\HOLOGO@true}
%    \begin{macrocode}
\def\HOLOGO@true{true}
%    \end{macrocode}
%    \end{macro}
%    \begin{macro}{\HOLOGO@false}
%    \begin{macrocode}
\def\HOLOGO@false{false}
%    \end{macrocode}
%    \end{macro}
%    \begin{macro}{\HOLOGO@break}
%    \begin{macrocode}
\def\HOLOGO@break{break}
%    \end{macrocode}
%    \end{macro}
%
%    \begin{macrocode}
\HOLOGO@DeclareBoolOption{break}
\HOLOGO@DeclareBoolOption{hyphenbreak}
\HOLOGO@DeclareBoolOption{spacebreak}
\HOLOGO@DeclareBoolOption{discretionarybreak}
%    \end{macrocode}
%
%    \begin{macrocode}
\kv@define@key{HoLogo}{variant}{%
  \ifx\HOLOGO@name\relax
    \@PackageError{hologo}{%
      Option `variant' is not available in \string\hologoSetup,%
      \MessageBreak
      Use \string\hologoLogoSetup\space instead%
    }\@ehc
  \else
    \edef\HOLOGO@temp{#1}%
    \ifx\HOLOGO@temp\ltx@empty
      \expandafter
      \let\csname HoLogoOpt@variant@\HOLOGO@name\endcsname\@undefined
    \else
      \ltx@IfUndefined{HoLogo@\HOLOGO@name @\HOLOGO@temp}{%
        \@PackageError{hologo}{%
          Unknown variant `\HOLOGO@temp' of logo `\HOLOGO@name'%
        }\@ehc
      }{%
        \expandafter
        \let\csname HoLogoOpt@variant@\HOLOGO@name\endcsname
            \HOLOGO@temp
      }%
    \fi
  \fi
}
%    \end{macrocode}
%
%    \begin{macro}{\HOLOGO@Variant}
%    \begin{macrocode}
\def\HOLOGO@Variant#1{%
  #1%
  \ltx@ifundefined{HoLogoOpt@variant@#1}{%
  }{%
    @\csname HoLogoOpt@variant@#1\endcsname
  }%
}
%    \end{macrocode}
%    \end{macro}
%
% \subsection{Break/no-break support}
%
%    \begin{macro}{\HOLOGO@space}
%    \begin{macrocode}
\def\HOLOGO@space{%
  \ltx@ifundefined{HoLogoOpt@spacebreak@\HOLOGO@name}{%
    \ltx@ifundefined{HoLogoOpt@break@\HOLOGO@name}{%
      \chardef\HOLOGO@temp=\HOLOGOOPT@spacebreak
    }{%
      \chardef\HOLOGO@temp=%
        \csname HoLogoOpt@break@\HOLOGO@name\endcsname
    }%
  }{%
    \chardef\HOLOGO@temp=%
      \csname HoLogoOpt@spacebreak@\HOLOGO@name\endcsname
  }%
  \ifcase\HOLOGO@temp
    \penalty10000 %
  \fi
  \ltx@space
}
%    \end{macrocode}
%    \end{macro}
%
%    \begin{macro}{\HOLOGO@hyphen}
%    \begin{macrocode}
\def\HOLOGO@hyphen{%
  \ltx@ifundefined{HoLogoOpt@hyphenbreak@\HOLOGO@name}{%
    \ltx@ifundefined{HoLogoOpt@break@\HOLOGO@name}{%
      \chardef\HOLOGO@temp=\HOLOGOOPT@hyphenbreak
    }{%
      \chardef\HOLOGO@temp=%
        \csname HoLogoOpt@break@\HOLOGO@name\endcsname
    }%
  }{%
    \chardef\HOLOGO@temp=%
      \csname HoLogoOpt@hyphenbreak@\HOLOGO@name\endcsname
  }%
  \ifcase\HOLOGO@temp
    \ltx@mbox{-}%
  \else
    -%
  \fi
}
%    \end{macrocode}
%    \end{macro}
%
%    \begin{macro}{\HOLOGO@discretionary}
%    \begin{macrocode}
\def\HOLOGO@discretionary{%
  \ltx@ifundefined{HoLogoOpt@discretionarybreak@\HOLOGO@name}{%
    \ltx@ifundefined{HoLogoOpt@break@\HOLOGO@name}{%
      \chardef\HOLOGO@temp=\HOLOGOOPT@discretionarybreak
    }{%
      \chardef\HOLOGO@temp=%
        \csname HoLogoOpt@break@\HOLOGO@name\endcsname
    }%
  }{%
    \chardef\HOLOGO@temp=%
      \csname HoLogoOpt@discretionarybreak@\HOLOGO@name\endcsname
  }%
  \ifcase\HOLOGO@temp
  \else
    \-%
  \fi
}
%    \end{macrocode}
%    \end{macro}
%
%    \begin{macro}{\HOLOGO@mbox}
%    \begin{macrocode}
\def\HOLOGO@mbox#1{%
  \ltx@ifundefined{HoLogoOpt@break@\HOLOGO@name}{%
    \chardef\HOLOGO@temp=\HOLOGOOPT@hyphenbreak
  }{%
    \chardef\HOLOGO@temp=%
      \csname HoLogoOpt@break@\HOLOGO@name\endcsname
  }%
  \ifcase\HOLOGO@temp
    \ltx@mbox{#1}%
  \else
    #1%
  \fi
}
%    \end{macrocode}
%    \end{macro}
%
% \subsection{Font support}
%
%    \begin{macro}{\HoLogoFont@font}
%    \begin{tabular}{@{}ll@{}}
%    |#1|:& logo name\\
%    |#2|:& font short name\\
%    |#3|:& text
%    \end{tabular}
%    \begin{macrocode}
\def\HoLogoFont@font#1#2#3{%
  \begingroup
    \ltx@IfUndefined{HoLogoFont@logo@#1.#2}{%
      \ltx@IfUndefined{HoLogoFont@font@#2}{%
        \@PackageWarning{hologo}{%
          Missing font `#2' for logo `#1'%
        }%
        #3%
      }{%
        \csname HoLogoFont@font@#2\endcsname{#3}%
      }%
    }{%
      \csname HoLogoFont@logo@#1.#2\endcsname{#3}%
    }%
  \endgroup
}
%    \end{macrocode}
%    \end{macro}
%
%    \begin{macro}{\HoLogoFont@Def}
%    \begin{macrocode}
\def\HoLogoFont@Def#1{%
  \expandafter\def\csname HoLogoFont@font@#1\endcsname
}
%    \end{macrocode}
%    \end{macro}
%    \begin{macro}{\HoLogoFont@LogoDef}
%    \begin{macrocode}
\def\HoLogoFont@LogoDef#1#2{%
  \expandafter\def\csname HoLogoFont@logo@#1.#2\endcsname
}
%    \end{macrocode}
%    \end{macro}
%
% \subsubsection{Font defaults}
%
%    \begin{macro}{\HoLogoFont@font@general}
%    \begin{macrocode}
\HoLogoFont@Def{general}{}%
%    \end{macrocode}
%    \end{macro}
%
%    \begin{macro}{\HoLogoFont@font@rm}
%    \begin{macrocode}
\ltx@IfUndefined{rmfamily}{%
  \ltx@IfUndefined{rm}{%
  }{%
    \HoLogoFont@Def{rm}{\rm}%
  }%
}{%
  \HoLogoFont@Def{rm}{\rmfamily}%
}
%    \end{macrocode}
%    \end{macro}
%
%    \begin{macro}{\HoLogoFont@font@sf}
%    \begin{macrocode}
\ltx@IfUndefined{sffamily}{%
  \ltx@IfUndefined{sf}{%
  }{%
    \HoLogoFont@Def{sf}{\sf}%
  }%
}{%
  \HoLogoFont@Def{sf}{\sffamily}%
}
%    \end{macrocode}
%    \end{macro}
%
%    \begin{macro}{\HoLogoFont@font@bibsf}
%    In case of \hologo{plainTeX} the original small caps
%    variant is used as default. In \hologo{LaTeX}
%    the definition of package \xpackage{dtklogos} \cite{dtklogos}
%    is used.
%\begin{quote}
%\begin{verbatim}
%\DeclareRobustCommand{\BibTeX}{%
%  B%
%  \kern-.05em%
%  \hbox{%
%    $\m@th$% %% force math size calculations
%    \csname S@\f@size\endcsname
%    \fontsize\sf@size\z@
%    \math@fontsfalse
%    \selectfont
%    I%
%    \kern-.025em%
%    B
%  }%
%  \kern-.08em%
%  \-%
%  \TeX
%}
%\end{verbatim}
%\end{quote}
%    \begin{macrocode}
\ltx@IfUndefined{selectfont}{%
  \ltx@IfUndefined{tensc}{%
    \font\tensc=cmcsc10\relax
  }{}%
  \HoLogoFont@Def{bibsf}{\tensc}%
}{%
  \HoLogoFont@Def{bibsf}{%
    $\mathsurround=0pt$%
    \csname S@\f@size\endcsname
    \fontsize\sf@size{0pt}%
    \math@fontsfalse
    \selectfont
  }%
}
%    \end{macrocode}
%    \end{macro}
%
%    \begin{macro}{\HoLogoFont@font@sc}
%    \begin{macrocode}
\ltx@IfUndefined{scshape}{%
  \ltx@IfUndefined{tensc}{%
    \font\tensc=cmcsc10\relax
  }{}%
  \HoLogoFont@Def{sc}{\tensc}%
}{%
  \HoLogoFont@Def{sc}{\scshape}%
}
%    \end{macrocode}
%    \end{macro}
%
%    \begin{macro}{\HoLogoFont@font@sy}
%    \begin{macrocode}
\ltx@IfUndefined{usefont}{%
  \ltx@IfUndefined{tensy}{%
  }{%
    \HoLogoFont@Def{sy}{\tensy}%
  }%
}{%
  \HoLogoFont@Def{sy}{%
    \usefont{OMS}{cmsy}{m}{n}%
  }%
}
%    \end{macrocode}
%    \end{macro}
%
%    \begin{macro}{\HoLogoFont@font@logo}
%    \begin{macrocode}
\begingroup
  \def\x{LaTeX2e}%
\expandafter\endgroup
\ifx\fmtname\x
  \ltx@IfUndefined{logofamily}{%
    \DeclareRobustCommand\logofamily{%
      \not@math@alphabet\logofamily\relax
      \fontencoding{U}%
      \fontfamily{logo}%
      \selectfont
    }%
  }{}%
  \ltx@IfUndefined{logofamily}{%
  }{%
    \HoLogoFont@Def{logo}{\logofamily}%
  }%
\else
  \ltx@IfUndefined{tenlogo}{%
    \font\tenlogo=logo10\relax
  }{}%
  \HoLogoFont@Def{logo}{\tenlogo}%
\fi
%    \end{macrocode}
%    \end{macro}
%
% \subsubsection{Font setup}
%
%    \begin{macro}{\hologoFontSetup}
%    \begin{macrocode}
\def\hologoFontSetup{%
  \let\HOLOGO@name\relax
  \HOLOGO@FontSetup
}
%    \end{macrocode}
%    \end{macro}
%
%    \begin{macro}{\hologoLogoFontSetup}
%    \begin{macrocode}
\def\hologoLogoFontSetup#1{%
  \edef\HOLOGO@name{#1}%
  \ltx@IfUndefined{HoLogo@\HOLOGO@name}{%
    \@PackageError{hologo}{%
      Unknown logo `\HOLOGO@name'%
    }\@ehc
    \ltx@gobble
  }{%
    \HOLOGO@FontSetup
  }%
}
%    \end{macrocode}
%    \end{macro}
%
%    \begin{macro}{\HOLOGO@FontSetup}
%    \begin{macrocode}
\def\HOLOGO@FontSetup{%
  \kvsetkeys{HoLogoFont}%
}
%    \end{macrocode}
%    \end{macro}
%
%    \begin{macrocode}
\def\HOLOGO@temp#1{%
  \kv@define@key{HoLogoFont}{#1}{%
    \ifx\HOLOGO@name\relax
      \HoLogoFont@Def{#1}{##1}%
    \else
      \HoLogoFont@LogoDef\HOLOGO@name{#1}{##1}%
    \fi
  }%
}
\HOLOGO@temp{general}
\HOLOGO@temp{sf}
%    \end{macrocode}
%
% \subsection{Generic logo commands}
%
%    \begin{macrocode}
\HOLOGO@IfExists\hologo{%
  \@PackageError{hologo}{%
    \string\hologo\ltx@space is already defined.\MessageBreak
    Package loading is aborted%
  }\@ehc
  \HOLOGO@AtEnd
}%
\HOLOGO@IfExists\hologoRobust{%
  \@PackageError{hologo}{%
    \string\hologoRobust\ltx@space is already defined.\MessageBreak
    Package loading is aborted%
  }\@ehc
  \HOLOGO@AtEnd
}%
%    \end{macrocode}
%
% \subsubsection{\cs{hologo} and friends}
%
%    \begin{macrocode}
\ifluatex
  \expandafter\ltx@firstofone
\else
  \expandafter\ltx@gobble
\fi
{%
  \ltx@IfUndefined{ifincsname}{%
    \ifnum\luatexversion<36 %
      \expandafter\ltx@gobble
    \else
      \expandafter\ltx@firstofone
    \fi
    {%
      \begingroup
        \ifcase0%
            \directlua{%
              if tex.enableprimitives then %
                tex.enableprimitives('HOLOGO@', {'ifincsname'})%
              else %
                tex.print('1')%
              end%
            }%
            \ifx\HOLOGO@ifincsname\@undefined 1\fi%
            \relax
          \expandafter\ltx@firstofone
        \else
          \endgroup
          \expandafter\ltx@gobble
        \fi
        {%
          \global\let\ifincsname\HOLOGO@ifincsname
        }%
      \HOLOGO@temp
    }%
  }{}%
}
%    \end{macrocode}
%    \begin{macrocode}
\ltx@IfUndefined{ifincsname}{%
  \catcode`$=14 %
}{%
  \catcode`$=9 %
}
%    \end{macrocode}
%
%    \begin{macro}{\hologo}
%    \begin{macrocode}
\def\hologo#1{%
$ \ifincsname
$   \ltx@ifundefined{HoLogoCs@\HOLOGO@Variant{#1}}{%
$     #1%
$   }{%
$     \csname HoLogoCs@\HOLOGO@Variant{#1}\endcsname\ltx@firstoftwo
$   }%
$ \else
    \HOLOGO@IfExists\texorpdfstring\texorpdfstring\ltx@firstoftwo
    {%
      \hologoRobust{#1}%
    }{%
      \ltx@ifundefined{HoLogoBkm@\HOLOGO@Variant{#1}}{%
        \ltx@ifundefined{HoLogo@#1}{?#1?}{#1}%
      }{%
        \csname HoLogoBkm@\HOLOGO@Variant{#1}\endcsname
        \ltx@firstoftwo
      }%
    }%
$ \fi
}
%    \end{macrocode}
%    \end{macro}
%    \begin{macro}{\Hologo}
%    \begin{macrocode}
\def\Hologo#1{%
$ \ifincsname
$   \ltx@ifundefined{HoLogoCs@\HOLOGO@Variant{#1}}{%
$     #1%
$   }{%
$     \csname HoLogoCs@\HOLOGO@Variant{#1}\endcsname\ltx@secondoftwo
$   }%
$ \else
    \HOLOGO@IfExists\texorpdfstring\texorpdfstring\ltx@firstoftwo
    {%
      \HologoRobust{#1}%
    }{%
      \ltx@ifundefined{HoLogoBkm@\HOLOGO@Variant{#1}}{%
        \ltx@ifundefined{HoLogo@#1}{?#1?}{#1}%
      }{%
        \csname HoLogoBkm@\HOLOGO@Variant{#1}\endcsname
        \ltx@secondoftwo
      }%
    }%
$ \fi
}
%    \end{macrocode}
%    \end{macro}
%
%    \begin{macro}{\hologoVariant}
%    \begin{macrocode}
\def\hologoVariant#1#2{%
  \ifx\relax#2\relax
    \hologo{#1}%
  \else
$   \ifincsname
$     \ltx@ifundefined{HoLogoCs@#1@#2}{%
$       #1%
$     }{%
$       \csname HoLogoCs@#1@#2\endcsname\ltx@firstoftwo
$     }%
$   \else
      \HOLOGO@IfExists\texorpdfstring\texorpdfstring\ltx@firstoftwo
      {%
        \hologoVariantRobust{#1}{#2}%
      }{%
        \ltx@ifundefined{HoLogoBkm@#1@#2}{%
          \ltx@ifundefined{HoLogo@#1}{?#1?}{#1}%
        }{%
          \csname HoLogoBkm@#1@#2\endcsname
          \ltx@firstoftwo
        }%
      }%
$   \fi
  \fi
}
%    \end{macrocode}
%    \end{macro}
%    \begin{macro}{\HologoVariant}
%    \begin{macrocode}
\def\HologoVariant#1#2{%
  \ifx\relax#2\relax
    \Hologo{#1}%
  \else
$   \ifincsname
$     \ltx@ifundefined{HoLogoCs@#1@#2}{%
$       #1%
$     }{%
$       \csname HoLogoCs@#1@#2\endcsname\ltx@secondoftwo
$     }%
$   \else
      \HOLOGO@IfExists\texorpdfstring\texorpdfstring\ltx@firstoftwo
      {%
        \HologoVariantRobust{#1}{#2}%
      }{%
        \ltx@ifundefined{HoLogoBkm@#1@#2}{%
          \ltx@ifundefined{HoLogo@#1}{?#1?}{#1}%
        }{%
          \csname HoLogoBkm@#1@#2\endcsname
          \ltx@secondoftwo
        }%
      }%
$   \fi
  \fi
}
%    \end{macrocode}
%    \end{macro}
%
%    \begin{macrocode}
\catcode`\$=3 %
%    \end{macrocode}
%
% \subsubsection{\cs{hologoRobust} and friends}
%
%    \begin{macro}{\hologoRobust}
%    \begin{macrocode}
\ltx@IfUndefined{protected}{%
  \ltx@IfUndefined{DeclareRobustCommand}{%
    \def\hologoRobust#1%
  }{%
    \DeclareRobustCommand*\hologoRobust[1]%
  }%
}{%
  \protected\def\hologoRobust#1%
}%
{%
  \edef\HOLOGO@name{#1}%
  \ltx@IfUndefined{HoLogo@\HOLOGO@Variant\HOLOGO@name}{%
    \@PackageError{hologo}{%
      Unknown logo `\HOLOGO@name'%
    }\@ehc
    ?\HOLOGO@name?%
  }{%
    \ltx@IfUndefined{ver@tex4ht.sty}{%
      \HoLogoFont@font\HOLOGO@name{general}{%
        \csname HoLogo@\HOLOGO@Variant\HOLOGO@name\endcsname
        \ltx@firstoftwo
      }%
    }{%
      \ltx@IfUndefined{HoLogoHtml@\HOLOGO@Variant\HOLOGO@name}{%
        \HOLOGO@name
      }{%
        \csname HoLogoHtml@\HOLOGO@Variant\HOLOGO@name\endcsname
        \ltx@firstoftwo
      }%
    }%
  }%
}
%    \end{macrocode}
%    \end{macro}
%    \begin{macro}{\HologoRobust}
%    \begin{macrocode}
\ltx@IfUndefined{protected}{%
  \ltx@IfUndefined{DeclareRobustCommand}{%
    \def\HologoRobust#1%
  }{%
    \DeclareRobustCommand*\HologoRobust[1]%
  }%
}{%
  \protected\def\HologoRobust#1%
}%
{%
  \edef\HOLOGO@name{#1}%
  \ltx@IfUndefined{HoLogo@\HOLOGO@Variant\HOLOGO@name}{%
    \@PackageError{hologo}{%
      Unknown logo `\HOLOGO@name'%
    }\@ehc
    ?\HOLOGO@name?%
  }{%
    \ltx@IfUndefined{ver@tex4ht.sty}{%
      \HoLogoFont@font\HOLOGO@name{general}{%
        \csname HoLogo@\HOLOGO@Variant\HOLOGO@name\endcsname
        \ltx@secondoftwo
      }%
    }{%
      \ltx@IfUndefined{HoLogoHtml@\HOLOGO@Variant\HOLOGO@name}{%
        \expandafter\HOLOGO@Uppercase\HOLOGO@name
      }{%
        \csname HoLogoHtml@\HOLOGO@Variant\HOLOGO@name\endcsname
        \ltx@secondoftwo
      }%
    }%
  }%
}
%    \end{macrocode}
%    \end{macro}
%    \begin{macro}{\hologoVariantRobust}
%    \begin{macrocode}
\ltx@IfUndefined{protected}{%
  \ltx@IfUndefined{DeclareRobustCommand}{%
    \def\hologoVariantRobust#1#2%
  }{%
    \DeclareRobustCommand*\hologoVariantRobust[2]%
  }%
}{%
  \protected\def\hologoVariantRobust#1#2%
}%
{%
  \begingroup
    \hologoLogoSetup{#1}{variant={#2}}%
    \hologoRobust{#1}%
  \endgroup
}
%    \end{macrocode}
%    \end{macro}
%    \begin{macro}{\HologoVariantRobust}
%    \begin{macrocode}
\ltx@IfUndefined{protected}{%
  \ltx@IfUndefined{DeclareRobustCommand}{%
    \def\HologoVariantRobust#1#2%
  }{%
    \DeclareRobustCommand*\HologoVariantRobust[2]%
  }%
}{%
  \protected\def\HologoVariantRobust#1#2%
}%
{%
  \begingroup
    \hologoLogoSetup{#1}{variant={#2}}%
    \HologoRobust{#1}%
  \endgroup
}
%    \end{macrocode}
%    \end{macro}
%
%    \begin{macro}{\hologorobust}
%    Macro \cs{hologorobust} is only defined for compatibility.
%    Its use is deprecated.
%    \begin{macrocode}
\def\hologorobust{\hologoRobust}
%    \end{macrocode}
%    \end{macro}
%
% \subsection{Helpers}
%
%    \begin{macro}{\HOLOGO@Uppercase}
%    Macro \cs{HOLOGO@Uppercase} is restricted to \cs{uppercase},
%    because \hologo{plainTeX} or \hologo{iniTeX} do not provide
%    \cs{MakeUppercase}.
%    \begin{macrocode}
\def\HOLOGO@Uppercase#1{\uppercase{#1}}
%    \end{macrocode}
%    \end{macro}
%
%    \begin{macro}{\HOLOGO@PdfdocUnicode}
%    \begin{macrocode}
\def\HOLOGO@PdfdocUnicode{%
  \ifx\ifHy@unicode\iftrue
    \expandafter\ltx@secondoftwo
  \else
    \expandafter\ltx@firstoftwo
  \fi
}
%    \end{macrocode}
%    \end{macro}
%
%    \begin{macro}{\HOLOGO@Math}
%    \begin{macrocode}
\def\HOLOGO@MathSetup{%
  \mathsurround0pt\relax
  \HOLOGO@IfExists\f@series{%
    \if b\expandafter\ltx@car\f@series x\@nil
      \csname boldmath\endcsname
   \fi
  }{}%
}
%    \end{macrocode}
%    \end{macro}
%
%    \begin{macro}{\HOLOGO@TempDimen}
%    \begin{macrocode}
\dimendef\HOLOGO@TempDimen=\ltx@zero
%    \end{macrocode}
%    \end{macro}
%    \begin{macro}{\HOLOGO@NegativeKerning}
%    \begin{macrocode}
\def\HOLOGO@NegativeKerning#1{%
  \begingroup
    \HOLOGO@TempDimen=0pt\relax
    \comma@parse@normalized{#1}{%
      \ifdim\HOLOGO@TempDimen=0pt %
        \expandafter\HOLOGO@@NegativeKerning\comma@entry
      \fi
      \ltx@gobble
    }%
    \ifdim\HOLOGO@TempDimen<0pt %
      \kern\HOLOGO@TempDimen
    \fi
  \endgroup
}
%    \end{macrocode}
%    \end{macro}
%    \begin{macro}{\HOLOGO@@NegativeKerning}
%    \begin{macrocode}
\def\HOLOGO@@NegativeKerning#1#2{%
  \setbox\ltx@zero\hbox{#1#2}%
  \HOLOGO@TempDimen=\wd\ltx@zero
  \setbox\ltx@zero\hbox{#1\kern0pt#2}%
  \advance\HOLOGO@TempDimen by -\wd\ltx@zero
}
%    \end{macrocode}
%    \end{macro}
%
%    \begin{macro}{\HOLOGO@SpaceFactor}
%    \begin{macrocode}
\def\HOLOGO@SpaceFactor{%
  \spacefactor1000 %
}
%    \end{macrocode}
%    \end{macro}
%
%    \begin{macro}{\HOLOGO@Span}
%    \begin{macrocode}
\def\HOLOGO@Span#1#2{%
  \HCode{<span class="HoLogo-#1">}%
  #2%
  \HCode{</span>}%
}
%    \end{macrocode}
%    \end{macro}
%
% \subsubsection{Text subscript}
%
%    \begin{macro}{\HOLOGO@SubScript}%
%    \begin{macrocode}
\def\HOLOGO@SubScript#1{%
  \ltx@IfUndefined{textsubscript}{%
    \ltx@IfUndefined{text}{%
      \ltx@mbox{%
        \mathsurround=0pt\relax
        $%
          _{%
            \ltx@IfUndefined{sf@size}{%
              \mathrm{#1}%
            }{%
              \mbox{%
                \fontsize\sf@size{0pt}\selectfont
                #1%
              }%
            }%
          }%
        $%
      }%
    }{%
      \ltx@mbox{%
        \mathsurround=0pt\relax
        $_{\text{#1}}$%
      }%
    }%
  }{%
    \textsubscript{#1}%
  }%
}
%    \end{macrocode}
%    \end{macro}
%
% \subsection{\hologo{TeX} and friends}
%
% \subsubsection{\hologo{TeX}}
%
%    \begin{macro}{\HoLogo@TeX}
%    Source: \hologo{LaTeX} kernel.
%    \begin{macrocode}
\def\HoLogo@TeX#1{%
  T\kern-.1667em\lower.5ex\hbox{E}\kern-.125emX\HOLOGO@SpaceFactor
}
%    \end{macrocode}
%    \end{macro}
%    \begin{macro}{\HoLogoHtml@TeX}
%    \begin{macrocode}
\def\HoLogoHtml@TeX#1{%
  \HoLogoCss@TeX
  \HOLOGO@Span{TeX}{%
    T%
    \HOLOGO@Span{e}{%
      E%
    }%
    X%
  }%
}
%    \end{macrocode}
%    \end{macro}
%    \begin{macro}{\HoLogoCss@TeX}
%    \begin{macrocode}
\def\HoLogoCss@TeX{%
  \Css{%
    span.HoLogo-TeX span.HoLogo-e{%
      position:relative;%
      top:.5ex;%
      margin-left:-.1667em;%
      margin-right:-.125em;%
    }%
  }%
  \Css{%
    a span.HoLogo-TeX span.HoLogo-e{%
      text-decoration:none;%
    }%
  }%
  \global\let\HoLogoCss@TeX\relax
}
%    \end{macrocode}
%    \end{macro}
%
% \subsubsection{\hologo{plainTeX}}
%
%    \begin{macro}{\HoLogo@plainTeX@space}
%    Source: ``The \hologo{TeX}book''
%    \begin{macrocode}
\def\HoLogo@plainTeX@space#1{%
  \HOLOGO@mbox{#1{p}{P}lain}\HOLOGO@space\hologo{TeX}%
}
%    \end{macrocode}
%    \end{macro}
%    \begin{macro}{\HoLogoCs@plainTeX@space}
%    \begin{macrocode}
\def\HoLogoCs@plainTeX@space#1{#1{p}{P}lain TeX}%
%    \end{macrocode}
%    \end{macro}
%    \begin{macro}{\HoLogoBkm@plainTeX@space}
%    \begin{macrocode}
\def\HoLogoBkm@plainTeX@space#1{%
  #1{p}{P}lain \hologo{TeX}%
}
%    \end{macrocode}
%    \end{macro}
%    \begin{macro}{\HoLogoHtml@plainTeX@space}
%    \begin{macrocode}
\def\HoLogoHtml@plainTeX@space#1{%
  #1{p}{P}lain \hologo{TeX}%
}
%    \end{macrocode}
%    \end{macro}
%
%    \begin{macro}{\HoLogo@plainTeX@hyphen}
%    \begin{macrocode}
\def\HoLogo@plainTeX@hyphen#1{%
  \HOLOGO@mbox{#1{p}{P}lain}\HOLOGO@hyphen\hologo{TeX}%
}
%    \end{macrocode}
%    \end{macro}
%    \begin{macro}{\HoLogoCs@plainTeX@hyphen}
%    \begin{macrocode}
\def\HoLogoCs@plainTeX@hyphen#1{#1{p}{P}lain-TeX}
%    \end{macrocode}
%    \end{macro}
%    \begin{macro}{\HoLogoBkm@plainTeX@hyphen}
%    \begin{macrocode}
\def\HoLogoBkm@plainTeX@hyphen#1{%
  #1{p}{P}lain-\hologo{TeX}%
}
%    \end{macrocode}
%    \end{macro}
%    \begin{macro}{\HoLogoHtml@plainTeX@hyphen}
%    \begin{macrocode}
\def\HoLogoHtml@plainTeX@hyphen#1{%
  #1{p}{P}lain-\hologo{TeX}%
}
%    \end{macrocode}
%    \end{macro}
%
%    \begin{macro}{\HoLogo@plainTeX@runtogether}
%    \begin{macrocode}
\def\HoLogo@plainTeX@runtogether#1{%
  \HOLOGO@mbox{#1{p}{P}lain\hologo{TeX}}%
}
%    \end{macrocode}
%    \end{macro}
%    \begin{macro}{\HoLogoCs@plainTeX@runtogether}
%    \begin{macrocode}
\def\HoLogoCs@plainTeX@runtogether#1{#1{p}{P}lainTeX}
%    \end{macrocode}
%    \end{macro}
%    \begin{macro}{\HoLogoBkm@plainTeX@runtogether}
%    \begin{macrocode}
\def\HoLogoBkm@plainTeX@runtogether#1{%
  #1{p}{P}lain\hologo{TeX}%
}
%    \end{macrocode}
%    \end{macro}
%    \begin{macro}{\HoLogoHtml@plainTeX@runtogether}
%    \begin{macrocode}
\def\HoLogoHtml@plainTeX@runtogether#1{%
  #1{p}{P}lain\hologo{TeX}%
}
%    \end{macrocode}
%    \end{macro}
%
%    \begin{macro}{\HoLogo@plainTeX}
%    \begin{macrocode}
\def\HoLogo@plainTeX{\HoLogo@plainTeX@space}
%    \end{macrocode}
%    \end{macro}
%    \begin{macro}{\HoLogoCs@plainTeX}
%    \begin{macrocode}
\def\HoLogoCs@plainTeX{\HoLogoCs@plainTeX@space}
%    \end{macrocode}
%    \end{macro}
%    \begin{macro}{\HoLogoBkm@plainTeX}
%    \begin{macrocode}
\def\HoLogoBkm@plainTeX{\HoLogoBkm@plainTeX@space}
%    \end{macrocode}
%    \end{macro}
%    \begin{macro}{\HoLogoHtml@plainTeX}
%    \begin{macrocode}
\def\HoLogoHtml@plainTeX{\HoLogoHtml@plainTeX@space}
%    \end{macrocode}
%    \end{macro}
%
% \subsubsection{\hologo{LaTeX}}
%
%    Source: \hologo{LaTeX} kernel.
%\begin{quote}
%\begin{verbatim}
%\DeclareRobustCommand{\LaTeX}{%
%  L%
%  \kern-.36em%
%  {%
%    \sbox\z@ T%
%    \vbox to\ht\z@{%
%      \hbox{%
%        \check@mathfonts
%        \fontsize\sf@size\z@
%        \math@fontsfalse
%        \selectfont
%        A%
%      }%
%      \vss
%    }%
%  }%
%  \kern-.15em%
%  \TeX
%}
%\end{verbatim}
%\end{quote}
%
%    \begin{macro}{\HoLogo@La}
%    \begin{macrocode}
\def\HoLogo@La#1{%
  L%
  \kern-.36em%
  \begingroup
    \setbox\ltx@zero\hbox{T}%
    \vbox to\ht\ltx@zero{%
      \hbox{%
        \ltx@ifundefined{check@mathfonts}{%
          \csname sevenrm\endcsname
        }{%
          \check@mathfonts
          \fontsize\sf@size{0pt}%
          \math@fontsfalse\selectfont
        }%
        A%
      }%
      \vss
    }%
  \endgroup
}
%    \end{macrocode}
%    \end{macro}
%
%    \begin{macro}{\HoLogo@LaTeX}
%    Source: \hologo{LaTeX} kernel.
%    \begin{macrocode}
\def\HoLogo@LaTeX#1{%
  \hologo{La}%
  \kern-.15em%
  \hologo{TeX}%
}
%    \end{macrocode}
%    \end{macro}
%    \begin{macro}{\HoLogoHtml@LaTeX}
%    \begin{macrocode}
\def\HoLogoHtml@LaTeX#1{%
  \HoLogoCss@LaTeX
  \HOLOGO@Span{LaTeX}{%
    L%
    \HOLOGO@Span{a}{%
      A%
    }%
    \hologo{TeX}%
  }%
}
%    \end{macrocode}
%    \end{macro}
%    \begin{macro}{\HoLogoCss@LaTeX}
%    \begin{macrocode}
\def\HoLogoCss@LaTeX{%
  \Css{%
    span.HoLogo-LaTeX span.HoLogo-a{%
      position:relative;%
      top:-.5ex;%
      margin-left:-.36em;%
      margin-right:-.15em;%
      font-size:85\%;%
    }%
  }%
  \global\let\HoLogoCss@LaTeX\relax
}
%    \end{macrocode}
%    \end{macro}
%
% \subsubsection{\hologo{(La)TeX}}
%
%    \begin{macro}{\HoLogo@LaTeXTeX}
%    The kerning around the parentheses is taken
%    from package \xpackage{dtklogos} \cite{dtklogos}.
%\begin{quote}
%\begin{verbatim}
%\DeclareRobustCommand{\LaTeXTeX}{%
%  (%
%  \kern-.15em%
%  L%
%  \kern-.36em%
%  {%
%    \sbox\z@ T%
%    \vbox to\ht0{%
%      \hbox{%
%        $\m@th$%
%        \csname S@\f@size\endcsname
%        \fontsize\sf@size\z@
%        \math@fontsfalse
%        \selectfont
%        A%
%      }%
%      \vss
%    }%
%  }%
%  \kern-.2em%
%  )%
%  \kern-.15em%
%  \TeX
%}
%\end{verbatim}
%\end{quote}
%    \begin{macrocode}
\def\HoLogo@LaTeXTeX#1{%
  (%
  \kern-.15em%
  \hologo{La}%
  \kern-.2em%
  )%
  \kern-.15em%
  \hologo{TeX}%
}
%    \end{macrocode}
%    \end{macro}
%    \begin{macro}{\HoLogoBkm@LaTeXTeX}
%    \begin{macrocode}
\def\HoLogoBkm@LaTeXTeX#1{(La)TeX}
%    \end{macrocode}
%    \end{macro}
%
%    \begin{macro}{\HoLogo@(La)TeX}
%    \begin{macrocode}
\expandafter
\let\csname HoLogo@(La)TeX\endcsname\HoLogo@LaTeXTeX
%    \end{macrocode}
%    \end{macro}
%    \begin{macro}{\HoLogoBkm@(La)TeX}
%    \begin{macrocode}
\expandafter
\let\csname HoLogoBkm@(La)TeX\endcsname\HoLogoBkm@LaTeXTeX
%    \end{macrocode}
%    \end{macro}
%    \begin{macro}{\HoLogoHtml@LaTeXTeX}
%    \begin{macrocode}
\def\HoLogoHtml@LaTeXTeX#1{%
  \HoLogoCss@LaTeXTeX
  \HOLOGO@Span{LaTeXTeX}{%
    (%
    \HOLOGO@Span{L}{L}%
    \HOLOGO@Span{a}{A}%
    \HOLOGO@Span{ParenRight}{)}%
    \hologo{TeX}%
  }%
}
%    \end{macrocode}
%    \end{macro}
%    \begin{macro}{\HoLogoHtml@(La)TeX}
%    Kerning after opening parentheses and before closing parentheses
%    is $-0.1$\,em. The original values $-0.15$\,em
%    looked too ugly for a serif font.
%    \begin{macrocode}
\expandafter
\let\csname HoLogoHtml@(La)TeX\endcsname\HoLogoHtml@LaTeXTeX
%    \end{macrocode}
%    \end{macro}
%    \begin{macro}{\HoLogoCss@LaTeXTeX}
%    \begin{macrocode}
\def\HoLogoCss@LaTeXTeX{%
  \Css{%
    span.HoLogo-LaTeXTeX span.HoLogo-L{%
      margin-left:-.1em;%
    }%
  }%
  \Css{%
    span.HoLogo-LaTeXTeX span.HoLogo-a{%
      position:relative;%
      top:-.5ex;%
      margin-left:-.36em;%
      margin-right:-.1em;%
      font-size:85\%;%
    }%
  }%
  \Css{%
    span.HoLogo-LaTeXTeX span.HoLogo-ParenRight{%
      margin-right:-.15em;%
    }%
  }%
  \global\let\HoLogoCss@LaTeXTeX\relax
}
%    \end{macrocode}
%    \end{macro}
%
% \subsubsection{\hologo{LaTeXe}}
%
%    \begin{macro}{\HoLogo@LaTeXe}
%    Source: \hologo{LaTeX} kernel
%    \begin{macrocode}
\def\HoLogo@LaTeXe#1{%
  \hologo{LaTeX}%
  \kern.15em%
  \hbox{%
    \HOLOGO@MathSetup
    2%
    $_{\textstyle\varepsilon}$%
  }%
}
%    \end{macrocode}
%    \end{macro}
%
%    \begin{macro}{\HoLogoCs@LaTeXe}
%    \begin{macrocode}
\ifnum64=`\^^^^0040\relax % test for big chars of LuaTeX/XeTeX
  \catcode`\$=9 %
  \catcode`\&=14 %
\else
  \catcode`\$=14 %
  \catcode`\&=9 %
\fi
\def\HoLogoCs@LaTeXe#1{%
  LaTeX2%
$ \string ^^^^0395%
& e%
}%
\catcode`\$=3 %
\catcode`\&=4 %
%    \end{macrocode}
%    \end{macro}
%
%    \begin{macro}{\HoLogoBkm@LaTeXe}
%    \begin{macrocode}
\def\HoLogoBkm@LaTeXe#1{%
  \hologo{LaTeX}%
  2%
  \HOLOGO@PdfdocUnicode{e}{\textepsilon}%
}
%    \end{macrocode}
%    \end{macro}
%
%    \begin{macro}{\HoLogoHtml@LaTeXe}
%    \begin{macrocode}
\def\HoLogoHtml@LaTeXe#1{%
  \HoLogoCss@LaTeXe
  \HOLOGO@Span{LaTeX2e}{%
    \hologo{LaTeX}%
    \HOLOGO@Span{2}{2}%
    \HOLOGO@Span{e}{%
      \HOLOGO@MathSetup
      \ensuremath{\textstyle\varepsilon}%
    }%
  }%
}
%    \end{macrocode}
%    \end{macro}
%    \begin{macro}{\HoLogoCss@LaTeXe}
%    \begin{macrocode}
\def\HoLogoCss@LaTeXe{%
  \Css{%
    span.HoLogo-LaTeX2e span.HoLogo-2{%
      padding-left:.15em;%
    }%
  }%
  \Css{%
    span.HoLogo-LaTeX2e span.HoLogo-e{%
      position:relative;%
      top:.35ex;%
      text-decoration:none;%
    }%
  }%
  \global\let\HoLogoCss@LaTeXe\relax
}
%    \end{macrocode}
%    \end{macro}
%
%    \begin{macro}{\HoLogo@LaTeX2e}
%    \begin{macrocode}
\expandafter
\let\csname HoLogo@LaTeX2e\endcsname\HoLogo@LaTeXe
%    \end{macrocode}
%    \end{macro}
%    \begin{macro}{\HoLogoCs@LaTeX2e}
%    \begin{macrocode}
\expandafter
\let\csname HoLogoCs@LaTeX2e\endcsname\HoLogoCs@LaTeXe
%    \end{macrocode}
%    \end{macro}
%    \begin{macro}{\HoLogoBkm@LaTeX2e}
%    \begin{macrocode}
\expandafter
\let\csname HoLogoBkm@LaTeX2e\endcsname\HoLogoBkm@LaTeXe
%    \end{macrocode}
%    \end{macro}
%    \begin{macro}{\HoLogoHtml@LaTeX2e}
%    \begin{macrocode}
\expandafter
\let\csname HoLogoHtml@LaTeX2e\endcsname\HoLogoHtml@LaTeXe
%    \end{macrocode}
%    \end{macro}
%
% \subsubsection{\hologo{LaTeX3}}
%
%    \begin{macro}{\HoLogo@LaTeX3}
%    Source: \hologo{LaTeX} kernel
%    \begin{macrocode}
\expandafter\def\csname HoLogo@LaTeX3\endcsname#1{%
  \hologo{LaTeX}%
  3%
}
%    \end{macrocode}
%    \end{macro}
%
%    \begin{macro}{\HoLogoBkm@LaTeX3}
%    \begin{macrocode}
\expandafter\def\csname HoLogoBkm@LaTeX3\endcsname#1{%
  \hologo{LaTeX}%
  3%
}
%    \end{macrocode}
%    \end{macro}
%    \begin{macro}{\HoLogoHtml@LaTeX3}
%    \begin{macrocode}
\expandafter
\let\csname HoLogoHtml@LaTeX3\expandafter\endcsname
\csname HoLogo@LaTeX3\endcsname
%    \end{macrocode}
%    \end{macro}
%
% \subsubsection{\hologo{LaTeXML}}
%
%    \begin{macro}{\HoLogo@LaTeXML}
%    \begin{macrocode}
\def\HoLogo@LaTeXML#1{%
  \HOLOGO@mbox{%
    \hologo{La}%
    \kern-.15em%
    T%
    \kern-.1667em%
    \lower.5ex\hbox{E}%
    \kern-.125em%
    \HoLogoFont@font{LaTeXML}{sc}{xml}%
  }%
}
%    \end{macrocode}
%    \end{macro}
%    \begin{macro}{\HoLogoHtml@pdfLaTeX}
%    \begin{macrocode}
\def\HoLogoHtml@LaTeXML#1{%
  \HOLOGO@Span{LaTeXML}{%
    \HoLogoCss@LaTeX
    \HoLogoCss@TeX
    \HOLOGO@Span{LaTeX}{%
      L%
      \HOLOGO@Span{a}{%
        A%
      }%
    }%
    \HOLOGO@Span{TeX}{%
      T%
      \HOLOGO@Span{e}{%
        E%
      }%
    }%
    \HCode{<span style="font-variant: small-caps;">}%
    xml%
    \HCode{</span>}%
  }%
}
%    \end{macrocode}
%    \end{macro}
%
% \subsubsection{\hologo{eTeX}}
%
%    \begin{macro}{\HoLogo@eTeX}
%    Source: package \xpackage{etex}
%    \begin{macrocode}
\def\HoLogo@eTeX#1{%
  \ltx@mbox{%
    \HOLOGO@MathSetup
    $\varepsilon$%
    -%
    \HOLOGO@NegativeKerning{-T,T-,To}%
    \hologo{TeX}%
  }%
}
%    \end{macrocode}
%    \end{macro}
%    \begin{macro}{\HoLogoCs@eTeX}
%    \begin{macrocode}
\ifnum64=`\^^^^0040\relax % test for big chars of LuaTeX/XeTeX
  \catcode`\$=9 %
  \catcode`\&=14 %
\else
  \catcode`\$=14 %
  \catcode`\&=9 %
\fi
\def\HoLogoCs@eTeX#1{%
$ #1{\string ^^^^0395}{\string ^^^^03b5}%
& #1{e}{E}%
  TeX%
}%
\catcode`\$=3 %
\catcode`\&=4 %
%    \end{macrocode}
%    \end{macro}
%    \begin{macro}{\HoLogoBkm@eTeX}
%    \begin{macrocode}
\def\HoLogoBkm@eTeX#1{%
  \HOLOGO@PdfdocUnicode{#1{e}{E}}{\textepsilon}%
  -%
  \hologo{TeX}%
}
%    \end{macrocode}
%    \end{macro}
%    \begin{macro}{\HoLogoHtml@eTeX}
%    \begin{macrocode}
\def\HoLogoHtml@eTeX#1{%
  \ltx@mbox{%
    \HOLOGO@MathSetup
    $\varepsilon$%
    -%
    \hologo{TeX}%
  }%
}
%    \end{macrocode}
%    \end{macro}
%
% \subsubsection{\hologo{iniTeX}}
%
%    \begin{macro}{\HoLogo@iniTeX}
%    \begin{macrocode}
\def\HoLogo@iniTeX#1{%
  \HOLOGO@mbox{%
    #1{i}{I}ni\hologo{TeX}%
  }%
}
%    \end{macrocode}
%    \end{macro}
%    \begin{macro}{\HoLogoCs@iniTeX}
%    \begin{macrocode}
\def\HoLogoCs@iniTeX#1{#1{i}{I}niTeX}
%    \end{macrocode}
%    \end{macro}
%    \begin{macro}{\HoLogoBkm@iniTeX}
%    \begin{macrocode}
\def\HoLogoBkm@iniTeX#1{%
  #1{i}{I}ni\hologo{TeX}%
}
%    \end{macrocode}
%    \end{macro}
%    \begin{macro}{\HoLogoHtml@iniTeX}
%    \begin{macrocode}
\let\HoLogoHtml@iniTeX\HoLogo@iniTeX
%    \end{macrocode}
%    \end{macro}
%
% \subsubsection{\hologo{virTeX}}
%
%    \begin{macro}{\HoLogo@virTeX}
%    \begin{macrocode}
\def\HoLogo@virTeX#1{%
  \HOLOGO@mbox{%
    #1{v}{V}ir\hologo{TeX}%
  }%
}
%    \end{macrocode}
%    \end{macro}
%    \begin{macro}{\HoLogoCs@virTeX}
%    \begin{macrocode}
\def\HoLogoCs@virTeX#1{#1{v}{V}irTeX}
%    \end{macrocode}
%    \end{macro}
%    \begin{macro}{\HoLogoBkm@virTeX}
%    \begin{macrocode}
\def\HoLogoBkm@virTeX#1{%
  #1{v}{V}ir\hologo{TeX}%
}
%    \end{macrocode}
%    \end{macro}
%    \begin{macro}{\HoLogoHtml@virTeX}
%    \begin{macrocode}
\let\HoLogoHtml@virTeX\HoLogo@virTeX
%    \end{macrocode}
%    \end{macro}
%
% \subsubsection{\hologo{SliTeX}}
%
% \paragraph{Definitions of the three variants.}
%
%    \begin{macro}{\HoLogo@SLiTeX@lift}
%    \begin{macrocode}
\def\HoLogo@SLiTeX@lift#1{%
  \HoLogoFont@font{SliTeX}{rm}{%
    S%
    \kern-.06em%
    L%
    \kern-.18em%
    \raise.32ex\hbox{\HoLogoFont@font{SliTeX}{sc}{i}}%
    \HOLOGO@discretionary
    \kern-.06em%
    \hologo{TeX}%
  }%
}
%    \end{macrocode}
%    \end{macro}
%    \begin{macro}{\HoLogoBkm@SLiTeX@lift}
%    \begin{macrocode}
\def\HoLogoBkm@SLiTeX@lift#1{SLiTeX}
%    \end{macrocode}
%    \end{macro}
%    \begin{macro}{\HoLogoHtml@SLiTeX@lift}
%    \begin{macrocode}
\def\HoLogoHtml@SLiTeX@lift#1{%
  \HoLogoCss@SLiTeX@lift
  \HOLOGO@Span{SLiTeX-lift}{%
    \HoLogoFont@font{SliTeX}{rm}{%
      S%
      \HOLOGO@Span{L}{L}%
      \HOLOGO@Span{i}{i}%
      \hologo{TeX}%
    }%
  }%
}
%    \end{macrocode}
%    \end{macro}
%    \begin{macro}{\HoLogoCss@SLiTeX@lift}
%    \begin{macrocode}
\def\HoLogoCss@SLiTeX@lift{%
  \Css{%
    span.HoLogo-SLiTeX-lift span.HoLogo-L{%
      margin-left:-.06em;%
      margin-right:-.18em;%
    }%
  }%
  \Css{%
    span.HoLogo-SLiTeX-lift span.HoLogo-i{%
      position:relative;%
      top:-.32ex;%
      margin-right:-.06em;%
      font-variant:small-caps;%
    }%
  }%
  \global\let\HoLogoCss@SLiTeX@lift\relax
}
%    \end{macrocode}
%    \end{macro}
%
%    \begin{macro}{\HoLogo@SliTeX@simple}
%    \begin{macrocode}
\def\HoLogo@SliTeX@simple#1{%
  \HoLogoFont@font{SliTeX}{rm}{%
    \ltx@mbox{%
      \HoLogoFont@font{SliTeX}{sc}{Sli}%
    }%
    \HOLOGO@discretionary
    \hologo{TeX}%
  }%
}
%    \end{macrocode}
%    \end{macro}
%    \begin{macro}{\HoLogoBkm@SliTeX@simple}
%    \begin{macrocode}
\def\HoLogoBkm@SliTeX@simple#1{SliTeX}
%    \end{macrocode}
%    \end{macro}
%    \begin{macro}{\HoLogoHtml@SliTeX@simple}
%    \begin{macrocode}
\let\HoLogoHtml@SliTeX@simple\HoLogo@SliTeX@simple
%    \end{macrocode}
%    \end{macro}
%
%    \begin{macro}{\HoLogo@SliTeX@narrow}
%    \begin{macrocode}
\def\HoLogo@SliTeX@narrow#1{%
  \HoLogoFont@font{SliTeX}{rm}{%
    \ltx@mbox{%
      S%
      \kern-.06em%
      \HoLogoFont@font{SliTeX}{sc}{%
        l%
        \kern-.035em%
        i%
      }%
    }%
    \HOLOGO@discretionary
    \kern-.06em%
    \hologo{TeX}%
  }%
}
%    \end{macrocode}
%    \end{macro}
%    \begin{macro}{\HoLogoBkm@SliTeX@narrow}
%    \begin{macrocode}
\def\HoLogoBkm@SliTeX@narrow#1{SliTeX}
%    \end{macrocode}
%    \end{macro}
%    \begin{macro}{\HoLogoHtml@SliTeX@narrow}
%    \begin{macrocode}
\def\HoLogoHtml@SliTeX@narrow#1{%
  \HoLogoCss@SliTeX@narrow
  \HOLOGO@Span{SliTeX-narrow}{%
    \HoLogoFont@font{SliTeX}{rm}{%
      S%
        \HOLOGO@Span{l}{l}%
        \HOLOGO@Span{i}{i}%
      \hologo{TeX}%
    }%
  }%
}
%    \end{macrocode}
%    \end{macro}
%    \begin{macro}{\HoLogoCss@SliTeX@narrow}
%    \begin{macrocode}
\def\HoLogoCss@SliTeX@narrow{%
  \Css{%
    span.HoLogo-SliTeX-narrow span.HoLogo-l{%
      margin-left:-.06em;%
      margin-right:-.035em;%
      font-variant:small-caps;%
    }%
  }%
  \Css{%
    span.HoLogo-SliTeX-narrow span.HoLogo-i{%
      margin-right:-.06em;%
      font-variant:small-caps;%
    }%
  }%
  \global\let\HoLogoCss@SliTeX@narrow\relax
}
%    \end{macrocode}
%    \end{macro}
%
% \paragraph{Macro set completion.}
%
%    \begin{macro}{\HoLogo@SLiTeX@simple}
%    \begin{macrocode}
\def\HoLogo@SLiTeX@simple{\HoLogo@SliTeX@simple}
%    \end{macrocode}
%    \end{macro}
%    \begin{macro}{\HoLogoBkm@SLiTeX@simple}
%    \begin{macrocode}
\def\HoLogoBkm@SLiTeX@simple{\HoLogoBkm@SliTeX@simple}
%    \end{macrocode}
%    \end{macro}
%    \begin{macro}{\HoLogoHtml@SLiTeX@simple}
%    \begin{macrocode}
\def\HoLogoHtml@SLiTeX@simple{\HoLogoHtml@SliTeX@simple}
%    \end{macrocode}
%    \end{macro}
%
%    \begin{macro}{\HoLogo@SLiTeX@narrow}
%    \begin{macrocode}
\def\HoLogo@SLiTeX@narrow{\HoLogo@SliTeX@narrow}
%    \end{macrocode}
%    \end{macro}
%    \begin{macro}{\HoLogoBkm@SLiTeX@narrow}
%    \begin{macrocode}
\def\HoLogoBkm@SLiTeX@narrow{\HoLogoBkm@SliTeX@narrow}
%    \end{macrocode}
%    \end{macro}
%    \begin{macro}{\HoLogoHtml@SLiTeX@narrow}
%    \begin{macrocode}
\def\HoLogoHtml@SLiTeX@narrow{\HoLogoHtml@SliTeX@narrow}
%    \end{macrocode}
%    \end{macro}
%
%    \begin{macro}{\HoLogo@SliTeX@lift}
%    \begin{macrocode}
\def\HoLogo@SliTeX@lift{\HoLogo@SLiTeX@lift}
%    \end{macrocode}
%    \end{macro}
%    \begin{macro}{\HoLogoBkm@SliTeX@lift}
%    \begin{macrocode}
\def\HoLogoBkm@SliTeX@lift{\HoLogoBkm@SLiTeX@lift}
%    \end{macrocode}
%    \end{macro}
%    \begin{macro}{\HoLogoHtml@SliTeX@lift}
%    \begin{macrocode}
\def\HoLogoHtml@SliTeX@lift{\HoLogoHtml@SLiTeX@lift}
%    \end{macrocode}
%    \end{macro}
%
% \paragraph{Defaults.}
%
%    \begin{macro}{\HoLogo@SLiTeX}
%    \begin{macrocode}
\def\HoLogo@SLiTeX{\HoLogo@SLiTeX@lift}
%    \end{macrocode}
%    \end{macro}
%    \begin{macro}{\HoLogoBkm@SLiTeX}
%    \begin{macrocode}
\def\HoLogoBkm@SLiTeX{\HoLogoBkm@SLiTeX@lift}
%    \end{macrocode}
%    \end{macro}
%    \begin{macro}{\HoLogoHtml@SLiTeX}
%    \begin{macrocode}
\def\HoLogoHtml@SLiTeX{\HoLogoHtml@SLiTeX@lift}
%    \end{macrocode}
%    \end{macro}
%
%    \begin{macro}{\HoLogo@SliTeX}
%    \begin{macrocode}
\def\HoLogo@SliTeX{\HoLogo@SliTeX@narrow}
%    \end{macrocode}
%    \end{macro}
%    \begin{macro}{\HoLogoBkm@SliTeX}
%    \begin{macrocode}
\def\HoLogoBkm@SliTeX{\HoLogoBkm@SliTeX@narrow}
%    \end{macrocode}
%    \end{macro}
%    \begin{macro}{\HoLogoHtml@SliTeX}
%    \begin{macrocode}
\def\HoLogoHtml@SliTeX{\HoLogoHtml@SliTeX@narrow}
%    \end{macrocode}
%    \end{macro}
%
% \subsubsection{\hologo{LuaTeX}}
%
%    \begin{macro}{\HoLogo@LuaTeX}
%    The kerning is an idea of Hans Hagen, see mailing list
%    `luatex at tug dot org' in March 2010.
%    \begin{macrocode}
\def\HoLogo@LuaTeX#1{%
  \HOLOGO@mbox{%
    Lua%
    \HOLOGO@NegativeKerning{aT,oT,To}%
    \hologo{TeX}%
  }%
}
%    \end{macrocode}
%    \end{macro}
%    \begin{macro}{\HoLogoHtml@LuaTeX}
%    \begin{macrocode}
\let\HoLogoHtml@LuaTeX\HoLogo@LuaTeX
%    \end{macrocode}
%    \end{macro}
%
% \subsubsection{\hologo{LuaLaTeX}}
%
%    \begin{macro}{\HoLogo@LuaLaTeX}
%    \begin{macrocode}
\def\HoLogo@LuaLaTeX#1{%
  \HOLOGO@mbox{%
    Lua%
    \hologo{LaTeX}%
  }%
}
%    \end{macrocode}
%    \end{macro}
%    \begin{macro}{\HoLogoHtml@LuaLaTeX}
%    \begin{macrocode}
\let\HoLogoHtml@LuaLaTeX\HoLogo@LuaLaTeX
%    \end{macrocode}
%    \end{macro}
%
% \subsubsection{\hologo{XeTeX}, \hologo{XeLaTeX}}
%
%    \begin{macro}{\HOLOGO@IfCharExists}
%    \begin{macrocode}
\ifluatex
  \ifnum\luatexversion<36 %
  \else
    \def\HOLOGO@IfCharExists#1{%
      \ifnum
        \directlua{%
           if luaotfload and luaotfload.aux then
             if luaotfload.aux.font_has_glyph(%
                    font.current(), \number#1) then % 	 
	       tex.print("1") % 	 
	     end % 	 
	   elseif font and font.fonts and font.current then %
            local f = font.fonts[font.current()]%
            if f.characters and f.characters[\number#1] then %
              tex.print("1")%
            end %
          end%
        }0=\ltx@zero
        \expandafter\ltx@secondoftwo
      \else
        \expandafter\ltx@firstoftwo
      \fi
    }%
  \fi
\fi
\ltx@IfUndefined{HOLOGO@IfCharExists}{%
  \def\HOLOGO@@IfCharExists#1{%
    \begingroup
      \tracinglostchars=\ltx@zero
      \setbox\ltx@zero=\hbox{%
        \kern7sp\char#1\relax
        \ifnum\lastkern>\ltx@zero
          \expandafter\aftergroup\csname iffalse\endcsname
        \else
          \expandafter\aftergroup\csname iftrue\endcsname
        \fi
      }%
      % \if{true|false} from \aftergroup
      \endgroup
      \expandafter\ltx@firstoftwo
    \else
      \endgroup
      \expandafter\ltx@secondoftwo
    \fi
  }%
  \ifxetex
    \ltx@IfUndefined{XeTeXfonttype}{}{%
      \ltx@IfUndefined{XeTeXcharglyph}{}{%
        \def\HOLOGO@IfCharExists#1{%
          \ifnum\XeTeXfonttype\font>\ltx@zero
            \expandafter\ltx@firstofthree
          \else
            \expandafter\ltx@gobble
          \fi
          {%
            \ifnum\XeTeXcharglyph#1>\ltx@zero
              \expandafter\ltx@firstoftwo
            \else
              \expandafter\ltx@secondoftwo
            \fi
          }%
          \HOLOGO@@IfCharExists{#1}%
        }%
      }%
    }%
  \fi
}{}
\ltx@ifundefined{HOLOGO@IfCharExists}{%
  \ifnum64=`\^^^^0040\relax % test for big chars of LuaTeX/XeTeX
    \let\HOLOGO@IfCharExists\HOLOGO@@IfCharExists
  \else
    \def\HOLOGO@IfCharExists#1{%
      \ifnum#1>255 %
        \expandafter\ltx@fourthoffour
      \fi
      \HOLOGO@@IfCharExists{#1}%
    }%
  \fi
}{}
%    \end{macrocode}
%    \end{macro}
%
%    \begin{macro}{\HoLogo@Xe}
%    Source: package \xpackage{dtklogos}
%    \begin{macrocode}
\def\HoLogo@Xe#1{%
  X%
  \kern-.1em\relax
  \HOLOGO@IfCharExists{"018E}{%
    \lower.5ex\hbox{\char"018E}%
  }{%
    \chardef\HOLOGO@choice=\ltx@zero
    \ifdim\fontdimen\ltx@one\font>0pt %
      \ltx@IfUndefined{rotatebox}{%
        \ltx@IfUndefined{pgftext}{%
          \ltx@IfUndefined{psscalebox}{%
            \ltx@IfUndefined{HOLOGO@ScaleBox@\hologoDriver}{%
            }{%
              \chardef\HOLOGO@choice=4 %
            }%
          }{%
            \chardef\HOLOGO@choice=3 %
          }%
        }{%
          \chardef\HOLOGO@choice=2 %
        }%
      }{%
        \chardef\HOLOGO@choice=1 %
      }%
      \ifcase\HOLOGO@choice
        \HOLOGO@WarningUnsupportedDriver{Xe}%
        e%
      \or % 1: \rotatebox
        \begingroup
          \setbox\ltx@zero\hbox{\rotatebox{180}{E}}%
          \ltx@LocDimenA=\dp\ltx@zero
          \advance\ltx@LocDimenA by -.5ex\relax
          \raise\ltx@LocDimenA\box\ltx@zero
        \endgroup
      \or % 2: \pgftext
        \lower.5ex\hbox{%
          \pgfpicture
            \pgftext[rotate=180]{E}%
          \endpgfpicture
        }%
      \or % 3: \psscalebox
        \begingroup
          \setbox\ltx@zero\hbox{\psscalebox{-1 -1}{E}}%
          \ltx@LocDimenA=\dp\ltx@zero
          \advance\ltx@LocDimenA by -.5ex\relax
          \raise\ltx@LocDimenA\box\ltx@zero
        \endgroup
      \or % 4: \HOLOGO@PointReflectBox
        \lower.5ex\hbox{\HOLOGO@PointReflectBox{E}}%
      \else
        \@PackageError{hologo}{Internal error (choice/it}\@ehc
      \fi
    \else
      \ltx@IfUndefined{reflectbox}{%
        \ltx@IfUndefined{pgftext}{%
          \ltx@IfUndefined{psscalebox}{%
            \ltx@IfUndefined{HOLOGO@ScaleBox@\hologoDriver}{%
            }{%
              \chardef\HOLOGO@choice=4 %
            }%
          }{%
            \chardef\HOLOGO@choice=3 %
          }%
        }{%
          \chardef\HOLOGO@choice=2 %
        }%
      }{%
        \chardef\HOLOGO@choice=1 %
      }%
      \ifcase\HOLOGO@choice
        \HOLOGO@WarningUnsupportedDriver{Xe}%
        e%
      \or % 1: reflectbox
        \lower.5ex\hbox{%
          \reflectbox{E}%
        }%
      \or % 2: \pgftext
        \lower.5ex\hbox{%
          \pgfpicture
            \pgftransformxscale{-1}%
            \pgftext{E}%
          \endpgfpicture
        }%
      \or % 3: \psscalebox
        \lower.5ex\hbox{%
          \psscalebox{-1 1}{E}%
        }%
      \or % 4: \HOLOGO@Reflectbox
        \lower.5ex\hbox{%
          \HOLOGO@ReflectBox{E}%
        }%
      \else
        \@PackageError{hologo}{Internal error (choice/up)}\@ehc
      \fi
    \fi
  }%
}
%    \end{macrocode}
%    \end{macro}
%    \begin{macro}{\HoLogoHtml@Xe}
%    \begin{macrocode}
\def\HoLogoHtml@Xe#1{%
  \HoLogoCss@Xe
  \HOLOGO@Span{Xe}{%
    X%
    \HOLOGO@Span{e}{%
      \HCode{&\ltx@hashchar x018e;}%
    }%
  }%
}
%    \end{macrocode}
%    \end{macro}
%    \begin{macro}{\HoLogoCss@Xe}
%    \begin{macrocode}
\def\HoLogoCss@Xe{%
  \Css{%
    span.HoLogo-Xe span.HoLogo-e{%
      position:relative;%
      top:.5ex;%
      left-margin:-.1em;%
    }%
  }%
  \global\let\HoLogoCss@Xe\relax
}
%    \end{macrocode}
%    \end{macro}
%
%    \begin{macro}{\HoLogo@XeTeX}
%    \begin{macrocode}
\def\HoLogo@XeTeX#1{%
  \hologo{Xe}%
  \kern-.15em\relax
  \hologo{TeX}%
}
%    \end{macrocode}
%    \end{macro}
%
%    \begin{macro}{\HoLogoHtml@XeTeX}
%    \begin{macrocode}
\def\HoLogoHtml@XeTeX#1{%
  \HoLogoCss@XeTeX
  \HOLOGO@Span{XeTeX}{%
    \hologo{Xe}%
    \hologo{TeX}%
  }%
}
%    \end{macrocode}
%    \end{macro}
%    \begin{macro}{\HoLogoCss@XeTeX}
%    \begin{macrocode}
\def\HoLogoCss@XeTeX{%
  \Css{%
    span.HoLogo-XeTeX span.HoLogo-TeX{%
      margin-left:-.15em;%
    }%
  }%
  \global\let\HoLogoCss@XeTeX\relax
}
%    \end{macrocode}
%    \end{macro}
%
%    \begin{macro}{\HoLogo@XeLaTeX}
%    \begin{macrocode}
\def\HoLogo@XeLaTeX#1{%
  \hologo{Xe}%
  \kern-.13em%
  \hologo{LaTeX}%
}
%    \end{macrocode}
%    \end{macro}
%    \begin{macro}{\HoLogoHtml@XeLaTeX}
%    \begin{macrocode}
\def\HoLogoHtml@XeLaTeX#1{%
  \HoLogoCss@XeLaTeX
  \HOLOGO@Span{XeLaTeX}{%
    \hologo{Xe}%
    \hologo{LaTeX}%
  }%
}
%    \end{macrocode}
%    \end{macro}
%    \begin{macro}{\HoLogoCss@XeLaTeX}
%    \begin{macrocode}
\def\HoLogoCss@XeLaTeX{%
  \Css{%
    span.HoLogo-XeLaTeX span.HoLogo-Xe{%
      margin-right:-.13em;%
    }%
  }%
  \global\let\HoLogoCss@XeLaTeX\relax
}
%    \end{macrocode}
%    \end{macro}
%
% \subsubsection{\hologo{pdfTeX}, \hologo{pdfLaTeX}}
%
%    \begin{macro}{\HoLogo@pdfTeX}
%    \begin{macrocode}
\def\HoLogo@pdfTeX#1{%
  \HOLOGO@mbox{%
    #1{p}{P}df\hologo{TeX}%
  }%
}
%    \end{macrocode}
%    \end{macro}
%    \begin{macro}{\HoLogoCs@pdfTeX}
%    \begin{macrocode}
\def\HoLogoCs@pdfTeX#1{#1{p}{P}dfTeX}
%    \end{macrocode}
%    \end{macro}
%    \begin{macro}{\HoLogoBkm@pdfTeX}
%    \begin{macrocode}
\def\HoLogoBkm@pdfTeX#1{%
  #1{p}{P}df\hologo{TeX}%
}
%    \end{macrocode}
%    \end{macro}
%    \begin{macro}{\HoLogoHtml@pdfTeX}
%    \begin{macrocode}
\let\HoLogoHtml@pdfTeX\HoLogo@pdfTeX
%    \end{macrocode}
%    \end{macro}
%
%    \begin{macro}{\HoLogo@pdfLaTeX}
%    \begin{macrocode}
\def\HoLogo@pdfLaTeX#1{%
  \HOLOGO@mbox{%
    #1{p}{P}df\hologo{LaTeX}%
  }%
}
%    \end{macrocode}
%    \end{macro}
%    \begin{macro}{\HoLogoCs@pdfLaTeX}
%    \begin{macrocode}
\def\HoLogoCs@pdfLaTeX#1{#1{p}{P}dfLaTeX}
%    \end{macrocode}
%    \end{macro}
%    \begin{macro}{\HoLogoBkm@pdfLaTeX}
%    \begin{macrocode}
\def\HoLogoBkm@pdfLaTeX#1{%
  #1{p}{P}df\hologo{LaTeX}%
}
%    \end{macrocode}
%    \end{macro}
%    \begin{macro}{\HoLogoHtml@pdfLaTeX}
%    \begin{macrocode}
\let\HoLogoHtml@pdfLaTeX\HoLogo@pdfLaTeX
%    \end{macrocode}
%    \end{macro}
%
% \subsubsection{\hologo{VTeX}}
%
%    \begin{macro}{\HoLogo@VTeX}
%    \begin{macrocode}
\def\HoLogo@VTeX#1{%
  \HOLOGO@mbox{%
    V\hologo{TeX}%
  }%
}
%    \end{macrocode}
%    \end{macro}
%    \begin{macro}{\HoLogoHtml@VTeX}
%    \begin{macrocode}
\let\HoLogoHtml@VTeX\HoLogo@VTeX
%    \end{macrocode}
%    \end{macro}
%
% \subsubsection{\hologo{AmS}, \dots}
%
%    Source: class \xclass{amsdtx}
%
%    \begin{macro}{\HoLogo@AmS}
%    \begin{macrocode}
\def\HoLogo@AmS#1{%
  \HoLogoFont@font{AmS}{sy}{%
    A%
    \kern-.1667em%
    \lower.5ex\hbox{M}%
    \kern-.125em%
    S%
  }%
}
%    \end{macrocode}
%    \end{macro}
%    \begin{macro}{\HoLogoBkm@AmS}
%    \begin{macrocode}
\def\HoLogoBkm@AmS#1{AmS}
%    \end{macrocode}
%    \end{macro}
%    \begin{macro}{\HoLogoHtml@AmS}
%    \begin{macrocode}
\def\HoLogoHtml@AmS#1{%
  \HoLogoCss@AmS
%  \HoLogoFont@font{AmS}{sy}{%
    \HOLOGO@Span{AmS}{%
      A%
      \HOLOGO@Span{M}{M}%
      S%
    }%
%   }%
}
%    \end{macrocode}
%    \end{macro}
%    \begin{macro}{\HoLogoCss@AmS}
%    \begin{macrocode}
\def\HoLogoCss@AmS{%
  \Css{%
    span.HoLogo-AmS span.HoLogo-M{%
      position:relative;%
      top:.5ex;%
      margin-left:-.1667em;%
      margin-right:-.125em;%
      text-decoration:none;%
    }%
  }%
  \global\let\HoLogoCss@AmS\relax
}
%    \end{macrocode}
%    \end{macro}
%
%    \begin{macro}{\HoLogo@AmSTeX}
%    \begin{macrocode}
\def\HoLogo@AmSTeX#1{%
  \hologo{AmS}%
  \HOLOGO@hyphen
  \hologo{TeX}%
}
%    \end{macrocode}
%    \end{macro}
%    \begin{macro}{\HoLogoBkm@AmSTeX}
%    \begin{macrocode}
\def\HoLogoBkm@AmSTeX#1{AmS-TeX}%
%    \end{macrocode}
%    \end{macro}
%    \begin{macro}{\HoLogoHtml@AmSTeX}
%    \begin{macrocode}
\let\HoLogoHtml@AmSTeX\HoLogo@AmSTeX
%    \end{macrocode}
%    \end{macro}
%
%    \begin{macro}{\HoLogo@AmSLaTeX}
%    \begin{macrocode}
\def\HoLogo@AmSLaTeX#1{%
  \hologo{AmS}%
  \HOLOGO@hyphen
  \hologo{LaTeX}%
}
%    \end{macrocode}
%    \end{macro}
%    \begin{macro}{\HoLogoBkm@AmSLaTeX}
%    \begin{macrocode}
\def\HoLogoBkm@AmSLaTeX#1{AmS-LaTeX}%
%    \end{macrocode}
%    \end{macro}
%    \begin{macro}{\HoLogoHtml@AmSLaTeX}
%    \begin{macrocode}
\let\HoLogoHtml@AmSLaTeX\HoLogo@AmSLaTeX
%    \end{macrocode}
%    \end{macro}
%
% \subsubsection{\hologo{BibTeX}}
%
%    \begin{macro}{\HoLogo@BibTeX@sc}
%    A definition of \hologo{BibTeX} is provided in
%    the documentation source for the manual of \hologo{BibTeX}
%    \cite{btxdoc}.
%\begin{quote}
%\begin{verbatim}
%\def\BibTeX{%
%  {%
%    \rm
%    B%
%    \kern-.05em%
%    {%
%      \sc
%      i%
%      \kern-.025em %
%      b%
%    }%
%    \kern-.08em
%    T%
%    \kern-.1667em%
%    \lower.7ex\hbox{E}%
%    \kern-.125em%
%    X%
%  }%
%}
%\end{verbatim}
%\end{quote}
%    \begin{macrocode}
\def\HoLogo@BibTeX@sc#1{%
  B%
  \kern-.05em%
  \HoLogoFont@font{BibTeX}{sc}{%
    i%
    \kern-.025em%
    b%
  }%
  \HOLOGO@discretionary
  \kern-.08em%
  \hologo{TeX}%
}
%    \end{macrocode}
%    \end{macro}
%    \begin{macro}{\HoLogoHtml@BibTeX@sc}
%    \begin{macrocode}
\def\HoLogoHtml@BibTeX@sc#1{%
  \HoLogoCss@BibTeX@sc
  \HOLOGO@Span{BibTeX-sc}{%
    B%
    \HOLOGO@Span{i}{i}%
    \HOLOGO@Span{b}{b}%
    \hologo{TeX}%
  }%
}
%    \end{macrocode}
%    \end{macro}
%    \begin{macro}{\HoLogoCss@BibTeX@sc}
%    \begin{macrocode}
\def\HoLogoCss@BibTeX@sc{%
  \Css{%
    span.HoLogo-BibTeX-sc span.HoLogo-i{%
      margin-left:-.05em;%
      margin-right:-.025em;%
      font-variant:small-caps;%
    }%
  }%
  \Css{%
    span.HoLogo-BibTeX-sc span.HoLogo-b{%
      margin-right:-.08em;%
      font-variant:small-caps;%
    }%
  }%
  \global\let\HoLogoCss@BibTeX@sc\relax
}
%    \end{macrocode}
%    \end{macro}
%
%    \begin{macro}{\HoLogo@BibTeX@sf}
%    Variant \xoption{sf} avoids trouble with unavailable
%    small caps fonts (e.g., bold versions of Computer Modern or
%    Latin Modern). The definition is taken from
%    package \xpackage{dtklogos} \cite{dtklogos}.
%\begin{quote}
%\begin{verbatim}
%\DeclareRobustCommand{\BibTeX}{%
%  B%
%  \kern-.05em%
%  \hbox{%
%    $\m@th$% %% force math size calculations
%    \csname S@\f@size\endcsname
%    \fontsize\sf@size\z@
%    \math@fontsfalse
%    \selectfont
%    I%
%    \kern-.025em%
%    B
%  }%
%  \kern-.08em%
%  \-%
%  \TeX
%}
%\end{verbatim}
%\end{quote}
%    \begin{macrocode}
\def\HoLogo@BibTeX@sf#1{%
  B%
  \kern-.05em%
  \HoLogoFont@font{BibTeX}{bibsf}{%
    I%
    \kern-.025em%
    B%
  }%
  \HOLOGO@discretionary
  \kern-.08em%
  \hologo{TeX}%
}
%    \end{macrocode}
%    \end{macro}
%    \begin{macro}{\HoLogoHtml@BibTeX@sf}
%    \begin{macrocode}
\def\HoLogoHtml@BibTeX@sf#1{%
  \HoLogoCss@BibTeX@sf
  \HOLOGO@Span{BibTeX-sf}{%
    B%
    \HoLogoFont@font{BibTeX}{bibsf}{%
      \HOLOGO@Span{i}{I}%
      B%
    }%
    \hologo{TeX}%
  }%
}
%    \end{macrocode}
%    \end{macro}
%    \begin{macro}{\HoLogoCss@BibTeX@sf}
%    \begin{macrocode}
\def\HoLogoCss@BibTeX@sf{%
  \Css{%
    span.HoLogo-BibTeX-sf span.HoLogo-i{%
      margin-left:-.05em;%
      margin-right:-.025em;%
    }%
  }%
  \Css{%
    span.HoLogo-BibTeX-sf span.HoLogo-TeX{%
      margin-left:-.08em;%
    }%
  }%
  \global\let\HoLogoCss@BibTeX@sf\relax
}
%    \end{macrocode}
%    \end{macro}
%
%    \begin{macro}{\HoLogo@BibTeX}
%    \begin{macrocode}
\def\HoLogo@BibTeX{\HoLogo@BibTeX@sf}
%    \end{macrocode}
%    \end{macro}
%    \begin{macro}{\HoLogoHtml@BibTeX}
%    \begin{macrocode}
\def\HoLogoHtml@BibTeX{\HoLogoHtml@BibTeX@sf}
%    \end{macrocode}
%    \end{macro}
%
% \subsubsection{\hologo{BibTeX8}}
%
%    \begin{macro}{\HoLogo@BibTeX8}
%    \begin{macrocode}
\expandafter\def\csname HoLogo@BibTeX8\endcsname#1{%
  \hologo{BibTeX}%
  8%
}
%    \end{macrocode}
%    \end{macro}
%
%    \begin{macro}{\HoLogoBkm@BibTeX8}
%    \begin{macrocode}
\expandafter\def\csname HoLogoBkm@BibTeX8\endcsname#1{%
  \hologo{BibTeX}%
  8%
}
%    \end{macrocode}
%    \end{macro}
%    \begin{macro}{\HoLogoHtml@BibTeX8}
%    \begin{macrocode}
\expandafter
\let\csname HoLogoHtml@BibTeX8\expandafter\endcsname
\csname HoLogo@BibTeX8\endcsname
%    \end{macrocode}
%    \end{macro}
%
% \subsubsection{\hologo{ConTeXt}}
%
%    \begin{macro}{\HoLogo@ConTeXt@simple}
%    \begin{macrocode}
\def\HoLogo@ConTeXt@simple#1{%
  \HOLOGO@mbox{Con}%
  \HOLOGO@discretionary
  \HOLOGO@mbox{\hologo{TeX}t}%
}
%    \end{macrocode}
%    \end{macro}
%    \begin{macro}{\HoLogoHtml@ConTeXt@simple}
%    \begin{macrocode}
\let\HoLogoHtml@ConTeXt@simple\HoLogo@ConTeXt@simple
%    \end{macrocode}
%    \end{macro}
%
%    \begin{macro}{\HoLogo@ConTeXt@narrow}
%    This definition of logo \hologo{ConTeXt} with variant \xoption{narrow}
%    comes from TUGboat's class \xclass{ltugboat} (version 2010/11/15 v2.8).
%    \begin{macrocode}
\def\HoLogo@ConTeXt@narrow#1{%
  \HOLOGO@mbox{C\kern-.0333emon}%
  \HOLOGO@discretionary
  \kern-.0667em%
  \HOLOGO@mbox{\hologo{TeX}\kern-.0333emt}%
}
%    \end{macrocode}
%    \end{macro}
%    \begin{macro}{\HoLogoHtml@ConTeXt@narrow}
%    \begin{macrocode}
\def\HoLogoHtml@ConTeXt@narrow#1{%
  \HoLogoCss@ConTeXt@narrow
  \HOLOGO@Span{ConTeXt-narrow}{%
    \HOLOGO@Span{C}{C}%
    on%
    \hologo{TeX}%
    t%
  }%
}
%    \end{macrocode}
%    \end{macro}
%    \begin{macro}{\HoLogoCss@ConTeXt@narrow}
%    \begin{macrocode}
\def\HoLogoCss@ConTeXt@narrow{%
  \Css{%
    span.HoLogo-ConTeXt-narrow span.HoLogo-C{%
      margin-left:-.0333em;%
    }%
  }%
  \Css{%
    span.HoLogo-ConTeXt-narrow span.HoLogo-TeX{%
      margin-left:-.0667em;%
      margin-right:-.0333em;%
    }%
  }%
  \global\let\HoLogoCss@ConTeXt@narrow\relax
}
%    \end{macrocode}
%    \end{macro}
%
%    \begin{macro}{\HoLogo@ConTeXt}
%    \begin{macrocode}
\def\HoLogo@ConTeXt{\HoLogo@ConTeXt@narrow}
%    \end{macrocode}
%    \end{macro}
%    \begin{macro}{\HoLogoHtml@ConTeXt}
%    \begin{macrocode}
\def\HoLogoHtml@ConTeXt{\HoLogoHtml@ConTeXt@narrow}
%    \end{macrocode}
%    \end{macro}
%
% \subsubsection{\hologo{emTeX}}
%
%    \begin{macro}{\HoLogo@emTeX}
%    \begin{macrocode}
\def\HoLogo@emTeX#1{%
  \HOLOGO@mbox{#1{e}{E}m}%
  \HOLOGO@discretionary
  \hologo{TeX}%
}
%    \end{macrocode}
%    \end{macro}
%    \begin{macro}{\HoLogoCs@emTeX}
%    \begin{macrocode}
\def\HoLogoCs@emTeX#1{#1{e}{E}mTeX}%
%    \end{macrocode}
%    \end{macro}
%    \begin{macro}{\HoLogoBkm@emTeX}
%    \begin{macrocode}
\def\HoLogoBkm@emTeX#1{%
  #1{e}{E}m\hologo{TeX}%
}
%    \end{macrocode}
%    \end{macro}
%    \begin{macro}{\HoLogoHtml@emTeX}
%    \begin{macrocode}
\let\HoLogoHtml@emTeX\HoLogo@emTeX
%    \end{macrocode}
%    \end{macro}
%
% \subsubsection{\hologo{ExTeX}}
%
%    \begin{macro}{\HoLogo@ExTeX}
%    The definition is taken from the FAQ of the
%    project \hologo{ExTeX}
%    \cite{ExTeX-FAQ}.
%\begin{quote}
%\begin{verbatim}
%\def\ExTeX{%
%  \textrm{% Logo always with serifs
%    \ensuremath{%
%      \textstyle
%      \varepsilon_{%
%        \kern-0.15em%
%        \mathcal{X}%
%      }%
%    }%
%    \kern-.15em%
%    \TeX
%  }%
%}
%\end{verbatim}
%\end{quote}
%    \begin{macrocode}
\def\HoLogo@ExTeX#1{%
  \HoLogoFont@font{ExTeX}{rm}{%
    \ltx@mbox{%
      \HOLOGO@MathSetup
      $%
        \textstyle
        \varepsilon_{%
          \kern-0.15em%
          \HoLogoFont@font{ExTeX}{sy}{X}%
        }%
      $%
    }%
    \HOLOGO@discretionary
    \kern-.15em%
    \hologo{TeX}%
  }%
}
%    \end{macrocode}
%    \end{macro}
%    \begin{macro}{\HoLogoHtml@ExTeX}
%    \begin{macrocode}
\def\HoLogoHtml@ExTeX#1{%
  \HoLogoCss@ExTeX
  \HoLogoFont@font{ExTeX}{rm}{%
    \HOLOGO@Span{ExTeX}{%
      \ltx@mbox{%
        \HOLOGO@MathSetup
        $\textstyle\varepsilon$%
        \HOLOGO@Span{X}{$\textstyle\chi$}%
        \hologo{TeX}%
      }%
    }%
  }%
}
%    \end{macrocode}
%    \end{macro}
%    \begin{macro}{\HoLogoBkm@ExTeX}
%    \begin{macrocode}
\def\HoLogoBkm@ExTeX#1{%
  \HOLOGO@PdfdocUnicode{#1{e}{E}x}{\textepsilon\textchi}%
  \hologo{TeX}%
}
%    \end{macrocode}
%    \end{macro}
%    \begin{macro}{\HoLogoCss@ExTeX}
%    \begin{macrocode}
\def\HoLogoCss@ExTeX{%
  \Css{%
    span.HoLogo-ExTeX{%
      font-family:serif;%
    }%
  }%
  \Css{%
    span.HoLogo-ExTeX span.HoLogo-TeX{%
      margin-left:-.15em;%
    }%
  }%
  \global\let\HoLogoCss@ExTeX\relax
}
%    \end{macrocode}
%    \end{macro}
%
% \subsubsection{\hologo{MiKTeX}}
%
%    \begin{macro}{\HoLogo@MiKTeX}
%    \begin{macrocode}
\def\HoLogo@MiKTeX#1{%
  \HOLOGO@mbox{MiK}%
  \HOLOGO@discretionary
  \hologo{TeX}%
}
%    \end{macrocode}
%    \end{macro}
%    \begin{macro}{\HoLogoHtml@MiKTeX}
%    \begin{macrocode}
\let\HoLogoHtml@MiKTeX\HoLogo@MiKTeX
%    \end{macrocode}
%    \end{macro}
%
% \subsubsection{\hologo{OzTeX} and friends}
%
%    Source: \hologo{OzTeX} FAQ \cite{OzTeX}:
%    \begin{quote}
%      |\def\OzTeX{O\kern-.03em z\kern-.15em\TeX}|\\
%      (There is no kerning in OzMF, OzMP and OzTtH.)
%    \end{quote}
%
%    \begin{macro}{\HoLogo@OzTeX}
%    \begin{macrocode}
\def\HoLogo@OzTeX#1{%
  O%
  \kern-.03em %
  z%
  \kern-.15em %
  \hologo{TeX}%
}
%    \end{macrocode}
%    \end{macro}
%    \begin{macro}{\HoLogoHtml@OzTeX}
%    \begin{macrocode}
\def\HoLogoHtml@OzTeX#1{%
  \HoLogoCss@OzTeX
  \HOLOGO@Span{OzTeX}{%
    O%
    \HOLOGO@Span{z}{z}%
    \hologo{TeX}%
  }%
}
%    \end{macrocode}
%    \end{macro}
%    \begin{macro}{\HoLogoCss@OzTeX}
%    \begin{macrocode}
\def\HoLogoCss@OzTeX{%
  \Css{%
    span.HoLogo-OzTeX span.HoLogo-z{%
      margin-left:-.03em;%
      margin-right:-.15em;%
    }%
  }%
  \global\let\HoLogoCss@OzTeX\relax
}
%    \end{macrocode}
%    \end{macro}
%
%    \begin{macro}{\HoLogo@OzMF}
%    \begin{macrocode}
\def\HoLogo@OzMF#1{%
  \HOLOGO@mbox{OzMF}%
}
%    \end{macrocode}
%    \end{macro}
%    \begin{macro}{\HoLogo@OzMP}
%    \begin{macrocode}
\def\HoLogo@OzMP#1{%
  \HOLOGO@mbox{OzMP}%
}
%    \end{macrocode}
%    \end{macro}
%    \begin{macro}{\HoLogo@OzTtH}
%    \begin{macrocode}
\def\HoLogo@OzTtH#1{%
  \HOLOGO@mbox{OzTtH}%
}
%    \end{macrocode}
%    \end{macro}
%
% \subsubsection{\hologo{PCTeX}}
%
%    \begin{macro}{\HoLogo@PCTeX}
%    \begin{macrocode}
\def\HoLogo@PCTeX#1{%
  \HOLOGO@mbox{PC}%
  \hologo{TeX}%
}
%    \end{macrocode}
%    \end{macro}
%    \begin{macro}{\HoLogoHtml@PCTeX}
%    \begin{macrocode}
\let\HoLogoHtml@PCTeX\HoLogo@PCTeX
%    \end{macrocode}
%    \end{macro}
%
% \subsubsection{\hologo{PiCTeX}}
%
%    The original definitions from \xfile{pictex.tex} \cite{PiCTeX}:
%\begin{quote}
%\begin{verbatim}
%\def\PiC{%
%  P%
%  \kern-.12em%
%  \lower.5ex\hbox{I}%
%  \kern-.075em%
%  C%
%}
%\def\PiCTeX{%
%  \PiC
%  \kern-.11em%
%  \TeX
%}
%\end{verbatim}
%\end{quote}
%
%    \begin{macro}{\HoLogo@PiC}
%    \begin{macrocode}
\def\HoLogo@PiC#1{%
  P%
  \kern-.12em%
  \lower.5ex\hbox{I}%
  \kern-.075em%
  C%
  \HOLOGO@SpaceFactor
}
%    \end{macrocode}
%    \end{macro}
%    \begin{macro}{\HoLogoHtml@PiC}
%    \begin{macrocode}
\def\HoLogoHtml@PiC#1{%
  \HoLogoCss@PiC
  \HOLOGO@Span{PiC}{%
    P%
    \HOLOGO@Span{i}{I}%
    C%
  }%
}
%    \end{macrocode}
%    \end{macro}
%    \begin{macro}{\HoLogoCss@PiC}
%    \begin{macrocode}
\def\HoLogoCss@PiC{%
  \Css{%
    span.HoLogo-PiC span.HoLogo-i{%
      position:relative;%
      top:.5ex;%
      margin-left:-.12em;%
      margin-right:-.075em;%
      text-decoration:none;%
    }%
  }%
  \global\let\HoLogoCss@PiC\relax
}
%    \end{macrocode}
%    \end{macro}
%
%    \begin{macro}{\HoLogo@PiCTeX}
%    \begin{macrocode}
\def\HoLogo@PiCTeX#1{%
  \hologo{PiC}%
  \HOLOGO@discretionary
  \kern-.11em%
  \hologo{TeX}%
}
%    \end{macrocode}
%    \end{macro}
%    \begin{macro}{\HoLogoHtml@PiCTeX}
%    \begin{macrocode}
\def\HoLogoHtml@PiCTeX#1{%
  \HoLogoCss@PiCTeX
  \HOLOGO@Span{PiCTeX}{%
    \hologo{PiC}%
    \hologo{TeX}%
  }%
}
%    \end{macrocode}
%    \end{macro}
%    \begin{macro}{\HoLogoCss@PiCTeX}
%    \begin{macrocode}
\def\HoLogoCss@PiCTeX{%
  \Css{%
    span.HoLogo-PiCTeX span.HoLogo-PiC{%
      margin-right:-.11em;%
    }%
  }%
  \global\let\HoLogoCss@PiCTeX\relax
}
%    \end{macrocode}
%    \end{macro}
%
% \subsubsection{\hologo{teTeX}}
%
%    \begin{macro}{\HoLogo@teTeX}
%    \begin{macrocode}
\def\HoLogo@teTeX#1{%
  \HOLOGO@mbox{#1{t}{T}e}%
  \HOLOGO@discretionary
  \hologo{TeX}%
}
%    \end{macrocode}
%    \end{macro}
%    \begin{macro}{\HoLogoCs@teTeX}
%    \begin{macrocode}
\def\HoLogoCs@teTeX#1{#1{t}{T}dfTeX}
%    \end{macrocode}
%    \end{macro}
%    \begin{macro}{\HoLogoBkm@teTeX}
%    \begin{macrocode}
\def\HoLogoBkm@teTeX#1{%
  #1{t}{T}e\hologo{TeX}%
}
%    \end{macrocode}
%    \end{macro}
%    \begin{macro}{\HoLogoHtml@teTeX}
%    \begin{macrocode}
\let\HoLogoHtml@teTeX\HoLogo@teTeX
%    \end{macrocode}
%    \end{macro}
%
% \subsubsection{\hologo{TeX4ht}}
%
%    \begin{macro}{\HoLogo@TeX4ht}
%    \begin{macrocode}
\expandafter\def\csname HoLogo@TeX4ht\endcsname#1{%
  \HOLOGO@mbox{\hologo{TeX}4ht}%
}
%    \end{macrocode}
%    \end{macro}
%    \begin{macro}{\HoLogoHtml@TeX4ht}
%    \begin{macrocode}
\expandafter
\let\csname HoLogoHtml@TeX4ht\expandafter\endcsname
\csname HoLogo@TeX4ht\endcsname
%    \end{macrocode}
%    \end{macro}
%
%
% \subsubsection{\hologo{SageTeX}}
%
%    \begin{macro}{\HoLogo@SageTeX}
%    \begin{macrocode}
\def\HoLogo@SageTeX#1{%
  \HOLOGO@mbox{Sage}%
  \HOLOGO@discretionary
  \HOLOGO@NegativeKerning{eT,oT,To}%
  \hologo{TeX}%
}
%    \end{macrocode}
%    \end{macro}
%    \begin{macro}{\HoLogoHtml@SageTeX}
%    \begin{macrocode}
\let\HoLogoHtml@SageTeX\HoLogo@SageTeX
%    \end{macrocode}
%    \end{macro}
%
% \subsection{\hologo{METAFONT} and friends}
%
%    \begin{macro}{\HoLogo@METAFONT}
%    \begin{macrocode}
\def\HoLogo@METAFONT#1{%
  \HoLogoFont@font{METAFONT}{logo}{%
    \HOLOGO@mbox{META}%
    \HOLOGO@discretionary
    \HOLOGO@mbox{FONT}%
  }%
}
%    \end{macrocode}
%    \end{macro}
%
%    \begin{macro}{\HoLogo@METAPOST}
%    \begin{macrocode}
\def\HoLogo@METAPOST#1{%
  \HoLogoFont@font{METAPOST}{logo}{%
    \HOLOGO@mbox{META}%
    \HOLOGO@discretionary
    \HOLOGO@mbox{POST}%
  }%
}
%    \end{macrocode}
%    \end{macro}
%
%    \begin{macro}{\HoLogo@MetaFun}
%    \begin{macrocode}
\def\HoLogo@MetaFun#1{%
  \HOLOGO@mbox{Meta}%
  \HOLOGO@discretionary
  \HOLOGO@mbox{Fun}%
}
%    \end{macrocode}
%    \end{macro}
%
%    \begin{macro}{\HoLogo@MetaPost}
%    \begin{macrocode}
\def\HoLogo@MetaPost#1{%
  \HOLOGO@mbox{Meta}%
  \HOLOGO@discretionary
  \HOLOGO@mbox{Post}%
}
%    \end{macrocode}
%    \end{macro}
%
% \subsection{Others}
%
% \subsubsection{\hologo{biber}}
%
%    \begin{macro}{\HoLogo@biber}
%    \begin{macrocode}
\def\HoLogo@biber#1{%
  \HOLOGO@mbox{#1{b}{B}i}%
  \HOLOGO@discretionary
  \HOLOGO@mbox{ber}%
}
%    \end{macrocode}
%    \end{macro}
%    \begin{macro}{\HoLogoCs@biber}
%    \begin{macrocode}
\def\HoLogoCs@biber#1{#1{b}{B}iber}
%    \end{macrocode}
%    \end{macro}
%    \begin{macro}{\HoLogoBkm@biber}
%    \begin{macrocode}
\def\HoLogoBkm@biber#1{%
  #1{b}{B}iber%
}
%    \end{macrocode}
%    \end{macro}
%    \begin{macro}{\HoLogoHtml@biber}
%    \begin{macrocode}
\let\HoLogoHtml@biber\HoLogo@biber
%    \end{macrocode}
%    \end{macro}
%
% \subsubsection{\hologo{KOMAScript}}
%
%    \begin{macro}{\HoLogo@KOMAScript}
%    The definition for \hologo{KOMAScript} is taken
%    from \hologo{KOMAScript} (\xfile{scrlogo.dtx}, reformatted) \cite{scrlogo}:
%\begin{quote}
%\begin{verbatim}
%\@ifundefined{KOMAScript}{%
%  \DeclareRobustCommand{\KOMAScript}{%
%    \textsf{%
%      K\kern.05em O\kern.05emM\kern.05em A%
%      \kern.1em-\kern.1em %
%      Script%
%    }%
%  }%
%}{}
%\end{verbatim}
%\end{quote}
%    \begin{macrocode}
\def\HoLogo@KOMAScript#1{%
  \HoLogoFont@font{KOMAScript}{sf}{%
    \HOLOGO@mbox{%
      K\kern.05em%
      O\kern.05em%
      M\kern.05em%
      A%
    }%
    \kern.1em%
    \HOLOGO@hyphen
    \kern.1em%
    \HOLOGO@mbox{Script}%
  }%
}
%    \end{macrocode}
%    \end{macro}
%    \begin{macro}{\HoLogoBkm@KOMAScript}
%    \begin{macrocode}
\def\HoLogoBkm@KOMAScript#1{%
  KOMA-Script%
}
%    \end{macrocode}
%    \end{macro}
%    \begin{macro}{\HoLogoHtml@KOMAScript}
%    \begin{macrocode}
\def\HoLogoHtml@KOMAScript#1{%
  \HoLogoCss@KOMAScript
  \HoLogoFont@font{KOMAScript}{sf}{%
    \HOLOGO@Span{KOMAScript}{%
      K%
      \HOLOGO@Span{O}{O}%
      M%
      \HOLOGO@Span{A}{A}%
      \HOLOGO@Span{hyphen}{-}%
      Script%
    }%
  }%
}
%    \end{macrocode}
%    \end{macro}
%    \begin{macro}{\HoLogoCss@KOMAScript}
%    \begin{macrocode}
\def\HoLogoCss@KOMAScript{%
  \Css{%
    span.HoLogo-KOMAScript{%
      font-family:sans-serif;%
    }%
  }%
  \Css{%
    span.HoLogo-KOMAScript span.HoLogo-O{%
      padding-left:.05em;%
      padding-right:.05em;%
    }%
  }%
  \Css{%
    span.HoLogo-KOMAScript span.HoLogo-A{%
      padding-left:.05em;%
    }%
  }%
  \Css{%
    span.HoLogo-KOMAScript span.HoLogo-hyphen{%
      padding-left:.1em;%
      padding-right:.1em;%
    }%
  }%
  \global\let\HoLogoCss@KOMAScript\relax
}
%    \end{macrocode}
%    \end{macro}
%
% \subsubsection{\hologo{LyX}}
%
%    \begin{macro}{\HoLogo@LyX}
%    The definition is taken from the documentation source files
%    of \hologo{LyX}, \xfile{Intro.lyx} \cite{LyX}:
%\begin{quote}
%\begin{verbatim}
%\def\LyX{%
%  \texorpdfstring{%
%    L\kern-.1667em\lower.25em\hbox{Y}\kern-.125emX\@%
%  }{%
%    LyX%
%  }%
%}
%\end{verbatim}
%\end{quote}
%    \begin{macrocode}
\def\HoLogo@LyX#1{%
  L%
  \kern-.1667em%
  \lower.25em\hbox{Y}%
  \kern-.125em%
  X%
  \HOLOGO@SpaceFactor
}
%    \end{macrocode}
%    \end{macro}
%    \begin{macro}{\HoLogoHtml@LyX}
%    \begin{macrocode}
\def\HoLogoHtml@LyX#1{%
  \HoLogoCss@LyX
  \HOLOGO@Span{LyX}{%
    L%
    \HOLOGO@Span{y}{Y}%
    X%
  }%
}
%    \end{macrocode}
%    \end{macro}
%    \begin{macro}{\HoLogoCss@LyX}
%    \begin{macrocode}
\def\HoLogoCss@LyX{%
  \Css{%
    span.HoLogo-LyX span.HoLogo-y{%
      position:relative;%
      top:.25em;%
      margin-left:-.1667em;%
      margin-right:-.125em;%
      text-decoration:none;%
    }%
  }%
  \global\let\HoLogoCss@LyX\relax
}
%    \end{macrocode}
%    \end{macro}
%
% \subsubsection{\hologo{NTS}}
%
%    \begin{macro}{\HoLogo@NTS}
%    Definition for \hologo{NTS} can be found in
%    package \xpackage{etex\textunderscore man} for the \hologo{eTeX} manual \cite{etexman}
%    and in package \xpackage{dtklogos} \cite{dtklogos}:
%\begin{quote}
%\begin{verbatim}
%\def\NTS{%
%  \leavevmode
%  \hbox{%
%    $%
%      \cal N%
%      \kern-0.35em%
%      \lower0.5ex\hbox{$\cal T$}%
%      \kern-0.2em%
%      S%
%    $%
%  }%
%}
%\end{verbatim}
%\end{quote}
%    \begin{macrocode}
\def\HoLogo@NTS#1{%
  \HoLogoFont@font{NTS}{sy}{%
    N\/%
    \kern-.35em%
    \lower.5ex\hbox{T\/}%
    \kern-.2em%
    S\/%
  }%
  \HOLOGO@SpaceFactor
}
%    \end{macrocode}
%    \end{macro}
%
% \subsubsection{\Hologo{TTH} (\hologo{TeX} to HTML translator)}
%
%    Source: \url{http://hutchinson.belmont.ma.us/tth/}
%    In the HTML source the second `T' is printed as subscript.
%\begin{quote}
%\begin{verbatim}
%T<sub>T</sub>H
%\end{verbatim}
%\end{quote}
%    \begin{macro}{\HoLogo@TTH}
%    \begin{macrocode}
\def\HoLogo@TTH#1{%
  \ltx@mbox{%
    T\HOLOGO@SubScript{T}H%
  }%
  \HOLOGO@SpaceFactor
}
%    \end{macrocode}
%    \end{macro}
%
%    \begin{macro}{\HoLogoHtml@TTH}
%    \begin{macrocode}
\def\HoLogoHtml@TTH#1{%
  T\HCode{<sub>}T\HCode{</sub>}H%
}
%    \end{macrocode}
%    \end{macro}
%
% \subsubsection{\Hologo{HanTheThanh}}
%
%    Partial source: Package \xpackage{dtklogos}.
%    The double accent is U+1EBF (latin small letter e with circumflex
%    and acute).
%    \begin{macro}{\HoLogo@HanTheThanh}
%    \begin{macrocode}
\def\HoLogo@HanTheThanh#1{%
  \ltx@mbox{H\`an}%
  \HOLOGO@space
  \ltx@mbox{%
    Th%
    \HOLOGO@IfCharExists{"1EBF}{%
      \char"1EBF\relax
    }{%
      \^e\hbox to 0pt{\hss\raise .5ex\hbox{\'{}}}%
    }%
  }%
  \HOLOGO@space
  \ltx@mbox{Th\`anh}%
}
%    \end{macrocode}
%    \end{macro}
%    \begin{macro}{\HoLogoBkm@HanTheThanh}
%    \begin{macrocode}
\def\HoLogoBkm@HanTheThanh#1{%
  H\`an %
  Th\HOLOGO@PdfdocUnicode{\^e}{\9036\277} %
  Th\`anh%
}
%    \end{macrocode}
%    \end{macro}
%    \begin{macro}{\HoLogoHtml@HanTheThanh}
%    \begin{macrocode}
\def\HoLogoHtml@HanTheThanh#1{%
  H\`an %
  Th\HCode{&\ltx@hashchar x1ebf;} %
  Th\`anh%
}
%    \end{macrocode}
%    \end{macro}
%
% \subsection{Driver detection}
%
%    \begin{macrocode}
\HOLOGO@IfExists\InputIfFileExists{%
  \InputIfFileExists{hologo.cfg}{}{}%
}{%
  \ltx@IfUndefined{pdf@filesize}{%
    \def\HOLOGO@InputIfExists{%
      \openin\HOLOGO@temp=hologo.cfg\relax
      \ifeof\HOLOGO@temp
        \closein\HOLOGO@temp
      \else
        \closein\HOLOGO@temp
        \begingroup
          \def\x{LaTeX2e}%
        \expandafter\endgroup
        \ifx\fmtname\x
          % \iffalse meta-comment
%
% File: hologo.dtx
% Version: 2016/05/12 v1.11
% Info: A logo collection with bookmark support
%
% Copyright (C) 2010-2012 by
%    Heiko Oberdiek <heiko.oberdiek at googlemail.com>
%
% This work may be distributed and/or modified under the
% conditions of the LaTeX Project Public License, either
% version 1.3c of this license or (at your option) any later
% version. This version of this license is in
%    http://www.latex-project.org/lppl/lppl-1-3c.txt
% and the latest version of this license is in
%    http://www.latex-project.org/lppl.txt
% and version 1.3 or later is part of all distributions of
% LaTeX version 2005/12/01 or later.
%
% This work has the LPPL maintenance status "maintained".
%
% This Current Maintainer of this work is Heiko Oberdiek.
%
% The Base Interpreter refers to any `TeX-Format',
% because some files are installed in TDS:tex/generic//.
%
% This work consists of the main source file hologo.dtx
% and the derived files
%    hologo.sty, hologo.pdf, hologo.ins, hologo.drv, hologo-example.tex,
%    hologo-test1.tex, hologo-test-spacefactor.tex,
%    hologo-test-list.tex.
%
% Distribution:
%    CTAN:macros/latex/contrib/oberdiek/hologo.dtx
%    CTAN:macros/latex/contrib/oberdiek/hologo.pdf
%
% Unpacking:
%    (a) If hologo.ins is present:
%           tex hologo.ins
%    (b) Without hologo.ins:
%           tex hologo.dtx
%    (c) If you insist on using LaTeX
%           latex \let\install=y\input{hologo.dtx}
%        (quote the arguments according to the demands of your shell)
%
% Documentation:
%    (a) If hologo.drv is present:
%           latex hologo.drv
%    (b) Without hologo.drv:
%           latex hologo.dtx; ...
%    The class ltxdoc loads the configuration file ltxdoc.cfg
%    if available. Here you can specify further options, e.g.
%    use A4 as paper format:
%       \PassOptionsToClass{a4paper}{article}
%
%    Programm calls to get the documentation (example):
%       pdflatex hologo.dtx
%       makeindex -s gind.ist hologo.idx
%       pdflatex hologo.dtx
%       makeindex -s gind.ist hologo.idx
%       pdflatex hologo.dtx
%
% Installation:
%    TDS:tex/generic/oberdiek/hologo.sty
%    TDS:doc/latex/oberdiek/hologo.pdf
%    TDS:doc/latex/oberdiek/example/hologo-example.tex
%    TDS:doc/latex/oberdiek/test/hologo-test1.tex
%    TDS:doc/latex/oberdiek/test/hologo-test-spacefactor.tex
%    TDS:doc/latex/oberdiek/test/hologo-test-list.tex
%    TDS:source/latex/oberdiek/hologo.dtx
%
%<*ignore>
\begingroup
  \catcode123=1 %
  \catcode125=2 %
  \def\x{LaTeX2e}%
\expandafter\endgroup
\ifcase 0\ifx\install y1\fi\expandafter
         \ifx\csname processbatchFile\endcsname\relax\else1\fi
         \ifx\fmtname\x\else 1\fi\relax
\else\csname fi\endcsname
%</ignore>
%<*install>
\input docstrip.tex
\Msg{************************************************************************}
\Msg{* Installation}
\Msg{* Package: hologo 2016/05/12 v1.11 A logo collection with bookmark support (HO)}
\Msg{************************************************************************}

\keepsilent
\askforoverwritefalse

\let\MetaPrefix\relax
\preamble

This is a generated file.

Project: hologo
Version: 2016/05/12 v1.11

Copyright (C) 2010-2012 by
   Heiko Oberdiek <heiko.oberdiek at googlemail.com>

This work may be distributed and/or modified under the
conditions of the LaTeX Project Public License, either
version 1.3c of this license or (at your option) any later
version. This version of this license is in
   http://www.latex-project.org/lppl/lppl-1-3c.txt
and the latest version of this license is in
   http://www.latex-project.org/lppl.txt
and version 1.3 or later is part of all distributions of
LaTeX version 2005/12/01 or later.

This work has the LPPL maintenance status "maintained".

This Current Maintainer of this work is Heiko Oberdiek.

The Base Interpreter refers to any `TeX-Format',
because some files are installed in TDS:tex/generic//.

This work consists of the main source file hologo.dtx
and the derived files
   hologo.sty, hologo.pdf, hologo.ins, hologo.drv, hologo-example.tex,
   hologo-test1.tex, hologo-test-spacefactor.tex,
   hologo-test-list.tex.

\endpreamble
\let\MetaPrefix\DoubleperCent

\generate{%
  \file{hologo.ins}{\from{hologo.dtx}{install}}%
  \file{hologo.drv}{\from{hologo.dtx}{driver}}%
  \usedir{tex/generic/oberdiek}%
  \file{hologo.sty}{\from{hologo.dtx}{package}}%
  \usedir{doc/latex/oberdiek/example}%
  \file{hologo-example.tex}{\from{hologo.dtx}{example}}%
  \usedir{doc/latex/oberdiek/test}%
  \file{hologo-test1.tex}{\from{hologo.dtx}{test1}}%
  \file{hologo-test-spacefactor.tex}{\from{hologo.dtx}{test-spacefactor}}%
  \file{hologo-test-list.tex}{\from{hologo.dtx}{test-list}}%
  \nopreamble
  \nopostamble
  \usedir{source/latex/oberdiek/catalogue}%
  \file{hologo.xml}{\from{hologo.dtx}{catalogue}}%
}

\catcode32=13\relax% active space
\let =\space%
\Msg{************************************************************************}
\Msg{*}
\Msg{* To finish the installation you have to move the following}
\Msg{* file into a directory searched by TeX:}
\Msg{*}
\Msg{*     hologo.sty}
\Msg{*}
\Msg{* To produce the documentation run the file `hologo.drv'}
\Msg{* through LaTeX.}
\Msg{*}
\Msg{* Happy TeXing!}
\Msg{*}
\Msg{************************************************************************}

\endbatchfile
%</install>
%<*ignore>
\fi
%</ignore>
%<*driver>
\NeedsTeXFormat{LaTeX2e}
\ProvidesFile{hologo.drv}%
  [2016/05/12 v1.11 A logo collection with bookmark support (HO)]%
\documentclass{ltxdoc}
\usepackage{holtxdoc}[2011/11/22]
\usepackage{hologo}[2016/05/12]
\usepackage{longtable}
\usepackage{array}
\usepackage{paralist}
%\usepackage[T1]{fontenc}
%\usepackage{lmodern}
\begin{document}
  \DocInput{hologo.dtx}%
\end{document}
%</driver>
% \fi
%
%
% \CharacterTable
%  {Upper-case    \A\B\C\D\E\F\G\H\I\J\K\L\M\N\O\P\Q\R\S\T\U\V\W\X\Y\Z
%   Lower-case    \a\b\c\d\e\f\g\h\i\j\k\l\m\n\o\p\q\r\s\t\u\v\w\x\y\z
%   Digits        \0\1\2\3\4\5\6\7\8\9
%   Exclamation   \!     Double quote  \"     Hash (number) \#
%   Dollar        \$     Percent       \%     Ampersand     \&
%   Acute accent  \'     Left paren    \(     Right paren   \)
%   Asterisk      \*     Plus          \+     Comma         \,
%   Minus         \-     Point         \.     Solidus       \/
%   Colon         \:     Semicolon     \;     Less than     \<
%   Equals        \=     Greater than  \>     Question mark \?
%   Commercial at \@     Left bracket  \[     Backslash     \\
%   Right bracket \]     Circumflex    \^     Underscore    \_
%   Grave accent  \`     Left brace    \{     Vertical bar  \|
%   Right brace   \}     Tilde         \~}
%
% \GetFileInfo{hologo.drv}
%
% \title{The \xpackage{hologo} package}
% \date{2016/05/12 v1.11}
% \author{Heiko Oberdiek\\\xemail{heiko.oberdiek at googlemail.com}}
%
% \maketitle
%
% \begin{abstract}
% This package starts a collection of logos with support for bookmarks
% strings.
% \end{abstract}
%
% \tableofcontents
%
% \section{Documentation}
%
% \subsection{Logo macros}
%
% \begin{declcs}{hologo} \M{name}
% \end{declcs}
% Macro \cs{hologo} sets the logo with name \meta{name}.
% The following table shows the supported names.
%
% \begingroup
%   \def\hologoEntry#1#2#3{^^A
%     #1&#2&\hologoLogoSetup{#1}{variant=#2}\hologo{#1}&#3\tabularnewline
%   }
%   \begin{longtable}{>{\ttfamily}l>{\ttfamily}lll}
%     \rmfamily\bfseries{name} & \rmfamily\bfseries variant
%     & \bfseries logo & \bfseries since\\
%     \hline
%     \endhead
%     \hologoList
%   \end{longtable}
% \endgroup
%
% \begin{declcs}{Hologo} \M{name}
% \end{declcs}
% Macro \cs{Hologo} starts the logo \meta{name} with an uppercase
% letter. As an exception small greek letters are not converted
% to uppercase. Examples, see \hologo{eTeX} and \hologo{ExTeX}.
%
% \subsection{Setup macros}
%
% The package does not support package options, but the following
% setup macros can be used to set options.
%
% \begin{declcs}{hologoSetup} \M{key value list}
% \end{declcs}
% Macro \cs{hologoSetup} sets global options.
%
% \begin{declcs}{hologoLogoSetup} \M{logo} \M{key value list}
% \end{declcs}
% Some options can also be used to configure a logo.
% These settings take precedence over global option settings.
%
% \subsection{Options}\label{sec:options}
%
% There are boolean and string options:
% \begin{description}
% \item[Boolean option:]
% It takes |true| or |false|
% as value. If the value is omitted, then |true| is used.
% \item[String option:]
% A value must be given as string. (But the string might be empty.)
% \end{description}
% The following options can be used both in \cs{hologoSetup}
% and \cs{hologoLogoSetup}:
% \begin{description}
% \def\entry#1{\item[\xoption{#1}:]}
% \entry{break}
%   enables or disables line breaks inside the logo. This setting is
%   refined by options \xoption{hyphenbreak}, \xoption{spacebreak}
%   or \xoption{discretionarybreak}.
%   Default is |false|.
% \entry{hyphenbreak}
%   enables or disables the line break right after the hyphen character.
% \entry{spacebreak}
%   enables or disables line breaks at space characters.
% \entry{discretionarybreak}
%   enables or disables line breaks at hyphenation points
%   (inserted by \cs{-}).
% \end{description}
% Macro \cs{hologoLogoSetup} also knows:
% \begin{description}
% \item[\xoption{variant}:]
%   This is a string option. It specifies a variant of a logo that
%   must exist. An empty string selects the package default variant.
% \end{description}
% Example:
% \begin{quote}
%   |\hologoSetup{break=false}|\\
%   |\hologoLogoSetup{plainTeX}{variant=hyphen,hyphenbreak}|\\
%   Then ``plain-\TeX'' contains one break point after the hyphen.
% \end{quote}
%
% \subsection{Driver options}
%
% Sometimes graphical operations are needed to construct some
% glyphs (e.g.\ \hologo{XeTeX}). If package \xpackage{graphics}
% or package \xpackage{pgf} are found, then the macros are taken
% from there. Otherwise the packge defines its own operations
% and therefore needs the driver information. Many drivers are
% detected automatically (\hologo{pdfTeX}/\hologo{LuaTeX}
% in PDF mode, \hologo{XeTeX}, \hologo{VTeX}). These have precedence
% over a driver option. The driver can be given as package option
% or using \cs{hologoDriverSetup}.
% The following list contains the recognized driver options:
% \begin{itemize}
% \item \xoption{pdftex}, \xoption{luatex}
% \item \xoption{dvipdfm}, \xoption{dvipdfmx}
% \item \xoption{dvips}, \xoption{dvipsone}, \xoption{xdvi}
% \item \xoption{xetex}
% \item \xoption{vtex}
% \end{itemize}
% The left driver of a line is the driver name that is used internally.
% The following names are aliases for drivers that use the
% same method. Therefore the entry in the \xext{log} file for
% the used driver prints the internally used driver name.
% \begin{description}
% \item[\xoption{driverfallback}:]
%   This option expects a driver that is used,
%   if the driver could not be detected automatically.
% \end{description}
%
% \begin{declcs}{hologoDriverSetup} \M{driver option}
% \end{declcs}
% The driver can also be configured after package loading
% using \cs{hologoDriverSetup}, also the way for \hologo{plainTeX}
% to setup the driver.
%
% \subsection{Font setup}
%
% Some logos require a special font, but should also be usable by
% \hologo{plainTeX}. Therefore the package provides some ways
% to influence the font settings. The options below
% take font settings as values. Both font commands
% such as \cs{sffamily} and macros that take one argument
% like \cs{textsf} can be used.
%
% \begin{declcs}{hologoFontSetup} \M{key value list}
% \end{declcs}
% Macro \cs{hologoFontSetup} sets the fonts for all logos.
% Supported keys:
% \begin{description}
% \def\entry#1{\item[\xoption{#1}:]}
% \entry{general}
%   This font is used for all logos. The default is empty.
%   That means no special font is used.
% \entry{bibsf}
%   This font is used for
%   {\hologoLogoSetup{BibTeX}{variant=sf}\hologo{BibTeX}}
%   with variant \xoption{sf}.
% \entry{rm}
%   This font is a serif font. It is used for \hologo{ExTeX}.
% \entry{sc}
%   This font specifies a small caps font. It is used for
%   {\hologoLogoSetup{BibTeX}{variant=sc}\hologo{BibTeX}}
%   with variant \xoption{sc}.
% \entry{sf}
%   This font specifies a sans serif font. The default
%   is \cs{sffamily}, then \cs{sf} is tried. Otherwise
%   a warning is given. It is used by \hologo{KOMAScript}.
% \entry{sy}
%   This is the font for math symbols (e.g. cmsy).
%   It is used by \hologo{AmS}, \hologo{NTS}, \hologo{ExTeX}.
% \entry{logo}
%   \hologo{METAFONT} and \hologo{METAPOST} are using that font.
%   In \hologo{LaTeX} \cs{logofamily} is used and
%   the definitions of package \xpackage{mflogo} are used
%   if the package is not loaded.
%   Otherwise the \cs{tenlogo} is used and defined
%   if it does not already exists.
% \end{description}
%
% \begin{declcs}{hologoLogoFontSetup} \M{logo} \M{key value list}
% \end{declcs}
% Fonts can also be set for a logo or logo component separately,
% see the following list.
% The keys are the same as for \cs{hologoFontSetup}.
%
% \begin{longtable}{>{\ttfamily}l>{\sffamily}ll}
%   \meta{logo} & keys & result\\
%   \hline
%   \endhead
%   BibTeX & bibsf & {\hologoLogoSetup{BibTeX}{variant=sf}\hologo{BibTeX}}\\[.5ex]
%   BibTeX & sc & {\hologoLogoSetup{BibTeX}{variant=sc}\hologo{BibTeX}}\\[.5ex]
%   ExTeX & rm & \hologo{ExTeX}\\
%   SliTeX & rm & \hologo{SliTeX}\\[.5ex]
%   AmS & sy & \hologo{AmS}\\
%   ExTeX & sy & \hologo{ExTeX}\\
%   NTS & sy & \hologo{NTS}\\[.5ex]
%   KOMAScript & sf & \hologo{KOMAScript}\\[.5ex]
%   METAFONT & logo & \hologo{METAFONT}\\
%   METAPOST & logo & \hologo{METAPOST}\\[.5ex]
%   SliTeX & sc \hologo{SliTeX}
% \end{longtable}
%
% \subsubsection{Font order}
%
% For all logos the font \xoption{general} is applied first.
% Example:
%\begin{quote}
%|\hologoFontSetup{general=\color{red}}|
%\end{quote}
% will print red logos.
% Then if the font uses a special font \xoption{sf}, for example,
% the font is applied that is setup by \cs{hologoLogoFontSetup}.
% If this font is not setup, then the common font setup
% by \cs{hologoFontSetup} is used. Otherwise a warning is given,
% that there is no font configured.
%
% \subsection{Additional user macros}
%
% Usually a variant of a logo is configured by using
% \cs{hologoLogoSetup}, because it is bad style to mix
% different variants of the same logo in the same text.
% There the following macros are a convenience for testing.
%
% \begin{declcs}{hologoVariant} \M{name} \M{variant}\\
%   \cs{HologoVariant} \M{name} \M{variant}
% \end{declcs}
% Logo \meta{name} is set using \meta{variant} that specifies
% explicitely which variant of the macro is used. If the argument
% is empty, then the default form of the logo is used
% (configurable by \cs{hologoLogoSetup}).
%
% \cs{HologoVariant} is used if the logo is set in a context
% that needs an uppercase first letter (beginning of a sentence, \dots).
%
% \begin{declcs}{hologoList}\\
%   \cs{hologoEntry} \M{logo} \M{variant} \M{since}
% \end{declcs}
% Macro \cs{hologoList} contains all logos that are provided
% by the package including variants. The list consists of calls
% of \cs{hologoEntry} with three arguments starting with the
% logo name \meta{logo} and its variant \meta{variant}. An empty
% variant means the current default. Argument \meta{since} specifies
% with version of the package \xpackage{hologo} is needed to get
% the logo. If the logo is fixed, then the date gets updated.
% Therefore the date \meta{since} is not exactly the date of
% the first introduction, but rather the date of the latest fix.
%
% Before \cs{hologoList} can be used, macro \cs{hologoEntry} needs
% a definition. The example file in section \ref{sec:example}
% shows applications of \cs{hologoList}.
%
% \subsection{Supported contexts}
%
% Macros \cs{hologo} and friends support special contexts:
% \begin{itemize}
% \item \hologo{LaTeX}'s protection mechanism.
% \item Bookmarks of package \xpackage{hyperref}.
% \item Package \xpackage{tex4ht}.
% \item The macros can be used inside \cs{csname} constructs,
%   if \cs{ifincsname} is available (\hologo{pdfTeX}, \hologo{XeTeX},
%   \hologo{LuaTeX}).
% \end{itemize}
%
% \subsection{Example}
% \label{sec:example}
%
% The following example prints the logos in different fonts.
%    \begin{macrocode}
%<*example>
%<<verbatim
\NeedsTeXFormat{LaTeX2e}
\documentclass[a4paper]{article}
\usepackage[
  hmargin=20mm,
  vmargin=20mm,
]{geometry}
\pagestyle{empty}
\usepackage{hologo}[2016/05/12]
\usepackage{longtable}
\usepackage{array}
\setlength{\extrarowheight}{2pt}
\usepackage[T1]{fontenc}
\usepackage{lmodern}
\usepackage{pdflscape}
\usepackage[
  pdfencoding=auto,
]{hyperref}
\hypersetup{
  pdfauthor={Heiko Oberdiek},
  pdftitle={Example for package `hologo'},
  pdfsubject={Logos with fonts lmr, lmss, qtm, qpl, qhv},
}
\usepackage{bookmark}

% Print the logo list on the console

\begingroup
  \typeout{}%
  \typeout{*** Begin of logo list ***}%
  \newcommand*{\hologoEntry}[3]{%
    \typeout{#1 \ifx\\#2\\\else(#2) \fi[#3]}%
  }%
  \hologoList
  \typeout{*** End of logo list ***}%
  \typeout{}%
\endgroup

\begin{document}
\begin{landscape}

  \section{Example file for package `hologo'}

  % Table for font names

  \begin{longtable}{>{\bfseries}ll}
    \textbf{font} & \textbf{Font name}\\
    \hline
    lmr & Latin Modern Roman\\
    lmss & Latin Modern Sans\\
    qtm & \TeX\ Gyre Termes\\
    qhv & \TeX\ Gyre Heros\\
    qpl & \TeX\ Gyre Pagella\\
  \end{longtable}

  % Logo list with logos in different fonts

  \begingroup
    \newcommand*{\SetVariant}[2]{%
      \ifx\\#2\\%
      \else
        \hologoLogoSetup{#1}{variant=#2}%
      \fi
    }%
    \newcommand*{\hologoEntry}[3]{%
      \SetVariant{#1}{#2}%
      \raisebox{1em}[0pt][0pt]{\hypertarget{#1@#2}{}}%
      \bookmark[%
        dest={#1@#2},%
      ]{%
        #1\ifx\\#2\\\else\space(#2)\fi: \Hologo{#1}, \hologo{#1} %
        [Unicode]%
      }%
      \hypersetup{unicode=false}%
      \bookmark[%
        dest={#1@#2},%
      ]{%
        #1\ifx\\#2\\\else\space(#2)\fi: \Hologo{#1}, \hologo{#1} %
        [PDFDocEncoding]%
      }%
      \texttt{#1}%
      &%
      \texttt{#2}%
      &%
      \Hologo{#1}%
      &%
      \SetVariant{#1}{#2}%
      \hologo{#1}%
      &%
      \SetVariant{#1}{#2}%
      \fontfamily{qtm}\selectfont
      \hologo{#1}%
      &%
      \SetVariant{#1}{#2}%
      \fontfamily{qpl}\selectfont
      \hologo{#1}%
      &%
      \SetVariant{#1}{#2}%
      \textsf{\hologo{#1}}%
      &%
      \SetVariant{#1}{#2}%
      \fontfamily{qhv}\selectfont
      \hologo{#1}%
      \tabularnewline
    }%
    \begin{longtable}{llllllll}%
      \textbf{\textit{logo}} & \textbf{\textit{variant}} &
      \texttt{\string\Hologo} &
      \textbf{lmr} & \textbf{qtm} & \textbf{qpl} &
      \textbf{lmss} & \textbf{qhv}
      \tabularnewline
      \hline
      \endhead
      \hologoList
    \end{longtable}%
  \endgroup

\end{landscape}
\end{document}
%verbatim
%</example>
%    \end{macrocode}
%
% \StopEventually{
% }
%
% \section{Implementation}
%    \begin{macrocode}
%<*package>
%    \end{macrocode}
%    Reload check, especially if the package is not used with \LaTeX.
%    \begin{macrocode}
\begingroup\catcode61\catcode48\catcode32=10\relax%
  \catcode13=5 % ^^M
  \endlinechar=13 %
  \catcode35=6 % #
  \catcode39=12 % '
  \catcode44=12 % ,
  \catcode45=12 % -
  \catcode46=12 % .
  \catcode58=12 % :
  \catcode64=11 % @
  \catcode123=1 % {
  \catcode125=2 % }
  \expandafter\let\expandafter\x\csname ver@hologo.sty\endcsname
  \ifx\x\relax % plain-TeX, first loading
  \else
    \def\empty{}%
    \ifx\x\empty % LaTeX, first loading,
      % variable is initialized, but \ProvidesPackage not yet seen
    \else
      \expandafter\ifx\csname PackageInfo\endcsname\relax
        \def\x#1#2{%
          \immediate\write-1{Package #1 Info: #2.}%
        }%
      \else
        \def\x#1#2{\PackageInfo{#1}{#2, stopped}}%
      \fi
      \x{hologo}{The package is already loaded}%
      \aftergroup\endinput
    \fi
  \fi
\endgroup%
%    \end{macrocode}
%    Package identification:
%    \begin{macrocode}
\begingroup\catcode61\catcode48\catcode32=10\relax%
  \catcode13=5 % ^^M
  \endlinechar=13 %
  \catcode35=6 % #
  \catcode39=12 % '
  \catcode40=12 % (
  \catcode41=12 % )
  \catcode44=12 % ,
  \catcode45=12 % -
  \catcode46=12 % .
  \catcode47=12 % /
  \catcode58=12 % :
  \catcode64=11 % @
  \catcode91=12 % [
  \catcode93=12 % ]
  \catcode123=1 % {
  \catcode125=2 % }
  \expandafter\ifx\csname ProvidesPackage\endcsname\relax
    \def\x#1#2#3[#4]{\endgroup
      \immediate\write-1{Package: #3 #4}%
      \xdef#1{#4}%
    }%
  \else
    \def\x#1#2[#3]{\endgroup
      #2[{#3}]%
      \ifx#1\@undefined
        \xdef#1{#3}%
      \fi
      \ifx#1\relax
        \xdef#1{#3}%
      \fi
    }%
  \fi
\expandafter\x\csname ver@hologo.sty\endcsname
\ProvidesPackage{hologo}%
  [2016/05/12 v1.11 A logo collection with bookmark support (HO)]%
%    \end{macrocode}
%
%    \begin{macrocode}
\begingroup\catcode61\catcode48\catcode32=10\relax%
  \catcode13=5 % ^^M
  \endlinechar=13 %
  \catcode123=1 % {
  \catcode125=2 % }
  \catcode64=11 % @
  \def\x{\endgroup
    \expandafter\edef\csname HOLOGO@AtEnd\endcsname{%
      \endlinechar=\the\endlinechar\relax
      \catcode13=\the\catcode13\relax
      \catcode32=\the\catcode32\relax
      \catcode35=\the\catcode35\relax
      \catcode61=\the\catcode61\relax
      \catcode64=\the\catcode64\relax
      \catcode123=\the\catcode123\relax
      \catcode125=\the\catcode125\relax
    }%
  }%
\x\catcode61\catcode48\catcode32=10\relax%
\catcode13=5 % ^^M
\endlinechar=13 %
\catcode35=6 % #
\catcode64=11 % @
\catcode123=1 % {
\catcode125=2 % }
\def\TMP@EnsureCode#1#2{%
  \edef\HOLOGO@AtEnd{%
    \HOLOGO@AtEnd
    \catcode#1=\the\catcode#1\relax
  }%
  \catcode#1=#2\relax
}
\TMP@EnsureCode{10}{12}% ^^J
\TMP@EnsureCode{33}{12}% !
\TMP@EnsureCode{34}{12}% "
\TMP@EnsureCode{36}{3}% $
\TMP@EnsureCode{38}{4}% &
\TMP@EnsureCode{39}{12}% '
\TMP@EnsureCode{40}{12}% (
\TMP@EnsureCode{41}{12}% )
\TMP@EnsureCode{42}{12}% *
\TMP@EnsureCode{43}{12}% +
\TMP@EnsureCode{44}{12}% ,
\TMP@EnsureCode{45}{12}% -
\TMP@EnsureCode{46}{12}% .
\TMP@EnsureCode{47}{12}% /
\TMP@EnsureCode{58}{12}% :
\TMP@EnsureCode{59}{12}% ;
\TMP@EnsureCode{60}{12}% <
\TMP@EnsureCode{62}{12}% >
\TMP@EnsureCode{63}{12}% ?
\TMP@EnsureCode{91}{12}% [
\TMP@EnsureCode{93}{12}% ]
\TMP@EnsureCode{94}{7}% ^ (superscript)
\TMP@EnsureCode{95}{8}% _ (subscript)
\TMP@EnsureCode{96}{12}% `
\TMP@EnsureCode{124}{12}% |
\edef\HOLOGO@AtEnd{%
  \HOLOGO@AtEnd
  \escapechar\the\escapechar\relax
  \noexpand\endinput
}
\escapechar=92 %
%    \end{macrocode}
%
% \subsection{Logo list}
%
%    \begin{macro}{\hologoList}
%    \begin{macrocode}
\def\hologoList{%
  \hologoEntry{(La)TeX}{}{2011/10/01}%
  \hologoEntry{AmSLaTeX}{}{2010/04/16}%
  \hologoEntry{AmSTeX}{}{2010/04/16}%
  \hologoEntry{biber}{}{2011/10/01}%
  \hologoEntry{BibTeX}{}{2011/10/01}%
  \hologoEntry{BibTeX}{sf}{2011/10/01}%
  \hologoEntry{BibTeX}{sc}{2011/10/01}%
  \hologoEntry{BibTeX8}{}{2011/11/22}%
  \hologoEntry{ConTeXt}{}{2011/03/25}%
  \hologoEntry{ConTeXt}{narrow}{2011/03/25}%
  \hologoEntry{ConTeXt}{simple}{2011/03/25}%
  \hologoEntry{emTeX}{}{2010/04/26}%
  \hologoEntry{eTeX}{}{2010/04/08}%
  \hologoEntry{ExTeX}{}{2011/10/01}%
  \hologoEntry{HanTheThanh}{}{2011/11/29}%
  \hologoEntry{iniTeX}{}{2011/10/01}%
  \hologoEntry{KOMAScript}{}{2011/10/01}%
  \hologoEntry{La}{}{2010/05/08}%
  \hologoEntry{LaTeX}{}{2010/04/08}%
  \hologoEntry{LaTeX2e}{}{2010/04/08}%
  \hologoEntry{LaTeX3}{}{2010/04/24}%
  \hologoEntry{LaTeXe}{}{2010/04/08}%
  \hologoEntry{LaTeXML}{}{2011/11/22}%
  \hologoEntry{LaTeXTeX}{}{2011/10/01}%
  \hologoEntry{LuaLaTeX}{}{2010/04/08}%
  \hologoEntry{LuaTeX}{}{2010/04/08}%
  \hologoEntry{LyX}{}{2011/10/01}%
  \hologoEntry{METAFONT}{}{2011/10/01}%
  \hologoEntry{MetaFun}{}{2011/10/01}%
  \hologoEntry{METAPOST}{}{2011/10/01}%
  \hologoEntry{MetaPost}{}{2011/10/01}%
  \hologoEntry{MiKTeX}{}{2011/10/01}%
  \hologoEntry{NTS}{}{2011/10/01}%
  \hologoEntry{OzMF}{}{2011/10/01}%
  \hologoEntry{OzMP}{}{2011/10/01}%
  \hologoEntry{OzTeX}{}{2011/10/01}%
  \hologoEntry{OzTtH}{}{2011/10/01}%
  \hologoEntry{PCTeX}{}{2011/10/01}%
  \hologoEntry{pdfTeX}{}{2011/10/01}%
  \hologoEntry{pdfLaTeX}{}{2011/10/01}%
  \hologoEntry{PiC}{}{2011/10/01}%
  \hologoEntry{PiCTeX}{}{2011/10/01}%
  \hologoEntry{plainTeX}{}{2010/04/08}%
  \hologoEntry{plainTeX}{space}{2010/04/16}%
  \hologoEntry{plainTeX}{hyphen}{2010/04/16}%
  \hologoEntry{plainTeX}{runtogether}{2010/04/16}%
  \hologoEntry{SageTeX}{}{2011/11/22}%
  \hologoEntry{SLiTeX}{}{2011/10/01}%
  \hologoEntry{SLiTeX}{lift}{2011/10/01}%
  \hologoEntry{SLiTeX}{narrow}{2011/10/01}%
  \hologoEntry{SLiTeX}{simple}{2011/10/01}%
  \hologoEntry{SliTeX}{}{2011/10/01}%
  \hologoEntry{SliTeX}{narrow}{2011/10/01}%
  \hologoEntry{SliTeX}{simple}{2011/10/01}%
  \hologoEntry{SliTeX}{lift}{2011/10/01}%
  \hologoEntry{teTeX}{}{2011/10/01}%
  \hologoEntry{TeX}{}{2010/04/08}%
  \hologoEntry{TeX4ht}{}{2011/11/22}%
  \hologoEntry{TTH}{}{2011/11/22}%
  \hologoEntry{virTeX}{}{2011/10/01}%
  \hologoEntry{VTeX}{}{2010/04/24}%
  \hologoEntry{Xe}{}{2010/04/08}%
  \hologoEntry{XeLaTeX}{}{2010/04/08}%
  \hologoEntry{XeTeX}{}{2010/04/08}%
}
%    \end{macrocode}
%    \end{macro}
%
% \subsection{Load resources}
%
%    \begin{macrocode}
\begingroup\expandafter\expandafter\expandafter\endgroup
\expandafter\ifx\csname RequirePackage\endcsname\relax
  \def\TMP@RequirePackage#1[#2]{%
    \begingroup\expandafter\expandafter\expandafter\endgroup
    \expandafter\ifx\csname ver@#1.sty\endcsname\relax
      \input #1.sty\relax
    \fi
  }%
  \TMP@RequirePackage{ltxcmds}[2011/02/04]%
  \TMP@RequirePackage{infwarerr}[2010/04/08]%
  \TMP@RequirePackage{kvsetkeys}[2010/03/01]%
  \TMP@RequirePackage{kvdefinekeys}[2010/03/01]%
  \TMP@RequirePackage{pdftexcmds}[2010/04/01]%
  \TMP@RequirePackage{ifpdf}[2010/01/28]%
  \TMP@RequirePackage{ifluatex}[2010/03/01]%
  \ltx@IfUndefined{newif}{%
    \expandafter\let\csname newif\endcsname\ltx@newif
  }{}%
  \TMP@RequirePackage{ifxetex}[2009/01/23]%
  \TMP@RequirePackage{ifvtex}[2010/03/01]%
\else
  \RequirePackage{ltxcmds}[2011/02/04]%
  \RequirePackage{infwarerr}[2010/04/08]%
  \RequirePackage{kvsetkeys}[2010/03/01]%
  \RequirePackage{kvdefinekeys}[2010/03/01]%
  \RequirePackage{pdftexcmds}[2010/04/01]%
  \RequirePackage{ifpdf}[2010/01/28]%
  \RequirePackage{ifluatex}[2010/03/01]%
  \RequirePackage{ifxetex}[2009/01/23]%
  \RequirePackage{ifvtex}[2010/03/01]%
\fi
%    \end{macrocode}
%
%    \begin{macro}{\HOLOGO@IfDefined}
%    \begin{macrocode}
\def\HOLOGO@IfExists#1{%
  \ifx\@undefined#1%
    \expandafter\ltx@secondoftwo
  \else
    \ifx\relax#1%
      \expandafter\ltx@secondoftwo
    \else
      \expandafter\expandafter\expandafter\ltx@firstoftwo
    \fi
  \fi
}
%    \end{macrocode}
%    \end{macro}
%
% \subsection{Setup macros}
%
%    \begin{macro}{\hologoSetup}
%    \begin{macrocode}
\def\hologoSetup{%
  \let\HOLOGO@name\relax
  \HOLOGO@Setup
}
%    \end{macrocode}
%    \end{macro}
%
%    \begin{macro}{\hologoLogoSetup}
%    \begin{macrocode}
\def\hologoLogoSetup#1{%
  \edef\HOLOGO@name{#1}%
  \ltx@IfUndefined{HoLogo@\HOLOGO@name}{%
    \@PackageError{hologo}{%
      Unknown logo `\HOLOGO@name'%
    }\@ehc
    \ltx@gobble
  }{%
    \HOLOGO@Setup
  }%
}
%    \end{macrocode}
%    \end{macro}
%
%    \begin{macro}{\HOLOGO@Setup}
%    \begin{macrocode}
\def\HOLOGO@Setup{%
  \kvsetkeys{HoLogo}%
}
%    \end{macrocode}
%    \end{macro}
%
% \subsection{Options}
%
%    \begin{macro}{\HOLOGO@DeclareBoolOption}
%    \begin{macrocode}
\def\HOLOGO@DeclareBoolOption#1{%
  \expandafter\chardef\csname HOLOGOOPT@#1\endcsname\ltx@zero
  \kv@define@key{HoLogo}{#1}[true]{%
    \def\HOLOGO@temp{##1}%
    \ifx\HOLOGO@temp\HOLOGO@true
      \ifx\HOLOGO@name\relax
        \expandafter\chardef\csname HOLOGOOPT@#1\endcsname=\ltx@one
      \else
        \expandafter\chardef\csname
        HoLogoOpt@#1@\HOLOGO@name\endcsname\ltx@one
      \fi
      \HOLOGO@SetBreakAll{#1}%
    \else
      \ifx\HOLOGO@temp\HOLOGO@false
        \ifx\HOLOGO@name\relax
          \expandafter\chardef\csname HOLOGOOPT@#1\endcsname=\ltx@zero
        \else
          \expandafter\chardef\csname
          HoLogoOpt@#1@\HOLOGO@name\endcsname=\ltx@zero
        \fi
        \HOLOGO@SetBreakAll{#1}%
      \else
        \@PackageError{hologo}{%
          Unknown value `##1' for boolean option `#1'.\MessageBreak
          Known values are `true' and `false'%
        }\@ehc
      \fi
    \fi
  }%
}
%    \end{macrocode}
%    \end{macro}
%
%    \begin{macro}{\HOLOGO@SetBreakAll}
%    \begin{macrocode}
\def\HOLOGO@SetBreakAll#1{%
  \def\HOLOGO@temp{#1}%
  \ifx\HOLOGO@temp\HOLOGO@break
    \ifx\HOLOGO@name\relax
      \chardef\HOLOGOOPT@hyphenbreak=\HOLOGOOPT@break
      \chardef\HOLOGOOPT@spacebreak=\HOLOGOOPT@break
      \chardef\HOLOGOOPT@discretionarybreak=\HOLOGOOPT@break
    \else
      \expandafter\chardef
         \csname HoLogoOpt@hyphenbreak@\HOLOGO@name\endcsname=%
         \csname HoLogoOpt@break@\HOLOGO@name\endcsname
      \expandafter\chardef
         \csname HoLogoOpt@spacebreak@\HOLOGO@name\endcsname=%
         \csname HoLogoOpt@break@\HOLOGO@name\endcsname
      \expandafter\chardef
         \csname HoLogoOpt@discretionarybreak@\HOLOGO@name
             \endcsname=%
         \csname HoLogoOpt@break@\HOLOGO@name\endcsname
    \fi
  \fi
}
%    \end{macrocode}
%    \end{macro}
%
%    \begin{macro}{\HOLOGO@true}
%    \begin{macrocode}
\def\HOLOGO@true{true}
%    \end{macrocode}
%    \end{macro}
%    \begin{macro}{\HOLOGO@false}
%    \begin{macrocode}
\def\HOLOGO@false{false}
%    \end{macrocode}
%    \end{macro}
%    \begin{macro}{\HOLOGO@break}
%    \begin{macrocode}
\def\HOLOGO@break{break}
%    \end{macrocode}
%    \end{macro}
%
%    \begin{macrocode}
\HOLOGO@DeclareBoolOption{break}
\HOLOGO@DeclareBoolOption{hyphenbreak}
\HOLOGO@DeclareBoolOption{spacebreak}
\HOLOGO@DeclareBoolOption{discretionarybreak}
%    \end{macrocode}
%
%    \begin{macrocode}
\kv@define@key{HoLogo}{variant}{%
  \ifx\HOLOGO@name\relax
    \@PackageError{hologo}{%
      Option `variant' is not available in \string\hologoSetup,%
      \MessageBreak
      Use \string\hologoLogoSetup\space instead%
    }\@ehc
  \else
    \edef\HOLOGO@temp{#1}%
    \ifx\HOLOGO@temp\ltx@empty
      \expandafter
      \let\csname HoLogoOpt@variant@\HOLOGO@name\endcsname\@undefined
    \else
      \ltx@IfUndefined{HoLogo@\HOLOGO@name @\HOLOGO@temp}{%
        \@PackageError{hologo}{%
          Unknown variant `\HOLOGO@temp' of logo `\HOLOGO@name'%
        }\@ehc
      }{%
        \expandafter
        \let\csname HoLogoOpt@variant@\HOLOGO@name\endcsname
            \HOLOGO@temp
      }%
    \fi
  \fi
}
%    \end{macrocode}
%
%    \begin{macro}{\HOLOGO@Variant}
%    \begin{macrocode}
\def\HOLOGO@Variant#1{%
  #1%
  \ltx@ifundefined{HoLogoOpt@variant@#1}{%
  }{%
    @\csname HoLogoOpt@variant@#1\endcsname
  }%
}
%    \end{macrocode}
%    \end{macro}
%
% \subsection{Break/no-break support}
%
%    \begin{macro}{\HOLOGO@space}
%    \begin{macrocode}
\def\HOLOGO@space{%
  \ltx@ifundefined{HoLogoOpt@spacebreak@\HOLOGO@name}{%
    \ltx@ifundefined{HoLogoOpt@break@\HOLOGO@name}{%
      \chardef\HOLOGO@temp=\HOLOGOOPT@spacebreak
    }{%
      \chardef\HOLOGO@temp=%
        \csname HoLogoOpt@break@\HOLOGO@name\endcsname
    }%
  }{%
    \chardef\HOLOGO@temp=%
      \csname HoLogoOpt@spacebreak@\HOLOGO@name\endcsname
  }%
  \ifcase\HOLOGO@temp
    \penalty10000 %
  \fi
  \ltx@space
}
%    \end{macrocode}
%    \end{macro}
%
%    \begin{macro}{\HOLOGO@hyphen}
%    \begin{macrocode}
\def\HOLOGO@hyphen{%
  \ltx@ifundefined{HoLogoOpt@hyphenbreak@\HOLOGO@name}{%
    \ltx@ifundefined{HoLogoOpt@break@\HOLOGO@name}{%
      \chardef\HOLOGO@temp=\HOLOGOOPT@hyphenbreak
    }{%
      \chardef\HOLOGO@temp=%
        \csname HoLogoOpt@break@\HOLOGO@name\endcsname
    }%
  }{%
    \chardef\HOLOGO@temp=%
      \csname HoLogoOpt@hyphenbreak@\HOLOGO@name\endcsname
  }%
  \ifcase\HOLOGO@temp
    \ltx@mbox{-}%
  \else
    -%
  \fi
}
%    \end{macrocode}
%    \end{macro}
%
%    \begin{macro}{\HOLOGO@discretionary}
%    \begin{macrocode}
\def\HOLOGO@discretionary{%
  \ltx@ifundefined{HoLogoOpt@discretionarybreak@\HOLOGO@name}{%
    \ltx@ifundefined{HoLogoOpt@break@\HOLOGO@name}{%
      \chardef\HOLOGO@temp=\HOLOGOOPT@discretionarybreak
    }{%
      \chardef\HOLOGO@temp=%
        \csname HoLogoOpt@break@\HOLOGO@name\endcsname
    }%
  }{%
    \chardef\HOLOGO@temp=%
      \csname HoLogoOpt@discretionarybreak@\HOLOGO@name\endcsname
  }%
  \ifcase\HOLOGO@temp
  \else
    \-%
  \fi
}
%    \end{macrocode}
%    \end{macro}
%
%    \begin{macro}{\HOLOGO@mbox}
%    \begin{macrocode}
\def\HOLOGO@mbox#1{%
  \ltx@ifundefined{HoLogoOpt@break@\HOLOGO@name}{%
    \chardef\HOLOGO@temp=\HOLOGOOPT@hyphenbreak
  }{%
    \chardef\HOLOGO@temp=%
      \csname HoLogoOpt@break@\HOLOGO@name\endcsname
  }%
  \ifcase\HOLOGO@temp
    \ltx@mbox{#1}%
  \else
    #1%
  \fi
}
%    \end{macrocode}
%    \end{macro}
%
% \subsection{Font support}
%
%    \begin{macro}{\HoLogoFont@font}
%    \begin{tabular}{@{}ll@{}}
%    |#1|:& logo name\\
%    |#2|:& font short name\\
%    |#3|:& text
%    \end{tabular}
%    \begin{macrocode}
\def\HoLogoFont@font#1#2#3{%
  \begingroup
    \ltx@IfUndefined{HoLogoFont@logo@#1.#2}{%
      \ltx@IfUndefined{HoLogoFont@font@#2}{%
        \@PackageWarning{hologo}{%
          Missing font `#2' for logo `#1'%
        }%
        #3%
      }{%
        \csname HoLogoFont@font@#2\endcsname{#3}%
      }%
    }{%
      \csname HoLogoFont@logo@#1.#2\endcsname{#3}%
    }%
  \endgroup
}
%    \end{macrocode}
%    \end{macro}
%
%    \begin{macro}{\HoLogoFont@Def}
%    \begin{macrocode}
\def\HoLogoFont@Def#1{%
  \expandafter\def\csname HoLogoFont@font@#1\endcsname
}
%    \end{macrocode}
%    \end{macro}
%    \begin{macro}{\HoLogoFont@LogoDef}
%    \begin{macrocode}
\def\HoLogoFont@LogoDef#1#2{%
  \expandafter\def\csname HoLogoFont@logo@#1.#2\endcsname
}
%    \end{macrocode}
%    \end{macro}
%
% \subsubsection{Font defaults}
%
%    \begin{macro}{\HoLogoFont@font@general}
%    \begin{macrocode}
\HoLogoFont@Def{general}{}%
%    \end{macrocode}
%    \end{macro}
%
%    \begin{macro}{\HoLogoFont@font@rm}
%    \begin{macrocode}
\ltx@IfUndefined{rmfamily}{%
  \ltx@IfUndefined{rm}{%
  }{%
    \HoLogoFont@Def{rm}{\rm}%
  }%
}{%
  \HoLogoFont@Def{rm}{\rmfamily}%
}
%    \end{macrocode}
%    \end{macro}
%
%    \begin{macro}{\HoLogoFont@font@sf}
%    \begin{macrocode}
\ltx@IfUndefined{sffamily}{%
  \ltx@IfUndefined{sf}{%
  }{%
    \HoLogoFont@Def{sf}{\sf}%
  }%
}{%
  \HoLogoFont@Def{sf}{\sffamily}%
}
%    \end{macrocode}
%    \end{macro}
%
%    \begin{macro}{\HoLogoFont@font@bibsf}
%    In case of \hologo{plainTeX} the original small caps
%    variant is used as default. In \hologo{LaTeX}
%    the definition of package \xpackage{dtklogos} \cite{dtklogos}
%    is used.
%\begin{quote}
%\begin{verbatim}
%\DeclareRobustCommand{\BibTeX}{%
%  B%
%  \kern-.05em%
%  \hbox{%
%    $\m@th$% %% force math size calculations
%    \csname S@\f@size\endcsname
%    \fontsize\sf@size\z@
%    \math@fontsfalse
%    \selectfont
%    I%
%    \kern-.025em%
%    B
%  }%
%  \kern-.08em%
%  \-%
%  \TeX
%}
%\end{verbatim}
%\end{quote}
%    \begin{macrocode}
\ltx@IfUndefined{selectfont}{%
  \ltx@IfUndefined{tensc}{%
    \font\tensc=cmcsc10\relax
  }{}%
  \HoLogoFont@Def{bibsf}{\tensc}%
}{%
  \HoLogoFont@Def{bibsf}{%
    $\mathsurround=0pt$%
    \csname S@\f@size\endcsname
    \fontsize\sf@size{0pt}%
    \math@fontsfalse
    \selectfont
  }%
}
%    \end{macrocode}
%    \end{macro}
%
%    \begin{macro}{\HoLogoFont@font@sc}
%    \begin{macrocode}
\ltx@IfUndefined{scshape}{%
  \ltx@IfUndefined{tensc}{%
    \font\tensc=cmcsc10\relax
  }{}%
  \HoLogoFont@Def{sc}{\tensc}%
}{%
  \HoLogoFont@Def{sc}{\scshape}%
}
%    \end{macrocode}
%    \end{macro}
%
%    \begin{macro}{\HoLogoFont@font@sy}
%    \begin{macrocode}
\ltx@IfUndefined{usefont}{%
  \ltx@IfUndefined{tensy}{%
  }{%
    \HoLogoFont@Def{sy}{\tensy}%
  }%
}{%
  \HoLogoFont@Def{sy}{%
    \usefont{OMS}{cmsy}{m}{n}%
  }%
}
%    \end{macrocode}
%    \end{macro}
%
%    \begin{macro}{\HoLogoFont@font@logo}
%    \begin{macrocode}
\begingroup
  \def\x{LaTeX2e}%
\expandafter\endgroup
\ifx\fmtname\x
  \ltx@IfUndefined{logofamily}{%
    \DeclareRobustCommand\logofamily{%
      \not@math@alphabet\logofamily\relax
      \fontencoding{U}%
      \fontfamily{logo}%
      \selectfont
    }%
  }{}%
  \ltx@IfUndefined{logofamily}{%
  }{%
    \HoLogoFont@Def{logo}{\logofamily}%
  }%
\else
  \ltx@IfUndefined{tenlogo}{%
    \font\tenlogo=logo10\relax
  }{}%
  \HoLogoFont@Def{logo}{\tenlogo}%
\fi
%    \end{macrocode}
%    \end{macro}
%
% \subsubsection{Font setup}
%
%    \begin{macro}{\hologoFontSetup}
%    \begin{macrocode}
\def\hologoFontSetup{%
  \let\HOLOGO@name\relax
  \HOLOGO@FontSetup
}
%    \end{macrocode}
%    \end{macro}
%
%    \begin{macro}{\hologoLogoFontSetup}
%    \begin{macrocode}
\def\hologoLogoFontSetup#1{%
  \edef\HOLOGO@name{#1}%
  \ltx@IfUndefined{HoLogo@\HOLOGO@name}{%
    \@PackageError{hologo}{%
      Unknown logo `\HOLOGO@name'%
    }\@ehc
    \ltx@gobble
  }{%
    \HOLOGO@FontSetup
  }%
}
%    \end{macrocode}
%    \end{macro}
%
%    \begin{macro}{\HOLOGO@FontSetup}
%    \begin{macrocode}
\def\HOLOGO@FontSetup{%
  \kvsetkeys{HoLogoFont}%
}
%    \end{macrocode}
%    \end{macro}
%
%    \begin{macrocode}
\def\HOLOGO@temp#1{%
  \kv@define@key{HoLogoFont}{#1}{%
    \ifx\HOLOGO@name\relax
      \HoLogoFont@Def{#1}{##1}%
    \else
      \HoLogoFont@LogoDef\HOLOGO@name{#1}{##1}%
    \fi
  }%
}
\HOLOGO@temp{general}
\HOLOGO@temp{sf}
%    \end{macrocode}
%
% \subsection{Generic logo commands}
%
%    \begin{macrocode}
\HOLOGO@IfExists\hologo{%
  \@PackageError{hologo}{%
    \string\hologo\ltx@space is already defined.\MessageBreak
    Package loading is aborted%
  }\@ehc
  \HOLOGO@AtEnd
}%
\HOLOGO@IfExists\hologoRobust{%
  \@PackageError{hologo}{%
    \string\hologoRobust\ltx@space is already defined.\MessageBreak
    Package loading is aborted%
  }\@ehc
  \HOLOGO@AtEnd
}%
%    \end{macrocode}
%
% \subsubsection{\cs{hologo} and friends}
%
%    \begin{macrocode}
\ifluatex
  \expandafter\ltx@firstofone
\else
  \expandafter\ltx@gobble
\fi
{%
  \ltx@IfUndefined{ifincsname}{%
    \ifnum\luatexversion<36 %
      \expandafter\ltx@gobble
    \else
      \expandafter\ltx@firstofone
    \fi
    {%
      \begingroup
        \ifcase0%
            \directlua{%
              if tex.enableprimitives then %
                tex.enableprimitives('HOLOGO@', {'ifincsname'})%
              else %
                tex.print('1')%
              end%
            }%
            \ifx\HOLOGO@ifincsname\@undefined 1\fi%
            \relax
          \expandafter\ltx@firstofone
        \else
          \endgroup
          \expandafter\ltx@gobble
        \fi
        {%
          \global\let\ifincsname\HOLOGO@ifincsname
        }%
      \HOLOGO@temp
    }%
  }{}%
}
%    \end{macrocode}
%    \begin{macrocode}
\ltx@IfUndefined{ifincsname}{%
  \catcode`$=14 %
}{%
  \catcode`$=9 %
}
%    \end{macrocode}
%
%    \begin{macro}{\hologo}
%    \begin{macrocode}
\def\hologo#1{%
$ \ifincsname
$   \ltx@ifundefined{HoLogoCs@\HOLOGO@Variant{#1}}{%
$     #1%
$   }{%
$     \csname HoLogoCs@\HOLOGO@Variant{#1}\endcsname\ltx@firstoftwo
$   }%
$ \else
    \HOLOGO@IfExists\texorpdfstring\texorpdfstring\ltx@firstoftwo
    {%
      \hologoRobust{#1}%
    }{%
      \ltx@ifundefined{HoLogoBkm@\HOLOGO@Variant{#1}}{%
        \ltx@ifundefined{HoLogo@#1}{?#1?}{#1}%
      }{%
        \csname HoLogoBkm@\HOLOGO@Variant{#1}\endcsname
        \ltx@firstoftwo
      }%
    }%
$ \fi
}
%    \end{macrocode}
%    \end{macro}
%    \begin{macro}{\Hologo}
%    \begin{macrocode}
\def\Hologo#1{%
$ \ifincsname
$   \ltx@ifundefined{HoLogoCs@\HOLOGO@Variant{#1}}{%
$     #1%
$   }{%
$     \csname HoLogoCs@\HOLOGO@Variant{#1}\endcsname\ltx@secondoftwo
$   }%
$ \else
    \HOLOGO@IfExists\texorpdfstring\texorpdfstring\ltx@firstoftwo
    {%
      \HologoRobust{#1}%
    }{%
      \ltx@ifundefined{HoLogoBkm@\HOLOGO@Variant{#1}}{%
        \ltx@ifundefined{HoLogo@#1}{?#1?}{#1}%
      }{%
        \csname HoLogoBkm@\HOLOGO@Variant{#1}\endcsname
        \ltx@secondoftwo
      }%
    }%
$ \fi
}
%    \end{macrocode}
%    \end{macro}
%
%    \begin{macro}{\hologoVariant}
%    \begin{macrocode}
\def\hologoVariant#1#2{%
  \ifx\relax#2\relax
    \hologo{#1}%
  \else
$   \ifincsname
$     \ltx@ifundefined{HoLogoCs@#1@#2}{%
$       #1%
$     }{%
$       \csname HoLogoCs@#1@#2\endcsname\ltx@firstoftwo
$     }%
$   \else
      \HOLOGO@IfExists\texorpdfstring\texorpdfstring\ltx@firstoftwo
      {%
        \hologoVariantRobust{#1}{#2}%
      }{%
        \ltx@ifundefined{HoLogoBkm@#1@#2}{%
          \ltx@ifundefined{HoLogo@#1}{?#1?}{#1}%
        }{%
          \csname HoLogoBkm@#1@#2\endcsname
          \ltx@firstoftwo
        }%
      }%
$   \fi
  \fi
}
%    \end{macrocode}
%    \end{macro}
%    \begin{macro}{\HologoVariant}
%    \begin{macrocode}
\def\HologoVariant#1#2{%
  \ifx\relax#2\relax
    \Hologo{#1}%
  \else
$   \ifincsname
$     \ltx@ifundefined{HoLogoCs@#1@#2}{%
$       #1%
$     }{%
$       \csname HoLogoCs@#1@#2\endcsname\ltx@secondoftwo
$     }%
$   \else
      \HOLOGO@IfExists\texorpdfstring\texorpdfstring\ltx@firstoftwo
      {%
        \HologoVariantRobust{#1}{#2}%
      }{%
        \ltx@ifundefined{HoLogoBkm@#1@#2}{%
          \ltx@ifundefined{HoLogo@#1}{?#1?}{#1}%
        }{%
          \csname HoLogoBkm@#1@#2\endcsname
          \ltx@secondoftwo
        }%
      }%
$   \fi
  \fi
}
%    \end{macrocode}
%    \end{macro}
%
%    \begin{macrocode}
\catcode`\$=3 %
%    \end{macrocode}
%
% \subsubsection{\cs{hologoRobust} and friends}
%
%    \begin{macro}{\hologoRobust}
%    \begin{macrocode}
\ltx@IfUndefined{protected}{%
  \ltx@IfUndefined{DeclareRobustCommand}{%
    \def\hologoRobust#1%
  }{%
    \DeclareRobustCommand*\hologoRobust[1]%
  }%
}{%
  \protected\def\hologoRobust#1%
}%
{%
  \edef\HOLOGO@name{#1}%
  \ltx@IfUndefined{HoLogo@\HOLOGO@Variant\HOLOGO@name}{%
    \@PackageError{hologo}{%
      Unknown logo `\HOLOGO@name'%
    }\@ehc
    ?\HOLOGO@name?%
  }{%
    \ltx@IfUndefined{ver@tex4ht.sty}{%
      \HoLogoFont@font\HOLOGO@name{general}{%
        \csname HoLogo@\HOLOGO@Variant\HOLOGO@name\endcsname
        \ltx@firstoftwo
      }%
    }{%
      \ltx@IfUndefined{HoLogoHtml@\HOLOGO@Variant\HOLOGO@name}{%
        \HOLOGO@name
      }{%
        \csname HoLogoHtml@\HOLOGO@Variant\HOLOGO@name\endcsname
        \ltx@firstoftwo
      }%
    }%
  }%
}
%    \end{macrocode}
%    \end{macro}
%    \begin{macro}{\HologoRobust}
%    \begin{macrocode}
\ltx@IfUndefined{protected}{%
  \ltx@IfUndefined{DeclareRobustCommand}{%
    \def\HologoRobust#1%
  }{%
    \DeclareRobustCommand*\HologoRobust[1]%
  }%
}{%
  \protected\def\HologoRobust#1%
}%
{%
  \edef\HOLOGO@name{#1}%
  \ltx@IfUndefined{HoLogo@\HOLOGO@Variant\HOLOGO@name}{%
    \@PackageError{hologo}{%
      Unknown logo `\HOLOGO@name'%
    }\@ehc
    ?\HOLOGO@name?%
  }{%
    \ltx@IfUndefined{ver@tex4ht.sty}{%
      \HoLogoFont@font\HOLOGO@name{general}{%
        \csname HoLogo@\HOLOGO@Variant\HOLOGO@name\endcsname
        \ltx@secondoftwo
      }%
    }{%
      \ltx@IfUndefined{HoLogoHtml@\HOLOGO@Variant\HOLOGO@name}{%
        \expandafter\HOLOGO@Uppercase\HOLOGO@name
      }{%
        \csname HoLogoHtml@\HOLOGO@Variant\HOLOGO@name\endcsname
        \ltx@secondoftwo
      }%
    }%
  }%
}
%    \end{macrocode}
%    \end{macro}
%    \begin{macro}{\hologoVariantRobust}
%    \begin{macrocode}
\ltx@IfUndefined{protected}{%
  \ltx@IfUndefined{DeclareRobustCommand}{%
    \def\hologoVariantRobust#1#2%
  }{%
    \DeclareRobustCommand*\hologoVariantRobust[2]%
  }%
}{%
  \protected\def\hologoVariantRobust#1#2%
}%
{%
  \begingroup
    \hologoLogoSetup{#1}{variant={#2}}%
    \hologoRobust{#1}%
  \endgroup
}
%    \end{macrocode}
%    \end{macro}
%    \begin{macro}{\HologoVariantRobust}
%    \begin{macrocode}
\ltx@IfUndefined{protected}{%
  \ltx@IfUndefined{DeclareRobustCommand}{%
    \def\HologoVariantRobust#1#2%
  }{%
    \DeclareRobustCommand*\HologoVariantRobust[2]%
  }%
}{%
  \protected\def\HologoVariantRobust#1#2%
}%
{%
  \begingroup
    \hologoLogoSetup{#1}{variant={#2}}%
    \HologoRobust{#1}%
  \endgroup
}
%    \end{macrocode}
%    \end{macro}
%
%    \begin{macro}{\hologorobust}
%    Macro \cs{hologorobust} is only defined for compatibility.
%    Its use is deprecated.
%    \begin{macrocode}
\def\hologorobust{\hologoRobust}
%    \end{macrocode}
%    \end{macro}
%
% \subsection{Helpers}
%
%    \begin{macro}{\HOLOGO@Uppercase}
%    Macro \cs{HOLOGO@Uppercase} is restricted to \cs{uppercase},
%    because \hologo{plainTeX} or \hologo{iniTeX} do not provide
%    \cs{MakeUppercase}.
%    \begin{macrocode}
\def\HOLOGO@Uppercase#1{\uppercase{#1}}
%    \end{macrocode}
%    \end{macro}
%
%    \begin{macro}{\HOLOGO@PdfdocUnicode}
%    \begin{macrocode}
\def\HOLOGO@PdfdocUnicode{%
  \ifx\ifHy@unicode\iftrue
    \expandafter\ltx@secondoftwo
  \else
    \expandafter\ltx@firstoftwo
  \fi
}
%    \end{macrocode}
%    \end{macro}
%
%    \begin{macro}{\HOLOGO@Math}
%    \begin{macrocode}
\def\HOLOGO@MathSetup{%
  \mathsurround0pt\relax
  \HOLOGO@IfExists\f@series{%
    \if b\expandafter\ltx@car\f@series x\@nil
      \csname boldmath\endcsname
   \fi
  }{}%
}
%    \end{macrocode}
%    \end{macro}
%
%    \begin{macro}{\HOLOGO@TempDimen}
%    \begin{macrocode}
\dimendef\HOLOGO@TempDimen=\ltx@zero
%    \end{macrocode}
%    \end{macro}
%    \begin{macro}{\HOLOGO@NegativeKerning}
%    \begin{macrocode}
\def\HOLOGO@NegativeKerning#1{%
  \begingroup
    \HOLOGO@TempDimen=0pt\relax
    \comma@parse@normalized{#1}{%
      \ifdim\HOLOGO@TempDimen=0pt %
        \expandafter\HOLOGO@@NegativeKerning\comma@entry
      \fi
      \ltx@gobble
    }%
    \ifdim\HOLOGO@TempDimen<0pt %
      \kern\HOLOGO@TempDimen
    \fi
  \endgroup
}
%    \end{macrocode}
%    \end{macro}
%    \begin{macro}{\HOLOGO@@NegativeKerning}
%    \begin{macrocode}
\def\HOLOGO@@NegativeKerning#1#2{%
  \setbox\ltx@zero\hbox{#1#2}%
  \HOLOGO@TempDimen=\wd\ltx@zero
  \setbox\ltx@zero\hbox{#1\kern0pt#2}%
  \advance\HOLOGO@TempDimen by -\wd\ltx@zero
}
%    \end{macrocode}
%    \end{macro}
%
%    \begin{macro}{\HOLOGO@SpaceFactor}
%    \begin{macrocode}
\def\HOLOGO@SpaceFactor{%
  \spacefactor1000 %
}
%    \end{macrocode}
%    \end{macro}
%
%    \begin{macro}{\HOLOGO@Span}
%    \begin{macrocode}
\def\HOLOGO@Span#1#2{%
  \HCode{<span class="HoLogo-#1">}%
  #2%
  \HCode{</span>}%
}
%    \end{macrocode}
%    \end{macro}
%
% \subsubsection{Text subscript}
%
%    \begin{macro}{\HOLOGO@SubScript}%
%    \begin{macrocode}
\def\HOLOGO@SubScript#1{%
  \ltx@IfUndefined{textsubscript}{%
    \ltx@IfUndefined{text}{%
      \ltx@mbox{%
        \mathsurround=0pt\relax
        $%
          _{%
            \ltx@IfUndefined{sf@size}{%
              \mathrm{#1}%
            }{%
              \mbox{%
                \fontsize\sf@size{0pt}\selectfont
                #1%
              }%
            }%
          }%
        $%
      }%
    }{%
      \ltx@mbox{%
        \mathsurround=0pt\relax
        $_{\text{#1}}$%
      }%
    }%
  }{%
    \textsubscript{#1}%
  }%
}
%    \end{macrocode}
%    \end{macro}
%
% \subsection{\hologo{TeX} and friends}
%
% \subsubsection{\hologo{TeX}}
%
%    \begin{macro}{\HoLogo@TeX}
%    Source: \hologo{LaTeX} kernel.
%    \begin{macrocode}
\def\HoLogo@TeX#1{%
  T\kern-.1667em\lower.5ex\hbox{E}\kern-.125emX\HOLOGO@SpaceFactor
}
%    \end{macrocode}
%    \end{macro}
%    \begin{macro}{\HoLogoHtml@TeX}
%    \begin{macrocode}
\def\HoLogoHtml@TeX#1{%
  \HoLogoCss@TeX
  \HOLOGO@Span{TeX}{%
    T%
    \HOLOGO@Span{e}{%
      E%
    }%
    X%
  }%
}
%    \end{macrocode}
%    \end{macro}
%    \begin{macro}{\HoLogoCss@TeX}
%    \begin{macrocode}
\def\HoLogoCss@TeX{%
  \Css{%
    span.HoLogo-TeX span.HoLogo-e{%
      position:relative;%
      top:.5ex;%
      margin-left:-.1667em;%
      margin-right:-.125em;%
    }%
  }%
  \Css{%
    a span.HoLogo-TeX span.HoLogo-e{%
      text-decoration:none;%
    }%
  }%
  \global\let\HoLogoCss@TeX\relax
}
%    \end{macrocode}
%    \end{macro}
%
% \subsubsection{\hologo{plainTeX}}
%
%    \begin{macro}{\HoLogo@plainTeX@space}
%    Source: ``The \hologo{TeX}book''
%    \begin{macrocode}
\def\HoLogo@plainTeX@space#1{%
  \HOLOGO@mbox{#1{p}{P}lain}\HOLOGO@space\hologo{TeX}%
}
%    \end{macrocode}
%    \end{macro}
%    \begin{macro}{\HoLogoCs@plainTeX@space}
%    \begin{macrocode}
\def\HoLogoCs@plainTeX@space#1{#1{p}{P}lain TeX}%
%    \end{macrocode}
%    \end{macro}
%    \begin{macro}{\HoLogoBkm@plainTeX@space}
%    \begin{macrocode}
\def\HoLogoBkm@plainTeX@space#1{%
  #1{p}{P}lain \hologo{TeX}%
}
%    \end{macrocode}
%    \end{macro}
%    \begin{macro}{\HoLogoHtml@plainTeX@space}
%    \begin{macrocode}
\def\HoLogoHtml@plainTeX@space#1{%
  #1{p}{P}lain \hologo{TeX}%
}
%    \end{macrocode}
%    \end{macro}
%
%    \begin{macro}{\HoLogo@plainTeX@hyphen}
%    \begin{macrocode}
\def\HoLogo@plainTeX@hyphen#1{%
  \HOLOGO@mbox{#1{p}{P}lain}\HOLOGO@hyphen\hologo{TeX}%
}
%    \end{macrocode}
%    \end{macro}
%    \begin{macro}{\HoLogoCs@plainTeX@hyphen}
%    \begin{macrocode}
\def\HoLogoCs@plainTeX@hyphen#1{#1{p}{P}lain-TeX}
%    \end{macrocode}
%    \end{macro}
%    \begin{macro}{\HoLogoBkm@plainTeX@hyphen}
%    \begin{macrocode}
\def\HoLogoBkm@plainTeX@hyphen#1{%
  #1{p}{P}lain-\hologo{TeX}%
}
%    \end{macrocode}
%    \end{macro}
%    \begin{macro}{\HoLogoHtml@plainTeX@hyphen}
%    \begin{macrocode}
\def\HoLogoHtml@plainTeX@hyphen#1{%
  #1{p}{P}lain-\hologo{TeX}%
}
%    \end{macrocode}
%    \end{macro}
%
%    \begin{macro}{\HoLogo@plainTeX@runtogether}
%    \begin{macrocode}
\def\HoLogo@plainTeX@runtogether#1{%
  \HOLOGO@mbox{#1{p}{P}lain\hologo{TeX}}%
}
%    \end{macrocode}
%    \end{macro}
%    \begin{macro}{\HoLogoCs@plainTeX@runtogether}
%    \begin{macrocode}
\def\HoLogoCs@plainTeX@runtogether#1{#1{p}{P}lainTeX}
%    \end{macrocode}
%    \end{macro}
%    \begin{macro}{\HoLogoBkm@plainTeX@runtogether}
%    \begin{macrocode}
\def\HoLogoBkm@plainTeX@runtogether#1{%
  #1{p}{P}lain\hologo{TeX}%
}
%    \end{macrocode}
%    \end{macro}
%    \begin{macro}{\HoLogoHtml@plainTeX@runtogether}
%    \begin{macrocode}
\def\HoLogoHtml@plainTeX@runtogether#1{%
  #1{p}{P}lain\hologo{TeX}%
}
%    \end{macrocode}
%    \end{macro}
%
%    \begin{macro}{\HoLogo@plainTeX}
%    \begin{macrocode}
\def\HoLogo@plainTeX{\HoLogo@plainTeX@space}
%    \end{macrocode}
%    \end{macro}
%    \begin{macro}{\HoLogoCs@plainTeX}
%    \begin{macrocode}
\def\HoLogoCs@plainTeX{\HoLogoCs@plainTeX@space}
%    \end{macrocode}
%    \end{macro}
%    \begin{macro}{\HoLogoBkm@plainTeX}
%    \begin{macrocode}
\def\HoLogoBkm@plainTeX{\HoLogoBkm@plainTeX@space}
%    \end{macrocode}
%    \end{macro}
%    \begin{macro}{\HoLogoHtml@plainTeX}
%    \begin{macrocode}
\def\HoLogoHtml@plainTeX{\HoLogoHtml@plainTeX@space}
%    \end{macrocode}
%    \end{macro}
%
% \subsubsection{\hologo{LaTeX}}
%
%    Source: \hologo{LaTeX} kernel.
%\begin{quote}
%\begin{verbatim}
%\DeclareRobustCommand{\LaTeX}{%
%  L%
%  \kern-.36em%
%  {%
%    \sbox\z@ T%
%    \vbox to\ht\z@{%
%      \hbox{%
%        \check@mathfonts
%        \fontsize\sf@size\z@
%        \math@fontsfalse
%        \selectfont
%        A%
%      }%
%      \vss
%    }%
%  }%
%  \kern-.15em%
%  \TeX
%}
%\end{verbatim}
%\end{quote}
%
%    \begin{macro}{\HoLogo@La}
%    \begin{macrocode}
\def\HoLogo@La#1{%
  L%
  \kern-.36em%
  \begingroup
    \setbox\ltx@zero\hbox{T}%
    \vbox to\ht\ltx@zero{%
      \hbox{%
        \ltx@ifundefined{check@mathfonts}{%
          \csname sevenrm\endcsname
        }{%
          \check@mathfonts
          \fontsize\sf@size{0pt}%
          \math@fontsfalse\selectfont
        }%
        A%
      }%
      \vss
    }%
  \endgroup
}
%    \end{macrocode}
%    \end{macro}
%
%    \begin{macro}{\HoLogo@LaTeX}
%    Source: \hologo{LaTeX} kernel.
%    \begin{macrocode}
\def\HoLogo@LaTeX#1{%
  \hologo{La}%
  \kern-.15em%
  \hologo{TeX}%
}
%    \end{macrocode}
%    \end{macro}
%    \begin{macro}{\HoLogoHtml@LaTeX}
%    \begin{macrocode}
\def\HoLogoHtml@LaTeX#1{%
  \HoLogoCss@LaTeX
  \HOLOGO@Span{LaTeX}{%
    L%
    \HOLOGO@Span{a}{%
      A%
    }%
    \hologo{TeX}%
  }%
}
%    \end{macrocode}
%    \end{macro}
%    \begin{macro}{\HoLogoCss@LaTeX}
%    \begin{macrocode}
\def\HoLogoCss@LaTeX{%
  \Css{%
    span.HoLogo-LaTeX span.HoLogo-a{%
      position:relative;%
      top:-.5ex;%
      margin-left:-.36em;%
      margin-right:-.15em;%
      font-size:85\%;%
    }%
  }%
  \global\let\HoLogoCss@LaTeX\relax
}
%    \end{macrocode}
%    \end{macro}
%
% \subsubsection{\hologo{(La)TeX}}
%
%    \begin{macro}{\HoLogo@LaTeXTeX}
%    The kerning around the parentheses is taken
%    from package \xpackage{dtklogos} \cite{dtklogos}.
%\begin{quote}
%\begin{verbatim}
%\DeclareRobustCommand{\LaTeXTeX}{%
%  (%
%  \kern-.15em%
%  L%
%  \kern-.36em%
%  {%
%    \sbox\z@ T%
%    \vbox to\ht0{%
%      \hbox{%
%        $\m@th$%
%        \csname S@\f@size\endcsname
%        \fontsize\sf@size\z@
%        \math@fontsfalse
%        \selectfont
%        A%
%      }%
%      \vss
%    }%
%  }%
%  \kern-.2em%
%  )%
%  \kern-.15em%
%  \TeX
%}
%\end{verbatim}
%\end{quote}
%    \begin{macrocode}
\def\HoLogo@LaTeXTeX#1{%
  (%
  \kern-.15em%
  \hologo{La}%
  \kern-.2em%
  )%
  \kern-.15em%
  \hologo{TeX}%
}
%    \end{macrocode}
%    \end{macro}
%    \begin{macro}{\HoLogoBkm@LaTeXTeX}
%    \begin{macrocode}
\def\HoLogoBkm@LaTeXTeX#1{(La)TeX}
%    \end{macrocode}
%    \end{macro}
%
%    \begin{macro}{\HoLogo@(La)TeX}
%    \begin{macrocode}
\expandafter
\let\csname HoLogo@(La)TeX\endcsname\HoLogo@LaTeXTeX
%    \end{macrocode}
%    \end{macro}
%    \begin{macro}{\HoLogoBkm@(La)TeX}
%    \begin{macrocode}
\expandafter
\let\csname HoLogoBkm@(La)TeX\endcsname\HoLogoBkm@LaTeXTeX
%    \end{macrocode}
%    \end{macro}
%    \begin{macro}{\HoLogoHtml@LaTeXTeX}
%    \begin{macrocode}
\def\HoLogoHtml@LaTeXTeX#1{%
  \HoLogoCss@LaTeXTeX
  \HOLOGO@Span{LaTeXTeX}{%
    (%
    \HOLOGO@Span{L}{L}%
    \HOLOGO@Span{a}{A}%
    \HOLOGO@Span{ParenRight}{)}%
    \hologo{TeX}%
  }%
}
%    \end{macrocode}
%    \end{macro}
%    \begin{macro}{\HoLogoHtml@(La)TeX}
%    Kerning after opening parentheses and before closing parentheses
%    is $-0.1$\,em. The original values $-0.15$\,em
%    looked too ugly for a serif font.
%    \begin{macrocode}
\expandafter
\let\csname HoLogoHtml@(La)TeX\endcsname\HoLogoHtml@LaTeXTeX
%    \end{macrocode}
%    \end{macro}
%    \begin{macro}{\HoLogoCss@LaTeXTeX}
%    \begin{macrocode}
\def\HoLogoCss@LaTeXTeX{%
  \Css{%
    span.HoLogo-LaTeXTeX span.HoLogo-L{%
      margin-left:-.1em;%
    }%
  }%
  \Css{%
    span.HoLogo-LaTeXTeX span.HoLogo-a{%
      position:relative;%
      top:-.5ex;%
      margin-left:-.36em;%
      margin-right:-.1em;%
      font-size:85\%;%
    }%
  }%
  \Css{%
    span.HoLogo-LaTeXTeX span.HoLogo-ParenRight{%
      margin-right:-.15em;%
    }%
  }%
  \global\let\HoLogoCss@LaTeXTeX\relax
}
%    \end{macrocode}
%    \end{macro}
%
% \subsubsection{\hologo{LaTeXe}}
%
%    \begin{macro}{\HoLogo@LaTeXe}
%    Source: \hologo{LaTeX} kernel
%    \begin{macrocode}
\def\HoLogo@LaTeXe#1{%
  \hologo{LaTeX}%
  \kern.15em%
  \hbox{%
    \HOLOGO@MathSetup
    2%
    $_{\textstyle\varepsilon}$%
  }%
}
%    \end{macrocode}
%    \end{macro}
%
%    \begin{macro}{\HoLogoCs@LaTeXe}
%    \begin{macrocode}
\ifnum64=`\^^^^0040\relax % test for big chars of LuaTeX/XeTeX
  \catcode`\$=9 %
  \catcode`\&=14 %
\else
  \catcode`\$=14 %
  \catcode`\&=9 %
\fi
\def\HoLogoCs@LaTeXe#1{%
  LaTeX2%
$ \string ^^^^0395%
& e%
}%
\catcode`\$=3 %
\catcode`\&=4 %
%    \end{macrocode}
%    \end{macro}
%
%    \begin{macro}{\HoLogoBkm@LaTeXe}
%    \begin{macrocode}
\def\HoLogoBkm@LaTeXe#1{%
  \hologo{LaTeX}%
  2%
  \HOLOGO@PdfdocUnicode{e}{\textepsilon}%
}
%    \end{macrocode}
%    \end{macro}
%
%    \begin{macro}{\HoLogoHtml@LaTeXe}
%    \begin{macrocode}
\def\HoLogoHtml@LaTeXe#1{%
  \HoLogoCss@LaTeXe
  \HOLOGO@Span{LaTeX2e}{%
    \hologo{LaTeX}%
    \HOLOGO@Span{2}{2}%
    \HOLOGO@Span{e}{%
      \HOLOGO@MathSetup
      \ensuremath{\textstyle\varepsilon}%
    }%
  }%
}
%    \end{macrocode}
%    \end{macro}
%    \begin{macro}{\HoLogoCss@LaTeXe}
%    \begin{macrocode}
\def\HoLogoCss@LaTeXe{%
  \Css{%
    span.HoLogo-LaTeX2e span.HoLogo-2{%
      padding-left:.15em;%
    }%
  }%
  \Css{%
    span.HoLogo-LaTeX2e span.HoLogo-e{%
      position:relative;%
      top:.35ex;%
      text-decoration:none;%
    }%
  }%
  \global\let\HoLogoCss@LaTeXe\relax
}
%    \end{macrocode}
%    \end{macro}
%
%    \begin{macro}{\HoLogo@LaTeX2e}
%    \begin{macrocode}
\expandafter
\let\csname HoLogo@LaTeX2e\endcsname\HoLogo@LaTeXe
%    \end{macrocode}
%    \end{macro}
%    \begin{macro}{\HoLogoCs@LaTeX2e}
%    \begin{macrocode}
\expandafter
\let\csname HoLogoCs@LaTeX2e\endcsname\HoLogoCs@LaTeXe
%    \end{macrocode}
%    \end{macro}
%    \begin{macro}{\HoLogoBkm@LaTeX2e}
%    \begin{macrocode}
\expandafter
\let\csname HoLogoBkm@LaTeX2e\endcsname\HoLogoBkm@LaTeXe
%    \end{macrocode}
%    \end{macro}
%    \begin{macro}{\HoLogoHtml@LaTeX2e}
%    \begin{macrocode}
\expandafter
\let\csname HoLogoHtml@LaTeX2e\endcsname\HoLogoHtml@LaTeXe
%    \end{macrocode}
%    \end{macro}
%
% \subsubsection{\hologo{LaTeX3}}
%
%    \begin{macro}{\HoLogo@LaTeX3}
%    Source: \hologo{LaTeX} kernel
%    \begin{macrocode}
\expandafter\def\csname HoLogo@LaTeX3\endcsname#1{%
  \hologo{LaTeX}%
  3%
}
%    \end{macrocode}
%    \end{macro}
%
%    \begin{macro}{\HoLogoBkm@LaTeX3}
%    \begin{macrocode}
\expandafter\def\csname HoLogoBkm@LaTeX3\endcsname#1{%
  \hologo{LaTeX}%
  3%
}
%    \end{macrocode}
%    \end{macro}
%    \begin{macro}{\HoLogoHtml@LaTeX3}
%    \begin{macrocode}
\expandafter
\let\csname HoLogoHtml@LaTeX3\expandafter\endcsname
\csname HoLogo@LaTeX3\endcsname
%    \end{macrocode}
%    \end{macro}
%
% \subsubsection{\hologo{LaTeXML}}
%
%    \begin{macro}{\HoLogo@LaTeXML}
%    \begin{macrocode}
\def\HoLogo@LaTeXML#1{%
  \HOLOGO@mbox{%
    \hologo{La}%
    \kern-.15em%
    T%
    \kern-.1667em%
    \lower.5ex\hbox{E}%
    \kern-.125em%
    \HoLogoFont@font{LaTeXML}{sc}{xml}%
  }%
}
%    \end{macrocode}
%    \end{macro}
%    \begin{macro}{\HoLogoHtml@pdfLaTeX}
%    \begin{macrocode}
\def\HoLogoHtml@LaTeXML#1{%
  \HOLOGO@Span{LaTeXML}{%
    \HoLogoCss@LaTeX
    \HoLogoCss@TeX
    \HOLOGO@Span{LaTeX}{%
      L%
      \HOLOGO@Span{a}{%
        A%
      }%
    }%
    \HOLOGO@Span{TeX}{%
      T%
      \HOLOGO@Span{e}{%
        E%
      }%
    }%
    \HCode{<span style="font-variant: small-caps;">}%
    xml%
    \HCode{</span>}%
  }%
}
%    \end{macrocode}
%    \end{macro}
%
% \subsubsection{\hologo{eTeX}}
%
%    \begin{macro}{\HoLogo@eTeX}
%    Source: package \xpackage{etex}
%    \begin{macrocode}
\def\HoLogo@eTeX#1{%
  \ltx@mbox{%
    \HOLOGO@MathSetup
    $\varepsilon$%
    -%
    \HOLOGO@NegativeKerning{-T,T-,To}%
    \hologo{TeX}%
  }%
}
%    \end{macrocode}
%    \end{macro}
%    \begin{macro}{\HoLogoCs@eTeX}
%    \begin{macrocode}
\ifnum64=`\^^^^0040\relax % test for big chars of LuaTeX/XeTeX
  \catcode`\$=9 %
  \catcode`\&=14 %
\else
  \catcode`\$=14 %
  \catcode`\&=9 %
\fi
\def\HoLogoCs@eTeX#1{%
$ #1{\string ^^^^0395}{\string ^^^^03b5}%
& #1{e}{E}%
  TeX%
}%
\catcode`\$=3 %
\catcode`\&=4 %
%    \end{macrocode}
%    \end{macro}
%    \begin{macro}{\HoLogoBkm@eTeX}
%    \begin{macrocode}
\def\HoLogoBkm@eTeX#1{%
  \HOLOGO@PdfdocUnicode{#1{e}{E}}{\textepsilon}%
  -%
  \hologo{TeX}%
}
%    \end{macrocode}
%    \end{macro}
%    \begin{macro}{\HoLogoHtml@eTeX}
%    \begin{macrocode}
\def\HoLogoHtml@eTeX#1{%
  \ltx@mbox{%
    \HOLOGO@MathSetup
    $\varepsilon$%
    -%
    \hologo{TeX}%
  }%
}
%    \end{macrocode}
%    \end{macro}
%
% \subsubsection{\hologo{iniTeX}}
%
%    \begin{macro}{\HoLogo@iniTeX}
%    \begin{macrocode}
\def\HoLogo@iniTeX#1{%
  \HOLOGO@mbox{%
    #1{i}{I}ni\hologo{TeX}%
  }%
}
%    \end{macrocode}
%    \end{macro}
%    \begin{macro}{\HoLogoCs@iniTeX}
%    \begin{macrocode}
\def\HoLogoCs@iniTeX#1{#1{i}{I}niTeX}
%    \end{macrocode}
%    \end{macro}
%    \begin{macro}{\HoLogoBkm@iniTeX}
%    \begin{macrocode}
\def\HoLogoBkm@iniTeX#1{%
  #1{i}{I}ni\hologo{TeX}%
}
%    \end{macrocode}
%    \end{macro}
%    \begin{macro}{\HoLogoHtml@iniTeX}
%    \begin{macrocode}
\let\HoLogoHtml@iniTeX\HoLogo@iniTeX
%    \end{macrocode}
%    \end{macro}
%
% \subsubsection{\hologo{virTeX}}
%
%    \begin{macro}{\HoLogo@virTeX}
%    \begin{macrocode}
\def\HoLogo@virTeX#1{%
  \HOLOGO@mbox{%
    #1{v}{V}ir\hologo{TeX}%
  }%
}
%    \end{macrocode}
%    \end{macro}
%    \begin{macro}{\HoLogoCs@virTeX}
%    \begin{macrocode}
\def\HoLogoCs@virTeX#1{#1{v}{V}irTeX}
%    \end{macrocode}
%    \end{macro}
%    \begin{macro}{\HoLogoBkm@virTeX}
%    \begin{macrocode}
\def\HoLogoBkm@virTeX#1{%
  #1{v}{V}ir\hologo{TeX}%
}
%    \end{macrocode}
%    \end{macro}
%    \begin{macro}{\HoLogoHtml@virTeX}
%    \begin{macrocode}
\let\HoLogoHtml@virTeX\HoLogo@virTeX
%    \end{macrocode}
%    \end{macro}
%
% \subsubsection{\hologo{SliTeX}}
%
% \paragraph{Definitions of the three variants.}
%
%    \begin{macro}{\HoLogo@SLiTeX@lift}
%    \begin{macrocode}
\def\HoLogo@SLiTeX@lift#1{%
  \HoLogoFont@font{SliTeX}{rm}{%
    S%
    \kern-.06em%
    L%
    \kern-.18em%
    \raise.32ex\hbox{\HoLogoFont@font{SliTeX}{sc}{i}}%
    \HOLOGO@discretionary
    \kern-.06em%
    \hologo{TeX}%
  }%
}
%    \end{macrocode}
%    \end{macro}
%    \begin{macro}{\HoLogoBkm@SLiTeX@lift}
%    \begin{macrocode}
\def\HoLogoBkm@SLiTeX@lift#1{SLiTeX}
%    \end{macrocode}
%    \end{macro}
%    \begin{macro}{\HoLogoHtml@SLiTeX@lift}
%    \begin{macrocode}
\def\HoLogoHtml@SLiTeX@lift#1{%
  \HoLogoCss@SLiTeX@lift
  \HOLOGO@Span{SLiTeX-lift}{%
    \HoLogoFont@font{SliTeX}{rm}{%
      S%
      \HOLOGO@Span{L}{L}%
      \HOLOGO@Span{i}{i}%
      \hologo{TeX}%
    }%
  }%
}
%    \end{macrocode}
%    \end{macro}
%    \begin{macro}{\HoLogoCss@SLiTeX@lift}
%    \begin{macrocode}
\def\HoLogoCss@SLiTeX@lift{%
  \Css{%
    span.HoLogo-SLiTeX-lift span.HoLogo-L{%
      margin-left:-.06em;%
      margin-right:-.18em;%
    }%
  }%
  \Css{%
    span.HoLogo-SLiTeX-lift span.HoLogo-i{%
      position:relative;%
      top:-.32ex;%
      margin-right:-.06em;%
      font-variant:small-caps;%
    }%
  }%
  \global\let\HoLogoCss@SLiTeX@lift\relax
}
%    \end{macrocode}
%    \end{macro}
%
%    \begin{macro}{\HoLogo@SliTeX@simple}
%    \begin{macrocode}
\def\HoLogo@SliTeX@simple#1{%
  \HoLogoFont@font{SliTeX}{rm}{%
    \ltx@mbox{%
      \HoLogoFont@font{SliTeX}{sc}{Sli}%
    }%
    \HOLOGO@discretionary
    \hologo{TeX}%
  }%
}
%    \end{macrocode}
%    \end{macro}
%    \begin{macro}{\HoLogoBkm@SliTeX@simple}
%    \begin{macrocode}
\def\HoLogoBkm@SliTeX@simple#1{SliTeX}
%    \end{macrocode}
%    \end{macro}
%    \begin{macro}{\HoLogoHtml@SliTeX@simple}
%    \begin{macrocode}
\let\HoLogoHtml@SliTeX@simple\HoLogo@SliTeX@simple
%    \end{macrocode}
%    \end{macro}
%
%    \begin{macro}{\HoLogo@SliTeX@narrow}
%    \begin{macrocode}
\def\HoLogo@SliTeX@narrow#1{%
  \HoLogoFont@font{SliTeX}{rm}{%
    \ltx@mbox{%
      S%
      \kern-.06em%
      \HoLogoFont@font{SliTeX}{sc}{%
        l%
        \kern-.035em%
        i%
      }%
    }%
    \HOLOGO@discretionary
    \kern-.06em%
    \hologo{TeX}%
  }%
}
%    \end{macrocode}
%    \end{macro}
%    \begin{macro}{\HoLogoBkm@SliTeX@narrow}
%    \begin{macrocode}
\def\HoLogoBkm@SliTeX@narrow#1{SliTeX}
%    \end{macrocode}
%    \end{macro}
%    \begin{macro}{\HoLogoHtml@SliTeX@narrow}
%    \begin{macrocode}
\def\HoLogoHtml@SliTeX@narrow#1{%
  \HoLogoCss@SliTeX@narrow
  \HOLOGO@Span{SliTeX-narrow}{%
    \HoLogoFont@font{SliTeX}{rm}{%
      S%
        \HOLOGO@Span{l}{l}%
        \HOLOGO@Span{i}{i}%
      \hologo{TeX}%
    }%
  }%
}
%    \end{macrocode}
%    \end{macro}
%    \begin{macro}{\HoLogoCss@SliTeX@narrow}
%    \begin{macrocode}
\def\HoLogoCss@SliTeX@narrow{%
  \Css{%
    span.HoLogo-SliTeX-narrow span.HoLogo-l{%
      margin-left:-.06em;%
      margin-right:-.035em;%
      font-variant:small-caps;%
    }%
  }%
  \Css{%
    span.HoLogo-SliTeX-narrow span.HoLogo-i{%
      margin-right:-.06em;%
      font-variant:small-caps;%
    }%
  }%
  \global\let\HoLogoCss@SliTeX@narrow\relax
}
%    \end{macrocode}
%    \end{macro}
%
% \paragraph{Macro set completion.}
%
%    \begin{macro}{\HoLogo@SLiTeX@simple}
%    \begin{macrocode}
\def\HoLogo@SLiTeX@simple{\HoLogo@SliTeX@simple}
%    \end{macrocode}
%    \end{macro}
%    \begin{macro}{\HoLogoBkm@SLiTeX@simple}
%    \begin{macrocode}
\def\HoLogoBkm@SLiTeX@simple{\HoLogoBkm@SliTeX@simple}
%    \end{macrocode}
%    \end{macro}
%    \begin{macro}{\HoLogoHtml@SLiTeX@simple}
%    \begin{macrocode}
\def\HoLogoHtml@SLiTeX@simple{\HoLogoHtml@SliTeX@simple}
%    \end{macrocode}
%    \end{macro}
%
%    \begin{macro}{\HoLogo@SLiTeX@narrow}
%    \begin{macrocode}
\def\HoLogo@SLiTeX@narrow{\HoLogo@SliTeX@narrow}
%    \end{macrocode}
%    \end{macro}
%    \begin{macro}{\HoLogoBkm@SLiTeX@narrow}
%    \begin{macrocode}
\def\HoLogoBkm@SLiTeX@narrow{\HoLogoBkm@SliTeX@narrow}
%    \end{macrocode}
%    \end{macro}
%    \begin{macro}{\HoLogoHtml@SLiTeX@narrow}
%    \begin{macrocode}
\def\HoLogoHtml@SLiTeX@narrow{\HoLogoHtml@SliTeX@narrow}
%    \end{macrocode}
%    \end{macro}
%
%    \begin{macro}{\HoLogo@SliTeX@lift}
%    \begin{macrocode}
\def\HoLogo@SliTeX@lift{\HoLogo@SLiTeX@lift}
%    \end{macrocode}
%    \end{macro}
%    \begin{macro}{\HoLogoBkm@SliTeX@lift}
%    \begin{macrocode}
\def\HoLogoBkm@SliTeX@lift{\HoLogoBkm@SLiTeX@lift}
%    \end{macrocode}
%    \end{macro}
%    \begin{macro}{\HoLogoHtml@SliTeX@lift}
%    \begin{macrocode}
\def\HoLogoHtml@SliTeX@lift{\HoLogoHtml@SLiTeX@lift}
%    \end{macrocode}
%    \end{macro}
%
% \paragraph{Defaults.}
%
%    \begin{macro}{\HoLogo@SLiTeX}
%    \begin{macrocode}
\def\HoLogo@SLiTeX{\HoLogo@SLiTeX@lift}
%    \end{macrocode}
%    \end{macro}
%    \begin{macro}{\HoLogoBkm@SLiTeX}
%    \begin{macrocode}
\def\HoLogoBkm@SLiTeX{\HoLogoBkm@SLiTeX@lift}
%    \end{macrocode}
%    \end{macro}
%    \begin{macro}{\HoLogoHtml@SLiTeX}
%    \begin{macrocode}
\def\HoLogoHtml@SLiTeX{\HoLogoHtml@SLiTeX@lift}
%    \end{macrocode}
%    \end{macro}
%
%    \begin{macro}{\HoLogo@SliTeX}
%    \begin{macrocode}
\def\HoLogo@SliTeX{\HoLogo@SliTeX@narrow}
%    \end{macrocode}
%    \end{macro}
%    \begin{macro}{\HoLogoBkm@SliTeX}
%    \begin{macrocode}
\def\HoLogoBkm@SliTeX{\HoLogoBkm@SliTeX@narrow}
%    \end{macrocode}
%    \end{macro}
%    \begin{macro}{\HoLogoHtml@SliTeX}
%    \begin{macrocode}
\def\HoLogoHtml@SliTeX{\HoLogoHtml@SliTeX@narrow}
%    \end{macrocode}
%    \end{macro}
%
% \subsubsection{\hologo{LuaTeX}}
%
%    \begin{macro}{\HoLogo@LuaTeX}
%    The kerning is an idea of Hans Hagen, see mailing list
%    `luatex at tug dot org' in March 2010.
%    \begin{macrocode}
\def\HoLogo@LuaTeX#1{%
  \HOLOGO@mbox{%
    Lua%
    \HOLOGO@NegativeKerning{aT,oT,To}%
    \hologo{TeX}%
  }%
}
%    \end{macrocode}
%    \end{macro}
%    \begin{macro}{\HoLogoHtml@LuaTeX}
%    \begin{macrocode}
\let\HoLogoHtml@LuaTeX\HoLogo@LuaTeX
%    \end{macrocode}
%    \end{macro}
%
% \subsubsection{\hologo{LuaLaTeX}}
%
%    \begin{macro}{\HoLogo@LuaLaTeX}
%    \begin{macrocode}
\def\HoLogo@LuaLaTeX#1{%
  \HOLOGO@mbox{%
    Lua%
    \hologo{LaTeX}%
  }%
}
%    \end{macrocode}
%    \end{macro}
%    \begin{macro}{\HoLogoHtml@LuaLaTeX}
%    \begin{macrocode}
\let\HoLogoHtml@LuaLaTeX\HoLogo@LuaLaTeX
%    \end{macrocode}
%    \end{macro}
%
% \subsubsection{\hologo{XeTeX}, \hologo{XeLaTeX}}
%
%    \begin{macro}{\HOLOGO@IfCharExists}
%    \begin{macrocode}
\ifluatex
  \ifnum\luatexversion<36 %
  \else
    \def\HOLOGO@IfCharExists#1{%
      \ifnum
        \directlua{%
           if luaotfload and luaotfload.aux then
             if luaotfload.aux.font_has_glyph(%
                    font.current(), \number#1) then % 	 
	       tex.print("1") % 	 
	     end % 	 
	   elseif font and font.fonts and font.current then %
            local f = font.fonts[font.current()]%
            if f.characters and f.characters[\number#1] then %
              tex.print("1")%
            end %
          end%
        }0=\ltx@zero
        \expandafter\ltx@secondoftwo
      \else
        \expandafter\ltx@firstoftwo
      \fi
    }%
  \fi
\fi
\ltx@IfUndefined{HOLOGO@IfCharExists}{%
  \def\HOLOGO@@IfCharExists#1{%
    \begingroup
      \tracinglostchars=\ltx@zero
      \setbox\ltx@zero=\hbox{%
        \kern7sp\char#1\relax
        \ifnum\lastkern>\ltx@zero
          \expandafter\aftergroup\csname iffalse\endcsname
        \else
          \expandafter\aftergroup\csname iftrue\endcsname
        \fi
      }%
      % \if{true|false} from \aftergroup
      \endgroup
      \expandafter\ltx@firstoftwo
    \else
      \endgroup
      \expandafter\ltx@secondoftwo
    \fi
  }%
  \ifxetex
    \ltx@IfUndefined{XeTeXfonttype}{}{%
      \ltx@IfUndefined{XeTeXcharglyph}{}{%
        \def\HOLOGO@IfCharExists#1{%
          \ifnum\XeTeXfonttype\font>\ltx@zero
            \expandafter\ltx@firstofthree
          \else
            \expandafter\ltx@gobble
          \fi
          {%
            \ifnum\XeTeXcharglyph#1>\ltx@zero
              \expandafter\ltx@firstoftwo
            \else
              \expandafter\ltx@secondoftwo
            \fi
          }%
          \HOLOGO@@IfCharExists{#1}%
        }%
      }%
    }%
  \fi
}{}
\ltx@ifundefined{HOLOGO@IfCharExists}{%
  \ifnum64=`\^^^^0040\relax % test for big chars of LuaTeX/XeTeX
    \let\HOLOGO@IfCharExists\HOLOGO@@IfCharExists
  \else
    \def\HOLOGO@IfCharExists#1{%
      \ifnum#1>255 %
        \expandafter\ltx@fourthoffour
      \fi
      \HOLOGO@@IfCharExists{#1}%
    }%
  \fi
}{}
%    \end{macrocode}
%    \end{macro}
%
%    \begin{macro}{\HoLogo@Xe}
%    Source: package \xpackage{dtklogos}
%    \begin{macrocode}
\def\HoLogo@Xe#1{%
  X%
  \kern-.1em\relax
  \HOLOGO@IfCharExists{"018E}{%
    \lower.5ex\hbox{\char"018E}%
  }{%
    \chardef\HOLOGO@choice=\ltx@zero
    \ifdim\fontdimen\ltx@one\font>0pt %
      \ltx@IfUndefined{rotatebox}{%
        \ltx@IfUndefined{pgftext}{%
          \ltx@IfUndefined{psscalebox}{%
            \ltx@IfUndefined{HOLOGO@ScaleBox@\hologoDriver}{%
            }{%
              \chardef\HOLOGO@choice=4 %
            }%
          }{%
            \chardef\HOLOGO@choice=3 %
          }%
        }{%
          \chardef\HOLOGO@choice=2 %
        }%
      }{%
        \chardef\HOLOGO@choice=1 %
      }%
      \ifcase\HOLOGO@choice
        \HOLOGO@WarningUnsupportedDriver{Xe}%
        e%
      \or % 1: \rotatebox
        \begingroup
          \setbox\ltx@zero\hbox{\rotatebox{180}{E}}%
          \ltx@LocDimenA=\dp\ltx@zero
          \advance\ltx@LocDimenA by -.5ex\relax
          \raise\ltx@LocDimenA\box\ltx@zero
        \endgroup
      \or % 2: \pgftext
        \lower.5ex\hbox{%
          \pgfpicture
            \pgftext[rotate=180]{E}%
          \endpgfpicture
        }%
      \or % 3: \psscalebox
        \begingroup
          \setbox\ltx@zero\hbox{\psscalebox{-1 -1}{E}}%
          \ltx@LocDimenA=\dp\ltx@zero
          \advance\ltx@LocDimenA by -.5ex\relax
          \raise\ltx@LocDimenA\box\ltx@zero
        \endgroup
      \or % 4: \HOLOGO@PointReflectBox
        \lower.5ex\hbox{\HOLOGO@PointReflectBox{E}}%
      \else
        \@PackageError{hologo}{Internal error (choice/it}\@ehc
      \fi
    \else
      \ltx@IfUndefined{reflectbox}{%
        \ltx@IfUndefined{pgftext}{%
          \ltx@IfUndefined{psscalebox}{%
            \ltx@IfUndefined{HOLOGO@ScaleBox@\hologoDriver}{%
            }{%
              \chardef\HOLOGO@choice=4 %
            }%
          }{%
            \chardef\HOLOGO@choice=3 %
          }%
        }{%
          \chardef\HOLOGO@choice=2 %
        }%
      }{%
        \chardef\HOLOGO@choice=1 %
      }%
      \ifcase\HOLOGO@choice
        \HOLOGO@WarningUnsupportedDriver{Xe}%
        e%
      \or % 1: reflectbox
        \lower.5ex\hbox{%
          \reflectbox{E}%
        }%
      \or % 2: \pgftext
        \lower.5ex\hbox{%
          \pgfpicture
            \pgftransformxscale{-1}%
            \pgftext{E}%
          \endpgfpicture
        }%
      \or % 3: \psscalebox
        \lower.5ex\hbox{%
          \psscalebox{-1 1}{E}%
        }%
      \or % 4: \HOLOGO@Reflectbox
        \lower.5ex\hbox{%
          \HOLOGO@ReflectBox{E}%
        }%
      \else
        \@PackageError{hologo}{Internal error (choice/up)}\@ehc
      \fi
    \fi
  }%
}
%    \end{macrocode}
%    \end{macro}
%    \begin{macro}{\HoLogoHtml@Xe}
%    \begin{macrocode}
\def\HoLogoHtml@Xe#1{%
  \HoLogoCss@Xe
  \HOLOGO@Span{Xe}{%
    X%
    \HOLOGO@Span{e}{%
      \HCode{&\ltx@hashchar x018e;}%
    }%
  }%
}
%    \end{macrocode}
%    \end{macro}
%    \begin{macro}{\HoLogoCss@Xe}
%    \begin{macrocode}
\def\HoLogoCss@Xe{%
  \Css{%
    span.HoLogo-Xe span.HoLogo-e{%
      position:relative;%
      top:.5ex;%
      left-margin:-.1em;%
    }%
  }%
  \global\let\HoLogoCss@Xe\relax
}
%    \end{macrocode}
%    \end{macro}
%
%    \begin{macro}{\HoLogo@XeTeX}
%    \begin{macrocode}
\def\HoLogo@XeTeX#1{%
  \hologo{Xe}%
  \kern-.15em\relax
  \hologo{TeX}%
}
%    \end{macrocode}
%    \end{macro}
%
%    \begin{macro}{\HoLogoHtml@XeTeX}
%    \begin{macrocode}
\def\HoLogoHtml@XeTeX#1{%
  \HoLogoCss@XeTeX
  \HOLOGO@Span{XeTeX}{%
    \hologo{Xe}%
    \hologo{TeX}%
  }%
}
%    \end{macrocode}
%    \end{macro}
%    \begin{macro}{\HoLogoCss@XeTeX}
%    \begin{macrocode}
\def\HoLogoCss@XeTeX{%
  \Css{%
    span.HoLogo-XeTeX span.HoLogo-TeX{%
      margin-left:-.15em;%
    }%
  }%
  \global\let\HoLogoCss@XeTeX\relax
}
%    \end{macrocode}
%    \end{macro}
%
%    \begin{macro}{\HoLogo@XeLaTeX}
%    \begin{macrocode}
\def\HoLogo@XeLaTeX#1{%
  \hologo{Xe}%
  \kern-.13em%
  \hologo{LaTeX}%
}
%    \end{macrocode}
%    \end{macro}
%    \begin{macro}{\HoLogoHtml@XeLaTeX}
%    \begin{macrocode}
\def\HoLogoHtml@XeLaTeX#1{%
  \HoLogoCss@XeLaTeX
  \HOLOGO@Span{XeLaTeX}{%
    \hologo{Xe}%
    \hologo{LaTeX}%
  }%
}
%    \end{macrocode}
%    \end{macro}
%    \begin{macro}{\HoLogoCss@XeLaTeX}
%    \begin{macrocode}
\def\HoLogoCss@XeLaTeX{%
  \Css{%
    span.HoLogo-XeLaTeX span.HoLogo-Xe{%
      margin-right:-.13em;%
    }%
  }%
  \global\let\HoLogoCss@XeLaTeX\relax
}
%    \end{macrocode}
%    \end{macro}
%
% \subsubsection{\hologo{pdfTeX}, \hologo{pdfLaTeX}}
%
%    \begin{macro}{\HoLogo@pdfTeX}
%    \begin{macrocode}
\def\HoLogo@pdfTeX#1{%
  \HOLOGO@mbox{%
    #1{p}{P}df\hologo{TeX}%
  }%
}
%    \end{macrocode}
%    \end{macro}
%    \begin{macro}{\HoLogoCs@pdfTeX}
%    \begin{macrocode}
\def\HoLogoCs@pdfTeX#1{#1{p}{P}dfTeX}
%    \end{macrocode}
%    \end{macro}
%    \begin{macro}{\HoLogoBkm@pdfTeX}
%    \begin{macrocode}
\def\HoLogoBkm@pdfTeX#1{%
  #1{p}{P}df\hologo{TeX}%
}
%    \end{macrocode}
%    \end{macro}
%    \begin{macro}{\HoLogoHtml@pdfTeX}
%    \begin{macrocode}
\let\HoLogoHtml@pdfTeX\HoLogo@pdfTeX
%    \end{macrocode}
%    \end{macro}
%
%    \begin{macro}{\HoLogo@pdfLaTeX}
%    \begin{macrocode}
\def\HoLogo@pdfLaTeX#1{%
  \HOLOGO@mbox{%
    #1{p}{P}df\hologo{LaTeX}%
  }%
}
%    \end{macrocode}
%    \end{macro}
%    \begin{macro}{\HoLogoCs@pdfLaTeX}
%    \begin{macrocode}
\def\HoLogoCs@pdfLaTeX#1{#1{p}{P}dfLaTeX}
%    \end{macrocode}
%    \end{macro}
%    \begin{macro}{\HoLogoBkm@pdfLaTeX}
%    \begin{macrocode}
\def\HoLogoBkm@pdfLaTeX#1{%
  #1{p}{P}df\hologo{LaTeX}%
}
%    \end{macrocode}
%    \end{macro}
%    \begin{macro}{\HoLogoHtml@pdfLaTeX}
%    \begin{macrocode}
\let\HoLogoHtml@pdfLaTeX\HoLogo@pdfLaTeX
%    \end{macrocode}
%    \end{macro}
%
% \subsubsection{\hologo{VTeX}}
%
%    \begin{macro}{\HoLogo@VTeX}
%    \begin{macrocode}
\def\HoLogo@VTeX#1{%
  \HOLOGO@mbox{%
    V\hologo{TeX}%
  }%
}
%    \end{macrocode}
%    \end{macro}
%    \begin{macro}{\HoLogoHtml@VTeX}
%    \begin{macrocode}
\let\HoLogoHtml@VTeX\HoLogo@VTeX
%    \end{macrocode}
%    \end{macro}
%
% \subsubsection{\hologo{AmS}, \dots}
%
%    Source: class \xclass{amsdtx}
%
%    \begin{macro}{\HoLogo@AmS}
%    \begin{macrocode}
\def\HoLogo@AmS#1{%
  \HoLogoFont@font{AmS}{sy}{%
    A%
    \kern-.1667em%
    \lower.5ex\hbox{M}%
    \kern-.125em%
    S%
  }%
}
%    \end{macrocode}
%    \end{macro}
%    \begin{macro}{\HoLogoBkm@AmS}
%    \begin{macrocode}
\def\HoLogoBkm@AmS#1{AmS}
%    \end{macrocode}
%    \end{macro}
%    \begin{macro}{\HoLogoHtml@AmS}
%    \begin{macrocode}
\def\HoLogoHtml@AmS#1{%
  \HoLogoCss@AmS
%  \HoLogoFont@font{AmS}{sy}{%
    \HOLOGO@Span{AmS}{%
      A%
      \HOLOGO@Span{M}{M}%
      S%
    }%
%   }%
}
%    \end{macrocode}
%    \end{macro}
%    \begin{macro}{\HoLogoCss@AmS}
%    \begin{macrocode}
\def\HoLogoCss@AmS{%
  \Css{%
    span.HoLogo-AmS span.HoLogo-M{%
      position:relative;%
      top:.5ex;%
      margin-left:-.1667em;%
      margin-right:-.125em;%
      text-decoration:none;%
    }%
  }%
  \global\let\HoLogoCss@AmS\relax
}
%    \end{macrocode}
%    \end{macro}
%
%    \begin{macro}{\HoLogo@AmSTeX}
%    \begin{macrocode}
\def\HoLogo@AmSTeX#1{%
  \hologo{AmS}%
  \HOLOGO@hyphen
  \hologo{TeX}%
}
%    \end{macrocode}
%    \end{macro}
%    \begin{macro}{\HoLogoBkm@AmSTeX}
%    \begin{macrocode}
\def\HoLogoBkm@AmSTeX#1{AmS-TeX}%
%    \end{macrocode}
%    \end{macro}
%    \begin{macro}{\HoLogoHtml@AmSTeX}
%    \begin{macrocode}
\let\HoLogoHtml@AmSTeX\HoLogo@AmSTeX
%    \end{macrocode}
%    \end{macro}
%
%    \begin{macro}{\HoLogo@AmSLaTeX}
%    \begin{macrocode}
\def\HoLogo@AmSLaTeX#1{%
  \hologo{AmS}%
  \HOLOGO@hyphen
  \hologo{LaTeX}%
}
%    \end{macrocode}
%    \end{macro}
%    \begin{macro}{\HoLogoBkm@AmSLaTeX}
%    \begin{macrocode}
\def\HoLogoBkm@AmSLaTeX#1{AmS-LaTeX}%
%    \end{macrocode}
%    \end{macro}
%    \begin{macro}{\HoLogoHtml@AmSLaTeX}
%    \begin{macrocode}
\let\HoLogoHtml@AmSLaTeX\HoLogo@AmSLaTeX
%    \end{macrocode}
%    \end{macro}
%
% \subsubsection{\hologo{BibTeX}}
%
%    \begin{macro}{\HoLogo@BibTeX@sc}
%    A definition of \hologo{BibTeX} is provided in
%    the documentation source for the manual of \hologo{BibTeX}
%    \cite{btxdoc}.
%\begin{quote}
%\begin{verbatim}
%\def\BibTeX{%
%  {%
%    \rm
%    B%
%    \kern-.05em%
%    {%
%      \sc
%      i%
%      \kern-.025em %
%      b%
%    }%
%    \kern-.08em
%    T%
%    \kern-.1667em%
%    \lower.7ex\hbox{E}%
%    \kern-.125em%
%    X%
%  }%
%}
%\end{verbatim}
%\end{quote}
%    \begin{macrocode}
\def\HoLogo@BibTeX@sc#1{%
  B%
  \kern-.05em%
  \HoLogoFont@font{BibTeX}{sc}{%
    i%
    \kern-.025em%
    b%
  }%
  \HOLOGO@discretionary
  \kern-.08em%
  \hologo{TeX}%
}
%    \end{macrocode}
%    \end{macro}
%    \begin{macro}{\HoLogoHtml@BibTeX@sc}
%    \begin{macrocode}
\def\HoLogoHtml@BibTeX@sc#1{%
  \HoLogoCss@BibTeX@sc
  \HOLOGO@Span{BibTeX-sc}{%
    B%
    \HOLOGO@Span{i}{i}%
    \HOLOGO@Span{b}{b}%
    \hologo{TeX}%
  }%
}
%    \end{macrocode}
%    \end{macro}
%    \begin{macro}{\HoLogoCss@BibTeX@sc}
%    \begin{macrocode}
\def\HoLogoCss@BibTeX@sc{%
  \Css{%
    span.HoLogo-BibTeX-sc span.HoLogo-i{%
      margin-left:-.05em;%
      margin-right:-.025em;%
      font-variant:small-caps;%
    }%
  }%
  \Css{%
    span.HoLogo-BibTeX-sc span.HoLogo-b{%
      margin-right:-.08em;%
      font-variant:small-caps;%
    }%
  }%
  \global\let\HoLogoCss@BibTeX@sc\relax
}
%    \end{macrocode}
%    \end{macro}
%
%    \begin{macro}{\HoLogo@BibTeX@sf}
%    Variant \xoption{sf} avoids trouble with unavailable
%    small caps fonts (e.g., bold versions of Computer Modern or
%    Latin Modern). The definition is taken from
%    package \xpackage{dtklogos} \cite{dtklogos}.
%\begin{quote}
%\begin{verbatim}
%\DeclareRobustCommand{\BibTeX}{%
%  B%
%  \kern-.05em%
%  \hbox{%
%    $\m@th$% %% force math size calculations
%    \csname S@\f@size\endcsname
%    \fontsize\sf@size\z@
%    \math@fontsfalse
%    \selectfont
%    I%
%    \kern-.025em%
%    B
%  }%
%  \kern-.08em%
%  \-%
%  \TeX
%}
%\end{verbatim}
%\end{quote}
%    \begin{macrocode}
\def\HoLogo@BibTeX@sf#1{%
  B%
  \kern-.05em%
  \HoLogoFont@font{BibTeX}{bibsf}{%
    I%
    \kern-.025em%
    B%
  }%
  \HOLOGO@discretionary
  \kern-.08em%
  \hologo{TeX}%
}
%    \end{macrocode}
%    \end{macro}
%    \begin{macro}{\HoLogoHtml@BibTeX@sf}
%    \begin{macrocode}
\def\HoLogoHtml@BibTeX@sf#1{%
  \HoLogoCss@BibTeX@sf
  \HOLOGO@Span{BibTeX-sf}{%
    B%
    \HoLogoFont@font{BibTeX}{bibsf}{%
      \HOLOGO@Span{i}{I}%
      B%
    }%
    \hologo{TeX}%
  }%
}
%    \end{macrocode}
%    \end{macro}
%    \begin{macro}{\HoLogoCss@BibTeX@sf}
%    \begin{macrocode}
\def\HoLogoCss@BibTeX@sf{%
  \Css{%
    span.HoLogo-BibTeX-sf span.HoLogo-i{%
      margin-left:-.05em;%
      margin-right:-.025em;%
    }%
  }%
  \Css{%
    span.HoLogo-BibTeX-sf span.HoLogo-TeX{%
      margin-left:-.08em;%
    }%
  }%
  \global\let\HoLogoCss@BibTeX@sf\relax
}
%    \end{macrocode}
%    \end{macro}
%
%    \begin{macro}{\HoLogo@BibTeX}
%    \begin{macrocode}
\def\HoLogo@BibTeX{\HoLogo@BibTeX@sf}
%    \end{macrocode}
%    \end{macro}
%    \begin{macro}{\HoLogoHtml@BibTeX}
%    \begin{macrocode}
\def\HoLogoHtml@BibTeX{\HoLogoHtml@BibTeX@sf}
%    \end{macrocode}
%    \end{macro}
%
% \subsubsection{\hologo{BibTeX8}}
%
%    \begin{macro}{\HoLogo@BibTeX8}
%    \begin{macrocode}
\expandafter\def\csname HoLogo@BibTeX8\endcsname#1{%
  \hologo{BibTeX}%
  8%
}
%    \end{macrocode}
%    \end{macro}
%
%    \begin{macro}{\HoLogoBkm@BibTeX8}
%    \begin{macrocode}
\expandafter\def\csname HoLogoBkm@BibTeX8\endcsname#1{%
  \hologo{BibTeX}%
  8%
}
%    \end{macrocode}
%    \end{macro}
%    \begin{macro}{\HoLogoHtml@BibTeX8}
%    \begin{macrocode}
\expandafter
\let\csname HoLogoHtml@BibTeX8\expandafter\endcsname
\csname HoLogo@BibTeX8\endcsname
%    \end{macrocode}
%    \end{macro}
%
% \subsubsection{\hologo{ConTeXt}}
%
%    \begin{macro}{\HoLogo@ConTeXt@simple}
%    \begin{macrocode}
\def\HoLogo@ConTeXt@simple#1{%
  \HOLOGO@mbox{Con}%
  \HOLOGO@discretionary
  \HOLOGO@mbox{\hologo{TeX}t}%
}
%    \end{macrocode}
%    \end{macro}
%    \begin{macro}{\HoLogoHtml@ConTeXt@simple}
%    \begin{macrocode}
\let\HoLogoHtml@ConTeXt@simple\HoLogo@ConTeXt@simple
%    \end{macrocode}
%    \end{macro}
%
%    \begin{macro}{\HoLogo@ConTeXt@narrow}
%    This definition of logo \hologo{ConTeXt} with variant \xoption{narrow}
%    comes from TUGboat's class \xclass{ltugboat} (version 2010/11/15 v2.8).
%    \begin{macrocode}
\def\HoLogo@ConTeXt@narrow#1{%
  \HOLOGO@mbox{C\kern-.0333emon}%
  \HOLOGO@discretionary
  \kern-.0667em%
  \HOLOGO@mbox{\hologo{TeX}\kern-.0333emt}%
}
%    \end{macrocode}
%    \end{macro}
%    \begin{macro}{\HoLogoHtml@ConTeXt@narrow}
%    \begin{macrocode}
\def\HoLogoHtml@ConTeXt@narrow#1{%
  \HoLogoCss@ConTeXt@narrow
  \HOLOGO@Span{ConTeXt-narrow}{%
    \HOLOGO@Span{C}{C}%
    on%
    \hologo{TeX}%
    t%
  }%
}
%    \end{macrocode}
%    \end{macro}
%    \begin{macro}{\HoLogoCss@ConTeXt@narrow}
%    \begin{macrocode}
\def\HoLogoCss@ConTeXt@narrow{%
  \Css{%
    span.HoLogo-ConTeXt-narrow span.HoLogo-C{%
      margin-left:-.0333em;%
    }%
  }%
  \Css{%
    span.HoLogo-ConTeXt-narrow span.HoLogo-TeX{%
      margin-left:-.0667em;%
      margin-right:-.0333em;%
    }%
  }%
  \global\let\HoLogoCss@ConTeXt@narrow\relax
}
%    \end{macrocode}
%    \end{macro}
%
%    \begin{macro}{\HoLogo@ConTeXt}
%    \begin{macrocode}
\def\HoLogo@ConTeXt{\HoLogo@ConTeXt@narrow}
%    \end{macrocode}
%    \end{macro}
%    \begin{macro}{\HoLogoHtml@ConTeXt}
%    \begin{macrocode}
\def\HoLogoHtml@ConTeXt{\HoLogoHtml@ConTeXt@narrow}
%    \end{macrocode}
%    \end{macro}
%
% \subsubsection{\hologo{emTeX}}
%
%    \begin{macro}{\HoLogo@emTeX}
%    \begin{macrocode}
\def\HoLogo@emTeX#1{%
  \HOLOGO@mbox{#1{e}{E}m}%
  \HOLOGO@discretionary
  \hologo{TeX}%
}
%    \end{macrocode}
%    \end{macro}
%    \begin{macro}{\HoLogoCs@emTeX}
%    \begin{macrocode}
\def\HoLogoCs@emTeX#1{#1{e}{E}mTeX}%
%    \end{macrocode}
%    \end{macro}
%    \begin{macro}{\HoLogoBkm@emTeX}
%    \begin{macrocode}
\def\HoLogoBkm@emTeX#1{%
  #1{e}{E}m\hologo{TeX}%
}
%    \end{macrocode}
%    \end{macro}
%    \begin{macro}{\HoLogoHtml@emTeX}
%    \begin{macrocode}
\let\HoLogoHtml@emTeX\HoLogo@emTeX
%    \end{macrocode}
%    \end{macro}
%
% \subsubsection{\hologo{ExTeX}}
%
%    \begin{macro}{\HoLogo@ExTeX}
%    The definition is taken from the FAQ of the
%    project \hologo{ExTeX}
%    \cite{ExTeX-FAQ}.
%\begin{quote}
%\begin{verbatim}
%\def\ExTeX{%
%  \textrm{% Logo always with serifs
%    \ensuremath{%
%      \textstyle
%      \varepsilon_{%
%        \kern-0.15em%
%        \mathcal{X}%
%      }%
%    }%
%    \kern-.15em%
%    \TeX
%  }%
%}
%\end{verbatim}
%\end{quote}
%    \begin{macrocode}
\def\HoLogo@ExTeX#1{%
  \HoLogoFont@font{ExTeX}{rm}{%
    \ltx@mbox{%
      \HOLOGO@MathSetup
      $%
        \textstyle
        \varepsilon_{%
          \kern-0.15em%
          \HoLogoFont@font{ExTeX}{sy}{X}%
        }%
      $%
    }%
    \HOLOGO@discretionary
    \kern-.15em%
    \hologo{TeX}%
  }%
}
%    \end{macrocode}
%    \end{macro}
%    \begin{macro}{\HoLogoHtml@ExTeX}
%    \begin{macrocode}
\def\HoLogoHtml@ExTeX#1{%
  \HoLogoCss@ExTeX
  \HoLogoFont@font{ExTeX}{rm}{%
    \HOLOGO@Span{ExTeX}{%
      \ltx@mbox{%
        \HOLOGO@MathSetup
        $\textstyle\varepsilon$%
        \HOLOGO@Span{X}{$\textstyle\chi$}%
        \hologo{TeX}%
      }%
    }%
  }%
}
%    \end{macrocode}
%    \end{macro}
%    \begin{macro}{\HoLogoBkm@ExTeX}
%    \begin{macrocode}
\def\HoLogoBkm@ExTeX#1{%
  \HOLOGO@PdfdocUnicode{#1{e}{E}x}{\textepsilon\textchi}%
  \hologo{TeX}%
}
%    \end{macrocode}
%    \end{macro}
%    \begin{macro}{\HoLogoCss@ExTeX}
%    \begin{macrocode}
\def\HoLogoCss@ExTeX{%
  \Css{%
    span.HoLogo-ExTeX{%
      font-family:serif;%
    }%
  }%
  \Css{%
    span.HoLogo-ExTeX span.HoLogo-TeX{%
      margin-left:-.15em;%
    }%
  }%
  \global\let\HoLogoCss@ExTeX\relax
}
%    \end{macrocode}
%    \end{macro}
%
% \subsubsection{\hologo{MiKTeX}}
%
%    \begin{macro}{\HoLogo@MiKTeX}
%    \begin{macrocode}
\def\HoLogo@MiKTeX#1{%
  \HOLOGO@mbox{MiK}%
  \HOLOGO@discretionary
  \hologo{TeX}%
}
%    \end{macrocode}
%    \end{macro}
%    \begin{macro}{\HoLogoHtml@MiKTeX}
%    \begin{macrocode}
\let\HoLogoHtml@MiKTeX\HoLogo@MiKTeX
%    \end{macrocode}
%    \end{macro}
%
% \subsubsection{\hologo{OzTeX} and friends}
%
%    Source: \hologo{OzTeX} FAQ \cite{OzTeX}:
%    \begin{quote}
%      |\def\OzTeX{O\kern-.03em z\kern-.15em\TeX}|\\
%      (There is no kerning in OzMF, OzMP and OzTtH.)
%    \end{quote}
%
%    \begin{macro}{\HoLogo@OzTeX}
%    \begin{macrocode}
\def\HoLogo@OzTeX#1{%
  O%
  \kern-.03em %
  z%
  \kern-.15em %
  \hologo{TeX}%
}
%    \end{macrocode}
%    \end{macro}
%    \begin{macro}{\HoLogoHtml@OzTeX}
%    \begin{macrocode}
\def\HoLogoHtml@OzTeX#1{%
  \HoLogoCss@OzTeX
  \HOLOGO@Span{OzTeX}{%
    O%
    \HOLOGO@Span{z}{z}%
    \hologo{TeX}%
  }%
}
%    \end{macrocode}
%    \end{macro}
%    \begin{macro}{\HoLogoCss@OzTeX}
%    \begin{macrocode}
\def\HoLogoCss@OzTeX{%
  \Css{%
    span.HoLogo-OzTeX span.HoLogo-z{%
      margin-left:-.03em;%
      margin-right:-.15em;%
    }%
  }%
  \global\let\HoLogoCss@OzTeX\relax
}
%    \end{macrocode}
%    \end{macro}
%
%    \begin{macro}{\HoLogo@OzMF}
%    \begin{macrocode}
\def\HoLogo@OzMF#1{%
  \HOLOGO@mbox{OzMF}%
}
%    \end{macrocode}
%    \end{macro}
%    \begin{macro}{\HoLogo@OzMP}
%    \begin{macrocode}
\def\HoLogo@OzMP#1{%
  \HOLOGO@mbox{OzMP}%
}
%    \end{macrocode}
%    \end{macro}
%    \begin{macro}{\HoLogo@OzTtH}
%    \begin{macrocode}
\def\HoLogo@OzTtH#1{%
  \HOLOGO@mbox{OzTtH}%
}
%    \end{macrocode}
%    \end{macro}
%
% \subsubsection{\hologo{PCTeX}}
%
%    \begin{macro}{\HoLogo@PCTeX}
%    \begin{macrocode}
\def\HoLogo@PCTeX#1{%
  \HOLOGO@mbox{PC}%
  \hologo{TeX}%
}
%    \end{macrocode}
%    \end{macro}
%    \begin{macro}{\HoLogoHtml@PCTeX}
%    \begin{macrocode}
\let\HoLogoHtml@PCTeX\HoLogo@PCTeX
%    \end{macrocode}
%    \end{macro}
%
% \subsubsection{\hologo{PiCTeX}}
%
%    The original definitions from \xfile{pictex.tex} \cite{PiCTeX}:
%\begin{quote}
%\begin{verbatim}
%\def\PiC{%
%  P%
%  \kern-.12em%
%  \lower.5ex\hbox{I}%
%  \kern-.075em%
%  C%
%}
%\def\PiCTeX{%
%  \PiC
%  \kern-.11em%
%  \TeX
%}
%\end{verbatim}
%\end{quote}
%
%    \begin{macro}{\HoLogo@PiC}
%    \begin{macrocode}
\def\HoLogo@PiC#1{%
  P%
  \kern-.12em%
  \lower.5ex\hbox{I}%
  \kern-.075em%
  C%
  \HOLOGO@SpaceFactor
}
%    \end{macrocode}
%    \end{macro}
%    \begin{macro}{\HoLogoHtml@PiC}
%    \begin{macrocode}
\def\HoLogoHtml@PiC#1{%
  \HoLogoCss@PiC
  \HOLOGO@Span{PiC}{%
    P%
    \HOLOGO@Span{i}{I}%
    C%
  }%
}
%    \end{macrocode}
%    \end{macro}
%    \begin{macro}{\HoLogoCss@PiC}
%    \begin{macrocode}
\def\HoLogoCss@PiC{%
  \Css{%
    span.HoLogo-PiC span.HoLogo-i{%
      position:relative;%
      top:.5ex;%
      margin-left:-.12em;%
      margin-right:-.075em;%
      text-decoration:none;%
    }%
  }%
  \global\let\HoLogoCss@PiC\relax
}
%    \end{macrocode}
%    \end{macro}
%
%    \begin{macro}{\HoLogo@PiCTeX}
%    \begin{macrocode}
\def\HoLogo@PiCTeX#1{%
  \hologo{PiC}%
  \HOLOGO@discretionary
  \kern-.11em%
  \hologo{TeX}%
}
%    \end{macrocode}
%    \end{macro}
%    \begin{macro}{\HoLogoHtml@PiCTeX}
%    \begin{macrocode}
\def\HoLogoHtml@PiCTeX#1{%
  \HoLogoCss@PiCTeX
  \HOLOGO@Span{PiCTeX}{%
    \hologo{PiC}%
    \hologo{TeX}%
  }%
}
%    \end{macrocode}
%    \end{macro}
%    \begin{macro}{\HoLogoCss@PiCTeX}
%    \begin{macrocode}
\def\HoLogoCss@PiCTeX{%
  \Css{%
    span.HoLogo-PiCTeX span.HoLogo-PiC{%
      margin-right:-.11em;%
    }%
  }%
  \global\let\HoLogoCss@PiCTeX\relax
}
%    \end{macrocode}
%    \end{macro}
%
% \subsubsection{\hologo{teTeX}}
%
%    \begin{macro}{\HoLogo@teTeX}
%    \begin{macrocode}
\def\HoLogo@teTeX#1{%
  \HOLOGO@mbox{#1{t}{T}e}%
  \HOLOGO@discretionary
  \hologo{TeX}%
}
%    \end{macrocode}
%    \end{macro}
%    \begin{macro}{\HoLogoCs@teTeX}
%    \begin{macrocode}
\def\HoLogoCs@teTeX#1{#1{t}{T}dfTeX}
%    \end{macrocode}
%    \end{macro}
%    \begin{macro}{\HoLogoBkm@teTeX}
%    \begin{macrocode}
\def\HoLogoBkm@teTeX#1{%
  #1{t}{T}e\hologo{TeX}%
}
%    \end{macrocode}
%    \end{macro}
%    \begin{macro}{\HoLogoHtml@teTeX}
%    \begin{macrocode}
\let\HoLogoHtml@teTeX\HoLogo@teTeX
%    \end{macrocode}
%    \end{macro}
%
% \subsubsection{\hologo{TeX4ht}}
%
%    \begin{macro}{\HoLogo@TeX4ht}
%    \begin{macrocode}
\expandafter\def\csname HoLogo@TeX4ht\endcsname#1{%
  \HOLOGO@mbox{\hologo{TeX}4ht}%
}
%    \end{macrocode}
%    \end{macro}
%    \begin{macro}{\HoLogoHtml@TeX4ht}
%    \begin{macrocode}
\expandafter
\let\csname HoLogoHtml@TeX4ht\expandafter\endcsname
\csname HoLogo@TeX4ht\endcsname
%    \end{macrocode}
%    \end{macro}
%
%
% \subsubsection{\hologo{SageTeX}}
%
%    \begin{macro}{\HoLogo@SageTeX}
%    \begin{macrocode}
\def\HoLogo@SageTeX#1{%
  \HOLOGO@mbox{Sage}%
  \HOLOGO@discretionary
  \HOLOGO@NegativeKerning{eT,oT,To}%
  \hologo{TeX}%
}
%    \end{macrocode}
%    \end{macro}
%    \begin{macro}{\HoLogoHtml@SageTeX}
%    \begin{macrocode}
\let\HoLogoHtml@SageTeX\HoLogo@SageTeX
%    \end{macrocode}
%    \end{macro}
%
% \subsection{\hologo{METAFONT} and friends}
%
%    \begin{macro}{\HoLogo@METAFONT}
%    \begin{macrocode}
\def\HoLogo@METAFONT#1{%
  \HoLogoFont@font{METAFONT}{logo}{%
    \HOLOGO@mbox{META}%
    \HOLOGO@discretionary
    \HOLOGO@mbox{FONT}%
  }%
}
%    \end{macrocode}
%    \end{macro}
%
%    \begin{macro}{\HoLogo@METAPOST}
%    \begin{macrocode}
\def\HoLogo@METAPOST#1{%
  \HoLogoFont@font{METAPOST}{logo}{%
    \HOLOGO@mbox{META}%
    \HOLOGO@discretionary
    \HOLOGO@mbox{POST}%
  }%
}
%    \end{macrocode}
%    \end{macro}
%
%    \begin{macro}{\HoLogo@MetaFun}
%    \begin{macrocode}
\def\HoLogo@MetaFun#1{%
  \HOLOGO@mbox{Meta}%
  \HOLOGO@discretionary
  \HOLOGO@mbox{Fun}%
}
%    \end{macrocode}
%    \end{macro}
%
%    \begin{macro}{\HoLogo@MetaPost}
%    \begin{macrocode}
\def\HoLogo@MetaPost#1{%
  \HOLOGO@mbox{Meta}%
  \HOLOGO@discretionary
  \HOLOGO@mbox{Post}%
}
%    \end{macrocode}
%    \end{macro}
%
% \subsection{Others}
%
% \subsubsection{\hologo{biber}}
%
%    \begin{macro}{\HoLogo@biber}
%    \begin{macrocode}
\def\HoLogo@biber#1{%
  \HOLOGO@mbox{#1{b}{B}i}%
  \HOLOGO@discretionary
  \HOLOGO@mbox{ber}%
}
%    \end{macrocode}
%    \end{macro}
%    \begin{macro}{\HoLogoCs@biber}
%    \begin{macrocode}
\def\HoLogoCs@biber#1{#1{b}{B}iber}
%    \end{macrocode}
%    \end{macro}
%    \begin{macro}{\HoLogoBkm@biber}
%    \begin{macrocode}
\def\HoLogoBkm@biber#1{%
  #1{b}{B}iber%
}
%    \end{macrocode}
%    \end{macro}
%    \begin{macro}{\HoLogoHtml@biber}
%    \begin{macrocode}
\let\HoLogoHtml@biber\HoLogo@biber
%    \end{macrocode}
%    \end{macro}
%
% \subsubsection{\hologo{KOMAScript}}
%
%    \begin{macro}{\HoLogo@KOMAScript}
%    The definition for \hologo{KOMAScript} is taken
%    from \hologo{KOMAScript} (\xfile{scrlogo.dtx}, reformatted) \cite{scrlogo}:
%\begin{quote}
%\begin{verbatim}
%\@ifundefined{KOMAScript}{%
%  \DeclareRobustCommand{\KOMAScript}{%
%    \textsf{%
%      K\kern.05em O\kern.05emM\kern.05em A%
%      \kern.1em-\kern.1em %
%      Script%
%    }%
%  }%
%}{}
%\end{verbatim}
%\end{quote}
%    \begin{macrocode}
\def\HoLogo@KOMAScript#1{%
  \HoLogoFont@font{KOMAScript}{sf}{%
    \HOLOGO@mbox{%
      K\kern.05em%
      O\kern.05em%
      M\kern.05em%
      A%
    }%
    \kern.1em%
    \HOLOGO@hyphen
    \kern.1em%
    \HOLOGO@mbox{Script}%
  }%
}
%    \end{macrocode}
%    \end{macro}
%    \begin{macro}{\HoLogoBkm@KOMAScript}
%    \begin{macrocode}
\def\HoLogoBkm@KOMAScript#1{%
  KOMA-Script%
}
%    \end{macrocode}
%    \end{macro}
%    \begin{macro}{\HoLogoHtml@KOMAScript}
%    \begin{macrocode}
\def\HoLogoHtml@KOMAScript#1{%
  \HoLogoCss@KOMAScript
  \HoLogoFont@font{KOMAScript}{sf}{%
    \HOLOGO@Span{KOMAScript}{%
      K%
      \HOLOGO@Span{O}{O}%
      M%
      \HOLOGO@Span{A}{A}%
      \HOLOGO@Span{hyphen}{-}%
      Script%
    }%
  }%
}
%    \end{macrocode}
%    \end{macro}
%    \begin{macro}{\HoLogoCss@KOMAScript}
%    \begin{macrocode}
\def\HoLogoCss@KOMAScript{%
  \Css{%
    span.HoLogo-KOMAScript{%
      font-family:sans-serif;%
    }%
  }%
  \Css{%
    span.HoLogo-KOMAScript span.HoLogo-O{%
      padding-left:.05em;%
      padding-right:.05em;%
    }%
  }%
  \Css{%
    span.HoLogo-KOMAScript span.HoLogo-A{%
      padding-left:.05em;%
    }%
  }%
  \Css{%
    span.HoLogo-KOMAScript span.HoLogo-hyphen{%
      padding-left:.1em;%
      padding-right:.1em;%
    }%
  }%
  \global\let\HoLogoCss@KOMAScript\relax
}
%    \end{macrocode}
%    \end{macro}
%
% \subsubsection{\hologo{LyX}}
%
%    \begin{macro}{\HoLogo@LyX}
%    The definition is taken from the documentation source files
%    of \hologo{LyX}, \xfile{Intro.lyx} \cite{LyX}:
%\begin{quote}
%\begin{verbatim}
%\def\LyX{%
%  \texorpdfstring{%
%    L\kern-.1667em\lower.25em\hbox{Y}\kern-.125emX\@%
%  }{%
%    LyX%
%  }%
%}
%\end{verbatim}
%\end{quote}
%    \begin{macrocode}
\def\HoLogo@LyX#1{%
  L%
  \kern-.1667em%
  \lower.25em\hbox{Y}%
  \kern-.125em%
  X%
  \HOLOGO@SpaceFactor
}
%    \end{macrocode}
%    \end{macro}
%    \begin{macro}{\HoLogoHtml@LyX}
%    \begin{macrocode}
\def\HoLogoHtml@LyX#1{%
  \HoLogoCss@LyX
  \HOLOGO@Span{LyX}{%
    L%
    \HOLOGO@Span{y}{Y}%
    X%
  }%
}
%    \end{macrocode}
%    \end{macro}
%    \begin{macro}{\HoLogoCss@LyX}
%    \begin{macrocode}
\def\HoLogoCss@LyX{%
  \Css{%
    span.HoLogo-LyX span.HoLogo-y{%
      position:relative;%
      top:.25em;%
      margin-left:-.1667em;%
      margin-right:-.125em;%
      text-decoration:none;%
    }%
  }%
  \global\let\HoLogoCss@LyX\relax
}
%    \end{macrocode}
%    \end{macro}
%
% \subsubsection{\hologo{NTS}}
%
%    \begin{macro}{\HoLogo@NTS}
%    Definition for \hologo{NTS} can be found in
%    package \xpackage{etex\textunderscore man} for the \hologo{eTeX} manual \cite{etexman}
%    and in package \xpackage{dtklogos} \cite{dtklogos}:
%\begin{quote}
%\begin{verbatim}
%\def\NTS{%
%  \leavevmode
%  \hbox{%
%    $%
%      \cal N%
%      \kern-0.35em%
%      \lower0.5ex\hbox{$\cal T$}%
%      \kern-0.2em%
%      S%
%    $%
%  }%
%}
%\end{verbatim}
%\end{quote}
%    \begin{macrocode}
\def\HoLogo@NTS#1{%
  \HoLogoFont@font{NTS}{sy}{%
    N\/%
    \kern-.35em%
    \lower.5ex\hbox{T\/}%
    \kern-.2em%
    S\/%
  }%
  \HOLOGO@SpaceFactor
}
%    \end{macrocode}
%    \end{macro}
%
% \subsubsection{\Hologo{TTH} (\hologo{TeX} to HTML translator)}
%
%    Source: \url{http://hutchinson.belmont.ma.us/tth/}
%    In the HTML source the second `T' is printed as subscript.
%\begin{quote}
%\begin{verbatim}
%T<sub>T</sub>H
%\end{verbatim}
%\end{quote}
%    \begin{macro}{\HoLogo@TTH}
%    \begin{macrocode}
\def\HoLogo@TTH#1{%
  \ltx@mbox{%
    T\HOLOGO@SubScript{T}H%
  }%
  \HOLOGO@SpaceFactor
}
%    \end{macrocode}
%    \end{macro}
%
%    \begin{macro}{\HoLogoHtml@TTH}
%    \begin{macrocode}
\def\HoLogoHtml@TTH#1{%
  T\HCode{<sub>}T\HCode{</sub>}H%
}
%    \end{macrocode}
%    \end{macro}
%
% \subsubsection{\Hologo{HanTheThanh}}
%
%    Partial source: Package \xpackage{dtklogos}.
%    The double accent is U+1EBF (latin small letter e with circumflex
%    and acute).
%    \begin{macro}{\HoLogo@HanTheThanh}
%    \begin{macrocode}
\def\HoLogo@HanTheThanh#1{%
  \ltx@mbox{H\`an}%
  \HOLOGO@space
  \ltx@mbox{%
    Th%
    \HOLOGO@IfCharExists{"1EBF}{%
      \char"1EBF\relax
    }{%
      \^e\hbox to 0pt{\hss\raise .5ex\hbox{\'{}}}%
    }%
  }%
  \HOLOGO@space
  \ltx@mbox{Th\`anh}%
}
%    \end{macrocode}
%    \end{macro}
%    \begin{macro}{\HoLogoBkm@HanTheThanh}
%    \begin{macrocode}
\def\HoLogoBkm@HanTheThanh#1{%
  H\`an %
  Th\HOLOGO@PdfdocUnicode{\^e}{\9036\277} %
  Th\`anh%
}
%    \end{macrocode}
%    \end{macro}
%    \begin{macro}{\HoLogoHtml@HanTheThanh}
%    \begin{macrocode}
\def\HoLogoHtml@HanTheThanh#1{%
  H\`an %
  Th\HCode{&\ltx@hashchar x1ebf;} %
  Th\`anh%
}
%    \end{macrocode}
%    \end{macro}
%
% \subsection{Driver detection}
%
%    \begin{macrocode}
\HOLOGO@IfExists\InputIfFileExists{%
  \InputIfFileExists{hologo.cfg}{}{}%
}{%
  \ltx@IfUndefined{pdf@filesize}{%
    \def\HOLOGO@InputIfExists{%
      \openin\HOLOGO@temp=hologo.cfg\relax
      \ifeof\HOLOGO@temp
        \closein\HOLOGO@temp
      \else
        \closein\HOLOGO@temp
        \begingroup
          \def\x{LaTeX2e}%
        \expandafter\endgroup
        \ifx\fmtname\x
          \input{hologo.cfg}%
        \else
          \input hologo.cfg\relax
        \fi
      \fi
    }%
    \ltx@IfUndefined{newread}{%
      \chardef\HOLOGO@temp=15 %
      \def\HOLOGO@CheckRead{%
        \ifeof\HOLOGO@temp
          \HOLOGO@InputIfExists
        \else
          \ifcase\HOLOGO@temp
            \@PackageWarningNoLine{hologo}{%
              Configuration file ignored, because\MessageBreak
              a free read register could not be found%
            }%
          \else
            \begingroup
              \count\ltx@cclv=\HOLOGO@temp
              \advance\ltx@cclv by \ltx@minusone
              \edef\x{\endgroup
                \chardef\noexpand\HOLOGO@temp=\the\count\ltx@cclv
                \relax
              }%
            \x
          \fi
        \fi
      }%
    }{%
      \csname newread\endcsname\HOLOGO@temp
      \HOLOGO@InputIfExists
    }%
  }{%
    \edef\HOLOGO@temp{\pdf@filesize{hologo.cfg}}%
    \ifx\HOLOGO@temp\ltx@empty
    \else
      \ifnum\HOLOGO@temp>0 %
        \begingroup
          \def\x{LaTeX2e}%
        \expandafter\endgroup
        \ifx\fmtname\x
          \input{hologo.cfg}%
        \else
          \input hologo.cfg\relax
        \fi
      \else
        \@PackageInfoNoLine{hologo}{%
          Empty configuration file `hologo.cfg' ignored%
        }%
      \fi
    \fi
  }%
}
%    \end{macrocode}
%
%    \begin{macrocode}
\def\HOLOGO@temp#1#2{%
  \kv@define@key{HoLogoDriver}{#1}[]{%
    \begingroup
      \def\HOLOGO@temp{##1}%
      \ltx@onelevel@sanitize\HOLOGO@temp
      \ifx\HOLOGO@temp\ltx@empty
      \else
        \@PackageError{hologo}{%
          Value (\HOLOGO@temp) not permitted for option `#1'%
        }%
        \@ehc
      \fi
    \endgroup
    \def\hologoDriver{#2}%
  }%
}%
\def\HOLOGO@@temp#1#2{%
  \ifx\kv@value\relax
    \HOLOGO@temp{#1}{#1}%
  \else
    \HOLOGO@temp{#1}{#2}%
  \fi
}%
\kv@parse@normalized{%
  pdftex,%
  luatex=pdftex,%
  dvipdfm,%
  dvipdfmx=dvipdfm,%
  dvips,%
  dvipsone=dvips,%
  xdvi=dvips,%
  xetex,%
  vtex,%
}\HOLOGO@@temp
%    \end{macrocode}
%
%    \begin{macrocode}
\kv@define@key{HoLogoDriver}{driverfallback}{%
  \def\HOLOGO@DriverFallback{#1}%
}
%    \end{macrocode}
%
%    \begin{macro}{\HOLOGO@DriverFallback}
%    \begin{macrocode}
\def\HOLOGO@DriverFallback{dvips}
%    \end{macrocode}
%    \end{macro}
%
%    \begin{macro}{\hologoDriverSetup}
%    \begin{macrocode}
\def\hologoDriverSetup{%
  \let\hologoDriver\ltx@undefined
  \HOLOGO@DriverSetup
}
%    \end{macrocode}
%    \end{macro}
%
%    \begin{macro}{\HOLOGO@DriverSetup}
%    \begin{macrocode}
\def\HOLOGO@DriverSetup#1{%
  \kvsetkeys{HoLogoDriver}{#1}%
  \HOLOGO@CheckDriver
  \ltx@ifundefined{hologoDriver}{%
    \begingroup
    \edef\x{\endgroup
      \noexpand\kvsetkeys{HoLogoDriver}{\HOLOGO@DriverFallback}%
    }\x
  }{}%
  \@PackageInfoNoLine{hologo}{Using driver `\hologoDriver'}%
}
%    \end{macrocode}
%    \end{macro}
%
%    \begin{macro}{\HOLOGO@CheckDriver}
%    \begin{macrocode}
\def\HOLOGO@CheckDriver{%
  \ifpdf
    \def\hologoDriver{pdftex}%
    \let\HOLOGO@pdfliteral\pdfliteral
    \ifluatex
      \ifx\pdfextension\@undefined\else
        \protected\def\pdfliteral{\pdfextension literal}%
        \let\HOLOGO@pdfliteral\pdfliteral
      \fi
      \ltx@IfUndefined{HOLOGO@pdfliteral}{%
        \ifnum\luatexversion<36 %
        \else
          \begingroup
            \let\HOLOGO@temp\endgroup
            \ifcase0%
                \directlua{%
                  if tex.enableprimitives then %
                    tex.enableprimitives('HOLOGO@', {'pdfliteral'})%
                  else %
                    tex.print('1')%
                  end%
                }%
                \ifx\HOLOGO@pdfliteral\@undefined 1\fi%
                \relax%
              \endgroup
              \let\HOLOGO@temp\relax
              \global\let\HOLOGO@pdfliteral\HOLOGO@pdfliteral
            \fi%
          \HOLOGO@temp
        \fi
      }{}%
    \fi
    \ltx@IfUndefined{HOLOGO@pdfliteral}{%
      \@PackageWarningNoLine{hologo}{%
        Cannot find \string\pdfliteral
      }%
    }{}%
  \else
    \ifxetex
      \def\hologoDriver{xetex}%
    \else
      \ifvtex
        \def\hologoDriver{vtex}%
      \fi
    \fi
  \fi
}
%    \end{macrocode}
%    \end{macro}
%
%    \begin{macro}{\HOLOGO@WarningUnsupportedDriver}
%    \begin{macrocode}
\def\HOLOGO@WarningUnsupportedDriver#1{%
  \@PackageWarningNoLine{hologo}{%
    Logo `#1' needs driver specific macros,\MessageBreak
    but driver `\hologoDriver' is not supported.\MessageBreak
    Use a different driver or\MessageBreak
    load package `graphics' or `pgf'%
  }%
}
%    \end{macrocode}
%    \end{macro}
%
% \subsubsection{Reflect box macros}
%
%    Skip driver part if not needed.
%    \begin{macrocode}
\ltx@IfUndefined{reflectbox}{}{%
  \ltx@IfUndefined{rotatebox}{}{%
    \HOLOGO@AtEnd
  }%
}
\ltx@IfUndefined{pgftext}{}{%
  \HOLOGO@AtEnd
}
\ltx@IfUndefined{psscalebox}{}{%
  \HOLOGO@AtEnd
}
%    \end{macrocode}
%
%    \begin{macrocode}
\def\HOLOGO@temp{LaTeX2e}
\ifx\fmtname\HOLOGO@temp
  \RequirePackage{kvoptions}[2011/06/30]%
  \ProcessKeyvalOptions{HoLogoDriver}%
\fi
\HOLOGO@DriverSetup{}
%    \end{macrocode}
%
%    \begin{macro}{\HOLOGO@ReflectBox}
%    \begin{macrocode}
\def\HOLOGO@ReflectBox#1{%
  \begingroup
    \setbox\ltx@zero\hbox{\begingroup#1\endgroup}%
    \setbox\ltx@two\hbox{%
      \kern\wd\ltx@zero
      \csname HOLOGO@ScaleBox@\hologoDriver\endcsname{-1}{1}{%
        \hbox to 0pt{\copy\ltx@zero\hss}%
      }%
    }%
    \wd\ltx@two=\wd\ltx@zero
    \box\ltx@two
  \endgroup
}
%    \end{macrocode}
%    \end{macro}
%
%    \begin{macro}{\HOLOGO@PointReflectBox}
%    \begin{macrocode}
\def\HOLOGO@PointReflectBox#1{%
  \begingroup
    \setbox\ltx@zero\hbox{\begingroup#1\endgroup}%
    \setbox\ltx@two\hbox{%
      \kern\wd\ltx@zero
      \raise\ht\ltx@zero\hbox{%
        \csname HOLOGO@ScaleBox@\hologoDriver\endcsname{-1}{-1}{%
          \hbox to 0pt{\copy\ltx@zero\hss}%
        }%
      }%
    }%
    \wd\ltx@two=\wd\ltx@zero
    \box\ltx@two
  \endgroup
}
%    \end{macrocode}
%    \end{macro}
%
%    We must define all variants because of dynamic driver setup.
%    \begin{macrocode}
\def\HOLOGO@temp#1#2{#2}
%    \end{macrocode}
%
%    \begin{macro}{\HOLOGO@ScaleBox@pdftex}
%    \begin{macrocode}
\HOLOGO@temp{pdftex}{%
  \def\HOLOGO@ScaleBox@pdftex#1#2#3{%
    \HOLOGO@pdfliteral{%
      q #1 0 0 #2 0 0 cm%
    }%
    #3%
    \HOLOGO@pdfliteral{%
      Q%
    }%
  }%
}
%    \end{macrocode}
%    \end{macro}
%    \begin{macro}{\HOLOGO@ScaleBox@dvips}
%    \begin{macrocode}
\HOLOGO@temp{dvips}{%
  \def\HOLOGO@ScaleBox@dvips#1#2#3{%
    \special{ps:%
      gsave %
      currentpoint %
      currentpoint translate %
      #1 #2 scale %
      neg exch neg exch translate%
    }%
    #3%
    \special{ps:%
      currentpoint %
      grestore %
      moveto%
    }%
  }%
}
%    \end{macrocode}
%    \end{macro}
%    \begin{macro}{\HOLOGO@ScaleBox@dvipdfm}
%    \begin{macrocode}
\HOLOGO@temp{dvipdfm}{%
  \let\HOLOGO@ScaleBox@dvipdfm\HOLOGO@ScaleBox@dvips
}
%    \end{macrocode}
%    \end{macro}
%    Since \hologo{XeTeX} v0.6.
%    \begin{macro}{\HOLOGO@ScaleBox@xetex}
%    \begin{macrocode}
\HOLOGO@temp{xetex}{%
  \def\HOLOGO@ScaleBox@xetex#1#2#3{%
    \special{x:gsave}%
    \special{x:scale #1 #2}%
    #3%
    \special{x:grestore}%
  }%
}
%    \end{macrocode}
%    \end{macro}
%    \begin{macro}{\HOLOGO@ScaleBox@vtex}
%    \begin{macrocode}
\HOLOGO@temp{vtex}{%
  \def\HOLOGO@ScaleBox@vtex#1#2#3{%
    \special{r(#1,0,0,#2,0,0}%
    #3%
    \special{r)}%
  }%
}
%    \end{macrocode}
%    \end{macro}
%
%    \begin{macrocode}
\HOLOGO@AtEnd%
%</package>
%    \end{macrocode}
%
% \section{Test}
%
% \subsection{Catcode checks for loading}
%
%    \begin{macrocode}
%<*test1>
%    \end{macrocode}
%    \begin{macrocode}
\catcode`\{=1 %
\catcode`\}=2 %
\catcode`\#=6 %
\catcode`\@=11 %
\expandafter\ifx\csname count@\endcsname\relax
  \countdef\count@=255 %
\fi
\expandafter\ifx\csname @gobble\endcsname\relax
  \long\def\@gobble#1{}%
\fi
\expandafter\ifx\csname @firstofone\endcsname\relax
  \long\def\@firstofone#1{#1}%
\fi
\expandafter\ifx\csname loop\endcsname\relax
  \expandafter\@firstofone
\else
  \expandafter\@gobble
\fi
{%
  \def\loop#1\repeat{%
    \def\body{#1}%
    \iterate
  }%
  \def\iterate{%
    \body
      \let\next\iterate
    \else
      \let\next\relax
    \fi
    \next
  }%
  \let\repeat=\fi
}%
\def\RestoreCatcodes{}
\count@=0 %
\loop
  \edef\RestoreCatcodes{%
    \RestoreCatcodes
    \catcode\the\count@=\the\catcode\count@\relax
  }%
\ifnum\count@<255 %
  \advance\count@ 1 %
\repeat

\def\RangeCatcodeInvalid#1#2{%
  \count@=#1\relax
  \loop
    \catcode\count@=15 %
  \ifnum\count@<#2\relax
    \advance\count@ 1 %
  \repeat
}
\def\RangeCatcodeCheck#1#2#3{%
  \count@=#1\relax
  \loop
    \ifnum#3=\catcode\count@
    \else
      \errmessage{%
        Character \the\count@\space
        with wrong catcode \the\catcode\count@\space
        instead of \number#3%
      }%
    \fi
  \ifnum\count@<#2\relax
    \advance\count@ 1 %
  \repeat
}
\def\space{ }
\expandafter\ifx\csname LoadCommand\endcsname\relax
  \def\LoadCommand{\input hologo.sty\relax}%
\fi
\def\Test{%
  \RangeCatcodeInvalid{0}{47}%
  \RangeCatcodeInvalid{58}{64}%
  \RangeCatcodeInvalid{91}{96}%
  \RangeCatcodeInvalid{123}{255}%
  \catcode`\@=12 %
  \catcode`\\=0 %
  \catcode`\%=14 %
  \LoadCommand
  \RangeCatcodeCheck{0}{36}{15}%
  \RangeCatcodeCheck{37}{37}{14}%
  \RangeCatcodeCheck{38}{47}{15}%
  \RangeCatcodeCheck{48}{57}{12}%
  \RangeCatcodeCheck{58}{63}{15}%
  \RangeCatcodeCheck{64}{64}{12}%
  \RangeCatcodeCheck{65}{90}{11}%
  \RangeCatcodeCheck{91}{91}{15}%
  \RangeCatcodeCheck{92}{92}{0}%
  \RangeCatcodeCheck{93}{96}{15}%
  \RangeCatcodeCheck{97}{122}{11}%
  \RangeCatcodeCheck{123}{255}{15}%
  \RestoreCatcodes
}
\Test
\csname @@end\endcsname
\end
%    \end{macrocode}
%    \begin{macrocode}
%</test1>
%    \end{macrocode}
%
% \subsection{Spacefactor}
%
%    The space factor must be 1000 after a logo. If it is greater 1000
%    then the following space is a space after a sentence closing point.
%    If the space factor is smaller 1000 then an immediate following
%    dot is interpreted as abbreviation, not sentence closing point.
%
%    \begin{macrocode}
%<*test-spacefactor>
\NeedsTeXFormat{LaTeX2e}
\documentclass{article}
\usepackage{hologo}[2016/05/12]
\usepackage{kvsetkeys}
\usepackage{qstest}
\IncludeTests{*}
\LogTests{log}{*}{*}
\begin{document}
\begin{qstest}{spacefactor}{spacefactor}
\newcommand*{\Test}[1]{%
  \sbox0{%
    \hologo{#1}%
    \Expect*{1000 (#1)}*{\the\spacefactor\space(#1)}%
  }%
}%
\makeatletter
\def\TestList{}
\def\hologoEntry#1#2#3{%
  \edef\TestList{%
    \ifx\TestList\@empty
    \else
      \TestList,%
    \fi
    #1%
    \ifx\\#2\\%
    \else
      ={variant=#2}%
    \fi
  }%
}
\hologoList
\expandafter\kv@parse@normalized\expandafter{%
  \TestList
}{%
  \begingroup
    \let\@logo=\kv@key
    \ifx\kv@value\relax
    \else
      \expandafter\hologoLogoSetup\expandafter\@logo\expandafter{%
        \kv@value
      }%
    \fi
    \Test\@logo
  \endgroup
  \@gobbletwo
}
\end{qstest}
\end{document}
%</test-spacefactor>
%    \end{macrocode}
%
% \subsection{Complete list}
%
%    \begin{macrocode}
%<*test-list>
\NeedsTeXFormat{LaTeX2e}
\documentclass[12pt,a4paper]{article}
\usepackage{hologo}[2016/05/12]
\usepackage[T1]{fontenc}
\usepackage{lmodern}
\usepackage{parskip}
\usepackage[unicode]{hyperref}[2011/09/28]
\usepackage{bookmark}[2011/09/19]
\bookmarksetup{%
  numbered,%
  open,%
  openlevel=2,%
}
\renewcommand*{\contentsname}{List of logos}
\begin{document}
\tableofcontents
\def\TestFont#1#2#3#4#5#6{%
  \begingroup
    \usefont{#3}{#4}{#5}{#6}%
    \HologoVariant{#1}{#2}/\hologoVariant{#1}{#2}%
    \quad
    \begingroup\scriptsize\hologoVariant{#1}{#2}\endgroup
    \quad
  \endgroup
  (#3/#4/#5/#6)%
  \par
}
\makeatletter
\def\hologoEntry#1#2#3{%
  \section{%
    \HologoVariant{#1}{#2}/\hologoVariant{#1}{#2} %
    {[#1\ifx\\#2\\\else\space(#2)\fi]}% hash-ok
  }% braces around [] because of bug in tex4ht
  \begingroup
    \hypersetup{unicode=false}%
    \bookmark[%
      dest=\@currentHref,%
      rellevel=1,%
      keeplevel,%
    ]{%
      \HologoVariant{#1}{#2}/\hologoVariant{#1}{#2} %
      (PDFDocEncoding)%
    }%
  \endgroup
  \TestFont{#1}{#2}{OT1}{cmr}{m}{n}%
  \TestFont{#1}{#2}{OT1}{cmss}{m}{n}%
  \TestFont{#1}{#2}{OT1}{cmr}{b}{n}%
  \TestFont{#1}{#2}{OT1}{cmr}{m}{it}%
  \TestFont{#1}{#2}{OT1}{cmtt}{m}{n}%
  \TestFont{#1}{#2}{T1}{lmr}{m}{n}%
  \TestFont{#1}{#2}{T1}{lmss}{m}{n}%
  \TestFont{#1}{#2}{T1}{lmr}{b}{n}%
  \TestFont{#1}{#2}{T1}{lmr}{m}{it}%
  \TestFont{#1}{#2}{T1}{lmtt}{m}{n}%
  \TestFont{#1}{#2}{T1}{lmvtt}{m}{n}%
  \TestFont{#1}{#2}{T1}{qtm}{m}{n}%
  \TestFont{#1}{#2}{T1}{qhv}{m}{n}%
  \TestFont{#1}{#2}{T1}{qtm}{b}{n}%
  \TestFont{#1}{#2}{T1}{qtm}{m}{it}%
  \TestFont{#1}{#2}{T1}{qcr}{m}{n}%
  \newpage
}
\makeatother
\hologoList
\end{document}
%</test-list>
%    \end{macrocode}
%
% \section{Installation}
%
% \subsection{Download}
%
% \paragraph{Package.} This package is available on
% CTAN\footnote{\url{ftp://ftp.ctan.org/tex-archive/}}:
% \begin{description}
% \item[\CTAN{macros/latex/contrib/oberdiek/hologo.dtx}] The source file.
% \item[\CTAN{macros/latex/contrib/oberdiek/hologo.pdf}] Documentation.
% \end{description}
%
%
% \paragraph{Bundle.} All the packages of the bundle `oberdiek'
% are also available in a TDS compliant ZIP archive. There
% the packages are already unpacked and the documentation files
% are generated. The files and directories obey the TDS standard.
% \begin{description}
% \item[\CTAN{install/macros/latex/contrib/oberdiek.tds.zip}]
% \end{description}
% \emph{TDS} refers to the standard ``A Directory Structure
% for \TeX\ Files'' (\CTAN{tds/tds.pdf}). Directories
% with \xfile{texmf} in their name are usually organized this way.
%
% \subsection{Bundle installation}
%
% \paragraph{Unpacking.} Unpack the \xfile{oberdiek.tds.zip} in the
% TDS tree (also known as \xfile{texmf} tree) of your choice.
% Example (linux):
% \begin{quote}
%   |unzip oberdiek.tds.zip -d ~/texmf|
% \end{quote}
%
% \paragraph{Script installation.}
% Check the directory \xfile{TDS:scripts/oberdiek/} for
% scripts that need further installation steps.
% Package \xpackage{attachfile2} comes with the Perl script
% \xfile{pdfatfi.pl} that should be installed in such a way
% that it can be called as \texttt{pdfatfi}.
% Example (linux):
% \begin{quote}
%   |chmod +x scripts/oberdiek/pdfatfi.pl|\\
%   |cp scripts/oberdiek/pdfatfi.pl /usr/local/bin/|
% \end{quote}
%
% \subsection{Package installation}
%
% \paragraph{Unpacking.} The \xfile{.dtx} file is a self-extracting
% \docstrip\ archive. The files are extracted by running the
% \xfile{.dtx} through \plainTeX:
% \begin{quote}
%   \verb|tex hologo.dtx|
% \end{quote}
%
% \paragraph{TDS.} Now the different files must be moved into
% the different directories in your installation TDS tree
% (also known as \xfile{texmf} tree):
% \begin{quote}
% \def\t{^^A
% \begin{tabular}{@{}>{\ttfamily}l@{ $\rightarrow$ }>{\ttfamily}l@{}}
%   hologo.sty & tex/generic/oberdiek/hologo.sty\\
%   hologo.pdf & doc/latex/oberdiek/hologo.pdf\\
%   example/hologo-example.tex & doc/latex/oberdiek/example/hologo-example.tex\\
%   test/hologo-test1.tex & doc/latex/oberdiek/test/hologo-test1.tex\\
%   test/hologo-test-spacefactor.tex & doc/latex/oberdiek/test/hologo-test-spacefactor.tex\\
%   test/hologo-test-list.tex & doc/latex/oberdiek/test/hologo-test-list.tex\\
%   hologo.dtx & source/latex/oberdiek/hologo.dtx\\
% \end{tabular}^^A
% }^^A
% \sbox0{\t}^^A
% \ifdim\wd0>\linewidth
%   \begingroup
%     \advance\linewidth by\leftmargin
%     \advance\linewidth by\rightmargin
%   \edef\x{\endgroup
%     \def\noexpand\lw{\the\linewidth}^^A
%   }\x
%   \def\lwbox{^^A
%     \leavevmode
%     \hbox to \linewidth{^^A
%       \kern-\leftmargin\relax
%       \hss
%       \usebox0
%       \hss
%       \kern-\rightmargin\relax
%     }^^A
%   }^^A
%   \ifdim\wd0>\lw
%     \sbox0{\small\t}^^A
%     \ifdim\wd0>\linewidth
%       \ifdim\wd0>\lw
%         \sbox0{\footnotesize\t}^^A
%         \ifdim\wd0>\linewidth
%           \ifdim\wd0>\lw
%             \sbox0{\scriptsize\t}^^A
%             \ifdim\wd0>\linewidth
%               \ifdim\wd0>\lw
%                 \sbox0{\tiny\t}^^A
%                 \ifdim\wd0>\linewidth
%                   \lwbox
%                 \else
%                   \usebox0
%                 \fi
%               \else
%                 \lwbox
%               \fi
%             \else
%               \usebox0
%             \fi
%           \else
%             \lwbox
%           \fi
%         \else
%           \usebox0
%         \fi
%       \else
%         \lwbox
%       \fi
%     \else
%       \usebox0
%     \fi
%   \else
%     \lwbox
%   \fi
% \else
%   \usebox0
% \fi
% \end{quote}
% If you have a \xfile{docstrip.cfg} that configures and enables \docstrip's
% TDS installing feature, then some files can already be in the right
% place, see the documentation of \docstrip.
%
% \subsection{Refresh file name databases}
%
% If your \TeX~distribution
% (\teTeX, \mikTeX, \dots) relies on file name databases, you must refresh
% these. For example, \teTeX\ users run \verb|texhash| or
% \verb|mktexlsr|.
%
% \subsection{Some details for the interested}
%
% \paragraph{Attached source.}
%
% The PDF documentation on CTAN also includes the
% \xfile{.dtx} source file. It can be extracted by
% AcrobatReader 6 or higher. Another option is \textsf{pdftk},
% e.g. unpack the file into the current directory:
% \begin{quote}
%   \verb|pdftk hologo.pdf unpack_files output .|
% \end{quote}
%
% \paragraph{Unpacking with \LaTeX.}
% The \xfile{.dtx} chooses its action depending on the format:
% \begin{description}
% \item[\plainTeX:] Run \docstrip\ and extract the files.
% \item[\LaTeX:] Generate the documentation.
% \end{description}
% If you insist on using \LaTeX\ for \docstrip\ (really,
% \docstrip\ does not need \LaTeX), then inform the autodetect routine
% about your intention:
% \begin{quote}
%   \verb|latex \let\install=y\input{hologo.dtx}|
% \end{quote}
% Do not forget to quote the argument according to the demands
% of your shell.
%
% \paragraph{Generating the documentation.}
% You can use both the \xfile{.dtx} or the \xfile{.drv} to generate
% the documentation. The process can be configured by the
% configuration file \xfile{ltxdoc.cfg}. For instance, put this
% line into this file, if you want to have A4 as paper format:
% \begin{quote}
%   \verb|\PassOptionsToClass{a4paper}{article}|
% \end{quote}
% An example follows how to generate the
% documentation with pdf\LaTeX:
% \begin{quote}
%\begin{verbatim}
%pdflatex hologo.dtx
%makeindex -s gind.ist hologo.idx
%pdflatex hologo.dtx
%makeindex -s gind.ist hologo.idx
%pdflatex hologo.dtx
%\end{verbatim}
% \end{quote}
%
% \section{Catalogue}
%
% The following XML file can be used as source for the
% \href{http://mirror.ctan.org/help/Catalogue/catalogue.html}{\TeX\ Catalogue}.
% The elements \texttt{caption} and \texttt{description} are imported
% from the original XML file from the Catalogue.
% The name of the XML file in the Catalogue is \xfile{hologo.xml}.
%    \begin{macrocode}
%<*catalogue>
<?xml version='1.0' encoding='us-ascii'?>
<!DOCTYPE entry SYSTEM 'catalogue.dtd'>
<entry datestamp='$Date$' modifier='$Author$' id='hologo'>
  <name>hologo</name>
  <caption>A collection of logos with bookmark support.</caption>
  <authorref id='auth:oberdiek'/>
  <copyright owner='Heiko Oberdiek' year='2010-2012'/>
  <license type='lppl1.3'/>
  <version number='1.10'/>
  <description>
    The package defines a single command <tt>\hologo</tt>, whose
    argument is the usual case-confused ASCII version of the logo.
    The command is bookmark-enabled, so that every logo becomes
    available in bookmarks without further work.
    <p/>
    The package is part of the <xref refid='oberdiek'>oberdiek</xref>
    bundle.
  </description>
  <documentation details='Package documentation'
      href='ctan:/macros/latex/contrib/oberdiek/hologo.pdf'/>
  <ctan file='true' path='/macros/latex/contrib/oberdiek/hologo.dtx'/>
  <miktex location='oberdiek'/>
  <texlive location='oberdiek'/>
  <install path='/macros/latex/contrib/oberdiek/oberdiek.tds.zip'/>
</entry>
%</catalogue>
%    \end{macrocode}
%
% \begin{thebibliography}{9}
% \raggedright
%
% \bibitem{btxdoc}
% Oren Patashnik,
% \textit{\hologo{BibTeX}ing},
% 1988-02-08.\\
% \CTAN{biblio/bibtex/base/}
%
% \bibitem{dtklogos}
% Gerd Neugebauer, DANTE,
% \textit{Package \xpackage{dtklogos}},
% 2011-04-25.\\
% \CTAN{usergrps/dante/dtk/dtklogos.sty}
%
% \bibitem{etexman}
% The \hologo{NTS} Team,
% \textit{The \hologo{eTeX} manual},
% 1998-02.\\
% \CTAN{systems/e-tex/v2/doc/}
%
% \bibitem{ExTeX-FAQ}
% The \hologo{ExTeX} group,
% \textit{\hologo{ExTeX}: FAQ -- How is \hologo{ExTeX} typeset?},
% 2007-04-14.\\
% \url{http://www.extex.org/documentation/faq.html}
%
% \bibitem{LyX}
% %@MISC{ LyX,
% %  title = {{LyX 2.0.0 -- The Document Processor [Computer software and manual]}},
% %  author = {{The LyX Team}},
% %  howpublished = {Internet: http://www.lyx.org},
% %  year = {2011-05-08},
% %  note = {Retrieved May 10, 2011, from http://www.lyx.org},
% %  url = {http://www.lyx.org/}
% %}
% The \hologo{LyX} Team,
% \textit{\hologo{LyX} -- The Document Processor},
% 2011-05-08.\\
% \url{http://www.lyx.org/}
%
% \bibitem{OzTeX}
% Andrew Trevorrow,
% \hologo{OzTeX} FAQ: What is the correct way to typeset ``\hologo{OzTeX}''?,
% 2011-09-15 (visited).
% \url{http://www.trevorrow.com/oztex/ozfaq.html#oztex-logo}
%
% \bibitem{PiCTeX}
% Michael Wichura,
% \textit{The \hologo{PiCTeX} macro package},
% 1987-09-21.
% \CTAN{graphics/pictex/}
%
% \bibitem{scrlogo}
% Markus Kohm,
% \textit{\hologo{KOMAScript} Datei \xfile{scrlogo.dtx}},
% 2009-01-30.\\
% \CTAN{install/macros/latex/contrib/komascript.tds.zip}
%
% \end{thebibliography}
%
% \begin{History}
%   \begin{Version}{2010/04/08 v1.0}
%   \item
%     The first version.
%   \end{Version}
%   \begin{Version}{2010/04/16 v1.1}
%   \item
%     \cs{Hologo} added for support of logos at start of a sentence.
%   \item
%     \cs{hologoSetup} and \cs{hologoLogoSetup} added.
%   \item
%     Options \xoption{break}, \xoption{hyphenbreak}, \xoption{spacebreak}
%     added.
%   \item
%     Variant support added by option \xoption{variant}.
%   \end{Version}
%   \begin{Version}{2010/04/24 v1.2}
%   \item
%     \hologo{LaTeX3} added.
%   \item
%     \hologo{VTeX} added.
%   \end{Version}
%   \begin{Version}{2010/11/21 v1.3}
%   \item
%     \hologo{iniTeX}, \hologo{virTeX} added.
%   \end{Version}
%   \begin{Version}{2011/03/25 v1.4}
%   \item
%     \hologo{ConTeXt} with variants added.
%   \item
%     Option \xoption{discretionarybreak} added as refinement for
%     option \xoption{break}.
%   \end{Version}
%   \begin{Version}{2011/04/21 v1.5}
%   \item
%     Wrong TDS directory for test files fixed.
%   \end{Version}
%   \begin{Version}{2011/10/01 v1.6}
%   \item
%     Support for package \xpackage{tex4ht} added.
%   \item
%     Support for \cs{csname} added if \cs{ifincsname} is available.
%   \item
%     New logos:
%     \hologo{(La)TeX},
%     \hologo{biber},
%     \hologo{BibTeX} (\xoption{sc}, \xoption{sf}),
%     \hologo{emTeX},
%     \hologo{ExTeX},
%     \hologo{KOMAScript},
%     \hologo{La},
%     \hologo{LyX},
%     \hologo{MiKTeX},
%     \hologo{NTS},
%     \hologo{OzMF},
%     \hologo{OzMP},
%     \hologo{OzTeX},
%     \hologo{OzTtH},
%     \hologo{PCTeX},
%     \hologo{PiC},
%     \hologo{PiCTeX},
%     \hologo{METAFONT},
%     \hologo{MetaFun},
%     \hologo{METAPOST},
%     \hologo{MetaPost},
%     \hologo{SLiTeX} (\xoption{lift}, \xoption{narrow}, \xoption{simple}),
%     \hologo{SliTeX} (\xoption{narrow}, \xoption{simple}, \xoption{lift}),
%     \hologo{teTeX}.
%   \item
%     Fixes:
%     \hologo{iniTeX},
%     \hologo{pdfLaTeX},
%     \hologo{pdfTeX},
%     \hologo{virTeX}.
%   \item
%     \cs{hologoFontSetup} and \cs{hologoLogoFontSetup} added.
%   \item
%     \cs{hologoVariant} and \cs{HologoVariant} added.
%   \end{Version}
%   \begin{Version}{2011/11/22 v1.7}
%   \item
%     New logos:
%     \hologo{BibTeX8},
%     \hologo{LaTeXML},
%     \hologo{SageTeX},
%     \hologo{TeX4ht},
%     \hologo{TTH}.
%   \item
%     \hologo{Xe} and friends: Driver stuff fixed.
%   \item
%     \hologo{Xe} and friends: Support for italic added.
%   \item
%     \hologo{Xe} and friends: Package support for \xpackage{pgf}
%     and \xpackage{pstricks} added.
%   \end{Version}
%   \begin{Version}{2011/11/29 v1.8}
%   \item
%     New logos:
%     \hologo{HanTheThanh}.
%   \end{Version}
%   \begin{Version}{2011/12/21 v1.9}
%   \item
%     Patch for package \xpackage{ifxetex} added for the case that
%     \cs{newif} is undefined in \hologo{iniTeX}.
%   \item
%     Some fixes for \hologo{iniTeX}.
%   \end{Version}
%   \begin{Version}{2012/04/26 v1.10}
%   \item
%     Fix in bookmark version of logo ``\hologo{HanTheThanh}''.
%   \end{Version}
%   \begin{Version}{2016/05/12 v1.11}
%   \item
%     Update HOLOGO@IfCharExists (previously in texlive)
%   \item define pdfliteral in current luatex.
%   \end{Version}
% \end{History}
%
% \PrintIndex
%
% \Finale
\endinput
%
        \else
          \input hologo.cfg\relax
        \fi
      \fi
    }%
    \ltx@IfUndefined{newread}{%
      \chardef\HOLOGO@temp=15 %
      \def\HOLOGO@CheckRead{%
        \ifeof\HOLOGO@temp
          \HOLOGO@InputIfExists
        \else
          \ifcase\HOLOGO@temp
            \@PackageWarningNoLine{hologo}{%
              Configuration file ignored, because\MessageBreak
              a free read register could not be found%
            }%
          \else
            \begingroup
              \count\ltx@cclv=\HOLOGO@temp
              \advance\ltx@cclv by \ltx@minusone
              \edef\x{\endgroup
                \chardef\noexpand\HOLOGO@temp=\the\count\ltx@cclv
                \relax
              }%
            \x
          \fi
        \fi
      }%
    }{%
      \csname newread\endcsname\HOLOGO@temp
      \HOLOGO@InputIfExists
    }%
  }{%
    \edef\HOLOGO@temp{\pdf@filesize{hologo.cfg}}%
    \ifx\HOLOGO@temp\ltx@empty
    \else
      \ifnum\HOLOGO@temp>0 %
        \begingroup
          \def\x{LaTeX2e}%
        \expandafter\endgroup
        \ifx\fmtname\x
          % \iffalse meta-comment
%
% File: hologo.dtx
% Version: 2016/05/12 v1.11
% Info: A logo collection with bookmark support
%
% Copyright (C) 2010-2012 by
%    Heiko Oberdiek <heiko.oberdiek at googlemail.com>
%
% This work may be distributed and/or modified under the
% conditions of the LaTeX Project Public License, either
% version 1.3c of this license or (at your option) any later
% version. This version of this license is in
%    http://www.latex-project.org/lppl/lppl-1-3c.txt
% and the latest version of this license is in
%    http://www.latex-project.org/lppl.txt
% and version 1.3 or later is part of all distributions of
% LaTeX version 2005/12/01 or later.
%
% This work has the LPPL maintenance status "maintained".
%
% This Current Maintainer of this work is Heiko Oberdiek.
%
% The Base Interpreter refers to any `TeX-Format',
% because some files are installed in TDS:tex/generic//.
%
% This work consists of the main source file hologo.dtx
% and the derived files
%    hologo.sty, hologo.pdf, hologo.ins, hologo.drv, hologo-example.tex,
%    hologo-test1.tex, hologo-test-spacefactor.tex,
%    hologo-test-list.tex.
%
% Distribution:
%    CTAN:macros/latex/contrib/oberdiek/hologo.dtx
%    CTAN:macros/latex/contrib/oberdiek/hologo.pdf
%
% Unpacking:
%    (a) If hologo.ins is present:
%           tex hologo.ins
%    (b) Without hologo.ins:
%           tex hologo.dtx
%    (c) If you insist on using LaTeX
%           latex \let\install=y\input{hologo.dtx}
%        (quote the arguments according to the demands of your shell)
%
% Documentation:
%    (a) If hologo.drv is present:
%           latex hologo.drv
%    (b) Without hologo.drv:
%           latex hologo.dtx; ...
%    The class ltxdoc loads the configuration file ltxdoc.cfg
%    if available. Here you can specify further options, e.g.
%    use A4 as paper format:
%       \PassOptionsToClass{a4paper}{article}
%
%    Programm calls to get the documentation (example):
%       pdflatex hologo.dtx
%       makeindex -s gind.ist hologo.idx
%       pdflatex hologo.dtx
%       makeindex -s gind.ist hologo.idx
%       pdflatex hologo.dtx
%
% Installation:
%    TDS:tex/generic/oberdiek/hologo.sty
%    TDS:doc/latex/oberdiek/hologo.pdf
%    TDS:doc/latex/oberdiek/example/hologo-example.tex
%    TDS:doc/latex/oberdiek/test/hologo-test1.tex
%    TDS:doc/latex/oberdiek/test/hologo-test-spacefactor.tex
%    TDS:doc/latex/oberdiek/test/hologo-test-list.tex
%    TDS:source/latex/oberdiek/hologo.dtx
%
%<*ignore>
\begingroup
  \catcode123=1 %
  \catcode125=2 %
  \def\x{LaTeX2e}%
\expandafter\endgroup
\ifcase 0\ifx\install y1\fi\expandafter
         \ifx\csname processbatchFile\endcsname\relax\else1\fi
         \ifx\fmtname\x\else 1\fi\relax
\else\csname fi\endcsname
%</ignore>
%<*install>
\input docstrip.tex
\Msg{************************************************************************}
\Msg{* Installation}
\Msg{* Package: hologo 2016/05/12 v1.11 A logo collection with bookmark support (HO)}
\Msg{************************************************************************}

\keepsilent
\askforoverwritefalse

\let\MetaPrefix\relax
\preamble

This is a generated file.

Project: hologo
Version: 2016/05/12 v1.11

Copyright (C) 2010-2012 by
   Heiko Oberdiek <heiko.oberdiek at googlemail.com>

This work may be distributed and/or modified under the
conditions of the LaTeX Project Public License, either
version 1.3c of this license or (at your option) any later
version. This version of this license is in
   http://www.latex-project.org/lppl/lppl-1-3c.txt
and the latest version of this license is in
   http://www.latex-project.org/lppl.txt
and version 1.3 or later is part of all distributions of
LaTeX version 2005/12/01 or later.

This work has the LPPL maintenance status "maintained".

This Current Maintainer of this work is Heiko Oberdiek.

The Base Interpreter refers to any `TeX-Format',
because some files are installed in TDS:tex/generic//.

This work consists of the main source file hologo.dtx
and the derived files
   hologo.sty, hologo.pdf, hologo.ins, hologo.drv, hologo-example.tex,
   hologo-test1.tex, hologo-test-spacefactor.tex,
   hologo-test-list.tex.

\endpreamble
\let\MetaPrefix\DoubleperCent

\generate{%
  \file{hologo.ins}{\from{hologo.dtx}{install}}%
  \file{hologo.drv}{\from{hologo.dtx}{driver}}%
  \usedir{tex/generic/oberdiek}%
  \file{hologo.sty}{\from{hologo.dtx}{package}}%
  \usedir{doc/latex/oberdiek/example}%
  \file{hologo-example.tex}{\from{hologo.dtx}{example}}%
  \usedir{doc/latex/oberdiek/test}%
  \file{hologo-test1.tex}{\from{hologo.dtx}{test1}}%
  \file{hologo-test-spacefactor.tex}{\from{hologo.dtx}{test-spacefactor}}%
  \file{hologo-test-list.tex}{\from{hologo.dtx}{test-list}}%
  \nopreamble
  \nopostamble
  \usedir{source/latex/oberdiek/catalogue}%
  \file{hologo.xml}{\from{hologo.dtx}{catalogue}}%
}

\catcode32=13\relax% active space
\let =\space%
\Msg{************************************************************************}
\Msg{*}
\Msg{* To finish the installation you have to move the following}
\Msg{* file into a directory searched by TeX:}
\Msg{*}
\Msg{*     hologo.sty}
\Msg{*}
\Msg{* To produce the documentation run the file `hologo.drv'}
\Msg{* through LaTeX.}
\Msg{*}
\Msg{* Happy TeXing!}
\Msg{*}
\Msg{************************************************************************}

\endbatchfile
%</install>
%<*ignore>
\fi
%</ignore>
%<*driver>
\NeedsTeXFormat{LaTeX2e}
\ProvidesFile{hologo.drv}%
  [2016/05/12 v1.11 A logo collection with bookmark support (HO)]%
\documentclass{ltxdoc}
\usepackage{holtxdoc}[2011/11/22]
\usepackage{hologo}[2016/05/12]
\usepackage{longtable}
\usepackage{array}
\usepackage{paralist}
%\usepackage[T1]{fontenc}
%\usepackage{lmodern}
\begin{document}
  \DocInput{hologo.dtx}%
\end{document}
%</driver>
% \fi
%
%
% \CharacterTable
%  {Upper-case    \A\B\C\D\E\F\G\H\I\J\K\L\M\N\O\P\Q\R\S\T\U\V\W\X\Y\Z
%   Lower-case    \a\b\c\d\e\f\g\h\i\j\k\l\m\n\o\p\q\r\s\t\u\v\w\x\y\z
%   Digits        \0\1\2\3\4\5\6\7\8\9
%   Exclamation   \!     Double quote  \"     Hash (number) \#
%   Dollar        \$     Percent       \%     Ampersand     \&
%   Acute accent  \'     Left paren    \(     Right paren   \)
%   Asterisk      \*     Plus          \+     Comma         \,
%   Minus         \-     Point         \.     Solidus       \/
%   Colon         \:     Semicolon     \;     Less than     \<
%   Equals        \=     Greater than  \>     Question mark \?
%   Commercial at \@     Left bracket  \[     Backslash     \\
%   Right bracket \]     Circumflex    \^     Underscore    \_
%   Grave accent  \`     Left brace    \{     Vertical bar  \|
%   Right brace   \}     Tilde         \~}
%
% \GetFileInfo{hologo.drv}
%
% \title{The \xpackage{hologo} package}
% \date{2016/05/12 v1.11}
% \author{Heiko Oberdiek\\\xemail{heiko.oberdiek at googlemail.com}}
%
% \maketitle
%
% \begin{abstract}
% This package starts a collection of logos with support for bookmarks
% strings.
% \end{abstract}
%
% \tableofcontents
%
% \section{Documentation}
%
% \subsection{Logo macros}
%
% \begin{declcs}{hologo} \M{name}
% \end{declcs}
% Macro \cs{hologo} sets the logo with name \meta{name}.
% The following table shows the supported names.
%
% \begingroup
%   \def\hologoEntry#1#2#3{^^A
%     #1&#2&\hologoLogoSetup{#1}{variant=#2}\hologo{#1}&#3\tabularnewline
%   }
%   \begin{longtable}{>{\ttfamily}l>{\ttfamily}lll}
%     \rmfamily\bfseries{name} & \rmfamily\bfseries variant
%     & \bfseries logo & \bfseries since\\
%     \hline
%     \endhead
%     \hologoList
%   \end{longtable}
% \endgroup
%
% \begin{declcs}{Hologo} \M{name}
% \end{declcs}
% Macro \cs{Hologo} starts the logo \meta{name} with an uppercase
% letter. As an exception small greek letters are not converted
% to uppercase. Examples, see \hologo{eTeX} and \hologo{ExTeX}.
%
% \subsection{Setup macros}
%
% The package does not support package options, but the following
% setup macros can be used to set options.
%
% \begin{declcs}{hologoSetup} \M{key value list}
% \end{declcs}
% Macro \cs{hologoSetup} sets global options.
%
% \begin{declcs}{hologoLogoSetup} \M{logo} \M{key value list}
% \end{declcs}
% Some options can also be used to configure a logo.
% These settings take precedence over global option settings.
%
% \subsection{Options}\label{sec:options}
%
% There are boolean and string options:
% \begin{description}
% \item[Boolean option:]
% It takes |true| or |false|
% as value. If the value is omitted, then |true| is used.
% \item[String option:]
% A value must be given as string. (But the string might be empty.)
% \end{description}
% The following options can be used both in \cs{hologoSetup}
% and \cs{hologoLogoSetup}:
% \begin{description}
% \def\entry#1{\item[\xoption{#1}:]}
% \entry{break}
%   enables or disables line breaks inside the logo. This setting is
%   refined by options \xoption{hyphenbreak}, \xoption{spacebreak}
%   or \xoption{discretionarybreak}.
%   Default is |false|.
% \entry{hyphenbreak}
%   enables or disables the line break right after the hyphen character.
% \entry{spacebreak}
%   enables or disables line breaks at space characters.
% \entry{discretionarybreak}
%   enables or disables line breaks at hyphenation points
%   (inserted by \cs{-}).
% \end{description}
% Macro \cs{hologoLogoSetup} also knows:
% \begin{description}
% \item[\xoption{variant}:]
%   This is a string option. It specifies a variant of a logo that
%   must exist. An empty string selects the package default variant.
% \end{description}
% Example:
% \begin{quote}
%   |\hologoSetup{break=false}|\\
%   |\hologoLogoSetup{plainTeX}{variant=hyphen,hyphenbreak}|\\
%   Then ``plain-\TeX'' contains one break point after the hyphen.
% \end{quote}
%
% \subsection{Driver options}
%
% Sometimes graphical operations are needed to construct some
% glyphs (e.g.\ \hologo{XeTeX}). If package \xpackage{graphics}
% or package \xpackage{pgf} are found, then the macros are taken
% from there. Otherwise the packge defines its own operations
% and therefore needs the driver information. Many drivers are
% detected automatically (\hologo{pdfTeX}/\hologo{LuaTeX}
% in PDF mode, \hologo{XeTeX}, \hologo{VTeX}). These have precedence
% over a driver option. The driver can be given as package option
% or using \cs{hologoDriverSetup}.
% The following list contains the recognized driver options:
% \begin{itemize}
% \item \xoption{pdftex}, \xoption{luatex}
% \item \xoption{dvipdfm}, \xoption{dvipdfmx}
% \item \xoption{dvips}, \xoption{dvipsone}, \xoption{xdvi}
% \item \xoption{xetex}
% \item \xoption{vtex}
% \end{itemize}
% The left driver of a line is the driver name that is used internally.
% The following names are aliases for drivers that use the
% same method. Therefore the entry in the \xext{log} file for
% the used driver prints the internally used driver name.
% \begin{description}
% \item[\xoption{driverfallback}:]
%   This option expects a driver that is used,
%   if the driver could not be detected automatically.
% \end{description}
%
% \begin{declcs}{hologoDriverSetup} \M{driver option}
% \end{declcs}
% The driver can also be configured after package loading
% using \cs{hologoDriverSetup}, also the way for \hologo{plainTeX}
% to setup the driver.
%
% \subsection{Font setup}
%
% Some logos require a special font, but should also be usable by
% \hologo{plainTeX}. Therefore the package provides some ways
% to influence the font settings. The options below
% take font settings as values. Both font commands
% such as \cs{sffamily} and macros that take one argument
% like \cs{textsf} can be used.
%
% \begin{declcs}{hologoFontSetup} \M{key value list}
% \end{declcs}
% Macro \cs{hologoFontSetup} sets the fonts for all logos.
% Supported keys:
% \begin{description}
% \def\entry#1{\item[\xoption{#1}:]}
% \entry{general}
%   This font is used for all logos. The default is empty.
%   That means no special font is used.
% \entry{bibsf}
%   This font is used for
%   {\hologoLogoSetup{BibTeX}{variant=sf}\hologo{BibTeX}}
%   with variant \xoption{sf}.
% \entry{rm}
%   This font is a serif font. It is used for \hologo{ExTeX}.
% \entry{sc}
%   This font specifies a small caps font. It is used for
%   {\hologoLogoSetup{BibTeX}{variant=sc}\hologo{BibTeX}}
%   with variant \xoption{sc}.
% \entry{sf}
%   This font specifies a sans serif font. The default
%   is \cs{sffamily}, then \cs{sf} is tried. Otherwise
%   a warning is given. It is used by \hologo{KOMAScript}.
% \entry{sy}
%   This is the font for math symbols (e.g. cmsy).
%   It is used by \hologo{AmS}, \hologo{NTS}, \hologo{ExTeX}.
% \entry{logo}
%   \hologo{METAFONT} and \hologo{METAPOST} are using that font.
%   In \hologo{LaTeX} \cs{logofamily} is used and
%   the definitions of package \xpackage{mflogo} are used
%   if the package is not loaded.
%   Otherwise the \cs{tenlogo} is used and defined
%   if it does not already exists.
% \end{description}
%
% \begin{declcs}{hologoLogoFontSetup} \M{logo} \M{key value list}
% \end{declcs}
% Fonts can also be set for a logo or logo component separately,
% see the following list.
% The keys are the same as for \cs{hologoFontSetup}.
%
% \begin{longtable}{>{\ttfamily}l>{\sffamily}ll}
%   \meta{logo} & keys & result\\
%   \hline
%   \endhead
%   BibTeX & bibsf & {\hologoLogoSetup{BibTeX}{variant=sf}\hologo{BibTeX}}\\[.5ex]
%   BibTeX & sc & {\hologoLogoSetup{BibTeX}{variant=sc}\hologo{BibTeX}}\\[.5ex]
%   ExTeX & rm & \hologo{ExTeX}\\
%   SliTeX & rm & \hologo{SliTeX}\\[.5ex]
%   AmS & sy & \hologo{AmS}\\
%   ExTeX & sy & \hologo{ExTeX}\\
%   NTS & sy & \hologo{NTS}\\[.5ex]
%   KOMAScript & sf & \hologo{KOMAScript}\\[.5ex]
%   METAFONT & logo & \hologo{METAFONT}\\
%   METAPOST & logo & \hologo{METAPOST}\\[.5ex]
%   SliTeX & sc \hologo{SliTeX}
% \end{longtable}
%
% \subsubsection{Font order}
%
% For all logos the font \xoption{general} is applied first.
% Example:
%\begin{quote}
%|\hologoFontSetup{general=\color{red}}|
%\end{quote}
% will print red logos.
% Then if the font uses a special font \xoption{sf}, for example,
% the font is applied that is setup by \cs{hologoLogoFontSetup}.
% If this font is not setup, then the common font setup
% by \cs{hologoFontSetup} is used. Otherwise a warning is given,
% that there is no font configured.
%
% \subsection{Additional user macros}
%
% Usually a variant of a logo is configured by using
% \cs{hologoLogoSetup}, because it is bad style to mix
% different variants of the same logo in the same text.
% There the following macros are a convenience for testing.
%
% \begin{declcs}{hologoVariant} \M{name} \M{variant}\\
%   \cs{HologoVariant} \M{name} \M{variant}
% \end{declcs}
% Logo \meta{name} is set using \meta{variant} that specifies
% explicitely which variant of the macro is used. If the argument
% is empty, then the default form of the logo is used
% (configurable by \cs{hologoLogoSetup}).
%
% \cs{HologoVariant} is used if the logo is set in a context
% that needs an uppercase first letter (beginning of a sentence, \dots).
%
% \begin{declcs}{hologoList}\\
%   \cs{hologoEntry} \M{logo} \M{variant} \M{since}
% \end{declcs}
% Macro \cs{hologoList} contains all logos that are provided
% by the package including variants. The list consists of calls
% of \cs{hologoEntry} with three arguments starting with the
% logo name \meta{logo} and its variant \meta{variant}. An empty
% variant means the current default. Argument \meta{since} specifies
% with version of the package \xpackage{hologo} is needed to get
% the logo. If the logo is fixed, then the date gets updated.
% Therefore the date \meta{since} is not exactly the date of
% the first introduction, but rather the date of the latest fix.
%
% Before \cs{hologoList} can be used, macro \cs{hologoEntry} needs
% a definition. The example file in section \ref{sec:example}
% shows applications of \cs{hologoList}.
%
% \subsection{Supported contexts}
%
% Macros \cs{hologo} and friends support special contexts:
% \begin{itemize}
% \item \hologo{LaTeX}'s protection mechanism.
% \item Bookmarks of package \xpackage{hyperref}.
% \item Package \xpackage{tex4ht}.
% \item The macros can be used inside \cs{csname} constructs,
%   if \cs{ifincsname} is available (\hologo{pdfTeX}, \hologo{XeTeX},
%   \hologo{LuaTeX}).
% \end{itemize}
%
% \subsection{Example}
% \label{sec:example}
%
% The following example prints the logos in different fonts.
%    \begin{macrocode}
%<*example>
%<<verbatim
\NeedsTeXFormat{LaTeX2e}
\documentclass[a4paper]{article}
\usepackage[
  hmargin=20mm,
  vmargin=20mm,
]{geometry}
\pagestyle{empty}
\usepackage{hologo}[2016/05/12]
\usepackage{longtable}
\usepackage{array}
\setlength{\extrarowheight}{2pt}
\usepackage[T1]{fontenc}
\usepackage{lmodern}
\usepackage{pdflscape}
\usepackage[
  pdfencoding=auto,
]{hyperref}
\hypersetup{
  pdfauthor={Heiko Oberdiek},
  pdftitle={Example for package `hologo'},
  pdfsubject={Logos with fonts lmr, lmss, qtm, qpl, qhv},
}
\usepackage{bookmark}

% Print the logo list on the console

\begingroup
  \typeout{}%
  \typeout{*** Begin of logo list ***}%
  \newcommand*{\hologoEntry}[3]{%
    \typeout{#1 \ifx\\#2\\\else(#2) \fi[#3]}%
  }%
  \hologoList
  \typeout{*** End of logo list ***}%
  \typeout{}%
\endgroup

\begin{document}
\begin{landscape}

  \section{Example file for package `hologo'}

  % Table for font names

  \begin{longtable}{>{\bfseries}ll}
    \textbf{font} & \textbf{Font name}\\
    \hline
    lmr & Latin Modern Roman\\
    lmss & Latin Modern Sans\\
    qtm & \TeX\ Gyre Termes\\
    qhv & \TeX\ Gyre Heros\\
    qpl & \TeX\ Gyre Pagella\\
  \end{longtable}

  % Logo list with logos in different fonts

  \begingroup
    \newcommand*{\SetVariant}[2]{%
      \ifx\\#2\\%
      \else
        \hologoLogoSetup{#1}{variant=#2}%
      \fi
    }%
    \newcommand*{\hologoEntry}[3]{%
      \SetVariant{#1}{#2}%
      \raisebox{1em}[0pt][0pt]{\hypertarget{#1@#2}{}}%
      \bookmark[%
        dest={#1@#2},%
      ]{%
        #1\ifx\\#2\\\else\space(#2)\fi: \Hologo{#1}, \hologo{#1} %
        [Unicode]%
      }%
      \hypersetup{unicode=false}%
      \bookmark[%
        dest={#1@#2},%
      ]{%
        #1\ifx\\#2\\\else\space(#2)\fi: \Hologo{#1}, \hologo{#1} %
        [PDFDocEncoding]%
      }%
      \texttt{#1}%
      &%
      \texttt{#2}%
      &%
      \Hologo{#1}%
      &%
      \SetVariant{#1}{#2}%
      \hologo{#1}%
      &%
      \SetVariant{#1}{#2}%
      \fontfamily{qtm}\selectfont
      \hologo{#1}%
      &%
      \SetVariant{#1}{#2}%
      \fontfamily{qpl}\selectfont
      \hologo{#1}%
      &%
      \SetVariant{#1}{#2}%
      \textsf{\hologo{#1}}%
      &%
      \SetVariant{#1}{#2}%
      \fontfamily{qhv}\selectfont
      \hologo{#1}%
      \tabularnewline
    }%
    \begin{longtable}{llllllll}%
      \textbf{\textit{logo}} & \textbf{\textit{variant}} &
      \texttt{\string\Hologo} &
      \textbf{lmr} & \textbf{qtm} & \textbf{qpl} &
      \textbf{lmss} & \textbf{qhv}
      \tabularnewline
      \hline
      \endhead
      \hologoList
    \end{longtable}%
  \endgroup

\end{landscape}
\end{document}
%verbatim
%</example>
%    \end{macrocode}
%
% \StopEventually{
% }
%
% \section{Implementation}
%    \begin{macrocode}
%<*package>
%    \end{macrocode}
%    Reload check, especially if the package is not used with \LaTeX.
%    \begin{macrocode}
\begingroup\catcode61\catcode48\catcode32=10\relax%
  \catcode13=5 % ^^M
  \endlinechar=13 %
  \catcode35=6 % #
  \catcode39=12 % '
  \catcode44=12 % ,
  \catcode45=12 % -
  \catcode46=12 % .
  \catcode58=12 % :
  \catcode64=11 % @
  \catcode123=1 % {
  \catcode125=2 % }
  \expandafter\let\expandafter\x\csname ver@hologo.sty\endcsname
  \ifx\x\relax % plain-TeX, first loading
  \else
    \def\empty{}%
    \ifx\x\empty % LaTeX, first loading,
      % variable is initialized, but \ProvidesPackage not yet seen
    \else
      \expandafter\ifx\csname PackageInfo\endcsname\relax
        \def\x#1#2{%
          \immediate\write-1{Package #1 Info: #2.}%
        }%
      \else
        \def\x#1#2{\PackageInfo{#1}{#2, stopped}}%
      \fi
      \x{hologo}{The package is already loaded}%
      \aftergroup\endinput
    \fi
  \fi
\endgroup%
%    \end{macrocode}
%    Package identification:
%    \begin{macrocode}
\begingroup\catcode61\catcode48\catcode32=10\relax%
  \catcode13=5 % ^^M
  \endlinechar=13 %
  \catcode35=6 % #
  \catcode39=12 % '
  \catcode40=12 % (
  \catcode41=12 % )
  \catcode44=12 % ,
  \catcode45=12 % -
  \catcode46=12 % .
  \catcode47=12 % /
  \catcode58=12 % :
  \catcode64=11 % @
  \catcode91=12 % [
  \catcode93=12 % ]
  \catcode123=1 % {
  \catcode125=2 % }
  \expandafter\ifx\csname ProvidesPackage\endcsname\relax
    \def\x#1#2#3[#4]{\endgroup
      \immediate\write-1{Package: #3 #4}%
      \xdef#1{#4}%
    }%
  \else
    \def\x#1#2[#3]{\endgroup
      #2[{#3}]%
      \ifx#1\@undefined
        \xdef#1{#3}%
      \fi
      \ifx#1\relax
        \xdef#1{#3}%
      \fi
    }%
  \fi
\expandafter\x\csname ver@hologo.sty\endcsname
\ProvidesPackage{hologo}%
  [2016/05/12 v1.11 A logo collection with bookmark support (HO)]%
%    \end{macrocode}
%
%    \begin{macrocode}
\begingroup\catcode61\catcode48\catcode32=10\relax%
  \catcode13=5 % ^^M
  \endlinechar=13 %
  \catcode123=1 % {
  \catcode125=2 % }
  \catcode64=11 % @
  \def\x{\endgroup
    \expandafter\edef\csname HOLOGO@AtEnd\endcsname{%
      \endlinechar=\the\endlinechar\relax
      \catcode13=\the\catcode13\relax
      \catcode32=\the\catcode32\relax
      \catcode35=\the\catcode35\relax
      \catcode61=\the\catcode61\relax
      \catcode64=\the\catcode64\relax
      \catcode123=\the\catcode123\relax
      \catcode125=\the\catcode125\relax
    }%
  }%
\x\catcode61\catcode48\catcode32=10\relax%
\catcode13=5 % ^^M
\endlinechar=13 %
\catcode35=6 % #
\catcode64=11 % @
\catcode123=1 % {
\catcode125=2 % }
\def\TMP@EnsureCode#1#2{%
  \edef\HOLOGO@AtEnd{%
    \HOLOGO@AtEnd
    \catcode#1=\the\catcode#1\relax
  }%
  \catcode#1=#2\relax
}
\TMP@EnsureCode{10}{12}% ^^J
\TMP@EnsureCode{33}{12}% !
\TMP@EnsureCode{34}{12}% "
\TMP@EnsureCode{36}{3}% $
\TMP@EnsureCode{38}{4}% &
\TMP@EnsureCode{39}{12}% '
\TMP@EnsureCode{40}{12}% (
\TMP@EnsureCode{41}{12}% )
\TMP@EnsureCode{42}{12}% *
\TMP@EnsureCode{43}{12}% +
\TMP@EnsureCode{44}{12}% ,
\TMP@EnsureCode{45}{12}% -
\TMP@EnsureCode{46}{12}% .
\TMP@EnsureCode{47}{12}% /
\TMP@EnsureCode{58}{12}% :
\TMP@EnsureCode{59}{12}% ;
\TMP@EnsureCode{60}{12}% <
\TMP@EnsureCode{62}{12}% >
\TMP@EnsureCode{63}{12}% ?
\TMP@EnsureCode{91}{12}% [
\TMP@EnsureCode{93}{12}% ]
\TMP@EnsureCode{94}{7}% ^ (superscript)
\TMP@EnsureCode{95}{8}% _ (subscript)
\TMP@EnsureCode{96}{12}% `
\TMP@EnsureCode{124}{12}% |
\edef\HOLOGO@AtEnd{%
  \HOLOGO@AtEnd
  \escapechar\the\escapechar\relax
  \noexpand\endinput
}
\escapechar=92 %
%    \end{macrocode}
%
% \subsection{Logo list}
%
%    \begin{macro}{\hologoList}
%    \begin{macrocode}
\def\hologoList{%
  \hologoEntry{(La)TeX}{}{2011/10/01}%
  \hologoEntry{AmSLaTeX}{}{2010/04/16}%
  \hologoEntry{AmSTeX}{}{2010/04/16}%
  \hologoEntry{biber}{}{2011/10/01}%
  \hologoEntry{BibTeX}{}{2011/10/01}%
  \hologoEntry{BibTeX}{sf}{2011/10/01}%
  \hologoEntry{BibTeX}{sc}{2011/10/01}%
  \hologoEntry{BibTeX8}{}{2011/11/22}%
  \hologoEntry{ConTeXt}{}{2011/03/25}%
  \hologoEntry{ConTeXt}{narrow}{2011/03/25}%
  \hologoEntry{ConTeXt}{simple}{2011/03/25}%
  \hologoEntry{emTeX}{}{2010/04/26}%
  \hologoEntry{eTeX}{}{2010/04/08}%
  \hologoEntry{ExTeX}{}{2011/10/01}%
  \hologoEntry{HanTheThanh}{}{2011/11/29}%
  \hologoEntry{iniTeX}{}{2011/10/01}%
  \hologoEntry{KOMAScript}{}{2011/10/01}%
  \hologoEntry{La}{}{2010/05/08}%
  \hologoEntry{LaTeX}{}{2010/04/08}%
  \hologoEntry{LaTeX2e}{}{2010/04/08}%
  \hologoEntry{LaTeX3}{}{2010/04/24}%
  \hologoEntry{LaTeXe}{}{2010/04/08}%
  \hologoEntry{LaTeXML}{}{2011/11/22}%
  \hologoEntry{LaTeXTeX}{}{2011/10/01}%
  \hologoEntry{LuaLaTeX}{}{2010/04/08}%
  \hologoEntry{LuaTeX}{}{2010/04/08}%
  \hologoEntry{LyX}{}{2011/10/01}%
  \hologoEntry{METAFONT}{}{2011/10/01}%
  \hologoEntry{MetaFun}{}{2011/10/01}%
  \hologoEntry{METAPOST}{}{2011/10/01}%
  \hologoEntry{MetaPost}{}{2011/10/01}%
  \hologoEntry{MiKTeX}{}{2011/10/01}%
  \hologoEntry{NTS}{}{2011/10/01}%
  \hologoEntry{OzMF}{}{2011/10/01}%
  \hologoEntry{OzMP}{}{2011/10/01}%
  \hologoEntry{OzTeX}{}{2011/10/01}%
  \hologoEntry{OzTtH}{}{2011/10/01}%
  \hologoEntry{PCTeX}{}{2011/10/01}%
  \hologoEntry{pdfTeX}{}{2011/10/01}%
  \hologoEntry{pdfLaTeX}{}{2011/10/01}%
  \hologoEntry{PiC}{}{2011/10/01}%
  \hologoEntry{PiCTeX}{}{2011/10/01}%
  \hologoEntry{plainTeX}{}{2010/04/08}%
  \hologoEntry{plainTeX}{space}{2010/04/16}%
  \hologoEntry{plainTeX}{hyphen}{2010/04/16}%
  \hologoEntry{plainTeX}{runtogether}{2010/04/16}%
  \hologoEntry{SageTeX}{}{2011/11/22}%
  \hologoEntry{SLiTeX}{}{2011/10/01}%
  \hologoEntry{SLiTeX}{lift}{2011/10/01}%
  \hologoEntry{SLiTeX}{narrow}{2011/10/01}%
  \hologoEntry{SLiTeX}{simple}{2011/10/01}%
  \hologoEntry{SliTeX}{}{2011/10/01}%
  \hologoEntry{SliTeX}{narrow}{2011/10/01}%
  \hologoEntry{SliTeX}{simple}{2011/10/01}%
  \hologoEntry{SliTeX}{lift}{2011/10/01}%
  \hologoEntry{teTeX}{}{2011/10/01}%
  \hologoEntry{TeX}{}{2010/04/08}%
  \hologoEntry{TeX4ht}{}{2011/11/22}%
  \hologoEntry{TTH}{}{2011/11/22}%
  \hologoEntry{virTeX}{}{2011/10/01}%
  \hologoEntry{VTeX}{}{2010/04/24}%
  \hologoEntry{Xe}{}{2010/04/08}%
  \hologoEntry{XeLaTeX}{}{2010/04/08}%
  \hologoEntry{XeTeX}{}{2010/04/08}%
}
%    \end{macrocode}
%    \end{macro}
%
% \subsection{Load resources}
%
%    \begin{macrocode}
\begingroup\expandafter\expandafter\expandafter\endgroup
\expandafter\ifx\csname RequirePackage\endcsname\relax
  \def\TMP@RequirePackage#1[#2]{%
    \begingroup\expandafter\expandafter\expandafter\endgroup
    \expandafter\ifx\csname ver@#1.sty\endcsname\relax
      \input #1.sty\relax
    \fi
  }%
  \TMP@RequirePackage{ltxcmds}[2011/02/04]%
  \TMP@RequirePackage{infwarerr}[2010/04/08]%
  \TMP@RequirePackage{kvsetkeys}[2010/03/01]%
  \TMP@RequirePackage{kvdefinekeys}[2010/03/01]%
  \TMP@RequirePackage{pdftexcmds}[2010/04/01]%
  \TMP@RequirePackage{ifpdf}[2010/01/28]%
  \TMP@RequirePackage{ifluatex}[2010/03/01]%
  \ltx@IfUndefined{newif}{%
    \expandafter\let\csname newif\endcsname\ltx@newif
  }{}%
  \TMP@RequirePackage{ifxetex}[2009/01/23]%
  \TMP@RequirePackage{ifvtex}[2010/03/01]%
\else
  \RequirePackage{ltxcmds}[2011/02/04]%
  \RequirePackage{infwarerr}[2010/04/08]%
  \RequirePackage{kvsetkeys}[2010/03/01]%
  \RequirePackage{kvdefinekeys}[2010/03/01]%
  \RequirePackage{pdftexcmds}[2010/04/01]%
  \RequirePackage{ifpdf}[2010/01/28]%
  \RequirePackage{ifluatex}[2010/03/01]%
  \RequirePackage{ifxetex}[2009/01/23]%
  \RequirePackage{ifvtex}[2010/03/01]%
\fi
%    \end{macrocode}
%
%    \begin{macro}{\HOLOGO@IfDefined}
%    \begin{macrocode}
\def\HOLOGO@IfExists#1{%
  \ifx\@undefined#1%
    \expandafter\ltx@secondoftwo
  \else
    \ifx\relax#1%
      \expandafter\ltx@secondoftwo
    \else
      \expandafter\expandafter\expandafter\ltx@firstoftwo
    \fi
  \fi
}
%    \end{macrocode}
%    \end{macro}
%
% \subsection{Setup macros}
%
%    \begin{macro}{\hologoSetup}
%    \begin{macrocode}
\def\hologoSetup{%
  \let\HOLOGO@name\relax
  \HOLOGO@Setup
}
%    \end{macrocode}
%    \end{macro}
%
%    \begin{macro}{\hologoLogoSetup}
%    \begin{macrocode}
\def\hologoLogoSetup#1{%
  \edef\HOLOGO@name{#1}%
  \ltx@IfUndefined{HoLogo@\HOLOGO@name}{%
    \@PackageError{hologo}{%
      Unknown logo `\HOLOGO@name'%
    }\@ehc
    \ltx@gobble
  }{%
    \HOLOGO@Setup
  }%
}
%    \end{macrocode}
%    \end{macro}
%
%    \begin{macro}{\HOLOGO@Setup}
%    \begin{macrocode}
\def\HOLOGO@Setup{%
  \kvsetkeys{HoLogo}%
}
%    \end{macrocode}
%    \end{macro}
%
% \subsection{Options}
%
%    \begin{macro}{\HOLOGO@DeclareBoolOption}
%    \begin{macrocode}
\def\HOLOGO@DeclareBoolOption#1{%
  \expandafter\chardef\csname HOLOGOOPT@#1\endcsname\ltx@zero
  \kv@define@key{HoLogo}{#1}[true]{%
    \def\HOLOGO@temp{##1}%
    \ifx\HOLOGO@temp\HOLOGO@true
      \ifx\HOLOGO@name\relax
        \expandafter\chardef\csname HOLOGOOPT@#1\endcsname=\ltx@one
      \else
        \expandafter\chardef\csname
        HoLogoOpt@#1@\HOLOGO@name\endcsname\ltx@one
      \fi
      \HOLOGO@SetBreakAll{#1}%
    \else
      \ifx\HOLOGO@temp\HOLOGO@false
        \ifx\HOLOGO@name\relax
          \expandafter\chardef\csname HOLOGOOPT@#1\endcsname=\ltx@zero
        \else
          \expandafter\chardef\csname
          HoLogoOpt@#1@\HOLOGO@name\endcsname=\ltx@zero
        \fi
        \HOLOGO@SetBreakAll{#1}%
      \else
        \@PackageError{hologo}{%
          Unknown value `##1' for boolean option `#1'.\MessageBreak
          Known values are `true' and `false'%
        }\@ehc
      \fi
    \fi
  }%
}
%    \end{macrocode}
%    \end{macro}
%
%    \begin{macro}{\HOLOGO@SetBreakAll}
%    \begin{macrocode}
\def\HOLOGO@SetBreakAll#1{%
  \def\HOLOGO@temp{#1}%
  \ifx\HOLOGO@temp\HOLOGO@break
    \ifx\HOLOGO@name\relax
      \chardef\HOLOGOOPT@hyphenbreak=\HOLOGOOPT@break
      \chardef\HOLOGOOPT@spacebreak=\HOLOGOOPT@break
      \chardef\HOLOGOOPT@discretionarybreak=\HOLOGOOPT@break
    \else
      \expandafter\chardef
         \csname HoLogoOpt@hyphenbreak@\HOLOGO@name\endcsname=%
         \csname HoLogoOpt@break@\HOLOGO@name\endcsname
      \expandafter\chardef
         \csname HoLogoOpt@spacebreak@\HOLOGO@name\endcsname=%
         \csname HoLogoOpt@break@\HOLOGO@name\endcsname
      \expandafter\chardef
         \csname HoLogoOpt@discretionarybreak@\HOLOGO@name
             \endcsname=%
         \csname HoLogoOpt@break@\HOLOGO@name\endcsname
    \fi
  \fi
}
%    \end{macrocode}
%    \end{macro}
%
%    \begin{macro}{\HOLOGO@true}
%    \begin{macrocode}
\def\HOLOGO@true{true}
%    \end{macrocode}
%    \end{macro}
%    \begin{macro}{\HOLOGO@false}
%    \begin{macrocode}
\def\HOLOGO@false{false}
%    \end{macrocode}
%    \end{macro}
%    \begin{macro}{\HOLOGO@break}
%    \begin{macrocode}
\def\HOLOGO@break{break}
%    \end{macrocode}
%    \end{macro}
%
%    \begin{macrocode}
\HOLOGO@DeclareBoolOption{break}
\HOLOGO@DeclareBoolOption{hyphenbreak}
\HOLOGO@DeclareBoolOption{spacebreak}
\HOLOGO@DeclareBoolOption{discretionarybreak}
%    \end{macrocode}
%
%    \begin{macrocode}
\kv@define@key{HoLogo}{variant}{%
  \ifx\HOLOGO@name\relax
    \@PackageError{hologo}{%
      Option `variant' is not available in \string\hologoSetup,%
      \MessageBreak
      Use \string\hologoLogoSetup\space instead%
    }\@ehc
  \else
    \edef\HOLOGO@temp{#1}%
    \ifx\HOLOGO@temp\ltx@empty
      \expandafter
      \let\csname HoLogoOpt@variant@\HOLOGO@name\endcsname\@undefined
    \else
      \ltx@IfUndefined{HoLogo@\HOLOGO@name @\HOLOGO@temp}{%
        \@PackageError{hologo}{%
          Unknown variant `\HOLOGO@temp' of logo `\HOLOGO@name'%
        }\@ehc
      }{%
        \expandafter
        \let\csname HoLogoOpt@variant@\HOLOGO@name\endcsname
            \HOLOGO@temp
      }%
    \fi
  \fi
}
%    \end{macrocode}
%
%    \begin{macro}{\HOLOGO@Variant}
%    \begin{macrocode}
\def\HOLOGO@Variant#1{%
  #1%
  \ltx@ifundefined{HoLogoOpt@variant@#1}{%
  }{%
    @\csname HoLogoOpt@variant@#1\endcsname
  }%
}
%    \end{macrocode}
%    \end{macro}
%
% \subsection{Break/no-break support}
%
%    \begin{macro}{\HOLOGO@space}
%    \begin{macrocode}
\def\HOLOGO@space{%
  \ltx@ifundefined{HoLogoOpt@spacebreak@\HOLOGO@name}{%
    \ltx@ifundefined{HoLogoOpt@break@\HOLOGO@name}{%
      \chardef\HOLOGO@temp=\HOLOGOOPT@spacebreak
    }{%
      \chardef\HOLOGO@temp=%
        \csname HoLogoOpt@break@\HOLOGO@name\endcsname
    }%
  }{%
    \chardef\HOLOGO@temp=%
      \csname HoLogoOpt@spacebreak@\HOLOGO@name\endcsname
  }%
  \ifcase\HOLOGO@temp
    \penalty10000 %
  \fi
  \ltx@space
}
%    \end{macrocode}
%    \end{macro}
%
%    \begin{macro}{\HOLOGO@hyphen}
%    \begin{macrocode}
\def\HOLOGO@hyphen{%
  \ltx@ifundefined{HoLogoOpt@hyphenbreak@\HOLOGO@name}{%
    \ltx@ifundefined{HoLogoOpt@break@\HOLOGO@name}{%
      \chardef\HOLOGO@temp=\HOLOGOOPT@hyphenbreak
    }{%
      \chardef\HOLOGO@temp=%
        \csname HoLogoOpt@break@\HOLOGO@name\endcsname
    }%
  }{%
    \chardef\HOLOGO@temp=%
      \csname HoLogoOpt@hyphenbreak@\HOLOGO@name\endcsname
  }%
  \ifcase\HOLOGO@temp
    \ltx@mbox{-}%
  \else
    -%
  \fi
}
%    \end{macrocode}
%    \end{macro}
%
%    \begin{macro}{\HOLOGO@discretionary}
%    \begin{macrocode}
\def\HOLOGO@discretionary{%
  \ltx@ifundefined{HoLogoOpt@discretionarybreak@\HOLOGO@name}{%
    \ltx@ifundefined{HoLogoOpt@break@\HOLOGO@name}{%
      \chardef\HOLOGO@temp=\HOLOGOOPT@discretionarybreak
    }{%
      \chardef\HOLOGO@temp=%
        \csname HoLogoOpt@break@\HOLOGO@name\endcsname
    }%
  }{%
    \chardef\HOLOGO@temp=%
      \csname HoLogoOpt@discretionarybreak@\HOLOGO@name\endcsname
  }%
  \ifcase\HOLOGO@temp
  \else
    \-%
  \fi
}
%    \end{macrocode}
%    \end{macro}
%
%    \begin{macro}{\HOLOGO@mbox}
%    \begin{macrocode}
\def\HOLOGO@mbox#1{%
  \ltx@ifundefined{HoLogoOpt@break@\HOLOGO@name}{%
    \chardef\HOLOGO@temp=\HOLOGOOPT@hyphenbreak
  }{%
    \chardef\HOLOGO@temp=%
      \csname HoLogoOpt@break@\HOLOGO@name\endcsname
  }%
  \ifcase\HOLOGO@temp
    \ltx@mbox{#1}%
  \else
    #1%
  \fi
}
%    \end{macrocode}
%    \end{macro}
%
% \subsection{Font support}
%
%    \begin{macro}{\HoLogoFont@font}
%    \begin{tabular}{@{}ll@{}}
%    |#1|:& logo name\\
%    |#2|:& font short name\\
%    |#3|:& text
%    \end{tabular}
%    \begin{macrocode}
\def\HoLogoFont@font#1#2#3{%
  \begingroup
    \ltx@IfUndefined{HoLogoFont@logo@#1.#2}{%
      \ltx@IfUndefined{HoLogoFont@font@#2}{%
        \@PackageWarning{hologo}{%
          Missing font `#2' for logo `#1'%
        }%
        #3%
      }{%
        \csname HoLogoFont@font@#2\endcsname{#3}%
      }%
    }{%
      \csname HoLogoFont@logo@#1.#2\endcsname{#3}%
    }%
  \endgroup
}
%    \end{macrocode}
%    \end{macro}
%
%    \begin{macro}{\HoLogoFont@Def}
%    \begin{macrocode}
\def\HoLogoFont@Def#1{%
  \expandafter\def\csname HoLogoFont@font@#1\endcsname
}
%    \end{macrocode}
%    \end{macro}
%    \begin{macro}{\HoLogoFont@LogoDef}
%    \begin{macrocode}
\def\HoLogoFont@LogoDef#1#2{%
  \expandafter\def\csname HoLogoFont@logo@#1.#2\endcsname
}
%    \end{macrocode}
%    \end{macro}
%
% \subsubsection{Font defaults}
%
%    \begin{macro}{\HoLogoFont@font@general}
%    \begin{macrocode}
\HoLogoFont@Def{general}{}%
%    \end{macrocode}
%    \end{macro}
%
%    \begin{macro}{\HoLogoFont@font@rm}
%    \begin{macrocode}
\ltx@IfUndefined{rmfamily}{%
  \ltx@IfUndefined{rm}{%
  }{%
    \HoLogoFont@Def{rm}{\rm}%
  }%
}{%
  \HoLogoFont@Def{rm}{\rmfamily}%
}
%    \end{macrocode}
%    \end{macro}
%
%    \begin{macro}{\HoLogoFont@font@sf}
%    \begin{macrocode}
\ltx@IfUndefined{sffamily}{%
  \ltx@IfUndefined{sf}{%
  }{%
    \HoLogoFont@Def{sf}{\sf}%
  }%
}{%
  \HoLogoFont@Def{sf}{\sffamily}%
}
%    \end{macrocode}
%    \end{macro}
%
%    \begin{macro}{\HoLogoFont@font@bibsf}
%    In case of \hologo{plainTeX} the original small caps
%    variant is used as default. In \hologo{LaTeX}
%    the definition of package \xpackage{dtklogos} \cite{dtklogos}
%    is used.
%\begin{quote}
%\begin{verbatim}
%\DeclareRobustCommand{\BibTeX}{%
%  B%
%  \kern-.05em%
%  \hbox{%
%    $\m@th$% %% force math size calculations
%    \csname S@\f@size\endcsname
%    \fontsize\sf@size\z@
%    \math@fontsfalse
%    \selectfont
%    I%
%    \kern-.025em%
%    B
%  }%
%  \kern-.08em%
%  \-%
%  \TeX
%}
%\end{verbatim}
%\end{quote}
%    \begin{macrocode}
\ltx@IfUndefined{selectfont}{%
  \ltx@IfUndefined{tensc}{%
    \font\tensc=cmcsc10\relax
  }{}%
  \HoLogoFont@Def{bibsf}{\tensc}%
}{%
  \HoLogoFont@Def{bibsf}{%
    $\mathsurround=0pt$%
    \csname S@\f@size\endcsname
    \fontsize\sf@size{0pt}%
    \math@fontsfalse
    \selectfont
  }%
}
%    \end{macrocode}
%    \end{macro}
%
%    \begin{macro}{\HoLogoFont@font@sc}
%    \begin{macrocode}
\ltx@IfUndefined{scshape}{%
  \ltx@IfUndefined{tensc}{%
    \font\tensc=cmcsc10\relax
  }{}%
  \HoLogoFont@Def{sc}{\tensc}%
}{%
  \HoLogoFont@Def{sc}{\scshape}%
}
%    \end{macrocode}
%    \end{macro}
%
%    \begin{macro}{\HoLogoFont@font@sy}
%    \begin{macrocode}
\ltx@IfUndefined{usefont}{%
  \ltx@IfUndefined{tensy}{%
  }{%
    \HoLogoFont@Def{sy}{\tensy}%
  }%
}{%
  \HoLogoFont@Def{sy}{%
    \usefont{OMS}{cmsy}{m}{n}%
  }%
}
%    \end{macrocode}
%    \end{macro}
%
%    \begin{macro}{\HoLogoFont@font@logo}
%    \begin{macrocode}
\begingroup
  \def\x{LaTeX2e}%
\expandafter\endgroup
\ifx\fmtname\x
  \ltx@IfUndefined{logofamily}{%
    \DeclareRobustCommand\logofamily{%
      \not@math@alphabet\logofamily\relax
      \fontencoding{U}%
      \fontfamily{logo}%
      \selectfont
    }%
  }{}%
  \ltx@IfUndefined{logofamily}{%
  }{%
    \HoLogoFont@Def{logo}{\logofamily}%
  }%
\else
  \ltx@IfUndefined{tenlogo}{%
    \font\tenlogo=logo10\relax
  }{}%
  \HoLogoFont@Def{logo}{\tenlogo}%
\fi
%    \end{macrocode}
%    \end{macro}
%
% \subsubsection{Font setup}
%
%    \begin{macro}{\hologoFontSetup}
%    \begin{macrocode}
\def\hologoFontSetup{%
  \let\HOLOGO@name\relax
  \HOLOGO@FontSetup
}
%    \end{macrocode}
%    \end{macro}
%
%    \begin{macro}{\hologoLogoFontSetup}
%    \begin{macrocode}
\def\hologoLogoFontSetup#1{%
  \edef\HOLOGO@name{#1}%
  \ltx@IfUndefined{HoLogo@\HOLOGO@name}{%
    \@PackageError{hologo}{%
      Unknown logo `\HOLOGO@name'%
    }\@ehc
    \ltx@gobble
  }{%
    \HOLOGO@FontSetup
  }%
}
%    \end{macrocode}
%    \end{macro}
%
%    \begin{macro}{\HOLOGO@FontSetup}
%    \begin{macrocode}
\def\HOLOGO@FontSetup{%
  \kvsetkeys{HoLogoFont}%
}
%    \end{macrocode}
%    \end{macro}
%
%    \begin{macrocode}
\def\HOLOGO@temp#1{%
  \kv@define@key{HoLogoFont}{#1}{%
    \ifx\HOLOGO@name\relax
      \HoLogoFont@Def{#1}{##1}%
    \else
      \HoLogoFont@LogoDef\HOLOGO@name{#1}{##1}%
    \fi
  }%
}
\HOLOGO@temp{general}
\HOLOGO@temp{sf}
%    \end{macrocode}
%
% \subsection{Generic logo commands}
%
%    \begin{macrocode}
\HOLOGO@IfExists\hologo{%
  \@PackageError{hologo}{%
    \string\hologo\ltx@space is already defined.\MessageBreak
    Package loading is aborted%
  }\@ehc
  \HOLOGO@AtEnd
}%
\HOLOGO@IfExists\hologoRobust{%
  \@PackageError{hologo}{%
    \string\hologoRobust\ltx@space is already defined.\MessageBreak
    Package loading is aborted%
  }\@ehc
  \HOLOGO@AtEnd
}%
%    \end{macrocode}
%
% \subsubsection{\cs{hologo} and friends}
%
%    \begin{macrocode}
\ifluatex
  \expandafter\ltx@firstofone
\else
  \expandafter\ltx@gobble
\fi
{%
  \ltx@IfUndefined{ifincsname}{%
    \ifnum\luatexversion<36 %
      \expandafter\ltx@gobble
    \else
      \expandafter\ltx@firstofone
    \fi
    {%
      \begingroup
        \ifcase0%
            \directlua{%
              if tex.enableprimitives then %
                tex.enableprimitives('HOLOGO@', {'ifincsname'})%
              else %
                tex.print('1')%
              end%
            }%
            \ifx\HOLOGO@ifincsname\@undefined 1\fi%
            \relax
          \expandafter\ltx@firstofone
        \else
          \endgroup
          \expandafter\ltx@gobble
        \fi
        {%
          \global\let\ifincsname\HOLOGO@ifincsname
        }%
      \HOLOGO@temp
    }%
  }{}%
}
%    \end{macrocode}
%    \begin{macrocode}
\ltx@IfUndefined{ifincsname}{%
  \catcode`$=14 %
}{%
  \catcode`$=9 %
}
%    \end{macrocode}
%
%    \begin{macro}{\hologo}
%    \begin{macrocode}
\def\hologo#1{%
$ \ifincsname
$   \ltx@ifundefined{HoLogoCs@\HOLOGO@Variant{#1}}{%
$     #1%
$   }{%
$     \csname HoLogoCs@\HOLOGO@Variant{#1}\endcsname\ltx@firstoftwo
$   }%
$ \else
    \HOLOGO@IfExists\texorpdfstring\texorpdfstring\ltx@firstoftwo
    {%
      \hologoRobust{#1}%
    }{%
      \ltx@ifundefined{HoLogoBkm@\HOLOGO@Variant{#1}}{%
        \ltx@ifundefined{HoLogo@#1}{?#1?}{#1}%
      }{%
        \csname HoLogoBkm@\HOLOGO@Variant{#1}\endcsname
        \ltx@firstoftwo
      }%
    }%
$ \fi
}
%    \end{macrocode}
%    \end{macro}
%    \begin{macro}{\Hologo}
%    \begin{macrocode}
\def\Hologo#1{%
$ \ifincsname
$   \ltx@ifundefined{HoLogoCs@\HOLOGO@Variant{#1}}{%
$     #1%
$   }{%
$     \csname HoLogoCs@\HOLOGO@Variant{#1}\endcsname\ltx@secondoftwo
$   }%
$ \else
    \HOLOGO@IfExists\texorpdfstring\texorpdfstring\ltx@firstoftwo
    {%
      \HologoRobust{#1}%
    }{%
      \ltx@ifundefined{HoLogoBkm@\HOLOGO@Variant{#1}}{%
        \ltx@ifundefined{HoLogo@#1}{?#1?}{#1}%
      }{%
        \csname HoLogoBkm@\HOLOGO@Variant{#1}\endcsname
        \ltx@secondoftwo
      }%
    }%
$ \fi
}
%    \end{macrocode}
%    \end{macro}
%
%    \begin{macro}{\hologoVariant}
%    \begin{macrocode}
\def\hologoVariant#1#2{%
  \ifx\relax#2\relax
    \hologo{#1}%
  \else
$   \ifincsname
$     \ltx@ifundefined{HoLogoCs@#1@#2}{%
$       #1%
$     }{%
$       \csname HoLogoCs@#1@#2\endcsname\ltx@firstoftwo
$     }%
$   \else
      \HOLOGO@IfExists\texorpdfstring\texorpdfstring\ltx@firstoftwo
      {%
        \hologoVariantRobust{#1}{#2}%
      }{%
        \ltx@ifundefined{HoLogoBkm@#1@#2}{%
          \ltx@ifundefined{HoLogo@#1}{?#1?}{#1}%
        }{%
          \csname HoLogoBkm@#1@#2\endcsname
          \ltx@firstoftwo
        }%
      }%
$   \fi
  \fi
}
%    \end{macrocode}
%    \end{macro}
%    \begin{macro}{\HologoVariant}
%    \begin{macrocode}
\def\HologoVariant#1#2{%
  \ifx\relax#2\relax
    \Hologo{#1}%
  \else
$   \ifincsname
$     \ltx@ifundefined{HoLogoCs@#1@#2}{%
$       #1%
$     }{%
$       \csname HoLogoCs@#1@#2\endcsname\ltx@secondoftwo
$     }%
$   \else
      \HOLOGO@IfExists\texorpdfstring\texorpdfstring\ltx@firstoftwo
      {%
        \HologoVariantRobust{#1}{#2}%
      }{%
        \ltx@ifundefined{HoLogoBkm@#1@#2}{%
          \ltx@ifundefined{HoLogo@#1}{?#1?}{#1}%
        }{%
          \csname HoLogoBkm@#1@#2\endcsname
          \ltx@secondoftwo
        }%
      }%
$   \fi
  \fi
}
%    \end{macrocode}
%    \end{macro}
%
%    \begin{macrocode}
\catcode`\$=3 %
%    \end{macrocode}
%
% \subsubsection{\cs{hologoRobust} and friends}
%
%    \begin{macro}{\hologoRobust}
%    \begin{macrocode}
\ltx@IfUndefined{protected}{%
  \ltx@IfUndefined{DeclareRobustCommand}{%
    \def\hologoRobust#1%
  }{%
    \DeclareRobustCommand*\hologoRobust[1]%
  }%
}{%
  \protected\def\hologoRobust#1%
}%
{%
  \edef\HOLOGO@name{#1}%
  \ltx@IfUndefined{HoLogo@\HOLOGO@Variant\HOLOGO@name}{%
    \@PackageError{hologo}{%
      Unknown logo `\HOLOGO@name'%
    }\@ehc
    ?\HOLOGO@name?%
  }{%
    \ltx@IfUndefined{ver@tex4ht.sty}{%
      \HoLogoFont@font\HOLOGO@name{general}{%
        \csname HoLogo@\HOLOGO@Variant\HOLOGO@name\endcsname
        \ltx@firstoftwo
      }%
    }{%
      \ltx@IfUndefined{HoLogoHtml@\HOLOGO@Variant\HOLOGO@name}{%
        \HOLOGO@name
      }{%
        \csname HoLogoHtml@\HOLOGO@Variant\HOLOGO@name\endcsname
        \ltx@firstoftwo
      }%
    }%
  }%
}
%    \end{macrocode}
%    \end{macro}
%    \begin{macro}{\HologoRobust}
%    \begin{macrocode}
\ltx@IfUndefined{protected}{%
  \ltx@IfUndefined{DeclareRobustCommand}{%
    \def\HologoRobust#1%
  }{%
    \DeclareRobustCommand*\HologoRobust[1]%
  }%
}{%
  \protected\def\HologoRobust#1%
}%
{%
  \edef\HOLOGO@name{#1}%
  \ltx@IfUndefined{HoLogo@\HOLOGO@Variant\HOLOGO@name}{%
    \@PackageError{hologo}{%
      Unknown logo `\HOLOGO@name'%
    }\@ehc
    ?\HOLOGO@name?%
  }{%
    \ltx@IfUndefined{ver@tex4ht.sty}{%
      \HoLogoFont@font\HOLOGO@name{general}{%
        \csname HoLogo@\HOLOGO@Variant\HOLOGO@name\endcsname
        \ltx@secondoftwo
      }%
    }{%
      \ltx@IfUndefined{HoLogoHtml@\HOLOGO@Variant\HOLOGO@name}{%
        \expandafter\HOLOGO@Uppercase\HOLOGO@name
      }{%
        \csname HoLogoHtml@\HOLOGO@Variant\HOLOGO@name\endcsname
        \ltx@secondoftwo
      }%
    }%
  }%
}
%    \end{macrocode}
%    \end{macro}
%    \begin{macro}{\hologoVariantRobust}
%    \begin{macrocode}
\ltx@IfUndefined{protected}{%
  \ltx@IfUndefined{DeclareRobustCommand}{%
    \def\hologoVariantRobust#1#2%
  }{%
    \DeclareRobustCommand*\hologoVariantRobust[2]%
  }%
}{%
  \protected\def\hologoVariantRobust#1#2%
}%
{%
  \begingroup
    \hologoLogoSetup{#1}{variant={#2}}%
    \hologoRobust{#1}%
  \endgroup
}
%    \end{macrocode}
%    \end{macro}
%    \begin{macro}{\HologoVariantRobust}
%    \begin{macrocode}
\ltx@IfUndefined{protected}{%
  \ltx@IfUndefined{DeclareRobustCommand}{%
    \def\HologoVariantRobust#1#2%
  }{%
    \DeclareRobustCommand*\HologoVariantRobust[2]%
  }%
}{%
  \protected\def\HologoVariantRobust#1#2%
}%
{%
  \begingroup
    \hologoLogoSetup{#1}{variant={#2}}%
    \HologoRobust{#1}%
  \endgroup
}
%    \end{macrocode}
%    \end{macro}
%
%    \begin{macro}{\hologorobust}
%    Macro \cs{hologorobust} is only defined for compatibility.
%    Its use is deprecated.
%    \begin{macrocode}
\def\hologorobust{\hologoRobust}
%    \end{macrocode}
%    \end{macro}
%
% \subsection{Helpers}
%
%    \begin{macro}{\HOLOGO@Uppercase}
%    Macro \cs{HOLOGO@Uppercase} is restricted to \cs{uppercase},
%    because \hologo{plainTeX} or \hologo{iniTeX} do not provide
%    \cs{MakeUppercase}.
%    \begin{macrocode}
\def\HOLOGO@Uppercase#1{\uppercase{#1}}
%    \end{macrocode}
%    \end{macro}
%
%    \begin{macro}{\HOLOGO@PdfdocUnicode}
%    \begin{macrocode}
\def\HOLOGO@PdfdocUnicode{%
  \ifx\ifHy@unicode\iftrue
    \expandafter\ltx@secondoftwo
  \else
    \expandafter\ltx@firstoftwo
  \fi
}
%    \end{macrocode}
%    \end{macro}
%
%    \begin{macro}{\HOLOGO@Math}
%    \begin{macrocode}
\def\HOLOGO@MathSetup{%
  \mathsurround0pt\relax
  \HOLOGO@IfExists\f@series{%
    \if b\expandafter\ltx@car\f@series x\@nil
      \csname boldmath\endcsname
   \fi
  }{}%
}
%    \end{macrocode}
%    \end{macro}
%
%    \begin{macro}{\HOLOGO@TempDimen}
%    \begin{macrocode}
\dimendef\HOLOGO@TempDimen=\ltx@zero
%    \end{macrocode}
%    \end{macro}
%    \begin{macro}{\HOLOGO@NegativeKerning}
%    \begin{macrocode}
\def\HOLOGO@NegativeKerning#1{%
  \begingroup
    \HOLOGO@TempDimen=0pt\relax
    \comma@parse@normalized{#1}{%
      \ifdim\HOLOGO@TempDimen=0pt %
        \expandafter\HOLOGO@@NegativeKerning\comma@entry
      \fi
      \ltx@gobble
    }%
    \ifdim\HOLOGO@TempDimen<0pt %
      \kern\HOLOGO@TempDimen
    \fi
  \endgroup
}
%    \end{macrocode}
%    \end{macro}
%    \begin{macro}{\HOLOGO@@NegativeKerning}
%    \begin{macrocode}
\def\HOLOGO@@NegativeKerning#1#2{%
  \setbox\ltx@zero\hbox{#1#2}%
  \HOLOGO@TempDimen=\wd\ltx@zero
  \setbox\ltx@zero\hbox{#1\kern0pt#2}%
  \advance\HOLOGO@TempDimen by -\wd\ltx@zero
}
%    \end{macrocode}
%    \end{macro}
%
%    \begin{macro}{\HOLOGO@SpaceFactor}
%    \begin{macrocode}
\def\HOLOGO@SpaceFactor{%
  \spacefactor1000 %
}
%    \end{macrocode}
%    \end{macro}
%
%    \begin{macro}{\HOLOGO@Span}
%    \begin{macrocode}
\def\HOLOGO@Span#1#2{%
  \HCode{<span class="HoLogo-#1">}%
  #2%
  \HCode{</span>}%
}
%    \end{macrocode}
%    \end{macro}
%
% \subsubsection{Text subscript}
%
%    \begin{macro}{\HOLOGO@SubScript}%
%    \begin{macrocode}
\def\HOLOGO@SubScript#1{%
  \ltx@IfUndefined{textsubscript}{%
    \ltx@IfUndefined{text}{%
      \ltx@mbox{%
        \mathsurround=0pt\relax
        $%
          _{%
            \ltx@IfUndefined{sf@size}{%
              \mathrm{#1}%
            }{%
              \mbox{%
                \fontsize\sf@size{0pt}\selectfont
                #1%
              }%
            }%
          }%
        $%
      }%
    }{%
      \ltx@mbox{%
        \mathsurround=0pt\relax
        $_{\text{#1}}$%
      }%
    }%
  }{%
    \textsubscript{#1}%
  }%
}
%    \end{macrocode}
%    \end{macro}
%
% \subsection{\hologo{TeX} and friends}
%
% \subsubsection{\hologo{TeX}}
%
%    \begin{macro}{\HoLogo@TeX}
%    Source: \hologo{LaTeX} kernel.
%    \begin{macrocode}
\def\HoLogo@TeX#1{%
  T\kern-.1667em\lower.5ex\hbox{E}\kern-.125emX\HOLOGO@SpaceFactor
}
%    \end{macrocode}
%    \end{macro}
%    \begin{macro}{\HoLogoHtml@TeX}
%    \begin{macrocode}
\def\HoLogoHtml@TeX#1{%
  \HoLogoCss@TeX
  \HOLOGO@Span{TeX}{%
    T%
    \HOLOGO@Span{e}{%
      E%
    }%
    X%
  }%
}
%    \end{macrocode}
%    \end{macro}
%    \begin{macro}{\HoLogoCss@TeX}
%    \begin{macrocode}
\def\HoLogoCss@TeX{%
  \Css{%
    span.HoLogo-TeX span.HoLogo-e{%
      position:relative;%
      top:.5ex;%
      margin-left:-.1667em;%
      margin-right:-.125em;%
    }%
  }%
  \Css{%
    a span.HoLogo-TeX span.HoLogo-e{%
      text-decoration:none;%
    }%
  }%
  \global\let\HoLogoCss@TeX\relax
}
%    \end{macrocode}
%    \end{macro}
%
% \subsubsection{\hologo{plainTeX}}
%
%    \begin{macro}{\HoLogo@plainTeX@space}
%    Source: ``The \hologo{TeX}book''
%    \begin{macrocode}
\def\HoLogo@plainTeX@space#1{%
  \HOLOGO@mbox{#1{p}{P}lain}\HOLOGO@space\hologo{TeX}%
}
%    \end{macrocode}
%    \end{macro}
%    \begin{macro}{\HoLogoCs@plainTeX@space}
%    \begin{macrocode}
\def\HoLogoCs@plainTeX@space#1{#1{p}{P}lain TeX}%
%    \end{macrocode}
%    \end{macro}
%    \begin{macro}{\HoLogoBkm@plainTeX@space}
%    \begin{macrocode}
\def\HoLogoBkm@plainTeX@space#1{%
  #1{p}{P}lain \hologo{TeX}%
}
%    \end{macrocode}
%    \end{macro}
%    \begin{macro}{\HoLogoHtml@plainTeX@space}
%    \begin{macrocode}
\def\HoLogoHtml@plainTeX@space#1{%
  #1{p}{P}lain \hologo{TeX}%
}
%    \end{macrocode}
%    \end{macro}
%
%    \begin{macro}{\HoLogo@plainTeX@hyphen}
%    \begin{macrocode}
\def\HoLogo@plainTeX@hyphen#1{%
  \HOLOGO@mbox{#1{p}{P}lain}\HOLOGO@hyphen\hologo{TeX}%
}
%    \end{macrocode}
%    \end{macro}
%    \begin{macro}{\HoLogoCs@plainTeX@hyphen}
%    \begin{macrocode}
\def\HoLogoCs@plainTeX@hyphen#1{#1{p}{P}lain-TeX}
%    \end{macrocode}
%    \end{macro}
%    \begin{macro}{\HoLogoBkm@plainTeX@hyphen}
%    \begin{macrocode}
\def\HoLogoBkm@plainTeX@hyphen#1{%
  #1{p}{P}lain-\hologo{TeX}%
}
%    \end{macrocode}
%    \end{macro}
%    \begin{macro}{\HoLogoHtml@plainTeX@hyphen}
%    \begin{macrocode}
\def\HoLogoHtml@plainTeX@hyphen#1{%
  #1{p}{P}lain-\hologo{TeX}%
}
%    \end{macrocode}
%    \end{macro}
%
%    \begin{macro}{\HoLogo@plainTeX@runtogether}
%    \begin{macrocode}
\def\HoLogo@plainTeX@runtogether#1{%
  \HOLOGO@mbox{#1{p}{P}lain\hologo{TeX}}%
}
%    \end{macrocode}
%    \end{macro}
%    \begin{macro}{\HoLogoCs@plainTeX@runtogether}
%    \begin{macrocode}
\def\HoLogoCs@plainTeX@runtogether#1{#1{p}{P}lainTeX}
%    \end{macrocode}
%    \end{macro}
%    \begin{macro}{\HoLogoBkm@plainTeX@runtogether}
%    \begin{macrocode}
\def\HoLogoBkm@plainTeX@runtogether#1{%
  #1{p}{P}lain\hologo{TeX}%
}
%    \end{macrocode}
%    \end{macro}
%    \begin{macro}{\HoLogoHtml@plainTeX@runtogether}
%    \begin{macrocode}
\def\HoLogoHtml@plainTeX@runtogether#1{%
  #1{p}{P}lain\hologo{TeX}%
}
%    \end{macrocode}
%    \end{macro}
%
%    \begin{macro}{\HoLogo@plainTeX}
%    \begin{macrocode}
\def\HoLogo@plainTeX{\HoLogo@plainTeX@space}
%    \end{macrocode}
%    \end{macro}
%    \begin{macro}{\HoLogoCs@plainTeX}
%    \begin{macrocode}
\def\HoLogoCs@plainTeX{\HoLogoCs@plainTeX@space}
%    \end{macrocode}
%    \end{macro}
%    \begin{macro}{\HoLogoBkm@plainTeX}
%    \begin{macrocode}
\def\HoLogoBkm@plainTeX{\HoLogoBkm@plainTeX@space}
%    \end{macrocode}
%    \end{macro}
%    \begin{macro}{\HoLogoHtml@plainTeX}
%    \begin{macrocode}
\def\HoLogoHtml@plainTeX{\HoLogoHtml@plainTeX@space}
%    \end{macrocode}
%    \end{macro}
%
% \subsubsection{\hologo{LaTeX}}
%
%    Source: \hologo{LaTeX} kernel.
%\begin{quote}
%\begin{verbatim}
%\DeclareRobustCommand{\LaTeX}{%
%  L%
%  \kern-.36em%
%  {%
%    \sbox\z@ T%
%    \vbox to\ht\z@{%
%      \hbox{%
%        \check@mathfonts
%        \fontsize\sf@size\z@
%        \math@fontsfalse
%        \selectfont
%        A%
%      }%
%      \vss
%    }%
%  }%
%  \kern-.15em%
%  \TeX
%}
%\end{verbatim}
%\end{quote}
%
%    \begin{macro}{\HoLogo@La}
%    \begin{macrocode}
\def\HoLogo@La#1{%
  L%
  \kern-.36em%
  \begingroup
    \setbox\ltx@zero\hbox{T}%
    \vbox to\ht\ltx@zero{%
      \hbox{%
        \ltx@ifundefined{check@mathfonts}{%
          \csname sevenrm\endcsname
        }{%
          \check@mathfonts
          \fontsize\sf@size{0pt}%
          \math@fontsfalse\selectfont
        }%
        A%
      }%
      \vss
    }%
  \endgroup
}
%    \end{macrocode}
%    \end{macro}
%
%    \begin{macro}{\HoLogo@LaTeX}
%    Source: \hologo{LaTeX} kernel.
%    \begin{macrocode}
\def\HoLogo@LaTeX#1{%
  \hologo{La}%
  \kern-.15em%
  \hologo{TeX}%
}
%    \end{macrocode}
%    \end{macro}
%    \begin{macro}{\HoLogoHtml@LaTeX}
%    \begin{macrocode}
\def\HoLogoHtml@LaTeX#1{%
  \HoLogoCss@LaTeX
  \HOLOGO@Span{LaTeX}{%
    L%
    \HOLOGO@Span{a}{%
      A%
    }%
    \hologo{TeX}%
  }%
}
%    \end{macrocode}
%    \end{macro}
%    \begin{macro}{\HoLogoCss@LaTeX}
%    \begin{macrocode}
\def\HoLogoCss@LaTeX{%
  \Css{%
    span.HoLogo-LaTeX span.HoLogo-a{%
      position:relative;%
      top:-.5ex;%
      margin-left:-.36em;%
      margin-right:-.15em;%
      font-size:85\%;%
    }%
  }%
  \global\let\HoLogoCss@LaTeX\relax
}
%    \end{macrocode}
%    \end{macro}
%
% \subsubsection{\hologo{(La)TeX}}
%
%    \begin{macro}{\HoLogo@LaTeXTeX}
%    The kerning around the parentheses is taken
%    from package \xpackage{dtklogos} \cite{dtklogos}.
%\begin{quote}
%\begin{verbatim}
%\DeclareRobustCommand{\LaTeXTeX}{%
%  (%
%  \kern-.15em%
%  L%
%  \kern-.36em%
%  {%
%    \sbox\z@ T%
%    \vbox to\ht0{%
%      \hbox{%
%        $\m@th$%
%        \csname S@\f@size\endcsname
%        \fontsize\sf@size\z@
%        \math@fontsfalse
%        \selectfont
%        A%
%      }%
%      \vss
%    }%
%  }%
%  \kern-.2em%
%  )%
%  \kern-.15em%
%  \TeX
%}
%\end{verbatim}
%\end{quote}
%    \begin{macrocode}
\def\HoLogo@LaTeXTeX#1{%
  (%
  \kern-.15em%
  \hologo{La}%
  \kern-.2em%
  )%
  \kern-.15em%
  \hologo{TeX}%
}
%    \end{macrocode}
%    \end{macro}
%    \begin{macro}{\HoLogoBkm@LaTeXTeX}
%    \begin{macrocode}
\def\HoLogoBkm@LaTeXTeX#1{(La)TeX}
%    \end{macrocode}
%    \end{macro}
%
%    \begin{macro}{\HoLogo@(La)TeX}
%    \begin{macrocode}
\expandafter
\let\csname HoLogo@(La)TeX\endcsname\HoLogo@LaTeXTeX
%    \end{macrocode}
%    \end{macro}
%    \begin{macro}{\HoLogoBkm@(La)TeX}
%    \begin{macrocode}
\expandafter
\let\csname HoLogoBkm@(La)TeX\endcsname\HoLogoBkm@LaTeXTeX
%    \end{macrocode}
%    \end{macro}
%    \begin{macro}{\HoLogoHtml@LaTeXTeX}
%    \begin{macrocode}
\def\HoLogoHtml@LaTeXTeX#1{%
  \HoLogoCss@LaTeXTeX
  \HOLOGO@Span{LaTeXTeX}{%
    (%
    \HOLOGO@Span{L}{L}%
    \HOLOGO@Span{a}{A}%
    \HOLOGO@Span{ParenRight}{)}%
    \hologo{TeX}%
  }%
}
%    \end{macrocode}
%    \end{macro}
%    \begin{macro}{\HoLogoHtml@(La)TeX}
%    Kerning after opening parentheses and before closing parentheses
%    is $-0.1$\,em. The original values $-0.15$\,em
%    looked too ugly for a serif font.
%    \begin{macrocode}
\expandafter
\let\csname HoLogoHtml@(La)TeX\endcsname\HoLogoHtml@LaTeXTeX
%    \end{macrocode}
%    \end{macro}
%    \begin{macro}{\HoLogoCss@LaTeXTeX}
%    \begin{macrocode}
\def\HoLogoCss@LaTeXTeX{%
  \Css{%
    span.HoLogo-LaTeXTeX span.HoLogo-L{%
      margin-left:-.1em;%
    }%
  }%
  \Css{%
    span.HoLogo-LaTeXTeX span.HoLogo-a{%
      position:relative;%
      top:-.5ex;%
      margin-left:-.36em;%
      margin-right:-.1em;%
      font-size:85\%;%
    }%
  }%
  \Css{%
    span.HoLogo-LaTeXTeX span.HoLogo-ParenRight{%
      margin-right:-.15em;%
    }%
  }%
  \global\let\HoLogoCss@LaTeXTeX\relax
}
%    \end{macrocode}
%    \end{macro}
%
% \subsubsection{\hologo{LaTeXe}}
%
%    \begin{macro}{\HoLogo@LaTeXe}
%    Source: \hologo{LaTeX} kernel
%    \begin{macrocode}
\def\HoLogo@LaTeXe#1{%
  \hologo{LaTeX}%
  \kern.15em%
  \hbox{%
    \HOLOGO@MathSetup
    2%
    $_{\textstyle\varepsilon}$%
  }%
}
%    \end{macrocode}
%    \end{macro}
%
%    \begin{macro}{\HoLogoCs@LaTeXe}
%    \begin{macrocode}
\ifnum64=`\^^^^0040\relax % test for big chars of LuaTeX/XeTeX
  \catcode`\$=9 %
  \catcode`\&=14 %
\else
  \catcode`\$=14 %
  \catcode`\&=9 %
\fi
\def\HoLogoCs@LaTeXe#1{%
  LaTeX2%
$ \string ^^^^0395%
& e%
}%
\catcode`\$=3 %
\catcode`\&=4 %
%    \end{macrocode}
%    \end{macro}
%
%    \begin{macro}{\HoLogoBkm@LaTeXe}
%    \begin{macrocode}
\def\HoLogoBkm@LaTeXe#1{%
  \hologo{LaTeX}%
  2%
  \HOLOGO@PdfdocUnicode{e}{\textepsilon}%
}
%    \end{macrocode}
%    \end{macro}
%
%    \begin{macro}{\HoLogoHtml@LaTeXe}
%    \begin{macrocode}
\def\HoLogoHtml@LaTeXe#1{%
  \HoLogoCss@LaTeXe
  \HOLOGO@Span{LaTeX2e}{%
    \hologo{LaTeX}%
    \HOLOGO@Span{2}{2}%
    \HOLOGO@Span{e}{%
      \HOLOGO@MathSetup
      \ensuremath{\textstyle\varepsilon}%
    }%
  }%
}
%    \end{macrocode}
%    \end{macro}
%    \begin{macro}{\HoLogoCss@LaTeXe}
%    \begin{macrocode}
\def\HoLogoCss@LaTeXe{%
  \Css{%
    span.HoLogo-LaTeX2e span.HoLogo-2{%
      padding-left:.15em;%
    }%
  }%
  \Css{%
    span.HoLogo-LaTeX2e span.HoLogo-e{%
      position:relative;%
      top:.35ex;%
      text-decoration:none;%
    }%
  }%
  \global\let\HoLogoCss@LaTeXe\relax
}
%    \end{macrocode}
%    \end{macro}
%
%    \begin{macro}{\HoLogo@LaTeX2e}
%    \begin{macrocode}
\expandafter
\let\csname HoLogo@LaTeX2e\endcsname\HoLogo@LaTeXe
%    \end{macrocode}
%    \end{macro}
%    \begin{macro}{\HoLogoCs@LaTeX2e}
%    \begin{macrocode}
\expandafter
\let\csname HoLogoCs@LaTeX2e\endcsname\HoLogoCs@LaTeXe
%    \end{macrocode}
%    \end{macro}
%    \begin{macro}{\HoLogoBkm@LaTeX2e}
%    \begin{macrocode}
\expandafter
\let\csname HoLogoBkm@LaTeX2e\endcsname\HoLogoBkm@LaTeXe
%    \end{macrocode}
%    \end{macro}
%    \begin{macro}{\HoLogoHtml@LaTeX2e}
%    \begin{macrocode}
\expandafter
\let\csname HoLogoHtml@LaTeX2e\endcsname\HoLogoHtml@LaTeXe
%    \end{macrocode}
%    \end{macro}
%
% \subsubsection{\hologo{LaTeX3}}
%
%    \begin{macro}{\HoLogo@LaTeX3}
%    Source: \hologo{LaTeX} kernel
%    \begin{macrocode}
\expandafter\def\csname HoLogo@LaTeX3\endcsname#1{%
  \hologo{LaTeX}%
  3%
}
%    \end{macrocode}
%    \end{macro}
%
%    \begin{macro}{\HoLogoBkm@LaTeX3}
%    \begin{macrocode}
\expandafter\def\csname HoLogoBkm@LaTeX3\endcsname#1{%
  \hologo{LaTeX}%
  3%
}
%    \end{macrocode}
%    \end{macro}
%    \begin{macro}{\HoLogoHtml@LaTeX3}
%    \begin{macrocode}
\expandafter
\let\csname HoLogoHtml@LaTeX3\expandafter\endcsname
\csname HoLogo@LaTeX3\endcsname
%    \end{macrocode}
%    \end{macro}
%
% \subsubsection{\hologo{LaTeXML}}
%
%    \begin{macro}{\HoLogo@LaTeXML}
%    \begin{macrocode}
\def\HoLogo@LaTeXML#1{%
  \HOLOGO@mbox{%
    \hologo{La}%
    \kern-.15em%
    T%
    \kern-.1667em%
    \lower.5ex\hbox{E}%
    \kern-.125em%
    \HoLogoFont@font{LaTeXML}{sc}{xml}%
  }%
}
%    \end{macrocode}
%    \end{macro}
%    \begin{macro}{\HoLogoHtml@pdfLaTeX}
%    \begin{macrocode}
\def\HoLogoHtml@LaTeXML#1{%
  \HOLOGO@Span{LaTeXML}{%
    \HoLogoCss@LaTeX
    \HoLogoCss@TeX
    \HOLOGO@Span{LaTeX}{%
      L%
      \HOLOGO@Span{a}{%
        A%
      }%
    }%
    \HOLOGO@Span{TeX}{%
      T%
      \HOLOGO@Span{e}{%
        E%
      }%
    }%
    \HCode{<span style="font-variant: small-caps;">}%
    xml%
    \HCode{</span>}%
  }%
}
%    \end{macrocode}
%    \end{macro}
%
% \subsubsection{\hologo{eTeX}}
%
%    \begin{macro}{\HoLogo@eTeX}
%    Source: package \xpackage{etex}
%    \begin{macrocode}
\def\HoLogo@eTeX#1{%
  \ltx@mbox{%
    \HOLOGO@MathSetup
    $\varepsilon$%
    -%
    \HOLOGO@NegativeKerning{-T,T-,To}%
    \hologo{TeX}%
  }%
}
%    \end{macrocode}
%    \end{macro}
%    \begin{macro}{\HoLogoCs@eTeX}
%    \begin{macrocode}
\ifnum64=`\^^^^0040\relax % test for big chars of LuaTeX/XeTeX
  \catcode`\$=9 %
  \catcode`\&=14 %
\else
  \catcode`\$=14 %
  \catcode`\&=9 %
\fi
\def\HoLogoCs@eTeX#1{%
$ #1{\string ^^^^0395}{\string ^^^^03b5}%
& #1{e}{E}%
  TeX%
}%
\catcode`\$=3 %
\catcode`\&=4 %
%    \end{macrocode}
%    \end{macro}
%    \begin{macro}{\HoLogoBkm@eTeX}
%    \begin{macrocode}
\def\HoLogoBkm@eTeX#1{%
  \HOLOGO@PdfdocUnicode{#1{e}{E}}{\textepsilon}%
  -%
  \hologo{TeX}%
}
%    \end{macrocode}
%    \end{macro}
%    \begin{macro}{\HoLogoHtml@eTeX}
%    \begin{macrocode}
\def\HoLogoHtml@eTeX#1{%
  \ltx@mbox{%
    \HOLOGO@MathSetup
    $\varepsilon$%
    -%
    \hologo{TeX}%
  }%
}
%    \end{macrocode}
%    \end{macro}
%
% \subsubsection{\hologo{iniTeX}}
%
%    \begin{macro}{\HoLogo@iniTeX}
%    \begin{macrocode}
\def\HoLogo@iniTeX#1{%
  \HOLOGO@mbox{%
    #1{i}{I}ni\hologo{TeX}%
  }%
}
%    \end{macrocode}
%    \end{macro}
%    \begin{macro}{\HoLogoCs@iniTeX}
%    \begin{macrocode}
\def\HoLogoCs@iniTeX#1{#1{i}{I}niTeX}
%    \end{macrocode}
%    \end{macro}
%    \begin{macro}{\HoLogoBkm@iniTeX}
%    \begin{macrocode}
\def\HoLogoBkm@iniTeX#1{%
  #1{i}{I}ni\hologo{TeX}%
}
%    \end{macrocode}
%    \end{macro}
%    \begin{macro}{\HoLogoHtml@iniTeX}
%    \begin{macrocode}
\let\HoLogoHtml@iniTeX\HoLogo@iniTeX
%    \end{macrocode}
%    \end{macro}
%
% \subsubsection{\hologo{virTeX}}
%
%    \begin{macro}{\HoLogo@virTeX}
%    \begin{macrocode}
\def\HoLogo@virTeX#1{%
  \HOLOGO@mbox{%
    #1{v}{V}ir\hologo{TeX}%
  }%
}
%    \end{macrocode}
%    \end{macro}
%    \begin{macro}{\HoLogoCs@virTeX}
%    \begin{macrocode}
\def\HoLogoCs@virTeX#1{#1{v}{V}irTeX}
%    \end{macrocode}
%    \end{macro}
%    \begin{macro}{\HoLogoBkm@virTeX}
%    \begin{macrocode}
\def\HoLogoBkm@virTeX#1{%
  #1{v}{V}ir\hologo{TeX}%
}
%    \end{macrocode}
%    \end{macro}
%    \begin{macro}{\HoLogoHtml@virTeX}
%    \begin{macrocode}
\let\HoLogoHtml@virTeX\HoLogo@virTeX
%    \end{macrocode}
%    \end{macro}
%
% \subsubsection{\hologo{SliTeX}}
%
% \paragraph{Definitions of the three variants.}
%
%    \begin{macro}{\HoLogo@SLiTeX@lift}
%    \begin{macrocode}
\def\HoLogo@SLiTeX@lift#1{%
  \HoLogoFont@font{SliTeX}{rm}{%
    S%
    \kern-.06em%
    L%
    \kern-.18em%
    \raise.32ex\hbox{\HoLogoFont@font{SliTeX}{sc}{i}}%
    \HOLOGO@discretionary
    \kern-.06em%
    \hologo{TeX}%
  }%
}
%    \end{macrocode}
%    \end{macro}
%    \begin{macro}{\HoLogoBkm@SLiTeX@lift}
%    \begin{macrocode}
\def\HoLogoBkm@SLiTeX@lift#1{SLiTeX}
%    \end{macrocode}
%    \end{macro}
%    \begin{macro}{\HoLogoHtml@SLiTeX@lift}
%    \begin{macrocode}
\def\HoLogoHtml@SLiTeX@lift#1{%
  \HoLogoCss@SLiTeX@lift
  \HOLOGO@Span{SLiTeX-lift}{%
    \HoLogoFont@font{SliTeX}{rm}{%
      S%
      \HOLOGO@Span{L}{L}%
      \HOLOGO@Span{i}{i}%
      \hologo{TeX}%
    }%
  }%
}
%    \end{macrocode}
%    \end{macro}
%    \begin{macro}{\HoLogoCss@SLiTeX@lift}
%    \begin{macrocode}
\def\HoLogoCss@SLiTeX@lift{%
  \Css{%
    span.HoLogo-SLiTeX-lift span.HoLogo-L{%
      margin-left:-.06em;%
      margin-right:-.18em;%
    }%
  }%
  \Css{%
    span.HoLogo-SLiTeX-lift span.HoLogo-i{%
      position:relative;%
      top:-.32ex;%
      margin-right:-.06em;%
      font-variant:small-caps;%
    }%
  }%
  \global\let\HoLogoCss@SLiTeX@lift\relax
}
%    \end{macrocode}
%    \end{macro}
%
%    \begin{macro}{\HoLogo@SliTeX@simple}
%    \begin{macrocode}
\def\HoLogo@SliTeX@simple#1{%
  \HoLogoFont@font{SliTeX}{rm}{%
    \ltx@mbox{%
      \HoLogoFont@font{SliTeX}{sc}{Sli}%
    }%
    \HOLOGO@discretionary
    \hologo{TeX}%
  }%
}
%    \end{macrocode}
%    \end{macro}
%    \begin{macro}{\HoLogoBkm@SliTeX@simple}
%    \begin{macrocode}
\def\HoLogoBkm@SliTeX@simple#1{SliTeX}
%    \end{macrocode}
%    \end{macro}
%    \begin{macro}{\HoLogoHtml@SliTeX@simple}
%    \begin{macrocode}
\let\HoLogoHtml@SliTeX@simple\HoLogo@SliTeX@simple
%    \end{macrocode}
%    \end{macro}
%
%    \begin{macro}{\HoLogo@SliTeX@narrow}
%    \begin{macrocode}
\def\HoLogo@SliTeX@narrow#1{%
  \HoLogoFont@font{SliTeX}{rm}{%
    \ltx@mbox{%
      S%
      \kern-.06em%
      \HoLogoFont@font{SliTeX}{sc}{%
        l%
        \kern-.035em%
        i%
      }%
    }%
    \HOLOGO@discretionary
    \kern-.06em%
    \hologo{TeX}%
  }%
}
%    \end{macrocode}
%    \end{macro}
%    \begin{macro}{\HoLogoBkm@SliTeX@narrow}
%    \begin{macrocode}
\def\HoLogoBkm@SliTeX@narrow#1{SliTeX}
%    \end{macrocode}
%    \end{macro}
%    \begin{macro}{\HoLogoHtml@SliTeX@narrow}
%    \begin{macrocode}
\def\HoLogoHtml@SliTeX@narrow#1{%
  \HoLogoCss@SliTeX@narrow
  \HOLOGO@Span{SliTeX-narrow}{%
    \HoLogoFont@font{SliTeX}{rm}{%
      S%
        \HOLOGO@Span{l}{l}%
        \HOLOGO@Span{i}{i}%
      \hologo{TeX}%
    }%
  }%
}
%    \end{macrocode}
%    \end{macro}
%    \begin{macro}{\HoLogoCss@SliTeX@narrow}
%    \begin{macrocode}
\def\HoLogoCss@SliTeX@narrow{%
  \Css{%
    span.HoLogo-SliTeX-narrow span.HoLogo-l{%
      margin-left:-.06em;%
      margin-right:-.035em;%
      font-variant:small-caps;%
    }%
  }%
  \Css{%
    span.HoLogo-SliTeX-narrow span.HoLogo-i{%
      margin-right:-.06em;%
      font-variant:small-caps;%
    }%
  }%
  \global\let\HoLogoCss@SliTeX@narrow\relax
}
%    \end{macrocode}
%    \end{macro}
%
% \paragraph{Macro set completion.}
%
%    \begin{macro}{\HoLogo@SLiTeX@simple}
%    \begin{macrocode}
\def\HoLogo@SLiTeX@simple{\HoLogo@SliTeX@simple}
%    \end{macrocode}
%    \end{macro}
%    \begin{macro}{\HoLogoBkm@SLiTeX@simple}
%    \begin{macrocode}
\def\HoLogoBkm@SLiTeX@simple{\HoLogoBkm@SliTeX@simple}
%    \end{macrocode}
%    \end{macro}
%    \begin{macro}{\HoLogoHtml@SLiTeX@simple}
%    \begin{macrocode}
\def\HoLogoHtml@SLiTeX@simple{\HoLogoHtml@SliTeX@simple}
%    \end{macrocode}
%    \end{macro}
%
%    \begin{macro}{\HoLogo@SLiTeX@narrow}
%    \begin{macrocode}
\def\HoLogo@SLiTeX@narrow{\HoLogo@SliTeX@narrow}
%    \end{macrocode}
%    \end{macro}
%    \begin{macro}{\HoLogoBkm@SLiTeX@narrow}
%    \begin{macrocode}
\def\HoLogoBkm@SLiTeX@narrow{\HoLogoBkm@SliTeX@narrow}
%    \end{macrocode}
%    \end{macro}
%    \begin{macro}{\HoLogoHtml@SLiTeX@narrow}
%    \begin{macrocode}
\def\HoLogoHtml@SLiTeX@narrow{\HoLogoHtml@SliTeX@narrow}
%    \end{macrocode}
%    \end{macro}
%
%    \begin{macro}{\HoLogo@SliTeX@lift}
%    \begin{macrocode}
\def\HoLogo@SliTeX@lift{\HoLogo@SLiTeX@lift}
%    \end{macrocode}
%    \end{macro}
%    \begin{macro}{\HoLogoBkm@SliTeX@lift}
%    \begin{macrocode}
\def\HoLogoBkm@SliTeX@lift{\HoLogoBkm@SLiTeX@lift}
%    \end{macrocode}
%    \end{macro}
%    \begin{macro}{\HoLogoHtml@SliTeX@lift}
%    \begin{macrocode}
\def\HoLogoHtml@SliTeX@lift{\HoLogoHtml@SLiTeX@lift}
%    \end{macrocode}
%    \end{macro}
%
% \paragraph{Defaults.}
%
%    \begin{macro}{\HoLogo@SLiTeX}
%    \begin{macrocode}
\def\HoLogo@SLiTeX{\HoLogo@SLiTeX@lift}
%    \end{macrocode}
%    \end{macro}
%    \begin{macro}{\HoLogoBkm@SLiTeX}
%    \begin{macrocode}
\def\HoLogoBkm@SLiTeX{\HoLogoBkm@SLiTeX@lift}
%    \end{macrocode}
%    \end{macro}
%    \begin{macro}{\HoLogoHtml@SLiTeX}
%    \begin{macrocode}
\def\HoLogoHtml@SLiTeX{\HoLogoHtml@SLiTeX@lift}
%    \end{macrocode}
%    \end{macro}
%
%    \begin{macro}{\HoLogo@SliTeX}
%    \begin{macrocode}
\def\HoLogo@SliTeX{\HoLogo@SliTeX@narrow}
%    \end{macrocode}
%    \end{macro}
%    \begin{macro}{\HoLogoBkm@SliTeX}
%    \begin{macrocode}
\def\HoLogoBkm@SliTeX{\HoLogoBkm@SliTeX@narrow}
%    \end{macrocode}
%    \end{macro}
%    \begin{macro}{\HoLogoHtml@SliTeX}
%    \begin{macrocode}
\def\HoLogoHtml@SliTeX{\HoLogoHtml@SliTeX@narrow}
%    \end{macrocode}
%    \end{macro}
%
% \subsubsection{\hologo{LuaTeX}}
%
%    \begin{macro}{\HoLogo@LuaTeX}
%    The kerning is an idea of Hans Hagen, see mailing list
%    `luatex at tug dot org' in March 2010.
%    \begin{macrocode}
\def\HoLogo@LuaTeX#1{%
  \HOLOGO@mbox{%
    Lua%
    \HOLOGO@NegativeKerning{aT,oT,To}%
    \hologo{TeX}%
  }%
}
%    \end{macrocode}
%    \end{macro}
%    \begin{macro}{\HoLogoHtml@LuaTeX}
%    \begin{macrocode}
\let\HoLogoHtml@LuaTeX\HoLogo@LuaTeX
%    \end{macrocode}
%    \end{macro}
%
% \subsubsection{\hologo{LuaLaTeX}}
%
%    \begin{macro}{\HoLogo@LuaLaTeX}
%    \begin{macrocode}
\def\HoLogo@LuaLaTeX#1{%
  \HOLOGO@mbox{%
    Lua%
    \hologo{LaTeX}%
  }%
}
%    \end{macrocode}
%    \end{macro}
%    \begin{macro}{\HoLogoHtml@LuaLaTeX}
%    \begin{macrocode}
\let\HoLogoHtml@LuaLaTeX\HoLogo@LuaLaTeX
%    \end{macrocode}
%    \end{macro}
%
% \subsubsection{\hologo{XeTeX}, \hologo{XeLaTeX}}
%
%    \begin{macro}{\HOLOGO@IfCharExists}
%    \begin{macrocode}
\ifluatex
  \ifnum\luatexversion<36 %
  \else
    \def\HOLOGO@IfCharExists#1{%
      \ifnum
        \directlua{%
           if luaotfload and luaotfload.aux then
             if luaotfload.aux.font_has_glyph(%
                    font.current(), \number#1) then % 	 
	       tex.print("1") % 	 
	     end % 	 
	   elseif font and font.fonts and font.current then %
            local f = font.fonts[font.current()]%
            if f.characters and f.characters[\number#1] then %
              tex.print("1")%
            end %
          end%
        }0=\ltx@zero
        \expandafter\ltx@secondoftwo
      \else
        \expandafter\ltx@firstoftwo
      \fi
    }%
  \fi
\fi
\ltx@IfUndefined{HOLOGO@IfCharExists}{%
  \def\HOLOGO@@IfCharExists#1{%
    \begingroup
      \tracinglostchars=\ltx@zero
      \setbox\ltx@zero=\hbox{%
        \kern7sp\char#1\relax
        \ifnum\lastkern>\ltx@zero
          \expandafter\aftergroup\csname iffalse\endcsname
        \else
          \expandafter\aftergroup\csname iftrue\endcsname
        \fi
      }%
      % \if{true|false} from \aftergroup
      \endgroup
      \expandafter\ltx@firstoftwo
    \else
      \endgroup
      \expandafter\ltx@secondoftwo
    \fi
  }%
  \ifxetex
    \ltx@IfUndefined{XeTeXfonttype}{}{%
      \ltx@IfUndefined{XeTeXcharglyph}{}{%
        \def\HOLOGO@IfCharExists#1{%
          \ifnum\XeTeXfonttype\font>\ltx@zero
            \expandafter\ltx@firstofthree
          \else
            \expandafter\ltx@gobble
          \fi
          {%
            \ifnum\XeTeXcharglyph#1>\ltx@zero
              \expandafter\ltx@firstoftwo
            \else
              \expandafter\ltx@secondoftwo
            \fi
          }%
          \HOLOGO@@IfCharExists{#1}%
        }%
      }%
    }%
  \fi
}{}
\ltx@ifundefined{HOLOGO@IfCharExists}{%
  \ifnum64=`\^^^^0040\relax % test for big chars of LuaTeX/XeTeX
    \let\HOLOGO@IfCharExists\HOLOGO@@IfCharExists
  \else
    \def\HOLOGO@IfCharExists#1{%
      \ifnum#1>255 %
        \expandafter\ltx@fourthoffour
      \fi
      \HOLOGO@@IfCharExists{#1}%
    }%
  \fi
}{}
%    \end{macrocode}
%    \end{macro}
%
%    \begin{macro}{\HoLogo@Xe}
%    Source: package \xpackage{dtklogos}
%    \begin{macrocode}
\def\HoLogo@Xe#1{%
  X%
  \kern-.1em\relax
  \HOLOGO@IfCharExists{"018E}{%
    \lower.5ex\hbox{\char"018E}%
  }{%
    \chardef\HOLOGO@choice=\ltx@zero
    \ifdim\fontdimen\ltx@one\font>0pt %
      \ltx@IfUndefined{rotatebox}{%
        \ltx@IfUndefined{pgftext}{%
          \ltx@IfUndefined{psscalebox}{%
            \ltx@IfUndefined{HOLOGO@ScaleBox@\hologoDriver}{%
            }{%
              \chardef\HOLOGO@choice=4 %
            }%
          }{%
            \chardef\HOLOGO@choice=3 %
          }%
        }{%
          \chardef\HOLOGO@choice=2 %
        }%
      }{%
        \chardef\HOLOGO@choice=1 %
      }%
      \ifcase\HOLOGO@choice
        \HOLOGO@WarningUnsupportedDriver{Xe}%
        e%
      \or % 1: \rotatebox
        \begingroup
          \setbox\ltx@zero\hbox{\rotatebox{180}{E}}%
          \ltx@LocDimenA=\dp\ltx@zero
          \advance\ltx@LocDimenA by -.5ex\relax
          \raise\ltx@LocDimenA\box\ltx@zero
        \endgroup
      \or % 2: \pgftext
        \lower.5ex\hbox{%
          \pgfpicture
            \pgftext[rotate=180]{E}%
          \endpgfpicture
        }%
      \or % 3: \psscalebox
        \begingroup
          \setbox\ltx@zero\hbox{\psscalebox{-1 -1}{E}}%
          \ltx@LocDimenA=\dp\ltx@zero
          \advance\ltx@LocDimenA by -.5ex\relax
          \raise\ltx@LocDimenA\box\ltx@zero
        \endgroup
      \or % 4: \HOLOGO@PointReflectBox
        \lower.5ex\hbox{\HOLOGO@PointReflectBox{E}}%
      \else
        \@PackageError{hologo}{Internal error (choice/it}\@ehc
      \fi
    \else
      \ltx@IfUndefined{reflectbox}{%
        \ltx@IfUndefined{pgftext}{%
          \ltx@IfUndefined{psscalebox}{%
            \ltx@IfUndefined{HOLOGO@ScaleBox@\hologoDriver}{%
            }{%
              \chardef\HOLOGO@choice=4 %
            }%
          }{%
            \chardef\HOLOGO@choice=3 %
          }%
        }{%
          \chardef\HOLOGO@choice=2 %
        }%
      }{%
        \chardef\HOLOGO@choice=1 %
      }%
      \ifcase\HOLOGO@choice
        \HOLOGO@WarningUnsupportedDriver{Xe}%
        e%
      \or % 1: reflectbox
        \lower.5ex\hbox{%
          \reflectbox{E}%
        }%
      \or % 2: \pgftext
        \lower.5ex\hbox{%
          \pgfpicture
            \pgftransformxscale{-1}%
            \pgftext{E}%
          \endpgfpicture
        }%
      \or % 3: \psscalebox
        \lower.5ex\hbox{%
          \psscalebox{-1 1}{E}%
        }%
      \or % 4: \HOLOGO@Reflectbox
        \lower.5ex\hbox{%
          \HOLOGO@ReflectBox{E}%
        }%
      \else
        \@PackageError{hologo}{Internal error (choice/up)}\@ehc
      \fi
    \fi
  }%
}
%    \end{macrocode}
%    \end{macro}
%    \begin{macro}{\HoLogoHtml@Xe}
%    \begin{macrocode}
\def\HoLogoHtml@Xe#1{%
  \HoLogoCss@Xe
  \HOLOGO@Span{Xe}{%
    X%
    \HOLOGO@Span{e}{%
      \HCode{&\ltx@hashchar x018e;}%
    }%
  }%
}
%    \end{macrocode}
%    \end{macro}
%    \begin{macro}{\HoLogoCss@Xe}
%    \begin{macrocode}
\def\HoLogoCss@Xe{%
  \Css{%
    span.HoLogo-Xe span.HoLogo-e{%
      position:relative;%
      top:.5ex;%
      left-margin:-.1em;%
    }%
  }%
  \global\let\HoLogoCss@Xe\relax
}
%    \end{macrocode}
%    \end{macro}
%
%    \begin{macro}{\HoLogo@XeTeX}
%    \begin{macrocode}
\def\HoLogo@XeTeX#1{%
  \hologo{Xe}%
  \kern-.15em\relax
  \hologo{TeX}%
}
%    \end{macrocode}
%    \end{macro}
%
%    \begin{macro}{\HoLogoHtml@XeTeX}
%    \begin{macrocode}
\def\HoLogoHtml@XeTeX#1{%
  \HoLogoCss@XeTeX
  \HOLOGO@Span{XeTeX}{%
    \hologo{Xe}%
    \hologo{TeX}%
  }%
}
%    \end{macrocode}
%    \end{macro}
%    \begin{macro}{\HoLogoCss@XeTeX}
%    \begin{macrocode}
\def\HoLogoCss@XeTeX{%
  \Css{%
    span.HoLogo-XeTeX span.HoLogo-TeX{%
      margin-left:-.15em;%
    }%
  }%
  \global\let\HoLogoCss@XeTeX\relax
}
%    \end{macrocode}
%    \end{macro}
%
%    \begin{macro}{\HoLogo@XeLaTeX}
%    \begin{macrocode}
\def\HoLogo@XeLaTeX#1{%
  \hologo{Xe}%
  \kern-.13em%
  \hologo{LaTeX}%
}
%    \end{macrocode}
%    \end{macro}
%    \begin{macro}{\HoLogoHtml@XeLaTeX}
%    \begin{macrocode}
\def\HoLogoHtml@XeLaTeX#1{%
  \HoLogoCss@XeLaTeX
  \HOLOGO@Span{XeLaTeX}{%
    \hologo{Xe}%
    \hologo{LaTeX}%
  }%
}
%    \end{macrocode}
%    \end{macro}
%    \begin{macro}{\HoLogoCss@XeLaTeX}
%    \begin{macrocode}
\def\HoLogoCss@XeLaTeX{%
  \Css{%
    span.HoLogo-XeLaTeX span.HoLogo-Xe{%
      margin-right:-.13em;%
    }%
  }%
  \global\let\HoLogoCss@XeLaTeX\relax
}
%    \end{macrocode}
%    \end{macro}
%
% \subsubsection{\hologo{pdfTeX}, \hologo{pdfLaTeX}}
%
%    \begin{macro}{\HoLogo@pdfTeX}
%    \begin{macrocode}
\def\HoLogo@pdfTeX#1{%
  \HOLOGO@mbox{%
    #1{p}{P}df\hologo{TeX}%
  }%
}
%    \end{macrocode}
%    \end{macro}
%    \begin{macro}{\HoLogoCs@pdfTeX}
%    \begin{macrocode}
\def\HoLogoCs@pdfTeX#1{#1{p}{P}dfTeX}
%    \end{macrocode}
%    \end{macro}
%    \begin{macro}{\HoLogoBkm@pdfTeX}
%    \begin{macrocode}
\def\HoLogoBkm@pdfTeX#1{%
  #1{p}{P}df\hologo{TeX}%
}
%    \end{macrocode}
%    \end{macro}
%    \begin{macro}{\HoLogoHtml@pdfTeX}
%    \begin{macrocode}
\let\HoLogoHtml@pdfTeX\HoLogo@pdfTeX
%    \end{macrocode}
%    \end{macro}
%
%    \begin{macro}{\HoLogo@pdfLaTeX}
%    \begin{macrocode}
\def\HoLogo@pdfLaTeX#1{%
  \HOLOGO@mbox{%
    #1{p}{P}df\hologo{LaTeX}%
  }%
}
%    \end{macrocode}
%    \end{macro}
%    \begin{macro}{\HoLogoCs@pdfLaTeX}
%    \begin{macrocode}
\def\HoLogoCs@pdfLaTeX#1{#1{p}{P}dfLaTeX}
%    \end{macrocode}
%    \end{macro}
%    \begin{macro}{\HoLogoBkm@pdfLaTeX}
%    \begin{macrocode}
\def\HoLogoBkm@pdfLaTeX#1{%
  #1{p}{P}df\hologo{LaTeX}%
}
%    \end{macrocode}
%    \end{macro}
%    \begin{macro}{\HoLogoHtml@pdfLaTeX}
%    \begin{macrocode}
\let\HoLogoHtml@pdfLaTeX\HoLogo@pdfLaTeX
%    \end{macrocode}
%    \end{macro}
%
% \subsubsection{\hologo{VTeX}}
%
%    \begin{macro}{\HoLogo@VTeX}
%    \begin{macrocode}
\def\HoLogo@VTeX#1{%
  \HOLOGO@mbox{%
    V\hologo{TeX}%
  }%
}
%    \end{macrocode}
%    \end{macro}
%    \begin{macro}{\HoLogoHtml@VTeX}
%    \begin{macrocode}
\let\HoLogoHtml@VTeX\HoLogo@VTeX
%    \end{macrocode}
%    \end{macro}
%
% \subsubsection{\hologo{AmS}, \dots}
%
%    Source: class \xclass{amsdtx}
%
%    \begin{macro}{\HoLogo@AmS}
%    \begin{macrocode}
\def\HoLogo@AmS#1{%
  \HoLogoFont@font{AmS}{sy}{%
    A%
    \kern-.1667em%
    \lower.5ex\hbox{M}%
    \kern-.125em%
    S%
  }%
}
%    \end{macrocode}
%    \end{macro}
%    \begin{macro}{\HoLogoBkm@AmS}
%    \begin{macrocode}
\def\HoLogoBkm@AmS#1{AmS}
%    \end{macrocode}
%    \end{macro}
%    \begin{macro}{\HoLogoHtml@AmS}
%    \begin{macrocode}
\def\HoLogoHtml@AmS#1{%
  \HoLogoCss@AmS
%  \HoLogoFont@font{AmS}{sy}{%
    \HOLOGO@Span{AmS}{%
      A%
      \HOLOGO@Span{M}{M}%
      S%
    }%
%   }%
}
%    \end{macrocode}
%    \end{macro}
%    \begin{macro}{\HoLogoCss@AmS}
%    \begin{macrocode}
\def\HoLogoCss@AmS{%
  \Css{%
    span.HoLogo-AmS span.HoLogo-M{%
      position:relative;%
      top:.5ex;%
      margin-left:-.1667em;%
      margin-right:-.125em;%
      text-decoration:none;%
    }%
  }%
  \global\let\HoLogoCss@AmS\relax
}
%    \end{macrocode}
%    \end{macro}
%
%    \begin{macro}{\HoLogo@AmSTeX}
%    \begin{macrocode}
\def\HoLogo@AmSTeX#1{%
  \hologo{AmS}%
  \HOLOGO@hyphen
  \hologo{TeX}%
}
%    \end{macrocode}
%    \end{macro}
%    \begin{macro}{\HoLogoBkm@AmSTeX}
%    \begin{macrocode}
\def\HoLogoBkm@AmSTeX#1{AmS-TeX}%
%    \end{macrocode}
%    \end{macro}
%    \begin{macro}{\HoLogoHtml@AmSTeX}
%    \begin{macrocode}
\let\HoLogoHtml@AmSTeX\HoLogo@AmSTeX
%    \end{macrocode}
%    \end{macro}
%
%    \begin{macro}{\HoLogo@AmSLaTeX}
%    \begin{macrocode}
\def\HoLogo@AmSLaTeX#1{%
  \hologo{AmS}%
  \HOLOGO@hyphen
  \hologo{LaTeX}%
}
%    \end{macrocode}
%    \end{macro}
%    \begin{macro}{\HoLogoBkm@AmSLaTeX}
%    \begin{macrocode}
\def\HoLogoBkm@AmSLaTeX#1{AmS-LaTeX}%
%    \end{macrocode}
%    \end{macro}
%    \begin{macro}{\HoLogoHtml@AmSLaTeX}
%    \begin{macrocode}
\let\HoLogoHtml@AmSLaTeX\HoLogo@AmSLaTeX
%    \end{macrocode}
%    \end{macro}
%
% \subsubsection{\hologo{BibTeX}}
%
%    \begin{macro}{\HoLogo@BibTeX@sc}
%    A definition of \hologo{BibTeX} is provided in
%    the documentation source for the manual of \hologo{BibTeX}
%    \cite{btxdoc}.
%\begin{quote}
%\begin{verbatim}
%\def\BibTeX{%
%  {%
%    \rm
%    B%
%    \kern-.05em%
%    {%
%      \sc
%      i%
%      \kern-.025em %
%      b%
%    }%
%    \kern-.08em
%    T%
%    \kern-.1667em%
%    \lower.7ex\hbox{E}%
%    \kern-.125em%
%    X%
%  }%
%}
%\end{verbatim}
%\end{quote}
%    \begin{macrocode}
\def\HoLogo@BibTeX@sc#1{%
  B%
  \kern-.05em%
  \HoLogoFont@font{BibTeX}{sc}{%
    i%
    \kern-.025em%
    b%
  }%
  \HOLOGO@discretionary
  \kern-.08em%
  \hologo{TeX}%
}
%    \end{macrocode}
%    \end{macro}
%    \begin{macro}{\HoLogoHtml@BibTeX@sc}
%    \begin{macrocode}
\def\HoLogoHtml@BibTeX@sc#1{%
  \HoLogoCss@BibTeX@sc
  \HOLOGO@Span{BibTeX-sc}{%
    B%
    \HOLOGO@Span{i}{i}%
    \HOLOGO@Span{b}{b}%
    \hologo{TeX}%
  }%
}
%    \end{macrocode}
%    \end{macro}
%    \begin{macro}{\HoLogoCss@BibTeX@sc}
%    \begin{macrocode}
\def\HoLogoCss@BibTeX@sc{%
  \Css{%
    span.HoLogo-BibTeX-sc span.HoLogo-i{%
      margin-left:-.05em;%
      margin-right:-.025em;%
      font-variant:small-caps;%
    }%
  }%
  \Css{%
    span.HoLogo-BibTeX-sc span.HoLogo-b{%
      margin-right:-.08em;%
      font-variant:small-caps;%
    }%
  }%
  \global\let\HoLogoCss@BibTeX@sc\relax
}
%    \end{macrocode}
%    \end{macro}
%
%    \begin{macro}{\HoLogo@BibTeX@sf}
%    Variant \xoption{sf} avoids trouble with unavailable
%    small caps fonts (e.g., bold versions of Computer Modern or
%    Latin Modern). The definition is taken from
%    package \xpackage{dtklogos} \cite{dtklogos}.
%\begin{quote}
%\begin{verbatim}
%\DeclareRobustCommand{\BibTeX}{%
%  B%
%  \kern-.05em%
%  \hbox{%
%    $\m@th$% %% force math size calculations
%    \csname S@\f@size\endcsname
%    \fontsize\sf@size\z@
%    \math@fontsfalse
%    \selectfont
%    I%
%    \kern-.025em%
%    B
%  }%
%  \kern-.08em%
%  \-%
%  \TeX
%}
%\end{verbatim}
%\end{quote}
%    \begin{macrocode}
\def\HoLogo@BibTeX@sf#1{%
  B%
  \kern-.05em%
  \HoLogoFont@font{BibTeX}{bibsf}{%
    I%
    \kern-.025em%
    B%
  }%
  \HOLOGO@discretionary
  \kern-.08em%
  \hologo{TeX}%
}
%    \end{macrocode}
%    \end{macro}
%    \begin{macro}{\HoLogoHtml@BibTeX@sf}
%    \begin{macrocode}
\def\HoLogoHtml@BibTeX@sf#1{%
  \HoLogoCss@BibTeX@sf
  \HOLOGO@Span{BibTeX-sf}{%
    B%
    \HoLogoFont@font{BibTeX}{bibsf}{%
      \HOLOGO@Span{i}{I}%
      B%
    }%
    \hologo{TeX}%
  }%
}
%    \end{macrocode}
%    \end{macro}
%    \begin{macro}{\HoLogoCss@BibTeX@sf}
%    \begin{macrocode}
\def\HoLogoCss@BibTeX@sf{%
  \Css{%
    span.HoLogo-BibTeX-sf span.HoLogo-i{%
      margin-left:-.05em;%
      margin-right:-.025em;%
    }%
  }%
  \Css{%
    span.HoLogo-BibTeX-sf span.HoLogo-TeX{%
      margin-left:-.08em;%
    }%
  }%
  \global\let\HoLogoCss@BibTeX@sf\relax
}
%    \end{macrocode}
%    \end{macro}
%
%    \begin{macro}{\HoLogo@BibTeX}
%    \begin{macrocode}
\def\HoLogo@BibTeX{\HoLogo@BibTeX@sf}
%    \end{macrocode}
%    \end{macro}
%    \begin{macro}{\HoLogoHtml@BibTeX}
%    \begin{macrocode}
\def\HoLogoHtml@BibTeX{\HoLogoHtml@BibTeX@sf}
%    \end{macrocode}
%    \end{macro}
%
% \subsubsection{\hologo{BibTeX8}}
%
%    \begin{macro}{\HoLogo@BibTeX8}
%    \begin{macrocode}
\expandafter\def\csname HoLogo@BibTeX8\endcsname#1{%
  \hologo{BibTeX}%
  8%
}
%    \end{macrocode}
%    \end{macro}
%
%    \begin{macro}{\HoLogoBkm@BibTeX8}
%    \begin{macrocode}
\expandafter\def\csname HoLogoBkm@BibTeX8\endcsname#1{%
  \hologo{BibTeX}%
  8%
}
%    \end{macrocode}
%    \end{macro}
%    \begin{macro}{\HoLogoHtml@BibTeX8}
%    \begin{macrocode}
\expandafter
\let\csname HoLogoHtml@BibTeX8\expandafter\endcsname
\csname HoLogo@BibTeX8\endcsname
%    \end{macrocode}
%    \end{macro}
%
% \subsubsection{\hologo{ConTeXt}}
%
%    \begin{macro}{\HoLogo@ConTeXt@simple}
%    \begin{macrocode}
\def\HoLogo@ConTeXt@simple#1{%
  \HOLOGO@mbox{Con}%
  \HOLOGO@discretionary
  \HOLOGO@mbox{\hologo{TeX}t}%
}
%    \end{macrocode}
%    \end{macro}
%    \begin{macro}{\HoLogoHtml@ConTeXt@simple}
%    \begin{macrocode}
\let\HoLogoHtml@ConTeXt@simple\HoLogo@ConTeXt@simple
%    \end{macrocode}
%    \end{macro}
%
%    \begin{macro}{\HoLogo@ConTeXt@narrow}
%    This definition of logo \hologo{ConTeXt} with variant \xoption{narrow}
%    comes from TUGboat's class \xclass{ltugboat} (version 2010/11/15 v2.8).
%    \begin{macrocode}
\def\HoLogo@ConTeXt@narrow#1{%
  \HOLOGO@mbox{C\kern-.0333emon}%
  \HOLOGO@discretionary
  \kern-.0667em%
  \HOLOGO@mbox{\hologo{TeX}\kern-.0333emt}%
}
%    \end{macrocode}
%    \end{macro}
%    \begin{macro}{\HoLogoHtml@ConTeXt@narrow}
%    \begin{macrocode}
\def\HoLogoHtml@ConTeXt@narrow#1{%
  \HoLogoCss@ConTeXt@narrow
  \HOLOGO@Span{ConTeXt-narrow}{%
    \HOLOGO@Span{C}{C}%
    on%
    \hologo{TeX}%
    t%
  }%
}
%    \end{macrocode}
%    \end{macro}
%    \begin{macro}{\HoLogoCss@ConTeXt@narrow}
%    \begin{macrocode}
\def\HoLogoCss@ConTeXt@narrow{%
  \Css{%
    span.HoLogo-ConTeXt-narrow span.HoLogo-C{%
      margin-left:-.0333em;%
    }%
  }%
  \Css{%
    span.HoLogo-ConTeXt-narrow span.HoLogo-TeX{%
      margin-left:-.0667em;%
      margin-right:-.0333em;%
    }%
  }%
  \global\let\HoLogoCss@ConTeXt@narrow\relax
}
%    \end{macrocode}
%    \end{macro}
%
%    \begin{macro}{\HoLogo@ConTeXt}
%    \begin{macrocode}
\def\HoLogo@ConTeXt{\HoLogo@ConTeXt@narrow}
%    \end{macrocode}
%    \end{macro}
%    \begin{macro}{\HoLogoHtml@ConTeXt}
%    \begin{macrocode}
\def\HoLogoHtml@ConTeXt{\HoLogoHtml@ConTeXt@narrow}
%    \end{macrocode}
%    \end{macro}
%
% \subsubsection{\hologo{emTeX}}
%
%    \begin{macro}{\HoLogo@emTeX}
%    \begin{macrocode}
\def\HoLogo@emTeX#1{%
  \HOLOGO@mbox{#1{e}{E}m}%
  \HOLOGO@discretionary
  \hologo{TeX}%
}
%    \end{macrocode}
%    \end{macro}
%    \begin{macro}{\HoLogoCs@emTeX}
%    \begin{macrocode}
\def\HoLogoCs@emTeX#1{#1{e}{E}mTeX}%
%    \end{macrocode}
%    \end{macro}
%    \begin{macro}{\HoLogoBkm@emTeX}
%    \begin{macrocode}
\def\HoLogoBkm@emTeX#1{%
  #1{e}{E}m\hologo{TeX}%
}
%    \end{macrocode}
%    \end{macro}
%    \begin{macro}{\HoLogoHtml@emTeX}
%    \begin{macrocode}
\let\HoLogoHtml@emTeX\HoLogo@emTeX
%    \end{macrocode}
%    \end{macro}
%
% \subsubsection{\hologo{ExTeX}}
%
%    \begin{macro}{\HoLogo@ExTeX}
%    The definition is taken from the FAQ of the
%    project \hologo{ExTeX}
%    \cite{ExTeX-FAQ}.
%\begin{quote}
%\begin{verbatim}
%\def\ExTeX{%
%  \textrm{% Logo always with serifs
%    \ensuremath{%
%      \textstyle
%      \varepsilon_{%
%        \kern-0.15em%
%        \mathcal{X}%
%      }%
%    }%
%    \kern-.15em%
%    \TeX
%  }%
%}
%\end{verbatim}
%\end{quote}
%    \begin{macrocode}
\def\HoLogo@ExTeX#1{%
  \HoLogoFont@font{ExTeX}{rm}{%
    \ltx@mbox{%
      \HOLOGO@MathSetup
      $%
        \textstyle
        \varepsilon_{%
          \kern-0.15em%
          \HoLogoFont@font{ExTeX}{sy}{X}%
        }%
      $%
    }%
    \HOLOGO@discretionary
    \kern-.15em%
    \hologo{TeX}%
  }%
}
%    \end{macrocode}
%    \end{macro}
%    \begin{macro}{\HoLogoHtml@ExTeX}
%    \begin{macrocode}
\def\HoLogoHtml@ExTeX#1{%
  \HoLogoCss@ExTeX
  \HoLogoFont@font{ExTeX}{rm}{%
    \HOLOGO@Span{ExTeX}{%
      \ltx@mbox{%
        \HOLOGO@MathSetup
        $\textstyle\varepsilon$%
        \HOLOGO@Span{X}{$\textstyle\chi$}%
        \hologo{TeX}%
      }%
    }%
  }%
}
%    \end{macrocode}
%    \end{macro}
%    \begin{macro}{\HoLogoBkm@ExTeX}
%    \begin{macrocode}
\def\HoLogoBkm@ExTeX#1{%
  \HOLOGO@PdfdocUnicode{#1{e}{E}x}{\textepsilon\textchi}%
  \hologo{TeX}%
}
%    \end{macrocode}
%    \end{macro}
%    \begin{macro}{\HoLogoCss@ExTeX}
%    \begin{macrocode}
\def\HoLogoCss@ExTeX{%
  \Css{%
    span.HoLogo-ExTeX{%
      font-family:serif;%
    }%
  }%
  \Css{%
    span.HoLogo-ExTeX span.HoLogo-TeX{%
      margin-left:-.15em;%
    }%
  }%
  \global\let\HoLogoCss@ExTeX\relax
}
%    \end{macrocode}
%    \end{macro}
%
% \subsubsection{\hologo{MiKTeX}}
%
%    \begin{macro}{\HoLogo@MiKTeX}
%    \begin{macrocode}
\def\HoLogo@MiKTeX#1{%
  \HOLOGO@mbox{MiK}%
  \HOLOGO@discretionary
  \hologo{TeX}%
}
%    \end{macrocode}
%    \end{macro}
%    \begin{macro}{\HoLogoHtml@MiKTeX}
%    \begin{macrocode}
\let\HoLogoHtml@MiKTeX\HoLogo@MiKTeX
%    \end{macrocode}
%    \end{macro}
%
% \subsubsection{\hologo{OzTeX} and friends}
%
%    Source: \hologo{OzTeX} FAQ \cite{OzTeX}:
%    \begin{quote}
%      |\def\OzTeX{O\kern-.03em z\kern-.15em\TeX}|\\
%      (There is no kerning in OzMF, OzMP and OzTtH.)
%    \end{quote}
%
%    \begin{macro}{\HoLogo@OzTeX}
%    \begin{macrocode}
\def\HoLogo@OzTeX#1{%
  O%
  \kern-.03em %
  z%
  \kern-.15em %
  \hologo{TeX}%
}
%    \end{macrocode}
%    \end{macro}
%    \begin{macro}{\HoLogoHtml@OzTeX}
%    \begin{macrocode}
\def\HoLogoHtml@OzTeX#1{%
  \HoLogoCss@OzTeX
  \HOLOGO@Span{OzTeX}{%
    O%
    \HOLOGO@Span{z}{z}%
    \hologo{TeX}%
  }%
}
%    \end{macrocode}
%    \end{macro}
%    \begin{macro}{\HoLogoCss@OzTeX}
%    \begin{macrocode}
\def\HoLogoCss@OzTeX{%
  \Css{%
    span.HoLogo-OzTeX span.HoLogo-z{%
      margin-left:-.03em;%
      margin-right:-.15em;%
    }%
  }%
  \global\let\HoLogoCss@OzTeX\relax
}
%    \end{macrocode}
%    \end{macro}
%
%    \begin{macro}{\HoLogo@OzMF}
%    \begin{macrocode}
\def\HoLogo@OzMF#1{%
  \HOLOGO@mbox{OzMF}%
}
%    \end{macrocode}
%    \end{macro}
%    \begin{macro}{\HoLogo@OzMP}
%    \begin{macrocode}
\def\HoLogo@OzMP#1{%
  \HOLOGO@mbox{OzMP}%
}
%    \end{macrocode}
%    \end{macro}
%    \begin{macro}{\HoLogo@OzTtH}
%    \begin{macrocode}
\def\HoLogo@OzTtH#1{%
  \HOLOGO@mbox{OzTtH}%
}
%    \end{macrocode}
%    \end{macro}
%
% \subsubsection{\hologo{PCTeX}}
%
%    \begin{macro}{\HoLogo@PCTeX}
%    \begin{macrocode}
\def\HoLogo@PCTeX#1{%
  \HOLOGO@mbox{PC}%
  \hologo{TeX}%
}
%    \end{macrocode}
%    \end{macro}
%    \begin{macro}{\HoLogoHtml@PCTeX}
%    \begin{macrocode}
\let\HoLogoHtml@PCTeX\HoLogo@PCTeX
%    \end{macrocode}
%    \end{macro}
%
% \subsubsection{\hologo{PiCTeX}}
%
%    The original definitions from \xfile{pictex.tex} \cite{PiCTeX}:
%\begin{quote}
%\begin{verbatim}
%\def\PiC{%
%  P%
%  \kern-.12em%
%  \lower.5ex\hbox{I}%
%  \kern-.075em%
%  C%
%}
%\def\PiCTeX{%
%  \PiC
%  \kern-.11em%
%  \TeX
%}
%\end{verbatim}
%\end{quote}
%
%    \begin{macro}{\HoLogo@PiC}
%    \begin{macrocode}
\def\HoLogo@PiC#1{%
  P%
  \kern-.12em%
  \lower.5ex\hbox{I}%
  \kern-.075em%
  C%
  \HOLOGO@SpaceFactor
}
%    \end{macrocode}
%    \end{macro}
%    \begin{macro}{\HoLogoHtml@PiC}
%    \begin{macrocode}
\def\HoLogoHtml@PiC#1{%
  \HoLogoCss@PiC
  \HOLOGO@Span{PiC}{%
    P%
    \HOLOGO@Span{i}{I}%
    C%
  }%
}
%    \end{macrocode}
%    \end{macro}
%    \begin{macro}{\HoLogoCss@PiC}
%    \begin{macrocode}
\def\HoLogoCss@PiC{%
  \Css{%
    span.HoLogo-PiC span.HoLogo-i{%
      position:relative;%
      top:.5ex;%
      margin-left:-.12em;%
      margin-right:-.075em;%
      text-decoration:none;%
    }%
  }%
  \global\let\HoLogoCss@PiC\relax
}
%    \end{macrocode}
%    \end{macro}
%
%    \begin{macro}{\HoLogo@PiCTeX}
%    \begin{macrocode}
\def\HoLogo@PiCTeX#1{%
  \hologo{PiC}%
  \HOLOGO@discretionary
  \kern-.11em%
  \hologo{TeX}%
}
%    \end{macrocode}
%    \end{macro}
%    \begin{macro}{\HoLogoHtml@PiCTeX}
%    \begin{macrocode}
\def\HoLogoHtml@PiCTeX#1{%
  \HoLogoCss@PiCTeX
  \HOLOGO@Span{PiCTeX}{%
    \hologo{PiC}%
    \hologo{TeX}%
  }%
}
%    \end{macrocode}
%    \end{macro}
%    \begin{macro}{\HoLogoCss@PiCTeX}
%    \begin{macrocode}
\def\HoLogoCss@PiCTeX{%
  \Css{%
    span.HoLogo-PiCTeX span.HoLogo-PiC{%
      margin-right:-.11em;%
    }%
  }%
  \global\let\HoLogoCss@PiCTeX\relax
}
%    \end{macrocode}
%    \end{macro}
%
% \subsubsection{\hologo{teTeX}}
%
%    \begin{macro}{\HoLogo@teTeX}
%    \begin{macrocode}
\def\HoLogo@teTeX#1{%
  \HOLOGO@mbox{#1{t}{T}e}%
  \HOLOGO@discretionary
  \hologo{TeX}%
}
%    \end{macrocode}
%    \end{macro}
%    \begin{macro}{\HoLogoCs@teTeX}
%    \begin{macrocode}
\def\HoLogoCs@teTeX#1{#1{t}{T}dfTeX}
%    \end{macrocode}
%    \end{macro}
%    \begin{macro}{\HoLogoBkm@teTeX}
%    \begin{macrocode}
\def\HoLogoBkm@teTeX#1{%
  #1{t}{T}e\hologo{TeX}%
}
%    \end{macrocode}
%    \end{macro}
%    \begin{macro}{\HoLogoHtml@teTeX}
%    \begin{macrocode}
\let\HoLogoHtml@teTeX\HoLogo@teTeX
%    \end{macrocode}
%    \end{macro}
%
% \subsubsection{\hologo{TeX4ht}}
%
%    \begin{macro}{\HoLogo@TeX4ht}
%    \begin{macrocode}
\expandafter\def\csname HoLogo@TeX4ht\endcsname#1{%
  \HOLOGO@mbox{\hologo{TeX}4ht}%
}
%    \end{macrocode}
%    \end{macro}
%    \begin{macro}{\HoLogoHtml@TeX4ht}
%    \begin{macrocode}
\expandafter
\let\csname HoLogoHtml@TeX4ht\expandafter\endcsname
\csname HoLogo@TeX4ht\endcsname
%    \end{macrocode}
%    \end{macro}
%
%
% \subsubsection{\hologo{SageTeX}}
%
%    \begin{macro}{\HoLogo@SageTeX}
%    \begin{macrocode}
\def\HoLogo@SageTeX#1{%
  \HOLOGO@mbox{Sage}%
  \HOLOGO@discretionary
  \HOLOGO@NegativeKerning{eT,oT,To}%
  \hologo{TeX}%
}
%    \end{macrocode}
%    \end{macro}
%    \begin{macro}{\HoLogoHtml@SageTeX}
%    \begin{macrocode}
\let\HoLogoHtml@SageTeX\HoLogo@SageTeX
%    \end{macrocode}
%    \end{macro}
%
% \subsection{\hologo{METAFONT} and friends}
%
%    \begin{macro}{\HoLogo@METAFONT}
%    \begin{macrocode}
\def\HoLogo@METAFONT#1{%
  \HoLogoFont@font{METAFONT}{logo}{%
    \HOLOGO@mbox{META}%
    \HOLOGO@discretionary
    \HOLOGO@mbox{FONT}%
  }%
}
%    \end{macrocode}
%    \end{macro}
%
%    \begin{macro}{\HoLogo@METAPOST}
%    \begin{macrocode}
\def\HoLogo@METAPOST#1{%
  \HoLogoFont@font{METAPOST}{logo}{%
    \HOLOGO@mbox{META}%
    \HOLOGO@discretionary
    \HOLOGO@mbox{POST}%
  }%
}
%    \end{macrocode}
%    \end{macro}
%
%    \begin{macro}{\HoLogo@MetaFun}
%    \begin{macrocode}
\def\HoLogo@MetaFun#1{%
  \HOLOGO@mbox{Meta}%
  \HOLOGO@discretionary
  \HOLOGO@mbox{Fun}%
}
%    \end{macrocode}
%    \end{macro}
%
%    \begin{macro}{\HoLogo@MetaPost}
%    \begin{macrocode}
\def\HoLogo@MetaPost#1{%
  \HOLOGO@mbox{Meta}%
  \HOLOGO@discretionary
  \HOLOGO@mbox{Post}%
}
%    \end{macrocode}
%    \end{macro}
%
% \subsection{Others}
%
% \subsubsection{\hologo{biber}}
%
%    \begin{macro}{\HoLogo@biber}
%    \begin{macrocode}
\def\HoLogo@biber#1{%
  \HOLOGO@mbox{#1{b}{B}i}%
  \HOLOGO@discretionary
  \HOLOGO@mbox{ber}%
}
%    \end{macrocode}
%    \end{macro}
%    \begin{macro}{\HoLogoCs@biber}
%    \begin{macrocode}
\def\HoLogoCs@biber#1{#1{b}{B}iber}
%    \end{macrocode}
%    \end{macro}
%    \begin{macro}{\HoLogoBkm@biber}
%    \begin{macrocode}
\def\HoLogoBkm@biber#1{%
  #1{b}{B}iber%
}
%    \end{macrocode}
%    \end{macro}
%    \begin{macro}{\HoLogoHtml@biber}
%    \begin{macrocode}
\let\HoLogoHtml@biber\HoLogo@biber
%    \end{macrocode}
%    \end{macro}
%
% \subsubsection{\hologo{KOMAScript}}
%
%    \begin{macro}{\HoLogo@KOMAScript}
%    The definition for \hologo{KOMAScript} is taken
%    from \hologo{KOMAScript} (\xfile{scrlogo.dtx}, reformatted) \cite{scrlogo}:
%\begin{quote}
%\begin{verbatim}
%\@ifundefined{KOMAScript}{%
%  \DeclareRobustCommand{\KOMAScript}{%
%    \textsf{%
%      K\kern.05em O\kern.05emM\kern.05em A%
%      \kern.1em-\kern.1em %
%      Script%
%    }%
%  }%
%}{}
%\end{verbatim}
%\end{quote}
%    \begin{macrocode}
\def\HoLogo@KOMAScript#1{%
  \HoLogoFont@font{KOMAScript}{sf}{%
    \HOLOGO@mbox{%
      K\kern.05em%
      O\kern.05em%
      M\kern.05em%
      A%
    }%
    \kern.1em%
    \HOLOGO@hyphen
    \kern.1em%
    \HOLOGO@mbox{Script}%
  }%
}
%    \end{macrocode}
%    \end{macro}
%    \begin{macro}{\HoLogoBkm@KOMAScript}
%    \begin{macrocode}
\def\HoLogoBkm@KOMAScript#1{%
  KOMA-Script%
}
%    \end{macrocode}
%    \end{macro}
%    \begin{macro}{\HoLogoHtml@KOMAScript}
%    \begin{macrocode}
\def\HoLogoHtml@KOMAScript#1{%
  \HoLogoCss@KOMAScript
  \HoLogoFont@font{KOMAScript}{sf}{%
    \HOLOGO@Span{KOMAScript}{%
      K%
      \HOLOGO@Span{O}{O}%
      M%
      \HOLOGO@Span{A}{A}%
      \HOLOGO@Span{hyphen}{-}%
      Script%
    }%
  }%
}
%    \end{macrocode}
%    \end{macro}
%    \begin{macro}{\HoLogoCss@KOMAScript}
%    \begin{macrocode}
\def\HoLogoCss@KOMAScript{%
  \Css{%
    span.HoLogo-KOMAScript{%
      font-family:sans-serif;%
    }%
  }%
  \Css{%
    span.HoLogo-KOMAScript span.HoLogo-O{%
      padding-left:.05em;%
      padding-right:.05em;%
    }%
  }%
  \Css{%
    span.HoLogo-KOMAScript span.HoLogo-A{%
      padding-left:.05em;%
    }%
  }%
  \Css{%
    span.HoLogo-KOMAScript span.HoLogo-hyphen{%
      padding-left:.1em;%
      padding-right:.1em;%
    }%
  }%
  \global\let\HoLogoCss@KOMAScript\relax
}
%    \end{macrocode}
%    \end{macro}
%
% \subsubsection{\hologo{LyX}}
%
%    \begin{macro}{\HoLogo@LyX}
%    The definition is taken from the documentation source files
%    of \hologo{LyX}, \xfile{Intro.lyx} \cite{LyX}:
%\begin{quote}
%\begin{verbatim}
%\def\LyX{%
%  \texorpdfstring{%
%    L\kern-.1667em\lower.25em\hbox{Y}\kern-.125emX\@%
%  }{%
%    LyX%
%  }%
%}
%\end{verbatim}
%\end{quote}
%    \begin{macrocode}
\def\HoLogo@LyX#1{%
  L%
  \kern-.1667em%
  \lower.25em\hbox{Y}%
  \kern-.125em%
  X%
  \HOLOGO@SpaceFactor
}
%    \end{macrocode}
%    \end{macro}
%    \begin{macro}{\HoLogoHtml@LyX}
%    \begin{macrocode}
\def\HoLogoHtml@LyX#1{%
  \HoLogoCss@LyX
  \HOLOGO@Span{LyX}{%
    L%
    \HOLOGO@Span{y}{Y}%
    X%
  }%
}
%    \end{macrocode}
%    \end{macro}
%    \begin{macro}{\HoLogoCss@LyX}
%    \begin{macrocode}
\def\HoLogoCss@LyX{%
  \Css{%
    span.HoLogo-LyX span.HoLogo-y{%
      position:relative;%
      top:.25em;%
      margin-left:-.1667em;%
      margin-right:-.125em;%
      text-decoration:none;%
    }%
  }%
  \global\let\HoLogoCss@LyX\relax
}
%    \end{macrocode}
%    \end{macro}
%
% \subsubsection{\hologo{NTS}}
%
%    \begin{macro}{\HoLogo@NTS}
%    Definition for \hologo{NTS} can be found in
%    package \xpackage{etex\textunderscore man} for the \hologo{eTeX} manual \cite{etexman}
%    and in package \xpackage{dtklogos} \cite{dtklogos}:
%\begin{quote}
%\begin{verbatim}
%\def\NTS{%
%  \leavevmode
%  \hbox{%
%    $%
%      \cal N%
%      \kern-0.35em%
%      \lower0.5ex\hbox{$\cal T$}%
%      \kern-0.2em%
%      S%
%    $%
%  }%
%}
%\end{verbatim}
%\end{quote}
%    \begin{macrocode}
\def\HoLogo@NTS#1{%
  \HoLogoFont@font{NTS}{sy}{%
    N\/%
    \kern-.35em%
    \lower.5ex\hbox{T\/}%
    \kern-.2em%
    S\/%
  }%
  \HOLOGO@SpaceFactor
}
%    \end{macrocode}
%    \end{macro}
%
% \subsubsection{\Hologo{TTH} (\hologo{TeX} to HTML translator)}
%
%    Source: \url{http://hutchinson.belmont.ma.us/tth/}
%    In the HTML source the second `T' is printed as subscript.
%\begin{quote}
%\begin{verbatim}
%T<sub>T</sub>H
%\end{verbatim}
%\end{quote}
%    \begin{macro}{\HoLogo@TTH}
%    \begin{macrocode}
\def\HoLogo@TTH#1{%
  \ltx@mbox{%
    T\HOLOGO@SubScript{T}H%
  }%
  \HOLOGO@SpaceFactor
}
%    \end{macrocode}
%    \end{macro}
%
%    \begin{macro}{\HoLogoHtml@TTH}
%    \begin{macrocode}
\def\HoLogoHtml@TTH#1{%
  T\HCode{<sub>}T\HCode{</sub>}H%
}
%    \end{macrocode}
%    \end{macro}
%
% \subsubsection{\Hologo{HanTheThanh}}
%
%    Partial source: Package \xpackage{dtklogos}.
%    The double accent is U+1EBF (latin small letter e with circumflex
%    and acute).
%    \begin{macro}{\HoLogo@HanTheThanh}
%    \begin{macrocode}
\def\HoLogo@HanTheThanh#1{%
  \ltx@mbox{H\`an}%
  \HOLOGO@space
  \ltx@mbox{%
    Th%
    \HOLOGO@IfCharExists{"1EBF}{%
      \char"1EBF\relax
    }{%
      \^e\hbox to 0pt{\hss\raise .5ex\hbox{\'{}}}%
    }%
  }%
  \HOLOGO@space
  \ltx@mbox{Th\`anh}%
}
%    \end{macrocode}
%    \end{macro}
%    \begin{macro}{\HoLogoBkm@HanTheThanh}
%    \begin{macrocode}
\def\HoLogoBkm@HanTheThanh#1{%
  H\`an %
  Th\HOLOGO@PdfdocUnicode{\^e}{\9036\277} %
  Th\`anh%
}
%    \end{macrocode}
%    \end{macro}
%    \begin{macro}{\HoLogoHtml@HanTheThanh}
%    \begin{macrocode}
\def\HoLogoHtml@HanTheThanh#1{%
  H\`an %
  Th\HCode{&\ltx@hashchar x1ebf;} %
  Th\`anh%
}
%    \end{macrocode}
%    \end{macro}
%
% \subsection{Driver detection}
%
%    \begin{macrocode}
\HOLOGO@IfExists\InputIfFileExists{%
  \InputIfFileExists{hologo.cfg}{}{}%
}{%
  \ltx@IfUndefined{pdf@filesize}{%
    \def\HOLOGO@InputIfExists{%
      \openin\HOLOGO@temp=hologo.cfg\relax
      \ifeof\HOLOGO@temp
        \closein\HOLOGO@temp
      \else
        \closein\HOLOGO@temp
        \begingroup
          \def\x{LaTeX2e}%
        \expandafter\endgroup
        \ifx\fmtname\x
          \input{hologo.cfg}%
        \else
          \input hologo.cfg\relax
        \fi
      \fi
    }%
    \ltx@IfUndefined{newread}{%
      \chardef\HOLOGO@temp=15 %
      \def\HOLOGO@CheckRead{%
        \ifeof\HOLOGO@temp
          \HOLOGO@InputIfExists
        \else
          \ifcase\HOLOGO@temp
            \@PackageWarningNoLine{hologo}{%
              Configuration file ignored, because\MessageBreak
              a free read register could not be found%
            }%
          \else
            \begingroup
              \count\ltx@cclv=\HOLOGO@temp
              \advance\ltx@cclv by \ltx@minusone
              \edef\x{\endgroup
                \chardef\noexpand\HOLOGO@temp=\the\count\ltx@cclv
                \relax
              }%
            \x
          \fi
        \fi
      }%
    }{%
      \csname newread\endcsname\HOLOGO@temp
      \HOLOGO@InputIfExists
    }%
  }{%
    \edef\HOLOGO@temp{\pdf@filesize{hologo.cfg}}%
    \ifx\HOLOGO@temp\ltx@empty
    \else
      \ifnum\HOLOGO@temp>0 %
        \begingroup
          \def\x{LaTeX2e}%
        \expandafter\endgroup
        \ifx\fmtname\x
          \input{hologo.cfg}%
        \else
          \input hologo.cfg\relax
        \fi
      \else
        \@PackageInfoNoLine{hologo}{%
          Empty configuration file `hologo.cfg' ignored%
        }%
      \fi
    \fi
  }%
}
%    \end{macrocode}
%
%    \begin{macrocode}
\def\HOLOGO@temp#1#2{%
  \kv@define@key{HoLogoDriver}{#1}[]{%
    \begingroup
      \def\HOLOGO@temp{##1}%
      \ltx@onelevel@sanitize\HOLOGO@temp
      \ifx\HOLOGO@temp\ltx@empty
      \else
        \@PackageError{hologo}{%
          Value (\HOLOGO@temp) not permitted for option `#1'%
        }%
        \@ehc
      \fi
    \endgroup
    \def\hologoDriver{#2}%
  }%
}%
\def\HOLOGO@@temp#1#2{%
  \ifx\kv@value\relax
    \HOLOGO@temp{#1}{#1}%
  \else
    \HOLOGO@temp{#1}{#2}%
  \fi
}%
\kv@parse@normalized{%
  pdftex,%
  luatex=pdftex,%
  dvipdfm,%
  dvipdfmx=dvipdfm,%
  dvips,%
  dvipsone=dvips,%
  xdvi=dvips,%
  xetex,%
  vtex,%
}\HOLOGO@@temp
%    \end{macrocode}
%
%    \begin{macrocode}
\kv@define@key{HoLogoDriver}{driverfallback}{%
  \def\HOLOGO@DriverFallback{#1}%
}
%    \end{macrocode}
%
%    \begin{macro}{\HOLOGO@DriverFallback}
%    \begin{macrocode}
\def\HOLOGO@DriverFallback{dvips}
%    \end{macrocode}
%    \end{macro}
%
%    \begin{macro}{\hologoDriverSetup}
%    \begin{macrocode}
\def\hologoDriverSetup{%
  \let\hologoDriver\ltx@undefined
  \HOLOGO@DriverSetup
}
%    \end{macrocode}
%    \end{macro}
%
%    \begin{macro}{\HOLOGO@DriverSetup}
%    \begin{macrocode}
\def\HOLOGO@DriverSetup#1{%
  \kvsetkeys{HoLogoDriver}{#1}%
  \HOLOGO@CheckDriver
  \ltx@ifundefined{hologoDriver}{%
    \begingroup
    \edef\x{\endgroup
      \noexpand\kvsetkeys{HoLogoDriver}{\HOLOGO@DriverFallback}%
    }\x
  }{}%
  \@PackageInfoNoLine{hologo}{Using driver `\hologoDriver'}%
}
%    \end{macrocode}
%    \end{macro}
%
%    \begin{macro}{\HOLOGO@CheckDriver}
%    \begin{macrocode}
\def\HOLOGO@CheckDriver{%
  \ifpdf
    \def\hologoDriver{pdftex}%
    \let\HOLOGO@pdfliteral\pdfliteral
    \ifluatex
      \ifx\pdfextension\@undefined\else
        \protected\def\pdfliteral{\pdfextension literal}%
        \let\HOLOGO@pdfliteral\pdfliteral
      \fi
      \ltx@IfUndefined{HOLOGO@pdfliteral}{%
        \ifnum\luatexversion<36 %
        \else
          \begingroup
            \let\HOLOGO@temp\endgroup
            \ifcase0%
                \directlua{%
                  if tex.enableprimitives then %
                    tex.enableprimitives('HOLOGO@', {'pdfliteral'})%
                  else %
                    tex.print('1')%
                  end%
                }%
                \ifx\HOLOGO@pdfliteral\@undefined 1\fi%
                \relax%
              \endgroup
              \let\HOLOGO@temp\relax
              \global\let\HOLOGO@pdfliteral\HOLOGO@pdfliteral
            \fi%
          \HOLOGO@temp
        \fi
      }{}%
    \fi
    \ltx@IfUndefined{HOLOGO@pdfliteral}{%
      \@PackageWarningNoLine{hologo}{%
        Cannot find \string\pdfliteral
      }%
    }{}%
  \else
    \ifxetex
      \def\hologoDriver{xetex}%
    \else
      \ifvtex
        \def\hologoDriver{vtex}%
      \fi
    \fi
  \fi
}
%    \end{macrocode}
%    \end{macro}
%
%    \begin{macro}{\HOLOGO@WarningUnsupportedDriver}
%    \begin{macrocode}
\def\HOLOGO@WarningUnsupportedDriver#1{%
  \@PackageWarningNoLine{hologo}{%
    Logo `#1' needs driver specific macros,\MessageBreak
    but driver `\hologoDriver' is not supported.\MessageBreak
    Use a different driver or\MessageBreak
    load package `graphics' or `pgf'%
  }%
}
%    \end{macrocode}
%    \end{macro}
%
% \subsubsection{Reflect box macros}
%
%    Skip driver part if not needed.
%    \begin{macrocode}
\ltx@IfUndefined{reflectbox}{}{%
  \ltx@IfUndefined{rotatebox}{}{%
    \HOLOGO@AtEnd
  }%
}
\ltx@IfUndefined{pgftext}{}{%
  \HOLOGO@AtEnd
}
\ltx@IfUndefined{psscalebox}{}{%
  \HOLOGO@AtEnd
}
%    \end{macrocode}
%
%    \begin{macrocode}
\def\HOLOGO@temp{LaTeX2e}
\ifx\fmtname\HOLOGO@temp
  \RequirePackage{kvoptions}[2011/06/30]%
  \ProcessKeyvalOptions{HoLogoDriver}%
\fi
\HOLOGO@DriverSetup{}
%    \end{macrocode}
%
%    \begin{macro}{\HOLOGO@ReflectBox}
%    \begin{macrocode}
\def\HOLOGO@ReflectBox#1{%
  \begingroup
    \setbox\ltx@zero\hbox{\begingroup#1\endgroup}%
    \setbox\ltx@two\hbox{%
      \kern\wd\ltx@zero
      \csname HOLOGO@ScaleBox@\hologoDriver\endcsname{-1}{1}{%
        \hbox to 0pt{\copy\ltx@zero\hss}%
      }%
    }%
    \wd\ltx@two=\wd\ltx@zero
    \box\ltx@two
  \endgroup
}
%    \end{macrocode}
%    \end{macro}
%
%    \begin{macro}{\HOLOGO@PointReflectBox}
%    \begin{macrocode}
\def\HOLOGO@PointReflectBox#1{%
  \begingroup
    \setbox\ltx@zero\hbox{\begingroup#1\endgroup}%
    \setbox\ltx@two\hbox{%
      \kern\wd\ltx@zero
      \raise\ht\ltx@zero\hbox{%
        \csname HOLOGO@ScaleBox@\hologoDriver\endcsname{-1}{-1}{%
          \hbox to 0pt{\copy\ltx@zero\hss}%
        }%
      }%
    }%
    \wd\ltx@two=\wd\ltx@zero
    \box\ltx@two
  \endgroup
}
%    \end{macrocode}
%    \end{macro}
%
%    We must define all variants because of dynamic driver setup.
%    \begin{macrocode}
\def\HOLOGO@temp#1#2{#2}
%    \end{macrocode}
%
%    \begin{macro}{\HOLOGO@ScaleBox@pdftex}
%    \begin{macrocode}
\HOLOGO@temp{pdftex}{%
  \def\HOLOGO@ScaleBox@pdftex#1#2#3{%
    \HOLOGO@pdfliteral{%
      q #1 0 0 #2 0 0 cm%
    }%
    #3%
    \HOLOGO@pdfliteral{%
      Q%
    }%
  }%
}
%    \end{macrocode}
%    \end{macro}
%    \begin{macro}{\HOLOGO@ScaleBox@dvips}
%    \begin{macrocode}
\HOLOGO@temp{dvips}{%
  \def\HOLOGO@ScaleBox@dvips#1#2#3{%
    \special{ps:%
      gsave %
      currentpoint %
      currentpoint translate %
      #1 #2 scale %
      neg exch neg exch translate%
    }%
    #3%
    \special{ps:%
      currentpoint %
      grestore %
      moveto%
    }%
  }%
}
%    \end{macrocode}
%    \end{macro}
%    \begin{macro}{\HOLOGO@ScaleBox@dvipdfm}
%    \begin{macrocode}
\HOLOGO@temp{dvipdfm}{%
  \let\HOLOGO@ScaleBox@dvipdfm\HOLOGO@ScaleBox@dvips
}
%    \end{macrocode}
%    \end{macro}
%    Since \hologo{XeTeX} v0.6.
%    \begin{macro}{\HOLOGO@ScaleBox@xetex}
%    \begin{macrocode}
\HOLOGO@temp{xetex}{%
  \def\HOLOGO@ScaleBox@xetex#1#2#3{%
    \special{x:gsave}%
    \special{x:scale #1 #2}%
    #3%
    \special{x:grestore}%
  }%
}
%    \end{macrocode}
%    \end{macro}
%    \begin{macro}{\HOLOGO@ScaleBox@vtex}
%    \begin{macrocode}
\HOLOGO@temp{vtex}{%
  \def\HOLOGO@ScaleBox@vtex#1#2#3{%
    \special{r(#1,0,0,#2,0,0}%
    #3%
    \special{r)}%
  }%
}
%    \end{macrocode}
%    \end{macro}
%
%    \begin{macrocode}
\HOLOGO@AtEnd%
%</package>
%    \end{macrocode}
%
% \section{Test}
%
% \subsection{Catcode checks for loading}
%
%    \begin{macrocode}
%<*test1>
%    \end{macrocode}
%    \begin{macrocode}
\catcode`\{=1 %
\catcode`\}=2 %
\catcode`\#=6 %
\catcode`\@=11 %
\expandafter\ifx\csname count@\endcsname\relax
  \countdef\count@=255 %
\fi
\expandafter\ifx\csname @gobble\endcsname\relax
  \long\def\@gobble#1{}%
\fi
\expandafter\ifx\csname @firstofone\endcsname\relax
  \long\def\@firstofone#1{#1}%
\fi
\expandafter\ifx\csname loop\endcsname\relax
  \expandafter\@firstofone
\else
  \expandafter\@gobble
\fi
{%
  \def\loop#1\repeat{%
    \def\body{#1}%
    \iterate
  }%
  \def\iterate{%
    \body
      \let\next\iterate
    \else
      \let\next\relax
    \fi
    \next
  }%
  \let\repeat=\fi
}%
\def\RestoreCatcodes{}
\count@=0 %
\loop
  \edef\RestoreCatcodes{%
    \RestoreCatcodes
    \catcode\the\count@=\the\catcode\count@\relax
  }%
\ifnum\count@<255 %
  \advance\count@ 1 %
\repeat

\def\RangeCatcodeInvalid#1#2{%
  \count@=#1\relax
  \loop
    \catcode\count@=15 %
  \ifnum\count@<#2\relax
    \advance\count@ 1 %
  \repeat
}
\def\RangeCatcodeCheck#1#2#3{%
  \count@=#1\relax
  \loop
    \ifnum#3=\catcode\count@
    \else
      \errmessage{%
        Character \the\count@\space
        with wrong catcode \the\catcode\count@\space
        instead of \number#3%
      }%
    \fi
  \ifnum\count@<#2\relax
    \advance\count@ 1 %
  \repeat
}
\def\space{ }
\expandafter\ifx\csname LoadCommand\endcsname\relax
  \def\LoadCommand{\input hologo.sty\relax}%
\fi
\def\Test{%
  \RangeCatcodeInvalid{0}{47}%
  \RangeCatcodeInvalid{58}{64}%
  \RangeCatcodeInvalid{91}{96}%
  \RangeCatcodeInvalid{123}{255}%
  \catcode`\@=12 %
  \catcode`\\=0 %
  \catcode`\%=14 %
  \LoadCommand
  \RangeCatcodeCheck{0}{36}{15}%
  \RangeCatcodeCheck{37}{37}{14}%
  \RangeCatcodeCheck{38}{47}{15}%
  \RangeCatcodeCheck{48}{57}{12}%
  \RangeCatcodeCheck{58}{63}{15}%
  \RangeCatcodeCheck{64}{64}{12}%
  \RangeCatcodeCheck{65}{90}{11}%
  \RangeCatcodeCheck{91}{91}{15}%
  \RangeCatcodeCheck{92}{92}{0}%
  \RangeCatcodeCheck{93}{96}{15}%
  \RangeCatcodeCheck{97}{122}{11}%
  \RangeCatcodeCheck{123}{255}{15}%
  \RestoreCatcodes
}
\Test
\csname @@end\endcsname
\end
%    \end{macrocode}
%    \begin{macrocode}
%</test1>
%    \end{macrocode}
%
% \subsection{Spacefactor}
%
%    The space factor must be 1000 after a logo. If it is greater 1000
%    then the following space is a space after a sentence closing point.
%    If the space factor is smaller 1000 then an immediate following
%    dot is interpreted as abbreviation, not sentence closing point.
%
%    \begin{macrocode}
%<*test-spacefactor>
\NeedsTeXFormat{LaTeX2e}
\documentclass{article}
\usepackage{hologo}[2016/05/12]
\usepackage{kvsetkeys}
\usepackage{qstest}
\IncludeTests{*}
\LogTests{log}{*}{*}
\begin{document}
\begin{qstest}{spacefactor}{spacefactor}
\newcommand*{\Test}[1]{%
  \sbox0{%
    \hologo{#1}%
    \Expect*{1000 (#1)}*{\the\spacefactor\space(#1)}%
  }%
}%
\makeatletter
\def\TestList{}
\def\hologoEntry#1#2#3{%
  \edef\TestList{%
    \ifx\TestList\@empty
    \else
      \TestList,%
    \fi
    #1%
    \ifx\\#2\\%
    \else
      ={variant=#2}%
    \fi
  }%
}
\hologoList
\expandafter\kv@parse@normalized\expandafter{%
  \TestList
}{%
  \begingroup
    \let\@logo=\kv@key
    \ifx\kv@value\relax
    \else
      \expandafter\hologoLogoSetup\expandafter\@logo\expandafter{%
        \kv@value
      }%
    \fi
    \Test\@logo
  \endgroup
  \@gobbletwo
}
\end{qstest}
\end{document}
%</test-spacefactor>
%    \end{macrocode}
%
% \subsection{Complete list}
%
%    \begin{macrocode}
%<*test-list>
\NeedsTeXFormat{LaTeX2e}
\documentclass[12pt,a4paper]{article}
\usepackage{hologo}[2016/05/12]
\usepackage[T1]{fontenc}
\usepackage{lmodern}
\usepackage{parskip}
\usepackage[unicode]{hyperref}[2011/09/28]
\usepackage{bookmark}[2011/09/19]
\bookmarksetup{%
  numbered,%
  open,%
  openlevel=2,%
}
\renewcommand*{\contentsname}{List of logos}
\begin{document}
\tableofcontents
\def\TestFont#1#2#3#4#5#6{%
  \begingroup
    \usefont{#3}{#4}{#5}{#6}%
    \HologoVariant{#1}{#2}/\hologoVariant{#1}{#2}%
    \quad
    \begingroup\scriptsize\hologoVariant{#1}{#2}\endgroup
    \quad
  \endgroup
  (#3/#4/#5/#6)%
  \par
}
\makeatletter
\def\hologoEntry#1#2#3{%
  \section{%
    \HologoVariant{#1}{#2}/\hologoVariant{#1}{#2} %
    {[#1\ifx\\#2\\\else\space(#2)\fi]}% hash-ok
  }% braces around [] because of bug in tex4ht
  \begingroup
    \hypersetup{unicode=false}%
    \bookmark[%
      dest=\@currentHref,%
      rellevel=1,%
      keeplevel,%
    ]{%
      \HologoVariant{#1}{#2}/\hologoVariant{#1}{#2} %
      (PDFDocEncoding)%
    }%
  \endgroup
  \TestFont{#1}{#2}{OT1}{cmr}{m}{n}%
  \TestFont{#1}{#2}{OT1}{cmss}{m}{n}%
  \TestFont{#1}{#2}{OT1}{cmr}{b}{n}%
  \TestFont{#1}{#2}{OT1}{cmr}{m}{it}%
  \TestFont{#1}{#2}{OT1}{cmtt}{m}{n}%
  \TestFont{#1}{#2}{T1}{lmr}{m}{n}%
  \TestFont{#1}{#2}{T1}{lmss}{m}{n}%
  \TestFont{#1}{#2}{T1}{lmr}{b}{n}%
  \TestFont{#1}{#2}{T1}{lmr}{m}{it}%
  \TestFont{#1}{#2}{T1}{lmtt}{m}{n}%
  \TestFont{#1}{#2}{T1}{lmvtt}{m}{n}%
  \TestFont{#1}{#2}{T1}{qtm}{m}{n}%
  \TestFont{#1}{#2}{T1}{qhv}{m}{n}%
  \TestFont{#1}{#2}{T1}{qtm}{b}{n}%
  \TestFont{#1}{#2}{T1}{qtm}{m}{it}%
  \TestFont{#1}{#2}{T1}{qcr}{m}{n}%
  \newpage
}
\makeatother
\hologoList
\end{document}
%</test-list>
%    \end{macrocode}
%
% \section{Installation}
%
% \subsection{Download}
%
% \paragraph{Package.} This package is available on
% CTAN\footnote{\url{ftp://ftp.ctan.org/tex-archive/}}:
% \begin{description}
% \item[\CTAN{macros/latex/contrib/oberdiek/hologo.dtx}] The source file.
% \item[\CTAN{macros/latex/contrib/oberdiek/hologo.pdf}] Documentation.
% \end{description}
%
%
% \paragraph{Bundle.} All the packages of the bundle `oberdiek'
% are also available in a TDS compliant ZIP archive. There
% the packages are already unpacked and the documentation files
% are generated. The files and directories obey the TDS standard.
% \begin{description}
% \item[\CTAN{install/macros/latex/contrib/oberdiek.tds.zip}]
% \end{description}
% \emph{TDS} refers to the standard ``A Directory Structure
% for \TeX\ Files'' (\CTAN{tds/tds.pdf}). Directories
% with \xfile{texmf} in their name are usually organized this way.
%
% \subsection{Bundle installation}
%
% \paragraph{Unpacking.} Unpack the \xfile{oberdiek.tds.zip} in the
% TDS tree (also known as \xfile{texmf} tree) of your choice.
% Example (linux):
% \begin{quote}
%   |unzip oberdiek.tds.zip -d ~/texmf|
% \end{quote}
%
% \paragraph{Script installation.}
% Check the directory \xfile{TDS:scripts/oberdiek/} for
% scripts that need further installation steps.
% Package \xpackage{attachfile2} comes with the Perl script
% \xfile{pdfatfi.pl} that should be installed in such a way
% that it can be called as \texttt{pdfatfi}.
% Example (linux):
% \begin{quote}
%   |chmod +x scripts/oberdiek/pdfatfi.pl|\\
%   |cp scripts/oberdiek/pdfatfi.pl /usr/local/bin/|
% \end{quote}
%
% \subsection{Package installation}
%
% \paragraph{Unpacking.} The \xfile{.dtx} file is a self-extracting
% \docstrip\ archive. The files are extracted by running the
% \xfile{.dtx} through \plainTeX:
% \begin{quote}
%   \verb|tex hologo.dtx|
% \end{quote}
%
% \paragraph{TDS.} Now the different files must be moved into
% the different directories in your installation TDS tree
% (also known as \xfile{texmf} tree):
% \begin{quote}
% \def\t{^^A
% \begin{tabular}{@{}>{\ttfamily}l@{ $\rightarrow$ }>{\ttfamily}l@{}}
%   hologo.sty & tex/generic/oberdiek/hologo.sty\\
%   hologo.pdf & doc/latex/oberdiek/hologo.pdf\\
%   example/hologo-example.tex & doc/latex/oberdiek/example/hologo-example.tex\\
%   test/hologo-test1.tex & doc/latex/oberdiek/test/hologo-test1.tex\\
%   test/hologo-test-spacefactor.tex & doc/latex/oberdiek/test/hologo-test-spacefactor.tex\\
%   test/hologo-test-list.tex & doc/latex/oberdiek/test/hologo-test-list.tex\\
%   hologo.dtx & source/latex/oberdiek/hologo.dtx\\
% \end{tabular}^^A
% }^^A
% \sbox0{\t}^^A
% \ifdim\wd0>\linewidth
%   \begingroup
%     \advance\linewidth by\leftmargin
%     \advance\linewidth by\rightmargin
%   \edef\x{\endgroup
%     \def\noexpand\lw{\the\linewidth}^^A
%   }\x
%   \def\lwbox{^^A
%     \leavevmode
%     \hbox to \linewidth{^^A
%       \kern-\leftmargin\relax
%       \hss
%       \usebox0
%       \hss
%       \kern-\rightmargin\relax
%     }^^A
%   }^^A
%   \ifdim\wd0>\lw
%     \sbox0{\small\t}^^A
%     \ifdim\wd0>\linewidth
%       \ifdim\wd0>\lw
%         \sbox0{\footnotesize\t}^^A
%         \ifdim\wd0>\linewidth
%           \ifdim\wd0>\lw
%             \sbox0{\scriptsize\t}^^A
%             \ifdim\wd0>\linewidth
%               \ifdim\wd0>\lw
%                 \sbox0{\tiny\t}^^A
%                 \ifdim\wd0>\linewidth
%                   \lwbox
%                 \else
%                   \usebox0
%                 \fi
%               \else
%                 \lwbox
%               \fi
%             \else
%               \usebox0
%             \fi
%           \else
%             \lwbox
%           \fi
%         \else
%           \usebox0
%         \fi
%       \else
%         \lwbox
%       \fi
%     \else
%       \usebox0
%     \fi
%   \else
%     \lwbox
%   \fi
% \else
%   \usebox0
% \fi
% \end{quote}
% If you have a \xfile{docstrip.cfg} that configures and enables \docstrip's
% TDS installing feature, then some files can already be in the right
% place, see the documentation of \docstrip.
%
% \subsection{Refresh file name databases}
%
% If your \TeX~distribution
% (\teTeX, \mikTeX, \dots) relies on file name databases, you must refresh
% these. For example, \teTeX\ users run \verb|texhash| or
% \verb|mktexlsr|.
%
% \subsection{Some details for the interested}
%
% \paragraph{Attached source.}
%
% The PDF documentation on CTAN also includes the
% \xfile{.dtx} source file. It can be extracted by
% AcrobatReader 6 or higher. Another option is \textsf{pdftk},
% e.g. unpack the file into the current directory:
% \begin{quote}
%   \verb|pdftk hologo.pdf unpack_files output .|
% \end{quote}
%
% \paragraph{Unpacking with \LaTeX.}
% The \xfile{.dtx} chooses its action depending on the format:
% \begin{description}
% \item[\plainTeX:] Run \docstrip\ and extract the files.
% \item[\LaTeX:] Generate the documentation.
% \end{description}
% If you insist on using \LaTeX\ for \docstrip\ (really,
% \docstrip\ does not need \LaTeX), then inform the autodetect routine
% about your intention:
% \begin{quote}
%   \verb|latex \let\install=y\input{hologo.dtx}|
% \end{quote}
% Do not forget to quote the argument according to the demands
% of your shell.
%
% \paragraph{Generating the documentation.}
% You can use both the \xfile{.dtx} or the \xfile{.drv} to generate
% the documentation. The process can be configured by the
% configuration file \xfile{ltxdoc.cfg}. For instance, put this
% line into this file, if you want to have A4 as paper format:
% \begin{quote}
%   \verb|\PassOptionsToClass{a4paper}{article}|
% \end{quote}
% An example follows how to generate the
% documentation with pdf\LaTeX:
% \begin{quote}
%\begin{verbatim}
%pdflatex hologo.dtx
%makeindex -s gind.ist hologo.idx
%pdflatex hologo.dtx
%makeindex -s gind.ist hologo.idx
%pdflatex hologo.dtx
%\end{verbatim}
% \end{quote}
%
% \section{Catalogue}
%
% The following XML file can be used as source for the
% \href{http://mirror.ctan.org/help/Catalogue/catalogue.html}{\TeX\ Catalogue}.
% The elements \texttt{caption} and \texttt{description} are imported
% from the original XML file from the Catalogue.
% The name of the XML file in the Catalogue is \xfile{hologo.xml}.
%    \begin{macrocode}
%<*catalogue>
<?xml version='1.0' encoding='us-ascii'?>
<!DOCTYPE entry SYSTEM 'catalogue.dtd'>
<entry datestamp='$Date$' modifier='$Author$' id='hologo'>
  <name>hologo</name>
  <caption>A collection of logos with bookmark support.</caption>
  <authorref id='auth:oberdiek'/>
  <copyright owner='Heiko Oberdiek' year='2010-2012'/>
  <license type='lppl1.3'/>
  <version number='1.10'/>
  <description>
    The package defines a single command <tt>\hologo</tt>, whose
    argument is the usual case-confused ASCII version of the logo.
    The command is bookmark-enabled, so that every logo becomes
    available in bookmarks without further work.
    <p/>
    The package is part of the <xref refid='oberdiek'>oberdiek</xref>
    bundle.
  </description>
  <documentation details='Package documentation'
      href='ctan:/macros/latex/contrib/oberdiek/hologo.pdf'/>
  <ctan file='true' path='/macros/latex/contrib/oberdiek/hologo.dtx'/>
  <miktex location='oberdiek'/>
  <texlive location='oberdiek'/>
  <install path='/macros/latex/contrib/oberdiek/oberdiek.tds.zip'/>
</entry>
%</catalogue>
%    \end{macrocode}
%
% \begin{thebibliography}{9}
% \raggedright
%
% \bibitem{btxdoc}
% Oren Patashnik,
% \textit{\hologo{BibTeX}ing},
% 1988-02-08.\\
% \CTAN{biblio/bibtex/base/}
%
% \bibitem{dtklogos}
% Gerd Neugebauer, DANTE,
% \textit{Package \xpackage{dtklogos}},
% 2011-04-25.\\
% \CTAN{usergrps/dante/dtk/dtklogos.sty}
%
% \bibitem{etexman}
% The \hologo{NTS} Team,
% \textit{The \hologo{eTeX} manual},
% 1998-02.\\
% \CTAN{systems/e-tex/v2/doc/}
%
% \bibitem{ExTeX-FAQ}
% The \hologo{ExTeX} group,
% \textit{\hologo{ExTeX}: FAQ -- How is \hologo{ExTeX} typeset?},
% 2007-04-14.\\
% \url{http://www.extex.org/documentation/faq.html}
%
% \bibitem{LyX}
% %@MISC{ LyX,
% %  title = {{LyX 2.0.0 -- The Document Processor [Computer software and manual]}},
% %  author = {{The LyX Team}},
% %  howpublished = {Internet: http://www.lyx.org},
% %  year = {2011-05-08},
% %  note = {Retrieved May 10, 2011, from http://www.lyx.org},
% %  url = {http://www.lyx.org/}
% %}
% The \hologo{LyX} Team,
% \textit{\hologo{LyX} -- The Document Processor},
% 2011-05-08.\\
% \url{http://www.lyx.org/}
%
% \bibitem{OzTeX}
% Andrew Trevorrow,
% \hologo{OzTeX} FAQ: What is the correct way to typeset ``\hologo{OzTeX}''?,
% 2011-09-15 (visited).
% \url{http://www.trevorrow.com/oztex/ozfaq.html#oztex-logo}
%
% \bibitem{PiCTeX}
% Michael Wichura,
% \textit{The \hologo{PiCTeX} macro package},
% 1987-09-21.
% \CTAN{graphics/pictex/}
%
% \bibitem{scrlogo}
% Markus Kohm,
% \textit{\hologo{KOMAScript} Datei \xfile{scrlogo.dtx}},
% 2009-01-30.\\
% \CTAN{install/macros/latex/contrib/komascript.tds.zip}
%
% \end{thebibliography}
%
% \begin{History}
%   \begin{Version}{2010/04/08 v1.0}
%   \item
%     The first version.
%   \end{Version}
%   \begin{Version}{2010/04/16 v1.1}
%   \item
%     \cs{Hologo} added for support of logos at start of a sentence.
%   \item
%     \cs{hologoSetup} and \cs{hologoLogoSetup} added.
%   \item
%     Options \xoption{break}, \xoption{hyphenbreak}, \xoption{spacebreak}
%     added.
%   \item
%     Variant support added by option \xoption{variant}.
%   \end{Version}
%   \begin{Version}{2010/04/24 v1.2}
%   \item
%     \hologo{LaTeX3} added.
%   \item
%     \hologo{VTeX} added.
%   \end{Version}
%   \begin{Version}{2010/11/21 v1.3}
%   \item
%     \hologo{iniTeX}, \hologo{virTeX} added.
%   \end{Version}
%   \begin{Version}{2011/03/25 v1.4}
%   \item
%     \hologo{ConTeXt} with variants added.
%   \item
%     Option \xoption{discretionarybreak} added as refinement for
%     option \xoption{break}.
%   \end{Version}
%   \begin{Version}{2011/04/21 v1.5}
%   \item
%     Wrong TDS directory for test files fixed.
%   \end{Version}
%   \begin{Version}{2011/10/01 v1.6}
%   \item
%     Support for package \xpackage{tex4ht} added.
%   \item
%     Support for \cs{csname} added if \cs{ifincsname} is available.
%   \item
%     New logos:
%     \hologo{(La)TeX},
%     \hologo{biber},
%     \hologo{BibTeX} (\xoption{sc}, \xoption{sf}),
%     \hologo{emTeX},
%     \hologo{ExTeX},
%     \hologo{KOMAScript},
%     \hologo{La},
%     \hologo{LyX},
%     \hologo{MiKTeX},
%     \hologo{NTS},
%     \hologo{OzMF},
%     \hologo{OzMP},
%     \hologo{OzTeX},
%     \hologo{OzTtH},
%     \hologo{PCTeX},
%     \hologo{PiC},
%     \hologo{PiCTeX},
%     \hologo{METAFONT},
%     \hologo{MetaFun},
%     \hologo{METAPOST},
%     \hologo{MetaPost},
%     \hologo{SLiTeX} (\xoption{lift}, \xoption{narrow}, \xoption{simple}),
%     \hologo{SliTeX} (\xoption{narrow}, \xoption{simple}, \xoption{lift}),
%     \hologo{teTeX}.
%   \item
%     Fixes:
%     \hologo{iniTeX},
%     \hologo{pdfLaTeX},
%     \hologo{pdfTeX},
%     \hologo{virTeX}.
%   \item
%     \cs{hologoFontSetup} and \cs{hologoLogoFontSetup} added.
%   \item
%     \cs{hologoVariant} and \cs{HologoVariant} added.
%   \end{Version}
%   \begin{Version}{2011/11/22 v1.7}
%   \item
%     New logos:
%     \hologo{BibTeX8},
%     \hologo{LaTeXML},
%     \hologo{SageTeX},
%     \hologo{TeX4ht},
%     \hologo{TTH}.
%   \item
%     \hologo{Xe} and friends: Driver stuff fixed.
%   \item
%     \hologo{Xe} and friends: Support for italic added.
%   \item
%     \hologo{Xe} and friends: Package support for \xpackage{pgf}
%     and \xpackage{pstricks} added.
%   \end{Version}
%   \begin{Version}{2011/11/29 v1.8}
%   \item
%     New logos:
%     \hologo{HanTheThanh}.
%   \end{Version}
%   \begin{Version}{2011/12/21 v1.9}
%   \item
%     Patch for package \xpackage{ifxetex} added for the case that
%     \cs{newif} is undefined in \hologo{iniTeX}.
%   \item
%     Some fixes for \hologo{iniTeX}.
%   \end{Version}
%   \begin{Version}{2012/04/26 v1.10}
%   \item
%     Fix in bookmark version of logo ``\hologo{HanTheThanh}''.
%   \end{Version}
%   \begin{Version}{2016/05/12 v1.11}
%   \item
%     Update HOLOGO@IfCharExists (previously in texlive)
%   \item define pdfliteral in current luatex.
%   \end{Version}
% \end{History}
%
% \PrintIndex
%
% \Finale
\endinput
%
        \else
          \input hologo.cfg\relax
        \fi
      \else
        \@PackageInfoNoLine{hologo}{%
          Empty configuration file `hologo.cfg' ignored%
        }%
      \fi
    \fi
  }%
}
%    \end{macrocode}
%
%    \begin{macrocode}
\def\HOLOGO@temp#1#2{%
  \kv@define@key{HoLogoDriver}{#1}[]{%
    \begingroup
      \def\HOLOGO@temp{##1}%
      \ltx@onelevel@sanitize\HOLOGO@temp
      \ifx\HOLOGO@temp\ltx@empty
      \else
        \@PackageError{hologo}{%
          Value (\HOLOGO@temp) not permitted for option `#1'%
        }%
        \@ehc
      \fi
    \endgroup
    \def\hologoDriver{#2}%
  }%
}%
\def\HOLOGO@@temp#1#2{%
  \ifx\kv@value\relax
    \HOLOGO@temp{#1}{#1}%
  \else
    \HOLOGO@temp{#1}{#2}%
  \fi
}%
\kv@parse@normalized{%
  pdftex,%
  luatex=pdftex,%
  dvipdfm,%
  dvipdfmx=dvipdfm,%
  dvips,%
  dvipsone=dvips,%
  xdvi=dvips,%
  xetex,%
  vtex,%
}\HOLOGO@@temp
%    \end{macrocode}
%
%    \begin{macrocode}
\kv@define@key{HoLogoDriver}{driverfallback}{%
  \def\HOLOGO@DriverFallback{#1}%
}
%    \end{macrocode}
%
%    \begin{macro}{\HOLOGO@DriverFallback}
%    \begin{macrocode}
\def\HOLOGO@DriverFallback{dvips}
%    \end{macrocode}
%    \end{macro}
%
%    \begin{macro}{\hologoDriverSetup}
%    \begin{macrocode}
\def\hologoDriverSetup{%
  \let\hologoDriver\ltx@undefined
  \HOLOGO@DriverSetup
}
%    \end{macrocode}
%    \end{macro}
%
%    \begin{macro}{\HOLOGO@DriverSetup}
%    \begin{macrocode}
\def\HOLOGO@DriverSetup#1{%
  \kvsetkeys{HoLogoDriver}{#1}%
  \HOLOGO@CheckDriver
  \ltx@ifundefined{hologoDriver}{%
    \begingroup
    \edef\x{\endgroup
      \noexpand\kvsetkeys{HoLogoDriver}{\HOLOGO@DriverFallback}%
    }\x
  }{}%
  \@PackageInfoNoLine{hologo}{Using driver `\hologoDriver'}%
}
%    \end{macrocode}
%    \end{macro}
%
%    \begin{macro}{\HOLOGO@CheckDriver}
%    \begin{macrocode}
\def\HOLOGO@CheckDriver{%
  \ifpdf
    \def\hologoDriver{pdftex}%
    \let\HOLOGO@pdfliteral\pdfliteral
    \ifluatex
      \ifx\pdfextension\@undefined\else
        \protected\def\pdfliteral{\pdfextension literal}%
        \let\HOLOGO@pdfliteral\pdfliteral
      \fi
      \ltx@IfUndefined{HOLOGO@pdfliteral}{%
        \ifnum\luatexversion<36 %
        \else
          \begingroup
            \let\HOLOGO@temp\endgroup
            \ifcase0%
                \directlua{%
                  if tex.enableprimitives then %
                    tex.enableprimitives('HOLOGO@', {'pdfliteral'})%
                  else %
                    tex.print('1')%
                  end%
                }%
                \ifx\HOLOGO@pdfliteral\@undefined 1\fi%
                \relax%
              \endgroup
              \let\HOLOGO@temp\relax
              \global\let\HOLOGO@pdfliteral\HOLOGO@pdfliteral
            \fi%
          \HOLOGO@temp
        \fi
      }{}%
    \fi
    \ltx@IfUndefined{HOLOGO@pdfliteral}{%
      \@PackageWarningNoLine{hologo}{%
        Cannot find \string\pdfliteral
      }%
    }{}%
  \else
    \ifxetex
      \def\hologoDriver{xetex}%
    \else
      \ifvtex
        \def\hologoDriver{vtex}%
      \fi
    \fi
  \fi
}
%    \end{macrocode}
%    \end{macro}
%
%    \begin{macro}{\HOLOGO@WarningUnsupportedDriver}
%    \begin{macrocode}
\def\HOLOGO@WarningUnsupportedDriver#1{%
  \@PackageWarningNoLine{hologo}{%
    Logo `#1' needs driver specific macros,\MessageBreak
    but driver `\hologoDriver' is not supported.\MessageBreak
    Use a different driver or\MessageBreak
    load package `graphics' or `pgf'%
  }%
}
%    \end{macrocode}
%    \end{macro}
%
% \subsubsection{Reflect box macros}
%
%    Skip driver part if not needed.
%    \begin{macrocode}
\ltx@IfUndefined{reflectbox}{}{%
  \ltx@IfUndefined{rotatebox}{}{%
    \HOLOGO@AtEnd
  }%
}
\ltx@IfUndefined{pgftext}{}{%
  \HOLOGO@AtEnd
}
\ltx@IfUndefined{psscalebox}{}{%
  \HOLOGO@AtEnd
}
%    \end{macrocode}
%
%    \begin{macrocode}
\def\HOLOGO@temp{LaTeX2e}
\ifx\fmtname\HOLOGO@temp
  \RequirePackage{kvoptions}[2011/06/30]%
  \ProcessKeyvalOptions{HoLogoDriver}%
\fi
\HOLOGO@DriverSetup{}
%    \end{macrocode}
%
%    \begin{macro}{\HOLOGO@ReflectBox}
%    \begin{macrocode}
\def\HOLOGO@ReflectBox#1{%
  \begingroup
    \setbox\ltx@zero\hbox{\begingroup#1\endgroup}%
    \setbox\ltx@two\hbox{%
      \kern\wd\ltx@zero
      \csname HOLOGO@ScaleBox@\hologoDriver\endcsname{-1}{1}{%
        \hbox to 0pt{\copy\ltx@zero\hss}%
      }%
    }%
    \wd\ltx@two=\wd\ltx@zero
    \box\ltx@two
  \endgroup
}
%    \end{macrocode}
%    \end{macro}
%
%    \begin{macro}{\HOLOGO@PointReflectBox}
%    \begin{macrocode}
\def\HOLOGO@PointReflectBox#1{%
  \begingroup
    \setbox\ltx@zero\hbox{\begingroup#1\endgroup}%
    \setbox\ltx@two\hbox{%
      \kern\wd\ltx@zero
      \raise\ht\ltx@zero\hbox{%
        \csname HOLOGO@ScaleBox@\hologoDriver\endcsname{-1}{-1}{%
          \hbox to 0pt{\copy\ltx@zero\hss}%
        }%
      }%
    }%
    \wd\ltx@two=\wd\ltx@zero
    \box\ltx@two
  \endgroup
}
%    \end{macrocode}
%    \end{macro}
%
%    We must define all variants because of dynamic driver setup.
%    \begin{macrocode}
\def\HOLOGO@temp#1#2{#2}
%    \end{macrocode}
%
%    \begin{macro}{\HOLOGO@ScaleBox@pdftex}
%    \begin{macrocode}
\HOLOGO@temp{pdftex}{%
  \def\HOLOGO@ScaleBox@pdftex#1#2#3{%
    \HOLOGO@pdfliteral{%
      q #1 0 0 #2 0 0 cm%
    }%
    #3%
    \HOLOGO@pdfliteral{%
      Q%
    }%
  }%
}
%    \end{macrocode}
%    \end{macro}
%    \begin{macro}{\HOLOGO@ScaleBox@dvips}
%    \begin{macrocode}
\HOLOGO@temp{dvips}{%
  \def\HOLOGO@ScaleBox@dvips#1#2#3{%
    \special{ps:%
      gsave %
      currentpoint %
      currentpoint translate %
      #1 #2 scale %
      neg exch neg exch translate%
    }%
    #3%
    \special{ps:%
      currentpoint %
      grestore %
      moveto%
    }%
  }%
}
%    \end{macrocode}
%    \end{macro}
%    \begin{macro}{\HOLOGO@ScaleBox@dvipdfm}
%    \begin{macrocode}
\HOLOGO@temp{dvipdfm}{%
  \let\HOLOGO@ScaleBox@dvipdfm\HOLOGO@ScaleBox@dvips
}
%    \end{macrocode}
%    \end{macro}
%    Since \hologo{XeTeX} v0.6.
%    \begin{macro}{\HOLOGO@ScaleBox@xetex}
%    \begin{macrocode}
\HOLOGO@temp{xetex}{%
  \def\HOLOGO@ScaleBox@xetex#1#2#3{%
    \special{x:gsave}%
    \special{x:scale #1 #2}%
    #3%
    \special{x:grestore}%
  }%
}
%    \end{macrocode}
%    \end{macro}
%    \begin{macro}{\HOLOGO@ScaleBox@vtex}
%    \begin{macrocode}
\HOLOGO@temp{vtex}{%
  \def\HOLOGO@ScaleBox@vtex#1#2#3{%
    \special{r(#1,0,0,#2,0,0}%
    #3%
    \special{r)}%
  }%
}
%    \end{macrocode}
%    \end{macro}
%
%    \begin{macrocode}
\HOLOGO@AtEnd%
%</package>
%    \end{macrocode}
%
% \section{Test}
%
% \subsection{Catcode checks for loading}
%
%    \begin{macrocode}
%<*test1>
%    \end{macrocode}
%    \begin{macrocode}
\catcode`\{=1 %
\catcode`\}=2 %
\catcode`\#=6 %
\catcode`\@=11 %
\expandafter\ifx\csname count@\endcsname\relax
  \countdef\count@=255 %
\fi
\expandafter\ifx\csname @gobble\endcsname\relax
  \long\def\@gobble#1{}%
\fi
\expandafter\ifx\csname @firstofone\endcsname\relax
  \long\def\@firstofone#1{#1}%
\fi
\expandafter\ifx\csname loop\endcsname\relax
  \expandafter\@firstofone
\else
  \expandafter\@gobble
\fi
{%
  \def\loop#1\repeat{%
    \def\body{#1}%
    \iterate
  }%
  \def\iterate{%
    \body
      \let\next\iterate
    \else
      \let\next\relax
    \fi
    \next
  }%
  \let\repeat=\fi
}%
\def\RestoreCatcodes{}
\count@=0 %
\loop
  \edef\RestoreCatcodes{%
    \RestoreCatcodes
    \catcode\the\count@=\the\catcode\count@\relax
  }%
\ifnum\count@<255 %
  \advance\count@ 1 %
\repeat

\def\RangeCatcodeInvalid#1#2{%
  \count@=#1\relax
  \loop
    \catcode\count@=15 %
  \ifnum\count@<#2\relax
    \advance\count@ 1 %
  \repeat
}
\def\RangeCatcodeCheck#1#2#3{%
  \count@=#1\relax
  \loop
    \ifnum#3=\catcode\count@
    \else
      \errmessage{%
        Character \the\count@\space
        with wrong catcode \the\catcode\count@\space
        instead of \number#3%
      }%
    \fi
  \ifnum\count@<#2\relax
    \advance\count@ 1 %
  \repeat
}
\def\space{ }
\expandafter\ifx\csname LoadCommand\endcsname\relax
  \def\LoadCommand{\input hologo.sty\relax}%
\fi
\def\Test{%
  \RangeCatcodeInvalid{0}{47}%
  \RangeCatcodeInvalid{58}{64}%
  \RangeCatcodeInvalid{91}{96}%
  \RangeCatcodeInvalid{123}{255}%
  \catcode`\@=12 %
  \catcode`\\=0 %
  \catcode`\%=14 %
  \LoadCommand
  \RangeCatcodeCheck{0}{36}{15}%
  \RangeCatcodeCheck{37}{37}{14}%
  \RangeCatcodeCheck{38}{47}{15}%
  \RangeCatcodeCheck{48}{57}{12}%
  \RangeCatcodeCheck{58}{63}{15}%
  \RangeCatcodeCheck{64}{64}{12}%
  \RangeCatcodeCheck{65}{90}{11}%
  \RangeCatcodeCheck{91}{91}{15}%
  \RangeCatcodeCheck{92}{92}{0}%
  \RangeCatcodeCheck{93}{96}{15}%
  \RangeCatcodeCheck{97}{122}{11}%
  \RangeCatcodeCheck{123}{255}{15}%
  \RestoreCatcodes
}
\Test
\csname @@end\endcsname
\end
%    \end{macrocode}
%    \begin{macrocode}
%</test1>
%    \end{macrocode}
%
% \subsection{Spacefactor}
%
%    The space factor must be 1000 after a logo. If it is greater 1000
%    then the following space is a space after a sentence closing point.
%    If the space factor is smaller 1000 then an immediate following
%    dot is interpreted as abbreviation, not sentence closing point.
%
%    \begin{macrocode}
%<*test-spacefactor>
\NeedsTeXFormat{LaTeX2e}
\documentclass{article}
\usepackage{hologo}[2016/05/12]
\usepackage{kvsetkeys}
\usepackage{qstest}
\IncludeTests{*}
\LogTests{log}{*}{*}
\begin{document}
\begin{qstest}{spacefactor}{spacefactor}
\newcommand*{\Test}[1]{%
  \sbox0{%
    \hologo{#1}%
    \Expect*{1000 (#1)}*{\the\spacefactor\space(#1)}%
  }%
}%
\makeatletter
\def\TestList{}
\def\hologoEntry#1#2#3{%
  \edef\TestList{%
    \ifx\TestList\@empty
    \else
      \TestList,%
    \fi
    #1%
    \ifx\\#2\\%
    \else
      ={variant=#2}%
    \fi
  }%
}
\hologoList
\expandafter\kv@parse@normalized\expandafter{%
  \TestList
}{%
  \begingroup
    \let\@logo=\kv@key
    \ifx\kv@value\relax
    \else
      \expandafter\hologoLogoSetup\expandafter\@logo\expandafter{%
        \kv@value
      }%
    \fi
    \Test\@logo
  \endgroup
  \@gobbletwo
}
\end{qstest}
\end{document}
%</test-spacefactor>
%    \end{macrocode}
%
% \subsection{Complete list}
%
%    \begin{macrocode}
%<*test-list>
\NeedsTeXFormat{LaTeX2e}
\documentclass[12pt,a4paper]{article}
\usepackage{hologo}[2016/05/12]
\usepackage[T1]{fontenc}
\usepackage{lmodern}
\usepackage{parskip}
\usepackage[unicode]{hyperref}[2011/09/28]
\usepackage{bookmark}[2011/09/19]
\bookmarksetup{%
  numbered,%
  open,%
  openlevel=2,%
}
\renewcommand*{\contentsname}{List of logos}
\begin{document}
\tableofcontents
\def\TestFont#1#2#3#4#5#6{%
  \begingroup
    \usefont{#3}{#4}{#5}{#6}%
    \HologoVariant{#1}{#2}/\hologoVariant{#1}{#2}%
    \quad
    \begingroup\scriptsize\hologoVariant{#1}{#2}\endgroup
    \quad
  \endgroup
  (#3/#4/#5/#6)%
  \par
}
\makeatletter
\def\hologoEntry#1#2#3{%
  \section{%
    \HologoVariant{#1}{#2}/\hologoVariant{#1}{#2} %
    {[#1\ifx\\#2\\\else\space(#2)\fi]}% hash-ok
  }% braces around [] because of bug in tex4ht
  \begingroup
    \hypersetup{unicode=false}%
    \bookmark[%
      dest=\@currentHref,%
      rellevel=1,%
      keeplevel,%
    ]{%
      \HologoVariant{#1}{#2}/\hologoVariant{#1}{#2} %
      (PDFDocEncoding)%
    }%
  \endgroup
  \TestFont{#1}{#2}{OT1}{cmr}{m}{n}%
  \TestFont{#1}{#2}{OT1}{cmss}{m}{n}%
  \TestFont{#1}{#2}{OT1}{cmr}{b}{n}%
  \TestFont{#1}{#2}{OT1}{cmr}{m}{it}%
  \TestFont{#1}{#2}{OT1}{cmtt}{m}{n}%
  \TestFont{#1}{#2}{T1}{lmr}{m}{n}%
  \TestFont{#1}{#2}{T1}{lmss}{m}{n}%
  \TestFont{#1}{#2}{T1}{lmr}{b}{n}%
  \TestFont{#1}{#2}{T1}{lmr}{m}{it}%
  \TestFont{#1}{#2}{T1}{lmtt}{m}{n}%
  \TestFont{#1}{#2}{T1}{lmvtt}{m}{n}%
  \TestFont{#1}{#2}{T1}{qtm}{m}{n}%
  \TestFont{#1}{#2}{T1}{qhv}{m}{n}%
  \TestFont{#1}{#2}{T1}{qtm}{b}{n}%
  \TestFont{#1}{#2}{T1}{qtm}{m}{it}%
  \TestFont{#1}{#2}{T1}{qcr}{m}{n}%
  \newpage
}
\makeatother
\hologoList
\end{document}
%</test-list>
%    \end{macrocode}
%
% \section{Installation}
%
% \subsection{Download}
%
% \paragraph{Package.} This package is available on
% CTAN\footnote{\url{ftp://ftp.ctan.org/tex-archive/}}:
% \begin{description}
% \item[\CTAN{macros/latex/contrib/oberdiek/hologo.dtx}] The source file.
% \item[\CTAN{macros/latex/contrib/oberdiek/hologo.pdf}] Documentation.
% \end{description}
%
%
% \paragraph{Bundle.} All the packages of the bundle `oberdiek'
% are also available in a TDS compliant ZIP archive. There
% the packages are already unpacked and the documentation files
% are generated. The files and directories obey the TDS standard.
% \begin{description}
% \item[\CTAN{install/macros/latex/contrib/oberdiek.tds.zip}]
% \end{description}
% \emph{TDS} refers to the standard ``A Directory Structure
% for \TeX\ Files'' (\CTAN{tds/tds.pdf}). Directories
% with \xfile{texmf} in their name are usually organized this way.
%
% \subsection{Bundle installation}
%
% \paragraph{Unpacking.} Unpack the \xfile{oberdiek.tds.zip} in the
% TDS tree (also known as \xfile{texmf} tree) of your choice.
% Example (linux):
% \begin{quote}
%   |unzip oberdiek.tds.zip -d ~/texmf|
% \end{quote}
%
% \paragraph{Script installation.}
% Check the directory \xfile{TDS:scripts/oberdiek/} for
% scripts that need further installation steps.
% Package \xpackage{attachfile2} comes with the Perl script
% \xfile{pdfatfi.pl} that should be installed in such a way
% that it can be called as \texttt{pdfatfi}.
% Example (linux):
% \begin{quote}
%   |chmod +x scripts/oberdiek/pdfatfi.pl|\\
%   |cp scripts/oberdiek/pdfatfi.pl /usr/local/bin/|
% \end{quote}
%
% \subsection{Package installation}
%
% \paragraph{Unpacking.} The \xfile{.dtx} file is a self-extracting
% \docstrip\ archive. The files are extracted by running the
% \xfile{.dtx} through \plainTeX:
% \begin{quote}
%   \verb|tex hologo.dtx|
% \end{quote}
%
% \paragraph{TDS.} Now the different files must be moved into
% the different directories in your installation TDS tree
% (also known as \xfile{texmf} tree):
% \begin{quote}
% \def\t{^^A
% \begin{tabular}{@{}>{\ttfamily}l@{ $\rightarrow$ }>{\ttfamily}l@{}}
%   hologo.sty & tex/generic/oberdiek/hologo.sty\\
%   hologo.pdf & doc/latex/oberdiek/hologo.pdf\\
%   example/hologo-example.tex & doc/latex/oberdiek/example/hologo-example.tex\\
%   test/hologo-test1.tex & doc/latex/oberdiek/test/hologo-test1.tex\\
%   test/hologo-test-spacefactor.tex & doc/latex/oberdiek/test/hologo-test-spacefactor.tex\\
%   test/hologo-test-list.tex & doc/latex/oberdiek/test/hologo-test-list.tex\\
%   hologo.dtx & source/latex/oberdiek/hologo.dtx\\
% \end{tabular}^^A
% }^^A
% \sbox0{\t}^^A
% \ifdim\wd0>\linewidth
%   \begingroup
%     \advance\linewidth by\leftmargin
%     \advance\linewidth by\rightmargin
%   \edef\x{\endgroup
%     \def\noexpand\lw{\the\linewidth}^^A
%   }\x
%   \def\lwbox{^^A
%     \leavevmode
%     \hbox to \linewidth{^^A
%       \kern-\leftmargin\relax
%       \hss
%       \usebox0
%       \hss
%       \kern-\rightmargin\relax
%     }^^A
%   }^^A
%   \ifdim\wd0>\lw
%     \sbox0{\small\t}^^A
%     \ifdim\wd0>\linewidth
%       \ifdim\wd0>\lw
%         \sbox0{\footnotesize\t}^^A
%         \ifdim\wd0>\linewidth
%           \ifdim\wd0>\lw
%             \sbox0{\scriptsize\t}^^A
%             \ifdim\wd0>\linewidth
%               \ifdim\wd0>\lw
%                 \sbox0{\tiny\t}^^A
%                 \ifdim\wd0>\linewidth
%                   \lwbox
%                 \else
%                   \usebox0
%                 \fi
%               \else
%                 \lwbox
%               \fi
%             \else
%               \usebox0
%             \fi
%           \else
%             \lwbox
%           \fi
%         \else
%           \usebox0
%         \fi
%       \else
%         \lwbox
%       \fi
%     \else
%       \usebox0
%     \fi
%   \else
%     \lwbox
%   \fi
% \else
%   \usebox0
% \fi
% \end{quote}
% If you have a \xfile{docstrip.cfg} that configures and enables \docstrip's
% TDS installing feature, then some files can already be in the right
% place, see the documentation of \docstrip.
%
% \subsection{Refresh file name databases}
%
% If your \TeX~distribution
% (\teTeX, \mikTeX, \dots) relies on file name databases, you must refresh
% these. For example, \teTeX\ users run \verb|texhash| or
% \verb|mktexlsr|.
%
% \subsection{Some details for the interested}
%
% \paragraph{Attached source.}
%
% The PDF documentation on CTAN also includes the
% \xfile{.dtx} source file. It can be extracted by
% AcrobatReader 6 or higher. Another option is \textsf{pdftk},
% e.g. unpack the file into the current directory:
% \begin{quote}
%   \verb|pdftk hologo.pdf unpack_files output .|
% \end{quote}
%
% \paragraph{Unpacking with \LaTeX.}
% The \xfile{.dtx} chooses its action depending on the format:
% \begin{description}
% \item[\plainTeX:] Run \docstrip\ and extract the files.
% \item[\LaTeX:] Generate the documentation.
% \end{description}
% If you insist on using \LaTeX\ for \docstrip\ (really,
% \docstrip\ does not need \LaTeX), then inform the autodetect routine
% about your intention:
% \begin{quote}
%   \verb|latex \let\install=y% \iffalse meta-comment
%
% File: hologo.dtx
% Version: 2016/05/12 v1.11
% Info: A logo collection with bookmark support
%
% Copyright (C) 2010-2012 by
%    Heiko Oberdiek <heiko.oberdiek at googlemail.com>
%
% This work may be distributed and/or modified under the
% conditions of the LaTeX Project Public License, either
% version 1.3c of this license or (at your option) any later
% version. This version of this license is in
%    http://www.latex-project.org/lppl/lppl-1-3c.txt
% and the latest version of this license is in
%    http://www.latex-project.org/lppl.txt
% and version 1.3 or later is part of all distributions of
% LaTeX version 2005/12/01 or later.
%
% This work has the LPPL maintenance status "maintained".
%
% This Current Maintainer of this work is Heiko Oberdiek.
%
% The Base Interpreter refers to any `TeX-Format',
% because some files are installed in TDS:tex/generic//.
%
% This work consists of the main source file hologo.dtx
% and the derived files
%    hologo.sty, hologo.pdf, hologo.ins, hologo.drv, hologo-example.tex,
%    hologo-test1.tex, hologo-test-spacefactor.tex,
%    hologo-test-list.tex.
%
% Distribution:
%    CTAN:macros/latex/contrib/oberdiek/hologo.dtx
%    CTAN:macros/latex/contrib/oberdiek/hologo.pdf
%
% Unpacking:
%    (a) If hologo.ins is present:
%           tex hologo.ins
%    (b) Without hologo.ins:
%           tex hologo.dtx
%    (c) If you insist on using LaTeX
%           latex \let\install=y\input{hologo.dtx}
%        (quote the arguments according to the demands of your shell)
%
% Documentation:
%    (a) If hologo.drv is present:
%           latex hologo.drv
%    (b) Without hologo.drv:
%           latex hologo.dtx; ...
%    The class ltxdoc loads the configuration file ltxdoc.cfg
%    if available. Here you can specify further options, e.g.
%    use A4 as paper format:
%       \PassOptionsToClass{a4paper}{article}
%
%    Programm calls to get the documentation (example):
%       pdflatex hologo.dtx
%       makeindex -s gind.ist hologo.idx
%       pdflatex hologo.dtx
%       makeindex -s gind.ist hologo.idx
%       pdflatex hologo.dtx
%
% Installation:
%    TDS:tex/generic/oberdiek/hologo.sty
%    TDS:doc/latex/oberdiek/hologo.pdf
%    TDS:doc/latex/oberdiek/example/hologo-example.tex
%    TDS:doc/latex/oberdiek/test/hologo-test1.tex
%    TDS:doc/latex/oberdiek/test/hologo-test-spacefactor.tex
%    TDS:doc/latex/oberdiek/test/hologo-test-list.tex
%    TDS:source/latex/oberdiek/hologo.dtx
%
%<*ignore>
\begingroup
  \catcode123=1 %
  \catcode125=2 %
  \def\x{LaTeX2e}%
\expandafter\endgroup
\ifcase 0\ifx\install y1\fi\expandafter
         \ifx\csname processbatchFile\endcsname\relax\else1\fi
         \ifx\fmtname\x\else 1\fi\relax
\else\csname fi\endcsname
%</ignore>
%<*install>
\input docstrip.tex
\Msg{************************************************************************}
\Msg{* Installation}
\Msg{* Package: hologo 2016/05/12 v1.11 A logo collection with bookmark support (HO)}
\Msg{************************************************************************}

\keepsilent
\askforoverwritefalse

\let\MetaPrefix\relax
\preamble

This is a generated file.

Project: hologo
Version: 2016/05/12 v1.11

Copyright (C) 2010-2012 by
   Heiko Oberdiek <heiko.oberdiek at googlemail.com>

This work may be distributed and/or modified under the
conditions of the LaTeX Project Public License, either
version 1.3c of this license or (at your option) any later
version. This version of this license is in
   http://www.latex-project.org/lppl/lppl-1-3c.txt
and the latest version of this license is in
   http://www.latex-project.org/lppl.txt
and version 1.3 or later is part of all distributions of
LaTeX version 2005/12/01 or later.

This work has the LPPL maintenance status "maintained".

This Current Maintainer of this work is Heiko Oberdiek.

The Base Interpreter refers to any `TeX-Format',
because some files are installed in TDS:tex/generic//.

This work consists of the main source file hologo.dtx
and the derived files
   hologo.sty, hologo.pdf, hologo.ins, hologo.drv, hologo-example.tex,
   hologo-test1.tex, hologo-test-spacefactor.tex,
   hologo-test-list.tex.

\endpreamble
\let\MetaPrefix\DoubleperCent

\generate{%
  \file{hologo.ins}{\from{hologo.dtx}{install}}%
  \file{hologo.drv}{\from{hologo.dtx}{driver}}%
  \usedir{tex/generic/oberdiek}%
  \file{hologo.sty}{\from{hologo.dtx}{package}}%
  \usedir{doc/latex/oberdiek/example}%
  \file{hologo-example.tex}{\from{hologo.dtx}{example}}%
  \usedir{doc/latex/oberdiek/test}%
  \file{hologo-test1.tex}{\from{hologo.dtx}{test1}}%
  \file{hologo-test-spacefactor.tex}{\from{hologo.dtx}{test-spacefactor}}%
  \file{hologo-test-list.tex}{\from{hologo.dtx}{test-list}}%
  \nopreamble
  \nopostamble
  \usedir{source/latex/oberdiek/catalogue}%
  \file{hologo.xml}{\from{hologo.dtx}{catalogue}}%
}

\catcode32=13\relax% active space
\let =\space%
\Msg{************************************************************************}
\Msg{*}
\Msg{* To finish the installation you have to move the following}
\Msg{* file into a directory searched by TeX:}
\Msg{*}
\Msg{*     hologo.sty}
\Msg{*}
\Msg{* To produce the documentation run the file `hologo.drv'}
\Msg{* through LaTeX.}
\Msg{*}
\Msg{* Happy TeXing!}
\Msg{*}
\Msg{************************************************************************}

\endbatchfile
%</install>
%<*ignore>
\fi
%</ignore>
%<*driver>
\NeedsTeXFormat{LaTeX2e}
\ProvidesFile{hologo.drv}%
  [2016/05/12 v1.11 A logo collection with bookmark support (HO)]%
\documentclass{ltxdoc}
\usepackage{holtxdoc}[2011/11/22]
\usepackage{hologo}[2016/05/12]
\usepackage{longtable}
\usepackage{array}
\usepackage{paralist}
%\usepackage[T1]{fontenc}
%\usepackage{lmodern}
\begin{document}
  \DocInput{hologo.dtx}%
\end{document}
%</driver>
% \fi
%
%
% \CharacterTable
%  {Upper-case    \A\B\C\D\E\F\G\H\I\J\K\L\M\N\O\P\Q\R\S\T\U\V\W\X\Y\Z
%   Lower-case    \a\b\c\d\e\f\g\h\i\j\k\l\m\n\o\p\q\r\s\t\u\v\w\x\y\z
%   Digits        \0\1\2\3\4\5\6\7\8\9
%   Exclamation   \!     Double quote  \"     Hash (number) \#
%   Dollar        \$     Percent       \%     Ampersand     \&
%   Acute accent  \'     Left paren    \(     Right paren   \)
%   Asterisk      \*     Plus          \+     Comma         \,
%   Minus         \-     Point         \.     Solidus       \/
%   Colon         \:     Semicolon     \;     Less than     \<
%   Equals        \=     Greater than  \>     Question mark \?
%   Commercial at \@     Left bracket  \[     Backslash     \\
%   Right bracket \]     Circumflex    \^     Underscore    \_
%   Grave accent  \`     Left brace    \{     Vertical bar  \|
%   Right brace   \}     Tilde         \~}
%
% \GetFileInfo{hologo.drv}
%
% \title{The \xpackage{hologo} package}
% \date{2016/05/12 v1.11}
% \author{Heiko Oberdiek\\\xemail{heiko.oberdiek at googlemail.com}}
%
% \maketitle
%
% \begin{abstract}
% This package starts a collection of logos with support for bookmarks
% strings.
% \end{abstract}
%
% \tableofcontents
%
% \section{Documentation}
%
% \subsection{Logo macros}
%
% \begin{declcs}{hologo} \M{name}
% \end{declcs}
% Macro \cs{hologo} sets the logo with name \meta{name}.
% The following table shows the supported names.
%
% \begingroup
%   \def\hologoEntry#1#2#3{^^A
%     #1&#2&\hologoLogoSetup{#1}{variant=#2}\hologo{#1}&#3\tabularnewline
%   }
%   \begin{longtable}{>{\ttfamily}l>{\ttfamily}lll}
%     \rmfamily\bfseries{name} & \rmfamily\bfseries variant
%     & \bfseries logo & \bfseries since\\
%     \hline
%     \endhead
%     \hologoList
%   \end{longtable}
% \endgroup
%
% \begin{declcs}{Hologo} \M{name}
% \end{declcs}
% Macro \cs{Hologo} starts the logo \meta{name} with an uppercase
% letter. As an exception small greek letters are not converted
% to uppercase. Examples, see \hologo{eTeX} and \hologo{ExTeX}.
%
% \subsection{Setup macros}
%
% The package does not support package options, but the following
% setup macros can be used to set options.
%
% \begin{declcs}{hologoSetup} \M{key value list}
% \end{declcs}
% Macro \cs{hologoSetup} sets global options.
%
% \begin{declcs}{hologoLogoSetup} \M{logo} \M{key value list}
% \end{declcs}
% Some options can also be used to configure a logo.
% These settings take precedence over global option settings.
%
% \subsection{Options}\label{sec:options}
%
% There are boolean and string options:
% \begin{description}
% \item[Boolean option:]
% It takes |true| or |false|
% as value. If the value is omitted, then |true| is used.
% \item[String option:]
% A value must be given as string. (But the string might be empty.)
% \end{description}
% The following options can be used both in \cs{hologoSetup}
% and \cs{hologoLogoSetup}:
% \begin{description}
% \def\entry#1{\item[\xoption{#1}:]}
% \entry{break}
%   enables or disables line breaks inside the logo. This setting is
%   refined by options \xoption{hyphenbreak}, \xoption{spacebreak}
%   or \xoption{discretionarybreak}.
%   Default is |false|.
% \entry{hyphenbreak}
%   enables or disables the line break right after the hyphen character.
% \entry{spacebreak}
%   enables or disables line breaks at space characters.
% \entry{discretionarybreak}
%   enables or disables line breaks at hyphenation points
%   (inserted by \cs{-}).
% \end{description}
% Macro \cs{hologoLogoSetup} also knows:
% \begin{description}
% \item[\xoption{variant}:]
%   This is a string option. It specifies a variant of a logo that
%   must exist. An empty string selects the package default variant.
% \end{description}
% Example:
% \begin{quote}
%   |\hologoSetup{break=false}|\\
%   |\hologoLogoSetup{plainTeX}{variant=hyphen,hyphenbreak}|\\
%   Then ``plain-\TeX'' contains one break point after the hyphen.
% \end{quote}
%
% \subsection{Driver options}
%
% Sometimes graphical operations are needed to construct some
% glyphs (e.g.\ \hologo{XeTeX}). If package \xpackage{graphics}
% or package \xpackage{pgf} are found, then the macros are taken
% from there. Otherwise the packge defines its own operations
% and therefore needs the driver information. Many drivers are
% detected automatically (\hologo{pdfTeX}/\hologo{LuaTeX}
% in PDF mode, \hologo{XeTeX}, \hologo{VTeX}). These have precedence
% over a driver option. The driver can be given as package option
% or using \cs{hologoDriverSetup}.
% The following list contains the recognized driver options:
% \begin{itemize}
% \item \xoption{pdftex}, \xoption{luatex}
% \item \xoption{dvipdfm}, \xoption{dvipdfmx}
% \item \xoption{dvips}, \xoption{dvipsone}, \xoption{xdvi}
% \item \xoption{xetex}
% \item \xoption{vtex}
% \end{itemize}
% The left driver of a line is the driver name that is used internally.
% The following names are aliases for drivers that use the
% same method. Therefore the entry in the \xext{log} file for
% the used driver prints the internally used driver name.
% \begin{description}
% \item[\xoption{driverfallback}:]
%   This option expects a driver that is used,
%   if the driver could not be detected automatically.
% \end{description}
%
% \begin{declcs}{hologoDriverSetup} \M{driver option}
% \end{declcs}
% The driver can also be configured after package loading
% using \cs{hologoDriverSetup}, also the way for \hologo{plainTeX}
% to setup the driver.
%
% \subsection{Font setup}
%
% Some logos require a special font, but should also be usable by
% \hologo{plainTeX}. Therefore the package provides some ways
% to influence the font settings. The options below
% take font settings as values. Both font commands
% such as \cs{sffamily} and macros that take one argument
% like \cs{textsf} can be used.
%
% \begin{declcs}{hologoFontSetup} \M{key value list}
% \end{declcs}
% Macro \cs{hologoFontSetup} sets the fonts for all logos.
% Supported keys:
% \begin{description}
% \def\entry#1{\item[\xoption{#1}:]}
% \entry{general}
%   This font is used for all logos. The default is empty.
%   That means no special font is used.
% \entry{bibsf}
%   This font is used for
%   {\hologoLogoSetup{BibTeX}{variant=sf}\hologo{BibTeX}}
%   with variant \xoption{sf}.
% \entry{rm}
%   This font is a serif font. It is used for \hologo{ExTeX}.
% \entry{sc}
%   This font specifies a small caps font. It is used for
%   {\hologoLogoSetup{BibTeX}{variant=sc}\hologo{BibTeX}}
%   with variant \xoption{sc}.
% \entry{sf}
%   This font specifies a sans serif font. The default
%   is \cs{sffamily}, then \cs{sf} is tried. Otherwise
%   a warning is given. It is used by \hologo{KOMAScript}.
% \entry{sy}
%   This is the font for math symbols (e.g. cmsy).
%   It is used by \hologo{AmS}, \hologo{NTS}, \hologo{ExTeX}.
% \entry{logo}
%   \hologo{METAFONT} and \hologo{METAPOST} are using that font.
%   In \hologo{LaTeX} \cs{logofamily} is used and
%   the definitions of package \xpackage{mflogo} are used
%   if the package is not loaded.
%   Otherwise the \cs{tenlogo} is used and defined
%   if it does not already exists.
% \end{description}
%
% \begin{declcs}{hologoLogoFontSetup} \M{logo} \M{key value list}
% \end{declcs}
% Fonts can also be set for a logo or logo component separately,
% see the following list.
% The keys are the same as for \cs{hologoFontSetup}.
%
% \begin{longtable}{>{\ttfamily}l>{\sffamily}ll}
%   \meta{logo} & keys & result\\
%   \hline
%   \endhead
%   BibTeX & bibsf & {\hologoLogoSetup{BibTeX}{variant=sf}\hologo{BibTeX}}\\[.5ex]
%   BibTeX & sc & {\hologoLogoSetup{BibTeX}{variant=sc}\hologo{BibTeX}}\\[.5ex]
%   ExTeX & rm & \hologo{ExTeX}\\
%   SliTeX & rm & \hologo{SliTeX}\\[.5ex]
%   AmS & sy & \hologo{AmS}\\
%   ExTeX & sy & \hologo{ExTeX}\\
%   NTS & sy & \hologo{NTS}\\[.5ex]
%   KOMAScript & sf & \hologo{KOMAScript}\\[.5ex]
%   METAFONT & logo & \hologo{METAFONT}\\
%   METAPOST & logo & \hologo{METAPOST}\\[.5ex]
%   SliTeX & sc \hologo{SliTeX}
% \end{longtable}
%
% \subsubsection{Font order}
%
% For all logos the font \xoption{general} is applied first.
% Example:
%\begin{quote}
%|\hologoFontSetup{general=\color{red}}|
%\end{quote}
% will print red logos.
% Then if the font uses a special font \xoption{sf}, for example,
% the font is applied that is setup by \cs{hologoLogoFontSetup}.
% If this font is not setup, then the common font setup
% by \cs{hologoFontSetup} is used. Otherwise a warning is given,
% that there is no font configured.
%
% \subsection{Additional user macros}
%
% Usually a variant of a logo is configured by using
% \cs{hologoLogoSetup}, because it is bad style to mix
% different variants of the same logo in the same text.
% There the following macros are a convenience for testing.
%
% \begin{declcs}{hologoVariant} \M{name} \M{variant}\\
%   \cs{HologoVariant} \M{name} \M{variant}
% \end{declcs}
% Logo \meta{name} is set using \meta{variant} that specifies
% explicitely which variant of the macro is used. If the argument
% is empty, then the default form of the logo is used
% (configurable by \cs{hologoLogoSetup}).
%
% \cs{HologoVariant} is used if the logo is set in a context
% that needs an uppercase first letter (beginning of a sentence, \dots).
%
% \begin{declcs}{hologoList}\\
%   \cs{hologoEntry} \M{logo} \M{variant} \M{since}
% \end{declcs}
% Macro \cs{hologoList} contains all logos that are provided
% by the package including variants. The list consists of calls
% of \cs{hologoEntry} with three arguments starting with the
% logo name \meta{logo} and its variant \meta{variant}. An empty
% variant means the current default. Argument \meta{since} specifies
% with version of the package \xpackage{hologo} is needed to get
% the logo. If the logo is fixed, then the date gets updated.
% Therefore the date \meta{since} is not exactly the date of
% the first introduction, but rather the date of the latest fix.
%
% Before \cs{hologoList} can be used, macro \cs{hologoEntry} needs
% a definition. The example file in section \ref{sec:example}
% shows applications of \cs{hologoList}.
%
% \subsection{Supported contexts}
%
% Macros \cs{hologo} and friends support special contexts:
% \begin{itemize}
% \item \hologo{LaTeX}'s protection mechanism.
% \item Bookmarks of package \xpackage{hyperref}.
% \item Package \xpackage{tex4ht}.
% \item The macros can be used inside \cs{csname} constructs,
%   if \cs{ifincsname} is available (\hologo{pdfTeX}, \hologo{XeTeX},
%   \hologo{LuaTeX}).
% \end{itemize}
%
% \subsection{Example}
% \label{sec:example}
%
% The following example prints the logos in different fonts.
%    \begin{macrocode}
%<*example>
%<<verbatim
\NeedsTeXFormat{LaTeX2e}
\documentclass[a4paper]{article}
\usepackage[
  hmargin=20mm,
  vmargin=20mm,
]{geometry}
\pagestyle{empty}
\usepackage{hologo}[2016/05/12]
\usepackage{longtable}
\usepackage{array}
\setlength{\extrarowheight}{2pt}
\usepackage[T1]{fontenc}
\usepackage{lmodern}
\usepackage{pdflscape}
\usepackage[
  pdfencoding=auto,
]{hyperref}
\hypersetup{
  pdfauthor={Heiko Oberdiek},
  pdftitle={Example for package `hologo'},
  pdfsubject={Logos with fonts lmr, lmss, qtm, qpl, qhv},
}
\usepackage{bookmark}

% Print the logo list on the console

\begingroup
  \typeout{}%
  \typeout{*** Begin of logo list ***}%
  \newcommand*{\hologoEntry}[3]{%
    \typeout{#1 \ifx\\#2\\\else(#2) \fi[#3]}%
  }%
  \hologoList
  \typeout{*** End of logo list ***}%
  \typeout{}%
\endgroup

\begin{document}
\begin{landscape}

  \section{Example file for package `hologo'}

  % Table for font names

  \begin{longtable}{>{\bfseries}ll}
    \textbf{font} & \textbf{Font name}\\
    \hline
    lmr & Latin Modern Roman\\
    lmss & Latin Modern Sans\\
    qtm & \TeX\ Gyre Termes\\
    qhv & \TeX\ Gyre Heros\\
    qpl & \TeX\ Gyre Pagella\\
  \end{longtable}

  % Logo list with logos in different fonts

  \begingroup
    \newcommand*{\SetVariant}[2]{%
      \ifx\\#2\\%
      \else
        \hologoLogoSetup{#1}{variant=#2}%
      \fi
    }%
    \newcommand*{\hologoEntry}[3]{%
      \SetVariant{#1}{#2}%
      \raisebox{1em}[0pt][0pt]{\hypertarget{#1@#2}{}}%
      \bookmark[%
        dest={#1@#2},%
      ]{%
        #1\ifx\\#2\\\else\space(#2)\fi: \Hologo{#1}, \hologo{#1} %
        [Unicode]%
      }%
      \hypersetup{unicode=false}%
      \bookmark[%
        dest={#1@#2},%
      ]{%
        #1\ifx\\#2\\\else\space(#2)\fi: \Hologo{#1}, \hologo{#1} %
        [PDFDocEncoding]%
      }%
      \texttt{#1}%
      &%
      \texttt{#2}%
      &%
      \Hologo{#1}%
      &%
      \SetVariant{#1}{#2}%
      \hologo{#1}%
      &%
      \SetVariant{#1}{#2}%
      \fontfamily{qtm}\selectfont
      \hologo{#1}%
      &%
      \SetVariant{#1}{#2}%
      \fontfamily{qpl}\selectfont
      \hologo{#1}%
      &%
      \SetVariant{#1}{#2}%
      \textsf{\hologo{#1}}%
      &%
      \SetVariant{#1}{#2}%
      \fontfamily{qhv}\selectfont
      \hologo{#1}%
      \tabularnewline
    }%
    \begin{longtable}{llllllll}%
      \textbf{\textit{logo}} & \textbf{\textit{variant}} &
      \texttt{\string\Hologo} &
      \textbf{lmr} & \textbf{qtm} & \textbf{qpl} &
      \textbf{lmss} & \textbf{qhv}
      \tabularnewline
      \hline
      \endhead
      \hologoList
    \end{longtable}%
  \endgroup

\end{landscape}
\end{document}
%verbatim
%</example>
%    \end{macrocode}
%
% \StopEventually{
% }
%
% \section{Implementation}
%    \begin{macrocode}
%<*package>
%    \end{macrocode}
%    Reload check, especially if the package is not used with \LaTeX.
%    \begin{macrocode}
\begingroup\catcode61\catcode48\catcode32=10\relax%
  \catcode13=5 % ^^M
  \endlinechar=13 %
  \catcode35=6 % #
  \catcode39=12 % '
  \catcode44=12 % ,
  \catcode45=12 % -
  \catcode46=12 % .
  \catcode58=12 % :
  \catcode64=11 % @
  \catcode123=1 % {
  \catcode125=2 % }
  \expandafter\let\expandafter\x\csname ver@hologo.sty\endcsname
  \ifx\x\relax % plain-TeX, first loading
  \else
    \def\empty{}%
    \ifx\x\empty % LaTeX, first loading,
      % variable is initialized, but \ProvidesPackage not yet seen
    \else
      \expandafter\ifx\csname PackageInfo\endcsname\relax
        \def\x#1#2{%
          \immediate\write-1{Package #1 Info: #2.}%
        }%
      \else
        \def\x#1#2{\PackageInfo{#1}{#2, stopped}}%
      \fi
      \x{hologo}{The package is already loaded}%
      \aftergroup\endinput
    \fi
  \fi
\endgroup%
%    \end{macrocode}
%    Package identification:
%    \begin{macrocode}
\begingroup\catcode61\catcode48\catcode32=10\relax%
  \catcode13=5 % ^^M
  \endlinechar=13 %
  \catcode35=6 % #
  \catcode39=12 % '
  \catcode40=12 % (
  \catcode41=12 % )
  \catcode44=12 % ,
  \catcode45=12 % -
  \catcode46=12 % .
  \catcode47=12 % /
  \catcode58=12 % :
  \catcode64=11 % @
  \catcode91=12 % [
  \catcode93=12 % ]
  \catcode123=1 % {
  \catcode125=2 % }
  \expandafter\ifx\csname ProvidesPackage\endcsname\relax
    \def\x#1#2#3[#4]{\endgroup
      \immediate\write-1{Package: #3 #4}%
      \xdef#1{#4}%
    }%
  \else
    \def\x#1#2[#3]{\endgroup
      #2[{#3}]%
      \ifx#1\@undefined
        \xdef#1{#3}%
      \fi
      \ifx#1\relax
        \xdef#1{#3}%
      \fi
    }%
  \fi
\expandafter\x\csname ver@hologo.sty\endcsname
\ProvidesPackage{hologo}%
  [2016/05/12 v1.11 A logo collection with bookmark support (HO)]%
%    \end{macrocode}
%
%    \begin{macrocode}
\begingroup\catcode61\catcode48\catcode32=10\relax%
  \catcode13=5 % ^^M
  \endlinechar=13 %
  \catcode123=1 % {
  \catcode125=2 % }
  \catcode64=11 % @
  \def\x{\endgroup
    \expandafter\edef\csname HOLOGO@AtEnd\endcsname{%
      \endlinechar=\the\endlinechar\relax
      \catcode13=\the\catcode13\relax
      \catcode32=\the\catcode32\relax
      \catcode35=\the\catcode35\relax
      \catcode61=\the\catcode61\relax
      \catcode64=\the\catcode64\relax
      \catcode123=\the\catcode123\relax
      \catcode125=\the\catcode125\relax
    }%
  }%
\x\catcode61\catcode48\catcode32=10\relax%
\catcode13=5 % ^^M
\endlinechar=13 %
\catcode35=6 % #
\catcode64=11 % @
\catcode123=1 % {
\catcode125=2 % }
\def\TMP@EnsureCode#1#2{%
  \edef\HOLOGO@AtEnd{%
    \HOLOGO@AtEnd
    \catcode#1=\the\catcode#1\relax
  }%
  \catcode#1=#2\relax
}
\TMP@EnsureCode{10}{12}% ^^J
\TMP@EnsureCode{33}{12}% !
\TMP@EnsureCode{34}{12}% "
\TMP@EnsureCode{36}{3}% $
\TMP@EnsureCode{38}{4}% &
\TMP@EnsureCode{39}{12}% '
\TMP@EnsureCode{40}{12}% (
\TMP@EnsureCode{41}{12}% )
\TMP@EnsureCode{42}{12}% *
\TMP@EnsureCode{43}{12}% +
\TMP@EnsureCode{44}{12}% ,
\TMP@EnsureCode{45}{12}% -
\TMP@EnsureCode{46}{12}% .
\TMP@EnsureCode{47}{12}% /
\TMP@EnsureCode{58}{12}% :
\TMP@EnsureCode{59}{12}% ;
\TMP@EnsureCode{60}{12}% <
\TMP@EnsureCode{62}{12}% >
\TMP@EnsureCode{63}{12}% ?
\TMP@EnsureCode{91}{12}% [
\TMP@EnsureCode{93}{12}% ]
\TMP@EnsureCode{94}{7}% ^ (superscript)
\TMP@EnsureCode{95}{8}% _ (subscript)
\TMP@EnsureCode{96}{12}% `
\TMP@EnsureCode{124}{12}% |
\edef\HOLOGO@AtEnd{%
  \HOLOGO@AtEnd
  \escapechar\the\escapechar\relax
  \noexpand\endinput
}
\escapechar=92 %
%    \end{macrocode}
%
% \subsection{Logo list}
%
%    \begin{macro}{\hologoList}
%    \begin{macrocode}
\def\hologoList{%
  \hologoEntry{(La)TeX}{}{2011/10/01}%
  \hologoEntry{AmSLaTeX}{}{2010/04/16}%
  \hologoEntry{AmSTeX}{}{2010/04/16}%
  \hologoEntry{biber}{}{2011/10/01}%
  \hologoEntry{BibTeX}{}{2011/10/01}%
  \hologoEntry{BibTeX}{sf}{2011/10/01}%
  \hologoEntry{BibTeX}{sc}{2011/10/01}%
  \hologoEntry{BibTeX8}{}{2011/11/22}%
  \hologoEntry{ConTeXt}{}{2011/03/25}%
  \hologoEntry{ConTeXt}{narrow}{2011/03/25}%
  \hologoEntry{ConTeXt}{simple}{2011/03/25}%
  \hologoEntry{emTeX}{}{2010/04/26}%
  \hologoEntry{eTeX}{}{2010/04/08}%
  \hologoEntry{ExTeX}{}{2011/10/01}%
  \hologoEntry{HanTheThanh}{}{2011/11/29}%
  \hologoEntry{iniTeX}{}{2011/10/01}%
  \hologoEntry{KOMAScript}{}{2011/10/01}%
  \hologoEntry{La}{}{2010/05/08}%
  \hologoEntry{LaTeX}{}{2010/04/08}%
  \hologoEntry{LaTeX2e}{}{2010/04/08}%
  \hologoEntry{LaTeX3}{}{2010/04/24}%
  \hologoEntry{LaTeXe}{}{2010/04/08}%
  \hologoEntry{LaTeXML}{}{2011/11/22}%
  \hologoEntry{LaTeXTeX}{}{2011/10/01}%
  \hologoEntry{LuaLaTeX}{}{2010/04/08}%
  \hologoEntry{LuaTeX}{}{2010/04/08}%
  \hologoEntry{LyX}{}{2011/10/01}%
  \hologoEntry{METAFONT}{}{2011/10/01}%
  \hologoEntry{MetaFun}{}{2011/10/01}%
  \hologoEntry{METAPOST}{}{2011/10/01}%
  \hologoEntry{MetaPost}{}{2011/10/01}%
  \hologoEntry{MiKTeX}{}{2011/10/01}%
  \hologoEntry{NTS}{}{2011/10/01}%
  \hologoEntry{OzMF}{}{2011/10/01}%
  \hologoEntry{OzMP}{}{2011/10/01}%
  \hologoEntry{OzTeX}{}{2011/10/01}%
  \hologoEntry{OzTtH}{}{2011/10/01}%
  \hologoEntry{PCTeX}{}{2011/10/01}%
  \hologoEntry{pdfTeX}{}{2011/10/01}%
  \hologoEntry{pdfLaTeX}{}{2011/10/01}%
  \hologoEntry{PiC}{}{2011/10/01}%
  \hologoEntry{PiCTeX}{}{2011/10/01}%
  \hologoEntry{plainTeX}{}{2010/04/08}%
  \hologoEntry{plainTeX}{space}{2010/04/16}%
  \hologoEntry{plainTeX}{hyphen}{2010/04/16}%
  \hologoEntry{plainTeX}{runtogether}{2010/04/16}%
  \hologoEntry{SageTeX}{}{2011/11/22}%
  \hologoEntry{SLiTeX}{}{2011/10/01}%
  \hologoEntry{SLiTeX}{lift}{2011/10/01}%
  \hologoEntry{SLiTeX}{narrow}{2011/10/01}%
  \hologoEntry{SLiTeX}{simple}{2011/10/01}%
  \hologoEntry{SliTeX}{}{2011/10/01}%
  \hologoEntry{SliTeX}{narrow}{2011/10/01}%
  \hologoEntry{SliTeX}{simple}{2011/10/01}%
  \hologoEntry{SliTeX}{lift}{2011/10/01}%
  \hologoEntry{teTeX}{}{2011/10/01}%
  \hologoEntry{TeX}{}{2010/04/08}%
  \hologoEntry{TeX4ht}{}{2011/11/22}%
  \hologoEntry{TTH}{}{2011/11/22}%
  \hologoEntry{virTeX}{}{2011/10/01}%
  \hologoEntry{VTeX}{}{2010/04/24}%
  \hologoEntry{Xe}{}{2010/04/08}%
  \hologoEntry{XeLaTeX}{}{2010/04/08}%
  \hologoEntry{XeTeX}{}{2010/04/08}%
}
%    \end{macrocode}
%    \end{macro}
%
% \subsection{Load resources}
%
%    \begin{macrocode}
\begingroup\expandafter\expandafter\expandafter\endgroup
\expandafter\ifx\csname RequirePackage\endcsname\relax
  \def\TMP@RequirePackage#1[#2]{%
    \begingroup\expandafter\expandafter\expandafter\endgroup
    \expandafter\ifx\csname ver@#1.sty\endcsname\relax
      \input #1.sty\relax
    \fi
  }%
  \TMP@RequirePackage{ltxcmds}[2011/02/04]%
  \TMP@RequirePackage{infwarerr}[2010/04/08]%
  \TMP@RequirePackage{kvsetkeys}[2010/03/01]%
  \TMP@RequirePackage{kvdefinekeys}[2010/03/01]%
  \TMP@RequirePackage{pdftexcmds}[2010/04/01]%
  \TMP@RequirePackage{ifpdf}[2010/01/28]%
  \TMP@RequirePackage{ifluatex}[2010/03/01]%
  \ltx@IfUndefined{newif}{%
    \expandafter\let\csname newif\endcsname\ltx@newif
  }{}%
  \TMP@RequirePackage{ifxetex}[2009/01/23]%
  \TMP@RequirePackage{ifvtex}[2010/03/01]%
\else
  \RequirePackage{ltxcmds}[2011/02/04]%
  \RequirePackage{infwarerr}[2010/04/08]%
  \RequirePackage{kvsetkeys}[2010/03/01]%
  \RequirePackage{kvdefinekeys}[2010/03/01]%
  \RequirePackage{pdftexcmds}[2010/04/01]%
  \RequirePackage{ifpdf}[2010/01/28]%
  \RequirePackage{ifluatex}[2010/03/01]%
  \RequirePackage{ifxetex}[2009/01/23]%
  \RequirePackage{ifvtex}[2010/03/01]%
\fi
%    \end{macrocode}
%
%    \begin{macro}{\HOLOGO@IfDefined}
%    \begin{macrocode}
\def\HOLOGO@IfExists#1{%
  \ifx\@undefined#1%
    \expandafter\ltx@secondoftwo
  \else
    \ifx\relax#1%
      \expandafter\ltx@secondoftwo
    \else
      \expandafter\expandafter\expandafter\ltx@firstoftwo
    \fi
  \fi
}
%    \end{macrocode}
%    \end{macro}
%
% \subsection{Setup macros}
%
%    \begin{macro}{\hologoSetup}
%    \begin{macrocode}
\def\hologoSetup{%
  \let\HOLOGO@name\relax
  \HOLOGO@Setup
}
%    \end{macrocode}
%    \end{macro}
%
%    \begin{macro}{\hologoLogoSetup}
%    \begin{macrocode}
\def\hologoLogoSetup#1{%
  \edef\HOLOGO@name{#1}%
  \ltx@IfUndefined{HoLogo@\HOLOGO@name}{%
    \@PackageError{hologo}{%
      Unknown logo `\HOLOGO@name'%
    }\@ehc
    \ltx@gobble
  }{%
    \HOLOGO@Setup
  }%
}
%    \end{macrocode}
%    \end{macro}
%
%    \begin{macro}{\HOLOGO@Setup}
%    \begin{macrocode}
\def\HOLOGO@Setup{%
  \kvsetkeys{HoLogo}%
}
%    \end{macrocode}
%    \end{macro}
%
% \subsection{Options}
%
%    \begin{macro}{\HOLOGO@DeclareBoolOption}
%    \begin{macrocode}
\def\HOLOGO@DeclareBoolOption#1{%
  \expandafter\chardef\csname HOLOGOOPT@#1\endcsname\ltx@zero
  \kv@define@key{HoLogo}{#1}[true]{%
    \def\HOLOGO@temp{##1}%
    \ifx\HOLOGO@temp\HOLOGO@true
      \ifx\HOLOGO@name\relax
        \expandafter\chardef\csname HOLOGOOPT@#1\endcsname=\ltx@one
      \else
        \expandafter\chardef\csname
        HoLogoOpt@#1@\HOLOGO@name\endcsname\ltx@one
      \fi
      \HOLOGO@SetBreakAll{#1}%
    \else
      \ifx\HOLOGO@temp\HOLOGO@false
        \ifx\HOLOGO@name\relax
          \expandafter\chardef\csname HOLOGOOPT@#1\endcsname=\ltx@zero
        \else
          \expandafter\chardef\csname
          HoLogoOpt@#1@\HOLOGO@name\endcsname=\ltx@zero
        \fi
        \HOLOGO@SetBreakAll{#1}%
      \else
        \@PackageError{hologo}{%
          Unknown value `##1' for boolean option `#1'.\MessageBreak
          Known values are `true' and `false'%
        }\@ehc
      \fi
    \fi
  }%
}
%    \end{macrocode}
%    \end{macro}
%
%    \begin{macro}{\HOLOGO@SetBreakAll}
%    \begin{macrocode}
\def\HOLOGO@SetBreakAll#1{%
  \def\HOLOGO@temp{#1}%
  \ifx\HOLOGO@temp\HOLOGO@break
    \ifx\HOLOGO@name\relax
      \chardef\HOLOGOOPT@hyphenbreak=\HOLOGOOPT@break
      \chardef\HOLOGOOPT@spacebreak=\HOLOGOOPT@break
      \chardef\HOLOGOOPT@discretionarybreak=\HOLOGOOPT@break
    \else
      \expandafter\chardef
         \csname HoLogoOpt@hyphenbreak@\HOLOGO@name\endcsname=%
         \csname HoLogoOpt@break@\HOLOGO@name\endcsname
      \expandafter\chardef
         \csname HoLogoOpt@spacebreak@\HOLOGO@name\endcsname=%
         \csname HoLogoOpt@break@\HOLOGO@name\endcsname
      \expandafter\chardef
         \csname HoLogoOpt@discretionarybreak@\HOLOGO@name
             \endcsname=%
         \csname HoLogoOpt@break@\HOLOGO@name\endcsname
    \fi
  \fi
}
%    \end{macrocode}
%    \end{macro}
%
%    \begin{macro}{\HOLOGO@true}
%    \begin{macrocode}
\def\HOLOGO@true{true}
%    \end{macrocode}
%    \end{macro}
%    \begin{macro}{\HOLOGO@false}
%    \begin{macrocode}
\def\HOLOGO@false{false}
%    \end{macrocode}
%    \end{macro}
%    \begin{macro}{\HOLOGO@break}
%    \begin{macrocode}
\def\HOLOGO@break{break}
%    \end{macrocode}
%    \end{macro}
%
%    \begin{macrocode}
\HOLOGO@DeclareBoolOption{break}
\HOLOGO@DeclareBoolOption{hyphenbreak}
\HOLOGO@DeclareBoolOption{spacebreak}
\HOLOGO@DeclareBoolOption{discretionarybreak}
%    \end{macrocode}
%
%    \begin{macrocode}
\kv@define@key{HoLogo}{variant}{%
  \ifx\HOLOGO@name\relax
    \@PackageError{hologo}{%
      Option `variant' is not available in \string\hologoSetup,%
      \MessageBreak
      Use \string\hologoLogoSetup\space instead%
    }\@ehc
  \else
    \edef\HOLOGO@temp{#1}%
    \ifx\HOLOGO@temp\ltx@empty
      \expandafter
      \let\csname HoLogoOpt@variant@\HOLOGO@name\endcsname\@undefined
    \else
      \ltx@IfUndefined{HoLogo@\HOLOGO@name @\HOLOGO@temp}{%
        \@PackageError{hologo}{%
          Unknown variant `\HOLOGO@temp' of logo `\HOLOGO@name'%
        }\@ehc
      }{%
        \expandafter
        \let\csname HoLogoOpt@variant@\HOLOGO@name\endcsname
            \HOLOGO@temp
      }%
    \fi
  \fi
}
%    \end{macrocode}
%
%    \begin{macro}{\HOLOGO@Variant}
%    \begin{macrocode}
\def\HOLOGO@Variant#1{%
  #1%
  \ltx@ifundefined{HoLogoOpt@variant@#1}{%
  }{%
    @\csname HoLogoOpt@variant@#1\endcsname
  }%
}
%    \end{macrocode}
%    \end{macro}
%
% \subsection{Break/no-break support}
%
%    \begin{macro}{\HOLOGO@space}
%    \begin{macrocode}
\def\HOLOGO@space{%
  \ltx@ifundefined{HoLogoOpt@spacebreak@\HOLOGO@name}{%
    \ltx@ifundefined{HoLogoOpt@break@\HOLOGO@name}{%
      \chardef\HOLOGO@temp=\HOLOGOOPT@spacebreak
    }{%
      \chardef\HOLOGO@temp=%
        \csname HoLogoOpt@break@\HOLOGO@name\endcsname
    }%
  }{%
    \chardef\HOLOGO@temp=%
      \csname HoLogoOpt@spacebreak@\HOLOGO@name\endcsname
  }%
  \ifcase\HOLOGO@temp
    \penalty10000 %
  \fi
  \ltx@space
}
%    \end{macrocode}
%    \end{macro}
%
%    \begin{macro}{\HOLOGO@hyphen}
%    \begin{macrocode}
\def\HOLOGO@hyphen{%
  \ltx@ifundefined{HoLogoOpt@hyphenbreak@\HOLOGO@name}{%
    \ltx@ifundefined{HoLogoOpt@break@\HOLOGO@name}{%
      \chardef\HOLOGO@temp=\HOLOGOOPT@hyphenbreak
    }{%
      \chardef\HOLOGO@temp=%
        \csname HoLogoOpt@break@\HOLOGO@name\endcsname
    }%
  }{%
    \chardef\HOLOGO@temp=%
      \csname HoLogoOpt@hyphenbreak@\HOLOGO@name\endcsname
  }%
  \ifcase\HOLOGO@temp
    \ltx@mbox{-}%
  \else
    -%
  \fi
}
%    \end{macrocode}
%    \end{macro}
%
%    \begin{macro}{\HOLOGO@discretionary}
%    \begin{macrocode}
\def\HOLOGO@discretionary{%
  \ltx@ifundefined{HoLogoOpt@discretionarybreak@\HOLOGO@name}{%
    \ltx@ifundefined{HoLogoOpt@break@\HOLOGO@name}{%
      \chardef\HOLOGO@temp=\HOLOGOOPT@discretionarybreak
    }{%
      \chardef\HOLOGO@temp=%
        \csname HoLogoOpt@break@\HOLOGO@name\endcsname
    }%
  }{%
    \chardef\HOLOGO@temp=%
      \csname HoLogoOpt@discretionarybreak@\HOLOGO@name\endcsname
  }%
  \ifcase\HOLOGO@temp
  \else
    \-%
  \fi
}
%    \end{macrocode}
%    \end{macro}
%
%    \begin{macro}{\HOLOGO@mbox}
%    \begin{macrocode}
\def\HOLOGO@mbox#1{%
  \ltx@ifundefined{HoLogoOpt@break@\HOLOGO@name}{%
    \chardef\HOLOGO@temp=\HOLOGOOPT@hyphenbreak
  }{%
    \chardef\HOLOGO@temp=%
      \csname HoLogoOpt@break@\HOLOGO@name\endcsname
  }%
  \ifcase\HOLOGO@temp
    \ltx@mbox{#1}%
  \else
    #1%
  \fi
}
%    \end{macrocode}
%    \end{macro}
%
% \subsection{Font support}
%
%    \begin{macro}{\HoLogoFont@font}
%    \begin{tabular}{@{}ll@{}}
%    |#1|:& logo name\\
%    |#2|:& font short name\\
%    |#3|:& text
%    \end{tabular}
%    \begin{macrocode}
\def\HoLogoFont@font#1#2#3{%
  \begingroup
    \ltx@IfUndefined{HoLogoFont@logo@#1.#2}{%
      \ltx@IfUndefined{HoLogoFont@font@#2}{%
        \@PackageWarning{hologo}{%
          Missing font `#2' for logo `#1'%
        }%
        #3%
      }{%
        \csname HoLogoFont@font@#2\endcsname{#3}%
      }%
    }{%
      \csname HoLogoFont@logo@#1.#2\endcsname{#3}%
    }%
  \endgroup
}
%    \end{macrocode}
%    \end{macro}
%
%    \begin{macro}{\HoLogoFont@Def}
%    \begin{macrocode}
\def\HoLogoFont@Def#1{%
  \expandafter\def\csname HoLogoFont@font@#1\endcsname
}
%    \end{macrocode}
%    \end{macro}
%    \begin{macro}{\HoLogoFont@LogoDef}
%    \begin{macrocode}
\def\HoLogoFont@LogoDef#1#2{%
  \expandafter\def\csname HoLogoFont@logo@#1.#2\endcsname
}
%    \end{macrocode}
%    \end{macro}
%
% \subsubsection{Font defaults}
%
%    \begin{macro}{\HoLogoFont@font@general}
%    \begin{macrocode}
\HoLogoFont@Def{general}{}%
%    \end{macrocode}
%    \end{macro}
%
%    \begin{macro}{\HoLogoFont@font@rm}
%    \begin{macrocode}
\ltx@IfUndefined{rmfamily}{%
  \ltx@IfUndefined{rm}{%
  }{%
    \HoLogoFont@Def{rm}{\rm}%
  }%
}{%
  \HoLogoFont@Def{rm}{\rmfamily}%
}
%    \end{macrocode}
%    \end{macro}
%
%    \begin{macro}{\HoLogoFont@font@sf}
%    \begin{macrocode}
\ltx@IfUndefined{sffamily}{%
  \ltx@IfUndefined{sf}{%
  }{%
    \HoLogoFont@Def{sf}{\sf}%
  }%
}{%
  \HoLogoFont@Def{sf}{\sffamily}%
}
%    \end{macrocode}
%    \end{macro}
%
%    \begin{macro}{\HoLogoFont@font@bibsf}
%    In case of \hologo{plainTeX} the original small caps
%    variant is used as default. In \hologo{LaTeX}
%    the definition of package \xpackage{dtklogos} \cite{dtklogos}
%    is used.
%\begin{quote}
%\begin{verbatim}
%\DeclareRobustCommand{\BibTeX}{%
%  B%
%  \kern-.05em%
%  \hbox{%
%    $\m@th$% %% force math size calculations
%    \csname S@\f@size\endcsname
%    \fontsize\sf@size\z@
%    \math@fontsfalse
%    \selectfont
%    I%
%    \kern-.025em%
%    B
%  }%
%  \kern-.08em%
%  \-%
%  \TeX
%}
%\end{verbatim}
%\end{quote}
%    \begin{macrocode}
\ltx@IfUndefined{selectfont}{%
  \ltx@IfUndefined{tensc}{%
    \font\tensc=cmcsc10\relax
  }{}%
  \HoLogoFont@Def{bibsf}{\tensc}%
}{%
  \HoLogoFont@Def{bibsf}{%
    $\mathsurround=0pt$%
    \csname S@\f@size\endcsname
    \fontsize\sf@size{0pt}%
    \math@fontsfalse
    \selectfont
  }%
}
%    \end{macrocode}
%    \end{macro}
%
%    \begin{macro}{\HoLogoFont@font@sc}
%    \begin{macrocode}
\ltx@IfUndefined{scshape}{%
  \ltx@IfUndefined{tensc}{%
    \font\tensc=cmcsc10\relax
  }{}%
  \HoLogoFont@Def{sc}{\tensc}%
}{%
  \HoLogoFont@Def{sc}{\scshape}%
}
%    \end{macrocode}
%    \end{macro}
%
%    \begin{macro}{\HoLogoFont@font@sy}
%    \begin{macrocode}
\ltx@IfUndefined{usefont}{%
  \ltx@IfUndefined{tensy}{%
  }{%
    \HoLogoFont@Def{sy}{\tensy}%
  }%
}{%
  \HoLogoFont@Def{sy}{%
    \usefont{OMS}{cmsy}{m}{n}%
  }%
}
%    \end{macrocode}
%    \end{macro}
%
%    \begin{macro}{\HoLogoFont@font@logo}
%    \begin{macrocode}
\begingroup
  \def\x{LaTeX2e}%
\expandafter\endgroup
\ifx\fmtname\x
  \ltx@IfUndefined{logofamily}{%
    \DeclareRobustCommand\logofamily{%
      \not@math@alphabet\logofamily\relax
      \fontencoding{U}%
      \fontfamily{logo}%
      \selectfont
    }%
  }{}%
  \ltx@IfUndefined{logofamily}{%
  }{%
    \HoLogoFont@Def{logo}{\logofamily}%
  }%
\else
  \ltx@IfUndefined{tenlogo}{%
    \font\tenlogo=logo10\relax
  }{}%
  \HoLogoFont@Def{logo}{\tenlogo}%
\fi
%    \end{macrocode}
%    \end{macro}
%
% \subsubsection{Font setup}
%
%    \begin{macro}{\hologoFontSetup}
%    \begin{macrocode}
\def\hologoFontSetup{%
  \let\HOLOGO@name\relax
  \HOLOGO@FontSetup
}
%    \end{macrocode}
%    \end{macro}
%
%    \begin{macro}{\hologoLogoFontSetup}
%    \begin{macrocode}
\def\hologoLogoFontSetup#1{%
  \edef\HOLOGO@name{#1}%
  \ltx@IfUndefined{HoLogo@\HOLOGO@name}{%
    \@PackageError{hologo}{%
      Unknown logo `\HOLOGO@name'%
    }\@ehc
    \ltx@gobble
  }{%
    \HOLOGO@FontSetup
  }%
}
%    \end{macrocode}
%    \end{macro}
%
%    \begin{macro}{\HOLOGO@FontSetup}
%    \begin{macrocode}
\def\HOLOGO@FontSetup{%
  \kvsetkeys{HoLogoFont}%
}
%    \end{macrocode}
%    \end{macro}
%
%    \begin{macrocode}
\def\HOLOGO@temp#1{%
  \kv@define@key{HoLogoFont}{#1}{%
    \ifx\HOLOGO@name\relax
      \HoLogoFont@Def{#1}{##1}%
    \else
      \HoLogoFont@LogoDef\HOLOGO@name{#1}{##1}%
    \fi
  }%
}
\HOLOGO@temp{general}
\HOLOGO@temp{sf}
%    \end{macrocode}
%
% \subsection{Generic logo commands}
%
%    \begin{macrocode}
\HOLOGO@IfExists\hologo{%
  \@PackageError{hologo}{%
    \string\hologo\ltx@space is already defined.\MessageBreak
    Package loading is aborted%
  }\@ehc
  \HOLOGO@AtEnd
}%
\HOLOGO@IfExists\hologoRobust{%
  \@PackageError{hologo}{%
    \string\hologoRobust\ltx@space is already defined.\MessageBreak
    Package loading is aborted%
  }\@ehc
  \HOLOGO@AtEnd
}%
%    \end{macrocode}
%
% \subsubsection{\cs{hologo} and friends}
%
%    \begin{macrocode}
\ifluatex
  \expandafter\ltx@firstofone
\else
  \expandafter\ltx@gobble
\fi
{%
  \ltx@IfUndefined{ifincsname}{%
    \ifnum\luatexversion<36 %
      \expandafter\ltx@gobble
    \else
      \expandafter\ltx@firstofone
    \fi
    {%
      \begingroup
        \ifcase0%
            \directlua{%
              if tex.enableprimitives then %
                tex.enableprimitives('HOLOGO@', {'ifincsname'})%
              else %
                tex.print('1')%
              end%
            }%
            \ifx\HOLOGO@ifincsname\@undefined 1\fi%
            \relax
          \expandafter\ltx@firstofone
        \else
          \endgroup
          \expandafter\ltx@gobble
        \fi
        {%
          \global\let\ifincsname\HOLOGO@ifincsname
        }%
      \HOLOGO@temp
    }%
  }{}%
}
%    \end{macrocode}
%    \begin{macrocode}
\ltx@IfUndefined{ifincsname}{%
  \catcode`$=14 %
}{%
  \catcode`$=9 %
}
%    \end{macrocode}
%
%    \begin{macro}{\hologo}
%    \begin{macrocode}
\def\hologo#1{%
$ \ifincsname
$   \ltx@ifundefined{HoLogoCs@\HOLOGO@Variant{#1}}{%
$     #1%
$   }{%
$     \csname HoLogoCs@\HOLOGO@Variant{#1}\endcsname\ltx@firstoftwo
$   }%
$ \else
    \HOLOGO@IfExists\texorpdfstring\texorpdfstring\ltx@firstoftwo
    {%
      \hologoRobust{#1}%
    }{%
      \ltx@ifundefined{HoLogoBkm@\HOLOGO@Variant{#1}}{%
        \ltx@ifundefined{HoLogo@#1}{?#1?}{#1}%
      }{%
        \csname HoLogoBkm@\HOLOGO@Variant{#1}\endcsname
        \ltx@firstoftwo
      }%
    }%
$ \fi
}
%    \end{macrocode}
%    \end{macro}
%    \begin{macro}{\Hologo}
%    \begin{macrocode}
\def\Hologo#1{%
$ \ifincsname
$   \ltx@ifundefined{HoLogoCs@\HOLOGO@Variant{#1}}{%
$     #1%
$   }{%
$     \csname HoLogoCs@\HOLOGO@Variant{#1}\endcsname\ltx@secondoftwo
$   }%
$ \else
    \HOLOGO@IfExists\texorpdfstring\texorpdfstring\ltx@firstoftwo
    {%
      \HologoRobust{#1}%
    }{%
      \ltx@ifundefined{HoLogoBkm@\HOLOGO@Variant{#1}}{%
        \ltx@ifundefined{HoLogo@#1}{?#1?}{#1}%
      }{%
        \csname HoLogoBkm@\HOLOGO@Variant{#1}\endcsname
        \ltx@secondoftwo
      }%
    }%
$ \fi
}
%    \end{macrocode}
%    \end{macro}
%
%    \begin{macro}{\hologoVariant}
%    \begin{macrocode}
\def\hologoVariant#1#2{%
  \ifx\relax#2\relax
    \hologo{#1}%
  \else
$   \ifincsname
$     \ltx@ifundefined{HoLogoCs@#1@#2}{%
$       #1%
$     }{%
$       \csname HoLogoCs@#1@#2\endcsname\ltx@firstoftwo
$     }%
$   \else
      \HOLOGO@IfExists\texorpdfstring\texorpdfstring\ltx@firstoftwo
      {%
        \hologoVariantRobust{#1}{#2}%
      }{%
        \ltx@ifundefined{HoLogoBkm@#1@#2}{%
          \ltx@ifundefined{HoLogo@#1}{?#1?}{#1}%
        }{%
          \csname HoLogoBkm@#1@#2\endcsname
          \ltx@firstoftwo
        }%
      }%
$   \fi
  \fi
}
%    \end{macrocode}
%    \end{macro}
%    \begin{macro}{\HologoVariant}
%    \begin{macrocode}
\def\HologoVariant#1#2{%
  \ifx\relax#2\relax
    \Hologo{#1}%
  \else
$   \ifincsname
$     \ltx@ifundefined{HoLogoCs@#1@#2}{%
$       #1%
$     }{%
$       \csname HoLogoCs@#1@#2\endcsname\ltx@secondoftwo
$     }%
$   \else
      \HOLOGO@IfExists\texorpdfstring\texorpdfstring\ltx@firstoftwo
      {%
        \HologoVariantRobust{#1}{#2}%
      }{%
        \ltx@ifundefined{HoLogoBkm@#1@#2}{%
          \ltx@ifundefined{HoLogo@#1}{?#1?}{#1}%
        }{%
          \csname HoLogoBkm@#1@#2\endcsname
          \ltx@secondoftwo
        }%
      }%
$   \fi
  \fi
}
%    \end{macrocode}
%    \end{macro}
%
%    \begin{macrocode}
\catcode`\$=3 %
%    \end{macrocode}
%
% \subsubsection{\cs{hologoRobust} and friends}
%
%    \begin{macro}{\hologoRobust}
%    \begin{macrocode}
\ltx@IfUndefined{protected}{%
  \ltx@IfUndefined{DeclareRobustCommand}{%
    \def\hologoRobust#1%
  }{%
    \DeclareRobustCommand*\hologoRobust[1]%
  }%
}{%
  \protected\def\hologoRobust#1%
}%
{%
  \edef\HOLOGO@name{#1}%
  \ltx@IfUndefined{HoLogo@\HOLOGO@Variant\HOLOGO@name}{%
    \@PackageError{hologo}{%
      Unknown logo `\HOLOGO@name'%
    }\@ehc
    ?\HOLOGO@name?%
  }{%
    \ltx@IfUndefined{ver@tex4ht.sty}{%
      \HoLogoFont@font\HOLOGO@name{general}{%
        \csname HoLogo@\HOLOGO@Variant\HOLOGO@name\endcsname
        \ltx@firstoftwo
      }%
    }{%
      \ltx@IfUndefined{HoLogoHtml@\HOLOGO@Variant\HOLOGO@name}{%
        \HOLOGO@name
      }{%
        \csname HoLogoHtml@\HOLOGO@Variant\HOLOGO@name\endcsname
        \ltx@firstoftwo
      }%
    }%
  }%
}
%    \end{macrocode}
%    \end{macro}
%    \begin{macro}{\HologoRobust}
%    \begin{macrocode}
\ltx@IfUndefined{protected}{%
  \ltx@IfUndefined{DeclareRobustCommand}{%
    \def\HologoRobust#1%
  }{%
    \DeclareRobustCommand*\HologoRobust[1]%
  }%
}{%
  \protected\def\HologoRobust#1%
}%
{%
  \edef\HOLOGO@name{#1}%
  \ltx@IfUndefined{HoLogo@\HOLOGO@Variant\HOLOGO@name}{%
    \@PackageError{hologo}{%
      Unknown logo `\HOLOGO@name'%
    }\@ehc
    ?\HOLOGO@name?%
  }{%
    \ltx@IfUndefined{ver@tex4ht.sty}{%
      \HoLogoFont@font\HOLOGO@name{general}{%
        \csname HoLogo@\HOLOGO@Variant\HOLOGO@name\endcsname
        \ltx@secondoftwo
      }%
    }{%
      \ltx@IfUndefined{HoLogoHtml@\HOLOGO@Variant\HOLOGO@name}{%
        \expandafter\HOLOGO@Uppercase\HOLOGO@name
      }{%
        \csname HoLogoHtml@\HOLOGO@Variant\HOLOGO@name\endcsname
        \ltx@secondoftwo
      }%
    }%
  }%
}
%    \end{macrocode}
%    \end{macro}
%    \begin{macro}{\hologoVariantRobust}
%    \begin{macrocode}
\ltx@IfUndefined{protected}{%
  \ltx@IfUndefined{DeclareRobustCommand}{%
    \def\hologoVariantRobust#1#2%
  }{%
    \DeclareRobustCommand*\hologoVariantRobust[2]%
  }%
}{%
  \protected\def\hologoVariantRobust#1#2%
}%
{%
  \begingroup
    \hologoLogoSetup{#1}{variant={#2}}%
    \hologoRobust{#1}%
  \endgroup
}
%    \end{macrocode}
%    \end{macro}
%    \begin{macro}{\HologoVariantRobust}
%    \begin{macrocode}
\ltx@IfUndefined{protected}{%
  \ltx@IfUndefined{DeclareRobustCommand}{%
    \def\HologoVariantRobust#1#2%
  }{%
    \DeclareRobustCommand*\HologoVariantRobust[2]%
  }%
}{%
  \protected\def\HologoVariantRobust#1#2%
}%
{%
  \begingroup
    \hologoLogoSetup{#1}{variant={#2}}%
    \HologoRobust{#1}%
  \endgroup
}
%    \end{macrocode}
%    \end{macro}
%
%    \begin{macro}{\hologorobust}
%    Macro \cs{hologorobust} is only defined for compatibility.
%    Its use is deprecated.
%    \begin{macrocode}
\def\hologorobust{\hologoRobust}
%    \end{macrocode}
%    \end{macro}
%
% \subsection{Helpers}
%
%    \begin{macro}{\HOLOGO@Uppercase}
%    Macro \cs{HOLOGO@Uppercase} is restricted to \cs{uppercase},
%    because \hologo{plainTeX} or \hologo{iniTeX} do not provide
%    \cs{MakeUppercase}.
%    \begin{macrocode}
\def\HOLOGO@Uppercase#1{\uppercase{#1}}
%    \end{macrocode}
%    \end{macro}
%
%    \begin{macro}{\HOLOGO@PdfdocUnicode}
%    \begin{macrocode}
\def\HOLOGO@PdfdocUnicode{%
  \ifx\ifHy@unicode\iftrue
    \expandafter\ltx@secondoftwo
  \else
    \expandafter\ltx@firstoftwo
  \fi
}
%    \end{macrocode}
%    \end{macro}
%
%    \begin{macro}{\HOLOGO@Math}
%    \begin{macrocode}
\def\HOLOGO@MathSetup{%
  \mathsurround0pt\relax
  \HOLOGO@IfExists\f@series{%
    \if b\expandafter\ltx@car\f@series x\@nil
      \csname boldmath\endcsname
   \fi
  }{}%
}
%    \end{macrocode}
%    \end{macro}
%
%    \begin{macro}{\HOLOGO@TempDimen}
%    \begin{macrocode}
\dimendef\HOLOGO@TempDimen=\ltx@zero
%    \end{macrocode}
%    \end{macro}
%    \begin{macro}{\HOLOGO@NegativeKerning}
%    \begin{macrocode}
\def\HOLOGO@NegativeKerning#1{%
  \begingroup
    \HOLOGO@TempDimen=0pt\relax
    \comma@parse@normalized{#1}{%
      \ifdim\HOLOGO@TempDimen=0pt %
        \expandafter\HOLOGO@@NegativeKerning\comma@entry
      \fi
      \ltx@gobble
    }%
    \ifdim\HOLOGO@TempDimen<0pt %
      \kern\HOLOGO@TempDimen
    \fi
  \endgroup
}
%    \end{macrocode}
%    \end{macro}
%    \begin{macro}{\HOLOGO@@NegativeKerning}
%    \begin{macrocode}
\def\HOLOGO@@NegativeKerning#1#2{%
  \setbox\ltx@zero\hbox{#1#2}%
  \HOLOGO@TempDimen=\wd\ltx@zero
  \setbox\ltx@zero\hbox{#1\kern0pt#2}%
  \advance\HOLOGO@TempDimen by -\wd\ltx@zero
}
%    \end{macrocode}
%    \end{macro}
%
%    \begin{macro}{\HOLOGO@SpaceFactor}
%    \begin{macrocode}
\def\HOLOGO@SpaceFactor{%
  \spacefactor1000 %
}
%    \end{macrocode}
%    \end{macro}
%
%    \begin{macro}{\HOLOGO@Span}
%    \begin{macrocode}
\def\HOLOGO@Span#1#2{%
  \HCode{<span class="HoLogo-#1">}%
  #2%
  \HCode{</span>}%
}
%    \end{macrocode}
%    \end{macro}
%
% \subsubsection{Text subscript}
%
%    \begin{macro}{\HOLOGO@SubScript}%
%    \begin{macrocode}
\def\HOLOGO@SubScript#1{%
  \ltx@IfUndefined{textsubscript}{%
    \ltx@IfUndefined{text}{%
      \ltx@mbox{%
        \mathsurround=0pt\relax
        $%
          _{%
            \ltx@IfUndefined{sf@size}{%
              \mathrm{#1}%
            }{%
              \mbox{%
                \fontsize\sf@size{0pt}\selectfont
                #1%
              }%
            }%
          }%
        $%
      }%
    }{%
      \ltx@mbox{%
        \mathsurround=0pt\relax
        $_{\text{#1}}$%
      }%
    }%
  }{%
    \textsubscript{#1}%
  }%
}
%    \end{macrocode}
%    \end{macro}
%
% \subsection{\hologo{TeX} and friends}
%
% \subsubsection{\hologo{TeX}}
%
%    \begin{macro}{\HoLogo@TeX}
%    Source: \hologo{LaTeX} kernel.
%    \begin{macrocode}
\def\HoLogo@TeX#1{%
  T\kern-.1667em\lower.5ex\hbox{E}\kern-.125emX\HOLOGO@SpaceFactor
}
%    \end{macrocode}
%    \end{macro}
%    \begin{macro}{\HoLogoHtml@TeX}
%    \begin{macrocode}
\def\HoLogoHtml@TeX#1{%
  \HoLogoCss@TeX
  \HOLOGO@Span{TeX}{%
    T%
    \HOLOGO@Span{e}{%
      E%
    }%
    X%
  }%
}
%    \end{macrocode}
%    \end{macro}
%    \begin{macro}{\HoLogoCss@TeX}
%    \begin{macrocode}
\def\HoLogoCss@TeX{%
  \Css{%
    span.HoLogo-TeX span.HoLogo-e{%
      position:relative;%
      top:.5ex;%
      margin-left:-.1667em;%
      margin-right:-.125em;%
    }%
  }%
  \Css{%
    a span.HoLogo-TeX span.HoLogo-e{%
      text-decoration:none;%
    }%
  }%
  \global\let\HoLogoCss@TeX\relax
}
%    \end{macrocode}
%    \end{macro}
%
% \subsubsection{\hologo{plainTeX}}
%
%    \begin{macro}{\HoLogo@plainTeX@space}
%    Source: ``The \hologo{TeX}book''
%    \begin{macrocode}
\def\HoLogo@plainTeX@space#1{%
  \HOLOGO@mbox{#1{p}{P}lain}\HOLOGO@space\hologo{TeX}%
}
%    \end{macrocode}
%    \end{macro}
%    \begin{macro}{\HoLogoCs@plainTeX@space}
%    \begin{macrocode}
\def\HoLogoCs@plainTeX@space#1{#1{p}{P}lain TeX}%
%    \end{macrocode}
%    \end{macro}
%    \begin{macro}{\HoLogoBkm@plainTeX@space}
%    \begin{macrocode}
\def\HoLogoBkm@plainTeX@space#1{%
  #1{p}{P}lain \hologo{TeX}%
}
%    \end{macrocode}
%    \end{macro}
%    \begin{macro}{\HoLogoHtml@plainTeX@space}
%    \begin{macrocode}
\def\HoLogoHtml@plainTeX@space#1{%
  #1{p}{P}lain \hologo{TeX}%
}
%    \end{macrocode}
%    \end{macro}
%
%    \begin{macro}{\HoLogo@plainTeX@hyphen}
%    \begin{macrocode}
\def\HoLogo@plainTeX@hyphen#1{%
  \HOLOGO@mbox{#1{p}{P}lain}\HOLOGO@hyphen\hologo{TeX}%
}
%    \end{macrocode}
%    \end{macro}
%    \begin{macro}{\HoLogoCs@plainTeX@hyphen}
%    \begin{macrocode}
\def\HoLogoCs@plainTeX@hyphen#1{#1{p}{P}lain-TeX}
%    \end{macrocode}
%    \end{macro}
%    \begin{macro}{\HoLogoBkm@plainTeX@hyphen}
%    \begin{macrocode}
\def\HoLogoBkm@plainTeX@hyphen#1{%
  #1{p}{P}lain-\hologo{TeX}%
}
%    \end{macrocode}
%    \end{macro}
%    \begin{macro}{\HoLogoHtml@plainTeX@hyphen}
%    \begin{macrocode}
\def\HoLogoHtml@plainTeX@hyphen#1{%
  #1{p}{P}lain-\hologo{TeX}%
}
%    \end{macrocode}
%    \end{macro}
%
%    \begin{macro}{\HoLogo@plainTeX@runtogether}
%    \begin{macrocode}
\def\HoLogo@plainTeX@runtogether#1{%
  \HOLOGO@mbox{#1{p}{P}lain\hologo{TeX}}%
}
%    \end{macrocode}
%    \end{macro}
%    \begin{macro}{\HoLogoCs@plainTeX@runtogether}
%    \begin{macrocode}
\def\HoLogoCs@plainTeX@runtogether#1{#1{p}{P}lainTeX}
%    \end{macrocode}
%    \end{macro}
%    \begin{macro}{\HoLogoBkm@plainTeX@runtogether}
%    \begin{macrocode}
\def\HoLogoBkm@plainTeX@runtogether#1{%
  #1{p}{P}lain\hologo{TeX}%
}
%    \end{macrocode}
%    \end{macro}
%    \begin{macro}{\HoLogoHtml@plainTeX@runtogether}
%    \begin{macrocode}
\def\HoLogoHtml@plainTeX@runtogether#1{%
  #1{p}{P}lain\hologo{TeX}%
}
%    \end{macrocode}
%    \end{macro}
%
%    \begin{macro}{\HoLogo@plainTeX}
%    \begin{macrocode}
\def\HoLogo@plainTeX{\HoLogo@plainTeX@space}
%    \end{macrocode}
%    \end{macro}
%    \begin{macro}{\HoLogoCs@plainTeX}
%    \begin{macrocode}
\def\HoLogoCs@plainTeX{\HoLogoCs@plainTeX@space}
%    \end{macrocode}
%    \end{macro}
%    \begin{macro}{\HoLogoBkm@plainTeX}
%    \begin{macrocode}
\def\HoLogoBkm@plainTeX{\HoLogoBkm@plainTeX@space}
%    \end{macrocode}
%    \end{macro}
%    \begin{macro}{\HoLogoHtml@plainTeX}
%    \begin{macrocode}
\def\HoLogoHtml@plainTeX{\HoLogoHtml@plainTeX@space}
%    \end{macrocode}
%    \end{macro}
%
% \subsubsection{\hologo{LaTeX}}
%
%    Source: \hologo{LaTeX} kernel.
%\begin{quote}
%\begin{verbatim}
%\DeclareRobustCommand{\LaTeX}{%
%  L%
%  \kern-.36em%
%  {%
%    \sbox\z@ T%
%    \vbox to\ht\z@{%
%      \hbox{%
%        \check@mathfonts
%        \fontsize\sf@size\z@
%        \math@fontsfalse
%        \selectfont
%        A%
%      }%
%      \vss
%    }%
%  }%
%  \kern-.15em%
%  \TeX
%}
%\end{verbatim}
%\end{quote}
%
%    \begin{macro}{\HoLogo@La}
%    \begin{macrocode}
\def\HoLogo@La#1{%
  L%
  \kern-.36em%
  \begingroup
    \setbox\ltx@zero\hbox{T}%
    \vbox to\ht\ltx@zero{%
      \hbox{%
        \ltx@ifundefined{check@mathfonts}{%
          \csname sevenrm\endcsname
        }{%
          \check@mathfonts
          \fontsize\sf@size{0pt}%
          \math@fontsfalse\selectfont
        }%
        A%
      }%
      \vss
    }%
  \endgroup
}
%    \end{macrocode}
%    \end{macro}
%
%    \begin{macro}{\HoLogo@LaTeX}
%    Source: \hologo{LaTeX} kernel.
%    \begin{macrocode}
\def\HoLogo@LaTeX#1{%
  \hologo{La}%
  \kern-.15em%
  \hologo{TeX}%
}
%    \end{macrocode}
%    \end{macro}
%    \begin{macro}{\HoLogoHtml@LaTeX}
%    \begin{macrocode}
\def\HoLogoHtml@LaTeX#1{%
  \HoLogoCss@LaTeX
  \HOLOGO@Span{LaTeX}{%
    L%
    \HOLOGO@Span{a}{%
      A%
    }%
    \hologo{TeX}%
  }%
}
%    \end{macrocode}
%    \end{macro}
%    \begin{macro}{\HoLogoCss@LaTeX}
%    \begin{macrocode}
\def\HoLogoCss@LaTeX{%
  \Css{%
    span.HoLogo-LaTeX span.HoLogo-a{%
      position:relative;%
      top:-.5ex;%
      margin-left:-.36em;%
      margin-right:-.15em;%
      font-size:85\%;%
    }%
  }%
  \global\let\HoLogoCss@LaTeX\relax
}
%    \end{macrocode}
%    \end{macro}
%
% \subsubsection{\hologo{(La)TeX}}
%
%    \begin{macro}{\HoLogo@LaTeXTeX}
%    The kerning around the parentheses is taken
%    from package \xpackage{dtklogos} \cite{dtklogos}.
%\begin{quote}
%\begin{verbatim}
%\DeclareRobustCommand{\LaTeXTeX}{%
%  (%
%  \kern-.15em%
%  L%
%  \kern-.36em%
%  {%
%    \sbox\z@ T%
%    \vbox to\ht0{%
%      \hbox{%
%        $\m@th$%
%        \csname S@\f@size\endcsname
%        \fontsize\sf@size\z@
%        \math@fontsfalse
%        \selectfont
%        A%
%      }%
%      \vss
%    }%
%  }%
%  \kern-.2em%
%  )%
%  \kern-.15em%
%  \TeX
%}
%\end{verbatim}
%\end{quote}
%    \begin{macrocode}
\def\HoLogo@LaTeXTeX#1{%
  (%
  \kern-.15em%
  \hologo{La}%
  \kern-.2em%
  )%
  \kern-.15em%
  \hologo{TeX}%
}
%    \end{macrocode}
%    \end{macro}
%    \begin{macro}{\HoLogoBkm@LaTeXTeX}
%    \begin{macrocode}
\def\HoLogoBkm@LaTeXTeX#1{(La)TeX}
%    \end{macrocode}
%    \end{macro}
%
%    \begin{macro}{\HoLogo@(La)TeX}
%    \begin{macrocode}
\expandafter
\let\csname HoLogo@(La)TeX\endcsname\HoLogo@LaTeXTeX
%    \end{macrocode}
%    \end{macro}
%    \begin{macro}{\HoLogoBkm@(La)TeX}
%    \begin{macrocode}
\expandafter
\let\csname HoLogoBkm@(La)TeX\endcsname\HoLogoBkm@LaTeXTeX
%    \end{macrocode}
%    \end{macro}
%    \begin{macro}{\HoLogoHtml@LaTeXTeX}
%    \begin{macrocode}
\def\HoLogoHtml@LaTeXTeX#1{%
  \HoLogoCss@LaTeXTeX
  \HOLOGO@Span{LaTeXTeX}{%
    (%
    \HOLOGO@Span{L}{L}%
    \HOLOGO@Span{a}{A}%
    \HOLOGO@Span{ParenRight}{)}%
    \hologo{TeX}%
  }%
}
%    \end{macrocode}
%    \end{macro}
%    \begin{macro}{\HoLogoHtml@(La)TeX}
%    Kerning after opening parentheses and before closing parentheses
%    is $-0.1$\,em. The original values $-0.15$\,em
%    looked too ugly for a serif font.
%    \begin{macrocode}
\expandafter
\let\csname HoLogoHtml@(La)TeX\endcsname\HoLogoHtml@LaTeXTeX
%    \end{macrocode}
%    \end{macro}
%    \begin{macro}{\HoLogoCss@LaTeXTeX}
%    \begin{macrocode}
\def\HoLogoCss@LaTeXTeX{%
  \Css{%
    span.HoLogo-LaTeXTeX span.HoLogo-L{%
      margin-left:-.1em;%
    }%
  }%
  \Css{%
    span.HoLogo-LaTeXTeX span.HoLogo-a{%
      position:relative;%
      top:-.5ex;%
      margin-left:-.36em;%
      margin-right:-.1em;%
      font-size:85\%;%
    }%
  }%
  \Css{%
    span.HoLogo-LaTeXTeX span.HoLogo-ParenRight{%
      margin-right:-.15em;%
    }%
  }%
  \global\let\HoLogoCss@LaTeXTeX\relax
}
%    \end{macrocode}
%    \end{macro}
%
% \subsubsection{\hologo{LaTeXe}}
%
%    \begin{macro}{\HoLogo@LaTeXe}
%    Source: \hologo{LaTeX} kernel
%    \begin{macrocode}
\def\HoLogo@LaTeXe#1{%
  \hologo{LaTeX}%
  \kern.15em%
  \hbox{%
    \HOLOGO@MathSetup
    2%
    $_{\textstyle\varepsilon}$%
  }%
}
%    \end{macrocode}
%    \end{macro}
%
%    \begin{macro}{\HoLogoCs@LaTeXe}
%    \begin{macrocode}
\ifnum64=`\^^^^0040\relax % test for big chars of LuaTeX/XeTeX
  \catcode`\$=9 %
  \catcode`\&=14 %
\else
  \catcode`\$=14 %
  \catcode`\&=9 %
\fi
\def\HoLogoCs@LaTeXe#1{%
  LaTeX2%
$ \string ^^^^0395%
& e%
}%
\catcode`\$=3 %
\catcode`\&=4 %
%    \end{macrocode}
%    \end{macro}
%
%    \begin{macro}{\HoLogoBkm@LaTeXe}
%    \begin{macrocode}
\def\HoLogoBkm@LaTeXe#1{%
  \hologo{LaTeX}%
  2%
  \HOLOGO@PdfdocUnicode{e}{\textepsilon}%
}
%    \end{macrocode}
%    \end{macro}
%
%    \begin{macro}{\HoLogoHtml@LaTeXe}
%    \begin{macrocode}
\def\HoLogoHtml@LaTeXe#1{%
  \HoLogoCss@LaTeXe
  \HOLOGO@Span{LaTeX2e}{%
    \hologo{LaTeX}%
    \HOLOGO@Span{2}{2}%
    \HOLOGO@Span{e}{%
      \HOLOGO@MathSetup
      \ensuremath{\textstyle\varepsilon}%
    }%
  }%
}
%    \end{macrocode}
%    \end{macro}
%    \begin{macro}{\HoLogoCss@LaTeXe}
%    \begin{macrocode}
\def\HoLogoCss@LaTeXe{%
  \Css{%
    span.HoLogo-LaTeX2e span.HoLogo-2{%
      padding-left:.15em;%
    }%
  }%
  \Css{%
    span.HoLogo-LaTeX2e span.HoLogo-e{%
      position:relative;%
      top:.35ex;%
      text-decoration:none;%
    }%
  }%
  \global\let\HoLogoCss@LaTeXe\relax
}
%    \end{macrocode}
%    \end{macro}
%
%    \begin{macro}{\HoLogo@LaTeX2e}
%    \begin{macrocode}
\expandafter
\let\csname HoLogo@LaTeX2e\endcsname\HoLogo@LaTeXe
%    \end{macrocode}
%    \end{macro}
%    \begin{macro}{\HoLogoCs@LaTeX2e}
%    \begin{macrocode}
\expandafter
\let\csname HoLogoCs@LaTeX2e\endcsname\HoLogoCs@LaTeXe
%    \end{macrocode}
%    \end{macro}
%    \begin{macro}{\HoLogoBkm@LaTeX2e}
%    \begin{macrocode}
\expandafter
\let\csname HoLogoBkm@LaTeX2e\endcsname\HoLogoBkm@LaTeXe
%    \end{macrocode}
%    \end{macro}
%    \begin{macro}{\HoLogoHtml@LaTeX2e}
%    \begin{macrocode}
\expandafter
\let\csname HoLogoHtml@LaTeX2e\endcsname\HoLogoHtml@LaTeXe
%    \end{macrocode}
%    \end{macro}
%
% \subsubsection{\hologo{LaTeX3}}
%
%    \begin{macro}{\HoLogo@LaTeX3}
%    Source: \hologo{LaTeX} kernel
%    \begin{macrocode}
\expandafter\def\csname HoLogo@LaTeX3\endcsname#1{%
  \hologo{LaTeX}%
  3%
}
%    \end{macrocode}
%    \end{macro}
%
%    \begin{macro}{\HoLogoBkm@LaTeX3}
%    \begin{macrocode}
\expandafter\def\csname HoLogoBkm@LaTeX3\endcsname#1{%
  \hologo{LaTeX}%
  3%
}
%    \end{macrocode}
%    \end{macro}
%    \begin{macro}{\HoLogoHtml@LaTeX3}
%    \begin{macrocode}
\expandafter
\let\csname HoLogoHtml@LaTeX3\expandafter\endcsname
\csname HoLogo@LaTeX3\endcsname
%    \end{macrocode}
%    \end{macro}
%
% \subsubsection{\hologo{LaTeXML}}
%
%    \begin{macro}{\HoLogo@LaTeXML}
%    \begin{macrocode}
\def\HoLogo@LaTeXML#1{%
  \HOLOGO@mbox{%
    \hologo{La}%
    \kern-.15em%
    T%
    \kern-.1667em%
    \lower.5ex\hbox{E}%
    \kern-.125em%
    \HoLogoFont@font{LaTeXML}{sc}{xml}%
  }%
}
%    \end{macrocode}
%    \end{macro}
%    \begin{macro}{\HoLogoHtml@pdfLaTeX}
%    \begin{macrocode}
\def\HoLogoHtml@LaTeXML#1{%
  \HOLOGO@Span{LaTeXML}{%
    \HoLogoCss@LaTeX
    \HoLogoCss@TeX
    \HOLOGO@Span{LaTeX}{%
      L%
      \HOLOGO@Span{a}{%
        A%
      }%
    }%
    \HOLOGO@Span{TeX}{%
      T%
      \HOLOGO@Span{e}{%
        E%
      }%
    }%
    \HCode{<span style="font-variant: small-caps;">}%
    xml%
    \HCode{</span>}%
  }%
}
%    \end{macrocode}
%    \end{macro}
%
% \subsubsection{\hologo{eTeX}}
%
%    \begin{macro}{\HoLogo@eTeX}
%    Source: package \xpackage{etex}
%    \begin{macrocode}
\def\HoLogo@eTeX#1{%
  \ltx@mbox{%
    \HOLOGO@MathSetup
    $\varepsilon$%
    -%
    \HOLOGO@NegativeKerning{-T,T-,To}%
    \hologo{TeX}%
  }%
}
%    \end{macrocode}
%    \end{macro}
%    \begin{macro}{\HoLogoCs@eTeX}
%    \begin{macrocode}
\ifnum64=`\^^^^0040\relax % test for big chars of LuaTeX/XeTeX
  \catcode`\$=9 %
  \catcode`\&=14 %
\else
  \catcode`\$=14 %
  \catcode`\&=9 %
\fi
\def\HoLogoCs@eTeX#1{%
$ #1{\string ^^^^0395}{\string ^^^^03b5}%
& #1{e}{E}%
  TeX%
}%
\catcode`\$=3 %
\catcode`\&=4 %
%    \end{macrocode}
%    \end{macro}
%    \begin{macro}{\HoLogoBkm@eTeX}
%    \begin{macrocode}
\def\HoLogoBkm@eTeX#1{%
  \HOLOGO@PdfdocUnicode{#1{e}{E}}{\textepsilon}%
  -%
  \hologo{TeX}%
}
%    \end{macrocode}
%    \end{macro}
%    \begin{macro}{\HoLogoHtml@eTeX}
%    \begin{macrocode}
\def\HoLogoHtml@eTeX#1{%
  \ltx@mbox{%
    \HOLOGO@MathSetup
    $\varepsilon$%
    -%
    \hologo{TeX}%
  }%
}
%    \end{macrocode}
%    \end{macro}
%
% \subsubsection{\hologo{iniTeX}}
%
%    \begin{macro}{\HoLogo@iniTeX}
%    \begin{macrocode}
\def\HoLogo@iniTeX#1{%
  \HOLOGO@mbox{%
    #1{i}{I}ni\hologo{TeX}%
  }%
}
%    \end{macrocode}
%    \end{macro}
%    \begin{macro}{\HoLogoCs@iniTeX}
%    \begin{macrocode}
\def\HoLogoCs@iniTeX#1{#1{i}{I}niTeX}
%    \end{macrocode}
%    \end{macro}
%    \begin{macro}{\HoLogoBkm@iniTeX}
%    \begin{macrocode}
\def\HoLogoBkm@iniTeX#1{%
  #1{i}{I}ni\hologo{TeX}%
}
%    \end{macrocode}
%    \end{macro}
%    \begin{macro}{\HoLogoHtml@iniTeX}
%    \begin{macrocode}
\let\HoLogoHtml@iniTeX\HoLogo@iniTeX
%    \end{macrocode}
%    \end{macro}
%
% \subsubsection{\hologo{virTeX}}
%
%    \begin{macro}{\HoLogo@virTeX}
%    \begin{macrocode}
\def\HoLogo@virTeX#1{%
  \HOLOGO@mbox{%
    #1{v}{V}ir\hologo{TeX}%
  }%
}
%    \end{macrocode}
%    \end{macro}
%    \begin{macro}{\HoLogoCs@virTeX}
%    \begin{macrocode}
\def\HoLogoCs@virTeX#1{#1{v}{V}irTeX}
%    \end{macrocode}
%    \end{macro}
%    \begin{macro}{\HoLogoBkm@virTeX}
%    \begin{macrocode}
\def\HoLogoBkm@virTeX#1{%
  #1{v}{V}ir\hologo{TeX}%
}
%    \end{macrocode}
%    \end{macro}
%    \begin{macro}{\HoLogoHtml@virTeX}
%    \begin{macrocode}
\let\HoLogoHtml@virTeX\HoLogo@virTeX
%    \end{macrocode}
%    \end{macro}
%
% \subsubsection{\hologo{SliTeX}}
%
% \paragraph{Definitions of the three variants.}
%
%    \begin{macro}{\HoLogo@SLiTeX@lift}
%    \begin{macrocode}
\def\HoLogo@SLiTeX@lift#1{%
  \HoLogoFont@font{SliTeX}{rm}{%
    S%
    \kern-.06em%
    L%
    \kern-.18em%
    \raise.32ex\hbox{\HoLogoFont@font{SliTeX}{sc}{i}}%
    \HOLOGO@discretionary
    \kern-.06em%
    \hologo{TeX}%
  }%
}
%    \end{macrocode}
%    \end{macro}
%    \begin{macro}{\HoLogoBkm@SLiTeX@lift}
%    \begin{macrocode}
\def\HoLogoBkm@SLiTeX@lift#1{SLiTeX}
%    \end{macrocode}
%    \end{macro}
%    \begin{macro}{\HoLogoHtml@SLiTeX@lift}
%    \begin{macrocode}
\def\HoLogoHtml@SLiTeX@lift#1{%
  \HoLogoCss@SLiTeX@lift
  \HOLOGO@Span{SLiTeX-lift}{%
    \HoLogoFont@font{SliTeX}{rm}{%
      S%
      \HOLOGO@Span{L}{L}%
      \HOLOGO@Span{i}{i}%
      \hologo{TeX}%
    }%
  }%
}
%    \end{macrocode}
%    \end{macro}
%    \begin{macro}{\HoLogoCss@SLiTeX@lift}
%    \begin{macrocode}
\def\HoLogoCss@SLiTeX@lift{%
  \Css{%
    span.HoLogo-SLiTeX-lift span.HoLogo-L{%
      margin-left:-.06em;%
      margin-right:-.18em;%
    }%
  }%
  \Css{%
    span.HoLogo-SLiTeX-lift span.HoLogo-i{%
      position:relative;%
      top:-.32ex;%
      margin-right:-.06em;%
      font-variant:small-caps;%
    }%
  }%
  \global\let\HoLogoCss@SLiTeX@lift\relax
}
%    \end{macrocode}
%    \end{macro}
%
%    \begin{macro}{\HoLogo@SliTeX@simple}
%    \begin{macrocode}
\def\HoLogo@SliTeX@simple#1{%
  \HoLogoFont@font{SliTeX}{rm}{%
    \ltx@mbox{%
      \HoLogoFont@font{SliTeX}{sc}{Sli}%
    }%
    \HOLOGO@discretionary
    \hologo{TeX}%
  }%
}
%    \end{macrocode}
%    \end{macro}
%    \begin{macro}{\HoLogoBkm@SliTeX@simple}
%    \begin{macrocode}
\def\HoLogoBkm@SliTeX@simple#1{SliTeX}
%    \end{macrocode}
%    \end{macro}
%    \begin{macro}{\HoLogoHtml@SliTeX@simple}
%    \begin{macrocode}
\let\HoLogoHtml@SliTeX@simple\HoLogo@SliTeX@simple
%    \end{macrocode}
%    \end{macro}
%
%    \begin{macro}{\HoLogo@SliTeX@narrow}
%    \begin{macrocode}
\def\HoLogo@SliTeX@narrow#1{%
  \HoLogoFont@font{SliTeX}{rm}{%
    \ltx@mbox{%
      S%
      \kern-.06em%
      \HoLogoFont@font{SliTeX}{sc}{%
        l%
        \kern-.035em%
        i%
      }%
    }%
    \HOLOGO@discretionary
    \kern-.06em%
    \hologo{TeX}%
  }%
}
%    \end{macrocode}
%    \end{macro}
%    \begin{macro}{\HoLogoBkm@SliTeX@narrow}
%    \begin{macrocode}
\def\HoLogoBkm@SliTeX@narrow#1{SliTeX}
%    \end{macrocode}
%    \end{macro}
%    \begin{macro}{\HoLogoHtml@SliTeX@narrow}
%    \begin{macrocode}
\def\HoLogoHtml@SliTeX@narrow#1{%
  \HoLogoCss@SliTeX@narrow
  \HOLOGO@Span{SliTeX-narrow}{%
    \HoLogoFont@font{SliTeX}{rm}{%
      S%
        \HOLOGO@Span{l}{l}%
        \HOLOGO@Span{i}{i}%
      \hologo{TeX}%
    }%
  }%
}
%    \end{macrocode}
%    \end{macro}
%    \begin{macro}{\HoLogoCss@SliTeX@narrow}
%    \begin{macrocode}
\def\HoLogoCss@SliTeX@narrow{%
  \Css{%
    span.HoLogo-SliTeX-narrow span.HoLogo-l{%
      margin-left:-.06em;%
      margin-right:-.035em;%
      font-variant:small-caps;%
    }%
  }%
  \Css{%
    span.HoLogo-SliTeX-narrow span.HoLogo-i{%
      margin-right:-.06em;%
      font-variant:small-caps;%
    }%
  }%
  \global\let\HoLogoCss@SliTeX@narrow\relax
}
%    \end{macrocode}
%    \end{macro}
%
% \paragraph{Macro set completion.}
%
%    \begin{macro}{\HoLogo@SLiTeX@simple}
%    \begin{macrocode}
\def\HoLogo@SLiTeX@simple{\HoLogo@SliTeX@simple}
%    \end{macrocode}
%    \end{macro}
%    \begin{macro}{\HoLogoBkm@SLiTeX@simple}
%    \begin{macrocode}
\def\HoLogoBkm@SLiTeX@simple{\HoLogoBkm@SliTeX@simple}
%    \end{macrocode}
%    \end{macro}
%    \begin{macro}{\HoLogoHtml@SLiTeX@simple}
%    \begin{macrocode}
\def\HoLogoHtml@SLiTeX@simple{\HoLogoHtml@SliTeX@simple}
%    \end{macrocode}
%    \end{macro}
%
%    \begin{macro}{\HoLogo@SLiTeX@narrow}
%    \begin{macrocode}
\def\HoLogo@SLiTeX@narrow{\HoLogo@SliTeX@narrow}
%    \end{macrocode}
%    \end{macro}
%    \begin{macro}{\HoLogoBkm@SLiTeX@narrow}
%    \begin{macrocode}
\def\HoLogoBkm@SLiTeX@narrow{\HoLogoBkm@SliTeX@narrow}
%    \end{macrocode}
%    \end{macro}
%    \begin{macro}{\HoLogoHtml@SLiTeX@narrow}
%    \begin{macrocode}
\def\HoLogoHtml@SLiTeX@narrow{\HoLogoHtml@SliTeX@narrow}
%    \end{macrocode}
%    \end{macro}
%
%    \begin{macro}{\HoLogo@SliTeX@lift}
%    \begin{macrocode}
\def\HoLogo@SliTeX@lift{\HoLogo@SLiTeX@lift}
%    \end{macrocode}
%    \end{macro}
%    \begin{macro}{\HoLogoBkm@SliTeX@lift}
%    \begin{macrocode}
\def\HoLogoBkm@SliTeX@lift{\HoLogoBkm@SLiTeX@lift}
%    \end{macrocode}
%    \end{macro}
%    \begin{macro}{\HoLogoHtml@SliTeX@lift}
%    \begin{macrocode}
\def\HoLogoHtml@SliTeX@lift{\HoLogoHtml@SLiTeX@lift}
%    \end{macrocode}
%    \end{macro}
%
% \paragraph{Defaults.}
%
%    \begin{macro}{\HoLogo@SLiTeX}
%    \begin{macrocode}
\def\HoLogo@SLiTeX{\HoLogo@SLiTeX@lift}
%    \end{macrocode}
%    \end{macro}
%    \begin{macro}{\HoLogoBkm@SLiTeX}
%    \begin{macrocode}
\def\HoLogoBkm@SLiTeX{\HoLogoBkm@SLiTeX@lift}
%    \end{macrocode}
%    \end{macro}
%    \begin{macro}{\HoLogoHtml@SLiTeX}
%    \begin{macrocode}
\def\HoLogoHtml@SLiTeX{\HoLogoHtml@SLiTeX@lift}
%    \end{macrocode}
%    \end{macro}
%
%    \begin{macro}{\HoLogo@SliTeX}
%    \begin{macrocode}
\def\HoLogo@SliTeX{\HoLogo@SliTeX@narrow}
%    \end{macrocode}
%    \end{macro}
%    \begin{macro}{\HoLogoBkm@SliTeX}
%    \begin{macrocode}
\def\HoLogoBkm@SliTeX{\HoLogoBkm@SliTeX@narrow}
%    \end{macrocode}
%    \end{macro}
%    \begin{macro}{\HoLogoHtml@SliTeX}
%    \begin{macrocode}
\def\HoLogoHtml@SliTeX{\HoLogoHtml@SliTeX@narrow}
%    \end{macrocode}
%    \end{macro}
%
% \subsubsection{\hologo{LuaTeX}}
%
%    \begin{macro}{\HoLogo@LuaTeX}
%    The kerning is an idea of Hans Hagen, see mailing list
%    `luatex at tug dot org' in March 2010.
%    \begin{macrocode}
\def\HoLogo@LuaTeX#1{%
  \HOLOGO@mbox{%
    Lua%
    \HOLOGO@NegativeKerning{aT,oT,To}%
    \hologo{TeX}%
  }%
}
%    \end{macrocode}
%    \end{macro}
%    \begin{macro}{\HoLogoHtml@LuaTeX}
%    \begin{macrocode}
\let\HoLogoHtml@LuaTeX\HoLogo@LuaTeX
%    \end{macrocode}
%    \end{macro}
%
% \subsubsection{\hologo{LuaLaTeX}}
%
%    \begin{macro}{\HoLogo@LuaLaTeX}
%    \begin{macrocode}
\def\HoLogo@LuaLaTeX#1{%
  \HOLOGO@mbox{%
    Lua%
    \hologo{LaTeX}%
  }%
}
%    \end{macrocode}
%    \end{macro}
%    \begin{macro}{\HoLogoHtml@LuaLaTeX}
%    \begin{macrocode}
\let\HoLogoHtml@LuaLaTeX\HoLogo@LuaLaTeX
%    \end{macrocode}
%    \end{macro}
%
% \subsubsection{\hologo{XeTeX}, \hologo{XeLaTeX}}
%
%    \begin{macro}{\HOLOGO@IfCharExists}
%    \begin{macrocode}
\ifluatex
  \ifnum\luatexversion<36 %
  \else
    \def\HOLOGO@IfCharExists#1{%
      \ifnum
        \directlua{%
           if luaotfload and luaotfload.aux then
             if luaotfload.aux.font_has_glyph(%
                    font.current(), \number#1) then % 	 
	       tex.print("1") % 	 
	     end % 	 
	   elseif font and font.fonts and font.current then %
            local f = font.fonts[font.current()]%
            if f.characters and f.characters[\number#1] then %
              tex.print("1")%
            end %
          end%
        }0=\ltx@zero
        \expandafter\ltx@secondoftwo
      \else
        \expandafter\ltx@firstoftwo
      \fi
    }%
  \fi
\fi
\ltx@IfUndefined{HOLOGO@IfCharExists}{%
  \def\HOLOGO@@IfCharExists#1{%
    \begingroup
      \tracinglostchars=\ltx@zero
      \setbox\ltx@zero=\hbox{%
        \kern7sp\char#1\relax
        \ifnum\lastkern>\ltx@zero
          \expandafter\aftergroup\csname iffalse\endcsname
        \else
          \expandafter\aftergroup\csname iftrue\endcsname
        \fi
      }%
      % \if{true|false} from \aftergroup
      \endgroup
      \expandafter\ltx@firstoftwo
    \else
      \endgroup
      \expandafter\ltx@secondoftwo
    \fi
  }%
  \ifxetex
    \ltx@IfUndefined{XeTeXfonttype}{}{%
      \ltx@IfUndefined{XeTeXcharglyph}{}{%
        \def\HOLOGO@IfCharExists#1{%
          \ifnum\XeTeXfonttype\font>\ltx@zero
            \expandafter\ltx@firstofthree
          \else
            \expandafter\ltx@gobble
          \fi
          {%
            \ifnum\XeTeXcharglyph#1>\ltx@zero
              \expandafter\ltx@firstoftwo
            \else
              \expandafter\ltx@secondoftwo
            \fi
          }%
          \HOLOGO@@IfCharExists{#1}%
        }%
      }%
    }%
  \fi
}{}
\ltx@ifundefined{HOLOGO@IfCharExists}{%
  \ifnum64=`\^^^^0040\relax % test for big chars of LuaTeX/XeTeX
    \let\HOLOGO@IfCharExists\HOLOGO@@IfCharExists
  \else
    \def\HOLOGO@IfCharExists#1{%
      \ifnum#1>255 %
        \expandafter\ltx@fourthoffour
      \fi
      \HOLOGO@@IfCharExists{#1}%
    }%
  \fi
}{}
%    \end{macrocode}
%    \end{macro}
%
%    \begin{macro}{\HoLogo@Xe}
%    Source: package \xpackage{dtklogos}
%    \begin{macrocode}
\def\HoLogo@Xe#1{%
  X%
  \kern-.1em\relax
  \HOLOGO@IfCharExists{"018E}{%
    \lower.5ex\hbox{\char"018E}%
  }{%
    \chardef\HOLOGO@choice=\ltx@zero
    \ifdim\fontdimen\ltx@one\font>0pt %
      \ltx@IfUndefined{rotatebox}{%
        \ltx@IfUndefined{pgftext}{%
          \ltx@IfUndefined{psscalebox}{%
            \ltx@IfUndefined{HOLOGO@ScaleBox@\hologoDriver}{%
            }{%
              \chardef\HOLOGO@choice=4 %
            }%
          }{%
            \chardef\HOLOGO@choice=3 %
          }%
        }{%
          \chardef\HOLOGO@choice=2 %
        }%
      }{%
        \chardef\HOLOGO@choice=1 %
      }%
      \ifcase\HOLOGO@choice
        \HOLOGO@WarningUnsupportedDriver{Xe}%
        e%
      \or % 1: \rotatebox
        \begingroup
          \setbox\ltx@zero\hbox{\rotatebox{180}{E}}%
          \ltx@LocDimenA=\dp\ltx@zero
          \advance\ltx@LocDimenA by -.5ex\relax
          \raise\ltx@LocDimenA\box\ltx@zero
        \endgroup
      \or % 2: \pgftext
        \lower.5ex\hbox{%
          \pgfpicture
            \pgftext[rotate=180]{E}%
          \endpgfpicture
        }%
      \or % 3: \psscalebox
        \begingroup
          \setbox\ltx@zero\hbox{\psscalebox{-1 -1}{E}}%
          \ltx@LocDimenA=\dp\ltx@zero
          \advance\ltx@LocDimenA by -.5ex\relax
          \raise\ltx@LocDimenA\box\ltx@zero
        \endgroup
      \or % 4: \HOLOGO@PointReflectBox
        \lower.5ex\hbox{\HOLOGO@PointReflectBox{E}}%
      \else
        \@PackageError{hologo}{Internal error (choice/it}\@ehc
      \fi
    \else
      \ltx@IfUndefined{reflectbox}{%
        \ltx@IfUndefined{pgftext}{%
          \ltx@IfUndefined{psscalebox}{%
            \ltx@IfUndefined{HOLOGO@ScaleBox@\hologoDriver}{%
            }{%
              \chardef\HOLOGO@choice=4 %
            }%
          }{%
            \chardef\HOLOGO@choice=3 %
          }%
        }{%
          \chardef\HOLOGO@choice=2 %
        }%
      }{%
        \chardef\HOLOGO@choice=1 %
      }%
      \ifcase\HOLOGO@choice
        \HOLOGO@WarningUnsupportedDriver{Xe}%
        e%
      \or % 1: reflectbox
        \lower.5ex\hbox{%
          \reflectbox{E}%
        }%
      \or % 2: \pgftext
        \lower.5ex\hbox{%
          \pgfpicture
            \pgftransformxscale{-1}%
            \pgftext{E}%
          \endpgfpicture
        }%
      \or % 3: \psscalebox
        \lower.5ex\hbox{%
          \psscalebox{-1 1}{E}%
        }%
      \or % 4: \HOLOGO@Reflectbox
        \lower.5ex\hbox{%
          \HOLOGO@ReflectBox{E}%
        }%
      \else
        \@PackageError{hologo}{Internal error (choice/up)}\@ehc
      \fi
    \fi
  }%
}
%    \end{macrocode}
%    \end{macro}
%    \begin{macro}{\HoLogoHtml@Xe}
%    \begin{macrocode}
\def\HoLogoHtml@Xe#1{%
  \HoLogoCss@Xe
  \HOLOGO@Span{Xe}{%
    X%
    \HOLOGO@Span{e}{%
      \HCode{&\ltx@hashchar x018e;}%
    }%
  }%
}
%    \end{macrocode}
%    \end{macro}
%    \begin{macro}{\HoLogoCss@Xe}
%    \begin{macrocode}
\def\HoLogoCss@Xe{%
  \Css{%
    span.HoLogo-Xe span.HoLogo-e{%
      position:relative;%
      top:.5ex;%
      left-margin:-.1em;%
    }%
  }%
  \global\let\HoLogoCss@Xe\relax
}
%    \end{macrocode}
%    \end{macro}
%
%    \begin{macro}{\HoLogo@XeTeX}
%    \begin{macrocode}
\def\HoLogo@XeTeX#1{%
  \hologo{Xe}%
  \kern-.15em\relax
  \hologo{TeX}%
}
%    \end{macrocode}
%    \end{macro}
%
%    \begin{macro}{\HoLogoHtml@XeTeX}
%    \begin{macrocode}
\def\HoLogoHtml@XeTeX#1{%
  \HoLogoCss@XeTeX
  \HOLOGO@Span{XeTeX}{%
    \hologo{Xe}%
    \hologo{TeX}%
  }%
}
%    \end{macrocode}
%    \end{macro}
%    \begin{macro}{\HoLogoCss@XeTeX}
%    \begin{macrocode}
\def\HoLogoCss@XeTeX{%
  \Css{%
    span.HoLogo-XeTeX span.HoLogo-TeX{%
      margin-left:-.15em;%
    }%
  }%
  \global\let\HoLogoCss@XeTeX\relax
}
%    \end{macrocode}
%    \end{macro}
%
%    \begin{macro}{\HoLogo@XeLaTeX}
%    \begin{macrocode}
\def\HoLogo@XeLaTeX#1{%
  \hologo{Xe}%
  \kern-.13em%
  \hologo{LaTeX}%
}
%    \end{macrocode}
%    \end{macro}
%    \begin{macro}{\HoLogoHtml@XeLaTeX}
%    \begin{macrocode}
\def\HoLogoHtml@XeLaTeX#1{%
  \HoLogoCss@XeLaTeX
  \HOLOGO@Span{XeLaTeX}{%
    \hologo{Xe}%
    \hologo{LaTeX}%
  }%
}
%    \end{macrocode}
%    \end{macro}
%    \begin{macro}{\HoLogoCss@XeLaTeX}
%    \begin{macrocode}
\def\HoLogoCss@XeLaTeX{%
  \Css{%
    span.HoLogo-XeLaTeX span.HoLogo-Xe{%
      margin-right:-.13em;%
    }%
  }%
  \global\let\HoLogoCss@XeLaTeX\relax
}
%    \end{macrocode}
%    \end{macro}
%
% \subsubsection{\hologo{pdfTeX}, \hologo{pdfLaTeX}}
%
%    \begin{macro}{\HoLogo@pdfTeX}
%    \begin{macrocode}
\def\HoLogo@pdfTeX#1{%
  \HOLOGO@mbox{%
    #1{p}{P}df\hologo{TeX}%
  }%
}
%    \end{macrocode}
%    \end{macro}
%    \begin{macro}{\HoLogoCs@pdfTeX}
%    \begin{macrocode}
\def\HoLogoCs@pdfTeX#1{#1{p}{P}dfTeX}
%    \end{macrocode}
%    \end{macro}
%    \begin{macro}{\HoLogoBkm@pdfTeX}
%    \begin{macrocode}
\def\HoLogoBkm@pdfTeX#1{%
  #1{p}{P}df\hologo{TeX}%
}
%    \end{macrocode}
%    \end{macro}
%    \begin{macro}{\HoLogoHtml@pdfTeX}
%    \begin{macrocode}
\let\HoLogoHtml@pdfTeX\HoLogo@pdfTeX
%    \end{macrocode}
%    \end{macro}
%
%    \begin{macro}{\HoLogo@pdfLaTeX}
%    \begin{macrocode}
\def\HoLogo@pdfLaTeX#1{%
  \HOLOGO@mbox{%
    #1{p}{P}df\hologo{LaTeX}%
  }%
}
%    \end{macrocode}
%    \end{macro}
%    \begin{macro}{\HoLogoCs@pdfLaTeX}
%    \begin{macrocode}
\def\HoLogoCs@pdfLaTeX#1{#1{p}{P}dfLaTeX}
%    \end{macrocode}
%    \end{macro}
%    \begin{macro}{\HoLogoBkm@pdfLaTeX}
%    \begin{macrocode}
\def\HoLogoBkm@pdfLaTeX#1{%
  #1{p}{P}df\hologo{LaTeX}%
}
%    \end{macrocode}
%    \end{macro}
%    \begin{macro}{\HoLogoHtml@pdfLaTeX}
%    \begin{macrocode}
\let\HoLogoHtml@pdfLaTeX\HoLogo@pdfLaTeX
%    \end{macrocode}
%    \end{macro}
%
% \subsubsection{\hologo{VTeX}}
%
%    \begin{macro}{\HoLogo@VTeX}
%    \begin{macrocode}
\def\HoLogo@VTeX#1{%
  \HOLOGO@mbox{%
    V\hologo{TeX}%
  }%
}
%    \end{macrocode}
%    \end{macro}
%    \begin{macro}{\HoLogoHtml@VTeX}
%    \begin{macrocode}
\let\HoLogoHtml@VTeX\HoLogo@VTeX
%    \end{macrocode}
%    \end{macro}
%
% \subsubsection{\hologo{AmS}, \dots}
%
%    Source: class \xclass{amsdtx}
%
%    \begin{macro}{\HoLogo@AmS}
%    \begin{macrocode}
\def\HoLogo@AmS#1{%
  \HoLogoFont@font{AmS}{sy}{%
    A%
    \kern-.1667em%
    \lower.5ex\hbox{M}%
    \kern-.125em%
    S%
  }%
}
%    \end{macrocode}
%    \end{macro}
%    \begin{macro}{\HoLogoBkm@AmS}
%    \begin{macrocode}
\def\HoLogoBkm@AmS#1{AmS}
%    \end{macrocode}
%    \end{macro}
%    \begin{macro}{\HoLogoHtml@AmS}
%    \begin{macrocode}
\def\HoLogoHtml@AmS#1{%
  \HoLogoCss@AmS
%  \HoLogoFont@font{AmS}{sy}{%
    \HOLOGO@Span{AmS}{%
      A%
      \HOLOGO@Span{M}{M}%
      S%
    }%
%   }%
}
%    \end{macrocode}
%    \end{macro}
%    \begin{macro}{\HoLogoCss@AmS}
%    \begin{macrocode}
\def\HoLogoCss@AmS{%
  \Css{%
    span.HoLogo-AmS span.HoLogo-M{%
      position:relative;%
      top:.5ex;%
      margin-left:-.1667em;%
      margin-right:-.125em;%
      text-decoration:none;%
    }%
  }%
  \global\let\HoLogoCss@AmS\relax
}
%    \end{macrocode}
%    \end{macro}
%
%    \begin{macro}{\HoLogo@AmSTeX}
%    \begin{macrocode}
\def\HoLogo@AmSTeX#1{%
  \hologo{AmS}%
  \HOLOGO@hyphen
  \hologo{TeX}%
}
%    \end{macrocode}
%    \end{macro}
%    \begin{macro}{\HoLogoBkm@AmSTeX}
%    \begin{macrocode}
\def\HoLogoBkm@AmSTeX#1{AmS-TeX}%
%    \end{macrocode}
%    \end{macro}
%    \begin{macro}{\HoLogoHtml@AmSTeX}
%    \begin{macrocode}
\let\HoLogoHtml@AmSTeX\HoLogo@AmSTeX
%    \end{macrocode}
%    \end{macro}
%
%    \begin{macro}{\HoLogo@AmSLaTeX}
%    \begin{macrocode}
\def\HoLogo@AmSLaTeX#1{%
  \hologo{AmS}%
  \HOLOGO@hyphen
  \hologo{LaTeX}%
}
%    \end{macrocode}
%    \end{macro}
%    \begin{macro}{\HoLogoBkm@AmSLaTeX}
%    \begin{macrocode}
\def\HoLogoBkm@AmSLaTeX#1{AmS-LaTeX}%
%    \end{macrocode}
%    \end{macro}
%    \begin{macro}{\HoLogoHtml@AmSLaTeX}
%    \begin{macrocode}
\let\HoLogoHtml@AmSLaTeX\HoLogo@AmSLaTeX
%    \end{macrocode}
%    \end{macro}
%
% \subsubsection{\hologo{BibTeX}}
%
%    \begin{macro}{\HoLogo@BibTeX@sc}
%    A definition of \hologo{BibTeX} is provided in
%    the documentation source for the manual of \hologo{BibTeX}
%    \cite{btxdoc}.
%\begin{quote}
%\begin{verbatim}
%\def\BibTeX{%
%  {%
%    \rm
%    B%
%    \kern-.05em%
%    {%
%      \sc
%      i%
%      \kern-.025em %
%      b%
%    }%
%    \kern-.08em
%    T%
%    \kern-.1667em%
%    \lower.7ex\hbox{E}%
%    \kern-.125em%
%    X%
%  }%
%}
%\end{verbatim}
%\end{quote}
%    \begin{macrocode}
\def\HoLogo@BibTeX@sc#1{%
  B%
  \kern-.05em%
  \HoLogoFont@font{BibTeX}{sc}{%
    i%
    \kern-.025em%
    b%
  }%
  \HOLOGO@discretionary
  \kern-.08em%
  \hologo{TeX}%
}
%    \end{macrocode}
%    \end{macro}
%    \begin{macro}{\HoLogoHtml@BibTeX@sc}
%    \begin{macrocode}
\def\HoLogoHtml@BibTeX@sc#1{%
  \HoLogoCss@BibTeX@sc
  \HOLOGO@Span{BibTeX-sc}{%
    B%
    \HOLOGO@Span{i}{i}%
    \HOLOGO@Span{b}{b}%
    \hologo{TeX}%
  }%
}
%    \end{macrocode}
%    \end{macro}
%    \begin{macro}{\HoLogoCss@BibTeX@sc}
%    \begin{macrocode}
\def\HoLogoCss@BibTeX@sc{%
  \Css{%
    span.HoLogo-BibTeX-sc span.HoLogo-i{%
      margin-left:-.05em;%
      margin-right:-.025em;%
      font-variant:small-caps;%
    }%
  }%
  \Css{%
    span.HoLogo-BibTeX-sc span.HoLogo-b{%
      margin-right:-.08em;%
      font-variant:small-caps;%
    }%
  }%
  \global\let\HoLogoCss@BibTeX@sc\relax
}
%    \end{macrocode}
%    \end{macro}
%
%    \begin{macro}{\HoLogo@BibTeX@sf}
%    Variant \xoption{sf} avoids trouble with unavailable
%    small caps fonts (e.g., bold versions of Computer Modern or
%    Latin Modern). The definition is taken from
%    package \xpackage{dtklogos} \cite{dtklogos}.
%\begin{quote}
%\begin{verbatim}
%\DeclareRobustCommand{\BibTeX}{%
%  B%
%  \kern-.05em%
%  \hbox{%
%    $\m@th$% %% force math size calculations
%    \csname S@\f@size\endcsname
%    \fontsize\sf@size\z@
%    \math@fontsfalse
%    \selectfont
%    I%
%    \kern-.025em%
%    B
%  }%
%  \kern-.08em%
%  \-%
%  \TeX
%}
%\end{verbatim}
%\end{quote}
%    \begin{macrocode}
\def\HoLogo@BibTeX@sf#1{%
  B%
  \kern-.05em%
  \HoLogoFont@font{BibTeX}{bibsf}{%
    I%
    \kern-.025em%
    B%
  }%
  \HOLOGO@discretionary
  \kern-.08em%
  \hologo{TeX}%
}
%    \end{macrocode}
%    \end{macro}
%    \begin{macro}{\HoLogoHtml@BibTeX@sf}
%    \begin{macrocode}
\def\HoLogoHtml@BibTeX@sf#1{%
  \HoLogoCss@BibTeX@sf
  \HOLOGO@Span{BibTeX-sf}{%
    B%
    \HoLogoFont@font{BibTeX}{bibsf}{%
      \HOLOGO@Span{i}{I}%
      B%
    }%
    \hologo{TeX}%
  }%
}
%    \end{macrocode}
%    \end{macro}
%    \begin{macro}{\HoLogoCss@BibTeX@sf}
%    \begin{macrocode}
\def\HoLogoCss@BibTeX@sf{%
  \Css{%
    span.HoLogo-BibTeX-sf span.HoLogo-i{%
      margin-left:-.05em;%
      margin-right:-.025em;%
    }%
  }%
  \Css{%
    span.HoLogo-BibTeX-sf span.HoLogo-TeX{%
      margin-left:-.08em;%
    }%
  }%
  \global\let\HoLogoCss@BibTeX@sf\relax
}
%    \end{macrocode}
%    \end{macro}
%
%    \begin{macro}{\HoLogo@BibTeX}
%    \begin{macrocode}
\def\HoLogo@BibTeX{\HoLogo@BibTeX@sf}
%    \end{macrocode}
%    \end{macro}
%    \begin{macro}{\HoLogoHtml@BibTeX}
%    \begin{macrocode}
\def\HoLogoHtml@BibTeX{\HoLogoHtml@BibTeX@sf}
%    \end{macrocode}
%    \end{macro}
%
% \subsubsection{\hologo{BibTeX8}}
%
%    \begin{macro}{\HoLogo@BibTeX8}
%    \begin{macrocode}
\expandafter\def\csname HoLogo@BibTeX8\endcsname#1{%
  \hologo{BibTeX}%
  8%
}
%    \end{macrocode}
%    \end{macro}
%
%    \begin{macro}{\HoLogoBkm@BibTeX8}
%    \begin{macrocode}
\expandafter\def\csname HoLogoBkm@BibTeX8\endcsname#1{%
  \hologo{BibTeX}%
  8%
}
%    \end{macrocode}
%    \end{macro}
%    \begin{macro}{\HoLogoHtml@BibTeX8}
%    \begin{macrocode}
\expandafter
\let\csname HoLogoHtml@BibTeX8\expandafter\endcsname
\csname HoLogo@BibTeX8\endcsname
%    \end{macrocode}
%    \end{macro}
%
% \subsubsection{\hologo{ConTeXt}}
%
%    \begin{macro}{\HoLogo@ConTeXt@simple}
%    \begin{macrocode}
\def\HoLogo@ConTeXt@simple#1{%
  \HOLOGO@mbox{Con}%
  \HOLOGO@discretionary
  \HOLOGO@mbox{\hologo{TeX}t}%
}
%    \end{macrocode}
%    \end{macro}
%    \begin{macro}{\HoLogoHtml@ConTeXt@simple}
%    \begin{macrocode}
\let\HoLogoHtml@ConTeXt@simple\HoLogo@ConTeXt@simple
%    \end{macrocode}
%    \end{macro}
%
%    \begin{macro}{\HoLogo@ConTeXt@narrow}
%    This definition of logo \hologo{ConTeXt} with variant \xoption{narrow}
%    comes from TUGboat's class \xclass{ltugboat} (version 2010/11/15 v2.8).
%    \begin{macrocode}
\def\HoLogo@ConTeXt@narrow#1{%
  \HOLOGO@mbox{C\kern-.0333emon}%
  \HOLOGO@discretionary
  \kern-.0667em%
  \HOLOGO@mbox{\hologo{TeX}\kern-.0333emt}%
}
%    \end{macrocode}
%    \end{macro}
%    \begin{macro}{\HoLogoHtml@ConTeXt@narrow}
%    \begin{macrocode}
\def\HoLogoHtml@ConTeXt@narrow#1{%
  \HoLogoCss@ConTeXt@narrow
  \HOLOGO@Span{ConTeXt-narrow}{%
    \HOLOGO@Span{C}{C}%
    on%
    \hologo{TeX}%
    t%
  }%
}
%    \end{macrocode}
%    \end{macro}
%    \begin{macro}{\HoLogoCss@ConTeXt@narrow}
%    \begin{macrocode}
\def\HoLogoCss@ConTeXt@narrow{%
  \Css{%
    span.HoLogo-ConTeXt-narrow span.HoLogo-C{%
      margin-left:-.0333em;%
    }%
  }%
  \Css{%
    span.HoLogo-ConTeXt-narrow span.HoLogo-TeX{%
      margin-left:-.0667em;%
      margin-right:-.0333em;%
    }%
  }%
  \global\let\HoLogoCss@ConTeXt@narrow\relax
}
%    \end{macrocode}
%    \end{macro}
%
%    \begin{macro}{\HoLogo@ConTeXt}
%    \begin{macrocode}
\def\HoLogo@ConTeXt{\HoLogo@ConTeXt@narrow}
%    \end{macrocode}
%    \end{macro}
%    \begin{macro}{\HoLogoHtml@ConTeXt}
%    \begin{macrocode}
\def\HoLogoHtml@ConTeXt{\HoLogoHtml@ConTeXt@narrow}
%    \end{macrocode}
%    \end{macro}
%
% \subsubsection{\hologo{emTeX}}
%
%    \begin{macro}{\HoLogo@emTeX}
%    \begin{macrocode}
\def\HoLogo@emTeX#1{%
  \HOLOGO@mbox{#1{e}{E}m}%
  \HOLOGO@discretionary
  \hologo{TeX}%
}
%    \end{macrocode}
%    \end{macro}
%    \begin{macro}{\HoLogoCs@emTeX}
%    \begin{macrocode}
\def\HoLogoCs@emTeX#1{#1{e}{E}mTeX}%
%    \end{macrocode}
%    \end{macro}
%    \begin{macro}{\HoLogoBkm@emTeX}
%    \begin{macrocode}
\def\HoLogoBkm@emTeX#1{%
  #1{e}{E}m\hologo{TeX}%
}
%    \end{macrocode}
%    \end{macro}
%    \begin{macro}{\HoLogoHtml@emTeX}
%    \begin{macrocode}
\let\HoLogoHtml@emTeX\HoLogo@emTeX
%    \end{macrocode}
%    \end{macro}
%
% \subsubsection{\hologo{ExTeX}}
%
%    \begin{macro}{\HoLogo@ExTeX}
%    The definition is taken from the FAQ of the
%    project \hologo{ExTeX}
%    \cite{ExTeX-FAQ}.
%\begin{quote}
%\begin{verbatim}
%\def\ExTeX{%
%  \textrm{% Logo always with serifs
%    \ensuremath{%
%      \textstyle
%      \varepsilon_{%
%        \kern-0.15em%
%        \mathcal{X}%
%      }%
%    }%
%    \kern-.15em%
%    \TeX
%  }%
%}
%\end{verbatim}
%\end{quote}
%    \begin{macrocode}
\def\HoLogo@ExTeX#1{%
  \HoLogoFont@font{ExTeX}{rm}{%
    \ltx@mbox{%
      \HOLOGO@MathSetup
      $%
        \textstyle
        \varepsilon_{%
          \kern-0.15em%
          \HoLogoFont@font{ExTeX}{sy}{X}%
        }%
      $%
    }%
    \HOLOGO@discretionary
    \kern-.15em%
    \hologo{TeX}%
  }%
}
%    \end{macrocode}
%    \end{macro}
%    \begin{macro}{\HoLogoHtml@ExTeX}
%    \begin{macrocode}
\def\HoLogoHtml@ExTeX#1{%
  \HoLogoCss@ExTeX
  \HoLogoFont@font{ExTeX}{rm}{%
    \HOLOGO@Span{ExTeX}{%
      \ltx@mbox{%
        \HOLOGO@MathSetup
        $\textstyle\varepsilon$%
        \HOLOGO@Span{X}{$\textstyle\chi$}%
        \hologo{TeX}%
      }%
    }%
  }%
}
%    \end{macrocode}
%    \end{macro}
%    \begin{macro}{\HoLogoBkm@ExTeX}
%    \begin{macrocode}
\def\HoLogoBkm@ExTeX#1{%
  \HOLOGO@PdfdocUnicode{#1{e}{E}x}{\textepsilon\textchi}%
  \hologo{TeX}%
}
%    \end{macrocode}
%    \end{macro}
%    \begin{macro}{\HoLogoCss@ExTeX}
%    \begin{macrocode}
\def\HoLogoCss@ExTeX{%
  \Css{%
    span.HoLogo-ExTeX{%
      font-family:serif;%
    }%
  }%
  \Css{%
    span.HoLogo-ExTeX span.HoLogo-TeX{%
      margin-left:-.15em;%
    }%
  }%
  \global\let\HoLogoCss@ExTeX\relax
}
%    \end{macrocode}
%    \end{macro}
%
% \subsubsection{\hologo{MiKTeX}}
%
%    \begin{macro}{\HoLogo@MiKTeX}
%    \begin{macrocode}
\def\HoLogo@MiKTeX#1{%
  \HOLOGO@mbox{MiK}%
  \HOLOGO@discretionary
  \hologo{TeX}%
}
%    \end{macrocode}
%    \end{macro}
%    \begin{macro}{\HoLogoHtml@MiKTeX}
%    \begin{macrocode}
\let\HoLogoHtml@MiKTeX\HoLogo@MiKTeX
%    \end{macrocode}
%    \end{macro}
%
% \subsubsection{\hologo{OzTeX} and friends}
%
%    Source: \hologo{OzTeX} FAQ \cite{OzTeX}:
%    \begin{quote}
%      |\def\OzTeX{O\kern-.03em z\kern-.15em\TeX}|\\
%      (There is no kerning in OzMF, OzMP and OzTtH.)
%    \end{quote}
%
%    \begin{macro}{\HoLogo@OzTeX}
%    \begin{macrocode}
\def\HoLogo@OzTeX#1{%
  O%
  \kern-.03em %
  z%
  \kern-.15em %
  \hologo{TeX}%
}
%    \end{macrocode}
%    \end{macro}
%    \begin{macro}{\HoLogoHtml@OzTeX}
%    \begin{macrocode}
\def\HoLogoHtml@OzTeX#1{%
  \HoLogoCss@OzTeX
  \HOLOGO@Span{OzTeX}{%
    O%
    \HOLOGO@Span{z}{z}%
    \hologo{TeX}%
  }%
}
%    \end{macrocode}
%    \end{macro}
%    \begin{macro}{\HoLogoCss@OzTeX}
%    \begin{macrocode}
\def\HoLogoCss@OzTeX{%
  \Css{%
    span.HoLogo-OzTeX span.HoLogo-z{%
      margin-left:-.03em;%
      margin-right:-.15em;%
    }%
  }%
  \global\let\HoLogoCss@OzTeX\relax
}
%    \end{macrocode}
%    \end{macro}
%
%    \begin{macro}{\HoLogo@OzMF}
%    \begin{macrocode}
\def\HoLogo@OzMF#1{%
  \HOLOGO@mbox{OzMF}%
}
%    \end{macrocode}
%    \end{macro}
%    \begin{macro}{\HoLogo@OzMP}
%    \begin{macrocode}
\def\HoLogo@OzMP#1{%
  \HOLOGO@mbox{OzMP}%
}
%    \end{macrocode}
%    \end{macro}
%    \begin{macro}{\HoLogo@OzTtH}
%    \begin{macrocode}
\def\HoLogo@OzTtH#1{%
  \HOLOGO@mbox{OzTtH}%
}
%    \end{macrocode}
%    \end{macro}
%
% \subsubsection{\hologo{PCTeX}}
%
%    \begin{macro}{\HoLogo@PCTeX}
%    \begin{macrocode}
\def\HoLogo@PCTeX#1{%
  \HOLOGO@mbox{PC}%
  \hologo{TeX}%
}
%    \end{macrocode}
%    \end{macro}
%    \begin{macro}{\HoLogoHtml@PCTeX}
%    \begin{macrocode}
\let\HoLogoHtml@PCTeX\HoLogo@PCTeX
%    \end{macrocode}
%    \end{macro}
%
% \subsubsection{\hologo{PiCTeX}}
%
%    The original definitions from \xfile{pictex.tex} \cite{PiCTeX}:
%\begin{quote}
%\begin{verbatim}
%\def\PiC{%
%  P%
%  \kern-.12em%
%  \lower.5ex\hbox{I}%
%  \kern-.075em%
%  C%
%}
%\def\PiCTeX{%
%  \PiC
%  \kern-.11em%
%  \TeX
%}
%\end{verbatim}
%\end{quote}
%
%    \begin{macro}{\HoLogo@PiC}
%    \begin{macrocode}
\def\HoLogo@PiC#1{%
  P%
  \kern-.12em%
  \lower.5ex\hbox{I}%
  \kern-.075em%
  C%
  \HOLOGO@SpaceFactor
}
%    \end{macrocode}
%    \end{macro}
%    \begin{macro}{\HoLogoHtml@PiC}
%    \begin{macrocode}
\def\HoLogoHtml@PiC#1{%
  \HoLogoCss@PiC
  \HOLOGO@Span{PiC}{%
    P%
    \HOLOGO@Span{i}{I}%
    C%
  }%
}
%    \end{macrocode}
%    \end{macro}
%    \begin{macro}{\HoLogoCss@PiC}
%    \begin{macrocode}
\def\HoLogoCss@PiC{%
  \Css{%
    span.HoLogo-PiC span.HoLogo-i{%
      position:relative;%
      top:.5ex;%
      margin-left:-.12em;%
      margin-right:-.075em;%
      text-decoration:none;%
    }%
  }%
  \global\let\HoLogoCss@PiC\relax
}
%    \end{macrocode}
%    \end{macro}
%
%    \begin{macro}{\HoLogo@PiCTeX}
%    \begin{macrocode}
\def\HoLogo@PiCTeX#1{%
  \hologo{PiC}%
  \HOLOGO@discretionary
  \kern-.11em%
  \hologo{TeX}%
}
%    \end{macrocode}
%    \end{macro}
%    \begin{macro}{\HoLogoHtml@PiCTeX}
%    \begin{macrocode}
\def\HoLogoHtml@PiCTeX#1{%
  \HoLogoCss@PiCTeX
  \HOLOGO@Span{PiCTeX}{%
    \hologo{PiC}%
    \hologo{TeX}%
  }%
}
%    \end{macrocode}
%    \end{macro}
%    \begin{macro}{\HoLogoCss@PiCTeX}
%    \begin{macrocode}
\def\HoLogoCss@PiCTeX{%
  \Css{%
    span.HoLogo-PiCTeX span.HoLogo-PiC{%
      margin-right:-.11em;%
    }%
  }%
  \global\let\HoLogoCss@PiCTeX\relax
}
%    \end{macrocode}
%    \end{macro}
%
% \subsubsection{\hologo{teTeX}}
%
%    \begin{macro}{\HoLogo@teTeX}
%    \begin{macrocode}
\def\HoLogo@teTeX#1{%
  \HOLOGO@mbox{#1{t}{T}e}%
  \HOLOGO@discretionary
  \hologo{TeX}%
}
%    \end{macrocode}
%    \end{macro}
%    \begin{macro}{\HoLogoCs@teTeX}
%    \begin{macrocode}
\def\HoLogoCs@teTeX#1{#1{t}{T}dfTeX}
%    \end{macrocode}
%    \end{macro}
%    \begin{macro}{\HoLogoBkm@teTeX}
%    \begin{macrocode}
\def\HoLogoBkm@teTeX#1{%
  #1{t}{T}e\hologo{TeX}%
}
%    \end{macrocode}
%    \end{macro}
%    \begin{macro}{\HoLogoHtml@teTeX}
%    \begin{macrocode}
\let\HoLogoHtml@teTeX\HoLogo@teTeX
%    \end{macrocode}
%    \end{macro}
%
% \subsubsection{\hologo{TeX4ht}}
%
%    \begin{macro}{\HoLogo@TeX4ht}
%    \begin{macrocode}
\expandafter\def\csname HoLogo@TeX4ht\endcsname#1{%
  \HOLOGO@mbox{\hologo{TeX}4ht}%
}
%    \end{macrocode}
%    \end{macro}
%    \begin{macro}{\HoLogoHtml@TeX4ht}
%    \begin{macrocode}
\expandafter
\let\csname HoLogoHtml@TeX4ht\expandafter\endcsname
\csname HoLogo@TeX4ht\endcsname
%    \end{macrocode}
%    \end{macro}
%
%
% \subsubsection{\hologo{SageTeX}}
%
%    \begin{macro}{\HoLogo@SageTeX}
%    \begin{macrocode}
\def\HoLogo@SageTeX#1{%
  \HOLOGO@mbox{Sage}%
  \HOLOGO@discretionary
  \HOLOGO@NegativeKerning{eT,oT,To}%
  \hologo{TeX}%
}
%    \end{macrocode}
%    \end{macro}
%    \begin{macro}{\HoLogoHtml@SageTeX}
%    \begin{macrocode}
\let\HoLogoHtml@SageTeX\HoLogo@SageTeX
%    \end{macrocode}
%    \end{macro}
%
% \subsection{\hologo{METAFONT} and friends}
%
%    \begin{macro}{\HoLogo@METAFONT}
%    \begin{macrocode}
\def\HoLogo@METAFONT#1{%
  \HoLogoFont@font{METAFONT}{logo}{%
    \HOLOGO@mbox{META}%
    \HOLOGO@discretionary
    \HOLOGO@mbox{FONT}%
  }%
}
%    \end{macrocode}
%    \end{macro}
%
%    \begin{macro}{\HoLogo@METAPOST}
%    \begin{macrocode}
\def\HoLogo@METAPOST#1{%
  \HoLogoFont@font{METAPOST}{logo}{%
    \HOLOGO@mbox{META}%
    \HOLOGO@discretionary
    \HOLOGO@mbox{POST}%
  }%
}
%    \end{macrocode}
%    \end{macro}
%
%    \begin{macro}{\HoLogo@MetaFun}
%    \begin{macrocode}
\def\HoLogo@MetaFun#1{%
  \HOLOGO@mbox{Meta}%
  \HOLOGO@discretionary
  \HOLOGO@mbox{Fun}%
}
%    \end{macrocode}
%    \end{macro}
%
%    \begin{macro}{\HoLogo@MetaPost}
%    \begin{macrocode}
\def\HoLogo@MetaPost#1{%
  \HOLOGO@mbox{Meta}%
  \HOLOGO@discretionary
  \HOLOGO@mbox{Post}%
}
%    \end{macrocode}
%    \end{macro}
%
% \subsection{Others}
%
% \subsubsection{\hologo{biber}}
%
%    \begin{macro}{\HoLogo@biber}
%    \begin{macrocode}
\def\HoLogo@biber#1{%
  \HOLOGO@mbox{#1{b}{B}i}%
  \HOLOGO@discretionary
  \HOLOGO@mbox{ber}%
}
%    \end{macrocode}
%    \end{macro}
%    \begin{macro}{\HoLogoCs@biber}
%    \begin{macrocode}
\def\HoLogoCs@biber#1{#1{b}{B}iber}
%    \end{macrocode}
%    \end{macro}
%    \begin{macro}{\HoLogoBkm@biber}
%    \begin{macrocode}
\def\HoLogoBkm@biber#1{%
  #1{b}{B}iber%
}
%    \end{macrocode}
%    \end{macro}
%    \begin{macro}{\HoLogoHtml@biber}
%    \begin{macrocode}
\let\HoLogoHtml@biber\HoLogo@biber
%    \end{macrocode}
%    \end{macro}
%
% \subsubsection{\hologo{KOMAScript}}
%
%    \begin{macro}{\HoLogo@KOMAScript}
%    The definition for \hologo{KOMAScript} is taken
%    from \hologo{KOMAScript} (\xfile{scrlogo.dtx}, reformatted) \cite{scrlogo}:
%\begin{quote}
%\begin{verbatim}
%\@ifundefined{KOMAScript}{%
%  \DeclareRobustCommand{\KOMAScript}{%
%    \textsf{%
%      K\kern.05em O\kern.05emM\kern.05em A%
%      \kern.1em-\kern.1em %
%      Script%
%    }%
%  }%
%}{}
%\end{verbatim}
%\end{quote}
%    \begin{macrocode}
\def\HoLogo@KOMAScript#1{%
  \HoLogoFont@font{KOMAScript}{sf}{%
    \HOLOGO@mbox{%
      K\kern.05em%
      O\kern.05em%
      M\kern.05em%
      A%
    }%
    \kern.1em%
    \HOLOGO@hyphen
    \kern.1em%
    \HOLOGO@mbox{Script}%
  }%
}
%    \end{macrocode}
%    \end{macro}
%    \begin{macro}{\HoLogoBkm@KOMAScript}
%    \begin{macrocode}
\def\HoLogoBkm@KOMAScript#1{%
  KOMA-Script%
}
%    \end{macrocode}
%    \end{macro}
%    \begin{macro}{\HoLogoHtml@KOMAScript}
%    \begin{macrocode}
\def\HoLogoHtml@KOMAScript#1{%
  \HoLogoCss@KOMAScript
  \HoLogoFont@font{KOMAScript}{sf}{%
    \HOLOGO@Span{KOMAScript}{%
      K%
      \HOLOGO@Span{O}{O}%
      M%
      \HOLOGO@Span{A}{A}%
      \HOLOGO@Span{hyphen}{-}%
      Script%
    }%
  }%
}
%    \end{macrocode}
%    \end{macro}
%    \begin{macro}{\HoLogoCss@KOMAScript}
%    \begin{macrocode}
\def\HoLogoCss@KOMAScript{%
  \Css{%
    span.HoLogo-KOMAScript{%
      font-family:sans-serif;%
    }%
  }%
  \Css{%
    span.HoLogo-KOMAScript span.HoLogo-O{%
      padding-left:.05em;%
      padding-right:.05em;%
    }%
  }%
  \Css{%
    span.HoLogo-KOMAScript span.HoLogo-A{%
      padding-left:.05em;%
    }%
  }%
  \Css{%
    span.HoLogo-KOMAScript span.HoLogo-hyphen{%
      padding-left:.1em;%
      padding-right:.1em;%
    }%
  }%
  \global\let\HoLogoCss@KOMAScript\relax
}
%    \end{macrocode}
%    \end{macro}
%
% \subsubsection{\hologo{LyX}}
%
%    \begin{macro}{\HoLogo@LyX}
%    The definition is taken from the documentation source files
%    of \hologo{LyX}, \xfile{Intro.lyx} \cite{LyX}:
%\begin{quote}
%\begin{verbatim}
%\def\LyX{%
%  \texorpdfstring{%
%    L\kern-.1667em\lower.25em\hbox{Y}\kern-.125emX\@%
%  }{%
%    LyX%
%  }%
%}
%\end{verbatim}
%\end{quote}
%    \begin{macrocode}
\def\HoLogo@LyX#1{%
  L%
  \kern-.1667em%
  \lower.25em\hbox{Y}%
  \kern-.125em%
  X%
  \HOLOGO@SpaceFactor
}
%    \end{macrocode}
%    \end{macro}
%    \begin{macro}{\HoLogoHtml@LyX}
%    \begin{macrocode}
\def\HoLogoHtml@LyX#1{%
  \HoLogoCss@LyX
  \HOLOGO@Span{LyX}{%
    L%
    \HOLOGO@Span{y}{Y}%
    X%
  }%
}
%    \end{macrocode}
%    \end{macro}
%    \begin{macro}{\HoLogoCss@LyX}
%    \begin{macrocode}
\def\HoLogoCss@LyX{%
  \Css{%
    span.HoLogo-LyX span.HoLogo-y{%
      position:relative;%
      top:.25em;%
      margin-left:-.1667em;%
      margin-right:-.125em;%
      text-decoration:none;%
    }%
  }%
  \global\let\HoLogoCss@LyX\relax
}
%    \end{macrocode}
%    \end{macro}
%
% \subsubsection{\hologo{NTS}}
%
%    \begin{macro}{\HoLogo@NTS}
%    Definition for \hologo{NTS} can be found in
%    package \xpackage{etex\textunderscore man} for the \hologo{eTeX} manual \cite{etexman}
%    and in package \xpackage{dtklogos} \cite{dtklogos}:
%\begin{quote}
%\begin{verbatim}
%\def\NTS{%
%  \leavevmode
%  \hbox{%
%    $%
%      \cal N%
%      \kern-0.35em%
%      \lower0.5ex\hbox{$\cal T$}%
%      \kern-0.2em%
%      S%
%    $%
%  }%
%}
%\end{verbatim}
%\end{quote}
%    \begin{macrocode}
\def\HoLogo@NTS#1{%
  \HoLogoFont@font{NTS}{sy}{%
    N\/%
    \kern-.35em%
    \lower.5ex\hbox{T\/}%
    \kern-.2em%
    S\/%
  }%
  \HOLOGO@SpaceFactor
}
%    \end{macrocode}
%    \end{macro}
%
% \subsubsection{\Hologo{TTH} (\hologo{TeX} to HTML translator)}
%
%    Source: \url{http://hutchinson.belmont.ma.us/tth/}
%    In the HTML source the second `T' is printed as subscript.
%\begin{quote}
%\begin{verbatim}
%T<sub>T</sub>H
%\end{verbatim}
%\end{quote}
%    \begin{macro}{\HoLogo@TTH}
%    \begin{macrocode}
\def\HoLogo@TTH#1{%
  \ltx@mbox{%
    T\HOLOGO@SubScript{T}H%
  }%
  \HOLOGO@SpaceFactor
}
%    \end{macrocode}
%    \end{macro}
%
%    \begin{macro}{\HoLogoHtml@TTH}
%    \begin{macrocode}
\def\HoLogoHtml@TTH#1{%
  T\HCode{<sub>}T\HCode{</sub>}H%
}
%    \end{macrocode}
%    \end{macro}
%
% \subsubsection{\Hologo{HanTheThanh}}
%
%    Partial source: Package \xpackage{dtklogos}.
%    The double accent is U+1EBF (latin small letter e with circumflex
%    and acute).
%    \begin{macro}{\HoLogo@HanTheThanh}
%    \begin{macrocode}
\def\HoLogo@HanTheThanh#1{%
  \ltx@mbox{H\`an}%
  \HOLOGO@space
  \ltx@mbox{%
    Th%
    \HOLOGO@IfCharExists{"1EBF}{%
      \char"1EBF\relax
    }{%
      \^e\hbox to 0pt{\hss\raise .5ex\hbox{\'{}}}%
    }%
  }%
  \HOLOGO@space
  \ltx@mbox{Th\`anh}%
}
%    \end{macrocode}
%    \end{macro}
%    \begin{macro}{\HoLogoBkm@HanTheThanh}
%    \begin{macrocode}
\def\HoLogoBkm@HanTheThanh#1{%
  H\`an %
  Th\HOLOGO@PdfdocUnicode{\^e}{\9036\277} %
  Th\`anh%
}
%    \end{macrocode}
%    \end{macro}
%    \begin{macro}{\HoLogoHtml@HanTheThanh}
%    \begin{macrocode}
\def\HoLogoHtml@HanTheThanh#1{%
  H\`an %
  Th\HCode{&\ltx@hashchar x1ebf;} %
  Th\`anh%
}
%    \end{macrocode}
%    \end{macro}
%
% \subsection{Driver detection}
%
%    \begin{macrocode}
\HOLOGO@IfExists\InputIfFileExists{%
  \InputIfFileExists{hologo.cfg}{}{}%
}{%
  \ltx@IfUndefined{pdf@filesize}{%
    \def\HOLOGO@InputIfExists{%
      \openin\HOLOGO@temp=hologo.cfg\relax
      \ifeof\HOLOGO@temp
        \closein\HOLOGO@temp
      \else
        \closein\HOLOGO@temp
        \begingroup
          \def\x{LaTeX2e}%
        \expandafter\endgroup
        \ifx\fmtname\x
          \input{hologo.cfg}%
        \else
          \input hologo.cfg\relax
        \fi
      \fi
    }%
    \ltx@IfUndefined{newread}{%
      \chardef\HOLOGO@temp=15 %
      \def\HOLOGO@CheckRead{%
        \ifeof\HOLOGO@temp
          \HOLOGO@InputIfExists
        \else
          \ifcase\HOLOGO@temp
            \@PackageWarningNoLine{hologo}{%
              Configuration file ignored, because\MessageBreak
              a free read register could not be found%
            }%
          \else
            \begingroup
              \count\ltx@cclv=\HOLOGO@temp
              \advance\ltx@cclv by \ltx@minusone
              \edef\x{\endgroup
                \chardef\noexpand\HOLOGO@temp=\the\count\ltx@cclv
                \relax
              }%
            \x
          \fi
        \fi
      }%
    }{%
      \csname newread\endcsname\HOLOGO@temp
      \HOLOGO@InputIfExists
    }%
  }{%
    \edef\HOLOGO@temp{\pdf@filesize{hologo.cfg}}%
    \ifx\HOLOGO@temp\ltx@empty
    \else
      \ifnum\HOLOGO@temp>0 %
        \begingroup
          \def\x{LaTeX2e}%
        \expandafter\endgroup
        \ifx\fmtname\x
          \input{hologo.cfg}%
        \else
          \input hologo.cfg\relax
        \fi
      \else
        \@PackageInfoNoLine{hologo}{%
          Empty configuration file `hologo.cfg' ignored%
        }%
      \fi
    \fi
  }%
}
%    \end{macrocode}
%
%    \begin{macrocode}
\def\HOLOGO@temp#1#2{%
  \kv@define@key{HoLogoDriver}{#1}[]{%
    \begingroup
      \def\HOLOGO@temp{##1}%
      \ltx@onelevel@sanitize\HOLOGO@temp
      \ifx\HOLOGO@temp\ltx@empty
      \else
        \@PackageError{hologo}{%
          Value (\HOLOGO@temp) not permitted for option `#1'%
        }%
        \@ehc
      \fi
    \endgroup
    \def\hologoDriver{#2}%
  }%
}%
\def\HOLOGO@@temp#1#2{%
  \ifx\kv@value\relax
    \HOLOGO@temp{#1}{#1}%
  \else
    \HOLOGO@temp{#1}{#2}%
  \fi
}%
\kv@parse@normalized{%
  pdftex,%
  luatex=pdftex,%
  dvipdfm,%
  dvipdfmx=dvipdfm,%
  dvips,%
  dvipsone=dvips,%
  xdvi=dvips,%
  xetex,%
  vtex,%
}\HOLOGO@@temp
%    \end{macrocode}
%
%    \begin{macrocode}
\kv@define@key{HoLogoDriver}{driverfallback}{%
  \def\HOLOGO@DriverFallback{#1}%
}
%    \end{macrocode}
%
%    \begin{macro}{\HOLOGO@DriverFallback}
%    \begin{macrocode}
\def\HOLOGO@DriverFallback{dvips}
%    \end{macrocode}
%    \end{macro}
%
%    \begin{macro}{\hologoDriverSetup}
%    \begin{macrocode}
\def\hologoDriverSetup{%
  \let\hologoDriver\ltx@undefined
  \HOLOGO@DriverSetup
}
%    \end{macrocode}
%    \end{macro}
%
%    \begin{macro}{\HOLOGO@DriverSetup}
%    \begin{macrocode}
\def\HOLOGO@DriverSetup#1{%
  \kvsetkeys{HoLogoDriver}{#1}%
  \HOLOGO@CheckDriver
  \ltx@ifundefined{hologoDriver}{%
    \begingroup
    \edef\x{\endgroup
      \noexpand\kvsetkeys{HoLogoDriver}{\HOLOGO@DriverFallback}%
    }\x
  }{}%
  \@PackageInfoNoLine{hologo}{Using driver `\hologoDriver'}%
}
%    \end{macrocode}
%    \end{macro}
%
%    \begin{macro}{\HOLOGO@CheckDriver}
%    \begin{macrocode}
\def\HOLOGO@CheckDriver{%
  \ifpdf
    \def\hologoDriver{pdftex}%
    \let\HOLOGO@pdfliteral\pdfliteral
    \ifluatex
      \ifx\pdfextension\@undefined\else
        \protected\def\pdfliteral{\pdfextension literal}%
        \let\HOLOGO@pdfliteral\pdfliteral
      \fi
      \ltx@IfUndefined{HOLOGO@pdfliteral}{%
        \ifnum\luatexversion<36 %
        \else
          \begingroup
            \let\HOLOGO@temp\endgroup
            \ifcase0%
                \directlua{%
                  if tex.enableprimitives then %
                    tex.enableprimitives('HOLOGO@', {'pdfliteral'})%
                  else %
                    tex.print('1')%
                  end%
                }%
                \ifx\HOLOGO@pdfliteral\@undefined 1\fi%
                \relax%
              \endgroup
              \let\HOLOGO@temp\relax
              \global\let\HOLOGO@pdfliteral\HOLOGO@pdfliteral
            \fi%
          \HOLOGO@temp
        \fi
      }{}%
    \fi
    \ltx@IfUndefined{HOLOGO@pdfliteral}{%
      \@PackageWarningNoLine{hologo}{%
        Cannot find \string\pdfliteral
      }%
    }{}%
  \else
    \ifxetex
      \def\hologoDriver{xetex}%
    \else
      \ifvtex
        \def\hologoDriver{vtex}%
      \fi
    \fi
  \fi
}
%    \end{macrocode}
%    \end{macro}
%
%    \begin{macro}{\HOLOGO@WarningUnsupportedDriver}
%    \begin{macrocode}
\def\HOLOGO@WarningUnsupportedDriver#1{%
  \@PackageWarningNoLine{hologo}{%
    Logo `#1' needs driver specific macros,\MessageBreak
    but driver `\hologoDriver' is not supported.\MessageBreak
    Use a different driver or\MessageBreak
    load package `graphics' or `pgf'%
  }%
}
%    \end{macrocode}
%    \end{macro}
%
% \subsubsection{Reflect box macros}
%
%    Skip driver part if not needed.
%    \begin{macrocode}
\ltx@IfUndefined{reflectbox}{}{%
  \ltx@IfUndefined{rotatebox}{}{%
    \HOLOGO@AtEnd
  }%
}
\ltx@IfUndefined{pgftext}{}{%
  \HOLOGO@AtEnd
}
\ltx@IfUndefined{psscalebox}{}{%
  \HOLOGO@AtEnd
}
%    \end{macrocode}
%
%    \begin{macrocode}
\def\HOLOGO@temp{LaTeX2e}
\ifx\fmtname\HOLOGO@temp
  \RequirePackage{kvoptions}[2011/06/30]%
  \ProcessKeyvalOptions{HoLogoDriver}%
\fi
\HOLOGO@DriverSetup{}
%    \end{macrocode}
%
%    \begin{macro}{\HOLOGO@ReflectBox}
%    \begin{macrocode}
\def\HOLOGO@ReflectBox#1{%
  \begingroup
    \setbox\ltx@zero\hbox{\begingroup#1\endgroup}%
    \setbox\ltx@two\hbox{%
      \kern\wd\ltx@zero
      \csname HOLOGO@ScaleBox@\hologoDriver\endcsname{-1}{1}{%
        \hbox to 0pt{\copy\ltx@zero\hss}%
      }%
    }%
    \wd\ltx@two=\wd\ltx@zero
    \box\ltx@two
  \endgroup
}
%    \end{macrocode}
%    \end{macro}
%
%    \begin{macro}{\HOLOGO@PointReflectBox}
%    \begin{macrocode}
\def\HOLOGO@PointReflectBox#1{%
  \begingroup
    \setbox\ltx@zero\hbox{\begingroup#1\endgroup}%
    \setbox\ltx@two\hbox{%
      \kern\wd\ltx@zero
      \raise\ht\ltx@zero\hbox{%
        \csname HOLOGO@ScaleBox@\hologoDriver\endcsname{-1}{-1}{%
          \hbox to 0pt{\copy\ltx@zero\hss}%
        }%
      }%
    }%
    \wd\ltx@two=\wd\ltx@zero
    \box\ltx@two
  \endgroup
}
%    \end{macrocode}
%    \end{macro}
%
%    We must define all variants because of dynamic driver setup.
%    \begin{macrocode}
\def\HOLOGO@temp#1#2{#2}
%    \end{macrocode}
%
%    \begin{macro}{\HOLOGO@ScaleBox@pdftex}
%    \begin{macrocode}
\HOLOGO@temp{pdftex}{%
  \def\HOLOGO@ScaleBox@pdftex#1#2#3{%
    \HOLOGO@pdfliteral{%
      q #1 0 0 #2 0 0 cm%
    }%
    #3%
    \HOLOGO@pdfliteral{%
      Q%
    }%
  }%
}
%    \end{macrocode}
%    \end{macro}
%    \begin{macro}{\HOLOGO@ScaleBox@dvips}
%    \begin{macrocode}
\HOLOGO@temp{dvips}{%
  \def\HOLOGO@ScaleBox@dvips#1#2#3{%
    \special{ps:%
      gsave %
      currentpoint %
      currentpoint translate %
      #1 #2 scale %
      neg exch neg exch translate%
    }%
    #3%
    \special{ps:%
      currentpoint %
      grestore %
      moveto%
    }%
  }%
}
%    \end{macrocode}
%    \end{macro}
%    \begin{macro}{\HOLOGO@ScaleBox@dvipdfm}
%    \begin{macrocode}
\HOLOGO@temp{dvipdfm}{%
  \let\HOLOGO@ScaleBox@dvipdfm\HOLOGO@ScaleBox@dvips
}
%    \end{macrocode}
%    \end{macro}
%    Since \hologo{XeTeX} v0.6.
%    \begin{macro}{\HOLOGO@ScaleBox@xetex}
%    \begin{macrocode}
\HOLOGO@temp{xetex}{%
  \def\HOLOGO@ScaleBox@xetex#1#2#3{%
    \special{x:gsave}%
    \special{x:scale #1 #2}%
    #3%
    \special{x:grestore}%
  }%
}
%    \end{macrocode}
%    \end{macro}
%    \begin{macro}{\HOLOGO@ScaleBox@vtex}
%    \begin{macrocode}
\HOLOGO@temp{vtex}{%
  \def\HOLOGO@ScaleBox@vtex#1#2#3{%
    \special{r(#1,0,0,#2,0,0}%
    #3%
    \special{r)}%
  }%
}
%    \end{macrocode}
%    \end{macro}
%
%    \begin{macrocode}
\HOLOGO@AtEnd%
%</package>
%    \end{macrocode}
%
% \section{Test}
%
% \subsection{Catcode checks for loading}
%
%    \begin{macrocode}
%<*test1>
%    \end{macrocode}
%    \begin{macrocode}
\catcode`\{=1 %
\catcode`\}=2 %
\catcode`\#=6 %
\catcode`\@=11 %
\expandafter\ifx\csname count@\endcsname\relax
  \countdef\count@=255 %
\fi
\expandafter\ifx\csname @gobble\endcsname\relax
  \long\def\@gobble#1{}%
\fi
\expandafter\ifx\csname @firstofone\endcsname\relax
  \long\def\@firstofone#1{#1}%
\fi
\expandafter\ifx\csname loop\endcsname\relax
  \expandafter\@firstofone
\else
  \expandafter\@gobble
\fi
{%
  \def\loop#1\repeat{%
    \def\body{#1}%
    \iterate
  }%
  \def\iterate{%
    \body
      \let\next\iterate
    \else
      \let\next\relax
    \fi
    \next
  }%
  \let\repeat=\fi
}%
\def\RestoreCatcodes{}
\count@=0 %
\loop
  \edef\RestoreCatcodes{%
    \RestoreCatcodes
    \catcode\the\count@=\the\catcode\count@\relax
  }%
\ifnum\count@<255 %
  \advance\count@ 1 %
\repeat

\def\RangeCatcodeInvalid#1#2{%
  \count@=#1\relax
  \loop
    \catcode\count@=15 %
  \ifnum\count@<#2\relax
    \advance\count@ 1 %
  \repeat
}
\def\RangeCatcodeCheck#1#2#3{%
  \count@=#1\relax
  \loop
    \ifnum#3=\catcode\count@
    \else
      \errmessage{%
        Character \the\count@\space
        with wrong catcode \the\catcode\count@\space
        instead of \number#3%
      }%
    \fi
  \ifnum\count@<#2\relax
    \advance\count@ 1 %
  \repeat
}
\def\space{ }
\expandafter\ifx\csname LoadCommand\endcsname\relax
  \def\LoadCommand{\input hologo.sty\relax}%
\fi
\def\Test{%
  \RangeCatcodeInvalid{0}{47}%
  \RangeCatcodeInvalid{58}{64}%
  \RangeCatcodeInvalid{91}{96}%
  \RangeCatcodeInvalid{123}{255}%
  \catcode`\@=12 %
  \catcode`\\=0 %
  \catcode`\%=14 %
  \LoadCommand
  \RangeCatcodeCheck{0}{36}{15}%
  \RangeCatcodeCheck{37}{37}{14}%
  \RangeCatcodeCheck{38}{47}{15}%
  \RangeCatcodeCheck{48}{57}{12}%
  \RangeCatcodeCheck{58}{63}{15}%
  \RangeCatcodeCheck{64}{64}{12}%
  \RangeCatcodeCheck{65}{90}{11}%
  \RangeCatcodeCheck{91}{91}{15}%
  \RangeCatcodeCheck{92}{92}{0}%
  \RangeCatcodeCheck{93}{96}{15}%
  \RangeCatcodeCheck{97}{122}{11}%
  \RangeCatcodeCheck{123}{255}{15}%
  \RestoreCatcodes
}
\Test
\csname @@end\endcsname
\end
%    \end{macrocode}
%    \begin{macrocode}
%</test1>
%    \end{macrocode}
%
% \subsection{Spacefactor}
%
%    The space factor must be 1000 after a logo. If it is greater 1000
%    then the following space is a space after a sentence closing point.
%    If the space factor is smaller 1000 then an immediate following
%    dot is interpreted as abbreviation, not sentence closing point.
%
%    \begin{macrocode}
%<*test-spacefactor>
\NeedsTeXFormat{LaTeX2e}
\documentclass{article}
\usepackage{hologo}[2016/05/12]
\usepackage{kvsetkeys}
\usepackage{qstest}
\IncludeTests{*}
\LogTests{log}{*}{*}
\begin{document}
\begin{qstest}{spacefactor}{spacefactor}
\newcommand*{\Test}[1]{%
  \sbox0{%
    \hologo{#1}%
    \Expect*{1000 (#1)}*{\the\spacefactor\space(#1)}%
  }%
}%
\makeatletter
\def\TestList{}
\def\hologoEntry#1#2#3{%
  \edef\TestList{%
    \ifx\TestList\@empty
    \else
      \TestList,%
    \fi
    #1%
    \ifx\\#2\\%
    \else
      ={variant=#2}%
    \fi
  }%
}
\hologoList
\expandafter\kv@parse@normalized\expandafter{%
  \TestList
}{%
  \begingroup
    \let\@logo=\kv@key
    \ifx\kv@value\relax
    \else
      \expandafter\hologoLogoSetup\expandafter\@logo\expandafter{%
        \kv@value
      }%
    \fi
    \Test\@logo
  \endgroup
  \@gobbletwo
}
\end{qstest}
\end{document}
%</test-spacefactor>
%    \end{macrocode}
%
% \subsection{Complete list}
%
%    \begin{macrocode}
%<*test-list>
\NeedsTeXFormat{LaTeX2e}
\documentclass[12pt,a4paper]{article}
\usepackage{hologo}[2016/05/12]
\usepackage[T1]{fontenc}
\usepackage{lmodern}
\usepackage{parskip}
\usepackage[unicode]{hyperref}[2011/09/28]
\usepackage{bookmark}[2011/09/19]
\bookmarksetup{%
  numbered,%
  open,%
  openlevel=2,%
}
\renewcommand*{\contentsname}{List of logos}
\begin{document}
\tableofcontents
\def\TestFont#1#2#3#4#5#6{%
  \begingroup
    \usefont{#3}{#4}{#5}{#6}%
    \HologoVariant{#1}{#2}/\hologoVariant{#1}{#2}%
    \quad
    \begingroup\scriptsize\hologoVariant{#1}{#2}\endgroup
    \quad
  \endgroup
  (#3/#4/#5/#6)%
  \par
}
\makeatletter
\def\hologoEntry#1#2#3{%
  \section{%
    \HologoVariant{#1}{#2}/\hologoVariant{#1}{#2} %
    {[#1\ifx\\#2\\\else\space(#2)\fi]}% hash-ok
  }% braces around [] because of bug in tex4ht
  \begingroup
    \hypersetup{unicode=false}%
    \bookmark[%
      dest=\@currentHref,%
      rellevel=1,%
      keeplevel,%
    ]{%
      \HologoVariant{#1}{#2}/\hologoVariant{#1}{#2} %
      (PDFDocEncoding)%
    }%
  \endgroup
  \TestFont{#1}{#2}{OT1}{cmr}{m}{n}%
  \TestFont{#1}{#2}{OT1}{cmss}{m}{n}%
  \TestFont{#1}{#2}{OT1}{cmr}{b}{n}%
  \TestFont{#1}{#2}{OT1}{cmr}{m}{it}%
  \TestFont{#1}{#2}{OT1}{cmtt}{m}{n}%
  \TestFont{#1}{#2}{T1}{lmr}{m}{n}%
  \TestFont{#1}{#2}{T1}{lmss}{m}{n}%
  \TestFont{#1}{#2}{T1}{lmr}{b}{n}%
  \TestFont{#1}{#2}{T1}{lmr}{m}{it}%
  \TestFont{#1}{#2}{T1}{lmtt}{m}{n}%
  \TestFont{#1}{#2}{T1}{lmvtt}{m}{n}%
  \TestFont{#1}{#2}{T1}{qtm}{m}{n}%
  \TestFont{#1}{#2}{T1}{qhv}{m}{n}%
  \TestFont{#1}{#2}{T1}{qtm}{b}{n}%
  \TestFont{#1}{#2}{T1}{qtm}{m}{it}%
  \TestFont{#1}{#2}{T1}{qcr}{m}{n}%
  \newpage
}
\makeatother
\hologoList
\end{document}
%</test-list>
%    \end{macrocode}
%
% \section{Installation}
%
% \subsection{Download}
%
% \paragraph{Package.} This package is available on
% CTAN\footnote{\url{ftp://ftp.ctan.org/tex-archive/}}:
% \begin{description}
% \item[\CTAN{macros/latex/contrib/oberdiek/hologo.dtx}] The source file.
% \item[\CTAN{macros/latex/contrib/oberdiek/hologo.pdf}] Documentation.
% \end{description}
%
%
% \paragraph{Bundle.} All the packages of the bundle `oberdiek'
% are also available in a TDS compliant ZIP archive. There
% the packages are already unpacked and the documentation files
% are generated. The files and directories obey the TDS standard.
% \begin{description}
% \item[\CTAN{install/macros/latex/contrib/oberdiek.tds.zip}]
% \end{description}
% \emph{TDS} refers to the standard ``A Directory Structure
% for \TeX\ Files'' (\CTAN{tds/tds.pdf}). Directories
% with \xfile{texmf} in their name are usually organized this way.
%
% \subsection{Bundle installation}
%
% \paragraph{Unpacking.} Unpack the \xfile{oberdiek.tds.zip} in the
% TDS tree (also known as \xfile{texmf} tree) of your choice.
% Example (linux):
% \begin{quote}
%   |unzip oberdiek.tds.zip -d ~/texmf|
% \end{quote}
%
% \paragraph{Script installation.}
% Check the directory \xfile{TDS:scripts/oberdiek/} for
% scripts that need further installation steps.
% Package \xpackage{attachfile2} comes with the Perl script
% \xfile{pdfatfi.pl} that should be installed in such a way
% that it can be called as \texttt{pdfatfi}.
% Example (linux):
% \begin{quote}
%   |chmod +x scripts/oberdiek/pdfatfi.pl|\\
%   |cp scripts/oberdiek/pdfatfi.pl /usr/local/bin/|
% \end{quote}
%
% \subsection{Package installation}
%
% \paragraph{Unpacking.} The \xfile{.dtx} file is a self-extracting
% \docstrip\ archive. The files are extracted by running the
% \xfile{.dtx} through \plainTeX:
% \begin{quote}
%   \verb|tex hologo.dtx|
% \end{quote}
%
% \paragraph{TDS.} Now the different files must be moved into
% the different directories in your installation TDS tree
% (also known as \xfile{texmf} tree):
% \begin{quote}
% \def\t{^^A
% \begin{tabular}{@{}>{\ttfamily}l@{ $\rightarrow$ }>{\ttfamily}l@{}}
%   hologo.sty & tex/generic/oberdiek/hologo.sty\\
%   hologo.pdf & doc/latex/oberdiek/hologo.pdf\\
%   example/hologo-example.tex & doc/latex/oberdiek/example/hologo-example.tex\\
%   test/hologo-test1.tex & doc/latex/oberdiek/test/hologo-test1.tex\\
%   test/hologo-test-spacefactor.tex & doc/latex/oberdiek/test/hologo-test-spacefactor.tex\\
%   test/hologo-test-list.tex & doc/latex/oberdiek/test/hologo-test-list.tex\\
%   hologo.dtx & source/latex/oberdiek/hologo.dtx\\
% \end{tabular}^^A
% }^^A
% \sbox0{\t}^^A
% \ifdim\wd0>\linewidth
%   \begingroup
%     \advance\linewidth by\leftmargin
%     \advance\linewidth by\rightmargin
%   \edef\x{\endgroup
%     \def\noexpand\lw{\the\linewidth}^^A
%   }\x
%   \def\lwbox{^^A
%     \leavevmode
%     \hbox to \linewidth{^^A
%       \kern-\leftmargin\relax
%       \hss
%       \usebox0
%       \hss
%       \kern-\rightmargin\relax
%     }^^A
%   }^^A
%   \ifdim\wd0>\lw
%     \sbox0{\small\t}^^A
%     \ifdim\wd0>\linewidth
%       \ifdim\wd0>\lw
%         \sbox0{\footnotesize\t}^^A
%         \ifdim\wd0>\linewidth
%           \ifdim\wd0>\lw
%             \sbox0{\scriptsize\t}^^A
%             \ifdim\wd0>\linewidth
%               \ifdim\wd0>\lw
%                 \sbox0{\tiny\t}^^A
%                 \ifdim\wd0>\linewidth
%                   \lwbox
%                 \else
%                   \usebox0
%                 \fi
%               \else
%                 \lwbox
%               \fi
%             \else
%               \usebox0
%             \fi
%           \else
%             \lwbox
%           \fi
%         \else
%           \usebox0
%         \fi
%       \else
%         \lwbox
%       \fi
%     \else
%       \usebox0
%     \fi
%   \else
%     \lwbox
%   \fi
% \else
%   \usebox0
% \fi
% \end{quote}
% If you have a \xfile{docstrip.cfg} that configures and enables \docstrip's
% TDS installing feature, then some files can already be in the right
% place, see the documentation of \docstrip.
%
% \subsection{Refresh file name databases}
%
% If your \TeX~distribution
% (\teTeX, \mikTeX, \dots) relies on file name databases, you must refresh
% these. For example, \teTeX\ users run \verb|texhash| or
% \verb|mktexlsr|.
%
% \subsection{Some details for the interested}
%
% \paragraph{Attached source.}
%
% The PDF documentation on CTAN also includes the
% \xfile{.dtx} source file. It can be extracted by
% AcrobatReader 6 or higher. Another option is \textsf{pdftk},
% e.g. unpack the file into the current directory:
% \begin{quote}
%   \verb|pdftk hologo.pdf unpack_files output .|
% \end{quote}
%
% \paragraph{Unpacking with \LaTeX.}
% The \xfile{.dtx} chooses its action depending on the format:
% \begin{description}
% \item[\plainTeX:] Run \docstrip\ and extract the files.
% \item[\LaTeX:] Generate the documentation.
% \end{description}
% If you insist on using \LaTeX\ for \docstrip\ (really,
% \docstrip\ does not need \LaTeX), then inform the autodetect routine
% about your intention:
% \begin{quote}
%   \verb|latex \let\install=y\input{hologo.dtx}|
% \end{quote}
% Do not forget to quote the argument according to the demands
% of your shell.
%
% \paragraph{Generating the documentation.}
% You can use both the \xfile{.dtx} or the \xfile{.drv} to generate
% the documentation. The process can be configured by the
% configuration file \xfile{ltxdoc.cfg}. For instance, put this
% line into this file, if you want to have A4 as paper format:
% \begin{quote}
%   \verb|\PassOptionsToClass{a4paper}{article}|
% \end{quote}
% An example follows how to generate the
% documentation with pdf\LaTeX:
% \begin{quote}
%\begin{verbatim}
%pdflatex hologo.dtx
%makeindex -s gind.ist hologo.idx
%pdflatex hologo.dtx
%makeindex -s gind.ist hologo.idx
%pdflatex hologo.dtx
%\end{verbatim}
% \end{quote}
%
% \section{Catalogue}
%
% The following XML file can be used as source for the
% \href{http://mirror.ctan.org/help/Catalogue/catalogue.html}{\TeX\ Catalogue}.
% The elements \texttt{caption} and \texttt{description} are imported
% from the original XML file from the Catalogue.
% The name of the XML file in the Catalogue is \xfile{hologo.xml}.
%    \begin{macrocode}
%<*catalogue>
<?xml version='1.0' encoding='us-ascii'?>
<!DOCTYPE entry SYSTEM 'catalogue.dtd'>
<entry datestamp='$Date$' modifier='$Author$' id='hologo'>
  <name>hologo</name>
  <caption>A collection of logos with bookmark support.</caption>
  <authorref id='auth:oberdiek'/>
  <copyright owner='Heiko Oberdiek' year='2010-2012'/>
  <license type='lppl1.3'/>
  <version number='1.10'/>
  <description>
    The package defines a single command <tt>\hologo</tt>, whose
    argument is the usual case-confused ASCII version of the logo.
    The command is bookmark-enabled, so that every logo becomes
    available in bookmarks without further work.
    <p/>
    The package is part of the <xref refid='oberdiek'>oberdiek</xref>
    bundle.
  </description>
  <documentation details='Package documentation'
      href='ctan:/macros/latex/contrib/oberdiek/hologo.pdf'/>
  <ctan file='true' path='/macros/latex/contrib/oberdiek/hologo.dtx'/>
  <miktex location='oberdiek'/>
  <texlive location='oberdiek'/>
  <install path='/macros/latex/contrib/oberdiek/oberdiek.tds.zip'/>
</entry>
%</catalogue>
%    \end{macrocode}
%
% \begin{thebibliography}{9}
% \raggedright
%
% \bibitem{btxdoc}
% Oren Patashnik,
% \textit{\hologo{BibTeX}ing},
% 1988-02-08.\\
% \CTAN{biblio/bibtex/base/}
%
% \bibitem{dtklogos}
% Gerd Neugebauer, DANTE,
% \textit{Package \xpackage{dtklogos}},
% 2011-04-25.\\
% \CTAN{usergrps/dante/dtk/dtklogos.sty}
%
% \bibitem{etexman}
% The \hologo{NTS} Team,
% \textit{The \hologo{eTeX} manual},
% 1998-02.\\
% \CTAN{systems/e-tex/v2/doc/}
%
% \bibitem{ExTeX-FAQ}
% The \hologo{ExTeX} group,
% \textit{\hologo{ExTeX}: FAQ -- How is \hologo{ExTeX} typeset?},
% 2007-04-14.\\
% \url{http://www.extex.org/documentation/faq.html}
%
% \bibitem{LyX}
% %@MISC{ LyX,
% %  title = {{LyX 2.0.0 -- The Document Processor [Computer software and manual]}},
% %  author = {{The LyX Team}},
% %  howpublished = {Internet: http://www.lyx.org},
% %  year = {2011-05-08},
% %  note = {Retrieved May 10, 2011, from http://www.lyx.org},
% %  url = {http://www.lyx.org/}
% %}
% The \hologo{LyX} Team,
% \textit{\hologo{LyX} -- The Document Processor},
% 2011-05-08.\\
% \url{http://www.lyx.org/}
%
% \bibitem{OzTeX}
% Andrew Trevorrow,
% \hologo{OzTeX} FAQ: What is the correct way to typeset ``\hologo{OzTeX}''?,
% 2011-09-15 (visited).
% \url{http://www.trevorrow.com/oztex/ozfaq.html#oztex-logo}
%
% \bibitem{PiCTeX}
% Michael Wichura,
% \textit{The \hologo{PiCTeX} macro package},
% 1987-09-21.
% \CTAN{graphics/pictex/}
%
% \bibitem{scrlogo}
% Markus Kohm,
% \textit{\hologo{KOMAScript} Datei \xfile{scrlogo.dtx}},
% 2009-01-30.\\
% \CTAN{install/macros/latex/contrib/komascript.tds.zip}
%
% \end{thebibliography}
%
% \begin{History}
%   \begin{Version}{2010/04/08 v1.0}
%   \item
%     The first version.
%   \end{Version}
%   \begin{Version}{2010/04/16 v1.1}
%   \item
%     \cs{Hologo} added for support of logos at start of a sentence.
%   \item
%     \cs{hologoSetup} and \cs{hologoLogoSetup} added.
%   \item
%     Options \xoption{break}, \xoption{hyphenbreak}, \xoption{spacebreak}
%     added.
%   \item
%     Variant support added by option \xoption{variant}.
%   \end{Version}
%   \begin{Version}{2010/04/24 v1.2}
%   \item
%     \hologo{LaTeX3} added.
%   \item
%     \hologo{VTeX} added.
%   \end{Version}
%   \begin{Version}{2010/11/21 v1.3}
%   \item
%     \hologo{iniTeX}, \hologo{virTeX} added.
%   \end{Version}
%   \begin{Version}{2011/03/25 v1.4}
%   \item
%     \hologo{ConTeXt} with variants added.
%   \item
%     Option \xoption{discretionarybreak} added as refinement for
%     option \xoption{break}.
%   \end{Version}
%   \begin{Version}{2011/04/21 v1.5}
%   \item
%     Wrong TDS directory for test files fixed.
%   \end{Version}
%   \begin{Version}{2011/10/01 v1.6}
%   \item
%     Support for package \xpackage{tex4ht} added.
%   \item
%     Support for \cs{csname} added if \cs{ifincsname} is available.
%   \item
%     New logos:
%     \hologo{(La)TeX},
%     \hologo{biber},
%     \hologo{BibTeX} (\xoption{sc}, \xoption{sf}),
%     \hologo{emTeX},
%     \hologo{ExTeX},
%     \hologo{KOMAScript},
%     \hologo{La},
%     \hologo{LyX},
%     \hologo{MiKTeX},
%     \hologo{NTS},
%     \hologo{OzMF},
%     \hologo{OzMP},
%     \hologo{OzTeX},
%     \hologo{OzTtH},
%     \hologo{PCTeX},
%     \hologo{PiC},
%     \hologo{PiCTeX},
%     \hologo{METAFONT},
%     \hologo{MetaFun},
%     \hologo{METAPOST},
%     \hologo{MetaPost},
%     \hologo{SLiTeX} (\xoption{lift}, \xoption{narrow}, \xoption{simple}),
%     \hologo{SliTeX} (\xoption{narrow}, \xoption{simple}, \xoption{lift}),
%     \hologo{teTeX}.
%   \item
%     Fixes:
%     \hologo{iniTeX},
%     \hologo{pdfLaTeX},
%     \hologo{pdfTeX},
%     \hologo{virTeX}.
%   \item
%     \cs{hologoFontSetup} and \cs{hologoLogoFontSetup} added.
%   \item
%     \cs{hologoVariant} and \cs{HologoVariant} added.
%   \end{Version}
%   \begin{Version}{2011/11/22 v1.7}
%   \item
%     New logos:
%     \hologo{BibTeX8},
%     \hologo{LaTeXML},
%     \hologo{SageTeX},
%     \hologo{TeX4ht},
%     \hologo{TTH}.
%   \item
%     \hologo{Xe} and friends: Driver stuff fixed.
%   \item
%     \hologo{Xe} and friends: Support for italic added.
%   \item
%     \hologo{Xe} and friends: Package support for \xpackage{pgf}
%     and \xpackage{pstricks} added.
%   \end{Version}
%   \begin{Version}{2011/11/29 v1.8}
%   \item
%     New logos:
%     \hologo{HanTheThanh}.
%   \end{Version}
%   \begin{Version}{2011/12/21 v1.9}
%   \item
%     Patch for package \xpackage{ifxetex} added for the case that
%     \cs{newif} is undefined in \hologo{iniTeX}.
%   \item
%     Some fixes for \hologo{iniTeX}.
%   \end{Version}
%   \begin{Version}{2012/04/26 v1.10}
%   \item
%     Fix in bookmark version of logo ``\hologo{HanTheThanh}''.
%   \end{Version}
%   \begin{Version}{2016/05/12 v1.11}
%   \item
%     Update HOLOGO@IfCharExists (previously in texlive)
%   \item define pdfliteral in current luatex.
%   \end{Version}
% \end{History}
%
% \PrintIndex
%
% \Finale
\endinput
|
% \end{quote}
% Do not forget to quote the argument according to the demands
% of your shell.
%
% \paragraph{Generating the documentation.}
% You can use both the \xfile{.dtx} or the \xfile{.drv} to generate
% the documentation. The process can be configured by the
% configuration file \xfile{ltxdoc.cfg}. For instance, put this
% line into this file, if you want to have A4 as paper format:
% \begin{quote}
%   \verb|\PassOptionsToClass{a4paper}{article}|
% \end{quote}
% An example follows how to generate the
% documentation with pdf\LaTeX:
% \begin{quote}
%\begin{verbatim}
%pdflatex hologo.dtx
%makeindex -s gind.ist hologo.idx
%pdflatex hologo.dtx
%makeindex -s gind.ist hologo.idx
%pdflatex hologo.dtx
%\end{verbatim}
% \end{quote}
%
% \section{Catalogue}
%
% The following XML file can be used as source for the
% \href{http://mirror.ctan.org/help/Catalogue/catalogue.html}{\TeX\ Catalogue}.
% The elements \texttt{caption} and \texttt{description} are imported
% from the original XML file from the Catalogue.
% The name of the XML file in the Catalogue is \xfile{hologo.xml}.
%    \begin{macrocode}
%<*catalogue>
<?xml version='1.0' encoding='us-ascii'?>
<!DOCTYPE entry SYSTEM 'catalogue.dtd'>
<entry datestamp='$Date$' modifier='$Author$' id='hologo'>
  <name>hologo</name>
  <caption>A collection of logos with bookmark support.</caption>
  <authorref id='auth:oberdiek'/>
  <copyright owner='Heiko Oberdiek' year='2010-2012'/>
  <license type='lppl1.3'/>
  <version number='1.10'/>
  <description>
    The package defines a single command <tt>\hologo</tt>, whose
    argument is the usual case-confused ASCII version of the logo.
    The command is bookmark-enabled, so that every logo becomes
    available in bookmarks without further work.
    <p/>
    The package is part of the <xref refid='oberdiek'>oberdiek</xref>
    bundle.
  </description>
  <documentation details='Package documentation'
      href='ctan:/macros/latex/contrib/oberdiek/hologo.pdf'/>
  <ctan file='true' path='/macros/latex/contrib/oberdiek/hologo.dtx'/>
  <miktex location='oberdiek'/>
  <texlive location='oberdiek'/>
  <install path='/macros/latex/contrib/oberdiek/oberdiek.tds.zip'/>
</entry>
%</catalogue>
%    \end{macrocode}
%
% \begin{thebibliography}{9}
% \raggedright
%
% \bibitem{btxdoc}
% Oren Patashnik,
% \textit{\hologo{BibTeX}ing},
% 1988-02-08.\\
% \CTAN{biblio/bibtex/base/}
%
% \bibitem{dtklogos}
% Gerd Neugebauer, DANTE,
% \textit{Package \xpackage{dtklogos}},
% 2011-04-25.\\
% \CTAN{usergrps/dante/dtk/dtklogos.sty}
%
% \bibitem{etexman}
% The \hologo{NTS} Team,
% \textit{The \hologo{eTeX} manual},
% 1998-02.\\
% \CTAN{systems/e-tex/v2/doc/}
%
% \bibitem{ExTeX-FAQ}
% The \hologo{ExTeX} group,
% \textit{\hologo{ExTeX}: FAQ -- How is \hologo{ExTeX} typeset?},
% 2007-04-14.\\
% \url{http://www.extex.org/documentation/faq.html}
%
% \bibitem{LyX}
% %@MISC{ LyX,
% %  title = {{LyX 2.0.0 -- The Document Processor [Computer software and manual]}},
% %  author = {{The LyX Team}},
% %  howpublished = {Internet: http://www.lyx.org},
% %  year = {2011-05-08},
% %  note = {Retrieved May 10, 2011, from http://www.lyx.org},
% %  url = {http://www.lyx.org/}
% %}
% The \hologo{LyX} Team,
% \textit{\hologo{LyX} -- The Document Processor},
% 2011-05-08.\\
% \url{http://www.lyx.org/}
%
% \bibitem{OzTeX}
% Andrew Trevorrow,
% \hologo{OzTeX} FAQ: What is the correct way to typeset ``\hologo{OzTeX}''?,
% 2011-09-15 (visited).
% \url{http://www.trevorrow.com/oztex/ozfaq.html#oztex-logo}
%
% \bibitem{PiCTeX}
% Michael Wichura,
% \textit{The \hologo{PiCTeX} macro package},
% 1987-09-21.
% \CTAN{graphics/pictex/}
%
% \bibitem{scrlogo}
% Markus Kohm,
% \textit{\hologo{KOMAScript} Datei \xfile{scrlogo.dtx}},
% 2009-01-30.\\
% \CTAN{install/macros/latex/contrib/komascript.tds.zip}
%
% \end{thebibliography}
%
% \begin{History}
%   \begin{Version}{2010/04/08 v1.0}
%   \item
%     The first version.
%   \end{Version}
%   \begin{Version}{2010/04/16 v1.1}
%   \item
%     \cs{Hologo} added for support of logos at start of a sentence.
%   \item
%     \cs{hologoSetup} and \cs{hologoLogoSetup} added.
%   \item
%     Options \xoption{break}, \xoption{hyphenbreak}, \xoption{spacebreak}
%     added.
%   \item
%     Variant support added by option \xoption{variant}.
%   \end{Version}
%   \begin{Version}{2010/04/24 v1.2}
%   \item
%     \hologo{LaTeX3} added.
%   \item
%     \hologo{VTeX} added.
%   \end{Version}
%   \begin{Version}{2010/11/21 v1.3}
%   \item
%     \hologo{iniTeX}, \hologo{virTeX} added.
%   \end{Version}
%   \begin{Version}{2011/03/25 v1.4}
%   \item
%     \hologo{ConTeXt} with variants added.
%   \item
%     Option \xoption{discretionarybreak} added as refinement for
%     option \xoption{break}.
%   \end{Version}
%   \begin{Version}{2011/04/21 v1.5}
%   \item
%     Wrong TDS directory for test files fixed.
%   \end{Version}
%   \begin{Version}{2011/10/01 v1.6}
%   \item
%     Support for package \xpackage{tex4ht} added.
%   \item
%     Support for \cs{csname} added if \cs{ifincsname} is available.
%   \item
%     New logos:
%     \hologo{(La)TeX},
%     \hologo{biber},
%     \hologo{BibTeX} (\xoption{sc}, \xoption{sf}),
%     \hologo{emTeX},
%     \hologo{ExTeX},
%     \hologo{KOMAScript},
%     \hologo{La},
%     \hologo{LyX},
%     \hologo{MiKTeX},
%     \hologo{NTS},
%     \hologo{OzMF},
%     \hologo{OzMP},
%     \hologo{OzTeX},
%     \hologo{OzTtH},
%     \hologo{PCTeX},
%     \hologo{PiC},
%     \hologo{PiCTeX},
%     \hologo{METAFONT},
%     \hologo{MetaFun},
%     \hologo{METAPOST},
%     \hologo{MetaPost},
%     \hologo{SLiTeX} (\xoption{lift}, \xoption{narrow}, \xoption{simple}),
%     \hologo{SliTeX} (\xoption{narrow}, \xoption{simple}, \xoption{lift}),
%     \hologo{teTeX}.
%   \item
%     Fixes:
%     \hologo{iniTeX},
%     \hologo{pdfLaTeX},
%     \hologo{pdfTeX},
%     \hologo{virTeX}.
%   \item
%     \cs{hologoFontSetup} and \cs{hologoLogoFontSetup} added.
%   \item
%     \cs{hologoVariant} and \cs{HologoVariant} added.
%   \end{Version}
%   \begin{Version}{2011/11/22 v1.7}
%   \item
%     New logos:
%     \hologo{BibTeX8},
%     \hologo{LaTeXML},
%     \hologo{SageTeX},
%     \hologo{TeX4ht},
%     \hologo{TTH}.
%   \item
%     \hologo{Xe} and friends: Driver stuff fixed.
%   \item
%     \hologo{Xe} and friends: Support for italic added.
%   \item
%     \hologo{Xe} and friends: Package support for \xpackage{pgf}
%     and \xpackage{pstricks} added.
%   \end{Version}
%   \begin{Version}{2011/11/29 v1.8}
%   \item
%     New logos:
%     \hologo{HanTheThanh}.
%   \end{Version}
%   \begin{Version}{2011/12/21 v1.9}
%   \item
%     Patch for package \xpackage{ifxetex} added for the case that
%     \cs{newif} is undefined in \hologo{iniTeX}.
%   \item
%     Some fixes for \hologo{iniTeX}.
%   \end{Version}
%   \begin{Version}{2012/04/26 v1.10}
%   \item
%     Fix in bookmark version of logo ``\hologo{HanTheThanh}''.
%   \end{Version}
%   \begin{Version}{2016/05/12 v1.11}
%   \item
%     Update HOLOGO@IfCharExists (previously in texlive)
%   \item define pdfliteral in current luatex.
%   \end{Version}
% \end{History}
%
% \PrintIndex
%
% \Finale
\endinput

%        (quote the arguments according to the demands of your shell)
%
% Documentation:
%    (a) If hologo.drv is present:
%           latex hologo.drv
%    (b) Without hologo.drv:
%           latex hologo.dtx; ...
%    The class ltxdoc loads the configuration file ltxdoc.cfg
%    if available. Here you can specify further options, e.g.
%    use A4 as paper format:
%       \PassOptionsToClass{a4paper}{article}
%
%    Programm calls to get the documentation (example):
%       pdflatex hologo.dtx
%       makeindex -s gind.ist hologo.idx
%       pdflatex hologo.dtx
%       makeindex -s gind.ist hologo.idx
%       pdflatex hologo.dtx
%
% Installation:
%    TDS:tex/generic/oberdiek/hologo.sty
%    TDS:doc/latex/oberdiek/hologo.pdf
%    TDS:doc/latex/oberdiek/example/hologo-example.tex
%    TDS:doc/latex/oberdiek/test/hologo-test1.tex
%    TDS:doc/latex/oberdiek/test/hologo-test-spacefactor.tex
%    TDS:doc/latex/oberdiek/test/hologo-test-list.tex
%    TDS:source/latex/oberdiek/hologo.dtx
%
%<*ignore>
\begingroup
  \catcode123=1 %
  \catcode125=2 %
  \def\x{LaTeX2e}%
\expandafter\endgroup
\ifcase 0\ifx\install y1\fi\expandafter
         \ifx\csname processbatchFile\endcsname\relax\else1\fi
         \ifx\fmtname\x\else 1\fi\relax
\else\csname fi\endcsname
%</ignore>
%<*install>
\input docstrip.tex
\Msg{************************************************************************}
\Msg{* Installation}
\Msg{* Package: hologo 2016/05/12 v1.11 A logo collection with bookmark support (HO)}
\Msg{************************************************************************}

\keepsilent
\askforoverwritefalse

\let\MetaPrefix\relax
\preamble

This is a generated file.

Project: hologo
Version: 2016/05/12 v1.11

Copyright (C) 2010-2012 by
   Heiko Oberdiek <heiko.oberdiek at googlemail.com>

This work may be distributed and/or modified under the
conditions of the LaTeX Project Public License, either
version 1.3c of this license or (at your option) any later
version. This version of this license is in
   http://www.latex-project.org/lppl/lppl-1-3c.txt
and the latest version of this license is in
   http://www.latex-project.org/lppl.txt
and version 1.3 or later is part of all distributions of
LaTeX version 2005/12/01 or later.

This work has the LPPL maintenance status "maintained".

This Current Maintainer of this work is Heiko Oberdiek.

The Base Interpreter refers to any `TeX-Format',
because some files are installed in TDS:tex/generic//.

This work consists of the main source file hologo.dtx
and the derived files
   hologo.sty, hologo.pdf, hologo.ins, hologo.drv, hologo-example.tex,
   hologo-test1.tex, hologo-test-spacefactor.tex,
   hologo-test-list.tex.

\endpreamble
\let\MetaPrefix\DoubleperCent

\generate{%
  \file{hologo.ins}{\from{hologo.dtx}{install}}%
  \file{hologo.drv}{\from{hologo.dtx}{driver}}%
  \usedir{tex/generic/oberdiek}%
  \file{hologo.sty}{\from{hologo.dtx}{package}}%
  \usedir{doc/latex/oberdiek/example}%
  \file{hologo-example.tex}{\from{hologo.dtx}{example}}%
  \usedir{doc/latex/oberdiek/test}%
  \file{hologo-test1.tex}{\from{hologo.dtx}{test1}}%
  \file{hologo-test-spacefactor.tex}{\from{hologo.dtx}{test-spacefactor}}%
  \file{hologo-test-list.tex}{\from{hologo.dtx}{test-list}}%
  \nopreamble
  \nopostamble
  \usedir{source/latex/oberdiek/catalogue}%
  \file{hologo.xml}{\from{hologo.dtx}{catalogue}}%
}

\catcode32=13\relax% active space
\let =\space%
\Msg{************************************************************************}
\Msg{*}
\Msg{* To finish the installation you have to move the following}
\Msg{* file into a directory searched by TeX:}
\Msg{*}
\Msg{*     hologo.sty}
\Msg{*}
\Msg{* To produce the documentation run the file `hologo.drv'}
\Msg{* through LaTeX.}
\Msg{*}
\Msg{* Happy TeXing!}
\Msg{*}
\Msg{************************************************************************}

\endbatchfile
%</install>
%<*ignore>
\fi
%</ignore>
%<*driver>
\NeedsTeXFormat{LaTeX2e}
\ProvidesFile{hologo.drv}%
  [2016/05/12 v1.11 A logo collection with bookmark support (HO)]%
\documentclass{ltxdoc}
\usepackage{holtxdoc}[2011/11/22]
\usepackage{hologo}[2016/05/12]
\usepackage{longtable}
\usepackage{array}
\usepackage{paralist}
%\usepackage[T1]{fontenc}
%\usepackage{lmodern}
\begin{document}
  \DocInput{hologo.dtx}%
\end{document}
%</driver>
% \fi
%
%
% \CharacterTable
%  {Upper-case    \A\B\C\D\E\F\G\H\I\J\K\L\M\N\O\P\Q\R\S\T\U\V\W\X\Y\Z
%   Lower-case    \a\b\c\d\e\f\g\h\i\j\k\l\m\n\o\p\q\r\s\t\u\v\w\x\y\z
%   Digits        \0\1\2\3\4\5\6\7\8\9
%   Exclamation   \!     Double quote  \"     Hash (number) \#
%   Dollar        \$     Percent       \%     Ampersand     \&
%   Acute accent  \'     Left paren    \(     Right paren   \)
%   Asterisk      \*     Plus          \+     Comma         \,
%   Minus         \-     Point         \.     Solidus       \/
%   Colon         \:     Semicolon     \;     Less than     \<
%   Equals        \=     Greater than  \>     Question mark \?
%   Commercial at \@     Left bracket  \[     Backslash     \\
%   Right bracket \]     Circumflex    \^     Underscore    \_
%   Grave accent  \`     Left brace    \{     Vertical bar  \|
%   Right brace   \}     Tilde         \~}
%
% \GetFileInfo{hologo.drv}
%
% \title{The \xpackage{hologo} package}
% \date{2016/05/12 v1.11}
% \author{Heiko Oberdiek\\\xemail{heiko.oberdiek at googlemail.com}}
%
% \maketitle
%
% \begin{abstract}
% This package starts a collection of logos with support for bookmarks
% strings.
% \end{abstract}
%
% \tableofcontents
%
% \section{Documentation}
%
% \subsection{Logo macros}
%
% \begin{declcs}{hologo} \M{name}
% \end{declcs}
% Macro \cs{hologo} sets the logo with name \meta{name}.
% The following table shows the supported names.
%
% \begingroup
%   \def\hologoEntry#1#2#3{^^A
%     #1&#2&\hologoLogoSetup{#1}{variant=#2}\hologo{#1}&#3\tabularnewline
%   }
%   \begin{longtable}{>{\ttfamily}l>{\ttfamily}lll}
%     \rmfamily\bfseries{name} & \rmfamily\bfseries variant
%     & \bfseries logo & \bfseries since\\
%     \hline
%     \endhead
%     \hologoList
%   \end{longtable}
% \endgroup
%
% \begin{declcs}{Hologo} \M{name}
% \end{declcs}
% Macro \cs{Hologo} starts the logo \meta{name} with an uppercase
% letter. As an exception small greek letters are not converted
% to uppercase. Examples, see \hologo{eTeX} and \hologo{ExTeX}.
%
% \subsection{Setup macros}
%
% The package does not support package options, but the following
% setup macros can be used to set options.
%
% \begin{declcs}{hologoSetup} \M{key value list}
% \end{declcs}
% Macro \cs{hologoSetup} sets global options.
%
% \begin{declcs}{hologoLogoSetup} \M{logo} \M{key value list}
% \end{declcs}
% Some options can also be used to configure a logo.
% These settings take precedence over global option settings.
%
% \subsection{Options}\label{sec:options}
%
% There are boolean and string options:
% \begin{description}
% \item[Boolean option:]
% It takes |true| or |false|
% as value. If the value is omitted, then |true| is used.
% \item[String option:]
% A value must be given as string. (But the string might be empty.)
% \end{description}
% The following options can be used both in \cs{hologoSetup}
% and \cs{hologoLogoSetup}:
% \begin{description}
% \def\entry#1{\item[\xoption{#1}:]}
% \entry{break}
%   enables or disables line breaks inside the logo. This setting is
%   refined by options \xoption{hyphenbreak}, \xoption{spacebreak}
%   or \xoption{discretionarybreak}.
%   Default is |false|.
% \entry{hyphenbreak}
%   enables or disables the line break right after the hyphen character.
% \entry{spacebreak}
%   enables or disables line breaks at space characters.
% \entry{discretionarybreak}
%   enables or disables line breaks at hyphenation points
%   (inserted by \cs{-}).
% \end{description}
% Macro \cs{hologoLogoSetup} also knows:
% \begin{description}
% \item[\xoption{variant}:]
%   This is a string option. It specifies a variant of a logo that
%   must exist. An empty string selects the package default variant.
% \end{description}
% Example:
% \begin{quote}
%   |\hologoSetup{break=false}|\\
%   |\hologoLogoSetup{plainTeX}{variant=hyphen,hyphenbreak}|\\
%   Then ``plain-\TeX'' contains one break point after the hyphen.
% \end{quote}
%
% \subsection{Driver options}
%
% Sometimes graphical operations are needed to construct some
% glyphs (e.g.\ \hologo{XeTeX}). If package \xpackage{graphics}
% or package \xpackage{pgf} are found, then the macros are taken
% from there. Otherwise the packge defines its own operations
% and therefore needs the driver information. Many drivers are
% detected automatically (\hologo{pdfTeX}/\hologo{LuaTeX}
% in PDF mode, \hologo{XeTeX}, \hologo{VTeX}). These have precedence
% over a driver option. The driver can be given as package option
% or using \cs{hologoDriverSetup}.
% The following list contains the recognized driver options:
% \begin{itemize}
% \item \xoption{pdftex}, \xoption{luatex}
% \item \xoption{dvipdfm}, \xoption{dvipdfmx}
% \item \xoption{dvips}, \xoption{dvipsone}, \xoption{xdvi}
% \item \xoption{xetex}
% \item \xoption{vtex}
% \end{itemize}
% The left driver of a line is the driver name that is used internally.
% The following names are aliases for drivers that use the
% same method. Therefore the entry in the \xext{log} file for
% the used driver prints the internally used driver name.
% \begin{description}
% \item[\xoption{driverfallback}:]
%   This option expects a driver that is used,
%   if the driver could not be detected automatically.
% \end{description}
%
% \begin{declcs}{hologoDriverSetup} \M{driver option}
% \end{declcs}
% The driver can also be configured after package loading
% using \cs{hologoDriverSetup}, also the way for \hologo{plainTeX}
% to setup the driver.
%
% \subsection{Font setup}
%
% Some logos require a special font, but should also be usable by
% \hologo{plainTeX}. Therefore the package provides some ways
% to influence the font settings. The options below
% take font settings as values. Both font commands
% such as \cs{sffamily} and macros that take one argument
% like \cs{textsf} can be used.
%
% \begin{declcs}{hologoFontSetup} \M{key value list}
% \end{declcs}
% Macro \cs{hologoFontSetup} sets the fonts for all logos.
% Supported keys:
% \begin{description}
% \def\entry#1{\item[\xoption{#1}:]}
% \entry{general}
%   This font is used for all logos. The default is empty.
%   That means no special font is used.
% \entry{bibsf}
%   This font is used for
%   {\hologoLogoSetup{BibTeX}{variant=sf}\hologo{BibTeX}}
%   with variant \xoption{sf}.
% \entry{rm}
%   This font is a serif font. It is used for \hologo{ExTeX}.
% \entry{sc}
%   This font specifies a small caps font. It is used for
%   {\hologoLogoSetup{BibTeX}{variant=sc}\hologo{BibTeX}}
%   with variant \xoption{sc}.
% \entry{sf}
%   This font specifies a sans serif font. The default
%   is \cs{sffamily}, then \cs{sf} is tried. Otherwise
%   a warning is given. It is used by \hologo{KOMAScript}.
% \entry{sy}
%   This is the font for math symbols (e.g. cmsy).
%   It is used by \hologo{AmS}, \hologo{NTS}, \hologo{ExTeX}.
% \entry{logo}
%   \hologo{METAFONT} and \hologo{METAPOST} are using that font.
%   In \hologo{LaTeX} \cs{logofamily} is used and
%   the definitions of package \xpackage{mflogo} are used
%   if the package is not loaded.
%   Otherwise the \cs{tenlogo} is used and defined
%   if it does not already exists.
% \end{description}
%
% \begin{declcs}{hologoLogoFontSetup} \M{logo} \M{key value list}
% \end{declcs}
% Fonts can also be set for a logo or logo component separately,
% see the following list.
% The keys are the same as for \cs{hologoFontSetup}.
%
% \begin{longtable}{>{\ttfamily}l>{\sffamily}ll}
%   \meta{logo} & keys & result\\
%   \hline
%   \endhead
%   BibTeX & bibsf & {\hologoLogoSetup{BibTeX}{variant=sf}\hologo{BibTeX}}\\[.5ex]
%   BibTeX & sc & {\hologoLogoSetup{BibTeX}{variant=sc}\hologo{BibTeX}}\\[.5ex]
%   ExTeX & rm & \hologo{ExTeX}\\
%   SliTeX & rm & \hologo{SliTeX}\\[.5ex]
%   AmS & sy & \hologo{AmS}\\
%   ExTeX & sy & \hologo{ExTeX}\\
%   NTS & sy & \hologo{NTS}\\[.5ex]
%   KOMAScript & sf & \hologo{KOMAScript}\\[.5ex]
%   METAFONT & logo & \hologo{METAFONT}\\
%   METAPOST & logo & \hologo{METAPOST}\\[.5ex]
%   SliTeX & sc \hologo{SliTeX}
% \end{longtable}
%
% \subsubsection{Font order}
%
% For all logos the font \xoption{general} is applied first.
% Example:
%\begin{quote}
%|\hologoFontSetup{general=\color{red}}|
%\end{quote}
% will print red logos.
% Then if the font uses a special font \xoption{sf}, for example,
% the font is applied that is setup by \cs{hologoLogoFontSetup}.
% If this font is not setup, then the common font setup
% by \cs{hologoFontSetup} is used. Otherwise a warning is given,
% that there is no font configured.
%
% \subsection{Additional user macros}
%
% Usually a variant of a logo is configured by using
% \cs{hologoLogoSetup}, because it is bad style to mix
% different variants of the same logo in the same text.
% There the following macros are a convenience for testing.
%
% \begin{declcs}{hologoVariant} \M{name} \M{variant}\\
%   \cs{HologoVariant} \M{name} \M{variant}
% \end{declcs}
% Logo \meta{name} is set using \meta{variant} that specifies
% explicitely which variant of the macro is used. If the argument
% is empty, then the default form of the logo is used
% (configurable by \cs{hologoLogoSetup}).
%
% \cs{HologoVariant} is used if the logo is set in a context
% that needs an uppercase first letter (beginning of a sentence, \dots).
%
% \begin{declcs}{hologoList}\\
%   \cs{hologoEntry} \M{logo} \M{variant} \M{since}
% \end{declcs}
% Macro \cs{hologoList} contains all logos that are provided
% by the package including variants. The list consists of calls
% of \cs{hologoEntry} with three arguments starting with the
% logo name \meta{logo} and its variant \meta{variant}. An empty
% variant means the current default. Argument \meta{since} specifies
% with version of the package \xpackage{hologo} is needed to get
% the logo. If the logo is fixed, then the date gets updated.
% Therefore the date \meta{since} is not exactly the date of
% the first introduction, but rather the date of the latest fix.
%
% Before \cs{hologoList} can be used, macro \cs{hologoEntry} needs
% a definition. The example file in section \ref{sec:example}
% shows applications of \cs{hologoList}.
%
% \subsection{Supported contexts}
%
% Macros \cs{hologo} and friends support special contexts:
% \begin{itemize}
% \item \hologo{LaTeX}'s protection mechanism.
% \item Bookmarks of package \xpackage{hyperref}.
% \item Package \xpackage{tex4ht}.
% \item The macros can be used inside \cs{csname} constructs,
%   if \cs{ifincsname} is available (\hologo{pdfTeX}, \hologo{XeTeX},
%   \hologo{LuaTeX}).
% \end{itemize}
%
% \subsection{Example}
% \label{sec:example}
%
% The following example prints the logos in different fonts.
%    \begin{macrocode}
%<*example>
%<<verbatim
\NeedsTeXFormat{LaTeX2e}
\documentclass[a4paper]{article}
\usepackage[
  hmargin=20mm,
  vmargin=20mm,
]{geometry}
\pagestyle{empty}
\usepackage{hologo}[2016/05/12]
\usepackage{longtable}
\usepackage{array}
\setlength{\extrarowheight}{2pt}
\usepackage[T1]{fontenc}
\usepackage{lmodern}
\usepackage{pdflscape}
\usepackage[
  pdfencoding=auto,
]{hyperref}
\hypersetup{
  pdfauthor={Heiko Oberdiek},
  pdftitle={Example for package `hologo'},
  pdfsubject={Logos with fonts lmr, lmss, qtm, qpl, qhv},
}
\usepackage{bookmark}

% Print the logo list on the console

\begingroup
  \typeout{}%
  \typeout{*** Begin of logo list ***}%
  \newcommand*{\hologoEntry}[3]{%
    \typeout{#1 \ifx\\#2\\\else(#2) \fi[#3]}%
  }%
  \hologoList
  \typeout{*** End of logo list ***}%
  \typeout{}%
\endgroup

\begin{document}
\begin{landscape}

  \section{Example file for package `hologo'}

  % Table for font names

  \begin{longtable}{>{\bfseries}ll}
    \textbf{font} & \textbf{Font name}\\
    \hline
    lmr & Latin Modern Roman\\
    lmss & Latin Modern Sans\\
    qtm & \TeX\ Gyre Termes\\
    qhv & \TeX\ Gyre Heros\\
    qpl & \TeX\ Gyre Pagella\\
  \end{longtable}

  % Logo list with logos in different fonts

  \begingroup
    \newcommand*{\SetVariant}[2]{%
      \ifx\\#2\\%
      \else
        \hologoLogoSetup{#1}{variant=#2}%
      \fi
    }%
    \newcommand*{\hologoEntry}[3]{%
      \SetVariant{#1}{#2}%
      \raisebox{1em}[0pt][0pt]{\hypertarget{#1@#2}{}}%
      \bookmark[%
        dest={#1@#2},%
      ]{%
        #1\ifx\\#2\\\else\space(#2)\fi: \Hologo{#1}, \hologo{#1} %
        [Unicode]%
      }%
      \hypersetup{unicode=false}%
      \bookmark[%
        dest={#1@#2},%
      ]{%
        #1\ifx\\#2\\\else\space(#2)\fi: \Hologo{#1}, \hologo{#1} %
        [PDFDocEncoding]%
      }%
      \texttt{#1}%
      &%
      \texttt{#2}%
      &%
      \Hologo{#1}%
      &%
      \SetVariant{#1}{#2}%
      \hologo{#1}%
      &%
      \SetVariant{#1}{#2}%
      \fontfamily{qtm}\selectfont
      \hologo{#1}%
      &%
      \SetVariant{#1}{#2}%
      \fontfamily{qpl}\selectfont
      \hologo{#1}%
      &%
      \SetVariant{#1}{#2}%
      \textsf{\hologo{#1}}%
      &%
      \SetVariant{#1}{#2}%
      \fontfamily{qhv}\selectfont
      \hologo{#1}%
      \tabularnewline
    }%
    \begin{longtable}{llllllll}%
      \textbf{\textit{logo}} & \textbf{\textit{variant}} &
      \texttt{\string\Hologo} &
      \textbf{lmr} & \textbf{qtm} & \textbf{qpl} &
      \textbf{lmss} & \textbf{qhv}
      \tabularnewline
      \hline
      \endhead
      \hologoList
    \end{longtable}%
  \endgroup

\end{landscape}
\end{document}
%verbatim
%</example>
%    \end{macrocode}
%
% \StopEventually{
% }
%
% \section{Implementation}
%    \begin{macrocode}
%<*package>
%    \end{macrocode}
%    Reload check, especially if the package is not used with \LaTeX.
%    \begin{macrocode}
\begingroup\catcode61\catcode48\catcode32=10\relax%
  \catcode13=5 % ^^M
  \endlinechar=13 %
  \catcode35=6 % #
  \catcode39=12 % '
  \catcode44=12 % ,
  \catcode45=12 % -
  \catcode46=12 % .
  \catcode58=12 % :
  \catcode64=11 % @
  \catcode123=1 % {
  \catcode125=2 % }
  \expandafter\let\expandafter\x\csname ver@hologo.sty\endcsname
  \ifx\x\relax % plain-TeX, first loading
  \else
    \def\empty{}%
    \ifx\x\empty % LaTeX, first loading,
      % variable is initialized, but \ProvidesPackage not yet seen
    \else
      \expandafter\ifx\csname PackageInfo\endcsname\relax
        \def\x#1#2{%
          \immediate\write-1{Package #1 Info: #2.}%
        }%
      \else
        \def\x#1#2{\PackageInfo{#1}{#2, stopped}}%
      \fi
      \x{hologo}{The package is already loaded}%
      \aftergroup\endinput
    \fi
  \fi
\endgroup%
%    \end{macrocode}
%    Package identification:
%    \begin{macrocode}
\begingroup\catcode61\catcode48\catcode32=10\relax%
  \catcode13=5 % ^^M
  \endlinechar=13 %
  \catcode35=6 % #
  \catcode39=12 % '
  \catcode40=12 % (
  \catcode41=12 % )
  \catcode44=12 % ,
  \catcode45=12 % -
  \catcode46=12 % .
  \catcode47=12 % /
  \catcode58=12 % :
  \catcode64=11 % @
  \catcode91=12 % [
  \catcode93=12 % ]
  \catcode123=1 % {
  \catcode125=2 % }
  \expandafter\ifx\csname ProvidesPackage\endcsname\relax
    \def\x#1#2#3[#4]{\endgroup
      \immediate\write-1{Package: #3 #4}%
      \xdef#1{#4}%
    }%
  \else
    \def\x#1#2[#3]{\endgroup
      #2[{#3}]%
      \ifx#1\@undefined
        \xdef#1{#3}%
      \fi
      \ifx#1\relax
        \xdef#1{#3}%
      \fi
    }%
  \fi
\expandafter\x\csname ver@hologo.sty\endcsname
\ProvidesPackage{hologo}%
  [2016/05/12 v1.11 A logo collection with bookmark support (HO)]%
%    \end{macrocode}
%
%    \begin{macrocode}
\begingroup\catcode61\catcode48\catcode32=10\relax%
  \catcode13=5 % ^^M
  \endlinechar=13 %
  \catcode123=1 % {
  \catcode125=2 % }
  \catcode64=11 % @
  \def\x{\endgroup
    \expandafter\edef\csname HOLOGO@AtEnd\endcsname{%
      \endlinechar=\the\endlinechar\relax
      \catcode13=\the\catcode13\relax
      \catcode32=\the\catcode32\relax
      \catcode35=\the\catcode35\relax
      \catcode61=\the\catcode61\relax
      \catcode64=\the\catcode64\relax
      \catcode123=\the\catcode123\relax
      \catcode125=\the\catcode125\relax
    }%
  }%
\x\catcode61\catcode48\catcode32=10\relax%
\catcode13=5 % ^^M
\endlinechar=13 %
\catcode35=6 % #
\catcode64=11 % @
\catcode123=1 % {
\catcode125=2 % }
\def\TMP@EnsureCode#1#2{%
  \edef\HOLOGO@AtEnd{%
    \HOLOGO@AtEnd
    \catcode#1=\the\catcode#1\relax
  }%
  \catcode#1=#2\relax
}
\TMP@EnsureCode{10}{12}% ^^J
\TMP@EnsureCode{33}{12}% !
\TMP@EnsureCode{34}{12}% "
\TMP@EnsureCode{36}{3}% $
\TMP@EnsureCode{38}{4}% &
\TMP@EnsureCode{39}{12}% '
\TMP@EnsureCode{40}{12}% (
\TMP@EnsureCode{41}{12}% )
\TMP@EnsureCode{42}{12}% *
\TMP@EnsureCode{43}{12}% +
\TMP@EnsureCode{44}{12}% ,
\TMP@EnsureCode{45}{12}% -
\TMP@EnsureCode{46}{12}% .
\TMP@EnsureCode{47}{12}% /
\TMP@EnsureCode{58}{12}% :
\TMP@EnsureCode{59}{12}% ;
\TMP@EnsureCode{60}{12}% <
\TMP@EnsureCode{62}{12}% >
\TMP@EnsureCode{63}{12}% ?
\TMP@EnsureCode{91}{12}% [
\TMP@EnsureCode{93}{12}% ]
\TMP@EnsureCode{94}{7}% ^ (superscript)
\TMP@EnsureCode{95}{8}% _ (subscript)
\TMP@EnsureCode{96}{12}% `
\TMP@EnsureCode{124}{12}% |
\edef\HOLOGO@AtEnd{%
  \HOLOGO@AtEnd
  \escapechar\the\escapechar\relax
  \noexpand\endinput
}
\escapechar=92 %
%    \end{macrocode}
%
% \subsection{Logo list}
%
%    \begin{macro}{\hologoList}
%    \begin{macrocode}
\def\hologoList{%
  \hologoEntry{(La)TeX}{}{2011/10/01}%
  \hologoEntry{AmSLaTeX}{}{2010/04/16}%
  \hologoEntry{AmSTeX}{}{2010/04/16}%
  \hologoEntry{biber}{}{2011/10/01}%
  \hologoEntry{BibTeX}{}{2011/10/01}%
  \hologoEntry{BibTeX}{sf}{2011/10/01}%
  \hologoEntry{BibTeX}{sc}{2011/10/01}%
  \hologoEntry{BibTeX8}{}{2011/11/22}%
  \hologoEntry{ConTeXt}{}{2011/03/25}%
  \hologoEntry{ConTeXt}{narrow}{2011/03/25}%
  \hologoEntry{ConTeXt}{simple}{2011/03/25}%
  \hologoEntry{emTeX}{}{2010/04/26}%
  \hologoEntry{eTeX}{}{2010/04/08}%
  \hologoEntry{ExTeX}{}{2011/10/01}%
  \hologoEntry{HanTheThanh}{}{2011/11/29}%
  \hologoEntry{iniTeX}{}{2011/10/01}%
  \hologoEntry{KOMAScript}{}{2011/10/01}%
  \hologoEntry{La}{}{2010/05/08}%
  \hologoEntry{LaTeX}{}{2010/04/08}%
  \hologoEntry{LaTeX2e}{}{2010/04/08}%
  \hologoEntry{LaTeX3}{}{2010/04/24}%
  \hologoEntry{LaTeXe}{}{2010/04/08}%
  \hologoEntry{LaTeXML}{}{2011/11/22}%
  \hologoEntry{LaTeXTeX}{}{2011/10/01}%
  \hologoEntry{LuaLaTeX}{}{2010/04/08}%
  \hologoEntry{LuaTeX}{}{2010/04/08}%
  \hologoEntry{LyX}{}{2011/10/01}%
  \hologoEntry{METAFONT}{}{2011/10/01}%
  \hologoEntry{MetaFun}{}{2011/10/01}%
  \hologoEntry{METAPOST}{}{2011/10/01}%
  \hologoEntry{MetaPost}{}{2011/10/01}%
  \hologoEntry{MiKTeX}{}{2011/10/01}%
  \hologoEntry{NTS}{}{2011/10/01}%
  \hologoEntry{OzMF}{}{2011/10/01}%
  \hologoEntry{OzMP}{}{2011/10/01}%
  \hologoEntry{OzTeX}{}{2011/10/01}%
  \hologoEntry{OzTtH}{}{2011/10/01}%
  \hologoEntry{PCTeX}{}{2011/10/01}%
  \hologoEntry{pdfTeX}{}{2011/10/01}%
  \hologoEntry{pdfLaTeX}{}{2011/10/01}%
  \hologoEntry{PiC}{}{2011/10/01}%
  \hologoEntry{PiCTeX}{}{2011/10/01}%
  \hologoEntry{plainTeX}{}{2010/04/08}%
  \hologoEntry{plainTeX}{space}{2010/04/16}%
  \hologoEntry{plainTeX}{hyphen}{2010/04/16}%
  \hologoEntry{plainTeX}{runtogether}{2010/04/16}%
  \hologoEntry{SageTeX}{}{2011/11/22}%
  \hologoEntry{SLiTeX}{}{2011/10/01}%
  \hologoEntry{SLiTeX}{lift}{2011/10/01}%
  \hologoEntry{SLiTeX}{narrow}{2011/10/01}%
  \hologoEntry{SLiTeX}{simple}{2011/10/01}%
  \hologoEntry{SliTeX}{}{2011/10/01}%
  \hologoEntry{SliTeX}{narrow}{2011/10/01}%
  \hologoEntry{SliTeX}{simple}{2011/10/01}%
  \hologoEntry{SliTeX}{lift}{2011/10/01}%
  \hologoEntry{teTeX}{}{2011/10/01}%
  \hologoEntry{TeX}{}{2010/04/08}%
  \hologoEntry{TeX4ht}{}{2011/11/22}%
  \hologoEntry{TTH}{}{2011/11/22}%
  \hologoEntry{virTeX}{}{2011/10/01}%
  \hologoEntry{VTeX}{}{2010/04/24}%
  \hologoEntry{Xe}{}{2010/04/08}%
  \hologoEntry{XeLaTeX}{}{2010/04/08}%
  \hologoEntry{XeTeX}{}{2010/04/08}%
}
%    \end{macrocode}
%    \end{macro}
%
% \subsection{Load resources}
%
%    \begin{macrocode}
\begingroup\expandafter\expandafter\expandafter\endgroup
\expandafter\ifx\csname RequirePackage\endcsname\relax
  \def\TMP@RequirePackage#1[#2]{%
    \begingroup\expandafter\expandafter\expandafter\endgroup
    \expandafter\ifx\csname ver@#1.sty\endcsname\relax
      \input #1.sty\relax
    \fi
  }%
  \TMP@RequirePackage{ltxcmds}[2011/02/04]%
  \TMP@RequirePackage{infwarerr}[2010/04/08]%
  \TMP@RequirePackage{kvsetkeys}[2010/03/01]%
  \TMP@RequirePackage{kvdefinekeys}[2010/03/01]%
  \TMP@RequirePackage{pdftexcmds}[2010/04/01]%
  \TMP@RequirePackage{ifpdf}[2010/01/28]%
  \TMP@RequirePackage{ifluatex}[2010/03/01]%
  \ltx@IfUndefined{newif}{%
    \expandafter\let\csname newif\endcsname\ltx@newif
  }{}%
  \TMP@RequirePackage{ifxetex}[2009/01/23]%
  \TMP@RequirePackage{ifvtex}[2010/03/01]%
\else
  \RequirePackage{ltxcmds}[2011/02/04]%
  \RequirePackage{infwarerr}[2010/04/08]%
  \RequirePackage{kvsetkeys}[2010/03/01]%
  \RequirePackage{kvdefinekeys}[2010/03/01]%
  \RequirePackage{pdftexcmds}[2010/04/01]%
  \RequirePackage{ifpdf}[2010/01/28]%
  \RequirePackage{ifluatex}[2010/03/01]%
  \RequirePackage{ifxetex}[2009/01/23]%
  \RequirePackage{ifvtex}[2010/03/01]%
\fi
%    \end{macrocode}
%
%    \begin{macro}{\HOLOGO@IfDefined}
%    \begin{macrocode}
\def\HOLOGO@IfExists#1{%
  \ifx\@undefined#1%
    \expandafter\ltx@secondoftwo
  \else
    \ifx\relax#1%
      \expandafter\ltx@secondoftwo
    \else
      \expandafter\expandafter\expandafter\ltx@firstoftwo
    \fi
  \fi
}
%    \end{macrocode}
%    \end{macro}
%
% \subsection{Setup macros}
%
%    \begin{macro}{\hologoSetup}
%    \begin{macrocode}
\def\hologoSetup{%
  \let\HOLOGO@name\relax
  \HOLOGO@Setup
}
%    \end{macrocode}
%    \end{macro}
%
%    \begin{macro}{\hologoLogoSetup}
%    \begin{macrocode}
\def\hologoLogoSetup#1{%
  \edef\HOLOGO@name{#1}%
  \ltx@IfUndefined{HoLogo@\HOLOGO@name}{%
    \@PackageError{hologo}{%
      Unknown logo `\HOLOGO@name'%
    }\@ehc
    \ltx@gobble
  }{%
    \HOLOGO@Setup
  }%
}
%    \end{macrocode}
%    \end{macro}
%
%    \begin{macro}{\HOLOGO@Setup}
%    \begin{macrocode}
\def\HOLOGO@Setup{%
  \kvsetkeys{HoLogo}%
}
%    \end{macrocode}
%    \end{macro}
%
% \subsection{Options}
%
%    \begin{macro}{\HOLOGO@DeclareBoolOption}
%    \begin{macrocode}
\def\HOLOGO@DeclareBoolOption#1{%
  \expandafter\chardef\csname HOLOGOOPT@#1\endcsname\ltx@zero
  \kv@define@key{HoLogo}{#1}[true]{%
    \def\HOLOGO@temp{##1}%
    \ifx\HOLOGO@temp\HOLOGO@true
      \ifx\HOLOGO@name\relax
        \expandafter\chardef\csname HOLOGOOPT@#1\endcsname=\ltx@one
      \else
        \expandafter\chardef\csname
        HoLogoOpt@#1@\HOLOGO@name\endcsname\ltx@one
      \fi
      \HOLOGO@SetBreakAll{#1}%
    \else
      \ifx\HOLOGO@temp\HOLOGO@false
        \ifx\HOLOGO@name\relax
          \expandafter\chardef\csname HOLOGOOPT@#1\endcsname=\ltx@zero
        \else
          \expandafter\chardef\csname
          HoLogoOpt@#1@\HOLOGO@name\endcsname=\ltx@zero
        \fi
        \HOLOGO@SetBreakAll{#1}%
      \else
        \@PackageError{hologo}{%
          Unknown value `##1' for boolean option `#1'.\MessageBreak
          Known values are `true' and `false'%
        }\@ehc
      \fi
    \fi
  }%
}
%    \end{macrocode}
%    \end{macro}
%
%    \begin{macro}{\HOLOGO@SetBreakAll}
%    \begin{macrocode}
\def\HOLOGO@SetBreakAll#1{%
  \def\HOLOGO@temp{#1}%
  \ifx\HOLOGO@temp\HOLOGO@break
    \ifx\HOLOGO@name\relax
      \chardef\HOLOGOOPT@hyphenbreak=\HOLOGOOPT@break
      \chardef\HOLOGOOPT@spacebreak=\HOLOGOOPT@break
      \chardef\HOLOGOOPT@discretionarybreak=\HOLOGOOPT@break
    \else
      \expandafter\chardef
         \csname HoLogoOpt@hyphenbreak@\HOLOGO@name\endcsname=%
         \csname HoLogoOpt@break@\HOLOGO@name\endcsname
      \expandafter\chardef
         \csname HoLogoOpt@spacebreak@\HOLOGO@name\endcsname=%
         \csname HoLogoOpt@break@\HOLOGO@name\endcsname
      \expandafter\chardef
         \csname HoLogoOpt@discretionarybreak@\HOLOGO@name
             \endcsname=%
         \csname HoLogoOpt@break@\HOLOGO@name\endcsname
    \fi
  \fi
}
%    \end{macrocode}
%    \end{macro}
%
%    \begin{macro}{\HOLOGO@true}
%    \begin{macrocode}
\def\HOLOGO@true{true}
%    \end{macrocode}
%    \end{macro}
%    \begin{macro}{\HOLOGO@false}
%    \begin{macrocode}
\def\HOLOGO@false{false}
%    \end{macrocode}
%    \end{macro}
%    \begin{macro}{\HOLOGO@break}
%    \begin{macrocode}
\def\HOLOGO@break{break}
%    \end{macrocode}
%    \end{macro}
%
%    \begin{macrocode}
\HOLOGO@DeclareBoolOption{break}
\HOLOGO@DeclareBoolOption{hyphenbreak}
\HOLOGO@DeclareBoolOption{spacebreak}
\HOLOGO@DeclareBoolOption{discretionarybreak}
%    \end{macrocode}
%
%    \begin{macrocode}
\kv@define@key{HoLogo}{variant}{%
  \ifx\HOLOGO@name\relax
    \@PackageError{hologo}{%
      Option `variant' is not available in \string\hologoSetup,%
      \MessageBreak
      Use \string\hologoLogoSetup\space instead%
    }\@ehc
  \else
    \edef\HOLOGO@temp{#1}%
    \ifx\HOLOGO@temp\ltx@empty
      \expandafter
      \let\csname HoLogoOpt@variant@\HOLOGO@name\endcsname\@undefined
    \else
      \ltx@IfUndefined{HoLogo@\HOLOGO@name @\HOLOGO@temp}{%
        \@PackageError{hologo}{%
          Unknown variant `\HOLOGO@temp' of logo `\HOLOGO@name'%
        }\@ehc
      }{%
        \expandafter
        \let\csname HoLogoOpt@variant@\HOLOGO@name\endcsname
            \HOLOGO@temp
      }%
    \fi
  \fi
}
%    \end{macrocode}
%
%    \begin{macro}{\HOLOGO@Variant}
%    \begin{macrocode}
\def\HOLOGO@Variant#1{%
  #1%
  \ltx@ifundefined{HoLogoOpt@variant@#1}{%
  }{%
    @\csname HoLogoOpt@variant@#1\endcsname
  }%
}
%    \end{macrocode}
%    \end{macro}
%
% \subsection{Break/no-break support}
%
%    \begin{macro}{\HOLOGO@space}
%    \begin{macrocode}
\def\HOLOGO@space{%
  \ltx@ifundefined{HoLogoOpt@spacebreak@\HOLOGO@name}{%
    \ltx@ifundefined{HoLogoOpt@break@\HOLOGO@name}{%
      \chardef\HOLOGO@temp=\HOLOGOOPT@spacebreak
    }{%
      \chardef\HOLOGO@temp=%
        \csname HoLogoOpt@break@\HOLOGO@name\endcsname
    }%
  }{%
    \chardef\HOLOGO@temp=%
      \csname HoLogoOpt@spacebreak@\HOLOGO@name\endcsname
  }%
  \ifcase\HOLOGO@temp
    \penalty10000 %
  \fi
  \ltx@space
}
%    \end{macrocode}
%    \end{macro}
%
%    \begin{macro}{\HOLOGO@hyphen}
%    \begin{macrocode}
\def\HOLOGO@hyphen{%
  \ltx@ifundefined{HoLogoOpt@hyphenbreak@\HOLOGO@name}{%
    \ltx@ifundefined{HoLogoOpt@break@\HOLOGO@name}{%
      \chardef\HOLOGO@temp=\HOLOGOOPT@hyphenbreak
    }{%
      \chardef\HOLOGO@temp=%
        \csname HoLogoOpt@break@\HOLOGO@name\endcsname
    }%
  }{%
    \chardef\HOLOGO@temp=%
      \csname HoLogoOpt@hyphenbreak@\HOLOGO@name\endcsname
  }%
  \ifcase\HOLOGO@temp
    \ltx@mbox{-}%
  \else
    -%
  \fi
}
%    \end{macrocode}
%    \end{macro}
%
%    \begin{macro}{\HOLOGO@discretionary}
%    \begin{macrocode}
\def\HOLOGO@discretionary{%
  \ltx@ifundefined{HoLogoOpt@discretionarybreak@\HOLOGO@name}{%
    \ltx@ifundefined{HoLogoOpt@break@\HOLOGO@name}{%
      \chardef\HOLOGO@temp=\HOLOGOOPT@discretionarybreak
    }{%
      \chardef\HOLOGO@temp=%
        \csname HoLogoOpt@break@\HOLOGO@name\endcsname
    }%
  }{%
    \chardef\HOLOGO@temp=%
      \csname HoLogoOpt@discretionarybreak@\HOLOGO@name\endcsname
  }%
  \ifcase\HOLOGO@temp
  \else
    \-%
  \fi
}
%    \end{macrocode}
%    \end{macro}
%
%    \begin{macro}{\HOLOGO@mbox}
%    \begin{macrocode}
\def\HOLOGO@mbox#1{%
  \ltx@ifundefined{HoLogoOpt@break@\HOLOGO@name}{%
    \chardef\HOLOGO@temp=\HOLOGOOPT@hyphenbreak
  }{%
    \chardef\HOLOGO@temp=%
      \csname HoLogoOpt@break@\HOLOGO@name\endcsname
  }%
  \ifcase\HOLOGO@temp
    \ltx@mbox{#1}%
  \else
    #1%
  \fi
}
%    \end{macrocode}
%    \end{macro}
%
% \subsection{Font support}
%
%    \begin{macro}{\HoLogoFont@font}
%    \begin{tabular}{@{}ll@{}}
%    |#1|:& logo name\\
%    |#2|:& font short name\\
%    |#3|:& text
%    \end{tabular}
%    \begin{macrocode}
\def\HoLogoFont@font#1#2#3{%
  \begingroup
    \ltx@IfUndefined{HoLogoFont@logo@#1.#2}{%
      \ltx@IfUndefined{HoLogoFont@font@#2}{%
        \@PackageWarning{hologo}{%
          Missing font `#2' for logo `#1'%
        }%
        #3%
      }{%
        \csname HoLogoFont@font@#2\endcsname{#3}%
      }%
    }{%
      \csname HoLogoFont@logo@#1.#2\endcsname{#3}%
    }%
  \endgroup
}
%    \end{macrocode}
%    \end{macro}
%
%    \begin{macro}{\HoLogoFont@Def}
%    \begin{macrocode}
\def\HoLogoFont@Def#1{%
  \expandafter\def\csname HoLogoFont@font@#1\endcsname
}
%    \end{macrocode}
%    \end{macro}
%    \begin{macro}{\HoLogoFont@LogoDef}
%    \begin{macrocode}
\def\HoLogoFont@LogoDef#1#2{%
  \expandafter\def\csname HoLogoFont@logo@#1.#2\endcsname
}
%    \end{macrocode}
%    \end{macro}
%
% \subsubsection{Font defaults}
%
%    \begin{macro}{\HoLogoFont@font@general}
%    \begin{macrocode}
\HoLogoFont@Def{general}{}%
%    \end{macrocode}
%    \end{macro}
%
%    \begin{macro}{\HoLogoFont@font@rm}
%    \begin{macrocode}
\ltx@IfUndefined{rmfamily}{%
  \ltx@IfUndefined{rm}{%
  }{%
    \HoLogoFont@Def{rm}{\rm}%
  }%
}{%
  \HoLogoFont@Def{rm}{\rmfamily}%
}
%    \end{macrocode}
%    \end{macro}
%
%    \begin{macro}{\HoLogoFont@font@sf}
%    \begin{macrocode}
\ltx@IfUndefined{sffamily}{%
  \ltx@IfUndefined{sf}{%
  }{%
    \HoLogoFont@Def{sf}{\sf}%
  }%
}{%
  \HoLogoFont@Def{sf}{\sffamily}%
}
%    \end{macrocode}
%    \end{macro}
%
%    \begin{macro}{\HoLogoFont@font@bibsf}
%    In case of \hologo{plainTeX} the original small caps
%    variant is used as default. In \hologo{LaTeX}
%    the definition of package \xpackage{dtklogos} \cite{dtklogos}
%    is used.
%\begin{quote}
%\begin{verbatim}
%\DeclareRobustCommand{\BibTeX}{%
%  B%
%  \kern-.05em%
%  \hbox{%
%    $\m@th$% %% force math size calculations
%    \csname S@\f@size\endcsname
%    \fontsize\sf@size\z@
%    \math@fontsfalse
%    \selectfont
%    I%
%    \kern-.025em%
%    B
%  }%
%  \kern-.08em%
%  \-%
%  \TeX
%}
%\end{verbatim}
%\end{quote}
%    \begin{macrocode}
\ltx@IfUndefined{selectfont}{%
  \ltx@IfUndefined{tensc}{%
    \font\tensc=cmcsc10\relax
  }{}%
  \HoLogoFont@Def{bibsf}{\tensc}%
}{%
  \HoLogoFont@Def{bibsf}{%
    $\mathsurround=0pt$%
    \csname S@\f@size\endcsname
    \fontsize\sf@size{0pt}%
    \math@fontsfalse
    \selectfont
  }%
}
%    \end{macrocode}
%    \end{macro}
%
%    \begin{macro}{\HoLogoFont@font@sc}
%    \begin{macrocode}
\ltx@IfUndefined{scshape}{%
  \ltx@IfUndefined{tensc}{%
    \font\tensc=cmcsc10\relax
  }{}%
  \HoLogoFont@Def{sc}{\tensc}%
}{%
  \HoLogoFont@Def{sc}{\scshape}%
}
%    \end{macrocode}
%    \end{macro}
%
%    \begin{macro}{\HoLogoFont@font@sy}
%    \begin{macrocode}
\ltx@IfUndefined{usefont}{%
  \ltx@IfUndefined{tensy}{%
  }{%
    \HoLogoFont@Def{sy}{\tensy}%
  }%
}{%
  \HoLogoFont@Def{sy}{%
    \usefont{OMS}{cmsy}{m}{n}%
  }%
}
%    \end{macrocode}
%    \end{macro}
%
%    \begin{macro}{\HoLogoFont@font@logo}
%    \begin{macrocode}
\begingroup
  \def\x{LaTeX2e}%
\expandafter\endgroup
\ifx\fmtname\x
  \ltx@IfUndefined{logofamily}{%
    \DeclareRobustCommand\logofamily{%
      \not@math@alphabet\logofamily\relax
      \fontencoding{U}%
      \fontfamily{logo}%
      \selectfont
    }%
  }{}%
  \ltx@IfUndefined{logofamily}{%
  }{%
    \HoLogoFont@Def{logo}{\logofamily}%
  }%
\else
  \ltx@IfUndefined{tenlogo}{%
    \font\tenlogo=logo10\relax
  }{}%
  \HoLogoFont@Def{logo}{\tenlogo}%
\fi
%    \end{macrocode}
%    \end{macro}
%
% \subsubsection{Font setup}
%
%    \begin{macro}{\hologoFontSetup}
%    \begin{macrocode}
\def\hologoFontSetup{%
  \let\HOLOGO@name\relax
  \HOLOGO@FontSetup
}
%    \end{macrocode}
%    \end{macro}
%
%    \begin{macro}{\hologoLogoFontSetup}
%    \begin{macrocode}
\def\hologoLogoFontSetup#1{%
  \edef\HOLOGO@name{#1}%
  \ltx@IfUndefined{HoLogo@\HOLOGO@name}{%
    \@PackageError{hologo}{%
      Unknown logo `\HOLOGO@name'%
    }\@ehc
    \ltx@gobble
  }{%
    \HOLOGO@FontSetup
  }%
}
%    \end{macrocode}
%    \end{macro}
%
%    \begin{macro}{\HOLOGO@FontSetup}
%    \begin{macrocode}
\def\HOLOGO@FontSetup{%
  \kvsetkeys{HoLogoFont}%
}
%    \end{macrocode}
%    \end{macro}
%
%    \begin{macrocode}
\def\HOLOGO@temp#1{%
  \kv@define@key{HoLogoFont}{#1}{%
    \ifx\HOLOGO@name\relax
      \HoLogoFont@Def{#1}{##1}%
    \else
      \HoLogoFont@LogoDef\HOLOGO@name{#1}{##1}%
    \fi
  }%
}
\HOLOGO@temp{general}
\HOLOGO@temp{sf}
%    \end{macrocode}
%
% \subsection{Generic logo commands}
%
%    \begin{macrocode}
\HOLOGO@IfExists\hologo{%
  \@PackageError{hologo}{%
    \string\hologo\ltx@space is already defined.\MessageBreak
    Package loading is aborted%
  }\@ehc
  \HOLOGO@AtEnd
}%
\HOLOGO@IfExists\hologoRobust{%
  \@PackageError{hologo}{%
    \string\hologoRobust\ltx@space is already defined.\MessageBreak
    Package loading is aborted%
  }\@ehc
  \HOLOGO@AtEnd
}%
%    \end{macrocode}
%
% \subsubsection{\cs{hologo} and friends}
%
%    \begin{macrocode}
\ifluatex
  \expandafter\ltx@firstofone
\else
  \expandafter\ltx@gobble
\fi
{%
  \ltx@IfUndefined{ifincsname}{%
    \ifnum\luatexversion<36 %
      \expandafter\ltx@gobble
    \else
      \expandafter\ltx@firstofone
    \fi
    {%
      \begingroup
        \ifcase0%
            \directlua{%
              if tex.enableprimitives then %
                tex.enableprimitives('HOLOGO@', {'ifincsname'})%
              else %
                tex.print('1')%
              end%
            }%
            \ifx\HOLOGO@ifincsname\@undefined 1\fi%
            \relax
          \expandafter\ltx@firstofone
        \else
          \endgroup
          \expandafter\ltx@gobble
        \fi
        {%
          \global\let\ifincsname\HOLOGO@ifincsname
        }%
      \HOLOGO@temp
    }%
  }{}%
}
%    \end{macrocode}
%    \begin{macrocode}
\ltx@IfUndefined{ifincsname}{%
  \catcode`$=14 %
}{%
  \catcode`$=9 %
}
%    \end{macrocode}
%
%    \begin{macro}{\hologo}
%    \begin{macrocode}
\def\hologo#1{%
$ \ifincsname
$   \ltx@ifundefined{HoLogoCs@\HOLOGO@Variant{#1}}{%
$     #1%
$   }{%
$     \csname HoLogoCs@\HOLOGO@Variant{#1}\endcsname\ltx@firstoftwo
$   }%
$ \else
    \HOLOGO@IfExists\texorpdfstring\texorpdfstring\ltx@firstoftwo
    {%
      \hologoRobust{#1}%
    }{%
      \ltx@ifundefined{HoLogoBkm@\HOLOGO@Variant{#1}}{%
        \ltx@ifundefined{HoLogo@#1}{?#1?}{#1}%
      }{%
        \csname HoLogoBkm@\HOLOGO@Variant{#1}\endcsname
        \ltx@firstoftwo
      }%
    }%
$ \fi
}
%    \end{macrocode}
%    \end{macro}
%    \begin{macro}{\Hologo}
%    \begin{macrocode}
\def\Hologo#1{%
$ \ifincsname
$   \ltx@ifundefined{HoLogoCs@\HOLOGO@Variant{#1}}{%
$     #1%
$   }{%
$     \csname HoLogoCs@\HOLOGO@Variant{#1}\endcsname\ltx@secondoftwo
$   }%
$ \else
    \HOLOGO@IfExists\texorpdfstring\texorpdfstring\ltx@firstoftwo
    {%
      \HologoRobust{#1}%
    }{%
      \ltx@ifundefined{HoLogoBkm@\HOLOGO@Variant{#1}}{%
        \ltx@ifundefined{HoLogo@#1}{?#1?}{#1}%
      }{%
        \csname HoLogoBkm@\HOLOGO@Variant{#1}\endcsname
        \ltx@secondoftwo
      }%
    }%
$ \fi
}
%    \end{macrocode}
%    \end{macro}
%
%    \begin{macro}{\hologoVariant}
%    \begin{macrocode}
\def\hologoVariant#1#2{%
  \ifx\relax#2\relax
    \hologo{#1}%
  \else
$   \ifincsname
$     \ltx@ifundefined{HoLogoCs@#1@#2}{%
$       #1%
$     }{%
$       \csname HoLogoCs@#1@#2\endcsname\ltx@firstoftwo
$     }%
$   \else
      \HOLOGO@IfExists\texorpdfstring\texorpdfstring\ltx@firstoftwo
      {%
        \hologoVariantRobust{#1}{#2}%
      }{%
        \ltx@ifundefined{HoLogoBkm@#1@#2}{%
          \ltx@ifundefined{HoLogo@#1}{?#1?}{#1}%
        }{%
          \csname HoLogoBkm@#1@#2\endcsname
          \ltx@firstoftwo
        }%
      }%
$   \fi
  \fi
}
%    \end{macrocode}
%    \end{macro}
%    \begin{macro}{\HologoVariant}
%    \begin{macrocode}
\def\HologoVariant#1#2{%
  \ifx\relax#2\relax
    \Hologo{#1}%
  \else
$   \ifincsname
$     \ltx@ifundefined{HoLogoCs@#1@#2}{%
$       #1%
$     }{%
$       \csname HoLogoCs@#1@#2\endcsname\ltx@secondoftwo
$     }%
$   \else
      \HOLOGO@IfExists\texorpdfstring\texorpdfstring\ltx@firstoftwo
      {%
        \HologoVariantRobust{#1}{#2}%
      }{%
        \ltx@ifundefined{HoLogoBkm@#1@#2}{%
          \ltx@ifundefined{HoLogo@#1}{?#1?}{#1}%
        }{%
          \csname HoLogoBkm@#1@#2\endcsname
          \ltx@secondoftwo
        }%
      }%
$   \fi
  \fi
}
%    \end{macrocode}
%    \end{macro}
%
%    \begin{macrocode}
\catcode`\$=3 %
%    \end{macrocode}
%
% \subsubsection{\cs{hologoRobust} and friends}
%
%    \begin{macro}{\hologoRobust}
%    \begin{macrocode}
\ltx@IfUndefined{protected}{%
  \ltx@IfUndefined{DeclareRobustCommand}{%
    \def\hologoRobust#1%
  }{%
    \DeclareRobustCommand*\hologoRobust[1]%
  }%
}{%
  \protected\def\hologoRobust#1%
}%
{%
  \edef\HOLOGO@name{#1}%
  \ltx@IfUndefined{HoLogo@\HOLOGO@Variant\HOLOGO@name}{%
    \@PackageError{hologo}{%
      Unknown logo `\HOLOGO@name'%
    }\@ehc
    ?\HOLOGO@name?%
  }{%
    \ltx@IfUndefined{ver@tex4ht.sty}{%
      \HoLogoFont@font\HOLOGO@name{general}{%
        \csname HoLogo@\HOLOGO@Variant\HOLOGO@name\endcsname
        \ltx@firstoftwo
      }%
    }{%
      \ltx@IfUndefined{HoLogoHtml@\HOLOGO@Variant\HOLOGO@name}{%
        \HOLOGO@name
      }{%
        \csname HoLogoHtml@\HOLOGO@Variant\HOLOGO@name\endcsname
        \ltx@firstoftwo
      }%
    }%
  }%
}
%    \end{macrocode}
%    \end{macro}
%    \begin{macro}{\HologoRobust}
%    \begin{macrocode}
\ltx@IfUndefined{protected}{%
  \ltx@IfUndefined{DeclareRobustCommand}{%
    \def\HologoRobust#1%
  }{%
    \DeclareRobustCommand*\HologoRobust[1]%
  }%
}{%
  \protected\def\HologoRobust#1%
}%
{%
  \edef\HOLOGO@name{#1}%
  \ltx@IfUndefined{HoLogo@\HOLOGO@Variant\HOLOGO@name}{%
    \@PackageError{hologo}{%
      Unknown logo `\HOLOGO@name'%
    }\@ehc
    ?\HOLOGO@name?%
  }{%
    \ltx@IfUndefined{ver@tex4ht.sty}{%
      \HoLogoFont@font\HOLOGO@name{general}{%
        \csname HoLogo@\HOLOGO@Variant\HOLOGO@name\endcsname
        \ltx@secondoftwo
      }%
    }{%
      \ltx@IfUndefined{HoLogoHtml@\HOLOGO@Variant\HOLOGO@name}{%
        \expandafter\HOLOGO@Uppercase\HOLOGO@name
      }{%
        \csname HoLogoHtml@\HOLOGO@Variant\HOLOGO@name\endcsname
        \ltx@secondoftwo
      }%
    }%
  }%
}
%    \end{macrocode}
%    \end{macro}
%    \begin{macro}{\hologoVariantRobust}
%    \begin{macrocode}
\ltx@IfUndefined{protected}{%
  \ltx@IfUndefined{DeclareRobustCommand}{%
    \def\hologoVariantRobust#1#2%
  }{%
    \DeclareRobustCommand*\hologoVariantRobust[2]%
  }%
}{%
  \protected\def\hologoVariantRobust#1#2%
}%
{%
  \begingroup
    \hologoLogoSetup{#1}{variant={#2}}%
    \hologoRobust{#1}%
  \endgroup
}
%    \end{macrocode}
%    \end{macro}
%    \begin{macro}{\HologoVariantRobust}
%    \begin{macrocode}
\ltx@IfUndefined{protected}{%
  \ltx@IfUndefined{DeclareRobustCommand}{%
    \def\HologoVariantRobust#1#2%
  }{%
    \DeclareRobustCommand*\HologoVariantRobust[2]%
  }%
}{%
  \protected\def\HologoVariantRobust#1#2%
}%
{%
  \begingroup
    \hologoLogoSetup{#1}{variant={#2}}%
    \HologoRobust{#1}%
  \endgroup
}
%    \end{macrocode}
%    \end{macro}
%
%    \begin{macro}{\hologorobust}
%    Macro \cs{hologorobust} is only defined for compatibility.
%    Its use is deprecated.
%    \begin{macrocode}
\def\hologorobust{\hologoRobust}
%    \end{macrocode}
%    \end{macro}
%
% \subsection{Helpers}
%
%    \begin{macro}{\HOLOGO@Uppercase}
%    Macro \cs{HOLOGO@Uppercase} is restricted to \cs{uppercase},
%    because \hologo{plainTeX} or \hologo{iniTeX} do not provide
%    \cs{MakeUppercase}.
%    \begin{macrocode}
\def\HOLOGO@Uppercase#1{\uppercase{#1}}
%    \end{macrocode}
%    \end{macro}
%
%    \begin{macro}{\HOLOGO@PdfdocUnicode}
%    \begin{macrocode}
\def\HOLOGO@PdfdocUnicode{%
  \ifx\ifHy@unicode\iftrue
    \expandafter\ltx@secondoftwo
  \else
    \expandafter\ltx@firstoftwo
  \fi
}
%    \end{macrocode}
%    \end{macro}
%
%    \begin{macro}{\HOLOGO@Math}
%    \begin{macrocode}
\def\HOLOGO@MathSetup{%
  \mathsurround0pt\relax
  \HOLOGO@IfExists\f@series{%
    \if b\expandafter\ltx@car\f@series x\@nil
      \csname boldmath\endcsname
   \fi
  }{}%
}
%    \end{macrocode}
%    \end{macro}
%
%    \begin{macro}{\HOLOGO@TempDimen}
%    \begin{macrocode}
\dimendef\HOLOGO@TempDimen=\ltx@zero
%    \end{macrocode}
%    \end{macro}
%    \begin{macro}{\HOLOGO@NegativeKerning}
%    \begin{macrocode}
\def\HOLOGO@NegativeKerning#1{%
  \begingroup
    \HOLOGO@TempDimen=0pt\relax
    \comma@parse@normalized{#1}{%
      \ifdim\HOLOGO@TempDimen=0pt %
        \expandafter\HOLOGO@@NegativeKerning\comma@entry
      \fi
      \ltx@gobble
    }%
    \ifdim\HOLOGO@TempDimen<0pt %
      \kern\HOLOGO@TempDimen
    \fi
  \endgroup
}
%    \end{macrocode}
%    \end{macro}
%    \begin{macro}{\HOLOGO@@NegativeKerning}
%    \begin{macrocode}
\def\HOLOGO@@NegativeKerning#1#2{%
  \setbox\ltx@zero\hbox{#1#2}%
  \HOLOGO@TempDimen=\wd\ltx@zero
  \setbox\ltx@zero\hbox{#1\kern0pt#2}%
  \advance\HOLOGO@TempDimen by -\wd\ltx@zero
}
%    \end{macrocode}
%    \end{macro}
%
%    \begin{macro}{\HOLOGO@SpaceFactor}
%    \begin{macrocode}
\def\HOLOGO@SpaceFactor{%
  \spacefactor1000 %
}
%    \end{macrocode}
%    \end{macro}
%
%    \begin{macro}{\HOLOGO@Span}
%    \begin{macrocode}
\def\HOLOGO@Span#1#2{%
  \HCode{<span class="HoLogo-#1">}%
  #2%
  \HCode{</span>}%
}
%    \end{macrocode}
%    \end{macro}
%
% \subsubsection{Text subscript}
%
%    \begin{macro}{\HOLOGO@SubScript}%
%    \begin{macrocode}
\def\HOLOGO@SubScript#1{%
  \ltx@IfUndefined{textsubscript}{%
    \ltx@IfUndefined{text}{%
      \ltx@mbox{%
        \mathsurround=0pt\relax
        $%
          _{%
            \ltx@IfUndefined{sf@size}{%
              \mathrm{#1}%
            }{%
              \mbox{%
                \fontsize\sf@size{0pt}\selectfont
                #1%
              }%
            }%
          }%
        $%
      }%
    }{%
      \ltx@mbox{%
        \mathsurround=0pt\relax
        $_{\text{#1}}$%
      }%
    }%
  }{%
    \textsubscript{#1}%
  }%
}
%    \end{macrocode}
%    \end{macro}
%
% \subsection{\hologo{TeX} and friends}
%
% \subsubsection{\hologo{TeX}}
%
%    \begin{macro}{\HoLogo@TeX}
%    Source: \hologo{LaTeX} kernel.
%    \begin{macrocode}
\def\HoLogo@TeX#1{%
  T\kern-.1667em\lower.5ex\hbox{E}\kern-.125emX\HOLOGO@SpaceFactor
}
%    \end{macrocode}
%    \end{macro}
%    \begin{macro}{\HoLogoHtml@TeX}
%    \begin{macrocode}
\def\HoLogoHtml@TeX#1{%
  \HoLogoCss@TeX
  \HOLOGO@Span{TeX}{%
    T%
    \HOLOGO@Span{e}{%
      E%
    }%
    X%
  }%
}
%    \end{macrocode}
%    \end{macro}
%    \begin{macro}{\HoLogoCss@TeX}
%    \begin{macrocode}
\def\HoLogoCss@TeX{%
  \Css{%
    span.HoLogo-TeX span.HoLogo-e{%
      position:relative;%
      top:.5ex;%
      margin-left:-.1667em;%
      margin-right:-.125em;%
    }%
  }%
  \Css{%
    a span.HoLogo-TeX span.HoLogo-e{%
      text-decoration:none;%
    }%
  }%
  \global\let\HoLogoCss@TeX\relax
}
%    \end{macrocode}
%    \end{macro}
%
% \subsubsection{\hologo{plainTeX}}
%
%    \begin{macro}{\HoLogo@plainTeX@space}
%    Source: ``The \hologo{TeX}book''
%    \begin{macrocode}
\def\HoLogo@plainTeX@space#1{%
  \HOLOGO@mbox{#1{p}{P}lain}\HOLOGO@space\hologo{TeX}%
}
%    \end{macrocode}
%    \end{macro}
%    \begin{macro}{\HoLogoCs@plainTeX@space}
%    \begin{macrocode}
\def\HoLogoCs@plainTeX@space#1{#1{p}{P}lain TeX}%
%    \end{macrocode}
%    \end{macro}
%    \begin{macro}{\HoLogoBkm@plainTeX@space}
%    \begin{macrocode}
\def\HoLogoBkm@plainTeX@space#1{%
  #1{p}{P}lain \hologo{TeX}%
}
%    \end{macrocode}
%    \end{macro}
%    \begin{macro}{\HoLogoHtml@plainTeX@space}
%    \begin{macrocode}
\def\HoLogoHtml@plainTeX@space#1{%
  #1{p}{P}lain \hologo{TeX}%
}
%    \end{macrocode}
%    \end{macro}
%
%    \begin{macro}{\HoLogo@plainTeX@hyphen}
%    \begin{macrocode}
\def\HoLogo@plainTeX@hyphen#1{%
  \HOLOGO@mbox{#1{p}{P}lain}\HOLOGO@hyphen\hologo{TeX}%
}
%    \end{macrocode}
%    \end{macro}
%    \begin{macro}{\HoLogoCs@plainTeX@hyphen}
%    \begin{macrocode}
\def\HoLogoCs@plainTeX@hyphen#1{#1{p}{P}lain-TeX}
%    \end{macrocode}
%    \end{macro}
%    \begin{macro}{\HoLogoBkm@plainTeX@hyphen}
%    \begin{macrocode}
\def\HoLogoBkm@plainTeX@hyphen#1{%
  #1{p}{P}lain-\hologo{TeX}%
}
%    \end{macrocode}
%    \end{macro}
%    \begin{macro}{\HoLogoHtml@plainTeX@hyphen}
%    \begin{macrocode}
\def\HoLogoHtml@plainTeX@hyphen#1{%
  #1{p}{P}lain-\hologo{TeX}%
}
%    \end{macrocode}
%    \end{macro}
%
%    \begin{macro}{\HoLogo@plainTeX@runtogether}
%    \begin{macrocode}
\def\HoLogo@plainTeX@runtogether#1{%
  \HOLOGO@mbox{#1{p}{P}lain\hologo{TeX}}%
}
%    \end{macrocode}
%    \end{macro}
%    \begin{macro}{\HoLogoCs@plainTeX@runtogether}
%    \begin{macrocode}
\def\HoLogoCs@plainTeX@runtogether#1{#1{p}{P}lainTeX}
%    \end{macrocode}
%    \end{macro}
%    \begin{macro}{\HoLogoBkm@plainTeX@runtogether}
%    \begin{macrocode}
\def\HoLogoBkm@plainTeX@runtogether#1{%
  #1{p}{P}lain\hologo{TeX}%
}
%    \end{macrocode}
%    \end{macro}
%    \begin{macro}{\HoLogoHtml@plainTeX@runtogether}
%    \begin{macrocode}
\def\HoLogoHtml@plainTeX@runtogether#1{%
  #1{p}{P}lain\hologo{TeX}%
}
%    \end{macrocode}
%    \end{macro}
%
%    \begin{macro}{\HoLogo@plainTeX}
%    \begin{macrocode}
\def\HoLogo@plainTeX{\HoLogo@plainTeX@space}
%    \end{macrocode}
%    \end{macro}
%    \begin{macro}{\HoLogoCs@plainTeX}
%    \begin{macrocode}
\def\HoLogoCs@plainTeX{\HoLogoCs@plainTeX@space}
%    \end{macrocode}
%    \end{macro}
%    \begin{macro}{\HoLogoBkm@plainTeX}
%    \begin{macrocode}
\def\HoLogoBkm@plainTeX{\HoLogoBkm@plainTeX@space}
%    \end{macrocode}
%    \end{macro}
%    \begin{macro}{\HoLogoHtml@plainTeX}
%    \begin{macrocode}
\def\HoLogoHtml@plainTeX{\HoLogoHtml@plainTeX@space}
%    \end{macrocode}
%    \end{macro}
%
% \subsubsection{\hologo{LaTeX}}
%
%    Source: \hologo{LaTeX} kernel.
%\begin{quote}
%\begin{verbatim}
%\DeclareRobustCommand{\LaTeX}{%
%  L%
%  \kern-.36em%
%  {%
%    \sbox\z@ T%
%    \vbox to\ht\z@{%
%      \hbox{%
%        \check@mathfonts
%        \fontsize\sf@size\z@
%        \math@fontsfalse
%        \selectfont
%        A%
%      }%
%      \vss
%    }%
%  }%
%  \kern-.15em%
%  \TeX
%}
%\end{verbatim}
%\end{quote}
%
%    \begin{macro}{\HoLogo@La}
%    \begin{macrocode}
\def\HoLogo@La#1{%
  L%
  \kern-.36em%
  \begingroup
    \setbox\ltx@zero\hbox{T}%
    \vbox to\ht\ltx@zero{%
      \hbox{%
        \ltx@ifundefined{check@mathfonts}{%
          \csname sevenrm\endcsname
        }{%
          \check@mathfonts
          \fontsize\sf@size{0pt}%
          \math@fontsfalse\selectfont
        }%
        A%
      }%
      \vss
    }%
  \endgroup
}
%    \end{macrocode}
%    \end{macro}
%
%    \begin{macro}{\HoLogo@LaTeX}
%    Source: \hologo{LaTeX} kernel.
%    \begin{macrocode}
\def\HoLogo@LaTeX#1{%
  \hologo{La}%
  \kern-.15em%
  \hologo{TeX}%
}
%    \end{macrocode}
%    \end{macro}
%    \begin{macro}{\HoLogoHtml@LaTeX}
%    \begin{macrocode}
\def\HoLogoHtml@LaTeX#1{%
  \HoLogoCss@LaTeX
  \HOLOGO@Span{LaTeX}{%
    L%
    \HOLOGO@Span{a}{%
      A%
    }%
    \hologo{TeX}%
  }%
}
%    \end{macrocode}
%    \end{macro}
%    \begin{macro}{\HoLogoCss@LaTeX}
%    \begin{macrocode}
\def\HoLogoCss@LaTeX{%
  \Css{%
    span.HoLogo-LaTeX span.HoLogo-a{%
      position:relative;%
      top:-.5ex;%
      margin-left:-.36em;%
      margin-right:-.15em;%
      font-size:85\%;%
    }%
  }%
  \global\let\HoLogoCss@LaTeX\relax
}
%    \end{macrocode}
%    \end{macro}
%
% \subsubsection{\hologo{(La)TeX}}
%
%    \begin{macro}{\HoLogo@LaTeXTeX}
%    The kerning around the parentheses is taken
%    from package \xpackage{dtklogos} \cite{dtklogos}.
%\begin{quote}
%\begin{verbatim}
%\DeclareRobustCommand{\LaTeXTeX}{%
%  (%
%  \kern-.15em%
%  L%
%  \kern-.36em%
%  {%
%    \sbox\z@ T%
%    \vbox to\ht0{%
%      \hbox{%
%        $\m@th$%
%        \csname S@\f@size\endcsname
%        \fontsize\sf@size\z@
%        \math@fontsfalse
%        \selectfont
%        A%
%      }%
%      \vss
%    }%
%  }%
%  \kern-.2em%
%  )%
%  \kern-.15em%
%  \TeX
%}
%\end{verbatim}
%\end{quote}
%    \begin{macrocode}
\def\HoLogo@LaTeXTeX#1{%
  (%
  \kern-.15em%
  \hologo{La}%
  \kern-.2em%
  )%
  \kern-.15em%
  \hologo{TeX}%
}
%    \end{macrocode}
%    \end{macro}
%    \begin{macro}{\HoLogoBkm@LaTeXTeX}
%    \begin{macrocode}
\def\HoLogoBkm@LaTeXTeX#1{(La)TeX}
%    \end{macrocode}
%    \end{macro}
%
%    \begin{macro}{\HoLogo@(La)TeX}
%    \begin{macrocode}
\expandafter
\let\csname HoLogo@(La)TeX\endcsname\HoLogo@LaTeXTeX
%    \end{macrocode}
%    \end{macro}
%    \begin{macro}{\HoLogoBkm@(La)TeX}
%    \begin{macrocode}
\expandafter
\let\csname HoLogoBkm@(La)TeX\endcsname\HoLogoBkm@LaTeXTeX
%    \end{macrocode}
%    \end{macro}
%    \begin{macro}{\HoLogoHtml@LaTeXTeX}
%    \begin{macrocode}
\def\HoLogoHtml@LaTeXTeX#1{%
  \HoLogoCss@LaTeXTeX
  \HOLOGO@Span{LaTeXTeX}{%
    (%
    \HOLOGO@Span{L}{L}%
    \HOLOGO@Span{a}{A}%
    \HOLOGO@Span{ParenRight}{)}%
    \hologo{TeX}%
  }%
}
%    \end{macrocode}
%    \end{macro}
%    \begin{macro}{\HoLogoHtml@(La)TeX}
%    Kerning after opening parentheses and before closing parentheses
%    is $-0.1$\,em. The original values $-0.15$\,em
%    looked too ugly for a serif font.
%    \begin{macrocode}
\expandafter
\let\csname HoLogoHtml@(La)TeX\endcsname\HoLogoHtml@LaTeXTeX
%    \end{macrocode}
%    \end{macro}
%    \begin{macro}{\HoLogoCss@LaTeXTeX}
%    \begin{macrocode}
\def\HoLogoCss@LaTeXTeX{%
  \Css{%
    span.HoLogo-LaTeXTeX span.HoLogo-L{%
      margin-left:-.1em;%
    }%
  }%
  \Css{%
    span.HoLogo-LaTeXTeX span.HoLogo-a{%
      position:relative;%
      top:-.5ex;%
      margin-left:-.36em;%
      margin-right:-.1em;%
      font-size:85\%;%
    }%
  }%
  \Css{%
    span.HoLogo-LaTeXTeX span.HoLogo-ParenRight{%
      margin-right:-.15em;%
    }%
  }%
  \global\let\HoLogoCss@LaTeXTeX\relax
}
%    \end{macrocode}
%    \end{macro}
%
% \subsubsection{\hologo{LaTeXe}}
%
%    \begin{macro}{\HoLogo@LaTeXe}
%    Source: \hologo{LaTeX} kernel
%    \begin{macrocode}
\def\HoLogo@LaTeXe#1{%
  \hologo{LaTeX}%
  \kern.15em%
  \hbox{%
    \HOLOGO@MathSetup
    2%
    $_{\textstyle\varepsilon}$%
  }%
}
%    \end{macrocode}
%    \end{macro}
%
%    \begin{macro}{\HoLogoCs@LaTeXe}
%    \begin{macrocode}
\ifnum64=`\^^^^0040\relax % test for big chars of LuaTeX/XeTeX
  \catcode`\$=9 %
  \catcode`\&=14 %
\else
  \catcode`\$=14 %
  \catcode`\&=9 %
\fi
\def\HoLogoCs@LaTeXe#1{%
  LaTeX2%
$ \string ^^^^0395%
& e%
}%
\catcode`\$=3 %
\catcode`\&=4 %
%    \end{macrocode}
%    \end{macro}
%
%    \begin{macro}{\HoLogoBkm@LaTeXe}
%    \begin{macrocode}
\def\HoLogoBkm@LaTeXe#1{%
  \hologo{LaTeX}%
  2%
  \HOLOGO@PdfdocUnicode{e}{\textepsilon}%
}
%    \end{macrocode}
%    \end{macro}
%
%    \begin{macro}{\HoLogoHtml@LaTeXe}
%    \begin{macrocode}
\def\HoLogoHtml@LaTeXe#1{%
  \HoLogoCss@LaTeXe
  \HOLOGO@Span{LaTeX2e}{%
    \hologo{LaTeX}%
    \HOLOGO@Span{2}{2}%
    \HOLOGO@Span{e}{%
      \HOLOGO@MathSetup
      \ensuremath{\textstyle\varepsilon}%
    }%
  }%
}
%    \end{macrocode}
%    \end{macro}
%    \begin{macro}{\HoLogoCss@LaTeXe}
%    \begin{macrocode}
\def\HoLogoCss@LaTeXe{%
  \Css{%
    span.HoLogo-LaTeX2e span.HoLogo-2{%
      padding-left:.15em;%
    }%
  }%
  \Css{%
    span.HoLogo-LaTeX2e span.HoLogo-e{%
      position:relative;%
      top:.35ex;%
      text-decoration:none;%
    }%
  }%
  \global\let\HoLogoCss@LaTeXe\relax
}
%    \end{macrocode}
%    \end{macro}
%
%    \begin{macro}{\HoLogo@LaTeX2e}
%    \begin{macrocode}
\expandafter
\let\csname HoLogo@LaTeX2e\endcsname\HoLogo@LaTeXe
%    \end{macrocode}
%    \end{macro}
%    \begin{macro}{\HoLogoCs@LaTeX2e}
%    \begin{macrocode}
\expandafter
\let\csname HoLogoCs@LaTeX2e\endcsname\HoLogoCs@LaTeXe
%    \end{macrocode}
%    \end{macro}
%    \begin{macro}{\HoLogoBkm@LaTeX2e}
%    \begin{macrocode}
\expandafter
\let\csname HoLogoBkm@LaTeX2e\endcsname\HoLogoBkm@LaTeXe
%    \end{macrocode}
%    \end{macro}
%    \begin{macro}{\HoLogoHtml@LaTeX2e}
%    \begin{macrocode}
\expandafter
\let\csname HoLogoHtml@LaTeX2e\endcsname\HoLogoHtml@LaTeXe
%    \end{macrocode}
%    \end{macro}
%
% \subsubsection{\hologo{LaTeX3}}
%
%    \begin{macro}{\HoLogo@LaTeX3}
%    Source: \hologo{LaTeX} kernel
%    \begin{macrocode}
\expandafter\def\csname HoLogo@LaTeX3\endcsname#1{%
  \hologo{LaTeX}%
  3%
}
%    \end{macrocode}
%    \end{macro}
%
%    \begin{macro}{\HoLogoBkm@LaTeX3}
%    \begin{macrocode}
\expandafter\def\csname HoLogoBkm@LaTeX3\endcsname#1{%
  \hologo{LaTeX}%
  3%
}
%    \end{macrocode}
%    \end{macro}
%    \begin{macro}{\HoLogoHtml@LaTeX3}
%    \begin{macrocode}
\expandafter
\let\csname HoLogoHtml@LaTeX3\expandafter\endcsname
\csname HoLogo@LaTeX3\endcsname
%    \end{macrocode}
%    \end{macro}
%
% \subsubsection{\hologo{LaTeXML}}
%
%    \begin{macro}{\HoLogo@LaTeXML}
%    \begin{macrocode}
\def\HoLogo@LaTeXML#1{%
  \HOLOGO@mbox{%
    \hologo{La}%
    \kern-.15em%
    T%
    \kern-.1667em%
    \lower.5ex\hbox{E}%
    \kern-.125em%
    \HoLogoFont@font{LaTeXML}{sc}{xml}%
  }%
}
%    \end{macrocode}
%    \end{macro}
%    \begin{macro}{\HoLogoHtml@pdfLaTeX}
%    \begin{macrocode}
\def\HoLogoHtml@LaTeXML#1{%
  \HOLOGO@Span{LaTeXML}{%
    \HoLogoCss@LaTeX
    \HoLogoCss@TeX
    \HOLOGO@Span{LaTeX}{%
      L%
      \HOLOGO@Span{a}{%
        A%
      }%
    }%
    \HOLOGO@Span{TeX}{%
      T%
      \HOLOGO@Span{e}{%
        E%
      }%
    }%
    \HCode{<span style="font-variant: small-caps;">}%
    xml%
    \HCode{</span>}%
  }%
}
%    \end{macrocode}
%    \end{macro}
%
% \subsubsection{\hologo{eTeX}}
%
%    \begin{macro}{\HoLogo@eTeX}
%    Source: package \xpackage{etex}
%    \begin{macrocode}
\def\HoLogo@eTeX#1{%
  \ltx@mbox{%
    \HOLOGO@MathSetup
    $\varepsilon$%
    -%
    \HOLOGO@NegativeKerning{-T,T-,To}%
    \hologo{TeX}%
  }%
}
%    \end{macrocode}
%    \end{macro}
%    \begin{macro}{\HoLogoCs@eTeX}
%    \begin{macrocode}
\ifnum64=`\^^^^0040\relax % test for big chars of LuaTeX/XeTeX
  \catcode`\$=9 %
  \catcode`\&=14 %
\else
  \catcode`\$=14 %
  \catcode`\&=9 %
\fi
\def\HoLogoCs@eTeX#1{%
$ #1{\string ^^^^0395}{\string ^^^^03b5}%
& #1{e}{E}%
  TeX%
}%
\catcode`\$=3 %
\catcode`\&=4 %
%    \end{macrocode}
%    \end{macro}
%    \begin{macro}{\HoLogoBkm@eTeX}
%    \begin{macrocode}
\def\HoLogoBkm@eTeX#1{%
  \HOLOGO@PdfdocUnicode{#1{e}{E}}{\textepsilon}%
  -%
  \hologo{TeX}%
}
%    \end{macrocode}
%    \end{macro}
%    \begin{macro}{\HoLogoHtml@eTeX}
%    \begin{macrocode}
\def\HoLogoHtml@eTeX#1{%
  \ltx@mbox{%
    \HOLOGO@MathSetup
    $\varepsilon$%
    -%
    \hologo{TeX}%
  }%
}
%    \end{macrocode}
%    \end{macro}
%
% \subsubsection{\hologo{iniTeX}}
%
%    \begin{macro}{\HoLogo@iniTeX}
%    \begin{macrocode}
\def\HoLogo@iniTeX#1{%
  \HOLOGO@mbox{%
    #1{i}{I}ni\hologo{TeX}%
  }%
}
%    \end{macrocode}
%    \end{macro}
%    \begin{macro}{\HoLogoCs@iniTeX}
%    \begin{macrocode}
\def\HoLogoCs@iniTeX#1{#1{i}{I}niTeX}
%    \end{macrocode}
%    \end{macro}
%    \begin{macro}{\HoLogoBkm@iniTeX}
%    \begin{macrocode}
\def\HoLogoBkm@iniTeX#1{%
  #1{i}{I}ni\hologo{TeX}%
}
%    \end{macrocode}
%    \end{macro}
%    \begin{macro}{\HoLogoHtml@iniTeX}
%    \begin{macrocode}
\let\HoLogoHtml@iniTeX\HoLogo@iniTeX
%    \end{macrocode}
%    \end{macro}
%
% \subsubsection{\hologo{virTeX}}
%
%    \begin{macro}{\HoLogo@virTeX}
%    \begin{macrocode}
\def\HoLogo@virTeX#1{%
  \HOLOGO@mbox{%
    #1{v}{V}ir\hologo{TeX}%
  }%
}
%    \end{macrocode}
%    \end{macro}
%    \begin{macro}{\HoLogoCs@virTeX}
%    \begin{macrocode}
\def\HoLogoCs@virTeX#1{#1{v}{V}irTeX}
%    \end{macrocode}
%    \end{macro}
%    \begin{macro}{\HoLogoBkm@virTeX}
%    \begin{macrocode}
\def\HoLogoBkm@virTeX#1{%
  #1{v}{V}ir\hologo{TeX}%
}
%    \end{macrocode}
%    \end{macro}
%    \begin{macro}{\HoLogoHtml@virTeX}
%    \begin{macrocode}
\let\HoLogoHtml@virTeX\HoLogo@virTeX
%    \end{macrocode}
%    \end{macro}
%
% \subsubsection{\hologo{SliTeX}}
%
% \paragraph{Definitions of the three variants.}
%
%    \begin{macro}{\HoLogo@SLiTeX@lift}
%    \begin{macrocode}
\def\HoLogo@SLiTeX@lift#1{%
  \HoLogoFont@font{SliTeX}{rm}{%
    S%
    \kern-.06em%
    L%
    \kern-.18em%
    \raise.32ex\hbox{\HoLogoFont@font{SliTeX}{sc}{i}}%
    \HOLOGO@discretionary
    \kern-.06em%
    \hologo{TeX}%
  }%
}
%    \end{macrocode}
%    \end{macro}
%    \begin{macro}{\HoLogoBkm@SLiTeX@lift}
%    \begin{macrocode}
\def\HoLogoBkm@SLiTeX@lift#1{SLiTeX}
%    \end{macrocode}
%    \end{macro}
%    \begin{macro}{\HoLogoHtml@SLiTeX@lift}
%    \begin{macrocode}
\def\HoLogoHtml@SLiTeX@lift#1{%
  \HoLogoCss@SLiTeX@lift
  \HOLOGO@Span{SLiTeX-lift}{%
    \HoLogoFont@font{SliTeX}{rm}{%
      S%
      \HOLOGO@Span{L}{L}%
      \HOLOGO@Span{i}{i}%
      \hologo{TeX}%
    }%
  }%
}
%    \end{macrocode}
%    \end{macro}
%    \begin{macro}{\HoLogoCss@SLiTeX@lift}
%    \begin{macrocode}
\def\HoLogoCss@SLiTeX@lift{%
  \Css{%
    span.HoLogo-SLiTeX-lift span.HoLogo-L{%
      margin-left:-.06em;%
      margin-right:-.18em;%
    }%
  }%
  \Css{%
    span.HoLogo-SLiTeX-lift span.HoLogo-i{%
      position:relative;%
      top:-.32ex;%
      margin-right:-.06em;%
      font-variant:small-caps;%
    }%
  }%
  \global\let\HoLogoCss@SLiTeX@lift\relax
}
%    \end{macrocode}
%    \end{macro}
%
%    \begin{macro}{\HoLogo@SliTeX@simple}
%    \begin{macrocode}
\def\HoLogo@SliTeX@simple#1{%
  \HoLogoFont@font{SliTeX}{rm}{%
    \ltx@mbox{%
      \HoLogoFont@font{SliTeX}{sc}{Sli}%
    }%
    \HOLOGO@discretionary
    \hologo{TeX}%
  }%
}
%    \end{macrocode}
%    \end{macro}
%    \begin{macro}{\HoLogoBkm@SliTeX@simple}
%    \begin{macrocode}
\def\HoLogoBkm@SliTeX@simple#1{SliTeX}
%    \end{macrocode}
%    \end{macro}
%    \begin{macro}{\HoLogoHtml@SliTeX@simple}
%    \begin{macrocode}
\let\HoLogoHtml@SliTeX@simple\HoLogo@SliTeX@simple
%    \end{macrocode}
%    \end{macro}
%
%    \begin{macro}{\HoLogo@SliTeX@narrow}
%    \begin{macrocode}
\def\HoLogo@SliTeX@narrow#1{%
  \HoLogoFont@font{SliTeX}{rm}{%
    \ltx@mbox{%
      S%
      \kern-.06em%
      \HoLogoFont@font{SliTeX}{sc}{%
        l%
        \kern-.035em%
        i%
      }%
    }%
    \HOLOGO@discretionary
    \kern-.06em%
    \hologo{TeX}%
  }%
}
%    \end{macrocode}
%    \end{macro}
%    \begin{macro}{\HoLogoBkm@SliTeX@narrow}
%    \begin{macrocode}
\def\HoLogoBkm@SliTeX@narrow#1{SliTeX}
%    \end{macrocode}
%    \end{macro}
%    \begin{macro}{\HoLogoHtml@SliTeX@narrow}
%    \begin{macrocode}
\def\HoLogoHtml@SliTeX@narrow#1{%
  \HoLogoCss@SliTeX@narrow
  \HOLOGO@Span{SliTeX-narrow}{%
    \HoLogoFont@font{SliTeX}{rm}{%
      S%
        \HOLOGO@Span{l}{l}%
        \HOLOGO@Span{i}{i}%
      \hologo{TeX}%
    }%
  }%
}
%    \end{macrocode}
%    \end{macro}
%    \begin{macro}{\HoLogoCss@SliTeX@narrow}
%    \begin{macrocode}
\def\HoLogoCss@SliTeX@narrow{%
  \Css{%
    span.HoLogo-SliTeX-narrow span.HoLogo-l{%
      margin-left:-.06em;%
      margin-right:-.035em;%
      font-variant:small-caps;%
    }%
  }%
  \Css{%
    span.HoLogo-SliTeX-narrow span.HoLogo-i{%
      margin-right:-.06em;%
      font-variant:small-caps;%
    }%
  }%
  \global\let\HoLogoCss@SliTeX@narrow\relax
}
%    \end{macrocode}
%    \end{macro}
%
% \paragraph{Macro set completion.}
%
%    \begin{macro}{\HoLogo@SLiTeX@simple}
%    \begin{macrocode}
\def\HoLogo@SLiTeX@simple{\HoLogo@SliTeX@simple}
%    \end{macrocode}
%    \end{macro}
%    \begin{macro}{\HoLogoBkm@SLiTeX@simple}
%    \begin{macrocode}
\def\HoLogoBkm@SLiTeX@simple{\HoLogoBkm@SliTeX@simple}
%    \end{macrocode}
%    \end{macro}
%    \begin{macro}{\HoLogoHtml@SLiTeX@simple}
%    \begin{macrocode}
\def\HoLogoHtml@SLiTeX@simple{\HoLogoHtml@SliTeX@simple}
%    \end{macrocode}
%    \end{macro}
%
%    \begin{macro}{\HoLogo@SLiTeX@narrow}
%    \begin{macrocode}
\def\HoLogo@SLiTeX@narrow{\HoLogo@SliTeX@narrow}
%    \end{macrocode}
%    \end{macro}
%    \begin{macro}{\HoLogoBkm@SLiTeX@narrow}
%    \begin{macrocode}
\def\HoLogoBkm@SLiTeX@narrow{\HoLogoBkm@SliTeX@narrow}
%    \end{macrocode}
%    \end{macro}
%    \begin{macro}{\HoLogoHtml@SLiTeX@narrow}
%    \begin{macrocode}
\def\HoLogoHtml@SLiTeX@narrow{\HoLogoHtml@SliTeX@narrow}
%    \end{macrocode}
%    \end{macro}
%
%    \begin{macro}{\HoLogo@SliTeX@lift}
%    \begin{macrocode}
\def\HoLogo@SliTeX@lift{\HoLogo@SLiTeX@lift}
%    \end{macrocode}
%    \end{macro}
%    \begin{macro}{\HoLogoBkm@SliTeX@lift}
%    \begin{macrocode}
\def\HoLogoBkm@SliTeX@lift{\HoLogoBkm@SLiTeX@lift}
%    \end{macrocode}
%    \end{macro}
%    \begin{macro}{\HoLogoHtml@SliTeX@lift}
%    \begin{macrocode}
\def\HoLogoHtml@SliTeX@lift{\HoLogoHtml@SLiTeX@lift}
%    \end{macrocode}
%    \end{macro}
%
% \paragraph{Defaults.}
%
%    \begin{macro}{\HoLogo@SLiTeX}
%    \begin{macrocode}
\def\HoLogo@SLiTeX{\HoLogo@SLiTeX@lift}
%    \end{macrocode}
%    \end{macro}
%    \begin{macro}{\HoLogoBkm@SLiTeX}
%    \begin{macrocode}
\def\HoLogoBkm@SLiTeX{\HoLogoBkm@SLiTeX@lift}
%    \end{macrocode}
%    \end{macro}
%    \begin{macro}{\HoLogoHtml@SLiTeX}
%    \begin{macrocode}
\def\HoLogoHtml@SLiTeX{\HoLogoHtml@SLiTeX@lift}
%    \end{macrocode}
%    \end{macro}
%
%    \begin{macro}{\HoLogo@SliTeX}
%    \begin{macrocode}
\def\HoLogo@SliTeX{\HoLogo@SliTeX@narrow}
%    \end{macrocode}
%    \end{macro}
%    \begin{macro}{\HoLogoBkm@SliTeX}
%    \begin{macrocode}
\def\HoLogoBkm@SliTeX{\HoLogoBkm@SliTeX@narrow}
%    \end{macrocode}
%    \end{macro}
%    \begin{macro}{\HoLogoHtml@SliTeX}
%    \begin{macrocode}
\def\HoLogoHtml@SliTeX{\HoLogoHtml@SliTeX@narrow}
%    \end{macrocode}
%    \end{macro}
%
% \subsubsection{\hologo{LuaTeX}}
%
%    \begin{macro}{\HoLogo@LuaTeX}
%    The kerning is an idea of Hans Hagen, see mailing list
%    `luatex at tug dot org' in March 2010.
%    \begin{macrocode}
\def\HoLogo@LuaTeX#1{%
  \HOLOGO@mbox{%
    Lua%
    \HOLOGO@NegativeKerning{aT,oT,To}%
    \hologo{TeX}%
  }%
}
%    \end{macrocode}
%    \end{macro}
%    \begin{macro}{\HoLogoHtml@LuaTeX}
%    \begin{macrocode}
\let\HoLogoHtml@LuaTeX\HoLogo@LuaTeX
%    \end{macrocode}
%    \end{macro}
%
% \subsubsection{\hologo{LuaLaTeX}}
%
%    \begin{macro}{\HoLogo@LuaLaTeX}
%    \begin{macrocode}
\def\HoLogo@LuaLaTeX#1{%
  \HOLOGO@mbox{%
    Lua%
    \hologo{LaTeX}%
  }%
}
%    \end{macrocode}
%    \end{macro}
%    \begin{macro}{\HoLogoHtml@LuaLaTeX}
%    \begin{macrocode}
\let\HoLogoHtml@LuaLaTeX\HoLogo@LuaLaTeX
%    \end{macrocode}
%    \end{macro}
%
% \subsubsection{\hologo{XeTeX}, \hologo{XeLaTeX}}
%
%    \begin{macro}{\HOLOGO@IfCharExists}
%    \begin{macrocode}
\ifluatex
  \ifnum\luatexversion<36 %
  \else
    \def\HOLOGO@IfCharExists#1{%
      \ifnum
        \directlua{%
           if luaotfload and luaotfload.aux then
             if luaotfload.aux.font_has_glyph(%
                    font.current(), \number#1) then % 	 
	       tex.print("1") % 	 
	     end % 	 
	   elseif font and font.fonts and font.current then %
            local f = font.fonts[font.current()]%
            if f.characters and f.characters[\number#1] then %
              tex.print("1")%
            end %
          end%
        }0=\ltx@zero
        \expandafter\ltx@secondoftwo
      \else
        \expandafter\ltx@firstoftwo
      \fi
    }%
  \fi
\fi
\ltx@IfUndefined{HOLOGO@IfCharExists}{%
  \def\HOLOGO@@IfCharExists#1{%
    \begingroup
      \tracinglostchars=\ltx@zero
      \setbox\ltx@zero=\hbox{%
        \kern7sp\char#1\relax
        \ifnum\lastkern>\ltx@zero
          \expandafter\aftergroup\csname iffalse\endcsname
        \else
          \expandafter\aftergroup\csname iftrue\endcsname
        \fi
      }%
      % \if{true|false} from \aftergroup
      \endgroup
      \expandafter\ltx@firstoftwo
    \else
      \endgroup
      \expandafter\ltx@secondoftwo
    \fi
  }%
  \ifxetex
    \ltx@IfUndefined{XeTeXfonttype}{}{%
      \ltx@IfUndefined{XeTeXcharglyph}{}{%
        \def\HOLOGO@IfCharExists#1{%
          \ifnum\XeTeXfonttype\font>\ltx@zero
            \expandafter\ltx@firstofthree
          \else
            \expandafter\ltx@gobble
          \fi
          {%
            \ifnum\XeTeXcharglyph#1>\ltx@zero
              \expandafter\ltx@firstoftwo
            \else
              \expandafter\ltx@secondoftwo
            \fi
          }%
          \HOLOGO@@IfCharExists{#1}%
        }%
      }%
    }%
  \fi
}{}
\ltx@ifundefined{HOLOGO@IfCharExists}{%
  \ifnum64=`\^^^^0040\relax % test for big chars of LuaTeX/XeTeX
    \let\HOLOGO@IfCharExists\HOLOGO@@IfCharExists
  \else
    \def\HOLOGO@IfCharExists#1{%
      \ifnum#1>255 %
        \expandafter\ltx@fourthoffour
      \fi
      \HOLOGO@@IfCharExists{#1}%
    }%
  \fi
}{}
%    \end{macrocode}
%    \end{macro}
%
%    \begin{macro}{\HoLogo@Xe}
%    Source: package \xpackage{dtklogos}
%    \begin{macrocode}
\def\HoLogo@Xe#1{%
  X%
  \kern-.1em\relax
  \HOLOGO@IfCharExists{"018E}{%
    \lower.5ex\hbox{\char"018E}%
  }{%
    \chardef\HOLOGO@choice=\ltx@zero
    \ifdim\fontdimen\ltx@one\font>0pt %
      \ltx@IfUndefined{rotatebox}{%
        \ltx@IfUndefined{pgftext}{%
          \ltx@IfUndefined{psscalebox}{%
            \ltx@IfUndefined{HOLOGO@ScaleBox@\hologoDriver}{%
            }{%
              \chardef\HOLOGO@choice=4 %
            }%
          }{%
            \chardef\HOLOGO@choice=3 %
          }%
        }{%
          \chardef\HOLOGO@choice=2 %
        }%
      }{%
        \chardef\HOLOGO@choice=1 %
      }%
      \ifcase\HOLOGO@choice
        \HOLOGO@WarningUnsupportedDriver{Xe}%
        e%
      \or % 1: \rotatebox
        \begingroup
          \setbox\ltx@zero\hbox{\rotatebox{180}{E}}%
          \ltx@LocDimenA=\dp\ltx@zero
          \advance\ltx@LocDimenA by -.5ex\relax
          \raise\ltx@LocDimenA\box\ltx@zero
        \endgroup
      \or % 2: \pgftext
        \lower.5ex\hbox{%
          \pgfpicture
            \pgftext[rotate=180]{E}%
          \endpgfpicture
        }%
      \or % 3: \psscalebox
        \begingroup
          \setbox\ltx@zero\hbox{\psscalebox{-1 -1}{E}}%
          \ltx@LocDimenA=\dp\ltx@zero
          \advance\ltx@LocDimenA by -.5ex\relax
          \raise\ltx@LocDimenA\box\ltx@zero
        \endgroup
      \or % 4: \HOLOGO@PointReflectBox
        \lower.5ex\hbox{\HOLOGO@PointReflectBox{E}}%
      \else
        \@PackageError{hologo}{Internal error (choice/it}\@ehc
      \fi
    \else
      \ltx@IfUndefined{reflectbox}{%
        \ltx@IfUndefined{pgftext}{%
          \ltx@IfUndefined{psscalebox}{%
            \ltx@IfUndefined{HOLOGO@ScaleBox@\hologoDriver}{%
            }{%
              \chardef\HOLOGO@choice=4 %
            }%
          }{%
            \chardef\HOLOGO@choice=3 %
          }%
        }{%
          \chardef\HOLOGO@choice=2 %
        }%
      }{%
        \chardef\HOLOGO@choice=1 %
      }%
      \ifcase\HOLOGO@choice
        \HOLOGO@WarningUnsupportedDriver{Xe}%
        e%
      \or % 1: reflectbox
        \lower.5ex\hbox{%
          \reflectbox{E}%
        }%
      \or % 2: \pgftext
        \lower.5ex\hbox{%
          \pgfpicture
            \pgftransformxscale{-1}%
            \pgftext{E}%
          \endpgfpicture
        }%
      \or % 3: \psscalebox
        \lower.5ex\hbox{%
          \psscalebox{-1 1}{E}%
        }%
      \or % 4: \HOLOGO@Reflectbox
        \lower.5ex\hbox{%
          \HOLOGO@ReflectBox{E}%
        }%
      \else
        \@PackageError{hologo}{Internal error (choice/up)}\@ehc
      \fi
    \fi
  }%
}
%    \end{macrocode}
%    \end{macro}
%    \begin{macro}{\HoLogoHtml@Xe}
%    \begin{macrocode}
\def\HoLogoHtml@Xe#1{%
  \HoLogoCss@Xe
  \HOLOGO@Span{Xe}{%
    X%
    \HOLOGO@Span{e}{%
      \HCode{&\ltx@hashchar x018e;}%
    }%
  }%
}
%    \end{macrocode}
%    \end{macro}
%    \begin{macro}{\HoLogoCss@Xe}
%    \begin{macrocode}
\def\HoLogoCss@Xe{%
  \Css{%
    span.HoLogo-Xe span.HoLogo-e{%
      position:relative;%
      top:.5ex;%
      left-margin:-.1em;%
    }%
  }%
  \global\let\HoLogoCss@Xe\relax
}
%    \end{macrocode}
%    \end{macro}
%
%    \begin{macro}{\HoLogo@XeTeX}
%    \begin{macrocode}
\def\HoLogo@XeTeX#1{%
  \hologo{Xe}%
  \kern-.15em\relax
  \hologo{TeX}%
}
%    \end{macrocode}
%    \end{macro}
%
%    \begin{macro}{\HoLogoHtml@XeTeX}
%    \begin{macrocode}
\def\HoLogoHtml@XeTeX#1{%
  \HoLogoCss@XeTeX
  \HOLOGO@Span{XeTeX}{%
    \hologo{Xe}%
    \hologo{TeX}%
  }%
}
%    \end{macrocode}
%    \end{macro}
%    \begin{macro}{\HoLogoCss@XeTeX}
%    \begin{macrocode}
\def\HoLogoCss@XeTeX{%
  \Css{%
    span.HoLogo-XeTeX span.HoLogo-TeX{%
      margin-left:-.15em;%
    }%
  }%
  \global\let\HoLogoCss@XeTeX\relax
}
%    \end{macrocode}
%    \end{macro}
%
%    \begin{macro}{\HoLogo@XeLaTeX}
%    \begin{macrocode}
\def\HoLogo@XeLaTeX#1{%
  \hologo{Xe}%
  \kern-.13em%
  \hologo{LaTeX}%
}
%    \end{macrocode}
%    \end{macro}
%    \begin{macro}{\HoLogoHtml@XeLaTeX}
%    \begin{macrocode}
\def\HoLogoHtml@XeLaTeX#1{%
  \HoLogoCss@XeLaTeX
  \HOLOGO@Span{XeLaTeX}{%
    \hologo{Xe}%
    \hologo{LaTeX}%
  }%
}
%    \end{macrocode}
%    \end{macro}
%    \begin{macro}{\HoLogoCss@XeLaTeX}
%    \begin{macrocode}
\def\HoLogoCss@XeLaTeX{%
  \Css{%
    span.HoLogo-XeLaTeX span.HoLogo-Xe{%
      margin-right:-.13em;%
    }%
  }%
  \global\let\HoLogoCss@XeLaTeX\relax
}
%    \end{macrocode}
%    \end{macro}
%
% \subsubsection{\hologo{pdfTeX}, \hologo{pdfLaTeX}}
%
%    \begin{macro}{\HoLogo@pdfTeX}
%    \begin{macrocode}
\def\HoLogo@pdfTeX#1{%
  \HOLOGO@mbox{%
    #1{p}{P}df\hologo{TeX}%
  }%
}
%    \end{macrocode}
%    \end{macro}
%    \begin{macro}{\HoLogoCs@pdfTeX}
%    \begin{macrocode}
\def\HoLogoCs@pdfTeX#1{#1{p}{P}dfTeX}
%    \end{macrocode}
%    \end{macro}
%    \begin{macro}{\HoLogoBkm@pdfTeX}
%    \begin{macrocode}
\def\HoLogoBkm@pdfTeX#1{%
  #1{p}{P}df\hologo{TeX}%
}
%    \end{macrocode}
%    \end{macro}
%    \begin{macro}{\HoLogoHtml@pdfTeX}
%    \begin{macrocode}
\let\HoLogoHtml@pdfTeX\HoLogo@pdfTeX
%    \end{macrocode}
%    \end{macro}
%
%    \begin{macro}{\HoLogo@pdfLaTeX}
%    \begin{macrocode}
\def\HoLogo@pdfLaTeX#1{%
  \HOLOGO@mbox{%
    #1{p}{P}df\hologo{LaTeX}%
  }%
}
%    \end{macrocode}
%    \end{macro}
%    \begin{macro}{\HoLogoCs@pdfLaTeX}
%    \begin{macrocode}
\def\HoLogoCs@pdfLaTeX#1{#1{p}{P}dfLaTeX}
%    \end{macrocode}
%    \end{macro}
%    \begin{macro}{\HoLogoBkm@pdfLaTeX}
%    \begin{macrocode}
\def\HoLogoBkm@pdfLaTeX#1{%
  #1{p}{P}df\hologo{LaTeX}%
}
%    \end{macrocode}
%    \end{macro}
%    \begin{macro}{\HoLogoHtml@pdfLaTeX}
%    \begin{macrocode}
\let\HoLogoHtml@pdfLaTeX\HoLogo@pdfLaTeX
%    \end{macrocode}
%    \end{macro}
%
% \subsubsection{\hologo{VTeX}}
%
%    \begin{macro}{\HoLogo@VTeX}
%    \begin{macrocode}
\def\HoLogo@VTeX#1{%
  \HOLOGO@mbox{%
    V\hologo{TeX}%
  }%
}
%    \end{macrocode}
%    \end{macro}
%    \begin{macro}{\HoLogoHtml@VTeX}
%    \begin{macrocode}
\let\HoLogoHtml@VTeX\HoLogo@VTeX
%    \end{macrocode}
%    \end{macro}
%
% \subsubsection{\hologo{AmS}, \dots}
%
%    Source: class \xclass{amsdtx}
%
%    \begin{macro}{\HoLogo@AmS}
%    \begin{macrocode}
\def\HoLogo@AmS#1{%
  \HoLogoFont@font{AmS}{sy}{%
    A%
    \kern-.1667em%
    \lower.5ex\hbox{M}%
    \kern-.125em%
    S%
  }%
}
%    \end{macrocode}
%    \end{macro}
%    \begin{macro}{\HoLogoBkm@AmS}
%    \begin{macrocode}
\def\HoLogoBkm@AmS#1{AmS}
%    \end{macrocode}
%    \end{macro}
%    \begin{macro}{\HoLogoHtml@AmS}
%    \begin{macrocode}
\def\HoLogoHtml@AmS#1{%
  \HoLogoCss@AmS
%  \HoLogoFont@font{AmS}{sy}{%
    \HOLOGO@Span{AmS}{%
      A%
      \HOLOGO@Span{M}{M}%
      S%
    }%
%   }%
}
%    \end{macrocode}
%    \end{macro}
%    \begin{macro}{\HoLogoCss@AmS}
%    \begin{macrocode}
\def\HoLogoCss@AmS{%
  \Css{%
    span.HoLogo-AmS span.HoLogo-M{%
      position:relative;%
      top:.5ex;%
      margin-left:-.1667em;%
      margin-right:-.125em;%
      text-decoration:none;%
    }%
  }%
  \global\let\HoLogoCss@AmS\relax
}
%    \end{macrocode}
%    \end{macro}
%
%    \begin{macro}{\HoLogo@AmSTeX}
%    \begin{macrocode}
\def\HoLogo@AmSTeX#1{%
  \hologo{AmS}%
  \HOLOGO@hyphen
  \hologo{TeX}%
}
%    \end{macrocode}
%    \end{macro}
%    \begin{macro}{\HoLogoBkm@AmSTeX}
%    \begin{macrocode}
\def\HoLogoBkm@AmSTeX#1{AmS-TeX}%
%    \end{macrocode}
%    \end{macro}
%    \begin{macro}{\HoLogoHtml@AmSTeX}
%    \begin{macrocode}
\let\HoLogoHtml@AmSTeX\HoLogo@AmSTeX
%    \end{macrocode}
%    \end{macro}
%
%    \begin{macro}{\HoLogo@AmSLaTeX}
%    \begin{macrocode}
\def\HoLogo@AmSLaTeX#1{%
  \hologo{AmS}%
  \HOLOGO@hyphen
  \hologo{LaTeX}%
}
%    \end{macrocode}
%    \end{macro}
%    \begin{macro}{\HoLogoBkm@AmSLaTeX}
%    \begin{macrocode}
\def\HoLogoBkm@AmSLaTeX#1{AmS-LaTeX}%
%    \end{macrocode}
%    \end{macro}
%    \begin{macro}{\HoLogoHtml@AmSLaTeX}
%    \begin{macrocode}
\let\HoLogoHtml@AmSLaTeX\HoLogo@AmSLaTeX
%    \end{macrocode}
%    \end{macro}
%
% \subsubsection{\hologo{BibTeX}}
%
%    \begin{macro}{\HoLogo@BibTeX@sc}
%    A definition of \hologo{BibTeX} is provided in
%    the documentation source for the manual of \hologo{BibTeX}
%    \cite{btxdoc}.
%\begin{quote}
%\begin{verbatim}
%\def\BibTeX{%
%  {%
%    \rm
%    B%
%    \kern-.05em%
%    {%
%      \sc
%      i%
%      \kern-.025em %
%      b%
%    }%
%    \kern-.08em
%    T%
%    \kern-.1667em%
%    \lower.7ex\hbox{E}%
%    \kern-.125em%
%    X%
%  }%
%}
%\end{verbatim}
%\end{quote}
%    \begin{macrocode}
\def\HoLogo@BibTeX@sc#1{%
  B%
  \kern-.05em%
  \HoLogoFont@font{BibTeX}{sc}{%
    i%
    \kern-.025em%
    b%
  }%
  \HOLOGO@discretionary
  \kern-.08em%
  \hologo{TeX}%
}
%    \end{macrocode}
%    \end{macro}
%    \begin{macro}{\HoLogoHtml@BibTeX@sc}
%    \begin{macrocode}
\def\HoLogoHtml@BibTeX@sc#1{%
  \HoLogoCss@BibTeX@sc
  \HOLOGO@Span{BibTeX-sc}{%
    B%
    \HOLOGO@Span{i}{i}%
    \HOLOGO@Span{b}{b}%
    \hologo{TeX}%
  }%
}
%    \end{macrocode}
%    \end{macro}
%    \begin{macro}{\HoLogoCss@BibTeX@sc}
%    \begin{macrocode}
\def\HoLogoCss@BibTeX@sc{%
  \Css{%
    span.HoLogo-BibTeX-sc span.HoLogo-i{%
      margin-left:-.05em;%
      margin-right:-.025em;%
      font-variant:small-caps;%
    }%
  }%
  \Css{%
    span.HoLogo-BibTeX-sc span.HoLogo-b{%
      margin-right:-.08em;%
      font-variant:small-caps;%
    }%
  }%
  \global\let\HoLogoCss@BibTeX@sc\relax
}
%    \end{macrocode}
%    \end{macro}
%
%    \begin{macro}{\HoLogo@BibTeX@sf}
%    Variant \xoption{sf} avoids trouble with unavailable
%    small caps fonts (e.g., bold versions of Computer Modern or
%    Latin Modern). The definition is taken from
%    package \xpackage{dtklogos} \cite{dtklogos}.
%\begin{quote}
%\begin{verbatim}
%\DeclareRobustCommand{\BibTeX}{%
%  B%
%  \kern-.05em%
%  \hbox{%
%    $\m@th$% %% force math size calculations
%    \csname S@\f@size\endcsname
%    \fontsize\sf@size\z@
%    \math@fontsfalse
%    \selectfont
%    I%
%    \kern-.025em%
%    B
%  }%
%  \kern-.08em%
%  \-%
%  \TeX
%}
%\end{verbatim}
%\end{quote}
%    \begin{macrocode}
\def\HoLogo@BibTeX@sf#1{%
  B%
  \kern-.05em%
  \HoLogoFont@font{BibTeX}{bibsf}{%
    I%
    \kern-.025em%
    B%
  }%
  \HOLOGO@discretionary
  \kern-.08em%
  \hologo{TeX}%
}
%    \end{macrocode}
%    \end{macro}
%    \begin{macro}{\HoLogoHtml@BibTeX@sf}
%    \begin{macrocode}
\def\HoLogoHtml@BibTeX@sf#1{%
  \HoLogoCss@BibTeX@sf
  \HOLOGO@Span{BibTeX-sf}{%
    B%
    \HoLogoFont@font{BibTeX}{bibsf}{%
      \HOLOGO@Span{i}{I}%
      B%
    }%
    \hologo{TeX}%
  }%
}
%    \end{macrocode}
%    \end{macro}
%    \begin{macro}{\HoLogoCss@BibTeX@sf}
%    \begin{macrocode}
\def\HoLogoCss@BibTeX@sf{%
  \Css{%
    span.HoLogo-BibTeX-sf span.HoLogo-i{%
      margin-left:-.05em;%
      margin-right:-.025em;%
    }%
  }%
  \Css{%
    span.HoLogo-BibTeX-sf span.HoLogo-TeX{%
      margin-left:-.08em;%
    }%
  }%
  \global\let\HoLogoCss@BibTeX@sf\relax
}
%    \end{macrocode}
%    \end{macro}
%
%    \begin{macro}{\HoLogo@BibTeX}
%    \begin{macrocode}
\def\HoLogo@BibTeX{\HoLogo@BibTeX@sf}
%    \end{macrocode}
%    \end{macro}
%    \begin{macro}{\HoLogoHtml@BibTeX}
%    \begin{macrocode}
\def\HoLogoHtml@BibTeX{\HoLogoHtml@BibTeX@sf}
%    \end{macrocode}
%    \end{macro}
%
% \subsubsection{\hologo{BibTeX8}}
%
%    \begin{macro}{\HoLogo@BibTeX8}
%    \begin{macrocode}
\expandafter\def\csname HoLogo@BibTeX8\endcsname#1{%
  \hologo{BibTeX}%
  8%
}
%    \end{macrocode}
%    \end{macro}
%
%    \begin{macro}{\HoLogoBkm@BibTeX8}
%    \begin{macrocode}
\expandafter\def\csname HoLogoBkm@BibTeX8\endcsname#1{%
  \hologo{BibTeX}%
  8%
}
%    \end{macrocode}
%    \end{macro}
%    \begin{macro}{\HoLogoHtml@BibTeX8}
%    \begin{macrocode}
\expandafter
\let\csname HoLogoHtml@BibTeX8\expandafter\endcsname
\csname HoLogo@BibTeX8\endcsname
%    \end{macrocode}
%    \end{macro}
%
% \subsubsection{\hologo{ConTeXt}}
%
%    \begin{macro}{\HoLogo@ConTeXt@simple}
%    \begin{macrocode}
\def\HoLogo@ConTeXt@simple#1{%
  \HOLOGO@mbox{Con}%
  \HOLOGO@discretionary
  \HOLOGO@mbox{\hologo{TeX}t}%
}
%    \end{macrocode}
%    \end{macro}
%    \begin{macro}{\HoLogoHtml@ConTeXt@simple}
%    \begin{macrocode}
\let\HoLogoHtml@ConTeXt@simple\HoLogo@ConTeXt@simple
%    \end{macrocode}
%    \end{macro}
%
%    \begin{macro}{\HoLogo@ConTeXt@narrow}
%    This definition of logo \hologo{ConTeXt} with variant \xoption{narrow}
%    comes from TUGboat's class \xclass{ltugboat} (version 2010/11/15 v2.8).
%    \begin{macrocode}
\def\HoLogo@ConTeXt@narrow#1{%
  \HOLOGO@mbox{C\kern-.0333emon}%
  \HOLOGO@discretionary
  \kern-.0667em%
  \HOLOGO@mbox{\hologo{TeX}\kern-.0333emt}%
}
%    \end{macrocode}
%    \end{macro}
%    \begin{macro}{\HoLogoHtml@ConTeXt@narrow}
%    \begin{macrocode}
\def\HoLogoHtml@ConTeXt@narrow#1{%
  \HoLogoCss@ConTeXt@narrow
  \HOLOGO@Span{ConTeXt-narrow}{%
    \HOLOGO@Span{C}{C}%
    on%
    \hologo{TeX}%
    t%
  }%
}
%    \end{macrocode}
%    \end{macro}
%    \begin{macro}{\HoLogoCss@ConTeXt@narrow}
%    \begin{macrocode}
\def\HoLogoCss@ConTeXt@narrow{%
  \Css{%
    span.HoLogo-ConTeXt-narrow span.HoLogo-C{%
      margin-left:-.0333em;%
    }%
  }%
  \Css{%
    span.HoLogo-ConTeXt-narrow span.HoLogo-TeX{%
      margin-left:-.0667em;%
      margin-right:-.0333em;%
    }%
  }%
  \global\let\HoLogoCss@ConTeXt@narrow\relax
}
%    \end{macrocode}
%    \end{macro}
%
%    \begin{macro}{\HoLogo@ConTeXt}
%    \begin{macrocode}
\def\HoLogo@ConTeXt{\HoLogo@ConTeXt@narrow}
%    \end{macrocode}
%    \end{macro}
%    \begin{macro}{\HoLogoHtml@ConTeXt}
%    \begin{macrocode}
\def\HoLogoHtml@ConTeXt{\HoLogoHtml@ConTeXt@narrow}
%    \end{macrocode}
%    \end{macro}
%
% \subsubsection{\hologo{emTeX}}
%
%    \begin{macro}{\HoLogo@emTeX}
%    \begin{macrocode}
\def\HoLogo@emTeX#1{%
  \HOLOGO@mbox{#1{e}{E}m}%
  \HOLOGO@discretionary
  \hologo{TeX}%
}
%    \end{macrocode}
%    \end{macro}
%    \begin{macro}{\HoLogoCs@emTeX}
%    \begin{macrocode}
\def\HoLogoCs@emTeX#1{#1{e}{E}mTeX}%
%    \end{macrocode}
%    \end{macro}
%    \begin{macro}{\HoLogoBkm@emTeX}
%    \begin{macrocode}
\def\HoLogoBkm@emTeX#1{%
  #1{e}{E}m\hologo{TeX}%
}
%    \end{macrocode}
%    \end{macro}
%    \begin{macro}{\HoLogoHtml@emTeX}
%    \begin{macrocode}
\let\HoLogoHtml@emTeX\HoLogo@emTeX
%    \end{macrocode}
%    \end{macro}
%
% \subsubsection{\hologo{ExTeX}}
%
%    \begin{macro}{\HoLogo@ExTeX}
%    The definition is taken from the FAQ of the
%    project \hologo{ExTeX}
%    \cite{ExTeX-FAQ}.
%\begin{quote}
%\begin{verbatim}
%\def\ExTeX{%
%  \textrm{% Logo always with serifs
%    \ensuremath{%
%      \textstyle
%      \varepsilon_{%
%        \kern-0.15em%
%        \mathcal{X}%
%      }%
%    }%
%    \kern-.15em%
%    \TeX
%  }%
%}
%\end{verbatim}
%\end{quote}
%    \begin{macrocode}
\def\HoLogo@ExTeX#1{%
  \HoLogoFont@font{ExTeX}{rm}{%
    \ltx@mbox{%
      \HOLOGO@MathSetup
      $%
        \textstyle
        \varepsilon_{%
          \kern-0.15em%
          \HoLogoFont@font{ExTeX}{sy}{X}%
        }%
      $%
    }%
    \HOLOGO@discretionary
    \kern-.15em%
    \hologo{TeX}%
  }%
}
%    \end{macrocode}
%    \end{macro}
%    \begin{macro}{\HoLogoHtml@ExTeX}
%    \begin{macrocode}
\def\HoLogoHtml@ExTeX#1{%
  \HoLogoCss@ExTeX
  \HoLogoFont@font{ExTeX}{rm}{%
    \HOLOGO@Span{ExTeX}{%
      \ltx@mbox{%
        \HOLOGO@MathSetup
        $\textstyle\varepsilon$%
        \HOLOGO@Span{X}{$\textstyle\chi$}%
        \hologo{TeX}%
      }%
    }%
  }%
}
%    \end{macrocode}
%    \end{macro}
%    \begin{macro}{\HoLogoBkm@ExTeX}
%    \begin{macrocode}
\def\HoLogoBkm@ExTeX#1{%
  \HOLOGO@PdfdocUnicode{#1{e}{E}x}{\textepsilon\textchi}%
  \hologo{TeX}%
}
%    \end{macrocode}
%    \end{macro}
%    \begin{macro}{\HoLogoCss@ExTeX}
%    \begin{macrocode}
\def\HoLogoCss@ExTeX{%
  \Css{%
    span.HoLogo-ExTeX{%
      font-family:serif;%
    }%
  }%
  \Css{%
    span.HoLogo-ExTeX span.HoLogo-TeX{%
      margin-left:-.15em;%
    }%
  }%
  \global\let\HoLogoCss@ExTeX\relax
}
%    \end{macrocode}
%    \end{macro}
%
% \subsubsection{\hologo{MiKTeX}}
%
%    \begin{macro}{\HoLogo@MiKTeX}
%    \begin{macrocode}
\def\HoLogo@MiKTeX#1{%
  \HOLOGO@mbox{MiK}%
  \HOLOGO@discretionary
  \hologo{TeX}%
}
%    \end{macrocode}
%    \end{macro}
%    \begin{macro}{\HoLogoHtml@MiKTeX}
%    \begin{macrocode}
\let\HoLogoHtml@MiKTeX\HoLogo@MiKTeX
%    \end{macrocode}
%    \end{macro}
%
% \subsubsection{\hologo{OzTeX} and friends}
%
%    Source: \hologo{OzTeX} FAQ \cite{OzTeX}:
%    \begin{quote}
%      |\def\OzTeX{O\kern-.03em z\kern-.15em\TeX}|\\
%      (There is no kerning in OzMF, OzMP and OzTtH.)
%    \end{quote}
%
%    \begin{macro}{\HoLogo@OzTeX}
%    \begin{macrocode}
\def\HoLogo@OzTeX#1{%
  O%
  \kern-.03em %
  z%
  \kern-.15em %
  \hologo{TeX}%
}
%    \end{macrocode}
%    \end{macro}
%    \begin{macro}{\HoLogoHtml@OzTeX}
%    \begin{macrocode}
\def\HoLogoHtml@OzTeX#1{%
  \HoLogoCss@OzTeX
  \HOLOGO@Span{OzTeX}{%
    O%
    \HOLOGO@Span{z}{z}%
    \hologo{TeX}%
  }%
}
%    \end{macrocode}
%    \end{macro}
%    \begin{macro}{\HoLogoCss@OzTeX}
%    \begin{macrocode}
\def\HoLogoCss@OzTeX{%
  \Css{%
    span.HoLogo-OzTeX span.HoLogo-z{%
      margin-left:-.03em;%
      margin-right:-.15em;%
    }%
  }%
  \global\let\HoLogoCss@OzTeX\relax
}
%    \end{macrocode}
%    \end{macro}
%
%    \begin{macro}{\HoLogo@OzMF}
%    \begin{macrocode}
\def\HoLogo@OzMF#1{%
  \HOLOGO@mbox{OzMF}%
}
%    \end{macrocode}
%    \end{macro}
%    \begin{macro}{\HoLogo@OzMP}
%    \begin{macrocode}
\def\HoLogo@OzMP#1{%
  \HOLOGO@mbox{OzMP}%
}
%    \end{macrocode}
%    \end{macro}
%    \begin{macro}{\HoLogo@OzTtH}
%    \begin{macrocode}
\def\HoLogo@OzTtH#1{%
  \HOLOGO@mbox{OzTtH}%
}
%    \end{macrocode}
%    \end{macro}
%
% \subsubsection{\hologo{PCTeX}}
%
%    \begin{macro}{\HoLogo@PCTeX}
%    \begin{macrocode}
\def\HoLogo@PCTeX#1{%
  \HOLOGO@mbox{PC}%
  \hologo{TeX}%
}
%    \end{macrocode}
%    \end{macro}
%    \begin{macro}{\HoLogoHtml@PCTeX}
%    \begin{macrocode}
\let\HoLogoHtml@PCTeX\HoLogo@PCTeX
%    \end{macrocode}
%    \end{macro}
%
% \subsubsection{\hologo{PiCTeX}}
%
%    The original definitions from \xfile{pictex.tex} \cite{PiCTeX}:
%\begin{quote}
%\begin{verbatim}
%\def\PiC{%
%  P%
%  \kern-.12em%
%  \lower.5ex\hbox{I}%
%  \kern-.075em%
%  C%
%}
%\def\PiCTeX{%
%  \PiC
%  \kern-.11em%
%  \TeX
%}
%\end{verbatim}
%\end{quote}
%
%    \begin{macro}{\HoLogo@PiC}
%    \begin{macrocode}
\def\HoLogo@PiC#1{%
  P%
  \kern-.12em%
  \lower.5ex\hbox{I}%
  \kern-.075em%
  C%
  \HOLOGO@SpaceFactor
}
%    \end{macrocode}
%    \end{macro}
%    \begin{macro}{\HoLogoHtml@PiC}
%    \begin{macrocode}
\def\HoLogoHtml@PiC#1{%
  \HoLogoCss@PiC
  \HOLOGO@Span{PiC}{%
    P%
    \HOLOGO@Span{i}{I}%
    C%
  }%
}
%    \end{macrocode}
%    \end{macro}
%    \begin{macro}{\HoLogoCss@PiC}
%    \begin{macrocode}
\def\HoLogoCss@PiC{%
  \Css{%
    span.HoLogo-PiC span.HoLogo-i{%
      position:relative;%
      top:.5ex;%
      margin-left:-.12em;%
      margin-right:-.075em;%
      text-decoration:none;%
    }%
  }%
  \global\let\HoLogoCss@PiC\relax
}
%    \end{macrocode}
%    \end{macro}
%
%    \begin{macro}{\HoLogo@PiCTeX}
%    \begin{macrocode}
\def\HoLogo@PiCTeX#1{%
  \hologo{PiC}%
  \HOLOGO@discretionary
  \kern-.11em%
  \hologo{TeX}%
}
%    \end{macrocode}
%    \end{macro}
%    \begin{macro}{\HoLogoHtml@PiCTeX}
%    \begin{macrocode}
\def\HoLogoHtml@PiCTeX#1{%
  \HoLogoCss@PiCTeX
  \HOLOGO@Span{PiCTeX}{%
    \hologo{PiC}%
    \hologo{TeX}%
  }%
}
%    \end{macrocode}
%    \end{macro}
%    \begin{macro}{\HoLogoCss@PiCTeX}
%    \begin{macrocode}
\def\HoLogoCss@PiCTeX{%
  \Css{%
    span.HoLogo-PiCTeX span.HoLogo-PiC{%
      margin-right:-.11em;%
    }%
  }%
  \global\let\HoLogoCss@PiCTeX\relax
}
%    \end{macrocode}
%    \end{macro}
%
% \subsubsection{\hologo{teTeX}}
%
%    \begin{macro}{\HoLogo@teTeX}
%    \begin{macrocode}
\def\HoLogo@teTeX#1{%
  \HOLOGO@mbox{#1{t}{T}e}%
  \HOLOGO@discretionary
  \hologo{TeX}%
}
%    \end{macrocode}
%    \end{macro}
%    \begin{macro}{\HoLogoCs@teTeX}
%    \begin{macrocode}
\def\HoLogoCs@teTeX#1{#1{t}{T}dfTeX}
%    \end{macrocode}
%    \end{macro}
%    \begin{macro}{\HoLogoBkm@teTeX}
%    \begin{macrocode}
\def\HoLogoBkm@teTeX#1{%
  #1{t}{T}e\hologo{TeX}%
}
%    \end{macrocode}
%    \end{macro}
%    \begin{macro}{\HoLogoHtml@teTeX}
%    \begin{macrocode}
\let\HoLogoHtml@teTeX\HoLogo@teTeX
%    \end{macrocode}
%    \end{macro}
%
% \subsubsection{\hologo{TeX4ht}}
%
%    \begin{macro}{\HoLogo@TeX4ht}
%    \begin{macrocode}
\expandafter\def\csname HoLogo@TeX4ht\endcsname#1{%
  \HOLOGO@mbox{\hologo{TeX}4ht}%
}
%    \end{macrocode}
%    \end{macro}
%    \begin{macro}{\HoLogoHtml@TeX4ht}
%    \begin{macrocode}
\expandafter
\let\csname HoLogoHtml@TeX4ht\expandafter\endcsname
\csname HoLogo@TeX4ht\endcsname
%    \end{macrocode}
%    \end{macro}
%
%
% \subsubsection{\hologo{SageTeX}}
%
%    \begin{macro}{\HoLogo@SageTeX}
%    \begin{macrocode}
\def\HoLogo@SageTeX#1{%
  \HOLOGO@mbox{Sage}%
  \HOLOGO@discretionary
  \HOLOGO@NegativeKerning{eT,oT,To}%
  \hologo{TeX}%
}
%    \end{macrocode}
%    \end{macro}
%    \begin{macro}{\HoLogoHtml@SageTeX}
%    \begin{macrocode}
\let\HoLogoHtml@SageTeX\HoLogo@SageTeX
%    \end{macrocode}
%    \end{macro}
%
% \subsection{\hologo{METAFONT} and friends}
%
%    \begin{macro}{\HoLogo@METAFONT}
%    \begin{macrocode}
\def\HoLogo@METAFONT#1{%
  \HoLogoFont@font{METAFONT}{logo}{%
    \HOLOGO@mbox{META}%
    \HOLOGO@discretionary
    \HOLOGO@mbox{FONT}%
  }%
}
%    \end{macrocode}
%    \end{macro}
%
%    \begin{macro}{\HoLogo@METAPOST}
%    \begin{macrocode}
\def\HoLogo@METAPOST#1{%
  \HoLogoFont@font{METAPOST}{logo}{%
    \HOLOGO@mbox{META}%
    \HOLOGO@discretionary
    \HOLOGO@mbox{POST}%
  }%
}
%    \end{macrocode}
%    \end{macro}
%
%    \begin{macro}{\HoLogo@MetaFun}
%    \begin{macrocode}
\def\HoLogo@MetaFun#1{%
  \HOLOGO@mbox{Meta}%
  \HOLOGO@discretionary
  \HOLOGO@mbox{Fun}%
}
%    \end{macrocode}
%    \end{macro}
%
%    \begin{macro}{\HoLogo@MetaPost}
%    \begin{macrocode}
\def\HoLogo@MetaPost#1{%
  \HOLOGO@mbox{Meta}%
  \HOLOGO@discretionary
  \HOLOGO@mbox{Post}%
}
%    \end{macrocode}
%    \end{macro}
%
% \subsection{Others}
%
% \subsubsection{\hologo{biber}}
%
%    \begin{macro}{\HoLogo@biber}
%    \begin{macrocode}
\def\HoLogo@biber#1{%
  \HOLOGO@mbox{#1{b}{B}i}%
  \HOLOGO@discretionary
  \HOLOGO@mbox{ber}%
}
%    \end{macrocode}
%    \end{macro}
%    \begin{macro}{\HoLogoCs@biber}
%    \begin{macrocode}
\def\HoLogoCs@biber#1{#1{b}{B}iber}
%    \end{macrocode}
%    \end{macro}
%    \begin{macro}{\HoLogoBkm@biber}
%    \begin{macrocode}
\def\HoLogoBkm@biber#1{%
  #1{b}{B}iber%
}
%    \end{macrocode}
%    \end{macro}
%    \begin{macro}{\HoLogoHtml@biber}
%    \begin{macrocode}
\let\HoLogoHtml@biber\HoLogo@biber
%    \end{macrocode}
%    \end{macro}
%
% \subsubsection{\hologo{KOMAScript}}
%
%    \begin{macro}{\HoLogo@KOMAScript}
%    The definition for \hologo{KOMAScript} is taken
%    from \hologo{KOMAScript} (\xfile{scrlogo.dtx}, reformatted) \cite{scrlogo}:
%\begin{quote}
%\begin{verbatim}
%\@ifundefined{KOMAScript}{%
%  \DeclareRobustCommand{\KOMAScript}{%
%    \textsf{%
%      K\kern.05em O\kern.05emM\kern.05em A%
%      \kern.1em-\kern.1em %
%      Script%
%    }%
%  }%
%}{}
%\end{verbatim}
%\end{quote}
%    \begin{macrocode}
\def\HoLogo@KOMAScript#1{%
  \HoLogoFont@font{KOMAScript}{sf}{%
    \HOLOGO@mbox{%
      K\kern.05em%
      O\kern.05em%
      M\kern.05em%
      A%
    }%
    \kern.1em%
    \HOLOGO@hyphen
    \kern.1em%
    \HOLOGO@mbox{Script}%
  }%
}
%    \end{macrocode}
%    \end{macro}
%    \begin{macro}{\HoLogoBkm@KOMAScript}
%    \begin{macrocode}
\def\HoLogoBkm@KOMAScript#1{%
  KOMA-Script%
}
%    \end{macrocode}
%    \end{macro}
%    \begin{macro}{\HoLogoHtml@KOMAScript}
%    \begin{macrocode}
\def\HoLogoHtml@KOMAScript#1{%
  \HoLogoCss@KOMAScript
  \HoLogoFont@font{KOMAScript}{sf}{%
    \HOLOGO@Span{KOMAScript}{%
      K%
      \HOLOGO@Span{O}{O}%
      M%
      \HOLOGO@Span{A}{A}%
      \HOLOGO@Span{hyphen}{-}%
      Script%
    }%
  }%
}
%    \end{macrocode}
%    \end{macro}
%    \begin{macro}{\HoLogoCss@KOMAScript}
%    \begin{macrocode}
\def\HoLogoCss@KOMAScript{%
  \Css{%
    span.HoLogo-KOMAScript{%
      font-family:sans-serif;%
    }%
  }%
  \Css{%
    span.HoLogo-KOMAScript span.HoLogo-O{%
      padding-left:.05em;%
      padding-right:.05em;%
    }%
  }%
  \Css{%
    span.HoLogo-KOMAScript span.HoLogo-A{%
      padding-left:.05em;%
    }%
  }%
  \Css{%
    span.HoLogo-KOMAScript span.HoLogo-hyphen{%
      padding-left:.1em;%
      padding-right:.1em;%
    }%
  }%
  \global\let\HoLogoCss@KOMAScript\relax
}
%    \end{macrocode}
%    \end{macro}
%
% \subsubsection{\hologo{LyX}}
%
%    \begin{macro}{\HoLogo@LyX}
%    The definition is taken from the documentation source files
%    of \hologo{LyX}, \xfile{Intro.lyx} \cite{LyX}:
%\begin{quote}
%\begin{verbatim}
%\def\LyX{%
%  \texorpdfstring{%
%    L\kern-.1667em\lower.25em\hbox{Y}\kern-.125emX\@%
%  }{%
%    LyX%
%  }%
%}
%\end{verbatim}
%\end{quote}
%    \begin{macrocode}
\def\HoLogo@LyX#1{%
  L%
  \kern-.1667em%
  \lower.25em\hbox{Y}%
  \kern-.125em%
  X%
  \HOLOGO@SpaceFactor
}
%    \end{macrocode}
%    \end{macro}
%    \begin{macro}{\HoLogoHtml@LyX}
%    \begin{macrocode}
\def\HoLogoHtml@LyX#1{%
  \HoLogoCss@LyX
  \HOLOGO@Span{LyX}{%
    L%
    \HOLOGO@Span{y}{Y}%
    X%
  }%
}
%    \end{macrocode}
%    \end{macro}
%    \begin{macro}{\HoLogoCss@LyX}
%    \begin{macrocode}
\def\HoLogoCss@LyX{%
  \Css{%
    span.HoLogo-LyX span.HoLogo-y{%
      position:relative;%
      top:.25em;%
      margin-left:-.1667em;%
      margin-right:-.125em;%
      text-decoration:none;%
    }%
  }%
  \global\let\HoLogoCss@LyX\relax
}
%    \end{macrocode}
%    \end{macro}
%
% \subsubsection{\hologo{NTS}}
%
%    \begin{macro}{\HoLogo@NTS}
%    Definition for \hologo{NTS} can be found in
%    package \xpackage{etex\textunderscore man} for the \hologo{eTeX} manual \cite{etexman}
%    and in package \xpackage{dtklogos} \cite{dtklogos}:
%\begin{quote}
%\begin{verbatim}
%\def\NTS{%
%  \leavevmode
%  \hbox{%
%    $%
%      \cal N%
%      \kern-0.35em%
%      \lower0.5ex\hbox{$\cal T$}%
%      \kern-0.2em%
%      S%
%    $%
%  }%
%}
%\end{verbatim}
%\end{quote}
%    \begin{macrocode}
\def\HoLogo@NTS#1{%
  \HoLogoFont@font{NTS}{sy}{%
    N\/%
    \kern-.35em%
    \lower.5ex\hbox{T\/}%
    \kern-.2em%
    S\/%
  }%
  \HOLOGO@SpaceFactor
}
%    \end{macrocode}
%    \end{macro}
%
% \subsubsection{\Hologo{TTH} (\hologo{TeX} to HTML translator)}
%
%    Source: \url{http://hutchinson.belmont.ma.us/tth/}
%    In the HTML source the second `T' is printed as subscript.
%\begin{quote}
%\begin{verbatim}
%T<sub>T</sub>H
%\end{verbatim}
%\end{quote}
%    \begin{macro}{\HoLogo@TTH}
%    \begin{macrocode}
\def\HoLogo@TTH#1{%
  \ltx@mbox{%
    T\HOLOGO@SubScript{T}H%
  }%
  \HOLOGO@SpaceFactor
}
%    \end{macrocode}
%    \end{macro}
%
%    \begin{macro}{\HoLogoHtml@TTH}
%    \begin{macrocode}
\def\HoLogoHtml@TTH#1{%
  T\HCode{<sub>}T\HCode{</sub>}H%
}
%    \end{macrocode}
%    \end{macro}
%
% \subsubsection{\Hologo{HanTheThanh}}
%
%    Partial source: Package \xpackage{dtklogos}.
%    The double accent is U+1EBF (latin small letter e with circumflex
%    and acute).
%    \begin{macro}{\HoLogo@HanTheThanh}
%    \begin{macrocode}
\def\HoLogo@HanTheThanh#1{%
  \ltx@mbox{H\`an}%
  \HOLOGO@space
  \ltx@mbox{%
    Th%
    \HOLOGO@IfCharExists{"1EBF}{%
      \char"1EBF\relax
    }{%
      \^e\hbox to 0pt{\hss\raise .5ex\hbox{\'{}}}%
    }%
  }%
  \HOLOGO@space
  \ltx@mbox{Th\`anh}%
}
%    \end{macrocode}
%    \end{macro}
%    \begin{macro}{\HoLogoBkm@HanTheThanh}
%    \begin{macrocode}
\def\HoLogoBkm@HanTheThanh#1{%
  H\`an %
  Th\HOLOGO@PdfdocUnicode{\^e}{\9036\277} %
  Th\`anh%
}
%    \end{macrocode}
%    \end{macro}
%    \begin{macro}{\HoLogoHtml@HanTheThanh}
%    \begin{macrocode}
\def\HoLogoHtml@HanTheThanh#1{%
  H\`an %
  Th\HCode{&\ltx@hashchar x1ebf;} %
  Th\`anh%
}
%    \end{macrocode}
%    \end{macro}
%
% \subsection{Driver detection}
%
%    \begin{macrocode}
\HOLOGO@IfExists\InputIfFileExists{%
  \InputIfFileExists{hologo.cfg}{}{}%
}{%
  \ltx@IfUndefined{pdf@filesize}{%
    \def\HOLOGO@InputIfExists{%
      \openin\HOLOGO@temp=hologo.cfg\relax
      \ifeof\HOLOGO@temp
        \closein\HOLOGO@temp
      \else
        \closein\HOLOGO@temp
        \begingroup
          \def\x{LaTeX2e}%
        \expandafter\endgroup
        \ifx\fmtname\x
          % \iffalse meta-comment
%
% File: hologo.dtx
% Version: 2016/05/12 v1.11
% Info: A logo collection with bookmark support
%
% Copyright (C) 2010-2012 by
%    Heiko Oberdiek <heiko.oberdiek at googlemail.com>
%
% This work may be distributed and/or modified under the
% conditions of the LaTeX Project Public License, either
% version 1.3c of this license or (at your option) any later
% version. This version of this license is in
%    http://www.latex-project.org/lppl/lppl-1-3c.txt
% and the latest version of this license is in
%    http://www.latex-project.org/lppl.txt
% and version 1.3 or later is part of all distributions of
% LaTeX version 2005/12/01 or later.
%
% This work has the LPPL maintenance status "maintained".
%
% This Current Maintainer of this work is Heiko Oberdiek.
%
% The Base Interpreter refers to any `TeX-Format',
% because some files are installed in TDS:tex/generic//.
%
% This work consists of the main source file hologo.dtx
% and the derived files
%    hologo.sty, hologo.pdf, hologo.ins, hologo.drv, hologo-example.tex,
%    hologo-test1.tex, hologo-test-spacefactor.tex,
%    hologo-test-list.tex.
%
% Distribution:
%    CTAN:macros/latex/contrib/oberdiek/hologo.dtx
%    CTAN:macros/latex/contrib/oberdiek/hologo.pdf
%
% Unpacking:
%    (a) If hologo.ins is present:
%           tex hologo.ins
%    (b) Without hologo.ins:
%           tex hologo.dtx
%    (c) If you insist on using LaTeX
%           latex \let\install=y% \iffalse meta-comment
%
% File: hologo.dtx
% Version: 2016/05/12 v1.11
% Info: A logo collection with bookmark support
%
% Copyright (C) 2010-2012 by
%    Heiko Oberdiek <heiko.oberdiek at googlemail.com>
%
% This work may be distributed and/or modified under the
% conditions of the LaTeX Project Public License, either
% version 1.3c of this license or (at your option) any later
% version. This version of this license is in
%    http://www.latex-project.org/lppl/lppl-1-3c.txt
% and the latest version of this license is in
%    http://www.latex-project.org/lppl.txt
% and version 1.3 or later is part of all distributions of
% LaTeX version 2005/12/01 or later.
%
% This work has the LPPL maintenance status "maintained".
%
% This Current Maintainer of this work is Heiko Oberdiek.
%
% The Base Interpreter refers to any `TeX-Format',
% because some files are installed in TDS:tex/generic//.
%
% This work consists of the main source file hologo.dtx
% and the derived files
%    hologo.sty, hologo.pdf, hologo.ins, hologo.drv, hologo-example.tex,
%    hologo-test1.tex, hologo-test-spacefactor.tex,
%    hologo-test-list.tex.
%
% Distribution:
%    CTAN:macros/latex/contrib/oberdiek/hologo.dtx
%    CTAN:macros/latex/contrib/oberdiek/hologo.pdf
%
% Unpacking:
%    (a) If hologo.ins is present:
%           tex hologo.ins
%    (b) Without hologo.ins:
%           tex hologo.dtx
%    (c) If you insist on using LaTeX
%           latex \let\install=y\input{hologo.dtx}
%        (quote the arguments according to the demands of your shell)
%
% Documentation:
%    (a) If hologo.drv is present:
%           latex hologo.drv
%    (b) Without hologo.drv:
%           latex hologo.dtx; ...
%    The class ltxdoc loads the configuration file ltxdoc.cfg
%    if available. Here you can specify further options, e.g.
%    use A4 as paper format:
%       \PassOptionsToClass{a4paper}{article}
%
%    Programm calls to get the documentation (example):
%       pdflatex hologo.dtx
%       makeindex -s gind.ist hologo.idx
%       pdflatex hologo.dtx
%       makeindex -s gind.ist hologo.idx
%       pdflatex hologo.dtx
%
% Installation:
%    TDS:tex/generic/oberdiek/hologo.sty
%    TDS:doc/latex/oberdiek/hologo.pdf
%    TDS:doc/latex/oberdiek/example/hologo-example.tex
%    TDS:doc/latex/oberdiek/test/hologo-test1.tex
%    TDS:doc/latex/oberdiek/test/hologo-test-spacefactor.tex
%    TDS:doc/latex/oberdiek/test/hologo-test-list.tex
%    TDS:source/latex/oberdiek/hologo.dtx
%
%<*ignore>
\begingroup
  \catcode123=1 %
  \catcode125=2 %
  \def\x{LaTeX2e}%
\expandafter\endgroup
\ifcase 0\ifx\install y1\fi\expandafter
         \ifx\csname processbatchFile\endcsname\relax\else1\fi
         \ifx\fmtname\x\else 1\fi\relax
\else\csname fi\endcsname
%</ignore>
%<*install>
\input docstrip.tex
\Msg{************************************************************************}
\Msg{* Installation}
\Msg{* Package: hologo 2016/05/12 v1.11 A logo collection with bookmark support (HO)}
\Msg{************************************************************************}

\keepsilent
\askforoverwritefalse

\let\MetaPrefix\relax
\preamble

This is a generated file.

Project: hologo
Version: 2016/05/12 v1.11

Copyright (C) 2010-2012 by
   Heiko Oberdiek <heiko.oberdiek at googlemail.com>

This work may be distributed and/or modified under the
conditions of the LaTeX Project Public License, either
version 1.3c of this license or (at your option) any later
version. This version of this license is in
   http://www.latex-project.org/lppl/lppl-1-3c.txt
and the latest version of this license is in
   http://www.latex-project.org/lppl.txt
and version 1.3 or later is part of all distributions of
LaTeX version 2005/12/01 or later.

This work has the LPPL maintenance status "maintained".

This Current Maintainer of this work is Heiko Oberdiek.

The Base Interpreter refers to any `TeX-Format',
because some files are installed in TDS:tex/generic//.

This work consists of the main source file hologo.dtx
and the derived files
   hologo.sty, hologo.pdf, hologo.ins, hologo.drv, hologo-example.tex,
   hologo-test1.tex, hologo-test-spacefactor.tex,
   hologo-test-list.tex.

\endpreamble
\let\MetaPrefix\DoubleperCent

\generate{%
  \file{hologo.ins}{\from{hologo.dtx}{install}}%
  \file{hologo.drv}{\from{hologo.dtx}{driver}}%
  \usedir{tex/generic/oberdiek}%
  \file{hologo.sty}{\from{hologo.dtx}{package}}%
  \usedir{doc/latex/oberdiek/example}%
  \file{hologo-example.tex}{\from{hologo.dtx}{example}}%
  \usedir{doc/latex/oberdiek/test}%
  \file{hologo-test1.tex}{\from{hologo.dtx}{test1}}%
  \file{hologo-test-spacefactor.tex}{\from{hologo.dtx}{test-spacefactor}}%
  \file{hologo-test-list.tex}{\from{hologo.dtx}{test-list}}%
  \nopreamble
  \nopostamble
  \usedir{source/latex/oberdiek/catalogue}%
  \file{hologo.xml}{\from{hologo.dtx}{catalogue}}%
}

\catcode32=13\relax% active space
\let =\space%
\Msg{************************************************************************}
\Msg{*}
\Msg{* To finish the installation you have to move the following}
\Msg{* file into a directory searched by TeX:}
\Msg{*}
\Msg{*     hologo.sty}
\Msg{*}
\Msg{* To produce the documentation run the file `hologo.drv'}
\Msg{* through LaTeX.}
\Msg{*}
\Msg{* Happy TeXing!}
\Msg{*}
\Msg{************************************************************************}

\endbatchfile
%</install>
%<*ignore>
\fi
%</ignore>
%<*driver>
\NeedsTeXFormat{LaTeX2e}
\ProvidesFile{hologo.drv}%
  [2016/05/12 v1.11 A logo collection with bookmark support (HO)]%
\documentclass{ltxdoc}
\usepackage{holtxdoc}[2011/11/22]
\usepackage{hologo}[2016/05/12]
\usepackage{longtable}
\usepackage{array}
\usepackage{paralist}
%\usepackage[T1]{fontenc}
%\usepackage{lmodern}
\begin{document}
  \DocInput{hologo.dtx}%
\end{document}
%</driver>
% \fi
%
%
% \CharacterTable
%  {Upper-case    \A\B\C\D\E\F\G\H\I\J\K\L\M\N\O\P\Q\R\S\T\U\V\W\X\Y\Z
%   Lower-case    \a\b\c\d\e\f\g\h\i\j\k\l\m\n\o\p\q\r\s\t\u\v\w\x\y\z
%   Digits        \0\1\2\3\4\5\6\7\8\9
%   Exclamation   \!     Double quote  \"     Hash (number) \#
%   Dollar        \$     Percent       \%     Ampersand     \&
%   Acute accent  \'     Left paren    \(     Right paren   \)
%   Asterisk      \*     Plus          \+     Comma         \,
%   Minus         \-     Point         \.     Solidus       \/
%   Colon         \:     Semicolon     \;     Less than     \<
%   Equals        \=     Greater than  \>     Question mark \?
%   Commercial at \@     Left bracket  \[     Backslash     \\
%   Right bracket \]     Circumflex    \^     Underscore    \_
%   Grave accent  \`     Left brace    \{     Vertical bar  \|
%   Right brace   \}     Tilde         \~}
%
% \GetFileInfo{hologo.drv}
%
% \title{The \xpackage{hologo} package}
% \date{2016/05/12 v1.11}
% \author{Heiko Oberdiek\\\xemail{heiko.oberdiek at googlemail.com}}
%
% \maketitle
%
% \begin{abstract}
% This package starts a collection of logos with support for bookmarks
% strings.
% \end{abstract}
%
% \tableofcontents
%
% \section{Documentation}
%
% \subsection{Logo macros}
%
% \begin{declcs}{hologo} \M{name}
% \end{declcs}
% Macro \cs{hologo} sets the logo with name \meta{name}.
% The following table shows the supported names.
%
% \begingroup
%   \def\hologoEntry#1#2#3{^^A
%     #1&#2&\hologoLogoSetup{#1}{variant=#2}\hologo{#1}&#3\tabularnewline
%   }
%   \begin{longtable}{>{\ttfamily}l>{\ttfamily}lll}
%     \rmfamily\bfseries{name} & \rmfamily\bfseries variant
%     & \bfseries logo & \bfseries since\\
%     \hline
%     \endhead
%     \hologoList
%   \end{longtable}
% \endgroup
%
% \begin{declcs}{Hologo} \M{name}
% \end{declcs}
% Macro \cs{Hologo} starts the logo \meta{name} with an uppercase
% letter. As an exception small greek letters are not converted
% to uppercase. Examples, see \hologo{eTeX} and \hologo{ExTeX}.
%
% \subsection{Setup macros}
%
% The package does not support package options, but the following
% setup macros can be used to set options.
%
% \begin{declcs}{hologoSetup} \M{key value list}
% \end{declcs}
% Macro \cs{hologoSetup} sets global options.
%
% \begin{declcs}{hologoLogoSetup} \M{logo} \M{key value list}
% \end{declcs}
% Some options can also be used to configure a logo.
% These settings take precedence over global option settings.
%
% \subsection{Options}\label{sec:options}
%
% There are boolean and string options:
% \begin{description}
% \item[Boolean option:]
% It takes |true| or |false|
% as value. If the value is omitted, then |true| is used.
% \item[String option:]
% A value must be given as string. (But the string might be empty.)
% \end{description}
% The following options can be used both in \cs{hologoSetup}
% and \cs{hologoLogoSetup}:
% \begin{description}
% \def\entry#1{\item[\xoption{#1}:]}
% \entry{break}
%   enables or disables line breaks inside the logo. This setting is
%   refined by options \xoption{hyphenbreak}, \xoption{spacebreak}
%   or \xoption{discretionarybreak}.
%   Default is |false|.
% \entry{hyphenbreak}
%   enables or disables the line break right after the hyphen character.
% \entry{spacebreak}
%   enables or disables line breaks at space characters.
% \entry{discretionarybreak}
%   enables or disables line breaks at hyphenation points
%   (inserted by \cs{-}).
% \end{description}
% Macro \cs{hologoLogoSetup} also knows:
% \begin{description}
% \item[\xoption{variant}:]
%   This is a string option. It specifies a variant of a logo that
%   must exist. An empty string selects the package default variant.
% \end{description}
% Example:
% \begin{quote}
%   |\hologoSetup{break=false}|\\
%   |\hologoLogoSetup{plainTeX}{variant=hyphen,hyphenbreak}|\\
%   Then ``plain-\TeX'' contains one break point after the hyphen.
% \end{quote}
%
% \subsection{Driver options}
%
% Sometimes graphical operations are needed to construct some
% glyphs (e.g.\ \hologo{XeTeX}). If package \xpackage{graphics}
% or package \xpackage{pgf} are found, then the macros are taken
% from there. Otherwise the packge defines its own operations
% and therefore needs the driver information. Many drivers are
% detected automatically (\hologo{pdfTeX}/\hologo{LuaTeX}
% in PDF mode, \hologo{XeTeX}, \hologo{VTeX}). These have precedence
% over a driver option. The driver can be given as package option
% or using \cs{hologoDriverSetup}.
% The following list contains the recognized driver options:
% \begin{itemize}
% \item \xoption{pdftex}, \xoption{luatex}
% \item \xoption{dvipdfm}, \xoption{dvipdfmx}
% \item \xoption{dvips}, \xoption{dvipsone}, \xoption{xdvi}
% \item \xoption{xetex}
% \item \xoption{vtex}
% \end{itemize}
% The left driver of a line is the driver name that is used internally.
% The following names are aliases for drivers that use the
% same method. Therefore the entry in the \xext{log} file for
% the used driver prints the internally used driver name.
% \begin{description}
% \item[\xoption{driverfallback}:]
%   This option expects a driver that is used,
%   if the driver could not be detected automatically.
% \end{description}
%
% \begin{declcs}{hologoDriverSetup} \M{driver option}
% \end{declcs}
% The driver can also be configured after package loading
% using \cs{hologoDriverSetup}, also the way for \hologo{plainTeX}
% to setup the driver.
%
% \subsection{Font setup}
%
% Some logos require a special font, but should also be usable by
% \hologo{plainTeX}. Therefore the package provides some ways
% to influence the font settings. The options below
% take font settings as values. Both font commands
% such as \cs{sffamily} and macros that take one argument
% like \cs{textsf} can be used.
%
% \begin{declcs}{hologoFontSetup} \M{key value list}
% \end{declcs}
% Macro \cs{hologoFontSetup} sets the fonts for all logos.
% Supported keys:
% \begin{description}
% \def\entry#1{\item[\xoption{#1}:]}
% \entry{general}
%   This font is used for all logos. The default is empty.
%   That means no special font is used.
% \entry{bibsf}
%   This font is used for
%   {\hologoLogoSetup{BibTeX}{variant=sf}\hologo{BibTeX}}
%   with variant \xoption{sf}.
% \entry{rm}
%   This font is a serif font. It is used for \hologo{ExTeX}.
% \entry{sc}
%   This font specifies a small caps font. It is used for
%   {\hologoLogoSetup{BibTeX}{variant=sc}\hologo{BibTeX}}
%   with variant \xoption{sc}.
% \entry{sf}
%   This font specifies a sans serif font. The default
%   is \cs{sffamily}, then \cs{sf} is tried. Otherwise
%   a warning is given. It is used by \hologo{KOMAScript}.
% \entry{sy}
%   This is the font for math symbols (e.g. cmsy).
%   It is used by \hologo{AmS}, \hologo{NTS}, \hologo{ExTeX}.
% \entry{logo}
%   \hologo{METAFONT} and \hologo{METAPOST} are using that font.
%   In \hologo{LaTeX} \cs{logofamily} is used and
%   the definitions of package \xpackage{mflogo} are used
%   if the package is not loaded.
%   Otherwise the \cs{tenlogo} is used and defined
%   if it does not already exists.
% \end{description}
%
% \begin{declcs}{hologoLogoFontSetup} \M{logo} \M{key value list}
% \end{declcs}
% Fonts can also be set for a logo or logo component separately,
% see the following list.
% The keys are the same as for \cs{hologoFontSetup}.
%
% \begin{longtable}{>{\ttfamily}l>{\sffamily}ll}
%   \meta{logo} & keys & result\\
%   \hline
%   \endhead
%   BibTeX & bibsf & {\hologoLogoSetup{BibTeX}{variant=sf}\hologo{BibTeX}}\\[.5ex]
%   BibTeX & sc & {\hologoLogoSetup{BibTeX}{variant=sc}\hologo{BibTeX}}\\[.5ex]
%   ExTeX & rm & \hologo{ExTeX}\\
%   SliTeX & rm & \hologo{SliTeX}\\[.5ex]
%   AmS & sy & \hologo{AmS}\\
%   ExTeX & sy & \hologo{ExTeX}\\
%   NTS & sy & \hologo{NTS}\\[.5ex]
%   KOMAScript & sf & \hologo{KOMAScript}\\[.5ex]
%   METAFONT & logo & \hologo{METAFONT}\\
%   METAPOST & logo & \hologo{METAPOST}\\[.5ex]
%   SliTeX & sc \hologo{SliTeX}
% \end{longtable}
%
% \subsubsection{Font order}
%
% For all logos the font \xoption{general} is applied first.
% Example:
%\begin{quote}
%|\hologoFontSetup{general=\color{red}}|
%\end{quote}
% will print red logos.
% Then if the font uses a special font \xoption{sf}, for example,
% the font is applied that is setup by \cs{hologoLogoFontSetup}.
% If this font is not setup, then the common font setup
% by \cs{hologoFontSetup} is used. Otherwise a warning is given,
% that there is no font configured.
%
% \subsection{Additional user macros}
%
% Usually a variant of a logo is configured by using
% \cs{hologoLogoSetup}, because it is bad style to mix
% different variants of the same logo in the same text.
% There the following macros are a convenience for testing.
%
% \begin{declcs}{hologoVariant} \M{name} \M{variant}\\
%   \cs{HologoVariant} \M{name} \M{variant}
% \end{declcs}
% Logo \meta{name} is set using \meta{variant} that specifies
% explicitely which variant of the macro is used. If the argument
% is empty, then the default form of the logo is used
% (configurable by \cs{hologoLogoSetup}).
%
% \cs{HologoVariant} is used if the logo is set in a context
% that needs an uppercase first letter (beginning of a sentence, \dots).
%
% \begin{declcs}{hologoList}\\
%   \cs{hologoEntry} \M{logo} \M{variant} \M{since}
% \end{declcs}
% Macro \cs{hologoList} contains all logos that are provided
% by the package including variants. The list consists of calls
% of \cs{hologoEntry} with three arguments starting with the
% logo name \meta{logo} and its variant \meta{variant}. An empty
% variant means the current default. Argument \meta{since} specifies
% with version of the package \xpackage{hologo} is needed to get
% the logo. If the logo is fixed, then the date gets updated.
% Therefore the date \meta{since} is not exactly the date of
% the first introduction, but rather the date of the latest fix.
%
% Before \cs{hologoList} can be used, macro \cs{hologoEntry} needs
% a definition. The example file in section \ref{sec:example}
% shows applications of \cs{hologoList}.
%
% \subsection{Supported contexts}
%
% Macros \cs{hologo} and friends support special contexts:
% \begin{itemize}
% \item \hologo{LaTeX}'s protection mechanism.
% \item Bookmarks of package \xpackage{hyperref}.
% \item Package \xpackage{tex4ht}.
% \item The macros can be used inside \cs{csname} constructs,
%   if \cs{ifincsname} is available (\hologo{pdfTeX}, \hologo{XeTeX},
%   \hologo{LuaTeX}).
% \end{itemize}
%
% \subsection{Example}
% \label{sec:example}
%
% The following example prints the logos in different fonts.
%    \begin{macrocode}
%<*example>
%<<verbatim
\NeedsTeXFormat{LaTeX2e}
\documentclass[a4paper]{article}
\usepackage[
  hmargin=20mm,
  vmargin=20mm,
]{geometry}
\pagestyle{empty}
\usepackage{hologo}[2016/05/12]
\usepackage{longtable}
\usepackage{array}
\setlength{\extrarowheight}{2pt}
\usepackage[T1]{fontenc}
\usepackage{lmodern}
\usepackage{pdflscape}
\usepackage[
  pdfencoding=auto,
]{hyperref}
\hypersetup{
  pdfauthor={Heiko Oberdiek},
  pdftitle={Example for package `hologo'},
  pdfsubject={Logos with fonts lmr, lmss, qtm, qpl, qhv},
}
\usepackage{bookmark}

% Print the logo list on the console

\begingroup
  \typeout{}%
  \typeout{*** Begin of logo list ***}%
  \newcommand*{\hologoEntry}[3]{%
    \typeout{#1 \ifx\\#2\\\else(#2) \fi[#3]}%
  }%
  \hologoList
  \typeout{*** End of logo list ***}%
  \typeout{}%
\endgroup

\begin{document}
\begin{landscape}

  \section{Example file for package `hologo'}

  % Table for font names

  \begin{longtable}{>{\bfseries}ll}
    \textbf{font} & \textbf{Font name}\\
    \hline
    lmr & Latin Modern Roman\\
    lmss & Latin Modern Sans\\
    qtm & \TeX\ Gyre Termes\\
    qhv & \TeX\ Gyre Heros\\
    qpl & \TeX\ Gyre Pagella\\
  \end{longtable}

  % Logo list with logos in different fonts

  \begingroup
    \newcommand*{\SetVariant}[2]{%
      \ifx\\#2\\%
      \else
        \hologoLogoSetup{#1}{variant=#2}%
      \fi
    }%
    \newcommand*{\hologoEntry}[3]{%
      \SetVariant{#1}{#2}%
      \raisebox{1em}[0pt][0pt]{\hypertarget{#1@#2}{}}%
      \bookmark[%
        dest={#1@#2},%
      ]{%
        #1\ifx\\#2\\\else\space(#2)\fi: \Hologo{#1}, \hologo{#1} %
        [Unicode]%
      }%
      \hypersetup{unicode=false}%
      \bookmark[%
        dest={#1@#2},%
      ]{%
        #1\ifx\\#2\\\else\space(#2)\fi: \Hologo{#1}, \hologo{#1} %
        [PDFDocEncoding]%
      }%
      \texttt{#1}%
      &%
      \texttt{#2}%
      &%
      \Hologo{#1}%
      &%
      \SetVariant{#1}{#2}%
      \hologo{#1}%
      &%
      \SetVariant{#1}{#2}%
      \fontfamily{qtm}\selectfont
      \hologo{#1}%
      &%
      \SetVariant{#1}{#2}%
      \fontfamily{qpl}\selectfont
      \hologo{#1}%
      &%
      \SetVariant{#1}{#2}%
      \textsf{\hologo{#1}}%
      &%
      \SetVariant{#1}{#2}%
      \fontfamily{qhv}\selectfont
      \hologo{#1}%
      \tabularnewline
    }%
    \begin{longtable}{llllllll}%
      \textbf{\textit{logo}} & \textbf{\textit{variant}} &
      \texttt{\string\Hologo} &
      \textbf{lmr} & \textbf{qtm} & \textbf{qpl} &
      \textbf{lmss} & \textbf{qhv}
      \tabularnewline
      \hline
      \endhead
      \hologoList
    \end{longtable}%
  \endgroup

\end{landscape}
\end{document}
%verbatim
%</example>
%    \end{macrocode}
%
% \StopEventually{
% }
%
% \section{Implementation}
%    \begin{macrocode}
%<*package>
%    \end{macrocode}
%    Reload check, especially if the package is not used with \LaTeX.
%    \begin{macrocode}
\begingroup\catcode61\catcode48\catcode32=10\relax%
  \catcode13=5 % ^^M
  \endlinechar=13 %
  \catcode35=6 % #
  \catcode39=12 % '
  \catcode44=12 % ,
  \catcode45=12 % -
  \catcode46=12 % .
  \catcode58=12 % :
  \catcode64=11 % @
  \catcode123=1 % {
  \catcode125=2 % }
  \expandafter\let\expandafter\x\csname ver@hologo.sty\endcsname
  \ifx\x\relax % plain-TeX, first loading
  \else
    \def\empty{}%
    \ifx\x\empty % LaTeX, first loading,
      % variable is initialized, but \ProvidesPackage not yet seen
    \else
      \expandafter\ifx\csname PackageInfo\endcsname\relax
        \def\x#1#2{%
          \immediate\write-1{Package #1 Info: #2.}%
        }%
      \else
        \def\x#1#2{\PackageInfo{#1}{#2, stopped}}%
      \fi
      \x{hologo}{The package is already loaded}%
      \aftergroup\endinput
    \fi
  \fi
\endgroup%
%    \end{macrocode}
%    Package identification:
%    \begin{macrocode}
\begingroup\catcode61\catcode48\catcode32=10\relax%
  \catcode13=5 % ^^M
  \endlinechar=13 %
  \catcode35=6 % #
  \catcode39=12 % '
  \catcode40=12 % (
  \catcode41=12 % )
  \catcode44=12 % ,
  \catcode45=12 % -
  \catcode46=12 % .
  \catcode47=12 % /
  \catcode58=12 % :
  \catcode64=11 % @
  \catcode91=12 % [
  \catcode93=12 % ]
  \catcode123=1 % {
  \catcode125=2 % }
  \expandafter\ifx\csname ProvidesPackage\endcsname\relax
    \def\x#1#2#3[#4]{\endgroup
      \immediate\write-1{Package: #3 #4}%
      \xdef#1{#4}%
    }%
  \else
    \def\x#1#2[#3]{\endgroup
      #2[{#3}]%
      \ifx#1\@undefined
        \xdef#1{#3}%
      \fi
      \ifx#1\relax
        \xdef#1{#3}%
      \fi
    }%
  \fi
\expandafter\x\csname ver@hologo.sty\endcsname
\ProvidesPackage{hologo}%
  [2016/05/12 v1.11 A logo collection with bookmark support (HO)]%
%    \end{macrocode}
%
%    \begin{macrocode}
\begingroup\catcode61\catcode48\catcode32=10\relax%
  \catcode13=5 % ^^M
  \endlinechar=13 %
  \catcode123=1 % {
  \catcode125=2 % }
  \catcode64=11 % @
  \def\x{\endgroup
    \expandafter\edef\csname HOLOGO@AtEnd\endcsname{%
      \endlinechar=\the\endlinechar\relax
      \catcode13=\the\catcode13\relax
      \catcode32=\the\catcode32\relax
      \catcode35=\the\catcode35\relax
      \catcode61=\the\catcode61\relax
      \catcode64=\the\catcode64\relax
      \catcode123=\the\catcode123\relax
      \catcode125=\the\catcode125\relax
    }%
  }%
\x\catcode61\catcode48\catcode32=10\relax%
\catcode13=5 % ^^M
\endlinechar=13 %
\catcode35=6 % #
\catcode64=11 % @
\catcode123=1 % {
\catcode125=2 % }
\def\TMP@EnsureCode#1#2{%
  \edef\HOLOGO@AtEnd{%
    \HOLOGO@AtEnd
    \catcode#1=\the\catcode#1\relax
  }%
  \catcode#1=#2\relax
}
\TMP@EnsureCode{10}{12}% ^^J
\TMP@EnsureCode{33}{12}% !
\TMP@EnsureCode{34}{12}% "
\TMP@EnsureCode{36}{3}% $
\TMP@EnsureCode{38}{4}% &
\TMP@EnsureCode{39}{12}% '
\TMP@EnsureCode{40}{12}% (
\TMP@EnsureCode{41}{12}% )
\TMP@EnsureCode{42}{12}% *
\TMP@EnsureCode{43}{12}% +
\TMP@EnsureCode{44}{12}% ,
\TMP@EnsureCode{45}{12}% -
\TMP@EnsureCode{46}{12}% .
\TMP@EnsureCode{47}{12}% /
\TMP@EnsureCode{58}{12}% :
\TMP@EnsureCode{59}{12}% ;
\TMP@EnsureCode{60}{12}% <
\TMP@EnsureCode{62}{12}% >
\TMP@EnsureCode{63}{12}% ?
\TMP@EnsureCode{91}{12}% [
\TMP@EnsureCode{93}{12}% ]
\TMP@EnsureCode{94}{7}% ^ (superscript)
\TMP@EnsureCode{95}{8}% _ (subscript)
\TMP@EnsureCode{96}{12}% `
\TMP@EnsureCode{124}{12}% |
\edef\HOLOGO@AtEnd{%
  \HOLOGO@AtEnd
  \escapechar\the\escapechar\relax
  \noexpand\endinput
}
\escapechar=92 %
%    \end{macrocode}
%
% \subsection{Logo list}
%
%    \begin{macro}{\hologoList}
%    \begin{macrocode}
\def\hologoList{%
  \hologoEntry{(La)TeX}{}{2011/10/01}%
  \hologoEntry{AmSLaTeX}{}{2010/04/16}%
  \hologoEntry{AmSTeX}{}{2010/04/16}%
  \hologoEntry{biber}{}{2011/10/01}%
  \hologoEntry{BibTeX}{}{2011/10/01}%
  \hologoEntry{BibTeX}{sf}{2011/10/01}%
  \hologoEntry{BibTeX}{sc}{2011/10/01}%
  \hologoEntry{BibTeX8}{}{2011/11/22}%
  \hologoEntry{ConTeXt}{}{2011/03/25}%
  \hologoEntry{ConTeXt}{narrow}{2011/03/25}%
  \hologoEntry{ConTeXt}{simple}{2011/03/25}%
  \hologoEntry{emTeX}{}{2010/04/26}%
  \hologoEntry{eTeX}{}{2010/04/08}%
  \hologoEntry{ExTeX}{}{2011/10/01}%
  \hologoEntry{HanTheThanh}{}{2011/11/29}%
  \hologoEntry{iniTeX}{}{2011/10/01}%
  \hologoEntry{KOMAScript}{}{2011/10/01}%
  \hologoEntry{La}{}{2010/05/08}%
  \hologoEntry{LaTeX}{}{2010/04/08}%
  \hologoEntry{LaTeX2e}{}{2010/04/08}%
  \hologoEntry{LaTeX3}{}{2010/04/24}%
  \hologoEntry{LaTeXe}{}{2010/04/08}%
  \hologoEntry{LaTeXML}{}{2011/11/22}%
  \hologoEntry{LaTeXTeX}{}{2011/10/01}%
  \hologoEntry{LuaLaTeX}{}{2010/04/08}%
  \hologoEntry{LuaTeX}{}{2010/04/08}%
  \hologoEntry{LyX}{}{2011/10/01}%
  \hologoEntry{METAFONT}{}{2011/10/01}%
  \hologoEntry{MetaFun}{}{2011/10/01}%
  \hologoEntry{METAPOST}{}{2011/10/01}%
  \hologoEntry{MetaPost}{}{2011/10/01}%
  \hologoEntry{MiKTeX}{}{2011/10/01}%
  \hologoEntry{NTS}{}{2011/10/01}%
  \hologoEntry{OzMF}{}{2011/10/01}%
  \hologoEntry{OzMP}{}{2011/10/01}%
  \hologoEntry{OzTeX}{}{2011/10/01}%
  \hologoEntry{OzTtH}{}{2011/10/01}%
  \hologoEntry{PCTeX}{}{2011/10/01}%
  \hologoEntry{pdfTeX}{}{2011/10/01}%
  \hologoEntry{pdfLaTeX}{}{2011/10/01}%
  \hologoEntry{PiC}{}{2011/10/01}%
  \hologoEntry{PiCTeX}{}{2011/10/01}%
  \hologoEntry{plainTeX}{}{2010/04/08}%
  \hologoEntry{plainTeX}{space}{2010/04/16}%
  \hologoEntry{plainTeX}{hyphen}{2010/04/16}%
  \hologoEntry{plainTeX}{runtogether}{2010/04/16}%
  \hologoEntry{SageTeX}{}{2011/11/22}%
  \hologoEntry{SLiTeX}{}{2011/10/01}%
  \hologoEntry{SLiTeX}{lift}{2011/10/01}%
  \hologoEntry{SLiTeX}{narrow}{2011/10/01}%
  \hologoEntry{SLiTeX}{simple}{2011/10/01}%
  \hologoEntry{SliTeX}{}{2011/10/01}%
  \hologoEntry{SliTeX}{narrow}{2011/10/01}%
  \hologoEntry{SliTeX}{simple}{2011/10/01}%
  \hologoEntry{SliTeX}{lift}{2011/10/01}%
  \hologoEntry{teTeX}{}{2011/10/01}%
  \hologoEntry{TeX}{}{2010/04/08}%
  \hologoEntry{TeX4ht}{}{2011/11/22}%
  \hologoEntry{TTH}{}{2011/11/22}%
  \hologoEntry{virTeX}{}{2011/10/01}%
  \hologoEntry{VTeX}{}{2010/04/24}%
  \hologoEntry{Xe}{}{2010/04/08}%
  \hologoEntry{XeLaTeX}{}{2010/04/08}%
  \hologoEntry{XeTeX}{}{2010/04/08}%
}
%    \end{macrocode}
%    \end{macro}
%
% \subsection{Load resources}
%
%    \begin{macrocode}
\begingroup\expandafter\expandafter\expandafter\endgroup
\expandafter\ifx\csname RequirePackage\endcsname\relax
  \def\TMP@RequirePackage#1[#2]{%
    \begingroup\expandafter\expandafter\expandafter\endgroup
    \expandafter\ifx\csname ver@#1.sty\endcsname\relax
      \input #1.sty\relax
    \fi
  }%
  \TMP@RequirePackage{ltxcmds}[2011/02/04]%
  \TMP@RequirePackage{infwarerr}[2010/04/08]%
  \TMP@RequirePackage{kvsetkeys}[2010/03/01]%
  \TMP@RequirePackage{kvdefinekeys}[2010/03/01]%
  \TMP@RequirePackage{pdftexcmds}[2010/04/01]%
  \TMP@RequirePackage{ifpdf}[2010/01/28]%
  \TMP@RequirePackage{ifluatex}[2010/03/01]%
  \ltx@IfUndefined{newif}{%
    \expandafter\let\csname newif\endcsname\ltx@newif
  }{}%
  \TMP@RequirePackage{ifxetex}[2009/01/23]%
  \TMP@RequirePackage{ifvtex}[2010/03/01]%
\else
  \RequirePackage{ltxcmds}[2011/02/04]%
  \RequirePackage{infwarerr}[2010/04/08]%
  \RequirePackage{kvsetkeys}[2010/03/01]%
  \RequirePackage{kvdefinekeys}[2010/03/01]%
  \RequirePackage{pdftexcmds}[2010/04/01]%
  \RequirePackage{ifpdf}[2010/01/28]%
  \RequirePackage{ifluatex}[2010/03/01]%
  \RequirePackage{ifxetex}[2009/01/23]%
  \RequirePackage{ifvtex}[2010/03/01]%
\fi
%    \end{macrocode}
%
%    \begin{macro}{\HOLOGO@IfDefined}
%    \begin{macrocode}
\def\HOLOGO@IfExists#1{%
  \ifx\@undefined#1%
    \expandafter\ltx@secondoftwo
  \else
    \ifx\relax#1%
      \expandafter\ltx@secondoftwo
    \else
      \expandafter\expandafter\expandafter\ltx@firstoftwo
    \fi
  \fi
}
%    \end{macrocode}
%    \end{macro}
%
% \subsection{Setup macros}
%
%    \begin{macro}{\hologoSetup}
%    \begin{macrocode}
\def\hologoSetup{%
  \let\HOLOGO@name\relax
  \HOLOGO@Setup
}
%    \end{macrocode}
%    \end{macro}
%
%    \begin{macro}{\hologoLogoSetup}
%    \begin{macrocode}
\def\hologoLogoSetup#1{%
  \edef\HOLOGO@name{#1}%
  \ltx@IfUndefined{HoLogo@\HOLOGO@name}{%
    \@PackageError{hologo}{%
      Unknown logo `\HOLOGO@name'%
    }\@ehc
    \ltx@gobble
  }{%
    \HOLOGO@Setup
  }%
}
%    \end{macrocode}
%    \end{macro}
%
%    \begin{macro}{\HOLOGO@Setup}
%    \begin{macrocode}
\def\HOLOGO@Setup{%
  \kvsetkeys{HoLogo}%
}
%    \end{macrocode}
%    \end{macro}
%
% \subsection{Options}
%
%    \begin{macro}{\HOLOGO@DeclareBoolOption}
%    \begin{macrocode}
\def\HOLOGO@DeclareBoolOption#1{%
  \expandafter\chardef\csname HOLOGOOPT@#1\endcsname\ltx@zero
  \kv@define@key{HoLogo}{#1}[true]{%
    \def\HOLOGO@temp{##1}%
    \ifx\HOLOGO@temp\HOLOGO@true
      \ifx\HOLOGO@name\relax
        \expandafter\chardef\csname HOLOGOOPT@#1\endcsname=\ltx@one
      \else
        \expandafter\chardef\csname
        HoLogoOpt@#1@\HOLOGO@name\endcsname\ltx@one
      \fi
      \HOLOGO@SetBreakAll{#1}%
    \else
      \ifx\HOLOGO@temp\HOLOGO@false
        \ifx\HOLOGO@name\relax
          \expandafter\chardef\csname HOLOGOOPT@#1\endcsname=\ltx@zero
        \else
          \expandafter\chardef\csname
          HoLogoOpt@#1@\HOLOGO@name\endcsname=\ltx@zero
        \fi
        \HOLOGO@SetBreakAll{#1}%
      \else
        \@PackageError{hologo}{%
          Unknown value `##1' for boolean option `#1'.\MessageBreak
          Known values are `true' and `false'%
        }\@ehc
      \fi
    \fi
  }%
}
%    \end{macrocode}
%    \end{macro}
%
%    \begin{macro}{\HOLOGO@SetBreakAll}
%    \begin{macrocode}
\def\HOLOGO@SetBreakAll#1{%
  \def\HOLOGO@temp{#1}%
  \ifx\HOLOGO@temp\HOLOGO@break
    \ifx\HOLOGO@name\relax
      \chardef\HOLOGOOPT@hyphenbreak=\HOLOGOOPT@break
      \chardef\HOLOGOOPT@spacebreak=\HOLOGOOPT@break
      \chardef\HOLOGOOPT@discretionarybreak=\HOLOGOOPT@break
    \else
      \expandafter\chardef
         \csname HoLogoOpt@hyphenbreak@\HOLOGO@name\endcsname=%
         \csname HoLogoOpt@break@\HOLOGO@name\endcsname
      \expandafter\chardef
         \csname HoLogoOpt@spacebreak@\HOLOGO@name\endcsname=%
         \csname HoLogoOpt@break@\HOLOGO@name\endcsname
      \expandafter\chardef
         \csname HoLogoOpt@discretionarybreak@\HOLOGO@name
             \endcsname=%
         \csname HoLogoOpt@break@\HOLOGO@name\endcsname
    \fi
  \fi
}
%    \end{macrocode}
%    \end{macro}
%
%    \begin{macro}{\HOLOGO@true}
%    \begin{macrocode}
\def\HOLOGO@true{true}
%    \end{macrocode}
%    \end{macro}
%    \begin{macro}{\HOLOGO@false}
%    \begin{macrocode}
\def\HOLOGO@false{false}
%    \end{macrocode}
%    \end{macro}
%    \begin{macro}{\HOLOGO@break}
%    \begin{macrocode}
\def\HOLOGO@break{break}
%    \end{macrocode}
%    \end{macro}
%
%    \begin{macrocode}
\HOLOGO@DeclareBoolOption{break}
\HOLOGO@DeclareBoolOption{hyphenbreak}
\HOLOGO@DeclareBoolOption{spacebreak}
\HOLOGO@DeclareBoolOption{discretionarybreak}
%    \end{macrocode}
%
%    \begin{macrocode}
\kv@define@key{HoLogo}{variant}{%
  \ifx\HOLOGO@name\relax
    \@PackageError{hologo}{%
      Option `variant' is not available in \string\hologoSetup,%
      \MessageBreak
      Use \string\hologoLogoSetup\space instead%
    }\@ehc
  \else
    \edef\HOLOGO@temp{#1}%
    \ifx\HOLOGO@temp\ltx@empty
      \expandafter
      \let\csname HoLogoOpt@variant@\HOLOGO@name\endcsname\@undefined
    \else
      \ltx@IfUndefined{HoLogo@\HOLOGO@name @\HOLOGO@temp}{%
        \@PackageError{hologo}{%
          Unknown variant `\HOLOGO@temp' of logo `\HOLOGO@name'%
        }\@ehc
      }{%
        \expandafter
        \let\csname HoLogoOpt@variant@\HOLOGO@name\endcsname
            \HOLOGO@temp
      }%
    \fi
  \fi
}
%    \end{macrocode}
%
%    \begin{macro}{\HOLOGO@Variant}
%    \begin{macrocode}
\def\HOLOGO@Variant#1{%
  #1%
  \ltx@ifundefined{HoLogoOpt@variant@#1}{%
  }{%
    @\csname HoLogoOpt@variant@#1\endcsname
  }%
}
%    \end{macrocode}
%    \end{macro}
%
% \subsection{Break/no-break support}
%
%    \begin{macro}{\HOLOGO@space}
%    \begin{macrocode}
\def\HOLOGO@space{%
  \ltx@ifundefined{HoLogoOpt@spacebreak@\HOLOGO@name}{%
    \ltx@ifundefined{HoLogoOpt@break@\HOLOGO@name}{%
      \chardef\HOLOGO@temp=\HOLOGOOPT@spacebreak
    }{%
      \chardef\HOLOGO@temp=%
        \csname HoLogoOpt@break@\HOLOGO@name\endcsname
    }%
  }{%
    \chardef\HOLOGO@temp=%
      \csname HoLogoOpt@spacebreak@\HOLOGO@name\endcsname
  }%
  \ifcase\HOLOGO@temp
    \penalty10000 %
  \fi
  \ltx@space
}
%    \end{macrocode}
%    \end{macro}
%
%    \begin{macro}{\HOLOGO@hyphen}
%    \begin{macrocode}
\def\HOLOGO@hyphen{%
  \ltx@ifundefined{HoLogoOpt@hyphenbreak@\HOLOGO@name}{%
    \ltx@ifundefined{HoLogoOpt@break@\HOLOGO@name}{%
      \chardef\HOLOGO@temp=\HOLOGOOPT@hyphenbreak
    }{%
      \chardef\HOLOGO@temp=%
        \csname HoLogoOpt@break@\HOLOGO@name\endcsname
    }%
  }{%
    \chardef\HOLOGO@temp=%
      \csname HoLogoOpt@hyphenbreak@\HOLOGO@name\endcsname
  }%
  \ifcase\HOLOGO@temp
    \ltx@mbox{-}%
  \else
    -%
  \fi
}
%    \end{macrocode}
%    \end{macro}
%
%    \begin{macro}{\HOLOGO@discretionary}
%    \begin{macrocode}
\def\HOLOGO@discretionary{%
  \ltx@ifundefined{HoLogoOpt@discretionarybreak@\HOLOGO@name}{%
    \ltx@ifundefined{HoLogoOpt@break@\HOLOGO@name}{%
      \chardef\HOLOGO@temp=\HOLOGOOPT@discretionarybreak
    }{%
      \chardef\HOLOGO@temp=%
        \csname HoLogoOpt@break@\HOLOGO@name\endcsname
    }%
  }{%
    \chardef\HOLOGO@temp=%
      \csname HoLogoOpt@discretionarybreak@\HOLOGO@name\endcsname
  }%
  \ifcase\HOLOGO@temp
  \else
    \-%
  \fi
}
%    \end{macrocode}
%    \end{macro}
%
%    \begin{macro}{\HOLOGO@mbox}
%    \begin{macrocode}
\def\HOLOGO@mbox#1{%
  \ltx@ifundefined{HoLogoOpt@break@\HOLOGO@name}{%
    \chardef\HOLOGO@temp=\HOLOGOOPT@hyphenbreak
  }{%
    \chardef\HOLOGO@temp=%
      \csname HoLogoOpt@break@\HOLOGO@name\endcsname
  }%
  \ifcase\HOLOGO@temp
    \ltx@mbox{#1}%
  \else
    #1%
  \fi
}
%    \end{macrocode}
%    \end{macro}
%
% \subsection{Font support}
%
%    \begin{macro}{\HoLogoFont@font}
%    \begin{tabular}{@{}ll@{}}
%    |#1|:& logo name\\
%    |#2|:& font short name\\
%    |#3|:& text
%    \end{tabular}
%    \begin{macrocode}
\def\HoLogoFont@font#1#2#3{%
  \begingroup
    \ltx@IfUndefined{HoLogoFont@logo@#1.#2}{%
      \ltx@IfUndefined{HoLogoFont@font@#2}{%
        \@PackageWarning{hologo}{%
          Missing font `#2' for logo `#1'%
        }%
        #3%
      }{%
        \csname HoLogoFont@font@#2\endcsname{#3}%
      }%
    }{%
      \csname HoLogoFont@logo@#1.#2\endcsname{#3}%
    }%
  \endgroup
}
%    \end{macrocode}
%    \end{macro}
%
%    \begin{macro}{\HoLogoFont@Def}
%    \begin{macrocode}
\def\HoLogoFont@Def#1{%
  \expandafter\def\csname HoLogoFont@font@#1\endcsname
}
%    \end{macrocode}
%    \end{macro}
%    \begin{macro}{\HoLogoFont@LogoDef}
%    \begin{macrocode}
\def\HoLogoFont@LogoDef#1#2{%
  \expandafter\def\csname HoLogoFont@logo@#1.#2\endcsname
}
%    \end{macrocode}
%    \end{macro}
%
% \subsubsection{Font defaults}
%
%    \begin{macro}{\HoLogoFont@font@general}
%    \begin{macrocode}
\HoLogoFont@Def{general}{}%
%    \end{macrocode}
%    \end{macro}
%
%    \begin{macro}{\HoLogoFont@font@rm}
%    \begin{macrocode}
\ltx@IfUndefined{rmfamily}{%
  \ltx@IfUndefined{rm}{%
  }{%
    \HoLogoFont@Def{rm}{\rm}%
  }%
}{%
  \HoLogoFont@Def{rm}{\rmfamily}%
}
%    \end{macrocode}
%    \end{macro}
%
%    \begin{macro}{\HoLogoFont@font@sf}
%    \begin{macrocode}
\ltx@IfUndefined{sffamily}{%
  \ltx@IfUndefined{sf}{%
  }{%
    \HoLogoFont@Def{sf}{\sf}%
  }%
}{%
  \HoLogoFont@Def{sf}{\sffamily}%
}
%    \end{macrocode}
%    \end{macro}
%
%    \begin{macro}{\HoLogoFont@font@bibsf}
%    In case of \hologo{plainTeX} the original small caps
%    variant is used as default. In \hologo{LaTeX}
%    the definition of package \xpackage{dtklogos} \cite{dtklogos}
%    is used.
%\begin{quote}
%\begin{verbatim}
%\DeclareRobustCommand{\BibTeX}{%
%  B%
%  \kern-.05em%
%  \hbox{%
%    $\m@th$% %% force math size calculations
%    \csname S@\f@size\endcsname
%    \fontsize\sf@size\z@
%    \math@fontsfalse
%    \selectfont
%    I%
%    \kern-.025em%
%    B
%  }%
%  \kern-.08em%
%  \-%
%  \TeX
%}
%\end{verbatim}
%\end{quote}
%    \begin{macrocode}
\ltx@IfUndefined{selectfont}{%
  \ltx@IfUndefined{tensc}{%
    \font\tensc=cmcsc10\relax
  }{}%
  \HoLogoFont@Def{bibsf}{\tensc}%
}{%
  \HoLogoFont@Def{bibsf}{%
    $\mathsurround=0pt$%
    \csname S@\f@size\endcsname
    \fontsize\sf@size{0pt}%
    \math@fontsfalse
    \selectfont
  }%
}
%    \end{macrocode}
%    \end{macro}
%
%    \begin{macro}{\HoLogoFont@font@sc}
%    \begin{macrocode}
\ltx@IfUndefined{scshape}{%
  \ltx@IfUndefined{tensc}{%
    \font\tensc=cmcsc10\relax
  }{}%
  \HoLogoFont@Def{sc}{\tensc}%
}{%
  \HoLogoFont@Def{sc}{\scshape}%
}
%    \end{macrocode}
%    \end{macro}
%
%    \begin{macro}{\HoLogoFont@font@sy}
%    \begin{macrocode}
\ltx@IfUndefined{usefont}{%
  \ltx@IfUndefined{tensy}{%
  }{%
    \HoLogoFont@Def{sy}{\tensy}%
  }%
}{%
  \HoLogoFont@Def{sy}{%
    \usefont{OMS}{cmsy}{m}{n}%
  }%
}
%    \end{macrocode}
%    \end{macro}
%
%    \begin{macro}{\HoLogoFont@font@logo}
%    \begin{macrocode}
\begingroup
  \def\x{LaTeX2e}%
\expandafter\endgroup
\ifx\fmtname\x
  \ltx@IfUndefined{logofamily}{%
    \DeclareRobustCommand\logofamily{%
      \not@math@alphabet\logofamily\relax
      \fontencoding{U}%
      \fontfamily{logo}%
      \selectfont
    }%
  }{}%
  \ltx@IfUndefined{logofamily}{%
  }{%
    \HoLogoFont@Def{logo}{\logofamily}%
  }%
\else
  \ltx@IfUndefined{tenlogo}{%
    \font\tenlogo=logo10\relax
  }{}%
  \HoLogoFont@Def{logo}{\tenlogo}%
\fi
%    \end{macrocode}
%    \end{macro}
%
% \subsubsection{Font setup}
%
%    \begin{macro}{\hologoFontSetup}
%    \begin{macrocode}
\def\hologoFontSetup{%
  \let\HOLOGO@name\relax
  \HOLOGO@FontSetup
}
%    \end{macrocode}
%    \end{macro}
%
%    \begin{macro}{\hologoLogoFontSetup}
%    \begin{macrocode}
\def\hologoLogoFontSetup#1{%
  \edef\HOLOGO@name{#1}%
  \ltx@IfUndefined{HoLogo@\HOLOGO@name}{%
    \@PackageError{hologo}{%
      Unknown logo `\HOLOGO@name'%
    }\@ehc
    \ltx@gobble
  }{%
    \HOLOGO@FontSetup
  }%
}
%    \end{macrocode}
%    \end{macro}
%
%    \begin{macro}{\HOLOGO@FontSetup}
%    \begin{macrocode}
\def\HOLOGO@FontSetup{%
  \kvsetkeys{HoLogoFont}%
}
%    \end{macrocode}
%    \end{macro}
%
%    \begin{macrocode}
\def\HOLOGO@temp#1{%
  \kv@define@key{HoLogoFont}{#1}{%
    \ifx\HOLOGO@name\relax
      \HoLogoFont@Def{#1}{##1}%
    \else
      \HoLogoFont@LogoDef\HOLOGO@name{#1}{##1}%
    \fi
  }%
}
\HOLOGO@temp{general}
\HOLOGO@temp{sf}
%    \end{macrocode}
%
% \subsection{Generic logo commands}
%
%    \begin{macrocode}
\HOLOGO@IfExists\hologo{%
  \@PackageError{hologo}{%
    \string\hologo\ltx@space is already defined.\MessageBreak
    Package loading is aborted%
  }\@ehc
  \HOLOGO@AtEnd
}%
\HOLOGO@IfExists\hologoRobust{%
  \@PackageError{hologo}{%
    \string\hologoRobust\ltx@space is already defined.\MessageBreak
    Package loading is aborted%
  }\@ehc
  \HOLOGO@AtEnd
}%
%    \end{macrocode}
%
% \subsubsection{\cs{hologo} and friends}
%
%    \begin{macrocode}
\ifluatex
  \expandafter\ltx@firstofone
\else
  \expandafter\ltx@gobble
\fi
{%
  \ltx@IfUndefined{ifincsname}{%
    \ifnum\luatexversion<36 %
      \expandafter\ltx@gobble
    \else
      \expandafter\ltx@firstofone
    \fi
    {%
      \begingroup
        \ifcase0%
            \directlua{%
              if tex.enableprimitives then %
                tex.enableprimitives('HOLOGO@', {'ifincsname'})%
              else %
                tex.print('1')%
              end%
            }%
            \ifx\HOLOGO@ifincsname\@undefined 1\fi%
            \relax
          \expandafter\ltx@firstofone
        \else
          \endgroup
          \expandafter\ltx@gobble
        \fi
        {%
          \global\let\ifincsname\HOLOGO@ifincsname
        }%
      \HOLOGO@temp
    }%
  }{}%
}
%    \end{macrocode}
%    \begin{macrocode}
\ltx@IfUndefined{ifincsname}{%
  \catcode`$=14 %
}{%
  \catcode`$=9 %
}
%    \end{macrocode}
%
%    \begin{macro}{\hologo}
%    \begin{macrocode}
\def\hologo#1{%
$ \ifincsname
$   \ltx@ifundefined{HoLogoCs@\HOLOGO@Variant{#1}}{%
$     #1%
$   }{%
$     \csname HoLogoCs@\HOLOGO@Variant{#1}\endcsname\ltx@firstoftwo
$   }%
$ \else
    \HOLOGO@IfExists\texorpdfstring\texorpdfstring\ltx@firstoftwo
    {%
      \hologoRobust{#1}%
    }{%
      \ltx@ifundefined{HoLogoBkm@\HOLOGO@Variant{#1}}{%
        \ltx@ifundefined{HoLogo@#1}{?#1?}{#1}%
      }{%
        \csname HoLogoBkm@\HOLOGO@Variant{#1}\endcsname
        \ltx@firstoftwo
      }%
    }%
$ \fi
}
%    \end{macrocode}
%    \end{macro}
%    \begin{macro}{\Hologo}
%    \begin{macrocode}
\def\Hologo#1{%
$ \ifincsname
$   \ltx@ifundefined{HoLogoCs@\HOLOGO@Variant{#1}}{%
$     #1%
$   }{%
$     \csname HoLogoCs@\HOLOGO@Variant{#1}\endcsname\ltx@secondoftwo
$   }%
$ \else
    \HOLOGO@IfExists\texorpdfstring\texorpdfstring\ltx@firstoftwo
    {%
      \HologoRobust{#1}%
    }{%
      \ltx@ifundefined{HoLogoBkm@\HOLOGO@Variant{#1}}{%
        \ltx@ifundefined{HoLogo@#1}{?#1?}{#1}%
      }{%
        \csname HoLogoBkm@\HOLOGO@Variant{#1}\endcsname
        \ltx@secondoftwo
      }%
    }%
$ \fi
}
%    \end{macrocode}
%    \end{macro}
%
%    \begin{macro}{\hologoVariant}
%    \begin{macrocode}
\def\hologoVariant#1#2{%
  \ifx\relax#2\relax
    \hologo{#1}%
  \else
$   \ifincsname
$     \ltx@ifundefined{HoLogoCs@#1@#2}{%
$       #1%
$     }{%
$       \csname HoLogoCs@#1@#2\endcsname\ltx@firstoftwo
$     }%
$   \else
      \HOLOGO@IfExists\texorpdfstring\texorpdfstring\ltx@firstoftwo
      {%
        \hologoVariantRobust{#1}{#2}%
      }{%
        \ltx@ifundefined{HoLogoBkm@#1@#2}{%
          \ltx@ifundefined{HoLogo@#1}{?#1?}{#1}%
        }{%
          \csname HoLogoBkm@#1@#2\endcsname
          \ltx@firstoftwo
        }%
      }%
$   \fi
  \fi
}
%    \end{macrocode}
%    \end{macro}
%    \begin{macro}{\HologoVariant}
%    \begin{macrocode}
\def\HologoVariant#1#2{%
  \ifx\relax#2\relax
    \Hologo{#1}%
  \else
$   \ifincsname
$     \ltx@ifundefined{HoLogoCs@#1@#2}{%
$       #1%
$     }{%
$       \csname HoLogoCs@#1@#2\endcsname\ltx@secondoftwo
$     }%
$   \else
      \HOLOGO@IfExists\texorpdfstring\texorpdfstring\ltx@firstoftwo
      {%
        \HologoVariantRobust{#1}{#2}%
      }{%
        \ltx@ifundefined{HoLogoBkm@#1@#2}{%
          \ltx@ifundefined{HoLogo@#1}{?#1?}{#1}%
        }{%
          \csname HoLogoBkm@#1@#2\endcsname
          \ltx@secondoftwo
        }%
      }%
$   \fi
  \fi
}
%    \end{macrocode}
%    \end{macro}
%
%    \begin{macrocode}
\catcode`\$=3 %
%    \end{macrocode}
%
% \subsubsection{\cs{hologoRobust} and friends}
%
%    \begin{macro}{\hologoRobust}
%    \begin{macrocode}
\ltx@IfUndefined{protected}{%
  \ltx@IfUndefined{DeclareRobustCommand}{%
    \def\hologoRobust#1%
  }{%
    \DeclareRobustCommand*\hologoRobust[1]%
  }%
}{%
  \protected\def\hologoRobust#1%
}%
{%
  \edef\HOLOGO@name{#1}%
  \ltx@IfUndefined{HoLogo@\HOLOGO@Variant\HOLOGO@name}{%
    \@PackageError{hologo}{%
      Unknown logo `\HOLOGO@name'%
    }\@ehc
    ?\HOLOGO@name?%
  }{%
    \ltx@IfUndefined{ver@tex4ht.sty}{%
      \HoLogoFont@font\HOLOGO@name{general}{%
        \csname HoLogo@\HOLOGO@Variant\HOLOGO@name\endcsname
        \ltx@firstoftwo
      }%
    }{%
      \ltx@IfUndefined{HoLogoHtml@\HOLOGO@Variant\HOLOGO@name}{%
        \HOLOGO@name
      }{%
        \csname HoLogoHtml@\HOLOGO@Variant\HOLOGO@name\endcsname
        \ltx@firstoftwo
      }%
    }%
  }%
}
%    \end{macrocode}
%    \end{macro}
%    \begin{macro}{\HologoRobust}
%    \begin{macrocode}
\ltx@IfUndefined{protected}{%
  \ltx@IfUndefined{DeclareRobustCommand}{%
    \def\HologoRobust#1%
  }{%
    \DeclareRobustCommand*\HologoRobust[1]%
  }%
}{%
  \protected\def\HologoRobust#1%
}%
{%
  \edef\HOLOGO@name{#1}%
  \ltx@IfUndefined{HoLogo@\HOLOGO@Variant\HOLOGO@name}{%
    \@PackageError{hologo}{%
      Unknown logo `\HOLOGO@name'%
    }\@ehc
    ?\HOLOGO@name?%
  }{%
    \ltx@IfUndefined{ver@tex4ht.sty}{%
      \HoLogoFont@font\HOLOGO@name{general}{%
        \csname HoLogo@\HOLOGO@Variant\HOLOGO@name\endcsname
        \ltx@secondoftwo
      }%
    }{%
      \ltx@IfUndefined{HoLogoHtml@\HOLOGO@Variant\HOLOGO@name}{%
        \expandafter\HOLOGO@Uppercase\HOLOGO@name
      }{%
        \csname HoLogoHtml@\HOLOGO@Variant\HOLOGO@name\endcsname
        \ltx@secondoftwo
      }%
    }%
  }%
}
%    \end{macrocode}
%    \end{macro}
%    \begin{macro}{\hologoVariantRobust}
%    \begin{macrocode}
\ltx@IfUndefined{protected}{%
  \ltx@IfUndefined{DeclareRobustCommand}{%
    \def\hologoVariantRobust#1#2%
  }{%
    \DeclareRobustCommand*\hologoVariantRobust[2]%
  }%
}{%
  \protected\def\hologoVariantRobust#1#2%
}%
{%
  \begingroup
    \hologoLogoSetup{#1}{variant={#2}}%
    \hologoRobust{#1}%
  \endgroup
}
%    \end{macrocode}
%    \end{macro}
%    \begin{macro}{\HologoVariantRobust}
%    \begin{macrocode}
\ltx@IfUndefined{protected}{%
  \ltx@IfUndefined{DeclareRobustCommand}{%
    \def\HologoVariantRobust#1#2%
  }{%
    \DeclareRobustCommand*\HologoVariantRobust[2]%
  }%
}{%
  \protected\def\HologoVariantRobust#1#2%
}%
{%
  \begingroup
    \hologoLogoSetup{#1}{variant={#2}}%
    \HologoRobust{#1}%
  \endgroup
}
%    \end{macrocode}
%    \end{macro}
%
%    \begin{macro}{\hologorobust}
%    Macro \cs{hologorobust} is only defined for compatibility.
%    Its use is deprecated.
%    \begin{macrocode}
\def\hologorobust{\hologoRobust}
%    \end{macrocode}
%    \end{macro}
%
% \subsection{Helpers}
%
%    \begin{macro}{\HOLOGO@Uppercase}
%    Macro \cs{HOLOGO@Uppercase} is restricted to \cs{uppercase},
%    because \hologo{plainTeX} or \hologo{iniTeX} do not provide
%    \cs{MakeUppercase}.
%    \begin{macrocode}
\def\HOLOGO@Uppercase#1{\uppercase{#1}}
%    \end{macrocode}
%    \end{macro}
%
%    \begin{macro}{\HOLOGO@PdfdocUnicode}
%    \begin{macrocode}
\def\HOLOGO@PdfdocUnicode{%
  \ifx\ifHy@unicode\iftrue
    \expandafter\ltx@secondoftwo
  \else
    \expandafter\ltx@firstoftwo
  \fi
}
%    \end{macrocode}
%    \end{macro}
%
%    \begin{macro}{\HOLOGO@Math}
%    \begin{macrocode}
\def\HOLOGO@MathSetup{%
  \mathsurround0pt\relax
  \HOLOGO@IfExists\f@series{%
    \if b\expandafter\ltx@car\f@series x\@nil
      \csname boldmath\endcsname
   \fi
  }{}%
}
%    \end{macrocode}
%    \end{macro}
%
%    \begin{macro}{\HOLOGO@TempDimen}
%    \begin{macrocode}
\dimendef\HOLOGO@TempDimen=\ltx@zero
%    \end{macrocode}
%    \end{macro}
%    \begin{macro}{\HOLOGO@NegativeKerning}
%    \begin{macrocode}
\def\HOLOGO@NegativeKerning#1{%
  \begingroup
    \HOLOGO@TempDimen=0pt\relax
    \comma@parse@normalized{#1}{%
      \ifdim\HOLOGO@TempDimen=0pt %
        \expandafter\HOLOGO@@NegativeKerning\comma@entry
      \fi
      \ltx@gobble
    }%
    \ifdim\HOLOGO@TempDimen<0pt %
      \kern\HOLOGO@TempDimen
    \fi
  \endgroup
}
%    \end{macrocode}
%    \end{macro}
%    \begin{macro}{\HOLOGO@@NegativeKerning}
%    \begin{macrocode}
\def\HOLOGO@@NegativeKerning#1#2{%
  \setbox\ltx@zero\hbox{#1#2}%
  \HOLOGO@TempDimen=\wd\ltx@zero
  \setbox\ltx@zero\hbox{#1\kern0pt#2}%
  \advance\HOLOGO@TempDimen by -\wd\ltx@zero
}
%    \end{macrocode}
%    \end{macro}
%
%    \begin{macro}{\HOLOGO@SpaceFactor}
%    \begin{macrocode}
\def\HOLOGO@SpaceFactor{%
  \spacefactor1000 %
}
%    \end{macrocode}
%    \end{macro}
%
%    \begin{macro}{\HOLOGO@Span}
%    \begin{macrocode}
\def\HOLOGO@Span#1#2{%
  \HCode{<span class="HoLogo-#1">}%
  #2%
  \HCode{</span>}%
}
%    \end{macrocode}
%    \end{macro}
%
% \subsubsection{Text subscript}
%
%    \begin{macro}{\HOLOGO@SubScript}%
%    \begin{macrocode}
\def\HOLOGO@SubScript#1{%
  \ltx@IfUndefined{textsubscript}{%
    \ltx@IfUndefined{text}{%
      \ltx@mbox{%
        \mathsurround=0pt\relax
        $%
          _{%
            \ltx@IfUndefined{sf@size}{%
              \mathrm{#1}%
            }{%
              \mbox{%
                \fontsize\sf@size{0pt}\selectfont
                #1%
              }%
            }%
          }%
        $%
      }%
    }{%
      \ltx@mbox{%
        \mathsurround=0pt\relax
        $_{\text{#1}}$%
      }%
    }%
  }{%
    \textsubscript{#1}%
  }%
}
%    \end{macrocode}
%    \end{macro}
%
% \subsection{\hologo{TeX} and friends}
%
% \subsubsection{\hologo{TeX}}
%
%    \begin{macro}{\HoLogo@TeX}
%    Source: \hologo{LaTeX} kernel.
%    \begin{macrocode}
\def\HoLogo@TeX#1{%
  T\kern-.1667em\lower.5ex\hbox{E}\kern-.125emX\HOLOGO@SpaceFactor
}
%    \end{macrocode}
%    \end{macro}
%    \begin{macro}{\HoLogoHtml@TeX}
%    \begin{macrocode}
\def\HoLogoHtml@TeX#1{%
  \HoLogoCss@TeX
  \HOLOGO@Span{TeX}{%
    T%
    \HOLOGO@Span{e}{%
      E%
    }%
    X%
  }%
}
%    \end{macrocode}
%    \end{macro}
%    \begin{macro}{\HoLogoCss@TeX}
%    \begin{macrocode}
\def\HoLogoCss@TeX{%
  \Css{%
    span.HoLogo-TeX span.HoLogo-e{%
      position:relative;%
      top:.5ex;%
      margin-left:-.1667em;%
      margin-right:-.125em;%
    }%
  }%
  \Css{%
    a span.HoLogo-TeX span.HoLogo-e{%
      text-decoration:none;%
    }%
  }%
  \global\let\HoLogoCss@TeX\relax
}
%    \end{macrocode}
%    \end{macro}
%
% \subsubsection{\hologo{plainTeX}}
%
%    \begin{macro}{\HoLogo@plainTeX@space}
%    Source: ``The \hologo{TeX}book''
%    \begin{macrocode}
\def\HoLogo@plainTeX@space#1{%
  \HOLOGO@mbox{#1{p}{P}lain}\HOLOGO@space\hologo{TeX}%
}
%    \end{macrocode}
%    \end{macro}
%    \begin{macro}{\HoLogoCs@plainTeX@space}
%    \begin{macrocode}
\def\HoLogoCs@plainTeX@space#1{#1{p}{P}lain TeX}%
%    \end{macrocode}
%    \end{macro}
%    \begin{macro}{\HoLogoBkm@plainTeX@space}
%    \begin{macrocode}
\def\HoLogoBkm@plainTeX@space#1{%
  #1{p}{P}lain \hologo{TeX}%
}
%    \end{macrocode}
%    \end{macro}
%    \begin{macro}{\HoLogoHtml@plainTeX@space}
%    \begin{macrocode}
\def\HoLogoHtml@plainTeX@space#1{%
  #1{p}{P}lain \hologo{TeX}%
}
%    \end{macrocode}
%    \end{macro}
%
%    \begin{macro}{\HoLogo@plainTeX@hyphen}
%    \begin{macrocode}
\def\HoLogo@plainTeX@hyphen#1{%
  \HOLOGO@mbox{#1{p}{P}lain}\HOLOGO@hyphen\hologo{TeX}%
}
%    \end{macrocode}
%    \end{macro}
%    \begin{macro}{\HoLogoCs@plainTeX@hyphen}
%    \begin{macrocode}
\def\HoLogoCs@plainTeX@hyphen#1{#1{p}{P}lain-TeX}
%    \end{macrocode}
%    \end{macro}
%    \begin{macro}{\HoLogoBkm@plainTeX@hyphen}
%    \begin{macrocode}
\def\HoLogoBkm@plainTeX@hyphen#1{%
  #1{p}{P}lain-\hologo{TeX}%
}
%    \end{macrocode}
%    \end{macro}
%    \begin{macro}{\HoLogoHtml@plainTeX@hyphen}
%    \begin{macrocode}
\def\HoLogoHtml@plainTeX@hyphen#1{%
  #1{p}{P}lain-\hologo{TeX}%
}
%    \end{macrocode}
%    \end{macro}
%
%    \begin{macro}{\HoLogo@plainTeX@runtogether}
%    \begin{macrocode}
\def\HoLogo@plainTeX@runtogether#1{%
  \HOLOGO@mbox{#1{p}{P}lain\hologo{TeX}}%
}
%    \end{macrocode}
%    \end{macro}
%    \begin{macro}{\HoLogoCs@plainTeX@runtogether}
%    \begin{macrocode}
\def\HoLogoCs@plainTeX@runtogether#1{#1{p}{P}lainTeX}
%    \end{macrocode}
%    \end{macro}
%    \begin{macro}{\HoLogoBkm@plainTeX@runtogether}
%    \begin{macrocode}
\def\HoLogoBkm@plainTeX@runtogether#1{%
  #1{p}{P}lain\hologo{TeX}%
}
%    \end{macrocode}
%    \end{macro}
%    \begin{macro}{\HoLogoHtml@plainTeX@runtogether}
%    \begin{macrocode}
\def\HoLogoHtml@plainTeX@runtogether#1{%
  #1{p}{P}lain\hologo{TeX}%
}
%    \end{macrocode}
%    \end{macro}
%
%    \begin{macro}{\HoLogo@plainTeX}
%    \begin{macrocode}
\def\HoLogo@plainTeX{\HoLogo@plainTeX@space}
%    \end{macrocode}
%    \end{macro}
%    \begin{macro}{\HoLogoCs@plainTeX}
%    \begin{macrocode}
\def\HoLogoCs@plainTeX{\HoLogoCs@plainTeX@space}
%    \end{macrocode}
%    \end{macro}
%    \begin{macro}{\HoLogoBkm@plainTeX}
%    \begin{macrocode}
\def\HoLogoBkm@plainTeX{\HoLogoBkm@plainTeX@space}
%    \end{macrocode}
%    \end{macro}
%    \begin{macro}{\HoLogoHtml@plainTeX}
%    \begin{macrocode}
\def\HoLogoHtml@plainTeX{\HoLogoHtml@plainTeX@space}
%    \end{macrocode}
%    \end{macro}
%
% \subsubsection{\hologo{LaTeX}}
%
%    Source: \hologo{LaTeX} kernel.
%\begin{quote}
%\begin{verbatim}
%\DeclareRobustCommand{\LaTeX}{%
%  L%
%  \kern-.36em%
%  {%
%    \sbox\z@ T%
%    \vbox to\ht\z@{%
%      \hbox{%
%        \check@mathfonts
%        \fontsize\sf@size\z@
%        \math@fontsfalse
%        \selectfont
%        A%
%      }%
%      \vss
%    }%
%  }%
%  \kern-.15em%
%  \TeX
%}
%\end{verbatim}
%\end{quote}
%
%    \begin{macro}{\HoLogo@La}
%    \begin{macrocode}
\def\HoLogo@La#1{%
  L%
  \kern-.36em%
  \begingroup
    \setbox\ltx@zero\hbox{T}%
    \vbox to\ht\ltx@zero{%
      \hbox{%
        \ltx@ifundefined{check@mathfonts}{%
          \csname sevenrm\endcsname
        }{%
          \check@mathfonts
          \fontsize\sf@size{0pt}%
          \math@fontsfalse\selectfont
        }%
        A%
      }%
      \vss
    }%
  \endgroup
}
%    \end{macrocode}
%    \end{macro}
%
%    \begin{macro}{\HoLogo@LaTeX}
%    Source: \hologo{LaTeX} kernel.
%    \begin{macrocode}
\def\HoLogo@LaTeX#1{%
  \hologo{La}%
  \kern-.15em%
  \hologo{TeX}%
}
%    \end{macrocode}
%    \end{macro}
%    \begin{macro}{\HoLogoHtml@LaTeX}
%    \begin{macrocode}
\def\HoLogoHtml@LaTeX#1{%
  \HoLogoCss@LaTeX
  \HOLOGO@Span{LaTeX}{%
    L%
    \HOLOGO@Span{a}{%
      A%
    }%
    \hologo{TeX}%
  }%
}
%    \end{macrocode}
%    \end{macro}
%    \begin{macro}{\HoLogoCss@LaTeX}
%    \begin{macrocode}
\def\HoLogoCss@LaTeX{%
  \Css{%
    span.HoLogo-LaTeX span.HoLogo-a{%
      position:relative;%
      top:-.5ex;%
      margin-left:-.36em;%
      margin-right:-.15em;%
      font-size:85\%;%
    }%
  }%
  \global\let\HoLogoCss@LaTeX\relax
}
%    \end{macrocode}
%    \end{macro}
%
% \subsubsection{\hologo{(La)TeX}}
%
%    \begin{macro}{\HoLogo@LaTeXTeX}
%    The kerning around the parentheses is taken
%    from package \xpackage{dtklogos} \cite{dtklogos}.
%\begin{quote}
%\begin{verbatim}
%\DeclareRobustCommand{\LaTeXTeX}{%
%  (%
%  \kern-.15em%
%  L%
%  \kern-.36em%
%  {%
%    \sbox\z@ T%
%    \vbox to\ht0{%
%      \hbox{%
%        $\m@th$%
%        \csname S@\f@size\endcsname
%        \fontsize\sf@size\z@
%        \math@fontsfalse
%        \selectfont
%        A%
%      }%
%      \vss
%    }%
%  }%
%  \kern-.2em%
%  )%
%  \kern-.15em%
%  \TeX
%}
%\end{verbatim}
%\end{quote}
%    \begin{macrocode}
\def\HoLogo@LaTeXTeX#1{%
  (%
  \kern-.15em%
  \hologo{La}%
  \kern-.2em%
  )%
  \kern-.15em%
  \hologo{TeX}%
}
%    \end{macrocode}
%    \end{macro}
%    \begin{macro}{\HoLogoBkm@LaTeXTeX}
%    \begin{macrocode}
\def\HoLogoBkm@LaTeXTeX#1{(La)TeX}
%    \end{macrocode}
%    \end{macro}
%
%    \begin{macro}{\HoLogo@(La)TeX}
%    \begin{macrocode}
\expandafter
\let\csname HoLogo@(La)TeX\endcsname\HoLogo@LaTeXTeX
%    \end{macrocode}
%    \end{macro}
%    \begin{macro}{\HoLogoBkm@(La)TeX}
%    \begin{macrocode}
\expandafter
\let\csname HoLogoBkm@(La)TeX\endcsname\HoLogoBkm@LaTeXTeX
%    \end{macrocode}
%    \end{macro}
%    \begin{macro}{\HoLogoHtml@LaTeXTeX}
%    \begin{macrocode}
\def\HoLogoHtml@LaTeXTeX#1{%
  \HoLogoCss@LaTeXTeX
  \HOLOGO@Span{LaTeXTeX}{%
    (%
    \HOLOGO@Span{L}{L}%
    \HOLOGO@Span{a}{A}%
    \HOLOGO@Span{ParenRight}{)}%
    \hologo{TeX}%
  }%
}
%    \end{macrocode}
%    \end{macro}
%    \begin{macro}{\HoLogoHtml@(La)TeX}
%    Kerning after opening parentheses and before closing parentheses
%    is $-0.1$\,em. The original values $-0.15$\,em
%    looked too ugly for a serif font.
%    \begin{macrocode}
\expandafter
\let\csname HoLogoHtml@(La)TeX\endcsname\HoLogoHtml@LaTeXTeX
%    \end{macrocode}
%    \end{macro}
%    \begin{macro}{\HoLogoCss@LaTeXTeX}
%    \begin{macrocode}
\def\HoLogoCss@LaTeXTeX{%
  \Css{%
    span.HoLogo-LaTeXTeX span.HoLogo-L{%
      margin-left:-.1em;%
    }%
  }%
  \Css{%
    span.HoLogo-LaTeXTeX span.HoLogo-a{%
      position:relative;%
      top:-.5ex;%
      margin-left:-.36em;%
      margin-right:-.1em;%
      font-size:85\%;%
    }%
  }%
  \Css{%
    span.HoLogo-LaTeXTeX span.HoLogo-ParenRight{%
      margin-right:-.15em;%
    }%
  }%
  \global\let\HoLogoCss@LaTeXTeX\relax
}
%    \end{macrocode}
%    \end{macro}
%
% \subsubsection{\hologo{LaTeXe}}
%
%    \begin{macro}{\HoLogo@LaTeXe}
%    Source: \hologo{LaTeX} kernel
%    \begin{macrocode}
\def\HoLogo@LaTeXe#1{%
  \hologo{LaTeX}%
  \kern.15em%
  \hbox{%
    \HOLOGO@MathSetup
    2%
    $_{\textstyle\varepsilon}$%
  }%
}
%    \end{macrocode}
%    \end{macro}
%
%    \begin{macro}{\HoLogoCs@LaTeXe}
%    \begin{macrocode}
\ifnum64=`\^^^^0040\relax % test for big chars of LuaTeX/XeTeX
  \catcode`\$=9 %
  \catcode`\&=14 %
\else
  \catcode`\$=14 %
  \catcode`\&=9 %
\fi
\def\HoLogoCs@LaTeXe#1{%
  LaTeX2%
$ \string ^^^^0395%
& e%
}%
\catcode`\$=3 %
\catcode`\&=4 %
%    \end{macrocode}
%    \end{macro}
%
%    \begin{macro}{\HoLogoBkm@LaTeXe}
%    \begin{macrocode}
\def\HoLogoBkm@LaTeXe#1{%
  \hologo{LaTeX}%
  2%
  \HOLOGO@PdfdocUnicode{e}{\textepsilon}%
}
%    \end{macrocode}
%    \end{macro}
%
%    \begin{macro}{\HoLogoHtml@LaTeXe}
%    \begin{macrocode}
\def\HoLogoHtml@LaTeXe#1{%
  \HoLogoCss@LaTeXe
  \HOLOGO@Span{LaTeX2e}{%
    \hologo{LaTeX}%
    \HOLOGO@Span{2}{2}%
    \HOLOGO@Span{e}{%
      \HOLOGO@MathSetup
      \ensuremath{\textstyle\varepsilon}%
    }%
  }%
}
%    \end{macrocode}
%    \end{macro}
%    \begin{macro}{\HoLogoCss@LaTeXe}
%    \begin{macrocode}
\def\HoLogoCss@LaTeXe{%
  \Css{%
    span.HoLogo-LaTeX2e span.HoLogo-2{%
      padding-left:.15em;%
    }%
  }%
  \Css{%
    span.HoLogo-LaTeX2e span.HoLogo-e{%
      position:relative;%
      top:.35ex;%
      text-decoration:none;%
    }%
  }%
  \global\let\HoLogoCss@LaTeXe\relax
}
%    \end{macrocode}
%    \end{macro}
%
%    \begin{macro}{\HoLogo@LaTeX2e}
%    \begin{macrocode}
\expandafter
\let\csname HoLogo@LaTeX2e\endcsname\HoLogo@LaTeXe
%    \end{macrocode}
%    \end{macro}
%    \begin{macro}{\HoLogoCs@LaTeX2e}
%    \begin{macrocode}
\expandafter
\let\csname HoLogoCs@LaTeX2e\endcsname\HoLogoCs@LaTeXe
%    \end{macrocode}
%    \end{macro}
%    \begin{macro}{\HoLogoBkm@LaTeX2e}
%    \begin{macrocode}
\expandafter
\let\csname HoLogoBkm@LaTeX2e\endcsname\HoLogoBkm@LaTeXe
%    \end{macrocode}
%    \end{macro}
%    \begin{macro}{\HoLogoHtml@LaTeX2e}
%    \begin{macrocode}
\expandafter
\let\csname HoLogoHtml@LaTeX2e\endcsname\HoLogoHtml@LaTeXe
%    \end{macrocode}
%    \end{macro}
%
% \subsubsection{\hologo{LaTeX3}}
%
%    \begin{macro}{\HoLogo@LaTeX3}
%    Source: \hologo{LaTeX} kernel
%    \begin{macrocode}
\expandafter\def\csname HoLogo@LaTeX3\endcsname#1{%
  \hologo{LaTeX}%
  3%
}
%    \end{macrocode}
%    \end{macro}
%
%    \begin{macro}{\HoLogoBkm@LaTeX3}
%    \begin{macrocode}
\expandafter\def\csname HoLogoBkm@LaTeX3\endcsname#1{%
  \hologo{LaTeX}%
  3%
}
%    \end{macrocode}
%    \end{macro}
%    \begin{macro}{\HoLogoHtml@LaTeX3}
%    \begin{macrocode}
\expandafter
\let\csname HoLogoHtml@LaTeX3\expandafter\endcsname
\csname HoLogo@LaTeX3\endcsname
%    \end{macrocode}
%    \end{macro}
%
% \subsubsection{\hologo{LaTeXML}}
%
%    \begin{macro}{\HoLogo@LaTeXML}
%    \begin{macrocode}
\def\HoLogo@LaTeXML#1{%
  \HOLOGO@mbox{%
    \hologo{La}%
    \kern-.15em%
    T%
    \kern-.1667em%
    \lower.5ex\hbox{E}%
    \kern-.125em%
    \HoLogoFont@font{LaTeXML}{sc}{xml}%
  }%
}
%    \end{macrocode}
%    \end{macro}
%    \begin{macro}{\HoLogoHtml@pdfLaTeX}
%    \begin{macrocode}
\def\HoLogoHtml@LaTeXML#1{%
  \HOLOGO@Span{LaTeXML}{%
    \HoLogoCss@LaTeX
    \HoLogoCss@TeX
    \HOLOGO@Span{LaTeX}{%
      L%
      \HOLOGO@Span{a}{%
        A%
      }%
    }%
    \HOLOGO@Span{TeX}{%
      T%
      \HOLOGO@Span{e}{%
        E%
      }%
    }%
    \HCode{<span style="font-variant: small-caps;">}%
    xml%
    \HCode{</span>}%
  }%
}
%    \end{macrocode}
%    \end{macro}
%
% \subsubsection{\hologo{eTeX}}
%
%    \begin{macro}{\HoLogo@eTeX}
%    Source: package \xpackage{etex}
%    \begin{macrocode}
\def\HoLogo@eTeX#1{%
  \ltx@mbox{%
    \HOLOGO@MathSetup
    $\varepsilon$%
    -%
    \HOLOGO@NegativeKerning{-T,T-,To}%
    \hologo{TeX}%
  }%
}
%    \end{macrocode}
%    \end{macro}
%    \begin{macro}{\HoLogoCs@eTeX}
%    \begin{macrocode}
\ifnum64=`\^^^^0040\relax % test for big chars of LuaTeX/XeTeX
  \catcode`\$=9 %
  \catcode`\&=14 %
\else
  \catcode`\$=14 %
  \catcode`\&=9 %
\fi
\def\HoLogoCs@eTeX#1{%
$ #1{\string ^^^^0395}{\string ^^^^03b5}%
& #1{e}{E}%
  TeX%
}%
\catcode`\$=3 %
\catcode`\&=4 %
%    \end{macrocode}
%    \end{macro}
%    \begin{macro}{\HoLogoBkm@eTeX}
%    \begin{macrocode}
\def\HoLogoBkm@eTeX#1{%
  \HOLOGO@PdfdocUnicode{#1{e}{E}}{\textepsilon}%
  -%
  \hologo{TeX}%
}
%    \end{macrocode}
%    \end{macro}
%    \begin{macro}{\HoLogoHtml@eTeX}
%    \begin{macrocode}
\def\HoLogoHtml@eTeX#1{%
  \ltx@mbox{%
    \HOLOGO@MathSetup
    $\varepsilon$%
    -%
    \hologo{TeX}%
  }%
}
%    \end{macrocode}
%    \end{macro}
%
% \subsubsection{\hologo{iniTeX}}
%
%    \begin{macro}{\HoLogo@iniTeX}
%    \begin{macrocode}
\def\HoLogo@iniTeX#1{%
  \HOLOGO@mbox{%
    #1{i}{I}ni\hologo{TeX}%
  }%
}
%    \end{macrocode}
%    \end{macro}
%    \begin{macro}{\HoLogoCs@iniTeX}
%    \begin{macrocode}
\def\HoLogoCs@iniTeX#1{#1{i}{I}niTeX}
%    \end{macrocode}
%    \end{macro}
%    \begin{macro}{\HoLogoBkm@iniTeX}
%    \begin{macrocode}
\def\HoLogoBkm@iniTeX#1{%
  #1{i}{I}ni\hologo{TeX}%
}
%    \end{macrocode}
%    \end{macro}
%    \begin{macro}{\HoLogoHtml@iniTeX}
%    \begin{macrocode}
\let\HoLogoHtml@iniTeX\HoLogo@iniTeX
%    \end{macrocode}
%    \end{macro}
%
% \subsubsection{\hologo{virTeX}}
%
%    \begin{macro}{\HoLogo@virTeX}
%    \begin{macrocode}
\def\HoLogo@virTeX#1{%
  \HOLOGO@mbox{%
    #1{v}{V}ir\hologo{TeX}%
  }%
}
%    \end{macrocode}
%    \end{macro}
%    \begin{macro}{\HoLogoCs@virTeX}
%    \begin{macrocode}
\def\HoLogoCs@virTeX#1{#1{v}{V}irTeX}
%    \end{macrocode}
%    \end{macro}
%    \begin{macro}{\HoLogoBkm@virTeX}
%    \begin{macrocode}
\def\HoLogoBkm@virTeX#1{%
  #1{v}{V}ir\hologo{TeX}%
}
%    \end{macrocode}
%    \end{macro}
%    \begin{macro}{\HoLogoHtml@virTeX}
%    \begin{macrocode}
\let\HoLogoHtml@virTeX\HoLogo@virTeX
%    \end{macrocode}
%    \end{macro}
%
% \subsubsection{\hologo{SliTeX}}
%
% \paragraph{Definitions of the three variants.}
%
%    \begin{macro}{\HoLogo@SLiTeX@lift}
%    \begin{macrocode}
\def\HoLogo@SLiTeX@lift#1{%
  \HoLogoFont@font{SliTeX}{rm}{%
    S%
    \kern-.06em%
    L%
    \kern-.18em%
    \raise.32ex\hbox{\HoLogoFont@font{SliTeX}{sc}{i}}%
    \HOLOGO@discretionary
    \kern-.06em%
    \hologo{TeX}%
  }%
}
%    \end{macrocode}
%    \end{macro}
%    \begin{macro}{\HoLogoBkm@SLiTeX@lift}
%    \begin{macrocode}
\def\HoLogoBkm@SLiTeX@lift#1{SLiTeX}
%    \end{macrocode}
%    \end{macro}
%    \begin{macro}{\HoLogoHtml@SLiTeX@lift}
%    \begin{macrocode}
\def\HoLogoHtml@SLiTeX@lift#1{%
  \HoLogoCss@SLiTeX@lift
  \HOLOGO@Span{SLiTeX-lift}{%
    \HoLogoFont@font{SliTeX}{rm}{%
      S%
      \HOLOGO@Span{L}{L}%
      \HOLOGO@Span{i}{i}%
      \hologo{TeX}%
    }%
  }%
}
%    \end{macrocode}
%    \end{macro}
%    \begin{macro}{\HoLogoCss@SLiTeX@lift}
%    \begin{macrocode}
\def\HoLogoCss@SLiTeX@lift{%
  \Css{%
    span.HoLogo-SLiTeX-lift span.HoLogo-L{%
      margin-left:-.06em;%
      margin-right:-.18em;%
    }%
  }%
  \Css{%
    span.HoLogo-SLiTeX-lift span.HoLogo-i{%
      position:relative;%
      top:-.32ex;%
      margin-right:-.06em;%
      font-variant:small-caps;%
    }%
  }%
  \global\let\HoLogoCss@SLiTeX@lift\relax
}
%    \end{macrocode}
%    \end{macro}
%
%    \begin{macro}{\HoLogo@SliTeX@simple}
%    \begin{macrocode}
\def\HoLogo@SliTeX@simple#1{%
  \HoLogoFont@font{SliTeX}{rm}{%
    \ltx@mbox{%
      \HoLogoFont@font{SliTeX}{sc}{Sli}%
    }%
    \HOLOGO@discretionary
    \hologo{TeX}%
  }%
}
%    \end{macrocode}
%    \end{macro}
%    \begin{macro}{\HoLogoBkm@SliTeX@simple}
%    \begin{macrocode}
\def\HoLogoBkm@SliTeX@simple#1{SliTeX}
%    \end{macrocode}
%    \end{macro}
%    \begin{macro}{\HoLogoHtml@SliTeX@simple}
%    \begin{macrocode}
\let\HoLogoHtml@SliTeX@simple\HoLogo@SliTeX@simple
%    \end{macrocode}
%    \end{macro}
%
%    \begin{macro}{\HoLogo@SliTeX@narrow}
%    \begin{macrocode}
\def\HoLogo@SliTeX@narrow#1{%
  \HoLogoFont@font{SliTeX}{rm}{%
    \ltx@mbox{%
      S%
      \kern-.06em%
      \HoLogoFont@font{SliTeX}{sc}{%
        l%
        \kern-.035em%
        i%
      }%
    }%
    \HOLOGO@discretionary
    \kern-.06em%
    \hologo{TeX}%
  }%
}
%    \end{macrocode}
%    \end{macro}
%    \begin{macro}{\HoLogoBkm@SliTeX@narrow}
%    \begin{macrocode}
\def\HoLogoBkm@SliTeX@narrow#1{SliTeX}
%    \end{macrocode}
%    \end{macro}
%    \begin{macro}{\HoLogoHtml@SliTeX@narrow}
%    \begin{macrocode}
\def\HoLogoHtml@SliTeX@narrow#1{%
  \HoLogoCss@SliTeX@narrow
  \HOLOGO@Span{SliTeX-narrow}{%
    \HoLogoFont@font{SliTeX}{rm}{%
      S%
        \HOLOGO@Span{l}{l}%
        \HOLOGO@Span{i}{i}%
      \hologo{TeX}%
    }%
  }%
}
%    \end{macrocode}
%    \end{macro}
%    \begin{macro}{\HoLogoCss@SliTeX@narrow}
%    \begin{macrocode}
\def\HoLogoCss@SliTeX@narrow{%
  \Css{%
    span.HoLogo-SliTeX-narrow span.HoLogo-l{%
      margin-left:-.06em;%
      margin-right:-.035em;%
      font-variant:small-caps;%
    }%
  }%
  \Css{%
    span.HoLogo-SliTeX-narrow span.HoLogo-i{%
      margin-right:-.06em;%
      font-variant:small-caps;%
    }%
  }%
  \global\let\HoLogoCss@SliTeX@narrow\relax
}
%    \end{macrocode}
%    \end{macro}
%
% \paragraph{Macro set completion.}
%
%    \begin{macro}{\HoLogo@SLiTeX@simple}
%    \begin{macrocode}
\def\HoLogo@SLiTeX@simple{\HoLogo@SliTeX@simple}
%    \end{macrocode}
%    \end{macro}
%    \begin{macro}{\HoLogoBkm@SLiTeX@simple}
%    \begin{macrocode}
\def\HoLogoBkm@SLiTeX@simple{\HoLogoBkm@SliTeX@simple}
%    \end{macrocode}
%    \end{macro}
%    \begin{macro}{\HoLogoHtml@SLiTeX@simple}
%    \begin{macrocode}
\def\HoLogoHtml@SLiTeX@simple{\HoLogoHtml@SliTeX@simple}
%    \end{macrocode}
%    \end{macro}
%
%    \begin{macro}{\HoLogo@SLiTeX@narrow}
%    \begin{macrocode}
\def\HoLogo@SLiTeX@narrow{\HoLogo@SliTeX@narrow}
%    \end{macrocode}
%    \end{macro}
%    \begin{macro}{\HoLogoBkm@SLiTeX@narrow}
%    \begin{macrocode}
\def\HoLogoBkm@SLiTeX@narrow{\HoLogoBkm@SliTeX@narrow}
%    \end{macrocode}
%    \end{macro}
%    \begin{macro}{\HoLogoHtml@SLiTeX@narrow}
%    \begin{macrocode}
\def\HoLogoHtml@SLiTeX@narrow{\HoLogoHtml@SliTeX@narrow}
%    \end{macrocode}
%    \end{macro}
%
%    \begin{macro}{\HoLogo@SliTeX@lift}
%    \begin{macrocode}
\def\HoLogo@SliTeX@lift{\HoLogo@SLiTeX@lift}
%    \end{macrocode}
%    \end{macro}
%    \begin{macro}{\HoLogoBkm@SliTeX@lift}
%    \begin{macrocode}
\def\HoLogoBkm@SliTeX@lift{\HoLogoBkm@SLiTeX@lift}
%    \end{macrocode}
%    \end{macro}
%    \begin{macro}{\HoLogoHtml@SliTeX@lift}
%    \begin{macrocode}
\def\HoLogoHtml@SliTeX@lift{\HoLogoHtml@SLiTeX@lift}
%    \end{macrocode}
%    \end{macro}
%
% \paragraph{Defaults.}
%
%    \begin{macro}{\HoLogo@SLiTeX}
%    \begin{macrocode}
\def\HoLogo@SLiTeX{\HoLogo@SLiTeX@lift}
%    \end{macrocode}
%    \end{macro}
%    \begin{macro}{\HoLogoBkm@SLiTeX}
%    \begin{macrocode}
\def\HoLogoBkm@SLiTeX{\HoLogoBkm@SLiTeX@lift}
%    \end{macrocode}
%    \end{macro}
%    \begin{macro}{\HoLogoHtml@SLiTeX}
%    \begin{macrocode}
\def\HoLogoHtml@SLiTeX{\HoLogoHtml@SLiTeX@lift}
%    \end{macrocode}
%    \end{macro}
%
%    \begin{macro}{\HoLogo@SliTeX}
%    \begin{macrocode}
\def\HoLogo@SliTeX{\HoLogo@SliTeX@narrow}
%    \end{macrocode}
%    \end{macro}
%    \begin{macro}{\HoLogoBkm@SliTeX}
%    \begin{macrocode}
\def\HoLogoBkm@SliTeX{\HoLogoBkm@SliTeX@narrow}
%    \end{macrocode}
%    \end{macro}
%    \begin{macro}{\HoLogoHtml@SliTeX}
%    \begin{macrocode}
\def\HoLogoHtml@SliTeX{\HoLogoHtml@SliTeX@narrow}
%    \end{macrocode}
%    \end{macro}
%
% \subsubsection{\hologo{LuaTeX}}
%
%    \begin{macro}{\HoLogo@LuaTeX}
%    The kerning is an idea of Hans Hagen, see mailing list
%    `luatex at tug dot org' in March 2010.
%    \begin{macrocode}
\def\HoLogo@LuaTeX#1{%
  \HOLOGO@mbox{%
    Lua%
    \HOLOGO@NegativeKerning{aT,oT,To}%
    \hologo{TeX}%
  }%
}
%    \end{macrocode}
%    \end{macro}
%    \begin{macro}{\HoLogoHtml@LuaTeX}
%    \begin{macrocode}
\let\HoLogoHtml@LuaTeX\HoLogo@LuaTeX
%    \end{macrocode}
%    \end{macro}
%
% \subsubsection{\hologo{LuaLaTeX}}
%
%    \begin{macro}{\HoLogo@LuaLaTeX}
%    \begin{macrocode}
\def\HoLogo@LuaLaTeX#1{%
  \HOLOGO@mbox{%
    Lua%
    \hologo{LaTeX}%
  }%
}
%    \end{macrocode}
%    \end{macro}
%    \begin{macro}{\HoLogoHtml@LuaLaTeX}
%    \begin{macrocode}
\let\HoLogoHtml@LuaLaTeX\HoLogo@LuaLaTeX
%    \end{macrocode}
%    \end{macro}
%
% \subsubsection{\hologo{XeTeX}, \hologo{XeLaTeX}}
%
%    \begin{macro}{\HOLOGO@IfCharExists}
%    \begin{macrocode}
\ifluatex
  \ifnum\luatexversion<36 %
  \else
    \def\HOLOGO@IfCharExists#1{%
      \ifnum
        \directlua{%
           if luaotfload and luaotfload.aux then
             if luaotfload.aux.font_has_glyph(%
                    font.current(), \number#1) then % 	 
	       tex.print("1") % 	 
	     end % 	 
	   elseif font and font.fonts and font.current then %
            local f = font.fonts[font.current()]%
            if f.characters and f.characters[\number#1] then %
              tex.print("1")%
            end %
          end%
        }0=\ltx@zero
        \expandafter\ltx@secondoftwo
      \else
        \expandafter\ltx@firstoftwo
      \fi
    }%
  \fi
\fi
\ltx@IfUndefined{HOLOGO@IfCharExists}{%
  \def\HOLOGO@@IfCharExists#1{%
    \begingroup
      \tracinglostchars=\ltx@zero
      \setbox\ltx@zero=\hbox{%
        \kern7sp\char#1\relax
        \ifnum\lastkern>\ltx@zero
          \expandafter\aftergroup\csname iffalse\endcsname
        \else
          \expandafter\aftergroup\csname iftrue\endcsname
        \fi
      }%
      % \if{true|false} from \aftergroup
      \endgroup
      \expandafter\ltx@firstoftwo
    \else
      \endgroup
      \expandafter\ltx@secondoftwo
    \fi
  }%
  \ifxetex
    \ltx@IfUndefined{XeTeXfonttype}{}{%
      \ltx@IfUndefined{XeTeXcharglyph}{}{%
        \def\HOLOGO@IfCharExists#1{%
          \ifnum\XeTeXfonttype\font>\ltx@zero
            \expandafter\ltx@firstofthree
          \else
            \expandafter\ltx@gobble
          \fi
          {%
            \ifnum\XeTeXcharglyph#1>\ltx@zero
              \expandafter\ltx@firstoftwo
            \else
              \expandafter\ltx@secondoftwo
            \fi
          }%
          \HOLOGO@@IfCharExists{#1}%
        }%
      }%
    }%
  \fi
}{}
\ltx@ifundefined{HOLOGO@IfCharExists}{%
  \ifnum64=`\^^^^0040\relax % test for big chars of LuaTeX/XeTeX
    \let\HOLOGO@IfCharExists\HOLOGO@@IfCharExists
  \else
    \def\HOLOGO@IfCharExists#1{%
      \ifnum#1>255 %
        \expandafter\ltx@fourthoffour
      \fi
      \HOLOGO@@IfCharExists{#1}%
    }%
  \fi
}{}
%    \end{macrocode}
%    \end{macro}
%
%    \begin{macro}{\HoLogo@Xe}
%    Source: package \xpackage{dtklogos}
%    \begin{macrocode}
\def\HoLogo@Xe#1{%
  X%
  \kern-.1em\relax
  \HOLOGO@IfCharExists{"018E}{%
    \lower.5ex\hbox{\char"018E}%
  }{%
    \chardef\HOLOGO@choice=\ltx@zero
    \ifdim\fontdimen\ltx@one\font>0pt %
      \ltx@IfUndefined{rotatebox}{%
        \ltx@IfUndefined{pgftext}{%
          \ltx@IfUndefined{psscalebox}{%
            \ltx@IfUndefined{HOLOGO@ScaleBox@\hologoDriver}{%
            }{%
              \chardef\HOLOGO@choice=4 %
            }%
          }{%
            \chardef\HOLOGO@choice=3 %
          }%
        }{%
          \chardef\HOLOGO@choice=2 %
        }%
      }{%
        \chardef\HOLOGO@choice=1 %
      }%
      \ifcase\HOLOGO@choice
        \HOLOGO@WarningUnsupportedDriver{Xe}%
        e%
      \or % 1: \rotatebox
        \begingroup
          \setbox\ltx@zero\hbox{\rotatebox{180}{E}}%
          \ltx@LocDimenA=\dp\ltx@zero
          \advance\ltx@LocDimenA by -.5ex\relax
          \raise\ltx@LocDimenA\box\ltx@zero
        \endgroup
      \or % 2: \pgftext
        \lower.5ex\hbox{%
          \pgfpicture
            \pgftext[rotate=180]{E}%
          \endpgfpicture
        }%
      \or % 3: \psscalebox
        \begingroup
          \setbox\ltx@zero\hbox{\psscalebox{-1 -1}{E}}%
          \ltx@LocDimenA=\dp\ltx@zero
          \advance\ltx@LocDimenA by -.5ex\relax
          \raise\ltx@LocDimenA\box\ltx@zero
        \endgroup
      \or % 4: \HOLOGO@PointReflectBox
        \lower.5ex\hbox{\HOLOGO@PointReflectBox{E}}%
      \else
        \@PackageError{hologo}{Internal error (choice/it}\@ehc
      \fi
    \else
      \ltx@IfUndefined{reflectbox}{%
        \ltx@IfUndefined{pgftext}{%
          \ltx@IfUndefined{psscalebox}{%
            \ltx@IfUndefined{HOLOGO@ScaleBox@\hologoDriver}{%
            }{%
              \chardef\HOLOGO@choice=4 %
            }%
          }{%
            \chardef\HOLOGO@choice=3 %
          }%
        }{%
          \chardef\HOLOGO@choice=2 %
        }%
      }{%
        \chardef\HOLOGO@choice=1 %
      }%
      \ifcase\HOLOGO@choice
        \HOLOGO@WarningUnsupportedDriver{Xe}%
        e%
      \or % 1: reflectbox
        \lower.5ex\hbox{%
          \reflectbox{E}%
        }%
      \or % 2: \pgftext
        \lower.5ex\hbox{%
          \pgfpicture
            \pgftransformxscale{-1}%
            \pgftext{E}%
          \endpgfpicture
        }%
      \or % 3: \psscalebox
        \lower.5ex\hbox{%
          \psscalebox{-1 1}{E}%
        }%
      \or % 4: \HOLOGO@Reflectbox
        \lower.5ex\hbox{%
          \HOLOGO@ReflectBox{E}%
        }%
      \else
        \@PackageError{hologo}{Internal error (choice/up)}\@ehc
      \fi
    \fi
  }%
}
%    \end{macrocode}
%    \end{macro}
%    \begin{macro}{\HoLogoHtml@Xe}
%    \begin{macrocode}
\def\HoLogoHtml@Xe#1{%
  \HoLogoCss@Xe
  \HOLOGO@Span{Xe}{%
    X%
    \HOLOGO@Span{e}{%
      \HCode{&\ltx@hashchar x018e;}%
    }%
  }%
}
%    \end{macrocode}
%    \end{macro}
%    \begin{macro}{\HoLogoCss@Xe}
%    \begin{macrocode}
\def\HoLogoCss@Xe{%
  \Css{%
    span.HoLogo-Xe span.HoLogo-e{%
      position:relative;%
      top:.5ex;%
      left-margin:-.1em;%
    }%
  }%
  \global\let\HoLogoCss@Xe\relax
}
%    \end{macrocode}
%    \end{macro}
%
%    \begin{macro}{\HoLogo@XeTeX}
%    \begin{macrocode}
\def\HoLogo@XeTeX#1{%
  \hologo{Xe}%
  \kern-.15em\relax
  \hologo{TeX}%
}
%    \end{macrocode}
%    \end{macro}
%
%    \begin{macro}{\HoLogoHtml@XeTeX}
%    \begin{macrocode}
\def\HoLogoHtml@XeTeX#1{%
  \HoLogoCss@XeTeX
  \HOLOGO@Span{XeTeX}{%
    \hologo{Xe}%
    \hologo{TeX}%
  }%
}
%    \end{macrocode}
%    \end{macro}
%    \begin{macro}{\HoLogoCss@XeTeX}
%    \begin{macrocode}
\def\HoLogoCss@XeTeX{%
  \Css{%
    span.HoLogo-XeTeX span.HoLogo-TeX{%
      margin-left:-.15em;%
    }%
  }%
  \global\let\HoLogoCss@XeTeX\relax
}
%    \end{macrocode}
%    \end{macro}
%
%    \begin{macro}{\HoLogo@XeLaTeX}
%    \begin{macrocode}
\def\HoLogo@XeLaTeX#1{%
  \hologo{Xe}%
  \kern-.13em%
  \hologo{LaTeX}%
}
%    \end{macrocode}
%    \end{macro}
%    \begin{macro}{\HoLogoHtml@XeLaTeX}
%    \begin{macrocode}
\def\HoLogoHtml@XeLaTeX#1{%
  \HoLogoCss@XeLaTeX
  \HOLOGO@Span{XeLaTeX}{%
    \hologo{Xe}%
    \hologo{LaTeX}%
  }%
}
%    \end{macrocode}
%    \end{macro}
%    \begin{macro}{\HoLogoCss@XeLaTeX}
%    \begin{macrocode}
\def\HoLogoCss@XeLaTeX{%
  \Css{%
    span.HoLogo-XeLaTeX span.HoLogo-Xe{%
      margin-right:-.13em;%
    }%
  }%
  \global\let\HoLogoCss@XeLaTeX\relax
}
%    \end{macrocode}
%    \end{macro}
%
% \subsubsection{\hologo{pdfTeX}, \hologo{pdfLaTeX}}
%
%    \begin{macro}{\HoLogo@pdfTeX}
%    \begin{macrocode}
\def\HoLogo@pdfTeX#1{%
  \HOLOGO@mbox{%
    #1{p}{P}df\hologo{TeX}%
  }%
}
%    \end{macrocode}
%    \end{macro}
%    \begin{macro}{\HoLogoCs@pdfTeX}
%    \begin{macrocode}
\def\HoLogoCs@pdfTeX#1{#1{p}{P}dfTeX}
%    \end{macrocode}
%    \end{macro}
%    \begin{macro}{\HoLogoBkm@pdfTeX}
%    \begin{macrocode}
\def\HoLogoBkm@pdfTeX#1{%
  #1{p}{P}df\hologo{TeX}%
}
%    \end{macrocode}
%    \end{macro}
%    \begin{macro}{\HoLogoHtml@pdfTeX}
%    \begin{macrocode}
\let\HoLogoHtml@pdfTeX\HoLogo@pdfTeX
%    \end{macrocode}
%    \end{macro}
%
%    \begin{macro}{\HoLogo@pdfLaTeX}
%    \begin{macrocode}
\def\HoLogo@pdfLaTeX#1{%
  \HOLOGO@mbox{%
    #1{p}{P}df\hologo{LaTeX}%
  }%
}
%    \end{macrocode}
%    \end{macro}
%    \begin{macro}{\HoLogoCs@pdfLaTeX}
%    \begin{macrocode}
\def\HoLogoCs@pdfLaTeX#1{#1{p}{P}dfLaTeX}
%    \end{macrocode}
%    \end{macro}
%    \begin{macro}{\HoLogoBkm@pdfLaTeX}
%    \begin{macrocode}
\def\HoLogoBkm@pdfLaTeX#1{%
  #1{p}{P}df\hologo{LaTeX}%
}
%    \end{macrocode}
%    \end{macro}
%    \begin{macro}{\HoLogoHtml@pdfLaTeX}
%    \begin{macrocode}
\let\HoLogoHtml@pdfLaTeX\HoLogo@pdfLaTeX
%    \end{macrocode}
%    \end{macro}
%
% \subsubsection{\hologo{VTeX}}
%
%    \begin{macro}{\HoLogo@VTeX}
%    \begin{macrocode}
\def\HoLogo@VTeX#1{%
  \HOLOGO@mbox{%
    V\hologo{TeX}%
  }%
}
%    \end{macrocode}
%    \end{macro}
%    \begin{macro}{\HoLogoHtml@VTeX}
%    \begin{macrocode}
\let\HoLogoHtml@VTeX\HoLogo@VTeX
%    \end{macrocode}
%    \end{macro}
%
% \subsubsection{\hologo{AmS}, \dots}
%
%    Source: class \xclass{amsdtx}
%
%    \begin{macro}{\HoLogo@AmS}
%    \begin{macrocode}
\def\HoLogo@AmS#1{%
  \HoLogoFont@font{AmS}{sy}{%
    A%
    \kern-.1667em%
    \lower.5ex\hbox{M}%
    \kern-.125em%
    S%
  }%
}
%    \end{macrocode}
%    \end{macro}
%    \begin{macro}{\HoLogoBkm@AmS}
%    \begin{macrocode}
\def\HoLogoBkm@AmS#1{AmS}
%    \end{macrocode}
%    \end{macro}
%    \begin{macro}{\HoLogoHtml@AmS}
%    \begin{macrocode}
\def\HoLogoHtml@AmS#1{%
  \HoLogoCss@AmS
%  \HoLogoFont@font{AmS}{sy}{%
    \HOLOGO@Span{AmS}{%
      A%
      \HOLOGO@Span{M}{M}%
      S%
    }%
%   }%
}
%    \end{macrocode}
%    \end{macro}
%    \begin{macro}{\HoLogoCss@AmS}
%    \begin{macrocode}
\def\HoLogoCss@AmS{%
  \Css{%
    span.HoLogo-AmS span.HoLogo-M{%
      position:relative;%
      top:.5ex;%
      margin-left:-.1667em;%
      margin-right:-.125em;%
      text-decoration:none;%
    }%
  }%
  \global\let\HoLogoCss@AmS\relax
}
%    \end{macrocode}
%    \end{macro}
%
%    \begin{macro}{\HoLogo@AmSTeX}
%    \begin{macrocode}
\def\HoLogo@AmSTeX#1{%
  \hologo{AmS}%
  \HOLOGO@hyphen
  \hologo{TeX}%
}
%    \end{macrocode}
%    \end{macro}
%    \begin{macro}{\HoLogoBkm@AmSTeX}
%    \begin{macrocode}
\def\HoLogoBkm@AmSTeX#1{AmS-TeX}%
%    \end{macrocode}
%    \end{macro}
%    \begin{macro}{\HoLogoHtml@AmSTeX}
%    \begin{macrocode}
\let\HoLogoHtml@AmSTeX\HoLogo@AmSTeX
%    \end{macrocode}
%    \end{macro}
%
%    \begin{macro}{\HoLogo@AmSLaTeX}
%    \begin{macrocode}
\def\HoLogo@AmSLaTeX#1{%
  \hologo{AmS}%
  \HOLOGO@hyphen
  \hologo{LaTeX}%
}
%    \end{macrocode}
%    \end{macro}
%    \begin{macro}{\HoLogoBkm@AmSLaTeX}
%    \begin{macrocode}
\def\HoLogoBkm@AmSLaTeX#1{AmS-LaTeX}%
%    \end{macrocode}
%    \end{macro}
%    \begin{macro}{\HoLogoHtml@AmSLaTeX}
%    \begin{macrocode}
\let\HoLogoHtml@AmSLaTeX\HoLogo@AmSLaTeX
%    \end{macrocode}
%    \end{macro}
%
% \subsubsection{\hologo{BibTeX}}
%
%    \begin{macro}{\HoLogo@BibTeX@sc}
%    A definition of \hologo{BibTeX} is provided in
%    the documentation source for the manual of \hologo{BibTeX}
%    \cite{btxdoc}.
%\begin{quote}
%\begin{verbatim}
%\def\BibTeX{%
%  {%
%    \rm
%    B%
%    \kern-.05em%
%    {%
%      \sc
%      i%
%      \kern-.025em %
%      b%
%    }%
%    \kern-.08em
%    T%
%    \kern-.1667em%
%    \lower.7ex\hbox{E}%
%    \kern-.125em%
%    X%
%  }%
%}
%\end{verbatim}
%\end{quote}
%    \begin{macrocode}
\def\HoLogo@BibTeX@sc#1{%
  B%
  \kern-.05em%
  \HoLogoFont@font{BibTeX}{sc}{%
    i%
    \kern-.025em%
    b%
  }%
  \HOLOGO@discretionary
  \kern-.08em%
  \hologo{TeX}%
}
%    \end{macrocode}
%    \end{macro}
%    \begin{macro}{\HoLogoHtml@BibTeX@sc}
%    \begin{macrocode}
\def\HoLogoHtml@BibTeX@sc#1{%
  \HoLogoCss@BibTeX@sc
  \HOLOGO@Span{BibTeX-sc}{%
    B%
    \HOLOGO@Span{i}{i}%
    \HOLOGO@Span{b}{b}%
    \hologo{TeX}%
  }%
}
%    \end{macrocode}
%    \end{macro}
%    \begin{macro}{\HoLogoCss@BibTeX@sc}
%    \begin{macrocode}
\def\HoLogoCss@BibTeX@sc{%
  \Css{%
    span.HoLogo-BibTeX-sc span.HoLogo-i{%
      margin-left:-.05em;%
      margin-right:-.025em;%
      font-variant:small-caps;%
    }%
  }%
  \Css{%
    span.HoLogo-BibTeX-sc span.HoLogo-b{%
      margin-right:-.08em;%
      font-variant:small-caps;%
    }%
  }%
  \global\let\HoLogoCss@BibTeX@sc\relax
}
%    \end{macrocode}
%    \end{macro}
%
%    \begin{macro}{\HoLogo@BibTeX@sf}
%    Variant \xoption{sf} avoids trouble with unavailable
%    small caps fonts (e.g., bold versions of Computer Modern or
%    Latin Modern). The definition is taken from
%    package \xpackage{dtklogos} \cite{dtklogos}.
%\begin{quote}
%\begin{verbatim}
%\DeclareRobustCommand{\BibTeX}{%
%  B%
%  \kern-.05em%
%  \hbox{%
%    $\m@th$% %% force math size calculations
%    \csname S@\f@size\endcsname
%    \fontsize\sf@size\z@
%    \math@fontsfalse
%    \selectfont
%    I%
%    \kern-.025em%
%    B
%  }%
%  \kern-.08em%
%  \-%
%  \TeX
%}
%\end{verbatim}
%\end{quote}
%    \begin{macrocode}
\def\HoLogo@BibTeX@sf#1{%
  B%
  \kern-.05em%
  \HoLogoFont@font{BibTeX}{bibsf}{%
    I%
    \kern-.025em%
    B%
  }%
  \HOLOGO@discretionary
  \kern-.08em%
  \hologo{TeX}%
}
%    \end{macrocode}
%    \end{macro}
%    \begin{macro}{\HoLogoHtml@BibTeX@sf}
%    \begin{macrocode}
\def\HoLogoHtml@BibTeX@sf#1{%
  \HoLogoCss@BibTeX@sf
  \HOLOGO@Span{BibTeX-sf}{%
    B%
    \HoLogoFont@font{BibTeX}{bibsf}{%
      \HOLOGO@Span{i}{I}%
      B%
    }%
    \hologo{TeX}%
  }%
}
%    \end{macrocode}
%    \end{macro}
%    \begin{macro}{\HoLogoCss@BibTeX@sf}
%    \begin{macrocode}
\def\HoLogoCss@BibTeX@sf{%
  \Css{%
    span.HoLogo-BibTeX-sf span.HoLogo-i{%
      margin-left:-.05em;%
      margin-right:-.025em;%
    }%
  }%
  \Css{%
    span.HoLogo-BibTeX-sf span.HoLogo-TeX{%
      margin-left:-.08em;%
    }%
  }%
  \global\let\HoLogoCss@BibTeX@sf\relax
}
%    \end{macrocode}
%    \end{macro}
%
%    \begin{macro}{\HoLogo@BibTeX}
%    \begin{macrocode}
\def\HoLogo@BibTeX{\HoLogo@BibTeX@sf}
%    \end{macrocode}
%    \end{macro}
%    \begin{macro}{\HoLogoHtml@BibTeX}
%    \begin{macrocode}
\def\HoLogoHtml@BibTeX{\HoLogoHtml@BibTeX@sf}
%    \end{macrocode}
%    \end{macro}
%
% \subsubsection{\hologo{BibTeX8}}
%
%    \begin{macro}{\HoLogo@BibTeX8}
%    \begin{macrocode}
\expandafter\def\csname HoLogo@BibTeX8\endcsname#1{%
  \hologo{BibTeX}%
  8%
}
%    \end{macrocode}
%    \end{macro}
%
%    \begin{macro}{\HoLogoBkm@BibTeX8}
%    \begin{macrocode}
\expandafter\def\csname HoLogoBkm@BibTeX8\endcsname#1{%
  \hologo{BibTeX}%
  8%
}
%    \end{macrocode}
%    \end{macro}
%    \begin{macro}{\HoLogoHtml@BibTeX8}
%    \begin{macrocode}
\expandafter
\let\csname HoLogoHtml@BibTeX8\expandafter\endcsname
\csname HoLogo@BibTeX8\endcsname
%    \end{macrocode}
%    \end{macro}
%
% \subsubsection{\hologo{ConTeXt}}
%
%    \begin{macro}{\HoLogo@ConTeXt@simple}
%    \begin{macrocode}
\def\HoLogo@ConTeXt@simple#1{%
  \HOLOGO@mbox{Con}%
  \HOLOGO@discretionary
  \HOLOGO@mbox{\hologo{TeX}t}%
}
%    \end{macrocode}
%    \end{macro}
%    \begin{macro}{\HoLogoHtml@ConTeXt@simple}
%    \begin{macrocode}
\let\HoLogoHtml@ConTeXt@simple\HoLogo@ConTeXt@simple
%    \end{macrocode}
%    \end{macro}
%
%    \begin{macro}{\HoLogo@ConTeXt@narrow}
%    This definition of logo \hologo{ConTeXt} with variant \xoption{narrow}
%    comes from TUGboat's class \xclass{ltugboat} (version 2010/11/15 v2.8).
%    \begin{macrocode}
\def\HoLogo@ConTeXt@narrow#1{%
  \HOLOGO@mbox{C\kern-.0333emon}%
  \HOLOGO@discretionary
  \kern-.0667em%
  \HOLOGO@mbox{\hologo{TeX}\kern-.0333emt}%
}
%    \end{macrocode}
%    \end{macro}
%    \begin{macro}{\HoLogoHtml@ConTeXt@narrow}
%    \begin{macrocode}
\def\HoLogoHtml@ConTeXt@narrow#1{%
  \HoLogoCss@ConTeXt@narrow
  \HOLOGO@Span{ConTeXt-narrow}{%
    \HOLOGO@Span{C}{C}%
    on%
    \hologo{TeX}%
    t%
  }%
}
%    \end{macrocode}
%    \end{macro}
%    \begin{macro}{\HoLogoCss@ConTeXt@narrow}
%    \begin{macrocode}
\def\HoLogoCss@ConTeXt@narrow{%
  \Css{%
    span.HoLogo-ConTeXt-narrow span.HoLogo-C{%
      margin-left:-.0333em;%
    }%
  }%
  \Css{%
    span.HoLogo-ConTeXt-narrow span.HoLogo-TeX{%
      margin-left:-.0667em;%
      margin-right:-.0333em;%
    }%
  }%
  \global\let\HoLogoCss@ConTeXt@narrow\relax
}
%    \end{macrocode}
%    \end{macro}
%
%    \begin{macro}{\HoLogo@ConTeXt}
%    \begin{macrocode}
\def\HoLogo@ConTeXt{\HoLogo@ConTeXt@narrow}
%    \end{macrocode}
%    \end{macro}
%    \begin{macro}{\HoLogoHtml@ConTeXt}
%    \begin{macrocode}
\def\HoLogoHtml@ConTeXt{\HoLogoHtml@ConTeXt@narrow}
%    \end{macrocode}
%    \end{macro}
%
% \subsubsection{\hologo{emTeX}}
%
%    \begin{macro}{\HoLogo@emTeX}
%    \begin{macrocode}
\def\HoLogo@emTeX#1{%
  \HOLOGO@mbox{#1{e}{E}m}%
  \HOLOGO@discretionary
  \hologo{TeX}%
}
%    \end{macrocode}
%    \end{macro}
%    \begin{macro}{\HoLogoCs@emTeX}
%    \begin{macrocode}
\def\HoLogoCs@emTeX#1{#1{e}{E}mTeX}%
%    \end{macrocode}
%    \end{macro}
%    \begin{macro}{\HoLogoBkm@emTeX}
%    \begin{macrocode}
\def\HoLogoBkm@emTeX#1{%
  #1{e}{E}m\hologo{TeX}%
}
%    \end{macrocode}
%    \end{macro}
%    \begin{macro}{\HoLogoHtml@emTeX}
%    \begin{macrocode}
\let\HoLogoHtml@emTeX\HoLogo@emTeX
%    \end{macrocode}
%    \end{macro}
%
% \subsubsection{\hologo{ExTeX}}
%
%    \begin{macro}{\HoLogo@ExTeX}
%    The definition is taken from the FAQ of the
%    project \hologo{ExTeX}
%    \cite{ExTeX-FAQ}.
%\begin{quote}
%\begin{verbatim}
%\def\ExTeX{%
%  \textrm{% Logo always with serifs
%    \ensuremath{%
%      \textstyle
%      \varepsilon_{%
%        \kern-0.15em%
%        \mathcal{X}%
%      }%
%    }%
%    \kern-.15em%
%    \TeX
%  }%
%}
%\end{verbatim}
%\end{quote}
%    \begin{macrocode}
\def\HoLogo@ExTeX#1{%
  \HoLogoFont@font{ExTeX}{rm}{%
    \ltx@mbox{%
      \HOLOGO@MathSetup
      $%
        \textstyle
        \varepsilon_{%
          \kern-0.15em%
          \HoLogoFont@font{ExTeX}{sy}{X}%
        }%
      $%
    }%
    \HOLOGO@discretionary
    \kern-.15em%
    \hologo{TeX}%
  }%
}
%    \end{macrocode}
%    \end{macro}
%    \begin{macro}{\HoLogoHtml@ExTeX}
%    \begin{macrocode}
\def\HoLogoHtml@ExTeX#1{%
  \HoLogoCss@ExTeX
  \HoLogoFont@font{ExTeX}{rm}{%
    \HOLOGO@Span{ExTeX}{%
      \ltx@mbox{%
        \HOLOGO@MathSetup
        $\textstyle\varepsilon$%
        \HOLOGO@Span{X}{$\textstyle\chi$}%
        \hologo{TeX}%
      }%
    }%
  }%
}
%    \end{macrocode}
%    \end{macro}
%    \begin{macro}{\HoLogoBkm@ExTeX}
%    \begin{macrocode}
\def\HoLogoBkm@ExTeX#1{%
  \HOLOGO@PdfdocUnicode{#1{e}{E}x}{\textepsilon\textchi}%
  \hologo{TeX}%
}
%    \end{macrocode}
%    \end{macro}
%    \begin{macro}{\HoLogoCss@ExTeX}
%    \begin{macrocode}
\def\HoLogoCss@ExTeX{%
  \Css{%
    span.HoLogo-ExTeX{%
      font-family:serif;%
    }%
  }%
  \Css{%
    span.HoLogo-ExTeX span.HoLogo-TeX{%
      margin-left:-.15em;%
    }%
  }%
  \global\let\HoLogoCss@ExTeX\relax
}
%    \end{macrocode}
%    \end{macro}
%
% \subsubsection{\hologo{MiKTeX}}
%
%    \begin{macro}{\HoLogo@MiKTeX}
%    \begin{macrocode}
\def\HoLogo@MiKTeX#1{%
  \HOLOGO@mbox{MiK}%
  \HOLOGO@discretionary
  \hologo{TeX}%
}
%    \end{macrocode}
%    \end{macro}
%    \begin{macro}{\HoLogoHtml@MiKTeX}
%    \begin{macrocode}
\let\HoLogoHtml@MiKTeX\HoLogo@MiKTeX
%    \end{macrocode}
%    \end{macro}
%
% \subsubsection{\hologo{OzTeX} and friends}
%
%    Source: \hologo{OzTeX} FAQ \cite{OzTeX}:
%    \begin{quote}
%      |\def\OzTeX{O\kern-.03em z\kern-.15em\TeX}|\\
%      (There is no kerning in OzMF, OzMP and OzTtH.)
%    \end{quote}
%
%    \begin{macro}{\HoLogo@OzTeX}
%    \begin{macrocode}
\def\HoLogo@OzTeX#1{%
  O%
  \kern-.03em %
  z%
  \kern-.15em %
  \hologo{TeX}%
}
%    \end{macrocode}
%    \end{macro}
%    \begin{macro}{\HoLogoHtml@OzTeX}
%    \begin{macrocode}
\def\HoLogoHtml@OzTeX#1{%
  \HoLogoCss@OzTeX
  \HOLOGO@Span{OzTeX}{%
    O%
    \HOLOGO@Span{z}{z}%
    \hologo{TeX}%
  }%
}
%    \end{macrocode}
%    \end{macro}
%    \begin{macro}{\HoLogoCss@OzTeX}
%    \begin{macrocode}
\def\HoLogoCss@OzTeX{%
  \Css{%
    span.HoLogo-OzTeX span.HoLogo-z{%
      margin-left:-.03em;%
      margin-right:-.15em;%
    }%
  }%
  \global\let\HoLogoCss@OzTeX\relax
}
%    \end{macrocode}
%    \end{macro}
%
%    \begin{macro}{\HoLogo@OzMF}
%    \begin{macrocode}
\def\HoLogo@OzMF#1{%
  \HOLOGO@mbox{OzMF}%
}
%    \end{macrocode}
%    \end{macro}
%    \begin{macro}{\HoLogo@OzMP}
%    \begin{macrocode}
\def\HoLogo@OzMP#1{%
  \HOLOGO@mbox{OzMP}%
}
%    \end{macrocode}
%    \end{macro}
%    \begin{macro}{\HoLogo@OzTtH}
%    \begin{macrocode}
\def\HoLogo@OzTtH#1{%
  \HOLOGO@mbox{OzTtH}%
}
%    \end{macrocode}
%    \end{macro}
%
% \subsubsection{\hologo{PCTeX}}
%
%    \begin{macro}{\HoLogo@PCTeX}
%    \begin{macrocode}
\def\HoLogo@PCTeX#1{%
  \HOLOGO@mbox{PC}%
  \hologo{TeX}%
}
%    \end{macrocode}
%    \end{macro}
%    \begin{macro}{\HoLogoHtml@PCTeX}
%    \begin{macrocode}
\let\HoLogoHtml@PCTeX\HoLogo@PCTeX
%    \end{macrocode}
%    \end{macro}
%
% \subsubsection{\hologo{PiCTeX}}
%
%    The original definitions from \xfile{pictex.tex} \cite{PiCTeX}:
%\begin{quote}
%\begin{verbatim}
%\def\PiC{%
%  P%
%  \kern-.12em%
%  \lower.5ex\hbox{I}%
%  \kern-.075em%
%  C%
%}
%\def\PiCTeX{%
%  \PiC
%  \kern-.11em%
%  \TeX
%}
%\end{verbatim}
%\end{quote}
%
%    \begin{macro}{\HoLogo@PiC}
%    \begin{macrocode}
\def\HoLogo@PiC#1{%
  P%
  \kern-.12em%
  \lower.5ex\hbox{I}%
  \kern-.075em%
  C%
  \HOLOGO@SpaceFactor
}
%    \end{macrocode}
%    \end{macro}
%    \begin{macro}{\HoLogoHtml@PiC}
%    \begin{macrocode}
\def\HoLogoHtml@PiC#1{%
  \HoLogoCss@PiC
  \HOLOGO@Span{PiC}{%
    P%
    \HOLOGO@Span{i}{I}%
    C%
  }%
}
%    \end{macrocode}
%    \end{macro}
%    \begin{macro}{\HoLogoCss@PiC}
%    \begin{macrocode}
\def\HoLogoCss@PiC{%
  \Css{%
    span.HoLogo-PiC span.HoLogo-i{%
      position:relative;%
      top:.5ex;%
      margin-left:-.12em;%
      margin-right:-.075em;%
      text-decoration:none;%
    }%
  }%
  \global\let\HoLogoCss@PiC\relax
}
%    \end{macrocode}
%    \end{macro}
%
%    \begin{macro}{\HoLogo@PiCTeX}
%    \begin{macrocode}
\def\HoLogo@PiCTeX#1{%
  \hologo{PiC}%
  \HOLOGO@discretionary
  \kern-.11em%
  \hologo{TeX}%
}
%    \end{macrocode}
%    \end{macro}
%    \begin{macro}{\HoLogoHtml@PiCTeX}
%    \begin{macrocode}
\def\HoLogoHtml@PiCTeX#1{%
  \HoLogoCss@PiCTeX
  \HOLOGO@Span{PiCTeX}{%
    \hologo{PiC}%
    \hologo{TeX}%
  }%
}
%    \end{macrocode}
%    \end{macro}
%    \begin{macro}{\HoLogoCss@PiCTeX}
%    \begin{macrocode}
\def\HoLogoCss@PiCTeX{%
  \Css{%
    span.HoLogo-PiCTeX span.HoLogo-PiC{%
      margin-right:-.11em;%
    }%
  }%
  \global\let\HoLogoCss@PiCTeX\relax
}
%    \end{macrocode}
%    \end{macro}
%
% \subsubsection{\hologo{teTeX}}
%
%    \begin{macro}{\HoLogo@teTeX}
%    \begin{macrocode}
\def\HoLogo@teTeX#1{%
  \HOLOGO@mbox{#1{t}{T}e}%
  \HOLOGO@discretionary
  \hologo{TeX}%
}
%    \end{macrocode}
%    \end{macro}
%    \begin{macro}{\HoLogoCs@teTeX}
%    \begin{macrocode}
\def\HoLogoCs@teTeX#1{#1{t}{T}dfTeX}
%    \end{macrocode}
%    \end{macro}
%    \begin{macro}{\HoLogoBkm@teTeX}
%    \begin{macrocode}
\def\HoLogoBkm@teTeX#1{%
  #1{t}{T}e\hologo{TeX}%
}
%    \end{macrocode}
%    \end{macro}
%    \begin{macro}{\HoLogoHtml@teTeX}
%    \begin{macrocode}
\let\HoLogoHtml@teTeX\HoLogo@teTeX
%    \end{macrocode}
%    \end{macro}
%
% \subsubsection{\hologo{TeX4ht}}
%
%    \begin{macro}{\HoLogo@TeX4ht}
%    \begin{macrocode}
\expandafter\def\csname HoLogo@TeX4ht\endcsname#1{%
  \HOLOGO@mbox{\hologo{TeX}4ht}%
}
%    \end{macrocode}
%    \end{macro}
%    \begin{macro}{\HoLogoHtml@TeX4ht}
%    \begin{macrocode}
\expandafter
\let\csname HoLogoHtml@TeX4ht\expandafter\endcsname
\csname HoLogo@TeX4ht\endcsname
%    \end{macrocode}
%    \end{macro}
%
%
% \subsubsection{\hologo{SageTeX}}
%
%    \begin{macro}{\HoLogo@SageTeX}
%    \begin{macrocode}
\def\HoLogo@SageTeX#1{%
  \HOLOGO@mbox{Sage}%
  \HOLOGO@discretionary
  \HOLOGO@NegativeKerning{eT,oT,To}%
  \hologo{TeX}%
}
%    \end{macrocode}
%    \end{macro}
%    \begin{macro}{\HoLogoHtml@SageTeX}
%    \begin{macrocode}
\let\HoLogoHtml@SageTeX\HoLogo@SageTeX
%    \end{macrocode}
%    \end{macro}
%
% \subsection{\hologo{METAFONT} and friends}
%
%    \begin{macro}{\HoLogo@METAFONT}
%    \begin{macrocode}
\def\HoLogo@METAFONT#1{%
  \HoLogoFont@font{METAFONT}{logo}{%
    \HOLOGO@mbox{META}%
    \HOLOGO@discretionary
    \HOLOGO@mbox{FONT}%
  }%
}
%    \end{macrocode}
%    \end{macro}
%
%    \begin{macro}{\HoLogo@METAPOST}
%    \begin{macrocode}
\def\HoLogo@METAPOST#1{%
  \HoLogoFont@font{METAPOST}{logo}{%
    \HOLOGO@mbox{META}%
    \HOLOGO@discretionary
    \HOLOGO@mbox{POST}%
  }%
}
%    \end{macrocode}
%    \end{macro}
%
%    \begin{macro}{\HoLogo@MetaFun}
%    \begin{macrocode}
\def\HoLogo@MetaFun#1{%
  \HOLOGO@mbox{Meta}%
  \HOLOGO@discretionary
  \HOLOGO@mbox{Fun}%
}
%    \end{macrocode}
%    \end{macro}
%
%    \begin{macro}{\HoLogo@MetaPost}
%    \begin{macrocode}
\def\HoLogo@MetaPost#1{%
  \HOLOGO@mbox{Meta}%
  \HOLOGO@discretionary
  \HOLOGO@mbox{Post}%
}
%    \end{macrocode}
%    \end{macro}
%
% \subsection{Others}
%
% \subsubsection{\hologo{biber}}
%
%    \begin{macro}{\HoLogo@biber}
%    \begin{macrocode}
\def\HoLogo@biber#1{%
  \HOLOGO@mbox{#1{b}{B}i}%
  \HOLOGO@discretionary
  \HOLOGO@mbox{ber}%
}
%    \end{macrocode}
%    \end{macro}
%    \begin{macro}{\HoLogoCs@biber}
%    \begin{macrocode}
\def\HoLogoCs@biber#1{#1{b}{B}iber}
%    \end{macrocode}
%    \end{macro}
%    \begin{macro}{\HoLogoBkm@biber}
%    \begin{macrocode}
\def\HoLogoBkm@biber#1{%
  #1{b}{B}iber%
}
%    \end{macrocode}
%    \end{macro}
%    \begin{macro}{\HoLogoHtml@biber}
%    \begin{macrocode}
\let\HoLogoHtml@biber\HoLogo@biber
%    \end{macrocode}
%    \end{macro}
%
% \subsubsection{\hologo{KOMAScript}}
%
%    \begin{macro}{\HoLogo@KOMAScript}
%    The definition for \hologo{KOMAScript} is taken
%    from \hologo{KOMAScript} (\xfile{scrlogo.dtx}, reformatted) \cite{scrlogo}:
%\begin{quote}
%\begin{verbatim}
%\@ifundefined{KOMAScript}{%
%  \DeclareRobustCommand{\KOMAScript}{%
%    \textsf{%
%      K\kern.05em O\kern.05emM\kern.05em A%
%      \kern.1em-\kern.1em %
%      Script%
%    }%
%  }%
%}{}
%\end{verbatim}
%\end{quote}
%    \begin{macrocode}
\def\HoLogo@KOMAScript#1{%
  \HoLogoFont@font{KOMAScript}{sf}{%
    \HOLOGO@mbox{%
      K\kern.05em%
      O\kern.05em%
      M\kern.05em%
      A%
    }%
    \kern.1em%
    \HOLOGO@hyphen
    \kern.1em%
    \HOLOGO@mbox{Script}%
  }%
}
%    \end{macrocode}
%    \end{macro}
%    \begin{macro}{\HoLogoBkm@KOMAScript}
%    \begin{macrocode}
\def\HoLogoBkm@KOMAScript#1{%
  KOMA-Script%
}
%    \end{macrocode}
%    \end{macro}
%    \begin{macro}{\HoLogoHtml@KOMAScript}
%    \begin{macrocode}
\def\HoLogoHtml@KOMAScript#1{%
  \HoLogoCss@KOMAScript
  \HoLogoFont@font{KOMAScript}{sf}{%
    \HOLOGO@Span{KOMAScript}{%
      K%
      \HOLOGO@Span{O}{O}%
      M%
      \HOLOGO@Span{A}{A}%
      \HOLOGO@Span{hyphen}{-}%
      Script%
    }%
  }%
}
%    \end{macrocode}
%    \end{macro}
%    \begin{macro}{\HoLogoCss@KOMAScript}
%    \begin{macrocode}
\def\HoLogoCss@KOMAScript{%
  \Css{%
    span.HoLogo-KOMAScript{%
      font-family:sans-serif;%
    }%
  }%
  \Css{%
    span.HoLogo-KOMAScript span.HoLogo-O{%
      padding-left:.05em;%
      padding-right:.05em;%
    }%
  }%
  \Css{%
    span.HoLogo-KOMAScript span.HoLogo-A{%
      padding-left:.05em;%
    }%
  }%
  \Css{%
    span.HoLogo-KOMAScript span.HoLogo-hyphen{%
      padding-left:.1em;%
      padding-right:.1em;%
    }%
  }%
  \global\let\HoLogoCss@KOMAScript\relax
}
%    \end{macrocode}
%    \end{macro}
%
% \subsubsection{\hologo{LyX}}
%
%    \begin{macro}{\HoLogo@LyX}
%    The definition is taken from the documentation source files
%    of \hologo{LyX}, \xfile{Intro.lyx} \cite{LyX}:
%\begin{quote}
%\begin{verbatim}
%\def\LyX{%
%  \texorpdfstring{%
%    L\kern-.1667em\lower.25em\hbox{Y}\kern-.125emX\@%
%  }{%
%    LyX%
%  }%
%}
%\end{verbatim}
%\end{quote}
%    \begin{macrocode}
\def\HoLogo@LyX#1{%
  L%
  \kern-.1667em%
  \lower.25em\hbox{Y}%
  \kern-.125em%
  X%
  \HOLOGO@SpaceFactor
}
%    \end{macrocode}
%    \end{macro}
%    \begin{macro}{\HoLogoHtml@LyX}
%    \begin{macrocode}
\def\HoLogoHtml@LyX#1{%
  \HoLogoCss@LyX
  \HOLOGO@Span{LyX}{%
    L%
    \HOLOGO@Span{y}{Y}%
    X%
  }%
}
%    \end{macrocode}
%    \end{macro}
%    \begin{macro}{\HoLogoCss@LyX}
%    \begin{macrocode}
\def\HoLogoCss@LyX{%
  \Css{%
    span.HoLogo-LyX span.HoLogo-y{%
      position:relative;%
      top:.25em;%
      margin-left:-.1667em;%
      margin-right:-.125em;%
      text-decoration:none;%
    }%
  }%
  \global\let\HoLogoCss@LyX\relax
}
%    \end{macrocode}
%    \end{macro}
%
% \subsubsection{\hologo{NTS}}
%
%    \begin{macro}{\HoLogo@NTS}
%    Definition for \hologo{NTS} can be found in
%    package \xpackage{etex\textunderscore man} for the \hologo{eTeX} manual \cite{etexman}
%    and in package \xpackage{dtklogos} \cite{dtklogos}:
%\begin{quote}
%\begin{verbatim}
%\def\NTS{%
%  \leavevmode
%  \hbox{%
%    $%
%      \cal N%
%      \kern-0.35em%
%      \lower0.5ex\hbox{$\cal T$}%
%      \kern-0.2em%
%      S%
%    $%
%  }%
%}
%\end{verbatim}
%\end{quote}
%    \begin{macrocode}
\def\HoLogo@NTS#1{%
  \HoLogoFont@font{NTS}{sy}{%
    N\/%
    \kern-.35em%
    \lower.5ex\hbox{T\/}%
    \kern-.2em%
    S\/%
  }%
  \HOLOGO@SpaceFactor
}
%    \end{macrocode}
%    \end{macro}
%
% \subsubsection{\Hologo{TTH} (\hologo{TeX} to HTML translator)}
%
%    Source: \url{http://hutchinson.belmont.ma.us/tth/}
%    In the HTML source the second `T' is printed as subscript.
%\begin{quote}
%\begin{verbatim}
%T<sub>T</sub>H
%\end{verbatim}
%\end{quote}
%    \begin{macro}{\HoLogo@TTH}
%    \begin{macrocode}
\def\HoLogo@TTH#1{%
  \ltx@mbox{%
    T\HOLOGO@SubScript{T}H%
  }%
  \HOLOGO@SpaceFactor
}
%    \end{macrocode}
%    \end{macro}
%
%    \begin{macro}{\HoLogoHtml@TTH}
%    \begin{macrocode}
\def\HoLogoHtml@TTH#1{%
  T\HCode{<sub>}T\HCode{</sub>}H%
}
%    \end{macrocode}
%    \end{macro}
%
% \subsubsection{\Hologo{HanTheThanh}}
%
%    Partial source: Package \xpackage{dtklogos}.
%    The double accent is U+1EBF (latin small letter e with circumflex
%    and acute).
%    \begin{macro}{\HoLogo@HanTheThanh}
%    \begin{macrocode}
\def\HoLogo@HanTheThanh#1{%
  \ltx@mbox{H\`an}%
  \HOLOGO@space
  \ltx@mbox{%
    Th%
    \HOLOGO@IfCharExists{"1EBF}{%
      \char"1EBF\relax
    }{%
      \^e\hbox to 0pt{\hss\raise .5ex\hbox{\'{}}}%
    }%
  }%
  \HOLOGO@space
  \ltx@mbox{Th\`anh}%
}
%    \end{macrocode}
%    \end{macro}
%    \begin{macro}{\HoLogoBkm@HanTheThanh}
%    \begin{macrocode}
\def\HoLogoBkm@HanTheThanh#1{%
  H\`an %
  Th\HOLOGO@PdfdocUnicode{\^e}{\9036\277} %
  Th\`anh%
}
%    \end{macrocode}
%    \end{macro}
%    \begin{macro}{\HoLogoHtml@HanTheThanh}
%    \begin{macrocode}
\def\HoLogoHtml@HanTheThanh#1{%
  H\`an %
  Th\HCode{&\ltx@hashchar x1ebf;} %
  Th\`anh%
}
%    \end{macrocode}
%    \end{macro}
%
% \subsection{Driver detection}
%
%    \begin{macrocode}
\HOLOGO@IfExists\InputIfFileExists{%
  \InputIfFileExists{hologo.cfg}{}{}%
}{%
  \ltx@IfUndefined{pdf@filesize}{%
    \def\HOLOGO@InputIfExists{%
      \openin\HOLOGO@temp=hologo.cfg\relax
      \ifeof\HOLOGO@temp
        \closein\HOLOGO@temp
      \else
        \closein\HOLOGO@temp
        \begingroup
          \def\x{LaTeX2e}%
        \expandafter\endgroup
        \ifx\fmtname\x
          \input{hologo.cfg}%
        \else
          \input hologo.cfg\relax
        \fi
      \fi
    }%
    \ltx@IfUndefined{newread}{%
      \chardef\HOLOGO@temp=15 %
      \def\HOLOGO@CheckRead{%
        \ifeof\HOLOGO@temp
          \HOLOGO@InputIfExists
        \else
          \ifcase\HOLOGO@temp
            \@PackageWarningNoLine{hologo}{%
              Configuration file ignored, because\MessageBreak
              a free read register could not be found%
            }%
          \else
            \begingroup
              \count\ltx@cclv=\HOLOGO@temp
              \advance\ltx@cclv by \ltx@minusone
              \edef\x{\endgroup
                \chardef\noexpand\HOLOGO@temp=\the\count\ltx@cclv
                \relax
              }%
            \x
          \fi
        \fi
      }%
    }{%
      \csname newread\endcsname\HOLOGO@temp
      \HOLOGO@InputIfExists
    }%
  }{%
    \edef\HOLOGO@temp{\pdf@filesize{hologo.cfg}}%
    \ifx\HOLOGO@temp\ltx@empty
    \else
      \ifnum\HOLOGO@temp>0 %
        \begingroup
          \def\x{LaTeX2e}%
        \expandafter\endgroup
        \ifx\fmtname\x
          \input{hologo.cfg}%
        \else
          \input hologo.cfg\relax
        \fi
      \else
        \@PackageInfoNoLine{hologo}{%
          Empty configuration file `hologo.cfg' ignored%
        }%
      \fi
    \fi
  }%
}
%    \end{macrocode}
%
%    \begin{macrocode}
\def\HOLOGO@temp#1#2{%
  \kv@define@key{HoLogoDriver}{#1}[]{%
    \begingroup
      \def\HOLOGO@temp{##1}%
      \ltx@onelevel@sanitize\HOLOGO@temp
      \ifx\HOLOGO@temp\ltx@empty
      \else
        \@PackageError{hologo}{%
          Value (\HOLOGO@temp) not permitted for option `#1'%
        }%
        \@ehc
      \fi
    \endgroup
    \def\hologoDriver{#2}%
  }%
}%
\def\HOLOGO@@temp#1#2{%
  \ifx\kv@value\relax
    \HOLOGO@temp{#1}{#1}%
  \else
    \HOLOGO@temp{#1}{#2}%
  \fi
}%
\kv@parse@normalized{%
  pdftex,%
  luatex=pdftex,%
  dvipdfm,%
  dvipdfmx=dvipdfm,%
  dvips,%
  dvipsone=dvips,%
  xdvi=dvips,%
  xetex,%
  vtex,%
}\HOLOGO@@temp
%    \end{macrocode}
%
%    \begin{macrocode}
\kv@define@key{HoLogoDriver}{driverfallback}{%
  \def\HOLOGO@DriverFallback{#1}%
}
%    \end{macrocode}
%
%    \begin{macro}{\HOLOGO@DriverFallback}
%    \begin{macrocode}
\def\HOLOGO@DriverFallback{dvips}
%    \end{macrocode}
%    \end{macro}
%
%    \begin{macro}{\hologoDriverSetup}
%    \begin{macrocode}
\def\hologoDriverSetup{%
  \let\hologoDriver\ltx@undefined
  \HOLOGO@DriverSetup
}
%    \end{macrocode}
%    \end{macro}
%
%    \begin{macro}{\HOLOGO@DriverSetup}
%    \begin{macrocode}
\def\HOLOGO@DriverSetup#1{%
  \kvsetkeys{HoLogoDriver}{#1}%
  \HOLOGO@CheckDriver
  \ltx@ifundefined{hologoDriver}{%
    \begingroup
    \edef\x{\endgroup
      \noexpand\kvsetkeys{HoLogoDriver}{\HOLOGO@DriverFallback}%
    }\x
  }{}%
  \@PackageInfoNoLine{hologo}{Using driver `\hologoDriver'}%
}
%    \end{macrocode}
%    \end{macro}
%
%    \begin{macro}{\HOLOGO@CheckDriver}
%    \begin{macrocode}
\def\HOLOGO@CheckDriver{%
  \ifpdf
    \def\hologoDriver{pdftex}%
    \let\HOLOGO@pdfliteral\pdfliteral
    \ifluatex
      \ifx\pdfextension\@undefined\else
        \protected\def\pdfliteral{\pdfextension literal}%
        \let\HOLOGO@pdfliteral\pdfliteral
      \fi
      \ltx@IfUndefined{HOLOGO@pdfliteral}{%
        \ifnum\luatexversion<36 %
        \else
          \begingroup
            \let\HOLOGO@temp\endgroup
            \ifcase0%
                \directlua{%
                  if tex.enableprimitives then %
                    tex.enableprimitives('HOLOGO@', {'pdfliteral'})%
                  else %
                    tex.print('1')%
                  end%
                }%
                \ifx\HOLOGO@pdfliteral\@undefined 1\fi%
                \relax%
              \endgroup
              \let\HOLOGO@temp\relax
              \global\let\HOLOGO@pdfliteral\HOLOGO@pdfliteral
            \fi%
          \HOLOGO@temp
        \fi
      }{}%
    \fi
    \ltx@IfUndefined{HOLOGO@pdfliteral}{%
      \@PackageWarningNoLine{hologo}{%
        Cannot find \string\pdfliteral
      }%
    }{}%
  \else
    \ifxetex
      \def\hologoDriver{xetex}%
    \else
      \ifvtex
        \def\hologoDriver{vtex}%
      \fi
    \fi
  \fi
}
%    \end{macrocode}
%    \end{macro}
%
%    \begin{macro}{\HOLOGO@WarningUnsupportedDriver}
%    \begin{macrocode}
\def\HOLOGO@WarningUnsupportedDriver#1{%
  \@PackageWarningNoLine{hologo}{%
    Logo `#1' needs driver specific macros,\MessageBreak
    but driver `\hologoDriver' is not supported.\MessageBreak
    Use a different driver or\MessageBreak
    load package `graphics' or `pgf'%
  }%
}
%    \end{macrocode}
%    \end{macro}
%
% \subsubsection{Reflect box macros}
%
%    Skip driver part if not needed.
%    \begin{macrocode}
\ltx@IfUndefined{reflectbox}{}{%
  \ltx@IfUndefined{rotatebox}{}{%
    \HOLOGO@AtEnd
  }%
}
\ltx@IfUndefined{pgftext}{}{%
  \HOLOGO@AtEnd
}
\ltx@IfUndefined{psscalebox}{}{%
  \HOLOGO@AtEnd
}
%    \end{macrocode}
%
%    \begin{macrocode}
\def\HOLOGO@temp{LaTeX2e}
\ifx\fmtname\HOLOGO@temp
  \RequirePackage{kvoptions}[2011/06/30]%
  \ProcessKeyvalOptions{HoLogoDriver}%
\fi
\HOLOGO@DriverSetup{}
%    \end{macrocode}
%
%    \begin{macro}{\HOLOGO@ReflectBox}
%    \begin{macrocode}
\def\HOLOGO@ReflectBox#1{%
  \begingroup
    \setbox\ltx@zero\hbox{\begingroup#1\endgroup}%
    \setbox\ltx@two\hbox{%
      \kern\wd\ltx@zero
      \csname HOLOGO@ScaleBox@\hologoDriver\endcsname{-1}{1}{%
        \hbox to 0pt{\copy\ltx@zero\hss}%
      }%
    }%
    \wd\ltx@two=\wd\ltx@zero
    \box\ltx@two
  \endgroup
}
%    \end{macrocode}
%    \end{macro}
%
%    \begin{macro}{\HOLOGO@PointReflectBox}
%    \begin{macrocode}
\def\HOLOGO@PointReflectBox#1{%
  \begingroup
    \setbox\ltx@zero\hbox{\begingroup#1\endgroup}%
    \setbox\ltx@two\hbox{%
      \kern\wd\ltx@zero
      \raise\ht\ltx@zero\hbox{%
        \csname HOLOGO@ScaleBox@\hologoDriver\endcsname{-1}{-1}{%
          \hbox to 0pt{\copy\ltx@zero\hss}%
        }%
      }%
    }%
    \wd\ltx@two=\wd\ltx@zero
    \box\ltx@two
  \endgroup
}
%    \end{macrocode}
%    \end{macro}
%
%    We must define all variants because of dynamic driver setup.
%    \begin{macrocode}
\def\HOLOGO@temp#1#2{#2}
%    \end{macrocode}
%
%    \begin{macro}{\HOLOGO@ScaleBox@pdftex}
%    \begin{macrocode}
\HOLOGO@temp{pdftex}{%
  \def\HOLOGO@ScaleBox@pdftex#1#2#3{%
    \HOLOGO@pdfliteral{%
      q #1 0 0 #2 0 0 cm%
    }%
    #3%
    \HOLOGO@pdfliteral{%
      Q%
    }%
  }%
}
%    \end{macrocode}
%    \end{macro}
%    \begin{macro}{\HOLOGO@ScaleBox@dvips}
%    \begin{macrocode}
\HOLOGO@temp{dvips}{%
  \def\HOLOGO@ScaleBox@dvips#1#2#3{%
    \special{ps:%
      gsave %
      currentpoint %
      currentpoint translate %
      #1 #2 scale %
      neg exch neg exch translate%
    }%
    #3%
    \special{ps:%
      currentpoint %
      grestore %
      moveto%
    }%
  }%
}
%    \end{macrocode}
%    \end{macro}
%    \begin{macro}{\HOLOGO@ScaleBox@dvipdfm}
%    \begin{macrocode}
\HOLOGO@temp{dvipdfm}{%
  \let\HOLOGO@ScaleBox@dvipdfm\HOLOGO@ScaleBox@dvips
}
%    \end{macrocode}
%    \end{macro}
%    Since \hologo{XeTeX} v0.6.
%    \begin{macro}{\HOLOGO@ScaleBox@xetex}
%    \begin{macrocode}
\HOLOGO@temp{xetex}{%
  \def\HOLOGO@ScaleBox@xetex#1#2#3{%
    \special{x:gsave}%
    \special{x:scale #1 #2}%
    #3%
    \special{x:grestore}%
  }%
}
%    \end{macrocode}
%    \end{macro}
%    \begin{macro}{\HOLOGO@ScaleBox@vtex}
%    \begin{macrocode}
\HOLOGO@temp{vtex}{%
  \def\HOLOGO@ScaleBox@vtex#1#2#3{%
    \special{r(#1,0,0,#2,0,0}%
    #3%
    \special{r)}%
  }%
}
%    \end{macrocode}
%    \end{macro}
%
%    \begin{macrocode}
\HOLOGO@AtEnd%
%</package>
%    \end{macrocode}
%
% \section{Test}
%
% \subsection{Catcode checks for loading}
%
%    \begin{macrocode}
%<*test1>
%    \end{macrocode}
%    \begin{macrocode}
\catcode`\{=1 %
\catcode`\}=2 %
\catcode`\#=6 %
\catcode`\@=11 %
\expandafter\ifx\csname count@\endcsname\relax
  \countdef\count@=255 %
\fi
\expandafter\ifx\csname @gobble\endcsname\relax
  \long\def\@gobble#1{}%
\fi
\expandafter\ifx\csname @firstofone\endcsname\relax
  \long\def\@firstofone#1{#1}%
\fi
\expandafter\ifx\csname loop\endcsname\relax
  \expandafter\@firstofone
\else
  \expandafter\@gobble
\fi
{%
  \def\loop#1\repeat{%
    \def\body{#1}%
    \iterate
  }%
  \def\iterate{%
    \body
      \let\next\iterate
    \else
      \let\next\relax
    \fi
    \next
  }%
  \let\repeat=\fi
}%
\def\RestoreCatcodes{}
\count@=0 %
\loop
  \edef\RestoreCatcodes{%
    \RestoreCatcodes
    \catcode\the\count@=\the\catcode\count@\relax
  }%
\ifnum\count@<255 %
  \advance\count@ 1 %
\repeat

\def\RangeCatcodeInvalid#1#2{%
  \count@=#1\relax
  \loop
    \catcode\count@=15 %
  \ifnum\count@<#2\relax
    \advance\count@ 1 %
  \repeat
}
\def\RangeCatcodeCheck#1#2#3{%
  \count@=#1\relax
  \loop
    \ifnum#3=\catcode\count@
    \else
      \errmessage{%
        Character \the\count@\space
        with wrong catcode \the\catcode\count@\space
        instead of \number#3%
      }%
    \fi
  \ifnum\count@<#2\relax
    \advance\count@ 1 %
  \repeat
}
\def\space{ }
\expandafter\ifx\csname LoadCommand\endcsname\relax
  \def\LoadCommand{\input hologo.sty\relax}%
\fi
\def\Test{%
  \RangeCatcodeInvalid{0}{47}%
  \RangeCatcodeInvalid{58}{64}%
  \RangeCatcodeInvalid{91}{96}%
  \RangeCatcodeInvalid{123}{255}%
  \catcode`\@=12 %
  \catcode`\\=0 %
  \catcode`\%=14 %
  \LoadCommand
  \RangeCatcodeCheck{0}{36}{15}%
  \RangeCatcodeCheck{37}{37}{14}%
  \RangeCatcodeCheck{38}{47}{15}%
  \RangeCatcodeCheck{48}{57}{12}%
  \RangeCatcodeCheck{58}{63}{15}%
  \RangeCatcodeCheck{64}{64}{12}%
  \RangeCatcodeCheck{65}{90}{11}%
  \RangeCatcodeCheck{91}{91}{15}%
  \RangeCatcodeCheck{92}{92}{0}%
  \RangeCatcodeCheck{93}{96}{15}%
  \RangeCatcodeCheck{97}{122}{11}%
  \RangeCatcodeCheck{123}{255}{15}%
  \RestoreCatcodes
}
\Test
\csname @@end\endcsname
\end
%    \end{macrocode}
%    \begin{macrocode}
%</test1>
%    \end{macrocode}
%
% \subsection{Spacefactor}
%
%    The space factor must be 1000 after a logo. If it is greater 1000
%    then the following space is a space after a sentence closing point.
%    If the space factor is smaller 1000 then an immediate following
%    dot is interpreted as abbreviation, not sentence closing point.
%
%    \begin{macrocode}
%<*test-spacefactor>
\NeedsTeXFormat{LaTeX2e}
\documentclass{article}
\usepackage{hologo}[2016/05/12]
\usepackage{kvsetkeys}
\usepackage{qstest}
\IncludeTests{*}
\LogTests{log}{*}{*}
\begin{document}
\begin{qstest}{spacefactor}{spacefactor}
\newcommand*{\Test}[1]{%
  \sbox0{%
    \hologo{#1}%
    \Expect*{1000 (#1)}*{\the\spacefactor\space(#1)}%
  }%
}%
\makeatletter
\def\TestList{}
\def\hologoEntry#1#2#3{%
  \edef\TestList{%
    \ifx\TestList\@empty
    \else
      \TestList,%
    \fi
    #1%
    \ifx\\#2\\%
    \else
      ={variant=#2}%
    \fi
  }%
}
\hologoList
\expandafter\kv@parse@normalized\expandafter{%
  \TestList
}{%
  \begingroup
    \let\@logo=\kv@key
    \ifx\kv@value\relax
    \else
      \expandafter\hologoLogoSetup\expandafter\@logo\expandafter{%
        \kv@value
      }%
    \fi
    \Test\@logo
  \endgroup
  \@gobbletwo
}
\end{qstest}
\end{document}
%</test-spacefactor>
%    \end{macrocode}
%
% \subsection{Complete list}
%
%    \begin{macrocode}
%<*test-list>
\NeedsTeXFormat{LaTeX2e}
\documentclass[12pt,a4paper]{article}
\usepackage{hologo}[2016/05/12]
\usepackage[T1]{fontenc}
\usepackage{lmodern}
\usepackage{parskip}
\usepackage[unicode]{hyperref}[2011/09/28]
\usepackage{bookmark}[2011/09/19]
\bookmarksetup{%
  numbered,%
  open,%
  openlevel=2,%
}
\renewcommand*{\contentsname}{List of logos}
\begin{document}
\tableofcontents
\def\TestFont#1#2#3#4#5#6{%
  \begingroup
    \usefont{#3}{#4}{#5}{#6}%
    \HologoVariant{#1}{#2}/\hologoVariant{#1}{#2}%
    \quad
    \begingroup\scriptsize\hologoVariant{#1}{#2}\endgroup
    \quad
  \endgroup
  (#3/#4/#5/#6)%
  \par
}
\makeatletter
\def\hologoEntry#1#2#3{%
  \section{%
    \HologoVariant{#1}{#2}/\hologoVariant{#1}{#2} %
    {[#1\ifx\\#2\\\else\space(#2)\fi]}% hash-ok
  }% braces around [] because of bug in tex4ht
  \begingroup
    \hypersetup{unicode=false}%
    \bookmark[%
      dest=\@currentHref,%
      rellevel=1,%
      keeplevel,%
    ]{%
      \HologoVariant{#1}{#2}/\hologoVariant{#1}{#2} %
      (PDFDocEncoding)%
    }%
  \endgroup
  \TestFont{#1}{#2}{OT1}{cmr}{m}{n}%
  \TestFont{#1}{#2}{OT1}{cmss}{m}{n}%
  \TestFont{#1}{#2}{OT1}{cmr}{b}{n}%
  \TestFont{#1}{#2}{OT1}{cmr}{m}{it}%
  \TestFont{#1}{#2}{OT1}{cmtt}{m}{n}%
  \TestFont{#1}{#2}{T1}{lmr}{m}{n}%
  \TestFont{#1}{#2}{T1}{lmss}{m}{n}%
  \TestFont{#1}{#2}{T1}{lmr}{b}{n}%
  \TestFont{#1}{#2}{T1}{lmr}{m}{it}%
  \TestFont{#1}{#2}{T1}{lmtt}{m}{n}%
  \TestFont{#1}{#2}{T1}{lmvtt}{m}{n}%
  \TestFont{#1}{#2}{T1}{qtm}{m}{n}%
  \TestFont{#1}{#2}{T1}{qhv}{m}{n}%
  \TestFont{#1}{#2}{T1}{qtm}{b}{n}%
  \TestFont{#1}{#2}{T1}{qtm}{m}{it}%
  \TestFont{#1}{#2}{T1}{qcr}{m}{n}%
  \newpage
}
\makeatother
\hologoList
\end{document}
%</test-list>
%    \end{macrocode}
%
% \section{Installation}
%
% \subsection{Download}
%
% \paragraph{Package.} This package is available on
% CTAN\footnote{\url{ftp://ftp.ctan.org/tex-archive/}}:
% \begin{description}
% \item[\CTAN{macros/latex/contrib/oberdiek/hologo.dtx}] The source file.
% \item[\CTAN{macros/latex/contrib/oberdiek/hologo.pdf}] Documentation.
% \end{description}
%
%
% \paragraph{Bundle.} All the packages of the bundle `oberdiek'
% are also available in a TDS compliant ZIP archive. There
% the packages are already unpacked and the documentation files
% are generated. The files and directories obey the TDS standard.
% \begin{description}
% \item[\CTAN{install/macros/latex/contrib/oberdiek.tds.zip}]
% \end{description}
% \emph{TDS} refers to the standard ``A Directory Structure
% for \TeX\ Files'' (\CTAN{tds/tds.pdf}). Directories
% with \xfile{texmf} in their name are usually organized this way.
%
% \subsection{Bundle installation}
%
% \paragraph{Unpacking.} Unpack the \xfile{oberdiek.tds.zip} in the
% TDS tree (also known as \xfile{texmf} tree) of your choice.
% Example (linux):
% \begin{quote}
%   |unzip oberdiek.tds.zip -d ~/texmf|
% \end{quote}
%
% \paragraph{Script installation.}
% Check the directory \xfile{TDS:scripts/oberdiek/} for
% scripts that need further installation steps.
% Package \xpackage{attachfile2} comes with the Perl script
% \xfile{pdfatfi.pl} that should be installed in such a way
% that it can be called as \texttt{pdfatfi}.
% Example (linux):
% \begin{quote}
%   |chmod +x scripts/oberdiek/pdfatfi.pl|\\
%   |cp scripts/oberdiek/pdfatfi.pl /usr/local/bin/|
% \end{quote}
%
% \subsection{Package installation}
%
% \paragraph{Unpacking.} The \xfile{.dtx} file is a self-extracting
% \docstrip\ archive. The files are extracted by running the
% \xfile{.dtx} through \plainTeX:
% \begin{quote}
%   \verb|tex hologo.dtx|
% \end{quote}
%
% \paragraph{TDS.} Now the different files must be moved into
% the different directories in your installation TDS tree
% (also known as \xfile{texmf} tree):
% \begin{quote}
% \def\t{^^A
% \begin{tabular}{@{}>{\ttfamily}l@{ $\rightarrow$ }>{\ttfamily}l@{}}
%   hologo.sty & tex/generic/oberdiek/hologo.sty\\
%   hologo.pdf & doc/latex/oberdiek/hologo.pdf\\
%   example/hologo-example.tex & doc/latex/oberdiek/example/hologo-example.tex\\
%   test/hologo-test1.tex & doc/latex/oberdiek/test/hologo-test1.tex\\
%   test/hologo-test-spacefactor.tex & doc/latex/oberdiek/test/hologo-test-spacefactor.tex\\
%   test/hologo-test-list.tex & doc/latex/oberdiek/test/hologo-test-list.tex\\
%   hologo.dtx & source/latex/oberdiek/hologo.dtx\\
% \end{tabular}^^A
% }^^A
% \sbox0{\t}^^A
% \ifdim\wd0>\linewidth
%   \begingroup
%     \advance\linewidth by\leftmargin
%     \advance\linewidth by\rightmargin
%   \edef\x{\endgroup
%     \def\noexpand\lw{\the\linewidth}^^A
%   }\x
%   \def\lwbox{^^A
%     \leavevmode
%     \hbox to \linewidth{^^A
%       \kern-\leftmargin\relax
%       \hss
%       \usebox0
%       \hss
%       \kern-\rightmargin\relax
%     }^^A
%   }^^A
%   \ifdim\wd0>\lw
%     \sbox0{\small\t}^^A
%     \ifdim\wd0>\linewidth
%       \ifdim\wd0>\lw
%         \sbox0{\footnotesize\t}^^A
%         \ifdim\wd0>\linewidth
%           \ifdim\wd0>\lw
%             \sbox0{\scriptsize\t}^^A
%             \ifdim\wd0>\linewidth
%               \ifdim\wd0>\lw
%                 \sbox0{\tiny\t}^^A
%                 \ifdim\wd0>\linewidth
%                   \lwbox
%                 \else
%                   \usebox0
%                 \fi
%               \else
%                 \lwbox
%               \fi
%             \else
%               \usebox0
%             \fi
%           \else
%             \lwbox
%           \fi
%         \else
%           \usebox0
%         \fi
%       \else
%         \lwbox
%       \fi
%     \else
%       \usebox0
%     \fi
%   \else
%     \lwbox
%   \fi
% \else
%   \usebox0
% \fi
% \end{quote}
% If you have a \xfile{docstrip.cfg} that configures and enables \docstrip's
% TDS installing feature, then some files can already be in the right
% place, see the documentation of \docstrip.
%
% \subsection{Refresh file name databases}
%
% If your \TeX~distribution
% (\teTeX, \mikTeX, \dots) relies on file name databases, you must refresh
% these. For example, \teTeX\ users run \verb|texhash| or
% \verb|mktexlsr|.
%
% \subsection{Some details for the interested}
%
% \paragraph{Attached source.}
%
% The PDF documentation on CTAN also includes the
% \xfile{.dtx} source file. It can be extracted by
% AcrobatReader 6 or higher. Another option is \textsf{pdftk},
% e.g. unpack the file into the current directory:
% \begin{quote}
%   \verb|pdftk hologo.pdf unpack_files output .|
% \end{quote}
%
% \paragraph{Unpacking with \LaTeX.}
% The \xfile{.dtx} chooses its action depending on the format:
% \begin{description}
% \item[\plainTeX:] Run \docstrip\ and extract the files.
% \item[\LaTeX:] Generate the documentation.
% \end{description}
% If you insist on using \LaTeX\ for \docstrip\ (really,
% \docstrip\ does not need \LaTeX), then inform the autodetect routine
% about your intention:
% \begin{quote}
%   \verb|latex \let\install=y\input{hologo.dtx}|
% \end{quote}
% Do not forget to quote the argument according to the demands
% of your shell.
%
% \paragraph{Generating the documentation.}
% You can use both the \xfile{.dtx} or the \xfile{.drv} to generate
% the documentation. The process can be configured by the
% configuration file \xfile{ltxdoc.cfg}. For instance, put this
% line into this file, if you want to have A4 as paper format:
% \begin{quote}
%   \verb|\PassOptionsToClass{a4paper}{article}|
% \end{quote}
% An example follows how to generate the
% documentation with pdf\LaTeX:
% \begin{quote}
%\begin{verbatim}
%pdflatex hologo.dtx
%makeindex -s gind.ist hologo.idx
%pdflatex hologo.dtx
%makeindex -s gind.ist hologo.idx
%pdflatex hologo.dtx
%\end{verbatim}
% \end{quote}
%
% \section{Catalogue}
%
% The following XML file can be used as source for the
% \href{http://mirror.ctan.org/help/Catalogue/catalogue.html}{\TeX\ Catalogue}.
% The elements \texttt{caption} and \texttt{description} are imported
% from the original XML file from the Catalogue.
% The name of the XML file in the Catalogue is \xfile{hologo.xml}.
%    \begin{macrocode}
%<*catalogue>
<?xml version='1.0' encoding='us-ascii'?>
<!DOCTYPE entry SYSTEM 'catalogue.dtd'>
<entry datestamp='$Date$' modifier='$Author$' id='hologo'>
  <name>hologo</name>
  <caption>A collection of logos with bookmark support.</caption>
  <authorref id='auth:oberdiek'/>
  <copyright owner='Heiko Oberdiek' year='2010-2012'/>
  <license type='lppl1.3'/>
  <version number='1.10'/>
  <description>
    The package defines a single command <tt>\hologo</tt>, whose
    argument is the usual case-confused ASCII version of the logo.
    The command is bookmark-enabled, so that every logo becomes
    available in bookmarks without further work.
    <p/>
    The package is part of the <xref refid='oberdiek'>oberdiek</xref>
    bundle.
  </description>
  <documentation details='Package documentation'
      href='ctan:/macros/latex/contrib/oberdiek/hologo.pdf'/>
  <ctan file='true' path='/macros/latex/contrib/oberdiek/hologo.dtx'/>
  <miktex location='oberdiek'/>
  <texlive location='oberdiek'/>
  <install path='/macros/latex/contrib/oberdiek/oberdiek.tds.zip'/>
</entry>
%</catalogue>
%    \end{macrocode}
%
% \begin{thebibliography}{9}
% \raggedright
%
% \bibitem{btxdoc}
% Oren Patashnik,
% \textit{\hologo{BibTeX}ing},
% 1988-02-08.\\
% \CTAN{biblio/bibtex/base/}
%
% \bibitem{dtklogos}
% Gerd Neugebauer, DANTE,
% \textit{Package \xpackage{dtklogos}},
% 2011-04-25.\\
% \CTAN{usergrps/dante/dtk/dtklogos.sty}
%
% \bibitem{etexman}
% The \hologo{NTS} Team,
% \textit{The \hologo{eTeX} manual},
% 1998-02.\\
% \CTAN{systems/e-tex/v2/doc/}
%
% \bibitem{ExTeX-FAQ}
% The \hologo{ExTeX} group,
% \textit{\hologo{ExTeX}: FAQ -- How is \hologo{ExTeX} typeset?},
% 2007-04-14.\\
% \url{http://www.extex.org/documentation/faq.html}
%
% \bibitem{LyX}
% %@MISC{ LyX,
% %  title = {{LyX 2.0.0 -- The Document Processor [Computer software and manual]}},
% %  author = {{The LyX Team}},
% %  howpublished = {Internet: http://www.lyx.org},
% %  year = {2011-05-08},
% %  note = {Retrieved May 10, 2011, from http://www.lyx.org},
% %  url = {http://www.lyx.org/}
% %}
% The \hologo{LyX} Team,
% \textit{\hologo{LyX} -- The Document Processor},
% 2011-05-08.\\
% \url{http://www.lyx.org/}
%
% \bibitem{OzTeX}
% Andrew Trevorrow,
% \hologo{OzTeX} FAQ: What is the correct way to typeset ``\hologo{OzTeX}''?,
% 2011-09-15 (visited).
% \url{http://www.trevorrow.com/oztex/ozfaq.html#oztex-logo}
%
% \bibitem{PiCTeX}
% Michael Wichura,
% \textit{The \hologo{PiCTeX} macro package},
% 1987-09-21.
% \CTAN{graphics/pictex/}
%
% \bibitem{scrlogo}
% Markus Kohm,
% \textit{\hologo{KOMAScript} Datei \xfile{scrlogo.dtx}},
% 2009-01-30.\\
% \CTAN{install/macros/latex/contrib/komascript.tds.zip}
%
% \end{thebibliography}
%
% \begin{History}
%   \begin{Version}{2010/04/08 v1.0}
%   \item
%     The first version.
%   \end{Version}
%   \begin{Version}{2010/04/16 v1.1}
%   \item
%     \cs{Hologo} added for support of logos at start of a sentence.
%   \item
%     \cs{hologoSetup} and \cs{hologoLogoSetup} added.
%   \item
%     Options \xoption{break}, \xoption{hyphenbreak}, \xoption{spacebreak}
%     added.
%   \item
%     Variant support added by option \xoption{variant}.
%   \end{Version}
%   \begin{Version}{2010/04/24 v1.2}
%   \item
%     \hologo{LaTeX3} added.
%   \item
%     \hologo{VTeX} added.
%   \end{Version}
%   \begin{Version}{2010/11/21 v1.3}
%   \item
%     \hologo{iniTeX}, \hologo{virTeX} added.
%   \end{Version}
%   \begin{Version}{2011/03/25 v1.4}
%   \item
%     \hologo{ConTeXt} with variants added.
%   \item
%     Option \xoption{discretionarybreak} added as refinement for
%     option \xoption{break}.
%   \end{Version}
%   \begin{Version}{2011/04/21 v1.5}
%   \item
%     Wrong TDS directory for test files fixed.
%   \end{Version}
%   \begin{Version}{2011/10/01 v1.6}
%   \item
%     Support for package \xpackage{tex4ht} added.
%   \item
%     Support for \cs{csname} added if \cs{ifincsname} is available.
%   \item
%     New logos:
%     \hologo{(La)TeX},
%     \hologo{biber},
%     \hologo{BibTeX} (\xoption{sc}, \xoption{sf}),
%     \hologo{emTeX},
%     \hologo{ExTeX},
%     \hologo{KOMAScript},
%     \hologo{La},
%     \hologo{LyX},
%     \hologo{MiKTeX},
%     \hologo{NTS},
%     \hologo{OzMF},
%     \hologo{OzMP},
%     \hologo{OzTeX},
%     \hologo{OzTtH},
%     \hologo{PCTeX},
%     \hologo{PiC},
%     \hologo{PiCTeX},
%     \hologo{METAFONT},
%     \hologo{MetaFun},
%     \hologo{METAPOST},
%     \hologo{MetaPost},
%     \hologo{SLiTeX} (\xoption{lift}, \xoption{narrow}, \xoption{simple}),
%     \hologo{SliTeX} (\xoption{narrow}, \xoption{simple}, \xoption{lift}),
%     \hologo{teTeX}.
%   \item
%     Fixes:
%     \hologo{iniTeX},
%     \hologo{pdfLaTeX},
%     \hologo{pdfTeX},
%     \hologo{virTeX}.
%   \item
%     \cs{hologoFontSetup} and \cs{hologoLogoFontSetup} added.
%   \item
%     \cs{hologoVariant} and \cs{HologoVariant} added.
%   \end{Version}
%   \begin{Version}{2011/11/22 v1.7}
%   \item
%     New logos:
%     \hologo{BibTeX8},
%     \hologo{LaTeXML},
%     \hologo{SageTeX},
%     \hologo{TeX4ht},
%     \hologo{TTH}.
%   \item
%     \hologo{Xe} and friends: Driver stuff fixed.
%   \item
%     \hologo{Xe} and friends: Support for italic added.
%   \item
%     \hologo{Xe} and friends: Package support for \xpackage{pgf}
%     and \xpackage{pstricks} added.
%   \end{Version}
%   \begin{Version}{2011/11/29 v1.8}
%   \item
%     New logos:
%     \hologo{HanTheThanh}.
%   \end{Version}
%   \begin{Version}{2011/12/21 v1.9}
%   \item
%     Patch for package \xpackage{ifxetex} added for the case that
%     \cs{newif} is undefined in \hologo{iniTeX}.
%   \item
%     Some fixes for \hologo{iniTeX}.
%   \end{Version}
%   \begin{Version}{2012/04/26 v1.10}
%   \item
%     Fix in bookmark version of logo ``\hologo{HanTheThanh}''.
%   \end{Version}
%   \begin{Version}{2016/05/12 v1.11}
%   \item
%     Update HOLOGO@IfCharExists (previously in texlive)
%   \item define pdfliteral in current luatex.
%   \end{Version}
% \end{History}
%
% \PrintIndex
%
% \Finale
\endinput

%        (quote the arguments according to the demands of your shell)
%
% Documentation:
%    (a) If hologo.drv is present:
%           latex hologo.drv
%    (b) Without hologo.drv:
%           latex hologo.dtx; ...
%    The class ltxdoc loads the configuration file ltxdoc.cfg
%    if available. Here you can specify further options, e.g.
%    use A4 as paper format:
%       \PassOptionsToClass{a4paper}{article}
%
%    Programm calls to get the documentation (example):
%       pdflatex hologo.dtx
%       makeindex -s gind.ist hologo.idx
%       pdflatex hologo.dtx
%       makeindex -s gind.ist hologo.idx
%       pdflatex hologo.dtx
%
% Installation:
%    TDS:tex/generic/oberdiek/hologo.sty
%    TDS:doc/latex/oberdiek/hologo.pdf
%    TDS:doc/latex/oberdiek/example/hologo-example.tex
%    TDS:doc/latex/oberdiek/test/hologo-test1.tex
%    TDS:doc/latex/oberdiek/test/hologo-test-spacefactor.tex
%    TDS:doc/latex/oberdiek/test/hologo-test-list.tex
%    TDS:source/latex/oberdiek/hologo.dtx
%
%<*ignore>
\begingroup
  \catcode123=1 %
  \catcode125=2 %
  \def\x{LaTeX2e}%
\expandafter\endgroup
\ifcase 0\ifx\install y1\fi\expandafter
         \ifx\csname processbatchFile\endcsname\relax\else1\fi
         \ifx\fmtname\x\else 1\fi\relax
\else\csname fi\endcsname
%</ignore>
%<*install>
\input docstrip.tex
\Msg{************************************************************************}
\Msg{* Installation}
\Msg{* Package: hologo 2016/05/12 v1.11 A logo collection with bookmark support (HO)}
\Msg{************************************************************************}

\keepsilent
\askforoverwritefalse

\let\MetaPrefix\relax
\preamble

This is a generated file.

Project: hologo
Version: 2016/05/12 v1.11

Copyright (C) 2010-2012 by
   Heiko Oberdiek <heiko.oberdiek at googlemail.com>

This work may be distributed and/or modified under the
conditions of the LaTeX Project Public License, either
version 1.3c of this license or (at your option) any later
version. This version of this license is in
   http://www.latex-project.org/lppl/lppl-1-3c.txt
and the latest version of this license is in
   http://www.latex-project.org/lppl.txt
and version 1.3 or later is part of all distributions of
LaTeX version 2005/12/01 or later.

This work has the LPPL maintenance status "maintained".

This Current Maintainer of this work is Heiko Oberdiek.

The Base Interpreter refers to any `TeX-Format',
because some files are installed in TDS:tex/generic//.

This work consists of the main source file hologo.dtx
and the derived files
   hologo.sty, hologo.pdf, hologo.ins, hologo.drv, hologo-example.tex,
   hologo-test1.tex, hologo-test-spacefactor.tex,
   hologo-test-list.tex.

\endpreamble
\let\MetaPrefix\DoubleperCent

\generate{%
  \file{hologo.ins}{\from{hologo.dtx}{install}}%
  \file{hologo.drv}{\from{hologo.dtx}{driver}}%
  \usedir{tex/generic/oberdiek}%
  \file{hologo.sty}{\from{hologo.dtx}{package}}%
  \usedir{doc/latex/oberdiek/example}%
  \file{hologo-example.tex}{\from{hologo.dtx}{example}}%
  \usedir{doc/latex/oberdiek/test}%
  \file{hologo-test1.tex}{\from{hologo.dtx}{test1}}%
  \file{hologo-test-spacefactor.tex}{\from{hologo.dtx}{test-spacefactor}}%
  \file{hologo-test-list.tex}{\from{hologo.dtx}{test-list}}%
  \nopreamble
  \nopostamble
  \usedir{source/latex/oberdiek/catalogue}%
  \file{hologo.xml}{\from{hologo.dtx}{catalogue}}%
}

\catcode32=13\relax% active space
\let =\space%
\Msg{************************************************************************}
\Msg{*}
\Msg{* To finish the installation you have to move the following}
\Msg{* file into a directory searched by TeX:}
\Msg{*}
\Msg{*     hologo.sty}
\Msg{*}
\Msg{* To produce the documentation run the file `hologo.drv'}
\Msg{* through LaTeX.}
\Msg{*}
\Msg{* Happy TeXing!}
\Msg{*}
\Msg{************************************************************************}

\endbatchfile
%</install>
%<*ignore>
\fi
%</ignore>
%<*driver>
\NeedsTeXFormat{LaTeX2e}
\ProvidesFile{hologo.drv}%
  [2016/05/12 v1.11 A logo collection with bookmark support (HO)]%
\documentclass{ltxdoc}
\usepackage{holtxdoc}[2011/11/22]
\usepackage{hologo}[2016/05/12]
\usepackage{longtable}
\usepackage{array}
\usepackage{paralist}
%\usepackage[T1]{fontenc}
%\usepackage{lmodern}
\begin{document}
  \DocInput{hologo.dtx}%
\end{document}
%</driver>
% \fi
%
%
% \CharacterTable
%  {Upper-case    \A\B\C\D\E\F\G\H\I\J\K\L\M\N\O\P\Q\R\S\T\U\V\W\X\Y\Z
%   Lower-case    \a\b\c\d\e\f\g\h\i\j\k\l\m\n\o\p\q\r\s\t\u\v\w\x\y\z
%   Digits        \0\1\2\3\4\5\6\7\8\9
%   Exclamation   \!     Double quote  \"     Hash (number) \#
%   Dollar        \$     Percent       \%     Ampersand     \&
%   Acute accent  \'     Left paren    \(     Right paren   \)
%   Asterisk      \*     Plus          \+     Comma         \,
%   Minus         \-     Point         \.     Solidus       \/
%   Colon         \:     Semicolon     \;     Less than     \<
%   Equals        \=     Greater than  \>     Question mark \?
%   Commercial at \@     Left bracket  \[     Backslash     \\
%   Right bracket \]     Circumflex    \^     Underscore    \_
%   Grave accent  \`     Left brace    \{     Vertical bar  \|
%   Right brace   \}     Tilde         \~}
%
% \GetFileInfo{hologo.drv}
%
% \title{The \xpackage{hologo} package}
% \date{2016/05/12 v1.11}
% \author{Heiko Oberdiek\\\xemail{heiko.oberdiek at googlemail.com}}
%
% \maketitle
%
% \begin{abstract}
% This package starts a collection of logos with support for bookmarks
% strings.
% \end{abstract}
%
% \tableofcontents
%
% \section{Documentation}
%
% \subsection{Logo macros}
%
% \begin{declcs}{hologo} \M{name}
% \end{declcs}
% Macro \cs{hologo} sets the logo with name \meta{name}.
% The following table shows the supported names.
%
% \begingroup
%   \def\hologoEntry#1#2#3{^^A
%     #1&#2&\hologoLogoSetup{#1}{variant=#2}\hologo{#1}&#3\tabularnewline
%   }
%   \begin{longtable}{>{\ttfamily}l>{\ttfamily}lll}
%     \rmfamily\bfseries{name} & \rmfamily\bfseries variant
%     & \bfseries logo & \bfseries since\\
%     \hline
%     \endhead
%     \hologoList
%   \end{longtable}
% \endgroup
%
% \begin{declcs}{Hologo} \M{name}
% \end{declcs}
% Macro \cs{Hologo} starts the logo \meta{name} with an uppercase
% letter. As an exception small greek letters are not converted
% to uppercase. Examples, see \hologo{eTeX} and \hologo{ExTeX}.
%
% \subsection{Setup macros}
%
% The package does not support package options, but the following
% setup macros can be used to set options.
%
% \begin{declcs}{hologoSetup} \M{key value list}
% \end{declcs}
% Macro \cs{hologoSetup} sets global options.
%
% \begin{declcs}{hologoLogoSetup} \M{logo} \M{key value list}
% \end{declcs}
% Some options can also be used to configure a logo.
% These settings take precedence over global option settings.
%
% \subsection{Options}\label{sec:options}
%
% There are boolean and string options:
% \begin{description}
% \item[Boolean option:]
% It takes |true| or |false|
% as value. If the value is omitted, then |true| is used.
% \item[String option:]
% A value must be given as string. (But the string might be empty.)
% \end{description}
% The following options can be used both in \cs{hologoSetup}
% and \cs{hologoLogoSetup}:
% \begin{description}
% \def\entry#1{\item[\xoption{#1}:]}
% \entry{break}
%   enables or disables line breaks inside the logo. This setting is
%   refined by options \xoption{hyphenbreak}, \xoption{spacebreak}
%   or \xoption{discretionarybreak}.
%   Default is |false|.
% \entry{hyphenbreak}
%   enables or disables the line break right after the hyphen character.
% \entry{spacebreak}
%   enables or disables line breaks at space characters.
% \entry{discretionarybreak}
%   enables or disables line breaks at hyphenation points
%   (inserted by \cs{-}).
% \end{description}
% Macro \cs{hologoLogoSetup} also knows:
% \begin{description}
% \item[\xoption{variant}:]
%   This is a string option. It specifies a variant of a logo that
%   must exist. An empty string selects the package default variant.
% \end{description}
% Example:
% \begin{quote}
%   |\hologoSetup{break=false}|\\
%   |\hologoLogoSetup{plainTeX}{variant=hyphen,hyphenbreak}|\\
%   Then ``plain-\TeX'' contains one break point after the hyphen.
% \end{quote}
%
% \subsection{Driver options}
%
% Sometimes graphical operations are needed to construct some
% glyphs (e.g.\ \hologo{XeTeX}). If package \xpackage{graphics}
% or package \xpackage{pgf} are found, then the macros are taken
% from there. Otherwise the packge defines its own operations
% and therefore needs the driver information. Many drivers are
% detected automatically (\hologo{pdfTeX}/\hologo{LuaTeX}
% in PDF mode, \hologo{XeTeX}, \hologo{VTeX}). These have precedence
% over a driver option. The driver can be given as package option
% or using \cs{hologoDriverSetup}.
% The following list contains the recognized driver options:
% \begin{itemize}
% \item \xoption{pdftex}, \xoption{luatex}
% \item \xoption{dvipdfm}, \xoption{dvipdfmx}
% \item \xoption{dvips}, \xoption{dvipsone}, \xoption{xdvi}
% \item \xoption{xetex}
% \item \xoption{vtex}
% \end{itemize}
% The left driver of a line is the driver name that is used internally.
% The following names are aliases for drivers that use the
% same method. Therefore the entry in the \xext{log} file for
% the used driver prints the internally used driver name.
% \begin{description}
% \item[\xoption{driverfallback}:]
%   This option expects a driver that is used,
%   if the driver could not be detected automatically.
% \end{description}
%
% \begin{declcs}{hologoDriverSetup} \M{driver option}
% \end{declcs}
% The driver can also be configured after package loading
% using \cs{hologoDriverSetup}, also the way for \hologo{plainTeX}
% to setup the driver.
%
% \subsection{Font setup}
%
% Some logos require a special font, but should also be usable by
% \hologo{plainTeX}. Therefore the package provides some ways
% to influence the font settings. The options below
% take font settings as values. Both font commands
% such as \cs{sffamily} and macros that take one argument
% like \cs{textsf} can be used.
%
% \begin{declcs}{hologoFontSetup} \M{key value list}
% \end{declcs}
% Macro \cs{hologoFontSetup} sets the fonts for all logos.
% Supported keys:
% \begin{description}
% \def\entry#1{\item[\xoption{#1}:]}
% \entry{general}
%   This font is used for all logos. The default is empty.
%   That means no special font is used.
% \entry{bibsf}
%   This font is used for
%   {\hologoLogoSetup{BibTeX}{variant=sf}\hologo{BibTeX}}
%   with variant \xoption{sf}.
% \entry{rm}
%   This font is a serif font. It is used for \hologo{ExTeX}.
% \entry{sc}
%   This font specifies a small caps font. It is used for
%   {\hologoLogoSetup{BibTeX}{variant=sc}\hologo{BibTeX}}
%   with variant \xoption{sc}.
% \entry{sf}
%   This font specifies a sans serif font. The default
%   is \cs{sffamily}, then \cs{sf} is tried. Otherwise
%   a warning is given. It is used by \hologo{KOMAScript}.
% \entry{sy}
%   This is the font for math symbols (e.g. cmsy).
%   It is used by \hologo{AmS}, \hologo{NTS}, \hologo{ExTeX}.
% \entry{logo}
%   \hologo{METAFONT} and \hologo{METAPOST} are using that font.
%   In \hologo{LaTeX} \cs{logofamily} is used and
%   the definitions of package \xpackage{mflogo} are used
%   if the package is not loaded.
%   Otherwise the \cs{tenlogo} is used and defined
%   if it does not already exists.
% \end{description}
%
% \begin{declcs}{hologoLogoFontSetup} \M{logo} \M{key value list}
% \end{declcs}
% Fonts can also be set for a logo or logo component separately,
% see the following list.
% The keys are the same as for \cs{hologoFontSetup}.
%
% \begin{longtable}{>{\ttfamily}l>{\sffamily}ll}
%   \meta{logo} & keys & result\\
%   \hline
%   \endhead
%   BibTeX & bibsf & {\hologoLogoSetup{BibTeX}{variant=sf}\hologo{BibTeX}}\\[.5ex]
%   BibTeX & sc & {\hologoLogoSetup{BibTeX}{variant=sc}\hologo{BibTeX}}\\[.5ex]
%   ExTeX & rm & \hologo{ExTeX}\\
%   SliTeX & rm & \hologo{SliTeX}\\[.5ex]
%   AmS & sy & \hologo{AmS}\\
%   ExTeX & sy & \hologo{ExTeX}\\
%   NTS & sy & \hologo{NTS}\\[.5ex]
%   KOMAScript & sf & \hologo{KOMAScript}\\[.5ex]
%   METAFONT & logo & \hologo{METAFONT}\\
%   METAPOST & logo & \hologo{METAPOST}\\[.5ex]
%   SliTeX & sc \hologo{SliTeX}
% \end{longtable}
%
% \subsubsection{Font order}
%
% For all logos the font \xoption{general} is applied first.
% Example:
%\begin{quote}
%|\hologoFontSetup{general=\color{red}}|
%\end{quote}
% will print red logos.
% Then if the font uses a special font \xoption{sf}, for example,
% the font is applied that is setup by \cs{hologoLogoFontSetup}.
% If this font is not setup, then the common font setup
% by \cs{hologoFontSetup} is used. Otherwise a warning is given,
% that there is no font configured.
%
% \subsection{Additional user macros}
%
% Usually a variant of a logo is configured by using
% \cs{hologoLogoSetup}, because it is bad style to mix
% different variants of the same logo in the same text.
% There the following macros are a convenience for testing.
%
% \begin{declcs}{hologoVariant} \M{name} \M{variant}\\
%   \cs{HologoVariant} \M{name} \M{variant}
% \end{declcs}
% Logo \meta{name} is set using \meta{variant} that specifies
% explicitely which variant of the macro is used. If the argument
% is empty, then the default form of the logo is used
% (configurable by \cs{hologoLogoSetup}).
%
% \cs{HologoVariant} is used if the logo is set in a context
% that needs an uppercase first letter (beginning of a sentence, \dots).
%
% \begin{declcs}{hologoList}\\
%   \cs{hologoEntry} \M{logo} \M{variant} \M{since}
% \end{declcs}
% Macro \cs{hologoList} contains all logos that are provided
% by the package including variants. The list consists of calls
% of \cs{hologoEntry} with three arguments starting with the
% logo name \meta{logo} and its variant \meta{variant}. An empty
% variant means the current default. Argument \meta{since} specifies
% with version of the package \xpackage{hologo} is needed to get
% the logo. If the logo is fixed, then the date gets updated.
% Therefore the date \meta{since} is not exactly the date of
% the first introduction, but rather the date of the latest fix.
%
% Before \cs{hologoList} can be used, macro \cs{hologoEntry} needs
% a definition. The example file in section \ref{sec:example}
% shows applications of \cs{hologoList}.
%
% \subsection{Supported contexts}
%
% Macros \cs{hologo} and friends support special contexts:
% \begin{itemize}
% \item \hologo{LaTeX}'s protection mechanism.
% \item Bookmarks of package \xpackage{hyperref}.
% \item Package \xpackage{tex4ht}.
% \item The macros can be used inside \cs{csname} constructs,
%   if \cs{ifincsname} is available (\hologo{pdfTeX}, \hologo{XeTeX},
%   \hologo{LuaTeX}).
% \end{itemize}
%
% \subsection{Example}
% \label{sec:example}
%
% The following example prints the logos in different fonts.
%    \begin{macrocode}
%<*example>
%<<verbatim
\NeedsTeXFormat{LaTeX2e}
\documentclass[a4paper]{article}
\usepackage[
  hmargin=20mm,
  vmargin=20mm,
]{geometry}
\pagestyle{empty}
\usepackage{hologo}[2016/05/12]
\usepackage{longtable}
\usepackage{array}
\setlength{\extrarowheight}{2pt}
\usepackage[T1]{fontenc}
\usepackage{lmodern}
\usepackage{pdflscape}
\usepackage[
  pdfencoding=auto,
]{hyperref}
\hypersetup{
  pdfauthor={Heiko Oberdiek},
  pdftitle={Example for package `hologo'},
  pdfsubject={Logos with fonts lmr, lmss, qtm, qpl, qhv},
}
\usepackage{bookmark}

% Print the logo list on the console

\begingroup
  \typeout{}%
  \typeout{*** Begin of logo list ***}%
  \newcommand*{\hologoEntry}[3]{%
    \typeout{#1 \ifx\\#2\\\else(#2) \fi[#3]}%
  }%
  \hologoList
  \typeout{*** End of logo list ***}%
  \typeout{}%
\endgroup

\begin{document}
\begin{landscape}

  \section{Example file for package `hologo'}

  % Table for font names

  \begin{longtable}{>{\bfseries}ll}
    \textbf{font} & \textbf{Font name}\\
    \hline
    lmr & Latin Modern Roman\\
    lmss & Latin Modern Sans\\
    qtm & \TeX\ Gyre Termes\\
    qhv & \TeX\ Gyre Heros\\
    qpl & \TeX\ Gyre Pagella\\
  \end{longtable}

  % Logo list with logos in different fonts

  \begingroup
    \newcommand*{\SetVariant}[2]{%
      \ifx\\#2\\%
      \else
        \hologoLogoSetup{#1}{variant=#2}%
      \fi
    }%
    \newcommand*{\hologoEntry}[3]{%
      \SetVariant{#1}{#2}%
      \raisebox{1em}[0pt][0pt]{\hypertarget{#1@#2}{}}%
      \bookmark[%
        dest={#1@#2},%
      ]{%
        #1\ifx\\#2\\\else\space(#2)\fi: \Hologo{#1}, \hologo{#1} %
        [Unicode]%
      }%
      \hypersetup{unicode=false}%
      \bookmark[%
        dest={#1@#2},%
      ]{%
        #1\ifx\\#2\\\else\space(#2)\fi: \Hologo{#1}, \hologo{#1} %
        [PDFDocEncoding]%
      }%
      \texttt{#1}%
      &%
      \texttt{#2}%
      &%
      \Hologo{#1}%
      &%
      \SetVariant{#1}{#2}%
      \hologo{#1}%
      &%
      \SetVariant{#1}{#2}%
      \fontfamily{qtm}\selectfont
      \hologo{#1}%
      &%
      \SetVariant{#1}{#2}%
      \fontfamily{qpl}\selectfont
      \hologo{#1}%
      &%
      \SetVariant{#1}{#2}%
      \textsf{\hologo{#1}}%
      &%
      \SetVariant{#1}{#2}%
      \fontfamily{qhv}\selectfont
      \hologo{#1}%
      \tabularnewline
    }%
    \begin{longtable}{llllllll}%
      \textbf{\textit{logo}} & \textbf{\textit{variant}} &
      \texttt{\string\Hologo} &
      \textbf{lmr} & \textbf{qtm} & \textbf{qpl} &
      \textbf{lmss} & \textbf{qhv}
      \tabularnewline
      \hline
      \endhead
      \hologoList
    \end{longtable}%
  \endgroup

\end{landscape}
\end{document}
%verbatim
%</example>
%    \end{macrocode}
%
% \StopEventually{
% }
%
% \section{Implementation}
%    \begin{macrocode}
%<*package>
%    \end{macrocode}
%    Reload check, especially if the package is not used with \LaTeX.
%    \begin{macrocode}
\begingroup\catcode61\catcode48\catcode32=10\relax%
  \catcode13=5 % ^^M
  \endlinechar=13 %
  \catcode35=6 % #
  \catcode39=12 % '
  \catcode44=12 % ,
  \catcode45=12 % -
  \catcode46=12 % .
  \catcode58=12 % :
  \catcode64=11 % @
  \catcode123=1 % {
  \catcode125=2 % }
  \expandafter\let\expandafter\x\csname ver@hologo.sty\endcsname
  \ifx\x\relax % plain-TeX, first loading
  \else
    \def\empty{}%
    \ifx\x\empty % LaTeX, first loading,
      % variable is initialized, but \ProvidesPackage not yet seen
    \else
      \expandafter\ifx\csname PackageInfo\endcsname\relax
        \def\x#1#2{%
          \immediate\write-1{Package #1 Info: #2.}%
        }%
      \else
        \def\x#1#2{\PackageInfo{#1}{#2, stopped}}%
      \fi
      \x{hologo}{The package is already loaded}%
      \aftergroup\endinput
    \fi
  \fi
\endgroup%
%    \end{macrocode}
%    Package identification:
%    \begin{macrocode}
\begingroup\catcode61\catcode48\catcode32=10\relax%
  \catcode13=5 % ^^M
  \endlinechar=13 %
  \catcode35=6 % #
  \catcode39=12 % '
  \catcode40=12 % (
  \catcode41=12 % )
  \catcode44=12 % ,
  \catcode45=12 % -
  \catcode46=12 % .
  \catcode47=12 % /
  \catcode58=12 % :
  \catcode64=11 % @
  \catcode91=12 % [
  \catcode93=12 % ]
  \catcode123=1 % {
  \catcode125=2 % }
  \expandafter\ifx\csname ProvidesPackage\endcsname\relax
    \def\x#1#2#3[#4]{\endgroup
      \immediate\write-1{Package: #3 #4}%
      \xdef#1{#4}%
    }%
  \else
    \def\x#1#2[#3]{\endgroup
      #2[{#3}]%
      \ifx#1\@undefined
        \xdef#1{#3}%
      \fi
      \ifx#1\relax
        \xdef#1{#3}%
      \fi
    }%
  \fi
\expandafter\x\csname ver@hologo.sty\endcsname
\ProvidesPackage{hologo}%
  [2016/05/12 v1.11 A logo collection with bookmark support (HO)]%
%    \end{macrocode}
%
%    \begin{macrocode}
\begingroup\catcode61\catcode48\catcode32=10\relax%
  \catcode13=5 % ^^M
  \endlinechar=13 %
  \catcode123=1 % {
  \catcode125=2 % }
  \catcode64=11 % @
  \def\x{\endgroup
    \expandafter\edef\csname HOLOGO@AtEnd\endcsname{%
      \endlinechar=\the\endlinechar\relax
      \catcode13=\the\catcode13\relax
      \catcode32=\the\catcode32\relax
      \catcode35=\the\catcode35\relax
      \catcode61=\the\catcode61\relax
      \catcode64=\the\catcode64\relax
      \catcode123=\the\catcode123\relax
      \catcode125=\the\catcode125\relax
    }%
  }%
\x\catcode61\catcode48\catcode32=10\relax%
\catcode13=5 % ^^M
\endlinechar=13 %
\catcode35=6 % #
\catcode64=11 % @
\catcode123=1 % {
\catcode125=2 % }
\def\TMP@EnsureCode#1#2{%
  \edef\HOLOGO@AtEnd{%
    \HOLOGO@AtEnd
    \catcode#1=\the\catcode#1\relax
  }%
  \catcode#1=#2\relax
}
\TMP@EnsureCode{10}{12}% ^^J
\TMP@EnsureCode{33}{12}% !
\TMP@EnsureCode{34}{12}% "
\TMP@EnsureCode{36}{3}% $
\TMP@EnsureCode{38}{4}% &
\TMP@EnsureCode{39}{12}% '
\TMP@EnsureCode{40}{12}% (
\TMP@EnsureCode{41}{12}% )
\TMP@EnsureCode{42}{12}% *
\TMP@EnsureCode{43}{12}% +
\TMP@EnsureCode{44}{12}% ,
\TMP@EnsureCode{45}{12}% -
\TMP@EnsureCode{46}{12}% .
\TMP@EnsureCode{47}{12}% /
\TMP@EnsureCode{58}{12}% :
\TMP@EnsureCode{59}{12}% ;
\TMP@EnsureCode{60}{12}% <
\TMP@EnsureCode{62}{12}% >
\TMP@EnsureCode{63}{12}% ?
\TMP@EnsureCode{91}{12}% [
\TMP@EnsureCode{93}{12}% ]
\TMP@EnsureCode{94}{7}% ^ (superscript)
\TMP@EnsureCode{95}{8}% _ (subscript)
\TMP@EnsureCode{96}{12}% `
\TMP@EnsureCode{124}{12}% |
\edef\HOLOGO@AtEnd{%
  \HOLOGO@AtEnd
  \escapechar\the\escapechar\relax
  \noexpand\endinput
}
\escapechar=92 %
%    \end{macrocode}
%
% \subsection{Logo list}
%
%    \begin{macro}{\hologoList}
%    \begin{macrocode}
\def\hologoList{%
  \hologoEntry{(La)TeX}{}{2011/10/01}%
  \hologoEntry{AmSLaTeX}{}{2010/04/16}%
  \hologoEntry{AmSTeX}{}{2010/04/16}%
  \hologoEntry{biber}{}{2011/10/01}%
  \hologoEntry{BibTeX}{}{2011/10/01}%
  \hologoEntry{BibTeX}{sf}{2011/10/01}%
  \hologoEntry{BibTeX}{sc}{2011/10/01}%
  \hologoEntry{BibTeX8}{}{2011/11/22}%
  \hologoEntry{ConTeXt}{}{2011/03/25}%
  \hologoEntry{ConTeXt}{narrow}{2011/03/25}%
  \hologoEntry{ConTeXt}{simple}{2011/03/25}%
  \hologoEntry{emTeX}{}{2010/04/26}%
  \hologoEntry{eTeX}{}{2010/04/08}%
  \hologoEntry{ExTeX}{}{2011/10/01}%
  \hologoEntry{HanTheThanh}{}{2011/11/29}%
  \hologoEntry{iniTeX}{}{2011/10/01}%
  \hologoEntry{KOMAScript}{}{2011/10/01}%
  \hologoEntry{La}{}{2010/05/08}%
  \hologoEntry{LaTeX}{}{2010/04/08}%
  \hologoEntry{LaTeX2e}{}{2010/04/08}%
  \hologoEntry{LaTeX3}{}{2010/04/24}%
  \hologoEntry{LaTeXe}{}{2010/04/08}%
  \hologoEntry{LaTeXML}{}{2011/11/22}%
  \hologoEntry{LaTeXTeX}{}{2011/10/01}%
  \hologoEntry{LuaLaTeX}{}{2010/04/08}%
  \hologoEntry{LuaTeX}{}{2010/04/08}%
  \hologoEntry{LyX}{}{2011/10/01}%
  \hologoEntry{METAFONT}{}{2011/10/01}%
  \hologoEntry{MetaFun}{}{2011/10/01}%
  \hologoEntry{METAPOST}{}{2011/10/01}%
  \hologoEntry{MetaPost}{}{2011/10/01}%
  \hologoEntry{MiKTeX}{}{2011/10/01}%
  \hologoEntry{NTS}{}{2011/10/01}%
  \hologoEntry{OzMF}{}{2011/10/01}%
  \hologoEntry{OzMP}{}{2011/10/01}%
  \hologoEntry{OzTeX}{}{2011/10/01}%
  \hologoEntry{OzTtH}{}{2011/10/01}%
  \hologoEntry{PCTeX}{}{2011/10/01}%
  \hologoEntry{pdfTeX}{}{2011/10/01}%
  \hologoEntry{pdfLaTeX}{}{2011/10/01}%
  \hologoEntry{PiC}{}{2011/10/01}%
  \hologoEntry{PiCTeX}{}{2011/10/01}%
  \hologoEntry{plainTeX}{}{2010/04/08}%
  \hologoEntry{plainTeX}{space}{2010/04/16}%
  \hologoEntry{plainTeX}{hyphen}{2010/04/16}%
  \hologoEntry{plainTeX}{runtogether}{2010/04/16}%
  \hologoEntry{SageTeX}{}{2011/11/22}%
  \hologoEntry{SLiTeX}{}{2011/10/01}%
  \hologoEntry{SLiTeX}{lift}{2011/10/01}%
  \hologoEntry{SLiTeX}{narrow}{2011/10/01}%
  \hologoEntry{SLiTeX}{simple}{2011/10/01}%
  \hologoEntry{SliTeX}{}{2011/10/01}%
  \hologoEntry{SliTeX}{narrow}{2011/10/01}%
  \hologoEntry{SliTeX}{simple}{2011/10/01}%
  \hologoEntry{SliTeX}{lift}{2011/10/01}%
  \hologoEntry{teTeX}{}{2011/10/01}%
  \hologoEntry{TeX}{}{2010/04/08}%
  \hologoEntry{TeX4ht}{}{2011/11/22}%
  \hologoEntry{TTH}{}{2011/11/22}%
  \hologoEntry{virTeX}{}{2011/10/01}%
  \hologoEntry{VTeX}{}{2010/04/24}%
  \hologoEntry{Xe}{}{2010/04/08}%
  \hologoEntry{XeLaTeX}{}{2010/04/08}%
  \hologoEntry{XeTeX}{}{2010/04/08}%
}
%    \end{macrocode}
%    \end{macro}
%
% \subsection{Load resources}
%
%    \begin{macrocode}
\begingroup\expandafter\expandafter\expandafter\endgroup
\expandafter\ifx\csname RequirePackage\endcsname\relax
  \def\TMP@RequirePackage#1[#2]{%
    \begingroup\expandafter\expandafter\expandafter\endgroup
    \expandafter\ifx\csname ver@#1.sty\endcsname\relax
      \input #1.sty\relax
    \fi
  }%
  \TMP@RequirePackage{ltxcmds}[2011/02/04]%
  \TMP@RequirePackage{infwarerr}[2010/04/08]%
  \TMP@RequirePackage{kvsetkeys}[2010/03/01]%
  \TMP@RequirePackage{kvdefinekeys}[2010/03/01]%
  \TMP@RequirePackage{pdftexcmds}[2010/04/01]%
  \TMP@RequirePackage{ifpdf}[2010/01/28]%
  \TMP@RequirePackage{ifluatex}[2010/03/01]%
  \ltx@IfUndefined{newif}{%
    \expandafter\let\csname newif\endcsname\ltx@newif
  }{}%
  \TMP@RequirePackage{ifxetex}[2009/01/23]%
  \TMP@RequirePackage{ifvtex}[2010/03/01]%
\else
  \RequirePackage{ltxcmds}[2011/02/04]%
  \RequirePackage{infwarerr}[2010/04/08]%
  \RequirePackage{kvsetkeys}[2010/03/01]%
  \RequirePackage{kvdefinekeys}[2010/03/01]%
  \RequirePackage{pdftexcmds}[2010/04/01]%
  \RequirePackage{ifpdf}[2010/01/28]%
  \RequirePackage{ifluatex}[2010/03/01]%
  \RequirePackage{ifxetex}[2009/01/23]%
  \RequirePackage{ifvtex}[2010/03/01]%
\fi
%    \end{macrocode}
%
%    \begin{macro}{\HOLOGO@IfDefined}
%    \begin{macrocode}
\def\HOLOGO@IfExists#1{%
  \ifx\@undefined#1%
    \expandafter\ltx@secondoftwo
  \else
    \ifx\relax#1%
      \expandafter\ltx@secondoftwo
    \else
      \expandafter\expandafter\expandafter\ltx@firstoftwo
    \fi
  \fi
}
%    \end{macrocode}
%    \end{macro}
%
% \subsection{Setup macros}
%
%    \begin{macro}{\hologoSetup}
%    \begin{macrocode}
\def\hologoSetup{%
  \let\HOLOGO@name\relax
  \HOLOGO@Setup
}
%    \end{macrocode}
%    \end{macro}
%
%    \begin{macro}{\hologoLogoSetup}
%    \begin{macrocode}
\def\hologoLogoSetup#1{%
  \edef\HOLOGO@name{#1}%
  \ltx@IfUndefined{HoLogo@\HOLOGO@name}{%
    \@PackageError{hologo}{%
      Unknown logo `\HOLOGO@name'%
    }\@ehc
    \ltx@gobble
  }{%
    \HOLOGO@Setup
  }%
}
%    \end{macrocode}
%    \end{macro}
%
%    \begin{macro}{\HOLOGO@Setup}
%    \begin{macrocode}
\def\HOLOGO@Setup{%
  \kvsetkeys{HoLogo}%
}
%    \end{macrocode}
%    \end{macro}
%
% \subsection{Options}
%
%    \begin{macro}{\HOLOGO@DeclareBoolOption}
%    \begin{macrocode}
\def\HOLOGO@DeclareBoolOption#1{%
  \expandafter\chardef\csname HOLOGOOPT@#1\endcsname\ltx@zero
  \kv@define@key{HoLogo}{#1}[true]{%
    \def\HOLOGO@temp{##1}%
    \ifx\HOLOGO@temp\HOLOGO@true
      \ifx\HOLOGO@name\relax
        \expandafter\chardef\csname HOLOGOOPT@#1\endcsname=\ltx@one
      \else
        \expandafter\chardef\csname
        HoLogoOpt@#1@\HOLOGO@name\endcsname\ltx@one
      \fi
      \HOLOGO@SetBreakAll{#1}%
    \else
      \ifx\HOLOGO@temp\HOLOGO@false
        \ifx\HOLOGO@name\relax
          \expandafter\chardef\csname HOLOGOOPT@#1\endcsname=\ltx@zero
        \else
          \expandafter\chardef\csname
          HoLogoOpt@#1@\HOLOGO@name\endcsname=\ltx@zero
        \fi
        \HOLOGO@SetBreakAll{#1}%
      \else
        \@PackageError{hologo}{%
          Unknown value `##1' for boolean option `#1'.\MessageBreak
          Known values are `true' and `false'%
        }\@ehc
      \fi
    \fi
  }%
}
%    \end{macrocode}
%    \end{macro}
%
%    \begin{macro}{\HOLOGO@SetBreakAll}
%    \begin{macrocode}
\def\HOLOGO@SetBreakAll#1{%
  \def\HOLOGO@temp{#1}%
  \ifx\HOLOGO@temp\HOLOGO@break
    \ifx\HOLOGO@name\relax
      \chardef\HOLOGOOPT@hyphenbreak=\HOLOGOOPT@break
      \chardef\HOLOGOOPT@spacebreak=\HOLOGOOPT@break
      \chardef\HOLOGOOPT@discretionarybreak=\HOLOGOOPT@break
    \else
      \expandafter\chardef
         \csname HoLogoOpt@hyphenbreak@\HOLOGO@name\endcsname=%
         \csname HoLogoOpt@break@\HOLOGO@name\endcsname
      \expandafter\chardef
         \csname HoLogoOpt@spacebreak@\HOLOGO@name\endcsname=%
         \csname HoLogoOpt@break@\HOLOGO@name\endcsname
      \expandafter\chardef
         \csname HoLogoOpt@discretionarybreak@\HOLOGO@name
             \endcsname=%
         \csname HoLogoOpt@break@\HOLOGO@name\endcsname
    \fi
  \fi
}
%    \end{macrocode}
%    \end{macro}
%
%    \begin{macro}{\HOLOGO@true}
%    \begin{macrocode}
\def\HOLOGO@true{true}
%    \end{macrocode}
%    \end{macro}
%    \begin{macro}{\HOLOGO@false}
%    \begin{macrocode}
\def\HOLOGO@false{false}
%    \end{macrocode}
%    \end{macro}
%    \begin{macro}{\HOLOGO@break}
%    \begin{macrocode}
\def\HOLOGO@break{break}
%    \end{macrocode}
%    \end{macro}
%
%    \begin{macrocode}
\HOLOGO@DeclareBoolOption{break}
\HOLOGO@DeclareBoolOption{hyphenbreak}
\HOLOGO@DeclareBoolOption{spacebreak}
\HOLOGO@DeclareBoolOption{discretionarybreak}
%    \end{macrocode}
%
%    \begin{macrocode}
\kv@define@key{HoLogo}{variant}{%
  \ifx\HOLOGO@name\relax
    \@PackageError{hologo}{%
      Option `variant' is not available in \string\hologoSetup,%
      \MessageBreak
      Use \string\hologoLogoSetup\space instead%
    }\@ehc
  \else
    \edef\HOLOGO@temp{#1}%
    \ifx\HOLOGO@temp\ltx@empty
      \expandafter
      \let\csname HoLogoOpt@variant@\HOLOGO@name\endcsname\@undefined
    \else
      \ltx@IfUndefined{HoLogo@\HOLOGO@name @\HOLOGO@temp}{%
        \@PackageError{hologo}{%
          Unknown variant `\HOLOGO@temp' of logo `\HOLOGO@name'%
        }\@ehc
      }{%
        \expandafter
        \let\csname HoLogoOpt@variant@\HOLOGO@name\endcsname
            \HOLOGO@temp
      }%
    \fi
  \fi
}
%    \end{macrocode}
%
%    \begin{macro}{\HOLOGO@Variant}
%    \begin{macrocode}
\def\HOLOGO@Variant#1{%
  #1%
  \ltx@ifundefined{HoLogoOpt@variant@#1}{%
  }{%
    @\csname HoLogoOpt@variant@#1\endcsname
  }%
}
%    \end{macrocode}
%    \end{macro}
%
% \subsection{Break/no-break support}
%
%    \begin{macro}{\HOLOGO@space}
%    \begin{macrocode}
\def\HOLOGO@space{%
  \ltx@ifundefined{HoLogoOpt@spacebreak@\HOLOGO@name}{%
    \ltx@ifundefined{HoLogoOpt@break@\HOLOGO@name}{%
      \chardef\HOLOGO@temp=\HOLOGOOPT@spacebreak
    }{%
      \chardef\HOLOGO@temp=%
        \csname HoLogoOpt@break@\HOLOGO@name\endcsname
    }%
  }{%
    \chardef\HOLOGO@temp=%
      \csname HoLogoOpt@spacebreak@\HOLOGO@name\endcsname
  }%
  \ifcase\HOLOGO@temp
    \penalty10000 %
  \fi
  \ltx@space
}
%    \end{macrocode}
%    \end{macro}
%
%    \begin{macro}{\HOLOGO@hyphen}
%    \begin{macrocode}
\def\HOLOGO@hyphen{%
  \ltx@ifundefined{HoLogoOpt@hyphenbreak@\HOLOGO@name}{%
    \ltx@ifundefined{HoLogoOpt@break@\HOLOGO@name}{%
      \chardef\HOLOGO@temp=\HOLOGOOPT@hyphenbreak
    }{%
      \chardef\HOLOGO@temp=%
        \csname HoLogoOpt@break@\HOLOGO@name\endcsname
    }%
  }{%
    \chardef\HOLOGO@temp=%
      \csname HoLogoOpt@hyphenbreak@\HOLOGO@name\endcsname
  }%
  \ifcase\HOLOGO@temp
    \ltx@mbox{-}%
  \else
    -%
  \fi
}
%    \end{macrocode}
%    \end{macro}
%
%    \begin{macro}{\HOLOGO@discretionary}
%    \begin{macrocode}
\def\HOLOGO@discretionary{%
  \ltx@ifundefined{HoLogoOpt@discretionarybreak@\HOLOGO@name}{%
    \ltx@ifundefined{HoLogoOpt@break@\HOLOGO@name}{%
      \chardef\HOLOGO@temp=\HOLOGOOPT@discretionarybreak
    }{%
      \chardef\HOLOGO@temp=%
        \csname HoLogoOpt@break@\HOLOGO@name\endcsname
    }%
  }{%
    \chardef\HOLOGO@temp=%
      \csname HoLogoOpt@discretionarybreak@\HOLOGO@name\endcsname
  }%
  \ifcase\HOLOGO@temp
  \else
    \-%
  \fi
}
%    \end{macrocode}
%    \end{macro}
%
%    \begin{macro}{\HOLOGO@mbox}
%    \begin{macrocode}
\def\HOLOGO@mbox#1{%
  \ltx@ifundefined{HoLogoOpt@break@\HOLOGO@name}{%
    \chardef\HOLOGO@temp=\HOLOGOOPT@hyphenbreak
  }{%
    \chardef\HOLOGO@temp=%
      \csname HoLogoOpt@break@\HOLOGO@name\endcsname
  }%
  \ifcase\HOLOGO@temp
    \ltx@mbox{#1}%
  \else
    #1%
  \fi
}
%    \end{macrocode}
%    \end{macro}
%
% \subsection{Font support}
%
%    \begin{macro}{\HoLogoFont@font}
%    \begin{tabular}{@{}ll@{}}
%    |#1|:& logo name\\
%    |#2|:& font short name\\
%    |#3|:& text
%    \end{tabular}
%    \begin{macrocode}
\def\HoLogoFont@font#1#2#3{%
  \begingroup
    \ltx@IfUndefined{HoLogoFont@logo@#1.#2}{%
      \ltx@IfUndefined{HoLogoFont@font@#2}{%
        \@PackageWarning{hologo}{%
          Missing font `#2' for logo `#1'%
        }%
        #3%
      }{%
        \csname HoLogoFont@font@#2\endcsname{#3}%
      }%
    }{%
      \csname HoLogoFont@logo@#1.#2\endcsname{#3}%
    }%
  \endgroup
}
%    \end{macrocode}
%    \end{macro}
%
%    \begin{macro}{\HoLogoFont@Def}
%    \begin{macrocode}
\def\HoLogoFont@Def#1{%
  \expandafter\def\csname HoLogoFont@font@#1\endcsname
}
%    \end{macrocode}
%    \end{macro}
%    \begin{macro}{\HoLogoFont@LogoDef}
%    \begin{macrocode}
\def\HoLogoFont@LogoDef#1#2{%
  \expandafter\def\csname HoLogoFont@logo@#1.#2\endcsname
}
%    \end{macrocode}
%    \end{macro}
%
% \subsubsection{Font defaults}
%
%    \begin{macro}{\HoLogoFont@font@general}
%    \begin{macrocode}
\HoLogoFont@Def{general}{}%
%    \end{macrocode}
%    \end{macro}
%
%    \begin{macro}{\HoLogoFont@font@rm}
%    \begin{macrocode}
\ltx@IfUndefined{rmfamily}{%
  \ltx@IfUndefined{rm}{%
  }{%
    \HoLogoFont@Def{rm}{\rm}%
  }%
}{%
  \HoLogoFont@Def{rm}{\rmfamily}%
}
%    \end{macrocode}
%    \end{macro}
%
%    \begin{macro}{\HoLogoFont@font@sf}
%    \begin{macrocode}
\ltx@IfUndefined{sffamily}{%
  \ltx@IfUndefined{sf}{%
  }{%
    \HoLogoFont@Def{sf}{\sf}%
  }%
}{%
  \HoLogoFont@Def{sf}{\sffamily}%
}
%    \end{macrocode}
%    \end{macro}
%
%    \begin{macro}{\HoLogoFont@font@bibsf}
%    In case of \hologo{plainTeX} the original small caps
%    variant is used as default. In \hologo{LaTeX}
%    the definition of package \xpackage{dtklogos} \cite{dtklogos}
%    is used.
%\begin{quote}
%\begin{verbatim}
%\DeclareRobustCommand{\BibTeX}{%
%  B%
%  \kern-.05em%
%  \hbox{%
%    $\m@th$% %% force math size calculations
%    \csname S@\f@size\endcsname
%    \fontsize\sf@size\z@
%    \math@fontsfalse
%    \selectfont
%    I%
%    \kern-.025em%
%    B
%  }%
%  \kern-.08em%
%  \-%
%  \TeX
%}
%\end{verbatim}
%\end{quote}
%    \begin{macrocode}
\ltx@IfUndefined{selectfont}{%
  \ltx@IfUndefined{tensc}{%
    \font\tensc=cmcsc10\relax
  }{}%
  \HoLogoFont@Def{bibsf}{\tensc}%
}{%
  \HoLogoFont@Def{bibsf}{%
    $\mathsurround=0pt$%
    \csname S@\f@size\endcsname
    \fontsize\sf@size{0pt}%
    \math@fontsfalse
    \selectfont
  }%
}
%    \end{macrocode}
%    \end{macro}
%
%    \begin{macro}{\HoLogoFont@font@sc}
%    \begin{macrocode}
\ltx@IfUndefined{scshape}{%
  \ltx@IfUndefined{tensc}{%
    \font\tensc=cmcsc10\relax
  }{}%
  \HoLogoFont@Def{sc}{\tensc}%
}{%
  \HoLogoFont@Def{sc}{\scshape}%
}
%    \end{macrocode}
%    \end{macro}
%
%    \begin{macro}{\HoLogoFont@font@sy}
%    \begin{macrocode}
\ltx@IfUndefined{usefont}{%
  \ltx@IfUndefined{tensy}{%
  }{%
    \HoLogoFont@Def{sy}{\tensy}%
  }%
}{%
  \HoLogoFont@Def{sy}{%
    \usefont{OMS}{cmsy}{m}{n}%
  }%
}
%    \end{macrocode}
%    \end{macro}
%
%    \begin{macro}{\HoLogoFont@font@logo}
%    \begin{macrocode}
\begingroup
  \def\x{LaTeX2e}%
\expandafter\endgroup
\ifx\fmtname\x
  \ltx@IfUndefined{logofamily}{%
    \DeclareRobustCommand\logofamily{%
      \not@math@alphabet\logofamily\relax
      \fontencoding{U}%
      \fontfamily{logo}%
      \selectfont
    }%
  }{}%
  \ltx@IfUndefined{logofamily}{%
  }{%
    \HoLogoFont@Def{logo}{\logofamily}%
  }%
\else
  \ltx@IfUndefined{tenlogo}{%
    \font\tenlogo=logo10\relax
  }{}%
  \HoLogoFont@Def{logo}{\tenlogo}%
\fi
%    \end{macrocode}
%    \end{macro}
%
% \subsubsection{Font setup}
%
%    \begin{macro}{\hologoFontSetup}
%    \begin{macrocode}
\def\hologoFontSetup{%
  \let\HOLOGO@name\relax
  \HOLOGO@FontSetup
}
%    \end{macrocode}
%    \end{macro}
%
%    \begin{macro}{\hologoLogoFontSetup}
%    \begin{macrocode}
\def\hologoLogoFontSetup#1{%
  \edef\HOLOGO@name{#1}%
  \ltx@IfUndefined{HoLogo@\HOLOGO@name}{%
    \@PackageError{hologo}{%
      Unknown logo `\HOLOGO@name'%
    }\@ehc
    \ltx@gobble
  }{%
    \HOLOGO@FontSetup
  }%
}
%    \end{macrocode}
%    \end{macro}
%
%    \begin{macro}{\HOLOGO@FontSetup}
%    \begin{macrocode}
\def\HOLOGO@FontSetup{%
  \kvsetkeys{HoLogoFont}%
}
%    \end{macrocode}
%    \end{macro}
%
%    \begin{macrocode}
\def\HOLOGO@temp#1{%
  \kv@define@key{HoLogoFont}{#1}{%
    \ifx\HOLOGO@name\relax
      \HoLogoFont@Def{#1}{##1}%
    \else
      \HoLogoFont@LogoDef\HOLOGO@name{#1}{##1}%
    \fi
  }%
}
\HOLOGO@temp{general}
\HOLOGO@temp{sf}
%    \end{macrocode}
%
% \subsection{Generic logo commands}
%
%    \begin{macrocode}
\HOLOGO@IfExists\hologo{%
  \@PackageError{hologo}{%
    \string\hologo\ltx@space is already defined.\MessageBreak
    Package loading is aborted%
  }\@ehc
  \HOLOGO@AtEnd
}%
\HOLOGO@IfExists\hologoRobust{%
  \@PackageError{hologo}{%
    \string\hologoRobust\ltx@space is already defined.\MessageBreak
    Package loading is aborted%
  }\@ehc
  \HOLOGO@AtEnd
}%
%    \end{macrocode}
%
% \subsubsection{\cs{hologo} and friends}
%
%    \begin{macrocode}
\ifluatex
  \expandafter\ltx@firstofone
\else
  \expandafter\ltx@gobble
\fi
{%
  \ltx@IfUndefined{ifincsname}{%
    \ifnum\luatexversion<36 %
      \expandafter\ltx@gobble
    \else
      \expandafter\ltx@firstofone
    \fi
    {%
      \begingroup
        \ifcase0%
            \directlua{%
              if tex.enableprimitives then %
                tex.enableprimitives('HOLOGO@', {'ifincsname'})%
              else %
                tex.print('1')%
              end%
            }%
            \ifx\HOLOGO@ifincsname\@undefined 1\fi%
            \relax
          \expandafter\ltx@firstofone
        \else
          \endgroup
          \expandafter\ltx@gobble
        \fi
        {%
          \global\let\ifincsname\HOLOGO@ifincsname
        }%
      \HOLOGO@temp
    }%
  }{}%
}
%    \end{macrocode}
%    \begin{macrocode}
\ltx@IfUndefined{ifincsname}{%
  \catcode`$=14 %
}{%
  \catcode`$=9 %
}
%    \end{macrocode}
%
%    \begin{macro}{\hologo}
%    \begin{macrocode}
\def\hologo#1{%
$ \ifincsname
$   \ltx@ifundefined{HoLogoCs@\HOLOGO@Variant{#1}}{%
$     #1%
$   }{%
$     \csname HoLogoCs@\HOLOGO@Variant{#1}\endcsname\ltx@firstoftwo
$   }%
$ \else
    \HOLOGO@IfExists\texorpdfstring\texorpdfstring\ltx@firstoftwo
    {%
      \hologoRobust{#1}%
    }{%
      \ltx@ifundefined{HoLogoBkm@\HOLOGO@Variant{#1}}{%
        \ltx@ifundefined{HoLogo@#1}{?#1?}{#1}%
      }{%
        \csname HoLogoBkm@\HOLOGO@Variant{#1}\endcsname
        \ltx@firstoftwo
      }%
    }%
$ \fi
}
%    \end{macrocode}
%    \end{macro}
%    \begin{macro}{\Hologo}
%    \begin{macrocode}
\def\Hologo#1{%
$ \ifincsname
$   \ltx@ifundefined{HoLogoCs@\HOLOGO@Variant{#1}}{%
$     #1%
$   }{%
$     \csname HoLogoCs@\HOLOGO@Variant{#1}\endcsname\ltx@secondoftwo
$   }%
$ \else
    \HOLOGO@IfExists\texorpdfstring\texorpdfstring\ltx@firstoftwo
    {%
      \HologoRobust{#1}%
    }{%
      \ltx@ifundefined{HoLogoBkm@\HOLOGO@Variant{#1}}{%
        \ltx@ifundefined{HoLogo@#1}{?#1?}{#1}%
      }{%
        \csname HoLogoBkm@\HOLOGO@Variant{#1}\endcsname
        \ltx@secondoftwo
      }%
    }%
$ \fi
}
%    \end{macrocode}
%    \end{macro}
%
%    \begin{macro}{\hologoVariant}
%    \begin{macrocode}
\def\hologoVariant#1#2{%
  \ifx\relax#2\relax
    \hologo{#1}%
  \else
$   \ifincsname
$     \ltx@ifundefined{HoLogoCs@#1@#2}{%
$       #1%
$     }{%
$       \csname HoLogoCs@#1@#2\endcsname\ltx@firstoftwo
$     }%
$   \else
      \HOLOGO@IfExists\texorpdfstring\texorpdfstring\ltx@firstoftwo
      {%
        \hologoVariantRobust{#1}{#2}%
      }{%
        \ltx@ifundefined{HoLogoBkm@#1@#2}{%
          \ltx@ifundefined{HoLogo@#1}{?#1?}{#1}%
        }{%
          \csname HoLogoBkm@#1@#2\endcsname
          \ltx@firstoftwo
        }%
      }%
$   \fi
  \fi
}
%    \end{macrocode}
%    \end{macro}
%    \begin{macro}{\HologoVariant}
%    \begin{macrocode}
\def\HologoVariant#1#2{%
  \ifx\relax#2\relax
    \Hologo{#1}%
  \else
$   \ifincsname
$     \ltx@ifundefined{HoLogoCs@#1@#2}{%
$       #1%
$     }{%
$       \csname HoLogoCs@#1@#2\endcsname\ltx@secondoftwo
$     }%
$   \else
      \HOLOGO@IfExists\texorpdfstring\texorpdfstring\ltx@firstoftwo
      {%
        \HologoVariantRobust{#1}{#2}%
      }{%
        \ltx@ifundefined{HoLogoBkm@#1@#2}{%
          \ltx@ifundefined{HoLogo@#1}{?#1?}{#1}%
        }{%
          \csname HoLogoBkm@#1@#2\endcsname
          \ltx@secondoftwo
        }%
      }%
$   \fi
  \fi
}
%    \end{macrocode}
%    \end{macro}
%
%    \begin{macrocode}
\catcode`\$=3 %
%    \end{macrocode}
%
% \subsubsection{\cs{hologoRobust} and friends}
%
%    \begin{macro}{\hologoRobust}
%    \begin{macrocode}
\ltx@IfUndefined{protected}{%
  \ltx@IfUndefined{DeclareRobustCommand}{%
    \def\hologoRobust#1%
  }{%
    \DeclareRobustCommand*\hologoRobust[1]%
  }%
}{%
  \protected\def\hologoRobust#1%
}%
{%
  \edef\HOLOGO@name{#1}%
  \ltx@IfUndefined{HoLogo@\HOLOGO@Variant\HOLOGO@name}{%
    \@PackageError{hologo}{%
      Unknown logo `\HOLOGO@name'%
    }\@ehc
    ?\HOLOGO@name?%
  }{%
    \ltx@IfUndefined{ver@tex4ht.sty}{%
      \HoLogoFont@font\HOLOGO@name{general}{%
        \csname HoLogo@\HOLOGO@Variant\HOLOGO@name\endcsname
        \ltx@firstoftwo
      }%
    }{%
      \ltx@IfUndefined{HoLogoHtml@\HOLOGO@Variant\HOLOGO@name}{%
        \HOLOGO@name
      }{%
        \csname HoLogoHtml@\HOLOGO@Variant\HOLOGO@name\endcsname
        \ltx@firstoftwo
      }%
    }%
  }%
}
%    \end{macrocode}
%    \end{macro}
%    \begin{macro}{\HologoRobust}
%    \begin{macrocode}
\ltx@IfUndefined{protected}{%
  \ltx@IfUndefined{DeclareRobustCommand}{%
    \def\HologoRobust#1%
  }{%
    \DeclareRobustCommand*\HologoRobust[1]%
  }%
}{%
  \protected\def\HologoRobust#1%
}%
{%
  \edef\HOLOGO@name{#1}%
  \ltx@IfUndefined{HoLogo@\HOLOGO@Variant\HOLOGO@name}{%
    \@PackageError{hologo}{%
      Unknown logo `\HOLOGO@name'%
    }\@ehc
    ?\HOLOGO@name?%
  }{%
    \ltx@IfUndefined{ver@tex4ht.sty}{%
      \HoLogoFont@font\HOLOGO@name{general}{%
        \csname HoLogo@\HOLOGO@Variant\HOLOGO@name\endcsname
        \ltx@secondoftwo
      }%
    }{%
      \ltx@IfUndefined{HoLogoHtml@\HOLOGO@Variant\HOLOGO@name}{%
        \expandafter\HOLOGO@Uppercase\HOLOGO@name
      }{%
        \csname HoLogoHtml@\HOLOGO@Variant\HOLOGO@name\endcsname
        \ltx@secondoftwo
      }%
    }%
  }%
}
%    \end{macrocode}
%    \end{macro}
%    \begin{macro}{\hologoVariantRobust}
%    \begin{macrocode}
\ltx@IfUndefined{protected}{%
  \ltx@IfUndefined{DeclareRobustCommand}{%
    \def\hologoVariantRobust#1#2%
  }{%
    \DeclareRobustCommand*\hologoVariantRobust[2]%
  }%
}{%
  \protected\def\hologoVariantRobust#1#2%
}%
{%
  \begingroup
    \hologoLogoSetup{#1}{variant={#2}}%
    \hologoRobust{#1}%
  \endgroup
}
%    \end{macrocode}
%    \end{macro}
%    \begin{macro}{\HologoVariantRobust}
%    \begin{macrocode}
\ltx@IfUndefined{protected}{%
  \ltx@IfUndefined{DeclareRobustCommand}{%
    \def\HologoVariantRobust#1#2%
  }{%
    \DeclareRobustCommand*\HologoVariantRobust[2]%
  }%
}{%
  \protected\def\HologoVariantRobust#1#2%
}%
{%
  \begingroup
    \hologoLogoSetup{#1}{variant={#2}}%
    \HologoRobust{#1}%
  \endgroup
}
%    \end{macrocode}
%    \end{macro}
%
%    \begin{macro}{\hologorobust}
%    Macro \cs{hologorobust} is only defined for compatibility.
%    Its use is deprecated.
%    \begin{macrocode}
\def\hologorobust{\hologoRobust}
%    \end{macrocode}
%    \end{macro}
%
% \subsection{Helpers}
%
%    \begin{macro}{\HOLOGO@Uppercase}
%    Macro \cs{HOLOGO@Uppercase} is restricted to \cs{uppercase},
%    because \hologo{plainTeX} or \hologo{iniTeX} do not provide
%    \cs{MakeUppercase}.
%    \begin{macrocode}
\def\HOLOGO@Uppercase#1{\uppercase{#1}}
%    \end{macrocode}
%    \end{macro}
%
%    \begin{macro}{\HOLOGO@PdfdocUnicode}
%    \begin{macrocode}
\def\HOLOGO@PdfdocUnicode{%
  \ifx\ifHy@unicode\iftrue
    \expandafter\ltx@secondoftwo
  \else
    \expandafter\ltx@firstoftwo
  \fi
}
%    \end{macrocode}
%    \end{macro}
%
%    \begin{macro}{\HOLOGO@Math}
%    \begin{macrocode}
\def\HOLOGO@MathSetup{%
  \mathsurround0pt\relax
  \HOLOGO@IfExists\f@series{%
    \if b\expandafter\ltx@car\f@series x\@nil
      \csname boldmath\endcsname
   \fi
  }{}%
}
%    \end{macrocode}
%    \end{macro}
%
%    \begin{macro}{\HOLOGO@TempDimen}
%    \begin{macrocode}
\dimendef\HOLOGO@TempDimen=\ltx@zero
%    \end{macrocode}
%    \end{macro}
%    \begin{macro}{\HOLOGO@NegativeKerning}
%    \begin{macrocode}
\def\HOLOGO@NegativeKerning#1{%
  \begingroup
    \HOLOGO@TempDimen=0pt\relax
    \comma@parse@normalized{#1}{%
      \ifdim\HOLOGO@TempDimen=0pt %
        \expandafter\HOLOGO@@NegativeKerning\comma@entry
      \fi
      \ltx@gobble
    }%
    \ifdim\HOLOGO@TempDimen<0pt %
      \kern\HOLOGO@TempDimen
    \fi
  \endgroup
}
%    \end{macrocode}
%    \end{macro}
%    \begin{macro}{\HOLOGO@@NegativeKerning}
%    \begin{macrocode}
\def\HOLOGO@@NegativeKerning#1#2{%
  \setbox\ltx@zero\hbox{#1#2}%
  \HOLOGO@TempDimen=\wd\ltx@zero
  \setbox\ltx@zero\hbox{#1\kern0pt#2}%
  \advance\HOLOGO@TempDimen by -\wd\ltx@zero
}
%    \end{macrocode}
%    \end{macro}
%
%    \begin{macro}{\HOLOGO@SpaceFactor}
%    \begin{macrocode}
\def\HOLOGO@SpaceFactor{%
  \spacefactor1000 %
}
%    \end{macrocode}
%    \end{macro}
%
%    \begin{macro}{\HOLOGO@Span}
%    \begin{macrocode}
\def\HOLOGO@Span#1#2{%
  \HCode{<span class="HoLogo-#1">}%
  #2%
  \HCode{</span>}%
}
%    \end{macrocode}
%    \end{macro}
%
% \subsubsection{Text subscript}
%
%    \begin{macro}{\HOLOGO@SubScript}%
%    \begin{macrocode}
\def\HOLOGO@SubScript#1{%
  \ltx@IfUndefined{textsubscript}{%
    \ltx@IfUndefined{text}{%
      \ltx@mbox{%
        \mathsurround=0pt\relax
        $%
          _{%
            \ltx@IfUndefined{sf@size}{%
              \mathrm{#1}%
            }{%
              \mbox{%
                \fontsize\sf@size{0pt}\selectfont
                #1%
              }%
            }%
          }%
        $%
      }%
    }{%
      \ltx@mbox{%
        \mathsurround=0pt\relax
        $_{\text{#1}}$%
      }%
    }%
  }{%
    \textsubscript{#1}%
  }%
}
%    \end{macrocode}
%    \end{macro}
%
% \subsection{\hologo{TeX} and friends}
%
% \subsubsection{\hologo{TeX}}
%
%    \begin{macro}{\HoLogo@TeX}
%    Source: \hologo{LaTeX} kernel.
%    \begin{macrocode}
\def\HoLogo@TeX#1{%
  T\kern-.1667em\lower.5ex\hbox{E}\kern-.125emX\HOLOGO@SpaceFactor
}
%    \end{macrocode}
%    \end{macro}
%    \begin{macro}{\HoLogoHtml@TeX}
%    \begin{macrocode}
\def\HoLogoHtml@TeX#1{%
  \HoLogoCss@TeX
  \HOLOGO@Span{TeX}{%
    T%
    \HOLOGO@Span{e}{%
      E%
    }%
    X%
  }%
}
%    \end{macrocode}
%    \end{macro}
%    \begin{macro}{\HoLogoCss@TeX}
%    \begin{macrocode}
\def\HoLogoCss@TeX{%
  \Css{%
    span.HoLogo-TeX span.HoLogo-e{%
      position:relative;%
      top:.5ex;%
      margin-left:-.1667em;%
      margin-right:-.125em;%
    }%
  }%
  \Css{%
    a span.HoLogo-TeX span.HoLogo-e{%
      text-decoration:none;%
    }%
  }%
  \global\let\HoLogoCss@TeX\relax
}
%    \end{macrocode}
%    \end{macro}
%
% \subsubsection{\hologo{plainTeX}}
%
%    \begin{macro}{\HoLogo@plainTeX@space}
%    Source: ``The \hologo{TeX}book''
%    \begin{macrocode}
\def\HoLogo@plainTeX@space#1{%
  \HOLOGO@mbox{#1{p}{P}lain}\HOLOGO@space\hologo{TeX}%
}
%    \end{macrocode}
%    \end{macro}
%    \begin{macro}{\HoLogoCs@plainTeX@space}
%    \begin{macrocode}
\def\HoLogoCs@plainTeX@space#1{#1{p}{P}lain TeX}%
%    \end{macrocode}
%    \end{macro}
%    \begin{macro}{\HoLogoBkm@plainTeX@space}
%    \begin{macrocode}
\def\HoLogoBkm@plainTeX@space#1{%
  #1{p}{P}lain \hologo{TeX}%
}
%    \end{macrocode}
%    \end{macro}
%    \begin{macro}{\HoLogoHtml@plainTeX@space}
%    \begin{macrocode}
\def\HoLogoHtml@plainTeX@space#1{%
  #1{p}{P}lain \hologo{TeX}%
}
%    \end{macrocode}
%    \end{macro}
%
%    \begin{macro}{\HoLogo@plainTeX@hyphen}
%    \begin{macrocode}
\def\HoLogo@plainTeX@hyphen#1{%
  \HOLOGO@mbox{#1{p}{P}lain}\HOLOGO@hyphen\hologo{TeX}%
}
%    \end{macrocode}
%    \end{macro}
%    \begin{macro}{\HoLogoCs@plainTeX@hyphen}
%    \begin{macrocode}
\def\HoLogoCs@plainTeX@hyphen#1{#1{p}{P}lain-TeX}
%    \end{macrocode}
%    \end{macro}
%    \begin{macro}{\HoLogoBkm@plainTeX@hyphen}
%    \begin{macrocode}
\def\HoLogoBkm@plainTeX@hyphen#1{%
  #1{p}{P}lain-\hologo{TeX}%
}
%    \end{macrocode}
%    \end{macro}
%    \begin{macro}{\HoLogoHtml@plainTeX@hyphen}
%    \begin{macrocode}
\def\HoLogoHtml@plainTeX@hyphen#1{%
  #1{p}{P}lain-\hologo{TeX}%
}
%    \end{macrocode}
%    \end{macro}
%
%    \begin{macro}{\HoLogo@plainTeX@runtogether}
%    \begin{macrocode}
\def\HoLogo@plainTeX@runtogether#1{%
  \HOLOGO@mbox{#1{p}{P}lain\hologo{TeX}}%
}
%    \end{macrocode}
%    \end{macro}
%    \begin{macro}{\HoLogoCs@plainTeX@runtogether}
%    \begin{macrocode}
\def\HoLogoCs@plainTeX@runtogether#1{#1{p}{P}lainTeX}
%    \end{macrocode}
%    \end{macro}
%    \begin{macro}{\HoLogoBkm@plainTeX@runtogether}
%    \begin{macrocode}
\def\HoLogoBkm@plainTeX@runtogether#1{%
  #1{p}{P}lain\hologo{TeX}%
}
%    \end{macrocode}
%    \end{macro}
%    \begin{macro}{\HoLogoHtml@plainTeX@runtogether}
%    \begin{macrocode}
\def\HoLogoHtml@plainTeX@runtogether#1{%
  #1{p}{P}lain\hologo{TeX}%
}
%    \end{macrocode}
%    \end{macro}
%
%    \begin{macro}{\HoLogo@plainTeX}
%    \begin{macrocode}
\def\HoLogo@plainTeX{\HoLogo@plainTeX@space}
%    \end{macrocode}
%    \end{macro}
%    \begin{macro}{\HoLogoCs@plainTeX}
%    \begin{macrocode}
\def\HoLogoCs@plainTeX{\HoLogoCs@plainTeX@space}
%    \end{macrocode}
%    \end{macro}
%    \begin{macro}{\HoLogoBkm@plainTeX}
%    \begin{macrocode}
\def\HoLogoBkm@plainTeX{\HoLogoBkm@plainTeX@space}
%    \end{macrocode}
%    \end{macro}
%    \begin{macro}{\HoLogoHtml@plainTeX}
%    \begin{macrocode}
\def\HoLogoHtml@plainTeX{\HoLogoHtml@plainTeX@space}
%    \end{macrocode}
%    \end{macro}
%
% \subsubsection{\hologo{LaTeX}}
%
%    Source: \hologo{LaTeX} kernel.
%\begin{quote}
%\begin{verbatim}
%\DeclareRobustCommand{\LaTeX}{%
%  L%
%  \kern-.36em%
%  {%
%    \sbox\z@ T%
%    \vbox to\ht\z@{%
%      \hbox{%
%        \check@mathfonts
%        \fontsize\sf@size\z@
%        \math@fontsfalse
%        \selectfont
%        A%
%      }%
%      \vss
%    }%
%  }%
%  \kern-.15em%
%  \TeX
%}
%\end{verbatim}
%\end{quote}
%
%    \begin{macro}{\HoLogo@La}
%    \begin{macrocode}
\def\HoLogo@La#1{%
  L%
  \kern-.36em%
  \begingroup
    \setbox\ltx@zero\hbox{T}%
    \vbox to\ht\ltx@zero{%
      \hbox{%
        \ltx@ifundefined{check@mathfonts}{%
          \csname sevenrm\endcsname
        }{%
          \check@mathfonts
          \fontsize\sf@size{0pt}%
          \math@fontsfalse\selectfont
        }%
        A%
      }%
      \vss
    }%
  \endgroup
}
%    \end{macrocode}
%    \end{macro}
%
%    \begin{macro}{\HoLogo@LaTeX}
%    Source: \hologo{LaTeX} kernel.
%    \begin{macrocode}
\def\HoLogo@LaTeX#1{%
  \hologo{La}%
  \kern-.15em%
  \hologo{TeX}%
}
%    \end{macrocode}
%    \end{macro}
%    \begin{macro}{\HoLogoHtml@LaTeX}
%    \begin{macrocode}
\def\HoLogoHtml@LaTeX#1{%
  \HoLogoCss@LaTeX
  \HOLOGO@Span{LaTeX}{%
    L%
    \HOLOGO@Span{a}{%
      A%
    }%
    \hologo{TeX}%
  }%
}
%    \end{macrocode}
%    \end{macro}
%    \begin{macro}{\HoLogoCss@LaTeX}
%    \begin{macrocode}
\def\HoLogoCss@LaTeX{%
  \Css{%
    span.HoLogo-LaTeX span.HoLogo-a{%
      position:relative;%
      top:-.5ex;%
      margin-left:-.36em;%
      margin-right:-.15em;%
      font-size:85\%;%
    }%
  }%
  \global\let\HoLogoCss@LaTeX\relax
}
%    \end{macrocode}
%    \end{macro}
%
% \subsubsection{\hologo{(La)TeX}}
%
%    \begin{macro}{\HoLogo@LaTeXTeX}
%    The kerning around the parentheses is taken
%    from package \xpackage{dtklogos} \cite{dtklogos}.
%\begin{quote}
%\begin{verbatim}
%\DeclareRobustCommand{\LaTeXTeX}{%
%  (%
%  \kern-.15em%
%  L%
%  \kern-.36em%
%  {%
%    \sbox\z@ T%
%    \vbox to\ht0{%
%      \hbox{%
%        $\m@th$%
%        \csname S@\f@size\endcsname
%        \fontsize\sf@size\z@
%        \math@fontsfalse
%        \selectfont
%        A%
%      }%
%      \vss
%    }%
%  }%
%  \kern-.2em%
%  )%
%  \kern-.15em%
%  \TeX
%}
%\end{verbatim}
%\end{quote}
%    \begin{macrocode}
\def\HoLogo@LaTeXTeX#1{%
  (%
  \kern-.15em%
  \hologo{La}%
  \kern-.2em%
  )%
  \kern-.15em%
  \hologo{TeX}%
}
%    \end{macrocode}
%    \end{macro}
%    \begin{macro}{\HoLogoBkm@LaTeXTeX}
%    \begin{macrocode}
\def\HoLogoBkm@LaTeXTeX#1{(La)TeX}
%    \end{macrocode}
%    \end{macro}
%
%    \begin{macro}{\HoLogo@(La)TeX}
%    \begin{macrocode}
\expandafter
\let\csname HoLogo@(La)TeX\endcsname\HoLogo@LaTeXTeX
%    \end{macrocode}
%    \end{macro}
%    \begin{macro}{\HoLogoBkm@(La)TeX}
%    \begin{macrocode}
\expandafter
\let\csname HoLogoBkm@(La)TeX\endcsname\HoLogoBkm@LaTeXTeX
%    \end{macrocode}
%    \end{macro}
%    \begin{macro}{\HoLogoHtml@LaTeXTeX}
%    \begin{macrocode}
\def\HoLogoHtml@LaTeXTeX#1{%
  \HoLogoCss@LaTeXTeX
  \HOLOGO@Span{LaTeXTeX}{%
    (%
    \HOLOGO@Span{L}{L}%
    \HOLOGO@Span{a}{A}%
    \HOLOGO@Span{ParenRight}{)}%
    \hologo{TeX}%
  }%
}
%    \end{macrocode}
%    \end{macro}
%    \begin{macro}{\HoLogoHtml@(La)TeX}
%    Kerning after opening parentheses and before closing parentheses
%    is $-0.1$\,em. The original values $-0.15$\,em
%    looked too ugly for a serif font.
%    \begin{macrocode}
\expandafter
\let\csname HoLogoHtml@(La)TeX\endcsname\HoLogoHtml@LaTeXTeX
%    \end{macrocode}
%    \end{macro}
%    \begin{macro}{\HoLogoCss@LaTeXTeX}
%    \begin{macrocode}
\def\HoLogoCss@LaTeXTeX{%
  \Css{%
    span.HoLogo-LaTeXTeX span.HoLogo-L{%
      margin-left:-.1em;%
    }%
  }%
  \Css{%
    span.HoLogo-LaTeXTeX span.HoLogo-a{%
      position:relative;%
      top:-.5ex;%
      margin-left:-.36em;%
      margin-right:-.1em;%
      font-size:85\%;%
    }%
  }%
  \Css{%
    span.HoLogo-LaTeXTeX span.HoLogo-ParenRight{%
      margin-right:-.15em;%
    }%
  }%
  \global\let\HoLogoCss@LaTeXTeX\relax
}
%    \end{macrocode}
%    \end{macro}
%
% \subsubsection{\hologo{LaTeXe}}
%
%    \begin{macro}{\HoLogo@LaTeXe}
%    Source: \hologo{LaTeX} kernel
%    \begin{macrocode}
\def\HoLogo@LaTeXe#1{%
  \hologo{LaTeX}%
  \kern.15em%
  \hbox{%
    \HOLOGO@MathSetup
    2%
    $_{\textstyle\varepsilon}$%
  }%
}
%    \end{macrocode}
%    \end{macro}
%
%    \begin{macro}{\HoLogoCs@LaTeXe}
%    \begin{macrocode}
\ifnum64=`\^^^^0040\relax % test for big chars of LuaTeX/XeTeX
  \catcode`\$=9 %
  \catcode`\&=14 %
\else
  \catcode`\$=14 %
  \catcode`\&=9 %
\fi
\def\HoLogoCs@LaTeXe#1{%
  LaTeX2%
$ \string ^^^^0395%
& e%
}%
\catcode`\$=3 %
\catcode`\&=4 %
%    \end{macrocode}
%    \end{macro}
%
%    \begin{macro}{\HoLogoBkm@LaTeXe}
%    \begin{macrocode}
\def\HoLogoBkm@LaTeXe#1{%
  \hologo{LaTeX}%
  2%
  \HOLOGO@PdfdocUnicode{e}{\textepsilon}%
}
%    \end{macrocode}
%    \end{macro}
%
%    \begin{macro}{\HoLogoHtml@LaTeXe}
%    \begin{macrocode}
\def\HoLogoHtml@LaTeXe#1{%
  \HoLogoCss@LaTeXe
  \HOLOGO@Span{LaTeX2e}{%
    \hologo{LaTeX}%
    \HOLOGO@Span{2}{2}%
    \HOLOGO@Span{e}{%
      \HOLOGO@MathSetup
      \ensuremath{\textstyle\varepsilon}%
    }%
  }%
}
%    \end{macrocode}
%    \end{macro}
%    \begin{macro}{\HoLogoCss@LaTeXe}
%    \begin{macrocode}
\def\HoLogoCss@LaTeXe{%
  \Css{%
    span.HoLogo-LaTeX2e span.HoLogo-2{%
      padding-left:.15em;%
    }%
  }%
  \Css{%
    span.HoLogo-LaTeX2e span.HoLogo-e{%
      position:relative;%
      top:.35ex;%
      text-decoration:none;%
    }%
  }%
  \global\let\HoLogoCss@LaTeXe\relax
}
%    \end{macrocode}
%    \end{macro}
%
%    \begin{macro}{\HoLogo@LaTeX2e}
%    \begin{macrocode}
\expandafter
\let\csname HoLogo@LaTeX2e\endcsname\HoLogo@LaTeXe
%    \end{macrocode}
%    \end{macro}
%    \begin{macro}{\HoLogoCs@LaTeX2e}
%    \begin{macrocode}
\expandafter
\let\csname HoLogoCs@LaTeX2e\endcsname\HoLogoCs@LaTeXe
%    \end{macrocode}
%    \end{macro}
%    \begin{macro}{\HoLogoBkm@LaTeX2e}
%    \begin{macrocode}
\expandafter
\let\csname HoLogoBkm@LaTeX2e\endcsname\HoLogoBkm@LaTeXe
%    \end{macrocode}
%    \end{macro}
%    \begin{macro}{\HoLogoHtml@LaTeX2e}
%    \begin{macrocode}
\expandafter
\let\csname HoLogoHtml@LaTeX2e\endcsname\HoLogoHtml@LaTeXe
%    \end{macrocode}
%    \end{macro}
%
% \subsubsection{\hologo{LaTeX3}}
%
%    \begin{macro}{\HoLogo@LaTeX3}
%    Source: \hologo{LaTeX} kernel
%    \begin{macrocode}
\expandafter\def\csname HoLogo@LaTeX3\endcsname#1{%
  \hologo{LaTeX}%
  3%
}
%    \end{macrocode}
%    \end{macro}
%
%    \begin{macro}{\HoLogoBkm@LaTeX3}
%    \begin{macrocode}
\expandafter\def\csname HoLogoBkm@LaTeX3\endcsname#1{%
  \hologo{LaTeX}%
  3%
}
%    \end{macrocode}
%    \end{macro}
%    \begin{macro}{\HoLogoHtml@LaTeX3}
%    \begin{macrocode}
\expandafter
\let\csname HoLogoHtml@LaTeX3\expandafter\endcsname
\csname HoLogo@LaTeX3\endcsname
%    \end{macrocode}
%    \end{macro}
%
% \subsubsection{\hologo{LaTeXML}}
%
%    \begin{macro}{\HoLogo@LaTeXML}
%    \begin{macrocode}
\def\HoLogo@LaTeXML#1{%
  \HOLOGO@mbox{%
    \hologo{La}%
    \kern-.15em%
    T%
    \kern-.1667em%
    \lower.5ex\hbox{E}%
    \kern-.125em%
    \HoLogoFont@font{LaTeXML}{sc}{xml}%
  }%
}
%    \end{macrocode}
%    \end{macro}
%    \begin{macro}{\HoLogoHtml@pdfLaTeX}
%    \begin{macrocode}
\def\HoLogoHtml@LaTeXML#1{%
  \HOLOGO@Span{LaTeXML}{%
    \HoLogoCss@LaTeX
    \HoLogoCss@TeX
    \HOLOGO@Span{LaTeX}{%
      L%
      \HOLOGO@Span{a}{%
        A%
      }%
    }%
    \HOLOGO@Span{TeX}{%
      T%
      \HOLOGO@Span{e}{%
        E%
      }%
    }%
    \HCode{<span style="font-variant: small-caps;">}%
    xml%
    \HCode{</span>}%
  }%
}
%    \end{macrocode}
%    \end{macro}
%
% \subsubsection{\hologo{eTeX}}
%
%    \begin{macro}{\HoLogo@eTeX}
%    Source: package \xpackage{etex}
%    \begin{macrocode}
\def\HoLogo@eTeX#1{%
  \ltx@mbox{%
    \HOLOGO@MathSetup
    $\varepsilon$%
    -%
    \HOLOGO@NegativeKerning{-T,T-,To}%
    \hologo{TeX}%
  }%
}
%    \end{macrocode}
%    \end{macro}
%    \begin{macro}{\HoLogoCs@eTeX}
%    \begin{macrocode}
\ifnum64=`\^^^^0040\relax % test for big chars of LuaTeX/XeTeX
  \catcode`\$=9 %
  \catcode`\&=14 %
\else
  \catcode`\$=14 %
  \catcode`\&=9 %
\fi
\def\HoLogoCs@eTeX#1{%
$ #1{\string ^^^^0395}{\string ^^^^03b5}%
& #1{e}{E}%
  TeX%
}%
\catcode`\$=3 %
\catcode`\&=4 %
%    \end{macrocode}
%    \end{macro}
%    \begin{macro}{\HoLogoBkm@eTeX}
%    \begin{macrocode}
\def\HoLogoBkm@eTeX#1{%
  \HOLOGO@PdfdocUnicode{#1{e}{E}}{\textepsilon}%
  -%
  \hologo{TeX}%
}
%    \end{macrocode}
%    \end{macro}
%    \begin{macro}{\HoLogoHtml@eTeX}
%    \begin{macrocode}
\def\HoLogoHtml@eTeX#1{%
  \ltx@mbox{%
    \HOLOGO@MathSetup
    $\varepsilon$%
    -%
    \hologo{TeX}%
  }%
}
%    \end{macrocode}
%    \end{macro}
%
% \subsubsection{\hologo{iniTeX}}
%
%    \begin{macro}{\HoLogo@iniTeX}
%    \begin{macrocode}
\def\HoLogo@iniTeX#1{%
  \HOLOGO@mbox{%
    #1{i}{I}ni\hologo{TeX}%
  }%
}
%    \end{macrocode}
%    \end{macro}
%    \begin{macro}{\HoLogoCs@iniTeX}
%    \begin{macrocode}
\def\HoLogoCs@iniTeX#1{#1{i}{I}niTeX}
%    \end{macrocode}
%    \end{macro}
%    \begin{macro}{\HoLogoBkm@iniTeX}
%    \begin{macrocode}
\def\HoLogoBkm@iniTeX#1{%
  #1{i}{I}ni\hologo{TeX}%
}
%    \end{macrocode}
%    \end{macro}
%    \begin{macro}{\HoLogoHtml@iniTeX}
%    \begin{macrocode}
\let\HoLogoHtml@iniTeX\HoLogo@iniTeX
%    \end{macrocode}
%    \end{macro}
%
% \subsubsection{\hologo{virTeX}}
%
%    \begin{macro}{\HoLogo@virTeX}
%    \begin{macrocode}
\def\HoLogo@virTeX#1{%
  \HOLOGO@mbox{%
    #1{v}{V}ir\hologo{TeX}%
  }%
}
%    \end{macrocode}
%    \end{macro}
%    \begin{macro}{\HoLogoCs@virTeX}
%    \begin{macrocode}
\def\HoLogoCs@virTeX#1{#1{v}{V}irTeX}
%    \end{macrocode}
%    \end{macro}
%    \begin{macro}{\HoLogoBkm@virTeX}
%    \begin{macrocode}
\def\HoLogoBkm@virTeX#1{%
  #1{v}{V}ir\hologo{TeX}%
}
%    \end{macrocode}
%    \end{macro}
%    \begin{macro}{\HoLogoHtml@virTeX}
%    \begin{macrocode}
\let\HoLogoHtml@virTeX\HoLogo@virTeX
%    \end{macrocode}
%    \end{macro}
%
% \subsubsection{\hologo{SliTeX}}
%
% \paragraph{Definitions of the three variants.}
%
%    \begin{macro}{\HoLogo@SLiTeX@lift}
%    \begin{macrocode}
\def\HoLogo@SLiTeX@lift#1{%
  \HoLogoFont@font{SliTeX}{rm}{%
    S%
    \kern-.06em%
    L%
    \kern-.18em%
    \raise.32ex\hbox{\HoLogoFont@font{SliTeX}{sc}{i}}%
    \HOLOGO@discretionary
    \kern-.06em%
    \hologo{TeX}%
  }%
}
%    \end{macrocode}
%    \end{macro}
%    \begin{macro}{\HoLogoBkm@SLiTeX@lift}
%    \begin{macrocode}
\def\HoLogoBkm@SLiTeX@lift#1{SLiTeX}
%    \end{macrocode}
%    \end{macro}
%    \begin{macro}{\HoLogoHtml@SLiTeX@lift}
%    \begin{macrocode}
\def\HoLogoHtml@SLiTeX@lift#1{%
  \HoLogoCss@SLiTeX@lift
  \HOLOGO@Span{SLiTeX-lift}{%
    \HoLogoFont@font{SliTeX}{rm}{%
      S%
      \HOLOGO@Span{L}{L}%
      \HOLOGO@Span{i}{i}%
      \hologo{TeX}%
    }%
  }%
}
%    \end{macrocode}
%    \end{macro}
%    \begin{macro}{\HoLogoCss@SLiTeX@lift}
%    \begin{macrocode}
\def\HoLogoCss@SLiTeX@lift{%
  \Css{%
    span.HoLogo-SLiTeX-lift span.HoLogo-L{%
      margin-left:-.06em;%
      margin-right:-.18em;%
    }%
  }%
  \Css{%
    span.HoLogo-SLiTeX-lift span.HoLogo-i{%
      position:relative;%
      top:-.32ex;%
      margin-right:-.06em;%
      font-variant:small-caps;%
    }%
  }%
  \global\let\HoLogoCss@SLiTeX@lift\relax
}
%    \end{macrocode}
%    \end{macro}
%
%    \begin{macro}{\HoLogo@SliTeX@simple}
%    \begin{macrocode}
\def\HoLogo@SliTeX@simple#1{%
  \HoLogoFont@font{SliTeX}{rm}{%
    \ltx@mbox{%
      \HoLogoFont@font{SliTeX}{sc}{Sli}%
    }%
    \HOLOGO@discretionary
    \hologo{TeX}%
  }%
}
%    \end{macrocode}
%    \end{macro}
%    \begin{macro}{\HoLogoBkm@SliTeX@simple}
%    \begin{macrocode}
\def\HoLogoBkm@SliTeX@simple#1{SliTeX}
%    \end{macrocode}
%    \end{macro}
%    \begin{macro}{\HoLogoHtml@SliTeX@simple}
%    \begin{macrocode}
\let\HoLogoHtml@SliTeX@simple\HoLogo@SliTeX@simple
%    \end{macrocode}
%    \end{macro}
%
%    \begin{macro}{\HoLogo@SliTeX@narrow}
%    \begin{macrocode}
\def\HoLogo@SliTeX@narrow#1{%
  \HoLogoFont@font{SliTeX}{rm}{%
    \ltx@mbox{%
      S%
      \kern-.06em%
      \HoLogoFont@font{SliTeX}{sc}{%
        l%
        \kern-.035em%
        i%
      }%
    }%
    \HOLOGO@discretionary
    \kern-.06em%
    \hologo{TeX}%
  }%
}
%    \end{macrocode}
%    \end{macro}
%    \begin{macro}{\HoLogoBkm@SliTeX@narrow}
%    \begin{macrocode}
\def\HoLogoBkm@SliTeX@narrow#1{SliTeX}
%    \end{macrocode}
%    \end{macro}
%    \begin{macro}{\HoLogoHtml@SliTeX@narrow}
%    \begin{macrocode}
\def\HoLogoHtml@SliTeX@narrow#1{%
  \HoLogoCss@SliTeX@narrow
  \HOLOGO@Span{SliTeX-narrow}{%
    \HoLogoFont@font{SliTeX}{rm}{%
      S%
        \HOLOGO@Span{l}{l}%
        \HOLOGO@Span{i}{i}%
      \hologo{TeX}%
    }%
  }%
}
%    \end{macrocode}
%    \end{macro}
%    \begin{macro}{\HoLogoCss@SliTeX@narrow}
%    \begin{macrocode}
\def\HoLogoCss@SliTeX@narrow{%
  \Css{%
    span.HoLogo-SliTeX-narrow span.HoLogo-l{%
      margin-left:-.06em;%
      margin-right:-.035em;%
      font-variant:small-caps;%
    }%
  }%
  \Css{%
    span.HoLogo-SliTeX-narrow span.HoLogo-i{%
      margin-right:-.06em;%
      font-variant:small-caps;%
    }%
  }%
  \global\let\HoLogoCss@SliTeX@narrow\relax
}
%    \end{macrocode}
%    \end{macro}
%
% \paragraph{Macro set completion.}
%
%    \begin{macro}{\HoLogo@SLiTeX@simple}
%    \begin{macrocode}
\def\HoLogo@SLiTeX@simple{\HoLogo@SliTeX@simple}
%    \end{macrocode}
%    \end{macro}
%    \begin{macro}{\HoLogoBkm@SLiTeX@simple}
%    \begin{macrocode}
\def\HoLogoBkm@SLiTeX@simple{\HoLogoBkm@SliTeX@simple}
%    \end{macrocode}
%    \end{macro}
%    \begin{macro}{\HoLogoHtml@SLiTeX@simple}
%    \begin{macrocode}
\def\HoLogoHtml@SLiTeX@simple{\HoLogoHtml@SliTeX@simple}
%    \end{macrocode}
%    \end{macro}
%
%    \begin{macro}{\HoLogo@SLiTeX@narrow}
%    \begin{macrocode}
\def\HoLogo@SLiTeX@narrow{\HoLogo@SliTeX@narrow}
%    \end{macrocode}
%    \end{macro}
%    \begin{macro}{\HoLogoBkm@SLiTeX@narrow}
%    \begin{macrocode}
\def\HoLogoBkm@SLiTeX@narrow{\HoLogoBkm@SliTeX@narrow}
%    \end{macrocode}
%    \end{macro}
%    \begin{macro}{\HoLogoHtml@SLiTeX@narrow}
%    \begin{macrocode}
\def\HoLogoHtml@SLiTeX@narrow{\HoLogoHtml@SliTeX@narrow}
%    \end{macrocode}
%    \end{macro}
%
%    \begin{macro}{\HoLogo@SliTeX@lift}
%    \begin{macrocode}
\def\HoLogo@SliTeX@lift{\HoLogo@SLiTeX@lift}
%    \end{macrocode}
%    \end{macro}
%    \begin{macro}{\HoLogoBkm@SliTeX@lift}
%    \begin{macrocode}
\def\HoLogoBkm@SliTeX@lift{\HoLogoBkm@SLiTeX@lift}
%    \end{macrocode}
%    \end{macro}
%    \begin{macro}{\HoLogoHtml@SliTeX@lift}
%    \begin{macrocode}
\def\HoLogoHtml@SliTeX@lift{\HoLogoHtml@SLiTeX@lift}
%    \end{macrocode}
%    \end{macro}
%
% \paragraph{Defaults.}
%
%    \begin{macro}{\HoLogo@SLiTeX}
%    \begin{macrocode}
\def\HoLogo@SLiTeX{\HoLogo@SLiTeX@lift}
%    \end{macrocode}
%    \end{macro}
%    \begin{macro}{\HoLogoBkm@SLiTeX}
%    \begin{macrocode}
\def\HoLogoBkm@SLiTeX{\HoLogoBkm@SLiTeX@lift}
%    \end{macrocode}
%    \end{macro}
%    \begin{macro}{\HoLogoHtml@SLiTeX}
%    \begin{macrocode}
\def\HoLogoHtml@SLiTeX{\HoLogoHtml@SLiTeX@lift}
%    \end{macrocode}
%    \end{macro}
%
%    \begin{macro}{\HoLogo@SliTeX}
%    \begin{macrocode}
\def\HoLogo@SliTeX{\HoLogo@SliTeX@narrow}
%    \end{macrocode}
%    \end{macro}
%    \begin{macro}{\HoLogoBkm@SliTeX}
%    \begin{macrocode}
\def\HoLogoBkm@SliTeX{\HoLogoBkm@SliTeX@narrow}
%    \end{macrocode}
%    \end{macro}
%    \begin{macro}{\HoLogoHtml@SliTeX}
%    \begin{macrocode}
\def\HoLogoHtml@SliTeX{\HoLogoHtml@SliTeX@narrow}
%    \end{macrocode}
%    \end{macro}
%
% \subsubsection{\hologo{LuaTeX}}
%
%    \begin{macro}{\HoLogo@LuaTeX}
%    The kerning is an idea of Hans Hagen, see mailing list
%    `luatex at tug dot org' in March 2010.
%    \begin{macrocode}
\def\HoLogo@LuaTeX#1{%
  \HOLOGO@mbox{%
    Lua%
    \HOLOGO@NegativeKerning{aT,oT,To}%
    \hologo{TeX}%
  }%
}
%    \end{macrocode}
%    \end{macro}
%    \begin{macro}{\HoLogoHtml@LuaTeX}
%    \begin{macrocode}
\let\HoLogoHtml@LuaTeX\HoLogo@LuaTeX
%    \end{macrocode}
%    \end{macro}
%
% \subsubsection{\hologo{LuaLaTeX}}
%
%    \begin{macro}{\HoLogo@LuaLaTeX}
%    \begin{macrocode}
\def\HoLogo@LuaLaTeX#1{%
  \HOLOGO@mbox{%
    Lua%
    \hologo{LaTeX}%
  }%
}
%    \end{macrocode}
%    \end{macro}
%    \begin{macro}{\HoLogoHtml@LuaLaTeX}
%    \begin{macrocode}
\let\HoLogoHtml@LuaLaTeX\HoLogo@LuaLaTeX
%    \end{macrocode}
%    \end{macro}
%
% \subsubsection{\hologo{XeTeX}, \hologo{XeLaTeX}}
%
%    \begin{macro}{\HOLOGO@IfCharExists}
%    \begin{macrocode}
\ifluatex
  \ifnum\luatexversion<36 %
  \else
    \def\HOLOGO@IfCharExists#1{%
      \ifnum
        \directlua{%
           if luaotfload and luaotfload.aux then
             if luaotfload.aux.font_has_glyph(%
                    font.current(), \number#1) then % 	 
	       tex.print("1") % 	 
	     end % 	 
	   elseif font and font.fonts and font.current then %
            local f = font.fonts[font.current()]%
            if f.characters and f.characters[\number#1] then %
              tex.print("1")%
            end %
          end%
        }0=\ltx@zero
        \expandafter\ltx@secondoftwo
      \else
        \expandafter\ltx@firstoftwo
      \fi
    }%
  \fi
\fi
\ltx@IfUndefined{HOLOGO@IfCharExists}{%
  \def\HOLOGO@@IfCharExists#1{%
    \begingroup
      \tracinglostchars=\ltx@zero
      \setbox\ltx@zero=\hbox{%
        \kern7sp\char#1\relax
        \ifnum\lastkern>\ltx@zero
          \expandafter\aftergroup\csname iffalse\endcsname
        \else
          \expandafter\aftergroup\csname iftrue\endcsname
        \fi
      }%
      % \if{true|false} from \aftergroup
      \endgroup
      \expandafter\ltx@firstoftwo
    \else
      \endgroup
      \expandafter\ltx@secondoftwo
    \fi
  }%
  \ifxetex
    \ltx@IfUndefined{XeTeXfonttype}{}{%
      \ltx@IfUndefined{XeTeXcharglyph}{}{%
        \def\HOLOGO@IfCharExists#1{%
          \ifnum\XeTeXfonttype\font>\ltx@zero
            \expandafter\ltx@firstofthree
          \else
            \expandafter\ltx@gobble
          \fi
          {%
            \ifnum\XeTeXcharglyph#1>\ltx@zero
              \expandafter\ltx@firstoftwo
            \else
              \expandafter\ltx@secondoftwo
            \fi
          }%
          \HOLOGO@@IfCharExists{#1}%
        }%
      }%
    }%
  \fi
}{}
\ltx@ifundefined{HOLOGO@IfCharExists}{%
  \ifnum64=`\^^^^0040\relax % test for big chars of LuaTeX/XeTeX
    \let\HOLOGO@IfCharExists\HOLOGO@@IfCharExists
  \else
    \def\HOLOGO@IfCharExists#1{%
      \ifnum#1>255 %
        \expandafter\ltx@fourthoffour
      \fi
      \HOLOGO@@IfCharExists{#1}%
    }%
  \fi
}{}
%    \end{macrocode}
%    \end{macro}
%
%    \begin{macro}{\HoLogo@Xe}
%    Source: package \xpackage{dtklogos}
%    \begin{macrocode}
\def\HoLogo@Xe#1{%
  X%
  \kern-.1em\relax
  \HOLOGO@IfCharExists{"018E}{%
    \lower.5ex\hbox{\char"018E}%
  }{%
    \chardef\HOLOGO@choice=\ltx@zero
    \ifdim\fontdimen\ltx@one\font>0pt %
      \ltx@IfUndefined{rotatebox}{%
        \ltx@IfUndefined{pgftext}{%
          \ltx@IfUndefined{psscalebox}{%
            \ltx@IfUndefined{HOLOGO@ScaleBox@\hologoDriver}{%
            }{%
              \chardef\HOLOGO@choice=4 %
            }%
          }{%
            \chardef\HOLOGO@choice=3 %
          }%
        }{%
          \chardef\HOLOGO@choice=2 %
        }%
      }{%
        \chardef\HOLOGO@choice=1 %
      }%
      \ifcase\HOLOGO@choice
        \HOLOGO@WarningUnsupportedDriver{Xe}%
        e%
      \or % 1: \rotatebox
        \begingroup
          \setbox\ltx@zero\hbox{\rotatebox{180}{E}}%
          \ltx@LocDimenA=\dp\ltx@zero
          \advance\ltx@LocDimenA by -.5ex\relax
          \raise\ltx@LocDimenA\box\ltx@zero
        \endgroup
      \or % 2: \pgftext
        \lower.5ex\hbox{%
          \pgfpicture
            \pgftext[rotate=180]{E}%
          \endpgfpicture
        }%
      \or % 3: \psscalebox
        \begingroup
          \setbox\ltx@zero\hbox{\psscalebox{-1 -1}{E}}%
          \ltx@LocDimenA=\dp\ltx@zero
          \advance\ltx@LocDimenA by -.5ex\relax
          \raise\ltx@LocDimenA\box\ltx@zero
        \endgroup
      \or % 4: \HOLOGO@PointReflectBox
        \lower.5ex\hbox{\HOLOGO@PointReflectBox{E}}%
      \else
        \@PackageError{hologo}{Internal error (choice/it}\@ehc
      \fi
    \else
      \ltx@IfUndefined{reflectbox}{%
        \ltx@IfUndefined{pgftext}{%
          \ltx@IfUndefined{psscalebox}{%
            \ltx@IfUndefined{HOLOGO@ScaleBox@\hologoDriver}{%
            }{%
              \chardef\HOLOGO@choice=4 %
            }%
          }{%
            \chardef\HOLOGO@choice=3 %
          }%
        }{%
          \chardef\HOLOGO@choice=2 %
        }%
      }{%
        \chardef\HOLOGO@choice=1 %
      }%
      \ifcase\HOLOGO@choice
        \HOLOGO@WarningUnsupportedDriver{Xe}%
        e%
      \or % 1: reflectbox
        \lower.5ex\hbox{%
          \reflectbox{E}%
        }%
      \or % 2: \pgftext
        \lower.5ex\hbox{%
          \pgfpicture
            \pgftransformxscale{-1}%
            \pgftext{E}%
          \endpgfpicture
        }%
      \or % 3: \psscalebox
        \lower.5ex\hbox{%
          \psscalebox{-1 1}{E}%
        }%
      \or % 4: \HOLOGO@Reflectbox
        \lower.5ex\hbox{%
          \HOLOGO@ReflectBox{E}%
        }%
      \else
        \@PackageError{hologo}{Internal error (choice/up)}\@ehc
      \fi
    \fi
  }%
}
%    \end{macrocode}
%    \end{macro}
%    \begin{macro}{\HoLogoHtml@Xe}
%    \begin{macrocode}
\def\HoLogoHtml@Xe#1{%
  \HoLogoCss@Xe
  \HOLOGO@Span{Xe}{%
    X%
    \HOLOGO@Span{e}{%
      \HCode{&\ltx@hashchar x018e;}%
    }%
  }%
}
%    \end{macrocode}
%    \end{macro}
%    \begin{macro}{\HoLogoCss@Xe}
%    \begin{macrocode}
\def\HoLogoCss@Xe{%
  \Css{%
    span.HoLogo-Xe span.HoLogo-e{%
      position:relative;%
      top:.5ex;%
      left-margin:-.1em;%
    }%
  }%
  \global\let\HoLogoCss@Xe\relax
}
%    \end{macrocode}
%    \end{macro}
%
%    \begin{macro}{\HoLogo@XeTeX}
%    \begin{macrocode}
\def\HoLogo@XeTeX#1{%
  \hologo{Xe}%
  \kern-.15em\relax
  \hologo{TeX}%
}
%    \end{macrocode}
%    \end{macro}
%
%    \begin{macro}{\HoLogoHtml@XeTeX}
%    \begin{macrocode}
\def\HoLogoHtml@XeTeX#1{%
  \HoLogoCss@XeTeX
  \HOLOGO@Span{XeTeX}{%
    \hologo{Xe}%
    \hologo{TeX}%
  }%
}
%    \end{macrocode}
%    \end{macro}
%    \begin{macro}{\HoLogoCss@XeTeX}
%    \begin{macrocode}
\def\HoLogoCss@XeTeX{%
  \Css{%
    span.HoLogo-XeTeX span.HoLogo-TeX{%
      margin-left:-.15em;%
    }%
  }%
  \global\let\HoLogoCss@XeTeX\relax
}
%    \end{macrocode}
%    \end{macro}
%
%    \begin{macro}{\HoLogo@XeLaTeX}
%    \begin{macrocode}
\def\HoLogo@XeLaTeX#1{%
  \hologo{Xe}%
  \kern-.13em%
  \hologo{LaTeX}%
}
%    \end{macrocode}
%    \end{macro}
%    \begin{macro}{\HoLogoHtml@XeLaTeX}
%    \begin{macrocode}
\def\HoLogoHtml@XeLaTeX#1{%
  \HoLogoCss@XeLaTeX
  \HOLOGO@Span{XeLaTeX}{%
    \hologo{Xe}%
    \hologo{LaTeX}%
  }%
}
%    \end{macrocode}
%    \end{macro}
%    \begin{macro}{\HoLogoCss@XeLaTeX}
%    \begin{macrocode}
\def\HoLogoCss@XeLaTeX{%
  \Css{%
    span.HoLogo-XeLaTeX span.HoLogo-Xe{%
      margin-right:-.13em;%
    }%
  }%
  \global\let\HoLogoCss@XeLaTeX\relax
}
%    \end{macrocode}
%    \end{macro}
%
% \subsubsection{\hologo{pdfTeX}, \hologo{pdfLaTeX}}
%
%    \begin{macro}{\HoLogo@pdfTeX}
%    \begin{macrocode}
\def\HoLogo@pdfTeX#1{%
  \HOLOGO@mbox{%
    #1{p}{P}df\hologo{TeX}%
  }%
}
%    \end{macrocode}
%    \end{macro}
%    \begin{macro}{\HoLogoCs@pdfTeX}
%    \begin{macrocode}
\def\HoLogoCs@pdfTeX#1{#1{p}{P}dfTeX}
%    \end{macrocode}
%    \end{macro}
%    \begin{macro}{\HoLogoBkm@pdfTeX}
%    \begin{macrocode}
\def\HoLogoBkm@pdfTeX#1{%
  #1{p}{P}df\hologo{TeX}%
}
%    \end{macrocode}
%    \end{macro}
%    \begin{macro}{\HoLogoHtml@pdfTeX}
%    \begin{macrocode}
\let\HoLogoHtml@pdfTeX\HoLogo@pdfTeX
%    \end{macrocode}
%    \end{macro}
%
%    \begin{macro}{\HoLogo@pdfLaTeX}
%    \begin{macrocode}
\def\HoLogo@pdfLaTeX#1{%
  \HOLOGO@mbox{%
    #1{p}{P}df\hologo{LaTeX}%
  }%
}
%    \end{macrocode}
%    \end{macro}
%    \begin{macro}{\HoLogoCs@pdfLaTeX}
%    \begin{macrocode}
\def\HoLogoCs@pdfLaTeX#1{#1{p}{P}dfLaTeX}
%    \end{macrocode}
%    \end{macro}
%    \begin{macro}{\HoLogoBkm@pdfLaTeX}
%    \begin{macrocode}
\def\HoLogoBkm@pdfLaTeX#1{%
  #1{p}{P}df\hologo{LaTeX}%
}
%    \end{macrocode}
%    \end{macro}
%    \begin{macro}{\HoLogoHtml@pdfLaTeX}
%    \begin{macrocode}
\let\HoLogoHtml@pdfLaTeX\HoLogo@pdfLaTeX
%    \end{macrocode}
%    \end{macro}
%
% \subsubsection{\hologo{VTeX}}
%
%    \begin{macro}{\HoLogo@VTeX}
%    \begin{macrocode}
\def\HoLogo@VTeX#1{%
  \HOLOGO@mbox{%
    V\hologo{TeX}%
  }%
}
%    \end{macrocode}
%    \end{macro}
%    \begin{macro}{\HoLogoHtml@VTeX}
%    \begin{macrocode}
\let\HoLogoHtml@VTeX\HoLogo@VTeX
%    \end{macrocode}
%    \end{macro}
%
% \subsubsection{\hologo{AmS}, \dots}
%
%    Source: class \xclass{amsdtx}
%
%    \begin{macro}{\HoLogo@AmS}
%    \begin{macrocode}
\def\HoLogo@AmS#1{%
  \HoLogoFont@font{AmS}{sy}{%
    A%
    \kern-.1667em%
    \lower.5ex\hbox{M}%
    \kern-.125em%
    S%
  }%
}
%    \end{macrocode}
%    \end{macro}
%    \begin{macro}{\HoLogoBkm@AmS}
%    \begin{macrocode}
\def\HoLogoBkm@AmS#1{AmS}
%    \end{macrocode}
%    \end{macro}
%    \begin{macro}{\HoLogoHtml@AmS}
%    \begin{macrocode}
\def\HoLogoHtml@AmS#1{%
  \HoLogoCss@AmS
%  \HoLogoFont@font{AmS}{sy}{%
    \HOLOGO@Span{AmS}{%
      A%
      \HOLOGO@Span{M}{M}%
      S%
    }%
%   }%
}
%    \end{macrocode}
%    \end{macro}
%    \begin{macro}{\HoLogoCss@AmS}
%    \begin{macrocode}
\def\HoLogoCss@AmS{%
  \Css{%
    span.HoLogo-AmS span.HoLogo-M{%
      position:relative;%
      top:.5ex;%
      margin-left:-.1667em;%
      margin-right:-.125em;%
      text-decoration:none;%
    }%
  }%
  \global\let\HoLogoCss@AmS\relax
}
%    \end{macrocode}
%    \end{macro}
%
%    \begin{macro}{\HoLogo@AmSTeX}
%    \begin{macrocode}
\def\HoLogo@AmSTeX#1{%
  \hologo{AmS}%
  \HOLOGO@hyphen
  \hologo{TeX}%
}
%    \end{macrocode}
%    \end{macro}
%    \begin{macro}{\HoLogoBkm@AmSTeX}
%    \begin{macrocode}
\def\HoLogoBkm@AmSTeX#1{AmS-TeX}%
%    \end{macrocode}
%    \end{macro}
%    \begin{macro}{\HoLogoHtml@AmSTeX}
%    \begin{macrocode}
\let\HoLogoHtml@AmSTeX\HoLogo@AmSTeX
%    \end{macrocode}
%    \end{macro}
%
%    \begin{macro}{\HoLogo@AmSLaTeX}
%    \begin{macrocode}
\def\HoLogo@AmSLaTeX#1{%
  \hologo{AmS}%
  \HOLOGO@hyphen
  \hologo{LaTeX}%
}
%    \end{macrocode}
%    \end{macro}
%    \begin{macro}{\HoLogoBkm@AmSLaTeX}
%    \begin{macrocode}
\def\HoLogoBkm@AmSLaTeX#1{AmS-LaTeX}%
%    \end{macrocode}
%    \end{macro}
%    \begin{macro}{\HoLogoHtml@AmSLaTeX}
%    \begin{macrocode}
\let\HoLogoHtml@AmSLaTeX\HoLogo@AmSLaTeX
%    \end{macrocode}
%    \end{macro}
%
% \subsubsection{\hologo{BibTeX}}
%
%    \begin{macro}{\HoLogo@BibTeX@sc}
%    A definition of \hologo{BibTeX} is provided in
%    the documentation source for the manual of \hologo{BibTeX}
%    \cite{btxdoc}.
%\begin{quote}
%\begin{verbatim}
%\def\BibTeX{%
%  {%
%    \rm
%    B%
%    \kern-.05em%
%    {%
%      \sc
%      i%
%      \kern-.025em %
%      b%
%    }%
%    \kern-.08em
%    T%
%    \kern-.1667em%
%    \lower.7ex\hbox{E}%
%    \kern-.125em%
%    X%
%  }%
%}
%\end{verbatim}
%\end{quote}
%    \begin{macrocode}
\def\HoLogo@BibTeX@sc#1{%
  B%
  \kern-.05em%
  \HoLogoFont@font{BibTeX}{sc}{%
    i%
    \kern-.025em%
    b%
  }%
  \HOLOGO@discretionary
  \kern-.08em%
  \hologo{TeX}%
}
%    \end{macrocode}
%    \end{macro}
%    \begin{macro}{\HoLogoHtml@BibTeX@sc}
%    \begin{macrocode}
\def\HoLogoHtml@BibTeX@sc#1{%
  \HoLogoCss@BibTeX@sc
  \HOLOGO@Span{BibTeX-sc}{%
    B%
    \HOLOGO@Span{i}{i}%
    \HOLOGO@Span{b}{b}%
    \hologo{TeX}%
  }%
}
%    \end{macrocode}
%    \end{macro}
%    \begin{macro}{\HoLogoCss@BibTeX@sc}
%    \begin{macrocode}
\def\HoLogoCss@BibTeX@sc{%
  \Css{%
    span.HoLogo-BibTeX-sc span.HoLogo-i{%
      margin-left:-.05em;%
      margin-right:-.025em;%
      font-variant:small-caps;%
    }%
  }%
  \Css{%
    span.HoLogo-BibTeX-sc span.HoLogo-b{%
      margin-right:-.08em;%
      font-variant:small-caps;%
    }%
  }%
  \global\let\HoLogoCss@BibTeX@sc\relax
}
%    \end{macrocode}
%    \end{macro}
%
%    \begin{macro}{\HoLogo@BibTeX@sf}
%    Variant \xoption{sf} avoids trouble with unavailable
%    small caps fonts (e.g., bold versions of Computer Modern or
%    Latin Modern). The definition is taken from
%    package \xpackage{dtklogos} \cite{dtklogos}.
%\begin{quote}
%\begin{verbatim}
%\DeclareRobustCommand{\BibTeX}{%
%  B%
%  \kern-.05em%
%  \hbox{%
%    $\m@th$% %% force math size calculations
%    \csname S@\f@size\endcsname
%    \fontsize\sf@size\z@
%    \math@fontsfalse
%    \selectfont
%    I%
%    \kern-.025em%
%    B
%  }%
%  \kern-.08em%
%  \-%
%  \TeX
%}
%\end{verbatim}
%\end{quote}
%    \begin{macrocode}
\def\HoLogo@BibTeX@sf#1{%
  B%
  \kern-.05em%
  \HoLogoFont@font{BibTeX}{bibsf}{%
    I%
    \kern-.025em%
    B%
  }%
  \HOLOGO@discretionary
  \kern-.08em%
  \hologo{TeX}%
}
%    \end{macrocode}
%    \end{macro}
%    \begin{macro}{\HoLogoHtml@BibTeX@sf}
%    \begin{macrocode}
\def\HoLogoHtml@BibTeX@sf#1{%
  \HoLogoCss@BibTeX@sf
  \HOLOGO@Span{BibTeX-sf}{%
    B%
    \HoLogoFont@font{BibTeX}{bibsf}{%
      \HOLOGO@Span{i}{I}%
      B%
    }%
    \hologo{TeX}%
  }%
}
%    \end{macrocode}
%    \end{macro}
%    \begin{macro}{\HoLogoCss@BibTeX@sf}
%    \begin{macrocode}
\def\HoLogoCss@BibTeX@sf{%
  \Css{%
    span.HoLogo-BibTeX-sf span.HoLogo-i{%
      margin-left:-.05em;%
      margin-right:-.025em;%
    }%
  }%
  \Css{%
    span.HoLogo-BibTeX-sf span.HoLogo-TeX{%
      margin-left:-.08em;%
    }%
  }%
  \global\let\HoLogoCss@BibTeX@sf\relax
}
%    \end{macrocode}
%    \end{macro}
%
%    \begin{macro}{\HoLogo@BibTeX}
%    \begin{macrocode}
\def\HoLogo@BibTeX{\HoLogo@BibTeX@sf}
%    \end{macrocode}
%    \end{macro}
%    \begin{macro}{\HoLogoHtml@BibTeX}
%    \begin{macrocode}
\def\HoLogoHtml@BibTeX{\HoLogoHtml@BibTeX@sf}
%    \end{macrocode}
%    \end{macro}
%
% \subsubsection{\hologo{BibTeX8}}
%
%    \begin{macro}{\HoLogo@BibTeX8}
%    \begin{macrocode}
\expandafter\def\csname HoLogo@BibTeX8\endcsname#1{%
  \hologo{BibTeX}%
  8%
}
%    \end{macrocode}
%    \end{macro}
%
%    \begin{macro}{\HoLogoBkm@BibTeX8}
%    \begin{macrocode}
\expandafter\def\csname HoLogoBkm@BibTeX8\endcsname#1{%
  \hologo{BibTeX}%
  8%
}
%    \end{macrocode}
%    \end{macro}
%    \begin{macro}{\HoLogoHtml@BibTeX8}
%    \begin{macrocode}
\expandafter
\let\csname HoLogoHtml@BibTeX8\expandafter\endcsname
\csname HoLogo@BibTeX8\endcsname
%    \end{macrocode}
%    \end{macro}
%
% \subsubsection{\hologo{ConTeXt}}
%
%    \begin{macro}{\HoLogo@ConTeXt@simple}
%    \begin{macrocode}
\def\HoLogo@ConTeXt@simple#1{%
  \HOLOGO@mbox{Con}%
  \HOLOGO@discretionary
  \HOLOGO@mbox{\hologo{TeX}t}%
}
%    \end{macrocode}
%    \end{macro}
%    \begin{macro}{\HoLogoHtml@ConTeXt@simple}
%    \begin{macrocode}
\let\HoLogoHtml@ConTeXt@simple\HoLogo@ConTeXt@simple
%    \end{macrocode}
%    \end{macro}
%
%    \begin{macro}{\HoLogo@ConTeXt@narrow}
%    This definition of logo \hologo{ConTeXt} with variant \xoption{narrow}
%    comes from TUGboat's class \xclass{ltugboat} (version 2010/11/15 v2.8).
%    \begin{macrocode}
\def\HoLogo@ConTeXt@narrow#1{%
  \HOLOGO@mbox{C\kern-.0333emon}%
  \HOLOGO@discretionary
  \kern-.0667em%
  \HOLOGO@mbox{\hologo{TeX}\kern-.0333emt}%
}
%    \end{macrocode}
%    \end{macro}
%    \begin{macro}{\HoLogoHtml@ConTeXt@narrow}
%    \begin{macrocode}
\def\HoLogoHtml@ConTeXt@narrow#1{%
  \HoLogoCss@ConTeXt@narrow
  \HOLOGO@Span{ConTeXt-narrow}{%
    \HOLOGO@Span{C}{C}%
    on%
    \hologo{TeX}%
    t%
  }%
}
%    \end{macrocode}
%    \end{macro}
%    \begin{macro}{\HoLogoCss@ConTeXt@narrow}
%    \begin{macrocode}
\def\HoLogoCss@ConTeXt@narrow{%
  \Css{%
    span.HoLogo-ConTeXt-narrow span.HoLogo-C{%
      margin-left:-.0333em;%
    }%
  }%
  \Css{%
    span.HoLogo-ConTeXt-narrow span.HoLogo-TeX{%
      margin-left:-.0667em;%
      margin-right:-.0333em;%
    }%
  }%
  \global\let\HoLogoCss@ConTeXt@narrow\relax
}
%    \end{macrocode}
%    \end{macro}
%
%    \begin{macro}{\HoLogo@ConTeXt}
%    \begin{macrocode}
\def\HoLogo@ConTeXt{\HoLogo@ConTeXt@narrow}
%    \end{macrocode}
%    \end{macro}
%    \begin{macro}{\HoLogoHtml@ConTeXt}
%    \begin{macrocode}
\def\HoLogoHtml@ConTeXt{\HoLogoHtml@ConTeXt@narrow}
%    \end{macrocode}
%    \end{macro}
%
% \subsubsection{\hologo{emTeX}}
%
%    \begin{macro}{\HoLogo@emTeX}
%    \begin{macrocode}
\def\HoLogo@emTeX#1{%
  \HOLOGO@mbox{#1{e}{E}m}%
  \HOLOGO@discretionary
  \hologo{TeX}%
}
%    \end{macrocode}
%    \end{macro}
%    \begin{macro}{\HoLogoCs@emTeX}
%    \begin{macrocode}
\def\HoLogoCs@emTeX#1{#1{e}{E}mTeX}%
%    \end{macrocode}
%    \end{macro}
%    \begin{macro}{\HoLogoBkm@emTeX}
%    \begin{macrocode}
\def\HoLogoBkm@emTeX#1{%
  #1{e}{E}m\hologo{TeX}%
}
%    \end{macrocode}
%    \end{macro}
%    \begin{macro}{\HoLogoHtml@emTeX}
%    \begin{macrocode}
\let\HoLogoHtml@emTeX\HoLogo@emTeX
%    \end{macrocode}
%    \end{macro}
%
% \subsubsection{\hologo{ExTeX}}
%
%    \begin{macro}{\HoLogo@ExTeX}
%    The definition is taken from the FAQ of the
%    project \hologo{ExTeX}
%    \cite{ExTeX-FAQ}.
%\begin{quote}
%\begin{verbatim}
%\def\ExTeX{%
%  \textrm{% Logo always with serifs
%    \ensuremath{%
%      \textstyle
%      \varepsilon_{%
%        \kern-0.15em%
%        \mathcal{X}%
%      }%
%    }%
%    \kern-.15em%
%    \TeX
%  }%
%}
%\end{verbatim}
%\end{quote}
%    \begin{macrocode}
\def\HoLogo@ExTeX#1{%
  \HoLogoFont@font{ExTeX}{rm}{%
    \ltx@mbox{%
      \HOLOGO@MathSetup
      $%
        \textstyle
        \varepsilon_{%
          \kern-0.15em%
          \HoLogoFont@font{ExTeX}{sy}{X}%
        }%
      $%
    }%
    \HOLOGO@discretionary
    \kern-.15em%
    \hologo{TeX}%
  }%
}
%    \end{macrocode}
%    \end{macro}
%    \begin{macro}{\HoLogoHtml@ExTeX}
%    \begin{macrocode}
\def\HoLogoHtml@ExTeX#1{%
  \HoLogoCss@ExTeX
  \HoLogoFont@font{ExTeX}{rm}{%
    \HOLOGO@Span{ExTeX}{%
      \ltx@mbox{%
        \HOLOGO@MathSetup
        $\textstyle\varepsilon$%
        \HOLOGO@Span{X}{$\textstyle\chi$}%
        \hologo{TeX}%
      }%
    }%
  }%
}
%    \end{macrocode}
%    \end{macro}
%    \begin{macro}{\HoLogoBkm@ExTeX}
%    \begin{macrocode}
\def\HoLogoBkm@ExTeX#1{%
  \HOLOGO@PdfdocUnicode{#1{e}{E}x}{\textepsilon\textchi}%
  \hologo{TeX}%
}
%    \end{macrocode}
%    \end{macro}
%    \begin{macro}{\HoLogoCss@ExTeX}
%    \begin{macrocode}
\def\HoLogoCss@ExTeX{%
  \Css{%
    span.HoLogo-ExTeX{%
      font-family:serif;%
    }%
  }%
  \Css{%
    span.HoLogo-ExTeX span.HoLogo-TeX{%
      margin-left:-.15em;%
    }%
  }%
  \global\let\HoLogoCss@ExTeX\relax
}
%    \end{macrocode}
%    \end{macro}
%
% \subsubsection{\hologo{MiKTeX}}
%
%    \begin{macro}{\HoLogo@MiKTeX}
%    \begin{macrocode}
\def\HoLogo@MiKTeX#1{%
  \HOLOGO@mbox{MiK}%
  \HOLOGO@discretionary
  \hologo{TeX}%
}
%    \end{macrocode}
%    \end{macro}
%    \begin{macro}{\HoLogoHtml@MiKTeX}
%    \begin{macrocode}
\let\HoLogoHtml@MiKTeX\HoLogo@MiKTeX
%    \end{macrocode}
%    \end{macro}
%
% \subsubsection{\hologo{OzTeX} and friends}
%
%    Source: \hologo{OzTeX} FAQ \cite{OzTeX}:
%    \begin{quote}
%      |\def\OzTeX{O\kern-.03em z\kern-.15em\TeX}|\\
%      (There is no kerning in OzMF, OzMP and OzTtH.)
%    \end{quote}
%
%    \begin{macro}{\HoLogo@OzTeX}
%    \begin{macrocode}
\def\HoLogo@OzTeX#1{%
  O%
  \kern-.03em %
  z%
  \kern-.15em %
  \hologo{TeX}%
}
%    \end{macrocode}
%    \end{macro}
%    \begin{macro}{\HoLogoHtml@OzTeX}
%    \begin{macrocode}
\def\HoLogoHtml@OzTeX#1{%
  \HoLogoCss@OzTeX
  \HOLOGO@Span{OzTeX}{%
    O%
    \HOLOGO@Span{z}{z}%
    \hologo{TeX}%
  }%
}
%    \end{macrocode}
%    \end{macro}
%    \begin{macro}{\HoLogoCss@OzTeX}
%    \begin{macrocode}
\def\HoLogoCss@OzTeX{%
  \Css{%
    span.HoLogo-OzTeX span.HoLogo-z{%
      margin-left:-.03em;%
      margin-right:-.15em;%
    }%
  }%
  \global\let\HoLogoCss@OzTeX\relax
}
%    \end{macrocode}
%    \end{macro}
%
%    \begin{macro}{\HoLogo@OzMF}
%    \begin{macrocode}
\def\HoLogo@OzMF#1{%
  \HOLOGO@mbox{OzMF}%
}
%    \end{macrocode}
%    \end{macro}
%    \begin{macro}{\HoLogo@OzMP}
%    \begin{macrocode}
\def\HoLogo@OzMP#1{%
  \HOLOGO@mbox{OzMP}%
}
%    \end{macrocode}
%    \end{macro}
%    \begin{macro}{\HoLogo@OzTtH}
%    \begin{macrocode}
\def\HoLogo@OzTtH#1{%
  \HOLOGO@mbox{OzTtH}%
}
%    \end{macrocode}
%    \end{macro}
%
% \subsubsection{\hologo{PCTeX}}
%
%    \begin{macro}{\HoLogo@PCTeX}
%    \begin{macrocode}
\def\HoLogo@PCTeX#1{%
  \HOLOGO@mbox{PC}%
  \hologo{TeX}%
}
%    \end{macrocode}
%    \end{macro}
%    \begin{macro}{\HoLogoHtml@PCTeX}
%    \begin{macrocode}
\let\HoLogoHtml@PCTeX\HoLogo@PCTeX
%    \end{macrocode}
%    \end{macro}
%
% \subsubsection{\hologo{PiCTeX}}
%
%    The original definitions from \xfile{pictex.tex} \cite{PiCTeX}:
%\begin{quote}
%\begin{verbatim}
%\def\PiC{%
%  P%
%  \kern-.12em%
%  \lower.5ex\hbox{I}%
%  \kern-.075em%
%  C%
%}
%\def\PiCTeX{%
%  \PiC
%  \kern-.11em%
%  \TeX
%}
%\end{verbatim}
%\end{quote}
%
%    \begin{macro}{\HoLogo@PiC}
%    \begin{macrocode}
\def\HoLogo@PiC#1{%
  P%
  \kern-.12em%
  \lower.5ex\hbox{I}%
  \kern-.075em%
  C%
  \HOLOGO@SpaceFactor
}
%    \end{macrocode}
%    \end{macro}
%    \begin{macro}{\HoLogoHtml@PiC}
%    \begin{macrocode}
\def\HoLogoHtml@PiC#1{%
  \HoLogoCss@PiC
  \HOLOGO@Span{PiC}{%
    P%
    \HOLOGO@Span{i}{I}%
    C%
  }%
}
%    \end{macrocode}
%    \end{macro}
%    \begin{macro}{\HoLogoCss@PiC}
%    \begin{macrocode}
\def\HoLogoCss@PiC{%
  \Css{%
    span.HoLogo-PiC span.HoLogo-i{%
      position:relative;%
      top:.5ex;%
      margin-left:-.12em;%
      margin-right:-.075em;%
      text-decoration:none;%
    }%
  }%
  \global\let\HoLogoCss@PiC\relax
}
%    \end{macrocode}
%    \end{macro}
%
%    \begin{macro}{\HoLogo@PiCTeX}
%    \begin{macrocode}
\def\HoLogo@PiCTeX#1{%
  \hologo{PiC}%
  \HOLOGO@discretionary
  \kern-.11em%
  \hologo{TeX}%
}
%    \end{macrocode}
%    \end{macro}
%    \begin{macro}{\HoLogoHtml@PiCTeX}
%    \begin{macrocode}
\def\HoLogoHtml@PiCTeX#1{%
  \HoLogoCss@PiCTeX
  \HOLOGO@Span{PiCTeX}{%
    \hologo{PiC}%
    \hologo{TeX}%
  }%
}
%    \end{macrocode}
%    \end{macro}
%    \begin{macro}{\HoLogoCss@PiCTeX}
%    \begin{macrocode}
\def\HoLogoCss@PiCTeX{%
  \Css{%
    span.HoLogo-PiCTeX span.HoLogo-PiC{%
      margin-right:-.11em;%
    }%
  }%
  \global\let\HoLogoCss@PiCTeX\relax
}
%    \end{macrocode}
%    \end{macro}
%
% \subsubsection{\hologo{teTeX}}
%
%    \begin{macro}{\HoLogo@teTeX}
%    \begin{macrocode}
\def\HoLogo@teTeX#1{%
  \HOLOGO@mbox{#1{t}{T}e}%
  \HOLOGO@discretionary
  \hologo{TeX}%
}
%    \end{macrocode}
%    \end{macro}
%    \begin{macro}{\HoLogoCs@teTeX}
%    \begin{macrocode}
\def\HoLogoCs@teTeX#1{#1{t}{T}dfTeX}
%    \end{macrocode}
%    \end{macro}
%    \begin{macro}{\HoLogoBkm@teTeX}
%    \begin{macrocode}
\def\HoLogoBkm@teTeX#1{%
  #1{t}{T}e\hologo{TeX}%
}
%    \end{macrocode}
%    \end{macro}
%    \begin{macro}{\HoLogoHtml@teTeX}
%    \begin{macrocode}
\let\HoLogoHtml@teTeX\HoLogo@teTeX
%    \end{macrocode}
%    \end{macro}
%
% \subsubsection{\hologo{TeX4ht}}
%
%    \begin{macro}{\HoLogo@TeX4ht}
%    \begin{macrocode}
\expandafter\def\csname HoLogo@TeX4ht\endcsname#1{%
  \HOLOGO@mbox{\hologo{TeX}4ht}%
}
%    \end{macrocode}
%    \end{macro}
%    \begin{macro}{\HoLogoHtml@TeX4ht}
%    \begin{macrocode}
\expandafter
\let\csname HoLogoHtml@TeX4ht\expandafter\endcsname
\csname HoLogo@TeX4ht\endcsname
%    \end{macrocode}
%    \end{macro}
%
%
% \subsubsection{\hologo{SageTeX}}
%
%    \begin{macro}{\HoLogo@SageTeX}
%    \begin{macrocode}
\def\HoLogo@SageTeX#1{%
  \HOLOGO@mbox{Sage}%
  \HOLOGO@discretionary
  \HOLOGO@NegativeKerning{eT,oT,To}%
  \hologo{TeX}%
}
%    \end{macrocode}
%    \end{macro}
%    \begin{macro}{\HoLogoHtml@SageTeX}
%    \begin{macrocode}
\let\HoLogoHtml@SageTeX\HoLogo@SageTeX
%    \end{macrocode}
%    \end{macro}
%
% \subsection{\hologo{METAFONT} and friends}
%
%    \begin{macro}{\HoLogo@METAFONT}
%    \begin{macrocode}
\def\HoLogo@METAFONT#1{%
  \HoLogoFont@font{METAFONT}{logo}{%
    \HOLOGO@mbox{META}%
    \HOLOGO@discretionary
    \HOLOGO@mbox{FONT}%
  }%
}
%    \end{macrocode}
%    \end{macro}
%
%    \begin{macro}{\HoLogo@METAPOST}
%    \begin{macrocode}
\def\HoLogo@METAPOST#1{%
  \HoLogoFont@font{METAPOST}{logo}{%
    \HOLOGO@mbox{META}%
    \HOLOGO@discretionary
    \HOLOGO@mbox{POST}%
  }%
}
%    \end{macrocode}
%    \end{macro}
%
%    \begin{macro}{\HoLogo@MetaFun}
%    \begin{macrocode}
\def\HoLogo@MetaFun#1{%
  \HOLOGO@mbox{Meta}%
  \HOLOGO@discretionary
  \HOLOGO@mbox{Fun}%
}
%    \end{macrocode}
%    \end{macro}
%
%    \begin{macro}{\HoLogo@MetaPost}
%    \begin{macrocode}
\def\HoLogo@MetaPost#1{%
  \HOLOGO@mbox{Meta}%
  \HOLOGO@discretionary
  \HOLOGO@mbox{Post}%
}
%    \end{macrocode}
%    \end{macro}
%
% \subsection{Others}
%
% \subsubsection{\hologo{biber}}
%
%    \begin{macro}{\HoLogo@biber}
%    \begin{macrocode}
\def\HoLogo@biber#1{%
  \HOLOGO@mbox{#1{b}{B}i}%
  \HOLOGO@discretionary
  \HOLOGO@mbox{ber}%
}
%    \end{macrocode}
%    \end{macro}
%    \begin{macro}{\HoLogoCs@biber}
%    \begin{macrocode}
\def\HoLogoCs@biber#1{#1{b}{B}iber}
%    \end{macrocode}
%    \end{macro}
%    \begin{macro}{\HoLogoBkm@biber}
%    \begin{macrocode}
\def\HoLogoBkm@biber#1{%
  #1{b}{B}iber%
}
%    \end{macrocode}
%    \end{macro}
%    \begin{macro}{\HoLogoHtml@biber}
%    \begin{macrocode}
\let\HoLogoHtml@biber\HoLogo@biber
%    \end{macrocode}
%    \end{macro}
%
% \subsubsection{\hologo{KOMAScript}}
%
%    \begin{macro}{\HoLogo@KOMAScript}
%    The definition for \hologo{KOMAScript} is taken
%    from \hologo{KOMAScript} (\xfile{scrlogo.dtx}, reformatted) \cite{scrlogo}:
%\begin{quote}
%\begin{verbatim}
%\@ifundefined{KOMAScript}{%
%  \DeclareRobustCommand{\KOMAScript}{%
%    \textsf{%
%      K\kern.05em O\kern.05emM\kern.05em A%
%      \kern.1em-\kern.1em %
%      Script%
%    }%
%  }%
%}{}
%\end{verbatim}
%\end{quote}
%    \begin{macrocode}
\def\HoLogo@KOMAScript#1{%
  \HoLogoFont@font{KOMAScript}{sf}{%
    \HOLOGO@mbox{%
      K\kern.05em%
      O\kern.05em%
      M\kern.05em%
      A%
    }%
    \kern.1em%
    \HOLOGO@hyphen
    \kern.1em%
    \HOLOGO@mbox{Script}%
  }%
}
%    \end{macrocode}
%    \end{macro}
%    \begin{macro}{\HoLogoBkm@KOMAScript}
%    \begin{macrocode}
\def\HoLogoBkm@KOMAScript#1{%
  KOMA-Script%
}
%    \end{macrocode}
%    \end{macro}
%    \begin{macro}{\HoLogoHtml@KOMAScript}
%    \begin{macrocode}
\def\HoLogoHtml@KOMAScript#1{%
  \HoLogoCss@KOMAScript
  \HoLogoFont@font{KOMAScript}{sf}{%
    \HOLOGO@Span{KOMAScript}{%
      K%
      \HOLOGO@Span{O}{O}%
      M%
      \HOLOGO@Span{A}{A}%
      \HOLOGO@Span{hyphen}{-}%
      Script%
    }%
  }%
}
%    \end{macrocode}
%    \end{macro}
%    \begin{macro}{\HoLogoCss@KOMAScript}
%    \begin{macrocode}
\def\HoLogoCss@KOMAScript{%
  \Css{%
    span.HoLogo-KOMAScript{%
      font-family:sans-serif;%
    }%
  }%
  \Css{%
    span.HoLogo-KOMAScript span.HoLogo-O{%
      padding-left:.05em;%
      padding-right:.05em;%
    }%
  }%
  \Css{%
    span.HoLogo-KOMAScript span.HoLogo-A{%
      padding-left:.05em;%
    }%
  }%
  \Css{%
    span.HoLogo-KOMAScript span.HoLogo-hyphen{%
      padding-left:.1em;%
      padding-right:.1em;%
    }%
  }%
  \global\let\HoLogoCss@KOMAScript\relax
}
%    \end{macrocode}
%    \end{macro}
%
% \subsubsection{\hologo{LyX}}
%
%    \begin{macro}{\HoLogo@LyX}
%    The definition is taken from the documentation source files
%    of \hologo{LyX}, \xfile{Intro.lyx} \cite{LyX}:
%\begin{quote}
%\begin{verbatim}
%\def\LyX{%
%  \texorpdfstring{%
%    L\kern-.1667em\lower.25em\hbox{Y}\kern-.125emX\@%
%  }{%
%    LyX%
%  }%
%}
%\end{verbatim}
%\end{quote}
%    \begin{macrocode}
\def\HoLogo@LyX#1{%
  L%
  \kern-.1667em%
  \lower.25em\hbox{Y}%
  \kern-.125em%
  X%
  \HOLOGO@SpaceFactor
}
%    \end{macrocode}
%    \end{macro}
%    \begin{macro}{\HoLogoHtml@LyX}
%    \begin{macrocode}
\def\HoLogoHtml@LyX#1{%
  \HoLogoCss@LyX
  \HOLOGO@Span{LyX}{%
    L%
    \HOLOGO@Span{y}{Y}%
    X%
  }%
}
%    \end{macrocode}
%    \end{macro}
%    \begin{macro}{\HoLogoCss@LyX}
%    \begin{macrocode}
\def\HoLogoCss@LyX{%
  \Css{%
    span.HoLogo-LyX span.HoLogo-y{%
      position:relative;%
      top:.25em;%
      margin-left:-.1667em;%
      margin-right:-.125em;%
      text-decoration:none;%
    }%
  }%
  \global\let\HoLogoCss@LyX\relax
}
%    \end{macrocode}
%    \end{macro}
%
% \subsubsection{\hologo{NTS}}
%
%    \begin{macro}{\HoLogo@NTS}
%    Definition for \hologo{NTS} can be found in
%    package \xpackage{etex\textunderscore man} for the \hologo{eTeX} manual \cite{etexman}
%    and in package \xpackage{dtklogos} \cite{dtklogos}:
%\begin{quote}
%\begin{verbatim}
%\def\NTS{%
%  \leavevmode
%  \hbox{%
%    $%
%      \cal N%
%      \kern-0.35em%
%      \lower0.5ex\hbox{$\cal T$}%
%      \kern-0.2em%
%      S%
%    $%
%  }%
%}
%\end{verbatim}
%\end{quote}
%    \begin{macrocode}
\def\HoLogo@NTS#1{%
  \HoLogoFont@font{NTS}{sy}{%
    N\/%
    \kern-.35em%
    \lower.5ex\hbox{T\/}%
    \kern-.2em%
    S\/%
  }%
  \HOLOGO@SpaceFactor
}
%    \end{macrocode}
%    \end{macro}
%
% \subsubsection{\Hologo{TTH} (\hologo{TeX} to HTML translator)}
%
%    Source: \url{http://hutchinson.belmont.ma.us/tth/}
%    In the HTML source the second `T' is printed as subscript.
%\begin{quote}
%\begin{verbatim}
%T<sub>T</sub>H
%\end{verbatim}
%\end{quote}
%    \begin{macro}{\HoLogo@TTH}
%    \begin{macrocode}
\def\HoLogo@TTH#1{%
  \ltx@mbox{%
    T\HOLOGO@SubScript{T}H%
  }%
  \HOLOGO@SpaceFactor
}
%    \end{macrocode}
%    \end{macro}
%
%    \begin{macro}{\HoLogoHtml@TTH}
%    \begin{macrocode}
\def\HoLogoHtml@TTH#1{%
  T\HCode{<sub>}T\HCode{</sub>}H%
}
%    \end{macrocode}
%    \end{macro}
%
% \subsubsection{\Hologo{HanTheThanh}}
%
%    Partial source: Package \xpackage{dtklogos}.
%    The double accent is U+1EBF (latin small letter e with circumflex
%    and acute).
%    \begin{macro}{\HoLogo@HanTheThanh}
%    \begin{macrocode}
\def\HoLogo@HanTheThanh#1{%
  \ltx@mbox{H\`an}%
  \HOLOGO@space
  \ltx@mbox{%
    Th%
    \HOLOGO@IfCharExists{"1EBF}{%
      \char"1EBF\relax
    }{%
      \^e\hbox to 0pt{\hss\raise .5ex\hbox{\'{}}}%
    }%
  }%
  \HOLOGO@space
  \ltx@mbox{Th\`anh}%
}
%    \end{macrocode}
%    \end{macro}
%    \begin{macro}{\HoLogoBkm@HanTheThanh}
%    \begin{macrocode}
\def\HoLogoBkm@HanTheThanh#1{%
  H\`an %
  Th\HOLOGO@PdfdocUnicode{\^e}{\9036\277} %
  Th\`anh%
}
%    \end{macrocode}
%    \end{macro}
%    \begin{macro}{\HoLogoHtml@HanTheThanh}
%    \begin{macrocode}
\def\HoLogoHtml@HanTheThanh#1{%
  H\`an %
  Th\HCode{&\ltx@hashchar x1ebf;} %
  Th\`anh%
}
%    \end{macrocode}
%    \end{macro}
%
% \subsection{Driver detection}
%
%    \begin{macrocode}
\HOLOGO@IfExists\InputIfFileExists{%
  \InputIfFileExists{hologo.cfg}{}{}%
}{%
  \ltx@IfUndefined{pdf@filesize}{%
    \def\HOLOGO@InputIfExists{%
      \openin\HOLOGO@temp=hologo.cfg\relax
      \ifeof\HOLOGO@temp
        \closein\HOLOGO@temp
      \else
        \closein\HOLOGO@temp
        \begingroup
          \def\x{LaTeX2e}%
        \expandafter\endgroup
        \ifx\fmtname\x
          % \iffalse meta-comment
%
% File: hologo.dtx
% Version: 2016/05/12 v1.11
% Info: A logo collection with bookmark support
%
% Copyright (C) 2010-2012 by
%    Heiko Oberdiek <heiko.oberdiek at googlemail.com>
%
% This work may be distributed and/or modified under the
% conditions of the LaTeX Project Public License, either
% version 1.3c of this license or (at your option) any later
% version. This version of this license is in
%    http://www.latex-project.org/lppl/lppl-1-3c.txt
% and the latest version of this license is in
%    http://www.latex-project.org/lppl.txt
% and version 1.3 or later is part of all distributions of
% LaTeX version 2005/12/01 or later.
%
% This work has the LPPL maintenance status "maintained".
%
% This Current Maintainer of this work is Heiko Oberdiek.
%
% The Base Interpreter refers to any `TeX-Format',
% because some files are installed in TDS:tex/generic//.
%
% This work consists of the main source file hologo.dtx
% and the derived files
%    hologo.sty, hologo.pdf, hologo.ins, hologo.drv, hologo-example.tex,
%    hologo-test1.tex, hologo-test-spacefactor.tex,
%    hologo-test-list.tex.
%
% Distribution:
%    CTAN:macros/latex/contrib/oberdiek/hologo.dtx
%    CTAN:macros/latex/contrib/oberdiek/hologo.pdf
%
% Unpacking:
%    (a) If hologo.ins is present:
%           tex hologo.ins
%    (b) Without hologo.ins:
%           tex hologo.dtx
%    (c) If you insist on using LaTeX
%           latex \let\install=y\input{hologo.dtx}
%        (quote the arguments according to the demands of your shell)
%
% Documentation:
%    (a) If hologo.drv is present:
%           latex hologo.drv
%    (b) Without hologo.drv:
%           latex hologo.dtx; ...
%    The class ltxdoc loads the configuration file ltxdoc.cfg
%    if available. Here you can specify further options, e.g.
%    use A4 as paper format:
%       \PassOptionsToClass{a4paper}{article}
%
%    Programm calls to get the documentation (example):
%       pdflatex hologo.dtx
%       makeindex -s gind.ist hologo.idx
%       pdflatex hologo.dtx
%       makeindex -s gind.ist hologo.idx
%       pdflatex hologo.dtx
%
% Installation:
%    TDS:tex/generic/oberdiek/hologo.sty
%    TDS:doc/latex/oberdiek/hologo.pdf
%    TDS:doc/latex/oberdiek/example/hologo-example.tex
%    TDS:doc/latex/oberdiek/test/hologo-test1.tex
%    TDS:doc/latex/oberdiek/test/hologo-test-spacefactor.tex
%    TDS:doc/latex/oberdiek/test/hologo-test-list.tex
%    TDS:source/latex/oberdiek/hologo.dtx
%
%<*ignore>
\begingroup
  \catcode123=1 %
  \catcode125=2 %
  \def\x{LaTeX2e}%
\expandafter\endgroup
\ifcase 0\ifx\install y1\fi\expandafter
         \ifx\csname processbatchFile\endcsname\relax\else1\fi
         \ifx\fmtname\x\else 1\fi\relax
\else\csname fi\endcsname
%</ignore>
%<*install>
\input docstrip.tex
\Msg{************************************************************************}
\Msg{* Installation}
\Msg{* Package: hologo 2016/05/12 v1.11 A logo collection with bookmark support (HO)}
\Msg{************************************************************************}

\keepsilent
\askforoverwritefalse

\let\MetaPrefix\relax
\preamble

This is a generated file.

Project: hologo
Version: 2016/05/12 v1.11

Copyright (C) 2010-2012 by
   Heiko Oberdiek <heiko.oberdiek at googlemail.com>

This work may be distributed and/or modified under the
conditions of the LaTeX Project Public License, either
version 1.3c of this license or (at your option) any later
version. This version of this license is in
   http://www.latex-project.org/lppl/lppl-1-3c.txt
and the latest version of this license is in
   http://www.latex-project.org/lppl.txt
and version 1.3 or later is part of all distributions of
LaTeX version 2005/12/01 or later.

This work has the LPPL maintenance status "maintained".

This Current Maintainer of this work is Heiko Oberdiek.

The Base Interpreter refers to any `TeX-Format',
because some files are installed in TDS:tex/generic//.

This work consists of the main source file hologo.dtx
and the derived files
   hologo.sty, hologo.pdf, hologo.ins, hologo.drv, hologo-example.tex,
   hologo-test1.tex, hologo-test-spacefactor.tex,
   hologo-test-list.tex.

\endpreamble
\let\MetaPrefix\DoubleperCent

\generate{%
  \file{hologo.ins}{\from{hologo.dtx}{install}}%
  \file{hologo.drv}{\from{hologo.dtx}{driver}}%
  \usedir{tex/generic/oberdiek}%
  \file{hologo.sty}{\from{hologo.dtx}{package}}%
  \usedir{doc/latex/oberdiek/example}%
  \file{hologo-example.tex}{\from{hologo.dtx}{example}}%
  \usedir{doc/latex/oberdiek/test}%
  \file{hologo-test1.tex}{\from{hologo.dtx}{test1}}%
  \file{hologo-test-spacefactor.tex}{\from{hologo.dtx}{test-spacefactor}}%
  \file{hologo-test-list.tex}{\from{hologo.dtx}{test-list}}%
  \nopreamble
  \nopostamble
  \usedir{source/latex/oberdiek/catalogue}%
  \file{hologo.xml}{\from{hologo.dtx}{catalogue}}%
}

\catcode32=13\relax% active space
\let =\space%
\Msg{************************************************************************}
\Msg{*}
\Msg{* To finish the installation you have to move the following}
\Msg{* file into a directory searched by TeX:}
\Msg{*}
\Msg{*     hologo.sty}
\Msg{*}
\Msg{* To produce the documentation run the file `hologo.drv'}
\Msg{* through LaTeX.}
\Msg{*}
\Msg{* Happy TeXing!}
\Msg{*}
\Msg{************************************************************************}

\endbatchfile
%</install>
%<*ignore>
\fi
%</ignore>
%<*driver>
\NeedsTeXFormat{LaTeX2e}
\ProvidesFile{hologo.drv}%
  [2016/05/12 v1.11 A logo collection with bookmark support (HO)]%
\documentclass{ltxdoc}
\usepackage{holtxdoc}[2011/11/22]
\usepackage{hologo}[2016/05/12]
\usepackage{longtable}
\usepackage{array}
\usepackage{paralist}
%\usepackage[T1]{fontenc}
%\usepackage{lmodern}
\begin{document}
  \DocInput{hologo.dtx}%
\end{document}
%</driver>
% \fi
%
%
% \CharacterTable
%  {Upper-case    \A\B\C\D\E\F\G\H\I\J\K\L\M\N\O\P\Q\R\S\T\U\V\W\X\Y\Z
%   Lower-case    \a\b\c\d\e\f\g\h\i\j\k\l\m\n\o\p\q\r\s\t\u\v\w\x\y\z
%   Digits        \0\1\2\3\4\5\6\7\8\9
%   Exclamation   \!     Double quote  \"     Hash (number) \#
%   Dollar        \$     Percent       \%     Ampersand     \&
%   Acute accent  \'     Left paren    \(     Right paren   \)
%   Asterisk      \*     Plus          \+     Comma         \,
%   Minus         \-     Point         \.     Solidus       \/
%   Colon         \:     Semicolon     \;     Less than     \<
%   Equals        \=     Greater than  \>     Question mark \?
%   Commercial at \@     Left bracket  \[     Backslash     \\
%   Right bracket \]     Circumflex    \^     Underscore    \_
%   Grave accent  \`     Left brace    \{     Vertical bar  \|
%   Right brace   \}     Tilde         \~}
%
% \GetFileInfo{hologo.drv}
%
% \title{The \xpackage{hologo} package}
% \date{2016/05/12 v1.11}
% \author{Heiko Oberdiek\\\xemail{heiko.oberdiek at googlemail.com}}
%
% \maketitle
%
% \begin{abstract}
% This package starts a collection of logos with support for bookmarks
% strings.
% \end{abstract}
%
% \tableofcontents
%
% \section{Documentation}
%
% \subsection{Logo macros}
%
% \begin{declcs}{hologo} \M{name}
% \end{declcs}
% Macro \cs{hologo} sets the logo with name \meta{name}.
% The following table shows the supported names.
%
% \begingroup
%   \def\hologoEntry#1#2#3{^^A
%     #1&#2&\hologoLogoSetup{#1}{variant=#2}\hologo{#1}&#3\tabularnewline
%   }
%   \begin{longtable}{>{\ttfamily}l>{\ttfamily}lll}
%     \rmfamily\bfseries{name} & \rmfamily\bfseries variant
%     & \bfseries logo & \bfseries since\\
%     \hline
%     \endhead
%     \hologoList
%   \end{longtable}
% \endgroup
%
% \begin{declcs}{Hologo} \M{name}
% \end{declcs}
% Macro \cs{Hologo} starts the logo \meta{name} with an uppercase
% letter. As an exception small greek letters are not converted
% to uppercase. Examples, see \hologo{eTeX} and \hologo{ExTeX}.
%
% \subsection{Setup macros}
%
% The package does not support package options, but the following
% setup macros can be used to set options.
%
% \begin{declcs}{hologoSetup} \M{key value list}
% \end{declcs}
% Macro \cs{hologoSetup} sets global options.
%
% \begin{declcs}{hologoLogoSetup} \M{logo} \M{key value list}
% \end{declcs}
% Some options can also be used to configure a logo.
% These settings take precedence over global option settings.
%
% \subsection{Options}\label{sec:options}
%
% There are boolean and string options:
% \begin{description}
% \item[Boolean option:]
% It takes |true| or |false|
% as value. If the value is omitted, then |true| is used.
% \item[String option:]
% A value must be given as string. (But the string might be empty.)
% \end{description}
% The following options can be used both in \cs{hologoSetup}
% and \cs{hologoLogoSetup}:
% \begin{description}
% \def\entry#1{\item[\xoption{#1}:]}
% \entry{break}
%   enables or disables line breaks inside the logo. This setting is
%   refined by options \xoption{hyphenbreak}, \xoption{spacebreak}
%   or \xoption{discretionarybreak}.
%   Default is |false|.
% \entry{hyphenbreak}
%   enables or disables the line break right after the hyphen character.
% \entry{spacebreak}
%   enables or disables line breaks at space characters.
% \entry{discretionarybreak}
%   enables or disables line breaks at hyphenation points
%   (inserted by \cs{-}).
% \end{description}
% Macro \cs{hologoLogoSetup} also knows:
% \begin{description}
% \item[\xoption{variant}:]
%   This is a string option. It specifies a variant of a logo that
%   must exist. An empty string selects the package default variant.
% \end{description}
% Example:
% \begin{quote}
%   |\hologoSetup{break=false}|\\
%   |\hologoLogoSetup{plainTeX}{variant=hyphen,hyphenbreak}|\\
%   Then ``plain-\TeX'' contains one break point after the hyphen.
% \end{quote}
%
% \subsection{Driver options}
%
% Sometimes graphical operations are needed to construct some
% glyphs (e.g.\ \hologo{XeTeX}). If package \xpackage{graphics}
% or package \xpackage{pgf} are found, then the macros are taken
% from there. Otherwise the packge defines its own operations
% and therefore needs the driver information. Many drivers are
% detected automatically (\hologo{pdfTeX}/\hologo{LuaTeX}
% in PDF mode, \hologo{XeTeX}, \hologo{VTeX}). These have precedence
% over a driver option. The driver can be given as package option
% or using \cs{hologoDriverSetup}.
% The following list contains the recognized driver options:
% \begin{itemize}
% \item \xoption{pdftex}, \xoption{luatex}
% \item \xoption{dvipdfm}, \xoption{dvipdfmx}
% \item \xoption{dvips}, \xoption{dvipsone}, \xoption{xdvi}
% \item \xoption{xetex}
% \item \xoption{vtex}
% \end{itemize}
% The left driver of a line is the driver name that is used internally.
% The following names are aliases for drivers that use the
% same method. Therefore the entry in the \xext{log} file for
% the used driver prints the internally used driver name.
% \begin{description}
% \item[\xoption{driverfallback}:]
%   This option expects a driver that is used,
%   if the driver could not be detected automatically.
% \end{description}
%
% \begin{declcs}{hologoDriverSetup} \M{driver option}
% \end{declcs}
% The driver can also be configured after package loading
% using \cs{hologoDriverSetup}, also the way for \hologo{plainTeX}
% to setup the driver.
%
% \subsection{Font setup}
%
% Some logos require a special font, but should also be usable by
% \hologo{plainTeX}. Therefore the package provides some ways
% to influence the font settings. The options below
% take font settings as values. Both font commands
% such as \cs{sffamily} and macros that take one argument
% like \cs{textsf} can be used.
%
% \begin{declcs}{hologoFontSetup} \M{key value list}
% \end{declcs}
% Macro \cs{hologoFontSetup} sets the fonts for all logos.
% Supported keys:
% \begin{description}
% \def\entry#1{\item[\xoption{#1}:]}
% \entry{general}
%   This font is used for all logos. The default is empty.
%   That means no special font is used.
% \entry{bibsf}
%   This font is used for
%   {\hologoLogoSetup{BibTeX}{variant=sf}\hologo{BibTeX}}
%   with variant \xoption{sf}.
% \entry{rm}
%   This font is a serif font. It is used for \hologo{ExTeX}.
% \entry{sc}
%   This font specifies a small caps font. It is used for
%   {\hologoLogoSetup{BibTeX}{variant=sc}\hologo{BibTeX}}
%   with variant \xoption{sc}.
% \entry{sf}
%   This font specifies a sans serif font. The default
%   is \cs{sffamily}, then \cs{sf} is tried. Otherwise
%   a warning is given. It is used by \hologo{KOMAScript}.
% \entry{sy}
%   This is the font for math symbols (e.g. cmsy).
%   It is used by \hologo{AmS}, \hologo{NTS}, \hologo{ExTeX}.
% \entry{logo}
%   \hologo{METAFONT} and \hologo{METAPOST} are using that font.
%   In \hologo{LaTeX} \cs{logofamily} is used and
%   the definitions of package \xpackage{mflogo} are used
%   if the package is not loaded.
%   Otherwise the \cs{tenlogo} is used and defined
%   if it does not already exists.
% \end{description}
%
% \begin{declcs}{hologoLogoFontSetup} \M{logo} \M{key value list}
% \end{declcs}
% Fonts can also be set for a logo or logo component separately,
% see the following list.
% The keys are the same as for \cs{hologoFontSetup}.
%
% \begin{longtable}{>{\ttfamily}l>{\sffamily}ll}
%   \meta{logo} & keys & result\\
%   \hline
%   \endhead
%   BibTeX & bibsf & {\hologoLogoSetup{BibTeX}{variant=sf}\hologo{BibTeX}}\\[.5ex]
%   BibTeX & sc & {\hologoLogoSetup{BibTeX}{variant=sc}\hologo{BibTeX}}\\[.5ex]
%   ExTeX & rm & \hologo{ExTeX}\\
%   SliTeX & rm & \hologo{SliTeX}\\[.5ex]
%   AmS & sy & \hologo{AmS}\\
%   ExTeX & sy & \hologo{ExTeX}\\
%   NTS & sy & \hologo{NTS}\\[.5ex]
%   KOMAScript & sf & \hologo{KOMAScript}\\[.5ex]
%   METAFONT & logo & \hologo{METAFONT}\\
%   METAPOST & logo & \hologo{METAPOST}\\[.5ex]
%   SliTeX & sc \hologo{SliTeX}
% \end{longtable}
%
% \subsubsection{Font order}
%
% For all logos the font \xoption{general} is applied first.
% Example:
%\begin{quote}
%|\hologoFontSetup{general=\color{red}}|
%\end{quote}
% will print red logos.
% Then if the font uses a special font \xoption{sf}, for example,
% the font is applied that is setup by \cs{hologoLogoFontSetup}.
% If this font is not setup, then the common font setup
% by \cs{hologoFontSetup} is used. Otherwise a warning is given,
% that there is no font configured.
%
% \subsection{Additional user macros}
%
% Usually a variant of a logo is configured by using
% \cs{hologoLogoSetup}, because it is bad style to mix
% different variants of the same logo in the same text.
% There the following macros are a convenience for testing.
%
% \begin{declcs}{hologoVariant} \M{name} \M{variant}\\
%   \cs{HologoVariant} \M{name} \M{variant}
% \end{declcs}
% Logo \meta{name} is set using \meta{variant} that specifies
% explicitely which variant of the macro is used. If the argument
% is empty, then the default form of the logo is used
% (configurable by \cs{hologoLogoSetup}).
%
% \cs{HologoVariant} is used if the logo is set in a context
% that needs an uppercase first letter (beginning of a sentence, \dots).
%
% \begin{declcs}{hologoList}\\
%   \cs{hologoEntry} \M{logo} \M{variant} \M{since}
% \end{declcs}
% Macro \cs{hologoList} contains all logos that are provided
% by the package including variants. The list consists of calls
% of \cs{hologoEntry} with three arguments starting with the
% logo name \meta{logo} and its variant \meta{variant}. An empty
% variant means the current default. Argument \meta{since} specifies
% with version of the package \xpackage{hologo} is needed to get
% the logo. If the logo is fixed, then the date gets updated.
% Therefore the date \meta{since} is not exactly the date of
% the first introduction, but rather the date of the latest fix.
%
% Before \cs{hologoList} can be used, macro \cs{hologoEntry} needs
% a definition. The example file in section \ref{sec:example}
% shows applications of \cs{hologoList}.
%
% \subsection{Supported contexts}
%
% Macros \cs{hologo} and friends support special contexts:
% \begin{itemize}
% \item \hologo{LaTeX}'s protection mechanism.
% \item Bookmarks of package \xpackage{hyperref}.
% \item Package \xpackage{tex4ht}.
% \item The macros can be used inside \cs{csname} constructs,
%   if \cs{ifincsname} is available (\hologo{pdfTeX}, \hologo{XeTeX},
%   \hologo{LuaTeX}).
% \end{itemize}
%
% \subsection{Example}
% \label{sec:example}
%
% The following example prints the logos in different fonts.
%    \begin{macrocode}
%<*example>
%<<verbatim
\NeedsTeXFormat{LaTeX2e}
\documentclass[a4paper]{article}
\usepackage[
  hmargin=20mm,
  vmargin=20mm,
]{geometry}
\pagestyle{empty}
\usepackage{hologo}[2016/05/12]
\usepackage{longtable}
\usepackage{array}
\setlength{\extrarowheight}{2pt}
\usepackage[T1]{fontenc}
\usepackage{lmodern}
\usepackage{pdflscape}
\usepackage[
  pdfencoding=auto,
]{hyperref}
\hypersetup{
  pdfauthor={Heiko Oberdiek},
  pdftitle={Example for package `hologo'},
  pdfsubject={Logos with fonts lmr, lmss, qtm, qpl, qhv},
}
\usepackage{bookmark}

% Print the logo list on the console

\begingroup
  \typeout{}%
  \typeout{*** Begin of logo list ***}%
  \newcommand*{\hologoEntry}[3]{%
    \typeout{#1 \ifx\\#2\\\else(#2) \fi[#3]}%
  }%
  \hologoList
  \typeout{*** End of logo list ***}%
  \typeout{}%
\endgroup

\begin{document}
\begin{landscape}

  \section{Example file for package `hologo'}

  % Table for font names

  \begin{longtable}{>{\bfseries}ll}
    \textbf{font} & \textbf{Font name}\\
    \hline
    lmr & Latin Modern Roman\\
    lmss & Latin Modern Sans\\
    qtm & \TeX\ Gyre Termes\\
    qhv & \TeX\ Gyre Heros\\
    qpl & \TeX\ Gyre Pagella\\
  \end{longtable}

  % Logo list with logos in different fonts

  \begingroup
    \newcommand*{\SetVariant}[2]{%
      \ifx\\#2\\%
      \else
        \hologoLogoSetup{#1}{variant=#2}%
      \fi
    }%
    \newcommand*{\hologoEntry}[3]{%
      \SetVariant{#1}{#2}%
      \raisebox{1em}[0pt][0pt]{\hypertarget{#1@#2}{}}%
      \bookmark[%
        dest={#1@#2},%
      ]{%
        #1\ifx\\#2\\\else\space(#2)\fi: \Hologo{#1}, \hologo{#1} %
        [Unicode]%
      }%
      \hypersetup{unicode=false}%
      \bookmark[%
        dest={#1@#2},%
      ]{%
        #1\ifx\\#2\\\else\space(#2)\fi: \Hologo{#1}, \hologo{#1} %
        [PDFDocEncoding]%
      }%
      \texttt{#1}%
      &%
      \texttt{#2}%
      &%
      \Hologo{#1}%
      &%
      \SetVariant{#1}{#2}%
      \hologo{#1}%
      &%
      \SetVariant{#1}{#2}%
      \fontfamily{qtm}\selectfont
      \hologo{#1}%
      &%
      \SetVariant{#1}{#2}%
      \fontfamily{qpl}\selectfont
      \hologo{#1}%
      &%
      \SetVariant{#1}{#2}%
      \textsf{\hologo{#1}}%
      &%
      \SetVariant{#1}{#2}%
      \fontfamily{qhv}\selectfont
      \hologo{#1}%
      \tabularnewline
    }%
    \begin{longtable}{llllllll}%
      \textbf{\textit{logo}} & \textbf{\textit{variant}} &
      \texttt{\string\Hologo} &
      \textbf{lmr} & \textbf{qtm} & \textbf{qpl} &
      \textbf{lmss} & \textbf{qhv}
      \tabularnewline
      \hline
      \endhead
      \hologoList
    \end{longtable}%
  \endgroup

\end{landscape}
\end{document}
%verbatim
%</example>
%    \end{macrocode}
%
% \StopEventually{
% }
%
% \section{Implementation}
%    \begin{macrocode}
%<*package>
%    \end{macrocode}
%    Reload check, especially if the package is not used with \LaTeX.
%    \begin{macrocode}
\begingroup\catcode61\catcode48\catcode32=10\relax%
  \catcode13=5 % ^^M
  \endlinechar=13 %
  \catcode35=6 % #
  \catcode39=12 % '
  \catcode44=12 % ,
  \catcode45=12 % -
  \catcode46=12 % .
  \catcode58=12 % :
  \catcode64=11 % @
  \catcode123=1 % {
  \catcode125=2 % }
  \expandafter\let\expandafter\x\csname ver@hologo.sty\endcsname
  \ifx\x\relax % plain-TeX, first loading
  \else
    \def\empty{}%
    \ifx\x\empty % LaTeX, first loading,
      % variable is initialized, but \ProvidesPackage not yet seen
    \else
      \expandafter\ifx\csname PackageInfo\endcsname\relax
        \def\x#1#2{%
          \immediate\write-1{Package #1 Info: #2.}%
        }%
      \else
        \def\x#1#2{\PackageInfo{#1}{#2, stopped}}%
      \fi
      \x{hologo}{The package is already loaded}%
      \aftergroup\endinput
    \fi
  \fi
\endgroup%
%    \end{macrocode}
%    Package identification:
%    \begin{macrocode}
\begingroup\catcode61\catcode48\catcode32=10\relax%
  \catcode13=5 % ^^M
  \endlinechar=13 %
  \catcode35=6 % #
  \catcode39=12 % '
  \catcode40=12 % (
  \catcode41=12 % )
  \catcode44=12 % ,
  \catcode45=12 % -
  \catcode46=12 % .
  \catcode47=12 % /
  \catcode58=12 % :
  \catcode64=11 % @
  \catcode91=12 % [
  \catcode93=12 % ]
  \catcode123=1 % {
  \catcode125=2 % }
  \expandafter\ifx\csname ProvidesPackage\endcsname\relax
    \def\x#1#2#3[#4]{\endgroup
      \immediate\write-1{Package: #3 #4}%
      \xdef#1{#4}%
    }%
  \else
    \def\x#1#2[#3]{\endgroup
      #2[{#3}]%
      \ifx#1\@undefined
        \xdef#1{#3}%
      \fi
      \ifx#1\relax
        \xdef#1{#3}%
      \fi
    }%
  \fi
\expandafter\x\csname ver@hologo.sty\endcsname
\ProvidesPackage{hologo}%
  [2016/05/12 v1.11 A logo collection with bookmark support (HO)]%
%    \end{macrocode}
%
%    \begin{macrocode}
\begingroup\catcode61\catcode48\catcode32=10\relax%
  \catcode13=5 % ^^M
  \endlinechar=13 %
  \catcode123=1 % {
  \catcode125=2 % }
  \catcode64=11 % @
  \def\x{\endgroup
    \expandafter\edef\csname HOLOGO@AtEnd\endcsname{%
      \endlinechar=\the\endlinechar\relax
      \catcode13=\the\catcode13\relax
      \catcode32=\the\catcode32\relax
      \catcode35=\the\catcode35\relax
      \catcode61=\the\catcode61\relax
      \catcode64=\the\catcode64\relax
      \catcode123=\the\catcode123\relax
      \catcode125=\the\catcode125\relax
    }%
  }%
\x\catcode61\catcode48\catcode32=10\relax%
\catcode13=5 % ^^M
\endlinechar=13 %
\catcode35=6 % #
\catcode64=11 % @
\catcode123=1 % {
\catcode125=2 % }
\def\TMP@EnsureCode#1#2{%
  \edef\HOLOGO@AtEnd{%
    \HOLOGO@AtEnd
    \catcode#1=\the\catcode#1\relax
  }%
  \catcode#1=#2\relax
}
\TMP@EnsureCode{10}{12}% ^^J
\TMP@EnsureCode{33}{12}% !
\TMP@EnsureCode{34}{12}% "
\TMP@EnsureCode{36}{3}% $
\TMP@EnsureCode{38}{4}% &
\TMP@EnsureCode{39}{12}% '
\TMP@EnsureCode{40}{12}% (
\TMP@EnsureCode{41}{12}% )
\TMP@EnsureCode{42}{12}% *
\TMP@EnsureCode{43}{12}% +
\TMP@EnsureCode{44}{12}% ,
\TMP@EnsureCode{45}{12}% -
\TMP@EnsureCode{46}{12}% .
\TMP@EnsureCode{47}{12}% /
\TMP@EnsureCode{58}{12}% :
\TMP@EnsureCode{59}{12}% ;
\TMP@EnsureCode{60}{12}% <
\TMP@EnsureCode{62}{12}% >
\TMP@EnsureCode{63}{12}% ?
\TMP@EnsureCode{91}{12}% [
\TMP@EnsureCode{93}{12}% ]
\TMP@EnsureCode{94}{7}% ^ (superscript)
\TMP@EnsureCode{95}{8}% _ (subscript)
\TMP@EnsureCode{96}{12}% `
\TMP@EnsureCode{124}{12}% |
\edef\HOLOGO@AtEnd{%
  \HOLOGO@AtEnd
  \escapechar\the\escapechar\relax
  \noexpand\endinput
}
\escapechar=92 %
%    \end{macrocode}
%
% \subsection{Logo list}
%
%    \begin{macro}{\hologoList}
%    \begin{macrocode}
\def\hologoList{%
  \hologoEntry{(La)TeX}{}{2011/10/01}%
  \hologoEntry{AmSLaTeX}{}{2010/04/16}%
  \hologoEntry{AmSTeX}{}{2010/04/16}%
  \hologoEntry{biber}{}{2011/10/01}%
  \hologoEntry{BibTeX}{}{2011/10/01}%
  \hologoEntry{BibTeX}{sf}{2011/10/01}%
  \hologoEntry{BibTeX}{sc}{2011/10/01}%
  \hologoEntry{BibTeX8}{}{2011/11/22}%
  \hologoEntry{ConTeXt}{}{2011/03/25}%
  \hologoEntry{ConTeXt}{narrow}{2011/03/25}%
  \hologoEntry{ConTeXt}{simple}{2011/03/25}%
  \hologoEntry{emTeX}{}{2010/04/26}%
  \hologoEntry{eTeX}{}{2010/04/08}%
  \hologoEntry{ExTeX}{}{2011/10/01}%
  \hologoEntry{HanTheThanh}{}{2011/11/29}%
  \hologoEntry{iniTeX}{}{2011/10/01}%
  \hologoEntry{KOMAScript}{}{2011/10/01}%
  \hologoEntry{La}{}{2010/05/08}%
  \hologoEntry{LaTeX}{}{2010/04/08}%
  \hologoEntry{LaTeX2e}{}{2010/04/08}%
  \hologoEntry{LaTeX3}{}{2010/04/24}%
  \hologoEntry{LaTeXe}{}{2010/04/08}%
  \hologoEntry{LaTeXML}{}{2011/11/22}%
  \hologoEntry{LaTeXTeX}{}{2011/10/01}%
  \hologoEntry{LuaLaTeX}{}{2010/04/08}%
  \hologoEntry{LuaTeX}{}{2010/04/08}%
  \hologoEntry{LyX}{}{2011/10/01}%
  \hologoEntry{METAFONT}{}{2011/10/01}%
  \hologoEntry{MetaFun}{}{2011/10/01}%
  \hologoEntry{METAPOST}{}{2011/10/01}%
  \hologoEntry{MetaPost}{}{2011/10/01}%
  \hologoEntry{MiKTeX}{}{2011/10/01}%
  \hologoEntry{NTS}{}{2011/10/01}%
  \hologoEntry{OzMF}{}{2011/10/01}%
  \hologoEntry{OzMP}{}{2011/10/01}%
  \hologoEntry{OzTeX}{}{2011/10/01}%
  \hologoEntry{OzTtH}{}{2011/10/01}%
  \hologoEntry{PCTeX}{}{2011/10/01}%
  \hologoEntry{pdfTeX}{}{2011/10/01}%
  \hologoEntry{pdfLaTeX}{}{2011/10/01}%
  \hologoEntry{PiC}{}{2011/10/01}%
  \hologoEntry{PiCTeX}{}{2011/10/01}%
  \hologoEntry{plainTeX}{}{2010/04/08}%
  \hologoEntry{plainTeX}{space}{2010/04/16}%
  \hologoEntry{plainTeX}{hyphen}{2010/04/16}%
  \hologoEntry{plainTeX}{runtogether}{2010/04/16}%
  \hologoEntry{SageTeX}{}{2011/11/22}%
  \hologoEntry{SLiTeX}{}{2011/10/01}%
  \hologoEntry{SLiTeX}{lift}{2011/10/01}%
  \hologoEntry{SLiTeX}{narrow}{2011/10/01}%
  \hologoEntry{SLiTeX}{simple}{2011/10/01}%
  \hologoEntry{SliTeX}{}{2011/10/01}%
  \hologoEntry{SliTeX}{narrow}{2011/10/01}%
  \hologoEntry{SliTeX}{simple}{2011/10/01}%
  \hologoEntry{SliTeX}{lift}{2011/10/01}%
  \hologoEntry{teTeX}{}{2011/10/01}%
  \hologoEntry{TeX}{}{2010/04/08}%
  \hologoEntry{TeX4ht}{}{2011/11/22}%
  \hologoEntry{TTH}{}{2011/11/22}%
  \hologoEntry{virTeX}{}{2011/10/01}%
  \hologoEntry{VTeX}{}{2010/04/24}%
  \hologoEntry{Xe}{}{2010/04/08}%
  \hologoEntry{XeLaTeX}{}{2010/04/08}%
  \hologoEntry{XeTeX}{}{2010/04/08}%
}
%    \end{macrocode}
%    \end{macro}
%
% \subsection{Load resources}
%
%    \begin{macrocode}
\begingroup\expandafter\expandafter\expandafter\endgroup
\expandafter\ifx\csname RequirePackage\endcsname\relax
  \def\TMP@RequirePackage#1[#2]{%
    \begingroup\expandafter\expandafter\expandafter\endgroup
    \expandafter\ifx\csname ver@#1.sty\endcsname\relax
      \input #1.sty\relax
    \fi
  }%
  \TMP@RequirePackage{ltxcmds}[2011/02/04]%
  \TMP@RequirePackage{infwarerr}[2010/04/08]%
  \TMP@RequirePackage{kvsetkeys}[2010/03/01]%
  \TMP@RequirePackage{kvdefinekeys}[2010/03/01]%
  \TMP@RequirePackage{pdftexcmds}[2010/04/01]%
  \TMP@RequirePackage{ifpdf}[2010/01/28]%
  \TMP@RequirePackage{ifluatex}[2010/03/01]%
  \ltx@IfUndefined{newif}{%
    \expandafter\let\csname newif\endcsname\ltx@newif
  }{}%
  \TMP@RequirePackage{ifxetex}[2009/01/23]%
  \TMP@RequirePackage{ifvtex}[2010/03/01]%
\else
  \RequirePackage{ltxcmds}[2011/02/04]%
  \RequirePackage{infwarerr}[2010/04/08]%
  \RequirePackage{kvsetkeys}[2010/03/01]%
  \RequirePackage{kvdefinekeys}[2010/03/01]%
  \RequirePackage{pdftexcmds}[2010/04/01]%
  \RequirePackage{ifpdf}[2010/01/28]%
  \RequirePackage{ifluatex}[2010/03/01]%
  \RequirePackage{ifxetex}[2009/01/23]%
  \RequirePackage{ifvtex}[2010/03/01]%
\fi
%    \end{macrocode}
%
%    \begin{macro}{\HOLOGO@IfDefined}
%    \begin{macrocode}
\def\HOLOGO@IfExists#1{%
  \ifx\@undefined#1%
    \expandafter\ltx@secondoftwo
  \else
    \ifx\relax#1%
      \expandafter\ltx@secondoftwo
    \else
      \expandafter\expandafter\expandafter\ltx@firstoftwo
    \fi
  \fi
}
%    \end{macrocode}
%    \end{macro}
%
% \subsection{Setup macros}
%
%    \begin{macro}{\hologoSetup}
%    \begin{macrocode}
\def\hologoSetup{%
  \let\HOLOGO@name\relax
  \HOLOGO@Setup
}
%    \end{macrocode}
%    \end{macro}
%
%    \begin{macro}{\hologoLogoSetup}
%    \begin{macrocode}
\def\hologoLogoSetup#1{%
  \edef\HOLOGO@name{#1}%
  \ltx@IfUndefined{HoLogo@\HOLOGO@name}{%
    \@PackageError{hologo}{%
      Unknown logo `\HOLOGO@name'%
    }\@ehc
    \ltx@gobble
  }{%
    \HOLOGO@Setup
  }%
}
%    \end{macrocode}
%    \end{macro}
%
%    \begin{macro}{\HOLOGO@Setup}
%    \begin{macrocode}
\def\HOLOGO@Setup{%
  \kvsetkeys{HoLogo}%
}
%    \end{macrocode}
%    \end{macro}
%
% \subsection{Options}
%
%    \begin{macro}{\HOLOGO@DeclareBoolOption}
%    \begin{macrocode}
\def\HOLOGO@DeclareBoolOption#1{%
  \expandafter\chardef\csname HOLOGOOPT@#1\endcsname\ltx@zero
  \kv@define@key{HoLogo}{#1}[true]{%
    \def\HOLOGO@temp{##1}%
    \ifx\HOLOGO@temp\HOLOGO@true
      \ifx\HOLOGO@name\relax
        \expandafter\chardef\csname HOLOGOOPT@#1\endcsname=\ltx@one
      \else
        \expandafter\chardef\csname
        HoLogoOpt@#1@\HOLOGO@name\endcsname\ltx@one
      \fi
      \HOLOGO@SetBreakAll{#1}%
    \else
      \ifx\HOLOGO@temp\HOLOGO@false
        \ifx\HOLOGO@name\relax
          \expandafter\chardef\csname HOLOGOOPT@#1\endcsname=\ltx@zero
        \else
          \expandafter\chardef\csname
          HoLogoOpt@#1@\HOLOGO@name\endcsname=\ltx@zero
        \fi
        \HOLOGO@SetBreakAll{#1}%
      \else
        \@PackageError{hologo}{%
          Unknown value `##1' for boolean option `#1'.\MessageBreak
          Known values are `true' and `false'%
        }\@ehc
      \fi
    \fi
  }%
}
%    \end{macrocode}
%    \end{macro}
%
%    \begin{macro}{\HOLOGO@SetBreakAll}
%    \begin{macrocode}
\def\HOLOGO@SetBreakAll#1{%
  \def\HOLOGO@temp{#1}%
  \ifx\HOLOGO@temp\HOLOGO@break
    \ifx\HOLOGO@name\relax
      \chardef\HOLOGOOPT@hyphenbreak=\HOLOGOOPT@break
      \chardef\HOLOGOOPT@spacebreak=\HOLOGOOPT@break
      \chardef\HOLOGOOPT@discretionarybreak=\HOLOGOOPT@break
    \else
      \expandafter\chardef
         \csname HoLogoOpt@hyphenbreak@\HOLOGO@name\endcsname=%
         \csname HoLogoOpt@break@\HOLOGO@name\endcsname
      \expandafter\chardef
         \csname HoLogoOpt@spacebreak@\HOLOGO@name\endcsname=%
         \csname HoLogoOpt@break@\HOLOGO@name\endcsname
      \expandafter\chardef
         \csname HoLogoOpt@discretionarybreak@\HOLOGO@name
             \endcsname=%
         \csname HoLogoOpt@break@\HOLOGO@name\endcsname
    \fi
  \fi
}
%    \end{macrocode}
%    \end{macro}
%
%    \begin{macro}{\HOLOGO@true}
%    \begin{macrocode}
\def\HOLOGO@true{true}
%    \end{macrocode}
%    \end{macro}
%    \begin{macro}{\HOLOGO@false}
%    \begin{macrocode}
\def\HOLOGO@false{false}
%    \end{macrocode}
%    \end{macro}
%    \begin{macro}{\HOLOGO@break}
%    \begin{macrocode}
\def\HOLOGO@break{break}
%    \end{macrocode}
%    \end{macro}
%
%    \begin{macrocode}
\HOLOGO@DeclareBoolOption{break}
\HOLOGO@DeclareBoolOption{hyphenbreak}
\HOLOGO@DeclareBoolOption{spacebreak}
\HOLOGO@DeclareBoolOption{discretionarybreak}
%    \end{macrocode}
%
%    \begin{macrocode}
\kv@define@key{HoLogo}{variant}{%
  \ifx\HOLOGO@name\relax
    \@PackageError{hologo}{%
      Option `variant' is not available in \string\hologoSetup,%
      \MessageBreak
      Use \string\hologoLogoSetup\space instead%
    }\@ehc
  \else
    \edef\HOLOGO@temp{#1}%
    \ifx\HOLOGO@temp\ltx@empty
      \expandafter
      \let\csname HoLogoOpt@variant@\HOLOGO@name\endcsname\@undefined
    \else
      \ltx@IfUndefined{HoLogo@\HOLOGO@name @\HOLOGO@temp}{%
        \@PackageError{hologo}{%
          Unknown variant `\HOLOGO@temp' of logo `\HOLOGO@name'%
        }\@ehc
      }{%
        \expandafter
        \let\csname HoLogoOpt@variant@\HOLOGO@name\endcsname
            \HOLOGO@temp
      }%
    \fi
  \fi
}
%    \end{macrocode}
%
%    \begin{macro}{\HOLOGO@Variant}
%    \begin{macrocode}
\def\HOLOGO@Variant#1{%
  #1%
  \ltx@ifundefined{HoLogoOpt@variant@#1}{%
  }{%
    @\csname HoLogoOpt@variant@#1\endcsname
  }%
}
%    \end{macrocode}
%    \end{macro}
%
% \subsection{Break/no-break support}
%
%    \begin{macro}{\HOLOGO@space}
%    \begin{macrocode}
\def\HOLOGO@space{%
  \ltx@ifundefined{HoLogoOpt@spacebreak@\HOLOGO@name}{%
    \ltx@ifundefined{HoLogoOpt@break@\HOLOGO@name}{%
      \chardef\HOLOGO@temp=\HOLOGOOPT@spacebreak
    }{%
      \chardef\HOLOGO@temp=%
        \csname HoLogoOpt@break@\HOLOGO@name\endcsname
    }%
  }{%
    \chardef\HOLOGO@temp=%
      \csname HoLogoOpt@spacebreak@\HOLOGO@name\endcsname
  }%
  \ifcase\HOLOGO@temp
    \penalty10000 %
  \fi
  \ltx@space
}
%    \end{macrocode}
%    \end{macro}
%
%    \begin{macro}{\HOLOGO@hyphen}
%    \begin{macrocode}
\def\HOLOGO@hyphen{%
  \ltx@ifundefined{HoLogoOpt@hyphenbreak@\HOLOGO@name}{%
    \ltx@ifundefined{HoLogoOpt@break@\HOLOGO@name}{%
      \chardef\HOLOGO@temp=\HOLOGOOPT@hyphenbreak
    }{%
      \chardef\HOLOGO@temp=%
        \csname HoLogoOpt@break@\HOLOGO@name\endcsname
    }%
  }{%
    \chardef\HOLOGO@temp=%
      \csname HoLogoOpt@hyphenbreak@\HOLOGO@name\endcsname
  }%
  \ifcase\HOLOGO@temp
    \ltx@mbox{-}%
  \else
    -%
  \fi
}
%    \end{macrocode}
%    \end{macro}
%
%    \begin{macro}{\HOLOGO@discretionary}
%    \begin{macrocode}
\def\HOLOGO@discretionary{%
  \ltx@ifundefined{HoLogoOpt@discretionarybreak@\HOLOGO@name}{%
    \ltx@ifundefined{HoLogoOpt@break@\HOLOGO@name}{%
      \chardef\HOLOGO@temp=\HOLOGOOPT@discretionarybreak
    }{%
      \chardef\HOLOGO@temp=%
        \csname HoLogoOpt@break@\HOLOGO@name\endcsname
    }%
  }{%
    \chardef\HOLOGO@temp=%
      \csname HoLogoOpt@discretionarybreak@\HOLOGO@name\endcsname
  }%
  \ifcase\HOLOGO@temp
  \else
    \-%
  \fi
}
%    \end{macrocode}
%    \end{macro}
%
%    \begin{macro}{\HOLOGO@mbox}
%    \begin{macrocode}
\def\HOLOGO@mbox#1{%
  \ltx@ifundefined{HoLogoOpt@break@\HOLOGO@name}{%
    \chardef\HOLOGO@temp=\HOLOGOOPT@hyphenbreak
  }{%
    \chardef\HOLOGO@temp=%
      \csname HoLogoOpt@break@\HOLOGO@name\endcsname
  }%
  \ifcase\HOLOGO@temp
    \ltx@mbox{#1}%
  \else
    #1%
  \fi
}
%    \end{macrocode}
%    \end{macro}
%
% \subsection{Font support}
%
%    \begin{macro}{\HoLogoFont@font}
%    \begin{tabular}{@{}ll@{}}
%    |#1|:& logo name\\
%    |#2|:& font short name\\
%    |#3|:& text
%    \end{tabular}
%    \begin{macrocode}
\def\HoLogoFont@font#1#2#3{%
  \begingroup
    \ltx@IfUndefined{HoLogoFont@logo@#1.#2}{%
      \ltx@IfUndefined{HoLogoFont@font@#2}{%
        \@PackageWarning{hologo}{%
          Missing font `#2' for logo `#1'%
        }%
        #3%
      }{%
        \csname HoLogoFont@font@#2\endcsname{#3}%
      }%
    }{%
      \csname HoLogoFont@logo@#1.#2\endcsname{#3}%
    }%
  \endgroup
}
%    \end{macrocode}
%    \end{macro}
%
%    \begin{macro}{\HoLogoFont@Def}
%    \begin{macrocode}
\def\HoLogoFont@Def#1{%
  \expandafter\def\csname HoLogoFont@font@#1\endcsname
}
%    \end{macrocode}
%    \end{macro}
%    \begin{macro}{\HoLogoFont@LogoDef}
%    \begin{macrocode}
\def\HoLogoFont@LogoDef#1#2{%
  \expandafter\def\csname HoLogoFont@logo@#1.#2\endcsname
}
%    \end{macrocode}
%    \end{macro}
%
% \subsubsection{Font defaults}
%
%    \begin{macro}{\HoLogoFont@font@general}
%    \begin{macrocode}
\HoLogoFont@Def{general}{}%
%    \end{macrocode}
%    \end{macro}
%
%    \begin{macro}{\HoLogoFont@font@rm}
%    \begin{macrocode}
\ltx@IfUndefined{rmfamily}{%
  \ltx@IfUndefined{rm}{%
  }{%
    \HoLogoFont@Def{rm}{\rm}%
  }%
}{%
  \HoLogoFont@Def{rm}{\rmfamily}%
}
%    \end{macrocode}
%    \end{macro}
%
%    \begin{macro}{\HoLogoFont@font@sf}
%    \begin{macrocode}
\ltx@IfUndefined{sffamily}{%
  \ltx@IfUndefined{sf}{%
  }{%
    \HoLogoFont@Def{sf}{\sf}%
  }%
}{%
  \HoLogoFont@Def{sf}{\sffamily}%
}
%    \end{macrocode}
%    \end{macro}
%
%    \begin{macro}{\HoLogoFont@font@bibsf}
%    In case of \hologo{plainTeX} the original small caps
%    variant is used as default. In \hologo{LaTeX}
%    the definition of package \xpackage{dtklogos} \cite{dtklogos}
%    is used.
%\begin{quote}
%\begin{verbatim}
%\DeclareRobustCommand{\BibTeX}{%
%  B%
%  \kern-.05em%
%  \hbox{%
%    $\m@th$% %% force math size calculations
%    \csname S@\f@size\endcsname
%    \fontsize\sf@size\z@
%    \math@fontsfalse
%    \selectfont
%    I%
%    \kern-.025em%
%    B
%  }%
%  \kern-.08em%
%  \-%
%  \TeX
%}
%\end{verbatim}
%\end{quote}
%    \begin{macrocode}
\ltx@IfUndefined{selectfont}{%
  \ltx@IfUndefined{tensc}{%
    \font\tensc=cmcsc10\relax
  }{}%
  \HoLogoFont@Def{bibsf}{\tensc}%
}{%
  \HoLogoFont@Def{bibsf}{%
    $\mathsurround=0pt$%
    \csname S@\f@size\endcsname
    \fontsize\sf@size{0pt}%
    \math@fontsfalse
    \selectfont
  }%
}
%    \end{macrocode}
%    \end{macro}
%
%    \begin{macro}{\HoLogoFont@font@sc}
%    \begin{macrocode}
\ltx@IfUndefined{scshape}{%
  \ltx@IfUndefined{tensc}{%
    \font\tensc=cmcsc10\relax
  }{}%
  \HoLogoFont@Def{sc}{\tensc}%
}{%
  \HoLogoFont@Def{sc}{\scshape}%
}
%    \end{macrocode}
%    \end{macro}
%
%    \begin{macro}{\HoLogoFont@font@sy}
%    \begin{macrocode}
\ltx@IfUndefined{usefont}{%
  \ltx@IfUndefined{tensy}{%
  }{%
    \HoLogoFont@Def{sy}{\tensy}%
  }%
}{%
  \HoLogoFont@Def{sy}{%
    \usefont{OMS}{cmsy}{m}{n}%
  }%
}
%    \end{macrocode}
%    \end{macro}
%
%    \begin{macro}{\HoLogoFont@font@logo}
%    \begin{macrocode}
\begingroup
  \def\x{LaTeX2e}%
\expandafter\endgroup
\ifx\fmtname\x
  \ltx@IfUndefined{logofamily}{%
    \DeclareRobustCommand\logofamily{%
      \not@math@alphabet\logofamily\relax
      \fontencoding{U}%
      \fontfamily{logo}%
      \selectfont
    }%
  }{}%
  \ltx@IfUndefined{logofamily}{%
  }{%
    \HoLogoFont@Def{logo}{\logofamily}%
  }%
\else
  \ltx@IfUndefined{tenlogo}{%
    \font\tenlogo=logo10\relax
  }{}%
  \HoLogoFont@Def{logo}{\tenlogo}%
\fi
%    \end{macrocode}
%    \end{macro}
%
% \subsubsection{Font setup}
%
%    \begin{macro}{\hologoFontSetup}
%    \begin{macrocode}
\def\hologoFontSetup{%
  \let\HOLOGO@name\relax
  \HOLOGO@FontSetup
}
%    \end{macrocode}
%    \end{macro}
%
%    \begin{macro}{\hologoLogoFontSetup}
%    \begin{macrocode}
\def\hologoLogoFontSetup#1{%
  \edef\HOLOGO@name{#1}%
  \ltx@IfUndefined{HoLogo@\HOLOGO@name}{%
    \@PackageError{hologo}{%
      Unknown logo `\HOLOGO@name'%
    }\@ehc
    \ltx@gobble
  }{%
    \HOLOGO@FontSetup
  }%
}
%    \end{macrocode}
%    \end{macro}
%
%    \begin{macro}{\HOLOGO@FontSetup}
%    \begin{macrocode}
\def\HOLOGO@FontSetup{%
  \kvsetkeys{HoLogoFont}%
}
%    \end{macrocode}
%    \end{macro}
%
%    \begin{macrocode}
\def\HOLOGO@temp#1{%
  \kv@define@key{HoLogoFont}{#1}{%
    \ifx\HOLOGO@name\relax
      \HoLogoFont@Def{#1}{##1}%
    \else
      \HoLogoFont@LogoDef\HOLOGO@name{#1}{##1}%
    \fi
  }%
}
\HOLOGO@temp{general}
\HOLOGO@temp{sf}
%    \end{macrocode}
%
% \subsection{Generic logo commands}
%
%    \begin{macrocode}
\HOLOGO@IfExists\hologo{%
  \@PackageError{hologo}{%
    \string\hologo\ltx@space is already defined.\MessageBreak
    Package loading is aborted%
  }\@ehc
  \HOLOGO@AtEnd
}%
\HOLOGO@IfExists\hologoRobust{%
  \@PackageError{hologo}{%
    \string\hologoRobust\ltx@space is already defined.\MessageBreak
    Package loading is aborted%
  }\@ehc
  \HOLOGO@AtEnd
}%
%    \end{macrocode}
%
% \subsubsection{\cs{hologo} and friends}
%
%    \begin{macrocode}
\ifluatex
  \expandafter\ltx@firstofone
\else
  \expandafter\ltx@gobble
\fi
{%
  \ltx@IfUndefined{ifincsname}{%
    \ifnum\luatexversion<36 %
      \expandafter\ltx@gobble
    \else
      \expandafter\ltx@firstofone
    \fi
    {%
      \begingroup
        \ifcase0%
            \directlua{%
              if tex.enableprimitives then %
                tex.enableprimitives('HOLOGO@', {'ifincsname'})%
              else %
                tex.print('1')%
              end%
            }%
            \ifx\HOLOGO@ifincsname\@undefined 1\fi%
            \relax
          \expandafter\ltx@firstofone
        \else
          \endgroup
          \expandafter\ltx@gobble
        \fi
        {%
          \global\let\ifincsname\HOLOGO@ifincsname
        }%
      \HOLOGO@temp
    }%
  }{}%
}
%    \end{macrocode}
%    \begin{macrocode}
\ltx@IfUndefined{ifincsname}{%
  \catcode`$=14 %
}{%
  \catcode`$=9 %
}
%    \end{macrocode}
%
%    \begin{macro}{\hologo}
%    \begin{macrocode}
\def\hologo#1{%
$ \ifincsname
$   \ltx@ifundefined{HoLogoCs@\HOLOGO@Variant{#1}}{%
$     #1%
$   }{%
$     \csname HoLogoCs@\HOLOGO@Variant{#1}\endcsname\ltx@firstoftwo
$   }%
$ \else
    \HOLOGO@IfExists\texorpdfstring\texorpdfstring\ltx@firstoftwo
    {%
      \hologoRobust{#1}%
    }{%
      \ltx@ifundefined{HoLogoBkm@\HOLOGO@Variant{#1}}{%
        \ltx@ifundefined{HoLogo@#1}{?#1?}{#1}%
      }{%
        \csname HoLogoBkm@\HOLOGO@Variant{#1}\endcsname
        \ltx@firstoftwo
      }%
    }%
$ \fi
}
%    \end{macrocode}
%    \end{macro}
%    \begin{macro}{\Hologo}
%    \begin{macrocode}
\def\Hologo#1{%
$ \ifincsname
$   \ltx@ifundefined{HoLogoCs@\HOLOGO@Variant{#1}}{%
$     #1%
$   }{%
$     \csname HoLogoCs@\HOLOGO@Variant{#1}\endcsname\ltx@secondoftwo
$   }%
$ \else
    \HOLOGO@IfExists\texorpdfstring\texorpdfstring\ltx@firstoftwo
    {%
      \HologoRobust{#1}%
    }{%
      \ltx@ifundefined{HoLogoBkm@\HOLOGO@Variant{#1}}{%
        \ltx@ifundefined{HoLogo@#1}{?#1?}{#1}%
      }{%
        \csname HoLogoBkm@\HOLOGO@Variant{#1}\endcsname
        \ltx@secondoftwo
      }%
    }%
$ \fi
}
%    \end{macrocode}
%    \end{macro}
%
%    \begin{macro}{\hologoVariant}
%    \begin{macrocode}
\def\hologoVariant#1#2{%
  \ifx\relax#2\relax
    \hologo{#1}%
  \else
$   \ifincsname
$     \ltx@ifundefined{HoLogoCs@#1@#2}{%
$       #1%
$     }{%
$       \csname HoLogoCs@#1@#2\endcsname\ltx@firstoftwo
$     }%
$   \else
      \HOLOGO@IfExists\texorpdfstring\texorpdfstring\ltx@firstoftwo
      {%
        \hologoVariantRobust{#1}{#2}%
      }{%
        \ltx@ifundefined{HoLogoBkm@#1@#2}{%
          \ltx@ifundefined{HoLogo@#1}{?#1?}{#1}%
        }{%
          \csname HoLogoBkm@#1@#2\endcsname
          \ltx@firstoftwo
        }%
      }%
$   \fi
  \fi
}
%    \end{macrocode}
%    \end{macro}
%    \begin{macro}{\HologoVariant}
%    \begin{macrocode}
\def\HologoVariant#1#2{%
  \ifx\relax#2\relax
    \Hologo{#1}%
  \else
$   \ifincsname
$     \ltx@ifundefined{HoLogoCs@#1@#2}{%
$       #1%
$     }{%
$       \csname HoLogoCs@#1@#2\endcsname\ltx@secondoftwo
$     }%
$   \else
      \HOLOGO@IfExists\texorpdfstring\texorpdfstring\ltx@firstoftwo
      {%
        \HologoVariantRobust{#1}{#2}%
      }{%
        \ltx@ifundefined{HoLogoBkm@#1@#2}{%
          \ltx@ifundefined{HoLogo@#1}{?#1?}{#1}%
        }{%
          \csname HoLogoBkm@#1@#2\endcsname
          \ltx@secondoftwo
        }%
      }%
$   \fi
  \fi
}
%    \end{macrocode}
%    \end{macro}
%
%    \begin{macrocode}
\catcode`\$=3 %
%    \end{macrocode}
%
% \subsubsection{\cs{hologoRobust} and friends}
%
%    \begin{macro}{\hologoRobust}
%    \begin{macrocode}
\ltx@IfUndefined{protected}{%
  \ltx@IfUndefined{DeclareRobustCommand}{%
    \def\hologoRobust#1%
  }{%
    \DeclareRobustCommand*\hologoRobust[1]%
  }%
}{%
  \protected\def\hologoRobust#1%
}%
{%
  \edef\HOLOGO@name{#1}%
  \ltx@IfUndefined{HoLogo@\HOLOGO@Variant\HOLOGO@name}{%
    \@PackageError{hologo}{%
      Unknown logo `\HOLOGO@name'%
    }\@ehc
    ?\HOLOGO@name?%
  }{%
    \ltx@IfUndefined{ver@tex4ht.sty}{%
      \HoLogoFont@font\HOLOGO@name{general}{%
        \csname HoLogo@\HOLOGO@Variant\HOLOGO@name\endcsname
        \ltx@firstoftwo
      }%
    }{%
      \ltx@IfUndefined{HoLogoHtml@\HOLOGO@Variant\HOLOGO@name}{%
        \HOLOGO@name
      }{%
        \csname HoLogoHtml@\HOLOGO@Variant\HOLOGO@name\endcsname
        \ltx@firstoftwo
      }%
    }%
  }%
}
%    \end{macrocode}
%    \end{macro}
%    \begin{macro}{\HologoRobust}
%    \begin{macrocode}
\ltx@IfUndefined{protected}{%
  \ltx@IfUndefined{DeclareRobustCommand}{%
    \def\HologoRobust#1%
  }{%
    \DeclareRobustCommand*\HologoRobust[1]%
  }%
}{%
  \protected\def\HologoRobust#1%
}%
{%
  \edef\HOLOGO@name{#1}%
  \ltx@IfUndefined{HoLogo@\HOLOGO@Variant\HOLOGO@name}{%
    \@PackageError{hologo}{%
      Unknown logo `\HOLOGO@name'%
    }\@ehc
    ?\HOLOGO@name?%
  }{%
    \ltx@IfUndefined{ver@tex4ht.sty}{%
      \HoLogoFont@font\HOLOGO@name{general}{%
        \csname HoLogo@\HOLOGO@Variant\HOLOGO@name\endcsname
        \ltx@secondoftwo
      }%
    }{%
      \ltx@IfUndefined{HoLogoHtml@\HOLOGO@Variant\HOLOGO@name}{%
        \expandafter\HOLOGO@Uppercase\HOLOGO@name
      }{%
        \csname HoLogoHtml@\HOLOGO@Variant\HOLOGO@name\endcsname
        \ltx@secondoftwo
      }%
    }%
  }%
}
%    \end{macrocode}
%    \end{macro}
%    \begin{macro}{\hologoVariantRobust}
%    \begin{macrocode}
\ltx@IfUndefined{protected}{%
  \ltx@IfUndefined{DeclareRobustCommand}{%
    \def\hologoVariantRobust#1#2%
  }{%
    \DeclareRobustCommand*\hologoVariantRobust[2]%
  }%
}{%
  \protected\def\hologoVariantRobust#1#2%
}%
{%
  \begingroup
    \hologoLogoSetup{#1}{variant={#2}}%
    \hologoRobust{#1}%
  \endgroup
}
%    \end{macrocode}
%    \end{macro}
%    \begin{macro}{\HologoVariantRobust}
%    \begin{macrocode}
\ltx@IfUndefined{protected}{%
  \ltx@IfUndefined{DeclareRobustCommand}{%
    \def\HologoVariantRobust#1#2%
  }{%
    \DeclareRobustCommand*\HologoVariantRobust[2]%
  }%
}{%
  \protected\def\HologoVariantRobust#1#2%
}%
{%
  \begingroup
    \hologoLogoSetup{#1}{variant={#2}}%
    \HologoRobust{#1}%
  \endgroup
}
%    \end{macrocode}
%    \end{macro}
%
%    \begin{macro}{\hologorobust}
%    Macro \cs{hologorobust} is only defined for compatibility.
%    Its use is deprecated.
%    \begin{macrocode}
\def\hologorobust{\hologoRobust}
%    \end{macrocode}
%    \end{macro}
%
% \subsection{Helpers}
%
%    \begin{macro}{\HOLOGO@Uppercase}
%    Macro \cs{HOLOGO@Uppercase} is restricted to \cs{uppercase},
%    because \hologo{plainTeX} or \hologo{iniTeX} do not provide
%    \cs{MakeUppercase}.
%    \begin{macrocode}
\def\HOLOGO@Uppercase#1{\uppercase{#1}}
%    \end{macrocode}
%    \end{macro}
%
%    \begin{macro}{\HOLOGO@PdfdocUnicode}
%    \begin{macrocode}
\def\HOLOGO@PdfdocUnicode{%
  \ifx\ifHy@unicode\iftrue
    \expandafter\ltx@secondoftwo
  \else
    \expandafter\ltx@firstoftwo
  \fi
}
%    \end{macrocode}
%    \end{macro}
%
%    \begin{macro}{\HOLOGO@Math}
%    \begin{macrocode}
\def\HOLOGO@MathSetup{%
  \mathsurround0pt\relax
  \HOLOGO@IfExists\f@series{%
    \if b\expandafter\ltx@car\f@series x\@nil
      \csname boldmath\endcsname
   \fi
  }{}%
}
%    \end{macrocode}
%    \end{macro}
%
%    \begin{macro}{\HOLOGO@TempDimen}
%    \begin{macrocode}
\dimendef\HOLOGO@TempDimen=\ltx@zero
%    \end{macrocode}
%    \end{macro}
%    \begin{macro}{\HOLOGO@NegativeKerning}
%    \begin{macrocode}
\def\HOLOGO@NegativeKerning#1{%
  \begingroup
    \HOLOGO@TempDimen=0pt\relax
    \comma@parse@normalized{#1}{%
      \ifdim\HOLOGO@TempDimen=0pt %
        \expandafter\HOLOGO@@NegativeKerning\comma@entry
      \fi
      \ltx@gobble
    }%
    \ifdim\HOLOGO@TempDimen<0pt %
      \kern\HOLOGO@TempDimen
    \fi
  \endgroup
}
%    \end{macrocode}
%    \end{macro}
%    \begin{macro}{\HOLOGO@@NegativeKerning}
%    \begin{macrocode}
\def\HOLOGO@@NegativeKerning#1#2{%
  \setbox\ltx@zero\hbox{#1#2}%
  \HOLOGO@TempDimen=\wd\ltx@zero
  \setbox\ltx@zero\hbox{#1\kern0pt#2}%
  \advance\HOLOGO@TempDimen by -\wd\ltx@zero
}
%    \end{macrocode}
%    \end{macro}
%
%    \begin{macro}{\HOLOGO@SpaceFactor}
%    \begin{macrocode}
\def\HOLOGO@SpaceFactor{%
  \spacefactor1000 %
}
%    \end{macrocode}
%    \end{macro}
%
%    \begin{macro}{\HOLOGO@Span}
%    \begin{macrocode}
\def\HOLOGO@Span#1#2{%
  \HCode{<span class="HoLogo-#1">}%
  #2%
  \HCode{</span>}%
}
%    \end{macrocode}
%    \end{macro}
%
% \subsubsection{Text subscript}
%
%    \begin{macro}{\HOLOGO@SubScript}%
%    \begin{macrocode}
\def\HOLOGO@SubScript#1{%
  \ltx@IfUndefined{textsubscript}{%
    \ltx@IfUndefined{text}{%
      \ltx@mbox{%
        \mathsurround=0pt\relax
        $%
          _{%
            \ltx@IfUndefined{sf@size}{%
              \mathrm{#1}%
            }{%
              \mbox{%
                \fontsize\sf@size{0pt}\selectfont
                #1%
              }%
            }%
          }%
        $%
      }%
    }{%
      \ltx@mbox{%
        \mathsurround=0pt\relax
        $_{\text{#1}}$%
      }%
    }%
  }{%
    \textsubscript{#1}%
  }%
}
%    \end{macrocode}
%    \end{macro}
%
% \subsection{\hologo{TeX} and friends}
%
% \subsubsection{\hologo{TeX}}
%
%    \begin{macro}{\HoLogo@TeX}
%    Source: \hologo{LaTeX} kernel.
%    \begin{macrocode}
\def\HoLogo@TeX#1{%
  T\kern-.1667em\lower.5ex\hbox{E}\kern-.125emX\HOLOGO@SpaceFactor
}
%    \end{macrocode}
%    \end{macro}
%    \begin{macro}{\HoLogoHtml@TeX}
%    \begin{macrocode}
\def\HoLogoHtml@TeX#1{%
  \HoLogoCss@TeX
  \HOLOGO@Span{TeX}{%
    T%
    \HOLOGO@Span{e}{%
      E%
    }%
    X%
  }%
}
%    \end{macrocode}
%    \end{macro}
%    \begin{macro}{\HoLogoCss@TeX}
%    \begin{macrocode}
\def\HoLogoCss@TeX{%
  \Css{%
    span.HoLogo-TeX span.HoLogo-e{%
      position:relative;%
      top:.5ex;%
      margin-left:-.1667em;%
      margin-right:-.125em;%
    }%
  }%
  \Css{%
    a span.HoLogo-TeX span.HoLogo-e{%
      text-decoration:none;%
    }%
  }%
  \global\let\HoLogoCss@TeX\relax
}
%    \end{macrocode}
%    \end{macro}
%
% \subsubsection{\hologo{plainTeX}}
%
%    \begin{macro}{\HoLogo@plainTeX@space}
%    Source: ``The \hologo{TeX}book''
%    \begin{macrocode}
\def\HoLogo@plainTeX@space#1{%
  \HOLOGO@mbox{#1{p}{P}lain}\HOLOGO@space\hologo{TeX}%
}
%    \end{macrocode}
%    \end{macro}
%    \begin{macro}{\HoLogoCs@plainTeX@space}
%    \begin{macrocode}
\def\HoLogoCs@plainTeX@space#1{#1{p}{P}lain TeX}%
%    \end{macrocode}
%    \end{macro}
%    \begin{macro}{\HoLogoBkm@plainTeX@space}
%    \begin{macrocode}
\def\HoLogoBkm@plainTeX@space#1{%
  #1{p}{P}lain \hologo{TeX}%
}
%    \end{macrocode}
%    \end{macro}
%    \begin{macro}{\HoLogoHtml@plainTeX@space}
%    \begin{macrocode}
\def\HoLogoHtml@plainTeX@space#1{%
  #1{p}{P}lain \hologo{TeX}%
}
%    \end{macrocode}
%    \end{macro}
%
%    \begin{macro}{\HoLogo@plainTeX@hyphen}
%    \begin{macrocode}
\def\HoLogo@plainTeX@hyphen#1{%
  \HOLOGO@mbox{#1{p}{P}lain}\HOLOGO@hyphen\hologo{TeX}%
}
%    \end{macrocode}
%    \end{macro}
%    \begin{macro}{\HoLogoCs@plainTeX@hyphen}
%    \begin{macrocode}
\def\HoLogoCs@plainTeX@hyphen#1{#1{p}{P}lain-TeX}
%    \end{macrocode}
%    \end{macro}
%    \begin{macro}{\HoLogoBkm@plainTeX@hyphen}
%    \begin{macrocode}
\def\HoLogoBkm@plainTeX@hyphen#1{%
  #1{p}{P}lain-\hologo{TeX}%
}
%    \end{macrocode}
%    \end{macro}
%    \begin{macro}{\HoLogoHtml@plainTeX@hyphen}
%    \begin{macrocode}
\def\HoLogoHtml@plainTeX@hyphen#1{%
  #1{p}{P}lain-\hologo{TeX}%
}
%    \end{macrocode}
%    \end{macro}
%
%    \begin{macro}{\HoLogo@plainTeX@runtogether}
%    \begin{macrocode}
\def\HoLogo@plainTeX@runtogether#1{%
  \HOLOGO@mbox{#1{p}{P}lain\hologo{TeX}}%
}
%    \end{macrocode}
%    \end{macro}
%    \begin{macro}{\HoLogoCs@plainTeX@runtogether}
%    \begin{macrocode}
\def\HoLogoCs@plainTeX@runtogether#1{#1{p}{P}lainTeX}
%    \end{macrocode}
%    \end{macro}
%    \begin{macro}{\HoLogoBkm@plainTeX@runtogether}
%    \begin{macrocode}
\def\HoLogoBkm@plainTeX@runtogether#1{%
  #1{p}{P}lain\hologo{TeX}%
}
%    \end{macrocode}
%    \end{macro}
%    \begin{macro}{\HoLogoHtml@plainTeX@runtogether}
%    \begin{macrocode}
\def\HoLogoHtml@plainTeX@runtogether#1{%
  #1{p}{P}lain\hologo{TeX}%
}
%    \end{macrocode}
%    \end{macro}
%
%    \begin{macro}{\HoLogo@plainTeX}
%    \begin{macrocode}
\def\HoLogo@plainTeX{\HoLogo@plainTeX@space}
%    \end{macrocode}
%    \end{macro}
%    \begin{macro}{\HoLogoCs@plainTeX}
%    \begin{macrocode}
\def\HoLogoCs@plainTeX{\HoLogoCs@plainTeX@space}
%    \end{macrocode}
%    \end{macro}
%    \begin{macro}{\HoLogoBkm@plainTeX}
%    \begin{macrocode}
\def\HoLogoBkm@plainTeX{\HoLogoBkm@plainTeX@space}
%    \end{macrocode}
%    \end{macro}
%    \begin{macro}{\HoLogoHtml@plainTeX}
%    \begin{macrocode}
\def\HoLogoHtml@plainTeX{\HoLogoHtml@plainTeX@space}
%    \end{macrocode}
%    \end{macro}
%
% \subsubsection{\hologo{LaTeX}}
%
%    Source: \hologo{LaTeX} kernel.
%\begin{quote}
%\begin{verbatim}
%\DeclareRobustCommand{\LaTeX}{%
%  L%
%  \kern-.36em%
%  {%
%    \sbox\z@ T%
%    \vbox to\ht\z@{%
%      \hbox{%
%        \check@mathfonts
%        \fontsize\sf@size\z@
%        \math@fontsfalse
%        \selectfont
%        A%
%      }%
%      \vss
%    }%
%  }%
%  \kern-.15em%
%  \TeX
%}
%\end{verbatim}
%\end{quote}
%
%    \begin{macro}{\HoLogo@La}
%    \begin{macrocode}
\def\HoLogo@La#1{%
  L%
  \kern-.36em%
  \begingroup
    \setbox\ltx@zero\hbox{T}%
    \vbox to\ht\ltx@zero{%
      \hbox{%
        \ltx@ifundefined{check@mathfonts}{%
          \csname sevenrm\endcsname
        }{%
          \check@mathfonts
          \fontsize\sf@size{0pt}%
          \math@fontsfalse\selectfont
        }%
        A%
      }%
      \vss
    }%
  \endgroup
}
%    \end{macrocode}
%    \end{macro}
%
%    \begin{macro}{\HoLogo@LaTeX}
%    Source: \hologo{LaTeX} kernel.
%    \begin{macrocode}
\def\HoLogo@LaTeX#1{%
  \hologo{La}%
  \kern-.15em%
  \hologo{TeX}%
}
%    \end{macrocode}
%    \end{macro}
%    \begin{macro}{\HoLogoHtml@LaTeX}
%    \begin{macrocode}
\def\HoLogoHtml@LaTeX#1{%
  \HoLogoCss@LaTeX
  \HOLOGO@Span{LaTeX}{%
    L%
    \HOLOGO@Span{a}{%
      A%
    }%
    \hologo{TeX}%
  }%
}
%    \end{macrocode}
%    \end{macro}
%    \begin{macro}{\HoLogoCss@LaTeX}
%    \begin{macrocode}
\def\HoLogoCss@LaTeX{%
  \Css{%
    span.HoLogo-LaTeX span.HoLogo-a{%
      position:relative;%
      top:-.5ex;%
      margin-left:-.36em;%
      margin-right:-.15em;%
      font-size:85\%;%
    }%
  }%
  \global\let\HoLogoCss@LaTeX\relax
}
%    \end{macrocode}
%    \end{macro}
%
% \subsubsection{\hologo{(La)TeX}}
%
%    \begin{macro}{\HoLogo@LaTeXTeX}
%    The kerning around the parentheses is taken
%    from package \xpackage{dtklogos} \cite{dtklogos}.
%\begin{quote}
%\begin{verbatim}
%\DeclareRobustCommand{\LaTeXTeX}{%
%  (%
%  \kern-.15em%
%  L%
%  \kern-.36em%
%  {%
%    \sbox\z@ T%
%    \vbox to\ht0{%
%      \hbox{%
%        $\m@th$%
%        \csname S@\f@size\endcsname
%        \fontsize\sf@size\z@
%        \math@fontsfalse
%        \selectfont
%        A%
%      }%
%      \vss
%    }%
%  }%
%  \kern-.2em%
%  )%
%  \kern-.15em%
%  \TeX
%}
%\end{verbatim}
%\end{quote}
%    \begin{macrocode}
\def\HoLogo@LaTeXTeX#1{%
  (%
  \kern-.15em%
  \hologo{La}%
  \kern-.2em%
  )%
  \kern-.15em%
  \hologo{TeX}%
}
%    \end{macrocode}
%    \end{macro}
%    \begin{macro}{\HoLogoBkm@LaTeXTeX}
%    \begin{macrocode}
\def\HoLogoBkm@LaTeXTeX#1{(La)TeX}
%    \end{macrocode}
%    \end{macro}
%
%    \begin{macro}{\HoLogo@(La)TeX}
%    \begin{macrocode}
\expandafter
\let\csname HoLogo@(La)TeX\endcsname\HoLogo@LaTeXTeX
%    \end{macrocode}
%    \end{macro}
%    \begin{macro}{\HoLogoBkm@(La)TeX}
%    \begin{macrocode}
\expandafter
\let\csname HoLogoBkm@(La)TeX\endcsname\HoLogoBkm@LaTeXTeX
%    \end{macrocode}
%    \end{macro}
%    \begin{macro}{\HoLogoHtml@LaTeXTeX}
%    \begin{macrocode}
\def\HoLogoHtml@LaTeXTeX#1{%
  \HoLogoCss@LaTeXTeX
  \HOLOGO@Span{LaTeXTeX}{%
    (%
    \HOLOGO@Span{L}{L}%
    \HOLOGO@Span{a}{A}%
    \HOLOGO@Span{ParenRight}{)}%
    \hologo{TeX}%
  }%
}
%    \end{macrocode}
%    \end{macro}
%    \begin{macro}{\HoLogoHtml@(La)TeX}
%    Kerning after opening parentheses and before closing parentheses
%    is $-0.1$\,em. The original values $-0.15$\,em
%    looked too ugly for a serif font.
%    \begin{macrocode}
\expandafter
\let\csname HoLogoHtml@(La)TeX\endcsname\HoLogoHtml@LaTeXTeX
%    \end{macrocode}
%    \end{macro}
%    \begin{macro}{\HoLogoCss@LaTeXTeX}
%    \begin{macrocode}
\def\HoLogoCss@LaTeXTeX{%
  \Css{%
    span.HoLogo-LaTeXTeX span.HoLogo-L{%
      margin-left:-.1em;%
    }%
  }%
  \Css{%
    span.HoLogo-LaTeXTeX span.HoLogo-a{%
      position:relative;%
      top:-.5ex;%
      margin-left:-.36em;%
      margin-right:-.1em;%
      font-size:85\%;%
    }%
  }%
  \Css{%
    span.HoLogo-LaTeXTeX span.HoLogo-ParenRight{%
      margin-right:-.15em;%
    }%
  }%
  \global\let\HoLogoCss@LaTeXTeX\relax
}
%    \end{macrocode}
%    \end{macro}
%
% \subsubsection{\hologo{LaTeXe}}
%
%    \begin{macro}{\HoLogo@LaTeXe}
%    Source: \hologo{LaTeX} kernel
%    \begin{macrocode}
\def\HoLogo@LaTeXe#1{%
  \hologo{LaTeX}%
  \kern.15em%
  \hbox{%
    \HOLOGO@MathSetup
    2%
    $_{\textstyle\varepsilon}$%
  }%
}
%    \end{macrocode}
%    \end{macro}
%
%    \begin{macro}{\HoLogoCs@LaTeXe}
%    \begin{macrocode}
\ifnum64=`\^^^^0040\relax % test for big chars of LuaTeX/XeTeX
  \catcode`\$=9 %
  \catcode`\&=14 %
\else
  \catcode`\$=14 %
  \catcode`\&=9 %
\fi
\def\HoLogoCs@LaTeXe#1{%
  LaTeX2%
$ \string ^^^^0395%
& e%
}%
\catcode`\$=3 %
\catcode`\&=4 %
%    \end{macrocode}
%    \end{macro}
%
%    \begin{macro}{\HoLogoBkm@LaTeXe}
%    \begin{macrocode}
\def\HoLogoBkm@LaTeXe#1{%
  \hologo{LaTeX}%
  2%
  \HOLOGO@PdfdocUnicode{e}{\textepsilon}%
}
%    \end{macrocode}
%    \end{macro}
%
%    \begin{macro}{\HoLogoHtml@LaTeXe}
%    \begin{macrocode}
\def\HoLogoHtml@LaTeXe#1{%
  \HoLogoCss@LaTeXe
  \HOLOGO@Span{LaTeX2e}{%
    \hologo{LaTeX}%
    \HOLOGO@Span{2}{2}%
    \HOLOGO@Span{e}{%
      \HOLOGO@MathSetup
      \ensuremath{\textstyle\varepsilon}%
    }%
  }%
}
%    \end{macrocode}
%    \end{macro}
%    \begin{macro}{\HoLogoCss@LaTeXe}
%    \begin{macrocode}
\def\HoLogoCss@LaTeXe{%
  \Css{%
    span.HoLogo-LaTeX2e span.HoLogo-2{%
      padding-left:.15em;%
    }%
  }%
  \Css{%
    span.HoLogo-LaTeX2e span.HoLogo-e{%
      position:relative;%
      top:.35ex;%
      text-decoration:none;%
    }%
  }%
  \global\let\HoLogoCss@LaTeXe\relax
}
%    \end{macrocode}
%    \end{macro}
%
%    \begin{macro}{\HoLogo@LaTeX2e}
%    \begin{macrocode}
\expandafter
\let\csname HoLogo@LaTeX2e\endcsname\HoLogo@LaTeXe
%    \end{macrocode}
%    \end{macro}
%    \begin{macro}{\HoLogoCs@LaTeX2e}
%    \begin{macrocode}
\expandafter
\let\csname HoLogoCs@LaTeX2e\endcsname\HoLogoCs@LaTeXe
%    \end{macrocode}
%    \end{macro}
%    \begin{macro}{\HoLogoBkm@LaTeX2e}
%    \begin{macrocode}
\expandafter
\let\csname HoLogoBkm@LaTeX2e\endcsname\HoLogoBkm@LaTeXe
%    \end{macrocode}
%    \end{macro}
%    \begin{macro}{\HoLogoHtml@LaTeX2e}
%    \begin{macrocode}
\expandafter
\let\csname HoLogoHtml@LaTeX2e\endcsname\HoLogoHtml@LaTeXe
%    \end{macrocode}
%    \end{macro}
%
% \subsubsection{\hologo{LaTeX3}}
%
%    \begin{macro}{\HoLogo@LaTeX3}
%    Source: \hologo{LaTeX} kernel
%    \begin{macrocode}
\expandafter\def\csname HoLogo@LaTeX3\endcsname#1{%
  \hologo{LaTeX}%
  3%
}
%    \end{macrocode}
%    \end{macro}
%
%    \begin{macro}{\HoLogoBkm@LaTeX3}
%    \begin{macrocode}
\expandafter\def\csname HoLogoBkm@LaTeX3\endcsname#1{%
  \hologo{LaTeX}%
  3%
}
%    \end{macrocode}
%    \end{macro}
%    \begin{macro}{\HoLogoHtml@LaTeX3}
%    \begin{macrocode}
\expandafter
\let\csname HoLogoHtml@LaTeX3\expandafter\endcsname
\csname HoLogo@LaTeX3\endcsname
%    \end{macrocode}
%    \end{macro}
%
% \subsubsection{\hologo{LaTeXML}}
%
%    \begin{macro}{\HoLogo@LaTeXML}
%    \begin{macrocode}
\def\HoLogo@LaTeXML#1{%
  \HOLOGO@mbox{%
    \hologo{La}%
    \kern-.15em%
    T%
    \kern-.1667em%
    \lower.5ex\hbox{E}%
    \kern-.125em%
    \HoLogoFont@font{LaTeXML}{sc}{xml}%
  }%
}
%    \end{macrocode}
%    \end{macro}
%    \begin{macro}{\HoLogoHtml@pdfLaTeX}
%    \begin{macrocode}
\def\HoLogoHtml@LaTeXML#1{%
  \HOLOGO@Span{LaTeXML}{%
    \HoLogoCss@LaTeX
    \HoLogoCss@TeX
    \HOLOGO@Span{LaTeX}{%
      L%
      \HOLOGO@Span{a}{%
        A%
      }%
    }%
    \HOLOGO@Span{TeX}{%
      T%
      \HOLOGO@Span{e}{%
        E%
      }%
    }%
    \HCode{<span style="font-variant: small-caps;">}%
    xml%
    \HCode{</span>}%
  }%
}
%    \end{macrocode}
%    \end{macro}
%
% \subsubsection{\hologo{eTeX}}
%
%    \begin{macro}{\HoLogo@eTeX}
%    Source: package \xpackage{etex}
%    \begin{macrocode}
\def\HoLogo@eTeX#1{%
  \ltx@mbox{%
    \HOLOGO@MathSetup
    $\varepsilon$%
    -%
    \HOLOGO@NegativeKerning{-T,T-,To}%
    \hologo{TeX}%
  }%
}
%    \end{macrocode}
%    \end{macro}
%    \begin{macro}{\HoLogoCs@eTeX}
%    \begin{macrocode}
\ifnum64=`\^^^^0040\relax % test for big chars of LuaTeX/XeTeX
  \catcode`\$=9 %
  \catcode`\&=14 %
\else
  \catcode`\$=14 %
  \catcode`\&=9 %
\fi
\def\HoLogoCs@eTeX#1{%
$ #1{\string ^^^^0395}{\string ^^^^03b5}%
& #1{e}{E}%
  TeX%
}%
\catcode`\$=3 %
\catcode`\&=4 %
%    \end{macrocode}
%    \end{macro}
%    \begin{macro}{\HoLogoBkm@eTeX}
%    \begin{macrocode}
\def\HoLogoBkm@eTeX#1{%
  \HOLOGO@PdfdocUnicode{#1{e}{E}}{\textepsilon}%
  -%
  \hologo{TeX}%
}
%    \end{macrocode}
%    \end{macro}
%    \begin{macro}{\HoLogoHtml@eTeX}
%    \begin{macrocode}
\def\HoLogoHtml@eTeX#1{%
  \ltx@mbox{%
    \HOLOGO@MathSetup
    $\varepsilon$%
    -%
    \hologo{TeX}%
  }%
}
%    \end{macrocode}
%    \end{macro}
%
% \subsubsection{\hologo{iniTeX}}
%
%    \begin{macro}{\HoLogo@iniTeX}
%    \begin{macrocode}
\def\HoLogo@iniTeX#1{%
  \HOLOGO@mbox{%
    #1{i}{I}ni\hologo{TeX}%
  }%
}
%    \end{macrocode}
%    \end{macro}
%    \begin{macro}{\HoLogoCs@iniTeX}
%    \begin{macrocode}
\def\HoLogoCs@iniTeX#1{#1{i}{I}niTeX}
%    \end{macrocode}
%    \end{macro}
%    \begin{macro}{\HoLogoBkm@iniTeX}
%    \begin{macrocode}
\def\HoLogoBkm@iniTeX#1{%
  #1{i}{I}ni\hologo{TeX}%
}
%    \end{macrocode}
%    \end{macro}
%    \begin{macro}{\HoLogoHtml@iniTeX}
%    \begin{macrocode}
\let\HoLogoHtml@iniTeX\HoLogo@iniTeX
%    \end{macrocode}
%    \end{macro}
%
% \subsubsection{\hologo{virTeX}}
%
%    \begin{macro}{\HoLogo@virTeX}
%    \begin{macrocode}
\def\HoLogo@virTeX#1{%
  \HOLOGO@mbox{%
    #1{v}{V}ir\hologo{TeX}%
  }%
}
%    \end{macrocode}
%    \end{macro}
%    \begin{macro}{\HoLogoCs@virTeX}
%    \begin{macrocode}
\def\HoLogoCs@virTeX#1{#1{v}{V}irTeX}
%    \end{macrocode}
%    \end{macro}
%    \begin{macro}{\HoLogoBkm@virTeX}
%    \begin{macrocode}
\def\HoLogoBkm@virTeX#1{%
  #1{v}{V}ir\hologo{TeX}%
}
%    \end{macrocode}
%    \end{macro}
%    \begin{macro}{\HoLogoHtml@virTeX}
%    \begin{macrocode}
\let\HoLogoHtml@virTeX\HoLogo@virTeX
%    \end{macrocode}
%    \end{macro}
%
% \subsubsection{\hologo{SliTeX}}
%
% \paragraph{Definitions of the three variants.}
%
%    \begin{macro}{\HoLogo@SLiTeX@lift}
%    \begin{macrocode}
\def\HoLogo@SLiTeX@lift#1{%
  \HoLogoFont@font{SliTeX}{rm}{%
    S%
    \kern-.06em%
    L%
    \kern-.18em%
    \raise.32ex\hbox{\HoLogoFont@font{SliTeX}{sc}{i}}%
    \HOLOGO@discretionary
    \kern-.06em%
    \hologo{TeX}%
  }%
}
%    \end{macrocode}
%    \end{macro}
%    \begin{macro}{\HoLogoBkm@SLiTeX@lift}
%    \begin{macrocode}
\def\HoLogoBkm@SLiTeX@lift#1{SLiTeX}
%    \end{macrocode}
%    \end{macro}
%    \begin{macro}{\HoLogoHtml@SLiTeX@lift}
%    \begin{macrocode}
\def\HoLogoHtml@SLiTeX@lift#1{%
  \HoLogoCss@SLiTeX@lift
  \HOLOGO@Span{SLiTeX-lift}{%
    \HoLogoFont@font{SliTeX}{rm}{%
      S%
      \HOLOGO@Span{L}{L}%
      \HOLOGO@Span{i}{i}%
      \hologo{TeX}%
    }%
  }%
}
%    \end{macrocode}
%    \end{macro}
%    \begin{macro}{\HoLogoCss@SLiTeX@lift}
%    \begin{macrocode}
\def\HoLogoCss@SLiTeX@lift{%
  \Css{%
    span.HoLogo-SLiTeX-lift span.HoLogo-L{%
      margin-left:-.06em;%
      margin-right:-.18em;%
    }%
  }%
  \Css{%
    span.HoLogo-SLiTeX-lift span.HoLogo-i{%
      position:relative;%
      top:-.32ex;%
      margin-right:-.06em;%
      font-variant:small-caps;%
    }%
  }%
  \global\let\HoLogoCss@SLiTeX@lift\relax
}
%    \end{macrocode}
%    \end{macro}
%
%    \begin{macro}{\HoLogo@SliTeX@simple}
%    \begin{macrocode}
\def\HoLogo@SliTeX@simple#1{%
  \HoLogoFont@font{SliTeX}{rm}{%
    \ltx@mbox{%
      \HoLogoFont@font{SliTeX}{sc}{Sli}%
    }%
    \HOLOGO@discretionary
    \hologo{TeX}%
  }%
}
%    \end{macrocode}
%    \end{macro}
%    \begin{macro}{\HoLogoBkm@SliTeX@simple}
%    \begin{macrocode}
\def\HoLogoBkm@SliTeX@simple#1{SliTeX}
%    \end{macrocode}
%    \end{macro}
%    \begin{macro}{\HoLogoHtml@SliTeX@simple}
%    \begin{macrocode}
\let\HoLogoHtml@SliTeX@simple\HoLogo@SliTeX@simple
%    \end{macrocode}
%    \end{macro}
%
%    \begin{macro}{\HoLogo@SliTeX@narrow}
%    \begin{macrocode}
\def\HoLogo@SliTeX@narrow#1{%
  \HoLogoFont@font{SliTeX}{rm}{%
    \ltx@mbox{%
      S%
      \kern-.06em%
      \HoLogoFont@font{SliTeX}{sc}{%
        l%
        \kern-.035em%
        i%
      }%
    }%
    \HOLOGO@discretionary
    \kern-.06em%
    \hologo{TeX}%
  }%
}
%    \end{macrocode}
%    \end{macro}
%    \begin{macro}{\HoLogoBkm@SliTeX@narrow}
%    \begin{macrocode}
\def\HoLogoBkm@SliTeX@narrow#1{SliTeX}
%    \end{macrocode}
%    \end{macro}
%    \begin{macro}{\HoLogoHtml@SliTeX@narrow}
%    \begin{macrocode}
\def\HoLogoHtml@SliTeX@narrow#1{%
  \HoLogoCss@SliTeX@narrow
  \HOLOGO@Span{SliTeX-narrow}{%
    \HoLogoFont@font{SliTeX}{rm}{%
      S%
        \HOLOGO@Span{l}{l}%
        \HOLOGO@Span{i}{i}%
      \hologo{TeX}%
    }%
  }%
}
%    \end{macrocode}
%    \end{macro}
%    \begin{macro}{\HoLogoCss@SliTeX@narrow}
%    \begin{macrocode}
\def\HoLogoCss@SliTeX@narrow{%
  \Css{%
    span.HoLogo-SliTeX-narrow span.HoLogo-l{%
      margin-left:-.06em;%
      margin-right:-.035em;%
      font-variant:small-caps;%
    }%
  }%
  \Css{%
    span.HoLogo-SliTeX-narrow span.HoLogo-i{%
      margin-right:-.06em;%
      font-variant:small-caps;%
    }%
  }%
  \global\let\HoLogoCss@SliTeX@narrow\relax
}
%    \end{macrocode}
%    \end{macro}
%
% \paragraph{Macro set completion.}
%
%    \begin{macro}{\HoLogo@SLiTeX@simple}
%    \begin{macrocode}
\def\HoLogo@SLiTeX@simple{\HoLogo@SliTeX@simple}
%    \end{macrocode}
%    \end{macro}
%    \begin{macro}{\HoLogoBkm@SLiTeX@simple}
%    \begin{macrocode}
\def\HoLogoBkm@SLiTeX@simple{\HoLogoBkm@SliTeX@simple}
%    \end{macrocode}
%    \end{macro}
%    \begin{macro}{\HoLogoHtml@SLiTeX@simple}
%    \begin{macrocode}
\def\HoLogoHtml@SLiTeX@simple{\HoLogoHtml@SliTeX@simple}
%    \end{macrocode}
%    \end{macro}
%
%    \begin{macro}{\HoLogo@SLiTeX@narrow}
%    \begin{macrocode}
\def\HoLogo@SLiTeX@narrow{\HoLogo@SliTeX@narrow}
%    \end{macrocode}
%    \end{macro}
%    \begin{macro}{\HoLogoBkm@SLiTeX@narrow}
%    \begin{macrocode}
\def\HoLogoBkm@SLiTeX@narrow{\HoLogoBkm@SliTeX@narrow}
%    \end{macrocode}
%    \end{macro}
%    \begin{macro}{\HoLogoHtml@SLiTeX@narrow}
%    \begin{macrocode}
\def\HoLogoHtml@SLiTeX@narrow{\HoLogoHtml@SliTeX@narrow}
%    \end{macrocode}
%    \end{macro}
%
%    \begin{macro}{\HoLogo@SliTeX@lift}
%    \begin{macrocode}
\def\HoLogo@SliTeX@lift{\HoLogo@SLiTeX@lift}
%    \end{macrocode}
%    \end{macro}
%    \begin{macro}{\HoLogoBkm@SliTeX@lift}
%    \begin{macrocode}
\def\HoLogoBkm@SliTeX@lift{\HoLogoBkm@SLiTeX@lift}
%    \end{macrocode}
%    \end{macro}
%    \begin{macro}{\HoLogoHtml@SliTeX@lift}
%    \begin{macrocode}
\def\HoLogoHtml@SliTeX@lift{\HoLogoHtml@SLiTeX@lift}
%    \end{macrocode}
%    \end{macro}
%
% \paragraph{Defaults.}
%
%    \begin{macro}{\HoLogo@SLiTeX}
%    \begin{macrocode}
\def\HoLogo@SLiTeX{\HoLogo@SLiTeX@lift}
%    \end{macrocode}
%    \end{macro}
%    \begin{macro}{\HoLogoBkm@SLiTeX}
%    \begin{macrocode}
\def\HoLogoBkm@SLiTeX{\HoLogoBkm@SLiTeX@lift}
%    \end{macrocode}
%    \end{macro}
%    \begin{macro}{\HoLogoHtml@SLiTeX}
%    \begin{macrocode}
\def\HoLogoHtml@SLiTeX{\HoLogoHtml@SLiTeX@lift}
%    \end{macrocode}
%    \end{macro}
%
%    \begin{macro}{\HoLogo@SliTeX}
%    \begin{macrocode}
\def\HoLogo@SliTeX{\HoLogo@SliTeX@narrow}
%    \end{macrocode}
%    \end{macro}
%    \begin{macro}{\HoLogoBkm@SliTeX}
%    \begin{macrocode}
\def\HoLogoBkm@SliTeX{\HoLogoBkm@SliTeX@narrow}
%    \end{macrocode}
%    \end{macro}
%    \begin{macro}{\HoLogoHtml@SliTeX}
%    \begin{macrocode}
\def\HoLogoHtml@SliTeX{\HoLogoHtml@SliTeX@narrow}
%    \end{macrocode}
%    \end{macro}
%
% \subsubsection{\hologo{LuaTeX}}
%
%    \begin{macro}{\HoLogo@LuaTeX}
%    The kerning is an idea of Hans Hagen, see mailing list
%    `luatex at tug dot org' in March 2010.
%    \begin{macrocode}
\def\HoLogo@LuaTeX#1{%
  \HOLOGO@mbox{%
    Lua%
    \HOLOGO@NegativeKerning{aT,oT,To}%
    \hologo{TeX}%
  }%
}
%    \end{macrocode}
%    \end{macro}
%    \begin{macro}{\HoLogoHtml@LuaTeX}
%    \begin{macrocode}
\let\HoLogoHtml@LuaTeX\HoLogo@LuaTeX
%    \end{macrocode}
%    \end{macro}
%
% \subsubsection{\hologo{LuaLaTeX}}
%
%    \begin{macro}{\HoLogo@LuaLaTeX}
%    \begin{macrocode}
\def\HoLogo@LuaLaTeX#1{%
  \HOLOGO@mbox{%
    Lua%
    \hologo{LaTeX}%
  }%
}
%    \end{macrocode}
%    \end{macro}
%    \begin{macro}{\HoLogoHtml@LuaLaTeX}
%    \begin{macrocode}
\let\HoLogoHtml@LuaLaTeX\HoLogo@LuaLaTeX
%    \end{macrocode}
%    \end{macro}
%
% \subsubsection{\hologo{XeTeX}, \hologo{XeLaTeX}}
%
%    \begin{macro}{\HOLOGO@IfCharExists}
%    \begin{macrocode}
\ifluatex
  \ifnum\luatexversion<36 %
  \else
    \def\HOLOGO@IfCharExists#1{%
      \ifnum
        \directlua{%
           if luaotfload and luaotfload.aux then
             if luaotfload.aux.font_has_glyph(%
                    font.current(), \number#1) then % 	 
	       tex.print("1") % 	 
	     end % 	 
	   elseif font and font.fonts and font.current then %
            local f = font.fonts[font.current()]%
            if f.characters and f.characters[\number#1] then %
              tex.print("1")%
            end %
          end%
        }0=\ltx@zero
        \expandafter\ltx@secondoftwo
      \else
        \expandafter\ltx@firstoftwo
      \fi
    }%
  \fi
\fi
\ltx@IfUndefined{HOLOGO@IfCharExists}{%
  \def\HOLOGO@@IfCharExists#1{%
    \begingroup
      \tracinglostchars=\ltx@zero
      \setbox\ltx@zero=\hbox{%
        \kern7sp\char#1\relax
        \ifnum\lastkern>\ltx@zero
          \expandafter\aftergroup\csname iffalse\endcsname
        \else
          \expandafter\aftergroup\csname iftrue\endcsname
        \fi
      }%
      % \if{true|false} from \aftergroup
      \endgroup
      \expandafter\ltx@firstoftwo
    \else
      \endgroup
      \expandafter\ltx@secondoftwo
    \fi
  }%
  \ifxetex
    \ltx@IfUndefined{XeTeXfonttype}{}{%
      \ltx@IfUndefined{XeTeXcharglyph}{}{%
        \def\HOLOGO@IfCharExists#1{%
          \ifnum\XeTeXfonttype\font>\ltx@zero
            \expandafter\ltx@firstofthree
          \else
            \expandafter\ltx@gobble
          \fi
          {%
            \ifnum\XeTeXcharglyph#1>\ltx@zero
              \expandafter\ltx@firstoftwo
            \else
              \expandafter\ltx@secondoftwo
            \fi
          }%
          \HOLOGO@@IfCharExists{#1}%
        }%
      }%
    }%
  \fi
}{}
\ltx@ifundefined{HOLOGO@IfCharExists}{%
  \ifnum64=`\^^^^0040\relax % test for big chars of LuaTeX/XeTeX
    \let\HOLOGO@IfCharExists\HOLOGO@@IfCharExists
  \else
    \def\HOLOGO@IfCharExists#1{%
      \ifnum#1>255 %
        \expandafter\ltx@fourthoffour
      \fi
      \HOLOGO@@IfCharExists{#1}%
    }%
  \fi
}{}
%    \end{macrocode}
%    \end{macro}
%
%    \begin{macro}{\HoLogo@Xe}
%    Source: package \xpackage{dtklogos}
%    \begin{macrocode}
\def\HoLogo@Xe#1{%
  X%
  \kern-.1em\relax
  \HOLOGO@IfCharExists{"018E}{%
    \lower.5ex\hbox{\char"018E}%
  }{%
    \chardef\HOLOGO@choice=\ltx@zero
    \ifdim\fontdimen\ltx@one\font>0pt %
      \ltx@IfUndefined{rotatebox}{%
        \ltx@IfUndefined{pgftext}{%
          \ltx@IfUndefined{psscalebox}{%
            \ltx@IfUndefined{HOLOGO@ScaleBox@\hologoDriver}{%
            }{%
              \chardef\HOLOGO@choice=4 %
            }%
          }{%
            \chardef\HOLOGO@choice=3 %
          }%
        }{%
          \chardef\HOLOGO@choice=2 %
        }%
      }{%
        \chardef\HOLOGO@choice=1 %
      }%
      \ifcase\HOLOGO@choice
        \HOLOGO@WarningUnsupportedDriver{Xe}%
        e%
      \or % 1: \rotatebox
        \begingroup
          \setbox\ltx@zero\hbox{\rotatebox{180}{E}}%
          \ltx@LocDimenA=\dp\ltx@zero
          \advance\ltx@LocDimenA by -.5ex\relax
          \raise\ltx@LocDimenA\box\ltx@zero
        \endgroup
      \or % 2: \pgftext
        \lower.5ex\hbox{%
          \pgfpicture
            \pgftext[rotate=180]{E}%
          \endpgfpicture
        }%
      \or % 3: \psscalebox
        \begingroup
          \setbox\ltx@zero\hbox{\psscalebox{-1 -1}{E}}%
          \ltx@LocDimenA=\dp\ltx@zero
          \advance\ltx@LocDimenA by -.5ex\relax
          \raise\ltx@LocDimenA\box\ltx@zero
        \endgroup
      \or % 4: \HOLOGO@PointReflectBox
        \lower.5ex\hbox{\HOLOGO@PointReflectBox{E}}%
      \else
        \@PackageError{hologo}{Internal error (choice/it}\@ehc
      \fi
    \else
      \ltx@IfUndefined{reflectbox}{%
        \ltx@IfUndefined{pgftext}{%
          \ltx@IfUndefined{psscalebox}{%
            \ltx@IfUndefined{HOLOGO@ScaleBox@\hologoDriver}{%
            }{%
              \chardef\HOLOGO@choice=4 %
            }%
          }{%
            \chardef\HOLOGO@choice=3 %
          }%
        }{%
          \chardef\HOLOGO@choice=2 %
        }%
      }{%
        \chardef\HOLOGO@choice=1 %
      }%
      \ifcase\HOLOGO@choice
        \HOLOGO@WarningUnsupportedDriver{Xe}%
        e%
      \or % 1: reflectbox
        \lower.5ex\hbox{%
          \reflectbox{E}%
        }%
      \or % 2: \pgftext
        \lower.5ex\hbox{%
          \pgfpicture
            \pgftransformxscale{-1}%
            \pgftext{E}%
          \endpgfpicture
        }%
      \or % 3: \psscalebox
        \lower.5ex\hbox{%
          \psscalebox{-1 1}{E}%
        }%
      \or % 4: \HOLOGO@Reflectbox
        \lower.5ex\hbox{%
          \HOLOGO@ReflectBox{E}%
        }%
      \else
        \@PackageError{hologo}{Internal error (choice/up)}\@ehc
      \fi
    \fi
  }%
}
%    \end{macrocode}
%    \end{macro}
%    \begin{macro}{\HoLogoHtml@Xe}
%    \begin{macrocode}
\def\HoLogoHtml@Xe#1{%
  \HoLogoCss@Xe
  \HOLOGO@Span{Xe}{%
    X%
    \HOLOGO@Span{e}{%
      \HCode{&\ltx@hashchar x018e;}%
    }%
  }%
}
%    \end{macrocode}
%    \end{macro}
%    \begin{macro}{\HoLogoCss@Xe}
%    \begin{macrocode}
\def\HoLogoCss@Xe{%
  \Css{%
    span.HoLogo-Xe span.HoLogo-e{%
      position:relative;%
      top:.5ex;%
      left-margin:-.1em;%
    }%
  }%
  \global\let\HoLogoCss@Xe\relax
}
%    \end{macrocode}
%    \end{macro}
%
%    \begin{macro}{\HoLogo@XeTeX}
%    \begin{macrocode}
\def\HoLogo@XeTeX#1{%
  \hologo{Xe}%
  \kern-.15em\relax
  \hologo{TeX}%
}
%    \end{macrocode}
%    \end{macro}
%
%    \begin{macro}{\HoLogoHtml@XeTeX}
%    \begin{macrocode}
\def\HoLogoHtml@XeTeX#1{%
  \HoLogoCss@XeTeX
  \HOLOGO@Span{XeTeX}{%
    \hologo{Xe}%
    \hologo{TeX}%
  }%
}
%    \end{macrocode}
%    \end{macro}
%    \begin{macro}{\HoLogoCss@XeTeX}
%    \begin{macrocode}
\def\HoLogoCss@XeTeX{%
  \Css{%
    span.HoLogo-XeTeX span.HoLogo-TeX{%
      margin-left:-.15em;%
    }%
  }%
  \global\let\HoLogoCss@XeTeX\relax
}
%    \end{macrocode}
%    \end{macro}
%
%    \begin{macro}{\HoLogo@XeLaTeX}
%    \begin{macrocode}
\def\HoLogo@XeLaTeX#1{%
  \hologo{Xe}%
  \kern-.13em%
  \hologo{LaTeX}%
}
%    \end{macrocode}
%    \end{macro}
%    \begin{macro}{\HoLogoHtml@XeLaTeX}
%    \begin{macrocode}
\def\HoLogoHtml@XeLaTeX#1{%
  \HoLogoCss@XeLaTeX
  \HOLOGO@Span{XeLaTeX}{%
    \hologo{Xe}%
    \hologo{LaTeX}%
  }%
}
%    \end{macrocode}
%    \end{macro}
%    \begin{macro}{\HoLogoCss@XeLaTeX}
%    \begin{macrocode}
\def\HoLogoCss@XeLaTeX{%
  \Css{%
    span.HoLogo-XeLaTeX span.HoLogo-Xe{%
      margin-right:-.13em;%
    }%
  }%
  \global\let\HoLogoCss@XeLaTeX\relax
}
%    \end{macrocode}
%    \end{macro}
%
% \subsubsection{\hologo{pdfTeX}, \hologo{pdfLaTeX}}
%
%    \begin{macro}{\HoLogo@pdfTeX}
%    \begin{macrocode}
\def\HoLogo@pdfTeX#1{%
  \HOLOGO@mbox{%
    #1{p}{P}df\hologo{TeX}%
  }%
}
%    \end{macrocode}
%    \end{macro}
%    \begin{macro}{\HoLogoCs@pdfTeX}
%    \begin{macrocode}
\def\HoLogoCs@pdfTeX#1{#1{p}{P}dfTeX}
%    \end{macrocode}
%    \end{macro}
%    \begin{macro}{\HoLogoBkm@pdfTeX}
%    \begin{macrocode}
\def\HoLogoBkm@pdfTeX#1{%
  #1{p}{P}df\hologo{TeX}%
}
%    \end{macrocode}
%    \end{macro}
%    \begin{macro}{\HoLogoHtml@pdfTeX}
%    \begin{macrocode}
\let\HoLogoHtml@pdfTeX\HoLogo@pdfTeX
%    \end{macrocode}
%    \end{macro}
%
%    \begin{macro}{\HoLogo@pdfLaTeX}
%    \begin{macrocode}
\def\HoLogo@pdfLaTeX#1{%
  \HOLOGO@mbox{%
    #1{p}{P}df\hologo{LaTeX}%
  }%
}
%    \end{macrocode}
%    \end{macro}
%    \begin{macro}{\HoLogoCs@pdfLaTeX}
%    \begin{macrocode}
\def\HoLogoCs@pdfLaTeX#1{#1{p}{P}dfLaTeX}
%    \end{macrocode}
%    \end{macro}
%    \begin{macro}{\HoLogoBkm@pdfLaTeX}
%    \begin{macrocode}
\def\HoLogoBkm@pdfLaTeX#1{%
  #1{p}{P}df\hologo{LaTeX}%
}
%    \end{macrocode}
%    \end{macro}
%    \begin{macro}{\HoLogoHtml@pdfLaTeX}
%    \begin{macrocode}
\let\HoLogoHtml@pdfLaTeX\HoLogo@pdfLaTeX
%    \end{macrocode}
%    \end{macro}
%
% \subsubsection{\hologo{VTeX}}
%
%    \begin{macro}{\HoLogo@VTeX}
%    \begin{macrocode}
\def\HoLogo@VTeX#1{%
  \HOLOGO@mbox{%
    V\hologo{TeX}%
  }%
}
%    \end{macrocode}
%    \end{macro}
%    \begin{macro}{\HoLogoHtml@VTeX}
%    \begin{macrocode}
\let\HoLogoHtml@VTeX\HoLogo@VTeX
%    \end{macrocode}
%    \end{macro}
%
% \subsubsection{\hologo{AmS}, \dots}
%
%    Source: class \xclass{amsdtx}
%
%    \begin{macro}{\HoLogo@AmS}
%    \begin{macrocode}
\def\HoLogo@AmS#1{%
  \HoLogoFont@font{AmS}{sy}{%
    A%
    \kern-.1667em%
    \lower.5ex\hbox{M}%
    \kern-.125em%
    S%
  }%
}
%    \end{macrocode}
%    \end{macro}
%    \begin{macro}{\HoLogoBkm@AmS}
%    \begin{macrocode}
\def\HoLogoBkm@AmS#1{AmS}
%    \end{macrocode}
%    \end{macro}
%    \begin{macro}{\HoLogoHtml@AmS}
%    \begin{macrocode}
\def\HoLogoHtml@AmS#1{%
  \HoLogoCss@AmS
%  \HoLogoFont@font{AmS}{sy}{%
    \HOLOGO@Span{AmS}{%
      A%
      \HOLOGO@Span{M}{M}%
      S%
    }%
%   }%
}
%    \end{macrocode}
%    \end{macro}
%    \begin{macro}{\HoLogoCss@AmS}
%    \begin{macrocode}
\def\HoLogoCss@AmS{%
  \Css{%
    span.HoLogo-AmS span.HoLogo-M{%
      position:relative;%
      top:.5ex;%
      margin-left:-.1667em;%
      margin-right:-.125em;%
      text-decoration:none;%
    }%
  }%
  \global\let\HoLogoCss@AmS\relax
}
%    \end{macrocode}
%    \end{macro}
%
%    \begin{macro}{\HoLogo@AmSTeX}
%    \begin{macrocode}
\def\HoLogo@AmSTeX#1{%
  \hologo{AmS}%
  \HOLOGO@hyphen
  \hologo{TeX}%
}
%    \end{macrocode}
%    \end{macro}
%    \begin{macro}{\HoLogoBkm@AmSTeX}
%    \begin{macrocode}
\def\HoLogoBkm@AmSTeX#1{AmS-TeX}%
%    \end{macrocode}
%    \end{macro}
%    \begin{macro}{\HoLogoHtml@AmSTeX}
%    \begin{macrocode}
\let\HoLogoHtml@AmSTeX\HoLogo@AmSTeX
%    \end{macrocode}
%    \end{macro}
%
%    \begin{macro}{\HoLogo@AmSLaTeX}
%    \begin{macrocode}
\def\HoLogo@AmSLaTeX#1{%
  \hologo{AmS}%
  \HOLOGO@hyphen
  \hologo{LaTeX}%
}
%    \end{macrocode}
%    \end{macro}
%    \begin{macro}{\HoLogoBkm@AmSLaTeX}
%    \begin{macrocode}
\def\HoLogoBkm@AmSLaTeX#1{AmS-LaTeX}%
%    \end{macrocode}
%    \end{macro}
%    \begin{macro}{\HoLogoHtml@AmSLaTeX}
%    \begin{macrocode}
\let\HoLogoHtml@AmSLaTeX\HoLogo@AmSLaTeX
%    \end{macrocode}
%    \end{macro}
%
% \subsubsection{\hologo{BibTeX}}
%
%    \begin{macro}{\HoLogo@BibTeX@sc}
%    A definition of \hologo{BibTeX} is provided in
%    the documentation source for the manual of \hologo{BibTeX}
%    \cite{btxdoc}.
%\begin{quote}
%\begin{verbatim}
%\def\BibTeX{%
%  {%
%    \rm
%    B%
%    \kern-.05em%
%    {%
%      \sc
%      i%
%      \kern-.025em %
%      b%
%    }%
%    \kern-.08em
%    T%
%    \kern-.1667em%
%    \lower.7ex\hbox{E}%
%    \kern-.125em%
%    X%
%  }%
%}
%\end{verbatim}
%\end{quote}
%    \begin{macrocode}
\def\HoLogo@BibTeX@sc#1{%
  B%
  \kern-.05em%
  \HoLogoFont@font{BibTeX}{sc}{%
    i%
    \kern-.025em%
    b%
  }%
  \HOLOGO@discretionary
  \kern-.08em%
  \hologo{TeX}%
}
%    \end{macrocode}
%    \end{macro}
%    \begin{macro}{\HoLogoHtml@BibTeX@sc}
%    \begin{macrocode}
\def\HoLogoHtml@BibTeX@sc#1{%
  \HoLogoCss@BibTeX@sc
  \HOLOGO@Span{BibTeX-sc}{%
    B%
    \HOLOGO@Span{i}{i}%
    \HOLOGO@Span{b}{b}%
    \hologo{TeX}%
  }%
}
%    \end{macrocode}
%    \end{macro}
%    \begin{macro}{\HoLogoCss@BibTeX@sc}
%    \begin{macrocode}
\def\HoLogoCss@BibTeX@sc{%
  \Css{%
    span.HoLogo-BibTeX-sc span.HoLogo-i{%
      margin-left:-.05em;%
      margin-right:-.025em;%
      font-variant:small-caps;%
    }%
  }%
  \Css{%
    span.HoLogo-BibTeX-sc span.HoLogo-b{%
      margin-right:-.08em;%
      font-variant:small-caps;%
    }%
  }%
  \global\let\HoLogoCss@BibTeX@sc\relax
}
%    \end{macrocode}
%    \end{macro}
%
%    \begin{macro}{\HoLogo@BibTeX@sf}
%    Variant \xoption{sf} avoids trouble with unavailable
%    small caps fonts (e.g., bold versions of Computer Modern or
%    Latin Modern). The definition is taken from
%    package \xpackage{dtklogos} \cite{dtklogos}.
%\begin{quote}
%\begin{verbatim}
%\DeclareRobustCommand{\BibTeX}{%
%  B%
%  \kern-.05em%
%  \hbox{%
%    $\m@th$% %% force math size calculations
%    \csname S@\f@size\endcsname
%    \fontsize\sf@size\z@
%    \math@fontsfalse
%    \selectfont
%    I%
%    \kern-.025em%
%    B
%  }%
%  \kern-.08em%
%  \-%
%  \TeX
%}
%\end{verbatim}
%\end{quote}
%    \begin{macrocode}
\def\HoLogo@BibTeX@sf#1{%
  B%
  \kern-.05em%
  \HoLogoFont@font{BibTeX}{bibsf}{%
    I%
    \kern-.025em%
    B%
  }%
  \HOLOGO@discretionary
  \kern-.08em%
  \hologo{TeX}%
}
%    \end{macrocode}
%    \end{macro}
%    \begin{macro}{\HoLogoHtml@BibTeX@sf}
%    \begin{macrocode}
\def\HoLogoHtml@BibTeX@sf#1{%
  \HoLogoCss@BibTeX@sf
  \HOLOGO@Span{BibTeX-sf}{%
    B%
    \HoLogoFont@font{BibTeX}{bibsf}{%
      \HOLOGO@Span{i}{I}%
      B%
    }%
    \hologo{TeX}%
  }%
}
%    \end{macrocode}
%    \end{macro}
%    \begin{macro}{\HoLogoCss@BibTeX@sf}
%    \begin{macrocode}
\def\HoLogoCss@BibTeX@sf{%
  \Css{%
    span.HoLogo-BibTeX-sf span.HoLogo-i{%
      margin-left:-.05em;%
      margin-right:-.025em;%
    }%
  }%
  \Css{%
    span.HoLogo-BibTeX-sf span.HoLogo-TeX{%
      margin-left:-.08em;%
    }%
  }%
  \global\let\HoLogoCss@BibTeX@sf\relax
}
%    \end{macrocode}
%    \end{macro}
%
%    \begin{macro}{\HoLogo@BibTeX}
%    \begin{macrocode}
\def\HoLogo@BibTeX{\HoLogo@BibTeX@sf}
%    \end{macrocode}
%    \end{macro}
%    \begin{macro}{\HoLogoHtml@BibTeX}
%    \begin{macrocode}
\def\HoLogoHtml@BibTeX{\HoLogoHtml@BibTeX@sf}
%    \end{macrocode}
%    \end{macro}
%
% \subsubsection{\hologo{BibTeX8}}
%
%    \begin{macro}{\HoLogo@BibTeX8}
%    \begin{macrocode}
\expandafter\def\csname HoLogo@BibTeX8\endcsname#1{%
  \hologo{BibTeX}%
  8%
}
%    \end{macrocode}
%    \end{macro}
%
%    \begin{macro}{\HoLogoBkm@BibTeX8}
%    \begin{macrocode}
\expandafter\def\csname HoLogoBkm@BibTeX8\endcsname#1{%
  \hologo{BibTeX}%
  8%
}
%    \end{macrocode}
%    \end{macro}
%    \begin{macro}{\HoLogoHtml@BibTeX8}
%    \begin{macrocode}
\expandafter
\let\csname HoLogoHtml@BibTeX8\expandafter\endcsname
\csname HoLogo@BibTeX8\endcsname
%    \end{macrocode}
%    \end{macro}
%
% \subsubsection{\hologo{ConTeXt}}
%
%    \begin{macro}{\HoLogo@ConTeXt@simple}
%    \begin{macrocode}
\def\HoLogo@ConTeXt@simple#1{%
  \HOLOGO@mbox{Con}%
  \HOLOGO@discretionary
  \HOLOGO@mbox{\hologo{TeX}t}%
}
%    \end{macrocode}
%    \end{macro}
%    \begin{macro}{\HoLogoHtml@ConTeXt@simple}
%    \begin{macrocode}
\let\HoLogoHtml@ConTeXt@simple\HoLogo@ConTeXt@simple
%    \end{macrocode}
%    \end{macro}
%
%    \begin{macro}{\HoLogo@ConTeXt@narrow}
%    This definition of logo \hologo{ConTeXt} with variant \xoption{narrow}
%    comes from TUGboat's class \xclass{ltugboat} (version 2010/11/15 v2.8).
%    \begin{macrocode}
\def\HoLogo@ConTeXt@narrow#1{%
  \HOLOGO@mbox{C\kern-.0333emon}%
  \HOLOGO@discretionary
  \kern-.0667em%
  \HOLOGO@mbox{\hologo{TeX}\kern-.0333emt}%
}
%    \end{macrocode}
%    \end{macro}
%    \begin{macro}{\HoLogoHtml@ConTeXt@narrow}
%    \begin{macrocode}
\def\HoLogoHtml@ConTeXt@narrow#1{%
  \HoLogoCss@ConTeXt@narrow
  \HOLOGO@Span{ConTeXt-narrow}{%
    \HOLOGO@Span{C}{C}%
    on%
    \hologo{TeX}%
    t%
  }%
}
%    \end{macrocode}
%    \end{macro}
%    \begin{macro}{\HoLogoCss@ConTeXt@narrow}
%    \begin{macrocode}
\def\HoLogoCss@ConTeXt@narrow{%
  \Css{%
    span.HoLogo-ConTeXt-narrow span.HoLogo-C{%
      margin-left:-.0333em;%
    }%
  }%
  \Css{%
    span.HoLogo-ConTeXt-narrow span.HoLogo-TeX{%
      margin-left:-.0667em;%
      margin-right:-.0333em;%
    }%
  }%
  \global\let\HoLogoCss@ConTeXt@narrow\relax
}
%    \end{macrocode}
%    \end{macro}
%
%    \begin{macro}{\HoLogo@ConTeXt}
%    \begin{macrocode}
\def\HoLogo@ConTeXt{\HoLogo@ConTeXt@narrow}
%    \end{macrocode}
%    \end{macro}
%    \begin{macro}{\HoLogoHtml@ConTeXt}
%    \begin{macrocode}
\def\HoLogoHtml@ConTeXt{\HoLogoHtml@ConTeXt@narrow}
%    \end{macrocode}
%    \end{macro}
%
% \subsubsection{\hologo{emTeX}}
%
%    \begin{macro}{\HoLogo@emTeX}
%    \begin{macrocode}
\def\HoLogo@emTeX#1{%
  \HOLOGO@mbox{#1{e}{E}m}%
  \HOLOGO@discretionary
  \hologo{TeX}%
}
%    \end{macrocode}
%    \end{macro}
%    \begin{macro}{\HoLogoCs@emTeX}
%    \begin{macrocode}
\def\HoLogoCs@emTeX#1{#1{e}{E}mTeX}%
%    \end{macrocode}
%    \end{macro}
%    \begin{macro}{\HoLogoBkm@emTeX}
%    \begin{macrocode}
\def\HoLogoBkm@emTeX#1{%
  #1{e}{E}m\hologo{TeX}%
}
%    \end{macrocode}
%    \end{macro}
%    \begin{macro}{\HoLogoHtml@emTeX}
%    \begin{macrocode}
\let\HoLogoHtml@emTeX\HoLogo@emTeX
%    \end{macrocode}
%    \end{macro}
%
% \subsubsection{\hologo{ExTeX}}
%
%    \begin{macro}{\HoLogo@ExTeX}
%    The definition is taken from the FAQ of the
%    project \hologo{ExTeX}
%    \cite{ExTeX-FAQ}.
%\begin{quote}
%\begin{verbatim}
%\def\ExTeX{%
%  \textrm{% Logo always with serifs
%    \ensuremath{%
%      \textstyle
%      \varepsilon_{%
%        \kern-0.15em%
%        \mathcal{X}%
%      }%
%    }%
%    \kern-.15em%
%    \TeX
%  }%
%}
%\end{verbatim}
%\end{quote}
%    \begin{macrocode}
\def\HoLogo@ExTeX#1{%
  \HoLogoFont@font{ExTeX}{rm}{%
    \ltx@mbox{%
      \HOLOGO@MathSetup
      $%
        \textstyle
        \varepsilon_{%
          \kern-0.15em%
          \HoLogoFont@font{ExTeX}{sy}{X}%
        }%
      $%
    }%
    \HOLOGO@discretionary
    \kern-.15em%
    \hologo{TeX}%
  }%
}
%    \end{macrocode}
%    \end{macro}
%    \begin{macro}{\HoLogoHtml@ExTeX}
%    \begin{macrocode}
\def\HoLogoHtml@ExTeX#1{%
  \HoLogoCss@ExTeX
  \HoLogoFont@font{ExTeX}{rm}{%
    \HOLOGO@Span{ExTeX}{%
      \ltx@mbox{%
        \HOLOGO@MathSetup
        $\textstyle\varepsilon$%
        \HOLOGO@Span{X}{$\textstyle\chi$}%
        \hologo{TeX}%
      }%
    }%
  }%
}
%    \end{macrocode}
%    \end{macro}
%    \begin{macro}{\HoLogoBkm@ExTeX}
%    \begin{macrocode}
\def\HoLogoBkm@ExTeX#1{%
  \HOLOGO@PdfdocUnicode{#1{e}{E}x}{\textepsilon\textchi}%
  \hologo{TeX}%
}
%    \end{macrocode}
%    \end{macro}
%    \begin{macro}{\HoLogoCss@ExTeX}
%    \begin{macrocode}
\def\HoLogoCss@ExTeX{%
  \Css{%
    span.HoLogo-ExTeX{%
      font-family:serif;%
    }%
  }%
  \Css{%
    span.HoLogo-ExTeX span.HoLogo-TeX{%
      margin-left:-.15em;%
    }%
  }%
  \global\let\HoLogoCss@ExTeX\relax
}
%    \end{macrocode}
%    \end{macro}
%
% \subsubsection{\hologo{MiKTeX}}
%
%    \begin{macro}{\HoLogo@MiKTeX}
%    \begin{macrocode}
\def\HoLogo@MiKTeX#1{%
  \HOLOGO@mbox{MiK}%
  \HOLOGO@discretionary
  \hologo{TeX}%
}
%    \end{macrocode}
%    \end{macro}
%    \begin{macro}{\HoLogoHtml@MiKTeX}
%    \begin{macrocode}
\let\HoLogoHtml@MiKTeX\HoLogo@MiKTeX
%    \end{macrocode}
%    \end{macro}
%
% \subsubsection{\hologo{OzTeX} and friends}
%
%    Source: \hologo{OzTeX} FAQ \cite{OzTeX}:
%    \begin{quote}
%      |\def\OzTeX{O\kern-.03em z\kern-.15em\TeX}|\\
%      (There is no kerning in OzMF, OzMP and OzTtH.)
%    \end{quote}
%
%    \begin{macro}{\HoLogo@OzTeX}
%    \begin{macrocode}
\def\HoLogo@OzTeX#1{%
  O%
  \kern-.03em %
  z%
  \kern-.15em %
  \hologo{TeX}%
}
%    \end{macrocode}
%    \end{macro}
%    \begin{macro}{\HoLogoHtml@OzTeX}
%    \begin{macrocode}
\def\HoLogoHtml@OzTeX#1{%
  \HoLogoCss@OzTeX
  \HOLOGO@Span{OzTeX}{%
    O%
    \HOLOGO@Span{z}{z}%
    \hologo{TeX}%
  }%
}
%    \end{macrocode}
%    \end{macro}
%    \begin{macro}{\HoLogoCss@OzTeX}
%    \begin{macrocode}
\def\HoLogoCss@OzTeX{%
  \Css{%
    span.HoLogo-OzTeX span.HoLogo-z{%
      margin-left:-.03em;%
      margin-right:-.15em;%
    }%
  }%
  \global\let\HoLogoCss@OzTeX\relax
}
%    \end{macrocode}
%    \end{macro}
%
%    \begin{macro}{\HoLogo@OzMF}
%    \begin{macrocode}
\def\HoLogo@OzMF#1{%
  \HOLOGO@mbox{OzMF}%
}
%    \end{macrocode}
%    \end{macro}
%    \begin{macro}{\HoLogo@OzMP}
%    \begin{macrocode}
\def\HoLogo@OzMP#1{%
  \HOLOGO@mbox{OzMP}%
}
%    \end{macrocode}
%    \end{macro}
%    \begin{macro}{\HoLogo@OzTtH}
%    \begin{macrocode}
\def\HoLogo@OzTtH#1{%
  \HOLOGO@mbox{OzTtH}%
}
%    \end{macrocode}
%    \end{macro}
%
% \subsubsection{\hologo{PCTeX}}
%
%    \begin{macro}{\HoLogo@PCTeX}
%    \begin{macrocode}
\def\HoLogo@PCTeX#1{%
  \HOLOGO@mbox{PC}%
  \hologo{TeX}%
}
%    \end{macrocode}
%    \end{macro}
%    \begin{macro}{\HoLogoHtml@PCTeX}
%    \begin{macrocode}
\let\HoLogoHtml@PCTeX\HoLogo@PCTeX
%    \end{macrocode}
%    \end{macro}
%
% \subsubsection{\hologo{PiCTeX}}
%
%    The original definitions from \xfile{pictex.tex} \cite{PiCTeX}:
%\begin{quote}
%\begin{verbatim}
%\def\PiC{%
%  P%
%  \kern-.12em%
%  \lower.5ex\hbox{I}%
%  \kern-.075em%
%  C%
%}
%\def\PiCTeX{%
%  \PiC
%  \kern-.11em%
%  \TeX
%}
%\end{verbatim}
%\end{quote}
%
%    \begin{macro}{\HoLogo@PiC}
%    \begin{macrocode}
\def\HoLogo@PiC#1{%
  P%
  \kern-.12em%
  \lower.5ex\hbox{I}%
  \kern-.075em%
  C%
  \HOLOGO@SpaceFactor
}
%    \end{macrocode}
%    \end{macro}
%    \begin{macro}{\HoLogoHtml@PiC}
%    \begin{macrocode}
\def\HoLogoHtml@PiC#1{%
  \HoLogoCss@PiC
  \HOLOGO@Span{PiC}{%
    P%
    \HOLOGO@Span{i}{I}%
    C%
  }%
}
%    \end{macrocode}
%    \end{macro}
%    \begin{macro}{\HoLogoCss@PiC}
%    \begin{macrocode}
\def\HoLogoCss@PiC{%
  \Css{%
    span.HoLogo-PiC span.HoLogo-i{%
      position:relative;%
      top:.5ex;%
      margin-left:-.12em;%
      margin-right:-.075em;%
      text-decoration:none;%
    }%
  }%
  \global\let\HoLogoCss@PiC\relax
}
%    \end{macrocode}
%    \end{macro}
%
%    \begin{macro}{\HoLogo@PiCTeX}
%    \begin{macrocode}
\def\HoLogo@PiCTeX#1{%
  \hologo{PiC}%
  \HOLOGO@discretionary
  \kern-.11em%
  \hologo{TeX}%
}
%    \end{macrocode}
%    \end{macro}
%    \begin{macro}{\HoLogoHtml@PiCTeX}
%    \begin{macrocode}
\def\HoLogoHtml@PiCTeX#1{%
  \HoLogoCss@PiCTeX
  \HOLOGO@Span{PiCTeX}{%
    \hologo{PiC}%
    \hologo{TeX}%
  }%
}
%    \end{macrocode}
%    \end{macro}
%    \begin{macro}{\HoLogoCss@PiCTeX}
%    \begin{macrocode}
\def\HoLogoCss@PiCTeX{%
  \Css{%
    span.HoLogo-PiCTeX span.HoLogo-PiC{%
      margin-right:-.11em;%
    }%
  }%
  \global\let\HoLogoCss@PiCTeX\relax
}
%    \end{macrocode}
%    \end{macro}
%
% \subsubsection{\hologo{teTeX}}
%
%    \begin{macro}{\HoLogo@teTeX}
%    \begin{macrocode}
\def\HoLogo@teTeX#1{%
  \HOLOGO@mbox{#1{t}{T}e}%
  \HOLOGO@discretionary
  \hologo{TeX}%
}
%    \end{macrocode}
%    \end{macro}
%    \begin{macro}{\HoLogoCs@teTeX}
%    \begin{macrocode}
\def\HoLogoCs@teTeX#1{#1{t}{T}dfTeX}
%    \end{macrocode}
%    \end{macro}
%    \begin{macro}{\HoLogoBkm@teTeX}
%    \begin{macrocode}
\def\HoLogoBkm@teTeX#1{%
  #1{t}{T}e\hologo{TeX}%
}
%    \end{macrocode}
%    \end{macro}
%    \begin{macro}{\HoLogoHtml@teTeX}
%    \begin{macrocode}
\let\HoLogoHtml@teTeX\HoLogo@teTeX
%    \end{macrocode}
%    \end{macro}
%
% \subsubsection{\hologo{TeX4ht}}
%
%    \begin{macro}{\HoLogo@TeX4ht}
%    \begin{macrocode}
\expandafter\def\csname HoLogo@TeX4ht\endcsname#1{%
  \HOLOGO@mbox{\hologo{TeX}4ht}%
}
%    \end{macrocode}
%    \end{macro}
%    \begin{macro}{\HoLogoHtml@TeX4ht}
%    \begin{macrocode}
\expandafter
\let\csname HoLogoHtml@TeX4ht\expandafter\endcsname
\csname HoLogo@TeX4ht\endcsname
%    \end{macrocode}
%    \end{macro}
%
%
% \subsubsection{\hologo{SageTeX}}
%
%    \begin{macro}{\HoLogo@SageTeX}
%    \begin{macrocode}
\def\HoLogo@SageTeX#1{%
  \HOLOGO@mbox{Sage}%
  \HOLOGO@discretionary
  \HOLOGO@NegativeKerning{eT,oT,To}%
  \hologo{TeX}%
}
%    \end{macrocode}
%    \end{macro}
%    \begin{macro}{\HoLogoHtml@SageTeX}
%    \begin{macrocode}
\let\HoLogoHtml@SageTeX\HoLogo@SageTeX
%    \end{macrocode}
%    \end{macro}
%
% \subsection{\hologo{METAFONT} and friends}
%
%    \begin{macro}{\HoLogo@METAFONT}
%    \begin{macrocode}
\def\HoLogo@METAFONT#1{%
  \HoLogoFont@font{METAFONT}{logo}{%
    \HOLOGO@mbox{META}%
    \HOLOGO@discretionary
    \HOLOGO@mbox{FONT}%
  }%
}
%    \end{macrocode}
%    \end{macro}
%
%    \begin{macro}{\HoLogo@METAPOST}
%    \begin{macrocode}
\def\HoLogo@METAPOST#1{%
  \HoLogoFont@font{METAPOST}{logo}{%
    \HOLOGO@mbox{META}%
    \HOLOGO@discretionary
    \HOLOGO@mbox{POST}%
  }%
}
%    \end{macrocode}
%    \end{macro}
%
%    \begin{macro}{\HoLogo@MetaFun}
%    \begin{macrocode}
\def\HoLogo@MetaFun#1{%
  \HOLOGO@mbox{Meta}%
  \HOLOGO@discretionary
  \HOLOGO@mbox{Fun}%
}
%    \end{macrocode}
%    \end{macro}
%
%    \begin{macro}{\HoLogo@MetaPost}
%    \begin{macrocode}
\def\HoLogo@MetaPost#1{%
  \HOLOGO@mbox{Meta}%
  \HOLOGO@discretionary
  \HOLOGO@mbox{Post}%
}
%    \end{macrocode}
%    \end{macro}
%
% \subsection{Others}
%
% \subsubsection{\hologo{biber}}
%
%    \begin{macro}{\HoLogo@biber}
%    \begin{macrocode}
\def\HoLogo@biber#1{%
  \HOLOGO@mbox{#1{b}{B}i}%
  \HOLOGO@discretionary
  \HOLOGO@mbox{ber}%
}
%    \end{macrocode}
%    \end{macro}
%    \begin{macro}{\HoLogoCs@biber}
%    \begin{macrocode}
\def\HoLogoCs@biber#1{#1{b}{B}iber}
%    \end{macrocode}
%    \end{macro}
%    \begin{macro}{\HoLogoBkm@biber}
%    \begin{macrocode}
\def\HoLogoBkm@biber#1{%
  #1{b}{B}iber%
}
%    \end{macrocode}
%    \end{macro}
%    \begin{macro}{\HoLogoHtml@biber}
%    \begin{macrocode}
\let\HoLogoHtml@biber\HoLogo@biber
%    \end{macrocode}
%    \end{macro}
%
% \subsubsection{\hologo{KOMAScript}}
%
%    \begin{macro}{\HoLogo@KOMAScript}
%    The definition for \hologo{KOMAScript} is taken
%    from \hologo{KOMAScript} (\xfile{scrlogo.dtx}, reformatted) \cite{scrlogo}:
%\begin{quote}
%\begin{verbatim}
%\@ifundefined{KOMAScript}{%
%  \DeclareRobustCommand{\KOMAScript}{%
%    \textsf{%
%      K\kern.05em O\kern.05emM\kern.05em A%
%      \kern.1em-\kern.1em %
%      Script%
%    }%
%  }%
%}{}
%\end{verbatim}
%\end{quote}
%    \begin{macrocode}
\def\HoLogo@KOMAScript#1{%
  \HoLogoFont@font{KOMAScript}{sf}{%
    \HOLOGO@mbox{%
      K\kern.05em%
      O\kern.05em%
      M\kern.05em%
      A%
    }%
    \kern.1em%
    \HOLOGO@hyphen
    \kern.1em%
    \HOLOGO@mbox{Script}%
  }%
}
%    \end{macrocode}
%    \end{macro}
%    \begin{macro}{\HoLogoBkm@KOMAScript}
%    \begin{macrocode}
\def\HoLogoBkm@KOMAScript#1{%
  KOMA-Script%
}
%    \end{macrocode}
%    \end{macro}
%    \begin{macro}{\HoLogoHtml@KOMAScript}
%    \begin{macrocode}
\def\HoLogoHtml@KOMAScript#1{%
  \HoLogoCss@KOMAScript
  \HoLogoFont@font{KOMAScript}{sf}{%
    \HOLOGO@Span{KOMAScript}{%
      K%
      \HOLOGO@Span{O}{O}%
      M%
      \HOLOGO@Span{A}{A}%
      \HOLOGO@Span{hyphen}{-}%
      Script%
    }%
  }%
}
%    \end{macrocode}
%    \end{macro}
%    \begin{macro}{\HoLogoCss@KOMAScript}
%    \begin{macrocode}
\def\HoLogoCss@KOMAScript{%
  \Css{%
    span.HoLogo-KOMAScript{%
      font-family:sans-serif;%
    }%
  }%
  \Css{%
    span.HoLogo-KOMAScript span.HoLogo-O{%
      padding-left:.05em;%
      padding-right:.05em;%
    }%
  }%
  \Css{%
    span.HoLogo-KOMAScript span.HoLogo-A{%
      padding-left:.05em;%
    }%
  }%
  \Css{%
    span.HoLogo-KOMAScript span.HoLogo-hyphen{%
      padding-left:.1em;%
      padding-right:.1em;%
    }%
  }%
  \global\let\HoLogoCss@KOMAScript\relax
}
%    \end{macrocode}
%    \end{macro}
%
% \subsubsection{\hologo{LyX}}
%
%    \begin{macro}{\HoLogo@LyX}
%    The definition is taken from the documentation source files
%    of \hologo{LyX}, \xfile{Intro.lyx} \cite{LyX}:
%\begin{quote}
%\begin{verbatim}
%\def\LyX{%
%  \texorpdfstring{%
%    L\kern-.1667em\lower.25em\hbox{Y}\kern-.125emX\@%
%  }{%
%    LyX%
%  }%
%}
%\end{verbatim}
%\end{quote}
%    \begin{macrocode}
\def\HoLogo@LyX#1{%
  L%
  \kern-.1667em%
  \lower.25em\hbox{Y}%
  \kern-.125em%
  X%
  \HOLOGO@SpaceFactor
}
%    \end{macrocode}
%    \end{macro}
%    \begin{macro}{\HoLogoHtml@LyX}
%    \begin{macrocode}
\def\HoLogoHtml@LyX#1{%
  \HoLogoCss@LyX
  \HOLOGO@Span{LyX}{%
    L%
    \HOLOGO@Span{y}{Y}%
    X%
  }%
}
%    \end{macrocode}
%    \end{macro}
%    \begin{macro}{\HoLogoCss@LyX}
%    \begin{macrocode}
\def\HoLogoCss@LyX{%
  \Css{%
    span.HoLogo-LyX span.HoLogo-y{%
      position:relative;%
      top:.25em;%
      margin-left:-.1667em;%
      margin-right:-.125em;%
      text-decoration:none;%
    }%
  }%
  \global\let\HoLogoCss@LyX\relax
}
%    \end{macrocode}
%    \end{macro}
%
% \subsubsection{\hologo{NTS}}
%
%    \begin{macro}{\HoLogo@NTS}
%    Definition for \hologo{NTS} can be found in
%    package \xpackage{etex\textunderscore man} for the \hologo{eTeX} manual \cite{etexman}
%    and in package \xpackage{dtklogos} \cite{dtklogos}:
%\begin{quote}
%\begin{verbatim}
%\def\NTS{%
%  \leavevmode
%  \hbox{%
%    $%
%      \cal N%
%      \kern-0.35em%
%      \lower0.5ex\hbox{$\cal T$}%
%      \kern-0.2em%
%      S%
%    $%
%  }%
%}
%\end{verbatim}
%\end{quote}
%    \begin{macrocode}
\def\HoLogo@NTS#1{%
  \HoLogoFont@font{NTS}{sy}{%
    N\/%
    \kern-.35em%
    \lower.5ex\hbox{T\/}%
    \kern-.2em%
    S\/%
  }%
  \HOLOGO@SpaceFactor
}
%    \end{macrocode}
%    \end{macro}
%
% \subsubsection{\Hologo{TTH} (\hologo{TeX} to HTML translator)}
%
%    Source: \url{http://hutchinson.belmont.ma.us/tth/}
%    In the HTML source the second `T' is printed as subscript.
%\begin{quote}
%\begin{verbatim}
%T<sub>T</sub>H
%\end{verbatim}
%\end{quote}
%    \begin{macro}{\HoLogo@TTH}
%    \begin{macrocode}
\def\HoLogo@TTH#1{%
  \ltx@mbox{%
    T\HOLOGO@SubScript{T}H%
  }%
  \HOLOGO@SpaceFactor
}
%    \end{macrocode}
%    \end{macro}
%
%    \begin{macro}{\HoLogoHtml@TTH}
%    \begin{macrocode}
\def\HoLogoHtml@TTH#1{%
  T\HCode{<sub>}T\HCode{</sub>}H%
}
%    \end{macrocode}
%    \end{macro}
%
% \subsubsection{\Hologo{HanTheThanh}}
%
%    Partial source: Package \xpackage{dtklogos}.
%    The double accent is U+1EBF (latin small letter e with circumflex
%    and acute).
%    \begin{macro}{\HoLogo@HanTheThanh}
%    \begin{macrocode}
\def\HoLogo@HanTheThanh#1{%
  \ltx@mbox{H\`an}%
  \HOLOGO@space
  \ltx@mbox{%
    Th%
    \HOLOGO@IfCharExists{"1EBF}{%
      \char"1EBF\relax
    }{%
      \^e\hbox to 0pt{\hss\raise .5ex\hbox{\'{}}}%
    }%
  }%
  \HOLOGO@space
  \ltx@mbox{Th\`anh}%
}
%    \end{macrocode}
%    \end{macro}
%    \begin{macro}{\HoLogoBkm@HanTheThanh}
%    \begin{macrocode}
\def\HoLogoBkm@HanTheThanh#1{%
  H\`an %
  Th\HOLOGO@PdfdocUnicode{\^e}{\9036\277} %
  Th\`anh%
}
%    \end{macrocode}
%    \end{macro}
%    \begin{macro}{\HoLogoHtml@HanTheThanh}
%    \begin{macrocode}
\def\HoLogoHtml@HanTheThanh#1{%
  H\`an %
  Th\HCode{&\ltx@hashchar x1ebf;} %
  Th\`anh%
}
%    \end{macrocode}
%    \end{macro}
%
% \subsection{Driver detection}
%
%    \begin{macrocode}
\HOLOGO@IfExists\InputIfFileExists{%
  \InputIfFileExists{hologo.cfg}{}{}%
}{%
  \ltx@IfUndefined{pdf@filesize}{%
    \def\HOLOGO@InputIfExists{%
      \openin\HOLOGO@temp=hologo.cfg\relax
      \ifeof\HOLOGO@temp
        \closein\HOLOGO@temp
      \else
        \closein\HOLOGO@temp
        \begingroup
          \def\x{LaTeX2e}%
        \expandafter\endgroup
        \ifx\fmtname\x
          \input{hologo.cfg}%
        \else
          \input hologo.cfg\relax
        \fi
      \fi
    }%
    \ltx@IfUndefined{newread}{%
      \chardef\HOLOGO@temp=15 %
      \def\HOLOGO@CheckRead{%
        \ifeof\HOLOGO@temp
          \HOLOGO@InputIfExists
        \else
          \ifcase\HOLOGO@temp
            \@PackageWarningNoLine{hologo}{%
              Configuration file ignored, because\MessageBreak
              a free read register could not be found%
            }%
          \else
            \begingroup
              \count\ltx@cclv=\HOLOGO@temp
              \advance\ltx@cclv by \ltx@minusone
              \edef\x{\endgroup
                \chardef\noexpand\HOLOGO@temp=\the\count\ltx@cclv
                \relax
              }%
            \x
          \fi
        \fi
      }%
    }{%
      \csname newread\endcsname\HOLOGO@temp
      \HOLOGO@InputIfExists
    }%
  }{%
    \edef\HOLOGO@temp{\pdf@filesize{hologo.cfg}}%
    \ifx\HOLOGO@temp\ltx@empty
    \else
      \ifnum\HOLOGO@temp>0 %
        \begingroup
          \def\x{LaTeX2e}%
        \expandafter\endgroup
        \ifx\fmtname\x
          \input{hologo.cfg}%
        \else
          \input hologo.cfg\relax
        \fi
      \else
        \@PackageInfoNoLine{hologo}{%
          Empty configuration file `hologo.cfg' ignored%
        }%
      \fi
    \fi
  }%
}
%    \end{macrocode}
%
%    \begin{macrocode}
\def\HOLOGO@temp#1#2{%
  \kv@define@key{HoLogoDriver}{#1}[]{%
    \begingroup
      \def\HOLOGO@temp{##1}%
      \ltx@onelevel@sanitize\HOLOGO@temp
      \ifx\HOLOGO@temp\ltx@empty
      \else
        \@PackageError{hologo}{%
          Value (\HOLOGO@temp) not permitted for option `#1'%
        }%
        \@ehc
      \fi
    \endgroup
    \def\hologoDriver{#2}%
  }%
}%
\def\HOLOGO@@temp#1#2{%
  \ifx\kv@value\relax
    \HOLOGO@temp{#1}{#1}%
  \else
    \HOLOGO@temp{#1}{#2}%
  \fi
}%
\kv@parse@normalized{%
  pdftex,%
  luatex=pdftex,%
  dvipdfm,%
  dvipdfmx=dvipdfm,%
  dvips,%
  dvipsone=dvips,%
  xdvi=dvips,%
  xetex,%
  vtex,%
}\HOLOGO@@temp
%    \end{macrocode}
%
%    \begin{macrocode}
\kv@define@key{HoLogoDriver}{driverfallback}{%
  \def\HOLOGO@DriverFallback{#1}%
}
%    \end{macrocode}
%
%    \begin{macro}{\HOLOGO@DriverFallback}
%    \begin{macrocode}
\def\HOLOGO@DriverFallback{dvips}
%    \end{macrocode}
%    \end{macro}
%
%    \begin{macro}{\hologoDriverSetup}
%    \begin{macrocode}
\def\hologoDriverSetup{%
  \let\hologoDriver\ltx@undefined
  \HOLOGO@DriverSetup
}
%    \end{macrocode}
%    \end{macro}
%
%    \begin{macro}{\HOLOGO@DriverSetup}
%    \begin{macrocode}
\def\HOLOGO@DriverSetup#1{%
  \kvsetkeys{HoLogoDriver}{#1}%
  \HOLOGO@CheckDriver
  \ltx@ifundefined{hologoDriver}{%
    \begingroup
    \edef\x{\endgroup
      \noexpand\kvsetkeys{HoLogoDriver}{\HOLOGO@DriverFallback}%
    }\x
  }{}%
  \@PackageInfoNoLine{hologo}{Using driver `\hologoDriver'}%
}
%    \end{macrocode}
%    \end{macro}
%
%    \begin{macro}{\HOLOGO@CheckDriver}
%    \begin{macrocode}
\def\HOLOGO@CheckDriver{%
  \ifpdf
    \def\hologoDriver{pdftex}%
    \let\HOLOGO@pdfliteral\pdfliteral
    \ifluatex
      \ifx\pdfextension\@undefined\else
        \protected\def\pdfliteral{\pdfextension literal}%
        \let\HOLOGO@pdfliteral\pdfliteral
      \fi
      \ltx@IfUndefined{HOLOGO@pdfliteral}{%
        \ifnum\luatexversion<36 %
        \else
          \begingroup
            \let\HOLOGO@temp\endgroup
            \ifcase0%
                \directlua{%
                  if tex.enableprimitives then %
                    tex.enableprimitives('HOLOGO@', {'pdfliteral'})%
                  else %
                    tex.print('1')%
                  end%
                }%
                \ifx\HOLOGO@pdfliteral\@undefined 1\fi%
                \relax%
              \endgroup
              \let\HOLOGO@temp\relax
              \global\let\HOLOGO@pdfliteral\HOLOGO@pdfliteral
            \fi%
          \HOLOGO@temp
        \fi
      }{}%
    \fi
    \ltx@IfUndefined{HOLOGO@pdfliteral}{%
      \@PackageWarningNoLine{hologo}{%
        Cannot find \string\pdfliteral
      }%
    }{}%
  \else
    \ifxetex
      \def\hologoDriver{xetex}%
    \else
      \ifvtex
        \def\hologoDriver{vtex}%
      \fi
    \fi
  \fi
}
%    \end{macrocode}
%    \end{macro}
%
%    \begin{macro}{\HOLOGO@WarningUnsupportedDriver}
%    \begin{macrocode}
\def\HOLOGO@WarningUnsupportedDriver#1{%
  \@PackageWarningNoLine{hologo}{%
    Logo `#1' needs driver specific macros,\MessageBreak
    but driver `\hologoDriver' is not supported.\MessageBreak
    Use a different driver or\MessageBreak
    load package `graphics' or `pgf'%
  }%
}
%    \end{macrocode}
%    \end{macro}
%
% \subsubsection{Reflect box macros}
%
%    Skip driver part if not needed.
%    \begin{macrocode}
\ltx@IfUndefined{reflectbox}{}{%
  \ltx@IfUndefined{rotatebox}{}{%
    \HOLOGO@AtEnd
  }%
}
\ltx@IfUndefined{pgftext}{}{%
  \HOLOGO@AtEnd
}
\ltx@IfUndefined{psscalebox}{}{%
  \HOLOGO@AtEnd
}
%    \end{macrocode}
%
%    \begin{macrocode}
\def\HOLOGO@temp{LaTeX2e}
\ifx\fmtname\HOLOGO@temp
  \RequirePackage{kvoptions}[2011/06/30]%
  \ProcessKeyvalOptions{HoLogoDriver}%
\fi
\HOLOGO@DriverSetup{}
%    \end{macrocode}
%
%    \begin{macro}{\HOLOGO@ReflectBox}
%    \begin{macrocode}
\def\HOLOGO@ReflectBox#1{%
  \begingroup
    \setbox\ltx@zero\hbox{\begingroup#1\endgroup}%
    \setbox\ltx@two\hbox{%
      \kern\wd\ltx@zero
      \csname HOLOGO@ScaleBox@\hologoDriver\endcsname{-1}{1}{%
        \hbox to 0pt{\copy\ltx@zero\hss}%
      }%
    }%
    \wd\ltx@two=\wd\ltx@zero
    \box\ltx@two
  \endgroup
}
%    \end{macrocode}
%    \end{macro}
%
%    \begin{macro}{\HOLOGO@PointReflectBox}
%    \begin{macrocode}
\def\HOLOGO@PointReflectBox#1{%
  \begingroup
    \setbox\ltx@zero\hbox{\begingroup#1\endgroup}%
    \setbox\ltx@two\hbox{%
      \kern\wd\ltx@zero
      \raise\ht\ltx@zero\hbox{%
        \csname HOLOGO@ScaleBox@\hologoDriver\endcsname{-1}{-1}{%
          \hbox to 0pt{\copy\ltx@zero\hss}%
        }%
      }%
    }%
    \wd\ltx@two=\wd\ltx@zero
    \box\ltx@two
  \endgroup
}
%    \end{macrocode}
%    \end{macro}
%
%    We must define all variants because of dynamic driver setup.
%    \begin{macrocode}
\def\HOLOGO@temp#1#2{#2}
%    \end{macrocode}
%
%    \begin{macro}{\HOLOGO@ScaleBox@pdftex}
%    \begin{macrocode}
\HOLOGO@temp{pdftex}{%
  \def\HOLOGO@ScaleBox@pdftex#1#2#3{%
    \HOLOGO@pdfliteral{%
      q #1 0 0 #2 0 0 cm%
    }%
    #3%
    \HOLOGO@pdfliteral{%
      Q%
    }%
  }%
}
%    \end{macrocode}
%    \end{macro}
%    \begin{macro}{\HOLOGO@ScaleBox@dvips}
%    \begin{macrocode}
\HOLOGO@temp{dvips}{%
  \def\HOLOGO@ScaleBox@dvips#1#2#3{%
    \special{ps:%
      gsave %
      currentpoint %
      currentpoint translate %
      #1 #2 scale %
      neg exch neg exch translate%
    }%
    #3%
    \special{ps:%
      currentpoint %
      grestore %
      moveto%
    }%
  }%
}
%    \end{macrocode}
%    \end{macro}
%    \begin{macro}{\HOLOGO@ScaleBox@dvipdfm}
%    \begin{macrocode}
\HOLOGO@temp{dvipdfm}{%
  \let\HOLOGO@ScaleBox@dvipdfm\HOLOGO@ScaleBox@dvips
}
%    \end{macrocode}
%    \end{macro}
%    Since \hologo{XeTeX} v0.6.
%    \begin{macro}{\HOLOGO@ScaleBox@xetex}
%    \begin{macrocode}
\HOLOGO@temp{xetex}{%
  \def\HOLOGO@ScaleBox@xetex#1#2#3{%
    \special{x:gsave}%
    \special{x:scale #1 #2}%
    #3%
    \special{x:grestore}%
  }%
}
%    \end{macrocode}
%    \end{macro}
%    \begin{macro}{\HOLOGO@ScaleBox@vtex}
%    \begin{macrocode}
\HOLOGO@temp{vtex}{%
  \def\HOLOGO@ScaleBox@vtex#1#2#3{%
    \special{r(#1,0,0,#2,0,0}%
    #3%
    \special{r)}%
  }%
}
%    \end{macrocode}
%    \end{macro}
%
%    \begin{macrocode}
\HOLOGO@AtEnd%
%</package>
%    \end{macrocode}
%
% \section{Test}
%
% \subsection{Catcode checks for loading}
%
%    \begin{macrocode}
%<*test1>
%    \end{macrocode}
%    \begin{macrocode}
\catcode`\{=1 %
\catcode`\}=2 %
\catcode`\#=6 %
\catcode`\@=11 %
\expandafter\ifx\csname count@\endcsname\relax
  \countdef\count@=255 %
\fi
\expandafter\ifx\csname @gobble\endcsname\relax
  \long\def\@gobble#1{}%
\fi
\expandafter\ifx\csname @firstofone\endcsname\relax
  \long\def\@firstofone#1{#1}%
\fi
\expandafter\ifx\csname loop\endcsname\relax
  \expandafter\@firstofone
\else
  \expandafter\@gobble
\fi
{%
  \def\loop#1\repeat{%
    \def\body{#1}%
    \iterate
  }%
  \def\iterate{%
    \body
      \let\next\iterate
    \else
      \let\next\relax
    \fi
    \next
  }%
  \let\repeat=\fi
}%
\def\RestoreCatcodes{}
\count@=0 %
\loop
  \edef\RestoreCatcodes{%
    \RestoreCatcodes
    \catcode\the\count@=\the\catcode\count@\relax
  }%
\ifnum\count@<255 %
  \advance\count@ 1 %
\repeat

\def\RangeCatcodeInvalid#1#2{%
  \count@=#1\relax
  \loop
    \catcode\count@=15 %
  \ifnum\count@<#2\relax
    \advance\count@ 1 %
  \repeat
}
\def\RangeCatcodeCheck#1#2#3{%
  \count@=#1\relax
  \loop
    \ifnum#3=\catcode\count@
    \else
      \errmessage{%
        Character \the\count@\space
        with wrong catcode \the\catcode\count@\space
        instead of \number#3%
      }%
    \fi
  \ifnum\count@<#2\relax
    \advance\count@ 1 %
  \repeat
}
\def\space{ }
\expandafter\ifx\csname LoadCommand\endcsname\relax
  \def\LoadCommand{\input hologo.sty\relax}%
\fi
\def\Test{%
  \RangeCatcodeInvalid{0}{47}%
  \RangeCatcodeInvalid{58}{64}%
  \RangeCatcodeInvalid{91}{96}%
  \RangeCatcodeInvalid{123}{255}%
  \catcode`\@=12 %
  \catcode`\\=0 %
  \catcode`\%=14 %
  \LoadCommand
  \RangeCatcodeCheck{0}{36}{15}%
  \RangeCatcodeCheck{37}{37}{14}%
  \RangeCatcodeCheck{38}{47}{15}%
  \RangeCatcodeCheck{48}{57}{12}%
  \RangeCatcodeCheck{58}{63}{15}%
  \RangeCatcodeCheck{64}{64}{12}%
  \RangeCatcodeCheck{65}{90}{11}%
  \RangeCatcodeCheck{91}{91}{15}%
  \RangeCatcodeCheck{92}{92}{0}%
  \RangeCatcodeCheck{93}{96}{15}%
  \RangeCatcodeCheck{97}{122}{11}%
  \RangeCatcodeCheck{123}{255}{15}%
  \RestoreCatcodes
}
\Test
\csname @@end\endcsname
\end
%    \end{macrocode}
%    \begin{macrocode}
%</test1>
%    \end{macrocode}
%
% \subsection{Spacefactor}
%
%    The space factor must be 1000 after a logo. If it is greater 1000
%    then the following space is a space after a sentence closing point.
%    If the space factor is smaller 1000 then an immediate following
%    dot is interpreted as abbreviation, not sentence closing point.
%
%    \begin{macrocode}
%<*test-spacefactor>
\NeedsTeXFormat{LaTeX2e}
\documentclass{article}
\usepackage{hologo}[2016/05/12]
\usepackage{kvsetkeys}
\usepackage{qstest}
\IncludeTests{*}
\LogTests{log}{*}{*}
\begin{document}
\begin{qstest}{spacefactor}{spacefactor}
\newcommand*{\Test}[1]{%
  \sbox0{%
    \hologo{#1}%
    \Expect*{1000 (#1)}*{\the\spacefactor\space(#1)}%
  }%
}%
\makeatletter
\def\TestList{}
\def\hologoEntry#1#2#3{%
  \edef\TestList{%
    \ifx\TestList\@empty
    \else
      \TestList,%
    \fi
    #1%
    \ifx\\#2\\%
    \else
      ={variant=#2}%
    \fi
  }%
}
\hologoList
\expandafter\kv@parse@normalized\expandafter{%
  \TestList
}{%
  \begingroup
    \let\@logo=\kv@key
    \ifx\kv@value\relax
    \else
      \expandafter\hologoLogoSetup\expandafter\@logo\expandafter{%
        \kv@value
      }%
    \fi
    \Test\@logo
  \endgroup
  \@gobbletwo
}
\end{qstest}
\end{document}
%</test-spacefactor>
%    \end{macrocode}
%
% \subsection{Complete list}
%
%    \begin{macrocode}
%<*test-list>
\NeedsTeXFormat{LaTeX2e}
\documentclass[12pt,a4paper]{article}
\usepackage{hologo}[2016/05/12]
\usepackage[T1]{fontenc}
\usepackage{lmodern}
\usepackage{parskip}
\usepackage[unicode]{hyperref}[2011/09/28]
\usepackage{bookmark}[2011/09/19]
\bookmarksetup{%
  numbered,%
  open,%
  openlevel=2,%
}
\renewcommand*{\contentsname}{List of logos}
\begin{document}
\tableofcontents
\def\TestFont#1#2#3#4#5#6{%
  \begingroup
    \usefont{#3}{#4}{#5}{#6}%
    \HologoVariant{#1}{#2}/\hologoVariant{#1}{#2}%
    \quad
    \begingroup\scriptsize\hologoVariant{#1}{#2}\endgroup
    \quad
  \endgroup
  (#3/#4/#5/#6)%
  \par
}
\makeatletter
\def\hologoEntry#1#2#3{%
  \section{%
    \HologoVariant{#1}{#2}/\hologoVariant{#1}{#2} %
    {[#1\ifx\\#2\\\else\space(#2)\fi]}% hash-ok
  }% braces around [] because of bug in tex4ht
  \begingroup
    \hypersetup{unicode=false}%
    \bookmark[%
      dest=\@currentHref,%
      rellevel=1,%
      keeplevel,%
    ]{%
      \HologoVariant{#1}{#2}/\hologoVariant{#1}{#2} %
      (PDFDocEncoding)%
    }%
  \endgroup
  \TestFont{#1}{#2}{OT1}{cmr}{m}{n}%
  \TestFont{#1}{#2}{OT1}{cmss}{m}{n}%
  \TestFont{#1}{#2}{OT1}{cmr}{b}{n}%
  \TestFont{#1}{#2}{OT1}{cmr}{m}{it}%
  \TestFont{#1}{#2}{OT1}{cmtt}{m}{n}%
  \TestFont{#1}{#2}{T1}{lmr}{m}{n}%
  \TestFont{#1}{#2}{T1}{lmss}{m}{n}%
  \TestFont{#1}{#2}{T1}{lmr}{b}{n}%
  \TestFont{#1}{#2}{T1}{lmr}{m}{it}%
  \TestFont{#1}{#2}{T1}{lmtt}{m}{n}%
  \TestFont{#1}{#2}{T1}{lmvtt}{m}{n}%
  \TestFont{#1}{#2}{T1}{qtm}{m}{n}%
  \TestFont{#1}{#2}{T1}{qhv}{m}{n}%
  \TestFont{#1}{#2}{T1}{qtm}{b}{n}%
  \TestFont{#1}{#2}{T1}{qtm}{m}{it}%
  \TestFont{#1}{#2}{T1}{qcr}{m}{n}%
  \newpage
}
\makeatother
\hologoList
\end{document}
%</test-list>
%    \end{macrocode}
%
% \section{Installation}
%
% \subsection{Download}
%
% \paragraph{Package.} This package is available on
% CTAN\footnote{\url{ftp://ftp.ctan.org/tex-archive/}}:
% \begin{description}
% \item[\CTAN{macros/latex/contrib/oberdiek/hologo.dtx}] The source file.
% \item[\CTAN{macros/latex/contrib/oberdiek/hologo.pdf}] Documentation.
% \end{description}
%
%
% \paragraph{Bundle.} All the packages of the bundle `oberdiek'
% are also available in a TDS compliant ZIP archive. There
% the packages are already unpacked and the documentation files
% are generated. The files and directories obey the TDS standard.
% \begin{description}
% \item[\CTAN{install/macros/latex/contrib/oberdiek.tds.zip}]
% \end{description}
% \emph{TDS} refers to the standard ``A Directory Structure
% for \TeX\ Files'' (\CTAN{tds/tds.pdf}). Directories
% with \xfile{texmf} in their name are usually organized this way.
%
% \subsection{Bundle installation}
%
% \paragraph{Unpacking.} Unpack the \xfile{oberdiek.tds.zip} in the
% TDS tree (also known as \xfile{texmf} tree) of your choice.
% Example (linux):
% \begin{quote}
%   |unzip oberdiek.tds.zip -d ~/texmf|
% \end{quote}
%
% \paragraph{Script installation.}
% Check the directory \xfile{TDS:scripts/oberdiek/} for
% scripts that need further installation steps.
% Package \xpackage{attachfile2} comes with the Perl script
% \xfile{pdfatfi.pl} that should be installed in such a way
% that it can be called as \texttt{pdfatfi}.
% Example (linux):
% \begin{quote}
%   |chmod +x scripts/oberdiek/pdfatfi.pl|\\
%   |cp scripts/oberdiek/pdfatfi.pl /usr/local/bin/|
% \end{quote}
%
% \subsection{Package installation}
%
% \paragraph{Unpacking.} The \xfile{.dtx} file is a self-extracting
% \docstrip\ archive. The files are extracted by running the
% \xfile{.dtx} through \plainTeX:
% \begin{quote}
%   \verb|tex hologo.dtx|
% \end{quote}
%
% \paragraph{TDS.} Now the different files must be moved into
% the different directories in your installation TDS tree
% (also known as \xfile{texmf} tree):
% \begin{quote}
% \def\t{^^A
% \begin{tabular}{@{}>{\ttfamily}l@{ $\rightarrow$ }>{\ttfamily}l@{}}
%   hologo.sty & tex/generic/oberdiek/hologo.sty\\
%   hologo.pdf & doc/latex/oberdiek/hologo.pdf\\
%   example/hologo-example.tex & doc/latex/oberdiek/example/hologo-example.tex\\
%   test/hologo-test1.tex & doc/latex/oberdiek/test/hologo-test1.tex\\
%   test/hologo-test-spacefactor.tex & doc/latex/oberdiek/test/hologo-test-spacefactor.tex\\
%   test/hologo-test-list.tex & doc/latex/oberdiek/test/hologo-test-list.tex\\
%   hologo.dtx & source/latex/oberdiek/hologo.dtx\\
% \end{tabular}^^A
% }^^A
% \sbox0{\t}^^A
% \ifdim\wd0>\linewidth
%   \begingroup
%     \advance\linewidth by\leftmargin
%     \advance\linewidth by\rightmargin
%   \edef\x{\endgroup
%     \def\noexpand\lw{\the\linewidth}^^A
%   }\x
%   \def\lwbox{^^A
%     \leavevmode
%     \hbox to \linewidth{^^A
%       \kern-\leftmargin\relax
%       \hss
%       \usebox0
%       \hss
%       \kern-\rightmargin\relax
%     }^^A
%   }^^A
%   \ifdim\wd0>\lw
%     \sbox0{\small\t}^^A
%     \ifdim\wd0>\linewidth
%       \ifdim\wd0>\lw
%         \sbox0{\footnotesize\t}^^A
%         \ifdim\wd0>\linewidth
%           \ifdim\wd0>\lw
%             \sbox0{\scriptsize\t}^^A
%             \ifdim\wd0>\linewidth
%               \ifdim\wd0>\lw
%                 \sbox0{\tiny\t}^^A
%                 \ifdim\wd0>\linewidth
%                   \lwbox
%                 \else
%                   \usebox0
%                 \fi
%               \else
%                 \lwbox
%               \fi
%             \else
%               \usebox0
%             \fi
%           \else
%             \lwbox
%           \fi
%         \else
%           \usebox0
%         \fi
%       \else
%         \lwbox
%       \fi
%     \else
%       \usebox0
%     \fi
%   \else
%     \lwbox
%   \fi
% \else
%   \usebox0
% \fi
% \end{quote}
% If you have a \xfile{docstrip.cfg} that configures and enables \docstrip's
% TDS installing feature, then some files can already be in the right
% place, see the documentation of \docstrip.
%
% \subsection{Refresh file name databases}
%
% If your \TeX~distribution
% (\teTeX, \mikTeX, \dots) relies on file name databases, you must refresh
% these. For example, \teTeX\ users run \verb|texhash| or
% \verb|mktexlsr|.
%
% \subsection{Some details for the interested}
%
% \paragraph{Attached source.}
%
% The PDF documentation on CTAN also includes the
% \xfile{.dtx} source file. It can be extracted by
% AcrobatReader 6 or higher. Another option is \textsf{pdftk},
% e.g. unpack the file into the current directory:
% \begin{quote}
%   \verb|pdftk hologo.pdf unpack_files output .|
% \end{quote}
%
% \paragraph{Unpacking with \LaTeX.}
% The \xfile{.dtx} chooses its action depending on the format:
% \begin{description}
% \item[\plainTeX:] Run \docstrip\ and extract the files.
% \item[\LaTeX:] Generate the documentation.
% \end{description}
% If you insist on using \LaTeX\ for \docstrip\ (really,
% \docstrip\ does not need \LaTeX), then inform the autodetect routine
% about your intention:
% \begin{quote}
%   \verb|latex \let\install=y\input{hologo.dtx}|
% \end{quote}
% Do not forget to quote the argument according to the demands
% of your shell.
%
% \paragraph{Generating the documentation.}
% You can use both the \xfile{.dtx} or the \xfile{.drv} to generate
% the documentation. The process can be configured by the
% configuration file \xfile{ltxdoc.cfg}. For instance, put this
% line into this file, if you want to have A4 as paper format:
% \begin{quote}
%   \verb|\PassOptionsToClass{a4paper}{article}|
% \end{quote}
% An example follows how to generate the
% documentation with pdf\LaTeX:
% \begin{quote}
%\begin{verbatim}
%pdflatex hologo.dtx
%makeindex -s gind.ist hologo.idx
%pdflatex hologo.dtx
%makeindex -s gind.ist hologo.idx
%pdflatex hologo.dtx
%\end{verbatim}
% \end{quote}
%
% \section{Catalogue}
%
% The following XML file can be used as source for the
% \href{http://mirror.ctan.org/help/Catalogue/catalogue.html}{\TeX\ Catalogue}.
% The elements \texttt{caption} and \texttt{description} are imported
% from the original XML file from the Catalogue.
% The name of the XML file in the Catalogue is \xfile{hologo.xml}.
%    \begin{macrocode}
%<*catalogue>
<?xml version='1.0' encoding='us-ascii'?>
<!DOCTYPE entry SYSTEM 'catalogue.dtd'>
<entry datestamp='$Date$' modifier='$Author$' id='hologo'>
  <name>hologo</name>
  <caption>A collection of logos with bookmark support.</caption>
  <authorref id='auth:oberdiek'/>
  <copyright owner='Heiko Oberdiek' year='2010-2012'/>
  <license type='lppl1.3'/>
  <version number='1.10'/>
  <description>
    The package defines a single command <tt>\hologo</tt>, whose
    argument is the usual case-confused ASCII version of the logo.
    The command is bookmark-enabled, so that every logo becomes
    available in bookmarks without further work.
    <p/>
    The package is part of the <xref refid='oberdiek'>oberdiek</xref>
    bundle.
  </description>
  <documentation details='Package documentation'
      href='ctan:/macros/latex/contrib/oberdiek/hologo.pdf'/>
  <ctan file='true' path='/macros/latex/contrib/oberdiek/hologo.dtx'/>
  <miktex location='oberdiek'/>
  <texlive location='oberdiek'/>
  <install path='/macros/latex/contrib/oberdiek/oberdiek.tds.zip'/>
</entry>
%</catalogue>
%    \end{macrocode}
%
% \begin{thebibliography}{9}
% \raggedright
%
% \bibitem{btxdoc}
% Oren Patashnik,
% \textit{\hologo{BibTeX}ing},
% 1988-02-08.\\
% \CTAN{biblio/bibtex/base/}
%
% \bibitem{dtklogos}
% Gerd Neugebauer, DANTE,
% \textit{Package \xpackage{dtklogos}},
% 2011-04-25.\\
% \CTAN{usergrps/dante/dtk/dtklogos.sty}
%
% \bibitem{etexman}
% The \hologo{NTS} Team,
% \textit{The \hologo{eTeX} manual},
% 1998-02.\\
% \CTAN{systems/e-tex/v2/doc/}
%
% \bibitem{ExTeX-FAQ}
% The \hologo{ExTeX} group,
% \textit{\hologo{ExTeX}: FAQ -- How is \hologo{ExTeX} typeset?},
% 2007-04-14.\\
% \url{http://www.extex.org/documentation/faq.html}
%
% \bibitem{LyX}
% %@MISC{ LyX,
% %  title = {{LyX 2.0.0 -- The Document Processor [Computer software and manual]}},
% %  author = {{The LyX Team}},
% %  howpublished = {Internet: http://www.lyx.org},
% %  year = {2011-05-08},
% %  note = {Retrieved May 10, 2011, from http://www.lyx.org},
% %  url = {http://www.lyx.org/}
% %}
% The \hologo{LyX} Team,
% \textit{\hologo{LyX} -- The Document Processor},
% 2011-05-08.\\
% \url{http://www.lyx.org/}
%
% \bibitem{OzTeX}
% Andrew Trevorrow,
% \hologo{OzTeX} FAQ: What is the correct way to typeset ``\hologo{OzTeX}''?,
% 2011-09-15 (visited).
% \url{http://www.trevorrow.com/oztex/ozfaq.html#oztex-logo}
%
% \bibitem{PiCTeX}
% Michael Wichura,
% \textit{The \hologo{PiCTeX} macro package},
% 1987-09-21.
% \CTAN{graphics/pictex/}
%
% \bibitem{scrlogo}
% Markus Kohm,
% \textit{\hologo{KOMAScript} Datei \xfile{scrlogo.dtx}},
% 2009-01-30.\\
% \CTAN{install/macros/latex/contrib/komascript.tds.zip}
%
% \end{thebibliography}
%
% \begin{History}
%   \begin{Version}{2010/04/08 v1.0}
%   \item
%     The first version.
%   \end{Version}
%   \begin{Version}{2010/04/16 v1.1}
%   \item
%     \cs{Hologo} added for support of logos at start of a sentence.
%   \item
%     \cs{hologoSetup} and \cs{hologoLogoSetup} added.
%   \item
%     Options \xoption{break}, \xoption{hyphenbreak}, \xoption{spacebreak}
%     added.
%   \item
%     Variant support added by option \xoption{variant}.
%   \end{Version}
%   \begin{Version}{2010/04/24 v1.2}
%   \item
%     \hologo{LaTeX3} added.
%   \item
%     \hologo{VTeX} added.
%   \end{Version}
%   \begin{Version}{2010/11/21 v1.3}
%   \item
%     \hologo{iniTeX}, \hologo{virTeX} added.
%   \end{Version}
%   \begin{Version}{2011/03/25 v1.4}
%   \item
%     \hologo{ConTeXt} with variants added.
%   \item
%     Option \xoption{discretionarybreak} added as refinement for
%     option \xoption{break}.
%   \end{Version}
%   \begin{Version}{2011/04/21 v1.5}
%   \item
%     Wrong TDS directory for test files fixed.
%   \end{Version}
%   \begin{Version}{2011/10/01 v1.6}
%   \item
%     Support for package \xpackage{tex4ht} added.
%   \item
%     Support for \cs{csname} added if \cs{ifincsname} is available.
%   \item
%     New logos:
%     \hologo{(La)TeX},
%     \hologo{biber},
%     \hologo{BibTeX} (\xoption{sc}, \xoption{sf}),
%     \hologo{emTeX},
%     \hologo{ExTeX},
%     \hologo{KOMAScript},
%     \hologo{La},
%     \hologo{LyX},
%     \hologo{MiKTeX},
%     \hologo{NTS},
%     \hologo{OzMF},
%     \hologo{OzMP},
%     \hologo{OzTeX},
%     \hologo{OzTtH},
%     \hologo{PCTeX},
%     \hologo{PiC},
%     \hologo{PiCTeX},
%     \hologo{METAFONT},
%     \hologo{MetaFun},
%     \hologo{METAPOST},
%     \hologo{MetaPost},
%     \hologo{SLiTeX} (\xoption{lift}, \xoption{narrow}, \xoption{simple}),
%     \hologo{SliTeX} (\xoption{narrow}, \xoption{simple}, \xoption{lift}),
%     \hologo{teTeX}.
%   \item
%     Fixes:
%     \hologo{iniTeX},
%     \hologo{pdfLaTeX},
%     \hologo{pdfTeX},
%     \hologo{virTeX}.
%   \item
%     \cs{hologoFontSetup} and \cs{hologoLogoFontSetup} added.
%   \item
%     \cs{hologoVariant} and \cs{HologoVariant} added.
%   \end{Version}
%   \begin{Version}{2011/11/22 v1.7}
%   \item
%     New logos:
%     \hologo{BibTeX8},
%     \hologo{LaTeXML},
%     \hologo{SageTeX},
%     \hologo{TeX4ht},
%     \hologo{TTH}.
%   \item
%     \hologo{Xe} and friends: Driver stuff fixed.
%   \item
%     \hologo{Xe} and friends: Support for italic added.
%   \item
%     \hologo{Xe} and friends: Package support for \xpackage{pgf}
%     and \xpackage{pstricks} added.
%   \end{Version}
%   \begin{Version}{2011/11/29 v1.8}
%   \item
%     New logos:
%     \hologo{HanTheThanh}.
%   \end{Version}
%   \begin{Version}{2011/12/21 v1.9}
%   \item
%     Patch for package \xpackage{ifxetex} added for the case that
%     \cs{newif} is undefined in \hologo{iniTeX}.
%   \item
%     Some fixes for \hologo{iniTeX}.
%   \end{Version}
%   \begin{Version}{2012/04/26 v1.10}
%   \item
%     Fix in bookmark version of logo ``\hologo{HanTheThanh}''.
%   \end{Version}
%   \begin{Version}{2016/05/12 v1.11}
%   \item
%     Update HOLOGO@IfCharExists (previously in texlive)
%   \item define pdfliteral in current luatex.
%   \end{Version}
% \end{History}
%
% \PrintIndex
%
% \Finale
\endinput
%
        \else
          \input hologo.cfg\relax
        \fi
      \fi
    }%
    \ltx@IfUndefined{newread}{%
      \chardef\HOLOGO@temp=15 %
      \def\HOLOGO@CheckRead{%
        \ifeof\HOLOGO@temp
          \HOLOGO@InputIfExists
        \else
          \ifcase\HOLOGO@temp
            \@PackageWarningNoLine{hologo}{%
              Configuration file ignored, because\MessageBreak
              a free read register could not be found%
            }%
          \else
            \begingroup
              \count\ltx@cclv=\HOLOGO@temp
              \advance\ltx@cclv by \ltx@minusone
              \edef\x{\endgroup
                \chardef\noexpand\HOLOGO@temp=\the\count\ltx@cclv
                \relax
              }%
            \x
          \fi
        \fi
      }%
    }{%
      \csname newread\endcsname\HOLOGO@temp
      \HOLOGO@InputIfExists
    }%
  }{%
    \edef\HOLOGO@temp{\pdf@filesize{hologo.cfg}}%
    \ifx\HOLOGO@temp\ltx@empty
    \else
      \ifnum\HOLOGO@temp>0 %
        \begingroup
          \def\x{LaTeX2e}%
        \expandafter\endgroup
        \ifx\fmtname\x
          % \iffalse meta-comment
%
% File: hologo.dtx
% Version: 2016/05/12 v1.11
% Info: A logo collection with bookmark support
%
% Copyright (C) 2010-2012 by
%    Heiko Oberdiek <heiko.oberdiek at googlemail.com>
%
% This work may be distributed and/or modified under the
% conditions of the LaTeX Project Public License, either
% version 1.3c of this license or (at your option) any later
% version. This version of this license is in
%    http://www.latex-project.org/lppl/lppl-1-3c.txt
% and the latest version of this license is in
%    http://www.latex-project.org/lppl.txt
% and version 1.3 or later is part of all distributions of
% LaTeX version 2005/12/01 or later.
%
% This work has the LPPL maintenance status "maintained".
%
% This Current Maintainer of this work is Heiko Oberdiek.
%
% The Base Interpreter refers to any `TeX-Format',
% because some files are installed in TDS:tex/generic//.
%
% This work consists of the main source file hologo.dtx
% and the derived files
%    hologo.sty, hologo.pdf, hologo.ins, hologo.drv, hologo-example.tex,
%    hologo-test1.tex, hologo-test-spacefactor.tex,
%    hologo-test-list.tex.
%
% Distribution:
%    CTAN:macros/latex/contrib/oberdiek/hologo.dtx
%    CTAN:macros/latex/contrib/oberdiek/hologo.pdf
%
% Unpacking:
%    (a) If hologo.ins is present:
%           tex hologo.ins
%    (b) Without hologo.ins:
%           tex hologo.dtx
%    (c) If you insist on using LaTeX
%           latex \let\install=y\input{hologo.dtx}
%        (quote the arguments according to the demands of your shell)
%
% Documentation:
%    (a) If hologo.drv is present:
%           latex hologo.drv
%    (b) Without hologo.drv:
%           latex hologo.dtx; ...
%    The class ltxdoc loads the configuration file ltxdoc.cfg
%    if available. Here you can specify further options, e.g.
%    use A4 as paper format:
%       \PassOptionsToClass{a4paper}{article}
%
%    Programm calls to get the documentation (example):
%       pdflatex hologo.dtx
%       makeindex -s gind.ist hologo.idx
%       pdflatex hologo.dtx
%       makeindex -s gind.ist hologo.idx
%       pdflatex hologo.dtx
%
% Installation:
%    TDS:tex/generic/oberdiek/hologo.sty
%    TDS:doc/latex/oberdiek/hologo.pdf
%    TDS:doc/latex/oberdiek/example/hologo-example.tex
%    TDS:doc/latex/oberdiek/test/hologo-test1.tex
%    TDS:doc/latex/oberdiek/test/hologo-test-spacefactor.tex
%    TDS:doc/latex/oberdiek/test/hologo-test-list.tex
%    TDS:source/latex/oberdiek/hologo.dtx
%
%<*ignore>
\begingroup
  \catcode123=1 %
  \catcode125=2 %
  \def\x{LaTeX2e}%
\expandafter\endgroup
\ifcase 0\ifx\install y1\fi\expandafter
         \ifx\csname processbatchFile\endcsname\relax\else1\fi
         \ifx\fmtname\x\else 1\fi\relax
\else\csname fi\endcsname
%</ignore>
%<*install>
\input docstrip.tex
\Msg{************************************************************************}
\Msg{* Installation}
\Msg{* Package: hologo 2016/05/12 v1.11 A logo collection with bookmark support (HO)}
\Msg{************************************************************************}

\keepsilent
\askforoverwritefalse

\let\MetaPrefix\relax
\preamble

This is a generated file.

Project: hologo
Version: 2016/05/12 v1.11

Copyright (C) 2010-2012 by
   Heiko Oberdiek <heiko.oberdiek at googlemail.com>

This work may be distributed and/or modified under the
conditions of the LaTeX Project Public License, either
version 1.3c of this license or (at your option) any later
version. This version of this license is in
   http://www.latex-project.org/lppl/lppl-1-3c.txt
and the latest version of this license is in
   http://www.latex-project.org/lppl.txt
and version 1.3 or later is part of all distributions of
LaTeX version 2005/12/01 or later.

This work has the LPPL maintenance status "maintained".

This Current Maintainer of this work is Heiko Oberdiek.

The Base Interpreter refers to any `TeX-Format',
because some files are installed in TDS:tex/generic//.

This work consists of the main source file hologo.dtx
and the derived files
   hologo.sty, hologo.pdf, hologo.ins, hologo.drv, hologo-example.tex,
   hologo-test1.tex, hologo-test-spacefactor.tex,
   hologo-test-list.tex.

\endpreamble
\let\MetaPrefix\DoubleperCent

\generate{%
  \file{hologo.ins}{\from{hologo.dtx}{install}}%
  \file{hologo.drv}{\from{hologo.dtx}{driver}}%
  \usedir{tex/generic/oberdiek}%
  \file{hologo.sty}{\from{hologo.dtx}{package}}%
  \usedir{doc/latex/oberdiek/example}%
  \file{hologo-example.tex}{\from{hologo.dtx}{example}}%
  \usedir{doc/latex/oberdiek/test}%
  \file{hologo-test1.tex}{\from{hologo.dtx}{test1}}%
  \file{hologo-test-spacefactor.tex}{\from{hologo.dtx}{test-spacefactor}}%
  \file{hologo-test-list.tex}{\from{hologo.dtx}{test-list}}%
  \nopreamble
  \nopostamble
  \usedir{source/latex/oberdiek/catalogue}%
  \file{hologo.xml}{\from{hologo.dtx}{catalogue}}%
}

\catcode32=13\relax% active space
\let =\space%
\Msg{************************************************************************}
\Msg{*}
\Msg{* To finish the installation you have to move the following}
\Msg{* file into a directory searched by TeX:}
\Msg{*}
\Msg{*     hologo.sty}
\Msg{*}
\Msg{* To produce the documentation run the file `hologo.drv'}
\Msg{* through LaTeX.}
\Msg{*}
\Msg{* Happy TeXing!}
\Msg{*}
\Msg{************************************************************************}

\endbatchfile
%</install>
%<*ignore>
\fi
%</ignore>
%<*driver>
\NeedsTeXFormat{LaTeX2e}
\ProvidesFile{hologo.drv}%
  [2016/05/12 v1.11 A logo collection with bookmark support (HO)]%
\documentclass{ltxdoc}
\usepackage{holtxdoc}[2011/11/22]
\usepackage{hologo}[2016/05/12]
\usepackage{longtable}
\usepackage{array}
\usepackage{paralist}
%\usepackage[T1]{fontenc}
%\usepackage{lmodern}
\begin{document}
  \DocInput{hologo.dtx}%
\end{document}
%</driver>
% \fi
%
%
% \CharacterTable
%  {Upper-case    \A\B\C\D\E\F\G\H\I\J\K\L\M\N\O\P\Q\R\S\T\U\V\W\X\Y\Z
%   Lower-case    \a\b\c\d\e\f\g\h\i\j\k\l\m\n\o\p\q\r\s\t\u\v\w\x\y\z
%   Digits        \0\1\2\3\4\5\6\7\8\9
%   Exclamation   \!     Double quote  \"     Hash (number) \#
%   Dollar        \$     Percent       \%     Ampersand     \&
%   Acute accent  \'     Left paren    \(     Right paren   \)
%   Asterisk      \*     Plus          \+     Comma         \,
%   Minus         \-     Point         \.     Solidus       \/
%   Colon         \:     Semicolon     \;     Less than     \<
%   Equals        \=     Greater than  \>     Question mark \?
%   Commercial at \@     Left bracket  \[     Backslash     \\
%   Right bracket \]     Circumflex    \^     Underscore    \_
%   Grave accent  \`     Left brace    \{     Vertical bar  \|
%   Right brace   \}     Tilde         \~}
%
% \GetFileInfo{hologo.drv}
%
% \title{The \xpackage{hologo} package}
% \date{2016/05/12 v1.11}
% \author{Heiko Oberdiek\\\xemail{heiko.oberdiek at googlemail.com}}
%
% \maketitle
%
% \begin{abstract}
% This package starts a collection of logos with support for bookmarks
% strings.
% \end{abstract}
%
% \tableofcontents
%
% \section{Documentation}
%
% \subsection{Logo macros}
%
% \begin{declcs}{hologo} \M{name}
% \end{declcs}
% Macro \cs{hologo} sets the logo with name \meta{name}.
% The following table shows the supported names.
%
% \begingroup
%   \def\hologoEntry#1#2#3{^^A
%     #1&#2&\hologoLogoSetup{#1}{variant=#2}\hologo{#1}&#3\tabularnewline
%   }
%   \begin{longtable}{>{\ttfamily}l>{\ttfamily}lll}
%     \rmfamily\bfseries{name} & \rmfamily\bfseries variant
%     & \bfseries logo & \bfseries since\\
%     \hline
%     \endhead
%     \hologoList
%   \end{longtable}
% \endgroup
%
% \begin{declcs}{Hologo} \M{name}
% \end{declcs}
% Macro \cs{Hologo} starts the logo \meta{name} with an uppercase
% letter. As an exception small greek letters are not converted
% to uppercase. Examples, see \hologo{eTeX} and \hologo{ExTeX}.
%
% \subsection{Setup macros}
%
% The package does not support package options, but the following
% setup macros can be used to set options.
%
% \begin{declcs}{hologoSetup} \M{key value list}
% \end{declcs}
% Macro \cs{hologoSetup} sets global options.
%
% \begin{declcs}{hologoLogoSetup} \M{logo} \M{key value list}
% \end{declcs}
% Some options can also be used to configure a logo.
% These settings take precedence over global option settings.
%
% \subsection{Options}\label{sec:options}
%
% There are boolean and string options:
% \begin{description}
% \item[Boolean option:]
% It takes |true| or |false|
% as value. If the value is omitted, then |true| is used.
% \item[String option:]
% A value must be given as string. (But the string might be empty.)
% \end{description}
% The following options can be used both in \cs{hologoSetup}
% and \cs{hologoLogoSetup}:
% \begin{description}
% \def\entry#1{\item[\xoption{#1}:]}
% \entry{break}
%   enables or disables line breaks inside the logo. This setting is
%   refined by options \xoption{hyphenbreak}, \xoption{spacebreak}
%   or \xoption{discretionarybreak}.
%   Default is |false|.
% \entry{hyphenbreak}
%   enables or disables the line break right after the hyphen character.
% \entry{spacebreak}
%   enables or disables line breaks at space characters.
% \entry{discretionarybreak}
%   enables or disables line breaks at hyphenation points
%   (inserted by \cs{-}).
% \end{description}
% Macro \cs{hologoLogoSetup} also knows:
% \begin{description}
% \item[\xoption{variant}:]
%   This is a string option. It specifies a variant of a logo that
%   must exist. An empty string selects the package default variant.
% \end{description}
% Example:
% \begin{quote}
%   |\hologoSetup{break=false}|\\
%   |\hologoLogoSetup{plainTeX}{variant=hyphen,hyphenbreak}|\\
%   Then ``plain-\TeX'' contains one break point after the hyphen.
% \end{quote}
%
% \subsection{Driver options}
%
% Sometimes graphical operations are needed to construct some
% glyphs (e.g.\ \hologo{XeTeX}). If package \xpackage{graphics}
% or package \xpackage{pgf} are found, then the macros are taken
% from there. Otherwise the packge defines its own operations
% and therefore needs the driver information. Many drivers are
% detected automatically (\hologo{pdfTeX}/\hologo{LuaTeX}
% in PDF mode, \hologo{XeTeX}, \hologo{VTeX}). These have precedence
% over a driver option. The driver can be given as package option
% or using \cs{hologoDriverSetup}.
% The following list contains the recognized driver options:
% \begin{itemize}
% \item \xoption{pdftex}, \xoption{luatex}
% \item \xoption{dvipdfm}, \xoption{dvipdfmx}
% \item \xoption{dvips}, \xoption{dvipsone}, \xoption{xdvi}
% \item \xoption{xetex}
% \item \xoption{vtex}
% \end{itemize}
% The left driver of a line is the driver name that is used internally.
% The following names are aliases for drivers that use the
% same method. Therefore the entry in the \xext{log} file for
% the used driver prints the internally used driver name.
% \begin{description}
% \item[\xoption{driverfallback}:]
%   This option expects a driver that is used,
%   if the driver could not be detected automatically.
% \end{description}
%
% \begin{declcs}{hologoDriverSetup} \M{driver option}
% \end{declcs}
% The driver can also be configured after package loading
% using \cs{hologoDriverSetup}, also the way for \hologo{plainTeX}
% to setup the driver.
%
% \subsection{Font setup}
%
% Some logos require a special font, but should also be usable by
% \hologo{plainTeX}. Therefore the package provides some ways
% to influence the font settings. The options below
% take font settings as values. Both font commands
% such as \cs{sffamily} and macros that take one argument
% like \cs{textsf} can be used.
%
% \begin{declcs}{hologoFontSetup} \M{key value list}
% \end{declcs}
% Macro \cs{hologoFontSetup} sets the fonts for all logos.
% Supported keys:
% \begin{description}
% \def\entry#1{\item[\xoption{#1}:]}
% \entry{general}
%   This font is used for all logos. The default is empty.
%   That means no special font is used.
% \entry{bibsf}
%   This font is used for
%   {\hologoLogoSetup{BibTeX}{variant=sf}\hologo{BibTeX}}
%   with variant \xoption{sf}.
% \entry{rm}
%   This font is a serif font. It is used for \hologo{ExTeX}.
% \entry{sc}
%   This font specifies a small caps font. It is used for
%   {\hologoLogoSetup{BibTeX}{variant=sc}\hologo{BibTeX}}
%   with variant \xoption{sc}.
% \entry{sf}
%   This font specifies a sans serif font. The default
%   is \cs{sffamily}, then \cs{sf} is tried. Otherwise
%   a warning is given. It is used by \hologo{KOMAScript}.
% \entry{sy}
%   This is the font for math symbols (e.g. cmsy).
%   It is used by \hologo{AmS}, \hologo{NTS}, \hologo{ExTeX}.
% \entry{logo}
%   \hologo{METAFONT} and \hologo{METAPOST} are using that font.
%   In \hologo{LaTeX} \cs{logofamily} is used and
%   the definitions of package \xpackage{mflogo} are used
%   if the package is not loaded.
%   Otherwise the \cs{tenlogo} is used and defined
%   if it does not already exists.
% \end{description}
%
% \begin{declcs}{hologoLogoFontSetup} \M{logo} \M{key value list}
% \end{declcs}
% Fonts can also be set for a logo or logo component separately,
% see the following list.
% The keys are the same as for \cs{hologoFontSetup}.
%
% \begin{longtable}{>{\ttfamily}l>{\sffamily}ll}
%   \meta{logo} & keys & result\\
%   \hline
%   \endhead
%   BibTeX & bibsf & {\hologoLogoSetup{BibTeX}{variant=sf}\hologo{BibTeX}}\\[.5ex]
%   BibTeX & sc & {\hologoLogoSetup{BibTeX}{variant=sc}\hologo{BibTeX}}\\[.5ex]
%   ExTeX & rm & \hologo{ExTeX}\\
%   SliTeX & rm & \hologo{SliTeX}\\[.5ex]
%   AmS & sy & \hologo{AmS}\\
%   ExTeX & sy & \hologo{ExTeX}\\
%   NTS & sy & \hologo{NTS}\\[.5ex]
%   KOMAScript & sf & \hologo{KOMAScript}\\[.5ex]
%   METAFONT & logo & \hologo{METAFONT}\\
%   METAPOST & logo & \hologo{METAPOST}\\[.5ex]
%   SliTeX & sc \hologo{SliTeX}
% \end{longtable}
%
% \subsubsection{Font order}
%
% For all logos the font \xoption{general} is applied first.
% Example:
%\begin{quote}
%|\hologoFontSetup{general=\color{red}}|
%\end{quote}
% will print red logos.
% Then if the font uses a special font \xoption{sf}, for example,
% the font is applied that is setup by \cs{hologoLogoFontSetup}.
% If this font is not setup, then the common font setup
% by \cs{hologoFontSetup} is used. Otherwise a warning is given,
% that there is no font configured.
%
% \subsection{Additional user macros}
%
% Usually a variant of a logo is configured by using
% \cs{hologoLogoSetup}, because it is bad style to mix
% different variants of the same logo in the same text.
% There the following macros are a convenience for testing.
%
% \begin{declcs}{hologoVariant} \M{name} \M{variant}\\
%   \cs{HologoVariant} \M{name} \M{variant}
% \end{declcs}
% Logo \meta{name} is set using \meta{variant} that specifies
% explicitely which variant of the macro is used. If the argument
% is empty, then the default form of the logo is used
% (configurable by \cs{hologoLogoSetup}).
%
% \cs{HologoVariant} is used if the logo is set in a context
% that needs an uppercase first letter (beginning of a sentence, \dots).
%
% \begin{declcs}{hologoList}\\
%   \cs{hologoEntry} \M{logo} \M{variant} \M{since}
% \end{declcs}
% Macro \cs{hologoList} contains all logos that are provided
% by the package including variants. The list consists of calls
% of \cs{hologoEntry} with three arguments starting with the
% logo name \meta{logo} and its variant \meta{variant}. An empty
% variant means the current default. Argument \meta{since} specifies
% with version of the package \xpackage{hologo} is needed to get
% the logo. If the logo is fixed, then the date gets updated.
% Therefore the date \meta{since} is not exactly the date of
% the first introduction, but rather the date of the latest fix.
%
% Before \cs{hologoList} can be used, macro \cs{hologoEntry} needs
% a definition. The example file in section \ref{sec:example}
% shows applications of \cs{hologoList}.
%
% \subsection{Supported contexts}
%
% Macros \cs{hologo} and friends support special contexts:
% \begin{itemize}
% \item \hologo{LaTeX}'s protection mechanism.
% \item Bookmarks of package \xpackage{hyperref}.
% \item Package \xpackage{tex4ht}.
% \item The macros can be used inside \cs{csname} constructs,
%   if \cs{ifincsname} is available (\hologo{pdfTeX}, \hologo{XeTeX},
%   \hologo{LuaTeX}).
% \end{itemize}
%
% \subsection{Example}
% \label{sec:example}
%
% The following example prints the logos in different fonts.
%    \begin{macrocode}
%<*example>
%<<verbatim
\NeedsTeXFormat{LaTeX2e}
\documentclass[a4paper]{article}
\usepackage[
  hmargin=20mm,
  vmargin=20mm,
]{geometry}
\pagestyle{empty}
\usepackage{hologo}[2016/05/12]
\usepackage{longtable}
\usepackage{array}
\setlength{\extrarowheight}{2pt}
\usepackage[T1]{fontenc}
\usepackage{lmodern}
\usepackage{pdflscape}
\usepackage[
  pdfencoding=auto,
]{hyperref}
\hypersetup{
  pdfauthor={Heiko Oberdiek},
  pdftitle={Example for package `hologo'},
  pdfsubject={Logos with fonts lmr, lmss, qtm, qpl, qhv},
}
\usepackage{bookmark}

% Print the logo list on the console

\begingroup
  \typeout{}%
  \typeout{*** Begin of logo list ***}%
  \newcommand*{\hologoEntry}[3]{%
    \typeout{#1 \ifx\\#2\\\else(#2) \fi[#3]}%
  }%
  \hologoList
  \typeout{*** End of logo list ***}%
  \typeout{}%
\endgroup

\begin{document}
\begin{landscape}

  \section{Example file for package `hologo'}

  % Table for font names

  \begin{longtable}{>{\bfseries}ll}
    \textbf{font} & \textbf{Font name}\\
    \hline
    lmr & Latin Modern Roman\\
    lmss & Latin Modern Sans\\
    qtm & \TeX\ Gyre Termes\\
    qhv & \TeX\ Gyre Heros\\
    qpl & \TeX\ Gyre Pagella\\
  \end{longtable}

  % Logo list with logos in different fonts

  \begingroup
    \newcommand*{\SetVariant}[2]{%
      \ifx\\#2\\%
      \else
        \hologoLogoSetup{#1}{variant=#2}%
      \fi
    }%
    \newcommand*{\hologoEntry}[3]{%
      \SetVariant{#1}{#2}%
      \raisebox{1em}[0pt][0pt]{\hypertarget{#1@#2}{}}%
      \bookmark[%
        dest={#1@#2},%
      ]{%
        #1\ifx\\#2\\\else\space(#2)\fi: \Hologo{#1}, \hologo{#1} %
        [Unicode]%
      }%
      \hypersetup{unicode=false}%
      \bookmark[%
        dest={#1@#2},%
      ]{%
        #1\ifx\\#2\\\else\space(#2)\fi: \Hologo{#1}, \hologo{#1} %
        [PDFDocEncoding]%
      }%
      \texttt{#1}%
      &%
      \texttt{#2}%
      &%
      \Hologo{#1}%
      &%
      \SetVariant{#1}{#2}%
      \hologo{#1}%
      &%
      \SetVariant{#1}{#2}%
      \fontfamily{qtm}\selectfont
      \hologo{#1}%
      &%
      \SetVariant{#1}{#2}%
      \fontfamily{qpl}\selectfont
      \hologo{#1}%
      &%
      \SetVariant{#1}{#2}%
      \textsf{\hologo{#1}}%
      &%
      \SetVariant{#1}{#2}%
      \fontfamily{qhv}\selectfont
      \hologo{#1}%
      \tabularnewline
    }%
    \begin{longtable}{llllllll}%
      \textbf{\textit{logo}} & \textbf{\textit{variant}} &
      \texttt{\string\Hologo} &
      \textbf{lmr} & \textbf{qtm} & \textbf{qpl} &
      \textbf{lmss} & \textbf{qhv}
      \tabularnewline
      \hline
      \endhead
      \hologoList
    \end{longtable}%
  \endgroup

\end{landscape}
\end{document}
%verbatim
%</example>
%    \end{macrocode}
%
% \StopEventually{
% }
%
% \section{Implementation}
%    \begin{macrocode}
%<*package>
%    \end{macrocode}
%    Reload check, especially if the package is not used with \LaTeX.
%    \begin{macrocode}
\begingroup\catcode61\catcode48\catcode32=10\relax%
  \catcode13=5 % ^^M
  \endlinechar=13 %
  \catcode35=6 % #
  \catcode39=12 % '
  \catcode44=12 % ,
  \catcode45=12 % -
  \catcode46=12 % .
  \catcode58=12 % :
  \catcode64=11 % @
  \catcode123=1 % {
  \catcode125=2 % }
  \expandafter\let\expandafter\x\csname ver@hologo.sty\endcsname
  \ifx\x\relax % plain-TeX, first loading
  \else
    \def\empty{}%
    \ifx\x\empty % LaTeX, first loading,
      % variable is initialized, but \ProvidesPackage not yet seen
    \else
      \expandafter\ifx\csname PackageInfo\endcsname\relax
        \def\x#1#2{%
          \immediate\write-1{Package #1 Info: #2.}%
        }%
      \else
        \def\x#1#2{\PackageInfo{#1}{#2, stopped}}%
      \fi
      \x{hologo}{The package is already loaded}%
      \aftergroup\endinput
    \fi
  \fi
\endgroup%
%    \end{macrocode}
%    Package identification:
%    \begin{macrocode}
\begingroup\catcode61\catcode48\catcode32=10\relax%
  \catcode13=5 % ^^M
  \endlinechar=13 %
  \catcode35=6 % #
  \catcode39=12 % '
  \catcode40=12 % (
  \catcode41=12 % )
  \catcode44=12 % ,
  \catcode45=12 % -
  \catcode46=12 % .
  \catcode47=12 % /
  \catcode58=12 % :
  \catcode64=11 % @
  \catcode91=12 % [
  \catcode93=12 % ]
  \catcode123=1 % {
  \catcode125=2 % }
  \expandafter\ifx\csname ProvidesPackage\endcsname\relax
    \def\x#1#2#3[#4]{\endgroup
      \immediate\write-1{Package: #3 #4}%
      \xdef#1{#4}%
    }%
  \else
    \def\x#1#2[#3]{\endgroup
      #2[{#3}]%
      \ifx#1\@undefined
        \xdef#1{#3}%
      \fi
      \ifx#1\relax
        \xdef#1{#3}%
      \fi
    }%
  \fi
\expandafter\x\csname ver@hologo.sty\endcsname
\ProvidesPackage{hologo}%
  [2016/05/12 v1.11 A logo collection with bookmark support (HO)]%
%    \end{macrocode}
%
%    \begin{macrocode}
\begingroup\catcode61\catcode48\catcode32=10\relax%
  \catcode13=5 % ^^M
  \endlinechar=13 %
  \catcode123=1 % {
  \catcode125=2 % }
  \catcode64=11 % @
  \def\x{\endgroup
    \expandafter\edef\csname HOLOGO@AtEnd\endcsname{%
      \endlinechar=\the\endlinechar\relax
      \catcode13=\the\catcode13\relax
      \catcode32=\the\catcode32\relax
      \catcode35=\the\catcode35\relax
      \catcode61=\the\catcode61\relax
      \catcode64=\the\catcode64\relax
      \catcode123=\the\catcode123\relax
      \catcode125=\the\catcode125\relax
    }%
  }%
\x\catcode61\catcode48\catcode32=10\relax%
\catcode13=5 % ^^M
\endlinechar=13 %
\catcode35=6 % #
\catcode64=11 % @
\catcode123=1 % {
\catcode125=2 % }
\def\TMP@EnsureCode#1#2{%
  \edef\HOLOGO@AtEnd{%
    \HOLOGO@AtEnd
    \catcode#1=\the\catcode#1\relax
  }%
  \catcode#1=#2\relax
}
\TMP@EnsureCode{10}{12}% ^^J
\TMP@EnsureCode{33}{12}% !
\TMP@EnsureCode{34}{12}% "
\TMP@EnsureCode{36}{3}% $
\TMP@EnsureCode{38}{4}% &
\TMP@EnsureCode{39}{12}% '
\TMP@EnsureCode{40}{12}% (
\TMP@EnsureCode{41}{12}% )
\TMP@EnsureCode{42}{12}% *
\TMP@EnsureCode{43}{12}% +
\TMP@EnsureCode{44}{12}% ,
\TMP@EnsureCode{45}{12}% -
\TMP@EnsureCode{46}{12}% .
\TMP@EnsureCode{47}{12}% /
\TMP@EnsureCode{58}{12}% :
\TMP@EnsureCode{59}{12}% ;
\TMP@EnsureCode{60}{12}% <
\TMP@EnsureCode{62}{12}% >
\TMP@EnsureCode{63}{12}% ?
\TMP@EnsureCode{91}{12}% [
\TMP@EnsureCode{93}{12}% ]
\TMP@EnsureCode{94}{7}% ^ (superscript)
\TMP@EnsureCode{95}{8}% _ (subscript)
\TMP@EnsureCode{96}{12}% `
\TMP@EnsureCode{124}{12}% |
\edef\HOLOGO@AtEnd{%
  \HOLOGO@AtEnd
  \escapechar\the\escapechar\relax
  \noexpand\endinput
}
\escapechar=92 %
%    \end{macrocode}
%
% \subsection{Logo list}
%
%    \begin{macro}{\hologoList}
%    \begin{macrocode}
\def\hologoList{%
  \hologoEntry{(La)TeX}{}{2011/10/01}%
  \hologoEntry{AmSLaTeX}{}{2010/04/16}%
  \hologoEntry{AmSTeX}{}{2010/04/16}%
  \hologoEntry{biber}{}{2011/10/01}%
  \hologoEntry{BibTeX}{}{2011/10/01}%
  \hologoEntry{BibTeX}{sf}{2011/10/01}%
  \hologoEntry{BibTeX}{sc}{2011/10/01}%
  \hologoEntry{BibTeX8}{}{2011/11/22}%
  \hologoEntry{ConTeXt}{}{2011/03/25}%
  \hologoEntry{ConTeXt}{narrow}{2011/03/25}%
  \hologoEntry{ConTeXt}{simple}{2011/03/25}%
  \hologoEntry{emTeX}{}{2010/04/26}%
  \hologoEntry{eTeX}{}{2010/04/08}%
  \hologoEntry{ExTeX}{}{2011/10/01}%
  \hologoEntry{HanTheThanh}{}{2011/11/29}%
  \hologoEntry{iniTeX}{}{2011/10/01}%
  \hologoEntry{KOMAScript}{}{2011/10/01}%
  \hologoEntry{La}{}{2010/05/08}%
  \hologoEntry{LaTeX}{}{2010/04/08}%
  \hologoEntry{LaTeX2e}{}{2010/04/08}%
  \hologoEntry{LaTeX3}{}{2010/04/24}%
  \hologoEntry{LaTeXe}{}{2010/04/08}%
  \hologoEntry{LaTeXML}{}{2011/11/22}%
  \hologoEntry{LaTeXTeX}{}{2011/10/01}%
  \hologoEntry{LuaLaTeX}{}{2010/04/08}%
  \hologoEntry{LuaTeX}{}{2010/04/08}%
  \hologoEntry{LyX}{}{2011/10/01}%
  \hologoEntry{METAFONT}{}{2011/10/01}%
  \hologoEntry{MetaFun}{}{2011/10/01}%
  \hologoEntry{METAPOST}{}{2011/10/01}%
  \hologoEntry{MetaPost}{}{2011/10/01}%
  \hologoEntry{MiKTeX}{}{2011/10/01}%
  \hologoEntry{NTS}{}{2011/10/01}%
  \hologoEntry{OzMF}{}{2011/10/01}%
  \hologoEntry{OzMP}{}{2011/10/01}%
  \hologoEntry{OzTeX}{}{2011/10/01}%
  \hologoEntry{OzTtH}{}{2011/10/01}%
  \hologoEntry{PCTeX}{}{2011/10/01}%
  \hologoEntry{pdfTeX}{}{2011/10/01}%
  \hologoEntry{pdfLaTeX}{}{2011/10/01}%
  \hologoEntry{PiC}{}{2011/10/01}%
  \hologoEntry{PiCTeX}{}{2011/10/01}%
  \hologoEntry{plainTeX}{}{2010/04/08}%
  \hologoEntry{plainTeX}{space}{2010/04/16}%
  \hologoEntry{plainTeX}{hyphen}{2010/04/16}%
  \hologoEntry{plainTeX}{runtogether}{2010/04/16}%
  \hologoEntry{SageTeX}{}{2011/11/22}%
  \hologoEntry{SLiTeX}{}{2011/10/01}%
  \hologoEntry{SLiTeX}{lift}{2011/10/01}%
  \hologoEntry{SLiTeX}{narrow}{2011/10/01}%
  \hologoEntry{SLiTeX}{simple}{2011/10/01}%
  \hologoEntry{SliTeX}{}{2011/10/01}%
  \hologoEntry{SliTeX}{narrow}{2011/10/01}%
  \hologoEntry{SliTeX}{simple}{2011/10/01}%
  \hologoEntry{SliTeX}{lift}{2011/10/01}%
  \hologoEntry{teTeX}{}{2011/10/01}%
  \hologoEntry{TeX}{}{2010/04/08}%
  \hologoEntry{TeX4ht}{}{2011/11/22}%
  \hologoEntry{TTH}{}{2011/11/22}%
  \hologoEntry{virTeX}{}{2011/10/01}%
  \hologoEntry{VTeX}{}{2010/04/24}%
  \hologoEntry{Xe}{}{2010/04/08}%
  \hologoEntry{XeLaTeX}{}{2010/04/08}%
  \hologoEntry{XeTeX}{}{2010/04/08}%
}
%    \end{macrocode}
%    \end{macro}
%
% \subsection{Load resources}
%
%    \begin{macrocode}
\begingroup\expandafter\expandafter\expandafter\endgroup
\expandafter\ifx\csname RequirePackage\endcsname\relax
  \def\TMP@RequirePackage#1[#2]{%
    \begingroup\expandafter\expandafter\expandafter\endgroup
    \expandafter\ifx\csname ver@#1.sty\endcsname\relax
      \input #1.sty\relax
    \fi
  }%
  \TMP@RequirePackage{ltxcmds}[2011/02/04]%
  \TMP@RequirePackage{infwarerr}[2010/04/08]%
  \TMP@RequirePackage{kvsetkeys}[2010/03/01]%
  \TMP@RequirePackage{kvdefinekeys}[2010/03/01]%
  \TMP@RequirePackage{pdftexcmds}[2010/04/01]%
  \TMP@RequirePackage{ifpdf}[2010/01/28]%
  \TMP@RequirePackage{ifluatex}[2010/03/01]%
  \ltx@IfUndefined{newif}{%
    \expandafter\let\csname newif\endcsname\ltx@newif
  }{}%
  \TMP@RequirePackage{ifxetex}[2009/01/23]%
  \TMP@RequirePackage{ifvtex}[2010/03/01]%
\else
  \RequirePackage{ltxcmds}[2011/02/04]%
  \RequirePackage{infwarerr}[2010/04/08]%
  \RequirePackage{kvsetkeys}[2010/03/01]%
  \RequirePackage{kvdefinekeys}[2010/03/01]%
  \RequirePackage{pdftexcmds}[2010/04/01]%
  \RequirePackage{ifpdf}[2010/01/28]%
  \RequirePackage{ifluatex}[2010/03/01]%
  \RequirePackage{ifxetex}[2009/01/23]%
  \RequirePackage{ifvtex}[2010/03/01]%
\fi
%    \end{macrocode}
%
%    \begin{macro}{\HOLOGO@IfDefined}
%    \begin{macrocode}
\def\HOLOGO@IfExists#1{%
  \ifx\@undefined#1%
    \expandafter\ltx@secondoftwo
  \else
    \ifx\relax#1%
      \expandafter\ltx@secondoftwo
    \else
      \expandafter\expandafter\expandafter\ltx@firstoftwo
    \fi
  \fi
}
%    \end{macrocode}
%    \end{macro}
%
% \subsection{Setup macros}
%
%    \begin{macro}{\hologoSetup}
%    \begin{macrocode}
\def\hologoSetup{%
  \let\HOLOGO@name\relax
  \HOLOGO@Setup
}
%    \end{macrocode}
%    \end{macro}
%
%    \begin{macro}{\hologoLogoSetup}
%    \begin{macrocode}
\def\hologoLogoSetup#1{%
  \edef\HOLOGO@name{#1}%
  \ltx@IfUndefined{HoLogo@\HOLOGO@name}{%
    \@PackageError{hologo}{%
      Unknown logo `\HOLOGO@name'%
    }\@ehc
    \ltx@gobble
  }{%
    \HOLOGO@Setup
  }%
}
%    \end{macrocode}
%    \end{macro}
%
%    \begin{macro}{\HOLOGO@Setup}
%    \begin{macrocode}
\def\HOLOGO@Setup{%
  \kvsetkeys{HoLogo}%
}
%    \end{macrocode}
%    \end{macro}
%
% \subsection{Options}
%
%    \begin{macro}{\HOLOGO@DeclareBoolOption}
%    \begin{macrocode}
\def\HOLOGO@DeclareBoolOption#1{%
  \expandafter\chardef\csname HOLOGOOPT@#1\endcsname\ltx@zero
  \kv@define@key{HoLogo}{#1}[true]{%
    \def\HOLOGO@temp{##1}%
    \ifx\HOLOGO@temp\HOLOGO@true
      \ifx\HOLOGO@name\relax
        \expandafter\chardef\csname HOLOGOOPT@#1\endcsname=\ltx@one
      \else
        \expandafter\chardef\csname
        HoLogoOpt@#1@\HOLOGO@name\endcsname\ltx@one
      \fi
      \HOLOGO@SetBreakAll{#1}%
    \else
      \ifx\HOLOGO@temp\HOLOGO@false
        \ifx\HOLOGO@name\relax
          \expandafter\chardef\csname HOLOGOOPT@#1\endcsname=\ltx@zero
        \else
          \expandafter\chardef\csname
          HoLogoOpt@#1@\HOLOGO@name\endcsname=\ltx@zero
        \fi
        \HOLOGO@SetBreakAll{#1}%
      \else
        \@PackageError{hologo}{%
          Unknown value `##1' for boolean option `#1'.\MessageBreak
          Known values are `true' and `false'%
        }\@ehc
      \fi
    \fi
  }%
}
%    \end{macrocode}
%    \end{macro}
%
%    \begin{macro}{\HOLOGO@SetBreakAll}
%    \begin{macrocode}
\def\HOLOGO@SetBreakAll#1{%
  \def\HOLOGO@temp{#1}%
  \ifx\HOLOGO@temp\HOLOGO@break
    \ifx\HOLOGO@name\relax
      \chardef\HOLOGOOPT@hyphenbreak=\HOLOGOOPT@break
      \chardef\HOLOGOOPT@spacebreak=\HOLOGOOPT@break
      \chardef\HOLOGOOPT@discretionarybreak=\HOLOGOOPT@break
    \else
      \expandafter\chardef
         \csname HoLogoOpt@hyphenbreak@\HOLOGO@name\endcsname=%
         \csname HoLogoOpt@break@\HOLOGO@name\endcsname
      \expandafter\chardef
         \csname HoLogoOpt@spacebreak@\HOLOGO@name\endcsname=%
         \csname HoLogoOpt@break@\HOLOGO@name\endcsname
      \expandafter\chardef
         \csname HoLogoOpt@discretionarybreak@\HOLOGO@name
             \endcsname=%
         \csname HoLogoOpt@break@\HOLOGO@name\endcsname
    \fi
  \fi
}
%    \end{macrocode}
%    \end{macro}
%
%    \begin{macro}{\HOLOGO@true}
%    \begin{macrocode}
\def\HOLOGO@true{true}
%    \end{macrocode}
%    \end{macro}
%    \begin{macro}{\HOLOGO@false}
%    \begin{macrocode}
\def\HOLOGO@false{false}
%    \end{macrocode}
%    \end{macro}
%    \begin{macro}{\HOLOGO@break}
%    \begin{macrocode}
\def\HOLOGO@break{break}
%    \end{macrocode}
%    \end{macro}
%
%    \begin{macrocode}
\HOLOGO@DeclareBoolOption{break}
\HOLOGO@DeclareBoolOption{hyphenbreak}
\HOLOGO@DeclareBoolOption{spacebreak}
\HOLOGO@DeclareBoolOption{discretionarybreak}
%    \end{macrocode}
%
%    \begin{macrocode}
\kv@define@key{HoLogo}{variant}{%
  \ifx\HOLOGO@name\relax
    \@PackageError{hologo}{%
      Option `variant' is not available in \string\hologoSetup,%
      \MessageBreak
      Use \string\hologoLogoSetup\space instead%
    }\@ehc
  \else
    \edef\HOLOGO@temp{#1}%
    \ifx\HOLOGO@temp\ltx@empty
      \expandafter
      \let\csname HoLogoOpt@variant@\HOLOGO@name\endcsname\@undefined
    \else
      \ltx@IfUndefined{HoLogo@\HOLOGO@name @\HOLOGO@temp}{%
        \@PackageError{hologo}{%
          Unknown variant `\HOLOGO@temp' of logo `\HOLOGO@name'%
        }\@ehc
      }{%
        \expandafter
        \let\csname HoLogoOpt@variant@\HOLOGO@name\endcsname
            \HOLOGO@temp
      }%
    \fi
  \fi
}
%    \end{macrocode}
%
%    \begin{macro}{\HOLOGO@Variant}
%    \begin{macrocode}
\def\HOLOGO@Variant#1{%
  #1%
  \ltx@ifundefined{HoLogoOpt@variant@#1}{%
  }{%
    @\csname HoLogoOpt@variant@#1\endcsname
  }%
}
%    \end{macrocode}
%    \end{macro}
%
% \subsection{Break/no-break support}
%
%    \begin{macro}{\HOLOGO@space}
%    \begin{macrocode}
\def\HOLOGO@space{%
  \ltx@ifundefined{HoLogoOpt@spacebreak@\HOLOGO@name}{%
    \ltx@ifundefined{HoLogoOpt@break@\HOLOGO@name}{%
      \chardef\HOLOGO@temp=\HOLOGOOPT@spacebreak
    }{%
      \chardef\HOLOGO@temp=%
        \csname HoLogoOpt@break@\HOLOGO@name\endcsname
    }%
  }{%
    \chardef\HOLOGO@temp=%
      \csname HoLogoOpt@spacebreak@\HOLOGO@name\endcsname
  }%
  \ifcase\HOLOGO@temp
    \penalty10000 %
  \fi
  \ltx@space
}
%    \end{macrocode}
%    \end{macro}
%
%    \begin{macro}{\HOLOGO@hyphen}
%    \begin{macrocode}
\def\HOLOGO@hyphen{%
  \ltx@ifundefined{HoLogoOpt@hyphenbreak@\HOLOGO@name}{%
    \ltx@ifundefined{HoLogoOpt@break@\HOLOGO@name}{%
      \chardef\HOLOGO@temp=\HOLOGOOPT@hyphenbreak
    }{%
      \chardef\HOLOGO@temp=%
        \csname HoLogoOpt@break@\HOLOGO@name\endcsname
    }%
  }{%
    \chardef\HOLOGO@temp=%
      \csname HoLogoOpt@hyphenbreak@\HOLOGO@name\endcsname
  }%
  \ifcase\HOLOGO@temp
    \ltx@mbox{-}%
  \else
    -%
  \fi
}
%    \end{macrocode}
%    \end{macro}
%
%    \begin{macro}{\HOLOGO@discretionary}
%    \begin{macrocode}
\def\HOLOGO@discretionary{%
  \ltx@ifundefined{HoLogoOpt@discretionarybreak@\HOLOGO@name}{%
    \ltx@ifundefined{HoLogoOpt@break@\HOLOGO@name}{%
      \chardef\HOLOGO@temp=\HOLOGOOPT@discretionarybreak
    }{%
      \chardef\HOLOGO@temp=%
        \csname HoLogoOpt@break@\HOLOGO@name\endcsname
    }%
  }{%
    \chardef\HOLOGO@temp=%
      \csname HoLogoOpt@discretionarybreak@\HOLOGO@name\endcsname
  }%
  \ifcase\HOLOGO@temp
  \else
    \-%
  \fi
}
%    \end{macrocode}
%    \end{macro}
%
%    \begin{macro}{\HOLOGO@mbox}
%    \begin{macrocode}
\def\HOLOGO@mbox#1{%
  \ltx@ifundefined{HoLogoOpt@break@\HOLOGO@name}{%
    \chardef\HOLOGO@temp=\HOLOGOOPT@hyphenbreak
  }{%
    \chardef\HOLOGO@temp=%
      \csname HoLogoOpt@break@\HOLOGO@name\endcsname
  }%
  \ifcase\HOLOGO@temp
    \ltx@mbox{#1}%
  \else
    #1%
  \fi
}
%    \end{macrocode}
%    \end{macro}
%
% \subsection{Font support}
%
%    \begin{macro}{\HoLogoFont@font}
%    \begin{tabular}{@{}ll@{}}
%    |#1|:& logo name\\
%    |#2|:& font short name\\
%    |#3|:& text
%    \end{tabular}
%    \begin{macrocode}
\def\HoLogoFont@font#1#2#3{%
  \begingroup
    \ltx@IfUndefined{HoLogoFont@logo@#1.#2}{%
      \ltx@IfUndefined{HoLogoFont@font@#2}{%
        \@PackageWarning{hologo}{%
          Missing font `#2' for logo `#1'%
        }%
        #3%
      }{%
        \csname HoLogoFont@font@#2\endcsname{#3}%
      }%
    }{%
      \csname HoLogoFont@logo@#1.#2\endcsname{#3}%
    }%
  \endgroup
}
%    \end{macrocode}
%    \end{macro}
%
%    \begin{macro}{\HoLogoFont@Def}
%    \begin{macrocode}
\def\HoLogoFont@Def#1{%
  \expandafter\def\csname HoLogoFont@font@#1\endcsname
}
%    \end{macrocode}
%    \end{macro}
%    \begin{macro}{\HoLogoFont@LogoDef}
%    \begin{macrocode}
\def\HoLogoFont@LogoDef#1#2{%
  \expandafter\def\csname HoLogoFont@logo@#1.#2\endcsname
}
%    \end{macrocode}
%    \end{macro}
%
% \subsubsection{Font defaults}
%
%    \begin{macro}{\HoLogoFont@font@general}
%    \begin{macrocode}
\HoLogoFont@Def{general}{}%
%    \end{macrocode}
%    \end{macro}
%
%    \begin{macro}{\HoLogoFont@font@rm}
%    \begin{macrocode}
\ltx@IfUndefined{rmfamily}{%
  \ltx@IfUndefined{rm}{%
  }{%
    \HoLogoFont@Def{rm}{\rm}%
  }%
}{%
  \HoLogoFont@Def{rm}{\rmfamily}%
}
%    \end{macrocode}
%    \end{macro}
%
%    \begin{macro}{\HoLogoFont@font@sf}
%    \begin{macrocode}
\ltx@IfUndefined{sffamily}{%
  \ltx@IfUndefined{sf}{%
  }{%
    \HoLogoFont@Def{sf}{\sf}%
  }%
}{%
  \HoLogoFont@Def{sf}{\sffamily}%
}
%    \end{macrocode}
%    \end{macro}
%
%    \begin{macro}{\HoLogoFont@font@bibsf}
%    In case of \hologo{plainTeX} the original small caps
%    variant is used as default. In \hologo{LaTeX}
%    the definition of package \xpackage{dtklogos} \cite{dtklogos}
%    is used.
%\begin{quote}
%\begin{verbatim}
%\DeclareRobustCommand{\BibTeX}{%
%  B%
%  \kern-.05em%
%  \hbox{%
%    $\m@th$% %% force math size calculations
%    \csname S@\f@size\endcsname
%    \fontsize\sf@size\z@
%    \math@fontsfalse
%    \selectfont
%    I%
%    \kern-.025em%
%    B
%  }%
%  \kern-.08em%
%  \-%
%  \TeX
%}
%\end{verbatim}
%\end{quote}
%    \begin{macrocode}
\ltx@IfUndefined{selectfont}{%
  \ltx@IfUndefined{tensc}{%
    \font\tensc=cmcsc10\relax
  }{}%
  \HoLogoFont@Def{bibsf}{\tensc}%
}{%
  \HoLogoFont@Def{bibsf}{%
    $\mathsurround=0pt$%
    \csname S@\f@size\endcsname
    \fontsize\sf@size{0pt}%
    \math@fontsfalse
    \selectfont
  }%
}
%    \end{macrocode}
%    \end{macro}
%
%    \begin{macro}{\HoLogoFont@font@sc}
%    \begin{macrocode}
\ltx@IfUndefined{scshape}{%
  \ltx@IfUndefined{tensc}{%
    \font\tensc=cmcsc10\relax
  }{}%
  \HoLogoFont@Def{sc}{\tensc}%
}{%
  \HoLogoFont@Def{sc}{\scshape}%
}
%    \end{macrocode}
%    \end{macro}
%
%    \begin{macro}{\HoLogoFont@font@sy}
%    \begin{macrocode}
\ltx@IfUndefined{usefont}{%
  \ltx@IfUndefined{tensy}{%
  }{%
    \HoLogoFont@Def{sy}{\tensy}%
  }%
}{%
  \HoLogoFont@Def{sy}{%
    \usefont{OMS}{cmsy}{m}{n}%
  }%
}
%    \end{macrocode}
%    \end{macro}
%
%    \begin{macro}{\HoLogoFont@font@logo}
%    \begin{macrocode}
\begingroup
  \def\x{LaTeX2e}%
\expandafter\endgroup
\ifx\fmtname\x
  \ltx@IfUndefined{logofamily}{%
    \DeclareRobustCommand\logofamily{%
      \not@math@alphabet\logofamily\relax
      \fontencoding{U}%
      \fontfamily{logo}%
      \selectfont
    }%
  }{}%
  \ltx@IfUndefined{logofamily}{%
  }{%
    \HoLogoFont@Def{logo}{\logofamily}%
  }%
\else
  \ltx@IfUndefined{tenlogo}{%
    \font\tenlogo=logo10\relax
  }{}%
  \HoLogoFont@Def{logo}{\tenlogo}%
\fi
%    \end{macrocode}
%    \end{macro}
%
% \subsubsection{Font setup}
%
%    \begin{macro}{\hologoFontSetup}
%    \begin{macrocode}
\def\hologoFontSetup{%
  \let\HOLOGO@name\relax
  \HOLOGO@FontSetup
}
%    \end{macrocode}
%    \end{macro}
%
%    \begin{macro}{\hologoLogoFontSetup}
%    \begin{macrocode}
\def\hologoLogoFontSetup#1{%
  \edef\HOLOGO@name{#1}%
  \ltx@IfUndefined{HoLogo@\HOLOGO@name}{%
    \@PackageError{hologo}{%
      Unknown logo `\HOLOGO@name'%
    }\@ehc
    \ltx@gobble
  }{%
    \HOLOGO@FontSetup
  }%
}
%    \end{macrocode}
%    \end{macro}
%
%    \begin{macro}{\HOLOGO@FontSetup}
%    \begin{macrocode}
\def\HOLOGO@FontSetup{%
  \kvsetkeys{HoLogoFont}%
}
%    \end{macrocode}
%    \end{macro}
%
%    \begin{macrocode}
\def\HOLOGO@temp#1{%
  \kv@define@key{HoLogoFont}{#1}{%
    \ifx\HOLOGO@name\relax
      \HoLogoFont@Def{#1}{##1}%
    \else
      \HoLogoFont@LogoDef\HOLOGO@name{#1}{##1}%
    \fi
  }%
}
\HOLOGO@temp{general}
\HOLOGO@temp{sf}
%    \end{macrocode}
%
% \subsection{Generic logo commands}
%
%    \begin{macrocode}
\HOLOGO@IfExists\hologo{%
  \@PackageError{hologo}{%
    \string\hologo\ltx@space is already defined.\MessageBreak
    Package loading is aborted%
  }\@ehc
  \HOLOGO@AtEnd
}%
\HOLOGO@IfExists\hologoRobust{%
  \@PackageError{hologo}{%
    \string\hologoRobust\ltx@space is already defined.\MessageBreak
    Package loading is aborted%
  }\@ehc
  \HOLOGO@AtEnd
}%
%    \end{macrocode}
%
% \subsubsection{\cs{hologo} and friends}
%
%    \begin{macrocode}
\ifluatex
  \expandafter\ltx@firstofone
\else
  \expandafter\ltx@gobble
\fi
{%
  \ltx@IfUndefined{ifincsname}{%
    \ifnum\luatexversion<36 %
      \expandafter\ltx@gobble
    \else
      \expandafter\ltx@firstofone
    \fi
    {%
      \begingroup
        \ifcase0%
            \directlua{%
              if tex.enableprimitives then %
                tex.enableprimitives('HOLOGO@', {'ifincsname'})%
              else %
                tex.print('1')%
              end%
            }%
            \ifx\HOLOGO@ifincsname\@undefined 1\fi%
            \relax
          \expandafter\ltx@firstofone
        \else
          \endgroup
          \expandafter\ltx@gobble
        \fi
        {%
          \global\let\ifincsname\HOLOGO@ifincsname
        }%
      \HOLOGO@temp
    }%
  }{}%
}
%    \end{macrocode}
%    \begin{macrocode}
\ltx@IfUndefined{ifincsname}{%
  \catcode`$=14 %
}{%
  \catcode`$=9 %
}
%    \end{macrocode}
%
%    \begin{macro}{\hologo}
%    \begin{macrocode}
\def\hologo#1{%
$ \ifincsname
$   \ltx@ifundefined{HoLogoCs@\HOLOGO@Variant{#1}}{%
$     #1%
$   }{%
$     \csname HoLogoCs@\HOLOGO@Variant{#1}\endcsname\ltx@firstoftwo
$   }%
$ \else
    \HOLOGO@IfExists\texorpdfstring\texorpdfstring\ltx@firstoftwo
    {%
      \hologoRobust{#1}%
    }{%
      \ltx@ifundefined{HoLogoBkm@\HOLOGO@Variant{#1}}{%
        \ltx@ifundefined{HoLogo@#1}{?#1?}{#1}%
      }{%
        \csname HoLogoBkm@\HOLOGO@Variant{#1}\endcsname
        \ltx@firstoftwo
      }%
    }%
$ \fi
}
%    \end{macrocode}
%    \end{macro}
%    \begin{macro}{\Hologo}
%    \begin{macrocode}
\def\Hologo#1{%
$ \ifincsname
$   \ltx@ifundefined{HoLogoCs@\HOLOGO@Variant{#1}}{%
$     #1%
$   }{%
$     \csname HoLogoCs@\HOLOGO@Variant{#1}\endcsname\ltx@secondoftwo
$   }%
$ \else
    \HOLOGO@IfExists\texorpdfstring\texorpdfstring\ltx@firstoftwo
    {%
      \HologoRobust{#1}%
    }{%
      \ltx@ifundefined{HoLogoBkm@\HOLOGO@Variant{#1}}{%
        \ltx@ifundefined{HoLogo@#1}{?#1?}{#1}%
      }{%
        \csname HoLogoBkm@\HOLOGO@Variant{#1}\endcsname
        \ltx@secondoftwo
      }%
    }%
$ \fi
}
%    \end{macrocode}
%    \end{macro}
%
%    \begin{macro}{\hologoVariant}
%    \begin{macrocode}
\def\hologoVariant#1#2{%
  \ifx\relax#2\relax
    \hologo{#1}%
  \else
$   \ifincsname
$     \ltx@ifundefined{HoLogoCs@#1@#2}{%
$       #1%
$     }{%
$       \csname HoLogoCs@#1@#2\endcsname\ltx@firstoftwo
$     }%
$   \else
      \HOLOGO@IfExists\texorpdfstring\texorpdfstring\ltx@firstoftwo
      {%
        \hologoVariantRobust{#1}{#2}%
      }{%
        \ltx@ifundefined{HoLogoBkm@#1@#2}{%
          \ltx@ifundefined{HoLogo@#1}{?#1?}{#1}%
        }{%
          \csname HoLogoBkm@#1@#2\endcsname
          \ltx@firstoftwo
        }%
      }%
$   \fi
  \fi
}
%    \end{macrocode}
%    \end{macro}
%    \begin{macro}{\HologoVariant}
%    \begin{macrocode}
\def\HologoVariant#1#2{%
  \ifx\relax#2\relax
    \Hologo{#1}%
  \else
$   \ifincsname
$     \ltx@ifundefined{HoLogoCs@#1@#2}{%
$       #1%
$     }{%
$       \csname HoLogoCs@#1@#2\endcsname\ltx@secondoftwo
$     }%
$   \else
      \HOLOGO@IfExists\texorpdfstring\texorpdfstring\ltx@firstoftwo
      {%
        \HologoVariantRobust{#1}{#2}%
      }{%
        \ltx@ifundefined{HoLogoBkm@#1@#2}{%
          \ltx@ifundefined{HoLogo@#1}{?#1?}{#1}%
        }{%
          \csname HoLogoBkm@#1@#2\endcsname
          \ltx@secondoftwo
        }%
      }%
$   \fi
  \fi
}
%    \end{macrocode}
%    \end{macro}
%
%    \begin{macrocode}
\catcode`\$=3 %
%    \end{macrocode}
%
% \subsubsection{\cs{hologoRobust} and friends}
%
%    \begin{macro}{\hologoRobust}
%    \begin{macrocode}
\ltx@IfUndefined{protected}{%
  \ltx@IfUndefined{DeclareRobustCommand}{%
    \def\hologoRobust#1%
  }{%
    \DeclareRobustCommand*\hologoRobust[1]%
  }%
}{%
  \protected\def\hologoRobust#1%
}%
{%
  \edef\HOLOGO@name{#1}%
  \ltx@IfUndefined{HoLogo@\HOLOGO@Variant\HOLOGO@name}{%
    \@PackageError{hologo}{%
      Unknown logo `\HOLOGO@name'%
    }\@ehc
    ?\HOLOGO@name?%
  }{%
    \ltx@IfUndefined{ver@tex4ht.sty}{%
      \HoLogoFont@font\HOLOGO@name{general}{%
        \csname HoLogo@\HOLOGO@Variant\HOLOGO@name\endcsname
        \ltx@firstoftwo
      }%
    }{%
      \ltx@IfUndefined{HoLogoHtml@\HOLOGO@Variant\HOLOGO@name}{%
        \HOLOGO@name
      }{%
        \csname HoLogoHtml@\HOLOGO@Variant\HOLOGO@name\endcsname
        \ltx@firstoftwo
      }%
    }%
  }%
}
%    \end{macrocode}
%    \end{macro}
%    \begin{macro}{\HologoRobust}
%    \begin{macrocode}
\ltx@IfUndefined{protected}{%
  \ltx@IfUndefined{DeclareRobustCommand}{%
    \def\HologoRobust#1%
  }{%
    \DeclareRobustCommand*\HologoRobust[1]%
  }%
}{%
  \protected\def\HologoRobust#1%
}%
{%
  \edef\HOLOGO@name{#1}%
  \ltx@IfUndefined{HoLogo@\HOLOGO@Variant\HOLOGO@name}{%
    \@PackageError{hologo}{%
      Unknown logo `\HOLOGO@name'%
    }\@ehc
    ?\HOLOGO@name?%
  }{%
    \ltx@IfUndefined{ver@tex4ht.sty}{%
      \HoLogoFont@font\HOLOGO@name{general}{%
        \csname HoLogo@\HOLOGO@Variant\HOLOGO@name\endcsname
        \ltx@secondoftwo
      }%
    }{%
      \ltx@IfUndefined{HoLogoHtml@\HOLOGO@Variant\HOLOGO@name}{%
        \expandafter\HOLOGO@Uppercase\HOLOGO@name
      }{%
        \csname HoLogoHtml@\HOLOGO@Variant\HOLOGO@name\endcsname
        \ltx@secondoftwo
      }%
    }%
  }%
}
%    \end{macrocode}
%    \end{macro}
%    \begin{macro}{\hologoVariantRobust}
%    \begin{macrocode}
\ltx@IfUndefined{protected}{%
  \ltx@IfUndefined{DeclareRobustCommand}{%
    \def\hologoVariantRobust#1#2%
  }{%
    \DeclareRobustCommand*\hologoVariantRobust[2]%
  }%
}{%
  \protected\def\hologoVariantRobust#1#2%
}%
{%
  \begingroup
    \hologoLogoSetup{#1}{variant={#2}}%
    \hologoRobust{#1}%
  \endgroup
}
%    \end{macrocode}
%    \end{macro}
%    \begin{macro}{\HologoVariantRobust}
%    \begin{macrocode}
\ltx@IfUndefined{protected}{%
  \ltx@IfUndefined{DeclareRobustCommand}{%
    \def\HologoVariantRobust#1#2%
  }{%
    \DeclareRobustCommand*\HologoVariantRobust[2]%
  }%
}{%
  \protected\def\HologoVariantRobust#1#2%
}%
{%
  \begingroup
    \hologoLogoSetup{#1}{variant={#2}}%
    \HologoRobust{#1}%
  \endgroup
}
%    \end{macrocode}
%    \end{macro}
%
%    \begin{macro}{\hologorobust}
%    Macro \cs{hologorobust} is only defined for compatibility.
%    Its use is deprecated.
%    \begin{macrocode}
\def\hologorobust{\hologoRobust}
%    \end{macrocode}
%    \end{macro}
%
% \subsection{Helpers}
%
%    \begin{macro}{\HOLOGO@Uppercase}
%    Macro \cs{HOLOGO@Uppercase} is restricted to \cs{uppercase},
%    because \hologo{plainTeX} or \hologo{iniTeX} do not provide
%    \cs{MakeUppercase}.
%    \begin{macrocode}
\def\HOLOGO@Uppercase#1{\uppercase{#1}}
%    \end{macrocode}
%    \end{macro}
%
%    \begin{macro}{\HOLOGO@PdfdocUnicode}
%    \begin{macrocode}
\def\HOLOGO@PdfdocUnicode{%
  \ifx\ifHy@unicode\iftrue
    \expandafter\ltx@secondoftwo
  \else
    \expandafter\ltx@firstoftwo
  \fi
}
%    \end{macrocode}
%    \end{macro}
%
%    \begin{macro}{\HOLOGO@Math}
%    \begin{macrocode}
\def\HOLOGO@MathSetup{%
  \mathsurround0pt\relax
  \HOLOGO@IfExists\f@series{%
    \if b\expandafter\ltx@car\f@series x\@nil
      \csname boldmath\endcsname
   \fi
  }{}%
}
%    \end{macrocode}
%    \end{macro}
%
%    \begin{macro}{\HOLOGO@TempDimen}
%    \begin{macrocode}
\dimendef\HOLOGO@TempDimen=\ltx@zero
%    \end{macrocode}
%    \end{macro}
%    \begin{macro}{\HOLOGO@NegativeKerning}
%    \begin{macrocode}
\def\HOLOGO@NegativeKerning#1{%
  \begingroup
    \HOLOGO@TempDimen=0pt\relax
    \comma@parse@normalized{#1}{%
      \ifdim\HOLOGO@TempDimen=0pt %
        \expandafter\HOLOGO@@NegativeKerning\comma@entry
      \fi
      \ltx@gobble
    }%
    \ifdim\HOLOGO@TempDimen<0pt %
      \kern\HOLOGO@TempDimen
    \fi
  \endgroup
}
%    \end{macrocode}
%    \end{macro}
%    \begin{macro}{\HOLOGO@@NegativeKerning}
%    \begin{macrocode}
\def\HOLOGO@@NegativeKerning#1#2{%
  \setbox\ltx@zero\hbox{#1#2}%
  \HOLOGO@TempDimen=\wd\ltx@zero
  \setbox\ltx@zero\hbox{#1\kern0pt#2}%
  \advance\HOLOGO@TempDimen by -\wd\ltx@zero
}
%    \end{macrocode}
%    \end{macro}
%
%    \begin{macro}{\HOLOGO@SpaceFactor}
%    \begin{macrocode}
\def\HOLOGO@SpaceFactor{%
  \spacefactor1000 %
}
%    \end{macrocode}
%    \end{macro}
%
%    \begin{macro}{\HOLOGO@Span}
%    \begin{macrocode}
\def\HOLOGO@Span#1#2{%
  \HCode{<span class="HoLogo-#1">}%
  #2%
  \HCode{</span>}%
}
%    \end{macrocode}
%    \end{macro}
%
% \subsubsection{Text subscript}
%
%    \begin{macro}{\HOLOGO@SubScript}%
%    \begin{macrocode}
\def\HOLOGO@SubScript#1{%
  \ltx@IfUndefined{textsubscript}{%
    \ltx@IfUndefined{text}{%
      \ltx@mbox{%
        \mathsurround=0pt\relax
        $%
          _{%
            \ltx@IfUndefined{sf@size}{%
              \mathrm{#1}%
            }{%
              \mbox{%
                \fontsize\sf@size{0pt}\selectfont
                #1%
              }%
            }%
          }%
        $%
      }%
    }{%
      \ltx@mbox{%
        \mathsurround=0pt\relax
        $_{\text{#1}}$%
      }%
    }%
  }{%
    \textsubscript{#1}%
  }%
}
%    \end{macrocode}
%    \end{macro}
%
% \subsection{\hologo{TeX} and friends}
%
% \subsubsection{\hologo{TeX}}
%
%    \begin{macro}{\HoLogo@TeX}
%    Source: \hologo{LaTeX} kernel.
%    \begin{macrocode}
\def\HoLogo@TeX#1{%
  T\kern-.1667em\lower.5ex\hbox{E}\kern-.125emX\HOLOGO@SpaceFactor
}
%    \end{macrocode}
%    \end{macro}
%    \begin{macro}{\HoLogoHtml@TeX}
%    \begin{macrocode}
\def\HoLogoHtml@TeX#1{%
  \HoLogoCss@TeX
  \HOLOGO@Span{TeX}{%
    T%
    \HOLOGO@Span{e}{%
      E%
    }%
    X%
  }%
}
%    \end{macrocode}
%    \end{macro}
%    \begin{macro}{\HoLogoCss@TeX}
%    \begin{macrocode}
\def\HoLogoCss@TeX{%
  \Css{%
    span.HoLogo-TeX span.HoLogo-e{%
      position:relative;%
      top:.5ex;%
      margin-left:-.1667em;%
      margin-right:-.125em;%
    }%
  }%
  \Css{%
    a span.HoLogo-TeX span.HoLogo-e{%
      text-decoration:none;%
    }%
  }%
  \global\let\HoLogoCss@TeX\relax
}
%    \end{macrocode}
%    \end{macro}
%
% \subsubsection{\hologo{plainTeX}}
%
%    \begin{macro}{\HoLogo@plainTeX@space}
%    Source: ``The \hologo{TeX}book''
%    \begin{macrocode}
\def\HoLogo@plainTeX@space#1{%
  \HOLOGO@mbox{#1{p}{P}lain}\HOLOGO@space\hologo{TeX}%
}
%    \end{macrocode}
%    \end{macro}
%    \begin{macro}{\HoLogoCs@plainTeX@space}
%    \begin{macrocode}
\def\HoLogoCs@plainTeX@space#1{#1{p}{P}lain TeX}%
%    \end{macrocode}
%    \end{macro}
%    \begin{macro}{\HoLogoBkm@plainTeX@space}
%    \begin{macrocode}
\def\HoLogoBkm@plainTeX@space#1{%
  #1{p}{P}lain \hologo{TeX}%
}
%    \end{macrocode}
%    \end{macro}
%    \begin{macro}{\HoLogoHtml@plainTeX@space}
%    \begin{macrocode}
\def\HoLogoHtml@plainTeX@space#1{%
  #1{p}{P}lain \hologo{TeX}%
}
%    \end{macrocode}
%    \end{macro}
%
%    \begin{macro}{\HoLogo@plainTeX@hyphen}
%    \begin{macrocode}
\def\HoLogo@plainTeX@hyphen#1{%
  \HOLOGO@mbox{#1{p}{P}lain}\HOLOGO@hyphen\hologo{TeX}%
}
%    \end{macrocode}
%    \end{macro}
%    \begin{macro}{\HoLogoCs@plainTeX@hyphen}
%    \begin{macrocode}
\def\HoLogoCs@plainTeX@hyphen#1{#1{p}{P}lain-TeX}
%    \end{macrocode}
%    \end{macro}
%    \begin{macro}{\HoLogoBkm@plainTeX@hyphen}
%    \begin{macrocode}
\def\HoLogoBkm@plainTeX@hyphen#1{%
  #1{p}{P}lain-\hologo{TeX}%
}
%    \end{macrocode}
%    \end{macro}
%    \begin{macro}{\HoLogoHtml@plainTeX@hyphen}
%    \begin{macrocode}
\def\HoLogoHtml@plainTeX@hyphen#1{%
  #1{p}{P}lain-\hologo{TeX}%
}
%    \end{macrocode}
%    \end{macro}
%
%    \begin{macro}{\HoLogo@plainTeX@runtogether}
%    \begin{macrocode}
\def\HoLogo@plainTeX@runtogether#1{%
  \HOLOGO@mbox{#1{p}{P}lain\hologo{TeX}}%
}
%    \end{macrocode}
%    \end{macro}
%    \begin{macro}{\HoLogoCs@plainTeX@runtogether}
%    \begin{macrocode}
\def\HoLogoCs@plainTeX@runtogether#1{#1{p}{P}lainTeX}
%    \end{macrocode}
%    \end{macro}
%    \begin{macro}{\HoLogoBkm@plainTeX@runtogether}
%    \begin{macrocode}
\def\HoLogoBkm@plainTeX@runtogether#1{%
  #1{p}{P}lain\hologo{TeX}%
}
%    \end{macrocode}
%    \end{macro}
%    \begin{macro}{\HoLogoHtml@plainTeX@runtogether}
%    \begin{macrocode}
\def\HoLogoHtml@plainTeX@runtogether#1{%
  #1{p}{P}lain\hologo{TeX}%
}
%    \end{macrocode}
%    \end{macro}
%
%    \begin{macro}{\HoLogo@plainTeX}
%    \begin{macrocode}
\def\HoLogo@plainTeX{\HoLogo@plainTeX@space}
%    \end{macrocode}
%    \end{macro}
%    \begin{macro}{\HoLogoCs@plainTeX}
%    \begin{macrocode}
\def\HoLogoCs@plainTeX{\HoLogoCs@plainTeX@space}
%    \end{macrocode}
%    \end{macro}
%    \begin{macro}{\HoLogoBkm@plainTeX}
%    \begin{macrocode}
\def\HoLogoBkm@plainTeX{\HoLogoBkm@plainTeX@space}
%    \end{macrocode}
%    \end{macro}
%    \begin{macro}{\HoLogoHtml@plainTeX}
%    \begin{macrocode}
\def\HoLogoHtml@plainTeX{\HoLogoHtml@plainTeX@space}
%    \end{macrocode}
%    \end{macro}
%
% \subsubsection{\hologo{LaTeX}}
%
%    Source: \hologo{LaTeX} kernel.
%\begin{quote}
%\begin{verbatim}
%\DeclareRobustCommand{\LaTeX}{%
%  L%
%  \kern-.36em%
%  {%
%    \sbox\z@ T%
%    \vbox to\ht\z@{%
%      \hbox{%
%        \check@mathfonts
%        \fontsize\sf@size\z@
%        \math@fontsfalse
%        \selectfont
%        A%
%      }%
%      \vss
%    }%
%  }%
%  \kern-.15em%
%  \TeX
%}
%\end{verbatim}
%\end{quote}
%
%    \begin{macro}{\HoLogo@La}
%    \begin{macrocode}
\def\HoLogo@La#1{%
  L%
  \kern-.36em%
  \begingroup
    \setbox\ltx@zero\hbox{T}%
    \vbox to\ht\ltx@zero{%
      \hbox{%
        \ltx@ifundefined{check@mathfonts}{%
          \csname sevenrm\endcsname
        }{%
          \check@mathfonts
          \fontsize\sf@size{0pt}%
          \math@fontsfalse\selectfont
        }%
        A%
      }%
      \vss
    }%
  \endgroup
}
%    \end{macrocode}
%    \end{macro}
%
%    \begin{macro}{\HoLogo@LaTeX}
%    Source: \hologo{LaTeX} kernel.
%    \begin{macrocode}
\def\HoLogo@LaTeX#1{%
  \hologo{La}%
  \kern-.15em%
  \hologo{TeX}%
}
%    \end{macrocode}
%    \end{macro}
%    \begin{macro}{\HoLogoHtml@LaTeX}
%    \begin{macrocode}
\def\HoLogoHtml@LaTeX#1{%
  \HoLogoCss@LaTeX
  \HOLOGO@Span{LaTeX}{%
    L%
    \HOLOGO@Span{a}{%
      A%
    }%
    \hologo{TeX}%
  }%
}
%    \end{macrocode}
%    \end{macro}
%    \begin{macro}{\HoLogoCss@LaTeX}
%    \begin{macrocode}
\def\HoLogoCss@LaTeX{%
  \Css{%
    span.HoLogo-LaTeX span.HoLogo-a{%
      position:relative;%
      top:-.5ex;%
      margin-left:-.36em;%
      margin-right:-.15em;%
      font-size:85\%;%
    }%
  }%
  \global\let\HoLogoCss@LaTeX\relax
}
%    \end{macrocode}
%    \end{macro}
%
% \subsubsection{\hologo{(La)TeX}}
%
%    \begin{macro}{\HoLogo@LaTeXTeX}
%    The kerning around the parentheses is taken
%    from package \xpackage{dtklogos} \cite{dtklogos}.
%\begin{quote}
%\begin{verbatim}
%\DeclareRobustCommand{\LaTeXTeX}{%
%  (%
%  \kern-.15em%
%  L%
%  \kern-.36em%
%  {%
%    \sbox\z@ T%
%    \vbox to\ht0{%
%      \hbox{%
%        $\m@th$%
%        \csname S@\f@size\endcsname
%        \fontsize\sf@size\z@
%        \math@fontsfalse
%        \selectfont
%        A%
%      }%
%      \vss
%    }%
%  }%
%  \kern-.2em%
%  )%
%  \kern-.15em%
%  \TeX
%}
%\end{verbatim}
%\end{quote}
%    \begin{macrocode}
\def\HoLogo@LaTeXTeX#1{%
  (%
  \kern-.15em%
  \hologo{La}%
  \kern-.2em%
  )%
  \kern-.15em%
  \hologo{TeX}%
}
%    \end{macrocode}
%    \end{macro}
%    \begin{macro}{\HoLogoBkm@LaTeXTeX}
%    \begin{macrocode}
\def\HoLogoBkm@LaTeXTeX#1{(La)TeX}
%    \end{macrocode}
%    \end{macro}
%
%    \begin{macro}{\HoLogo@(La)TeX}
%    \begin{macrocode}
\expandafter
\let\csname HoLogo@(La)TeX\endcsname\HoLogo@LaTeXTeX
%    \end{macrocode}
%    \end{macro}
%    \begin{macro}{\HoLogoBkm@(La)TeX}
%    \begin{macrocode}
\expandafter
\let\csname HoLogoBkm@(La)TeX\endcsname\HoLogoBkm@LaTeXTeX
%    \end{macrocode}
%    \end{macro}
%    \begin{macro}{\HoLogoHtml@LaTeXTeX}
%    \begin{macrocode}
\def\HoLogoHtml@LaTeXTeX#1{%
  \HoLogoCss@LaTeXTeX
  \HOLOGO@Span{LaTeXTeX}{%
    (%
    \HOLOGO@Span{L}{L}%
    \HOLOGO@Span{a}{A}%
    \HOLOGO@Span{ParenRight}{)}%
    \hologo{TeX}%
  }%
}
%    \end{macrocode}
%    \end{macro}
%    \begin{macro}{\HoLogoHtml@(La)TeX}
%    Kerning after opening parentheses and before closing parentheses
%    is $-0.1$\,em. The original values $-0.15$\,em
%    looked too ugly for a serif font.
%    \begin{macrocode}
\expandafter
\let\csname HoLogoHtml@(La)TeX\endcsname\HoLogoHtml@LaTeXTeX
%    \end{macrocode}
%    \end{macro}
%    \begin{macro}{\HoLogoCss@LaTeXTeX}
%    \begin{macrocode}
\def\HoLogoCss@LaTeXTeX{%
  \Css{%
    span.HoLogo-LaTeXTeX span.HoLogo-L{%
      margin-left:-.1em;%
    }%
  }%
  \Css{%
    span.HoLogo-LaTeXTeX span.HoLogo-a{%
      position:relative;%
      top:-.5ex;%
      margin-left:-.36em;%
      margin-right:-.1em;%
      font-size:85\%;%
    }%
  }%
  \Css{%
    span.HoLogo-LaTeXTeX span.HoLogo-ParenRight{%
      margin-right:-.15em;%
    }%
  }%
  \global\let\HoLogoCss@LaTeXTeX\relax
}
%    \end{macrocode}
%    \end{macro}
%
% \subsubsection{\hologo{LaTeXe}}
%
%    \begin{macro}{\HoLogo@LaTeXe}
%    Source: \hologo{LaTeX} kernel
%    \begin{macrocode}
\def\HoLogo@LaTeXe#1{%
  \hologo{LaTeX}%
  \kern.15em%
  \hbox{%
    \HOLOGO@MathSetup
    2%
    $_{\textstyle\varepsilon}$%
  }%
}
%    \end{macrocode}
%    \end{macro}
%
%    \begin{macro}{\HoLogoCs@LaTeXe}
%    \begin{macrocode}
\ifnum64=`\^^^^0040\relax % test for big chars of LuaTeX/XeTeX
  \catcode`\$=9 %
  \catcode`\&=14 %
\else
  \catcode`\$=14 %
  \catcode`\&=9 %
\fi
\def\HoLogoCs@LaTeXe#1{%
  LaTeX2%
$ \string ^^^^0395%
& e%
}%
\catcode`\$=3 %
\catcode`\&=4 %
%    \end{macrocode}
%    \end{macro}
%
%    \begin{macro}{\HoLogoBkm@LaTeXe}
%    \begin{macrocode}
\def\HoLogoBkm@LaTeXe#1{%
  \hologo{LaTeX}%
  2%
  \HOLOGO@PdfdocUnicode{e}{\textepsilon}%
}
%    \end{macrocode}
%    \end{macro}
%
%    \begin{macro}{\HoLogoHtml@LaTeXe}
%    \begin{macrocode}
\def\HoLogoHtml@LaTeXe#1{%
  \HoLogoCss@LaTeXe
  \HOLOGO@Span{LaTeX2e}{%
    \hologo{LaTeX}%
    \HOLOGO@Span{2}{2}%
    \HOLOGO@Span{e}{%
      \HOLOGO@MathSetup
      \ensuremath{\textstyle\varepsilon}%
    }%
  }%
}
%    \end{macrocode}
%    \end{macro}
%    \begin{macro}{\HoLogoCss@LaTeXe}
%    \begin{macrocode}
\def\HoLogoCss@LaTeXe{%
  \Css{%
    span.HoLogo-LaTeX2e span.HoLogo-2{%
      padding-left:.15em;%
    }%
  }%
  \Css{%
    span.HoLogo-LaTeX2e span.HoLogo-e{%
      position:relative;%
      top:.35ex;%
      text-decoration:none;%
    }%
  }%
  \global\let\HoLogoCss@LaTeXe\relax
}
%    \end{macrocode}
%    \end{macro}
%
%    \begin{macro}{\HoLogo@LaTeX2e}
%    \begin{macrocode}
\expandafter
\let\csname HoLogo@LaTeX2e\endcsname\HoLogo@LaTeXe
%    \end{macrocode}
%    \end{macro}
%    \begin{macro}{\HoLogoCs@LaTeX2e}
%    \begin{macrocode}
\expandafter
\let\csname HoLogoCs@LaTeX2e\endcsname\HoLogoCs@LaTeXe
%    \end{macrocode}
%    \end{macro}
%    \begin{macro}{\HoLogoBkm@LaTeX2e}
%    \begin{macrocode}
\expandafter
\let\csname HoLogoBkm@LaTeX2e\endcsname\HoLogoBkm@LaTeXe
%    \end{macrocode}
%    \end{macro}
%    \begin{macro}{\HoLogoHtml@LaTeX2e}
%    \begin{macrocode}
\expandafter
\let\csname HoLogoHtml@LaTeX2e\endcsname\HoLogoHtml@LaTeXe
%    \end{macrocode}
%    \end{macro}
%
% \subsubsection{\hologo{LaTeX3}}
%
%    \begin{macro}{\HoLogo@LaTeX3}
%    Source: \hologo{LaTeX} kernel
%    \begin{macrocode}
\expandafter\def\csname HoLogo@LaTeX3\endcsname#1{%
  \hologo{LaTeX}%
  3%
}
%    \end{macrocode}
%    \end{macro}
%
%    \begin{macro}{\HoLogoBkm@LaTeX3}
%    \begin{macrocode}
\expandafter\def\csname HoLogoBkm@LaTeX3\endcsname#1{%
  \hologo{LaTeX}%
  3%
}
%    \end{macrocode}
%    \end{macro}
%    \begin{macro}{\HoLogoHtml@LaTeX3}
%    \begin{macrocode}
\expandafter
\let\csname HoLogoHtml@LaTeX3\expandafter\endcsname
\csname HoLogo@LaTeX3\endcsname
%    \end{macrocode}
%    \end{macro}
%
% \subsubsection{\hologo{LaTeXML}}
%
%    \begin{macro}{\HoLogo@LaTeXML}
%    \begin{macrocode}
\def\HoLogo@LaTeXML#1{%
  \HOLOGO@mbox{%
    \hologo{La}%
    \kern-.15em%
    T%
    \kern-.1667em%
    \lower.5ex\hbox{E}%
    \kern-.125em%
    \HoLogoFont@font{LaTeXML}{sc}{xml}%
  }%
}
%    \end{macrocode}
%    \end{macro}
%    \begin{macro}{\HoLogoHtml@pdfLaTeX}
%    \begin{macrocode}
\def\HoLogoHtml@LaTeXML#1{%
  \HOLOGO@Span{LaTeXML}{%
    \HoLogoCss@LaTeX
    \HoLogoCss@TeX
    \HOLOGO@Span{LaTeX}{%
      L%
      \HOLOGO@Span{a}{%
        A%
      }%
    }%
    \HOLOGO@Span{TeX}{%
      T%
      \HOLOGO@Span{e}{%
        E%
      }%
    }%
    \HCode{<span style="font-variant: small-caps;">}%
    xml%
    \HCode{</span>}%
  }%
}
%    \end{macrocode}
%    \end{macro}
%
% \subsubsection{\hologo{eTeX}}
%
%    \begin{macro}{\HoLogo@eTeX}
%    Source: package \xpackage{etex}
%    \begin{macrocode}
\def\HoLogo@eTeX#1{%
  \ltx@mbox{%
    \HOLOGO@MathSetup
    $\varepsilon$%
    -%
    \HOLOGO@NegativeKerning{-T,T-,To}%
    \hologo{TeX}%
  }%
}
%    \end{macrocode}
%    \end{macro}
%    \begin{macro}{\HoLogoCs@eTeX}
%    \begin{macrocode}
\ifnum64=`\^^^^0040\relax % test for big chars of LuaTeX/XeTeX
  \catcode`\$=9 %
  \catcode`\&=14 %
\else
  \catcode`\$=14 %
  \catcode`\&=9 %
\fi
\def\HoLogoCs@eTeX#1{%
$ #1{\string ^^^^0395}{\string ^^^^03b5}%
& #1{e}{E}%
  TeX%
}%
\catcode`\$=3 %
\catcode`\&=4 %
%    \end{macrocode}
%    \end{macro}
%    \begin{macro}{\HoLogoBkm@eTeX}
%    \begin{macrocode}
\def\HoLogoBkm@eTeX#1{%
  \HOLOGO@PdfdocUnicode{#1{e}{E}}{\textepsilon}%
  -%
  \hologo{TeX}%
}
%    \end{macrocode}
%    \end{macro}
%    \begin{macro}{\HoLogoHtml@eTeX}
%    \begin{macrocode}
\def\HoLogoHtml@eTeX#1{%
  \ltx@mbox{%
    \HOLOGO@MathSetup
    $\varepsilon$%
    -%
    \hologo{TeX}%
  }%
}
%    \end{macrocode}
%    \end{macro}
%
% \subsubsection{\hologo{iniTeX}}
%
%    \begin{macro}{\HoLogo@iniTeX}
%    \begin{macrocode}
\def\HoLogo@iniTeX#1{%
  \HOLOGO@mbox{%
    #1{i}{I}ni\hologo{TeX}%
  }%
}
%    \end{macrocode}
%    \end{macro}
%    \begin{macro}{\HoLogoCs@iniTeX}
%    \begin{macrocode}
\def\HoLogoCs@iniTeX#1{#1{i}{I}niTeX}
%    \end{macrocode}
%    \end{macro}
%    \begin{macro}{\HoLogoBkm@iniTeX}
%    \begin{macrocode}
\def\HoLogoBkm@iniTeX#1{%
  #1{i}{I}ni\hologo{TeX}%
}
%    \end{macrocode}
%    \end{macro}
%    \begin{macro}{\HoLogoHtml@iniTeX}
%    \begin{macrocode}
\let\HoLogoHtml@iniTeX\HoLogo@iniTeX
%    \end{macrocode}
%    \end{macro}
%
% \subsubsection{\hologo{virTeX}}
%
%    \begin{macro}{\HoLogo@virTeX}
%    \begin{macrocode}
\def\HoLogo@virTeX#1{%
  \HOLOGO@mbox{%
    #1{v}{V}ir\hologo{TeX}%
  }%
}
%    \end{macrocode}
%    \end{macro}
%    \begin{macro}{\HoLogoCs@virTeX}
%    \begin{macrocode}
\def\HoLogoCs@virTeX#1{#1{v}{V}irTeX}
%    \end{macrocode}
%    \end{macro}
%    \begin{macro}{\HoLogoBkm@virTeX}
%    \begin{macrocode}
\def\HoLogoBkm@virTeX#1{%
  #1{v}{V}ir\hologo{TeX}%
}
%    \end{macrocode}
%    \end{macro}
%    \begin{macro}{\HoLogoHtml@virTeX}
%    \begin{macrocode}
\let\HoLogoHtml@virTeX\HoLogo@virTeX
%    \end{macrocode}
%    \end{macro}
%
% \subsubsection{\hologo{SliTeX}}
%
% \paragraph{Definitions of the three variants.}
%
%    \begin{macro}{\HoLogo@SLiTeX@lift}
%    \begin{macrocode}
\def\HoLogo@SLiTeX@lift#1{%
  \HoLogoFont@font{SliTeX}{rm}{%
    S%
    \kern-.06em%
    L%
    \kern-.18em%
    \raise.32ex\hbox{\HoLogoFont@font{SliTeX}{sc}{i}}%
    \HOLOGO@discretionary
    \kern-.06em%
    \hologo{TeX}%
  }%
}
%    \end{macrocode}
%    \end{macro}
%    \begin{macro}{\HoLogoBkm@SLiTeX@lift}
%    \begin{macrocode}
\def\HoLogoBkm@SLiTeX@lift#1{SLiTeX}
%    \end{macrocode}
%    \end{macro}
%    \begin{macro}{\HoLogoHtml@SLiTeX@lift}
%    \begin{macrocode}
\def\HoLogoHtml@SLiTeX@lift#1{%
  \HoLogoCss@SLiTeX@lift
  \HOLOGO@Span{SLiTeX-lift}{%
    \HoLogoFont@font{SliTeX}{rm}{%
      S%
      \HOLOGO@Span{L}{L}%
      \HOLOGO@Span{i}{i}%
      \hologo{TeX}%
    }%
  }%
}
%    \end{macrocode}
%    \end{macro}
%    \begin{macro}{\HoLogoCss@SLiTeX@lift}
%    \begin{macrocode}
\def\HoLogoCss@SLiTeX@lift{%
  \Css{%
    span.HoLogo-SLiTeX-lift span.HoLogo-L{%
      margin-left:-.06em;%
      margin-right:-.18em;%
    }%
  }%
  \Css{%
    span.HoLogo-SLiTeX-lift span.HoLogo-i{%
      position:relative;%
      top:-.32ex;%
      margin-right:-.06em;%
      font-variant:small-caps;%
    }%
  }%
  \global\let\HoLogoCss@SLiTeX@lift\relax
}
%    \end{macrocode}
%    \end{macro}
%
%    \begin{macro}{\HoLogo@SliTeX@simple}
%    \begin{macrocode}
\def\HoLogo@SliTeX@simple#1{%
  \HoLogoFont@font{SliTeX}{rm}{%
    \ltx@mbox{%
      \HoLogoFont@font{SliTeX}{sc}{Sli}%
    }%
    \HOLOGO@discretionary
    \hologo{TeX}%
  }%
}
%    \end{macrocode}
%    \end{macro}
%    \begin{macro}{\HoLogoBkm@SliTeX@simple}
%    \begin{macrocode}
\def\HoLogoBkm@SliTeX@simple#1{SliTeX}
%    \end{macrocode}
%    \end{macro}
%    \begin{macro}{\HoLogoHtml@SliTeX@simple}
%    \begin{macrocode}
\let\HoLogoHtml@SliTeX@simple\HoLogo@SliTeX@simple
%    \end{macrocode}
%    \end{macro}
%
%    \begin{macro}{\HoLogo@SliTeX@narrow}
%    \begin{macrocode}
\def\HoLogo@SliTeX@narrow#1{%
  \HoLogoFont@font{SliTeX}{rm}{%
    \ltx@mbox{%
      S%
      \kern-.06em%
      \HoLogoFont@font{SliTeX}{sc}{%
        l%
        \kern-.035em%
        i%
      }%
    }%
    \HOLOGO@discretionary
    \kern-.06em%
    \hologo{TeX}%
  }%
}
%    \end{macrocode}
%    \end{macro}
%    \begin{macro}{\HoLogoBkm@SliTeX@narrow}
%    \begin{macrocode}
\def\HoLogoBkm@SliTeX@narrow#1{SliTeX}
%    \end{macrocode}
%    \end{macro}
%    \begin{macro}{\HoLogoHtml@SliTeX@narrow}
%    \begin{macrocode}
\def\HoLogoHtml@SliTeX@narrow#1{%
  \HoLogoCss@SliTeX@narrow
  \HOLOGO@Span{SliTeX-narrow}{%
    \HoLogoFont@font{SliTeX}{rm}{%
      S%
        \HOLOGO@Span{l}{l}%
        \HOLOGO@Span{i}{i}%
      \hologo{TeX}%
    }%
  }%
}
%    \end{macrocode}
%    \end{macro}
%    \begin{macro}{\HoLogoCss@SliTeX@narrow}
%    \begin{macrocode}
\def\HoLogoCss@SliTeX@narrow{%
  \Css{%
    span.HoLogo-SliTeX-narrow span.HoLogo-l{%
      margin-left:-.06em;%
      margin-right:-.035em;%
      font-variant:small-caps;%
    }%
  }%
  \Css{%
    span.HoLogo-SliTeX-narrow span.HoLogo-i{%
      margin-right:-.06em;%
      font-variant:small-caps;%
    }%
  }%
  \global\let\HoLogoCss@SliTeX@narrow\relax
}
%    \end{macrocode}
%    \end{macro}
%
% \paragraph{Macro set completion.}
%
%    \begin{macro}{\HoLogo@SLiTeX@simple}
%    \begin{macrocode}
\def\HoLogo@SLiTeX@simple{\HoLogo@SliTeX@simple}
%    \end{macrocode}
%    \end{macro}
%    \begin{macro}{\HoLogoBkm@SLiTeX@simple}
%    \begin{macrocode}
\def\HoLogoBkm@SLiTeX@simple{\HoLogoBkm@SliTeX@simple}
%    \end{macrocode}
%    \end{macro}
%    \begin{macro}{\HoLogoHtml@SLiTeX@simple}
%    \begin{macrocode}
\def\HoLogoHtml@SLiTeX@simple{\HoLogoHtml@SliTeX@simple}
%    \end{macrocode}
%    \end{macro}
%
%    \begin{macro}{\HoLogo@SLiTeX@narrow}
%    \begin{macrocode}
\def\HoLogo@SLiTeX@narrow{\HoLogo@SliTeX@narrow}
%    \end{macrocode}
%    \end{macro}
%    \begin{macro}{\HoLogoBkm@SLiTeX@narrow}
%    \begin{macrocode}
\def\HoLogoBkm@SLiTeX@narrow{\HoLogoBkm@SliTeX@narrow}
%    \end{macrocode}
%    \end{macro}
%    \begin{macro}{\HoLogoHtml@SLiTeX@narrow}
%    \begin{macrocode}
\def\HoLogoHtml@SLiTeX@narrow{\HoLogoHtml@SliTeX@narrow}
%    \end{macrocode}
%    \end{macro}
%
%    \begin{macro}{\HoLogo@SliTeX@lift}
%    \begin{macrocode}
\def\HoLogo@SliTeX@lift{\HoLogo@SLiTeX@lift}
%    \end{macrocode}
%    \end{macro}
%    \begin{macro}{\HoLogoBkm@SliTeX@lift}
%    \begin{macrocode}
\def\HoLogoBkm@SliTeX@lift{\HoLogoBkm@SLiTeX@lift}
%    \end{macrocode}
%    \end{macro}
%    \begin{macro}{\HoLogoHtml@SliTeX@lift}
%    \begin{macrocode}
\def\HoLogoHtml@SliTeX@lift{\HoLogoHtml@SLiTeX@lift}
%    \end{macrocode}
%    \end{macro}
%
% \paragraph{Defaults.}
%
%    \begin{macro}{\HoLogo@SLiTeX}
%    \begin{macrocode}
\def\HoLogo@SLiTeX{\HoLogo@SLiTeX@lift}
%    \end{macrocode}
%    \end{macro}
%    \begin{macro}{\HoLogoBkm@SLiTeX}
%    \begin{macrocode}
\def\HoLogoBkm@SLiTeX{\HoLogoBkm@SLiTeX@lift}
%    \end{macrocode}
%    \end{macro}
%    \begin{macro}{\HoLogoHtml@SLiTeX}
%    \begin{macrocode}
\def\HoLogoHtml@SLiTeX{\HoLogoHtml@SLiTeX@lift}
%    \end{macrocode}
%    \end{macro}
%
%    \begin{macro}{\HoLogo@SliTeX}
%    \begin{macrocode}
\def\HoLogo@SliTeX{\HoLogo@SliTeX@narrow}
%    \end{macrocode}
%    \end{macro}
%    \begin{macro}{\HoLogoBkm@SliTeX}
%    \begin{macrocode}
\def\HoLogoBkm@SliTeX{\HoLogoBkm@SliTeX@narrow}
%    \end{macrocode}
%    \end{macro}
%    \begin{macro}{\HoLogoHtml@SliTeX}
%    \begin{macrocode}
\def\HoLogoHtml@SliTeX{\HoLogoHtml@SliTeX@narrow}
%    \end{macrocode}
%    \end{macro}
%
% \subsubsection{\hologo{LuaTeX}}
%
%    \begin{macro}{\HoLogo@LuaTeX}
%    The kerning is an idea of Hans Hagen, see mailing list
%    `luatex at tug dot org' in March 2010.
%    \begin{macrocode}
\def\HoLogo@LuaTeX#1{%
  \HOLOGO@mbox{%
    Lua%
    \HOLOGO@NegativeKerning{aT,oT,To}%
    \hologo{TeX}%
  }%
}
%    \end{macrocode}
%    \end{macro}
%    \begin{macro}{\HoLogoHtml@LuaTeX}
%    \begin{macrocode}
\let\HoLogoHtml@LuaTeX\HoLogo@LuaTeX
%    \end{macrocode}
%    \end{macro}
%
% \subsubsection{\hologo{LuaLaTeX}}
%
%    \begin{macro}{\HoLogo@LuaLaTeX}
%    \begin{macrocode}
\def\HoLogo@LuaLaTeX#1{%
  \HOLOGO@mbox{%
    Lua%
    \hologo{LaTeX}%
  }%
}
%    \end{macrocode}
%    \end{macro}
%    \begin{macro}{\HoLogoHtml@LuaLaTeX}
%    \begin{macrocode}
\let\HoLogoHtml@LuaLaTeX\HoLogo@LuaLaTeX
%    \end{macrocode}
%    \end{macro}
%
% \subsubsection{\hologo{XeTeX}, \hologo{XeLaTeX}}
%
%    \begin{macro}{\HOLOGO@IfCharExists}
%    \begin{macrocode}
\ifluatex
  \ifnum\luatexversion<36 %
  \else
    \def\HOLOGO@IfCharExists#1{%
      \ifnum
        \directlua{%
           if luaotfload and luaotfload.aux then
             if luaotfload.aux.font_has_glyph(%
                    font.current(), \number#1) then % 	 
	       tex.print("1") % 	 
	     end % 	 
	   elseif font and font.fonts and font.current then %
            local f = font.fonts[font.current()]%
            if f.characters and f.characters[\number#1] then %
              tex.print("1")%
            end %
          end%
        }0=\ltx@zero
        \expandafter\ltx@secondoftwo
      \else
        \expandafter\ltx@firstoftwo
      \fi
    }%
  \fi
\fi
\ltx@IfUndefined{HOLOGO@IfCharExists}{%
  \def\HOLOGO@@IfCharExists#1{%
    \begingroup
      \tracinglostchars=\ltx@zero
      \setbox\ltx@zero=\hbox{%
        \kern7sp\char#1\relax
        \ifnum\lastkern>\ltx@zero
          \expandafter\aftergroup\csname iffalse\endcsname
        \else
          \expandafter\aftergroup\csname iftrue\endcsname
        \fi
      }%
      % \if{true|false} from \aftergroup
      \endgroup
      \expandafter\ltx@firstoftwo
    \else
      \endgroup
      \expandafter\ltx@secondoftwo
    \fi
  }%
  \ifxetex
    \ltx@IfUndefined{XeTeXfonttype}{}{%
      \ltx@IfUndefined{XeTeXcharglyph}{}{%
        \def\HOLOGO@IfCharExists#1{%
          \ifnum\XeTeXfonttype\font>\ltx@zero
            \expandafter\ltx@firstofthree
          \else
            \expandafter\ltx@gobble
          \fi
          {%
            \ifnum\XeTeXcharglyph#1>\ltx@zero
              \expandafter\ltx@firstoftwo
            \else
              \expandafter\ltx@secondoftwo
            \fi
          }%
          \HOLOGO@@IfCharExists{#1}%
        }%
      }%
    }%
  \fi
}{}
\ltx@ifundefined{HOLOGO@IfCharExists}{%
  \ifnum64=`\^^^^0040\relax % test for big chars of LuaTeX/XeTeX
    \let\HOLOGO@IfCharExists\HOLOGO@@IfCharExists
  \else
    \def\HOLOGO@IfCharExists#1{%
      \ifnum#1>255 %
        \expandafter\ltx@fourthoffour
      \fi
      \HOLOGO@@IfCharExists{#1}%
    }%
  \fi
}{}
%    \end{macrocode}
%    \end{macro}
%
%    \begin{macro}{\HoLogo@Xe}
%    Source: package \xpackage{dtklogos}
%    \begin{macrocode}
\def\HoLogo@Xe#1{%
  X%
  \kern-.1em\relax
  \HOLOGO@IfCharExists{"018E}{%
    \lower.5ex\hbox{\char"018E}%
  }{%
    \chardef\HOLOGO@choice=\ltx@zero
    \ifdim\fontdimen\ltx@one\font>0pt %
      \ltx@IfUndefined{rotatebox}{%
        \ltx@IfUndefined{pgftext}{%
          \ltx@IfUndefined{psscalebox}{%
            \ltx@IfUndefined{HOLOGO@ScaleBox@\hologoDriver}{%
            }{%
              \chardef\HOLOGO@choice=4 %
            }%
          }{%
            \chardef\HOLOGO@choice=3 %
          }%
        }{%
          \chardef\HOLOGO@choice=2 %
        }%
      }{%
        \chardef\HOLOGO@choice=1 %
      }%
      \ifcase\HOLOGO@choice
        \HOLOGO@WarningUnsupportedDriver{Xe}%
        e%
      \or % 1: \rotatebox
        \begingroup
          \setbox\ltx@zero\hbox{\rotatebox{180}{E}}%
          \ltx@LocDimenA=\dp\ltx@zero
          \advance\ltx@LocDimenA by -.5ex\relax
          \raise\ltx@LocDimenA\box\ltx@zero
        \endgroup
      \or % 2: \pgftext
        \lower.5ex\hbox{%
          \pgfpicture
            \pgftext[rotate=180]{E}%
          \endpgfpicture
        }%
      \or % 3: \psscalebox
        \begingroup
          \setbox\ltx@zero\hbox{\psscalebox{-1 -1}{E}}%
          \ltx@LocDimenA=\dp\ltx@zero
          \advance\ltx@LocDimenA by -.5ex\relax
          \raise\ltx@LocDimenA\box\ltx@zero
        \endgroup
      \or % 4: \HOLOGO@PointReflectBox
        \lower.5ex\hbox{\HOLOGO@PointReflectBox{E}}%
      \else
        \@PackageError{hologo}{Internal error (choice/it}\@ehc
      \fi
    \else
      \ltx@IfUndefined{reflectbox}{%
        \ltx@IfUndefined{pgftext}{%
          \ltx@IfUndefined{psscalebox}{%
            \ltx@IfUndefined{HOLOGO@ScaleBox@\hologoDriver}{%
            }{%
              \chardef\HOLOGO@choice=4 %
            }%
          }{%
            \chardef\HOLOGO@choice=3 %
          }%
        }{%
          \chardef\HOLOGO@choice=2 %
        }%
      }{%
        \chardef\HOLOGO@choice=1 %
      }%
      \ifcase\HOLOGO@choice
        \HOLOGO@WarningUnsupportedDriver{Xe}%
        e%
      \or % 1: reflectbox
        \lower.5ex\hbox{%
          \reflectbox{E}%
        }%
      \or % 2: \pgftext
        \lower.5ex\hbox{%
          \pgfpicture
            \pgftransformxscale{-1}%
            \pgftext{E}%
          \endpgfpicture
        }%
      \or % 3: \psscalebox
        \lower.5ex\hbox{%
          \psscalebox{-1 1}{E}%
        }%
      \or % 4: \HOLOGO@Reflectbox
        \lower.5ex\hbox{%
          \HOLOGO@ReflectBox{E}%
        }%
      \else
        \@PackageError{hologo}{Internal error (choice/up)}\@ehc
      \fi
    \fi
  }%
}
%    \end{macrocode}
%    \end{macro}
%    \begin{macro}{\HoLogoHtml@Xe}
%    \begin{macrocode}
\def\HoLogoHtml@Xe#1{%
  \HoLogoCss@Xe
  \HOLOGO@Span{Xe}{%
    X%
    \HOLOGO@Span{e}{%
      \HCode{&\ltx@hashchar x018e;}%
    }%
  }%
}
%    \end{macrocode}
%    \end{macro}
%    \begin{macro}{\HoLogoCss@Xe}
%    \begin{macrocode}
\def\HoLogoCss@Xe{%
  \Css{%
    span.HoLogo-Xe span.HoLogo-e{%
      position:relative;%
      top:.5ex;%
      left-margin:-.1em;%
    }%
  }%
  \global\let\HoLogoCss@Xe\relax
}
%    \end{macrocode}
%    \end{macro}
%
%    \begin{macro}{\HoLogo@XeTeX}
%    \begin{macrocode}
\def\HoLogo@XeTeX#1{%
  \hologo{Xe}%
  \kern-.15em\relax
  \hologo{TeX}%
}
%    \end{macrocode}
%    \end{macro}
%
%    \begin{macro}{\HoLogoHtml@XeTeX}
%    \begin{macrocode}
\def\HoLogoHtml@XeTeX#1{%
  \HoLogoCss@XeTeX
  \HOLOGO@Span{XeTeX}{%
    \hologo{Xe}%
    \hologo{TeX}%
  }%
}
%    \end{macrocode}
%    \end{macro}
%    \begin{macro}{\HoLogoCss@XeTeX}
%    \begin{macrocode}
\def\HoLogoCss@XeTeX{%
  \Css{%
    span.HoLogo-XeTeX span.HoLogo-TeX{%
      margin-left:-.15em;%
    }%
  }%
  \global\let\HoLogoCss@XeTeX\relax
}
%    \end{macrocode}
%    \end{macro}
%
%    \begin{macro}{\HoLogo@XeLaTeX}
%    \begin{macrocode}
\def\HoLogo@XeLaTeX#1{%
  \hologo{Xe}%
  \kern-.13em%
  \hologo{LaTeX}%
}
%    \end{macrocode}
%    \end{macro}
%    \begin{macro}{\HoLogoHtml@XeLaTeX}
%    \begin{macrocode}
\def\HoLogoHtml@XeLaTeX#1{%
  \HoLogoCss@XeLaTeX
  \HOLOGO@Span{XeLaTeX}{%
    \hologo{Xe}%
    \hologo{LaTeX}%
  }%
}
%    \end{macrocode}
%    \end{macro}
%    \begin{macro}{\HoLogoCss@XeLaTeX}
%    \begin{macrocode}
\def\HoLogoCss@XeLaTeX{%
  \Css{%
    span.HoLogo-XeLaTeX span.HoLogo-Xe{%
      margin-right:-.13em;%
    }%
  }%
  \global\let\HoLogoCss@XeLaTeX\relax
}
%    \end{macrocode}
%    \end{macro}
%
% \subsubsection{\hologo{pdfTeX}, \hologo{pdfLaTeX}}
%
%    \begin{macro}{\HoLogo@pdfTeX}
%    \begin{macrocode}
\def\HoLogo@pdfTeX#1{%
  \HOLOGO@mbox{%
    #1{p}{P}df\hologo{TeX}%
  }%
}
%    \end{macrocode}
%    \end{macro}
%    \begin{macro}{\HoLogoCs@pdfTeX}
%    \begin{macrocode}
\def\HoLogoCs@pdfTeX#1{#1{p}{P}dfTeX}
%    \end{macrocode}
%    \end{macro}
%    \begin{macro}{\HoLogoBkm@pdfTeX}
%    \begin{macrocode}
\def\HoLogoBkm@pdfTeX#1{%
  #1{p}{P}df\hologo{TeX}%
}
%    \end{macrocode}
%    \end{macro}
%    \begin{macro}{\HoLogoHtml@pdfTeX}
%    \begin{macrocode}
\let\HoLogoHtml@pdfTeX\HoLogo@pdfTeX
%    \end{macrocode}
%    \end{macro}
%
%    \begin{macro}{\HoLogo@pdfLaTeX}
%    \begin{macrocode}
\def\HoLogo@pdfLaTeX#1{%
  \HOLOGO@mbox{%
    #1{p}{P}df\hologo{LaTeX}%
  }%
}
%    \end{macrocode}
%    \end{macro}
%    \begin{macro}{\HoLogoCs@pdfLaTeX}
%    \begin{macrocode}
\def\HoLogoCs@pdfLaTeX#1{#1{p}{P}dfLaTeX}
%    \end{macrocode}
%    \end{macro}
%    \begin{macro}{\HoLogoBkm@pdfLaTeX}
%    \begin{macrocode}
\def\HoLogoBkm@pdfLaTeX#1{%
  #1{p}{P}df\hologo{LaTeX}%
}
%    \end{macrocode}
%    \end{macro}
%    \begin{macro}{\HoLogoHtml@pdfLaTeX}
%    \begin{macrocode}
\let\HoLogoHtml@pdfLaTeX\HoLogo@pdfLaTeX
%    \end{macrocode}
%    \end{macro}
%
% \subsubsection{\hologo{VTeX}}
%
%    \begin{macro}{\HoLogo@VTeX}
%    \begin{macrocode}
\def\HoLogo@VTeX#1{%
  \HOLOGO@mbox{%
    V\hologo{TeX}%
  }%
}
%    \end{macrocode}
%    \end{macro}
%    \begin{macro}{\HoLogoHtml@VTeX}
%    \begin{macrocode}
\let\HoLogoHtml@VTeX\HoLogo@VTeX
%    \end{macrocode}
%    \end{macro}
%
% \subsubsection{\hologo{AmS}, \dots}
%
%    Source: class \xclass{amsdtx}
%
%    \begin{macro}{\HoLogo@AmS}
%    \begin{macrocode}
\def\HoLogo@AmS#1{%
  \HoLogoFont@font{AmS}{sy}{%
    A%
    \kern-.1667em%
    \lower.5ex\hbox{M}%
    \kern-.125em%
    S%
  }%
}
%    \end{macrocode}
%    \end{macro}
%    \begin{macro}{\HoLogoBkm@AmS}
%    \begin{macrocode}
\def\HoLogoBkm@AmS#1{AmS}
%    \end{macrocode}
%    \end{macro}
%    \begin{macro}{\HoLogoHtml@AmS}
%    \begin{macrocode}
\def\HoLogoHtml@AmS#1{%
  \HoLogoCss@AmS
%  \HoLogoFont@font{AmS}{sy}{%
    \HOLOGO@Span{AmS}{%
      A%
      \HOLOGO@Span{M}{M}%
      S%
    }%
%   }%
}
%    \end{macrocode}
%    \end{macro}
%    \begin{macro}{\HoLogoCss@AmS}
%    \begin{macrocode}
\def\HoLogoCss@AmS{%
  \Css{%
    span.HoLogo-AmS span.HoLogo-M{%
      position:relative;%
      top:.5ex;%
      margin-left:-.1667em;%
      margin-right:-.125em;%
      text-decoration:none;%
    }%
  }%
  \global\let\HoLogoCss@AmS\relax
}
%    \end{macrocode}
%    \end{macro}
%
%    \begin{macro}{\HoLogo@AmSTeX}
%    \begin{macrocode}
\def\HoLogo@AmSTeX#1{%
  \hologo{AmS}%
  \HOLOGO@hyphen
  \hologo{TeX}%
}
%    \end{macrocode}
%    \end{macro}
%    \begin{macro}{\HoLogoBkm@AmSTeX}
%    \begin{macrocode}
\def\HoLogoBkm@AmSTeX#1{AmS-TeX}%
%    \end{macrocode}
%    \end{macro}
%    \begin{macro}{\HoLogoHtml@AmSTeX}
%    \begin{macrocode}
\let\HoLogoHtml@AmSTeX\HoLogo@AmSTeX
%    \end{macrocode}
%    \end{macro}
%
%    \begin{macro}{\HoLogo@AmSLaTeX}
%    \begin{macrocode}
\def\HoLogo@AmSLaTeX#1{%
  \hologo{AmS}%
  \HOLOGO@hyphen
  \hologo{LaTeX}%
}
%    \end{macrocode}
%    \end{macro}
%    \begin{macro}{\HoLogoBkm@AmSLaTeX}
%    \begin{macrocode}
\def\HoLogoBkm@AmSLaTeX#1{AmS-LaTeX}%
%    \end{macrocode}
%    \end{macro}
%    \begin{macro}{\HoLogoHtml@AmSLaTeX}
%    \begin{macrocode}
\let\HoLogoHtml@AmSLaTeX\HoLogo@AmSLaTeX
%    \end{macrocode}
%    \end{macro}
%
% \subsubsection{\hologo{BibTeX}}
%
%    \begin{macro}{\HoLogo@BibTeX@sc}
%    A definition of \hologo{BibTeX} is provided in
%    the documentation source for the manual of \hologo{BibTeX}
%    \cite{btxdoc}.
%\begin{quote}
%\begin{verbatim}
%\def\BibTeX{%
%  {%
%    \rm
%    B%
%    \kern-.05em%
%    {%
%      \sc
%      i%
%      \kern-.025em %
%      b%
%    }%
%    \kern-.08em
%    T%
%    \kern-.1667em%
%    \lower.7ex\hbox{E}%
%    \kern-.125em%
%    X%
%  }%
%}
%\end{verbatim}
%\end{quote}
%    \begin{macrocode}
\def\HoLogo@BibTeX@sc#1{%
  B%
  \kern-.05em%
  \HoLogoFont@font{BibTeX}{sc}{%
    i%
    \kern-.025em%
    b%
  }%
  \HOLOGO@discretionary
  \kern-.08em%
  \hologo{TeX}%
}
%    \end{macrocode}
%    \end{macro}
%    \begin{macro}{\HoLogoHtml@BibTeX@sc}
%    \begin{macrocode}
\def\HoLogoHtml@BibTeX@sc#1{%
  \HoLogoCss@BibTeX@sc
  \HOLOGO@Span{BibTeX-sc}{%
    B%
    \HOLOGO@Span{i}{i}%
    \HOLOGO@Span{b}{b}%
    \hologo{TeX}%
  }%
}
%    \end{macrocode}
%    \end{macro}
%    \begin{macro}{\HoLogoCss@BibTeX@sc}
%    \begin{macrocode}
\def\HoLogoCss@BibTeX@sc{%
  \Css{%
    span.HoLogo-BibTeX-sc span.HoLogo-i{%
      margin-left:-.05em;%
      margin-right:-.025em;%
      font-variant:small-caps;%
    }%
  }%
  \Css{%
    span.HoLogo-BibTeX-sc span.HoLogo-b{%
      margin-right:-.08em;%
      font-variant:small-caps;%
    }%
  }%
  \global\let\HoLogoCss@BibTeX@sc\relax
}
%    \end{macrocode}
%    \end{macro}
%
%    \begin{macro}{\HoLogo@BibTeX@sf}
%    Variant \xoption{sf} avoids trouble with unavailable
%    small caps fonts (e.g., bold versions of Computer Modern or
%    Latin Modern). The definition is taken from
%    package \xpackage{dtklogos} \cite{dtklogos}.
%\begin{quote}
%\begin{verbatim}
%\DeclareRobustCommand{\BibTeX}{%
%  B%
%  \kern-.05em%
%  \hbox{%
%    $\m@th$% %% force math size calculations
%    \csname S@\f@size\endcsname
%    \fontsize\sf@size\z@
%    \math@fontsfalse
%    \selectfont
%    I%
%    \kern-.025em%
%    B
%  }%
%  \kern-.08em%
%  \-%
%  \TeX
%}
%\end{verbatim}
%\end{quote}
%    \begin{macrocode}
\def\HoLogo@BibTeX@sf#1{%
  B%
  \kern-.05em%
  \HoLogoFont@font{BibTeX}{bibsf}{%
    I%
    \kern-.025em%
    B%
  }%
  \HOLOGO@discretionary
  \kern-.08em%
  \hologo{TeX}%
}
%    \end{macrocode}
%    \end{macro}
%    \begin{macro}{\HoLogoHtml@BibTeX@sf}
%    \begin{macrocode}
\def\HoLogoHtml@BibTeX@sf#1{%
  \HoLogoCss@BibTeX@sf
  \HOLOGO@Span{BibTeX-sf}{%
    B%
    \HoLogoFont@font{BibTeX}{bibsf}{%
      \HOLOGO@Span{i}{I}%
      B%
    }%
    \hologo{TeX}%
  }%
}
%    \end{macrocode}
%    \end{macro}
%    \begin{macro}{\HoLogoCss@BibTeX@sf}
%    \begin{macrocode}
\def\HoLogoCss@BibTeX@sf{%
  \Css{%
    span.HoLogo-BibTeX-sf span.HoLogo-i{%
      margin-left:-.05em;%
      margin-right:-.025em;%
    }%
  }%
  \Css{%
    span.HoLogo-BibTeX-sf span.HoLogo-TeX{%
      margin-left:-.08em;%
    }%
  }%
  \global\let\HoLogoCss@BibTeX@sf\relax
}
%    \end{macrocode}
%    \end{macro}
%
%    \begin{macro}{\HoLogo@BibTeX}
%    \begin{macrocode}
\def\HoLogo@BibTeX{\HoLogo@BibTeX@sf}
%    \end{macrocode}
%    \end{macro}
%    \begin{macro}{\HoLogoHtml@BibTeX}
%    \begin{macrocode}
\def\HoLogoHtml@BibTeX{\HoLogoHtml@BibTeX@sf}
%    \end{macrocode}
%    \end{macro}
%
% \subsubsection{\hologo{BibTeX8}}
%
%    \begin{macro}{\HoLogo@BibTeX8}
%    \begin{macrocode}
\expandafter\def\csname HoLogo@BibTeX8\endcsname#1{%
  \hologo{BibTeX}%
  8%
}
%    \end{macrocode}
%    \end{macro}
%
%    \begin{macro}{\HoLogoBkm@BibTeX8}
%    \begin{macrocode}
\expandafter\def\csname HoLogoBkm@BibTeX8\endcsname#1{%
  \hologo{BibTeX}%
  8%
}
%    \end{macrocode}
%    \end{macro}
%    \begin{macro}{\HoLogoHtml@BibTeX8}
%    \begin{macrocode}
\expandafter
\let\csname HoLogoHtml@BibTeX8\expandafter\endcsname
\csname HoLogo@BibTeX8\endcsname
%    \end{macrocode}
%    \end{macro}
%
% \subsubsection{\hologo{ConTeXt}}
%
%    \begin{macro}{\HoLogo@ConTeXt@simple}
%    \begin{macrocode}
\def\HoLogo@ConTeXt@simple#1{%
  \HOLOGO@mbox{Con}%
  \HOLOGO@discretionary
  \HOLOGO@mbox{\hologo{TeX}t}%
}
%    \end{macrocode}
%    \end{macro}
%    \begin{macro}{\HoLogoHtml@ConTeXt@simple}
%    \begin{macrocode}
\let\HoLogoHtml@ConTeXt@simple\HoLogo@ConTeXt@simple
%    \end{macrocode}
%    \end{macro}
%
%    \begin{macro}{\HoLogo@ConTeXt@narrow}
%    This definition of logo \hologo{ConTeXt} with variant \xoption{narrow}
%    comes from TUGboat's class \xclass{ltugboat} (version 2010/11/15 v2.8).
%    \begin{macrocode}
\def\HoLogo@ConTeXt@narrow#1{%
  \HOLOGO@mbox{C\kern-.0333emon}%
  \HOLOGO@discretionary
  \kern-.0667em%
  \HOLOGO@mbox{\hologo{TeX}\kern-.0333emt}%
}
%    \end{macrocode}
%    \end{macro}
%    \begin{macro}{\HoLogoHtml@ConTeXt@narrow}
%    \begin{macrocode}
\def\HoLogoHtml@ConTeXt@narrow#1{%
  \HoLogoCss@ConTeXt@narrow
  \HOLOGO@Span{ConTeXt-narrow}{%
    \HOLOGO@Span{C}{C}%
    on%
    \hologo{TeX}%
    t%
  }%
}
%    \end{macrocode}
%    \end{macro}
%    \begin{macro}{\HoLogoCss@ConTeXt@narrow}
%    \begin{macrocode}
\def\HoLogoCss@ConTeXt@narrow{%
  \Css{%
    span.HoLogo-ConTeXt-narrow span.HoLogo-C{%
      margin-left:-.0333em;%
    }%
  }%
  \Css{%
    span.HoLogo-ConTeXt-narrow span.HoLogo-TeX{%
      margin-left:-.0667em;%
      margin-right:-.0333em;%
    }%
  }%
  \global\let\HoLogoCss@ConTeXt@narrow\relax
}
%    \end{macrocode}
%    \end{macro}
%
%    \begin{macro}{\HoLogo@ConTeXt}
%    \begin{macrocode}
\def\HoLogo@ConTeXt{\HoLogo@ConTeXt@narrow}
%    \end{macrocode}
%    \end{macro}
%    \begin{macro}{\HoLogoHtml@ConTeXt}
%    \begin{macrocode}
\def\HoLogoHtml@ConTeXt{\HoLogoHtml@ConTeXt@narrow}
%    \end{macrocode}
%    \end{macro}
%
% \subsubsection{\hologo{emTeX}}
%
%    \begin{macro}{\HoLogo@emTeX}
%    \begin{macrocode}
\def\HoLogo@emTeX#1{%
  \HOLOGO@mbox{#1{e}{E}m}%
  \HOLOGO@discretionary
  \hologo{TeX}%
}
%    \end{macrocode}
%    \end{macro}
%    \begin{macro}{\HoLogoCs@emTeX}
%    \begin{macrocode}
\def\HoLogoCs@emTeX#1{#1{e}{E}mTeX}%
%    \end{macrocode}
%    \end{macro}
%    \begin{macro}{\HoLogoBkm@emTeX}
%    \begin{macrocode}
\def\HoLogoBkm@emTeX#1{%
  #1{e}{E}m\hologo{TeX}%
}
%    \end{macrocode}
%    \end{macro}
%    \begin{macro}{\HoLogoHtml@emTeX}
%    \begin{macrocode}
\let\HoLogoHtml@emTeX\HoLogo@emTeX
%    \end{macrocode}
%    \end{macro}
%
% \subsubsection{\hologo{ExTeX}}
%
%    \begin{macro}{\HoLogo@ExTeX}
%    The definition is taken from the FAQ of the
%    project \hologo{ExTeX}
%    \cite{ExTeX-FAQ}.
%\begin{quote}
%\begin{verbatim}
%\def\ExTeX{%
%  \textrm{% Logo always with serifs
%    \ensuremath{%
%      \textstyle
%      \varepsilon_{%
%        \kern-0.15em%
%        \mathcal{X}%
%      }%
%    }%
%    \kern-.15em%
%    \TeX
%  }%
%}
%\end{verbatim}
%\end{quote}
%    \begin{macrocode}
\def\HoLogo@ExTeX#1{%
  \HoLogoFont@font{ExTeX}{rm}{%
    \ltx@mbox{%
      \HOLOGO@MathSetup
      $%
        \textstyle
        \varepsilon_{%
          \kern-0.15em%
          \HoLogoFont@font{ExTeX}{sy}{X}%
        }%
      $%
    }%
    \HOLOGO@discretionary
    \kern-.15em%
    \hologo{TeX}%
  }%
}
%    \end{macrocode}
%    \end{macro}
%    \begin{macro}{\HoLogoHtml@ExTeX}
%    \begin{macrocode}
\def\HoLogoHtml@ExTeX#1{%
  \HoLogoCss@ExTeX
  \HoLogoFont@font{ExTeX}{rm}{%
    \HOLOGO@Span{ExTeX}{%
      \ltx@mbox{%
        \HOLOGO@MathSetup
        $\textstyle\varepsilon$%
        \HOLOGO@Span{X}{$\textstyle\chi$}%
        \hologo{TeX}%
      }%
    }%
  }%
}
%    \end{macrocode}
%    \end{macro}
%    \begin{macro}{\HoLogoBkm@ExTeX}
%    \begin{macrocode}
\def\HoLogoBkm@ExTeX#1{%
  \HOLOGO@PdfdocUnicode{#1{e}{E}x}{\textepsilon\textchi}%
  \hologo{TeX}%
}
%    \end{macrocode}
%    \end{macro}
%    \begin{macro}{\HoLogoCss@ExTeX}
%    \begin{macrocode}
\def\HoLogoCss@ExTeX{%
  \Css{%
    span.HoLogo-ExTeX{%
      font-family:serif;%
    }%
  }%
  \Css{%
    span.HoLogo-ExTeX span.HoLogo-TeX{%
      margin-left:-.15em;%
    }%
  }%
  \global\let\HoLogoCss@ExTeX\relax
}
%    \end{macrocode}
%    \end{macro}
%
% \subsubsection{\hologo{MiKTeX}}
%
%    \begin{macro}{\HoLogo@MiKTeX}
%    \begin{macrocode}
\def\HoLogo@MiKTeX#1{%
  \HOLOGO@mbox{MiK}%
  \HOLOGO@discretionary
  \hologo{TeX}%
}
%    \end{macrocode}
%    \end{macro}
%    \begin{macro}{\HoLogoHtml@MiKTeX}
%    \begin{macrocode}
\let\HoLogoHtml@MiKTeX\HoLogo@MiKTeX
%    \end{macrocode}
%    \end{macro}
%
% \subsubsection{\hologo{OzTeX} and friends}
%
%    Source: \hologo{OzTeX} FAQ \cite{OzTeX}:
%    \begin{quote}
%      |\def\OzTeX{O\kern-.03em z\kern-.15em\TeX}|\\
%      (There is no kerning in OzMF, OzMP and OzTtH.)
%    \end{quote}
%
%    \begin{macro}{\HoLogo@OzTeX}
%    \begin{macrocode}
\def\HoLogo@OzTeX#1{%
  O%
  \kern-.03em %
  z%
  \kern-.15em %
  \hologo{TeX}%
}
%    \end{macrocode}
%    \end{macro}
%    \begin{macro}{\HoLogoHtml@OzTeX}
%    \begin{macrocode}
\def\HoLogoHtml@OzTeX#1{%
  \HoLogoCss@OzTeX
  \HOLOGO@Span{OzTeX}{%
    O%
    \HOLOGO@Span{z}{z}%
    \hologo{TeX}%
  }%
}
%    \end{macrocode}
%    \end{macro}
%    \begin{macro}{\HoLogoCss@OzTeX}
%    \begin{macrocode}
\def\HoLogoCss@OzTeX{%
  \Css{%
    span.HoLogo-OzTeX span.HoLogo-z{%
      margin-left:-.03em;%
      margin-right:-.15em;%
    }%
  }%
  \global\let\HoLogoCss@OzTeX\relax
}
%    \end{macrocode}
%    \end{macro}
%
%    \begin{macro}{\HoLogo@OzMF}
%    \begin{macrocode}
\def\HoLogo@OzMF#1{%
  \HOLOGO@mbox{OzMF}%
}
%    \end{macrocode}
%    \end{macro}
%    \begin{macro}{\HoLogo@OzMP}
%    \begin{macrocode}
\def\HoLogo@OzMP#1{%
  \HOLOGO@mbox{OzMP}%
}
%    \end{macrocode}
%    \end{macro}
%    \begin{macro}{\HoLogo@OzTtH}
%    \begin{macrocode}
\def\HoLogo@OzTtH#1{%
  \HOLOGO@mbox{OzTtH}%
}
%    \end{macrocode}
%    \end{macro}
%
% \subsubsection{\hologo{PCTeX}}
%
%    \begin{macro}{\HoLogo@PCTeX}
%    \begin{macrocode}
\def\HoLogo@PCTeX#1{%
  \HOLOGO@mbox{PC}%
  \hologo{TeX}%
}
%    \end{macrocode}
%    \end{macro}
%    \begin{macro}{\HoLogoHtml@PCTeX}
%    \begin{macrocode}
\let\HoLogoHtml@PCTeX\HoLogo@PCTeX
%    \end{macrocode}
%    \end{macro}
%
% \subsubsection{\hologo{PiCTeX}}
%
%    The original definitions from \xfile{pictex.tex} \cite{PiCTeX}:
%\begin{quote}
%\begin{verbatim}
%\def\PiC{%
%  P%
%  \kern-.12em%
%  \lower.5ex\hbox{I}%
%  \kern-.075em%
%  C%
%}
%\def\PiCTeX{%
%  \PiC
%  \kern-.11em%
%  \TeX
%}
%\end{verbatim}
%\end{quote}
%
%    \begin{macro}{\HoLogo@PiC}
%    \begin{macrocode}
\def\HoLogo@PiC#1{%
  P%
  \kern-.12em%
  \lower.5ex\hbox{I}%
  \kern-.075em%
  C%
  \HOLOGO@SpaceFactor
}
%    \end{macrocode}
%    \end{macro}
%    \begin{macro}{\HoLogoHtml@PiC}
%    \begin{macrocode}
\def\HoLogoHtml@PiC#1{%
  \HoLogoCss@PiC
  \HOLOGO@Span{PiC}{%
    P%
    \HOLOGO@Span{i}{I}%
    C%
  }%
}
%    \end{macrocode}
%    \end{macro}
%    \begin{macro}{\HoLogoCss@PiC}
%    \begin{macrocode}
\def\HoLogoCss@PiC{%
  \Css{%
    span.HoLogo-PiC span.HoLogo-i{%
      position:relative;%
      top:.5ex;%
      margin-left:-.12em;%
      margin-right:-.075em;%
      text-decoration:none;%
    }%
  }%
  \global\let\HoLogoCss@PiC\relax
}
%    \end{macrocode}
%    \end{macro}
%
%    \begin{macro}{\HoLogo@PiCTeX}
%    \begin{macrocode}
\def\HoLogo@PiCTeX#1{%
  \hologo{PiC}%
  \HOLOGO@discretionary
  \kern-.11em%
  \hologo{TeX}%
}
%    \end{macrocode}
%    \end{macro}
%    \begin{macro}{\HoLogoHtml@PiCTeX}
%    \begin{macrocode}
\def\HoLogoHtml@PiCTeX#1{%
  \HoLogoCss@PiCTeX
  \HOLOGO@Span{PiCTeX}{%
    \hologo{PiC}%
    \hologo{TeX}%
  }%
}
%    \end{macrocode}
%    \end{macro}
%    \begin{macro}{\HoLogoCss@PiCTeX}
%    \begin{macrocode}
\def\HoLogoCss@PiCTeX{%
  \Css{%
    span.HoLogo-PiCTeX span.HoLogo-PiC{%
      margin-right:-.11em;%
    }%
  }%
  \global\let\HoLogoCss@PiCTeX\relax
}
%    \end{macrocode}
%    \end{macro}
%
% \subsubsection{\hologo{teTeX}}
%
%    \begin{macro}{\HoLogo@teTeX}
%    \begin{macrocode}
\def\HoLogo@teTeX#1{%
  \HOLOGO@mbox{#1{t}{T}e}%
  \HOLOGO@discretionary
  \hologo{TeX}%
}
%    \end{macrocode}
%    \end{macro}
%    \begin{macro}{\HoLogoCs@teTeX}
%    \begin{macrocode}
\def\HoLogoCs@teTeX#1{#1{t}{T}dfTeX}
%    \end{macrocode}
%    \end{macro}
%    \begin{macro}{\HoLogoBkm@teTeX}
%    \begin{macrocode}
\def\HoLogoBkm@teTeX#1{%
  #1{t}{T}e\hologo{TeX}%
}
%    \end{macrocode}
%    \end{macro}
%    \begin{macro}{\HoLogoHtml@teTeX}
%    \begin{macrocode}
\let\HoLogoHtml@teTeX\HoLogo@teTeX
%    \end{macrocode}
%    \end{macro}
%
% \subsubsection{\hologo{TeX4ht}}
%
%    \begin{macro}{\HoLogo@TeX4ht}
%    \begin{macrocode}
\expandafter\def\csname HoLogo@TeX4ht\endcsname#1{%
  \HOLOGO@mbox{\hologo{TeX}4ht}%
}
%    \end{macrocode}
%    \end{macro}
%    \begin{macro}{\HoLogoHtml@TeX4ht}
%    \begin{macrocode}
\expandafter
\let\csname HoLogoHtml@TeX4ht\expandafter\endcsname
\csname HoLogo@TeX4ht\endcsname
%    \end{macrocode}
%    \end{macro}
%
%
% \subsubsection{\hologo{SageTeX}}
%
%    \begin{macro}{\HoLogo@SageTeX}
%    \begin{macrocode}
\def\HoLogo@SageTeX#1{%
  \HOLOGO@mbox{Sage}%
  \HOLOGO@discretionary
  \HOLOGO@NegativeKerning{eT,oT,To}%
  \hologo{TeX}%
}
%    \end{macrocode}
%    \end{macro}
%    \begin{macro}{\HoLogoHtml@SageTeX}
%    \begin{macrocode}
\let\HoLogoHtml@SageTeX\HoLogo@SageTeX
%    \end{macrocode}
%    \end{macro}
%
% \subsection{\hologo{METAFONT} and friends}
%
%    \begin{macro}{\HoLogo@METAFONT}
%    \begin{macrocode}
\def\HoLogo@METAFONT#1{%
  \HoLogoFont@font{METAFONT}{logo}{%
    \HOLOGO@mbox{META}%
    \HOLOGO@discretionary
    \HOLOGO@mbox{FONT}%
  }%
}
%    \end{macrocode}
%    \end{macro}
%
%    \begin{macro}{\HoLogo@METAPOST}
%    \begin{macrocode}
\def\HoLogo@METAPOST#1{%
  \HoLogoFont@font{METAPOST}{logo}{%
    \HOLOGO@mbox{META}%
    \HOLOGO@discretionary
    \HOLOGO@mbox{POST}%
  }%
}
%    \end{macrocode}
%    \end{macro}
%
%    \begin{macro}{\HoLogo@MetaFun}
%    \begin{macrocode}
\def\HoLogo@MetaFun#1{%
  \HOLOGO@mbox{Meta}%
  \HOLOGO@discretionary
  \HOLOGO@mbox{Fun}%
}
%    \end{macrocode}
%    \end{macro}
%
%    \begin{macro}{\HoLogo@MetaPost}
%    \begin{macrocode}
\def\HoLogo@MetaPost#1{%
  \HOLOGO@mbox{Meta}%
  \HOLOGO@discretionary
  \HOLOGO@mbox{Post}%
}
%    \end{macrocode}
%    \end{macro}
%
% \subsection{Others}
%
% \subsubsection{\hologo{biber}}
%
%    \begin{macro}{\HoLogo@biber}
%    \begin{macrocode}
\def\HoLogo@biber#1{%
  \HOLOGO@mbox{#1{b}{B}i}%
  \HOLOGO@discretionary
  \HOLOGO@mbox{ber}%
}
%    \end{macrocode}
%    \end{macro}
%    \begin{macro}{\HoLogoCs@biber}
%    \begin{macrocode}
\def\HoLogoCs@biber#1{#1{b}{B}iber}
%    \end{macrocode}
%    \end{macro}
%    \begin{macro}{\HoLogoBkm@biber}
%    \begin{macrocode}
\def\HoLogoBkm@biber#1{%
  #1{b}{B}iber%
}
%    \end{macrocode}
%    \end{macro}
%    \begin{macro}{\HoLogoHtml@biber}
%    \begin{macrocode}
\let\HoLogoHtml@biber\HoLogo@biber
%    \end{macrocode}
%    \end{macro}
%
% \subsubsection{\hologo{KOMAScript}}
%
%    \begin{macro}{\HoLogo@KOMAScript}
%    The definition for \hologo{KOMAScript} is taken
%    from \hologo{KOMAScript} (\xfile{scrlogo.dtx}, reformatted) \cite{scrlogo}:
%\begin{quote}
%\begin{verbatim}
%\@ifundefined{KOMAScript}{%
%  \DeclareRobustCommand{\KOMAScript}{%
%    \textsf{%
%      K\kern.05em O\kern.05emM\kern.05em A%
%      \kern.1em-\kern.1em %
%      Script%
%    }%
%  }%
%}{}
%\end{verbatim}
%\end{quote}
%    \begin{macrocode}
\def\HoLogo@KOMAScript#1{%
  \HoLogoFont@font{KOMAScript}{sf}{%
    \HOLOGO@mbox{%
      K\kern.05em%
      O\kern.05em%
      M\kern.05em%
      A%
    }%
    \kern.1em%
    \HOLOGO@hyphen
    \kern.1em%
    \HOLOGO@mbox{Script}%
  }%
}
%    \end{macrocode}
%    \end{macro}
%    \begin{macro}{\HoLogoBkm@KOMAScript}
%    \begin{macrocode}
\def\HoLogoBkm@KOMAScript#1{%
  KOMA-Script%
}
%    \end{macrocode}
%    \end{macro}
%    \begin{macro}{\HoLogoHtml@KOMAScript}
%    \begin{macrocode}
\def\HoLogoHtml@KOMAScript#1{%
  \HoLogoCss@KOMAScript
  \HoLogoFont@font{KOMAScript}{sf}{%
    \HOLOGO@Span{KOMAScript}{%
      K%
      \HOLOGO@Span{O}{O}%
      M%
      \HOLOGO@Span{A}{A}%
      \HOLOGO@Span{hyphen}{-}%
      Script%
    }%
  }%
}
%    \end{macrocode}
%    \end{macro}
%    \begin{macro}{\HoLogoCss@KOMAScript}
%    \begin{macrocode}
\def\HoLogoCss@KOMAScript{%
  \Css{%
    span.HoLogo-KOMAScript{%
      font-family:sans-serif;%
    }%
  }%
  \Css{%
    span.HoLogo-KOMAScript span.HoLogo-O{%
      padding-left:.05em;%
      padding-right:.05em;%
    }%
  }%
  \Css{%
    span.HoLogo-KOMAScript span.HoLogo-A{%
      padding-left:.05em;%
    }%
  }%
  \Css{%
    span.HoLogo-KOMAScript span.HoLogo-hyphen{%
      padding-left:.1em;%
      padding-right:.1em;%
    }%
  }%
  \global\let\HoLogoCss@KOMAScript\relax
}
%    \end{macrocode}
%    \end{macro}
%
% \subsubsection{\hologo{LyX}}
%
%    \begin{macro}{\HoLogo@LyX}
%    The definition is taken from the documentation source files
%    of \hologo{LyX}, \xfile{Intro.lyx} \cite{LyX}:
%\begin{quote}
%\begin{verbatim}
%\def\LyX{%
%  \texorpdfstring{%
%    L\kern-.1667em\lower.25em\hbox{Y}\kern-.125emX\@%
%  }{%
%    LyX%
%  }%
%}
%\end{verbatim}
%\end{quote}
%    \begin{macrocode}
\def\HoLogo@LyX#1{%
  L%
  \kern-.1667em%
  \lower.25em\hbox{Y}%
  \kern-.125em%
  X%
  \HOLOGO@SpaceFactor
}
%    \end{macrocode}
%    \end{macro}
%    \begin{macro}{\HoLogoHtml@LyX}
%    \begin{macrocode}
\def\HoLogoHtml@LyX#1{%
  \HoLogoCss@LyX
  \HOLOGO@Span{LyX}{%
    L%
    \HOLOGO@Span{y}{Y}%
    X%
  }%
}
%    \end{macrocode}
%    \end{macro}
%    \begin{macro}{\HoLogoCss@LyX}
%    \begin{macrocode}
\def\HoLogoCss@LyX{%
  \Css{%
    span.HoLogo-LyX span.HoLogo-y{%
      position:relative;%
      top:.25em;%
      margin-left:-.1667em;%
      margin-right:-.125em;%
      text-decoration:none;%
    }%
  }%
  \global\let\HoLogoCss@LyX\relax
}
%    \end{macrocode}
%    \end{macro}
%
% \subsubsection{\hologo{NTS}}
%
%    \begin{macro}{\HoLogo@NTS}
%    Definition for \hologo{NTS} can be found in
%    package \xpackage{etex\textunderscore man} for the \hologo{eTeX} manual \cite{etexman}
%    and in package \xpackage{dtklogos} \cite{dtklogos}:
%\begin{quote}
%\begin{verbatim}
%\def\NTS{%
%  \leavevmode
%  \hbox{%
%    $%
%      \cal N%
%      \kern-0.35em%
%      \lower0.5ex\hbox{$\cal T$}%
%      \kern-0.2em%
%      S%
%    $%
%  }%
%}
%\end{verbatim}
%\end{quote}
%    \begin{macrocode}
\def\HoLogo@NTS#1{%
  \HoLogoFont@font{NTS}{sy}{%
    N\/%
    \kern-.35em%
    \lower.5ex\hbox{T\/}%
    \kern-.2em%
    S\/%
  }%
  \HOLOGO@SpaceFactor
}
%    \end{macrocode}
%    \end{macro}
%
% \subsubsection{\Hologo{TTH} (\hologo{TeX} to HTML translator)}
%
%    Source: \url{http://hutchinson.belmont.ma.us/tth/}
%    In the HTML source the second `T' is printed as subscript.
%\begin{quote}
%\begin{verbatim}
%T<sub>T</sub>H
%\end{verbatim}
%\end{quote}
%    \begin{macro}{\HoLogo@TTH}
%    \begin{macrocode}
\def\HoLogo@TTH#1{%
  \ltx@mbox{%
    T\HOLOGO@SubScript{T}H%
  }%
  \HOLOGO@SpaceFactor
}
%    \end{macrocode}
%    \end{macro}
%
%    \begin{macro}{\HoLogoHtml@TTH}
%    \begin{macrocode}
\def\HoLogoHtml@TTH#1{%
  T\HCode{<sub>}T\HCode{</sub>}H%
}
%    \end{macrocode}
%    \end{macro}
%
% \subsubsection{\Hologo{HanTheThanh}}
%
%    Partial source: Package \xpackage{dtklogos}.
%    The double accent is U+1EBF (latin small letter e with circumflex
%    and acute).
%    \begin{macro}{\HoLogo@HanTheThanh}
%    \begin{macrocode}
\def\HoLogo@HanTheThanh#1{%
  \ltx@mbox{H\`an}%
  \HOLOGO@space
  \ltx@mbox{%
    Th%
    \HOLOGO@IfCharExists{"1EBF}{%
      \char"1EBF\relax
    }{%
      \^e\hbox to 0pt{\hss\raise .5ex\hbox{\'{}}}%
    }%
  }%
  \HOLOGO@space
  \ltx@mbox{Th\`anh}%
}
%    \end{macrocode}
%    \end{macro}
%    \begin{macro}{\HoLogoBkm@HanTheThanh}
%    \begin{macrocode}
\def\HoLogoBkm@HanTheThanh#1{%
  H\`an %
  Th\HOLOGO@PdfdocUnicode{\^e}{\9036\277} %
  Th\`anh%
}
%    \end{macrocode}
%    \end{macro}
%    \begin{macro}{\HoLogoHtml@HanTheThanh}
%    \begin{macrocode}
\def\HoLogoHtml@HanTheThanh#1{%
  H\`an %
  Th\HCode{&\ltx@hashchar x1ebf;} %
  Th\`anh%
}
%    \end{macrocode}
%    \end{macro}
%
% \subsection{Driver detection}
%
%    \begin{macrocode}
\HOLOGO@IfExists\InputIfFileExists{%
  \InputIfFileExists{hologo.cfg}{}{}%
}{%
  \ltx@IfUndefined{pdf@filesize}{%
    \def\HOLOGO@InputIfExists{%
      \openin\HOLOGO@temp=hologo.cfg\relax
      \ifeof\HOLOGO@temp
        \closein\HOLOGO@temp
      \else
        \closein\HOLOGO@temp
        \begingroup
          \def\x{LaTeX2e}%
        \expandafter\endgroup
        \ifx\fmtname\x
          \input{hologo.cfg}%
        \else
          \input hologo.cfg\relax
        \fi
      \fi
    }%
    \ltx@IfUndefined{newread}{%
      \chardef\HOLOGO@temp=15 %
      \def\HOLOGO@CheckRead{%
        \ifeof\HOLOGO@temp
          \HOLOGO@InputIfExists
        \else
          \ifcase\HOLOGO@temp
            \@PackageWarningNoLine{hologo}{%
              Configuration file ignored, because\MessageBreak
              a free read register could not be found%
            }%
          \else
            \begingroup
              \count\ltx@cclv=\HOLOGO@temp
              \advance\ltx@cclv by \ltx@minusone
              \edef\x{\endgroup
                \chardef\noexpand\HOLOGO@temp=\the\count\ltx@cclv
                \relax
              }%
            \x
          \fi
        \fi
      }%
    }{%
      \csname newread\endcsname\HOLOGO@temp
      \HOLOGO@InputIfExists
    }%
  }{%
    \edef\HOLOGO@temp{\pdf@filesize{hologo.cfg}}%
    \ifx\HOLOGO@temp\ltx@empty
    \else
      \ifnum\HOLOGO@temp>0 %
        \begingroup
          \def\x{LaTeX2e}%
        \expandafter\endgroup
        \ifx\fmtname\x
          \input{hologo.cfg}%
        \else
          \input hologo.cfg\relax
        \fi
      \else
        \@PackageInfoNoLine{hologo}{%
          Empty configuration file `hologo.cfg' ignored%
        }%
      \fi
    \fi
  }%
}
%    \end{macrocode}
%
%    \begin{macrocode}
\def\HOLOGO@temp#1#2{%
  \kv@define@key{HoLogoDriver}{#1}[]{%
    \begingroup
      \def\HOLOGO@temp{##1}%
      \ltx@onelevel@sanitize\HOLOGO@temp
      \ifx\HOLOGO@temp\ltx@empty
      \else
        \@PackageError{hologo}{%
          Value (\HOLOGO@temp) not permitted for option `#1'%
        }%
        \@ehc
      \fi
    \endgroup
    \def\hologoDriver{#2}%
  }%
}%
\def\HOLOGO@@temp#1#2{%
  \ifx\kv@value\relax
    \HOLOGO@temp{#1}{#1}%
  \else
    \HOLOGO@temp{#1}{#2}%
  \fi
}%
\kv@parse@normalized{%
  pdftex,%
  luatex=pdftex,%
  dvipdfm,%
  dvipdfmx=dvipdfm,%
  dvips,%
  dvipsone=dvips,%
  xdvi=dvips,%
  xetex,%
  vtex,%
}\HOLOGO@@temp
%    \end{macrocode}
%
%    \begin{macrocode}
\kv@define@key{HoLogoDriver}{driverfallback}{%
  \def\HOLOGO@DriverFallback{#1}%
}
%    \end{macrocode}
%
%    \begin{macro}{\HOLOGO@DriverFallback}
%    \begin{macrocode}
\def\HOLOGO@DriverFallback{dvips}
%    \end{macrocode}
%    \end{macro}
%
%    \begin{macro}{\hologoDriverSetup}
%    \begin{macrocode}
\def\hologoDriverSetup{%
  \let\hologoDriver\ltx@undefined
  \HOLOGO@DriverSetup
}
%    \end{macrocode}
%    \end{macro}
%
%    \begin{macro}{\HOLOGO@DriverSetup}
%    \begin{macrocode}
\def\HOLOGO@DriverSetup#1{%
  \kvsetkeys{HoLogoDriver}{#1}%
  \HOLOGO@CheckDriver
  \ltx@ifundefined{hologoDriver}{%
    \begingroup
    \edef\x{\endgroup
      \noexpand\kvsetkeys{HoLogoDriver}{\HOLOGO@DriverFallback}%
    }\x
  }{}%
  \@PackageInfoNoLine{hologo}{Using driver `\hologoDriver'}%
}
%    \end{macrocode}
%    \end{macro}
%
%    \begin{macro}{\HOLOGO@CheckDriver}
%    \begin{macrocode}
\def\HOLOGO@CheckDriver{%
  \ifpdf
    \def\hologoDriver{pdftex}%
    \let\HOLOGO@pdfliteral\pdfliteral
    \ifluatex
      \ifx\pdfextension\@undefined\else
        \protected\def\pdfliteral{\pdfextension literal}%
        \let\HOLOGO@pdfliteral\pdfliteral
      \fi
      \ltx@IfUndefined{HOLOGO@pdfliteral}{%
        \ifnum\luatexversion<36 %
        \else
          \begingroup
            \let\HOLOGO@temp\endgroup
            \ifcase0%
                \directlua{%
                  if tex.enableprimitives then %
                    tex.enableprimitives('HOLOGO@', {'pdfliteral'})%
                  else %
                    tex.print('1')%
                  end%
                }%
                \ifx\HOLOGO@pdfliteral\@undefined 1\fi%
                \relax%
              \endgroup
              \let\HOLOGO@temp\relax
              \global\let\HOLOGO@pdfliteral\HOLOGO@pdfliteral
            \fi%
          \HOLOGO@temp
        \fi
      }{}%
    \fi
    \ltx@IfUndefined{HOLOGO@pdfliteral}{%
      \@PackageWarningNoLine{hologo}{%
        Cannot find \string\pdfliteral
      }%
    }{}%
  \else
    \ifxetex
      \def\hologoDriver{xetex}%
    \else
      \ifvtex
        \def\hologoDriver{vtex}%
      \fi
    \fi
  \fi
}
%    \end{macrocode}
%    \end{macro}
%
%    \begin{macro}{\HOLOGO@WarningUnsupportedDriver}
%    \begin{macrocode}
\def\HOLOGO@WarningUnsupportedDriver#1{%
  \@PackageWarningNoLine{hologo}{%
    Logo `#1' needs driver specific macros,\MessageBreak
    but driver `\hologoDriver' is not supported.\MessageBreak
    Use a different driver or\MessageBreak
    load package `graphics' or `pgf'%
  }%
}
%    \end{macrocode}
%    \end{macro}
%
% \subsubsection{Reflect box macros}
%
%    Skip driver part if not needed.
%    \begin{macrocode}
\ltx@IfUndefined{reflectbox}{}{%
  \ltx@IfUndefined{rotatebox}{}{%
    \HOLOGO@AtEnd
  }%
}
\ltx@IfUndefined{pgftext}{}{%
  \HOLOGO@AtEnd
}
\ltx@IfUndefined{psscalebox}{}{%
  \HOLOGO@AtEnd
}
%    \end{macrocode}
%
%    \begin{macrocode}
\def\HOLOGO@temp{LaTeX2e}
\ifx\fmtname\HOLOGO@temp
  \RequirePackage{kvoptions}[2011/06/30]%
  \ProcessKeyvalOptions{HoLogoDriver}%
\fi
\HOLOGO@DriverSetup{}
%    \end{macrocode}
%
%    \begin{macro}{\HOLOGO@ReflectBox}
%    \begin{macrocode}
\def\HOLOGO@ReflectBox#1{%
  \begingroup
    \setbox\ltx@zero\hbox{\begingroup#1\endgroup}%
    \setbox\ltx@two\hbox{%
      \kern\wd\ltx@zero
      \csname HOLOGO@ScaleBox@\hologoDriver\endcsname{-1}{1}{%
        \hbox to 0pt{\copy\ltx@zero\hss}%
      }%
    }%
    \wd\ltx@two=\wd\ltx@zero
    \box\ltx@two
  \endgroup
}
%    \end{macrocode}
%    \end{macro}
%
%    \begin{macro}{\HOLOGO@PointReflectBox}
%    \begin{macrocode}
\def\HOLOGO@PointReflectBox#1{%
  \begingroup
    \setbox\ltx@zero\hbox{\begingroup#1\endgroup}%
    \setbox\ltx@two\hbox{%
      \kern\wd\ltx@zero
      \raise\ht\ltx@zero\hbox{%
        \csname HOLOGO@ScaleBox@\hologoDriver\endcsname{-1}{-1}{%
          \hbox to 0pt{\copy\ltx@zero\hss}%
        }%
      }%
    }%
    \wd\ltx@two=\wd\ltx@zero
    \box\ltx@two
  \endgroup
}
%    \end{macrocode}
%    \end{macro}
%
%    We must define all variants because of dynamic driver setup.
%    \begin{macrocode}
\def\HOLOGO@temp#1#2{#2}
%    \end{macrocode}
%
%    \begin{macro}{\HOLOGO@ScaleBox@pdftex}
%    \begin{macrocode}
\HOLOGO@temp{pdftex}{%
  \def\HOLOGO@ScaleBox@pdftex#1#2#3{%
    \HOLOGO@pdfliteral{%
      q #1 0 0 #2 0 0 cm%
    }%
    #3%
    \HOLOGO@pdfliteral{%
      Q%
    }%
  }%
}
%    \end{macrocode}
%    \end{macro}
%    \begin{macro}{\HOLOGO@ScaleBox@dvips}
%    \begin{macrocode}
\HOLOGO@temp{dvips}{%
  \def\HOLOGO@ScaleBox@dvips#1#2#3{%
    \special{ps:%
      gsave %
      currentpoint %
      currentpoint translate %
      #1 #2 scale %
      neg exch neg exch translate%
    }%
    #3%
    \special{ps:%
      currentpoint %
      grestore %
      moveto%
    }%
  }%
}
%    \end{macrocode}
%    \end{macro}
%    \begin{macro}{\HOLOGO@ScaleBox@dvipdfm}
%    \begin{macrocode}
\HOLOGO@temp{dvipdfm}{%
  \let\HOLOGO@ScaleBox@dvipdfm\HOLOGO@ScaleBox@dvips
}
%    \end{macrocode}
%    \end{macro}
%    Since \hologo{XeTeX} v0.6.
%    \begin{macro}{\HOLOGO@ScaleBox@xetex}
%    \begin{macrocode}
\HOLOGO@temp{xetex}{%
  \def\HOLOGO@ScaleBox@xetex#1#2#3{%
    \special{x:gsave}%
    \special{x:scale #1 #2}%
    #3%
    \special{x:grestore}%
  }%
}
%    \end{macrocode}
%    \end{macro}
%    \begin{macro}{\HOLOGO@ScaleBox@vtex}
%    \begin{macrocode}
\HOLOGO@temp{vtex}{%
  \def\HOLOGO@ScaleBox@vtex#1#2#3{%
    \special{r(#1,0,0,#2,0,0}%
    #3%
    \special{r)}%
  }%
}
%    \end{macrocode}
%    \end{macro}
%
%    \begin{macrocode}
\HOLOGO@AtEnd%
%</package>
%    \end{macrocode}
%
% \section{Test}
%
% \subsection{Catcode checks for loading}
%
%    \begin{macrocode}
%<*test1>
%    \end{macrocode}
%    \begin{macrocode}
\catcode`\{=1 %
\catcode`\}=2 %
\catcode`\#=6 %
\catcode`\@=11 %
\expandafter\ifx\csname count@\endcsname\relax
  \countdef\count@=255 %
\fi
\expandafter\ifx\csname @gobble\endcsname\relax
  \long\def\@gobble#1{}%
\fi
\expandafter\ifx\csname @firstofone\endcsname\relax
  \long\def\@firstofone#1{#1}%
\fi
\expandafter\ifx\csname loop\endcsname\relax
  \expandafter\@firstofone
\else
  \expandafter\@gobble
\fi
{%
  \def\loop#1\repeat{%
    \def\body{#1}%
    \iterate
  }%
  \def\iterate{%
    \body
      \let\next\iterate
    \else
      \let\next\relax
    \fi
    \next
  }%
  \let\repeat=\fi
}%
\def\RestoreCatcodes{}
\count@=0 %
\loop
  \edef\RestoreCatcodes{%
    \RestoreCatcodes
    \catcode\the\count@=\the\catcode\count@\relax
  }%
\ifnum\count@<255 %
  \advance\count@ 1 %
\repeat

\def\RangeCatcodeInvalid#1#2{%
  \count@=#1\relax
  \loop
    \catcode\count@=15 %
  \ifnum\count@<#2\relax
    \advance\count@ 1 %
  \repeat
}
\def\RangeCatcodeCheck#1#2#3{%
  \count@=#1\relax
  \loop
    \ifnum#3=\catcode\count@
    \else
      \errmessage{%
        Character \the\count@\space
        with wrong catcode \the\catcode\count@\space
        instead of \number#3%
      }%
    \fi
  \ifnum\count@<#2\relax
    \advance\count@ 1 %
  \repeat
}
\def\space{ }
\expandafter\ifx\csname LoadCommand\endcsname\relax
  \def\LoadCommand{\input hologo.sty\relax}%
\fi
\def\Test{%
  \RangeCatcodeInvalid{0}{47}%
  \RangeCatcodeInvalid{58}{64}%
  \RangeCatcodeInvalid{91}{96}%
  \RangeCatcodeInvalid{123}{255}%
  \catcode`\@=12 %
  \catcode`\\=0 %
  \catcode`\%=14 %
  \LoadCommand
  \RangeCatcodeCheck{0}{36}{15}%
  \RangeCatcodeCheck{37}{37}{14}%
  \RangeCatcodeCheck{38}{47}{15}%
  \RangeCatcodeCheck{48}{57}{12}%
  \RangeCatcodeCheck{58}{63}{15}%
  \RangeCatcodeCheck{64}{64}{12}%
  \RangeCatcodeCheck{65}{90}{11}%
  \RangeCatcodeCheck{91}{91}{15}%
  \RangeCatcodeCheck{92}{92}{0}%
  \RangeCatcodeCheck{93}{96}{15}%
  \RangeCatcodeCheck{97}{122}{11}%
  \RangeCatcodeCheck{123}{255}{15}%
  \RestoreCatcodes
}
\Test
\csname @@end\endcsname
\end
%    \end{macrocode}
%    \begin{macrocode}
%</test1>
%    \end{macrocode}
%
% \subsection{Spacefactor}
%
%    The space factor must be 1000 after a logo. If it is greater 1000
%    then the following space is a space after a sentence closing point.
%    If the space factor is smaller 1000 then an immediate following
%    dot is interpreted as abbreviation, not sentence closing point.
%
%    \begin{macrocode}
%<*test-spacefactor>
\NeedsTeXFormat{LaTeX2e}
\documentclass{article}
\usepackage{hologo}[2016/05/12]
\usepackage{kvsetkeys}
\usepackage{qstest}
\IncludeTests{*}
\LogTests{log}{*}{*}
\begin{document}
\begin{qstest}{spacefactor}{spacefactor}
\newcommand*{\Test}[1]{%
  \sbox0{%
    \hologo{#1}%
    \Expect*{1000 (#1)}*{\the\spacefactor\space(#1)}%
  }%
}%
\makeatletter
\def\TestList{}
\def\hologoEntry#1#2#3{%
  \edef\TestList{%
    \ifx\TestList\@empty
    \else
      \TestList,%
    \fi
    #1%
    \ifx\\#2\\%
    \else
      ={variant=#2}%
    \fi
  }%
}
\hologoList
\expandafter\kv@parse@normalized\expandafter{%
  \TestList
}{%
  \begingroup
    \let\@logo=\kv@key
    \ifx\kv@value\relax
    \else
      \expandafter\hologoLogoSetup\expandafter\@logo\expandafter{%
        \kv@value
      }%
    \fi
    \Test\@logo
  \endgroup
  \@gobbletwo
}
\end{qstest}
\end{document}
%</test-spacefactor>
%    \end{macrocode}
%
% \subsection{Complete list}
%
%    \begin{macrocode}
%<*test-list>
\NeedsTeXFormat{LaTeX2e}
\documentclass[12pt,a4paper]{article}
\usepackage{hologo}[2016/05/12]
\usepackage[T1]{fontenc}
\usepackage{lmodern}
\usepackage{parskip}
\usepackage[unicode]{hyperref}[2011/09/28]
\usepackage{bookmark}[2011/09/19]
\bookmarksetup{%
  numbered,%
  open,%
  openlevel=2,%
}
\renewcommand*{\contentsname}{List of logos}
\begin{document}
\tableofcontents
\def\TestFont#1#2#3#4#5#6{%
  \begingroup
    \usefont{#3}{#4}{#5}{#6}%
    \HologoVariant{#1}{#2}/\hologoVariant{#1}{#2}%
    \quad
    \begingroup\scriptsize\hologoVariant{#1}{#2}\endgroup
    \quad
  \endgroup
  (#3/#4/#5/#6)%
  \par
}
\makeatletter
\def\hologoEntry#1#2#3{%
  \section{%
    \HologoVariant{#1}{#2}/\hologoVariant{#1}{#2} %
    {[#1\ifx\\#2\\\else\space(#2)\fi]}% hash-ok
  }% braces around [] because of bug in tex4ht
  \begingroup
    \hypersetup{unicode=false}%
    \bookmark[%
      dest=\@currentHref,%
      rellevel=1,%
      keeplevel,%
    ]{%
      \HologoVariant{#1}{#2}/\hologoVariant{#1}{#2} %
      (PDFDocEncoding)%
    }%
  \endgroup
  \TestFont{#1}{#2}{OT1}{cmr}{m}{n}%
  \TestFont{#1}{#2}{OT1}{cmss}{m}{n}%
  \TestFont{#1}{#2}{OT1}{cmr}{b}{n}%
  \TestFont{#1}{#2}{OT1}{cmr}{m}{it}%
  \TestFont{#1}{#2}{OT1}{cmtt}{m}{n}%
  \TestFont{#1}{#2}{T1}{lmr}{m}{n}%
  \TestFont{#1}{#2}{T1}{lmss}{m}{n}%
  \TestFont{#1}{#2}{T1}{lmr}{b}{n}%
  \TestFont{#1}{#2}{T1}{lmr}{m}{it}%
  \TestFont{#1}{#2}{T1}{lmtt}{m}{n}%
  \TestFont{#1}{#2}{T1}{lmvtt}{m}{n}%
  \TestFont{#1}{#2}{T1}{qtm}{m}{n}%
  \TestFont{#1}{#2}{T1}{qhv}{m}{n}%
  \TestFont{#1}{#2}{T1}{qtm}{b}{n}%
  \TestFont{#1}{#2}{T1}{qtm}{m}{it}%
  \TestFont{#1}{#2}{T1}{qcr}{m}{n}%
  \newpage
}
\makeatother
\hologoList
\end{document}
%</test-list>
%    \end{macrocode}
%
% \section{Installation}
%
% \subsection{Download}
%
% \paragraph{Package.} This package is available on
% CTAN\footnote{\url{ftp://ftp.ctan.org/tex-archive/}}:
% \begin{description}
% \item[\CTAN{macros/latex/contrib/oberdiek/hologo.dtx}] The source file.
% \item[\CTAN{macros/latex/contrib/oberdiek/hologo.pdf}] Documentation.
% \end{description}
%
%
% \paragraph{Bundle.} All the packages of the bundle `oberdiek'
% are also available in a TDS compliant ZIP archive. There
% the packages are already unpacked and the documentation files
% are generated. The files and directories obey the TDS standard.
% \begin{description}
% \item[\CTAN{install/macros/latex/contrib/oberdiek.tds.zip}]
% \end{description}
% \emph{TDS} refers to the standard ``A Directory Structure
% for \TeX\ Files'' (\CTAN{tds/tds.pdf}). Directories
% with \xfile{texmf} in their name are usually organized this way.
%
% \subsection{Bundle installation}
%
% \paragraph{Unpacking.} Unpack the \xfile{oberdiek.tds.zip} in the
% TDS tree (also known as \xfile{texmf} tree) of your choice.
% Example (linux):
% \begin{quote}
%   |unzip oberdiek.tds.zip -d ~/texmf|
% \end{quote}
%
% \paragraph{Script installation.}
% Check the directory \xfile{TDS:scripts/oberdiek/} for
% scripts that need further installation steps.
% Package \xpackage{attachfile2} comes with the Perl script
% \xfile{pdfatfi.pl} that should be installed in such a way
% that it can be called as \texttt{pdfatfi}.
% Example (linux):
% \begin{quote}
%   |chmod +x scripts/oberdiek/pdfatfi.pl|\\
%   |cp scripts/oberdiek/pdfatfi.pl /usr/local/bin/|
% \end{quote}
%
% \subsection{Package installation}
%
% \paragraph{Unpacking.} The \xfile{.dtx} file is a self-extracting
% \docstrip\ archive. The files are extracted by running the
% \xfile{.dtx} through \plainTeX:
% \begin{quote}
%   \verb|tex hologo.dtx|
% \end{quote}
%
% \paragraph{TDS.} Now the different files must be moved into
% the different directories in your installation TDS tree
% (also known as \xfile{texmf} tree):
% \begin{quote}
% \def\t{^^A
% \begin{tabular}{@{}>{\ttfamily}l@{ $\rightarrow$ }>{\ttfamily}l@{}}
%   hologo.sty & tex/generic/oberdiek/hologo.sty\\
%   hologo.pdf & doc/latex/oberdiek/hologo.pdf\\
%   example/hologo-example.tex & doc/latex/oberdiek/example/hologo-example.tex\\
%   test/hologo-test1.tex & doc/latex/oberdiek/test/hologo-test1.tex\\
%   test/hologo-test-spacefactor.tex & doc/latex/oberdiek/test/hologo-test-spacefactor.tex\\
%   test/hologo-test-list.tex & doc/latex/oberdiek/test/hologo-test-list.tex\\
%   hologo.dtx & source/latex/oberdiek/hologo.dtx\\
% \end{tabular}^^A
% }^^A
% \sbox0{\t}^^A
% \ifdim\wd0>\linewidth
%   \begingroup
%     \advance\linewidth by\leftmargin
%     \advance\linewidth by\rightmargin
%   \edef\x{\endgroup
%     \def\noexpand\lw{\the\linewidth}^^A
%   }\x
%   \def\lwbox{^^A
%     \leavevmode
%     \hbox to \linewidth{^^A
%       \kern-\leftmargin\relax
%       \hss
%       \usebox0
%       \hss
%       \kern-\rightmargin\relax
%     }^^A
%   }^^A
%   \ifdim\wd0>\lw
%     \sbox0{\small\t}^^A
%     \ifdim\wd0>\linewidth
%       \ifdim\wd0>\lw
%         \sbox0{\footnotesize\t}^^A
%         \ifdim\wd0>\linewidth
%           \ifdim\wd0>\lw
%             \sbox0{\scriptsize\t}^^A
%             \ifdim\wd0>\linewidth
%               \ifdim\wd0>\lw
%                 \sbox0{\tiny\t}^^A
%                 \ifdim\wd0>\linewidth
%                   \lwbox
%                 \else
%                   \usebox0
%                 \fi
%               \else
%                 \lwbox
%               \fi
%             \else
%               \usebox0
%             \fi
%           \else
%             \lwbox
%           \fi
%         \else
%           \usebox0
%         \fi
%       \else
%         \lwbox
%       \fi
%     \else
%       \usebox0
%     \fi
%   \else
%     \lwbox
%   \fi
% \else
%   \usebox0
% \fi
% \end{quote}
% If you have a \xfile{docstrip.cfg} that configures and enables \docstrip's
% TDS installing feature, then some files can already be in the right
% place, see the documentation of \docstrip.
%
% \subsection{Refresh file name databases}
%
% If your \TeX~distribution
% (\teTeX, \mikTeX, \dots) relies on file name databases, you must refresh
% these. For example, \teTeX\ users run \verb|texhash| or
% \verb|mktexlsr|.
%
% \subsection{Some details for the interested}
%
% \paragraph{Attached source.}
%
% The PDF documentation on CTAN also includes the
% \xfile{.dtx} source file. It can be extracted by
% AcrobatReader 6 or higher. Another option is \textsf{pdftk},
% e.g. unpack the file into the current directory:
% \begin{quote}
%   \verb|pdftk hologo.pdf unpack_files output .|
% \end{quote}
%
% \paragraph{Unpacking with \LaTeX.}
% The \xfile{.dtx} chooses its action depending on the format:
% \begin{description}
% \item[\plainTeX:] Run \docstrip\ and extract the files.
% \item[\LaTeX:] Generate the documentation.
% \end{description}
% If you insist on using \LaTeX\ for \docstrip\ (really,
% \docstrip\ does not need \LaTeX), then inform the autodetect routine
% about your intention:
% \begin{quote}
%   \verb|latex \let\install=y\input{hologo.dtx}|
% \end{quote}
% Do not forget to quote the argument according to the demands
% of your shell.
%
% \paragraph{Generating the documentation.}
% You can use both the \xfile{.dtx} or the \xfile{.drv} to generate
% the documentation. The process can be configured by the
% configuration file \xfile{ltxdoc.cfg}. For instance, put this
% line into this file, if you want to have A4 as paper format:
% \begin{quote}
%   \verb|\PassOptionsToClass{a4paper}{article}|
% \end{quote}
% An example follows how to generate the
% documentation with pdf\LaTeX:
% \begin{quote}
%\begin{verbatim}
%pdflatex hologo.dtx
%makeindex -s gind.ist hologo.idx
%pdflatex hologo.dtx
%makeindex -s gind.ist hologo.idx
%pdflatex hologo.dtx
%\end{verbatim}
% \end{quote}
%
% \section{Catalogue}
%
% The following XML file can be used as source for the
% \href{http://mirror.ctan.org/help/Catalogue/catalogue.html}{\TeX\ Catalogue}.
% The elements \texttt{caption} and \texttt{description} are imported
% from the original XML file from the Catalogue.
% The name of the XML file in the Catalogue is \xfile{hologo.xml}.
%    \begin{macrocode}
%<*catalogue>
<?xml version='1.0' encoding='us-ascii'?>
<!DOCTYPE entry SYSTEM 'catalogue.dtd'>
<entry datestamp='$Date$' modifier='$Author$' id='hologo'>
  <name>hologo</name>
  <caption>A collection of logos with bookmark support.</caption>
  <authorref id='auth:oberdiek'/>
  <copyright owner='Heiko Oberdiek' year='2010-2012'/>
  <license type='lppl1.3'/>
  <version number='1.10'/>
  <description>
    The package defines a single command <tt>\hologo</tt>, whose
    argument is the usual case-confused ASCII version of the logo.
    The command is bookmark-enabled, so that every logo becomes
    available in bookmarks without further work.
    <p/>
    The package is part of the <xref refid='oberdiek'>oberdiek</xref>
    bundle.
  </description>
  <documentation details='Package documentation'
      href='ctan:/macros/latex/contrib/oberdiek/hologo.pdf'/>
  <ctan file='true' path='/macros/latex/contrib/oberdiek/hologo.dtx'/>
  <miktex location='oberdiek'/>
  <texlive location='oberdiek'/>
  <install path='/macros/latex/contrib/oberdiek/oberdiek.tds.zip'/>
</entry>
%</catalogue>
%    \end{macrocode}
%
% \begin{thebibliography}{9}
% \raggedright
%
% \bibitem{btxdoc}
% Oren Patashnik,
% \textit{\hologo{BibTeX}ing},
% 1988-02-08.\\
% \CTAN{biblio/bibtex/base/}
%
% \bibitem{dtklogos}
% Gerd Neugebauer, DANTE,
% \textit{Package \xpackage{dtklogos}},
% 2011-04-25.\\
% \CTAN{usergrps/dante/dtk/dtklogos.sty}
%
% \bibitem{etexman}
% The \hologo{NTS} Team,
% \textit{The \hologo{eTeX} manual},
% 1998-02.\\
% \CTAN{systems/e-tex/v2/doc/}
%
% \bibitem{ExTeX-FAQ}
% The \hologo{ExTeX} group,
% \textit{\hologo{ExTeX}: FAQ -- How is \hologo{ExTeX} typeset?},
% 2007-04-14.\\
% \url{http://www.extex.org/documentation/faq.html}
%
% \bibitem{LyX}
% %@MISC{ LyX,
% %  title = {{LyX 2.0.0 -- The Document Processor [Computer software and manual]}},
% %  author = {{The LyX Team}},
% %  howpublished = {Internet: http://www.lyx.org},
% %  year = {2011-05-08},
% %  note = {Retrieved May 10, 2011, from http://www.lyx.org},
% %  url = {http://www.lyx.org/}
% %}
% The \hologo{LyX} Team,
% \textit{\hologo{LyX} -- The Document Processor},
% 2011-05-08.\\
% \url{http://www.lyx.org/}
%
% \bibitem{OzTeX}
% Andrew Trevorrow,
% \hologo{OzTeX} FAQ: What is the correct way to typeset ``\hologo{OzTeX}''?,
% 2011-09-15 (visited).
% \url{http://www.trevorrow.com/oztex/ozfaq.html#oztex-logo}
%
% \bibitem{PiCTeX}
% Michael Wichura,
% \textit{The \hologo{PiCTeX} macro package},
% 1987-09-21.
% \CTAN{graphics/pictex/}
%
% \bibitem{scrlogo}
% Markus Kohm,
% \textit{\hologo{KOMAScript} Datei \xfile{scrlogo.dtx}},
% 2009-01-30.\\
% \CTAN{install/macros/latex/contrib/komascript.tds.zip}
%
% \end{thebibliography}
%
% \begin{History}
%   \begin{Version}{2010/04/08 v1.0}
%   \item
%     The first version.
%   \end{Version}
%   \begin{Version}{2010/04/16 v1.1}
%   \item
%     \cs{Hologo} added for support of logos at start of a sentence.
%   \item
%     \cs{hologoSetup} and \cs{hologoLogoSetup} added.
%   \item
%     Options \xoption{break}, \xoption{hyphenbreak}, \xoption{spacebreak}
%     added.
%   \item
%     Variant support added by option \xoption{variant}.
%   \end{Version}
%   \begin{Version}{2010/04/24 v1.2}
%   \item
%     \hologo{LaTeX3} added.
%   \item
%     \hologo{VTeX} added.
%   \end{Version}
%   \begin{Version}{2010/11/21 v1.3}
%   \item
%     \hologo{iniTeX}, \hologo{virTeX} added.
%   \end{Version}
%   \begin{Version}{2011/03/25 v1.4}
%   \item
%     \hologo{ConTeXt} with variants added.
%   \item
%     Option \xoption{discretionarybreak} added as refinement for
%     option \xoption{break}.
%   \end{Version}
%   \begin{Version}{2011/04/21 v1.5}
%   \item
%     Wrong TDS directory for test files fixed.
%   \end{Version}
%   \begin{Version}{2011/10/01 v1.6}
%   \item
%     Support for package \xpackage{tex4ht} added.
%   \item
%     Support for \cs{csname} added if \cs{ifincsname} is available.
%   \item
%     New logos:
%     \hologo{(La)TeX},
%     \hologo{biber},
%     \hologo{BibTeX} (\xoption{sc}, \xoption{sf}),
%     \hologo{emTeX},
%     \hologo{ExTeX},
%     \hologo{KOMAScript},
%     \hologo{La},
%     \hologo{LyX},
%     \hologo{MiKTeX},
%     \hologo{NTS},
%     \hologo{OzMF},
%     \hologo{OzMP},
%     \hologo{OzTeX},
%     \hologo{OzTtH},
%     \hologo{PCTeX},
%     \hologo{PiC},
%     \hologo{PiCTeX},
%     \hologo{METAFONT},
%     \hologo{MetaFun},
%     \hologo{METAPOST},
%     \hologo{MetaPost},
%     \hologo{SLiTeX} (\xoption{lift}, \xoption{narrow}, \xoption{simple}),
%     \hologo{SliTeX} (\xoption{narrow}, \xoption{simple}, \xoption{lift}),
%     \hologo{teTeX}.
%   \item
%     Fixes:
%     \hologo{iniTeX},
%     \hologo{pdfLaTeX},
%     \hologo{pdfTeX},
%     \hologo{virTeX}.
%   \item
%     \cs{hologoFontSetup} and \cs{hologoLogoFontSetup} added.
%   \item
%     \cs{hologoVariant} and \cs{HologoVariant} added.
%   \end{Version}
%   \begin{Version}{2011/11/22 v1.7}
%   \item
%     New logos:
%     \hologo{BibTeX8},
%     \hologo{LaTeXML},
%     \hologo{SageTeX},
%     \hologo{TeX4ht},
%     \hologo{TTH}.
%   \item
%     \hologo{Xe} and friends: Driver stuff fixed.
%   \item
%     \hologo{Xe} and friends: Support for italic added.
%   \item
%     \hologo{Xe} and friends: Package support for \xpackage{pgf}
%     and \xpackage{pstricks} added.
%   \end{Version}
%   \begin{Version}{2011/11/29 v1.8}
%   \item
%     New logos:
%     \hologo{HanTheThanh}.
%   \end{Version}
%   \begin{Version}{2011/12/21 v1.9}
%   \item
%     Patch for package \xpackage{ifxetex} added for the case that
%     \cs{newif} is undefined in \hologo{iniTeX}.
%   \item
%     Some fixes for \hologo{iniTeX}.
%   \end{Version}
%   \begin{Version}{2012/04/26 v1.10}
%   \item
%     Fix in bookmark version of logo ``\hologo{HanTheThanh}''.
%   \end{Version}
%   \begin{Version}{2016/05/12 v1.11}
%   \item
%     Update HOLOGO@IfCharExists (previously in texlive)
%   \item define pdfliteral in current luatex.
%   \end{Version}
% \end{History}
%
% \PrintIndex
%
% \Finale
\endinput
%
        \else
          \input hologo.cfg\relax
        \fi
      \else
        \@PackageInfoNoLine{hologo}{%
          Empty configuration file `hologo.cfg' ignored%
        }%
      \fi
    \fi
  }%
}
%    \end{macrocode}
%
%    \begin{macrocode}
\def\HOLOGO@temp#1#2{%
  \kv@define@key{HoLogoDriver}{#1}[]{%
    \begingroup
      \def\HOLOGO@temp{##1}%
      \ltx@onelevel@sanitize\HOLOGO@temp
      \ifx\HOLOGO@temp\ltx@empty
      \else
        \@PackageError{hologo}{%
          Value (\HOLOGO@temp) not permitted for option `#1'%
        }%
        \@ehc
      \fi
    \endgroup
    \def\hologoDriver{#2}%
  }%
}%
\def\HOLOGO@@temp#1#2{%
  \ifx\kv@value\relax
    \HOLOGO@temp{#1}{#1}%
  \else
    \HOLOGO@temp{#1}{#2}%
  \fi
}%
\kv@parse@normalized{%
  pdftex,%
  luatex=pdftex,%
  dvipdfm,%
  dvipdfmx=dvipdfm,%
  dvips,%
  dvipsone=dvips,%
  xdvi=dvips,%
  xetex,%
  vtex,%
}\HOLOGO@@temp
%    \end{macrocode}
%
%    \begin{macrocode}
\kv@define@key{HoLogoDriver}{driverfallback}{%
  \def\HOLOGO@DriverFallback{#1}%
}
%    \end{macrocode}
%
%    \begin{macro}{\HOLOGO@DriverFallback}
%    \begin{macrocode}
\def\HOLOGO@DriverFallback{dvips}
%    \end{macrocode}
%    \end{macro}
%
%    \begin{macro}{\hologoDriverSetup}
%    \begin{macrocode}
\def\hologoDriverSetup{%
  \let\hologoDriver\ltx@undefined
  \HOLOGO@DriverSetup
}
%    \end{macrocode}
%    \end{macro}
%
%    \begin{macro}{\HOLOGO@DriverSetup}
%    \begin{macrocode}
\def\HOLOGO@DriverSetup#1{%
  \kvsetkeys{HoLogoDriver}{#1}%
  \HOLOGO@CheckDriver
  \ltx@ifundefined{hologoDriver}{%
    \begingroup
    \edef\x{\endgroup
      \noexpand\kvsetkeys{HoLogoDriver}{\HOLOGO@DriverFallback}%
    }\x
  }{}%
  \@PackageInfoNoLine{hologo}{Using driver `\hologoDriver'}%
}
%    \end{macrocode}
%    \end{macro}
%
%    \begin{macro}{\HOLOGO@CheckDriver}
%    \begin{macrocode}
\def\HOLOGO@CheckDriver{%
  \ifpdf
    \def\hologoDriver{pdftex}%
    \let\HOLOGO@pdfliteral\pdfliteral
    \ifluatex
      \ifx\pdfextension\@undefined\else
        \protected\def\pdfliteral{\pdfextension literal}%
        \let\HOLOGO@pdfliteral\pdfliteral
      \fi
      \ltx@IfUndefined{HOLOGO@pdfliteral}{%
        \ifnum\luatexversion<36 %
        \else
          \begingroup
            \let\HOLOGO@temp\endgroup
            \ifcase0%
                \directlua{%
                  if tex.enableprimitives then %
                    tex.enableprimitives('HOLOGO@', {'pdfliteral'})%
                  else %
                    tex.print('1')%
                  end%
                }%
                \ifx\HOLOGO@pdfliteral\@undefined 1\fi%
                \relax%
              \endgroup
              \let\HOLOGO@temp\relax
              \global\let\HOLOGO@pdfliteral\HOLOGO@pdfliteral
            \fi%
          \HOLOGO@temp
        \fi
      }{}%
    \fi
    \ltx@IfUndefined{HOLOGO@pdfliteral}{%
      \@PackageWarningNoLine{hologo}{%
        Cannot find \string\pdfliteral
      }%
    }{}%
  \else
    \ifxetex
      \def\hologoDriver{xetex}%
    \else
      \ifvtex
        \def\hologoDriver{vtex}%
      \fi
    \fi
  \fi
}
%    \end{macrocode}
%    \end{macro}
%
%    \begin{macro}{\HOLOGO@WarningUnsupportedDriver}
%    \begin{macrocode}
\def\HOLOGO@WarningUnsupportedDriver#1{%
  \@PackageWarningNoLine{hologo}{%
    Logo `#1' needs driver specific macros,\MessageBreak
    but driver `\hologoDriver' is not supported.\MessageBreak
    Use a different driver or\MessageBreak
    load package `graphics' or `pgf'%
  }%
}
%    \end{macrocode}
%    \end{macro}
%
% \subsubsection{Reflect box macros}
%
%    Skip driver part if not needed.
%    \begin{macrocode}
\ltx@IfUndefined{reflectbox}{}{%
  \ltx@IfUndefined{rotatebox}{}{%
    \HOLOGO@AtEnd
  }%
}
\ltx@IfUndefined{pgftext}{}{%
  \HOLOGO@AtEnd
}
\ltx@IfUndefined{psscalebox}{}{%
  \HOLOGO@AtEnd
}
%    \end{macrocode}
%
%    \begin{macrocode}
\def\HOLOGO@temp{LaTeX2e}
\ifx\fmtname\HOLOGO@temp
  \RequirePackage{kvoptions}[2011/06/30]%
  \ProcessKeyvalOptions{HoLogoDriver}%
\fi
\HOLOGO@DriverSetup{}
%    \end{macrocode}
%
%    \begin{macro}{\HOLOGO@ReflectBox}
%    \begin{macrocode}
\def\HOLOGO@ReflectBox#1{%
  \begingroup
    \setbox\ltx@zero\hbox{\begingroup#1\endgroup}%
    \setbox\ltx@two\hbox{%
      \kern\wd\ltx@zero
      \csname HOLOGO@ScaleBox@\hologoDriver\endcsname{-1}{1}{%
        \hbox to 0pt{\copy\ltx@zero\hss}%
      }%
    }%
    \wd\ltx@two=\wd\ltx@zero
    \box\ltx@two
  \endgroup
}
%    \end{macrocode}
%    \end{macro}
%
%    \begin{macro}{\HOLOGO@PointReflectBox}
%    \begin{macrocode}
\def\HOLOGO@PointReflectBox#1{%
  \begingroup
    \setbox\ltx@zero\hbox{\begingroup#1\endgroup}%
    \setbox\ltx@two\hbox{%
      \kern\wd\ltx@zero
      \raise\ht\ltx@zero\hbox{%
        \csname HOLOGO@ScaleBox@\hologoDriver\endcsname{-1}{-1}{%
          \hbox to 0pt{\copy\ltx@zero\hss}%
        }%
      }%
    }%
    \wd\ltx@two=\wd\ltx@zero
    \box\ltx@two
  \endgroup
}
%    \end{macrocode}
%    \end{macro}
%
%    We must define all variants because of dynamic driver setup.
%    \begin{macrocode}
\def\HOLOGO@temp#1#2{#2}
%    \end{macrocode}
%
%    \begin{macro}{\HOLOGO@ScaleBox@pdftex}
%    \begin{macrocode}
\HOLOGO@temp{pdftex}{%
  \def\HOLOGO@ScaleBox@pdftex#1#2#3{%
    \HOLOGO@pdfliteral{%
      q #1 0 0 #2 0 0 cm%
    }%
    #3%
    \HOLOGO@pdfliteral{%
      Q%
    }%
  }%
}
%    \end{macrocode}
%    \end{macro}
%    \begin{macro}{\HOLOGO@ScaleBox@dvips}
%    \begin{macrocode}
\HOLOGO@temp{dvips}{%
  \def\HOLOGO@ScaleBox@dvips#1#2#3{%
    \special{ps:%
      gsave %
      currentpoint %
      currentpoint translate %
      #1 #2 scale %
      neg exch neg exch translate%
    }%
    #3%
    \special{ps:%
      currentpoint %
      grestore %
      moveto%
    }%
  }%
}
%    \end{macrocode}
%    \end{macro}
%    \begin{macro}{\HOLOGO@ScaleBox@dvipdfm}
%    \begin{macrocode}
\HOLOGO@temp{dvipdfm}{%
  \let\HOLOGO@ScaleBox@dvipdfm\HOLOGO@ScaleBox@dvips
}
%    \end{macrocode}
%    \end{macro}
%    Since \hologo{XeTeX} v0.6.
%    \begin{macro}{\HOLOGO@ScaleBox@xetex}
%    \begin{macrocode}
\HOLOGO@temp{xetex}{%
  \def\HOLOGO@ScaleBox@xetex#1#2#3{%
    \special{x:gsave}%
    \special{x:scale #1 #2}%
    #3%
    \special{x:grestore}%
  }%
}
%    \end{macrocode}
%    \end{macro}
%    \begin{macro}{\HOLOGO@ScaleBox@vtex}
%    \begin{macrocode}
\HOLOGO@temp{vtex}{%
  \def\HOLOGO@ScaleBox@vtex#1#2#3{%
    \special{r(#1,0,0,#2,0,0}%
    #3%
    \special{r)}%
  }%
}
%    \end{macrocode}
%    \end{macro}
%
%    \begin{macrocode}
\HOLOGO@AtEnd%
%</package>
%    \end{macrocode}
%
% \section{Test}
%
% \subsection{Catcode checks for loading}
%
%    \begin{macrocode}
%<*test1>
%    \end{macrocode}
%    \begin{macrocode}
\catcode`\{=1 %
\catcode`\}=2 %
\catcode`\#=6 %
\catcode`\@=11 %
\expandafter\ifx\csname count@\endcsname\relax
  \countdef\count@=255 %
\fi
\expandafter\ifx\csname @gobble\endcsname\relax
  \long\def\@gobble#1{}%
\fi
\expandafter\ifx\csname @firstofone\endcsname\relax
  \long\def\@firstofone#1{#1}%
\fi
\expandafter\ifx\csname loop\endcsname\relax
  \expandafter\@firstofone
\else
  \expandafter\@gobble
\fi
{%
  \def\loop#1\repeat{%
    \def\body{#1}%
    \iterate
  }%
  \def\iterate{%
    \body
      \let\next\iterate
    \else
      \let\next\relax
    \fi
    \next
  }%
  \let\repeat=\fi
}%
\def\RestoreCatcodes{}
\count@=0 %
\loop
  \edef\RestoreCatcodes{%
    \RestoreCatcodes
    \catcode\the\count@=\the\catcode\count@\relax
  }%
\ifnum\count@<255 %
  \advance\count@ 1 %
\repeat

\def\RangeCatcodeInvalid#1#2{%
  \count@=#1\relax
  \loop
    \catcode\count@=15 %
  \ifnum\count@<#2\relax
    \advance\count@ 1 %
  \repeat
}
\def\RangeCatcodeCheck#1#2#3{%
  \count@=#1\relax
  \loop
    \ifnum#3=\catcode\count@
    \else
      \errmessage{%
        Character \the\count@\space
        with wrong catcode \the\catcode\count@\space
        instead of \number#3%
      }%
    \fi
  \ifnum\count@<#2\relax
    \advance\count@ 1 %
  \repeat
}
\def\space{ }
\expandafter\ifx\csname LoadCommand\endcsname\relax
  \def\LoadCommand{\input hologo.sty\relax}%
\fi
\def\Test{%
  \RangeCatcodeInvalid{0}{47}%
  \RangeCatcodeInvalid{58}{64}%
  \RangeCatcodeInvalid{91}{96}%
  \RangeCatcodeInvalid{123}{255}%
  \catcode`\@=12 %
  \catcode`\\=0 %
  \catcode`\%=14 %
  \LoadCommand
  \RangeCatcodeCheck{0}{36}{15}%
  \RangeCatcodeCheck{37}{37}{14}%
  \RangeCatcodeCheck{38}{47}{15}%
  \RangeCatcodeCheck{48}{57}{12}%
  \RangeCatcodeCheck{58}{63}{15}%
  \RangeCatcodeCheck{64}{64}{12}%
  \RangeCatcodeCheck{65}{90}{11}%
  \RangeCatcodeCheck{91}{91}{15}%
  \RangeCatcodeCheck{92}{92}{0}%
  \RangeCatcodeCheck{93}{96}{15}%
  \RangeCatcodeCheck{97}{122}{11}%
  \RangeCatcodeCheck{123}{255}{15}%
  \RestoreCatcodes
}
\Test
\csname @@end\endcsname
\end
%    \end{macrocode}
%    \begin{macrocode}
%</test1>
%    \end{macrocode}
%
% \subsection{Spacefactor}
%
%    The space factor must be 1000 after a logo. If it is greater 1000
%    then the following space is a space after a sentence closing point.
%    If the space factor is smaller 1000 then an immediate following
%    dot is interpreted as abbreviation, not sentence closing point.
%
%    \begin{macrocode}
%<*test-spacefactor>
\NeedsTeXFormat{LaTeX2e}
\documentclass{article}
\usepackage{hologo}[2016/05/12]
\usepackage{kvsetkeys}
\usepackage{qstest}
\IncludeTests{*}
\LogTests{log}{*}{*}
\begin{document}
\begin{qstest}{spacefactor}{spacefactor}
\newcommand*{\Test}[1]{%
  \sbox0{%
    \hologo{#1}%
    \Expect*{1000 (#1)}*{\the\spacefactor\space(#1)}%
  }%
}%
\makeatletter
\def\TestList{}
\def\hologoEntry#1#2#3{%
  \edef\TestList{%
    \ifx\TestList\@empty
    \else
      \TestList,%
    \fi
    #1%
    \ifx\\#2\\%
    \else
      ={variant=#2}%
    \fi
  }%
}
\hologoList
\expandafter\kv@parse@normalized\expandafter{%
  \TestList
}{%
  \begingroup
    \let\@logo=\kv@key
    \ifx\kv@value\relax
    \else
      \expandafter\hologoLogoSetup\expandafter\@logo\expandafter{%
        \kv@value
      }%
    \fi
    \Test\@logo
  \endgroup
  \@gobbletwo
}
\end{qstest}
\end{document}
%</test-spacefactor>
%    \end{macrocode}
%
% \subsection{Complete list}
%
%    \begin{macrocode}
%<*test-list>
\NeedsTeXFormat{LaTeX2e}
\documentclass[12pt,a4paper]{article}
\usepackage{hologo}[2016/05/12]
\usepackage[T1]{fontenc}
\usepackage{lmodern}
\usepackage{parskip}
\usepackage[unicode]{hyperref}[2011/09/28]
\usepackage{bookmark}[2011/09/19]
\bookmarksetup{%
  numbered,%
  open,%
  openlevel=2,%
}
\renewcommand*{\contentsname}{List of logos}
\begin{document}
\tableofcontents
\def\TestFont#1#2#3#4#5#6{%
  \begingroup
    \usefont{#3}{#4}{#5}{#6}%
    \HologoVariant{#1}{#2}/\hologoVariant{#1}{#2}%
    \quad
    \begingroup\scriptsize\hologoVariant{#1}{#2}\endgroup
    \quad
  \endgroup
  (#3/#4/#5/#6)%
  \par
}
\makeatletter
\def\hologoEntry#1#2#3{%
  \section{%
    \HologoVariant{#1}{#2}/\hologoVariant{#1}{#2} %
    {[#1\ifx\\#2\\\else\space(#2)\fi]}% hash-ok
  }% braces around [] because of bug in tex4ht
  \begingroup
    \hypersetup{unicode=false}%
    \bookmark[%
      dest=\@currentHref,%
      rellevel=1,%
      keeplevel,%
    ]{%
      \HologoVariant{#1}{#2}/\hologoVariant{#1}{#2} %
      (PDFDocEncoding)%
    }%
  \endgroup
  \TestFont{#1}{#2}{OT1}{cmr}{m}{n}%
  \TestFont{#1}{#2}{OT1}{cmss}{m}{n}%
  \TestFont{#1}{#2}{OT1}{cmr}{b}{n}%
  \TestFont{#1}{#2}{OT1}{cmr}{m}{it}%
  \TestFont{#1}{#2}{OT1}{cmtt}{m}{n}%
  \TestFont{#1}{#2}{T1}{lmr}{m}{n}%
  \TestFont{#1}{#2}{T1}{lmss}{m}{n}%
  \TestFont{#1}{#2}{T1}{lmr}{b}{n}%
  \TestFont{#1}{#2}{T1}{lmr}{m}{it}%
  \TestFont{#1}{#2}{T1}{lmtt}{m}{n}%
  \TestFont{#1}{#2}{T1}{lmvtt}{m}{n}%
  \TestFont{#1}{#2}{T1}{qtm}{m}{n}%
  \TestFont{#1}{#2}{T1}{qhv}{m}{n}%
  \TestFont{#1}{#2}{T1}{qtm}{b}{n}%
  \TestFont{#1}{#2}{T1}{qtm}{m}{it}%
  \TestFont{#1}{#2}{T1}{qcr}{m}{n}%
  \newpage
}
\makeatother
\hologoList
\end{document}
%</test-list>
%    \end{macrocode}
%
% \section{Installation}
%
% \subsection{Download}
%
% \paragraph{Package.} This package is available on
% CTAN\footnote{\url{ftp://ftp.ctan.org/tex-archive/}}:
% \begin{description}
% \item[\CTAN{macros/latex/contrib/oberdiek/hologo.dtx}] The source file.
% \item[\CTAN{macros/latex/contrib/oberdiek/hologo.pdf}] Documentation.
% \end{description}
%
%
% \paragraph{Bundle.} All the packages of the bundle `oberdiek'
% are also available in a TDS compliant ZIP archive. There
% the packages are already unpacked and the documentation files
% are generated. The files and directories obey the TDS standard.
% \begin{description}
% \item[\CTAN{install/macros/latex/contrib/oberdiek.tds.zip}]
% \end{description}
% \emph{TDS} refers to the standard ``A Directory Structure
% for \TeX\ Files'' (\CTAN{tds/tds.pdf}). Directories
% with \xfile{texmf} in their name are usually organized this way.
%
% \subsection{Bundle installation}
%
% \paragraph{Unpacking.} Unpack the \xfile{oberdiek.tds.zip} in the
% TDS tree (also known as \xfile{texmf} tree) of your choice.
% Example (linux):
% \begin{quote}
%   |unzip oberdiek.tds.zip -d ~/texmf|
% \end{quote}
%
% \paragraph{Script installation.}
% Check the directory \xfile{TDS:scripts/oberdiek/} for
% scripts that need further installation steps.
% Package \xpackage{attachfile2} comes with the Perl script
% \xfile{pdfatfi.pl} that should be installed in such a way
% that it can be called as \texttt{pdfatfi}.
% Example (linux):
% \begin{quote}
%   |chmod +x scripts/oberdiek/pdfatfi.pl|\\
%   |cp scripts/oberdiek/pdfatfi.pl /usr/local/bin/|
% \end{quote}
%
% \subsection{Package installation}
%
% \paragraph{Unpacking.} The \xfile{.dtx} file is a self-extracting
% \docstrip\ archive. The files are extracted by running the
% \xfile{.dtx} through \plainTeX:
% \begin{quote}
%   \verb|tex hologo.dtx|
% \end{quote}
%
% \paragraph{TDS.} Now the different files must be moved into
% the different directories in your installation TDS tree
% (also known as \xfile{texmf} tree):
% \begin{quote}
% \def\t{^^A
% \begin{tabular}{@{}>{\ttfamily}l@{ $\rightarrow$ }>{\ttfamily}l@{}}
%   hologo.sty & tex/generic/oberdiek/hologo.sty\\
%   hologo.pdf & doc/latex/oberdiek/hologo.pdf\\
%   example/hologo-example.tex & doc/latex/oberdiek/example/hologo-example.tex\\
%   test/hologo-test1.tex & doc/latex/oberdiek/test/hologo-test1.tex\\
%   test/hologo-test-spacefactor.tex & doc/latex/oberdiek/test/hologo-test-spacefactor.tex\\
%   test/hologo-test-list.tex & doc/latex/oberdiek/test/hologo-test-list.tex\\
%   hologo.dtx & source/latex/oberdiek/hologo.dtx\\
% \end{tabular}^^A
% }^^A
% \sbox0{\t}^^A
% \ifdim\wd0>\linewidth
%   \begingroup
%     \advance\linewidth by\leftmargin
%     \advance\linewidth by\rightmargin
%   \edef\x{\endgroup
%     \def\noexpand\lw{\the\linewidth}^^A
%   }\x
%   \def\lwbox{^^A
%     \leavevmode
%     \hbox to \linewidth{^^A
%       \kern-\leftmargin\relax
%       \hss
%       \usebox0
%       \hss
%       \kern-\rightmargin\relax
%     }^^A
%   }^^A
%   \ifdim\wd0>\lw
%     \sbox0{\small\t}^^A
%     \ifdim\wd0>\linewidth
%       \ifdim\wd0>\lw
%         \sbox0{\footnotesize\t}^^A
%         \ifdim\wd0>\linewidth
%           \ifdim\wd0>\lw
%             \sbox0{\scriptsize\t}^^A
%             \ifdim\wd0>\linewidth
%               \ifdim\wd0>\lw
%                 \sbox0{\tiny\t}^^A
%                 \ifdim\wd0>\linewidth
%                   \lwbox
%                 \else
%                   \usebox0
%                 \fi
%               \else
%                 \lwbox
%               \fi
%             \else
%               \usebox0
%             \fi
%           \else
%             \lwbox
%           \fi
%         \else
%           \usebox0
%         \fi
%       \else
%         \lwbox
%       \fi
%     \else
%       \usebox0
%     \fi
%   \else
%     \lwbox
%   \fi
% \else
%   \usebox0
% \fi
% \end{quote}
% If you have a \xfile{docstrip.cfg} that configures and enables \docstrip's
% TDS installing feature, then some files can already be in the right
% place, see the documentation of \docstrip.
%
% \subsection{Refresh file name databases}
%
% If your \TeX~distribution
% (\teTeX, \mikTeX, \dots) relies on file name databases, you must refresh
% these. For example, \teTeX\ users run \verb|texhash| or
% \verb|mktexlsr|.
%
% \subsection{Some details for the interested}
%
% \paragraph{Attached source.}
%
% The PDF documentation on CTAN also includes the
% \xfile{.dtx} source file. It can be extracted by
% AcrobatReader 6 or higher. Another option is \textsf{pdftk},
% e.g. unpack the file into the current directory:
% \begin{quote}
%   \verb|pdftk hologo.pdf unpack_files output .|
% \end{quote}
%
% \paragraph{Unpacking with \LaTeX.}
% The \xfile{.dtx} chooses its action depending on the format:
% \begin{description}
% \item[\plainTeX:] Run \docstrip\ and extract the files.
% \item[\LaTeX:] Generate the documentation.
% \end{description}
% If you insist on using \LaTeX\ for \docstrip\ (really,
% \docstrip\ does not need \LaTeX), then inform the autodetect routine
% about your intention:
% \begin{quote}
%   \verb|latex \let\install=y% \iffalse meta-comment
%
% File: hologo.dtx
% Version: 2016/05/12 v1.11
% Info: A logo collection with bookmark support
%
% Copyright (C) 2010-2012 by
%    Heiko Oberdiek <heiko.oberdiek at googlemail.com>
%
% This work may be distributed and/or modified under the
% conditions of the LaTeX Project Public License, either
% version 1.3c of this license or (at your option) any later
% version. This version of this license is in
%    http://www.latex-project.org/lppl/lppl-1-3c.txt
% and the latest version of this license is in
%    http://www.latex-project.org/lppl.txt
% and version 1.3 or later is part of all distributions of
% LaTeX version 2005/12/01 or later.
%
% This work has the LPPL maintenance status "maintained".
%
% This Current Maintainer of this work is Heiko Oberdiek.
%
% The Base Interpreter refers to any `TeX-Format',
% because some files are installed in TDS:tex/generic//.
%
% This work consists of the main source file hologo.dtx
% and the derived files
%    hologo.sty, hologo.pdf, hologo.ins, hologo.drv, hologo-example.tex,
%    hologo-test1.tex, hologo-test-spacefactor.tex,
%    hologo-test-list.tex.
%
% Distribution:
%    CTAN:macros/latex/contrib/oberdiek/hologo.dtx
%    CTAN:macros/latex/contrib/oberdiek/hologo.pdf
%
% Unpacking:
%    (a) If hologo.ins is present:
%           tex hologo.ins
%    (b) Without hologo.ins:
%           tex hologo.dtx
%    (c) If you insist on using LaTeX
%           latex \let\install=y\input{hologo.dtx}
%        (quote the arguments according to the demands of your shell)
%
% Documentation:
%    (a) If hologo.drv is present:
%           latex hologo.drv
%    (b) Without hologo.drv:
%           latex hologo.dtx; ...
%    The class ltxdoc loads the configuration file ltxdoc.cfg
%    if available. Here you can specify further options, e.g.
%    use A4 as paper format:
%       \PassOptionsToClass{a4paper}{article}
%
%    Programm calls to get the documentation (example):
%       pdflatex hologo.dtx
%       makeindex -s gind.ist hologo.idx
%       pdflatex hologo.dtx
%       makeindex -s gind.ist hologo.idx
%       pdflatex hologo.dtx
%
% Installation:
%    TDS:tex/generic/oberdiek/hologo.sty
%    TDS:doc/latex/oberdiek/hologo.pdf
%    TDS:doc/latex/oberdiek/example/hologo-example.tex
%    TDS:doc/latex/oberdiek/test/hologo-test1.tex
%    TDS:doc/latex/oberdiek/test/hologo-test-spacefactor.tex
%    TDS:doc/latex/oberdiek/test/hologo-test-list.tex
%    TDS:source/latex/oberdiek/hologo.dtx
%
%<*ignore>
\begingroup
  \catcode123=1 %
  \catcode125=2 %
  \def\x{LaTeX2e}%
\expandafter\endgroup
\ifcase 0\ifx\install y1\fi\expandafter
         \ifx\csname processbatchFile\endcsname\relax\else1\fi
         \ifx\fmtname\x\else 1\fi\relax
\else\csname fi\endcsname
%</ignore>
%<*install>
\input docstrip.tex
\Msg{************************************************************************}
\Msg{* Installation}
\Msg{* Package: hologo 2016/05/12 v1.11 A logo collection with bookmark support (HO)}
\Msg{************************************************************************}

\keepsilent
\askforoverwritefalse

\let\MetaPrefix\relax
\preamble

This is a generated file.

Project: hologo
Version: 2016/05/12 v1.11

Copyright (C) 2010-2012 by
   Heiko Oberdiek <heiko.oberdiek at googlemail.com>

This work may be distributed and/or modified under the
conditions of the LaTeX Project Public License, either
version 1.3c of this license or (at your option) any later
version. This version of this license is in
   http://www.latex-project.org/lppl/lppl-1-3c.txt
and the latest version of this license is in
   http://www.latex-project.org/lppl.txt
and version 1.3 or later is part of all distributions of
LaTeX version 2005/12/01 or later.

This work has the LPPL maintenance status "maintained".

This Current Maintainer of this work is Heiko Oberdiek.

The Base Interpreter refers to any `TeX-Format',
because some files are installed in TDS:tex/generic//.

This work consists of the main source file hologo.dtx
and the derived files
   hologo.sty, hologo.pdf, hologo.ins, hologo.drv, hologo-example.tex,
   hologo-test1.tex, hologo-test-spacefactor.tex,
   hologo-test-list.tex.

\endpreamble
\let\MetaPrefix\DoubleperCent

\generate{%
  \file{hologo.ins}{\from{hologo.dtx}{install}}%
  \file{hologo.drv}{\from{hologo.dtx}{driver}}%
  \usedir{tex/generic/oberdiek}%
  \file{hologo.sty}{\from{hologo.dtx}{package}}%
  \usedir{doc/latex/oberdiek/example}%
  \file{hologo-example.tex}{\from{hologo.dtx}{example}}%
  \usedir{doc/latex/oberdiek/test}%
  \file{hologo-test1.tex}{\from{hologo.dtx}{test1}}%
  \file{hologo-test-spacefactor.tex}{\from{hologo.dtx}{test-spacefactor}}%
  \file{hologo-test-list.tex}{\from{hologo.dtx}{test-list}}%
  \nopreamble
  \nopostamble
  \usedir{source/latex/oberdiek/catalogue}%
  \file{hologo.xml}{\from{hologo.dtx}{catalogue}}%
}

\catcode32=13\relax% active space
\let =\space%
\Msg{************************************************************************}
\Msg{*}
\Msg{* To finish the installation you have to move the following}
\Msg{* file into a directory searched by TeX:}
\Msg{*}
\Msg{*     hologo.sty}
\Msg{*}
\Msg{* To produce the documentation run the file `hologo.drv'}
\Msg{* through LaTeX.}
\Msg{*}
\Msg{* Happy TeXing!}
\Msg{*}
\Msg{************************************************************************}

\endbatchfile
%</install>
%<*ignore>
\fi
%</ignore>
%<*driver>
\NeedsTeXFormat{LaTeX2e}
\ProvidesFile{hologo.drv}%
  [2016/05/12 v1.11 A logo collection with bookmark support (HO)]%
\documentclass{ltxdoc}
\usepackage{holtxdoc}[2011/11/22]
\usepackage{hologo}[2016/05/12]
\usepackage{longtable}
\usepackage{array}
\usepackage{paralist}
%\usepackage[T1]{fontenc}
%\usepackage{lmodern}
\begin{document}
  \DocInput{hologo.dtx}%
\end{document}
%</driver>
% \fi
%
%
% \CharacterTable
%  {Upper-case    \A\B\C\D\E\F\G\H\I\J\K\L\M\N\O\P\Q\R\S\T\U\V\W\X\Y\Z
%   Lower-case    \a\b\c\d\e\f\g\h\i\j\k\l\m\n\o\p\q\r\s\t\u\v\w\x\y\z
%   Digits        \0\1\2\3\4\5\6\7\8\9
%   Exclamation   \!     Double quote  \"     Hash (number) \#
%   Dollar        \$     Percent       \%     Ampersand     \&
%   Acute accent  \'     Left paren    \(     Right paren   \)
%   Asterisk      \*     Plus          \+     Comma         \,
%   Minus         \-     Point         \.     Solidus       \/
%   Colon         \:     Semicolon     \;     Less than     \<
%   Equals        \=     Greater than  \>     Question mark \?
%   Commercial at \@     Left bracket  \[     Backslash     \\
%   Right bracket \]     Circumflex    \^     Underscore    \_
%   Grave accent  \`     Left brace    \{     Vertical bar  \|
%   Right brace   \}     Tilde         \~}
%
% \GetFileInfo{hologo.drv}
%
% \title{The \xpackage{hologo} package}
% \date{2016/05/12 v1.11}
% \author{Heiko Oberdiek\\\xemail{heiko.oberdiek at googlemail.com}}
%
% \maketitle
%
% \begin{abstract}
% This package starts a collection of logos with support for bookmarks
% strings.
% \end{abstract}
%
% \tableofcontents
%
% \section{Documentation}
%
% \subsection{Logo macros}
%
% \begin{declcs}{hologo} \M{name}
% \end{declcs}
% Macro \cs{hologo} sets the logo with name \meta{name}.
% The following table shows the supported names.
%
% \begingroup
%   \def\hologoEntry#1#2#3{^^A
%     #1&#2&\hologoLogoSetup{#1}{variant=#2}\hologo{#1}&#3\tabularnewline
%   }
%   \begin{longtable}{>{\ttfamily}l>{\ttfamily}lll}
%     \rmfamily\bfseries{name} & \rmfamily\bfseries variant
%     & \bfseries logo & \bfseries since\\
%     \hline
%     \endhead
%     \hologoList
%   \end{longtable}
% \endgroup
%
% \begin{declcs}{Hologo} \M{name}
% \end{declcs}
% Macro \cs{Hologo} starts the logo \meta{name} with an uppercase
% letter. As an exception small greek letters are not converted
% to uppercase. Examples, see \hologo{eTeX} and \hologo{ExTeX}.
%
% \subsection{Setup macros}
%
% The package does not support package options, but the following
% setup macros can be used to set options.
%
% \begin{declcs}{hologoSetup} \M{key value list}
% \end{declcs}
% Macro \cs{hologoSetup} sets global options.
%
% \begin{declcs}{hologoLogoSetup} \M{logo} \M{key value list}
% \end{declcs}
% Some options can also be used to configure a logo.
% These settings take precedence over global option settings.
%
% \subsection{Options}\label{sec:options}
%
% There are boolean and string options:
% \begin{description}
% \item[Boolean option:]
% It takes |true| or |false|
% as value. If the value is omitted, then |true| is used.
% \item[String option:]
% A value must be given as string. (But the string might be empty.)
% \end{description}
% The following options can be used both in \cs{hologoSetup}
% and \cs{hologoLogoSetup}:
% \begin{description}
% \def\entry#1{\item[\xoption{#1}:]}
% \entry{break}
%   enables or disables line breaks inside the logo. This setting is
%   refined by options \xoption{hyphenbreak}, \xoption{spacebreak}
%   or \xoption{discretionarybreak}.
%   Default is |false|.
% \entry{hyphenbreak}
%   enables or disables the line break right after the hyphen character.
% \entry{spacebreak}
%   enables or disables line breaks at space characters.
% \entry{discretionarybreak}
%   enables or disables line breaks at hyphenation points
%   (inserted by \cs{-}).
% \end{description}
% Macro \cs{hologoLogoSetup} also knows:
% \begin{description}
% \item[\xoption{variant}:]
%   This is a string option. It specifies a variant of a logo that
%   must exist. An empty string selects the package default variant.
% \end{description}
% Example:
% \begin{quote}
%   |\hologoSetup{break=false}|\\
%   |\hologoLogoSetup{plainTeX}{variant=hyphen,hyphenbreak}|\\
%   Then ``plain-\TeX'' contains one break point after the hyphen.
% \end{quote}
%
% \subsection{Driver options}
%
% Sometimes graphical operations are needed to construct some
% glyphs (e.g.\ \hologo{XeTeX}). If package \xpackage{graphics}
% or package \xpackage{pgf} are found, then the macros are taken
% from there. Otherwise the packge defines its own operations
% and therefore needs the driver information. Many drivers are
% detected automatically (\hologo{pdfTeX}/\hologo{LuaTeX}
% in PDF mode, \hologo{XeTeX}, \hologo{VTeX}). These have precedence
% over a driver option. The driver can be given as package option
% or using \cs{hologoDriverSetup}.
% The following list contains the recognized driver options:
% \begin{itemize}
% \item \xoption{pdftex}, \xoption{luatex}
% \item \xoption{dvipdfm}, \xoption{dvipdfmx}
% \item \xoption{dvips}, \xoption{dvipsone}, \xoption{xdvi}
% \item \xoption{xetex}
% \item \xoption{vtex}
% \end{itemize}
% The left driver of a line is the driver name that is used internally.
% The following names are aliases for drivers that use the
% same method. Therefore the entry in the \xext{log} file for
% the used driver prints the internally used driver name.
% \begin{description}
% \item[\xoption{driverfallback}:]
%   This option expects a driver that is used,
%   if the driver could not be detected automatically.
% \end{description}
%
% \begin{declcs}{hologoDriverSetup} \M{driver option}
% \end{declcs}
% The driver can also be configured after package loading
% using \cs{hologoDriverSetup}, also the way for \hologo{plainTeX}
% to setup the driver.
%
% \subsection{Font setup}
%
% Some logos require a special font, but should also be usable by
% \hologo{plainTeX}. Therefore the package provides some ways
% to influence the font settings. The options below
% take font settings as values. Both font commands
% such as \cs{sffamily} and macros that take one argument
% like \cs{textsf} can be used.
%
% \begin{declcs}{hologoFontSetup} \M{key value list}
% \end{declcs}
% Macro \cs{hologoFontSetup} sets the fonts for all logos.
% Supported keys:
% \begin{description}
% \def\entry#1{\item[\xoption{#1}:]}
% \entry{general}
%   This font is used for all logos. The default is empty.
%   That means no special font is used.
% \entry{bibsf}
%   This font is used for
%   {\hologoLogoSetup{BibTeX}{variant=sf}\hologo{BibTeX}}
%   with variant \xoption{sf}.
% \entry{rm}
%   This font is a serif font. It is used for \hologo{ExTeX}.
% \entry{sc}
%   This font specifies a small caps font. It is used for
%   {\hologoLogoSetup{BibTeX}{variant=sc}\hologo{BibTeX}}
%   with variant \xoption{sc}.
% \entry{sf}
%   This font specifies a sans serif font. The default
%   is \cs{sffamily}, then \cs{sf} is tried. Otherwise
%   a warning is given. It is used by \hologo{KOMAScript}.
% \entry{sy}
%   This is the font for math symbols (e.g. cmsy).
%   It is used by \hologo{AmS}, \hologo{NTS}, \hologo{ExTeX}.
% \entry{logo}
%   \hologo{METAFONT} and \hologo{METAPOST} are using that font.
%   In \hologo{LaTeX} \cs{logofamily} is used and
%   the definitions of package \xpackage{mflogo} are used
%   if the package is not loaded.
%   Otherwise the \cs{tenlogo} is used and defined
%   if it does not already exists.
% \end{description}
%
% \begin{declcs}{hologoLogoFontSetup} \M{logo} \M{key value list}
% \end{declcs}
% Fonts can also be set for a logo or logo component separately,
% see the following list.
% The keys are the same as for \cs{hologoFontSetup}.
%
% \begin{longtable}{>{\ttfamily}l>{\sffamily}ll}
%   \meta{logo} & keys & result\\
%   \hline
%   \endhead
%   BibTeX & bibsf & {\hologoLogoSetup{BibTeX}{variant=sf}\hologo{BibTeX}}\\[.5ex]
%   BibTeX & sc & {\hologoLogoSetup{BibTeX}{variant=sc}\hologo{BibTeX}}\\[.5ex]
%   ExTeX & rm & \hologo{ExTeX}\\
%   SliTeX & rm & \hologo{SliTeX}\\[.5ex]
%   AmS & sy & \hologo{AmS}\\
%   ExTeX & sy & \hologo{ExTeX}\\
%   NTS & sy & \hologo{NTS}\\[.5ex]
%   KOMAScript & sf & \hologo{KOMAScript}\\[.5ex]
%   METAFONT & logo & \hologo{METAFONT}\\
%   METAPOST & logo & \hologo{METAPOST}\\[.5ex]
%   SliTeX & sc \hologo{SliTeX}
% \end{longtable}
%
% \subsubsection{Font order}
%
% For all logos the font \xoption{general} is applied first.
% Example:
%\begin{quote}
%|\hologoFontSetup{general=\color{red}}|
%\end{quote}
% will print red logos.
% Then if the font uses a special font \xoption{sf}, for example,
% the font is applied that is setup by \cs{hologoLogoFontSetup}.
% If this font is not setup, then the common font setup
% by \cs{hologoFontSetup} is used. Otherwise a warning is given,
% that there is no font configured.
%
% \subsection{Additional user macros}
%
% Usually a variant of a logo is configured by using
% \cs{hologoLogoSetup}, because it is bad style to mix
% different variants of the same logo in the same text.
% There the following macros are a convenience for testing.
%
% \begin{declcs}{hologoVariant} \M{name} \M{variant}\\
%   \cs{HologoVariant} \M{name} \M{variant}
% \end{declcs}
% Logo \meta{name} is set using \meta{variant} that specifies
% explicitely which variant of the macro is used. If the argument
% is empty, then the default form of the logo is used
% (configurable by \cs{hologoLogoSetup}).
%
% \cs{HologoVariant} is used if the logo is set in a context
% that needs an uppercase first letter (beginning of a sentence, \dots).
%
% \begin{declcs}{hologoList}\\
%   \cs{hologoEntry} \M{logo} \M{variant} \M{since}
% \end{declcs}
% Macro \cs{hologoList} contains all logos that are provided
% by the package including variants. The list consists of calls
% of \cs{hologoEntry} with three arguments starting with the
% logo name \meta{logo} and its variant \meta{variant}. An empty
% variant means the current default. Argument \meta{since} specifies
% with version of the package \xpackage{hologo} is needed to get
% the logo. If the logo is fixed, then the date gets updated.
% Therefore the date \meta{since} is not exactly the date of
% the first introduction, but rather the date of the latest fix.
%
% Before \cs{hologoList} can be used, macro \cs{hologoEntry} needs
% a definition. The example file in section \ref{sec:example}
% shows applications of \cs{hologoList}.
%
% \subsection{Supported contexts}
%
% Macros \cs{hologo} and friends support special contexts:
% \begin{itemize}
% \item \hologo{LaTeX}'s protection mechanism.
% \item Bookmarks of package \xpackage{hyperref}.
% \item Package \xpackage{tex4ht}.
% \item The macros can be used inside \cs{csname} constructs,
%   if \cs{ifincsname} is available (\hologo{pdfTeX}, \hologo{XeTeX},
%   \hologo{LuaTeX}).
% \end{itemize}
%
% \subsection{Example}
% \label{sec:example}
%
% The following example prints the logos in different fonts.
%    \begin{macrocode}
%<*example>
%<<verbatim
\NeedsTeXFormat{LaTeX2e}
\documentclass[a4paper]{article}
\usepackage[
  hmargin=20mm,
  vmargin=20mm,
]{geometry}
\pagestyle{empty}
\usepackage{hologo}[2016/05/12]
\usepackage{longtable}
\usepackage{array}
\setlength{\extrarowheight}{2pt}
\usepackage[T1]{fontenc}
\usepackage{lmodern}
\usepackage{pdflscape}
\usepackage[
  pdfencoding=auto,
]{hyperref}
\hypersetup{
  pdfauthor={Heiko Oberdiek},
  pdftitle={Example for package `hologo'},
  pdfsubject={Logos with fonts lmr, lmss, qtm, qpl, qhv},
}
\usepackage{bookmark}

% Print the logo list on the console

\begingroup
  \typeout{}%
  \typeout{*** Begin of logo list ***}%
  \newcommand*{\hologoEntry}[3]{%
    \typeout{#1 \ifx\\#2\\\else(#2) \fi[#3]}%
  }%
  \hologoList
  \typeout{*** End of logo list ***}%
  \typeout{}%
\endgroup

\begin{document}
\begin{landscape}

  \section{Example file for package `hologo'}

  % Table for font names

  \begin{longtable}{>{\bfseries}ll}
    \textbf{font} & \textbf{Font name}\\
    \hline
    lmr & Latin Modern Roman\\
    lmss & Latin Modern Sans\\
    qtm & \TeX\ Gyre Termes\\
    qhv & \TeX\ Gyre Heros\\
    qpl & \TeX\ Gyre Pagella\\
  \end{longtable}

  % Logo list with logos in different fonts

  \begingroup
    \newcommand*{\SetVariant}[2]{%
      \ifx\\#2\\%
      \else
        \hologoLogoSetup{#1}{variant=#2}%
      \fi
    }%
    \newcommand*{\hologoEntry}[3]{%
      \SetVariant{#1}{#2}%
      \raisebox{1em}[0pt][0pt]{\hypertarget{#1@#2}{}}%
      \bookmark[%
        dest={#1@#2},%
      ]{%
        #1\ifx\\#2\\\else\space(#2)\fi: \Hologo{#1}, \hologo{#1} %
        [Unicode]%
      }%
      \hypersetup{unicode=false}%
      \bookmark[%
        dest={#1@#2},%
      ]{%
        #1\ifx\\#2\\\else\space(#2)\fi: \Hologo{#1}, \hologo{#1} %
        [PDFDocEncoding]%
      }%
      \texttt{#1}%
      &%
      \texttt{#2}%
      &%
      \Hologo{#1}%
      &%
      \SetVariant{#1}{#2}%
      \hologo{#1}%
      &%
      \SetVariant{#1}{#2}%
      \fontfamily{qtm}\selectfont
      \hologo{#1}%
      &%
      \SetVariant{#1}{#2}%
      \fontfamily{qpl}\selectfont
      \hologo{#1}%
      &%
      \SetVariant{#1}{#2}%
      \textsf{\hologo{#1}}%
      &%
      \SetVariant{#1}{#2}%
      \fontfamily{qhv}\selectfont
      \hologo{#1}%
      \tabularnewline
    }%
    \begin{longtable}{llllllll}%
      \textbf{\textit{logo}} & \textbf{\textit{variant}} &
      \texttt{\string\Hologo} &
      \textbf{lmr} & \textbf{qtm} & \textbf{qpl} &
      \textbf{lmss} & \textbf{qhv}
      \tabularnewline
      \hline
      \endhead
      \hologoList
    \end{longtable}%
  \endgroup

\end{landscape}
\end{document}
%verbatim
%</example>
%    \end{macrocode}
%
% \StopEventually{
% }
%
% \section{Implementation}
%    \begin{macrocode}
%<*package>
%    \end{macrocode}
%    Reload check, especially if the package is not used with \LaTeX.
%    \begin{macrocode}
\begingroup\catcode61\catcode48\catcode32=10\relax%
  \catcode13=5 % ^^M
  \endlinechar=13 %
  \catcode35=6 % #
  \catcode39=12 % '
  \catcode44=12 % ,
  \catcode45=12 % -
  \catcode46=12 % .
  \catcode58=12 % :
  \catcode64=11 % @
  \catcode123=1 % {
  \catcode125=2 % }
  \expandafter\let\expandafter\x\csname ver@hologo.sty\endcsname
  \ifx\x\relax % plain-TeX, first loading
  \else
    \def\empty{}%
    \ifx\x\empty % LaTeX, first loading,
      % variable is initialized, but \ProvidesPackage not yet seen
    \else
      \expandafter\ifx\csname PackageInfo\endcsname\relax
        \def\x#1#2{%
          \immediate\write-1{Package #1 Info: #2.}%
        }%
      \else
        \def\x#1#2{\PackageInfo{#1}{#2, stopped}}%
      \fi
      \x{hologo}{The package is already loaded}%
      \aftergroup\endinput
    \fi
  \fi
\endgroup%
%    \end{macrocode}
%    Package identification:
%    \begin{macrocode}
\begingroup\catcode61\catcode48\catcode32=10\relax%
  \catcode13=5 % ^^M
  \endlinechar=13 %
  \catcode35=6 % #
  \catcode39=12 % '
  \catcode40=12 % (
  \catcode41=12 % )
  \catcode44=12 % ,
  \catcode45=12 % -
  \catcode46=12 % .
  \catcode47=12 % /
  \catcode58=12 % :
  \catcode64=11 % @
  \catcode91=12 % [
  \catcode93=12 % ]
  \catcode123=1 % {
  \catcode125=2 % }
  \expandafter\ifx\csname ProvidesPackage\endcsname\relax
    \def\x#1#2#3[#4]{\endgroup
      \immediate\write-1{Package: #3 #4}%
      \xdef#1{#4}%
    }%
  \else
    \def\x#1#2[#3]{\endgroup
      #2[{#3}]%
      \ifx#1\@undefined
        \xdef#1{#3}%
      \fi
      \ifx#1\relax
        \xdef#1{#3}%
      \fi
    }%
  \fi
\expandafter\x\csname ver@hologo.sty\endcsname
\ProvidesPackage{hologo}%
  [2016/05/12 v1.11 A logo collection with bookmark support (HO)]%
%    \end{macrocode}
%
%    \begin{macrocode}
\begingroup\catcode61\catcode48\catcode32=10\relax%
  \catcode13=5 % ^^M
  \endlinechar=13 %
  \catcode123=1 % {
  \catcode125=2 % }
  \catcode64=11 % @
  \def\x{\endgroup
    \expandafter\edef\csname HOLOGO@AtEnd\endcsname{%
      \endlinechar=\the\endlinechar\relax
      \catcode13=\the\catcode13\relax
      \catcode32=\the\catcode32\relax
      \catcode35=\the\catcode35\relax
      \catcode61=\the\catcode61\relax
      \catcode64=\the\catcode64\relax
      \catcode123=\the\catcode123\relax
      \catcode125=\the\catcode125\relax
    }%
  }%
\x\catcode61\catcode48\catcode32=10\relax%
\catcode13=5 % ^^M
\endlinechar=13 %
\catcode35=6 % #
\catcode64=11 % @
\catcode123=1 % {
\catcode125=2 % }
\def\TMP@EnsureCode#1#2{%
  \edef\HOLOGO@AtEnd{%
    \HOLOGO@AtEnd
    \catcode#1=\the\catcode#1\relax
  }%
  \catcode#1=#2\relax
}
\TMP@EnsureCode{10}{12}% ^^J
\TMP@EnsureCode{33}{12}% !
\TMP@EnsureCode{34}{12}% "
\TMP@EnsureCode{36}{3}% $
\TMP@EnsureCode{38}{4}% &
\TMP@EnsureCode{39}{12}% '
\TMP@EnsureCode{40}{12}% (
\TMP@EnsureCode{41}{12}% )
\TMP@EnsureCode{42}{12}% *
\TMP@EnsureCode{43}{12}% +
\TMP@EnsureCode{44}{12}% ,
\TMP@EnsureCode{45}{12}% -
\TMP@EnsureCode{46}{12}% .
\TMP@EnsureCode{47}{12}% /
\TMP@EnsureCode{58}{12}% :
\TMP@EnsureCode{59}{12}% ;
\TMP@EnsureCode{60}{12}% <
\TMP@EnsureCode{62}{12}% >
\TMP@EnsureCode{63}{12}% ?
\TMP@EnsureCode{91}{12}% [
\TMP@EnsureCode{93}{12}% ]
\TMP@EnsureCode{94}{7}% ^ (superscript)
\TMP@EnsureCode{95}{8}% _ (subscript)
\TMP@EnsureCode{96}{12}% `
\TMP@EnsureCode{124}{12}% |
\edef\HOLOGO@AtEnd{%
  \HOLOGO@AtEnd
  \escapechar\the\escapechar\relax
  \noexpand\endinput
}
\escapechar=92 %
%    \end{macrocode}
%
% \subsection{Logo list}
%
%    \begin{macro}{\hologoList}
%    \begin{macrocode}
\def\hologoList{%
  \hologoEntry{(La)TeX}{}{2011/10/01}%
  \hologoEntry{AmSLaTeX}{}{2010/04/16}%
  \hologoEntry{AmSTeX}{}{2010/04/16}%
  \hologoEntry{biber}{}{2011/10/01}%
  \hologoEntry{BibTeX}{}{2011/10/01}%
  \hologoEntry{BibTeX}{sf}{2011/10/01}%
  \hologoEntry{BibTeX}{sc}{2011/10/01}%
  \hologoEntry{BibTeX8}{}{2011/11/22}%
  \hologoEntry{ConTeXt}{}{2011/03/25}%
  \hologoEntry{ConTeXt}{narrow}{2011/03/25}%
  \hologoEntry{ConTeXt}{simple}{2011/03/25}%
  \hologoEntry{emTeX}{}{2010/04/26}%
  \hologoEntry{eTeX}{}{2010/04/08}%
  \hologoEntry{ExTeX}{}{2011/10/01}%
  \hologoEntry{HanTheThanh}{}{2011/11/29}%
  \hologoEntry{iniTeX}{}{2011/10/01}%
  \hologoEntry{KOMAScript}{}{2011/10/01}%
  \hologoEntry{La}{}{2010/05/08}%
  \hologoEntry{LaTeX}{}{2010/04/08}%
  \hologoEntry{LaTeX2e}{}{2010/04/08}%
  \hologoEntry{LaTeX3}{}{2010/04/24}%
  \hologoEntry{LaTeXe}{}{2010/04/08}%
  \hologoEntry{LaTeXML}{}{2011/11/22}%
  \hologoEntry{LaTeXTeX}{}{2011/10/01}%
  \hologoEntry{LuaLaTeX}{}{2010/04/08}%
  \hologoEntry{LuaTeX}{}{2010/04/08}%
  \hologoEntry{LyX}{}{2011/10/01}%
  \hologoEntry{METAFONT}{}{2011/10/01}%
  \hologoEntry{MetaFun}{}{2011/10/01}%
  \hologoEntry{METAPOST}{}{2011/10/01}%
  \hologoEntry{MetaPost}{}{2011/10/01}%
  \hologoEntry{MiKTeX}{}{2011/10/01}%
  \hologoEntry{NTS}{}{2011/10/01}%
  \hologoEntry{OzMF}{}{2011/10/01}%
  \hologoEntry{OzMP}{}{2011/10/01}%
  \hologoEntry{OzTeX}{}{2011/10/01}%
  \hologoEntry{OzTtH}{}{2011/10/01}%
  \hologoEntry{PCTeX}{}{2011/10/01}%
  \hologoEntry{pdfTeX}{}{2011/10/01}%
  \hologoEntry{pdfLaTeX}{}{2011/10/01}%
  \hologoEntry{PiC}{}{2011/10/01}%
  \hologoEntry{PiCTeX}{}{2011/10/01}%
  \hologoEntry{plainTeX}{}{2010/04/08}%
  \hologoEntry{plainTeX}{space}{2010/04/16}%
  \hologoEntry{plainTeX}{hyphen}{2010/04/16}%
  \hologoEntry{plainTeX}{runtogether}{2010/04/16}%
  \hologoEntry{SageTeX}{}{2011/11/22}%
  \hologoEntry{SLiTeX}{}{2011/10/01}%
  \hologoEntry{SLiTeX}{lift}{2011/10/01}%
  \hologoEntry{SLiTeX}{narrow}{2011/10/01}%
  \hologoEntry{SLiTeX}{simple}{2011/10/01}%
  \hologoEntry{SliTeX}{}{2011/10/01}%
  \hologoEntry{SliTeX}{narrow}{2011/10/01}%
  \hologoEntry{SliTeX}{simple}{2011/10/01}%
  \hologoEntry{SliTeX}{lift}{2011/10/01}%
  \hologoEntry{teTeX}{}{2011/10/01}%
  \hologoEntry{TeX}{}{2010/04/08}%
  \hologoEntry{TeX4ht}{}{2011/11/22}%
  \hologoEntry{TTH}{}{2011/11/22}%
  \hologoEntry{virTeX}{}{2011/10/01}%
  \hologoEntry{VTeX}{}{2010/04/24}%
  \hologoEntry{Xe}{}{2010/04/08}%
  \hologoEntry{XeLaTeX}{}{2010/04/08}%
  \hologoEntry{XeTeX}{}{2010/04/08}%
}
%    \end{macrocode}
%    \end{macro}
%
% \subsection{Load resources}
%
%    \begin{macrocode}
\begingroup\expandafter\expandafter\expandafter\endgroup
\expandafter\ifx\csname RequirePackage\endcsname\relax
  \def\TMP@RequirePackage#1[#2]{%
    \begingroup\expandafter\expandafter\expandafter\endgroup
    \expandafter\ifx\csname ver@#1.sty\endcsname\relax
      \input #1.sty\relax
    \fi
  }%
  \TMP@RequirePackage{ltxcmds}[2011/02/04]%
  \TMP@RequirePackage{infwarerr}[2010/04/08]%
  \TMP@RequirePackage{kvsetkeys}[2010/03/01]%
  \TMP@RequirePackage{kvdefinekeys}[2010/03/01]%
  \TMP@RequirePackage{pdftexcmds}[2010/04/01]%
  \TMP@RequirePackage{ifpdf}[2010/01/28]%
  \TMP@RequirePackage{ifluatex}[2010/03/01]%
  \ltx@IfUndefined{newif}{%
    \expandafter\let\csname newif\endcsname\ltx@newif
  }{}%
  \TMP@RequirePackage{ifxetex}[2009/01/23]%
  \TMP@RequirePackage{ifvtex}[2010/03/01]%
\else
  \RequirePackage{ltxcmds}[2011/02/04]%
  \RequirePackage{infwarerr}[2010/04/08]%
  \RequirePackage{kvsetkeys}[2010/03/01]%
  \RequirePackage{kvdefinekeys}[2010/03/01]%
  \RequirePackage{pdftexcmds}[2010/04/01]%
  \RequirePackage{ifpdf}[2010/01/28]%
  \RequirePackage{ifluatex}[2010/03/01]%
  \RequirePackage{ifxetex}[2009/01/23]%
  \RequirePackage{ifvtex}[2010/03/01]%
\fi
%    \end{macrocode}
%
%    \begin{macro}{\HOLOGO@IfDefined}
%    \begin{macrocode}
\def\HOLOGO@IfExists#1{%
  \ifx\@undefined#1%
    \expandafter\ltx@secondoftwo
  \else
    \ifx\relax#1%
      \expandafter\ltx@secondoftwo
    \else
      \expandafter\expandafter\expandafter\ltx@firstoftwo
    \fi
  \fi
}
%    \end{macrocode}
%    \end{macro}
%
% \subsection{Setup macros}
%
%    \begin{macro}{\hologoSetup}
%    \begin{macrocode}
\def\hologoSetup{%
  \let\HOLOGO@name\relax
  \HOLOGO@Setup
}
%    \end{macrocode}
%    \end{macro}
%
%    \begin{macro}{\hologoLogoSetup}
%    \begin{macrocode}
\def\hologoLogoSetup#1{%
  \edef\HOLOGO@name{#1}%
  \ltx@IfUndefined{HoLogo@\HOLOGO@name}{%
    \@PackageError{hologo}{%
      Unknown logo `\HOLOGO@name'%
    }\@ehc
    \ltx@gobble
  }{%
    \HOLOGO@Setup
  }%
}
%    \end{macrocode}
%    \end{macro}
%
%    \begin{macro}{\HOLOGO@Setup}
%    \begin{macrocode}
\def\HOLOGO@Setup{%
  \kvsetkeys{HoLogo}%
}
%    \end{macrocode}
%    \end{macro}
%
% \subsection{Options}
%
%    \begin{macro}{\HOLOGO@DeclareBoolOption}
%    \begin{macrocode}
\def\HOLOGO@DeclareBoolOption#1{%
  \expandafter\chardef\csname HOLOGOOPT@#1\endcsname\ltx@zero
  \kv@define@key{HoLogo}{#1}[true]{%
    \def\HOLOGO@temp{##1}%
    \ifx\HOLOGO@temp\HOLOGO@true
      \ifx\HOLOGO@name\relax
        \expandafter\chardef\csname HOLOGOOPT@#1\endcsname=\ltx@one
      \else
        \expandafter\chardef\csname
        HoLogoOpt@#1@\HOLOGO@name\endcsname\ltx@one
      \fi
      \HOLOGO@SetBreakAll{#1}%
    \else
      \ifx\HOLOGO@temp\HOLOGO@false
        \ifx\HOLOGO@name\relax
          \expandafter\chardef\csname HOLOGOOPT@#1\endcsname=\ltx@zero
        \else
          \expandafter\chardef\csname
          HoLogoOpt@#1@\HOLOGO@name\endcsname=\ltx@zero
        \fi
        \HOLOGO@SetBreakAll{#1}%
      \else
        \@PackageError{hologo}{%
          Unknown value `##1' for boolean option `#1'.\MessageBreak
          Known values are `true' and `false'%
        }\@ehc
      \fi
    \fi
  }%
}
%    \end{macrocode}
%    \end{macro}
%
%    \begin{macro}{\HOLOGO@SetBreakAll}
%    \begin{macrocode}
\def\HOLOGO@SetBreakAll#1{%
  \def\HOLOGO@temp{#1}%
  \ifx\HOLOGO@temp\HOLOGO@break
    \ifx\HOLOGO@name\relax
      \chardef\HOLOGOOPT@hyphenbreak=\HOLOGOOPT@break
      \chardef\HOLOGOOPT@spacebreak=\HOLOGOOPT@break
      \chardef\HOLOGOOPT@discretionarybreak=\HOLOGOOPT@break
    \else
      \expandafter\chardef
         \csname HoLogoOpt@hyphenbreak@\HOLOGO@name\endcsname=%
         \csname HoLogoOpt@break@\HOLOGO@name\endcsname
      \expandafter\chardef
         \csname HoLogoOpt@spacebreak@\HOLOGO@name\endcsname=%
         \csname HoLogoOpt@break@\HOLOGO@name\endcsname
      \expandafter\chardef
         \csname HoLogoOpt@discretionarybreak@\HOLOGO@name
             \endcsname=%
         \csname HoLogoOpt@break@\HOLOGO@name\endcsname
    \fi
  \fi
}
%    \end{macrocode}
%    \end{macro}
%
%    \begin{macro}{\HOLOGO@true}
%    \begin{macrocode}
\def\HOLOGO@true{true}
%    \end{macrocode}
%    \end{macro}
%    \begin{macro}{\HOLOGO@false}
%    \begin{macrocode}
\def\HOLOGO@false{false}
%    \end{macrocode}
%    \end{macro}
%    \begin{macro}{\HOLOGO@break}
%    \begin{macrocode}
\def\HOLOGO@break{break}
%    \end{macrocode}
%    \end{macro}
%
%    \begin{macrocode}
\HOLOGO@DeclareBoolOption{break}
\HOLOGO@DeclareBoolOption{hyphenbreak}
\HOLOGO@DeclareBoolOption{spacebreak}
\HOLOGO@DeclareBoolOption{discretionarybreak}
%    \end{macrocode}
%
%    \begin{macrocode}
\kv@define@key{HoLogo}{variant}{%
  \ifx\HOLOGO@name\relax
    \@PackageError{hologo}{%
      Option `variant' is not available in \string\hologoSetup,%
      \MessageBreak
      Use \string\hologoLogoSetup\space instead%
    }\@ehc
  \else
    \edef\HOLOGO@temp{#1}%
    \ifx\HOLOGO@temp\ltx@empty
      \expandafter
      \let\csname HoLogoOpt@variant@\HOLOGO@name\endcsname\@undefined
    \else
      \ltx@IfUndefined{HoLogo@\HOLOGO@name @\HOLOGO@temp}{%
        \@PackageError{hologo}{%
          Unknown variant `\HOLOGO@temp' of logo `\HOLOGO@name'%
        }\@ehc
      }{%
        \expandafter
        \let\csname HoLogoOpt@variant@\HOLOGO@name\endcsname
            \HOLOGO@temp
      }%
    \fi
  \fi
}
%    \end{macrocode}
%
%    \begin{macro}{\HOLOGO@Variant}
%    \begin{macrocode}
\def\HOLOGO@Variant#1{%
  #1%
  \ltx@ifundefined{HoLogoOpt@variant@#1}{%
  }{%
    @\csname HoLogoOpt@variant@#1\endcsname
  }%
}
%    \end{macrocode}
%    \end{macro}
%
% \subsection{Break/no-break support}
%
%    \begin{macro}{\HOLOGO@space}
%    \begin{macrocode}
\def\HOLOGO@space{%
  \ltx@ifundefined{HoLogoOpt@spacebreak@\HOLOGO@name}{%
    \ltx@ifundefined{HoLogoOpt@break@\HOLOGO@name}{%
      \chardef\HOLOGO@temp=\HOLOGOOPT@spacebreak
    }{%
      \chardef\HOLOGO@temp=%
        \csname HoLogoOpt@break@\HOLOGO@name\endcsname
    }%
  }{%
    \chardef\HOLOGO@temp=%
      \csname HoLogoOpt@spacebreak@\HOLOGO@name\endcsname
  }%
  \ifcase\HOLOGO@temp
    \penalty10000 %
  \fi
  \ltx@space
}
%    \end{macrocode}
%    \end{macro}
%
%    \begin{macro}{\HOLOGO@hyphen}
%    \begin{macrocode}
\def\HOLOGO@hyphen{%
  \ltx@ifundefined{HoLogoOpt@hyphenbreak@\HOLOGO@name}{%
    \ltx@ifundefined{HoLogoOpt@break@\HOLOGO@name}{%
      \chardef\HOLOGO@temp=\HOLOGOOPT@hyphenbreak
    }{%
      \chardef\HOLOGO@temp=%
        \csname HoLogoOpt@break@\HOLOGO@name\endcsname
    }%
  }{%
    \chardef\HOLOGO@temp=%
      \csname HoLogoOpt@hyphenbreak@\HOLOGO@name\endcsname
  }%
  \ifcase\HOLOGO@temp
    \ltx@mbox{-}%
  \else
    -%
  \fi
}
%    \end{macrocode}
%    \end{macro}
%
%    \begin{macro}{\HOLOGO@discretionary}
%    \begin{macrocode}
\def\HOLOGO@discretionary{%
  \ltx@ifundefined{HoLogoOpt@discretionarybreak@\HOLOGO@name}{%
    \ltx@ifundefined{HoLogoOpt@break@\HOLOGO@name}{%
      \chardef\HOLOGO@temp=\HOLOGOOPT@discretionarybreak
    }{%
      \chardef\HOLOGO@temp=%
        \csname HoLogoOpt@break@\HOLOGO@name\endcsname
    }%
  }{%
    \chardef\HOLOGO@temp=%
      \csname HoLogoOpt@discretionarybreak@\HOLOGO@name\endcsname
  }%
  \ifcase\HOLOGO@temp
  \else
    \-%
  \fi
}
%    \end{macrocode}
%    \end{macro}
%
%    \begin{macro}{\HOLOGO@mbox}
%    \begin{macrocode}
\def\HOLOGO@mbox#1{%
  \ltx@ifundefined{HoLogoOpt@break@\HOLOGO@name}{%
    \chardef\HOLOGO@temp=\HOLOGOOPT@hyphenbreak
  }{%
    \chardef\HOLOGO@temp=%
      \csname HoLogoOpt@break@\HOLOGO@name\endcsname
  }%
  \ifcase\HOLOGO@temp
    \ltx@mbox{#1}%
  \else
    #1%
  \fi
}
%    \end{macrocode}
%    \end{macro}
%
% \subsection{Font support}
%
%    \begin{macro}{\HoLogoFont@font}
%    \begin{tabular}{@{}ll@{}}
%    |#1|:& logo name\\
%    |#2|:& font short name\\
%    |#3|:& text
%    \end{tabular}
%    \begin{macrocode}
\def\HoLogoFont@font#1#2#3{%
  \begingroup
    \ltx@IfUndefined{HoLogoFont@logo@#1.#2}{%
      \ltx@IfUndefined{HoLogoFont@font@#2}{%
        \@PackageWarning{hologo}{%
          Missing font `#2' for logo `#1'%
        }%
        #3%
      }{%
        \csname HoLogoFont@font@#2\endcsname{#3}%
      }%
    }{%
      \csname HoLogoFont@logo@#1.#2\endcsname{#3}%
    }%
  \endgroup
}
%    \end{macrocode}
%    \end{macro}
%
%    \begin{macro}{\HoLogoFont@Def}
%    \begin{macrocode}
\def\HoLogoFont@Def#1{%
  \expandafter\def\csname HoLogoFont@font@#1\endcsname
}
%    \end{macrocode}
%    \end{macro}
%    \begin{macro}{\HoLogoFont@LogoDef}
%    \begin{macrocode}
\def\HoLogoFont@LogoDef#1#2{%
  \expandafter\def\csname HoLogoFont@logo@#1.#2\endcsname
}
%    \end{macrocode}
%    \end{macro}
%
% \subsubsection{Font defaults}
%
%    \begin{macro}{\HoLogoFont@font@general}
%    \begin{macrocode}
\HoLogoFont@Def{general}{}%
%    \end{macrocode}
%    \end{macro}
%
%    \begin{macro}{\HoLogoFont@font@rm}
%    \begin{macrocode}
\ltx@IfUndefined{rmfamily}{%
  \ltx@IfUndefined{rm}{%
  }{%
    \HoLogoFont@Def{rm}{\rm}%
  }%
}{%
  \HoLogoFont@Def{rm}{\rmfamily}%
}
%    \end{macrocode}
%    \end{macro}
%
%    \begin{macro}{\HoLogoFont@font@sf}
%    \begin{macrocode}
\ltx@IfUndefined{sffamily}{%
  \ltx@IfUndefined{sf}{%
  }{%
    \HoLogoFont@Def{sf}{\sf}%
  }%
}{%
  \HoLogoFont@Def{sf}{\sffamily}%
}
%    \end{macrocode}
%    \end{macro}
%
%    \begin{macro}{\HoLogoFont@font@bibsf}
%    In case of \hologo{plainTeX} the original small caps
%    variant is used as default. In \hologo{LaTeX}
%    the definition of package \xpackage{dtklogos} \cite{dtklogos}
%    is used.
%\begin{quote}
%\begin{verbatim}
%\DeclareRobustCommand{\BibTeX}{%
%  B%
%  \kern-.05em%
%  \hbox{%
%    $\m@th$% %% force math size calculations
%    \csname S@\f@size\endcsname
%    \fontsize\sf@size\z@
%    \math@fontsfalse
%    \selectfont
%    I%
%    \kern-.025em%
%    B
%  }%
%  \kern-.08em%
%  \-%
%  \TeX
%}
%\end{verbatim}
%\end{quote}
%    \begin{macrocode}
\ltx@IfUndefined{selectfont}{%
  \ltx@IfUndefined{tensc}{%
    \font\tensc=cmcsc10\relax
  }{}%
  \HoLogoFont@Def{bibsf}{\tensc}%
}{%
  \HoLogoFont@Def{bibsf}{%
    $\mathsurround=0pt$%
    \csname S@\f@size\endcsname
    \fontsize\sf@size{0pt}%
    \math@fontsfalse
    \selectfont
  }%
}
%    \end{macrocode}
%    \end{macro}
%
%    \begin{macro}{\HoLogoFont@font@sc}
%    \begin{macrocode}
\ltx@IfUndefined{scshape}{%
  \ltx@IfUndefined{tensc}{%
    \font\tensc=cmcsc10\relax
  }{}%
  \HoLogoFont@Def{sc}{\tensc}%
}{%
  \HoLogoFont@Def{sc}{\scshape}%
}
%    \end{macrocode}
%    \end{macro}
%
%    \begin{macro}{\HoLogoFont@font@sy}
%    \begin{macrocode}
\ltx@IfUndefined{usefont}{%
  \ltx@IfUndefined{tensy}{%
  }{%
    \HoLogoFont@Def{sy}{\tensy}%
  }%
}{%
  \HoLogoFont@Def{sy}{%
    \usefont{OMS}{cmsy}{m}{n}%
  }%
}
%    \end{macrocode}
%    \end{macro}
%
%    \begin{macro}{\HoLogoFont@font@logo}
%    \begin{macrocode}
\begingroup
  \def\x{LaTeX2e}%
\expandafter\endgroup
\ifx\fmtname\x
  \ltx@IfUndefined{logofamily}{%
    \DeclareRobustCommand\logofamily{%
      \not@math@alphabet\logofamily\relax
      \fontencoding{U}%
      \fontfamily{logo}%
      \selectfont
    }%
  }{}%
  \ltx@IfUndefined{logofamily}{%
  }{%
    \HoLogoFont@Def{logo}{\logofamily}%
  }%
\else
  \ltx@IfUndefined{tenlogo}{%
    \font\tenlogo=logo10\relax
  }{}%
  \HoLogoFont@Def{logo}{\tenlogo}%
\fi
%    \end{macrocode}
%    \end{macro}
%
% \subsubsection{Font setup}
%
%    \begin{macro}{\hologoFontSetup}
%    \begin{macrocode}
\def\hologoFontSetup{%
  \let\HOLOGO@name\relax
  \HOLOGO@FontSetup
}
%    \end{macrocode}
%    \end{macro}
%
%    \begin{macro}{\hologoLogoFontSetup}
%    \begin{macrocode}
\def\hologoLogoFontSetup#1{%
  \edef\HOLOGO@name{#1}%
  \ltx@IfUndefined{HoLogo@\HOLOGO@name}{%
    \@PackageError{hologo}{%
      Unknown logo `\HOLOGO@name'%
    }\@ehc
    \ltx@gobble
  }{%
    \HOLOGO@FontSetup
  }%
}
%    \end{macrocode}
%    \end{macro}
%
%    \begin{macro}{\HOLOGO@FontSetup}
%    \begin{macrocode}
\def\HOLOGO@FontSetup{%
  \kvsetkeys{HoLogoFont}%
}
%    \end{macrocode}
%    \end{macro}
%
%    \begin{macrocode}
\def\HOLOGO@temp#1{%
  \kv@define@key{HoLogoFont}{#1}{%
    \ifx\HOLOGO@name\relax
      \HoLogoFont@Def{#1}{##1}%
    \else
      \HoLogoFont@LogoDef\HOLOGO@name{#1}{##1}%
    \fi
  }%
}
\HOLOGO@temp{general}
\HOLOGO@temp{sf}
%    \end{macrocode}
%
% \subsection{Generic logo commands}
%
%    \begin{macrocode}
\HOLOGO@IfExists\hologo{%
  \@PackageError{hologo}{%
    \string\hologo\ltx@space is already defined.\MessageBreak
    Package loading is aborted%
  }\@ehc
  \HOLOGO@AtEnd
}%
\HOLOGO@IfExists\hologoRobust{%
  \@PackageError{hologo}{%
    \string\hologoRobust\ltx@space is already defined.\MessageBreak
    Package loading is aborted%
  }\@ehc
  \HOLOGO@AtEnd
}%
%    \end{macrocode}
%
% \subsubsection{\cs{hologo} and friends}
%
%    \begin{macrocode}
\ifluatex
  \expandafter\ltx@firstofone
\else
  \expandafter\ltx@gobble
\fi
{%
  \ltx@IfUndefined{ifincsname}{%
    \ifnum\luatexversion<36 %
      \expandafter\ltx@gobble
    \else
      \expandafter\ltx@firstofone
    \fi
    {%
      \begingroup
        \ifcase0%
            \directlua{%
              if tex.enableprimitives then %
                tex.enableprimitives('HOLOGO@', {'ifincsname'})%
              else %
                tex.print('1')%
              end%
            }%
            \ifx\HOLOGO@ifincsname\@undefined 1\fi%
            \relax
          \expandafter\ltx@firstofone
        \else
          \endgroup
          \expandafter\ltx@gobble
        \fi
        {%
          \global\let\ifincsname\HOLOGO@ifincsname
        }%
      \HOLOGO@temp
    }%
  }{}%
}
%    \end{macrocode}
%    \begin{macrocode}
\ltx@IfUndefined{ifincsname}{%
  \catcode`$=14 %
}{%
  \catcode`$=9 %
}
%    \end{macrocode}
%
%    \begin{macro}{\hologo}
%    \begin{macrocode}
\def\hologo#1{%
$ \ifincsname
$   \ltx@ifundefined{HoLogoCs@\HOLOGO@Variant{#1}}{%
$     #1%
$   }{%
$     \csname HoLogoCs@\HOLOGO@Variant{#1}\endcsname\ltx@firstoftwo
$   }%
$ \else
    \HOLOGO@IfExists\texorpdfstring\texorpdfstring\ltx@firstoftwo
    {%
      \hologoRobust{#1}%
    }{%
      \ltx@ifundefined{HoLogoBkm@\HOLOGO@Variant{#1}}{%
        \ltx@ifundefined{HoLogo@#1}{?#1?}{#1}%
      }{%
        \csname HoLogoBkm@\HOLOGO@Variant{#1}\endcsname
        \ltx@firstoftwo
      }%
    }%
$ \fi
}
%    \end{macrocode}
%    \end{macro}
%    \begin{macro}{\Hologo}
%    \begin{macrocode}
\def\Hologo#1{%
$ \ifincsname
$   \ltx@ifundefined{HoLogoCs@\HOLOGO@Variant{#1}}{%
$     #1%
$   }{%
$     \csname HoLogoCs@\HOLOGO@Variant{#1}\endcsname\ltx@secondoftwo
$   }%
$ \else
    \HOLOGO@IfExists\texorpdfstring\texorpdfstring\ltx@firstoftwo
    {%
      \HologoRobust{#1}%
    }{%
      \ltx@ifundefined{HoLogoBkm@\HOLOGO@Variant{#1}}{%
        \ltx@ifundefined{HoLogo@#1}{?#1?}{#1}%
      }{%
        \csname HoLogoBkm@\HOLOGO@Variant{#1}\endcsname
        \ltx@secondoftwo
      }%
    }%
$ \fi
}
%    \end{macrocode}
%    \end{macro}
%
%    \begin{macro}{\hologoVariant}
%    \begin{macrocode}
\def\hologoVariant#1#2{%
  \ifx\relax#2\relax
    \hologo{#1}%
  \else
$   \ifincsname
$     \ltx@ifundefined{HoLogoCs@#1@#2}{%
$       #1%
$     }{%
$       \csname HoLogoCs@#1@#2\endcsname\ltx@firstoftwo
$     }%
$   \else
      \HOLOGO@IfExists\texorpdfstring\texorpdfstring\ltx@firstoftwo
      {%
        \hologoVariantRobust{#1}{#2}%
      }{%
        \ltx@ifundefined{HoLogoBkm@#1@#2}{%
          \ltx@ifundefined{HoLogo@#1}{?#1?}{#1}%
        }{%
          \csname HoLogoBkm@#1@#2\endcsname
          \ltx@firstoftwo
        }%
      }%
$   \fi
  \fi
}
%    \end{macrocode}
%    \end{macro}
%    \begin{macro}{\HologoVariant}
%    \begin{macrocode}
\def\HologoVariant#1#2{%
  \ifx\relax#2\relax
    \Hologo{#1}%
  \else
$   \ifincsname
$     \ltx@ifundefined{HoLogoCs@#1@#2}{%
$       #1%
$     }{%
$       \csname HoLogoCs@#1@#2\endcsname\ltx@secondoftwo
$     }%
$   \else
      \HOLOGO@IfExists\texorpdfstring\texorpdfstring\ltx@firstoftwo
      {%
        \HologoVariantRobust{#1}{#2}%
      }{%
        \ltx@ifundefined{HoLogoBkm@#1@#2}{%
          \ltx@ifundefined{HoLogo@#1}{?#1?}{#1}%
        }{%
          \csname HoLogoBkm@#1@#2\endcsname
          \ltx@secondoftwo
        }%
      }%
$   \fi
  \fi
}
%    \end{macrocode}
%    \end{macro}
%
%    \begin{macrocode}
\catcode`\$=3 %
%    \end{macrocode}
%
% \subsubsection{\cs{hologoRobust} and friends}
%
%    \begin{macro}{\hologoRobust}
%    \begin{macrocode}
\ltx@IfUndefined{protected}{%
  \ltx@IfUndefined{DeclareRobustCommand}{%
    \def\hologoRobust#1%
  }{%
    \DeclareRobustCommand*\hologoRobust[1]%
  }%
}{%
  \protected\def\hologoRobust#1%
}%
{%
  \edef\HOLOGO@name{#1}%
  \ltx@IfUndefined{HoLogo@\HOLOGO@Variant\HOLOGO@name}{%
    \@PackageError{hologo}{%
      Unknown logo `\HOLOGO@name'%
    }\@ehc
    ?\HOLOGO@name?%
  }{%
    \ltx@IfUndefined{ver@tex4ht.sty}{%
      \HoLogoFont@font\HOLOGO@name{general}{%
        \csname HoLogo@\HOLOGO@Variant\HOLOGO@name\endcsname
        \ltx@firstoftwo
      }%
    }{%
      \ltx@IfUndefined{HoLogoHtml@\HOLOGO@Variant\HOLOGO@name}{%
        \HOLOGO@name
      }{%
        \csname HoLogoHtml@\HOLOGO@Variant\HOLOGO@name\endcsname
        \ltx@firstoftwo
      }%
    }%
  }%
}
%    \end{macrocode}
%    \end{macro}
%    \begin{macro}{\HologoRobust}
%    \begin{macrocode}
\ltx@IfUndefined{protected}{%
  \ltx@IfUndefined{DeclareRobustCommand}{%
    \def\HologoRobust#1%
  }{%
    \DeclareRobustCommand*\HologoRobust[1]%
  }%
}{%
  \protected\def\HologoRobust#1%
}%
{%
  \edef\HOLOGO@name{#1}%
  \ltx@IfUndefined{HoLogo@\HOLOGO@Variant\HOLOGO@name}{%
    \@PackageError{hologo}{%
      Unknown logo `\HOLOGO@name'%
    }\@ehc
    ?\HOLOGO@name?%
  }{%
    \ltx@IfUndefined{ver@tex4ht.sty}{%
      \HoLogoFont@font\HOLOGO@name{general}{%
        \csname HoLogo@\HOLOGO@Variant\HOLOGO@name\endcsname
        \ltx@secondoftwo
      }%
    }{%
      \ltx@IfUndefined{HoLogoHtml@\HOLOGO@Variant\HOLOGO@name}{%
        \expandafter\HOLOGO@Uppercase\HOLOGO@name
      }{%
        \csname HoLogoHtml@\HOLOGO@Variant\HOLOGO@name\endcsname
        \ltx@secondoftwo
      }%
    }%
  }%
}
%    \end{macrocode}
%    \end{macro}
%    \begin{macro}{\hologoVariantRobust}
%    \begin{macrocode}
\ltx@IfUndefined{protected}{%
  \ltx@IfUndefined{DeclareRobustCommand}{%
    \def\hologoVariantRobust#1#2%
  }{%
    \DeclareRobustCommand*\hologoVariantRobust[2]%
  }%
}{%
  \protected\def\hologoVariantRobust#1#2%
}%
{%
  \begingroup
    \hologoLogoSetup{#1}{variant={#2}}%
    \hologoRobust{#1}%
  \endgroup
}
%    \end{macrocode}
%    \end{macro}
%    \begin{macro}{\HologoVariantRobust}
%    \begin{macrocode}
\ltx@IfUndefined{protected}{%
  \ltx@IfUndefined{DeclareRobustCommand}{%
    \def\HologoVariantRobust#1#2%
  }{%
    \DeclareRobustCommand*\HologoVariantRobust[2]%
  }%
}{%
  \protected\def\HologoVariantRobust#1#2%
}%
{%
  \begingroup
    \hologoLogoSetup{#1}{variant={#2}}%
    \HologoRobust{#1}%
  \endgroup
}
%    \end{macrocode}
%    \end{macro}
%
%    \begin{macro}{\hologorobust}
%    Macro \cs{hologorobust} is only defined for compatibility.
%    Its use is deprecated.
%    \begin{macrocode}
\def\hologorobust{\hologoRobust}
%    \end{macrocode}
%    \end{macro}
%
% \subsection{Helpers}
%
%    \begin{macro}{\HOLOGO@Uppercase}
%    Macro \cs{HOLOGO@Uppercase} is restricted to \cs{uppercase},
%    because \hologo{plainTeX} or \hologo{iniTeX} do not provide
%    \cs{MakeUppercase}.
%    \begin{macrocode}
\def\HOLOGO@Uppercase#1{\uppercase{#1}}
%    \end{macrocode}
%    \end{macro}
%
%    \begin{macro}{\HOLOGO@PdfdocUnicode}
%    \begin{macrocode}
\def\HOLOGO@PdfdocUnicode{%
  \ifx\ifHy@unicode\iftrue
    \expandafter\ltx@secondoftwo
  \else
    \expandafter\ltx@firstoftwo
  \fi
}
%    \end{macrocode}
%    \end{macro}
%
%    \begin{macro}{\HOLOGO@Math}
%    \begin{macrocode}
\def\HOLOGO@MathSetup{%
  \mathsurround0pt\relax
  \HOLOGO@IfExists\f@series{%
    \if b\expandafter\ltx@car\f@series x\@nil
      \csname boldmath\endcsname
   \fi
  }{}%
}
%    \end{macrocode}
%    \end{macro}
%
%    \begin{macro}{\HOLOGO@TempDimen}
%    \begin{macrocode}
\dimendef\HOLOGO@TempDimen=\ltx@zero
%    \end{macrocode}
%    \end{macro}
%    \begin{macro}{\HOLOGO@NegativeKerning}
%    \begin{macrocode}
\def\HOLOGO@NegativeKerning#1{%
  \begingroup
    \HOLOGO@TempDimen=0pt\relax
    \comma@parse@normalized{#1}{%
      \ifdim\HOLOGO@TempDimen=0pt %
        \expandafter\HOLOGO@@NegativeKerning\comma@entry
      \fi
      \ltx@gobble
    }%
    \ifdim\HOLOGO@TempDimen<0pt %
      \kern\HOLOGO@TempDimen
    \fi
  \endgroup
}
%    \end{macrocode}
%    \end{macro}
%    \begin{macro}{\HOLOGO@@NegativeKerning}
%    \begin{macrocode}
\def\HOLOGO@@NegativeKerning#1#2{%
  \setbox\ltx@zero\hbox{#1#2}%
  \HOLOGO@TempDimen=\wd\ltx@zero
  \setbox\ltx@zero\hbox{#1\kern0pt#2}%
  \advance\HOLOGO@TempDimen by -\wd\ltx@zero
}
%    \end{macrocode}
%    \end{macro}
%
%    \begin{macro}{\HOLOGO@SpaceFactor}
%    \begin{macrocode}
\def\HOLOGO@SpaceFactor{%
  \spacefactor1000 %
}
%    \end{macrocode}
%    \end{macro}
%
%    \begin{macro}{\HOLOGO@Span}
%    \begin{macrocode}
\def\HOLOGO@Span#1#2{%
  \HCode{<span class="HoLogo-#1">}%
  #2%
  \HCode{</span>}%
}
%    \end{macrocode}
%    \end{macro}
%
% \subsubsection{Text subscript}
%
%    \begin{macro}{\HOLOGO@SubScript}%
%    \begin{macrocode}
\def\HOLOGO@SubScript#1{%
  \ltx@IfUndefined{textsubscript}{%
    \ltx@IfUndefined{text}{%
      \ltx@mbox{%
        \mathsurround=0pt\relax
        $%
          _{%
            \ltx@IfUndefined{sf@size}{%
              \mathrm{#1}%
            }{%
              \mbox{%
                \fontsize\sf@size{0pt}\selectfont
                #1%
              }%
            }%
          }%
        $%
      }%
    }{%
      \ltx@mbox{%
        \mathsurround=0pt\relax
        $_{\text{#1}}$%
      }%
    }%
  }{%
    \textsubscript{#1}%
  }%
}
%    \end{macrocode}
%    \end{macro}
%
% \subsection{\hologo{TeX} and friends}
%
% \subsubsection{\hologo{TeX}}
%
%    \begin{macro}{\HoLogo@TeX}
%    Source: \hologo{LaTeX} kernel.
%    \begin{macrocode}
\def\HoLogo@TeX#1{%
  T\kern-.1667em\lower.5ex\hbox{E}\kern-.125emX\HOLOGO@SpaceFactor
}
%    \end{macrocode}
%    \end{macro}
%    \begin{macro}{\HoLogoHtml@TeX}
%    \begin{macrocode}
\def\HoLogoHtml@TeX#1{%
  \HoLogoCss@TeX
  \HOLOGO@Span{TeX}{%
    T%
    \HOLOGO@Span{e}{%
      E%
    }%
    X%
  }%
}
%    \end{macrocode}
%    \end{macro}
%    \begin{macro}{\HoLogoCss@TeX}
%    \begin{macrocode}
\def\HoLogoCss@TeX{%
  \Css{%
    span.HoLogo-TeX span.HoLogo-e{%
      position:relative;%
      top:.5ex;%
      margin-left:-.1667em;%
      margin-right:-.125em;%
    }%
  }%
  \Css{%
    a span.HoLogo-TeX span.HoLogo-e{%
      text-decoration:none;%
    }%
  }%
  \global\let\HoLogoCss@TeX\relax
}
%    \end{macrocode}
%    \end{macro}
%
% \subsubsection{\hologo{plainTeX}}
%
%    \begin{macro}{\HoLogo@plainTeX@space}
%    Source: ``The \hologo{TeX}book''
%    \begin{macrocode}
\def\HoLogo@plainTeX@space#1{%
  \HOLOGO@mbox{#1{p}{P}lain}\HOLOGO@space\hologo{TeX}%
}
%    \end{macrocode}
%    \end{macro}
%    \begin{macro}{\HoLogoCs@plainTeX@space}
%    \begin{macrocode}
\def\HoLogoCs@plainTeX@space#1{#1{p}{P}lain TeX}%
%    \end{macrocode}
%    \end{macro}
%    \begin{macro}{\HoLogoBkm@plainTeX@space}
%    \begin{macrocode}
\def\HoLogoBkm@plainTeX@space#1{%
  #1{p}{P}lain \hologo{TeX}%
}
%    \end{macrocode}
%    \end{macro}
%    \begin{macro}{\HoLogoHtml@plainTeX@space}
%    \begin{macrocode}
\def\HoLogoHtml@plainTeX@space#1{%
  #1{p}{P}lain \hologo{TeX}%
}
%    \end{macrocode}
%    \end{macro}
%
%    \begin{macro}{\HoLogo@plainTeX@hyphen}
%    \begin{macrocode}
\def\HoLogo@plainTeX@hyphen#1{%
  \HOLOGO@mbox{#1{p}{P}lain}\HOLOGO@hyphen\hologo{TeX}%
}
%    \end{macrocode}
%    \end{macro}
%    \begin{macro}{\HoLogoCs@plainTeX@hyphen}
%    \begin{macrocode}
\def\HoLogoCs@plainTeX@hyphen#1{#1{p}{P}lain-TeX}
%    \end{macrocode}
%    \end{macro}
%    \begin{macro}{\HoLogoBkm@plainTeX@hyphen}
%    \begin{macrocode}
\def\HoLogoBkm@plainTeX@hyphen#1{%
  #1{p}{P}lain-\hologo{TeX}%
}
%    \end{macrocode}
%    \end{macro}
%    \begin{macro}{\HoLogoHtml@plainTeX@hyphen}
%    \begin{macrocode}
\def\HoLogoHtml@plainTeX@hyphen#1{%
  #1{p}{P}lain-\hologo{TeX}%
}
%    \end{macrocode}
%    \end{macro}
%
%    \begin{macro}{\HoLogo@plainTeX@runtogether}
%    \begin{macrocode}
\def\HoLogo@plainTeX@runtogether#1{%
  \HOLOGO@mbox{#1{p}{P}lain\hologo{TeX}}%
}
%    \end{macrocode}
%    \end{macro}
%    \begin{macro}{\HoLogoCs@plainTeX@runtogether}
%    \begin{macrocode}
\def\HoLogoCs@plainTeX@runtogether#1{#1{p}{P}lainTeX}
%    \end{macrocode}
%    \end{macro}
%    \begin{macro}{\HoLogoBkm@plainTeX@runtogether}
%    \begin{macrocode}
\def\HoLogoBkm@plainTeX@runtogether#1{%
  #1{p}{P}lain\hologo{TeX}%
}
%    \end{macrocode}
%    \end{macro}
%    \begin{macro}{\HoLogoHtml@plainTeX@runtogether}
%    \begin{macrocode}
\def\HoLogoHtml@plainTeX@runtogether#1{%
  #1{p}{P}lain\hologo{TeX}%
}
%    \end{macrocode}
%    \end{macro}
%
%    \begin{macro}{\HoLogo@plainTeX}
%    \begin{macrocode}
\def\HoLogo@plainTeX{\HoLogo@plainTeX@space}
%    \end{macrocode}
%    \end{macro}
%    \begin{macro}{\HoLogoCs@plainTeX}
%    \begin{macrocode}
\def\HoLogoCs@plainTeX{\HoLogoCs@plainTeX@space}
%    \end{macrocode}
%    \end{macro}
%    \begin{macro}{\HoLogoBkm@plainTeX}
%    \begin{macrocode}
\def\HoLogoBkm@plainTeX{\HoLogoBkm@plainTeX@space}
%    \end{macrocode}
%    \end{macro}
%    \begin{macro}{\HoLogoHtml@plainTeX}
%    \begin{macrocode}
\def\HoLogoHtml@plainTeX{\HoLogoHtml@plainTeX@space}
%    \end{macrocode}
%    \end{macro}
%
% \subsubsection{\hologo{LaTeX}}
%
%    Source: \hologo{LaTeX} kernel.
%\begin{quote}
%\begin{verbatim}
%\DeclareRobustCommand{\LaTeX}{%
%  L%
%  \kern-.36em%
%  {%
%    \sbox\z@ T%
%    \vbox to\ht\z@{%
%      \hbox{%
%        \check@mathfonts
%        \fontsize\sf@size\z@
%        \math@fontsfalse
%        \selectfont
%        A%
%      }%
%      \vss
%    }%
%  }%
%  \kern-.15em%
%  \TeX
%}
%\end{verbatim}
%\end{quote}
%
%    \begin{macro}{\HoLogo@La}
%    \begin{macrocode}
\def\HoLogo@La#1{%
  L%
  \kern-.36em%
  \begingroup
    \setbox\ltx@zero\hbox{T}%
    \vbox to\ht\ltx@zero{%
      \hbox{%
        \ltx@ifundefined{check@mathfonts}{%
          \csname sevenrm\endcsname
        }{%
          \check@mathfonts
          \fontsize\sf@size{0pt}%
          \math@fontsfalse\selectfont
        }%
        A%
      }%
      \vss
    }%
  \endgroup
}
%    \end{macrocode}
%    \end{macro}
%
%    \begin{macro}{\HoLogo@LaTeX}
%    Source: \hologo{LaTeX} kernel.
%    \begin{macrocode}
\def\HoLogo@LaTeX#1{%
  \hologo{La}%
  \kern-.15em%
  \hologo{TeX}%
}
%    \end{macrocode}
%    \end{macro}
%    \begin{macro}{\HoLogoHtml@LaTeX}
%    \begin{macrocode}
\def\HoLogoHtml@LaTeX#1{%
  \HoLogoCss@LaTeX
  \HOLOGO@Span{LaTeX}{%
    L%
    \HOLOGO@Span{a}{%
      A%
    }%
    \hologo{TeX}%
  }%
}
%    \end{macrocode}
%    \end{macro}
%    \begin{macro}{\HoLogoCss@LaTeX}
%    \begin{macrocode}
\def\HoLogoCss@LaTeX{%
  \Css{%
    span.HoLogo-LaTeX span.HoLogo-a{%
      position:relative;%
      top:-.5ex;%
      margin-left:-.36em;%
      margin-right:-.15em;%
      font-size:85\%;%
    }%
  }%
  \global\let\HoLogoCss@LaTeX\relax
}
%    \end{macrocode}
%    \end{macro}
%
% \subsubsection{\hologo{(La)TeX}}
%
%    \begin{macro}{\HoLogo@LaTeXTeX}
%    The kerning around the parentheses is taken
%    from package \xpackage{dtklogos} \cite{dtklogos}.
%\begin{quote}
%\begin{verbatim}
%\DeclareRobustCommand{\LaTeXTeX}{%
%  (%
%  \kern-.15em%
%  L%
%  \kern-.36em%
%  {%
%    \sbox\z@ T%
%    \vbox to\ht0{%
%      \hbox{%
%        $\m@th$%
%        \csname S@\f@size\endcsname
%        \fontsize\sf@size\z@
%        \math@fontsfalse
%        \selectfont
%        A%
%      }%
%      \vss
%    }%
%  }%
%  \kern-.2em%
%  )%
%  \kern-.15em%
%  \TeX
%}
%\end{verbatim}
%\end{quote}
%    \begin{macrocode}
\def\HoLogo@LaTeXTeX#1{%
  (%
  \kern-.15em%
  \hologo{La}%
  \kern-.2em%
  )%
  \kern-.15em%
  \hologo{TeX}%
}
%    \end{macrocode}
%    \end{macro}
%    \begin{macro}{\HoLogoBkm@LaTeXTeX}
%    \begin{macrocode}
\def\HoLogoBkm@LaTeXTeX#1{(La)TeX}
%    \end{macrocode}
%    \end{macro}
%
%    \begin{macro}{\HoLogo@(La)TeX}
%    \begin{macrocode}
\expandafter
\let\csname HoLogo@(La)TeX\endcsname\HoLogo@LaTeXTeX
%    \end{macrocode}
%    \end{macro}
%    \begin{macro}{\HoLogoBkm@(La)TeX}
%    \begin{macrocode}
\expandafter
\let\csname HoLogoBkm@(La)TeX\endcsname\HoLogoBkm@LaTeXTeX
%    \end{macrocode}
%    \end{macro}
%    \begin{macro}{\HoLogoHtml@LaTeXTeX}
%    \begin{macrocode}
\def\HoLogoHtml@LaTeXTeX#1{%
  \HoLogoCss@LaTeXTeX
  \HOLOGO@Span{LaTeXTeX}{%
    (%
    \HOLOGO@Span{L}{L}%
    \HOLOGO@Span{a}{A}%
    \HOLOGO@Span{ParenRight}{)}%
    \hologo{TeX}%
  }%
}
%    \end{macrocode}
%    \end{macro}
%    \begin{macro}{\HoLogoHtml@(La)TeX}
%    Kerning after opening parentheses and before closing parentheses
%    is $-0.1$\,em. The original values $-0.15$\,em
%    looked too ugly for a serif font.
%    \begin{macrocode}
\expandafter
\let\csname HoLogoHtml@(La)TeX\endcsname\HoLogoHtml@LaTeXTeX
%    \end{macrocode}
%    \end{macro}
%    \begin{macro}{\HoLogoCss@LaTeXTeX}
%    \begin{macrocode}
\def\HoLogoCss@LaTeXTeX{%
  \Css{%
    span.HoLogo-LaTeXTeX span.HoLogo-L{%
      margin-left:-.1em;%
    }%
  }%
  \Css{%
    span.HoLogo-LaTeXTeX span.HoLogo-a{%
      position:relative;%
      top:-.5ex;%
      margin-left:-.36em;%
      margin-right:-.1em;%
      font-size:85\%;%
    }%
  }%
  \Css{%
    span.HoLogo-LaTeXTeX span.HoLogo-ParenRight{%
      margin-right:-.15em;%
    }%
  }%
  \global\let\HoLogoCss@LaTeXTeX\relax
}
%    \end{macrocode}
%    \end{macro}
%
% \subsubsection{\hologo{LaTeXe}}
%
%    \begin{macro}{\HoLogo@LaTeXe}
%    Source: \hologo{LaTeX} kernel
%    \begin{macrocode}
\def\HoLogo@LaTeXe#1{%
  \hologo{LaTeX}%
  \kern.15em%
  \hbox{%
    \HOLOGO@MathSetup
    2%
    $_{\textstyle\varepsilon}$%
  }%
}
%    \end{macrocode}
%    \end{macro}
%
%    \begin{macro}{\HoLogoCs@LaTeXe}
%    \begin{macrocode}
\ifnum64=`\^^^^0040\relax % test for big chars of LuaTeX/XeTeX
  \catcode`\$=9 %
  \catcode`\&=14 %
\else
  \catcode`\$=14 %
  \catcode`\&=9 %
\fi
\def\HoLogoCs@LaTeXe#1{%
  LaTeX2%
$ \string ^^^^0395%
& e%
}%
\catcode`\$=3 %
\catcode`\&=4 %
%    \end{macrocode}
%    \end{macro}
%
%    \begin{macro}{\HoLogoBkm@LaTeXe}
%    \begin{macrocode}
\def\HoLogoBkm@LaTeXe#1{%
  \hologo{LaTeX}%
  2%
  \HOLOGO@PdfdocUnicode{e}{\textepsilon}%
}
%    \end{macrocode}
%    \end{macro}
%
%    \begin{macro}{\HoLogoHtml@LaTeXe}
%    \begin{macrocode}
\def\HoLogoHtml@LaTeXe#1{%
  \HoLogoCss@LaTeXe
  \HOLOGO@Span{LaTeX2e}{%
    \hologo{LaTeX}%
    \HOLOGO@Span{2}{2}%
    \HOLOGO@Span{e}{%
      \HOLOGO@MathSetup
      \ensuremath{\textstyle\varepsilon}%
    }%
  }%
}
%    \end{macrocode}
%    \end{macro}
%    \begin{macro}{\HoLogoCss@LaTeXe}
%    \begin{macrocode}
\def\HoLogoCss@LaTeXe{%
  \Css{%
    span.HoLogo-LaTeX2e span.HoLogo-2{%
      padding-left:.15em;%
    }%
  }%
  \Css{%
    span.HoLogo-LaTeX2e span.HoLogo-e{%
      position:relative;%
      top:.35ex;%
      text-decoration:none;%
    }%
  }%
  \global\let\HoLogoCss@LaTeXe\relax
}
%    \end{macrocode}
%    \end{macro}
%
%    \begin{macro}{\HoLogo@LaTeX2e}
%    \begin{macrocode}
\expandafter
\let\csname HoLogo@LaTeX2e\endcsname\HoLogo@LaTeXe
%    \end{macrocode}
%    \end{macro}
%    \begin{macro}{\HoLogoCs@LaTeX2e}
%    \begin{macrocode}
\expandafter
\let\csname HoLogoCs@LaTeX2e\endcsname\HoLogoCs@LaTeXe
%    \end{macrocode}
%    \end{macro}
%    \begin{macro}{\HoLogoBkm@LaTeX2e}
%    \begin{macrocode}
\expandafter
\let\csname HoLogoBkm@LaTeX2e\endcsname\HoLogoBkm@LaTeXe
%    \end{macrocode}
%    \end{macro}
%    \begin{macro}{\HoLogoHtml@LaTeX2e}
%    \begin{macrocode}
\expandafter
\let\csname HoLogoHtml@LaTeX2e\endcsname\HoLogoHtml@LaTeXe
%    \end{macrocode}
%    \end{macro}
%
% \subsubsection{\hologo{LaTeX3}}
%
%    \begin{macro}{\HoLogo@LaTeX3}
%    Source: \hologo{LaTeX} kernel
%    \begin{macrocode}
\expandafter\def\csname HoLogo@LaTeX3\endcsname#1{%
  \hologo{LaTeX}%
  3%
}
%    \end{macrocode}
%    \end{macro}
%
%    \begin{macro}{\HoLogoBkm@LaTeX3}
%    \begin{macrocode}
\expandafter\def\csname HoLogoBkm@LaTeX3\endcsname#1{%
  \hologo{LaTeX}%
  3%
}
%    \end{macrocode}
%    \end{macro}
%    \begin{macro}{\HoLogoHtml@LaTeX3}
%    \begin{macrocode}
\expandafter
\let\csname HoLogoHtml@LaTeX3\expandafter\endcsname
\csname HoLogo@LaTeX3\endcsname
%    \end{macrocode}
%    \end{macro}
%
% \subsubsection{\hologo{LaTeXML}}
%
%    \begin{macro}{\HoLogo@LaTeXML}
%    \begin{macrocode}
\def\HoLogo@LaTeXML#1{%
  \HOLOGO@mbox{%
    \hologo{La}%
    \kern-.15em%
    T%
    \kern-.1667em%
    \lower.5ex\hbox{E}%
    \kern-.125em%
    \HoLogoFont@font{LaTeXML}{sc}{xml}%
  }%
}
%    \end{macrocode}
%    \end{macro}
%    \begin{macro}{\HoLogoHtml@pdfLaTeX}
%    \begin{macrocode}
\def\HoLogoHtml@LaTeXML#1{%
  \HOLOGO@Span{LaTeXML}{%
    \HoLogoCss@LaTeX
    \HoLogoCss@TeX
    \HOLOGO@Span{LaTeX}{%
      L%
      \HOLOGO@Span{a}{%
        A%
      }%
    }%
    \HOLOGO@Span{TeX}{%
      T%
      \HOLOGO@Span{e}{%
        E%
      }%
    }%
    \HCode{<span style="font-variant: small-caps;">}%
    xml%
    \HCode{</span>}%
  }%
}
%    \end{macrocode}
%    \end{macro}
%
% \subsubsection{\hologo{eTeX}}
%
%    \begin{macro}{\HoLogo@eTeX}
%    Source: package \xpackage{etex}
%    \begin{macrocode}
\def\HoLogo@eTeX#1{%
  \ltx@mbox{%
    \HOLOGO@MathSetup
    $\varepsilon$%
    -%
    \HOLOGO@NegativeKerning{-T,T-,To}%
    \hologo{TeX}%
  }%
}
%    \end{macrocode}
%    \end{macro}
%    \begin{macro}{\HoLogoCs@eTeX}
%    \begin{macrocode}
\ifnum64=`\^^^^0040\relax % test for big chars of LuaTeX/XeTeX
  \catcode`\$=9 %
  \catcode`\&=14 %
\else
  \catcode`\$=14 %
  \catcode`\&=9 %
\fi
\def\HoLogoCs@eTeX#1{%
$ #1{\string ^^^^0395}{\string ^^^^03b5}%
& #1{e}{E}%
  TeX%
}%
\catcode`\$=3 %
\catcode`\&=4 %
%    \end{macrocode}
%    \end{macro}
%    \begin{macro}{\HoLogoBkm@eTeX}
%    \begin{macrocode}
\def\HoLogoBkm@eTeX#1{%
  \HOLOGO@PdfdocUnicode{#1{e}{E}}{\textepsilon}%
  -%
  \hologo{TeX}%
}
%    \end{macrocode}
%    \end{macro}
%    \begin{macro}{\HoLogoHtml@eTeX}
%    \begin{macrocode}
\def\HoLogoHtml@eTeX#1{%
  \ltx@mbox{%
    \HOLOGO@MathSetup
    $\varepsilon$%
    -%
    \hologo{TeX}%
  }%
}
%    \end{macrocode}
%    \end{macro}
%
% \subsubsection{\hologo{iniTeX}}
%
%    \begin{macro}{\HoLogo@iniTeX}
%    \begin{macrocode}
\def\HoLogo@iniTeX#1{%
  \HOLOGO@mbox{%
    #1{i}{I}ni\hologo{TeX}%
  }%
}
%    \end{macrocode}
%    \end{macro}
%    \begin{macro}{\HoLogoCs@iniTeX}
%    \begin{macrocode}
\def\HoLogoCs@iniTeX#1{#1{i}{I}niTeX}
%    \end{macrocode}
%    \end{macro}
%    \begin{macro}{\HoLogoBkm@iniTeX}
%    \begin{macrocode}
\def\HoLogoBkm@iniTeX#1{%
  #1{i}{I}ni\hologo{TeX}%
}
%    \end{macrocode}
%    \end{macro}
%    \begin{macro}{\HoLogoHtml@iniTeX}
%    \begin{macrocode}
\let\HoLogoHtml@iniTeX\HoLogo@iniTeX
%    \end{macrocode}
%    \end{macro}
%
% \subsubsection{\hologo{virTeX}}
%
%    \begin{macro}{\HoLogo@virTeX}
%    \begin{macrocode}
\def\HoLogo@virTeX#1{%
  \HOLOGO@mbox{%
    #1{v}{V}ir\hologo{TeX}%
  }%
}
%    \end{macrocode}
%    \end{macro}
%    \begin{macro}{\HoLogoCs@virTeX}
%    \begin{macrocode}
\def\HoLogoCs@virTeX#1{#1{v}{V}irTeX}
%    \end{macrocode}
%    \end{macro}
%    \begin{macro}{\HoLogoBkm@virTeX}
%    \begin{macrocode}
\def\HoLogoBkm@virTeX#1{%
  #1{v}{V}ir\hologo{TeX}%
}
%    \end{macrocode}
%    \end{macro}
%    \begin{macro}{\HoLogoHtml@virTeX}
%    \begin{macrocode}
\let\HoLogoHtml@virTeX\HoLogo@virTeX
%    \end{macrocode}
%    \end{macro}
%
% \subsubsection{\hologo{SliTeX}}
%
% \paragraph{Definitions of the three variants.}
%
%    \begin{macro}{\HoLogo@SLiTeX@lift}
%    \begin{macrocode}
\def\HoLogo@SLiTeX@lift#1{%
  \HoLogoFont@font{SliTeX}{rm}{%
    S%
    \kern-.06em%
    L%
    \kern-.18em%
    \raise.32ex\hbox{\HoLogoFont@font{SliTeX}{sc}{i}}%
    \HOLOGO@discretionary
    \kern-.06em%
    \hologo{TeX}%
  }%
}
%    \end{macrocode}
%    \end{macro}
%    \begin{macro}{\HoLogoBkm@SLiTeX@lift}
%    \begin{macrocode}
\def\HoLogoBkm@SLiTeX@lift#1{SLiTeX}
%    \end{macrocode}
%    \end{macro}
%    \begin{macro}{\HoLogoHtml@SLiTeX@lift}
%    \begin{macrocode}
\def\HoLogoHtml@SLiTeX@lift#1{%
  \HoLogoCss@SLiTeX@lift
  \HOLOGO@Span{SLiTeX-lift}{%
    \HoLogoFont@font{SliTeX}{rm}{%
      S%
      \HOLOGO@Span{L}{L}%
      \HOLOGO@Span{i}{i}%
      \hologo{TeX}%
    }%
  }%
}
%    \end{macrocode}
%    \end{macro}
%    \begin{macro}{\HoLogoCss@SLiTeX@lift}
%    \begin{macrocode}
\def\HoLogoCss@SLiTeX@lift{%
  \Css{%
    span.HoLogo-SLiTeX-lift span.HoLogo-L{%
      margin-left:-.06em;%
      margin-right:-.18em;%
    }%
  }%
  \Css{%
    span.HoLogo-SLiTeX-lift span.HoLogo-i{%
      position:relative;%
      top:-.32ex;%
      margin-right:-.06em;%
      font-variant:small-caps;%
    }%
  }%
  \global\let\HoLogoCss@SLiTeX@lift\relax
}
%    \end{macrocode}
%    \end{macro}
%
%    \begin{macro}{\HoLogo@SliTeX@simple}
%    \begin{macrocode}
\def\HoLogo@SliTeX@simple#1{%
  \HoLogoFont@font{SliTeX}{rm}{%
    \ltx@mbox{%
      \HoLogoFont@font{SliTeX}{sc}{Sli}%
    }%
    \HOLOGO@discretionary
    \hologo{TeX}%
  }%
}
%    \end{macrocode}
%    \end{macro}
%    \begin{macro}{\HoLogoBkm@SliTeX@simple}
%    \begin{macrocode}
\def\HoLogoBkm@SliTeX@simple#1{SliTeX}
%    \end{macrocode}
%    \end{macro}
%    \begin{macro}{\HoLogoHtml@SliTeX@simple}
%    \begin{macrocode}
\let\HoLogoHtml@SliTeX@simple\HoLogo@SliTeX@simple
%    \end{macrocode}
%    \end{macro}
%
%    \begin{macro}{\HoLogo@SliTeX@narrow}
%    \begin{macrocode}
\def\HoLogo@SliTeX@narrow#1{%
  \HoLogoFont@font{SliTeX}{rm}{%
    \ltx@mbox{%
      S%
      \kern-.06em%
      \HoLogoFont@font{SliTeX}{sc}{%
        l%
        \kern-.035em%
        i%
      }%
    }%
    \HOLOGO@discretionary
    \kern-.06em%
    \hologo{TeX}%
  }%
}
%    \end{macrocode}
%    \end{macro}
%    \begin{macro}{\HoLogoBkm@SliTeX@narrow}
%    \begin{macrocode}
\def\HoLogoBkm@SliTeX@narrow#1{SliTeX}
%    \end{macrocode}
%    \end{macro}
%    \begin{macro}{\HoLogoHtml@SliTeX@narrow}
%    \begin{macrocode}
\def\HoLogoHtml@SliTeX@narrow#1{%
  \HoLogoCss@SliTeX@narrow
  \HOLOGO@Span{SliTeX-narrow}{%
    \HoLogoFont@font{SliTeX}{rm}{%
      S%
        \HOLOGO@Span{l}{l}%
        \HOLOGO@Span{i}{i}%
      \hologo{TeX}%
    }%
  }%
}
%    \end{macrocode}
%    \end{macro}
%    \begin{macro}{\HoLogoCss@SliTeX@narrow}
%    \begin{macrocode}
\def\HoLogoCss@SliTeX@narrow{%
  \Css{%
    span.HoLogo-SliTeX-narrow span.HoLogo-l{%
      margin-left:-.06em;%
      margin-right:-.035em;%
      font-variant:small-caps;%
    }%
  }%
  \Css{%
    span.HoLogo-SliTeX-narrow span.HoLogo-i{%
      margin-right:-.06em;%
      font-variant:small-caps;%
    }%
  }%
  \global\let\HoLogoCss@SliTeX@narrow\relax
}
%    \end{macrocode}
%    \end{macro}
%
% \paragraph{Macro set completion.}
%
%    \begin{macro}{\HoLogo@SLiTeX@simple}
%    \begin{macrocode}
\def\HoLogo@SLiTeX@simple{\HoLogo@SliTeX@simple}
%    \end{macrocode}
%    \end{macro}
%    \begin{macro}{\HoLogoBkm@SLiTeX@simple}
%    \begin{macrocode}
\def\HoLogoBkm@SLiTeX@simple{\HoLogoBkm@SliTeX@simple}
%    \end{macrocode}
%    \end{macro}
%    \begin{macro}{\HoLogoHtml@SLiTeX@simple}
%    \begin{macrocode}
\def\HoLogoHtml@SLiTeX@simple{\HoLogoHtml@SliTeX@simple}
%    \end{macrocode}
%    \end{macro}
%
%    \begin{macro}{\HoLogo@SLiTeX@narrow}
%    \begin{macrocode}
\def\HoLogo@SLiTeX@narrow{\HoLogo@SliTeX@narrow}
%    \end{macrocode}
%    \end{macro}
%    \begin{macro}{\HoLogoBkm@SLiTeX@narrow}
%    \begin{macrocode}
\def\HoLogoBkm@SLiTeX@narrow{\HoLogoBkm@SliTeX@narrow}
%    \end{macrocode}
%    \end{macro}
%    \begin{macro}{\HoLogoHtml@SLiTeX@narrow}
%    \begin{macrocode}
\def\HoLogoHtml@SLiTeX@narrow{\HoLogoHtml@SliTeX@narrow}
%    \end{macrocode}
%    \end{macro}
%
%    \begin{macro}{\HoLogo@SliTeX@lift}
%    \begin{macrocode}
\def\HoLogo@SliTeX@lift{\HoLogo@SLiTeX@lift}
%    \end{macrocode}
%    \end{macro}
%    \begin{macro}{\HoLogoBkm@SliTeX@lift}
%    \begin{macrocode}
\def\HoLogoBkm@SliTeX@lift{\HoLogoBkm@SLiTeX@lift}
%    \end{macrocode}
%    \end{macro}
%    \begin{macro}{\HoLogoHtml@SliTeX@lift}
%    \begin{macrocode}
\def\HoLogoHtml@SliTeX@lift{\HoLogoHtml@SLiTeX@lift}
%    \end{macrocode}
%    \end{macro}
%
% \paragraph{Defaults.}
%
%    \begin{macro}{\HoLogo@SLiTeX}
%    \begin{macrocode}
\def\HoLogo@SLiTeX{\HoLogo@SLiTeX@lift}
%    \end{macrocode}
%    \end{macro}
%    \begin{macro}{\HoLogoBkm@SLiTeX}
%    \begin{macrocode}
\def\HoLogoBkm@SLiTeX{\HoLogoBkm@SLiTeX@lift}
%    \end{macrocode}
%    \end{macro}
%    \begin{macro}{\HoLogoHtml@SLiTeX}
%    \begin{macrocode}
\def\HoLogoHtml@SLiTeX{\HoLogoHtml@SLiTeX@lift}
%    \end{macrocode}
%    \end{macro}
%
%    \begin{macro}{\HoLogo@SliTeX}
%    \begin{macrocode}
\def\HoLogo@SliTeX{\HoLogo@SliTeX@narrow}
%    \end{macrocode}
%    \end{macro}
%    \begin{macro}{\HoLogoBkm@SliTeX}
%    \begin{macrocode}
\def\HoLogoBkm@SliTeX{\HoLogoBkm@SliTeX@narrow}
%    \end{macrocode}
%    \end{macro}
%    \begin{macro}{\HoLogoHtml@SliTeX}
%    \begin{macrocode}
\def\HoLogoHtml@SliTeX{\HoLogoHtml@SliTeX@narrow}
%    \end{macrocode}
%    \end{macro}
%
% \subsubsection{\hologo{LuaTeX}}
%
%    \begin{macro}{\HoLogo@LuaTeX}
%    The kerning is an idea of Hans Hagen, see mailing list
%    `luatex at tug dot org' in March 2010.
%    \begin{macrocode}
\def\HoLogo@LuaTeX#1{%
  \HOLOGO@mbox{%
    Lua%
    \HOLOGO@NegativeKerning{aT,oT,To}%
    \hologo{TeX}%
  }%
}
%    \end{macrocode}
%    \end{macro}
%    \begin{macro}{\HoLogoHtml@LuaTeX}
%    \begin{macrocode}
\let\HoLogoHtml@LuaTeX\HoLogo@LuaTeX
%    \end{macrocode}
%    \end{macro}
%
% \subsubsection{\hologo{LuaLaTeX}}
%
%    \begin{macro}{\HoLogo@LuaLaTeX}
%    \begin{macrocode}
\def\HoLogo@LuaLaTeX#1{%
  \HOLOGO@mbox{%
    Lua%
    \hologo{LaTeX}%
  }%
}
%    \end{macrocode}
%    \end{macro}
%    \begin{macro}{\HoLogoHtml@LuaLaTeX}
%    \begin{macrocode}
\let\HoLogoHtml@LuaLaTeX\HoLogo@LuaLaTeX
%    \end{macrocode}
%    \end{macro}
%
% \subsubsection{\hologo{XeTeX}, \hologo{XeLaTeX}}
%
%    \begin{macro}{\HOLOGO@IfCharExists}
%    \begin{macrocode}
\ifluatex
  \ifnum\luatexversion<36 %
  \else
    \def\HOLOGO@IfCharExists#1{%
      \ifnum
        \directlua{%
           if luaotfload and luaotfload.aux then
             if luaotfload.aux.font_has_glyph(%
                    font.current(), \number#1) then % 	 
	       tex.print("1") % 	 
	     end % 	 
	   elseif font and font.fonts and font.current then %
            local f = font.fonts[font.current()]%
            if f.characters and f.characters[\number#1] then %
              tex.print("1")%
            end %
          end%
        }0=\ltx@zero
        \expandafter\ltx@secondoftwo
      \else
        \expandafter\ltx@firstoftwo
      \fi
    }%
  \fi
\fi
\ltx@IfUndefined{HOLOGO@IfCharExists}{%
  \def\HOLOGO@@IfCharExists#1{%
    \begingroup
      \tracinglostchars=\ltx@zero
      \setbox\ltx@zero=\hbox{%
        \kern7sp\char#1\relax
        \ifnum\lastkern>\ltx@zero
          \expandafter\aftergroup\csname iffalse\endcsname
        \else
          \expandafter\aftergroup\csname iftrue\endcsname
        \fi
      }%
      % \if{true|false} from \aftergroup
      \endgroup
      \expandafter\ltx@firstoftwo
    \else
      \endgroup
      \expandafter\ltx@secondoftwo
    \fi
  }%
  \ifxetex
    \ltx@IfUndefined{XeTeXfonttype}{}{%
      \ltx@IfUndefined{XeTeXcharglyph}{}{%
        \def\HOLOGO@IfCharExists#1{%
          \ifnum\XeTeXfonttype\font>\ltx@zero
            \expandafter\ltx@firstofthree
          \else
            \expandafter\ltx@gobble
          \fi
          {%
            \ifnum\XeTeXcharglyph#1>\ltx@zero
              \expandafter\ltx@firstoftwo
            \else
              \expandafter\ltx@secondoftwo
            \fi
          }%
          \HOLOGO@@IfCharExists{#1}%
        }%
      }%
    }%
  \fi
}{}
\ltx@ifundefined{HOLOGO@IfCharExists}{%
  \ifnum64=`\^^^^0040\relax % test for big chars of LuaTeX/XeTeX
    \let\HOLOGO@IfCharExists\HOLOGO@@IfCharExists
  \else
    \def\HOLOGO@IfCharExists#1{%
      \ifnum#1>255 %
        \expandafter\ltx@fourthoffour
      \fi
      \HOLOGO@@IfCharExists{#1}%
    }%
  \fi
}{}
%    \end{macrocode}
%    \end{macro}
%
%    \begin{macro}{\HoLogo@Xe}
%    Source: package \xpackage{dtklogos}
%    \begin{macrocode}
\def\HoLogo@Xe#1{%
  X%
  \kern-.1em\relax
  \HOLOGO@IfCharExists{"018E}{%
    \lower.5ex\hbox{\char"018E}%
  }{%
    \chardef\HOLOGO@choice=\ltx@zero
    \ifdim\fontdimen\ltx@one\font>0pt %
      \ltx@IfUndefined{rotatebox}{%
        \ltx@IfUndefined{pgftext}{%
          \ltx@IfUndefined{psscalebox}{%
            \ltx@IfUndefined{HOLOGO@ScaleBox@\hologoDriver}{%
            }{%
              \chardef\HOLOGO@choice=4 %
            }%
          }{%
            \chardef\HOLOGO@choice=3 %
          }%
        }{%
          \chardef\HOLOGO@choice=2 %
        }%
      }{%
        \chardef\HOLOGO@choice=1 %
      }%
      \ifcase\HOLOGO@choice
        \HOLOGO@WarningUnsupportedDriver{Xe}%
        e%
      \or % 1: \rotatebox
        \begingroup
          \setbox\ltx@zero\hbox{\rotatebox{180}{E}}%
          \ltx@LocDimenA=\dp\ltx@zero
          \advance\ltx@LocDimenA by -.5ex\relax
          \raise\ltx@LocDimenA\box\ltx@zero
        \endgroup
      \or % 2: \pgftext
        \lower.5ex\hbox{%
          \pgfpicture
            \pgftext[rotate=180]{E}%
          \endpgfpicture
        }%
      \or % 3: \psscalebox
        \begingroup
          \setbox\ltx@zero\hbox{\psscalebox{-1 -1}{E}}%
          \ltx@LocDimenA=\dp\ltx@zero
          \advance\ltx@LocDimenA by -.5ex\relax
          \raise\ltx@LocDimenA\box\ltx@zero
        \endgroup
      \or % 4: \HOLOGO@PointReflectBox
        \lower.5ex\hbox{\HOLOGO@PointReflectBox{E}}%
      \else
        \@PackageError{hologo}{Internal error (choice/it}\@ehc
      \fi
    \else
      \ltx@IfUndefined{reflectbox}{%
        \ltx@IfUndefined{pgftext}{%
          \ltx@IfUndefined{psscalebox}{%
            \ltx@IfUndefined{HOLOGO@ScaleBox@\hologoDriver}{%
            }{%
              \chardef\HOLOGO@choice=4 %
            }%
          }{%
            \chardef\HOLOGO@choice=3 %
          }%
        }{%
          \chardef\HOLOGO@choice=2 %
        }%
      }{%
        \chardef\HOLOGO@choice=1 %
      }%
      \ifcase\HOLOGO@choice
        \HOLOGO@WarningUnsupportedDriver{Xe}%
        e%
      \or % 1: reflectbox
        \lower.5ex\hbox{%
          \reflectbox{E}%
        }%
      \or % 2: \pgftext
        \lower.5ex\hbox{%
          \pgfpicture
            \pgftransformxscale{-1}%
            \pgftext{E}%
          \endpgfpicture
        }%
      \or % 3: \psscalebox
        \lower.5ex\hbox{%
          \psscalebox{-1 1}{E}%
        }%
      \or % 4: \HOLOGO@Reflectbox
        \lower.5ex\hbox{%
          \HOLOGO@ReflectBox{E}%
        }%
      \else
        \@PackageError{hologo}{Internal error (choice/up)}\@ehc
      \fi
    \fi
  }%
}
%    \end{macrocode}
%    \end{macro}
%    \begin{macro}{\HoLogoHtml@Xe}
%    \begin{macrocode}
\def\HoLogoHtml@Xe#1{%
  \HoLogoCss@Xe
  \HOLOGO@Span{Xe}{%
    X%
    \HOLOGO@Span{e}{%
      \HCode{&\ltx@hashchar x018e;}%
    }%
  }%
}
%    \end{macrocode}
%    \end{macro}
%    \begin{macro}{\HoLogoCss@Xe}
%    \begin{macrocode}
\def\HoLogoCss@Xe{%
  \Css{%
    span.HoLogo-Xe span.HoLogo-e{%
      position:relative;%
      top:.5ex;%
      left-margin:-.1em;%
    }%
  }%
  \global\let\HoLogoCss@Xe\relax
}
%    \end{macrocode}
%    \end{macro}
%
%    \begin{macro}{\HoLogo@XeTeX}
%    \begin{macrocode}
\def\HoLogo@XeTeX#1{%
  \hologo{Xe}%
  \kern-.15em\relax
  \hologo{TeX}%
}
%    \end{macrocode}
%    \end{macro}
%
%    \begin{macro}{\HoLogoHtml@XeTeX}
%    \begin{macrocode}
\def\HoLogoHtml@XeTeX#1{%
  \HoLogoCss@XeTeX
  \HOLOGO@Span{XeTeX}{%
    \hologo{Xe}%
    \hologo{TeX}%
  }%
}
%    \end{macrocode}
%    \end{macro}
%    \begin{macro}{\HoLogoCss@XeTeX}
%    \begin{macrocode}
\def\HoLogoCss@XeTeX{%
  \Css{%
    span.HoLogo-XeTeX span.HoLogo-TeX{%
      margin-left:-.15em;%
    }%
  }%
  \global\let\HoLogoCss@XeTeX\relax
}
%    \end{macrocode}
%    \end{macro}
%
%    \begin{macro}{\HoLogo@XeLaTeX}
%    \begin{macrocode}
\def\HoLogo@XeLaTeX#1{%
  \hologo{Xe}%
  \kern-.13em%
  \hologo{LaTeX}%
}
%    \end{macrocode}
%    \end{macro}
%    \begin{macro}{\HoLogoHtml@XeLaTeX}
%    \begin{macrocode}
\def\HoLogoHtml@XeLaTeX#1{%
  \HoLogoCss@XeLaTeX
  \HOLOGO@Span{XeLaTeX}{%
    \hologo{Xe}%
    \hologo{LaTeX}%
  }%
}
%    \end{macrocode}
%    \end{macro}
%    \begin{macro}{\HoLogoCss@XeLaTeX}
%    \begin{macrocode}
\def\HoLogoCss@XeLaTeX{%
  \Css{%
    span.HoLogo-XeLaTeX span.HoLogo-Xe{%
      margin-right:-.13em;%
    }%
  }%
  \global\let\HoLogoCss@XeLaTeX\relax
}
%    \end{macrocode}
%    \end{macro}
%
% \subsubsection{\hologo{pdfTeX}, \hologo{pdfLaTeX}}
%
%    \begin{macro}{\HoLogo@pdfTeX}
%    \begin{macrocode}
\def\HoLogo@pdfTeX#1{%
  \HOLOGO@mbox{%
    #1{p}{P}df\hologo{TeX}%
  }%
}
%    \end{macrocode}
%    \end{macro}
%    \begin{macro}{\HoLogoCs@pdfTeX}
%    \begin{macrocode}
\def\HoLogoCs@pdfTeX#1{#1{p}{P}dfTeX}
%    \end{macrocode}
%    \end{macro}
%    \begin{macro}{\HoLogoBkm@pdfTeX}
%    \begin{macrocode}
\def\HoLogoBkm@pdfTeX#1{%
  #1{p}{P}df\hologo{TeX}%
}
%    \end{macrocode}
%    \end{macro}
%    \begin{macro}{\HoLogoHtml@pdfTeX}
%    \begin{macrocode}
\let\HoLogoHtml@pdfTeX\HoLogo@pdfTeX
%    \end{macrocode}
%    \end{macro}
%
%    \begin{macro}{\HoLogo@pdfLaTeX}
%    \begin{macrocode}
\def\HoLogo@pdfLaTeX#1{%
  \HOLOGO@mbox{%
    #1{p}{P}df\hologo{LaTeX}%
  }%
}
%    \end{macrocode}
%    \end{macro}
%    \begin{macro}{\HoLogoCs@pdfLaTeX}
%    \begin{macrocode}
\def\HoLogoCs@pdfLaTeX#1{#1{p}{P}dfLaTeX}
%    \end{macrocode}
%    \end{macro}
%    \begin{macro}{\HoLogoBkm@pdfLaTeX}
%    \begin{macrocode}
\def\HoLogoBkm@pdfLaTeX#1{%
  #1{p}{P}df\hologo{LaTeX}%
}
%    \end{macrocode}
%    \end{macro}
%    \begin{macro}{\HoLogoHtml@pdfLaTeX}
%    \begin{macrocode}
\let\HoLogoHtml@pdfLaTeX\HoLogo@pdfLaTeX
%    \end{macrocode}
%    \end{macro}
%
% \subsubsection{\hologo{VTeX}}
%
%    \begin{macro}{\HoLogo@VTeX}
%    \begin{macrocode}
\def\HoLogo@VTeX#1{%
  \HOLOGO@mbox{%
    V\hologo{TeX}%
  }%
}
%    \end{macrocode}
%    \end{macro}
%    \begin{macro}{\HoLogoHtml@VTeX}
%    \begin{macrocode}
\let\HoLogoHtml@VTeX\HoLogo@VTeX
%    \end{macrocode}
%    \end{macro}
%
% \subsubsection{\hologo{AmS}, \dots}
%
%    Source: class \xclass{amsdtx}
%
%    \begin{macro}{\HoLogo@AmS}
%    \begin{macrocode}
\def\HoLogo@AmS#1{%
  \HoLogoFont@font{AmS}{sy}{%
    A%
    \kern-.1667em%
    \lower.5ex\hbox{M}%
    \kern-.125em%
    S%
  }%
}
%    \end{macrocode}
%    \end{macro}
%    \begin{macro}{\HoLogoBkm@AmS}
%    \begin{macrocode}
\def\HoLogoBkm@AmS#1{AmS}
%    \end{macrocode}
%    \end{macro}
%    \begin{macro}{\HoLogoHtml@AmS}
%    \begin{macrocode}
\def\HoLogoHtml@AmS#1{%
  \HoLogoCss@AmS
%  \HoLogoFont@font{AmS}{sy}{%
    \HOLOGO@Span{AmS}{%
      A%
      \HOLOGO@Span{M}{M}%
      S%
    }%
%   }%
}
%    \end{macrocode}
%    \end{macro}
%    \begin{macro}{\HoLogoCss@AmS}
%    \begin{macrocode}
\def\HoLogoCss@AmS{%
  \Css{%
    span.HoLogo-AmS span.HoLogo-M{%
      position:relative;%
      top:.5ex;%
      margin-left:-.1667em;%
      margin-right:-.125em;%
      text-decoration:none;%
    }%
  }%
  \global\let\HoLogoCss@AmS\relax
}
%    \end{macrocode}
%    \end{macro}
%
%    \begin{macro}{\HoLogo@AmSTeX}
%    \begin{macrocode}
\def\HoLogo@AmSTeX#1{%
  \hologo{AmS}%
  \HOLOGO@hyphen
  \hologo{TeX}%
}
%    \end{macrocode}
%    \end{macro}
%    \begin{macro}{\HoLogoBkm@AmSTeX}
%    \begin{macrocode}
\def\HoLogoBkm@AmSTeX#1{AmS-TeX}%
%    \end{macrocode}
%    \end{macro}
%    \begin{macro}{\HoLogoHtml@AmSTeX}
%    \begin{macrocode}
\let\HoLogoHtml@AmSTeX\HoLogo@AmSTeX
%    \end{macrocode}
%    \end{macro}
%
%    \begin{macro}{\HoLogo@AmSLaTeX}
%    \begin{macrocode}
\def\HoLogo@AmSLaTeX#1{%
  \hologo{AmS}%
  \HOLOGO@hyphen
  \hologo{LaTeX}%
}
%    \end{macrocode}
%    \end{macro}
%    \begin{macro}{\HoLogoBkm@AmSLaTeX}
%    \begin{macrocode}
\def\HoLogoBkm@AmSLaTeX#1{AmS-LaTeX}%
%    \end{macrocode}
%    \end{macro}
%    \begin{macro}{\HoLogoHtml@AmSLaTeX}
%    \begin{macrocode}
\let\HoLogoHtml@AmSLaTeX\HoLogo@AmSLaTeX
%    \end{macrocode}
%    \end{macro}
%
% \subsubsection{\hologo{BibTeX}}
%
%    \begin{macro}{\HoLogo@BibTeX@sc}
%    A definition of \hologo{BibTeX} is provided in
%    the documentation source for the manual of \hologo{BibTeX}
%    \cite{btxdoc}.
%\begin{quote}
%\begin{verbatim}
%\def\BibTeX{%
%  {%
%    \rm
%    B%
%    \kern-.05em%
%    {%
%      \sc
%      i%
%      \kern-.025em %
%      b%
%    }%
%    \kern-.08em
%    T%
%    \kern-.1667em%
%    \lower.7ex\hbox{E}%
%    \kern-.125em%
%    X%
%  }%
%}
%\end{verbatim}
%\end{quote}
%    \begin{macrocode}
\def\HoLogo@BibTeX@sc#1{%
  B%
  \kern-.05em%
  \HoLogoFont@font{BibTeX}{sc}{%
    i%
    \kern-.025em%
    b%
  }%
  \HOLOGO@discretionary
  \kern-.08em%
  \hologo{TeX}%
}
%    \end{macrocode}
%    \end{macro}
%    \begin{macro}{\HoLogoHtml@BibTeX@sc}
%    \begin{macrocode}
\def\HoLogoHtml@BibTeX@sc#1{%
  \HoLogoCss@BibTeX@sc
  \HOLOGO@Span{BibTeX-sc}{%
    B%
    \HOLOGO@Span{i}{i}%
    \HOLOGO@Span{b}{b}%
    \hologo{TeX}%
  }%
}
%    \end{macrocode}
%    \end{macro}
%    \begin{macro}{\HoLogoCss@BibTeX@sc}
%    \begin{macrocode}
\def\HoLogoCss@BibTeX@sc{%
  \Css{%
    span.HoLogo-BibTeX-sc span.HoLogo-i{%
      margin-left:-.05em;%
      margin-right:-.025em;%
      font-variant:small-caps;%
    }%
  }%
  \Css{%
    span.HoLogo-BibTeX-sc span.HoLogo-b{%
      margin-right:-.08em;%
      font-variant:small-caps;%
    }%
  }%
  \global\let\HoLogoCss@BibTeX@sc\relax
}
%    \end{macrocode}
%    \end{macro}
%
%    \begin{macro}{\HoLogo@BibTeX@sf}
%    Variant \xoption{sf} avoids trouble with unavailable
%    small caps fonts (e.g., bold versions of Computer Modern or
%    Latin Modern). The definition is taken from
%    package \xpackage{dtklogos} \cite{dtklogos}.
%\begin{quote}
%\begin{verbatim}
%\DeclareRobustCommand{\BibTeX}{%
%  B%
%  \kern-.05em%
%  \hbox{%
%    $\m@th$% %% force math size calculations
%    \csname S@\f@size\endcsname
%    \fontsize\sf@size\z@
%    \math@fontsfalse
%    \selectfont
%    I%
%    \kern-.025em%
%    B
%  }%
%  \kern-.08em%
%  \-%
%  \TeX
%}
%\end{verbatim}
%\end{quote}
%    \begin{macrocode}
\def\HoLogo@BibTeX@sf#1{%
  B%
  \kern-.05em%
  \HoLogoFont@font{BibTeX}{bibsf}{%
    I%
    \kern-.025em%
    B%
  }%
  \HOLOGO@discretionary
  \kern-.08em%
  \hologo{TeX}%
}
%    \end{macrocode}
%    \end{macro}
%    \begin{macro}{\HoLogoHtml@BibTeX@sf}
%    \begin{macrocode}
\def\HoLogoHtml@BibTeX@sf#1{%
  \HoLogoCss@BibTeX@sf
  \HOLOGO@Span{BibTeX-sf}{%
    B%
    \HoLogoFont@font{BibTeX}{bibsf}{%
      \HOLOGO@Span{i}{I}%
      B%
    }%
    \hologo{TeX}%
  }%
}
%    \end{macrocode}
%    \end{macro}
%    \begin{macro}{\HoLogoCss@BibTeX@sf}
%    \begin{macrocode}
\def\HoLogoCss@BibTeX@sf{%
  \Css{%
    span.HoLogo-BibTeX-sf span.HoLogo-i{%
      margin-left:-.05em;%
      margin-right:-.025em;%
    }%
  }%
  \Css{%
    span.HoLogo-BibTeX-sf span.HoLogo-TeX{%
      margin-left:-.08em;%
    }%
  }%
  \global\let\HoLogoCss@BibTeX@sf\relax
}
%    \end{macrocode}
%    \end{macro}
%
%    \begin{macro}{\HoLogo@BibTeX}
%    \begin{macrocode}
\def\HoLogo@BibTeX{\HoLogo@BibTeX@sf}
%    \end{macrocode}
%    \end{macro}
%    \begin{macro}{\HoLogoHtml@BibTeX}
%    \begin{macrocode}
\def\HoLogoHtml@BibTeX{\HoLogoHtml@BibTeX@sf}
%    \end{macrocode}
%    \end{macro}
%
% \subsubsection{\hologo{BibTeX8}}
%
%    \begin{macro}{\HoLogo@BibTeX8}
%    \begin{macrocode}
\expandafter\def\csname HoLogo@BibTeX8\endcsname#1{%
  \hologo{BibTeX}%
  8%
}
%    \end{macrocode}
%    \end{macro}
%
%    \begin{macro}{\HoLogoBkm@BibTeX8}
%    \begin{macrocode}
\expandafter\def\csname HoLogoBkm@BibTeX8\endcsname#1{%
  \hologo{BibTeX}%
  8%
}
%    \end{macrocode}
%    \end{macro}
%    \begin{macro}{\HoLogoHtml@BibTeX8}
%    \begin{macrocode}
\expandafter
\let\csname HoLogoHtml@BibTeX8\expandafter\endcsname
\csname HoLogo@BibTeX8\endcsname
%    \end{macrocode}
%    \end{macro}
%
% \subsubsection{\hologo{ConTeXt}}
%
%    \begin{macro}{\HoLogo@ConTeXt@simple}
%    \begin{macrocode}
\def\HoLogo@ConTeXt@simple#1{%
  \HOLOGO@mbox{Con}%
  \HOLOGO@discretionary
  \HOLOGO@mbox{\hologo{TeX}t}%
}
%    \end{macrocode}
%    \end{macro}
%    \begin{macro}{\HoLogoHtml@ConTeXt@simple}
%    \begin{macrocode}
\let\HoLogoHtml@ConTeXt@simple\HoLogo@ConTeXt@simple
%    \end{macrocode}
%    \end{macro}
%
%    \begin{macro}{\HoLogo@ConTeXt@narrow}
%    This definition of logo \hologo{ConTeXt} with variant \xoption{narrow}
%    comes from TUGboat's class \xclass{ltugboat} (version 2010/11/15 v2.8).
%    \begin{macrocode}
\def\HoLogo@ConTeXt@narrow#1{%
  \HOLOGO@mbox{C\kern-.0333emon}%
  \HOLOGO@discretionary
  \kern-.0667em%
  \HOLOGO@mbox{\hologo{TeX}\kern-.0333emt}%
}
%    \end{macrocode}
%    \end{macro}
%    \begin{macro}{\HoLogoHtml@ConTeXt@narrow}
%    \begin{macrocode}
\def\HoLogoHtml@ConTeXt@narrow#1{%
  \HoLogoCss@ConTeXt@narrow
  \HOLOGO@Span{ConTeXt-narrow}{%
    \HOLOGO@Span{C}{C}%
    on%
    \hologo{TeX}%
    t%
  }%
}
%    \end{macrocode}
%    \end{macro}
%    \begin{macro}{\HoLogoCss@ConTeXt@narrow}
%    \begin{macrocode}
\def\HoLogoCss@ConTeXt@narrow{%
  \Css{%
    span.HoLogo-ConTeXt-narrow span.HoLogo-C{%
      margin-left:-.0333em;%
    }%
  }%
  \Css{%
    span.HoLogo-ConTeXt-narrow span.HoLogo-TeX{%
      margin-left:-.0667em;%
      margin-right:-.0333em;%
    }%
  }%
  \global\let\HoLogoCss@ConTeXt@narrow\relax
}
%    \end{macrocode}
%    \end{macro}
%
%    \begin{macro}{\HoLogo@ConTeXt}
%    \begin{macrocode}
\def\HoLogo@ConTeXt{\HoLogo@ConTeXt@narrow}
%    \end{macrocode}
%    \end{macro}
%    \begin{macro}{\HoLogoHtml@ConTeXt}
%    \begin{macrocode}
\def\HoLogoHtml@ConTeXt{\HoLogoHtml@ConTeXt@narrow}
%    \end{macrocode}
%    \end{macro}
%
% \subsubsection{\hologo{emTeX}}
%
%    \begin{macro}{\HoLogo@emTeX}
%    \begin{macrocode}
\def\HoLogo@emTeX#1{%
  \HOLOGO@mbox{#1{e}{E}m}%
  \HOLOGO@discretionary
  \hologo{TeX}%
}
%    \end{macrocode}
%    \end{macro}
%    \begin{macro}{\HoLogoCs@emTeX}
%    \begin{macrocode}
\def\HoLogoCs@emTeX#1{#1{e}{E}mTeX}%
%    \end{macrocode}
%    \end{macro}
%    \begin{macro}{\HoLogoBkm@emTeX}
%    \begin{macrocode}
\def\HoLogoBkm@emTeX#1{%
  #1{e}{E}m\hologo{TeX}%
}
%    \end{macrocode}
%    \end{macro}
%    \begin{macro}{\HoLogoHtml@emTeX}
%    \begin{macrocode}
\let\HoLogoHtml@emTeX\HoLogo@emTeX
%    \end{macrocode}
%    \end{macro}
%
% \subsubsection{\hologo{ExTeX}}
%
%    \begin{macro}{\HoLogo@ExTeX}
%    The definition is taken from the FAQ of the
%    project \hologo{ExTeX}
%    \cite{ExTeX-FAQ}.
%\begin{quote}
%\begin{verbatim}
%\def\ExTeX{%
%  \textrm{% Logo always with serifs
%    \ensuremath{%
%      \textstyle
%      \varepsilon_{%
%        \kern-0.15em%
%        \mathcal{X}%
%      }%
%    }%
%    \kern-.15em%
%    \TeX
%  }%
%}
%\end{verbatim}
%\end{quote}
%    \begin{macrocode}
\def\HoLogo@ExTeX#1{%
  \HoLogoFont@font{ExTeX}{rm}{%
    \ltx@mbox{%
      \HOLOGO@MathSetup
      $%
        \textstyle
        \varepsilon_{%
          \kern-0.15em%
          \HoLogoFont@font{ExTeX}{sy}{X}%
        }%
      $%
    }%
    \HOLOGO@discretionary
    \kern-.15em%
    \hologo{TeX}%
  }%
}
%    \end{macrocode}
%    \end{macro}
%    \begin{macro}{\HoLogoHtml@ExTeX}
%    \begin{macrocode}
\def\HoLogoHtml@ExTeX#1{%
  \HoLogoCss@ExTeX
  \HoLogoFont@font{ExTeX}{rm}{%
    \HOLOGO@Span{ExTeX}{%
      \ltx@mbox{%
        \HOLOGO@MathSetup
        $\textstyle\varepsilon$%
        \HOLOGO@Span{X}{$\textstyle\chi$}%
        \hologo{TeX}%
      }%
    }%
  }%
}
%    \end{macrocode}
%    \end{macro}
%    \begin{macro}{\HoLogoBkm@ExTeX}
%    \begin{macrocode}
\def\HoLogoBkm@ExTeX#1{%
  \HOLOGO@PdfdocUnicode{#1{e}{E}x}{\textepsilon\textchi}%
  \hologo{TeX}%
}
%    \end{macrocode}
%    \end{macro}
%    \begin{macro}{\HoLogoCss@ExTeX}
%    \begin{macrocode}
\def\HoLogoCss@ExTeX{%
  \Css{%
    span.HoLogo-ExTeX{%
      font-family:serif;%
    }%
  }%
  \Css{%
    span.HoLogo-ExTeX span.HoLogo-TeX{%
      margin-left:-.15em;%
    }%
  }%
  \global\let\HoLogoCss@ExTeX\relax
}
%    \end{macrocode}
%    \end{macro}
%
% \subsubsection{\hologo{MiKTeX}}
%
%    \begin{macro}{\HoLogo@MiKTeX}
%    \begin{macrocode}
\def\HoLogo@MiKTeX#1{%
  \HOLOGO@mbox{MiK}%
  \HOLOGO@discretionary
  \hologo{TeX}%
}
%    \end{macrocode}
%    \end{macro}
%    \begin{macro}{\HoLogoHtml@MiKTeX}
%    \begin{macrocode}
\let\HoLogoHtml@MiKTeX\HoLogo@MiKTeX
%    \end{macrocode}
%    \end{macro}
%
% \subsubsection{\hologo{OzTeX} and friends}
%
%    Source: \hologo{OzTeX} FAQ \cite{OzTeX}:
%    \begin{quote}
%      |\def\OzTeX{O\kern-.03em z\kern-.15em\TeX}|\\
%      (There is no kerning in OzMF, OzMP and OzTtH.)
%    \end{quote}
%
%    \begin{macro}{\HoLogo@OzTeX}
%    \begin{macrocode}
\def\HoLogo@OzTeX#1{%
  O%
  \kern-.03em %
  z%
  \kern-.15em %
  \hologo{TeX}%
}
%    \end{macrocode}
%    \end{macro}
%    \begin{macro}{\HoLogoHtml@OzTeX}
%    \begin{macrocode}
\def\HoLogoHtml@OzTeX#1{%
  \HoLogoCss@OzTeX
  \HOLOGO@Span{OzTeX}{%
    O%
    \HOLOGO@Span{z}{z}%
    \hologo{TeX}%
  }%
}
%    \end{macrocode}
%    \end{macro}
%    \begin{macro}{\HoLogoCss@OzTeX}
%    \begin{macrocode}
\def\HoLogoCss@OzTeX{%
  \Css{%
    span.HoLogo-OzTeX span.HoLogo-z{%
      margin-left:-.03em;%
      margin-right:-.15em;%
    }%
  }%
  \global\let\HoLogoCss@OzTeX\relax
}
%    \end{macrocode}
%    \end{macro}
%
%    \begin{macro}{\HoLogo@OzMF}
%    \begin{macrocode}
\def\HoLogo@OzMF#1{%
  \HOLOGO@mbox{OzMF}%
}
%    \end{macrocode}
%    \end{macro}
%    \begin{macro}{\HoLogo@OzMP}
%    \begin{macrocode}
\def\HoLogo@OzMP#1{%
  \HOLOGO@mbox{OzMP}%
}
%    \end{macrocode}
%    \end{macro}
%    \begin{macro}{\HoLogo@OzTtH}
%    \begin{macrocode}
\def\HoLogo@OzTtH#1{%
  \HOLOGO@mbox{OzTtH}%
}
%    \end{macrocode}
%    \end{macro}
%
% \subsubsection{\hologo{PCTeX}}
%
%    \begin{macro}{\HoLogo@PCTeX}
%    \begin{macrocode}
\def\HoLogo@PCTeX#1{%
  \HOLOGO@mbox{PC}%
  \hologo{TeX}%
}
%    \end{macrocode}
%    \end{macro}
%    \begin{macro}{\HoLogoHtml@PCTeX}
%    \begin{macrocode}
\let\HoLogoHtml@PCTeX\HoLogo@PCTeX
%    \end{macrocode}
%    \end{macro}
%
% \subsubsection{\hologo{PiCTeX}}
%
%    The original definitions from \xfile{pictex.tex} \cite{PiCTeX}:
%\begin{quote}
%\begin{verbatim}
%\def\PiC{%
%  P%
%  \kern-.12em%
%  \lower.5ex\hbox{I}%
%  \kern-.075em%
%  C%
%}
%\def\PiCTeX{%
%  \PiC
%  \kern-.11em%
%  \TeX
%}
%\end{verbatim}
%\end{quote}
%
%    \begin{macro}{\HoLogo@PiC}
%    \begin{macrocode}
\def\HoLogo@PiC#1{%
  P%
  \kern-.12em%
  \lower.5ex\hbox{I}%
  \kern-.075em%
  C%
  \HOLOGO@SpaceFactor
}
%    \end{macrocode}
%    \end{macro}
%    \begin{macro}{\HoLogoHtml@PiC}
%    \begin{macrocode}
\def\HoLogoHtml@PiC#1{%
  \HoLogoCss@PiC
  \HOLOGO@Span{PiC}{%
    P%
    \HOLOGO@Span{i}{I}%
    C%
  }%
}
%    \end{macrocode}
%    \end{macro}
%    \begin{macro}{\HoLogoCss@PiC}
%    \begin{macrocode}
\def\HoLogoCss@PiC{%
  \Css{%
    span.HoLogo-PiC span.HoLogo-i{%
      position:relative;%
      top:.5ex;%
      margin-left:-.12em;%
      margin-right:-.075em;%
      text-decoration:none;%
    }%
  }%
  \global\let\HoLogoCss@PiC\relax
}
%    \end{macrocode}
%    \end{macro}
%
%    \begin{macro}{\HoLogo@PiCTeX}
%    \begin{macrocode}
\def\HoLogo@PiCTeX#1{%
  \hologo{PiC}%
  \HOLOGO@discretionary
  \kern-.11em%
  \hologo{TeX}%
}
%    \end{macrocode}
%    \end{macro}
%    \begin{macro}{\HoLogoHtml@PiCTeX}
%    \begin{macrocode}
\def\HoLogoHtml@PiCTeX#1{%
  \HoLogoCss@PiCTeX
  \HOLOGO@Span{PiCTeX}{%
    \hologo{PiC}%
    \hologo{TeX}%
  }%
}
%    \end{macrocode}
%    \end{macro}
%    \begin{macro}{\HoLogoCss@PiCTeX}
%    \begin{macrocode}
\def\HoLogoCss@PiCTeX{%
  \Css{%
    span.HoLogo-PiCTeX span.HoLogo-PiC{%
      margin-right:-.11em;%
    }%
  }%
  \global\let\HoLogoCss@PiCTeX\relax
}
%    \end{macrocode}
%    \end{macro}
%
% \subsubsection{\hologo{teTeX}}
%
%    \begin{macro}{\HoLogo@teTeX}
%    \begin{macrocode}
\def\HoLogo@teTeX#1{%
  \HOLOGO@mbox{#1{t}{T}e}%
  \HOLOGO@discretionary
  \hologo{TeX}%
}
%    \end{macrocode}
%    \end{macro}
%    \begin{macro}{\HoLogoCs@teTeX}
%    \begin{macrocode}
\def\HoLogoCs@teTeX#1{#1{t}{T}dfTeX}
%    \end{macrocode}
%    \end{macro}
%    \begin{macro}{\HoLogoBkm@teTeX}
%    \begin{macrocode}
\def\HoLogoBkm@teTeX#1{%
  #1{t}{T}e\hologo{TeX}%
}
%    \end{macrocode}
%    \end{macro}
%    \begin{macro}{\HoLogoHtml@teTeX}
%    \begin{macrocode}
\let\HoLogoHtml@teTeX\HoLogo@teTeX
%    \end{macrocode}
%    \end{macro}
%
% \subsubsection{\hologo{TeX4ht}}
%
%    \begin{macro}{\HoLogo@TeX4ht}
%    \begin{macrocode}
\expandafter\def\csname HoLogo@TeX4ht\endcsname#1{%
  \HOLOGO@mbox{\hologo{TeX}4ht}%
}
%    \end{macrocode}
%    \end{macro}
%    \begin{macro}{\HoLogoHtml@TeX4ht}
%    \begin{macrocode}
\expandafter
\let\csname HoLogoHtml@TeX4ht\expandafter\endcsname
\csname HoLogo@TeX4ht\endcsname
%    \end{macrocode}
%    \end{macro}
%
%
% \subsubsection{\hologo{SageTeX}}
%
%    \begin{macro}{\HoLogo@SageTeX}
%    \begin{macrocode}
\def\HoLogo@SageTeX#1{%
  \HOLOGO@mbox{Sage}%
  \HOLOGO@discretionary
  \HOLOGO@NegativeKerning{eT,oT,To}%
  \hologo{TeX}%
}
%    \end{macrocode}
%    \end{macro}
%    \begin{macro}{\HoLogoHtml@SageTeX}
%    \begin{macrocode}
\let\HoLogoHtml@SageTeX\HoLogo@SageTeX
%    \end{macrocode}
%    \end{macro}
%
% \subsection{\hologo{METAFONT} and friends}
%
%    \begin{macro}{\HoLogo@METAFONT}
%    \begin{macrocode}
\def\HoLogo@METAFONT#1{%
  \HoLogoFont@font{METAFONT}{logo}{%
    \HOLOGO@mbox{META}%
    \HOLOGO@discretionary
    \HOLOGO@mbox{FONT}%
  }%
}
%    \end{macrocode}
%    \end{macro}
%
%    \begin{macro}{\HoLogo@METAPOST}
%    \begin{macrocode}
\def\HoLogo@METAPOST#1{%
  \HoLogoFont@font{METAPOST}{logo}{%
    \HOLOGO@mbox{META}%
    \HOLOGO@discretionary
    \HOLOGO@mbox{POST}%
  }%
}
%    \end{macrocode}
%    \end{macro}
%
%    \begin{macro}{\HoLogo@MetaFun}
%    \begin{macrocode}
\def\HoLogo@MetaFun#1{%
  \HOLOGO@mbox{Meta}%
  \HOLOGO@discretionary
  \HOLOGO@mbox{Fun}%
}
%    \end{macrocode}
%    \end{macro}
%
%    \begin{macro}{\HoLogo@MetaPost}
%    \begin{macrocode}
\def\HoLogo@MetaPost#1{%
  \HOLOGO@mbox{Meta}%
  \HOLOGO@discretionary
  \HOLOGO@mbox{Post}%
}
%    \end{macrocode}
%    \end{macro}
%
% \subsection{Others}
%
% \subsubsection{\hologo{biber}}
%
%    \begin{macro}{\HoLogo@biber}
%    \begin{macrocode}
\def\HoLogo@biber#1{%
  \HOLOGO@mbox{#1{b}{B}i}%
  \HOLOGO@discretionary
  \HOLOGO@mbox{ber}%
}
%    \end{macrocode}
%    \end{macro}
%    \begin{macro}{\HoLogoCs@biber}
%    \begin{macrocode}
\def\HoLogoCs@biber#1{#1{b}{B}iber}
%    \end{macrocode}
%    \end{macro}
%    \begin{macro}{\HoLogoBkm@biber}
%    \begin{macrocode}
\def\HoLogoBkm@biber#1{%
  #1{b}{B}iber%
}
%    \end{macrocode}
%    \end{macro}
%    \begin{macro}{\HoLogoHtml@biber}
%    \begin{macrocode}
\let\HoLogoHtml@biber\HoLogo@biber
%    \end{macrocode}
%    \end{macro}
%
% \subsubsection{\hologo{KOMAScript}}
%
%    \begin{macro}{\HoLogo@KOMAScript}
%    The definition for \hologo{KOMAScript} is taken
%    from \hologo{KOMAScript} (\xfile{scrlogo.dtx}, reformatted) \cite{scrlogo}:
%\begin{quote}
%\begin{verbatim}
%\@ifundefined{KOMAScript}{%
%  \DeclareRobustCommand{\KOMAScript}{%
%    \textsf{%
%      K\kern.05em O\kern.05emM\kern.05em A%
%      \kern.1em-\kern.1em %
%      Script%
%    }%
%  }%
%}{}
%\end{verbatim}
%\end{quote}
%    \begin{macrocode}
\def\HoLogo@KOMAScript#1{%
  \HoLogoFont@font{KOMAScript}{sf}{%
    \HOLOGO@mbox{%
      K\kern.05em%
      O\kern.05em%
      M\kern.05em%
      A%
    }%
    \kern.1em%
    \HOLOGO@hyphen
    \kern.1em%
    \HOLOGO@mbox{Script}%
  }%
}
%    \end{macrocode}
%    \end{macro}
%    \begin{macro}{\HoLogoBkm@KOMAScript}
%    \begin{macrocode}
\def\HoLogoBkm@KOMAScript#1{%
  KOMA-Script%
}
%    \end{macrocode}
%    \end{macro}
%    \begin{macro}{\HoLogoHtml@KOMAScript}
%    \begin{macrocode}
\def\HoLogoHtml@KOMAScript#1{%
  \HoLogoCss@KOMAScript
  \HoLogoFont@font{KOMAScript}{sf}{%
    \HOLOGO@Span{KOMAScript}{%
      K%
      \HOLOGO@Span{O}{O}%
      M%
      \HOLOGO@Span{A}{A}%
      \HOLOGO@Span{hyphen}{-}%
      Script%
    }%
  }%
}
%    \end{macrocode}
%    \end{macro}
%    \begin{macro}{\HoLogoCss@KOMAScript}
%    \begin{macrocode}
\def\HoLogoCss@KOMAScript{%
  \Css{%
    span.HoLogo-KOMAScript{%
      font-family:sans-serif;%
    }%
  }%
  \Css{%
    span.HoLogo-KOMAScript span.HoLogo-O{%
      padding-left:.05em;%
      padding-right:.05em;%
    }%
  }%
  \Css{%
    span.HoLogo-KOMAScript span.HoLogo-A{%
      padding-left:.05em;%
    }%
  }%
  \Css{%
    span.HoLogo-KOMAScript span.HoLogo-hyphen{%
      padding-left:.1em;%
      padding-right:.1em;%
    }%
  }%
  \global\let\HoLogoCss@KOMAScript\relax
}
%    \end{macrocode}
%    \end{macro}
%
% \subsubsection{\hologo{LyX}}
%
%    \begin{macro}{\HoLogo@LyX}
%    The definition is taken from the documentation source files
%    of \hologo{LyX}, \xfile{Intro.lyx} \cite{LyX}:
%\begin{quote}
%\begin{verbatim}
%\def\LyX{%
%  \texorpdfstring{%
%    L\kern-.1667em\lower.25em\hbox{Y}\kern-.125emX\@%
%  }{%
%    LyX%
%  }%
%}
%\end{verbatim}
%\end{quote}
%    \begin{macrocode}
\def\HoLogo@LyX#1{%
  L%
  \kern-.1667em%
  \lower.25em\hbox{Y}%
  \kern-.125em%
  X%
  \HOLOGO@SpaceFactor
}
%    \end{macrocode}
%    \end{macro}
%    \begin{macro}{\HoLogoHtml@LyX}
%    \begin{macrocode}
\def\HoLogoHtml@LyX#1{%
  \HoLogoCss@LyX
  \HOLOGO@Span{LyX}{%
    L%
    \HOLOGO@Span{y}{Y}%
    X%
  }%
}
%    \end{macrocode}
%    \end{macro}
%    \begin{macro}{\HoLogoCss@LyX}
%    \begin{macrocode}
\def\HoLogoCss@LyX{%
  \Css{%
    span.HoLogo-LyX span.HoLogo-y{%
      position:relative;%
      top:.25em;%
      margin-left:-.1667em;%
      margin-right:-.125em;%
      text-decoration:none;%
    }%
  }%
  \global\let\HoLogoCss@LyX\relax
}
%    \end{macrocode}
%    \end{macro}
%
% \subsubsection{\hologo{NTS}}
%
%    \begin{macro}{\HoLogo@NTS}
%    Definition for \hologo{NTS} can be found in
%    package \xpackage{etex\textunderscore man} for the \hologo{eTeX} manual \cite{etexman}
%    and in package \xpackage{dtklogos} \cite{dtklogos}:
%\begin{quote}
%\begin{verbatim}
%\def\NTS{%
%  \leavevmode
%  \hbox{%
%    $%
%      \cal N%
%      \kern-0.35em%
%      \lower0.5ex\hbox{$\cal T$}%
%      \kern-0.2em%
%      S%
%    $%
%  }%
%}
%\end{verbatim}
%\end{quote}
%    \begin{macrocode}
\def\HoLogo@NTS#1{%
  \HoLogoFont@font{NTS}{sy}{%
    N\/%
    \kern-.35em%
    \lower.5ex\hbox{T\/}%
    \kern-.2em%
    S\/%
  }%
  \HOLOGO@SpaceFactor
}
%    \end{macrocode}
%    \end{macro}
%
% \subsubsection{\Hologo{TTH} (\hologo{TeX} to HTML translator)}
%
%    Source: \url{http://hutchinson.belmont.ma.us/tth/}
%    In the HTML source the second `T' is printed as subscript.
%\begin{quote}
%\begin{verbatim}
%T<sub>T</sub>H
%\end{verbatim}
%\end{quote}
%    \begin{macro}{\HoLogo@TTH}
%    \begin{macrocode}
\def\HoLogo@TTH#1{%
  \ltx@mbox{%
    T\HOLOGO@SubScript{T}H%
  }%
  \HOLOGO@SpaceFactor
}
%    \end{macrocode}
%    \end{macro}
%
%    \begin{macro}{\HoLogoHtml@TTH}
%    \begin{macrocode}
\def\HoLogoHtml@TTH#1{%
  T\HCode{<sub>}T\HCode{</sub>}H%
}
%    \end{macrocode}
%    \end{macro}
%
% \subsubsection{\Hologo{HanTheThanh}}
%
%    Partial source: Package \xpackage{dtklogos}.
%    The double accent is U+1EBF (latin small letter e with circumflex
%    and acute).
%    \begin{macro}{\HoLogo@HanTheThanh}
%    \begin{macrocode}
\def\HoLogo@HanTheThanh#1{%
  \ltx@mbox{H\`an}%
  \HOLOGO@space
  \ltx@mbox{%
    Th%
    \HOLOGO@IfCharExists{"1EBF}{%
      \char"1EBF\relax
    }{%
      \^e\hbox to 0pt{\hss\raise .5ex\hbox{\'{}}}%
    }%
  }%
  \HOLOGO@space
  \ltx@mbox{Th\`anh}%
}
%    \end{macrocode}
%    \end{macro}
%    \begin{macro}{\HoLogoBkm@HanTheThanh}
%    \begin{macrocode}
\def\HoLogoBkm@HanTheThanh#1{%
  H\`an %
  Th\HOLOGO@PdfdocUnicode{\^e}{\9036\277} %
  Th\`anh%
}
%    \end{macrocode}
%    \end{macro}
%    \begin{macro}{\HoLogoHtml@HanTheThanh}
%    \begin{macrocode}
\def\HoLogoHtml@HanTheThanh#1{%
  H\`an %
  Th\HCode{&\ltx@hashchar x1ebf;} %
  Th\`anh%
}
%    \end{macrocode}
%    \end{macro}
%
% \subsection{Driver detection}
%
%    \begin{macrocode}
\HOLOGO@IfExists\InputIfFileExists{%
  \InputIfFileExists{hologo.cfg}{}{}%
}{%
  \ltx@IfUndefined{pdf@filesize}{%
    \def\HOLOGO@InputIfExists{%
      \openin\HOLOGO@temp=hologo.cfg\relax
      \ifeof\HOLOGO@temp
        \closein\HOLOGO@temp
      \else
        \closein\HOLOGO@temp
        \begingroup
          \def\x{LaTeX2e}%
        \expandafter\endgroup
        \ifx\fmtname\x
          \input{hologo.cfg}%
        \else
          \input hologo.cfg\relax
        \fi
      \fi
    }%
    \ltx@IfUndefined{newread}{%
      \chardef\HOLOGO@temp=15 %
      \def\HOLOGO@CheckRead{%
        \ifeof\HOLOGO@temp
          \HOLOGO@InputIfExists
        \else
          \ifcase\HOLOGO@temp
            \@PackageWarningNoLine{hologo}{%
              Configuration file ignored, because\MessageBreak
              a free read register could not be found%
            }%
          \else
            \begingroup
              \count\ltx@cclv=\HOLOGO@temp
              \advance\ltx@cclv by \ltx@minusone
              \edef\x{\endgroup
                \chardef\noexpand\HOLOGO@temp=\the\count\ltx@cclv
                \relax
              }%
            \x
          \fi
        \fi
      }%
    }{%
      \csname newread\endcsname\HOLOGO@temp
      \HOLOGO@InputIfExists
    }%
  }{%
    \edef\HOLOGO@temp{\pdf@filesize{hologo.cfg}}%
    \ifx\HOLOGO@temp\ltx@empty
    \else
      \ifnum\HOLOGO@temp>0 %
        \begingroup
          \def\x{LaTeX2e}%
        \expandafter\endgroup
        \ifx\fmtname\x
          \input{hologo.cfg}%
        \else
          \input hologo.cfg\relax
        \fi
      \else
        \@PackageInfoNoLine{hologo}{%
          Empty configuration file `hologo.cfg' ignored%
        }%
      \fi
    \fi
  }%
}
%    \end{macrocode}
%
%    \begin{macrocode}
\def\HOLOGO@temp#1#2{%
  \kv@define@key{HoLogoDriver}{#1}[]{%
    \begingroup
      \def\HOLOGO@temp{##1}%
      \ltx@onelevel@sanitize\HOLOGO@temp
      \ifx\HOLOGO@temp\ltx@empty
      \else
        \@PackageError{hologo}{%
          Value (\HOLOGO@temp) not permitted for option `#1'%
        }%
        \@ehc
      \fi
    \endgroup
    \def\hologoDriver{#2}%
  }%
}%
\def\HOLOGO@@temp#1#2{%
  \ifx\kv@value\relax
    \HOLOGO@temp{#1}{#1}%
  \else
    \HOLOGO@temp{#1}{#2}%
  \fi
}%
\kv@parse@normalized{%
  pdftex,%
  luatex=pdftex,%
  dvipdfm,%
  dvipdfmx=dvipdfm,%
  dvips,%
  dvipsone=dvips,%
  xdvi=dvips,%
  xetex,%
  vtex,%
}\HOLOGO@@temp
%    \end{macrocode}
%
%    \begin{macrocode}
\kv@define@key{HoLogoDriver}{driverfallback}{%
  \def\HOLOGO@DriverFallback{#1}%
}
%    \end{macrocode}
%
%    \begin{macro}{\HOLOGO@DriverFallback}
%    \begin{macrocode}
\def\HOLOGO@DriverFallback{dvips}
%    \end{macrocode}
%    \end{macro}
%
%    \begin{macro}{\hologoDriverSetup}
%    \begin{macrocode}
\def\hologoDriverSetup{%
  \let\hologoDriver\ltx@undefined
  \HOLOGO@DriverSetup
}
%    \end{macrocode}
%    \end{macro}
%
%    \begin{macro}{\HOLOGO@DriverSetup}
%    \begin{macrocode}
\def\HOLOGO@DriverSetup#1{%
  \kvsetkeys{HoLogoDriver}{#1}%
  \HOLOGO@CheckDriver
  \ltx@ifundefined{hologoDriver}{%
    \begingroup
    \edef\x{\endgroup
      \noexpand\kvsetkeys{HoLogoDriver}{\HOLOGO@DriverFallback}%
    }\x
  }{}%
  \@PackageInfoNoLine{hologo}{Using driver `\hologoDriver'}%
}
%    \end{macrocode}
%    \end{macro}
%
%    \begin{macro}{\HOLOGO@CheckDriver}
%    \begin{macrocode}
\def\HOLOGO@CheckDriver{%
  \ifpdf
    \def\hologoDriver{pdftex}%
    \let\HOLOGO@pdfliteral\pdfliteral
    \ifluatex
      \ifx\pdfextension\@undefined\else
        \protected\def\pdfliteral{\pdfextension literal}%
        \let\HOLOGO@pdfliteral\pdfliteral
      \fi
      \ltx@IfUndefined{HOLOGO@pdfliteral}{%
        \ifnum\luatexversion<36 %
        \else
          \begingroup
            \let\HOLOGO@temp\endgroup
            \ifcase0%
                \directlua{%
                  if tex.enableprimitives then %
                    tex.enableprimitives('HOLOGO@', {'pdfliteral'})%
                  else %
                    tex.print('1')%
                  end%
                }%
                \ifx\HOLOGO@pdfliteral\@undefined 1\fi%
                \relax%
              \endgroup
              \let\HOLOGO@temp\relax
              \global\let\HOLOGO@pdfliteral\HOLOGO@pdfliteral
            \fi%
          \HOLOGO@temp
        \fi
      }{}%
    \fi
    \ltx@IfUndefined{HOLOGO@pdfliteral}{%
      \@PackageWarningNoLine{hologo}{%
        Cannot find \string\pdfliteral
      }%
    }{}%
  \else
    \ifxetex
      \def\hologoDriver{xetex}%
    \else
      \ifvtex
        \def\hologoDriver{vtex}%
      \fi
    \fi
  \fi
}
%    \end{macrocode}
%    \end{macro}
%
%    \begin{macro}{\HOLOGO@WarningUnsupportedDriver}
%    \begin{macrocode}
\def\HOLOGO@WarningUnsupportedDriver#1{%
  \@PackageWarningNoLine{hologo}{%
    Logo `#1' needs driver specific macros,\MessageBreak
    but driver `\hologoDriver' is not supported.\MessageBreak
    Use a different driver or\MessageBreak
    load package `graphics' or `pgf'%
  }%
}
%    \end{macrocode}
%    \end{macro}
%
% \subsubsection{Reflect box macros}
%
%    Skip driver part if not needed.
%    \begin{macrocode}
\ltx@IfUndefined{reflectbox}{}{%
  \ltx@IfUndefined{rotatebox}{}{%
    \HOLOGO@AtEnd
  }%
}
\ltx@IfUndefined{pgftext}{}{%
  \HOLOGO@AtEnd
}
\ltx@IfUndefined{psscalebox}{}{%
  \HOLOGO@AtEnd
}
%    \end{macrocode}
%
%    \begin{macrocode}
\def\HOLOGO@temp{LaTeX2e}
\ifx\fmtname\HOLOGO@temp
  \RequirePackage{kvoptions}[2011/06/30]%
  \ProcessKeyvalOptions{HoLogoDriver}%
\fi
\HOLOGO@DriverSetup{}
%    \end{macrocode}
%
%    \begin{macro}{\HOLOGO@ReflectBox}
%    \begin{macrocode}
\def\HOLOGO@ReflectBox#1{%
  \begingroup
    \setbox\ltx@zero\hbox{\begingroup#1\endgroup}%
    \setbox\ltx@two\hbox{%
      \kern\wd\ltx@zero
      \csname HOLOGO@ScaleBox@\hologoDriver\endcsname{-1}{1}{%
        \hbox to 0pt{\copy\ltx@zero\hss}%
      }%
    }%
    \wd\ltx@two=\wd\ltx@zero
    \box\ltx@two
  \endgroup
}
%    \end{macrocode}
%    \end{macro}
%
%    \begin{macro}{\HOLOGO@PointReflectBox}
%    \begin{macrocode}
\def\HOLOGO@PointReflectBox#1{%
  \begingroup
    \setbox\ltx@zero\hbox{\begingroup#1\endgroup}%
    \setbox\ltx@two\hbox{%
      \kern\wd\ltx@zero
      \raise\ht\ltx@zero\hbox{%
        \csname HOLOGO@ScaleBox@\hologoDriver\endcsname{-1}{-1}{%
          \hbox to 0pt{\copy\ltx@zero\hss}%
        }%
      }%
    }%
    \wd\ltx@two=\wd\ltx@zero
    \box\ltx@two
  \endgroup
}
%    \end{macrocode}
%    \end{macro}
%
%    We must define all variants because of dynamic driver setup.
%    \begin{macrocode}
\def\HOLOGO@temp#1#2{#2}
%    \end{macrocode}
%
%    \begin{macro}{\HOLOGO@ScaleBox@pdftex}
%    \begin{macrocode}
\HOLOGO@temp{pdftex}{%
  \def\HOLOGO@ScaleBox@pdftex#1#2#3{%
    \HOLOGO@pdfliteral{%
      q #1 0 0 #2 0 0 cm%
    }%
    #3%
    \HOLOGO@pdfliteral{%
      Q%
    }%
  }%
}
%    \end{macrocode}
%    \end{macro}
%    \begin{macro}{\HOLOGO@ScaleBox@dvips}
%    \begin{macrocode}
\HOLOGO@temp{dvips}{%
  \def\HOLOGO@ScaleBox@dvips#1#2#3{%
    \special{ps:%
      gsave %
      currentpoint %
      currentpoint translate %
      #1 #2 scale %
      neg exch neg exch translate%
    }%
    #3%
    \special{ps:%
      currentpoint %
      grestore %
      moveto%
    }%
  }%
}
%    \end{macrocode}
%    \end{macro}
%    \begin{macro}{\HOLOGO@ScaleBox@dvipdfm}
%    \begin{macrocode}
\HOLOGO@temp{dvipdfm}{%
  \let\HOLOGO@ScaleBox@dvipdfm\HOLOGO@ScaleBox@dvips
}
%    \end{macrocode}
%    \end{macro}
%    Since \hologo{XeTeX} v0.6.
%    \begin{macro}{\HOLOGO@ScaleBox@xetex}
%    \begin{macrocode}
\HOLOGO@temp{xetex}{%
  \def\HOLOGO@ScaleBox@xetex#1#2#3{%
    \special{x:gsave}%
    \special{x:scale #1 #2}%
    #3%
    \special{x:grestore}%
  }%
}
%    \end{macrocode}
%    \end{macro}
%    \begin{macro}{\HOLOGO@ScaleBox@vtex}
%    \begin{macrocode}
\HOLOGO@temp{vtex}{%
  \def\HOLOGO@ScaleBox@vtex#1#2#3{%
    \special{r(#1,0,0,#2,0,0}%
    #3%
    \special{r)}%
  }%
}
%    \end{macrocode}
%    \end{macro}
%
%    \begin{macrocode}
\HOLOGO@AtEnd%
%</package>
%    \end{macrocode}
%
% \section{Test}
%
% \subsection{Catcode checks for loading}
%
%    \begin{macrocode}
%<*test1>
%    \end{macrocode}
%    \begin{macrocode}
\catcode`\{=1 %
\catcode`\}=2 %
\catcode`\#=6 %
\catcode`\@=11 %
\expandafter\ifx\csname count@\endcsname\relax
  \countdef\count@=255 %
\fi
\expandafter\ifx\csname @gobble\endcsname\relax
  \long\def\@gobble#1{}%
\fi
\expandafter\ifx\csname @firstofone\endcsname\relax
  \long\def\@firstofone#1{#1}%
\fi
\expandafter\ifx\csname loop\endcsname\relax
  \expandafter\@firstofone
\else
  \expandafter\@gobble
\fi
{%
  \def\loop#1\repeat{%
    \def\body{#1}%
    \iterate
  }%
  \def\iterate{%
    \body
      \let\next\iterate
    \else
      \let\next\relax
    \fi
    \next
  }%
  \let\repeat=\fi
}%
\def\RestoreCatcodes{}
\count@=0 %
\loop
  \edef\RestoreCatcodes{%
    \RestoreCatcodes
    \catcode\the\count@=\the\catcode\count@\relax
  }%
\ifnum\count@<255 %
  \advance\count@ 1 %
\repeat

\def\RangeCatcodeInvalid#1#2{%
  \count@=#1\relax
  \loop
    \catcode\count@=15 %
  \ifnum\count@<#2\relax
    \advance\count@ 1 %
  \repeat
}
\def\RangeCatcodeCheck#1#2#3{%
  \count@=#1\relax
  \loop
    \ifnum#3=\catcode\count@
    \else
      \errmessage{%
        Character \the\count@\space
        with wrong catcode \the\catcode\count@\space
        instead of \number#3%
      }%
    \fi
  \ifnum\count@<#2\relax
    \advance\count@ 1 %
  \repeat
}
\def\space{ }
\expandafter\ifx\csname LoadCommand\endcsname\relax
  \def\LoadCommand{\input hologo.sty\relax}%
\fi
\def\Test{%
  \RangeCatcodeInvalid{0}{47}%
  \RangeCatcodeInvalid{58}{64}%
  \RangeCatcodeInvalid{91}{96}%
  \RangeCatcodeInvalid{123}{255}%
  \catcode`\@=12 %
  \catcode`\\=0 %
  \catcode`\%=14 %
  \LoadCommand
  \RangeCatcodeCheck{0}{36}{15}%
  \RangeCatcodeCheck{37}{37}{14}%
  \RangeCatcodeCheck{38}{47}{15}%
  \RangeCatcodeCheck{48}{57}{12}%
  \RangeCatcodeCheck{58}{63}{15}%
  \RangeCatcodeCheck{64}{64}{12}%
  \RangeCatcodeCheck{65}{90}{11}%
  \RangeCatcodeCheck{91}{91}{15}%
  \RangeCatcodeCheck{92}{92}{0}%
  \RangeCatcodeCheck{93}{96}{15}%
  \RangeCatcodeCheck{97}{122}{11}%
  \RangeCatcodeCheck{123}{255}{15}%
  \RestoreCatcodes
}
\Test
\csname @@end\endcsname
\end
%    \end{macrocode}
%    \begin{macrocode}
%</test1>
%    \end{macrocode}
%
% \subsection{Spacefactor}
%
%    The space factor must be 1000 after a logo. If it is greater 1000
%    then the following space is a space after a sentence closing point.
%    If the space factor is smaller 1000 then an immediate following
%    dot is interpreted as abbreviation, not sentence closing point.
%
%    \begin{macrocode}
%<*test-spacefactor>
\NeedsTeXFormat{LaTeX2e}
\documentclass{article}
\usepackage{hologo}[2016/05/12]
\usepackage{kvsetkeys}
\usepackage{qstest}
\IncludeTests{*}
\LogTests{log}{*}{*}
\begin{document}
\begin{qstest}{spacefactor}{spacefactor}
\newcommand*{\Test}[1]{%
  \sbox0{%
    \hologo{#1}%
    \Expect*{1000 (#1)}*{\the\spacefactor\space(#1)}%
  }%
}%
\makeatletter
\def\TestList{}
\def\hologoEntry#1#2#3{%
  \edef\TestList{%
    \ifx\TestList\@empty
    \else
      \TestList,%
    \fi
    #1%
    \ifx\\#2\\%
    \else
      ={variant=#2}%
    \fi
  }%
}
\hologoList
\expandafter\kv@parse@normalized\expandafter{%
  \TestList
}{%
  \begingroup
    \let\@logo=\kv@key
    \ifx\kv@value\relax
    \else
      \expandafter\hologoLogoSetup\expandafter\@logo\expandafter{%
        \kv@value
      }%
    \fi
    \Test\@logo
  \endgroup
  \@gobbletwo
}
\end{qstest}
\end{document}
%</test-spacefactor>
%    \end{macrocode}
%
% \subsection{Complete list}
%
%    \begin{macrocode}
%<*test-list>
\NeedsTeXFormat{LaTeX2e}
\documentclass[12pt,a4paper]{article}
\usepackage{hologo}[2016/05/12]
\usepackage[T1]{fontenc}
\usepackage{lmodern}
\usepackage{parskip}
\usepackage[unicode]{hyperref}[2011/09/28]
\usepackage{bookmark}[2011/09/19]
\bookmarksetup{%
  numbered,%
  open,%
  openlevel=2,%
}
\renewcommand*{\contentsname}{List of logos}
\begin{document}
\tableofcontents
\def\TestFont#1#2#3#4#5#6{%
  \begingroup
    \usefont{#3}{#4}{#5}{#6}%
    \HologoVariant{#1}{#2}/\hologoVariant{#1}{#2}%
    \quad
    \begingroup\scriptsize\hologoVariant{#1}{#2}\endgroup
    \quad
  \endgroup
  (#3/#4/#5/#6)%
  \par
}
\makeatletter
\def\hologoEntry#1#2#3{%
  \section{%
    \HologoVariant{#1}{#2}/\hologoVariant{#1}{#2} %
    {[#1\ifx\\#2\\\else\space(#2)\fi]}% hash-ok
  }% braces around [] because of bug in tex4ht
  \begingroup
    \hypersetup{unicode=false}%
    \bookmark[%
      dest=\@currentHref,%
      rellevel=1,%
      keeplevel,%
    ]{%
      \HologoVariant{#1}{#2}/\hologoVariant{#1}{#2} %
      (PDFDocEncoding)%
    }%
  \endgroup
  \TestFont{#1}{#2}{OT1}{cmr}{m}{n}%
  \TestFont{#1}{#2}{OT1}{cmss}{m}{n}%
  \TestFont{#1}{#2}{OT1}{cmr}{b}{n}%
  \TestFont{#1}{#2}{OT1}{cmr}{m}{it}%
  \TestFont{#1}{#2}{OT1}{cmtt}{m}{n}%
  \TestFont{#1}{#2}{T1}{lmr}{m}{n}%
  \TestFont{#1}{#2}{T1}{lmss}{m}{n}%
  \TestFont{#1}{#2}{T1}{lmr}{b}{n}%
  \TestFont{#1}{#2}{T1}{lmr}{m}{it}%
  \TestFont{#1}{#2}{T1}{lmtt}{m}{n}%
  \TestFont{#1}{#2}{T1}{lmvtt}{m}{n}%
  \TestFont{#1}{#2}{T1}{qtm}{m}{n}%
  \TestFont{#1}{#2}{T1}{qhv}{m}{n}%
  \TestFont{#1}{#2}{T1}{qtm}{b}{n}%
  \TestFont{#1}{#2}{T1}{qtm}{m}{it}%
  \TestFont{#1}{#2}{T1}{qcr}{m}{n}%
  \newpage
}
\makeatother
\hologoList
\end{document}
%</test-list>
%    \end{macrocode}
%
% \section{Installation}
%
% \subsection{Download}
%
% \paragraph{Package.} This package is available on
% CTAN\footnote{\url{ftp://ftp.ctan.org/tex-archive/}}:
% \begin{description}
% \item[\CTAN{macros/latex/contrib/oberdiek/hologo.dtx}] The source file.
% \item[\CTAN{macros/latex/contrib/oberdiek/hologo.pdf}] Documentation.
% \end{description}
%
%
% \paragraph{Bundle.} All the packages of the bundle `oberdiek'
% are also available in a TDS compliant ZIP archive. There
% the packages are already unpacked and the documentation files
% are generated. The files and directories obey the TDS standard.
% \begin{description}
% \item[\CTAN{install/macros/latex/contrib/oberdiek.tds.zip}]
% \end{description}
% \emph{TDS} refers to the standard ``A Directory Structure
% for \TeX\ Files'' (\CTAN{tds/tds.pdf}). Directories
% with \xfile{texmf} in their name are usually organized this way.
%
% \subsection{Bundle installation}
%
% \paragraph{Unpacking.} Unpack the \xfile{oberdiek.tds.zip} in the
% TDS tree (also known as \xfile{texmf} tree) of your choice.
% Example (linux):
% \begin{quote}
%   |unzip oberdiek.tds.zip -d ~/texmf|
% \end{quote}
%
% \paragraph{Script installation.}
% Check the directory \xfile{TDS:scripts/oberdiek/} for
% scripts that need further installation steps.
% Package \xpackage{attachfile2} comes with the Perl script
% \xfile{pdfatfi.pl} that should be installed in such a way
% that it can be called as \texttt{pdfatfi}.
% Example (linux):
% \begin{quote}
%   |chmod +x scripts/oberdiek/pdfatfi.pl|\\
%   |cp scripts/oberdiek/pdfatfi.pl /usr/local/bin/|
% \end{quote}
%
% \subsection{Package installation}
%
% \paragraph{Unpacking.} The \xfile{.dtx} file is a self-extracting
% \docstrip\ archive. The files are extracted by running the
% \xfile{.dtx} through \plainTeX:
% \begin{quote}
%   \verb|tex hologo.dtx|
% \end{quote}
%
% \paragraph{TDS.} Now the different files must be moved into
% the different directories in your installation TDS tree
% (also known as \xfile{texmf} tree):
% \begin{quote}
% \def\t{^^A
% \begin{tabular}{@{}>{\ttfamily}l@{ $\rightarrow$ }>{\ttfamily}l@{}}
%   hologo.sty & tex/generic/oberdiek/hologo.sty\\
%   hologo.pdf & doc/latex/oberdiek/hologo.pdf\\
%   example/hologo-example.tex & doc/latex/oberdiek/example/hologo-example.tex\\
%   test/hologo-test1.tex & doc/latex/oberdiek/test/hologo-test1.tex\\
%   test/hologo-test-spacefactor.tex & doc/latex/oberdiek/test/hologo-test-spacefactor.tex\\
%   test/hologo-test-list.tex & doc/latex/oberdiek/test/hologo-test-list.tex\\
%   hologo.dtx & source/latex/oberdiek/hologo.dtx\\
% \end{tabular}^^A
% }^^A
% \sbox0{\t}^^A
% \ifdim\wd0>\linewidth
%   \begingroup
%     \advance\linewidth by\leftmargin
%     \advance\linewidth by\rightmargin
%   \edef\x{\endgroup
%     \def\noexpand\lw{\the\linewidth}^^A
%   }\x
%   \def\lwbox{^^A
%     \leavevmode
%     \hbox to \linewidth{^^A
%       \kern-\leftmargin\relax
%       \hss
%       \usebox0
%       \hss
%       \kern-\rightmargin\relax
%     }^^A
%   }^^A
%   \ifdim\wd0>\lw
%     \sbox0{\small\t}^^A
%     \ifdim\wd0>\linewidth
%       \ifdim\wd0>\lw
%         \sbox0{\footnotesize\t}^^A
%         \ifdim\wd0>\linewidth
%           \ifdim\wd0>\lw
%             \sbox0{\scriptsize\t}^^A
%             \ifdim\wd0>\linewidth
%               \ifdim\wd0>\lw
%                 \sbox0{\tiny\t}^^A
%                 \ifdim\wd0>\linewidth
%                   \lwbox
%                 \else
%                   \usebox0
%                 \fi
%               \else
%                 \lwbox
%               \fi
%             \else
%               \usebox0
%             \fi
%           \else
%             \lwbox
%           \fi
%         \else
%           \usebox0
%         \fi
%       \else
%         \lwbox
%       \fi
%     \else
%       \usebox0
%     \fi
%   \else
%     \lwbox
%   \fi
% \else
%   \usebox0
% \fi
% \end{quote}
% If you have a \xfile{docstrip.cfg} that configures and enables \docstrip's
% TDS installing feature, then some files can already be in the right
% place, see the documentation of \docstrip.
%
% \subsection{Refresh file name databases}
%
% If your \TeX~distribution
% (\teTeX, \mikTeX, \dots) relies on file name databases, you must refresh
% these. For example, \teTeX\ users run \verb|texhash| or
% \verb|mktexlsr|.
%
% \subsection{Some details for the interested}
%
% \paragraph{Attached source.}
%
% The PDF documentation on CTAN also includes the
% \xfile{.dtx} source file. It can be extracted by
% AcrobatReader 6 or higher. Another option is \textsf{pdftk},
% e.g. unpack the file into the current directory:
% \begin{quote}
%   \verb|pdftk hologo.pdf unpack_files output .|
% \end{quote}
%
% \paragraph{Unpacking with \LaTeX.}
% The \xfile{.dtx} chooses its action depending on the format:
% \begin{description}
% \item[\plainTeX:] Run \docstrip\ and extract the files.
% \item[\LaTeX:] Generate the documentation.
% \end{description}
% If you insist on using \LaTeX\ for \docstrip\ (really,
% \docstrip\ does not need \LaTeX), then inform the autodetect routine
% about your intention:
% \begin{quote}
%   \verb|latex \let\install=y\input{hologo.dtx}|
% \end{quote}
% Do not forget to quote the argument according to the demands
% of your shell.
%
% \paragraph{Generating the documentation.}
% You can use both the \xfile{.dtx} or the \xfile{.drv} to generate
% the documentation. The process can be configured by the
% configuration file \xfile{ltxdoc.cfg}. For instance, put this
% line into this file, if you want to have A4 as paper format:
% \begin{quote}
%   \verb|\PassOptionsToClass{a4paper}{article}|
% \end{quote}
% An example follows how to generate the
% documentation with pdf\LaTeX:
% \begin{quote}
%\begin{verbatim}
%pdflatex hologo.dtx
%makeindex -s gind.ist hologo.idx
%pdflatex hologo.dtx
%makeindex -s gind.ist hologo.idx
%pdflatex hologo.dtx
%\end{verbatim}
% \end{quote}
%
% \section{Catalogue}
%
% The following XML file can be used as source for the
% \href{http://mirror.ctan.org/help/Catalogue/catalogue.html}{\TeX\ Catalogue}.
% The elements \texttt{caption} and \texttt{description} are imported
% from the original XML file from the Catalogue.
% The name of the XML file in the Catalogue is \xfile{hologo.xml}.
%    \begin{macrocode}
%<*catalogue>
<?xml version='1.0' encoding='us-ascii'?>
<!DOCTYPE entry SYSTEM 'catalogue.dtd'>
<entry datestamp='$Date$' modifier='$Author$' id='hologo'>
  <name>hologo</name>
  <caption>A collection of logos with bookmark support.</caption>
  <authorref id='auth:oberdiek'/>
  <copyright owner='Heiko Oberdiek' year='2010-2012'/>
  <license type='lppl1.3'/>
  <version number='1.10'/>
  <description>
    The package defines a single command <tt>\hologo</tt>, whose
    argument is the usual case-confused ASCII version of the logo.
    The command is bookmark-enabled, so that every logo becomes
    available in bookmarks without further work.
    <p/>
    The package is part of the <xref refid='oberdiek'>oberdiek</xref>
    bundle.
  </description>
  <documentation details='Package documentation'
      href='ctan:/macros/latex/contrib/oberdiek/hologo.pdf'/>
  <ctan file='true' path='/macros/latex/contrib/oberdiek/hologo.dtx'/>
  <miktex location='oberdiek'/>
  <texlive location='oberdiek'/>
  <install path='/macros/latex/contrib/oberdiek/oberdiek.tds.zip'/>
</entry>
%</catalogue>
%    \end{macrocode}
%
% \begin{thebibliography}{9}
% \raggedright
%
% \bibitem{btxdoc}
% Oren Patashnik,
% \textit{\hologo{BibTeX}ing},
% 1988-02-08.\\
% \CTAN{biblio/bibtex/base/}
%
% \bibitem{dtklogos}
% Gerd Neugebauer, DANTE,
% \textit{Package \xpackage{dtklogos}},
% 2011-04-25.\\
% \CTAN{usergrps/dante/dtk/dtklogos.sty}
%
% \bibitem{etexman}
% The \hologo{NTS} Team,
% \textit{The \hologo{eTeX} manual},
% 1998-02.\\
% \CTAN{systems/e-tex/v2/doc/}
%
% \bibitem{ExTeX-FAQ}
% The \hologo{ExTeX} group,
% \textit{\hologo{ExTeX}: FAQ -- How is \hologo{ExTeX} typeset?},
% 2007-04-14.\\
% \url{http://www.extex.org/documentation/faq.html}
%
% \bibitem{LyX}
% %@MISC{ LyX,
% %  title = {{LyX 2.0.0 -- The Document Processor [Computer software and manual]}},
% %  author = {{The LyX Team}},
% %  howpublished = {Internet: http://www.lyx.org},
% %  year = {2011-05-08},
% %  note = {Retrieved May 10, 2011, from http://www.lyx.org},
% %  url = {http://www.lyx.org/}
% %}
% The \hologo{LyX} Team,
% \textit{\hologo{LyX} -- The Document Processor},
% 2011-05-08.\\
% \url{http://www.lyx.org/}
%
% \bibitem{OzTeX}
% Andrew Trevorrow,
% \hologo{OzTeX} FAQ: What is the correct way to typeset ``\hologo{OzTeX}''?,
% 2011-09-15 (visited).
% \url{http://www.trevorrow.com/oztex/ozfaq.html#oztex-logo}
%
% \bibitem{PiCTeX}
% Michael Wichura,
% \textit{The \hologo{PiCTeX} macro package},
% 1987-09-21.
% \CTAN{graphics/pictex/}
%
% \bibitem{scrlogo}
% Markus Kohm,
% \textit{\hologo{KOMAScript} Datei \xfile{scrlogo.dtx}},
% 2009-01-30.\\
% \CTAN{install/macros/latex/contrib/komascript.tds.zip}
%
% \end{thebibliography}
%
% \begin{History}
%   \begin{Version}{2010/04/08 v1.0}
%   \item
%     The first version.
%   \end{Version}
%   \begin{Version}{2010/04/16 v1.1}
%   \item
%     \cs{Hologo} added for support of logos at start of a sentence.
%   \item
%     \cs{hologoSetup} and \cs{hologoLogoSetup} added.
%   \item
%     Options \xoption{break}, \xoption{hyphenbreak}, \xoption{spacebreak}
%     added.
%   \item
%     Variant support added by option \xoption{variant}.
%   \end{Version}
%   \begin{Version}{2010/04/24 v1.2}
%   \item
%     \hologo{LaTeX3} added.
%   \item
%     \hologo{VTeX} added.
%   \end{Version}
%   \begin{Version}{2010/11/21 v1.3}
%   \item
%     \hologo{iniTeX}, \hologo{virTeX} added.
%   \end{Version}
%   \begin{Version}{2011/03/25 v1.4}
%   \item
%     \hologo{ConTeXt} with variants added.
%   \item
%     Option \xoption{discretionarybreak} added as refinement for
%     option \xoption{break}.
%   \end{Version}
%   \begin{Version}{2011/04/21 v1.5}
%   \item
%     Wrong TDS directory for test files fixed.
%   \end{Version}
%   \begin{Version}{2011/10/01 v1.6}
%   \item
%     Support for package \xpackage{tex4ht} added.
%   \item
%     Support for \cs{csname} added if \cs{ifincsname} is available.
%   \item
%     New logos:
%     \hologo{(La)TeX},
%     \hologo{biber},
%     \hologo{BibTeX} (\xoption{sc}, \xoption{sf}),
%     \hologo{emTeX},
%     \hologo{ExTeX},
%     \hologo{KOMAScript},
%     \hologo{La},
%     \hologo{LyX},
%     \hologo{MiKTeX},
%     \hologo{NTS},
%     \hologo{OzMF},
%     \hologo{OzMP},
%     \hologo{OzTeX},
%     \hologo{OzTtH},
%     \hologo{PCTeX},
%     \hologo{PiC},
%     \hologo{PiCTeX},
%     \hologo{METAFONT},
%     \hologo{MetaFun},
%     \hologo{METAPOST},
%     \hologo{MetaPost},
%     \hologo{SLiTeX} (\xoption{lift}, \xoption{narrow}, \xoption{simple}),
%     \hologo{SliTeX} (\xoption{narrow}, \xoption{simple}, \xoption{lift}),
%     \hologo{teTeX}.
%   \item
%     Fixes:
%     \hologo{iniTeX},
%     \hologo{pdfLaTeX},
%     \hologo{pdfTeX},
%     \hologo{virTeX}.
%   \item
%     \cs{hologoFontSetup} and \cs{hologoLogoFontSetup} added.
%   \item
%     \cs{hologoVariant} and \cs{HologoVariant} added.
%   \end{Version}
%   \begin{Version}{2011/11/22 v1.7}
%   \item
%     New logos:
%     \hologo{BibTeX8},
%     \hologo{LaTeXML},
%     \hologo{SageTeX},
%     \hologo{TeX4ht},
%     \hologo{TTH}.
%   \item
%     \hologo{Xe} and friends: Driver stuff fixed.
%   \item
%     \hologo{Xe} and friends: Support for italic added.
%   \item
%     \hologo{Xe} and friends: Package support for \xpackage{pgf}
%     and \xpackage{pstricks} added.
%   \end{Version}
%   \begin{Version}{2011/11/29 v1.8}
%   \item
%     New logos:
%     \hologo{HanTheThanh}.
%   \end{Version}
%   \begin{Version}{2011/12/21 v1.9}
%   \item
%     Patch for package \xpackage{ifxetex} added for the case that
%     \cs{newif} is undefined in \hologo{iniTeX}.
%   \item
%     Some fixes for \hologo{iniTeX}.
%   \end{Version}
%   \begin{Version}{2012/04/26 v1.10}
%   \item
%     Fix in bookmark version of logo ``\hologo{HanTheThanh}''.
%   \end{Version}
%   \begin{Version}{2016/05/12 v1.11}
%   \item
%     Update HOLOGO@IfCharExists (previously in texlive)
%   \item define pdfliteral in current luatex.
%   \end{Version}
% \end{History}
%
% \PrintIndex
%
% \Finale
\endinput
|
% \end{quote}
% Do not forget to quote the argument according to the demands
% of your shell.
%
% \paragraph{Generating the documentation.}
% You can use both the \xfile{.dtx} or the \xfile{.drv} to generate
% the documentation. The process can be configured by the
% configuration file \xfile{ltxdoc.cfg}. For instance, put this
% line into this file, if you want to have A4 as paper format:
% \begin{quote}
%   \verb|\PassOptionsToClass{a4paper}{article}|
% \end{quote}
% An example follows how to generate the
% documentation with pdf\LaTeX:
% \begin{quote}
%\begin{verbatim}
%pdflatex hologo.dtx
%makeindex -s gind.ist hologo.idx
%pdflatex hologo.dtx
%makeindex -s gind.ist hologo.idx
%pdflatex hologo.dtx
%\end{verbatim}
% \end{quote}
%
% \section{Catalogue}
%
% The following XML file can be used as source for the
% \href{http://mirror.ctan.org/help/Catalogue/catalogue.html}{\TeX\ Catalogue}.
% The elements \texttt{caption} and \texttt{description} are imported
% from the original XML file from the Catalogue.
% The name of the XML file in the Catalogue is \xfile{hologo.xml}.
%    \begin{macrocode}
%<*catalogue>
<?xml version='1.0' encoding='us-ascii'?>
<!DOCTYPE entry SYSTEM 'catalogue.dtd'>
<entry datestamp='$Date$' modifier='$Author$' id='hologo'>
  <name>hologo</name>
  <caption>A collection of logos with bookmark support.</caption>
  <authorref id='auth:oberdiek'/>
  <copyright owner='Heiko Oberdiek' year='2010-2012'/>
  <license type='lppl1.3'/>
  <version number='1.10'/>
  <description>
    The package defines a single command <tt>\hologo</tt>, whose
    argument is the usual case-confused ASCII version of the logo.
    The command is bookmark-enabled, so that every logo becomes
    available in bookmarks without further work.
    <p/>
    The package is part of the <xref refid='oberdiek'>oberdiek</xref>
    bundle.
  </description>
  <documentation details='Package documentation'
      href='ctan:/macros/latex/contrib/oberdiek/hologo.pdf'/>
  <ctan file='true' path='/macros/latex/contrib/oberdiek/hologo.dtx'/>
  <miktex location='oberdiek'/>
  <texlive location='oberdiek'/>
  <install path='/macros/latex/contrib/oberdiek/oberdiek.tds.zip'/>
</entry>
%</catalogue>
%    \end{macrocode}
%
% \begin{thebibliography}{9}
% \raggedright
%
% \bibitem{btxdoc}
% Oren Patashnik,
% \textit{\hologo{BibTeX}ing},
% 1988-02-08.\\
% \CTAN{biblio/bibtex/base/}
%
% \bibitem{dtklogos}
% Gerd Neugebauer, DANTE,
% \textit{Package \xpackage{dtklogos}},
% 2011-04-25.\\
% \CTAN{usergrps/dante/dtk/dtklogos.sty}
%
% \bibitem{etexman}
% The \hologo{NTS} Team,
% \textit{The \hologo{eTeX} manual},
% 1998-02.\\
% \CTAN{systems/e-tex/v2/doc/}
%
% \bibitem{ExTeX-FAQ}
% The \hologo{ExTeX} group,
% \textit{\hologo{ExTeX}: FAQ -- How is \hologo{ExTeX} typeset?},
% 2007-04-14.\\
% \url{http://www.extex.org/documentation/faq.html}
%
% \bibitem{LyX}
% %@MISC{ LyX,
% %  title = {{LyX 2.0.0 -- The Document Processor [Computer software and manual]}},
% %  author = {{The LyX Team}},
% %  howpublished = {Internet: http://www.lyx.org},
% %  year = {2011-05-08},
% %  note = {Retrieved May 10, 2011, from http://www.lyx.org},
% %  url = {http://www.lyx.org/}
% %}
% The \hologo{LyX} Team,
% \textit{\hologo{LyX} -- The Document Processor},
% 2011-05-08.\\
% \url{http://www.lyx.org/}
%
% \bibitem{OzTeX}
% Andrew Trevorrow,
% \hologo{OzTeX} FAQ: What is the correct way to typeset ``\hologo{OzTeX}''?,
% 2011-09-15 (visited).
% \url{http://www.trevorrow.com/oztex/ozfaq.html#oztex-logo}
%
% \bibitem{PiCTeX}
% Michael Wichura,
% \textit{The \hologo{PiCTeX} macro package},
% 1987-09-21.
% \CTAN{graphics/pictex/}
%
% \bibitem{scrlogo}
% Markus Kohm,
% \textit{\hologo{KOMAScript} Datei \xfile{scrlogo.dtx}},
% 2009-01-30.\\
% \CTAN{install/macros/latex/contrib/komascript.tds.zip}
%
% \end{thebibliography}
%
% \begin{History}
%   \begin{Version}{2010/04/08 v1.0}
%   \item
%     The first version.
%   \end{Version}
%   \begin{Version}{2010/04/16 v1.1}
%   \item
%     \cs{Hologo} added for support of logos at start of a sentence.
%   \item
%     \cs{hologoSetup} and \cs{hologoLogoSetup} added.
%   \item
%     Options \xoption{break}, \xoption{hyphenbreak}, \xoption{spacebreak}
%     added.
%   \item
%     Variant support added by option \xoption{variant}.
%   \end{Version}
%   \begin{Version}{2010/04/24 v1.2}
%   \item
%     \hologo{LaTeX3} added.
%   \item
%     \hologo{VTeX} added.
%   \end{Version}
%   \begin{Version}{2010/11/21 v1.3}
%   \item
%     \hologo{iniTeX}, \hologo{virTeX} added.
%   \end{Version}
%   \begin{Version}{2011/03/25 v1.4}
%   \item
%     \hologo{ConTeXt} with variants added.
%   \item
%     Option \xoption{discretionarybreak} added as refinement for
%     option \xoption{break}.
%   \end{Version}
%   \begin{Version}{2011/04/21 v1.5}
%   \item
%     Wrong TDS directory for test files fixed.
%   \end{Version}
%   \begin{Version}{2011/10/01 v1.6}
%   \item
%     Support for package \xpackage{tex4ht} added.
%   \item
%     Support for \cs{csname} added if \cs{ifincsname} is available.
%   \item
%     New logos:
%     \hologo{(La)TeX},
%     \hologo{biber},
%     \hologo{BibTeX} (\xoption{sc}, \xoption{sf}),
%     \hologo{emTeX},
%     \hologo{ExTeX},
%     \hologo{KOMAScript},
%     \hologo{La},
%     \hologo{LyX},
%     \hologo{MiKTeX},
%     \hologo{NTS},
%     \hologo{OzMF},
%     \hologo{OzMP},
%     \hologo{OzTeX},
%     \hologo{OzTtH},
%     \hologo{PCTeX},
%     \hologo{PiC},
%     \hologo{PiCTeX},
%     \hologo{METAFONT},
%     \hologo{MetaFun},
%     \hologo{METAPOST},
%     \hologo{MetaPost},
%     \hologo{SLiTeX} (\xoption{lift}, \xoption{narrow}, \xoption{simple}),
%     \hologo{SliTeX} (\xoption{narrow}, \xoption{simple}, \xoption{lift}),
%     \hologo{teTeX}.
%   \item
%     Fixes:
%     \hologo{iniTeX},
%     \hologo{pdfLaTeX},
%     \hologo{pdfTeX},
%     \hologo{virTeX}.
%   \item
%     \cs{hologoFontSetup} and \cs{hologoLogoFontSetup} added.
%   \item
%     \cs{hologoVariant} and \cs{HologoVariant} added.
%   \end{Version}
%   \begin{Version}{2011/11/22 v1.7}
%   \item
%     New logos:
%     \hologo{BibTeX8},
%     \hologo{LaTeXML},
%     \hologo{SageTeX},
%     \hologo{TeX4ht},
%     \hologo{TTH}.
%   \item
%     \hologo{Xe} and friends: Driver stuff fixed.
%   \item
%     \hologo{Xe} and friends: Support for italic added.
%   \item
%     \hologo{Xe} and friends: Package support for \xpackage{pgf}
%     and \xpackage{pstricks} added.
%   \end{Version}
%   \begin{Version}{2011/11/29 v1.8}
%   \item
%     New logos:
%     \hologo{HanTheThanh}.
%   \end{Version}
%   \begin{Version}{2011/12/21 v1.9}
%   \item
%     Patch for package \xpackage{ifxetex} added for the case that
%     \cs{newif} is undefined in \hologo{iniTeX}.
%   \item
%     Some fixes for \hologo{iniTeX}.
%   \end{Version}
%   \begin{Version}{2012/04/26 v1.10}
%   \item
%     Fix in bookmark version of logo ``\hologo{HanTheThanh}''.
%   \end{Version}
%   \begin{Version}{2016/05/12 v1.11}
%   \item
%     Update HOLOGO@IfCharExists (previously in texlive)
%   \item define pdfliteral in current luatex.
%   \end{Version}
% \end{History}
%
% \PrintIndex
%
% \Finale
\endinput
%
        \else
          \input hologo.cfg\relax
        \fi
      \fi
    }%
    \ltx@IfUndefined{newread}{%
      \chardef\HOLOGO@temp=15 %
      \def\HOLOGO@CheckRead{%
        \ifeof\HOLOGO@temp
          \HOLOGO@InputIfExists
        \else
          \ifcase\HOLOGO@temp
            \@PackageWarningNoLine{hologo}{%
              Configuration file ignored, because\MessageBreak
              a free read register could not be found%
            }%
          \else
            \begingroup
              \count\ltx@cclv=\HOLOGO@temp
              \advance\ltx@cclv by \ltx@minusone
              \edef\x{\endgroup
                \chardef\noexpand\HOLOGO@temp=\the\count\ltx@cclv
                \relax
              }%
            \x
          \fi
        \fi
      }%
    }{%
      \csname newread\endcsname\HOLOGO@temp
      \HOLOGO@InputIfExists
    }%
  }{%
    \edef\HOLOGO@temp{\pdf@filesize{hologo.cfg}}%
    \ifx\HOLOGO@temp\ltx@empty
    \else
      \ifnum\HOLOGO@temp>0 %
        \begingroup
          \def\x{LaTeX2e}%
        \expandafter\endgroup
        \ifx\fmtname\x
          % \iffalse meta-comment
%
% File: hologo.dtx
% Version: 2016/05/12 v1.11
% Info: A logo collection with bookmark support
%
% Copyright (C) 2010-2012 by
%    Heiko Oberdiek <heiko.oberdiek at googlemail.com>
%
% This work may be distributed and/or modified under the
% conditions of the LaTeX Project Public License, either
% version 1.3c of this license or (at your option) any later
% version. This version of this license is in
%    http://www.latex-project.org/lppl/lppl-1-3c.txt
% and the latest version of this license is in
%    http://www.latex-project.org/lppl.txt
% and version 1.3 or later is part of all distributions of
% LaTeX version 2005/12/01 or later.
%
% This work has the LPPL maintenance status "maintained".
%
% This Current Maintainer of this work is Heiko Oberdiek.
%
% The Base Interpreter refers to any `TeX-Format',
% because some files are installed in TDS:tex/generic//.
%
% This work consists of the main source file hologo.dtx
% and the derived files
%    hologo.sty, hologo.pdf, hologo.ins, hologo.drv, hologo-example.tex,
%    hologo-test1.tex, hologo-test-spacefactor.tex,
%    hologo-test-list.tex.
%
% Distribution:
%    CTAN:macros/latex/contrib/oberdiek/hologo.dtx
%    CTAN:macros/latex/contrib/oberdiek/hologo.pdf
%
% Unpacking:
%    (a) If hologo.ins is present:
%           tex hologo.ins
%    (b) Without hologo.ins:
%           tex hologo.dtx
%    (c) If you insist on using LaTeX
%           latex \let\install=y% \iffalse meta-comment
%
% File: hologo.dtx
% Version: 2016/05/12 v1.11
% Info: A logo collection with bookmark support
%
% Copyright (C) 2010-2012 by
%    Heiko Oberdiek <heiko.oberdiek at googlemail.com>
%
% This work may be distributed and/or modified under the
% conditions of the LaTeX Project Public License, either
% version 1.3c of this license or (at your option) any later
% version. This version of this license is in
%    http://www.latex-project.org/lppl/lppl-1-3c.txt
% and the latest version of this license is in
%    http://www.latex-project.org/lppl.txt
% and version 1.3 or later is part of all distributions of
% LaTeX version 2005/12/01 or later.
%
% This work has the LPPL maintenance status "maintained".
%
% This Current Maintainer of this work is Heiko Oberdiek.
%
% The Base Interpreter refers to any `TeX-Format',
% because some files are installed in TDS:tex/generic//.
%
% This work consists of the main source file hologo.dtx
% and the derived files
%    hologo.sty, hologo.pdf, hologo.ins, hologo.drv, hologo-example.tex,
%    hologo-test1.tex, hologo-test-spacefactor.tex,
%    hologo-test-list.tex.
%
% Distribution:
%    CTAN:macros/latex/contrib/oberdiek/hologo.dtx
%    CTAN:macros/latex/contrib/oberdiek/hologo.pdf
%
% Unpacking:
%    (a) If hologo.ins is present:
%           tex hologo.ins
%    (b) Without hologo.ins:
%           tex hologo.dtx
%    (c) If you insist on using LaTeX
%           latex \let\install=y\input{hologo.dtx}
%        (quote the arguments according to the demands of your shell)
%
% Documentation:
%    (a) If hologo.drv is present:
%           latex hologo.drv
%    (b) Without hologo.drv:
%           latex hologo.dtx; ...
%    The class ltxdoc loads the configuration file ltxdoc.cfg
%    if available. Here you can specify further options, e.g.
%    use A4 as paper format:
%       \PassOptionsToClass{a4paper}{article}
%
%    Programm calls to get the documentation (example):
%       pdflatex hologo.dtx
%       makeindex -s gind.ist hologo.idx
%       pdflatex hologo.dtx
%       makeindex -s gind.ist hologo.idx
%       pdflatex hologo.dtx
%
% Installation:
%    TDS:tex/generic/oberdiek/hologo.sty
%    TDS:doc/latex/oberdiek/hologo.pdf
%    TDS:doc/latex/oberdiek/example/hologo-example.tex
%    TDS:doc/latex/oberdiek/test/hologo-test1.tex
%    TDS:doc/latex/oberdiek/test/hologo-test-spacefactor.tex
%    TDS:doc/latex/oberdiek/test/hologo-test-list.tex
%    TDS:source/latex/oberdiek/hologo.dtx
%
%<*ignore>
\begingroup
  \catcode123=1 %
  \catcode125=2 %
  \def\x{LaTeX2e}%
\expandafter\endgroup
\ifcase 0\ifx\install y1\fi\expandafter
         \ifx\csname processbatchFile\endcsname\relax\else1\fi
         \ifx\fmtname\x\else 1\fi\relax
\else\csname fi\endcsname
%</ignore>
%<*install>
\input docstrip.tex
\Msg{************************************************************************}
\Msg{* Installation}
\Msg{* Package: hologo 2016/05/12 v1.11 A logo collection with bookmark support (HO)}
\Msg{************************************************************************}

\keepsilent
\askforoverwritefalse

\let\MetaPrefix\relax
\preamble

This is a generated file.

Project: hologo
Version: 2016/05/12 v1.11

Copyright (C) 2010-2012 by
   Heiko Oberdiek <heiko.oberdiek at googlemail.com>

This work may be distributed and/or modified under the
conditions of the LaTeX Project Public License, either
version 1.3c of this license or (at your option) any later
version. This version of this license is in
   http://www.latex-project.org/lppl/lppl-1-3c.txt
and the latest version of this license is in
   http://www.latex-project.org/lppl.txt
and version 1.3 or later is part of all distributions of
LaTeX version 2005/12/01 or later.

This work has the LPPL maintenance status "maintained".

This Current Maintainer of this work is Heiko Oberdiek.

The Base Interpreter refers to any `TeX-Format',
because some files are installed in TDS:tex/generic//.

This work consists of the main source file hologo.dtx
and the derived files
   hologo.sty, hologo.pdf, hologo.ins, hologo.drv, hologo-example.tex,
   hologo-test1.tex, hologo-test-spacefactor.tex,
   hologo-test-list.tex.

\endpreamble
\let\MetaPrefix\DoubleperCent

\generate{%
  \file{hologo.ins}{\from{hologo.dtx}{install}}%
  \file{hologo.drv}{\from{hologo.dtx}{driver}}%
  \usedir{tex/generic/oberdiek}%
  \file{hologo.sty}{\from{hologo.dtx}{package}}%
  \usedir{doc/latex/oberdiek/example}%
  \file{hologo-example.tex}{\from{hologo.dtx}{example}}%
  \usedir{doc/latex/oberdiek/test}%
  \file{hologo-test1.tex}{\from{hologo.dtx}{test1}}%
  \file{hologo-test-spacefactor.tex}{\from{hologo.dtx}{test-spacefactor}}%
  \file{hologo-test-list.tex}{\from{hologo.dtx}{test-list}}%
  \nopreamble
  \nopostamble
  \usedir{source/latex/oberdiek/catalogue}%
  \file{hologo.xml}{\from{hologo.dtx}{catalogue}}%
}

\catcode32=13\relax% active space
\let =\space%
\Msg{************************************************************************}
\Msg{*}
\Msg{* To finish the installation you have to move the following}
\Msg{* file into a directory searched by TeX:}
\Msg{*}
\Msg{*     hologo.sty}
\Msg{*}
\Msg{* To produce the documentation run the file `hologo.drv'}
\Msg{* through LaTeX.}
\Msg{*}
\Msg{* Happy TeXing!}
\Msg{*}
\Msg{************************************************************************}

\endbatchfile
%</install>
%<*ignore>
\fi
%</ignore>
%<*driver>
\NeedsTeXFormat{LaTeX2e}
\ProvidesFile{hologo.drv}%
  [2016/05/12 v1.11 A logo collection with bookmark support (HO)]%
\documentclass{ltxdoc}
\usepackage{holtxdoc}[2011/11/22]
\usepackage{hologo}[2016/05/12]
\usepackage{longtable}
\usepackage{array}
\usepackage{paralist}
%\usepackage[T1]{fontenc}
%\usepackage{lmodern}
\begin{document}
  \DocInput{hologo.dtx}%
\end{document}
%</driver>
% \fi
%
%
% \CharacterTable
%  {Upper-case    \A\B\C\D\E\F\G\H\I\J\K\L\M\N\O\P\Q\R\S\T\U\V\W\X\Y\Z
%   Lower-case    \a\b\c\d\e\f\g\h\i\j\k\l\m\n\o\p\q\r\s\t\u\v\w\x\y\z
%   Digits        \0\1\2\3\4\5\6\7\8\9
%   Exclamation   \!     Double quote  \"     Hash (number) \#
%   Dollar        \$     Percent       \%     Ampersand     \&
%   Acute accent  \'     Left paren    \(     Right paren   \)
%   Asterisk      \*     Plus          \+     Comma         \,
%   Minus         \-     Point         \.     Solidus       \/
%   Colon         \:     Semicolon     \;     Less than     \<
%   Equals        \=     Greater than  \>     Question mark \?
%   Commercial at \@     Left bracket  \[     Backslash     \\
%   Right bracket \]     Circumflex    \^     Underscore    \_
%   Grave accent  \`     Left brace    \{     Vertical bar  \|
%   Right brace   \}     Tilde         \~}
%
% \GetFileInfo{hologo.drv}
%
% \title{The \xpackage{hologo} package}
% \date{2016/05/12 v1.11}
% \author{Heiko Oberdiek\\\xemail{heiko.oberdiek at googlemail.com}}
%
% \maketitle
%
% \begin{abstract}
% This package starts a collection of logos with support for bookmarks
% strings.
% \end{abstract}
%
% \tableofcontents
%
% \section{Documentation}
%
% \subsection{Logo macros}
%
% \begin{declcs}{hologo} \M{name}
% \end{declcs}
% Macro \cs{hologo} sets the logo with name \meta{name}.
% The following table shows the supported names.
%
% \begingroup
%   \def\hologoEntry#1#2#3{^^A
%     #1&#2&\hologoLogoSetup{#1}{variant=#2}\hologo{#1}&#3\tabularnewline
%   }
%   \begin{longtable}{>{\ttfamily}l>{\ttfamily}lll}
%     \rmfamily\bfseries{name} & \rmfamily\bfseries variant
%     & \bfseries logo & \bfseries since\\
%     \hline
%     \endhead
%     \hologoList
%   \end{longtable}
% \endgroup
%
% \begin{declcs}{Hologo} \M{name}
% \end{declcs}
% Macro \cs{Hologo} starts the logo \meta{name} with an uppercase
% letter. As an exception small greek letters are not converted
% to uppercase. Examples, see \hologo{eTeX} and \hologo{ExTeX}.
%
% \subsection{Setup macros}
%
% The package does not support package options, but the following
% setup macros can be used to set options.
%
% \begin{declcs}{hologoSetup} \M{key value list}
% \end{declcs}
% Macro \cs{hologoSetup} sets global options.
%
% \begin{declcs}{hologoLogoSetup} \M{logo} \M{key value list}
% \end{declcs}
% Some options can also be used to configure a logo.
% These settings take precedence over global option settings.
%
% \subsection{Options}\label{sec:options}
%
% There are boolean and string options:
% \begin{description}
% \item[Boolean option:]
% It takes |true| or |false|
% as value. If the value is omitted, then |true| is used.
% \item[String option:]
% A value must be given as string. (But the string might be empty.)
% \end{description}
% The following options can be used both in \cs{hologoSetup}
% and \cs{hologoLogoSetup}:
% \begin{description}
% \def\entry#1{\item[\xoption{#1}:]}
% \entry{break}
%   enables or disables line breaks inside the logo. This setting is
%   refined by options \xoption{hyphenbreak}, \xoption{spacebreak}
%   or \xoption{discretionarybreak}.
%   Default is |false|.
% \entry{hyphenbreak}
%   enables or disables the line break right after the hyphen character.
% \entry{spacebreak}
%   enables or disables line breaks at space characters.
% \entry{discretionarybreak}
%   enables or disables line breaks at hyphenation points
%   (inserted by \cs{-}).
% \end{description}
% Macro \cs{hologoLogoSetup} also knows:
% \begin{description}
% \item[\xoption{variant}:]
%   This is a string option. It specifies a variant of a logo that
%   must exist. An empty string selects the package default variant.
% \end{description}
% Example:
% \begin{quote}
%   |\hologoSetup{break=false}|\\
%   |\hologoLogoSetup{plainTeX}{variant=hyphen,hyphenbreak}|\\
%   Then ``plain-\TeX'' contains one break point after the hyphen.
% \end{quote}
%
% \subsection{Driver options}
%
% Sometimes graphical operations are needed to construct some
% glyphs (e.g.\ \hologo{XeTeX}). If package \xpackage{graphics}
% or package \xpackage{pgf} are found, then the macros are taken
% from there. Otherwise the packge defines its own operations
% and therefore needs the driver information. Many drivers are
% detected automatically (\hologo{pdfTeX}/\hologo{LuaTeX}
% in PDF mode, \hologo{XeTeX}, \hologo{VTeX}). These have precedence
% over a driver option. The driver can be given as package option
% or using \cs{hologoDriverSetup}.
% The following list contains the recognized driver options:
% \begin{itemize}
% \item \xoption{pdftex}, \xoption{luatex}
% \item \xoption{dvipdfm}, \xoption{dvipdfmx}
% \item \xoption{dvips}, \xoption{dvipsone}, \xoption{xdvi}
% \item \xoption{xetex}
% \item \xoption{vtex}
% \end{itemize}
% The left driver of a line is the driver name that is used internally.
% The following names are aliases for drivers that use the
% same method. Therefore the entry in the \xext{log} file for
% the used driver prints the internally used driver name.
% \begin{description}
% \item[\xoption{driverfallback}:]
%   This option expects a driver that is used,
%   if the driver could not be detected automatically.
% \end{description}
%
% \begin{declcs}{hologoDriverSetup} \M{driver option}
% \end{declcs}
% The driver can also be configured after package loading
% using \cs{hologoDriverSetup}, also the way for \hologo{plainTeX}
% to setup the driver.
%
% \subsection{Font setup}
%
% Some logos require a special font, but should also be usable by
% \hologo{plainTeX}. Therefore the package provides some ways
% to influence the font settings. The options below
% take font settings as values. Both font commands
% such as \cs{sffamily} and macros that take one argument
% like \cs{textsf} can be used.
%
% \begin{declcs}{hologoFontSetup} \M{key value list}
% \end{declcs}
% Macro \cs{hologoFontSetup} sets the fonts for all logos.
% Supported keys:
% \begin{description}
% \def\entry#1{\item[\xoption{#1}:]}
% \entry{general}
%   This font is used for all logos. The default is empty.
%   That means no special font is used.
% \entry{bibsf}
%   This font is used for
%   {\hologoLogoSetup{BibTeX}{variant=sf}\hologo{BibTeX}}
%   with variant \xoption{sf}.
% \entry{rm}
%   This font is a serif font. It is used for \hologo{ExTeX}.
% \entry{sc}
%   This font specifies a small caps font. It is used for
%   {\hologoLogoSetup{BibTeX}{variant=sc}\hologo{BibTeX}}
%   with variant \xoption{sc}.
% \entry{sf}
%   This font specifies a sans serif font. The default
%   is \cs{sffamily}, then \cs{sf} is tried. Otherwise
%   a warning is given. It is used by \hologo{KOMAScript}.
% \entry{sy}
%   This is the font for math symbols (e.g. cmsy).
%   It is used by \hologo{AmS}, \hologo{NTS}, \hologo{ExTeX}.
% \entry{logo}
%   \hologo{METAFONT} and \hologo{METAPOST} are using that font.
%   In \hologo{LaTeX} \cs{logofamily} is used and
%   the definitions of package \xpackage{mflogo} are used
%   if the package is not loaded.
%   Otherwise the \cs{tenlogo} is used and defined
%   if it does not already exists.
% \end{description}
%
% \begin{declcs}{hologoLogoFontSetup} \M{logo} \M{key value list}
% \end{declcs}
% Fonts can also be set for a logo or logo component separately,
% see the following list.
% The keys are the same as for \cs{hologoFontSetup}.
%
% \begin{longtable}{>{\ttfamily}l>{\sffamily}ll}
%   \meta{logo} & keys & result\\
%   \hline
%   \endhead
%   BibTeX & bibsf & {\hologoLogoSetup{BibTeX}{variant=sf}\hologo{BibTeX}}\\[.5ex]
%   BibTeX & sc & {\hologoLogoSetup{BibTeX}{variant=sc}\hologo{BibTeX}}\\[.5ex]
%   ExTeX & rm & \hologo{ExTeX}\\
%   SliTeX & rm & \hologo{SliTeX}\\[.5ex]
%   AmS & sy & \hologo{AmS}\\
%   ExTeX & sy & \hologo{ExTeX}\\
%   NTS & sy & \hologo{NTS}\\[.5ex]
%   KOMAScript & sf & \hologo{KOMAScript}\\[.5ex]
%   METAFONT & logo & \hologo{METAFONT}\\
%   METAPOST & logo & \hologo{METAPOST}\\[.5ex]
%   SliTeX & sc \hologo{SliTeX}
% \end{longtable}
%
% \subsubsection{Font order}
%
% For all logos the font \xoption{general} is applied first.
% Example:
%\begin{quote}
%|\hologoFontSetup{general=\color{red}}|
%\end{quote}
% will print red logos.
% Then if the font uses a special font \xoption{sf}, for example,
% the font is applied that is setup by \cs{hologoLogoFontSetup}.
% If this font is not setup, then the common font setup
% by \cs{hologoFontSetup} is used. Otherwise a warning is given,
% that there is no font configured.
%
% \subsection{Additional user macros}
%
% Usually a variant of a logo is configured by using
% \cs{hologoLogoSetup}, because it is bad style to mix
% different variants of the same logo in the same text.
% There the following macros are a convenience for testing.
%
% \begin{declcs}{hologoVariant} \M{name} \M{variant}\\
%   \cs{HologoVariant} \M{name} \M{variant}
% \end{declcs}
% Logo \meta{name} is set using \meta{variant} that specifies
% explicitely which variant of the macro is used. If the argument
% is empty, then the default form of the logo is used
% (configurable by \cs{hologoLogoSetup}).
%
% \cs{HologoVariant} is used if the logo is set in a context
% that needs an uppercase first letter (beginning of a sentence, \dots).
%
% \begin{declcs}{hologoList}\\
%   \cs{hologoEntry} \M{logo} \M{variant} \M{since}
% \end{declcs}
% Macro \cs{hologoList} contains all logos that are provided
% by the package including variants. The list consists of calls
% of \cs{hologoEntry} with three arguments starting with the
% logo name \meta{logo} and its variant \meta{variant}. An empty
% variant means the current default. Argument \meta{since} specifies
% with version of the package \xpackage{hologo} is needed to get
% the logo. If the logo is fixed, then the date gets updated.
% Therefore the date \meta{since} is not exactly the date of
% the first introduction, but rather the date of the latest fix.
%
% Before \cs{hologoList} can be used, macro \cs{hologoEntry} needs
% a definition. The example file in section \ref{sec:example}
% shows applications of \cs{hologoList}.
%
% \subsection{Supported contexts}
%
% Macros \cs{hologo} and friends support special contexts:
% \begin{itemize}
% \item \hologo{LaTeX}'s protection mechanism.
% \item Bookmarks of package \xpackage{hyperref}.
% \item Package \xpackage{tex4ht}.
% \item The macros can be used inside \cs{csname} constructs,
%   if \cs{ifincsname} is available (\hologo{pdfTeX}, \hologo{XeTeX},
%   \hologo{LuaTeX}).
% \end{itemize}
%
% \subsection{Example}
% \label{sec:example}
%
% The following example prints the logos in different fonts.
%    \begin{macrocode}
%<*example>
%<<verbatim
\NeedsTeXFormat{LaTeX2e}
\documentclass[a4paper]{article}
\usepackage[
  hmargin=20mm,
  vmargin=20mm,
]{geometry}
\pagestyle{empty}
\usepackage{hologo}[2016/05/12]
\usepackage{longtable}
\usepackage{array}
\setlength{\extrarowheight}{2pt}
\usepackage[T1]{fontenc}
\usepackage{lmodern}
\usepackage{pdflscape}
\usepackage[
  pdfencoding=auto,
]{hyperref}
\hypersetup{
  pdfauthor={Heiko Oberdiek},
  pdftitle={Example for package `hologo'},
  pdfsubject={Logos with fonts lmr, lmss, qtm, qpl, qhv},
}
\usepackage{bookmark}

% Print the logo list on the console

\begingroup
  \typeout{}%
  \typeout{*** Begin of logo list ***}%
  \newcommand*{\hologoEntry}[3]{%
    \typeout{#1 \ifx\\#2\\\else(#2) \fi[#3]}%
  }%
  \hologoList
  \typeout{*** End of logo list ***}%
  \typeout{}%
\endgroup

\begin{document}
\begin{landscape}

  \section{Example file for package `hologo'}

  % Table for font names

  \begin{longtable}{>{\bfseries}ll}
    \textbf{font} & \textbf{Font name}\\
    \hline
    lmr & Latin Modern Roman\\
    lmss & Latin Modern Sans\\
    qtm & \TeX\ Gyre Termes\\
    qhv & \TeX\ Gyre Heros\\
    qpl & \TeX\ Gyre Pagella\\
  \end{longtable}

  % Logo list with logos in different fonts

  \begingroup
    \newcommand*{\SetVariant}[2]{%
      \ifx\\#2\\%
      \else
        \hologoLogoSetup{#1}{variant=#2}%
      \fi
    }%
    \newcommand*{\hologoEntry}[3]{%
      \SetVariant{#1}{#2}%
      \raisebox{1em}[0pt][0pt]{\hypertarget{#1@#2}{}}%
      \bookmark[%
        dest={#1@#2},%
      ]{%
        #1\ifx\\#2\\\else\space(#2)\fi: \Hologo{#1}, \hologo{#1} %
        [Unicode]%
      }%
      \hypersetup{unicode=false}%
      \bookmark[%
        dest={#1@#2},%
      ]{%
        #1\ifx\\#2\\\else\space(#2)\fi: \Hologo{#1}, \hologo{#1} %
        [PDFDocEncoding]%
      }%
      \texttt{#1}%
      &%
      \texttt{#2}%
      &%
      \Hologo{#1}%
      &%
      \SetVariant{#1}{#2}%
      \hologo{#1}%
      &%
      \SetVariant{#1}{#2}%
      \fontfamily{qtm}\selectfont
      \hologo{#1}%
      &%
      \SetVariant{#1}{#2}%
      \fontfamily{qpl}\selectfont
      \hologo{#1}%
      &%
      \SetVariant{#1}{#2}%
      \textsf{\hologo{#1}}%
      &%
      \SetVariant{#1}{#2}%
      \fontfamily{qhv}\selectfont
      \hologo{#1}%
      \tabularnewline
    }%
    \begin{longtable}{llllllll}%
      \textbf{\textit{logo}} & \textbf{\textit{variant}} &
      \texttt{\string\Hologo} &
      \textbf{lmr} & \textbf{qtm} & \textbf{qpl} &
      \textbf{lmss} & \textbf{qhv}
      \tabularnewline
      \hline
      \endhead
      \hologoList
    \end{longtable}%
  \endgroup

\end{landscape}
\end{document}
%verbatim
%</example>
%    \end{macrocode}
%
% \StopEventually{
% }
%
% \section{Implementation}
%    \begin{macrocode}
%<*package>
%    \end{macrocode}
%    Reload check, especially if the package is not used with \LaTeX.
%    \begin{macrocode}
\begingroup\catcode61\catcode48\catcode32=10\relax%
  \catcode13=5 % ^^M
  \endlinechar=13 %
  \catcode35=6 % #
  \catcode39=12 % '
  \catcode44=12 % ,
  \catcode45=12 % -
  \catcode46=12 % .
  \catcode58=12 % :
  \catcode64=11 % @
  \catcode123=1 % {
  \catcode125=2 % }
  \expandafter\let\expandafter\x\csname ver@hologo.sty\endcsname
  \ifx\x\relax % plain-TeX, first loading
  \else
    \def\empty{}%
    \ifx\x\empty % LaTeX, first loading,
      % variable is initialized, but \ProvidesPackage not yet seen
    \else
      \expandafter\ifx\csname PackageInfo\endcsname\relax
        \def\x#1#2{%
          \immediate\write-1{Package #1 Info: #2.}%
        }%
      \else
        \def\x#1#2{\PackageInfo{#1}{#2, stopped}}%
      \fi
      \x{hologo}{The package is already loaded}%
      \aftergroup\endinput
    \fi
  \fi
\endgroup%
%    \end{macrocode}
%    Package identification:
%    \begin{macrocode}
\begingroup\catcode61\catcode48\catcode32=10\relax%
  \catcode13=5 % ^^M
  \endlinechar=13 %
  \catcode35=6 % #
  \catcode39=12 % '
  \catcode40=12 % (
  \catcode41=12 % )
  \catcode44=12 % ,
  \catcode45=12 % -
  \catcode46=12 % .
  \catcode47=12 % /
  \catcode58=12 % :
  \catcode64=11 % @
  \catcode91=12 % [
  \catcode93=12 % ]
  \catcode123=1 % {
  \catcode125=2 % }
  \expandafter\ifx\csname ProvidesPackage\endcsname\relax
    \def\x#1#2#3[#4]{\endgroup
      \immediate\write-1{Package: #3 #4}%
      \xdef#1{#4}%
    }%
  \else
    \def\x#1#2[#3]{\endgroup
      #2[{#3}]%
      \ifx#1\@undefined
        \xdef#1{#3}%
      \fi
      \ifx#1\relax
        \xdef#1{#3}%
      \fi
    }%
  \fi
\expandafter\x\csname ver@hologo.sty\endcsname
\ProvidesPackage{hologo}%
  [2016/05/12 v1.11 A logo collection with bookmark support (HO)]%
%    \end{macrocode}
%
%    \begin{macrocode}
\begingroup\catcode61\catcode48\catcode32=10\relax%
  \catcode13=5 % ^^M
  \endlinechar=13 %
  \catcode123=1 % {
  \catcode125=2 % }
  \catcode64=11 % @
  \def\x{\endgroup
    \expandafter\edef\csname HOLOGO@AtEnd\endcsname{%
      \endlinechar=\the\endlinechar\relax
      \catcode13=\the\catcode13\relax
      \catcode32=\the\catcode32\relax
      \catcode35=\the\catcode35\relax
      \catcode61=\the\catcode61\relax
      \catcode64=\the\catcode64\relax
      \catcode123=\the\catcode123\relax
      \catcode125=\the\catcode125\relax
    }%
  }%
\x\catcode61\catcode48\catcode32=10\relax%
\catcode13=5 % ^^M
\endlinechar=13 %
\catcode35=6 % #
\catcode64=11 % @
\catcode123=1 % {
\catcode125=2 % }
\def\TMP@EnsureCode#1#2{%
  \edef\HOLOGO@AtEnd{%
    \HOLOGO@AtEnd
    \catcode#1=\the\catcode#1\relax
  }%
  \catcode#1=#2\relax
}
\TMP@EnsureCode{10}{12}% ^^J
\TMP@EnsureCode{33}{12}% !
\TMP@EnsureCode{34}{12}% "
\TMP@EnsureCode{36}{3}% $
\TMP@EnsureCode{38}{4}% &
\TMP@EnsureCode{39}{12}% '
\TMP@EnsureCode{40}{12}% (
\TMP@EnsureCode{41}{12}% )
\TMP@EnsureCode{42}{12}% *
\TMP@EnsureCode{43}{12}% +
\TMP@EnsureCode{44}{12}% ,
\TMP@EnsureCode{45}{12}% -
\TMP@EnsureCode{46}{12}% .
\TMP@EnsureCode{47}{12}% /
\TMP@EnsureCode{58}{12}% :
\TMP@EnsureCode{59}{12}% ;
\TMP@EnsureCode{60}{12}% <
\TMP@EnsureCode{62}{12}% >
\TMP@EnsureCode{63}{12}% ?
\TMP@EnsureCode{91}{12}% [
\TMP@EnsureCode{93}{12}% ]
\TMP@EnsureCode{94}{7}% ^ (superscript)
\TMP@EnsureCode{95}{8}% _ (subscript)
\TMP@EnsureCode{96}{12}% `
\TMP@EnsureCode{124}{12}% |
\edef\HOLOGO@AtEnd{%
  \HOLOGO@AtEnd
  \escapechar\the\escapechar\relax
  \noexpand\endinput
}
\escapechar=92 %
%    \end{macrocode}
%
% \subsection{Logo list}
%
%    \begin{macro}{\hologoList}
%    \begin{macrocode}
\def\hologoList{%
  \hologoEntry{(La)TeX}{}{2011/10/01}%
  \hologoEntry{AmSLaTeX}{}{2010/04/16}%
  \hologoEntry{AmSTeX}{}{2010/04/16}%
  \hologoEntry{biber}{}{2011/10/01}%
  \hologoEntry{BibTeX}{}{2011/10/01}%
  \hologoEntry{BibTeX}{sf}{2011/10/01}%
  \hologoEntry{BibTeX}{sc}{2011/10/01}%
  \hologoEntry{BibTeX8}{}{2011/11/22}%
  \hologoEntry{ConTeXt}{}{2011/03/25}%
  \hologoEntry{ConTeXt}{narrow}{2011/03/25}%
  \hologoEntry{ConTeXt}{simple}{2011/03/25}%
  \hologoEntry{emTeX}{}{2010/04/26}%
  \hologoEntry{eTeX}{}{2010/04/08}%
  \hologoEntry{ExTeX}{}{2011/10/01}%
  \hologoEntry{HanTheThanh}{}{2011/11/29}%
  \hologoEntry{iniTeX}{}{2011/10/01}%
  \hologoEntry{KOMAScript}{}{2011/10/01}%
  \hologoEntry{La}{}{2010/05/08}%
  \hologoEntry{LaTeX}{}{2010/04/08}%
  \hologoEntry{LaTeX2e}{}{2010/04/08}%
  \hologoEntry{LaTeX3}{}{2010/04/24}%
  \hologoEntry{LaTeXe}{}{2010/04/08}%
  \hologoEntry{LaTeXML}{}{2011/11/22}%
  \hologoEntry{LaTeXTeX}{}{2011/10/01}%
  \hologoEntry{LuaLaTeX}{}{2010/04/08}%
  \hologoEntry{LuaTeX}{}{2010/04/08}%
  \hologoEntry{LyX}{}{2011/10/01}%
  \hologoEntry{METAFONT}{}{2011/10/01}%
  \hologoEntry{MetaFun}{}{2011/10/01}%
  \hologoEntry{METAPOST}{}{2011/10/01}%
  \hologoEntry{MetaPost}{}{2011/10/01}%
  \hologoEntry{MiKTeX}{}{2011/10/01}%
  \hologoEntry{NTS}{}{2011/10/01}%
  \hologoEntry{OzMF}{}{2011/10/01}%
  \hologoEntry{OzMP}{}{2011/10/01}%
  \hologoEntry{OzTeX}{}{2011/10/01}%
  \hologoEntry{OzTtH}{}{2011/10/01}%
  \hologoEntry{PCTeX}{}{2011/10/01}%
  \hologoEntry{pdfTeX}{}{2011/10/01}%
  \hologoEntry{pdfLaTeX}{}{2011/10/01}%
  \hologoEntry{PiC}{}{2011/10/01}%
  \hologoEntry{PiCTeX}{}{2011/10/01}%
  \hologoEntry{plainTeX}{}{2010/04/08}%
  \hologoEntry{plainTeX}{space}{2010/04/16}%
  \hologoEntry{plainTeX}{hyphen}{2010/04/16}%
  \hologoEntry{plainTeX}{runtogether}{2010/04/16}%
  \hologoEntry{SageTeX}{}{2011/11/22}%
  \hologoEntry{SLiTeX}{}{2011/10/01}%
  \hologoEntry{SLiTeX}{lift}{2011/10/01}%
  \hologoEntry{SLiTeX}{narrow}{2011/10/01}%
  \hologoEntry{SLiTeX}{simple}{2011/10/01}%
  \hologoEntry{SliTeX}{}{2011/10/01}%
  \hologoEntry{SliTeX}{narrow}{2011/10/01}%
  \hologoEntry{SliTeX}{simple}{2011/10/01}%
  \hologoEntry{SliTeX}{lift}{2011/10/01}%
  \hologoEntry{teTeX}{}{2011/10/01}%
  \hologoEntry{TeX}{}{2010/04/08}%
  \hologoEntry{TeX4ht}{}{2011/11/22}%
  \hologoEntry{TTH}{}{2011/11/22}%
  \hologoEntry{virTeX}{}{2011/10/01}%
  \hologoEntry{VTeX}{}{2010/04/24}%
  \hologoEntry{Xe}{}{2010/04/08}%
  \hologoEntry{XeLaTeX}{}{2010/04/08}%
  \hologoEntry{XeTeX}{}{2010/04/08}%
}
%    \end{macrocode}
%    \end{macro}
%
% \subsection{Load resources}
%
%    \begin{macrocode}
\begingroup\expandafter\expandafter\expandafter\endgroup
\expandafter\ifx\csname RequirePackage\endcsname\relax
  \def\TMP@RequirePackage#1[#2]{%
    \begingroup\expandafter\expandafter\expandafter\endgroup
    \expandafter\ifx\csname ver@#1.sty\endcsname\relax
      \input #1.sty\relax
    \fi
  }%
  \TMP@RequirePackage{ltxcmds}[2011/02/04]%
  \TMP@RequirePackage{infwarerr}[2010/04/08]%
  \TMP@RequirePackage{kvsetkeys}[2010/03/01]%
  \TMP@RequirePackage{kvdefinekeys}[2010/03/01]%
  \TMP@RequirePackage{pdftexcmds}[2010/04/01]%
  \TMP@RequirePackage{ifpdf}[2010/01/28]%
  \TMP@RequirePackage{ifluatex}[2010/03/01]%
  \ltx@IfUndefined{newif}{%
    \expandafter\let\csname newif\endcsname\ltx@newif
  }{}%
  \TMP@RequirePackage{ifxetex}[2009/01/23]%
  \TMP@RequirePackage{ifvtex}[2010/03/01]%
\else
  \RequirePackage{ltxcmds}[2011/02/04]%
  \RequirePackage{infwarerr}[2010/04/08]%
  \RequirePackage{kvsetkeys}[2010/03/01]%
  \RequirePackage{kvdefinekeys}[2010/03/01]%
  \RequirePackage{pdftexcmds}[2010/04/01]%
  \RequirePackage{ifpdf}[2010/01/28]%
  \RequirePackage{ifluatex}[2010/03/01]%
  \RequirePackage{ifxetex}[2009/01/23]%
  \RequirePackage{ifvtex}[2010/03/01]%
\fi
%    \end{macrocode}
%
%    \begin{macro}{\HOLOGO@IfDefined}
%    \begin{macrocode}
\def\HOLOGO@IfExists#1{%
  \ifx\@undefined#1%
    \expandafter\ltx@secondoftwo
  \else
    \ifx\relax#1%
      \expandafter\ltx@secondoftwo
    \else
      \expandafter\expandafter\expandafter\ltx@firstoftwo
    \fi
  \fi
}
%    \end{macrocode}
%    \end{macro}
%
% \subsection{Setup macros}
%
%    \begin{macro}{\hologoSetup}
%    \begin{macrocode}
\def\hologoSetup{%
  \let\HOLOGO@name\relax
  \HOLOGO@Setup
}
%    \end{macrocode}
%    \end{macro}
%
%    \begin{macro}{\hologoLogoSetup}
%    \begin{macrocode}
\def\hologoLogoSetup#1{%
  \edef\HOLOGO@name{#1}%
  \ltx@IfUndefined{HoLogo@\HOLOGO@name}{%
    \@PackageError{hologo}{%
      Unknown logo `\HOLOGO@name'%
    }\@ehc
    \ltx@gobble
  }{%
    \HOLOGO@Setup
  }%
}
%    \end{macrocode}
%    \end{macro}
%
%    \begin{macro}{\HOLOGO@Setup}
%    \begin{macrocode}
\def\HOLOGO@Setup{%
  \kvsetkeys{HoLogo}%
}
%    \end{macrocode}
%    \end{macro}
%
% \subsection{Options}
%
%    \begin{macro}{\HOLOGO@DeclareBoolOption}
%    \begin{macrocode}
\def\HOLOGO@DeclareBoolOption#1{%
  \expandafter\chardef\csname HOLOGOOPT@#1\endcsname\ltx@zero
  \kv@define@key{HoLogo}{#1}[true]{%
    \def\HOLOGO@temp{##1}%
    \ifx\HOLOGO@temp\HOLOGO@true
      \ifx\HOLOGO@name\relax
        \expandafter\chardef\csname HOLOGOOPT@#1\endcsname=\ltx@one
      \else
        \expandafter\chardef\csname
        HoLogoOpt@#1@\HOLOGO@name\endcsname\ltx@one
      \fi
      \HOLOGO@SetBreakAll{#1}%
    \else
      \ifx\HOLOGO@temp\HOLOGO@false
        \ifx\HOLOGO@name\relax
          \expandafter\chardef\csname HOLOGOOPT@#1\endcsname=\ltx@zero
        \else
          \expandafter\chardef\csname
          HoLogoOpt@#1@\HOLOGO@name\endcsname=\ltx@zero
        \fi
        \HOLOGO@SetBreakAll{#1}%
      \else
        \@PackageError{hologo}{%
          Unknown value `##1' for boolean option `#1'.\MessageBreak
          Known values are `true' and `false'%
        }\@ehc
      \fi
    \fi
  }%
}
%    \end{macrocode}
%    \end{macro}
%
%    \begin{macro}{\HOLOGO@SetBreakAll}
%    \begin{macrocode}
\def\HOLOGO@SetBreakAll#1{%
  \def\HOLOGO@temp{#1}%
  \ifx\HOLOGO@temp\HOLOGO@break
    \ifx\HOLOGO@name\relax
      \chardef\HOLOGOOPT@hyphenbreak=\HOLOGOOPT@break
      \chardef\HOLOGOOPT@spacebreak=\HOLOGOOPT@break
      \chardef\HOLOGOOPT@discretionarybreak=\HOLOGOOPT@break
    \else
      \expandafter\chardef
         \csname HoLogoOpt@hyphenbreak@\HOLOGO@name\endcsname=%
         \csname HoLogoOpt@break@\HOLOGO@name\endcsname
      \expandafter\chardef
         \csname HoLogoOpt@spacebreak@\HOLOGO@name\endcsname=%
         \csname HoLogoOpt@break@\HOLOGO@name\endcsname
      \expandafter\chardef
         \csname HoLogoOpt@discretionarybreak@\HOLOGO@name
             \endcsname=%
         \csname HoLogoOpt@break@\HOLOGO@name\endcsname
    \fi
  \fi
}
%    \end{macrocode}
%    \end{macro}
%
%    \begin{macro}{\HOLOGO@true}
%    \begin{macrocode}
\def\HOLOGO@true{true}
%    \end{macrocode}
%    \end{macro}
%    \begin{macro}{\HOLOGO@false}
%    \begin{macrocode}
\def\HOLOGO@false{false}
%    \end{macrocode}
%    \end{macro}
%    \begin{macro}{\HOLOGO@break}
%    \begin{macrocode}
\def\HOLOGO@break{break}
%    \end{macrocode}
%    \end{macro}
%
%    \begin{macrocode}
\HOLOGO@DeclareBoolOption{break}
\HOLOGO@DeclareBoolOption{hyphenbreak}
\HOLOGO@DeclareBoolOption{spacebreak}
\HOLOGO@DeclareBoolOption{discretionarybreak}
%    \end{macrocode}
%
%    \begin{macrocode}
\kv@define@key{HoLogo}{variant}{%
  \ifx\HOLOGO@name\relax
    \@PackageError{hologo}{%
      Option `variant' is not available in \string\hologoSetup,%
      \MessageBreak
      Use \string\hologoLogoSetup\space instead%
    }\@ehc
  \else
    \edef\HOLOGO@temp{#1}%
    \ifx\HOLOGO@temp\ltx@empty
      \expandafter
      \let\csname HoLogoOpt@variant@\HOLOGO@name\endcsname\@undefined
    \else
      \ltx@IfUndefined{HoLogo@\HOLOGO@name @\HOLOGO@temp}{%
        \@PackageError{hologo}{%
          Unknown variant `\HOLOGO@temp' of logo `\HOLOGO@name'%
        }\@ehc
      }{%
        \expandafter
        \let\csname HoLogoOpt@variant@\HOLOGO@name\endcsname
            \HOLOGO@temp
      }%
    \fi
  \fi
}
%    \end{macrocode}
%
%    \begin{macro}{\HOLOGO@Variant}
%    \begin{macrocode}
\def\HOLOGO@Variant#1{%
  #1%
  \ltx@ifundefined{HoLogoOpt@variant@#1}{%
  }{%
    @\csname HoLogoOpt@variant@#1\endcsname
  }%
}
%    \end{macrocode}
%    \end{macro}
%
% \subsection{Break/no-break support}
%
%    \begin{macro}{\HOLOGO@space}
%    \begin{macrocode}
\def\HOLOGO@space{%
  \ltx@ifundefined{HoLogoOpt@spacebreak@\HOLOGO@name}{%
    \ltx@ifundefined{HoLogoOpt@break@\HOLOGO@name}{%
      \chardef\HOLOGO@temp=\HOLOGOOPT@spacebreak
    }{%
      \chardef\HOLOGO@temp=%
        \csname HoLogoOpt@break@\HOLOGO@name\endcsname
    }%
  }{%
    \chardef\HOLOGO@temp=%
      \csname HoLogoOpt@spacebreak@\HOLOGO@name\endcsname
  }%
  \ifcase\HOLOGO@temp
    \penalty10000 %
  \fi
  \ltx@space
}
%    \end{macrocode}
%    \end{macro}
%
%    \begin{macro}{\HOLOGO@hyphen}
%    \begin{macrocode}
\def\HOLOGO@hyphen{%
  \ltx@ifundefined{HoLogoOpt@hyphenbreak@\HOLOGO@name}{%
    \ltx@ifundefined{HoLogoOpt@break@\HOLOGO@name}{%
      \chardef\HOLOGO@temp=\HOLOGOOPT@hyphenbreak
    }{%
      \chardef\HOLOGO@temp=%
        \csname HoLogoOpt@break@\HOLOGO@name\endcsname
    }%
  }{%
    \chardef\HOLOGO@temp=%
      \csname HoLogoOpt@hyphenbreak@\HOLOGO@name\endcsname
  }%
  \ifcase\HOLOGO@temp
    \ltx@mbox{-}%
  \else
    -%
  \fi
}
%    \end{macrocode}
%    \end{macro}
%
%    \begin{macro}{\HOLOGO@discretionary}
%    \begin{macrocode}
\def\HOLOGO@discretionary{%
  \ltx@ifundefined{HoLogoOpt@discretionarybreak@\HOLOGO@name}{%
    \ltx@ifundefined{HoLogoOpt@break@\HOLOGO@name}{%
      \chardef\HOLOGO@temp=\HOLOGOOPT@discretionarybreak
    }{%
      \chardef\HOLOGO@temp=%
        \csname HoLogoOpt@break@\HOLOGO@name\endcsname
    }%
  }{%
    \chardef\HOLOGO@temp=%
      \csname HoLogoOpt@discretionarybreak@\HOLOGO@name\endcsname
  }%
  \ifcase\HOLOGO@temp
  \else
    \-%
  \fi
}
%    \end{macrocode}
%    \end{macro}
%
%    \begin{macro}{\HOLOGO@mbox}
%    \begin{macrocode}
\def\HOLOGO@mbox#1{%
  \ltx@ifundefined{HoLogoOpt@break@\HOLOGO@name}{%
    \chardef\HOLOGO@temp=\HOLOGOOPT@hyphenbreak
  }{%
    \chardef\HOLOGO@temp=%
      \csname HoLogoOpt@break@\HOLOGO@name\endcsname
  }%
  \ifcase\HOLOGO@temp
    \ltx@mbox{#1}%
  \else
    #1%
  \fi
}
%    \end{macrocode}
%    \end{macro}
%
% \subsection{Font support}
%
%    \begin{macro}{\HoLogoFont@font}
%    \begin{tabular}{@{}ll@{}}
%    |#1|:& logo name\\
%    |#2|:& font short name\\
%    |#3|:& text
%    \end{tabular}
%    \begin{macrocode}
\def\HoLogoFont@font#1#2#3{%
  \begingroup
    \ltx@IfUndefined{HoLogoFont@logo@#1.#2}{%
      \ltx@IfUndefined{HoLogoFont@font@#2}{%
        \@PackageWarning{hologo}{%
          Missing font `#2' for logo `#1'%
        }%
        #3%
      }{%
        \csname HoLogoFont@font@#2\endcsname{#3}%
      }%
    }{%
      \csname HoLogoFont@logo@#1.#2\endcsname{#3}%
    }%
  \endgroup
}
%    \end{macrocode}
%    \end{macro}
%
%    \begin{macro}{\HoLogoFont@Def}
%    \begin{macrocode}
\def\HoLogoFont@Def#1{%
  \expandafter\def\csname HoLogoFont@font@#1\endcsname
}
%    \end{macrocode}
%    \end{macro}
%    \begin{macro}{\HoLogoFont@LogoDef}
%    \begin{macrocode}
\def\HoLogoFont@LogoDef#1#2{%
  \expandafter\def\csname HoLogoFont@logo@#1.#2\endcsname
}
%    \end{macrocode}
%    \end{macro}
%
% \subsubsection{Font defaults}
%
%    \begin{macro}{\HoLogoFont@font@general}
%    \begin{macrocode}
\HoLogoFont@Def{general}{}%
%    \end{macrocode}
%    \end{macro}
%
%    \begin{macro}{\HoLogoFont@font@rm}
%    \begin{macrocode}
\ltx@IfUndefined{rmfamily}{%
  \ltx@IfUndefined{rm}{%
  }{%
    \HoLogoFont@Def{rm}{\rm}%
  }%
}{%
  \HoLogoFont@Def{rm}{\rmfamily}%
}
%    \end{macrocode}
%    \end{macro}
%
%    \begin{macro}{\HoLogoFont@font@sf}
%    \begin{macrocode}
\ltx@IfUndefined{sffamily}{%
  \ltx@IfUndefined{sf}{%
  }{%
    \HoLogoFont@Def{sf}{\sf}%
  }%
}{%
  \HoLogoFont@Def{sf}{\sffamily}%
}
%    \end{macrocode}
%    \end{macro}
%
%    \begin{macro}{\HoLogoFont@font@bibsf}
%    In case of \hologo{plainTeX} the original small caps
%    variant is used as default. In \hologo{LaTeX}
%    the definition of package \xpackage{dtklogos} \cite{dtklogos}
%    is used.
%\begin{quote}
%\begin{verbatim}
%\DeclareRobustCommand{\BibTeX}{%
%  B%
%  \kern-.05em%
%  \hbox{%
%    $\m@th$% %% force math size calculations
%    \csname S@\f@size\endcsname
%    \fontsize\sf@size\z@
%    \math@fontsfalse
%    \selectfont
%    I%
%    \kern-.025em%
%    B
%  }%
%  \kern-.08em%
%  \-%
%  \TeX
%}
%\end{verbatim}
%\end{quote}
%    \begin{macrocode}
\ltx@IfUndefined{selectfont}{%
  \ltx@IfUndefined{tensc}{%
    \font\tensc=cmcsc10\relax
  }{}%
  \HoLogoFont@Def{bibsf}{\tensc}%
}{%
  \HoLogoFont@Def{bibsf}{%
    $\mathsurround=0pt$%
    \csname S@\f@size\endcsname
    \fontsize\sf@size{0pt}%
    \math@fontsfalse
    \selectfont
  }%
}
%    \end{macrocode}
%    \end{macro}
%
%    \begin{macro}{\HoLogoFont@font@sc}
%    \begin{macrocode}
\ltx@IfUndefined{scshape}{%
  \ltx@IfUndefined{tensc}{%
    \font\tensc=cmcsc10\relax
  }{}%
  \HoLogoFont@Def{sc}{\tensc}%
}{%
  \HoLogoFont@Def{sc}{\scshape}%
}
%    \end{macrocode}
%    \end{macro}
%
%    \begin{macro}{\HoLogoFont@font@sy}
%    \begin{macrocode}
\ltx@IfUndefined{usefont}{%
  \ltx@IfUndefined{tensy}{%
  }{%
    \HoLogoFont@Def{sy}{\tensy}%
  }%
}{%
  \HoLogoFont@Def{sy}{%
    \usefont{OMS}{cmsy}{m}{n}%
  }%
}
%    \end{macrocode}
%    \end{macro}
%
%    \begin{macro}{\HoLogoFont@font@logo}
%    \begin{macrocode}
\begingroup
  \def\x{LaTeX2e}%
\expandafter\endgroup
\ifx\fmtname\x
  \ltx@IfUndefined{logofamily}{%
    \DeclareRobustCommand\logofamily{%
      \not@math@alphabet\logofamily\relax
      \fontencoding{U}%
      \fontfamily{logo}%
      \selectfont
    }%
  }{}%
  \ltx@IfUndefined{logofamily}{%
  }{%
    \HoLogoFont@Def{logo}{\logofamily}%
  }%
\else
  \ltx@IfUndefined{tenlogo}{%
    \font\tenlogo=logo10\relax
  }{}%
  \HoLogoFont@Def{logo}{\tenlogo}%
\fi
%    \end{macrocode}
%    \end{macro}
%
% \subsubsection{Font setup}
%
%    \begin{macro}{\hologoFontSetup}
%    \begin{macrocode}
\def\hologoFontSetup{%
  \let\HOLOGO@name\relax
  \HOLOGO@FontSetup
}
%    \end{macrocode}
%    \end{macro}
%
%    \begin{macro}{\hologoLogoFontSetup}
%    \begin{macrocode}
\def\hologoLogoFontSetup#1{%
  \edef\HOLOGO@name{#1}%
  \ltx@IfUndefined{HoLogo@\HOLOGO@name}{%
    \@PackageError{hologo}{%
      Unknown logo `\HOLOGO@name'%
    }\@ehc
    \ltx@gobble
  }{%
    \HOLOGO@FontSetup
  }%
}
%    \end{macrocode}
%    \end{macro}
%
%    \begin{macro}{\HOLOGO@FontSetup}
%    \begin{macrocode}
\def\HOLOGO@FontSetup{%
  \kvsetkeys{HoLogoFont}%
}
%    \end{macrocode}
%    \end{macro}
%
%    \begin{macrocode}
\def\HOLOGO@temp#1{%
  \kv@define@key{HoLogoFont}{#1}{%
    \ifx\HOLOGO@name\relax
      \HoLogoFont@Def{#1}{##1}%
    \else
      \HoLogoFont@LogoDef\HOLOGO@name{#1}{##1}%
    \fi
  }%
}
\HOLOGO@temp{general}
\HOLOGO@temp{sf}
%    \end{macrocode}
%
% \subsection{Generic logo commands}
%
%    \begin{macrocode}
\HOLOGO@IfExists\hologo{%
  \@PackageError{hologo}{%
    \string\hologo\ltx@space is already defined.\MessageBreak
    Package loading is aborted%
  }\@ehc
  \HOLOGO@AtEnd
}%
\HOLOGO@IfExists\hologoRobust{%
  \@PackageError{hologo}{%
    \string\hologoRobust\ltx@space is already defined.\MessageBreak
    Package loading is aborted%
  }\@ehc
  \HOLOGO@AtEnd
}%
%    \end{macrocode}
%
% \subsubsection{\cs{hologo} and friends}
%
%    \begin{macrocode}
\ifluatex
  \expandafter\ltx@firstofone
\else
  \expandafter\ltx@gobble
\fi
{%
  \ltx@IfUndefined{ifincsname}{%
    \ifnum\luatexversion<36 %
      \expandafter\ltx@gobble
    \else
      \expandafter\ltx@firstofone
    \fi
    {%
      \begingroup
        \ifcase0%
            \directlua{%
              if tex.enableprimitives then %
                tex.enableprimitives('HOLOGO@', {'ifincsname'})%
              else %
                tex.print('1')%
              end%
            }%
            \ifx\HOLOGO@ifincsname\@undefined 1\fi%
            \relax
          \expandafter\ltx@firstofone
        \else
          \endgroup
          \expandafter\ltx@gobble
        \fi
        {%
          \global\let\ifincsname\HOLOGO@ifincsname
        }%
      \HOLOGO@temp
    }%
  }{}%
}
%    \end{macrocode}
%    \begin{macrocode}
\ltx@IfUndefined{ifincsname}{%
  \catcode`$=14 %
}{%
  \catcode`$=9 %
}
%    \end{macrocode}
%
%    \begin{macro}{\hologo}
%    \begin{macrocode}
\def\hologo#1{%
$ \ifincsname
$   \ltx@ifundefined{HoLogoCs@\HOLOGO@Variant{#1}}{%
$     #1%
$   }{%
$     \csname HoLogoCs@\HOLOGO@Variant{#1}\endcsname\ltx@firstoftwo
$   }%
$ \else
    \HOLOGO@IfExists\texorpdfstring\texorpdfstring\ltx@firstoftwo
    {%
      \hologoRobust{#1}%
    }{%
      \ltx@ifundefined{HoLogoBkm@\HOLOGO@Variant{#1}}{%
        \ltx@ifundefined{HoLogo@#1}{?#1?}{#1}%
      }{%
        \csname HoLogoBkm@\HOLOGO@Variant{#1}\endcsname
        \ltx@firstoftwo
      }%
    }%
$ \fi
}
%    \end{macrocode}
%    \end{macro}
%    \begin{macro}{\Hologo}
%    \begin{macrocode}
\def\Hologo#1{%
$ \ifincsname
$   \ltx@ifundefined{HoLogoCs@\HOLOGO@Variant{#1}}{%
$     #1%
$   }{%
$     \csname HoLogoCs@\HOLOGO@Variant{#1}\endcsname\ltx@secondoftwo
$   }%
$ \else
    \HOLOGO@IfExists\texorpdfstring\texorpdfstring\ltx@firstoftwo
    {%
      \HologoRobust{#1}%
    }{%
      \ltx@ifundefined{HoLogoBkm@\HOLOGO@Variant{#1}}{%
        \ltx@ifundefined{HoLogo@#1}{?#1?}{#1}%
      }{%
        \csname HoLogoBkm@\HOLOGO@Variant{#1}\endcsname
        \ltx@secondoftwo
      }%
    }%
$ \fi
}
%    \end{macrocode}
%    \end{macro}
%
%    \begin{macro}{\hologoVariant}
%    \begin{macrocode}
\def\hologoVariant#1#2{%
  \ifx\relax#2\relax
    \hologo{#1}%
  \else
$   \ifincsname
$     \ltx@ifundefined{HoLogoCs@#1@#2}{%
$       #1%
$     }{%
$       \csname HoLogoCs@#1@#2\endcsname\ltx@firstoftwo
$     }%
$   \else
      \HOLOGO@IfExists\texorpdfstring\texorpdfstring\ltx@firstoftwo
      {%
        \hologoVariantRobust{#1}{#2}%
      }{%
        \ltx@ifundefined{HoLogoBkm@#1@#2}{%
          \ltx@ifundefined{HoLogo@#1}{?#1?}{#1}%
        }{%
          \csname HoLogoBkm@#1@#2\endcsname
          \ltx@firstoftwo
        }%
      }%
$   \fi
  \fi
}
%    \end{macrocode}
%    \end{macro}
%    \begin{macro}{\HologoVariant}
%    \begin{macrocode}
\def\HologoVariant#1#2{%
  \ifx\relax#2\relax
    \Hologo{#1}%
  \else
$   \ifincsname
$     \ltx@ifundefined{HoLogoCs@#1@#2}{%
$       #1%
$     }{%
$       \csname HoLogoCs@#1@#2\endcsname\ltx@secondoftwo
$     }%
$   \else
      \HOLOGO@IfExists\texorpdfstring\texorpdfstring\ltx@firstoftwo
      {%
        \HologoVariantRobust{#1}{#2}%
      }{%
        \ltx@ifundefined{HoLogoBkm@#1@#2}{%
          \ltx@ifundefined{HoLogo@#1}{?#1?}{#1}%
        }{%
          \csname HoLogoBkm@#1@#2\endcsname
          \ltx@secondoftwo
        }%
      }%
$   \fi
  \fi
}
%    \end{macrocode}
%    \end{macro}
%
%    \begin{macrocode}
\catcode`\$=3 %
%    \end{macrocode}
%
% \subsubsection{\cs{hologoRobust} and friends}
%
%    \begin{macro}{\hologoRobust}
%    \begin{macrocode}
\ltx@IfUndefined{protected}{%
  \ltx@IfUndefined{DeclareRobustCommand}{%
    \def\hologoRobust#1%
  }{%
    \DeclareRobustCommand*\hologoRobust[1]%
  }%
}{%
  \protected\def\hologoRobust#1%
}%
{%
  \edef\HOLOGO@name{#1}%
  \ltx@IfUndefined{HoLogo@\HOLOGO@Variant\HOLOGO@name}{%
    \@PackageError{hologo}{%
      Unknown logo `\HOLOGO@name'%
    }\@ehc
    ?\HOLOGO@name?%
  }{%
    \ltx@IfUndefined{ver@tex4ht.sty}{%
      \HoLogoFont@font\HOLOGO@name{general}{%
        \csname HoLogo@\HOLOGO@Variant\HOLOGO@name\endcsname
        \ltx@firstoftwo
      }%
    }{%
      \ltx@IfUndefined{HoLogoHtml@\HOLOGO@Variant\HOLOGO@name}{%
        \HOLOGO@name
      }{%
        \csname HoLogoHtml@\HOLOGO@Variant\HOLOGO@name\endcsname
        \ltx@firstoftwo
      }%
    }%
  }%
}
%    \end{macrocode}
%    \end{macro}
%    \begin{macro}{\HologoRobust}
%    \begin{macrocode}
\ltx@IfUndefined{protected}{%
  \ltx@IfUndefined{DeclareRobustCommand}{%
    \def\HologoRobust#1%
  }{%
    \DeclareRobustCommand*\HologoRobust[1]%
  }%
}{%
  \protected\def\HologoRobust#1%
}%
{%
  \edef\HOLOGO@name{#1}%
  \ltx@IfUndefined{HoLogo@\HOLOGO@Variant\HOLOGO@name}{%
    \@PackageError{hologo}{%
      Unknown logo `\HOLOGO@name'%
    }\@ehc
    ?\HOLOGO@name?%
  }{%
    \ltx@IfUndefined{ver@tex4ht.sty}{%
      \HoLogoFont@font\HOLOGO@name{general}{%
        \csname HoLogo@\HOLOGO@Variant\HOLOGO@name\endcsname
        \ltx@secondoftwo
      }%
    }{%
      \ltx@IfUndefined{HoLogoHtml@\HOLOGO@Variant\HOLOGO@name}{%
        \expandafter\HOLOGO@Uppercase\HOLOGO@name
      }{%
        \csname HoLogoHtml@\HOLOGO@Variant\HOLOGO@name\endcsname
        \ltx@secondoftwo
      }%
    }%
  }%
}
%    \end{macrocode}
%    \end{macro}
%    \begin{macro}{\hologoVariantRobust}
%    \begin{macrocode}
\ltx@IfUndefined{protected}{%
  \ltx@IfUndefined{DeclareRobustCommand}{%
    \def\hologoVariantRobust#1#2%
  }{%
    \DeclareRobustCommand*\hologoVariantRobust[2]%
  }%
}{%
  \protected\def\hologoVariantRobust#1#2%
}%
{%
  \begingroup
    \hologoLogoSetup{#1}{variant={#2}}%
    \hologoRobust{#1}%
  \endgroup
}
%    \end{macrocode}
%    \end{macro}
%    \begin{macro}{\HologoVariantRobust}
%    \begin{macrocode}
\ltx@IfUndefined{protected}{%
  \ltx@IfUndefined{DeclareRobustCommand}{%
    \def\HologoVariantRobust#1#2%
  }{%
    \DeclareRobustCommand*\HologoVariantRobust[2]%
  }%
}{%
  \protected\def\HologoVariantRobust#1#2%
}%
{%
  \begingroup
    \hologoLogoSetup{#1}{variant={#2}}%
    \HologoRobust{#1}%
  \endgroup
}
%    \end{macrocode}
%    \end{macro}
%
%    \begin{macro}{\hologorobust}
%    Macro \cs{hologorobust} is only defined for compatibility.
%    Its use is deprecated.
%    \begin{macrocode}
\def\hologorobust{\hologoRobust}
%    \end{macrocode}
%    \end{macro}
%
% \subsection{Helpers}
%
%    \begin{macro}{\HOLOGO@Uppercase}
%    Macro \cs{HOLOGO@Uppercase} is restricted to \cs{uppercase},
%    because \hologo{plainTeX} or \hologo{iniTeX} do not provide
%    \cs{MakeUppercase}.
%    \begin{macrocode}
\def\HOLOGO@Uppercase#1{\uppercase{#1}}
%    \end{macrocode}
%    \end{macro}
%
%    \begin{macro}{\HOLOGO@PdfdocUnicode}
%    \begin{macrocode}
\def\HOLOGO@PdfdocUnicode{%
  \ifx\ifHy@unicode\iftrue
    \expandafter\ltx@secondoftwo
  \else
    \expandafter\ltx@firstoftwo
  \fi
}
%    \end{macrocode}
%    \end{macro}
%
%    \begin{macro}{\HOLOGO@Math}
%    \begin{macrocode}
\def\HOLOGO@MathSetup{%
  \mathsurround0pt\relax
  \HOLOGO@IfExists\f@series{%
    \if b\expandafter\ltx@car\f@series x\@nil
      \csname boldmath\endcsname
   \fi
  }{}%
}
%    \end{macrocode}
%    \end{macro}
%
%    \begin{macro}{\HOLOGO@TempDimen}
%    \begin{macrocode}
\dimendef\HOLOGO@TempDimen=\ltx@zero
%    \end{macrocode}
%    \end{macro}
%    \begin{macro}{\HOLOGO@NegativeKerning}
%    \begin{macrocode}
\def\HOLOGO@NegativeKerning#1{%
  \begingroup
    \HOLOGO@TempDimen=0pt\relax
    \comma@parse@normalized{#1}{%
      \ifdim\HOLOGO@TempDimen=0pt %
        \expandafter\HOLOGO@@NegativeKerning\comma@entry
      \fi
      \ltx@gobble
    }%
    \ifdim\HOLOGO@TempDimen<0pt %
      \kern\HOLOGO@TempDimen
    \fi
  \endgroup
}
%    \end{macrocode}
%    \end{macro}
%    \begin{macro}{\HOLOGO@@NegativeKerning}
%    \begin{macrocode}
\def\HOLOGO@@NegativeKerning#1#2{%
  \setbox\ltx@zero\hbox{#1#2}%
  \HOLOGO@TempDimen=\wd\ltx@zero
  \setbox\ltx@zero\hbox{#1\kern0pt#2}%
  \advance\HOLOGO@TempDimen by -\wd\ltx@zero
}
%    \end{macrocode}
%    \end{macro}
%
%    \begin{macro}{\HOLOGO@SpaceFactor}
%    \begin{macrocode}
\def\HOLOGO@SpaceFactor{%
  \spacefactor1000 %
}
%    \end{macrocode}
%    \end{macro}
%
%    \begin{macro}{\HOLOGO@Span}
%    \begin{macrocode}
\def\HOLOGO@Span#1#2{%
  \HCode{<span class="HoLogo-#1">}%
  #2%
  \HCode{</span>}%
}
%    \end{macrocode}
%    \end{macro}
%
% \subsubsection{Text subscript}
%
%    \begin{macro}{\HOLOGO@SubScript}%
%    \begin{macrocode}
\def\HOLOGO@SubScript#1{%
  \ltx@IfUndefined{textsubscript}{%
    \ltx@IfUndefined{text}{%
      \ltx@mbox{%
        \mathsurround=0pt\relax
        $%
          _{%
            \ltx@IfUndefined{sf@size}{%
              \mathrm{#1}%
            }{%
              \mbox{%
                \fontsize\sf@size{0pt}\selectfont
                #1%
              }%
            }%
          }%
        $%
      }%
    }{%
      \ltx@mbox{%
        \mathsurround=0pt\relax
        $_{\text{#1}}$%
      }%
    }%
  }{%
    \textsubscript{#1}%
  }%
}
%    \end{macrocode}
%    \end{macro}
%
% \subsection{\hologo{TeX} and friends}
%
% \subsubsection{\hologo{TeX}}
%
%    \begin{macro}{\HoLogo@TeX}
%    Source: \hologo{LaTeX} kernel.
%    \begin{macrocode}
\def\HoLogo@TeX#1{%
  T\kern-.1667em\lower.5ex\hbox{E}\kern-.125emX\HOLOGO@SpaceFactor
}
%    \end{macrocode}
%    \end{macro}
%    \begin{macro}{\HoLogoHtml@TeX}
%    \begin{macrocode}
\def\HoLogoHtml@TeX#1{%
  \HoLogoCss@TeX
  \HOLOGO@Span{TeX}{%
    T%
    \HOLOGO@Span{e}{%
      E%
    }%
    X%
  }%
}
%    \end{macrocode}
%    \end{macro}
%    \begin{macro}{\HoLogoCss@TeX}
%    \begin{macrocode}
\def\HoLogoCss@TeX{%
  \Css{%
    span.HoLogo-TeX span.HoLogo-e{%
      position:relative;%
      top:.5ex;%
      margin-left:-.1667em;%
      margin-right:-.125em;%
    }%
  }%
  \Css{%
    a span.HoLogo-TeX span.HoLogo-e{%
      text-decoration:none;%
    }%
  }%
  \global\let\HoLogoCss@TeX\relax
}
%    \end{macrocode}
%    \end{macro}
%
% \subsubsection{\hologo{plainTeX}}
%
%    \begin{macro}{\HoLogo@plainTeX@space}
%    Source: ``The \hologo{TeX}book''
%    \begin{macrocode}
\def\HoLogo@plainTeX@space#1{%
  \HOLOGO@mbox{#1{p}{P}lain}\HOLOGO@space\hologo{TeX}%
}
%    \end{macrocode}
%    \end{macro}
%    \begin{macro}{\HoLogoCs@plainTeX@space}
%    \begin{macrocode}
\def\HoLogoCs@plainTeX@space#1{#1{p}{P}lain TeX}%
%    \end{macrocode}
%    \end{macro}
%    \begin{macro}{\HoLogoBkm@plainTeX@space}
%    \begin{macrocode}
\def\HoLogoBkm@plainTeX@space#1{%
  #1{p}{P}lain \hologo{TeX}%
}
%    \end{macrocode}
%    \end{macro}
%    \begin{macro}{\HoLogoHtml@plainTeX@space}
%    \begin{macrocode}
\def\HoLogoHtml@plainTeX@space#1{%
  #1{p}{P}lain \hologo{TeX}%
}
%    \end{macrocode}
%    \end{macro}
%
%    \begin{macro}{\HoLogo@plainTeX@hyphen}
%    \begin{macrocode}
\def\HoLogo@plainTeX@hyphen#1{%
  \HOLOGO@mbox{#1{p}{P}lain}\HOLOGO@hyphen\hologo{TeX}%
}
%    \end{macrocode}
%    \end{macro}
%    \begin{macro}{\HoLogoCs@plainTeX@hyphen}
%    \begin{macrocode}
\def\HoLogoCs@plainTeX@hyphen#1{#1{p}{P}lain-TeX}
%    \end{macrocode}
%    \end{macro}
%    \begin{macro}{\HoLogoBkm@plainTeX@hyphen}
%    \begin{macrocode}
\def\HoLogoBkm@plainTeX@hyphen#1{%
  #1{p}{P}lain-\hologo{TeX}%
}
%    \end{macrocode}
%    \end{macro}
%    \begin{macro}{\HoLogoHtml@plainTeX@hyphen}
%    \begin{macrocode}
\def\HoLogoHtml@plainTeX@hyphen#1{%
  #1{p}{P}lain-\hologo{TeX}%
}
%    \end{macrocode}
%    \end{macro}
%
%    \begin{macro}{\HoLogo@plainTeX@runtogether}
%    \begin{macrocode}
\def\HoLogo@plainTeX@runtogether#1{%
  \HOLOGO@mbox{#1{p}{P}lain\hologo{TeX}}%
}
%    \end{macrocode}
%    \end{macro}
%    \begin{macro}{\HoLogoCs@plainTeX@runtogether}
%    \begin{macrocode}
\def\HoLogoCs@plainTeX@runtogether#1{#1{p}{P}lainTeX}
%    \end{macrocode}
%    \end{macro}
%    \begin{macro}{\HoLogoBkm@plainTeX@runtogether}
%    \begin{macrocode}
\def\HoLogoBkm@plainTeX@runtogether#1{%
  #1{p}{P}lain\hologo{TeX}%
}
%    \end{macrocode}
%    \end{macro}
%    \begin{macro}{\HoLogoHtml@plainTeX@runtogether}
%    \begin{macrocode}
\def\HoLogoHtml@plainTeX@runtogether#1{%
  #1{p}{P}lain\hologo{TeX}%
}
%    \end{macrocode}
%    \end{macro}
%
%    \begin{macro}{\HoLogo@plainTeX}
%    \begin{macrocode}
\def\HoLogo@plainTeX{\HoLogo@plainTeX@space}
%    \end{macrocode}
%    \end{macro}
%    \begin{macro}{\HoLogoCs@plainTeX}
%    \begin{macrocode}
\def\HoLogoCs@plainTeX{\HoLogoCs@plainTeX@space}
%    \end{macrocode}
%    \end{macro}
%    \begin{macro}{\HoLogoBkm@plainTeX}
%    \begin{macrocode}
\def\HoLogoBkm@plainTeX{\HoLogoBkm@plainTeX@space}
%    \end{macrocode}
%    \end{macro}
%    \begin{macro}{\HoLogoHtml@plainTeX}
%    \begin{macrocode}
\def\HoLogoHtml@plainTeX{\HoLogoHtml@plainTeX@space}
%    \end{macrocode}
%    \end{macro}
%
% \subsubsection{\hologo{LaTeX}}
%
%    Source: \hologo{LaTeX} kernel.
%\begin{quote}
%\begin{verbatim}
%\DeclareRobustCommand{\LaTeX}{%
%  L%
%  \kern-.36em%
%  {%
%    \sbox\z@ T%
%    \vbox to\ht\z@{%
%      \hbox{%
%        \check@mathfonts
%        \fontsize\sf@size\z@
%        \math@fontsfalse
%        \selectfont
%        A%
%      }%
%      \vss
%    }%
%  }%
%  \kern-.15em%
%  \TeX
%}
%\end{verbatim}
%\end{quote}
%
%    \begin{macro}{\HoLogo@La}
%    \begin{macrocode}
\def\HoLogo@La#1{%
  L%
  \kern-.36em%
  \begingroup
    \setbox\ltx@zero\hbox{T}%
    \vbox to\ht\ltx@zero{%
      \hbox{%
        \ltx@ifundefined{check@mathfonts}{%
          \csname sevenrm\endcsname
        }{%
          \check@mathfonts
          \fontsize\sf@size{0pt}%
          \math@fontsfalse\selectfont
        }%
        A%
      }%
      \vss
    }%
  \endgroup
}
%    \end{macrocode}
%    \end{macro}
%
%    \begin{macro}{\HoLogo@LaTeX}
%    Source: \hologo{LaTeX} kernel.
%    \begin{macrocode}
\def\HoLogo@LaTeX#1{%
  \hologo{La}%
  \kern-.15em%
  \hologo{TeX}%
}
%    \end{macrocode}
%    \end{macro}
%    \begin{macro}{\HoLogoHtml@LaTeX}
%    \begin{macrocode}
\def\HoLogoHtml@LaTeX#1{%
  \HoLogoCss@LaTeX
  \HOLOGO@Span{LaTeX}{%
    L%
    \HOLOGO@Span{a}{%
      A%
    }%
    \hologo{TeX}%
  }%
}
%    \end{macrocode}
%    \end{macro}
%    \begin{macro}{\HoLogoCss@LaTeX}
%    \begin{macrocode}
\def\HoLogoCss@LaTeX{%
  \Css{%
    span.HoLogo-LaTeX span.HoLogo-a{%
      position:relative;%
      top:-.5ex;%
      margin-left:-.36em;%
      margin-right:-.15em;%
      font-size:85\%;%
    }%
  }%
  \global\let\HoLogoCss@LaTeX\relax
}
%    \end{macrocode}
%    \end{macro}
%
% \subsubsection{\hologo{(La)TeX}}
%
%    \begin{macro}{\HoLogo@LaTeXTeX}
%    The kerning around the parentheses is taken
%    from package \xpackage{dtklogos} \cite{dtklogos}.
%\begin{quote}
%\begin{verbatim}
%\DeclareRobustCommand{\LaTeXTeX}{%
%  (%
%  \kern-.15em%
%  L%
%  \kern-.36em%
%  {%
%    \sbox\z@ T%
%    \vbox to\ht0{%
%      \hbox{%
%        $\m@th$%
%        \csname S@\f@size\endcsname
%        \fontsize\sf@size\z@
%        \math@fontsfalse
%        \selectfont
%        A%
%      }%
%      \vss
%    }%
%  }%
%  \kern-.2em%
%  )%
%  \kern-.15em%
%  \TeX
%}
%\end{verbatim}
%\end{quote}
%    \begin{macrocode}
\def\HoLogo@LaTeXTeX#1{%
  (%
  \kern-.15em%
  \hologo{La}%
  \kern-.2em%
  )%
  \kern-.15em%
  \hologo{TeX}%
}
%    \end{macrocode}
%    \end{macro}
%    \begin{macro}{\HoLogoBkm@LaTeXTeX}
%    \begin{macrocode}
\def\HoLogoBkm@LaTeXTeX#1{(La)TeX}
%    \end{macrocode}
%    \end{macro}
%
%    \begin{macro}{\HoLogo@(La)TeX}
%    \begin{macrocode}
\expandafter
\let\csname HoLogo@(La)TeX\endcsname\HoLogo@LaTeXTeX
%    \end{macrocode}
%    \end{macro}
%    \begin{macro}{\HoLogoBkm@(La)TeX}
%    \begin{macrocode}
\expandafter
\let\csname HoLogoBkm@(La)TeX\endcsname\HoLogoBkm@LaTeXTeX
%    \end{macrocode}
%    \end{macro}
%    \begin{macro}{\HoLogoHtml@LaTeXTeX}
%    \begin{macrocode}
\def\HoLogoHtml@LaTeXTeX#1{%
  \HoLogoCss@LaTeXTeX
  \HOLOGO@Span{LaTeXTeX}{%
    (%
    \HOLOGO@Span{L}{L}%
    \HOLOGO@Span{a}{A}%
    \HOLOGO@Span{ParenRight}{)}%
    \hologo{TeX}%
  }%
}
%    \end{macrocode}
%    \end{macro}
%    \begin{macro}{\HoLogoHtml@(La)TeX}
%    Kerning after opening parentheses and before closing parentheses
%    is $-0.1$\,em. The original values $-0.15$\,em
%    looked too ugly for a serif font.
%    \begin{macrocode}
\expandafter
\let\csname HoLogoHtml@(La)TeX\endcsname\HoLogoHtml@LaTeXTeX
%    \end{macrocode}
%    \end{macro}
%    \begin{macro}{\HoLogoCss@LaTeXTeX}
%    \begin{macrocode}
\def\HoLogoCss@LaTeXTeX{%
  \Css{%
    span.HoLogo-LaTeXTeX span.HoLogo-L{%
      margin-left:-.1em;%
    }%
  }%
  \Css{%
    span.HoLogo-LaTeXTeX span.HoLogo-a{%
      position:relative;%
      top:-.5ex;%
      margin-left:-.36em;%
      margin-right:-.1em;%
      font-size:85\%;%
    }%
  }%
  \Css{%
    span.HoLogo-LaTeXTeX span.HoLogo-ParenRight{%
      margin-right:-.15em;%
    }%
  }%
  \global\let\HoLogoCss@LaTeXTeX\relax
}
%    \end{macrocode}
%    \end{macro}
%
% \subsubsection{\hologo{LaTeXe}}
%
%    \begin{macro}{\HoLogo@LaTeXe}
%    Source: \hologo{LaTeX} kernel
%    \begin{macrocode}
\def\HoLogo@LaTeXe#1{%
  \hologo{LaTeX}%
  \kern.15em%
  \hbox{%
    \HOLOGO@MathSetup
    2%
    $_{\textstyle\varepsilon}$%
  }%
}
%    \end{macrocode}
%    \end{macro}
%
%    \begin{macro}{\HoLogoCs@LaTeXe}
%    \begin{macrocode}
\ifnum64=`\^^^^0040\relax % test for big chars of LuaTeX/XeTeX
  \catcode`\$=9 %
  \catcode`\&=14 %
\else
  \catcode`\$=14 %
  \catcode`\&=9 %
\fi
\def\HoLogoCs@LaTeXe#1{%
  LaTeX2%
$ \string ^^^^0395%
& e%
}%
\catcode`\$=3 %
\catcode`\&=4 %
%    \end{macrocode}
%    \end{macro}
%
%    \begin{macro}{\HoLogoBkm@LaTeXe}
%    \begin{macrocode}
\def\HoLogoBkm@LaTeXe#1{%
  \hologo{LaTeX}%
  2%
  \HOLOGO@PdfdocUnicode{e}{\textepsilon}%
}
%    \end{macrocode}
%    \end{macro}
%
%    \begin{macro}{\HoLogoHtml@LaTeXe}
%    \begin{macrocode}
\def\HoLogoHtml@LaTeXe#1{%
  \HoLogoCss@LaTeXe
  \HOLOGO@Span{LaTeX2e}{%
    \hologo{LaTeX}%
    \HOLOGO@Span{2}{2}%
    \HOLOGO@Span{e}{%
      \HOLOGO@MathSetup
      \ensuremath{\textstyle\varepsilon}%
    }%
  }%
}
%    \end{macrocode}
%    \end{macro}
%    \begin{macro}{\HoLogoCss@LaTeXe}
%    \begin{macrocode}
\def\HoLogoCss@LaTeXe{%
  \Css{%
    span.HoLogo-LaTeX2e span.HoLogo-2{%
      padding-left:.15em;%
    }%
  }%
  \Css{%
    span.HoLogo-LaTeX2e span.HoLogo-e{%
      position:relative;%
      top:.35ex;%
      text-decoration:none;%
    }%
  }%
  \global\let\HoLogoCss@LaTeXe\relax
}
%    \end{macrocode}
%    \end{macro}
%
%    \begin{macro}{\HoLogo@LaTeX2e}
%    \begin{macrocode}
\expandafter
\let\csname HoLogo@LaTeX2e\endcsname\HoLogo@LaTeXe
%    \end{macrocode}
%    \end{macro}
%    \begin{macro}{\HoLogoCs@LaTeX2e}
%    \begin{macrocode}
\expandafter
\let\csname HoLogoCs@LaTeX2e\endcsname\HoLogoCs@LaTeXe
%    \end{macrocode}
%    \end{macro}
%    \begin{macro}{\HoLogoBkm@LaTeX2e}
%    \begin{macrocode}
\expandafter
\let\csname HoLogoBkm@LaTeX2e\endcsname\HoLogoBkm@LaTeXe
%    \end{macrocode}
%    \end{macro}
%    \begin{macro}{\HoLogoHtml@LaTeX2e}
%    \begin{macrocode}
\expandafter
\let\csname HoLogoHtml@LaTeX2e\endcsname\HoLogoHtml@LaTeXe
%    \end{macrocode}
%    \end{macro}
%
% \subsubsection{\hologo{LaTeX3}}
%
%    \begin{macro}{\HoLogo@LaTeX3}
%    Source: \hologo{LaTeX} kernel
%    \begin{macrocode}
\expandafter\def\csname HoLogo@LaTeX3\endcsname#1{%
  \hologo{LaTeX}%
  3%
}
%    \end{macrocode}
%    \end{macro}
%
%    \begin{macro}{\HoLogoBkm@LaTeX3}
%    \begin{macrocode}
\expandafter\def\csname HoLogoBkm@LaTeX3\endcsname#1{%
  \hologo{LaTeX}%
  3%
}
%    \end{macrocode}
%    \end{macro}
%    \begin{macro}{\HoLogoHtml@LaTeX3}
%    \begin{macrocode}
\expandafter
\let\csname HoLogoHtml@LaTeX3\expandafter\endcsname
\csname HoLogo@LaTeX3\endcsname
%    \end{macrocode}
%    \end{macro}
%
% \subsubsection{\hologo{LaTeXML}}
%
%    \begin{macro}{\HoLogo@LaTeXML}
%    \begin{macrocode}
\def\HoLogo@LaTeXML#1{%
  \HOLOGO@mbox{%
    \hologo{La}%
    \kern-.15em%
    T%
    \kern-.1667em%
    \lower.5ex\hbox{E}%
    \kern-.125em%
    \HoLogoFont@font{LaTeXML}{sc}{xml}%
  }%
}
%    \end{macrocode}
%    \end{macro}
%    \begin{macro}{\HoLogoHtml@pdfLaTeX}
%    \begin{macrocode}
\def\HoLogoHtml@LaTeXML#1{%
  \HOLOGO@Span{LaTeXML}{%
    \HoLogoCss@LaTeX
    \HoLogoCss@TeX
    \HOLOGO@Span{LaTeX}{%
      L%
      \HOLOGO@Span{a}{%
        A%
      }%
    }%
    \HOLOGO@Span{TeX}{%
      T%
      \HOLOGO@Span{e}{%
        E%
      }%
    }%
    \HCode{<span style="font-variant: small-caps;">}%
    xml%
    \HCode{</span>}%
  }%
}
%    \end{macrocode}
%    \end{macro}
%
% \subsubsection{\hologo{eTeX}}
%
%    \begin{macro}{\HoLogo@eTeX}
%    Source: package \xpackage{etex}
%    \begin{macrocode}
\def\HoLogo@eTeX#1{%
  \ltx@mbox{%
    \HOLOGO@MathSetup
    $\varepsilon$%
    -%
    \HOLOGO@NegativeKerning{-T,T-,To}%
    \hologo{TeX}%
  }%
}
%    \end{macrocode}
%    \end{macro}
%    \begin{macro}{\HoLogoCs@eTeX}
%    \begin{macrocode}
\ifnum64=`\^^^^0040\relax % test for big chars of LuaTeX/XeTeX
  \catcode`\$=9 %
  \catcode`\&=14 %
\else
  \catcode`\$=14 %
  \catcode`\&=9 %
\fi
\def\HoLogoCs@eTeX#1{%
$ #1{\string ^^^^0395}{\string ^^^^03b5}%
& #1{e}{E}%
  TeX%
}%
\catcode`\$=3 %
\catcode`\&=4 %
%    \end{macrocode}
%    \end{macro}
%    \begin{macro}{\HoLogoBkm@eTeX}
%    \begin{macrocode}
\def\HoLogoBkm@eTeX#1{%
  \HOLOGO@PdfdocUnicode{#1{e}{E}}{\textepsilon}%
  -%
  \hologo{TeX}%
}
%    \end{macrocode}
%    \end{macro}
%    \begin{macro}{\HoLogoHtml@eTeX}
%    \begin{macrocode}
\def\HoLogoHtml@eTeX#1{%
  \ltx@mbox{%
    \HOLOGO@MathSetup
    $\varepsilon$%
    -%
    \hologo{TeX}%
  }%
}
%    \end{macrocode}
%    \end{macro}
%
% \subsubsection{\hologo{iniTeX}}
%
%    \begin{macro}{\HoLogo@iniTeX}
%    \begin{macrocode}
\def\HoLogo@iniTeX#1{%
  \HOLOGO@mbox{%
    #1{i}{I}ni\hologo{TeX}%
  }%
}
%    \end{macrocode}
%    \end{macro}
%    \begin{macro}{\HoLogoCs@iniTeX}
%    \begin{macrocode}
\def\HoLogoCs@iniTeX#1{#1{i}{I}niTeX}
%    \end{macrocode}
%    \end{macro}
%    \begin{macro}{\HoLogoBkm@iniTeX}
%    \begin{macrocode}
\def\HoLogoBkm@iniTeX#1{%
  #1{i}{I}ni\hologo{TeX}%
}
%    \end{macrocode}
%    \end{macro}
%    \begin{macro}{\HoLogoHtml@iniTeX}
%    \begin{macrocode}
\let\HoLogoHtml@iniTeX\HoLogo@iniTeX
%    \end{macrocode}
%    \end{macro}
%
% \subsubsection{\hologo{virTeX}}
%
%    \begin{macro}{\HoLogo@virTeX}
%    \begin{macrocode}
\def\HoLogo@virTeX#1{%
  \HOLOGO@mbox{%
    #1{v}{V}ir\hologo{TeX}%
  }%
}
%    \end{macrocode}
%    \end{macro}
%    \begin{macro}{\HoLogoCs@virTeX}
%    \begin{macrocode}
\def\HoLogoCs@virTeX#1{#1{v}{V}irTeX}
%    \end{macrocode}
%    \end{macro}
%    \begin{macro}{\HoLogoBkm@virTeX}
%    \begin{macrocode}
\def\HoLogoBkm@virTeX#1{%
  #1{v}{V}ir\hologo{TeX}%
}
%    \end{macrocode}
%    \end{macro}
%    \begin{macro}{\HoLogoHtml@virTeX}
%    \begin{macrocode}
\let\HoLogoHtml@virTeX\HoLogo@virTeX
%    \end{macrocode}
%    \end{macro}
%
% \subsubsection{\hologo{SliTeX}}
%
% \paragraph{Definitions of the three variants.}
%
%    \begin{macro}{\HoLogo@SLiTeX@lift}
%    \begin{macrocode}
\def\HoLogo@SLiTeX@lift#1{%
  \HoLogoFont@font{SliTeX}{rm}{%
    S%
    \kern-.06em%
    L%
    \kern-.18em%
    \raise.32ex\hbox{\HoLogoFont@font{SliTeX}{sc}{i}}%
    \HOLOGO@discretionary
    \kern-.06em%
    \hologo{TeX}%
  }%
}
%    \end{macrocode}
%    \end{macro}
%    \begin{macro}{\HoLogoBkm@SLiTeX@lift}
%    \begin{macrocode}
\def\HoLogoBkm@SLiTeX@lift#1{SLiTeX}
%    \end{macrocode}
%    \end{macro}
%    \begin{macro}{\HoLogoHtml@SLiTeX@lift}
%    \begin{macrocode}
\def\HoLogoHtml@SLiTeX@lift#1{%
  \HoLogoCss@SLiTeX@lift
  \HOLOGO@Span{SLiTeX-lift}{%
    \HoLogoFont@font{SliTeX}{rm}{%
      S%
      \HOLOGO@Span{L}{L}%
      \HOLOGO@Span{i}{i}%
      \hologo{TeX}%
    }%
  }%
}
%    \end{macrocode}
%    \end{macro}
%    \begin{macro}{\HoLogoCss@SLiTeX@lift}
%    \begin{macrocode}
\def\HoLogoCss@SLiTeX@lift{%
  \Css{%
    span.HoLogo-SLiTeX-lift span.HoLogo-L{%
      margin-left:-.06em;%
      margin-right:-.18em;%
    }%
  }%
  \Css{%
    span.HoLogo-SLiTeX-lift span.HoLogo-i{%
      position:relative;%
      top:-.32ex;%
      margin-right:-.06em;%
      font-variant:small-caps;%
    }%
  }%
  \global\let\HoLogoCss@SLiTeX@lift\relax
}
%    \end{macrocode}
%    \end{macro}
%
%    \begin{macro}{\HoLogo@SliTeX@simple}
%    \begin{macrocode}
\def\HoLogo@SliTeX@simple#1{%
  \HoLogoFont@font{SliTeX}{rm}{%
    \ltx@mbox{%
      \HoLogoFont@font{SliTeX}{sc}{Sli}%
    }%
    \HOLOGO@discretionary
    \hologo{TeX}%
  }%
}
%    \end{macrocode}
%    \end{macro}
%    \begin{macro}{\HoLogoBkm@SliTeX@simple}
%    \begin{macrocode}
\def\HoLogoBkm@SliTeX@simple#1{SliTeX}
%    \end{macrocode}
%    \end{macro}
%    \begin{macro}{\HoLogoHtml@SliTeX@simple}
%    \begin{macrocode}
\let\HoLogoHtml@SliTeX@simple\HoLogo@SliTeX@simple
%    \end{macrocode}
%    \end{macro}
%
%    \begin{macro}{\HoLogo@SliTeX@narrow}
%    \begin{macrocode}
\def\HoLogo@SliTeX@narrow#1{%
  \HoLogoFont@font{SliTeX}{rm}{%
    \ltx@mbox{%
      S%
      \kern-.06em%
      \HoLogoFont@font{SliTeX}{sc}{%
        l%
        \kern-.035em%
        i%
      }%
    }%
    \HOLOGO@discretionary
    \kern-.06em%
    \hologo{TeX}%
  }%
}
%    \end{macrocode}
%    \end{macro}
%    \begin{macro}{\HoLogoBkm@SliTeX@narrow}
%    \begin{macrocode}
\def\HoLogoBkm@SliTeX@narrow#1{SliTeX}
%    \end{macrocode}
%    \end{macro}
%    \begin{macro}{\HoLogoHtml@SliTeX@narrow}
%    \begin{macrocode}
\def\HoLogoHtml@SliTeX@narrow#1{%
  \HoLogoCss@SliTeX@narrow
  \HOLOGO@Span{SliTeX-narrow}{%
    \HoLogoFont@font{SliTeX}{rm}{%
      S%
        \HOLOGO@Span{l}{l}%
        \HOLOGO@Span{i}{i}%
      \hologo{TeX}%
    }%
  }%
}
%    \end{macrocode}
%    \end{macro}
%    \begin{macro}{\HoLogoCss@SliTeX@narrow}
%    \begin{macrocode}
\def\HoLogoCss@SliTeX@narrow{%
  \Css{%
    span.HoLogo-SliTeX-narrow span.HoLogo-l{%
      margin-left:-.06em;%
      margin-right:-.035em;%
      font-variant:small-caps;%
    }%
  }%
  \Css{%
    span.HoLogo-SliTeX-narrow span.HoLogo-i{%
      margin-right:-.06em;%
      font-variant:small-caps;%
    }%
  }%
  \global\let\HoLogoCss@SliTeX@narrow\relax
}
%    \end{macrocode}
%    \end{macro}
%
% \paragraph{Macro set completion.}
%
%    \begin{macro}{\HoLogo@SLiTeX@simple}
%    \begin{macrocode}
\def\HoLogo@SLiTeX@simple{\HoLogo@SliTeX@simple}
%    \end{macrocode}
%    \end{macro}
%    \begin{macro}{\HoLogoBkm@SLiTeX@simple}
%    \begin{macrocode}
\def\HoLogoBkm@SLiTeX@simple{\HoLogoBkm@SliTeX@simple}
%    \end{macrocode}
%    \end{macro}
%    \begin{macro}{\HoLogoHtml@SLiTeX@simple}
%    \begin{macrocode}
\def\HoLogoHtml@SLiTeX@simple{\HoLogoHtml@SliTeX@simple}
%    \end{macrocode}
%    \end{macro}
%
%    \begin{macro}{\HoLogo@SLiTeX@narrow}
%    \begin{macrocode}
\def\HoLogo@SLiTeX@narrow{\HoLogo@SliTeX@narrow}
%    \end{macrocode}
%    \end{macro}
%    \begin{macro}{\HoLogoBkm@SLiTeX@narrow}
%    \begin{macrocode}
\def\HoLogoBkm@SLiTeX@narrow{\HoLogoBkm@SliTeX@narrow}
%    \end{macrocode}
%    \end{macro}
%    \begin{macro}{\HoLogoHtml@SLiTeX@narrow}
%    \begin{macrocode}
\def\HoLogoHtml@SLiTeX@narrow{\HoLogoHtml@SliTeX@narrow}
%    \end{macrocode}
%    \end{macro}
%
%    \begin{macro}{\HoLogo@SliTeX@lift}
%    \begin{macrocode}
\def\HoLogo@SliTeX@lift{\HoLogo@SLiTeX@lift}
%    \end{macrocode}
%    \end{macro}
%    \begin{macro}{\HoLogoBkm@SliTeX@lift}
%    \begin{macrocode}
\def\HoLogoBkm@SliTeX@lift{\HoLogoBkm@SLiTeX@lift}
%    \end{macrocode}
%    \end{macro}
%    \begin{macro}{\HoLogoHtml@SliTeX@lift}
%    \begin{macrocode}
\def\HoLogoHtml@SliTeX@lift{\HoLogoHtml@SLiTeX@lift}
%    \end{macrocode}
%    \end{macro}
%
% \paragraph{Defaults.}
%
%    \begin{macro}{\HoLogo@SLiTeX}
%    \begin{macrocode}
\def\HoLogo@SLiTeX{\HoLogo@SLiTeX@lift}
%    \end{macrocode}
%    \end{macro}
%    \begin{macro}{\HoLogoBkm@SLiTeX}
%    \begin{macrocode}
\def\HoLogoBkm@SLiTeX{\HoLogoBkm@SLiTeX@lift}
%    \end{macrocode}
%    \end{macro}
%    \begin{macro}{\HoLogoHtml@SLiTeX}
%    \begin{macrocode}
\def\HoLogoHtml@SLiTeX{\HoLogoHtml@SLiTeX@lift}
%    \end{macrocode}
%    \end{macro}
%
%    \begin{macro}{\HoLogo@SliTeX}
%    \begin{macrocode}
\def\HoLogo@SliTeX{\HoLogo@SliTeX@narrow}
%    \end{macrocode}
%    \end{macro}
%    \begin{macro}{\HoLogoBkm@SliTeX}
%    \begin{macrocode}
\def\HoLogoBkm@SliTeX{\HoLogoBkm@SliTeX@narrow}
%    \end{macrocode}
%    \end{macro}
%    \begin{macro}{\HoLogoHtml@SliTeX}
%    \begin{macrocode}
\def\HoLogoHtml@SliTeX{\HoLogoHtml@SliTeX@narrow}
%    \end{macrocode}
%    \end{macro}
%
% \subsubsection{\hologo{LuaTeX}}
%
%    \begin{macro}{\HoLogo@LuaTeX}
%    The kerning is an idea of Hans Hagen, see mailing list
%    `luatex at tug dot org' in March 2010.
%    \begin{macrocode}
\def\HoLogo@LuaTeX#1{%
  \HOLOGO@mbox{%
    Lua%
    \HOLOGO@NegativeKerning{aT,oT,To}%
    \hologo{TeX}%
  }%
}
%    \end{macrocode}
%    \end{macro}
%    \begin{macro}{\HoLogoHtml@LuaTeX}
%    \begin{macrocode}
\let\HoLogoHtml@LuaTeX\HoLogo@LuaTeX
%    \end{macrocode}
%    \end{macro}
%
% \subsubsection{\hologo{LuaLaTeX}}
%
%    \begin{macro}{\HoLogo@LuaLaTeX}
%    \begin{macrocode}
\def\HoLogo@LuaLaTeX#1{%
  \HOLOGO@mbox{%
    Lua%
    \hologo{LaTeX}%
  }%
}
%    \end{macrocode}
%    \end{macro}
%    \begin{macro}{\HoLogoHtml@LuaLaTeX}
%    \begin{macrocode}
\let\HoLogoHtml@LuaLaTeX\HoLogo@LuaLaTeX
%    \end{macrocode}
%    \end{macro}
%
% \subsubsection{\hologo{XeTeX}, \hologo{XeLaTeX}}
%
%    \begin{macro}{\HOLOGO@IfCharExists}
%    \begin{macrocode}
\ifluatex
  \ifnum\luatexversion<36 %
  \else
    \def\HOLOGO@IfCharExists#1{%
      \ifnum
        \directlua{%
           if luaotfload and luaotfload.aux then
             if luaotfload.aux.font_has_glyph(%
                    font.current(), \number#1) then % 	 
	       tex.print("1") % 	 
	     end % 	 
	   elseif font and font.fonts and font.current then %
            local f = font.fonts[font.current()]%
            if f.characters and f.characters[\number#1] then %
              tex.print("1")%
            end %
          end%
        }0=\ltx@zero
        \expandafter\ltx@secondoftwo
      \else
        \expandafter\ltx@firstoftwo
      \fi
    }%
  \fi
\fi
\ltx@IfUndefined{HOLOGO@IfCharExists}{%
  \def\HOLOGO@@IfCharExists#1{%
    \begingroup
      \tracinglostchars=\ltx@zero
      \setbox\ltx@zero=\hbox{%
        \kern7sp\char#1\relax
        \ifnum\lastkern>\ltx@zero
          \expandafter\aftergroup\csname iffalse\endcsname
        \else
          \expandafter\aftergroup\csname iftrue\endcsname
        \fi
      }%
      % \if{true|false} from \aftergroup
      \endgroup
      \expandafter\ltx@firstoftwo
    \else
      \endgroup
      \expandafter\ltx@secondoftwo
    \fi
  }%
  \ifxetex
    \ltx@IfUndefined{XeTeXfonttype}{}{%
      \ltx@IfUndefined{XeTeXcharglyph}{}{%
        \def\HOLOGO@IfCharExists#1{%
          \ifnum\XeTeXfonttype\font>\ltx@zero
            \expandafter\ltx@firstofthree
          \else
            \expandafter\ltx@gobble
          \fi
          {%
            \ifnum\XeTeXcharglyph#1>\ltx@zero
              \expandafter\ltx@firstoftwo
            \else
              \expandafter\ltx@secondoftwo
            \fi
          }%
          \HOLOGO@@IfCharExists{#1}%
        }%
      }%
    }%
  \fi
}{}
\ltx@ifundefined{HOLOGO@IfCharExists}{%
  \ifnum64=`\^^^^0040\relax % test for big chars of LuaTeX/XeTeX
    \let\HOLOGO@IfCharExists\HOLOGO@@IfCharExists
  \else
    \def\HOLOGO@IfCharExists#1{%
      \ifnum#1>255 %
        \expandafter\ltx@fourthoffour
      \fi
      \HOLOGO@@IfCharExists{#1}%
    }%
  \fi
}{}
%    \end{macrocode}
%    \end{macro}
%
%    \begin{macro}{\HoLogo@Xe}
%    Source: package \xpackage{dtklogos}
%    \begin{macrocode}
\def\HoLogo@Xe#1{%
  X%
  \kern-.1em\relax
  \HOLOGO@IfCharExists{"018E}{%
    \lower.5ex\hbox{\char"018E}%
  }{%
    \chardef\HOLOGO@choice=\ltx@zero
    \ifdim\fontdimen\ltx@one\font>0pt %
      \ltx@IfUndefined{rotatebox}{%
        \ltx@IfUndefined{pgftext}{%
          \ltx@IfUndefined{psscalebox}{%
            \ltx@IfUndefined{HOLOGO@ScaleBox@\hologoDriver}{%
            }{%
              \chardef\HOLOGO@choice=4 %
            }%
          }{%
            \chardef\HOLOGO@choice=3 %
          }%
        }{%
          \chardef\HOLOGO@choice=2 %
        }%
      }{%
        \chardef\HOLOGO@choice=1 %
      }%
      \ifcase\HOLOGO@choice
        \HOLOGO@WarningUnsupportedDriver{Xe}%
        e%
      \or % 1: \rotatebox
        \begingroup
          \setbox\ltx@zero\hbox{\rotatebox{180}{E}}%
          \ltx@LocDimenA=\dp\ltx@zero
          \advance\ltx@LocDimenA by -.5ex\relax
          \raise\ltx@LocDimenA\box\ltx@zero
        \endgroup
      \or % 2: \pgftext
        \lower.5ex\hbox{%
          \pgfpicture
            \pgftext[rotate=180]{E}%
          \endpgfpicture
        }%
      \or % 3: \psscalebox
        \begingroup
          \setbox\ltx@zero\hbox{\psscalebox{-1 -1}{E}}%
          \ltx@LocDimenA=\dp\ltx@zero
          \advance\ltx@LocDimenA by -.5ex\relax
          \raise\ltx@LocDimenA\box\ltx@zero
        \endgroup
      \or % 4: \HOLOGO@PointReflectBox
        \lower.5ex\hbox{\HOLOGO@PointReflectBox{E}}%
      \else
        \@PackageError{hologo}{Internal error (choice/it}\@ehc
      \fi
    \else
      \ltx@IfUndefined{reflectbox}{%
        \ltx@IfUndefined{pgftext}{%
          \ltx@IfUndefined{psscalebox}{%
            \ltx@IfUndefined{HOLOGO@ScaleBox@\hologoDriver}{%
            }{%
              \chardef\HOLOGO@choice=4 %
            }%
          }{%
            \chardef\HOLOGO@choice=3 %
          }%
        }{%
          \chardef\HOLOGO@choice=2 %
        }%
      }{%
        \chardef\HOLOGO@choice=1 %
      }%
      \ifcase\HOLOGO@choice
        \HOLOGO@WarningUnsupportedDriver{Xe}%
        e%
      \or % 1: reflectbox
        \lower.5ex\hbox{%
          \reflectbox{E}%
        }%
      \or % 2: \pgftext
        \lower.5ex\hbox{%
          \pgfpicture
            \pgftransformxscale{-1}%
            \pgftext{E}%
          \endpgfpicture
        }%
      \or % 3: \psscalebox
        \lower.5ex\hbox{%
          \psscalebox{-1 1}{E}%
        }%
      \or % 4: \HOLOGO@Reflectbox
        \lower.5ex\hbox{%
          \HOLOGO@ReflectBox{E}%
        }%
      \else
        \@PackageError{hologo}{Internal error (choice/up)}\@ehc
      \fi
    \fi
  }%
}
%    \end{macrocode}
%    \end{macro}
%    \begin{macro}{\HoLogoHtml@Xe}
%    \begin{macrocode}
\def\HoLogoHtml@Xe#1{%
  \HoLogoCss@Xe
  \HOLOGO@Span{Xe}{%
    X%
    \HOLOGO@Span{e}{%
      \HCode{&\ltx@hashchar x018e;}%
    }%
  }%
}
%    \end{macrocode}
%    \end{macro}
%    \begin{macro}{\HoLogoCss@Xe}
%    \begin{macrocode}
\def\HoLogoCss@Xe{%
  \Css{%
    span.HoLogo-Xe span.HoLogo-e{%
      position:relative;%
      top:.5ex;%
      left-margin:-.1em;%
    }%
  }%
  \global\let\HoLogoCss@Xe\relax
}
%    \end{macrocode}
%    \end{macro}
%
%    \begin{macro}{\HoLogo@XeTeX}
%    \begin{macrocode}
\def\HoLogo@XeTeX#1{%
  \hologo{Xe}%
  \kern-.15em\relax
  \hologo{TeX}%
}
%    \end{macrocode}
%    \end{macro}
%
%    \begin{macro}{\HoLogoHtml@XeTeX}
%    \begin{macrocode}
\def\HoLogoHtml@XeTeX#1{%
  \HoLogoCss@XeTeX
  \HOLOGO@Span{XeTeX}{%
    \hologo{Xe}%
    \hologo{TeX}%
  }%
}
%    \end{macrocode}
%    \end{macro}
%    \begin{macro}{\HoLogoCss@XeTeX}
%    \begin{macrocode}
\def\HoLogoCss@XeTeX{%
  \Css{%
    span.HoLogo-XeTeX span.HoLogo-TeX{%
      margin-left:-.15em;%
    }%
  }%
  \global\let\HoLogoCss@XeTeX\relax
}
%    \end{macrocode}
%    \end{macro}
%
%    \begin{macro}{\HoLogo@XeLaTeX}
%    \begin{macrocode}
\def\HoLogo@XeLaTeX#1{%
  \hologo{Xe}%
  \kern-.13em%
  \hologo{LaTeX}%
}
%    \end{macrocode}
%    \end{macro}
%    \begin{macro}{\HoLogoHtml@XeLaTeX}
%    \begin{macrocode}
\def\HoLogoHtml@XeLaTeX#1{%
  \HoLogoCss@XeLaTeX
  \HOLOGO@Span{XeLaTeX}{%
    \hologo{Xe}%
    \hologo{LaTeX}%
  }%
}
%    \end{macrocode}
%    \end{macro}
%    \begin{macro}{\HoLogoCss@XeLaTeX}
%    \begin{macrocode}
\def\HoLogoCss@XeLaTeX{%
  \Css{%
    span.HoLogo-XeLaTeX span.HoLogo-Xe{%
      margin-right:-.13em;%
    }%
  }%
  \global\let\HoLogoCss@XeLaTeX\relax
}
%    \end{macrocode}
%    \end{macro}
%
% \subsubsection{\hologo{pdfTeX}, \hologo{pdfLaTeX}}
%
%    \begin{macro}{\HoLogo@pdfTeX}
%    \begin{macrocode}
\def\HoLogo@pdfTeX#1{%
  \HOLOGO@mbox{%
    #1{p}{P}df\hologo{TeX}%
  }%
}
%    \end{macrocode}
%    \end{macro}
%    \begin{macro}{\HoLogoCs@pdfTeX}
%    \begin{macrocode}
\def\HoLogoCs@pdfTeX#1{#1{p}{P}dfTeX}
%    \end{macrocode}
%    \end{macro}
%    \begin{macro}{\HoLogoBkm@pdfTeX}
%    \begin{macrocode}
\def\HoLogoBkm@pdfTeX#1{%
  #1{p}{P}df\hologo{TeX}%
}
%    \end{macrocode}
%    \end{macro}
%    \begin{macro}{\HoLogoHtml@pdfTeX}
%    \begin{macrocode}
\let\HoLogoHtml@pdfTeX\HoLogo@pdfTeX
%    \end{macrocode}
%    \end{macro}
%
%    \begin{macro}{\HoLogo@pdfLaTeX}
%    \begin{macrocode}
\def\HoLogo@pdfLaTeX#1{%
  \HOLOGO@mbox{%
    #1{p}{P}df\hologo{LaTeX}%
  }%
}
%    \end{macrocode}
%    \end{macro}
%    \begin{macro}{\HoLogoCs@pdfLaTeX}
%    \begin{macrocode}
\def\HoLogoCs@pdfLaTeX#1{#1{p}{P}dfLaTeX}
%    \end{macrocode}
%    \end{macro}
%    \begin{macro}{\HoLogoBkm@pdfLaTeX}
%    \begin{macrocode}
\def\HoLogoBkm@pdfLaTeX#1{%
  #1{p}{P}df\hologo{LaTeX}%
}
%    \end{macrocode}
%    \end{macro}
%    \begin{macro}{\HoLogoHtml@pdfLaTeX}
%    \begin{macrocode}
\let\HoLogoHtml@pdfLaTeX\HoLogo@pdfLaTeX
%    \end{macrocode}
%    \end{macro}
%
% \subsubsection{\hologo{VTeX}}
%
%    \begin{macro}{\HoLogo@VTeX}
%    \begin{macrocode}
\def\HoLogo@VTeX#1{%
  \HOLOGO@mbox{%
    V\hologo{TeX}%
  }%
}
%    \end{macrocode}
%    \end{macro}
%    \begin{macro}{\HoLogoHtml@VTeX}
%    \begin{macrocode}
\let\HoLogoHtml@VTeX\HoLogo@VTeX
%    \end{macrocode}
%    \end{macro}
%
% \subsubsection{\hologo{AmS}, \dots}
%
%    Source: class \xclass{amsdtx}
%
%    \begin{macro}{\HoLogo@AmS}
%    \begin{macrocode}
\def\HoLogo@AmS#1{%
  \HoLogoFont@font{AmS}{sy}{%
    A%
    \kern-.1667em%
    \lower.5ex\hbox{M}%
    \kern-.125em%
    S%
  }%
}
%    \end{macrocode}
%    \end{macro}
%    \begin{macro}{\HoLogoBkm@AmS}
%    \begin{macrocode}
\def\HoLogoBkm@AmS#1{AmS}
%    \end{macrocode}
%    \end{macro}
%    \begin{macro}{\HoLogoHtml@AmS}
%    \begin{macrocode}
\def\HoLogoHtml@AmS#1{%
  \HoLogoCss@AmS
%  \HoLogoFont@font{AmS}{sy}{%
    \HOLOGO@Span{AmS}{%
      A%
      \HOLOGO@Span{M}{M}%
      S%
    }%
%   }%
}
%    \end{macrocode}
%    \end{macro}
%    \begin{macro}{\HoLogoCss@AmS}
%    \begin{macrocode}
\def\HoLogoCss@AmS{%
  \Css{%
    span.HoLogo-AmS span.HoLogo-M{%
      position:relative;%
      top:.5ex;%
      margin-left:-.1667em;%
      margin-right:-.125em;%
      text-decoration:none;%
    }%
  }%
  \global\let\HoLogoCss@AmS\relax
}
%    \end{macrocode}
%    \end{macro}
%
%    \begin{macro}{\HoLogo@AmSTeX}
%    \begin{macrocode}
\def\HoLogo@AmSTeX#1{%
  \hologo{AmS}%
  \HOLOGO@hyphen
  \hologo{TeX}%
}
%    \end{macrocode}
%    \end{macro}
%    \begin{macro}{\HoLogoBkm@AmSTeX}
%    \begin{macrocode}
\def\HoLogoBkm@AmSTeX#1{AmS-TeX}%
%    \end{macrocode}
%    \end{macro}
%    \begin{macro}{\HoLogoHtml@AmSTeX}
%    \begin{macrocode}
\let\HoLogoHtml@AmSTeX\HoLogo@AmSTeX
%    \end{macrocode}
%    \end{macro}
%
%    \begin{macro}{\HoLogo@AmSLaTeX}
%    \begin{macrocode}
\def\HoLogo@AmSLaTeX#1{%
  \hologo{AmS}%
  \HOLOGO@hyphen
  \hologo{LaTeX}%
}
%    \end{macrocode}
%    \end{macro}
%    \begin{macro}{\HoLogoBkm@AmSLaTeX}
%    \begin{macrocode}
\def\HoLogoBkm@AmSLaTeX#1{AmS-LaTeX}%
%    \end{macrocode}
%    \end{macro}
%    \begin{macro}{\HoLogoHtml@AmSLaTeX}
%    \begin{macrocode}
\let\HoLogoHtml@AmSLaTeX\HoLogo@AmSLaTeX
%    \end{macrocode}
%    \end{macro}
%
% \subsubsection{\hologo{BibTeX}}
%
%    \begin{macro}{\HoLogo@BibTeX@sc}
%    A definition of \hologo{BibTeX} is provided in
%    the documentation source for the manual of \hologo{BibTeX}
%    \cite{btxdoc}.
%\begin{quote}
%\begin{verbatim}
%\def\BibTeX{%
%  {%
%    \rm
%    B%
%    \kern-.05em%
%    {%
%      \sc
%      i%
%      \kern-.025em %
%      b%
%    }%
%    \kern-.08em
%    T%
%    \kern-.1667em%
%    \lower.7ex\hbox{E}%
%    \kern-.125em%
%    X%
%  }%
%}
%\end{verbatim}
%\end{quote}
%    \begin{macrocode}
\def\HoLogo@BibTeX@sc#1{%
  B%
  \kern-.05em%
  \HoLogoFont@font{BibTeX}{sc}{%
    i%
    \kern-.025em%
    b%
  }%
  \HOLOGO@discretionary
  \kern-.08em%
  \hologo{TeX}%
}
%    \end{macrocode}
%    \end{macro}
%    \begin{macro}{\HoLogoHtml@BibTeX@sc}
%    \begin{macrocode}
\def\HoLogoHtml@BibTeX@sc#1{%
  \HoLogoCss@BibTeX@sc
  \HOLOGO@Span{BibTeX-sc}{%
    B%
    \HOLOGO@Span{i}{i}%
    \HOLOGO@Span{b}{b}%
    \hologo{TeX}%
  }%
}
%    \end{macrocode}
%    \end{macro}
%    \begin{macro}{\HoLogoCss@BibTeX@sc}
%    \begin{macrocode}
\def\HoLogoCss@BibTeX@sc{%
  \Css{%
    span.HoLogo-BibTeX-sc span.HoLogo-i{%
      margin-left:-.05em;%
      margin-right:-.025em;%
      font-variant:small-caps;%
    }%
  }%
  \Css{%
    span.HoLogo-BibTeX-sc span.HoLogo-b{%
      margin-right:-.08em;%
      font-variant:small-caps;%
    }%
  }%
  \global\let\HoLogoCss@BibTeX@sc\relax
}
%    \end{macrocode}
%    \end{macro}
%
%    \begin{macro}{\HoLogo@BibTeX@sf}
%    Variant \xoption{sf} avoids trouble with unavailable
%    small caps fonts (e.g., bold versions of Computer Modern or
%    Latin Modern). The definition is taken from
%    package \xpackage{dtklogos} \cite{dtklogos}.
%\begin{quote}
%\begin{verbatim}
%\DeclareRobustCommand{\BibTeX}{%
%  B%
%  \kern-.05em%
%  \hbox{%
%    $\m@th$% %% force math size calculations
%    \csname S@\f@size\endcsname
%    \fontsize\sf@size\z@
%    \math@fontsfalse
%    \selectfont
%    I%
%    \kern-.025em%
%    B
%  }%
%  \kern-.08em%
%  \-%
%  \TeX
%}
%\end{verbatim}
%\end{quote}
%    \begin{macrocode}
\def\HoLogo@BibTeX@sf#1{%
  B%
  \kern-.05em%
  \HoLogoFont@font{BibTeX}{bibsf}{%
    I%
    \kern-.025em%
    B%
  }%
  \HOLOGO@discretionary
  \kern-.08em%
  \hologo{TeX}%
}
%    \end{macrocode}
%    \end{macro}
%    \begin{macro}{\HoLogoHtml@BibTeX@sf}
%    \begin{macrocode}
\def\HoLogoHtml@BibTeX@sf#1{%
  \HoLogoCss@BibTeX@sf
  \HOLOGO@Span{BibTeX-sf}{%
    B%
    \HoLogoFont@font{BibTeX}{bibsf}{%
      \HOLOGO@Span{i}{I}%
      B%
    }%
    \hologo{TeX}%
  }%
}
%    \end{macrocode}
%    \end{macro}
%    \begin{macro}{\HoLogoCss@BibTeX@sf}
%    \begin{macrocode}
\def\HoLogoCss@BibTeX@sf{%
  \Css{%
    span.HoLogo-BibTeX-sf span.HoLogo-i{%
      margin-left:-.05em;%
      margin-right:-.025em;%
    }%
  }%
  \Css{%
    span.HoLogo-BibTeX-sf span.HoLogo-TeX{%
      margin-left:-.08em;%
    }%
  }%
  \global\let\HoLogoCss@BibTeX@sf\relax
}
%    \end{macrocode}
%    \end{macro}
%
%    \begin{macro}{\HoLogo@BibTeX}
%    \begin{macrocode}
\def\HoLogo@BibTeX{\HoLogo@BibTeX@sf}
%    \end{macrocode}
%    \end{macro}
%    \begin{macro}{\HoLogoHtml@BibTeX}
%    \begin{macrocode}
\def\HoLogoHtml@BibTeX{\HoLogoHtml@BibTeX@sf}
%    \end{macrocode}
%    \end{macro}
%
% \subsubsection{\hologo{BibTeX8}}
%
%    \begin{macro}{\HoLogo@BibTeX8}
%    \begin{macrocode}
\expandafter\def\csname HoLogo@BibTeX8\endcsname#1{%
  \hologo{BibTeX}%
  8%
}
%    \end{macrocode}
%    \end{macro}
%
%    \begin{macro}{\HoLogoBkm@BibTeX8}
%    \begin{macrocode}
\expandafter\def\csname HoLogoBkm@BibTeX8\endcsname#1{%
  \hologo{BibTeX}%
  8%
}
%    \end{macrocode}
%    \end{macro}
%    \begin{macro}{\HoLogoHtml@BibTeX8}
%    \begin{macrocode}
\expandafter
\let\csname HoLogoHtml@BibTeX8\expandafter\endcsname
\csname HoLogo@BibTeX8\endcsname
%    \end{macrocode}
%    \end{macro}
%
% \subsubsection{\hologo{ConTeXt}}
%
%    \begin{macro}{\HoLogo@ConTeXt@simple}
%    \begin{macrocode}
\def\HoLogo@ConTeXt@simple#1{%
  \HOLOGO@mbox{Con}%
  \HOLOGO@discretionary
  \HOLOGO@mbox{\hologo{TeX}t}%
}
%    \end{macrocode}
%    \end{macro}
%    \begin{macro}{\HoLogoHtml@ConTeXt@simple}
%    \begin{macrocode}
\let\HoLogoHtml@ConTeXt@simple\HoLogo@ConTeXt@simple
%    \end{macrocode}
%    \end{macro}
%
%    \begin{macro}{\HoLogo@ConTeXt@narrow}
%    This definition of logo \hologo{ConTeXt} with variant \xoption{narrow}
%    comes from TUGboat's class \xclass{ltugboat} (version 2010/11/15 v2.8).
%    \begin{macrocode}
\def\HoLogo@ConTeXt@narrow#1{%
  \HOLOGO@mbox{C\kern-.0333emon}%
  \HOLOGO@discretionary
  \kern-.0667em%
  \HOLOGO@mbox{\hologo{TeX}\kern-.0333emt}%
}
%    \end{macrocode}
%    \end{macro}
%    \begin{macro}{\HoLogoHtml@ConTeXt@narrow}
%    \begin{macrocode}
\def\HoLogoHtml@ConTeXt@narrow#1{%
  \HoLogoCss@ConTeXt@narrow
  \HOLOGO@Span{ConTeXt-narrow}{%
    \HOLOGO@Span{C}{C}%
    on%
    \hologo{TeX}%
    t%
  }%
}
%    \end{macrocode}
%    \end{macro}
%    \begin{macro}{\HoLogoCss@ConTeXt@narrow}
%    \begin{macrocode}
\def\HoLogoCss@ConTeXt@narrow{%
  \Css{%
    span.HoLogo-ConTeXt-narrow span.HoLogo-C{%
      margin-left:-.0333em;%
    }%
  }%
  \Css{%
    span.HoLogo-ConTeXt-narrow span.HoLogo-TeX{%
      margin-left:-.0667em;%
      margin-right:-.0333em;%
    }%
  }%
  \global\let\HoLogoCss@ConTeXt@narrow\relax
}
%    \end{macrocode}
%    \end{macro}
%
%    \begin{macro}{\HoLogo@ConTeXt}
%    \begin{macrocode}
\def\HoLogo@ConTeXt{\HoLogo@ConTeXt@narrow}
%    \end{macrocode}
%    \end{macro}
%    \begin{macro}{\HoLogoHtml@ConTeXt}
%    \begin{macrocode}
\def\HoLogoHtml@ConTeXt{\HoLogoHtml@ConTeXt@narrow}
%    \end{macrocode}
%    \end{macro}
%
% \subsubsection{\hologo{emTeX}}
%
%    \begin{macro}{\HoLogo@emTeX}
%    \begin{macrocode}
\def\HoLogo@emTeX#1{%
  \HOLOGO@mbox{#1{e}{E}m}%
  \HOLOGO@discretionary
  \hologo{TeX}%
}
%    \end{macrocode}
%    \end{macro}
%    \begin{macro}{\HoLogoCs@emTeX}
%    \begin{macrocode}
\def\HoLogoCs@emTeX#1{#1{e}{E}mTeX}%
%    \end{macrocode}
%    \end{macro}
%    \begin{macro}{\HoLogoBkm@emTeX}
%    \begin{macrocode}
\def\HoLogoBkm@emTeX#1{%
  #1{e}{E}m\hologo{TeX}%
}
%    \end{macrocode}
%    \end{macro}
%    \begin{macro}{\HoLogoHtml@emTeX}
%    \begin{macrocode}
\let\HoLogoHtml@emTeX\HoLogo@emTeX
%    \end{macrocode}
%    \end{macro}
%
% \subsubsection{\hologo{ExTeX}}
%
%    \begin{macro}{\HoLogo@ExTeX}
%    The definition is taken from the FAQ of the
%    project \hologo{ExTeX}
%    \cite{ExTeX-FAQ}.
%\begin{quote}
%\begin{verbatim}
%\def\ExTeX{%
%  \textrm{% Logo always with serifs
%    \ensuremath{%
%      \textstyle
%      \varepsilon_{%
%        \kern-0.15em%
%        \mathcal{X}%
%      }%
%    }%
%    \kern-.15em%
%    \TeX
%  }%
%}
%\end{verbatim}
%\end{quote}
%    \begin{macrocode}
\def\HoLogo@ExTeX#1{%
  \HoLogoFont@font{ExTeX}{rm}{%
    \ltx@mbox{%
      \HOLOGO@MathSetup
      $%
        \textstyle
        \varepsilon_{%
          \kern-0.15em%
          \HoLogoFont@font{ExTeX}{sy}{X}%
        }%
      $%
    }%
    \HOLOGO@discretionary
    \kern-.15em%
    \hologo{TeX}%
  }%
}
%    \end{macrocode}
%    \end{macro}
%    \begin{macro}{\HoLogoHtml@ExTeX}
%    \begin{macrocode}
\def\HoLogoHtml@ExTeX#1{%
  \HoLogoCss@ExTeX
  \HoLogoFont@font{ExTeX}{rm}{%
    \HOLOGO@Span{ExTeX}{%
      \ltx@mbox{%
        \HOLOGO@MathSetup
        $\textstyle\varepsilon$%
        \HOLOGO@Span{X}{$\textstyle\chi$}%
        \hologo{TeX}%
      }%
    }%
  }%
}
%    \end{macrocode}
%    \end{macro}
%    \begin{macro}{\HoLogoBkm@ExTeX}
%    \begin{macrocode}
\def\HoLogoBkm@ExTeX#1{%
  \HOLOGO@PdfdocUnicode{#1{e}{E}x}{\textepsilon\textchi}%
  \hologo{TeX}%
}
%    \end{macrocode}
%    \end{macro}
%    \begin{macro}{\HoLogoCss@ExTeX}
%    \begin{macrocode}
\def\HoLogoCss@ExTeX{%
  \Css{%
    span.HoLogo-ExTeX{%
      font-family:serif;%
    }%
  }%
  \Css{%
    span.HoLogo-ExTeX span.HoLogo-TeX{%
      margin-left:-.15em;%
    }%
  }%
  \global\let\HoLogoCss@ExTeX\relax
}
%    \end{macrocode}
%    \end{macro}
%
% \subsubsection{\hologo{MiKTeX}}
%
%    \begin{macro}{\HoLogo@MiKTeX}
%    \begin{macrocode}
\def\HoLogo@MiKTeX#1{%
  \HOLOGO@mbox{MiK}%
  \HOLOGO@discretionary
  \hologo{TeX}%
}
%    \end{macrocode}
%    \end{macro}
%    \begin{macro}{\HoLogoHtml@MiKTeX}
%    \begin{macrocode}
\let\HoLogoHtml@MiKTeX\HoLogo@MiKTeX
%    \end{macrocode}
%    \end{macro}
%
% \subsubsection{\hologo{OzTeX} and friends}
%
%    Source: \hologo{OzTeX} FAQ \cite{OzTeX}:
%    \begin{quote}
%      |\def\OzTeX{O\kern-.03em z\kern-.15em\TeX}|\\
%      (There is no kerning in OzMF, OzMP and OzTtH.)
%    \end{quote}
%
%    \begin{macro}{\HoLogo@OzTeX}
%    \begin{macrocode}
\def\HoLogo@OzTeX#1{%
  O%
  \kern-.03em %
  z%
  \kern-.15em %
  \hologo{TeX}%
}
%    \end{macrocode}
%    \end{macro}
%    \begin{macro}{\HoLogoHtml@OzTeX}
%    \begin{macrocode}
\def\HoLogoHtml@OzTeX#1{%
  \HoLogoCss@OzTeX
  \HOLOGO@Span{OzTeX}{%
    O%
    \HOLOGO@Span{z}{z}%
    \hologo{TeX}%
  }%
}
%    \end{macrocode}
%    \end{macro}
%    \begin{macro}{\HoLogoCss@OzTeX}
%    \begin{macrocode}
\def\HoLogoCss@OzTeX{%
  \Css{%
    span.HoLogo-OzTeX span.HoLogo-z{%
      margin-left:-.03em;%
      margin-right:-.15em;%
    }%
  }%
  \global\let\HoLogoCss@OzTeX\relax
}
%    \end{macrocode}
%    \end{macro}
%
%    \begin{macro}{\HoLogo@OzMF}
%    \begin{macrocode}
\def\HoLogo@OzMF#1{%
  \HOLOGO@mbox{OzMF}%
}
%    \end{macrocode}
%    \end{macro}
%    \begin{macro}{\HoLogo@OzMP}
%    \begin{macrocode}
\def\HoLogo@OzMP#1{%
  \HOLOGO@mbox{OzMP}%
}
%    \end{macrocode}
%    \end{macro}
%    \begin{macro}{\HoLogo@OzTtH}
%    \begin{macrocode}
\def\HoLogo@OzTtH#1{%
  \HOLOGO@mbox{OzTtH}%
}
%    \end{macrocode}
%    \end{macro}
%
% \subsubsection{\hologo{PCTeX}}
%
%    \begin{macro}{\HoLogo@PCTeX}
%    \begin{macrocode}
\def\HoLogo@PCTeX#1{%
  \HOLOGO@mbox{PC}%
  \hologo{TeX}%
}
%    \end{macrocode}
%    \end{macro}
%    \begin{macro}{\HoLogoHtml@PCTeX}
%    \begin{macrocode}
\let\HoLogoHtml@PCTeX\HoLogo@PCTeX
%    \end{macrocode}
%    \end{macro}
%
% \subsubsection{\hologo{PiCTeX}}
%
%    The original definitions from \xfile{pictex.tex} \cite{PiCTeX}:
%\begin{quote}
%\begin{verbatim}
%\def\PiC{%
%  P%
%  \kern-.12em%
%  \lower.5ex\hbox{I}%
%  \kern-.075em%
%  C%
%}
%\def\PiCTeX{%
%  \PiC
%  \kern-.11em%
%  \TeX
%}
%\end{verbatim}
%\end{quote}
%
%    \begin{macro}{\HoLogo@PiC}
%    \begin{macrocode}
\def\HoLogo@PiC#1{%
  P%
  \kern-.12em%
  \lower.5ex\hbox{I}%
  \kern-.075em%
  C%
  \HOLOGO@SpaceFactor
}
%    \end{macrocode}
%    \end{macro}
%    \begin{macro}{\HoLogoHtml@PiC}
%    \begin{macrocode}
\def\HoLogoHtml@PiC#1{%
  \HoLogoCss@PiC
  \HOLOGO@Span{PiC}{%
    P%
    \HOLOGO@Span{i}{I}%
    C%
  }%
}
%    \end{macrocode}
%    \end{macro}
%    \begin{macro}{\HoLogoCss@PiC}
%    \begin{macrocode}
\def\HoLogoCss@PiC{%
  \Css{%
    span.HoLogo-PiC span.HoLogo-i{%
      position:relative;%
      top:.5ex;%
      margin-left:-.12em;%
      margin-right:-.075em;%
      text-decoration:none;%
    }%
  }%
  \global\let\HoLogoCss@PiC\relax
}
%    \end{macrocode}
%    \end{macro}
%
%    \begin{macro}{\HoLogo@PiCTeX}
%    \begin{macrocode}
\def\HoLogo@PiCTeX#1{%
  \hologo{PiC}%
  \HOLOGO@discretionary
  \kern-.11em%
  \hologo{TeX}%
}
%    \end{macrocode}
%    \end{macro}
%    \begin{macro}{\HoLogoHtml@PiCTeX}
%    \begin{macrocode}
\def\HoLogoHtml@PiCTeX#1{%
  \HoLogoCss@PiCTeX
  \HOLOGO@Span{PiCTeX}{%
    \hologo{PiC}%
    \hologo{TeX}%
  }%
}
%    \end{macrocode}
%    \end{macro}
%    \begin{macro}{\HoLogoCss@PiCTeX}
%    \begin{macrocode}
\def\HoLogoCss@PiCTeX{%
  \Css{%
    span.HoLogo-PiCTeX span.HoLogo-PiC{%
      margin-right:-.11em;%
    }%
  }%
  \global\let\HoLogoCss@PiCTeX\relax
}
%    \end{macrocode}
%    \end{macro}
%
% \subsubsection{\hologo{teTeX}}
%
%    \begin{macro}{\HoLogo@teTeX}
%    \begin{macrocode}
\def\HoLogo@teTeX#1{%
  \HOLOGO@mbox{#1{t}{T}e}%
  \HOLOGO@discretionary
  \hologo{TeX}%
}
%    \end{macrocode}
%    \end{macro}
%    \begin{macro}{\HoLogoCs@teTeX}
%    \begin{macrocode}
\def\HoLogoCs@teTeX#1{#1{t}{T}dfTeX}
%    \end{macrocode}
%    \end{macro}
%    \begin{macro}{\HoLogoBkm@teTeX}
%    \begin{macrocode}
\def\HoLogoBkm@teTeX#1{%
  #1{t}{T}e\hologo{TeX}%
}
%    \end{macrocode}
%    \end{macro}
%    \begin{macro}{\HoLogoHtml@teTeX}
%    \begin{macrocode}
\let\HoLogoHtml@teTeX\HoLogo@teTeX
%    \end{macrocode}
%    \end{macro}
%
% \subsubsection{\hologo{TeX4ht}}
%
%    \begin{macro}{\HoLogo@TeX4ht}
%    \begin{macrocode}
\expandafter\def\csname HoLogo@TeX4ht\endcsname#1{%
  \HOLOGO@mbox{\hologo{TeX}4ht}%
}
%    \end{macrocode}
%    \end{macro}
%    \begin{macro}{\HoLogoHtml@TeX4ht}
%    \begin{macrocode}
\expandafter
\let\csname HoLogoHtml@TeX4ht\expandafter\endcsname
\csname HoLogo@TeX4ht\endcsname
%    \end{macrocode}
%    \end{macro}
%
%
% \subsubsection{\hologo{SageTeX}}
%
%    \begin{macro}{\HoLogo@SageTeX}
%    \begin{macrocode}
\def\HoLogo@SageTeX#1{%
  \HOLOGO@mbox{Sage}%
  \HOLOGO@discretionary
  \HOLOGO@NegativeKerning{eT,oT,To}%
  \hologo{TeX}%
}
%    \end{macrocode}
%    \end{macro}
%    \begin{macro}{\HoLogoHtml@SageTeX}
%    \begin{macrocode}
\let\HoLogoHtml@SageTeX\HoLogo@SageTeX
%    \end{macrocode}
%    \end{macro}
%
% \subsection{\hologo{METAFONT} and friends}
%
%    \begin{macro}{\HoLogo@METAFONT}
%    \begin{macrocode}
\def\HoLogo@METAFONT#1{%
  \HoLogoFont@font{METAFONT}{logo}{%
    \HOLOGO@mbox{META}%
    \HOLOGO@discretionary
    \HOLOGO@mbox{FONT}%
  }%
}
%    \end{macrocode}
%    \end{macro}
%
%    \begin{macro}{\HoLogo@METAPOST}
%    \begin{macrocode}
\def\HoLogo@METAPOST#1{%
  \HoLogoFont@font{METAPOST}{logo}{%
    \HOLOGO@mbox{META}%
    \HOLOGO@discretionary
    \HOLOGO@mbox{POST}%
  }%
}
%    \end{macrocode}
%    \end{macro}
%
%    \begin{macro}{\HoLogo@MetaFun}
%    \begin{macrocode}
\def\HoLogo@MetaFun#1{%
  \HOLOGO@mbox{Meta}%
  \HOLOGO@discretionary
  \HOLOGO@mbox{Fun}%
}
%    \end{macrocode}
%    \end{macro}
%
%    \begin{macro}{\HoLogo@MetaPost}
%    \begin{macrocode}
\def\HoLogo@MetaPost#1{%
  \HOLOGO@mbox{Meta}%
  \HOLOGO@discretionary
  \HOLOGO@mbox{Post}%
}
%    \end{macrocode}
%    \end{macro}
%
% \subsection{Others}
%
% \subsubsection{\hologo{biber}}
%
%    \begin{macro}{\HoLogo@biber}
%    \begin{macrocode}
\def\HoLogo@biber#1{%
  \HOLOGO@mbox{#1{b}{B}i}%
  \HOLOGO@discretionary
  \HOLOGO@mbox{ber}%
}
%    \end{macrocode}
%    \end{macro}
%    \begin{macro}{\HoLogoCs@biber}
%    \begin{macrocode}
\def\HoLogoCs@biber#1{#1{b}{B}iber}
%    \end{macrocode}
%    \end{macro}
%    \begin{macro}{\HoLogoBkm@biber}
%    \begin{macrocode}
\def\HoLogoBkm@biber#1{%
  #1{b}{B}iber%
}
%    \end{macrocode}
%    \end{macro}
%    \begin{macro}{\HoLogoHtml@biber}
%    \begin{macrocode}
\let\HoLogoHtml@biber\HoLogo@biber
%    \end{macrocode}
%    \end{macro}
%
% \subsubsection{\hologo{KOMAScript}}
%
%    \begin{macro}{\HoLogo@KOMAScript}
%    The definition for \hologo{KOMAScript} is taken
%    from \hologo{KOMAScript} (\xfile{scrlogo.dtx}, reformatted) \cite{scrlogo}:
%\begin{quote}
%\begin{verbatim}
%\@ifundefined{KOMAScript}{%
%  \DeclareRobustCommand{\KOMAScript}{%
%    \textsf{%
%      K\kern.05em O\kern.05emM\kern.05em A%
%      \kern.1em-\kern.1em %
%      Script%
%    }%
%  }%
%}{}
%\end{verbatim}
%\end{quote}
%    \begin{macrocode}
\def\HoLogo@KOMAScript#1{%
  \HoLogoFont@font{KOMAScript}{sf}{%
    \HOLOGO@mbox{%
      K\kern.05em%
      O\kern.05em%
      M\kern.05em%
      A%
    }%
    \kern.1em%
    \HOLOGO@hyphen
    \kern.1em%
    \HOLOGO@mbox{Script}%
  }%
}
%    \end{macrocode}
%    \end{macro}
%    \begin{macro}{\HoLogoBkm@KOMAScript}
%    \begin{macrocode}
\def\HoLogoBkm@KOMAScript#1{%
  KOMA-Script%
}
%    \end{macrocode}
%    \end{macro}
%    \begin{macro}{\HoLogoHtml@KOMAScript}
%    \begin{macrocode}
\def\HoLogoHtml@KOMAScript#1{%
  \HoLogoCss@KOMAScript
  \HoLogoFont@font{KOMAScript}{sf}{%
    \HOLOGO@Span{KOMAScript}{%
      K%
      \HOLOGO@Span{O}{O}%
      M%
      \HOLOGO@Span{A}{A}%
      \HOLOGO@Span{hyphen}{-}%
      Script%
    }%
  }%
}
%    \end{macrocode}
%    \end{macro}
%    \begin{macro}{\HoLogoCss@KOMAScript}
%    \begin{macrocode}
\def\HoLogoCss@KOMAScript{%
  \Css{%
    span.HoLogo-KOMAScript{%
      font-family:sans-serif;%
    }%
  }%
  \Css{%
    span.HoLogo-KOMAScript span.HoLogo-O{%
      padding-left:.05em;%
      padding-right:.05em;%
    }%
  }%
  \Css{%
    span.HoLogo-KOMAScript span.HoLogo-A{%
      padding-left:.05em;%
    }%
  }%
  \Css{%
    span.HoLogo-KOMAScript span.HoLogo-hyphen{%
      padding-left:.1em;%
      padding-right:.1em;%
    }%
  }%
  \global\let\HoLogoCss@KOMAScript\relax
}
%    \end{macrocode}
%    \end{macro}
%
% \subsubsection{\hologo{LyX}}
%
%    \begin{macro}{\HoLogo@LyX}
%    The definition is taken from the documentation source files
%    of \hologo{LyX}, \xfile{Intro.lyx} \cite{LyX}:
%\begin{quote}
%\begin{verbatim}
%\def\LyX{%
%  \texorpdfstring{%
%    L\kern-.1667em\lower.25em\hbox{Y}\kern-.125emX\@%
%  }{%
%    LyX%
%  }%
%}
%\end{verbatim}
%\end{quote}
%    \begin{macrocode}
\def\HoLogo@LyX#1{%
  L%
  \kern-.1667em%
  \lower.25em\hbox{Y}%
  \kern-.125em%
  X%
  \HOLOGO@SpaceFactor
}
%    \end{macrocode}
%    \end{macro}
%    \begin{macro}{\HoLogoHtml@LyX}
%    \begin{macrocode}
\def\HoLogoHtml@LyX#1{%
  \HoLogoCss@LyX
  \HOLOGO@Span{LyX}{%
    L%
    \HOLOGO@Span{y}{Y}%
    X%
  }%
}
%    \end{macrocode}
%    \end{macro}
%    \begin{macro}{\HoLogoCss@LyX}
%    \begin{macrocode}
\def\HoLogoCss@LyX{%
  \Css{%
    span.HoLogo-LyX span.HoLogo-y{%
      position:relative;%
      top:.25em;%
      margin-left:-.1667em;%
      margin-right:-.125em;%
      text-decoration:none;%
    }%
  }%
  \global\let\HoLogoCss@LyX\relax
}
%    \end{macrocode}
%    \end{macro}
%
% \subsubsection{\hologo{NTS}}
%
%    \begin{macro}{\HoLogo@NTS}
%    Definition for \hologo{NTS} can be found in
%    package \xpackage{etex\textunderscore man} for the \hologo{eTeX} manual \cite{etexman}
%    and in package \xpackage{dtklogos} \cite{dtklogos}:
%\begin{quote}
%\begin{verbatim}
%\def\NTS{%
%  \leavevmode
%  \hbox{%
%    $%
%      \cal N%
%      \kern-0.35em%
%      \lower0.5ex\hbox{$\cal T$}%
%      \kern-0.2em%
%      S%
%    $%
%  }%
%}
%\end{verbatim}
%\end{quote}
%    \begin{macrocode}
\def\HoLogo@NTS#1{%
  \HoLogoFont@font{NTS}{sy}{%
    N\/%
    \kern-.35em%
    \lower.5ex\hbox{T\/}%
    \kern-.2em%
    S\/%
  }%
  \HOLOGO@SpaceFactor
}
%    \end{macrocode}
%    \end{macro}
%
% \subsubsection{\Hologo{TTH} (\hologo{TeX} to HTML translator)}
%
%    Source: \url{http://hutchinson.belmont.ma.us/tth/}
%    In the HTML source the second `T' is printed as subscript.
%\begin{quote}
%\begin{verbatim}
%T<sub>T</sub>H
%\end{verbatim}
%\end{quote}
%    \begin{macro}{\HoLogo@TTH}
%    \begin{macrocode}
\def\HoLogo@TTH#1{%
  \ltx@mbox{%
    T\HOLOGO@SubScript{T}H%
  }%
  \HOLOGO@SpaceFactor
}
%    \end{macrocode}
%    \end{macro}
%
%    \begin{macro}{\HoLogoHtml@TTH}
%    \begin{macrocode}
\def\HoLogoHtml@TTH#1{%
  T\HCode{<sub>}T\HCode{</sub>}H%
}
%    \end{macrocode}
%    \end{macro}
%
% \subsubsection{\Hologo{HanTheThanh}}
%
%    Partial source: Package \xpackage{dtklogos}.
%    The double accent is U+1EBF (latin small letter e with circumflex
%    and acute).
%    \begin{macro}{\HoLogo@HanTheThanh}
%    \begin{macrocode}
\def\HoLogo@HanTheThanh#1{%
  \ltx@mbox{H\`an}%
  \HOLOGO@space
  \ltx@mbox{%
    Th%
    \HOLOGO@IfCharExists{"1EBF}{%
      \char"1EBF\relax
    }{%
      \^e\hbox to 0pt{\hss\raise .5ex\hbox{\'{}}}%
    }%
  }%
  \HOLOGO@space
  \ltx@mbox{Th\`anh}%
}
%    \end{macrocode}
%    \end{macro}
%    \begin{macro}{\HoLogoBkm@HanTheThanh}
%    \begin{macrocode}
\def\HoLogoBkm@HanTheThanh#1{%
  H\`an %
  Th\HOLOGO@PdfdocUnicode{\^e}{\9036\277} %
  Th\`anh%
}
%    \end{macrocode}
%    \end{macro}
%    \begin{macro}{\HoLogoHtml@HanTheThanh}
%    \begin{macrocode}
\def\HoLogoHtml@HanTheThanh#1{%
  H\`an %
  Th\HCode{&\ltx@hashchar x1ebf;} %
  Th\`anh%
}
%    \end{macrocode}
%    \end{macro}
%
% \subsection{Driver detection}
%
%    \begin{macrocode}
\HOLOGO@IfExists\InputIfFileExists{%
  \InputIfFileExists{hologo.cfg}{}{}%
}{%
  \ltx@IfUndefined{pdf@filesize}{%
    \def\HOLOGO@InputIfExists{%
      \openin\HOLOGO@temp=hologo.cfg\relax
      \ifeof\HOLOGO@temp
        \closein\HOLOGO@temp
      \else
        \closein\HOLOGO@temp
        \begingroup
          \def\x{LaTeX2e}%
        \expandafter\endgroup
        \ifx\fmtname\x
          \input{hologo.cfg}%
        \else
          \input hologo.cfg\relax
        \fi
      \fi
    }%
    \ltx@IfUndefined{newread}{%
      \chardef\HOLOGO@temp=15 %
      \def\HOLOGO@CheckRead{%
        \ifeof\HOLOGO@temp
          \HOLOGO@InputIfExists
        \else
          \ifcase\HOLOGO@temp
            \@PackageWarningNoLine{hologo}{%
              Configuration file ignored, because\MessageBreak
              a free read register could not be found%
            }%
          \else
            \begingroup
              \count\ltx@cclv=\HOLOGO@temp
              \advance\ltx@cclv by \ltx@minusone
              \edef\x{\endgroup
                \chardef\noexpand\HOLOGO@temp=\the\count\ltx@cclv
                \relax
              }%
            \x
          \fi
        \fi
      }%
    }{%
      \csname newread\endcsname\HOLOGO@temp
      \HOLOGO@InputIfExists
    }%
  }{%
    \edef\HOLOGO@temp{\pdf@filesize{hologo.cfg}}%
    \ifx\HOLOGO@temp\ltx@empty
    \else
      \ifnum\HOLOGO@temp>0 %
        \begingroup
          \def\x{LaTeX2e}%
        \expandafter\endgroup
        \ifx\fmtname\x
          \input{hologo.cfg}%
        \else
          \input hologo.cfg\relax
        \fi
      \else
        \@PackageInfoNoLine{hologo}{%
          Empty configuration file `hologo.cfg' ignored%
        }%
      \fi
    \fi
  }%
}
%    \end{macrocode}
%
%    \begin{macrocode}
\def\HOLOGO@temp#1#2{%
  \kv@define@key{HoLogoDriver}{#1}[]{%
    \begingroup
      \def\HOLOGO@temp{##1}%
      \ltx@onelevel@sanitize\HOLOGO@temp
      \ifx\HOLOGO@temp\ltx@empty
      \else
        \@PackageError{hologo}{%
          Value (\HOLOGO@temp) not permitted for option `#1'%
        }%
        \@ehc
      \fi
    \endgroup
    \def\hologoDriver{#2}%
  }%
}%
\def\HOLOGO@@temp#1#2{%
  \ifx\kv@value\relax
    \HOLOGO@temp{#1}{#1}%
  \else
    \HOLOGO@temp{#1}{#2}%
  \fi
}%
\kv@parse@normalized{%
  pdftex,%
  luatex=pdftex,%
  dvipdfm,%
  dvipdfmx=dvipdfm,%
  dvips,%
  dvipsone=dvips,%
  xdvi=dvips,%
  xetex,%
  vtex,%
}\HOLOGO@@temp
%    \end{macrocode}
%
%    \begin{macrocode}
\kv@define@key{HoLogoDriver}{driverfallback}{%
  \def\HOLOGO@DriverFallback{#1}%
}
%    \end{macrocode}
%
%    \begin{macro}{\HOLOGO@DriverFallback}
%    \begin{macrocode}
\def\HOLOGO@DriverFallback{dvips}
%    \end{macrocode}
%    \end{macro}
%
%    \begin{macro}{\hologoDriverSetup}
%    \begin{macrocode}
\def\hologoDriverSetup{%
  \let\hologoDriver\ltx@undefined
  \HOLOGO@DriverSetup
}
%    \end{macrocode}
%    \end{macro}
%
%    \begin{macro}{\HOLOGO@DriverSetup}
%    \begin{macrocode}
\def\HOLOGO@DriverSetup#1{%
  \kvsetkeys{HoLogoDriver}{#1}%
  \HOLOGO@CheckDriver
  \ltx@ifundefined{hologoDriver}{%
    \begingroup
    \edef\x{\endgroup
      \noexpand\kvsetkeys{HoLogoDriver}{\HOLOGO@DriverFallback}%
    }\x
  }{}%
  \@PackageInfoNoLine{hologo}{Using driver `\hologoDriver'}%
}
%    \end{macrocode}
%    \end{macro}
%
%    \begin{macro}{\HOLOGO@CheckDriver}
%    \begin{macrocode}
\def\HOLOGO@CheckDriver{%
  \ifpdf
    \def\hologoDriver{pdftex}%
    \let\HOLOGO@pdfliteral\pdfliteral
    \ifluatex
      \ifx\pdfextension\@undefined\else
        \protected\def\pdfliteral{\pdfextension literal}%
        \let\HOLOGO@pdfliteral\pdfliteral
      \fi
      \ltx@IfUndefined{HOLOGO@pdfliteral}{%
        \ifnum\luatexversion<36 %
        \else
          \begingroup
            \let\HOLOGO@temp\endgroup
            \ifcase0%
                \directlua{%
                  if tex.enableprimitives then %
                    tex.enableprimitives('HOLOGO@', {'pdfliteral'})%
                  else %
                    tex.print('1')%
                  end%
                }%
                \ifx\HOLOGO@pdfliteral\@undefined 1\fi%
                \relax%
              \endgroup
              \let\HOLOGO@temp\relax
              \global\let\HOLOGO@pdfliteral\HOLOGO@pdfliteral
            \fi%
          \HOLOGO@temp
        \fi
      }{}%
    \fi
    \ltx@IfUndefined{HOLOGO@pdfliteral}{%
      \@PackageWarningNoLine{hologo}{%
        Cannot find \string\pdfliteral
      }%
    }{}%
  \else
    \ifxetex
      \def\hologoDriver{xetex}%
    \else
      \ifvtex
        \def\hologoDriver{vtex}%
      \fi
    \fi
  \fi
}
%    \end{macrocode}
%    \end{macro}
%
%    \begin{macro}{\HOLOGO@WarningUnsupportedDriver}
%    \begin{macrocode}
\def\HOLOGO@WarningUnsupportedDriver#1{%
  \@PackageWarningNoLine{hologo}{%
    Logo `#1' needs driver specific macros,\MessageBreak
    but driver `\hologoDriver' is not supported.\MessageBreak
    Use a different driver or\MessageBreak
    load package `graphics' or `pgf'%
  }%
}
%    \end{macrocode}
%    \end{macro}
%
% \subsubsection{Reflect box macros}
%
%    Skip driver part if not needed.
%    \begin{macrocode}
\ltx@IfUndefined{reflectbox}{}{%
  \ltx@IfUndefined{rotatebox}{}{%
    \HOLOGO@AtEnd
  }%
}
\ltx@IfUndefined{pgftext}{}{%
  \HOLOGO@AtEnd
}
\ltx@IfUndefined{psscalebox}{}{%
  \HOLOGO@AtEnd
}
%    \end{macrocode}
%
%    \begin{macrocode}
\def\HOLOGO@temp{LaTeX2e}
\ifx\fmtname\HOLOGO@temp
  \RequirePackage{kvoptions}[2011/06/30]%
  \ProcessKeyvalOptions{HoLogoDriver}%
\fi
\HOLOGO@DriverSetup{}
%    \end{macrocode}
%
%    \begin{macro}{\HOLOGO@ReflectBox}
%    \begin{macrocode}
\def\HOLOGO@ReflectBox#1{%
  \begingroup
    \setbox\ltx@zero\hbox{\begingroup#1\endgroup}%
    \setbox\ltx@two\hbox{%
      \kern\wd\ltx@zero
      \csname HOLOGO@ScaleBox@\hologoDriver\endcsname{-1}{1}{%
        \hbox to 0pt{\copy\ltx@zero\hss}%
      }%
    }%
    \wd\ltx@two=\wd\ltx@zero
    \box\ltx@two
  \endgroup
}
%    \end{macrocode}
%    \end{macro}
%
%    \begin{macro}{\HOLOGO@PointReflectBox}
%    \begin{macrocode}
\def\HOLOGO@PointReflectBox#1{%
  \begingroup
    \setbox\ltx@zero\hbox{\begingroup#1\endgroup}%
    \setbox\ltx@two\hbox{%
      \kern\wd\ltx@zero
      \raise\ht\ltx@zero\hbox{%
        \csname HOLOGO@ScaleBox@\hologoDriver\endcsname{-1}{-1}{%
          \hbox to 0pt{\copy\ltx@zero\hss}%
        }%
      }%
    }%
    \wd\ltx@two=\wd\ltx@zero
    \box\ltx@two
  \endgroup
}
%    \end{macrocode}
%    \end{macro}
%
%    We must define all variants because of dynamic driver setup.
%    \begin{macrocode}
\def\HOLOGO@temp#1#2{#2}
%    \end{macrocode}
%
%    \begin{macro}{\HOLOGO@ScaleBox@pdftex}
%    \begin{macrocode}
\HOLOGO@temp{pdftex}{%
  \def\HOLOGO@ScaleBox@pdftex#1#2#3{%
    \HOLOGO@pdfliteral{%
      q #1 0 0 #2 0 0 cm%
    }%
    #3%
    \HOLOGO@pdfliteral{%
      Q%
    }%
  }%
}
%    \end{macrocode}
%    \end{macro}
%    \begin{macro}{\HOLOGO@ScaleBox@dvips}
%    \begin{macrocode}
\HOLOGO@temp{dvips}{%
  \def\HOLOGO@ScaleBox@dvips#1#2#3{%
    \special{ps:%
      gsave %
      currentpoint %
      currentpoint translate %
      #1 #2 scale %
      neg exch neg exch translate%
    }%
    #3%
    \special{ps:%
      currentpoint %
      grestore %
      moveto%
    }%
  }%
}
%    \end{macrocode}
%    \end{macro}
%    \begin{macro}{\HOLOGO@ScaleBox@dvipdfm}
%    \begin{macrocode}
\HOLOGO@temp{dvipdfm}{%
  \let\HOLOGO@ScaleBox@dvipdfm\HOLOGO@ScaleBox@dvips
}
%    \end{macrocode}
%    \end{macro}
%    Since \hologo{XeTeX} v0.6.
%    \begin{macro}{\HOLOGO@ScaleBox@xetex}
%    \begin{macrocode}
\HOLOGO@temp{xetex}{%
  \def\HOLOGO@ScaleBox@xetex#1#2#3{%
    \special{x:gsave}%
    \special{x:scale #1 #2}%
    #3%
    \special{x:grestore}%
  }%
}
%    \end{macrocode}
%    \end{macro}
%    \begin{macro}{\HOLOGO@ScaleBox@vtex}
%    \begin{macrocode}
\HOLOGO@temp{vtex}{%
  \def\HOLOGO@ScaleBox@vtex#1#2#3{%
    \special{r(#1,0,0,#2,0,0}%
    #3%
    \special{r)}%
  }%
}
%    \end{macrocode}
%    \end{macro}
%
%    \begin{macrocode}
\HOLOGO@AtEnd%
%</package>
%    \end{macrocode}
%
% \section{Test}
%
% \subsection{Catcode checks for loading}
%
%    \begin{macrocode}
%<*test1>
%    \end{macrocode}
%    \begin{macrocode}
\catcode`\{=1 %
\catcode`\}=2 %
\catcode`\#=6 %
\catcode`\@=11 %
\expandafter\ifx\csname count@\endcsname\relax
  \countdef\count@=255 %
\fi
\expandafter\ifx\csname @gobble\endcsname\relax
  \long\def\@gobble#1{}%
\fi
\expandafter\ifx\csname @firstofone\endcsname\relax
  \long\def\@firstofone#1{#1}%
\fi
\expandafter\ifx\csname loop\endcsname\relax
  \expandafter\@firstofone
\else
  \expandafter\@gobble
\fi
{%
  \def\loop#1\repeat{%
    \def\body{#1}%
    \iterate
  }%
  \def\iterate{%
    \body
      \let\next\iterate
    \else
      \let\next\relax
    \fi
    \next
  }%
  \let\repeat=\fi
}%
\def\RestoreCatcodes{}
\count@=0 %
\loop
  \edef\RestoreCatcodes{%
    \RestoreCatcodes
    \catcode\the\count@=\the\catcode\count@\relax
  }%
\ifnum\count@<255 %
  \advance\count@ 1 %
\repeat

\def\RangeCatcodeInvalid#1#2{%
  \count@=#1\relax
  \loop
    \catcode\count@=15 %
  \ifnum\count@<#2\relax
    \advance\count@ 1 %
  \repeat
}
\def\RangeCatcodeCheck#1#2#3{%
  \count@=#1\relax
  \loop
    \ifnum#3=\catcode\count@
    \else
      \errmessage{%
        Character \the\count@\space
        with wrong catcode \the\catcode\count@\space
        instead of \number#3%
      }%
    \fi
  \ifnum\count@<#2\relax
    \advance\count@ 1 %
  \repeat
}
\def\space{ }
\expandafter\ifx\csname LoadCommand\endcsname\relax
  \def\LoadCommand{\input hologo.sty\relax}%
\fi
\def\Test{%
  \RangeCatcodeInvalid{0}{47}%
  \RangeCatcodeInvalid{58}{64}%
  \RangeCatcodeInvalid{91}{96}%
  \RangeCatcodeInvalid{123}{255}%
  \catcode`\@=12 %
  \catcode`\\=0 %
  \catcode`\%=14 %
  \LoadCommand
  \RangeCatcodeCheck{0}{36}{15}%
  \RangeCatcodeCheck{37}{37}{14}%
  \RangeCatcodeCheck{38}{47}{15}%
  \RangeCatcodeCheck{48}{57}{12}%
  \RangeCatcodeCheck{58}{63}{15}%
  \RangeCatcodeCheck{64}{64}{12}%
  \RangeCatcodeCheck{65}{90}{11}%
  \RangeCatcodeCheck{91}{91}{15}%
  \RangeCatcodeCheck{92}{92}{0}%
  \RangeCatcodeCheck{93}{96}{15}%
  \RangeCatcodeCheck{97}{122}{11}%
  \RangeCatcodeCheck{123}{255}{15}%
  \RestoreCatcodes
}
\Test
\csname @@end\endcsname
\end
%    \end{macrocode}
%    \begin{macrocode}
%</test1>
%    \end{macrocode}
%
% \subsection{Spacefactor}
%
%    The space factor must be 1000 after a logo. If it is greater 1000
%    then the following space is a space after a sentence closing point.
%    If the space factor is smaller 1000 then an immediate following
%    dot is interpreted as abbreviation, not sentence closing point.
%
%    \begin{macrocode}
%<*test-spacefactor>
\NeedsTeXFormat{LaTeX2e}
\documentclass{article}
\usepackage{hologo}[2016/05/12]
\usepackage{kvsetkeys}
\usepackage{qstest}
\IncludeTests{*}
\LogTests{log}{*}{*}
\begin{document}
\begin{qstest}{spacefactor}{spacefactor}
\newcommand*{\Test}[1]{%
  \sbox0{%
    \hologo{#1}%
    \Expect*{1000 (#1)}*{\the\spacefactor\space(#1)}%
  }%
}%
\makeatletter
\def\TestList{}
\def\hologoEntry#1#2#3{%
  \edef\TestList{%
    \ifx\TestList\@empty
    \else
      \TestList,%
    \fi
    #1%
    \ifx\\#2\\%
    \else
      ={variant=#2}%
    \fi
  }%
}
\hologoList
\expandafter\kv@parse@normalized\expandafter{%
  \TestList
}{%
  \begingroup
    \let\@logo=\kv@key
    \ifx\kv@value\relax
    \else
      \expandafter\hologoLogoSetup\expandafter\@logo\expandafter{%
        \kv@value
      }%
    \fi
    \Test\@logo
  \endgroup
  \@gobbletwo
}
\end{qstest}
\end{document}
%</test-spacefactor>
%    \end{macrocode}
%
% \subsection{Complete list}
%
%    \begin{macrocode}
%<*test-list>
\NeedsTeXFormat{LaTeX2e}
\documentclass[12pt,a4paper]{article}
\usepackage{hologo}[2016/05/12]
\usepackage[T1]{fontenc}
\usepackage{lmodern}
\usepackage{parskip}
\usepackage[unicode]{hyperref}[2011/09/28]
\usepackage{bookmark}[2011/09/19]
\bookmarksetup{%
  numbered,%
  open,%
  openlevel=2,%
}
\renewcommand*{\contentsname}{List of logos}
\begin{document}
\tableofcontents
\def\TestFont#1#2#3#4#5#6{%
  \begingroup
    \usefont{#3}{#4}{#5}{#6}%
    \HologoVariant{#1}{#2}/\hologoVariant{#1}{#2}%
    \quad
    \begingroup\scriptsize\hologoVariant{#1}{#2}\endgroup
    \quad
  \endgroup
  (#3/#4/#5/#6)%
  \par
}
\makeatletter
\def\hologoEntry#1#2#3{%
  \section{%
    \HologoVariant{#1}{#2}/\hologoVariant{#1}{#2} %
    {[#1\ifx\\#2\\\else\space(#2)\fi]}% hash-ok
  }% braces around [] because of bug in tex4ht
  \begingroup
    \hypersetup{unicode=false}%
    \bookmark[%
      dest=\@currentHref,%
      rellevel=1,%
      keeplevel,%
    ]{%
      \HologoVariant{#1}{#2}/\hologoVariant{#1}{#2} %
      (PDFDocEncoding)%
    }%
  \endgroup
  \TestFont{#1}{#2}{OT1}{cmr}{m}{n}%
  \TestFont{#1}{#2}{OT1}{cmss}{m}{n}%
  \TestFont{#1}{#2}{OT1}{cmr}{b}{n}%
  \TestFont{#1}{#2}{OT1}{cmr}{m}{it}%
  \TestFont{#1}{#2}{OT1}{cmtt}{m}{n}%
  \TestFont{#1}{#2}{T1}{lmr}{m}{n}%
  \TestFont{#1}{#2}{T1}{lmss}{m}{n}%
  \TestFont{#1}{#2}{T1}{lmr}{b}{n}%
  \TestFont{#1}{#2}{T1}{lmr}{m}{it}%
  \TestFont{#1}{#2}{T1}{lmtt}{m}{n}%
  \TestFont{#1}{#2}{T1}{lmvtt}{m}{n}%
  \TestFont{#1}{#2}{T1}{qtm}{m}{n}%
  \TestFont{#1}{#2}{T1}{qhv}{m}{n}%
  \TestFont{#1}{#2}{T1}{qtm}{b}{n}%
  \TestFont{#1}{#2}{T1}{qtm}{m}{it}%
  \TestFont{#1}{#2}{T1}{qcr}{m}{n}%
  \newpage
}
\makeatother
\hologoList
\end{document}
%</test-list>
%    \end{macrocode}
%
% \section{Installation}
%
% \subsection{Download}
%
% \paragraph{Package.} This package is available on
% CTAN\footnote{\url{ftp://ftp.ctan.org/tex-archive/}}:
% \begin{description}
% \item[\CTAN{macros/latex/contrib/oberdiek/hologo.dtx}] The source file.
% \item[\CTAN{macros/latex/contrib/oberdiek/hologo.pdf}] Documentation.
% \end{description}
%
%
% \paragraph{Bundle.} All the packages of the bundle `oberdiek'
% are also available in a TDS compliant ZIP archive. There
% the packages are already unpacked and the documentation files
% are generated. The files and directories obey the TDS standard.
% \begin{description}
% \item[\CTAN{install/macros/latex/contrib/oberdiek.tds.zip}]
% \end{description}
% \emph{TDS} refers to the standard ``A Directory Structure
% for \TeX\ Files'' (\CTAN{tds/tds.pdf}). Directories
% with \xfile{texmf} in their name are usually organized this way.
%
% \subsection{Bundle installation}
%
% \paragraph{Unpacking.} Unpack the \xfile{oberdiek.tds.zip} in the
% TDS tree (also known as \xfile{texmf} tree) of your choice.
% Example (linux):
% \begin{quote}
%   |unzip oberdiek.tds.zip -d ~/texmf|
% \end{quote}
%
% \paragraph{Script installation.}
% Check the directory \xfile{TDS:scripts/oberdiek/} for
% scripts that need further installation steps.
% Package \xpackage{attachfile2} comes with the Perl script
% \xfile{pdfatfi.pl} that should be installed in such a way
% that it can be called as \texttt{pdfatfi}.
% Example (linux):
% \begin{quote}
%   |chmod +x scripts/oberdiek/pdfatfi.pl|\\
%   |cp scripts/oberdiek/pdfatfi.pl /usr/local/bin/|
% \end{quote}
%
% \subsection{Package installation}
%
% \paragraph{Unpacking.} The \xfile{.dtx} file is a self-extracting
% \docstrip\ archive. The files are extracted by running the
% \xfile{.dtx} through \plainTeX:
% \begin{quote}
%   \verb|tex hologo.dtx|
% \end{quote}
%
% \paragraph{TDS.} Now the different files must be moved into
% the different directories in your installation TDS tree
% (also known as \xfile{texmf} tree):
% \begin{quote}
% \def\t{^^A
% \begin{tabular}{@{}>{\ttfamily}l@{ $\rightarrow$ }>{\ttfamily}l@{}}
%   hologo.sty & tex/generic/oberdiek/hologo.sty\\
%   hologo.pdf & doc/latex/oberdiek/hologo.pdf\\
%   example/hologo-example.tex & doc/latex/oberdiek/example/hologo-example.tex\\
%   test/hologo-test1.tex & doc/latex/oberdiek/test/hologo-test1.tex\\
%   test/hologo-test-spacefactor.tex & doc/latex/oberdiek/test/hologo-test-spacefactor.tex\\
%   test/hologo-test-list.tex & doc/latex/oberdiek/test/hologo-test-list.tex\\
%   hologo.dtx & source/latex/oberdiek/hologo.dtx\\
% \end{tabular}^^A
% }^^A
% \sbox0{\t}^^A
% \ifdim\wd0>\linewidth
%   \begingroup
%     \advance\linewidth by\leftmargin
%     \advance\linewidth by\rightmargin
%   \edef\x{\endgroup
%     \def\noexpand\lw{\the\linewidth}^^A
%   }\x
%   \def\lwbox{^^A
%     \leavevmode
%     \hbox to \linewidth{^^A
%       \kern-\leftmargin\relax
%       \hss
%       \usebox0
%       \hss
%       \kern-\rightmargin\relax
%     }^^A
%   }^^A
%   \ifdim\wd0>\lw
%     \sbox0{\small\t}^^A
%     \ifdim\wd0>\linewidth
%       \ifdim\wd0>\lw
%         \sbox0{\footnotesize\t}^^A
%         \ifdim\wd0>\linewidth
%           \ifdim\wd0>\lw
%             \sbox0{\scriptsize\t}^^A
%             \ifdim\wd0>\linewidth
%               \ifdim\wd0>\lw
%                 \sbox0{\tiny\t}^^A
%                 \ifdim\wd0>\linewidth
%                   \lwbox
%                 \else
%                   \usebox0
%                 \fi
%               \else
%                 \lwbox
%               \fi
%             \else
%               \usebox0
%             \fi
%           \else
%             \lwbox
%           \fi
%         \else
%           \usebox0
%         \fi
%       \else
%         \lwbox
%       \fi
%     \else
%       \usebox0
%     \fi
%   \else
%     \lwbox
%   \fi
% \else
%   \usebox0
% \fi
% \end{quote}
% If you have a \xfile{docstrip.cfg} that configures and enables \docstrip's
% TDS installing feature, then some files can already be in the right
% place, see the documentation of \docstrip.
%
% \subsection{Refresh file name databases}
%
% If your \TeX~distribution
% (\teTeX, \mikTeX, \dots) relies on file name databases, you must refresh
% these. For example, \teTeX\ users run \verb|texhash| or
% \verb|mktexlsr|.
%
% \subsection{Some details for the interested}
%
% \paragraph{Attached source.}
%
% The PDF documentation on CTAN also includes the
% \xfile{.dtx} source file. It can be extracted by
% AcrobatReader 6 or higher. Another option is \textsf{pdftk},
% e.g. unpack the file into the current directory:
% \begin{quote}
%   \verb|pdftk hologo.pdf unpack_files output .|
% \end{quote}
%
% \paragraph{Unpacking with \LaTeX.}
% The \xfile{.dtx} chooses its action depending on the format:
% \begin{description}
% \item[\plainTeX:] Run \docstrip\ and extract the files.
% \item[\LaTeX:] Generate the documentation.
% \end{description}
% If you insist on using \LaTeX\ for \docstrip\ (really,
% \docstrip\ does not need \LaTeX), then inform the autodetect routine
% about your intention:
% \begin{quote}
%   \verb|latex \let\install=y\input{hologo.dtx}|
% \end{quote}
% Do not forget to quote the argument according to the demands
% of your shell.
%
% \paragraph{Generating the documentation.}
% You can use both the \xfile{.dtx} or the \xfile{.drv} to generate
% the documentation. The process can be configured by the
% configuration file \xfile{ltxdoc.cfg}. For instance, put this
% line into this file, if you want to have A4 as paper format:
% \begin{quote}
%   \verb|\PassOptionsToClass{a4paper}{article}|
% \end{quote}
% An example follows how to generate the
% documentation with pdf\LaTeX:
% \begin{quote}
%\begin{verbatim}
%pdflatex hologo.dtx
%makeindex -s gind.ist hologo.idx
%pdflatex hologo.dtx
%makeindex -s gind.ist hologo.idx
%pdflatex hologo.dtx
%\end{verbatim}
% \end{quote}
%
% \section{Catalogue}
%
% The following XML file can be used as source for the
% \href{http://mirror.ctan.org/help/Catalogue/catalogue.html}{\TeX\ Catalogue}.
% The elements \texttt{caption} and \texttt{description} are imported
% from the original XML file from the Catalogue.
% The name of the XML file in the Catalogue is \xfile{hologo.xml}.
%    \begin{macrocode}
%<*catalogue>
<?xml version='1.0' encoding='us-ascii'?>
<!DOCTYPE entry SYSTEM 'catalogue.dtd'>
<entry datestamp='$Date$' modifier='$Author$' id='hologo'>
  <name>hologo</name>
  <caption>A collection of logos with bookmark support.</caption>
  <authorref id='auth:oberdiek'/>
  <copyright owner='Heiko Oberdiek' year='2010-2012'/>
  <license type='lppl1.3'/>
  <version number='1.10'/>
  <description>
    The package defines a single command <tt>\hologo</tt>, whose
    argument is the usual case-confused ASCII version of the logo.
    The command is bookmark-enabled, so that every logo becomes
    available in bookmarks without further work.
    <p/>
    The package is part of the <xref refid='oberdiek'>oberdiek</xref>
    bundle.
  </description>
  <documentation details='Package documentation'
      href='ctan:/macros/latex/contrib/oberdiek/hologo.pdf'/>
  <ctan file='true' path='/macros/latex/contrib/oberdiek/hologo.dtx'/>
  <miktex location='oberdiek'/>
  <texlive location='oberdiek'/>
  <install path='/macros/latex/contrib/oberdiek/oberdiek.tds.zip'/>
</entry>
%</catalogue>
%    \end{macrocode}
%
% \begin{thebibliography}{9}
% \raggedright
%
% \bibitem{btxdoc}
% Oren Patashnik,
% \textit{\hologo{BibTeX}ing},
% 1988-02-08.\\
% \CTAN{biblio/bibtex/base/}
%
% \bibitem{dtklogos}
% Gerd Neugebauer, DANTE,
% \textit{Package \xpackage{dtklogos}},
% 2011-04-25.\\
% \CTAN{usergrps/dante/dtk/dtklogos.sty}
%
% \bibitem{etexman}
% The \hologo{NTS} Team,
% \textit{The \hologo{eTeX} manual},
% 1998-02.\\
% \CTAN{systems/e-tex/v2/doc/}
%
% \bibitem{ExTeX-FAQ}
% The \hologo{ExTeX} group,
% \textit{\hologo{ExTeX}: FAQ -- How is \hologo{ExTeX} typeset?},
% 2007-04-14.\\
% \url{http://www.extex.org/documentation/faq.html}
%
% \bibitem{LyX}
% %@MISC{ LyX,
% %  title = {{LyX 2.0.0 -- The Document Processor [Computer software and manual]}},
% %  author = {{The LyX Team}},
% %  howpublished = {Internet: http://www.lyx.org},
% %  year = {2011-05-08},
% %  note = {Retrieved May 10, 2011, from http://www.lyx.org},
% %  url = {http://www.lyx.org/}
% %}
% The \hologo{LyX} Team,
% \textit{\hologo{LyX} -- The Document Processor},
% 2011-05-08.\\
% \url{http://www.lyx.org/}
%
% \bibitem{OzTeX}
% Andrew Trevorrow,
% \hologo{OzTeX} FAQ: What is the correct way to typeset ``\hologo{OzTeX}''?,
% 2011-09-15 (visited).
% \url{http://www.trevorrow.com/oztex/ozfaq.html#oztex-logo}
%
% \bibitem{PiCTeX}
% Michael Wichura,
% \textit{The \hologo{PiCTeX} macro package},
% 1987-09-21.
% \CTAN{graphics/pictex/}
%
% \bibitem{scrlogo}
% Markus Kohm,
% \textit{\hologo{KOMAScript} Datei \xfile{scrlogo.dtx}},
% 2009-01-30.\\
% \CTAN{install/macros/latex/contrib/komascript.tds.zip}
%
% \end{thebibliography}
%
% \begin{History}
%   \begin{Version}{2010/04/08 v1.0}
%   \item
%     The first version.
%   \end{Version}
%   \begin{Version}{2010/04/16 v1.1}
%   \item
%     \cs{Hologo} added for support of logos at start of a sentence.
%   \item
%     \cs{hologoSetup} and \cs{hologoLogoSetup} added.
%   \item
%     Options \xoption{break}, \xoption{hyphenbreak}, \xoption{spacebreak}
%     added.
%   \item
%     Variant support added by option \xoption{variant}.
%   \end{Version}
%   \begin{Version}{2010/04/24 v1.2}
%   \item
%     \hologo{LaTeX3} added.
%   \item
%     \hologo{VTeX} added.
%   \end{Version}
%   \begin{Version}{2010/11/21 v1.3}
%   \item
%     \hologo{iniTeX}, \hologo{virTeX} added.
%   \end{Version}
%   \begin{Version}{2011/03/25 v1.4}
%   \item
%     \hologo{ConTeXt} with variants added.
%   \item
%     Option \xoption{discretionarybreak} added as refinement for
%     option \xoption{break}.
%   \end{Version}
%   \begin{Version}{2011/04/21 v1.5}
%   \item
%     Wrong TDS directory for test files fixed.
%   \end{Version}
%   \begin{Version}{2011/10/01 v1.6}
%   \item
%     Support for package \xpackage{tex4ht} added.
%   \item
%     Support for \cs{csname} added if \cs{ifincsname} is available.
%   \item
%     New logos:
%     \hologo{(La)TeX},
%     \hologo{biber},
%     \hologo{BibTeX} (\xoption{sc}, \xoption{sf}),
%     \hologo{emTeX},
%     \hologo{ExTeX},
%     \hologo{KOMAScript},
%     \hologo{La},
%     \hologo{LyX},
%     \hologo{MiKTeX},
%     \hologo{NTS},
%     \hologo{OzMF},
%     \hologo{OzMP},
%     \hologo{OzTeX},
%     \hologo{OzTtH},
%     \hologo{PCTeX},
%     \hologo{PiC},
%     \hologo{PiCTeX},
%     \hologo{METAFONT},
%     \hologo{MetaFun},
%     \hologo{METAPOST},
%     \hologo{MetaPost},
%     \hologo{SLiTeX} (\xoption{lift}, \xoption{narrow}, \xoption{simple}),
%     \hologo{SliTeX} (\xoption{narrow}, \xoption{simple}, \xoption{lift}),
%     \hologo{teTeX}.
%   \item
%     Fixes:
%     \hologo{iniTeX},
%     \hologo{pdfLaTeX},
%     \hologo{pdfTeX},
%     \hologo{virTeX}.
%   \item
%     \cs{hologoFontSetup} and \cs{hologoLogoFontSetup} added.
%   \item
%     \cs{hologoVariant} and \cs{HologoVariant} added.
%   \end{Version}
%   \begin{Version}{2011/11/22 v1.7}
%   \item
%     New logos:
%     \hologo{BibTeX8},
%     \hologo{LaTeXML},
%     \hologo{SageTeX},
%     \hologo{TeX4ht},
%     \hologo{TTH}.
%   \item
%     \hologo{Xe} and friends: Driver stuff fixed.
%   \item
%     \hologo{Xe} and friends: Support for italic added.
%   \item
%     \hologo{Xe} and friends: Package support for \xpackage{pgf}
%     and \xpackage{pstricks} added.
%   \end{Version}
%   \begin{Version}{2011/11/29 v1.8}
%   \item
%     New logos:
%     \hologo{HanTheThanh}.
%   \end{Version}
%   \begin{Version}{2011/12/21 v1.9}
%   \item
%     Patch for package \xpackage{ifxetex} added for the case that
%     \cs{newif} is undefined in \hologo{iniTeX}.
%   \item
%     Some fixes for \hologo{iniTeX}.
%   \end{Version}
%   \begin{Version}{2012/04/26 v1.10}
%   \item
%     Fix in bookmark version of logo ``\hologo{HanTheThanh}''.
%   \end{Version}
%   \begin{Version}{2016/05/12 v1.11}
%   \item
%     Update HOLOGO@IfCharExists (previously in texlive)
%   \item define pdfliteral in current luatex.
%   \end{Version}
% \end{History}
%
% \PrintIndex
%
% \Finale
\endinput

%        (quote the arguments according to the demands of your shell)
%
% Documentation:
%    (a) If hologo.drv is present:
%           latex hologo.drv
%    (b) Without hologo.drv:
%           latex hologo.dtx; ...
%    The class ltxdoc loads the configuration file ltxdoc.cfg
%    if available. Here you can specify further options, e.g.
%    use A4 as paper format:
%       \PassOptionsToClass{a4paper}{article}
%
%    Programm calls to get the documentation (example):
%       pdflatex hologo.dtx
%       makeindex -s gind.ist hologo.idx
%       pdflatex hologo.dtx
%       makeindex -s gind.ist hologo.idx
%       pdflatex hologo.dtx
%
% Installation:
%    TDS:tex/generic/oberdiek/hologo.sty
%    TDS:doc/latex/oberdiek/hologo.pdf
%    TDS:doc/latex/oberdiek/example/hologo-example.tex
%    TDS:doc/latex/oberdiek/test/hologo-test1.tex
%    TDS:doc/latex/oberdiek/test/hologo-test-spacefactor.tex
%    TDS:doc/latex/oberdiek/test/hologo-test-list.tex
%    TDS:source/latex/oberdiek/hologo.dtx
%
%<*ignore>
\begingroup
  \catcode123=1 %
  \catcode125=2 %
  \def\x{LaTeX2e}%
\expandafter\endgroup
\ifcase 0\ifx\install y1\fi\expandafter
         \ifx\csname processbatchFile\endcsname\relax\else1\fi
         \ifx\fmtname\x\else 1\fi\relax
\else\csname fi\endcsname
%</ignore>
%<*install>
\input docstrip.tex
\Msg{************************************************************************}
\Msg{* Installation}
\Msg{* Package: hologo 2016/05/12 v1.11 A logo collection with bookmark support (HO)}
\Msg{************************************************************************}

\keepsilent
\askforoverwritefalse

\let\MetaPrefix\relax
\preamble

This is a generated file.

Project: hologo
Version: 2016/05/12 v1.11

Copyright (C) 2010-2012 by
   Heiko Oberdiek <heiko.oberdiek at googlemail.com>

This work may be distributed and/or modified under the
conditions of the LaTeX Project Public License, either
version 1.3c of this license or (at your option) any later
version. This version of this license is in
   http://www.latex-project.org/lppl/lppl-1-3c.txt
and the latest version of this license is in
   http://www.latex-project.org/lppl.txt
and version 1.3 or later is part of all distributions of
LaTeX version 2005/12/01 or later.

This work has the LPPL maintenance status "maintained".

This Current Maintainer of this work is Heiko Oberdiek.

The Base Interpreter refers to any `TeX-Format',
because some files are installed in TDS:tex/generic//.

This work consists of the main source file hologo.dtx
and the derived files
   hologo.sty, hologo.pdf, hologo.ins, hologo.drv, hologo-example.tex,
   hologo-test1.tex, hologo-test-spacefactor.tex,
   hologo-test-list.tex.

\endpreamble
\let\MetaPrefix\DoubleperCent

\generate{%
  \file{hologo.ins}{\from{hologo.dtx}{install}}%
  \file{hologo.drv}{\from{hologo.dtx}{driver}}%
  \usedir{tex/generic/oberdiek}%
  \file{hologo.sty}{\from{hologo.dtx}{package}}%
  \usedir{doc/latex/oberdiek/example}%
  \file{hologo-example.tex}{\from{hologo.dtx}{example}}%
  \usedir{doc/latex/oberdiek/test}%
  \file{hologo-test1.tex}{\from{hologo.dtx}{test1}}%
  \file{hologo-test-spacefactor.tex}{\from{hologo.dtx}{test-spacefactor}}%
  \file{hologo-test-list.tex}{\from{hologo.dtx}{test-list}}%
  \nopreamble
  \nopostamble
  \usedir{source/latex/oberdiek/catalogue}%
  \file{hologo.xml}{\from{hologo.dtx}{catalogue}}%
}

\catcode32=13\relax% active space
\let =\space%
\Msg{************************************************************************}
\Msg{*}
\Msg{* To finish the installation you have to move the following}
\Msg{* file into a directory searched by TeX:}
\Msg{*}
\Msg{*     hologo.sty}
\Msg{*}
\Msg{* To produce the documentation run the file `hologo.drv'}
\Msg{* through LaTeX.}
\Msg{*}
\Msg{* Happy TeXing!}
\Msg{*}
\Msg{************************************************************************}

\endbatchfile
%</install>
%<*ignore>
\fi
%</ignore>
%<*driver>
\NeedsTeXFormat{LaTeX2e}
\ProvidesFile{hologo.drv}%
  [2016/05/12 v1.11 A logo collection with bookmark support (HO)]%
\documentclass{ltxdoc}
\usepackage{holtxdoc}[2011/11/22]
\usepackage{hologo}[2016/05/12]
\usepackage{longtable}
\usepackage{array}
\usepackage{paralist}
%\usepackage[T1]{fontenc}
%\usepackage{lmodern}
\begin{document}
  \DocInput{hologo.dtx}%
\end{document}
%</driver>
% \fi
%
%
% \CharacterTable
%  {Upper-case    \A\B\C\D\E\F\G\H\I\J\K\L\M\N\O\P\Q\R\S\T\U\V\W\X\Y\Z
%   Lower-case    \a\b\c\d\e\f\g\h\i\j\k\l\m\n\o\p\q\r\s\t\u\v\w\x\y\z
%   Digits        \0\1\2\3\4\5\6\7\8\9
%   Exclamation   \!     Double quote  \"     Hash (number) \#
%   Dollar        \$     Percent       \%     Ampersand     \&
%   Acute accent  \'     Left paren    \(     Right paren   \)
%   Asterisk      \*     Plus          \+     Comma         \,
%   Minus         \-     Point         \.     Solidus       \/
%   Colon         \:     Semicolon     \;     Less than     \<
%   Equals        \=     Greater than  \>     Question mark \?
%   Commercial at \@     Left bracket  \[     Backslash     \\
%   Right bracket \]     Circumflex    \^     Underscore    \_
%   Grave accent  \`     Left brace    \{     Vertical bar  \|
%   Right brace   \}     Tilde         \~}
%
% \GetFileInfo{hologo.drv}
%
% \title{The \xpackage{hologo} package}
% \date{2016/05/12 v1.11}
% \author{Heiko Oberdiek\\\xemail{heiko.oberdiek at googlemail.com}}
%
% \maketitle
%
% \begin{abstract}
% This package starts a collection of logos with support for bookmarks
% strings.
% \end{abstract}
%
% \tableofcontents
%
% \section{Documentation}
%
% \subsection{Logo macros}
%
% \begin{declcs}{hologo} \M{name}
% \end{declcs}
% Macro \cs{hologo} sets the logo with name \meta{name}.
% The following table shows the supported names.
%
% \begingroup
%   \def\hologoEntry#1#2#3{^^A
%     #1&#2&\hologoLogoSetup{#1}{variant=#2}\hologo{#1}&#3\tabularnewline
%   }
%   \begin{longtable}{>{\ttfamily}l>{\ttfamily}lll}
%     \rmfamily\bfseries{name} & \rmfamily\bfseries variant
%     & \bfseries logo & \bfseries since\\
%     \hline
%     \endhead
%     \hologoList
%   \end{longtable}
% \endgroup
%
% \begin{declcs}{Hologo} \M{name}
% \end{declcs}
% Macro \cs{Hologo} starts the logo \meta{name} with an uppercase
% letter. As an exception small greek letters are not converted
% to uppercase. Examples, see \hologo{eTeX} and \hologo{ExTeX}.
%
% \subsection{Setup macros}
%
% The package does not support package options, but the following
% setup macros can be used to set options.
%
% \begin{declcs}{hologoSetup} \M{key value list}
% \end{declcs}
% Macro \cs{hologoSetup} sets global options.
%
% \begin{declcs}{hologoLogoSetup} \M{logo} \M{key value list}
% \end{declcs}
% Some options can also be used to configure a logo.
% These settings take precedence over global option settings.
%
% \subsection{Options}\label{sec:options}
%
% There are boolean and string options:
% \begin{description}
% \item[Boolean option:]
% It takes |true| or |false|
% as value. If the value is omitted, then |true| is used.
% \item[String option:]
% A value must be given as string. (But the string might be empty.)
% \end{description}
% The following options can be used both in \cs{hologoSetup}
% and \cs{hologoLogoSetup}:
% \begin{description}
% \def\entry#1{\item[\xoption{#1}:]}
% \entry{break}
%   enables or disables line breaks inside the logo. This setting is
%   refined by options \xoption{hyphenbreak}, \xoption{spacebreak}
%   or \xoption{discretionarybreak}.
%   Default is |false|.
% \entry{hyphenbreak}
%   enables or disables the line break right after the hyphen character.
% \entry{spacebreak}
%   enables or disables line breaks at space characters.
% \entry{discretionarybreak}
%   enables or disables line breaks at hyphenation points
%   (inserted by \cs{-}).
% \end{description}
% Macro \cs{hologoLogoSetup} also knows:
% \begin{description}
% \item[\xoption{variant}:]
%   This is a string option. It specifies a variant of a logo that
%   must exist. An empty string selects the package default variant.
% \end{description}
% Example:
% \begin{quote}
%   |\hologoSetup{break=false}|\\
%   |\hologoLogoSetup{plainTeX}{variant=hyphen,hyphenbreak}|\\
%   Then ``plain-\TeX'' contains one break point after the hyphen.
% \end{quote}
%
% \subsection{Driver options}
%
% Sometimes graphical operations are needed to construct some
% glyphs (e.g.\ \hologo{XeTeX}). If package \xpackage{graphics}
% or package \xpackage{pgf} are found, then the macros are taken
% from there. Otherwise the packge defines its own operations
% and therefore needs the driver information. Many drivers are
% detected automatically (\hologo{pdfTeX}/\hologo{LuaTeX}
% in PDF mode, \hologo{XeTeX}, \hologo{VTeX}). These have precedence
% over a driver option. The driver can be given as package option
% or using \cs{hologoDriverSetup}.
% The following list contains the recognized driver options:
% \begin{itemize}
% \item \xoption{pdftex}, \xoption{luatex}
% \item \xoption{dvipdfm}, \xoption{dvipdfmx}
% \item \xoption{dvips}, \xoption{dvipsone}, \xoption{xdvi}
% \item \xoption{xetex}
% \item \xoption{vtex}
% \end{itemize}
% The left driver of a line is the driver name that is used internally.
% The following names are aliases for drivers that use the
% same method. Therefore the entry in the \xext{log} file for
% the used driver prints the internally used driver name.
% \begin{description}
% \item[\xoption{driverfallback}:]
%   This option expects a driver that is used,
%   if the driver could not be detected automatically.
% \end{description}
%
% \begin{declcs}{hologoDriverSetup} \M{driver option}
% \end{declcs}
% The driver can also be configured after package loading
% using \cs{hologoDriverSetup}, also the way for \hologo{plainTeX}
% to setup the driver.
%
% \subsection{Font setup}
%
% Some logos require a special font, but should also be usable by
% \hologo{plainTeX}. Therefore the package provides some ways
% to influence the font settings. The options below
% take font settings as values. Both font commands
% such as \cs{sffamily} and macros that take one argument
% like \cs{textsf} can be used.
%
% \begin{declcs}{hologoFontSetup} \M{key value list}
% \end{declcs}
% Macro \cs{hologoFontSetup} sets the fonts for all logos.
% Supported keys:
% \begin{description}
% \def\entry#1{\item[\xoption{#1}:]}
% \entry{general}
%   This font is used for all logos. The default is empty.
%   That means no special font is used.
% \entry{bibsf}
%   This font is used for
%   {\hologoLogoSetup{BibTeX}{variant=sf}\hologo{BibTeX}}
%   with variant \xoption{sf}.
% \entry{rm}
%   This font is a serif font. It is used for \hologo{ExTeX}.
% \entry{sc}
%   This font specifies a small caps font. It is used for
%   {\hologoLogoSetup{BibTeX}{variant=sc}\hologo{BibTeX}}
%   with variant \xoption{sc}.
% \entry{sf}
%   This font specifies a sans serif font. The default
%   is \cs{sffamily}, then \cs{sf} is tried. Otherwise
%   a warning is given. It is used by \hologo{KOMAScript}.
% \entry{sy}
%   This is the font for math symbols (e.g. cmsy).
%   It is used by \hologo{AmS}, \hologo{NTS}, \hologo{ExTeX}.
% \entry{logo}
%   \hologo{METAFONT} and \hologo{METAPOST} are using that font.
%   In \hologo{LaTeX} \cs{logofamily} is used and
%   the definitions of package \xpackage{mflogo} are used
%   if the package is not loaded.
%   Otherwise the \cs{tenlogo} is used and defined
%   if it does not already exists.
% \end{description}
%
% \begin{declcs}{hologoLogoFontSetup} \M{logo} \M{key value list}
% \end{declcs}
% Fonts can also be set for a logo or logo component separately,
% see the following list.
% The keys are the same as for \cs{hologoFontSetup}.
%
% \begin{longtable}{>{\ttfamily}l>{\sffamily}ll}
%   \meta{logo} & keys & result\\
%   \hline
%   \endhead
%   BibTeX & bibsf & {\hologoLogoSetup{BibTeX}{variant=sf}\hologo{BibTeX}}\\[.5ex]
%   BibTeX & sc & {\hologoLogoSetup{BibTeX}{variant=sc}\hologo{BibTeX}}\\[.5ex]
%   ExTeX & rm & \hologo{ExTeX}\\
%   SliTeX & rm & \hologo{SliTeX}\\[.5ex]
%   AmS & sy & \hologo{AmS}\\
%   ExTeX & sy & \hologo{ExTeX}\\
%   NTS & sy & \hologo{NTS}\\[.5ex]
%   KOMAScript & sf & \hologo{KOMAScript}\\[.5ex]
%   METAFONT & logo & \hologo{METAFONT}\\
%   METAPOST & logo & \hologo{METAPOST}\\[.5ex]
%   SliTeX & sc \hologo{SliTeX}
% \end{longtable}
%
% \subsubsection{Font order}
%
% For all logos the font \xoption{general} is applied first.
% Example:
%\begin{quote}
%|\hologoFontSetup{general=\color{red}}|
%\end{quote}
% will print red logos.
% Then if the font uses a special font \xoption{sf}, for example,
% the font is applied that is setup by \cs{hologoLogoFontSetup}.
% If this font is not setup, then the common font setup
% by \cs{hologoFontSetup} is used. Otherwise a warning is given,
% that there is no font configured.
%
% \subsection{Additional user macros}
%
% Usually a variant of a logo is configured by using
% \cs{hologoLogoSetup}, because it is bad style to mix
% different variants of the same logo in the same text.
% There the following macros are a convenience for testing.
%
% \begin{declcs}{hologoVariant} \M{name} \M{variant}\\
%   \cs{HologoVariant} \M{name} \M{variant}
% \end{declcs}
% Logo \meta{name} is set using \meta{variant} that specifies
% explicitely which variant of the macro is used. If the argument
% is empty, then the default form of the logo is used
% (configurable by \cs{hologoLogoSetup}).
%
% \cs{HologoVariant} is used if the logo is set in a context
% that needs an uppercase first letter (beginning of a sentence, \dots).
%
% \begin{declcs}{hologoList}\\
%   \cs{hologoEntry} \M{logo} \M{variant} \M{since}
% \end{declcs}
% Macro \cs{hologoList} contains all logos that are provided
% by the package including variants. The list consists of calls
% of \cs{hologoEntry} with three arguments starting with the
% logo name \meta{logo} and its variant \meta{variant}. An empty
% variant means the current default. Argument \meta{since} specifies
% with version of the package \xpackage{hologo} is needed to get
% the logo. If the logo is fixed, then the date gets updated.
% Therefore the date \meta{since} is not exactly the date of
% the first introduction, but rather the date of the latest fix.
%
% Before \cs{hologoList} can be used, macro \cs{hologoEntry} needs
% a definition. The example file in section \ref{sec:example}
% shows applications of \cs{hologoList}.
%
% \subsection{Supported contexts}
%
% Macros \cs{hologo} and friends support special contexts:
% \begin{itemize}
% \item \hologo{LaTeX}'s protection mechanism.
% \item Bookmarks of package \xpackage{hyperref}.
% \item Package \xpackage{tex4ht}.
% \item The macros can be used inside \cs{csname} constructs,
%   if \cs{ifincsname} is available (\hologo{pdfTeX}, \hologo{XeTeX},
%   \hologo{LuaTeX}).
% \end{itemize}
%
% \subsection{Example}
% \label{sec:example}
%
% The following example prints the logos in different fonts.
%    \begin{macrocode}
%<*example>
%<<verbatim
\NeedsTeXFormat{LaTeX2e}
\documentclass[a4paper]{article}
\usepackage[
  hmargin=20mm,
  vmargin=20mm,
]{geometry}
\pagestyle{empty}
\usepackage{hologo}[2016/05/12]
\usepackage{longtable}
\usepackage{array}
\setlength{\extrarowheight}{2pt}
\usepackage[T1]{fontenc}
\usepackage{lmodern}
\usepackage{pdflscape}
\usepackage[
  pdfencoding=auto,
]{hyperref}
\hypersetup{
  pdfauthor={Heiko Oberdiek},
  pdftitle={Example for package `hologo'},
  pdfsubject={Logos with fonts lmr, lmss, qtm, qpl, qhv},
}
\usepackage{bookmark}

% Print the logo list on the console

\begingroup
  \typeout{}%
  \typeout{*** Begin of logo list ***}%
  \newcommand*{\hologoEntry}[3]{%
    \typeout{#1 \ifx\\#2\\\else(#2) \fi[#3]}%
  }%
  \hologoList
  \typeout{*** End of logo list ***}%
  \typeout{}%
\endgroup

\begin{document}
\begin{landscape}

  \section{Example file for package `hologo'}

  % Table for font names

  \begin{longtable}{>{\bfseries}ll}
    \textbf{font} & \textbf{Font name}\\
    \hline
    lmr & Latin Modern Roman\\
    lmss & Latin Modern Sans\\
    qtm & \TeX\ Gyre Termes\\
    qhv & \TeX\ Gyre Heros\\
    qpl & \TeX\ Gyre Pagella\\
  \end{longtable}

  % Logo list with logos in different fonts

  \begingroup
    \newcommand*{\SetVariant}[2]{%
      \ifx\\#2\\%
      \else
        \hologoLogoSetup{#1}{variant=#2}%
      \fi
    }%
    \newcommand*{\hologoEntry}[3]{%
      \SetVariant{#1}{#2}%
      \raisebox{1em}[0pt][0pt]{\hypertarget{#1@#2}{}}%
      \bookmark[%
        dest={#1@#2},%
      ]{%
        #1\ifx\\#2\\\else\space(#2)\fi: \Hologo{#1}, \hologo{#1} %
        [Unicode]%
      }%
      \hypersetup{unicode=false}%
      \bookmark[%
        dest={#1@#2},%
      ]{%
        #1\ifx\\#2\\\else\space(#2)\fi: \Hologo{#1}, \hologo{#1} %
        [PDFDocEncoding]%
      }%
      \texttt{#1}%
      &%
      \texttt{#2}%
      &%
      \Hologo{#1}%
      &%
      \SetVariant{#1}{#2}%
      \hologo{#1}%
      &%
      \SetVariant{#1}{#2}%
      \fontfamily{qtm}\selectfont
      \hologo{#1}%
      &%
      \SetVariant{#1}{#2}%
      \fontfamily{qpl}\selectfont
      \hologo{#1}%
      &%
      \SetVariant{#1}{#2}%
      \textsf{\hologo{#1}}%
      &%
      \SetVariant{#1}{#2}%
      \fontfamily{qhv}\selectfont
      \hologo{#1}%
      \tabularnewline
    }%
    \begin{longtable}{llllllll}%
      \textbf{\textit{logo}} & \textbf{\textit{variant}} &
      \texttt{\string\Hologo} &
      \textbf{lmr} & \textbf{qtm} & \textbf{qpl} &
      \textbf{lmss} & \textbf{qhv}
      \tabularnewline
      \hline
      \endhead
      \hologoList
    \end{longtable}%
  \endgroup

\end{landscape}
\end{document}
%verbatim
%</example>
%    \end{macrocode}
%
% \StopEventually{
% }
%
% \section{Implementation}
%    \begin{macrocode}
%<*package>
%    \end{macrocode}
%    Reload check, especially if the package is not used with \LaTeX.
%    \begin{macrocode}
\begingroup\catcode61\catcode48\catcode32=10\relax%
  \catcode13=5 % ^^M
  \endlinechar=13 %
  \catcode35=6 % #
  \catcode39=12 % '
  \catcode44=12 % ,
  \catcode45=12 % -
  \catcode46=12 % .
  \catcode58=12 % :
  \catcode64=11 % @
  \catcode123=1 % {
  \catcode125=2 % }
  \expandafter\let\expandafter\x\csname ver@hologo.sty\endcsname
  \ifx\x\relax % plain-TeX, first loading
  \else
    \def\empty{}%
    \ifx\x\empty % LaTeX, first loading,
      % variable is initialized, but \ProvidesPackage not yet seen
    \else
      \expandafter\ifx\csname PackageInfo\endcsname\relax
        \def\x#1#2{%
          \immediate\write-1{Package #1 Info: #2.}%
        }%
      \else
        \def\x#1#2{\PackageInfo{#1}{#2, stopped}}%
      \fi
      \x{hologo}{The package is already loaded}%
      \aftergroup\endinput
    \fi
  \fi
\endgroup%
%    \end{macrocode}
%    Package identification:
%    \begin{macrocode}
\begingroup\catcode61\catcode48\catcode32=10\relax%
  \catcode13=5 % ^^M
  \endlinechar=13 %
  \catcode35=6 % #
  \catcode39=12 % '
  \catcode40=12 % (
  \catcode41=12 % )
  \catcode44=12 % ,
  \catcode45=12 % -
  \catcode46=12 % .
  \catcode47=12 % /
  \catcode58=12 % :
  \catcode64=11 % @
  \catcode91=12 % [
  \catcode93=12 % ]
  \catcode123=1 % {
  \catcode125=2 % }
  \expandafter\ifx\csname ProvidesPackage\endcsname\relax
    \def\x#1#2#3[#4]{\endgroup
      \immediate\write-1{Package: #3 #4}%
      \xdef#1{#4}%
    }%
  \else
    \def\x#1#2[#3]{\endgroup
      #2[{#3}]%
      \ifx#1\@undefined
        \xdef#1{#3}%
      \fi
      \ifx#1\relax
        \xdef#1{#3}%
      \fi
    }%
  \fi
\expandafter\x\csname ver@hologo.sty\endcsname
\ProvidesPackage{hologo}%
  [2016/05/12 v1.11 A logo collection with bookmark support (HO)]%
%    \end{macrocode}
%
%    \begin{macrocode}
\begingroup\catcode61\catcode48\catcode32=10\relax%
  \catcode13=5 % ^^M
  \endlinechar=13 %
  \catcode123=1 % {
  \catcode125=2 % }
  \catcode64=11 % @
  \def\x{\endgroup
    \expandafter\edef\csname HOLOGO@AtEnd\endcsname{%
      \endlinechar=\the\endlinechar\relax
      \catcode13=\the\catcode13\relax
      \catcode32=\the\catcode32\relax
      \catcode35=\the\catcode35\relax
      \catcode61=\the\catcode61\relax
      \catcode64=\the\catcode64\relax
      \catcode123=\the\catcode123\relax
      \catcode125=\the\catcode125\relax
    }%
  }%
\x\catcode61\catcode48\catcode32=10\relax%
\catcode13=5 % ^^M
\endlinechar=13 %
\catcode35=6 % #
\catcode64=11 % @
\catcode123=1 % {
\catcode125=2 % }
\def\TMP@EnsureCode#1#2{%
  \edef\HOLOGO@AtEnd{%
    \HOLOGO@AtEnd
    \catcode#1=\the\catcode#1\relax
  }%
  \catcode#1=#2\relax
}
\TMP@EnsureCode{10}{12}% ^^J
\TMP@EnsureCode{33}{12}% !
\TMP@EnsureCode{34}{12}% "
\TMP@EnsureCode{36}{3}% $
\TMP@EnsureCode{38}{4}% &
\TMP@EnsureCode{39}{12}% '
\TMP@EnsureCode{40}{12}% (
\TMP@EnsureCode{41}{12}% )
\TMP@EnsureCode{42}{12}% *
\TMP@EnsureCode{43}{12}% +
\TMP@EnsureCode{44}{12}% ,
\TMP@EnsureCode{45}{12}% -
\TMP@EnsureCode{46}{12}% .
\TMP@EnsureCode{47}{12}% /
\TMP@EnsureCode{58}{12}% :
\TMP@EnsureCode{59}{12}% ;
\TMP@EnsureCode{60}{12}% <
\TMP@EnsureCode{62}{12}% >
\TMP@EnsureCode{63}{12}% ?
\TMP@EnsureCode{91}{12}% [
\TMP@EnsureCode{93}{12}% ]
\TMP@EnsureCode{94}{7}% ^ (superscript)
\TMP@EnsureCode{95}{8}% _ (subscript)
\TMP@EnsureCode{96}{12}% `
\TMP@EnsureCode{124}{12}% |
\edef\HOLOGO@AtEnd{%
  \HOLOGO@AtEnd
  \escapechar\the\escapechar\relax
  \noexpand\endinput
}
\escapechar=92 %
%    \end{macrocode}
%
% \subsection{Logo list}
%
%    \begin{macro}{\hologoList}
%    \begin{macrocode}
\def\hologoList{%
  \hologoEntry{(La)TeX}{}{2011/10/01}%
  \hologoEntry{AmSLaTeX}{}{2010/04/16}%
  \hologoEntry{AmSTeX}{}{2010/04/16}%
  \hologoEntry{biber}{}{2011/10/01}%
  \hologoEntry{BibTeX}{}{2011/10/01}%
  \hologoEntry{BibTeX}{sf}{2011/10/01}%
  \hologoEntry{BibTeX}{sc}{2011/10/01}%
  \hologoEntry{BibTeX8}{}{2011/11/22}%
  \hologoEntry{ConTeXt}{}{2011/03/25}%
  \hologoEntry{ConTeXt}{narrow}{2011/03/25}%
  \hologoEntry{ConTeXt}{simple}{2011/03/25}%
  \hologoEntry{emTeX}{}{2010/04/26}%
  \hologoEntry{eTeX}{}{2010/04/08}%
  \hologoEntry{ExTeX}{}{2011/10/01}%
  \hologoEntry{HanTheThanh}{}{2011/11/29}%
  \hologoEntry{iniTeX}{}{2011/10/01}%
  \hologoEntry{KOMAScript}{}{2011/10/01}%
  \hologoEntry{La}{}{2010/05/08}%
  \hologoEntry{LaTeX}{}{2010/04/08}%
  \hologoEntry{LaTeX2e}{}{2010/04/08}%
  \hologoEntry{LaTeX3}{}{2010/04/24}%
  \hologoEntry{LaTeXe}{}{2010/04/08}%
  \hologoEntry{LaTeXML}{}{2011/11/22}%
  \hologoEntry{LaTeXTeX}{}{2011/10/01}%
  \hologoEntry{LuaLaTeX}{}{2010/04/08}%
  \hologoEntry{LuaTeX}{}{2010/04/08}%
  \hologoEntry{LyX}{}{2011/10/01}%
  \hologoEntry{METAFONT}{}{2011/10/01}%
  \hologoEntry{MetaFun}{}{2011/10/01}%
  \hologoEntry{METAPOST}{}{2011/10/01}%
  \hologoEntry{MetaPost}{}{2011/10/01}%
  \hologoEntry{MiKTeX}{}{2011/10/01}%
  \hologoEntry{NTS}{}{2011/10/01}%
  \hologoEntry{OzMF}{}{2011/10/01}%
  \hologoEntry{OzMP}{}{2011/10/01}%
  \hologoEntry{OzTeX}{}{2011/10/01}%
  \hologoEntry{OzTtH}{}{2011/10/01}%
  \hologoEntry{PCTeX}{}{2011/10/01}%
  \hologoEntry{pdfTeX}{}{2011/10/01}%
  \hologoEntry{pdfLaTeX}{}{2011/10/01}%
  \hologoEntry{PiC}{}{2011/10/01}%
  \hologoEntry{PiCTeX}{}{2011/10/01}%
  \hologoEntry{plainTeX}{}{2010/04/08}%
  \hologoEntry{plainTeX}{space}{2010/04/16}%
  \hologoEntry{plainTeX}{hyphen}{2010/04/16}%
  \hologoEntry{plainTeX}{runtogether}{2010/04/16}%
  \hologoEntry{SageTeX}{}{2011/11/22}%
  \hologoEntry{SLiTeX}{}{2011/10/01}%
  \hologoEntry{SLiTeX}{lift}{2011/10/01}%
  \hologoEntry{SLiTeX}{narrow}{2011/10/01}%
  \hologoEntry{SLiTeX}{simple}{2011/10/01}%
  \hologoEntry{SliTeX}{}{2011/10/01}%
  \hologoEntry{SliTeX}{narrow}{2011/10/01}%
  \hologoEntry{SliTeX}{simple}{2011/10/01}%
  \hologoEntry{SliTeX}{lift}{2011/10/01}%
  \hologoEntry{teTeX}{}{2011/10/01}%
  \hologoEntry{TeX}{}{2010/04/08}%
  \hologoEntry{TeX4ht}{}{2011/11/22}%
  \hologoEntry{TTH}{}{2011/11/22}%
  \hologoEntry{virTeX}{}{2011/10/01}%
  \hologoEntry{VTeX}{}{2010/04/24}%
  \hologoEntry{Xe}{}{2010/04/08}%
  \hologoEntry{XeLaTeX}{}{2010/04/08}%
  \hologoEntry{XeTeX}{}{2010/04/08}%
}
%    \end{macrocode}
%    \end{macro}
%
% \subsection{Load resources}
%
%    \begin{macrocode}
\begingroup\expandafter\expandafter\expandafter\endgroup
\expandafter\ifx\csname RequirePackage\endcsname\relax
  \def\TMP@RequirePackage#1[#2]{%
    \begingroup\expandafter\expandafter\expandafter\endgroup
    \expandafter\ifx\csname ver@#1.sty\endcsname\relax
      \input #1.sty\relax
    \fi
  }%
  \TMP@RequirePackage{ltxcmds}[2011/02/04]%
  \TMP@RequirePackage{infwarerr}[2010/04/08]%
  \TMP@RequirePackage{kvsetkeys}[2010/03/01]%
  \TMP@RequirePackage{kvdefinekeys}[2010/03/01]%
  \TMP@RequirePackage{pdftexcmds}[2010/04/01]%
  \TMP@RequirePackage{ifpdf}[2010/01/28]%
  \TMP@RequirePackage{ifluatex}[2010/03/01]%
  \ltx@IfUndefined{newif}{%
    \expandafter\let\csname newif\endcsname\ltx@newif
  }{}%
  \TMP@RequirePackage{ifxetex}[2009/01/23]%
  \TMP@RequirePackage{ifvtex}[2010/03/01]%
\else
  \RequirePackage{ltxcmds}[2011/02/04]%
  \RequirePackage{infwarerr}[2010/04/08]%
  \RequirePackage{kvsetkeys}[2010/03/01]%
  \RequirePackage{kvdefinekeys}[2010/03/01]%
  \RequirePackage{pdftexcmds}[2010/04/01]%
  \RequirePackage{ifpdf}[2010/01/28]%
  \RequirePackage{ifluatex}[2010/03/01]%
  \RequirePackage{ifxetex}[2009/01/23]%
  \RequirePackage{ifvtex}[2010/03/01]%
\fi
%    \end{macrocode}
%
%    \begin{macro}{\HOLOGO@IfDefined}
%    \begin{macrocode}
\def\HOLOGO@IfExists#1{%
  \ifx\@undefined#1%
    \expandafter\ltx@secondoftwo
  \else
    \ifx\relax#1%
      \expandafter\ltx@secondoftwo
    \else
      \expandafter\expandafter\expandafter\ltx@firstoftwo
    \fi
  \fi
}
%    \end{macrocode}
%    \end{macro}
%
% \subsection{Setup macros}
%
%    \begin{macro}{\hologoSetup}
%    \begin{macrocode}
\def\hologoSetup{%
  \let\HOLOGO@name\relax
  \HOLOGO@Setup
}
%    \end{macrocode}
%    \end{macro}
%
%    \begin{macro}{\hologoLogoSetup}
%    \begin{macrocode}
\def\hologoLogoSetup#1{%
  \edef\HOLOGO@name{#1}%
  \ltx@IfUndefined{HoLogo@\HOLOGO@name}{%
    \@PackageError{hologo}{%
      Unknown logo `\HOLOGO@name'%
    }\@ehc
    \ltx@gobble
  }{%
    \HOLOGO@Setup
  }%
}
%    \end{macrocode}
%    \end{macro}
%
%    \begin{macro}{\HOLOGO@Setup}
%    \begin{macrocode}
\def\HOLOGO@Setup{%
  \kvsetkeys{HoLogo}%
}
%    \end{macrocode}
%    \end{macro}
%
% \subsection{Options}
%
%    \begin{macro}{\HOLOGO@DeclareBoolOption}
%    \begin{macrocode}
\def\HOLOGO@DeclareBoolOption#1{%
  \expandafter\chardef\csname HOLOGOOPT@#1\endcsname\ltx@zero
  \kv@define@key{HoLogo}{#1}[true]{%
    \def\HOLOGO@temp{##1}%
    \ifx\HOLOGO@temp\HOLOGO@true
      \ifx\HOLOGO@name\relax
        \expandafter\chardef\csname HOLOGOOPT@#1\endcsname=\ltx@one
      \else
        \expandafter\chardef\csname
        HoLogoOpt@#1@\HOLOGO@name\endcsname\ltx@one
      \fi
      \HOLOGO@SetBreakAll{#1}%
    \else
      \ifx\HOLOGO@temp\HOLOGO@false
        \ifx\HOLOGO@name\relax
          \expandafter\chardef\csname HOLOGOOPT@#1\endcsname=\ltx@zero
        \else
          \expandafter\chardef\csname
          HoLogoOpt@#1@\HOLOGO@name\endcsname=\ltx@zero
        \fi
        \HOLOGO@SetBreakAll{#1}%
      \else
        \@PackageError{hologo}{%
          Unknown value `##1' for boolean option `#1'.\MessageBreak
          Known values are `true' and `false'%
        }\@ehc
      \fi
    \fi
  }%
}
%    \end{macrocode}
%    \end{macro}
%
%    \begin{macro}{\HOLOGO@SetBreakAll}
%    \begin{macrocode}
\def\HOLOGO@SetBreakAll#1{%
  \def\HOLOGO@temp{#1}%
  \ifx\HOLOGO@temp\HOLOGO@break
    \ifx\HOLOGO@name\relax
      \chardef\HOLOGOOPT@hyphenbreak=\HOLOGOOPT@break
      \chardef\HOLOGOOPT@spacebreak=\HOLOGOOPT@break
      \chardef\HOLOGOOPT@discretionarybreak=\HOLOGOOPT@break
    \else
      \expandafter\chardef
         \csname HoLogoOpt@hyphenbreak@\HOLOGO@name\endcsname=%
         \csname HoLogoOpt@break@\HOLOGO@name\endcsname
      \expandafter\chardef
         \csname HoLogoOpt@spacebreak@\HOLOGO@name\endcsname=%
         \csname HoLogoOpt@break@\HOLOGO@name\endcsname
      \expandafter\chardef
         \csname HoLogoOpt@discretionarybreak@\HOLOGO@name
             \endcsname=%
         \csname HoLogoOpt@break@\HOLOGO@name\endcsname
    \fi
  \fi
}
%    \end{macrocode}
%    \end{macro}
%
%    \begin{macro}{\HOLOGO@true}
%    \begin{macrocode}
\def\HOLOGO@true{true}
%    \end{macrocode}
%    \end{macro}
%    \begin{macro}{\HOLOGO@false}
%    \begin{macrocode}
\def\HOLOGO@false{false}
%    \end{macrocode}
%    \end{macro}
%    \begin{macro}{\HOLOGO@break}
%    \begin{macrocode}
\def\HOLOGO@break{break}
%    \end{macrocode}
%    \end{macro}
%
%    \begin{macrocode}
\HOLOGO@DeclareBoolOption{break}
\HOLOGO@DeclareBoolOption{hyphenbreak}
\HOLOGO@DeclareBoolOption{spacebreak}
\HOLOGO@DeclareBoolOption{discretionarybreak}
%    \end{macrocode}
%
%    \begin{macrocode}
\kv@define@key{HoLogo}{variant}{%
  \ifx\HOLOGO@name\relax
    \@PackageError{hologo}{%
      Option `variant' is not available in \string\hologoSetup,%
      \MessageBreak
      Use \string\hologoLogoSetup\space instead%
    }\@ehc
  \else
    \edef\HOLOGO@temp{#1}%
    \ifx\HOLOGO@temp\ltx@empty
      \expandafter
      \let\csname HoLogoOpt@variant@\HOLOGO@name\endcsname\@undefined
    \else
      \ltx@IfUndefined{HoLogo@\HOLOGO@name @\HOLOGO@temp}{%
        \@PackageError{hologo}{%
          Unknown variant `\HOLOGO@temp' of logo `\HOLOGO@name'%
        }\@ehc
      }{%
        \expandafter
        \let\csname HoLogoOpt@variant@\HOLOGO@name\endcsname
            \HOLOGO@temp
      }%
    \fi
  \fi
}
%    \end{macrocode}
%
%    \begin{macro}{\HOLOGO@Variant}
%    \begin{macrocode}
\def\HOLOGO@Variant#1{%
  #1%
  \ltx@ifundefined{HoLogoOpt@variant@#1}{%
  }{%
    @\csname HoLogoOpt@variant@#1\endcsname
  }%
}
%    \end{macrocode}
%    \end{macro}
%
% \subsection{Break/no-break support}
%
%    \begin{macro}{\HOLOGO@space}
%    \begin{macrocode}
\def\HOLOGO@space{%
  \ltx@ifundefined{HoLogoOpt@spacebreak@\HOLOGO@name}{%
    \ltx@ifundefined{HoLogoOpt@break@\HOLOGO@name}{%
      \chardef\HOLOGO@temp=\HOLOGOOPT@spacebreak
    }{%
      \chardef\HOLOGO@temp=%
        \csname HoLogoOpt@break@\HOLOGO@name\endcsname
    }%
  }{%
    \chardef\HOLOGO@temp=%
      \csname HoLogoOpt@spacebreak@\HOLOGO@name\endcsname
  }%
  \ifcase\HOLOGO@temp
    \penalty10000 %
  \fi
  \ltx@space
}
%    \end{macrocode}
%    \end{macro}
%
%    \begin{macro}{\HOLOGO@hyphen}
%    \begin{macrocode}
\def\HOLOGO@hyphen{%
  \ltx@ifundefined{HoLogoOpt@hyphenbreak@\HOLOGO@name}{%
    \ltx@ifundefined{HoLogoOpt@break@\HOLOGO@name}{%
      \chardef\HOLOGO@temp=\HOLOGOOPT@hyphenbreak
    }{%
      \chardef\HOLOGO@temp=%
        \csname HoLogoOpt@break@\HOLOGO@name\endcsname
    }%
  }{%
    \chardef\HOLOGO@temp=%
      \csname HoLogoOpt@hyphenbreak@\HOLOGO@name\endcsname
  }%
  \ifcase\HOLOGO@temp
    \ltx@mbox{-}%
  \else
    -%
  \fi
}
%    \end{macrocode}
%    \end{macro}
%
%    \begin{macro}{\HOLOGO@discretionary}
%    \begin{macrocode}
\def\HOLOGO@discretionary{%
  \ltx@ifundefined{HoLogoOpt@discretionarybreak@\HOLOGO@name}{%
    \ltx@ifundefined{HoLogoOpt@break@\HOLOGO@name}{%
      \chardef\HOLOGO@temp=\HOLOGOOPT@discretionarybreak
    }{%
      \chardef\HOLOGO@temp=%
        \csname HoLogoOpt@break@\HOLOGO@name\endcsname
    }%
  }{%
    \chardef\HOLOGO@temp=%
      \csname HoLogoOpt@discretionarybreak@\HOLOGO@name\endcsname
  }%
  \ifcase\HOLOGO@temp
  \else
    \-%
  \fi
}
%    \end{macrocode}
%    \end{macro}
%
%    \begin{macro}{\HOLOGO@mbox}
%    \begin{macrocode}
\def\HOLOGO@mbox#1{%
  \ltx@ifundefined{HoLogoOpt@break@\HOLOGO@name}{%
    \chardef\HOLOGO@temp=\HOLOGOOPT@hyphenbreak
  }{%
    \chardef\HOLOGO@temp=%
      \csname HoLogoOpt@break@\HOLOGO@name\endcsname
  }%
  \ifcase\HOLOGO@temp
    \ltx@mbox{#1}%
  \else
    #1%
  \fi
}
%    \end{macrocode}
%    \end{macro}
%
% \subsection{Font support}
%
%    \begin{macro}{\HoLogoFont@font}
%    \begin{tabular}{@{}ll@{}}
%    |#1|:& logo name\\
%    |#2|:& font short name\\
%    |#3|:& text
%    \end{tabular}
%    \begin{macrocode}
\def\HoLogoFont@font#1#2#3{%
  \begingroup
    \ltx@IfUndefined{HoLogoFont@logo@#1.#2}{%
      \ltx@IfUndefined{HoLogoFont@font@#2}{%
        \@PackageWarning{hologo}{%
          Missing font `#2' for logo `#1'%
        }%
        #3%
      }{%
        \csname HoLogoFont@font@#2\endcsname{#3}%
      }%
    }{%
      \csname HoLogoFont@logo@#1.#2\endcsname{#3}%
    }%
  \endgroup
}
%    \end{macrocode}
%    \end{macro}
%
%    \begin{macro}{\HoLogoFont@Def}
%    \begin{macrocode}
\def\HoLogoFont@Def#1{%
  \expandafter\def\csname HoLogoFont@font@#1\endcsname
}
%    \end{macrocode}
%    \end{macro}
%    \begin{macro}{\HoLogoFont@LogoDef}
%    \begin{macrocode}
\def\HoLogoFont@LogoDef#1#2{%
  \expandafter\def\csname HoLogoFont@logo@#1.#2\endcsname
}
%    \end{macrocode}
%    \end{macro}
%
% \subsubsection{Font defaults}
%
%    \begin{macro}{\HoLogoFont@font@general}
%    \begin{macrocode}
\HoLogoFont@Def{general}{}%
%    \end{macrocode}
%    \end{macro}
%
%    \begin{macro}{\HoLogoFont@font@rm}
%    \begin{macrocode}
\ltx@IfUndefined{rmfamily}{%
  \ltx@IfUndefined{rm}{%
  }{%
    \HoLogoFont@Def{rm}{\rm}%
  }%
}{%
  \HoLogoFont@Def{rm}{\rmfamily}%
}
%    \end{macrocode}
%    \end{macro}
%
%    \begin{macro}{\HoLogoFont@font@sf}
%    \begin{macrocode}
\ltx@IfUndefined{sffamily}{%
  \ltx@IfUndefined{sf}{%
  }{%
    \HoLogoFont@Def{sf}{\sf}%
  }%
}{%
  \HoLogoFont@Def{sf}{\sffamily}%
}
%    \end{macrocode}
%    \end{macro}
%
%    \begin{macro}{\HoLogoFont@font@bibsf}
%    In case of \hologo{plainTeX} the original small caps
%    variant is used as default. In \hologo{LaTeX}
%    the definition of package \xpackage{dtklogos} \cite{dtklogos}
%    is used.
%\begin{quote}
%\begin{verbatim}
%\DeclareRobustCommand{\BibTeX}{%
%  B%
%  \kern-.05em%
%  \hbox{%
%    $\m@th$% %% force math size calculations
%    \csname S@\f@size\endcsname
%    \fontsize\sf@size\z@
%    \math@fontsfalse
%    \selectfont
%    I%
%    \kern-.025em%
%    B
%  }%
%  \kern-.08em%
%  \-%
%  \TeX
%}
%\end{verbatim}
%\end{quote}
%    \begin{macrocode}
\ltx@IfUndefined{selectfont}{%
  \ltx@IfUndefined{tensc}{%
    \font\tensc=cmcsc10\relax
  }{}%
  \HoLogoFont@Def{bibsf}{\tensc}%
}{%
  \HoLogoFont@Def{bibsf}{%
    $\mathsurround=0pt$%
    \csname S@\f@size\endcsname
    \fontsize\sf@size{0pt}%
    \math@fontsfalse
    \selectfont
  }%
}
%    \end{macrocode}
%    \end{macro}
%
%    \begin{macro}{\HoLogoFont@font@sc}
%    \begin{macrocode}
\ltx@IfUndefined{scshape}{%
  \ltx@IfUndefined{tensc}{%
    \font\tensc=cmcsc10\relax
  }{}%
  \HoLogoFont@Def{sc}{\tensc}%
}{%
  \HoLogoFont@Def{sc}{\scshape}%
}
%    \end{macrocode}
%    \end{macro}
%
%    \begin{macro}{\HoLogoFont@font@sy}
%    \begin{macrocode}
\ltx@IfUndefined{usefont}{%
  \ltx@IfUndefined{tensy}{%
  }{%
    \HoLogoFont@Def{sy}{\tensy}%
  }%
}{%
  \HoLogoFont@Def{sy}{%
    \usefont{OMS}{cmsy}{m}{n}%
  }%
}
%    \end{macrocode}
%    \end{macro}
%
%    \begin{macro}{\HoLogoFont@font@logo}
%    \begin{macrocode}
\begingroup
  \def\x{LaTeX2e}%
\expandafter\endgroup
\ifx\fmtname\x
  \ltx@IfUndefined{logofamily}{%
    \DeclareRobustCommand\logofamily{%
      \not@math@alphabet\logofamily\relax
      \fontencoding{U}%
      \fontfamily{logo}%
      \selectfont
    }%
  }{}%
  \ltx@IfUndefined{logofamily}{%
  }{%
    \HoLogoFont@Def{logo}{\logofamily}%
  }%
\else
  \ltx@IfUndefined{tenlogo}{%
    \font\tenlogo=logo10\relax
  }{}%
  \HoLogoFont@Def{logo}{\tenlogo}%
\fi
%    \end{macrocode}
%    \end{macro}
%
% \subsubsection{Font setup}
%
%    \begin{macro}{\hologoFontSetup}
%    \begin{macrocode}
\def\hologoFontSetup{%
  \let\HOLOGO@name\relax
  \HOLOGO@FontSetup
}
%    \end{macrocode}
%    \end{macro}
%
%    \begin{macro}{\hologoLogoFontSetup}
%    \begin{macrocode}
\def\hologoLogoFontSetup#1{%
  \edef\HOLOGO@name{#1}%
  \ltx@IfUndefined{HoLogo@\HOLOGO@name}{%
    \@PackageError{hologo}{%
      Unknown logo `\HOLOGO@name'%
    }\@ehc
    \ltx@gobble
  }{%
    \HOLOGO@FontSetup
  }%
}
%    \end{macrocode}
%    \end{macro}
%
%    \begin{macro}{\HOLOGO@FontSetup}
%    \begin{macrocode}
\def\HOLOGO@FontSetup{%
  \kvsetkeys{HoLogoFont}%
}
%    \end{macrocode}
%    \end{macro}
%
%    \begin{macrocode}
\def\HOLOGO@temp#1{%
  \kv@define@key{HoLogoFont}{#1}{%
    \ifx\HOLOGO@name\relax
      \HoLogoFont@Def{#1}{##1}%
    \else
      \HoLogoFont@LogoDef\HOLOGO@name{#1}{##1}%
    \fi
  }%
}
\HOLOGO@temp{general}
\HOLOGO@temp{sf}
%    \end{macrocode}
%
% \subsection{Generic logo commands}
%
%    \begin{macrocode}
\HOLOGO@IfExists\hologo{%
  \@PackageError{hologo}{%
    \string\hologo\ltx@space is already defined.\MessageBreak
    Package loading is aborted%
  }\@ehc
  \HOLOGO@AtEnd
}%
\HOLOGO@IfExists\hologoRobust{%
  \@PackageError{hologo}{%
    \string\hologoRobust\ltx@space is already defined.\MessageBreak
    Package loading is aborted%
  }\@ehc
  \HOLOGO@AtEnd
}%
%    \end{macrocode}
%
% \subsubsection{\cs{hologo} and friends}
%
%    \begin{macrocode}
\ifluatex
  \expandafter\ltx@firstofone
\else
  \expandafter\ltx@gobble
\fi
{%
  \ltx@IfUndefined{ifincsname}{%
    \ifnum\luatexversion<36 %
      \expandafter\ltx@gobble
    \else
      \expandafter\ltx@firstofone
    \fi
    {%
      \begingroup
        \ifcase0%
            \directlua{%
              if tex.enableprimitives then %
                tex.enableprimitives('HOLOGO@', {'ifincsname'})%
              else %
                tex.print('1')%
              end%
            }%
            \ifx\HOLOGO@ifincsname\@undefined 1\fi%
            \relax
          \expandafter\ltx@firstofone
        \else
          \endgroup
          \expandafter\ltx@gobble
        \fi
        {%
          \global\let\ifincsname\HOLOGO@ifincsname
        }%
      \HOLOGO@temp
    }%
  }{}%
}
%    \end{macrocode}
%    \begin{macrocode}
\ltx@IfUndefined{ifincsname}{%
  \catcode`$=14 %
}{%
  \catcode`$=9 %
}
%    \end{macrocode}
%
%    \begin{macro}{\hologo}
%    \begin{macrocode}
\def\hologo#1{%
$ \ifincsname
$   \ltx@ifundefined{HoLogoCs@\HOLOGO@Variant{#1}}{%
$     #1%
$   }{%
$     \csname HoLogoCs@\HOLOGO@Variant{#1}\endcsname\ltx@firstoftwo
$   }%
$ \else
    \HOLOGO@IfExists\texorpdfstring\texorpdfstring\ltx@firstoftwo
    {%
      \hologoRobust{#1}%
    }{%
      \ltx@ifundefined{HoLogoBkm@\HOLOGO@Variant{#1}}{%
        \ltx@ifundefined{HoLogo@#1}{?#1?}{#1}%
      }{%
        \csname HoLogoBkm@\HOLOGO@Variant{#1}\endcsname
        \ltx@firstoftwo
      }%
    }%
$ \fi
}
%    \end{macrocode}
%    \end{macro}
%    \begin{macro}{\Hologo}
%    \begin{macrocode}
\def\Hologo#1{%
$ \ifincsname
$   \ltx@ifundefined{HoLogoCs@\HOLOGO@Variant{#1}}{%
$     #1%
$   }{%
$     \csname HoLogoCs@\HOLOGO@Variant{#1}\endcsname\ltx@secondoftwo
$   }%
$ \else
    \HOLOGO@IfExists\texorpdfstring\texorpdfstring\ltx@firstoftwo
    {%
      \HologoRobust{#1}%
    }{%
      \ltx@ifundefined{HoLogoBkm@\HOLOGO@Variant{#1}}{%
        \ltx@ifundefined{HoLogo@#1}{?#1?}{#1}%
      }{%
        \csname HoLogoBkm@\HOLOGO@Variant{#1}\endcsname
        \ltx@secondoftwo
      }%
    }%
$ \fi
}
%    \end{macrocode}
%    \end{macro}
%
%    \begin{macro}{\hologoVariant}
%    \begin{macrocode}
\def\hologoVariant#1#2{%
  \ifx\relax#2\relax
    \hologo{#1}%
  \else
$   \ifincsname
$     \ltx@ifundefined{HoLogoCs@#1@#2}{%
$       #1%
$     }{%
$       \csname HoLogoCs@#1@#2\endcsname\ltx@firstoftwo
$     }%
$   \else
      \HOLOGO@IfExists\texorpdfstring\texorpdfstring\ltx@firstoftwo
      {%
        \hologoVariantRobust{#1}{#2}%
      }{%
        \ltx@ifundefined{HoLogoBkm@#1@#2}{%
          \ltx@ifundefined{HoLogo@#1}{?#1?}{#1}%
        }{%
          \csname HoLogoBkm@#1@#2\endcsname
          \ltx@firstoftwo
        }%
      }%
$   \fi
  \fi
}
%    \end{macrocode}
%    \end{macro}
%    \begin{macro}{\HologoVariant}
%    \begin{macrocode}
\def\HologoVariant#1#2{%
  \ifx\relax#2\relax
    \Hologo{#1}%
  \else
$   \ifincsname
$     \ltx@ifundefined{HoLogoCs@#1@#2}{%
$       #1%
$     }{%
$       \csname HoLogoCs@#1@#2\endcsname\ltx@secondoftwo
$     }%
$   \else
      \HOLOGO@IfExists\texorpdfstring\texorpdfstring\ltx@firstoftwo
      {%
        \HologoVariantRobust{#1}{#2}%
      }{%
        \ltx@ifundefined{HoLogoBkm@#1@#2}{%
          \ltx@ifundefined{HoLogo@#1}{?#1?}{#1}%
        }{%
          \csname HoLogoBkm@#1@#2\endcsname
          \ltx@secondoftwo
        }%
      }%
$   \fi
  \fi
}
%    \end{macrocode}
%    \end{macro}
%
%    \begin{macrocode}
\catcode`\$=3 %
%    \end{macrocode}
%
% \subsubsection{\cs{hologoRobust} and friends}
%
%    \begin{macro}{\hologoRobust}
%    \begin{macrocode}
\ltx@IfUndefined{protected}{%
  \ltx@IfUndefined{DeclareRobustCommand}{%
    \def\hologoRobust#1%
  }{%
    \DeclareRobustCommand*\hologoRobust[1]%
  }%
}{%
  \protected\def\hologoRobust#1%
}%
{%
  \edef\HOLOGO@name{#1}%
  \ltx@IfUndefined{HoLogo@\HOLOGO@Variant\HOLOGO@name}{%
    \@PackageError{hologo}{%
      Unknown logo `\HOLOGO@name'%
    }\@ehc
    ?\HOLOGO@name?%
  }{%
    \ltx@IfUndefined{ver@tex4ht.sty}{%
      \HoLogoFont@font\HOLOGO@name{general}{%
        \csname HoLogo@\HOLOGO@Variant\HOLOGO@name\endcsname
        \ltx@firstoftwo
      }%
    }{%
      \ltx@IfUndefined{HoLogoHtml@\HOLOGO@Variant\HOLOGO@name}{%
        \HOLOGO@name
      }{%
        \csname HoLogoHtml@\HOLOGO@Variant\HOLOGO@name\endcsname
        \ltx@firstoftwo
      }%
    }%
  }%
}
%    \end{macrocode}
%    \end{macro}
%    \begin{macro}{\HologoRobust}
%    \begin{macrocode}
\ltx@IfUndefined{protected}{%
  \ltx@IfUndefined{DeclareRobustCommand}{%
    \def\HologoRobust#1%
  }{%
    \DeclareRobustCommand*\HologoRobust[1]%
  }%
}{%
  \protected\def\HologoRobust#1%
}%
{%
  \edef\HOLOGO@name{#1}%
  \ltx@IfUndefined{HoLogo@\HOLOGO@Variant\HOLOGO@name}{%
    \@PackageError{hologo}{%
      Unknown logo `\HOLOGO@name'%
    }\@ehc
    ?\HOLOGO@name?%
  }{%
    \ltx@IfUndefined{ver@tex4ht.sty}{%
      \HoLogoFont@font\HOLOGO@name{general}{%
        \csname HoLogo@\HOLOGO@Variant\HOLOGO@name\endcsname
        \ltx@secondoftwo
      }%
    }{%
      \ltx@IfUndefined{HoLogoHtml@\HOLOGO@Variant\HOLOGO@name}{%
        \expandafter\HOLOGO@Uppercase\HOLOGO@name
      }{%
        \csname HoLogoHtml@\HOLOGO@Variant\HOLOGO@name\endcsname
        \ltx@secondoftwo
      }%
    }%
  }%
}
%    \end{macrocode}
%    \end{macro}
%    \begin{macro}{\hologoVariantRobust}
%    \begin{macrocode}
\ltx@IfUndefined{protected}{%
  \ltx@IfUndefined{DeclareRobustCommand}{%
    \def\hologoVariantRobust#1#2%
  }{%
    \DeclareRobustCommand*\hologoVariantRobust[2]%
  }%
}{%
  \protected\def\hologoVariantRobust#1#2%
}%
{%
  \begingroup
    \hologoLogoSetup{#1}{variant={#2}}%
    \hologoRobust{#1}%
  \endgroup
}
%    \end{macrocode}
%    \end{macro}
%    \begin{macro}{\HologoVariantRobust}
%    \begin{macrocode}
\ltx@IfUndefined{protected}{%
  \ltx@IfUndefined{DeclareRobustCommand}{%
    \def\HologoVariantRobust#1#2%
  }{%
    \DeclareRobustCommand*\HologoVariantRobust[2]%
  }%
}{%
  \protected\def\HologoVariantRobust#1#2%
}%
{%
  \begingroup
    \hologoLogoSetup{#1}{variant={#2}}%
    \HologoRobust{#1}%
  \endgroup
}
%    \end{macrocode}
%    \end{macro}
%
%    \begin{macro}{\hologorobust}
%    Macro \cs{hologorobust} is only defined for compatibility.
%    Its use is deprecated.
%    \begin{macrocode}
\def\hologorobust{\hologoRobust}
%    \end{macrocode}
%    \end{macro}
%
% \subsection{Helpers}
%
%    \begin{macro}{\HOLOGO@Uppercase}
%    Macro \cs{HOLOGO@Uppercase} is restricted to \cs{uppercase},
%    because \hologo{plainTeX} or \hologo{iniTeX} do not provide
%    \cs{MakeUppercase}.
%    \begin{macrocode}
\def\HOLOGO@Uppercase#1{\uppercase{#1}}
%    \end{macrocode}
%    \end{macro}
%
%    \begin{macro}{\HOLOGO@PdfdocUnicode}
%    \begin{macrocode}
\def\HOLOGO@PdfdocUnicode{%
  \ifx\ifHy@unicode\iftrue
    \expandafter\ltx@secondoftwo
  \else
    \expandafter\ltx@firstoftwo
  \fi
}
%    \end{macrocode}
%    \end{macro}
%
%    \begin{macro}{\HOLOGO@Math}
%    \begin{macrocode}
\def\HOLOGO@MathSetup{%
  \mathsurround0pt\relax
  \HOLOGO@IfExists\f@series{%
    \if b\expandafter\ltx@car\f@series x\@nil
      \csname boldmath\endcsname
   \fi
  }{}%
}
%    \end{macrocode}
%    \end{macro}
%
%    \begin{macro}{\HOLOGO@TempDimen}
%    \begin{macrocode}
\dimendef\HOLOGO@TempDimen=\ltx@zero
%    \end{macrocode}
%    \end{macro}
%    \begin{macro}{\HOLOGO@NegativeKerning}
%    \begin{macrocode}
\def\HOLOGO@NegativeKerning#1{%
  \begingroup
    \HOLOGO@TempDimen=0pt\relax
    \comma@parse@normalized{#1}{%
      \ifdim\HOLOGO@TempDimen=0pt %
        \expandafter\HOLOGO@@NegativeKerning\comma@entry
      \fi
      \ltx@gobble
    }%
    \ifdim\HOLOGO@TempDimen<0pt %
      \kern\HOLOGO@TempDimen
    \fi
  \endgroup
}
%    \end{macrocode}
%    \end{macro}
%    \begin{macro}{\HOLOGO@@NegativeKerning}
%    \begin{macrocode}
\def\HOLOGO@@NegativeKerning#1#2{%
  \setbox\ltx@zero\hbox{#1#2}%
  \HOLOGO@TempDimen=\wd\ltx@zero
  \setbox\ltx@zero\hbox{#1\kern0pt#2}%
  \advance\HOLOGO@TempDimen by -\wd\ltx@zero
}
%    \end{macrocode}
%    \end{macro}
%
%    \begin{macro}{\HOLOGO@SpaceFactor}
%    \begin{macrocode}
\def\HOLOGO@SpaceFactor{%
  \spacefactor1000 %
}
%    \end{macrocode}
%    \end{macro}
%
%    \begin{macro}{\HOLOGO@Span}
%    \begin{macrocode}
\def\HOLOGO@Span#1#2{%
  \HCode{<span class="HoLogo-#1">}%
  #2%
  \HCode{</span>}%
}
%    \end{macrocode}
%    \end{macro}
%
% \subsubsection{Text subscript}
%
%    \begin{macro}{\HOLOGO@SubScript}%
%    \begin{macrocode}
\def\HOLOGO@SubScript#1{%
  \ltx@IfUndefined{textsubscript}{%
    \ltx@IfUndefined{text}{%
      \ltx@mbox{%
        \mathsurround=0pt\relax
        $%
          _{%
            \ltx@IfUndefined{sf@size}{%
              \mathrm{#1}%
            }{%
              \mbox{%
                \fontsize\sf@size{0pt}\selectfont
                #1%
              }%
            }%
          }%
        $%
      }%
    }{%
      \ltx@mbox{%
        \mathsurround=0pt\relax
        $_{\text{#1}}$%
      }%
    }%
  }{%
    \textsubscript{#1}%
  }%
}
%    \end{macrocode}
%    \end{macro}
%
% \subsection{\hologo{TeX} and friends}
%
% \subsubsection{\hologo{TeX}}
%
%    \begin{macro}{\HoLogo@TeX}
%    Source: \hologo{LaTeX} kernel.
%    \begin{macrocode}
\def\HoLogo@TeX#1{%
  T\kern-.1667em\lower.5ex\hbox{E}\kern-.125emX\HOLOGO@SpaceFactor
}
%    \end{macrocode}
%    \end{macro}
%    \begin{macro}{\HoLogoHtml@TeX}
%    \begin{macrocode}
\def\HoLogoHtml@TeX#1{%
  \HoLogoCss@TeX
  \HOLOGO@Span{TeX}{%
    T%
    \HOLOGO@Span{e}{%
      E%
    }%
    X%
  }%
}
%    \end{macrocode}
%    \end{macro}
%    \begin{macro}{\HoLogoCss@TeX}
%    \begin{macrocode}
\def\HoLogoCss@TeX{%
  \Css{%
    span.HoLogo-TeX span.HoLogo-e{%
      position:relative;%
      top:.5ex;%
      margin-left:-.1667em;%
      margin-right:-.125em;%
    }%
  }%
  \Css{%
    a span.HoLogo-TeX span.HoLogo-e{%
      text-decoration:none;%
    }%
  }%
  \global\let\HoLogoCss@TeX\relax
}
%    \end{macrocode}
%    \end{macro}
%
% \subsubsection{\hologo{plainTeX}}
%
%    \begin{macro}{\HoLogo@plainTeX@space}
%    Source: ``The \hologo{TeX}book''
%    \begin{macrocode}
\def\HoLogo@plainTeX@space#1{%
  \HOLOGO@mbox{#1{p}{P}lain}\HOLOGO@space\hologo{TeX}%
}
%    \end{macrocode}
%    \end{macro}
%    \begin{macro}{\HoLogoCs@plainTeX@space}
%    \begin{macrocode}
\def\HoLogoCs@plainTeX@space#1{#1{p}{P}lain TeX}%
%    \end{macrocode}
%    \end{macro}
%    \begin{macro}{\HoLogoBkm@plainTeX@space}
%    \begin{macrocode}
\def\HoLogoBkm@plainTeX@space#1{%
  #1{p}{P}lain \hologo{TeX}%
}
%    \end{macrocode}
%    \end{macro}
%    \begin{macro}{\HoLogoHtml@plainTeX@space}
%    \begin{macrocode}
\def\HoLogoHtml@plainTeX@space#1{%
  #1{p}{P}lain \hologo{TeX}%
}
%    \end{macrocode}
%    \end{macro}
%
%    \begin{macro}{\HoLogo@plainTeX@hyphen}
%    \begin{macrocode}
\def\HoLogo@plainTeX@hyphen#1{%
  \HOLOGO@mbox{#1{p}{P}lain}\HOLOGO@hyphen\hologo{TeX}%
}
%    \end{macrocode}
%    \end{macro}
%    \begin{macro}{\HoLogoCs@plainTeX@hyphen}
%    \begin{macrocode}
\def\HoLogoCs@plainTeX@hyphen#1{#1{p}{P}lain-TeX}
%    \end{macrocode}
%    \end{macro}
%    \begin{macro}{\HoLogoBkm@plainTeX@hyphen}
%    \begin{macrocode}
\def\HoLogoBkm@plainTeX@hyphen#1{%
  #1{p}{P}lain-\hologo{TeX}%
}
%    \end{macrocode}
%    \end{macro}
%    \begin{macro}{\HoLogoHtml@plainTeX@hyphen}
%    \begin{macrocode}
\def\HoLogoHtml@plainTeX@hyphen#1{%
  #1{p}{P}lain-\hologo{TeX}%
}
%    \end{macrocode}
%    \end{macro}
%
%    \begin{macro}{\HoLogo@plainTeX@runtogether}
%    \begin{macrocode}
\def\HoLogo@plainTeX@runtogether#1{%
  \HOLOGO@mbox{#1{p}{P}lain\hologo{TeX}}%
}
%    \end{macrocode}
%    \end{macro}
%    \begin{macro}{\HoLogoCs@plainTeX@runtogether}
%    \begin{macrocode}
\def\HoLogoCs@plainTeX@runtogether#1{#1{p}{P}lainTeX}
%    \end{macrocode}
%    \end{macro}
%    \begin{macro}{\HoLogoBkm@plainTeX@runtogether}
%    \begin{macrocode}
\def\HoLogoBkm@plainTeX@runtogether#1{%
  #1{p}{P}lain\hologo{TeX}%
}
%    \end{macrocode}
%    \end{macro}
%    \begin{macro}{\HoLogoHtml@plainTeX@runtogether}
%    \begin{macrocode}
\def\HoLogoHtml@plainTeX@runtogether#1{%
  #1{p}{P}lain\hologo{TeX}%
}
%    \end{macrocode}
%    \end{macro}
%
%    \begin{macro}{\HoLogo@plainTeX}
%    \begin{macrocode}
\def\HoLogo@plainTeX{\HoLogo@plainTeX@space}
%    \end{macrocode}
%    \end{macro}
%    \begin{macro}{\HoLogoCs@plainTeX}
%    \begin{macrocode}
\def\HoLogoCs@plainTeX{\HoLogoCs@plainTeX@space}
%    \end{macrocode}
%    \end{macro}
%    \begin{macro}{\HoLogoBkm@plainTeX}
%    \begin{macrocode}
\def\HoLogoBkm@plainTeX{\HoLogoBkm@plainTeX@space}
%    \end{macrocode}
%    \end{macro}
%    \begin{macro}{\HoLogoHtml@plainTeX}
%    \begin{macrocode}
\def\HoLogoHtml@plainTeX{\HoLogoHtml@plainTeX@space}
%    \end{macrocode}
%    \end{macro}
%
% \subsubsection{\hologo{LaTeX}}
%
%    Source: \hologo{LaTeX} kernel.
%\begin{quote}
%\begin{verbatim}
%\DeclareRobustCommand{\LaTeX}{%
%  L%
%  \kern-.36em%
%  {%
%    \sbox\z@ T%
%    \vbox to\ht\z@{%
%      \hbox{%
%        \check@mathfonts
%        \fontsize\sf@size\z@
%        \math@fontsfalse
%        \selectfont
%        A%
%      }%
%      \vss
%    }%
%  }%
%  \kern-.15em%
%  \TeX
%}
%\end{verbatim}
%\end{quote}
%
%    \begin{macro}{\HoLogo@La}
%    \begin{macrocode}
\def\HoLogo@La#1{%
  L%
  \kern-.36em%
  \begingroup
    \setbox\ltx@zero\hbox{T}%
    \vbox to\ht\ltx@zero{%
      \hbox{%
        \ltx@ifundefined{check@mathfonts}{%
          \csname sevenrm\endcsname
        }{%
          \check@mathfonts
          \fontsize\sf@size{0pt}%
          \math@fontsfalse\selectfont
        }%
        A%
      }%
      \vss
    }%
  \endgroup
}
%    \end{macrocode}
%    \end{macro}
%
%    \begin{macro}{\HoLogo@LaTeX}
%    Source: \hologo{LaTeX} kernel.
%    \begin{macrocode}
\def\HoLogo@LaTeX#1{%
  \hologo{La}%
  \kern-.15em%
  \hologo{TeX}%
}
%    \end{macrocode}
%    \end{macro}
%    \begin{macro}{\HoLogoHtml@LaTeX}
%    \begin{macrocode}
\def\HoLogoHtml@LaTeX#1{%
  \HoLogoCss@LaTeX
  \HOLOGO@Span{LaTeX}{%
    L%
    \HOLOGO@Span{a}{%
      A%
    }%
    \hologo{TeX}%
  }%
}
%    \end{macrocode}
%    \end{macro}
%    \begin{macro}{\HoLogoCss@LaTeX}
%    \begin{macrocode}
\def\HoLogoCss@LaTeX{%
  \Css{%
    span.HoLogo-LaTeX span.HoLogo-a{%
      position:relative;%
      top:-.5ex;%
      margin-left:-.36em;%
      margin-right:-.15em;%
      font-size:85\%;%
    }%
  }%
  \global\let\HoLogoCss@LaTeX\relax
}
%    \end{macrocode}
%    \end{macro}
%
% \subsubsection{\hologo{(La)TeX}}
%
%    \begin{macro}{\HoLogo@LaTeXTeX}
%    The kerning around the parentheses is taken
%    from package \xpackage{dtklogos} \cite{dtklogos}.
%\begin{quote}
%\begin{verbatim}
%\DeclareRobustCommand{\LaTeXTeX}{%
%  (%
%  \kern-.15em%
%  L%
%  \kern-.36em%
%  {%
%    \sbox\z@ T%
%    \vbox to\ht0{%
%      \hbox{%
%        $\m@th$%
%        \csname S@\f@size\endcsname
%        \fontsize\sf@size\z@
%        \math@fontsfalse
%        \selectfont
%        A%
%      }%
%      \vss
%    }%
%  }%
%  \kern-.2em%
%  )%
%  \kern-.15em%
%  \TeX
%}
%\end{verbatim}
%\end{quote}
%    \begin{macrocode}
\def\HoLogo@LaTeXTeX#1{%
  (%
  \kern-.15em%
  \hologo{La}%
  \kern-.2em%
  )%
  \kern-.15em%
  \hologo{TeX}%
}
%    \end{macrocode}
%    \end{macro}
%    \begin{macro}{\HoLogoBkm@LaTeXTeX}
%    \begin{macrocode}
\def\HoLogoBkm@LaTeXTeX#1{(La)TeX}
%    \end{macrocode}
%    \end{macro}
%
%    \begin{macro}{\HoLogo@(La)TeX}
%    \begin{macrocode}
\expandafter
\let\csname HoLogo@(La)TeX\endcsname\HoLogo@LaTeXTeX
%    \end{macrocode}
%    \end{macro}
%    \begin{macro}{\HoLogoBkm@(La)TeX}
%    \begin{macrocode}
\expandafter
\let\csname HoLogoBkm@(La)TeX\endcsname\HoLogoBkm@LaTeXTeX
%    \end{macrocode}
%    \end{macro}
%    \begin{macro}{\HoLogoHtml@LaTeXTeX}
%    \begin{macrocode}
\def\HoLogoHtml@LaTeXTeX#1{%
  \HoLogoCss@LaTeXTeX
  \HOLOGO@Span{LaTeXTeX}{%
    (%
    \HOLOGO@Span{L}{L}%
    \HOLOGO@Span{a}{A}%
    \HOLOGO@Span{ParenRight}{)}%
    \hologo{TeX}%
  }%
}
%    \end{macrocode}
%    \end{macro}
%    \begin{macro}{\HoLogoHtml@(La)TeX}
%    Kerning after opening parentheses and before closing parentheses
%    is $-0.1$\,em. The original values $-0.15$\,em
%    looked too ugly for a serif font.
%    \begin{macrocode}
\expandafter
\let\csname HoLogoHtml@(La)TeX\endcsname\HoLogoHtml@LaTeXTeX
%    \end{macrocode}
%    \end{macro}
%    \begin{macro}{\HoLogoCss@LaTeXTeX}
%    \begin{macrocode}
\def\HoLogoCss@LaTeXTeX{%
  \Css{%
    span.HoLogo-LaTeXTeX span.HoLogo-L{%
      margin-left:-.1em;%
    }%
  }%
  \Css{%
    span.HoLogo-LaTeXTeX span.HoLogo-a{%
      position:relative;%
      top:-.5ex;%
      margin-left:-.36em;%
      margin-right:-.1em;%
      font-size:85\%;%
    }%
  }%
  \Css{%
    span.HoLogo-LaTeXTeX span.HoLogo-ParenRight{%
      margin-right:-.15em;%
    }%
  }%
  \global\let\HoLogoCss@LaTeXTeX\relax
}
%    \end{macrocode}
%    \end{macro}
%
% \subsubsection{\hologo{LaTeXe}}
%
%    \begin{macro}{\HoLogo@LaTeXe}
%    Source: \hologo{LaTeX} kernel
%    \begin{macrocode}
\def\HoLogo@LaTeXe#1{%
  \hologo{LaTeX}%
  \kern.15em%
  \hbox{%
    \HOLOGO@MathSetup
    2%
    $_{\textstyle\varepsilon}$%
  }%
}
%    \end{macrocode}
%    \end{macro}
%
%    \begin{macro}{\HoLogoCs@LaTeXe}
%    \begin{macrocode}
\ifnum64=`\^^^^0040\relax % test for big chars of LuaTeX/XeTeX
  \catcode`\$=9 %
  \catcode`\&=14 %
\else
  \catcode`\$=14 %
  \catcode`\&=9 %
\fi
\def\HoLogoCs@LaTeXe#1{%
  LaTeX2%
$ \string ^^^^0395%
& e%
}%
\catcode`\$=3 %
\catcode`\&=4 %
%    \end{macrocode}
%    \end{macro}
%
%    \begin{macro}{\HoLogoBkm@LaTeXe}
%    \begin{macrocode}
\def\HoLogoBkm@LaTeXe#1{%
  \hologo{LaTeX}%
  2%
  \HOLOGO@PdfdocUnicode{e}{\textepsilon}%
}
%    \end{macrocode}
%    \end{macro}
%
%    \begin{macro}{\HoLogoHtml@LaTeXe}
%    \begin{macrocode}
\def\HoLogoHtml@LaTeXe#1{%
  \HoLogoCss@LaTeXe
  \HOLOGO@Span{LaTeX2e}{%
    \hologo{LaTeX}%
    \HOLOGO@Span{2}{2}%
    \HOLOGO@Span{e}{%
      \HOLOGO@MathSetup
      \ensuremath{\textstyle\varepsilon}%
    }%
  }%
}
%    \end{macrocode}
%    \end{macro}
%    \begin{macro}{\HoLogoCss@LaTeXe}
%    \begin{macrocode}
\def\HoLogoCss@LaTeXe{%
  \Css{%
    span.HoLogo-LaTeX2e span.HoLogo-2{%
      padding-left:.15em;%
    }%
  }%
  \Css{%
    span.HoLogo-LaTeX2e span.HoLogo-e{%
      position:relative;%
      top:.35ex;%
      text-decoration:none;%
    }%
  }%
  \global\let\HoLogoCss@LaTeXe\relax
}
%    \end{macrocode}
%    \end{macro}
%
%    \begin{macro}{\HoLogo@LaTeX2e}
%    \begin{macrocode}
\expandafter
\let\csname HoLogo@LaTeX2e\endcsname\HoLogo@LaTeXe
%    \end{macrocode}
%    \end{macro}
%    \begin{macro}{\HoLogoCs@LaTeX2e}
%    \begin{macrocode}
\expandafter
\let\csname HoLogoCs@LaTeX2e\endcsname\HoLogoCs@LaTeXe
%    \end{macrocode}
%    \end{macro}
%    \begin{macro}{\HoLogoBkm@LaTeX2e}
%    \begin{macrocode}
\expandafter
\let\csname HoLogoBkm@LaTeX2e\endcsname\HoLogoBkm@LaTeXe
%    \end{macrocode}
%    \end{macro}
%    \begin{macro}{\HoLogoHtml@LaTeX2e}
%    \begin{macrocode}
\expandafter
\let\csname HoLogoHtml@LaTeX2e\endcsname\HoLogoHtml@LaTeXe
%    \end{macrocode}
%    \end{macro}
%
% \subsubsection{\hologo{LaTeX3}}
%
%    \begin{macro}{\HoLogo@LaTeX3}
%    Source: \hologo{LaTeX} kernel
%    \begin{macrocode}
\expandafter\def\csname HoLogo@LaTeX3\endcsname#1{%
  \hologo{LaTeX}%
  3%
}
%    \end{macrocode}
%    \end{macro}
%
%    \begin{macro}{\HoLogoBkm@LaTeX3}
%    \begin{macrocode}
\expandafter\def\csname HoLogoBkm@LaTeX3\endcsname#1{%
  \hologo{LaTeX}%
  3%
}
%    \end{macrocode}
%    \end{macro}
%    \begin{macro}{\HoLogoHtml@LaTeX3}
%    \begin{macrocode}
\expandafter
\let\csname HoLogoHtml@LaTeX3\expandafter\endcsname
\csname HoLogo@LaTeX3\endcsname
%    \end{macrocode}
%    \end{macro}
%
% \subsubsection{\hologo{LaTeXML}}
%
%    \begin{macro}{\HoLogo@LaTeXML}
%    \begin{macrocode}
\def\HoLogo@LaTeXML#1{%
  \HOLOGO@mbox{%
    \hologo{La}%
    \kern-.15em%
    T%
    \kern-.1667em%
    \lower.5ex\hbox{E}%
    \kern-.125em%
    \HoLogoFont@font{LaTeXML}{sc}{xml}%
  }%
}
%    \end{macrocode}
%    \end{macro}
%    \begin{macro}{\HoLogoHtml@pdfLaTeX}
%    \begin{macrocode}
\def\HoLogoHtml@LaTeXML#1{%
  \HOLOGO@Span{LaTeXML}{%
    \HoLogoCss@LaTeX
    \HoLogoCss@TeX
    \HOLOGO@Span{LaTeX}{%
      L%
      \HOLOGO@Span{a}{%
        A%
      }%
    }%
    \HOLOGO@Span{TeX}{%
      T%
      \HOLOGO@Span{e}{%
        E%
      }%
    }%
    \HCode{<span style="font-variant: small-caps;">}%
    xml%
    \HCode{</span>}%
  }%
}
%    \end{macrocode}
%    \end{macro}
%
% \subsubsection{\hologo{eTeX}}
%
%    \begin{macro}{\HoLogo@eTeX}
%    Source: package \xpackage{etex}
%    \begin{macrocode}
\def\HoLogo@eTeX#1{%
  \ltx@mbox{%
    \HOLOGO@MathSetup
    $\varepsilon$%
    -%
    \HOLOGO@NegativeKerning{-T,T-,To}%
    \hologo{TeX}%
  }%
}
%    \end{macrocode}
%    \end{macro}
%    \begin{macro}{\HoLogoCs@eTeX}
%    \begin{macrocode}
\ifnum64=`\^^^^0040\relax % test for big chars of LuaTeX/XeTeX
  \catcode`\$=9 %
  \catcode`\&=14 %
\else
  \catcode`\$=14 %
  \catcode`\&=9 %
\fi
\def\HoLogoCs@eTeX#1{%
$ #1{\string ^^^^0395}{\string ^^^^03b5}%
& #1{e}{E}%
  TeX%
}%
\catcode`\$=3 %
\catcode`\&=4 %
%    \end{macrocode}
%    \end{macro}
%    \begin{macro}{\HoLogoBkm@eTeX}
%    \begin{macrocode}
\def\HoLogoBkm@eTeX#1{%
  \HOLOGO@PdfdocUnicode{#1{e}{E}}{\textepsilon}%
  -%
  \hologo{TeX}%
}
%    \end{macrocode}
%    \end{macro}
%    \begin{macro}{\HoLogoHtml@eTeX}
%    \begin{macrocode}
\def\HoLogoHtml@eTeX#1{%
  \ltx@mbox{%
    \HOLOGO@MathSetup
    $\varepsilon$%
    -%
    \hologo{TeX}%
  }%
}
%    \end{macrocode}
%    \end{macro}
%
% \subsubsection{\hologo{iniTeX}}
%
%    \begin{macro}{\HoLogo@iniTeX}
%    \begin{macrocode}
\def\HoLogo@iniTeX#1{%
  \HOLOGO@mbox{%
    #1{i}{I}ni\hologo{TeX}%
  }%
}
%    \end{macrocode}
%    \end{macro}
%    \begin{macro}{\HoLogoCs@iniTeX}
%    \begin{macrocode}
\def\HoLogoCs@iniTeX#1{#1{i}{I}niTeX}
%    \end{macrocode}
%    \end{macro}
%    \begin{macro}{\HoLogoBkm@iniTeX}
%    \begin{macrocode}
\def\HoLogoBkm@iniTeX#1{%
  #1{i}{I}ni\hologo{TeX}%
}
%    \end{macrocode}
%    \end{macro}
%    \begin{macro}{\HoLogoHtml@iniTeX}
%    \begin{macrocode}
\let\HoLogoHtml@iniTeX\HoLogo@iniTeX
%    \end{macrocode}
%    \end{macro}
%
% \subsubsection{\hologo{virTeX}}
%
%    \begin{macro}{\HoLogo@virTeX}
%    \begin{macrocode}
\def\HoLogo@virTeX#1{%
  \HOLOGO@mbox{%
    #1{v}{V}ir\hologo{TeX}%
  }%
}
%    \end{macrocode}
%    \end{macro}
%    \begin{macro}{\HoLogoCs@virTeX}
%    \begin{macrocode}
\def\HoLogoCs@virTeX#1{#1{v}{V}irTeX}
%    \end{macrocode}
%    \end{macro}
%    \begin{macro}{\HoLogoBkm@virTeX}
%    \begin{macrocode}
\def\HoLogoBkm@virTeX#1{%
  #1{v}{V}ir\hologo{TeX}%
}
%    \end{macrocode}
%    \end{macro}
%    \begin{macro}{\HoLogoHtml@virTeX}
%    \begin{macrocode}
\let\HoLogoHtml@virTeX\HoLogo@virTeX
%    \end{macrocode}
%    \end{macro}
%
% \subsubsection{\hologo{SliTeX}}
%
% \paragraph{Definitions of the three variants.}
%
%    \begin{macro}{\HoLogo@SLiTeX@lift}
%    \begin{macrocode}
\def\HoLogo@SLiTeX@lift#1{%
  \HoLogoFont@font{SliTeX}{rm}{%
    S%
    \kern-.06em%
    L%
    \kern-.18em%
    \raise.32ex\hbox{\HoLogoFont@font{SliTeX}{sc}{i}}%
    \HOLOGO@discretionary
    \kern-.06em%
    \hologo{TeX}%
  }%
}
%    \end{macrocode}
%    \end{macro}
%    \begin{macro}{\HoLogoBkm@SLiTeX@lift}
%    \begin{macrocode}
\def\HoLogoBkm@SLiTeX@lift#1{SLiTeX}
%    \end{macrocode}
%    \end{macro}
%    \begin{macro}{\HoLogoHtml@SLiTeX@lift}
%    \begin{macrocode}
\def\HoLogoHtml@SLiTeX@lift#1{%
  \HoLogoCss@SLiTeX@lift
  \HOLOGO@Span{SLiTeX-lift}{%
    \HoLogoFont@font{SliTeX}{rm}{%
      S%
      \HOLOGO@Span{L}{L}%
      \HOLOGO@Span{i}{i}%
      \hologo{TeX}%
    }%
  }%
}
%    \end{macrocode}
%    \end{macro}
%    \begin{macro}{\HoLogoCss@SLiTeX@lift}
%    \begin{macrocode}
\def\HoLogoCss@SLiTeX@lift{%
  \Css{%
    span.HoLogo-SLiTeX-lift span.HoLogo-L{%
      margin-left:-.06em;%
      margin-right:-.18em;%
    }%
  }%
  \Css{%
    span.HoLogo-SLiTeX-lift span.HoLogo-i{%
      position:relative;%
      top:-.32ex;%
      margin-right:-.06em;%
      font-variant:small-caps;%
    }%
  }%
  \global\let\HoLogoCss@SLiTeX@lift\relax
}
%    \end{macrocode}
%    \end{macro}
%
%    \begin{macro}{\HoLogo@SliTeX@simple}
%    \begin{macrocode}
\def\HoLogo@SliTeX@simple#1{%
  \HoLogoFont@font{SliTeX}{rm}{%
    \ltx@mbox{%
      \HoLogoFont@font{SliTeX}{sc}{Sli}%
    }%
    \HOLOGO@discretionary
    \hologo{TeX}%
  }%
}
%    \end{macrocode}
%    \end{macro}
%    \begin{macro}{\HoLogoBkm@SliTeX@simple}
%    \begin{macrocode}
\def\HoLogoBkm@SliTeX@simple#1{SliTeX}
%    \end{macrocode}
%    \end{macro}
%    \begin{macro}{\HoLogoHtml@SliTeX@simple}
%    \begin{macrocode}
\let\HoLogoHtml@SliTeX@simple\HoLogo@SliTeX@simple
%    \end{macrocode}
%    \end{macro}
%
%    \begin{macro}{\HoLogo@SliTeX@narrow}
%    \begin{macrocode}
\def\HoLogo@SliTeX@narrow#1{%
  \HoLogoFont@font{SliTeX}{rm}{%
    \ltx@mbox{%
      S%
      \kern-.06em%
      \HoLogoFont@font{SliTeX}{sc}{%
        l%
        \kern-.035em%
        i%
      }%
    }%
    \HOLOGO@discretionary
    \kern-.06em%
    \hologo{TeX}%
  }%
}
%    \end{macrocode}
%    \end{macro}
%    \begin{macro}{\HoLogoBkm@SliTeX@narrow}
%    \begin{macrocode}
\def\HoLogoBkm@SliTeX@narrow#1{SliTeX}
%    \end{macrocode}
%    \end{macro}
%    \begin{macro}{\HoLogoHtml@SliTeX@narrow}
%    \begin{macrocode}
\def\HoLogoHtml@SliTeX@narrow#1{%
  \HoLogoCss@SliTeX@narrow
  \HOLOGO@Span{SliTeX-narrow}{%
    \HoLogoFont@font{SliTeX}{rm}{%
      S%
        \HOLOGO@Span{l}{l}%
        \HOLOGO@Span{i}{i}%
      \hologo{TeX}%
    }%
  }%
}
%    \end{macrocode}
%    \end{macro}
%    \begin{macro}{\HoLogoCss@SliTeX@narrow}
%    \begin{macrocode}
\def\HoLogoCss@SliTeX@narrow{%
  \Css{%
    span.HoLogo-SliTeX-narrow span.HoLogo-l{%
      margin-left:-.06em;%
      margin-right:-.035em;%
      font-variant:small-caps;%
    }%
  }%
  \Css{%
    span.HoLogo-SliTeX-narrow span.HoLogo-i{%
      margin-right:-.06em;%
      font-variant:small-caps;%
    }%
  }%
  \global\let\HoLogoCss@SliTeX@narrow\relax
}
%    \end{macrocode}
%    \end{macro}
%
% \paragraph{Macro set completion.}
%
%    \begin{macro}{\HoLogo@SLiTeX@simple}
%    \begin{macrocode}
\def\HoLogo@SLiTeX@simple{\HoLogo@SliTeX@simple}
%    \end{macrocode}
%    \end{macro}
%    \begin{macro}{\HoLogoBkm@SLiTeX@simple}
%    \begin{macrocode}
\def\HoLogoBkm@SLiTeX@simple{\HoLogoBkm@SliTeX@simple}
%    \end{macrocode}
%    \end{macro}
%    \begin{macro}{\HoLogoHtml@SLiTeX@simple}
%    \begin{macrocode}
\def\HoLogoHtml@SLiTeX@simple{\HoLogoHtml@SliTeX@simple}
%    \end{macrocode}
%    \end{macro}
%
%    \begin{macro}{\HoLogo@SLiTeX@narrow}
%    \begin{macrocode}
\def\HoLogo@SLiTeX@narrow{\HoLogo@SliTeX@narrow}
%    \end{macrocode}
%    \end{macro}
%    \begin{macro}{\HoLogoBkm@SLiTeX@narrow}
%    \begin{macrocode}
\def\HoLogoBkm@SLiTeX@narrow{\HoLogoBkm@SliTeX@narrow}
%    \end{macrocode}
%    \end{macro}
%    \begin{macro}{\HoLogoHtml@SLiTeX@narrow}
%    \begin{macrocode}
\def\HoLogoHtml@SLiTeX@narrow{\HoLogoHtml@SliTeX@narrow}
%    \end{macrocode}
%    \end{macro}
%
%    \begin{macro}{\HoLogo@SliTeX@lift}
%    \begin{macrocode}
\def\HoLogo@SliTeX@lift{\HoLogo@SLiTeX@lift}
%    \end{macrocode}
%    \end{macro}
%    \begin{macro}{\HoLogoBkm@SliTeX@lift}
%    \begin{macrocode}
\def\HoLogoBkm@SliTeX@lift{\HoLogoBkm@SLiTeX@lift}
%    \end{macrocode}
%    \end{macro}
%    \begin{macro}{\HoLogoHtml@SliTeX@lift}
%    \begin{macrocode}
\def\HoLogoHtml@SliTeX@lift{\HoLogoHtml@SLiTeX@lift}
%    \end{macrocode}
%    \end{macro}
%
% \paragraph{Defaults.}
%
%    \begin{macro}{\HoLogo@SLiTeX}
%    \begin{macrocode}
\def\HoLogo@SLiTeX{\HoLogo@SLiTeX@lift}
%    \end{macrocode}
%    \end{macro}
%    \begin{macro}{\HoLogoBkm@SLiTeX}
%    \begin{macrocode}
\def\HoLogoBkm@SLiTeX{\HoLogoBkm@SLiTeX@lift}
%    \end{macrocode}
%    \end{macro}
%    \begin{macro}{\HoLogoHtml@SLiTeX}
%    \begin{macrocode}
\def\HoLogoHtml@SLiTeX{\HoLogoHtml@SLiTeX@lift}
%    \end{macrocode}
%    \end{macro}
%
%    \begin{macro}{\HoLogo@SliTeX}
%    \begin{macrocode}
\def\HoLogo@SliTeX{\HoLogo@SliTeX@narrow}
%    \end{macrocode}
%    \end{macro}
%    \begin{macro}{\HoLogoBkm@SliTeX}
%    \begin{macrocode}
\def\HoLogoBkm@SliTeX{\HoLogoBkm@SliTeX@narrow}
%    \end{macrocode}
%    \end{macro}
%    \begin{macro}{\HoLogoHtml@SliTeX}
%    \begin{macrocode}
\def\HoLogoHtml@SliTeX{\HoLogoHtml@SliTeX@narrow}
%    \end{macrocode}
%    \end{macro}
%
% \subsubsection{\hologo{LuaTeX}}
%
%    \begin{macro}{\HoLogo@LuaTeX}
%    The kerning is an idea of Hans Hagen, see mailing list
%    `luatex at tug dot org' in March 2010.
%    \begin{macrocode}
\def\HoLogo@LuaTeX#1{%
  \HOLOGO@mbox{%
    Lua%
    \HOLOGO@NegativeKerning{aT,oT,To}%
    \hologo{TeX}%
  }%
}
%    \end{macrocode}
%    \end{macro}
%    \begin{macro}{\HoLogoHtml@LuaTeX}
%    \begin{macrocode}
\let\HoLogoHtml@LuaTeX\HoLogo@LuaTeX
%    \end{macrocode}
%    \end{macro}
%
% \subsubsection{\hologo{LuaLaTeX}}
%
%    \begin{macro}{\HoLogo@LuaLaTeX}
%    \begin{macrocode}
\def\HoLogo@LuaLaTeX#1{%
  \HOLOGO@mbox{%
    Lua%
    \hologo{LaTeX}%
  }%
}
%    \end{macrocode}
%    \end{macro}
%    \begin{macro}{\HoLogoHtml@LuaLaTeX}
%    \begin{macrocode}
\let\HoLogoHtml@LuaLaTeX\HoLogo@LuaLaTeX
%    \end{macrocode}
%    \end{macro}
%
% \subsubsection{\hologo{XeTeX}, \hologo{XeLaTeX}}
%
%    \begin{macro}{\HOLOGO@IfCharExists}
%    \begin{macrocode}
\ifluatex
  \ifnum\luatexversion<36 %
  \else
    \def\HOLOGO@IfCharExists#1{%
      \ifnum
        \directlua{%
           if luaotfload and luaotfload.aux then
             if luaotfload.aux.font_has_glyph(%
                    font.current(), \number#1) then % 	 
	       tex.print("1") % 	 
	     end % 	 
	   elseif font and font.fonts and font.current then %
            local f = font.fonts[font.current()]%
            if f.characters and f.characters[\number#1] then %
              tex.print("1")%
            end %
          end%
        }0=\ltx@zero
        \expandafter\ltx@secondoftwo
      \else
        \expandafter\ltx@firstoftwo
      \fi
    }%
  \fi
\fi
\ltx@IfUndefined{HOLOGO@IfCharExists}{%
  \def\HOLOGO@@IfCharExists#1{%
    \begingroup
      \tracinglostchars=\ltx@zero
      \setbox\ltx@zero=\hbox{%
        \kern7sp\char#1\relax
        \ifnum\lastkern>\ltx@zero
          \expandafter\aftergroup\csname iffalse\endcsname
        \else
          \expandafter\aftergroup\csname iftrue\endcsname
        \fi
      }%
      % \if{true|false} from \aftergroup
      \endgroup
      \expandafter\ltx@firstoftwo
    \else
      \endgroup
      \expandafter\ltx@secondoftwo
    \fi
  }%
  \ifxetex
    \ltx@IfUndefined{XeTeXfonttype}{}{%
      \ltx@IfUndefined{XeTeXcharglyph}{}{%
        \def\HOLOGO@IfCharExists#1{%
          \ifnum\XeTeXfonttype\font>\ltx@zero
            \expandafter\ltx@firstofthree
          \else
            \expandafter\ltx@gobble
          \fi
          {%
            \ifnum\XeTeXcharglyph#1>\ltx@zero
              \expandafter\ltx@firstoftwo
            \else
              \expandafter\ltx@secondoftwo
            \fi
          }%
          \HOLOGO@@IfCharExists{#1}%
        }%
      }%
    }%
  \fi
}{}
\ltx@ifundefined{HOLOGO@IfCharExists}{%
  \ifnum64=`\^^^^0040\relax % test for big chars of LuaTeX/XeTeX
    \let\HOLOGO@IfCharExists\HOLOGO@@IfCharExists
  \else
    \def\HOLOGO@IfCharExists#1{%
      \ifnum#1>255 %
        \expandafter\ltx@fourthoffour
      \fi
      \HOLOGO@@IfCharExists{#1}%
    }%
  \fi
}{}
%    \end{macrocode}
%    \end{macro}
%
%    \begin{macro}{\HoLogo@Xe}
%    Source: package \xpackage{dtklogos}
%    \begin{macrocode}
\def\HoLogo@Xe#1{%
  X%
  \kern-.1em\relax
  \HOLOGO@IfCharExists{"018E}{%
    \lower.5ex\hbox{\char"018E}%
  }{%
    \chardef\HOLOGO@choice=\ltx@zero
    \ifdim\fontdimen\ltx@one\font>0pt %
      \ltx@IfUndefined{rotatebox}{%
        \ltx@IfUndefined{pgftext}{%
          \ltx@IfUndefined{psscalebox}{%
            \ltx@IfUndefined{HOLOGO@ScaleBox@\hologoDriver}{%
            }{%
              \chardef\HOLOGO@choice=4 %
            }%
          }{%
            \chardef\HOLOGO@choice=3 %
          }%
        }{%
          \chardef\HOLOGO@choice=2 %
        }%
      }{%
        \chardef\HOLOGO@choice=1 %
      }%
      \ifcase\HOLOGO@choice
        \HOLOGO@WarningUnsupportedDriver{Xe}%
        e%
      \or % 1: \rotatebox
        \begingroup
          \setbox\ltx@zero\hbox{\rotatebox{180}{E}}%
          \ltx@LocDimenA=\dp\ltx@zero
          \advance\ltx@LocDimenA by -.5ex\relax
          \raise\ltx@LocDimenA\box\ltx@zero
        \endgroup
      \or % 2: \pgftext
        \lower.5ex\hbox{%
          \pgfpicture
            \pgftext[rotate=180]{E}%
          \endpgfpicture
        }%
      \or % 3: \psscalebox
        \begingroup
          \setbox\ltx@zero\hbox{\psscalebox{-1 -1}{E}}%
          \ltx@LocDimenA=\dp\ltx@zero
          \advance\ltx@LocDimenA by -.5ex\relax
          \raise\ltx@LocDimenA\box\ltx@zero
        \endgroup
      \or % 4: \HOLOGO@PointReflectBox
        \lower.5ex\hbox{\HOLOGO@PointReflectBox{E}}%
      \else
        \@PackageError{hologo}{Internal error (choice/it}\@ehc
      \fi
    \else
      \ltx@IfUndefined{reflectbox}{%
        \ltx@IfUndefined{pgftext}{%
          \ltx@IfUndefined{psscalebox}{%
            \ltx@IfUndefined{HOLOGO@ScaleBox@\hologoDriver}{%
            }{%
              \chardef\HOLOGO@choice=4 %
            }%
          }{%
            \chardef\HOLOGO@choice=3 %
          }%
        }{%
          \chardef\HOLOGO@choice=2 %
        }%
      }{%
        \chardef\HOLOGO@choice=1 %
      }%
      \ifcase\HOLOGO@choice
        \HOLOGO@WarningUnsupportedDriver{Xe}%
        e%
      \or % 1: reflectbox
        \lower.5ex\hbox{%
          \reflectbox{E}%
        }%
      \or % 2: \pgftext
        \lower.5ex\hbox{%
          \pgfpicture
            \pgftransformxscale{-1}%
            \pgftext{E}%
          \endpgfpicture
        }%
      \or % 3: \psscalebox
        \lower.5ex\hbox{%
          \psscalebox{-1 1}{E}%
        }%
      \or % 4: \HOLOGO@Reflectbox
        \lower.5ex\hbox{%
          \HOLOGO@ReflectBox{E}%
        }%
      \else
        \@PackageError{hologo}{Internal error (choice/up)}\@ehc
      \fi
    \fi
  }%
}
%    \end{macrocode}
%    \end{macro}
%    \begin{macro}{\HoLogoHtml@Xe}
%    \begin{macrocode}
\def\HoLogoHtml@Xe#1{%
  \HoLogoCss@Xe
  \HOLOGO@Span{Xe}{%
    X%
    \HOLOGO@Span{e}{%
      \HCode{&\ltx@hashchar x018e;}%
    }%
  }%
}
%    \end{macrocode}
%    \end{macro}
%    \begin{macro}{\HoLogoCss@Xe}
%    \begin{macrocode}
\def\HoLogoCss@Xe{%
  \Css{%
    span.HoLogo-Xe span.HoLogo-e{%
      position:relative;%
      top:.5ex;%
      left-margin:-.1em;%
    }%
  }%
  \global\let\HoLogoCss@Xe\relax
}
%    \end{macrocode}
%    \end{macro}
%
%    \begin{macro}{\HoLogo@XeTeX}
%    \begin{macrocode}
\def\HoLogo@XeTeX#1{%
  \hologo{Xe}%
  \kern-.15em\relax
  \hologo{TeX}%
}
%    \end{macrocode}
%    \end{macro}
%
%    \begin{macro}{\HoLogoHtml@XeTeX}
%    \begin{macrocode}
\def\HoLogoHtml@XeTeX#1{%
  \HoLogoCss@XeTeX
  \HOLOGO@Span{XeTeX}{%
    \hologo{Xe}%
    \hologo{TeX}%
  }%
}
%    \end{macrocode}
%    \end{macro}
%    \begin{macro}{\HoLogoCss@XeTeX}
%    \begin{macrocode}
\def\HoLogoCss@XeTeX{%
  \Css{%
    span.HoLogo-XeTeX span.HoLogo-TeX{%
      margin-left:-.15em;%
    }%
  }%
  \global\let\HoLogoCss@XeTeX\relax
}
%    \end{macrocode}
%    \end{macro}
%
%    \begin{macro}{\HoLogo@XeLaTeX}
%    \begin{macrocode}
\def\HoLogo@XeLaTeX#1{%
  \hologo{Xe}%
  \kern-.13em%
  \hologo{LaTeX}%
}
%    \end{macrocode}
%    \end{macro}
%    \begin{macro}{\HoLogoHtml@XeLaTeX}
%    \begin{macrocode}
\def\HoLogoHtml@XeLaTeX#1{%
  \HoLogoCss@XeLaTeX
  \HOLOGO@Span{XeLaTeX}{%
    \hologo{Xe}%
    \hologo{LaTeX}%
  }%
}
%    \end{macrocode}
%    \end{macro}
%    \begin{macro}{\HoLogoCss@XeLaTeX}
%    \begin{macrocode}
\def\HoLogoCss@XeLaTeX{%
  \Css{%
    span.HoLogo-XeLaTeX span.HoLogo-Xe{%
      margin-right:-.13em;%
    }%
  }%
  \global\let\HoLogoCss@XeLaTeX\relax
}
%    \end{macrocode}
%    \end{macro}
%
% \subsubsection{\hologo{pdfTeX}, \hologo{pdfLaTeX}}
%
%    \begin{macro}{\HoLogo@pdfTeX}
%    \begin{macrocode}
\def\HoLogo@pdfTeX#1{%
  \HOLOGO@mbox{%
    #1{p}{P}df\hologo{TeX}%
  }%
}
%    \end{macrocode}
%    \end{macro}
%    \begin{macro}{\HoLogoCs@pdfTeX}
%    \begin{macrocode}
\def\HoLogoCs@pdfTeX#1{#1{p}{P}dfTeX}
%    \end{macrocode}
%    \end{macro}
%    \begin{macro}{\HoLogoBkm@pdfTeX}
%    \begin{macrocode}
\def\HoLogoBkm@pdfTeX#1{%
  #1{p}{P}df\hologo{TeX}%
}
%    \end{macrocode}
%    \end{macro}
%    \begin{macro}{\HoLogoHtml@pdfTeX}
%    \begin{macrocode}
\let\HoLogoHtml@pdfTeX\HoLogo@pdfTeX
%    \end{macrocode}
%    \end{macro}
%
%    \begin{macro}{\HoLogo@pdfLaTeX}
%    \begin{macrocode}
\def\HoLogo@pdfLaTeX#1{%
  \HOLOGO@mbox{%
    #1{p}{P}df\hologo{LaTeX}%
  }%
}
%    \end{macrocode}
%    \end{macro}
%    \begin{macro}{\HoLogoCs@pdfLaTeX}
%    \begin{macrocode}
\def\HoLogoCs@pdfLaTeX#1{#1{p}{P}dfLaTeX}
%    \end{macrocode}
%    \end{macro}
%    \begin{macro}{\HoLogoBkm@pdfLaTeX}
%    \begin{macrocode}
\def\HoLogoBkm@pdfLaTeX#1{%
  #1{p}{P}df\hologo{LaTeX}%
}
%    \end{macrocode}
%    \end{macro}
%    \begin{macro}{\HoLogoHtml@pdfLaTeX}
%    \begin{macrocode}
\let\HoLogoHtml@pdfLaTeX\HoLogo@pdfLaTeX
%    \end{macrocode}
%    \end{macro}
%
% \subsubsection{\hologo{VTeX}}
%
%    \begin{macro}{\HoLogo@VTeX}
%    \begin{macrocode}
\def\HoLogo@VTeX#1{%
  \HOLOGO@mbox{%
    V\hologo{TeX}%
  }%
}
%    \end{macrocode}
%    \end{macro}
%    \begin{macro}{\HoLogoHtml@VTeX}
%    \begin{macrocode}
\let\HoLogoHtml@VTeX\HoLogo@VTeX
%    \end{macrocode}
%    \end{macro}
%
% \subsubsection{\hologo{AmS}, \dots}
%
%    Source: class \xclass{amsdtx}
%
%    \begin{macro}{\HoLogo@AmS}
%    \begin{macrocode}
\def\HoLogo@AmS#1{%
  \HoLogoFont@font{AmS}{sy}{%
    A%
    \kern-.1667em%
    \lower.5ex\hbox{M}%
    \kern-.125em%
    S%
  }%
}
%    \end{macrocode}
%    \end{macro}
%    \begin{macro}{\HoLogoBkm@AmS}
%    \begin{macrocode}
\def\HoLogoBkm@AmS#1{AmS}
%    \end{macrocode}
%    \end{macro}
%    \begin{macro}{\HoLogoHtml@AmS}
%    \begin{macrocode}
\def\HoLogoHtml@AmS#1{%
  \HoLogoCss@AmS
%  \HoLogoFont@font{AmS}{sy}{%
    \HOLOGO@Span{AmS}{%
      A%
      \HOLOGO@Span{M}{M}%
      S%
    }%
%   }%
}
%    \end{macrocode}
%    \end{macro}
%    \begin{macro}{\HoLogoCss@AmS}
%    \begin{macrocode}
\def\HoLogoCss@AmS{%
  \Css{%
    span.HoLogo-AmS span.HoLogo-M{%
      position:relative;%
      top:.5ex;%
      margin-left:-.1667em;%
      margin-right:-.125em;%
      text-decoration:none;%
    }%
  }%
  \global\let\HoLogoCss@AmS\relax
}
%    \end{macrocode}
%    \end{macro}
%
%    \begin{macro}{\HoLogo@AmSTeX}
%    \begin{macrocode}
\def\HoLogo@AmSTeX#1{%
  \hologo{AmS}%
  \HOLOGO@hyphen
  \hologo{TeX}%
}
%    \end{macrocode}
%    \end{macro}
%    \begin{macro}{\HoLogoBkm@AmSTeX}
%    \begin{macrocode}
\def\HoLogoBkm@AmSTeX#1{AmS-TeX}%
%    \end{macrocode}
%    \end{macro}
%    \begin{macro}{\HoLogoHtml@AmSTeX}
%    \begin{macrocode}
\let\HoLogoHtml@AmSTeX\HoLogo@AmSTeX
%    \end{macrocode}
%    \end{macro}
%
%    \begin{macro}{\HoLogo@AmSLaTeX}
%    \begin{macrocode}
\def\HoLogo@AmSLaTeX#1{%
  \hologo{AmS}%
  \HOLOGO@hyphen
  \hologo{LaTeX}%
}
%    \end{macrocode}
%    \end{macro}
%    \begin{macro}{\HoLogoBkm@AmSLaTeX}
%    \begin{macrocode}
\def\HoLogoBkm@AmSLaTeX#1{AmS-LaTeX}%
%    \end{macrocode}
%    \end{macro}
%    \begin{macro}{\HoLogoHtml@AmSLaTeX}
%    \begin{macrocode}
\let\HoLogoHtml@AmSLaTeX\HoLogo@AmSLaTeX
%    \end{macrocode}
%    \end{macro}
%
% \subsubsection{\hologo{BibTeX}}
%
%    \begin{macro}{\HoLogo@BibTeX@sc}
%    A definition of \hologo{BibTeX} is provided in
%    the documentation source for the manual of \hologo{BibTeX}
%    \cite{btxdoc}.
%\begin{quote}
%\begin{verbatim}
%\def\BibTeX{%
%  {%
%    \rm
%    B%
%    \kern-.05em%
%    {%
%      \sc
%      i%
%      \kern-.025em %
%      b%
%    }%
%    \kern-.08em
%    T%
%    \kern-.1667em%
%    \lower.7ex\hbox{E}%
%    \kern-.125em%
%    X%
%  }%
%}
%\end{verbatim}
%\end{quote}
%    \begin{macrocode}
\def\HoLogo@BibTeX@sc#1{%
  B%
  \kern-.05em%
  \HoLogoFont@font{BibTeX}{sc}{%
    i%
    \kern-.025em%
    b%
  }%
  \HOLOGO@discretionary
  \kern-.08em%
  \hologo{TeX}%
}
%    \end{macrocode}
%    \end{macro}
%    \begin{macro}{\HoLogoHtml@BibTeX@sc}
%    \begin{macrocode}
\def\HoLogoHtml@BibTeX@sc#1{%
  \HoLogoCss@BibTeX@sc
  \HOLOGO@Span{BibTeX-sc}{%
    B%
    \HOLOGO@Span{i}{i}%
    \HOLOGO@Span{b}{b}%
    \hologo{TeX}%
  }%
}
%    \end{macrocode}
%    \end{macro}
%    \begin{macro}{\HoLogoCss@BibTeX@sc}
%    \begin{macrocode}
\def\HoLogoCss@BibTeX@sc{%
  \Css{%
    span.HoLogo-BibTeX-sc span.HoLogo-i{%
      margin-left:-.05em;%
      margin-right:-.025em;%
      font-variant:small-caps;%
    }%
  }%
  \Css{%
    span.HoLogo-BibTeX-sc span.HoLogo-b{%
      margin-right:-.08em;%
      font-variant:small-caps;%
    }%
  }%
  \global\let\HoLogoCss@BibTeX@sc\relax
}
%    \end{macrocode}
%    \end{macro}
%
%    \begin{macro}{\HoLogo@BibTeX@sf}
%    Variant \xoption{sf} avoids trouble with unavailable
%    small caps fonts (e.g., bold versions of Computer Modern or
%    Latin Modern). The definition is taken from
%    package \xpackage{dtklogos} \cite{dtklogos}.
%\begin{quote}
%\begin{verbatim}
%\DeclareRobustCommand{\BibTeX}{%
%  B%
%  \kern-.05em%
%  \hbox{%
%    $\m@th$% %% force math size calculations
%    \csname S@\f@size\endcsname
%    \fontsize\sf@size\z@
%    \math@fontsfalse
%    \selectfont
%    I%
%    \kern-.025em%
%    B
%  }%
%  \kern-.08em%
%  \-%
%  \TeX
%}
%\end{verbatim}
%\end{quote}
%    \begin{macrocode}
\def\HoLogo@BibTeX@sf#1{%
  B%
  \kern-.05em%
  \HoLogoFont@font{BibTeX}{bibsf}{%
    I%
    \kern-.025em%
    B%
  }%
  \HOLOGO@discretionary
  \kern-.08em%
  \hologo{TeX}%
}
%    \end{macrocode}
%    \end{macro}
%    \begin{macro}{\HoLogoHtml@BibTeX@sf}
%    \begin{macrocode}
\def\HoLogoHtml@BibTeX@sf#1{%
  \HoLogoCss@BibTeX@sf
  \HOLOGO@Span{BibTeX-sf}{%
    B%
    \HoLogoFont@font{BibTeX}{bibsf}{%
      \HOLOGO@Span{i}{I}%
      B%
    }%
    \hologo{TeX}%
  }%
}
%    \end{macrocode}
%    \end{macro}
%    \begin{macro}{\HoLogoCss@BibTeX@sf}
%    \begin{macrocode}
\def\HoLogoCss@BibTeX@sf{%
  \Css{%
    span.HoLogo-BibTeX-sf span.HoLogo-i{%
      margin-left:-.05em;%
      margin-right:-.025em;%
    }%
  }%
  \Css{%
    span.HoLogo-BibTeX-sf span.HoLogo-TeX{%
      margin-left:-.08em;%
    }%
  }%
  \global\let\HoLogoCss@BibTeX@sf\relax
}
%    \end{macrocode}
%    \end{macro}
%
%    \begin{macro}{\HoLogo@BibTeX}
%    \begin{macrocode}
\def\HoLogo@BibTeX{\HoLogo@BibTeX@sf}
%    \end{macrocode}
%    \end{macro}
%    \begin{macro}{\HoLogoHtml@BibTeX}
%    \begin{macrocode}
\def\HoLogoHtml@BibTeX{\HoLogoHtml@BibTeX@sf}
%    \end{macrocode}
%    \end{macro}
%
% \subsubsection{\hologo{BibTeX8}}
%
%    \begin{macro}{\HoLogo@BibTeX8}
%    \begin{macrocode}
\expandafter\def\csname HoLogo@BibTeX8\endcsname#1{%
  \hologo{BibTeX}%
  8%
}
%    \end{macrocode}
%    \end{macro}
%
%    \begin{macro}{\HoLogoBkm@BibTeX8}
%    \begin{macrocode}
\expandafter\def\csname HoLogoBkm@BibTeX8\endcsname#1{%
  \hologo{BibTeX}%
  8%
}
%    \end{macrocode}
%    \end{macro}
%    \begin{macro}{\HoLogoHtml@BibTeX8}
%    \begin{macrocode}
\expandafter
\let\csname HoLogoHtml@BibTeX8\expandafter\endcsname
\csname HoLogo@BibTeX8\endcsname
%    \end{macrocode}
%    \end{macro}
%
% \subsubsection{\hologo{ConTeXt}}
%
%    \begin{macro}{\HoLogo@ConTeXt@simple}
%    \begin{macrocode}
\def\HoLogo@ConTeXt@simple#1{%
  \HOLOGO@mbox{Con}%
  \HOLOGO@discretionary
  \HOLOGO@mbox{\hologo{TeX}t}%
}
%    \end{macrocode}
%    \end{macro}
%    \begin{macro}{\HoLogoHtml@ConTeXt@simple}
%    \begin{macrocode}
\let\HoLogoHtml@ConTeXt@simple\HoLogo@ConTeXt@simple
%    \end{macrocode}
%    \end{macro}
%
%    \begin{macro}{\HoLogo@ConTeXt@narrow}
%    This definition of logo \hologo{ConTeXt} with variant \xoption{narrow}
%    comes from TUGboat's class \xclass{ltugboat} (version 2010/11/15 v2.8).
%    \begin{macrocode}
\def\HoLogo@ConTeXt@narrow#1{%
  \HOLOGO@mbox{C\kern-.0333emon}%
  \HOLOGO@discretionary
  \kern-.0667em%
  \HOLOGO@mbox{\hologo{TeX}\kern-.0333emt}%
}
%    \end{macrocode}
%    \end{macro}
%    \begin{macro}{\HoLogoHtml@ConTeXt@narrow}
%    \begin{macrocode}
\def\HoLogoHtml@ConTeXt@narrow#1{%
  \HoLogoCss@ConTeXt@narrow
  \HOLOGO@Span{ConTeXt-narrow}{%
    \HOLOGO@Span{C}{C}%
    on%
    \hologo{TeX}%
    t%
  }%
}
%    \end{macrocode}
%    \end{macro}
%    \begin{macro}{\HoLogoCss@ConTeXt@narrow}
%    \begin{macrocode}
\def\HoLogoCss@ConTeXt@narrow{%
  \Css{%
    span.HoLogo-ConTeXt-narrow span.HoLogo-C{%
      margin-left:-.0333em;%
    }%
  }%
  \Css{%
    span.HoLogo-ConTeXt-narrow span.HoLogo-TeX{%
      margin-left:-.0667em;%
      margin-right:-.0333em;%
    }%
  }%
  \global\let\HoLogoCss@ConTeXt@narrow\relax
}
%    \end{macrocode}
%    \end{macro}
%
%    \begin{macro}{\HoLogo@ConTeXt}
%    \begin{macrocode}
\def\HoLogo@ConTeXt{\HoLogo@ConTeXt@narrow}
%    \end{macrocode}
%    \end{macro}
%    \begin{macro}{\HoLogoHtml@ConTeXt}
%    \begin{macrocode}
\def\HoLogoHtml@ConTeXt{\HoLogoHtml@ConTeXt@narrow}
%    \end{macrocode}
%    \end{macro}
%
% \subsubsection{\hologo{emTeX}}
%
%    \begin{macro}{\HoLogo@emTeX}
%    \begin{macrocode}
\def\HoLogo@emTeX#1{%
  \HOLOGO@mbox{#1{e}{E}m}%
  \HOLOGO@discretionary
  \hologo{TeX}%
}
%    \end{macrocode}
%    \end{macro}
%    \begin{macro}{\HoLogoCs@emTeX}
%    \begin{macrocode}
\def\HoLogoCs@emTeX#1{#1{e}{E}mTeX}%
%    \end{macrocode}
%    \end{macro}
%    \begin{macro}{\HoLogoBkm@emTeX}
%    \begin{macrocode}
\def\HoLogoBkm@emTeX#1{%
  #1{e}{E}m\hologo{TeX}%
}
%    \end{macrocode}
%    \end{macro}
%    \begin{macro}{\HoLogoHtml@emTeX}
%    \begin{macrocode}
\let\HoLogoHtml@emTeX\HoLogo@emTeX
%    \end{macrocode}
%    \end{macro}
%
% \subsubsection{\hologo{ExTeX}}
%
%    \begin{macro}{\HoLogo@ExTeX}
%    The definition is taken from the FAQ of the
%    project \hologo{ExTeX}
%    \cite{ExTeX-FAQ}.
%\begin{quote}
%\begin{verbatim}
%\def\ExTeX{%
%  \textrm{% Logo always with serifs
%    \ensuremath{%
%      \textstyle
%      \varepsilon_{%
%        \kern-0.15em%
%        \mathcal{X}%
%      }%
%    }%
%    \kern-.15em%
%    \TeX
%  }%
%}
%\end{verbatim}
%\end{quote}
%    \begin{macrocode}
\def\HoLogo@ExTeX#1{%
  \HoLogoFont@font{ExTeX}{rm}{%
    \ltx@mbox{%
      \HOLOGO@MathSetup
      $%
        \textstyle
        \varepsilon_{%
          \kern-0.15em%
          \HoLogoFont@font{ExTeX}{sy}{X}%
        }%
      $%
    }%
    \HOLOGO@discretionary
    \kern-.15em%
    \hologo{TeX}%
  }%
}
%    \end{macrocode}
%    \end{macro}
%    \begin{macro}{\HoLogoHtml@ExTeX}
%    \begin{macrocode}
\def\HoLogoHtml@ExTeX#1{%
  \HoLogoCss@ExTeX
  \HoLogoFont@font{ExTeX}{rm}{%
    \HOLOGO@Span{ExTeX}{%
      \ltx@mbox{%
        \HOLOGO@MathSetup
        $\textstyle\varepsilon$%
        \HOLOGO@Span{X}{$\textstyle\chi$}%
        \hologo{TeX}%
      }%
    }%
  }%
}
%    \end{macrocode}
%    \end{macro}
%    \begin{macro}{\HoLogoBkm@ExTeX}
%    \begin{macrocode}
\def\HoLogoBkm@ExTeX#1{%
  \HOLOGO@PdfdocUnicode{#1{e}{E}x}{\textepsilon\textchi}%
  \hologo{TeX}%
}
%    \end{macrocode}
%    \end{macro}
%    \begin{macro}{\HoLogoCss@ExTeX}
%    \begin{macrocode}
\def\HoLogoCss@ExTeX{%
  \Css{%
    span.HoLogo-ExTeX{%
      font-family:serif;%
    }%
  }%
  \Css{%
    span.HoLogo-ExTeX span.HoLogo-TeX{%
      margin-left:-.15em;%
    }%
  }%
  \global\let\HoLogoCss@ExTeX\relax
}
%    \end{macrocode}
%    \end{macro}
%
% \subsubsection{\hologo{MiKTeX}}
%
%    \begin{macro}{\HoLogo@MiKTeX}
%    \begin{macrocode}
\def\HoLogo@MiKTeX#1{%
  \HOLOGO@mbox{MiK}%
  \HOLOGO@discretionary
  \hologo{TeX}%
}
%    \end{macrocode}
%    \end{macro}
%    \begin{macro}{\HoLogoHtml@MiKTeX}
%    \begin{macrocode}
\let\HoLogoHtml@MiKTeX\HoLogo@MiKTeX
%    \end{macrocode}
%    \end{macro}
%
% \subsubsection{\hologo{OzTeX} and friends}
%
%    Source: \hologo{OzTeX} FAQ \cite{OzTeX}:
%    \begin{quote}
%      |\def\OzTeX{O\kern-.03em z\kern-.15em\TeX}|\\
%      (There is no kerning in OzMF, OzMP and OzTtH.)
%    \end{quote}
%
%    \begin{macro}{\HoLogo@OzTeX}
%    \begin{macrocode}
\def\HoLogo@OzTeX#1{%
  O%
  \kern-.03em %
  z%
  \kern-.15em %
  \hologo{TeX}%
}
%    \end{macrocode}
%    \end{macro}
%    \begin{macro}{\HoLogoHtml@OzTeX}
%    \begin{macrocode}
\def\HoLogoHtml@OzTeX#1{%
  \HoLogoCss@OzTeX
  \HOLOGO@Span{OzTeX}{%
    O%
    \HOLOGO@Span{z}{z}%
    \hologo{TeX}%
  }%
}
%    \end{macrocode}
%    \end{macro}
%    \begin{macro}{\HoLogoCss@OzTeX}
%    \begin{macrocode}
\def\HoLogoCss@OzTeX{%
  \Css{%
    span.HoLogo-OzTeX span.HoLogo-z{%
      margin-left:-.03em;%
      margin-right:-.15em;%
    }%
  }%
  \global\let\HoLogoCss@OzTeX\relax
}
%    \end{macrocode}
%    \end{macro}
%
%    \begin{macro}{\HoLogo@OzMF}
%    \begin{macrocode}
\def\HoLogo@OzMF#1{%
  \HOLOGO@mbox{OzMF}%
}
%    \end{macrocode}
%    \end{macro}
%    \begin{macro}{\HoLogo@OzMP}
%    \begin{macrocode}
\def\HoLogo@OzMP#1{%
  \HOLOGO@mbox{OzMP}%
}
%    \end{macrocode}
%    \end{macro}
%    \begin{macro}{\HoLogo@OzTtH}
%    \begin{macrocode}
\def\HoLogo@OzTtH#1{%
  \HOLOGO@mbox{OzTtH}%
}
%    \end{macrocode}
%    \end{macro}
%
% \subsubsection{\hologo{PCTeX}}
%
%    \begin{macro}{\HoLogo@PCTeX}
%    \begin{macrocode}
\def\HoLogo@PCTeX#1{%
  \HOLOGO@mbox{PC}%
  \hologo{TeX}%
}
%    \end{macrocode}
%    \end{macro}
%    \begin{macro}{\HoLogoHtml@PCTeX}
%    \begin{macrocode}
\let\HoLogoHtml@PCTeX\HoLogo@PCTeX
%    \end{macrocode}
%    \end{macro}
%
% \subsubsection{\hologo{PiCTeX}}
%
%    The original definitions from \xfile{pictex.tex} \cite{PiCTeX}:
%\begin{quote}
%\begin{verbatim}
%\def\PiC{%
%  P%
%  \kern-.12em%
%  \lower.5ex\hbox{I}%
%  \kern-.075em%
%  C%
%}
%\def\PiCTeX{%
%  \PiC
%  \kern-.11em%
%  \TeX
%}
%\end{verbatim}
%\end{quote}
%
%    \begin{macro}{\HoLogo@PiC}
%    \begin{macrocode}
\def\HoLogo@PiC#1{%
  P%
  \kern-.12em%
  \lower.5ex\hbox{I}%
  \kern-.075em%
  C%
  \HOLOGO@SpaceFactor
}
%    \end{macrocode}
%    \end{macro}
%    \begin{macro}{\HoLogoHtml@PiC}
%    \begin{macrocode}
\def\HoLogoHtml@PiC#1{%
  \HoLogoCss@PiC
  \HOLOGO@Span{PiC}{%
    P%
    \HOLOGO@Span{i}{I}%
    C%
  }%
}
%    \end{macrocode}
%    \end{macro}
%    \begin{macro}{\HoLogoCss@PiC}
%    \begin{macrocode}
\def\HoLogoCss@PiC{%
  \Css{%
    span.HoLogo-PiC span.HoLogo-i{%
      position:relative;%
      top:.5ex;%
      margin-left:-.12em;%
      margin-right:-.075em;%
      text-decoration:none;%
    }%
  }%
  \global\let\HoLogoCss@PiC\relax
}
%    \end{macrocode}
%    \end{macro}
%
%    \begin{macro}{\HoLogo@PiCTeX}
%    \begin{macrocode}
\def\HoLogo@PiCTeX#1{%
  \hologo{PiC}%
  \HOLOGO@discretionary
  \kern-.11em%
  \hologo{TeX}%
}
%    \end{macrocode}
%    \end{macro}
%    \begin{macro}{\HoLogoHtml@PiCTeX}
%    \begin{macrocode}
\def\HoLogoHtml@PiCTeX#1{%
  \HoLogoCss@PiCTeX
  \HOLOGO@Span{PiCTeX}{%
    \hologo{PiC}%
    \hologo{TeX}%
  }%
}
%    \end{macrocode}
%    \end{macro}
%    \begin{macro}{\HoLogoCss@PiCTeX}
%    \begin{macrocode}
\def\HoLogoCss@PiCTeX{%
  \Css{%
    span.HoLogo-PiCTeX span.HoLogo-PiC{%
      margin-right:-.11em;%
    }%
  }%
  \global\let\HoLogoCss@PiCTeX\relax
}
%    \end{macrocode}
%    \end{macro}
%
% \subsubsection{\hologo{teTeX}}
%
%    \begin{macro}{\HoLogo@teTeX}
%    \begin{macrocode}
\def\HoLogo@teTeX#1{%
  \HOLOGO@mbox{#1{t}{T}e}%
  \HOLOGO@discretionary
  \hologo{TeX}%
}
%    \end{macrocode}
%    \end{macro}
%    \begin{macro}{\HoLogoCs@teTeX}
%    \begin{macrocode}
\def\HoLogoCs@teTeX#1{#1{t}{T}dfTeX}
%    \end{macrocode}
%    \end{macro}
%    \begin{macro}{\HoLogoBkm@teTeX}
%    \begin{macrocode}
\def\HoLogoBkm@teTeX#1{%
  #1{t}{T}e\hologo{TeX}%
}
%    \end{macrocode}
%    \end{macro}
%    \begin{macro}{\HoLogoHtml@teTeX}
%    \begin{macrocode}
\let\HoLogoHtml@teTeX\HoLogo@teTeX
%    \end{macrocode}
%    \end{macro}
%
% \subsubsection{\hologo{TeX4ht}}
%
%    \begin{macro}{\HoLogo@TeX4ht}
%    \begin{macrocode}
\expandafter\def\csname HoLogo@TeX4ht\endcsname#1{%
  \HOLOGO@mbox{\hologo{TeX}4ht}%
}
%    \end{macrocode}
%    \end{macro}
%    \begin{macro}{\HoLogoHtml@TeX4ht}
%    \begin{macrocode}
\expandafter
\let\csname HoLogoHtml@TeX4ht\expandafter\endcsname
\csname HoLogo@TeX4ht\endcsname
%    \end{macrocode}
%    \end{macro}
%
%
% \subsubsection{\hologo{SageTeX}}
%
%    \begin{macro}{\HoLogo@SageTeX}
%    \begin{macrocode}
\def\HoLogo@SageTeX#1{%
  \HOLOGO@mbox{Sage}%
  \HOLOGO@discretionary
  \HOLOGO@NegativeKerning{eT,oT,To}%
  \hologo{TeX}%
}
%    \end{macrocode}
%    \end{macro}
%    \begin{macro}{\HoLogoHtml@SageTeX}
%    \begin{macrocode}
\let\HoLogoHtml@SageTeX\HoLogo@SageTeX
%    \end{macrocode}
%    \end{macro}
%
% \subsection{\hologo{METAFONT} and friends}
%
%    \begin{macro}{\HoLogo@METAFONT}
%    \begin{macrocode}
\def\HoLogo@METAFONT#1{%
  \HoLogoFont@font{METAFONT}{logo}{%
    \HOLOGO@mbox{META}%
    \HOLOGO@discretionary
    \HOLOGO@mbox{FONT}%
  }%
}
%    \end{macrocode}
%    \end{macro}
%
%    \begin{macro}{\HoLogo@METAPOST}
%    \begin{macrocode}
\def\HoLogo@METAPOST#1{%
  \HoLogoFont@font{METAPOST}{logo}{%
    \HOLOGO@mbox{META}%
    \HOLOGO@discretionary
    \HOLOGO@mbox{POST}%
  }%
}
%    \end{macrocode}
%    \end{macro}
%
%    \begin{macro}{\HoLogo@MetaFun}
%    \begin{macrocode}
\def\HoLogo@MetaFun#1{%
  \HOLOGO@mbox{Meta}%
  \HOLOGO@discretionary
  \HOLOGO@mbox{Fun}%
}
%    \end{macrocode}
%    \end{macro}
%
%    \begin{macro}{\HoLogo@MetaPost}
%    \begin{macrocode}
\def\HoLogo@MetaPost#1{%
  \HOLOGO@mbox{Meta}%
  \HOLOGO@discretionary
  \HOLOGO@mbox{Post}%
}
%    \end{macrocode}
%    \end{macro}
%
% \subsection{Others}
%
% \subsubsection{\hologo{biber}}
%
%    \begin{macro}{\HoLogo@biber}
%    \begin{macrocode}
\def\HoLogo@biber#1{%
  \HOLOGO@mbox{#1{b}{B}i}%
  \HOLOGO@discretionary
  \HOLOGO@mbox{ber}%
}
%    \end{macrocode}
%    \end{macro}
%    \begin{macro}{\HoLogoCs@biber}
%    \begin{macrocode}
\def\HoLogoCs@biber#1{#1{b}{B}iber}
%    \end{macrocode}
%    \end{macro}
%    \begin{macro}{\HoLogoBkm@biber}
%    \begin{macrocode}
\def\HoLogoBkm@biber#1{%
  #1{b}{B}iber%
}
%    \end{macrocode}
%    \end{macro}
%    \begin{macro}{\HoLogoHtml@biber}
%    \begin{macrocode}
\let\HoLogoHtml@biber\HoLogo@biber
%    \end{macrocode}
%    \end{macro}
%
% \subsubsection{\hologo{KOMAScript}}
%
%    \begin{macro}{\HoLogo@KOMAScript}
%    The definition for \hologo{KOMAScript} is taken
%    from \hologo{KOMAScript} (\xfile{scrlogo.dtx}, reformatted) \cite{scrlogo}:
%\begin{quote}
%\begin{verbatim}
%\@ifundefined{KOMAScript}{%
%  \DeclareRobustCommand{\KOMAScript}{%
%    \textsf{%
%      K\kern.05em O\kern.05emM\kern.05em A%
%      \kern.1em-\kern.1em %
%      Script%
%    }%
%  }%
%}{}
%\end{verbatim}
%\end{quote}
%    \begin{macrocode}
\def\HoLogo@KOMAScript#1{%
  \HoLogoFont@font{KOMAScript}{sf}{%
    \HOLOGO@mbox{%
      K\kern.05em%
      O\kern.05em%
      M\kern.05em%
      A%
    }%
    \kern.1em%
    \HOLOGO@hyphen
    \kern.1em%
    \HOLOGO@mbox{Script}%
  }%
}
%    \end{macrocode}
%    \end{macro}
%    \begin{macro}{\HoLogoBkm@KOMAScript}
%    \begin{macrocode}
\def\HoLogoBkm@KOMAScript#1{%
  KOMA-Script%
}
%    \end{macrocode}
%    \end{macro}
%    \begin{macro}{\HoLogoHtml@KOMAScript}
%    \begin{macrocode}
\def\HoLogoHtml@KOMAScript#1{%
  \HoLogoCss@KOMAScript
  \HoLogoFont@font{KOMAScript}{sf}{%
    \HOLOGO@Span{KOMAScript}{%
      K%
      \HOLOGO@Span{O}{O}%
      M%
      \HOLOGO@Span{A}{A}%
      \HOLOGO@Span{hyphen}{-}%
      Script%
    }%
  }%
}
%    \end{macrocode}
%    \end{macro}
%    \begin{macro}{\HoLogoCss@KOMAScript}
%    \begin{macrocode}
\def\HoLogoCss@KOMAScript{%
  \Css{%
    span.HoLogo-KOMAScript{%
      font-family:sans-serif;%
    }%
  }%
  \Css{%
    span.HoLogo-KOMAScript span.HoLogo-O{%
      padding-left:.05em;%
      padding-right:.05em;%
    }%
  }%
  \Css{%
    span.HoLogo-KOMAScript span.HoLogo-A{%
      padding-left:.05em;%
    }%
  }%
  \Css{%
    span.HoLogo-KOMAScript span.HoLogo-hyphen{%
      padding-left:.1em;%
      padding-right:.1em;%
    }%
  }%
  \global\let\HoLogoCss@KOMAScript\relax
}
%    \end{macrocode}
%    \end{macro}
%
% \subsubsection{\hologo{LyX}}
%
%    \begin{macro}{\HoLogo@LyX}
%    The definition is taken from the documentation source files
%    of \hologo{LyX}, \xfile{Intro.lyx} \cite{LyX}:
%\begin{quote}
%\begin{verbatim}
%\def\LyX{%
%  \texorpdfstring{%
%    L\kern-.1667em\lower.25em\hbox{Y}\kern-.125emX\@%
%  }{%
%    LyX%
%  }%
%}
%\end{verbatim}
%\end{quote}
%    \begin{macrocode}
\def\HoLogo@LyX#1{%
  L%
  \kern-.1667em%
  \lower.25em\hbox{Y}%
  \kern-.125em%
  X%
  \HOLOGO@SpaceFactor
}
%    \end{macrocode}
%    \end{macro}
%    \begin{macro}{\HoLogoHtml@LyX}
%    \begin{macrocode}
\def\HoLogoHtml@LyX#1{%
  \HoLogoCss@LyX
  \HOLOGO@Span{LyX}{%
    L%
    \HOLOGO@Span{y}{Y}%
    X%
  }%
}
%    \end{macrocode}
%    \end{macro}
%    \begin{macro}{\HoLogoCss@LyX}
%    \begin{macrocode}
\def\HoLogoCss@LyX{%
  \Css{%
    span.HoLogo-LyX span.HoLogo-y{%
      position:relative;%
      top:.25em;%
      margin-left:-.1667em;%
      margin-right:-.125em;%
      text-decoration:none;%
    }%
  }%
  \global\let\HoLogoCss@LyX\relax
}
%    \end{macrocode}
%    \end{macro}
%
% \subsubsection{\hologo{NTS}}
%
%    \begin{macro}{\HoLogo@NTS}
%    Definition for \hologo{NTS} can be found in
%    package \xpackage{etex\textunderscore man} for the \hologo{eTeX} manual \cite{etexman}
%    and in package \xpackage{dtklogos} \cite{dtklogos}:
%\begin{quote}
%\begin{verbatim}
%\def\NTS{%
%  \leavevmode
%  \hbox{%
%    $%
%      \cal N%
%      \kern-0.35em%
%      \lower0.5ex\hbox{$\cal T$}%
%      \kern-0.2em%
%      S%
%    $%
%  }%
%}
%\end{verbatim}
%\end{quote}
%    \begin{macrocode}
\def\HoLogo@NTS#1{%
  \HoLogoFont@font{NTS}{sy}{%
    N\/%
    \kern-.35em%
    \lower.5ex\hbox{T\/}%
    \kern-.2em%
    S\/%
  }%
  \HOLOGO@SpaceFactor
}
%    \end{macrocode}
%    \end{macro}
%
% \subsubsection{\Hologo{TTH} (\hologo{TeX} to HTML translator)}
%
%    Source: \url{http://hutchinson.belmont.ma.us/tth/}
%    In the HTML source the second `T' is printed as subscript.
%\begin{quote}
%\begin{verbatim}
%T<sub>T</sub>H
%\end{verbatim}
%\end{quote}
%    \begin{macro}{\HoLogo@TTH}
%    \begin{macrocode}
\def\HoLogo@TTH#1{%
  \ltx@mbox{%
    T\HOLOGO@SubScript{T}H%
  }%
  \HOLOGO@SpaceFactor
}
%    \end{macrocode}
%    \end{macro}
%
%    \begin{macro}{\HoLogoHtml@TTH}
%    \begin{macrocode}
\def\HoLogoHtml@TTH#1{%
  T\HCode{<sub>}T\HCode{</sub>}H%
}
%    \end{macrocode}
%    \end{macro}
%
% \subsubsection{\Hologo{HanTheThanh}}
%
%    Partial source: Package \xpackage{dtklogos}.
%    The double accent is U+1EBF (latin small letter e with circumflex
%    and acute).
%    \begin{macro}{\HoLogo@HanTheThanh}
%    \begin{macrocode}
\def\HoLogo@HanTheThanh#1{%
  \ltx@mbox{H\`an}%
  \HOLOGO@space
  \ltx@mbox{%
    Th%
    \HOLOGO@IfCharExists{"1EBF}{%
      \char"1EBF\relax
    }{%
      \^e\hbox to 0pt{\hss\raise .5ex\hbox{\'{}}}%
    }%
  }%
  \HOLOGO@space
  \ltx@mbox{Th\`anh}%
}
%    \end{macrocode}
%    \end{macro}
%    \begin{macro}{\HoLogoBkm@HanTheThanh}
%    \begin{macrocode}
\def\HoLogoBkm@HanTheThanh#1{%
  H\`an %
  Th\HOLOGO@PdfdocUnicode{\^e}{\9036\277} %
  Th\`anh%
}
%    \end{macrocode}
%    \end{macro}
%    \begin{macro}{\HoLogoHtml@HanTheThanh}
%    \begin{macrocode}
\def\HoLogoHtml@HanTheThanh#1{%
  H\`an %
  Th\HCode{&\ltx@hashchar x1ebf;} %
  Th\`anh%
}
%    \end{macrocode}
%    \end{macro}
%
% \subsection{Driver detection}
%
%    \begin{macrocode}
\HOLOGO@IfExists\InputIfFileExists{%
  \InputIfFileExists{hologo.cfg}{}{}%
}{%
  \ltx@IfUndefined{pdf@filesize}{%
    \def\HOLOGO@InputIfExists{%
      \openin\HOLOGO@temp=hologo.cfg\relax
      \ifeof\HOLOGO@temp
        \closein\HOLOGO@temp
      \else
        \closein\HOLOGO@temp
        \begingroup
          \def\x{LaTeX2e}%
        \expandafter\endgroup
        \ifx\fmtname\x
          % \iffalse meta-comment
%
% File: hologo.dtx
% Version: 2016/05/12 v1.11
% Info: A logo collection with bookmark support
%
% Copyright (C) 2010-2012 by
%    Heiko Oberdiek <heiko.oberdiek at googlemail.com>
%
% This work may be distributed and/or modified under the
% conditions of the LaTeX Project Public License, either
% version 1.3c of this license or (at your option) any later
% version. This version of this license is in
%    http://www.latex-project.org/lppl/lppl-1-3c.txt
% and the latest version of this license is in
%    http://www.latex-project.org/lppl.txt
% and version 1.3 or later is part of all distributions of
% LaTeX version 2005/12/01 or later.
%
% This work has the LPPL maintenance status "maintained".
%
% This Current Maintainer of this work is Heiko Oberdiek.
%
% The Base Interpreter refers to any `TeX-Format',
% because some files are installed in TDS:tex/generic//.
%
% This work consists of the main source file hologo.dtx
% and the derived files
%    hologo.sty, hologo.pdf, hologo.ins, hologo.drv, hologo-example.tex,
%    hologo-test1.tex, hologo-test-spacefactor.tex,
%    hologo-test-list.tex.
%
% Distribution:
%    CTAN:macros/latex/contrib/oberdiek/hologo.dtx
%    CTAN:macros/latex/contrib/oberdiek/hologo.pdf
%
% Unpacking:
%    (a) If hologo.ins is present:
%           tex hologo.ins
%    (b) Without hologo.ins:
%           tex hologo.dtx
%    (c) If you insist on using LaTeX
%           latex \let\install=y\input{hologo.dtx}
%        (quote the arguments according to the demands of your shell)
%
% Documentation:
%    (a) If hologo.drv is present:
%           latex hologo.drv
%    (b) Without hologo.drv:
%           latex hologo.dtx; ...
%    The class ltxdoc loads the configuration file ltxdoc.cfg
%    if available. Here you can specify further options, e.g.
%    use A4 as paper format:
%       \PassOptionsToClass{a4paper}{article}
%
%    Programm calls to get the documentation (example):
%       pdflatex hologo.dtx
%       makeindex -s gind.ist hologo.idx
%       pdflatex hologo.dtx
%       makeindex -s gind.ist hologo.idx
%       pdflatex hologo.dtx
%
% Installation:
%    TDS:tex/generic/oberdiek/hologo.sty
%    TDS:doc/latex/oberdiek/hologo.pdf
%    TDS:doc/latex/oberdiek/example/hologo-example.tex
%    TDS:doc/latex/oberdiek/test/hologo-test1.tex
%    TDS:doc/latex/oberdiek/test/hologo-test-spacefactor.tex
%    TDS:doc/latex/oberdiek/test/hologo-test-list.tex
%    TDS:source/latex/oberdiek/hologo.dtx
%
%<*ignore>
\begingroup
  \catcode123=1 %
  \catcode125=2 %
  \def\x{LaTeX2e}%
\expandafter\endgroup
\ifcase 0\ifx\install y1\fi\expandafter
         \ifx\csname processbatchFile\endcsname\relax\else1\fi
         \ifx\fmtname\x\else 1\fi\relax
\else\csname fi\endcsname
%</ignore>
%<*install>
\input docstrip.tex
\Msg{************************************************************************}
\Msg{* Installation}
\Msg{* Package: hologo 2016/05/12 v1.11 A logo collection with bookmark support (HO)}
\Msg{************************************************************************}

\keepsilent
\askforoverwritefalse

\let\MetaPrefix\relax
\preamble

This is a generated file.

Project: hologo
Version: 2016/05/12 v1.11

Copyright (C) 2010-2012 by
   Heiko Oberdiek <heiko.oberdiek at googlemail.com>

This work may be distributed and/or modified under the
conditions of the LaTeX Project Public License, either
version 1.3c of this license or (at your option) any later
version. This version of this license is in
   http://www.latex-project.org/lppl/lppl-1-3c.txt
and the latest version of this license is in
   http://www.latex-project.org/lppl.txt
and version 1.3 or later is part of all distributions of
LaTeX version 2005/12/01 or later.

This work has the LPPL maintenance status "maintained".

This Current Maintainer of this work is Heiko Oberdiek.

The Base Interpreter refers to any `TeX-Format',
because some files are installed in TDS:tex/generic//.

This work consists of the main source file hologo.dtx
and the derived files
   hologo.sty, hologo.pdf, hologo.ins, hologo.drv, hologo-example.tex,
   hologo-test1.tex, hologo-test-spacefactor.tex,
   hologo-test-list.tex.

\endpreamble
\let\MetaPrefix\DoubleperCent

\generate{%
  \file{hologo.ins}{\from{hologo.dtx}{install}}%
  \file{hologo.drv}{\from{hologo.dtx}{driver}}%
  \usedir{tex/generic/oberdiek}%
  \file{hologo.sty}{\from{hologo.dtx}{package}}%
  \usedir{doc/latex/oberdiek/example}%
  \file{hologo-example.tex}{\from{hologo.dtx}{example}}%
  \usedir{doc/latex/oberdiek/test}%
  \file{hologo-test1.tex}{\from{hologo.dtx}{test1}}%
  \file{hologo-test-spacefactor.tex}{\from{hologo.dtx}{test-spacefactor}}%
  \file{hologo-test-list.tex}{\from{hologo.dtx}{test-list}}%
  \nopreamble
  \nopostamble
  \usedir{source/latex/oberdiek/catalogue}%
  \file{hologo.xml}{\from{hologo.dtx}{catalogue}}%
}

\catcode32=13\relax% active space
\let =\space%
\Msg{************************************************************************}
\Msg{*}
\Msg{* To finish the installation you have to move the following}
\Msg{* file into a directory searched by TeX:}
\Msg{*}
\Msg{*     hologo.sty}
\Msg{*}
\Msg{* To produce the documentation run the file `hologo.drv'}
\Msg{* through LaTeX.}
\Msg{*}
\Msg{* Happy TeXing!}
\Msg{*}
\Msg{************************************************************************}

\endbatchfile
%</install>
%<*ignore>
\fi
%</ignore>
%<*driver>
\NeedsTeXFormat{LaTeX2e}
\ProvidesFile{hologo.drv}%
  [2016/05/12 v1.11 A logo collection with bookmark support (HO)]%
\documentclass{ltxdoc}
\usepackage{holtxdoc}[2011/11/22]
\usepackage{hologo}[2016/05/12]
\usepackage{longtable}
\usepackage{array}
\usepackage{paralist}
%\usepackage[T1]{fontenc}
%\usepackage{lmodern}
\begin{document}
  \DocInput{hologo.dtx}%
\end{document}
%</driver>
% \fi
%
%
% \CharacterTable
%  {Upper-case    \A\B\C\D\E\F\G\H\I\J\K\L\M\N\O\P\Q\R\S\T\U\V\W\X\Y\Z
%   Lower-case    \a\b\c\d\e\f\g\h\i\j\k\l\m\n\o\p\q\r\s\t\u\v\w\x\y\z
%   Digits        \0\1\2\3\4\5\6\7\8\9
%   Exclamation   \!     Double quote  \"     Hash (number) \#
%   Dollar        \$     Percent       \%     Ampersand     \&
%   Acute accent  \'     Left paren    \(     Right paren   \)
%   Asterisk      \*     Plus          \+     Comma         \,
%   Minus         \-     Point         \.     Solidus       \/
%   Colon         \:     Semicolon     \;     Less than     \<
%   Equals        \=     Greater than  \>     Question mark \?
%   Commercial at \@     Left bracket  \[     Backslash     \\
%   Right bracket \]     Circumflex    \^     Underscore    \_
%   Grave accent  \`     Left brace    \{     Vertical bar  \|
%   Right brace   \}     Tilde         \~}
%
% \GetFileInfo{hologo.drv}
%
% \title{The \xpackage{hologo} package}
% \date{2016/05/12 v1.11}
% \author{Heiko Oberdiek\\\xemail{heiko.oberdiek at googlemail.com}}
%
% \maketitle
%
% \begin{abstract}
% This package starts a collection of logos with support for bookmarks
% strings.
% \end{abstract}
%
% \tableofcontents
%
% \section{Documentation}
%
% \subsection{Logo macros}
%
% \begin{declcs}{hologo} \M{name}
% \end{declcs}
% Macro \cs{hologo} sets the logo with name \meta{name}.
% The following table shows the supported names.
%
% \begingroup
%   \def\hologoEntry#1#2#3{^^A
%     #1&#2&\hologoLogoSetup{#1}{variant=#2}\hologo{#1}&#3\tabularnewline
%   }
%   \begin{longtable}{>{\ttfamily}l>{\ttfamily}lll}
%     \rmfamily\bfseries{name} & \rmfamily\bfseries variant
%     & \bfseries logo & \bfseries since\\
%     \hline
%     \endhead
%     \hologoList
%   \end{longtable}
% \endgroup
%
% \begin{declcs}{Hologo} \M{name}
% \end{declcs}
% Macro \cs{Hologo} starts the logo \meta{name} with an uppercase
% letter. As an exception small greek letters are not converted
% to uppercase. Examples, see \hologo{eTeX} and \hologo{ExTeX}.
%
% \subsection{Setup macros}
%
% The package does not support package options, but the following
% setup macros can be used to set options.
%
% \begin{declcs}{hologoSetup} \M{key value list}
% \end{declcs}
% Macro \cs{hologoSetup} sets global options.
%
% \begin{declcs}{hologoLogoSetup} \M{logo} \M{key value list}
% \end{declcs}
% Some options can also be used to configure a logo.
% These settings take precedence over global option settings.
%
% \subsection{Options}\label{sec:options}
%
% There are boolean and string options:
% \begin{description}
% \item[Boolean option:]
% It takes |true| or |false|
% as value. If the value is omitted, then |true| is used.
% \item[String option:]
% A value must be given as string. (But the string might be empty.)
% \end{description}
% The following options can be used both in \cs{hologoSetup}
% and \cs{hologoLogoSetup}:
% \begin{description}
% \def\entry#1{\item[\xoption{#1}:]}
% \entry{break}
%   enables or disables line breaks inside the logo. This setting is
%   refined by options \xoption{hyphenbreak}, \xoption{spacebreak}
%   or \xoption{discretionarybreak}.
%   Default is |false|.
% \entry{hyphenbreak}
%   enables or disables the line break right after the hyphen character.
% \entry{spacebreak}
%   enables or disables line breaks at space characters.
% \entry{discretionarybreak}
%   enables or disables line breaks at hyphenation points
%   (inserted by \cs{-}).
% \end{description}
% Macro \cs{hologoLogoSetup} also knows:
% \begin{description}
% \item[\xoption{variant}:]
%   This is a string option. It specifies a variant of a logo that
%   must exist. An empty string selects the package default variant.
% \end{description}
% Example:
% \begin{quote}
%   |\hologoSetup{break=false}|\\
%   |\hologoLogoSetup{plainTeX}{variant=hyphen,hyphenbreak}|\\
%   Then ``plain-\TeX'' contains one break point after the hyphen.
% \end{quote}
%
% \subsection{Driver options}
%
% Sometimes graphical operations are needed to construct some
% glyphs (e.g.\ \hologo{XeTeX}). If package \xpackage{graphics}
% or package \xpackage{pgf} are found, then the macros are taken
% from there. Otherwise the packge defines its own operations
% and therefore needs the driver information. Many drivers are
% detected automatically (\hologo{pdfTeX}/\hologo{LuaTeX}
% in PDF mode, \hologo{XeTeX}, \hologo{VTeX}). These have precedence
% over a driver option. The driver can be given as package option
% or using \cs{hologoDriverSetup}.
% The following list contains the recognized driver options:
% \begin{itemize}
% \item \xoption{pdftex}, \xoption{luatex}
% \item \xoption{dvipdfm}, \xoption{dvipdfmx}
% \item \xoption{dvips}, \xoption{dvipsone}, \xoption{xdvi}
% \item \xoption{xetex}
% \item \xoption{vtex}
% \end{itemize}
% The left driver of a line is the driver name that is used internally.
% The following names are aliases for drivers that use the
% same method. Therefore the entry in the \xext{log} file for
% the used driver prints the internally used driver name.
% \begin{description}
% \item[\xoption{driverfallback}:]
%   This option expects a driver that is used,
%   if the driver could not be detected automatically.
% \end{description}
%
% \begin{declcs}{hologoDriverSetup} \M{driver option}
% \end{declcs}
% The driver can also be configured after package loading
% using \cs{hologoDriverSetup}, also the way for \hologo{plainTeX}
% to setup the driver.
%
% \subsection{Font setup}
%
% Some logos require a special font, but should also be usable by
% \hologo{plainTeX}. Therefore the package provides some ways
% to influence the font settings. The options below
% take font settings as values. Both font commands
% such as \cs{sffamily} and macros that take one argument
% like \cs{textsf} can be used.
%
% \begin{declcs}{hologoFontSetup} \M{key value list}
% \end{declcs}
% Macro \cs{hologoFontSetup} sets the fonts for all logos.
% Supported keys:
% \begin{description}
% \def\entry#1{\item[\xoption{#1}:]}
% \entry{general}
%   This font is used for all logos. The default is empty.
%   That means no special font is used.
% \entry{bibsf}
%   This font is used for
%   {\hologoLogoSetup{BibTeX}{variant=sf}\hologo{BibTeX}}
%   with variant \xoption{sf}.
% \entry{rm}
%   This font is a serif font. It is used for \hologo{ExTeX}.
% \entry{sc}
%   This font specifies a small caps font. It is used for
%   {\hologoLogoSetup{BibTeX}{variant=sc}\hologo{BibTeX}}
%   with variant \xoption{sc}.
% \entry{sf}
%   This font specifies a sans serif font. The default
%   is \cs{sffamily}, then \cs{sf} is tried. Otherwise
%   a warning is given. It is used by \hologo{KOMAScript}.
% \entry{sy}
%   This is the font for math symbols (e.g. cmsy).
%   It is used by \hologo{AmS}, \hologo{NTS}, \hologo{ExTeX}.
% \entry{logo}
%   \hologo{METAFONT} and \hologo{METAPOST} are using that font.
%   In \hologo{LaTeX} \cs{logofamily} is used and
%   the definitions of package \xpackage{mflogo} are used
%   if the package is not loaded.
%   Otherwise the \cs{tenlogo} is used and defined
%   if it does not already exists.
% \end{description}
%
% \begin{declcs}{hologoLogoFontSetup} \M{logo} \M{key value list}
% \end{declcs}
% Fonts can also be set for a logo or logo component separately,
% see the following list.
% The keys are the same as for \cs{hologoFontSetup}.
%
% \begin{longtable}{>{\ttfamily}l>{\sffamily}ll}
%   \meta{logo} & keys & result\\
%   \hline
%   \endhead
%   BibTeX & bibsf & {\hologoLogoSetup{BibTeX}{variant=sf}\hologo{BibTeX}}\\[.5ex]
%   BibTeX & sc & {\hologoLogoSetup{BibTeX}{variant=sc}\hologo{BibTeX}}\\[.5ex]
%   ExTeX & rm & \hologo{ExTeX}\\
%   SliTeX & rm & \hologo{SliTeX}\\[.5ex]
%   AmS & sy & \hologo{AmS}\\
%   ExTeX & sy & \hologo{ExTeX}\\
%   NTS & sy & \hologo{NTS}\\[.5ex]
%   KOMAScript & sf & \hologo{KOMAScript}\\[.5ex]
%   METAFONT & logo & \hologo{METAFONT}\\
%   METAPOST & logo & \hologo{METAPOST}\\[.5ex]
%   SliTeX & sc \hologo{SliTeX}
% \end{longtable}
%
% \subsubsection{Font order}
%
% For all logos the font \xoption{general} is applied first.
% Example:
%\begin{quote}
%|\hologoFontSetup{general=\color{red}}|
%\end{quote}
% will print red logos.
% Then if the font uses a special font \xoption{sf}, for example,
% the font is applied that is setup by \cs{hologoLogoFontSetup}.
% If this font is not setup, then the common font setup
% by \cs{hologoFontSetup} is used. Otherwise a warning is given,
% that there is no font configured.
%
% \subsection{Additional user macros}
%
% Usually a variant of a logo is configured by using
% \cs{hologoLogoSetup}, because it is bad style to mix
% different variants of the same logo in the same text.
% There the following macros are a convenience for testing.
%
% \begin{declcs}{hologoVariant} \M{name} \M{variant}\\
%   \cs{HologoVariant} \M{name} \M{variant}
% \end{declcs}
% Logo \meta{name} is set using \meta{variant} that specifies
% explicitely which variant of the macro is used. If the argument
% is empty, then the default form of the logo is used
% (configurable by \cs{hologoLogoSetup}).
%
% \cs{HologoVariant} is used if the logo is set in a context
% that needs an uppercase first letter (beginning of a sentence, \dots).
%
% \begin{declcs}{hologoList}\\
%   \cs{hologoEntry} \M{logo} \M{variant} \M{since}
% \end{declcs}
% Macro \cs{hologoList} contains all logos that are provided
% by the package including variants. The list consists of calls
% of \cs{hologoEntry} with three arguments starting with the
% logo name \meta{logo} and its variant \meta{variant}. An empty
% variant means the current default. Argument \meta{since} specifies
% with version of the package \xpackage{hologo} is needed to get
% the logo. If the logo is fixed, then the date gets updated.
% Therefore the date \meta{since} is not exactly the date of
% the first introduction, but rather the date of the latest fix.
%
% Before \cs{hologoList} can be used, macro \cs{hologoEntry} needs
% a definition. The example file in section \ref{sec:example}
% shows applications of \cs{hologoList}.
%
% \subsection{Supported contexts}
%
% Macros \cs{hologo} and friends support special contexts:
% \begin{itemize}
% \item \hologo{LaTeX}'s protection mechanism.
% \item Bookmarks of package \xpackage{hyperref}.
% \item Package \xpackage{tex4ht}.
% \item The macros can be used inside \cs{csname} constructs,
%   if \cs{ifincsname} is available (\hologo{pdfTeX}, \hologo{XeTeX},
%   \hologo{LuaTeX}).
% \end{itemize}
%
% \subsection{Example}
% \label{sec:example}
%
% The following example prints the logos in different fonts.
%    \begin{macrocode}
%<*example>
%<<verbatim
\NeedsTeXFormat{LaTeX2e}
\documentclass[a4paper]{article}
\usepackage[
  hmargin=20mm,
  vmargin=20mm,
]{geometry}
\pagestyle{empty}
\usepackage{hologo}[2016/05/12]
\usepackage{longtable}
\usepackage{array}
\setlength{\extrarowheight}{2pt}
\usepackage[T1]{fontenc}
\usepackage{lmodern}
\usepackage{pdflscape}
\usepackage[
  pdfencoding=auto,
]{hyperref}
\hypersetup{
  pdfauthor={Heiko Oberdiek},
  pdftitle={Example for package `hologo'},
  pdfsubject={Logos with fonts lmr, lmss, qtm, qpl, qhv},
}
\usepackage{bookmark}

% Print the logo list on the console

\begingroup
  \typeout{}%
  \typeout{*** Begin of logo list ***}%
  \newcommand*{\hologoEntry}[3]{%
    \typeout{#1 \ifx\\#2\\\else(#2) \fi[#3]}%
  }%
  \hologoList
  \typeout{*** End of logo list ***}%
  \typeout{}%
\endgroup

\begin{document}
\begin{landscape}

  \section{Example file for package `hologo'}

  % Table for font names

  \begin{longtable}{>{\bfseries}ll}
    \textbf{font} & \textbf{Font name}\\
    \hline
    lmr & Latin Modern Roman\\
    lmss & Latin Modern Sans\\
    qtm & \TeX\ Gyre Termes\\
    qhv & \TeX\ Gyre Heros\\
    qpl & \TeX\ Gyre Pagella\\
  \end{longtable}

  % Logo list with logos in different fonts

  \begingroup
    \newcommand*{\SetVariant}[2]{%
      \ifx\\#2\\%
      \else
        \hologoLogoSetup{#1}{variant=#2}%
      \fi
    }%
    \newcommand*{\hologoEntry}[3]{%
      \SetVariant{#1}{#2}%
      \raisebox{1em}[0pt][0pt]{\hypertarget{#1@#2}{}}%
      \bookmark[%
        dest={#1@#2},%
      ]{%
        #1\ifx\\#2\\\else\space(#2)\fi: \Hologo{#1}, \hologo{#1} %
        [Unicode]%
      }%
      \hypersetup{unicode=false}%
      \bookmark[%
        dest={#1@#2},%
      ]{%
        #1\ifx\\#2\\\else\space(#2)\fi: \Hologo{#1}, \hologo{#1} %
        [PDFDocEncoding]%
      }%
      \texttt{#1}%
      &%
      \texttt{#2}%
      &%
      \Hologo{#1}%
      &%
      \SetVariant{#1}{#2}%
      \hologo{#1}%
      &%
      \SetVariant{#1}{#2}%
      \fontfamily{qtm}\selectfont
      \hologo{#1}%
      &%
      \SetVariant{#1}{#2}%
      \fontfamily{qpl}\selectfont
      \hologo{#1}%
      &%
      \SetVariant{#1}{#2}%
      \textsf{\hologo{#1}}%
      &%
      \SetVariant{#1}{#2}%
      \fontfamily{qhv}\selectfont
      \hologo{#1}%
      \tabularnewline
    }%
    \begin{longtable}{llllllll}%
      \textbf{\textit{logo}} & \textbf{\textit{variant}} &
      \texttt{\string\Hologo} &
      \textbf{lmr} & \textbf{qtm} & \textbf{qpl} &
      \textbf{lmss} & \textbf{qhv}
      \tabularnewline
      \hline
      \endhead
      \hologoList
    \end{longtable}%
  \endgroup

\end{landscape}
\end{document}
%verbatim
%</example>
%    \end{macrocode}
%
% \StopEventually{
% }
%
% \section{Implementation}
%    \begin{macrocode}
%<*package>
%    \end{macrocode}
%    Reload check, especially if the package is not used with \LaTeX.
%    \begin{macrocode}
\begingroup\catcode61\catcode48\catcode32=10\relax%
  \catcode13=5 % ^^M
  \endlinechar=13 %
  \catcode35=6 % #
  \catcode39=12 % '
  \catcode44=12 % ,
  \catcode45=12 % -
  \catcode46=12 % .
  \catcode58=12 % :
  \catcode64=11 % @
  \catcode123=1 % {
  \catcode125=2 % }
  \expandafter\let\expandafter\x\csname ver@hologo.sty\endcsname
  \ifx\x\relax % plain-TeX, first loading
  \else
    \def\empty{}%
    \ifx\x\empty % LaTeX, first loading,
      % variable is initialized, but \ProvidesPackage not yet seen
    \else
      \expandafter\ifx\csname PackageInfo\endcsname\relax
        \def\x#1#2{%
          \immediate\write-1{Package #1 Info: #2.}%
        }%
      \else
        \def\x#1#2{\PackageInfo{#1}{#2, stopped}}%
      \fi
      \x{hologo}{The package is already loaded}%
      \aftergroup\endinput
    \fi
  \fi
\endgroup%
%    \end{macrocode}
%    Package identification:
%    \begin{macrocode}
\begingroup\catcode61\catcode48\catcode32=10\relax%
  \catcode13=5 % ^^M
  \endlinechar=13 %
  \catcode35=6 % #
  \catcode39=12 % '
  \catcode40=12 % (
  \catcode41=12 % )
  \catcode44=12 % ,
  \catcode45=12 % -
  \catcode46=12 % .
  \catcode47=12 % /
  \catcode58=12 % :
  \catcode64=11 % @
  \catcode91=12 % [
  \catcode93=12 % ]
  \catcode123=1 % {
  \catcode125=2 % }
  \expandafter\ifx\csname ProvidesPackage\endcsname\relax
    \def\x#1#2#3[#4]{\endgroup
      \immediate\write-1{Package: #3 #4}%
      \xdef#1{#4}%
    }%
  \else
    \def\x#1#2[#3]{\endgroup
      #2[{#3}]%
      \ifx#1\@undefined
        \xdef#1{#3}%
      \fi
      \ifx#1\relax
        \xdef#1{#3}%
      \fi
    }%
  \fi
\expandafter\x\csname ver@hologo.sty\endcsname
\ProvidesPackage{hologo}%
  [2016/05/12 v1.11 A logo collection with bookmark support (HO)]%
%    \end{macrocode}
%
%    \begin{macrocode}
\begingroup\catcode61\catcode48\catcode32=10\relax%
  \catcode13=5 % ^^M
  \endlinechar=13 %
  \catcode123=1 % {
  \catcode125=2 % }
  \catcode64=11 % @
  \def\x{\endgroup
    \expandafter\edef\csname HOLOGO@AtEnd\endcsname{%
      \endlinechar=\the\endlinechar\relax
      \catcode13=\the\catcode13\relax
      \catcode32=\the\catcode32\relax
      \catcode35=\the\catcode35\relax
      \catcode61=\the\catcode61\relax
      \catcode64=\the\catcode64\relax
      \catcode123=\the\catcode123\relax
      \catcode125=\the\catcode125\relax
    }%
  }%
\x\catcode61\catcode48\catcode32=10\relax%
\catcode13=5 % ^^M
\endlinechar=13 %
\catcode35=6 % #
\catcode64=11 % @
\catcode123=1 % {
\catcode125=2 % }
\def\TMP@EnsureCode#1#2{%
  \edef\HOLOGO@AtEnd{%
    \HOLOGO@AtEnd
    \catcode#1=\the\catcode#1\relax
  }%
  \catcode#1=#2\relax
}
\TMP@EnsureCode{10}{12}% ^^J
\TMP@EnsureCode{33}{12}% !
\TMP@EnsureCode{34}{12}% "
\TMP@EnsureCode{36}{3}% $
\TMP@EnsureCode{38}{4}% &
\TMP@EnsureCode{39}{12}% '
\TMP@EnsureCode{40}{12}% (
\TMP@EnsureCode{41}{12}% )
\TMP@EnsureCode{42}{12}% *
\TMP@EnsureCode{43}{12}% +
\TMP@EnsureCode{44}{12}% ,
\TMP@EnsureCode{45}{12}% -
\TMP@EnsureCode{46}{12}% .
\TMP@EnsureCode{47}{12}% /
\TMP@EnsureCode{58}{12}% :
\TMP@EnsureCode{59}{12}% ;
\TMP@EnsureCode{60}{12}% <
\TMP@EnsureCode{62}{12}% >
\TMP@EnsureCode{63}{12}% ?
\TMP@EnsureCode{91}{12}% [
\TMP@EnsureCode{93}{12}% ]
\TMP@EnsureCode{94}{7}% ^ (superscript)
\TMP@EnsureCode{95}{8}% _ (subscript)
\TMP@EnsureCode{96}{12}% `
\TMP@EnsureCode{124}{12}% |
\edef\HOLOGO@AtEnd{%
  \HOLOGO@AtEnd
  \escapechar\the\escapechar\relax
  \noexpand\endinput
}
\escapechar=92 %
%    \end{macrocode}
%
% \subsection{Logo list}
%
%    \begin{macro}{\hologoList}
%    \begin{macrocode}
\def\hologoList{%
  \hologoEntry{(La)TeX}{}{2011/10/01}%
  \hologoEntry{AmSLaTeX}{}{2010/04/16}%
  \hologoEntry{AmSTeX}{}{2010/04/16}%
  \hologoEntry{biber}{}{2011/10/01}%
  \hologoEntry{BibTeX}{}{2011/10/01}%
  \hologoEntry{BibTeX}{sf}{2011/10/01}%
  \hologoEntry{BibTeX}{sc}{2011/10/01}%
  \hologoEntry{BibTeX8}{}{2011/11/22}%
  \hologoEntry{ConTeXt}{}{2011/03/25}%
  \hologoEntry{ConTeXt}{narrow}{2011/03/25}%
  \hologoEntry{ConTeXt}{simple}{2011/03/25}%
  \hologoEntry{emTeX}{}{2010/04/26}%
  \hologoEntry{eTeX}{}{2010/04/08}%
  \hologoEntry{ExTeX}{}{2011/10/01}%
  \hologoEntry{HanTheThanh}{}{2011/11/29}%
  \hologoEntry{iniTeX}{}{2011/10/01}%
  \hologoEntry{KOMAScript}{}{2011/10/01}%
  \hologoEntry{La}{}{2010/05/08}%
  \hologoEntry{LaTeX}{}{2010/04/08}%
  \hologoEntry{LaTeX2e}{}{2010/04/08}%
  \hologoEntry{LaTeX3}{}{2010/04/24}%
  \hologoEntry{LaTeXe}{}{2010/04/08}%
  \hologoEntry{LaTeXML}{}{2011/11/22}%
  \hologoEntry{LaTeXTeX}{}{2011/10/01}%
  \hologoEntry{LuaLaTeX}{}{2010/04/08}%
  \hologoEntry{LuaTeX}{}{2010/04/08}%
  \hologoEntry{LyX}{}{2011/10/01}%
  \hologoEntry{METAFONT}{}{2011/10/01}%
  \hologoEntry{MetaFun}{}{2011/10/01}%
  \hologoEntry{METAPOST}{}{2011/10/01}%
  \hologoEntry{MetaPost}{}{2011/10/01}%
  \hologoEntry{MiKTeX}{}{2011/10/01}%
  \hologoEntry{NTS}{}{2011/10/01}%
  \hologoEntry{OzMF}{}{2011/10/01}%
  \hologoEntry{OzMP}{}{2011/10/01}%
  \hologoEntry{OzTeX}{}{2011/10/01}%
  \hologoEntry{OzTtH}{}{2011/10/01}%
  \hologoEntry{PCTeX}{}{2011/10/01}%
  \hologoEntry{pdfTeX}{}{2011/10/01}%
  \hologoEntry{pdfLaTeX}{}{2011/10/01}%
  \hologoEntry{PiC}{}{2011/10/01}%
  \hologoEntry{PiCTeX}{}{2011/10/01}%
  \hologoEntry{plainTeX}{}{2010/04/08}%
  \hologoEntry{plainTeX}{space}{2010/04/16}%
  \hologoEntry{plainTeX}{hyphen}{2010/04/16}%
  \hologoEntry{plainTeX}{runtogether}{2010/04/16}%
  \hologoEntry{SageTeX}{}{2011/11/22}%
  \hologoEntry{SLiTeX}{}{2011/10/01}%
  \hologoEntry{SLiTeX}{lift}{2011/10/01}%
  \hologoEntry{SLiTeX}{narrow}{2011/10/01}%
  \hologoEntry{SLiTeX}{simple}{2011/10/01}%
  \hologoEntry{SliTeX}{}{2011/10/01}%
  \hologoEntry{SliTeX}{narrow}{2011/10/01}%
  \hologoEntry{SliTeX}{simple}{2011/10/01}%
  \hologoEntry{SliTeX}{lift}{2011/10/01}%
  \hologoEntry{teTeX}{}{2011/10/01}%
  \hologoEntry{TeX}{}{2010/04/08}%
  \hologoEntry{TeX4ht}{}{2011/11/22}%
  \hologoEntry{TTH}{}{2011/11/22}%
  \hologoEntry{virTeX}{}{2011/10/01}%
  \hologoEntry{VTeX}{}{2010/04/24}%
  \hologoEntry{Xe}{}{2010/04/08}%
  \hologoEntry{XeLaTeX}{}{2010/04/08}%
  \hologoEntry{XeTeX}{}{2010/04/08}%
}
%    \end{macrocode}
%    \end{macro}
%
% \subsection{Load resources}
%
%    \begin{macrocode}
\begingroup\expandafter\expandafter\expandafter\endgroup
\expandafter\ifx\csname RequirePackage\endcsname\relax
  \def\TMP@RequirePackage#1[#2]{%
    \begingroup\expandafter\expandafter\expandafter\endgroup
    \expandafter\ifx\csname ver@#1.sty\endcsname\relax
      \input #1.sty\relax
    \fi
  }%
  \TMP@RequirePackage{ltxcmds}[2011/02/04]%
  \TMP@RequirePackage{infwarerr}[2010/04/08]%
  \TMP@RequirePackage{kvsetkeys}[2010/03/01]%
  \TMP@RequirePackage{kvdefinekeys}[2010/03/01]%
  \TMP@RequirePackage{pdftexcmds}[2010/04/01]%
  \TMP@RequirePackage{ifpdf}[2010/01/28]%
  \TMP@RequirePackage{ifluatex}[2010/03/01]%
  \ltx@IfUndefined{newif}{%
    \expandafter\let\csname newif\endcsname\ltx@newif
  }{}%
  \TMP@RequirePackage{ifxetex}[2009/01/23]%
  \TMP@RequirePackage{ifvtex}[2010/03/01]%
\else
  \RequirePackage{ltxcmds}[2011/02/04]%
  \RequirePackage{infwarerr}[2010/04/08]%
  \RequirePackage{kvsetkeys}[2010/03/01]%
  \RequirePackage{kvdefinekeys}[2010/03/01]%
  \RequirePackage{pdftexcmds}[2010/04/01]%
  \RequirePackage{ifpdf}[2010/01/28]%
  \RequirePackage{ifluatex}[2010/03/01]%
  \RequirePackage{ifxetex}[2009/01/23]%
  \RequirePackage{ifvtex}[2010/03/01]%
\fi
%    \end{macrocode}
%
%    \begin{macro}{\HOLOGO@IfDefined}
%    \begin{macrocode}
\def\HOLOGO@IfExists#1{%
  \ifx\@undefined#1%
    \expandafter\ltx@secondoftwo
  \else
    \ifx\relax#1%
      \expandafter\ltx@secondoftwo
    \else
      \expandafter\expandafter\expandafter\ltx@firstoftwo
    \fi
  \fi
}
%    \end{macrocode}
%    \end{macro}
%
% \subsection{Setup macros}
%
%    \begin{macro}{\hologoSetup}
%    \begin{macrocode}
\def\hologoSetup{%
  \let\HOLOGO@name\relax
  \HOLOGO@Setup
}
%    \end{macrocode}
%    \end{macro}
%
%    \begin{macro}{\hologoLogoSetup}
%    \begin{macrocode}
\def\hologoLogoSetup#1{%
  \edef\HOLOGO@name{#1}%
  \ltx@IfUndefined{HoLogo@\HOLOGO@name}{%
    \@PackageError{hologo}{%
      Unknown logo `\HOLOGO@name'%
    }\@ehc
    \ltx@gobble
  }{%
    \HOLOGO@Setup
  }%
}
%    \end{macrocode}
%    \end{macro}
%
%    \begin{macro}{\HOLOGO@Setup}
%    \begin{macrocode}
\def\HOLOGO@Setup{%
  \kvsetkeys{HoLogo}%
}
%    \end{macrocode}
%    \end{macro}
%
% \subsection{Options}
%
%    \begin{macro}{\HOLOGO@DeclareBoolOption}
%    \begin{macrocode}
\def\HOLOGO@DeclareBoolOption#1{%
  \expandafter\chardef\csname HOLOGOOPT@#1\endcsname\ltx@zero
  \kv@define@key{HoLogo}{#1}[true]{%
    \def\HOLOGO@temp{##1}%
    \ifx\HOLOGO@temp\HOLOGO@true
      \ifx\HOLOGO@name\relax
        \expandafter\chardef\csname HOLOGOOPT@#1\endcsname=\ltx@one
      \else
        \expandafter\chardef\csname
        HoLogoOpt@#1@\HOLOGO@name\endcsname\ltx@one
      \fi
      \HOLOGO@SetBreakAll{#1}%
    \else
      \ifx\HOLOGO@temp\HOLOGO@false
        \ifx\HOLOGO@name\relax
          \expandafter\chardef\csname HOLOGOOPT@#1\endcsname=\ltx@zero
        \else
          \expandafter\chardef\csname
          HoLogoOpt@#1@\HOLOGO@name\endcsname=\ltx@zero
        \fi
        \HOLOGO@SetBreakAll{#1}%
      \else
        \@PackageError{hologo}{%
          Unknown value `##1' for boolean option `#1'.\MessageBreak
          Known values are `true' and `false'%
        }\@ehc
      \fi
    \fi
  }%
}
%    \end{macrocode}
%    \end{macro}
%
%    \begin{macro}{\HOLOGO@SetBreakAll}
%    \begin{macrocode}
\def\HOLOGO@SetBreakAll#1{%
  \def\HOLOGO@temp{#1}%
  \ifx\HOLOGO@temp\HOLOGO@break
    \ifx\HOLOGO@name\relax
      \chardef\HOLOGOOPT@hyphenbreak=\HOLOGOOPT@break
      \chardef\HOLOGOOPT@spacebreak=\HOLOGOOPT@break
      \chardef\HOLOGOOPT@discretionarybreak=\HOLOGOOPT@break
    \else
      \expandafter\chardef
         \csname HoLogoOpt@hyphenbreak@\HOLOGO@name\endcsname=%
         \csname HoLogoOpt@break@\HOLOGO@name\endcsname
      \expandafter\chardef
         \csname HoLogoOpt@spacebreak@\HOLOGO@name\endcsname=%
         \csname HoLogoOpt@break@\HOLOGO@name\endcsname
      \expandafter\chardef
         \csname HoLogoOpt@discretionarybreak@\HOLOGO@name
             \endcsname=%
         \csname HoLogoOpt@break@\HOLOGO@name\endcsname
    \fi
  \fi
}
%    \end{macrocode}
%    \end{macro}
%
%    \begin{macro}{\HOLOGO@true}
%    \begin{macrocode}
\def\HOLOGO@true{true}
%    \end{macrocode}
%    \end{macro}
%    \begin{macro}{\HOLOGO@false}
%    \begin{macrocode}
\def\HOLOGO@false{false}
%    \end{macrocode}
%    \end{macro}
%    \begin{macro}{\HOLOGO@break}
%    \begin{macrocode}
\def\HOLOGO@break{break}
%    \end{macrocode}
%    \end{macro}
%
%    \begin{macrocode}
\HOLOGO@DeclareBoolOption{break}
\HOLOGO@DeclareBoolOption{hyphenbreak}
\HOLOGO@DeclareBoolOption{spacebreak}
\HOLOGO@DeclareBoolOption{discretionarybreak}
%    \end{macrocode}
%
%    \begin{macrocode}
\kv@define@key{HoLogo}{variant}{%
  \ifx\HOLOGO@name\relax
    \@PackageError{hologo}{%
      Option `variant' is not available in \string\hologoSetup,%
      \MessageBreak
      Use \string\hologoLogoSetup\space instead%
    }\@ehc
  \else
    \edef\HOLOGO@temp{#1}%
    \ifx\HOLOGO@temp\ltx@empty
      \expandafter
      \let\csname HoLogoOpt@variant@\HOLOGO@name\endcsname\@undefined
    \else
      \ltx@IfUndefined{HoLogo@\HOLOGO@name @\HOLOGO@temp}{%
        \@PackageError{hologo}{%
          Unknown variant `\HOLOGO@temp' of logo `\HOLOGO@name'%
        }\@ehc
      }{%
        \expandafter
        \let\csname HoLogoOpt@variant@\HOLOGO@name\endcsname
            \HOLOGO@temp
      }%
    \fi
  \fi
}
%    \end{macrocode}
%
%    \begin{macro}{\HOLOGO@Variant}
%    \begin{macrocode}
\def\HOLOGO@Variant#1{%
  #1%
  \ltx@ifundefined{HoLogoOpt@variant@#1}{%
  }{%
    @\csname HoLogoOpt@variant@#1\endcsname
  }%
}
%    \end{macrocode}
%    \end{macro}
%
% \subsection{Break/no-break support}
%
%    \begin{macro}{\HOLOGO@space}
%    \begin{macrocode}
\def\HOLOGO@space{%
  \ltx@ifundefined{HoLogoOpt@spacebreak@\HOLOGO@name}{%
    \ltx@ifundefined{HoLogoOpt@break@\HOLOGO@name}{%
      \chardef\HOLOGO@temp=\HOLOGOOPT@spacebreak
    }{%
      \chardef\HOLOGO@temp=%
        \csname HoLogoOpt@break@\HOLOGO@name\endcsname
    }%
  }{%
    \chardef\HOLOGO@temp=%
      \csname HoLogoOpt@spacebreak@\HOLOGO@name\endcsname
  }%
  \ifcase\HOLOGO@temp
    \penalty10000 %
  \fi
  \ltx@space
}
%    \end{macrocode}
%    \end{macro}
%
%    \begin{macro}{\HOLOGO@hyphen}
%    \begin{macrocode}
\def\HOLOGO@hyphen{%
  \ltx@ifundefined{HoLogoOpt@hyphenbreak@\HOLOGO@name}{%
    \ltx@ifundefined{HoLogoOpt@break@\HOLOGO@name}{%
      \chardef\HOLOGO@temp=\HOLOGOOPT@hyphenbreak
    }{%
      \chardef\HOLOGO@temp=%
        \csname HoLogoOpt@break@\HOLOGO@name\endcsname
    }%
  }{%
    \chardef\HOLOGO@temp=%
      \csname HoLogoOpt@hyphenbreak@\HOLOGO@name\endcsname
  }%
  \ifcase\HOLOGO@temp
    \ltx@mbox{-}%
  \else
    -%
  \fi
}
%    \end{macrocode}
%    \end{macro}
%
%    \begin{macro}{\HOLOGO@discretionary}
%    \begin{macrocode}
\def\HOLOGO@discretionary{%
  \ltx@ifundefined{HoLogoOpt@discretionarybreak@\HOLOGO@name}{%
    \ltx@ifundefined{HoLogoOpt@break@\HOLOGO@name}{%
      \chardef\HOLOGO@temp=\HOLOGOOPT@discretionarybreak
    }{%
      \chardef\HOLOGO@temp=%
        \csname HoLogoOpt@break@\HOLOGO@name\endcsname
    }%
  }{%
    \chardef\HOLOGO@temp=%
      \csname HoLogoOpt@discretionarybreak@\HOLOGO@name\endcsname
  }%
  \ifcase\HOLOGO@temp
  \else
    \-%
  \fi
}
%    \end{macrocode}
%    \end{macro}
%
%    \begin{macro}{\HOLOGO@mbox}
%    \begin{macrocode}
\def\HOLOGO@mbox#1{%
  \ltx@ifundefined{HoLogoOpt@break@\HOLOGO@name}{%
    \chardef\HOLOGO@temp=\HOLOGOOPT@hyphenbreak
  }{%
    \chardef\HOLOGO@temp=%
      \csname HoLogoOpt@break@\HOLOGO@name\endcsname
  }%
  \ifcase\HOLOGO@temp
    \ltx@mbox{#1}%
  \else
    #1%
  \fi
}
%    \end{macrocode}
%    \end{macro}
%
% \subsection{Font support}
%
%    \begin{macro}{\HoLogoFont@font}
%    \begin{tabular}{@{}ll@{}}
%    |#1|:& logo name\\
%    |#2|:& font short name\\
%    |#3|:& text
%    \end{tabular}
%    \begin{macrocode}
\def\HoLogoFont@font#1#2#3{%
  \begingroup
    \ltx@IfUndefined{HoLogoFont@logo@#1.#2}{%
      \ltx@IfUndefined{HoLogoFont@font@#2}{%
        \@PackageWarning{hologo}{%
          Missing font `#2' for logo `#1'%
        }%
        #3%
      }{%
        \csname HoLogoFont@font@#2\endcsname{#3}%
      }%
    }{%
      \csname HoLogoFont@logo@#1.#2\endcsname{#3}%
    }%
  \endgroup
}
%    \end{macrocode}
%    \end{macro}
%
%    \begin{macro}{\HoLogoFont@Def}
%    \begin{macrocode}
\def\HoLogoFont@Def#1{%
  \expandafter\def\csname HoLogoFont@font@#1\endcsname
}
%    \end{macrocode}
%    \end{macro}
%    \begin{macro}{\HoLogoFont@LogoDef}
%    \begin{macrocode}
\def\HoLogoFont@LogoDef#1#2{%
  \expandafter\def\csname HoLogoFont@logo@#1.#2\endcsname
}
%    \end{macrocode}
%    \end{macro}
%
% \subsubsection{Font defaults}
%
%    \begin{macro}{\HoLogoFont@font@general}
%    \begin{macrocode}
\HoLogoFont@Def{general}{}%
%    \end{macrocode}
%    \end{macro}
%
%    \begin{macro}{\HoLogoFont@font@rm}
%    \begin{macrocode}
\ltx@IfUndefined{rmfamily}{%
  \ltx@IfUndefined{rm}{%
  }{%
    \HoLogoFont@Def{rm}{\rm}%
  }%
}{%
  \HoLogoFont@Def{rm}{\rmfamily}%
}
%    \end{macrocode}
%    \end{macro}
%
%    \begin{macro}{\HoLogoFont@font@sf}
%    \begin{macrocode}
\ltx@IfUndefined{sffamily}{%
  \ltx@IfUndefined{sf}{%
  }{%
    \HoLogoFont@Def{sf}{\sf}%
  }%
}{%
  \HoLogoFont@Def{sf}{\sffamily}%
}
%    \end{macrocode}
%    \end{macro}
%
%    \begin{macro}{\HoLogoFont@font@bibsf}
%    In case of \hologo{plainTeX} the original small caps
%    variant is used as default. In \hologo{LaTeX}
%    the definition of package \xpackage{dtklogos} \cite{dtklogos}
%    is used.
%\begin{quote}
%\begin{verbatim}
%\DeclareRobustCommand{\BibTeX}{%
%  B%
%  \kern-.05em%
%  \hbox{%
%    $\m@th$% %% force math size calculations
%    \csname S@\f@size\endcsname
%    \fontsize\sf@size\z@
%    \math@fontsfalse
%    \selectfont
%    I%
%    \kern-.025em%
%    B
%  }%
%  \kern-.08em%
%  \-%
%  \TeX
%}
%\end{verbatim}
%\end{quote}
%    \begin{macrocode}
\ltx@IfUndefined{selectfont}{%
  \ltx@IfUndefined{tensc}{%
    \font\tensc=cmcsc10\relax
  }{}%
  \HoLogoFont@Def{bibsf}{\tensc}%
}{%
  \HoLogoFont@Def{bibsf}{%
    $\mathsurround=0pt$%
    \csname S@\f@size\endcsname
    \fontsize\sf@size{0pt}%
    \math@fontsfalse
    \selectfont
  }%
}
%    \end{macrocode}
%    \end{macro}
%
%    \begin{macro}{\HoLogoFont@font@sc}
%    \begin{macrocode}
\ltx@IfUndefined{scshape}{%
  \ltx@IfUndefined{tensc}{%
    \font\tensc=cmcsc10\relax
  }{}%
  \HoLogoFont@Def{sc}{\tensc}%
}{%
  \HoLogoFont@Def{sc}{\scshape}%
}
%    \end{macrocode}
%    \end{macro}
%
%    \begin{macro}{\HoLogoFont@font@sy}
%    \begin{macrocode}
\ltx@IfUndefined{usefont}{%
  \ltx@IfUndefined{tensy}{%
  }{%
    \HoLogoFont@Def{sy}{\tensy}%
  }%
}{%
  \HoLogoFont@Def{sy}{%
    \usefont{OMS}{cmsy}{m}{n}%
  }%
}
%    \end{macrocode}
%    \end{macro}
%
%    \begin{macro}{\HoLogoFont@font@logo}
%    \begin{macrocode}
\begingroup
  \def\x{LaTeX2e}%
\expandafter\endgroup
\ifx\fmtname\x
  \ltx@IfUndefined{logofamily}{%
    \DeclareRobustCommand\logofamily{%
      \not@math@alphabet\logofamily\relax
      \fontencoding{U}%
      \fontfamily{logo}%
      \selectfont
    }%
  }{}%
  \ltx@IfUndefined{logofamily}{%
  }{%
    \HoLogoFont@Def{logo}{\logofamily}%
  }%
\else
  \ltx@IfUndefined{tenlogo}{%
    \font\tenlogo=logo10\relax
  }{}%
  \HoLogoFont@Def{logo}{\tenlogo}%
\fi
%    \end{macrocode}
%    \end{macro}
%
% \subsubsection{Font setup}
%
%    \begin{macro}{\hologoFontSetup}
%    \begin{macrocode}
\def\hologoFontSetup{%
  \let\HOLOGO@name\relax
  \HOLOGO@FontSetup
}
%    \end{macrocode}
%    \end{macro}
%
%    \begin{macro}{\hologoLogoFontSetup}
%    \begin{macrocode}
\def\hologoLogoFontSetup#1{%
  \edef\HOLOGO@name{#1}%
  \ltx@IfUndefined{HoLogo@\HOLOGO@name}{%
    \@PackageError{hologo}{%
      Unknown logo `\HOLOGO@name'%
    }\@ehc
    \ltx@gobble
  }{%
    \HOLOGO@FontSetup
  }%
}
%    \end{macrocode}
%    \end{macro}
%
%    \begin{macro}{\HOLOGO@FontSetup}
%    \begin{macrocode}
\def\HOLOGO@FontSetup{%
  \kvsetkeys{HoLogoFont}%
}
%    \end{macrocode}
%    \end{macro}
%
%    \begin{macrocode}
\def\HOLOGO@temp#1{%
  \kv@define@key{HoLogoFont}{#1}{%
    \ifx\HOLOGO@name\relax
      \HoLogoFont@Def{#1}{##1}%
    \else
      \HoLogoFont@LogoDef\HOLOGO@name{#1}{##1}%
    \fi
  }%
}
\HOLOGO@temp{general}
\HOLOGO@temp{sf}
%    \end{macrocode}
%
% \subsection{Generic logo commands}
%
%    \begin{macrocode}
\HOLOGO@IfExists\hologo{%
  \@PackageError{hologo}{%
    \string\hologo\ltx@space is already defined.\MessageBreak
    Package loading is aborted%
  }\@ehc
  \HOLOGO@AtEnd
}%
\HOLOGO@IfExists\hologoRobust{%
  \@PackageError{hologo}{%
    \string\hologoRobust\ltx@space is already defined.\MessageBreak
    Package loading is aborted%
  }\@ehc
  \HOLOGO@AtEnd
}%
%    \end{macrocode}
%
% \subsubsection{\cs{hologo} and friends}
%
%    \begin{macrocode}
\ifluatex
  \expandafter\ltx@firstofone
\else
  \expandafter\ltx@gobble
\fi
{%
  \ltx@IfUndefined{ifincsname}{%
    \ifnum\luatexversion<36 %
      \expandafter\ltx@gobble
    \else
      \expandafter\ltx@firstofone
    \fi
    {%
      \begingroup
        \ifcase0%
            \directlua{%
              if tex.enableprimitives then %
                tex.enableprimitives('HOLOGO@', {'ifincsname'})%
              else %
                tex.print('1')%
              end%
            }%
            \ifx\HOLOGO@ifincsname\@undefined 1\fi%
            \relax
          \expandafter\ltx@firstofone
        \else
          \endgroup
          \expandafter\ltx@gobble
        \fi
        {%
          \global\let\ifincsname\HOLOGO@ifincsname
        }%
      \HOLOGO@temp
    }%
  }{}%
}
%    \end{macrocode}
%    \begin{macrocode}
\ltx@IfUndefined{ifincsname}{%
  \catcode`$=14 %
}{%
  \catcode`$=9 %
}
%    \end{macrocode}
%
%    \begin{macro}{\hologo}
%    \begin{macrocode}
\def\hologo#1{%
$ \ifincsname
$   \ltx@ifundefined{HoLogoCs@\HOLOGO@Variant{#1}}{%
$     #1%
$   }{%
$     \csname HoLogoCs@\HOLOGO@Variant{#1}\endcsname\ltx@firstoftwo
$   }%
$ \else
    \HOLOGO@IfExists\texorpdfstring\texorpdfstring\ltx@firstoftwo
    {%
      \hologoRobust{#1}%
    }{%
      \ltx@ifundefined{HoLogoBkm@\HOLOGO@Variant{#1}}{%
        \ltx@ifundefined{HoLogo@#1}{?#1?}{#1}%
      }{%
        \csname HoLogoBkm@\HOLOGO@Variant{#1}\endcsname
        \ltx@firstoftwo
      }%
    }%
$ \fi
}
%    \end{macrocode}
%    \end{macro}
%    \begin{macro}{\Hologo}
%    \begin{macrocode}
\def\Hologo#1{%
$ \ifincsname
$   \ltx@ifundefined{HoLogoCs@\HOLOGO@Variant{#1}}{%
$     #1%
$   }{%
$     \csname HoLogoCs@\HOLOGO@Variant{#1}\endcsname\ltx@secondoftwo
$   }%
$ \else
    \HOLOGO@IfExists\texorpdfstring\texorpdfstring\ltx@firstoftwo
    {%
      \HologoRobust{#1}%
    }{%
      \ltx@ifundefined{HoLogoBkm@\HOLOGO@Variant{#1}}{%
        \ltx@ifundefined{HoLogo@#1}{?#1?}{#1}%
      }{%
        \csname HoLogoBkm@\HOLOGO@Variant{#1}\endcsname
        \ltx@secondoftwo
      }%
    }%
$ \fi
}
%    \end{macrocode}
%    \end{macro}
%
%    \begin{macro}{\hologoVariant}
%    \begin{macrocode}
\def\hologoVariant#1#2{%
  \ifx\relax#2\relax
    \hologo{#1}%
  \else
$   \ifincsname
$     \ltx@ifundefined{HoLogoCs@#1@#2}{%
$       #1%
$     }{%
$       \csname HoLogoCs@#1@#2\endcsname\ltx@firstoftwo
$     }%
$   \else
      \HOLOGO@IfExists\texorpdfstring\texorpdfstring\ltx@firstoftwo
      {%
        \hologoVariantRobust{#1}{#2}%
      }{%
        \ltx@ifundefined{HoLogoBkm@#1@#2}{%
          \ltx@ifundefined{HoLogo@#1}{?#1?}{#1}%
        }{%
          \csname HoLogoBkm@#1@#2\endcsname
          \ltx@firstoftwo
        }%
      }%
$   \fi
  \fi
}
%    \end{macrocode}
%    \end{macro}
%    \begin{macro}{\HologoVariant}
%    \begin{macrocode}
\def\HologoVariant#1#2{%
  \ifx\relax#2\relax
    \Hologo{#1}%
  \else
$   \ifincsname
$     \ltx@ifundefined{HoLogoCs@#1@#2}{%
$       #1%
$     }{%
$       \csname HoLogoCs@#1@#2\endcsname\ltx@secondoftwo
$     }%
$   \else
      \HOLOGO@IfExists\texorpdfstring\texorpdfstring\ltx@firstoftwo
      {%
        \HologoVariantRobust{#1}{#2}%
      }{%
        \ltx@ifundefined{HoLogoBkm@#1@#2}{%
          \ltx@ifundefined{HoLogo@#1}{?#1?}{#1}%
        }{%
          \csname HoLogoBkm@#1@#2\endcsname
          \ltx@secondoftwo
        }%
      }%
$   \fi
  \fi
}
%    \end{macrocode}
%    \end{macro}
%
%    \begin{macrocode}
\catcode`\$=3 %
%    \end{macrocode}
%
% \subsubsection{\cs{hologoRobust} and friends}
%
%    \begin{macro}{\hologoRobust}
%    \begin{macrocode}
\ltx@IfUndefined{protected}{%
  \ltx@IfUndefined{DeclareRobustCommand}{%
    \def\hologoRobust#1%
  }{%
    \DeclareRobustCommand*\hologoRobust[1]%
  }%
}{%
  \protected\def\hologoRobust#1%
}%
{%
  \edef\HOLOGO@name{#1}%
  \ltx@IfUndefined{HoLogo@\HOLOGO@Variant\HOLOGO@name}{%
    \@PackageError{hologo}{%
      Unknown logo `\HOLOGO@name'%
    }\@ehc
    ?\HOLOGO@name?%
  }{%
    \ltx@IfUndefined{ver@tex4ht.sty}{%
      \HoLogoFont@font\HOLOGO@name{general}{%
        \csname HoLogo@\HOLOGO@Variant\HOLOGO@name\endcsname
        \ltx@firstoftwo
      }%
    }{%
      \ltx@IfUndefined{HoLogoHtml@\HOLOGO@Variant\HOLOGO@name}{%
        \HOLOGO@name
      }{%
        \csname HoLogoHtml@\HOLOGO@Variant\HOLOGO@name\endcsname
        \ltx@firstoftwo
      }%
    }%
  }%
}
%    \end{macrocode}
%    \end{macro}
%    \begin{macro}{\HologoRobust}
%    \begin{macrocode}
\ltx@IfUndefined{protected}{%
  \ltx@IfUndefined{DeclareRobustCommand}{%
    \def\HologoRobust#1%
  }{%
    \DeclareRobustCommand*\HologoRobust[1]%
  }%
}{%
  \protected\def\HologoRobust#1%
}%
{%
  \edef\HOLOGO@name{#1}%
  \ltx@IfUndefined{HoLogo@\HOLOGO@Variant\HOLOGO@name}{%
    \@PackageError{hologo}{%
      Unknown logo `\HOLOGO@name'%
    }\@ehc
    ?\HOLOGO@name?%
  }{%
    \ltx@IfUndefined{ver@tex4ht.sty}{%
      \HoLogoFont@font\HOLOGO@name{general}{%
        \csname HoLogo@\HOLOGO@Variant\HOLOGO@name\endcsname
        \ltx@secondoftwo
      }%
    }{%
      \ltx@IfUndefined{HoLogoHtml@\HOLOGO@Variant\HOLOGO@name}{%
        \expandafter\HOLOGO@Uppercase\HOLOGO@name
      }{%
        \csname HoLogoHtml@\HOLOGO@Variant\HOLOGO@name\endcsname
        \ltx@secondoftwo
      }%
    }%
  }%
}
%    \end{macrocode}
%    \end{macro}
%    \begin{macro}{\hologoVariantRobust}
%    \begin{macrocode}
\ltx@IfUndefined{protected}{%
  \ltx@IfUndefined{DeclareRobustCommand}{%
    \def\hologoVariantRobust#1#2%
  }{%
    \DeclareRobustCommand*\hologoVariantRobust[2]%
  }%
}{%
  \protected\def\hologoVariantRobust#1#2%
}%
{%
  \begingroup
    \hologoLogoSetup{#1}{variant={#2}}%
    \hologoRobust{#1}%
  \endgroup
}
%    \end{macrocode}
%    \end{macro}
%    \begin{macro}{\HologoVariantRobust}
%    \begin{macrocode}
\ltx@IfUndefined{protected}{%
  \ltx@IfUndefined{DeclareRobustCommand}{%
    \def\HologoVariantRobust#1#2%
  }{%
    \DeclareRobustCommand*\HologoVariantRobust[2]%
  }%
}{%
  \protected\def\HologoVariantRobust#1#2%
}%
{%
  \begingroup
    \hologoLogoSetup{#1}{variant={#2}}%
    \HologoRobust{#1}%
  \endgroup
}
%    \end{macrocode}
%    \end{macro}
%
%    \begin{macro}{\hologorobust}
%    Macro \cs{hologorobust} is only defined for compatibility.
%    Its use is deprecated.
%    \begin{macrocode}
\def\hologorobust{\hologoRobust}
%    \end{macrocode}
%    \end{macro}
%
% \subsection{Helpers}
%
%    \begin{macro}{\HOLOGO@Uppercase}
%    Macro \cs{HOLOGO@Uppercase} is restricted to \cs{uppercase},
%    because \hologo{plainTeX} or \hologo{iniTeX} do not provide
%    \cs{MakeUppercase}.
%    \begin{macrocode}
\def\HOLOGO@Uppercase#1{\uppercase{#1}}
%    \end{macrocode}
%    \end{macro}
%
%    \begin{macro}{\HOLOGO@PdfdocUnicode}
%    \begin{macrocode}
\def\HOLOGO@PdfdocUnicode{%
  \ifx\ifHy@unicode\iftrue
    \expandafter\ltx@secondoftwo
  \else
    \expandafter\ltx@firstoftwo
  \fi
}
%    \end{macrocode}
%    \end{macro}
%
%    \begin{macro}{\HOLOGO@Math}
%    \begin{macrocode}
\def\HOLOGO@MathSetup{%
  \mathsurround0pt\relax
  \HOLOGO@IfExists\f@series{%
    \if b\expandafter\ltx@car\f@series x\@nil
      \csname boldmath\endcsname
   \fi
  }{}%
}
%    \end{macrocode}
%    \end{macro}
%
%    \begin{macro}{\HOLOGO@TempDimen}
%    \begin{macrocode}
\dimendef\HOLOGO@TempDimen=\ltx@zero
%    \end{macrocode}
%    \end{macro}
%    \begin{macro}{\HOLOGO@NegativeKerning}
%    \begin{macrocode}
\def\HOLOGO@NegativeKerning#1{%
  \begingroup
    \HOLOGO@TempDimen=0pt\relax
    \comma@parse@normalized{#1}{%
      \ifdim\HOLOGO@TempDimen=0pt %
        \expandafter\HOLOGO@@NegativeKerning\comma@entry
      \fi
      \ltx@gobble
    }%
    \ifdim\HOLOGO@TempDimen<0pt %
      \kern\HOLOGO@TempDimen
    \fi
  \endgroup
}
%    \end{macrocode}
%    \end{macro}
%    \begin{macro}{\HOLOGO@@NegativeKerning}
%    \begin{macrocode}
\def\HOLOGO@@NegativeKerning#1#2{%
  \setbox\ltx@zero\hbox{#1#2}%
  \HOLOGO@TempDimen=\wd\ltx@zero
  \setbox\ltx@zero\hbox{#1\kern0pt#2}%
  \advance\HOLOGO@TempDimen by -\wd\ltx@zero
}
%    \end{macrocode}
%    \end{macro}
%
%    \begin{macro}{\HOLOGO@SpaceFactor}
%    \begin{macrocode}
\def\HOLOGO@SpaceFactor{%
  \spacefactor1000 %
}
%    \end{macrocode}
%    \end{macro}
%
%    \begin{macro}{\HOLOGO@Span}
%    \begin{macrocode}
\def\HOLOGO@Span#1#2{%
  \HCode{<span class="HoLogo-#1">}%
  #2%
  \HCode{</span>}%
}
%    \end{macrocode}
%    \end{macro}
%
% \subsubsection{Text subscript}
%
%    \begin{macro}{\HOLOGO@SubScript}%
%    \begin{macrocode}
\def\HOLOGO@SubScript#1{%
  \ltx@IfUndefined{textsubscript}{%
    \ltx@IfUndefined{text}{%
      \ltx@mbox{%
        \mathsurround=0pt\relax
        $%
          _{%
            \ltx@IfUndefined{sf@size}{%
              \mathrm{#1}%
            }{%
              \mbox{%
                \fontsize\sf@size{0pt}\selectfont
                #1%
              }%
            }%
          }%
        $%
      }%
    }{%
      \ltx@mbox{%
        \mathsurround=0pt\relax
        $_{\text{#1}}$%
      }%
    }%
  }{%
    \textsubscript{#1}%
  }%
}
%    \end{macrocode}
%    \end{macro}
%
% \subsection{\hologo{TeX} and friends}
%
% \subsubsection{\hologo{TeX}}
%
%    \begin{macro}{\HoLogo@TeX}
%    Source: \hologo{LaTeX} kernel.
%    \begin{macrocode}
\def\HoLogo@TeX#1{%
  T\kern-.1667em\lower.5ex\hbox{E}\kern-.125emX\HOLOGO@SpaceFactor
}
%    \end{macrocode}
%    \end{macro}
%    \begin{macro}{\HoLogoHtml@TeX}
%    \begin{macrocode}
\def\HoLogoHtml@TeX#1{%
  \HoLogoCss@TeX
  \HOLOGO@Span{TeX}{%
    T%
    \HOLOGO@Span{e}{%
      E%
    }%
    X%
  }%
}
%    \end{macrocode}
%    \end{macro}
%    \begin{macro}{\HoLogoCss@TeX}
%    \begin{macrocode}
\def\HoLogoCss@TeX{%
  \Css{%
    span.HoLogo-TeX span.HoLogo-e{%
      position:relative;%
      top:.5ex;%
      margin-left:-.1667em;%
      margin-right:-.125em;%
    }%
  }%
  \Css{%
    a span.HoLogo-TeX span.HoLogo-e{%
      text-decoration:none;%
    }%
  }%
  \global\let\HoLogoCss@TeX\relax
}
%    \end{macrocode}
%    \end{macro}
%
% \subsubsection{\hologo{plainTeX}}
%
%    \begin{macro}{\HoLogo@plainTeX@space}
%    Source: ``The \hologo{TeX}book''
%    \begin{macrocode}
\def\HoLogo@plainTeX@space#1{%
  \HOLOGO@mbox{#1{p}{P}lain}\HOLOGO@space\hologo{TeX}%
}
%    \end{macrocode}
%    \end{macro}
%    \begin{macro}{\HoLogoCs@plainTeX@space}
%    \begin{macrocode}
\def\HoLogoCs@plainTeX@space#1{#1{p}{P}lain TeX}%
%    \end{macrocode}
%    \end{macro}
%    \begin{macro}{\HoLogoBkm@plainTeX@space}
%    \begin{macrocode}
\def\HoLogoBkm@plainTeX@space#1{%
  #1{p}{P}lain \hologo{TeX}%
}
%    \end{macrocode}
%    \end{macro}
%    \begin{macro}{\HoLogoHtml@plainTeX@space}
%    \begin{macrocode}
\def\HoLogoHtml@plainTeX@space#1{%
  #1{p}{P}lain \hologo{TeX}%
}
%    \end{macrocode}
%    \end{macro}
%
%    \begin{macro}{\HoLogo@plainTeX@hyphen}
%    \begin{macrocode}
\def\HoLogo@plainTeX@hyphen#1{%
  \HOLOGO@mbox{#1{p}{P}lain}\HOLOGO@hyphen\hologo{TeX}%
}
%    \end{macrocode}
%    \end{macro}
%    \begin{macro}{\HoLogoCs@plainTeX@hyphen}
%    \begin{macrocode}
\def\HoLogoCs@plainTeX@hyphen#1{#1{p}{P}lain-TeX}
%    \end{macrocode}
%    \end{macro}
%    \begin{macro}{\HoLogoBkm@plainTeX@hyphen}
%    \begin{macrocode}
\def\HoLogoBkm@plainTeX@hyphen#1{%
  #1{p}{P}lain-\hologo{TeX}%
}
%    \end{macrocode}
%    \end{macro}
%    \begin{macro}{\HoLogoHtml@plainTeX@hyphen}
%    \begin{macrocode}
\def\HoLogoHtml@plainTeX@hyphen#1{%
  #1{p}{P}lain-\hologo{TeX}%
}
%    \end{macrocode}
%    \end{macro}
%
%    \begin{macro}{\HoLogo@plainTeX@runtogether}
%    \begin{macrocode}
\def\HoLogo@plainTeX@runtogether#1{%
  \HOLOGO@mbox{#1{p}{P}lain\hologo{TeX}}%
}
%    \end{macrocode}
%    \end{macro}
%    \begin{macro}{\HoLogoCs@plainTeX@runtogether}
%    \begin{macrocode}
\def\HoLogoCs@plainTeX@runtogether#1{#1{p}{P}lainTeX}
%    \end{macrocode}
%    \end{macro}
%    \begin{macro}{\HoLogoBkm@plainTeX@runtogether}
%    \begin{macrocode}
\def\HoLogoBkm@plainTeX@runtogether#1{%
  #1{p}{P}lain\hologo{TeX}%
}
%    \end{macrocode}
%    \end{macro}
%    \begin{macro}{\HoLogoHtml@plainTeX@runtogether}
%    \begin{macrocode}
\def\HoLogoHtml@plainTeX@runtogether#1{%
  #1{p}{P}lain\hologo{TeX}%
}
%    \end{macrocode}
%    \end{macro}
%
%    \begin{macro}{\HoLogo@plainTeX}
%    \begin{macrocode}
\def\HoLogo@plainTeX{\HoLogo@plainTeX@space}
%    \end{macrocode}
%    \end{macro}
%    \begin{macro}{\HoLogoCs@plainTeX}
%    \begin{macrocode}
\def\HoLogoCs@plainTeX{\HoLogoCs@plainTeX@space}
%    \end{macrocode}
%    \end{macro}
%    \begin{macro}{\HoLogoBkm@plainTeX}
%    \begin{macrocode}
\def\HoLogoBkm@plainTeX{\HoLogoBkm@plainTeX@space}
%    \end{macrocode}
%    \end{macro}
%    \begin{macro}{\HoLogoHtml@plainTeX}
%    \begin{macrocode}
\def\HoLogoHtml@plainTeX{\HoLogoHtml@plainTeX@space}
%    \end{macrocode}
%    \end{macro}
%
% \subsubsection{\hologo{LaTeX}}
%
%    Source: \hologo{LaTeX} kernel.
%\begin{quote}
%\begin{verbatim}
%\DeclareRobustCommand{\LaTeX}{%
%  L%
%  \kern-.36em%
%  {%
%    \sbox\z@ T%
%    \vbox to\ht\z@{%
%      \hbox{%
%        \check@mathfonts
%        \fontsize\sf@size\z@
%        \math@fontsfalse
%        \selectfont
%        A%
%      }%
%      \vss
%    }%
%  }%
%  \kern-.15em%
%  \TeX
%}
%\end{verbatim}
%\end{quote}
%
%    \begin{macro}{\HoLogo@La}
%    \begin{macrocode}
\def\HoLogo@La#1{%
  L%
  \kern-.36em%
  \begingroup
    \setbox\ltx@zero\hbox{T}%
    \vbox to\ht\ltx@zero{%
      \hbox{%
        \ltx@ifundefined{check@mathfonts}{%
          \csname sevenrm\endcsname
        }{%
          \check@mathfonts
          \fontsize\sf@size{0pt}%
          \math@fontsfalse\selectfont
        }%
        A%
      }%
      \vss
    }%
  \endgroup
}
%    \end{macrocode}
%    \end{macro}
%
%    \begin{macro}{\HoLogo@LaTeX}
%    Source: \hologo{LaTeX} kernel.
%    \begin{macrocode}
\def\HoLogo@LaTeX#1{%
  \hologo{La}%
  \kern-.15em%
  \hologo{TeX}%
}
%    \end{macrocode}
%    \end{macro}
%    \begin{macro}{\HoLogoHtml@LaTeX}
%    \begin{macrocode}
\def\HoLogoHtml@LaTeX#1{%
  \HoLogoCss@LaTeX
  \HOLOGO@Span{LaTeX}{%
    L%
    \HOLOGO@Span{a}{%
      A%
    }%
    \hologo{TeX}%
  }%
}
%    \end{macrocode}
%    \end{macro}
%    \begin{macro}{\HoLogoCss@LaTeX}
%    \begin{macrocode}
\def\HoLogoCss@LaTeX{%
  \Css{%
    span.HoLogo-LaTeX span.HoLogo-a{%
      position:relative;%
      top:-.5ex;%
      margin-left:-.36em;%
      margin-right:-.15em;%
      font-size:85\%;%
    }%
  }%
  \global\let\HoLogoCss@LaTeX\relax
}
%    \end{macrocode}
%    \end{macro}
%
% \subsubsection{\hologo{(La)TeX}}
%
%    \begin{macro}{\HoLogo@LaTeXTeX}
%    The kerning around the parentheses is taken
%    from package \xpackage{dtklogos} \cite{dtklogos}.
%\begin{quote}
%\begin{verbatim}
%\DeclareRobustCommand{\LaTeXTeX}{%
%  (%
%  \kern-.15em%
%  L%
%  \kern-.36em%
%  {%
%    \sbox\z@ T%
%    \vbox to\ht0{%
%      \hbox{%
%        $\m@th$%
%        \csname S@\f@size\endcsname
%        \fontsize\sf@size\z@
%        \math@fontsfalse
%        \selectfont
%        A%
%      }%
%      \vss
%    }%
%  }%
%  \kern-.2em%
%  )%
%  \kern-.15em%
%  \TeX
%}
%\end{verbatim}
%\end{quote}
%    \begin{macrocode}
\def\HoLogo@LaTeXTeX#1{%
  (%
  \kern-.15em%
  \hologo{La}%
  \kern-.2em%
  )%
  \kern-.15em%
  \hologo{TeX}%
}
%    \end{macrocode}
%    \end{macro}
%    \begin{macro}{\HoLogoBkm@LaTeXTeX}
%    \begin{macrocode}
\def\HoLogoBkm@LaTeXTeX#1{(La)TeX}
%    \end{macrocode}
%    \end{macro}
%
%    \begin{macro}{\HoLogo@(La)TeX}
%    \begin{macrocode}
\expandafter
\let\csname HoLogo@(La)TeX\endcsname\HoLogo@LaTeXTeX
%    \end{macrocode}
%    \end{macro}
%    \begin{macro}{\HoLogoBkm@(La)TeX}
%    \begin{macrocode}
\expandafter
\let\csname HoLogoBkm@(La)TeX\endcsname\HoLogoBkm@LaTeXTeX
%    \end{macrocode}
%    \end{macro}
%    \begin{macro}{\HoLogoHtml@LaTeXTeX}
%    \begin{macrocode}
\def\HoLogoHtml@LaTeXTeX#1{%
  \HoLogoCss@LaTeXTeX
  \HOLOGO@Span{LaTeXTeX}{%
    (%
    \HOLOGO@Span{L}{L}%
    \HOLOGO@Span{a}{A}%
    \HOLOGO@Span{ParenRight}{)}%
    \hologo{TeX}%
  }%
}
%    \end{macrocode}
%    \end{macro}
%    \begin{macro}{\HoLogoHtml@(La)TeX}
%    Kerning after opening parentheses and before closing parentheses
%    is $-0.1$\,em. The original values $-0.15$\,em
%    looked too ugly for a serif font.
%    \begin{macrocode}
\expandafter
\let\csname HoLogoHtml@(La)TeX\endcsname\HoLogoHtml@LaTeXTeX
%    \end{macrocode}
%    \end{macro}
%    \begin{macro}{\HoLogoCss@LaTeXTeX}
%    \begin{macrocode}
\def\HoLogoCss@LaTeXTeX{%
  \Css{%
    span.HoLogo-LaTeXTeX span.HoLogo-L{%
      margin-left:-.1em;%
    }%
  }%
  \Css{%
    span.HoLogo-LaTeXTeX span.HoLogo-a{%
      position:relative;%
      top:-.5ex;%
      margin-left:-.36em;%
      margin-right:-.1em;%
      font-size:85\%;%
    }%
  }%
  \Css{%
    span.HoLogo-LaTeXTeX span.HoLogo-ParenRight{%
      margin-right:-.15em;%
    }%
  }%
  \global\let\HoLogoCss@LaTeXTeX\relax
}
%    \end{macrocode}
%    \end{macro}
%
% \subsubsection{\hologo{LaTeXe}}
%
%    \begin{macro}{\HoLogo@LaTeXe}
%    Source: \hologo{LaTeX} kernel
%    \begin{macrocode}
\def\HoLogo@LaTeXe#1{%
  \hologo{LaTeX}%
  \kern.15em%
  \hbox{%
    \HOLOGO@MathSetup
    2%
    $_{\textstyle\varepsilon}$%
  }%
}
%    \end{macrocode}
%    \end{macro}
%
%    \begin{macro}{\HoLogoCs@LaTeXe}
%    \begin{macrocode}
\ifnum64=`\^^^^0040\relax % test for big chars of LuaTeX/XeTeX
  \catcode`\$=9 %
  \catcode`\&=14 %
\else
  \catcode`\$=14 %
  \catcode`\&=9 %
\fi
\def\HoLogoCs@LaTeXe#1{%
  LaTeX2%
$ \string ^^^^0395%
& e%
}%
\catcode`\$=3 %
\catcode`\&=4 %
%    \end{macrocode}
%    \end{macro}
%
%    \begin{macro}{\HoLogoBkm@LaTeXe}
%    \begin{macrocode}
\def\HoLogoBkm@LaTeXe#1{%
  \hologo{LaTeX}%
  2%
  \HOLOGO@PdfdocUnicode{e}{\textepsilon}%
}
%    \end{macrocode}
%    \end{macro}
%
%    \begin{macro}{\HoLogoHtml@LaTeXe}
%    \begin{macrocode}
\def\HoLogoHtml@LaTeXe#1{%
  \HoLogoCss@LaTeXe
  \HOLOGO@Span{LaTeX2e}{%
    \hologo{LaTeX}%
    \HOLOGO@Span{2}{2}%
    \HOLOGO@Span{e}{%
      \HOLOGO@MathSetup
      \ensuremath{\textstyle\varepsilon}%
    }%
  }%
}
%    \end{macrocode}
%    \end{macro}
%    \begin{macro}{\HoLogoCss@LaTeXe}
%    \begin{macrocode}
\def\HoLogoCss@LaTeXe{%
  \Css{%
    span.HoLogo-LaTeX2e span.HoLogo-2{%
      padding-left:.15em;%
    }%
  }%
  \Css{%
    span.HoLogo-LaTeX2e span.HoLogo-e{%
      position:relative;%
      top:.35ex;%
      text-decoration:none;%
    }%
  }%
  \global\let\HoLogoCss@LaTeXe\relax
}
%    \end{macrocode}
%    \end{macro}
%
%    \begin{macro}{\HoLogo@LaTeX2e}
%    \begin{macrocode}
\expandafter
\let\csname HoLogo@LaTeX2e\endcsname\HoLogo@LaTeXe
%    \end{macrocode}
%    \end{macro}
%    \begin{macro}{\HoLogoCs@LaTeX2e}
%    \begin{macrocode}
\expandafter
\let\csname HoLogoCs@LaTeX2e\endcsname\HoLogoCs@LaTeXe
%    \end{macrocode}
%    \end{macro}
%    \begin{macro}{\HoLogoBkm@LaTeX2e}
%    \begin{macrocode}
\expandafter
\let\csname HoLogoBkm@LaTeX2e\endcsname\HoLogoBkm@LaTeXe
%    \end{macrocode}
%    \end{macro}
%    \begin{macro}{\HoLogoHtml@LaTeX2e}
%    \begin{macrocode}
\expandafter
\let\csname HoLogoHtml@LaTeX2e\endcsname\HoLogoHtml@LaTeXe
%    \end{macrocode}
%    \end{macro}
%
% \subsubsection{\hologo{LaTeX3}}
%
%    \begin{macro}{\HoLogo@LaTeX3}
%    Source: \hologo{LaTeX} kernel
%    \begin{macrocode}
\expandafter\def\csname HoLogo@LaTeX3\endcsname#1{%
  \hologo{LaTeX}%
  3%
}
%    \end{macrocode}
%    \end{macro}
%
%    \begin{macro}{\HoLogoBkm@LaTeX3}
%    \begin{macrocode}
\expandafter\def\csname HoLogoBkm@LaTeX3\endcsname#1{%
  \hologo{LaTeX}%
  3%
}
%    \end{macrocode}
%    \end{macro}
%    \begin{macro}{\HoLogoHtml@LaTeX3}
%    \begin{macrocode}
\expandafter
\let\csname HoLogoHtml@LaTeX3\expandafter\endcsname
\csname HoLogo@LaTeX3\endcsname
%    \end{macrocode}
%    \end{macro}
%
% \subsubsection{\hologo{LaTeXML}}
%
%    \begin{macro}{\HoLogo@LaTeXML}
%    \begin{macrocode}
\def\HoLogo@LaTeXML#1{%
  \HOLOGO@mbox{%
    \hologo{La}%
    \kern-.15em%
    T%
    \kern-.1667em%
    \lower.5ex\hbox{E}%
    \kern-.125em%
    \HoLogoFont@font{LaTeXML}{sc}{xml}%
  }%
}
%    \end{macrocode}
%    \end{macro}
%    \begin{macro}{\HoLogoHtml@pdfLaTeX}
%    \begin{macrocode}
\def\HoLogoHtml@LaTeXML#1{%
  \HOLOGO@Span{LaTeXML}{%
    \HoLogoCss@LaTeX
    \HoLogoCss@TeX
    \HOLOGO@Span{LaTeX}{%
      L%
      \HOLOGO@Span{a}{%
        A%
      }%
    }%
    \HOLOGO@Span{TeX}{%
      T%
      \HOLOGO@Span{e}{%
        E%
      }%
    }%
    \HCode{<span style="font-variant: small-caps;">}%
    xml%
    \HCode{</span>}%
  }%
}
%    \end{macrocode}
%    \end{macro}
%
% \subsubsection{\hologo{eTeX}}
%
%    \begin{macro}{\HoLogo@eTeX}
%    Source: package \xpackage{etex}
%    \begin{macrocode}
\def\HoLogo@eTeX#1{%
  \ltx@mbox{%
    \HOLOGO@MathSetup
    $\varepsilon$%
    -%
    \HOLOGO@NegativeKerning{-T,T-,To}%
    \hologo{TeX}%
  }%
}
%    \end{macrocode}
%    \end{macro}
%    \begin{macro}{\HoLogoCs@eTeX}
%    \begin{macrocode}
\ifnum64=`\^^^^0040\relax % test for big chars of LuaTeX/XeTeX
  \catcode`\$=9 %
  \catcode`\&=14 %
\else
  \catcode`\$=14 %
  \catcode`\&=9 %
\fi
\def\HoLogoCs@eTeX#1{%
$ #1{\string ^^^^0395}{\string ^^^^03b5}%
& #1{e}{E}%
  TeX%
}%
\catcode`\$=3 %
\catcode`\&=4 %
%    \end{macrocode}
%    \end{macro}
%    \begin{macro}{\HoLogoBkm@eTeX}
%    \begin{macrocode}
\def\HoLogoBkm@eTeX#1{%
  \HOLOGO@PdfdocUnicode{#1{e}{E}}{\textepsilon}%
  -%
  \hologo{TeX}%
}
%    \end{macrocode}
%    \end{macro}
%    \begin{macro}{\HoLogoHtml@eTeX}
%    \begin{macrocode}
\def\HoLogoHtml@eTeX#1{%
  \ltx@mbox{%
    \HOLOGO@MathSetup
    $\varepsilon$%
    -%
    \hologo{TeX}%
  }%
}
%    \end{macrocode}
%    \end{macro}
%
% \subsubsection{\hologo{iniTeX}}
%
%    \begin{macro}{\HoLogo@iniTeX}
%    \begin{macrocode}
\def\HoLogo@iniTeX#1{%
  \HOLOGO@mbox{%
    #1{i}{I}ni\hologo{TeX}%
  }%
}
%    \end{macrocode}
%    \end{macro}
%    \begin{macro}{\HoLogoCs@iniTeX}
%    \begin{macrocode}
\def\HoLogoCs@iniTeX#1{#1{i}{I}niTeX}
%    \end{macrocode}
%    \end{macro}
%    \begin{macro}{\HoLogoBkm@iniTeX}
%    \begin{macrocode}
\def\HoLogoBkm@iniTeX#1{%
  #1{i}{I}ni\hologo{TeX}%
}
%    \end{macrocode}
%    \end{macro}
%    \begin{macro}{\HoLogoHtml@iniTeX}
%    \begin{macrocode}
\let\HoLogoHtml@iniTeX\HoLogo@iniTeX
%    \end{macrocode}
%    \end{macro}
%
% \subsubsection{\hologo{virTeX}}
%
%    \begin{macro}{\HoLogo@virTeX}
%    \begin{macrocode}
\def\HoLogo@virTeX#1{%
  \HOLOGO@mbox{%
    #1{v}{V}ir\hologo{TeX}%
  }%
}
%    \end{macrocode}
%    \end{macro}
%    \begin{macro}{\HoLogoCs@virTeX}
%    \begin{macrocode}
\def\HoLogoCs@virTeX#1{#1{v}{V}irTeX}
%    \end{macrocode}
%    \end{macro}
%    \begin{macro}{\HoLogoBkm@virTeX}
%    \begin{macrocode}
\def\HoLogoBkm@virTeX#1{%
  #1{v}{V}ir\hologo{TeX}%
}
%    \end{macrocode}
%    \end{macro}
%    \begin{macro}{\HoLogoHtml@virTeX}
%    \begin{macrocode}
\let\HoLogoHtml@virTeX\HoLogo@virTeX
%    \end{macrocode}
%    \end{macro}
%
% \subsubsection{\hologo{SliTeX}}
%
% \paragraph{Definitions of the three variants.}
%
%    \begin{macro}{\HoLogo@SLiTeX@lift}
%    \begin{macrocode}
\def\HoLogo@SLiTeX@lift#1{%
  \HoLogoFont@font{SliTeX}{rm}{%
    S%
    \kern-.06em%
    L%
    \kern-.18em%
    \raise.32ex\hbox{\HoLogoFont@font{SliTeX}{sc}{i}}%
    \HOLOGO@discretionary
    \kern-.06em%
    \hologo{TeX}%
  }%
}
%    \end{macrocode}
%    \end{macro}
%    \begin{macro}{\HoLogoBkm@SLiTeX@lift}
%    \begin{macrocode}
\def\HoLogoBkm@SLiTeX@lift#1{SLiTeX}
%    \end{macrocode}
%    \end{macro}
%    \begin{macro}{\HoLogoHtml@SLiTeX@lift}
%    \begin{macrocode}
\def\HoLogoHtml@SLiTeX@lift#1{%
  \HoLogoCss@SLiTeX@lift
  \HOLOGO@Span{SLiTeX-lift}{%
    \HoLogoFont@font{SliTeX}{rm}{%
      S%
      \HOLOGO@Span{L}{L}%
      \HOLOGO@Span{i}{i}%
      \hologo{TeX}%
    }%
  }%
}
%    \end{macrocode}
%    \end{macro}
%    \begin{macro}{\HoLogoCss@SLiTeX@lift}
%    \begin{macrocode}
\def\HoLogoCss@SLiTeX@lift{%
  \Css{%
    span.HoLogo-SLiTeX-lift span.HoLogo-L{%
      margin-left:-.06em;%
      margin-right:-.18em;%
    }%
  }%
  \Css{%
    span.HoLogo-SLiTeX-lift span.HoLogo-i{%
      position:relative;%
      top:-.32ex;%
      margin-right:-.06em;%
      font-variant:small-caps;%
    }%
  }%
  \global\let\HoLogoCss@SLiTeX@lift\relax
}
%    \end{macrocode}
%    \end{macro}
%
%    \begin{macro}{\HoLogo@SliTeX@simple}
%    \begin{macrocode}
\def\HoLogo@SliTeX@simple#1{%
  \HoLogoFont@font{SliTeX}{rm}{%
    \ltx@mbox{%
      \HoLogoFont@font{SliTeX}{sc}{Sli}%
    }%
    \HOLOGO@discretionary
    \hologo{TeX}%
  }%
}
%    \end{macrocode}
%    \end{macro}
%    \begin{macro}{\HoLogoBkm@SliTeX@simple}
%    \begin{macrocode}
\def\HoLogoBkm@SliTeX@simple#1{SliTeX}
%    \end{macrocode}
%    \end{macro}
%    \begin{macro}{\HoLogoHtml@SliTeX@simple}
%    \begin{macrocode}
\let\HoLogoHtml@SliTeX@simple\HoLogo@SliTeX@simple
%    \end{macrocode}
%    \end{macro}
%
%    \begin{macro}{\HoLogo@SliTeX@narrow}
%    \begin{macrocode}
\def\HoLogo@SliTeX@narrow#1{%
  \HoLogoFont@font{SliTeX}{rm}{%
    \ltx@mbox{%
      S%
      \kern-.06em%
      \HoLogoFont@font{SliTeX}{sc}{%
        l%
        \kern-.035em%
        i%
      }%
    }%
    \HOLOGO@discretionary
    \kern-.06em%
    \hologo{TeX}%
  }%
}
%    \end{macrocode}
%    \end{macro}
%    \begin{macro}{\HoLogoBkm@SliTeX@narrow}
%    \begin{macrocode}
\def\HoLogoBkm@SliTeX@narrow#1{SliTeX}
%    \end{macrocode}
%    \end{macro}
%    \begin{macro}{\HoLogoHtml@SliTeX@narrow}
%    \begin{macrocode}
\def\HoLogoHtml@SliTeX@narrow#1{%
  \HoLogoCss@SliTeX@narrow
  \HOLOGO@Span{SliTeX-narrow}{%
    \HoLogoFont@font{SliTeX}{rm}{%
      S%
        \HOLOGO@Span{l}{l}%
        \HOLOGO@Span{i}{i}%
      \hologo{TeX}%
    }%
  }%
}
%    \end{macrocode}
%    \end{macro}
%    \begin{macro}{\HoLogoCss@SliTeX@narrow}
%    \begin{macrocode}
\def\HoLogoCss@SliTeX@narrow{%
  \Css{%
    span.HoLogo-SliTeX-narrow span.HoLogo-l{%
      margin-left:-.06em;%
      margin-right:-.035em;%
      font-variant:small-caps;%
    }%
  }%
  \Css{%
    span.HoLogo-SliTeX-narrow span.HoLogo-i{%
      margin-right:-.06em;%
      font-variant:small-caps;%
    }%
  }%
  \global\let\HoLogoCss@SliTeX@narrow\relax
}
%    \end{macrocode}
%    \end{macro}
%
% \paragraph{Macro set completion.}
%
%    \begin{macro}{\HoLogo@SLiTeX@simple}
%    \begin{macrocode}
\def\HoLogo@SLiTeX@simple{\HoLogo@SliTeX@simple}
%    \end{macrocode}
%    \end{macro}
%    \begin{macro}{\HoLogoBkm@SLiTeX@simple}
%    \begin{macrocode}
\def\HoLogoBkm@SLiTeX@simple{\HoLogoBkm@SliTeX@simple}
%    \end{macrocode}
%    \end{macro}
%    \begin{macro}{\HoLogoHtml@SLiTeX@simple}
%    \begin{macrocode}
\def\HoLogoHtml@SLiTeX@simple{\HoLogoHtml@SliTeX@simple}
%    \end{macrocode}
%    \end{macro}
%
%    \begin{macro}{\HoLogo@SLiTeX@narrow}
%    \begin{macrocode}
\def\HoLogo@SLiTeX@narrow{\HoLogo@SliTeX@narrow}
%    \end{macrocode}
%    \end{macro}
%    \begin{macro}{\HoLogoBkm@SLiTeX@narrow}
%    \begin{macrocode}
\def\HoLogoBkm@SLiTeX@narrow{\HoLogoBkm@SliTeX@narrow}
%    \end{macrocode}
%    \end{macro}
%    \begin{macro}{\HoLogoHtml@SLiTeX@narrow}
%    \begin{macrocode}
\def\HoLogoHtml@SLiTeX@narrow{\HoLogoHtml@SliTeX@narrow}
%    \end{macrocode}
%    \end{macro}
%
%    \begin{macro}{\HoLogo@SliTeX@lift}
%    \begin{macrocode}
\def\HoLogo@SliTeX@lift{\HoLogo@SLiTeX@lift}
%    \end{macrocode}
%    \end{macro}
%    \begin{macro}{\HoLogoBkm@SliTeX@lift}
%    \begin{macrocode}
\def\HoLogoBkm@SliTeX@lift{\HoLogoBkm@SLiTeX@lift}
%    \end{macrocode}
%    \end{macro}
%    \begin{macro}{\HoLogoHtml@SliTeX@lift}
%    \begin{macrocode}
\def\HoLogoHtml@SliTeX@lift{\HoLogoHtml@SLiTeX@lift}
%    \end{macrocode}
%    \end{macro}
%
% \paragraph{Defaults.}
%
%    \begin{macro}{\HoLogo@SLiTeX}
%    \begin{macrocode}
\def\HoLogo@SLiTeX{\HoLogo@SLiTeX@lift}
%    \end{macrocode}
%    \end{macro}
%    \begin{macro}{\HoLogoBkm@SLiTeX}
%    \begin{macrocode}
\def\HoLogoBkm@SLiTeX{\HoLogoBkm@SLiTeX@lift}
%    \end{macrocode}
%    \end{macro}
%    \begin{macro}{\HoLogoHtml@SLiTeX}
%    \begin{macrocode}
\def\HoLogoHtml@SLiTeX{\HoLogoHtml@SLiTeX@lift}
%    \end{macrocode}
%    \end{macro}
%
%    \begin{macro}{\HoLogo@SliTeX}
%    \begin{macrocode}
\def\HoLogo@SliTeX{\HoLogo@SliTeX@narrow}
%    \end{macrocode}
%    \end{macro}
%    \begin{macro}{\HoLogoBkm@SliTeX}
%    \begin{macrocode}
\def\HoLogoBkm@SliTeX{\HoLogoBkm@SliTeX@narrow}
%    \end{macrocode}
%    \end{macro}
%    \begin{macro}{\HoLogoHtml@SliTeX}
%    \begin{macrocode}
\def\HoLogoHtml@SliTeX{\HoLogoHtml@SliTeX@narrow}
%    \end{macrocode}
%    \end{macro}
%
% \subsubsection{\hologo{LuaTeX}}
%
%    \begin{macro}{\HoLogo@LuaTeX}
%    The kerning is an idea of Hans Hagen, see mailing list
%    `luatex at tug dot org' in March 2010.
%    \begin{macrocode}
\def\HoLogo@LuaTeX#1{%
  \HOLOGO@mbox{%
    Lua%
    \HOLOGO@NegativeKerning{aT,oT,To}%
    \hologo{TeX}%
  }%
}
%    \end{macrocode}
%    \end{macro}
%    \begin{macro}{\HoLogoHtml@LuaTeX}
%    \begin{macrocode}
\let\HoLogoHtml@LuaTeX\HoLogo@LuaTeX
%    \end{macrocode}
%    \end{macro}
%
% \subsubsection{\hologo{LuaLaTeX}}
%
%    \begin{macro}{\HoLogo@LuaLaTeX}
%    \begin{macrocode}
\def\HoLogo@LuaLaTeX#1{%
  \HOLOGO@mbox{%
    Lua%
    \hologo{LaTeX}%
  }%
}
%    \end{macrocode}
%    \end{macro}
%    \begin{macro}{\HoLogoHtml@LuaLaTeX}
%    \begin{macrocode}
\let\HoLogoHtml@LuaLaTeX\HoLogo@LuaLaTeX
%    \end{macrocode}
%    \end{macro}
%
% \subsubsection{\hologo{XeTeX}, \hologo{XeLaTeX}}
%
%    \begin{macro}{\HOLOGO@IfCharExists}
%    \begin{macrocode}
\ifluatex
  \ifnum\luatexversion<36 %
  \else
    \def\HOLOGO@IfCharExists#1{%
      \ifnum
        \directlua{%
           if luaotfload and luaotfload.aux then
             if luaotfload.aux.font_has_glyph(%
                    font.current(), \number#1) then % 	 
	       tex.print("1") % 	 
	     end % 	 
	   elseif font and font.fonts and font.current then %
            local f = font.fonts[font.current()]%
            if f.characters and f.characters[\number#1] then %
              tex.print("1")%
            end %
          end%
        }0=\ltx@zero
        \expandafter\ltx@secondoftwo
      \else
        \expandafter\ltx@firstoftwo
      \fi
    }%
  \fi
\fi
\ltx@IfUndefined{HOLOGO@IfCharExists}{%
  \def\HOLOGO@@IfCharExists#1{%
    \begingroup
      \tracinglostchars=\ltx@zero
      \setbox\ltx@zero=\hbox{%
        \kern7sp\char#1\relax
        \ifnum\lastkern>\ltx@zero
          \expandafter\aftergroup\csname iffalse\endcsname
        \else
          \expandafter\aftergroup\csname iftrue\endcsname
        \fi
      }%
      % \if{true|false} from \aftergroup
      \endgroup
      \expandafter\ltx@firstoftwo
    \else
      \endgroup
      \expandafter\ltx@secondoftwo
    \fi
  }%
  \ifxetex
    \ltx@IfUndefined{XeTeXfonttype}{}{%
      \ltx@IfUndefined{XeTeXcharglyph}{}{%
        \def\HOLOGO@IfCharExists#1{%
          \ifnum\XeTeXfonttype\font>\ltx@zero
            \expandafter\ltx@firstofthree
          \else
            \expandafter\ltx@gobble
          \fi
          {%
            \ifnum\XeTeXcharglyph#1>\ltx@zero
              \expandafter\ltx@firstoftwo
            \else
              \expandafter\ltx@secondoftwo
            \fi
          }%
          \HOLOGO@@IfCharExists{#1}%
        }%
      }%
    }%
  \fi
}{}
\ltx@ifundefined{HOLOGO@IfCharExists}{%
  \ifnum64=`\^^^^0040\relax % test for big chars of LuaTeX/XeTeX
    \let\HOLOGO@IfCharExists\HOLOGO@@IfCharExists
  \else
    \def\HOLOGO@IfCharExists#1{%
      \ifnum#1>255 %
        \expandafter\ltx@fourthoffour
      \fi
      \HOLOGO@@IfCharExists{#1}%
    }%
  \fi
}{}
%    \end{macrocode}
%    \end{macro}
%
%    \begin{macro}{\HoLogo@Xe}
%    Source: package \xpackage{dtklogos}
%    \begin{macrocode}
\def\HoLogo@Xe#1{%
  X%
  \kern-.1em\relax
  \HOLOGO@IfCharExists{"018E}{%
    \lower.5ex\hbox{\char"018E}%
  }{%
    \chardef\HOLOGO@choice=\ltx@zero
    \ifdim\fontdimen\ltx@one\font>0pt %
      \ltx@IfUndefined{rotatebox}{%
        \ltx@IfUndefined{pgftext}{%
          \ltx@IfUndefined{psscalebox}{%
            \ltx@IfUndefined{HOLOGO@ScaleBox@\hologoDriver}{%
            }{%
              \chardef\HOLOGO@choice=4 %
            }%
          }{%
            \chardef\HOLOGO@choice=3 %
          }%
        }{%
          \chardef\HOLOGO@choice=2 %
        }%
      }{%
        \chardef\HOLOGO@choice=1 %
      }%
      \ifcase\HOLOGO@choice
        \HOLOGO@WarningUnsupportedDriver{Xe}%
        e%
      \or % 1: \rotatebox
        \begingroup
          \setbox\ltx@zero\hbox{\rotatebox{180}{E}}%
          \ltx@LocDimenA=\dp\ltx@zero
          \advance\ltx@LocDimenA by -.5ex\relax
          \raise\ltx@LocDimenA\box\ltx@zero
        \endgroup
      \or % 2: \pgftext
        \lower.5ex\hbox{%
          \pgfpicture
            \pgftext[rotate=180]{E}%
          \endpgfpicture
        }%
      \or % 3: \psscalebox
        \begingroup
          \setbox\ltx@zero\hbox{\psscalebox{-1 -1}{E}}%
          \ltx@LocDimenA=\dp\ltx@zero
          \advance\ltx@LocDimenA by -.5ex\relax
          \raise\ltx@LocDimenA\box\ltx@zero
        \endgroup
      \or % 4: \HOLOGO@PointReflectBox
        \lower.5ex\hbox{\HOLOGO@PointReflectBox{E}}%
      \else
        \@PackageError{hologo}{Internal error (choice/it}\@ehc
      \fi
    \else
      \ltx@IfUndefined{reflectbox}{%
        \ltx@IfUndefined{pgftext}{%
          \ltx@IfUndefined{psscalebox}{%
            \ltx@IfUndefined{HOLOGO@ScaleBox@\hologoDriver}{%
            }{%
              \chardef\HOLOGO@choice=4 %
            }%
          }{%
            \chardef\HOLOGO@choice=3 %
          }%
        }{%
          \chardef\HOLOGO@choice=2 %
        }%
      }{%
        \chardef\HOLOGO@choice=1 %
      }%
      \ifcase\HOLOGO@choice
        \HOLOGO@WarningUnsupportedDriver{Xe}%
        e%
      \or % 1: reflectbox
        \lower.5ex\hbox{%
          \reflectbox{E}%
        }%
      \or % 2: \pgftext
        \lower.5ex\hbox{%
          \pgfpicture
            \pgftransformxscale{-1}%
            \pgftext{E}%
          \endpgfpicture
        }%
      \or % 3: \psscalebox
        \lower.5ex\hbox{%
          \psscalebox{-1 1}{E}%
        }%
      \or % 4: \HOLOGO@Reflectbox
        \lower.5ex\hbox{%
          \HOLOGO@ReflectBox{E}%
        }%
      \else
        \@PackageError{hologo}{Internal error (choice/up)}\@ehc
      \fi
    \fi
  }%
}
%    \end{macrocode}
%    \end{macro}
%    \begin{macro}{\HoLogoHtml@Xe}
%    \begin{macrocode}
\def\HoLogoHtml@Xe#1{%
  \HoLogoCss@Xe
  \HOLOGO@Span{Xe}{%
    X%
    \HOLOGO@Span{e}{%
      \HCode{&\ltx@hashchar x018e;}%
    }%
  }%
}
%    \end{macrocode}
%    \end{macro}
%    \begin{macro}{\HoLogoCss@Xe}
%    \begin{macrocode}
\def\HoLogoCss@Xe{%
  \Css{%
    span.HoLogo-Xe span.HoLogo-e{%
      position:relative;%
      top:.5ex;%
      left-margin:-.1em;%
    }%
  }%
  \global\let\HoLogoCss@Xe\relax
}
%    \end{macrocode}
%    \end{macro}
%
%    \begin{macro}{\HoLogo@XeTeX}
%    \begin{macrocode}
\def\HoLogo@XeTeX#1{%
  \hologo{Xe}%
  \kern-.15em\relax
  \hologo{TeX}%
}
%    \end{macrocode}
%    \end{macro}
%
%    \begin{macro}{\HoLogoHtml@XeTeX}
%    \begin{macrocode}
\def\HoLogoHtml@XeTeX#1{%
  \HoLogoCss@XeTeX
  \HOLOGO@Span{XeTeX}{%
    \hologo{Xe}%
    \hologo{TeX}%
  }%
}
%    \end{macrocode}
%    \end{macro}
%    \begin{macro}{\HoLogoCss@XeTeX}
%    \begin{macrocode}
\def\HoLogoCss@XeTeX{%
  \Css{%
    span.HoLogo-XeTeX span.HoLogo-TeX{%
      margin-left:-.15em;%
    }%
  }%
  \global\let\HoLogoCss@XeTeX\relax
}
%    \end{macrocode}
%    \end{macro}
%
%    \begin{macro}{\HoLogo@XeLaTeX}
%    \begin{macrocode}
\def\HoLogo@XeLaTeX#1{%
  \hologo{Xe}%
  \kern-.13em%
  \hologo{LaTeX}%
}
%    \end{macrocode}
%    \end{macro}
%    \begin{macro}{\HoLogoHtml@XeLaTeX}
%    \begin{macrocode}
\def\HoLogoHtml@XeLaTeX#1{%
  \HoLogoCss@XeLaTeX
  \HOLOGO@Span{XeLaTeX}{%
    \hologo{Xe}%
    \hologo{LaTeX}%
  }%
}
%    \end{macrocode}
%    \end{macro}
%    \begin{macro}{\HoLogoCss@XeLaTeX}
%    \begin{macrocode}
\def\HoLogoCss@XeLaTeX{%
  \Css{%
    span.HoLogo-XeLaTeX span.HoLogo-Xe{%
      margin-right:-.13em;%
    }%
  }%
  \global\let\HoLogoCss@XeLaTeX\relax
}
%    \end{macrocode}
%    \end{macro}
%
% \subsubsection{\hologo{pdfTeX}, \hologo{pdfLaTeX}}
%
%    \begin{macro}{\HoLogo@pdfTeX}
%    \begin{macrocode}
\def\HoLogo@pdfTeX#1{%
  \HOLOGO@mbox{%
    #1{p}{P}df\hologo{TeX}%
  }%
}
%    \end{macrocode}
%    \end{macro}
%    \begin{macro}{\HoLogoCs@pdfTeX}
%    \begin{macrocode}
\def\HoLogoCs@pdfTeX#1{#1{p}{P}dfTeX}
%    \end{macrocode}
%    \end{macro}
%    \begin{macro}{\HoLogoBkm@pdfTeX}
%    \begin{macrocode}
\def\HoLogoBkm@pdfTeX#1{%
  #1{p}{P}df\hologo{TeX}%
}
%    \end{macrocode}
%    \end{macro}
%    \begin{macro}{\HoLogoHtml@pdfTeX}
%    \begin{macrocode}
\let\HoLogoHtml@pdfTeX\HoLogo@pdfTeX
%    \end{macrocode}
%    \end{macro}
%
%    \begin{macro}{\HoLogo@pdfLaTeX}
%    \begin{macrocode}
\def\HoLogo@pdfLaTeX#1{%
  \HOLOGO@mbox{%
    #1{p}{P}df\hologo{LaTeX}%
  }%
}
%    \end{macrocode}
%    \end{macro}
%    \begin{macro}{\HoLogoCs@pdfLaTeX}
%    \begin{macrocode}
\def\HoLogoCs@pdfLaTeX#1{#1{p}{P}dfLaTeX}
%    \end{macrocode}
%    \end{macro}
%    \begin{macro}{\HoLogoBkm@pdfLaTeX}
%    \begin{macrocode}
\def\HoLogoBkm@pdfLaTeX#1{%
  #1{p}{P}df\hologo{LaTeX}%
}
%    \end{macrocode}
%    \end{macro}
%    \begin{macro}{\HoLogoHtml@pdfLaTeX}
%    \begin{macrocode}
\let\HoLogoHtml@pdfLaTeX\HoLogo@pdfLaTeX
%    \end{macrocode}
%    \end{macro}
%
% \subsubsection{\hologo{VTeX}}
%
%    \begin{macro}{\HoLogo@VTeX}
%    \begin{macrocode}
\def\HoLogo@VTeX#1{%
  \HOLOGO@mbox{%
    V\hologo{TeX}%
  }%
}
%    \end{macrocode}
%    \end{macro}
%    \begin{macro}{\HoLogoHtml@VTeX}
%    \begin{macrocode}
\let\HoLogoHtml@VTeX\HoLogo@VTeX
%    \end{macrocode}
%    \end{macro}
%
% \subsubsection{\hologo{AmS}, \dots}
%
%    Source: class \xclass{amsdtx}
%
%    \begin{macro}{\HoLogo@AmS}
%    \begin{macrocode}
\def\HoLogo@AmS#1{%
  \HoLogoFont@font{AmS}{sy}{%
    A%
    \kern-.1667em%
    \lower.5ex\hbox{M}%
    \kern-.125em%
    S%
  }%
}
%    \end{macrocode}
%    \end{macro}
%    \begin{macro}{\HoLogoBkm@AmS}
%    \begin{macrocode}
\def\HoLogoBkm@AmS#1{AmS}
%    \end{macrocode}
%    \end{macro}
%    \begin{macro}{\HoLogoHtml@AmS}
%    \begin{macrocode}
\def\HoLogoHtml@AmS#1{%
  \HoLogoCss@AmS
%  \HoLogoFont@font{AmS}{sy}{%
    \HOLOGO@Span{AmS}{%
      A%
      \HOLOGO@Span{M}{M}%
      S%
    }%
%   }%
}
%    \end{macrocode}
%    \end{macro}
%    \begin{macro}{\HoLogoCss@AmS}
%    \begin{macrocode}
\def\HoLogoCss@AmS{%
  \Css{%
    span.HoLogo-AmS span.HoLogo-M{%
      position:relative;%
      top:.5ex;%
      margin-left:-.1667em;%
      margin-right:-.125em;%
      text-decoration:none;%
    }%
  }%
  \global\let\HoLogoCss@AmS\relax
}
%    \end{macrocode}
%    \end{macro}
%
%    \begin{macro}{\HoLogo@AmSTeX}
%    \begin{macrocode}
\def\HoLogo@AmSTeX#1{%
  \hologo{AmS}%
  \HOLOGO@hyphen
  \hologo{TeX}%
}
%    \end{macrocode}
%    \end{macro}
%    \begin{macro}{\HoLogoBkm@AmSTeX}
%    \begin{macrocode}
\def\HoLogoBkm@AmSTeX#1{AmS-TeX}%
%    \end{macrocode}
%    \end{macro}
%    \begin{macro}{\HoLogoHtml@AmSTeX}
%    \begin{macrocode}
\let\HoLogoHtml@AmSTeX\HoLogo@AmSTeX
%    \end{macrocode}
%    \end{macro}
%
%    \begin{macro}{\HoLogo@AmSLaTeX}
%    \begin{macrocode}
\def\HoLogo@AmSLaTeX#1{%
  \hologo{AmS}%
  \HOLOGO@hyphen
  \hologo{LaTeX}%
}
%    \end{macrocode}
%    \end{macro}
%    \begin{macro}{\HoLogoBkm@AmSLaTeX}
%    \begin{macrocode}
\def\HoLogoBkm@AmSLaTeX#1{AmS-LaTeX}%
%    \end{macrocode}
%    \end{macro}
%    \begin{macro}{\HoLogoHtml@AmSLaTeX}
%    \begin{macrocode}
\let\HoLogoHtml@AmSLaTeX\HoLogo@AmSLaTeX
%    \end{macrocode}
%    \end{macro}
%
% \subsubsection{\hologo{BibTeX}}
%
%    \begin{macro}{\HoLogo@BibTeX@sc}
%    A definition of \hologo{BibTeX} is provided in
%    the documentation source for the manual of \hologo{BibTeX}
%    \cite{btxdoc}.
%\begin{quote}
%\begin{verbatim}
%\def\BibTeX{%
%  {%
%    \rm
%    B%
%    \kern-.05em%
%    {%
%      \sc
%      i%
%      \kern-.025em %
%      b%
%    }%
%    \kern-.08em
%    T%
%    \kern-.1667em%
%    \lower.7ex\hbox{E}%
%    \kern-.125em%
%    X%
%  }%
%}
%\end{verbatim}
%\end{quote}
%    \begin{macrocode}
\def\HoLogo@BibTeX@sc#1{%
  B%
  \kern-.05em%
  \HoLogoFont@font{BibTeX}{sc}{%
    i%
    \kern-.025em%
    b%
  }%
  \HOLOGO@discretionary
  \kern-.08em%
  \hologo{TeX}%
}
%    \end{macrocode}
%    \end{macro}
%    \begin{macro}{\HoLogoHtml@BibTeX@sc}
%    \begin{macrocode}
\def\HoLogoHtml@BibTeX@sc#1{%
  \HoLogoCss@BibTeX@sc
  \HOLOGO@Span{BibTeX-sc}{%
    B%
    \HOLOGO@Span{i}{i}%
    \HOLOGO@Span{b}{b}%
    \hologo{TeX}%
  }%
}
%    \end{macrocode}
%    \end{macro}
%    \begin{macro}{\HoLogoCss@BibTeX@sc}
%    \begin{macrocode}
\def\HoLogoCss@BibTeX@sc{%
  \Css{%
    span.HoLogo-BibTeX-sc span.HoLogo-i{%
      margin-left:-.05em;%
      margin-right:-.025em;%
      font-variant:small-caps;%
    }%
  }%
  \Css{%
    span.HoLogo-BibTeX-sc span.HoLogo-b{%
      margin-right:-.08em;%
      font-variant:small-caps;%
    }%
  }%
  \global\let\HoLogoCss@BibTeX@sc\relax
}
%    \end{macrocode}
%    \end{macro}
%
%    \begin{macro}{\HoLogo@BibTeX@sf}
%    Variant \xoption{sf} avoids trouble with unavailable
%    small caps fonts (e.g., bold versions of Computer Modern or
%    Latin Modern). The definition is taken from
%    package \xpackage{dtklogos} \cite{dtklogos}.
%\begin{quote}
%\begin{verbatim}
%\DeclareRobustCommand{\BibTeX}{%
%  B%
%  \kern-.05em%
%  \hbox{%
%    $\m@th$% %% force math size calculations
%    \csname S@\f@size\endcsname
%    \fontsize\sf@size\z@
%    \math@fontsfalse
%    \selectfont
%    I%
%    \kern-.025em%
%    B
%  }%
%  \kern-.08em%
%  \-%
%  \TeX
%}
%\end{verbatim}
%\end{quote}
%    \begin{macrocode}
\def\HoLogo@BibTeX@sf#1{%
  B%
  \kern-.05em%
  \HoLogoFont@font{BibTeX}{bibsf}{%
    I%
    \kern-.025em%
    B%
  }%
  \HOLOGO@discretionary
  \kern-.08em%
  \hologo{TeX}%
}
%    \end{macrocode}
%    \end{macro}
%    \begin{macro}{\HoLogoHtml@BibTeX@sf}
%    \begin{macrocode}
\def\HoLogoHtml@BibTeX@sf#1{%
  \HoLogoCss@BibTeX@sf
  \HOLOGO@Span{BibTeX-sf}{%
    B%
    \HoLogoFont@font{BibTeX}{bibsf}{%
      \HOLOGO@Span{i}{I}%
      B%
    }%
    \hologo{TeX}%
  }%
}
%    \end{macrocode}
%    \end{macro}
%    \begin{macro}{\HoLogoCss@BibTeX@sf}
%    \begin{macrocode}
\def\HoLogoCss@BibTeX@sf{%
  \Css{%
    span.HoLogo-BibTeX-sf span.HoLogo-i{%
      margin-left:-.05em;%
      margin-right:-.025em;%
    }%
  }%
  \Css{%
    span.HoLogo-BibTeX-sf span.HoLogo-TeX{%
      margin-left:-.08em;%
    }%
  }%
  \global\let\HoLogoCss@BibTeX@sf\relax
}
%    \end{macrocode}
%    \end{macro}
%
%    \begin{macro}{\HoLogo@BibTeX}
%    \begin{macrocode}
\def\HoLogo@BibTeX{\HoLogo@BibTeX@sf}
%    \end{macrocode}
%    \end{macro}
%    \begin{macro}{\HoLogoHtml@BibTeX}
%    \begin{macrocode}
\def\HoLogoHtml@BibTeX{\HoLogoHtml@BibTeX@sf}
%    \end{macrocode}
%    \end{macro}
%
% \subsubsection{\hologo{BibTeX8}}
%
%    \begin{macro}{\HoLogo@BibTeX8}
%    \begin{macrocode}
\expandafter\def\csname HoLogo@BibTeX8\endcsname#1{%
  \hologo{BibTeX}%
  8%
}
%    \end{macrocode}
%    \end{macro}
%
%    \begin{macro}{\HoLogoBkm@BibTeX8}
%    \begin{macrocode}
\expandafter\def\csname HoLogoBkm@BibTeX8\endcsname#1{%
  \hologo{BibTeX}%
  8%
}
%    \end{macrocode}
%    \end{macro}
%    \begin{macro}{\HoLogoHtml@BibTeX8}
%    \begin{macrocode}
\expandafter
\let\csname HoLogoHtml@BibTeX8\expandafter\endcsname
\csname HoLogo@BibTeX8\endcsname
%    \end{macrocode}
%    \end{macro}
%
% \subsubsection{\hologo{ConTeXt}}
%
%    \begin{macro}{\HoLogo@ConTeXt@simple}
%    \begin{macrocode}
\def\HoLogo@ConTeXt@simple#1{%
  \HOLOGO@mbox{Con}%
  \HOLOGO@discretionary
  \HOLOGO@mbox{\hologo{TeX}t}%
}
%    \end{macrocode}
%    \end{macro}
%    \begin{macro}{\HoLogoHtml@ConTeXt@simple}
%    \begin{macrocode}
\let\HoLogoHtml@ConTeXt@simple\HoLogo@ConTeXt@simple
%    \end{macrocode}
%    \end{macro}
%
%    \begin{macro}{\HoLogo@ConTeXt@narrow}
%    This definition of logo \hologo{ConTeXt} with variant \xoption{narrow}
%    comes from TUGboat's class \xclass{ltugboat} (version 2010/11/15 v2.8).
%    \begin{macrocode}
\def\HoLogo@ConTeXt@narrow#1{%
  \HOLOGO@mbox{C\kern-.0333emon}%
  \HOLOGO@discretionary
  \kern-.0667em%
  \HOLOGO@mbox{\hologo{TeX}\kern-.0333emt}%
}
%    \end{macrocode}
%    \end{macro}
%    \begin{macro}{\HoLogoHtml@ConTeXt@narrow}
%    \begin{macrocode}
\def\HoLogoHtml@ConTeXt@narrow#1{%
  \HoLogoCss@ConTeXt@narrow
  \HOLOGO@Span{ConTeXt-narrow}{%
    \HOLOGO@Span{C}{C}%
    on%
    \hologo{TeX}%
    t%
  }%
}
%    \end{macrocode}
%    \end{macro}
%    \begin{macro}{\HoLogoCss@ConTeXt@narrow}
%    \begin{macrocode}
\def\HoLogoCss@ConTeXt@narrow{%
  \Css{%
    span.HoLogo-ConTeXt-narrow span.HoLogo-C{%
      margin-left:-.0333em;%
    }%
  }%
  \Css{%
    span.HoLogo-ConTeXt-narrow span.HoLogo-TeX{%
      margin-left:-.0667em;%
      margin-right:-.0333em;%
    }%
  }%
  \global\let\HoLogoCss@ConTeXt@narrow\relax
}
%    \end{macrocode}
%    \end{macro}
%
%    \begin{macro}{\HoLogo@ConTeXt}
%    \begin{macrocode}
\def\HoLogo@ConTeXt{\HoLogo@ConTeXt@narrow}
%    \end{macrocode}
%    \end{macro}
%    \begin{macro}{\HoLogoHtml@ConTeXt}
%    \begin{macrocode}
\def\HoLogoHtml@ConTeXt{\HoLogoHtml@ConTeXt@narrow}
%    \end{macrocode}
%    \end{macro}
%
% \subsubsection{\hologo{emTeX}}
%
%    \begin{macro}{\HoLogo@emTeX}
%    \begin{macrocode}
\def\HoLogo@emTeX#1{%
  \HOLOGO@mbox{#1{e}{E}m}%
  \HOLOGO@discretionary
  \hologo{TeX}%
}
%    \end{macrocode}
%    \end{macro}
%    \begin{macro}{\HoLogoCs@emTeX}
%    \begin{macrocode}
\def\HoLogoCs@emTeX#1{#1{e}{E}mTeX}%
%    \end{macrocode}
%    \end{macro}
%    \begin{macro}{\HoLogoBkm@emTeX}
%    \begin{macrocode}
\def\HoLogoBkm@emTeX#1{%
  #1{e}{E}m\hologo{TeX}%
}
%    \end{macrocode}
%    \end{macro}
%    \begin{macro}{\HoLogoHtml@emTeX}
%    \begin{macrocode}
\let\HoLogoHtml@emTeX\HoLogo@emTeX
%    \end{macrocode}
%    \end{macro}
%
% \subsubsection{\hologo{ExTeX}}
%
%    \begin{macro}{\HoLogo@ExTeX}
%    The definition is taken from the FAQ of the
%    project \hologo{ExTeX}
%    \cite{ExTeX-FAQ}.
%\begin{quote}
%\begin{verbatim}
%\def\ExTeX{%
%  \textrm{% Logo always with serifs
%    \ensuremath{%
%      \textstyle
%      \varepsilon_{%
%        \kern-0.15em%
%        \mathcal{X}%
%      }%
%    }%
%    \kern-.15em%
%    \TeX
%  }%
%}
%\end{verbatim}
%\end{quote}
%    \begin{macrocode}
\def\HoLogo@ExTeX#1{%
  \HoLogoFont@font{ExTeX}{rm}{%
    \ltx@mbox{%
      \HOLOGO@MathSetup
      $%
        \textstyle
        \varepsilon_{%
          \kern-0.15em%
          \HoLogoFont@font{ExTeX}{sy}{X}%
        }%
      $%
    }%
    \HOLOGO@discretionary
    \kern-.15em%
    \hologo{TeX}%
  }%
}
%    \end{macrocode}
%    \end{macro}
%    \begin{macro}{\HoLogoHtml@ExTeX}
%    \begin{macrocode}
\def\HoLogoHtml@ExTeX#1{%
  \HoLogoCss@ExTeX
  \HoLogoFont@font{ExTeX}{rm}{%
    \HOLOGO@Span{ExTeX}{%
      \ltx@mbox{%
        \HOLOGO@MathSetup
        $\textstyle\varepsilon$%
        \HOLOGO@Span{X}{$\textstyle\chi$}%
        \hologo{TeX}%
      }%
    }%
  }%
}
%    \end{macrocode}
%    \end{macro}
%    \begin{macro}{\HoLogoBkm@ExTeX}
%    \begin{macrocode}
\def\HoLogoBkm@ExTeX#1{%
  \HOLOGO@PdfdocUnicode{#1{e}{E}x}{\textepsilon\textchi}%
  \hologo{TeX}%
}
%    \end{macrocode}
%    \end{macro}
%    \begin{macro}{\HoLogoCss@ExTeX}
%    \begin{macrocode}
\def\HoLogoCss@ExTeX{%
  \Css{%
    span.HoLogo-ExTeX{%
      font-family:serif;%
    }%
  }%
  \Css{%
    span.HoLogo-ExTeX span.HoLogo-TeX{%
      margin-left:-.15em;%
    }%
  }%
  \global\let\HoLogoCss@ExTeX\relax
}
%    \end{macrocode}
%    \end{macro}
%
% \subsubsection{\hologo{MiKTeX}}
%
%    \begin{macro}{\HoLogo@MiKTeX}
%    \begin{macrocode}
\def\HoLogo@MiKTeX#1{%
  \HOLOGO@mbox{MiK}%
  \HOLOGO@discretionary
  \hologo{TeX}%
}
%    \end{macrocode}
%    \end{macro}
%    \begin{macro}{\HoLogoHtml@MiKTeX}
%    \begin{macrocode}
\let\HoLogoHtml@MiKTeX\HoLogo@MiKTeX
%    \end{macrocode}
%    \end{macro}
%
% \subsubsection{\hologo{OzTeX} and friends}
%
%    Source: \hologo{OzTeX} FAQ \cite{OzTeX}:
%    \begin{quote}
%      |\def\OzTeX{O\kern-.03em z\kern-.15em\TeX}|\\
%      (There is no kerning in OzMF, OzMP and OzTtH.)
%    \end{quote}
%
%    \begin{macro}{\HoLogo@OzTeX}
%    \begin{macrocode}
\def\HoLogo@OzTeX#1{%
  O%
  \kern-.03em %
  z%
  \kern-.15em %
  \hologo{TeX}%
}
%    \end{macrocode}
%    \end{macro}
%    \begin{macro}{\HoLogoHtml@OzTeX}
%    \begin{macrocode}
\def\HoLogoHtml@OzTeX#1{%
  \HoLogoCss@OzTeX
  \HOLOGO@Span{OzTeX}{%
    O%
    \HOLOGO@Span{z}{z}%
    \hologo{TeX}%
  }%
}
%    \end{macrocode}
%    \end{macro}
%    \begin{macro}{\HoLogoCss@OzTeX}
%    \begin{macrocode}
\def\HoLogoCss@OzTeX{%
  \Css{%
    span.HoLogo-OzTeX span.HoLogo-z{%
      margin-left:-.03em;%
      margin-right:-.15em;%
    }%
  }%
  \global\let\HoLogoCss@OzTeX\relax
}
%    \end{macrocode}
%    \end{macro}
%
%    \begin{macro}{\HoLogo@OzMF}
%    \begin{macrocode}
\def\HoLogo@OzMF#1{%
  \HOLOGO@mbox{OzMF}%
}
%    \end{macrocode}
%    \end{macro}
%    \begin{macro}{\HoLogo@OzMP}
%    \begin{macrocode}
\def\HoLogo@OzMP#1{%
  \HOLOGO@mbox{OzMP}%
}
%    \end{macrocode}
%    \end{macro}
%    \begin{macro}{\HoLogo@OzTtH}
%    \begin{macrocode}
\def\HoLogo@OzTtH#1{%
  \HOLOGO@mbox{OzTtH}%
}
%    \end{macrocode}
%    \end{macro}
%
% \subsubsection{\hologo{PCTeX}}
%
%    \begin{macro}{\HoLogo@PCTeX}
%    \begin{macrocode}
\def\HoLogo@PCTeX#1{%
  \HOLOGO@mbox{PC}%
  \hologo{TeX}%
}
%    \end{macrocode}
%    \end{macro}
%    \begin{macro}{\HoLogoHtml@PCTeX}
%    \begin{macrocode}
\let\HoLogoHtml@PCTeX\HoLogo@PCTeX
%    \end{macrocode}
%    \end{macro}
%
% \subsubsection{\hologo{PiCTeX}}
%
%    The original definitions from \xfile{pictex.tex} \cite{PiCTeX}:
%\begin{quote}
%\begin{verbatim}
%\def\PiC{%
%  P%
%  \kern-.12em%
%  \lower.5ex\hbox{I}%
%  \kern-.075em%
%  C%
%}
%\def\PiCTeX{%
%  \PiC
%  \kern-.11em%
%  \TeX
%}
%\end{verbatim}
%\end{quote}
%
%    \begin{macro}{\HoLogo@PiC}
%    \begin{macrocode}
\def\HoLogo@PiC#1{%
  P%
  \kern-.12em%
  \lower.5ex\hbox{I}%
  \kern-.075em%
  C%
  \HOLOGO@SpaceFactor
}
%    \end{macrocode}
%    \end{macro}
%    \begin{macro}{\HoLogoHtml@PiC}
%    \begin{macrocode}
\def\HoLogoHtml@PiC#1{%
  \HoLogoCss@PiC
  \HOLOGO@Span{PiC}{%
    P%
    \HOLOGO@Span{i}{I}%
    C%
  }%
}
%    \end{macrocode}
%    \end{macro}
%    \begin{macro}{\HoLogoCss@PiC}
%    \begin{macrocode}
\def\HoLogoCss@PiC{%
  \Css{%
    span.HoLogo-PiC span.HoLogo-i{%
      position:relative;%
      top:.5ex;%
      margin-left:-.12em;%
      margin-right:-.075em;%
      text-decoration:none;%
    }%
  }%
  \global\let\HoLogoCss@PiC\relax
}
%    \end{macrocode}
%    \end{macro}
%
%    \begin{macro}{\HoLogo@PiCTeX}
%    \begin{macrocode}
\def\HoLogo@PiCTeX#1{%
  \hologo{PiC}%
  \HOLOGO@discretionary
  \kern-.11em%
  \hologo{TeX}%
}
%    \end{macrocode}
%    \end{macro}
%    \begin{macro}{\HoLogoHtml@PiCTeX}
%    \begin{macrocode}
\def\HoLogoHtml@PiCTeX#1{%
  \HoLogoCss@PiCTeX
  \HOLOGO@Span{PiCTeX}{%
    \hologo{PiC}%
    \hologo{TeX}%
  }%
}
%    \end{macrocode}
%    \end{macro}
%    \begin{macro}{\HoLogoCss@PiCTeX}
%    \begin{macrocode}
\def\HoLogoCss@PiCTeX{%
  \Css{%
    span.HoLogo-PiCTeX span.HoLogo-PiC{%
      margin-right:-.11em;%
    }%
  }%
  \global\let\HoLogoCss@PiCTeX\relax
}
%    \end{macrocode}
%    \end{macro}
%
% \subsubsection{\hologo{teTeX}}
%
%    \begin{macro}{\HoLogo@teTeX}
%    \begin{macrocode}
\def\HoLogo@teTeX#1{%
  \HOLOGO@mbox{#1{t}{T}e}%
  \HOLOGO@discretionary
  \hologo{TeX}%
}
%    \end{macrocode}
%    \end{macro}
%    \begin{macro}{\HoLogoCs@teTeX}
%    \begin{macrocode}
\def\HoLogoCs@teTeX#1{#1{t}{T}dfTeX}
%    \end{macrocode}
%    \end{macro}
%    \begin{macro}{\HoLogoBkm@teTeX}
%    \begin{macrocode}
\def\HoLogoBkm@teTeX#1{%
  #1{t}{T}e\hologo{TeX}%
}
%    \end{macrocode}
%    \end{macro}
%    \begin{macro}{\HoLogoHtml@teTeX}
%    \begin{macrocode}
\let\HoLogoHtml@teTeX\HoLogo@teTeX
%    \end{macrocode}
%    \end{macro}
%
% \subsubsection{\hologo{TeX4ht}}
%
%    \begin{macro}{\HoLogo@TeX4ht}
%    \begin{macrocode}
\expandafter\def\csname HoLogo@TeX4ht\endcsname#1{%
  \HOLOGO@mbox{\hologo{TeX}4ht}%
}
%    \end{macrocode}
%    \end{macro}
%    \begin{macro}{\HoLogoHtml@TeX4ht}
%    \begin{macrocode}
\expandafter
\let\csname HoLogoHtml@TeX4ht\expandafter\endcsname
\csname HoLogo@TeX4ht\endcsname
%    \end{macrocode}
%    \end{macro}
%
%
% \subsubsection{\hologo{SageTeX}}
%
%    \begin{macro}{\HoLogo@SageTeX}
%    \begin{macrocode}
\def\HoLogo@SageTeX#1{%
  \HOLOGO@mbox{Sage}%
  \HOLOGO@discretionary
  \HOLOGO@NegativeKerning{eT,oT,To}%
  \hologo{TeX}%
}
%    \end{macrocode}
%    \end{macro}
%    \begin{macro}{\HoLogoHtml@SageTeX}
%    \begin{macrocode}
\let\HoLogoHtml@SageTeX\HoLogo@SageTeX
%    \end{macrocode}
%    \end{macro}
%
% \subsection{\hologo{METAFONT} and friends}
%
%    \begin{macro}{\HoLogo@METAFONT}
%    \begin{macrocode}
\def\HoLogo@METAFONT#1{%
  \HoLogoFont@font{METAFONT}{logo}{%
    \HOLOGO@mbox{META}%
    \HOLOGO@discretionary
    \HOLOGO@mbox{FONT}%
  }%
}
%    \end{macrocode}
%    \end{macro}
%
%    \begin{macro}{\HoLogo@METAPOST}
%    \begin{macrocode}
\def\HoLogo@METAPOST#1{%
  \HoLogoFont@font{METAPOST}{logo}{%
    \HOLOGO@mbox{META}%
    \HOLOGO@discretionary
    \HOLOGO@mbox{POST}%
  }%
}
%    \end{macrocode}
%    \end{macro}
%
%    \begin{macro}{\HoLogo@MetaFun}
%    \begin{macrocode}
\def\HoLogo@MetaFun#1{%
  \HOLOGO@mbox{Meta}%
  \HOLOGO@discretionary
  \HOLOGO@mbox{Fun}%
}
%    \end{macrocode}
%    \end{macro}
%
%    \begin{macro}{\HoLogo@MetaPost}
%    \begin{macrocode}
\def\HoLogo@MetaPost#1{%
  \HOLOGO@mbox{Meta}%
  \HOLOGO@discretionary
  \HOLOGO@mbox{Post}%
}
%    \end{macrocode}
%    \end{macro}
%
% \subsection{Others}
%
% \subsubsection{\hologo{biber}}
%
%    \begin{macro}{\HoLogo@biber}
%    \begin{macrocode}
\def\HoLogo@biber#1{%
  \HOLOGO@mbox{#1{b}{B}i}%
  \HOLOGO@discretionary
  \HOLOGO@mbox{ber}%
}
%    \end{macrocode}
%    \end{macro}
%    \begin{macro}{\HoLogoCs@biber}
%    \begin{macrocode}
\def\HoLogoCs@biber#1{#1{b}{B}iber}
%    \end{macrocode}
%    \end{macro}
%    \begin{macro}{\HoLogoBkm@biber}
%    \begin{macrocode}
\def\HoLogoBkm@biber#1{%
  #1{b}{B}iber%
}
%    \end{macrocode}
%    \end{macro}
%    \begin{macro}{\HoLogoHtml@biber}
%    \begin{macrocode}
\let\HoLogoHtml@biber\HoLogo@biber
%    \end{macrocode}
%    \end{macro}
%
% \subsubsection{\hologo{KOMAScript}}
%
%    \begin{macro}{\HoLogo@KOMAScript}
%    The definition for \hologo{KOMAScript} is taken
%    from \hologo{KOMAScript} (\xfile{scrlogo.dtx}, reformatted) \cite{scrlogo}:
%\begin{quote}
%\begin{verbatim}
%\@ifundefined{KOMAScript}{%
%  \DeclareRobustCommand{\KOMAScript}{%
%    \textsf{%
%      K\kern.05em O\kern.05emM\kern.05em A%
%      \kern.1em-\kern.1em %
%      Script%
%    }%
%  }%
%}{}
%\end{verbatim}
%\end{quote}
%    \begin{macrocode}
\def\HoLogo@KOMAScript#1{%
  \HoLogoFont@font{KOMAScript}{sf}{%
    \HOLOGO@mbox{%
      K\kern.05em%
      O\kern.05em%
      M\kern.05em%
      A%
    }%
    \kern.1em%
    \HOLOGO@hyphen
    \kern.1em%
    \HOLOGO@mbox{Script}%
  }%
}
%    \end{macrocode}
%    \end{macro}
%    \begin{macro}{\HoLogoBkm@KOMAScript}
%    \begin{macrocode}
\def\HoLogoBkm@KOMAScript#1{%
  KOMA-Script%
}
%    \end{macrocode}
%    \end{macro}
%    \begin{macro}{\HoLogoHtml@KOMAScript}
%    \begin{macrocode}
\def\HoLogoHtml@KOMAScript#1{%
  \HoLogoCss@KOMAScript
  \HoLogoFont@font{KOMAScript}{sf}{%
    \HOLOGO@Span{KOMAScript}{%
      K%
      \HOLOGO@Span{O}{O}%
      M%
      \HOLOGO@Span{A}{A}%
      \HOLOGO@Span{hyphen}{-}%
      Script%
    }%
  }%
}
%    \end{macrocode}
%    \end{macro}
%    \begin{macro}{\HoLogoCss@KOMAScript}
%    \begin{macrocode}
\def\HoLogoCss@KOMAScript{%
  \Css{%
    span.HoLogo-KOMAScript{%
      font-family:sans-serif;%
    }%
  }%
  \Css{%
    span.HoLogo-KOMAScript span.HoLogo-O{%
      padding-left:.05em;%
      padding-right:.05em;%
    }%
  }%
  \Css{%
    span.HoLogo-KOMAScript span.HoLogo-A{%
      padding-left:.05em;%
    }%
  }%
  \Css{%
    span.HoLogo-KOMAScript span.HoLogo-hyphen{%
      padding-left:.1em;%
      padding-right:.1em;%
    }%
  }%
  \global\let\HoLogoCss@KOMAScript\relax
}
%    \end{macrocode}
%    \end{macro}
%
% \subsubsection{\hologo{LyX}}
%
%    \begin{macro}{\HoLogo@LyX}
%    The definition is taken from the documentation source files
%    of \hologo{LyX}, \xfile{Intro.lyx} \cite{LyX}:
%\begin{quote}
%\begin{verbatim}
%\def\LyX{%
%  \texorpdfstring{%
%    L\kern-.1667em\lower.25em\hbox{Y}\kern-.125emX\@%
%  }{%
%    LyX%
%  }%
%}
%\end{verbatim}
%\end{quote}
%    \begin{macrocode}
\def\HoLogo@LyX#1{%
  L%
  \kern-.1667em%
  \lower.25em\hbox{Y}%
  \kern-.125em%
  X%
  \HOLOGO@SpaceFactor
}
%    \end{macrocode}
%    \end{macro}
%    \begin{macro}{\HoLogoHtml@LyX}
%    \begin{macrocode}
\def\HoLogoHtml@LyX#1{%
  \HoLogoCss@LyX
  \HOLOGO@Span{LyX}{%
    L%
    \HOLOGO@Span{y}{Y}%
    X%
  }%
}
%    \end{macrocode}
%    \end{macro}
%    \begin{macro}{\HoLogoCss@LyX}
%    \begin{macrocode}
\def\HoLogoCss@LyX{%
  \Css{%
    span.HoLogo-LyX span.HoLogo-y{%
      position:relative;%
      top:.25em;%
      margin-left:-.1667em;%
      margin-right:-.125em;%
      text-decoration:none;%
    }%
  }%
  \global\let\HoLogoCss@LyX\relax
}
%    \end{macrocode}
%    \end{macro}
%
% \subsubsection{\hologo{NTS}}
%
%    \begin{macro}{\HoLogo@NTS}
%    Definition for \hologo{NTS} can be found in
%    package \xpackage{etex\textunderscore man} for the \hologo{eTeX} manual \cite{etexman}
%    and in package \xpackage{dtklogos} \cite{dtklogos}:
%\begin{quote}
%\begin{verbatim}
%\def\NTS{%
%  \leavevmode
%  \hbox{%
%    $%
%      \cal N%
%      \kern-0.35em%
%      \lower0.5ex\hbox{$\cal T$}%
%      \kern-0.2em%
%      S%
%    $%
%  }%
%}
%\end{verbatim}
%\end{quote}
%    \begin{macrocode}
\def\HoLogo@NTS#1{%
  \HoLogoFont@font{NTS}{sy}{%
    N\/%
    \kern-.35em%
    \lower.5ex\hbox{T\/}%
    \kern-.2em%
    S\/%
  }%
  \HOLOGO@SpaceFactor
}
%    \end{macrocode}
%    \end{macro}
%
% \subsubsection{\Hologo{TTH} (\hologo{TeX} to HTML translator)}
%
%    Source: \url{http://hutchinson.belmont.ma.us/tth/}
%    In the HTML source the second `T' is printed as subscript.
%\begin{quote}
%\begin{verbatim}
%T<sub>T</sub>H
%\end{verbatim}
%\end{quote}
%    \begin{macro}{\HoLogo@TTH}
%    \begin{macrocode}
\def\HoLogo@TTH#1{%
  \ltx@mbox{%
    T\HOLOGO@SubScript{T}H%
  }%
  \HOLOGO@SpaceFactor
}
%    \end{macrocode}
%    \end{macro}
%
%    \begin{macro}{\HoLogoHtml@TTH}
%    \begin{macrocode}
\def\HoLogoHtml@TTH#1{%
  T\HCode{<sub>}T\HCode{</sub>}H%
}
%    \end{macrocode}
%    \end{macro}
%
% \subsubsection{\Hologo{HanTheThanh}}
%
%    Partial source: Package \xpackage{dtklogos}.
%    The double accent is U+1EBF (latin small letter e with circumflex
%    and acute).
%    \begin{macro}{\HoLogo@HanTheThanh}
%    \begin{macrocode}
\def\HoLogo@HanTheThanh#1{%
  \ltx@mbox{H\`an}%
  \HOLOGO@space
  \ltx@mbox{%
    Th%
    \HOLOGO@IfCharExists{"1EBF}{%
      \char"1EBF\relax
    }{%
      \^e\hbox to 0pt{\hss\raise .5ex\hbox{\'{}}}%
    }%
  }%
  \HOLOGO@space
  \ltx@mbox{Th\`anh}%
}
%    \end{macrocode}
%    \end{macro}
%    \begin{macro}{\HoLogoBkm@HanTheThanh}
%    \begin{macrocode}
\def\HoLogoBkm@HanTheThanh#1{%
  H\`an %
  Th\HOLOGO@PdfdocUnicode{\^e}{\9036\277} %
  Th\`anh%
}
%    \end{macrocode}
%    \end{macro}
%    \begin{macro}{\HoLogoHtml@HanTheThanh}
%    \begin{macrocode}
\def\HoLogoHtml@HanTheThanh#1{%
  H\`an %
  Th\HCode{&\ltx@hashchar x1ebf;} %
  Th\`anh%
}
%    \end{macrocode}
%    \end{macro}
%
% \subsection{Driver detection}
%
%    \begin{macrocode}
\HOLOGO@IfExists\InputIfFileExists{%
  \InputIfFileExists{hologo.cfg}{}{}%
}{%
  \ltx@IfUndefined{pdf@filesize}{%
    \def\HOLOGO@InputIfExists{%
      \openin\HOLOGO@temp=hologo.cfg\relax
      \ifeof\HOLOGO@temp
        \closein\HOLOGO@temp
      \else
        \closein\HOLOGO@temp
        \begingroup
          \def\x{LaTeX2e}%
        \expandafter\endgroup
        \ifx\fmtname\x
          \input{hologo.cfg}%
        \else
          \input hologo.cfg\relax
        \fi
      \fi
    }%
    \ltx@IfUndefined{newread}{%
      \chardef\HOLOGO@temp=15 %
      \def\HOLOGO@CheckRead{%
        \ifeof\HOLOGO@temp
          \HOLOGO@InputIfExists
        \else
          \ifcase\HOLOGO@temp
            \@PackageWarningNoLine{hologo}{%
              Configuration file ignored, because\MessageBreak
              a free read register could not be found%
            }%
          \else
            \begingroup
              \count\ltx@cclv=\HOLOGO@temp
              \advance\ltx@cclv by \ltx@minusone
              \edef\x{\endgroup
                \chardef\noexpand\HOLOGO@temp=\the\count\ltx@cclv
                \relax
              }%
            \x
          \fi
        \fi
      }%
    }{%
      \csname newread\endcsname\HOLOGO@temp
      \HOLOGO@InputIfExists
    }%
  }{%
    \edef\HOLOGO@temp{\pdf@filesize{hologo.cfg}}%
    \ifx\HOLOGO@temp\ltx@empty
    \else
      \ifnum\HOLOGO@temp>0 %
        \begingroup
          \def\x{LaTeX2e}%
        \expandafter\endgroup
        \ifx\fmtname\x
          \input{hologo.cfg}%
        \else
          \input hologo.cfg\relax
        \fi
      \else
        \@PackageInfoNoLine{hologo}{%
          Empty configuration file `hologo.cfg' ignored%
        }%
      \fi
    \fi
  }%
}
%    \end{macrocode}
%
%    \begin{macrocode}
\def\HOLOGO@temp#1#2{%
  \kv@define@key{HoLogoDriver}{#1}[]{%
    \begingroup
      \def\HOLOGO@temp{##1}%
      \ltx@onelevel@sanitize\HOLOGO@temp
      \ifx\HOLOGO@temp\ltx@empty
      \else
        \@PackageError{hologo}{%
          Value (\HOLOGO@temp) not permitted for option `#1'%
        }%
        \@ehc
      \fi
    \endgroup
    \def\hologoDriver{#2}%
  }%
}%
\def\HOLOGO@@temp#1#2{%
  \ifx\kv@value\relax
    \HOLOGO@temp{#1}{#1}%
  \else
    \HOLOGO@temp{#1}{#2}%
  \fi
}%
\kv@parse@normalized{%
  pdftex,%
  luatex=pdftex,%
  dvipdfm,%
  dvipdfmx=dvipdfm,%
  dvips,%
  dvipsone=dvips,%
  xdvi=dvips,%
  xetex,%
  vtex,%
}\HOLOGO@@temp
%    \end{macrocode}
%
%    \begin{macrocode}
\kv@define@key{HoLogoDriver}{driverfallback}{%
  \def\HOLOGO@DriverFallback{#1}%
}
%    \end{macrocode}
%
%    \begin{macro}{\HOLOGO@DriverFallback}
%    \begin{macrocode}
\def\HOLOGO@DriverFallback{dvips}
%    \end{macrocode}
%    \end{macro}
%
%    \begin{macro}{\hologoDriverSetup}
%    \begin{macrocode}
\def\hologoDriverSetup{%
  \let\hologoDriver\ltx@undefined
  \HOLOGO@DriverSetup
}
%    \end{macrocode}
%    \end{macro}
%
%    \begin{macro}{\HOLOGO@DriverSetup}
%    \begin{macrocode}
\def\HOLOGO@DriverSetup#1{%
  \kvsetkeys{HoLogoDriver}{#1}%
  \HOLOGO@CheckDriver
  \ltx@ifundefined{hologoDriver}{%
    \begingroup
    \edef\x{\endgroup
      \noexpand\kvsetkeys{HoLogoDriver}{\HOLOGO@DriverFallback}%
    }\x
  }{}%
  \@PackageInfoNoLine{hologo}{Using driver `\hologoDriver'}%
}
%    \end{macrocode}
%    \end{macro}
%
%    \begin{macro}{\HOLOGO@CheckDriver}
%    \begin{macrocode}
\def\HOLOGO@CheckDriver{%
  \ifpdf
    \def\hologoDriver{pdftex}%
    \let\HOLOGO@pdfliteral\pdfliteral
    \ifluatex
      \ifx\pdfextension\@undefined\else
        \protected\def\pdfliteral{\pdfextension literal}%
        \let\HOLOGO@pdfliteral\pdfliteral
      \fi
      \ltx@IfUndefined{HOLOGO@pdfliteral}{%
        \ifnum\luatexversion<36 %
        \else
          \begingroup
            \let\HOLOGO@temp\endgroup
            \ifcase0%
                \directlua{%
                  if tex.enableprimitives then %
                    tex.enableprimitives('HOLOGO@', {'pdfliteral'})%
                  else %
                    tex.print('1')%
                  end%
                }%
                \ifx\HOLOGO@pdfliteral\@undefined 1\fi%
                \relax%
              \endgroup
              \let\HOLOGO@temp\relax
              \global\let\HOLOGO@pdfliteral\HOLOGO@pdfliteral
            \fi%
          \HOLOGO@temp
        \fi
      }{}%
    \fi
    \ltx@IfUndefined{HOLOGO@pdfliteral}{%
      \@PackageWarningNoLine{hologo}{%
        Cannot find \string\pdfliteral
      }%
    }{}%
  \else
    \ifxetex
      \def\hologoDriver{xetex}%
    \else
      \ifvtex
        \def\hologoDriver{vtex}%
      \fi
    \fi
  \fi
}
%    \end{macrocode}
%    \end{macro}
%
%    \begin{macro}{\HOLOGO@WarningUnsupportedDriver}
%    \begin{macrocode}
\def\HOLOGO@WarningUnsupportedDriver#1{%
  \@PackageWarningNoLine{hologo}{%
    Logo `#1' needs driver specific macros,\MessageBreak
    but driver `\hologoDriver' is not supported.\MessageBreak
    Use a different driver or\MessageBreak
    load package `graphics' or `pgf'%
  }%
}
%    \end{macrocode}
%    \end{macro}
%
% \subsubsection{Reflect box macros}
%
%    Skip driver part if not needed.
%    \begin{macrocode}
\ltx@IfUndefined{reflectbox}{}{%
  \ltx@IfUndefined{rotatebox}{}{%
    \HOLOGO@AtEnd
  }%
}
\ltx@IfUndefined{pgftext}{}{%
  \HOLOGO@AtEnd
}
\ltx@IfUndefined{psscalebox}{}{%
  \HOLOGO@AtEnd
}
%    \end{macrocode}
%
%    \begin{macrocode}
\def\HOLOGO@temp{LaTeX2e}
\ifx\fmtname\HOLOGO@temp
  \RequirePackage{kvoptions}[2011/06/30]%
  \ProcessKeyvalOptions{HoLogoDriver}%
\fi
\HOLOGO@DriverSetup{}
%    \end{macrocode}
%
%    \begin{macro}{\HOLOGO@ReflectBox}
%    \begin{macrocode}
\def\HOLOGO@ReflectBox#1{%
  \begingroup
    \setbox\ltx@zero\hbox{\begingroup#1\endgroup}%
    \setbox\ltx@two\hbox{%
      \kern\wd\ltx@zero
      \csname HOLOGO@ScaleBox@\hologoDriver\endcsname{-1}{1}{%
        \hbox to 0pt{\copy\ltx@zero\hss}%
      }%
    }%
    \wd\ltx@two=\wd\ltx@zero
    \box\ltx@two
  \endgroup
}
%    \end{macrocode}
%    \end{macro}
%
%    \begin{macro}{\HOLOGO@PointReflectBox}
%    \begin{macrocode}
\def\HOLOGO@PointReflectBox#1{%
  \begingroup
    \setbox\ltx@zero\hbox{\begingroup#1\endgroup}%
    \setbox\ltx@two\hbox{%
      \kern\wd\ltx@zero
      \raise\ht\ltx@zero\hbox{%
        \csname HOLOGO@ScaleBox@\hologoDriver\endcsname{-1}{-1}{%
          \hbox to 0pt{\copy\ltx@zero\hss}%
        }%
      }%
    }%
    \wd\ltx@two=\wd\ltx@zero
    \box\ltx@two
  \endgroup
}
%    \end{macrocode}
%    \end{macro}
%
%    We must define all variants because of dynamic driver setup.
%    \begin{macrocode}
\def\HOLOGO@temp#1#2{#2}
%    \end{macrocode}
%
%    \begin{macro}{\HOLOGO@ScaleBox@pdftex}
%    \begin{macrocode}
\HOLOGO@temp{pdftex}{%
  \def\HOLOGO@ScaleBox@pdftex#1#2#3{%
    \HOLOGO@pdfliteral{%
      q #1 0 0 #2 0 0 cm%
    }%
    #3%
    \HOLOGO@pdfliteral{%
      Q%
    }%
  }%
}
%    \end{macrocode}
%    \end{macro}
%    \begin{macro}{\HOLOGO@ScaleBox@dvips}
%    \begin{macrocode}
\HOLOGO@temp{dvips}{%
  \def\HOLOGO@ScaleBox@dvips#1#2#3{%
    \special{ps:%
      gsave %
      currentpoint %
      currentpoint translate %
      #1 #2 scale %
      neg exch neg exch translate%
    }%
    #3%
    \special{ps:%
      currentpoint %
      grestore %
      moveto%
    }%
  }%
}
%    \end{macrocode}
%    \end{macro}
%    \begin{macro}{\HOLOGO@ScaleBox@dvipdfm}
%    \begin{macrocode}
\HOLOGO@temp{dvipdfm}{%
  \let\HOLOGO@ScaleBox@dvipdfm\HOLOGO@ScaleBox@dvips
}
%    \end{macrocode}
%    \end{macro}
%    Since \hologo{XeTeX} v0.6.
%    \begin{macro}{\HOLOGO@ScaleBox@xetex}
%    \begin{macrocode}
\HOLOGO@temp{xetex}{%
  \def\HOLOGO@ScaleBox@xetex#1#2#3{%
    \special{x:gsave}%
    \special{x:scale #1 #2}%
    #3%
    \special{x:grestore}%
  }%
}
%    \end{macrocode}
%    \end{macro}
%    \begin{macro}{\HOLOGO@ScaleBox@vtex}
%    \begin{macrocode}
\HOLOGO@temp{vtex}{%
  \def\HOLOGO@ScaleBox@vtex#1#2#3{%
    \special{r(#1,0,0,#2,0,0}%
    #3%
    \special{r)}%
  }%
}
%    \end{macrocode}
%    \end{macro}
%
%    \begin{macrocode}
\HOLOGO@AtEnd%
%</package>
%    \end{macrocode}
%
% \section{Test}
%
% \subsection{Catcode checks for loading}
%
%    \begin{macrocode}
%<*test1>
%    \end{macrocode}
%    \begin{macrocode}
\catcode`\{=1 %
\catcode`\}=2 %
\catcode`\#=6 %
\catcode`\@=11 %
\expandafter\ifx\csname count@\endcsname\relax
  \countdef\count@=255 %
\fi
\expandafter\ifx\csname @gobble\endcsname\relax
  \long\def\@gobble#1{}%
\fi
\expandafter\ifx\csname @firstofone\endcsname\relax
  \long\def\@firstofone#1{#1}%
\fi
\expandafter\ifx\csname loop\endcsname\relax
  \expandafter\@firstofone
\else
  \expandafter\@gobble
\fi
{%
  \def\loop#1\repeat{%
    \def\body{#1}%
    \iterate
  }%
  \def\iterate{%
    \body
      \let\next\iterate
    \else
      \let\next\relax
    \fi
    \next
  }%
  \let\repeat=\fi
}%
\def\RestoreCatcodes{}
\count@=0 %
\loop
  \edef\RestoreCatcodes{%
    \RestoreCatcodes
    \catcode\the\count@=\the\catcode\count@\relax
  }%
\ifnum\count@<255 %
  \advance\count@ 1 %
\repeat

\def\RangeCatcodeInvalid#1#2{%
  \count@=#1\relax
  \loop
    \catcode\count@=15 %
  \ifnum\count@<#2\relax
    \advance\count@ 1 %
  \repeat
}
\def\RangeCatcodeCheck#1#2#3{%
  \count@=#1\relax
  \loop
    \ifnum#3=\catcode\count@
    \else
      \errmessage{%
        Character \the\count@\space
        with wrong catcode \the\catcode\count@\space
        instead of \number#3%
      }%
    \fi
  \ifnum\count@<#2\relax
    \advance\count@ 1 %
  \repeat
}
\def\space{ }
\expandafter\ifx\csname LoadCommand\endcsname\relax
  \def\LoadCommand{\input hologo.sty\relax}%
\fi
\def\Test{%
  \RangeCatcodeInvalid{0}{47}%
  \RangeCatcodeInvalid{58}{64}%
  \RangeCatcodeInvalid{91}{96}%
  \RangeCatcodeInvalid{123}{255}%
  \catcode`\@=12 %
  \catcode`\\=0 %
  \catcode`\%=14 %
  \LoadCommand
  \RangeCatcodeCheck{0}{36}{15}%
  \RangeCatcodeCheck{37}{37}{14}%
  \RangeCatcodeCheck{38}{47}{15}%
  \RangeCatcodeCheck{48}{57}{12}%
  \RangeCatcodeCheck{58}{63}{15}%
  \RangeCatcodeCheck{64}{64}{12}%
  \RangeCatcodeCheck{65}{90}{11}%
  \RangeCatcodeCheck{91}{91}{15}%
  \RangeCatcodeCheck{92}{92}{0}%
  \RangeCatcodeCheck{93}{96}{15}%
  \RangeCatcodeCheck{97}{122}{11}%
  \RangeCatcodeCheck{123}{255}{15}%
  \RestoreCatcodes
}
\Test
\csname @@end\endcsname
\end
%    \end{macrocode}
%    \begin{macrocode}
%</test1>
%    \end{macrocode}
%
% \subsection{Spacefactor}
%
%    The space factor must be 1000 after a logo. If it is greater 1000
%    then the following space is a space after a sentence closing point.
%    If the space factor is smaller 1000 then an immediate following
%    dot is interpreted as abbreviation, not sentence closing point.
%
%    \begin{macrocode}
%<*test-spacefactor>
\NeedsTeXFormat{LaTeX2e}
\documentclass{article}
\usepackage{hologo}[2016/05/12]
\usepackage{kvsetkeys}
\usepackage{qstest}
\IncludeTests{*}
\LogTests{log}{*}{*}
\begin{document}
\begin{qstest}{spacefactor}{spacefactor}
\newcommand*{\Test}[1]{%
  \sbox0{%
    \hologo{#1}%
    \Expect*{1000 (#1)}*{\the\spacefactor\space(#1)}%
  }%
}%
\makeatletter
\def\TestList{}
\def\hologoEntry#1#2#3{%
  \edef\TestList{%
    \ifx\TestList\@empty
    \else
      \TestList,%
    \fi
    #1%
    \ifx\\#2\\%
    \else
      ={variant=#2}%
    \fi
  }%
}
\hologoList
\expandafter\kv@parse@normalized\expandafter{%
  \TestList
}{%
  \begingroup
    \let\@logo=\kv@key
    \ifx\kv@value\relax
    \else
      \expandafter\hologoLogoSetup\expandafter\@logo\expandafter{%
        \kv@value
      }%
    \fi
    \Test\@logo
  \endgroup
  \@gobbletwo
}
\end{qstest}
\end{document}
%</test-spacefactor>
%    \end{macrocode}
%
% \subsection{Complete list}
%
%    \begin{macrocode}
%<*test-list>
\NeedsTeXFormat{LaTeX2e}
\documentclass[12pt,a4paper]{article}
\usepackage{hologo}[2016/05/12]
\usepackage[T1]{fontenc}
\usepackage{lmodern}
\usepackage{parskip}
\usepackage[unicode]{hyperref}[2011/09/28]
\usepackage{bookmark}[2011/09/19]
\bookmarksetup{%
  numbered,%
  open,%
  openlevel=2,%
}
\renewcommand*{\contentsname}{List of logos}
\begin{document}
\tableofcontents
\def\TestFont#1#2#3#4#5#6{%
  \begingroup
    \usefont{#3}{#4}{#5}{#6}%
    \HologoVariant{#1}{#2}/\hologoVariant{#1}{#2}%
    \quad
    \begingroup\scriptsize\hologoVariant{#1}{#2}\endgroup
    \quad
  \endgroup
  (#3/#4/#5/#6)%
  \par
}
\makeatletter
\def\hologoEntry#1#2#3{%
  \section{%
    \HologoVariant{#1}{#2}/\hologoVariant{#1}{#2} %
    {[#1\ifx\\#2\\\else\space(#2)\fi]}% hash-ok
  }% braces around [] because of bug in tex4ht
  \begingroup
    \hypersetup{unicode=false}%
    \bookmark[%
      dest=\@currentHref,%
      rellevel=1,%
      keeplevel,%
    ]{%
      \HologoVariant{#1}{#2}/\hologoVariant{#1}{#2} %
      (PDFDocEncoding)%
    }%
  \endgroup
  \TestFont{#1}{#2}{OT1}{cmr}{m}{n}%
  \TestFont{#1}{#2}{OT1}{cmss}{m}{n}%
  \TestFont{#1}{#2}{OT1}{cmr}{b}{n}%
  \TestFont{#1}{#2}{OT1}{cmr}{m}{it}%
  \TestFont{#1}{#2}{OT1}{cmtt}{m}{n}%
  \TestFont{#1}{#2}{T1}{lmr}{m}{n}%
  \TestFont{#1}{#2}{T1}{lmss}{m}{n}%
  \TestFont{#1}{#2}{T1}{lmr}{b}{n}%
  \TestFont{#1}{#2}{T1}{lmr}{m}{it}%
  \TestFont{#1}{#2}{T1}{lmtt}{m}{n}%
  \TestFont{#1}{#2}{T1}{lmvtt}{m}{n}%
  \TestFont{#1}{#2}{T1}{qtm}{m}{n}%
  \TestFont{#1}{#2}{T1}{qhv}{m}{n}%
  \TestFont{#1}{#2}{T1}{qtm}{b}{n}%
  \TestFont{#1}{#2}{T1}{qtm}{m}{it}%
  \TestFont{#1}{#2}{T1}{qcr}{m}{n}%
  \newpage
}
\makeatother
\hologoList
\end{document}
%</test-list>
%    \end{macrocode}
%
% \section{Installation}
%
% \subsection{Download}
%
% \paragraph{Package.} This package is available on
% CTAN\footnote{\url{ftp://ftp.ctan.org/tex-archive/}}:
% \begin{description}
% \item[\CTAN{macros/latex/contrib/oberdiek/hologo.dtx}] The source file.
% \item[\CTAN{macros/latex/contrib/oberdiek/hologo.pdf}] Documentation.
% \end{description}
%
%
% \paragraph{Bundle.} All the packages of the bundle `oberdiek'
% are also available in a TDS compliant ZIP archive. There
% the packages are already unpacked and the documentation files
% are generated. The files and directories obey the TDS standard.
% \begin{description}
% \item[\CTAN{install/macros/latex/contrib/oberdiek.tds.zip}]
% \end{description}
% \emph{TDS} refers to the standard ``A Directory Structure
% for \TeX\ Files'' (\CTAN{tds/tds.pdf}). Directories
% with \xfile{texmf} in their name are usually organized this way.
%
% \subsection{Bundle installation}
%
% \paragraph{Unpacking.} Unpack the \xfile{oberdiek.tds.zip} in the
% TDS tree (also known as \xfile{texmf} tree) of your choice.
% Example (linux):
% \begin{quote}
%   |unzip oberdiek.tds.zip -d ~/texmf|
% \end{quote}
%
% \paragraph{Script installation.}
% Check the directory \xfile{TDS:scripts/oberdiek/} for
% scripts that need further installation steps.
% Package \xpackage{attachfile2} comes with the Perl script
% \xfile{pdfatfi.pl} that should be installed in such a way
% that it can be called as \texttt{pdfatfi}.
% Example (linux):
% \begin{quote}
%   |chmod +x scripts/oberdiek/pdfatfi.pl|\\
%   |cp scripts/oberdiek/pdfatfi.pl /usr/local/bin/|
% \end{quote}
%
% \subsection{Package installation}
%
% \paragraph{Unpacking.} The \xfile{.dtx} file is a self-extracting
% \docstrip\ archive. The files are extracted by running the
% \xfile{.dtx} through \plainTeX:
% \begin{quote}
%   \verb|tex hologo.dtx|
% \end{quote}
%
% \paragraph{TDS.} Now the different files must be moved into
% the different directories in your installation TDS tree
% (also known as \xfile{texmf} tree):
% \begin{quote}
% \def\t{^^A
% \begin{tabular}{@{}>{\ttfamily}l@{ $\rightarrow$ }>{\ttfamily}l@{}}
%   hologo.sty & tex/generic/oberdiek/hologo.sty\\
%   hologo.pdf & doc/latex/oberdiek/hologo.pdf\\
%   example/hologo-example.tex & doc/latex/oberdiek/example/hologo-example.tex\\
%   test/hologo-test1.tex & doc/latex/oberdiek/test/hologo-test1.tex\\
%   test/hologo-test-spacefactor.tex & doc/latex/oberdiek/test/hologo-test-spacefactor.tex\\
%   test/hologo-test-list.tex & doc/latex/oberdiek/test/hologo-test-list.tex\\
%   hologo.dtx & source/latex/oberdiek/hologo.dtx\\
% \end{tabular}^^A
% }^^A
% \sbox0{\t}^^A
% \ifdim\wd0>\linewidth
%   \begingroup
%     \advance\linewidth by\leftmargin
%     \advance\linewidth by\rightmargin
%   \edef\x{\endgroup
%     \def\noexpand\lw{\the\linewidth}^^A
%   }\x
%   \def\lwbox{^^A
%     \leavevmode
%     \hbox to \linewidth{^^A
%       \kern-\leftmargin\relax
%       \hss
%       \usebox0
%       \hss
%       \kern-\rightmargin\relax
%     }^^A
%   }^^A
%   \ifdim\wd0>\lw
%     \sbox0{\small\t}^^A
%     \ifdim\wd0>\linewidth
%       \ifdim\wd0>\lw
%         \sbox0{\footnotesize\t}^^A
%         \ifdim\wd0>\linewidth
%           \ifdim\wd0>\lw
%             \sbox0{\scriptsize\t}^^A
%             \ifdim\wd0>\linewidth
%               \ifdim\wd0>\lw
%                 \sbox0{\tiny\t}^^A
%                 \ifdim\wd0>\linewidth
%                   \lwbox
%                 \else
%                   \usebox0
%                 \fi
%               \else
%                 \lwbox
%               \fi
%             \else
%               \usebox0
%             \fi
%           \else
%             \lwbox
%           \fi
%         \else
%           \usebox0
%         \fi
%       \else
%         \lwbox
%       \fi
%     \else
%       \usebox0
%     \fi
%   \else
%     \lwbox
%   \fi
% \else
%   \usebox0
% \fi
% \end{quote}
% If you have a \xfile{docstrip.cfg} that configures and enables \docstrip's
% TDS installing feature, then some files can already be in the right
% place, see the documentation of \docstrip.
%
% \subsection{Refresh file name databases}
%
% If your \TeX~distribution
% (\teTeX, \mikTeX, \dots) relies on file name databases, you must refresh
% these. For example, \teTeX\ users run \verb|texhash| or
% \verb|mktexlsr|.
%
% \subsection{Some details for the interested}
%
% \paragraph{Attached source.}
%
% The PDF documentation on CTAN also includes the
% \xfile{.dtx} source file. It can be extracted by
% AcrobatReader 6 or higher. Another option is \textsf{pdftk},
% e.g. unpack the file into the current directory:
% \begin{quote}
%   \verb|pdftk hologo.pdf unpack_files output .|
% \end{quote}
%
% \paragraph{Unpacking with \LaTeX.}
% The \xfile{.dtx} chooses its action depending on the format:
% \begin{description}
% \item[\plainTeX:] Run \docstrip\ and extract the files.
% \item[\LaTeX:] Generate the documentation.
% \end{description}
% If you insist on using \LaTeX\ for \docstrip\ (really,
% \docstrip\ does not need \LaTeX), then inform the autodetect routine
% about your intention:
% \begin{quote}
%   \verb|latex \let\install=y\input{hologo.dtx}|
% \end{quote}
% Do not forget to quote the argument according to the demands
% of your shell.
%
% \paragraph{Generating the documentation.}
% You can use both the \xfile{.dtx} or the \xfile{.drv} to generate
% the documentation. The process can be configured by the
% configuration file \xfile{ltxdoc.cfg}. For instance, put this
% line into this file, if you want to have A4 as paper format:
% \begin{quote}
%   \verb|\PassOptionsToClass{a4paper}{article}|
% \end{quote}
% An example follows how to generate the
% documentation with pdf\LaTeX:
% \begin{quote}
%\begin{verbatim}
%pdflatex hologo.dtx
%makeindex -s gind.ist hologo.idx
%pdflatex hologo.dtx
%makeindex -s gind.ist hologo.idx
%pdflatex hologo.dtx
%\end{verbatim}
% \end{quote}
%
% \section{Catalogue}
%
% The following XML file can be used as source for the
% \href{http://mirror.ctan.org/help/Catalogue/catalogue.html}{\TeX\ Catalogue}.
% The elements \texttt{caption} and \texttt{description} are imported
% from the original XML file from the Catalogue.
% The name of the XML file in the Catalogue is \xfile{hologo.xml}.
%    \begin{macrocode}
%<*catalogue>
<?xml version='1.0' encoding='us-ascii'?>
<!DOCTYPE entry SYSTEM 'catalogue.dtd'>
<entry datestamp='$Date$' modifier='$Author$' id='hologo'>
  <name>hologo</name>
  <caption>A collection of logos with bookmark support.</caption>
  <authorref id='auth:oberdiek'/>
  <copyright owner='Heiko Oberdiek' year='2010-2012'/>
  <license type='lppl1.3'/>
  <version number='1.10'/>
  <description>
    The package defines a single command <tt>\hologo</tt>, whose
    argument is the usual case-confused ASCII version of the logo.
    The command is bookmark-enabled, so that every logo becomes
    available in bookmarks without further work.
    <p/>
    The package is part of the <xref refid='oberdiek'>oberdiek</xref>
    bundle.
  </description>
  <documentation details='Package documentation'
      href='ctan:/macros/latex/contrib/oberdiek/hologo.pdf'/>
  <ctan file='true' path='/macros/latex/contrib/oberdiek/hologo.dtx'/>
  <miktex location='oberdiek'/>
  <texlive location='oberdiek'/>
  <install path='/macros/latex/contrib/oberdiek/oberdiek.tds.zip'/>
</entry>
%</catalogue>
%    \end{macrocode}
%
% \begin{thebibliography}{9}
% \raggedright
%
% \bibitem{btxdoc}
% Oren Patashnik,
% \textit{\hologo{BibTeX}ing},
% 1988-02-08.\\
% \CTAN{biblio/bibtex/base/}
%
% \bibitem{dtklogos}
% Gerd Neugebauer, DANTE,
% \textit{Package \xpackage{dtklogos}},
% 2011-04-25.\\
% \CTAN{usergrps/dante/dtk/dtklogos.sty}
%
% \bibitem{etexman}
% The \hologo{NTS} Team,
% \textit{The \hologo{eTeX} manual},
% 1998-02.\\
% \CTAN{systems/e-tex/v2/doc/}
%
% \bibitem{ExTeX-FAQ}
% The \hologo{ExTeX} group,
% \textit{\hologo{ExTeX}: FAQ -- How is \hologo{ExTeX} typeset?},
% 2007-04-14.\\
% \url{http://www.extex.org/documentation/faq.html}
%
% \bibitem{LyX}
% %@MISC{ LyX,
% %  title = {{LyX 2.0.0 -- The Document Processor [Computer software and manual]}},
% %  author = {{The LyX Team}},
% %  howpublished = {Internet: http://www.lyx.org},
% %  year = {2011-05-08},
% %  note = {Retrieved May 10, 2011, from http://www.lyx.org},
% %  url = {http://www.lyx.org/}
% %}
% The \hologo{LyX} Team,
% \textit{\hologo{LyX} -- The Document Processor},
% 2011-05-08.\\
% \url{http://www.lyx.org/}
%
% \bibitem{OzTeX}
% Andrew Trevorrow,
% \hologo{OzTeX} FAQ: What is the correct way to typeset ``\hologo{OzTeX}''?,
% 2011-09-15 (visited).
% \url{http://www.trevorrow.com/oztex/ozfaq.html#oztex-logo}
%
% \bibitem{PiCTeX}
% Michael Wichura,
% \textit{The \hologo{PiCTeX} macro package},
% 1987-09-21.
% \CTAN{graphics/pictex/}
%
% \bibitem{scrlogo}
% Markus Kohm,
% \textit{\hologo{KOMAScript} Datei \xfile{scrlogo.dtx}},
% 2009-01-30.\\
% \CTAN{install/macros/latex/contrib/komascript.tds.zip}
%
% \end{thebibliography}
%
% \begin{History}
%   \begin{Version}{2010/04/08 v1.0}
%   \item
%     The first version.
%   \end{Version}
%   \begin{Version}{2010/04/16 v1.1}
%   \item
%     \cs{Hologo} added for support of logos at start of a sentence.
%   \item
%     \cs{hologoSetup} and \cs{hologoLogoSetup} added.
%   \item
%     Options \xoption{break}, \xoption{hyphenbreak}, \xoption{spacebreak}
%     added.
%   \item
%     Variant support added by option \xoption{variant}.
%   \end{Version}
%   \begin{Version}{2010/04/24 v1.2}
%   \item
%     \hologo{LaTeX3} added.
%   \item
%     \hologo{VTeX} added.
%   \end{Version}
%   \begin{Version}{2010/11/21 v1.3}
%   \item
%     \hologo{iniTeX}, \hologo{virTeX} added.
%   \end{Version}
%   \begin{Version}{2011/03/25 v1.4}
%   \item
%     \hologo{ConTeXt} with variants added.
%   \item
%     Option \xoption{discretionarybreak} added as refinement for
%     option \xoption{break}.
%   \end{Version}
%   \begin{Version}{2011/04/21 v1.5}
%   \item
%     Wrong TDS directory for test files fixed.
%   \end{Version}
%   \begin{Version}{2011/10/01 v1.6}
%   \item
%     Support for package \xpackage{tex4ht} added.
%   \item
%     Support for \cs{csname} added if \cs{ifincsname} is available.
%   \item
%     New logos:
%     \hologo{(La)TeX},
%     \hologo{biber},
%     \hologo{BibTeX} (\xoption{sc}, \xoption{sf}),
%     \hologo{emTeX},
%     \hologo{ExTeX},
%     \hologo{KOMAScript},
%     \hologo{La},
%     \hologo{LyX},
%     \hologo{MiKTeX},
%     \hologo{NTS},
%     \hologo{OzMF},
%     \hologo{OzMP},
%     \hologo{OzTeX},
%     \hologo{OzTtH},
%     \hologo{PCTeX},
%     \hologo{PiC},
%     \hologo{PiCTeX},
%     \hologo{METAFONT},
%     \hologo{MetaFun},
%     \hologo{METAPOST},
%     \hologo{MetaPost},
%     \hologo{SLiTeX} (\xoption{lift}, \xoption{narrow}, \xoption{simple}),
%     \hologo{SliTeX} (\xoption{narrow}, \xoption{simple}, \xoption{lift}),
%     \hologo{teTeX}.
%   \item
%     Fixes:
%     \hologo{iniTeX},
%     \hologo{pdfLaTeX},
%     \hologo{pdfTeX},
%     \hologo{virTeX}.
%   \item
%     \cs{hologoFontSetup} and \cs{hologoLogoFontSetup} added.
%   \item
%     \cs{hologoVariant} and \cs{HologoVariant} added.
%   \end{Version}
%   \begin{Version}{2011/11/22 v1.7}
%   \item
%     New logos:
%     \hologo{BibTeX8},
%     \hologo{LaTeXML},
%     \hologo{SageTeX},
%     \hologo{TeX4ht},
%     \hologo{TTH}.
%   \item
%     \hologo{Xe} and friends: Driver stuff fixed.
%   \item
%     \hologo{Xe} and friends: Support for italic added.
%   \item
%     \hologo{Xe} and friends: Package support for \xpackage{pgf}
%     and \xpackage{pstricks} added.
%   \end{Version}
%   \begin{Version}{2011/11/29 v1.8}
%   \item
%     New logos:
%     \hologo{HanTheThanh}.
%   \end{Version}
%   \begin{Version}{2011/12/21 v1.9}
%   \item
%     Patch for package \xpackage{ifxetex} added for the case that
%     \cs{newif} is undefined in \hologo{iniTeX}.
%   \item
%     Some fixes for \hologo{iniTeX}.
%   \end{Version}
%   \begin{Version}{2012/04/26 v1.10}
%   \item
%     Fix in bookmark version of logo ``\hologo{HanTheThanh}''.
%   \end{Version}
%   \begin{Version}{2016/05/12 v1.11}
%   \item
%     Update HOLOGO@IfCharExists (previously in texlive)
%   \item define pdfliteral in current luatex.
%   \end{Version}
% \end{History}
%
% \PrintIndex
%
% \Finale
\endinput
%
        \else
          \input hologo.cfg\relax
        \fi
      \fi
    }%
    \ltx@IfUndefined{newread}{%
      \chardef\HOLOGO@temp=15 %
      \def\HOLOGO@CheckRead{%
        \ifeof\HOLOGO@temp
          \HOLOGO@InputIfExists
        \else
          \ifcase\HOLOGO@temp
            \@PackageWarningNoLine{hologo}{%
              Configuration file ignored, because\MessageBreak
              a free read register could not be found%
            }%
          \else
            \begingroup
              \count\ltx@cclv=\HOLOGO@temp
              \advance\ltx@cclv by \ltx@minusone
              \edef\x{\endgroup
                \chardef\noexpand\HOLOGO@temp=\the\count\ltx@cclv
                \relax
              }%
            \x
          \fi
        \fi
      }%
    }{%
      \csname newread\endcsname\HOLOGO@temp
      \HOLOGO@InputIfExists
    }%
  }{%
    \edef\HOLOGO@temp{\pdf@filesize{hologo.cfg}}%
    \ifx\HOLOGO@temp\ltx@empty
    \else
      \ifnum\HOLOGO@temp>0 %
        \begingroup
          \def\x{LaTeX2e}%
        \expandafter\endgroup
        \ifx\fmtname\x
          % \iffalse meta-comment
%
% File: hologo.dtx
% Version: 2016/05/12 v1.11
% Info: A logo collection with bookmark support
%
% Copyright (C) 2010-2012 by
%    Heiko Oberdiek <heiko.oberdiek at googlemail.com>
%
% This work may be distributed and/or modified under the
% conditions of the LaTeX Project Public License, either
% version 1.3c of this license or (at your option) any later
% version. This version of this license is in
%    http://www.latex-project.org/lppl/lppl-1-3c.txt
% and the latest version of this license is in
%    http://www.latex-project.org/lppl.txt
% and version 1.3 or later is part of all distributions of
% LaTeX version 2005/12/01 or later.
%
% This work has the LPPL maintenance status "maintained".
%
% This Current Maintainer of this work is Heiko Oberdiek.
%
% The Base Interpreter refers to any `TeX-Format',
% because some files are installed in TDS:tex/generic//.
%
% This work consists of the main source file hologo.dtx
% and the derived files
%    hologo.sty, hologo.pdf, hologo.ins, hologo.drv, hologo-example.tex,
%    hologo-test1.tex, hologo-test-spacefactor.tex,
%    hologo-test-list.tex.
%
% Distribution:
%    CTAN:macros/latex/contrib/oberdiek/hologo.dtx
%    CTAN:macros/latex/contrib/oberdiek/hologo.pdf
%
% Unpacking:
%    (a) If hologo.ins is present:
%           tex hologo.ins
%    (b) Without hologo.ins:
%           tex hologo.dtx
%    (c) If you insist on using LaTeX
%           latex \let\install=y\input{hologo.dtx}
%        (quote the arguments according to the demands of your shell)
%
% Documentation:
%    (a) If hologo.drv is present:
%           latex hologo.drv
%    (b) Without hologo.drv:
%           latex hologo.dtx; ...
%    The class ltxdoc loads the configuration file ltxdoc.cfg
%    if available. Here you can specify further options, e.g.
%    use A4 as paper format:
%       \PassOptionsToClass{a4paper}{article}
%
%    Programm calls to get the documentation (example):
%       pdflatex hologo.dtx
%       makeindex -s gind.ist hologo.idx
%       pdflatex hologo.dtx
%       makeindex -s gind.ist hologo.idx
%       pdflatex hologo.dtx
%
% Installation:
%    TDS:tex/generic/oberdiek/hologo.sty
%    TDS:doc/latex/oberdiek/hologo.pdf
%    TDS:doc/latex/oberdiek/example/hologo-example.tex
%    TDS:doc/latex/oberdiek/test/hologo-test1.tex
%    TDS:doc/latex/oberdiek/test/hologo-test-spacefactor.tex
%    TDS:doc/latex/oberdiek/test/hologo-test-list.tex
%    TDS:source/latex/oberdiek/hologo.dtx
%
%<*ignore>
\begingroup
  \catcode123=1 %
  \catcode125=2 %
  \def\x{LaTeX2e}%
\expandafter\endgroup
\ifcase 0\ifx\install y1\fi\expandafter
         \ifx\csname processbatchFile\endcsname\relax\else1\fi
         \ifx\fmtname\x\else 1\fi\relax
\else\csname fi\endcsname
%</ignore>
%<*install>
\input docstrip.tex
\Msg{************************************************************************}
\Msg{* Installation}
\Msg{* Package: hologo 2016/05/12 v1.11 A logo collection with bookmark support (HO)}
\Msg{************************************************************************}

\keepsilent
\askforoverwritefalse

\let\MetaPrefix\relax
\preamble

This is a generated file.

Project: hologo
Version: 2016/05/12 v1.11

Copyright (C) 2010-2012 by
   Heiko Oberdiek <heiko.oberdiek at googlemail.com>

This work may be distributed and/or modified under the
conditions of the LaTeX Project Public License, either
version 1.3c of this license or (at your option) any later
version. This version of this license is in
   http://www.latex-project.org/lppl/lppl-1-3c.txt
and the latest version of this license is in
   http://www.latex-project.org/lppl.txt
and version 1.3 or later is part of all distributions of
LaTeX version 2005/12/01 or later.

This work has the LPPL maintenance status "maintained".

This Current Maintainer of this work is Heiko Oberdiek.

The Base Interpreter refers to any `TeX-Format',
because some files are installed in TDS:tex/generic//.

This work consists of the main source file hologo.dtx
and the derived files
   hologo.sty, hologo.pdf, hologo.ins, hologo.drv, hologo-example.tex,
   hologo-test1.tex, hologo-test-spacefactor.tex,
   hologo-test-list.tex.

\endpreamble
\let\MetaPrefix\DoubleperCent

\generate{%
  \file{hologo.ins}{\from{hologo.dtx}{install}}%
  \file{hologo.drv}{\from{hologo.dtx}{driver}}%
  \usedir{tex/generic/oberdiek}%
  \file{hologo.sty}{\from{hologo.dtx}{package}}%
  \usedir{doc/latex/oberdiek/example}%
  \file{hologo-example.tex}{\from{hologo.dtx}{example}}%
  \usedir{doc/latex/oberdiek/test}%
  \file{hologo-test1.tex}{\from{hologo.dtx}{test1}}%
  \file{hologo-test-spacefactor.tex}{\from{hologo.dtx}{test-spacefactor}}%
  \file{hologo-test-list.tex}{\from{hologo.dtx}{test-list}}%
  \nopreamble
  \nopostamble
  \usedir{source/latex/oberdiek/catalogue}%
  \file{hologo.xml}{\from{hologo.dtx}{catalogue}}%
}

\catcode32=13\relax% active space
\let =\space%
\Msg{************************************************************************}
\Msg{*}
\Msg{* To finish the installation you have to move the following}
\Msg{* file into a directory searched by TeX:}
\Msg{*}
\Msg{*     hologo.sty}
\Msg{*}
\Msg{* To produce the documentation run the file `hologo.drv'}
\Msg{* through LaTeX.}
\Msg{*}
\Msg{* Happy TeXing!}
\Msg{*}
\Msg{************************************************************************}

\endbatchfile
%</install>
%<*ignore>
\fi
%</ignore>
%<*driver>
\NeedsTeXFormat{LaTeX2e}
\ProvidesFile{hologo.drv}%
  [2016/05/12 v1.11 A logo collection with bookmark support (HO)]%
\documentclass{ltxdoc}
\usepackage{holtxdoc}[2011/11/22]
\usepackage{hologo}[2016/05/12]
\usepackage{longtable}
\usepackage{array}
\usepackage{paralist}
%\usepackage[T1]{fontenc}
%\usepackage{lmodern}
\begin{document}
  \DocInput{hologo.dtx}%
\end{document}
%</driver>
% \fi
%
%
% \CharacterTable
%  {Upper-case    \A\B\C\D\E\F\G\H\I\J\K\L\M\N\O\P\Q\R\S\T\U\V\W\X\Y\Z
%   Lower-case    \a\b\c\d\e\f\g\h\i\j\k\l\m\n\o\p\q\r\s\t\u\v\w\x\y\z
%   Digits        \0\1\2\3\4\5\6\7\8\9
%   Exclamation   \!     Double quote  \"     Hash (number) \#
%   Dollar        \$     Percent       \%     Ampersand     \&
%   Acute accent  \'     Left paren    \(     Right paren   \)
%   Asterisk      \*     Plus          \+     Comma         \,
%   Minus         \-     Point         \.     Solidus       \/
%   Colon         \:     Semicolon     \;     Less than     \<
%   Equals        \=     Greater than  \>     Question mark \?
%   Commercial at \@     Left bracket  \[     Backslash     \\
%   Right bracket \]     Circumflex    \^     Underscore    \_
%   Grave accent  \`     Left brace    \{     Vertical bar  \|
%   Right brace   \}     Tilde         \~}
%
% \GetFileInfo{hologo.drv}
%
% \title{The \xpackage{hologo} package}
% \date{2016/05/12 v1.11}
% \author{Heiko Oberdiek\\\xemail{heiko.oberdiek at googlemail.com}}
%
% \maketitle
%
% \begin{abstract}
% This package starts a collection of logos with support for bookmarks
% strings.
% \end{abstract}
%
% \tableofcontents
%
% \section{Documentation}
%
% \subsection{Logo macros}
%
% \begin{declcs}{hologo} \M{name}
% \end{declcs}
% Macro \cs{hologo} sets the logo with name \meta{name}.
% The following table shows the supported names.
%
% \begingroup
%   \def\hologoEntry#1#2#3{^^A
%     #1&#2&\hologoLogoSetup{#1}{variant=#2}\hologo{#1}&#3\tabularnewline
%   }
%   \begin{longtable}{>{\ttfamily}l>{\ttfamily}lll}
%     \rmfamily\bfseries{name} & \rmfamily\bfseries variant
%     & \bfseries logo & \bfseries since\\
%     \hline
%     \endhead
%     \hologoList
%   \end{longtable}
% \endgroup
%
% \begin{declcs}{Hologo} \M{name}
% \end{declcs}
% Macro \cs{Hologo} starts the logo \meta{name} with an uppercase
% letter. As an exception small greek letters are not converted
% to uppercase. Examples, see \hologo{eTeX} and \hologo{ExTeX}.
%
% \subsection{Setup macros}
%
% The package does not support package options, but the following
% setup macros can be used to set options.
%
% \begin{declcs}{hologoSetup} \M{key value list}
% \end{declcs}
% Macro \cs{hologoSetup} sets global options.
%
% \begin{declcs}{hologoLogoSetup} \M{logo} \M{key value list}
% \end{declcs}
% Some options can also be used to configure a logo.
% These settings take precedence over global option settings.
%
% \subsection{Options}\label{sec:options}
%
% There are boolean and string options:
% \begin{description}
% \item[Boolean option:]
% It takes |true| or |false|
% as value. If the value is omitted, then |true| is used.
% \item[String option:]
% A value must be given as string. (But the string might be empty.)
% \end{description}
% The following options can be used both in \cs{hologoSetup}
% and \cs{hologoLogoSetup}:
% \begin{description}
% \def\entry#1{\item[\xoption{#1}:]}
% \entry{break}
%   enables or disables line breaks inside the logo. This setting is
%   refined by options \xoption{hyphenbreak}, \xoption{spacebreak}
%   or \xoption{discretionarybreak}.
%   Default is |false|.
% \entry{hyphenbreak}
%   enables or disables the line break right after the hyphen character.
% \entry{spacebreak}
%   enables or disables line breaks at space characters.
% \entry{discretionarybreak}
%   enables or disables line breaks at hyphenation points
%   (inserted by \cs{-}).
% \end{description}
% Macro \cs{hologoLogoSetup} also knows:
% \begin{description}
% \item[\xoption{variant}:]
%   This is a string option. It specifies a variant of a logo that
%   must exist. An empty string selects the package default variant.
% \end{description}
% Example:
% \begin{quote}
%   |\hologoSetup{break=false}|\\
%   |\hologoLogoSetup{plainTeX}{variant=hyphen,hyphenbreak}|\\
%   Then ``plain-\TeX'' contains one break point after the hyphen.
% \end{quote}
%
% \subsection{Driver options}
%
% Sometimes graphical operations are needed to construct some
% glyphs (e.g.\ \hologo{XeTeX}). If package \xpackage{graphics}
% or package \xpackage{pgf} are found, then the macros are taken
% from there. Otherwise the packge defines its own operations
% and therefore needs the driver information. Many drivers are
% detected automatically (\hologo{pdfTeX}/\hologo{LuaTeX}
% in PDF mode, \hologo{XeTeX}, \hologo{VTeX}). These have precedence
% over a driver option. The driver can be given as package option
% or using \cs{hologoDriverSetup}.
% The following list contains the recognized driver options:
% \begin{itemize}
% \item \xoption{pdftex}, \xoption{luatex}
% \item \xoption{dvipdfm}, \xoption{dvipdfmx}
% \item \xoption{dvips}, \xoption{dvipsone}, \xoption{xdvi}
% \item \xoption{xetex}
% \item \xoption{vtex}
% \end{itemize}
% The left driver of a line is the driver name that is used internally.
% The following names are aliases for drivers that use the
% same method. Therefore the entry in the \xext{log} file for
% the used driver prints the internally used driver name.
% \begin{description}
% \item[\xoption{driverfallback}:]
%   This option expects a driver that is used,
%   if the driver could not be detected automatically.
% \end{description}
%
% \begin{declcs}{hologoDriverSetup} \M{driver option}
% \end{declcs}
% The driver can also be configured after package loading
% using \cs{hologoDriverSetup}, also the way for \hologo{plainTeX}
% to setup the driver.
%
% \subsection{Font setup}
%
% Some logos require a special font, but should also be usable by
% \hologo{plainTeX}. Therefore the package provides some ways
% to influence the font settings. The options below
% take font settings as values. Both font commands
% such as \cs{sffamily} and macros that take one argument
% like \cs{textsf} can be used.
%
% \begin{declcs}{hologoFontSetup} \M{key value list}
% \end{declcs}
% Macro \cs{hologoFontSetup} sets the fonts for all logos.
% Supported keys:
% \begin{description}
% \def\entry#1{\item[\xoption{#1}:]}
% \entry{general}
%   This font is used for all logos. The default is empty.
%   That means no special font is used.
% \entry{bibsf}
%   This font is used for
%   {\hologoLogoSetup{BibTeX}{variant=sf}\hologo{BibTeX}}
%   with variant \xoption{sf}.
% \entry{rm}
%   This font is a serif font. It is used for \hologo{ExTeX}.
% \entry{sc}
%   This font specifies a small caps font. It is used for
%   {\hologoLogoSetup{BibTeX}{variant=sc}\hologo{BibTeX}}
%   with variant \xoption{sc}.
% \entry{sf}
%   This font specifies a sans serif font. The default
%   is \cs{sffamily}, then \cs{sf} is tried. Otherwise
%   a warning is given. It is used by \hologo{KOMAScript}.
% \entry{sy}
%   This is the font for math symbols (e.g. cmsy).
%   It is used by \hologo{AmS}, \hologo{NTS}, \hologo{ExTeX}.
% \entry{logo}
%   \hologo{METAFONT} and \hologo{METAPOST} are using that font.
%   In \hologo{LaTeX} \cs{logofamily} is used and
%   the definitions of package \xpackage{mflogo} are used
%   if the package is not loaded.
%   Otherwise the \cs{tenlogo} is used and defined
%   if it does not already exists.
% \end{description}
%
% \begin{declcs}{hologoLogoFontSetup} \M{logo} \M{key value list}
% \end{declcs}
% Fonts can also be set for a logo or logo component separately,
% see the following list.
% The keys are the same as for \cs{hologoFontSetup}.
%
% \begin{longtable}{>{\ttfamily}l>{\sffamily}ll}
%   \meta{logo} & keys & result\\
%   \hline
%   \endhead
%   BibTeX & bibsf & {\hologoLogoSetup{BibTeX}{variant=sf}\hologo{BibTeX}}\\[.5ex]
%   BibTeX & sc & {\hologoLogoSetup{BibTeX}{variant=sc}\hologo{BibTeX}}\\[.5ex]
%   ExTeX & rm & \hologo{ExTeX}\\
%   SliTeX & rm & \hologo{SliTeX}\\[.5ex]
%   AmS & sy & \hologo{AmS}\\
%   ExTeX & sy & \hologo{ExTeX}\\
%   NTS & sy & \hologo{NTS}\\[.5ex]
%   KOMAScript & sf & \hologo{KOMAScript}\\[.5ex]
%   METAFONT & logo & \hologo{METAFONT}\\
%   METAPOST & logo & \hologo{METAPOST}\\[.5ex]
%   SliTeX & sc \hologo{SliTeX}
% \end{longtable}
%
% \subsubsection{Font order}
%
% For all logos the font \xoption{general} is applied first.
% Example:
%\begin{quote}
%|\hologoFontSetup{general=\color{red}}|
%\end{quote}
% will print red logos.
% Then if the font uses a special font \xoption{sf}, for example,
% the font is applied that is setup by \cs{hologoLogoFontSetup}.
% If this font is not setup, then the common font setup
% by \cs{hologoFontSetup} is used. Otherwise a warning is given,
% that there is no font configured.
%
% \subsection{Additional user macros}
%
% Usually a variant of a logo is configured by using
% \cs{hologoLogoSetup}, because it is bad style to mix
% different variants of the same logo in the same text.
% There the following macros are a convenience for testing.
%
% \begin{declcs}{hologoVariant} \M{name} \M{variant}\\
%   \cs{HologoVariant} \M{name} \M{variant}
% \end{declcs}
% Logo \meta{name} is set using \meta{variant} that specifies
% explicitely which variant of the macro is used. If the argument
% is empty, then the default form of the logo is used
% (configurable by \cs{hologoLogoSetup}).
%
% \cs{HologoVariant} is used if the logo is set in a context
% that needs an uppercase first letter (beginning of a sentence, \dots).
%
% \begin{declcs}{hologoList}\\
%   \cs{hologoEntry} \M{logo} \M{variant} \M{since}
% \end{declcs}
% Macro \cs{hologoList} contains all logos that are provided
% by the package including variants. The list consists of calls
% of \cs{hologoEntry} with three arguments starting with the
% logo name \meta{logo} and its variant \meta{variant}. An empty
% variant means the current default. Argument \meta{since} specifies
% with version of the package \xpackage{hologo} is needed to get
% the logo. If the logo is fixed, then the date gets updated.
% Therefore the date \meta{since} is not exactly the date of
% the first introduction, but rather the date of the latest fix.
%
% Before \cs{hologoList} can be used, macro \cs{hologoEntry} needs
% a definition. The example file in section \ref{sec:example}
% shows applications of \cs{hologoList}.
%
% \subsection{Supported contexts}
%
% Macros \cs{hologo} and friends support special contexts:
% \begin{itemize}
% \item \hologo{LaTeX}'s protection mechanism.
% \item Bookmarks of package \xpackage{hyperref}.
% \item Package \xpackage{tex4ht}.
% \item The macros can be used inside \cs{csname} constructs,
%   if \cs{ifincsname} is available (\hologo{pdfTeX}, \hologo{XeTeX},
%   \hologo{LuaTeX}).
% \end{itemize}
%
% \subsection{Example}
% \label{sec:example}
%
% The following example prints the logos in different fonts.
%    \begin{macrocode}
%<*example>
%<<verbatim
\NeedsTeXFormat{LaTeX2e}
\documentclass[a4paper]{article}
\usepackage[
  hmargin=20mm,
  vmargin=20mm,
]{geometry}
\pagestyle{empty}
\usepackage{hologo}[2016/05/12]
\usepackage{longtable}
\usepackage{array}
\setlength{\extrarowheight}{2pt}
\usepackage[T1]{fontenc}
\usepackage{lmodern}
\usepackage{pdflscape}
\usepackage[
  pdfencoding=auto,
]{hyperref}
\hypersetup{
  pdfauthor={Heiko Oberdiek},
  pdftitle={Example for package `hologo'},
  pdfsubject={Logos with fonts lmr, lmss, qtm, qpl, qhv},
}
\usepackage{bookmark}

% Print the logo list on the console

\begingroup
  \typeout{}%
  \typeout{*** Begin of logo list ***}%
  \newcommand*{\hologoEntry}[3]{%
    \typeout{#1 \ifx\\#2\\\else(#2) \fi[#3]}%
  }%
  \hologoList
  \typeout{*** End of logo list ***}%
  \typeout{}%
\endgroup

\begin{document}
\begin{landscape}

  \section{Example file for package `hologo'}

  % Table for font names

  \begin{longtable}{>{\bfseries}ll}
    \textbf{font} & \textbf{Font name}\\
    \hline
    lmr & Latin Modern Roman\\
    lmss & Latin Modern Sans\\
    qtm & \TeX\ Gyre Termes\\
    qhv & \TeX\ Gyre Heros\\
    qpl & \TeX\ Gyre Pagella\\
  \end{longtable}

  % Logo list with logos in different fonts

  \begingroup
    \newcommand*{\SetVariant}[2]{%
      \ifx\\#2\\%
      \else
        \hologoLogoSetup{#1}{variant=#2}%
      \fi
    }%
    \newcommand*{\hologoEntry}[3]{%
      \SetVariant{#1}{#2}%
      \raisebox{1em}[0pt][0pt]{\hypertarget{#1@#2}{}}%
      \bookmark[%
        dest={#1@#2},%
      ]{%
        #1\ifx\\#2\\\else\space(#2)\fi: \Hologo{#1}, \hologo{#1} %
        [Unicode]%
      }%
      \hypersetup{unicode=false}%
      \bookmark[%
        dest={#1@#2},%
      ]{%
        #1\ifx\\#2\\\else\space(#2)\fi: \Hologo{#1}, \hologo{#1} %
        [PDFDocEncoding]%
      }%
      \texttt{#1}%
      &%
      \texttt{#2}%
      &%
      \Hologo{#1}%
      &%
      \SetVariant{#1}{#2}%
      \hologo{#1}%
      &%
      \SetVariant{#1}{#2}%
      \fontfamily{qtm}\selectfont
      \hologo{#1}%
      &%
      \SetVariant{#1}{#2}%
      \fontfamily{qpl}\selectfont
      \hologo{#1}%
      &%
      \SetVariant{#1}{#2}%
      \textsf{\hologo{#1}}%
      &%
      \SetVariant{#1}{#2}%
      \fontfamily{qhv}\selectfont
      \hologo{#1}%
      \tabularnewline
    }%
    \begin{longtable}{llllllll}%
      \textbf{\textit{logo}} & \textbf{\textit{variant}} &
      \texttt{\string\Hologo} &
      \textbf{lmr} & \textbf{qtm} & \textbf{qpl} &
      \textbf{lmss} & \textbf{qhv}
      \tabularnewline
      \hline
      \endhead
      \hologoList
    \end{longtable}%
  \endgroup

\end{landscape}
\end{document}
%verbatim
%</example>
%    \end{macrocode}
%
% \StopEventually{
% }
%
% \section{Implementation}
%    \begin{macrocode}
%<*package>
%    \end{macrocode}
%    Reload check, especially if the package is not used with \LaTeX.
%    \begin{macrocode}
\begingroup\catcode61\catcode48\catcode32=10\relax%
  \catcode13=5 % ^^M
  \endlinechar=13 %
  \catcode35=6 % #
  \catcode39=12 % '
  \catcode44=12 % ,
  \catcode45=12 % -
  \catcode46=12 % .
  \catcode58=12 % :
  \catcode64=11 % @
  \catcode123=1 % {
  \catcode125=2 % }
  \expandafter\let\expandafter\x\csname ver@hologo.sty\endcsname
  \ifx\x\relax % plain-TeX, first loading
  \else
    \def\empty{}%
    \ifx\x\empty % LaTeX, first loading,
      % variable is initialized, but \ProvidesPackage not yet seen
    \else
      \expandafter\ifx\csname PackageInfo\endcsname\relax
        \def\x#1#2{%
          \immediate\write-1{Package #1 Info: #2.}%
        }%
      \else
        \def\x#1#2{\PackageInfo{#1}{#2, stopped}}%
      \fi
      \x{hologo}{The package is already loaded}%
      \aftergroup\endinput
    \fi
  \fi
\endgroup%
%    \end{macrocode}
%    Package identification:
%    \begin{macrocode}
\begingroup\catcode61\catcode48\catcode32=10\relax%
  \catcode13=5 % ^^M
  \endlinechar=13 %
  \catcode35=6 % #
  \catcode39=12 % '
  \catcode40=12 % (
  \catcode41=12 % )
  \catcode44=12 % ,
  \catcode45=12 % -
  \catcode46=12 % .
  \catcode47=12 % /
  \catcode58=12 % :
  \catcode64=11 % @
  \catcode91=12 % [
  \catcode93=12 % ]
  \catcode123=1 % {
  \catcode125=2 % }
  \expandafter\ifx\csname ProvidesPackage\endcsname\relax
    \def\x#1#2#3[#4]{\endgroup
      \immediate\write-1{Package: #3 #4}%
      \xdef#1{#4}%
    }%
  \else
    \def\x#1#2[#3]{\endgroup
      #2[{#3}]%
      \ifx#1\@undefined
        \xdef#1{#3}%
      \fi
      \ifx#1\relax
        \xdef#1{#3}%
      \fi
    }%
  \fi
\expandafter\x\csname ver@hologo.sty\endcsname
\ProvidesPackage{hologo}%
  [2016/05/12 v1.11 A logo collection with bookmark support (HO)]%
%    \end{macrocode}
%
%    \begin{macrocode}
\begingroup\catcode61\catcode48\catcode32=10\relax%
  \catcode13=5 % ^^M
  \endlinechar=13 %
  \catcode123=1 % {
  \catcode125=2 % }
  \catcode64=11 % @
  \def\x{\endgroup
    \expandafter\edef\csname HOLOGO@AtEnd\endcsname{%
      \endlinechar=\the\endlinechar\relax
      \catcode13=\the\catcode13\relax
      \catcode32=\the\catcode32\relax
      \catcode35=\the\catcode35\relax
      \catcode61=\the\catcode61\relax
      \catcode64=\the\catcode64\relax
      \catcode123=\the\catcode123\relax
      \catcode125=\the\catcode125\relax
    }%
  }%
\x\catcode61\catcode48\catcode32=10\relax%
\catcode13=5 % ^^M
\endlinechar=13 %
\catcode35=6 % #
\catcode64=11 % @
\catcode123=1 % {
\catcode125=2 % }
\def\TMP@EnsureCode#1#2{%
  \edef\HOLOGO@AtEnd{%
    \HOLOGO@AtEnd
    \catcode#1=\the\catcode#1\relax
  }%
  \catcode#1=#2\relax
}
\TMP@EnsureCode{10}{12}% ^^J
\TMP@EnsureCode{33}{12}% !
\TMP@EnsureCode{34}{12}% "
\TMP@EnsureCode{36}{3}% $
\TMP@EnsureCode{38}{4}% &
\TMP@EnsureCode{39}{12}% '
\TMP@EnsureCode{40}{12}% (
\TMP@EnsureCode{41}{12}% )
\TMP@EnsureCode{42}{12}% *
\TMP@EnsureCode{43}{12}% +
\TMP@EnsureCode{44}{12}% ,
\TMP@EnsureCode{45}{12}% -
\TMP@EnsureCode{46}{12}% .
\TMP@EnsureCode{47}{12}% /
\TMP@EnsureCode{58}{12}% :
\TMP@EnsureCode{59}{12}% ;
\TMP@EnsureCode{60}{12}% <
\TMP@EnsureCode{62}{12}% >
\TMP@EnsureCode{63}{12}% ?
\TMP@EnsureCode{91}{12}% [
\TMP@EnsureCode{93}{12}% ]
\TMP@EnsureCode{94}{7}% ^ (superscript)
\TMP@EnsureCode{95}{8}% _ (subscript)
\TMP@EnsureCode{96}{12}% `
\TMP@EnsureCode{124}{12}% |
\edef\HOLOGO@AtEnd{%
  \HOLOGO@AtEnd
  \escapechar\the\escapechar\relax
  \noexpand\endinput
}
\escapechar=92 %
%    \end{macrocode}
%
% \subsection{Logo list}
%
%    \begin{macro}{\hologoList}
%    \begin{macrocode}
\def\hologoList{%
  \hologoEntry{(La)TeX}{}{2011/10/01}%
  \hologoEntry{AmSLaTeX}{}{2010/04/16}%
  \hologoEntry{AmSTeX}{}{2010/04/16}%
  \hologoEntry{biber}{}{2011/10/01}%
  \hologoEntry{BibTeX}{}{2011/10/01}%
  \hologoEntry{BibTeX}{sf}{2011/10/01}%
  \hologoEntry{BibTeX}{sc}{2011/10/01}%
  \hologoEntry{BibTeX8}{}{2011/11/22}%
  \hologoEntry{ConTeXt}{}{2011/03/25}%
  \hologoEntry{ConTeXt}{narrow}{2011/03/25}%
  \hologoEntry{ConTeXt}{simple}{2011/03/25}%
  \hologoEntry{emTeX}{}{2010/04/26}%
  \hologoEntry{eTeX}{}{2010/04/08}%
  \hologoEntry{ExTeX}{}{2011/10/01}%
  \hologoEntry{HanTheThanh}{}{2011/11/29}%
  \hologoEntry{iniTeX}{}{2011/10/01}%
  \hologoEntry{KOMAScript}{}{2011/10/01}%
  \hologoEntry{La}{}{2010/05/08}%
  \hologoEntry{LaTeX}{}{2010/04/08}%
  \hologoEntry{LaTeX2e}{}{2010/04/08}%
  \hologoEntry{LaTeX3}{}{2010/04/24}%
  \hologoEntry{LaTeXe}{}{2010/04/08}%
  \hologoEntry{LaTeXML}{}{2011/11/22}%
  \hologoEntry{LaTeXTeX}{}{2011/10/01}%
  \hologoEntry{LuaLaTeX}{}{2010/04/08}%
  \hologoEntry{LuaTeX}{}{2010/04/08}%
  \hologoEntry{LyX}{}{2011/10/01}%
  \hologoEntry{METAFONT}{}{2011/10/01}%
  \hologoEntry{MetaFun}{}{2011/10/01}%
  \hologoEntry{METAPOST}{}{2011/10/01}%
  \hologoEntry{MetaPost}{}{2011/10/01}%
  \hologoEntry{MiKTeX}{}{2011/10/01}%
  \hologoEntry{NTS}{}{2011/10/01}%
  \hologoEntry{OzMF}{}{2011/10/01}%
  \hologoEntry{OzMP}{}{2011/10/01}%
  \hologoEntry{OzTeX}{}{2011/10/01}%
  \hologoEntry{OzTtH}{}{2011/10/01}%
  \hologoEntry{PCTeX}{}{2011/10/01}%
  \hologoEntry{pdfTeX}{}{2011/10/01}%
  \hologoEntry{pdfLaTeX}{}{2011/10/01}%
  \hologoEntry{PiC}{}{2011/10/01}%
  \hologoEntry{PiCTeX}{}{2011/10/01}%
  \hologoEntry{plainTeX}{}{2010/04/08}%
  \hologoEntry{plainTeX}{space}{2010/04/16}%
  \hologoEntry{plainTeX}{hyphen}{2010/04/16}%
  \hologoEntry{plainTeX}{runtogether}{2010/04/16}%
  \hologoEntry{SageTeX}{}{2011/11/22}%
  \hologoEntry{SLiTeX}{}{2011/10/01}%
  \hologoEntry{SLiTeX}{lift}{2011/10/01}%
  \hologoEntry{SLiTeX}{narrow}{2011/10/01}%
  \hologoEntry{SLiTeX}{simple}{2011/10/01}%
  \hologoEntry{SliTeX}{}{2011/10/01}%
  \hologoEntry{SliTeX}{narrow}{2011/10/01}%
  \hologoEntry{SliTeX}{simple}{2011/10/01}%
  \hologoEntry{SliTeX}{lift}{2011/10/01}%
  \hologoEntry{teTeX}{}{2011/10/01}%
  \hologoEntry{TeX}{}{2010/04/08}%
  \hologoEntry{TeX4ht}{}{2011/11/22}%
  \hologoEntry{TTH}{}{2011/11/22}%
  \hologoEntry{virTeX}{}{2011/10/01}%
  \hologoEntry{VTeX}{}{2010/04/24}%
  \hologoEntry{Xe}{}{2010/04/08}%
  \hologoEntry{XeLaTeX}{}{2010/04/08}%
  \hologoEntry{XeTeX}{}{2010/04/08}%
}
%    \end{macrocode}
%    \end{macro}
%
% \subsection{Load resources}
%
%    \begin{macrocode}
\begingroup\expandafter\expandafter\expandafter\endgroup
\expandafter\ifx\csname RequirePackage\endcsname\relax
  \def\TMP@RequirePackage#1[#2]{%
    \begingroup\expandafter\expandafter\expandafter\endgroup
    \expandafter\ifx\csname ver@#1.sty\endcsname\relax
      \input #1.sty\relax
    \fi
  }%
  \TMP@RequirePackage{ltxcmds}[2011/02/04]%
  \TMP@RequirePackage{infwarerr}[2010/04/08]%
  \TMP@RequirePackage{kvsetkeys}[2010/03/01]%
  \TMP@RequirePackage{kvdefinekeys}[2010/03/01]%
  \TMP@RequirePackage{pdftexcmds}[2010/04/01]%
  \TMP@RequirePackage{ifpdf}[2010/01/28]%
  \TMP@RequirePackage{ifluatex}[2010/03/01]%
  \ltx@IfUndefined{newif}{%
    \expandafter\let\csname newif\endcsname\ltx@newif
  }{}%
  \TMP@RequirePackage{ifxetex}[2009/01/23]%
  \TMP@RequirePackage{ifvtex}[2010/03/01]%
\else
  \RequirePackage{ltxcmds}[2011/02/04]%
  \RequirePackage{infwarerr}[2010/04/08]%
  \RequirePackage{kvsetkeys}[2010/03/01]%
  \RequirePackage{kvdefinekeys}[2010/03/01]%
  \RequirePackage{pdftexcmds}[2010/04/01]%
  \RequirePackage{ifpdf}[2010/01/28]%
  \RequirePackage{ifluatex}[2010/03/01]%
  \RequirePackage{ifxetex}[2009/01/23]%
  \RequirePackage{ifvtex}[2010/03/01]%
\fi
%    \end{macrocode}
%
%    \begin{macro}{\HOLOGO@IfDefined}
%    \begin{macrocode}
\def\HOLOGO@IfExists#1{%
  \ifx\@undefined#1%
    \expandafter\ltx@secondoftwo
  \else
    \ifx\relax#1%
      \expandafter\ltx@secondoftwo
    \else
      \expandafter\expandafter\expandafter\ltx@firstoftwo
    \fi
  \fi
}
%    \end{macrocode}
%    \end{macro}
%
% \subsection{Setup macros}
%
%    \begin{macro}{\hologoSetup}
%    \begin{macrocode}
\def\hologoSetup{%
  \let\HOLOGO@name\relax
  \HOLOGO@Setup
}
%    \end{macrocode}
%    \end{macro}
%
%    \begin{macro}{\hologoLogoSetup}
%    \begin{macrocode}
\def\hologoLogoSetup#1{%
  \edef\HOLOGO@name{#1}%
  \ltx@IfUndefined{HoLogo@\HOLOGO@name}{%
    \@PackageError{hologo}{%
      Unknown logo `\HOLOGO@name'%
    }\@ehc
    \ltx@gobble
  }{%
    \HOLOGO@Setup
  }%
}
%    \end{macrocode}
%    \end{macro}
%
%    \begin{macro}{\HOLOGO@Setup}
%    \begin{macrocode}
\def\HOLOGO@Setup{%
  \kvsetkeys{HoLogo}%
}
%    \end{macrocode}
%    \end{macro}
%
% \subsection{Options}
%
%    \begin{macro}{\HOLOGO@DeclareBoolOption}
%    \begin{macrocode}
\def\HOLOGO@DeclareBoolOption#1{%
  \expandafter\chardef\csname HOLOGOOPT@#1\endcsname\ltx@zero
  \kv@define@key{HoLogo}{#1}[true]{%
    \def\HOLOGO@temp{##1}%
    \ifx\HOLOGO@temp\HOLOGO@true
      \ifx\HOLOGO@name\relax
        \expandafter\chardef\csname HOLOGOOPT@#1\endcsname=\ltx@one
      \else
        \expandafter\chardef\csname
        HoLogoOpt@#1@\HOLOGO@name\endcsname\ltx@one
      \fi
      \HOLOGO@SetBreakAll{#1}%
    \else
      \ifx\HOLOGO@temp\HOLOGO@false
        \ifx\HOLOGO@name\relax
          \expandafter\chardef\csname HOLOGOOPT@#1\endcsname=\ltx@zero
        \else
          \expandafter\chardef\csname
          HoLogoOpt@#1@\HOLOGO@name\endcsname=\ltx@zero
        \fi
        \HOLOGO@SetBreakAll{#1}%
      \else
        \@PackageError{hologo}{%
          Unknown value `##1' for boolean option `#1'.\MessageBreak
          Known values are `true' and `false'%
        }\@ehc
      \fi
    \fi
  }%
}
%    \end{macrocode}
%    \end{macro}
%
%    \begin{macro}{\HOLOGO@SetBreakAll}
%    \begin{macrocode}
\def\HOLOGO@SetBreakAll#1{%
  \def\HOLOGO@temp{#1}%
  \ifx\HOLOGO@temp\HOLOGO@break
    \ifx\HOLOGO@name\relax
      \chardef\HOLOGOOPT@hyphenbreak=\HOLOGOOPT@break
      \chardef\HOLOGOOPT@spacebreak=\HOLOGOOPT@break
      \chardef\HOLOGOOPT@discretionarybreak=\HOLOGOOPT@break
    \else
      \expandafter\chardef
         \csname HoLogoOpt@hyphenbreak@\HOLOGO@name\endcsname=%
         \csname HoLogoOpt@break@\HOLOGO@name\endcsname
      \expandafter\chardef
         \csname HoLogoOpt@spacebreak@\HOLOGO@name\endcsname=%
         \csname HoLogoOpt@break@\HOLOGO@name\endcsname
      \expandafter\chardef
         \csname HoLogoOpt@discretionarybreak@\HOLOGO@name
             \endcsname=%
         \csname HoLogoOpt@break@\HOLOGO@name\endcsname
    \fi
  \fi
}
%    \end{macrocode}
%    \end{macro}
%
%    \begin{macro}{\HOLOGO@true}
%    \begin{macrocode}
\def\HOLOGO@true{true}
%    \end{macrocode}
%    \end{macro}
%    \begin{macro}{\HOLOGO@false}
%    \begin{macrocode}
\def\HOLOGO@false{false}
%    \end{macrocode}
%    \end{macro}
%    \begin{macro}{\HOLOGO@break}
%    \begin{macrocode}
\def\HOLOGO@break{break}
%    \end{macrocode}
%    \end{macro}
%
%    \begin{macrocode}
\HOLOGO@DeclareBoolOption{break}
\HOLOGO@DeclareBoolOption{hyphenbreak}
\HOLOGO@DeclareBoolOption{spacebreak}
\HOLOGO@DeclareBoolOption{discretionarybreak}
%    \end{macrocode}
%
%    \begin{macrocode}
\kv@define@key{HoLogo}{variant}{%
  \ifx\HOLOGO@name\relax
    \@PackageError{hologo}{%
      Option `variant' is not available in \string\hologoSetup,%
      \MessageBreak
      Use \string\hologoLogoSetup\space instead%
    }\@ehc
  \else
    \edef\HOLOGO@temp{#1}%
    \ifx\HOLOGO@temp\ltx@empty
      \expandafter
      \let\csname HoLogoOpt@variant@\HOLOGO@name\endcsname\@undefined
    \else
      \ltx@IfUndefined{HoLogo@\HOLOGO@name @\HOLOGO@temp}{%
        \@PackageError{hologo}{%
          Unknown variant `\HOLOGO@temp' of logo `\HOLOGO@name'%
        }\@ehc
      }{%
        \expandafter
        \let\csname HoLogoOpt@variant@\HOLOGO@name\endcsname
            \HOLOGO@temp
      }%
    \fi
  \fi
}
%    \end{macrocode}
%
%    \begin{macro}{\HOLOGO@Variant}
%    \begin{macrocode}
\def\HOLOGO@Variant#1{%
  #1%
  \ltx@ifundefined{HoLogoOpt@variant@#1}{%
  }{%
    @\csname HoLogoOpt@variant@#1\endcsname
  }%
}
%    \end{macrocode}
%    \end{macro}
%
% \subsection{Break/no-break support}
%
%    \begin{macro}{\HOLOGO@space}
%    \begin{macrocode}
\def\HOLOGO@space{%
  \ltx@ifundefined{HoLogoOpt@spacebreak@\HOLOGO@name}{%
    \ltx@ifundefined{HoLogoOpt@break@\HOLOGO@name}{%
      \chardef\HOLOGO@temp=\HOLOGOOPT@spacebreak
    }{%
      \chardef\HOLOGO@temp=%
        \csname HoLogoOpt@break@\HOLOGO@name\endcsname
    }%
  }{%
    \chardef\HOLOGO@temp=%
      \csname HoLogoOpt@spacebreak@\HOLOGO@name\endcsname
  }%
  \ifcase\HOLOGO@temp
    \penalty10000 %
  \fi
  \ltx@space
}
%    \end{macrocode}
%    \end{macro}
%
%    \begin{macro}{\HOLOGO@hyphen}
%    \begin{macrocode}
\def\HOLOGO@hyphen{%
  \ltx@ifundefined{HoLogoOpt@hyphenbreak@\HOLOGO@name}{%
    \ltx@ifundefined{HoLogoOpt@break@\HOLOGO@name}{%
      \chardef\HOLOGO@temp=\HOLOGOOPT@hyphenbreak
    }{%
      \chardef\HOLOGO@temp=%
        \csname HoLogoOpt@break@\HOLOGO@name\endcsname
    }%
  }{%
    \chardef\HOLOGO@temp=%
      \csname HoLogoOpt@hyphenbreak@\HOLOGO@name\endcsname
  }%
  \ifcase\HOLOGO@temp
    \ltx@mbox{-}%
  \else
    -%
  \fi
}
%    \end{macrocode}
%    \end{macro}
%
%    \begin{macro}{\HOLOGO@discretionary}
%    \begin{macrocode}
\def\HOLOGO@discretionary{%
  \ltx@ifundefined{HoLogoOpt@discretionarybreak@\HOLOGO@name}{%
    \ltx@ifundefined{HoLogoOpt@break@\HOLOGO@name}{%
      \chardef\HOLOGO@temp=\HOLOGOOPT@discretionarybreak
    }{%
      \chardef\HOLOGO@temp=%
        \csname HoLogoOpt@break@\HOLOGO@name\endcsname
    }%
  }{%
    \chardef\HOLOGO@temp=%
      \csname HoLogoOpt@discretionarybreak@\HOLOGO@name\endcsname
  }%
  \ifcase\HOLOGO@temp
  \else
    \-%
  \fi
}
%    \end{macrocode}
%    \end{macro}
%
%    \begin{macro}{\HOLOGO@mbox}
%    \begin{macrocode}
\def\HOLOGO@mbox#1{%
  \ltx@ifundefined{HoLogoOpt@break@\HOLOGO@name}{%
    \chardef\HOLOGO@temp=\HOLOGOOPT@hyphenbreak
  }{%
    \chardef\HOLOGO@temp=%
      \csname HoLogoOpt@break@\HOLOGO@name\endcsname
  }%
  \ifcase\HOLOGO@temp
    \ltx@mbox{#1}%
  \else
    #1%
  \fi
}
%    \end{macrocode}
%    \end{macro}
%
% \subsection{Font support}
%
%    \begin{macro}{\HoLogoFont@font}
%    \begin{tabular}{@{}ll@{}}
%    |#1|:& logo name\\
%    |#2|:& font short name\\
%    |#3|:& text
%    \end{tabular}
%    \begin{macrocode}
\def\HoLogoFont@font#1#2#3{%
  \begingroup
    \ltx@IfUndefined{HoLogoFont@logo@#1.#2}{%
      \ltx@IfUndefined{HoLogoFont@font@#2}{%
        \@PackageWarning{hologo}{%
          Missing font `#2' for logo `#1'%
        }%
        #3%
      }{%
        \csname HoLogoFont@font@#2\endcsname{#3}%
      }%
    }{%
      \csname HoLogoFont@logo@#1.#2\endcsname{#3}%
    }%
  \endgroup
}
%    \end{macrocode}
%    \end{macro}
%
%    \begin{macro}{\HoLogoFont@Def}
%    \begin{macrocode}
\def\HoLogoFont@Def#1{%
  \expandafter\def\csname HoLogoFont@font@#1\endcsname
}
%    \end{macrocode}
%    \end{macro}
%    \begin{macro}{\HoLogoFont@LogoDef}
%    \begin{macrocode}
\def\HoLogoFont@LogoDef#1#2{%
  \expandafter\def\csname HoLogoFont@logo@#1.#2\endcsname
}
%    \end{macrocode}
%    \end{macro}
%
% \subsubsection{Font defaults}
%
%    \begin{macro}{\HoLogoFont@font@general}
%    \begin{macrocode}
\HoLogoFont@Def{general}{}%
%    \end{macrocode}
%    \end{macro}
%
%    \begin{macro}{\HoLogoFont@font@rm}
%    \begin{macrocode}
\ltx@IfUndefined{rmfamily}{%
  \ltx@IfUndefined{rm}{%
  }{%
    \HoLogoFont@Def{rm}{\rm}%
  }%
}{%
  \HoLogoFont@Def{rm}{\rmfamily}%
}
%    \end{macrocode}
%    \end{macro}
%
%    \begin{macro}{\HoLogoFont@font@sf}
%    \begin{macrocode}
\ltx@IfUndefined{sffamily}{%
  \ltx@IfUndefined{sf}{%
  }{%
    \HoLogoFont@Def{sf}{\sf}%
  }%
}{%
  \HoLogoFont@Def{sf}{\sffamily}%
}
%    \end{macrocode}
%    \end{macro}
%
%    \begin{macro}{\HoLogoFont@font@bibsf}
%    In case of \hologo{plainTeX} the original small caps
%    variant is used as default. In \hologo{LaTeX}
%    the definition of package \xpackage{dtklogos} \cite{dtklogos}
%    is used.
%\begin{quote}
%\begin{verbatim}
%\DeclareRobustCommand{\BibTeX}{%
%  B%
%  \kern-.05em%
%  \hbox{%
%    $\m@th$% %% force math size calculations
%    \csname S@\f@size\endcsname
%    \fontsize\sf@size\z@
%    \math@fontsfalse
%    \selectfont
%    I%
%    \kern-.025em%
%    B
%  }%
%  \kern-.08em%
%  \-%
%  \TeX
%}
%\end{verbatim}
%\end{quote}
%    \begin{macrocode}
\ltx@IfUndefined{selectfont}{%
  \ltx@IfUndefined{tensc}{%
    \font\tensc=cmcsc10\relax
  }{}%
  \HoLogoFont@Def{bibsf}{\tensc}%
}{%
  \HoLogoFont@Def{bibsf}{%
    $\mathsurround=0pt$%
    \csname S@\f@size\endcsname
    \fontsize\sf@size{0pt}%
    \math@fontsfalse
    \selectfont
  }%
}
%    \end{macrocode}
%    \end{macro}
%
%    \begin{macro}{\HoLogoFont@font@sc}
%    \begin{macrocode}
\ltx@IfUndefined{scshape}{%
  \ltx@IfUndefined{tensc}{%
    \font\tensc=cmcsc10\relax
  }{}%
  \HoLogoFont@Def{sc}{\tensc}%
}{%
  \HoLogoFont@Def{sc}{\scshape}%
}
%    \end{macrocode}
%    \end{macro}
%
%    \begin{macro}{\HoLogoFont@font@sy}
%    \begin{macrocode}
\ltx@IfUndefined{usefont}{%
  \ltx@IfUndefined{tensy}{%
  }{%
    \HoLogoFont@Def{sy}{\tensy}%
  }%
}{%
  \HoLogoFont@Def{sy}{%
    \usefont{OMS}{cmsy}{m}{n}%
  }%
}
%    \end{macrocode}
%    \end{macro}
%
%    \begin{macro}{\HoLogoFont@font@logo}
%    \begin{macrocode}
\begingroup
  \def\x{LaTeX2e}%
\expandafter\endgroup
\ifx\fmtname\x
  \ltx@IfUndefined{logofamily}{%
    \DeclareRobustCommand\logofamily{%
      \not@math@alphabet\logofamily\relax
      \fontencoding{U}%
      \fontfamily{logo}%
      \selectfont
    }%
  }{}%
  \ltx@IfUndefined{logofamily}{%
  }{%
    \HoLogoFont@Def{logo}{\logofamily}%
  }%
\else
  \ltx@IfUndefined{tenlogo}{%
    \font\tenlogo=logo10\relax
  }{}%
  \HoLogoFont@Def{logo}{\tenlogo}%
\fi
%    \end{macrocode}
%    \end{macro}
%
% \subsubsection{Font setup}
%
%    \begin{macro}{\hologoFontSetup}
%    \begin{macrocode}
\def\hologoFontSetup{%
  \let\HOLOGO@name\relax
  \HOLOGO@FontSetup
}
%    \end{macrocode}
%    \end{macro}
%
%    \begin{macro}{\hologoLogoFontSetup}
%    \begin{macrocode}
\def\hologoLogoFontSetup#1{%
  \edef\HOLOGO@name{#1}%
  \ltx@IfUndefined{HoLogo@\HOLOGO@name}{%
    \@PackageError{hologo}{%
      Unknown logo `\HOLOGO@name'%
    }\@ehc
    \ltx@gobble
  }{%
    \HOLOGO@FontSetup
  }%
}
%    \end{macrocode}
%    \end{macro}
%
%    \begin{macro}{\HOLOGO@FontSetup}
%    \begin{macrocode}
\def\HOLOGO@FontSetup{%
  \kvsetkeys{HoLogoFont}%
}
%    \end{macrocode}
%    \end{macro}
%
%    \begin{macrocode}
\def\HOLOGO@temp#1{%
  \kv@define@key{HoLogoFont}{#1}{%
    \ifx\HOLOGO@name\relax
      \HoLogoFont@Def{#1}{##1}%
    \else
      \HoLogoFont@LogoDef\HOLOGO@name{#1}{##1}%
    \fi
  }%
}
\HOLOGO@temp{general}
\HOLOGO@temp{sf}
%    \end{macrocode}
%
% \subsection{Generic logo commands}
%
%    \begin{macrocode}
\HOLOGO@IfExists\hologo{%
  \@PackageError{hologo}{%
    \string\hologo\ltx@space is already defined.\MessageBreak
    Package loading is aborted%
  }\@ehc
  \HOLOGO@AtEnd
}%
\HOLOGO@IfExists\hologoRobust{%
  \@PackageError{hologo}{%
    \string\hologoRobust\ltx@space is already defined.\MessageBreak
    Package loading is aborted%
  }\@ehc
  \HOLOGO@AtEnd
}%
%    \end{macrocode}
%
% \subsubsection{\cs{hologo} and friends}
%
%    \begin{macrocode}
\ifluatex
  \expandafter\ltx@firstofone
\else
  \expandafter\ltx@gobble
\fi
{%
  \ltx@IfUndefined{ifincsname}{%
    \ifnum\luatexversion<36 %
      \expandafter\ltx@gobble
    \else
      \expandafter\ltx@firstofone
    \fi
    {%
      \begingroup
        \ifcase0%
            \directlua{%
              if tex.enableprimitives then %
                tex.enableprimitives('HOLOGO@', {'ifincsname'})%
              else %
                tex.print('1')%
              end%
            }%
            \ifx\HOLOGO@ifincsname\@undefined 1\fi%
            \relax
          \expandafter\ltx@firstofone
        \else
          \endgroup
          \expandafter\ltx@gobble
        \fi
        {%
          \global\let\ifincsname\HOLOGO@ifincsname
        }%
      \HOLOGO@temp
    }%
  }{}%
}
%    \end{macrocode}
%    \begin{macrocode}
\ltx@IfUndefined{ifincsname}{%
  \catcode`$=14 %
}{%
  \catcode`$=9 %
}
%    \end{macrocode}
%
%    \begin{macro}{\hologo}
%    \begin{macrocode}
\def\hologo#1{%
$ \ifincsname
$   \ltx@ifundefined{HoLogoCs@\HOLOGO@Variant{#1}}{%
$     #1%
$   }{%
$     \csname HoLogoCs@\HOLOGO@Variant{#1}\endcsname\ltx@firstoftwo
$   }%
$ \else
    \HOLOGO@IfExists\texorpdfstring\texorpdfstring\ltx@firstoftwo
    {%
      \hologoRobust{#1}%
    }{%
      \ltx@ifundefined{HoLogoBkm@\HOLOGO@Variant{#1}}{%
        \ltx@ifundefined{HoLogo@#1}{?#1?}{#1}%
      }{%
        \csname HoLogoBkm@\HOLOGO@Variant{#1}\endcsname
        \ltx@firstoftwo
      }%
    }%
$ \fi
}
%    \end{macrocode}
%    \end{macro}
%    \begin{macro}{\Hologo}
%    \begin{macrocode}
\def\Hologo#1{%
$ \ifincsname
$   \ltx@ifundefined{HoLogoCs@\HOLOGO@Variant{#1}}{%
$     #1%
$   }{%
$     \csname HoLogoCs@\HOLOGO@Variant{#1}\endcsname\ltx@secondoftwo
$   }%
$ \else
    \HOLOGO@IfExists\texorpdfstring\texorpdfstring\ltx@firstoftwo
    {%
      \HologoRobust{#1}%
    }{%
      \ltx@ifundefined{HoLogoBkm@\HOLOGO@Variant{#1}}{%
        \ltx@ifundefined{HoLogo@#1}{?#1?}{#1}%
      }{%
        \csname HoLogoBkm@\HOLOGO@Variant{#1}\endcsname
        \ltx@secondoftwo
      }%
    }%
$ \fi
}
%    \end{macrocode}
%    \end{macro}
%
%    \begin{macro}{\hologoVariant}
%    \begin{macrocode}
\def\hologoVariant#1#2{%
  \ifx\relax#2\relax
    \hologo{#1}%
  \else
$   \ifincsname
$     \ltx@ifundefined{HoLogoCs@#1@#2}{%
$       #1%
$     }{%
$       \csname HoLogoCs@#1@#2\endcsname\ltx@firstoftwo
$     }%
$   \else
      \HOLOGO@IfExists\texorpdfstring\texorpdfstring\ltx@firstoftwo
      {%
        \hologoVariantRobust{#1}{#2}%
      }{%
        \ltx@ifundefined{HoLogoBkm@#1@#2}{%
          \ltx@ifundefined{HoLogo@#1}{?#1?}{#1}%
        }{%
          \csname HoLogoBkm@#1@#2\endcsname
          \ltx@firstoftwo
        }%
      }%
$   \fi
  \fi
}
%    \end{macrocode}
%    \end{macro}
%    \begin{macro}{\HologoVariant}
%    \begin{macrocode}
\def\HologoVariant#1#2{%
  \ifx\relax#2\relax
    \Hologo{#1}%
  \else
$   \ifincsname
$     \ltx@ifundefined{HoLogoCs@#1@#2}{%
$       #1%
$     }{%
$       \csname HoLogoCs@#1@#2\endcsname\ltx@secondoftwo
$     }%
$   \else
      \HOLOGO@IfExists\texorpdfstring\texorpdfstring\ltx@firstoftwo
      {%
        \HologoVariantRobust{#1}{#2}%
      }{%
        \ltx@ifundefined{HoLogoBkm@#1@#2}{%
          \ltx@ifundefined{HoLogo@#1}{?#1?}{#1}%
        }{%
          \csname HoLogoBkm@#1@#2\endcsname
          \ltx@secondoftwo
        }%
      }%
$   \fi
  \fi
}
%    \end{macrocode}
%    \end{macro}
%
%    \begin{macrocode}
\catcode`\$=3 %
%    \end{macrocode}
%
% \subsubsection{\cs{hologoRobust} and friends}
%
%    \begin{macro}{\hologoRobust}
%    \begin{macrocode}
\ltx@IfUndefined{protected}{%
  \ltx@IfUndefined{DeclareRobustCommand}{%
    \def\hologoRobust#1%
  }{%
    \DeclareRobustCommand*\hologoRobust[1]%
  }%
}{%
  \protected\def\hologoRobust#1%
}%
{%
  \edef\HOLOGO@name{#1}%
  \ltx@IfUndefined{HoLogo@\HOLOGO@Variant\HOLOGO@name}{%
    \@PackageError{hologo}{%
      Unknown logo `\HOLOGO@name'%
    }\@ehc
    ?\HOLOGO@name?%
  }{%
    \ltx@IfUndefined{ver@tex4ht.sty}{%
      \HoLogoFont@font\HOLOGO@name{general}{%
        \csname HoLogo@\HOLOGO@Variant\HOLOGO@name\endcsname
        \ltx@firstoftwo
      }%
    }{%
      \ltx@IfUndefined{HoLogoHtml@\HOLOGO@Variant\HOLOGO@name}{%
        \HOLOGO@name
      }{%
        \csname HoLogoHtml@\HOLOGO@Variant\HOLOGO@name\endcsname
        \ltx@firstoftwo
      }%
    }%
  }%
}
%    \end{macrocode}
%    \end{macro}
%    \begin{macro}{\HologoRobust}
%    \begin{macrocode}
\ltx@IfUndefined{protected}{%
  \ltx@IfUndefined{DeclareRobustCommand}{%
    \def\HologoRobust#1%
  }{%
    \DeclareRobustCommand*\HologoRobust[1]%
  }%
}{%
  \protected\def\HologoRobust#1%
}%
{%
  \edef\HOLOGO@name{#1}%
  \ltx@IfUndefined{HoLogo@\HOLOGO@Variant\HOLOGO@name}{%
    \@PackageError{hologo}{%
      Unknown logo `\HOLOGO@name'%
    }\@ehc
    ?\HOLOGO@name?%
  }{%
    \ltx@IfUndefined{ver@tex4ht.sty}{%
      \HoLogoFont@font\HOLOGO@name{general}{%
        \csname HoLogo@\HOLOGO@Variant\HOLOGO@name\endcsname
        \ltx@secondoftwo
      }%
    }{%
      \ltx@IfUndefined{HoLogoHtml@\HOLOGO@Variant\HOLOGO@name}{%
        \expandafter\HOLOGO@Uppercase\HOLOGO@name
      }{%
        \csname HoLogoHtml@\HOLOGO@Variant\HOLOGO@name\endcsname
        \ltx@secondoftwo
      }%
    }%
  }%
}
%    \end{macrocode}
%    \end{macro}
%    \begin{macro}{\hologoVariantRobust}
%    \begin{macrocode}
\ltx@IfUndefined{protected}{%
  \ltx@IfUndefined{DeclareRobustCommand}{%
    \def\hologoVariantRobust#1#2%
  }{%
    \DeclareRobustCommand*\hologoVariantRobust[2]%
  }%
}{%
  \protected\def\hologoVariantRobust#1#2%
}%
{%
  \begingroup
    \hologoLogoSetup{#1}{variant={#2}}%
    \hologoRobust{#1}%
  \endgroup
}
%    \end{macrocode}
%    \end{macro}
%    \begin{macro}{\HologoVariantRobust}
%    \begin{macrocode}
\ltx@IfUndefined{protected}{%
  \ltx@IfUndefined{DeclareRobustCommand}{%
    \def\HologoVariantRobust#1#2%
  }{%
    \DeclareRobustCommand*\HologoVariantRobust[2]%
  }%
}{%
  \protected\def\HologoVariantRobust#1#2%
}%
{%
  \begingroup
    \hologoLogoSetup{#1}{variant={#2}}%
    \HologoRobust{#1}%
  \endgroup
}
%    \end{macrocode}
%    \end{macro}
%
%    \begin{macro}{\hologorobust}
%    Macro \cs{hologorobust} is only defined for compatibility.
%    Its use is deprecated.
%    \begin{macrocode}
\def\hologorobust{\hologoRobust}
%    \end{macrocode}
%    \end{macro}
%
% \subsection{Helpers}
%
%    \begin{macro}{\HOLOGO@Uppercase}
%    Macro \cs{HOLOGO@Uppercase} is restricted to \cs{uppercase},
%    because \hologo{plainTeX} or \hologo{iniTeX} do not provide
%    \cs{MakeUppercase}.
%    \begin{macrocode}
\def\HOLOGO@Uppercase#1{\uppercase{#1}}
%    \end{macrocode}
%    \end{macro}
%
%    \begin{macro}{\HOLOGO@PdfdocUnicode}
%    \begin{macrocode}
\def\HOLOGO@PdfdocUnicode{%
  \ifx\ifHy@unicode\iftrue
    \expandafter\ltx@secondoftwo
  \else
    \expandafter\ltx@firstoftwo
  \fi
}
%    \end{macrocode}
%    \end{macro}
%
%    \begin{macro}{\HOLOGO@Math}
%    \begin{macrocode}
\def\HOLOGO@MathSetup{%
  \mathsurround0pt\relax
  \HOLOGO@IfExists\f@series{%
    \if b\expandafter\ltx@car\f@series x\@nil
      \csname boldmath\endcsname
   \fi
  }{}%
}
%    \end{macrocode}
%    \end{macro}
%
%    \begin{macro}{\HOLOGO@TempDimen}
%    \begin{macrocode}
\dimendef\HOLOGO@TempDimen=\ltx@zero
%    \end{macrocode}
%    \end{macro}
%    \begin{macro}{\HOLOGO@NegativeKerning}
%    \begin{macrocode}
\def\HOLOGO@NegativeKerning#1{%
  \begingroup
    \HOLOGO@TempDimen=0pt\relax
    \comma@parse@normalized{#1}{%
      \ifdim\HOLOGO@TempDimen=0pt %
        \expandafter\HOLOGO@@NegativeKerning\comma@entry
      \fi
      \ltx@gobble
    }%
    \ifdim\HOLOGO@TempDimen<0pt %
      \kern\HOLOGO@TempDimen
    \fi
  \endgroup
}
%    \end{macrocode}
%    \end{macro}
%    \begin{macro}{\HOLOGO@@NegativeKerning}
%    \begin{macrocode}
\def\HOLOGO@@NegativeKerning#1#2{%
  \setbox\ltx@zero\hbox{#1#2}%
  \HOLOGO@TempDimen=\wd\ltx@zero
  \setbox\ltx@zero\hbox{#1\kern0pt#2}%
  \advance\HOLOGO@TempDimen by -\wd\ltx@zero
}
%    \end{macrocode}
%    \end{macro}
%
%    \begin{macro}{\HOLOGO@SpaceFactor}
%    \begin{macrocode}
\def\HOLOGO@SpaceFactor{%
  \spacefactor1000 %
}
%    \end{macrocode}
%    \end{macro}
%
%    \begin{macro}{\HOLOGO@Span}
%    \begin{macrocode}
\def\HOLOGO@Span#1#2{%
  \HCode{<span class="HoLogo-#1">}%
  #2%
  \HCode{</span>}%
}
%    \end{macrocode}
%    \end{macro}
%
% \subsubsection{Text subscript}
%
%    \begin{macro}{\HOLOGO@SubScript}%
%    \begin{macrocode}
\def\HOLOGO@SubScript#1{%
  \ltx@IfUndefined{textsubscript}{%
    \ltx@IfUndefined{text}{%
      \ltx@mbox{%
        \mathsurround=0pt\relax
        $%
          _{%
            \ltx@IfUndefined{sf@size}{%
              \mathrm{#1}%
            }{%
              \mbox{%
                \fontsize\sf@size{0pt}\selectfont
                #1%
              }%
            }%
          }%
        $%
      }%
    }{%
      \ltx@mbox{%
        \mathsurround=0pt\relax
        $_{\text{#1}}$%
      }%
    }%
  }{%
    \textsubscript{#1}%
  }%
}
%    \end{macrocode}
%    \end{macro}
%
% \subsection{\hologo{TeX} and friends}
%
% \subsubsection{\hologo{TeX}}
%
%    \begin{macro}{\HoLogo@TeX}
%    Source: \hologo{LaTeX} kernel.
%    \begin{macrocode}
\def\HoLogo@TeX#1{%
  T\kern-.1667em\lower.5ex\hbox{E}\kern-.125emX\HOLOGO@SpaceFactor
}
%    \end{macrocode}
%    \end{macro}
%    \begin{macro}{\HoLogoHtml@TeX}
%    \begin{macrocode}
\def\HoLogoHtml@TeX#1{%
  \HoLogoCss@TeX
  \HOLOGO@Span{TeX}{%
    T%
    \HOLOGO@Span{e}{%
      E%
    }%
    X%
  }%
}
%    \end{macrocode}
%    \end{macro}
%    \begin{macro}{\HoLogoCss@TeX}
%    \begin{macrocode}
\def\HoLogoCss@TeX{%
  \Css{%
    span.HoLogo-TeX span.HoLogo-e{%
      position:relative;%
      top:.5ex;%
      margin-left:-.1667em;%
      margin-right:-.125em;%
    }%
  }%
  \Css{%
    a span.HoLogo-TeX span.HoLogo-e{%
      text-decoration:none;%
    }%
  }%
  \global\let\HoLogoCss@TeX\relax
}
%    \end{macrocode}
%    \end{macro}
%
% \subsubsection{\hologo{plainTeX}}
%
%    \begin{macro}{\HoLogo@plainTeX@space}
%    Source: ``The \hologo{TeX}book''
%    \begin{macrocode}
\def\HoLogo@plainTeX@space#1{%
  \HOLOGO@mbox{#1{p}{P}lain}\HOLOGO@space\hologo{TeX}%
}
%    \end{macrocode}
%    \end{macro}
%    \begin{macro}{\HoLogoCs@plainTeX@space}
%    \begin{macrocode}
\def\HoLogoCs@plainTeX@space#1{#1{p}{P}lain TeX}%
%    \end{macrocode}
%    \end{macro}
%    \begin{macro}{\HoLogoBkm@plainTeX@space}
%    \begin{macrocode}
\def\HoLogoBkm@plainTeX@space#1{%
  #1{p}{P}lain \hologo{TeX}%
}
%    \end{macrocode}
%    \end{macro}
%    \begin{macro}{\HoLogoHtml@plainTeX@space}
%    \begin{macrocode}
\def\HoLogoHtml@plainTeX@space#1{%
  #1{p}{P}lain \hologo{TeX}%
}
%    \end{macrocode}
%    \end{macro}
%
%    \begin{macro}{\HoLogo@plainTeX@hyphen}
%    \begin{macrocode}
\def\HoLogo@plainTeX@hyphen#1{%
  \HOLOGO@mbox{#1{p}{P}lain}\HOLOGO@hyphen\hologo{TeX}%
}
%    \end{macrocode}
%    \end{macro}
%    \begin{macro}{\HoLogoCs@plainTeX@hyphen}
%    \begin{macrocode}
\def\HoLogoCs@plainTeX@hyphen#1{#1{p}{P}lain-TeX}
%    \end{macrocode}
%    \end{macro}
%    \begin{macro}{\HoLogoBkm@plainTeX@hyphen}
%    \begin{macrocode}
\def\HoLogoBkm@plainTeX@hyphen#1{%
  #1{p}{P}lain-\hologo{TeX}%
}
%    \end{macrocode}
%    \end{macro}
%    \begin{macro}{\HoLogoHtml@plainTeX@hyphen}
%    \begin{macrocode}
\def\HoLogoHtml@plainTeX@hyphen#1{%
  #1{p}{P}lain-\hologo{TeX}%
}
%    \end{macrocode}
%    \end{macro}
%
%    \begin{macro}{\HoLogo@plainTeX@runtogether}
%    \begin{macrocode}
\def\HoLogo@plainTeX@runtogether#1{%
  \HOLOGO@mbox{#1{p}{P}lain\hologo{TeX}}%
}
%    \end{macrocode}
%    \end{macro}
%    \begin{macro}{\HoLogoCs@plainTeX@runtogether}
%    \begin{macrocode}
\def\HoLogoCs@plainTeX@runtogether#1{#1{p}{P}lainTeX}
%    \end{macrocode}
%    \end{macro}
%    \begin{macro}{\HoLogoBkm@plainTeX@runtogether}
%    \begin{macrocode}
\def\HoLogoBkm@plainTeX@runtogether#1{%
  #1{p}{P}lain\hologo{TeX}%
}
%    \end{macrocode}
%    \end{macro}
%    \begin{macro}{\HoLogoHtml@plainTeX@runtogether}
%    \begin{macrocode}
\def\HoLogoHtml@plainTeX@runtogether#1{%
  #1{p}{P}lain\hologo{TeX}%
}
%    \end{macrocode}
%    \end{macro}
%
%    \begin{macro}{\HoLogo@plainTeX}
%    \begin{macrocode}
\def\HoLogo@plainTeX{\HoLogo@plainTeX@space}
%    \end{macrocode}
%    \end{macro}
%    \begin{macro}{\HoLogoCs@plainTeX}
%    \begin{macrocode}
\def\HoLogoCs@plainTeX{\HoLogoCs@plainTeX@space}
%    \end{macrocode}
%    \end{macro}
%    \begin{macro}{\HoLogoBkm@plainTeX}
%    \begin{macrocode}
\def\HoLogoBkm@plainTeX{\HoLogoBkm@plainTeX@space}
%    \end{macrocode}
%    \end{macro}
%    \begin{macro}{\HoLogoHtml@plainTeX}
%    \begin{macrocode}
\def\HoLogoHtml@plainTeX{\HoLogoHtml@plainTeX@space}
%    \end{macrocode}
%    \end{macro}
%
% \subsubsection{\hologo{LaTeX}}
%
%    Source: \hologo{LaTeX} kernel.
%\begin{quote}
%\begin{verbatim}
%\DeclareRobustCommand{\LaTeX}{%
%  L%
%  \kern-.36em%
%  {%
%    \sbox\z@ T%
%    \vbox to\ht\z@{%
%      \hbox{%
%        \check@mathfonts
%        \fontsize\sf@size\z@
%        \math@fontsfalse
%        \selectfont
%        A%
%      }%
%      \vss
%    }%
%  }%
%  \kern-.15em%
%  \TeX
%}
%\end{verbatim}
%\end{quote}
%
%    \begin{macro}{\HoLogo@La}
%    \begin{macrocode}
\def\HoLogo@La#1{%
  L%
  \kern-.36em%
  \begingroup
    \setbox\ltx@zero\hbox{T}%
    \vbox to\ht\ltx@zero{%
      \hbox{%
        \ltx@ifundefined{check@mathfonts}{%
          \csname sevenrm\endcsname
        }{%
          \check@mathfonts
          \fontsize\sf@size{0pt}%
          \math@fontsfalse\selectfont
        }%
        A%
      }%
      \vss
    }%
  \endgroup
}
%    \end{macrocode}
%    \end{macro}
%
%    \begin{macro}{\HoLogo@LaTeX}
%    Source: \hologo{LaTeX} kernel.
%    \begin{macrocode}
\def\HoLogo@LaTeX#1{%
  \hologo{La}%
  \kern-.15em%
  \hologo{TeX}%
}
%    \end{macrocode}
%    \end{macro}
%    \begin{macro}{\HoLogoHtml@LaTeX}
%    \begin{macrocode}
\def\HoLogoHtml@LaTeX#1{%
  \HoLogoCss@LaTeX
  \HOLOGO@Span{LaTeX}{%
    L%
    \HOLOGO@Span{a}{%
      A%
    }%
    \hologo{TeX}%
  }%
}
%    \end{macrocode}
%    \end{macro}
%    \begin{macro}{\HoLogoCss@LaTeX}
%    \begin{macrocode}
\def\HoLogoCss@LaTeX{%
  \Css{%
    span.HoLogo-LaTeX span.HoLogo-a{%
      position:relative;%
      top:-.5ex;%
      margin-left:-.36em;%
      margin-right:-.15em;%
      font-size:85\%;%
    }%
  }%
  \global\let\HoLogoCss@LaTeX\relax
}
%    \end{macrocode}
%    \end{macro}
%
% \subsubsection{\hologo{(La)TeX}}
%
%    \begin{macro}{\HoLogo@LaTeXTeX}
%    The kerning around the parentheses is taken
%    from package \xpackage{dtklogos} \cite{dtklogos}.
%\begin{quote}
%\begin{verbatim}
%\DeclareRobustCommand{\LaTeXTeX}{%
%  (%
%  \kern-.15em%
%  L%
%  \kern-.36em%
%  {%
%    \sbox\z@ T%
%    \vbox to\ht0{%
%      \hbox{%
%        $\m@th$%
%        \csname S@\f@size\endcsname
%        \fontsize\sf@size\z@
%        \math@fontsfalse
%        \selectfont
%        A%
%      }%
%      \vss
%    }%
%  }%
%  \kern-.2em%
%  )%
%  \kern-.15em%
%  \TeX
%}
%\end{verbatim}
%\end{quote}
%    \begin{macrocode}
\def\HoLogo@LaTeXTeX#1{%
  (%
  \kern-.15em%
  \hologo{La}%
  \kern-.2em%
  )%
  \kern-.15em%
  \hologo{TeX}%
}
%    \end{macrocode}
%    \end{macro}
%    \begin{macro}{\HoLogoBkm@LaTeXTeX}
%    \begin{macrocode}
\def\HoLogoBkm@LaTeXTeX#1{(La)TeX}
%    \end{macrocode}
%    \end{macro}
%
%    \begin{macro}{\HoLogo@(La)TeX}
%    \begin{macrocode}
\expandafter
\let\csname HoLogo@(La)TeX\endcsname\HoLogo@LaTeXTeX
%    \end{macrocode}
%    \end{macro}
%    \begin{macro}{\HoLogoBkm@(La)TeX}
%    \begin{macrocode}
\expandafter
\let\csname HoLogoBkm@(La)TeX\endcsname\HoLogoBkm@LaTeXTeX
%    \end{macrocode}
%    \end{macro}
%    \begin{macro}{\HoLogoHtml@LaTeXTeX}
%    \begin{macrocode}
\def\HoLogoHtml@LaTeXTeX#1{%
  \HoLogoCss@LaTeXTeX
  \HOLOGO@Span{LaTeXTeX}{%
    (%
    \HOLOGO@Span{L}{L}%
    \HOLOGO@Span{a}{A}%
    \HOLOGO@Span{ParenRight}{)}%
    \hologo{TeX}%
  }%
}
%    \end{macrocode}
%    \end{macro}
%    \begin{macro}{\HoLogoHtml@(La)TeX}
%    Kerning after opening parentheses and before closing parentheses
%    is $-0.1$\,em. The original values $-0.15$\,em
%    looked too ugly for a serif font.
%    \begin{macrocode}
\expandafter
\let\csname HoLogoHtml@(La)TeX\endcsname\HoLogoHtml@LaTeXTeX
%    \end{macrocode}
%    \end{macro}
%    \begin{macro}{\HoLogoCss@LaTeXTeX}
%    \begin{macrocode}
\def\HoLogoCss@LaTeXTeX{%
  \Css{%
    span.HoLogo-LaTeXTeX span.HoLogo-L{%
      margin-left:-.1em;%
    }%
  }%
  \Css{%
    span.HoLogo-LaTeXTeX span.HoLogo-a{%
      position:relative;%
      top:-.5ex;%
      margin-left:-.36em;%
      margin-right:-.1em;%
      font-size:85\%;%
    }%
  }%
  \Css{%
    span.HoLogo-LaTeXTeX span.HoLogo-ParenRight{%
      margin-right:-.15em;%
    }%
  }%
  \global\let\HoLogoCss@LaTeXTeX\relax
}
%    \end{macrocode}
%    \end{macro}
%
% \subsubsection{\hologo{LaTeXe}}
%
%    \begin{macro}{\HoLogo@LaTeXe}
%    Source: \hologo{LaTeX} kernel
%    \begin{macrocode}
\def\HoLogo@LaTeXe#1{%
  \hologo{LaTeX}%
  \kern.15em%
  \hbox{%
    \HOLOGO@MathSetup
    2%
    $_{\textstyle\varepsilon}$%
  }%
}
%    \end{macrocode}
%    \end{macro}
%
%    \begin{macro}{\HoLogoCs@LaTeXe}
%    \begin{macrocode}
\ifnum64=`\^^^^0040\relax % test for big chars of LuaTeX/XeTeX
  \catcode`\$=9 %
  \catcode`\&=14 %
\else
  \catcode`\$=14 %
  \catcode`\&=9 %
\fi
\def\HoLogoCs@LaTeXe#1{%
  LaTeX2%
$ \string ^^^^0395%
& e%
}%
\catcode`\$=3 %
\catcode`\&=4 %
%    \end{macrocode}
%    \end{macro}
%
%    \begin{macro}{\HoLogoBkm@LaTeXe}
%    \begin{macrocode}
\def\HoLogoBkm@LaTeXe#1{%
  \hologo{LaTeX}%
  2%
  \HOLOGO@PdfdocUnicode{e}{\textepsilon}%
}
%    \end{macrocode}
%    \end{macro}
%
%    \begin{macro}{\HoLogoHtml@LaTeXe}
%    \begin{macrocode}
\def\HoLogoHtml@LaTeXe#1{%
  \HoLogoCss@LaTeXe
  \HOLOGO@Span{LaTeX2e}{%
    \hologo{LaTeX}%
    \HOLOGO@Span{2}{2}%
    \HOLOGO@Span{e}{%
      \HOLOGO@MathSetup
      \ensuremath{\textstyle\varepsilon}%
    }%
  }%
}
%    \end{macrocode}
%    \end{macro}
%    \begin{macro}{\HoLogoCss@LaTeXe}
%    \begin{macrocode}
\def\HoLogoCss@LaTeXe{%
  \Css{%
    span.HoLogo-LaTeX2e span.HoLogo-2{%
      padding-left:.15em;%
    }%
  }%
  \Css{%
    span.HoLogo-LaTeX2e span.HoLogo-e{%
      position:relative;%
      top:.35ex;%
      text-decoration:none;%
    }%
  }%
  \global\let\HoLogoCss@LaTeXe\relax
}
%    \end{macrocode}
%    \end{macro}
%
%    \begin{macro}{\HoLogo@LaTeX2e}
%    \begin{macrocode}
\expandafter
\let\csname HoLogo@LaTeX2e\endcsname\HoLogo@LaTeXe
%    \end{macrocode}
%    \end{macro}
%    \begin{macro}{\HoLogoCs@LaTeX2e}
%    \begin{macrocode}
\expandafter
\let\csname HoLogoCs@LaTeX2e\endcsname\HoLogoCs@LaTeXe
%    \end{macrocode}
%    \end{macro}
%    \begin{macro}{\HoLogoBkm@LaTeX2e}
%    \begin{macrocode}
\expandafter
\let\csname HoLogoBkm@LaTeX2e\endcsname\HoLogoBkm@LaTeXe
%    \end{macrocode}
%    \end{macro}
%    \begin{macro}{\HoLogoHtml@LaTeX2e}
%    \begin{macrocode}
\expandafter
\let\csname HoLogoHtml@LaTeX2e\endcsname\HoLogoHtml@LaTeXe
%    \end{macrocode}
%    \end{macro}
%
% \subsubsection{\hologo{LaTeX3}}
%
%    \begin{macro}{\HoLogo@LaTeX3}
%    Source: \hologo{LaTeX} kernel
%    \begin{macrocode}
\expandafter\def\csname HoLogo@LaTeX3\endcsname#1{%
  \hologo{LaTeX}%
  3%
}
%    \end{macrocode}
%    \end{macro}
%
%    \begin{macro}{\HoLogoBkm@LaTeX3}
%    \begin{macrocode}
\expandafter\def\csname HoLogoBkm@LaTeX3\endcsname#1{%
  \hologo{LaTeX}%
  3%
}
%    \end{macrocode}
%    \end{macro}
%    \begin{macro}{\HoLogoHtml@LaTeX3}
%    \begin{macrocode}
\expandafter
\let\csname HoLogoHtml@LaTeX3\expandafter\endcsname
\csname HoLogo@LaTeX3\endcsname
%    \end{macrocode}
%    \end{macro}
%
% \subsubsection{\hologo{LaTeXML}}
%
%    \begin{macro}{\HoLogo@LaTeXML}
%    \begin{macrocode}
\def\HoLogo@LaTeXML#1{%
  \HOLOGO@mbox{%
    \hologo{La}%
    \kern-.15em%
    T%
    \kern-.1667em%
    \lower.5ex\hbox{E}%
    \kern-.125em%
    \HoLogoFont@font{LaTeXML}{sc}{xml}%
  }%
}
%    \end{macrocode}
%    \end{macro}
%    \begin{macro}{\HoLogoHtml@pdfLaTeX}
%    \begin{macrocode}
\def\HoLogoHtml@LaTeXML#1{%
  \HOLOGO@Span{LaTeXML}{%
    \HoLogoCss@LaTeX
    \HoLogoCss@TeX
    \HOLOGO@Span{LaTeX}{%
      L%
      \HOLOGO@Span{a}{%
        A%
      }%
    }%
    \HOLOGO@Span{TeX}{%
      T%
      \HOLOGO@Span{e}{%
        E%
      }%
    }%
    \HCode{<span style="font-variant: small-caps;">}%
    xml%
    \HCode{</span>}%
  }%
}
%    \end{macrocode}
%    \end{macro}
%
% \subsubsection{\hologo{eTeX}}
%
%    \begin{macro}{\HoLogo@eTeX}
%    Source: package \xpackage{etex}
%    \begin{macrocode}
\def\HoLogo@eTeX#1{%
  \ltx@mbox{%
    \HOLOGO@MathSetup
    $\varepsilon$%
    -%
    \HOLOGO@NegativeKerning{-T,T-,To}%
    \hologo{TeX}%
  }%
}
%    \end{macrocode}
%    \end{macro}
%    \begin{macro}{\HoLogoCs@eTeX}
%    \begin{macrocode}
\ifnum64=`\^^^^0040\relax % test for big chars of LuaTeX/XeTeX
  \catcode`\$=9 %
  \catcode`\&=14 %
\else
  \catcode`\$=14 %
  \catcode`\&=9 %
\fi
\def\HoLogoCs@eTeX#1{%
$ #1{\string ^^^^0395}{\string ^^^^03b5}%
& #1{e}{E}%
  TeX%
}%
\catcode`\$=3 %
\catcode`\&=4 %
%    \end{macrocode}
%    \end{macro}
%    \begin{macro}{\HoLogoBkm@eTeX}
%    \begin{macrocode}
\def\HoLogoBkm@eTeX#1{%
  \HOLOGO@PdfdocUnicode{#1{e}{E}}{\textepsilon}%
  -%
  \hologo{TeX}%
}
%    \end{macrocode}
%    \end{macro}
%    \begin{macro}{\HoLogoHtml@eTeX}
%    \begin{macrocode}
\def\HoLogoHtml@eTeX#1{%
  \ltx@mbox{%
    \HOLOGO@MathSetup
    $\varepsilon$%
    -%
    \hologo{TeX}%
  }%
}
%    \end{macrocode}
%    \end{macro}
%
% \subsubsection{\hologo{iniTeX}}
%
%    \begin{macro}{\HoLogo@iniTeX}
%    \begin{macrocode}
\def\HoLogo@iniTeX#1{%
  \HOLOGO@mbox{%
    #1{i}{I}ni\hologo{TeX}%
  }%
}
%    \end{macrocode}
%    \end{macro}
%    \begin{macro}{\HoLogoCs@iniTeX}
%    \begin{macrocode}
\def\HoLogoCs@iniTeX#1{#1{i}{I}niTeX}
%    \end{macrocode}
%    \end{macro}
%    \begin{macro}{\HoLogoBkm@iniTeX}
%    \begin{macrocode}
\def\HoLogoBkm@iniTeX#1{%
  #1{i}{I}ni\hologo{TeX}%
}
%    \end{macrocode}
%    \end{macro}
%    \begin{macro}{\HoLogoHtml@iniTeX}
%    \begin{macrocode}
\let\HoLogoHtml@iniTeX\HoLogo@iniTeX
%    \end{macrocode}
%    \end{macro}
%
% \subsubsection{\hologo{virTeX}}
%
%    \begin{macro}{\HoLogo@virTeX}
%    \begin{macrocode}
\def\HoLogo@virTeX#1{%
  \HOLOGO@mbox{%
    #1{v}{V}ir\hologo{TeX}%
  }%
}
%    \end{macrocode}
%    \end{macro}
%    \begin{macro}{\HoLogoCs@virTeX}
%    \begin{macrocode}
\def\HoLogoCs@virTeX#1{#1{v}{V}irTeX}
%    \end{macrocode}
%    \end{macro}
%    \begin{macro}{\HoLogoBkm@virTeX}
%    \begin{macrocode}
\def\HoLogoBkm@virTeX#1{%
  #1{v}{V}ir\hologo{TeX}%
}
%    \end{macrocode}
%    \end{macro}
%    \begin{macro}{\HoLogoHtml@virTeX}
%    \begin{macrocode}
\let\HoLogoHtml@virTeX\HoLogo@virTeX
%    \end{macrocode}
%    \end{macro}
%
% \subsubsection{\hologo{SliTeX}}
%
% \paragraph{Definitions of the three variants.}
%
%    \begin{macro}{\HoLogo@SLiTeX@lift}
%    \begin{macrocode}
\def\HoLogo@SLiTeX@lift#1{%
  \HoLogoFont@font{SliTeX}{rm}{%
    S%
    \kern-.06em%
    L%
    \kern-.18em%
    \raise.32ex\hbox{\HoLogoFont@font{SliTeX}{sc}{i}}%
    \HOLOGO@discretionary
    \kern-.06em%
    \hologo{TeX}%
  }%
}
%    \end{macrocode}
%    \end{macro}
%    \begin{macro}{\HoLogoBkm@SLiTeX@lift}
%    \begin{macrocode}
\def\HoLogoBkm@SLiTeX@lift#1{SLiTeX}
%    \end{macrocode}
%    \end{macro}
%    \begin{macro}{\HoLogoHtml@SLiTeX@lift}
%    \begin{macrocode}
\def\HoLogoHtml@SLiTeX@lift#1{%
  \HoLogoCss@SLiTeX@lift
  \HOLOGO@Span{SLiTeX-lift}{%
    \HoLogoFont@font{SliTeX}{rm}{%
      S%
      \HOLOGO@Span{L}{L}%
      \HOLOGO@Span{i}{i}%
      \hologo{TeX}%
    }%
  }%
}
%    \end{macrocode}
%    \end{macro}
%    \begin{macro}{\HoLogoCss@SLiTeX@lift}
%    \begin{macrocode}
\def\HoLogoCss@SLiTeX@lift{%
  \Css{%
    span.HoLogo-SLiTeX-lift span.HoLogo-L{%
      margin-left:-.06em;%
      margin-right:-.18em;%
    }%
  }%
  \Css{%
    span.HoLogo-SLiTeX-lift span.HoLogo-i{%
      position:relative;%
      top:-.32ex;%
      margin-right:-.06em;%
      font-variant:small-caps;%
    }%
  }%
  \global\let\HoLogoCss@SLiTeX@lift\relax
}
%    \end{macrocode}
%    \end{macro}
%
%    \begin{macro}{\HoLogo@SliTeX@simple}
%    \begin{macrocode}
\def\HoLogo@SliTeX@simple#1{%
  \HoLogoFont@font{SliTeX}{rm}{%
    \ltx@mbox{%
      \HoLogoFont@font{SliTeX}{sc}{Sli}%
    }%
    \HOLOGO@discretionary
    \hologo{TeX}%
  }%
}
%    \end{macrocode}
%    \end{macro}
%    \begin{macro}{\HoLogoBkm@SliTeX@simple}
%    \begin{macrocode}
\def\HoLogoBkm@SliTeX@simple#1{SliTeX}
%    \end{macrocode}
%    \end{macro}
%    \begin{macro}{\HoLogoHtml@SliTeX@simple}
%    \begin{macrocode}
\let\HoLogoHtml@SliTeX@simple\HoLogo@SliTeX@simple
%    \end{macrocode}
%    \end{macro}
%
%    \begin{macro}{\HoLogo@SliTeX@narrow}
%    \begin{macrocode}
\def\HoLogo@SliTeX@narrow#1{%
  \HoLogoFont@font{SliTeX}{rm}{%
    \ltx@mbox{%
      S%
      \kern-.06em%
      \HoLogoFont@font{SliTeX}{sc}{%
        l%
        \kern-.035em%
        i%
      }%
    }%
    \HOLOGO@discretionary
    \kern-.06em%
    \hologo{TeX}%
  }%
}
%    \end{macrocode}
%    \end{macro}
%    \begin{macro}{\HoLogoBkm@SliTeX@narrow}
%    \begin{macrocode}
\def\HoLogoBkm@SliTeX@narrow#1{SliTeX}
%    \end{macrocode}
%    \end{macro}
%    \begin{macro}{\HoLogoHtml@SliTeX@narrow}
%    \begin{macrocode}
\def\HoLogoHtml@SliTeX@narrow#1{%
  \HoLogoCss@SliTeX@narrow
  \HOLOGO@Span{SliTeX-narrow}{%
    \HoLogoFont@font{SliTeX}{rm}{%
      S%
        \HOLOGO@Span{l}{l}%
        \HOLOGO@Span{i}{i}%
      \hologo{TeX}%
    }%
  }%
}
%    \end{macrocode}
%    \end{macro}
%    \begin{macro}{\HoLogoCss@SliTeX@narrow}
%    \begin{macrocode}
\def\HoLogoCss@SliTeX@narrow{%
  \Css{%
    span.HoLogo-SliTeX-narrow span.HoLogo-l{%
      margin-left:-.06em;%
      margin-right:-.035em;%
      font-variant:small-caps;%
    }%
  }%
  \Css{%
    span.HoLogo-SliTeX-narrow span.HoLogo-i{%
      margin-right:-.06em;%
      font-variant:small-caps;%
    }%
  }%
  \global\let\HoLogoCss@SliTeX@narrow\relax
}
%    \end{macrocode}
%    \end{macro}
%
% \paragraph{Macro set completion.}
%
%    \begin{macro}{\HoLogo@SLiTeX@simple}
%    \begin{macrocode}
\def\HoLogo@SLiTeX@simple{\HoLogo@SliTeX@simple}
%    \end{macrocode}
%    \end{macro}
%    \begin{macro}{\HoLogoBkm@SLiTeX@simple}
%    \begin{macrocode}
\def\HoLogoBkm@SLiTeX@simple{\HoLogoBkm@SliTeX@simple}
%    \end{macrocode}
%    \end{macro}
%    \begin{macro}{\HoLogoHtml@SLiTeX@simple}
%    \begin{macrocode}
\def\HoLogoHtml@SLiTeX@simple{\HoLogoHtml@SliTeX@simple}
%    \end{macrocode}
%    \end{macro}
%
%    \begin{macro}{\HoLogo@SLiTeX@narrow}
%    \begin{macrocode}
\def\HoLogo@SLiTeX@narrow{\HoLogo@SliTeX@narrow}
%    \end{macrocode}
%    \end{macro}
%    \begin{macro}{\HoLogoBkm@SLiTeX@narrow}
%    \begin{macrocode}
\def\HoLogoBkm@SLiTeX@narrow{\HoLogoBkm@SliTeX@narrow}
%    \end{macrocode}
%    \end{macro}
%    \begin{macro}{\HoLogoHtml@SLiTeX@narrow}
%    \begin{macrocode}
\def\HoLogoHtml@SLiTeX@narrow{\HoLogoHtml@SliTeX@narrow}
%    \end{macrocode}
%    \end{macro}
%
%    \begin{macro}{\HoLogo@SliTeX@lift}
%    \begin{macrocode}
\def\HoLogo@SliTeX@lift{\HoLogo@SLiTeX@lift}
%    \end{macrocode}
%    \end{macro}
%    \begin{macro}{\HoLogoBkm@SliTeX@lift}
%    \begin{macrocode}
\def\HoLogoBkm@SliTeX@lift{\HoLogoBkm@SLiTeX@lift}
%    \end{macrocode}
%    \end{macro}
%    \begin{macro}{\HoLogoHtml@SliTeX@lift}
%    \begin{macrocode}
\def\HoLogoHtml@SliTeX@lift{\HoLogoHtml@SLiTeX@lift}
%    \end{macrocode}
%    \end{macro}
%
% \paragraph{Defaults.}
%
%    \begin{macro}{\HoLogo@SLiTeX}
%    \begin{macrocode}
\def\HoLogo@SLiTeX{\HoLogo@SLiTeX@lift}
%    \end{macrocode}
%    \end{macro}
%    \begin{macro}{\HoLogoBkm@SLiTeX}
%    \begin{macrocode}
\def\HoLogoBkm@SLiTeX{\HoLogoBkm@SLiTeX@lift}
%    \end{macrocode}
%    \end{macro}
%    \begin{macro}{\HoLogoHtml@SLiTeX}
%    \begin{macrocode}
\def\HoLogoHtml@SLiTeX{\HoLogoHtml@SLiTeX@lift}
%    \end{macrocode}
%    \end{macro}
%
%    \begin{macro}{\HoLogo@SliTeX}
%    \begin{macrocode}
\def\HoLogo@SliTeX{\HoLogo@SliTeX@narrow}
%    \end{macrocode}
%    \end{macro}
%    \begin{macro}{\HoLogoBkm@SliTeX}
%    \begin{macrocode}
\def\HoLogoBkm@SliTeX{\HoLogoBkm@SliTeX@narrow}
%    \end{macrocode}
%    \end{macro}
%    \begin{macro}{\HoLogoHtml@SliTeX}
%    \begin{macrocode}
\def\HoLogoHtml@SliTeX{\HoLogoHtml@SliTeX@narrow}
%    \end{macrocode}
%    \end{macro}
%
% \subsubsection{\hologo{LuaTeX}}
%
%    \begin{macro}{\HoLogo@LuaTeX}
%    The kerning is an idea of Hans Hagen, see mailing list
%    `luatex at tug dot org' in March 2010.
%    \begin{macrocode}
\def\HoLogo@LuaTeX#1{%
  \HOLOGO@mbox{%
    Lua%
    \HOLOGO@NegativeKerning{aT,oT,To}%
    \hologo{TeX}%
  }%
}
%    \end{macrocode}
%    \end{macro}
%    \begin{macro}{\HoLogoHtml@LuaTeX}
%    \begin{macrocode}
\let\HoLogoHtml@LuaTeX\HoLogo@LuaTeX
%    \end{macrocode}
%    \end{macro}
%
% \subsubsection{\hologo{LuaLaTeX}}
%
%    \begin{macro}{\HoLogo@LuaLaTeX}
%    \begin{macrocode}
\def\HoLogo@LuaLaTeX#1{%
  \HOLOGO@mbox{%
    Lua%
    \hologo{LaTeX}%
  }%
}
%    \end{macrocode}
%    \end{macro}
%    \begin{macro}{\HoLogoHtml@LuaLaTeX}
%    \begin{macrocode}
\let\HoLogoHtml@LuaLaTeX\HoLogo@LuaLaTeX
%    \end{macrocode}
%    \end{macro}
%
% \subsubsection{\hologo{XeTeX}, \hologo{XeLaTeX}}
%
%    \begin{macro}{\HOLOGO@IfCharExists}
%    \begin{macrocode}
\ifluatex
  \ifnum\luatexversion<36 %
  \else
    \def\HOLOGO@IfCharExists#1{%
      \ifnum
        \directlua{%
           if luaotfload and luaotfload.aux then
             if luaotfload.aux.font_has_glyph(%
                    font.current(), \number#1) then % 	 
	       tex.print("1") % 	 
	     end % 	 
	   elseif font and font.fonts and font.current then %
            local f = font.fonts[font.current()]%
            if f.characters and f.characters[\number#1] then %
              tex.print("1")%
            end %
          end%
        }0=\ltx@zero
        \expandafter\ltx@secondoftwo
      \else
        \expandafter\ltx@firstoftwo
      \fi
    }%
  \fi
\fi
\ltx@IfUndefined{HOLOGO@IfCharExists}{%
  \def\HOLOGO@@IfCharExists#1{%
    \begingroup
      \tracinglostchars=\ltx@zero
      \setbox\ltx@zero=\hbox{%
        \kern7sp\char#1\relax
        \ifnum\lastkern>\ltx@zero
          \expandafter\aftergroup\csname iffalse\endcsname
        \else
          \expandafter\aftergroup\csname iftrue\endcsname
        \fi
      }%
      % \if{true|false} from \aftergroup
      \endgroup
      \expandafter\ltx@firstoftwo
    \else
      \endgroup
      \expandafter\ltx@secondoftwo
    \fi
  }%
  \ifxetex
    \ltx@IfUndefined{XeTeXfonttype}{}{%
      \ltx@IfUndefined{XeTeXcharglyph}{}{%
        \def\HOLOGO@IfCharExists#1{%
          \ifnum\XeTeXfonttype\font>\ltx@zero
            \expandafter\ltx@firstofthree
          \else
            \expandafter\ltx@gobble
          \fi
          {%
            \ifnum\XeTeXcharglyph#1>\ltx@zero
              \expandafter\ltx@firstoftwo
            \else
              \expandafter\ltx@secondoftwo
            \fi
          }%
          \HOLOGO@@IfCharExists{#1}%
        }%
      }%
    }%
  \fi
}{}
\ltx@ifundefined{HOLOGO@IfCharExists}{%
  \ifnum64=`\^^^^0040\relax % test for big chars of LuaTeX/XeTeX
    \let\HOLOGO@IfCharExists\HOLOGO@@IfCharExists
  \else
    \def\HOLOGO@IfCharExists#1{%
      \ifnum#1>255 %
        \expandafter\ltx@fourthoffour
      \fi
      \HOLOGO@@IfCharExists{#1}%
    }%
  \fi
}{}
%    \end{macrocode}
%    \end{macro}
%
%    \begin{macro}{\HoLogo@Xe}
%    Source: package \xpackage{dtklogos}
%    \begin{macrocode}
\def\HoLogo@Xe#1{%
  X%
  \kern-.1em\relax
  \HOLOGO@IfCharExists{"018E}{%
    \lower.5ex\hbox{\char"018E}%
  }{%
    \chardef\HOLOGO@choice=\ltx@zero
    \ifdim\fontdimen\ltx@one\font>0pt %
      \ltx@IfUndefined{rotatebox}{%
        \ltx@IfUndefined{pgftext}{%
          \ltx@IfUndefined{psscalebox}{%
            \ltx@IfUndefined{HOLOGO@ScaleBox@\hologoDriver}{%
            }{%
              \chardef\HOLOGO@choice=4 %
            }%
          }{%
            \chardef\HOLOGO@choice=3 %
          }%
        }{%
          \chardef\HOLOGO@choice=2 %
        }%
      }{%
        \chardef\HOLOGO@choice=1 %
      }%
      \ifcase\HOLOGO@choice
        \HOLOGO@WarningUnsupportedDriver{Xe}%
        e%
      \or % 1: \rotatebox
        \begingroup
          \setbox\ltx@zero\hbox{\rotatebox{180}{E}}%
          \ltx@LocDimenA=\dp\ltx@zero
          \advance\ltx@LocDimenA by -.5ex\relax
          \raise\ltx@LocDimenA\box\ltx@zero
        \endgroup
      \or % 2: \pgftext
        \lower.5ex\hbox{%
          \pgfpicture
            \pgftext[rotate=180]{E}%
          \endpgfpicture
        }%
      \or % 3: \psscalebox
        \begingroup
          \setbox\ltx@zero\hbox{\psscalebox{-1 -1}{E}}%
          \ltx@LocDimenA=\dp\ltx@zero
          \advance\ltx@LocDimenA by -.5ex\relax
          \raise\ltx@LocDimenA\box\ltx@zero
        \endgroup
      \or % 4: \HOLOGO@PointReflectBox
        \lower.5ex\hbox{\HOLOGO@PointReflectBox{E}}%
      \else
        \@PackageError{hologo}{Internal error (choice/it}\@ehc
      \fi
    \else
      \ltx@IfUndefined{reflectbox}{%
        \ltx@IfUndefined{pgftext}{%
          \ltx@IfUndefined{psscalebox}{%
            \ltx@IfUndefined{HOLOGO@ScaleBox@\hologoDriver}{%
            }{%
              \chardef\HOLOGO@choice=4 %
            }%
          }{%
            \chardef\HOLOGO@choice=3 %
          }%
        }{%
          \chardef\HOLOGO@choice=2 %
        }%
      }{%
        \chardef\HOLOGO@choice=1 %
      }%
      \ifcase\HOLOGO@choice
        \HOLOGO@WarningUnsupportedDriver{Xe}%
        e%
      \or % 1: reflectbox
        \lower.5ex\hbox{%
          \reflectbox{E}%
        }%
      \or % 2: \pgftext
        \lower.5ex\hbox{%
          \pgfpicture
            \pgftransformxscale{-1}%
            \pgftext{E}%
          \endpgfpicture
        }%
      \or % 3: \psscalebox
        \lower.5ex\hbox{%
          \psscalebox{-1 1}{E}%
        }%
      \or % 4: \HOLOGO@Reflectbox
        \lower.5ex\hbox{%
          \HOLOGO@ReflectBox{E}%
        }%
      \else
        \@PackageError{hologo}{Internal error (choice/up)}\@ehc
      \fi
    \fi
  }%
}
%    \end{macrocode}
%    \end{macro}
%    \begin{macro}{\HoLogoHtml@Xe}
%    \begin{macrocode}
\def\HoLogoHtml@Xe#1{%
  \HoLogoCss@Xe
  \HOLOGO@Span{Xe}{%
    X%
    \HOLOGO@Span{e}{%
      \HCode{&\ltx@hashchar x018e;}%
    }%
  }%
}
%    \end{macrocode}
%    \end{macro}
%    \begin{macro}{\HoLogoCss@Xe}
%    \begin{macrocode}
\def\HoLogoCss@Xe{%
  \Css{%
    span.HoLogo-Xe span.HoLogo-e{%
      position:relative;%
      top:.5ex;%
      left-margin:-.1em;%
    }%
  }%
  \global\let\HoLogoCss@Xe\relax
}
%    \end{macrocode}
%    \end{macro}
%
%    \begin{macro}{\HoLogo@XeTeX}
%    \begin{macrocode}
\def\HoLogo@XeTeX#1{%
  \hologo{Xe}%
  \kern-.15em\relax
  \hologo{TeX}%
}
%    \end{macrocode}
%    \end{macro}
%
%    \begin{macro}{\HoLogoHtml@XeTeX}
%    \begin{macrocode}
\def\HoLogoHtml@XeTeX#1{%
  \HoLogoCss@XeTeX
  \HOLOGO@Span{XeTeX}{%
    \hologo{Xe}%
    \hologo{TeX}%
  }%
}
%    \end{macrocode}
%    \end{macro}
%    \begin{macro}{\HoLogoCss@XeTeX}
%    \begin{macrocode}
\def\HoLogoCss@XeTeX{%
  \Css{%
    span.HoLogo-XeTeX span.HoLogo-TeX{%
      margin-left:-.15em;%
    }%
  }%
  \global\let\HoLogoCss@XeTeX\relax
}
%    \end{macrocode}
%    \end{macro}
%
%    \begin{macro}{\HoLogo@XeLaTeX}
%    \begin{macrocode}
\def\HoLogo@XeLaTeX#1{%
  \hologo{Xe}%
  \kern-.13em%
  \hologo{LaTeX}%
}
%    \end{macrocode}
%    \end{macro}
%    \begin{macro}{\HoLogoHtml@XeLaTeX}
%    \begin{macrocode}
\def\HoLogoHtml@XeLaTeX#1{%
  \HoLogoCss@XeLaTeX
  \HOLOGO@Span{XeLaTeX}{%
    \hologo{Xe}%
    \hologo{LaTeX}%
  }%
}
%    \end{macrocode}
%    \end{macro}
%    \begin{macro}{\HoLogoCss@XeLaTeX}
%    \begin{macrocode}
\def\HoLogoCss@XeLaTeX{%
  \Css{%
    span.HoLogo-XeLaTeX span.HoLogo-Xe{%
      margin-right:-.13em;%
    }%
  }%
  \global\let\HoLogoCss@XeLaTeX\relax
}
%    \end{macrocode}
%    \end{macro}
%
% \subsubsection{\hologo{pdfTeX}, \hologo{pdfLaTeX}}
%
%    \begin{macro}{\HoLogo@pdfTeX}
%    \begin{macrocode}
\def\HoLogo@pdfTeX#1{%
  \HOLOGO@mbox{%
    #1{p}{P}df\hologo{TeX}%
  }%
}
%    \end{macrocode}
%    \end{macro}
%    \begin{macro}{\HoLogoCs@pdfTeX}
%    \begin{macrocode}
\def\HoLogoCs@pdfTeX#1{#1{p}{P}dfTeX}
%    \end{macrocode}
%    \end{macro}
%    \begin{macro}{\HoLogoBkm@pdfTeX}
%    \begin{macrocode}
\def\HoLogoBkm@pdfTeX#1{%
  #1{p}{P}df\hologo{TeX}%
}
%    \end{macrocode}
%    \end{macro}
%    \begin{macro}{\HoLogoHtml@pdfTeX}
%    \begin{macrocode}
\let\HoLogoHtml@pdfTeX\HoLogo@pdfTeX
%    \end{macrocode}
%    \end{macro}
%
%    \begin{macro}{\HoLogo@pdfLaTeX}
%    \begin{macrocode}
\def\HoLogo@pdfLaTeX#1{%
  \HOLOGO@mbox{%
    #1{p}{P}df\hologo{LaTeX}%
  }%
}
%    \end{macrocode}
%    \end{macro}
%    \begin{macro}{\HoLogoCs@pdfLaTeX}
%    \begin{macrocode}
\def\HoLogoCs@pdfLaTeX#1{#1{p}{P}dfLaTeX}
%    \end{macrocode}
%    \end{macro}
%    \begin{macro}{\HoLogoBkm@pdfLaTeX}
%    \begin{macrocode}
\def\HoLogoBkm@pdfLaTeX#1{%
  #1{p}{P}df\hologo{LaTeX}%
}
%    \end{macrocode}
%    \end{macro}
%    \begin{macro}{\HoLogoHtml@pdfLaTeX}
%    \begin{macrocode}
\let\HoLogoHtml@pdfLaTeX\HoLogo@pdfLaTeX
%    \end{macrocode}
%    \end{macro}
%
% \subsubsection{\hologo{VTeX}}
%
%    \begin{macro}{\HoLogo@VTeX}
%    \begin{macrocode}
\def\HoLogo@VTeX#1{%
  \HOLOGO@mbox{%
    V\hologo{TeX}%
  }%
}
%    \end{macrocode}
%    \end{macro}
%    \begin{macro}{\HoLogoHtml@VTeX}
%    \begin{macrocode}
\let\HoLogoHtml@VTeX\HoLogo@VTeX
%    \end{macrocode}
%    \end{macro}
%
% \subsubsection{\hologo{AmS}, \dots}
%
%    Source: class \xclass{amsdtx}
%
%    \begin{macro}{\HoLogo@AmS}
%    \begin{macrocode}
\def\HoLogo@AmS#1{%
  \HoLogoFont@font{AmS}{sy}{%
    A%
    \kern-.1667em%
    \lower.5ex\hbox{M}%
    \kern-.125em%
    S%
  }%
}
%    \end{macrocode}
%    \end{macro}
%    \begin{macro}{\HoLogoBkm@AmS}
%    \begin{macrocode}
\def\HoLogoBkm@AmS#1{AmS}
%    \end{macrocode}
%    \end{macro}
%    \begin{macro}{\HoLogoHtml@AmS}
%    \begin{macrocode}
\def\HoLogoHtml@AmS#1{%
  \HoLogoCss@AmS
%  \HoLogoFont@font{AmS}{sy}{%
    \HOLOGO@Span{AmS}{%
      A%
      \HOLOGO@Span{M}{M}%
      S%
    }%
%   }%
}
%    \end{macrocode}
%    \end{macro}
%    \begin{macro}{\HoLogoCss@AmS}
%    \begin{macrocode}
\def\HoLogoCss@AmS{%
  \Css{%
    span.HoLogo-AmS span.HoLogo-M{%
      position:relative;%
      top:.5ex;%
      margin-left:-.1667em;%
      margin-right:-.125em;%
      text-decoration:none;%
    }%
  }%
  \global\let\HoLogoCss@AmS\relax
}
%    \end{macrocode}
%    \end{macro}
%
%    \begin{macro}{\HoLogo@AmSTeX}
%    \begin{macrocode}
\def\HoLogo@AmSTeX#1{%
  \hologo{AmS}%
  \HOLOGO@hyphen
  \hologo{TeX}%
}
%    \end{macrocode}
%    \end{macro}
%    \begin{macro}{\HoLogoBkm@AmSTeX}
%    \begin{macrocode}
\def\HoLogoBkm@AmSTeX#1{AmS-TeX}%
%    \end{macrocode}
%    \end{macro}
%    \begin{macro}{\HoLogoHtml@AmSTeX}
%    \begin{macrocode}
\let\HoLogoHtml@AmSTeX\HoLogo@AmSTeX
%    \end{macrocode}
%    \end{macro}
%
%    \begin{macro}{\HoLogo@AmSLaTeX}
%    \begin{macrocode}
\def\HoLogo@AmSLaTeX#1{%
  \hologo{AmS}%
  \HOLOGO@hyphen
  \hologo{LaTeX}%
}
%    \end{macrocode}
%    \end{macro}
%    \begin{macro}{\HoLogoBkm@AmSLaTeX}
%    \begin{macrocode}
\def\HoLogoBkm@AmSLaTeX#1{AmS-LaTeX}%
%    \end{macrocode}
%    \end{macro}
%    \begin{macro}{\HoLogoHtml@AmSLaTeX}
%    \begin{macrocode}
\let\HoLogoHtml@AmSLaTeX\HoLogo@AmSLaTeX
%    \end{macrocode}
%    \end{macro}
%
% \subsubsection{\hologo{BibTeX}}
%
%    \begin{macro}{\HoLogo@BibTeX@sc}
%    A definition of \hologo{BibTeX} is provided in
%    the documentation source for the manual of \hologo{BibTeX}
%    \cite{btxdoc}.
%\begin{quote}
%\begin{verbatim}
%\def\BibTeX{%
%  {%
%    \rm
%    B%
%    \kern-.05em%
%    {%
%      \sc
%      i%
%      \kern-.025em %
%      b%
%    }%
%    \kern-.08em
%    T%
%    \kern-.1667em%
%    \lower.7ex\hbox{E}%
%    \kern-.125em%
%    X%
%  }%
%}
%\end{verbatim}
%\end{quote}
%    \begin{macrocode}
\def\HoLogo@BibTeX@sc#1{%
  B%
  \kern-.05em%
  \HoLogoFont@font{BibTeX}{sc}{%
    i%
    \kern-.025em%
    b%
  }%
  \HOLOGO@discretionary
  \kern-.08em%
  \hologo{TeX}%
}
%    \end{macrocode}
%    \end{macro}
%    \begin{macro}{\HoLogoHtml@BibTeX@sc}
%    \begin{macrocode}
\def\HoLogoHtml@BibTeX@sc#1{%
  \HoLogoCss@BibTeX@sc
  \HOLOGO@Span{BibTeX-sc}{%
    B%
    \HOLOGO@Span{i}{i}%
    \HOLOGO@Span{b}{b}%
    \hologo{TeX}%
  }%
}
%    \end{macrocode}
%    \end{macro}
%    \begin{macro}{\HoLogoCss@BibTeX@sc}
%    \begin{macrocode}
\def\HoLogoCss@BibTeX@sc{%
  \Css{%
    span.HoLogo-BibTeX-sc span.HoLogo-i{%
      margin-left:-.05em;%
      margin-right:-.025em;%
      font-variant:small-caps;%
    }%
  }%
  \Css{%
    span.HoLogo-BibTeX-sc span.HoLogo-b{%
      margin-right:-.08em;%
      font-variant:small-caps;%
    }%
  }%
  \global\let\HoLogoCss@BibTeX@sc\relax
}
%    \end{macrocode}
%    \end{macro}
%
%    \begin{macro}{\HoLogo@BibTeX@sf}
%    Variant \xoption{sf} avoids trouble with unavailable
%    small caps fonts (e.g., bold versions of Computer Modern or
%    Latin Modern). The definition is taken from
%    package \xpackage{dtklogos} \cite{dtklogos}.
%\begin{quote}
%\begin{verbatim}
%\DeclareRobustCommand{\BibTeX}{%
%  B%
%  \kern-.05em%
%  \hbox{%
%    $\m@th$% %% force math size calculations
%    \csname S@\f@size\endcsname
%    \fontsize\sf@size\z@
%    \math@fontsfalse
%    \selectfont
%    I%
%    \kern-.025em%
%    B
%  }%
%  \kern-.08em%
%  \-%
%  \TeX
%}
%\end{verbatim}
%\end{quote}
%    \begin{macrocode}
\def\HoLogo@BibTeX@sf#1{%
  B%
  \kern-.05em%
  \HoLogoFont@font{BibTeX}{bibsf}{%
    I%
    \kern-.025em%
    B%
  }%
  \HOLOGO@discretionary
  \kern-.08em%
  \hologo{TeX}%
}
%    \end{macrocode}
%    \end{macro}
%    \begin{macro}{\HoLogoHtml@BibTeX@sf}
%    \begin{macrocode}
\def\HoLogoHtml@BibTeX@sf#1{%
  \HoLogoCss@BibTeX@sf
  \HOLOGO@Span{BibTeX-sf}{%
    B%
    \HoLogoFont@font{BibTeX}{bibsf}{%
      \HOLOGO@Span{i}{I}%
      B%
    }%
    \hologo{TeX}%
  }%
}
%    \end{macrocode}
%    \end{macro}
%    \begin{macro}{\HoLogoCss@BibTeX@sf}
%    \begin{macrocode}
\def\HoLogoCss@BibTeX@sf{%
  \Css{%
    span.HoLogo-BibTeX-sf span.HoLogo-i{%
      margin-left:-.05em;%
      margin-right:-.025em;%
    }%
  }%
  \Css{%
    span.HoLogo-BibTeX-sf span.HoLogo-TeX{%
      margin-left:-.08em;%
    }%
  }%
  \global\let\HoLogoCss@BibTeX@sf\relax
}
%    \end{macrocode}
%    \end{macro}
%
%    \begin{macro}{\HoLogo@BibTeX}
%    \begin{macrocode}
\def\HoLogo@BibTeX{\HoLogo@BibTeX@sf}
%    \end{macrocode}
%    \end{macro}
%    \begin{macro}{\HoLogoHtml@BibTeX}
%    \begin{macrocode}
\def\HoLogoHtml@BibTeX{\HoLogoHtml@BibTeX@sf}
%    \end{macrocode}
%    \end{macro}
%
% \subsubsection{\hologo{BibTeX8}}
%
%    \begin{macro}{\HoLogo@BibTeX8}
%    \begin{macrocode}
\expandafter\def\csname HoLogo@BibTeX8\endcsname#1{%
  \hologo{BibTeX}%
  8%
}
%    \end{macrocode}
%    \end{macro}
%
%    \begin{macro}{\HoLogoBkm@BibTeX8}
%    \begin{macrocode}
\expandafter\def\csname HoLogoBkm@BibTeX8\endcsname#1{%
  \hologo{BibTeX}%
  8%
}
%    \end{macrocode}
%    \end{macro}
%    \begin{macro}{\HoLogoHtml@BibTeX8}
%    \begin{macrocode}
\expandafter
\let\csname HoLogoHtml@BibTeX8\expandafter\endcsname
\csname HoLogo@BibTeX8\endcsname
%    \end{macrocode}
%    \end{macro}
%
% \subsubsection{\hologo{ConTeXt}}
%
%    \begin{macro}{\HoLogo@ConTeXt@simple}
%    \begin{macrocode}
\def\HoLogo@ConTeXt@simple#1{%
  \HOLOGO@mbox{Con}%
  \HOLOGO@discretionary
  \HOLOGO@mbox{\hologo{TeX}t}%
}
%    \end{macrocode}
%    \end{macro}
%    \begin{macro}{\HoLogoHtml@ConTeXt@simple}
%    \begin{macrocode}
\let\HoLogoHtml@ConTeXt@simple\HoLogo@ConTeXt@simple
%    \end{macrocode}
%    \end{macro}
%
%    \begin{macro}{\HoLogo@ConTeXt@narrow}
%    This definition of logo \hologo{ConTeXt} with variant \xoption{narrow}
%    comes from TUGboat's class \xclass{ltugboat} (version 2010/11/15 v2.8).
%    \begin{macrocode}
\def\HoLogo@ConTeXt@narrow#1{%
  \HOLOGO@mbox{C\kern-.0333emon}%
  \HOLOGO@discretionary
  \kern-.0667em%
  \HOLOGO@mbox{\hologo{TeX}\kern-.0333emt}%
}
%    \end{macrocode}
%    \end{macro}
%    \begin{macro}{\HoLogoHtml@ConTeXt@narrow}
%    \begin{macrocode}
\def\HoLogoHtml@ConTeXt@narrow#1{%
  \HoLogoCss@ConTeXt@narrow
  \HOLOGO@Span{ConTeXt-narrow}{%
    \HOLOGO@Span{C}{C}%
    on%
    \hologo{TeX}%
    t%
  }%
}
%    \end{macrocode}
%    \end{macro}
%    \begin{macro}{\HoLogoCss@ConTeXt@narrow}
%    \begin{macrocode}
\def\HoLogoCss@ConTeXt@narrow{%
  \Css{%
    span.HoLogo-ConTeXt-narrow span.HoLogo-C{%
      margin-left:-.0333em;%
    }%
  }%
  \Css{%
    span.HoLogo-ConTeXt-narrow span.HoLogo-TeX{%
      margin-left:-.0667em;%
      margin-right:-.0333em;%
    }%
  }%
  \global\let\HoLogoCss@ConTeXt@narrow\relax
}
%    \end{macrocode}
%    \end{macro}
%
%    \begin{macro}{\HoLogo@ConTeXt}
%    \begin{macrocode}
\def\HoLogo@ConTeXt{\HoLogo@ConTeXt@narrow}
%    \end{macrocode}
%    \end{macro}
%    \begin{macro}{\HoLogoHtml@ConTeXt}
%    \begin{macrocode}
\def\HoLogoHtml@ConTeXt{\HoLogoHtml@ConTeXt@narrow}
%    \end{macrocode}
%    \end{macro}
%
% \subsubsection{\hologo{emTeX}}
%
%    \begin{macro}{\HoLogo@emTeX}
%    \begin{macrocode}
\def\HoLogo@emTeX#1{%
  \HOLOGO@mbox{#1{e}{E}m}%
  \HOLOGO@discretionary
  \hologo{TeX}%
}
%    \end{macrocode}
%    \end{macro}
%    \begin{macro}{\HoLogoCs@emTeX}
%    \begin{macrocode}
\def\HoLogoCs@emTeX#1{#1{e}{E}mTeX}%
%    \end{macrocode}
%    \end{macro}
%    \begin{macro}{\HoLogoBkm@emTeX}
%    \begin{macrocode}
\def\HoLogoBkm@emTeX#1{%
  #1{e}{E}m\hologo{TeX}%
}
%    \end{macrocode}
%    \end{macro}
%    \begin{macro}{\HoLogoHtml@emTeX}
%    \begin{macrocode}
\let\HoLogoHtml@emTeX\HoLogo@emTeX
%    \end{macrocode}
%    \end{macro}
%
% \subsubsection{\hologo{ExTeX}}
%
%    \begin{macro}{\HoLogo@ExTeX}
%    The definition is taken from the FAQ of the
%    project \hologo{ExTeX}
%    \cite{ExTeX-FAQ}.
%\begin{quote}
%\begin{verbatim}
%\def\ExTeX{%
%  \textrm{% Logo always with serifs
%    \ensuremath{%
%      \textstyle
%      \varepsilon_{%
%        \kern-0.15em%
%        \mathcal{X}%
%      }%
%    }%
%    \kern-.15em%
%    \TeX
%  }%
%}
%\end{verbatim}
%\end{quote}
%    \begin{macrocode}
\def\HoLogo@ExTeX#1{%
  \HoLogoFont@font{ExTeX}{rm}{%
    \ltx@mbox{%
      \HOLOGO@MathSetup
      $%
        \textstyle
        \varepsilon_{%
          \kern-0.15em%
          \HoLogoFont@font{ExTeX}{sy}{X}%
        }%
      $%
    }%
    \HOLOGO@discretionary
    \kern-.15em%
    \hologo{TeX}%
  }%
}
%    \end{macrocode}
%    \end{macro}
%    \begin{macro}{\HoLogoHtml@ExTeX}
%    \begin{macrocode}
\def\HoLogoHtml@ExTeX#1{%
  \HoLogoCss@ExTeX
  \HoLogoFont@font{ExTeX}{rm}{%
    \HOLOGO@Span{ExTeX}{%
      \ltx@mbox{%
        \HOLOGO@MathSetup
        $\textstyle\varepsilon$%
        \HOLOGO@Span{X}{$\textstyle\chi$}%
        \hologo{TeX}%
      }%
    }%
  }%
}
%    \end{macrocode}
%    \end{macro}
%    \begin{macro}{\HoLogoBkm@ExTeX}
%    \begin{macrocode}
\def\HoLogoBkm@ExTeX#1{%
  \HOLOGO@PdfdocUnicode{#1{e}{E}x}{\textepsilon\textchi}%
  \hologo{TeX}%
}
%    \end{macrocode}
%    \end{macro}
%    \begin{macro}{\HoLogoCss@ExTeX}
%    \begin{macrocode}
\def\HoLogoCss@ExTeX{%
  \Css{%
    span.HoLogo-ExTeX{%
      font-family:serif;%
    }%
  }%
  \Css{%
    span.HoLogo-ExTeX span.HoLogo-TeX{%
      margin-left:-.15em;%
    }%
  }%
  \global\let\HoLogoCss@ExTeX\relax
}
%    \end{macrocode}
%    \end{macro}
%
% \subsubsection{\hologo{MiKTeX}}
%
%    \begin{macro}{\HoLogo@MiKTeX}
%    \begin{macrocode}
\def\HoLogo@MiKTeX#1{%
  \HOLOGO@mbox{MiK}%
  \HOLOGO@discretionary
  \hologo{TeX}%
}
%    \end{macrocode}
%    \end{macro}
%    \begin{macro}{\HoLogoHtml@MiKTeX}
%    \begin{macrocode}
\let\HoLogoHtml@MiKTeX\HoLogo@MiKTeX
%    \end{macrocode}
%    \end{macro}
%
% \subsubsection{\hologo{OzTeX} and friends}
%
%    Source: \hologo{OzTeX} FAQ \cite{OzTeX}:
%    \begin{quote}
%      |\def\OzTeX{O\kern-.03em z\kern-.15em\TeX}|\\
%      (There is no kerning in OzMF, OzMP and OzTtH.)
%    \end{quote}
%
%    \begin{macro}{\HoLogo@OzTeX}
%    \begin{macrocode}
\def\HoLogo@OzTeX#1{%
  O%
  \kern-.03em %
  z%
  \kern-.15em %
  \hologo{TeX}%
}
%    \end{macrocode}
%    \end{macro}
%    \begin{macro}{\HoLogoHtml@OzTeX}
%    \begin{macrocode}
\def\HoLogoHtml@OzTeX#1{%
  \HoLogoCss@OzTeX
  \HOLOGO@Span{OzTeX}{%
    O%
    \HOLOGO@Span{z}{z}%
    \hologo{TeX}%
  }%
}
%    \end{macrocode}
%    \end{macro}
%    \begin{macro}{\HoLogoCss@OzTeX}
%    \begin{macrocode}
\def\HoLogoCss@OzTeX{%
  \Css{%
    span.HoLogo-OzTeX span.HoLogo-z{%
      margin-left:-.03em;%
      margin-right:-.15em;%
    }%
  }%
  \global\let\HoLogoCss@OzTeX\relax
}
%    \end{macrocode}
%    \end{macro}
%
%    \begin{macro}{\HoLogo@OzMF}
%    \begin{macrocode}
\def\HoLogo@OzMF#1{%
  \HOLOGO@mbox{OzMF}%
}
%    \end{macrocode}
%    \end{macro}
%    \begin{macro}{\HoLogo@OzMP}
%    \begin{macrocode}
\def\HoLogo@OzMP#1{%
  \HOLOGO@mbox{OzMP}%
}
%    \end{macrocode}
%    \end{macro}
%    \begin{macro}{\HoLogo@OzTtH}
%    \begin{macrocode}
\def\HoLogo@OzTtH#1{%
  \HOLOGO@mbox{OzTtH}%
}
%    \end{macrocode}
%    \end{macro}
%
% \subsubsection{\hologo{PCTeX}}
%
%    \begin{macro}{\HoLogo@PCTeX}
%    \begin{macrocode}
\def\HoLogo@PCTeX#1{%
  \HOLOGO@mbox{PC}%
  \hologo{TeX}%
}
%    \end{macrocode}
%    \end{macro}
%    \begin{macro}{\HoLogoHtml@PCTeX}
%    \begin{macrocode}
\let\HoLogoHtml@PCTeX\HoLogo@PCTeX
%    \end{macrocode}
%    \end{macro}
%
% \subsubsection{\hologo{PiCTeX}}
%
%    The original definitions from \xfile{pictex.tex} \cite{PiCTeX}:
%\begin{quote}
%\begin{verbatim}
%\def\PiC{%
%  P%
%  \kern-.12em%
%  \lower.5ex\hbox{I}%
%  \kern-.075em%
%  C%
%}
%\def\PiCTeX{%
%  \PiC
%  \kern-.11em%
%  \TeX
%}
%\end{verbatim}
%\end{quote}
%
%    \begin{macro}{\HoLogo@PiC}
%    \begin{macrocode}
\def\HoLogo@PiC#1{%
  P%
  \kern-.12em%
  \lower.5ex\hbox{I}%
  \kern-.075em%
  C%
  \HOLOGO@SpaceFactor
}
%    \end{macrocode}
%    \end{macro}
%    \begin{macro}{\HoLogoHtml@PiC}
%    \begin{macrocode}
\def\HoLogoHtml@PiC#1{%
  \HoLogoCss@PiC
  \HOLOGO@Span{PiC}{%
    P%
    \HOLOGO@Span{i}{I}%
    C%
  }%
}
%    \end{macrocode}
%    \end{macro}
%    \begin{macro}{\HoLogoCss@PiC}
%    \begin{macrocode}
\def\HoLogoCss@PiC{%
  \Css{%
    span.HoLogo-PiC span.HoLogo-i{%
      position:relative;%
      top:.5ex;%
      margin-left:-.12em;%
      margin-right:-.075em;%
      text-decoration:none;%
    }%
  }%
  \global\let\HoLogoCss@PiC\relax
}
%    \end{macrocode}
%    \end{macro}
%
%    \begin{macro}{\HoLogo@PiCTeX}
%    \begin{macrocode}
\def\HoLogo@PiCTeX#1{%
  \hologo{PiC}%
  \HOLOGO@discretionary
  \kern-.11em%
  \hologo{TeX}%
}
%    \end{macrocode}
%    \end{macro}
%    \begin{macro}{\HoLogoHtml@PiCTeX}
%    \begin{macrocode}
\def\HoLogoHtml@PiCTeX#1{%
  \HoLogoCss@PiCTeX
  \HOLOGO@Span{PiCTeX}{%
    \hologo{PiC}%
    \hologo{TeX}%
  }%
}
%    \end{macrocode}
%    \end{macro}
%    \begin{macro}{\HoLogoCss@PiCTeX}
%    \begin{macrocode}
\def\HoLogoCss@PiCTeX{%
  \Css{%
    span.HoLogo-PiCTeX span.HoLogo-PiC{%
      margin-right:-.11em;%
    }%
  }%
  \global\let\HoLogoCss@PiCTeX\relax
}
%    \end{macrocode}
%    \end{macro}
%
% \subsubsection{\hologo{teTeX}}
%
%    \begin{macro}{\HoLogo@teTeX}
%    \begin{macrocode}
\def\HoLogo@teTeX#1{%
  \HOLOGO@mbox{#1{t}{T}e}%
  \HOLOGO@discretionary
  \hologo{TeX}%
}
%    \end{macrocode}
%    \end{macro}
%    \begin{macro}{\HoLogoCs@teTeX}
%    \begin{macrocode}
\def\HoLogoCs@teTeX#1{#1{t}{T}dfTeX}
%    \end{macrocode}
%    \end{macro}
%    \begin{macro}{\HoLogoBkm@teTeX}
%    \begin{macrocode}
\def\HoLogoBkm@teTeX#1{%
  #1{t}{T}e\hologo{TeX}%
}
%    \end{macrocode}
%    \end{macro}
%    \begin{macro}{\HoLogoHtml@teTeX}
%    \begin{macrocode}
\let\HoLogoHtml@teTeX\HoLogo@teTeX
%    \end{macrocode}
%    \end{macro}
%
% \subsubsection{\hologo{TeX4ht}}
%
%    \begin{macro}{\HoLogo@TeX4ht}
%    \begin{macrocode}
\expandafter\def\csname HoLogo@TeX4ht\endcsname#1{%
  \HOLOGO@mbox{\hologo{TeX}4ht}%
}
%    \end{macrocode}
%    \end{macro}
%    \begin{macro}{\HoLogoHtml@TeX4ht}
%    \begin{macrocode}
\expandafter
\let\csname HoLogoHtml@TeX4ht\expandafter\endcsname
\csname HoLogo@TeX4ht\endcsname
%    \end{macrocode}
%    \end{macro}
%
%
% \subsubsection{\hologo{SageTeX}}
%
%    \begin{macro}{\HoLogo@SageTeX}
%    \begin{macrocode}
\def\HoLogo@SageTeX#1{%
  \HOLOGO@mbox{Sage}%
  \HOLOGO@discretionary
  \HOLOGO@NegativeKerning{eT,oT,To}%
  \hologo{TeX}%
}
%    \end{macrocode}
%    \end{macro}
%    \begin{macro}{\HoLogoHtml@SageTeX}
%    \begin{macrocode}
\let\HoLogoHtml@SageTeX\HoLogo@SageTeX
%    \end{macrocode}
%    \end{macro}
%
% \subsection{\hologo{METAFONT} and friends}
%
%    \begin{macro}{\HoLogo@METAFONT}
%    \begin{macrocode}
\def\HoLogo@METAFONT#1{%
  \HoLogoFont@font{METAFONT}{logo}{%
    \HOLOGO@mbox{META}%
    \HOLOGO@discretionary
    \HOLOGO@mbox{FONT}%
  }%
}
%    \end{macrocode}
%    \end{macro}
%
%    \begin{macro}{\HoLogo@METAPOST}
%    \begin{macrocode}
\def\HoLogo@METAPOST#1{%
  \HoLogoFont@font{METAPOST}{logo}{%
    \HOLOGO@mbox{META}%
    \HOLOGO@discretionary
    \HOLOGO@mbox{POST}%
  }%
}
%    \end{macrocode}
%    \end{macro}
%
%    \begin{macro}{\HoLogo@MetaFun}
%    \begin{macrocode}
\def\HoLogo@MetaFun#1{%
  \HOLOGO@mbox{Meta}%
  \HOLOGO@discretionary
  \HOLOGO@mbox{Fun}%
}
%    \end{macrocode}
%    \end{macro}
%
%    \begin{macro}{\HoLogo@MetaPost}
%    \begin{macrocode}
\def\HoLogo@MetaPost#1{%
  \HOLOGO@mbox{Meta}%
  \HOLOGO@discretionary
  \HOLOGO@mbox{Post}%
}
%    \end{macrocode}
%    \end{macro}
%
% \subsection{Others}
%
% \subsubsection{\hologo{biber}}
%
%    \begin{macro}{\HoLogo@biber}
%    \begin{macrocode}
\def\HoLogo@biber#1{%
  \HOLOGO@mbox{#1{b}{B}i}%
  \HOLOGO@discretionary
  \HOLOGO@mbox{ber}%
}
%    \end{macrocode}
%    \end{macro}
%    \begin{macro}{\HoLogoCs@biber}
%    \begin{macrocode}
\def\HoLogoCs@biber#1{#1{b}{B}iber}
%    \end{macrocode}
%    \end{macro}
%    \begin{macro}{\HoLogoBkm@biber}
%    \begin{macrocode}
\def\HoLogoBkm@biber#1{%
  #1{b}{B}iber%
}
%    \end{macrocode}
%    \end{macro}
%    \begin{macro}{\HoLogoHtml@biber}
%    \begin{macrocode}
\let\HoLogoHtml@biber\HoLogo@biber
%    \end{macrocode}
%    \end{macro}
%
% \subsubsection{\hologo{KOMAScript}}
%
%    \begin{macro}{\HoLogo@KOMAScript}
%    The definition for \hologo{KOMAScript} is taken
%    from \hologo{KOMAScript} (\xfile{scrlogo.dtx}, reformatted) \cite{scrlogo}:
%\begin{quote}
%\begin{verbatim}
%\@ifundefined{KOMAScript}{%
%  \DeclareRobustCommand{\KOMAScript}{%
%    \textsf{%
%      K\kern.05em O\kern.05emM\kern.05em A%
%      \kern.1em-\kern.1em %
%      Script%
%    }%
%  }%
%}{}
%\end{verbatim}
%\end{quote}
%    \begin{macrocode}
\def\HoLogo@KOMAScript#1{%
  \HoLogoFont@font{KOMAScript}{sf}{%
    \HOLOGO@mbox{%
      K\kern.05em%
      O\kern.05em%
      M\kern.05em%
      A%
    }%
    \kern.1em%
    \HOLOGO@hyphen
    \kern.1em%
    \HOLOGO@mbox{Script}%
  }%
}
%    \end{macrocode}
%    \end{macro}
%    \begin{macro}{\HoLogoBkm@KOMAScript}
%    \begin{macrocode}
\def\HoLogoBkm@KOMAScript#1{%
  KOMA-Script%
}
%    \end{macrocode}
%    \end{macro}
%    \begin{macro}{\HoLogoHtml@KOMAScript}
%    \begin{macrocode}
\def\HoLogoHtml@KOMAScript#1{%
  \HoLogoCss@KOMAScript
  \HoLogoFont@font{KOMAScript}{sf}{%
    \HOLOGO@Span{KOMAScript}{%
      K%
      \HOLOGO@Span{O}{O}%
      M%
      \HOLOGO@Span{A}{A}%
      \HOLOGO@Span{hyphen}{-}%
      Script%
    }%
  }%
}
%    \end{macrocode}
%    \end{macro}
%    \begin{macro}{\HoLogoCss@KOMAScript}
%    \begin{macrocode}
\def\HoLogoCss@KOMAScript{%
  \Css{%
    span.HoLogo-KOMAScript{%
      font-family:sans-serif;%
    }%
  }%
  \Css{%
    span.HoLogo-KOMAScript span.HoLogo-O{%
      padding-left:.05em;%
      padding-right:.05em;%
    }%
  }%
  \Css{%
    span.HoLogo-KOMAScript span.HoLogo-A{%
      padding-left:.05em;%
    }%
  }%
  \Css{%
    span.HoLogo-KOMAScript span.HoLogo-hyphen{%
      padding-left:.1em;%
      padding-right:.1em;%
    }%
  }%
  \global\let\HoLogoCss@KOMAScript\relax
}
%    \end{macrocode}
%    \end{macro}
%
% \subsubsection{\hologo{LyX}}
%
%    \begin{macro}{\HoLogo@LyX}
%    The definition is taken from the documentation source files
%    of \hologo{LyX}, \xfile{Intro.lyx} \cite{LyX}:
%\begin{quote}
%\begin{verbatim}
%\def\LyX{%
%  \texorpdfstring{%
%    L\kern-.1667em\lower.25em\hbox{Y}\kern-.125emX\@%
%  }{%
%    LyX%
%  }%
%}
%\end{verbatim}
%\end{quote}
%    \begin{macrocode}
\def\HoLogo@LyX#1{%
  L%
  \kern-.1667em%
  \lower.25em\hbox{Y}%
  \kern-.125em%
  X%
  \HOLOGO@SpaceFactor
}
%    \end{macrocode}
%    \end{macro}
%    \begin{macro}{\HoLogoHtml@LyX}
%    \begin{macrocode}
\def\HoLogoHtml@LyX#1{%
  \HoLogoCss@LyX
  \HOLOGO@Span{LyX}{%
    L%
    \HOLOGO@Span{y}{Y}%
    X%
  }%
}
%    \end{macrocode}
%    \end{macro}
%    \begin{macro}{\HoLogoCss@LyX}
%    \begin{macrocode}
\def\HoLogoCss@LyX{%
  \Css{%
    span.HoLogo-LyX span.HoLogo-y{%
      position:relative;%
      top:.25em;%
      margin-left:-.1667em;%
      margin-right:-.125em;%
      text-decoration:none;%
    }%
  }%
  \global\let\HoLogoCss@LyX\relax
}
%    \end{macrocode}
%    \end{macro}
%
% \subsubsection{\hologo{NTS}}
%
%    \begin{macro}{\HoLogo@NTS}
%    Definition for \hologo{NTS} can be found in
%    package \xpackage{etex\textunderscore man} for the \hologo{eTeX} manual \cite{etexman}
%    and in package \xpackage{dtklogos} \cite{dtklogos}:
%\begin{quote}
%\begin{verbatim}
%\def\NTS{%
%  \leavevmode
%  \hbox{%
%    $%
%      \cal N%
%      \kern-0.35em%
%      \lower0.5ex\hbox{$\cal T$}%
%      \kern-0.2em%
%      S%
%    $%
%  }%
%}
%\end{verbatim}
%\end{quote}
%    \begin{macrocode}
\def\HoLogo@NTS#1{%
  \HoLogoFont@font{NTS}{sy}{%
    N\/%
    \kern-.35em%
    \lower.5ex\hbox{T\/}%
    \kern-.2em%
    S\/%
  }%
  \HOLOGO@SpaceFactor
}
%    \end{macrocode}
%    \end{macro}
%
% \subsubsection{\Hologo{TTH} (\hologo{TeX} to HTML translator)}
%
%    Source: \url{http://hutchinson.belmont.ma.us/tth/}
%    In the HTML source the second `T' is printed as subscript.
%\begin{quote}
%\begin{verbatim}
%T<sub>T</sub>H
%\end{verbatim}
%\end{quote}
%    \begin{macro}{\HoLogo@TTH}
%    \begin{macrocode}
\def\HoLogo@TTH#1{%
  \ltx@mbox{%
    T\HOLOGO@SubScript{T}H%
  }%
  \HOLOGO@SpaceFactor
}
%    \end{macrocode}
%    \end{macro}
%
%    \begin{macro}{\HoLogoHtml@TTH}
%    \begin{macrocode}
\def\HoLogoHtml@TTH#1{%
  T\HCode{<sub>}T\HCode{</sub>}H%
}
%    \end{macrocode}
%    \end{macro}
%
% \subsubsection{\Hologo{HanTheThanh}}
%
%    Partial source: Package \xpackage{dtklogos}.
%    The double accent is U+1EBF (latin small letter e with circumflex
%    and acute).
%    \begin{macro}{\HoLogo@HanTheThanh}
%    \begin{macrocode}
\def\HoLogo@HanTheThanh#1{%
  \ltx@mbox{H\`an}%
  \HOLOGO@space
  \ltx@mbox{%
    Th%
    \HOLOGO@IfCharExists{"1EBF}{%
      \char"1EBF\relax
    }{%
      \^e\hbox to 0pt{\hss\raise .5ex\hbox{\'{}}}%
    }%
  }%
  \HOLOGO@space
  \ltx@mbox{Th\`anh}%
}
%    \end{macrocode}
%    \end{macro}
%    \begin{macro}{\HoLogoBkm@HanTheThanh}
%    \begin{macrocode}
\def\HoLogoBkm@HanTheThanh#1{%
  H\`an %
  Th\HOLOGO@PdfdocUnicode{\^e}{\9036\277} %
  Th\`anh%
}
%    \end{macrocode}
%    \end{macro}
%    \begin{macro}{\HoLogoHtml@HanTheThanh}
%    \begin{macrocode}
\def\HoLogoHtml@HanTheThanh#1{%
  H\`an %
  Th\HCode{&\ltx@hashchar x1ebf;} %
  Th\`anh%
}
%    \end{macrocode}
%    \end{macro}
%
% \subsection{Driver detection}
%
%    \begin{macrocode}
\HOLOGO@IfExists\InputIfFileExists{%
  \InputIfFileExists{hologo.cfg}{}{}%
}{%
  \ltx@IfUndefined{pdf@filesize}{%
    \def\HOLOGO@InputIfExists{%
      \openin\HOLOGO@temp=hologo.cfg\relax
      \ifeof\HOLOGO@temp
        \closein\HOLOGO@temp
      \else
        \closein\HOLOGO@temp
        \begingroup
          \def\x{LaTeX2e}%
        \expandafter\endgroup
        \ifx\fmtname\x
          \input{hologo.cfg}%
        \else
          \input hologo.cfg\relax
        \fi
      \fi
    }%
    \ltx@IfUndefined{newread}{%
      \chardef\HOLOGO@temp=15 %
      \def\HOLOGO@CheckRead{%
        \ifeof\HOLOGO@temp
          \HOLOGO@InputIfExists
        \else
          \ifcase\HOLOGO@temp
            \@PackageWarningNoLine{hologo}{%
              Configuration file ignored, because\MessageBreak
              a free read register could not be found%
            }%
          \else
            \begingroup
              \count\ltx@cclv=\HOLOGO@temp
              \advance\ltx@cclv by \ltx@minusone
              \edef\x{\endgroup
                \chardef\noexpand\HOLOGO@temp=\the\count\ltx@cclv
                \relax
              }%
            \x
          \fi
        \fi
      }%
    }{%
      \csname newread\endcsname\HOLOGO@temp
      \HOLOGO@InputIfExists
    }%
  }{%
    \edef\HOLOGO@temp{\pdf@filesize{hologo.cfg}}%
    \ifx\HOLOGO@temp\ltx@empty
    \else
      \ifnum\HOLOGO@temp>0 %
        \begingroup
          \def\x{LaTeX2e}%
        \expandafter\endgroup
        \ifx\fmtname\x
          \input{hologo.cfg}%
        \else
          \input hologo.cfg\relax
        \fi
      \else
        \@PackageInfoNoLine{hologo}{%
          Empty configuration file `hologo.cfg' ignored%
        }%
      \fi
    \fi
  }%
}
%    \end{macrocode}
%
%    \begin{macrocode}
\def\HOLOGO@temp#1#2{%
  \kv@define@key{HoLogoDriver}{#1}[]{%
    \begingroup
      \def\HOLOGO@temp{##1}%
      \ltx@onelevel@sanitize\HOLOGO@temp
      \ifx\HOLOGO@temp\ltx@empty
      \else
        \@PackageError{hologo}{%
          Value (\HOLOGO@temp) not permitted for option `#1'%
        }%
        \@ehc
      \fi
    \endgroup
    \def\hologoDriver{#2}%
  }%
}%
\def\HOLOGO@@temp#1#2{%
  \ifx\kv@value\relax
    \HOLOGO@temp{#1}{#1}%
  \else
    \HOLOGO@temp{#1}{#2}%
  \fi
}%
\kv@parse@normalized{%
  pdftex,%
  luatex=pdftex,%
  dvipdfm,%
  dvipdfmx=dvipdfm,%
  dvips,%
  dvipsone=dvips,%
  xdvi=dvips,%
  xetex,%
  vtex,%
}\HOLOGO@@temp
%    \end{macrocode}
%
%    \begin{macrocode}
\kv@define@key{HoLogoDriver}{driverfallback}{%
  \def\HOLOGO@DriverFallback{#1}%
}
%    \end{macrocode}
%
%    \begin{macro}{\HOLOGO@DriverFallback}
%    \begin{macrocode}
\def\HOLOGO@DriverFallback{dvips}
%    \end{macrocode}
%    \end{macro}
%
%    \begin{macro}{\hologoDriverSetup}
%    \begin{macrocode}
\def\hologoDriverSetup{%
  \let\hologoDriver\ltx@undefined
  \HOLOGO@DriverSetup
}
%    \end{macrocode}
%    \end{macro}
%
%    \begin{macro}{\HOLOGO@DriverSetup}
%    \begin{macrocode}
\def\HOLOGO@DriverSetup#1{%
  \kvsetkeys{HoLogoDriver}{#1}%
  \HOLOGO@CheckDriver
  \ltx@ifundefined{hologoDriver}{%
    \begingroup
    \edef\x{\endgroup
      \noexpand\kvsetkeys{HoLogoDriver}{\HOLOGO@DriverFallback}%
    }\x
  }{}%
  \@PackageInfoNoLine{hologo}{Using driver `\hologoDriver'}%
}
%    \end{macrocode}
%    \end{macro}
%
%    \begin{macro}{\HOLOGO@CheckDriver}
%    \begin{macrocode}
\def\HOLOGO@CheckDriver{%
  \ifpdf
    \def\hologoDriver{pdftex}%
    \let\HOLOGO@pdfliteral\pdfliteral
    \ifluatex
      \ifx\pdfextension\@undefined\else
        \protected\def\pdfliteral{\pdfextension literal}%
        \let\HOLOGO@pdfliteral\pdfliteral
      \fi
      \ltx@IfUndefined{HOLOGO@pdfliteral}{%
        \ifnum\luatexversion<36 %
        \else
          \begingroup
            \let\HOLOGO@temp\endgroup
            \ifcase0%
                \directlua{%
                  if tex.enableprimitives then %
                    tex.enableprimitives('HOLOGO@', {'pdfliteral'})%
                  else %
                    tex.print('1')%
                  end%
                }%
                \ifx\HOLOGO@pdfliteral\@undefined 1\fi%
                \relax%
              \endgroup
              \let\HOLOGO@temp\relax
              \global\let\HOLOGO@pdfliteral\HOLOGO@pdfliteral
            \fi%
          \HOLOGO@temp
        \fi
      }{}%
    \fi
    \ltx@IfUndefined{HOLOGO@pdfliteral}{%
      \@PackageWarningNoLine{hologo}{%
        Cannot find \string\pdfliteral
      }%
    }{}%
  \else
    \ifxetex
      \def\hologoDriver{xetex}%
    \else
      \ifvtex
        \def\hologoDriver{vtex}%
      \fi
    \fi
  \fi
}
%    \end{macrocode}
%    \end{macro}
%
%    \begin{macro}{\HOLOGO@WarningUnsupportedDriver}
%    \begin{macrocode}
\def\HOLOGO@WarningUnsupportedDriver#1{%
  \@PackageWarningNoLine{hologo}{%
    Logo `#1' needs driver specific macros,\MessageBreak
    but driver `\hologoDriver' is not supported.\MessageBreak
    Use a different driver or\MessageBreak
    load package `graphics' or `pgf'%
  }%
}
%    \end{macrocode}
%    \end{macro}
%
% \subsubsection{Reflect box macros}
%
%    Skip driver part if not needed.
%    \begin{macrocode}
\ltx@IfUndefined{reflectbox}{}{%
  \ltx@IfUndefined{rotatebox}{}{%
    \HOLOGO@AtEnd
  }%
}
\ltx@IfUndefined{pgftext}{}{%
  \HOLOGO@AtEnd
}
\ltx@IfUndefined{psscalebox}{}{%
  \HOLOGO@AtEnd
}
%    \end{macrocode}
%
%    \begin{macrocode}
\def\HOLOGO@temp{LaTeX2e}
\ifx\fmtname\HOLOGO@temp
  \RequirePackage{kvoptions}[2011/06/30]%
  \ProcessKeyvalOptions{HoLogoDriver}%
\fi
\HOLOGO@DriverSetup{}
%    \end{macrocode}
%
%    \begin{macro}{\HOLOGO@ReflectBox}
%    \begin{macrocode}
\def\HOLOGO@ReflectBox#1{%
  \begingroup
    \setbox\ltx@zero\hbox{\begingroup#1\endgroup}%
    \setbox\ltx@two\hbox{%
      \kern\wd\ltx@zero
      \csname HOLOGO@ScaleBox@\hologoDriver\endcsname{-1}{1}{%
        \hbox to 0pt{\copy\ltx@zero\hss}%
      }%
    }%
    \wd\ltx@two=\wd\ltx@zero
    \box\ltx@two
  \endgroup
}
%    \end{macrocode}
%    \end{macro}
%
%    \begin{macro}{\HOLOGO@PointReflectBox}
%    \begin{macrocode}
\def\HOLOGO@PointReflectBox#1{%
  \begingroup
    \setbox\ltx@zero\hbox{\begingroup#1\endgroup}%
    \setbox\ltx@two\hbox{%
      \kern\wd\ltx@zero
      \raise\ht\ltx@zero\hbox{%
        \csname HOLOGO@ScaleBox@\hologoDriver\endcsname{-1}{-1}{%
          \hbox to 0pt{\copy\ltx@zero\hss}%
        }%
      }%
    }%
    \wd\ltx@two=\wd\ltx@zero
    \box\ltx@two
  \endgroup
}
%    \end{macrocode}
%    \end{macro}
%
%    We must define all variants because of dynamic driver setup.
%    \begin{macrocode}
\def\HOLOGO@temp#1#2{#2}
%    \end{macrocode}
%
%    \begin{macro}{\HOLOGO@ScaleBox@pdftex}
%    \begin{macrocode}
\HOLOGO@temp{pdftex}{%
  \def\HOLOGO@ScaleBox@pdftex#1#2#3{%
    \HOLOGO@pdfliteral{%
      q #1 0 0 #2 0 0 cm%
    }%
    #3%
    \HOLOGO@pdfliteral{%
      Q%
    }%
  }%
}
%    \end{macrocode}
%    \end{macro}
%    \begin{macro}{\HOLOGO@ScaleBox@dvips}
%    \begin{macrocode}
\HOLOGO@temp{dvips}{%
  \def\HOLOGO@ScaleBox@dvips#1#2#3{%
    \special{ps:%
      gsave %
      currentpoint %
      currentpoint translate %
      #1 #2 scale %
      neg exch neg exch translate%
    }%
    #3%
    \special{ps:%
      currentpoint %
      grestore %
      moveto%
    }%
  }%
}
%    \end{macrocode}
%    \end{macro}
%    \begin{macro}{\HOLOGO@ScaleBox@dvipdfm}
%    \begin{macrocode}
\HOLOGO@temp{dvipdfm}{%
  \let\HOLOGO@ScaleBox@dvipdfm\HOLOGO@ScaleBox@dvips
}
%    \end{macrocode}
%    \end{macro}
%    Since \hologo{XeTeX} v0.6.
%    \begin{macro}{\HOLOGO@ScaleBox@xetex}
%    \begin{macrocode}
\HOLOGO@temp{xetex}{%
  \def\HOLOGO@ScaleBox@xetex#1#2#3{%
    \special{x:gsave}%
    \special{x:scale #1 #2}%
    #3%
    \special{x:grestore}%
  }%
}
%    \end{macrocode}
%    \end{macro}
%    \begin{macro}{\HOLOGO@ScaleBox@vtex}
%    \begin{macrocode}
\HOLOGO@temp{vtex}{%
  \def\HOLOGO@ScaleBox@vtex#1#2#3{%
    \special{r(#1,0,0,#2,0,0}%
    #3%
    \special{r)}%
  }%
}
%    \end{macrocode}
%    \end{macro}
%
%    \begin{macrocode}
\HOLOGO@AtEnd%
%</package>
%    \end{macrocode}
%
% \section{Test}
%
% \subsection{Catcode checks for loading}
%
%    \begin{macrocode}
%<*test1>
%    \end{macrocode}
%    \begin{macrocode}
\catcode`\{=1 %
\catcode`\}=2 %
\catcode`\#=6 %
\catcode`\@=11 %
\expandafter\ifx\csname count@\endcsname\relax
  \countdef\count@=255 %
\fi
\expandafter\ifx\csname @gobble\endcsname\relax
  \long\def\@gobble#1{}%
\fi
\expandafter\ifx\csname @firstofone\endcsname\relax
  \long\def\@firstofone#1{#1}%
\fi
\expandafter\ifx\csname loop\endcsname\relax
  \expandafter\@firstofone
\else
  \expandafter\@gobble
\fi
{%
  \def\loop#1\repeat{%
    \def\body{#1}%
    \iterate
  }%
  \def\iterate{%
    \body
      \let\next\iterate
    \else
      \let\next\relax
    \fi
    \next
  }%
  \let\repeat=\fi
}%
\def\RestoreCatcodes{}
\count@=0 %
\loop
  \edef\RestoreCatcodes{%
    \RestoreCatcodes
    \catcode\the\count@=\the\catcode\count@\relax
  }%
\ifnum\count@<255 %
  \advance\count@ 1 %
\repeat

\def\RangeCatcodeInvalid#1#2{%
  \count@=#1\relax
  \loop
    \catcode\count@=15 %
  \ifnum\count@<#2\relax
    \advance\count@ 1 %
  \repeat
}
\def\RangeCatcodeCheck#1#2#3{%
  \count@=#1\relax
  \loop
    \ifnum#3=\catcode\count@
    \else
      \errmessage{%
        Character \the\count@\space
        with wrong catcode \the\catcode\count@\space
        instead of \number#3%
      }%
    \fi
  \ifnum\count@<#2\relax
    \advance\count@ 1 %
  \repeat
}
\def\space{ }
\expandafter\ifx\csname LoadCommand\endcsname\relax
  \def\LoadCommand{\input hologo.sty\relax}%
\fi
\def\Test{%
  \RangeCatcodeInvalid{0}{47}%
  \RangeCatcodeInvalid{58}{64}%
  \RangeCatcodeInvalid{91}{96}%
  \RangeCatcodeInvalid{123}{255}%
  \catcode`\@=12 %
  \catcode`\\=0 %
  \catcode`\%=14 %
  \LoadCommand
  \RangeCatcodeCheck{0}{36}{15}%
  \RangeCatcodeCheck{37}{37}{14}%
  \RangeCatcodeCheck{38}{47}{15}%
  \RangeCatcodeCheck{48}{57}{12}%
  \RangeCatcodeCheck{58}{63}{15}%
  \RangeCatcodeCheck{64}{64}{12}%
  \RangeCatcodeCheck{65}{90}{11}%
  \RangeCatcodeCheck{91}{91}{15}%
  \RangeCatcodeCheck{92}{92}{0}%
  \RangeCatcodeCheck{93}{96}{15}%
  \RangeCatcodeCheck{97}{122}{11}%
  \RangeCatcodeCheck{123}{255}{15}%
  \RestoreCatcodes
}
\Test
\csname @@end\endcsname
\end
%    \end{macrocode}
%    \begin{macrocode}
%</test1>
%    \end{macrocode}
%
% \subsection{Spacefactor}
%
%    The space factor must be 1000 after a logo. If it is greater 1000
%    then the following space is a space after a sentence closing point.
%    If the space factor is smaller 1000 then an immediate following
%    dot is interpreted as abbreviation, not sentence closing point.
%
%    \begin{macrocode}
%<*test-spacefactor>
\NeedsTeXFormat{LaTeX2e}
\documentclass{article}
\usepackage{hologo}[2016/05/12]
\usepackage{kvsetkeys}
\usepackage{qstest}
\IncludeTests{*}
\LogTests{log}{*}{*}
\begin{document}
\begin{qstest}{spacefactor}{spacefactor}
\newcommand*{\Test}[1]{%
  \sbox0{%
    \hologo{#1}%
    \Expect*{1000 (#1)}*{\the\spacefactor\space(#1)}%
  }%
}%
\makeatletter
\def\TestList{}
\def\hologoEntry#1#2#3{%
  \edef\TestList{%
    \ifx\TestList\@empty
    \else
      \TestList,%
    \fi
    #1%
    \ifx\\#2\\%
    \else
      ={variant=#2}%
    \fi
  }%
}
\hologoList
\expandafter\kv@parse@normalized\expandafter{%
  \TestList
}{%
  \begingroup
    \let\@logo=\kv@key
    \ifx\kv@value\relax
    \else
      \expandafter\hologoLogoSetup\expandafter\@logo\expandafter{%
        \kv@value
      }%
    \fi
    \Test\@logo
  \endgroup
  \@gobbletwo
}
\end{qstest}
\end{document}
%</test-spacefactor>
%    \end{macrocode}
%
% \subsection{Complete list}
%
%    \begin{macrocode}
%<*test-list>
\NeedsTeXFormat{LaTeX2e}
\documentclass[12pt,a4paper]{article}
\usepackage{hologo}[2016/05/12]
\usepackage[T1]{fontenc}
\usepackage{lmodern}
\usepackage{parskip}
\usepackage[unicode]{hyperref}[2011/09/28]
\usepackage{bookmark}[2011/09/19]
\bookmarksetup{%
  numbered,%
  open,%
  openlevel=2,%
}
\renewcommand*{\contentsname}{List of logos}
\begin{document}
\tableofcontents
\def\TestFont#1#2#3#4#5#6{%
  \begingroup
    \usefont{#3}{#4}{#5}{#6}%
    \HologoVariant{#1}{#2}/\hologoVariant{#1}{#2}%
    \quad
    \begingroup\scriptsize\hologoVariant{#1}{#2}\endgroup
    \quad
  \endgroup
  (#3/#4/#5/#6)%
  \par
}
\makeatletter
\def\hologoEntry#1#2#3{%
  \section{%
    \HologoVariant{#1}{#2}/\hologoVariant{#1}{#2} %
    {[#1\ifx\\#2\\\else\space(#2)\fi]}% hash-ok
  }% braces around [] because of bug in tex4ht
  \begingroup
    \hypersetup{unicode=false}%
    \bookmark[%
      dest=\@currentHref,%
      rellevel=1,%
      keeplevel,%
    ]{%
      \HologoVariant{#1}{#2}/\hologoVariant{#1}{#2} %
      (PDFDocEncoding)%
    }%
  \endgroup
  \TestFont{#1}{#2}{OT1}{cmr}{m}{n}%
  \TestFont{#1}{#2}{OT1}{cmss}{m}{n}%
  \TestFont{#1}{#2}{OT1}{cmr}{b}{n}%
  \TestFont{#1}{#2}{OT1}{cmr}{m}{it}%
  \TestFont{#1}{#2}{OT1}{cmtt}{m}{n}%
  \TestFont{#1}{#2}{T1}{lmr}{m}{n}%
  \TestFont{#1}{#2}{T1}{lmss}{m}{n}%
  \TestFont{#1}{#2}{T1}{lmr}{b}{n}%
  \TestFont{#1}{#2}{T1}{lmr}{m}{it}%
  \TestFont{#1}{#2}{T1}{lmtt}{m}{n}%
  \TestFont{#1}{#2}{T1}{lmvtt}{m}{n}%
  \TestFont{#1}{#2}{T1}{qtm}{m}{n}%
  \TestFont{#1}{#2}{T1}{qhv}{m}{n}%
  \TestFont{#1}{#2}{T1}{qtm}{b}{n}%
  \TestFont{#1}{#2}{T1}{qtm}{m}{it}%
  \TestFont{#1}{#2}{T1}{qcr}{m}{n}%
  \newpage
}
\makeatother
\hologoList
\end{document}
%</test-list>
%    \end{macrocode}
%
% \section{Installation}
%
% \subsection{Download}
%
% \paragraph{Package.} This package is available on
% CTAN\footnote{\url{ftp://ftp.ctan.org/tex-archive/}}:
% \begin{description}
% \item[\CTAN{macros/latex/contrib/oberdiek/hologo.dtx}] The source file.
% \item[\CTAN{macros/latex/contrib/oberdiek/hologo.pdf}] Documentation.
% \end{description}
%
%
% \paragraph{Bundle.} All the packages of the bundle `oberdiek'
% are also available in a TDS compliant ZIP archive. There
% the packages are already unpacked and the documentation files
% are generated. The files and directories obey the TDS standard.
% \begin{description}
% \item[\CTAN{install/macros/latex/contrib/oberdiek.tds.zip}]
% \end{description}
% \emph{TDS} refers to the standard ``A Directory Structure
% for \TeX\ Files'' (\CTAN{tds/tds.pdf}). Directories
% with \xfile{texmf} in their name are usually organized this way.
%
% \subsection{Bundle installation}
%
% \paragraph{Unpacking.} Unpack the \xfile{oberdiek.tds.zip} in the
% TDS tree (also known as \xfile{texmf} tree) of your choice.
% Example (linux):
% \begin{quote}
%   |unzip oberdiek.tds.zip -d ~/texmf|
% \end{quote}
%
% \paragraph{Script installation.}
% Check the directory \xfile{TDS:scripts/oberdiek/} for
% scripts that need further installation steps.
% Package \xpackage{attachfile2} comes with the Perl script
% \xfile{pdfatfi.pl} that should be installed in such a way
% that it can be called as \texttt{pdfatfi}.
% Example (linux):
% \begin{quote}
%   |chmod +x scripts/oberdiek/pdfatfi.pl|\\
%   |cp scripts/oberdiek/pdfatfi.pl /usr/local/bin/|
% \end{quote}
%
% \subsection{Package installation}
%
% \paragraph{Unpacking.} The \xfile{.dtx} file is a self-extracting
% \docstrip\ archive. The files are extracted by running the
% \xfile{.dtx} through \plainTeX:
% \begin{quote}
%   \verb|tex hologo.dtx|
% \end{quote}
%
% \paragraph{TDS.} Now the different files must be moved into
% the different directories in your installation TDS tree
% (also known as \xfile{texmf} tree):
% \begin{quote}
% \def\t{^^A
% \begin{tabular}{@{}>{\ttfamily}l@{ $\rightarrow$ }>{\ttfamily}l@{}}
%   hologo.sty & tex/generic/oberdiek/hologo.sty\\
%   hologo.pdf & doc/latex/oberdiek/hologo.pdf\\
%   example/hologo-example.tex & doc/latex/oberdiek/example/hologo-example.tex\\
%   test/hologo-test1.tex & doc/latex/oberdiek/test/hologo-test1.tex\\
%   test/hologo-test-spacefactor.tex & doc/latex/oberdiek/test/hologo-test-spacefactor.tex\\
%   test/hologo-test-list.tex & doc/latex/oberdiek/test/hologo-test-list.tex\\
%   hologo.dtx & source/latex/oberdiek/hologo.dtx\\
% \end{tabular}^^A
% }^^A
% \sbox0{\t}^^A
% \ifdim\wd0>\linewidth
%   \begingroup
%     \advance\linewidth by\leftmargin
%     \advance\linewidth by\rightmargin
%   \edef\x{\endgroup
%     \def\noexpand\lw{\the\linewidth}^^A
%   }\x
%   \def\lwbox{^^A
%     \leavevmode
%     \hbox to \linewidth{^^A
%       \kern-\leftmargin\relax
%       \hss
%       \usebox0
%       \hss
%       \kern-\rightmargin\relax
%     }^^A
%   }^^A
%   \ifdim\wd0>\lw
%     \sbox0{\small\t}^^A
%     \ifdim\wd0>\linewidth
%       \ifdim\wd0>\lw
%         \sbox0{\footnotesize\t}^^A
%         \ifdim\wd0>\linewidth
%           \ifdim\wd0>\lw
%             \sbox0{\scriptsize\t}^^A
%             \ifdim\wd0>\linewidth
%               \ifdim\wd0>\lw
%                 \sbox0{\tiny\t}^^A
%                 \ifdim\wd0>\linewidth
%                   \lwbox
%                 \else
%                   \usebox0
%                 \fi
%               \else
%                 \lwbox
%               \fi
%             \else
%               \usebox0
%             \fi
%           \else
%             \lwbox
%           \fi
%         \else
%           \usebox0
%         \fi
%       \else
%         \lwbox
%       \fi
%     \else
%       \usebox0
%     \fi
%   \else
%     \lwbox
%   \fi
% \else
%   \usebox0
% \fi
% \end{quote}
% If you have a \xfile{docstrip.cfg} that configures and enables \docstrip's
% TDS installing feature, then some files can already be in the right
% place, see the documentation of \docstrip.
%
% \subsection{Refresh file name databases}
%
% If your \TeX~distribution
% (\teTeX, \mikTeX, \dots) relies on file name databases, you must refresh
% these. For example, \teTeX\ users run \verb|texhash| or
% \verb|mktexlsr|.
%
% \subsection{Some details for the interested}
%
% \paragraph{Attached source.}
%
% The PDF documentation on CTAN also includes the
% \xfile{.dtx} source file. It can be extracted by
% AcrobatReader 6 or higher. Another option is \textsf{pdftk},
% e.g. unpack the file into the current directory:
% \begin{quote}
%   \verb|pdftk hologo.pdf unpack_files output .|
% \end{quote}
%
% \paragraph{Unpacking with \LaTeX.}
% The \xfile{.dtx} chooses its action depending on the format:
% \begin{description}
% \item[\plainTeX:] Run \docstrip\ and extract the files.
% \item[\LaTeX:] Generate the documentation.
% \end{description}
% If you insist on using \LaTeX\ for \docstrip\ (really,
% \docstrip\ does not need \LaTeX), then inform the autodetect routine
% about your intention:
% \begin{quote}
%   \verb|latex \let\install=y\input{hologo.dtx}|
% \end{quote}
% Do not forget to quote the argument according to the demands
% of your shell.
%
% \paragraph{Generating the documentation.}
% You can use both the \xfile{.dtx} or the \xfile{.drv} to generate
% the documentation. The process can be configured by the
% configuration file \xfile{ltxdoc.cfg}. For instance, put this
% line into this file, if you want to have A4 as paper format:
% \begin{quote}
%   \verb|\PassOptionsToClass{a4paper}{article}|
% \end{quote}
% An example follows how to generate the
% documentation with pdf\LaTeX:
% \begin{quote}
%\begin{verbatim}
%pdflatex hologo.dtx
%makeindex -s gind.ist hologo.idx
%pdflatex hologo.dtx
%makeindex -s gind.ist hologo.idx
%pdflatex hologo.dtx
%\end{verbatim}
% \end{quote}
%
% \section{Catalogue}
%
% The following XML file can be used as source for the
% \href{http://mirror.ctan.org/help/Catalogue/catalogue.html}{\TeX\ Catalogue}.
% The elements \texttt{caption} and \texttt{description} are imported
% from the original XML file from the Catalogue.
% The name of the XML file in the Catalogue is \xfile{hologo.xml}.
%    \begin{macrocode}
%<*catalogue>
<?xml version='1.0' encoding='us-ascii'?>
<!DOCTYPE entry SYSTEM 'catalogue.dtd'>
<entry datestamp='$Date$' modifier='$Author$' id='hologo'>
  <name>hologo</name>
  <caption>A collection of logos with bookmark support.</caption>
  <authorref id='auth:oberdiek'/>
  <copyright owner='Heiko Oberdiek' year='2010-2012'/>
  <license type='lppl1.3'/>
  <version number='1.10'/>
  <description>
    The package defines a single command <tt>\hologo</tt>, whose
    argument is the usual case-confused ASCII version of the logo.
    The command is bookmark-enabled, so that every logo becomes
    available in bookmarks without further work.
    <p/>
    The package is part of the <xref refid='oberdiek'>oberdiek</xref>
    bundle.
  </description>
  <documentation details='Package documentation'
      href='ctan:/macros/latex/contrib/oberdiek/hologo.pdf'/>
  <ctan file='true' path='/macros/latex/contrib/oberdiek/hologo.dtx'/>
  <miktex location='oberdiek'/>
  <texlive location='oberdiek'/>
  <install path='/macros/latex/contrib/oberdiek/oberdiek.tds.zip'/>
</entry>
%</catalogue>
%    \end{macrocode}
%
% \begin{thebibliography}{9}
% \raggedright
%
% \bibitem{btxdoc}
% Oren Patashnik,
% \textit{\hologo{BibTeX}ing},
% 1988-02-08.\\
% \CTAN{biblio/bibtex/base/}
%
% \bibitem{dtklogos}
% Gerd Neugebauer, DANTE,
% \textit{Package \xpackage{dtklogos}},
% 2011-04-25.\\
% \CTAN{usergrps/dante/dtk/dtklogos.sty}
%
% \bibitem{etexman}
% The \hologo{NTS} Team,
% \textit{The \hologo{eTeX} manual},
% 1998-02.\\
% \CTAN{systems/e-tex/v2/doc/}
%
% \bibitem{ExTeX-FAQ}
% The \hologo{ExTeX} group,
% \textit{\hologo{ExTeX}: FAQ -- How is \hologo{ExTeX} typeset?},
% 2007-04-14.\\
% \url{http://www.extex.org/documentation/faq.html}
%
% \bibitem{LyX}
% %@MISC{ LyX,
% %  title = {{LyX 2.0.0 -- The Document Processor [Computer software and manual]}},
% %  author = {{The LyX Team}},
% %  howpublished = {Internet: http://www.lyx.org},
% %  year = {2011-05-08},
% %  note = {Retrieved May 10, 2011, from http://www.lyx.org},
% %  url = {http://www.lyx.org/}
% %}
% The \hologo{LyX} Team,
% \textit{\hologo{LyX} -- The Document Processor},
% 2011-05-08.\\
% \url{http://www.lyx.org/}
%
% \bibitem{OzTeX}
% Andrew Trevorrow,
% \hologo{OzTeX} FAQ: What is the correct way to typeset ``\hologo{OzTeX}''?,
% 2011-09-15 (visited).
% \url{http://www.trevorrow.com/oztex/ozfaq.html#oztex-logo}
%
% \bibitem{PiCTeX}
% Michael Wichura,
% \textit{The \hologo{PiCTeX} macro package},
% 1987-09-21.
% \CTAN{graphics/pictex/}
%
% \bibitem{scrlogo}
% Markus Kohm,
% \textit{\hologo{KOMAScript} Datei \xfile{scrlogo.dtx}},
% 2009-01-30.\\
% \CTAN{install/macros/latex/contrib/komascript.tds.zip}
%
% \end{thebibliography}
%
% \begin{History}
%   \begin{Version}{2010/04/08 v1.0}
%   \item
%     The first version.
%   \end{Version}
%   \begin{Version}{2010/04/16 v1.1}
%   \item
%     \cs{Hologo} added for support of logos at start of a sentence.
%   \item
%     \cs{hologoSetup} and \cs{hologoLogoSetup} added.
%   \item
%     Options \xoption{break}, \xoption{hyphenbreak}, \xoption{spacebreak}
%     added.
%   \item
%     Variant support added by option \xoption{variant}.
%   \end{Version}
%   \begin{Version}{2010/04/24 v1.2}
%   \item
%     \hologo{LaTeX3} added.
%   \item
%     \hologo{VTeX} added.
%   \end{Version}
%   \begin{Version}{2010/11/21 v1.3}
%   \item
%     \hologo{iniTeX}, \hologo{virTeX} added.
%   \end{Version}
%   \begin{Version}{2011/03/25 v1.4}
%   \item
%     \hologo{ConTeXt} with variants added.
%   \item
%     Option \xoption{discretionarybreak} added as refinement for
%     option \xoption{break}.
%   \end{Version}
%   \begin{Version}{2011/04/21 v1.5}
%   \item
%     Wrong TDS directory for test files fixed.
%   \end{Version}
%   \begin{Version}{2011/10/01 v1.6}
%   \item
%     Support for package \xpackage{tex4ht} added.
%   \item
%     Support for \cs{csname} added if \cs{ifincsname} is available.
%   \item
%     New logos:
%     \hologo{(La)TeX},
%     \hologo{biber},
%     \hologo{BibTeX} (\xoption{sc}, \xoption{sf}),
%     \hologo{emTeX},
%     \hologo{ExTeX},
%     \hologo{KOMAScript},
%     \hologo{La},
%     \hologo{LyX},
%     \hologo{MiKTeX},
%     \hologo{NTS},
%     \hologo{OzMF},
%     \hologo{OzMP},
%     \hologo{OzTeX},
%     \hologo{OzTtH},
%     \hologo{PCTeX},
%     \hologo{PiC},
%     \hologo{PiCTeX},
%     \hologo{METAFONT},
%     \hologo{MetaFun},
%     \hologo{METAPOST},
%     \hologo{MetaPost},
%     \hologo{SLiTeX} (\xoption{lift}, \xoption{narrow}, \xoption{simple}),
%     \hologo{SliTeX} (\xoption{narrow}, \xoption{simple}, \xoption{lift}),
%     \hologo{teTeX}.
%   \item
%     Fixes:
%     \hologo{iniTeX},
%     \hologo{pdfLaTeX},
%     \hologo{pdfTeX},
%     \hologo{virTeX}.
%   \item
%     \cs{hologoFontSetup} and \cs{hologoLogoFontSetup} added.
%   \item
%     \cs{hologoVariant} and \cs{HologoVariant} added.
%   \end{Version}
%   \begin{Version}{2011/11/22 v1.7}
%   \item
%     New logos:
%     \hologo{BibTeX8},
%     \hologo{LaTeXML},
%     \hologo{SageTeX},
%     \hologo{TeX4ht},
%     \hologo{TTH}.
%   \item
%     \hologo{Xe} and friends: Driver stuff fixed.
%   \item
%     \hologo{Xe} and friends: Support for italic added.
%   \item
%     \hologo{Xe} and friends: Package support for \xpackage{pgf}
%     and \xpackage{pstricks} added.
%   \end{Version}
%   \begin{Version}{2011/11/29 v1.8}
%   \item
%     New logos:
%     \hologo{HanTheThanh}.
%   \end{Version}
%   \begin{Version}{2011/12/21 v1.9}
%   \item
%     Patch for package \xpackage{ifxetex} added for the case that
%     \cs{newif} is undefined in \hologo{iniTeX}.
%   \item
%     Some fixes for \hologo{iniTeX}.
%   \end{Version}
%   \begin{Version}{2012/04/26 v1.10}
%   \item
%     Fix in bookmark version of logo ``\hologo{HanTheThanh}''.
%   \end{Version}
%   \begin{Version}{2016/05/12 v1.11}
%   \item
%     Update HOLOGO@IfCharExists (previously in texlive)
%   \item define pdfliteral in current luatex.
%   \end{Version}
% \end{History}
%
% \PrintIndex
%
% \Finale
\endinput
%
        \else
          \input hologo.cfg\relax
        \fi
      \else
        \@PackageInfoNoLine{hologo}{%
          Empty configuration file `hologo.cfg' ignored%
        }%
      \fi
    \fi
  }%
}
%    \end{macrocode}
%
%    \begin{macrocode}
\def\HOLOGO@temp#1#2{%
  \kv@define@key{HoLogoDriver}{#1}[]{%
    \begingroup
      \def\HOLOGO@temp{##1}%
      \ltx@onelevel@sanitize\HOLOGO@temp
      \ifx\HOLOGO@temp\ltx@empty
      \else
        \@PackageError{hologo}{%
          Value (\HOLOGO@temp) not permitted for option `#1'%
        }%
        \@ehc
      \fi
    \endgroup
    \def\hologoDriver{#2}%
  }%
}%
\def\HOLOGO@@temp#1#2{%
  \ifx\kv@value\relax
    \HOLOGO@temp{#1}{#1}%
  \else
    \HOLOGO@temp{#1}{#2}%
  \fi
}%
\kv@parse@normalized{%
  pdftex,%
  luatex=pdftex,%
  dvipdfm,%
  dvipdfmx=dvipdfm,%
  dvips,%
  dvipsone=dvips,%
  xdvi=dvips,%
  xetex,%
  vtex,%
}\HOLOGO@@temp
%    \end{macrocode}
%
%    \begin{macrocode}
\kv@define@key{HoLogoDriver}{driverfallback}{%
  \def\HOLOGO@DriverFallback{#1}%
}
%    \end{macrocode}
%
%    \begin{macro}{\HOLOGO@DriverFallback}
%    \begin{macrocode}
\def\HOLOGO@DriverFallback{dvips}
%    \end{macrocode}
%    \end{macro}
%
%    \begin{macro}{\hologoDriverSetup}
%    \begin{macrocode}
\def\hologoDriverSetup{%
  \let\hologoDriver\ltx@undefined
  \HOLOGO@DriverSetup
}
%    \end{macrocode}
%    \end{macro}
%
%    \begin{macro}{\HOLOGO@DriverSetup}
%    \begin{macrocode}
\def\HOLOGO@DriverSetup#1{%
  \kvsetkeys{HoLogoDriver}{#1}%
  \HOLOGO@CheckDriver
  \ltx@ifundefined{hologoDriver}{%
    \begingroup
    \edef\x{\endgroup
      \noexpand\kvsetkeys{HoLogoDriver}{\HOLOGO@DriverFallback}%
    }\x
  }{}%
  \@PackageInfoNoLine{hologo}{Using driver `\hologoDriver'}%
}
%    \end{macrocode}
%    \end{macro}
%
%    \begin{macro}{\HOLOGO@CheckDriver}
%    \begin{macrocode}
\def\HOLOGO@CheckDriver{%
  \ifpdf
    \def\hologoDriver{pdftex}%
    \let\HOLOGO@pdfliteral\pdfliteral
    \ifluatex
      \ifx\pdfextension\@undefined\else
        \protected\def\pdfliteral{\pdfextension literal}%
        \let\HOLOGO@pdfliteral\pdfliteral
      \fi
      \ltx@IfUndefined{HOLOGO@pdfliteral}{%
        \ifnum\luatexversion<36 %
        \else
          \begingroup
            \let\HOLOGO@temp\endgroup
            \ifcase0%
                \directlua{%
                  if tex.enableprimitives then %
                    tex.enableprimitives('HOLOGO@', {'pdfliteral'})%
                  else %
                    tex.print('1')%
                  end%
                }%
                \ifx\HOLOGO@pdfliteral\@undefined 1\fi%
                \relax%
              \endgroup
              \let\HOLOGO@temp\relax
              \global\let\HOLOGO@pdfliteral\HOLOGO@pdfliteral
            \fi%
          \HOLOGO@temp
        \fi
      }{}%
    \fi
    \ltx@IfUndefined{HOLOGO@pdfliteral}{%
      \@PackageWarningNoLine{hologo}{%
        Cannot find \string\pdfliteral
      }%
    }{}%
  \else
    \ifxetex
      \def\hologoDriver{xetex}%
    \else
      \ifvtex
        \def\hologoDriver{vtex}%
      \fi
    \fi
  \fi
}
%    \end{macrocode}
%    \end{macro}
%
%    \begin{macro}{\HOLOGO@WarningUnsupportedDriver}
%    \begin{macrocode}
\def\HOLOGO@WarningUnsupportedDriver#1{%
  \@PackageWarningNoLine{hologo}{%
    Logo `#1' needs driver specific macros,\MessageBreak
    but driver `\hologoDriver' is not supported.\MessageBreak
    Use a different driver or\MessageBreak
    load package `graphics' or `pgf'%
  }%
}
%    \end{macrocode}
%    \end{macro}
%
% \subsubsection{Reflect box macros}
%
%    Skip driver part if not needed.
%    \begin{macrocode}
\ltx@IfUndefined{reflectbox}{}{%
  \ltx@IfUndefined{rotatebox}{}{%
    \HOLOGO@AtEnd
  }%
}
\ltx@IfUndefined{pgftext}{}{%
  \HOLOGO@AtEnd
}
\ltx@IfUndefined{psscalebox}{}{%
  \HOLOGO@AtEnd
}
%    \end{macrocode}
%
%    \begin{macrocode}
\def\HOLOGO@temp{LaTeX2e}
\ifx\fmtname\HOLOGO@temp
  \RequirePackage{kvoptions}[2011/06/30]%
  \ProcessKeyvalOptions{HoLogoDriver}%
\fi
\HOLOGO@DriverSetup{}
%    \end{macrocode}
%
%    \begin{macro}{\HOLOGO@ReflectBox}
%    \begin{macrocode}
\def\HOLOGO@ReflectBox#1{%
  \begingroup
    \setbox\ltx@zero\hbox{\begingroup#1\endgroup}%
    \setbox\ltx@two\hbox{%
      \kern\wd\ltx@zero
      \csname HOLOGO@ScaleBox@\hologoDriver\endcsname{-1}{1}{%
        \hbox to 0pt{\copy\ltx@zero\hss}%
      }%
    }%
    \wd\ltx@two=\wd\ltx@zero
    \box\ltx@two
  \endgroup
}
%    \end{macrocode}
%    \end{macro}
%
%    \begin{macro}{\HOLOGO@PointReflectBox}
%    \begin{macrocode}
\def\HOLOGO@PointReflectBox#1{%
  \begingroup
    \setbox\ltx@zero\hbox{\begingroup#1\endgroup}%
    \setbox\ltx@two\hbox{%
      \kern\wd\ltx@zero
      \raise\ht\ltx@zero\hbox{%
        \csname HOLOGO@ScaleBox@\hologoDriver\endcsname{-1}{-1}{%
          \hbox to 0pt{\copy\ltx@zero\hss}%
        }%
      }%
    }%
    \wd\ltx@two=\wd\ltx@zero
    \box\ltx@two
  \endgroup
}
%    \end{macrocode}
%    \end{macro}
%
%    We must define all variants because of dynamic driver setup.
%    \begin{macrocode}
\def\HOLOGO@temp#1#2{#2}
%    \end{macrocode}
%
%    \begin{macro}{\HOLOGO@ScaleBox@pdftex}
%    \begin{macrocode}
\HOLOGO@temp{pdftex}{%
  \def\HOLOGO@ScaleBox@pdftex#1#2#3{%
    \HOLOGO@pdfliteral{%
      q #1 0 0 #2 0 0 cm%
    }%
    #3%
    \HOLOGO@pdfliteral{%
      Q%
    }%
  }%
}
%    \end{macrocode}
%    \end{macro}
%    \begin{macro}{\HOLOGO@ScaleBox@dvips}
%    \begin{macrocode}
\HOLOGO@temp{dvips}{%
  \def\HOLOGO@ScaleBox@dvips#1#2#3{%
    \special{ps:%
      gsave %
      currentpoint %
      currentpoint translate %
      #1 #2 scale %
      neg exch neg exch translate%
    }%
    #3%
    \special{ps:%
      currentpoint %
      grestore %
      moveto%
    }%
  }%
}
%    \end{macrocode}
%    \end{macro}
%    \begin{macro}{\HOLOGO@ScaleBox@dvipdfm}
%    \begin{macrocode}
\HOLOGO@temp{dvipdfm}{%
  \let\HOLOGO@ScaleBox@dvipdfm\HOLOGO@ScaleBox@dvips
}
%    \end{macrocode}
%    \end{macro}
%    Since \hologo{XeTeX} v0.6.
%    \begin{macro}{\HOLOGO@ScaleBox@xetex}
%    \begin{macrocode}
\HOLOGO@temp{xetex}{%
  \def\HOLOGO@ScaleBox@xetex#1#2#3{%
    \special{x:gsave}%
    \special{x:scale #1 #2}%
    #3%
    \special{x:grestore}%
  }%
}
%    \end{macrocode}
%    \end{macro}
%    \begin{macro}{\HOLOGO@ScaleBox@vtex}
%    \begin{macrocode}
\HOLOGO@temp{vtex}{%
  \def\HOLOGO@ScaleBox@vtex#1#2#3{%
    \special{r(#1,0,0,#2,0,0}%
    #3%
    \special{r)}%
  }%
}
%    \end{macrocode}
%    \end{macro}
%
%    \begin{macrocode}
\HOLOGO@AtEnd%
%</package>
%    \end{macrocode}
%
% \section{Test}
%
% \subsection{Catcode checks for loading}
%
%    \begin{macrocode}
%<*test1>
%    \end{macrocode}
%    \begin{macrocode}
\catcode`\{=1 %
\catcode`\}=2 %
\catcode`\#=6 %
\catcode`\@=11 %
\expandafter\ifx\csname count@\endcsname\relax
  \countdef\count@=255 %
\fi
\expandafter\ifx\csname @gobble\endcsname\relax
  \long\def\@gobble#1{}%
\fi
\expandafter\ifx\csname @firstofone\endcsname\relax
  \long\def\@firstofone#1{#1}%
\fi
\expandafter\ifx\csname loop\endcsname\relax
  \expandafter\@firstofone
\else
  \expandafter\@gobble
\fi
{%
  \def\loop#1\repeat{%
    \def\body{#1}%
    \iterate
  }%
  \def\iterate{%
    \body
      \let\next\iterate
    \else
      \let\next\relax
    \fi
    \next
  }%
  \let\repeat=\fi
}%
\def\RestoreCatcodes{}
\count@=0 %
\loop
  \edef\RestoreCatcodes{%
    \RestoreCatcodes
    \catcode\the\count@=\the\catcode\count@\relax
  }%
\ifnum\count@<255 %
  \advance\count@ 1 %
\repeat

\def\RangeCatcodeInvalid#1#2{%
  \count@=#1\relax
  \loop
    \catcode\count@=15 %
  \ifnum\count@<#2\relax
    \advance\count@ 1 %
  \repeat
}
\def\RangeCatcodeCheck#1#2#3{%
  \count@=#1\relax
  \loop
    \ifnum#3=\catcode\count@
    \else
      \errmessage{%
        Character \the\count@\space
        with wrong catcode \the\catcode\count@\space
        instead of \number#3%
      }%
    \fi
  \ifnum\count@<#2\relax
    \advance\count@ 1 %
  \repeat
}
\def\space{ }
\expandafter\ifx\csname LoadCommand\endcsname\relax
  \def\LoadCommand{\input hologo.sty\relax}%
\fi
\def\Test{%
  \RangeCatcodeInvalid{0}{47}%
  \RangeCatcodeInvalid{58}{64}%
  \RangeCatcodeInvalid{91}{96}%
  \RangeCatcodeInvalid{123}{255}%
  \catcode`\@=12 %
  \catcode`\\=0 %
  \catcode`\%=14 %
  \LoadCommand
  \RangeCatcodeCheck{0}{36}{15}%
  \RangeCatcodeCheck{37}{37}{14}%
  \RangeCatcodeCheck{38}{47}{15}%
  \RangeCatcodeCheck{48}{57}{12}%
  \RangeCatcodeCheck{58}{63}{15}%
  \RangeCatcodeCheck{64}{64}{12}%
  \RangeCatcodeCheck{65}{90}{11}%
  \RangeCatcodeCheck{91}{91}{15}%
  \RangeCatcodeCheck{92}{92}{0}%
  \RangeCatcodeCheck{93}{96}{15}%
  \RangeCatcodeCheck{97}{122}{11}%
  \RangeCatcodeCheck{123}{255}{15}%
  \RestoreCatcodes
}
\Test
\csname @@end\endcsname
\end
%    \end{macrocode}
%    \begin{macrocode}
%</test1>
%    \end{macrocode}
%
% \subsection{Spacefactor}
%
%    The space factor must be 1000 after a logo. If it is greater 1000
%    then the following space is a space after a sentence closing point.
%    If the space factor is smaller 1000 then an immediate following
%    dot is interpreted as abbreviation, not sentence closing point.
%
%    \begin{macrocode}
%<*test-spacefactor>
\NeedsTeXFormat{LaTeX2e}
\documentclass{article}
\usepackage{hologo}[2016/05/12]
\usepackage{kvsetkeys}
\usepackage{qstest}
\IncludeTests{*}
\LogTests{log}{*}{*}
\begin{document}
\begin{qstest}{spacefactor}{spacefactor}
\newcommand*{\Test}[1]{%
  \sbox0{%
    \hologo{#1}%
    \Expect*{1000 (#1)}*{\the\spacefactor\space(#1)}%
  }%
}%
\makeatletter
\def\TestList{}
\def\hologoEntry#1#2#3{%
  \edef\TestList{%
    \ifx\TestList\@empty
    \else
      \TestList,%
    \fi
    #1%
    \ifx\\#2\\%
    \else
      ={variant=#2}%
    \fi
  }%
}
\hologoList
\expandafter\kv@parse@normalized\expandafter{%
  \TestList
}{%
  \begingroup
    \let\@logo=\kv@key
    \ifx\kv@value\relax
    \else
      \expandafter\hologoLogoSetup\expandafter\@logo\expandafter{%
        \kv@value
      }%
    \fi
    \Test\@logo
  \endgroup
  \@gobbletwo
}
\end{qstest}
\end{document}
%</test-spacefactor>
%    \end{macrocode}
%
% \subsection{Complete list}
%
%    \begin{macrocode}
%<*test-list>
\NeedsTeXFormat{LaTeX2e}
\documentclass[12pt,a4paper]{article}
\usepackage{hologo}[2016/05/12]
\usepackage[T1]{fontenc}
\usepackage{lmodern}
\usepackage{parskip}
\usepackage[unicode]{hyperref}[2011/09/28]
\usepackage{bookmark}[2011/09/19]
\bookmarksetup{%
  numbered,%
  open,%
  openlevel=2,%
}
\renewcommand*{\contentsname}{List of logos}
\begin{document}
\tableofcontents
\def\TestFont#1#2#3#4#5#6{%
  \begingroup
    \usefont{#3}{#4}{#5}{#6}%
    \HologoVariant{#1}{#2}/\hologoVariant{#1}{#2}%
    \quad
    \begingroup\scriptsize\hologoVariant{#1}{#2}\endgroup
    \quad
  \endgroup
  (#3/#4/#5/#6)%
  \par
}
\makeatletter
\def\hologoEntry#1#2#3{%
  \section{%
    \HologoVariant{#1}{#2}/\hologoVariant{#1}{#2} %
    {[#1\ifx\\#2\\\else\space(#2)\fi]}% hash-ok
  }% braces around [] because of bug in tex4ht
  \begingroup
    \hypersetup{unicode=false}%
    \bookmark[%
      dest=\@currentHref,%
      rellevel=1,%
      keeplevel,%
    ]{%
      \HologoVariant{#1}{#2}/\hologoVariant{#1}{#2} %
      (PDFDocEncoding)%
    }%
  \endgroup
  \TestFont{#1}{#2}{OT1}{cmr}{m}{n}%
  \TestFont{#1}{#2}{OT1}{cmss}{m}{n}%
  \TestFont{#1}{#2}{OT1}{cmr}{b}{n}%
  \TestFont{#1}{#2}{OT1}{cmr}{m}{it}%
  \TestFont{#1}{#2}{OT1}{cmtt}{m}{n}%
  \TestFont{#1}{#2}{T1}{lmr}{m}{n}%
  \TestFont{#1}{#2}{T1}{lmss}{m}{n}%
  \TestFont{#1}{#2}{T1}{lmr}{b}{n}%
  \TestFont{#1}{#2}{T1}{lmr}{m}{it}%
  \TestFont{#1}{#2}{T1}{lmtt}{m}{n}%
  \TestFont{#1}{#2}{T1}{lmvtt}{m}{n}%
  \TestFont{#1}{#2}{T1}{qtm}{m}{n}%
  \TestFont{#1}{#2}{T1}{qhv}{m}{n}%
  \TestFont{#1}{#2}{T1}{qtm}{b}{n}%
  \TestFont{#1}{#2}{T1}{qtm}{m}{it}%
  \TestFont{#1}{#2}{T1}{qcr}{m}{n}%
  \newpage
}
\makeatother
\hologoList
\end{document}
%</test-list>
%    \end{macrocode}
%
% \section{Installation}
%
% \subsection{Download}
%
% \paragraph{Package.} This package is available on
% CTAN\footnote{\url{ftp://ftp.ctan.org/tex-archive/}}:
% \begin{description}
% \item[\CTAN{macros/latex/contrib/oberdiek/hologo.dtx}] The source file.
% \item[\CTAN{macros/latex/contrib/oberdiek/hologo.pdf}] Documentation.
% \end{description}
%
%
% \paragraph{Bundle.} All the packages of the bundle `oberdiek'
% are also available in a TDS compliant ZIP archive. There
% the packages are already unpacked and the documentation files
% are generated. The files and directories obey the TDS standard.
% \begin{description}
% \item[\CTAN{install/macros/latex/contrib/oberdiek.tds.zip}]
% \end{description}
% \emph{TDS} refers to the standard ``A Directory Structure
% for \TeX\ Files'' (\CTAN{tds/tds.pdf}). Directories
% with \xfile{texmf} in their name are usually organized this way.
%
% \subsection{Bundle installation}
%
% \paragraph{Unpacking.} Unpack the \xfile{oberdiek.tds.zip} in the
% TDS tree (also known as \xfile{texmf} tree) of your choice.
% Example (linux):
% \begin{quote}
%   |unzip oberdiek.tds.zip -d ~/texmf|
% \end{quote}
%
% \paragraph{Script installation.}
% Check the directory \xfile{TDS:scripts/oberdiek/} for
% scripts that need further installation steps.
% Package \xpackage{attachfile2} comes with the Perl script
% \xfile{pdfatfi.pl} that should be installed in such a way
% that it can be called as \texttt{pdfatfi}.
% Example (linux):
% \begin{quote}
%   |chmod +x scripts/oberdiek/pdfatfi.pl|\\
%   |cp scripts/oberdiek/pdfatfi.pl /usr/local/bin/|
% \end{quote}
%
% \subsection{Package installation}
%
% \paragraph{Unpacking.} The \xfile{.dtx} file is a self-extracting
% \docstrip\ archive. The files are extracted by running the
% \xfile{.dtx} through \plainTeX:
% \begin{quote}
%   \verb|tex hologo.dtx|
% \end{quote}
%
% \paragraph{TDS.} Now the different files must be moved into
% the different directories in your installation TDS tree
% (also known as \xfile{texmf} tree):
% \begin{quote}
% \def\t{^^A
% \begin{tabular}{@{}>{\ttfamily}l@{ $\rightarrow$ }>{\ttfamily}l@{}}
%   hologo.sty & tex/generic/oberdiek/hologo.sty\\
%   hologo.pdf & doc/latex/oberdiek/hologo.pdf\\
%   example/hologo-example.tex & doc/latex/oberdiek/example/hologo-example.tex\\
%   test/hologo-test1.tex & doc/latex/oberdiek/test/hologo-test1.tex\\
%   test/hologo-test-spacefactor.tex & doc/latex/oberdiek/test/hologo-test-spacefactor.tex\\
%   test/hologo-test-list.tex & doc/latex/oberdiek/test/hologo-test-list.tex\\
%   hologo.dtx & source/latex/oberdiek/hologo.dtx\\
% \end{tabular}^^A
% }^^A
% \sbox0{\t}^^A
% \ifdim\wd0>\linewidth
%   \begingroup
%     \advance\linewidth by\leftmargin
%     \advance\linewidth by\rightmargin
%   \edef\x{\endgroup
%     \def\noexpand\lw{\the\linewidth}^^A
%   }\x
%   \def\lwbox{^^A
%     \leavevmode
%     \hbox to \linewidth{^^A
%       \kern-\leftmargin\relax
%       \hss
%       \usebox0
%       \hss
%       \kern-\rightmargin\relax
%     }^^A
%   }^^A
%   \ifdim\wd0>\lw
%     \sbox0{\small\t}^^A
%     \ifdim\wd0>\linewidth
%       \ifdim\wd0>\lw
%         \sbox0{\footnotesize\t}^^A
%         \ifdim\wd0>\linewidth
%           \ifdim\wd0>\lw
%             \sbox0{\scriptsize\t}^^A
%             \ifdim\wd0>\linewidth
%               \ifdim\wd0>\lw
%                 \sbox0{\tiny\t}^^A
%                 \ifdim\wd0>\linewidth
%                   \lwbox
%                 \else
%                   \usebox0
%                 \fi
%               \else
%                 \lwbox
%               \fi
%             \else
%               \usebox0
%             \fi
%           \else
%             \lwbox
%           \fi
%         \else
%           \usebox0
%         \fi
%       \else
%         \lwbox
%       \fi
%     \else
%       \usebox0
%     \fi
%   \else
%     \lwbox
%   \fi
% \else
%   \usebox0
% \fi
% \end{quote}
% If you have a \xfile{docstrip.cfg} that configures and enables \docstrip's
% TDS installing feature, then some files can already be in the right
% place, see the documentation of \docstrip.
%
% \subsection{Refresh file name databases}
%
% If your \TeX~distribution
% (\teTeX, \mikTeX, \dots) relies on file name databases, you must refresh
% these. For example, \teTeX\ users run \verb|texhash| or
% \verb|mktexlsr|.
%
% \subsection{Some details for the interested}
%
% \paragraph{Attached source.}
%
% The PDF documentation on CTAN also includes the
% \xfile{.dtx} source file. It can be extracted by
% AcrobatReader 6 or higher. Another option is \textsf{pdftk},
% e.g. unpack the file into the current directory:
% \begin{quote}
%   \verb|pdftk hologo.pdf unpack_files output .|
% \end{quote}
%
% \paragraph{Unpacking with \LaTeX.}
% The \xfile{.dtx} chooses its action depending on the format:
% \begin{description}
% \item[\plainTeX:] Run \docstrip\ and extract the files.
% \item[\LaTeX:] Generate the documentation.
% \end{description}
% If you insist on using \LaTeX\ for \docstrip\ (really,
% \docstrip\ does not need \LaTeX), then inform the autodetect routine
% about your intention:
% \begin{quote}
%   \verb|latex \let\install=y% \iffalse meta-comment
%
% File: hologo.dtx
% Version: 2016/05/12 v1.11
% Info: A logo collection with bookmark support
%
% Copyright (C) 2010-2012 by
%    Heiko Oberdiek <heiko.oberdiek at googlemail.com>
%
% This work may be distributed and/or modified under the
% conditions of the LaTeX Project Public License, either
% version 1.3c of this license or (at your option) any later
% version. This version of this license is in
%    http://www.latex-project.org/lppl/lppl-1-3c.txt
% and the latest version of this license is in
%    http://www.latex-project.org/lppl.txt
% and version 1.3 or later is part of all distributions of
% LaTeX version 2005/12/01 or later.
%
% This work has the LPPL maintenance status "maintained".
%
% This Current Maintainer of this work is Heiko Oberdiek.
%
% The Base Interpreter refers to any `TeX-Format',
% because some files are installed in TDS:tex/generic//.
%
% This work consists of the main source file hologo.dtx
% and the derived files
%    hologo.sty, hologo.pdf, hologo.ins, hologo.drv, hologo-example.tex,
%    hologo-test1.tex, hologo-test-spacefactor.tex,
%    hologo-test-list.tex.
%
% Distribution:
%    CTAN:macros/latex/contrib/oberdiek/hologo.dtx
%    CTAN:macros/latex/contrib/oberdiek/hologo.pdf
%
% Unpacking:
%    (a) If hologo.ins is present:
%           tex hologo.ins
%    (b) Without hologo.ins:
%           tex hologo.dtx
%    (c) If you insist on using LaTeX
%           latex \let\install=y\input{hologo.dtx}
%        (quote the arguments according to the demands of your shell)
%
% Documentation:
%    (a) If hologo.drv is present:
%           latex hologo.drv
%    (b) Without hologo.drv:
%           latex hologo.dtx; ...
%    The class ltxdoc loads the configuration file ltxdoc.cfg
%    if available. Here you can specify further options, e.g.
%    use A4 as paper format:
%       \PassOptionsToClass{a4paper}{article}
%
%    Programm calls to get the documentation (example):
%       pdflatex hologo.dtx
%       makeindex -s gind.ist hologo.idx
%       pdflatex hologo.dtx
%       makeindex -s gind.ist hologo.idx
%       pdflatex hologo.dtx
%
% Installation:
%    TDS:tex/generic/oberdiek/hologo.sty
%    TDS:doc/latex/oberdiek/hologo.pdf
%    TDS:doc/latex/oberdiek/example/hologo-example.tex
%    TDS:doc/latex/oberdiek/test/hologo-test1.tex
%    TDS:doc/latex/oberdiek/test/hologo-test-spacefactor.tex
%    TDS:doc/latex/oberdiek/test/hologo-test-list.tex
%    TDS:source/latex/oberdiek/hologo.dtx
%
%<*ignore>
\begingroup
  \catcode123=1 %
  \catcode125=2 %
  \def\x{LaTeX2e}%
\expandafter\endgroup
\ifcase 0\ifx\install y1\fi\expandafter
         \ifx\csname processbatchFile\endcsname\relax\else1\fi
         \ifx\fmtname\x\else 1\fi\relax
\else\csname fi\endcsname
%</ignore>
%<*install>
\input docstrip.tex
\Msg{************************************************************************}
\Msg{* Installation}
\Msg{* Package: hologo 2016/05/12 v1.11 A logo collection with bookmark support (HO)}
\Msg{************************************************************************}

\keepsilent
\askforoverwritefalse

\let\MetaPrefix\relax
\preamble

This is a generated file.

Project: hologo
Version: 2016/05/12 v1.11

Copyright (C) 2010-2012 by
   Heiko Oberdiek <heiko.oberdiek at googlemail.com>

This work may be distributed and/or modified under the
conditions of the LaTeX Project Public License, either
version 1.3c of this license or (at your option) any later
version. This version of this license is in
   http://www.latex-project.org/lppl/lppl-1-3c.txt
and the latest version of this license is in
   http://www.latex-project.org/lppl.txt
and version 1.3 or later is part of all distributions of
LaTeX version 2005/12/01 or later.

This work has the LPPL maintenance status "maintained".

This Current Maintainer of this work is Heiko Oberdiek.

The Base Interpreter refers to any `TeX-Format',
because some files are installed in TDS:tex/generic//.

This work consists of the main source file hologo.dtx
and the derived files
   hologo.sty, hologo.pdf, hologo.ins, hologo.drv, hologo-example.tex,
   hologo-test1.tex, hologo-test-spacefactor.tex,
   hologo-test-list.tex.

\endpreamble
\let\MetaPrefix\DoubleperCent

\generate{%
  \file{hologo.ins}{\from{hologo.dtx}{install}}%
  \file{hologo.drv}{\from{hologo.dtx}{driver}}%
  \usedir{tex/generic/oberdiek}%
  \file{hologo.sty}{\from{hologo.dtx}{package}}%
  \usedir{doc/latex/oberdiek/example}%
  \file{hologo-example.tex}{\from{hologo.dtx}{example}}%
  \usedir{doc/latex/oberdiek/test}%
  \file{hologo-test1.tex}{\from{hologo.dtx}{test1}}%
  \file{hologo-test-spacefactor.tex}{\from{hologo.dtx}{test-spacefactor}}%
  \file{hologo-test-list.tex}{\from{hologo.dtx}{test-list}}%
  \nopreamble
  \nopostamble
  \usedir{source/latex/oberdiek/catalogue}%
  \file{hologo.xml}{\from{hologo.dtx}{catalogue}}%
}

\catcode32=13\relax% active space
\let =\space%
\Msg{************************************************************************}
\Msg{*}
\Msg{* To finish the installation you have to move the following}
\Msg{* file into a directory searched by TeX:}
\Msg{*}
\Msg{*     hologo.sty}
\Msg{*}
\Msg{* To produce the documentation run the file `hologo.drv'}
\Msg{* through LaTeX.}
\Msg{*}
\Msg{* Happy TeXing!}
\Msg{*}
\Msg{************************************************************************}

\endbatchfile
%</install>
%<*ignore>
\fi
%</ignore>
%<*driver>
\NeedsTeXFormat{LaTeX2e}
\ProvidesFile{hologo.drv}%
  [2016/05/12 v1.11 A logo collection with bookmark support (HO)]%
\documentclass{ltxdoc}
\usepackage{holtxdoc}[2011/11/22]
\usepackage{hologo}[2016/05/12]
\usepackage{longtable}
\usepackage{array}
\usepackage{paralist}
%\usepackage[T1]{fontenc}
%\usepackage{lmodern}
\begin{document}
  \DocInput{hologo.dtx}%
\end{document}
%</driver>
% \fi
%
%
% \CharacterTable
%  {Upper-case    \A\B\C\D\E\F\G\H\I\J\K\L\M\N\O\P\Q\R\S\T\U\V\W\X\Y\Z
%   Lower-case    \a\b\c\d\e\f\g\h\i\j\k\l\m\n\o\p\q\r\s\t\u\v\w\x\y\z
%   Digits        \0\1\2\3\4\5\6\7\8\9
%   Exclamation   \!     Double quote  \"     Hash (number) \#
%   Dollar        \$     Percent       \%     Ampersand     \&
%   Acute accent  \'     Left paren    \(     Right paren   \)
%   Asterisk      \*     Plus          \+     Comma         \,
%   Minus         \-     Point         \.     Solidus       \/
%   Colon         \:     Semicolon     \;     Less than     \<
%   Equals        \=     Greater than  \>     Question mark \?
%   Commercial at \@     Left bracket  \[     Backslash     \\
%   Right bracket \]     Circumflex    \^     Underscore    \_
%   Grave accent  \`     Left brace    \{     Vertical bar  \|
%   Right brace   \}     Tilde         \~}
%
% \GetFileInfo{hologo.drv}
%
% \title{The \xpackage{hologo} package}
% \date{2016/05/12 v1.11}
% \author{Heiko Oberdiek\\\xemail{heiko.oberdiek at googlemail.com}}
%
% \maketitle
%
% \begin{abstract}
% This package starts a collection of logos with support for bookmarks
% strings.
% \end{abstract}
%
% \tableofcontents
%
% \section{Documentation}
%
% \subsection{Logo macros}
%
% \begin{declcs}{hologo} \M{name}
% \end{declcs}
% Macro \cs{hologo} sets the logo with name \meta{name}.
% The following table shows the supported names.
%
% \begingroup
%   \def\hologoEntry#1#2#3{^^A
%     #1&#2&\hologoLogoSetup{#1}{variant=#2}\hologo{#1}&#3\tabularnewline
%   }
%   \begin{longtable}{>{\ttfamily}l>{\ttfamily}lll}
%     \rmfamily\bfseries{name} & \rmfamily\bfseries variant
%     & \bfseries logo & \bfseries since\\
%     \hline
%     \endhead
%     \hologoList
%   \end{longtable}
% \endgroup
%
% \begin{declcs}{Hologo} \M{name}
% \end{declcs}
% Macro \cs{Hologo} starts the logo \meta{name} with an uppercase
% letter. As an exception small greek letters are not converted
% to uppercase. Examples, see \hologo{eTeX} and \hologo{ExTeX}.
%
% \subsection{Setup macros}
%
% The package does not support package options, but the following
% setup macros can be used to set options.
%
% \begin{declcs}{hologoSetup} \M{key value list}
% \end{declcs}
% Macro \cs{hologoSetup} sets global options.
%
% \begin{declcs}{hologoLogoSetup} \M{logo} \M{key value list}
% \end{declcs}
% Some options can also be used to configure a logo.
% These settings take precedence over global option settings.
%
% \subsection{Options}\label{sec:options}
%
% There are boolean and string options:
% \begin{description}
% \item[Boolean option:]
% It takes |true| or |false|
% as value. If the value is omitted, then |true| is used.
% \item[String option:]
% A value must be given as string. (But the string might be empty.)
% \end{description}
% The following options can be used both in \cs{hologoSetup}
% and \cs{hologoLogoSetup}:
% \begin{description}
% \def\entry#1{\item[\xoption{#1}:]}
% \entry{break}
%   enables or disables line breaks inside the logo. This setting is
%   refined by options \xoption{hyphenbreak}, \xoption{spacebreak}
%   or \xoption{discretionarybreak}.
%   Default is |false|.
% \entry{hyphenbreak}
%   enables or disables the line break right after the hyphen character.
% \entry{spacebreak}
%   enables or disables line breaks at space characters.
% \entry{discretionarybreak}
%   enables or disables line breaks at hyphenation points
%   (inserted by \cs{-}).
% \end{description}
% Macro \cs{hologoLogoSetup} also knows:
% \begin{description}
% \item[\xoption{variant}:]
%   This is a string option. It specifies a variant of a logo that
%   must exist. An empty string selects the package default variant.
% \end{description}
% Example:
% \begin{quote}
%   |\hologoSetup{break=false}|\\
%   |\hologoLogoSetup{plainTeX}{variant=hyphen,hyphenbreak}|\\
%   Then ``plain-\TeX'' contains one break point after the hyphen.
% \end{quote}
%
% \subsection{Driver options}
%
% Sometimes graphical operations are needed to construct some
% glyphs (e.g.\ \hologo{XeTeX}). If package \xpackage{graphics}
% or package \xpackage{pgf} are found, then the macros are taken
% from there. Otherwise the packge defines its own operations
% and therefore needs the driver information. Many drivers are
% detected automatically (\hologo{pdfTeX}/\hologo{LuaTeX}
% in PDF mode, \hologo{XeTeX}, \hologo{VTeX}). These have precedence
% over a driver option. The driver can be given as package option
% or using \cs{hologoDriverSetup}.
% The following list contains the recognized driver options:
% \begin{itemize}
% \item \xoption{pdftex}, \xoption{luatex}
% \item \xoption{dvipdfm}, \xoption{dvipdfmx}
% \item \xoption{dvips}, \xoption{dvipsone}, \xoption{xdvi}
% \item \xoption{xetex}
% \item \xoption{vtex}
% \end{itemize}
% The left driver of a line is the driver name that is used internally.
% The following names are aliases for drivers that use the
% same method. Therefore the entry in the \xext{log} file for
% the used driver prints the internally used driver name.
% \begin{description}
% \item[\xoption{driverfallback}:]
%   This option expects a driver that is used,
%   if the driver could not be detected automatically.
% \end{description}
%
% \begin{declcs}{hologoDriverSetup} \M{driver option}
% \end{declcs}
% The driver can also be configured after package loading
% using \cs{hologoDriverSetup}, also the way for \hologo{plainTeX}
% to setup the driver.
%
% \subsection{Font setup}
%
% Some logos require a special font, but should also be usable by
% \hologo{plainTeX}. Therefore the package provides some ways
% to influence the font settings. The options below
% take font settings as values. Both font commands
% such as \cs{sffamily} and macros that take one argument
% like \cs{textsf} can be used.
%
% \begin{declcs}{hologoFontSetup} \M{key value list}
% \end{declcs}
% Macro \cs{hologoFontSetup} sets the fonts for all logos.
% Supported keys:
% \begin{description}
% \def\entry#1{\item[\xoption{#1}:]}
% \entry{general}
%   This font is used for all logos. The default is empty.
%   That means no special font is used.
% \entry{bibsf}
%   This font is used for
%   {\hologoLogoSetup{BibTeX}{variant=sf}\hologo{BibTeX}}
%   with variant \xoption{sf}.
% \entry{rm}
%   This font is a serif font. It is used for \hologo{ExTeX}.
% \entry{sc}
%   This font specifies a small caps font. It is used for
%   {\hologoLogoSetup{BibTeX}{variant=sc}\hologo{BibTeX}}
%   with variant \xoption{sc}.
% \entry{sf}
%   This font specifies a sans serif font. The default
%   is \cs{sffamily}, then \cs{sf} is tried. Otherwise
%   a warning is given. It is used by \hologo{KOMAScript}.
% \entry{sy}
%   This is the font for math symbols (e.g. cmsy).
%   It is used by \hologo{AmS}, \hologo{NTS}, \hologo{ExTeX}.
% \entry{logo}
%   \hologo{METAFONT} and \hologo{METAPOST} are using that font.
%   In \hologo{LaTeX} \cs{logofamily} is used and
%   the definitions of package \xpackage{mflogo} are used
%   if the package is not loaded.
%   Otherwise the \cs{tenlogo} is used and defined
%   if it does not already exists.
% \end{description}
%
% \begin{declcs}{hologoLogoFontSetup} \M{logo} \M{key value list}
% \end{declcs}
% Fonts can also be set for a logo or logo component separately,
% see the following list.
% The keys are the same as for \cs{hologoFontSetup}.
%
% \begin{longtable}{>{\ttfamily}l>{\sffamily}ll}
%   \meta{logo} & keys & result\\
%   \hline
%   \endhead
%   BibTeX & bibsf & {\hologoLogoSetup{BibTeX}{variant=sf}\hologo{BibTeX}}\\[.5ex]
%   BibTeX & sc & {\hologoLogoSetup{BibTeX}{variant=sc}\hologo{BibTeX}}\\[.5ex]
%   ExTeX & rm & \hologo{ExTeX}\\
%   SliTeX & rm & \hologo{SliTeX}\\[.5ex]
%   AmS & sy & \hologo{AmS}\\
%   ExTeX & sy & \hologo{ExTeX}\\
%   NTS & sy & \hologo{NTS}\\[.5ex]
%   KOMAScript & sf & \hologo{KOMAScript}\\[.5ex]
%   METAFONT & logo & \hologo{METAFONT}\\
%   METAPOST & logo & \hologo{METAPOST}\\[.5ex]
%   SliTeX & sc \hologo{SliTeX}
% \end{longtable}
%
% \subsubsection{Font order}
%
% For all logos the font \xoption{general} is applied first.
% Example:
%\begin{quote}
%|\hologoFontSetup{general=\color{red}}|
%\end{quote}
% will print red logos.
% Then if the font uses a special font \xoption{sf}, for example,
% the font is applied that is setup by \cs{hologoLogoFontSetup}.
% If this font is not setup, then the common font setup
% by \cs{hologoFontSetup} is used. Otherwise a warning is given,
% that there is no font configured.
%
% \subsection{Additional user macros}
%
% Usually a variant of a logo is configured by using
% \cs{hologoLogoSetup}, because it is bad style to mix
% different variants of the same logo in the same text.
% There the following macros are a convenience for testing.
%
% \begin{declcs}{hologoVariant} \M{name} \M{variant}\\
%   \cs{HologoVariant} \M{name} \M{variant}
% \end{declcs}
% Logo \meta{name} is set using \meta{variant} that specifies
% explicitely which variant of the macro is used. If the argument
% is empty, then the default form of the logo is used
% (configurable by \cs{hologoLogoSetup}).
%
% \cs{HologoVariant} is used if the logo is set in a context
% that needs an uppercase first letter (beginning of a sentence, \dots).
%
% \begin{declcs}{hologoList}\\
%   \cs{hologoEntry} \M{logo} \M{variant} \M{since}
% \end{declcs}
% Macro \cs{hologoList} contains all logos that are provided
% by the package including variants. The list consists of calls
% of \cs{hologoEntry} with three arguments starting with the
% logo name \meta{logo} and its variant \meta{variant}. An empty
% variant means the current default. Argument \meta{since} specifies
% with version of the package \xpackage{hologo} is needed to get
% the logo. If the logo is fixed, then the date gets updated.
% Therefore the date \meta{since} is not exactly the date of
% the first introduction, but rather the date of the latest fix.
%
% Before \cs{hologoList} can be used, macro \cs{hologoEntry} needs
% a definition. The example file in section \ref{sec:example}
% shows applications of \cs{hologoList}.
%
% \subsection{Supported contexts}
%
% Macros \cs{hologo} and friends support special contexts:
% \begin{itemize}
% \item \hologo{LaTeX}'s protection mechanism.
% \item Bookmarks of package \xpackage{hyperref}.
% \item Package \xpackage{tex4ht}.
% \item The macros can be used inside \cs{csname} constructs,
%   if \cs{ifincsname} is available (\hologo{pdfTeX}, \hologo{XeTeX},
%   \hologo{LuaTeX}).
% \end{itemize}
%
% \subsection{Example}
% \label{sec:example}
%
% The following example prints the logos in different fonts.
%    \begin{macrocode}
%<*example>
%<<verbatim
\NeedsTeXFormat{LaTeX2e}
\documentclass[a4paper]{article}
\usepackage[
  hmargin=20mm,
  vmargin=20mm,
]{geometry}
\pagestyle{empty}
\usepackage{hologo}[2016/05/12]
\usepackage{longtable}
\usepackage{array}
\setlength{\extrarowheight}{2pt}
\usepackage[T1]{fontenc}
\usepackage{lmodern}
\usepackage{pdflscape}
\usepackage[
  pdfencoding=auto,
]{hyperref}
\hypersetup{
  pdfauthor={Heiko Oberdiek},
  pdftitle={Example for package `hologo'},
  pdfsubject={Logos with fonts lmr, lmss, qtm, qpl, qhv},
}
\usepackage{bookmark}

% Print the logo list on the console

\begingroup
  \typeout{}%
  \typeout{*** Begin of logo list ***}%
  \newcommand*{\hologoEntry}[3]{%
    \typeout{#1 \ifx\\#2\\\else(#2) \fi[#3]}%
  }%
  \hologoList
  \typeout{*** End of logo list ***}%
  \typeout{}%
\endgroup

\begin{document}
\begin{landscape}

  \section{Example file for package `hologo'}

  % Table for font names

  \begin{longtable}{>{\bfseries}ll}
    \textbf{font} & \textbf{Font name}\\
    \hline
    lmr & Latin Modern Roman\\
    lmss & Latin Modern Sans\\
    qtm & \TeX\ Gyre Termes\\
    qhv & \TeX\ Gyre Heros\\
    qpl & \TeX\ Gyre Pagella\\
  \end{longtable}

  % Logo list with logos in different fonts

  \begingroup
    \newcommand*{\SetVariant}[2]{%
      \ifx\\#2\\%
      \else
        \hologoLogoSetup{#1}{variant=#2}%
      \fi
    }%
    \newcommand*{\hologoEntry}[3]{%
      \SetVariant{#1}{#2}%
      \raisebox{1em}[0pt][0pt]{\hypertarget{#1@#2}{}}%
      \bookmark[%
        dest={#1@#2},%
      ]{%
        #1\ifx\\#2\\\else\space(#2)\fi: \Hologo{#1}, \hologo{#1} %
        [Unicode]%
      }%
      \hypersetup{unicode=false}%
      \bookmark[%
        dest={#1@#2},%
      ]{%
        #1\ifx\\#2\\\else\space(#2)\fi: \Hologo{#1}, \hologo{#1} %
        [PDFDocEncoding]%
      }%
      \texttt{#1}%
      &%
      \texttt{#2}%
      &%
      \Hologo{#1}%
      &%
      \SetVariant{#1}{#2}%
      \hologo{#1}%
      &%
      \SetVariant{#1}{#2}%
      \fontfamily{qtm}\selectfont
      \hologo{#1}%
      &%
      \SetVariant{#1}{#2}%
      \fontfamily{qpl}\selectfont
      \hologo{#1}%
      &%
      \SetVariant{#1}{#2}%
      \textsf{\hologo{#1}}%
      &%
      \SetVariant{#1}{#2}%
      \fontfamily{qhv}\selectfont
      \hologo{#1}%
      \tabularnewline
    }%
    \begin{longtable}{llllllll}%
      \textbf{\textit{logo}} & \textbf{\textit{variant}} &
      \texttt{\string\Hologo} &
      \textbf{lmr} & \textbf{qtm} & \textbf{qpl} &
      \textbf{lmss} & \textbf{qhv}
      \tabularnewline
      \hline
      \endhead
      \hologoList
    \end{longtable}%
  \endgroup

\end{landscape}
\end{document}
%verbatim
%</example>
%    \end{macrocode}
%
% \StopEventually{
% }
%
% \section{Implementation}
%    \begin{macrocode}
%<*package>
%    \end{macrocode}
%    Reload check, especially if the package is not used with \LaTeX.
%    \begin{macrocode}
\begingroup\catcode61\catcode48\catcode32=10\relax%
  \catcode13=5 % ^^M
  \endlinechar=13 %
  \catcode35=6 % #
  \catcode39=12 % '
  \catcode44=12 % ,
  \catcode45=12 % -
  \catcode46=12 % .
  \catcode58=12 % :
  \catcode64=11 % @
  \catcode123=1 % {
  \catcode125=2 % }
  \expandafter\let\expandafter\x\csname ver@hologo.sty\endcsname
  \ifx\x\relax % plain-TeX, first loading
  \else
    \def\empty{}%
    \ifx\x\empty % LaTeX, first loading,
      % variable is initialized, but \ProvidesPackage not yet seen
    \else
      \expandafter\ifx\csname PackageInfo\endcsname\relax
        \def\x#1#2{%
          \immediate\write-1{Package #1 Info: #2.}%
        }%
      \else
        \def\x#1#2{\PackageInfo{#1}{#2, stopped}}%
      \fi
      \x{hologo}{The package is already loaded}%
      \aftergroup\endinput
    \fi
  \fi
\endgroup%
%    \end{macrocode}
%    Package identification:
%    \begin{macrocode}
\begingroup\catcode61\catcode48\catcode32=10\relax%
  \catcode13=5 % ^^M
  \endlinechar=13 %
  \catcode35=6 % #
  \catcode39=12 % '
  \catcode40=12 % (
  \catcode41=12 % )
  \catcode44=12 % ,
  \catcode45=12 % -
  \catcode46=12 % .
  \catcode47=12 % /
  \catcode58=12 % :
  \catcode64=11 % @
  \catcode91=12 % [
  \catcode93=12 % ]
  \catcode123=1 % {
  \catcode125=2 % }
  \expandafter\ifx\csname ProvidesPackage\endcsname\relax
    \def\x#1#2#3[#4]{\endgroup
      \immediate\write-1{Package: #3 #4}%
      \xdef#1{#4}%
    }%
  \else
    \def\x#1#2[#3]{\endgroup
      #2[{#3}]%
      \ifx#1\@undefined
        \xdef#1{#3}%
      \fi
      \ifx#1\relax
        \xdef#1{#3}%
      \fi
    }%
  \fi
\expandafter\x\csname ver@hologo.sty\endcsname
\ProvidesPackage{hologo}%
  [2016/05/12 v1.11 A logo collection with bookmark support (HO)]%
%    \end{macrocode}
%
%    \begin{macrocode}
\begingroup\catcode61\catcode48\catcode32=10\relax%
  \catcode13=5 % ^^M
  \endlinechar=13 %
  \catcode123=1 % {
  \catcode125=2 % }
  \catcode64=11 % @
  \def\x{\endgroup
    \expandafter\edef\csname HOLOGO@AtEnd\endcsname{%
      \endlinechar=\the\endlinechar\relax
      \catcode13=\the\catcode13\relax
      \catcode32=\the\catcode32\relax
      \catcode35=\the\catcode35\relax
      \catcode61=\the\catcode61\relax
      \catcode64=\the\catcode64\relax
      \catcode123=\the\catcode123\relax
      \catcode125=\the\catcode125\relax
    }%
  }%
\x\catcode61\catcode48\catcode32=10\relax%
\catcode13=5 % ^^M
\endlinechar=13 %
\catcode35=6 % #
\catcode64=11 % @
\catcode123=1 % {
\catcode125=2 % }
\def\TMP@EnsureCode#1#2{%
  \edef\HOLOGO@AtEnd{%
    \HOLOGO@AtEnd
    \catcode#1=\the\catcode#1\relax
  }%
  \catcode#1=#2\relax
}
\TMP@EnsureCode{10}{12}% ^^J
\TMP@EnsureCode{33}{12}% !
\TMP@EnsureCode{34}{12}% "
\TMP@EnsureCode{36}{3}% $
\TMP@EnsureCode{38}{4}% &
\TMP@EnsureCode{39}{12}% '
\TMP@EnsureCode{40}{12}% (
\TMP@EnsureCode{41}{12}% )
\TMP@EnsureCode{42}{12}% *
\TMP@EnsureCode{43}{12}% +
\TMP@EnsureCode{44}{12}% ,
\TMP@EnsureCode{45}{12}% -
\TMP@EnsureCode{46}{12}% .
\TMP@EnsureCode{47}{12}% /
\TMP@EnsureCode{58}{12}% :
\TMP@EnsureCode{59}{12}% ;
\TMP@EnsureCode{60}{12}% <
\TMP@EnsureCode{62}{12}% >
\TMP@EnsureCode{63}{12}% ?
\TMP@EnsureCode{91}{12}% [
\TMP@EnsureCode{93}{12}% ]
\TMP@EnsureCode{94}{7}% ^ (superscript)
\TMP@EnsureCode{95}{8}% _ (subscript)
\TMP@EnsureCode{96}{12}% `
\TMP@EnsureCode{124}{12}% |
\edef\HOLOGO@AtEnd{%
  \HOLOGO@AtEnd
  \escapechar\the\escapechar\relax
  \noexpand\endinput
}
\escapechar=92 %
%    \end{macrocode}
%
% \subsection{Logo list}
%
%    \begin{macro}{\hologoList}
%    \begin{macrocode}
\def\hologoList{%
  \hologoEntry{(La)TeX}{}{2011/10/01}%
  \hologoEntry{AmSLaTeX}{}{2010/04/16}%
  \hologoEntry{AmSTeX}{}{2010/04/16}%
  \hologoEntry{biber}{}{2011/10/01}%
  \hologoEntry{BibTeX}{}{2011/10/01}%
  \hologoEntry{BibTeX}{sf}{2011/10/01}%
  \hologoEntry{BibTeX}{sc}{2011/10/01}%
  \hologoEntry{BibTeX8}{}{2011/11/22}%
  \hologoEntry{ConTeXt}{}{2011/03/25}%
  \hologoEntry{ConTeXt}{narrow}{2011/03/25}%
  \hologoEntry{ConTeXt}{simple}{2011/03/25}%
  \hologoEntry{emTeX}{}{2010/04/26}%
  \hologoEntry{eTeX}{}{2010/04/08}%
  \hologoEntry{ExTeX}{}{2011/10/01}%
  \hologoEntry{HanTheThanh}{}{2011/11/29}%
  \hologoEntry{iniTeX}{}{2011/10/01}%
  \hologoEntry{KOMAScript}{}{2011/10/01}%
  \hologoEntry{La}{}{2010/05/08}%
  \hologoEntry{LaTeX}{}{2010/04/08}%
  \hologoEntry{LaTeX2e}{}{2010/04/08}%
  \hologoEntry{LaTeX3}{}{2010/04/24}%
  \hologoEntry{LaTeXe}{}{2010/04/08}%
  \hologoEntry{LaTeXML}{}{2011/11/22}%
  \hologoEntry{LaTeXTeX}{}{2011/10/01}%
  \hologoEntry{LuaLaTeX}{}{2010/04/08}%
  \hologoEntry{LuaTeX}{}{2010/04/08}%
  \hologoEntry{LyX}{}{2011/10/01}%
  \hologoEntry{METAFONT}{}{2011/10/01}%
  \hologoEntry{MetaFun}{}{2011/10/01}%
  \hologoEntry{METAPOST}{}{2011/10/01}%
  \hologoEntry{MetaPost}{}{2011/10/01}%
  \hologoEntry{MiKTeX}{}{2011/10/01}%
  \hologoEntry{NTS}{}{2011/10/01}%
  \hologoEntry{OzMF}{}{2011/10/01}%
  \hologoEntry{OzMP}{}{2011/10/01}%
  \hologoEntry{OzTeX}{}{2011/10/01}%
  \hologoEntry{OzTtH}{}{2011/10/01}%
  \hologoEntry{PCTeX}{}{2011/10/01}%
  \hologoEntry{pdfTeX}{}{2011/10/01}%
  \hologoEntry{pdfLaTeX}{}{2011/10/01}%
  \hologoEntry{PiC}{}{2011/10/01}%
  \hologoEntry{PiCTeX}{}{2011/10/01}%
  \hologoEntry{plainTeX}{}{2010/04/08}%
  \hologoEntry{plainTeX}{space}{2010/04/16}%
  \hologoEntry{plainTeX}{hyphen}{2010/04/16}%
  \hologoEntry{plainTeX}{runtogether}{2010/04/16}%
  \hologoEntry{SageTeX}{}{2011/11/22}%
  \hologoEntry{SLiTeX}{}{2011/10/01}%
  \hologoEntry{SLiTeX}{lift}{2011/10/01}%
  \hologoEntry{SLiTeX}{narrow}{2011/10/01}%
  \hologoEntry{SLiTeX}{simple}{2011/10/01}%
  \hologoEntry{SliTeX}{}{2011/10/01}%
  \hologoEntry{SliTeX}{narrow}{2011/10/01}%
  \hologoEntry{SliTeX}{simple}{2011/10/01}%
  \hologoEntry{SliTeX}{lift}{2011/10/01}%
  \hologoEntry{teTeX}{}{2011/10/01}%
  \hologoEntry{TeX}{}{2010/04/08}%
  \hologoEntry{TeX4ht}{}{2011/11/22}%
  \hologoEntry{TTH}{}{2011/11/22}%
  \hologoEntry{virTeX}{}{2011/10/01}%
  \hologoEntry{VTeX}{}{2010/04/24}%
  \hologoEntry{Xe}{}{2010/04/08}%
  \hologoEntry{XeLaTeX}{}{2010/04/08}%
  \hologoEntry{XeTeX}{}{2010/04/08}%
}
%    \end{macrocode}
%    \end{macro}
%
% \subsection{Load resources}
%
%    \begin{macrocode}
\begingroup\expandafter\expandafter\expandafter\endgroup
\expandafter\ifx\csname RequirePackage\endcsname\relax
  \def\TMP@RequirePackage#1[#2]{%
    \begingroup\expandafter\expandafter\expandafter\endgroup
    \expandafter\ifx\csname ver@#1.sty\endcsname\relax
      \input #1.sty\relax
    \fi
  }%
  \TMP@RequirePackage{ltxcmds}[2011/02/04]%
  \TMP@RequirePackage{infwarerr}[2010/04/08]%
  \TMP@RequirePackage{kvsetkeys}[2010/03/01]%
  \TMP@RequirePackage{kvdefinekeys}[2010/03/01]%
  \TMP@RequirePackage{pdftexcmds}[2010/04/01]%
  \TMP@RequirePackage{ifpdf}[2010/01/28]%
  \TMP@RequirePackage{ifluatex}[2010/03/01]%
  \ltx@IfUndefined{newif}{%
    \expandafter\let\csname newif\endcsname\ltx@newif
  }{}%
  \TMP@RequirePackage{ifxetex}[2009/01/23]%
  \TMP@RequirePackage{ifvtex}[2010/03/01]%
\else
  \RequirePackage{ltxcmds}[2011/02/04]%
  \RequirePackage{infwarerr}[2010/04/08]%
  \RequirePackage{kvsetkeys}[2010/03/01]%
  \RequirePackage{kvdefinekeys}[2010/03/01]%
  \RequirePackage{pdftexcmds}[2010/04/01]%
  \RequirePackage{ifpdf}[2010/01/28]%
  \RequirePackage{ifluatex}[2010/03/01]%
  \RequirePackage{ifxetex}[2009/01/23]%
  \RequirePackage{ifvtex}[2010/03/01]%
\fi
%    \end{macrocode}
%
%    \begin{macro}{\HOLOGO@IfDefined}
%    \begin{macrocode}
\def\HOLOGO@IfExists#1{%
  \ifx\@undefined#1%
    \expandafter\ltx@secondoftwo
  \else
    \ifx\relax#1%
      \expandafter\ltx@secondoftwo
    \else
      \expandafter\expandafter\expandafter\ltx@firstoftwo
    \fi
  \fi
}
%    \end{macrocode}
%    \end{macro}
%
% \subsection{Setup macros}
%
%    \begin{macro}{\hologoSetup}
%    \begin{macrocode}
\def\hologoSetup{%
  \let\HOLOGO@name\relax
  \HOLOGO@Setup
}
%    \end{macrocode}
%    \end{macro}
%
%    \begin{macro}{\hologoLogoSetup}
%    \begin{macrocode}
\def\hologoLogoSetup#1{%
  \edef\HOLOGO@name{#1}%
  \ltx@IfUndefined{HoLogo@\HOLOGO@name}{%
    \@PackageError{hologo}{%
      Unknown logo `\HOLOGO@name'%
    }\@ehc
    \ltx@gobble
  }{%
    \HOLOGO@Setup
  }%
}
%    \end{macrocode}
%    \end{macro}
%
%    \begin{macro}{\HOLOGO@Setup}
%    \begin{macrocode}
\def\HOLOGO@Setup{%
  \kvsetkeys{HoLogo}%
}
%    \end{macrocode}
%    \end{macro}
%
% \subsection{Options}
%
%    \begin{macro}{\HOLOGO@DeclareBoolOption}
%    \begin{macrocode}
\def\HOLOGO@DeclareBoolOption#1{%
  \expandafter\chardef\csname HOLOGOOPT@#1\endcsname\ltx@zero
  \kv@define@key{HoLogo}{#1}[true]{%
    \def\HOLOGO@temp{##1}%
    \ifx\HOLOGO@temp\HOLOGO@true
      \ifx\HOLOGO@name\relax
        \expandafter\chardef\csname HOLOGOOPT@#1\endcsname=\ltx@one
      \else
        \expandafter\chardef\csname
        HoLogoOpt@#1@\HOLOGO@name\endcsname\ltx@one
      \fi
      \HOLOGO@SetBreakAll{#1}%
    \else
      \ifx\HOLOGO@temp\HOLOGO@false
        \ifx\HOLOGO@name\relax
          \expandafter\chardef\csname HOLOGOOPT@#1\endcsname=\ltx@zero
        \else
          \expandafter\chardef\csname
          HoLogoOpt@#1@\HOLOGO@name\endcsname=\ltx@zero
        \fi
        \HOLOGO@SetBreakAll{#1}%
      \else
        \@PackageError{hologo}{%
          Unknown value `##1' for boolean option `#1'.\MessageBreak
          Known values are `true' and `false'%
        }\@ehc
      \fi
    \fi
  }%
}
%    \end{macrocode}
%    \end{macro}
%
%    \begin{macro}{\HOLOGO@SetBreakAll}
%    \begin{macrocode}
\def\HOLOGO@SetBreakAll#1{%
  \def\HOLOGO@temp{#1}%
  \ifx\HOLOGO@temp\HOLOGO@break
    \ifx\HOLOGO@name\relax
      \chardef\HOLOGOOPT@hyphenbreak=\HOLOGOOPT@break
      \chardef\HOLOGOOPT@spacebreak=\HOLOGOOPT@break
      \chardef\HOLOGOOPT@discretionarybreak=\HOLOGOOPT@break
    \else
      \expandafter\chardef
         \csname HoLogoOpt@hyphenbreak@\HOLOGO@name\endcsname=%
         \csname HoLogoOpt@break@\HOLOGO@name\endcsname
      \expandafter\chardef
         \csname HoLogoOpt@spacebreak@\HOLOGO@name\endcsname=%
         \csname HoLogoOpt@break@\HOLOGO@name\endcsname
      \expandafter\chardef
         \csname HoLogoOpt@discretionarybreak@\HOLOGO@name
             \endcsname=%
         \csname HoLogoOpt@break@\HOLOGO@name\endcsname
    \fi
  \fi
}
%    \end{macrocode}
%    \end{macro}
%
%    \begin{macro}{\HOLOGO@true}
%    \begin{macrocode}
\def\HOLOGO@true{true}
%    \end{macrocode}
%    \end{macro}
%    \begin{macro}{\HOLOGO@false}
%    \begin{macrocode}
\def\HOLOGO@false{false}
%    \end{macrocode}
%    \end{macro}
%    \begin{macro}{\HOLOGO@break}
%    \begin{macrocode}
\def\HOLOGO@break{break}
%    \end{macrocode}
%    \end{macro}
%
%    \begin{macrocode}
\HOLOGO@DeclareBoolOption{break}
\HOLOGO@DeclareBoolOption{hyphenbreak}
\HOLOGO@DeclareBoolOption{spacebreak}
\HOLOGO@DeclareBoolOption{discretionarybreak}
%    \end{macrocode}
%
%    \begin{macrocode}
\kv@define@key{HoLogo}{variant}{%
  \ifx\HOLOGO@name\relax
    \@PackageError{hologo}{%
      Option `variant' is not available in \string\hologoSetup,%
      \MessageBreak
      Use \string\hologoLogoSetup\space instead%
    }\@ehc
  \else
    \edef\HOLOGO@temp{#1}%
    \ifx\HOLOGO@temp\ltx@empty
      \expandafter
      \let\csname HoLogoOpt@variant@\HOLOGO@name\endcsname\@undefined
    \else
      \ltx@IfUndefined{HoLogo@\HOLOGO@name @\HOLOGO@temp}{%
        \@PackageError{hologo}{%
          Unknown variant `\HOLOGO@temp' of logo `\HOLOGO@name'%
        }\@ehc
      }{%
        \expandafter
        \let\csname HoLogoOpt@variant@\HOLOGO@name\endcsname
            \HOLOGO@temp
      }%
    \fi
  \fi
}
%    \end{macrocode}
%
%    \begin{macro}{\HOLOGO@Variant}
%    \begin{macrocode}
\def\HOLOGO@Variant#1{%
  #1%
  \ltx@ifundefined{HoLogoOpt@variant@#1}{%
  }{%
    @\csname HoLogoOpt@variant@#1\endcsname
  }%
}
%    \end{macrocode}
%    \end{macro}
%
% \subsection{Break/no-break support}
%
%    \begin{macro}{\HOLOGO@space}
%    \begin{macrocode}
\def\HOLOGO@space{%
  \ltx@ifundefined{HoLogoOpt@spacebreak@\HOLOGO@name}{%
    \ltx@ifundefined{HoLogoOpt@break@\HOLOGO@name}{%
      \chardef\HOLOGO@temp=\HOLOGOOPT@spacebreak
    }{%
      \chardef\HOLOGO@temp=%
        \csname HoLogoOpt@break@\HOLOGO@name\endcsname
    }%
  }{%
    \chardef\HOLOGO@temp=%
      \csname HoLogoOpt@spacebreak@\HOLOGO@name\endcsname
  }%
  \ifcase\HOLOGO@temp
    \penalty10000 %
  \fi
  \ltx@space
}
%    \end{macrocode}
%    \end{macro}
%
%    \begin{macro}{\HOLOGO@hyphen}
%    \begin{macrocode}
\def\HOLOGO@hyphen{%
  \ltx@ifundefined{HoLogoOpt@hyphenbreak@\HOLOGO@name}{%
    \ltx@ifundefined{HoLogoOpt@break@\HOLOGO@name}{%
      \chardef\HOLOGO@temp=\HOLOGOOPT@hyphenbreak
    }{%
      \chardef\HOLOGO@temp=%
        \csname HoLogoOpt@break@\HOLOGO@name\endcsname
    }%
  }{%
    \chardef\HOLOGO@temp=%
      \csname HoLogoOpt@hyphenbreak@\HOLOGO@name\endcsname
  }%
  \ifcase\HOLOGO@temp
    \ltx@mbox{-}%
  \else
    -%
  \fi
}
%    \end{macrocode}
%    \end{macro}
%
%    \begin{macro}{\HOLOGO@discretionary}
%    \begin{macrocode}
\def\HOLOGO@discretionary{%
  \ltx@ifundefined{HoLogoOpt@discretionarybreak@\HOLOGO@name}{%
    \ltx@ifundefined{HoLogoOpt@break@\HOLOGO@name}{%
      \chardef\HOLOGO@temp=\HOLOGOOPT@discretionarybreak
    }{%
      \chardef\HOLOGO@temp=%
        \csname HoLogoOpt@break@\HOLOGO@name\endcsname
    }%
  }{%
    \chardef\HOLOGO@temp=%
      \csname HoLogoOpt@discretionarybreak@\HOLOGO@name\endcsname
  }%
  \ifcase\HOLOGO@temp
  \else
    \-%
  \fi
}
%    \end{macrocode}
%    \end{macro}
%
%    \begin{macro}{\HOLOGO@mbox}
%    \begin{macrocode}
\def\HOLOGO@mbox#1{%
  \ltx@ifundefined{HoLogoOpt@break@\HOLOGO@name}{%
    \chardef\HOLOGO@temp=\HOLOGOOPT@hyphenbreak
  }{%
    \chardef\HOLOGO@temp=%
      \csname HoLogoOpt@break@\HOLOGO@name\endcsname
  }%
  \ifcase\HOLOGO@temp
    \ltx@mbox{#1}%
  \else
    #1%
  \fi
}
%    \end{macrocode}
%    \end{macro}
%
% \subsection{Font support}
%
%    \begin{macro}{\HoLogoFont@font}
%    \begin{tabular}{@{}ll@{}}
%    |#1|:& logo name\\
%    |#2|:& font short name\\
%    |#3|:& text
%    \end{tabular}
%    \begin{macrocode}
\def\HoLogoFont@font#1#2#3{%
  \begingroup
    \ltx@IfUndefined{HoLogoFont@logo@#1.#2}{%
      \ltx@IfUndefined{HoLogoFont@font@#2}{%
        \@PackageWarning{hologo}{%
          Missing font `#2' for logo `#1'%
        }%
        #3%
      }{%
        \csname HoLogoFont@font@#2\endcsname{#3}%
      }%
    }{%
      \csname HoLogoFont@logo@#1.#2\endcsname{#3}%
    }%
  \endgroup
}
%    \end{macrocode}
%    \end{macro}
%
%    \begin{macro}{\HoLogoFont@Def}
%    \begin{macrocode}
\def\HoLogoFont@Def#1{%
  \expandafter\def\csname HoLogoFont@font@#1\endcsname
}
%    \end{macrocode}
%    \end{macro}
%    \begin{macro}{\HoLogoFont@LogoDef}
%    \begin{macrocode}
\def\HoLogoFont@LogoDef#1#2{%
  \expandafter\def\csname HoLogoFont@logo@#1.#2\endcsname
}
%    \end{macrocode}
%    \end{macro}
%
% \subsubsection{Font defaults}
%
%    \begin{macro}{\HoLogoFont@font@general}
%    \begin{macrocode}
\HoLogoFont@Def{general}{}%
%    \end{macrocode}
%    \end{macro}
%
%    \begin{macro}{\HoLogoFont@font@rm}
%    \begin{macrocode}
\ltx@IfUndefined{rmfamily}{%
  \ltx@IfUndefined{rm}{%
  }{%
    \HoLogoFont@Def{rm}{\rm}%
  }%
}{%
  \HoLogoFont@Def{rm}{\rmfamily}%
}
%    \end{macrocode}
%    \end{macro}
%
%    \begin{macro}{\HoLogoFont@font@sf}
%    \begin{macrocode}
\ltx@IfUndefined{sffamily}{%
  \ltx@IfUndefined{sf}{%
  }{%
    \HoLogoFont@Def{sf}{\sf}%
  }%
}{%
  \HoLogoFont@Def{sf}{\sffamily}%
}
%    \end{macrocode}
%    \end{macro}
%
%    \begin{macro}{\HoLogoFont@font@bibsf}
%    In case of \hologo{plainTeX} the original small caps
%    variant is used as default. In \hologo{LaTeX}
%    the definition of package \xpackage{dtklogos} \cite{dtklogos}
%    is used.
%\begin{quote}
%\begin{verbatim}
%\DeclareRobustCommand{\BibTeX}{%
%  B%
%  \kern-.05em%
%  \hbox{%
%    $\m@th$% %% force math size calculations
%    \csname S@\f@size\endcsname
%    \fontsize\sf@size\z@
%    \math@fontsfalse
%    \selectfont
%    I%
%    \kern-.025em%
%    B
%  }%
%  \kern-.08em%
%  \-%
%  \TeX
%}
%\end{verbatim}
%\end{quote}
%    \begin{macrocode}
\ltx@IfUndefined{selectfont}{%
  \ltx@IfUndefined{tensc}{%
    \font\tensc=cmcsc10\relax
  }{}%
  \HoLogoFont@Def{bibsf}{\tensc}%
}{%
  \HoLogoFont@Def{bibsf}{%
    $\mathsurround=0pt$%
    \csname S@\f@size\endcsname
    \fontsize\sf@size{0pt}%
    \math@fontsfalse
    \selectfont
  }%
}
%    \end{macrocode}
%    \end{macro}
%
%    \begin{macro}{\HoLogoFont@font@sc}
%    \begin{macrocode}
\ltx@IfUndefined{scshape}{%
  \ltx@IfUndefined{tensc}{%
    \font\tensc=cmcsc10\relax
  }{}%
  \HoLogoFont@Def{sc}{\tensc}%
}{%
  \HoLogoFont@Def{sc}{\scshape}%
}
%    \end{macrocode}
%    \end{macro}
%
%    \begin{macro}{\HoLogoFont@font@sy}
%    \begin{macrocode}
\ltx@IfUndefined{usefont}{%
  \ltx@IfUndefined{tensy}{%
  }{%
    \HoLogoFont@Def{sy}{\tensy}%
  }%
}{%
  \HoLogoFont@Def{sy}{%
    \usefont{OMS}{cmsy}{m}{n}%
  }%
}
%    \end{macrocode}
%    \end{macro}
%
%    \begin{macro}{\HoLogoFont@font@logo}
%    \begin{macrocode}
\begingroup
  \def\x{LaTeX2e}%
\expandafter\endgroup
\ifx\fmtname\x
  \ltx@IfUndefined{logofamily}{%
    \DeclareRobustCommand\logofamily{%
      \not@math@alphabet\logofamily\relax
      \fontencoding{U}%
      \fontfamily{logo}%
      \selectfont
    }%
  }{}%
  \ltx@IfUndefined{logofamily}{%
  }{%
    \HoLogoFont@Def{logo}{\logofamily}%
  }%
\else
  \ltx@IfUndefined{tenlogo}{%
    \font\tenlogo=logo10\relax
  }{}%
  \HoLogoFont@Def{logo}{\tenlogo}%
\fi
%    \end{macrocode}
%    \end{macro}
%
% \subsubsection{Font setup}
%
%    \begin{macro}{\hologoFontSetup}
%    \begin{macrocode}
\def\hologoFontSetup{%
  \let\HOLOGO@name\relax
  \HOLOGO@FontSetup
}
%    \end{macrocode}
%    \end{macro}
%
%    \begin{macro}{\hologoLogoFontSetup}
%    \begin{macrocode}
\def\hologoLogoFontSetup#1{%
  \edef\HOLOGO@name{#1}%
  \ltx@IfUndefined{HoLogo@\HOLOGO@name}{%
    \@PackageError{hologo}{%
      Unknown logo `\HOLOGO@name'%
    }\@ehc
    \ltx@gobble
  }{%
    \HOLOGO@FontSetup
  }%
}
%    \end{macrocode}
%    \end{macro}
%
%    \begin{macro}{\HOLOGO@FontSetup}
%    \begin{macrocode}
\def\HOLOGO@FontSetup{%
  \kvsetkeys{HoLogoFont}%
}
%    \end{macrocode}
%    \end{macro}
%
%    \begin{macrocode}
\def\HOLOGO@temp#1{%
  \kv@define@key{HoLogoFont}{#1}{%
    \ifx\HOLOGO@name\relax
      \HoLogoFont@Def{#1}{##1}%
    \else
      \HoLogoFont@LogoDef\HOLOGO@name{#1}{##1}%
    \fi
  }%
}
\HOLOGO@temp{general}
\HOLOGO@temp{sf}
%    \end{macrocode}
%
% \subsection{Generic logo commands}
%
%    \begin{macrocode}
\HOLOGO@IfExists\hologo{%
  \@PackageError{hologo}{%
    \string\hologo\ltx@space is already defined.\MessageBreak
    Package loading is aborted%
  }\@ehc
  \HOLOGO@AtEnd
}%
\HOLOGO@IfExists\hologoRobust{%
  \@PackageError{hologo}{%
    \string\hologoRobust\ltx@space is already defined.\MessageBreak
    Package loading is aborted%
  }\@ehc
  \HOLOGO@AtEnd
}%
%    \end{macrocode}
%
% \subsubsection{\cs{hologo} and friends}
%
%    \begin{macrocode}
\ifluatex
  \expandafter\ltx@firstofone
\else
  \expandafter\ltx@gobble
\fi
{%
  \ltx@IfUndefined{ifincsname}{%
    \ifnum\luatexversion<36 %
      \expandafter\ltx@gobble
    \else
      \expandafter\ltx@firstofone
    \fi
    {%
      \begingroup
        \ifcase0%
            \directlua{%
              if tex.enableprimitives then %
                tex.enableprimitives('HOLOGO@', {'ifincsname'})%
              else %
                tex.print('1')%
              end%
            }%
            \ifx\HOLOGO@ifincsname\@undefined 1\fi%
            \relax
          \expandafter\ltx@firstofone
        \else
          \endgroup
          \expandafter\ltx@gobble
        \fi
        {%
          \global\let\ifincsname\HOLOGO@ifincsname
        }%
      \HOLOGO@temp
    }%
  }{}%
}
%    \end{macrocode}
%    \begin{macrocode}
\ltx@IfUndefined{ifincsname}{%
  \catcode`$=14 %
}{%
  \catcode`$=9 %
}
%    \end{macrocode}
%
%    \begin{macro}{\hologo}
%    \begin{macrocode}
\def\hologo#1{%
$ \ifincsname
$   \ltx@ifundefined{HoLogoCs@\HOLOGO@Variant{#1}}{%
$     #1%
$   }{%
$     \csname HoLogoCs@\HOLOGO@Variant{#1}\endcsname\ltx@firstoftwo
$   }%
$ \else
    \HOLOGO@IfExists\texorpdfstring\texorpdfstring\ltx@firstoftwo
    {%
      \hologoRobust{#1}%
    }{%
      \ltx@ifundefined{HoLogoBkm@\HOLOGO@Variant{#1}}{%
        \ltx@ifundefined{HoLogo@#1}{?#1?}{#1}%
      }{%
        \csname HoLogoBkm@\HOLOGO@Variant{#1}\endcsname
        \ltx@firstoftwo
      }%
    }%
$ \fi
}
%    \end{macrocode}
%    \end{macro}
%    \begin{macro}{\Hologo}
%    \begin{macrocode}
\def\Hologo#1{%
$ \ifincsname
$   \ltx@ifundefined{HoLogoCs@\HOLOGO@Variant{#1}}{%
$     #1%
$   }{%
$     \csname HoLogoCs@\HOLOGO@Variant{#1}\endcsname\ltx@secondoftwo
$   }%
$ \else
    \HOLOGO@IfExists\texorpdfstring\texorpdfstring\ltx@firstoftwo
    {%
      \HologoRobust{#1}%
    }{%
      \ltx@ifundefined{HoLogoBkm@\HOLOGO@Variant{#1}}{%
        \ltx@ifundefined{HoLogo@#1}{?#1?}{#1}%
      }{%
        \csname HoLogoBkm@\HOLOGO@Variant{#1}\endcsname
        \ltx@secondoftwo
      }%
    }%
$ \fi
}
%    \end{macrocode}
%    \end{macro}
%
%    \begin{macro}{\hologoVariant}
%    \begin{macrocode}
\def\hologoVariant#1#2{%
  \ifx\relax#2\relax
    \hologo{#1}%
  \else
$   \ifincsname
$     \ltx@ifundefined{HoLogoCs@#1@#2}{%
$       #1%
$     }{%
$       \csname HoLogoCs@#1@#2\endcsname\ltx@firstoftwo
$     }%
$   \else
      \HOLOGO@IfExists\texorpdfstring\texorpdfstring\ltx@firstoftwo
      {%
        \hologoVariantRobust{#1}{#2}%
      }{%
        \ltx@ifundefined{HoLogoBkm@#1@#2}{%
          \ltx@ifundefined{HoLogo@#1}{?#1?}{#1}%
        }{%
          \csname HoLogoBkm@#1@#2\endcsname
          \ltx@firstoftwo
        }%
      }%
$   \fi
  \fi
}
%    \end{macrocode}
%    \end{macro}
%    \begin{macro}{\HologoVariant}
%    \begin{macrocode}
\def\HologoVariant#1#2{%
  \ifx\relax#2\relax
    \Hologo{#1}%
  \else
$   \ifincsname
$     \ltx@ifundefined{HoLogoCs@#1@#2}{%
$       #1%
$     }{%
$       \csname HoLogoCs@#1@#2\endcsname\ltx@secondoftwo
$     }%
$   \else
      \HOLOGO@IfExists\texorpdfstring\texorpdfstring\ltx@firstoftwo
      {%
        \HologoVariantRobust{#1}{#2}%
      }{%
        \ltx@ifundefined{HoLogoBkm@#1@#2}{%
          \ltx@ifundefined{HoLogo@#1}{?#1?}{#1}%
        }{%
          \csname HoLogoBkm@#1@#2\endcsname
          \ltx@secondoftwo
        }%
      }%
$   \fi
  \fi
}
%    \end{macrocode}
%    \end{macro}
%
%    \begin{macrocode}
\catcode`\$=3 %
%    \end{macrocode}
%
% \subsubsection{\cs{hologoRobust} and friends}
%
%    \begin{macro}{\hologoRobust}
%    \begin{macrocode}
\ltx@IfUndefined{protected}{%
  \ltx@IfUndefined{DeclareRobustCommand}{%
    \def\hologoRobust#1%
  }{%
    \DeclareRobustCommand*\hologoRobust[1]%
  }%
}{%
  \protected\def\hologoRobust#1%
}%
{%
  \edef\HOLOGO@name{#1}%
  \ltx@IfUndefined{HoLogo@\HOLOGO@Variant\HOLOGO@name}{%
    \@PackageError{hologo}{%
      Unknown logo `\HOLOGO@name'%
    }\@ehc
    ?\HOLOGO@name?%
  }{%
    \ltx@IfUndefined{ver@tex4ht.sty}{%
      \HoLogoFont@font\HOLOGO@name{general}{%
        \csname HoLogo@\HOLOGO@Variant\HOLOGO@name\endcsname
        \ltx@firstoftwo
      }%
    }{%
      \ltx@IfUndefined{HoLogoHtml@\HOLOGO@Variant\HOLOGO@name}{%
        \HOLOGO@name
      }{%
        \csname HoLogoHtml@\HOLOGO@Variant\HOLOGO@name\endcsname
        \ltx@firstoftwo
      }%
    }%
  }%
}
%    \end{macrocode}
%    \end{macro}
%    \begin{macro}{\HologoRobust}
%    \begin{macrocode}
\ltx@IfUndefined{protected}{%
  \ltx@IfUndefined{DeclareRobustCommand}{%
    \def\HologoRobust#1%
  }{%
    \DeclareRobustCommand*\HologoRobust[1]%
  }%
}{%
  \protected\def\HologoRobust#1%
}%
{%
  \edef\HOLOGO@name{#1}%
  \ltx@IfUndefined{HoLogo@\HOLOGO@Variant\HOLOGO@name}{%
    \@PackageError{hologo}{%
      Unknown logo `\HOLOGO@name'%
    }\@ehc
    ?\HOLOGO@name?%
  }{%
    \ltx@IfUndefined{ver@tex4ht.sty}{%
      \HoLogoFont@font\HOLOGO@name{general}{%
        \csname HoLogo@\HOLOGO@Variant\HOLOGO@name\endcsname
        \ltx@secondoftwo
      }%
    }{%
      \ltx@IfUndefined{HoLogoHtml@\HOLOGO@Variant\HOLOGO@name}{%
        \expandafter\HOLOGO@Uppercase\HOLOGO@name
      }{%
        \csname HoLogoHtml@\HOLOGO@Variant\HOLOGO@name\endcsname
        \ltx@secondoftwo
      }%
    }%
  }%
}
%    \end{macrocode}
%    \end{macro}
%    \begin{macro}{\hologoVariantRobust}
%    \begin{macrocode}
\ltx@IfUndefined{protected}{%
  \ltx@IfUndefined{DeclareRobustCommand}{%
    \def\hologoVariantRobust#1#2%
  }{%
    \DeclareRobustCommand*\hologoVariantRobust[2]%
  }%
}{%
  \protected\def\hologoVariantRobust#1#2%
}%
{%
  \begingroup
    \hologoLogoSetup{#1}{variant={#2}}%
    \hologoRobust{#1}%
  \endgroup
}
%    \end{macrocode}
%    \end{macro}
%    \begin{macro}{\HologoVariantRobust}
%    \begin{macrocode}
\ltx@IfUndefined{protected}{%
  \ltx@IfUndefined{DeclareRobustCommand}{%
    \def\HologoVariantRobust#1#2%
  }{%
    \DeclareRobustCommand*\HologoVariantRobust[2]%
  }%
}{%
  \protected\def\HologoVariantRobust#1#2%
}%
{%
  \begingroup
    \hologoLogoSetup{#1}{variant={#2}}%
    \HologoRobust{#1}%
  \endgroup
}
%    \end{macrocode}
%    \end{macro}
%
%    \begin{macro}{\hologorobust}
%    Macro \cs{hologorobust} is only defined for compatibility.
%    Its use is deprecated.
%    \begin{macrocode}
\def\hologorobust{\hologoRobust}
%    \end{macrocode}
%    \end{macro}
%
% \subsection{Helpers}
%
%    \begin{macro}{\HOLOGO@Uppercase}
%    Macro \cs{HOLOGO@Uppercase} is restricted to \cs{uppercase},
%    because \hologo{plainTeX} or \hologo{iniTeX} do not provide
%    \cs{MakeUppercase}.
%    \begin{macrocode}
\def\HOLOGO@Uppercase#1{\uppercase{#1}}
%    \end{macrocode}
%    \end{macro}
%
%    \begin{macro}{\HOLOGO@PdfdocUnicode}
%    \begin{macrocode}
\def\HOLOGO@PdfdocUnicode{%
  \ifx\ifHy@unicode\iftrue
    \expandafter\ltx@secondoftwo
  \else
    \expandafter\ltx@firstoftwo
  \fi
}
%    \end{macrocode}
%    \end{macro}
%
%    \begin{macro}{\HOLOGO@Math}
%    \begin{macrocode}
\def\HOLOGO@MathSetup{%
  \mathsurround0pt\relax
  \HOLOGO@IfExists\f@series{%
    \if b\expandafter\ltx@car\f@series x\@nil
      \csname boldmath\endcsname
   \fi
  }{}%
}
%    \end{macrocode}
%    \end{macro}
%
%    \begin{macro}{\HOLOGO@TempDimen}
%    \begin{macrocode}
\dimendef\HOLOGO@TempDimen=\ltx@zero
%    \end{macrocode}
%    \end{macro}
%    \begin{macro}{\HOLOGO@NegativeKerning}
%    \begin{macrocode}
\def\HOLOGO@NegativeKerning#1{%
  \begingroup
    \HOLOGO@TempDimen=0pt\relax
    \comma@parse@normalized{#1}{%
      \ifdim\HOLOGO@TempDimen=0pt %
        \expandafter\HOLOGO@@NegativeKerning\comma@entry
      \fi
      \ltx@gobble
    }%
    \ifdim\HOLOGO@TempDimen<0pt %
      \kern\HOLOGO@TempDimen
    \fi
  \endgroup
}
%    \end{macrocode}
%    \end{macro}
%    \begin{macro}{\HOLOGO@@NegativeKerning}
%    \begin{macrocode}
\def\HOLOGO@@NegativeKerning#1#2{%
  \setbox\ltx@zero\hbox{#1#2}%
  \HOLOGO@TempDimen=\wd\ltx@zero
  \setbox\ltx@zero\hbox{#1\kern0pt#2}%
  \advance\HOLOGO@TempDimen by -\wd\ltx@zero
}
%    \end{macrocode}
%    \end{macro}
%
%    \begin{macro}{\HOLOGO@SpaceFactor}
%    \begin{macrocode}
\def\HOLOGO@SpaceFactor{%
  \spacefactor1000 %
}
%    \end{macrocode}
%    \end{macro}
%
%    \begin{macro}{\HOLOGO@Span}
%    \begin{macrocode}
\def\HOLOGO@Span#1#2{%
  \HCode{<span class="HoLogo-#1">}%
  #2%
  \HCode{</span>}%
}
%    \end{macrocode}
%    \end{macro}
%
% \subsubsection{Text subscript}
%
%    \begin{macro}{\HOLOGO@SubScript}%
%    \begin{macrocode}
\def\HOLOGO@SubScript#1{%
  \ltx@IfUndefined{textsubscript}{%
    \ltx@IfUndefined{text}{%
      \ltx@mbox{%
        \mathsurround=0pt\relax
        $%
          _{%
            \ltx@IfUndefined{sf@size}{%
              \mathrm{#1}%
            }{%
              \mbox{%
                \fontsize\sf@size{0pt}\selectfont
                #1%
              }%
            }%
          }%
        $%
      }%
    }{%
      \ltx@mbox{%
        \mathsurround=0pt\relax
        $_{\text{#1}}$%
      }%
    }%
  }{%
    \textsubscript{#1}%
  }%
}
%    \end{macrocode}
%    \end{macro}
%
% \subsection{\hologo{TeX} and friends}
%
% \subsubsection{\hologo{TeX}}
%
%    \begin{macro}{\HoLogo@TeX}
%    Source: \hologo{LaTeX} kernel.
%    \begin{macrocode}
\def\HoLogo@TeX#1{%
  T\kern-.1667em\lower.5ex\hbox{E}\kern-.125emX\HOLOGO@SpaceFactor
}
%    \end{macrocode}
%    \end{macro}
%    \begin{macro}{\HoLogoHtml@TeX}
%    \begin{macrocode}
\def\HoLogoHtml@TeX#1{%
  \HoLogoCss@TeX
  \HOLOGO@Span{TeX}{%
    T%
    \HOLOGO@Span{e}{%
      E%
    }%
    X%
  }%
}
%    \end{macrocode}
%    \end{macro}
%    \begin{macro}{\HoLogoCss@TeX}
%    \begin{macrocode}
\def\HoLogoCss@TeX{%
  \Css{%
    span.HoLogo-TeX span.HoLogo-e{%
      position:relative;%
      top:.5ex;%
      margin-left:-.1667em;%
      margin-right:-.125em;%
    }%
  }%
  \Css{%
    a span.HoLogo-TeX span.HoLogo-e{%
      text-decoration:none;%
    }%
  }%
  \global\let\HoLogoCss@TeX\relax
}
%    \end{macrocode}
%    \end{macro}
%
% \subsubsection{\hologo{plainTeX}}
%
%    \begin{macro}{\HoLogo@plainTeX@space}
%    Source: ``The \hologo{TeX}book''
%    \begin{macrocode}
\def\HoLogo@plainTeX@space#1{%
  \HOLOGO@mbox{#1{p}{P}lain}\HOLOGO@space\hologo{TeX}%
}
%    \end{macrocode}
%    \end{macro}
%    \begin{macro}{\HoLogoCs@plainTeX@space}
%    \begin{macrocode}
\def\HoLogoCs@plainTeX@space#1{#1{p}{P}lain TeX}%
%    \end{macrocode}
%    \end{macro}
%    \begin{macro}{\HoLogoBkm@plainTeX@space}
%    \begin{macrocode}
\def\HoLogoBkm@plainTeX@space#1{%
  #1{p}{P}lain \hologo{TeX}%
}
%    \end{macrocode}
%    \end{macro}
%    \begin{macro}{\HoLogoHtml@plainTeX@space}
%    \begin{macrocode}
\def\HoLogoHtml@plainTeX@space#1{%
  #1{p}{P}lain \hologo{TeX}%
}
%    \end{macrocode}
%    \end{macro}
%
%    \begin{macro}{\HoLogo@plainTeX@hyphen}
%    \begin{macrocode}
\def\HoLogo@plainTeX@hyphen#1{%
  \HOLOGO@mbox{#1{p}{P}lain}\HOLOGO@hyphen\hologo{TeX}%
}
%    \end{macrocode}
%    \end{macro}
%    \begin{macro}{\HoLogoCs@plainTeX@hyphen}
%    \begin{macrocode}
\def\HoLogoCs@plainTeX@hyphen#1{#1{p}{P}lain-TeX}
%    \end{macrocode}
%    \end{macro}
%    \begin{macro}{\HoLogoBkm@plainTeX@hyphen}
%    \begin{macrocode}
\def\HoLogoBkm@plainTeX@hyphen#1{%
  #1{p}{P}lain-\hologo{TeX}%
}
%    \end{macrocode}
%    \end{macro}
%    \begin{macro}{\HoLogoHtml@plainTeX@hyphen}
%    \begin{macrocode}
\def\HoLogoHtml@plainTeX@hyphen#1{%
  #1{p}{P}lain-\hologo{TeX}%
}
%    \end{macrocode}
%    \end{macro}
%
%    \begin{macro}{\HoLogo@plainTeX@runtogether}
%    \begin{macrocode}
\def\HoLogo@plainTeX@runtogether#1{%
  \HOLOGO@mbox{#1{p}{P}lain\hologo{TeX}}%
}
%    \end{macrocode}
%    \end{macro}
%    \begin{macro}{\HoLogoCs@plainTeX@runtogether}
%    \begin{macrocode}
\def\HoLogoCs@plainTeX@runtogether#1{#1{p}{P}lainTeX}
%    \end{macrocode}
%    \end{macro}
%    \begin{macro}{\HoLogoBkm@plainTeX@runtogether}
%    \begin{macrocode}
\def\HoLogoBkm@plainTeX@runtogether#1{%
  #1{p}{P}lain\hologo{TeX}%
}
%    \end{macrocode}
%    \end{macro}
%    \begin{macro}{\HoLogoHtml@plainTeX@runtogether}
%    \begin{macrocode}
\def\HoLogoHtml@plainTeX@runtogether#1{%
  #1{p}{P}lain\hologo{TeX}%
}
%    \end{macrocode}
%    \end{macro}
%
%    \begin{macro}{\HoLogo@plainTeX}
%    \begin{macrocode}
\def\HoLogo@plainTeX{\HoLogo@plainTeX@space}
%    \end{macrocode}
%    \end{macro}
%    \begin{macro}{\HoLogoCs@plainTeX}
%    \begin{macrocode}
\def\HoLogoCs@plainTeX{\HoLogoCs@plainTeX@space}
%    \end{macrocode}
%    \end{macro}
%    \begin{macro}{\HoLogoBkm@plainTeX}
%    \begin{macrocode}
\def\HoLogoBkm@plainTeX{\HoLogoBkm@plainTeX@space}
%    \end{macrocode}
%    \end{macro}
%    \begin{macro}{\HoLogoHtml@plainTeX}
%    \begin{macrocode}
\def\HoLogoHtml@plainTeX{\HoLogoHtml@plainTeX@space}
%    \end{macrocode}
%    \end{macro}
%
% \subsubsection{\hologo{LaTeX}}
%
%    Source: \hologo{LaTeX} kernel.
%\begin{quote}
%\begin{verbatim}
%\DeclareRobustCommand{\LaTeX}{%
%  L%
%  \kern-.36em%
%  {%
%    \sbox\z@ T%
%    \vbox to\ht\z@{%
%      \hbox{%
%        \check@mathfonts
%        \fontsize\sf@size\z@
%        \math@fontsfalse
%        \selectfont
%        A%
%      }%
%      \vss
%    }%
%  }%
%  \kern-.15em%
%  \TeX
%}
%\end{verbatim}
%\end{quote}
%
%    \begin{macro}{\HoLogo@La}
%    \begin{macrocode}
\def\HoLogo@La#1{%
  L%
  \kern-.36em%
  \begingroup
    \setbox\ltx@zero\hbox{T}%
    \vbox to\ht\ltx@zero{%
      \hbox{%
        \ltx@ifundefined{check@mathfonts}{%
          \csname sevenrm\endcsname
        }{%
          \check@mathfonts
          \fontsize\sf@size{0pt}%
          \math@fontsfalse\selectfont
        }%
        A%
      }%
      \vss
    }%
  \endgroup
}
%    \end{macrocode}
%    \end{macro}
%
%    \begin{macro}{\HoLogo@LaTeX}
%    Source: \hologo{LaTeX} kernel.
%    \begin{macrocode}
\def\HoLogo@LaTeX#1{%
  \hologo{La}%
  \kern-.15em%
  \hologo{TeX}%
}
%    \end{macrocode}
%    \end{macro}
%    \begin{macro}{\HoLogoHtml@LaTeX}
%    \begin{macrocode}
\def\HoLogoHtml@LaTeX#1{%
  \HoLogoCss@LaTeX
  \HOLOGO@Span{LaTeX}{%
    L%
    \HOLOGO@Span{a}{%
      A%
    }%
    \hologo{TeX}%
  }%
}
%    \end{macrocode}
%    \end{macro}
%    \begin{macro}{\HoLogoCss@LaTeX}
%    \begin{macrocode}
\def\HoLogoCss@LaTeX{%
  \Css{%
    span.HoLogo-LaTeX span.HoLogo-a{%
      position:relative;%
      top:-.5ex;%
      margin-left:-.36em;%
      margin-right:-.15em;%
      font-size:85\%;%
    }%
  }%
  \global\let\HoLogoCss@LaTeX\relax
}
%    \end{macrocode}
%    \end{macro}
%
% \subsubsection{\hologo{(La)TeX}}
%
%    \begin{macro}{\HoLogo@LaTeXTeX}
%    The kerning around the parentheses is taken
%    from package \xpackage{dtklogos} \cite{dtklogos}.
%\begin{quote}
%\begin{verbatim}
%\DeclareRobustCommand{\LaTeXTeX}{%
%  (%
%  \kern-.15em%
%  L%
%  \kern-.36em%
%  {%
%    \sbox\z@ T%
%    \vbox to\ht0{%
%      \hbox{%
%        $\m@th$%
%        \csname S@\f@size\endcsname
%        \fontsize\sf@size\z@
%        \math@fontsfalse
%        \selectfont
%        A%
%      }%
%      \vss
%    }%
%  }%
%  \kern-.2em%
%  )%
%  \kern-.15em%
%  \TeX
%}
%\end{verbatim}
%\end{quote}
%    \begin{macrocode}
\def\HoLogo@LaTeXTeX#1{%
  (%
  \kern-.15em%
  \hologo{La}%
  \kern-.2em%
  )%
  \kern-.15em%
  \hologo{TeX}%
}
%    \end{macrocode}
%    \end{macro}
%    \begin{macro}{\HoLogoBkm@LaTeXTeX}
%    \begin{macrocode}
\def\HoLogoBkm@LaTeXTeX#1{(La)TeX}
%    \end{macrocode}
%    \end{macro}
%
%    \begin{macro}{\HoLogo@(La)TeX}
%    \begin{macrocode}
\expandafter
\let\csname HoLogo@(La)TeX\endcsname\HoLogo@LaTeXTeX
%    \end{macrocode}
%    \end{macro}
%    \begin{macro}{\HoLogoBkm@(La)TeX}
%    \begin{macrocode}
\expandafter
\let\csname HoLogoBkm@(La)TeX\endcsname\HoLogoBkm@LaTeXTeX
%    \end{macrocode}
%    \end{macro}
%    \begin{macro}{\HoLogoHtml@LaTeXTeX}
%    \begin{macrocode}
\def\HoLogoHtml@LaTeXTeX#1{%
  \HoLogoCss@LaTeXTeX
  \HOLOGO@Span{LaTeXTeX}{%
    (%
    \HOLOGO@Span{L}{L}%
    \HOLOGO@Span{a}{A}%
    \HOLOGO@Span{ParenRight}{)}%
    \hologo{TeX}%
  }%
}
%    \end{macrocode}
%    \end{macro}
%    \begin{macro}{\HoLogoHtml@(La)TeX}
%    Kerning after opening parentheses and before closing parentheses
%    is $-0.1$\,em. The original values $-0.15$\,em
%    looked too ugly for a serif font.
%    \begin{macrocode}
\expandafter
\let\csname HoLogoHtml@(La)TeX\endcsname\HoLogoHtml@LaTeXTeX
%    \end{macrocode}
%    \end{macro}
%    \begin{macro}{\HoLogoCss@LaTeXTeX}
%    \begin{macrocode}
\def\HoLogoCss@LaTeXTeX{%
  \Css{%
    span.HoLogo-LaTeXTeX span.HoLogo-L{%
      margin-left:-.1em;%
    }%
  }%
  \Css{%
    span.HoLogo-LaTeXTeX span.HoLogo-a{%
      position:relative;%
      top:-.5ex;%
      margin-left:-.36em;%
      margin-right:-.1em;%
      font-size:85\%;%
    }%
  }%
  \Css{%
    span.HoLogo-LaTeXTeX span.HoLogo-ParenRight{%
      margin-right:-.15em;%
    }%
  }%
  \global\let\HoLogoCss@LaTeXTeX\relax
}
%    \end{macrocode}
%    \end{macro}
%
% \subsubsection{\hologo{LaTeXe}}
%
%    \begin{macro}{\HoLogo@LaTeXe}
%    Source: \hologo{LaTeX} kernel
%    \begin{macrocode}
\def\HoLogo@LaTeXe#1{%
  \hologo{LaTeX}%
  \kern.15em%
  \hbox{%
    \HOLOGO@MathSetup
    2%
    $_{\textstyle\varepsilon}$%
  }%
}
%    \end{macrocode}
%    \end{macro}
%
%    \begin{macro}{\HoLogoCs@LaTeXe}
%    \begin{macrocode}
\ifnum64=`\^^^^0040\relax % test for big chars of LuaTeX/XeTeX
  \catcode`\$=9 %
  \catcode`\&=14 %
\else
  \catcode`\$=14 %
  \catcode`\&=9 %
\fi
\def\HoLogoCs@LaTeXe#1{%
  LaTeX2%
$ \string ^^^^0395%
& e%
}%
\catcode`\$=3 %
\catcode`\&=4 %
%    \end{macrocode}
%    \end{macro}
%
%    \begin{macro}{\HoLogoBkm@LaTeXe}
%    \begin{macrocode}
\def\HoLogoBkm@LaTeXe#1{%
  \hologo{LaTeX}%
  2%
  \HOLOGO@PdfdocUnicode{e}{\textepsilon}%
}
%    \end{macrocode}
%    \end{macro}
%
%    \begin{macro}{\HoLogoHtml@LaTeXe}
%    \begin{macrocode}
\def\HoLogoHtml@LaTeXe#1{%
  \HoLogoCss@LaTeXe
  \HOLOGO@Span{LaTeX2e}{%
    \hologo{LaTeX}%
    \HOLOGO@Span{2}{2}%
    \HOLOGO@Span{e}{%
      \HOLOGO@MathSetup
      \ensuremath{\textstyle\varepsilon}%
    }%
  }%
}
%    \end{macrocode}
%    \end{macro}
%    \begin{macro}{\HoLogoCss@LaTeXe}
%    \begin{macrocode}
\def\HoLogoCss@LaTeXe{%
  \Css{%
    span.HoLogo-LaTeX2e span.HoLogo-2{%
      padding-left:.15em;%
    }%
  }%
  \Css{%
    span.HoLogo-LaTeX2e span.HoLogo-e{%
      position:relative;%
      top:.35ex;%
      text-decoration:none;%
    }%
  }%
  \global\let\HoLogoCss@LaTeXe\relax
}
%    \end{macrocode}
%    \end{macro}
%
%    \begin{macro}{\HoLogo@LaTeX2e}
%    \begin{macrocode}
\expandafter
\let\csname HoLogo@LaTeX2e\endcsname\HoLogo@LaTeXe
%    \end{macrocode}
%    \end{macro}
%    \begin{macro}{\HoLogoCs@LaTeX2e}
%    \begin{macrocode}
\expandafter
\let\csname HoLogoCs@LaTeX2e\endcsname\HoLogoCs@LaTeXe
%    \end{macrocode}
%    \end{macro}
%    \begin{macro}{\HoLogoBkm@LaTeX2e}
%    \begin{macrocode}
\expandafter
\let\csname HoLogoBkm@LaTeX2e\endcsname\HoLogoBkm@LaTeXe
%    \end{macrocode}
%    \end{macro}
%    \begin{macro}{\HoLogoHtml@LaTeX2e}
%    \begin{macrocode}
\expandafter
\let\csname HoLogoHtml@LaTeX2e\endcsname\HoLogoHtml@LaTeXe
%    \end{macrocode}
%    \end{macro}
%
% \subsubsection{\hologo{LaTeX3}}
%
%    \begin{macro}{\HoLogo@LaTeX3}
%    Source: \hologo{LaTeX} kernel
%    \begin{macrocode}
\expandafter\def\csname HoLogo@LaTeX3\endcsname#1{%
  \hologo{LaTeX}%
  3%
}
%    \end{macrocode}
%    \end{macro}
%
%    \begin{macro}{\HoLogoBkm@LaTeX3}
%    \begin{macrocode}
\expandafter\def\csname HoLogoBkm@LaTeX3\endcsname#1{%
  \hologo{LaTeX}%
  3%
}
%    \end{macrocode}
%    \end{macro}
%    \begin{macro}{\HoLogoHtml@LaTeX3}
%    \begin{macrocode}
\expandafter
\let\csname HoLogoHtml@LaTeX3\expandafter\endcsname
\csname HoLogo@LaTeX3\endcsname
%    \end{macrocode}
%    \end{macro}
%
% \subsubsection{\hologo{LaTeXML}}
%
%    \begin{macro}{\HoLogo@LaTeXML}
%    \begin{macrocode}
\def\HoLogo@LaTeXML#1{%
  \HOLOGO@mbox{%
    \hologo{La}%
    \kern-.15em%
    T%
    \kern-.1667em%
    \lower.5ex\hbox{E}%
    \kern-.125em%
    \HoLogoFont@font{LaTeXML}{sc}{xml}%
  }%
}
%    \end{macrocode}
%    \end{macro}
%    \begin{macro}{\HoLogoHtml@pdfLaTeX}
%    \begin{macrocode}
\def\HoLogoHtml@LaTeXML#1{%
  \HOLOGO@Span{LaTeXML}{%
    \HoLogoCss@LaTeX
    \HoLogoCss@TeX
    \HOLOGO@Span{LaTeX}{%
      L%
      \HOLOGO@Span{a}{%
        A%
      }%
    }%
    \HOLOGO@Span{TeX}{%
      T%
      \HOLOGO@Span{e}{%
        E%
      }%
    }%
    \HCode{<span style="font-variant: small-caps;">}%
    xml%
    \HCode{</span>}%
  }%
}
%    \end{macrocode}
%    \end{macro}
%
% \subsubsection{\hologo{eTeX}}
%
%    \begin{macro}{\HoLogo@eTeX}
%    Source: package \xpackage{etex}
%    \begin{macrocode}
\def\HoLogo@eTeX#1{%
  \ltx@mbox{%
    \HOLOGO@MathSetup
    $\varepsilon$%
    -%
    \HOLOGO@NegativeKerning{-T,T-,To}%
    \hologo{TeX}%
  }%
}
%    \end{macrocode}
%    \end{macro}
%    \begin{macro}{\HoLogoCs@eTeX}
%    \begin{macrocode}
\ifnum64=`\^^^^0040\relax % test for big chars of LuaTeX/XeTeX
  \catcode`\$=9 %
  \catcode`\&=14 %
\else
  \catcode`\$=14 %
  \catcode`\&=9 %
\fi
\def\HoLogoCs@eTeX#1{%
$ #1{\string ^^^^0395}{\string ^^^^03b5}%
& #1{e}{E}%
  TeX%
}%
\catcode`\$=3 %
\catcode`\&=4 %
%    \end{macrocode}
%    \end{macro}
%    \begin{macro}{\HoLogoBkm@eTeX}
%    \begin{macrocode}
\def\HoLogoBkm@eTeX#1{%
  \HOLOGO@PdfdocUnicode{#1{e}{E}}{\textepsilon}%
  -%
  \hologo{TeX}%
}
%    \end{macrocode}
%    \end{macro}
%    \begin{macro}{\HoLogoHtml@eTeX}
%    \begin{macrocode}
\def\HoLogoHtml@eTeX#1{%
  \ltx@mbox{%
    \HOLOGO@MathSetup
    $\varepsilon$%
    -%
    \hologo{TeX}%
  }%
}
%    \end{macrocode}
%    \end{macro}
%
% \subsubsection{\hologo{iniTeX}}
%
%    \begin{macro}{\HoLogo@iniTeX}
%    \begin{macrocode}
\def\HoLogo@iniTeX#1{%
  \HOLOGO@mbox{%
    #1{i}{I}ni\hologo{TeX}%
  }%
}
%    \end{macrocode}
%    \end{macro}
%    \begin{macro}{\HoLogoCs@iniTeX}
%    \begin{macrocode}
\def\HoLogoCs@iniTeX#1{#1{i}{I}niTeX}
%    \end{macrocode}
%    \end{macro}
%    \begin{macro}{\HoLogoBkm@iniTeX}
%    \begin{macrocode}
\def\HoLogoBkm@iniTeX#1{%
  #1{i}{I}ni\hologo{TeX}%
}
%    \end{macrocode}
%    \end{macro}
%    \begin{macro}{\HoLogoHtml@iniTeX}
%    \begin{macrocode}
\let\HoLogoHtml@iniTeX\HoLogo@iniTeX
%    \end{macrocode}
%    \end{macro}
%
% \subsubsection{\hologo{virTeX}}
%
%    \begin{macro}{\HoLogo@virTeX}
%    \begin{macrocode}
\def\HoLogo@virTeX#1{%
  \HOLOGO@mbox{%
    #1{v}{V}ir\hologo{TeX}%
  }%
}
%    \end{macrocode}
%    \end{macro}
%    \begin{macro}{\HoLogoCs@virTeX}
%    \begin{macrocode}
\def\HoLogoCs@virTeX#1{#1{v}{V}irTeX}
%    \end{macrocode}
%    \end{macro}
%    \begin{macro}{\HoLogoBkm@virTeX}
%    \begin{macrocode}
\def\HoLogoBkm@virTeX#1{%
  #1{v}{V}ir\hologo{TeX}%
}
%    \end{macrocode}
%    \end{macro}
%    \begin{macro}{\HoLogoHtml@virTeX}
%    \begin{macrocode}
\let\HoLogoHtml@virTeX\HoLogo@virTeX
%    \end{macrocode}
%    \end{macro}
%
% \subsubsection{\hologo{SliTeX}}
%
% \paragraph{Definitions of the three variants.}
%
%    \begin{macro}{\HoLogo@SLiTeX@lift}
%    \begin{macrocode}
\def\HoLogo@SLiTeX@lift#1{%
  \HoLogoFont@font{SliTeX}{rm}{%
    S%
    \kern-.06em%
    L%
    \kern-.18em%
    \raise.32ex\hbox{\HoLogoFont@font{SliTeX}{sc}{i}}%
    \HOLOGO@discretionary
    \kern-.06em%
    \hologo{TeX}%
  }%
}
%    \end{macrocode}
%    \end{macro}
%    \begin{macro}{\HoLogoBkm@SLiTeX@lift}
%    \begin{macrocode}
\def\HoLogoBkm@SLiTeX@lift#1{SLiTeX}
%    \end{macrocode}
%    \end{macro}
%    \begin{macro}{\HoLogoHtml@SLiTeX@lift}
%    \begin{macrocode}
\def\HoLogoHtml@SLiTeX@lift#1{%
  \HoLogoCss@SLiTeX@lift
  \HOLOGO@Span{SLiTeX-lift}{%
    \HoLogoFont@font{SliTeX}{rm}{%
      S%
      \HOLOGO@Span{L}{L}%
      \HOLOGO@Span{i}{i}%
      \hologo{TeX}%
    }%
  }%
}
%    \end{macrocode}
%    \end{macro}
%    \begin{macro}{\HoLogoCss@SLiTeX@lift}
%    \begin{macrocode}
\def\HoLogoCss@SLiTeX@lift{%
  \Css{%
    span.HoLogo-SLiTeX-lift span.HoLogo-L{%
      margin-left:-.06em;%
      margin-right:-.18em;%
    }%
  }%
  \Css{%
    span.HoLogo-SLiTeX-lift span.HoLogo-i{%
      position:relative;%
      top:-.32ex;%
      margin-right:-.06em;%
      font-variant:small-caps;%
    }%
  }%
  \global\let\HoLogoCss@SLiTeX@lift\relax
}
%    \end{macrocode}
%    \end{macro}
%
%    \begin{macro}{\HoLogo@SliTeX@simple}
%    \begin{macrocode}
\def\HoLogo@SliTeX@simple#1{%
  \HoLogoFont@font{SliTeX}{rm}{%
    \ltx@mbox{%
      \HoLogoFont@font{SliTeX}{sc}{Sli}%
    }%
    \HOLOGO@discretionary
    \hologo{TeX}%
  }%
}
%    \end{macrocode}
%    \end{macro}
%    \begin{macro}{\HoLogoBkm@SliTeX@simple}
%    \begin{macrocode}
\def\HoLogoBkm@SliTeX@simple#1{SliTeX}
%    \end{macrocode}
%    \end{macro}
%    \begin{macro}{\HoLogoHtml@SliTeX@simple}
%    \begin{macrocode}
\let\HoLogoHtml@SliTeX@simple\HoLogo@SliTeX@simple
%    \end{macrocode}
%    \end{macro}
%
%    \begin{macro}{\HoLogo@SliTeX@narrow}
%    \begin{macrocode}
\def\HoLogo@SliTeX@narrow#1{%
  \HoLogoFont@font{SliTeX}{rm}{%
    \ltx@mbox{%
      S%
      \kern-.06em%
      \HoLogoFont@font{SliTeX}{sc}{%
        l%
        \kern-.035em%
        i%
      }%
    }%
    \HOLOGO@discretionary
    \kern-.06em%
    \hologo{TeX}%
  }%
}
%    \end{macrocode}
%    \end{macro}
%    \begin{macro}{\HoLogoBkm@SliTeX@narrow}
%    \begin{macrocode}
\def\HoLogoBkm@SliTeX@narrow#1{SliTeX}
%    \end{macrocode}
%    \end{macro}
%    \begin{macro}{\HoLogoHtml@SliTeX@narrow}
%    \begin{macrocode}
\def\HoLogoHtml@SliTeX@narrow#1{%
  \HoLogoCss@SliTeX@narrow
  \HOLOGO@Span{SliTeX-narrow}{%
    \HoLogoFont@font{SliTeX}{rm}{%
      S%
        \HOLOGO@Span{l}{l}%
        \HOLOGO@Span{i}{i}%
      \hologo{TeX}%
    }%
  }%
}
%    \end{macrocode}
%    \end{macro}
%    \begin{macro}{\HoLogoCss@SliTeX@narrow}
%    \begin{macrocode}
\def\HoLogoCss@SliTeX@narrow{%
  \Css{%
    span.HoLogo-SliTeX-narrow span.HoLogo-l{%
      margin-left:-.06em;%
      margin-right:-.035em;%
      font-variant:small-caps;%
    }%
  }%
  \Css{%
    span.HoLogo-SliTeX-narrow span.HoLogo-i{%
      margin-right:-.06em;%
      font-variant:small-caps;%
    }%
  }%
  \global\let\HoLogoCss@SliTeX@narrow\relax
}
%    \end{macrocode}
%    \end{macro}
%
% \paragraph{Macro set completion.}
%
%    \begin{macro}{\HoLogo@SLiTeX@simple}
%    \begin{macrocode}
\def\HoLogo@SLiTeX@simple{\HoLogo@SliTeX@simple}
%    \end{macrocode}
%    \end{macro}
%    \begin{macro}{\HoLogoBkm@SLiTeX@simple}
%    \begin{macrocode}
\def\HoLogoBkm@SLiTeX@simple{\HoLogoBkm@SliTeX@simple}
%    \end{macrocode}
%    \end{macro}
%    \begin{macro}{\HoLogoHtml@SLiTeX@simple}
%    \begin{macrocode}
\def\HoLogoHtml@SLiTeX@simple{\HoLogoHtml@SliTeX@simple}
%    \end{macrocode}
%    \end{macro}
%
%    \begin{macro}{\HoLogo@SLiTeX@narrow}
%    \begin{macrocode}
\def\HoLogo@SLiTeX@narrow{\HoLogo@SliTeX@narrow}
%    \end{macrocode}
%    \end{macro}
%    \begin{macro}{\HoLogoBkm@SLiTeX@narrow}
%    \begin{macrocode}
\def\HoLogoBkm@SLiTeX@narrow{\HoLogoBkm@SliTeX@narrow}
%    \end{macrocode}
%    \end{macro}
%    \begin{macro}{\HoLogoHtml@SLiTeX@narrow}
%    \begin{macrocode}
\def\HoLogoHtml@SLiTeX@narrow{\HoLogoHtml@SliTeX@narrow}
%    \end{macrocode}
%    \end{macro}
%
%    \begin{macro}{\HoLogo@SliTeX@lift}
%    \begin{macrocode}
\def\HoLogo@SliTeX@lift{\HoLogo@SLiTeX@lift}
%    \end{macrocode}
%    \end{macro}
%    \begin{macro}{\HoLogoBkm@SliTeX@lift}
%    \begin{macrocode}
\def\HoLogoBkm@SliTeX@lift{\HoLogoBkm@SLiTeX@lift}
%    \end{macrocode}
%    \end{macro}
%    \begin{macro}{\HoLogoHtml@SliTeX@lift}
%    \begin{macrocode}
\def\HoLogoHtml@SliTeX@lift{\HoLogoHtml@SLiTeX@lift}
%    \end{macrocode}
%    \end{macro}
%
% \paragraph{Defaults.}
%
%    \begin{macro}{\HoLogo@SLiTeX}
%    \begin{macrocode}
\def\HoLogo@SLiTeX{\HoLogo@SLiTeX@lift}
%    \end{macrocode}
%    \end{macro}
%    \begin{macro}{\HoLogoBkm@SLiTeX}
%    \begin{macrocode}
\def\HoLogoBkm@SLiTeX{\HoLogoBkm@SLiTeX@lift}
%    \end{macrocode}
%    \end{macro}
%    \begin{macro}{\HoLogoHtml@SLiTeX}
%    \begin{macrocode}
\def\HoLogoHtml@SLiTeX{\HoLogoHtml@SLiTeX@lift}
%    \end{macrocode}
%    \end{macro}
%
%    \begin{macro}{\HoLogo@SliTeX}
%    \begin{macrocode}
\def\HoLogo@SliTeX{\HoLogo@SliTeX@narrow}
%    \end{macrocode}
%    \end{macro}
%    \begin{macro}{\HoLogoBkm@SliTeX}
%    \begin{macrocode}
\def\HoLogoBkm@SliTeX{\HoLogoBkm@SliTeX@narrow}
%    \end{macrocode}
%    \end{macro}
%    \begin{macro}{\HoLogoHtml@SliTeX}
%    \begin{macrocode}
\def\HoLogoHtml@SliTeX{\HoLogoHtml@SliTeX@narrow}
%    \end{macrocode}
%    \end{macro}
%
% \subsubsection{\hologo{LuaTeX}}
%
%    \begin{macro}{\HoLogo@LuaTeX}
%    The kerning is an idea of Hans Hagen, see mailing list
%    `luatex at tug dot org' in March 2010.
%    \begin{macrocode}
\def\HoLogo@LuaTeX#1{%
  \HOLOGO@mbox{%
    Lua%
    \HOLOGO@NegativeKerning{aT,oT,To}%
    \hologo{TeX}%
  }%
}
%    \end{macrocode}
%    \end{macro}
%    \begin{macro}{\HoLogoHtml@LuaTeX}
%    \begin{macrocode}
\let\HoLogoHtml@LuaTeX\HoLogo@LuaTeX
%    \end{macrocode}
%    \end{macro}
%
% \subsubsection{\hologo{LuaLaTeX}}
%
%    \begin{macro}{\HoLogo@LuaLaTeX}
%    \begin{macrocode}
\def\HoLogo@LuaLaTeX#1{%
  \HOLOGO@mbox{%
    Lua%
    \hologo{LaTeX}%
  }%
}
%    \end{macrocode}
%    \end{macro}
%    \begin{macro}{\HoLogoHtml@LuaLaTeX}
%    \begin{macrocode}
\let\HoLogoHtml@LuaLaTeX\HoLogo@LuaLaTeX
%    \end{macrocode}
%    \end{macro}
%
% \subsubsection{\hologo{XeTeX}, \hologo{XeLaTeX}}
%
%    \begin{macro}{\HOLOGO@IfCharExists}
%    \begin{macrocode}
\ifluatex
  \ifnum\luatexversion<36 %
  \else
    \def\HOLOGO@IfCharExists#1{%
      \ifnum
        \directlua{%
           if luaotfload and luaotfload.aux then
             if luaotfload.aux.font_has_glyph(%
                    font.current(), \number#1) then % 	 
	       tex.print("1") % 	 
	     end % 	 
	   elseif font and font.fonts and font.current then %
            local f = font.fonts[font.current()]%
            if f.characters and f.characters[\number#1] then %
              tex.print("1")%
            end %
          end%
        }0=\ltx@zero
        \expandafter\ltx@secondoftwo
      \else
        \expandafter\ltx@firstoftwo
      \fi
    }%
  \fi
\fi
\ltx@IfUndefined{HOLOGO@IfCharExists}{%
  \def\HOLOGO@@IfCharExists#1{%
    \begingroup
      \tracinglostchars=\ltx@zero
      \setbox\ltx@zero=\hbox{%
        \kern7sp\char#1\relax
        \ifnum\lastkern>\ltx@zero
          \expandafter\aftergroup\csname iffalse\endcsname
        \else
          \expandafter\aftergroup\csname iftrue\endcsname
        \fi
      }%
      % \if{true|false} from \aftergroup
      \endgroup
      \expandafter\ltx@firstoftwo
    \else
      \endgroup
      \expandafter\ltx@secondoftwo
    \fi
  }%
  \ifxetex
    \ltx@IfUndefined{XeTeXfonttype}{}{%
      \ltx@IfUndefined{XeTeXcharglyph}{}{%
        \def\HOLOGO@IfCharExists#1{%
          \ifnum\XeTeXfonttype\font>\ltx@zero
            \expandafter\ltx@firstofthree
          \else
            \expandafter\ltx@gobble
          \fi
          {%
            \ifnum\XeTeXcharglyph#1>\ltx@zero
              \expandafter\ltx@firstoftwo
            \else
              \expandafter\ltx@secondoftwo
            \fi
          }%
          \HOLOGO@@IfCharExists{#1}%
        }%
      }%
    }%
  \fi
}{}
\ltx@ifundefined{HOLOGO@IfCharExists}{%
  \ifnum64=`\^^^^0040\relax % test for big chars of LuaTeX/XeTeX
    \let\HOLOGO@IfCharExists\HOLOGO@@IfCharExists
  \else
    \def\HOLOGO@IfCharExists#1{%
      \ifnum#1>255 %
        \expandafter\ltx@fourthoffour
      \fi
      \HOLOGO@@IfCharExists{#1}%
    }%
  \fi
}{}
%    \end{macrocode}
%    \end{macro}
%
%    \begin{macro}{\HoLogo@Xe}
%    Source: package \xpackage{dtklogos}
%    \begin{macrocode}
\def\HoLogo@Xe#1{%
  X%
  \kern-.1em\relax
  \HOLOGO@IfCharExists{"018E}{%
    \lower.5ex\hbox{\char"018E}%
  }{%
    \chardef\HOLOGO@choice=\ltx@zero
    \ifdim\fontdimen\ltx@one\font>0pt %
      \ltx@IfUndefined{rotatebox}{%
        \ltx@IfUndefined{pgftext}{%
          \ltx@IfUndefined{psscalebox}{%
            \ltx@IfUndefined{HOLOGO@ScaleBox@\hologoDriver}{%
            }{%
              \chardef\HOLOGO@choice=4 %
            }%
          }{%
            \chardef\HOLOGO@choice=3 %
          }%
        }{%
          \chardef\HOLOGO@choice=2 %
        }%
      }{%
        \chardef\HOLOGO@choice=1 %
      }%
      \ifcase\HOLOGO@choice
        \HOLOGO@WarningUnsupportedDriver{Xe}%
        e%
      \or % 1: \rotatebox
        \begingroup
          \setbox\ltx@zero\hbox{\rotatebox{180}{E}}%
          \ltx@LocDimenA=\dp\ltx@zero
          \advance\ltx@LocDimenA by -.5ex\relax
          \raise\ltx@LocDimenA\box\ltx@zero
        \endgroup
      \or % 2: \pgftext
        \lower.5ex\hbox{%
          \pgfpicture
            \pgftext[rotate=180]{E}%
          \endpgfpicture
        }%
      \or % 3: \psscalebox
        \begingroup
          \setbox\ltx@zero\hbox{\psscalebox{-1 -1}{E}}%
          \ltx@LocDimenA=\dp\ltx@zero
          \advance\ltx@LocDimenA by -.5ex\relax
          \raise\ltx@LocDimenA\box\ltx@zero
        \endgroup
      \or % 4: \HOLOGO@PointReflectBox
        \lower.5ex\hbox{\HOLOGO@PointReflectBox{E}}%
      \else
        \@PackageError{hologo}{Internal error (choice/it}\@ehc
      \fi
    \else
      \ltx@IfUndefined{reflectbox}{%
        \ltx@IfUndefined{pgftext}{%
          \ltx@IfUndefined{psscalebox}{%
            \ltx@IfUndefined{HOLOGO@ScaleBox@\hologoDriver}{%
            }{%
              \chardef\HOLOGO@choice=4 %
            }%
          }{%
            \chardef\HOLOGO@choice=3 %
          }%
        }{%
          \chardef\HOLOGO@choice=2 %
        }%
      }{%
        \chardef\HOLOGO@choice=1 %
      }%
      \ifcase\HOLOGO@choice
        \HOLOGO@WarningUnsupportedDriver{Xe}%
        e%
      \or % 1: reflectbox
        \lower.5ex\hbox{%
          \reflectbox{E}%
        }%
      \or % 2: \pgftext
        \lower.5ex\hbox{%
          \pgfpicture
            \pgftransformxscale{-1}%
            \pgftext{E}%
          \endpgfpicture
        }%
      \or % 3: \psscalebox
        \lower.5ex\hbox{%
          \psscalebox{-1 1}{E}%
        }%
      \or % 4: \HOLOGO@Reflectbox
        \lower.5ex\hbox{%
          \HOLOGO@ReflectBox{E}%
        }%
      \else
        \@PackageError{hologo}{Internal error (choice/up)}\@ehc
      \fi
    \fi
  }%
}
%    \end{macrocode}
%    \end{macro}
%    \begin{macro}{\HoLogoHtml@Xe}
%    \begin{macrocode}
\def\HoLogoHtml@Xe#1{%
  \HoLogoCss@Xe
  \HOLOGO@Span{Xe}{%
    X%
    \HOLOGO@Span{e}{%
      \HCode{&\ltx@hashchar x018e;}%
    }%
  }%
}
%    \end{macrocode}
%    \end{macro}
%    \begin{macro}{\HoLogoCss@Xe}
%    \begin{macrocode}
\def\HoLogoCss@Xe{%
  \Css{%
    span.HoLogo-Xe span.HoLogo-e{%
      position:relative;%
      top:.5ex;%
      left-margin:-.1em;%
    }%
  }%
  \global\let\HoLogoCss@Xe\relax
}
%    \end{macrocode}
%    \end{macro}
%
%    \begin{macro}{\HoLogo@XeTeX}
%    \begin{macrocode}
\def\HoLogo@XeTeX#1{%
  \hologo{Xe}%
  \kern-.15em\relax
  \hologo{TeX}%
}
%    \end{macrocode}
%    \end{macro}
%
%    \begin{macro}{\HoLogoHtml@XeTeX}
%    \begin{macrocode}
\def\HoLogoHtml@XeTeX#1{%
  \HoLogoCss@XeTeX
  \HOLOGO@Span{XeTeX}{%
    \hologo{Xe}%
    \hologo{TeX}%
  }%
}
%    \end{macrocode}
%    \end{macro}
%    \begin{macro}{\HoLogoCss@XeTeX}
%    \begin{macrocode}
\def\HoLogoCss@XeTeX{%
  \Css{%
    span.HoLogo-XeTeX span.HoLogo-TeX{%
      margin-left:-.15em;%
    }%
  }%
  \global\let\HoLogoCss@XeTeX\relax
}
%    \end{macrocode}
%    \end{macro}
%
%    \begin{macro}{\HoLogo@XeLaTeX}
%    \begin{macrocode}
\def\HoLogo@XeLaTeX#1{%
  \hologo{Xe}%
  \kern-.13em%
  \hologo{LaTeX}%
}
%    \end{macrocode}
%    \end{macro}
%    \begin{macro}{\HoLogoHtml@XeLaTeX}
%    \begin{macrocode}
\def\HoLogoHtml@XeLaTeX#1{%
  \HoLogoCss@XeLaTeX
  \HOLOGO@Span{XeLaTeX}{%
    \hologo{Xe}%
    \hologo{LaTeX}%
  }%
}
%    \end{macrocode}
%    \end{macro}
%    \begin{macro}{\HoLogoCss@XeLaTeX}
%    \begin{macrocode}
\def\HoLogoCss@XeLaTeX{%
  \Css{%
    span.HoLogo-XeLaTeX span.HoLogo-Xe{%
      margin-right:-.13em;%
    }%
  }%
  \global\let\HoLogoCss@XeLaTeX\relax
}
%    \end{macrocode}
%    \end{macro}
%
% \subsubsection{\hologo{pdfTeX}, \hologo{pdfLaTeX}}
%
%    \begin{macro}{\HoLogo@pdfTeX}
%    \begin{macrocode}
\def\HoLogo@pdfTeX#1{%
  \HOLOGO@mbox{%
    #1{p}{P}df\hologo{TeX}%
  }%
}
%    \end{macrocode}
%    \end{macro}
%    \begin{macro}{\HoLogoCs@pdfTeX}
%    \begin{macrocode}
\def\HoLogoCs@pdfTeX#1{#1{p}{P}dfTeX}
%    \end{macrocode}
%    \end{macro}
%    \begin{macro}{\HoLogoBkm@pdfTeX}
%    \begin{macrocode}
\def\HoLogoBkm@pdfTeX#1{%
  #1{p}{P}df\hologo{TeX}%
}
%    \end{macrocode}
%    \end{macro}
%    \begin{macro}{\HoLogoHtml@pdfTeX}
%    \begin{macrocode}
\let\HoLogoHtml@pdfTeX\HoLogo@pdfTeX
%    \end{macrocode}
%    \end{macro}
%
%    \begin{macro}{\HoLogo@pdfLaTeX}
%    \begin{macrocode}
\def\HoLogo@pdfLaTeX#1{%
  \HOLOGO@mbox{%
    #1{p}{P}df\hologo{LaTeX}%
  }%
}
%    \end{macrocode}
%    \end{macro}
%    \begin{macro}{\HoLogoCs@pdfLaTeX}
%    \begin{macrocode}
\def\HoLogoCs@pdfLaTeX#1{#1{p}{P}dfLaTeX}
%    \end{macrocode}
%    \end{macro}
%    \begin{macro}{\HoLogoBkm@pdfLaTeX}
%    \begin{macrocode}
\def\HoLogoBkm@pdfLaTeX#1{%
  #1{p}{P}df\hologo{LaTeX}%
}
%    \end{macrocode}
%    \end{macro}
%    \begin{macro}{\HoLogoHtml@pdfLaTeX}
%    \begin{macrocode}
\let\HoLogoHtml@pdfLaTeX\HoLogo@pdfLaTeX
%    \end{macrocode}
%    \end{macro}
%
% \subsubsection{\hologo{VTeX}}
%
%    \begin{macro}{\HoLogo@VTeX}
%    \begin{macrocode}
\def\HoLogo@VTeX#1{%
  \HOLOGO@mbox{%
    V\hologo{TeX}%
  }%
}
%    \end{macrocode}
%    \end{macro}
%    \begin{macro}{\HoLogoHtml@VTeX}
%    \begin{macrocode}
\let\HoLogoHtml@VTeX\HoLogo@VTeX
%    \end{macrocode}
%    \end{macro}
%
% \subsubsection{\hologo{AmS}, \dots}
%
%    Source: class \xclass{amsdtx}
%
%    \begin{macro}{\HoLogo@AmS}
%    \begin{macrocode}
\def\HoLogo@AmS#1{%
  \HoLogoFont@font{AmS}{sy}{%
    A%
    \kern-.1667em%
    \lower.5ex\hbox{M}%
    \kern-.125em%
    S%
  }%
}
%    \end{macrocode}
%    \end{macro}
%    \begin{macro}{\HoLogoBkm@AmS}
%    \begin{macrocode}
\def\HoLogoBkm@AmS#1{AmS}
%    \end{macrocode}
%    \end{macro}
%    \begin{macro}{\HoLogoHtml@AmS}
%    \begin{macrocode}
\def\HoLogoHtml@AmS#1{%
  \HoLogoCss@AmS
%  \HoLogoFont@font{AmS}{sy}{%
    \HOLOGO@Span{AmS}{%
      A%
      \HOLOGO@Span{M}{M}%
      S%
    }%
%   }%
}
%    \end{macrocode}
%    \end{macro}
%    \begin{macro}{\HoLogoCss@AmS}
%    \begin{macrocode}
\def\HoLogoCss@AmS{%
  \Css{%
    span.HoLogo-AmS span.HoLogo-M{%
      position:relative;%
      top:.5ex;%
      margin-left:-.1667em;%
      margin-right:-.125em;%
      text-decoration:none;%
    }%
  }%
  \global\let\HoLogoCss@AmS\relax
}
%    \end{macrocode}
%    \end{macro}
%
%    \begin{macro}{\HoLogo@AmSTeX}
%    \begin{macrocode}
\def\HoLogo@AmSTeX#1{%
  \hologo{AmS}%
  \HOLOGO@hyphen
  \hologo{TeX}%
}
%    \end{macrocode}
%    \end{macro}
%    \begin{macro}{\HoLogoBkm@AmSTeX}
%    \begin{macrocode}
\def\HoLogoBkm@AmSTeX#1{AmS-TeX}%
%    \end{macrocode}
%    \end{macro}
%    \begin{macro}{\HoLogoHtml@AmSTeX}
%    \begin{macrocode}
\let\HoLogoHtml@AmSTeX\HoLogo@AmSTeX
%    \end{macrocode}
%    \end{macro}
%
%    \begin{macro}{\HoLogo@AmSLaTeX}
%    \begin{macrocode}
\def\HoLogo@AmSLaTeX#1{%
  \hologo{AmS}%
  \HOLOGO@hyphen
  \hologo{LaTeX}%
}
%    \end{macrocode}
%    \end{macro}
%    \begin{macro}{\HoLogoBkm@AmSLaTeX}
%    \begin{macrocode}
\def\HoLogoBkm@AmSLaTeX#1{AmS-LaTeX}%
%    \end{macrocode}
%    \end{macro}
%    \begin{macro}{\HoLogoHtml@AmSLaTeX}
%    \begin{macrocode}
\let\HoLogoHtml@AmSLaTeX\HoLogo@AmSLaTeX
%    \end{macrocode}
%    \end{macro}
%
% \subsubsection{\hologo{BibTeX}}
%
%    \begin{macro}{\HoLogo@BibTeX@sc}
%    A definition of \hologo{BibTeX} is provided in
%    the documentation source for the manual of \hologo{BibTeX}
%    \cite{btxdoc}.
%\begin{quote}
%\begin{verbatim}
%\def\BibTeX{%
%  {%
%    \rm
%    B%
%    \kern-.05em%
%    {%
%      \sc
%      i%
%      \kern-.025em %
%      b%
%    }%
%    \kern-.08em
%    T%
%    \kern-.1667em%
%    \lower.7ex\hbox{E}%
%    \kern-.125em%
%    X%
%  }%
%}
%\end{verbatim}
%\end{quote}
%    \begin{macrocode}
\def\HoLogo@BibTeX@sc#1{%
  B%
  \kern-.05em%
  \HoLogoFont@font{BibTeX}{sc}{%
    i%
    \kern-.025em%
    b%
  }%
  \HOLOGO@discretionary
  \kern-.08em%
  \hologo{TeX}%
}
%    \end{macrocode}
%    \end{macro}
%    \begin{macro}{\HoLogoHtml@BibTeX@sc}
%    \begin{macrocode}
\def\HoLogoHtml@BibTeX@sc#1{%
  \HoLogoCss@BibTeX@sc
  \HOLOGO@Span{BibTeX-sc}{%
    B%
    \HOLOGO@Span{i}{i}%
    \HOLOGO@Span{b}{b}%
    \hologo{TeX}%
  }%
}
%    \end{macrocode}
%    \end{macro}
%    \begin{macro}{\HoLogoCss@BibTeX@sc}
%    \begin{macrocode}
\def\HoLogoCss@BibTeX@sc{%
  \Css{%
    span.HoLogo-BibTeX-sc span.HoLogo-i{%
      margin-left:-.05em;%
      margin-right:-.025em;%
      font-variant:small-caps;%
    }%
  }%
  \Css{%
    span.HoLogo-BibTeX-sc span.HoLogo-b{%
      margin-right:-.08em;%
      font-variant:small-caps;%
    }%
  }%
  \global\let\HoLogoCss@BibTeX@sc\relax
}
%    \end{macrocode}
%    \end{macro}
%
%    \begin{macro}{\HoLogo@BibTeX@sf}
%    Variant \xoption{sf} avoids trouble with unavailable
%    small caps fonts (e.g., bold versions of Computer Modern or
%    Latin Modern). The definition is taken from
%    package \xpackage{dtklogos} \cite{dtklogos}.
%\begin{quote}
%\begin{verbatim}
%\DeclareRobustCommand{\BibTeX}{%
%  B%
%  \kern-.05em%
%  \hbox{%
%    $\m@th$% %% force math size calculations
%    \csname S@\f@size\endcsname
%    \fontsize\sf@size\z@
%    \math@fontsfalse
%    \selectfont
%    I%
%    \kern-.025em%
%    B
%  }%
%  \kern-.08em%
%  \-%
%  \TeX
%}
%\end{verbatim}
%\end{quote}
%    \begin{macrocode}
\def\HoLogo@BibTeX@sf#1{%
  B%
  \kern-.05em%
  \HoLogoFont@font{BibTeX}{bibsf}{%
    I%
    \kern-.025em%
    B%
  }%
  \HOLOGO@discretionary
  \kern-.08em%
  \hologo{TeX}%
}
%    \end{macrocode}
%    \end{macro}
%    \begin{macro}{\HoLogoHtml@BibTeX@sf}
%    \begin{macrocode}
\def\HoLogoHtml@BibTeX@sf#1{%
  \HoLogoCss@BibTeX@sf
  \HOLOGO@Span{BibTeX-sf}{%
    B%
    \HoLogoFont@font{BibTeX}{bibsf}{%
      \HOLOGO@Span{i}{I}%
      B%
    }%
    \hologo{TeX}%
  }%
}
%    \end{macrocode}
%    \end{macro}
%    \begin{macro}{\HoLogoCss@BibTeX@sf}
%    \begin{macrocode}
\def\HoLogoCss@BibTeX@sf{%
  \Css{%
    span.HoLogo-BibTeX-sf span.HoLogo-i{%
      margin-left:-.05em;%
      margin-right:-.025em;%
    }%
  }%
  \Css{%
    span.HoLogo-BibTeX-sf span.HoLogo-TeX{%
      margin-left:-.08em;%
    }%
  }%
  \global\let\HoLogoCss@BibTeX@sf\relax
}
%    \end{macrocode}
%    \end{macro}
%
%    \begin{macro}{\HoLogo@BibTeX}
%    \begin{macrocode}
\def\HoLogo@BibTeX{\HoLogo@BibTeX@sf}
%    \end{macrocode}
%    \end{macro}
%    \begin{macro}{\HoLogoHtml@BibTeX}
%    \begin{macrocode}
\def\HoLogoHtml@BibTeX{\HoLogoHtml@BibTeX@sf}
%    \end{macrocode}
%    \end{macro}
%
% \subsubsection{\hologo{BibTeX8}}
%
%    \begin{macro}{\HoLogo@BibTeX8}
%    \begin{macrocode}
\expandafter\def\csname HoLogo@BibTeX8\endcsname#1{%
  \hologo{BibTeX}%
  8%
}
%    \end{macrocode}
%    \end{macro}
%
%    \begin{macro}{\HoLogoBkm@BibTeX8}
%    \begin{macrocode}
\expandafter\def\csname HoLogoBkm@BibTeX8\endcsname#1{%
  \hologo{BibTeX}%
  8%
}
%    \end{macrocode}
%    \end{macro}
%    \begin{macro}{\HoLogoHtml@BibTeX8}
%    \begin{macrocode}
\expandafter
\let\csname HoLogoHtml@BibTeX8\expandafter\endcsname
\csname HoLogo@BibTeX8\endcsname
%    \end{macrocode}
%    \end{macro}
%
% \subsubsection{\hologo{ConTeXt}}
%
%    \begin{macro}{\HoLogo@ConTeXt@simple}
%    \begin{macrocode}
\def\HoLogo@ConTeXt@simple#1{%
  \HOLOGO@mbox{Con}%
  \HOLOGO@discretionary
  \HOLOGO@mbox{\hologo{TeX}t}%
}
%    \end{macrocode}
%    \end{macro}
%    \begin{macro}{\HoLogoHtml@ConTeXt@simple}
%    \begin{macrocode}
\let\HoLogoHtml@ConTeXt@simple\HoLogo@ConTeXt@simple
%    \end{macrocode}
%    \end{macro}
%
%    \begin{macro}{\HoLogo@ConTeXt@narrow}
%    This definition of logo \hologo{ConTeXt} with variant \xoption{narrow}
%    comes from TUGboat's class \xclass{ltugboat} (version 2010/11/15 v2.8).
%    \begin{macrocode}
\def\HoLogo@ConTeXt@narrow#1{%
  \HOLOGO@mbox{C\kern-.0333emon}%
  \HOLOGO@discretionary
  \kern-.0667em%
  \HOLOGO@mbox{\hologo{TeX}\kern-.0333emt}%
}
%    \end{macrocode}
%    \end{macro}
%    \begin{macro}{\HoLogoHtml@ConTeXt@narrow}
%    \begin{macrocode}
\def\HoLogoHtml@ConTeXt@narrow#1{%
  \HoLogoCss@ConTeXt@narrow
  \HOLOGO@Span{ConTeXt-narrow}{%
    \HOLOGO@Span{C}{C}%
    on%
    \hologo{TeX}%
    t%
  }%
}
%    \end{macrocode}
%    \end{macro}
%    \begin{macro}{\HoLogoCss@ConTeXt@narrow}
%    \begin{macrocode}
\def\HoLogoCss@ConTeXt@narrow{%
  \Css{%
    span.HoLogo-ConTeXt-narrow span.HoLogo-C{%
      margin-left:-.0333em;%
    }%
  }%
  \Css{%
    span.HoLogo-ConTeXt-narrow span.HoLogo-TeX{%
      margin-left:-.0667em;%
      margin-right:-.0333em;%
    }%
  }%
  \global\let\HoLogoCss@ConTeXt@narrow\relax
}
%    \end{macrocode}
%    \end{macro}
%
%    \begin{macro}{\HoLogo@ConTeXt}
%    \begin{macrocode}
\def\HoLogo@ConTeXt{\HoLogo@ConTeXt@narrow}
%    \end{macrocode}
%    \end{macro}
%    \begin{macro}{\HoLogoHtml@ConTeXt}
%    \begin{macrocode}
\def\HoLogoHtml@ConTeXt{\HoLogoHtml@ConTeXt@narrow}
%    \end{macrocode}
%    \end{macro}
%
% \subsubsection{\hologo{emTeX}}
%
%    \begin{macro}{\HoLogo@emTeX}
%    \begin{macrocode}
\def\HoLogo@emTeX#1{%
  \HOLOGO@mbox{#1{e}{E}m}%
  \HOLOGO@discretionary
  \hologo{TeX}%
}
%    \end{macrocode}
%    \end{macro}
%    \begin{macro}{\HoLogoCs@emTeX}
%    \begin{macrocode}
\def\HoLogoCs@emTeX#1{#1{e}{E}mTeX}%
%    \end{macrocode}
%    \end{macro}
%    \begin{macro}{\HoLogoBkm@emTeX}
%    \begin{macrocode}
\def\HoLogoBkm@emTeX#1{%
  #1{e}{E}m\hologo{TeX}%
}
%    \end{macrocode}
%    \end{macro}
%    \begin{macro}{\HoLogoHtml@emTeX}
%    \begin{macrocode}
\let\HoLogoHtml@emTeX\HoLogo@emTeX
%    \end{macrocode}
%    \end{macro}
%
% \subsubsection{\hologo{ExTeX}}
%
%    \begin{macro}{\HoLogo@ExTeX}
%    The definition is taken from the FAQ of the
%    project \hologo{ExTeX}
%    \cite{ExTeX-FAQ}.
%\begin{quote}
%\begin{verbatim}
%\def\ExTeX{%
%  \textrm{% Logo always with serifs
%    \ensuremath{%
%      \textstyle
%      \varepsilon_{%
%        \kern-0.15em%
%        \mathcal{X}%
%      }%
%    }%
%    \kern-.15em%
%    \TeX
%  }%
%}
%\end{verbatim}
%\end{quote}
%    \begin{macrocode}
\def\HoLogo@ExTeX#1{%
  \HoLogoFont@font{ExTeX}{rm}{%
    \ltx@mbox{%
      \HOLOGO@MathSetup
      $%
        \textstyle
        \varepsilon_{%
          \kern-0.15em%
          \HoLogoFont@font{ExTeX}{sy}{X}%
        }%
      $%
    }%
    \HOLOGO@discretionary
    \kern-.15em%
    \hologo{TeX}%
  }%
}
%    \end{macrocode}
%    \end{macro}
%    \begin{macro}{\HoLogoHtml@ExTeX}
%    \begin{macrocode}
\def\HoLogoHtml@ExTeX#1{%
  \HoLogoCss@ExTeX
  \HoLogoFont@font{ExTeX}{rm}{%
    \HOLOGO@Span{ExTeX}{%
      \ltx@mbox{%
        \HOLOGO@MathSetup
        $\textstyle\varepsilon$%
        \HOLOGO@Span{X}{$\textstyle\chi$}%
        \hologo{TeX}%
      }%
    }%
  }%
}
%    \end{macrocode}
%    \end{macro}
%    \begin{macro}{\HoLogoBkm@ExTeX}
%    \begin{macrocode}
\def\HoLogoBkm@ExTeX#1{%
  \HOLOGO@PdfdocUnicode{#1{e}{E}x}{\textepsilon\textchi}%
  \hologo{TeX}%
}
%    \end{macrocode}
%    \end{macro}
%    \begin{macro}{\HoLogoCss@ExTeX}
%    \begin{macrocode}
\def\HoLogoCss@ExTeX{%
  \Css{%
    span.HoLogo-ExTeX{%
      font-family:serif;%
    }%
  }%
  \Css{%
    span.HoLogo-ExTeX span.HoLogo-TeX{%
      margin-left:-.15em;%
    }%
  }%
  \global\let\HoLogoCss@ExTeX\relax
}
%    \end{macrocode}
%    \end{macro}
%
% \subsubsection{\hologo{MiKTeX}}
%
%    \begin{macro}{\HoLogo@MiKTeX}
%    \begin{macrocode}
\def\HoLogo@MiKTeX#1{%
  \HOLOGO@mbox{MiK}%
  \HOLOGO@discretionary
  \hologo{TeX}%
}
%    \end{macrocode}
%    \end{macro}
%    \begin{macro}{\HoLogoHtml@MiKTeX}
%    \begin{macrocode}
\let\HoLogoHtml@MiKTeX\HoLogo@MiKTeX
%    \end{macrocode}
%    \end{macro}
%
% \subsubsection{\hologo{OzTeX} and friends}
%
%    Source: \hologo{OzTeX} FAQ \cite{OzTeX}:
%    \begin{quote}
%      |\def\OzTeX{O\kern-.03em z\kern-.15em\TeX}|\\
%      (There is no kerning in OzMF, OzMP and OzTtH.)
%    \end{quote}
%
%    \begin{macro}{\HoLogo@OzTeX}
%    \begin{macrocode}
\def\HoLogo@OzTeX#1{%
  O%
  \kern-.03em %
  z%
  \kern-.15em %
  \hologo{TeX}%
}
%    \end{macrocode}
%    \end{macro}
%    \begin{macro}{\HoLogoHtml@OzTeX}
%    \begin{macrocode}
\def\HoLogoHtml@OzTeX#1{%
  \HoLogoCss@OzTeX
  \HOLOGO@Span{OzTeX}{%
    O%
    \HOLOGO@Span{z}{z}%
    \hologo{TeX}%
  }%
}
%    \end{macrocode}
%    \end{macro}
%    \begin{macro}{\HoLogoCss@OzTeX}
%    \begin{macrocode}
\def\HoLogoCss@OzTeX{%
  \Css{%
    span.HoLogo-OzTeX span.HoLogo-z{%
      margin-left:-.03em;%
      margin-right:-.15em;%
    }%
  }%
  \global\let\HoLogoCss@OzTeX\relax
}
%    \end{macrocode}
%    \end{macro}
%
%    \begin{macro}{\HoLogo@OzMF}
%    \begin{macrocode}
\def\HoLogo@OzMF#1{%
  \HOLOGO@mbox{OzMF}%
}
%    \end{macrocode}
%    \end{macro}
%    \begin{macro}{\HoLogo@OzMP}
%    \begin{macrocode}
\def\HoLogo@OzMP#1{%
  \HOLOGO@mbox{OzMP}%
}
%    \end{macrocode}
%    \end{macro}
%    \begin{macro}{\HoLogo@OzTtH}
%    \begin{macrocode}
\def\HoLogo@OzTtH#1{%
  \HOLOGO@mbox{OzTtH}%
}
%    \end{macrocode}
%    \end{macro}
%
% \subsubsection{\hologo{PCTeX}}
%
%    \begin{macro}{\HoLogo@PCTeX}
%    \begin{macrocode}
\def\HoLogo@PCTeX#1{%
  \HOLOGO@mbox{PC}%
  \hologo{TeX}%
}
%    \end{macrocode}
%    \end{macro}
%    \begin{macro}{\HoLogoHtml@PCTeX}
%    \begin{macrocode}
\let\HoLogoHtml@PCTeX\HoLogo@PCTeX
%    \end{macrocode}
%    \end{macro}
%
% \subsubsection{\hologo{PiCTeX}}
%
%    The original definitions from \xfile{pictex.tex} \cite{PiCTeX}:
%\begin{quote}
%\begin{verbatim}
%\def\PiC{%
%  P%
%  \kern-.12em%
%  \lower.5ex\hbox{I}%
%  \kern-.075em%
%  C%
%}
%\def\PiCTeX{%
%  \PiC
%  \kern-.11em%
%  \TeX
%}
%\end{verbatim}
%\end{quote}
%
%    \begin{macro}{\HoLogo@PiC}
%    \begin{macrocode}
\def\HoLogo@PiC#1{%
  P%
  \kern-.12em%
  \lower.5ex\hbox{I}%
  \kern-.075em%
  C%
  \HOLOGO@SpaceFactor
}
%    \end{macrocode}
%    \end{macro}
%    \begin{macro}{\HoLogoHtml@PiC}
%    \begin{macrocode}
\def\HoLogoHtml@PiC#1{%
  \HoLogoCss@PiC
  \HOLOGO@Span{PiC}{%
    P%
    \HOLOGO@Span{i}{I}%
    C%
  }%
}
%    \end{macrocode}
%    \end{macro}
%    \begin{macro}{\HoLogoCss@PiC}
%    \begin{macrocode}
\def\HoLogoCss@PiC{%
  \Css{%
    span.HoLogo-PiC span.HoLogo-i{%
      position:relative;%
      top:.5ex;%
      margin-left:-.12em;%
      margin-right:-.075em;%
      text-decoration:none;%
    }%
  }%
  \global\let\HoLogoCss@PiC\relax
}
%    \end{macrocode}
%    \end{macro}
%
%    \begin{macro}{\HoLogo@PiCTeX}
%    \begin{macrocode}
\def\HoLogo@PiCTeX#1{%
  \hologo{PiC}%
  \HOLOGO@discretionary
  \kern-.11em%
  \hologo{TeX}%
}
%    \end{macrocode}
%    \end{macro}
%    \begin{macro}{\HoLogoHtml@PiCTeX}
%    \begin{macrocode}
\def\HoLogoHtml@PiCTeX#1{%
  \HoLogoCss@PiCTeX
  \HOLOGO@Span{PiCTeX}{%
    \hologo{PiC}%
    \hologo{TeX}%
  }%
}
%    \end{macrocode}
%    \end{macro}
%    \begin{macro}{\HoLogoCss@PiCTeX}
%    \begin{macrocode}
\def\HoLogoCss@PiCTeX{%
  \Css{%
    span.HoLogo-PiCTeX span.HoLogo-PiC{%
      margin-right:-.11em;%
    }%
  }%
  \global\let\HoLogoCss@PiCTeX\relax
}
%    \end{macrocode}
%    \end{macro}
%
% \subsubsection{\hologo{teTeX}}
%
%    \begin{macro}{\HoLogo@teTeX}
%    \begin{macrocode}
\def\HoLogo@teTeX#1{%
  \HOLOGO@mbox{#1{t}{T}e}%
  \HOLOGO@discretionary
  \hologo{TeX}%
}
%    \end{macrocode}
%    \end{macro}
%    \begin{macro}{\HoLogoCs@teTeX}
%    \begin{macrocode}
\def\HoLogoCs@teTeX#1{#1{t}{T}dfTeX}
%    \end{macrocode}
%    \end{macro}
%    \begin{macro}{\HoLogoBkm@teTeX}
%    \begin{macrocode}
\def\HoLogoBkm@teTeX#1{%
  #1{t}{T}e\hologo{TeX}%
}
%    \end{macrocode}
%    \end{macro}
%    \begin{macro}{\HoLogoHtml@teTeX}
%    \begin{macrocode}
\let\HoLogoHtml@teTeX\HoLogo@teTeX
%    \end{macrocode}
%    \end{macro}
%
% \subsubsection{\hologo{TeX4ht}}
%
%    \begin{macro}{\HoLogo@TeX4ht}
%    \begin{macrocode}
\expandafter\def\csname HoLogo@TeX4ht\endcsname#1{%
  \HOLOGO@mbox{\hologo{TeX}4ht}%
}
%    \end{macrocode}
%    \end{macro}
%    \begin{macro}{\HoLogoHtml@TeX4ht}
%    \begin{macrocode}
\expandafter
\let\csname HoLogoHtml@TeX4ht\expandafter\endcsname
\csname HoLogo@TeX4ht\endcsname
%    \end{macrocode}
%    \end{macro}
%
%
% \subsubsection{\hologo{SageTeX}}
%
%    \begin{macro}{\HoLogo@SageTeX}
%    \begin{macrocode}
\def\HoLogo@SageTeX#1{%
  \HOLOGO@mbox{Sage}%
  \HOLOGO@discretionary
  \HOLOGO@NegativeKerning{eT,oT,To}%
  \hologo{TeX}%
}
%    \end{macrocode}
%    \end{macro}
%    \begin{macro}{\HoLogoHtml@SageTeX}
%    \begin{macrocode}
\let\HoLogoHtml@SageTeX\HoLogo@SageTeX
%    \end{macrocode}
%    \end{macro}
%
% \subsection{\hologo{METAFONT} and friends}
%
%    \begin{macro}{\HoLogo@METAFONT}
%    \begin{macrocode}
\def\HoLogo@METAFONT#1{%
  \HoLogoFont@font{METAFONT}{logo}{%
    \HOLOGO@mbox{META}%
    \HOLOGO@discretionary
    \HOLOGO@mbox{FONT}%
  }%
}
%    \end{macrocode}
%    \end{macro}
%
%    \begin{macro}{\HoLogo@METAPOST}
%    \begin{macrocode}
\def\HoLogo@METAPOST#1{%
  \HoLogoFont@font{METAPOST}{logo}{%
    \HOLOGO@mbox{META}%
    \HOLOGO@discretionary
    \HOLOGO@mbox{POST}%
  }%
}
%    \end{macrocode}
%    \end{macro}
%
%    \begin{macro}{\HoLogo@MetaFun}
%    \begin{macrocode}
\def\HoLogo@MetaFun#1{%
  \HOLOGO@mbox{Meta}%
  \HOLOGO@discretionary
  \HOLOGO@mbox{Fun}%
}
%    \end{macrocode}
%    \end{macro}
%
%    \begin{macro}{\HoLogo@MetaPost}
%    \begin{macrocode}
\def\HoLogo@MetaPost#1{%
  \HOLOGO@mbox{Meta}%
  \HOLOGO@discretionary
  \HOLOGO@mbox{Post}%
}
%    \end{macrocode}
%    \end{macro}
%
% \subsection{Others}
%
% \subsubsection{\hologo{biber}}
%
%    \begin{macro}{\HoLogo@biber}
%    \begin{macrocode}
\def\HoLogo@biber#1{%
  \HOLOGO@mbox{#1{b}{B}i}%
  \HOLOGO@discretionary
  \HOLOGO@mbox{ber}%
}
%    \end{macrocode}
%    \end{macro}
%    \begin{macro}{\HoLogoCs@biber}
%    \begin{macrocode}
\def\HoLogoCs@biber#1{#1{b}{B}iber}
%    \end{macrocode}
%    \end{macro}
%    \begin{macro}{\HoLogoBkm@biber}
%    \begin{macrocode}
\def\HoLogoBkm@biber#1{%
  #1{b}{B}iber%
}
%    \end{macrocode}
%    \end{macro}
%    \begin{macro}{\HoLogoHtml@biber}
%    \begin{macrocode}
\let\HoLogoHtml@biber\HoLogo@biber
%    \end{macrocode}
%    \end{macro}
%
% \subsubsection{\hologo{KOMAScript}}
%
%    \begin{macro}{\HoLogo@KOMAScript}
%    The definition for \hologo{KOMAScript} is taken
%    from \hologo{KOMAScript} (\xfile{scrlogo.dtx}, reformatted) \cite{scrlogo}:
%\begin{quote}
%\begin{verbatim}
%\@ifundefined{KOMAScript}{%
%  \DeclareRobustCommand{\KOMAScript}{%
%    \textsf{%
%      K\kern.05em O\kern.05emM\kern.05em A%
%      \kern.1em-\kern.1em %
%      Script%
%    }%
%  }%
%}{}
%\end{verbatim}
%\end{quote}
%    \begin{macrocode}
\def\HoLogo@KOMAScript#1{%
  \HoLogoFont@font{KOMAScript}{sf}{%
    \HOLOGO@mbox{%
      K\kern.05em%
      O\kern.05em%
      M\kern.05em%
      A%
    }%
    \kern.1em%
    \HOLOGO@hyphen
    \kern.1em%
    \HOLOGO@mbox{Script}%
  }%
}
%    \end{macrocode}
%    \end{macro}
%    \begin{macro}{\HoLogoBkm@KOMAScript}
%    \begin{macrocode}
\def\HoLogoBkm@KOMAScript#1{%
  KOMA-Script%
}
%    \end{macrocode}
%    \end{macro}
%    \begin{macro}{\HoLogoHtml@KOMAScript}
%    \begin{macrocode}
\def\HoLogoHtml@KOMAScript#1{%
  \HoLogoCss@KOMAScript
  \HoLogoFont@font{KOMAScript}{sf}{%
    \HOLOGO@Span{KOMAScript}{%
      K%
      \HOLOGO@Span{O}{O}%
      M%
      \HOLOGO@Span{A}{A}%
      \HOLOGO@Span{hyphen}{-}%
      Script%
    }%
  }%
}
%    \end{macrocode}
%    \end{macro}
%    \begin{macro}{\HoLogoCss@KOMAScript}
%    \begin{macrocode}
\def\HoLogoCss@KOMAScript{%
  \Css{%
    span.HoLogo-KOMAScript{%
      font-family:sans-serif;%
    }%
  }%
  \Css{%
    span.HoLogo-KOMAScript span.HoLogo-O{%
      padding-left:.05em;%
      padding-right:.05em;%
    }%
  }%
  \Css{%
    span.HoLogo-KOMAScript span.HoLogo-A{%
      padding-left:.05em;%
    }%
  }%
  \Css{%
    span.HoLogo-KOMAScript span.HoLogo-hyphen{%
      padding-left:.1em;%
      padding-right:.1em;%
    }%
  }%
  \global\let\HoLogoCss@KOMAScript\relax
}
%    \end{macrocode}
%    \end{macro}
%
% \subsubsection{\hologo{LyX}}
%
%    \begin{macro}{\HoLogo@LyX}
%    The definition is taken from the documentation source files
%    of \hologo{LyX}, \xfile{Intro.lyx} \cite{LyX}:
%\begin{quote}
%\begin{verbatim}
%\def\LyX{%
%  \texorpdfstring{%
%    L\kern-.1667em\lower.25em\hbox{Y}\kern-.125emX\@%
%  }{%
%    LyX%
%  }%
%}
%\end{verbatim}
%\end{quote}
%    \begin{macrocode}
\def\HoLogo@LyX#1{%
  L%
  \kern-.1667em%
  \lower.25em\hbox{Y}%
  \kern-.125em%
  X%
  \HOLOGO@SpaceFactor
}
%    \end{macrocode}
%    \end{macro}
%    \begin{macro}{\HoLogoHtml@LyX}
%    \begin{macrocode}
\def\HoLogoHtml@LyX#1{%
  \HoLogoCss@LyX
  \HOLOGO@Span{LyX}{%
    L%
    \HOLOGO@Span{y}{Y}%
    X%
  }%
}
%    \end{macrocode}
%    \end{macro}
%    \begin{macro}{\HoLogoCss@LyX}
%    \begin{macrocode}
\def\HoLogoCss@LyX{%
  \Css{%
    span.HoLogo-LyX span.HoLogo-y{%
      position:relative;%
      top:.25em;%
      margin-left:-.1667em;%
      margin-right:-.125em;%
      text-decoration:none;%
    }%
  }%
  \global\let\HoLogoCss@LyX\relax
}
%    \end{macrocode}
%    \end{macro}
%
% \subsubsection{\hologo{NTS}}
%
%    \begin{macro}{\HoLogo@NTS}
%    Definition for \hologo{NTS} can be found in
%    package \xpackage{etex\textunderscore man} for the \hologo{eTeX} manual \cite{etexman}
%    and in package \xpackage{dtklogos} \cite{dtklogos}:
%\begin{quote}
%\begin{verbatim}
%\def\NTS{%
%  \leavevmode
%  \hbox{%
%    $%
%      \cal N%
%      \kern-0.35em%
%      \lower0.5ex\hbox{$\cal T$}%
%      \kern-0.2em%
%      S%
%    $%
%  }%
%}
%\end{verbatim}
%\end{quote}
%    \begin{macrocode}
\def\HoLogo@NTS#1{%
  \HoLogoFont@font{NTS}{sy}{%
    N\/%
    \kern-.35em%
    \lower.5ex\hbox{T\/}%
    \kern-.2em%
    S\/%
  }%
  \HOLOGO@SpaceFactor
}
%    \end{macrocode}
%    \end{macro}
%
% \subsubsection{\Hologo{TTH} (\hologo{TeX} to HTML translator)}
%
%    Source: \url{http://hutchinson.belmont.ma.us/tth/}
%    In the HTML source the second `T' is printed as subscript.
%\begin{quote}
%\begin{verbatim}
%T<sub>T</sub>H
%\end{verbatim}
%\end{quote}
%    \begin{macro}{\HoLogo@TTH}
%    \begin{macrocode}
\def\HoLogo@TTH#1{%
  \ltx@mbox{%
    T\HOLOGO@SubScript{T}H%
  }%
  \HOLOGO@SpaceFactor
}
%    \end{macrocode}
%    \end{macro}
%
%    \begin{macro}{\HoLogoHtml@TTH}
%    \begin{macrocode}
\def\HoLogoHtml@TTH#1{%
  T\HCode{<sub>}T\HCode{</sub>}H%
}
%    \end{macrocode}
%    \end{macro}
%
% \subsubsection{\Hologo{HanTheThanh}}
%
%    Partial source: Package \xpackage{dtklogos}.
%    The double accent is U+1EBF (latin small letter e with circumflex
%    and acute).
%    \begin{macro}{\HoLogo@HanTheThanh}
%    \begin{macrocode}
\def\HoLogo@HanTheThanh#1{%
  \ltx@mbox{H\`an}%
  \HOLOGO@space
  \ltx@mbox{%
    Th%
    \HOLOGO@IfCharExists{"1EBF}{%
      \char"1EBF\relax
    }{%
      \^e\hbox to 0pt{\hss\raise .5ex\hbox{\'{}}}%
    }%
  }%
  \HOLOGO@space
  \ltx@mbox{Th\`anh}%
}
%    \end{macrocode}
%    \end{macro}
%    \begin{macro}{\HoLogoBkm@HanTheThanh}
%    \begin{macrocode}
\def\HoLogoBkm@HanTheThanh#1{%
  H\`an %
  Th\HOLOGO@PdfdocUnicode{\^e}{\9036\277} %
  Th\`anh%
}
%    \end{macrocode}
%    \end{macro}
%    \begin{macro}{\HoLogoHtml@HanTheThanh}
%    \begin{macrocode}
\def\HoLogoHtml@HanTheThanh#1{%
  H\`an %
  Th\HCode{&\ltx@hashchar x1ebf;} %
  Th\`anh%
}
%    \end{macrocode}
%    \end{macro}
%
% \subsection{Driver detection}
%
%    \begin{macrocode}
\HOLOGO@IfExists\InputIfFileExists{%
  \InputIfFileExists{hologo.cfg}{}{}%
}{%
  \ltx@IfUndefined{pdf@filesize}{%
    \def\HOLOGO@InputIfExists{%
      \openin\HOLOGO@temp=hologo.cfg\relax
      \ifeof\HOLOGO@temp
        \closein\HOLOGO@temp
      \else
        \closein\HOLOGO@temp
        \begingroup
          \def\x{LaTeX2e}%
        \expandafter\endgroup
        \ifx\fmtname\x
          \input{hologo.cfg}%
        \else
          \input hologo.cfg\relax
        \fi
      \fi
    }%
    \ltx@IfUndefined{newread}{%
      \chardef\HOLOGO@temp=15 %
      \def\HOLOGO@CheckRead{%
        \ifeof\HOLOGO@temp
          \HOLOGO@InputIfExists
        \else
          \ifcase\HOLOGO@temp
            \@PackageWarningNoLine{hologo}{%
              Configuration file ignored, because\MessageBreak
              a free read register could not be found%
            }%
          \else
            \begingroup
              \count\ltx@cclv=\HOLOGO@temp
              \advance\ltx@cclv by \ltx@minusone
              \edef\x{\endgroup
                \chardef\noexpand\HOLOGO@temp=\the\count\ltx@cclv
                \relax
              }%
            \x
          \fi
        \fi
      }%
    }{%
      \csname newread\endcsname\HOLOGO@temp
      \HOLOGO@InputIfExists
    }%
  }{%
    \edef\HOLOGO@temp{\pdf@filesize{hologo.cfg}}%
    \ifx\HOLOGO@temp\ltx@empty
    \else
      \ifnum\HOLOGO@temp>0 %
        \begingroup
          \def\x{LaTeX2e}%
        \expandafter\endgroup
        \ifx\fmtname\x
          \input{hologo.cfg}%
        \else
          \input hologo.cfg\relax
        \fi
      \else
        \@PackageInfoNoLine{hologo}{%
          Empty configuration file `hologo.cfg' ignored%
        }%
      \fi
    \fi
  }%
}
%    \end{macrocode}
%
%    \begin{macrocode}
\def\HOLOGO@temp#1#2{%
  \kv@define@key{HoLogoDriver}{#1}[]{%
    \begingroup
      \def\HOLOGO@temp{##1}%
      \ltx@onelevel@sanitize\HOLOGO@temp
      \ifx\HOLOGO@temp\ltx@empty
      \else
        \@PackageError{hologo}{%
          Value (\HOLOGO@temp) not permitted for option `#1'%
        }%
        \@ehc
      \fi
    \endgroup
    \def\hologoDriver{#2}%
  }%
}%
\def\HOLOGO@@temp#1#2{%
  \ifx\kv@value\relax
    \HOLOGO@temp{#1}{#1}%
  \else
    \HOLOGO@temp{#1}{#2}%
  \fi
}%
\kv@parse@normalized{%
  pdftex,%
  luatex=pdftex,%
  dvipdfm,%
  dvipdfmx=dvipdfm,%
  dvips,%
  dvipsone=dvips,%
  xdvi=dvips,%
  xetex,%
  vtex,%
}\HOLOGO@@temp
%    \end{macrocode}
%
%    \begin{macrocode}
\kv@define@key{HoLogoDriver}{driverfallback}{%
  \def\HOLOGO@DriverFallback{#1}%
}
%    \end{macrocode}
%
%    \begin{macro}{\HOLOGO@DriverFallback}
%    \begin{macrocode}
\def\HOLOGO@DriverFallback{dvips}
%    \end{macrocode}
%    \end{macro}
%
%    \begin{macro}{\hologoDriverSetup}
%    \begin{macrocode}
\def\hologoDriverSetup{%
  \let\hologoDriver\ltx@undefined
  \HOLOGO@DriverSetup
}
%    \end{macrocode}
%    \end{macro}
%
%    \begin{macro}{\HOLOGO@DriverSetup}
%    \begin{macrocode}
\def\HOLOGO@DriverSetup#1{%
  \kvsetkeys{HoLogoDriver}{#1}%
  \HOLOGO@CheckDriver
  \ltx@ifundefined{hologoDriver}{%
    \begingroup
    \edef\x{\endgroup
      \noexpand\kvsetkeys{HoLogoDriver}{\HOLOGO@DriverFallback}%
    }\x
  }{}%
  \@PackageInfoNoLine{hologo}{Using driver `\hologoDriver'}%
}
%    \end{macrocode}
%    \end{macro}
%
%    \begin{macro}{\HOLOGO@CheckDriver}
%    \begin{macrocode}
\def\HOLOGO@CheckDriver{%
  \ifpdf
    \def\hologoDriver{pdftex}%
    \let\HOLOGO@pdfliteral\pdfliteral
    \ifluatex
      \ifx\pdfextension\@undefined\else
        \protected\def\pdfliteral{\pdfextension literal}%
        \let\HOLOGO@pdfliteral\pdfliteral
      \fi
      \ltx@IfUndefined{HOLOGO@pdfliteral}{%
        \ifnum\luatexversion<36 %
        \else
          \begingroup
            \let\HOLOGO@temp\endgroup
            \ifcase0%
                \directlua{%
                  if tex.enableprimitives then %
                    tex.enableprimitives('HOLOGO@', {'pdfliteral'})%
                  else %
                    tex.print('1')%
                  end%
                }%
                \ifx\HOLOGO@pdfliteral\@undefined 1\fi%
                \relax%
              \endgroup
              \let\HOLOGO@temp\relax
              \global\let\HOLOGO@pdfliteral\HOLOGO@pdfliteral
            \fi%
          \HOLOGO@temp
        \fi
      }{}%
    \fi
    \ltx@IfUndefined{HOLOGO@pdfliteral}{%
      \@PackageWarningNoLine{hologo}{%
        Cannot find \string\pdfliteral
      }%
    }{}%
  \else
    \ifxetex
      \def\hologoDriver{xetex}%
    \else
      \ifvtex
        \def\hologoDriver{vtex}%
      \fi
    \fi
  \fi
}
%    \end{macrocode}
%    \end{macro}
%
%    \begin{macro}{\HOLOGO@WarningUnsupportedDriver}
%    \begin{macrocode}
\def\HOLOGO@WarningUnsupportedDriver#1{%
  \@PackageWarningNoLine{hologo}{%
    Logo `#1' needs driver specific macros,\MessageBreak
    but driver `\hologoDriver' is not supported.\MessageBreak
    Use a different driver or\MessageBreak
    load package `graphics' or `pgf'%
  }%
}
%    \end{macrocode}
%    \end{macro}
%
% \subsubsection{Reflect box macros}
%
%    Skip driver part if not needed.
%    \begin{macrocode}
\ltx@IfUndefined{reflectbox}{}{%
  \ltx@IfUndefined{rotatebox}{}{%
    \HOLOGO@AtEnd
  }%
}
\ltx@IfUndefined{pgftext}{}{%
  \HOLOGO@AtEnd
}
\ltx@IfUndefined{psscalebox}{}{%
  \HOLOGO@AtEnd
}
%    \end{macrocode}
%
%    \begin{macrocode}
\def\HOLOGO@temp{LaTeX2e}
\ifx\fmtname\HOLOGO@temp
  \RequirePackage{kvoptions}[2011/06/30]%
  \ProcessKeyvalOptions{HoLogoDriver}%
\fi
\HOLOGO@DriverSetup{}
%    \end{macrocode}
%
%    \begin{macro}{\HOLOGO@ReflectBox}
%    \begin{macrocode}
\def\HOLOGO@ReflectBox#1{%
  \begingroup
    \setbox\ltx@zero\hbox{\begingroup#1\endgroup}%
    \setbox\ltx@two\hbox{%
      \kern\wd\ltx@zero
      \csname HOLOGO@ScaleBox@\hologoDriver\endcsname{-1}{1}{%
        \hbox to 0pt{\copy\ltx@zero\hss}%
      }%
    }%
    \wd\ltx@two=\wd\ltx@zero
    \box\ltx@two
  \endgroup
}
%    \end{macrocode}
%    \end{macro}
%
%    \begin{macro}{\HOLOGO@PointReflectBox}
%    \begin{macrocode}
\def\HOLOGO@PointReflectBox#1{%
  \begingroup
    \setbox\ltx@zero\hbox{\begingroup#1\endgroup}%
    \setbox\ltx@two\hbox{%
      \kern\wd\ltx@zero
      \raise\ht\ltx@zero\hbox{%
        \csname HOLOGO@ScaleBox@\hologoDriver\endcsname{-1}{-1}{%
          \hbox to 0pt{\copy\ltx@zero\hss}%
        }%
      }%
    }%
    \wd\ltx@two=\wd\ltx@zero
    \box\ltx@two
  \endgroup
}
%    \end{macrocode}
%    \end{macro}
%
%    We must define all variants because of dynamic driver setup.
%    \begin{macrocode}
\def\HOLOGO@temp#1#2{#2}
%    \end{macrocode}
%
%    \begin{macro}{\HOLOGO@ScaleBox@pdftex}
%    \begin{macrocode}
\HOLOGO@temp{pdftex}{%
  \def\HOLOGO@ScaleBox@pdftex#1#2#3{%
    \HOLOGO@pdfliteral{%
      q #1 0 0 #2 0 0 cm%
    }%
    #3%
    \HOLOGO@pdfliteral{%
      Q%
    }%
  }%
}
%    \end{macrocode}
%    \end{macro}
%    \begin{macro}{\HOLOGO@ScaleBox@dvips}
%    \begin{macrocode}
\HOLOGO@temp{dvips}{%
  \def\HOLOGO@ScaleBox@dvips#1#2#3{%
    \special{ps:%
      gsave %
      currentpoint %
      currentpoint translate %
      #1 #2 scale %
      neg exch neg exch translate%
    }%
    #3%
    \special{ps:%
      currentpoint %
      grestore %
      moveto%
    }%
  }%
}
%    \end{macrocode}
%    \end{macro}
%    \begin{macro}{\HOLOGO@ScaleBox@dvipdfm}
%    \begin{macrocode}
\HOLOGO@temp{dvipdfm}{%
  \let\HOLOGO@ScaleBox@dvipdfm\HOLOGO@ScaleBox@dvips
}
%    \end{macrocode}
%    \end{macro}
%    Since \hologo{XeTeX} v0.6.
%    \begin{macro}{\HOLOGO@ScaleBox@xetex}
%    \begin{macrocode}
\HOLOGO@temp{xetex}{%
  \def\HOLOGO@ScaleBox@xetex#1#2#3{%
    \special{x:gsave}%
    \special{x:scale #1 #2}%
    #3%
    \special{x:grestore}%
  }%
}
%    \end{macrocode}
%    \end{macro}
%    \begin{macro}{\HOLOGO@ScaleBox@vtex}
%    \begin{macrocode}
\HOLOGO@temp{vtex}{%
  \def\HOLOGO@ScaleBox@vtex#1#2#3{%
    \special{r(#1,0,0,#2,0,0}%
    #3%
    \special{r)}%
  }%
}
%    \end{macrocode}
%    \end{macro}
%
%    \begin{macrocode}
\HOLOGO@AtEnd%
%</package>
%    \end{macrocode}
%
% \section{Test}
%
% \subsection{Catcode checks for loading}
%
%    \begin{macrocode}
%<*test1>
%    \end{macrocode}
%    \begin{macrocode}
\catcode`\{=1 %
\catcode`\}=2 %
\catcode`\#=6 %
\catcode`\@=11 %
\expandafter\ifx\csname count@\endcsname\relax
  \countdef\count@=255 %
\fi
\expandafter\ifx\csname @gobble\endcsname\relax
  \long\def\@gobble#1{}%
\fi
\expandafter\ifx\csname @firstofone\endcsname\relax
  \long\def\@firstofone#1{#1}%
\fi
\expandafter\ifx\csname loop\endcsname\relax
  \expandafter\@firstofone
\else
  \expandafter\@gobble
\fi
{%
  \def\loop#1\repeat{%
    \def\body{#1}%
    \iterate
  }%
  \def\iterate{%
    \body
      \let\next\iterate
    \else
      \let\next\relax
    \fi
    \next
  }%
  \let\repeat=\fi
}%
\def\RestoreCatcodes{}
\count@=0 %
\loop
  \edef\RestoreCatcodes{%
    \RestoreCatcodes
    \catcode\the\count@=\the\catcode\count@\relax
  }%
\ifnum\count@<255 %
  \advance\count@ 1 %
\repeat

\def\RangeCatcodeInvalid#1#2{%
  \count@=#1\relax
  \loop
    \catcode\count@=15 %
  \ifnum\count@<#2\relax
    \advance\count@ 1 %
  \repeat
}
\def\RangeCatcodeCheck#1#2#3{%
  \count@=#1\relax
  \loop
    \ifnum#3=\catcode\count@
    \else
      \errmessage{%
        Character \the\count@\space
        with wrong catcode \the\catcode\count@\space
        instead of \number#3%
      }%
    \fi
  \ifnum\count@<#2\relax
    \advance\count@ 1 %
  \repeat
}
\def\space{ }
\expandafter\ifx\csname LoadCommand\endcsname\relax
  \def\LoadCommand{\input hologo.sty\relax}%
\fi
\def\Test{%
  \RangeCatcodeInvalid{0}{47}%
  \RangeCatcodeInvalid{58}{64}%
  \RangeCatcodeInvalid{91}{96}%
  \RangeCatcodeInvalid{123}{255}%
  \catcode`\@=12 %
  \catcode`\\=0 %
  \catcode`\%=14 %
  \LoadCommand
  \RangeCatcodeCheck{0}{36}{15}%
  \RangeCatcodeCheck{37}{37}{14}%
  \RangeCatcodeCheck{38}{47}{15}%
  \RangeCatcodeCheck{48}{57}{12}%
  \RangeCatcodeCheck{58}{63}{15}%
  \RangeCatcodeCheck{64}{64}{12}%
  \RangeCatcodeCheck{65}{90}{11}%
  \RangeCatcodeCheck{91}{91}{15}%
  \RangeCatcodeCheck{92}{92}{0}%
  \RangeCatcodeCheck{93}{96}{15}%
  \RangeCatcodeCheck{97}{122}{11}%
  \RangeCatcodeCheck{123}{255}{15}%
  \RestoreCatcodes
}
\Test
\csname @@end\endcsname
\end
%    \end{macrocode}
%    \begin{macrocode}
%</test1>
%    \end{macrocode}
%
% \subsection{Spacefactor}
%
%    The space factor must be 1000 after a logo. If it is greater 1000
%    then the following space is a space after a sentence closing point.
%    If the space factor is smaller 1000 then an immediate following
%    dot is interpreted as abbreviation, not sentence closing point.
%
%    \begin{macrocode}
%<*test-spacefactor>
\NeedsTeXFormat{LaTeX2e}
\documentclass{article}
\usepackage{hologo}[2016/05/12]
\usepackage{kvsetkeys}
\usepackage{qstest}
\IncludeTests{*}
\LogTests{log}{*}{*}
\begin{document}
\begin{qstest}{spacefactor}{spacefactor}
\newcommand*{\Test}[1]{%
  \sbox0{%
    \hologo{#1}%
    \Expect*{1000 (#1)}*{\the\spacefactor\space(#1)}%
  }%
}%
\makeatletter
\def\TestList{}
\def\hologoEntry#1#2#3{%
  \edef\TestList{%
    \ifx\TestList\@empty
    \else
      \TestList,%
    \fi
    #1%
    \ifx\\#2\\%
    \else
      ={variant=#2}%
    \fi
  }%
}
\hologoList
\expandafter\kv@parse@normalized\expandafter{%
  \TestList
}{%
  \begingroup
    \let\@logo=\kv@key
    \ifx\kv@value\relax
    \else
      \expandafter\hologoLogoSetup\expandafter\@logo\expandafter{%
        \kv@value
      }%
    \fi
    \Test\@logo
  \endgroup
  \@gobbletwo
}
\end{qstest}
\end{document}
%</test-spacefactor>
%    \end{macrocode}
%
% \subsection{Complete list}
%
%    \begin{macrocode}
%<*test-list>
\NeedsTeXFormat{LaTeX2e}
\documentclass[12pt,a4paper]{article}
\usepackage{hologo}[2016/05/12]
\usepackage[T1]{fontenc}
\usepackage{lmodern}
\usepackage{parskip}
\usepackage[unicode]{hyperref}[2011/09/28]
\usepackage{bookmark}[2011/09/19]
\bookmarksetup{%
  numbered,%
  open,%
  openlevel=2,%
}
\renewcommand*{\contentsname}{List of logos}
\begin{document}
\tableofcontents
\def\TestFont#1#2#3#4#5#6{%
  \begingroup
    \usefont{#3}{#4}{#5}{#6}%
    \HologoVariant{#1}{#2}/\hologoVariant{#1}{#2}%
    \quad
    \begingroup\scriptsize\hologoVariant{#1}{#2}\endgroup
    \quad
  \endgroup
  (#3/#4/#5/#6)%
  \par
}
\makeatletter
\def\hologoEntry#1#2#3{%
  \section{%
    \HologoVariant{#1}{#2}/\hologoVariant{#1}{#2} %
    {[#1\ifx\\#2\\\else\space(#2)\fi]}% hash-ok
  }% braces around [] because of bug in tex4ht
  \begingroup
    \hypersetup{unicode=false}%
    \bookmark[%
      dest=\@currentHref,%
      rellevel=1,%
      keeplevel,%
    ]{%
      \HologoVariant{#1}{#2}/\hologoVariant{#1}{#2} %
      (PDFDocEncoding)%
    }%
  \endgroup
  \TestFont{#1}{#2}{OT1}{cmr}{m}{n}%
  \TestFont{#1}{#2}{OT1}{cmss}{m}{n}%
  \TestFont{#1}{#2}{OT1}{cmr}{b}{n}%
  \TestFont{#1}{#2}{OT1}{cmr}{m}{it}%
  \TestFont{#1}{#2}{OT1}{cmtt}{m}{n}%
  \TestFont{#1}{#2}{T1}{lmr}{m}{n}%
  \TestFont{#1}{#2}{T1}{lmss}{m}{n}%
  \TestFont{#1}{#2}{T1}{lmr}{b}{n}%
  \TestFont{#1}{#2}{T1}{lmr}{m}{it}%
  \TestFont{#1}{#2}{T1}{lmtt}{m}{n}%
  \TestFont{#1}{#2}{T1}{lmvtt}{m}{n}%
  \TestFont{#1}{#2}{T1}{qtm}{m}{n}%
  \TestFont{#1}{#2}{T1}{qhv}{m}{n}%
  \TestFont{#1}{#2}{T1}{qtm}{b}{n}%
  \TestFont{#1}{#2}{T1}{qtm}{m}{it}%
  \TestFont{#1}{#2}{T1}{qcr}{m}{n}%
  \newpage
}
\makeatother
\hologoList
\end{document}
%</test-list>
%    \end{macrocode}
%
% \section{Installation}
%
% \subsection{Download}
%
% \paragraph{Package.} This package is available on
% CTAN\footnote{\url{ftp://ftp.ctan.org/tex-archive/}}:
% \begin{description}
% \item[\CTAN{macros/latex/contrib/oberdiek/hologo.dtx}] The source file.
% \item[\CTAN{macros/latex/contrib/oberdiek/hologo.pdf}] Documentation.
% \end{description}
%
%
% \paragraph{Bundle.} All the packages of the bundle `oberdiek'
% are also available in a TDS compliant ZIP archive. There
% the packages are already unpacked and the documentation files
% are generated. The files and directories obey the TDS standard.
% \begin{description}
% \item[\CTAN{install/macros/latex/contrib/oberdiek.tds.zip}]
% \end{description}
% \emph{TDS} refers to the standard ``A Directory Structure
% for \TeX\ Files'' (\CTAN{tds/tds.pdf}). Directories
% with \xfile{texmf} in their name are usually organized this way.
%
% \subsection{Bundle installation}
%
% \paragraph{Unpacking.} Unpack the \xfile{oberdiek.tds.zip} in the
% TDS tree (also known as \xfile{texmf} tree) of your choice.
% Example (linux):
% \begin{quote}
%   |unzip oberdiek.tds.zip -d ~/texmf|
% \end{quote}
%
% \paragraph{Script installation.}
% Check the directory \xfile{TDS:scripts/oberdiek/} for
% scripts that need further installation steps.
% Package \xpackage{attachfile2} comes with the Perl script
% \xfile{pdfatfi.pl} that should be installed in such a way
% that it can be called as \texttt{pdfatfi}.
% Example (linux):
% \begin{quote}
%   |chmod +x scripts/oberdiek/pdfatfi.pl|\\
%   |cp scripts/oberdiek/pdfatfi.pl /usr/local/bin/|
% \end{quote}
%
% \subsection{Package installation}
%
% \paragraph{Unpacking.} The \xfile{.dtx} file is a self-extracting
% \docstrip\ archive. The files are extracted by running the
% \xfile{.dtx} through \plainTeX:
% \begin{quote}
%   \verb|tex hologo.dtx|
% \end{quote}
%
% \paragraph{TDS.} Now the different files must be moved into
% the different directories in your installation TDS tree
% (also known as \xfile{texmf} tree):
% \begin{quote}
% \def\t{^^A
% \begin{tabular}{@{}>{\ttfamily}l@{ $\rightarrow$ }>{\ttfamily}l@{}}
%   hologo.sty & tex/generic/oberdiek/hologo.sty\\
%   hologo.pdf & doc/latex/oberdiek/hologo.pdf\\
%   example/hologo-example.tex & doc/latex/oberdiek/example/hologo-example.tex\\
%   test/hologo-test1.tex & doc/latex/oberdiek/test/hologo-test1.tex\\
%   test/hologo-test-spacefactor.tex & doc/latex/oberdiek/test/hologo-test-spacefactor.tex\\
%   test/hologo-test-list.tex & doc/latex/oberdiek/test/hologo-test-list.tex\\
%   hologo.dtx & source/latex/oberdiek/hologo.dtx\\
% \end{tabular}^^A
% }^^A
% \sbox0{\t}^^A
% \ifdim\wd0>\linewidth
%   \begingroup
%     \advance\linewidth by\leftmargin
%     \advance\linewidth by\rightmargin
%   \edef\x{\endgroup
%     \def\noexpand\lw{\the\linewidth}^^A
%   }\x
%   \def\lwbox{^^A
%     \leavevmode
%     \hbox to \linewidth{^^A
%       \kern-\leftmargin\relax
%       \hss
%       \usebox0
%       \hss
%       \kern-\rightmargin\relax
%     }^^A
%   }^^A
%   \ifdim\wd0>\lw
%     \sbox0{\small\t}^^A
%     \ifdim\wd0>\linewidth
%       \ifdim\wd0>\lw
%         \sbox0{\footnotesize\t}^^A
%         \ifdim\wd0>\linewidth
%           \ifdim\wd0>\lw
%             \sbox0{\scriptsize\t}^^A
%             \ifdim\wd0>\linewidth
%               \ifdim\wd0>\lw
%                 \sbox0{\tiny\t}^^A
%                 \ifdim\wd0>\linewidth
%                   \lwbox
%                 \else
%                   \usebox0
%                 \fi
%               \else
%                 \lwbox
%               \fi
%             \else
%               \usebox0
%             \fi
%           \else
%             \lwbox
%           \fi
%         \else
%           \usebox0
%         \fi
%       \else
%         \lwbox
%       \fi
%     \else
%       \usebox0
%     \fi
%   \else
%     \lwbox
%   \fi
% \else
%   \usebox0
% \fi
% \end{quote}
% If you have a \xfile{docstrip.cfg} that configures and enables \docstrip's
% TDS installing feature, then some files can already be in the right
% place, see the documentation of \docstrip.
%
% \subsection{Refresh file name databases}
%
% If your \TeX~distribution
% (\teTeX, \mikTeX, \dots) relies on file name databases, you must refresh
% these. For example, \teTeX\ users run \verb|texhash| or
% \verb|mktexlsr|.
%
% \subsection{Some details for the interested}
%
% \paragraph{Attached source.}
%
% The PDF documentation on CTAN also includes the
% \xfile{.dtx} source file. It can be extracted by
% AcrobatReader 6 or higher. Another option is \textsf{pdftk},
% e.g. unpack the file into the current directory:
% \begin{quote}
%   \verb|pdftk hologo.pdf unpack_files output .|
% \end{quote}
%
% \paragraph{Unpacking with \LaTeX.}
% The \xfile{.dtx} chooses its action depending on the format:
% \begin{description}
% \item[\plainTeX:] Run \docstrip\ and extract the files.
% \item[\LaTeX:] Generate the documentation.
% \end{description}
% If you insist on using \LaTeX\ for \docstrip\ (really,
% \docstrip\ does not need \LaTeX), then inform the autodetect routine
% about your intention:
% \begin{quote}
%   \verb|latex \let\install=y\input{hologo.dtx}|
% \end{quote}
% Do not forget to quote the argument according to the demands
% of your shell.
%
% \paragraph{Generating the documentation.}
% You can use both the \xfile{.dtx} or the \xfile{.drv} to generate
% the documentation. The process can be configured by the
% configuration file \xfile{ltxdoc.cfg}. For instance, put this
% line into this file, if you want to have A4 as paper format:
% \begin{quote}
%   \verb|\PassOptionsToClass{a4paper}{article}|
% \end{quote}
% An example follows how to generate the
% documentation with pdf\LaTeX:
% \begin{quote}
%\begin{verbatim}
%pdflatex hologo.dtx
%makeindex -s gind.ist hologo.idx
%pdflatex hologo.dtx
%makeindex -s gind.ist hologo.idx
%pdflatex hologo.dtx
%\end{verbatim}
% \end{quote}
%
% \section{Catalogue}
%
% The following XML file can be used as source for the
% \href{http://mirror.ctan.org/help/Catalogue/catalogue.html}{\TeX\ Catalogue}.
% The elements \texttt{caption} and \texttt{description} are imported
% from the original XML file from the Catalogue.
% The name of the XML file in the Catalogue is \xfile{hologo.xml}.
%    \begin{macrocode}
%<*catalogue>
<?xml version='1.0' encoding='us-ascii'?>
<!DOCTYPE entry SYSTEM 'catalogue.dtd'>
<entry datestamp='$Date$' modifier='$Author$' id='hologo'>
  <name>hologo</name>
  <caption>A collection of logos with bookmark support.</caption>
  <authorref id='auth:oberdiek'/>
  <copyright owner='Heiko Oberdiek' year='2010-2012'/>
  <license type='lppl1.3'/>
  <version number='1.10'/>
  <description>
    The package defines a single command <tt>\hologo</tt>, whose
    argument is the usual case-confused ASCII version of the logo.
    The command is bookmark-enabled, so that every logo becomes
    available in bookmarks without further work.
    <p/>
    The package is part of the <xref refid='oberdiek'>oberdiek</xref>
    bundle.
  </description>
  <documentation details='Package documentation'
      href='ctan:/macros/latex/contrib/oberdiek/hologo.pdf'/>
  <ctan file='true' path='/macros/latex/contrib/oberdiek/hologo.dtx'/>
  <miktex location='oberdiek'/>
  <texlive location='oberdiek'/>
  <install path='/macros/latex/contrib/oberdiek/oberdiek.tds.zip'/>
</entry>
%</catalogue>
%    \end{macrocode}
%
% \begin{thebibliography}{9}
% \raggedright
%
% \bibitem{btxdoc}
% Oren Patashnik,
% \textit{\hologo{BibTeX}ing},
% 1988-02-08.\\
% \CTAN{biblio/bibtex/base/}
%
% \bibitem{dtklogos}
% Gerd Neugebauer, DANTE,
% \textit{Package \xpackage{dtklogos}},
% 2011-04-25.\\
% \CTAN{usergrps/dante/dtk/dtklogos.sty}
%
% \bibitem{etexman}
% The \hologo{NTS} Team,
% \textit{The \hologo{eTeX} manual},
% 1998-02.\\
% \CTAN{systems/e-tex/v2/doc/}
%
% \bibitem{ExTeX-FAQ}
% The \hologo{ExTeX} group,
% \textit{\hologo{ExTeX}: FAQ -- How is \hologo{ExTeX} typeset?},
% 2007-04-14.\\
% \url{http://www.extex.org/documentation/faq.html}
%
% \bibitem{LyX}
% %@MISC{ LyX,
% %  title = {{LyX 2.0.0 -- The Document Processor [Computer software and manual]}},
% %  author = {{The LyX Team}},
% %  howpublished = {Internet: http://www.lyx.org},
% %  year = {2011-05-08},
% %  note = {Retrieved May 10, 2011, from http://www.lyx.org},
% %  url = {http://www.lyx.org/}
% %}
% The \hologo{LyX} Team,
% \textit{\hologo{LyX} -- The Document Processor},
% 2011-05-08.\\
% \url{http://www.lyx.org/}
%
% \bibitem{OzTeX}
% Andrew Trevorrow,
% \hologo{OzTeX} FAQ: What is the correct way to typeset ``\hologo{OzTeX}''?,
% 2011-09-15 (visited).
% \url{http://www.trevorrow.com/oztex/ozfaq.html#oztex-logo}
%
% \bibitem{PiCTeX}
% Michael Wichura,
% \textit{The \hologo{PiCTeX} macro package},
% 1987-09-21.
% \CTAN{graphics/pictex/}
%
% \bibitem{scrlogo}
% Markus Kohm,
% \textit{\hologo{KOMAScript} Datei \xfile{scrlogo.dtx}},
% 2009-01-30.\\
% \CTAN{install/macros/latex/contrib/komascript.tds.zip}
%
% \end{thebibliography}
%
% \begin{History}
%   \begin{Version}{2010/04/08 v1.0}
%   \item
%     The first version.
%   \end{Version}
%   \begin{Version}{2010/04/16 v1.1}
%   \item
%     \cs{Hologo} added for support of logos at start of a sentence.
%   \item
%     \cs{hologoSetup} and \cs{hologoLogoSetup} added.
%   \item
%     Options \xoption{break}, \xoption{hyphenbreak}, \xoption{spacebreak}
%     added.
%   \item
%     Variant support added by option \xoption{variant}.
%   \end{Version}
%   \begin{Version}{2010/04/24 v1.2}
%   \item
%     \hologo{LaTeX3} added.
%   \item
%     \hologo{VTeX} added.
%   \end{Version}
%   \begin{Version}{2010/11/21 v1.3}
%   \item
%     \hologo{iniTeX}, \hologo{virTeX} added.
%   \end{Version}
%   \begin{Version}{2011/03/25 v1.4}
%   \item
%     \hologo{ConTeXt} with variants added.
%   \item
%     Option \xoption{discretionarybreak} added as refinement for
%     option \xoption{break}.
%   \end{Version}
%   \begin{Version}{2011/04/21 v1.5}
%   \item
%     Wrong TDS directory for test files fixed.
%   \end{Version}
%   \begin{Version}{2011/10/01 v1.6}
%   \item
%     Support for package \xpackage{tex4ht} added.
%   \item
%     Support for \cs{csname} added if \cs{ifincsname} is available.
%   \item
%     New logos:
%     \hologo{(La)TeX},
%     \hologo{biber},
%     \hologo{BibTeX} (\xoption{sc}, \xoption{sf}),
%     \hologo{emTeX},
%     \hologo{ExTeX},
%     \hologo{KOMAScript},
%     \hologo{La},
%     \hologo{LyX},
%     \hologo{MiKTeX},
%     \hologo{NTS},
%     \hologo{OzMF},
%     \hologo{OzMP},
%     \hologo{OzTeX},
%     \hologo{OzTtH},
%     \hologo{PCTeX},
%     \hologo{PiC},
%     \hologo{PiCTeX},
%     \hologo{METAFONT},
%     \hologo{MetaFun},
%     \hologo{METAPOST},
%     \hologo{MetaPost},
%     \hologo{SLiTeX} (\xoption{lift}, \xoption{narrow}, \xoption{simple}),
%     \hologo{SliTeX} (\xoption{narrow}, \xoption{simple}, \xoption{lift}),
%     \hologo{teTeX}.
%   \item
%     Fixes:
%     \hologo{iniTeX},
%     \hologo{pdfLaTeX},
%     \hologo{pdfTeX},
%     \hologo{virTeX}.
%   \item
%     \cs{hologoFontSetup} and \cs{hologoLogoFontSetup} added.
%   \item
%     \cs{hologoVariant} and \cs{HologoVariant} added.
%   \end{Version}
%   \begin{Version}{2011/11/22 v1.7}
%   \item
%     New logos:
%     \hologo{BibTeX8},
%     \hologo{LaTeXML},
%     \hologo{SageTeX},
%     \hologo{TeX4ht},
%     \hologo{TTH}.
%   \item
%     \hologo{Xe} and friends: Driver stuff fixed.
%   \item
%     \hologo{Xe} and friends: Support for italic added.
%   \item
%     \hologo{Xe} and friends: Package support for \xpackage{pgf}
%     and \xpackage{pstricks} added.
%   \end{Version}
%   \begin{Version}{2011/11/29 v1.8}
%   \item
%     New logos:
%     \hologo{HanTheThanh}.
%   \end{Version}
%   \begin{Version}{2011/12/21 v1.9}
%   \item
%     Patch for package \xpackage{ifxetex} added for the case that
%     \cs{newif} is undefined in \hologo{iniTeX}.
%   \item
%     Some fixes for \hologo{iniTeX}.
%   \end{Version}
%   \begin{Version}{2012/04/26 v1.10}
%   \item
%     Fix in bookmark version of logo ``\hologo{HanTheThanh}''.
%   \end{Version}
%   \begin{Version}{2016/05/12 v1.11}
%   \item
%     Update HOLOGO@IfCharExists (previously in texlive)
%   \item define pdfliteral in current luatex.
%   \end{Version}
% \end{History}
%
% \PrintIndex
%
% \Finale
\endinput
|
% \end{quote}
% Do not forget to quote the argument according to the demands
% of your shell.
%
% \paragraph{Generating the documentation.}
% You can use both the \xfile{.dtx} or the \xfile{.drv} to generate
% the documentation. The process can be configured by the
% configuration file \xfile{ltxdoc.cfg}. For instance, put this
% line into this file, if you want to have A4 as paper format:
% \begin{quote}
%   \verb|\PassOptionsToClass{a4paper}{article}|
% \end{quote}
% An example follows how to generate the
% documentation with pdf\LaTeX:
% \begin{quote}
%\begin{verbatim}
%pdflatex hologo.dtx
%makeindex -s gind.ist hologo.idx
%pdflatex hologo.dtx
%makeindex -s gind.ist hologo.idx
%pdflatex hologo.dtx
%\end{verbatim}
% \end{quote}
%
% \section{Catalogue}
%
% The following XML file can be used as source for the
% \href{http://mirror.ctan.org/help/Catalogue/catalogue.html}{\TeX\ Catalogue}.
% The elements \texttt{caption} and \texttt{description} are imported
% from the original XML file from the Catalogue.
% The name of the XML file in the Catalogue is \xfile{hologo.xml}.
%    \begin{macrocode}
%<*catalogue>
<?xml version='1.0' encoding='us-ascii'?>
<!DOCTYPE entry SYSTEM 'catalogue.dtd'>
<entry datestamp='$Date$' modifier='$Author$' id='hologo'>
  <name>hologo</name>
  <caption>A collection of logos with bookmark support.</caption>
  <authorref id='auth:oberdiek'/>
  <copyright owner='Heiko Oberdiek' year='2010-2012'/>
  <license type='lppl1.3'/>
  <version number='1.10'/>
  <description>
    The package defines a single command <tt>\hologo</tt>, whose
    argument is the usual case-confused ASCII version of the logo.
    The command is bookmark-enabled, so that every logo becomes
    available in bookmarks without further work.
    <p/>
    The package is part of the <xref refid='oberdiek'>oberdiek</xref>
    bundle.
  </description>
  <documentation details='Package documentation'
      href='ctan:/macros/latex/contrib/oberdiek/hologo.pdf'/>
  <ctan file='true' path='/macros/latex/contrib/oberdiek/hologo.dtx'/>
  <miktex location='oberdiek'/>
  <texlive location='oberdiek'/>
  <install path='/macros/latex/contrib/oberdiek/oberdiek.tds.zip'/>
</entry>
%</catalogue>
%    \end{macrocode}
%
% \begin{thebibliography}{9}
% \raggedright
%
% \bibitem{btxdoc}
% Oren Patashnik,
% \textit{\hologo{BibTeX}ing},
% 1988-02-08.\\
% \CTAN{biblio/bibtex/base/}
%
% \bibitem{dtklogos}
% Gerd Neugebauer, DANTE,
% \textit{Package \xpackage{dtklogos}},
% 2011-04-25.\\
% \CTAN{usergrps/dante/dtk/dtklogos.sty}
%
% \bibitem{etexman}
% The \hologo{NTS} Team,
% \textit{The \hologo{eTeX} manual},
% 1998-02.\\
% \CTAN{systems/e-tex/v2/doc/}
%
% \bibitem{ExTeX-FAQ}
% The \hologo{ExTeX} group,
% \textit{\hologo{ExTeX}: FAQ -- How is \hologo{ExTeX} typeset?},
% 2007-04-14.\\
% \url{http://www.extex.org/documentation/faq.html}
%
% \bibitem{LyX}
% %@MISC{ LyX,
% %  title = {{LyX 2.0.0 -- The Document Processor [Computer software and manual]}},
% %  author = {{The LyX Team}},
% %  howpublished = {Internet: http://www.lyx.org},
% %  year = {2011-05-08},
% %  note = {Retrieved May 10, 2011, from http://www.lyx.org},
% %  url = {http://www.lyx.org/}
% %}
% The \hologo{LyX} Team,
% \textit{\hologo{LyX} -- The Document Processor},
% 2011-05-08.\\
% \url{http://www.lyx.org/}
%
% \bibitem{OzTeX}
% Andrew Trevorrow,
% \hologo{OzTeX} FAQ: What is the correct way to typeset ``\hologo{OzTeX}''?,
% 2011-09-15 (visited).
% \url{http://www.trevorrow.com/oztex/ozfaq.html#oztex-logo}
%
% \bibitem{PiCTeX}
% Michael Wichura,
% \textit{The \hologo{PiCTeX} macro package},
% 1987-09-21.
% \CTAN{graphics/pictex/}
%
% \bibitem{scrlogo}
% Markus Kohm,
% \textit{\hologo{KOMAScript} Datei \xfile{scrlogo.dtx}},
% 2009-01-30.\\
% \CTAN{install/macros/latex/contrib/komascript.tds.zip}
%
% \end{thebibliography}
%
% \begin{History}
%   \begin{Version}{2010/04/08 v1.0}
%   \item
%     The first version.
%   \end{Version}
%   \begin{Version}{2010/04/16 v1.1}
%   \item
%     \cs{Hologo} added for support of logos at start of a sentence.
%   \item
%     \cs{hologoSetup} and \cs{hologoLogoSetup} added.
%   \item
%     Options \xoption{break}, \xoption{hyphenbreak}, \xoption{spacebreak}
%     added.
%   \item
%     Variant support added by option \xoption{variant}.
%   \end{Version}
%   \begin{Version}{2010/04/24 v1.2}
%   \item
%     \hologo{LaTeX3} added.
%   \item
%     \hologo{VTeX} added.
%   \end{Version}
%   \begin{Version}{2010/11/21 v1.3}
%   \item
%     \hologo{iniTeX}, \hologo{virTeX} added.
%   \end{Version}
%   \begin{Version}{2011/03/25 v1.4}
%   \item
%     \hologo{ConTeXt} with variants added.
%   \item
%     Option \xoption{discretionarybreak} added as refinement for
%     option \xoption{break}.
%   \end{Version}
%   \begin{Version}{2011/04/21 v1.5}
%   \item
%     Wrong TDS directory for test files fixed.
%   \end{Version}
%   \begin{Version}{2011/10/01 v1.6}
%   \item
%     Support for package \xpackage{tex4ht} added.
%   \item
%     Support for \cs{csname} added if \cs{ifincsname} is available.
%   \item
%     New logos:
%     \hologo{(La)TeX},
%     \hologo{biber},
%     \hologo{BibTeX} (\xoption{sc}, \xoption{sf}),
%     \hologo{emTeX},
%     \hologo{ExTeX},
%     \hologo{KOMAScript},
%     \hologo{La},
%     \hologo{LyX},
%     \hologo{MiKTeX},
%     \hologo{NTS},
%     \hologo{OzMF},
%     \hologo{OzMP},
%     \hologo{OzTeX},
%     \hologo{OzTtH},
%     \hologo{PCTeX},
%     \hologo{PiC},
%     \hologo{PiCTeX},
%     \hologo{METAFONT},
%     \hologo{MetaFun},
%     \hologo{METAPOST},
%     \hologo{MetaPost},
%     \hologo{SLiTeX} (\xoption{lift}, \xoption{narrow}, \xoption{simple}),
%     \hologo{SliTeX} (\xoption{narrow}, \xoption{simple}, \xoption{lift}),
%     \hologo{teTeX}.
%   \item
%     Fixes:
%     \hologo{iniTeX},
%     \hologo{pdfLaTeX},
%     \hologo{pdfTeX},
%     \hologo{virTeX}.
%   \item
%     \cs{hologoFontSetup} and \cs{hologoLogoFontSetup} added.
%   \item
%     \cs{hologoVariant} and \cs{HologoVariant} added.
%   \end{Version}
%   \begin{Version}{2011/11/22 v1.7}
%   \item
%     New logos:
%     \hologo{BibTeX8},
%     \hologo{LaTeXML},
%     \hologo{SageTeX},
%     \hologo{TeX4ht},
%     \hologo{TTH}.
%   \item
%     \hologo{Xe} and friends: Driver stuff fixed.
%   \item
%     \hologo{Xe} and friends: Support for italic added.
%   \item
%     \hologo{Xe} and friends: Package support for \xpackage{pgf}
%     and \xpackage{pstricks} added.
%   \end{Version}
%   \begin{Version}{2011/11/29 v1.8}
%   \item
%     New logos:
%     \hologo{HanTheThanh}.
%   \end{Version}
%   \begin{Version}{2011/12/21 v1.9}
%   \item
%     Patch for package \xpackage{ifxetex} added for the case that
%     \cs{newif} is undefined in \hologo{iniTeX}.
%   \item
%     Some fixes for \hologo{iniTeX}.
%   \end{Version}
%   \begin{Version}{2012/04/26 v1.10}
%   \item
%     Fix in bookmark version of logo ``\hologo{HanTheThanh}''.
%   \end{Version}
%   \begin{Version}{2016/05/12 v1.11}
%   \item
%     Update HOLOGO@IfCharExists (previously in texlive)
%   \item define pdfliteral in current luatex.
%   \end{Version}
% \end{History}
%
% \PrintIndex
%
% \Finale
\endinput
%
        \else
          \input hologo.cfg\relax
        \fi
      \else
        \@PackageInfoNoLine{hologo}{%
          Empty configuration file `hologo.cfg' ignored%
        }%
      \fi
    \fi
  }%
}
%    \end{macrocode}
%
%    \begin{macrocode}
\def\HOLOGO@temp#1#2{%
  \kv@define@key{HoLogoDriver}{#1}[]{%
    \begingroup
      \def\HOLOGO@temp{##1}%
      \ltx@onelevel@sanitize\HOLOGO@temp
      \ifx\HOLOGO@temp\ltx@empty
      \else
        \@PackageError{hologo}{%
          Value (\HOLOGO@temp) not permitted for option `#1'%
        }%
        \@ehc
      \fi
    \endgroup
    \def\hologoDriver{#2}%
  }%
}%
\def\HOLOGO@@temp#1#2{%
  \ifx\kv@value\relax
    \HOLOGO@temp{#1}{#1}%
  \else
    \HOLOGO@temp{#1}{#2}%
  \fi
}%
\kv@parse@normalized{%
  pdftex,%
  luatex=pdftex,%
  dvipdfm,%
  dvipdfmx=dvipdfm,%
  dvips,%
  dvipsone=dvips,%
  xdvi=dvips,%
  xetex,%
  vtex,%
}\HOLOGO@@temp
%    \end{macrocode}
%
%    \begin{macrocode}
\kv@define@key{HoLogoDriver}{driverfallback}{%
  \def\HOLOGO@DriverFallback{#1}%
}
%    \end{macrocode}
%
%    \begin{macro}{\HOLOGO@DriverFallback}
%    \begin{macrocode}
\def\HOLOGO@DriverFallback{dvips}
%    \end{macrocode}
%    \end{macro}
%
%    \begin{macro}{\hologoDriverSetup}
%    \begin{macrocode}
\def\hologoDriverSetup{%
  \let\hologoDriver\ltx@undefined
  \HOLOGO@DriverSetup
}
%    \end{macrocode}
%    \end{macro}
%
%    \begin{macro}{\HOLOGO@DriverSetup}
%    \begin{macrocode}
\def\HOLOGO@DriverSetup#1{%
  \kvsetkeys{HoLogoDriver}{#1}%
  \HOLOGO@CheckDriver
  \ltx@ifundefined{hologoDriver}{%
    \begingroup
    \edef\x{\endgroup
      \noexpand\kvsetkeys{HoLogoDriver}{\HOLOGO@DriverFallback}%
    }\x
  }{}%
  \@PackageInfoNoLine{hologo}{Using driver `\hologoDriver'}%
}
%    \end{macrocode}
%    \end{macro}
%
%    \begin{macro}{\HOLOGO@CheckDriver}
%    \begin{macrocode}
\def\HOLOGO@CheckDriver{%
  \ifpdf
    \def\hologoDriver{pdftex}%
    \let\HOLOGO@pdfliteral\pdfliteral
    \ifluatex
      \ifx\pdfextension\@undefined\else
        \protected\def\pdfliteral{\pdfextension literal}%
        \let\HOLOGO@pdfliteral\pdfliteral
      \fi
      \ltx@IfUndefined{HOLOGO@pdfliteral}{%
        \ifnum\luatexversion<36 %
        \else
          \begingroup
            \let\HOLOGO@temp\endgroup
            \ifcase0%
                \directlua{%
                  if tex.enableprimitives then %
                    tex.enableprimitives('HOLOGO@', {'pdfliteral'})%
                  else %
                    tex.print('1')%
                  end%
                }%
                \ifx\HOLOGO@pdfliteral\@undefined 1\fi%
                \relax%
              \endgroup
              \let\HOLOGO@temp\relax
              \global\let\HOLOGO@pdfliteral\HOLOGO@pdfliteral
            \fi%
          \HOLOGO@temp
        \fi
      }{}%
    \fi
    \ltx@IfUndefined{HOLOGO@pdfliteral}{%
      \@PackageWarningNoLine{hologo}{%
        Cannot find \string\pdfliteral
      }%
    }{}%
  \else
    \ifxetex
      \def\hologoDriver{xetex}%
    \else
      \ifvtex
        \def\hologoDriver{vtex}%
      \fi
    \fi
  \fi
}
%    \end{macrocode}
%    \end{macro}
%
%    \begin{macro}{\HOLOGO@WarningUnsupportedDriver}
%    \begin{macrocode}
\def\HOLOGO@WarningUnsupportedDriver#1{%
  \@PackageWarningNoLine{hologo}{%
    Logo `#1' needs driver specific macros,\MessageBreak
    but driver `\hologoDriver' is not supported.\MessageBreak
    Use a different driver or\MessageBreak
    load package `graphics' or `pgf'%
  }%
}
%    \end{macrocode}
%    \end{macro}
%
% \subsubsection{Reflect box macros}
%
%    Skip driver part if not needed.
%    \begin{macrocode}
\ltx@IfUndefined{reflectbox}{}{%
  \ltx@IfUndefined{rotatebox}{}{%
    \HOLOGO@AtEnd
  }%
}
\ltx@IfUndefined{pgftext}{}{%
  \HOLOGO@AtEnd
}
\ltx@IfUndefined{psscalebox}{}{%
  \HOLOGO@AtEnd
}
%    \end{macrocode}
%
%    \begin{macrocode}
\def\HOLOGO@temp{LaTeX2e}
\ifx\fmtname\HOLOGO@temp
  \RequirePackage{kvoptions}[2011/06/30]%
  \ProcessKeyvalOptions{HoLogoDriver}%
\fi
\HOLOGO@DriverSetup{}
%    \end{macrocode}
%
%    \begin{macro}{\HOLOGO@ReflectBox}
%    \begin{macrocode}
\def\HOLOGO@ReflectBox#1{%
  \begingroup
    \setbox\ltx@zero\hbox{\begingroup#1\endgroup}%
    \setbox\ltx@two\hbox{%
      \kern\wd\ltx@zero
      \csname HOLOGO@ScaleBox@\hologoDriver\endcsname{-1}{1}{%
        \hbox to 0pt{\copy\ltx@zero\hss}%
      }%
    }%
    \wd\ltx@two=\wd\ltx@zero
    \box\ltx@two
  \endgroup
}
%    \end{macrocode}
%    \end{macro}
%
%    \begin{macro}{\HOLOGO@PointReflectBox}
%    \begin{macrocode}
\def\HOLOGO@PointReflectBox#1{%
  \begingroup
    \setbox\ltx@zero\hbox{\begingroup#1\endgroup}%
    \setbox\ltx@two\hbox{%
      \kern\wd\ltx@zero
      \raise\ht\ltx@zero\hbox{%
        \csname HOLOGO@ScaleBox@\hologoDriver\endcsname{-1}{-1}{%
          \hbox to 0pt{\copy\ltx@zero\hss}%
        }%
      }%
    }%
    \wd\ltx@two=\wd\ltx@zero
    \box\ltx@two
  \endgroup
}
%    \end{macrocode}
%    \end{macro}
%
%    We must define all variants because of dynamic driver setup.
%    \begin{macrocode}
\def\HOLOGO@temp#1#2{#2}
%    \end{macrocode}
%
%    \begin{macro}{\HOLOGO@ScaleBox@pdftex}
%    \begin{macrocode}
\HOLOGO@temp{pdftex}{%
  \def\HOLOGO@ScaleBox@pdftex#1#2#3{%
    \HOLOGO@pdfliteral{%
      q #1 0 0 #2 0 0 cm%
    }%
    #3%
    \HOLOGO@pdfliteral{%
      Q%
    }%
  }%
}
%    \end{macrocode}
%    \end{macro}
%    \begin{macro}{\HOLOGO@ScaleBox@dvips}
%    \begin{macrocode}
\HOLOGO@temp{dvips}{%
  \def\HOLOGO@ScaleBox@dvips#1#2#3{%
    \special{ps:%
      gsave %
      currentpoint %
      currentpoint translate %
      #1 #2 scale %
      neg exch neg exch translate%
    }%
    #3%
    \special{ps:%
      currentpoint %
      grestore %
      moveto%
    }%
  }%
}
%    \end{macrocode}
%    \end{macro}
%    \begin{macro}{\HOLOGO@ScaleBox@dvipdfm}
%    \begin{macrocode}
\HOLOGO@temp{dvipdfm}{%
  \let\HOLOGO@ScaleBox@dvipdfm\HOLOGO@ScaleBox@dvips
}
%    \end{macrocode}
%    \end{macro}
%    Since \hologo{XeTeX} v0.6.
%    \begin{macro}{\HOLOGO@ScaleBox@xetex}
%    \begin{macrocode}
\HOLOGO@temp{xetex}{%
  \def\HOLOGO@ScaleBox@xetex#1#2#3{%
    \special{x:gsave}%
    \special{x:scale #1 #2}%
    #3%
    \special{x:grestore}%
  }%
}
%    \end{macrocode}
%    \end{macro}
%    \begin{macro}{\HOLOGO@ScaleBox@vtex}
%    \begin{macrocode}
\HOLOGO@temp{vtex}{%
  \def\HOLOGO@ScaleBox@vtex#1#2#3{%
    \special{r(#1,0,0,#2,0,0}%
    #3%
    \special{r)}%
  }%
}
%    \end{macrocode}
%    \end{macro}
%
%    \begin{macrocode}
\HOLOGO@AtEnd%
%</package>
%    \end{macrocode}
%
% \section{Test}
%
% \subsection{Catcode checks for loading}
%
%    \begin{macrocode}
%<*test1>
%    \end{macrocode}
%    \begin{macrocode}
\catcode`\{=1 %
\catcode`\}=2 %
\catcode`\#=6 %
\catcode`\@=11 %
\expandafter\ifx\csname count@\endcsname\relax
  \countdef\count@=255 %
\fi
\expandafter\ifx\csname @gobble\endcsname\relax
  \long\def\@gobble#1{}%
\fi
\expandafter\ifx\csname @firstofone\endcsname\relax
  \long\def\@firstofone#1{#1}%
\fi
\expandafter\ifx\csname loop\endcsname\relax
  \expandafter\@firstofone
\else
  \expandafter\@gobble
\fi
{%
  \def\loop#1\repeat{%
    \def\body{#1}%
    \iterate
  }%
  \def\iterate{%
    \body
      \let\next\iterate
    \else
      \let\next\relax
    \fi
    \next
  }%
  \let\repeat=\fi
}%
\def\RestoreCatcodes{}
\count@=0 %
\loop
  \edef\RestoreCatcodes{%
    \RestoreCatcodes
    \catcode\the\count@=\the\catcode\count@\relax
  }%
\ifnum\count@<255 %
  \advance\count@ 1 %
\repeat

\def\RangeCatcodeInvalid#1#2{%
  \count@=#1\relax
  \loop
    \catcode\count@=15 %
  \ifnum\count@<#2\relax
    \advance\count@ 1 %
  \repeat
}
\def\RangeCatcodeCheck#1#2#3{%
  \count@=#1\relax
  \loop
    \ifnum#3=\catcode\count@
    \else
      \errmessage{%
        Character \the\count@\space
        with wrong catcode \the\catcode\count@\space
        instead of \number#3%
      }%
    \fi
  \ifnum\count@<#2\relax
    \advance\count@ 1 %
  \repeat
}
\def\space{ }
\expandafter\ifx\csname LoadCommand\endcsname\relax
  \def\LoadCommand{\input hologo.sty\relax}%
\fi
\def\Test{%
  \RangeCatcodeInvalid{0}{47}%
  \RangeCatcodeInvalid{58}{64}%
  \RangeCatcodeInvalid{91}{96}%
  \RangeCatcodeInvalid{123}{255}%
  \catcode`\@=12 %
  \catcode`\\=0 %
  \catcode`\%=14 %
  \LoadCommand
  \RangeCatcodeCheck{0}{36}{15}%
  \RangeCatcodeCheck{37}{37}{14}%
  \RangeCatcodeCheck{38}{47}{15}%
  \RangeCatcodeCheck{48}{57}{12}%
  \RangeCatcodeCheck{58}{63}{15}%
  \RangeCatcodeCheck{64}{64}{12}%
  \RangeCatcodeCheck{65}{90}{11}%
  \RangeCatcodeCheck{91}{91}{15}%
  \RangeCatcodeCheck{92}{92}{0}%
  \RangeCatcodeCheck{93}{96}{15}%
  \RangeCatcodeCheck{97}{122}{11}%
  \RangeCatcodeCheck{123}{255}{15}%
  \RestoreCatcodes
}
\Test
\csname @@end\endcsname
\end
%    \end{macrocode}
%    \begin{macrocode}
%</test1>
%    \end{macrocode}
%
% \subsection{Spacefactor}
%
%    The space factor must be 1000 after a logo. If it is greater 1000
%    then the following space is a space after a sentence closing point.
%    If the space factor is smaller 1000 then an immediate following
%    dot is interpreted as abbreviation, not sentence closing point.
%
%    \begin{macrocode}
%<*test-spacefactor>
\NeedsTeXFormat{LaTeX2e}
\documentclass{article}
\usepackage{hologo}[2016/05/12]
\usepackage{kvsetkeys}
\usepackage{qstest}
\IncludeTests{*}
\LogTests{log}{*}{*}
\begin{document}
\begin{qstest}{spacefactor}{spacefactor}
\newcommand*{\Test}[1]{%
  \sbox0{%
    \hologo{#1}%
    \Expect*{1000 (#1)}*{\the\spacefactor\space(#1)}%
  }%
}%
\makeatletter
\def\TestList{}
\def\hologoEntry#1#2#3{%
  \edef\TestList{%
    \ifx\TestList\@empty
    \else
      \TestList,%
    \fi
    #1%
    \ifx\\#2\\%
    \else
      ={variant=#2}%
    \fi
  }%
}
\hologoList
\expandafter\kv@parse@normalized\expandafter{%
  \TestList
}{%
  \begingroup
    \let\@logo=\kv@key
    \ifx\kv@value\relax
    \else
      \expandafter\hologoLogoSetup\expandafter\@logo\expandafter{%
        \kv@value
      }%
    \fi
    \Test\@logo
  \endgroup
  \@gobbletwo
}
\end{qstest}
\end{document}
%</test-spacefactor>
%    \end{macrocode}
%
% \subsection{Complete list}
%
%    \begin{macrocode}
%<*test-list>
\NeedsTeXFormat{LaTeX2e}
\documentclass[12pt,a4paper]{article}
\usepackage{hologo}[2016/05/12]
\usepackage[T1]{fontenc}
\usepackage{lmodern}
\usepackage{parskip}
\usepackage[unicode]{hyperref}[2011/09/28]
\usepackage{bookmark}[2011/09/19]
\bookmarksetup{%
  numbered,%
  open,%
  openlevel=2,%
}
\renewcommand*{\contentsname}{List of logos}
\begin{document}
\tableofcontents
\def\TestFont#1#2#3#4#5#6{%
  \begingroup
    \usefont{#3}{#4}{#5}{#6}%
    \HologoVariant{#1}{#2}/\hologoVariant{#1}{#2}%
    \quad
    \begingroup\scriptsize\hologoVariant{#1}{#2}\endgroup
    \quad
  \endgroup
  (#3/#4/#5/#6)%
  \par
}
\makeatletter
\def\hologoEntry#1#2#3{%
  \section{%
    \HologoVariant{#1}{#2}/\hologoVariant{#1}{#2} %
    {[#1\ifx\\#2\\\else\space(#2)\fi]}% hash-ok
  }% braces around [] because of bug in tex4ht
  \begingroup
    \hypersetup{unicode=false}%
    \bookmark[%
      dest=\@currentHref,%
      rellevel=1,%
      keeplevel,%
    ]{%
      \HologoVariant{#1}{#2}/\hologoVariant{#1}{#2} %
      (PDFDocEncoding)%
    }%
  \endgroup
  \TestFont{#1}{#2}{OT1}{cmr}{m}{n}%
  \TestFont{#1}{#2}{OT1}{cmss}{m}{n}%
  \TestFont{#1}{#2}{OT1}{cmr}{b}{n}%
  \TestFont{#1}{#2}{OT1}{cmr}{m}{it}%
  \TestFont{#1}{#2}{OT1}{cmtt}{m}{n}%
  \TestFont{#1}{#2}{T1}{lmr}{m}{n}%
  \TestFont{#1}{#2}{T1}{lmss}{m}{n}%
  \TestFont{#1}{#2}{T1}{lmr}{b}{n}%
  \TestFont{#1}{#2}{T1}{lmr}{m}{it}%
  \TestFont{#1}{#2}{T1}{lmtt}{m}{n}%
  \TestFont{#1}{#2}{T1}{lmvtt}{m}{n}%
  \TestFont{#1}{#2}{T1}{qtm}{m}{n}%
  \TestFont{#1}{#2}{T1}{qhv}{m}{n}%
  \TestFont{#1}{#2}{T1}{qtm}{b}{n}%
  \TestFont{#1}{#2}{T1}{qtm}{m}{it}%
  \TestFont{#1}{#2}{T1}{qcr}{m}{n}%
  \newpage
}
\makeatother
\hologoList
\end{document}
%</test-list>
%    \end{macrocode}
%
% \section{Installation}
%
% \subsection{Download}
%
% \paragraph{Package.} This package is available on
% CTAN\footnote{\url{ftp://ftp.ctan.org/tex-archive/}}:
% \begin{description}
% \item[\CTAN{macros/latex/contrib/oberdiek/hologo.dtx}] The source file.
% \item[\CTAN{macros/latex/contrib/oberdiek/hologo.pdf}] Documentation.
% \end{description}
%
%
% \paragraph{Bundle.} All the packages of the bundle `oberdiek'
% are also available in a TDS compliant ZIP archive. There
% the packages are already unpacked and the documentation files
% are generated. The files and directories obey the TDS standard.
% \begin{description}
% \item[\CTAN{install/macros/latex/contrib/oberdiek.tds.zip}]
% \end{description}
% \emph{TDS} refers to the standard ``A Directory Structure
% for \TeX\ Files'' (\CTAN{tds/tds.pdf}). Directories
% with \xfile{texmf} in their name are usually organized this way.
%
% \subsection{Bundle installation}
%
% \paragraph{Unpacking.} Unpack the \xfile{oberdiek.tds.zip} in the
% TDS tree (also known as \xfile{texmf} tree) of your choice.
% Example (linux):
% \begin{quote}
%   |unzip oberdiek.tds.zip -d ~/texmf|
% \end{quote}
%
% \paragraph{Script installation.}
% Check the directory \xfile{TDS:scripts/oberdiek/} for
% scripts that need further installation steps.
% Package \xpackage{attachfile2} comes with the Perl script
% \xfile{pdfatfi.pl} that should be installed in such a way
% that it can be called as \texttt{pdfatfi}.
% Example (linux):
% \begin{quote}
%   |chmod +x scripts/oberdiek/pdfatfi.pl|\\
%   |cp scripts/oberdiek/pdfatfi.pl /usr/local/bin/|
% \end{quote}
%
% \subsection{Package installation}
%
% \paragraph{Unpacking.} The \xfile{.dtx} file is a self-extracting
% \docstrip\ archive. The files are extracted by running the
% \xfile{.dtx} through \plainTeX:
% \begin{quote}
%   \verb|tex hologo.dtx|
% \end{quote}
%
% \paragraph{TDS.} Now the different files must be moved into
% the different directories in your installation TDS tree
% (also known as \xfile{texmf} tree):
% \begin{quote}
% \def\t{^^A
% \begin{tabular}{@{}>{\ttfamily}l@{ $\rightarrow$ }>{\ttfamily}l@{}}
%   hologo.sty & tex/generic/oberdiek/hologo.sty\\
%   hologo.pdf & doc/latex/oberdiek/hologo.pdf\\
%   example/hologo-example.tex & doc/latex/oberdiek/example/hologo-example.tex\\
%   test/hologo-test1.tex & doc/latex/oberdiek/test/hologo-test1.tex\\
%   test/hologo-test-spacefactor.tex & doc/latex/oberdiek/test/hologo-test-spacefactor.tex\\
%   test/hologo-test-list.tex & doc/latex/oberdiek/test/hologo-test-list.tex\\
%   hologo.dtx & source/latex/oberdiek/hologo.dtx\\
% \end{tabular}^^A
% }^^A
% \sbox0{\t}^^A
% \ifdim\wd0>\linewidth
%   \begingroup
%     \advance\linewidth by\leftmargin
%     \advance\linewidth by\rightmargin
%   \edef\x{\endgroup
%     \def\noexpand\lw{\the\linewidth}^^A
%   }\x
%   \def\lwbox{^^A
%     \leavevmode
%     \hbox to \linewidth{^^A
%       \kern-\leftmargin\relax
%       \hss
%       \usebox0
%       \hss
%       \kern-\rightmargin\relax
%     }^^A
%   }^^A
%   \ifdim\wd0>\lw
%     \sbox0{\small\t}^^A
%     \ifdim\wd0>\linewidth
%       \ifdim\wd0>\lw
%         \sbox0{\footnotesize\t}^^A
%         \ifdim\wd0>\linewidth
%           \ifdim\wd0>\lw
%             \sbox0{\scriptsize\t}^^A
%             \ifdim\wd0>\linewidth
%               \ifdim\wd0>\lw
%                 \sbox0{\tiny\t}^^A
%                 \ifdim\wd0>\linewidth
%                   \lwbox
%                 \else
%                   \usebox0
%                 \fi
%               \else
%                 \lwbox
%               \fi
%             \else
%               \usebox0
%             \fi
%           \else
%             \lwbox
%           \fi
%         \else
%           \usebox0
%         \fi
%       \else
%         \lwbox
%       \fi
%     \else
%       \usebox0
%     \fi
%   \else
%     \lwbox
%   \fi
% \else
%   \usebox0
% \fi
% \end{quote}
% If you have a \xfile{docstrip.cfg} that configures and enables \docstrip's
% TDS installing feature, then some files can already be in the right
% place, see the documentation of \docstrip.
%
% \subsection{Refresh file name databases}
%
% If your \TeX~distribution
% (\teTeX, \mikTeX, \dots) relies on file name databases, you must refresh
% these. For example, \teTeX\ users run \verb|texhash| or
% \verb|mktexlsr|.
%
% \subsection{Some details for the interested}
%
% \paragraph{Attached source.}
%
% The PDF documentation on CTAN also includes the
% \xfile{.dtx} source file. It can be extracted by
% AcrobatReader 6 or higher. Another option is \textsf{pdftk},
% e.g. unpack the file into the current directory:
% \begin{quote}
%   \verb|pdftk hologo.pdf unpack_files output .|
% \end{quote}
%
% \paragraph{Unpacking with \LaTeX.}
% The \xfile{.dtx} chooses its action depending on the format:
% \begin{description}
% \item[\plainTeX:] Run \docstrip\ and extract the files.
% \item[\LaTeX:] Generate the documentation.
% \end{description}
% If you insist on using \LaTeX\ for \docstrip\ (really,
% \docstrip\ does not need \LaTeX), then inform the autodetect routine
% about your intention:
% \begin{quote}
%   \verb|latex \let\install=y% \iffalse meta-comment
%
% File: hologo.dtx
% Version: 2016/05/12 v1.11
% Info: A logo collection with bookmark support
%
% Copyright (C) 2010-2012 by
%    Heiko Oberdiek <heiko.oberdiek at googlemail.com>
%
% This work may be distributed and/or modified under the
% conditions of the LaTeX Project Public License, either
% version 1.3c of this license or (at your option) any later
% version. This version of this license is in
%    http://www.latex-project.org/lppl/lppl-1-3c.txt
% and the latest version of this license is in
%    http://www.latex-project.org/lppl.txt
% and version 1.3 or later is part of all distributions of
% LaTeX version 2005/12/01 or later.
%
% This work has the LPPL maintenance status "maintained".
%
% This Current Maintainer of this work is Heiko Oberdiek.
%
% The Base Interpreter refers to any `TeX-Format',
% because some files are installed in TDS:tex/generic//.
%
% This work consists of the main source file hologo.dtx
% and the derived files
%    hologo.sty, hologo.pdf, hologo.ins, hologo.drv, hologo-example.tex,
%    hologo-test1.tex, hologo-test-spacefactor.tex,
%    hologo-test-list.tex.
%
% Distribution:
%    CTAN:macros/latex/contrib/oberdiek/hologo.dtx
%    CTAN:macros/latex/contrib/oberdiek/hologo.pdf
%
% Unpacking:
%    (a) If hologo.ins is present:
%           tex hologo.ins
%    (b) Without hologo.ins:
%           tex hologo.dtx
%    (c) If you insist on using LaTeX
%           latex \let\install=y% \iffalse meta-comment
%
% File: hologo.dtx
% Version: 2016/05/12 v1.11
% Info: A logo collection with bookmark support
%
% Copyright (C) 2010-2012 by
%    Heiko Oberdiek <heiko.oberdiek at googlemail.com>
%
% This work may be distributed and/or modified under the
% conditions of the LaTeX Project Public License, either
% version 1.3c of this license or (at your option) any later
% version. This version of this license is in
%    http://www.latex-project.org/lppl/lppl-1-3c.txt
% and the latest version of this license is in
%    http://www.latex-project.org/lppl.txt
% and version 1.3 or later is part of all distributions of
% LaTeX version 2005/12/01 or later.
%
% This work has the LPPL maintenance status "maintained".
%
% This Current Maintainer of this work is Heiko Oberdiek.
%
% The Base Interpreter refers to any `TeX-Format',
% because some files are installed in TDS:tex/generic//.
%
% This work consists of the main source file hologo.dtx
% and the derived files
%    hologo.sty, hologo.pdf, hologo.ins, hologo.drv, hologo-example.tex,
%    hologo-test1.tex, hologo-test-spacefactor.tex,
%    hologo-test-list.tex.
%
% Distribution:
%    CTAN:macros/latex/contrib/oberdiek/hologo.dtx
%    CTAN:macros/latex/contrib/oberdiek/hologo.pdf
%
% Unpacking:
%    (a) If hologo.ins is present:
%           tex hologo.ins
%    (b) Without hologo.ins:
%           tex hologo.dtx
%    (c) If you insist on using LaTeX
%           latex \let\install=y\input{hologo.dtx}
%        (quote the arguments according to the demands of your shell)
%
% Documentation:
%    (a) If hologo.drv is present:
%           latex hologo.drv
%    (b) Without hologo.drv:
%           latex hologo.dtx; ...
%    The class ltxdoc loads the configuration file ltxdoc.cfg
%    if available. Here you can specify further options, e.g.
%    use A4 as paper format:
%       \PassOptionsToClass{a4paper}{article}
%
%    Programm calls to get the documentation (example):
%       pdflatex hologo.dtx
%       makeindex -s gind.ist hologo.idx
%       pdflatex hologo.dtx
%       makeindex -s gind.ist hologo.idx
%       pdflatex hologo.dtx
%
% Installation:
%    TDS:tex/generic/oberdiek/hologo.sty
%    TDS:doc/latex/oberdiek/hologo.pdf
%    TDS:doc/latex/oberdiek/example/hologo-example.tex
%    TDS:doc/latex/oberdiek/test/hologo-test1.tex
%    TDS:doc/latex/oberdiek/test/hologo-test-spacefactor.tex
%    TDS:doc/latex/oberdiek/test/hologo-test-list.tex
%    TDS:source/latex/oberdiek/hologo.dtx
%
%<*ignore>
\begingroup
  \catcode123=1 %
  \catcode125=2 %
  \def\x{LaTeX2e}%
\expandafter\endgroup
\ifcase 0\ifx\install y1\fi\expandafter
         \ifx\csname processbatchFile\endcsname\relax\else1\fi
         \ifx\fmtname\x\else 1\fi\relax
\else\csname fi\endcsname
%</ignore>
%<*install>
\input docstrip.tex
\Msg{************************************************************************}
\Msg{* Installation}
\Msg{* Package: hologo 2016/05/12 v1.11 A logo collection with bookmark support (HO)}
\Msg{************************************************************************}

\keepsilent
\askforoverwritefalse

\let\MetaPrefix\relax
\preamble

This is a generated file.

Project: hologo
Version: 2016/05/12 v1.11

Copyright (C) 2010-2012 by
   Heiko Oberdiek <heiko.oberdiek at googlemail.com>

This work may be distributed and/or modified under the
conditions of the LaTeX Project Public License, either
version 1.3c of this license or (at your option) any later
version. This version of this license is in
   http://www.latex-project.org/lppl/lppl-1-3c.txt
and the latest version of this license is in
   http://www.latex-project.org/lppl.txt
and version 1.3 or later is part of all distributions of
LaTeX version 2005/12/01 or later.

This work has the LPPL maintenance status "maintained".

This Current Maintainer of this work is Heiko Oberdiek.

The Base Interpreter refers to any `TeX-Format',
because some files are installed in TDS:tex/generic//.

This work consists of the main source file hologo.dtx
and the derived files
   hologo.sty, hologo.pdf, hologo.ins, hologo.drv, hologo-example.tex,
   hologo-test1.tex, hologo-test-spacefactor.tex,
   hologo-test-list.tex.

\endpreamble
\let\MetaPrefix\DoubleperCent

\generate{%
  \file{hologo.ins}{\from{hologo.dtx}{install}}%
  \file{hologo.drv}{\from{hologo.dtx}{driver}}%
  \usedir{tex/generic/oberdiek}%
  \file{hologo.sty}{\from{hologo.dtx}{package}}%
  \usedir{doc/latex/oberdiek/example}%
  \file{hologo-example.tex}{\from{hologo.dtx}{example}}%
  \usedir{doc/latex/oberdiek/test}%
  \file{hologo-test1.tex}{\from{hologo.dtx}{test1}}%
  \file{hologo-test-spacefactor.tex}{\from{hologo.dtx}{test-spacefactor}}%
  \file{hologo-test-list.tex}{\from{hologo.dtx}{test-list}}%
  \nopreamble
  \nopostamble
  \usedir{source/latex/oberdiek/catalogue}%
  \file{hologo.xml}{\from{hologo.dtx}{catalogue}}%
}

\catcode32=13\relax% active space
\let =\space%
\Msg{************************************************************************}
\Msg{*}
\Msg{* To finish the installation you have to move the following}
\Msg{* file into a directory searched by TeX:}
\Msg{*}
\Msg{*     hologo.sty}
\Msg{*}
\Msg{* To produce the documentation run the file `hologo.drv'}
\Msg{* through LaTeX.}
\Msg{*}
\Msg{* Happy TeXing!}
\Msg{*}
\Msg{************************************************************************}

\endbatchfile
%</install>
%<*ignore>
\fi
%</ignore>
%<*driver>
\NeedsTeXFormat{LaTeX2e}
\ProvidesFile{hologo.drv}%
  [2016/05/12 v1.11 A logo collection with bookmark support (HO)]%
\documentclass{ltxdoc}
\usepackage{holtxdoc}[2011/11/22]
\usepackage{hologo}[2016/05/12]
\usepackage{longtable}
\usepackage{array}
\usepackage{paralist}
%\usepackage[T1]{fontenc}
%\usepackage{lmodern}
\begin{document}
  \DocInput{hologo.dtx}%
\end{document}
%</driver>
% \fi
%
%
% \CharacterTable
%  {Upper-case    \A\B\C\D\E\F\G\H\I\J\K\L\M\N\O\P\Q\R\S\T\U\V\W\X\Y\Z
%   Lower-case    \a\b\c\d\e\f\g\h\i\j\k\l\m\n\o\p\q\r\s\t\u\v\w\x\y\z
%   Digits        \0\1\2\3\4\5\6\7\8\9
%   Exclamation   \!     Double quote  \"     Hash (number) \#
%   Dollar        \$     Percent       \%     Ampersand     \&
%   Acute accent  \'     Left paren    \(     Right paren   \)
%   Asterisk      \*     Plus          \+     Comma         \,
%   Minus         \-     Point         \.     Solidus       \/
%   Colon         \:     Semicolon     \;     Less than     \<
%   Equals        \=     Greater than  \>     Question mark \?
%   Commercial at \@     Left bracket  \[     Backslash     \\
%   Right bracket \]     Circumflex    \^     Underscore    \_
%   Grave accent  \`     Left brace    \{     Vertical bar  \|
%   Right brace   \}     Tilde         \~}
%
% \GetFileInfo{hologo.drv}
%
% \title{The \xpackage{hologo} package}
% \date{2016/05/12 v1.11}
% \author{Heiko Oberdiek\\\xemail{heiko.oberdiek at googlemail.com}}
%
% \maketitle
%
% \begin{abstract}
% This package starts a collection of logos with support for bookmarks
% strings.
% \end{abstract}
%
% \tableofcontents
%
% \section{Documentation}
%
% \subsection{Logo macros}
%
% \begin{declcs}{hologo} \M{name}
% \end{declcs}
% Macro \cs{hologo} sets the logo with name \meta{name}.
% The following table shows the supported names.
%
% \begingroup
%   \def\hologoEntry#1#2#3{^^A
%     #1&#2&\hologoLogoSetup{#1}{variant=#2}\hologo{#1}&#3\tabularnewline
%   }
%   \begin{longtable}{>{\ttfamily}l>{\ttfamily}lll}
%     \rmfamily\bfseries{name} & \rmfamily\bfseries variant
%     & \bfseries logo & \bfseries since\\
%     \hline
%     \endhead
%     \hologoList
%   \end{longtable}
% \endgroup
%
% \begin{declcs}{Hologo} \M{name}
% \end{declcs}
% Macro \cs{Hologo} starts the logo \meta{name} with an uppercase
% letter. As an exception small greek letters are not converted
% to uppercase. Examples, see \hologo{eTeX} and \hologo{ExTeX}.
%
% \subsection{Setup macros}
%
% The package does not support package options, but the following
% setup macros can be used to set options.
%
% \begin{declcs}{hologoSetup} \M{key value list}
% \end{declcs}
% Macro \cs{hologoSetup} sets global options.
%
% \begin{declcs}{hologoLogoSetup} \M{logo} \M{key value list}
% \end{declcs}
% Some options can also be used to configure a logo.
% These settings take precedence over global option settings.
%
% \subsection{Options}\label{sec:options}
%
% There are boolean and string options:
% \begin{description}
% \item[Boolean option:]
% It takes |true| or |false|
% as value. If the value is omitted, then |true| is used.
% \item[String option:]
% A value must be given as string. (But the string might be empty.)
% \end{description}
% The following options can be used both in \cs{hologoSetup}
% and \cs{hologoLogoSetup}:
% \begin{description}
% \def\entry#1{\item[\xoption{#1}:]}
% \entry{break}
%   enables or disables line breaks inside the logo. This setting is
%   refined by options \xoption{hyphenbreak}, \xoption{spacebreak}
%   or \xoption{discretionarybreak}.
%   Default is |false|.
% \entry{hyphenbreak}
%   enables or disables the line break right after the hyphen character.
% \entry{spacebreak}
%   enables or disables line breaks at space characters.
% \entry{discretionarybreak}
%   enables or disables line breaks at hyphenation points
%   (inserted by \cs{-}).
% \end{description}
% Macro \cs{hologoLogoSetup} also knows:
% \begin{description}
% \item[\xoption{variant}:]
%   This is a string option. It specifies a variant of a logo that
%   must exist. An empty string selects the package default variant.
% \end{description}
% Example:
% \begin{quote}
%   |\hologoSetup{break=false}|\\
%   |\hologoLogoSetup{plainTeX}{variant=hyphen,hyphenbreak}|\\
%   Then ``plain-\TeX'' contains one break point after the hyphen.
% \end{quote}
%
% \subsection{Driver options}
%
% Sometimes graphical operations are needed to construct some
% glyphs (e.g.\ \hologo{XeTeX}). If package \xpackage{graphics}
% or package \xpackage{pgf} are found, then the macros are taken
% from there. Otherwise the packge defines its own operations
% and therefore needs the driver information. Many drivers are
% detected automatically (\hologo{pdfTeX}/\hologo{LuaTeX}
% in PDF mode, \hologo{XeTeX}, \hologo{VTeX}). These have precedence
% over a driver option. The driver can be given as package option
% or using \cs{hologoDriverSetup}.
% The following list contains the recognized driver options:
% \begin{itemize}
% \item \xoption{pdftex}, \xoption{luatex}
% \item \xoption{dvipdfm}, \xoption{dvipdfmx}
% \item \xoption{dvips}, \xoption{dvipsone}, \xoption{xdvi}
% \item \xoption{xetex}
% \item \xoption{vtex}
% \end{itemize}
% The left driver of a line is the driver name that is used internally.
% The following names are aliases for drivers that use the
% same method. Therefore the entry in the \xext{log} file for
% the used driver prints the internally used driver name.
% \begin{description}
% \item[\xoption{driverfallback}:]
%   This option expects a driver that is used,
%   if the driver could not be detected automatically.
% \end{description}
%
% \begin{declcs}{hologoDriverSetup} \M{driver option}
% \end{declcs}
% The driver can also be configured after package loading
% using \cs{hologoDriverSetup}, also the way for \hologo{plainTeX}
% to setup the driver.
%
% \subsection{Font setup}
%
% Some logos require a special font, but should also be usable by
% \hologo{plainTeX}. Therefore the package provides some ways
% to influence the font settings. The options below
% take font settings as values. Both font commands
% such as \cs{sffamily} and macros that take one argument
% like \cs{textsf} can be used.
%
% \begin{declcs}{hologoFontSetup} \M{key value list}
% \end{declcs}
% Macro \cs{hologoFontSetup} sets the fonts for all logos.
% Supported keys:
% \begin{description}
% \def\entry#1{\item[\xoption{#1}:]}
% \entry{general}
%   This font is used for all logos. The default is empty.
%   That means no special font is used.
% \entry{bibsf}
%   This font is used for
%   {\hologoLogoSetup{BibTeX}{variant=sf}\hologo{BibTeX}}
%   with variant \xoption{sf}.
% \entry{rm}
%   This font is a serif font. It is used for \hologo{ExTeX}.
% \entry{sc}
%   This font specifies a small caps font. It is used for
%   {\hologoLogoSetup{BibTeX}{variant=sc}\hologo{BibTeX}}
%   with variant \xoption{sc}.
% \entry{sf}
%   This font specifies a sans serif font. The default
%   is \cs{sffamily}, then \cs{sf} is tried. Otherwise
%   a warning is given. It is used by \hologo{KOMAScript}.
% \entry{sy}
%   This is the font for math symbols (e.g. cmsy).
%   It is used by \hologo{AmS}, \hologo{NTS}, \hologo{ExTeX}.
% \entry{logo}
%   \hologo{METAFONT} and \hologo{METAPOST} are using that font.
%   In \hologo{LaTeX} \cs{logofamily} is used and
%   the definitions of package \xpackage{mflogo} are used
%   if the package is not loaded.
%   Otherwise the \cs{tenlogo} is used and defined
%   if it does not already exists.
% \end{description}
%
% \begin{declcs}{hologoLogoFontSetup} \M{logo} \M{key value list}
% \end{declcs}
% Fonts can also be set for a logo or logo component separately,
% see the following list.
% The keys are the same as for \cs{hologoFontSetup}.
%
% \begin{longtable}{>{\ttfamily}l>{\sffamily}ll}
%   \meta{logo} & keys & result\\
%   \hline
%   \endhead
%   BibTeX & bibsf & {\hologoLogoSetup{BibTeX}{variant=sf}\hologo{BibTeX}}\\[.5ex]
%   BibTeX & sc & {\hologoLogoSetup{BibTeX}{variant=sc}\hologo{BibTeX}}\\[.5ex]
%   ExTeX & rm & \hologo{ExTeX}\\
%   SliTeX & rm & \hologo{SliTeX}\\[.5ex]
%   AmS & sy & \hologo{AmS}\\
%   ExTeX & sy & \hologo{ExTeX}\\
%   NTS & sy & \hologo{NTS}\\[.5ex]
%   KOMAScript & sf & \hologo{KOMAScript}\\[.5ex]
%   METAFONT & logo & \hologo{METAFONT}\\
%   METAPOST & logo & \hologo{METAPOST}\\[.5ex]
%   SliTeX & sc \hologo{SliTeX}
% \end{longtable}
%
% \subsubsection{Font order}
%
% For all logos the font \xoption{general} is applied first.
% Example:
%\begin{quote}
%|\hologoFontSetup{general=\color{red}}|
%\end{quote}
% will print red logos.
% Then if the font uses a special font \xoption{sf}, for example,
% the font is applied that is setup by \cs{hologoLogoFontSetup}.
% If this font is not setup, then the common font setup
% by \cs{hologoFontSetup} is used. Otherwise a warning is given,
% that there is no font configured.
%
% \subsection{Additional user macros}
%
% Usually a variant of a logo is configured by using
% \cs{hologoLogoSetup}, because it is bad style to mix
% different variants of the same logo in the same text.
% There the following macros are a convenience for testing.
%
% \begin{declcs}{hologoVariant} \M{name} \M{variant}\\
%   \cs{HologoVariant} \M{name} \M{variant}
% \end{declcs}
% Logo \meta{name} is set using \meta{variant} that specifies
% explicitely which variant of the macro is used. If the argument
% is empty, then the default form of the logo is used
% (configurable by \cs{hologoLogoSetup}).
%
% \cs{HologoVariant} is used if the logo is set in a context
% that needs an uppercase first letter (beginning of a sentence, \dots).
%
% \begin{declcs}{hologoList}\\
%   \cs{hologoEntry} \M{logo} \M{variant} \M{since}
% \end{declcs}
% Macro \cs{hologoList} contains all logos that are provided
% by the package including variants. The list consists of calls
% of \cs{hologoEntry} with three arguments starting with the
% logo name \meta{logo} and its variant \meta{variant}. An empty
% variant means the current default. Argument \meta{since} specifies
% with version of the package \xpackage{hologo} is needed to get
% the logo. If the logo is fixed, then the date gets updated.
% Therefore the date \meta{since} is not exactly the date of
% the first introduction, but rather the date of the latest fix.
%
% Before \cs{hologoList} can be used, macro \cs{hologoEntry} needs
% a definition. The example file in section \ref{sec:example}
% shows applications of \cs{hologoList}.
%
% \subsection{Supported contexts}
%
% Macros \cs{hologo} and friends support special contexts:
% \begin{itemize}
% \item \hologo{LaTeX}'s protection mechanism.
% \item Bookmarks of package \xpackage{hyperref}.
% \item Package \xpackage{tex4ht}.
% \item The macros can be used inside \cs{csname} constructs,
%   if \cs{ifincsname} is available (\hologo{pdfTeX}, \hologo{XeTeX},
%   \hologo{LuaTeX}).
% \end{itemize}
%
% \subsection{Example}
% \label{sec:example}
%
% The following example prints the logos in different fonts.
%    \begin{macrocode}
%<*example>
%<<verbatim
\NeedsTeXFormat{LaTeX2e}
\documentclass[a4paper]{article}
\usepackage[
  hmargin=20mm,
  vmargin=20mm,
]{geometry}
\pagestyle{empty}
\usepackage{hologo}[2016/05/12]
\usepackage{longtable}
\usepackage{array}
\setlength{\extrarowheight}{2pt}
\usepackage[T1]{fontenc}
\usepackage{lmodern}
\usepackage{pdflscape}
\usepackage[
  pdfencoding=auto,
]{hyperref}
\hypersetup{
  pdfauthor={Heiko Oberdiek},
  pdftitle={Example for package `hologo'},
  pdfsubject={Logos with fonts lmr, lmss, qtm, qpl, qhv},
}
\usepackage{bookmark}

% Print the logo list on the console

\begingroup
  \typeout{}%
  \typeout{*** Begin of logo list ***}%
  \newcommand*{\hologoEntry}[3]{%
    \typeout{#1 \ifx\\#2\\\else(#2) \fi[#3]}%
  }%
  \hologoList
  \typeout{*** End of logo list ***}%
  \typeout{}%
\endgroup

\begin{document}
\begin{landscape}

  \section{Example file for package `hologo'}

  % Table for font names

  \begin{longtable}{>{\bfseries}ll}
    \textbf{font} & \textbf{Font name}\\
    \hline
    lmr & Latin Modern Roman\\
    lmss & Latin Modern Sans\\
    qtm & \TeX\ Gyre Termes\\
    qhv & \TeX\ Gyre Heros\\
    qpl & \TeX\ Gyre Pagella\\
  \end{longtable}

  % Logo list with logos in different fonts

  \begingroup
    \newcommand*{\SetVariant}[2]{%
      \ifx\\#2\\%
      \else
        \hologoLogoSetup{#1}{variant=#2}%
      \fi
    }%
    \newcommand*{\hologoEntry}[3]{%
      \SetVariant{#1}{#2}%
      \raisebox{1em}[0pt][0pt]{\hypertarget{#1@#2}{}}%
      \bookmark[%
        dest={#1@#2},%
      ]{%
        #1\ifx\\#2\\\else\space(#2)\fi: \Hologo{#1}, \hologo{#1} %
        [Unicode]%
      }%
      \hypersetup{unicode=false}%
      \bookmark[%
        dest={#1@#2},%
      ]{%
        #1\ifx\\#2\\\else\space(#2)\fi: \Hologo{#1}, \hologo{#1} %
        [PDFDocEncoding]%
      }%
      \texttt{#1}%
      &%
      \texttt{#2}%
      &%
      \Hologo{#1}%
      &%
      \SetVariant{#1}{#2}%
      \hologo{#1}%
      &%
      \SetVariant{#1}{#2}%
      \fontfamily{qtm}\selectfont
      \hologo{#1}%
      &%
      \SetVariant{#1}{#2}%
      \fontfamily{qpl}\selectfont
      \hologo{#1}%
      &%
      \SetVariant{#1}{#2}%
      \textsf{\hologo{#1}}%
      &%
      \SetVariant{#1}{#2}%
      \fontfamily{qhv}\selectfont
      \hologo{#1}%
      \tabularnewline
    }%
    \begin{longtable}{llllllll}%
      \textbf{\textit{logo}} & \textbf{\textit{variant}} &
      \texttt{\string\Hologo} &
      \textbf{lmr} & \textbf{qtm} & \textbf{qpl} &
      \textbf{lmss} & \textbf{qhv}
      \tabularnewline
      \hline
      \endhead
      \hologoList
    \end{longtable}%
  \endgroup

\end{landscape}
\end{document}
%verbatim
%</example>
%    \end{macrocode}
%
% \StopEventually{
% }
%
% \section{Implementation}
%    \begin{macrocode}
%<*package>
%    \end{macrocode}
%    Reload check, especially if the package is not used with \LaTeX.
%    \begin{macrocode}
\begingroup\catcode61\catcode48\catcode32=10\relax%
  \catcode13=5 % ^^M
  \endlinechar=13 %
  \catcode35=6 % #
  \catcode39=12 % '
  \catcode44=12 % ,
  \catcode45=12 % -
  \catcode46=12 % .
  \catcode58=12 % :
  \catcode64=11 % @
  \catcode123=1 % {
  \catcode125=2 % }
  \expandafter\let\expandafter\x\csname ver@hologo.sty\endcsname
  \ifx\x\relax % plain-TeX, first loading
  \else
    \def\empty{}%
    \ifx\x\empty % LaTeX, first loading,
      % variable is initialized, but \ProvidesPackage not yet seen
    \else
      \expandafter\ifx\csname PackageInfo\endcsname\relax
        \def\x#1#2{%
          \immediate\write-1{Package #1 Info: #2.}%
        }%
      \else
        \def\x#1#2{\PackageInfo{#1}{#2, stopped}}%
      \fi
      \x{hologo}{The package is already loaded}%
      \aftergroup\endinput
    \fi
  \fi
\endgroup%
%    \end{macrocode}
%    Package identification:
%    \begin{macrocode}
\begingroup\catcode61\catcode48\catcode32=10\relax%
  \catcode13=5 % ^^M
  \endlinechar=13 %
  \catcode35=6 % #
  \catcode39=12 % '
  \catcode40=12 % (
  \catcode41=12 % )
  \catcode44=12 % ,
  \catcode45=12 % -
  \catcode46=12 % .
  \catcode47=12 % /
  \catcode58=12 % :
  \catcode64=11 % @
  \catcode91=12 % [
  \catcode93=12 % ]
  \catcode123=1 % {
  \catcode125=2 % }
  \expandafter\ifx\csname ProvidesPackage\endcsname\relax
    \def\x#1#2#3[#4]{\endgroup
      \immediate\write-1{Package: #3 #4}%
      \xdef#1{#4}%
    }%
  \else
    \def\x#1#2[#3]{\endgroup
      #2[{#3}]%
      \ifx#1\@undefined
        \xdef#1{#3}%
      \fi
      \ifx#1\relax
        \xdef#1{#3}%
      \fi
    }%
  \fi
\expandafter\x\csname ver@hologo.sty\endcsname
\ProvidesPackage{hologo}%
  [2016/05/12 v1.11 A logo collection with bookmark support (HO)]%
%    \end{macrocode}
%
%    \begin{macrocode}
\begingroup\catcode61\catcode48\catcode32=10\relax%
  \catcode13=5 % ^^M
  \endlinechar=13 %
  \catcode123=1 % {
  \catcode125=2 % }
  \catcode64=11 % @
  \def\x{\endgroup
    \expandafter\edef\csname HOLOGO@AtEnd\endcsname{%
      \endlinechar=\the\endlinechar\relax
      \catcode13=\the\catcode13\relax
      \catcode32=\the\catcode32\relax
      \catcode35=\the\catcode35\relax
      \catcode61=\the\catcode61\relax
      \catcode64=\the\catcode64\relax
      \catcode123=\the\catcode123\relax
      \catcode125=\the\catcode125\relax
    }%
  }%
\x\catcode61\catcode48\catcode32=10\relax%
\catcode13=5 % ^^M
\endlinechar=13 %
\catcode35=6 % #
\catcode64=11 % @
\catcode123=1 % {
\catcode125=2 % }
\def\TMP@EnsureCode#1#2{%
  \edef\HOLOGO@AtEnd{%
    \HOLOGO@AtEnd
    \catcode#1=\the\catcode#1\relax
  }%
  \catcode#1=#2\relax
}
\TMP@EnsureCode{10}{12}% ^^J
\TMP@EnsureCode{33}{12}% !
\TMP@EnsureCode{34}{12}% "
\TMP@EnsureCode{36}{3}% $
\TMP@EnsureCode{38}{4}% &
\TMP@EnsureCode{39}{12}% '
\TMP@EnsureCode{40}{12}% (
\TMP@EnsureCode{41}{12}% )
\TMP@EnsureCode{42}{12}% *
\TMP@EnsureCode{43}{12}% +
\TMP@EnsureCode{44}{12}% ,
\TMP@EnsureCode{45}{12}% -
\TMP@EnsureCode{46}{12}% .
\TMP@EnsureCode{47}{12}% /
\TMP@EnsureCode{58}{12}% :
\TMP@EnsureCode{59}{12}% ;
\TMP@EnsureCode{60}{12}% <
\TMP@EnsureCode{62}{12}% >
\TMP@EnsureCode{63}{12}% ?
\TMP@EnsureCode{91}{12}% [
\TMP@EnsureCode{93}{12}% ]
\TMP@EnsureCode{94}{7}% ^ (superscript)
\TMP@EnsureCode{95}{8}% _ (subscript)
\TMP@EnsureCode{96}{12}% `
\TMP@EnsureCode{124}{12}% |
\edef\HOLOGO@AtEnd{%
  \HOLOGO@AtEnd
  \escapechar\the\escapechar\relax
  \noexpand\endinput
}
\escapechar=92 %
%    \end{macrocode}
%
% \subsection{Logo list}
%
%    \begin{macro}{\hologoList}
%    \begin{macrocode}
\def\hologoList{%
  \hologoEntry{(La)TeX}{}{2011/10/01}%
  \hologoEntry{AmSLaTeX}{}{2010/04/16}%
  \hologoEntry{AmSTeX}{}{2010/04/16}%
  \hologoEntry{biber}{}{2011/10/01}%
  \hologoEntry{BibTeX}{}{2011/10/01}%
  \hologoEntry{BibTeX}{sf}{2011/10/01}%
  \hologoEntry{BibTeX}{sc}{2011/10/01}%
  \hologoEntry{BibTeX8}{}{2011/11/22}%
  \hologoEntry{ConTeXt}{}{2011/03/25}%
  \hologoEntry{ConTeXt}{narrow}{2011/03/25}%
  \hologoEntry{ConTeXt}{simple}{2011/03/25}%
  \hologoEntry{emTeX}{}{2010/04/26}%
  \hologoEntry{eTeX}{}{2010/04/08}%
  \hologoEntry{ExTeX}{}{2011/10/01}%
  \hologoEntry{HanTheThanh}{}{2011/11/29}%
  \hologoEntry{iniTeX}{}{2011/10/01}%
  \hologoEntry{KOMAScript}{}{2011/10/01}%
  \hologoEntry{La}{}{2010/05/08}%
  \hologoEntry{LaTeX}{}{2010/04/08}%
  \hologoEntry{LaTeX2e}{}{2010/04/08}%
  \hologoEntry{LaTeX3}{}{2010/04/24}%
  \hologoEntry{LaTeXe}{}{2010/04/08}%
  \hologoEntry{LaTeXML}{}{2011/11/22}%
  \hologoEntry{LaTeXTeX}{}{2011/10/01}%
  \hologoEntry{LuaLaTeX}{}{2010/04/08}%
  \hologoEntry{LuaTeX}{}{2010/04/08}%
  \hologoEntry{LyX}{}{2011/10/01}%
  \hologoEntry{METAFONT}{}{2011/10/01}%
  \hologoEntry{MetaFun}{}{2011/10/01}%
  \hologoEntry{METAPOST}{}{2011/10/01}%
  \hologoEntry{MetaPost}{}{2011/10/01}%
  \hologoEntry{MiKTeX}{}{2011/10/01}%
  \hologoEntry{NTS}{}{2011/10/01}%
  \hologoEntry{OzMF}{}{2011/10/01}%
  \hologoEntry{OzMP}{}{2011/10/01}%
  \hologoEntry{OzTeX}{}{2011/10/01}%
  \hologoEntry{OzTtH}{}{2011/10/01}%
  \hologoEntry{PCTeX}{}{2011/10/01}%
  \hologoEntry{pdfTeX}{}{2011/10/01}%
  \hologoEntry{pdfLaTeX}{}{2011/10/01}%
  \hologoEntry{PiC}{}{2011/10/01}%
  \hologoEntry{PiCTeX}{}{2011/10/01}%
  \hologoEntry{plainTeX}{}{2010/04/08}%
  \hologoEntry{plainTeX}{space}{2010/04/16}%
  \hologoEntry{plainTeX}{hyphen}{2010/04/16}%
  \hologoEntry{plainTeX}{runtogether}{2010/04/16}%
  \hologoEntry{SageTeX}{}{2011/11/22}%
  \hologoEntry{SLiTeX}{}{2011/10/01}%
  \hologoEntry{SLiTeX}{lift}{2011/10/01}%
  \hologoEntry{SLiTeX}{narrow}{2011/10/01}%
  \hologoEntry{SLiTeX}{simple}{2011/10/01}%
  \hologoEntry{SliTeX}{}{2011/10/01}%
  \hologoEntry{SliTeX}{narrow}{2011/10/01}%
  \hologoEntry{SliTeX}{simple}{2011/10/01}%
  \hologoEntry{SliTeX}{lift}{2011/10/01}%
  \hologoEntry{teTeX}{}{2011/10/01}%
  \hologoEntry{TeX}{}{2010/04/08}%
  \hologoEntry{TeX4ht}{}{2011/11/22}%
  \hologoEntry{TTH}{}{2011/11/22}%
  \hologoEntry{virTeX}{}{2011/10/01}%
  \hologoEntry{VTeX}{}{2010/04/24}%
  \hologoEntry{Xe}{}{2010/04/08}%
  \hologoEntry{XeLaTeX}{}{2010/04/08}%
  \hologoEntry{XeTeX}{}{2010/04/08}%
}
%    \end{macrocode}
%    \end{macro}
%
% \subsection{Load resources}
%
%    \begin{macrocode}
\begingroup\expandafter\expandafter\expandafter\endgroup
\expandafter\ifx\csname RequirePackage\endcsname\relax
  \def\TMP@RequirePackage#1[#2]{%
    \begingroup\expandafter\expandafter\expandafter\endgroup
    \expandafter\ifx\csname ver@#1.sty\endcsname\relax
      \input #1.sty\relax
    \fi
  }%
  \TMP@RequirePackage{ltxcmds}[2011/02/04]%
  \TMP@RequirePackage{infwarerr}[2010/04/08]%
  \TMP@RequirePackage{kvsetkeys}[2010/03/01]%
  \TMP@RequirePackage{kvdefinekeys}[2010/03/01]%
  \TMP@RequirePackage{pdftexcmds}[2010/04/01]%
  \TMP@RequirePackage{ifpdf}[2010/01/28]%
  \TMP@RequirePackage{ifluatex}[2010/03/01]%
  \ltx@IfUndefined{newif}{%
    \expandafter\let\csname newif\endcsname\ltx@newif
  }{}%
  \TMP@RequirePackage{ifxetex}[2009/01/23]%
  \TMP@RequirePackage{ifvtex}[2010/03/01]%
\else
  \RequirePackage{ltxcmds}[2011/02/04]%
  \RequirePackage{infwarerr}[2010/04/08]%
  \RequirePackage{kvsetkeys}[2010/03/01]%
  \RequirePackage{kvdefinekeys}[2010/03/01]%
  \RequirePackage{pdftexcmds}[2010/04/01]%
  \RequirePackage{ifpdf}[2010/01/28]%
  \RequirePackage{ifluatex}[2010/03/01]%
  \RequirePackage{ifxetex}[2009/01/23]%
  \RequirePackage{ifvtex}[2010/03/01]%
\fi
%    \end{macrocode}
%
%    \begin{macro}{\HOLOGO@IfDefined}
%    \begin{macrocode}
\def\HOLOGO@IfExists#1{%
  \ifx\@undefined#1%
    \expandafter\ltx@secondoftwo
  \else
    \ifx\relax#1%
      \expandafter\ltx@secondoftwo
    \else
      \expandafter\expandafter\expandafter\ltx@firstoftwo
    \fi
  \fi
}
%    \end{macrocode}
%    \end{macro}
%
% \subsection{Setup macros}
%
%    \begin{macro}{\hologoSetup}
%    \begin{macrocode}
\def\hologoSetup{%
  \let\HOLOGO@name\relax
  \HOLOGO@Setup
}
%    \end{macrocode}
%    \end{macro}
%
%    \begin{macro}{\hologoLogoSetup}
%    \begin{macrocode}
\def\hologoLogoSetup#1{%
  \edef\HOLOGO@name{#1}%
  \ltx@IfUndefined{HoLogo@\HOLOGO@name}{%
    \@PackageError{hologo}{%
      Unknown logo `\HOLOGO@name'%
    }\@ehc
    \ltx@gobble
  }{%
    \HOLOGO@Setup
  }%
}
%    \end{macrocode}
%    \end{macro}
%
%    \begin{macro}{\HOLOGO@Setup}
%    \begin{macrocode}
\def\HOLOGO@Setup{%
  \kvsetkeys{HoLogo}%
}
%    \end{macrocode}
%    \end{macro}
%
% \subsection{Options}
%
%    \begin{macro}{\HOLOGO@DeclareBoolOption}
%    \begin{macrocode}
\def\HOLOGO@DeclareBoolOption#1{%
  \expandafter\chardef\csname HOLOGOOPT@#1\endcsname\ltx@zero
  \kv@define@key{HoLogo}{#1}[true]{%
    \def\HOLOGO@temp{##1}%
    \ifx\HOLOGO@temp\HOLOGO@true
      \ifx\HOLOGO@name\relax
        \expandafter\chardef\csname HOLOGOOPT@#1\endcsname=\ltx@one
      \else
        \expandafter\chardef\csname
        HoLogoOpt@#1@\HOLOGO@name\endcsname\ltx@one
      \fi
      \HOLOGO@SetBreakAll{#1}%
    \else
      \ifx\HOLOGO@temp\HOLOGO@false
        \ifx\HOLOGO@name\relax
          \expandafter\chardef\csname HOLOGOOPT@#1\endcsname=\ltx@zero
        \else
          \expandafter\chardef\csname
          HoLogoOpt@#1@\HOLOGO@name\endcsname=\ltx@zero
        \fi
        \HOLOGO@SetBreakAll{#1}%
      \else
        \@PackageError{hologo}{%
          Unknown value `##1' for boolean option `#1'.\MessageBreak
          Known values are `true' and `false'%
        }\@ehc
      \fi
    \fi
  }%
}
%    \end{macrocode}
%    \end{macro}
%
%    \begin{macro}{\HOLOGO@SetBreakAll}
%    \begin{macrocode}
\def\HOLOGO@SetBreakAll#1{%
  \def\HOLOGO@temp{#1}%
  \ifx\HOLOGO@temp\HOLOGO@break
    \ifx\HOLOGO@name\relax
      \chardef\HOLOGOOPT@hyphenbreak=\HOLOGOOPT@break
      \chardef\HOLOGOOPT@spacebreak=\HOLOGOOPT@break
      \chardef\HOLOGOOPT@discretionarybreak=\HOLOGOOPT@break
    \else
      \expandafter\chardef
         \csname HoLogoOpt@hyphenbreak@\HOLOGO@name\endcsname=%
         \csname HoLogoOpt@break@\HOLOGO@name\endcsname
      \expandafter\chardef
         \csname HoLogoOpt@spacebreak@\HOLOGO@name\endcsname=%
         \csname HoLogoOpt@break@\HOLOGO@name\endcsname
      \expandafter\chardef
         \csname HoLogoOpt@discretionarybreak@\HOLOGO@name
             \endcsname=%
         \csname HoLogoOpt@break@\HOLOGO@name\endcsname
    \fi
  \fi
}
%    \end{macrocode}
%    \end{macro}
%
%    \begin{macro}{\HOLOGO@true}
%    \begin{macrocode}
\def\HOLOGO@true{true}
%    \end{macrocode}
%    \end{macro}
%    \begin{macro}{\HOLOGO@false}
%    \begin{macrocode}
\def\HOLOGO@false{false}
%    \end{macrocode}
%    \end{macro}
%    \begin{macro}{\HOLOGO@break}
%    \begin{macrocode}
\def\HOLOGO@break{break}
%    \end{macrocode}
%    \end{macro}
%
%    \begin{macrocode}
\HOLOGO@DeclareBoolOption{break}
\HOLOGO@DeclareBoolOption{hyphenbreak}
\HOLOGO@DeclareBoolOption{spacebreak}
\HOLOGO@DeclareBoolOption{discretionarybreak}
%    \end{macrocode}
%
%    \begin{macrocode}
\kv@define@key{HoLogo}{variant}{%
  \ifx\HOLOGO@name\relax
    \@PackageError{hologo}{%
      Option `variant' is not available in \string\hologoSetup,%
      \MessageBreak
      Use \string\hologoLogoSetup\space instead%
    }\@ehc
  \else
    \edef\HOLOGO@temp{#1}%
    \ifx\HOLOGO@temp\ltx@empty
      \expandafter
      \let\csname HoLogoOpt@variant@\HOLOGO@name\endcsname\@undefined
    \else
      \ltx@IfUndefined{HoLogo@\HOLOGO@name @\HOLOGO@temp}{%
        \@PackageError{hologo}{%
          Unknown variant `\HOLOGO@temp' of logo `\HOLOGO@name'%
        }\@ehc
      }{%
        \expandafter
        \let\csname HoLogoOpt@variant@\HOLOGO@name\endcsname
            \HOLOGO@temp
      }%
    \fi
  \fi
}
%    \end{macrocode}
%
%    \begin{macro}{\HOLOGO@Variant}
%    \begin{macrocode}
\def\HOLOGO@Variant#1{%
  #1%
  \ltx@ifundefined{HoLogoOpt@variant@#1}{%
  }{%
    @\csname HoLogoOpt@variant@#1\endcsname
  }%
}
%    \end{macrocode}
%    \end{macro}
%
% \subsection{Break/no-break support}
%
%    \begin{macro}{\HOLOGO@space}
%    \begin{macrocode}
\def\HOLOGO@space{%
  \ltx@ifundefined{HoLogoOpt@spacebreak@\HOLOGO@name}{%
    \ltx@ifundefined{HoLogoOpt@break@\HOLOGO@name}{%
      \chardef\HOLOGO@temp=\HOLOGOOPT@spacebreak
    }{%
      \chardef\HOLOGO@temp=%
        \csname HoLogoOpt@break@\HOLOGO@name\endcsname
    }%
  }{%
    \chardef\HOLOGO@temp=%
      \csname HoLogoOpt@spacebreak@\HOLOGO@name\endcsname
  }%
  \ifcase\HOLOGO@temp
    \penalty10000 %
  \fi
  \ltx@space
}
%    \end{macrocode}
%    \end{macro}
%
%    \begin{macro}{\HOLOGO@hyphen}
%    \begin{macrocode}
\def\HOLOGO@hyphen{%
  \ltx@ifundefined{HoLogoOpt@hyphenbreak@\HOLOGO@name}{%
    \ltx@ifundefined{HoLogoOpt@break@\HOLOGO@name}{%
      \chardef\HOLOGO@temp=\HOLOGOOPT@hyphenbreak
    }{%
      \chardef\HOLOGO@temp=%
        \csname HoLogoOpt@break@\HOLOGO@name\endcsname
    }%
  }{%
    \chardef\HOLOGO@temp=%
      \csname HoLogoOpt@hyphenbreak@\HOLOGO@name\endcsname
  }%
  \ifcase\HOLOGO@temp
    \ltx@mbox{-}%
  \else
    -%
  \fi
}
%    \end{macrocode}
%    \end{macro}
%
%    \begin{macro}{\HOLOGO@discretionary}
%    \begin{macrocode}
\def\HOLOGO@discretionary{%
  \ltx@ifundefined{HoLogoOpt@discretionarybreak@\HOLOGO@name}{%
    \ltx@ifundefined{HoLogoOpt@break@\HOLOGO@name}{%
      \chardef\HOLOGO@temp=\HOLOGOOPT@discretionarybreak
    }{%
      \chardef\HOLOGO@temp=%
        \csname HoLogoOpt@break@\HOLOGO@name\endcsname
    }%
  }{%
    \chardef\HOLOGO@temp=%
      \csname HoLogoOpt@discretionarybreak@\HOLOGO@name\endcsname
  }%
  \ifcase\HOLOGO@temp
  \else
    \-%
  \fi
}
%    \end{macrocode}
%    \end{macro}
%
%    \begin{macro}{\HOLOGO@mbox}
%    \begin{macrocode}
\def\HOLOGO@mbox#1{%
  \ltx@ifundefined{HoLogoOpt@break@\HOLOGO@name}{%
    \chardef\HOLOGO@temp=\HOLOGOOPT@hyphenbreak
  }{%
    \chardef\HOLOGO@temp=%
      \csname HoLogoOpt@break@\HOLOGO@name\endcsname
  }%
  \ifcase\HOLOGO@temp
    \ltx@mbox{#1}%
  \else
    #1%
  \fi
}
%    \end{macrocode}
%    \end{macro}
%
% \subsection{Font support}
%
%    \begin{macro}{\HoLogoFont@font}
%    \begin{tabular}{@{}ll@{}}
%    |#1|:& logo name\\
%    |#2|:& font short name\\
%    |#3|:& text
%    \end{tabular}
%    \begin{macrocode}
\def\HoLogoFont@font#1#2#3{%
  \begingroup
    \ltx@IfUndefined{HoLogoFont@logo@#1.#2}{%
      \ltx@IfUndefined{HoLogoFont@font@#2}{%
        \@PackageWarning{hologo}{%
          Missing font `#2' for logo `#1'%
        }%
        #3%
      }{%
        \csname HoLogoFont@font@#2\endcsname{#3}%
      }%
    }{%
      \csname HoLogoFont@logo@#1.#2\endcsname{#3}%
    }%
  \endgroup
}
%    \end{macrocode}
%    \end{macro}
%
%    \begin{macro}{\HoLogoFont@Def}
%    \begin{macrocode}
\def\HoLogoFont@Def#1{%
  \expandafter\def\csname HoLogoFont@font@#1\endcsname
}
%    \end{macrocode}
%    \end{macro}
%    \begin{macro}{\HoLogoFont@LogoDef}
%    \begin{macrocode}
\def\HoLogoFont@LogoDef#1#2{%
  \expandafter\def\csname HoLogoFont@logo@#1.#2\endcsname
}
%    \end{macrocode}
%    \end{macro}
%
% \subsubsection{Font defaults}
%
%    \begin{macro}{\HoLogoFont@font@general}
%    \begin{macrocode}
\HoLogoFont@Def{general}{}%
%    \end{macrocode}
%    \end{macro}
%
%    \begin{macro}{\HoLogoFont@font@rm}
%    \begin{macrocode}
\ltx@IfUndefined{rmfamily}{%
  \ltx@IfUndefined{rm}{%
  }{%
    \HoLogoFont@Def{rm}{\rm}%
  }%
}{%
  \HoLogoFont@Def{rm}{\rmfamily}%
}
%    \end{macrocode}
%    \end{macro}
%
%    \begin{macro}{\HoLogoFont@font@sf}
%    \begin{macrocode}
\ltx@IfUndefined{sffamily}{%
  \ltx@IfUndefined{sf}{%
  }{%
    \HoLogoFont@Def{sf}{\sf}%
  }%
}{%
  \HoLogoFont@Def{sf}{\sffamily}%
}
%    \end{macrocode}
%    \end{macro}
%
%    \begin{macro}{\HoLogoFont@font@bibsf}
%    In case of \hologo{plainTeX} the original small caps
%    variant is used as default. In \hologo{LaTeX}
%    the definition of package \xpackage{dtklogos} \cite{dtklogos}
%    is used.
%\begin{quote}
%\begin{verbatim}
%\DeclareRobustCommand{\BibTeX}{%
%  B%
%  \kern-.05em%
%  \hbox{%
%    $\m@th$% %% force math size calculations
%    \csname S@\f@size\endcsname
%    \fontsize\sf@size\z@
%    \math@fontsfalse
%    \selectfont
%    I%
%    \kern-.025em%
%    B
%  }%
%  \kern-.08em%
%  \-%
%  \TeX
%}
%\end{verbatim}
%\end{quote}
%    \begin{macrocode}
\ltx@IfUndefined{selectfont}{%
  \ltx@IfUndefined{tensc}{%
    \font\tensc=cmcsc10\relax
  }{}%
  \HoLogoFont@Def{bibsf}{\tensc}%
}{%
  \HoLogoFont@Def{bibsf}{%
    $\mathsurround=0pt$%
    \csname S@\f@size\endcsname
    \fontsize\sf@size{0pt}%
    \math@fontsfalse
    \selectfont
  }%
}
%    \end{macrocode}
%    \end{macro}
%
%    \begin{macro}{\HoLogoFont@font@sc}
%    \begin{macrocode}
\ltx@IfUndefined{scshape}{%
  \ltx@IfUndefined{tensc}{%
    \font\tensc=cmcsc10\relax
  }{}%
  \HoLogoFont@Def{sc}{\tensc}%
}{%
  \HoLogoFont@Def{sc}{\scshape}%
}
%    \end{macrocode}
%    \end{macro}
%
%    \begin{macro}{\HoLogoFont@font@sy}
%    \begin{macrocode}
\ltx@IfUndefined{usefont}{%
  \ltx@IfUndefined{tensy}{%
  }{%
    \HoLogoFont@Def{sy}{\tensy}%
  }%
}{%
  \HoLogoFont@Def{sy}{%
    \usefont{OMS}{cmsy}{m}{n}%
  }%
}
%    \end{macrocode}
%    \end{macro}
%
%    \begin{macro}{\HoLogoFont@font@logo}
%    \begin{macrocode}
\begingroup
  \def\x{LaTeX2e}%
\expandafter\endgroup
\ifx\fmtname\x
  \ltx@IfUndefined{logofamily}{%
    \DeclareRobustCommand\logofamily{%
      \not@math@alphabet\logofamily\relax
      \fontencoding{U}%
      \fontfamily{logo}%
      \selectfont
    }%
  }{}%
  \ltx@IfUndefined{logofamily}{%
  }{%
    \HoLogoFont@Def{logo}{\logofamily}%
  }%
\else
  \ltx@IfUndefined{tenlogo}{%
    \font\tenlogo=logo10\relax
  }{}%
  \HoLogoFont@Def{logo}{\tenlogo}%
\fi
%    \end{macrocode}
%    \end{macro}
%
% \subsubsection{Font setup}
%
%    \begin{macro}{\hologoFontSetup}
%    \begin{macrocode}
\def\hologoFontSetup{%
  \let\HOLOGO@name\relax
  \HOLOGO@FontSetup
}
%    \end{macrocode}
%    \end{macro}
%
%    \begin{macro}{\hologoLogoFontSetup}
%    \begin{macrocode}
\def\hologoLogoFontSetup#1{%
  \edef\HOLOGO@name{#1}%
  \ltx@IfUndefined{HoLogo@\HOLOGO@name}{%
    \@PackageError{hologo}{%
      Unknown logo `\HOLOGO@name'%
    }\@ehc
    \ltx@gobble
  }{%
    \HOLOGO@FontSetup
  }%
}
%    \end{macrocode}
%    \end{macro}
%
%    \begin{macro}{\HOLOGO@FontSetup}
%    \begin{macrocode}
\def\HOLOGO@FontSetup{%
  \kvsetkeys{HoLogoFont}%
}
%    \end{macrocode}
%    \end{macro}
%
%    \begin{macrocode}
\def\HOLOGO@temp#1{%
  \kv@define@key{HoLogoFont}{#1}{%
    \ifx\HOLOGO@name\relax
      \HoLogoFont@Def{#1}{##1}%
    \else
      \HoLogoFont@LogoDef\HOLOGO@name{#1}{##1}%
    \fi
  }%
}
\HOLOGO@temp{general}
\HOLOGO@temp{sf}
%    \end{macrocode}
%
% \subsection{Generic logo commands}
%
%    \begin{macrocode}
\HOLOGO@IfExists\hologo{%
  \@PackageError{hologo}{%
    \string\hologo\ltx@space is already defined.\MessageBreak
    Package loading is aborted%
  }\@ehc
  \HOLOGO@AtEnd
}%
\HOLOGO@IfExists\hologoRobust{%
  \@PackageError{hologo}{%
    \string\hologoRobust\ltx@space is already defined.\MessageBreak
    Package loading is aborted%
  }\@ehc
  \HOLOGO@AtEnd
}%
%    \end{macrocode}
%
% \subsubsection{\cs{hologo} and friends}
%
%    \begin{macrocode}
\ifluatex
  \expandafter\ltx@firstofone
\else
  \expandafter\ltx@gobble
\fi
{%
  \ltx@IfUndefined{ifincsname}{%
    \ifnum\luatexversion<36 %
      \expandafter\ltx@gobble
    \else
      \expandafter\ltx@firstofone
    \fi
    {%
      \begingroup
        \ifcase0%
            \directlua{%
              if tex.enableprimitives then %
                tex.enableprimitives('HOLOGO@', {'ifincsname'})%
              else %
                tex.print('1')%
              end%
            }%
            \ifx\HOLOGO@ifincsname\@undefined 1\fi%
            \relax
          \expandafter\ltx@firstofone
        \else
          \endgroup
          \expandafter\ltx@gobble
        \fi
        {%
          \global\let\ifincsname\HOLOGO@ifincsname
        }%
      \HOLOGO@temp
    }%
  }{}%
}
%    \end{macrocode}
%    \begin{macrocode}
\ltx@IfUndefined{ifincsname}{%
  \catcode`$=14 %
}{%
  \catcode`$=9 %
}
%    \end{macrocode}
%
%    \begin{macro}{\hologo}
%    \begin{macrocode}
\def\hologo#1{%
$ \ifincsname
$   \ltx@ifundefined{HoLogoCs@\HOLOGO@Variant{#1}}{%
$     #1%
$   }{%
$     \csname HoLogoCs@\HOLOGO@Variant{#1}\endcsname\ltx@firstoftwo
$   }%
$ \else
    \HOLOGO@IfExists\texorpdfstring\texorpdfstring\ltx@firstoftwo
    {%
      \hologoRobust{#1}%
    }{%
      \ltx@ifundefined{HoLogoBkm@\HOLOGO@Variant{#1}}{%
        \ltx@ifundefined{HoLogo@#1}{?#1?}{#1}%
      }{%
        \csname HoLogoBkm@\HOLOGO@Variant{#1}\endcsname
        \ltx@firstoftwo
      }%
    }%
$ \fi
}
%    \end{macrocode}
%    \end{macro}
%    \begin{macro}{\Hologo}
%    \begin{macrocode}
\def\Hologo#1{%
$ \ifincsname
$   \ltx@ifundefined{HoLogoCs@\HOLOGO@Variant{#1}}{%
$     #1%
$   }{%
$     \csname HoLogoCs@\HOLOGO@Variant{#1}\endcsname\ltx@secondoftwo
$   }%
$ \else
    \HOLOGO@IfExists\texorpdfstring\texorpdfstring\ltx@firstoftwo
    {%
      \HologoRobust{#1}%
    }{%
      \ltx@ifundefined{HoLogoBkm@\HOLOGO@Variant{#1}}{%
        \ltx@ifundefined{HoLogo@#1}{?#1?}{#1}%
      }{%
        \csname HoLogoBkm@\HOLOGO@Variant{#1}\endcsname
        \ltx@secondoftwo
      }%
    }%
$ \fi
}
%    \end{macrocode}
%    \end{macro}
%
%    \begin{macro}{\hologoVariant}
%    \begin{macrocode}
\def\hologoVariant#1#2{%
  \ifx\relax#2\relax
    \hologo{#1}%
  \else
$   \ifincsname
$     \ltx@ifundefined{HoLogoCs@#1@#2}{%
$       #1%
$     }{%
$       \csname HoLogoCs@#1@#2\endcsname\ltx@firstoftwo
$     }%
$   \else
      \HOLOGO@IfExists\texorpdfstring\texorpdfstring\ltx@firstoftwo
      {%
        \hologoVariantRobust{#1}{#2}%
      }{%
        \ltx@ifundefined{HoLogoBkm@#1@#2}{%
          \ltx@ifundefined{HoLogo@#1}{?#1?}{#1}%
        }{%
          \csname HoLogoBkm@#1@#2\endcsname
          \ltx@firstoftwo
        }%
      }%
$   \fi
  \fi
}
%    \end{macrocode}
%    \end{macro}
%    \begin{macro}{\HologoVariant}
%    \begin{macrocode}
\def\HologoVariant#1#2{%
  \ifx\relax#2\relax
    \Hologo{#1}%
  \else
$   \ifincsname
$     \ltx@ifundefined{HoLogoCs@#1@#2}{%
$       #1%
$     }{%
$       \csname HoLogoCs@#1@#2\endcsname\ltx@secondoftwo
$     }%
$   \else
      \HOLOGO@IfExists\texorpdfstring\texorpdfstring\ltx@firstoftwo
      {%
        \HologoVariantRobust{#1}{#2}%
      }{%
        \ltx@ifundefined{HoLogoBkm@#1@#2}{%
          \ltx@ifundefined{HoLogo@#1}{?#1?}{#1}%
        }{%
          \csname HoLogoBkm@#1@#2\endcsname
          \ltx@secondoftwo
        }%
      }%
$   \fi
  \fi
}
%    \end{macrocode}
%    \end{macro}
%
%    \begin{macrocode}
\catcode`\$=3 %
%    \end{macrocode}
%
% \subsubsection{\cs{hologoRobust} and friends}
%
%    \begin{macro}{\hologoRobust}
%    \begin{macrocode}
\ltx@IfUndefined{protected}{%
  \ltx@IfUndefined{DeclareRobustCommand}{%
    \def\hologoRobust#1%
  }{%
    \DeclareRobustCommand*\hologoRobust[1]%
  }%
}{%
  \protected\def\hologoRobust#1%
}%
{%
  \edef\HOLOGO@name{#1}%
  \ltx@IfUndefined{HoLogo@\HOLOGO@Variant\HOLOGO@name}{%
    \@PackageError{hologo}{%
      Unknown logo `\HOLOGO@name'%
    }\@ehc
    ?\HOLOGO@name?%
  }{%
    \ltx@IfUndefined{ver@tex4ht.sty}{%
      \HoLogoFont@font\HOLOGO@name{general}{%
        \csname HoLogo@\HOLOGO@Variant\HOLOGO@name\endcsname
        \ltx@firstoftwo
      }%
    }{%
      \ltx@IfUndefined{HoLogoHtml@\HOLOGO@Variant\HOLOGO@name}{%
        \HOLOGO@name
      }{%
        \csname HoLogoHtml@\HOLOGO@Variant\HOLOGO@name\endcsname
        \ltx@firstoftwo
      }%
    }%
  }%
}
%    \end{macrocode}
%    \end{macro}
%    \begin{macro}{\HologoRobust}
%    \begin{macrocode}
\ltx@IfUndefined{protected}{%
  \ltx@IfUndefined{DeclareRobustCommand}{%
    \def\HologoRobust#1%
  }{%
    \DeclareRobustCommand*\HologoRobust[1]%
  }%
}{%
  \protected\def\HologoRobust#1%
}%
{%
  \edef\HOLOGO@name{#1}%
  \ltx@IfUndefined{HoLogo@\HOLOGO@Variant\HOLOGO@name}{%
    \@PackageError{hologo}{%
      Unknown logo `\HOLOGO@name'%
    }\@ehc
    ?\HOLOGO@name?%
  }{%
    \ltx@IfUndefined{ver@tex4ht.sty}{%
      \HoLogoFont@font\HOLOGO@name{general}{%
        \csname HoLogo@\HOLOGO@Variant\HOLOGO@name\endcsname
        \ltx@secondoftwo
      }%
    }{%
      \ltx@IfUndefined{HoLogoHtml@\HOLOGO@Variant\HOLOGO@name}{%
        \expandafter\HOLOGO@Uppercase\HOLOGO@name
      }{%
        \csname HoLogoHtml@\HOLOGO@Variant\HOLOGO@name\endcsname
        \ltx@secondoftwo
      }%
    }%
  }%
}
%    \end{macrocode}
%    \end{macro}
%    \begin{macro}{\hologoVariantRobust}
%    \begin{macrocode}
\ltx@IfUndefined{protected}{%
  \ltx@IfUndefined{DeclareRobustCommand}{%
    \def\hologoVariantRobust#1#2%
  }{%
    \DeclareRobustCommand*\hologoVariantRobust[2]%
  }%
}{%
  \protected\def\hologoVariantRobust#1#2%
}%
{%
  \begingroup
    \hologoLogoSetup{#1}{variant={#2}}%
    \hologoRobust{#1}%
  \endgroup
}
%    \end{macrocode}
%    \end{macro}
%    \begin{macro}{\HologoVariantRobust}
%    \begin{macrocode}
\ltx@IfUndefined{protected}{%
  \ltx@IfUndefined{DeclareRobustCommand}{%
    \def\HologoVariantRobust#1#2%
  }{%
    \DeclareRobustCommand*\HologoVariantRobust[2]%
  }%
}{%
  \protected\def\HologoVariantRobust#1#2%
}%
{%
  \begingroup
    \hologoLogoSetup{#1}{variant={#2}}%
    \HologoRobust{#1}%
  \endgroup
}
%    \end{macrocode}
%    \end{macro}
%
%    \begin{macro}{\hologorobust}
%    Macro \cs{hologorobust} is only defined for compatibility.
%    Its use is deprecated.
%    \begin{macrocode}
\def\hologorobust{\hologoRobust}
%    \end{macrocode}
%    \end{macro}
%
% \subsection{Helpers}
%
%    \begin{macro}{\HOLOGO@Uppercase}
%    Macro \cs{HOLOGO@Uppercase} is restricted to \cs{uppercase},
%    because \hologo{plainTeX} or \hologo{iniTeX} do not provide
%    \cs{MakeUppercase}.
%    \begin{macrocode}
\def\HOLOGO@Uppercase#1{\uppercase{#1}}
%    \end{macrocode}
%    \end{macro}
%
%    \begin{macro}{\HOLOGO@PdfdocUnicode}
%    \begin{macrocode}
\def\HOLOGO@PdfdocUnicode{%
  \ifx\ifHy@unicode\iftrue
    \expandafter\ltx@secondoftwo
  \else
    \expandafter\ltx@firstoftwo
  \fi
}
%    \end{macrocode}
%    \end{macro}
%
%    \begin{macro}{\HOLOGO@Math}
%    \begin{macrocode}
\def\HOLOGO@MathSetup{%
  \mathsurround0pt\relax
  \HOLOGO@IfExists\f@series{%
    \if b\expandafter\ltx@car\f@series x\@nil
      \csname boldmath\endcsname
   \fi
  }{}%
}
%    \end{macrocode}
%    \end{macro}
%
%    \begin{macro}{\HOLOGO@TempDimen}
%    \begin{macrocode}
\dimendef\HOLOGO@TempDimen=\ltx@zero
%    \end{macrocode}
%    \end{macro}
%    \begin{macro}{\HOLOGO@NegativeKerning}
%    \begin{macrocode}
\def\HOLOGO@NegativeKerning#1{%
  \begingroup
    \HOLOGO@TempDimen=0pt\relax
    \comma@parse@normalized{#1}{%
      \ifdim\HOLOGO@TempDimen=0pt %
        \expandafter\HOLOGO@@NegativeKerning\comma@entry
      \fi
      \ltx@gobble
    }%
    \ifdim\HOLOGO@TempDimen<0pt %
      \kern\HOLOGO@TempDimen
    \fi
  \endgroup
}
%    \end{macrocode}
%    \end{macro}
%    \begin{macro}{\HOLOGO@@NegativeKerning}
%    \begin{macrocode}
\def\HOLOGO@@NegativeKerning#1#2{%
  \setbox\ltx@zero\hbox{#1#2}%
  \HOLOGO@TempDimen=\wd\ltx@zero
  \setbox\ltx@zero\hbox{#1\kern0pt#2}%
  \advance\HOLOGO@TempDimen by -\wd\ltx@zero
}
%    \end{macrocode}
%    \end{macro}
%
%    \begin{macro}{\HOLOGO@SpaceFactor}
%    \begin{macrocode}
\def\HOLOGO@SpaceFactor{%
  \spacefactor1000 %
}
%    \end{macrocode}
%    \end{macro}
%
%    \begin{macro}{\HOLOGO@Span}
%    \begin{macrocode}
\def\HOLOGO@Span#1#2{%
  \HCode{<span class="HoLogo-#1">}%
  #2%
  \HCode{</span>}%
}
%    \end{macrocode}
%    \end{macro}
%
% \subsubsection{Text subscript}
%
%    \begin{macro}{\HOLOGO@SubScript}%
%    \begin{macrocode}
\def\HOLOGO@SubScript#1{%
  \ltx@IfUndefined{textsubscript}{%
    \ltx@IfUndefined{text}{%
      \ltx@mbox{%
        \mathsurround=0pt\relax
        $%
          _{%
            \ltx@IfUndefined{sf@size}{%
              \mathrm{#1}%
            }{%
              \mbox{%
                \fontsize\sf@size{0pt}\selectfont
                #1%
              }%
            }%
          }%
        $%
      }%
    }{%
      \ltx@mbox{%
        \mathsurround=0pt\relax
        $_{\text{#1}}$%
      }%
    }%
  }{%
    \textsubscript{#1}%
  }%
}
%    \end{macrocode}
%    \end{macro}
%
% \subsection{\hologo{TeX} and friends}
%
% \subsubsection{\hologo{TeX}}
%
%    \begin{macro}{\HoLogo@TeX}
%    Source: \hologo{LaTeX} kernel.
%    \begin{macrocode}
\def\HoLogo@TeX#1{%
  T\kern-.1667em\lower.5ex\hbox{E}\kern-.125emX\HOLOGO@SpaceFactor
}
%    \end{macrocode}
%    \end{macro}
%    \begin{macro}{\HoLogoHtml@TeX}
%    \begin{macrocode}
\def\HoLogoHtml@TeX#1{%
  \HoLogoCss@TeX
  \HOLOGO@Span{TeX}{%
    T%
    \HOLOGO@Span{e}{%
      E%
    }%
    X%
  }%
}
%    \end{macrocode}
%    \end{macro}
%    \begin{macro}{\HoLogoCss@TeX}
%    \begin{macrocode}
\def\HoLogoCss@TeX{%
  \Css{%
    span.HoLogo-TeX span.HoLogo-e{%
      position:relative;%
      top:.5ex;%
      margin-left:-.1667em;%
      margin-right:-.125em;%
    }%
  }%
  \Css{%
    a span.HoLogo-TeX span.HoLogo-e{%
      text-decoration:none;%
    }%
  }%
  \global\let\HoLogoCss@TeX\relax
}
%    \end{macrocode}
%    \end{macro}
%
% \subsubsection{\hologo{plainTeX}}
%
%    \begin{macro}{\HoLogo@plainTeX@space}
%    Source: ``The \hologo{TeX}book''
%    \begin{macrocode}
\def\HoLogo@plainTeX@space#1{%
  \HOLOGO@mbox{#1{p}{P}lain}\HOLOGO@space\hologo{TeX}%
}
%    \end{macrocode}
%    \end{macro}
%    \begin{macro}{\HoLogoCs@plainTeX@space}
%    \begin{macrocode}
\def\HoLogoCs@plainTeX@space#1{#1{p}{P}lain TeX}%
%    \end{macrocode}
%    \end{macro}
%    \begin{macro}{\HoLogoBkm@plainTeX@space}
%    \begin{macrocode}
\def\HoLogoBkm@plainTeX@space#1{%
  #1{p}{P}lain \hologo{TeX}%
}
%    \end{macrocode}
%    \end{macro}
%    \begin{macro}{\HoLogoHtml@plainTeX@space}
%    \begin{macrocode}
\def\HoLogoHtml@plainTeX@space#1{%
  #1{p}{P}lain \hologo{TeX}%
}
%    \end{macrocode}
%    \end{macro}
%
%    \begin{macro}{\HoLogo@plainTeX@hyphen}
%    \begin{macrocode}
\def\HoLogo@plainTeX@hyphen#1{%
  \HOLOGO@mbox{#1{p}{P}lain}\HOLOGO@hyphen\hologo{TeX}%
}
%    \end{macrocode}
%    \end{macro}
%    \begin{macro}{\HoLogoCs@plainTeX@hyphen}
%    \begin{macrocode}
\def\HoLogoCs@plainTeX@hyphen#1{#1{p}{P}lain-TeX}
%    \end{macrocode}
%    \end{macro}
%    \begin{macro}{\HoLogoBkm@plainTeX@hyphen}
%    \begin{macrocode}
\def\HoLogoBkm@plainTeX@hyphen#1{%
  #1{p}{P}lain-\hologo{TeX}%
}
%    \end{macrocode}
%    \end{macro}
%    \begin{macro}{\HoLogoHtml@plainTeX@hyphen}
%    \begin{macrocode}
\def\HoLogoHtml@plainTeX@hyphen#1{%
  #1{p}{P}lain-\hologo{TeX}%
}
%    \end{macrocode}
%    \end{macro}
%
%    \begin{macro}{\HoLogo@plainTeX@runtogether}
%    \begin{macrocode}
\def\HoLogo@plainTeX@runtogether#1{%
  \HOLOGO@mbox{#1{p}{P}lain\hologo{TeX}}%
}
%    \end{macrocode}
%    \end{macro}
%    \begin{macro}{\HoLogoCs@plainTeX@runtogether}
%    \begin{macrocode}
\def\HoLogoCs@plainTeX@runtogether#1{#1{p}{P}lainTeX}
%    \end{macrocode}
%    \end{macro}
%    \begin{macro}{\HoLogoBkm@plainTeX@runtogether}
%    \begin{macrocode}
\def\HoLogoBkm@plainTeX@runtogether#1{%
  #1{p}{P}lain\hologo{TeX}%
}
%    \end{macrocode}
%    \end{macro}
%    \begin{macro}{\HoLogoHtml@plainTeX@runtogether}
%    \begin{macrocode}
\def\HoLogoHtml@plainTeX@runtogether#1{%
  #1{p}{P}lain\hologo{TeX}%
}
%    \end{macrocode}
%    \end{macro}
%
%    \begin{macro}{\HoLogo@plainTeX}
%    \begin{macrocode}
\def\HoLogo@plainTeX{\HoLogo@plainTeX@space}
%    \end{macrocode}
%    \end{macro}
%    \begin{macro}{\HoLogoCs@plainTeX}
%    \begin{macrocode}
\def\HoLogoCs@plainTeX{\HoLogoCs@plainTeX@space}
%    \end{macrocode}
%    \end{macro}
%    \begin{macro}{\HoLogoBkm@plainTeX}
%    \begin{macrocode}
\def\HoLogoBkm@plainTeX{\HoLogoBkm@plainTeX@space}
%    \end{macrocode}
%    \end{macro}
%    \begin{macro}{\HoLogoHtml@plainTeX}
%    \begin{macrocode}
\def\HoLogoHtml@plainTeX{\HoLogoHtml@plainTeX@space}
%    \end{macrocode}
%    \end{macro}
%
% \subsubsection{\hologo{LaTeX}}
%
%    Source: \hologo{LaTeX} kernel.
%\begin{quote}
%\begin{verbatim}
%\DeclareRobustCommand{\LaTeX}{%
%  L%
%  \kern-.36em%
%  {%
%    \sbox\z@ T%
%    \vbox to\ht\z@{%
%      \hbox{%
%        \check@mathfonts
%        \fontsize\sf@size\z@
%        \math@fontsfalse
%        \selectfont
%        A%
%      }%
%      \vss
%    }%
%  }%
%  \kern-.15em%
%  \TeX
%}
%\end{verbatim}
%\end{quote}
%
%    \begin{macro}{\HoLogo@La}
%    \begin{macrocode}
\def\HoLogo@La#1{%
  L%
  \kern-.36em%
  \begingroup
    \setbox\ltx@zero\hbox{T}%
    \vbox to\ht\ltx@zero{%
      \hbox{%
        \ltx@ifundefined{check@mathfonts}{%
          \csname sevenrm\endcsname
        }{%
          \check@mathfonts
          \fontsize\sf@size{0pt}%
          \math@fontsfalse\selectfont
        }%
        A%
      }%
      \vss
    }%
  \endgroup
}
%    \end{macrocode}
%    \end{macro}
%
%    \begin{macro}{\HoLogo@LaTeX}
%    Source: \hologo{LaTeX} kernel.
%    \begin{macrocode}
\def\HoLogo@LaTeX#1{%
  \hologo{La}%
  \kern-.15em%
  \hologo{TeX}%
}
%    \end{macrocode}
%    \end{macro}
%    \begin{macro}{\HoLogoHtml@LaTeX}
%    \begin{macrocode}
\def\HoLogoHtml@LaTeX#1{%
  \HoLogoCss@LaTeX
  \HOLOGO@Span{LaTeX}{%
    L%
    \HOLOGO@Span{a}{%
      A%
    }%
    \hologo{TeX}%
  }%
}
%    \end{macrocode}
%    \end{macro}
%    \begin{macro}{\HoLogoCss@LaTeX}
%    \begin{macrocode}
\def\HoLogoCss@LaTeX{%
  \Css{%
    span.HoLogo-LaTeX span.HoLogo-a{%
      position:relative;%
      top:-.5ex;%
      margin-left:-.36em;%
      margin-right:-.15em;%
      font-size:85\%;%
    }%
  }%
  \global\let\HoLogoCss@LaTeX\relax
}
%    \end{macrocode}
%    \end{macro}
%
% \subsubsection{\hologo{(La)TeX}}
%
%    \begin{macro}{\HoLogo@LaTeXTeX}
%    The kerning around the parentheses is taken
%    from package \xpackage{dtklogos} \cite{dtklogos}.
%\begin{quote}
%\begin{verbatim}
%\DeclareRobustCommand{\LaTeXTeX}{%
%  (%
%  \kern-.15em%
%  L%
%  \kern-.36em%
%  {%
%    \sbox\z@ T%
%    \vbox to\ht0{%
%      \hbox{%
%        $\m@th$%
%        \csname S@\f@size\endcsname
%        \fontsize\sf@size\z@
%        \math@fontsfalse
%        \selectfont
%        A%
%      }%
%      \vss
%    }%
%  }%
%  \kern-.2em%
%  )%
%  \kern-.15em%
%  \TeX
%}
%\end{verbatim}
%\end{quote}
%    \begin{macrocode}
\def\HoLogo@LaTeXTeX#1{%
  (%
  \kern-.15em%
  \hologo{La}%
  \kern-.2em%
  )%
  \kern-.15em%
  \hologo{TeX}%
}
%    \end{macrocode}
%    \end{macro}
%    \begin{macro}{\HoLogoBkm@LaTeXTeX}
%    \begin{macrocode}
\def\HoLogoBkm@LaTeXTeX#1{(La)TeX}
%    \end{macrocode}
%    \end{macro}
%
%    \begin{macro}{\HoLogo@(La)TeX}
%    \begin{macrocode}
\expandafter
\let\csname HoLogo@(La)TeX\endcsname\HoLogo@LaTeXTeX
%    \end{macrocode}
%    \end{macro}
%    \begin{macro}{\HoLogoBkm@(La)TeX}
%    \begin{macrocode}
\expandafter
\let\csname HoLogoBkm@(La)TeX\endcsname\HoLogoBkm@LaTeXTeX
%    \end{macrocode}
%    \end{macro}
%    \begin{macro}{\HoLogoHtml@LaTeXTeX}
%    \begin{macrocode}
\def\HoLogoHtml@LaTeXTeX#1{%
  \HoLogoCss@LaTeXTeX
  \HOLOGO@Span{LaTeXTeX}{%
    (%
    \HOLOGO@Span{L}{L}%
    \HOLOGO@Span{a}{A}%
    \HOLOGO@Span{ParenRight}{)}%
    \hologo{TeX}%
  }%
}
%    \end{macrocode}
%    \end{macro}
%    \begin{macro}{\HoLogoHtml@(La)TeX}
%    Kerning after opening parentheses and before closing parentheses
%    is $-0.1$\,em. The original values $-0.15$\,em
%    looked too ugly for a serif font.
%    \begin{macrocode}
\expandafter
\let\csname HoLogoHtml@(La)TeX\endcsname\HoLogoHtml@LaTeXTeX
%    \end{macrocode}
%    \end{macro}
%    \begin{macro}{\HoLogoCss@LaTeXTeX}
%    \begin{macrocode}
\def\HoLogoCss@LaTeXTeX{%
  \Css{%
    span.HoLogo-LaTeXTeX span.HoLogo-L{%
      margin-left:-.1em;%
    }%
  }%
  \Css{%
    span.HoLogo-LaTeXTeX span.HoLogo-a{%
      position:relative;%
      top:-.5ex;%
      margin-left:-.36em;%
      margin-right:-.1em;%
      font-size:85\%;%
    }%
  }%
  \Css{%
    span.HoLogo-LaTeXTeX span.HoLogo-ParenRight{%
      margin-right:-.15em;%
    }%
  }%
  \global\let\HoLogoCss@LaTeXTeX\relax
}
%    \end{macrocode}
%    \end{macro}
%
% \subsubsection{\hologo{LaTeXe}}
%
%    \begin{macro}{\HoLogo@LaTeXe}
%    Source: \hologo{LaTeX} kernel
%    \begin{macrocode}
\def\HoLogo@LaTeXe#1{%
  \hologo{LaTeX}%
  \kern.15em%
  \hbox{%
    \HOLOGO@MathSetup
    2%
    $_{\textstyle\varepsilon}$%
  }%
}
%    \end{macrocode}
%    \end{macro}
%
%    \begin{macro}{\HoLogoCs@LaTeXe}
%    \begin{macrocode}
\ifnum64=`\^^^^0040\relax % test for big chars of LuaTeX/XeTeX
  \catcode`\$=9 %
  \catcode`\&=14 %
\else
  \catcode`\$=14 %
  \catcode`\&=9 %
\fi
\def\HoLogoCs@LaTeXe#1{%
  LaTeX2%
$ \string ^^^^0395%
& e%
}%
\catcode`\$=3 %
\catcode`\&=4 %
%    \end{macrocode}
%    \end{macro}
%
%    \begin{macro}{\HoLogoBkm@LaTeXe}
%    \begin{macrocode}
\def\HoLogoBkm@LaTeXe#1{%
  \hologo{LaTeX}%
  2%
  \HOLOGO@PdfdocUnicode{e}{\textepsilon}%
}
%    \end{macrocode}
%    \end{macro}
%
%    \begin{macro}{\HoLogoHtml@LaTeXe}
%    \begin{macrocode}
\def\HoLogoHtml@LaTeXe#1{%
  \HoLogoCss@LaTeXe
  \HOLOGO@Span{LaTeX2e}{%
    \hologo{LaTeX}%
    \HOLOGO@Span{2}{2}%
    \HOLOGO@Span{e}{%
      \HOLOGO@MathSetup
      \ensuremath{\textstyle\varepsilon}%
    }%
  }%
}
%    \end{macrocode}
%    \end{macro}
%    \begin{macro}{\HoLogoCss@LaTeXe}
%    \begin{macrocode}
\def\HoLogoCss@LaTeXe{%
  \Css{%
    span.HoLogo-LaTeX2e span.HoLogo-2{%
      padding-left:.15em;%
    }%
  }%
  \Css{%
    span.HoLogo-LaTeX2e span.HoLogo-e{%
      position:relative;%
      top:.35ex;%
      text-decoration:none;%
    }%
  }%
  \global\let\HoLogoCss@LaTeXe\relax
}
%    \end{macrocode}
%    \end{macro}
%
%    \begin{macro}{\HoLogo@LaTeX2e}
%    \begin{macrocode}
\expandafter
\let\csname HoLogo@LaTeX2e\endcsname\HoLogo@LaTeXe
%    \end{macrocode}
%    \end{macro}
%    \begin{macro}{\HoLogoCs@LaTeX2e}
%    \begin{macrocode}
\expandafter
\let\csname HoLogoCs@LaTeX2e\endcsname\HoLogoCs@LaTeXe
%    \end{macrocode}
%    \end{macro}
%    \begin{macro}{\HoLogoBkm@LaTeX2e}
%    \begin{macrocode}
\expandafter
\let\csname HoLogoBkm@LaTeX2e\endcsname\HoLogoBkm@LaTeXe
%    \end{macrocode}
%    \end{macro}
%    \begin{macro}{\HoLogoHtml@LaTeX2e}
%    \begin{macrocode}
\expandafter
\let\csname HoLogoHtml@LaTeX2e\endcsname\HoLogoHtml@LaTeXe
%    \end{macrocode}
%    \end{macro}
%
% \subsubsection{\hologo{LaTeX3}}
%
%    \begin{macro}{\HoLogo@LaTeX3}
%    Source: \hologo{LaTeX} kernel
%    \begin{macrocode}
\expandafter\def\csname HoLogo@LaTeX3\endcsname#1{%
  \hologo{LaTeX}%
  3%
}
%    \end{macrocode}
%    \end{macro}
%
%    \begin{macro}{\HoLogoBkm@LaTeX3}
%    \begin{macrocode}
\expandafter\def\csname HoLogoBkm@LaTeX3\endcsname#1{%
  \hologo{LaTeX}%
  3%
}
%    \end{macrocode}
%    \end{macro}
%    \begin{macro}{\HoLogoHtml@LaTeX3}
%    \begin{macrocode}
\expandafter
\let\csname HoLogoHtml@LaTeX3\expandafter\endcsname
\csname HoLogo@LaTeX3\endcsname
%    \end{macrocode}
%    \end{macro}
%
% \subsubsection{\hologo{LaTeXML}}
%
%    \begin{macro}{\HoLogo@LaTeXML}
%    \begin{macrocode}
\def\HoLogo@LaTeXML#1{%
  \HOLOGO@mbox{%
    \hologo{La}%
    \kern-.15em%
    T%
    \kern-.1667em%
    \lower.5ex\hbox{E}%
    \kern-.125em%
    \HoLogoFont@font{LaTeXML}{sc}{xml}%
  }%
}
%    \end{macrocode}
%    \end{macro}
%    \begin{macro}{\HoLogoHtml@pdfLaTeX}
%    \begin{macrocode}
\def\HoLogoHtml@LaTeXML#1{%
  \HOLOGO@Span{LaTeXML}{%
    \HoLogoCss@LaTeX
    \HoLogoCss@TeX
    \HOLOGO@Span{LaTeX}{%
      L%
      \HOLOGO@Span{a}{%
        A%
      }%
    }%
    \HOLOGO@Span{TeX}{%
      T%
      \HOLOGO@Span{e}{%
        E%
      }%
    }%
    \HCode{<span style="font-variant: small-caps;">}%
    xml%
    \HCode{</span>}%
  }%
}
%    \end{macrocode}
%    \end{macro}
%
% \subsubsection{\hologo{eTeX}}
%
%    \begin{macro}{\HoLogo@eTeX}
%    Source: package \xpackage{etex}
%    \begin{macrocode}
\def\HoLogo@eTeX#1{%
  \ltx@mbox{%
    \HOLOGO@MathSetup
    $\varepsilon$%
    -%
    \HOLOGO@NegativeKerning{-T,T-,To}%
    \hologo{TeX}%
  }%
}
%    \end{macrocode}
%    \end{macro}
%    \begin{macro}{\HoLogoCs@eTeX}
%    \begin{macrocode}
\ifnum64=`\^^^^0040\relax % test for big chars of LuaTeX/XeTeX
  \catcode`\$=9 %
  \catcode`\&=14 %
\else
  \catcode`\$=14 %
  \catcode`\&=9 %
\fi
\def\HoLogoCs@eTeX#1{%
$ #1{\string ^^^^0395}{\string ^^^^03b5}%
& #1{e}{E}%
  TeX%
}%
\catcode`\$=3 %
\catcode`\&=4 %
%    \end{macrocode}
%    \end{macro}
%    \begin{macro}{\HoLogoBkm@eTeX}
%    \begin{macrocode}
\def\HoLogoBkm@eTeX#1{%
  \HOLOGO@PdfdocUnicode{#1{e}{E}}{\textepsilon}%
  -%
  \hologo{TeX}%
}
%    \end{macrocode}
%    \end{macro}
%    \begin{macro}{\HoLogoHtml@eTeX}
%    \begin{macrocode}
\def\HoLogoHtml@eTeX#1{%
  \ltx@mbox{%
    \HOLOGO@MathSetup
    $\varepsilon$%
    -%
    \hologo{TeX}%
  }%
}
%    \end{macrocode}
%    \end{macro}
%
% \subsubsection{\hologo{iniTeX}}
%
%    \begin{macro}{\HoLogo@iniTeX}
%    \begin{macrocode}
\def\HoLogo@iniTeX#1{%
  \HOLOGO@mbox{%
    #1{i}{I}ni\hologo{TeX}%
  }%
}
%    \end{macrocode}
%    \end{macro}
%    \begin{macro}{\HoLogoCs@iniTeX}
%    \begin{macrocode}
\def\HoLogoCs@iniTeX#1{#1{i}{I}niTeX}
%    \end{macrocode}
%    \end{macro}
%    \begin{macro}{\HoLogoBkm@iniTeX}
%    \begin{macrocode}
\def\HoLogoBkm@iniTeX#1{%
  #1{i}{I}ni\hologo{TeX}%
}
%    \end{macrocode}
%    \end{macro}
%    \begin{macro}{\HoLogoHtml@iniTeX}
%    \begin{macrocode}
\let\HoLogoHtml@iniTeX\HoLogo@iniTeX
%    \end{macrocode}
%    \end{macro}
%
% \subsubsection{\hologo{virTeX}}
%
%    \begin{macro}{\HoLogo@virTeX}
%    \begin{macrocode}
\def\HoLogo@virTeX#1{%
  \HOLOGO@mbox{%
    #1{v}{V}ir\hologo{TeX}%
  }%
}
%    \end{macrocode}
%    \end{macro}
%    \begin{macro}{\HoLogoCs@virTeX}
%    \begin{macrocode}
\def\HoLogoCs@virTeX#1{#1{v}{V}irTeX}
%    \end{macrocode}
%    \end{macro}
%    \begin{macro}{\HoLogoBkm@virTeX}
%    \begin{macrocode}
\def\HoLogoBkm@virTeX#1{%
  #1{v}{V}ir\hologo{TeX}%
}
%    \end{macrocode}
%    \end{macro}
%    \begin{macro}{\HoLogoHtml@virTeX}
%    \begin{macrocode}
\let\HoLogoHtml@virTeX\HoLogo@virTeX
%    \end{macrocode}
%    \end{macro}
%
% \subsubsection{\hologo{SliTeX}}
%
% \paragraph{Definitions of the three variants.}
%
%    \begin{macro}{\HoLogo@SLiTeX@lift}
%    \begin{macrocode}
\def\HoLogo@SLiTeX@lift#1{%
  \HoLogoFont@font{SliTeX}{rm}{%
    S%
    \kern-.06em%
    L%
    \kern-.18em%
    \raise.32ex\hbox{\HoLogoFont@font{SliTeX}{sc}{i}}%
    \HOLOGO@discretionary
    \kern-.06em%
    \hologo{TeX}%
  }%
}
%    \end{macrocode}
%    \end{macro}
%    \begin{macro}{\HoLogoBkm@SLiTeX@lift}
%    \begin{macrocode}
\def\HoLogoBkm@SLiTeX@lift#1{SLiTeX}
%    \end{macrocode}
%    \end{macro}
%    \begin{macro}{\HoLogoHtml@SLiTeX@lift}
%    \begin{macrocode}
\def\HoLogoHtml@SLiTeX@lift#1{%
  \HoLogoCss@SLiTeX@lift
  \HOLOGO@Span{SLiTeX-lift}{%
    \HoLogoFont@font{SliTeX}{rm}{%
      S%
      \HOLOGO@Span{L}{L}%
      \HOLOGO@Span{i}{i}%
      \hologo{TeX}%
    }%
  }%
}
%    \end{macrocode}
%    \end{macro}
%    \begin{macro}{\HoLogoCss@SLiTeX@lift}
%    \begin{macrocode}
\def\HoLogoCss@SLiTeX@lift{%
  \Css{%
    span.HoLogo-SLiTeX-lift span.HoLogo-L{%
      margin-left:-.06em;%
      margin-right:-.18em;%
    }%
  }%
  \Css{%
    span.HoLogo-SLiTeX-lift span.HoLogo-i{%
      position:relative;%
      top:-.32ex;%
      margin-right:-.06em;%
      font-variant:small-caps;%
    }%
  }%
  \global\let\HoLogoCss@SLiTeX@lift\relax
}
%    \end{macrocode}
%    \end{macro}
%
%    \begin{macro}{\HoLogo@SliTeX@simple}
%    \begin{macrocode}
\def\HoLogo@SliTeX@simple#1{%
  \HoLogoFont@font{SliTeX}{rm}{%
    \ltx@mbox{%
      \HoLogoFont@font{SliTeX}{sc}{Sli}%
    }%
    \HOLOGO@discretionary
    \hologo{TeX}%
  }%
}
%    \end{macrocode}
%    \end{macro}
%    \begin{macro}{\HoLogoBkm@SliTeX@simple}
%    \begin{macrocode}
\def\HoLogoBkm@SliTeX@simple#1{SliTeX}
%    \end{macrocode}
%    \end{macro}
%    \begin{macro}{\HoLogoHtml@SliTeX@simple}
%    \begin{macrocode}
\let\HoLogoHtml@SliTeX@simple\HoLogo@SliTeX@simple
%    \end{macrocode}
%    \end{macro}
%
%    \begin{macro}{\HoLogo@SliTeX@narrow}
%    \begin{macrocode}
\def\HoLogo@SliTeX@narrow#1{%
  \HoLogoFont@font{SliTeX}{rm}{%
    \ltx@mbox{%
      S%
      \kern-.06em%
      \HoLogoFont@font{SliTeX}{sc}{%
        l%
        \kern-.035em%
        i%
      }%
    }%
    \HOLOGO@discretionary
    \kern-.06em%
    \hologo{TeX}%
  }%
}
%    \end{macrocode}
%    \end{macro}
%    \begin{macro}{\HoLogoBkm@SliTeX@narrow}
%    \begin{macrocode}
\def\HoLogoBkm@SliTeX@narrow#1{SliTeX}
%    \end{macrocode}
%    \end{macro}
%    \begin{macro}{\HoLogoHtml@SliTeX@narrow}
%    \begin{macrocode}
\def\HoLogoHtml@SliTeX@narrow#1{%
  \HoLogoCss@SliTeX@narrow
  \HOLOGO@Span{SliTeX-narrow}{%
    \HoLogoFont@font{SliTeX}{rm}{%
      S%
        \HOLOGO@Span{l}{l}%
        \HOLOGO@Span{i}{i}%
      \hologo{TeX}%
    }%
  }%
}
%    \end{macrocode}
%    \end{macro}
%    \begin{macro}{\HoLogoCss@SliTeX@narrow}
%    \begin{macrocode}
\def\HoLogoCss@SliTeX@narrow{%
  \Css{%
    span.HoLogo-SliTeX-narrow span.HoLogo-l{%
      margin-left:-.06em;%
      margin-right:-.035em;%
      font-variant:small-caps;%
    }%
  }%
  \Css{%
    span.HoLogo-SliTeX-narrow span.HoLogo-i{%
      margin-right:-.06em;%
      font-variant:small-caps;%
    }%
  }%
  \global\let\HoLogoCss@SliTeX@narrow\relax
}
%    \end{macrocode}
%    \end{macro}
%
% \paragraph{Macro set completion.}
%
%    \begin{macro}{\HoLogo@SLiTeX@simple}
%    \begin{macrocode}
\def\HoLogo@SLiTeX@simple{\HoLogo@SliTeX@simple}
%    \end{macrocode}
%    \end{macro}
%    \begin{macro}{\HoLogoBkm@SLiTeX@simple}
%    \begin{macrocode}
\def\HoLogoBkm@SLiTeX@simple{\HoLogoBkm@SliTeX@simple}
%    \end{macrocode}
%    \end{macro}
%    \begin{macro}{\HoLogoHtml@SLiTeX@simple}
%    \begin{macrocode}
\def\HoLogoHtml@SLiTeX@simple{\HoLogoHtml@SliTeX@simple}
%    \end{macrocode}
%    \end{macro}
%
%    \begin{macro}{\HoLogo@SLiTeX@narrow}
%    \begin{macrocode}
\def\HoLogo@SLiTeX@narrow{\HoLogo@SliTeX@narrow}
%    \end{macrocode}
%    \end{macro}
%    \begin{macro}{\HoLogoBkm@SLiTeX@narrow}
%    \begin{macrocode}
\def\HoLogoBkm@SLiTeX@narrow{\HoLogoBkm@SliTeX@narrow}
%    \end{macrocode}
%    \end{macro}
%    \begin{macro}{\HoLogoHtml@SLiTeX@narrow}
%    \begin{macrocode}
\def\HoLogoHtml@SLiTeX@narrow{\HoLogoHtml@SliTeX@narrow}
%    \end{macrocode}
%    \end{macro}
%
%    \begin{macro}{\HoLogo@SliTeX@lift}
%    \begin{macrocode}
\def\HoLogo@SliTeX@lift{\HoLogo@SLiTeX@lift}
%    \end{macrocode}
%    \end{macro}
%    \begin{macro}{\HoLogoBkm@SliTeX@lift}
%    \begin{macrocode}
\def\HoLogoBkm@SliTeX@lift{\HoLogoBkm@SLiTeX@lift}
%    \end{macrocode}
%    \end{macro}
%    \begin{macro}{\HoLogoHtml@SliTeX@lift}
%    \begin{macrocode}
\def\HoLogoHtml@SliTeX@lift{\HoLogoHtml@SLiTeX@lift}
%    \end{macrocode}
%    \end{macro}
%
% \paragraph{Defaults.}
%
%    \begin{macro}{\HoLogo@SLiTeX}
%    \begin{macrocode}
\def\HoLogo@SLiTeX{\HoLogo@SLiTeX@lift}
%    \end{macrocode}
%    \end{macro}
%    \begin{macro}{\HoLogoBkm@SLiTeX}
%    \begin{macrocode}
\def\HoLogoBkm@SLiTeX{\HoLogoBkm@SLiTeX@lift}
%    \end{macrocode}
%    \end{macro}
%    \begin{macro}{\HoLogoHtml@SLiTeX}
%    \begin{macrocode}
\def\HoLogoHtml@SLiTeX{\HoLogoHtml@SLiTeX@lift}
%    \end{macrocode}
%    \end{macro}
%
%    \begin{macro}{\HoLogo@SliTeX}
%    \begin{macrocode}
\def\HoLogo@SliTeX{\HoLogo@SliTeX@narrow}
%    \end{macrocode}
%    \end{macro}
%    \begin{macro}{\HoLogoBkm@SliTeX}
%    \begin{macrocode}
\def\HoLogoBkm@SliTeX{\HoLogoBkm@SliTeX@narrow}
%    \end{macrocode}
%    \end{macro}
%    \begin{macro}{\HoLogoHtml@SliTeX}
%    \begin{macrocode}
\def\HoLogoHtml@SliTeX{\HoLogoHtml@SliTeX@narrow}
%    \end{macrocode}
%    \end{macro}
%
% \subsubsection{\hologo{LuaTeX}}
%
%    \begin{macro}{\HoLogo@LuaTeX}
%    The kerning is an idea of Hans Hagen, see mailing list
%    `luatex at tug dot org' in March 2010.
%    \begin{macrocode}
\def\HoLogo@LuaTeX#1{%
  \HOLOGO@mbox{%
    Lua%
    \HOLOGO@NegativeKerning{aT,oT,To}%
    \hologo{TeX}%
  }%
}
%    \end{macrocode}
%    \end{macro}
%    \begin{macro}{\HoLogoHtml@LuaTeX}
%    \begin{macrocode}
\let\HoLogoHtml@LuaTeX\HoLogo@LuaTeX
%    \end{macrocode}
%    \end{macro}
%
% \subsubsection{\hologo{LuaLaTeX}}
%
%    \begin{macro}{\HoLogo@LuaLaTeX}
%    \begin{macrocode}
\def\HoLogo@LuaLaTeX#1{%
  \HOLOGO@mbox{%
    Lua%
    \hologo{LaTeX}%
  }%
}
%    \end{macrocode}
%    \end{macro}
%    \begin{macro}{\HoLogoHtml@LuaLaTeX}
%    \begin{macrocode}
\let\HoLogoHtml@LuaLaTeX\HoLogo@LuaLaTeX
%    \end{macrocode}
%    \end{macro}
%
% \subsubsection{\hologo{XeTeX}, \hologo{XeLaTeX}}
%
%    \begin{macro}{\HOLOGO@IfCharExists}
%    \begin{macrocode}
\ifluatex
  \ifnum\luatexversion<36 %
  \else
    \def\HOLOGO@IfCharExists#1{%
      \ifnum
        \directlua{%
           if luaotfload and luaotfload.aux then
             if luaotfload.aux.font_has_glyph(%
                    font.current(), \number#1) then % 	 
	       tex.print("1") % 	 
	     end % 	 
	   elseif font and font.fonts and font.current then %
            local f = font.fonts[font.current()]%
            if f.characters and f.characters[\number#1] then %
              tex.print("1")%
            end %
          end%
        }0=\ltx@zero
        \expandafter\ltx@secondoftwo
      \else
        \expandafter\ltx@firstoftwo
      \fi
    }%
  \fi
\fi
\ltx@IfUndefined{HOLOGO@IfCharExists}{%
  \def\HOLOGO@@IfCharExists#1{%
    \begingroup
      \tracinglostchars=\ltx@zero
      \setbox\ltx@zero=\hbox{%
        \kern7sp\char#1\relax
        \ifnum\lastkern>\ltx@zero
          \expandafter\aftergroup\csname iffalse\endcsname
        \else
          \expandafter\aftergroup\csname iftrue\endcsname
        \fi
      }%
      % \if{true|false} from \aftergroup
      \endgroup
      \expandafter\ltx@firstoftwo
    \else
      \endgroup
      \expandafter\ltx@secondoftwo
    \fi
  }%
  \ifxetex
    \ltx@IfUndefined{XeTeXfonttype}{}{%
      \ltx@IfUndefined{XeTeXcharglyph}{}{%
        \def\HOLOGO@IfCharExists#1{%
          \ifnum\XeTeXfonttype\font>\ltx@zero
            \expandafter\ltx@firstofthree
          \else
            \expandafter\ltx@gobble
          \fi
          {%
            \ifnum\XeTeXcharglyph#1>\ltx@zero
              \expandafter\ltx@firstoftwo
            \else
              \expandafter\ltx@secondoftwo
            \fi
          }%
          \HOLOGO@@IfCharExists{#1}%
        }%
      }%
    }%
  \fi
}{}
\ltx@ifundefined{HOLOGO@IfCharExists}{%
  \ifnum64=`\^^^^0040\relax % test for big chars of LuaTeX/XeTeX
    \let\HOLOGO@IfCharExists\HOLOGO@@IfCharExists
  \else
    \def\HOLOGO@IfCharExists#1{%
      \ifnum#1>255 %
        \expandafter\ltx@fourthoffour
      \fi
      \HOLOGO@@IfCharExists{#1}%
    }%
  \fi
}{}
%    \end{macrocode}
%    \end{macro}
%
%    \begin{macro}{\HoLogo@Xe}
%    Source: package \xpackage{dtklogos}
%    \begin{macrocode}
\def\HoLogo@Xe#1{%
  X%
  \kern-.1em\relax
  \HOLOGO@IfCharExists{"018E}{%
    \lower.5ex\hbox{\char"018E}%
  }{%
    \chardef\HOLOGO@choice=\ltx@zero
    \ifdim\fontdimen\ltx@one\font>0pt %
      \ltx@IfUndefined{rotatebox}{%
        \ltx@IfUndefined{pgftext}{%
          \ltx@IfUndefined{psscalebox}{%
            \ltx@IfUndefined{HOLOGO@ScaleBox@\hologoDriver}{%
            }{%
              \chardef\HOLOGO@choice=4 %
            }%
          }{%
            \chardef\HOLOGO@choice=3 %
          }%
        }{%
          \chardef\HOLOGO@choice=2 %
        }%
      }{%
        \chardef\HOLOGO@choice=1 %
      }%
      \ifcase\HOLOGO@choice
        \HOLOGO@WarningUnsupportedDriver{Xe}%
        e%
      \or % 1: \rotatebox
        \begingroup
          \setbox\ltx@zero\hbox{\rotatebox{180}{E}}%
          \ltx@LocDimenA=\dp\ltx@zero
          \advance\ltx@LocDimenA by -.5ex\relax
          \raise\ltx@LocDimenA\box\ltx@zero
        \endgroup
      \or % 2: \pgftext
        \lower.5ex\hbox{%
          \pgfpicture
            \pgftext[rotate=180]{E}%
          \endpgfpicture
        }%
      \or % 3: \psscalebox
        \begingroup
          \setbox\ltx@zero\hbox{\psscalebox{-1 -1}{E}}%
          \ltx@LocDimenA=\dp\ltx@zero
          \advance\ltx@LocDimenA by -.5ex\relax
          \raise\ltx@LocDimenA\box\ltx@zero
        \endgroup
      \or % 4: \HOLOGO@PointReflectBox
        \lower.5ex\hbox{\HOLOGO@PointReflectBox{E}}%
      \else
        \@PackageError{hologo}{Internal error (choice/it}\@ehc
      \fi
    \else
      \ltx@IfUndefined{reflectbox}{%
        \ltx@IfUndefined{pgftext}{%
          \ltx@IfUndefined{psscalebox}{%
            \ltx@IfUndefined{HOLOGO@ScaleBox@\hologoDriver}{%
            }{%
              \chardef\HOLOGO@choice=4 %
            }%
          }{%
            \chardef\HOLOGO@choice=3 %
          }%
        }{%
          \chardef\HOLOGO@choice=2 %
        }%
      }{%
        \chardef\HOLOGO@choice=1 %
      }%
      \ifcase\HOLOGO@choice
        \HOLOGO@WarningUnsupportedDriver{Xe}%
        e%
      \or % 1: reflectbox
        \lower.5ex\hbox{%
          \reflectbox{E}%
        }%
      \or % 2: \pgftext
        \lower.5ex\hbox{%
          \pgfpicture
            \pgftransformxscale{-1}%
            \pgftext{E}%
          \endpgfpicture
        }%
      \or % 3: \psscalebox
        \lower.5ex\hbox{%
          \psscalebox{-1 1}{E}%
        }%
      \or % 4: \HOLOGO@Reflectbox
        \lower.5ex\hbox{%
          \HOLOGO@ReflectBox{E}%
        }%
      \else
        \@PackageError{hologo}{Internal error (choice/up)}\@ehc
      \fi
    \fi
  }%
}
%    \end{macrocode}
%    \end{macro}
%    \begin{macro}{\HoLogoHtml@Xe}
%    \begin{macrocode}
\def\HoLogoHtml@Xe#1{%
  \HoLogoCss@Xe
  \HOLOGO@Span{Xe}{%
    X%
    \HOLOGO@Span{e}{%
      \HCode{&\ltx@hashchar x018e;}%
    }%
  }%
}
%    \end{macrocode}
%    \end{macro}
%    \begin{macro}{\HoLogoCss@Xe}
%    \begin{macrocode}
\def\HoLogoCss@Xe{%
  \Css{%
    span.HoLogo-Xe span.HoLogo-e{%
      position:relative;%
      top:.5ex;%
      left-margin:-.1em;%
    }%
  }%
  \global\let\HoLogoCss@Xe\relax
}
%    \end{macrocode}
%    \end{macro}
%
%    \begin{macro}{\HoLogo@XeTeX}
%    \begin{macrocode}
\def\HoLogo@XeTeX#1{%
  \hologo{Xe}%
  \kern-.15em\relax
  \hologo{TeX}%
}
%    \end{macrocode}
%    \end{macro}
%
%    \begin{macro}{\HoLogoHtml@XeTeX}
%    \begin{macrocode}
\def\HoLogoHtml@XeTeX#1{%
  \HoLogoCss@XeTeX
  \HOLOGO@Span{XeTeX}{%
    \hologo{Xe}%
    \hologo{TeX}%
  }%
}
%    \end{macrocode}
%    \end{macro}
%    \begin{macro}{\HoLogoCss@XeTeX}
%    \begin{macrocode}
\def\HoLogoCss@XeTeX{%
  \Css{%
    span.HoLogo-XeTeX span.HoLogo-TeX{%
      margin-left:-.15em;%
    }%
  }%
  \global\let\HoLogoCss@XeTeX\relax
}
%    \end{macrocode}
%    \end{macro}
%
%    \begin{macro}{\HoLogo@XeLaTeX}
%    \begin{macrocode}
\def\HoLogo@XeLaTeX#1{%
  \hologo{Xe}%
  \kern-.13em%
  \hologo{LaTeX}%
}
%    \end{macrocode}
%    \end{macro}
%    \begin{macro}{\HoLogoHtml@XeLaTeX}
%    \begin{macrocode}
\def\HoLogoHtml@XeLaTeX#1{%
  \HoLogoCss@XeLaTeX
  \HOLOGO@Span{XeLaTeX}{%
    \hologo{Xe}%
    \hologo{LaTeX}%
  }%
}
%    \end{macrocode}
%    \end{macro}
%    \begin{macro}{\HoLogoCss@XeLaTeX}
%    \begin{macrocode}
\def\HoLogoCss@XeLaTeX{%
  \Css{%
    span.HoLogo-XeLaTeX span.HoLogo-Xe{%
      margin-right:-.13em;%
    }%
  }%
  \global\let\HoLogoCss@XeLaTeX\relax
}
%    \end{macrocode}
%    \end{macro}
%
% \subsubsection{\hologo{pdfTeX}, \hologo{pdfLaTeX}}
%
%    \begin{macro}{\HoLogo@pdfTeX}
%    \begin{macrocode}
\def\HoLogo@pdfTeX#1{%
  \HOLOGO@mbox{%
    #1{p}{P}df\hologo{TeX}%
  }%
}
%    \end{macrocode}
%    \end{macro}
%    \begin{macro}{\HoLogoCs@pdfTeX}
%    \begin{macrocode}
\def\HoLogoCs@pdfTeX#1{#1{p}{P}dfTeX}
%    \end{macrocode}
%    \end{macro}
%    \begin{macro}{\HoLogoBkm@pdfTeX}
%    \begin{macrocode}
\def\HoLogoBkm@pdfTeX#1{%
  #1{p}{P}df\hologo{TeX}%
}
%    \end{macrocode}
%    \end{macro}
%    \begin{macro}{\HoLogoHtml@pdfTeX}
%    \begin{macrocode}
\let\HoLogoHtml@pdfTeX\HoLogo@pdfTeX
%    \end{macrocode}
%    \end{macro}
%
%    \begin{macro}{\HoLogo@pdfLaTeX}
%    \begin{macrocode}
\def\HoLogo@pdfLaTeX#1{%
  \HOLOGO@mbox{%
    #1{p}{P}df\hologo{LaTeX}%
  }%
}
%    \end{macrocode}
%    \end{macro}
%    \begin{macro}{\HoLogoCs@pdfLaTeX}
%    \begin{macrocode}
\def\HoLogoCs@pdfLaTeX#1{#1{p}{P}dfLaTeX}
%    \end{macrocode}
%    \end{macro}
%    \begin{macro}{\HoLogoBkm@pdfLaTeX}
%    \begin{macrocode}
\def\HoLogoBkm@pdfLaTeX#1{%
  #1{p}{P}df\hologo{LaTeX}%
}
%    \end{macrocode}
%    \end{macro}
%    \begin{macro}{\HoLogoHtml@pdfLaTeX}
%    \begin{macrocode}
\let\HoLogoHtml@pdfLaTeX\HoLogo@pdfLaTeX
%    \end{macrocode}
%    \end{macro}
%
% \subsubsection{\hologo{VTeX}}
%
%    \begin{macro}{\HoLogo@VTeX}
%    \begin{macrocode}
\def\HoLogo@VTeX#1{%
  \HOLOGO@mbox{%
    V\hologo{TeX}%
  }%
}
%    \end{macrocode}
%    \end{macro}
%    \begin{macro}{\HoLogoHtml@VTeX}
%    \begin{macrocode}
\let\HoLogoHtml@VTeX\HoLogo@VTeX
%    \end{macrocode}
%    \end{macro}
%
% \subsubsection{\hologo{AmS}, \dots}
%
%    Source: class \xclass{amsdtx}
%
%    \begin{macro}{\HoLogo@AmS}
%    \begin{macrocode}
\def\HoLogo@AmS#1{%
  \HoLogoFont@font{AmS}{sy}{%
    A%
    \kern-.1667em%
    \lower.5ex\hbox{M}%
    \kern-.125em%
    S%
  }%
}
%    \end{macrocode}
%    \end{macro}
%    \begin{macro}{\HoLogoBkm@AmS}
%    \begin{macrocode}
\def\HoLogoBkm@AmS#1{AmS}
%    \end{macrocode}
%    \end{macro}
%    \begin{macro}{\HoLogoHtml@AmS}
%    \begin{macrocode}
\def\HoLogoHtml@AmS#1{%
  \HoLogoCss@AmS
%  \HoLogoFont@font{AmS}{sy}{%
    \HOLOGO@Span{AmS}{%
      A%
      \HOLOGO@Span{M}{M}%
      S%
    }%
%   }%
}
%    \end{macrocode}
%    \end{macro}
%    \begin{macro}{\HoLogoCss@AmS}
%    \begin{macrocode}
\def\HoLogoCss@AmS{%
  \Css{%
    span.HoLogo-AmS span.HoLogo-M{%
      position:relative;%
      top:.5ex;%
      margin-left:-.1667em;%
      margin-right:-.125em;%
      text-decoration:none;%
    }%
  }%
  \global\let\HoLogoCss@AmS\relax
}
%    \end{macrocode}
%    \end{macro}
%
%    \begin{macro}{\HoLogo@AmSTeX}
%    \begin{macrocode}
\def\HoLogo@AmSTeX#1{%
  \hologo{AmS}%
  \HOLOGO@hyphen
  \hologo{TeX}%
}
%    \end{macrocode}
%    \end{macro}
%    \begin{macro}{\HoLogoBkm@AmSTeX}
%    \begin{macrocode}
\def\HoLogoBkm@AmSTeX#1{AmS-TeX}%
%    \end{macrocode}
%    \end{macro}
%    \begin{macro}{\HoLogoHtml@AmSTeX}
%    \begin{macrocode}
\let\HoLogoHtml@AmSTeX\HoLogo@AmSTeX
%    \end{macrocode}
%    \end{macro}
%
%    \begin{macro}{\HoLogo@AmSLaTeX}
%    \begin{macrocode}
\def\HoLogo@AmSLaTeX#1{%
  \hologo{AmS}%
  \HOLOGO@hyphen
  \hologo{LaTeX}%
}
%    \end{macrocode}
%    \end{macro}
%    \begin{macro}{\HoLogoBkm@AmSLaTeX}
%    \begin{macrocode}
\def\HoLogoBkm@AmSLaTeX#1{AmS-LaTeX}%
%    \end{macrocode}
%    \end{macro}
%    \begin{macro}{\HoLogoHtml@AmSLaTeX}
%    \begin{macrocode}
\let\HoLogoHtml@AmSLaTeX\HoLogo@AmSLaTeX
%    \end{macrocode}
%    \end{macro}
%
% \subsubsection{\hologo{BibTeX}}
%
%    \begin{macro}{\HoLogo@BibTeX@sc}
%    A definition of \hologo{BibTeX} is provided in
%    the documentation source for the manual of \hologo{BibTeX}
%    \cite{btxdoc}.
%\begin{quote}
%\begin{verbatim}
%\def\BibTeX{%
%  {%
%    \rm
%    B%
%    \kern-.05em%
%    {%
%      \sc
%      i%
%      \kern-.025em %
%      b%
%    }%
%    \kern-.08em
%    T%
%    \kern-.1667em%
%    \lower.7ex\hbox{E}%
%    \kern-.125em%
%    X%
%  }%
%}
%\end{verbatim}
%\end{quote}
%    \begin{macrocode}
\def\HoLogo@BibTeX@sc#1{%
  B%
  \kern-.05em%
  \HoLogoFont@font{BibTeX}{sc}{%
    i%
    \kern-.025em%
    b%
  }%
  \HOLOGO@discretionary
  \kern-.08em%
  \hologo{TeX}%
}
%    \end{macrocode}
%    \end{macro}
%    \begin{macro}{\HoLogoHtml@BibTeX@sc}
%    \begin{macrocode}
\def\HoLogoHtml@BibTeX@sc#1{%
  \HoLogoCss@BibTeX@sc
  \HOLOGO@Span{BibTeX-sc}{%
    B%
    \HOLOGO@Span{i}{i}%
    \HOLOGO@Span{b}{b}%
    \hologo{TeX}%
  }%
}
%    \end{macrocode}
%    \end{macro}
%    \begin{macro}{\HoLogoCss@BibTeX@sc}
%    \begin{macrocode}
\def\HoLogoCss@BibTeX@sc{%
  \Css{%
    span.HoLogo-BibTeX-sc span.HoLogo-i{%
      margin-left:-.05em;%
      margin-right:-.025em;%
      font-variant:small-caps;%
    }%
  }%
  \Css{%
    span.HoLogo-BibTeX-sc span.HoLogo-b{%
      margin-right:-.08em;%
      font-variant:small-caps;%
    }%
  }%
  \global\let\HoLogoCss@BibTeX@sc\relax
}
%    \end{macrocode}
%    \end{macro}
%
%    \begin{macro}{\HoLogo@BibTeX@sf}
%    Variant \xoption{sf} avoids trouble with unavailable
%    small caps fonts (e.g., bold versions of Computer Modern or
%    Latin Modern). The definition is taken from
%    package \xpackage{dtklogos} \cite{dtklogos}.
%\begin{quote}
%\begin{verbatim}
%\DeclareRobustCommand{\BibTeX}{%
%  B%
%  \kern-.05em%
%  \hbox{%
%    $\m@th$% %% force math size calculations
%    \csname S@\f@size\endcsname
%    \fontsize\sf@size\z@
%    \math@fontsfalse
%    \selectfont
%    I%
%    \kern-.025em%
%    B
%  }%
%  \kern-.08em%
%  \-%
%  \TeX
%}
%\end{verbatim}
%\end{quote}
%    \begin{macrocode}
\def\HoLogo@BibTeX@sf#1{%
  B%
  \kern-.05em%
  \HoLogoFont@font{BibTeX}{bibsf}{%
    I%
    \kern-.025em%
    B%
  }%
  \HOLOGO@discretionary
  \kern-.08em%
  \hologo{TeX}%
}
%    \end{macrocode}
%    \end{macro}
%    \begin{macro}{\HoLogoHtml@BibTeX@sf}
%    \begin{macrocode}
\def\HoLogoHtml@BibTeX@sf#1{%
  \HoLogoCss@BibTeX@sf
  \HOLOGO@Span{BibTeX-sf}{%
    B%
    \HoLogoFont@font{BibTeX}{bibsf}{%
      \HOLOGO@Span{i}{I}%
      B%
    }%
    \hologo{TeX}%
  }%
}
%    \end{macrocode}
%    \end{macro}
%    \begin{macro}{\HoLogoCss@BibTeX@sf}
%    \begin{macrocode}
\def\HoLogoCss@BibTeX@sf{%
  \Css{%
    span.HoLogo-BibTeX-sf span.HoLogo-i{%
      margin-left:-.05em;%
      margin-right:-.025em;%
    }%
  }%
  \Css{%
    span.HoLogo-BibTeX-sf span.HoLogo-TeX{%
      margin-left:-.08em;%
    }%
  }%
  \global\let\HoLogoCss@BibTeX@sf\relax
}
%    \end{macrocode}
%    \end{macro}
%
%    \begin{macro}{\HoLogo@BibTeX}
%    \begin{macrocode}
\def\HoLogo@BibTeX{\HoLogo@BibTeX@sf}
%    \end{macrocode}
%    \end{macro}
%    \begin{macro}{\HoLogoHtml@BibTeX}
%    \begin{macrocode}
\def\HoLogoHtml@BibTeX{\HoLogoHtml@BibTeX@sf}
%    \end{macrocode}
%    \end{macro}
%
% \subsubsection{\hologo{BibTeX8}}
%
%    \begin{macro}{\HoLogo@BibTeX8}
%    \begin{macrocode}
\expandafter\def\csname HoLogo@BibTeX8\endcsname#1{%
  \hologo{BibTeX}%
  8%
}
%    \end{macrocode}
%    \end{macro}
%
%    \begin{macro}{\HoLogoBkm@BibTeX8}
%    \begin{macrocode}
\expandafter\def\csname HoLogoBkm@BibTeX8\endcsname#1{%
  \hologo{BibTeX}%
  8%
}
%    \end{macrocode}
%    \end{macro}
%    \begin{macro}{\HoLogoHtml@BibTeX8}
%    \begin{macrocode}
\expandafter
\let\csname HoLogoHtml@BibTeX8\expandafter\endcsname
\csname HoLogo@BibTeX8\endcsname
%    \end{macrocode}
%    \end{macro}
%
% \subsubsection{\hologo{ConTeXt}}
%
%    \begin{macro}{\HoLogo@ConTeXt@simple}
%    \begin{macrocode}
\def\HoLogo@ConTeXt@simple#1{%
  \HOLOGO@mbox{Con}%
  \HOLOGO@discretionary
  \HOLOGO@mbox{\hologo{TeX}t}%
}
%    \end{macrocode}
%    \end{macro}
%    \begin{macro}{\HoLogoHtml@ConTeXt@simple}
%    \begin{macrocode}
\let\HoLogoHtml@ConTeXt@simple\HoLogo@ConTeXt@simple
%    \end{macrocode}
%    \end{macro}
%
%    \begin{macro}{\HoLogo@ConTeXt@narrow}
%    This definition of logo \hologo{ConTeXt} with variant \xoption{narrow}
%    comes from TUGboat's class \xclass{ltugboat} (version 2010/11/15 v2.8).
%    \begin{macrocode}
\def\HoLogo@ConTeXt@narrow#1{%
  \HOLOGO@mbox{C\kern-.0333emon}%
  \HOLOGO@discretionary
  \kern-.0667em%
  \HOLOGO@mbox{\hologo{TeX}\kern-.0333emt}%
}
%    \end{macrocode}
%    \end{macro}
%    \begin{macro}{\HoLogoHtml@ConTeXt@narrow}
%    \begin{macrocode}
\def\HoLogoHtml@ConTeXt@narrow#1{%
  \HoLogoCss@ConTeXt@narrow
  \HOLOGO@Span{ConTeXt-narrow}{%
    \HOLOGO@Span{C}{C}%
    on%
    \hologo{TeX}%
    t%
  }%
}
%    \end{macrocode}
%    \end{macro}
%    \begin{macro}{\HoLogoCss@ConTeXt@narrow}
%    \begin{macrocode}
\def\HoLogoCss@ConTeXt@narrow{%
  \Css{%
    span.HoLogo-ConTeXt-narrow span.HoLogo-C{%
      margin-left:-.0333em;%
    }%
  }%
  \Css{%
    span.HoLogo-ConTeXt-narrow span.HoLogo-TeX{%
      margin-left:-.0667em;%
      margin-right:-.0333em;%
    }%
  }%
  \global\let\HoLogoCss@ConTeXt@narrow\relax
}
%    \end{macrocode}
%    \end{macro}
%
%    \begin{macro}{\HoLogo@ConTeXt}
%    \begin{macrocode}
\def\HoLogo@ConTeXt{\HoLogo@ConTeXt@narrow}
%    \end{macrocode}
%    \end{macro}
%    \begin{macro}{\HoLogoHtml@ConTeXt}
%    \begin{macrocode}
\def\HoLogoHtml@ConTeXt{\HoLogoHtml@ConTeXt@narrow}
%    \end{macrocode}
%    \end{macro}
%
% \subsubsection{\hologo{emTeX}}
%
%    \begin{macro}{\HoLogo@emTeX}
%    \begin{macrocode}
\def\HoLogo@emTeX#1{%
  \HOLOGO@mbox{#1{e}{E}m}%
  \HOLOGO@discretionary
  \hologo{TeX}%
}
%    \end{macrocode}
%    \end{macro}
%    \begin{macro}{\HoLogoCs@emTeX}
%    \begin{macrocode}
\def\HoLogoCs@emTeX#1{#1{e}{E}mTeX}%
%    \end{macrocode}
%    \end{macro}
%    \begin{macro}{\HoLogoBkm@emTeX}
%    \begin{macrocode}
\def\HoLogoBkm@emTeX#1{%
  #1{e}{E}m\hologo{TeX}%
}
%    \end{macrocode}
%    \end{macro}
%    \begin{macro}{\HoLogoHtml@emTeX}
%    \begin{macrocode}
\let\HoLogoHtml@emTeX\HoLogo@emTeX
%    \end{macrocode}
%    \end{macro}
%
% \subsubsection{\hologo{ExTeX}}
%
%    \begin{macro}{\HoLogo@ExTeX}
%    The definition is taken from the FAQ of the
%    project \hologo{ExTeX}
%    \cite{ExTeX-FAQ}.
%\begin{quote}
%\begin{verbatim}
%\def\ExTeX{%
%  \textrm{% Logo always with serifs
%    \ensuremath{%
%      \textstyle
%      \varepsilon_{%
%        \kern-0.15em%
%        \mathcal{X}%
%      }%
%    }%
%    \kern-.15em%
%    \TeX
%  }%
%}
%\end{verbatim}
%\end{quote}
%    \begin{macrocode}
\def\HoLogo@ExTeX#1{%
  \HoLogoFont@font{ExTeX}{rm}{%
    \ltx@mbox{%
      \HOLOGO@MathSetup
      $%
        \textstyle
        \varepsilon_{%
          \kern-0.15em%
          \HoLogoFont@font{ExTeX}{sy}{X}%
        }%
      $%
    }%
    \HOLOGO@discretionary
    \kern-.15em%
    \hologo{TeX}%
  }%
}
%    \end{macrocode}
%    \end{macro}
%    \begin{macro}{\HoLogoHtml@ExTeX}
%    \begin{macrocode}
\def\HoLogoHtml@ExTeX#1{%
  \HoLogoCss@ExTeX
  \HoLogoFont@font{ExTeX}{rm}{%
    \HOLOGO@Span{ExTeX}{%
      \ltx@mbox{%
        \HOLOGO@MathSetup
        $\textstyle\varepsilon$%
        \HOLOGO@Span{X}{$\textstyle\chi$}%
        \hologo{TeX}%
      }%
    }%
  }%
}
%    \end{macrocode}
%    \end{macro}
%    \begin{macro}{\HoLogoBkm@ExTeX}
%    \begin{macrocode}
\def\HoLogoBkm@ExTeX#1{%
  \HOLOGO@PdfdocUnicode{#1{e}{E}x}{\textepsilon\textchi}%
  \hologo{TeX}%
}
%    \end{macrocode}
%    \end{macro}
%    \begin{macro}{\HoLogoCss@ExTeX}
%    \begin{macrocode}
\def\HoLogoCss@ExTeX{%
  \Css{%
    span.HoLogo-ExTeX{%
      font-family:serif;%
    }%
  }%
  \Css{%
    span.HoLogo-ExTeX span.HoLogo-TeX{%
      margin-left:-.15em;%
    }%
  }%
  \global\let\HoLogoCss@ExTeX\relax
}
%    \end{macrocode}
%    \end{macro}
%
% \subsubsection{\hologo{MiKTeX}}
%
%    \begin{macro}{\HoLogo@MiKTeX}
%    \begin{macrocode}
\def\HoLogo@MiKTeX#1{%
  \HOLOGO@mbox{MiK}%
  \HOLOGO@discretionary
  \hologo{TeX}%
}
%    \end{macrocode}
%    \end{macro}
%    \begin{macro}{\HoLogoHtml@MiKTeX}
%    \begin{macrocode}
\let\HoLogoHtml@MiKTeX\HoLogo@MiKTeX
%    \end{macrocode}
%    \end{macro}
%
% \subsubsection{\hologo{OzTeX} and friends}
%
%    Source: \hologo{OzTeX} FAQ \cite{OzTeX}:
%    \begin{quote}
%      |\def\OzTeX{O\kern-.03em z\kern-.15em\TeX}|\\
%      (There is no kerning in OzMF, OzMP and OzTtH.)
%    \end{quote}
%
%    \begin{macro}{\HoLogo@OzTeX}
%    \begin{macrocode}
\def\HoLogo@OzTeX#1{%
  O%
  \kern-.03em %
  z%
  \kern-.15em %
  \hologo{TeX}%
}
%    \end{macrocode}
%    \end{macro}
%    \begin{macro}{\HoLogoHtml@OzTeX}
%    \begin{macrocode}
\def\HoLogoHtml@OzTeX#1{%
  \HoLogoCss@OzTeX
  \HOLOGO@Span{OzTeX}{%
    O%
    \HOLOGO@Span{z}{z}%
    \hologo{TeX}%
  }%
}
%    \end{macrocode}
%    \end{macro}
%    \begin{macro}{\HoLogoCss@OzTeX}
%    \begin{macrocode}
\def\HoLogoCss@OzTeX{%
  \Css{%
    span.HoLogo-OzTeX span.HoLogo-z{%
      margin-left:-.03em;%
      margin-right:-.15em;%
    }%
  }%
  \global\let\HoLogoCss@OzTeX\relax
}
%    \end{macrocode}
%    \end{macro}
%
%    \begin{macro}{\HoLogo@OzMF}
%    \begin{macrocode}
\def\HoLogo@OzMF#1{%
  \HOLOGO@mbox{OzMF}%
}
%    \end{macrocode}
%    \end{macro}
%    \begin{macro}{\HoLogo@OzMP}
%    \begin{macrocode}
\def\HoLogo@OzMP#1{%
  \HOLOGO@mbox{OzMP}%
}
%    \end{macrocode}
%    \end{macro}
%    \begin{macro}{\HoLogo@OzTtH}
%    \begin{macrocode}
\def\HoLogo@OzTtH#1{%
  \HOLOGO@mbox{OzTtH}%
}
%    \end{macrocode}
%    \end{macro}
%
% \subsubsection{\hologo{PCTeX}}
%
%    \begin{macro}{\HoLogo@PCTeX}
%    \begin{macrocode}
\def\HoLogo@PCTeX#1{%
  \HOLOGO@mbox{PC}%
  \hologo{TeX}%
}
%    \end{macrocode}
%    \end{macro}
%    \begin{macro}{\HoLogoHtml@PCTeX}
%    \begin{macrocode}
\let\HoLogoHtml@PCTeX\HoLogo@PCTeX
%    \end{macrocode}
%    \end{macro}
%
% \subsubsection{\hologo{PiCTeX}}
%
%    The original definitions from \xfile{pictex.tex} \cite{PiCTeX}:
%\begin{quote}
%\begin{verbatim}
%\def\PiC{%
%  P%
%  \kern-.12em%
%  \lower.5ex\hbox{I}%
%  \kern-.075em%
%  C%
%}
%\def\PiCTeX{%
%  \PiC
%  \kern-.11em%
%  \TeX
%}
%\end{verbatim}
%\end{quote}
%
%    \begin{macro}{\HoLogo@PiC}
%    \begin{macrocode}
\def\HoLogo@PiC#1{%
  P%
  \kern-.12em%
  \lower.5ex\hbox{I}%
  \kern-.075em%
  C%
  \HOLOGO@SpaceFactor
}
%    \end{macrocode}
%    \end{macro}
%    \begin{macro}{\HoLogoHtml@PiC}
%    \begin{macrocode}
\def\HoLogoHtml@PiC#1{%
  \HoLogoCss@PiC
  \HOLOGO@Span{PiC}{%
    P%
    \HOLOGO@Span{i}{I}%
    C%
  }%
}
%    \end{macrocode}
%    \end{macro}
%    \begin{macro}{\HoLogoCss@PiC}
%    \begin{macrocode}
\def\HoLogoCss@PiC{%
  \Css{%
    span.HoLogo-PiC span.HoLogo-i{%
      position:relative;%
      top:.5ex;%
      margin-left:-.12em;%
      margin-right:-.075em;%
      text-decoration:none;%
    }%
  }%
  \global\let\HoLogoCss@PiC\relax
}
%    \end{macrocode}
%    \end{macro}
%
%    \begin{macro}{\HoLogo@PiCTeX}
%    \begin{macrocode}
\def\HoLogo@PiCTeX#1{%
  \hologo{PiC}%
  \HOLOGO@discretionary
  \kern-.11em%
  \hologo{TeX}%
}
%    \end{macrocode}
%    \end{macro}
%    \begin{macro}{\HoLogoHtml@PiCTeX}
%    \begin{macrocode}
\def\HoLogoHtml@PiCTeX#1{%
  \HoLogoCss@PiCTeX
  \HOLOGO@Span{PiCTeX}{%
    \hologo{PiC}%
    \hologo{TeX}%
  }%
}
%    \end{macrocode}
%    \end{macro}
%    \begin{macro}{\HoLogoCss@PiCTeX}
%    \begin{macrocode}
\def\HoLogoCss@PiCTeX{%
  \Css{%
    span.HoLogo-PiCTeX span.HoLogo-PiC{%
      margin-right:-.11em;%
    }%
  }%
  \global\let\HoLogoCss@PiCTeX\relax
}
%    \end{macrocode}
%    \end{macro}
%
% \subsubsection{\hologo{teTeX}}
%
%    \begin{macro}{\HoLogo@teTeX}
%    \begin{macrocode}
\def\HoLogo@teTeX#1{%
  \HOLOGO@mbox{#1{t}{T}e}%
  \HOLOGO@discretionary
  \hologo{TeX}%
}
%    \end{macrocode}
%    \end{macro}
%    \begin{macro}{\HoLogoCs@teTeX}
%    \begin{macrocode}
\def\HoLogoCs@teTeX#1{#1{t}{T}dfTeX}
%    \end{macrocode}
%    \end{macro}
%    \begin{macro}{\HoLogoBkm@teTeX}
%    \begin{macrocode}
\def\HoLogoBkm@teTeX#1{%
  #1{t}{T}e\hologo{TeX}%
}
%    \end{macrocode}
%    \end{macro}
%    \begin{macro}{\HoLogoHtml@teTeX}
%    \begin{macrocode}
\let\HoLogoHtml@teTeX\HoLogo@teTeX
%    \end{macrocode}
%    \end{macro}
%
% \subsubsection{\hologo{TeX4ht}}
%
%    \begin{macro}{\HoLogo@TeX4ht}
%    \begin{macrocode}
\expandafter\def\csname HoLogo@TeX4ht\endcsname#1{%
  \HOLOGO@mbox{\hologo{TeX}4ht}%
}
%    \end{macrocode}
%    \end{macro}
%    \begin{macro}{\HoLogoHtml@TeX4ht}
%    \begin{macrocode}
\expandafter
\let\csname HoLogoHtml@TeX4ht\expandafter\endcsname
\csname HoLogo@TeX4ht\endcsname
%    \end{macrocode}
%    \end{macro}
%
%
% \subsubsection{\hologo{SageTeX}}
%
%    \begin{macro}{\HoLogo@SageTeX}
%    \begin{macrocode}
\def\HoLogo@SageTeX#1{%
  \HOLOGO@mbox{Sage}%
  \HOLOGO@discretionary
  \HOLOGO@NegativeKerning{eT,oT,To}%
  \hologo{TeX}%
}
%    \end{macrocode}
%    \end{macro}
%    \begin{macro}{\HoLogoHtml@SageTeX}
%    \begin{macrocode}
\let\HoLogoHtml@SageTeX\HoLogo@SageTeX
%    \end{macrocode}
%    \end{macro}
%
% \subsection{\hologo{METAFONT} and friends}
%
%    \begin{macro}{\HoLogo@METAFONT}
%    \begin{macrocode}
\def\HoLogo@METAFONT#1{%
  \HoLogoFont@font{METAFONT}{logo}{%
    \HOLOGO@mbox{META}%
    \HOLOGO@discretionary
    \HOLOGO@mbox{FONT}%
  }%
}
%    \end{macrocode}
%    \end{macro}
%
%    \begin{macro}{\HoLogo@METAPOST}
%    \begin{macrocode}
\def\HoLogo@METAPOST#1{%
  \HoLogoFont@font{METAPOST}{logo}{%
    \HOLOGO@mbox{META}%
    \HOLOGO@discretionary
    \HOLOGO@mbox{POST}%
  }%
}
%    \end{macrocode}
%    \end{macro}
%
%    \begin{macro}{\HoLogo@MetaFun}
%    \begin{macrocode}
\def\HoLogo@MetaFun#1{%
  \HOLOGO@mbox{Meta}%
  \HOLOGO@discretionary
  \HOLOGO@mbox{Fun}%
}
%    \end{macrocode}
%    \end{macro}
%
%    \begin{macro}{\HoLogo@MetaPost}
%    \begin{macrocode}
\def\HoLogo@MetaPost#1{%
  \HOLOGO@mbox{Meta}%
  \HOLOGO@discretionary
  \HOLOGO@mbox{Post}%
}
%    \end{macrocode}
%    \end{macro}
%
% \subsection{Others}
%
% \subsubsection{\hologo{biber}}
%
%    \begin{macro}{\HoLogo@biber}
%    \begin{macrocode}
\def\HoLogo@biber#1{%
  \HOLOGO@mbox{#1{b}{B}i}%
  \HOLOGO@discretionary
  \HOLOGO@mbox{ber}%
}
%    \end{macrocode}
%    \end{macro}
%    \begin{macro}{\HoLogoCs@biber}
%    \begin{macrocode}
\def\HoLogoCs@biber#1{#1{b}{B}iber}
%    \end{macrocode}
%    \end{macro}
%    \begin{macro}{\HoLogoBkm@biber}
%    \begin{macrocode}
\def\HoLogoBkm@biber#1{%
  #1{b}{B}iber%
}
%    \end{macrocode}
%    \end{macro}
%    \begin{macro}{\HoLogoHtml@biber}
%    \begin{macrocode}
\let\HoLogoHtml@biber\HoLogo@biber
%    \end{macrocode}
%    \end{macro}
%
% \subsubsection{\hologo{KOMAScript}}
%
%    \begin{macro}{\HoLogo@KOMAScript}
%    The definition for \hologo{KOMAScript} is taken
%    from \hologo{KOMAScript} (\xfile{scrlogo.dtx}, reformatted) \cite{scrlogo}:
%\begin{quote}
%\begin{verbatim}
%\@ifundefined{KOMAScript}{%
%  \DeclareRobustCommand{\KOMAScript}{%
%    \textsf{%
%      K\kern.05em O\kern.05emM\kern.05em A%
%      \kern.1em-\kern.1em %
%      Script%
%    }%
%  }%
%}{}
%\end{verbatim}
%\end{quote}
%    \begin{macrocode}
\def\HoLogo@KOMAScript#1{%
  \HoLogoFont@font{KOMAScript}{sf}{%
    \HOLOGO@mbox{%
      K\kern.05em%
      O\kern.05em%
      M\kern.05em%
      A%
    }%
    \kern.1em%
    \HOLOGO@hyphen
    \kern.1em%
    \HOLOGO@mbox{Script}%
  }%
}
%    \end{macrocode}
%    \end{macro}
%    \begin{macro}{\HoLogoBkm@KOMAScript}
%    \begin{macrocode}
\def\HoLogoBkm@KOMAScript#1{%
  KOMA-Script%
}
%    \end{macrocode}
%    \end{macro}
%    \begin{macro}{\HoLogoHtml@KOMAScript}
%    \begin{macrocode}
\def\HoLogoHtml@KOMAScript#1{%
  \HoLogoCss@KOMAScript
  \HoLogoFont@font{KOMAScript}{sf}{%
    \HOLOGO@Span{KOMAScript}{%
      K%
      \HOLOGO@Span{O}{O}%
      M%
      \HOLOGO@Span{A}{A}%
      \HOLOGO@Span{hyphen}{-}%
      Script%
    }%
  }%
}
%    \end{macrocode}
%    \end{macro}
%    \begin{macro}{\HoLogoCss@KOMAScript}
%    \begin{macrocode}
\def\HoLogoCss@KOMAScript{%
  \Css{%
    span.HoLogo-KOMAScript{%
      font-family:sans-serif;%
    }%
  }%
  \Css{%
    span.HoLogo-KOMAScript span.HoLogo-O{%
      padding-left:.05em;%
      padding-right:.05em;%
    }%
  }%
  \Css{%
    span.HoLogo-KOMAScript span.HoLogo-A{%
      padding-left:.05em;%
    }%
  }%
  \Css{%
    span.HoLogo-KOMAScript span.HoLogo-hyphen{%
      padding-left:.1em;%
      padding-right:.1em;%
    }%
  }%
  \global\let\HoLogoCss@KOMAScript\relax
}
%    \end{macrocode}
%    \end{macro}
%
% \subsubsection{\hologo{LyX}}
%
%    \begin{macro}{\HoLogo@LyX}
%    The definition is taken from the documentation source files
%    of \hologo{LyX}, \xfile{Intro.lyx} \cite{LyX}:
%\begin{quote}
%\begin{verbatim}
%\def\LyX{%
%  \texorpdfstring{%
%    L\kern-.1667em\lower.25em\hbox{Y}\kern-.125emX\@%
%  }{%
%    LyX%
%  }%
%}
%\end{verbatim}
%\end{quote}
%    \begin{macrocode}
\def\HoLogo@LyX#1{%
  L%
  \kern-.1667em%
  \lower.25em\hbox{Y}%
  \kern-.125em%
  X%
  \HOLOGO@SpaceFactor
}
%    \end{macrocode}
%    \end{macro}
%    \begin{macro}{\HoLogoHtml@LyX}
%    \begin{macrocode}
\def\HoLogoHtml@LyX#1{%
  \HoLogoCss@LyX
  \HOLOGO@Span{LyX}{%
    L%
    \HOLOGO@Span{y}{Y}%
    X%
  }%
}
%    \end{macrocode}
%    \end{macro}
%    \begin{macro}{\HoLogoCss@LyX}
%    \begin{macrocode}
\def\HoLogoCss@LyX{%
  \Css{%
    span.HoLogo-LyX span.HoLogo-y{%
      position:relative;%
      top:.25em;%
      margin-left:-.1667em;%
      margin-right:-.125em;%
      text-decoration:none;%
    }%
  }%
  \global\let\HoLogoCss@LyX\relax
}
%    \end{macrocode}
%    \end{macro}
%
% \subsubsection{\hologo{NTS}}
%
%    \begin{macro}{\HoLogo@NTS}
%    Definition for \hologo{NTS} can be found in
%    package \xpackage{etex\textunderscore man} for the \hologo{eTeX} manual \cite{etexman}
%    and in package \xpackage{dtklogos} \cite{dtklogos}:
%\begin{quote}
%\begin{verbatim}
%\def\NTS{%
%  \leavevmode
%  \hbox{%
%    $%
%      \cal N%
%      \kern-0.35em%
%      \lower0.5ex\hbox{$\cal T$}%
%      \kern-0.2em%
%      S%
%    $%
%  }%
%}
%\end{verbatim}
%\end{quote}
%    \begin{macrocode}
\def\HoLogo@NTS#1{%
  \HoLogoFont@font{NTS}{sy}{%
    N\/%
    \kern-.35em%
    \lower.5ex\hbox{T\/}%
    \kern-.2em%
    S\/%
  }%
  \HOLOGO@SpaceFactor
}
%    \end{macrocode}
%    \end{macro}
%
% \subsubsection{\Hologo{TTH} (\hologo{TeX} to HTML translator)}
%
%    Source: \url{http://hutchinson.belmont.ma.us/tth/}
%    In the HTML source the second `T' is printed as subscript.
%\begin{quote}
%\begin{verbatim}
%T<sub>T</sub>H
%\end{verbatim}
%\end{quote}
%    \begin{macro}{\HoLogo@TTH}
%    \begin{macrocode}
\def\HoLogo@TTH#1{%
  \ltx@mbox{%
    T\HOLOGO@SubScript{T}H%
  }%
  \HOLOGO@SpaceFactor
}
%    \end{macrocode}
%    \end{macro}
%
%    \begin{macro}{\HoLogoHtml@TTH}
%    \begin{macrocode}
\def\HoLogoHtml@TTH#1{%
  T\HCode{<sub>}T\HCode{</sub>}H%
}
%    \end{macrocode}
%    \end{macro}
%
% \subsubsection{\Hologo{HanTheThanh}}
%
%    Partial source: Package \xpackage{dtklogos}.
%    The double accent is U+1EBF (latin small letter e with circumflex
%    and acute).
%    \begin{macro}{\HoLogo@HanTheThanh}
%    \begin{macrocode}
\def\HoLogo@HanTheThanh#1{%
  \ltx@mbox{H\`an}%
  \HOLOGO@space
  \ltx@mbox{%
    Th%
    \HOLOGO@IfCharExists{"1EBF}{%
      \char"1EBF\relax
    }{%
      \^e\hbox to 0pt{\hss\raise .5ex\hbox{\'{}}}%
    }%
  }%
  \HOLOGO@space
  \ltx@mbox{Th\`anh}%
}
%    \end{macrocode}
%    \end{macro}
%    \begin{macro}{\HoLogoBkm@HanTheThanh}
%    \begin{macrocode}
\def\HoLogoBkm@HanTheThanh#1{%
  H\`an %
  Th\HOLOGO@PdfdocUnicode{\^e}{\9036\277} %
  Th\`anh%
}
%    \end{macrocode}
%    \end{macro}
%    \begin{macro}{\HoLogoHtml@HanTheThanh}
%    \begin{macrocode}
\def\HoLogoHtml@HanTheThanh#1{%
  H\`an %
  Th\HCode{&\ltx@hashchar x1ebf;} %
  Th\`anh%
}
%    \end{macrocode}
%    \end{macro}
%
% \subsection{Driver detection}
%
%    \begin{macrocode}
\HOLOGO@IfExists\InputIfFileExists{%
  \InputIfFileExists{hologo.cfg}{}{}%
}{%
  \ltx@IfUndefined{pdf@filesize}{%
    \def\HOLOGO@InputIfExists{%
      \openin\HOLOGO@temp=hologo.cfg\relax
      \ifeof\HOLOGO@temp
        \closein\HOLOGO@temp
      \else
        \closein\HOLOGO@temp
        \begingroup
          \def\x{LaTeX2e}%
        \expandafter\endgroup
        \ifx\fmtname\x
          \input{hologo.cfg}%
        \else
          \input hologo.cfg\relax
        \fi
      \fi
    }%
    \ltx@IfUndefined{newread}{%
      \chardef\HOLOGO@temp=15 %
      \def\HOLOGO@CheckRead{%
        \ifeof\HOLOGO@temp
          \HOLOGO@InputIfExists
        \else
          \ifcase\HOLOGO@temp
            \@PackageWarningNoLine{hologo}{%
              Configuration file ignored, because\MessageBreak
              a free read register could not be found%
            }%
          \else
            \begingroup
              \count\ltx@cclv=\HOLOGO@temp
              \advance\ltx@cclv by \ltx@minusone
              \edef\x{\endgroup
                \chardef\noexpand\HOLOGO@temp=\the\count\ltx@cclv
                \relax
              }%
            \x
          \fi
        \fi
      }%
    }{%
      \csname newread\endcsname\HOLOGO@temp
      \HOLOGO@InputIfExists
    }%
  }{%
    \edef\HOLOGO@temp{\pdf@filesize{hologo.cfg}}%
    \ifx\HOLOGO@temp\ltx@empty
    \else
      \ifnum\HOLOGO@temp>0 %
        \begingroup
          \def\x{LaTeX2e}%
        \expandafter\endgroup
        \ifx\fmtname\x
          \input{hologo.cfg}%
        \else
          \input hologo.cfg\relax
        \fi
      \else
        \@PackageInfoNoLine{hologo}{%
          Empty configuration file `hologo.cfg' ignored%
        }%
      \fi
    \fi
  }%
}
%    \end{macrocode}
%
%    \begin{macrocode}
\def\HOLOGO@temp#1#2{%
  \kv@define@key{HoLogoDriver}{#1}[]{%
    \begingroup
      \def\HOLOGO@temp{##1}%
      \ltx@onelevel@sanitize\HOLOGO@temp
      \ifx\HOLOGO@temp\ltx@empty
      \else
        \@PackageError{hologo}{%
          Value (\HOLOGO@temp) not permitted for option `#1'%
        }%
        \@ehc
      \fi
    \endgroup
    \def\hologoDriver{#2}%
  }%
}%
\def\HOLOGO@@temp#1#2{%
  \ifx\kv@value\relax
    \HOLOGO@temp{#1}{#1}%
  \else
    \HOLOGO@temp{#1}{#2}%
  \fi
}%
\kv@parse@normalized{%
  pdftex,%
  luatex=pdftex,%
  dvipdfm,%
  dvipdfmx=dvipdfm,%
  dvips,%
  dvipsone=dvips,%
  xdvi=dvips,%
  xetex,%
  vtex,%
}\HOLOGO@@temp
%    \end{macrocode}
%
%    \begin{macrocode}
\kv@define@key{HoLogoDriver}{driverfallback}{%
  \def\HOLOGO@DriverFallback{#1}%
}
%    \end{macrocode}
%
%    \begin{macro}{\HOLOGO@DriverFallback}
%    \begin{macrocode}
\def\HOLOGO@DriverFallback{dvips}
%    \end{macrocode}
%    \end{macro}
%
%    \begin{macro}{\hologoDriverSetup}
%    \begin{macrocode}
\def\hologoDriverSetup{%
  \let\hologoDriver\ltx@undefined
  \HOLOGO@DriverSetup
}
%    \end{macrocode}
%    \end{macro}
%
%    \begin{macro}{\HOLOGO@DriverSetup}
%    \begin{macrocode}
\def\HOLOGO@DriverSetup#1{%
  \kvsetkeys{HoLogoDriver}{#1}%
  \HOLOGO@CheckDriver
  \ltx@ifundefined{hologoDriver}{%
    \begingroup
    \edef\x{\endgroup
      \noexpand\kvsetkeys{HoLogoDriver}{\HOLOGO@DriverFallback}%
    }\x
  }{}%
  \@PackageInfoNoLine{hologo}{Using driver `\hologoDriver'}%
}
%    \end{macrocode}
%    \end{macro}
%
%    \begin{macro}{\HOLOGO@CheckDriver}
%    \begin{macrocode}
\def\HOLOGO@CheckDriver{%
  \ifpdf
    \def\hologoDriver{pdftex}%
    \let\HOLOGO@pdfliteral\pdfliteral
    \ifluatex
      \ifx\pdfextension\@undefined\else
        \protected\def\pdfliteral{\pdfextension literal}%
        \let\HOLOGO@pdfliteral\pdfliteral
      \fi
      \ltx@IfUndefined{HOLOGO@pdfliteral}{%
        \ifnum\luatexversion<36 %
        \else
          \begingroup
            \let\HOLOGO@temp\endgroup
            \ifcase0%
                \directlua{%
                  if tex.enableprimitives then %
                    tex.enableprimitives('HOLOGO@', {'pdfliteral'})%
                  else %
                    tex.print('1')%
                  end%
                }%
                \ifx\HOLOGO@pdfliteral\@undefined 1\fi%
                \relax%
              \endgroup
              \let\HOLOGO@temp\relax
              \global\let\HOLOGO@pdfliteral\HOLOGO@pdfliteral
            \fi%
          \HOLOGO@temp
        \fi
      }{}%
    \fi
    \ltx@IfUndefined{HOLOGO@pdfliteral}{%
      \@PackageWarningNoLine{hologo}{%
        Cannot find \string\pdfliteral
      }%
    }{}%
  \else
    \ifxetex
      \def\hologoDriver{xetex}%
    \else
      \ifvtex
        \def\hologoDriver{vtex}%
      \fi
    \fi
  \fi
}
%    \end{macrocode}
%    \end{macro}
%
%    \begin{macro}{\HOLOGO@WarningUnsupportedDriver}
%    \begin{macrocode}
\def\HOLOGO@WarningUnsupportedDriver#1{%
  \@PackageWarningNoLine{hologo}{%
    Logo `#1' needs driver specific macros,\MessageBreak
    but driver `\hologoDriver' is not supported.\MessageBreak
    Use a different driver or\MessageBreak
    load package `graphics' or `pgf'%
  }%
}
%    \end{macrocode}
%    \end{macro}
%
% \subsubsection{Reflect box macros}
%
%    Skip driver part if not needed.
%    \begin{macrocode}
\ltx@IfUndefined{reflectbox}{}{%
  \ltx@IfUndefined{rotatebox}{}{%
    \HOLOGO@AtEnd
  }%
}
\ltx@IfUndefined{pgftext}{}{%
  \HOLOGO@AtEnd
}
\ltx@IfUndefined{psscalebox}{}{%
  \HOLOGO@AtEnd
}
%    \end{macrocode}
%
%    \begin{macrocode}
\def\HOLOGO@temp{LaTeX2e}
\ifx\fmtname\HOLOGO@temp
  \RequirePackage{kvoptions}[2011/06/30]%
  \ProcessKeyvalOptions{HoLogoDriver}%
\fi
\HOLOGO@DriverSetup{}
%    \end{macrocode}
%
%    \begin{macro}{\HOLOGO@ReflectBox}
%    \begin{macrocode}
\def\HOLOGO@ReflectBox#1{%
  \begingroup
    \setbox\ltx@zero\hbox{\begingroup#1\endgroup}%
    \setbox\ltx@two\hbox{%
      \kern\wd\ltx@zero
      \csname HOLOGO@ScaleBox@\hologoDriver\endcsname{-1}{1}{%
        \hbox to 0pt{\copy\ltx@zero\hss}%
      }%
    }%
    \wd\ltx@two=\wd\ltx@zero
    \box\ltx@two
  \endgroup
}
%    \end{macrocode}
%    \end{macro}
%
%    \begin{macro}{\HOLOGO@PointReflectBox}
%    \begin{macrocode}
\def\HOLOGO@PointReflectBox#1{%
  \begingroup
    \setbox\ltx@zero\hbox{\begingroup#1\endgroup}%
    \setbox\ltx@two\hbox{%
      \kern\wd\ltx@zero
      \raise\ht\ltx@zero\hbox{%
        \csname HOLOGO@ScaleBox@\hologoDriver\endcsname{-1}{-1}{%
          \hbox to 0pt{\copy\ltx@zero\hss}%
        }%
      }%
    }%
    \wd\ltx@two=\wd\ltx@zero
    \box\ltx@two
  \endgroup
}
%    \end{macrocode}
%    \end{macro}
%
%    We must define all variants because of dynamic driver setup.
%    \begin{macrocode}
\def\HOLOGO@temp#1#2{#2}
%    \end{macrocode}
%
%    \begin{macro}{\HOLOGO@ScaleBox@pdftex}
%    \begin{macrocode}
\HOLOGO@temp{pdftex}{%
  \def\HOLOGO@ScaleBox@pdftex#1#2#3{%
    \HOLOGO@pdfliteral{%
      q #1 0 0 #2 0 0 cm%
    }%
    #3%
    \HOLOGO@pdfliteral{%
      Q%
    }%
  }%
}
%    \end{macrocode}
%    \end{macro}
%    \begin{macro}{\HOLOGO@ScaleBox@dvips}
%    \begin{macrocode}
\HOLOGO@temp{dvips}{%
  \def\HOLOGO@ScaleBox@dvips#1#2#3{%
    \special{ps:%
      gsave %
      currentpoint %
      currentpoint translate %
      #1 #2 scale %
      neg exch neg exch translate%
    }%
    #3%
    \special{ps:%
      currentpoint %
      grestore %
      moveto%
    }%
  }%
}
%    \end{macrocode}
%    \end{macro}
%    \begin{macro}{\HOLOGO@ScaleBox@dvipdfm}
%    \begin{macrocode}
\HOLOGO@temp{dvipdfm}{%
  \let\HOLOGO@ScaleBox@dvipdfm\HOLOGO@ScaleBox@dvips
}
%    \end{macrocode}
%    \end{macro}
%    Since \hologo{XeTeX} v0.6.
%    \begin{macro}{\HOLOGO@ScaleBox@xetex}
%    \begin{macrocode}
\HOLOGO@temp{xetex}{%
  \def\HOLOGO@ScaleBox@xetex#1#2#3{%
    \special{x:gsave}%
    \special{x:scale #1 #2}%
    #3%
    \special{x:grestore}%
  }%
}
%    \end{macrocode}
%    \end{macro}
%    \begin{macro}{\HOLOGO@ScaleBox@vtex}
%    \begin{macrocode}
\HOLOGO@temp{vtex}{%
  \def\HOLOGO@ScaleBox@vtex#1#2#3{%
    \special{r(#1,0,0,#2,0,0}%
    #3%
    \special{r)}%
  }%
}
%    \end{macrocode}
%    \end{macro}
%
%    \begin{macrocode}
\HOLOGO@AtEnd%
%</package>
%    \end{macrocode}
%
% \section{Test}
%
% \subsection{Catcode checks for loading}
%
%    \begin{macrocode}
%<*test1>
%    \end{macrocode}
%    \begin{macrocode}
\catcode`\{=1 %
\catcode`\}=2 %
\catcode`\#=6 %
\catcode`\@=11 %
\expandafter\ifx\csname count@\endcsname\relax
  \countdef\count@=255 %
\fi
\expandafter\ifx\csname @gobble\endcsname\relax
  \long\def\@gobble#1{}%
\fi
\expandafter\ifx\csname @firstofone\endcsname\relax
  \long\def\@firstofone#1{#1}%
\fi
\expandafter\ifx\csname loop\endcsname\relax
  \expandafter\@firstofone
\else
  \expandafter\@gobble
\fi
{%
  \def\loop#1\repeat{%
    \def\body{#1}%
    \iterate
  }%
  \def\iterate{%
    \body
      \let\next\iterate
    \else
      \let\next\relax
    \fi
    \next
  }%
  \let\repeat=\fi
}%
\def\RestoreCatcodes{}
\count@=0 %
\loop
  \edef\RestoreCatcodes{%
    \RestoreCatcodes
    \catcode\the\count@=\the\catcode\count@\relax
  }%
\ifnum\count@<255 %
  \advance\count@ 1 %
\repeat

\def\RangeCatcodeInvalid#1#2{%
  \count@=#1\relax
  \loop
    \catcode\count@=15 %
  \ifnum\count@<#2\relax
    \advance\count@ 1 %
  \repeat
}
\def\RangeCatcodeCheck#1#2#3{%
  \count@=#1\relax
  \loop
    \ifnum#3=\catcode\count@
    \else
      \errmessage{%
        Character \the\count@\space
        with wrong catcode \the\catcode\count@\space
        instead of \number#3%
      }%
    \fi
  \ifnum\count@<#2\relax
    \advance\count@ 1 %
  \repeat
}
\def\space{ }
\expandafter\ifx\csname LoadCommand\endcsname\relax
  \def\LoadCommand{\input hologo.sty\relax}%
\fi
\def\Test{%
  \RangeCatcodeInvalid{0}{47}%
  \RangeCatcodeInvalid{58}{64}%
  \RangeCatcodeInvalid{91}{96}%
  \RangeCatcodeInvalid{123}{255}%
  \catcode`\@=12 %
  \catcode`\\=0 %
  \catcode`\%=14 %
  \LoadCommand
  \RangeCatcodeCheck{0}{36}{15}%
  \RangeCatcodeCheck{37}{37}{14}%
  \RangeCatcodeCheck{38}{47}{15}%
  \RangeCatcodeCheck{48}{57}{12}%
  \RangeCatcodeCheck{58}{63}{15}%
  \RangeCatcodeCheck{64}{64}{12}%
  \RangeCatcodeCheck{65}{90}{11}%
  \RangeCatcodeCheck{91}{91}{15}%
  \RangeCatcodeCheck{92}{92}{0}%
  \RangeCatcodeCheck{93}{96}{15}%
  \RangeCatcodeCheck{97}{122}{11}%
  \RangeCatcodeCheck{123}{255}{15}%
  \RestoreCatcodes
}
\Test
\csname @@end\endcsname
\end
%    \end{macrocode}
%    \begin{macrocode}
%</test1>
%    \end{macrocode}
%
% \subsection{Spacefactor}
%
%    The space factor must be 1000 after a logo. If it is greater 1000
%    then the following space is a space after a sentence closing point.
%    If the space factor is smaller 1000 then an immediate following
%    dot is interpreted as abbreviation, not sentence closing point.
%
%    \begin{macrocode}
%<*test-spacefactor>
\NeedsTeXFormat{LaTeX2e}
\documentclass{article}
\usepackage{hologo}[2016/05/12]
\usepackage{kvsetkeys}
\usepackage{qstest}
\IncludeTests{*}
\LogTests{log}{*}{*}
\begin{document}
\begin{qstest}{spacefactor}{spacefactor}
\newcommand*{\Test}[1]{%
  \sbox0{%
    \hologo{#1}%
    \Expect*{1000 (#1)}*{\the\spacefactor\space(#1)}%
  }%
}%
\makeatletter
\def\TestList{}
\def\hologoEntry#1#2#3{%
  \edef\TestList{%
    \ifx\TestList\@empty
    \else
      \TestList,%
    \fi
    #1%
    \ifx\\#2\\%
    \else
      ={variant=#2}%
    \fi
  }%
}
\hologoList
\expandafter\kv@parse@normalized\expandafter{%
  \TestList
}{%
  \begingroup
    \let\@logo=\kv@key
    \ifx\kv@value\relax
    \else
      \expandafter\hologoLogoSetup\expandafter\@logo\expandafter{%
        \kv@value
      }%
    \fi
    \Test\@logo
  \endgroup
  \@gobbletwo
}
\end{qstest}
\end{document}
%</test-spacefactor>
%    \end{macrocode}
%
% \subsection{Complete list}
%
%    \begin{macrocode}
%<*test-list>
\NeedsTeXFormat{LaTeX2e}
\documentclass[12pt,a4paper]{article}
\usepackage{hologo}[2016/05/12]
\usepackage[T1]{fontenc}
\usepackage{lmodern}
\usepackage{parskip}
\usepackage[unicode]{hyperref}[2011/09/28]
\usepackage{bookmark}[2011/09/19]
\bookmarksetup{%
  numbered,%
  open,%
  openlevel=2,%
}
\renewcommand*{\contentsname}{List of logos}
\begin{document}
\tableofcontents
\def\TestFont#1#2#3#4#5#6{%
  \begingroup
    \usefont{#3}{#4}{#5}{#6}%
    \HologoVariant{#1}{#2}/\hologoVariant{#1}{#2}%
    \quad
    \begingroup\scriptsize\hologoVariant{#1}{#2}\endgroup
    \quad
  \endgroup
  (#3/#4/#5/#6)%
  \par
}
\makeatletter
\def\hologoEntry#1#2#3{%
  \section{%
    \HologoVariant{#1}{#2}/\hologoVariant{#1}{#2} %
    {[#1\ifx\\#2\\\else\space(#2)\fi]}% hash-ok
  }% braces around [] because of bug in tex4ht
  \begingroup
    \hypersetup{unicode=false}%
    \bookmark[%
      dest=\@currentHref,%
      rellevel=1,%
      keeplevel,%
    ]{%
      \HologoVariant{#1}{#2}/\hologoVariant{#1}{#2} %
      (PDFDocEncoding)%
    }%
  \endgroup
  \TestFont{#1}{#2}{OT1}{cmr}{m}{n}%
  \TestFont{#1}{#2}{OT1}{cmss}{m}{n}%
  \TestFont{#1}{#2}{OT1}{cmr}{b}{n}%
  \TestFont{#1}{#2}{OT1}{cmr}{m}{it}%
  \TestFont{#1}{#2}{OT1}{cmtt}{m}{n}%
  \TestFont{#1}{#2}{T1}{lmr}{m}{n}%
  \TestFont{#1}{#2}{T1}{lmss}{m}{n}%
  \TestFont{#1}{#2}{T1}{lmr}{b}{n}%
  \TestFont{#1}{#2}{T1}{lmr}{m}{it}%
  \TestFont{#1}{#2}{T1}{lmtt}{m}{n}%
  \TestFont{#1}{#2}{T1}{lmvtt}{m}{n}%
  \TestFont{#1}{#2}{T1}{qtm}{m}{n}%
  \TestFont{#1}{#2}{T1}{qhv}{m}{n}%
  \TestFont{#1}{#2}{T1}{qtm}{b}{n}%
  \TestFont{#1}{#2}{T1}{qtm}{m}{it}%
  \TestFont{#1}{#2}{T1}{qcr}{m}{n}%
  \newpage
}
\makeatother
\hologoList
\end{document}
%</test-list>
%    \end{macrocode}
%
% \section{Installation}
%
% \subsection{Download}
%
% \paragraph{Package.} This package is available on
% CTAN\footnote{\url{ftp://ftp.ctan.org/tex-archive/}}:
% \begin{description}
% \item[\CTAN{macros/latex/contrib/oberdiek/hologo.dtx}] The source file.
% \item[\CTAN{macros/latex/contrib/oberdiek/hologo.pdf}] Documentation.
% \end{description}
%
%
% \paragraph{Bundle.} All the packages of the bundle `oberdiek'
% are also available in a TDS compliant ZIP archive. There
% the packages are already unpacked and the documentation files
% are generated. The files and directories obey the TDS standard.
% \begin{description}
% \item[\CTAN{install/macros/latex/contrib/oberdiek.tds.zip}]
% \end{description}
% \emph{TDS} refers to the standard ``A Directory Structure
% for \TeX\ Files'' (\CTAN{tds/tds.pdf}). Directories
% with \xfile{texmf} in their name are usually organized this way.
%
% \subsection{Bundle installation}
%
% \paragraph{Unpacking.} Unpack the \xfile{oberdiek.tds.zip} in the
% TDS tree (also known as \xfile{texmf} tree) of your choice.
% Example (linux):
% \begin{quote}
%   |unzip oberdiek.tds.zip -d ~/texmf|
% \end{quote}
%
% \paragraph{Script installation.}
% Check the directory \xfile{TDS:scripts/oberdiek/} for
% scripts that need further installation steps.
% Package \xpackage{attachfile2} comes with the Perl script
% \xfile{pdfatfi.pl} that should be installed in such a way
% that it can be called as \texttt{pdfatfi}.
% Example (linux):
% \begin{quote}
%   |chmod +x scripts/oberdiek/pdfatfi.pl|\\
%   |cp scripts/oberdiek/pdfatfi.pl /usr/local/bin/|
% \end{quote}
%
% \subsection{Package installation}
%
% \paragraph{Unpacking.} The \xfile{.dtx} file is a self-extracting
% \docstrip\ archive. The files are extracted by running the
% \xfile{.dtx} through \plainTeX:
% \begin{quote}
%   \verb|tex hologo.dtx|
% \end{quote}
%
% \paragraph{TDS.} Now the different files must be moved into
% the different directories in your installation TDS tree
% (also known as \xfile{texmf} tree):
% \begin{quote}
% \def\t{^^A
% \begin{tabular}{@{}>{\ttfamily}l@{ $\rightarrow$ }>{\ttfamily}l@{}}
%   hologo.sty & tex/generic/oberdiek/hologo.sty\\
%   hologo.pdf & doc/latex/oberdiek/hologo.pdf\\
%   example/hologo-example.tex & doc/latex/oberdiek/example/hologo-example.tex\\
%   test/hologo-test1.tex & doc/latex/oberdiek/test/hologo-test1.tex\\
%   test/hologo-test-spacefactor.tex & doc/latex/oberdiek/test/hologo-test-spacefactor.tex\\
%   test/hologo-test-list.tex & doc/latex/oberdiek/test/hologo-test-list.tex\\
%   hologo.dtx & source/latex/oberdiek/hologo.dtx\\
% \end{tabular}^^A
% }^^A
% \sbox0{\t}^^A
% \ifdim\wd0>\linewidth
%   \begingroup
%     \advance\linewidth by\leftmargin
%     \advance\linewidth by\rightmargin
%   \edef\x{\endgroup
%     \def\noexpand\lw{\the\linewidth}^^A
%   }\x
%   \def\lwbox{^^A
%     \leavevmode
%     \hbox to \linewidth{^^A
%       \kern-\leftmargin\relax
%       \hss
%       \usebox0
%       \hss
%       \kern-\rightmargin\relax
%     }^^A
%   }^^A
%   \ifdim\wd0>\lw
%     \sbox0{\small\t}^^A
%     \ifdim\wd0>\linewidth
%       \ifdim\wd0>\lw
%         \sbox0{\footnotesize\t}^^A
%         \ifdim\wd0>\linewidth
%           \ifdim\wd0>\lw
%             \sbox0{\scriptsize\t}^^A
%             \ifdim\wd0>\linewidth
%               \ifdim\wd0>\lw
%                 \sbox0{\tiny\t}^^A
%                 \ifdim\wd0>\linewidth
%                   \lwbox
%                 \else
%                   \usebox0
%                 \fi
%               \else
%                 \lwbox
%               \fi
%             \else
%               \usebox0
%             \fi
%           \else
%             \lwbox
%           \fi
%         \else
%           \usebox0
%         \fi
%       \else
%         \lwbox
%       \fi
%     \else
%       \usebox0
%     \fi
%   \else
%     \lwbox
%   \fi
% \else
%   \usebox0
% \fi
% \end{quote}
% If you have a \xfile{docstrip.cfg} that configures and enables \docstrip's
% TDS installing feature, then some files can already be in the right
% place, see the documentation of \docstrip.
%
% \subsection{Refresh file name databases}
%
% If your \TeX~distribution
% (\teTeX, \mikTeX, \dots) relies on file name databases, you must refresh
% these. For example, \teTeX\ users run \verb|texhash| or
% \verb|mktexlsr|.
%
% \subsection{Some details for the interested}
%
% \paragraph{Attached source.}
%
% The PDF documentation on CTAN also includes the
% \xfile{.dtx} source file. It can be extracted by
% AcrobatReader 6 or higher. Another option is \textsf{pdftk},
% e.g. unpack the file into the current directory:
% \begin{quote}
%   \verb|pdftk hologo.pdf unpack_files output .|
% \end{quote}
%
% \paragraph{Unpacking with \LaTeX.}
% The \xfile{.dtx} chooses its action depending on the format:
% \begin{description}
% \item[\plainTeX:] Run \docstrip\ and extract the files.
% \item[\LaTeX:] Generate the documentation.
% \end{description}
% If you insist on using \LaTeX\ for \docstrip\ (really,
% \docstrip\ does not need \LaTeX), then inform the autodetect routine
% about your intention:
% \begin{quote}
%   \verb|latex \let\install=y\input{hologo.dtx}|
% \end{quote}
% Do not forget to quote the argument according to the demands
% of your shell.
%
% \paragraph{Generating the documentation.}
% You can use both the \xfile{.dtx} or the \xfile{.drv} to generate
% the documentation. The process can be configured by the
% configuration file \xfile{ltxdoc.cfg}. For instance, put this
% line into this file, if you want to have A4 as paper format:
% \begin{quote}
%   \verb|\PassOptionsToClass{a4paper}{article}|
% \end{quote}
% An example follows how to generate the
% documentation with pdf\LaTeX:
% \begin{quote}
%\begin{verbatim}
%pdflatex hologo.dtx
%makeindex -s gind.ist hologo.idx
%pdflatex hologo.dtx
%makeindex -s gind.ist hologo.idx
%pdflatex hologo.dtx
%\end{verbatim}
% \end{quote}
%
% \section{Catalogue}
%
% The following XML file can be used as source for the
% \href{http://mirror.ctan.org/help/Catalogue/catalogue.html}{\TeX\ Catalogue}.
% The elements \texttt{caption} and \texttt{description} are imported
% from the original XML file from the Catalogue.
% The name of the XML file in the Catalogue is \xfile{hologo.xml}.
%    \begin{macrocode}
%<*catalogue>
<?xml version='1.0' encoding='us-ascii'?>
<!DOCTYPE entry SYSTEM 'catalogue.dtd'>
<entry datestamp='$Date$' modifier='$Author$' id='hologo'>
  <name>hologo</name>
  <caption>A collection of logos with bookmark support.</caption>
  <authorref id='auth:oberdiek'/>
  <copyright owner='Heiko Oberdiek' year='2010-2012'/>
  <license type='lppl1.3'/>
  <version number='1.10'/>
  <description>
    The package defines a single command <tt>\hologo</tt>, whose
    argument is the usual case-confused ASCII version of the logo.
    The command is bookmark-enabled, so that every logo becomes
    available in bookmarks without further work.
    <p/>
    The package is part of the <xref refid='oberdiek'>oberdiek</xref>
    bundle.
  </description>
  <documentation details='Package documentation'
      href='ctan:/macros/latex/contrib/oberdiek/hologo.pdf'/>
  <ctan file='true' path='/macros/latex/contrib/oberdiek/hologo.dtx'/>
  <miktex location='oberdiek'/>
  <texlive location='oberdiek'/>
  <install path='/macros/latex/contrib/oberdiek/oberdiek.tds.zip'/>
</entry>
%</catalogue>
%    \end{macrocode}
%
% \begin{thebibliography}{9}
% \raggedright
%
% \bibitem{btxdoc}
% Oren Patashnik,
% \textit{\hologo{BibTeX}ing},
% 1988-02-08.\\
% \CTAN{biblio/bibtex/base/}
%
% \bibitem{dtklogos}
% Gerd Neugebauer, DANTE,
% \textit{Package \xpackage{dtklogos}},
% 2011-04-25.\\
% \CTAN{usergrps/dante/dtk/dtklogos.sty}
%
% \bibitem{etexman}
% The \hologo{NTS} Team,
% \textit{The \hologo{eTeX} manual},
% 1998-02.\\
% \CTAN{systems/e-tex/v2/doc/}
%
% \bibitem{ExTeX-FAQ}
% The \hologo{ExTeX} group,
% \textit{\hologo{ExTeX}: FAQ -- How is \hologo{ExTeX} typeset?},
% 2007-04-14.\\
% \url{http://www.extex.org/documentation/faq.html}
%
% \bibitem{LyX}
% %@MISC{ LyX,
% %  title = {{LyX 2.0.0 -- The Document Processor [Computer software and manual]}},
% %  author = {{The LyX Team}},
% %  howpublished = {Internet: http://www.lyx.org},
% %  year = {2011-05-08},
% %  note = {Retrieved May 10, 2011, from http://www.lyx.org},
% %  url = {http://www.lyx.org/}
% %}
% The \hologo{LyX} Team,
% \textit{\hologo{LyX} -- The Document Processor},
% 2011-05-08.\\
% \url{http://www.lyx.org/}
%
% \bibitem{OzTeX}
% Andrew Trevorrow,
% \hologo{OzTeX} FAQ: What is the correct way to typeset ``\hologo{OzTeX}''?,
% 2011-09-15 (visited).
% \url{http://www.trevorrow.com/oztex/ozfaq.html#oztex-logo}
%
% \bibitem{PiCTeX}
% Michael Wichura,
% \textit{The \hologo{PiCTeX} macro package},
% 1987-09-21.
% \CTAN{graphics/pictex/}
%
% \bibitem{scrlogo}
% Markus Kohm,
% \textit{\hologo{KOMAScript} Datei \xfile{scrlogo.dtx}},
% 2009-01-30.\\
% \CTAN{install/macros/latex/contrib/komascript.tds.zip}
%
% \end{thebibliography}
%
% \begin{History}
%   \begin{Version}{2010/04/08 v1.0}
%   \item
%     The first version.
%   \end{Version}
%   \begin{Version}{2010/04/16 v1.1}
%   \item
%     \cs{Hologo} added for support of logos at start of a sentence.
%   \item
%     \cs{hologoSetup} and \cs{hologoLogoSetup} added.
%   \item
%     Options \xoption{break}, \xoption{hyphenbreak}, \xoption{spacebreak}
%     added.
%   \item
%     Variant support added by option \xoption{variant}.
%   \end{Version}
%   \begin{Version}{2010/04/24 v1.2}
%   \item
%     \hologo{LaTeX3} added.
%   \item
%     \hologo{VTeX} added.
%   \end{Version}
%   \begin{Version}{2010/11/21 v1.3}
%   \item
%     \hologo{iniTeX}, \hologo{virTeX} added.
%   \end{Version}
%   \begin{Version}{2011/03/25 v1.4}
%   \item
%     \hologo{ConTeXt} with variants added.
%   \item
%     Option \xoption{discretionarybreak} added as refinement for
%     option \xoption{break}.
%   \end{Version}
%   \begin{Version}{2011/04/21 v1.5}
%   \item
%     Wrong TDS directory for test files fixed.
%   \end{Version}
%   \begin{Version}{2011/10/01 v1.6}
%   \item
%     Support for package \xpackage{tex4ht} added.
%   \item
%     Support for \cs{csname} added if \cs{ifincsname} is available.
%   \item
%     New logos:
%     \hologo{(La)TeX},
%     \hologo{biber},
%     \hologo{BibTeX} (\xoption{sc}, \xoption{sf}),
%     \hologo{emTeX},
%     \hologo{ExTeX},
%     \hologo{KOMAScript},
%     \hologo{La},
%     \hologo{LyX},
%     \hologo{MiKTeX},
%     \hologo{NTS},
%     \hologo{OzMF},
%     \hologo{OzMP},
%     \hologo{OzTeX},
%     \hologo{OzTtH},
%     \hologo{PCTeX},
%     \hologo{PiC},
%     \hologo{PiCTeX},
%     \hologo{METAFONT},
%     \hologo{MetaFun},
%     \hologo{METAPOST},
%     \hologo{MetaPost},
%     \hologo{SLiTeX} (\xoption{lift}, \xoption{narrow}, \xoption{simple}),
%     \hologo{SliTeX} (\xoption{narrow}, \xoption{simple}, \xoption{lift}),
%     \hologo{teTeX}.
%   \item
%     Fixes:
%     \hologo{iniTeX},
%     \hologo{pdfLaTeX},
%     \hologo{pdfTeX},
%     \hologo{virTeX}.
%   \item
%     \cs{hologoFontSetup} and \cs{hologoLogoFontSetup} added.
%   \item
%     \cs{hologoVariant} and \cs{HologoVariant} added.
%   \end{Version}
%   \begin{Version}{2011/11/22 v1.7}
%   \item
%     New logos:
%     \hologo{BibTeX8},
%     \hologo{LaTeXML},
%     \hologo{SageTeX},
%     \hologo{TeX4ht},
%     \hologo{TTH}.
%   \item
%     \hologo{Xe} and friends: Driver stuff fixed.
%   \item
%     \hologo{Xe} and friends: Support for italic added.
%   \item
%     \hologo{Xe} and friends: Package support for \xpackage{pgf}
%     and \xpackage{pstricks} added.
%   \end{Version}
%   \begin{Version}{2011/11/29 v1.8}
%   \item
%     New logos:
%     \hologo{HanTheThanh}.
%   \end{Version}
%   \begin{Version}{2011/12/21 v1.9}
%   \item
%     Patch for package \xpackage{ifxetex} added for the case that
%     \cs{newif} is undefined in \hologo{iniTeX}.
%   \item
%     Some fixes for \hologo{iniTeX}.
%   \end{Version}
%   \begin{Version}{2012/04/26 v1.10}
%   \item
%     Fix in bookmark version of logo ``\hologo{HanTheThanh}''.
%   \end{Version}
%   \begin{Version}{2016/05/12 v1.11}
%   \item
%     Update HOLOGO@IfCharExists (previously in texlive)
%   \item define pdfliteral in current luatex.
%   \end{Version}
% \end{History}
%
% \PrintIndex
%
% \Finale
\endinput

%        (quote the arguments according to the demands of your shell)
%
% Documentation:
%    (a) If hologo.drv is present:
%           latex hologo.drv
%    (b) Without hologo.drv:
%           latex hologo.dtx; ...
%    The class ltxdoc loads the configuration file ltxdoc.cfg
%    if available. Here you can specify further options, e.g.
%    use A4 as paper format:
%       \PassOptionsToClass{a4paper}{article}
%
%    Programm calls to get the documentation (example):
%       pdflatex hologo.dtx
%       makeindex -s gind.ist hologo.idx
%       pdflatex hologo.dtx
%       makeindex -s gind.ist hologo.idx
%       pdflatex hologo.dtx
%
% Installation:
%    TDS:tex/generic/oberdiek/hologo.sty
%    TDS:doc/latex/oberdiek/hologo.pdf
%    TDS:doc/latex/oberdiek/example/hologo-example.tex
%    TDS:doc/latex/oberdiek/test/hologo-test1.tex
%    TDS:doc/latex/oberdiek/test/hologo-test-spacefactor.tex
%    TDS:doc/latex/oberdiek/test/hologo-test-list.tex
%    TDS:source/latex/oberdiek/hologo.dtx
%
%<*ignore>
\begingroup
  \catcode123=1 %
  \catcode125=2 %
  \def\x{LaTeX2e}%
\expandafter\endgroup
\ifcase 0\ifx\install y1\fi\expandafter
         \ifx\csname processbatchFile\endcsname\relax\else1\fi
         \ifx\fmtname\x\else 1\fi\relax
\else\csname fi\endcsname
%</ignore>
%<*install>
\input docstrip.tex
\Msg{************************************************************************}
\Msg{* Installation}
\Msg{* Package: hologo 2016/05/12 v1.11 A logo collection with bookmark support (HO)}
\Msg{************************************************************************}

\keepsilent
\askforoverwritefalse

\let\MetaPrefix\relax
\preamble

This is a generated file.

Project: hologo
Version: 2016/05/12 v1.11

Copyright (C) 2010-2012 by
   Heiko Oberdiek <heiko.oberdiek at googlemail.com>

This work may be distributed and/or modified under the
conditions of the LaTeX Project Public License, either
version 1.3c of this license or (at your option) any later
version. This version of this license is in
   http://www.latex-project.org/lppl/lppl-1-3c.txt
and the latest version of this license is in
   http://www.latex-project.org/lppl.txt
and version 1.3 or later is part of all distributions of
LaTeX version 2005/12/01 or later.

This work has the LPPL maintenance status "maintained".

This Current Maintainer of this work is Heiko Oberdiek.

The Base Interpreter refers to any `TeX-Format',
because some files are installed in TDS:tex/generic//.

This work consists of the main source file hologo.dtx
and the derived files
   hologo.sty, hologo.pdf, hologo.ins, hologo.drv, hologo-example.tex,
   hologo-test1.tex, hologo-test-spacefactor.tex,
   hologo-test-list.tex.

\endpreamble
\let\MetaPrefix\DoubleperCent

\generate{%
  \file{hologo.ins}{\from{hologo.dtx}{install}}%
  \file{hologo.drv}{\from{hologo.dtx}{driver}}%
  \usedir{tex/generic/oberdiek}%
  \file{hologo.sty}{\from{hologo.dtx}{package}}%
  \usedir{doc/latex/oberdiek/example}%
  \file{hologo-example.tex}{\from{hologo.dtx}{example}}%
  \usedir{doc/latex/oberdiek/test}%
  \file{hologo-test1.tex}{\from{hologo.dtx}{test1}}%
  \file{hologo-test-spacefactor.tex}{\from{hologo.dtx}{test-spacefactor}}%
  \file{hologo-test-list.tex}{\from{hologo.dtx}{test-list}}%
  \nopreamble
  \nopostamble
  \usedir{source/latex/oberdiek/catalogue}%
  \file{hologo.xml}{\from{hologo.dtx}{catalogue}}%
}

\catcode32=13\relax% active space
\let =\space%
\Msg{************************************************************************}
\Msg{*}
\Msg{* To finish the installation you have to move the following}
\Msg{* file into a directory searched by TeX:}
\Msg{*}
\Msg{*     hologo.sty}
\Msg{*}
\Msg{* To produce the documentation run the file `hologo.drv'}
\Msg{* through LaTeX.}
\Msg{*}
\Msg{* Happy TeXing!}
\Msg{*}
\Msg{************************************************************************}

\endbatchfile
%</install>
%<*ignore>
\fi
%</ignore>
%<*driver>
\NeedsTeXFormat{LaTeX2e}
\ProvidesFile{hologo.drv}%
  [2016/05/12 v1.11 A logo collection with bookmark support (HO)]%
\documentclass{ltxdoc}
\usepackage{holtxdoc}[2011/11/22]
\usepackage{hologo}[2016/05/12]
\usepackage{longtable}
\usepackage{array}
\usepackage{paralist}
%\usepackage[T1]{fontenc}
%\usepackage{lmodern}
\begin{document}
  \DocInput{hologo.dtx}%
\end{document}
%</driver>
% \fi
%
%
% \CharacterTable
%  {Upper-case    \A\B\C\D\E\F\G\H\I\J\K\L\M\N\O\P\Q\R\S\T\U\V\W\X\Y\Z
%   Lower-case    \a\b\c\d\e\f\g\h\i\j\k\l\m\n\o\p\q\r\s\t\u\v\w\x\y\z
%   Digits        \0\1\2\3\4\5\6\7\8\9
%   Exclamation   \!     Double quote  \"     Hash (number) \#
%   Dollar        \$     Percent       \%     Ampersand     \&
%   Acute accent  \'     Left paren    \(     Right paren   \)
%   Asterisk      \*     Plus          \+     Comma         \,
%   Minus         \-     Point         \.     Solidus       \/
%   Colon         \:     Semicolon     \;     Less than     \<
%   Equals        \=     Greater than  \>     Question mark \?
%   Commercial at \@     Left bracket  \[     Backslash     \\
%   Right bracket \]     Circumflex    \^     Underscore    \_
%   Grave accent  \`     Left brace    \{     Vertical bar  \|
%   Right brace   \}     Tilde         \~}
%
% \GetFileInfo{hologo.drv}
%
% \title{The \xpackage{hologo} package}
% \date{2016/05/12 v1.11}
% \author{Heiko Oberdiek\\\xemail{heiko.oberdiek at googlemail.com}}
%
% \maketitle
%
% \begin{abstract}
% This package starts a collection of logos with support for bookmarks
% strings.
% \end{abstract}
%
% \tableofcontents
%
% \section{Documentation}
%
% \subsection{Logo macros}
%
% \begin{declcs}{hologo} \M{name}
% \end{declcs}
% Macro \cs{hologo} sets the logo with name \meta{name}.
% The following table shows the supported names.
%
% \begingroup
%   \def\hologoEntry#1#2#3{^^A
%     #1&#2&\hologoLogoSetup{#1}{variant=#2}\hologo{#1}&#3\tabularnewline
%   }
%   \begin{longtable}{>{\ttfamily}l>{\ttfamily}lll}
%     \rmfamily\bfseries{name} & \rmfamily\bfseries variant
%     & \bfseries logo & \bfseries since\\
%     \hline
%     \endhead
%     \hologoList
%   \end{longtable}
% \endgroup
%
% \begin{declcs}{Hologo} \M{name}
% \end{declcs}
% Macro \cs{Hologo} starts the logo \meta{name} with an uppercase
% letter. As an exception small greek letters are not converted
% to uppercase. Examples, see \hologo{eTeX} and \hologo{ExTeX}.
%
% \subsection{Setup macros}
%
% The package does not support package options, but the following
% setup macros can be used to set options.
%
% \begin{declcs}{hologoSetup} \M{key value list}
% \end{declcs}
% Macro \cs{hologoSetup} sets global options.
%
% \begin{declcs}{hologoLogoSetup} \M{logo} \M{key value list}
% \end{declcs}
% Some options can also be used to configure a logo.
% These settings take precedence over global option settings.
%
% \subsection{Options}\label{sec:options}
%
% There are boolean and string options:
% \begin{description}
% \item[Boolean option:]
% It takes |true| or |false|
% as value. If the value is omitted, then |true| is used.
% \item[String option:]
% A value must be given as string. (But the string might be empty.)
% \end{description}
% The following options can be used both in \cs{hologoSetup}
% and \cs{hologoLogoSetup}:
% \begin{description}
% \def\entry#1{\item[\xoption{#1}:]}
% \entry{break}
%   enables or disables line breaks inside the logo. This setting is
%   refined by options \xoption{hyphenbreak}, \xoption{spacebreak}
%   or \xoption{discretionarybreak}.
%   Default is |false|.
% \entry{hyphenbreak}
%   enables or disables the line break right after the hyphen character.
% \entry{spacebreak}
%   enables or disables line breaks at space characters.
% \entry{discretionarybreak}
%   enables or disables line breaks at hyphenation points
%   (inserted by \cs{-}).
% \end{description}
% Macro \cs{hologoLogoSetup} also knows:
% \begin{description}
% \item[\xoption{variant}:]
%   This is a string option. It specifies a variant of a logo that
%   must exist. An empty string selects the package default variant.
% \end{description}
% Example:
% \begin{quote}
%   |\hologoSetup{break=false}|\\
%   |\hologoLogoSetup{plainTeX}{variant=hyphen,hyphenbreak}|\\
%   Then ``plain-\TeX'' contains one break point after the hyphen.
% \end{quote}
%
% \subsection{Driver options}
%
% Sometimes graphical operations are needed to construct some
% glyphs (e.g.\ \hologo{XeTeX}). If package \xpackage{graphics}
% or package \xpackage{pgf} are found, then the macros are taken
% from there. Otherwise the packge defines its own operations
% and therefore needs the driver information. Many drivers are
% detected automatically (\hologo{pdfTeX}/\hologo{LuaTeX}
% in PDF mode, \hologo{XeTeX}, \hologo{VTeX}). These have precedence
% over a driver option. The driver can be given as package option
% or using \cs{hologoDriverSetup}.
% The following list contains the recognized driver options:
% \begin{itemize}
% \item \xoption{pdftex}, \xoption{luatex}
% \item \xoption{dvipdfm}, \xoption{dvipdfmx}
% \item \xoption{dvips}, \xoption{dvipsone}, \xoption{xdvi}
% \item \xoption{xetex}
% \item \xoption{vtex}
% \end{itemize}
% The left driver of a line is the driver name that is used internally.
% The following names are aliases for drivers that use the
% same method. Therefore the entry in the \xext{log} file for
% the used driver prints the internally used driver name.
% \begin{description}
% \item[\xoption{driverfallback}:]
%   This option expects a driver that is used,
%   if the driver could not be detected automatically.
% \end{description}
%
% \begin{declcs}{hologoDriverSetup} \M{driver option}
% \end{declcs}
% The driver can also be configured after package loading
% using \cs{hologoDriverSetup}, also the way for \hologo{plainTeX}
% to setup the driver.
%
% \subsection{Font setup}
%
% Some logos require a special font, but should also be usable by
% \hologo{plainTeX}. Therefore the package provides some ways
% to influence the font settings. The options below
% take font settings as values. Both font commands
% such as \cs{sffamily} and macros that take one argument
% like \cs{textsf} can be used.
%
% \begin{declcs}{hologoFontSetup} \M{key value list}
% \end{declcs}
% Macro \cs{hologoFontSetup} sets the fonts for all logos.
% Supported keys:
% \begin{description}
% \def\entry#1{\item[\xoption{#1}:]}
% \entry{general}
%   This font is used for all logos. The default is empty.
%   That means no special font is used.
% \entry{bibsf}
%   This font is used for
%   {\hologoLogoSetup{BibTeX}{variant=sf}\hologo{BibTeX}}
%   with variant \xoption{sf}.
% \entry{rm}
%   This font is a serif font. It is used for \hologo{ExTeX}.
% \entry{sc}
%   This font specifies a small caps font. It is used for
%   {\hologoLogoSetup{BibTeX}{variant=sc}\hologo{BibTeX}}
%   with variant \xoption{sc}.
% \entry{sf}
%   This font specifies a sans serif font. The default
%   is \cs{sffamily}, then \cs{sf} is tried. Otherwise
%   a warning is given. It is used by \hologo{KOMAScript}.
% \entry{sy}
%   This is the font for math symbols (e.g. cmsy).
%   It is used by \hologo{AmS}, \hologo{NTS}, \hologo{ExTeX}.
% \entry{logo}
%   \hologo{METAFONT} and \hologo{METAPOST} are using that font.
%   In \hologo{LaTeX} \cs{logofamily} is used and
%   the definitions of package \xpackage{mflogo} are used
%   if the package is not loaded.
%   Otherwise the \cs{tenlogo} is used and defined
%   if it does not already exists.
% \end{description}
%
% \begin{declcs}{hologoLogoFontSetup} \M{logo} \M{key value list}
% \end{declcs}
% Fonts can also be set for a logo or logo component separately,
% see the following list.
% The keys are the same as for \cs{hologoFontSetup}.
%
% \begin{longtable}{>{\ttfamily}l>{\sffamily}ll}
%   \meta{logo} & keys & result\\
%   \hline
%   \endhead
%   BibTeX & bibsf & {\hologoLogoSetup{BibTeX}{variant=sf}\hologo{BibTeX}}\\[.5ex]
%   BibTeX & sc & {\hologoLogoSetup{BibTeX}{variant=sc}\hologo{BibTeX}}\\[.5ex]
%   ExTeX & rm & \hologo{ExTeX}\\
%   SliTeX & rm & \hologo{SliTeX}\\[.5ex]
%   AmS & sy & \hologo{AmS}\\
%   ExTeX & sy & \hologo{ExTeX}\\
%   NTS & sy & \hologo{NTS}\\[.5ex]
%   KOMAScript & sf & \hologo{KOMAScript}\\[.5ex]
%   METAFONT & logo & \hologo{METAFONT}\\
%   METAPOST & logo & \hologo{METAPOST}\\[.5ex]
%   SliTeX & sc \hologo{SliTeX}
% \end{longtable}
%
% \subsubsection{Font order}
%
% For all logos the font \xoption{general} is applied first.
% Example:
%\begin{quote}
%|\hologoFontSetup{general=\color{red}}|
%\end{quote}
% will print red logos.
% Then if the font uses a special font \xoption{sf}, for example,
% the font is applied that is setup by \cs{hologoLogoFontSetup}.
% If this font is not setup, then the common font setup
% by \cs{hologoFontSetup} is used. Otherwise a warning is given,
% that there is no font configured.
%
% \subsection{Additional user macros}
%
% Usually a variant of a logo is configured by using
% \cs{hologoLogoSetup}, because it is bad style to mix
% different variants of the same logo in the same text.
% There the following macros are a convenience for testing.
%
% \begin{declcs}{hologoVariant} \M{name} \M{variant}\\
%   \cs{HologoVariant} \M{name} \M{variant}
% \end{declcs}
% Logo \meta{name} is set using \meta{variant} that specifies
% explicitely which variant of the macro is used. If the argument
% is empty, then the default form of the logo is used
% (configurable by \cs{hologoLogoSetup}).
%
% \cs{HologoVariant} is used if the logo is set in a context
% that needs an uppercase first letter (beginning of a sentence, \dots).
%
% \begin{declcs}{hologoList}\\
%   \cs{hologoEntry} \M{logo} \M{variant} \M{since}
% \end{declcs}
% Macro \cs{hologoList} contains all logos that are provided
% by the package including variants. The list consists of calls
% of \cs{hologoEntry} with three arguments starting with the
% logo name \meta{logo} and its variant \meta{variant}. An empty
% variant means the current default. Argument \meta{since} specifies
% with version of the package \xpackage{hologo} is needed to get
% the logo. If the logo is fixed, then the date gets updated.
% Therefore the date \meta{since} is not exactly the date of
% the first introduction, but rather the date of the latest fix.
%
% Before \cs{hologoList} can be used, macro \cs{hologoEntry} needs
% a definition. The example file in section \ref{sec:example}
% shows applications of \cs{hologoList}.
%
% \subsection{Supported contexts}
%
% Macros \cs{hologo} and friends support special contexts:
% \begin{itemize}
% \item \hologo{LaTeX}'s protection mechanism.
% \item Bookmarks of package \xpackage{hyperref}.
% \item Package \xpackage{tex4ht}.
% \item The macros can be used inside \cs{csname} constructs,
%   if \cs{ifincsname} is available (\hologo{pdfTeX}, \hologo{XeTeX},
%   \hologo{LuaTeX}).
% \end{itemize}
%
% \subsection{Example}
% \label{sec:example}
%
% The following example prints the logos in different fonts.
%    \begin{macrocode}
%<*example>
%<<verbatim
\NeedsTeXFormat{LaTeX2e}
\documentclass[a4paper]{article}
\usepackage[
  hmargin=20mm,
  vmargin=20mm,
]{geometry}
\pagestyle{empty}
\usepackage{hologo}[2016/05/12]
\usepackage{longtable}
\usepackage{array}
\setlength{\extrarowheight}{2pt}
\usepackage[T1]{fontenc}
\usepackage{lmodern}
\usepackage{pdflscape}
\usepackage[
  pdfencoding=auto,
]{hyperref}
\hypersetup{
  pdfauthor={Heiko Oberdiek},
  pdftitle={Example for package `hologo'},
  pdfsubject={Logos with fonts lmr, lmss, qtm, qpl, qhv},
}
\usepackage{bookmark}

% Print the logo list on the console

\begingroup
  \typeout{}%
  \typeout{*** Begin of logo list ***}%
  \newcommand*{\hologoEntry}[3]{%
    \typeout{#1 \ifx\\#2\\\else(#2) \fi[#3]}%
  }%
  \hologoList
  \typeout{*** End of logo list ***}%
  \typeout{}%
\endgroup

\begin{document}
\begin{landscape}

  \section{Example file for package `hologo'}

  % Table for font names

  \begin{longtable}{>{\bfseries}ll}
    \textbf{font} & \textbf{Font name}\\
    \hline
    lmr & Latin Modern Roman\\
    lmss & Latin Modern Sans\\
    qtm & \TeX\ Gyre Termes\\
    qhv & \TeX\ Gyre Heros\\
    qpl & \TeX\ Gyre Pagella\\
  \end{longtable}

  % Logo list with logos in different fonts

  \begingroup
    \newcommand*{\SetVariant}[2]{%
      \ifx\\#2\\%
      \else
        \hologoLogoSetup{#1}{variant=#2}%
      \fi
    }%
    \newcommand*{\hologoEntry}[3]{%
      \SetVariant{#1}{#2}%
      \raisebox{1em}[0pt][0pt]{\hypertarget{#1@#2}{}}%
      \bookmark[%
        dest={#1@#2},%
      ]{%
        #1\ifx\\#2\\\else\space(#2)\fi: \Hologo{#1}, \hologo{#1} %
        [Unicode]%
      }%
      \hypersetup{unicode=false}%
      \bookmark[%
        dest={#1@#2},%
      ]{%
        #1\ifx\\#2\\\else\space(#2)\fi: \Hologo{#1}, \hologo{#1} %
        [PDFDocEncoding]%
      }%
      \texttt{#1}%
      &%
      \texttt{#2}%
      &%
      \Hologo{#1}%
      &%
      \SetVariant{#1}{#2}%
      \hologo{#1}%
      &%
      \SetVariant{#1}{#2}%
      \fontfamily{qtm}\selectfont
      \hologo{#1}%
      &%
      \SetVariant{#1}{#2}%
      \fontfamily{qpl}\selectfont
      \hologo{#1}%
      &%
      \SetVariant{#1}{#2}%
      \textsf{\hologo{#1}}%
      &%
      \SetVariant{#1}{#2}%
      \fontfamily{qhv}\selectfont
      \hologo{#1}%
      \tabularnewline
    }%
    \begin{longtable}{llllllll}%
      \textbf{\textit{logo}} & \textbf{\textit{variant}} &
      \texttt{\string\Hologo} &
      \textbf{lmr} & \textbf{qtm} & \textbf{qpl} &
      \textbf{lmss} & \textbf{qhv}
      \tabularnewline
      \hline
      \endhead
      \hologoList
    \end{longtable}%
  \endgroup

\end{landscape}
\end{document}
%verbatim
%</example>
%    \end{macrocode}
%
% \StopEventually{
% }
%
% \section{Implementation}
%    \begin{macrocode}
%<*package>
%    \end{macrocode}
%    Reload check, especially if the package is not used with \LaTeX.
%    \begin{macrocode}
\begingroup\catcode61\catcode48\catcode32=10\relax%
  \catcode13=5 % ^^M
  \endlinechar=13 %
  \catcode35=6 % #
  \catcode39=12 % '
  \catcode44=12 % ,
  \catcode45=12 % -
  \catcode46=12 % .
  \catcode58=12 % :
  \catcode64=11 % @
  \catcode123=1 % {
  \catcode125=2 % }
  \expandafter\let\expandafter\x\csname ver@hologo.sty\endcsname
  \ifx\x\relax % plain-TeX, first loading
  \else
    \def\empty{}%
    \ifx\x\empty % LaTeX, first loading,
      % variable is initialized, but \ProvidesPackage not yet seen
    \else
      \expandafter\ifx\csname PackageInfo\endcsname\relax
        \def\x#1#2{%
          \immediate\write-1{Package #1 Info: #2.}%
        }%
      \else
        \def\x#1#2{\PackageInfo{#1}{#2, stopped}}%
      \fi
      \x{hologo}{The package is already loaded}%
      \aftergroup\endinput
    \fi
  \fi
\endgroup%
%    \end{macrocode}
%    Package identification:
%    \begin{macrocode}
\begingroup\catcode61\catcode48\catcode32=10\relax%
  \catcode13=5 % ^^M
  \endlinechar=13 %
  \catcode35=6 % #
  \catcode39=12 % '
  \catcode40=12 % (
  \catcode41=12 % )
  \catcode44=12 % ,
  \catcode45=12 % -
  \catcode46=12 % .
  \catcode47=12 % /
  \catcode58=12 % :
  \catcode64=11 % @
  \catcode91=12 % [
  \catcode93=12 % ]
  \catcode123=1 % {
  \catcode125=2 % }
  \expandafter\ifx\csname ProvidesPackage\endcsname\relax
    \def\x#1#2#3[#4]{\endgroup
      \immediate\write-1{Package: #3 #4}%
      \xdef#1{#4}%
    }%
  \else
    \def\x#1#2[#3]{\endgroup
      #2[{#3}]%
      \ifx#1\@undefined
        \xdef#1{#3}%
      \fi
      \ifx#1\relax
        \xdef#1{#3}%
      \fi
    }%
  \fi
\expandafter\x\csname ver@hologo.sty\endcsname
\ProvidesPackage{hologo}%
  [2016/05/12 v1.11 A logo collection with bookmark support (HO)]%
%    \end{macrocode}
%
%    \begin{macrocode}
\begingroup\catcode61\catcode48\catcode32=10\relax%
  \catcode13=5 % ^^M
  \endlinechar=13 %
  \catcode123=1 % {
  \catcode125=2 % }
  \catcode64=11 % @
  \def\x{\endgroup
    \expandafter\edef\csname HOLOGO@AtEnd\endcsname{%
      \endlinechar=\the\endlinechar\relax
      \catcode13=\the\catcode13\relax
      \catcode32=\the\catcode32\relax
      \catcode35=\the\catcode35\relax
      \catcode61=\the\catcode61\relax
      \catcode64=\the\catcode64\relax
      \catcode123=\the\catcode123\relax
      \catcode125=\the\catcode125\relax
    }%
  }%
\x\catcode61\catcode48\catcode32=10\relax%
\catcode13=5 % ^^M
\endlinechar=13 %
\catcode35=6 % #
\catcode64=11 % @
\catcode123=1 % {
\catcode125=2 % }
\def\TMP@EnsureCode#1#2{%
  \edef\HOLOGO@AtEnd{%
    \HOLOGO@AtEnd
    \catcode#1=\the\catcode#1\relax
  }%
  \catcode#1=#2\relax
}
\TMP@EnsureCode{10}{12}% ^^J
\TMP@EnsureCode{33}{12}% !
\TMP@EnsureCode{34}{12}% "
\TMP@EnsureCode{36}{3}% $
\TMP@EnsureCode{38}{4}% &
\TMP@EnsureCode{39}{12}% '
\TMP@EnsureCode{40}{12}% (
\TMP@EnsureCode{41}{12}% )
\TMP@EnsureCode{42}{12}% *
\TMP@EnsureCode{43}{12}% +
\TMP@EnsureCode{44}{12}% ,
\TMP@EnsureCode{45}{12}% -
\TMP@EnsureCode{46}{12}% .
\TMP@EnsureCode{47}{12}% /
\TMP@EnsureCode{58}{12}% :
\TMP@EnsureCode{59}{12}% ;
\TMP@EnsureCode{60}{12}% <
\TMP@EnsureCode{62}{12}% >
\TMP@EnsureCode{63}{12}% ?
\TMP@EnsureCode{91}{12}% [
\TMP@EnsureCode{93}{12}% ]
\TMP@EnsureCode{94}{7}% ^ (superscript)
\TMP@EnsureCode{95}{8}% _ (subscript)
\TMP@EnsureCode{96}{12}% `
\TMP@EnsureCode{124}{12}% |
\edef\HOLOGO@AtEnd{%
  \HOLOGO@AtEnd
  \escapechar\the\escapechar\relax
  \noexpand\endinput
}
\escapechar=92 %
%    \end{macrocode}
%
% \subsection{Logo list}
%
%    \begin{macro}{\hologoList}
%    \begin{macrocode}
\def\hologoList{%
  \hologoEntry{(La)TeX}{}{2011/10/01}%
  \hologoEntry{AmSLaTeX}{}{2010/04/16}%
  \hologoEntry{AmSTeX}{}{2010/04/16}%
  \hologoEntry{biber}{}{2011/10/01}%
  \hologoEntry{BibTeX}{}{2011/10/01}%
  \hologoEntry{BibTeX}{sf}{2011/10/01}%
  \hologoEntry{BibTeX}{sc}{2011/10/01}%
  \hologoEntry{BibTeX8}{}{2011/11/22}%
  \hologoEntry{ConTeXt}{}{2011/03/25}%
  \hologoEntry{ConTeXt}{narrow}{2011/03/25}%
  \hologoEntry{ConTeXt}{simple}{2011/03/25}%
  \hologoEntry{emTeX}{}{2010/04/26}%
  \hologoEntry{eTeX}{}{2010/04/08}%
  \hologoEntry{ExTeX}{}{2011/10/01}%
  \hologoEntry{HanTheThanh}{}{2011/11/29}%
  \hologoEntry{iniTeX}{}{2011/10/01}%
  \hologoEntry{KOMAScript}{}{2011/10/01}%
  \hologoEntry{La}{}{2010/05/08}%
  \hologoEntry{LaTeX}{}{2010/04/08}%
  \hologoEntry{LaTeX2e}{}{2010/04/08}%
  \hologoEntry{LaTeX3}{}{2010/04/24}%
  \hologoEntry{LaTeXe}{}{2010/04/08}%
  \hologoEntry{LaTeXML}{}{2011/11/22}%
  \hologoEntry{LaTeXTeX}{}{2011/10/01}%
  \hologoEntry{LuaLaTeX}{}{2010/04/08}%
  \hologoEntry{LuaTeX}{}{2010/04/08}%
  \hologoEntry{LyX}{}{2011/10/01}%
  \hologoEntry{METAFONT}{}{2011/10/01}%
  \hologoEntry{MetaFun}{}{2011/10/01}%
  \hologoEntry{METAPOST}{}{2011/10/01}%
  \hologoEntry{MetaPost}{}{2011/10/01}%
  \hologoEntry{MiKTeX}{}{2011/10/01}%
  \hologoEntry{NTS}{}{2011/10/01}%
  \hologoEntry{OzMF}{}{2011/10/01}%
  \hologoEntry{OzMP}{}{2011/10/01}%
  \hologoEntry{OzTeX}{}{2011/10/01}%
  \hologoEntry{OzTtH}{}{2011/10/01}%
  \hologoEntry{PCTeX}{}{2011/10/01}%
  \hologoEntry{pdfTeX}{}{2011/10/01}%
  \hologoEntry{pdfLaTeX}{}{2011/10/01}%
  \hologoEntry{PiC}{}{2011/10/01}%
  \hologoEntry{PiCTeX}{}{2011/10/01}%
  \hologoEntry{plainTeX}{}{2010/04/08}%
  \hologoEntry{plainTeX}{space}{2010/04/16}%
  \hologoEntry{plainTeX}{hyphen}{2010/04/16}%
  \hologoEntry{plainTeX}{runtogether}{2010/04/16}%
  \hologoEntry{SageTeX}{}{2011/11/22}%
  \hologoEntry{SLiTeX}{}{2011/10/01}%
  \hologoEntry{SLiTeX}{lift}{2011/10/01}%
  \hologoEntry{SLiTeX}{narrow}{2011/10/01}%
  \hologoEntry{SLiTeX}{simple}{2011/10/01}%
  \hologoEntry{SliTeX}{}{2011/10/01}%
  \hologoEntry{SliTeX}{narrow}{2011/10/01}%
  \hologoEntry{SliTeX}{simple}{2011/10/01}%
  \hologoEntry{SliTeX}{lift}{2011/10/01}%
  \hologoEntry{teTeX}{}{2011/10/01}%
  \hologoEntry{TeX}{}{2010/04/08}%
  \hologoEntry{TeX4ht}{}{2011/11/22}%
  \hologoEntry{TTH}{}{2011/11/22}%
  \hologoEntry{virTeX}{}{2011/10/01}%
  \hologoEntry{VTeX}{}{2010/04/24}%
  \hologoEntry{Xe}{}{2010/04/08}%
  \hologoEntry{XeLaTeX}{}{2010/04/08}%
  \hologoEntry{XeTeX}{}{2010/04/08}%
}
%    \end{macrocode}
%    \end{macro}
%
% \subsection{Load resources}
%
%    \begin{macrocode}
\begingroup\expandafter\expandafter\expandafter\endgroup
\expandafter\ifx\csname RequirePackage\endcsname\relax
  \def\TMP@RequirePackage#1[#2]{%
    \begingroup\expandafter\expandafter\expandafter\endgroup
    \expandafter\ifx\csname ver@#1.sty\endcsname\relax
      \input #1.sty\relax
    \fi
  }%
  \TMP@RequirePackage{ltxcmds}[2011/02/04]%
  \TMP@RequirePackage{infwarerr}[2010/04/08]%
  \TMP@RequirePackage{kvsetkeys}[2010/03/01]%
  \TMP@RequirePackage{kvdefinekeys}[2010/03/01]%
  \TMP@RequirePackage{pdftexcmds}[2010/04/01]%
  \TMP@RequirePackage{ifpdf}[2010/01/28]%
  \TMP@RequirePackage{ifluatex}[2010/03/01]%
  \ltx@IfUndefined{newif}{%
    \expandafter\let\csname newif\endcsname\ltx@newif
  }{}%
  \TMP@RequirePackage{ifxetex}[2009/01/23]%
  \TMP@RequirePackage{ifvtex}[2010/03/01]%
\else
  \RequirePackage{ltxcmds}[2011/02/04]%
  \RequirePackage{infwarerr}[2010/04/08]%
  \RequirePackage{kvsetkeys}[2010/03/01]%
  \RequirePackage{kvdefinekeys}[2010/03/01]%
  \RequirePackage{pdftexcmds}[2010/04/01]%
  \RequirePackage{ifpdf}[2010/01/28]%
  \RequirePackage{ifluatex}[2010/03/01]%
  \RequirePackage{ifxetex}[2009/01/23]%
  \RequirePackage{ifvtex}[2010/03/01]%
\fi
%    \end{macrocode}
%
%    \begin{macro}{\HOLOGO@IfDefined}
%    \begin{macrocode}
\def\HOLOGO@IfExists#1{%
  \ifx\@undefined#1%
    \expandafter\ltx@secondoftwo
  \else
    \ifx\relax#1%
      \expandafter\ltx@secondoftwo
    \else
      \expandafter\expandafter\expandafter\ltx@firstoftwo
    \fi
  \fi
}
%    \end{macrocode}
%    \end{macro}
%
% \subsection{Setup macros}
%
%    \begin{macro}{\hologoSetup}
%    \begin{macrocode}
\def\hologoSetup{%
  \let\HOLOGO@name\relax
  \HOLOGO@Setup
}
%    \end{macrocode}
%    \end{macro}
%
%    \begin{macro}{\hologoLogoSetup}
%    \begin{macrocode}
\def\hologoLogoSetup#1{%
  \edef\HOLOGO@name{#1}%
  \ltx@IfUndefined{HoLogo@\HOLOGO@name}{%
    \@PackageError{hologo}{%
      Unknown logo `\HOLOGO@name'%
    }\@ehc
    \ltx@gobble
  }{%
    \HOLOGO@Setup
  }%
}
%    \end{macrocode}
%    \end{macro}
%
%    \begin{macro}{\HOLOGO@Setup}
%    \begin{macrocode}
\def\HOLOGO@Setup{%
  \kvsetkeys{HoLogo}%
}
%    \end{macrocode}
%    \end{macro}
%
% \subsection{Options}
%
%    \begin{macro}{\HOLOGO@DeclareBoolOption}
%    \begin{macrocode}
\def\HOLOGO@DeclareBoolOption#1{%
  \expandafter\chardef\csname HOLOGOOPT@#1\endcsname\ltx@zero
  \kv@define@key{HoLogo}{#1}[true]{%
    \def\HOLOGO@temp{##1}%
    \ifx\HOLOGO@temp\HOLOGO@true
      \ifx\HOLOGO@name\relax
        \expandafter\chardef\csname HOLOGOOPT@#1\endcsname=\ltx@one
      \else
        \expandafter\chardef\csname
        HoLogoOpt@#1@\HOLOGO@name\endcsname\ltx@one
      \fi
      \HOLOGO@SetBreakAll{#1}%
    \else
      \ifx\HOLOGO@temp\HOLOGO@false
        \ifx\HOLOGO@name\relax
          \expandafter\chardef\csname HOLOGOOPT@#1\endcsname=\ltx@zero
        \else
          \expandafter\chardef\csname
          HoLogoOpt@#1@\HOLOGO@name\endcsname=\ltx@zero
        \fi
        \HOLOGO@SetBreakAll{#1}%
      \else
        \@PackageError{hologo}{%
          Unknown value `##1' for boolean option `#1'.\MessageBreak
          Known values are `true' and `false'%
        }\@ehc
      \fi
    \fi
  }%
}
%    \end{macrocode}
%    \end{macro}
%
%    \begin{macro}{\HOLOGO@SetBreakAll}
%    \begin{macrocode}
\def\HOLOGO@SetBreakAll#1{%
  \def\HOLOGO@temp{#1}%
  \ifx\HOLOGO@temp\HOLOGO@break
    \ifx\HOLOGO@name\relax
      \chardef\HOLOGOOPT@hyphenbreak=\HOLOGOOPT@break
      \chardef\HOLOGOOPT@spacebreak=\HOLOGOOPT@break
      \chardef\HOLOGOOPT@discretionarybreak=\HOLOGOOPT@break
    \else
      \expandafter\chardef
         \csname HoLogoOpt@hyphenbreak@\HOLOGO@name\endcsname=%
         \csname HoLogoOpt@break@\HOLOGO@name\endcsname
      \expandafter\chardef
         \csname HoLogoOpt@spacebreak@\HOLOGO@name\endcsname=%
         \csname HoLogoOpt@break@\HOLOGO@name\endcsname
      \expandafter\chardef
         \csname HoLogoOpt@discretionarybreak@\HOLOGO@name
             \endcsname=%
         \csname HoLogoOpt@break@\HOLOGO@name\endcsname
    \fi
  \fi
}
%    \end{macrocode}
%    \end{macro}
%
%    \begin{macro}{\HOLOGO@true}
%    \begin{macrocode}
\def\HOLOGO@true{true}
%    \end{macrocode}
%    \end{macro}
%    \begin{macro}{\HOLOGO@false}
%    \begin{macrocode}
\def\HOLOGO@false{false}
%    \end{macrocode}
%    \end{macro}
%    \begin{macro}{\HOLOGO@break}
%    \begin{macrocode}
\def\HOLOGO@break{break}
%    \end{macrocode}
%    \end{macro}
%
%    \begin{macrocode}
\HOLOGO@DeclareBoolOption{break}
\HOLOGO@DeclareBoolOption{hyphenbreak}
\HOLOGO@DeclareBoolOption{spacebreak}
\HOLOGO@DeclareBoolOption{discretionarybreak}
%    \end{macrocode}
%
%    \begin{macrocode}
\kv@define@key{HoLogo}{variant}{%
  \ifx\HOLOGO@name\relax
    \@PackageError{hologo}{%
      Option `variant' is not available in \string\hologoSetup,%
      \MessageBreak
      Use \string\hologoLogoSetup\space instead%
    }\@ehc
  \else
    \edef\HOLOGO@temp{#1}%
    \ifx\HOLOGO@temp\ltx@empty
      \expandafter
      \let\csname HoLogoOpt@variant@\HOLOGO@name\endcsname\@undefined
    \else
      \ltx@IfUndefined{HoLogo@\HOLOGO@name @\HOLOGO@temp}{%
        \@PackageError{hologo}{%
          Unknown variant `\HOLOGO@temp' of logo `\HOLOGO@name'%
        }\@ehc
      }{%
        \expandafter
        \let\csname HoLogoOpt@variant@\HOLOGO@name\endcsname
            \HOLOGO@temp
      }%
    \fi
  \fi
}
%    \end{macrocode}
%
%    \begin{macro}{\HOLOGO@Variant}
%    \begin{macrocode}
\def\HOLOGO@Variant#1{%
  #1%
  \ltx@ifundefined{HoLogoOpt@variant@#1}{%
  }{%
    @\csname HoLogoOpt@variant@#1\endcsname
  }%
}
%    \end{macrocode}
%    \end{macro}
%
% \subsection{Break/no-break support}
%
%    \begin{macro}{\HOLOGO@space}
%    \begin{macrocode}
\def\HOLOGO@space{%
  \ltx@ifundefined{HoLogoOpt@spacebreak@\HOLOGO@name}{%
    \ltx@ifundefined{HoLogoOpt@break@\HOLOGO@name}{%
      \chardef\HOLOGO@temp=\HOLOGOOPT@spacebreak
    }{%
      \chardef\HOLOGO@temp=%
        \csname HoLogoOpt@break@\HOLOGO@name\endcsname
    }%
  }{%
    \chardef\HOLOGO@temp=%
      \csname HoLogoOpt@spacebreak@\HOLOGO@name\endcsname
  }%
  \ifcase\HOLOGO@temp
    \penalty10000 %
  \fi
  \ltx@space
}
%    \end{macrocode}
%    \end{macro}
%
%    \begin{macro}{\HOLOGO@hyphen}
%    \begin{macrocode}
\def\HOLOGO@hyphen{%
  \ltx@ifundefined{HoLogoOpt@hyphenbreak@\HOLOGO@name}{%
    \ltx@ifundefined{HoLogoOpt@break@\HOLOGO@name}{%
      \chardef\HOLOGO@temp=\HOLOGOOPT@hyphenbreak
    }{%
      \chardef\HOLOGO@temp=%
        \csname HoLogoOpt@break@\HOLOGO@name\endcsname
    }%
  }{%
    \chardef\HOLOGO@temp=%
      \csname HoLogoOpt@hyphenbreak@\HOLOGO@name\endcsname
  }%
  \ifcase\HOLOGO@temp
    \ltx@mbox{-}%
  \else
    -%
  \fi
}
%    \end{macrocode}
%    \end{macro}
%
%    \begin{macro}{\HOLOGO@discretionary}
%    \begin{macrocode}
\def\HOLOGO@discretionary{%
  \ltx@ifundefined{HoLogoOpt@discretionarybreak@\HOLOGO@name}{%
    \ltx@ifundefined{HoLogoOpt@break@\HOLOGO@name}{%
      \chardef\HOLOGO@temp=\HOLOGOOPT@discretionarybreak
    }{%
      \chardef\HOLOGO@temp=%
        \csname HoLogoOpt@break@\HOLOGO@name\endcsname
    }%
  }{%
    \chardef\HOLOGO@temp=%
      \csname HoLogoOpt@discretionarybreak@\HOLOGO@name\endcsname
  }%
  \ifcase\HOLOGO@temp
  \else
    \-%
  \fi
}
%    \end{macrocode}
%    \end{macro}
%
%    \begin{macro}{\HOLOGO@mbox}
%    \begin{macrocode}
\def\HOLOGO@mbox#1{%
  \ltx@ifundefined{HoLogoOpt@break@\HOLOGO@name}{%
    \chardef\HOLOGO@temp=\HOLOGOOPT@hyphenbreak
  }{%
    \chardef\HOLOGO@temp=%
      \csname HoLogoOpt@break@\HOLOGO@name\endcsname
  }%
  \ifcase\HOLOGO@temp
    \ltx@mbox{#1}%
  \else
    #1%
  \fi
}
%    \end{macrocode}
%    \end{macro}
%
% \subsection{Font support}
%
%    \begin{macro}{\HoLogoFont@font}
%    \begin{tabular}{@{}ll@{}}
%    |#1|:& logo name\\
%    |#2|:& font short name\\
%    |#3|:& text
%    \end{tabular}
%    \begin{macrocode}
\def\HoLogoFont@font#1#2#3{%
  \begingroup
    \ltx@IfUndefined{HoLogoFont@logo@#1.#2}{%
      \ltx@IfUndefined{HoLogoFont@font@#2}{%
        \@PackageWarning{hologo}{%
          Missing font `#2' for logo `#1'%
        }%
        #3%
      }{%
        \csname HoLogoFont@font@#2\endcsname{#3}%
      }%
    }{%
      \csname HoLogoFont@logo@#1.#2\endcsname{#3}%
    }%
  \endgroup
}
%    \end{macrocode}
%    \end{macro}
%
%    \begin{macro}{\HoLogoFont@Def}
%    \begin{macrocode}
\def\HoLogoFont@Def#1{%
  \expandafter\def\csname HoLogoFont@font@#1\endcsname
}
%    \end{macrocode}
%    \end{macro}
%    \begin{macro}{\HoLogoFont@LogoDef}
%    \begin{macrocode}
\def\HoLogoFont@LogoDef#1#2{%
  \expandafter\def\csname HoLogoFont@logo@#1.#2\endcsname
}
%    \end{macrocode}
%    \end{macro}
%
% \subsubsection{Font defaults}
%
%    \begin{macro}{\HoLogoFont@font@general}
%    \begin{macrocode}
\HoLogoFont@Def{general}{}%
%    \end{macrocode}
%    \end{macro}
%
%    \begin{macro}{\HoLogoFont@font@rm}
%    \begin{macrocode}
\ltx@IfUndefined{rmfamily}{%
  \ltx@IfUndefined{rm}{%
  }{%
    \HoLogoFont@Def{rm}{\rm}%
  }%
}{%
  \HoLogoFont@Def{rm}{\rmfamily}%
}
%    \end{macrocode}
%    \end{macro}
%
%    \begin{macro}{\HoLogoFont@font@sf}
%    \begin{macrocode}
\ltx@IfUndefined{sffamily}{%
  \ltx@IfUndefined{sf}{%
  }{%
    \HoLogoFont@Def{sf}{\sf}%
  }%
}{%
  \HoLogoFont@Def{sf}{\sffamily}%
}
%    \end{macrocode}
%    \end{macro}
%
%    \begin{macro}{\HoLogoFont@font@bibsf}
%    In case of \hologo{plainTeX} the original small caps
%    variant is used as default. In \hologo{LaTeX}
%    the definition of package \xpackage{dtklogos} \cite{dtklogos}
%    is used.
%\begin{quote}
%\begin{verbatim}
%\DeclareRobustCommand{\BibTeX}{%
%  B%
%  \kern-.05em%
%  \hbox{%
%    $\m@th$% %% force math size calculations
%    \csname S@\f@size\endcsname
%    \fontsize\sf@size\z@
%    \math@fontsfalse
%    \selectfont
%    I%
%    \kern-.025em%
%    B
%  }%
%  \kern-.08em%
%  \-%
%  \TeX
%}
%\end{verbatim}
%\end{quote}
%    \begin{macrocode}
\ltx@IfUndefined{selectfont}{%
  \ltx@IfUndefined{tensc}{%
    \font\tensc=cmcsc10\relax
  }{}%
  \HoLogoFont@Def{bibsf}{\tensc}%
}{%
  \HoLogoFont@Def{bibsf}{%
    $\mathsurround=0pt$%
    \csname S@\f@size\endcsname
    \fontsize\sf@size{0pt}%
    \math@fontsfalse
    \selectfont
  }%
}
%    \end{macrocode}
%    \end{macro}
%
%    \begin{macro}{\HoLogoFont@font@sc}
%    \begin{macrocode}
\ltx@IfUndefined{scshape}{%
  \ltx@IfUndefined{tensc}{%
    \font\tensc=cmcsc10\relax
  }{}%
  \HoLogoFont@Def{sc}{\tensc}%
}{%
  \HoLogoFont@Def{sc}{\scshape}%
}
%    \end{macrocode}
%    \end{macro}
%
%    \begin{macro}{\HoLogoFont@font@sy}
%    \begin{macrocode}
\ltx@IfUndefined{usefont}{%
  \ltx@IfUndefined{tensy}{%
  }{%
    \HoLogoFont@Def{sy}{\tensy}%
  }%
}{%
  \HoLogoFont@Def{sy}{%
    \usefont{OMS}{cmsy}{m}{n}%
  }%
}
%    \end{macrocode}
%    \end{macro}
%
%    \begin{macro}{\HoLogoFont@font@logo}
%    \begin{macrocode}
\begingroup
  \def\x{LaTeX2e}%
\expandafter\endgroup
\ifx\fmtname\x
  \ltx@IfUndefined{logofamily}{%
    \DeclareRobustCommand\logofamily{%
      \not@math@alphabet\logofamily\relax
      \fontencoding{U}%
      \fontfamily{logo}%
      \selectfont
    }%
  }{}%
  \ltx@IfUndefined{logofamily}{%
  }{%
    \HoLogoFont@Def{logo}{\logofamily}%
  }%
\else
  \ltx@IfUndefined{tenlogo}{%
    \font\tenlogo=logo10\relax
  }{}%
  \HoLogoFont@Def{logo}{\tenlogo}%
\fi
%    \end{macrocode}
%    \end{macro}
%
% \subsubsection{Font setup}
%
%    \begin{macro}{\hologoFontSetup}
%    \begin{macrocode}
\def\hologoFontSetup{%
  \let\HOLOGO@name\relax
  \HOLOGO@FontSetup
}
%    \end{macrocode}
%    \end{macro}
%
%    \begin{macro}{\hologoLogoFontSetup}
%    \begin{macrocode}
\def\hologoLogoFontSetup#1{%
  \edef\HOLOGO@name{#1}%
  \ltx@IfUndefined{HoLogo@\HOLOGO@name}{%
    \@PackageError{hologo}{%
      Unknown logo `\HOLOGO@name'%
    }\@ehc
    \ltx@gobble
  }{%
    \HOLOGO@FontSetup
  }%
}
%    \end{macrocode}
%    \end{macro}
%
%    \begin{macro}{\HOLOGO@FontSetup}
%    \begin{macrocode}
\def\HOLOGO@FontSetup{%
  \kvsetkeys{HoLogoFont}%
}
%    \end{macrocode}
%    \end{macro}
%
%    \begin{macrocode}
\def\HOLOGO@temp#1{%
  \kv@define@key{HoLogoFont}{#1}{%
    \ifx\HOLOGO@name\relax
      \HoLogoFont@Def{#1}{##1}%
    \else
      \HoLogoFont@LogoDef\HOLOGO@name{#1}{##1}%
    \fi
  }%
}
\HOLOGO@temp{general}
\HOLOGO@temp{sf}
%    \end{macrocode}
%
% \subsection{Generic logo commands}
%
%    \begin{macrocode}
\HOLOGO@IfExists\hologo{%
  \@PackageError{hologo}{%
    \string\hologo\ltx@space is already defined.\MessageBreak
    Package loading is aborted%
  }\@ehc
  \HOLOGO@AtEnd
}%
\HOLOGO@IfExists\hologoRobust{%
  \@PackageError{hologo}{%
    \string\hologoRobust\ltx@space is already defined.\MessageBreak
    Package loading is aborted%
  }\@ehc
  \HOLOGO@AtEnd
}%
%    \end{macrocode}
%
% \subsubsection{\cs{hologo} and friends}
%
%    \begin{macrocode}
\ifluatex
  \expandafter\ltx@firstofone
\else
  \expandafter\ltx@gobble
\fi
{%
  \ltx@IfUndefined{ifincsname}{%
    \ifnum\luatexversion<36 %
      \expandafter\ltx@gobble
    \else
      \expandafter\ltx@firstofone
    \fi
    {%
      \begingroup
        \ifcase0%
            \directlua{%
              if tex.enableprimitives then %
                tex.enableprimitives('HOLOGO@', {'ifincsname'})%
              else %
                tex.print('1')%
              end%
            }%
            \ifx\HOLOGO@ifincsname\@undefined 1\fi%
            \relax
          \expandafter\ltx@firstofone
        \else
          \endgroup
          \expandafter\ltx@gobble
        \fi
        {%
          \global\let\ifincsname\HOLOGO@ifincsname
        }%
      \HOLOGO@temp
    }%
  }{}%
}
%    \end{macrocode}
%    \begin{macrocode}
\ltx@IfUndefined{ifincsname}{%
  \catcode`$=14 %
}{%
  \catcode`$=9 %
}
%    \end{macrocode}
%
%    \begin{macro}{\hologo}
%    \begin{macrocode}
\def\hologo#1{%
$ \ifincsname
$   \ltx@ifundefined{HoLogoCs@\HOLOGO@Variant{#1}}{%
$     #1%
$   }{%
$     \csname HoLogoCs@\HOLOGO@Variant{#1}\endcsname\ltx@firstoftwo
$   }%
$ \else
    \HOLOGO@IfExists\texorpdfstring\texorpdfstring\ltx@firstoftwo
    {%
      \hologoRobust{#1}%
    }{%
      \ltx@ifundefined{HoLogoBkm@\HOLOGO@Variant{#1}}{%
        \ltx@ifundefined{HoLogo@#1}{?#1?}{#1}%
      }{%
        \csname HoLogoBkm@\HOLOGO@Variant{#1}\endcsname
        \ltx@firstoftwo
      }%
    }%
$ \fi
}
%    \end{macrocode}
%    \end{macro}
%    \begin{macro}{\Hologo}
%    \begin{macrocode}
\def\Hologo#1{%
$ \ifincsname
$   \ltx@ifundefined{HoLogoCs@\HOLOGO@Variant{#1}}{%
$     #1%
$   }{%
$     \csname HoLogoCs@\HOLOGO@Variant{#1}\endcsname\ltx@secondoftwo
$   }%
$ \else
    \HOLOGO@IfExists\texorpdfstring\texorpdfstring\ltx@firstoftwo
    {%
      \HologoRobust{#1}%
    }{%
      \ltx@ifundefined{HoLogoBkm@\HOLOGO@Variant{#1}}{%
        \ltx@ifundefined{HoLogo@#1}{?#1?}{#1}%
      }{%
        \csname HoLogoBkm@\HOLOGO@Variant{#1}\endcsname
        \ltx@secondoftwo
      }%
    }%
$ \fi
}
%    \end{macrocode}
%    \end{macro}
%
%    \begin{macro}{\hologoVariant}
%    \begin{macrocode}
\def\hologoVariant#1#2{%
  \ifx\relax#2\relax
    \hologo{#1}%
  \else
$   \ifincsname
$     \ltx@ifundefined{HoLogoCs@#1@#2}{%
$       #1%
$     }{%
$       \csname HoLogoCs@#1@#2\endcsname\ltx@firstoftwo
$     }%
$   \else
      \HOLOGO@IfExists\texorpdfstring\texorpdfstring\ltx@firstoftwo
      {%
        \hologoVariantRobust{#1}{#2}%
      }{%
        \ltx@ifundefined{HoLogoBkm@#1@#2}{%
          \ltx@ifundefined{HoLogo@#1}{?#1?}{#1}%
        }{%
          \csname HoLogoBkm@#1@#2\endcsname
          \ltx@firstoftwo
        }%
      }%
$   \fi
  \fi
}
%    \end{macrocode}
%    \end{macro}
%    \begin{macro}{\HologoVariant}
%    \begin{macrocode}
\def\HologoVariant#1#2{%
  \ifx\relax#2\relax
    \Hologo{#1}%
  \else
$   \ifincsname
$     \ltx@ifundefined{HoLogoCs@#1@#2}{%
$       #1%
$     }{%
$       \csname HoLogoCs@#1@#2\endcsname\ltx@secondoftwo
$     }%
$   \else
      \HOLOGO@IfExists\texorpdfstring\texorpdfstring\ltx@firstoftwo
      {%
        \HologoVariantRobust{#1}{#2}%
      }{%
        \ltx@ifundefined{HoLogoBkm@#1@#2}{%
          \ltx@ifundefined{HoLogo@#1}{?#1?}{#1}%
        }{%
          \csname HoLogoBkm@#1@#2\endcsname
          \ltx@secondoftwo
        }%
      }%
$   \fi
  \fi
}
%    \end{macrocode}
%    \end{macro}
%
%    \begin{macrocode}
\catcode`\$=3 %
%    \end{macrocode}
%
% \subsubsection{\cs{hologoRobust} and friends}
%
%    \begin{macro}{\hologoRobust}
%    \begin{macrocode}
\ltx@IfUndefined{protected}{%
  \ltx@IfUndefined{DeclareRobustCommand}{%
    \def\hologoRobust#1%
  }{%
    \DeclareRobustCommand*\hologoRobust[1]%
  }%
}{%
  \protected\def\hologoRobust#1%
}%
{%
  \edef\HOLOGO@name{#1}%
  \ltx@IfUndefined{HoLogo@\HOLOGO@Variant\HOLOGO@name}{%
    \@PackageError{hologo}{%
      Unknown logo `\HOLOGO@name'%
    }\@ehc
    ?\HOLOGO@name?%
  }{%
    \ltx@IfUndefined{ver@tex4ht.sty}{%
      \HoLogoFont@font\HOLOGO@name{general}{%
        \csname HoLogo@\HOLOGO@Variant\HOLOGO@name\endcsname
        \ltx@firstoftwo
      }%
    }{%
      \ltx@IfUndefined{HoLogoHtml@\HOLOGO@Variant\HOLOGO@name}{%
        \HOLOGO@name
      }{%
        \csname HoLogoHtml@\HOLOGO@Variant\HOLOGO@name\endcsname
        \ltx@firstoftwo
      }%
    }%
  }%
}
%    \end{macrocode}
%    \end{macro}
%    \begin{macro}{\HologoRobust}
%    \begin{macrocode}
\ltx@IfUndefined{protected}{%
  \ltx@IfUndefined{DeclareRobustCommand}{%
    \def\HologoRobust#1%
  }{%
    \DeclareRobustCommand*\HologoRobust[1]%
  }%
}{%
  \protected\def\HologoRobust#1%
}%
{%
  \edef\HOLOGO@name{#1}%
  \ltx@IfUndefined{HoLogo@\HOLOGO@Variant\HOLOGO@name}{%
    \@PackageError{hologo}{%
      Unknown logo `\HOLOGO@name'%
    }\@ehc
    ?\HOLOGO@name?%
  }{%
    \ltx@IfUndefined{ver@tex4ht.sty}{%
      \HoLogoFont@font\HOLOGO@name{general}{%
        \csname HoLogo@\HOLOGO@Variant\HOLOGO@name\endcsname
        \ltx@secondoftwo
      }%
    }{%
      \ltx@IfUndefined{HoLogoHtml@\HOLOGO@Variant\HOLOGO@name}{%
        \expandafter\HOLOGO@Uppercase\HOLOGO@name
      }{%
        \csname HoLogoHtml@\HOLOGO@Variant\HOLOGO@name\endcsname
        \ltx@secondoftwo
      }%
    }%
  }%
}
%    \end{macrocode}
%    \end{macro}
%    \begin{macro}{\hologoVariantRobust}
%    \begin{macrocode}
\ltx@IfUndefined{protected}{%
  \ltx@IfUndefined{DeclareRobustCommand}{%
    \def\hologoVariantRobust#1#2%
  }{%
    \DeclareRobustCommand*\hologoVariantRobust[2]%
  }%
}{%
  \protected\def\hologoVariantRobust#1#2%
}%
{%
  \begingroup
    \hologoLogoSetup{#1}{variant={#2}}%
    \hologoRobust{#1}%
  \endgroup
}
%    \end{macrocode}
%    \end{macro}
%    \begin{macro}{\HologoVariantRobust}
%    \begin{macrocode}
\ltx@IfUndefined{protected}{%
  \ltx@IfUndefined{DeclareRobustCommand}{%
    \def\HologoVariantRobust#1#2%
  }{%
    \DeclareRobustCommand*\HologoVariantRobust[2]%
  }%
}{%
  \protected\def\HologoVariantRobust#1#2%
}%
{%
  \begingroup
    \hologoLogoSetup{#1}{variant={#2}}%
    \HologoRobust{#1}%
  \endgroup
}
%    \end{macrocode}
%    \end{macro}
%
%    \begin{macro}{\hologorobust}
%    Macro \cs{hologorobust} is only defined for compatibility.
%    Its use is deprecated.
%    \begin{macrocode}
\def\hologorobust{\hologoRobust}
%    \end{macrocode}
%    \end{macro}
%
% \subsection{Helpers}
%
%    \begin{macro}{\HOLOGO@Uppercase}
%    Macro \cs{HOLOGO@Uppercase} is restricted to \cs{uppercase},
%    because \hologo{plainTeX} or \hologo{iniTeX} do not provide
%    \cs{MakeUppercase}.
%    \begin{macrocode}
\def\HOLOGO@Uppercase#1{\uppercase{#1}}
%    \end{macrocode}
%    \end{macro}
%
%    \begin{macro}{\HOLOGO@PdfdocUnicode}
%    \begin{macrocode}
\def\HOLOGO@PdfdocUnicode{%
  \ifx\ifHy@unicode\iftrue
    \expandafter\ltx@secondoftwo
  \else
    \expandafter\ltx@firstoftwo
  \fi
}
%    \end{macrocode}
%    \end{macro}
%
%    \begin{macro}{\HOLOGO@Math}
%    \begin{macrocode}
\def\HOLOGO@MathSetup{%
  \mathsurround0pt\relax
  \HOLOGO@IfExists\f@series{%
    \if b\expandafter\ltx@car\f@series x\@nil
      \csname boldmath\endcsname
   \fi
  }{}%
}
%    \end{macrocode}
%    \end{macro}
%
%    \begin{macro}{\HOLOGO@TempDimen}
%    \begin{macrocode}
\dimendef\HOLOGO@TempDimen=\ltx@zero
%    \end{macrocode}
%    \end{macro}
%    \begin{macro}{\HOLOGO@NegativeKerning}
%    \begin{macrocode}
\def\HOLOGO@NegativeKerning#1{%
  \begingroup
    \HOLOGO@TempDimen=0pt\relax
    \comma@parse@normalized{#1}{%
      \ifdim\HOLOGO@TempDimen=0pt %
        \expandafter\HOLOGO@@NegativeKerning\comma@entry
      \fi
      \ltx@gobble
    }%
    \ifdim\HOLOGO@TempDimen<0pt %
      \kern\HOLOGO@TempDimen
    \fi
  \endgroup
}
%    \end{macrocode}
%    \end{macro}
%    \begin{macro}{\HOLOGO@@NegativeKerning}
%    \begin{macrocode}
\def\HOLOGO@@NegativeKerning#1#2{%
  \setbox\ltx@zero\hbox{#1#2}%
  \HOLOGO@TempDimen=\wd\ltx@zero
  \setbox\ltx@zero\hbox{#1\kern0pt#2}%
  \advance\HOLOGO@TempDimen by -\wd\ltx@zero
}
%    \end{macrocode}
%    \end{macro}
%
%    \begin{macro}{\HOLOGO@SpaceFactor}
%    \begin{macrocode}
\def\HOLOGO@SpaceFactor{%
  \spacefactor1000 %
}
%    \end{macrocode}
%    \end{macro}
%
%    \begin{macro}{\HOLOGO@Span}
%    \begin{macrocode}
\def\HOLOGO@Span#1#2{%
  \HCode{<span class="HoLogo-#1">}%
  #2%
  \HCode{</span>}%
}
%    \end{macrocode}
%    \end{macro}
%
% \subsubsection{Text subscript}
%
%    \begin{macro}{\HOLOGO@SubScript}%
%    \begin{macrocode}
\def\HOLOGO@SubScript#1{%
  \ltx@IfUndefined{textsubscript}{%
    \ltx@IfUndefined{text}{%
      \ltx@mbox{%
        \mathsurround=0pt\relax
        $%
          _{%
            \ltx@IfUndefined{sf@size}{%
              \mathrm{#1}%
            }{%
              \mbox{%
                \fontsize\sf@size{0pt}\selectfont
                #1%
              }%
            }%
          }%
        $%
      }%
    }{%
      \ltx@mbox{%
        \mathsurround=0pt\relax
        $_{\text{#1}}$%
      }%
    }%
  }{%
    \textsubscript{#1}%
  }%
}
%    \end{macrocode}
%    \end{macro}
%
% \subsection{\hologo{TeX} and friends}
%
% \subsubsection{\hologo{TeX}}
%
%    \begin{macro}{\HoLogo@TeX}
%    Source: \hologo{LaTeX} kernel.
%    \begin{macrocode}
\def\HoLogo@TeX#1{%
  T\kern-.1667em\lower.5ex\hbox{E}\kern-.125emX\HOLOGO@SpaceFactor
}
%    \end{macrocode}
%    \end{macro}
%    \begin{macro}{\HoLogoHtml@TeX}
%    \begin{macrocode}
\def\HoLogoHtml@TeX#1{%
  \HoLogoCss@TeX
  \HOLOGO@Span{TeX}{%
    T%
    \HOLOGO@Span{e}{%
      E%
    }%
    X%
  }%
}
%    \end{macrocode}
%    \end{macro}
%    \begin{macro}{\HoLogoCss@TeX}
%    \begin{macrocode}
\def\HoLogoCss@TeX{%
  \Css{%
    span.HoLogo-TeX span.HoLogo-e{%
      position:relative;%
      top:.5ex;%
      margin-left:-.1667em;%
      margin-right:-.125em;%
    }%
  }%
  \Css{%
    a span.HoLogo-TeX span.HoLogo-e{%
      text-decoration:none;%
    }%
  }%
  \global\let\HoLogoCss@TeX\relax
}
%    \end{macrocode}
%    \end{macro}
%
% \subsubsection{\hologo{plainTeX}}
%
%    \begin{macro}{\HoLogo@plainTeX@space}
%    Source: ``The \hologo{TeX}book''
%    \begin{macrocode}
\def\HoLogo@plainTeX@space#1{%
  \HOLOGO@mbox{#1{p}{P}lain}\HOLOGO@space\hologo{TeX}%
}
%    \end{macrocode}
%    \end{macro}
%    \begin{macro}{\HoLogoCs@plainTeX@space}
%    \begin{macrocode}
\def\HoLogoCs@plainTeX@space#1{#1{p}{P}lain TeX}%
%    \end{macrocode}
%    \end{macro}
%    \begin{macro}{\HoLogoBkm@plainTeX@space}
%    \begin{macrocode}
\def\HoLogoBkm@plainTeX@space#1{%
  #1{p}{P}lain \hologo{TeX}%
}
%    \end{macrocode}
%    \end{macro}
%    \begin{macro}{\HoLogoHtml@plainTeX@space}
%    \begin{macrocode}
\def\HoLogoHtml@plainTeX@space#1{%
  #1{p}{P}lain \hologo{TeX}%
}
%    \end{macrocode}
%    \end{macro}
%
%    \begin{macro}{\HoLogo@plainTeX@hyphen}
%    \begin{macrocode}
\def\HoLogo@plainTeX@hyphen#1{%
  \HOLOGO@mbox{#1{p}{P}lain}\HOLOGO@hyphen\hologo{TeX}%
}
%    \end{macrocode}
%    \end{macro}
%    \begin{macro}{\HoLogoCs@plainTeX@hyphen}
%    \begin{macrocode}
\def\HoLogoCs@plainTeX@hyphen#1{#1{p}{P}lain-TeX}
%    \end{macrocode}
%    \end{macro}
%    \begin{macro}{\HoLogoBkm@plainTeX@hyphen}
%    \begin{macrocode}
\def\HoLogoBkm@plainTeX@hyphen#1{%
  #1{p}{P}lain-\hologo{TeX}%
}
%    \end{macrocode}
%    \end{macro}
%    \begin{macro}{\HoLogoHtml@plainTeX@hyphen}
%    \begin{macrocode}
\def\HoLogoHtml@plainTeX@hyphen#1{%
  #1{p}{P}lain-\hologo{TeX}%
}
%    \end{macrocode}
%    \end{macro}
%
%    \begin{macro}{\HoLogo@plainTeX@runtogether}
%    \begin{macrocode}
\def\HoLogo@plainTeX@runtogether#1{%
  \HOLOGO@mbox{#1{p}{P}lain\hologo{TeX}}%
}
%    \end{macrocode}
%    \end{macro}
%    \begin{macro}{\HoLogoCs@plainTeX@runtogether}
%    \begin{macrocode}
\def\HoLogoCs@plainTeX@runtogether#1{#1{p}{P}lainTeX}
%    \end{macrocode}
%    \end{macro}
%    \begin{macro}{\HoLogoBkm@plainTeX@runtogether}
%    \begin{macrocode}
\def\HoLogoBkm@plainTeX@runtogether#1{%
  #1{p}{P}lain\hologo{TeX}%
}
%    \end{macrocode}
%    \end{macro}
%    \begin{macro}{\HoLogoHtml@plainTeX@runtogether}
%    \begin{macrocode}
\def\HoLogoHtml@plainTeX@runtogether#1{%
  #1{p}{P}lain\hologo{TeX}%
}
%    \end{macrocode}
%    \end{macro}
%
%    \begin{macro}{\HoLogo@plainTeX}
%    \begin{macrocode}
\def\HoLogo@plainTeX{\HoLogo@plainTeX@space}
%    \end{macrocode}
%    \end{macro}
%    \begin{macro}{\HoLogoCs@plainTeX}
%    \begin{macrocode}
\def\HoLogoCs@plainTeX{\HoLogoCs@plainTeX@space}
%    \end{macrocode}
%    \end{macro}
%    \begin{macro}{\HoLogoBkm@plainTeX}
%    \begin{macrocode}
\def\HoLogoBkm@plainTeX{\HoLogoBkm@plainTeX@space}
%    \end{macrocode}
%    \end{macro}
%    \begin{macro}{\HoLogoHtml@plainTeX}
%    \begin{macrocode}
\def\HoLogoHtml@plainTeX{\HoLogoHtml@plainTeX@space}
%    \end{macrocode}
%    \end{macro}
%
% \subsubsection{\hologo{LaTeX}}
%
%    Source: \hologo{LaTeX} kernel.
%\begin{quote}
%\begin{verbatim}
%\DeclareRobustCommand{\LaTeX}{%
%  L%
%  \kern-.36em%
%  {%
%    \sbox\z@ T%
%    \vbox to\ht\z@{%
%      \hbox{%
%        \check@mathfonts
%        \fontsize\sf@size\z@
%        \math@fontsfalse
%        \selectfont
%        A%
%      }%
%      \vss
%    }%
%  }%
%  \kern-.15em%
%  \TeX
%}
%\end{verbatim}
%\end{quote}
%
%    \begin{macro}{\HoLogo@La}
%    \begin{macrocode}
\def\HoLogo@La#1{%
  L%
  \kern-.36em%
  \begingroup
    \setbox\ltx@zero\hbox{T}%
    \vbox to\ht\ltx@zero{%
      \hbox{%
        \ltx@ifundefined{check@mathfonts}{%
          \csname sevenrm\endcsname
        }{%
          \check@mathfonts
          \fontsize\sf@size{0pt}%
          \math@fontsfalse\selectfont
        }%
        A%
      }%
      \vss
    }%
  \endgroup
}
%    \end{macrocode}
%    \end{macro}
%
%    \begin{macro}{\HoLogo@LaTeX}
%    Source: \hologo{LaTeX} kernel.
%    \begin{macrocode}
\def\HoLogo@LaTeX#1{%
  \hologo{La}%
  \kern-.15em%
  \hologo{TeX}%
}
%    \end{macrocode}
%    \end{macro}
%    \begin{macro}{\HoLogoHtml@LaTeX}
%    \begin{macrocode}
\def\HoLogoHtml@LaTeX#1{%
  \HoLogoCss@LaTeX
  \HOLOGO@Span{LaTeX}{%
    L%
    \HOLOGO@Span{a}{%
      A%
    }%
    \hologo{TeX}%
  }%
}
%    \end{macrocode}
%    \end{macro}
%    \begin{macro}{\HoLogoCss@LaTeX}
%    \begin{macrocode}
\def\HoLogoCss@LaTeX{%
  \Css{%
    span.HoLogo-LaTeX span.HoLogo-a{%
      position:relative;%
      top:-.5ex;%
      margin-left:-.36em;%
      margin-right:-.15em;%
      font-size:85\%;%
    }%
  }%
  \global\let\HoLogoCss@LaTeX\relax
}
%    \end{macrocode}
%    \end{macro}
%
% \subsubsection{\hologo{(La)TeX}}
%
%    \begin{macro}{\HoLogo@LaTeXTeX}
%    The kerning around the parentheses is taken
%    from package \xpackage{dtklogos} \cite{dtklogos}.
%\begin{quote}
%\begin{verbatim}
%\DeclareRobustCommand{\LaTeXTeX}{%
%  (%
%  \kern-.15em%
%  L%
%  \kern-.36em%
%  {%
%    \sbox\z@ T%
%    \vbox to\ht0{%
%      \hbox{%
%        $\m@th$%
%        \csname S@\f@size\endcsname
%        \fontsize\sf@size\z@
%        \math@fontsfalse
%        \selectfont
%        A%
%      }%
%      \vss
%    }%
%  }%
%  \kern-.2em%
%  )%
%  \kern-.15em%
%  \TeX
%}
%\end{verbatim}
%\end{quote}
%    \begin{macrocode}
\def\HoLogo@LaTeXTeX#1{%
  (%
  \kern-.15em%
  \hologo{La}%
  \kern-.2em%
  )%
  \kern-.15em%
  \hologo{TeX}%
}
%    \end{macrocode}
%    \end{macro}
%    \begin{macro}{\HoLogoBkm@LaTeXTeX}
%    \begin{macrocode}
\def\HoLogoBkm@LaTeXTeX#1{(La)TeX}
%    \end{macrocode}
%    \end{macro}
%
%    \begin{macro}{\HoLogo@(La)TeX}
%    \begin{macrocode}
\expandafter
\let\csname HoLogo@(La)TeX\endcsname\HoLogo@LaTeXTeX
%    \end{macrocode}
%    \end{macro}
%    \begin{macro}{\HoLogoBkm@(La)TeX}
%    \begin{macrocode}
\expandafter
\let\csname HoLogoBkm@(La)TeX\endcsname\HoLogoBkm@LaTeXTeX
%    \end{macrocode}
%    \end{macro}
%    \begin{macro}{\HoLogoHtml@LaTeXTeX}
%    \begin{macrocode}
\def\HoLogoHtml@LaTeXTeX#1{%
  \HoLogoCss@LaTeXTeX
  \HOLOGO@Span{LaTeXTeX}{%
    (%
    \HOLOGO@Span{L}{L}%
    \HOLOGO@Span{a}{A}%
    \HOLOGO@Span{ParenRight}{)}%
    \hologo{TeX}%
  }%
}
%    \end{macrocode}
%    \end{macro}
%    \begin{macro}{\HoLogoHtml@(La)TeX}
%    Kerning after opening parentheses and before closing parentheses
%    is $-0.1$\,em. The original values $-0.15$\,em
%    looked too ugly for a serif font.
%    \begin{macrocode}
\expandafter
\let\csname HoLogoHtml@(La)TeX\endcsname\HoLogoHtml@LaTeXTeX
%    \end{macrocode}
%    \end{macro}
%    \begin{macro}{\HoLogoCss@LaTeXTeX}
%    \begin{macrocode}
\def\HoLogoCss@LaTeXTeX{%
  \Css{%
    span.HoLogo-LaTeXTeX span.HoLogo-L{%
      margin-left:-.1em;%
    }%
  }%
  \Css{%
    span.HoLogo-LaTeXTeX span.HoLogo-a{%
      position:relative;%
      top:-.5ex;%
      margin-left:-.36em;%
      margin-right:-.1em;%
      font-size:85\%;%
    }%
  }%
  \Css{%
    span.HoLogo-LaTeXTeX span.HoLogo-ParenRight{%
      margin-right:-.15em;%
    }%
  }%
  \global\let\HoLogoCss@LaTeXTeX\relax
}
%    \end{macrocode}
%    \end{macro}
%
% \subsubsection{\hologo{LaTeXe}}
%
%    \begin{macro}{\HoLogo@LaTeXe}
%    Source: \hologo{LaTeX} kernel
%    \begin{macrocode}
\def\HoLogo@LaTeXe#1{%
  \hologo{LaTeX}%
  \kern.15em%
  \hbox{%
    \HOLOGO@MathSetup
    2%
    $_{\textstyle\varepsilon}$%
  }%
}
%    \end{macrocode}
%    \end{macro}
%
%    \begin{macro}{\HoLogoCs@LaTeXe}
%    \begin{macrocode}
\ifnum64=`\^^^^0040\relax % test for big chars of LuaTeX/XeTeX
  \catcode`\$=9 %
  \catcode`\&=14 %
\else
  \catcode`\$=14 %
  \catcode`\&=9 %
\fi
\def\HoLogoCs@LaTeXe#1{%
  LaTeX2%
$ \string ^^^^0395%
& e%
}%
\catcode`\$=3 %
\catcode`\&=4 %
%    \end{macrocode}
%    \end{macro}
%
%    \begin{macro}{\HoLogoBkm@LaTeXe}
%    \begin{macrocode}
\def\HoLogoBkm@LaTeXe#1{%
  \hologo{LaTeX}%
  2%
  \HOLOGO@PdfdocUnicode{e}{\textepsilon}%
}
%    \end{macrocode}
%    \end{macro}
%
%    \begin{macro}{\HoLogoHtml@LaTeXe}
%    \begin{macrocode}
\def\HoLogoHtml@LaTeXe#1{%
  \HoLogoCss@LaTeXe
  \HOLOGO@Span{LaTeX2e}{%
    \hologo{LaTeX}%
    \HOLOGO@Span{2}{2}%
    \HOLOGO@Span{e}{%
      \HOLOGO@MathSetup
      \ensuremath{\textstyle\varepsilon}%
    }%
  }%
}
%    \end{macrocode}
%    \end{macro}
%    \begin{macro}{\HoLogoCss@LaTeXe}
%    \begin{macrocode}
\def\HoLogoCss@LaTeXe{%
  \Css{%
    span.HoLogo-LaTeX2e span.HoLogo-2{%
      padding-left:.15em;%
    }%
  }%
  \Css{%
    span.HoLogo-LaTeX2e span.HoLogo-e{%
      position:relative;%
      top:.35ex;%
      text-decoration:none;%
    }%
  }%
  \global\let\HoLogoCss@LaTeXe\relax
}
%    \end{macrocode}
%    \end{macro}
%
%    \begin{macro}{\HoLogo@LaTeX2e}
%    \begin{macrocode}
\expandafter
\let\csname HoLogo@LaTeX2e\endcsname\HoLogo@LaTeXe
%    \end{macrocode}
%    \end{macro}
%    \begin{macro}{\HoLogoCs@LaTeX2e}
%    \begin{macrocode}
\expandafter
\let\csname HoLogoCs@LaTeX2e\endcsname\HoLogoCs@LaTeXe
%    \end{macrocode}
%    \end{macro}
%    \begin{macro}{\HoLogoBkm@LaTeX2e}
%    \begin{macrocode}
\expandafter
\let\csname HoLogoBkm@LaTeX2e\endcsname\HoLogoBkm@LaTeXe
%    \end{macrocode}
%    \end{macro}
%    \begin{macro}{\HoLogoHtml@LaTeX2e}
%    \begin{macrocode}
\expandafter
\let\csname HoLogoHtml@LaTeX2e\endcsname\HoLogoHtml@LaTeXe
%    \end{macrocode}
%    \end{macro}
%
% \subsubsection{\hologo{LaTeX3}}
%
%    \begin{macro}{\HoLogo@LaTeX3}
%    Source: \hologo{LaTeX} kernel
%    \begin{macrocode}
\expandafter\def\csname HoLogo@LaTeX3\endcsname#1{%
  \hologo{LaTeX}%
  3%
}
%    \end{macrocode}
%    \end{macro}
%
%    \begin{macro}{\HoLogoBkm@LaTeX3}
%    \begin{macrocode}
\expandafter\def\csname HoLogoBkm@LaTeX3\endcsname#1{%
  \hologo{LaTeX}%
  3%
}
%    \end{macrocode}
%    \end{macro}
%    \begin{macro}{\HoLogoHtml@LaTeX3}
%    \begin{macrocode}
\expandafter
\let\csname HoLogoHtml@LaTeX3\expandafter\endcsname
\csname HoLogo@LaTeX3\endcsname
%    \end{macrocode}
%    \end{macro}
%
% \subsubsection{\hologo{LaTeXML}}
%
%    \begin{macro}{\HoLogo@LaTeXML}
%    \begin{macrocode}
\def\HoLogo@LaTeXML#1{%
  \HOLOGO@mbox{%
    \hologo{La}%
    \kern-.15em%
    T%
    \kern-.1667em%
    \lower.5ex\hbox{E}%
    \kern-.125em%
    \HoLogoFont@font{LaTeXML}{sc}{xml}%
  }%
}
%    \end{macrocode}
%    \end{macro}
%    \begin{macro}{\HoLogoHtml@pdfLaTeX}
%    \begin{macrocode}
\def\HoLogoHtml@LaTeXML#1{%
  \HOLOGO@Span{LaTeXML}{%
    \HoLogoCss@LaTeX
    \HoLogoCss@TeX
    \HOLOGO@Span{LaTeX}{%
      L%
      \HOLOGO@Span{a}{%
        A%
      }%
    }%
    \HOLOGO@Span{TeX}{%
      T%
      \HOLOGO@Span{e}{%
        E%
      }%
    }%
    \HCode{<span style="font-variant: small-caps;">}%
    xml%
    \HCode{</span>}%
  }%
}
%    \end{macrocode}
%    \end{macro}
%
% \subsubsection{\hologo{eTeX}}
%
%    \begin{macro}{\HoLogo@eTeX}
%    Source: package \xpackage{etex}
%    \begin{macrocode}
\def\HoLogo@eTeX#1{%
  \ltx@mbox{%
    \HOLOGO@MathSetup
    $\varepsilon$%
    -%
    \HOLOGO@NegativeKerning{-T,T-,To}%
    \hologo{TeX}%
  }%
}
%    \end{macrocode}
%    \end{macro}
%    \begin{macro}{\HoLogoCs@eTeX}
%    \begin{macrocode}
\ifnum64=`\^^^^0040\relax % test for big chars of LuaTeX/XeTeX
  \catcode`\$=9 %
  \catcode`\&=14 %
\else
  \catcode`\$=14 %
  \catcode`\&=9 %
\fi
\def\HoLogoCs@eTeX#1{%
$ #1{\string ^^^^0395}{\string ^^^^03b5}%
& #1{e}{E}%
  TeX%
}%
\catcode`\$=3 %
\catcode`\&=4 %
%    \end{macrocode}
%    \end{macro}
%    \begin{macro}{\HoLogoBkm@eTeX}
%    \begin{macrocode}
\def\HoLogoBkm@eTeX#1{%
  \HOLOGO@PdfdocUnicode{#1{e}{E}}{\textepsilon}%
  -%
  \hologo{TeX}%
}
%    \end{macrocode}
%    \end{macro}
%    \begin{macro}{\HoLogoHtml@eTeX}
%    \begin{macrocode}
\def\HoLogoHtml@eTeX#1{%
  \ltx@mbox{%
    \HOLOGO@MathSetup
    $\varepsilon$%
    -%
    \hologo{TeX}%
  }%
}
%    \end{macrocode}
%    \end{macro}
%
% \subsubsection{\hologo{iniTeX}}
%
%    \begin{macro}{\HoLogo@iniTeX}
%    \begin{macrocode}
\def\HoLogo@iniTeX#1{%
  \HOLOGO@mbox{%
    #1{i}{I}ni\hologo{TeX}%
  }%
}
%    \end{macrocode}
%    \end{macro}
%    \begin{macro}{\HoLogoCs@iniTeX}
%    \begin{macrocode}
\def\HoLogoCs@iniTeX#1{#1{i}{I}niTeX}
%    \end{macrocode}
%    \end{macro}
%    \begin{macro}{\HoLogoBkm@iniTeX}
%    \begin{macrocode}
\def\HoLogoBkm@iniTeX#1{%
  #1{i}{I}ni\hologo{TeX}%
}
%    \end{macrocode}
%    \end{macro}
%    \begin{macro}{\HoLogoHtml@iniTeX}
%    \begin{macrocode}
\let\HoLogoHtml@iniTeX\HoLogo@iniTeX
%    \end{macrocode}
%    \end{macro}
%
% \subsubsection{\hologo{virTeX}}
%
%    \begin{macro}{\HoLogo@virTeX}
%    \begin{macrocode}
\def\HoLogo@virTeX#1{%
  \HOLOGO@mbox{%
    #1{v}{V}ir\hologo{TeX}%
  }%
}
%    \end{macrocode}
%    \end{macro}
%    \begin{macro}{\HoLogoCs@virTeX}
%    \begin{macrocode}
\def\HoLogoCs@virTeX#1{#1{v}{V}irTeX}
%    \end{macrocode}
%    \end{macro}
%    \begin{macro}{\HoLogoBkm@virTeX}
%    \begin{macrocode}
\def\HoLogoBkm@virTeX#1{%
  #1{v}{V}ir\hologo{TeX}%
}
%    \end{macrocode}
%    \end{macro}
%    \begin{macro}{\HoLogoHtml@virTeX}
%    \begin{macrocode}
\let\HoLogoHtml@virTeX\HoLogo@virTeX
%    \end{macrocode}
%    \end{macro}
%
% \subsubsection{\hologo{SliTeX}}
%
% \paragraph{Definitions of the three variants.}
%
%    \begin{macro}{\HoLogo@SLiTeX@lift}
%    \begin{macrocode}
\def\HoLogo@SLiTeX@lift#1{%
  \HoLogoFont@font{SliTeX}{rm}{%
    S%
    \kern-.06em%
    L%
    \kern-.18em%
    \raise.32ex\hbox{\HoLogoFont@font{SliTeX}{sc}{i}}%
    \HOLOGO@discretionary
    \kern-.06em%
    \hologo{TeX}%
  }%
}
%    \end{macrocode}
%    \end{macro}
%    \begin{macro}{\HoLogoBkm@SLiTeX@lift}
%    \begin{macrocode}
\def\HoLogoBkm@SLiTeX@lift#1{SLiTeX}
%    \end{macrocode}
%    \end{macro}
%    \begin{macro}{\HoLogoHtml@SLiTeX@lift}
%    \begin{macrocode}
\def\HoLogoHtml@SLiTeX@lift#1{%
  \HoLogoCss@SLiTeX@lift
  \HOLOGO@Span{SLiTeX-lift}{%
    \HoLogoFont@font{SliTeX}{rm}{%
      S%
      \HOLOGO@Span{L}{L}%
      \HOLOGO@Span{i}{i}%
      \hologo{TeX}%
    }%
  }%
}
%    \end{macrocode}
%    \end{macro}
%    \begin{macro}{\HoLogoCss@SLiTeX@lift}
%    \begin{macrocode}
\def\HoLogoCss@SLiTeX@lift{%
  \Css{%
    span.HoLogo-SLiTeX-lift span.HoLogo-L{%
      margin-left:-.06em;%
      margin-right:-.18em;%
    }%
  }%
  \Css{%
    span.HoLogo-SLiTeX-lift span.HoLogo-i{%
      position:relative;%
      top:-.32ex;%
      margin-right:-.06em;%
      font-variant:small-caps;%
    }%
  }%
  \global\let\HoLogoCss@SLiTeX@lift\relax
}
%    \end{macrocode}
%    \end{macro}
%
%    \begin{macro}{\HoLogo@SliTeX@simple}
%    \begin{macrocode}
\def\HoLogo@SliTeX@simple#1{%
  \HoLogoFont@font{SliTeX}{rm}{%
    \ltx@mbox{%
      \HoLogoFont@font{SliTeX}{sc}{Sli}%
    }%
    \HOLOGO@discretionary
    \hologo{TeX}%
  }%
}
%    \end{macrocode}
%    \end{macro}
%    \begin{macro}{\HoLogoBkm@SliTeX@simple}
%    \begin{macrocode}
\def\HoLogoBkm@SliTeX@simple#1{SliTeX}
%    \end{macrocode}
%    \end{macro}
%    \begin{macro}{\HoLogoHtml@SliTeX@simple}
%    \begin{macrocode}
\let\HoLogoHtml@SliTeX@simple\HoLogo@SliTeX@simple
%    \end{macrocode}
%    \end{macro}
%
%    \begin{macro}{\HoLogo@SliTeX@narrow}
%    \begin{macrocode}
\def\HoLogo@SliTeX@narrow#1{%
  \HoLogoFont@font{SliTeX}{rm}{%
    \ltx@mbox{%
      S%
      \kern-.06em%
      \HoLogoFont@font{SliTeX}{sc}{%
        l%
        \kern-.035em%
        i%
      }%
    }%
    \HOLOGO@discretionary
    \kern-.06em%
    \hologo{TeX}%
  }%
}
%    \end{macrocode}
%    \end{macro}
%    \begin{macro}{\HoLogoBkm@SliTeX@narrow}
%    \begin{macrocode}
\def\HoLogoBkm@SliTeX@narrow#1{SliTeX}
%    \end{macrocode}
%    \end{macro}
%    \begin{macro}{\HoLogoHtml@SliTeX@narrow}
%    \begin{macrocode}
\def\HoLogoHtml@SliTeX@narrow#1{%
  \HoLogoCss@SliTeX@narrow
  \HOLOGO@Span{SliTeX-narrow}{%
    \HoLogoFont@font{SliTeX}{rm}{%
      S%
        \HOLOGO@Span{l}{l}%
        \HOLOGO@Span{i}{i}%
      \hologo{TeX}%
    }%
  }%
}
%    \end{macrocode}
%    \end{macro}
%    \begin{macro}{\HoLogoCss@SliTeX@narrow}
%    \begin{macrocode}
\def\HoLogoCss@SliTeX@narrow{%
  \Css{%
    span.HoLogo-SliTeX-narrow span.HoLogo-l{%
      margin-left:-.06em;%
      margin-right:-.035em;%
      font-variant:small-caps;%
    }%
  }%
  \Css{%
    span.HoLogo-SliTeX-narrow span.HoLogo-i{%
      margin-right:-.06em;%
      font-variant:small-caps;%
    }%
  }%
  \global\let\HoLogoCss@SliTeX@narrow\relax
}
%    \end{macrocode}
%    \end{macro}
%
% \paragraph{Macro set completion.}
%
%    \begin{macro}{\HoLogo@SLiTeX@simple}
%    \begin{macrocode}
\def\HoLogo@SLiTeX@simple{\HoLogo@SliTeX@simple}
%    \end{macrocode}
%    \end{macro}
%    \begin{macro}{\HoLogoBkm@SLiTeX@simple}
%    \begin{macrocode}
\def\HoLogoBkm@SLiTeX@simple{\HoLogoBkm@SliTeX@simple}
%    \end{macrocode}
%    \end{macro}
%    \begin{macro}{\HoLogoHtml@SLiTeX@simple}
%    \begin{macrocode}
\def\HoLogoHtml@SLiTeX@simple{\HoLogoHtml@SliTeX@simple}
%    \end{macrocode}
%    \end{macro}
%
%    \begin{macro}{\HoLogo@SLiTeX@narrow}
%    \begin{macrocode}
\def\HoLogo@SLiTeX@narrow{\HoLogo@SliTeX@narrow}
%    \end{macrocode}
%    \end{macro}
%    \begin{macro}{\HoLogoBkm@SLiTeX@narrow}
%    \begin{macrocode}
\def\HoLogoBkm@SLiTeX@narrow{\HoLogoBkm@SliTeX@narrow}
%    \end{macrocode}
%    \end{macro}
%    \begin{macro}{\HoLogoHtml@SLiTeX@narrow}
%    \begin{macrocode}
\def\HoLogoHtml@SLiTeX@narrow{\HoLogoHtml@SliTeX@narrow}
%    \end{macrocode}
%    \end{macro}
%
%    \begin{macro}{\HoLogo@SliTeX@lift}
%    \begin{macrocode}
\def\HoLogo@SliTeX@lift{\HoLogo@SLiTeX@lift}
%    \end{macrocode}
%    \end{macro}
%    \begin{macro}{\HoLogoBkm@SliTeX@lift}
%    \begin{macrocode}
\def\HoLogoBkm@SliTeX@lift{\HoLogoBkm@SLiTeX@lift}
%    \end{macrocode}
%    \end{macro}
%    \begin{macro}{\HoLogoHtml@SliTeX@lift}
%    \begin{macrocode}
\def\HoLogoHtml@SliTeX@lift{\HoLogoHtml@SLiTeX@lift}
%    \end{macrocode}
%    \end{macro}
%
% \paragraph{Defaults.}
%
%    \begin{macro}{\HoLogo@SLiTeX}
%    \begin{macrocode}
\def\HoLogo@SLiTeX{\HoLogo@SLiTeX@lift}
%    \end{macrocode}
%    \end{macro}
%    \begin{macro}{\HoLogoBkm@SLiTeX}
%    \begin{macrocode}
\def\HoLogoBkm@SLiTeX{\HoLogoBkm@SLiTeX@lift}
%    \end{macrocode}
%    \end{macro}
%    \begin{macro}{\HoLogoHtml@SLiTeX}
%    \begin{macrocode}
\def\HoLogoHtml@SLiTeX{\HoLogoHtml@SLiTeX@lift}
%    \end{macrocode}
%    \end{macro}
%
%    \begin{macro}{\HoLogo@SliTeX}
%    \begin{macrocode}
\def\HoLogo@SliTeX{\HoLogo@SliTeX@narrow}
%    \end{macrocode}
%    \end{macro}
%    \begin{macro}{\HoLogoBkm@SliTeX}
%    \begin{macrocode}
\def\HoLogoBkm@SliTeX{\HoLogoBkm@SliTeX@narrow}
%    \end{macrocode}
%    \end{macro}
%    \begin{macro}{\HoLogoHtml@SliTeX}
%    \begin{macrocode}
\def\HoLogoHtml@SliTeX{\HoLogoHtml@SliTeX@narrow}
%    \end{macrocode}
%    \end{macro}
%
% \subsubsection{\hologo{LuaTeX}}
%
%    \begin{macro}{\HoLogo@LuaTeX}
%    The kerning is an idea of Hans Hagen, see mailing list
%    `luatex at tug dot org' in March 2010.
%    \begin{macrocode}
\def\HoLogo@LuaTeX#1{%
  \HOLOGO@mbox{%
    Lua%
    \HOLOGO@NegativeKerning{aT,oT,To}%
    \hologo{TeX}%
  }%
}
%    \end{macrocode}
%    \end{macro}
%    \begin{macro}{\HoLogoHtml@LuaTeX}
%    \begin{macrocode}
\let\HoLogoHtml@LuaTeX\HoLogo@LuaTeX
%    \end{macrocode}
%    \end{macro}
%
% \subsubsection{\hologo{LuaLaTeX}}
%
%    \begin{macro}{\HoLogo@LuaLaTeX}
%    \begin{macrocode}
\def\HoLogo@LuaLaTeX#1{%
  \HOLOGO@mbox{%
    Lua%
    \hologo{LaTeX}%
  }%
}
%    \end{macrocode}
%    \end{macro}
%    \begin{macro}{\HoLogoHtml@LuaLaTeX}
%    \begin{macrocode}
\let\HoLogoHtml@LuaLaTeX\HoLogo@LuaLaTeX
%    \end{macrocode}
%    \end{macro}
%
% \subsubsection{\hologo{XeTeX}, \hologo{XeLaTeX}}
%
%    \begin{macro}{\HOLOGO@IfCharExists}
%    \begin{macrocode}
\ifluatex
  \ifnum\luatexversion<36 %
  \else
    \def\HOLOGO@IfCharExists#1{%
      \ifnum
        \directlua{%
           if luaotfload and luaotfload.aux then
             if luaotfload.aux.font_has_glyph(%
                    font.current(), \number#1) then % 	 
	       tex.print("1") % 	 
	     end % 	 
	   elseif font and font.fonts and font.current then %
            local f = font.fonts[font.current()]%
            if f.characters and f.characters[\number#1] then %
              tex.print("1")%
            end %
          end%
        }0=\ltx@zero
        \expandafter\ltx@secondoftwo
      \else
        \expandafter\ltx@firstoftwo
      \fi
    }%
  \fi
\fi
\ltx@IfUndefined{HOLOGO@IfCharExists}{%
  \def\HOLOGO@@IfCharExists#1{%
    \begingroup
      \tracinglostchars=\ltx@zero
      \setbox\ltx@zero=\hbox{%
        \kern7sp\char#1\relax
        \ifnum\lastkern>\ltx@zero
          \expandafter\aftergroup\csname iffalse\endcsname
        \else
          \expandafter\aftergroup\csname iftrue\endcsname
        \fi
      }%
      % \if{true|false} from \aftergroup
      \endgroup
      \expandafter\ltx@firstoftwo
    \else
      \endgroup
      \expandafter\ltx@secondoftwo
    \fi
  }%
  \ifxetex
    \ltx@IfUndefined{XeTeXfonttype}{}{%
      \ltx@IfUndefined{XeTeXcharglyph}{}{%
        \def\HOLOGO@IfCharExists#1{%
          \ifnum\XeTeXfonttype\font>\ltx@zero
            \expandafter\ltx@firstofthree
          \else
            \expandafter\ltx@gobble
          \fi
          {%
            \ifnum\XeTeXcharglyph#1>\ltx@zero
              \expandafter\ltx@firstoftwo
            \else
              \expandafter\ltx@secondoftwo
            \fi
          }%
          \HOLOGO@@IfCharExists{#1}%
        }%
      }%
    }%
  \fi
}{}
\ltx@ifundefined{HOLOGO@IfCharExists}{%
  \ifnum64=`\^^^^0040\relax % test for big chars of LuaTeX/XeTeX
    \let\HOLOGO@IfCharExists\HOLOGO@@IfCharExists
  \else
    \def\HOLOGO@IfCharExists#1{%
      \ifnum#1>255 %
        \expandafter\ltx@fourthoffour
      \fi
      \HOLOGO@@IfCharExists{#1}%
    }%
  \fi
}{}
%    \end{macrocode}
%    \end{macro}
%
%    \begin{macro}{\HoLogo@Xe}
%    Source: package \xpackage{dtklogos}
%    \begin{macrocode}
\def\HoLogo@Xe#1{%
  X%
  \kern-.1em\relax
  \HOLOGO@IfCharExists{"018E}{%
    \lower.5ex\hbox{\char"018E}%
  }{%
    \chardef\HOLOGO@choice=\ltx@zero
    \ifdim\fontdimen\ltx@one\font>0pt %
      \ltx@IfUndefined{rotatebox}{%
        \ltx@IfUndefined{pgftext}{%
          \ltx@IfUndefined{psscalebox}{%
            \ltx@IfUndefined{HOLOGO@ScaleBox@\hologoDriver}{%
            }{%
              \chardef\HOLOGO@choice=4 %
            }%
          }{%
            \chardef\HOLOGO@choice=3 %
          }%
        }{%
          \chardef\HOLOGO@choice=2 %
        }%
      }{%
        \chardef\HOLOGO@choice=1 %
      }%
      \ifcase\HOLOGO@choice
        \HOLOGO@WarningUnsupportedDriver{Xe}%
        e%
      \or % 1: \rotatebox
        \begingroup
          \setbox\ltx@zero\hbox{\rotatebox{180}{E}}%
          \ltx@LocDimenA=\dp\ltx@zero
          \advance\ltx@LocDimenA by -.5ex\relax
          \raise\ltx@LocDimenA\box\ltx@zero
        \endgroup
      \or % 2: \pgftext
        \lower.5ex\hbox{%
          \pgfpicture
            \pgftext[rotate=180]{E}%
          \endpgfpicture
        }%
      \or % 3: \psscalebox
        \begingroup
          \setbox\ltx@zero\hbox{\psscalebox{-1 -1}{E}}%
          \ltx@LocDimenA=\dp\ltx@zero
          \advance\ltx@LocDimenA by -.5ex\relax
          \raise\ltx@LocDimenA\box\ltx@zero
        \endgroup
      \or % 4: \HOLOGO@PointReflectBox
        \lower.5ex\hbox{\HOLOGO@PointReflectBox{E}}%
      \else
        \@PackageError{hologo}{Internal error (choice/it}\@ehc
      \fi
    \else
      \ltx@IfUndefined{reflectbox}{%
        \ltx@IfUndefined{pgftext}{%
          \ltx@IfUndefined{psscalebox}{%
            \ltx@IfUndefined{HOLOGO@ScaleBox@\hologoDriver}{%
            }{%
              \chardef\HOLOGO@choice=4 %
            }%
          }{%
            \chardef\HOLOGO@choice=3 %
          }%
        }{%
          \chardef\HOLOGO@choice=2 %
        }%
      }{%
        \chardef\HOLOGO@choice=1 %
      }%
      \ifcase\HOLOGO@choice
        \HOLOGO@WarningUnsupportedDriver{Xe}%
        e%
      \or % 1: reflectbox
        \lower.5ex\hbox{%
          \reflectbox{E}%
        }%
      \or % 2: \pgftext
        \lower.5ex\hbox{%
          \pgfpicture
            \pgftransformxscale{-1}%
            \pgftext{E}%
          \endpgfpicture
        }%
      \or % 3: \psscalebox
        \lower.5ex\hbox{%
          \psscalebox{-1 1}{E}%
        }%
      \or % 4: \HOLOGO@Reflectbox
        \lower.5ex\hbox{%
          \HOLOGO@ReflectBox{E}%
        }%
      \else
        \@PackageError{hologo}{Internal error (choice/up)}\@ehc
      \fi
    \fi
  }%
}
%    \end{macrocode}
%    \end{macro}
%    \begin{macro}{\HoLogoHtml@Xe}
%    \begin{macrocode}
\def\HoLogoHtml@Xe#1{%
  \HoLogoCss@Xe
  \HOLOGO@Span{Xe}{%
    X%
    \HOLOGO@Span{e}{%
      \HCode{&\ltx@hashchar x018e;}%
    }%
  }%
}
%    \end{macrocode}
%    \end{macro}
%    \begin{macro}{\HoLogoCss@Xe}
%    \begin{macrocode}
\def\HoLogoCss@Xe{%
  \Css{%
    span.HoLogo-Xe span.HoLogo-e{%
      position:relative;%
      top:.5ex;%
      left-margin:-.1em;%
    }%
  }%
  \global\let\HoLogoCss@Xe\relax
}
%    \end{macrocode}
%    \end{macro}
%
%    \begin{macro}{\HoLogo@XeTeX}
%    \begin{macrocode}
\def\HoLogo@XeTeX#1{%
  \hologo{Xe}%
  \kern-.15em\relax
  \hologo{TeX}%
}
%    \end{macrocode}
%    \end{macro}
%
%    \begin{macro}{\HoLogoHtml@XeTeX}
%    \begin{macrocode}
\def\HoLogoHtml@XeTeX#1{%
  \HoLogoCss@XeTeX
  \HOLOGO@Span{XeTeX}{%
    \hologo{Xe}%
    \hologo{TeX}%
  }%
}
%    \end{macrocode}
%    \end{macro}
%    \begin{macro}{\HoLogoCss@XeTeX}
%    \begin{macrocode}
\def\HoLogoCss@XeTeX{%
  \Css{%
    span.HoLogo-XeTeX span.HoLogo-TeX{%
      margin-left:-.15em;%
    }%
  }%
  \global\let\HoLogoCss@XeTeX\relax
}
%    \end{macrocode}
%    \end{macro}
%
%    \begin{macro}{\HoLogo@XeLaTeX}
%    \begin{macrocode}
\def\HoLogo@XeLaTeX#1{%
  \hologo{Xe}%
  \kern-.13em%
  \hologo{LaTeX}%
}
%    \end{macrocode}
%    \end{macro}
%    \begin{macro}{\HoLogoHtml@XeLaTeX}
%    \begin{macrocode}
\def\HoLogoHtml@XeLaTeX#1{%
  \HoLogoCss@XeLaTeX
  \HOLOGO@Span{XeLaTeX}{%
    \hologo{Xe}%
    \hologo{LaTeX}%
  }%
}
%    \end{macrocode}
%    \end{macro}
%    \begin{macro}{\HoLogoCss@XeLaTeX}
%    \begin{macrocode}
\def\HoLogoCss@XeLaTeX{%
  \Css{%
    span.HoLogo-XeLaTeX span.HoLogo-Xe{%
      margin-right:-.13em;%
    }%
  }%
  \global\let\HoLogoCss@XeLaTeX\relax
}
%    \end{macrocode}
%    \end{macro}
%
% \subsubsection{\hologo{pdfTeX}, \hologo{pdfLaTeX}}
%
%    \begin{macro}{\HoLogo@pdfTeX}
%    \begin{macrocode}
\def\HoLogo@pdfTeX#1{%
  \HOLOGO@mbox{%
    #1{p}{P}df\hologo{TeX}%
  }%
}
%    \end{macrocode}
%    \end{macro}
%    \begin{macro}{\HoLogoCs@pdfTeX}
%    \begin{macrocode}
\def\HoLogoCs@pdfTeX#1{#1{p}{P}dfTeX}
%    \end{macrocode}
%    \end{macro}
%    \begin{macro}{\HoLogoBkm@pdfTeX}
%    \begin{macrocode}
\def\HoLogoBkm@pdfTeX#1{%
  #1{p}{P}df\hologo{TeX}%
}
%    \end{macrocode}
%    \end{macro}
%    \begin{macro}{\HoLogoHtml@pdfTeX}
%    \begin{macrocode}
\let\HoLogoHtml@pdfTeX\HoLogo@pdfTeX
%    \end{macrocode}
%    \end{macro}
%
%    \begin{macro}{\HoLogo@pdfLaTeX}
%    \begin{macrocode}
\def\HoLogo@pdfLaTeX#1{%
  \HOLOGO@mbox{%
    #1{p}{P}df\hologo{LaTeX}%
  }%
}
%    \end{macrocode}
%    \end{macro}
%    \begin{macro}{\HoLogoCs@pdfLaTeX}
%    \begin{macrocode}
\def\HoLogoCs@pdfLaTeX#1{#1{p}{P}dfLaTeX}
%    \end{macrocode}
%    \end{macro}
%    \begin{macro}{\HoLogoBkm@pdfLaTeX}
%    \begin{macrocode}
\def\HoLogoBkm@pdfLaTeX#1{%
  #1{p}{P}df\hologo{LaTeX}%
}
%    \end{macrocode}
%    \end{macro}
%    \begin{macro}{\HoLogoHtml@pdfLaTeX}
%    \begin{macrocode}
\let\HoLogoHtml@pdfLaTeX\HoLogo@pdfLaTeX
%    \end{macrocode}
%    \end{macro}
%
% \subsubsection{\hologo{VTeX}}
%
%    \begin{macro}{\HoLogo@VTeX}
%    \begin{macrocode}
\def\HoLogo@VTeX#1{%
  \HOLOGO@mbox{%
    V\hologo{TeX}%
  }%
}
%    \end{macrocode}
%    \end{macro}
%    \begin{macro}{\HoLogoHtml@VTeX}
%    \begin{macrocode}
\let\HoLogoHtml@VTeX\HoLogo@VTeX
%    \end{macrocode}
%    \end{macro}
%
% \subsubsection{\hologo{AmS}, \dots}
%
%    Source: class \xclass{amsdtx}
%
%    \begin{macro}{\HoLogo@AmS}
%    \begin{macrocode}
\def\HoLogo@AmS#1{%
  \HoLogoFont@font{AmS}{sy}{%
    A%
    \kern-.1667em%
    \lower.5ex\hbox{M}%
    \kern-.125em%
    S%
  }%
}
%    \end{macrocode}
%    \end{macro}
%    \begin{macro}{\HoLogoBkm@AmS}
%    \begin{macrocode}
\def\HoLogoBkm@AmS#1{AmS}
%    \end{macrocode}
%    \end{macro}
%    \begin{macro}{\HoLogoHtml@AmS}
%    \begin{macrocode}
\def\HoLogoHtml@AmS#1{%
  \HoLogoCss@AmS
%  \HoLogoFont@font{AmS}{sy}{%
    \HOLOGO@Span{AmS}{%
      A%
      \HOLOGO@Span{M}{M}%
      S%
    }%
%   }%
}
%    \end{macrocode}
%    \end{macro}
%    \begin{macro}{\HoLogoCss@AmS}
%    \begin{macrocode}
\def\HoLogoCss@AmS{%
  \Css{%
    span.HoLogo-AmS span.HoLogo-M{%
      position:relative;%
      top:.5ex;%
      margin-left:-.1667em;%
      margin-right:-.125em;%
      text-decoration:none;%
    }%
  }%
  \global\let\HoLogoCss@AmS\relax
}
%    \end{macrocode}
%    \end{macro}
%
%    \begin{macro}{\HoLogo@AmSTeX}
%    \begin{macrocode}
\def\HoLogo@AmSTeX#1{%
  \hologo{AmS}%
  \HOLOGO@hyphen
  \hologo{TeX}%
}
%    \end{macrocode}
%    \end{macro}
%    \begin{macro}{\HoLogoBkm@AmSTeX}
%    \begin{macrocode}
\def\HoLogoBkm@AmSTeX#1{AmS-TeX}%
%    \end{macrocode}
%    \end{macro}
%    \begin{macro}{\HoLogoHtml@AmSTeX}
%    \begin{macrocode}
\let\HoLogoHtml@AmSTeX\HoLogo@AmSTeX
%    \end{macrocode}
%    \end{macro}
%
%    \begin{macro}{\HoLogo@AmSLaTeX}
%    \begin{macrocode}
\def\HoLogo@AmSLaTeX#1{%
  \hologo{AmS}%
  \HOLOGO@hyphen
  \hologo{LaTeX}%
}
%    \end{macrocode}
%    \end{macro}
%    \begin{macro}{\HoLogoBkm@AmSLaTeX}
%    \begin{macrocode}
\def\HoLogoBkm@AmSLaTeX#1{AmS-LaTeX}%
%    \end{macrocode}
%    \end{macro}
%    \begin{macro}{\HoLogoHtml@AmSLaTeX}
%    \begin{macrocode}
\let\HoLogoHtml@AmSLaTeX\HoLogo@AmSLaTeX
%    \end{macrocode}
%    \end{macro}
%
% \subsubsection{\hologo{BibTeX}}
%
%    \begin{macro}{\HoLogo@BibTeX@sc}
%    A definition of \hologo{BibTeX} is provided in
%    the documentation source for the manual of \hologo{BibTeX}
%    \cite{btxdoc}.
%\begin{quote}
%\begin{verbatim}
%\def\BibTeX{%
%  {%
%    \rm
%    B%
%    \kern-.05em%
%    {%
%      \sc
%      i%
%      \kern-.025em %
%      b%
%    }%
%    \kern-.08em
%    T%
%    \kern-.1667em%
%    \lower.7ex\hbox{E}%
%    \kern-.125em%
%    X%
%  }%
%}
%\end{verbatim}
%\end{quote}
%    \begin{macrocode}
\def\HoLogo@BibTeX@sc#1{%
  B%
  \kern-.05em%
  \HoLogoFont@font{BibTeX}{sc}{%
    i%
    \kern-.025em%
    b%
  }%
  \HOLOGO@discretionary
  \kern-.08em%
  \hologo{TeX}%
}
%    \end{macrocode}
%    \end{macro}
%    \begin{macro}{\HoLogoHtml@BibTeX@sc}
%    \begin{macrocode}
\def\HoLogoHtml@BibTeX@sc#1{%
  \HoLogoCss@BibTeX@sc
  \HOLOGO@Span{BibTeX-sc}{%
    B%
    \HOLOGO@Span{i}{i}%
    \HOLOGO@Span{b}{b}%
    \hologo{TeX}%
  }%
}
%    \end{macrocode}
%    \end{macro}
%    \begin{macro}{\HoLogoCss@BibTeX@sc}
%    \begin{macrocode}
\def\HoLogoCss@BibTeX@sc{%
  \Css{%
    span.HoLogo-BibTeX-sc span.HoLogo-i{%
      margin-left:-.05em;%
      margin-right:-.025em;%
      font-variant:small-caps;%
    }%
  }%
  \Css{%
    span.HoLogo-BibTeX-sc span.HoLogo-b{%
      margin-right:-.08em;%
      font-variant:small-caps;%
    }%
  }%
  \global\let\HoLogoCss@BibTeX@sc\relax
}
%    \end{macrocode}
%    \end{macro}
%
%    \begin{macro}{\HoLogo@BibTeX@sf}
%    Variant \xoption{sf} avoids trouble with unavailable
%    small caps fonts (e.g., bold versions of Computer Modern or
%    Latin Modern). The definition is taken from
%    package \xpackage{dtklogos} \cite{dtklogos}.
%\begin{quote}
%\begin{verbatim}
%\DeclareRobustCommand{\BibTeX}{%
%  B%
%  \kern-.05em%
%  \hbox{%
%    $\m@th$% %% force math size calculations
%    \csname S@\f@size\endcsname
%    \fontsize\sf@size\z@
%    \math@fontsfalse
%    \selectfont
%    I%
%    \kern-.025em%
%    B
%  }%
%  \kern-.08em%
%  \-%
%  \TeX
%}
%\end{verbatim}
%\end{quote}
%    \begin{macrocode}
\def\HoLogo@BibTeX@sf#1{%
  B%
  \kern-.05em%
  \HoLogoFont@font{BibTeX}{bibsf}{%
    I%
    \kern-.025em%
    B%
  }%
  \HOLOGO@discretionary
  \kern-.08em%
  \hologo{TeX}%
}
%    \end{macrocode}
%    \end{macro}
%    \begin{macro}{\HoLogoHtml@BibTeX@sf}
%    \begin{macrocode}
\def\HoLogoHtml@BibTeX@sf#1{%
  \HoLogoCss@BibTeX@sf
  \HOLOGO@Span{BibTeX-sf}{%
    B%
    \HoLogoFont@font{BibTeX}{bibsf}{%
      \HOLOGO@Span{i}{I}%
      B%
    }%
    \hologo{TeX}%
  }%
}
%    \end{macrocode}
%    \end{macro}
%    \begin{macro}{\HoLogoCss@BibTeX@sf}
%    \begin{macrocode}
\def\HoLogoCss@BibTeX@sf{%
  \Css{%
    span.HoLogo-BibTeX-sf span.HoLogo-i{%
      margin-left:-.05em;%
      margin-right:-.025em;%
    }%
  }%
  \Css{%
    span.HoLogo-BibTeX-sf span.HoLogo-TeX{%
      margin-left:-.08em;%
    }%
  }%
  \global\let\HoLogoCss@BibTeX@sf\relax
}
%    \end{macrocode}
%    \end{macro}
%
%    \begin{macro}{\HoLogo@BibTeX}
%    \begin{macrocode}
\def\HoLogo@BibTeX{\HoLogo@BibTeX@sf}
%    \end{macrocode}
%    \end{macro}
%    \begin{macro}{\HoLogoHtml@BibTeX}
%    \begin{macrocode}
\def\HoLogoHtml@BibTeX{\HoLogoHtml@BibTeX@sf}
%    \end{macrocode}
%    \end{macro}
%
% \subsubsection{\hologo{BibTeX8}}
%
%    \begin{macro}{\HoLogo@BibTeX8}
%    \begin{macrocode}
\expandafter\def\csname HoLogo@BibTeX8\endcsname#1{%
  \hologo{BibTeX}%
  8%
}
%    \end{macrocode}
%    \end{macro}
%
%    \begin{macro}{\HoLogoBkm@BibTeX8}
%    \begin{macrocode}
\expandafter\def\csname HoLogoBkm@BibTeX8\endcsname#1{%
  \hologo{BibTeX}%
  8%
}
%    \end{macrocode}
%    \end{macro}
%    \begin{macro}{\HoLogoHtml@BibTeX8}
%    \begin{macrocode}
\expandafter
\let\csname HoLogoHtml@BibTeX8\expandafter\endcsname
\csname HoLogo@BibTeX8\endcsname
%    \end{macrocode}
%    \end{macro}
%
% \subsubsection{\hologo{ConTeXt}}
%
%    \begin{macro}{\HoLogo@ConTeXt@simple}
%    \begin{macrocode}
\def\HoLogo@ConTeXt@simple#1{%
  \HOLOGO@mbox{Con}%
  \HOLOGO@discretionary
  \HOLOGO@mbox{\hologo{TeX}t}%
}
%    \end{macrocode}
%    \end{macro}
%    \begin{macro}{\HoLogoHtml@ConTeXt@simple}
%    \begin{macrocode}
\let\HoLogoHtml@ConTeXt@simple\HoLogo@ConTeXt@simple
%    \end{macrocode}
%    \end{macro}
%
%    \begin{macro}{\HoLogo@ConTeXt@narrow}
%    This definition of logo \hologo{ConTeXt} with variant \xoption{narrow}
%    comes from TUGboat's class \xclass{ltugboat} (version 2010/11/15 v2.8).
%    \begin{macrocode}
\def\HoLogo@ConTeXt@narrow#1{%
  \HOLOGO@mbox{C\kern-.0333emon}%
  \HOLOGO@discretionary
  \kern-.0667em%
  \HOLOGO@mbox{\hologo{TeX}\kern-.0333emt}%
}
%    \end{macrocode}
%    \end{macro}
%    \begin{macro}{\HoLogoHtml@ConTeXt@narrow}
%    \begin{macrocode}
\def\HoLogoHtml@ConTeXt@narrow#1{%
  \HoLogoCss@ConTeXt@narrow
  \HOLOGO@Span{ConTeXt-narrow}{%
    \HOLOGO@Span{C}{C}%
    on%
    \hologo{TeX}%
    t%
  }%
}
%    \end{macrocode}
%    \end{macro}
%    \begin{macro}{\HoLogoCss@ConTeXt@narrow}
%    \begin{macrocode}
\def\HoLogoCss@ConTeXt@narrow{%
  \Css{%
    span.HoLogo-ConTeXt-narrow span.HoLogo-C{%
      margin-left:-.0333em;%
    }%
  }%
  \Css{%
    span.HoLogo-ConTeXt-narrow span.HoLogo-TeX{%
      margin-left:-.0667em;%
      margin-right:-.0333em;%
    }%
  }%
  \global\let\HoLogoCss@ConTeXt@narrow\relax
}
%    \end{macrocode}
%    \end{macro}
%
%    \begin{macro}{\HoLogo@ConTeXt}
%    \begin{macrocode}
\def\HoLogo@ConTeXt{\HoLogo@ConTeXt@narrow}
%    \end{macrocode}
%    \end{macro}
%    \begin{macro}{\HoLogoHtml@ConTeXt}
%    \begin{macrocode}
\def\HoLogoHtml@ConTeXt{\HoLogoHtml@ConTeXt@narrow}
%    \end{macrocode}
%    \end{macro}
%
% \subsubsection{\hologo{emTeX}}
%
%    \begin{macro}{\HoLogo@emTeX}
%    \begin{macrocode}
\def\HoLogo@emTeX#1{%
  \HOLOGO@mbox{#1{e}{E}m}%
  \HOLOGO@discretionary
  \hologo{TeX}%
}
%    \end{macrocode}
%    \end{macro}
%    \begin{macro}{\HoLogoCs@emTeX}
%    \begin{macrocode}
\def\HoLogoCs@emTeX#1{#1{e}{E}mTeX}%
%    \end{macrocode}
%    \end{macro}
%    \begin{macro}{\HoLogoBkm@emTeX}
%    \begin{macrocode}
\def\HoLogoBkm@emTeX#1{%
  #1{e}{E}m\hologo{TeX}%
}
%    \end{macrocode}
%    \end{macro}
%    \begin{macro}{\HoLogoHtml@emTeX}
%    \begin{macrocode}
\let\HoLogoHtml@emTeX\HoLogo@emTeX
%    \end{macrocode}
%    \end{macro}
%
% \subsubsection{\hologo{ExTeX}}
%
%    \begin{macro}{\HoLogo@ExTeX}
%    The definition is taken from the FAQ of the
%    project \hologo{ExTeX}
%    \cite{ExTeX-FAQ}.
%\begin{quote}
%\begin{verbatim}
%\def\ExTeX{%
%  \textrm{% Logo always with serifs
%    \ensuremath{%
%      \textstyle
%      \varepsilon_{%
%        \kern-0.15em%
%        \mathcal{X}%
%      }%
%    }%
%    \kern-.15em%
%    \TeX
%  }%
%}
%\end{verbatim}
%\end{quote}
%    \begin{macrocode}
\def\HoLogo@ExTeX#1{%
  \HoLogoFont@font{ExTeX}{rm}{%
    \ltx@mbox{%
      \HOLOGO@MathSetup
      $%
        \textstyle
        \varepsilon_{%
          \kern-0.15em%
          \HoLogoFont@font{ExTeX}{sy}{X}%
        }%
      $%
    }%
    \HOLOGO@discretionary
    \kern-.15em%
    \hologo{TeX}%
  }%
}
%    \end{macrocode}
%    \end{macro}
%    \begin{macro}{\HoLogoHtml@ExTeX}
%    \begin{macrocode}
\def\HoLogoHtml@ExTeX#1{%
  \HoLogoCss@ExTeX
  \HoLogoFont@font{ExTeX}{rm}{%
    \HOLOGO@Span{ExTeX}{%
      \ltx@mbox{%
        \HOLOGO@MathSetup
        $\textstyle\varepsilon$%
        \HOLOGO@Span{X}{$\textstyle\chi$}%
        \hologo{TeX}%
      }%
    }%
  }%
}
%    \end{macrocode}
%    \end{macro}
%    \begin{macro}{\HoLogoBkm@ExTeX}
%    \begin{macrocode}
\def\HoLogoBkm@ExTeX#1{%
  \HOLOGO@PdfdocUnicode{#1{e}{E}x}{\textepsilon\textchi}%
  \hologo{TeX}%
}
%    \end{macrocode}
%    \end{macro}
%    \begin{macro}{\HoLogoCss@ExTeX}
%    \begin{macrocode}
\def\HoLogoCss@ExTeX{%
  \Css{%
    span.HoLogo-ExTeX{%
      font-family:serif;%
    }%
  }%
  \Css{%
    span.HoLogo-ExTeX span.HoLogo-TeX{%
      margin-left:-.15em;%
    }%
  }%
  \global\let\HoLogoCss@ExTeX\relax
}
%    \end{macrocode}
%    \end{macro}
%
% \subsubsection{\hologo{MiKTeX}}
%
%    \begin{macro}{\HoLogo@MiKTeX}
%    \begin{macrocode}
\def\HoLogo@MiKTeX#1{%
  \HOLOGO@mbox{MiK}%
  \HOLOGO@discretionary
  \hologo{TeX}%
}
%    \end{macrocode}
%    \end{macro}
%    \begin{macro}{\HoLogoHtml@MiKTeX}
%    \begin{macrocode}
\let\HoLogoHtml@MiKTeX\HoLogo@MiKTeX
%    \end{macrocode}
%    \end{macro}
%
% \subsubsection{\hologo{OzTeX} and friends}
%
%    Source: \hologo{OzTeX} FAQ \cite{OzTeX}:
%    \begin{quote}
%      |\def\OzTeX{O\kern-.03em z\kern-.15em\TeX}|\\
%      (There is no kerning in OzMF, OzMP and OzTtH.)
%    \end{quote}
%
%    \begin{macro}{\HoLogo@OzTeX}
%    \begin{macrocode}
\def\HoLogo@OzTeX#1{%
  O%
  \kern-.03em %
  z%
  \kern-.15em %
  \hologo{TeX}%
}
%    \end{macrocode}
%    \end{macro}
%    \begin{macro}{\HoLogoHtml@OzTeX}
%    \begin{macrocode}
\def\HoLogoHtml@OzTeX#1{%
  \HoLogoCss@OzTeX
  \HOLOGO@Span{OzTeX}{%
    O%
    \HOLOGO@Span{z}{z}%
    \hologo{TeX}%
  }%
}
%    \end{macrocode}
%    \end{macro}
%    \begin{macro}{\HoLogoCss@OzTeX}
%    \begin{macrocode}
\def\HoLogoCss@OzTeX{%
  \Css{%
    span.HoLogo-OzTeX span.HoLogo-z{%
      margin-left:-.03em;%
      margin-right:-.15em;%
    }%
  }%
  \global\let\HoLogoCss@OzTeX\relax
}
%    \end{macrocode}
%    \end{macro}
%
%    \begin{macro}{\HoLogo@OzMF}
%    \begin{macrocode}
\def\HoLogo@OzMF#1{%
  \HOLOGO@mbox{OzMF}%
}
%    \end{macrocode}
%    \end{macro}
%    \begin{macro}{\HoLogo@OzMP}
%    \begin{macrocode}
\def\HoLogo@OzMP#1{%
  \HOLOGO@mbox{OzMP}%
}
%    \end{macrocode}
%    \end{macro}
%    \begin{macro}{\HoLogo@OzTtH}
%    \begin{macrocode}
\def\HoLogo@OzTtH#1{%
  \HOLOGO@mbox{OzTtH}%
}
%    \end{macrocode}
%    \end{macro}
%
% \subsubsection{\hologo{PCTeX}}
%
%    \begin{macro}{\HoLogo@PCTeX}
%    \begin{macrocode}
\def\HoLogo@PCTeX#1{%
  \HOLOGO@mbox{PC}%
  \hologo{TeX}%
}
%    \end{macrocode}
%    \end{macro}
%    \begin{macro}{\HoLogoHtml@PCTeX}
%    \begin{macrocode}
\let\HoLogoHtml@PCTeX\HoLogo@PCTeX
%    \end{macrocode}
%    \end{macro}
%
% \subsubsection{\hologo{PiCTeX}}
%
%    The original definitions from \xfile{pictex.tex} \cite{PiCTeX}:
%\begin{quote}
%\begin{verbatim}
%\def\PiC{%
%  P%
%  \kern-.12em%
%  \lower.5ex\hbox{I}%
%  \kern-.075em%
%  C%
%}
%\def\PiCTeX{%
%  \PiC
%  \kern-.11em%
%  \TeX
%}
%\end{verbatim}
%\end{quote}
%
%    \begin{macro}{\HoLogo@PiC}
%    \begin{macrocode}
\def\HoLogo@PiC#1{%
  P%
  \kern-.12em%
  \lower.5ex\hbox{I}%
  \kern-.075em%
  C%
  \HOLOGO@SpaceFactor
}
%    \end{macrocode}
%    \end{macro}
%    \begin{macro}{\HoLogoHtml@PiC}
%    \begin{macrocode}
\def\HoLogoHtml@PiC#1{%
  \HoLogoCss@PiC
  \HOLOGO@Span{PiC}{%
    P%
    \HOLOGO@Span{i}{I}%
    C%
  }%
}
%    \end{macrocode}
%    \end{macro}
%    \begin{macro}{\HoLogoCss@PiC}
%    \begin{macrocode}
\def\HoLogoCss@PiC{%
  \Css{%
    span.HoLogo-PiC span.HoLogo-i{%
      position:relative;%
      top:.5ex;%
      margin-left:-.12em;%
      margin-right:-.075em;%
      text-decoration:none;%
    }%
  }%
  \global\let\HoLogoCss@PiC\relax
}
%    \end{macrocode}
%    \end{macro}
%
%    \begin{macro}{\HoLogo@PiCTeX}
%    \begin{macrocode}
\def\HoLogo@PiCTeX#1{%
  \hologo{PiC}%
  \HOLOGO@discretionary
  \kern-.11em%
  \hologo{TeX}%
}
%    \end{macrocode}
%    \end{macro}
%    \begin{macro}{\HoLogoHtml@PiCTeX}
%    \begin{macrocode}
\def\HoLogoHtml@PiCTeX#1{%
  \HoLogoCss@PiCTeX
  \HOLOGO@Span{PiCTeX}{%
    \hologo{PiC}%
    \hologo{TeX}%
  }%
}
%    \end{macrocode}
%    \end{macro}
%    \begin{macro}{\HoLogoCss@PiCTeX}
%    \begin{macrocode}
\def\HoLogoCss@PiCTeX{%
  \Css{%
    span.HoLogo-PiCTeX span.HoLogo-PiC{%
      margin-right:-.11em;%
    }%
  }%
  \global\let\HoLogoCss@PiCTeX\relax
}
%    \end{macrocode}
%    \end{macro}
%
% \subsubsection{\hologo{teTeX}}
%
%    \begin{macro}{\HoLogo@teTeX}
%    \begin{macrocode}
\def\HoLogo@teTeX#1{%
  \HOLOGO@mbox{#1{t}{T}e}%
  \HOLOGO@discretionary
  \hologo{TeX}%
}
%    \end{macrocode}
%    \end{macro}
%    \begin{macro}{\HoLogoCs@teTeX}
%    \begin{macrocode}
\def\HoLogoCs@teTeX#1{#1{t}{T}dfTeX}
%    \end{macrocode}
%    \end{macro}
%    \begin{macro}{\HoLogoBkm@teTeX}
%    \begin{macrocode}
\def\HoLogoBkm@teTeX#1{%
  #1{t}{T}e\hologo{TeX}%
}
%    \end{macrocode}
%    \end{macro}
%    \begin{macro}{\HoLogoHtml@teTeX}
%    \begin{macrocode}
\let\HoLogoHtml@teTeX\HoLogo@teTeX
%    \end{macrocode}
%    \end{macro}
%
% \subsubsection{\hologo{TeX4ht}}
%
%    \begin{macro}{\HoLogo@TeX4ht}
%    \begin{macrocode}
\expandafter\def\csname HoLogo@TeX4ht\endcsname#1{%
  \HOLOGO@mbox{\hologo{TeX}4ht}%
}
%    \end{macrocode}
%    \end{macro}
%    \begin{macro}{\HoLogoHtml@TeX4ht}
%    \begin{macrocode}
\expandafter
\let\csname HoLogoHtml@TeX4ht\expandafter\endcsname
\csname HoLogo@TeX4ht\endcsname
%    \end{macrocode}
%    \end{macro}
%
%
% \subsubsection{\hologo{SageTeX}}
%
%    \begin{macro}{\HoLogo@SageTeX}
%    \begin{macrocode}
\def\HoLogo@SageTeX#1{%
  \HOLOGO@mbox{Sage}%
  \HOLOGO@discretionary
  \HOLOGO@NegativeKerning{eT,oT,To}%
  \hologo{TeX}%
}
%    \end{macrocode}
%    \end{macro}
%    \begin{macro}{\HoLogoHtml@SageTeX}
%    \begin{macrocode}
\let\HoLogoHtml@SageTeX\HoLogo@SageTeX
%    \end{macrocode}
%    \end{macro}
%
% \subsection{\hologo{METAFONT} and friends}
%
%    \begin{macro}{\HoLogo@METAFONT}
%    \begin{macrocode}
\def\HoLogo@METAFONT#1{%
  \HoLogoFont@font{METAFONT}{logo}{%
    \HOLOGO@mbox{META}%
    \HOLOGO@discretionary
    \HOLOGO@mbox{FONT}%
  }%
}
%    \end{macrocode}
%    \end{macro}
%
%    \begin{macro}{\HoLogo@METAPOST}
%    \begin{macrocode}
\def\HoLogo@METAPOST#1{%
  \HoLogoFont@font{METAPOST}{logo}{%
    \HOLOGO@mbox{META}%
    \HOLOGO@discretionary
    \HOLOGO@mbox{POST}%
  }%
}
%    \end{macrocode}
%    \end{macro}
%
%    \begin{macro}{\HoLogo@MetaFun}
%    \begin{macrocode}
\def\HoLogo@MetaFun#1{%
  \HOLOGO@mbox{Meta}%
  \HOLOGO@discretionary
  \HOLOGO@mbox{Fun}%
}
%    \end{macrocode}
%    \end{macro}
%
%    \begin{macro}{\HoLogo@MetaPost}
%    \begin{macrocode}
\def\HoLogo@MetaPost#1{%
  \HOLOGO@mbox{Meta}%
  \HOLOGO@discretionary
  \HOLOGO@mbox{Post}%
}
%    \end{macrocode}
%    \end{macro}
%
% \subsection{Others}
%
% \subsubsection{\hologo{biber}}
%
%    \begin{macro}{\HoLogo@biber}
%    \begin{macrocode}
\def\HoLogo@biber#1{%
  \HOLOGO@mbox{#1{b}{B}i}%
  \HOLOGO@discretionary
  \HOLOGO@mbox{ber}%
}
%    \end{macrocode}
%    \end{macro}
%    \begin{macro}{\HoLogoCs@biber}
%    \begin{macrocode}
\def\HoLogoCs@biber#1{#1{b}{B}iber}
%    \end{macrocode}
%    \end{macro}
%    \begin{macro}{\HoLogoBkm@biber}
%    \begin{macrocode}
\def\HoLogoBkm@biber#1{%
  #1{b}{B}iber%
}
%    \end{macrocode}
%    \end{macro}
%    \begin{macro}{\HoLogoHtml@biber}
%    \begin{macrocode}
\let\HoLogoHtml@biber\HoLogo@biber
%    \end{macrocode}
%    \end{macro}
%
% \subsubsection{\hologo{KOMAScript}}
%
%    \begin{macro}{\HoLogo@KOMAScript}
%    The definition for \hologo{KOMAScript} is taken
%    from \hologo{KOMAScript} (\xfile{scrlogo.dtx}, reformatted) \cite{scrlogo}:
%\begin{quote}
%\begin{verbatim}
%\@ifundefined{KOMAScript}{%
%  \DeclareRobustCommand{\KOMAScript}{%
%    \textsf{%
%      K\kern.05em O\kern.05emM\kern.05em A%
%      \kern.1em-\kern.1em %
%      Script%
%    }%
%  }%
%}{}
%\end{verbatim}
%\end{quote}
%    \begin{macrocode}
\def\HoLogo@KOMAScript#1{%
  \HoLogoFont@font{KOMAScript}{sf}{%
    \HOLOGO@mbox{%
      K\kern.05em%
      O\kern.05em%
      M\kern.05em%
      A%
    }%
    \kern.1em%
    \HOLOGO@hyphen
    \kern.1em%
    \HOLOGO@mbox{Script}%
  }%
}
%    \end{macrocode}
%    \end{macro}
%    \begin{macro}{\HoLogoBkm@KOMAScript}
%    \begin{macrocode}
\def\HoLogoBkm@KOMAScript#1{%
  KOMA-Script%
}
%    \end{macrocode}
%    \end{macro}
%    \begin{macro}{\HoLogoHtml@KOMAScript}
%    \begin{macrocode}
\def\HoLogoHtml@KOMAScript#1{%
  \HoLogoCss@KOMAScript
  \HoLogoFont@font{KOMAScript}{sf}{%
    \HOLOGO@Span{KOMAScript}{%
      K%
      \HOLOGO@Span{O}{O}%
      M%
      \HOLOGO@Span{A}{A}%
      \HOLOGO@Span{hyphen}{-}%
      Script%
    }%
  }%
}
%    \end{macrocode}
%    \end{macro}
%    \begin{macro}{\HoLogoCss@KOMAScript}
%    \begin{macrocode}
\def\HoLogoCss@KOMAScript{%
  \Css{%
    span.HoLogo-KOMAScript{%
      font-family:sans-serif;%
    }%
  }%
  \Css{%
    span.HoLogo-KOMAScript span.HoLogo-O{%
      padding-left:.05em;%
      padding-right:.05em;%
    }%
  }%
  \Css{%
    span.HoLogo-KOMAScript span.HoLogo-A{%
      padding-left:.05em;%
    }%
  }%
  \Css{%
    span.HoLogo-KOMAScript span.HoLogo-hyphen{%
      padding-left:.1em;%
      padding-right:.1em;%
    }%
  }%
  \global\let\HoLogoCss@KOMAScript\relax
}
%    \end{macrocode}
%    \end{macro}
%
% \subsubsection{\hologo{LyX}}
%
%    \begin{macro}{\HoLogo@LyX}
%    The definition is taken from the documentation source files
%    of \hologo{LyX}, \xfile{Intro.lyx} \cite{LyX}:
%\begin{quote}
%\begin{verbatim}
%\def\LyX{%
%  \texorpdfstring{%
%    L\kern-.1667em\lower.25em\hbox{Y}\kern-.125emX\@%
%  }{%
%    LyX%
%  }%
%}
%\end{verbatim}
%\end{quote}
%    \begin{macrocode}
\def\HoLogo@LyX#1{%
  L%
  \kern-.1667em%
  \lower.25em\hbox{Y}%
  \kern-.125em%
  X%
  \HOLOGO@SpaceFactor
}
%    \end{macrocode}
%    \end{macro}
%    \begin{macro}{\HoLogoHtml@LyX}
%    \begin{macrocode}
\def\HoLogoHtml@LyX#1{%
  \HoLogoCss@LyX
  \HOLOGO@Span{LyX}{%
    L%
    \HOLOGO@Span{y}{Y}%
    X%
  }%
}
%    \end{macrocode}
%    \end{macro}
%    \begin{macro}{\HoLogoCss@LyX}
%    \begin{macrocode}
\def\HoLogoCss@LyX{%
  \Css{%
    span.HoLogo-LyX span.HoLogo-y{%
      position:relative;%
      top:.25em;%
      margin-left:-.1667em;%
      margin-right:-.125em;%
      text-decoration:none;%
    }%
  }%
  \global\let\HoLogoCss@LyX\relax
}
%    \end{macrocode}
%    \end{macro}
%
% \subsubsection{\hologo{NTS}}
%
%    \begin{macro}{\HoLogo@NTS}
%    Definition for \hologo{NTS} can be found in
%    package \xpackage{etex\textunderscore man} for the \hologo{eTeX} manual \cite{etexman}
%    and in package \xpackage{dtklogos} \cite{dtklogos}:
%\begin{quote}
%\begin{verbatim}
%\def\NTS{%
%  \leavevmode
%  \hbox{%
%    $%
%      \cal N%
%      \kern-0.35em%
%      \lower0.5ex\hbox{$\cal T$}%
%      \kern-0.2em%
%      S%
%    $%
%  }%
%}
%\end{verbatim}
%\end{quote}
%    \begin{macrocode}
\def\HoLogo@NTS#1{%
  \HoLogoFont@font{NTS}{sy}{%
    N\/%
    \kern-.35em%
    \lower.5ex\hbox{T\/}%
    \kern-.2em%
    S\/%
  }%
  \HOLOGO@SpaceFactor
}
%    \end{macrocode}
%    \end{macro}
%
% \subsubsection{\Hologo{TTH} (\hologo{TeX} to HTML translator)}
%
%    Source: \url{http://hutchinson.belmont.ma.us/tth/}
%    In the HTML source the second `T' is printed as subscript.
%\begin{quote}
%\begin{verbatim}
%T<sub>T</sub>H
%\end{verbatim}
%\end{quote}
%    \begin{macro}{\HoLogo@TTH}
%    \begin{macrocode}
\def\HoLogo@TTH#1{%
  \ltx@mbox{%
    T\HOLOGO@SubScript{T}H%
  }%
  \HOLOGO@SpaceFactor
}
%    \end{macrocode}
%    \end{macro}
%
%    \begin{macro}{\HoLogoHtml@TTH}
%    \begin{macrocode}
\def\HoLogoHtml@TTH#1{%
  T\HCode{<sub>}T\HCode{</sub>}H%
}
%    \end{macrocode}
%    \end{macro}
%
% \subsubsection{\Hologo{HanTheThanh}}
%
%    Partial source: Package \xpackage{dtklogos}.
%    The double accent is U+1EBF (latin small letter e with circumflex
%    and acute).
%    \begin{macro}{\HoLogo@HanTheThanh}
%    \begin{macrocode}
\def\HoLogo@HanTheThanh#1{%
  \ltx@mbox{H\`an}%
  \HOLOGO@space
  \ltx@mbox{%
    Th%
    \HOLOGO@IfCharExists{"1EBF}{%
      \char"1EBF\relax
    }{%
      \^e\hbox to 0pt{\hss\raise .5ex\hbox{\'{}}}%
    }%
  }%
  \HOLOGO@space
  \ltx@mbox{Th\`anh}%
}
%    \end{macrocode}
%    \end{macro}
%    \begin{macro}{\HoLogoBkm@HanTheThanh}
%    \begin{macrocode}
\def\HoLogoBkm@HanTheThanh#1{%
  H\`an %
  Th\HOLOGO@PdfdocUnicode{\^e}{\9036\277} %
  Th\`anh%
}
%    \end{macrocode}
%    \end{macro}
%    \begin{macro}{\HoLogoHtml@HanTheThanh}
%    \begin{macrocode}
\def\HoLogoHtml@HanTheThanh#1{%
  H\`an %
  Th\HCode{&\ltx@hashchar x1ebf;} %
  Th\`anh%
}
%    \end{macrocode}
%    \end{macro}
%
% \subsection{Driver detection}
%
%    \begin{macrocode}
\HOLOGO@IfExists\InputIfFileExists{%
  \InputIfFileExists{hologo.cfg}{}{}%
}{%
  \ltx@IfUndefined{pdf@filesize}{%
    \def\HOLOGO@InputIfExists{%
      \openin\HOLOGO@temp=hologo.cfg\relax
      \ifeof\HOLOGO@temp
        \closein\HOLOGO@temp
      \else
        \closein\HOLOGO@temp
        \begingroup
          \def\x{LaTeX2e}%
        \expandafter\endgroup
        \ifx\fmtname\x
          % \iffalse meta-comment
%
% File: hologo.dtx
% Version: 2016/05/12 v1.11
% Info: A logo collection with bookmark support
%
% Copyright (C) 2010-2012 by
%    Heiko Oberdiek <heiko.oberdiek at googlemail.com>
%
% This work may be distributed and/or modified under the
% conditions of the LaTeX Project Public License, either
% version 1.3c of this license or (at your option) any later
% version. This version of this license is in
%    http://www.latex-project.org/lppl/lppl-1-3c.txt
% and the latest version of this license is in
%    http://www.latex-project.org/lppl.txt
% and version 1.3 or later is part of all distributions of
% LaTeX version 2005/12/01 or later.
%
% This work has the LPPL maintenance status "maintained".
%
% This Current Maintainer of this work is Heiko Oberdiek.
%
% The Base Interpreter refers to any `TeX-Format',
% because some files are installed in TDS:tex/generic//.
%
% This work consists of the main source file hologo.dtx
% and the derived files
%    hologo.sty, hologo.pdf, hologo.ins, hologo.drv, hologo-example.tex,
%    hologo-test1.tex, hologo-test-spacefactor.tex,
%    hologo-test-list.tex.
%
% Distribution:
%    CTAN:macros/latex/contrib/oberdiek/hologo.dtx
%    CTAN:macros/latex/contrib/oberdiek/hologo.pdf
%
% Unpacking:
%    (a) If hologo.ins is present:
%           tex hologo.ins
%    (b) Without hologo.ins:
%           tex hologo.dtx
%    (c) If you insist on using LaTeX
%           latex \let\install=y\input{hologo.dtx}
%        (quote the arguments according to the demands of your shell)
%
% Documentation:
%    (a) If hologo.drv is present:
%           latex hologo.drv
%    (b) Without hologo.drv:
%           latex hologo.dtx; ...
%    The class ltxdoc loads the configuration file ltxdoc.cfg
%    if available. Here you can specify further options, e.g.
%    use A4 as paper format:
%       \PassOptionsToClass{a4paper}{article}
%
%    Programm calls to get the documentation (example):
%       pdflatex hologo.dtx
%       makeindex -s gind.ist hologo.idx
%       pdflatex hologo.dtx
%       makeindex -s gind.ist hologo.idx
%       pdflatex hologo.dtx
%
% Installation:
%    TDS:tex/generic/oberdiek/hologo.sty
%    TDS:doc/latex/oberdiek/hologo.pdf
%    TDS:doc/latex/oberdiek/example/hologo-example.tex
%    TDS:doc/latex/oberdiek/test/hologo-test1.tex
%    TDS:doc/latex/oberdiek/test/hologo-test-spacefactor.tex
%    TDS:doc/latex/oberdiek/test/hologo-test-list.tex
%    TDS:source/latex/oberdiek/hologo.dtx
%
%<*ignore>
\begingroup
  \catcode123=1 %
  \catcode125=2 %
  \def\x{LaTeX2e}%
\expandafter\endgroup
\ifcase 0\ifx\install y1\fi\expandafter
         \ifx\csname processbatchFile\endcsname\relax\else1\fi
         \ifx\fmtname\x\else 1\fi\relax
\else\csname fi\endcsname
%</ignore>
%<*install>
\input docstrip.tex
\Msg{************************************************************************}
\Msg{* Installation}
\Msg{* Package: hologo 2016/05/12 v1.11 A logo collection with bookmark support (HO)}
\Msg{************************************************************************}

\keepsilent
\askforoverwritefalse

\let\MetaPrefix\relax
\preamble

This is a generated file.

Project: hologo
Version: 2016/05/12 v1.11

Copyright (C) 2010-2012 by
   Heiko Oberdiek <heiko.oberdiek at googlemail.com>

This work may be distributed and/or modified under the
conditions of the LaTeX Project Public License, either
version 1.3c of this license or (at your option) any later
version. This version of this license is in
   http://www.latex-project.org/lppl/lppl-1-3c.txt
and the latest version of this license is in
   http://www.latex-project.org/lppl.txt
and version 1.3 or later is part of all distributions of
LaTeX version 2005/12/01 or later.

This work has the LPPL maintenance status "maintained".

This Current Maintainer of this work is Heiko Oberdiek.

The Base Interpreter refers to any `TeX-Format',
because some files are installed in TDS:tex/generic//.

This work consists of the main source file hologo.dtx
and the derived files
   hologo.sty, hologo.pdf, hologo.ins, hologo.drv, hologo-example.tex,
   hologo-test1.tex, hologo-test-spacefactor.tex,
   hologo-test-list.tex.

\endpreamble
\let\MetaPrefix\DoubleperCent

\generate{%
  \file{hologo.ins}{\from{hologo.dtx}{install}}%
  \file{hologo.drv}{\from{hologo.dtx}{driver}}%
  \usedir{tex/generic/oberdiek}%
  \file{hologo.sty}{\from{hologo.dtx}{package}}%
  \usedir{doc/latex/oberdiek/example}%
  \file{hologo-example.tex}{\from{hologo.dtx}{example}}%
  \usedir{doc/latex/oberdiek/test}%
  \file{hologo-test1.tex}{\from{hologo.dtx}{test1}}%
  \file{hologo-test-spacefactor.tex}{\from{hologo.dtx}{test-spacefactor}}%
  \file{hologo-test-list.tex}{\from{hologo.dtx}{test-list}}%
  \nopreamble
  \nopostamble
  \usedir{source/latex/oberdiek/catalogue}%
  \file{hologo.xml}{\from{hologo.dtx}{catalogue}}%
}

\catcode32=13\relax% active space
\let =\space%
\Msg{************************************************************************}
\Msg{*}
\Msg{* To finish the installation you have to move the following}
\Msg{* file into a directory searched by TeX:}
\Msg{*}
\Msg{*     hologo.sty}
\Msg{*}
\Msg{* To produce the documentation run the file `hologo.drv'}
\Msg{* through LaTeX.}
\Msg{*}
\Msg{* Happy TeXing!}
\Msg{*}
\Msg{************************************************************************}

\endbatchfile
%</install>
%<*ignore>
\fi
%</ignore>
%<*driver>
\NeedsTeXFormat{LaTeX2e}
\ProvidesFile{hologo.drv}%
  [2016/05/12 v1.11 A logo collection with bookmark support (HO)]%
\documentclass{ltxdoc}
\usepackage{holtxdoc}[2011/11/22]
\usepackage{hologo}[2016/05/12]
\usepackage{longtable}
\usepackage{array}
\usepackage{paralist}
%\usepackage[T1]{fontenc}
%\usepackage{lmodern}
\begin{document}
  \DocInput{hologo.dtx}%
\end{document}
%</driver>
% \fi
%
%
% \CharacterTable
%  {Upper-case    \A\B\C\D\E\F\G\H\I\J\K\L\M\N\O\P\Q\R\S\T\U\V\W\X\Y\Z
%   Lower-case    \a\b\c\d\e\f\g\h\i\j\k\l\m\n\o\p\q\r\s\t\u\v\w\x\y\z
%   Digits        \0\1\2\3\4\5\6\7\8\9
%   Exclamation   \!     Double quote  \"     Hash (number) \#
%   Dollar        \$     Percent       \%     Ampersand     \&
%   Acute accent  \'     Left paren    \(     Right paren   \)
%   Asterisk      \*     Plus          \+     Comma         \,
%   Minus         \-     Point         \.     Solidus       \/
%   Colon         \:     Semicolon     \;     Less than     \<
%   Equals        \=     Greater than  \>     Question mark \?
%   Commercial at \@     Left bracket  \[     Backslash     \\
%   Right bracket \]     Circumflex    \^     Underscore    \_
%   Grave accent  \`     Left brace    \{     Vertical bar  \|
%   Right brace   \}     Tilde         \~}
%
% \GetFileInfo{hologo.drv}
%
% \title{The \xpackage{hologo} package}
% \date{2016/05/12 v1.11}
% \author{Heiko Oberdiek\\\xemail{heiko.oberdiek at googlemail.com}}
%
% \maketitle
%
% \begin{abstract}
% This package starts a collection of logos with support for bookmarks
% strings.
% \end{abstract}
%
% \tableofcontents
%
% \section{Documentation}
%
% \subsection{Logo macros}
%
% \begin{declcs}{hologo} \M{name}
% \end{declcs}
% Macro \cs{hologo} sets the logo with name \meta{name}.
% The following table shows the supported names.
%
% \begingroup
%   \def\hologoEntry#1#2#3{^^A
%     #1&#2&\hologoLogoSetup{#1}{variant=#2}\hologo{#1}&#3\tabularnewline
%   }
%   \begin{longtable}{>{\ttfamily}l>{\ttfamily}lll}
%     \rmfamily\bfseries{name} & \rmfamily\bfseries variant
%     & \bfseries logo & \bfseries since\\
%     \hline
%     \endhead
%     \hologoList
%   \end{longtable}
% \endgroup
%
% \begin{declcs}{Hologo} \M{name}
% \end{declcs}
% Macro \cs{Hologo} starts the logo \meta{name} with an uppercase
% letter. As an exception small greek letters are not converted
% to uppercase. Examples, see \hologo{eTeX} and \hologo{ExTeX}.
%
% \subsection{Setup macros}
%
% The package does not support package options, but the following
% setup macros can be used to set options.
%
% \begin{declcs}{hologoSetup} \M{key value list}
% \end{declcs}
% Macro \cs{hologoSetup} sets global options.
%
% \begin{declcs}{hologoLogoSetup} \M{logo} \M{key value list}
% \end{declcs}
% Some options can also be used to configure a logo.
% These settings take precedence over global option settings.
%
% \subsection{Options}\label{sec:options}
%
% There are boolean and string options:
% \begin{description}
% \item[Boolean option:]
% It takes |true| or |false|
% as value. If the value is omitted, then |true| is used.
% \item[String option:]
% A value must be given as string. (But the string might be empty.)
% \end{description}
% The following options can be used both in \cs{hologoSetup}
% and \cs{hologoLogoSetup}:
% \begin{description}
% \def\entry#1{\item[\xoption{#1}:]}
% \entry{break}
%   enables or disables line breaks inside the logo. This setting is
%   refined by options \xoption{hyphenbreak}, \xoption{spacebreak}
%   or \xoption{discretionarybreak}.
%   Default is |false|.
% \entry{hyphenbreak}
%   enables or disables the line break right after the hyphen character.
% \entry{spacebreak}
%   enables or disables line breaks at space characters.
% \entry{discretionarybreak}
%   enables or disables line breaks at hyphenation points
%   (inserted by \cs{-}).
% \end{description}
% Macro \cs{hologoLogoSetup} also knows:
% \begin{description}
% \item[\xoption{variant}:]
%   This is a string option. It specifies a variant of a logo that
%   must exist. An empty string selects the package default variant.
% \end{description}
% Example:
% \begin{quote}
%   |\hologoSetup{break=false}|\\
%   |\hologoLogoSetup{plainTeX}{variant=hyphen,hyphenbreak}|\\
%   Then ``plain-\TeX'' contains one break point after the hyphen.
% \end{quote}
%
% \subsection{Driver options}
%
% Sometimes graphical operations are needed to construct some
% glyphs (e.g.\ \hologo{XeTeX}). If package \xpackage{graphics}
% or package \xpackage{pgf} are found, then the macros are taken
% from there. Otherwise the packge defines its own operations
% and therefore needs the driver information. Many drivers are
% detected automatically (\hologo{pdfTeX}/\hologo{LuaTeX}
% in PDF mode, \hologo{XeTeX}, \hologo{VTeX}). These have precedence
% over a driver option. The driver can be given as package option
% or using \cs{hologoDriverSetup}.
% The following list contains the recognized driver options:
% \begin{itemize}
% \item \xoption{pdftex}, \xoption{luatex}
% \item \xoption{dvipdfm}, \xoption{dvipdfmx}
% \item \xoption{dvips}, \xoption{dvipsone}, \xoption{xdvi}
% \item \xoption{xetex}
% \item \xoption{vtex}
% \end{itemize}
% The left driver of a line is the driver name that is used internally.
% The following names are aliases for drivers that use the
% same method. Therefore the entry in the \xext{log} file for
% the used driver prints the internally used driver name.
% \begin{description}
% \item[\xoption{driverfallback}:]
%   This option expects a driver that is used,
%   if the driver could not be detected automatically.
% \end{description}
%
% \begin{declcs}{hologoDriverSetup} \M{driver option}
% \end{declcs}
% The driver can also be configured after package loading
% using \cs{hologoDriverSetup}, also the way for \hologo{plainTeX}
% to setup the driver.
%
% \subsection{Font setup}
%
% Some logos require a special font, but should also be usable by
% \hologo{plainTeX}. Therefore the package provides some ways
% to influence the font settings. The options below
% take font settings as values. Both font commands
% such as \cs{sffamily} and macros that take one argument
% like \cs{textsf} can be used.
%
% \begin{declcs}{hologoFontSetup} \M{key value list}
% \end{declcs}
% Macro \cs{hologoFontSetup} sets the fonts for all logos.
% Supported keys:
% \begin{description}
% \def\entry#1{\item[\xoption{#1}:]}
% \entry{general}
%   This font is used for all logos. The default is empty.
%   That means no special font is used.
% \entry{bibsf}
%   This font is used for
%   {\hologoLogoSetup{BibTeX}{variant=sf}\hologo{BibTeX}}
%   with variant \xoption{sf}.
% \entry{rm}
%   This font is a serif font. It is used for \hologo{ExTeX}.
% \entry{sc}
%   This font specifies a small caps font. It is used for
%   {\hologoLogoSetup{BibTeX}{variant=sc}\hologo{BibTeX}}
%   with variant \xoption{sc}.
% \entry{sf}
%   This font specifies a sans serif font. The default
%   is \cs{sffamily}, then \cs{sf} is tried. Otherwise
%   a warning is given. It is used by \hologo{KOMAScript}.
% \entry{sy}
%   This is the font for math symbols (e.g. cmsy).
%   It is used by \hologo{AmS}, \hologo{NTS}, \hologo{ExTeX}.
% \entry{logo}
%   \hologo{METAFONT} and \hologo{METAPOST} are using that font.
%   In \hologo{LaTeX} \cs{logofamily} is used and
%   the definitions of package \xpackage{mflogo} are used
%   if the package is not loaded.
%   Otherwise the \cs{tenlogo} is used and defined
%   if it does not already exists.
% \end{description}
%
% \begin{declcs}{hologoLogoFontSetup} \M{logo} \M{key value list}
% \end{declcs}
% Fonts can also be set for a logo or logo component separately,
% see the following list.
% The keys are the same as for \cs{hologoFontSetup}.
%
% \begin{longtable}{>{\ttfamily}l>{\sffamily}ll}
%   \meta{logo} & keys & result\\
%   \hline
%   \endhead
%   BibTeX & bibsf & {\hologoLogoSetup{BibTeX}{variant=sf}\hologo{BibTeX}}\\[.5ex]
%   BibTeX & sc & {\hologoLogoSetup{BibTeX}{variant=sc}\hologo{BibTeX}}\\[.5ex]
%   ExTeX & rm & \hologo{ExTeX}\\
%   SliTeX & rm & \hologo{SliTeX}\\[.5ex]
%   AmS & sy & \hologo{AmS}\\
%   ExTeX & sy & \hologo{ExTeX}\\
%   NTS & sy & \hologo{NTS}\\[.5ex]
%   KOMAScript & sf & \hologo{KOMAScript}\\[.5ex]
%   METAFONT & logo & \hologo{METAFONT}\\
%   METAPOST & logo & \hologo{METAPOST}\\[.5ex]
%   SliTeX & sc \hologo{SliTeX}
% \end{longtable}
%
% \subsubsection{Font order}
%
% For all logos the font \xoption{general} is applied first.
% Example:
%\begin{quote}
%|\hologoFontSetup{general=\color{red}}|
%\end{quote}
% will print red logos.
% Then if the font uses a special font \xoption{sf}, for example,
% the font is applied that is setup by \cs{hologoLogoFontSetup}.
% If this font is not setup, then the common font setup
% by \cs{hologoFontSetup} is used. Otherwise a warning is given,
% that there is no font configured.
%
% \subsection{Additional user macros}
%
% Usually a variant of a logo is configured by using
% \cs{hologoLogoSetup}, because it is bad style to mix
% different variants of the same logo in the same text.
% There the following macros are a convenience for testing.
%
% \begin{declcs}{hologoVariant} \M{name} \M{variant}\\
%   \cs{HologoVariant} \M{name} \M{variant}
% \end{declcs}
% Logo \meta{name} is set using \meta{variant} that specifies
% explicitely which variant of the macro is used. If the argument
% is empty, then the default form of the logo is used
% (configurable by \cs{hologoLogoSetup}).
%
% \cs{HologoVariant} is used if the logo is set in a context
% that needs an uppercase first letter (beginning of a sentence, \dots).
%
% \begin{declcs}{hologoList}\\
%   \cs{hologoEntry} \M{logo} \M{variant} \M{since}
% \end{declcs}
% Macro \cs{hologoList} contains all logos that are provided
% by the package including variants. The list consists of calls
% of \cs{hologoEntry} with three arguments starting with the
% logo name \meta{logo} and its variant \meta{variant}. An empty
% variant means the current default. Argument \meta{since} specifies
% with version of the package \xpackage{hologo} is needed to get
% the logo. If the logo is fixed, then the date gets updated.
% Therefore the date \meta{since} is not exactly the date of
% the first introduction, but rather the date of the latest fix.
%
% Before \cs{hologoList} can be used, macro \cs{hologoEntry} needs
% a definition. The example file in section \ref{sec:example}
% shows applications of \cs{hologoList}.
%
% \subsection{Supported contexts}
%
% Macros \cs{hologo} and friends support special contexts:
% \begin{itemize}
% \item \hologo{LaTeX}'s protection mechanism.
% \item Bookmarks of package \xpackage{hyperref}.
% \item Package \xpackage{tex4ht}.
% \item The macros can be used inside \cs{csname} constructs,
%   if \cs{ifincsname} is available (\hologo{pdfTeX}, \hologo{XeTeX},
%   \hologo{LuaTeX}).
% \end{itemize}
%
% \subsection{Example}
% \label{sec:example}
%
% The following example prints the logos in different fonts.
%    \begin{macrocode}
%<*example>
%<<verbatim
\NeedsTeXFormat{LaTeX2e}
\documentclass[a4paper]{article}
\usepackage[
  hmargin=20mm,
  vmargin=20mm,
]{geometry}
\pagestyle{empty}
\usepackage{hologo}[2016/05/12]
\usepackage{longtable}
\usepackage{array}
\setlength{\extrarowheight}{2pt}
\usepackage[T1]{fontenc}
\usepackage{lmodern}
\usepackage{pdflscape}
\usepackage[
  pdfencoding=auto,
]{hyperref}
\hypersetup{
  pdfauthor={Heiko Oberdiek},
  pdftitle={Example for package `hologo'},
  pdfsubject={Logos with fonts lmr, lmss, qtm, qpl, qhv},
}
\usepackage{bookmark}

% Print the logo list on the console

\begingroup
  \typeout{}%
  \typeout{*** Begin of logo list ***}%
  \newcommand*{\hologoEntry}[3]{%
    \typeout{#1 \ifx\\#2\\\else(#2) \fi[#3]}%
  }%
  \hologoList
  \typeout{*** End of logo list ***}%
  \typeout{}%
\endgroup

\begin{document}
\begin{landscape}

  \section{Example file for package `hologo'}

  % Table for font names

  \begin{longtable}{>{\bfseries}ll}
    \textbf{font} & \textbf{Font name}\\
    \hline
    lmr & Latin Modern Roman\\
    lmss & Latin Modern Sans\\
    qtm & \TeX\ Gyre Termes\\
    qhv & \TeX\ Gyre Heros\\
    qpl & \TeX\ Gyre Pagella\\
  \end{longtable}

  % Logo list with logos in different fonts

  \begingroup
    \newcommand*{\SetVariant}[2]{%
      \ifx\\#2\\%
      \else
        \hologoLogoSetup{#1}{variant=#2}%
      \fi
    }%
    \newcommand*{\hologoEntry}[3]{%
      \SetVariant{#1}{#2}%
      \raisebox{1em}[0pt][0pt]{\hypertarget{#1@#2}{}}%
      \bookmark[%
        dest={#1@#2},%
      ]{%
        #1\ifx\\#2\\\else\space(#2)\fi: \Hologo{#1}, \hologo{#1} %
        [Unicode]%
      }%
      \hypersetup{unicode=false}%
      \bookmark[%
        dest={#1@#2},%
      ]{%
        #1\ifx\\#2\\\else\space(#2)\fi: \Hologo{#1}, \hologo{#1} %
        [PDFDocEncoding]%
      }%
      \texttt{#1}%
      &%
      \texttt{#2}%
      &%
      \Hologo{#1}%
      &%
      \SetVariant{#1}{#2}%
      \hologo{#1}%
      &%
      \SetVariant{#1}{#2}%
      \fontfamily{qtm}\selectfont
      \hologo{#1}%
      &%
      \SetVariant{#1}{#2}%
      \fontfamily{qpl}\selectfont
      \hologo{#1}%
      &%
      \SetVariant{#1}{#2}%
      \textsf{\hologo{#1}}%
      &%
      \SetVariant{#1}{#2}%
      \fontfamily{qhv}\selectfont
      \hologo{#1}%
      \tabularnewline
    }%
    \begin{longtable}{llllllll}%
      \textbf{\textit{logo}} & \textbf{\textit{variant}} &
      \texttt{\string\Hologo} &
      \textbf{lmr} & \textbf{qtm} & \textbf{qpl} &
      \textbf{lmss} & \textbf{qhv}
      \tabularnewline
      \hline
      \endhead
      \hologoList
    \end{longtable}%
  \endgroup

\end{landscape}
\end{document}
%verbatim
%</example>
%    \end{macrocode}
%
% \StopEventually{
% }
%
% \section{Implementation}
%    \begin{macrocode}
%<*package>
%    \end{macrocode}
%    Reload check, especially if the package is not used with \LaTeX.
%    \begin{macrocode}
\begingroup\catcode61\catcode48\catcode32=10\relax%
  \catcode13=5 % ^^M
  \endlinechar=13 %
  \catcode35=6 % #
  \catcode39=12 % '
  \catcode44=12 % ,
  \catcode45=12 % -
  \catcode46=12 % .
  \catcode58=12 % :
  \catcode64=11 % @
  \catcode123=1 % {
  \catcode125=2 % }
  \expandafter\let\expandafter\x\csname ver@hologo.sty\endcsname
  \ifx\x\relax % plain-TeX, first loading
  \else
    \def\empty{}%
    \ifx\x\empty % LaTeX, first loading,
      % variable is initialized, but \ProvidesPackage not yet seen
    \else
      \expandafter\ifx\csname PackageInfo\endcsname\relax
        \def\x#1#2{%
          \immediate\write-1{Package #1 Info: #2.}%
        }%
      \else
        \def\x#1#2{\PackageInfo{#1}{#2, stopped}}%
      \fi
      \x{hologo}{The package is already loaded}%
      \aftergroup\endinput
    \fi
  \fi
\endgroup%
%    \end{macrocode}
%    Package identification:
%    \begin{macrocode}
\begingroup\catcode61\catcode48\catcode32=10\relax%
  \catcode13=5 % ^^M
  \endlinechar=13 %
  \catcode35=6 % #
  \catcode39=12 % '
  \catcode40=12 % (
  \catcode41=12 % )
  \catcode44=12 % ,
  \catcode45=12 % -
  \catcode46=12 % .
  \catcode47=12 % /
  \catcode58=12 % :
  \catcode64=11 % @
  \catcode91=12 % [
  \catcode93=12 % ]
  \catcode123=1 % {
  \catcode125=2 % }
  \expandafter\ifx\csname ProvidesPackage\endcsname\relax
    \def\x#1#2#3[#4]{\endgroup
      \immediate\write-1{Package: #3 #4}%
      \xdef#1{#4}%
    }%
  \else
    \def\x#1#2[#3]{\endgroup
      #2[{#3}]%
      \ifx#1\@undefined
        \xdef#1{#3}%
      \fi
      \ifx#1\relax
        \xdef#1{#3}%
      \fi
    }%
  \fi
\expandafter\x\csname ver@hologo.sty\endcsname
\ProvidesPackage{hologo}%
  [2016/05/12 v1.11 A logo collection with bookmark support (HO)]%
%    \end{macrocode}
%
%    \begin{macrocode}
\begingroup\catcode61\catcode48\catcode32=10\relax%
  \catcode13=5 % ^^M
  \endlinechar=13 %
  \catcode123=1 % {
  \catcode125=2 % }
  \catcode64=11 % @
  \def\x{\endgroup
    \expandafter\edef\csname HOLOGO@AtEnd\endcsname{%
      \endlinechar=\the\endlinechar\relax
      \catcode13=\the\catcode13\relax
      \catcode32=\the\catcode32\relax
      \catcode35=\the\catcode35\relax
      \catcode61=\the\catcode61\relax
      \catcode64=\the\catcode64\relax
      \catcode123=\the\catcode123\relax
      \catcode125=\the\catcode125\relax
    }%
  }%
\x\catcode61\catcode48\catcode32=10\relax%
\catcode13=5 % ^^M
\endlinechar=13 %
\catcode35=6 % #
\catcode64=11 % @
\catcode123=1 % {
\catcode125=2 % }
\def\TMP@EnsureCode#1#2{%
  \edef\HOLOGO@AtEnd{%
    \HOLOGO@AtEnd
    \catcode#1=\the\catcode#1\relax
  }%
  \catcode#1=#2\relax
}
\TMP@EnsureCode{10}{12}% ^^J
\TMP@EnsureCode{33}{12}% !
\TMP@EnsureCode{34}{12}% "
\TMP@EnsureCode{36}{3}% $
\TMP@EnsureCode{38}{4}% &
\TMP@EnsureCode{39}{12}% '
\TMP@EnsureCode{40}{12}% (
\TMP@EnsureCode{41}{12}% )
\TMP@EnsureCode{42}{12}% *
\TMP@EnsureCode{43}{12}% +
\TMP@EnsureCode{44}{12}% ,
\TMP@EnsureCode{45}{12}% -
\TMP@EnsureCode{46}{12}% .
\TMP@EnsureCode{47}{12}% /
\TMP@EnsureCode{58}{12}% :
\TMP@EnsureCode{59}{12}% ;
\TMP@EnsureCode{60}{12}% <
\TMP@EnsureCode{62}{12}% >
\TMP@EnsureCode{63}{12}% ?
\TMP@EnsureCode{91}{12}% [
\TMP@EnsureCode{93}{12}% ]
\TMP@EnsureCode{94}{7}% ^ (superscript)
\TMP@EnsureCode{95}{8}% _ (subscript)
\TMP@EnsureCode{96}{12}% `
\TMP@EnsureCode{124}{12}% |
\edef\HOLOGO@AtEnd{%
  \HOLOGO@AtEnd
  \escapechar\the\escapechar\relax
  \noexpand\endinput
}
\escapechar=92 %
%    \end{macrocode}
%
% \subsection{Logo list}
%
%    \begin{macro}{\hologoList}
%    \begin{macrocode}
\def\hologoList{%
  \hologoEntry{(La)TeX}{}{2011/10/01}%
  \hologoEntry{AmSLaTeX}{}{2010/04/16}%
  \hologoEntry{AmSTeX}{}{2010/04/16}%
  \hologoEntry{biber}{}{2011/10/01}%
  \hologoEntry{BibTeX}{}{2011/10/01}%
  \hologoEntry{BibTeX}{sf}{2011/10/01}%
  \hologoEntry{BibTeX}{sc}{2011/10/01}%
  \hologoEntry{BibTeX8}{}{2011/11/22}%
  \hologoEntry{ConTeXt}{}{2011/03/25}%
  \hologoEntry{ConTeXt}{narrow}{2011/03/25}%
  \hologoEntry{ConTeXt}{simple}{2011/03/25}%
  \hologoEntry{emTeX}{}{2010/04/26}%
  \hologoEntry{eTeX}{}{2010/04/08}%
  \hologoEntry{ExTeX}{}{2011/10/01}%
  \hologoEntry{HanTheThanh}{}{2011/11/29}%
  \hologoEntry{iniTeX}{}{2011/10/01}%
  \hologoEntry{KOMAScript}{}{2011/10/01}%
  \hologoEntry{La}{}{2010/05/08}%
  \hologoEntry{LaTeX}{}{2010/04/08}%
  \hologoEntry{LaTeX2e}{}{2010/04/08}%
  \hologoEntry{LaTeX3}{}{2010/04/24}%
  \hologoEntry{LaTeXe}{}{2010/04/08}%
  \hologoEntry{LaTeXML}{}{2011/11/22}%
  \hologoEntry{LaTeXTeX}{}{2011/10/01}%
  \hologoEntry{LuaLaTeX}{}{2010/04/08}%
  \hologoEntry{LuaTeX}{}{2010/04/08}%
  \hologoEntry{LyX}{}{2011/10/01}%
  \hologoEntry{METAFONT}{}{2011/10/01}%
  \hologoEntry{MetaFun}{}{2011/10/01}%
  \hologoEntry{METAPOST}{}{2011/10/01}%
  \hologoEntry{MetaPost}{}{2011/10/01}%
  \hologoEntry{MiKTeX}{}{2011/10/01}%
  \hologoEntry{NTS}{}{2011/10/01}%
  \hologoEntry{OzMF}{}{2011/10/01}%
  \hologoEntry{OzMP}{}{2011/10/01}%
  \hologoEntry{OzTeX}{}{2011/10/01}%
  \hologoEntry{OzTtH}{}{2011/10/01}%
  \hologoEntry{PCTeX}{}{2011/10/01}%
  \hologoEntry{pdfTeX}{}{2011/10/01}%
  \hologoEntry{pdfLaTeX}{}{2011/10/01}%
  \hologoEntry{PiC}{}{2011/10/01}%
  \hologoEntry{PiCTeX}{}{2011/10/01}%
  \hologoEntry{plainTeX}{}{2010/04/08}%
  \hologoEntry{plainTeX}{space}{2010/04/16}%
  \hologoEntry{plainTeX}{hyphen}{2010/04/16}%
  \hologoEntry{plainTeX}{runtogether}{2010/04/16}%
  \hologoEntry{SageTeX}{}{2011/11/22}%
  \hologoEntry{SLiTeX}{}{2011/10/01}%
  \hologoEntry{SLiTeX}{lift}{2011/10/01}%
  \hologoEntry{SLiTeX}{narrow}{2011/10/01}%
  \hologoEntry{SLiTeX}{simple}{2011/10/01}%
  \hologoEntry{SliTeX}{}{2011/10/01}%
  \hologoEntry{SliTeX}{narrow}{2011/10/01}%
  \hologoEntry{SliTeX}{simple}{2011/10/01}%
  \hologoEntry{SliTeX}{lift}{2011/10/01}%
  \hologoEntry{teTeX}{}{2011/10/01}%
  \hologoEntry{TeX}{}{2010/04/08}%
  \hologoEntry{TeX4ht}{}{2011/11/22}%
  \hologoEntry{TTH}{}{2011/11/22}%
  \hologoEntry{virTeX}{}{2011/10/01}%
  \hologoEntry{VTeX}{}{2010/04/24}%
  \hologoEntry{Xe}{}{2010/04/08}%
  \hologoEntry{XeLaTeX}{}{2010/04/08}%
  \hologoEntry{XeTeX}{}{2010/04/08}%
}
%    \end{macrocode}
%    \end{macro}
%
% \subsection{Load resources}
%
%    \begin{macrocode}
\begingroup\expandafter\expandafter\expandafter\endgroup
\expandafter\ifx\csname RequirePackage\endcsname\relax
  \def\TMP@RequirePackage#1[#2]{%
    \begingroup\expandafter\expandafter\expandafter\endgroup
    \expandafter\ifx\csname ver@#1.sty\endcsname\relax
      \input #1.sty\relax
    \fi
  }%
  \TMP@RequirePackage{ltxcmds}[2011/02/04]%
  \TMP@RequirePackage{infwarerr}[2010/04/08]%
  \TMP@RequirePackage{kvsetkeys}[2010/03/01]%
  \TMP@RequirePackage{kvdefinekeys}[2010/03/01]%
  \TMP@RequirePackage{pdftexcmds}[2010/04/01]%
  \TMP@RequirePackage{ifpdf}[2010/01/28]%
  \TMP@RequirePackage{ifluatex}[2010/03/01]%
  \ltx@IfUndefined{newif}{%
    \expandafter\let\csname newif\endcsname\ltx@newif
  }{}%
  \TMP@RequirePackage{ifxetex}[2009/01/23]%
  \TMP@RequirePackage{ifvtex}[2010/03/01]%
\else
  \RequirePackage{ltxcmds}[2011/02/04]%
  \RequirePackage{infwarerr}[2010/04/08]%
  \RequirePackage{kvsetkeys}[2010/03/01]%
  \RequirePackage{kvdefinekeys}[2010/03/01]%
  \RequirePackage{pdftexcmds}[2010/04/01]%
  \RequirePackage{ifpdf}[2010/01/28]%
  \RequirePackage{ifluatex}[2010/03/01]%
  \RequirePackage{ifxetex}[2009/01/23]%
  \RequirePackage{ifvtex}[2010/03/01]%
\fi
%    \end{macrocode}
%
%    \begin{macro}{\HOLOGO@IfDefined}
%    \begin{macrocode}
\def\HOLOGO@IfExists#1{%
  \ifx\@undefined#1%
    \expandafter\ltx@secondoftwo
  \else
    \ifx\relax#1%
      \expandafter\ltx@secondoftwo
    \else
      \expandafter\expandafter\expandafter\ltx@firstoftwo
    \fi
  \fi
}
%    \end{macrocode}
%    \end{macro}
%
% \subsection{Setup macros}
%
%    \begin{macro}{\hologoSetup}
%    \begin{macrocode}
\def\hologoSetup{%
  \let\HOLOGO@name\relax
  \HOLOGO@Setup
}
%    \end{macrocode}
%    \end{macro}
%
%    \begin{macro}{\hologoLogoSetup}
%    \begin{macrocode}
\def\hologoLogoSetup#1{%
  \edef\HOLOGO@name{#1}%
  \ltx@IfUndefined{HoLogo@\HOLOGO@name}{%
    \@PackageError{hologo}{%
      Unknown logo `\HOLOGO@name'%
    }\@ehc
    \ltx@gobble
  }{%
    \HOLOGO@Setup
  }%
}
%    \end{macrocode}
%    \end{macro}
%
%    \begin{macro}{\HOLOGO@Setup}
%    \begin{macrocode}
\def\HOLOGO@Setup{%
  \kvsetkeys{HoLogo}%
}
%    \end{macrocode}
%    \end{macro}
%
% \subsection{Options}
%
%    \begin{macro}{\HOLOGO@DeclareBoolOption}
%    \begin{macrocode}
\def\HOLOGO@DeclareBoolOption#1{%
  \expandafter\chardef\csname HOLOGOOPT@#1\endcsname\ltx@zero
  \kv@define@key{HoLogo}{#1}[true]{%
    \def\HOLOGO@temp{##1}%
    \ifx\HOLOGO@temp\HOLOGO@true
      \ifx\HOLOGO@name\relax
        \expandafter\chardef\csname HOLOGOOPT@#1\endcsname=\ltx@one
      \else
        \expandafter\chardef\csname
        HoLogoOpt@#1@\HOLOGO@name\endcsname\ltx@one
      \fi
      \HOLOGO@SetBreakAll{#1}%
    \else
      \ifx\HOLOGO@temp\HOLOGO@false
        \ifx\HOLOGO@name\relax
          \expandafter\chardef\csname HOLOGOOPT@#1\endcsname=\ltx@zero
        \else
          \expandafter\chardef\csname
          HoLogoOpt@#1@\HOLOGO@name\endcsname=\ltx@zero
        \fi
        \HOLOGO@SetBreakAll{#1}%
      \else
        \@PackageError{hologo}{%
          Unknown value `##1' for boolean option `#1'.\MessageBreak
          Known values are `true' and `false'%
        }\@ehc
      \fi
    \fi
  }%
}
%    \end{macrocode}
%    \end{macro}
%
%    \begin{macro}{\HOLOGO@SetBreakAll}
%    \begin{macrocode}
\def\HOLOGO@SetBreakAll#1{%
  \def\HOLOGO@temp{#1}%
  \ifx\HOLOGO@temp\HOLOGO@break
    \ifx\HOLOGO@name\relax
      \chardef\HOLOGOOPT@hyphenbreak=\HOLOGOOPT@break
      \chardef\HOLOGOOPT@spacebreak=\HOLOGOOPT@break
      \chardef\HOLOGOOPT@discretionarybreak=\HOLOGOOPT@break
    \else
      \expandafter\chardef
         \csname HoLogoOpt@hyphenbreak@\HOLOGO@name\endcsname=%
         \csname HoLogoOpt@break@\HOLOGO@name\endcsname
      \expandafter\chardef
         \csname HoLogoOpt@spacebreak@\HOLOGO@name\endcsname=%
         \csname HoLogoOpt@break@\HOLOGO@name\endcsname
      \expandafter\chardef
         \csname HoLogoOpt@discretionarybreak@\HOLOGO@name
             \endcsname=%
         \csname HoLogoOpt@break@\HOLOGO@name\endcsname
    \fi
  \fi
}
%    \end{macrocode}
%    \end{macro}
%
%    \begin{macro}{\HOLOGO@true}
%    \begin{macrocode}
\def\HOLOGO@true{true}
%    \end{macrocode}
%    \end{macro}
%    \begin{macro}{\HOLOGO@false}
%    \begin{macrocode}
\def\HOLOGO@false{false}
%    \end{macrocode}
%    \end{macro}
%    \begin{macro}{\HOLOGO@break}
%    \begin{macrocode}
\def\HOLOGO@break{break}
%    \end{macrocode}
%    \end{macro}
%
%    \begin{macrocode}
\HOLOGO@DeclareBoolOption{break}
\HOLOGO@DeclareBoolOption{hyphenbreak}
\HOLOGO@DeclareBoolOption{spacebreak}
\HOLOGO@DeclareBoolOption{discretionarybreak}
%    \end{macrocode}
%
%    \begin{macrocode}
\kv@define@key{HoLogo}{variant}{%
  \ifx\HOLOGO@name\relax
    \@PackageError{hologo}{%
      Option `variant' is not available in \string\hologoSetup,%
      \MessageBreak
      Use \string\hologoLogoSetup\space instead%
    }\@ehc
  \else
    \edef\HOLOGO@temp{#1}%
    \ifx\HOLOGO@temp\ltx@empty
      \expandafter
      \let\csname HoLogoOpt@variant@\HOLOGO@name\endcsname\@undefined
    \else
      \ltx@IfUndefined{HoLogo@\HOLOGO@name @\HOLOGO@temp}{%
        \@PackageError{hologo}{%
          Unknown variant `\HOLOGO@temp' of logo `\HOLOGO@name'%
        }\@ehc
      }{%
        \expandafter
        \let\csname HoLogoOpt@variant@\HOLOGO@name\endcsname
            \HOLOGO@temp
      }%
    \fi
  \fi
}
%    \end{macrocode}
%
%    \begin{macro}{\HOLOGO@Variant}
%    \begin{macrocode}
\def\HOLOGO@Variant#1{%
  #1%
  \ltx@ifundefined{HoLogoOpt@variant@#1}{%
  }{%
    @\csname HoLogoOpt@variant@#1\endcsname
  }%
}
%    \end{macrocode}
%    \end{macro}
%
% \subsection{Break/no-break support}
%
%    \begin{macro}{\HOLOGO@space}
%    \begin{macrocode}
\def\HOLOGO@space{%
  \ltx@ifundefined{HoLogoOpt@spacebreak@\HOLOGO@name}{%
    \ltx@ifundefined{HoLogoOpt@break@\HOLOGO@name}{%
      \chardef\HOLOGO@temp=\HOLOGOOPT@spacebreak
    }{%
      \chardef\HOLOGO@temp=%
        \csname HoLogoOpt@break@\HOLOGO@name\endcsname
    }%
  }{%
    \chardef\HOLOGO@temp=%
      \csname HoLogoOpt@spacebreak@\HOLOGO@name\endcsname
  }%
  \ifcase\HOLOGO@temp
    \penalty10000 %
  \fi
  \ltx@space
}
%    \end{macrocode}
%    \end{macro}
%
%    \begin{macro}{\HOLOGO@hyphen}
%    \begin{macrocode}
\def\HOLOGO@hyphen{%
  \ltx@ifundefined{HoLogoOpt@hyphenbreak@\HOLOGO@name}{%
    \ltx@ifundefined{HoLogoOpt@break@\HOLOGO@name}{%
      \chardef\HOLOGO@temp=\HOLOGOOPT@hyphenbreak
    }{%
      \chardef\HOLOGO@temp=%
        \csname HoLogoOpt@break@\HOLOGO@name\endcsname
    }%
  }{%
    \chardef\HOLOGO@temp=%
      \csname HoLogoOpt@hyphenbreak@\HOLOGO@name\endcsname
  }%
  \ifcase\HOLOGO@temp
    \ltx@mbox{-}%
  \else
    -%
  \fi
}
%    \end{macrocode}
%    \end{macro}
%
%    \begin{macro}{\HOLOGO@discretionary}
%    \begin{macrocode}
\def\HOLOGO@discretionary{%
  \ltx@ifundefined{HoLogoOpt@discretionarybreak@\HOLOGO@name}{%
    \ltx@ifundefined{HoLogoOpt@break@\HOLOGO@name}{%
      \chardef\HOLOGO@temp=\HOLOGOOPT@discretionarybreak
    }{%
      \chardef\HOLOGO@temp=%
        \csname HoLogoOpt@break@\HOLOGO@name\endcsname
    }%
  }{%
    \chardef\HOLOGO@temp=%
      \csname HoLogoOpt@discretionarybreak@\HOLOGO@name\endcsname
  }%
  \ifcase\HOLOGO@temp
  \else
    \-%
  \fi
}
%    \end{macrocode}
%    \end{macro}
%
%    \begin{macro}{\HOLOGO@mbox}
%    \begin{macrocode}
\def\HOLOGO@mbox#1{%
  \ltx@ifundefined{HoLogoOpt@break@\HOLOGO@name}{%
    \chardef\HOLOGO@temp=\HOLOGOOPT@hyphenbreak
  }{%
    \chardef\HOLOGO@temp=%
      \csname HoLogoOpt@break@\HOLOGO@name\endcsname
  }%
  \ifcase\HOLOGO@temp
    \ltx@mbox{#1}%
  \else
    #1%
  \fi
}
%    \end{macrocode}
%    \end{macro}
%
% \subsection{Font support}
%
%    \begin{macro}{\HoLogoFont@font}
%    \begin{tabular}{@{}ll@{}}
%    |#1|:& logo name\\
%    |#2|:& font short name\\
%    |#3|:& text
%    \end{tabular}
%    \begin{macrocode}
\def\HoLogoFont@font#1#2#3{%
  \begingroup
    \ltx@IfUndefined{HoLogoFont@logo@#1.#2}{%
      \ltx@IfUndefined{HoLogoFont@font@#2}{%
        \@PackageWarning{hologo}{%
          Missing font `#2' for logo `#1'%
        }%
        #3%
      }{%
        \csname HoLogoFont@font@#2\endcsname{#3}%
      }%
    }{%
      \csname HoLogoFont@logo@#1.#2\endcsname{#3}%
    }%
  \endgroup
}
%    \end{macrocode}
%    \end{macro}
%
%    \begin{macro}{\HoLogoFont@Def}
%    \begin{macrocode}
\def\HoLogoFont@Def#1{%
  \expandafter\def\csname HoLogoFont@font@#1\endcsname
}
%    \end{macrocode}
%    \end{macro}
%    \begin{macro}{\HoLogoFont@LogoDef}
%    \begin{macrocode}
\def\HoLogoFont@LogoDef#1#2{%
  \expandafter\def\csname HoLogoFont@logo@#1.#2\endcsname
}
%    \end{macrocode}
%    \end{macro}
%
% \subsubsection{Font defaults}
%
%    \begin{macro}{\HoLogoFont@font@general}
%    \begin{macrocode}
\HoLogoFont@Def{general}{}%
%    \end{macrocode}
%    \end{macro}
%
%    \begin{macro}{\HoLogoFont@font@rm}
%    \begin{macrocode}
\ltx@IfUndefined{rmfamily}{%
  \ltx@IfUndefined{rm}{%
  }{%
    \HoLogoFont@Def{rm}{\rm}%
  }%
}{%
  \HoLogoFont@Def{rm}{\rmfamily}%
}
%    \end{macrocode}
%    \end{macro}
%
%    \begin{macro}{\HoLogoFont@font@sf}
%    \begin{macrocode}
\ltx@IfUndefined{sffamily}{%
  \ltx@IfUndefined{sf}{%
  }{%
    \HoLogoFont@Def{sf}{\sf}%
  }%
}{%
  \HoLogoFont@Def{sf}{\sffamily}%
}
%    \end{macrocode}
%    \end{macro}
%
%    \begin{macro}{\HoLogoFont@font@bibsf}
%    In case of \hologo{plainTeX} the original small caps
%    variant is used as default. In \hologo{LaTeX}
%    the definition of package \xpackage{dtklogos} \cite{dtklogos}
%    is used.
%\begin{quote}
%\begin{verbatim}
%\DeclareRobustCommand{\BibTeX}{%
%  B%
%  \kern-.05em%
%  \hbox{%
%    $\m@th$% %% force math size calculations
%    \csname S@\f@size\endcsname
%    \fontsize\sf@size\z@
%    \math@fontsfalse
%    \selectfont
%    I%
%    \kern-.025em%
%    B
%  }%
%  \kern-.08em%
%  \-%
%  \TeX
%}
%\end{verbatim}
%\end{quote}
%    \begin{macrocode}
\ltx@IfUndefined{selectfont}{%
  \ltx@IfUndefined{tensc}{%
    \font\tensc=cmcsc10\relax
  }{}%
  \HoLogoFont@Def{bibsf}{\tensc}%
}{%
  \HoLogoFont@Def{bibsf}{%
    $\mathsurround=0pt$%
    \csname S@\f@size\endcsname
    \fontsize\sf@size{0pt}%
    \math@fontsfalse
    \selectfont
  }%
}
%    \end{macrocode}
%    \end{macro}
%
%    \begin{macro}{\HoLogoFont@font@sc}
%    \begin{macrocode}
\ltx@IfUndefined{scshape}{%
  \ltx@IfUndefined{tensc}{%
    \font\tensc=cmcsc10\relax
  }{}%
  \HoLogoFont@Def{sc}{\tensc}%
}{%
  \HoLogoFont@Def{sc}{\scshape}%
}
%    \end{macrocode}
%    \end{macro}
%
%    \begin{macro}{\HoLogoFont@font@sy}
%    \begin{macrocode}
\ltx@IfUndefined{usefont}{%
  \ltx@IfUndefined{tensy}{%
  }{%
    \HoLogoFont@Def{sy}{\tensy}%
  }%
}{%
  \HoLogoFont@Def{sy}{%
    \usefont{OMS}{cmsy}{m}{n}%
  }%
}
%    \end{macrocode}
%    \end{macro}
%
%    \begin{macro}{\HoLogoFont@font@logo}
%    \begin{macrocode}
\begingroup
  \def\x{LaTeX2e}%
\expandafter\endgroup
\ifx\fmtname\x
  \ltx@IfUndefined{logofamily}{%
    \DeclareRobustCommand\logofamily{%
      \not@math@alphabet\logofamily\relax
      \fontencoding{U}%
      \fontfamily{logo}%
      \selectfont
    }%
  }{}%
  \ltx@IfUndefined{logofamily}{%
  }{%
    \HoLogoFont@Def{logo}{\logofamily}%
  }%
\else
  \ltx@IfUndefined{tenlogo}{%
    \font\tenlogo=logo10\relax
  }{}%
  \HoLogoFont@Def{logo}{\tenlogo}%
\fi
%    \end{macrocode}
%    \end{macro}
%
% \subsubsection{Font setup}
%
%    \begin{macro}{\hologoFontSetup}
%    \begin{macrocode}
\def\hologoFontSetup{%
  \let\HOLOGO@name\relax
  \HOLOGO@FontSetup
}
%    \end{macrocode}
%    \end{macro}
%
%    \begin{macro}{\hologoLogoFontSetup}
%    \begin{macrocode}
\def\hologoLogoFontSetup#1{%
  \edef\HOLOGO@name{#1}%
  \ltx@IfUndefined{HoLogo@\HOLOGO@name}{%
    \@PackageError{hologo}{%
      Unknown logo `\HOLOGO@name'%
    }\@ehc
    \ltx@gobble
  }{%
    \HOLOGO@FontSetup
  }%
}
%    \end{macrocode}
%    \end{macro}
%
%    \begin{macro}{\HOLOGO@FontSetup}
%    \begin{macrocode}
\def\HOLOGO@FontSetup{%
  \kvsetkeys{HoLogoFont}%
}
%    \end{macrocode}
%    \end{macro}
%
%    \begin{macrocode}
\def\HOLOGO@temp#1{%
  \kv@define@key{HoLogoFont}{#1}{%
    \ifx\HOLOGO@name\relax
      \HoLogoFont@Def{#1}{##1}%
    \else
      \HoLogoFont@LogoDef\HOLOGO@name{#1}{##1}%
    \fi
  }%
}
\HOLOGO@temp{general}
\HOLOGO@temp{sf}
%    \end{macrocode}
%
% \subsection{Generic logo commands}
%
%    \begin{macrocode}
\HOLOGO@IfExists\hologo{%
  \@PackageError{hologo}{%
    \string\hologo\ltx@space is already defined.\MessageBreak
    Package loading is aborted%
  }\@ehc
  \HOLOGO@AtEnd
}%
\HOLOGO@IfExists\hologoRobust{%
  \@PackageError{hologo}{%
    \string\hologoRobust\ltx@space is already defined.\MessageBreak
    Package loading is aborted%
  }\@ehc
  \HOLOGO@AtEnd
}%
%    \end{macrocode}
%
% \subsubsection{\cs{hologo} and friends}
%
%    \begin{macrocode}
\ifluatex
  \expandafter\ltx@firstofone
\else
  \expandafter\ltx@gobble
\fi
{%
  \ltx@IfUndefined{ifincsname}{%
    \ifnum\luatexversion<36 %
      \expandafter\ltx@gobble
    \else
      \expandafter\ltx@firstofone
    \fi
    {%
      \begingroup
        \ifcase0%
            \directlua{%
              if tex.enableprimitives then %
                tex.enableprimitives('HOLOGO@', {'ifincsname'})%
              else %
                tex.print('1')%
              end%
            }%
            \ifx\HOLOGO@ifincsname\@undefined 1\fi%
            \relax
          \expandafter\ltx@firstofone
        \else
          \endgroup
          \expandafter\ltx@gobble
        \fi
        {%
          \global\let\ifincsname\HOLOGO@ifincsname
        }%
      \HOLOGO@temp
    }%
  }{}%
}
%    \end{macrocode}
%    \begin{macrocode}
\ltx@IfUndefined{ifincsname}{%
  \catcode`$=14 %
}{%
  \catcode`$=9 %
}
%    \end{macrocode}
%
%    \begin{macro}{\hologo}
%    \begin{macrocode}
\def\hologo#1{%
$ \ifincsname
$   \ltx@ifundefined{HoLogoCs@\HOLOGO@Variant{#1}}{%
$     #1%
$   }{%
$     \csname HoLogoCs@\HOLOGO@Variant{#1}\endcsname\ltx@firstoftwo
$   }%
$ \else
    \HOLOGO@IfExists\texorpdfstring\texorpdfstring\ltx@firstoftwo
    {%
      \hologoRobust{#1}%
    }{%
      \ltx@ifundefined{HoLogoBkm@\HOLOGO@Variant{#1}}{%
        \ltx@ifundefined{HoLogo@#1}{?#1?}{#1}%
      }{%
        \csname HoLogoBkm@\HOLOGO@Variant{#1}\endcsname
        \ltx@firstoftwo
      }%
    }%
$ \fi
}
%    \end{macrocode}
%    \end{macro}
%    \begin{macro}{\Hologo}
%    \begin{macrocode}
\def\Hologo#1{%
$ \ifincsname
$   \ltx@ifundefined{HoLogoCs@\HOLOGO@Variant{#1}}{%
$     #1%
$   }{%
$     \csname HoLogoCs@\HOLOGO@Variant{#1}\endcsname\ltx@secondoftwo
$   }%
$ \else
    \HOLOGO@IfExists\texorpdfstring\texorpdfstring\ltx@firstoftwo
    {%
      \HologoRobust{#1}%
    }{%
      \ltx@ifundefined{HoLogoBkm@\HOLOGO@Variant{#1}}{%
        \ltx@ifundefined{HoLogo@#1}{?#1?}{#1}%
      }{%
        \csname HoLogoBkm@\HOLOGO@Variant{#1}\endcsname
        \ltx@secondoftwo
      }%
    }%
$ \fi
}
%    \end{macrocode}
%    \end{macro}
%
%    \begin{macro}{\hologoVariant}
%    \begin{macrocode}
\def\hologoVariant#1#2{%
  \ifx\relax#2\relax
    \hologo{#1}%
  \else
$   \ifincsname
$     \ltx@ifundefined{HoLogoCs@#1@#2}{%
$       #1%
$     }{%
$       \csname HoLogoCs@#1@#2\endcsname\ltx@firstoftwo
$     }%
$   \else
      \HOLOGO@IfExists\texorpdfstring\texorpdfstring\ltx@firstoftwo
      {%
        \hologoVariantRobust{#1}{#2}%
      }{%
        \ltx@ifundefined{HoLogoBkm@#1@#2}{%
          \ltx@ifundefined{HoLogo@#1}{?#1?}{#1}%
        }{%
          \csname HoLogoBkm@#1@#2\endcsname
          \ltx@firstoftwo
        }%
      }%
$   \fi
  \fi
}
%    \end{macrocode}
%    \end{macro}
%    \begin{macro}{\HologoVariant}
%    \begin{macrocode}
\def\HologoVariant#1#2{%
  \ifx\relax#2\relax
    \Hologo{#1}%
  \else
$   \ifincsname
$     \ltx@ifundefined{HoLogoCs@#1@#2}{%
$       #1%
$     }{%
$       \csname HoLogoCs@#1@#2\endcsname\ltx@secondoftwo
$     }%
$   \else
      \HOLOGO@IfExists\texorpdfstring\texorpdfstring\ltx@firstoftwo
      {%
        \HologoVariantRobust{#1}{#2}%
      }{%
        \ltx@ifundefined{HoLogoBkm@#1@#2}{%
          \ltx@ifundefined{HoLogo@#1}{?#1?}{#1}%
        }{%
          \csname HoLogoBkm@#1@#2\endcsname
          \ltx@secondoftwo
        }%
      }%
$   \fi
  \fi
}
%    \end{macrocode}
%    \end{macro}
%
%    \begin{macrocode}
\catcode`\$=3 %
%    \end{macrocode}
%
% \subsubsection{\cs{hologoRobust} and friends}
%
%    \begin{macro}{\hologoRobust}
%    \begin{macrocode}
\ltx@IfUndefined{protected}{%
  \ltx@IfUndefined{DeclareRobustCommand}{%
    \def\hologoRobust#1%
  }{%
    \DeclareRobustCommand*\hologoRobust[1]%
  }%
}{%
  \protected\def\hologoRobust#1%
}%
{%
  \edef\HOLOGO@name{#1}%
  \ltx@IfUndefined{HoLogo@\HOLOGO@Variant\HOLOGO@name}{%
    \@PackageError{hologo}{%
      Unknown logo `\HOLOGO@name'%
    }\@ehc
    ?\HOLOGO@name?%
  }{%
    \ltx@IfUndefined{ver@tex4ht.sty}{%
      \HoLogoFont@font\HOLOGO@name{general}{%
        \csname HoLogo@\HOLOGO@Variant\HOLOGO@name\endcsname
        \ltx@firstoftwo
      }%
    }{%
      \ltx@IfUndefined{HoLogoHtml@\HOLOGO@Variant\HOLOGO@name}{%
        \HOLOGO@name
      }{%
        \csname HoLogoHtml@\HOLOGO@Variant\HOLOGO@name\endcsname
        \ltx@firstoftwo
      }%
    }%
  }%
}
%    \end{macrocode}
%    \end{macro}
%    \begin{macro}{\HologoRobust}
%    \begin{macrocode}
\ltx@IfUndefined{protected}{%
  \ltx@IfUndefined{DeclareRobustCommand}{%
    \def\HologoRobust#1%
  }{%
    \DeclareRobustCommand*\HologoRobust[1]%
  }%
}{%
  \protected\def\HologoRobust#1%
}%
{%
  \edef\HOLOGO@name{#1}%
  \ltx@IfUndefined{HoLogo@\HOLOGO@Variant\HOLOGO@name}{%
    \@PackageError{hologo}{%
      Unknown logo `\HOLOGO@name'%
    }\@ehc
    ?\HOLOGO@name?%
  }{%
    \ltx@IfUndefined{ver@tex4ht.sty}{%
      \HoLogoFont@font\HOLOGO@name{general}{%
        \csname HoLogo@\HOLOGO@Variant\HOLOGO@name\endcsname
        \ltx@secondoftwo
      }%
    }{%
      \ltx@IfUndefined{HoLogoHtml@\HOLOGO@Variant\HOLOGO@name}{%
        \expandafter\HOLOGO@Uppercase\HOLOGO@name
      }{%
        \csname HoLogoHtml@\HOLOGO@Variant\HOLOGO@name\endcsname
        \ltx@secondoftwo
      }%
    }%
  }%
}
%    \end{macrocode}
%    \end{macro}
%    \begin{macro}{\hologoVariantRobust}
%    \begin{macrocode}
\ltx@IfUndefined{protected}{%
  \ltx@IfUndefined{DeclareRobustCommand}{%
    \def\hologoVariantRobust#1#2%
  }{%
    \DeclareRobustCommand*\hologoVariantRobust[2]%
  }%
}{%
  \protected\def\hologoVariantRobust#1#2%
}%
{%
  \begingroup
    \hologoLogoSetup{#1}{variant={#2}}%
    \hologoRobust{#1}%
  \endgroup
}
%    \end{macrocode}
%    \end{macro}
%    \begin{macro}{\HologoVariantRobust}
%    \begin{macrocode}
\ltx@IfUndefined{protected}{%
  \ltx@IfUndefined{DeclareRobustCommand}{%
    \def\HologoVariantRobust#1#2%
  }{%
    \DeclareRobustCommand*\HologoVariantRobust[2]%
  }%
}{%
  \protected\def\HologoVariantRobust#1#2%
}%
{%
  \begingroup
    \hologoLogoSetup{#1}{variant={#2}}%
    \HologoRobust{#1}%
  \endgroup
}
%    \end{macrocode}
%    \end{macro}
%
%    \begin{macro}{\hologorobust}
%    Macro \cs{hologorobust} is only defined for compatibility.
%    Its use is deprecated.
%    \begin{macrocode}
\def\hologorobust{\hologoRobust}
%    \end{macrocode}
%    \end{macro}
%
% \subsection{Helpers}
%
%    \begin{macro}{\HOLOGO@Uppercase}
%    Macro \cs{HOLOGO@Uppercase} is restricted to \cs{uppercase},
%    because \hologo{plainTeX} or \hologo{iniTeX} do not provide
%    \cs{MakeUppercase}.
%    \begin{macrocode}
\def\HOLOGO@Uppercase#1{\uppercase{#1}}
%    \end{macrocode}
%    \end{macro}
%
%    \begin{macro}{\HOLOGO@PdfdocUnicode}
%    \begin{macrocode}
\def\HOLOGO@PdfdocUnicode{%
  \ifx\ifHy@unicode\iftrue
    \expandafter\ltx@secondoftwo
  \else
    \expandafter\ltx@firstoftwo
  \fi
}
%    \end{macrocode}
%    \end{macro}
%
%    \begin{macro}{\HOLOGO@Math}
%    \begin{macrocode}
\def\HOLOGO@MathSetup{%
  \mathsurround0pt\relax
  \HOLOGO@IfExists\f@series{%
    \if b\expandafter\ltx@car\f@series x\@nil
      \csname boldmath\endcsname
   \fi
  }{}%
}
%    \end{macrocode}
%    \end{macro}
%
%    \begin{macro}{\HOLOGO@TempDimen}
%    \begin{macrocode}
\dimendef\HOLOGO@TempDimen=\ltx@zero
%    \end{macrocode}
%    \end{macro}
%    \begin{macro}{\HOLOGO@NegativeKerning}
%    \begin{macrocode}
\def\HOLOGO@NegativeKerning#1{%
  \begingroup
    \HOLOGO@TempDimen=0pt\relax
    \comma@parse@normalized{#1}{%
      \ifdim\HOLOGO@TempDimen=0pt %
        \expandafter\HOLOGO@@NegativeKerning\comma@entry
      \fi
      \ltx@gobble
    }%
    \ifdim\HOLOGO@TempDimen<0pt %
      \kern\HOLOGO@TempDimen
    \fi
  \endgroup
}
%    \end{macrocode}
%    \end{macro}
%    \begin{macro}{\HOLOGO@@NegativeKerning}
%    \begin{macrocode}
\def\HOLOGO@@NegativeKerning#1#2{%
  \setbox\ltx@zero\hbox{#1#2}%
  \HOLOGO@TempDimen=\wd\ltx@zero
  \setbox\ltx@zero\hbox{#1\kern0pt#2}%
  \advance\HOLOGO@TempDimen by -\wd\ltx@zero
}
%    \end{macrocode}
%    \end{macro}
%
%    \begin{macro}{\HOLOGO@SpaceFactor}
%    \begin{macrocode}
\def\HOLOGO@SpaceFactor{%
  \spacefactor1000 %
}
%    \end{macrocode}
%    \end{macro}
%
%    \begin{macro}{\HOLOGO@Span}
%    \begin{macrocode}
\def\HOLOGO@Span#1#2{%
  \HCode{<span class="HoLogo-#1">}%
  #2%
  \HCode{</span>}%
}
%    \end{macrocode}
%    \end{macro}
%
% \subsubsection{Text subscript}
%
%    \begin{macro}{\HOLOGO@SubScript}%
%    \begin{macrocode}
\def\HOLOGO@SubScript#1{%
  \ltx@IfUndefined{textsubscript}{%
    \ltx@IfUndefined{text}{%
      \ltx@mbox{%
        \mathsurround=0pt\relax
        $%
          _{%
            \ltx@IfUndefined{sf@size}{%
              \mathrm{#1}%
            }{%
              \mbox{%
                \fontsize\sf@size{0pt}\selectfont
                #1%
              }%
            }%
          }%
        $%
      }%
    }{%
      \ltx@mbox{%
        \mathsurround=0pt\relax
        $_{\text{#1}}$%
      }%
    }%
  }{%
    \textsubscript{#1}%
  }%
}
%    \end{macrocode}
%    \end{macro}
%
% \subsection{\hologo{TeX} and friends}
%
% \subsubsection{\hologo{TeX}}
%
%    \begin{macro}{\HoLogo@TeX}
%    Source: \hologo{LaTeX} kernel.
%    \begin{macrocode}
\def\HoLogo@TeX#1{%
  T\kern-.1667em\lower.5ex\hbox{E}\kern-.125emX\HOLOGO@SpaceFactor
}
%    \end{macrocode}
%    \end{macro}
%    \begin{macro}{\HoLogoHtml@TeX}
%    \begin{macrocode}
\def\HoLogoHtml@TeX#1{%
  \HoLogoCss@TeX
  \HOLOGO@Span{TeX}{%
    T%
    \HOLOGO@Span{e}{%
      E%
    }%
    X%
  }%
}
%    \end{macrocode}
%    \end{macro}
%    \begin{macro}{\HoLogoCss@TeX}
%    \begin{macrocode}
\def\HoLogoCss@TeX{%
  \Css{%
    span.HoLogo-TeX span.HoLogo-e{%
      position:relative;%
      top:.5ex;%
      margin-left:-.1667em;%
      margin-right:-.125em;%
    }%
  }%
  \Css{%
    a span.HoLogo-TeX span.HoLogo-e{%
      text-decoration:none;%
    }%
  }%
  \global\let\HoLogoCss@TeX\relax
}
%    \end{macrocode}
%    \end{macro}
%
% \subsubsection{\hologo{plainTeX}}
%
%    \begin{macro}{\HoLogo@plainTeX@space}
%    Source: ``The \hologo{TeX}book''
%    \begin{macrocode}
\def\HoLogo@plainTeX@space#1{%
  \HOLOGO@mbox{#1{p}{P}lain}\HOLOGO@space\hologo{TeX}%
}
%    \end{macrocode}
%    \end{macro}
%    \begin{macro}{\HoLogoCs@plainTeX@space}
%    \begin{macrocode}
\def\HoLogoCs@plainTeX@space#1{#1{p}{P}lain TeX}%
%    \end{macrocode}
%    \end{macro}
%    \begin{macro}{\HoLogoBkm@plainTeX@space}
%    \begin{macrocode}
\def\HoLogoBkm@plainTeX@space#1{%
  #1{p}{P}lain \hologo{TeX}%
}
%    \end{macrocode}
%    \end{macro}
%    \begin{macro}{\HoLogoHtml@plainTeX@space}
%    \begin{macrocode}
\def\HoLogoHtml@plainTeX@space#1{%
  #1{p}{P}lain \hologo{TeX}%
}
%    \end{macrocode}
%    \end{macro}
%
%    \begin{macro}{\HoLogo@plainTeX@hyphen}
%    \begin{macrocode}
\def\HoLogo@plainTeX@hyphen#1{%
  \HOLOGO@mbox{#1{p}{P}lain}\HOLOGO@hyphen\hologo{TeX}%
}
%    \end{macrocode}
%    \end{macro}
%    \begin{macro}{\HoLogoCs@plainTeX@hyphen}
%    \begin{macrocode}
\def\HoLogoCs@plainTeX@hyphen#1{#1{p}{P}lain-TeX}
%    \end{macrocode}
%    \end{macro}
%    \begin{macro}{\HoLogoBkm@plainTeX@hyphen}
%    \begin{macrocode}
\def\HoLogoBkm@plainTeX@hyphen#1{%
  #1{p}{P}lain-\hologo{TeX}%
}
%    \end{macrocode}
%    \end{macro}
%    \begin{macro}{\HoLogoHtml@plainTeX@hyphen}
%    \begin{macrocode}
\def\HoLogoHtml@plainTeX@hyphen#1{%
  #1{p}{P}lain-\hologo{TeX}%
}
%    \end{macrocode}
%    \end{macro}
%
%    \begin{macro}{\HoLogo@plainTeX@runtogether}
%    \begin{macrocode}
\def\HoLogo@plainTeX@runtogether#1{%
  \HOLOGO@mbox{#1{p}{P}lain\hologo{TeX}}%
}
%    \end{macrocode}
%    \end{macro}
%    \begin{macro}{\HoLogoCs@plainTeX@runtogether}
%    \begin{macrocode}
\def\HoLogoCs@plainTeX@runtogether#1{#1{p}{P}lainTeX}
%    \end{macrocode}
%    \end{macro}
%    \begin{macro}{\HoLogoBkm@plainTeX@runtogether}
%    \begin{macrocode}
\def\HoLogoBkm@plainTeX@runtogether#1{%
  #1{p}{P}lain\hologo{TeX}%
}
%    \end{macrocode}
%    \end{macro}
%    \begin{macro}{\HoLogoHtml@plainTeX@runtogether}
%    \begin{macrocode}
\def\HoLogoHtml@plainTeX@runtogether#1{%
  #1{p}{P}lain\hologo{TeX}%
}
%    \end{macrocode}
%    \end{macro}
%
%    \begin{macro}{\HoLogo@plainTeX}
%    \begin{macrocode}
\def\HoLogo@plainTeX{\HoLogo@plainTeX@space}
%    \end{macrocode}
%    \end{macro}
%    \begin{macro}{\HoLogoCs@plainTeX}
%    \begin{macrocode}
\def\HoLogoCs@plainTeX{\HoLogoCs@plainTeX@space}
%    \end{macrocode}
%    \end{macro}
%    \begin{macro}{\HoLogoBkm@plainTeX}
%    \begin{macrocode}
\def\HoLogoBkm@plainTeX{\HoLogoBkm@plainTeX@space}
%    \end{macrocode}
%    \end{macro}
%    \begin{macro}{\HoLogoHtml@plainTeX}
%    \begin{macrocode}
\def\HoLogoHtml@plainTeX{\HoLogoHtml@plainTeX@space}
%    \end{macrocode}
%    \end{macro}
%
% \subsubsection{\hologo{LaTeX}}
%
%    Source: \hologo{LaTeX} kernel.
%\begin{quote}
%\begin{verbatim}
%\DeclareRobustCommand{\LaTeX}{%
%  L%
%  \kern-.36em%
%  {%
%    \sbox\z@ T%
%    \vbox to\ht\z@{%
%      \hbox{%
%        \check@mathfonts
%        \fontsize\sf@size\z@
%        \math@fontsfalse
%        \selectfont
%        A%
%      }%
%      \vss
%    }%
%  }%
%  \kern-.15em%
%  \TeX
%}
%\end{verbatim}
%\end{quote}
%
%    \begin{macro}{\HoLogo@La}
%    \begin{macrocode}
\def\HoLogo@La#1{%
  L%
  \kern-.36em%
  \begingroup
    \setbox\ltx@zero\hbox{T}%
    \vbox to\ht\ltx@zero{%
      \hbox{%
        \ltx@ifundefined{check@mathfonts}{%
          \csname sevenrm\endcsname
        }{%
          \check@mathfonts
          \fontsize\sf@size{0pt}%
          \math@fontsfalse\selectfont
        }%
        A%
      }%
      \vss
    }%
  \endgroup
}
%    \end{macrocode}
%    \end{macro}
%
%    \begin{macro}{\HoLogo@LaTeX}
%    Source: \hologo{LaTeX} kernel.
%    \begin{macrocode}
\def\HoLogo@LaTeX#1{%
  \hologo{La}%
  \kern-.15em%
  \hologo{TeX}%
}
%    \end{macrocode}
%    \end{macro}
%    \begin{macro}{\HoLogoHtml@LaTeX}
%    \begin{macrocode}
\def\HoLogoHtml@LaTeX#1{%
  \HoLogoCss@LaTeX
  \HOLOGO@Span{LaTeX}{%
    L%
    \HOLOGO@Span{a}{%
      A%
    }%
    \hologo{TeX}%
  }%
}
%    \end{macrocode}
%    \end{macro}
%    \begin{macro}{\HoLogoCss@LaTeX}
%    \begin{macrocode}
\def\HoLogoCss@LaTeX{%
  \Css{%
    span.HoLogo-LaTeX span.HoLogo-a{%
      position:relative;%
      top:-.5ex;%
      margin-left:-.36em;%
      margin-right:-.15em;%
      font-size:85\%;%
    }%
  }%
  \global\let\HoLogoCss@LaTeX\relax
}
%    \end{macrocode}
%    \end{macro}
%
% \subsubsection{\hologo{(La)TeX}}
%
%    \begin{macro}{\HoLogo@LaTeXTeX}
%    The kerning around the parentheses is taken
%    from package \xpackage{dtklogos} \cite{dtklogos}.
%\begin{quote}
%\begin{verbatim}
%\DeclareRobustCommand{\LaTeXTeX}{%
%  (%
%  \kern-.15em%
%  L%
%  \kern-.36em%
%  {%
%    \sbox\z@ T%
%    \vbox to\ht0{%
%      \hbox{%
%        $\m@th$%
%        \csname S@\f@size\endcsname
%        \fontsize\sf@size\z@
%        \math@fontsfalse
%        \selectfont
%        A%
%      }%
%      \vss
%    }%
%  }%
%  \kern-.2em%
%  )%
%  \kern-.15em%
%  \TeX
%}
%\end{verbatim}
%\end{quote}
%    \begin{macrocode}
\def\HoLogo@LaTeXTeX#1{%
  (%
  \kern-.15em%
  \hologo{La}%
  \kern-.2em%
  )%
  \kern-.15em%
  \hologo{TeX}%
}
%    \end{macrocode}
%    \end{macro}
%    \begin{macro}{\HoLogoBkm@LaTeXTeX}
%    \begin{macrocode}
\def\HoLogoBkm@LaTeXTeX#1{(La)TeX}
%    \end{macrocode}
%    \end{macro}
%
%    \begin{macro}{\HoLogo@(La)TeX}
%    \begin{macrocode}
\expandafter
\let\csname HoLogo@(La)TeX\endcsname\HoLogo@LaTeXTeX
%    \end{macrocode}
%    \end{macro}
%    \begin{macro}{\HoLogoBkm@(La)TeX}
%    \begin{macrocode}
\expandafter
\let\csname HoLogoBkm@(La)TeX\endcsname\HoLogoBkm@LaTeXTeX
%    \end{macrocode}
%    \end{macro}
%    \begin{macro}{\HoLogoHtml@LaTeXTeX}
%    \begin{macrocode}
\def\HoLogoHtml@LaTeXTeX#1{%
  \HoLogoCss@LaTeXTeX
  \HOLOGO@Span{LaTeXTeX}{%
    (%
    \HOLOGO@Span{L}{L}%
    \HOLOGO@Span{a}{A}%
    \HOLOGO@Span{ParenRight}{)}%
    \hologo{TeX}%
  }%
}
%    \end{macrocode}
%    \end{macro}
%    \begin{macro}{\HoLogoHtml@(La)TeX}
%    Kerning after opening parentheses and before closing parentheses
%    is $-0.1$\,em. The original values $-0.15$\,em
%    looked too ugly for a serif font.
%    \begin{macrocode}
\expandafter
\let\csname HoLogoHtml@(La)TeX\endcsname\HoLogoHtml@LaTeXTeX
%    \end{macrocode}
%    \end{macro}
%    \begin{macro}{\HoLogoCss@LaTeXTeX}
%    \begin{macrocode}
\def\HoLogoCss@LaTeXTeX{%
  \Css{%
    span.HoLogo-LaTeXTeX span.HoLogo-L{%
      margin-left:-.1em;%
    }%
  }%
  \Css{%
    span.HoLogo-LaTeXTeX span.HoLogo-a{%
      position:relative;%
      top:-.5ex;%
      margin-left:-.36em;%
      margin-right:-.1em;%
      font-size:85\%;%
    }%
  }%
  \Css{%
    span.HoLogo-LaTeXTeX span.HoLogo-ParenRight{%
      margin-right:-.15em;%
    }%
  }%
  \global\let\HoLogoCss@LaTeXTeX\relax
}
%    \end{macrocode}
%    \end{macro}
%
% \subsubsection{\hologo{LaTeXe}}
%
%    \begin{macro}{\HoLogo@LaTeXe}
%    Source: \hologo{LaTeX} kernel
%    \begin{macrocode}
\def\HoLogo@LaTeXe#1{%
  \hologo{LaTeX}%
  \kern.15em%
  \hbox{%
    \HOLOGO@MathSetup
    2%
    $_{\textstyle\varepsilon}$%
  }%
}
%    \end{macrocode}
%    \end{macro}
%
%    \begin{macro}{\HoLogoCs@LaTeXe}
%    \begin{macrocode}
\ifnum64=`\^^^^0040\relax % test for big chars of LuaTeX/XeTeX
  \catcode`\$=9 %
  \catcode`\&=14 %
\else
  \catcode`\$=14 %
  \catcode`\&=9 %
\fi
\def\HoLogoCs@LaTeXe#1{%
  LaTeX2%
$ \string ^^^^0395%
& e%
}%
\catcode`\$=3 %
\catcode`\&=4 %
%    \end{macrocode}
%    \end{macro}
%
%    \begin{macro}{\HoLogoBkm@LaTeXe}
%    \begin{macrocode}
\def\HoLogoBkm@LaTeXe#1{%
  \hologo{LaTeX}%
  2%
  \HOLOGO@PdfdocUnicode{e}{\textepsilon}%
}
%    \end{macrocode}
%    \end{macro}
%
%    \begin{macro}{\HoLogoHtml@LaTeXe}
%    \begin{macrocode}
\def\HoLogoHtml@LaTeXe#1{%
  \HoLogoCss@LaTeXe
  \HOLOGO@Span{LaTeX2e}{%
    \hologo{LaTeX}%
    \HOLOGO@Span{2}{2}%
    \HOLOGO@Span{e}{%
      \HOLOGO@MathSetup
      \ensuremath{\textstyle\varepsilon}%
    }%
  }%
}
%    \end{macrocode}
%    \end{macro}
%    \begin{macro}{\HoLogoCss@LaTeXe}
%    \begin{macrocode}
\def\HoLogoCss@LaTeXe{%
  \Css{%
    span.HoLogo-LaTeX2e span.HoLogo-2{%
      padding-left:.15em;%
    }%
  }%
  \Css{%
    span.HoLogo-LaTeX2e span.HoLogo-e{%
      position:relative;%
      top:.35ex;%
      text-decoration:none;%
    }%
  }%
  \global\let\HoLogoCss@LaTeXe\relax
}
%    \end{macrocode}
%    \end{macro}
%
%    \begin{macro}{\HoLogo@LaTeX2e}
%    \begin{macrocode}
\expandafter
\let\csname HoLogo@LaTeX2e\endcsname\HoLogo@LaTeXe
%    \end{macrocode}
%    \end{macro}
%    \begin{macro}{\HoLogoCs@LaTeX2e}
%    \begin{macrocode}
\expandafter
\let\csname HoLogoCs@LaTeX2e\endcsname\HoLogoCs@LaTeXe
%    \end{macrocode}
%    \end{macro}
%    \begin{macro}{\HoLogoBkm@LaTeX2e}
%    \begin{macrocode}
\expandafter
\let\csname HoLogoBkm@LaTeX2e\endcsname\HoLogoBkm@LaTeXe
%    \end{macrocode}
%    \end{macro}
%    \begin{macro}{\HoLogoHtml@LaTeX2e}
%    \begin{macrocode}
\expandafter
\let\csname HoLogoHtml@LaTeX2e\endcsname\HoLogoHtml@LaTeXe
%    \end{macrocode}
%    \end{macro}
%
% \subsubsection{\hologo{LaTeX3}}
%
%    \begin{macro}{\HoLogo@LaTeX3}
%    Source: \hologo{LaTeX} kernel
%    \begin{macrocode}
\expandafter\def\csname HoLogo@LaTeX3\endcsname#1{%
  \hologo{LaTeX}%
  3%
}
%    \end{macrocode}
%    \end{macro}
%
%    \begin{macro}{\HoLogoBkm@LaTeX3}
%    \begin{macrocode}
\expandafter\def\csname HoLogoBkm@LaTeX3\endcsname#1{%
  \hologo{LaTeX}%
  3%
}
%    \end{macrocode}
%    \end{macro}
%    \begin{macro}{\HoLogoHtml@LaTeX3}
%    \begin{macrocode}
\expandafter
\let\csname HoLogoHtml@LaTeX3\expandafter\endcsname
\csname HoLogo@LaTeX3\endcsname
%    \end{macrocode}
%    \end{macro}
%
% \subsubsection{\hologo{LaTeXML}}
%
%    \begin{macro}{\HoLogo@LaTeXML}
%    \begin{macrocode}
\def\HoLogo@LaTeXML#1{%
  \HOLOGO@mbox{%
    \hologo{La}%
    \kern-.15em%
    T%
    \kern-.1667em%
    \lower.5ex\hbox{E}%
    \kern-.125em%
    \HoLogoFont@font{LaTeXML}{sc}{xml}%
  }%
}
%    \end{macrocode}
%    \end{macro}
%    \begin{macro}{\HoLogoHtml@pdfLaTeX}
%    \begin{macrocode}
\def\HoLogoHtml@LaTeXML#1{%
  \HOLOGO@Span{LaTeXML}{%
    \HoLogoCss@LaTeX
    \HoLogoCss@TeX
    \HOLOGO@Span{LaTeX}{%
      L%
      \HOLOGO@Span{a}{%
        A%
      }%
    }%
    \HOLOGO@Span{TeX}{%
      T%
      \HOLOGO@Span{e}{%
        E%
      }%
    }%
    \HCode{<span style="font-variant: small-caps;">}%
    xml%
    \HCode{</span>}%
  }%
}
%    \end{macrocode}
%    \end{macro}
%
% \subsubsection{\hologo{eTeX}}
%
%    \begin{macro}{\HoLogo@eTeX}
%    Source: package \xpackage{etex}
%    \begin{macrocode}
\def\HoLogo@eTeX#1{%
  \ltx@mbox{%
    \HOLOGO@MathSetup
    $\varepsilon$%
    -%
    \HOLOGO@NegativeKerning{-T,T-,To}%
    \hologo{TeX}%
  }%
}
%    \end{macrocode}
%    \end{macro}
%    \begin{macro}{\HoLogoCs@eTeX}
%    \begin{macrocode}
\ifnum64=`\^^^^0040\relax % test for big chars of LuaTeX/XeTeX
  \catcode`\$=9 %
  \catcode`\&=14 %
\else
  \catcode`\$=14 %
  \catcode`\&=9 %
\fi
\def\HoLogoCs@eTeX#1{%
$ #1{\string ^^^^0395}{\string ^^^^03b5}%
& #1{e}{E}%
  TeX%
}%
\catcode`\$=3 %
\catcode`\&=4 %
%    \end{macrocode}
%    \end{macro}
%    \begin{macro}{\HoLogoBkm@eTeX}
%    \begin{macrocode}
\def\HoLogoBkm@eTeX#1{%
  \HOLOGO@PdfdocUnicode{#1{e}{E}}{\textepsilon}%
  -%
  \hologo{TeX}%
}
%    \end{macrocode}
%    \end{macro}
%    \begin{macro}{\HoLogoHtml@eTeX}
%    \begin{macrocode}
\def\HoLogoHtml@eTeX#1{%
  \ltx@mbox{%
    \HOLOGO@MathSetup
    $\varepsilon$%
    -%
    \hologo{TeX}%
  }%
}
%    \end{macrocode}
%    \end{macro}
%
% \subsubsection{\hologo{iniTeX}}
%
%    \begin{macro}{\HoLogo@iniTeX}
%    \begin{macrocode}
\def\HoLogo@iniTeX#1{%
  \HOLOGO@mbox{%
    #1{i}{I}ni\hologo{TeX}%
  }%
}
%    \end{macrocode}
%    \end{macro}
%    \begin{macro}{\HoLogoCs@iniTeX}
%    \begin{macrocode}
\def\HoLogoCs@iniTeX#1{#1{i}{I}niTeX}
%    \end{macrocode}
%    \end{macro}
%    \begin{macro}{\HoLogoBkm@iniTeX}
%    \begin{macrocode}
\def\HoLogoBkm@iniTeX#1{%
  #1{i}{I}ni\hologo{TeX}%
}
%    \end{macrocode}
%    \end{macro}
%    \begin{macro}{\HoLogoHtml@iniTeX}
%    \begin{macrocode}
\let\HoLogoHtml@iniTeX\HoLogo@iniTeX
%    \end{macrocode}
%    \end{macro}
%
% \subsubsection{\hologo{virTeX}}
%
%    \begin{macro}{\HoLogo@virTeX}
%    \begin{macrocode}
\def\HoLogo@virTeX#1{%
  \HOLOGO@mbox{%
    #1{v}{V}ir\hologo{TeX}%
  }%
}
%    \end{macrocode}
%    \end{macro}
%    \begin{macro}{\HoLogoCs@virTeX}
%    \begin{macrocode}
\def\HoLogoCs@virTeX#1{#1{v}{V}irTeX}
%    \end{macrocode}
%    \end{macro}
%    \begin{macro}{\HoLogoBkm@virTeX}
%    \begin{macrocode}
\def\HoLogoBkm@virTeX#1{%
  #1{v}{V}ir\hologo{TeX}%
}
%    \end{macrocode}
%    \end{macro}
%    \begin{macro}{\HoLogoHtml@virTeX}
%    \begin{macrocode}
\let\HoLogoHtml@virTeX\HoLogo@virTeX
%    \end{macrocode}
%    \end{macro}
%
% \subsubsection{\hologo{SliTeX}}
%
% \paragraph{Definitions of the three variants.}
%
%    \begin{macro}{\HoLogo@SLiTeX@lift}
%    \begin{macrocode}
\def\HoLogo@SLiTeX@lift#1{%
  \HoLogoFont@font{SliTeX}{rm}{%
    S%
    \kern-.06em%
    L%
    \kern-.18em%
    \raise.32ex\hbox{\HoLogoFont@font{SliTeX}{sc}{i}}%
    \HOLOGO@discretionary
    \kern-.06em%
    \hologo{TeX}%
  }%
}
%    \end{macrocode}
%    \end{macro}
%    \begin{macro}{\HoLogoBkm@SLiTeX@lift}
%    \begin{macrocode}
\def\HoLogoBkm@SLiTeX@lift#1{SLiTeX}
%    \end{macrocode}
%    \end{macro}
%    \begin{macro}{\HoLogoHtml@SLiTeX@lift}
%    \begin{macrocode}
\def\HoLogoHtml@SLiTeX@lift#1{%
  \HoLogoCss@SLiTeX@lift
  \HOLOGO@Span{SLiTeX-lift}{%
    \HoLogoFont@font{SliTeX}{rm}{%
      S%
      \HOLOGO@Span{L}{L}%
      \HOLOGO@Span{i}{i}%
      \hologo{TeX}%
    }%
  }%
}
%    \end{macrocode}
%    \end{macro}
%    \begin{macro}{\HoLogoCss@SLiTeX@lift}
%    \begin{macrocode}
\def\HoLogoCss@SLiTeX@lift{%
  \Css{%
    span.HoLogo-SLiTeX-lift span.HoLogo-L{%
      margin-left:-.06em;%
      margin-right:-.18em;%
    }%
  }%
  \Css{%
    span.HoLogo-SLiTeX-lift span.HoLogo-i{%
      position:relative;%
      top:-.32ex;%
      margin-right:-.06em;%
      font-variant:small-caps;%
    }%
  }%
  \global\let\HoLogoCss@SLiTeX@lift\relax
}
%    \end{macrocode}
%    \end{macro}
%
%    \begin{macro}{\HoLogo@SliTeX@simple}
%    \begin{macrocode}
\def\HoLogo@SliTeX@simple#1{%
  \HoLogoFont@font{SliTeX}{rm}{%
    \ltx@mbox{%
      \HoLogoFont@font{SliTeX}{sc}{Sli}%
    }%
    \HOLOGO@discretionary
    \hologo{TeX}%
  }%
}
%    \end{macrocode}
%    \end{macro}
%    \begin{macro}{\HoLogoBkm@SliTeX@simple}
%    \begin{macrocode}
\def\HoLogoBkm@SliTeX@simple#1{SliTeX}
%    \end{macrocode}
%    \end{macro}
%    \begin{macro}{\HoLogoHtml@SliTeX@simple}
%    \begin{macrocode}
\let\HoLogoHtml@SliTeX@simple\HoLogo@SliTeX@simple
%    \end{macrocode}
%    \end{macro}
%
%    \begin{macro}{\HoLogo@SliTeX@narrow}
%    \begin{macrocode}
\def\HoLogo@SliTeX@narrow#1{%
  \HoLogoFont@font{SliTeX}{rm}{%
    \ltx@mbox{%
      S%
      \kern-.06em%
      \HoLogoFont@font{SliTeX}{sc}{%
        l%
        \kern-.035em%
        i%
      }%
    }%
    \HOLOGO@discretionary
    \kern-.06em%
    \hologo{TeX}%
  }%
}
%    \end{macrocode}
%    \end{macro}
%    \begin{macro}{\HoLogoBkm@SliTeX@narrow}
%    \begin{macrocode}
\def\HoLogoBkm@SliTeX@narrow#1{SliTeX}
%    \end{macrocode}
%    \end{macro}
%    \begin{macro}{\HoLogoHtml@SliTeX@narrow}
%    \begin{macrocode}
\def\HoLogoHtml@SliTeX@narrow#1{%
  \HoLogoCss@SliTeX@narrow
  \HOLOGO@Span{SliTeX-narrow}{%
    \HoLogoFont@font{SliTeX}{rm}{%
      S%
        \HOLOGO@Span{l}{l}%
        \HOLOGO@Span{i}{i}%
      \hologo{TeX}%
    }%
  }%
}
%    \end{macrocode}
%    \end{macro}
%    \begin{macro}{\HoLogoCss@SliTeX@narrow}
%    \begin{macrocode}
\def\HoLogoCss@SliTeX@narrow{%
  \Css{%
    span.HoLogo-SliTeX-narrow span.HoLogo-l{%
      margin-left:-.06em;%
      margin-right:-.035em;%
      font-variant:small-caps;%
    }%
  }%
  \Css{%
    span.HoLogo-SliTeX-narrow span.HoLogo-i{%
      margin-right:-.06em;%
      font-variant:small-caps;%
    }%
  }%
  \global\let\HoLogoCss@SliTeX@narrow\relax
}
%    \end{macrocode}
%    \end{macro}
%
% \paragraph{Macro set completion.}
%
%    \begin{macro}{\HoLogo@SLiTeX@simple}
%    \begin{macrocode}
\def\HoLogo@SLiTeX@simple{\HoLogo@SliTeX@simple}
%    \end{macrocode}
%    \end{macro}
%    \begin{macro}{\HoLogoBkm@SLiTeX@simple}
%    \begin{macrocode}
\def\HoLogoBkm@SLiTeX@simple{\HoLogoBkm@SliTeX@simple}
%    \end{macrocode}
%    \end{macro}
%    \begin{macro}{\HoLogoHtml@SLiTeX@simple}
%    \begin{macrocode}
\def\HoLogoHtml@SLiTeX@simple{\HoLogoHtml@SliTeX@simple}
%    \end{macrocode}
%    \end{macro}
%
%    \begin{macro}{\HoLogo@SLiTeX@narrow}
%    \begin{macrocode}
\def\HoLogo@SLiTeX@narrow{\HoLogo@SliTeX@narrow}
%    \end{macrocode}
%    \end{macro}
%    \begin{macro}{\HoLogoBkm@SLiTeX@narrow}
%    \begin{macrocode}
\def\HoLogoBkm@SLiTeX@narrow{\HoLogoBkm@SliTeX@narrow}
%    \end{macrocode}
%    \end{macro}
%    \begin{macro}{\HoLogoHtml@SLiTeX@narrow}
%    \begin{macrocode}
\def\HoLogoHtml@SLiTeX@narrow{\HoLogoHtml@SliTeX@narrow}
%    \end{macrocode}
%    \end{macro}
%
%    \begin{macro}{\HoLogo@SliTeX@lift}
%    \begin{macrocode}
\def\HoLogo@SliTeX@lift{\HoLogo@SLiTeX@lift}
%    \end{macrocode}
%    \end{macro}
%    \begin{macro}{\HoLogoBkm@SliTeX@lift}
%    \begin{macrocode}
\def\HoLogoBkm@SliTeX@lift{\HoLogoBkm@SLiTeX@lift}
%    \end{macrocode}
%    \end{macro}
%    \begin{macro}{\HoLogoHtml@SliTeX@lift}
%    \begin{macrocode}
\def\HoLogoHtml@SliTeX@lift{\HoLogoHtml@SLiTeX@lift}
%    \end{macrocode}
%    \end{macro}
%
% \paragraph{Defaults.}
%
%    \begin{macro}{\HoLogo@SLiTeX}
%    \begin{macrocode}
\def\HoLogo@SLiTeX{\HoLogo@SLiTeX@lift}
%    \end{macrocode}
%    \end{macro}
%    \begin{macro}{\HoLogoBkm@SLiTeX}
%    \begin{macrocode}
\def\HoLogoBkm@SLiTeX{\HoLogoBkm@SLiTeX@lift}
%    \end{macrocode}
%    \end{macro}
%    \begin{macro}{\HoLogoHtml@SLiTeX}
%    \begin{macrocode}
\def\HoLogoHtml@SLiTeX{\HoLogoHtml@SLiTeX@lift}
%    \end{macrocode}
%    \end{macro}
%
%    \begin{macro}{\HoLogo@SliTeX}
%    \begin{macrocode}
\def\HoLogo@SliTeX{\HoLogo@SliTeX@narrow}
%    \end{macrocode}
%    \end{macro}
%    \begin{macro}{\HoLogoBkm@SliTeX}
%    \begin{macrocode}
\def\HoLogoBkm@SliTeX{\HoLogoBkm@SliTeX@narrow}
%    \end{macrocode}
%    \end{macro}
%    \begin{macro}{\HoLogoHtml@SliTeX}
%    \begin{macrocode}
\def\HoLogoHtml@SliTeX{\HoLogoHtml@SliTeX@narrow}
%    \end{macrocode}
%    \end{macro}
%
% \subsubsection{\hologo{LuaTeX}}
%
%    \begin{macro}{\HoLogo@LuaTeX}
%    The kerning is an idea of Hans Hagen, see mailing list
%    `luatex at tug dot org' in March 2010.
%    \begin{macrocode}
\def\HoLogo@LuaTeX#1{%
  \HOLOGO@mbox{%
    Lua%
    \HOLOGO@NegativeKerning{aT,oT,To}%
    \hologo{TeX}%
  }%
}
%    \end{macrocode}
%    \end{macro}
%    \begin{macro}{\HoLogoHtml@LuaTeX}
%    \begin{macrocode}
\let\HoLogoHtml@LuaTeX\HoLogo@LuaTeX
%    \end{macrocode}
%    \end{macro}
%
% \subsubsection{\hologo{LuaLaTeX}}
%
%    \begin{macro}{\HoLogo@LuaLaTeX}
%    \begin{macrocode}
\def\HoLogo@LuaLaTeX#1{%
  \HOLOGO@mbox{%
    Lua%
    \hologo{LaTeX}%
  }%
}
%    \end{macrocode}
%    \end{macro}
%    \begin{macro}{\HoLogoHtml@LuaLaTeX}
%    \begin{macrocode}
\let\HoLogoHtml@LuaLaTeX\HoLogo@LuaLaTeX
%    \end{macrocode}
%    \end{macro}
%
% \subsubsection{\hologo{XeTeX}, \hologo{XeLaTeX}}
%
%    \begin{macro}{\HOLOGO@IfCharExists}
%    \begin{macrocode}
\ifluatex
  \ifnum\luatexversion<36 %
  \else
    \def\HOLOGO@IfCharExists#1{%
      \ifnum
        \directlua{%
           if luaotfload and luaotfload.aux then
             if luaotfload.aux.font_has_glyph(%
                    font.current(), \number#1) then % 	 
	       tex.print("1") % 	 
	     end % 	 
	   elseif font and font.fonts and font.current then %
            local f = font.fonts[font.current()]%
            if f.characters and f.characters[\number#1] then %
              tex.print("1")%
            end %
          end%
        }0=\ltx@zero
        \expandafter\ltx@secondoftwo
      \else
        \expandafter\ltx@firstoftwo
      \fi
    }%
  \fi
\fi
\ltx@IfUndefined{HOLOGO@IfCharExists}{%
  \def\HOLOGO@@IfCharExists#1{%
    \begingroup
      \tracinglostchars=\ltx@zero
      \setbox\ltx@zero=\hbox{%
        \kern7sp\char#1\relax
        \ifnum\lastkern>\ltx@zero
          \expandafter\aftergroup\csname iffalse\endcsname
        \else
          \expandafter\aftergroup\csname iftrue\endcsname
        \fi
      }%
      % \if{true|false} from \aftergroup
      \endgroup
      \expandafter\ltx@firstoftwo
    \else
      \endgroup
      \expandafter\ltx@secondoftwo
    \fi
  }%
  \ifxetex
    \ltx@IfUndefined{XeTeXfonttype}{}{%
      \ltx@IfUndefined{XeTeXcharglyph}{}{%
        \def\HOLOGO@IfCharExists#1{%
          \ifnum\XeTeXfonttype\font>\ltx@zero
            \expandafter\ltx@firstofthree
          \else
            \expandafter\ltx@gobble
          \fi
          {%
            \ifnum\XeTeXcharglyph#1>\ltx@zero
              \expandafter\ltx@firstoftwo
            \else
              \expandafter\ltx@secondoftwo
            \fi
          }%
          \HOLOGO@@IfCharExists{#1}%
        }%
      }%
    }%
  \fi
}{}
\ltx@ifundefined{HOLOGO@IfCharExists}{%
  \ifnum64=`\^^^^0040\relax % test for big chars of LuaTeX/XeTeX
    \let\HOLOGO@IfCharExists\HOLOGO@@IfCharExists
  \else
    \def\HOLOGO@IfCharExists#1{%
      \ifnum#1>255 %
        \expandafter\ltx@fourthoffour
      \fi
      \HOLOGO@@IfCharExists{#1}%
    }%
  \fi
}{}
%    \end{macrocode}
%    \end{macro}
%
%    \begin{macro}{\HoLogo@Xe}
%    Source: package \xpackage{dtklogos}
%    \begin{macrocode}
\def\HoLogo@Xe#1{%
  X%
  \kern-.1em\relax
  \HOLOGO@IfCharExists{"018E}{%
    \lower.5ex\hbox{\char"018E}%
  }{%
    \chardef\HOLOGO@choice=\ltx@zero
    \ifdim\fontdimen\ltx@one\font>0pt %
      \ltx@IfUndefined{rotatebox}{%
        \ltx@IfUndefined{pgftext}{%
          \ltx@IfUndefined{psscalebox}{%
            \ltx@IfUndefined{HOLOGO@ScaleBox@\hologoDriver}{%
            }{%
              \chardef\HOLOGO@choice=4 %
            }%
          }{%
            \chardef\HOLOGO@choice=3 %
          }%
        }{%
          \chardef\HOLOGO@choice=2 %
        }%
      }{%
        \chardef\HOLOGO@choice=1 %
      }%
      \ifcase\HOLOGO@choice
        \HOLOGO@WarningUnsupportedDriver{Xe}%
        e%
      \or % 1: \rotatebox
        \begingroup
          \setbox\ltx@zero\hbox{\rotatebox{180}{E}}%
          \ltx@LocDimenA=\dp\ltx@zero
          \advance\ltx@LocDimenA by -.5ex\relax
          \raise\ltx@LocDimenA\box\ltx@zero
        \endgroup
      \or % 2: \pgftext
        \lower.5ex\hbox{%
          \pgfpicture
            \pgftext[rotate=180]{E}%
          \endpgfpicture
        }%
      \or % 3: \psscalebox
        \begingroup
          \setbox\ltx@zero\hbox{\psscalebox{-1 -1}{E}}%
          \ltx@LocDimenA=\dp\ltx@zero
          \advance\ltx@LocDimenA by -.5ex\relax
          \raise\ltx@LocDimenA\box\ltx@zero
        \endgroup
      \or % 4: \HOLOGO@PointReflectBox
        \lower.5ex\hbox{\HOLOGO@PointReflectBox{E}}%
      \else
        \@PackageError{hologo}{Internal error (choice/it}\@ehc
      \fi
    \else
      \ltx@IfUndefined{reflectbox}{%
        \ltx@IfUndefined{pgftext}{%
          \ltx@IfUndefined{psscalebox}{%
            \ltx@IfUndefined{HOLOGO@ScaleBox@\hologoDriver}{%
            }{%
              \chardef\HOLOGO@choice=4 %
            }%
          }{%
            \chardef\HOLOGO@choice=3 %
          }%
        }{%
          \chardef\HOLOGO@choice=2 %
        }%
      }{%
        \chardef\HOLOGO@choice=1 %
      }%
      \ifcase\HOLOGO@choice
        \HOLOGO@WarningUnsupportedDriver{Xe}%
        e%
      \or % 1: reflectbox
        \lower.5ex\hbox{%
          \reflectbox{E}%
        }%
      \or % 2: \pgftext
        \lower.5ex\hbox{%
          \pgfpicture
            \pgftransformxscale{-1}%
            \pgftext{E}%
          \endpgfpicture
        }%
      \or % 3: \psscalebox
        \lower.5ex\hbox{%
          \psscalebox{-1 1}{E}%
        }%
      \or % 4: \HOLOGO@Reflectbox
        \lower.5ex\hbox{%
          \HOLOGO@ReflectBox{E}%
        }%
      \else
        \@PackageError{hologo}{Internal error (choice/up)}\@ehc
      \fi
    \fi
  }%
}
%    \end{macrocode}
%    \end{macro}
%    \begin{macro}{\HoLogoHtml@Xe}
%    \begin{macrocode}
\def\HoLogoHtml@Xe#1{%
  \HoLogoCss@Xe
  \HOLOGO@Span{Xe}{%
    X%
    \HOLOGO@Span{e}{%
      \HCode{&\ltx@hashchar x018e;}%
    }%
  }%
}
%    \end{macrocode}
%    \end{macro}
%    \begin{macro}{\HoLogoCss@Xe}
%    \begin{macrocode}
\def\HoLogoCss@Xe{%
  \Css{%
    span.HoLogo-Xe span.HoLogo-e{%
      position:relative;%
      top:.5ex;%
      left-margin:-.1em;%
    }%
  }%
  \global\let\HoLogoCss@Xe\relax
}
%    \end{macrocode}
%    \end{macro}
%
%    \begin{macro}{\HoLogo@XeTeX}
%    \begin{macrocode}
\def\HoLogo@XeTeX#1{%
  \hologo{Xe}%
  \kern-.15em\relax
  \hologo{TeX}%
}
%    \end{macrocode}
%    \end{macro}
%
%    \begin{macro}{\HoLogoHtml@XeTeX}
%    \begin{macrocode}
\def\HoLogoHtml@XeTeX#1{%
  \HoLogoCss@XeTeX
  \HOLOGO@Span{XeTeX}{%
    \hologo{Xe}%
    \hologo{TeX}%
  }%
}
%    \end{macrocode}
%    \end{macro}
%    \begin{macro}{\HoLogoCss@XeTeX}
%    \begin{macrocode}
\def\HoLogoCss@XeTeX{%
  \Css{%
    span.HoLogo-XeTeX span.HoLogo-TeX{%
      margin-left:-.15em;%
    }%
  }%
  \global\let\HoLogoCss@XeTeX\relax
}
%    \end{macrocode}
%    \end{macro}
%
%    \begin{macro}{\HoLogo@XeLaTeX}
%    \begin{macrocode}
\def\HoLogo@XeLaTeX#1{%
  \hologo{Xe}%
  \kern-.13em%
  \hologo{LaTeX}%
}
%    \end{macrocode}
%    \end{macro}
%    \begin{macro}{\HoLogoHtml@XeLaTeX}
%    \begin{macrocode}
\def\HoLogoHtml@XeLaTeX#1{%
  \HoLogoCss@XeLaTeX
  \HOLOGO@Span{XeLaTeX}{%
    \hologo{Xe}%
    \hologo{LaTeX}%
  }%
}
%    \end{macrocode}
%    \end{macro}
%    \begin{macro}{\HoLogoCss@XeLaTeX}
%    \begin{macrocode}
\def\HoLogoCss@XeLaTeX{%
  \Css{%
    span.HoLogo-XeLaTeX span.HoLogo-Xe{%
      margin-right:-.13em;%
    }%
  }%
  \global\let\HoLogoCss@XeLaTeX\relax
}
%    \end{macrocode}
%    \end{macro}
%
% \subsubsection{\hologo{pdfTeX}, \hologo{pdfLaTeX}}
%
%    \begin{macro}{\HoLogo@pdfTeX}
%    \begin{macrocode}
\def\HoLogo@pdfTeX#1{%
  \HOLOGO@mbox{%
    #1{p}{P}df\hologo{TeX}%
  }%
}
%    \end{macrocode}
%    \end{macro}
%    \begin{macro}{\HoLogoCs@pdfTeX}
%    \begin{macrocode}
\def\HoLogoCs@pdfTeX#1{#1{p}{P}dfTeX}
%    \end{macrocode}
%    \end{macro}
%    \begin{macro}{\HoLogoBkm@pdfTeX}
%    \begin{macrocode}
\def\HoLogoBkm@pdfTeX#1{%
  #1{p}{P}df\hologo{TeX}%
}
%    \end{macrocode}
%    \end{macro}
%    \begin{macro}{\HoLogoHtml@pdfTeX}
%    \begin{macrocode}
\let\HoLogoHtml@pdfTeX\HoLogo@pdfTeX
%    \end{macrocode}
%    \end{macro}
%
%    \begin{macro}{\HoLogo@pdfLaTeX}
%    \begin{macrocode}
\def\HoLogo@pdfLaTeX#1{%
  \HOLOGO@mbox{%
    #1{p}{P}df\hologo{LaTeX}%
  }%
}
%    \end{macrocode}
%    \end{macro}
%    \begin{macro}{\HoLogoCs@pdfLaTeX}
%    \begin{macrocode}
\def\HoLogoCs@pdfLaTeX#1{#1{p}{P}dfLaTeX}
%    \end{macrocode}
%    \end{macro}
%    \begin{macro}{\HoLogoBkm@pdfLaTeX}
%    \begin{macrocode}
\def\HoLogoBkm@pdfLaTeX#1{%
  #1{p}{P}df\hologo{LaTeX}%
}
%    \end{macrocode}
%    \end{macro}
%    \begin{macro}{\HoLogoHtml@pdfLaTeX}
%    \begin{macrocode}
\let\HoLogoHtml@pdfLaTeX\HoLogo@pdfLaTeX
%    \end{macrocode}
%    \end{macro}
%
% \subsubsection{\hologo{VTeX}}
%
%    \begin{macro}{\HoLogo@VTeX}
%    \begin{macrocode}
\def\HoLogo@VTeX#1{%
  \HOLOGO@mbox{%
    V\hologo{TeX}%
  }%
}
%    \end{macrocode}
%    \end{macro}
%    \begin{macro}{\HoLogoHtml@VTeX}
%    \begin{macrocode}
\let\HoLogoHtml@VTeX\HoLogo@VTeX
%    \end{macrocode}
%    \end{macro}
%
% \subsubsection{\hologo{AmS}, \dots}
%
%    Source: class \xclass{amsdtx}
%
%    \begin{macro}{\HoLogo@AmS}
%    \begin{macrocode}
\def\HoLogo@AmS#1{%
  \HoLogoFont@font{AmS}{sy}{%
    A%
    \kern-.1667em%
    \lower.5ex\hbox{M}%
    \kern-.125em%
    S%
  }%
}
%    \end{macrocode}
%    \end{macro}
%    \begin{macro}{\HoLogoBkm@AmS}
%    \begin{macrocode}
\def\HoLogoBkm@AmS#1{AmS}
%    \end{macrocode}
%    \end{macro}
%    \begin{macro}{\HoLogoHtml@AmS}
%    \begin{macrocode}
\def\HoLogoHtml@AmS#1{%
  \HoLogoCss@AmS
%  \HoLogoFont@font{AmS}{sy}{%
    \HOLOGO@Span{AmS}{%
      A%
      \HOLOGO@Span{M}{M}%
      S%
    }%
%   }%
}
%    \end{macrocode}
%    \end{macro}
%    \begin{macro}{\HoLogoCss@AmS}
%    \begin{macrocode}
\def\HoLogoCss@AmS{%
  \Css{%
    span.HoLogo-AmS span.HoLogo-M{%
      position:relative;%
      top:.5ex;%
      margin-left:-.1667em;%
      margin-right:-.125em;%
      text-decoration:none;%
    }%
  }%
  \global\let\HoLogoCss@AmS\relax
}
%    \end{macrocode}
%    \end{macro}
%
%    \begin{macro}{\HoLogo@AmSTeX}
%    \begin{macrocode}
\def\HoLogo@AmSTeX#1{%
  \hologo{AmS}%
  \HOLOGO@hyphen
  \hologo{TeX}%
}
%    \end{macrocode}
%    \end{macro}
%    \begin{macro}{\HoLogoBkm@AmSTeX}
%    \begin{macrocode}
\def\HoLogoBkm@AmSTeX#1{AmS-TeX}%
%    \end{macrocode}
%    \end{macro}
%    \begin{macro}{\HoLogoHtml@AmSTeX}
%    \begin{macrocode}
\let\HoLogoHtml@AmSTeX\HoLogo@AmSTeX
%    \end{macrocode}
%    \end{macro}
%
%    \begin{macro}{\HoLogo@AmSLaTeX}
%    \begin{macrocode}
\def\HoLogo@AmSLaTeX#1{%
  \hologo{AmS}%
  \HOLOGO@hyphen
  \hologo{LaTeX}%
}
%    \end{macrocode}
%    \end{macro}
%    \begin{macro}{\HoLogoBkm@AmSLaTeX}
%    \begin{macrocode}
\def\HoLogoBkm@AmSLaTeX#1{AmS-LaTeX}%
%    \end{macrocode}
%    \end{macro}
%    \begin{macro}{\HoLogoHtml@AmSLaTeX}
%    \begin{macrocode}
\let\HoLogoHtml@AmSLaTeX\HoLogo@AmSLaTeX
%    \end{macrocode}
%    \end{macro}
%
% \subsubsection{\hologo{BibTeX}}
%
%    \begin{macro}{\HoLogo@BibTeX@sc}
%    A definition of \hologo{BibTeX} is provided in
%    the documentation source for the manual of \hologo{BibTeX}
%    \cite{btxdoc}.
%\begin{quote}
%\begin{verbatim}
%\def\BibTeX{%
%  {%
%    \rm
%    B%
%    \kern-.05em%
%    {%
%      \sc
%      i%
%      \kern-.025em %
%      b%
%    }%
%    \kern-.08em
%    T%
%    \kern-.1667em%
%    \lower.7ex\hbox{E}%
%    \kern-.125em%
%    X%
%  }%
%}
%\end{verbatim}
%\end{quote}
%    \begin{macrocode}
\def\HoLogo@BibTeX@sc#1{%
  B%
  \kern-.05em%
  \HoLogoFont@font{BibTeX}{sc}{%
    i%
    \kern-.025em%
    b%
  }%
  \HOLOGO@discretionary
  \kern-.08em%
  \hologo{TeX}%
}
%    \end{macrocode}
%    \end{macro}
%    \begin{macro}{\HoLogoHtml@BibTeX@sc}
%    \begin{macrocode}
\def\HoLogoHtml@BibTeX@sc#1{%
  \HoLogoCss@BibTeX@sc
  \HOLOGO@Span{BibTeX-sc}{%
    B%
    \HOLOGO@Span{i}{i}%
    \HOLOGO@Span{b}{b}%
    \hologo{TeX}%
  }%
}
%    \end{macrocode}
%    \end{macro}
%    \begin{macro}{\HoLogoCss@BibTeX@sc}
%    \begin{macrocode}
\def\HoLogoCss@BibTeX@sc{%
  \Css{%
    span.HoLogo-BibTeX-sc span.HoLogo-i{%
      margin-left:-.05em;%
      margin-right:-.025em;%
      font-variant:small-caps;%
    }%
  }%
  \Css{%
    span.HoLogo-BibTeX-sc span.HoLogo-b{%
      margin-right:-.08em;%
      font-variant:small-caps;%
    }%
  }%
  \global\let\HoLogoCss@BibTeX@sc\relax
}
%    \end{macrocode}
%    \end{macro}
%
%    \begin{macro}{\HoLogo@BibTeX@sf}
%    Variant \xoption{sf} avoids trouble with unavailable
%    small caps fonts (e.g., bold versions of Computer Modern or
%    Latin Modern). The definition is taken from
%    package \xpackage{dtklogos} \cite{dtklogos}.
%\begin{quote}
%\begin{verbatim}
%\DeclareRobustCommand{\BibTeX}{%
%  B%
%  \kern-.05em%
%  \hbox{%
%    $\m@th$% %% force math size calculations
%    \csname S@\f@size\endcsname
%    \fontsize\sf@size\z@
%    \math@fontsfalse
%    \selectfont
%    I%
%    \kern-.025em%
%    B
%  }%
%  \kern-.08em%
%  \-%
%  \TeX
%}
%\end{verbatim}
%\end{quote}
%    \begin{macrocode}
\def\HoLogo@BibTeX@sf#1{%
  B%
  \kern-.05em%
  \HoLogoFont@font{BibTeX}{bibsf}{%
    I%
    \kern-.025em%
    B%
  }%
  \HOLOGO@discretionary
  \kern-.08em%
  \hologo{TeX}%
}
%    \end{macrocode}
%    \end{macro}
%    \begin{macro}{\HoLogoHtml@BibTeX@sf}
%    \begin{macrocode}
\def\HoLogoHtml@BibTeX@sf#1{%
  \HoLogoCss@BibTeX@sf
  \HOLOGO@Span{BibTeX-sf}{%
    B%
    \HoLogoFont@font{BibTeX}{bibsf}{%
      \HOLOGO@Span{i}{I}%
      B%
    }%
    \hologo{TeX}%
  }%
}
%    \end{macrocode}
%    \end{macro}
%    \begin{macro}{\HoLogoCss@BibTeX@sf}
%    \begin{macrocode}
\def\HoLogoCss@BibTeX@sf{%
  \Css{%
    span.HoLogo-BibTeX-sf span.HoLogo-i{%
      margin-left:-.05em;%
      margin-right:-.025em;%
    }%
  }%
  \Css{%
    span.HoLogo-BibTeX-sf span.HoLogo-TeX{%
      margin-left:-.08em;%
    }%
  }%
  \global\let\HoLogoCss@BibTeX@sf\relax
}
%    \end{macrocode}
%    \end{macro}
%
%    \begin{macro}{\HoLogo@BibTeX}
%    \begin{macrocode}
\def\HoLogo@BibTeX{\HoLogo@BibTeX@sf}
%    \end{macrocode}
%    \end{macro}
%    \begin{macro}{\HoLogoHtml@BibTeX}
%    \begin{macrocode}
\def\HoLogoHtml@BibTeX{\HoLogoHtml@BibTeX@sf}
%    \end{macrocode}
%    \end{macro}
%
% \subsubsection{\hologo{BibTeX8}}
%
%    \begin{macro}{\HoLogo@BibTeX8}
%    \begin{macrocode}
\expandafter\def\csname HoLogo@BibTeX8\endcsname#1{%
  \hologo{BibTeX}%
  8%
}
%    \end{macrocode}
%    \end{macro}
%
%    \begin{macro}{\HoLogoBkm@BibTeX8}
%    \begin{macrocode}
\expandafter\def\csname HoLogoBkm@BibTeX8\endcsname#1{%
  \hologo{BibTeX}%
  8%
}
%    \end{macrocode}
%    \end{macro}
%    \begin{macro}{\HoLogoHtml@BibTeX8}
%    \begin{macrocode}
\expandafter
\let\csname HoLogoHtml@BibTeX8\expandafter\endcsname
\csname HoLogo@BibTeX8\endcsname
%    \end{macrocode}
%    \end{macro}
%
% \subsubsection{\hologo{ConTeXt}}
%
%    \begin{macro}{\HoLogo@ConTeXt@simple}
%    \begin{macrocode}
\def\HoLogo@ConTeXt@simple#1{%
  \HOLOGO@mbox{Con}%
  \HOLOGO@discretionary
  \HOLOGO@mbox{\hologo{TeX}t}%
}
%    \end{macrocode}
%    \end{macro}
%    \begin{macro}{\HoLogoHtml@ConTeXt@simple}
%    \begin{macrocode}
\let\HoLogoHtml@ConTeXt@simple\HoLogo@ConTeXt@simple
%    \end{macrocode}
%    \end{macro}
%
%    \begin{macro}{\HoLogo@ConTeXt@narrow}
%    This definition of logo \hologo{ConTeXt} with variant \xoption{narrow}
%    comes from TUGboat's class \xclass{ltugboat} (version 2010/11/15 v2.8).
%    \begin{macrocode}
\def\HoLogo@ConTeXt@narrow#1{%
  \HOLOGO@mbox{C\kern-.0333emon}%
  \HOLOGO@discretionary
  \kern-.0667em%
  \HOLOGO@mbox{\hologo{TeX}\kern-.0333emt}%
}
%    \end{macrocode}
%    \end{macro}
%    \begin{macro}{\HoLogoHtml@ConTeXt@narrow}
%    \begin{macrocode}
\def\HoLogoHtml@ConTeXt@narrow#1{%
  \HoLogoCss@ConTeXt@narrow
  \HOLOGO@Span{ConTeXt-narrow}{%
    \HOLOGO@Span{C}{C}%
    on%
    \hologo{TeX}%
    t%
  }%
}
%    \end{macrocode}
%    \end{macro}
%    \begin{macro}{\HoLogoCss@ConTeXt@narrow}
%    \begin{macrocode}
\def\HoLogoCss@ConTeXt@narrow{%
  \Css{%
    span.HoLogo-ConTeXt-narrow span.HoLogo-C{%
      margin-left:-.0333em;%
    }%
  }%
  \Css{%
    span.HoLogo-ConTeXt-narrow span.HoLogo-TeX{%
      margin-left:-.0667em;%
      margin-right:-.0333em;%
    }%
  }%
  \global\let\HoLogoCss@ConTeXt@narrow\relax
}
%    \end{macrocode}
%    \end{macro}
%
%    \begin{macro}{\HoLogo@ConTeXt}
%    \begin{macrocode}
\def\HoLogo@ConTeXt{\HoLogo@ConTeXt@narrow}
%    \end{macrocode}
%    \end{macro}
%    \begin{macro}{\HoLogoHtml@ConTeXt}
%    \begin{macrocode}
\def\HoLogoHtml@ConTeXt{\HoLogoHtml@ConTeXt@narrow}
%    \end{macrocode}
%    \end{macro}
%
% \subsubsection{\hologo{emTeX}}
%
%    \begin{macro}{\HoLogo@emTeX}
%    \begin{macrocode}
\def\HoLogo@emTeX#1{%
  \HOLOGO@mbox{#1{e}{E}m}%
  \HOLOGO@discretionary
  \hologo{TeX}%
}
%    \end{macrocode}
%    \end{macro}
%    \begin{macro}{\HoLogoCs@emTeX}
%    \begin{macrocode}
\def\HoLogoCs@emTeX#1{#1{e}{E}mTeX}%
%    \end{macrocode}
%    \end{macro}
%    \begin{macro}{\HoLogoBkm@emTeX}
%    \begin{macrocode}
\def\HoLogoBkm@emTeX#1{%
  #1{e}{E}m\hologo{TeX}%
}
%    \end{macrocode}
%    \end{macro}
%    \begin{macro}{\HoLogoHtml@emTeX}
%    \begin{macrocode}
\let\HoLogoHtml@emTeX\HoLogo@emTeX
%    \end{macrocode}
%    \end{macro}
%
% \subsubsection{\hologo{ExTeX}}
%
%    \begin{macro}{\HoLogo@ExTeX}
%    The definition is taken from the FAQ of the
%    project \hologo{ExTeX}
%    \cite{ExTeX-FAQ}.
%\begin{quote}
%\begin{verbatim}
%\def\ExTeX{%
%  \textrm{% Logo always with serifs
%    \ensuremath{%
%      \textstyle
%      \varepsilon_{%
%        \kern-0.15em%
%        \mathcal{X}%
%      }%
%    }%
%    \kern-.15em%
%    \TeX
%  }%
%}
%\end{verbatim}
%\end{quote}
%    \begin{macrocode}
\def\HoLogo@ExTeX#1{%
  \HoLogoFont@font{ExTeX}{rm}{%
    \ltx@mbox{%
      \HOLOGO@MathSetup
      $%
        \textstyle
        \varepsilon_{%
          \kern-0.15em%
          \HoLogoFont@font{ExTeX}{sy}{X}%
        }%
      $%
    }%
    \HOLOGO@discretionary
    \kern-.15em%
    \hologo{TeX}%
  }%
}
%    \end{macrocode}
%    \end{macro}
%    \begin{macro}{\HoLogoHtml@ExTeX}
%    \begin{macrocode}
\def\HoLogoHtml@ExTeX#1{%
  \HoLogoCss@ExTeX
  \HoLogoFont@font{ExTeX}{rm}{%
    \HOLOGO@Span{ExTeX}{%
      \ltx@mbox{%
        \HOLOGO@MathSetup
        $\textstyle\varepsilon$%
        \HOLOGO@Span{X}{$\textstyle\chi$}%
        \hologo{TeX}%
      }%
    }%
  }%
}
%    \end{macrocode}
%    \end{macro}
%    \begin{macro}{\HoLogoBkm@ExTeX}
%    \begin{macrocode}
\def\HoLogoBkm@ExTeX#1{%
  \HOLOGO@PdfdocUnicode{#1{e}{E}x}{\textepsilon\textchi}%
  \hologo{TeX}%
}
%    \end{macrocode}
%    \end{macro}
%    \begin{macro}{\HoLogoCss@ExTeX}
%    \begin{macrocode}
\def\HoLogoCss@ExTeX{%
  \Css{%
    span.HoLogo-ExTeX{%
      font-family:serif;%
    }%
  }%
  \Css{%
    span.HoLogo-ExTeX span.HoLogo-TeX{%
      margin-left:-.15em;%
    }%
  }%
  \global\let\HoLogoCss@ExTeX\relax
}
%    \end{macrocode}
%    \end{macro}
%
% \subsubsection{\hologo{MiKTeX}}
%
%    \begin{macro}{\HoLogo@MiKTeX}
%    \begin{macrocode}
\def\HoLogo@MiKTeX#1{%
  \HOLOGO@mbox{MiK}%
  \HOLOGO@discretionary
  \hologo{TeX}%
}
%    \end{macrocode}
%    \end{macro}
%    \begin{macro}{\HoLogoHtml@MiKTeX}
%    \begin{macrocode}
\let\HoLogoHtml@MiKTeX\HoLogo@MiKTeX
%    \end{macrocode}
%    \end{macro}
%
% \subsubsection{\hologo{OzTeX} and friends}
%
%    Source: \hologo{OzTeX} FAQ \cite{OzTeX}:
%    \begin{quote}
%      |\def\OzTeX{O\kern-.03em z\kern-.15em\TeX}|\\
%      (There is no kerning in OzMF, OzMP and OzTtH.)
%    \end{quote}
%
%    \begin{macro}{\HoLogo@OzTeX}
%    \begin{macrocode}
\def\HoLogo@OzTeX#1{%
  O%
  \kern-.03em %
  z%
  \kern-.15em %
  \hologo{TeX}%
}
%    \end{macrocode}
%    \end{macro}
%    \begin{macro}{\HoLogoHtml@OzTeX}
%    \begin{macrocode}
\def\HoLogoHtml@OzTeX#1{%
  \HoLogoCss@OzTeX
  \HOLOGO@Span{OzTeX}{%
    O%
    \HOLOGO@Span{z}{z}%
    \hologo{TeX}%
  }%
}
%    \end{macrocode}
%    \end{macro}
%    \begin{macro}{\HoLogoCss@OzTeX}
%    \begin{macrocode}
\def\HoLogoCss@OzTeX{%
  \Css{%
    span.HoLogo-OzTeX span.HoLogo-z{%
      margin-left:-.03em;%
      margin-right:-.15em;%
    }%
  }%
  \global\let\HoLogoCss@OzTeX\relax
}
%    \end{macrocode}
%    \end{macro}
%
%    \begin{macro}{\HoLogo@OzMF}
%    \begin{macrocode}
\def\HoLogo@OzMF#1{%
  \HOLOGO@mbox{OzMF}%
}
%    \end{macrocode}
%    \end{macro}
%    \begin{macro}{\HoLogo@OzMP}
%    \begin{macrocode}
\def\HoLogo@OzMP#1{%
  \HOLOGO@mbox{OzMP}%
}
%    \end{macrocode}
%    \end{macro}
%    \begin{macro}{\HoLogo@OzTtH}
%    \begin{macrocode}
\def\HoLogo@OzTtH#1{%
  \HOLOGO@mbox{OzTtH}%
}
%    \end{macrocode}
%    \end{macro}
%
% \subsubsection{\hologo{PCTeX}}
%
%    \begin{macro}{\HoLogo@PCTeX}
%    \begin{macrocode}
\def\HoLogo@PCTeX#1{%
  \HOLOGO@mbox{PC}%
  \hologo{TeX}%
}
%    \end{macrocode}
%    \end{macro}
%    \begin{macro}{\HoLogoHtml@PCTeX}
%    \begin{macrocode}
\let\HoLogoHtml@PCTeX\HoLogo@PCTeX
%    \end{macrocode}
%    \end{macro}
%
% \subsubsection{\hologo{PiCTeX}}
%
%    The original definitions from \xfile{pictex.tex} \cite{PiCTeX}:
%\begin{quote}
%\begin{verbatim}
%\def\PiC{%
%  P%
%  \kern-.12em%
%  \lower.5ex\hbox{I}%
%  \kern-.075em%
%  C%
%}
%\def\PiCTeX{%
%  \PiC
%  \kern-.11em%
%  \TeX
%}
%\end{verbatim}
%\end{quote}
%
%    \begin{macro}{\HoLogo@PiC}
%    \begin{macrocode}
\def\HoLogo@PiC#1{%
  P%
  \kern-.12em%
  \lower.5ex\hbox{I}%
  \kern-.075em%
  C%
  \HOLOGO@SpaceFactor
}
%    \end{macrocode}
%    \end{macro}
%    \begin{macro}{\HoLogoHtml@PiC}
%    \begin{macrocode}
\def\HoLogoHtml@PiC#1{%
  \HoLogoCss@PiC
  \HOLOGO@Span{PiC}{%
    P%
    \HOLOGO@Span{i}{I}%
    C%
  }%
}
%    \end{macrocode}
%    \end{macro}
%    \begin{macro}{\HoLogoCss@PiC}
%    \begin{macrocode}
\def\HoLogoCss@PiC{%
  \Css{%
    span.HoLogo-PiC span.HoLogo-i{%
      position:relative;%
      top:.5ex;%
      margin-left:-.12em;%
      margin-right:-.075em;%
      text-decoration:none;%
    }%
  }%
  \global\let\HoLogoCss@PiC\relax
}
%    \end{macrocode}
%    \end{macro}
%
%    \begin{macro}{\HoLogo@PiCTeX}
%    \begin{macrocode}
\def\HoLogo@PiCTeX#1{%
  \hologo{PiC}%
  \HOLOGO@discretionary
  \kern-.11em%
  \hologo{TeX}%
}
%    \end{macrocode}
%    \end{macro}
%    \begin{macro}{\HoLogoHtml@PiCTeX}
%    \begin{macrocode}
\def\HoLogoHtml@PiCTeX#1{%
  \HoLogoCss@PiCTeX
  \HOLOGO@Span{PiCTeX}{%
    \hologo{PiC}%
    \hologo{TeX}%
  }%
}
%    \end{macrocode}
%    \end{macro}
%    \begin{macro}{\HoLogoCss@PiCTeX}
%    \begin{macrocode}
\def\HoLogoCss@PiCTeX{%
  \Css{%
    span.HoLogo-PiCTeX span.HoLogo-PiC{%
      margin-right:-.11em;%
    }%
  }%
  \global\let\HoLogoCss@PiCTeX\relax
}
%    \end{macrocode}
%    \end{macro}
%
% \subsubsection{\hologo{teTeX}}
%
%    \begin{macro}{\HoLogo@teTeX}
%    \begin{macrocode}
\def\HoLogo@teTeX#1{%
  \HOLOGO@mbox{#1{t}{T}e}%
  \HOLOGO@discretionary
  \hologo{TeX}%
}
%    \end{macrocode}
%    \end{macro}
%    \begin{macro}{\HoLogoCs@teTeX}
%    \begin{macrocode}
\def\HoLogoCs@teTeX#1{#1{t}{T}dfTeX}
%    \end{macrocode}
%    \end{macro}
%    \begin{macro}{\HoLogoBkm@teTeX}
%    \begin{macrocode}
\def\HoLogoBkm@teTeX#1{%
  #1{t}{T}e\hologo{TeX}%
}
%    \end{macrocode}
%    \end{macro}
%    \begin{macro}{\HoLogoHtml@teTeX}
%    \begin{macrocode}
\let\HoLogoHtml@teTeX\HoLogo@teTeX
%    \end{macrocode}
%    \end{macro}
%
% \subsubsection{\hologo{TeX4ht}}
%
%    \begin{macro}{\HoLogo@TeX4ht}
%    \begin{macrocode}
\expandafter\def\csname HoLogo@TeX4ht\endcsname#1{%
  \HOLOGO@mbox{\hologo{TeX}4ht}%
}
%    \end{macrocode}
%    \end{macro}
%    \begin{macro}{\HoLogoHtml@TeX4ht}
%    \begin{macrocode}
\expandafter
\let\csname HoLogoHtml@TeX4ht\expandafter\endcsname
\csname HoLogo@TeX4ht\endcsname
%    \end{macrocode}
%    \end{macro}
%
%
% \subsubsection{\hologo{SageTeX}}
%
%    \begin{macro}{\HoLogo@SageTeX}
%    \begin{macrocode}
\def\HoLogo@SageTeX#1{%
  \HOLOGO@mbox{Sage}%
  \HOLOGO@discretionary
  \HOLOGO@NegativeKerning{eT,oT,To}%
  \hologo{TeX}%
}
%    \end{macrocode}
%    \end{macro}
%    \begin{macro}{\HoLogoHtml@SageTeX}
%    \begin{macrocode}
\let\HoLogoHtml@SageTeX\HoLogo@SageTeX
%    \end{macrocode}
%    \end{macro}
%
% \subsection{\hologo{METAFONT} and friends}
%
%    \begin{macro}{\HoLogo@METAFONT}
%    \begin{macrocode}
\def\HoLogo@METAFONT#1{%
  \HoLogoFont@font{METAFONT}{logo}{%
    \HOLOGO@mbox{META}%
    \HOLOGO@discretionary
    \HOLOGO@mbox{FONT}%
  }%
}
%    \end{macrocode}
%    \end{macro}
%
%    \begin{macro}{\HoLogo@METAPOST}
%    \begin{macrocode}
\def\HoLogo@METAPOST#1{%
  \HoLogoFont@font{METAPOST}{logo}{%
    \HOLOGO@mbox{META}%
    \HOLOGO@discretionary
    \HOLOGO@mbox{POST}%
  }%
}
%    \end{macrocode}
%    \end{macro}
%
%    \begin{macro}{\HoLogo@MetaFun}
%    \begin{macrocode}
\def\HoLogo@MetaFun#1{%
  \HOLOGO@mbox{Meta}%
  \HOLOGO@discretionary
  \HOLOGO@mbox{Fun}%
}
%    \end{macrocode}
%    \end{macro}
%
%    \begin{macro}{\HoLogo@MetaPost}
%    \begin{macrocode}
\def\HoLogo@MetaPost#1{%
  \HOLOGO@mbox{Meta}%
  \HOLOGO@discretionary
  \HOLOGO@mbox{Post}%
}
%    \end{macrocode}
%    \end{macro}
%
% \subsection{Others}
%
% \subsubsection{\hologo{biber}}
%
%    \begin{macro}{\HoLogo@biber}
%    \begin{macrocode}
\def\HoLogo@biber#1{%
  \HOLOGO@mbox{#1{b}{B}i}%
  \HOLOGO@discretionary
  \HOLOGO@mbox{ber}%
}
%    \end{macrocode}
%    \end{macro}
%    \begin{macro}{\HoLogoCs@biber}
%    \begin{macrocode}
\def\HoLogoCs@biber#1{#1{b}{B}iber}
%    \end{macrocode}
%    \end{macro}
%    \begin{macro}{\HoLogoBkm@biber}
%    \begin{macrocode}
\def\HoLogoBkm@biber#1{%
  #1{b}{B}iber%
}
%    \end{macrocode}
%    \end{macro}
%    \begin{macro}{\HoLogoHtml@biber}
%    \begin{macrocode}
\let\HoLogoHtml@biber\HoLogo@biber
%    \end{macrocode}
%    \end{macro}
%
% \subsubsection{\hologo{KOMAScript}}
%
%    \begin{macro}{\HoLogo@KOMAScript}
%    The definition for \hologo{KOMAScript} is taken
%    from \hologo{KOMAScript} (\xfile{scrlogo.dtx}, reformatted) \cite{scrlogo}:
%\begin{quote}
%\begin{verbatim}
%\@ifundefined{KOMAScript}{%
%  \DeclareRobustCommand{\KOMAScript}{%
%    \textsf{%
%      K\kern.05em O\kern.05emM\kern.05em A%
%      \kern.1em-\kern.1em %
%      Script%
%    }%
%  }%
%}{}
%\end{verbatim}
%\end{quote}
%    \begin{macrocode}
\def\HoLogo@KOMAScript#1{%
  \HoLogoFont@font{KOMAScript}{sf}{%
    \HOLOGO@mbox{%
      K\kern.05em%
      O\kern.05em%
      M\kern.05em%
      A%
    }%
    \kern.1em%
    \HOLOGO@hyphen
    \kern.1em%
    \HOLOGO@mbox{Script}%
  }%
}
%    \end{macrocode}
%    \end{macro}
%    \begin{macro}{\HoLogoBkm@KOMAScript}
%    \begin{macrocode}
\def\HoLogoBkm@KOMAScript#1{%
  KOMA-Script%
}
%    \end{macrocode}
%    \end{macro}
%    \begin{macro}{\HoLogoHtml@KOMAScript}
%    \begin{macrocode}
\def\HoLogoHtml@KOMAScript#1{%
  \HoLogoCss@KOMAScript
  \HoLogoFont@font{KOMAScript}{sf}{%
    \HOLOGO@Span{KOMAScript}{%
      K%
      \HOLOGO@Span{O}{O}%
      M%
      \HOLOGO@Span{A}{A}%
      \HOLOGO@Span{hyphen}{-}%
      Script%
    }%
  }%
}
%    \end{macrocode}
%    \end{macro}
%    \begin{macro}{\HoLogoCss@KOMAScript}
%    \begin{macrocode}
\def\HoLogoCss@KOMAScript{%
  \Css{%
    span.HoLogo-KOMAScript{%
      font-family:sans-serif;%
    }%
  }%
  \Css{%
    span.HoLogo-KOMAScript span.HoLogo-O{%
      padding-left:.05em;%
      padding-right:.05em;%
    }%
  }%
  \Css{%
    span.HoLogo-KOMAScript span.HoLogo-A{%
      padding-left:.05em;%
    }%
  }%
  \Css{%
    span.HoLogo-KOMAScript span.HoLogo-hyphen{%
      padding-left:.1em;%
      padding-right:.1em;%
    }%
  }%
  \global\let\HoLogoCss@KOMAScript\relax
}
%    \end{macrocode}
%    \end{macro}
%
% \subsubsection{\hologo{LyX}}
%
%    \begin{macro}{\HoLogo@LyX}
%    The definition is taken from the documentation source files
%    of \hologo{LyX}, \xfile{Intro.lyx} \cite{LyX}:
%\begin{quote}
%\begin{verbatim}
%\def\LyX{%
%  \texorpdfstring{%
%    L\kern-.1667em\lower.25em\hbox{Y}\kern-.125emX\@%
%  }{%
%    LyX%
%  }%
%}
%\end{verbatim}
%\end{quote}
%    \begin{macrocode}
\def\HoLogo@LyX#1{%
  L%
  \kern-.1667em%
  \lower.25em\hbox{Y}%
  \kern-.125em%
  X%
  \HOLOGO@SpaceFactor
}
%    \end{macrocode}
%    \end{macro}
%    \begin{macro}{\HoLogoHtml@LyX}
%    \begin{macrocode}
\def\HoLogoHtml@LyX#1{%
  \HoLogoCss@LyX
  \HOLOGO@Span{LyX}{%
    L%
    \HOLOGO@Span{y}{Y}%
    X%
  }%
}
%    \end{macrocode}
%    \end{macro}
%    \begin{macro}{\HoLogoCss@LyX}
%    \begin{macrocode}
\def\HoLogoCss@LyX{%
  \Css{%
    span.HoLogo-LyX span.HoLogo-y{%
      position:relative;%
      top:.25em;%
      margin-left:-.1667em;%
      margin-right:-.125em;%
      text-decoration:none;%
    }%
  }%
  \global\let\HoLogoCss@LyX\relax
}
%    \end{macrocode}
%    \end{macro}
%
% \subsubsection{\hologo{NTS}}
%
%    \begin{macro}{\HoLogo@NTS}
%    Definition for \hologo{NTS} can be found in
%    package \xpackage{etex\textunderscore man} for the \hologo{eTeX} manual \cite{etexman}
%    and in package \xpackage{dtklogos} \cite{dtklogos}:
%\begin{quote}
%\begin{verbatim}
%\def\NTS{%
%  \leavevmode
%  \hbox{%
%    $%
%      \cal N%
%      \kern-0.35em%
%      \lower0.5ex\hbox{$\cal T$}%
%      \kern-0.2em%
%      S%
%    $%
%  }%
%}
%\end{verbatim}
%\end{quote}
%    \begin{macrocode}
\def\HoLogo@NTS#1{%
  \HoLogoFont@font{NTS}{sy}{%
    N\/%
    \kern-.35em%
    \lower.5ex\hbox{T\/}%
    \kern-.2em%
    S\/%
  }%
  \HOLOGO@SpaceFactor
}
%    \end{macrocode}
%    \end{macro}
%
% \subsubsection{\Hologo{TTH} (\hologo{TeX} to HTML translator)}
%
%    Source: \url{http://hutchinson.belmont.ma.us/tth/}
%    In the HTML source the second `T' is printed as subscript.
%\begin{quote}
%\begin{verbatim}
%T<sub>T</sub>H
%\end{verbatim}
%\end{quote}
%    \begin{macro}{\HoLogo@TTH}
%    \begin{macrocode}
\def\HoLogo@TTH#1{%
  \ltx@mbox{%
    T\HOLOGO@SubScript{T}H%
  }%
  \HOLOGO@SpaceFactor
}
%    \end{macrocode}
%    \end{macro}
%
%    \begin{macro}{\HoLogoHtml@TTH}
%    \begin{macrocode}
\def\HoLogoHtml@TTH#1{%
  T\HCode{<sub>}T\HCode{</sub>}H%
}
%    \end{macrocode}
%    \end{macro}
%
% \subsubsection{\Hologo{HanTheThanh}}
%
%    Partial source: Package \xpackage{dtklogos}.
%    The double accent is U+1EBF (latin small letter e with circumflex
%    and acute).
%    \begin{macro}{\HoLogo@HanTheThanh}
%    \begin{macrocode}
\def\HoLogo@HanTheThanh#1{%
  \ltx@mbox{H\`an}%
  \HOLOGO@space
  \ltx@mbox{%
    Th%
    \HOLOGO@IfCharExists{"1EBF}{%
      \char"1EBF\relax
    }{%
      \^e\hbox to 0pt{\hss\raise .5ex\hbox{\'{}}}%
    }%
  }%
  \HOLOGO@space
  \ltx@mbox{Th\`anh}%
}
%    \end{macrocode}
%    \end{macro}
%    \begin{macro}{\HoLogoBkm@HanTheThanh}
%    \begin{macrocode}
\def\HoLogoBkm@HanTheThanh#1{%
  H\`an %
  Th\HOLOGO@PdfdocUnicode{\^e}{\9036\277} %
  Th\`anh%
}
%    \end{macrocode}
%    \end{macro}
%    \begin{macro}{\HoLogoHtml@HanTheThanh}
%    \begin{macrocode}
\def\HoLogoHtml@HanTheThanh#1{%
  H\`an %
  Th\HCode{&\ltx@hashchar x1ebf;} %
  Th\`anh%
}
%    \end{macrocode}
%    \end{macro}
%
% \subsection{Driver detection}
%
%    \begin{macrocode}
\HOLOGO@IfExists\InputIfFileExists{%
  \InputIfFileExists{hologo.cfg}{}{}%
}{%
  \ltx@IfUndefined{pdf@filesize}{%
    \def\HOLOGO@InputIfExists{%
      \openin\HOLOGO@temp=hologo.cfg\relax
      \ifeof\HOLOGO@temp
        \closein\HOLOGO@temp
      \else
        \closein\HOLOGO@temp
        \begingroup
          \def\x{LaTeX2e}%
        \expandafter\endgroup
        \ifx\fmtname\x
          \input{hologo.cfg}%
        \else
          \input hologo.cfg\relax
        \fi
      \fi
    }%
    \ltx@IfUndefined{newread}{%
      \chardef\HOLOGO@temp=15 %
      \def\HOLOGO@CheckRead{%
        \ifeof\HOLOGO@temp
          \HOLOGO@InputIfExists
        \else
          \ifcase\HOLOGO@temp
            \@PackageWarningNoLine{hologo}{%
              Configuration file ignored, because\MessageBreak
              a free read register could not be found%
            }%
          \else
            \begingroup
              \count\ltx@cclv=\HOLOGO@temp
              \advance\ltx@cclv by \ltx@minusone
              \edef\x{\endgroup
                \chardef\noexpand\HOLOGO@temp=\the\count\ltx@cclv
                \relax
              }%
            \x
          \fi
        \fi
      }%
    }{%
      \csname newread\endcsname\HOLOGO@temp
      \HOLOGO@InputIfExists
    }%
  }{%
    \edef\HOLOGO@temp{\pdf@filesize{hologo.cfg}}%
    \ifx\HOLOGO@temp\ltx@empty
    \else
      \ifnum\HOLOGO@temp>0 %
        \begingroup
          \def\x{LaTeX2e}%
        \expandafter\endgroup
        \ifx\fmtname\x
          \input{hologo.cfg}%
        \else
          \input hologo.cfg\relax
        \fi
      \else
        \@PackageInfoNoLine{hologo}{%
          Empty configuration file `hologo.cfg' ignored%
        }%
      \fi
    \fi
  }%
}
%    \end{macrocode}
%
%    \begin{macrocode}
\def\HOLOGO@temp#1#2{%
  \kv@define@key{HoLogoDriver}{#1}[]{%
    \begingroup
      \def\HOLOGO@temp{##1}%
      \ltx@onelevel@sanitize\HOLOGO@temp
      \ifx\HOLOGO@temp\ltx@empty
      \else
        \@PackageError{hologo}{%
          Value (\HOLOGO@temp) not permitted for option `#1'%
        }%
        \@ehc
      \fi
    \endgroup
    \def\hologoDriver{#2}%
  }%
}%
\def\HOLOGO@@temp#1#2{%
  \ifx\kv@value\relax
    \HOLOGO@temp{#1}{#1}%
  \else
    \HOLOGO@temp{#1}{#2}%
  \fi
}%
\kv@parse@normalized{%
  pdftex,%
  luatex=pdftex,%
  dvipdfm,%
  dvipdfmx=dvipdfm,%
  dvips,%
  dvipsone=dvips,%
  xdvi=dvips,%
  xetex,%
  vtex,%
}\HOLOGO@@temp
%    \end{macrocode}
%
%    \begin{macrocode}
\kv@define@key{HoLogoDriver}{driverfallback}{%
  \def\HOLOGO@DriverFallback{#1}%
}
%    \end{macrocode}
%
%    \begin{macro}{\HOLOGO@DriverFallback}
%    \begin{macrocode}
\def\HOLOGO@DriverFallback{dvips}
%    \end{macrocode}
%    \end{macro}
%
%    \begin{macro}{\hologoDriverSetup}
%    \begin{macrocode}
\def\hologoDriverSetup{%
  \let\hologoDriver\ltx@undefined
  \HOLOGO@DriverSetup
}
%    \end{macrocode}
%    \end{macro}
%
%    \begin{macro}{\HOLOGO@DriverSetup}
%    \begin{macrocode}
\def\HOLOGO@DriverSetup#1{%
  \kvsetkeys{HoLogoDriver}{#1}%
  \HOLOGO@CheckDriver
  \ltx@ifundefined{hologoDriver}{%
    \begingroup
    \edef\x{\endgroup
      \noexpand\kvsetkeys{HoLogoDriver}{\HOLOGO@DriverFallback}%
    }\x
  }{}%
  \@PackageInfoNoLine{hologo}{Using driver `\hologoDriver'}%
}
%    \end{macrocode}
%    \end{macro}
%
%    \begin{macro}{\HOLOGO@CheckDriver}
%    \begin{macrocode}
\def\HOLOGO@CheckDriver{%
  \ifpdf
    \def\hologoDriver{pdftex}%
    \let\HOLOGO@pdfliteral\pdfliteral
    \ifluatex
      \ifx\pdfextension\@undefined\else
        \protected\def\pdfliteral{\pdfextension literal}%
        \let\HOLOGO@pdfliteral\pdfliteral
      \fi
      \ltx@IfUndefined{HOLOGO@pdfliteral}{%
        \ifnum\luatexversion<36 %
        \else
          \begingroup
            \let\HOLOGO@temp\endgroup
            \ifcase0%
                \directlua{%
                  if tex.enableprimitives then %
                    tex.enableprimitives('HOLOGO@', {'pdfliteral'})%
                  else %
                    tex.print('1')%
                  end%
                }%
                \ifx\HOLOGO@pdfliteral\@undefined 1\fi%
                \relax%
              \endgroup
              \let\HOLOGO@temp\relax
              \global\let\HOLOGO@pdfliteral\HOLOGO@pdfliteral
            \fi%
          \HOLOGO@temp
        \fi
      }{}%
    \fi
    \ltx@IfUndefined{HOLOGO@pdfliteral}{%
      \@PackageWarningNoLine{hologo}{%
        Cannot find \string\pdfliteral
      }%
    }{}%
  \else
    \ifxetex
      \def\hologoDriver{xetex}%
    \else
      \ifvtex
        \def\hologoDriver{vtex}%
      \fi
    \fi
  \fi
}
%    \end{macrocode}
%    \end{macro}
%
%    \begin{macro}{\HOLOGO@WarningUnsupportedDriver}
%    \begin{macrocode}
\def\HOLOGO@WarningUnsupportedDriver#1{%
  \@PackageWarningNoLine{hologo}{%
    Logo `#1' needs driver specific macros,\MessageBreak
    but driver `\hologoDriver' is not supported.\MessageBreak
    Use a different driver or\MessageBreak
    load package `graphics' or `pgf'%
  }%
}
%    \end{macrocode}
%    \end{macro}
%
% \subsubsection{Reflect box macros}
%
%    Skip driver part if not needed.
%    \begin{macrocode}
\ltx@IfUndefined{reflectbox}{}{%
  \ltx@IfUndefined{rotatebox}{}{%
    \HOLOGO@AtEnd
  }%
}
\ltx@IfUndefined{pgftext}{}{%
  \HOLOGO@AtEnd
}
\ltx@IfUndefined{psscalebox}{}{%
  \HOLOGO@AtEnd
}
%    \end{macrocode}
%
%    \begin{macrocode}
\def\HOLOGO@temp{LaTeX2e}
\ifx\fmtname\HOLOGO@temp
  \RequirePackage{kvoptions}[2011/06/30]%
  \ProcessKeyvalOptions{HoLogoDriver}%
\fi
\HOLOGO@DriverSetup{}
%    \end{macrocode}
%
%    \begin{macro}{\HOLOGO@ReflectBox}
%    \begin{macrocode}
\def\HOLOGO@ReflectBox#1{%
  \begingroup
    \setbox\ltx@zero\hbox{\begingroup#1\endgroup}%
    \setbox\ltx@two\hbox{%
      \kern\wd\ltx@zero
      \csname HOLOGO@ScaleBox@\hologoDriver\endcsname{-1}{1}{%
        \hbox to 0pt{\copy\ltx@zero\hss}%
      }%
    }%
    \wd\ltx@two=\wd\ltx@zero
    \box\ltx@two
  \endgroup
}
%    \end{macrocode}
%    \end{macro}
%
%    \begin{macro}{\HOLOGO@PointReflectBox}
%    \begin{macrocode}
\def\HOLOGO@PointReflectBox#1{%
  \begingroup
    \setbox\ltx@zero\hbox{\begingroup#1\endgroup}%
    \setbox\ltx@two\hbox{%
      \kern\wd\ltx@zero
      \raise\ht\ltx@zero\hbox{%
        \csname HOLOGO@ScaleBox@\hologoDriver\endcsname{-1}{-1}{%
          \hbox to 0pt{\copy\ltx@zero\hss}%
        }%
      }%
    }%
    \wd\ltx@two=\wd\ltx@zero
    \box\ltx@two
  \endgroup
}
%    \end{macrocode}
%    \end{macro}
%
%    We must define all variants because of dynamic driver setup.
%    \begin{macrocode}
\def\HOLOGO@temp#1#2{#2}
%    \end{macrocode}
%
%    \begin{macro}{\HOLOGO@ScaleBox@pdftex}
%    \begin{macrocode}
\HOLOGO@temp{pdftex}{%
  \def\HOLOGO@ScaleBox@pdftex#1#2#3{%
    \HOLOGO@pdfliteral{%
      q #1 0 0 #2 0 0 cm%
    }%
    #3%
    \HOLOGO@pdfliteral{%
      Q%
    }%
  }%
}
%    \end{macrocode}
%    \end{macro}
%    \begin{macro}{\HOLOGO@ScaleBox@dvips}
%    \begin{macrocode}
\HOLOGO@temp{dvips}{%
  \def\HOLOGO@ScaleBox@dvips#1#2#3{%
    \special{ps:%
      gsave %
      currentpoint %
      currentpoint translate %
      #1 #2 scale %
      neg exch neg exch translate%
    }%
    #3%
    \special{ps:%
      currentpoint %
      grestore %
      moveto%
    }%
  }%
}
%    \end{macrocode}
%    \end{macro}
%    \begin{macro}{\HOLOGO@ScaleBox@dvipdfm}
%    \begin{macrocode}
\HOLOGO@temp{dvipdfm}{%
  \let\HOLOGO@ScaleBox@dvipdfm\HOLOGO@ScaleBox@dvips
}
%    \end{macrocode}
%    \end{macro}
%    Since \hologo{XeTeX} v0.6.
%    \begin{macro}{\HOLOGO@ScaleBox@xetex}
%    \begin{macrocode}
\HOLOGO@temp{xetex}{%
  \def\HOLOGO@ScaleBox@xetex#1#2#3{%
    \special{x:gsave}%
    \special{x:scale #1 #2}%
    #3%
    \special{x:grestore}%
  }%
}
%    \end{macrocode}
%    \end{macro}
%    \begin{macro}{\HOLOGO@ScaleBox@vtex}
%    \begin{macrocode}
\HOLOGO@temp{vtex}{%
  \def\HOLOGO@ScaleBox@vtex#1#2#3{%
    \special{r(#1,0,0,#2,0,0}%
    #3%
    \special{r)}%
  }%
}
%    \end{macrocode}
%    \end{macro}
%
%    \begin{macrocode}
\HOLOGO@AtEnd%
%</package>
%    \end{macrocode}
%
% \section{Test}
%
% \subsection{Catcode checks for loading}
%
%    \begin{macrocode}
%<*test1>
%    \end{macrocode}
%    \begin{macrocode}
\catcode`\{=1 %
\catcode`\}=2 %
\catcode`\#=6 %
\catcode`\@=11 %
\expandafter\ifx\csname count@\endcsname\relax
  \countdef\count@=255 %
\fi
\expandafter\ifx\csname @gobble\endcsname\relax
  \long\def\@gobble#1{}%
\fi
\expandafter\ifx\csname @firstofone\endcsname\relax
  \long\def\@firstofone#1{#1}%
\fi
\expandafter\ifx\csname loop\endcsname\relax
  \expandafter\@firstofone
\else
  \expandafter\@gobble
\fi
{%
  \def\loop#1\repeat{%
    \def\body{#1}%
    \iterate
  }%
  \def\iterate{%
    \body
      \let\next\iterate
    \else
      \let\next\relax
    \fi
    \next
  }%
  \let\repeat=\fi
}%
\def\RestoreCatcodes{}
\count@=0 %
\loop
  \edef\RestoreCatcodes{%
    \RestoreCatcodes
    \catcode\the\count@=\the\catcode\count@\relax
  }%
\ifnum\count@<255 %
  \advance\count@ 1 %
\repeat

\def\RangeCatcodeInvalid#1#2{%
  \count@=#1\relax
  \loop
    \catcode\count@=15 %
  \ifnum\count@<#2\relax
    \advance\count@ 1 %
  \repeat
}
\def\RangeCatcodeCheck#1#2#3{%
  \count@=#1\relax
  \loop
    \ifnum#3=\catcode\count@
    \else
      \errmessage{%
        Character \the\count@\space
        with wrong catcode \the\catcode\count@\space
        instead of \number#3%
      }%
    \fi
  \ifnum\count@<#2\relax
    \advance\count@ 1 %
  \repeat
}
\def\space{ }
\expandafter\ifx\csname LoadCommand\endcsname\relax
  \def\LoadCommand{\input hologo.sty\relax}%
\fi
\def\Test{%
  \RangeCatcodeInvalid{0}{47}%
  \RangeCatcodeInvalid{58}{64}%
  \RangeCatcodeInvalid{91}{96}%
  \RangeCatcodeInvalid{123}{255}%
  \catcode`\@=12 %
  \catcode`\\=0 %
  \catcode`\%=14 %
  \LoadCommand
  \RangeCatcodeCheck{0}{36}{15}%
  \RangeCatcodeCheck{37}{37}{14}%
  \RangeCatcodeCheck{38}{47}{15}%
  \RangeCatcodeCheck{48}{57}{12}%
  \RangeCatcodeCheck{58}{63}{15}%
  \RangeCatcodeCheck{64}{64}{12}%
  \RangeCatcodeCheck{65}{90}{11}%
  \RangeCatcodeCheck{91}{91}{15}%
  \RangeCatcodeCheck{92}{92}{0}%
  \RangeCatcodeCheck{93}{96}{15}%
  \RangeCatcodeCheck{97}{122}{11}%
  \RangeCatcodeCheck{123}{255}{15}%
  \RestoreCatcodes
}
\Test
\csname @@end\endcsname
\end
%    \end{macrocode}
%    \begin{macrocode}
%</test1>
%    \end{macrocode}
%
% \subsection{Spacefactor}
%
%    The space factor must be 1000 after a logo. If it is greater 1000
%    then the following space is a space after a sentence closing point.
%    If the space factor is smaller 1000 then an immediate following
%    dot is interpreted as abbreviation, not sentence closing point.
%
%    \begin{macrocode}
%<*test-spacefactor>
\NeedsTeXFormat{LaTeX2e}
\documentclass{article}
\usepackage{hologo}[2016/05/12]
\usepackage{kvsetkeys}
\usepackage{qstest}
\IncludeTests{*}
\LogTests{log}{*}{*}
\begin{document}
\begin{qstest}{spacefactor}{spacefactor}
\newcommand*{\Test}[1]{%
  \sbox0{%
    \hologo{#1}%
    \Expect*{1000 (#1)}*{\the\spacefactor\space(#1)}%
  }%
}%
\makeatletter
\def\TestList{}
\def\hologoEntry#1#2#3{%
  \edef\TestList{%
    \ifx\TestList\@empty
    \else
      \TestList,%
    \fi
    #1%
    \ifx\\#2\\%
    \else
      ={variant=#2}%
    \fi
  }%
}
\hologoList
\expandafter\kv@parse@normalized\expandafter{%
  \TestList
}{%
  \begingroup
    \let\@logo=\kv@key
    \ifx\kv@value\relax
    \else
      \expandafter\hologoLogoSetup\expandafter\@logo\expandafter{%
        \kv@value
      }%
    \fi
    \Test\@logo
  \endgroup
  \@gobbletwo
}
\end{qstest}
\end{document}
%</test-spacefactor>
%    \end{macrocode}
%
% \subsection{Complete list}
%
%    \begin{macrocode}
%<*test-list>
\NeedsTeXFormat{LaTeX2e}
\documentclass[12pt,a4paper]{article}
\usepackage{hologo}[2016/05/12]
\usepackage[T1]{fontenc}
\usepackage{lmodern}
\usepackage{parskip}
\usepackage[unicode]{hyperref}[2011/09/28]
\usepackage{bookmark}[2011/09/19]
\bookmarksetup{%
  numbered,%
  open,%
  openlevel=2,%
}
\renewcommand*{\contentsname}{List of logos}
\begin{document}
\tableofcontents
\def\TestFont#1#2#3#4#5#6{%
  \begingroup
    \usefont{#3}{#4}{#5}{#6}%
    \HologoVariant{#1}{#2}/\hologoVariant{#1}{#2}%
    \quad
    \begingroup\scriptsize\hologoVariant{#1}{#2}\endgroup
    \quad
  \endgroup
  (#3/#4/#5/#6)%
  \par
}
\makeatletter
\def\hologoEntry#1#2#3{%
  \section{%
    \HologoVariant{#1}{#2}/\hologoVariant{#1}{#2} %
    {[#1\ifx\\#2\\\else\space(#2)\fi]}% hash-ok
  }% braces around [] because of bug in tex4ht
  \begingroup
    \hypersetup{unicode=false}%
    \bookmark[%
      dest=\@currentHref,%
      rellevel=1,%
      keeplevel,%
    ]{%
      \HologoVariant{#1}{#2}/\hologoVariant{#1}{#2} %
      (PDFDocEncoding)%
    }%
  \endgroup
  \TestFont{#1}{#2}{OT1}{cmr}{m}{n}%
  \TestFont{#1}{#2}{OT1}{cmss}{m}{n}%
  \TestFont{#1}{#2}{OT1}{cmr}{b}{n}%
  \TestFont{#1}{#2}{OT1}{cmr}{m}{it}%
  \TestFont{#1}{#2}{OT1}{cmtt}{m}{n}%
  \TestFont{#1}{#2}{T1}{lmr}{m}{n}%
  \TestFont{#1}{#2}{T1}{lmss}{m}{n}%
  \TestFont{#1}{#2}{T1}{lmr}{b}{n}%
  \TestFont{#1}{#2}{T1}{lmr}{m}{it}%
  \TestFont{#1}{#2}{T1}{lmtt}{m}{n}%
  \TestFont{#1}{#2}{T1}{lmvtt}{m}{n}%
  \TestFont{#1}{#2}{T1}{qtm}{m}{n}%
  \TestFont{#1}{#2}{T1}{qhv}{m}{n}%
  \TestFont{#1}{#2}{T1}{qtm}{b}{n}%
  \TestFont{#1}{#2}{T1}{qtm}{m}{it}%
  \TestFont{#1}{#2}{T1}{qcr}{m}{n}%
  \newpage
}
\makeatother
\hologoList
\end{document}
%</test-list>
%    \end{macrocode}
%
% \section{Installation}
%
% \subsection{Download}
%
% \paragraph{Package.} This package is available on
% CTAN\footnote{\url{ftp://ftp.ctan.org/tex-archive/}}:
% \begin{description}
% \item[\CTAN{macros/latex/contrib/oberdiek/hologo.dtx}] The source file.
% \item[\CTAN{macros/latex/contrib/oberdiek/hologo.pdf}] Documentation.
% \end{description}
%
%
% \paragraph{Bundle.} All the packages of the bundle `oberdiek'
% are also available in a TDS compliant ZIP archive. There
% the packages are already unpacked and the documentation files
% are generated. The files and directories obey the TDS standard.
% \begin{description}
% \item[\CTAN{install/macros/latex/contrib/oberdiek.tds.zip}]
% \end{description}
% \emph{TDS} refers to the standard ``A Directory Structure
% for \TeX\ Files'' (\CTAN{tds/tds.pdf}). Directories
% with \xfile{texmf} in their name are usually organized this way.
%
% \subsection{Bundle installation}
%
% \paragraph{Unpacking.} Unpack the \xfile{oberdiek.tds.zip} in the
% TDS tree (also known as \xfile{texmf} tree) of your choice.
% Example (linux):
% \begin{quote}
%   |unzip oberdiek.tds.zip -d ~/texmf|
% \end{quote}
%
% \paragraph{Script installation.}
% Check the directory \xfile{TDS:scripts/oberdiek/} for
% scripts that need further installation steps.
% Package \xpackage{attachfile2} comes with the Perl script
% \xfile{pdfatfi.pl} that should be installed in such a way
% that it can be called as \texttt{pdfatfi}.
% Example (linux):
% \begin{quote}
%   |chmod +x scripts/oberdiek/pdfatfi.pl|\\
%   |cp scripts/oberdiek/pdfatfi.pl /usr/local/bin/|
% \end{quote}
%
% \subsection{Package installation}
%
% \paragraph{Unpacking.} The \xfile{.dtx} file is a self-extracting
% \docstrip\ archive. The files are extracted by running the
% \xfile{.dtx} through \plainTeX:
% \begin{quote}
%   \verb|tex hologo.dtx|
% \end{quote}
%
% \paragraph{TDS.} Now the different files must be moved into
% the different directories in your installation TDS tree
% (also known as \xfile{texmf} tree):
% \begin{quote}
% \def\t{^^A
% \begin{tabular}{@{}>{\ttfamily}l@{ $\rightarrow$ }>{\ttfamily}l@{}}
%   hologo.sty & tex/generic/oberdiek/hologo.sty\\
%   hologo.pdf & doc/latex/oberdiek/hologo.pdf\\
%   example/hologo-example.tex & doc/latex/oberdiek/example/hologo-example.tex\\
%   test/hologo-test1.tex & doc/latex/oberdiek/test/hologo-test1.tex\\
%   test/hologo-test-spacefactor.tex & doc/latex/oberdiek/test/hologo-test-spacefactor.tex\\
%   test/hologo-test-list.tex & doc/latex/oberdiek/test/hologo-test-list.tex\\
%   hologo.dtx & source/latex/oberdiek/hologo.dtx\\
% \end{tabular}^^A
% }^^A
% \sbox0{\t}^^A
% \ifdim\wd0>\linewidth
%   \begingroup
%     \advance\linewidth by\leftmargin
%     \advance\linewidth by\rightmargin
%   \edef\x{\endgroup
%     \def\noexpand\lw{\the\linewidth}^^A
%   }\x
%   \def\lwbox{^^A
%     \leavevmode
%     \hbox to \linewidth{^^A
%       \kern-\leftmargin\relax
%       \hss
%       \usebox0
%       \hss
%       \kern-\rightmargin\relax
%     }^^A
%   }^^A
%   \ifdim\wd0>\lw
%     \sbox0{\small\t}^^A
%     \ifdim\wd0>\linewidth
%       \ifdim\wd0>\lw
%         \sbox0{\footnotesize\t}^^A
%         \ifdim\wd0>\linewidth
%           \ifdim\wd0>\lw
%             \sbox0{\scriptsize\t}^^A
%             \ifdim\wd0>\linewidth
%               \ifdim\wd0>\lw
%                 \sbox0{\tiny\t}^^A
%                 \ifdim\wd0>\linewidth
%                   \lwbox
%                 \else
%                   \usebox0
%                 \fi
%               \else
%                 \lwbox
%               \fi
%             \else
%               \usebox0
%             \fi
%           \else
%             \lwbox
%           \fi
%         \else
%           \usebox0
%         \fi
%       \else
%         \lwbox
%       \fi
%     \else
%       \usebox0
%     \fi
%   \else
%     \lwbox
%   \fi
% \else
%   \usebox0
% \fi
% \end{quote}
% If you have a \xfile{docstrip.cfg} that configures and enables \docstrip's
% TDS installing feature, then some files can already be in the right
% place, see the documentation of \docstrip.
%
% \subsection{Refresh file name databases}
%
% If your \TeX~distribution
% (\teTeX, \mikTeX, \dots) relies on file name databases, you must refresh
% these. For example, \teTeX\ users run \verb|texhash| or
% \verb|mktexlsr|.
%
% \subsection{Some details for the interested}
%
% \paragraph{Attached source.}
%
% The PDF documentation on CTAN also includes the
% \xfile{.dtx} source file. It can be extracted by
% AcrobatReader 6 or higher. Another option is \textsf{pdftk},
% e.g. unpack the file into the current directory:
% \begin{quote}
%   \verb|pdftk hologo.pdf unpack_files output .|
% \end{quote}
%
% \paragraph{Unpacking with \LaTeX.}
% The \xfile{.dtx} chooses its action depending on the format:
% \begin{description}
% \item[\plainTeX:] Run \docstrip\ and extract the files.
% \item[\LaTeX:] Generate the documentation.
% \end{description}
% If you insist on using \LaTeX\ for \docstrip\ (really,
% \docstrip\ does not need \LaTeX), then inform the autodetect routine
% about your intention:
% \begin{quote}
%   \verb|latex \let\install=y\input{hologo.dtx}|
% \end{quote}
% Do not forget to quote the argument according to the demands
% of your shell.
%
% \paragraph{Generating the documentation.}
% You can use both the \xfile{.dtx} or the \xfile{.drv} to generate
% the documentation. The process can be configured by the
% configuration file \xfile{ltxdoc.cfg}. For instance, put this
% line into this file, if you want to have A4 as paper format:
% \begin{quote}
%   \verb|\PassOptionsToClass{a4paper}{article}|
% \end{quote}
% An example follows how to generate the
% documentation with pdf\LaTeX:
% \begin{quote}
%\begin{verbatim}
%pdflatex hologo.dtx
%makeindex -s gind.ist hologo.idx
%pdflatex hologo.dtx
%makeindex -s gind.ist hologo.idx
%pdflatex hologo.dtx
%\end{verbatim}
% \end{quote}
%
% \section{Catalogue}
%
% The following XML file can be used as source for the
% \href{http://mirror.ctan.org/help/Catalogue/catalogue.html}{\TeX\ Catalogue}.
% The elements \texttt{caption} and \texttt{description} are imported
% from the original XML file from the Catalogue.
% The name of the XML file in the Catalogue is \xfile{hologo.xml}.
%    \begin{macrocode}
%<*catalogue>
<?xml version='1.0' encoding='us-ascii'?>
<!DOCTYPE entry SYSTEM 'catalogue.dtd'>
<entry datestamp='$Date$' modifier='$Author$' id='hologo'>
  <name>hologo</name>
  <caption>A collection of logos with bookmark support.</caption>
  <authorref id='auth:oberdiek'/>
  <copyright owner='Heiko Oberdiek' year='2010-2012'/>
  <license type='lppl1.3'/>
  <version number='1.10'/>
  <description>
    The package defines a single command <tt>\hologo</tt>, whose
    argument is the usual case-confused ASCII version of the logo.
    The command is bookmark-enabled, so that every logo becomes
    available in bookmarks without further work.
    <p/>
    The package is part of the <xref refid='oberdiek'>oberdiek</xref>
    bundle.
  </description>
  <documentation details='Package documentation'
      href='ctan:/macros/latex/contrib/oberdiek/hologo.pdf'/>
  <ctan file='true' path='/macros/latex/contrib/oberdiek/hologo.dtx'/>
  <miktex location='oberdiek'/>
  <texlive location='oberdiek'/>
  <install path='/macros/latex/contrib/oberdiek/oberdiek.tds.zip'/>
</entry>
%</catalogue>
%    \end{macrocode}
%
% \begin{thebibliography}{9}
% \raggedright
%
% \bibitem{btxdoc}
% Oren Patashnik,
% \textit{\hologo{BibTeX}ing},
% 1988-02-08.\\
% \CTAN{biblio/bibtex/base/}
%
% \bibitem{dtklogos}
% Gerd Neugebauer, DANTE,
% \textit{Package \xpackage{dtklogos}},
% 2011-04-25.\\
% \CTAN{usergrps/dante/dtk/dtklogos.sty}
%
% \bibitem{etexman}
% The \hologo{NTS} Team,
% \textit{The \hologo{eTeX} manual},
% 1998-02.\\
% \CTAN{systems/e-tex/v2/doc/}
%
% \bibitem{ExTeX-FAQ}
% The \hologo{ExTeX} group,
% \textit{\hologo{ExTeX}: FAQ -- How is \hologo{ExTeX} typeset?},
% 2007-04-14.\\
% \url{http://www.extex.org/documentation/faq.html}
%
% \bibitem{LyX}
% %@MISC{ LyX,
% %  title = {{LyX 2.0.0 -- The Document Processor [Computer software and manual]}},
% %  author = {{The LyX Team}},
% %  howpublished = {Internet: http://www.lyx.org},
% %  year = {2011-05-08},
% %  note = {Retrieved May 10, 2011, from http://www.lyx.org},
% %  url = {http://www.lyx.org/}
% %}
% The \hologo{LyX} Team,
% \textit{\hologo{LyX} -- The Document Processor},
% 2011-05-08.\\
% \url{http://www.lyx.org/}
%
% \bibitem{OzTeX}
% Andrew Trevorrow,
% \hologo{OzTeX} FAQ: What is the correct way to typeset ``\hologo{OzTeX}''?,
% 2011-09-15 (visited).
% \url{http://www.trevorrow.com/oztex/ozfaq.html#oztex-logo}
%
% \bibitem{PiCTeX}
% Michael Wichura,
% \textit{The \hologo{PiCTeX} macro package},
% 1987-09-21.
% \CTAN{graphics/pictex/}
%
% \bibitem{scrlogo}
% Markus Kohm,
% \textit{\hologo{KOMAScript} Datei \xfile{scrlogo.dtx}},
% 2009-01-30.\\
% \CTAN{install/macros/latex/contrib/komascript.tds.zip}
%
% \end{thebibliography}
%
% \begin{History}
%   \begin{Version}{2010/04/08 v1.0}
%   \item
%     The first version.
%   \end{Version}
%   \begin{Version}{2010/04/16 v1.1}
%   \item
%     \cs{Hologo} added for support of logos at start of a sentence.
%   \item
%     \cs{hologoSetup} and \cs{hologoLogoSetup} added.
%   \item
%     Options \xoption{break}, \xoption{hyphenbreak}, \xoption{spacebreak}
%     added.
%   \item
%     Variant support added by option \xoption{variant}.
%   \end{Version}
%   \begin{Version}{2010/04/24 v1.2}
%   \item
%     \hologo{LaTeX3} added.
%   \item
%     \hologo{VTeX} added.
%   \end{Version}
%   \begin{Version}{2010/11/21 v1.3}
%   \item
%     \hologo{iniTeX}, \hologo{virTeX} added.
%   \end{Version}
%   \begin{Version}{2011/03/25 v1.4}
%   \item
%     \hologo{ConTeXt} with variants added.
%   \item
%     Option \xoption{discretionarybreak} added as refinement for
%     option \xoption{break}.
%   \end{Version}
%   \begin{Version}{2011/04/21 v1.5}
%   \item
%     Wrong TDS directory for test files fixed.
%   \end{Version}
%   \begin{Version}{2011/10/01 v1.6}
%   \item
%     Support for package \xpackage{tex4ht} added.
%   \item
%     Support for \cs{csname} added if \cs{ifincsname} is available.
%   \item
%     New logos:
%     \hologo{(La)TeX},
%     \hologo{biber},
%     \hologo{BibTeX} (\xoption{sc}, \xoption{sf}),
%     \hologo{emTeX},
%     \hologo{ExTeX},
%     \hologo{KOMAScript},
%     \hologo{La},
%     \hologo{LyX},
%     \hologo{MiKTeX},
%     \hologo{NTS},
%     \hologo{OzMF},
%     \hologo{OzMP},
%     \hologo{OzTeX},
%     \hologo{OzTtH},
%     \hologo{PCTeX},
%     \hologo{PiC},
%     \hologo{PiCTeX},
%     \hologo{METAFONT},
%     \hologo{MetaFun},
%     \hologo{METAPOST},
%     \hologo{MetaPost},
%     \hologo{SLiTeX} (\xoption{lift}, \xoption{narrow}, \xoption{simple}),
%     \hologo{SliTeX} (\xoption{narrow}, \xoption{simple}, \xoption{lift}),
%     \hologo{teTeX}.
%   \item
%     Fixes:
%     \hologo{iniTeX},
%     \hologo{pdfLaTeX},
%     \hologo{pdfTeX},
%     \hologo{virTeX}.
%   \item
%     \cs{hologoFontSetup} and \cs{hologoLogoFontSetup} added.
%   \item
%     \cs{hologoVariant} and \cs{HologoVariant} added.
%   \end{Version}
%   \begin{Version}{2011/11/22 v1.7}
%   \item
%     New logos:
%     \hologo{BibTeX8},
%     \hologo{LaTeXML},
%     \hologo{SageTeX},
%     \hologo{TeX4ht},
%     \hologo{TTH}.
%   \item
%     \hologo{Xe} and friends: Driver stuff fixed.
%   \item
%     \hologo{Xe} and friends: Support for italic added.
%   \item
%     \hologo{Xe} and friends: Package support for \xpackage{pgf}
%     and \xpackage{pstricks} added.
%   \end{Version}
%   \begin{Version}{2011/11/29 v1.8}
%   \item
%     New logos:
%     \hologo{HanTheThanh}.
%   \end{Version}
%   \begin{Version}{2011/12/21 v1.9}
%   \item
%     Patch for package \xpackage{ifxetex} added for the case that
%     \cs{newif} is undefined in \hologo{iniTeX}.
%   \item
%     Some fixes for \hologo{iniTeX}.
%   \end{Version}
%   \begin{Version}{2012/04/26 v1.10}
%   \item
%     Fix in bookmark version of logo ``\hologo{HanTheThanh}''.
%   \end{Version}
%   \begin{Version}{2016/05/12 v1.11}
%   \item
%     Update HOLOGO@IfCharExists (previously in texlive)
%   \item define pdfliteral in current luatex.
%   \end{Version}
% \end{History}
%
% \PrintIndex
%
% \Finale
\endinput
%
        \else
          \input hologo.cfg\relax
        \fi
      \fi
    }%
    \ltx@IfUndefined{newread}{%
      \chardef\HOLOGO@temp=15 %
      \def\HOLOGO@CheckRead{%
        \ifeof\HOLOGO@temp
          \HOLOGO@InputIfExists
        \else
          \ifcase\HOLOGO@temp
            \@PackageWarningNoLine{hologo}{%
              Configuration file ignored, because\MessageBreak
              a free read register could not be found%
            }%
          \else
            \begingroup
              \count\ltx@cclv=\HOLOGO@temp
              \advance\ltx@cclv by \ltx@minusone
              \edef\x{\endgroup
                \chardef\noexpand\HOLOGO@temp=\the\count\ltx@cclv
                \relax
              }%
            \x
          \fi
        \fi
      }%
    }{%
      \csname newread\endcsname\HOLOGO@temp
      \HOLOGO@InputIfExists
    }%
  }{%
    \edef\HOLOGO@temp{\pdf@filesize{hologo.cfg}}%
    \ifx\HOLOGO@temp\ltx@empty
    \else
      \ifnum\HOLOGO@temp>0 %
        \begingroup
          \def\x{LaTeX2e}%
        \expandafter\endgroup
        \ifx\fmtname\x
          % \iffalse meta-comment
%
% File: hologo.dtx
% Version: 2016/05/12 v1.11
% Info: A logo collection with bookmark support
%
% Copyright (C) 2010-2012 by
%    Heiko Oberdiek <heiko.oberdiek at googlemail.com>
%
% This work may be distributed and/or modified under the
% conditions of the LaTeX Project Public License, either
% version 1.3c of this license or (at your option) any later
% version. This version of this license is in
%    http://www.latex-project.org/lppl/lppl-1-3c.txt
% and the latest version of this license is in
%    http://www.latex-project.org/lppl.txt
% and version 1.3 or later is part of all distributions of
% LaTeX version 2005/12/01 or later.
%
% This work has the LPPL maintenance status "maintained".
%
% This Current Maintainer of this work is Heiko Oberdiek.
%
% The Base Interpreter refers to any `TeX-Format',
% because some files are installed in TDS:tex/generic//.
%
% This work consists of the main source file hologo.dtx
% and the derived files
%    hologo.sty, hologo.pdf, hologo.ins, hologo.drv, hologo-example.tex,
%    hologo-test1.tex, hologo-test-spacefactor.tex,
%    hologo-test-list.tex.
%
% Distribution:
%    CTAN:macros/latex/contrib/oberdiek/hologo.dtx
%    CTAN:macros/latex/contrib/oberdiek/hologo.pdf
%
% Unpacking:
%    (a) If hologo.ins is present:
%           tex hologo.ins
%    (b) Without hologo.ins:
%           tex hologo.dtx
%    (c) If you insist on using LaTeX
%           latex \let\install=y\input{hologo.dtx}
%        (quote the arguments according to the demands of your shell)
%
% Documentation:
%    (a) If hologo.drv is present:
%           latex hologo.drv
%    (b) Without hologo.drv:
%           latex hologo.dtx; ...
%    The class ltxdoc loads the configuration file ltxdoc.cfg
%    if available. Here you can specify further options, e.g.
%    use A4 as paper format:
%       \PassOptionsToClass{a4paper}{article}
%
%    Programm calls to get the documentation (example):
%       pdflatex hologo.dtx
%       makeindex -s gind.ist hologo.idx
%       pdflatex hologo.dtx
%       makeindex -s gind.ist hologo.idx
%       pdflatex hologo.dtx
%
% Installation:
%    TDS:tex/generic/oberdiek/hologo.sty
%    TDS:doc/latex/oberdiek/hologo.pdf
%    TDS:doc/latex/oberdiek/example/hologo-example.tex
%    TDS:doc/latex/oberdiek/test/hologo-test1.tex
%    TDS:doc/latex/oberdiek/test/hologo-test-spacefactor.tex
%    TDS:doc/latex/oberdiek/test/hologo-test-list.tex
%    TDS:source/latex/oberdiek/hologo.dtx
%
%<*ignore>
\begingroup
  \catcode123=1 %
  \catcode125=2 %
  \def\x{LaTeX2e}%
\expandafter\endgroup
\ifcase 0\ifx\install y1\fi\expandafter
         \ifx\csname processbatchFile\endcsname\relax\else1\fi
         \ifx\fmtname\x\else 1\fi\relax
\else\csname fi\endcsname
%</ignore>
%<*install>
\input docstrip.tex
\Msg{************************************************************************}
\Msg{* Installation}
\Msg{* Package: hologo 2016/05/12 v1.11 A logo collection with bookmark support (HO)}
\Msg{************************************************************************}

\keepsilent
\askforoverwritefalse

\let\MetaPrefix\relax
\preamble

This is a generated file.

Project: hologo
Version: 2016/05/12 v1.11

Copyright (C) 2010-2012 by
   Heiko Oberdiek <heiko.oberdiek at googlemail.com>

This work may be distributed and/or modified under the
conditions of the LaTeX Project Public License, either
version 1.3c of this license or (at your option) any later
version. This version of this license is in
   http://www.latex-project.org/lppl/lppl-1-3c.txt
and the latest version of this license is in
   http://www.latex-project.org/lppl.txt
and version 1.3 or later is part of all distributions of
LaTeX version 2005/12/01 or later.

This work has the LPPL maintenance status "maintained".

This Current Maintainer of this work is Heiko Oberdiek.

The Base Interpreter refers to any `TeX-Format',
because some files are installed in TDS:tex/generic//.

This work consists of the main source file hologo.dtx
and the derived files
   hologo.sty, hologo.pdf, hologo.ins, hologo.drv, hologo-example.tex,
   hologo-test1.tex, hologo-test-spacefactor.tex,
   hologo-test-list.tex.

\endpreamble
\let\MetaPrefix\DoubleperCent

\generate{%
  \file{hologo.ins}{\from{hologo.dtx}{install}}%
  \file{hologo.drv}{\from{hologo.dtx}{driver}}%
  \usedir{tex/generic/oberdiek}%
  \file{hologo.sty}{\from{hologo.dtx}{package}}%
  \usedir{doc/latex/oberdiek/example}%
  \file{hologo-example.tex}{\from{hologo.dtx}{example}}%
  \usedir{doc/latex/oberdiek/test}%
  \file{hologo-test1.tex}{\from{hologo.dtx}{test1}}%
  \file{hologo-test-spacefactor.tex}{\from{hologo.dtx}{test-spacefactor}}%
  \file{hologo-test-list.tex}{\from{hologo.dtx}{test-list}}%
  \nopreamble
  \nopostamble
  \usedir{source/latex/oberdiek/catalogue}%
  \file{hologo.xml}{\from{hologo.dtx}{catalogue}}%
}

\catcode32=13\relax% active space
\let =\space%
\Msg{************************************************************************}
\Msg{*}
\Msg{* To finish the installation you have to move the following}
\Msg{* file into a directory searched by TeX:}
\Msg{*}
\Msg{*     hologo.sty}
\Msg{*}
\Msg{* To produce the documentation run the file `hologo.drv'}
\Msg{* through LaTeX.}
\Msg{*}
\Msg{* Happy TeXing!}
\Msg{*}
\Msg{************************************************************************}

\endbatchfile
%</install>
%<*ignore>
\fi
%</ignore>
%<*driver>
\NeedsTeXFormat{LaTeX2e}
\ProvidesFile{hologo.drv}%
  [2016/05/12 v1.11 A logo collection with bookmark support (HO)]%
\documentclass{ltxdoc}
\usepackage{holtxdoc}[2011/11/22]
\usepackage{hologo}[2016/05/12]
\usepackage{longtable}
\usepackage{array}
\usepackage{paralist}
%\usepackage[T1]{fontenc}
%\usepackage{lmodern}
\begin{document}
  \DocInput{hologo.dtx}%
\end{document}
%</driver>
% \fi
%
%
% \CharacterTable
%  {Upper-case    \A\B\C\D\E\F\G\H\I\J\K\L\M\N\O\P\Q\R\S\T\U\V\W\X\Y\Z
%   Lower-case    \a\b\c\d\e\f\g\h\i\j\k\l\m\n\o\p\q\r\s\t\u\v\w\x\y\z
%   Digits        \0\1\2\3\4\5\6\7\8\9
%   Exclamation   \!     Double quote  \"     Hash (number) \#
%   Dollar        \$     Percent       \%     Ampersand     \&
%   Acute accent  \'     Left paren    \(     Right paren   \)
%   Asterisk      \*     Plus          \+     Comma         \,
%   Minus         \-     Point         \.     Solidus       \/
%   Colon         \:     Semicolon     \;     Less than     \<
%   Equals        \=     Greater than  \>     Question mark \?
%   Commercial at \@     Left bracket  \[     Backslash     \\
%   Right bracket \]     Circumflex    \^     Underscore    \_
%   Grave accent  \`     Left brace    \{     Vertical bar  \|
%   Right brace   \}     Tilde         \~}
%
% \GetFileInfo{hologo.drv}
%
% \title{The \xpackage{hologo} package}
% \date{2016/05/12 v1.11}
% \author{Heiko Oberdiek\\\xemail{heiko.oberdiek at googlemail.com}}
%
% \maketitle
%
% \begin{abstract}
% This package starts a collection of logos with support for bookmarks
% strings.
% \end{abstract}
%
% \tableofcontents
%
% \section{Documentation}
%
% \subsection{Logo macros}
%
% \begin{declcs}{hologo} \M{name}
% \end{declcs}
% Macro \cs{hologo} sets the logo with name \meta{name}.
% The following table shows the supported names.
%
% \begingroup
%   \def\hologoEntry#1#2#3{^^A
%     #1&#2&\hologoLogoSetup{#1}{variant=#2}\hologo{#1}&#3\tabularnewline
%   }
%   \begin{longtable}{>{\ttfamily}l>{\ttfamily}lll}
%     \rmfamily\bfseries{name} & \rmfamily\bfseries variant
%     & \bfseries logo & \bfseries since\\
%     \hline
%     \endhead
%     \hologoList
%   \end{longtable}
% \endgroup
%
% \begin{declcs}{Hologo} \M{name}
% \end{declcs}
% Macro \cs{Hologo} starts the logo \meta{name} with an uppercase
% letter. As an exception small greek letters are not converted
% to uppercase. Examples, see \hologo{eTeX} and \hologo{ExTeX}.
%
% \subsection{Setup macros}
%
% The package does not support package options, but the following
% setup macros can be used to set options.
%
% \begin{declcs}{hologoSetup} \M{key value list}
% \end{declcs}
% Macro \cs{hologoSetup} sets global options.
%
% \begin{declcs}{hologoLogoSetup} \M{logo} \M{key value list}
% \end{declcs}
% Some options can also be used to configure a logo.
% These settings take precedence over global option settings.
%
% \subsection{Options}\label{sec:options}
%
% There are boolean and string options:
% \begin{description}
% \item[Boolean option:]
% It takes |true| or |false|
% as value. If the value is omitted, then |true| is used.
% \item[String option:]
% A value must be given as string. (But the string might be empty.)
% \end{description}
% The following options can be used both in \cs{hologoSetup}
% and \cs{hologoLogoSetup}:
% \begin{description}
% \def\entry#1{\item[\xoption{#1}:]}
% \entry{break}
%   enables or disables line breaks inside the logo. This setting is
%   refined by options \xoption{hyphenbreak}, \xoption{spacebreak}
%   or \xoption{discretionarybreak}.
%   Default is |false|.
% \entry{hyphenbreak}
%   enables or disables the line break right after the hyphen character.
% \entry{spacebreak}
%   enables or disables line breaks at space characters.
% \entry{discretionarybreak}
%   enables or disables line breaks at hyphenation points
%   (inserted by \cs{-}).
% \end{description}
% Macro \cs{hologoLogoSetup} also knows:
% \begin{description}
% \item[\xoption{variant}:]
%   This is a string option. It specifies a variant of a logo that
%   must exist. An empty string selects the package default variant.
% \end{description}
% Example:
% \begin{quote}
%   |\hologoSetup{break=false}|\\
%   |\hologoLogoSetup{plainTeX}{variant=hyphen,hyphenbreak}|\\
%   Then ``plain-\TeX'' contains one break point after the hyphen.
% \end{quote}
%
% \subsection{Driver options}
%
% Sometimes graphical operations are needed to construct some
% glyphs (e.g.\ \hologo{XeTeX}). If package \xpackage{graphics}
% or package \xpackage{pgf} are found, then the macros are taken
% from there. Otherwise the packge defines its own operations
% and therefore needs the driver information. Many drivers are
% detected automatically (\hologo{pdfTeX}/\hologo{LuaTeX}
% in PDF mode, \hologo{XeTeX}, \hologo{VTeX}). These have precedence
% over a driver option. The driver can be given as package option
% or using \cs{hologoDriverSetup}.
% The following list contains the recognized driver options:
% \begin{itemize}
% \item \xoption{pdftex}, \xoption{luatex}
% \item \xoption{dvipdfm}, \xoption{dvipdfmx}
% \item \xoption{dvips}, \xoption{dvipsone}, \xoption{xdvi}
% \item \xoption{xetex}
% \item \xoption{vtex}
% \end{itemize}
% The left driver of a line is the driver name that is used internally.
% The following names are aliases for drivers that use the
% same method. Therefore the entry in the \xext{log} file for
% the used driver prints the internally used driver name.
% \begin{description}
% \item[\xoption{driverfallback}:]
%   This option expects a driver that is used,
%   if the driver could not be detected automatically.
% \end{description}
%
% \begin{declcs}{hologoDriverSetup} \M{driver option}
% \end{declcs}
% The driver can also be configured after package loading
% using \cs{hologoDriverSetup}, also the way for \hologo{plainTeX}
% to setup the driver.
%
% \subsection{Font setup}
%
% Some logos require a special font, but should also be usable by
% \hologo{plainTeX}. Therefore the package provides some ways
% to influence the font settings. The options below
% take font settings as values. Both font commands
% such as \cs{sffamily} and macros that take one argument
% like \cs{textsf} can be used.
%
% \begin{declcs}{hologoFontSetup} \M{key value list}
% \end{declcs}
% Macro \cs{hologoFontSetup} sets the fonts for all logos.
% Supported keys:
% \begin{description}
% \def\entry#1{\item[\xoption{#1}:]}
% \entry{general}
%   This font is used for all logos. The default is empty.
%   That means no special font is used.
% \entry{bibsf}
%   This font is used for
%   {\hologoLogoSetup{BibTeX}{variant=sf}\hologo{BibTeX}}
%   with variant \xoption{sf}.
% \entry{rm}
%   This font is a serif font. It is used for \hologo{ExTeX}.
% \entry{sc}
%   This font specifies a small caps font. It is used for
%   {\hologoLogoSetup{BibTeX}{variant=sc}\hologo{BibTeX}}
%   with variant \xoption{sc}.
% \entry{sf}
%   This font specifies a sans serif font. The default
%   is \cs{sffamily}, then \cs{sf} is tried. Otherwise
%   a warning is given. It is used by \hologo{KOMAScript}.
% \entry{sy}
%   This is the font for math symbols (e.g. cmsy).
%   It is used by \hologo{AmS}, \hologo{NTS}, \hologo{ExTeX}.
% \entry{logo}
%   \hologo{METAFONT} and \hologo{METAPOST} are using that font.
%   In \hologo{LaTeX} \cs{logofamily} is used and
%   the definitions of package \xpackage{mflogo} are used
%   if the package is not loaded.
%   Otherwise the \cs{tenlogo} is used and defined
%   if it does not already exists.
% \end{description}
%
% \begin{declcs}{hologoLogoFontSetup} \M{logo} \M{key value list}
% \end{declcs}
% Fonts can also be set for a logo or logo component separately,
% see the following list.
% The keys are the same as for \cs{hologoFontSetup}.
%
% \begin{longtable}{>{\ttfamily}l>{\sffamily}ll}
%   \meta{logo} & keys & result\\
%   \hline
%   \endhead
%   BibTeX & bibsf & {\hologoLogoSetup{BibTeX}{variant=sf}\hologo{BibTeX}}\\[.5ex]
%   BibTeX & sc & {\hologoLogoSetup{BibTeX}{variant=sc}\hologo{BibTeX}}\\[.5ex]
%   ExTeX & rm & \hologo{ExTeX}\\
%   SliTeX & rm & \hologo{SliTeX}\\[.5ex]
%   AmS & sy & \hologo{AmS}\\
%   ExTeX & sy & \hologo{ExTeX}\\
%   NTS & sy & \hologo{NTS}\\[.5ex]
%   KOMAScript & sf & \hologo{KOMAScript}\\[.5ex]
%   METAFONT & logo & \hologo{METAFONT}\\
%   METAPOST & logo & \hologo{METAPOST}\\[.5ex]
%   SliTeX & sc \hologo{SliTeX}
% \end{longtable}
%
% \subsubsection{Font order}
%
% For all logos the font \xoption{general} is applied first.
% Example:
%\begin{quote}
%|\hologoFontSetup{general=\color{red}}|
%\end{quote}
% will print red logos.
% Then if the font uses a special font \xoption{sf}, for example,
% the font is applied that is setup by \cs{hologoLogoFontSetup}.
% If this font is not setup, then the common font setup
% by \cs{hologoFontSetup} is used. Otherwise a warning is given,
% that there is no font configured.
%
% \subsection{Additional user macros}
%
% Usually a variant of a logo is configured by using
% \cs{hologoLogoSetup}, because it is bad style to mix
% different variants of the same logo in the same text.
% There the following macros are a convenience for testing.
%
% \begin{declcs}{hologoVariant} \M{name} \M{variant}\\
%   \cs{HologoVariant} \M{name} \M{variant}
% \end{declcs}
% Logo \meta{name} is set using \meta{variant} that specifies
% explicitely which variant of the macro is used. If the argument
% is empty, then the default form of the logo is used
% (configurable by \cs{hologoLogoSetup}).
%
% \cs{HologoVariant} is used if the logo is set in a context
% that needs an uppercase first letter (beginning of a sentence, \dots).
%
% \begin{declcs}{hologoList}\\
%   \cs{hologoEntry} \M{logo} \M{variant} \M{since}
% \end{declcs}
% Macro \cs{hologoList} contains all logos that are provided
% by the package including variants. The list consists of calls
% of \cs{hologoEntry} with three arguments starting with the
% logo name \meta{logo} and its variant \meta{variant}. An empty
% variant means the current default. Argument \meta{since} specifies
% with version of the package \xpackage{hologo} is needed to get
% the logo. If the logo is fixed, then the date gets updated.
% Therefore the date \meta{since} is not exactly the date of
% the first introduction, but rather the date of the latest fix.
%
% Before \cs{hologoList} can be used, macro \cs{hologoEntry} needs
% a definition. The example file in section \ref{sec:example}
% shows applications of \cs{hologoList}.
%
% \subsection{Supported contexts}
%
% Macros \cs{hologo} and friends support special contexts:
% \begin{itemize}
% \item \hologo{LaTeX}'s protection mechanism.
% \item Bookmarks of package \xpackage{hyperref}.
% \item Package \xpackage{tex4ht}.
% \item The macros can be used inside \cs{csname} constructs,
%   if \cs{ifincsname} is available (\hologo{pdfTeX}, \hologo{XeTeX},
%   \hologo{LuaTeX}).
% \end{itemize}
%
% \subsection{Example}
% \label{sec:example}
%
% The following example prints the logos in different fonts.
%    \begin{macrocode}
%<*example>
%<<verbatim
\NeedsTeXFormat{LaTeX2e}
\documentclass[a4paper]{article}
\usepackage[
  hmargin=20mm,
  vmargin=20mm,
]{geometry}
\pagestyle{empty}
\usepackage{hologo}[2016/05/12]
\usepackage{longtable}
\usepackage{array}
\setlength{\extrarowheight}{2pt}
\usepackage[T1]{fontenc}
\usepackage{lmodern}
\usepackage{pdflscape}
\usepackage[
  pdfencoding=auto,
]{hyperref}
\hypersetup{
  pdfauthor={Heiko Oberdiek},
  pdftitle={Example for package `hologo'},
  pdfsubject={Logos with fonts lmr, lmss, qtm, qpl, qhv},
}
\usepackage{bookmark}

% Print the logo list on the console

\begingroup
  \typeout{}%
  \typeout{*** Begin of logo list ***}%
  \newcommand*{\hologoEntry}[3]{%
    \typeout{#1 \ifx\\#2\\\else(#2) \fi[#3]}%
  }%
  \hologoList
  \typeout{*** End of logo list ***}%
  \typeout{}%
\endgroup

\begin{document}
\begin{landscape}

  \section{Example file for package `hologo'}

  % Table for font names

  \begin{longtable}{>{\bfseries}ll}
    \textbf{font} & \textbf{Font name}\\
    \hline
    lmr & Latin Modern Roman\\
    lmss & Latin Modern Sans\\
    qtm & \TeX\ Gyre Termes\\
    qhv & \TeX\ Gyre Heros\\
    qpl & \TeX\ Gyre Pagella\\
  \end{longtable}

  % Logo list with logos in different fonts

  \begingroup
    \newcommand*{\SetVariant}[2]{%
      \ifx\\#2\\%
      \else
        \hologoLogoSetup{#1}{variant=#2}%
      \fi
    }%
    \newcommand*{\hologoEntry}[3]{%
      \SetVariant{#1}{#2}%
      \raisebox{1em}[0pt][0pt]{\hypertarget{#1@#2}{}}%
      \bookmark[%
        dest={#1@#2},%
      ]{%
        #1\ifx\\#2\\\else\space(#2)\fi: \Hologo{#1}, \hologo{#1} %
        [Unicode]%
      }%
      \hypersetup{unicode=false}%
      \bookmark[%
        dest={#1@#2},%
      ]{%
        #1\ifx\\#2\\\else\space(#2)\fi: \Hologo{#1}, \hologo{#1} %
        [PDFDocEncoding]%
      }%
      \texttt{#1}%
      &%
      \texttt{#2}%
      &%
      \Hologo{#1}%
      &%
      \SetVariant{#1}{#2}%
      \hologo{#1}%
      &%
      \SetVariant{#1}{#2}%
      \fontfamily{qtm}\selectfont
      \hologo{#1}%
      &%
      \SetVariant{#1}{#2}%
      \fontfamily{qpl}\selectfont
      \hologo{#1}%
      &%
      \SetVariant{#1}{#2}%
      \textsf{\hologo{#1}}%
      &%
      \SetVariant{#1}{#2}%
      \fontfamily{qhv}\selectfont
      \hologo{#1}%
      \tabularnewline
    }%
    \begin{longtable}{llllllll}%
      \textbf{\textit{logo}} & \textbf{\textit{variant}} &
      \texttt{\string\Hologo} &
      \textbf{lmr} & \textbf{qtm} & \textbf{qpl} &
      \textbf{lmss} & \textbf{qhv}
      \tabularnewline
      \hline
      \endhead
      \hologoList
    \end{longtable}%
  \endgroup

\end{landscape}
\end{document}
%verbatim
%</example>
%    \end{macrocode}
%
% \StopEventually{
% }
%
% \section{Implementation}
%    \begin{macrocode}
%<*package>
%    \end{macrocode}
%    Reload check, especially if the package is not used with \LaTeX.
%    \begin{macrocode}
\begingroup\catcode61\catcode48\catcode32=10\relax%
  \catcode13=5 % ^^M
  \endlinechar=13 %
  \catcode35=6 % #
  \catcode39=12 % '
  \catcode44=12 % ,
  \catcode45=12 % -
  \catcode46=12 % .
  \catcode58=12 % :
  \catcode64=11 % @
  \catcode123=1 % {
  \catcode125=2 % }
  \expandafter\let\expandafter\x\csname ver@hologo.sty\endcsname
  \ifx\x\relax % plain-TeX, first loading
  \else
    \def\empty{}%
    \ifx\x\empty % LaTeX, first loading,
      % variable is initialized, but \ProvidesPackage not yet seen
    \else
      \expandafter\ifx\csname PackageInfo\endcsname\relax
        \def\x#1#2{%
          \immediate\write-1{Package #1 Info: #2.}%
        }%
      \else
        \def\x#1#2{\PackageInfo{#1}{#2, stopped}}%
      \fi
      \x{hologo}{The package is already loaded}%
      \aftergroup\endinput
    \fi
  \fi
\endgroup%
%    \end{macrocode}
%    Package identification:
%    \begin{macrocode}
\begingroup\catcode61\catcode48\catcode32=10\relax%
  \catcode13=5 % ^^M
  \endlinechar=13 %
  \catcode35=6 % #
  \catcode39=12 % '
  \catcode40=12 % (
  \catcode41=12 % )
  \catcode44=12 % ,
  \catcode45=12 % -
  \catcode46=12 % .
  \catcode47=12 % /
  \catcode58=12 % :
  \catcode64=11 % @
  \catcode91=12 % [
  \catcode93=12 % ]
  \catcode123=1 % {
  \catcode125=2 % }
  \expandafter\ifx\csname ProvidesPackage\endcsname\relax
    \def\x#1#2#3[#4]{\endgroup
      \immediate\write-1{Package: #3 #4}%
      \xdef#1{#4}%
    }%
  \else
    \def\x#1#2[#3]{\endgroup
      #2[{#3}]%
      \ifx#1\@undefined
        \xdef#1{#3}%
      \fi
      \ifx#1\relax
        \xdef#1{#3}%
      \fi
    }%
  \fi
\expandafter\x\csname ver@hologo.sty\endcsname
\ProvidesPackage{hologo}%
  [2016/05/12 v1.11 A logo collection with bookmark support (HO)]%
%    \end{macrocode}
%
%    \begin{macrocode}
\begingroup\catcode61\catcode48\catcode32=10\relax%
  \catcode13=5 % ^^M
  \endlinechar=13 %
  \catcode123=1 % {
  \catcode125=2 % }
  \catcode64=11 % @
  \def\x{\endgroup
    \expandafter\edef\csname HOLOGO@AtEnd\endcsname{%
      \endlinechar=\the\endlinechar\relax
      \catcode13=\the\catcode13\relax
      \catcode32=\the\catcode32\relax
      \catcode35=\the\catcode35\relax
      \catcode61=\the\catcode61\relax
      \catcode64=\the\catcode64\relax
      \catcode123=\the\catcode123\relax
      \catcode125=\the\catcode125\relax
    }%
  }%
\x\catcode61\catcode48\catcode32=10\relax%
\catcode13=5 % ^^M
\endlinechar=13 %
\catcode35=6 % #
\catcode64=11 % @
\catcode123=1 % {
\catcode125=2 % }
\def\TMP@EnsureCode#1#2{%
  \edef\HOLOGO@AtEnd{%
    \HOLOGO@AtEnd
    \catcode#1=\the\catcode#1\relax
  }%
  \catcode#1=#2\relax
}
\TMP@EnsureCode{10}{12}% ^^J
\TMP@EnsureCode{33}{12}% !
\TMP@EnsureCode{34}{12}% "
\TMP@EnsureCode{36}{3}% $
\TMP@EnsureCode{38}{4}% &
\TMP@EnsureCode{39}{12}% '
\TMP@EnsureCode{40}{12}% (
\TMP@EnsureCode{41}{12}% )
\TMP@EnsureCode{42}{12}% *
\TMP@EnsureCode{43}{12}% +
\TMP@EnsureCode{44}{12}% ,
\TMP@EnsureCode{45}{12}% -
\TMP@EnsureCode{46}{12}% .
\TMP@EnsureCode{47}{12}% /
\TMP@EnsureCode{58}{12}% :
\TMP@EnsureCode{59}{12}% ;
\TMP@EnsureCode{60}{12}% <
\TMP@EnsureCode{62}{12}% >
\TMP@EnsureCode{63}{12}% ?
\TMP@EnsureCode{91}{12}% [
\TMP@EnsureCode{93}{12}% ]
\TMP@EnsureCode{94}{7}% ^ (superscript)
\TMP@EnsureCode{95}{8}% _ (subscript)
\TMP@EnsureCode{96}{12}% `
\TMP@EnsureCode{124}{12}% |
\edef\HOLOGO@AtEnd{%
  \HOLOGO@AtEnd
  \escapechar\the\escapechar\relax
  \noexpand\endinput
}
\escapechar=92 %
%    \end{macrocode}
%
% \subsection{Logo list}
%
%    \begin{macro}{\hologoList}
%    \begin{macrocode}
\def\hologoList{%
  \hologoEntry{(La)TeX}{}{2011/10/01}%
  \hologoEntry{AmSLaTeX}{}{2010/04/16}%
  \hologoEntry{AmSTeX}{}{2010/04/16}%
  \hologoEntry{biber}{}{2011/10/01}%
  \hologoEntry{BibTeX}{}{2011/10/01}%
  \hologoEntry{BibTeX}{sf}{2011/10/01}%
  \hologoEntry{BibTeX}{sc}{2011/10/01}%
  \hologoEntry{BibTeX8}{}{2011/11/22}%
  \hologoEntry{ConTeXt}{}{2011/03/25}%
  \hologoEntry{ConTeXt}{narrow}{2011/03/25}%
  \hologoEntry{ConTeXt}{simple}{2011/03/25}%
  \hologoEntry{emTeX}{}{2010/04/26}%
  \hologoEntry{eTeX}{}{2010/04/08}%
  \hologoEntry{ExTeX}{}{2011/10/01}%
  \hologoEntry{HanTheThanh}{}{2011/11/29}%
  \hologoEntry{iniTeX}{}{2011/10/01}%
  \hologoEntry{KOMAScript}{}{2011/10/01}%
  \hologoEntry{La}{}{2010/05/08}%
  \hologoEntry{LaTeX}{}{2010/04/08}%
  \hologoEntry{LaTeX2e}{}{2010/04/08}%
  \hologoEntry{LaTeX3}{}{2010/04/24}%
  \hologoEntry{LaTeXe}{}{2010/04/08}%
  \hologoEntry{LaTeXML}{}{2011/11/22}%
  \hologoEntry{LaTeXTeX}{}{2011/10/01}%
  \hologoEntry{LuaLaTeX}{}{2010/04/08}%
  \hologoEntry{LuaTeX}{}{2010/04/08}%
  \hologoEntry{LyX}{}{2011/10/01}%
  \hologoEntry{METAFONT}{}{2011/10/01}%
  \hologoEntry{MetaFun}{}{2011/10/01}%
  \hologoEntry{METAPOST}{}{2011/10/01}%
  \hologoEntry{MetaPost}{}{2011/10/01}%
  \hologoEntry{MiKTeX}{}{2011/10/01}%
  \hologoEntry{NTS}{}{2011/10/01}%
  \hologoEntry{OzMF}{}{2011/10/01}%
  \hologoEntry{OzMP}{}{2011/10/01}%
  \hologoEntry{OzTeX}{}{2011/10/01}%
  \hologoEntry{OzTtH}{}{2011/10/01}%
  \hologoEntry{PCTeX}{}{2011/10/01}%
  \hologoEntry{pdfTeX}{}{2011/10/01}%
  \hologoEntry{pdfLaTeX}{}{2011/10/01}%
  \hologoEntry{PiC}{}{2011/10/01}%
  \hologoEntry{PiCTeX}{}{2011/10/01}%
  \hologoEntry{plainTeX}{}{2010/04/08}%
  \hologoEntry{plainTeX}{space}{2010/04/16}%
  \hologoEntry{plainTeX}{hyphen}{2010/04/16}%
  \hologoEntry{plainTeX}{runtogether}{2010/04/16}%
  \hologoEntry{SageTeX}{}{2011/11/22}%
  \hologoEntry{SLiTeX}{}{2011/10/01}%
  \hologoEntry{SLiTeX}{lift}{2011/10/01}%
  \hologoEntry{SLiTeX}{narrow}{2011/10/01}%
  \hologoEntry{SLiTeX}{simple}{2011/10/01}%
  \hologoEntry{SliTeX}{}{2011/10/01}%
  \hologoEntry{SliTeX}{narrow}{2011/10/01}%
  \hologoEntry{SliTeX}{simple}{2011/10/01}%
  \hologoEntry{SliTeX}{lift}{2011/10/01}%
  \hologoEntry{teTeX}{}{2011/10/01}%
  \hologoEntry{TeX}{}{2010/04/08}%
  \hologoEntry{TeX4ht}{}{2011/11/22}%
  \hologoEntry{TTH}{}{2011/11/22}%
  \hologoEntry{virTeX}{}{2011/10/01}%
  \hologoEntry{VTeX}{}{2010/04/24}%
  \hologoEntry{Xe}{}{2010/04/08}%
  \hologoEntry{XeLaTeX}{}{2010/04/08}%
  \hologoEntry{XeTeX}{}{2010/04/08}%
}
%    \end{macrocode}
%    \end{macro}
%
% \subsection{Load resources}
%
%    \begin{macrocode}
\begingroup\expandafter\expandafter\expandafter\endgroup
\expandafter\ifx\csname RequirePackage\endcsname\relax
  \def\TMP@RequirePackage#1[#2]{%
    \begingroup\expandafter\expandafter\expandafter\endgroup
    \expandafter\ifx\csname ver@#1.sty\endcsname\relax
      \input #1.sty\relax
    \fi
  }%
  \TMP@RequirePackage{ltxcmds}[2011/02/04]%
  \TMP@RequirePackage{infwarerr}[2010/04/08]%
  \TMP@RequirePackage{kvsetkeys}[2010/03/01]%
  \TMP@RequirePackage{kvdefinekeys}[2010/03/01]%
  \TMP@RequirePackage{pdftexcmds}[2010/04/01]%
  \TMP@RequirePackage{ifpdf}[2010/01/28]%
  \TMP@RequirePackage{ifluatex}[2010/03/01]%
  \ltx@IfUndefined{newif}{%
    \expandafter\let\csname newif\endcsname\ltx@newif
  }{}%
  \TMP@RequirePackage{ifxetex}[2009/01/23]%
  \TMP@RequirePackage{ifvtex}[2010/03/01]%
\else
  \RequirePackage{ltxcmds}[2011/02/04]%
  \RequirePackage{infwarerr}[2010/04/08]%
  \RequirePackage{kvsetkeys}[2010/03/01]%
  \RequirePackage{kvdefinekeys}[2010/03/01]%
  \RequirePackage{pdftexcmds}[2010/04/01]%
  \RequirePackage{ifpdf}[2010/01/28]%
  \RequirePackage{ifluatex}[2010/03/01]%
  \RequirePackage{ifxetex}[2009/01/23]%
  \RequirePackage{ifvtex}[2010/03/01]%
\fi
%    \end{macrocode}
%
%    \begin{macro}{\HOLOGO@IfDefined}
%    \begin{macrocode}
\def\HOLOGO@IfExists#1{%
  \ifx\@undefined#1%
    \expandafter\ltx@secondoftwo
  \else
    \ifx\relax#1%
      \expandafter\ltx@secondoftwo
    \else
      \expandafter\expandafter\expandafter\ltx@firstoftwo
    \fi
  \fi
}
%    \end{macrocode}
%    \end{macro}
%
% \subsection{Setup macros}
%
%    \begin{macro}{\hologoSetup}
%    \begin{macrocode}
\def\hologoSetup{%
  \let\HOLOGO@name\relax
  \HOLOGO@Setup
}
%    \end{macrocode}
%    \end{macro}
%
%    \begin{macro}{\hologoLogoSetup}
%    \begin{macrocode}
\def\hologoLogoSetup#1{%
  \edef\HOLOGO@name{#1}%
  \ltx@IfUndefined{HoLogo@\HOLOGO@name}{%
    \@PackageError{hologo}{%
      Unknown logo `\HOLOGO@name'%
    }\@ehc
    \ltx@gobble
  }{%
    \HOLOGO@Setup
  }%
}
%    \end{macrocode}
%    \end{macro}
%
%    \begin{macro}{\HOLOGO@Setup}
%    \begin{macrocode}
\def\HOLOGO@Setup{%
  \kvsetkeys{HoLogo}%
}
%    \end{macrocode}
%    \end{macro}
%
% \subsection{Options}
%
%    \begin{macro}{\HOLOGO@DeclareBoolOption}
%    \begin{macrocode}
\def\HOLOGO@DeclareBoolOption#1{%
  \expandafter\chardef\csname HOLOGOOPT@#1\endcsname\ltx@zero
  \kv@define@key{HoLogo}{#1}[true]{%
    \def\HOLOGO@temp{##1}%
    \ifx\HOLOGO@temp\HOLOGO@true
      \ifx\HOLOGO@name\relax
        \expandafter\chardef\csname HOLOGOOPT@#1\endcsname=\ltx@one
      \else
        \expandafter\chardef\csname
        HoLogoOpt@#1@\HOLOGO@name\endcsname\ltx@one
      \fi
      \HOLOGO@SetBreakAll{#1}%
    \else
      \ifx\HOLOGO@temp\HOLOGO@false
        \ifx\HOLOGO@name\relax
          \expandafter\chardef\csname HOLOGOOPT@#1\endcsname=\ltx@zero
        \else
          \expandafter\chardef\csname
          HoLogoOpt@#1@\HOLOGO@name\endcsname=\ltx@zero
        \fi
        \HOLOGO@SetBreakAll{#1}%
      \else
        \@PackageError{hologo}{%
          Unknown value `##1' for boolean option `#1'.\MessageBreak
          Known values are `true' and `false'%
        }\@ehc
      \fi
    \fi
  }%
}
%    \end{macrocode}
%    \end{macro}
%
%    \begin{macro}{\HOLOGO@SetBreakAll}
%    \begin{macrocode}
\def\HOLOGO@SetBreakAll#1{%
  \def\HOLOGO@temp{#1}%
  \ifx\HOLOGO@temp\HOLOGO@break
    \ifx\HOLOGO@name\relax
      \chardef\HOLOGOOPT@hyphenbreak=\HOLOGOOPT@break
      \chardef\HOLOGOOPT@spacebreak=\HOLOGOOPT@break
      \chardef\HOLOGOOPT@discretionarybreak=\HOLOGOOPT@break
    \else
      \expandafter\chardef
         \csname HoLogoOpt@hyphenbreak@\HOLOGO@name\endcsname=%
         \csname HoLogoOpt@break@\HOLOGO@name\endcsname
      \expandafter\chardef
         \csname HoLogoOpt@spacebreak@\HOLOGO@name\endcsname=%
         \csname HoLogoOpt@break@\HOLOGO@name\endcsname
      \expandafter\chardef
         \csname HoLogoOpt@discretionarybreak@\HOLOGO@name
             \endcsname=%
         \csname HoLogoOpt@break@\HOLOGO@name\endcsname
    \fi
  \fi
}
%    \end{macrocode}
%    \end{macro}
%
%    \begin{macro}{\HOLOGO@true}
%    \begin{macrocode}
\def\HOLOGO@true{true}
%    \end{macrocode}
%    \end{macro}
%    \begin{macro}{\HOLOGO@false}
%    \begin{macrocode}
\def\HOLOGO@false{false}
%    \end{macrocode}
%    \end{macro}
%    \begin{macro}{\HOLOGO@break}
%    \begin{macrocode}
\def\HOLOGO@break{break}
%    \end{macrocode}
%    \end{macro}
%
%    \begin{macrocode}
\HOLOGO@DeclareBoolOption{break}
\HOLOGO@DeclareBoolOption{hyphenbreak}
\HOLOGO@DeclareBoolOption{spacebreak}
\HOLOGO@DeclareBoolOption{discretionarybreak}
%    \end{macrocode}
%
%    \begin{macrocode}
\kv@define@key{HoLogo}{variant}{%
  \ifx\HOLOGO@name\relax
    \@PackageError{hologo}{%
      Option `variant' is not available in \string\hologoSetup,%
      \MessageBreak
      Use \string\hologoLogoSetup\space instead%
    }\@ehc
  \else
    \edef\HOLOGO@temp{#1}%
    \ifx\HOLOGO@temp\ltx@empty
      \expandafter
      \let\csname HoLogoOpt@variant@\HOLOGO@name\endcsname\@undefined
    \else
      \ltx@IfUndefined{HoLogo@\HOLOGO@name @\HOLOGO@temp}{%
        \@PackageError{hologo}{%
          Unknown variant `\HOLOGO@temp' of logo `\HOLOGO@name'%
        }\@ehc
      }{%
        \expandafter
        \let\csname HoLogoOpt@variant@\HOLOGO@name\endcsname
            \HOLOGO@temp
      }%
    \fi
  \fi
}
%    \end{macrocode}
%
%    \begin{macro}{\HOLOGO@Variant}
%    \begin{macrocode}
\def\HOLOGO@Variant#1{%
  #1%
  \ltx@ifundefined{HoLogoOpt@variant@#1}{%
  }{%
    @\csname HoLogoOpt@variant@#1\endcsname
  }%
}
%    \end{macrocode}
%    \end{macro}
%
% \subsection{Break/no-break support}
%
%    \begin{macro}{\HOLOGO@space}
%    \begin{macrocode}
\def\HOLOGO@space{%
  \ltx@ifundefined{HoLogoOpt@spacebreak@\HOLOGO@name}{%
    \ltx@ifundefined{HoLogoOpt@break@\HOLOGO@name}{%
      \chardef\HOLOGO@temp=\HOLOGOOPT@spacebreak
    }{%
      \chardef\HOLOGO@temp=%
        \csname HoLogoOpt@break@\HOLOGO@name\endcsname
    }%
  }{%
    \chardef\HOLOGO@temp=%
      \csname HoLogoOpt@spacebreak@\HOLOGO@name\endcsname
  }%
  \ifcase\HOLOGO@temp
    \penalty10000 %
  \fi
  \ltx@space
}
%    \end{macrocode}
%    \end{macro}
%
%    \begin{macro}{\HOLOGO@hyphen}
%    \begin{macrocode}
\def\HOLOGO@hyphen{%
  \ltx@ifundefined{HoLogoOpt@hyphenbreak@\HOLOGO@name}{%
    \ltx@ifundefined{HoLogoOpt@break@\HOLOGO@name}{%
      \chardef\HOLOGO@temp=\HOLOGOOPT@hyphenbreak
    }{%
      \chardef\HOLOGO@temp=%
        \csname HoLogoOpt@break@\HOLOGO@name\endcsname
    }%
  }{%
    \chardef\HOLOGO@temp=%
      \csname HoLogoOpt@hyphenbreak@\HOLOGO@name\endcsname
  }%
  \ifcase\HOLOGO@temp
    \ltx@mbox{-}%
  \else
    -%
  \fi
}
%    \end{macrocode}
%    \end{macro}
%
%    \begin{macro}{\HOLOGO@discretionary}
%    \begin{macrocode}
\def\HOLOGO@discretionary{%
  \ltx@ifundefined{HoLogoOpt@discretionarybreak@\HOLOGO@name}{%
    \ltx@ifundefined{HoLogoOpt@break@\HOLOGO@name}{%
      \chardef\HOLOGO@temp=\HOLOGOOPT@discretionarybreak
    }{%
      \chardef\HOLOGO@temp=%
        \csname HoLogoOpt@break@\HOLOGO@name\endcsname
    }%
  }{%
    \chardef\HOLOGO@temp=%
      \csname HoLogoOpt@discretionarybreak@\HOLOGO@name\endcsname
  }%
  \ifcase\HOLOGO@temp
  \else
    \-%
  \fi
}
%    \end{macrocode}
%    \end{macro}
%
%    \begin{macro}{\HOLOGO@mbox}
%    \begin{macrocode}
\def\HOLOGO@mbox#1{%
  \ltx@ifundefined{HoLogoOpt@break@\HOLOGO@name}{%
    \chardef\HOLOGO@temp=\HOLOGOOPT@hyphenbreak
  }{%
    \chardef\HOLOGO@temp=%
      \csname HoLogoOpt@break@\HOLOGO@name\endcsname
  }%
  \ifcase\HOLOGO@temp
    \ltx@mbox{#1}%
  \else
    #1%
  \fi
}
%    \end{macrocode}
%    \end{macro}
%
% \subsection{Font support}
%
%    \begin{macro}{\HoLogoFont@font}
%    \begin{tabular}{@{}ll@{}}
%    |#1|:& logo name\\
%    |#2|:& font short name\\
%    |#3|:& text
%    \end{tabular}
%    \begin{macrocode}
\def\HoLogoFont@font#1#2#3{%
  \begingroup
    \ltx@IfUndefined{HoLogoFont@logo@#1.#2}{%
      \ltx@IfUndefined{HoLogoFont@font@#2}{%
        \@PackageWarning{hologo}{%
          Missing font `#2' for logo `#1'%
        }%
        #3%
      }{%
        \csname HoLogoFont@font@#2\endcsname{#3}%
      }%
    }{%
      \csname HoLogoFont@logo@#1.#2\endcsname{#3}%
    }%
  \endgroup
}
%    \end{macrocode}
%    \end{macro}
%
%    \begin{macro}{\HoLogoFont@Def}
%    \begin{macrocode}
\def\HoLogoFont@Def#1{%
  \expandafter\def\csname HoLogoFont@font@#1\endcsname
}
%    \end{macrocode}
%    \end{macro}
%    \begin{macro}{\HoLogoFont@LogoDef}
%    \begin{macrocode}
\def\HoLogoFont@LogoDef#1#2{%
  \expandafter\def\csname HoLogoFont@logo@#1.#2\endcsname
}
%    \end{macrocode}
%    \end{macro}
%
% \subsubsection{Font defaults}
%
%    \begin{macro}{\HoLogoFont@font@general}
%    \begin{macrocode}
\HoLogoFont@Def{general}{}%
%    \end{macrocode}
%    \end{macro}
%
%    \begin{macro}{\HoLogoFont@font@rm}
%    \begin{macrocode}
\ltx@IfUndefined{rmfamily}{%
  \ltx@IfUndefined{rm}{%
  }{%
    \HoLogoFont@Def{rm}{\rm}%
  }%
}{%
  \HoLogoFont@Def{rm}{\rmfamily}%
}
%    \end{macrocode}
%    \end{macro}
%
%    \begin{macro}{\HoLogoFont@font@sf}
%    \begin{macrocode}
\ltx@IfUndefined{sffamily}{%
  \ltx@IfUndefined{sf}{%
  }{%
    \HoLogoFont@Def{sf}{\sf}%
  }%
}{%
  \HoLogoFont@Def{sf}{\sffamily}%
}
%    \end{macrocode}
%    \end{macro}
%
%    \begin{macro}{\HoLogoFont@font@bibsf}
%    In case of \hologo{plainTeX} the original small caps
%    variant is used as default. In \hologo{LaTeX}
%    the definition of package \xpackage{dtklogos} \cite{dtklogos}
%    is used.
%\begin{quote}
%\begin{verbatim}
%\DeclareRobustCommand{\BibTeX}{%
%  B%
%  \kern-.05em%
%  \hbox{%
%    $\m@th$% %% force math size calculations
%    \csname S@\f@size\endcsname
%    \fontsize\sf@size\z@
%    \math@fontsfalse
%    \selectfont
%    I%
%    \kern-.025em%
%    B
%  }%
%  \kern-.08em%
%  \-%
%  \TeX
%}
%\end{verbatim}
%\end{quote}
%    \begin{macrocode}
\ltx@IfUndefined{selectfont}{%
  \ltx@IfUndefined{tensc}{%
    \font\tensc=cmcsc10\relax
  }{}%
  \HoLogoFont@Def{bibsf}{\tensc}%
}{%
  \HoLogoFont@Def{bibsf}{%
    $\mathsurround=0pt$%
    \csname S@\f@size\endcsname
    \fontsize\sf@size{0pt}%
    \math@fontsfalse
    \selectfont
  }%
}
%    \end{macrocode}
%    \end{macro}
%
%    \begin{macro}{\HoLogoFont@font@sc}
%    \begin{macrocode}
\ltx@IfUndefined{scshape}{%
  \ltx@IfUndefined{tensc}{%
    \font\tensc=cmcsc10\relax
  }{}%
  \HoLogoFont@Def{sc}{\tensc}%
}{%
  \HoLogoFont@Def{sc}{\scshape}%
}
%    \end{macrocode}
%    \end{macro}
%
%    \begin{macro}{\HoLogoFont@font@sy}
%    \begin{macrocode}
\ltx@IfUndefined{usefont}{%
  \ltx@IfUndefined{tensy}{%
  }{%
    \HoLogoFont@Def{sy}{\tensy}%
  }%
}{%
  \HoLogoFont@Def{sy}{%
    \usefont{OMS}{cmsy}{m}{n}%
  }%
}
%    \end{macrocode}
%    \end{macro}
%
%    \begin{macro}{\HoLogoFont@font@logo}
%    \begin{macrocode}
\begingroup
  \def\x{LaTeX2e}%
\expandafter\endgroup
\ifx\fmtname\x
  \ltx@IfUndefined{logofamily}{%
    \DeclareRobustCommand\logofamily{%
      \not@math@alphabet\logofamily\relax
      \fontencoding{U}%
      \fontfamily{logo}%
      \selectfont
    }%
  }{}%
  \ltx@IfUndefined{logofamily}{%
  }{%
    \HoLogoFont@Def{logo}{\logofamily}%
  }%
\else
  \ltx@IfUndefined{tenlogo}{%
    \font\tenlogo=logo10\relax
  }{}%
  \HoLogoFont@Def{logo}{\tenlogo}%
\fi
%    \end{macrocode}
%    \end{macro}
%
% \subsubsection{Font setup}
%
%    \begin{macro}{\hologoFontSetup}
%    \begin{macrocode}
\def\hologoFontSetup{%
  \let\HOLOGO@name\relax
  \HOLOGO@FontSetup
}
%    \end{macrocode}
%    \end{macro}
%
%    \begin{macro}{\hologoLogoFontSetup}
%    \begin{macrocode}
\def\hologoLogoFontSetup#1{%
  \edef\HOLOGO@name{#1}%
  \ltx@IfUndefined{HoLogo@\HOLOGO@name}{%
    \@PackageError{hologo}{%
      Unknown logo `\HOLOGO@name'%
    }\@ehc
    \ltx@gobble
  }{%
    \HOLOGO@FontSetup
  }%
}
%    \end{macrocode}
%    \end{macro}
%
%    \begin{macro}{\HOLOGO@FontSetup}
%    \begin{macrocode}
\def\HOLOGO@FontSetup{%
  \kvsetkeys{HoLogoFont}%
}
%    \end{macrocode}
%    \end{macro}
%
%    \begin{macrocode}
\def\HOLOGO@temp#1{%
  \kv@define@key{HoLogoFont}{#1}{%
    \ifx\HOLOGO@name\relax
      \HoLogoFont@Def{#1}{##1}%
    \else
      \HoLogoFont@LogoDef\HOLOGO@name{#1}{##1}%
    \fi
  }%
}
\HOLOGO@temp{general}
\HOLOGO@temp{sf}
%    \end{macrocode}
%
% \subsection{Generic logo commands}
%
%    \begin{macrocode}
\HOLOGO@IfExists\hologo{%
  \@PackageError{hologo}{%
    \string\hologo\ltx@space is already defined.\MessageBreak
    Package loading is aborted%
  }\@ehc
  \HOLOGO@AtEnd
}%
\HOLOGO@IfExists\hologoRobust{%
  \@PackageError{hologo}{%
    \string\hologoRobust\ltx@space is already defined.\MessageBreak
    Package loading is aborted%
  }\@ehc
  \HOLOGO@AtEnd
}%
%    \end{macrocode}
%
% \subsubsection{\cs{hologo} and friends}
%
%    \begin{macrocode}
\ifluatex
  \expandafter\ltx@firstofone
\else
  \expandafter\ltx@gobble
\fi
{%
  \ltx@IfUndefined{ifincsname}{%
    \ifnum\luatexversion<36 %
      \expandafter\ltx@gobble
    \else
      \expandafter\ltx@firstofone
    \fi
    {%
      \begingroup
        \ifcase0%
            \directlua{%
              if tex.enableprimitives then %
                tex.enableprimitives('HOLOGO@', {'ifincsname'})%
              else %
                tex.print('1')%
              end%
            }%
            \ifx\HOLOGO@ifincsname\@undefined 1\fi%
            \relax
          \expandafter\ltx@firstofone
        \else
          \endgroup
          \expandafter\ltx@gobble
        \fi
        {%
          \global\let\ifincsname\HOLOGO@ifincsname
        }%
      \HOLOGO@temp
    }%
  }{}%
}
%    \end{macrocode}
%    \begin{macrocode}
\ltx@IfUndefined{ifincsname}{%
  \catcode`$=14 %
}{%
  \catcode`$=9 %
}
%    \end{macrocode}
%
%    \begin{macro}{\hologo}
%    \begin{macrocode}
\def\hologo#1{%
$ \ifincsname
$   \ltx@ifundefined{HoLogoCs@\HOLOGO@Variant{#1}}{%
$     #1%
$   }{%
$     \csname HoLogoCs@\HOLOGO@Variant{#1}\endcsname\ltx@firstoftwo
$   }%
$ \else
    \HOLOGO@IfExists\texorpdfstring\texorpdfstring\ltx@firstoftwo
    {%
      \hologoRobust{#1}%
    }{%
      \ltx@ifundefined{HoLogoBkm@\HOLOGO@Variant{#1}}{%
        \ltx@ifundefined{HoLogo@#1}{?#1?}{#1}%
      }{%
        \csname HoLogoBkm@\HOLOGO@Variant{#1}\endcsname
        \ltx@firstoftwo
      }%
    }%
$ \fi
}
%    \end{macrocode}
%    \end{macro}
%    \begin{macro}{\Hologo}
%    \begin{macrocode}
\def\Hologo#1{%
$ \ifincsname
$   \ltx@ifundefined{HoLogoCs@\HOLOGO@Variant{#1}}{%
$     #1%
$   }{%
$     \csname HoLogoCs@\HOLOGO@Variant{#1}\endcsname\ltx@secondoftwo
$   }%
$ \else
    \HOLOGO@IfExists\texorpdfstring\texorpdfstring\ltx@firstoftwo
    {%
      \HologoRobust{#1}%
    }{%
      \ltx@ifundefined{HoLogoBkm@\HOLOGO@Variant{#1}}{%
        \ltx@ifundefined{HoLogo@#1}{?#1?}{#1}%
      }{%
        \csname HoLogoBkm@\HOLOGO@Variant{#1}\endcsname
        \ltx@secondoftwo
      }%
    }%
$ \fi
}
%    \end{macrocode}
%    \end{macro}
%
%    \begin{macro}{\hologoVariant}
%    \begin{macrocode}
\def\hologoVariant#1#2{%
  \ifx\relax#2\relax
    \hologo{#1}%
  \else
$   \ifincsname
$     \ltx@ifundefined{HoLogoCs@#1@#2}{%
$       #1%
$     }{%
$       \csname HoLogoCs@#1@#2\endcsname\ltx@firstoftwo
$     }%
$   \else
      \HOLOGO@IfExists\texorpdfstring\texorpdfstring\ltx@firstoftwo
      {%
        \hologoVariantRobust{#1}{#2}%
      }{%
        \ltx@ifundefined{HoLogoBkm@#1@#2}{%
          \ltx@ifundefined{HoLogo@#1}{?#1?}{#1}%
        }{%
          \csname HoLogoBkm@#1@#2\endcsname
          \ltx@firstoftwo
        }%
      }%
$   \fi
  \fi
}
%    \end{macrocode}
%    \end{macro}
%    \begin{macro}{\HologoVariant}
%    \begin{macrocode}
\def\HologoVariant#1#2{%
  \ifx\relax#2\relax
    \Hologo{#1}%
  \else
$   \ifincsname
$     \ltx@ifundefined{HoLogoCs@#1@#2}{%
$       #1%
$     }{%
$       \csname HoLogoCs@#1@#2\endcsname\ltx@secondoftwo
$     }%
$   \else
      \HOLOGO@IfExists\texorpdfstring\texorpdfstring\ltx@firstoftwo
      {%
        \HologoVariantRobust{#1}{#2}%
      }{%
        \ltx@ifundefined{HoLogoBkm@#1@#2}{%
          \ltx@ifundefined{HoLogo@#1}{?#1?}{#1}%
        }{%
          \csname HoLogoBkm@#1@#2\endcsname
          \ltx@secondoftwo
        }%
      }%
$   \fi
  \fi
}
%    \end{macrocode}
%    \end{macro}
%
%    \begin{macrocode}
\catcode`\$=3 %
%    \end{macrocode}
%
% \subsubsection{\cs{hologoRobust} and friends}
%
%    \begin{macro}{\hologoRobust}
%    \begin{macrocode}
\ltx@IfUndefined{protected}{%
  \ltx@IfUndefined{DeclareRobustCommand}{%
    \def\hologoRobust#1%
  }{%
    \DeclareRobustCommand*\hologoRobust[1]%
  }%
}{%
  \protected\def\hologoRobust#1%
}%
{%
  \edef\HOLOGO@name{#1}%
  \ltx@IfUndefined{HoLogo@\HOLOGO@Variant\HOLOGO@name}{%
    \@PackageError{hologo}{%
      Unknown logo `\HOLOGO@name'%
    }\@ehc
    ?\HOLOGO@name?%
  }{%
    \ltx@IfUndefined{ver@tex4ht.sty}{%
      \HoLogoFont@font\HOLOGO@name{general}{%
        \csname HoLogo@\HOLOGO@Variant\HOLOGO@name\endcsname
        \ltx@firstoftwo
      }%
    }{%
      \ltx@IfUndefined{HoLogoHtml@\HOLOGO@Variant\HOLOGO@name}{%
        \HOLOGO@name
      }{%
        \csname HoLogoHtml@\HOLOGO@Variant\HOLOGO@name\endcsname
        \ltx@firstoftwo
      }%
    }%
  }%
}
%    \end{macrocode}
%    \end{macro}
%    \begin{macro}{\HologoRobust}
%    \begin{macrocode}
\ltx@IfUndefined{protected}{%
  \ltx@IfUndefined{DeclareRobustCommand}{%
    \def\HologoRobust#1%
  }{%
    \DeclareRobustCommand*\HologoRobust[1]%
  }%
}{%
  \protected\def\HologoRobust#1%
}%
{%
  \edef\HOLOGO@name{#1}%
  \ltx@IfUndefined{HoLogo@\HOLOGO@Variant\HOLOGO@name}{%
    \@PackageError{hologo}{%
      Unknown logo `\HOLOGO@name'%
    }\@ehc
    ?\HOLOGO@name?%
  }{%
    \ltx@IfUndefined{ver@tex4ht.sty}{%
      \HoLogoFont@font\HOLOGO@name{general}{%
        \csname HoLogo@\HOLOGO@Variant\HOLOGO@name\endcsname
        \ltx@secondoftwo
      }%
    }{%
      \ltx@IfUndefined{HoLogoHtml@\HOLOGO@Variant\HOLOGO@name}{%
        \expandafter\HOLOGO@Uppercase\HOLOGO@name
      }{%
        \csname HoLogoHtml@\HOLOGO@Variant\HOLOGO@name\endcsname
        \ltx@secondoftwo
      }%
    }%
  }%
}
%    \end{macrocode}
%    \end{macro}
%    \begin{macro}{\hologoVariantRobust}
%    \begin{macrocode}
\ltx@IfUndefined{protected}{%
  \ltx@IfUndefined{DeclareRobustCommand}{%
    \def\hologoVariantRobust#1#2%
  }{%
    \DeclareRobustCommand*\hologoVariantRobust[2]%
  }%
}{%
  \protected\def\hologoVariantRobust#1#2%
}%
{%
  \begingroup
    \hologoLogoSetup{#1}{variant={#2}}%
    \hologoRobust{#1}%
  \endgroup
}
%    \end{macrocode}
%    \end{macro}
%    \begin{macro}{\HologoVariantRobust}
%    \begin{macrocode}
\ltx@IfUndefined{protected}{%
  \ltx@IfUndefined{DeclareRobustCommand}{%
    \def\HologoVariantRobust#1#2%
  }{%
    \DeclareRobustCommand*\HologoVariantRobust[2]%
  }%
}{%
  \protected\def\HologoVariantRobust#1#2%
}%
{%
  \begingroup
    \hologoLogoSetup{#1}{variant={#2}}%
    \HologoRobust{#1}%
  \endgroup
}
%    \end{macrocode}
%    \end{macro}
%
%    \begin{macro}{\hologorobust}
%    Macro \cs{hologorobust} is only defined for compatibility.
%    Its use is deprecated.
%    \begin{macrocode}
\def\hologorobust{\hologoRobust}
%    \end{macrocode}
%    \end{macro}
%
% \subsection{Helpers}
%
%    \begin{macro}{\HOLOGO@Uppercase}
%    Macro \cs{HOLOGO@Uppercase} is restricted to \cs{uppercase},
%    because \hologo{plainTeX} or \hologo{iniTeX} do not provide
%    \cs{MakeUppercase}.
%    \begin{macrocode}
\def\HOLOGO@Uppercase#1{\uppercase{#1}}
%    \end{macrocode}
%    \end{macro}
%
%    \begin{macro}{\HOLOGO@PdfdocUnicode}
%    \begin{macrocode}
\def\HOLOGO@PdfdocUnicode{%
  \ifx\ifHy@unicode\iftrue
    \expandafter\ltx@secondoftwo
  \else
    \expandafter\ltx@firstoftwo
  \fi
}
%    \end{macrocode}
%    \end{macro}
%
%    \begin{macro}{\HOLOGO@Math}
%    \begin{macrocode}
\def\HOLOGO@MathSetup{%
  \mathsurround0pt\relax
  \HOLOGO@IfExists\f@series{%
    \if b\expandafter\ltx@car\f@series x\@nil
      \csname boldmath\endcsname
   \fi
  }{}%
}
%    \end{macrocode}
%    \end{macro}
%
%    \begin{macro}{\HOLOGO@TempDimen}
%    \begin{macrocode}
\dimendef\HOLOGO@TempDimen=\ltx@zero
%    \end{macrocode}
%    \end{macro}
%    \begin{macro}{\HOLOGO@NegativeKerning}
%    \begin{macrocode}
\def\HOLOGO@NegativeKerning#1{%
  \begingroup
    \HOLOGO@TempDimen=0pt\relax
    \comma@parse@normalized{#1}{%
      \ifdim\HOLOGO@TempDimen=0pt %
        \expandafter\HOLOGO@@NegativeKerning\comma@entry
      \fi
      \ltx@gobble
    }%
    \ifdim\HOLOGO@TempDimen<0pt %
      \kern\HOLOGO@TempDimen
    \fi
  \endgroup
}
%    \end{macrocode}
%    \end{macro}
%    \begin{macro}{\HOLOGO@@NegativeKerning}
%    \begin{macrocode}
\def\HOLOGO@@NegativeKerning#1#2{%
  \setbox\ltx@zero\hbox{#1#2}%
  \HOLOGO@TempDimen=\wd\ltx@zero
  \setbox\ltx@zero\hbox{#1\kern0pt#2}%
  \advance\HOLOGO@TempDimen by -\wd\ltx@zero
}
%    \end{macrocode}
%    \end{macro}
%
%    \begin{macro}{\HOLOGO@SpaceFactor}
%    \begin{macrocode}
\def\HOLOGO@SpaceFactor{%
  \spacefactor1000 %
}
%    \end{macrocode}
%    \end{macro}
%
%    \begin{macro}{\HOLOGO@Span}
%    \begin{macrocode}
\def\HOLOGO@Span#1#2{%
  \HCode{<span class="HoLogo-#1">}%
  #2%
  \HCode{</span>}%
}
%    \end{macrocode}
%    \end{macro}
%
% \subsubsection{Text subscript}
%
%    \begin{macro}{\HOLOGO@SubScript}%
%    \begin{macrocode}
\def\HOLOGO@SubScript#1{%
  \ltx@IfUndefined{textsubscript}{%
    \ltx@IfUndefined{text}{%
      \ltx@mbox{%
        \mathsurround=0pt\relax
        $%
          _{%
            \ltx@IfUndefined{sf@size}{%
              \mathrm{#1}%
            }{%
              \mbox{%
                \fontsize\sf@size{0pt}\selectfont
                #1%
              }%
            }%
          }%
        $%
      }%
    }{%
      \ltx@mbox{%
        \mathsurround=0pt\relax
        $_{\text{#1}}$%
      }%
    }%
  }{%
    \textsubscript{#1}%
  }%
}
%    \end{macrocode}
%    \end{macro}
%
% \subsection{\hologo{TeX} and friends}
%
% \subsubsection{\hologo{TeX}}
%
%    \begin{macro}{\HoLogo@TeX}
%    Source: \hologo{LaTeX} kernel.
%    \begin{macrocode}
\def\HoLogo@TeX#1{%
  T\kern-.1667em\lower.5ex\hbox{E}\kern-.125emX\HOLOGO@SpaceFactor
}
%    \end{macrocode}
%    \end{macro}
%    \begin{macro}{\HoLogoHtml@TeX}
%    \begin{macrocode}
\def\HoLogoHtml@TeX#1{%
  \HoLogoCss@TeX
  \HOLOGO@Span{TeX}{%
    T%
    \HOLOGO@Span{e}{%
      E%
    }%
    X%
  }%
}
%    \end{macrocode}
%    \end{macro}
%    \begin{macro}{\HoLogoCss@TeX}
%    \begin{macrocode}
\def\HoLogoCss@TeX{%
  \Css{%
    span.HoLogo-TeX span.HoLogo-e{%
      position:relative;%
      top:.5ex;%
      margin-left:-.1667em;%
      margin-right:-.125em;%
    }%
  }%
  \Css{%
    a span.HoLogo-TeX span.HoLogo-e{%
      text-decoration:none;%
    }%
  }%
  \global\let\HoLogoCss@TeX\relax
}
%    \end{macrocode}
%    \end{macro}
%
% \subsubsection{\hologo{plainTeX}}
%
%    \begin{macro}{\HoLogo@plainTeX@space}
%    Source: ``The \hologo{TeX}book''
%    \begin{macrocode}
\def\HoLogo@plainTeX@space#1{%
  \HOLOGO@mbox{#1{p}{P}lain}\HOLOGO@space\hologo{TeX}%
}
%    \end{macrocode}
%    \end{macro}
%    \begin{macro}{\HoLogoCs@plainTeX@space}
%    \begin{macrocode}
\def\HoLogoCs@plainTeX@space#1{#1{p}{P}lain TeX}%
%    \end{macrocode}
%    \end{macro}
%    \begin{macro}{\HoLogoBkm@plainTeX@space}
%    \begin{macrocode}
\def\HoLogoBkm@plainTeX@space#1{%
  #1{p}{P}lain \hologo{TeX}%
}
%    \end{macrocode}
%    \end{macro}
%    \begin{macro}{\HoLogoHtml@plainTeX@space}
%    \begin{macrocode}
\def\HoLogoHtml@plainTeX@space#1{%
  #1{p}{P}lain \hologo{TeX}%
}
%    \end{macrocode}
%    \end{macro}
%
%    \begin{macro}{\HoLogo@plainTeX@hyphen}
%    \begin{macrocode}
\def\HoLogo@plainTeX@hyphen#1{%
  \HOLOGO@mbox{#1{p}{P}lain}\HOLOGO@hyphen\hologo{TeX}%
}
%    \end{macrocode}
%    \end{macro}
%    \begin{macro}{\HoLogoCs@plainTeX@hyphen}
%    \begin{macrocode}
\def\HoLogoCs@plainTeX@hyphen#1{#1{p}{P}lain-TeX}
%    \end{macrocode}
%    \end{macro}
%    \begin{macro}{\HoLogoBkm@plainTeX@hyphen}
%    \begin{macrocode}
\def\HoLogoBkm@plainTeX@hyphen#1{%
  #1{p}{P}lain-\hologo{TeX}%
}
%    \end{macrocode}
%    \end{macro}
%    \begin{macro}{\HoLogoHtml@plainTeX@hyphen}
%    \begin{macrocode}
\def\HoLogoHtml@plainTeX@hyphen#1{%
  #1{p}{P}lain-\hologo{TeX}%
}
%    \end{macrocode}
%    \end{macro}
%
%    \begin{macro}{\HoLogo@plainTeX@runtogether}
%    \begin{macrocode}
\def\HoLogo@plainTeX@runtogether#1{%
  \HOLOGO@mbox{#1{p}{P}lain\hologo{TeX}}%
}
%    \end{macrocode}
%    \end{macro}
%    \begin{macro}{\HoLogoCs@plainTeX@runtogether}
%    \begin{macrocode}
\def\HoLogoCs@plainTeX@runtogether#1{#1{p}{P}lainTeX}
%    \end{macrocode}
%    \end{macro}
%    \begin{macro}{\HoLogoBkm@plainTeX@runtogether}
%    \begin{macrocode}
\def\HoLogoBkm@plainTeX@runtogether#1{%
  #1{p}{P}lain\hologo{TeX}%
}
%    \end{macrocode}
%    \end{macro}
%    \begin{macro}{\HoLogoHtml@plainTeX@runtogether}
%    \begin{macrocode}
\def\HoLogoHtml@plainTeX@runtogether#1{%
  #1{p}{P}lain\hologo{TeX}%
}
%    \end{macrocode}
%    \end{macro}
%
%    \begin{macro}{\HoLogo@plainTeX}
%    \begin{macrocode}
\def\HoLogo@plainTeX{\HoLogo@plainTeX@space}
%    \end{macrocode}
%    \end{macro}
%    \begin{macro}{\HoLogoCs@plainTeX}
%    \begin{macrocode}
\def\HoLogoCs@plainTeX{\HoLogoCs@plainTeX@space}
%    \end{macrocode}
%    \end{macro}
%    \begin{macro}{\HoLogoBkm@plainTeX}
%    \begin{macrocode}
\def\HoLogoBkm@plainTeX{\HoLogoBkm@plainTeX@space}
%    \end{macrocode}
%    \end{macro}
%    \begin{macro}{\HoLogoHtml@plainTeX}
%    \begin{macrocode}
\def\HoLogoHtml@plainTeX{\HoLogoHtml@plainTeX@space}
%    \end{macrocode}
%    \end{macro}
%
% \subsubsection{\hologo{LaTeX}}
%
%    Source: \hologo{LaTeX} kernel.
%\begin{quote}
%\begin{verbatim}
%\DeclareRobustCommand{\LaTeX}{%
%  L%
%  \kern-.36em%
%  {%
%    \sbox\z@ T%
%    \vbox to\ht\z@{%
%      \hbox{%
%        \check@mathfonts
%        \fontsize\sf@size\z@
%        \math@fontsfalse
%        \selectfont
%        A%
%      }%
%      \vss
%    }%
%  }%
%  \kern-.15em%
%  \TeX
%}
%\end{verbatim}
%\end{quote}
%
%    \begin{macro}{\HoLogo@La}
%    \begin{macrocode}
\def\HoLogo@La#1{%
  L%
  \kern-.36em%
  \begingroup
    \setbox\ltx@zero\hbox{T}%
    \vbox to\ht\ltx@zero{%
      \hbox{%
        \ltx@ifundefined{check@mathfonts}{%
          \csname sevenrm\endcsname
        }{%
          \check@mathfonts
          \fontsize\sf@size{0pt}%
          \math@fontsfalse\selectfont
        }%
        A%
      }%
      \vss
    }%
  \endgroup
}
%    \end{macrocode}
%    \end{macro}
%
%    \begin{macro}{\HoLogo@LaTeX}
%    Source: \hologo{LaTeX} kernel.
%    \begin{macrocode}
\def\HoLogo@LaTeX#1{%
  \hologo{La}%
  \kern-.15em%
  \hologo{TeX}%
}
%    \end{macrocode}
%    \end{macro}
%    \begin{macro}{\HoLogoHtml@LaTeX}
%    \begin{macrocode}
\def\HoLogoHtml@LaTeX#1{%
  \HoLogoCss@LaTeX
  \HOLOGO@Span{LaTeX}{%
    L%
    \HOLOGO@Span{a}{%
      A%
    }%
    \hologo{TeX}%
  }%
}
%    \end{macrocode}
%    \end{macro}
%    \begin{macro}{\HoLogoCss@LaTeX}
%    \begin{macrocode}
\def\HoLogoCss@LaTeX{%
  \Css{%
    span.HoLogo-LaTeX span.HoLogo-a{%
      position:relative;%
      top:-.5ex;%
      margin-left:-.36em;%
      margin-right:-.15em;%
      font-size:85\%;%
    }%
  }%
  \global\let\HoLogoCss@LaTeX\relax
}
%    \end{macrocode}
%    \end{macro}
%
% \subsubsection{\hologo{(La)TeX}}
%
%    \begin{macro}{\HoLogo@LaTeXTeX}
%    The kerning around the parentheses is taken
%    from package \xpackage{dtklogos} \cite{dtklogos}.
%\begin{quote}
%\begin{verbatim}
%\DeclareRobustCommand{\LaTeXTeX}{%
%  (%
%  \kern-.15em%
%  L%
%  \kern-.36em%
%  {%
%    \sbox\z@ T%
%    \vbox to\ht0{%
%      \hbox{%
%        $\m@th$%
%        \csname S@\f@size\endcsname
%        \fontsize\sf@size\z@
%        \math@fontsfalse
%        \selectfont
%        A%
%      }%
%      \vss
%    }%
%  }%
%  \kern-.2em%
%  )%
%  \kern-.15em%
%  \TeX
%}
%\end{verbatim}
%\end{quote}
%    \begin{macrocode}
\def\HoLogo@LaTeXTeX#1{%
  (%
  \kern-.15em%
  \hologo{La}%
  \kern-.2em%
  )%
  \kern-.15em%
  \hologo{TeX}%
}
%    \end{macrocode}
%    \end{macro}
%    \begin{macro}{\HoLogoBkm@LaTeXTeX}
%    \begin{macrocode}
\def\HoLogoBkm@LaTeXTeX#1{(La)TeX}
%    \end{macrocode}
%    \end{macro}
%
%    \begin{macro}{\HoLogo@(La)TeX}
%    \begin{macrocode}
\expandafter
\let\csname HoLogo@(La)TeX\endcsname\HoLogo@LaTeXTeX
%    \end{macrocode}
%    \end{macro}
%    \begin{macro}{\HoLogoBkm@(La)TeX}
%    \begin{macrocode}
\expandafter
\let\csname HoLogoBkm@(La)TeX\endcsname\HoLogoBkm@LaTeXTeX
%    \end{macrocode}
%    \end{macro}
%    \begin{macro}{\HoLogoHtml@LaTeXTeX}
%    \begin{macrocode}
\def\HoLogoHtml@LaTeXTeX#1{%
  \HoLogoCss@LaTeXTeX
  \HOLOGO@Span{LaTeXTeX}{%
    (%
    \HOLOGO@Span{L}{L}%
    \HOLOGO@Span{a}{A}%
    \HOLOGO@Span{ParenRight}{)}%
    \hologo{TeX}%
  }%
}
%    \end{macrocode}
%    \end{macro}
%    \begin{macro}{\HoLogoHtml@(La)TeX}
%    Kerning after opening parentheses and before closing parentheses
%    is $-0.1$\,em. The original values $-0.15$\,em
%    looked too ugly for a serif font.
%    \begin{macrocode}
\expandafter
\let\csname HoLogoHtml@(La)TeX\endcsname\HoLogoHtml@LaTeXTeX
%    \end{macrocode}
%    \end{macro}
%    \begin{macro}{\HoLogoCss@LaTeXTeX}
%    \begin{macrocode}
\def\HoLogoCss@LaTeXTeX{%
  \Css{%
    span.HoLogo-LaTeXTeX span.HoLogo-L{%
      margin-left:-.1em;%
    }%
  }%
  \Css{%
    span.HoLogo-LaTeXTeX span.HoLogo-a{%
      position:relative;%
      top:-.5ex;%
      margin-left:-.36em;%
      margin-right:-.1em;%
      font-size:85\%;%
    }%
  }%
  \Css{%
    span.HoLogo-LaTeXTeX span.HoLogo-ParenRight{%
      margin-right:-.15em;%
    }%
  }%
  \global\let\HoLogoCss@LaTeXTeX\relax
}
%    \end{macrocode}
%    \end{macro}
%
% \subsubsection{\hologo{LaTeXe}}
%
%    \begin{macro}{\HoLogo@LaTeXe}
%    Source: \hologo{LaTeX} kernel
%    \begin{macrocode}
\def\HoLogo@LaTeXe#1{%
  \hologo{LaTeX}%
  \kern.15em%
  \hbox{%
    \HOLOGO@MathSetup
    2%
    $_{\textstyle\varepsilon}$%
  }%
}
%    \end{macrocode}
%    \end{macro}
%
%    \begin{macro}{\HoLogoCs@LaTeXe}
%    \begin{macrocode}
\ifnum64=`\^^^^0040\relax % test for big chars of LuaTeX/XeTeX
  \catcode`\$=9 %
  \catcode`\&=14 %
\else
  \catcode`\$=14 %
  \catcode`\&=9 %
\fi
\def\HoLogoCs@LaTeXe#1{%
  LaTeX2%
$ \string ^^^^0395%
& e%
}%
\catcode`\$=3 %
\catcode`\&=4 %
%    \end{macrocode}
%    \end{macro}
%
%    \begin{macro}{\HoLogoBkm@LaTeXe}
%    \begin{macrocode}
\def\HoLogoBkm@LaTeXe#1{%
  \hologo{LaTeX}%
  2%
  \HOLOGO@PdfdocUnicode{e}{\textepsilon}%
}
%    \end{macrocode}
%    \end{macro}
%
%    \begin{macro}{\HoLogoHtml@LaTeXe}
%    \begin{macrocode}
\def\HoLogoHtml@LaTeXe#1{%
  \HoLogoCss@LaTeXe
  \HOLOGO@Span{LaTeX2e}{%
    \hologo{LaTeX}%
    \HOLOGO@Span{2}{2}%
    \HOLOGO@Span{e}{%
      \HOLOGO@MathSetup
      \ensuremath{\textstyle\varepsilon}%
    }%
  }%
}
%    \end{macrocode}
%    \end{macro}
%    \begin{macro}{\HoLogoCss@LaTeXe}
%    \begin{macrocode}
\def\HoLogoCss@LaTeXe{%
  \Css{%
    span.HoLogo-LaTeX2e span.HoLogo-2{%
      padding-left:.15em;%
    }%
  }%
  \Css{%
    span.HoLogo-LaTeX2e span.HoLogo-e{%
      position:relative;%
      top:.35ex;%
      text-decoration:none;%
    }%
  }%
  \global\let\HoLogoCss@LaTeXe\relax
}
%    \end{macrocode}
%    \end{macro}
%
%    \begin{macro}{\HoLogo@LaTeX2e}
%    \begin{macrocode}
\expandafter
\let\csname HoLogo@LaTeX2e\endcsname\HoLogo@LaTeXe
%    \end{macrocode}
%    \end{macro}
%    \begin{macro}{\HoLogoCs@LaTeX2e}
%    \begin{macrocode}
\expandafter
\let\csname HoLogoCs@LaTeX2e\endcsname\HoLogoCs@LaTeXe
%    \end{macrocode}
%    \end{macro}
%    \begin{macro}{\HoLogoBkm@LaTeX2e}
%    \begin{macrocode}
\expandafter
\let\csname HoLogoBkm@LaTeX2e\endcsname\HoLogoBkm@LaTeXe
%    \end{macrocode}
%    \end{macro}
%    \begin{macro}{\HoLogoHtml@LaTeX2e}
%    \begin{macrocode}
\expandafter
\let\csname HoLogoHtml@LaTeX2e\endcsname\HoLogoHtml@LaTeXe
%    \end{macrocode}
%    \end{macro}
%
% \subsubsection{\hologo{LaTeX3}}
%
%    \begin{macro}{\HoLogo@LaTeX3}
%    Source: \hologo{LaTeX} kernel
%    \begin{macrocode}
\expandafter\def\csname HoLogo@LaTeX3\endcsname#1{%
  \hologo{LaTeX}%
  3%
}
%    \end{macrocode}
%    \end{macro}
%
%    \begin{macro}{\HoLogoBkm@LaTeX3}
%    \begin{macrocode}
\expandafter\def\csname HoLogoBkm@LaTeX3\endcsname#1{%
  \hologo{LaTeX}%
  3%
}
%    \end{macrocode}
%    \end{macro}
%    \begin{macro}{\HoLogoHtml@LaTeX3}
%    \begin{macrocode}
\expandafter
\let\csname HoLogoHtml@LaTeX3\expandafter\endcsname
\csname HoLogo@LaTeX3\endcsname
%    \end{macrocode}
%    \end{macro}
%
% \subsubsection{\hologo{LaTeXML}}
%
%    \begin{macro}{\HoLogo@LaTeXML}
%    \begin{macrocode}
\def\HoLogo@LaTeXML#1{%
  \HOLOGO@mbox{%
    \hologo{La}%
    \kern-.15em%
    T%
    \kern-.1667em%
    \lower.5ex\hbox{E}%
    \kern-.125em%
    \HoLogoFont@font{LaTeXML}{sc}{xml}%
  }%
}
%    \end{macrocode}
%    \end{macro}
%    \begin{macro}{\HoLogoHtml@pdfLaTeX}
%    \begin{macrocode}
\def\HoLogoHtml@LaTeXML#1{%
  \HOLOGO@Span{LaTeXML}{%
    \HoLogoCss@LaTeX
    \HoLogoCss@TeX
    \HOLOGO@Span{LaTeX}{%
      L%
      \HOLOGO@Span{a}{%
        A%
      }%
    }%
    \HOLOGO@Span{TeX}{%
      T%
      \HOLOGO@Span{e}{%
        E%
      }%
    }%
    \HCode{<span style="font-variant: small-caps;">}%
    xml%
    \HCode{</span>}%
  }%
}
%    \end{macrocode}
%    \end{macro}
%
% \subsubsection{\hologo{eTeX}}
%
%    \begin{macro}{\HoLogo@eTeX}
%    Source: package \xpackage{etex}
%    \begin{macrocode}
\def\HoLogo@eTeX#1{%
  \ltx@mbox{%
    \HOLOGO@MathSetup
    $\varepsilon$%
    -%
    \HOLOGO@NegativeKerning{-T,T-,To}%
    \hologo{TeX}%
  }%
}
%    \end{macrocode}
%    \end{macro}
%    \begin{macro}{\HoLogoCs@eTeX}
%    \begin{macrocode}
\ifnum64=`\^^^^0040\relax % test for big chars of LuaTeX/XeTeX
  \catcode`\$=9 %
  \catcode`\&=14 %
\else
  \catcode`\$=14 %
  \catcode`\&=9 %
\fi
\def\HoLogoCs@eTeX#1{%
$ #1{\string ^^^^0395}{\string ^^^^03b5}%
& #1{e}{E}%
  TeX%
}%
\catcode`\$=3 %
\catcode`\&=4 %
%    \end{macrocode}
%    \end{macro}
%    \begin{macro}{\HoLogoBkm@eTeX}
%    \begin{macrocode}
\def\HoLogoBkm@eTeX#1{%
  \HOLOGO@PdfdocUnicode{#1{e}{E}}{\textepsilon}%
  -%
  \hologo{TeX}%
}
%    \end{macrocode}
%    \end{macro}
%    \begin{macro}{\HoLogoHtml@eTeX}
%    \begin{macrocode}
\def\HoLogoHtml@eTeX#1{%
  \ltx@mbox{%
    \HOLOGO@MathSetup
    $\varepsilon$%
    -%
    \hologo{TeX}%
  }%
}
%    \end{macrocode}
%    \end{macro}
%
% \subsubsection{\hologo{iniTeX}}
%
%    \begin{macro}{\HoLogo@iniTeX}
%    \begin{macrocode}
\def\HoLogo@iniTeX#1{%
  \HOLOGO@mbox{%
    #1{i}{I}ni\hologo{TeX}%
  }%
}
%    \end{macrocode}
%    \end{macro}
%    \begin{macro}{\HoLogoCs@iniTeX}
%    \begin{macrocode}
\def\HoLogoCs@iniTeX#1{#1{i}{I}niTeX}
%    \end{macrocode}
%    \end{macro}
%    \begin{macro}{\HoLogoBkm@iniTeX}
%    \begin{macrocode}
\def\HoLogoBkm@iniTeX#1{%
  #1{i}{I}ni\hologo{TeX}%
}
%    \end{macrocode}
%    \end{macro}
%    \begin{macro}{\HoLogoHtml@iniTeX}
%    \begin{macrocode}
\let\HoLogoHtml@iniTeX\HoLogo@iniTeX
%    \end{macrocode}
%    \end{macro}
%
% \subsubsection{\hologo{virTeX}}
%
%    \begin{macro}{\HoLogo@virTeX}
%    \begin{macrocode}
\def\HoLogo@virTeX#1{%
  \HOLOGO@mbox{%
    #1{v}{V}ir\hologo{TeX}%
  }%
}
%    \end{macrocode}
%    \end{macro}
%    \begin{macro}{\HoLogoCs@virTeX}
%    \begin{macrocode}
\def\HoLogoCs@virTeX#1{#1{v}{V}irTeX}
%    \end{macrocode}
%    \end{macro}
%    \begin{macro}{\HoLogoBkm@virTeX}
%    \begin{macrocode}
\def\HoLogoBkm@virTeX#1{%
  #1{v}{V}ir\hologo{TeX}%
}
%    \end{macrocode}
%    \end{macro}
%    \begin{macro}{\HoLogoHtml@virTeX}
%    \begin{macrocode}
\let\HoLogoHtml@virTeX\HoLogo@virTeX
%    \end{macrocode}
%    \end{macro}
%
% \subsubsection{\hologo{SliTeX}}
%
% \paragraph{Definitions of the three variants.}
%
%    \begin{macro}{\HoLogo@SLiTeX@lift}
%    \begin{macrocode}
\def\HoLogo@SLiTeX@lift#1{%
  \HoLogoFont@font{SliTeX}{rm}{%
    S%
    \kern-.06em%
    L%
    \kern-.18em%
    \raise.32ex\hbox{\HoLogoFont@font{SliTeX}{sc}{i}}%
    \HOLOGO@discretionary
    \kern-.06em%
    \hologo{TeX}%
  }%
}
%    \end{macrocode}
%    \end{macro}
%    \begin{macro}{\HoLogoBkm@SLiTeX@lift}
%    \begin{macrocode}
\def\HoLogoBkm@SLiTeX@lift#1{SLiTeX}
%    \end{macrocode}
%    \end{macro}
%    \begin{macro}{\HoLogoHtml@SLiTeX@lift}
%    \begin{macrocode}
\def\HoLogoHtml@SLiTeX@lift#1{%
  \HoLogoCss@SLiTeX@lift
  \HOLOGO@Span{SLiTeX-lift}{%
    \HoLogoFont@font{SliTeX}{rm}{%
      S%
      \HOLOGO@Span{L}{L}%
      \HOLOGO@Span{i}{i}%
      \hologo{TeX}%
    }%
  }%
}
%    \end{macrocode}
%    \end{macro}
%    \begin{macro}{\HoLogoCss@SLiTeX@lift}
%    \begin{macrocode}
\def\HoLogoCss@SLiTeX@lift{%
  \Css{%
    span.HoLogo-SLiTeX-lift span.HoLogo-L{%
      margin-left:-.06em;%
      margin-right:-.18em;%
    }%
  }%
  \Css{%
    span.HoLogo-SLiTeX-lift span.HoLogo-i{%
      position:relative;%
      top:-.32ex;%
      margin-right:-.06em;%
      font-variant:small-caps;%
    }%
  }%
  \global\let\HoLogoCss@SLiTeX@lift\relax
}
%    \end{macrocode}
%    \end{macro}
%
%    \begin{macro}{\HoLogo@SliTeX@simple}
%    \begin{macrocode}
\def\HoLogo@SliTeX@simple#1{%
  \HoLogoFont@font{SliTeX}{rm}{%
    \ltx@mbox{%
      \HoLogoFont@font{SliTeX}{sc}{Sli}%
    }%
    \HOLOGO@discretionary
    \hologo{TeX}%
  }%
}
%    \end{macrocode}
%    \end{macro}
%    \begin{macro}{\HoLogoBkm@SliTeX@simple}
%    \begin{macrocode}
\def\HoLogoBkm@SliTeX@simple#1{SliTeX}
%    \end{macrocode}
%    \end{macro}
%    \begin{macro}{\HoLogoHtml@SliTeX@simple}
%    \begin{macrocode}
\let\HoLogoHtml@SliTeX@simple\HoLogo@SliTeX@simple
%    \end{macrocode}
%    \end{macro}
%
%    \begin{macro}{\HoLogo@SliTeX@narrow}
%    \begin{macrocode}
\def\HoLogo@SliTeX@narrow#1{%
  \HoLogoFont@font{SliTeX}{rm}{%
    \ltx@mbox{%
      S%
      \kern-.06em%
      \HoLogoFont@font{SliTeX}{sc}{%
        l%
        \kern-.035em%
        i%
      }%
    }%
    \HOLOGO@discretionary
    \kern-.06em%
    \hologo{TeX}%
  }%
}
%    \end{macrocode}
%    \end{macro}
%    \begin{macro}{\HoLogoBkm@SliTeX@narrow}
%    \begin{macrocode}
\def\HoLogoBkm@SliTeX@narrow#1{SliTeX}
%    \end{macrocode}
%    \end{macro}
%    \begin{macro}{\HoLogoHtml@SliTeX@narrow}
%    \begin{macrocode}
\def\HoLogoHtml@SliTeX@narrow#1{%
  \HoLogoCss@SliTeX@narrow
  \HOLOGO@Span{SliTeX-narrow}{%
    \HoLogoFont@font{SliTeX}{rm}{%
      S%
        \HOLOGO@Span{l}{l}%
        \HOLOGO@Span{i}{i}%
      \hologo{TeX}%
    }%
  }%
}
%    \end{macrocode}
%    \end{macro}
%    \begin{macro}{\HoLogoCss@SliTeX@narrow}
%    \begin{macrocode}
\def\HoLogoCss@SliTeX@narrow{%
  \Css{%
    span.HoLogo-SliTeX-narrow span.HoLogo-l{%
      margin-left:-.06em;%
      margin-right:-.035em;%
      font-variant:small-caps;%
    }%
  }%
  \Css{%
    span.HoLogo-SliTeX-narrow span.HoLogo-i{%
      margin-right:-.06em;%
      font-variant:small-caps;%
    }%
  }%
  \global\let\HoLogoCss@SliTeX@narrow\relax
}
%    \end{macrocode}
%    \end{macro}
%
% \paragraph{Macro set completion.}
%
%    \begin{macro}{\HoLogo@SLiTeX@simple}
%    \begin{macrocode}
\def\HoLogo@SLiTeX@simple{\HoLogo@SliTeX@simple}
%    \end{macrocode}
%    \end{macro}
%    \begin{macro}{\HoLogoBkm@SLiTeX@simple}
%    \begin{macrocode}
\def\HoLogoBkm@SLiTeX@simple{\HoLogoBkm@SliTeX@simple}
%    \end{macrocode}
%    \end{macro}
%    \begin{macro}{\HoLogoHtml@SLiTeX@simple}
%    \begin{macrocode}
\def\HoLogoHtml@SLiTeX@simple{\HoLogoHtml@SliTeX@simple}
%    \end{macrocode}
%    \end{macro}
%
%    \begin{macro}{\HoLogo@SLiTeX@narrow}
%    \begin{macrocode}
\def\HoLogo@SLiTeX@narrow{\HoLogo@SliTeX@narrow}
%    \end{macrocode}
%    \end{macro}
%    \begin{macro}{\HoLogoBkm@SLiTeX@narrow}
%    \begin{macrocode}
\def\HoLogoBkm@SLiTeX@narrow{\HoLogoBkm@SliTeX@narrow}
%    \end{macrocode}
%    \end{macro}
%    \begin{macro}{\HoLogoHtml@SLiTeX@narrow}
%    \begin{macrocode}
\def\HoLogoHtml@SLiTeX@narrow{\HoLogoHtml@SliTeX@narrow}
%    \end{macrocode}
%    \end{macro}
%
%    \begin{macro}{\HoLogo@SliTeX@lift}
%    \begin{macrocode}
\def\HoLogo@SliTeX@lift{\HoLogo@SLiTeX@lift}
%    \end{macrocode}
%    \end{macro}
%    \begin{macro}{\HoLogoBkm@SliTeX@lift}
%    \begin{macrocode}
\def\HoLogoBkm@SliTeX@lift{\HoLogoBkm@SLiTeX@lift}
%    \end{macrocode}
%    \end{macro}
%    \begin{macro}{\HoLogoHtml@SliTeX@lift}
%    \begin{macrocode}
\def\HoLogoHtml@SliTeX@lift{\HoLogoHtml@SLiTeX@lift}
%    \end{macrocode}
%    \end{macro}
%
% \paragraph{Defaults.}
%
%    \begin{macro}{\HoLogo@SLiTeX}
%    \begin{macrocode}
\def\HoLogo@SLiTeX{\HoLogo@SLiTeX@lift}
%    \end{macrocode}
%    \end{macro}
%    \begin{macro}{\HoLogoBkm@SLiTeX}
%    \begin{macrocode}
\def\HoLogoBkm@SLiTeX{\HoLogoBkm@SLiTeX@lift}
%    \end{macrocode}
%    \end{macro}
%    \begin{macro}{\HoLogoHtml@SLiTeX}
%    \begin{macrocode}
\def\HoLogoHtml@SLiTeX{\HoLogoHtml@SLiTeX@lift}
%    \end{macrocode}
%    \end{macro}
%
%    \begin{macro}{\HoLogo@SliTeX}
%    \begin{macrocode}
\def\HoLogo@SliTeX{\HoLogo@SliTeX@narrow}
%    \end{macrocode}
%    \end{macro}
%    \begin{macro}{\HoLogoBkm@SliTeX}
%    \begin{macrocode}
\def\HoLogoBkm@SliTeX{\HoLogoBkm@SliTeX@narrow}
%    \end{macrocode}
%    \end{macro}
%    \begin{macro}{\HoLogoHtml@SliTeX}
%    \begin{macrocode}
\def\HoLogoHtml@SliTeX{\HoLogoHtml@SliTeX@narrow}
%    \end{macrocode}
%    \end{macro}
%
% \subsubsection{\hologo{LuaTeX}}
%
%    \begin{macro}{\HoLogo@LuaTeX}
%    The kerning is an idea of Hans Hagen, see mailing list
%    `luatex at tug dot org' in March 2010.
%    \begin{macrocode}
\def\HoLogo@LuaTeX#1{%
  \HOLOGO@mbox{%
    Lua%
    \HOLOGO@NegativeKerning{aT,oT,To}%
    \hologo{TeX}%
  }%
}
%    \end{macrocode}
%    \end{macro}
%    \begin{macro}{\HoLogoHtml@LuaTeX}
%    \begin{macrocode}
\let\HoLogoHtml@LuaTeX\HoLogo@LuaTeX
%    \end{macrocode}
%    \end{macro}
%
% \subsubsection{\hologo{LuaLaTeX}}
%
%    \begin{macro}{\HoLogo@LuaLaTeX}
%    \begin{macrocode}
\def\HoLogo@LuaLaTeX#1{%
  \HOLOGO@mbox{%
    Lua%
    \hologo{LaTeX}%
  }%
}
%    \end{macrocode}
%    \end{macro}
%    \begin{macro}{\HoLogoHtml@LuaLaTeX}
%    \begin{macrocode}
\let\HoLogoHtml@LuaLaTeX\HoLogo@LuaLaTeX
%    \end{macrocode}
%    \end{macro}
%
% \subsubsection{\hologo{XeTeX}, \hologo{XeLaTeX}}
%
%    \begin{macro}{\HOLOGO@IfCharExists}
%    \begin{macrocode}
\ifluatex
  \ifnum\luatexversion<36 %
  \else
    \def\HOLOGO@IfCharExists#1{%
      \ifnum
        \directlua{%
           if luaotfload and luaotfload.aux then
             if luaotfload.aux.font_has_glyph(%
                    font.current(), \number#1) then % 	 
	       tex.print("1") % 	 
	     end % 	 
	   elseif font and font.fonts and font.current then %
            local f = font.fonts[font.current()]%
            if f.characters and f.characters[\number#1] then %
              tex.print("1")%
            end %
          end%
        }0=\ltx@zero
        \expandafter\ltx@secondoftwo
      \else
        \expandafter\ltx@firstoftwo
      \fi
    }%
  \fi
\fi
\ltx@IfUndefined{HOLOGO@IfCharExists}{%
  \def\HOLOGO@@IfCharExists#1{%
    \begingroup
      \tracinglostchars=\ltx@zero
      \setbox\ltx@zero=\hbox{%
        \kern7sp\char#1\relax
        \ifnum\lastkern>\ltx@zero
          \expandafter\aftergroup\csname iffalse\endcsname
        \else
          \expandafter\aftergroup\csname iftrue\endcsname
        \fi
      }%
      % \if{true|false} from \aftergroup
      \endgroup
      \expandafter\ltx@firstoftwo
    \else
      \endgroup
      \expandafter\ltx@secondoftwo
    \fi
  }%
  \ifxetex
    \ltx@IfUndefined{XeTeXfonttype}{}{%
      \ltx@IfUndefined{XeTeXcharglyph}{}{%
        \def\HOLOGO@IfCharExists#1{%
          \ifnum\XeTeXfonttype\font>\ltx@zero
            \expandafter\ltx@firstofthree
          \else
            \expandafter\ltx@gobble
          \fi
          {%
            \ifnum\XeTeXcharglyph#1>\ltx@zero
              \expandafter\ltx@firstoftwo
            \else
              \expandafter\ltx@secondoftwo
            \fi
          }%
          \HOLOGO@@IfCharExists{#1}%
        }%
      }%
    }%
  \fi
}{}
\ltx@ifundefined{HOLOGO@IfCharExists}{%
  \ifnum64=`\^^^^0040\relax % test for big chars of LuaTeX/XeTeX
    \let\HOLOGO@IfCharExists\HOLOGO@@IfCharExists
  \else
    \def\HOLOGO@IfCharExists#1{%
      \ifnum#1>255 %
        \expandafter\ltx@fourthoffour
      \fi
      \HOLOGO@@IfCharExists{#1}%
    }%
  \fi
}{}
%    \end{macrocode}
%    \end{macro}
%
%    \begin{macro}{\HoLogo@Xe}
%    Source: package \xpackage{dtklogos}
%    \begin{macrocode}
\def\HoLogo@Xe#1{%
  X%
  \kern-.1em\relax
  \HOLOGO@IfCharExists{"018E}{%
    \lower.5ex\hbox{\char"018E}%
  }{%
    \chardef\HOLOGO@choice=\ltx@zero
    \ifdim\fontdimen\ltx@one\font>0pt %
      \ltx@IfUndefined{rotatebox}{%
        \ltx@IfUndefined{pgftext}{%
          \ltx@IfUndefined{psscalebox}{%
            \ltx@IfUndefined{HOLOGO@ScaleBox@\hologoDriver}{%
            }{%
              \chardef\HOLOGO@choice=4 %
            }%
          }{%
            \chardef\HOLOGO@choice=3 %
          }%
        }{%
          \chardef\HOLOGO@choice=2 %
        }%
      }{%
        \chardef\HOLOGO@choice=1 %
      }%
      \ifcase\HOLOGO@choice
        \HOLOGO@WarningUnsupportedDriver{Xe}%
        e%
      \or % 1: \rotatebox
        \begingroup
          \setbox\ltx@zero\hbox{\rotatebox{180}{E}}%
          \ltx@LocDimenA=\dp\ltx@zero
          \advance\ltx@LocDimenA by -.5ex\relax
          \raise\ltx@LocDimenA\box\ltx@zero
        \endgroup
      \or % 2: \pgftext
        \lower.5ex\hbox{%
          \pgfpicture
            \pgftext[rotate=180]{E}%
          \endpgfpicture
        }%
      \or % 3: \psscalebox
        \begingroup
          \setbox\ltx@zero\hbox{\psscalebox{-1 -1}{E}}%
          \ltx@LocDimenA=\dp\ltx@zero
          \advance\ltx@LocDimenA by -.5ex\relax
          \raise\ltx@LocDimenA\box\ltx@zero
        \endgroup
      \or % 4: \HOLOGO@PointReflectBox
        \lower.5ex\hbox{\HOLOGO@PointReflectBox{E}}%
      \else
        \@PackageError{hologo}{Internal error (choice/it}\@ehc
      \fi
    \else
      \ltx@IfUndefined{reflectbox}{%
        \ltx@IfUndefined{pgftext}{%
          \ltx@IfUndefined{psscalebox}{%
            \ltx@IfUndefined{HOLOGO@ScaleBox@\hologoDriver}{%
            }{%
              \chardef\HOLOGO@choice=4 %
            }%
          }{%
            \chardef\HOLOGO@choice=3 %
          }%
        }{%
          \chardef\HOLOGO@choice=2 %
        }%
      }{%
        \chardef\HOLOGO@choice=1 %
      }%
      \ifcase\HOLOGO@choice
        \HOLOGO@WarningUnsupportedDriver{Xe}%
        e%
      \or % 1: reflectbox
        \lower.5ex\hbox{%
          \reflectbox{E}%
        }%
      \or % 2: \pgftext
        \lower.5ex\hbox{%
          \pgfpicture
            \pgftransformxscale{-1}%
            \pgftext{E}%
          \endpgfpicture
        }%
      \or % 3: \psscalebox
        \lower.5ex\hbox{%
          \psscalebox{-1 1}{E}%
        }%
      \or % 4: \HOLOGO@Reflectbox
        \lower.5ex\hbox{%
          \HOLOGO@ReflectBox{E}%
        }%
      \else
        \@PackageError{hologo}{Internal error (choice/up)}\@ehc
      \fi
    \fi
  }%
}
%    \end{macrocode}
%    \end{macro}
%    \begin{macro}{\HoLogoHtml@Xe}
%    \begin{macrocode}
\def\HoLogoHtml@Xe#1{%
  \HoLogoCss@Xe
  \HOLOGO@Span{Xe}{%
    X%
    \HOLOGO@Span{e}{%
      \HCode{&\ltx@hashchar x018e;}%
    }%
  }%
}
%    \end{macrocode}
%    \end{macro}
%    \begin{macro}{\HoLogoCss@Xe}
%    \begin{macrocode}
\def\HoLogoCss@Xe{%
  \Css{%
    span.HoLogo-Xe span.HoLogo-e{%
      position:relative;%
      top:.5ex;%
      left-margin:-.1em;%
    }%
  }%
  \global\let\HoLogoCss@Xe\relax
}
%    \end{macrocode}
%    \end{macro}
%
%    \begin{macro}{\HoLogo@XeTeX}
%    \begin{macrocode}
\def\HoLogo@XeTeX#1{%
  \hologo{Xe}%
  \kern-.15em\relax
  \hologo{TeX}%
}
%    \end{macrocode}
%    \end{macro}
%
%    \begin{macro}{\HoLogoHtml@XeTeX}
%    \begin{macrocode}
\def\HoLogoHtml@XeTeX#1{%
  \HoLogoCss@XeTeX
  \HOLOGO@Span{XeTeX}{%
    \hologo{Xe}%
    \hologo{TeX}%
  }%
}
%    \end{macrocode}
%    \end{macro}
%    \begin{macro}{\HoLogoCss@XeTeX}
%    \begin{macrocode}
\def\HoLogoCss@XeTeX{%
  \Css{%
    span.HoLogo-XeTeX span.HoLogo-TeX{%
      margin-left:-.15em;%
    }%
  }%
  \global\let\HoLogoCss@XeTeX\relax
}
%    \end{macrocode}
%    \end{macro}
%
%    \begin{macro}{\HoLogo@XeLaTeX}
%    \begin{macrocode}
\def\HoLogo@XeLaTeX#1{%
  \hologo{Xe}%
  \kern-.13em%
  \hologo{LaTeX}%
}
%    \end{macrocode}
%    \end{macro}
%    \begin{macro}{\HoLogoHtml@XeLaTeX}
%    \begin{macrocode}
\def\HoLogoHtml@XeLaTeX#1{%
  \HoLogoCss@XeLaTeX
  \HOLOGO@Span{XeLaTeX}{%
    \hologo{Xe}%
    \hologo{LaTeX}%
  }%
}
%    \end{macrocode}
%    \end{macro}
%    \begin{macro}{\HoLogoCss@XeLaTeX}
%    \begin{macrocode}
\def\HoLogoCss@XeLaTeX{%
  \Css{%
    span.HoLogo-XeLaTeX span.HoLogo-Xe{%
      margin-right:-.13em;%
    }%
  }%
  \global\let\HoLogoCss@XeLaTeX\relax
}
%    \end{macrocode}
%    \end{macro}
%
% \subsubsection{\hologo{pdfTeX}, \hologo{pdfLaTeX}}
%
%    \begin{macro}{\HoLogo@pdfTeX}
%    \begin{macrocode}
\def\HoLogo@pdfTeX#1{%
  \HOLOGO@mbox{%
    #1{p}{P}df\hologo{TeX}%
  }%
}
%    \end{macrocode}
%    \end{macro}
%    \begin{macro}{\HoLogoCs@pdfTeX}
%    \begin{macrocode}
\def\HoLogoCs@pdfTeX#1{#1{p}{P}dfTeX}
%    \end{macrocode}
%    \end{macro}
%    \begin{macro}{\HoLogoBkm@pdfTeX}
%    \begin{macrocode}
\def\HoLogoBkm@pdfTeX#1{%
  #1{p}{P}df\hologo{TeX}%
}
%    \end{macrocode}
%    \end{macro}
%    \begin{macro}{\HoLogoHtml@pdfTeX}
%    \begin{macrocode}
\let\HoLogoHtml@pdfTeX\HoLogo@pdfTeX
%    \end{macrocode}
%    \end{macro}
%
%    \begin{macro}{\HoLogo@pdfLaTeX}
%    \begin{macrocode}
\def\HoLogo@pdfLaTeX#1{%
  \HOLOGO@mbox{%
    #1{p}{P}df\hologo{LaTeX}%
  }%
}
%    \end{macrocode}
%    \end{macro}
%    \begin{macro}{\HoLogoCs@pdfLaTeX}
%    \begin{macrocode}
\def\HoLogoCs@pdfLaTeX#1{#1{p}{P}dfLaTeX}
%    \end{macrocode}
%    \end{macro}
%    \begin{macro}{\HoLogoBkm@pdfLaTeX}
%    \begin{macrocode}
\def\HoLogoBkm@pdfLaTeX#1{%
  #1{p}{P}df\hologo{LaTeX}%
}
%    \end{macrocode}
%    \end{macro}
%    \begin{macro}{\HoLogoHtml@pdfLaTeX}
%    \begin{macrocode}
\let\HoLogoHtml@pdfLaTeX\HoLogo@pdfLaTeX
%    \end{macrocode}
%    \end{macro}
%
% \subsubsection{\hologo{VTeX}}
%
%    \begin{macro}{\HoLogo@VTeX}
%    \begin{macrocode}
\def\HoLogo@VTeX#1{%
  \HOLOGO@mbox{%
    V\hologo{TeX}%
  }%
}
%    \end{macrocode}
%    \end{macro}
%    \begin{macro}{\HoLogoHtml@VTeX}
%    \begin{macrocode}
\let\HoLogoHtml@VTeX\HoLogo@VTeX
%    \end{macrocode}
%    \end{macro}
%
% \subsubsection{\hologo{AmS}, \dots}
%
%    Source: class \xclass{amsdtx}
%
%    \begin{macro}{\HoLogo@AmS}
%    \begin{macrocode}
\def\HoLogo@AmS#1{%
  \HoLogoFont@font{AmS}{sy}{%
    A%
    \kern-.1667em%
    \lower.5ex\hbox{M}%
    \kern-.125em%
    S%
  }%
}
%    \end{macrocode}
%    \end{macro}
%    \begin{macro}{\HoLogoBkm@AmS}
%    \begin{macrocode}
\def\HoLogoBkm@AmS#1{AmS}
%    \end{macrocode}
%    \end{macro}
%    \begin{macro}{\HoLogoHtml@AmS}
%    \begin{macrocode}
\def\HoLogoHtml@AmS#1{%
  \HoLogoCss@AmS
%  \HoLogoFont@font{AmS}{sy}{%
    \HOLOGO@Span{AmS}{%
      A%
      \HOLOGO@Span{M}{M}%
      S%
    }%
%   }%
}
%    \end{macrocode}
%    \end{macro}
%    \begin{macro}{\HoLogoCss@AmS}
%    \begin{macrocode}
\def\HoLogoCss@AmS{%
  \Css{%
    span.HoLogo-AmS span.HoLogo-M{%
      position:relative;%
      top:.5ex;%
      margin-left:-.1667em;%
      margin-right:-.125em;%
      text-decoration:none;%
    }%
  }%
  \global\let\HoLogoCss@AmS\relax
}
%    \end{macrocode}
%    \end{macro}
%
%    \begin{macro}{\HoLogo@AmSTeX}
%    \begin{macrocode}
\def\HoLogo@AmSTeX#1{%
  \hologo{AmS}%
  \HOLOGO@hyphen
  \hologo{TeX}%
}
%    \end{macrocode}
%    \end{macro}
%    \begin{macro}{\HoLogoBkm@AmSTeX}
%    \begin{macrocode}
\def\HoLogoBkm@AmSTeX#1{AmS-TeX}%
%    \end{macrocode}
%    \end{macro}
%    \begin{macro}{\HoLogoHtml@AmSTeX}
%    \begin{macrocode}
\let\HoLogoHtml@AmSTeX\HoLogo@AmSTeX
%    \end{macrocode}
%    \end{macro}
%
%    \begin{macro}{\HoLogo@AmSLaTeX}
%    \begin{macrocode}
\def\HoLogo@AmSLaTeX#1{%
  \hologo{AmS}%
  \HOLOGO@hyphen
  \hologo{LaTeX}%
}
%    \end{macrocode}
%    \end{macro}
%    \begin{macro}{\HoLogoBkm@AmSLaTeX}
%    \begin{macrocode}
\def\HoLogoBkm@AmSLaTeX#1{AmS-LaTeX}%
%    \end{macrocode}
%    \end{macro}
%    \begin{macro}{\HoLogoHtml@AmSLaTeX}
%    \begin{macrocode}
\let\HoLogoHtml@AmSLaTeX\HoLogo@AmSLaTeX
%    \end{macrocode}
%    \end{macro}
%
% \subsubsection{\hologo{BibTeX}}
%
%    \begin{macro}{\HoLogo@BibTeX@sc}
%    A definition of \hologo{BibTeX} is provided in
%    the documentation source for the manual of \hologo{BibTeX}
%    \cite{btxdoc}.
%\begin{quote}
%\begin{verbatim}
%\def\BibTeX{%
%  {%
%    \rm
%    B%
%    \kern-.05em%
%    {%
%      \sc
%      i%
%      \kern-.025em %
%      b%
%    }%
%    \kern-.08em
%    T%
%    \kern-.1667em%
%    \lower.7ex\hbox{E}%
%    \kern-.125em%
%    X%
%  }%
%}
%\end{verbatim}
%\end{quote}
%    \begin{macrocode}
\def\HoLogo@BibTeX@sc#1{%
  B%
  \kern-.05em%
  \HoLogoFont@font{BibTeX}{sc}{%
    i%
    \kern-.025em%
    b%
  }%
  \HOLOGO@discretionary
  \kern-.08em%
  \hologo{TeX}%
}
%    \end{macrocode}
%    \end{macro}
%    \begin{macro}{\HoLogoHtml@BibTeX@sc}
%    \begin{macrocode}
\def\HoLogoHtml@BibTeX@sc#1{%
  \HoLogoCss@BibTeX@sc
  \HOLOGO@Span{BibTeX-sc}{%
    B%
    \HOLOGO@Span{i}{i}%
    \HOLOGO@Span{b}{b}%
    \hologo{TeX}%
  }%
}
%    \end{macrocode}
%    \end{macro}
%    \begin{macro}{\HoLogoCss@BibTeX@sc}
%    \begin{macrocode}
\def\HoLogoCss@BibTeX@sc{%
  \Css{%
    span.HoLogo-BibTeX-sc span.HoLogo-i{%
      margin-left:-.05em;%
      margin-right:-.025em;%
      font-variant:small-caps;%
    }%
  }%
  \Css{%
    span.HoLogo-BibTeX-sc span.HoLogo-b{%
      margin-right:-.08em;%
      font-variant:small-caps;%
    }%
  }%
  \global\let\HoLogoCss@BibTeX@sc\relax
}
%    \end{macrocode}
%    \end{macro}
%
%    \begin{macro}{\HoLogo@BibTeX@sf}
%    Variant \xoption{sf} avoids trouble with unavailable
%    small caps fonts (e.g., bold versions of Computer Modern or
%    Latin Modern). The definition is taken from
%    package \xpackage{dtklogos} \cite{dtklogos}.
%\begin{quote}
%\begin{verbatim}
%\DeclareRobustCommand{\BibTeX}{%
%  B%
%  \kern-.05em%
%  \hbox{%
%    $\m@th$% %% force math size calculations
%    \csname S@\f@size\endcsname
%    \fontsize\sf@size\z@
%    \math@fontsfalse
%    \selectfont
%    I%
%    \kern-.025em%
%    B
%  }%
%  \kern-.08em%
%  \-%
%  \TeX
%}
%\end{verbatim}
%\end{quote}
%    \begin{macrocode}
\def\HoLogo@BibTeX@sf#1{%
  B%
  \kern-.05em%
  \HoLogoFont@font{BibTeX}{bibsf}{%
    I%
    \kern-.025em%
    B%
  }%
  \HOLOGO@discretionary
  \kern-.08em%
  \hologo{TeX}%
}
%    \end{macrocode}
%    \end{macro}
%    \begin{macro}{\HoLogoHtml@BibTeX@sf}
%    \begin{macrocode}
\def\HoLogoHtml@BibTeX@sf#1{%
  \HoLogoCss@BibTeX@sf
  \HOLOGO@Span{BibTeX-sf}{%
    B%
    \HoLogoFont@font{BibTeX}{bibsf}{%
      \HOLOGO@Span{i}{I}%
      B%
    }%
    \hologo{TeX}%
  }%
}
%    \end{macrocode}
%    \end{macro}
%    \begin{macro}{\HoLogoCss@BibTeX@sf}
%    \begin{macrocode}
\def\HoLogoCss@BibTeX@sf{%
  \Css{%
    span.HoLogo-BibTeX-sf span.HoLogo-i{%
      margin-left:-.05em;%
      margin-right:-.025em;%
    }%
  }%
  \Css{%
    span.HoLogo-BibTeX-sf span.HoLogo-TeX{%
      margin-left:-.08em;%
    }%
  }%
  \global\let\HoLogoCss@BibTeX@sf\relax
}
%    \end{macrocode}
%    \end{macro}
%
%    \begin{macro}{\HoLogo@BibTeX}
%    \begin{macrocode}
\def\HoLogo@BibTeX{\HoLogo@BibTeX@sf}
%    \end{macrocode}
%    \end{macro}
%    \begin{macro}{\HoLogoHtml@BibTeX}
%    \begin{macrocode}
\def\HoLogoHtml@BibTeX{\HoLogoHtml@BibTeX@sf}
%    \end{macrocode}
%    \end{macro}
%
% \subsubsection{\hologo{BibTeX8}}
%
%    \begin{macro}{\HoLogo@BibTeX8}
%    \begin{macrocode}
\expandafter\def\csname HoLogo@BibTeX8\endcsname#1{%
  \hologo{BibTeX}%
  8%
}
%    \end{macrocode}
%    \end{macro}
%
%    \begin{macro}{\HoLogoBkm@BibTeX8}
%    \begin{macrocode}
\expandafter\def\csname HoLogoBkm@BibTeX8\endcsname#1{%
  \hologo{BibTeX}%
  8%
}
%    \end{macrocode}
%    \end{macro}
%    \begin{macro}{\HoLogoHtml@BibTeX8}
%    \begin{macrocode}
\expandafter
\let\csname HoLogoHtml@BibTeX8\expandafter\endcsname
\csname HoLogo@BibTeX8\endcsname
%    \end{macrocode}
%    \end{macro}
%
% \subsubsection{\hologo{ConTeXt}}
%
%    \begin{macro}{\HoLogo@ConTeXt@simple}
%    \begin{macrocode}
\def\HoLogo@ConTeXt@simple#1{%
  \HOLOGO@mbox{Con}%
  \HOLOGO@discretionary
  \HOLOGO@mbox{\hologo{TeX}t}%
}
%    \end{macrocode}
%    \end{macro}
%    \begin{macro}{\HoLogoHtml@ConTeXt@simple}
%    \begin{macrocode}
\let\HoLogoHtml@ConTeXt@simple\HoLogo@ConTeXt@simple
%    \end{macrocode}
%    \end{macro}
%
%    \begin{macro}{\HoLogo@ConTeXt@narrow}
%    This definition of logo \hologo{ConTeXt} with variant \xoption{narrow}
%    comes from TUGboat's class \xclass{ltugboat} (version 2010/11/15 v2.8).
%    \begin{macrocode}
\def\HoLogo@ConTeXt@narrow#1{%
  \HOLOGO@mbox{C\kern-.0333emon}%
  \HOLOGO@discretionary
  \kern-.0667em%
  \HOLOGO@mbox{\hologo{TeX}\kern-.0333emt}%
}
%    \end{macrocode}
%    \end{macro}
%    \begin{macro}{\HoLogoHtml@ConTeXt@narrow}
%    \begin{macrocode}
\def\HoLogoHtml@ConTeXt@narrow#1{%
  \HoLogoCss@ConTeXt@narrow
  \HOLOGO@Span{ConTeXt-narrow}{%
    \HOLOGO@Span{C}{C}%
    on%
    \hologo{TeX}%
    t%
  }%
}
%    \end{macrocode}
%    \end{macro}
%    \begin{macro}{\HoLogoCss@ConTeXt@narrow}
%    \begin{macrocode}
\def\HoLogoCss@ConTeXt@narrow{%
  \Css{%
    span.HoLogo-ConTeXt-narrow span.HoLogo-C{%
      margin-left:-.0333em;%
    }%
  }%
  \Css{%
    span.HoLogo-ConTeXt-narrow span.HoLogo-TeX{%
      margin-left:-.0667em;%
      margin-right:-.0333em;%
    }%
  }%
  \global\let\HoLogoCss@ConTeXt@narrow\relax
}
%    \end{macrocode}
%    \end{macro}
%
%    \begin{macro}{\HoLogo@ConTeXt}
%    \begin{macrocode}
\def\HoLogo@ConTeXt{\HoLogo@ConTeXt@narrow}
%    \end{macrocode}
%    \end{macro}
%    \begin{macro}{\HoLogoHtml@ConTeXt}
%    \begin{macrocode}
\def\HoLogoHtml@ConTeXt{\HoLogoHtml@ConTeXt@narrow}
%    \end{macrocode}
%    \end{macro}
%
% \subsubsection{\hologo{emTeX}}
%
%    \begin{macro}{\HoLogo@emTeX}
%    \begin{macrocode}
\def\HoLogo@emTeX#1{%
  \HOLOGO@mbox{#1{e}{E}m}%
  \HOLOGO@discretionary
  \hologo{TeX}%
}
%    \end{macrocode}
%    \end{macro}
%    \begin{macro}{\HoLogoCs@emTeX}
%    \begin{macrocode}
\def\HoLogoCs@emTeX#1{#1{e}{E}mTeX}%
%    \end{macrocode}
%    \end{macro}
%    \begin{macro}{\HoLogoBkm@emTeX}
%    \begin{macrocode}
\def\HoLogoBkm@emTeX#1{%
  #1{e}{E}m\hologo{TeX}%
}
%    \end{macrocode}
%    \end{macro}
%    \begin{macro}{\HoLogoHtml@emTeX}
%    \begin{macrocode}
\let\HoLogoHtml@emTeX\HoLogo@emTeX
%    \end{macrocode}
%    \end{macro}
%
% \subsubsection{\hologo{ExTeX}}
%
%    \begin{macro}{\HoLogo@ExTeX}
%    The definition is taken from the FAQ of the
%    project \hologo{ExTeX}
%    \cite{ExTeX-FAQ}.
%\begin{quote}
%\begin{verbatim}
%\def\ExTeX{%
%  \textrm{% Logo always with serifs
%    \ensuremath{%
%      \textstyle
%      \varepsilon_{%
%        \kern-0.15em%
%        \mathcal{X}%
%      }%
%    }%
%    \kern-.15em%
%    \TeX
%  }%
%}
%\end{verbatim}
%\end{quote}
%    \begin{macrocode}
\def\HoLogo@ExTeX#1{%
  \HoLogoFont@font{ExTeX}{rm}{%
    \ltx@mbox{%
      \HOLOGO@MathSetup
      $%
        \textstyle
        \varepsilon_{%
          \kern-0.15em%
          \HoLogoFont@font{ExTeX}{sy}{X}%
        }%
      $%
    }%
    \HOLOGO@discretionary
    \kern-.15em%
    \hologo{TeX}%
  }%
}
%    \end{macrocode}
%    \end{macro}
%    \begin{macro}{\HoLogoHtml@ExTeX}
%    \begin{macrocode}
\def\HoLogoHtml@ExTeX#1{%
  \HoLogoCss@ExTeX
  \HoLogoFont@font{ExTeX}{rm}{%
    \HOLOGO@Span{ExTeX}{%
      \ltx@mbox{%
        \HOLOGO@MathSetup
        $\textstyle\varepsilon$%
        \HOLOGO@Span{X}{$\textstyle\chi$}%
        \hologo{TeX}%
      }%
    }%
  }%
}
%    \end{macrocode}
%    \end{macro}
%    \begin{macro}{\HoLogoBkm@ExTeX}
%    \begin{macrocode}
\def\HoLogoBkm@ExTeX#1{%
  \HOLOGO@PdfdocUnicode{#1{e}{E}x}{\textepsilon\textchi}%
  \hologo{TeX}%
}
%    \end{macrocode}
%    \end{macro}
%    \begin{macro}{\HoLogoCss@ExTeX}
%    \begin{macrocode}
\def\HoLogoCss@ExTeX{%
  \Css{%
    span.HoLogo-ExTeX{%
      font-family:serif;%
    }%
  }%
  \Css{%
    span.HoLogo-ExTeX span.HoLogo-TeX{%
      margin-left:-.15em;%
    }%
  }%
  \global\let\HoLogoCss@ExTeX\relax
}
%    \end{macrocode}
%    \end{macro}
%
% \subsubsection{\hologo{MiKTeX}}
%
%    \begin{macro}{\HoLogo@MiKTeX}
%    \begin{macrocode}
\def\HoLogo@MiKTeX#1{%
  \HOLOGO@mbox{MiK}%
  \HOLOGO@discretionary
  \hologo{TeX}%
}
%    \end{macrocode}
%    \end{macro}
%    \begin{macro}{\HoLogoHtml@MiKTeX}
%    \begin{macrocode}
\let\HoLogoHtml@MiKTeX\HoLogo@MiKTeX
%    \end{macrocode}
%    \end{macro}
%
% \subsubsection{\hologo{OzTeX} and friends}
%
%    Source: \hologo{OzTeX} FAQ \cite{OzTeX}:
%    \begin{quote}
%      |\def\OzTeX{O\kern-.03em z\kern-.15em\TeX}|\\
%      (There is no kerning in OzMF, OzMP and OzTtH.)
%    \end{quote}
%
%    \begin{macro}{\HoLogo@OzTeX}
%    \begin{macrocode}
\def\HoLogo@OzTeX#1{%
  O%
  \kern-.03em %
  z%
  \kern-.15em %
  \hologo{TeX}%
}
%    \end{macrocode}
%    \end{macro}
%    \begin{macro}{\HoLogoHtml@OzTeX}
%    \begin{macrocode}
\def\HoLogoHtml@OzTeX#1{%
  \HoLogoCss@OzTeX
  \HOLOGO@Span{OzTeX}{%
    O%
    \HOLOGO@Span{z}{z}%
    \hologo{TeX}%
  }%
}
%    \end{macrocode}
%    \end{macro}
%    \begin{macro}{\HoLogoCss@OzTeX}
%    \begin{macrocode}
\def\HoLogoCss@OzTeX{%
  \Css{%
    span.HoLogo-OzTeX span.HoLogo-z{%
      margin-left:-.03em;%
      margin-right:-.15em;%
    }%
  }%
  \global\let\HoLogoCss@OzTeX\relax
}
%    \end{macrocode}
%    \end{macro}
%
%    \begin{macro}{\HoLogo@OzMF}
%    \begin{macrocode}
\def\HoLogo@OzMF#1{%
  \HOLOGO@mbox{OzMF}%
}
%    \end{macrocode}
%    \end{macro}
%    \begin{macro}{\HoLogo@OzMP}
%    \begin{macrocode}
\def\HoLogo@OzMP#1{%
  \HOLOGO@mbox{OzMP}%
}
%    \end{macrocode}
%    \end{macro}
%    \begin{macro}{\HoLogo@OzTtH}
%    \begin{macrocode}
\def\HoLogo@OzTtH#1{%
  \HOLOGO@mbox{OzTtH}%
}
%    \end{macrocode}
%    \end{macro}
%
% \subsubsection{\hologo{PCTeX}}
%
%    \begin{macro}{\HoLogo@PCTeX}
%    \begin{macrocode}
\def\HoLogo@PCTeX#1{%
  \HOLOGO@mbox{PC}%
  \hologo{TeX}%
}
%    \end{macrocode}
%    \end{macro}
%    \begin{macro}{\HoLogoHtml@PCTeX}
%    \begin{macrocode}
\let\HoLogoHtml@PCTeX\HoLogo@PCTeX
%    \end{macrocode}
%    \end{macro}
%
% \subsubsection{\hologo{PiCTeX}}
%
%    The original definitions from \xfile{pictex.tex} \cite{PiCTeX}:
%\begin{quote}
%\begin{verbatim}
%\def\PiC{%
%  P%
%  \kern-.12em%
%  \lower.5ex\hbox{I}%
%  \kern-.075em%
%  C%
%}
%\def\PiCTeX{%
%  \PiC
%  \kern-.11em%
%  \TeX
%}
%\end{verbatim}
%\end{quote}
%
%    \begin{macro}{\HoLogo@PiC}
%    \begin{macrocode}
\def\HoLogo@PiC#1{%
  P%
  \kern-.12em%
  \lower.5ex\hbox{I}%
  \kern-.075em%
  C%
  \HOLOGO@SpaceFactor
}
%    \end{macrocode}
%    \end{macro}
%    \begin{macro}{\HoLogoHtml@PiC}
%    \begin{macrocode}
\def\HoLogoHtml@PiC#1{%
  \HoLogoCss@PiC
  \HOLOGO@Span{PiC}{%
    P%
    \HOLOGO@Span{i}{I}%
    C%
  }%
}
%    \end{macrocode}
%    \end{macro}
%    \begin{macro}{\HoLogoCss@PiC}
%    \begin{macrocode}
\def\HoLogoCss@PiC{%
  \Css{%
    span.HoLogo-PiC span.HoLogo-i{%
      position:relative;%
      top:.5ex;%
      margin-left:-.12em;%
      margin-right:-.075em;%
      text-decoration:none;%
    }%
  }%
  \global\let\HoLogoCss@PiC\relax
}
%    \end{macrocode}
%    \end{macro}
%
%    \begin{macro}{\HoLogo@PiCTeX}
%    \begin{macrocode}
\def\HoLogo@PiCTeX#1{%
  \hologo{PiC}%
  \HOLOGO@discretionary
  \kern-.11em%
  \hologo{TeX}%
}
%    \end{macrocode}
%    \end{macro}
%    \begin{macro}{\HoLogoHtml@PiCTeX}
%    \begin{macrocode}
\def\HoLogoHtml@PiCTeX#1{%
  \HoLogoCss@PiCTeX
  \HOLOGO@Span{PiCTeX}{%
    \hologo{PiC}%
    \hologo{TeX}%
  }%
}
%    \end{macrocode}
%    \end{macro}
%    \begin{macro}{\HoLogoCss@PiCTeX}
%    \begin{macrocode}
\def\HoLogoCss@PiCTeX{%
  \Css{%
    span.HoLogo-PiCTeX span.HoLogo-PiC{%
      margin-right:-.11em;%
    }%
  }%
  \global\let\HoLogoCss@PiCTeX\relax
}
%    \end{macrocode}
%    \end{macro}
%
% \subsubsection{\hologo{teTeX}}
%
%    \begin{macro}{\HoLogo@teTeX}
%    \begin{macrocode}
\def\HoLogo@teTeX#1{%
  \HOLOGO@mbox{#1{t}{T}e}%
  \HOLOGO@discretionary
  \hologo{TeX}%
}
%    \end{macrocode}
%    \end{macro}
%    \begin{macro}{\HoLogoCs@teTeX}
%    \begin{macrocode}
\def\HoLogoCs@teTeX#1{#1{t}{T}dfTeX}
%    \end{macrocode}
%    \end{macro}
%    \begin{macro}{\HoLogoBkm@teTeX}
%    \begin{macrocode}
\def\HoLogoBkm@teTeX#1{%
  #1{t}{T}e\hologo{TeX}%
}
%    \end{macrocode}
%    \end{macro}
%    \begin{macro}{\HoLogoHtml@teTeX}
%    \begin{macrocode}
\let\HoLogoHtml@teTeX\HoLogo@teTeX
%    \end{macrocode}
%    \end{macro}
%
% \subsubsection{\hologo{TeX4ht}}
%
%    \begin{macro}{\HoLogo@TeX4ht}
%    \begin{macrocode}
\expandafter\def\csname HoLogo@TeX4ht\endcsname#1{%
  \HOLOGO@mbox{\hologo{TeX}4ht}%
}
%    \end{macrocode}
%    \end{macro}
%    \begin{macro}{\HoLogoHtml@TeX4ht}
%    \begin{macrocode}
\expandafter
\let\csname HoLogoHtml@TeX4ht\expandafter\endcsname
\csname HoLogo@TeX4ht\endcsname
%    \end{macrocode}
%    \end{macro}
%
%
% \subsubsection{\hologo{SageTeX}}
%
%    \begin{macro}{\HoLogo@SageTeX}
%    \begin{macrocode}
\def\HoLogo@SageTeX#1{%
  \HOLOGO@mbox{Sage}%
  \HOLOGO@discretionary
  \HOLOGO@NegativeKerning{eT,oT,To}%
  \hologo{TeX}%
}
%    \end{macrocode}
%    \end{macro}
%    \begin{macro}{\HoLogoHtml@SageTeX}
%    \begin{macrocode}
\let\HoLogoHtml@SageTeX\HoLogo@SageTeX
%    \end{macrocode}
%    \end{macro}
%
% \subsection{\hologo{METAFONT} and friends}
%
%    \begin{macro}{\HoLogo@METAFONT}
%    \begin{macrocode}
\def\HoLogo@METAFONT#1{%
  \HoLogoFont@font{METAFONT}{logo}{%
    \HOLOGO@mbox{META}%
    \HOLOGO@discretionary
    \HOLOGO@mbox{FONT}%
  }%
}
%    \end{macrocode}
%    \end{macro}
%
%    \begin{macro}{\HoLogo@METAPOST}
%    \begin{macrocode}
\def\HoLogo@METAPOST#1{%
  \HoLogoFont@font{METAPOST}{logo}{%
    \HOLOGO@mbox{META}%
    \HOLOGO@discretionary
    \HOLOGO@mbox{POST}%
  }%
}
%    \end{macrocode}
%    \end{macro}
%
%    \begin{macro}{\HoLogo@MetaFun}
%    \begin{macrocode}
\def\HoLogo@MetaFun#1{%
  \HOLOGO@mbox{Meta}%
  \HOLOGO@discretionary
  \HOLOGO@mbox{Fun}%
}
%    \end{macrocode}
%    \end{macro}
%
%    \begin{macro}{\HoLogo@MetaPost}
%    \begin{macrocode}
\def\HoLogo@MetaPost#1{%
  \HOLOGO@mbox{Meta}%
  \HOLOGO@discretionary
  \HOLOGO@mbox{Post}%
}
%    \end{macrocode}
%    \end{macro}
%
% \subsection{Others}
%
% \subsubsection{\hologo{biber}}
%
%    \begin{macro}{\HoLogo@biber}
%    \begin{macrocode}
\def\HoLogo@biber#1{%
  \HOLOGO@mbox{#1{b}{B}i}%
  \HOLOGO@discretionary
  \HOLOGO@mbox{ber}%
}
%    \end{macrocode}
%    \end{macro}
%    \begin{macro}{\HoLogoCs@biber}
%    \begin{macrocode}
\def\HoLogoCs@biber#1{#1{b}{B}iber}
%    \end{macrocode}
%    \end{macro}
%    \begin{macro}{\HoLogoBkm@biber}
%    \begin{macrocode}
\def\HoLogoBkm@biber#1{%
  #1{b}{B}iber%
}
%    \end{macrocode}
%    \end{macro}
%    \begin{macro}{\HoLogoHtml@biber}
%    \begin{macrocode}
\let\HoLogoHtml@biber\HoLogo@biber
%    \end{macrocode}
%    \end{macro}
%
% \subsubsection{\hologo{KOMAScript}}
%
%    \begin{macro}{\HoLogo@KOMAScript}
%    The definition for \hologo{KOMAScript} is taken
%    from \hologo{KOMAScript} (\xfile{scrlogo.dtx}, reformatted) \cite{scrlogo}:
%\begin{quote}
%\begin{verbatim}
%\@ifundefined{KOMAScript}{%
%  \DeclareRobustCommand{\KOMAScript}{%
%    \textsf{%
%      K\kern.05em O\kern.05emM\kern.05em A%
%      \kern.1em-\kern.1em %
%      Script%
%    }%
%  }%
%}{}
%\end{verbatim}
%\end{quote}
%    \begin{macrocode}
\def\HoLogo@KOMAScript#1{%
  \HoLogoFont@font{KOMAScript}{sf}{%
    \HOLOGO@mbox{%
      K\kern.05em%
      O\kern.05em%
      M\kern.05em%
      A%
    }%
    \kern.1em%
    \HOLOGO@hyphen
    \kern.1em%
    \HOLOGO@mbox{Script}%
  }%
}
%    \end{macrocode}
%    \end{macro}
%    \begin{macro}{\HoLogoBkm@KOMAScript}
%    \begin{macrocode}
\def\HoLogoBkm@KOMAScript#1{%
  KOMA-Script%
}
%    \end{macrocode}
%    \end{macro}
%    \begin{macro}{\HoLogoHtml@KOMAScript}
%    \begin{macrocode}
\def\HoLogoHtml@KOMAScript#1{%
  \HoLogoCss@KOMAScript
  \HoLogoFont@font{KOMAScript}{sf}{%
    \HOLOGO@Span{KOMAScript}{%
      K%
      \HOLOGO@Span{O}{O}%
      M%
      \HOLOGO@Span{A}{A}%
      \HOLOGO@Span{hyphen}{-}%
      Script%
    }%
  }%
}
%    \end{macrocode}
%    \end{macro}
%    \begin{macro}{\HoLogoCss@KOMAScript}
%    \begin{macrocode}
\def\HoLogoCss@KOMAScript{%
  \Css{%
    span.HoLogo-KOMAScript{%
      font-family:sans-serif;%
    }%
  }%
  \Css{%
    span.HoLogo-KOMAScript span.HoLogo-O{%
      padding-left:.05em;%
      padding-right:.05em;%
    }%
  }%
  \Css{%
    span.HoLogo-KOMAScript span.HoLogo-A{%
      padding-left:.05em;%
    }%
  }%
  \Css{%
    span.HoLogo-KOMAScript span.HoLogo-hyphen{%
      padding-left:.1em;%
      padding-right:.1em;%
    }%
  }%
  \global\let\HoLogoCss@KOMAScript\relax
}
%    \end{macrocode}
%    \end{macro}
%
% \subsubsection{\hologo{LyX}}
%
%    \begin{macro}{\HoLogo@LyX}
%    The definition is taken from the documentation source files
%    of \hologo{LyX}, \xfile{Intro.lyx} \cite{LyX}:
%\begin{quote}
%\begin{verbatim}
%\def\LyX{%
%  \texorpdfstring{%
%    L\kern-.1667em\lower.25em\hbox{Y}\kern-.125emX\@%
%  }{%
%    LyX%
%  }%
%}
%\end{verbatim}
%\end{quote}
%    \begin{macrocode}
\def\HoLogo@LyX#1{%
  L%
  \kern-.1667em%
  \lower.25em\hbox{Y}%
  \kern-.125em%
  X%
  \HOLOGO@SpaceFactor
}
%    \end{macrocode}
%    \end{macro}
%    \begin{macro}{\HoLogoHtml@LyX}
%    \begin{macrocode}
\def\HoLogoHtml@LyX#1{%
  \HoLogoCss@LyX
  \HOLOGO@Span{LyX}{%
    L%
    \HOLOGO@Span{y}{Y}%
    X%
  }%
}
%    \end{macrocode}
%    \end{macro}
%    \begin{macro}{\HoLogoCss@LyX}
%    \begin{macrocode}
\def\HoLogoCss@LyX{%
  \Css{%
    span.HoLogo-LyX span.HoLogo-y{%
      position:relative;%
      top:.25em;%
      margin-left:-.1667em;%
      margin-right:-.125em;%
      text-decoration:none;%
    }%
  }%
  \global\let\HoLogoCss@LyX\relax
}
%    \end{macrocode}
%    \end{macro}
%
% \subsubsection{\hologo{NTS}}
%
%    \begin{macro}{\HoLogo@NTS}
%    Definition for \hologo{NTS} can be found in
%    package \xpackage{etex\textunderscore man} for the \hologo{eTeX} manual \cite{etexman}
%    and in package \xpackage{dtklogos} \cite{dtklogos}:
%\begin{quote}
%\begin{verbatim}
%\def\NTS{%
%  \leavevmode
%  \hbox{%
%    $%
%      \cal N%
%      \kern-0.35em%
%      \lower0.5ex\hbox{$\cal T$}%
%      \kern-0.2em%
%      S%
%    $%
%  }%
%}
%\end{verbatim}
%\end{quote}
%    \begin{macrocode}
\def\HoLogo@NTS#1{%
  \HoLogoFont@font{NTS}{sy}{%
    N\/%
    \kern-.35em%
    \lower.5ex\hbox{T\/}%
    \kern-.2em%
    S\/%
  }%
  \HOLOGO@SpaceFactor
}
%    \end{macrocode}
%    \end{macro}
%
% \subsubsection{\Hologo{TTH} (\hologo{TeX} to HTML translator)}
%
%    Source: \url{http://hutchinson.belmont.ma.us/tth/}
%    In the HTML source the second `T' is printed as subscript.
%\begin{quote}
%\begin{verbatim}
%T<sub>T</sub>H
%\end{verbatim}
%\end{quote}
%    \begin{macro}{\HoLogo@TTH}
%    \begin{macrocode}
\def\HoLogo@TTH#1{%
  \ltx@mbox{%
    T\HOLOGO@SubScript{T}H%
  }%
  \HOLOGO@SpaceFactor
}
%    \end{macrocode}
%    \end{macro}
%
%    \begin{macro}{\HoLogoHtml@TTH}
%    \begin{macrocode}
\def\HoLogoHtml@TTH#1{%
  T\HCode{<sub>}T\HCode{</sub>}H%
}
%    \end{macrocode}
%    \end{macro}
%
% \subsubsection{\Hologo{HanTheThanh}}
%
%    Partial source: Package \xpackage{dtklogos}.
%    The double accent is U+1EBF (latin small letter e with circumflex
%    and acute).
%    \begin{macro}{\HoLogo@HanTheThanh}
%    \begin{macrocode}
\def\HoLogo@HanTheThanh#1{%
  \ltx@mbox{H\`an}%
  \HOLOGO@space
  \ltx@mbox{%
    Th%
    \HOLOGO@IfCharExists{"1EBF}{%
      \char"1EBF\relax
    }{%
      \^e\hbox to 0pt{\hss\raise .5ex\hbox{\'{}}}%
    }%
  }%
  \HOLOGO@space
  \ltx@mbox{Th\`anh}%
}
%    \end{macrocode}
%    \end{macro}
%    \begin{macro}{\HoLogoBkm@HanTheThanh}
%    \begin{macrocode}
\def\HoLogoBkm@HanTheThanh#1{%
  H\`an %
  Th\HOLOGO@PdfdocUnicode{\^e}{\9036\277} %
  Th\`anh%
}
%    \end{macrocode}
%    \end{macro}
%    \begin{macro}{\HoLogoHtml@HanTheThanh}
%    \begin{macrocode}
\def\HoLogoHtml@HanTheThanh#1{%
  H\`an %
  Th\HCode{&\ltx@hashchar x1ebf;} %
  Th\`anh%
}
%    \end{macrocode}
%    \end{macro}
%
% \subsection{Driver detection}
%
%    \begin{macrocode}
\HOLOGO@IfExists\InputIfFileExists{%
  \InputIfFileExists{hologo.cfg}{}{}%
}{%
  \ltx@IfUndefined{pdf@filesize}{%
    \def\HOLOGO@InputIfExists{%
      \openin\HOLOGO@temp=hologo.cfg\relax
      \ifeof\HOLOGO@temp
        \closein\HOLOGO@temp
      \else
        \closein\HOLOGO@temp
        \begingroup
          \def\x{LaTeX2e}%
        \expandafter\endgroup
        \ifx\fmtname\x
          \input{hologo.cfg}%
        \else
          \input hologo.cfg\relax
        \fi
      \fi
    }%
    \ltx@IfUndefined{newread}{%
      \chardef\HOLOGO@temp=15 %
      \def\HOLOGO@CheckRead{%
        \ifeof\HOLOGO@temp
          \HOLOGO@InputIfExists
        \else
          \ifcase\HOLOGO@temp
            \@PackageWarningNoLine{hologo}{%
              Configuration file ignored, because\MessageBreak
              a free read register could not be found%
            }%
          \else
            \begingroup
              \count\ltx@cclv=\HOLOGO@temp
              \advance\ltx@cclv by \ltx@minusone
              \edef\x{\endgroup
                \chardef\noexpand\HOLOGO@temp=\the\count\ltx@cclv
                \relax
              }%
            \x
          \fi
        \fi
      }%
    }{%
      \csname newread\endcsname\HOLOGO@temp
      \HOLOGO@InputIfExists
    }%
  }{%
    \edef\HOLOGO@temp{\pdf@filesize{hologo.cfg}}%
    \ifx\HOLOGO@temp\ltx@empty
    \else
      \ifnum\HOLOGO@temp>0 %
        \begingroup
          \def\x{LaTeX2e}%
        \expandafter\endgroup
        \ifx\fmtname\x
          \input{hologo.cfg}%
        \else
          \input hologo.cfg\relax
        \fi
      \else
        \@PackageInfoNoLine{hologo}{%
          Empty configuration file `hologo.cfg' ignored%
        }%
      \fi
    \fi
  }%
}
%    \end{macrocode}
%
%    \begin{macrocode}
\def\HOLOGO@temp#1#2{%
  \kv@define@key{HoLogoDriver}{#1}[]{%
    \begingroup
      \def\HOLOGO@temp{##1}%
      \ltx@onelevel@sanitize\HOLOGO@temp
      \ifx\HOLOGO@temp\ltx@empty
      \else
        \@PackageError{hologo}{%
          Value (\HOLOGO@temp) not permitted for option `#1'%
        }%
        \@ehc
      \fi
    \endgroup
    \def\hologoDriver{#2}%
  }%
}%
\def\HOLOGO@@temp#1#2{%
  \ifx\kv@value\relax
    \HOLOGO@temp{#1}{#1}%
  \else
    \HOLOGO@temp{#1}{#2}%
  \fi
}%
\kv@parse@normalized{%
  pdftex,%
  luatex=pdftex,%
  dvipdfm,%
  dvipdfmx=dvipdfm,%
  dvips,%
  dvipsone=dvips,%
  xdvi=dvips,%
  xetex,%
  vtex,%
}\HOLOGO@@temp
%    \end{macrocode}
%
%    \begin{macrocode}
\kv@define@key{HoLogoDriver}{driverfallback}{%
  \def\HOLOGO@DriverFallback{#1}%
}
%    \end{macrocode}
%
%    \begin{macro}{\HOLOGO@DriverFallback}
%    \begin{macrocode}
\def\HOLOGO@DriverFallback{dvips}
%    \end{macrocode}
%    \end{macro}
%
%    \begin{macro}{\hologoDriverSetup}
%    \begin{macrocode}
\def\hologoDriverSetup{%
  \let\hologoDriver\ltx@undefined
  \HOLOGO@DriverSetup
}
%    \end{macrocode}
%    \end{macro}
%
%    \begin{macro}{\HOLOGO@DriverSetup}
%    \begin{macrocode}
\def\HOLOGO@DriverSetup#1{%
  \kvsetkeys{HoLogoDriver}{#1}%
  \HOLOGO@CheckDriver
  \ltx@ifundefined{hologoDriver}{%
    \begingroup
    \edef\x{\endgroup
      \noexpand\kvsetkeys{HoLogoDriver}{\HOLOGO@DriverFallback}%
    }\x
  }{}%
  \@PackageInfoNoLine{hologo}{Using driver `\hologoDriver'}%
}
%    \end{macrocode}
%    \end{macro}
%
%    \begin{macro}{\HOLOGO@CheckDriver}
%    \begin{macrocode}
\def\HOLOGO@CheckDriver{%
  \ifpdf
    \def\hologoDriver{pdftex}%
    \let\HOLOGO@pdfliteral\pdfliteral
    \ifluatex
      \ifx\pdfextension\@undefined\else
        \protected\def\pdfliteral{\pdfextension literal}%
        \let\HOLOGO@pdfliteral\pdfliteral
      \fi
      \ltx@IfUndefined{HOLOGO@pdfliteral}{%
        \ifnum\luatexversion<36 %
        \else
          \begingroup
            \let\HOLOGO@temp\endgroup
            \ifcase0%
                \directlua{%
                  if tex.enableprimitives then %
                    tex.enableprimitives('HOLOGO@', {'pdfliteral'})%
                  else %
                    tex.print('1')%
                  end%
                }%
                \ifx\HOLOGO@pdfliteral\@undefined 1\fi%
                \relax%
              \endgroup
              \let\HOLOGO@temp\relax
              \global\let\HOLOGO@pdfliteral\HOLOGO@pdfliteral
            \fi%
          \HOLOGO@temp
        \fi
      }{}%
    \fi
    \ltx@IfUndefined{HOLOGO@pdfliteral}{%
      \@PackageWarningNoLine{hologo}{%
        Cannot find \string\pdfliteral
      }%
    }{}%
  \else
    \ifxetex
      \def\hologoDriver{xetex}%
    \else
      \ifvtex
        \def\hologoDriver{vtex}%
      \fi
    \fi
  \fi
}
%    \end{macrocode}
%    \end{macro}
%
%    \begin{macro}{\HOLOGO@WarningUnsupportedDriver}
%    \begin{macrocode}
\def\HOLOGO@WarningUnsupportedDriver#1{%
  \@PackageWarningNoLine{hologo}{%
    Logo `#1' needs driver specific macros,\MessageBreak
    but driver `\hologoDriver' is not supported.\MessageBreak
    Use a different driver or\MessageBreak
    load package `graphics' or `pgf'%
  }%
}
%    \end{macrocode}
%    \end{macro}
%
% \subsubsection{Reflect box macros}
%
%    Skip driver part if not needed.
%    \begin{macrocode}
\ltx@IfUndefined{reflectbox}{}{%
  \ltx@IfUndefined{rotatebox}{}{%
    \HOLOGO@AtEnd
  }%
}
\ltx@IfUndefined{pgftext}{}{%
  \HOLOGO@AtEnd
}
\ltx@IfUndefined{psscalebox}{}{%
  \HOLOGO@AtEnd
}
%    \end{macrocode}
%
%    \begin{macrocode}
\def\HOLOGO@temp{LaTeX2e}
\ifx\fmtname\HOLOGO@temp
  \RequirePackage{kvoptions}[2011/06/30]%
  \ProcessKeyvalOptions{HoLogoDriver}%
\fi
\HOLOGO@DriverSetup{}
%    \end{macrocode}
%
%    \begin{macro}{\HOLOGO@ReflectBox}
%    \begin{macrocode}
\def\HOLOGO@ReflectBox#1{%
  \begingroup
    \setbox\ltx@zero\hbox{\begingroup#1\endgroup}%
    \setbox\ltx@two\hbox{%
      \kern\wd\ltx@zero
      \csname HOLOGO@ScaleBox@\hologoDriver\endcsname{-1}{1}{%
        \hbox to 0pt{\copy\ltx@zero\hss}%
      }%
    }%
    \wd\ltx@two=\wd\ltx@zero
    \box\ltx@two
  \endgroup
}
%    \end{macrocode}
%    \end{macro}
%
%    \begin{macro}{\HOLOGO@PointReflectBox}
%    \begin{macrocode}
\def\HOLOGO@PointReflectBox#1{%
  \begingroup
    \setbox\ltx@zero\hbox{\begingroup#1\endgroup}%
    \setbox\ltx@two\hbox{%
      \kern\wd\ltx@zero
      \raise\ht\ltx@zero\hbox{%
        \csname HOLOGO@ScaleBox@\hologoDriver\endcsname{-1}{-1}{%
          \hbox to 0pt{\copy\ltx@zero\hss}%
        }%
      }%
    }%
    \wd\ltx@two=\wd\ltx@zero
    \box\ltx@two
  \endgroup
}
%    \end{macrocode}
%    \end{macro}
%
%    We must define all variants because of dynamic driver setup.
%    \begin{macrocode}
\def\HOLOGO@temp#1#2{#2}
%    \end{macrocode}
%
%    \begin{macro}{\HOLOGO@ScaleBox@pdftex}
%    \begin{macrocode}
\HOLOGO@temp{pdftex}{%
  \def\HOLOGO@ScaleBox@pdftex#1#2#3{%
    \HOLOGO@pdfliteral{%
      q #1 0 0 #2 0 0 cm%
    }%
    #3%
    \HOLOGO@pdfliteral{%
      Q%
    }%
  }%
}
%    \end{macrocode}
%    \end{macro}
%    \begin{macro}{\HOLOGO@ScaleBox@dvips}
%    \begin{macrocode}
\HOLOGO@temp{dvips}{%
  \def\HOLOGO@ScaleBox@dvips#1#2#3{%
    \special{ps:%
      gsave %
      currentpoint %
      currentpoint translate %
      #1 #2 scale %
      neg exch neg exch translate%
    }%
    #3%
    \special{ps:%
      currentpoint %
      grestore %
      moveto%
    }%
  }%
}
%    \end{macrocode}
%    \end{macro}
%    \begin{macro}{\HOLOGO@ScaleBox@dvipdfm}
%    \begin{macrocode}
\HOLOGO@temp{dvipdfm}{%
  \let\HOLOGO@ScaleBox@dvipdfm\HOLOGO@ScaleBox@dvips
}
%    \end{macrocode}
%    \end{macro}
%    Since \hologo{XeTeX} v0.6.
%    \begin{macro}{\HOLOGO@ScaleBox@xetex}
%    \begin{macrocode}
\HOLOGO@temp{xetex}{%
  \def\HOLOGO@ScaleBox@xetex#1#2#3{%
    \special{x:gsave}%
    \special{x:scale #1 #2}%
    #3%
    \special{x:grestore}%
  }%
}
%    \end{macrocode}
%    \end{macro}
%    \begin{macro}{\HOLOGO@ScaleBox@vtex}
%    \begin{macrocode}
\HOLOGO@temp{vtex}{%
  \def\HOLOGO@ScaleBox@vtex#1#2#3{%
    \special{r(#1,0,0,#2,0,0}%
    #3%
    \special{r)}%
  }%
}
%    \end{macrocode}
%    \end{macro}
%
%    \begin{macrocode}
\HOLOGO@AtEnd%
%</package>
%    \end{macrocode}
%
% \section{Test}
%
% \subsection{Catcode checks for loading}
%
%    \begin{macrocode}
%<*test1>
%    \end{macrocode}
%    \begin{macrocode}
\catcode`\{=1 %
\catcode`\}=2 %
\catcode`\#=6 %
\catcode`\@=11 %
\expandafter\ifx\csname count@\endcsname\relax
  \countdef\count@=255 %
\fi
\expandafter\ifx\csname @gobble\endcsname\relax
  \long\def\@gobble#1{}%
\fi
\expandafter\ifx\csname @firstofone\endcsname\relax
  \long\def\@firstofone#1{#1}%
\fi
\expandafter\ifx\csname loop\endcsname\relax
  \expandafter\@firstofone
\else
  \expandafter\@gobble
\fi
{%
  \def\loop#1\repeat{%
    \def\body{#1}%
    \iterate
  }%
  \def\iterate{%
    \body
      \let\next\iterate
    \else
      \let\next\relax
    \fi
    \next
  }%
  \let\repeat=\fi
}%
\def\RestoreCatcodes{}
\count@=0 %
\loop
  \edef\RestoreCatcodes{%
    \RestoreCatcodes
    \catcode\the\count@=\the\catcode\count@\relax
  }%
\ifnum\count@<255 %
  \advance\count@ 1 %
\repeat

\def\RangeCatcodeInvalid#1#2{%
  \count@=#1\relax
  \loop
    \catcode\count@=15 %
  \ifnum\count@<#2\relax
    \advance\count@ 1 %
  \repeat
}
\def\RangeCatcodeCheck#1#2#3{%
  \count@=#1\relax
  \loop
    \ifnum#3=\catcode\count@
    \else
      \errmessage{%
        Character \the\count@\space
        with wrong catcode \the\catcode\count@\space
        instead of \number#3%
      }%
    \fi
  \ifnum\count@<#2\relax
    \advance\count@ 1 %
  \repeat
}
\def\space{ }
\expandafter\ifx\csname LoadCommand\endcsname\relax
  \def\LoadCommand{\input hologo.sty\relax}%
\fi
\def\Test{%
  \RangeCatcodeInvalid{0}{47}%
  \RangeCatcodeInvalid{58}{64}%
  \RangeCatcodeInvalid{91}{96}%
  \RangeCatcodeInvalid{123}{255}%
  \catcode`\@=12 %
  \catcode`\\=0 %
  \catcode`\%=14 %
  \LoadCommand
  \RangeCatcodeCheck{0}{36}{15}%
  \RangeCatcodeCheck{37}{37}{14}%
  \RangeCatcodeCheck{38}{47}{15}%
  \RangeCatcodeCheck{48}{57}{12}%
  \RangeCatcodeCheck{58}{63}{15}%
  \RangeCatcodeCheck{64}{64}{12}%
  \RangeCatcodeCheck{65}{90}{11}%
  \RangeCatcodeCheck{91}{91}{15}%
  \RangeCatcodeCheck{92}{92}{0}%
  \RangeCatcodeCheck{93}{96}{15}%
  \RangeCatcodeCheck{97}{122}{11}%
  \RangeCatcodeCheck{123}{255}{15}%
  \RestoreCatcodes
}
\Test
\csname @@end\endcsname
\end
%    \end{macrocode}
%    \begin{macrocode}
%</test1>
%    \end{macrocode}
%
% \subsection{Spacefactor}
%
%    The space factor must be 1000 after a logo. If it is greater 1000
%    then the following space is a space after a sentence closing point.
%    If the space factor is smaller 1000 then an immediate following
%    dot is interpreted as abbreviation, not sentence closing point.
%
%    \begin{macrocode}
%<*test-spacefactor>
\NeedsTeXFormat{LaTeX2e}
\documentclass{article}
\usepackage{hologo}[2016/05/12]
\usepackage{kvsetkeys}
\usepackage{qstest}
\IncludeTests{*}
\LogTests{log}{*}{*}
\begin{document}
\begin{qstest}{spacefactor}{spacefactor}
\newcommand*{\Test}[1]{%
  \sbox0{%
    \hologo{#1}%
    \Expect*{1000 (#1)}*{\the\spacefactor\space(#1)}%
  }%
}%
\makeatletter
\def\TestList{}
\def\hologoEntry#1#2#3{%
  \edef\TestList{%
    \ifx\TestList\@empty
    \else
      \TestList,%
    \fi
    #1%
    \ifx\\#2\\%
    \else
      ={variant=#2}%
    \fi
  }%
}
\hologoList
\expandafter\kv@parse@normalized\expandafter{%
  \TestList
}{%
  \begingroup
    \let\@logo=\kv@key
    \ifx\kv@value\relax
    \else
      \expandafter\hologoLogoSetup\expandafter\@logo\expandafter{%
        \kv@value
      }%
    \fi
    \Test\@logo
  \endgroup
  \@gobbletwo
}
\end{qstest}
\end{document}
%</test-spacefactor>
%    \end{macrocode}
%
% \subsection{Complete list}
%
%    \begin{macrocode}
%<*test-list>
\NeedsTeXFormat{LaTeX2e}
\documentclass[12pt,a4paper]{article}
\usepackage{hologo}[2016/05/12]
\usepackage[T1]{fontenc}
\usepackage{lmodern}
\usepackage{parskip}
\usepackage[unicode]{hyperref}[2011/09/28]
\usepackage{bookmark}[2011/09/19]
\bookmarksetup{%
  numbered,%
  open,%
  openlevel=2,%
}
\renewcommand*{\contentsname}{List of logos}
\begin{document}
\tableofcontents
\def\TestFont#1#2#3#4#5#6{%
  \begingroup
    \usefont{#3}{#4}{#5}{#6}%
    \HologoVariant{#1}{#2}/\hologoVariant{#1}{#2}%
    \quad
    \begingroup\scriptsize\hologoVariant{#1}{#2}\endgroup
    \quad
  \endgroup
  (#3/#4/#5/#6)%
  \par
}
\makeatletter
\def\hologoEntry#1#2#3{%
  \section{%
    \HologoVariant{#1}{#2}/\hologoVariant{#1}{#2} %
    {[#1\ifx\\#2\\\else\space(#2)\fi]}% hash-ok
  }% braces around [] because of bug in tex4ht
  \begingroup
    \hypersetup{unicode=false}%
    \bookmark[%
      dest=\@currentHref,%
      rellevel=1,%
      keeplevel,%
    ]{%
      \HologoVariant{#1}{#2}/\hologoVariant{#1}{#2} %
      (PDFDocEncoding)%
    }%
  \endgroup
  \TestFont{#1}{#2}{OT1}{cmr}{m}{n}%
  \TestFont{#1}{#2}{OT1}{cmss}{m}{n}%
  \TestFont{#1}{#2}{OT1}{cmr}{b}{n}%
  \TestFont{#1}{#2}{OT1}{cmr}{m}{it}%
  \TestFont{#1}{#2}{OT1}{cmtt}{m}{n}%
  \TestFont{#1}{#2}{T1}{lmr}{m}{n}%
  \TestFont{#1}{#2}{T1}{lmss}{m}{n}%
  \TestFont{#1}{#2}{T1}{lmr}{b}{n}%
  \TestFont{#1}{#2}{T1}{lmr}{m}{it}%
  \TestFont{#1}{#2}{T1}{lmtt}{m}{n}%
  \TestFont{#1}{#2}{T1}{lmvtt}{m}{n}%
  \TestFont{#1}{#2}{T1}{qtm}{m}{n}%
  \TestFont{#1}{#2}{T1}{qhv}{m}{n}%
  \TestFont{#1}{#2}{T1}{qtm}{b}{n}%
  \TestFont{#1}{#2}{T1}{qtm}{m}{it}%
  \TestFont{#1}{#2}{T1}{qcr}{m}{n}%
  \newpage
}
\makeatother
\hologoList
\end{document}
%</test-list>
%    \end{macrocode}
%
% \section{Installation}
%
% \subsection{Download}
%
% \paragraph{Package.} This package is available on
% CTAN\footnote{\url{ftp://ftp.ctan.org/tex-archive/}}:
% \begin{description}
% \item[\CTAN{macros/latex/contrib/oberdiek/hologo.dtx}] The source file.
% \item[\CTAN{macros/latex/contrib/oberdiek/hologo.pdf}] Documentation.
% \end{description}
%
%
% \paragraph{Bundle.} All the packages of the bundle `oberdiek'
% are also available in a TDS compliant ZIP archive. There
% the packages are already unpacked and the documentation files
% are generated. The files and directories obey the TDS standard.
% \begin{description}
% \item[\CTAN{install/macros/latex/contrib/oberdiek.tds.zip}]
% \end{description}
% \emph{TDS} refers to the standard ``A Directory Structure
% for \TeX\ Files'' (\CTAN{tds/tds.pdf}). Directories
% with \xfile{texmf} in their name are usually organized this way.
%
% \subsection{Bundle installation}
%
% \paragraph{Unpacking.} Unpack the \xfile{oberdiek.tds.zip} in the
% TDS tree (also known as \xfile{texmf} tree) of your choice.
% Example (linux):
% \begin{quote}
%   |unzip oberdiek.tds.zip -d ~/texmf|
% \end{quote}
%
% \paragraph{Script installation.}
% Check the directory \xfile{TDS:scripts/oberdiek/} for
% scripts that need further installation steps.
% Package \xpackage{attachfile2} comes with the Perl script
% \xfile{pdfatfi.pl} that should be installed in such a way
% that it can be called as \texttt{pdfatfi}.
% Example (linux):
% \begin{quote}
%   |chmod +x scripts/oberdiek/pdfatfi.pl|\\
%   |cp scripts/oberdiek/pdfatfi.pl /usr/local/bin/|
% \end{quote}
%
% \subsection{Package installation}
%
% \paragraph{Unpacking.} The \xfile{.dtx} file is a self-extracting
% \docstrip\ archive. The files are extracted by running the
% \xfile{.dtx} through \plainTeX:
% \begin{quote}
%   \verb|tex hologo.dtx|
% \end{quote}
%
% \paragraph{TDS.} Now the different files must be moved into
% the different directories in your installation TDS tree
% (also known as \xfile{texmf} tree):
% \begin{quote}
% \def\t{^^A
% \begin{tabular}{@{}>{\ttfamily}l@{ $\rightarrow$ }>{\ttfamily}l@{}}
%   hologo.sty & tex/generic/oberdiek/hologo.sty\\
%   hologo.pdf & doc/latex/oberdiek/hologo.pdf\\
%   example/hologo-example.tex & doc/latex/oberdiek/example/hologo-example.tex\\
%   test/hologo-test1.tex & doc/latex/oberdiek/test/hologo-test1.tex\\
%   test/hologo-test-spacefactor.tex & doc/latex/oberdiek/test/hologo-test-spacefactor.tex\\
%   test/hologo-test-list.tex & doc/latex/oberdiek/test/hologo-test-list.tex\\
%   hologo.dtx & source/latex/oberdiek/hologo.dtx\\
% \end{tabular}^^A
% }^^A
% \sbox0{\t}^^A
% \ifdim\wd0>\linewidth
%   \begingroup
%     \advance\linewidth by\leftmargin
%     \advance\linewidth by\rightmargin
%   \edef\x{\endgroup
%     \def\noexpand\lw{\the\linewidth}^^A
%   }\x
%   \def\lwbox{^^A
%     \leavevmode
%     \hbox to \linewidth{^^A
%       \kern-\leftmargin\relax
%       \hss
%       \usebox0
%       \hss
%       \kern-\rightmargin\relax
%     }^^A
%   }^^A
%   \ifdim\wd0>\lw
%     \sbox0{\small\t}^^A
%     \ifdim\wd0>\linewidth
%       \ifdim\wd0>\lw
%         \sbox0{\footnotesize\t}^^A
%         \ifdim\wd0>\linewidth
%           \ifdim\wd0>\lw
%             \sbox0{\scriptsize\t}^^A
%             \ifdim\wd0>\linewidth
%               \ifdim\wd0>\lw
%                 \sbox0{\tiny\t}^^A
%                 \ifdim\wd0>\linewidth
%                   \lwbox
%                 \else
%                   \usebox0
%                 \fi
%               \else
%                 \lwbox
%               \fi
%             \else
%               \usebox0
%             \fi
%           \else
%             \lwbox
%           \fi
%         \else
%           \usebox0
%         \fi
%       \else
%         \lwbox
%       \fi
%     \else
%       \usebox0
%     \fi
%   \else
%     \lwbox
%   \fi
% \else
%   \usebox0
% \fi
% \end{quote}
% If you have a \xfile{docstrip.cfg} that configures and enables \docstrip's
% TDS installing feature, then some files can already be in the right
% place, see the documentation of \docstrip.
%
% \subsection{Refresh file name databases}
%
% If your \TeX~distribution
% (\teTeX, \mikTeX, \dots) relies on file name databases, you must refresh
% these. For example, \teTeX\ users run \verb|texhash| or
% \verb|mktexlsr|.
%
% \subsection{Some details for the interested}
%
% \paragraph{Attached source.}
%
% The PDF documentation on CTAN also includes the
% \xfile{.dtx} source file. It can be extracted by
% AcrobatReader 6 or higher. Another option is \textsf{pdftk},
% e.g. unpack the file into the current directory:
% \begin{quote}
%   \verb|pdftk hologo.pdf unpack_files output .|
% \end{quote}
%
% \paragraph{Unpacking with \LaTeX.}
% The \xfile{.dtx} chooses its action depending on the format:
% \begin{description}
% \item[\plainTeX:] Run \docstrip\ and extract the files.
% \item[\LaTeX:] Generate the documentation.
% \end{description}
% If you insist on using \LaTeX\ for \docstrip\ (really,
% \docstrip\ does not need \LaTeX), then inform the autodetect routine
% about your intention:
% \begin{quote}
%   \verb|latex \let\install=y\input{hologo.dtx}|
% \end{quote}
% Do not forget to quote the argument according to the demands
% of your shell.
%
% \paragraph{Generating the documentation.}
% You can use both the \xfile{.dtx} or the \xfile{.drv} to generate
% the documentation. The process can be configured by the
% configuration file \xfile{ltxdoc.cfg}. For instance, put this
% line into this file, if you want to have A4 as paper format:
% \begin{quote}
%   \verb|\PassOptionsToClass{a4paper}{article}|
% \end{quote}
% An example follows how to generate the
% documentation with pdf\LaTeX:
% \begin{quote}
%\begin{verbatim}
%pdflatex hologo.dtx
%makeindex -s gind.ist hologo.idx
%pdflatex hologo.dtx
%makeindex -s gind.ist hologo.idx
%pdflatex hologo.dtx
%\end{verbatim}
% \end{quote}
%
% \section{Catalogue}
%
% The following XML file can be used as source for the
% \href{http://mirror.ctan.org/help/Catalogue/catalogue.html}{\TeX\ Catalogue}.
% The elements \texttt{caption} and \texttt{description} are imported
% from the original XML file from the Catalogue.
% The name of the XML file in the Catalogue is \xfile{hologo.xml}.
%    \begin{macrocode}
%<*catalogue>
<?xml version='1.0' encoding='us-ascii'?>
<!DOCTYPE entry SYSTEM 'catalogue.dtd'>
<entry datestamp='$Date$' modifier='$Author$' id='hologo'>
  <name>hologo</name>
  <caption>A collection of logos with bookmark support.</caption>
  <authorref id='auth:oberdiek'/>
  <copyright owner='Heiko Oberdiek' year='2010-2012'/>
  <license type='lppl1.3'/>
  <version number='1.10'/>
  <description>
    The package defines a single command <tt>\hologo</tt>, whose
    argument is the usual case-confused ASCII version of the logo.
    The command is bookmark-enabled, so that every logo becomes
    available in bookmarks without further work.
    <p/>
    The package is part of the <xref refid='oberdiek'>oberdiek</xref>
    bundle.
  </description>
  <documentation details='Package documentation'
      href='ctan:/macros/latex/contrib/oberdiek/hologo.pdf'/>
  <ctan file='true' path='/macros/latex/contrib/oberdiek/hologo.dtx'/>
  <miktex location='oberdiek'/>
  <texlive location='oberdiek'/>
  <install path='/macros/latex/contrib/oberdiek/oberdiek.tds.zip'/>
</entry>
%</catalogue>
%    \end{macrocode}
%
% \begin{thebibliography}{9}
% \raggedright
%
% \bibitem{btxdoc}
% Oren Patashnik,
% \textit{\hologo{BibTeX}ing},
% 1988-02-08.\\
% \CTAN{biblio/bibtex/base/}
%
% \bibitem{dtklogos}
% Gerd Neugebauer, DANTE,
% \textit{Package \xpackage{dtklogos}},
% 2011-04-25.\\
% \CTAN{usergrps/dante/dtk/dtklogos.sty}
%
% \bibitem{etexman}
% The \hologo{NTS} Team,
% \textit{The \hologo{eTeX} manual},
% 1998-02.\\
% \CTAN{systems/e-tex/v2/doc/}
%
% \bibitem{ExTeX-FAQ}
% The \hologo{ExTeX} group,
% \textit{\hologo{ExTeX}: FAQ -- How is \hologo{ExTeX} typeset?},
% 2007-04-14.\\
% \url{http://www.extex.org/documentation/faq.html}
%
% \bibitem{LyX}
% %@MISC{ LyX,
% %  title = {{LyX 2.0.0 -- The Document Processor [Computer software and manual]}},
% %  author = {{The LyX Team}},
% %  howpublished = {Internet: http://www.lyx.org},
% %  year = {2011-05-08},
% %  note = {Retrieved May 10, 2011, from http://www.lyx.org},
% %  url = {http://www.lyx.org/}
% %}
% The \hologo{LyX} Team,
% \textit{\hologo{LyX} -- The Document Processor},
% 2011-05-08.\\
% \url{http://www.lyx.org/}
%
% \bibitem{OzTeX}
% Andrew Trevorrow,
% \hologo{OzTeX} FAQ: What is the correct way to typeset ``\hologo{OzTeX}''?,
% 2011-09-15 (visited).
% \url{http://www.trevorrow.com/oztex/ozfaq.html#oztex-logo}
%
% \bibitem{PiCTeX}
% Michael Wichura,
% \textit{The \hologo{PiCTeX} macro package},
% 1987-09-21.
% \CTAN{graphics/pictex/}
%
% \bibitem{scrlogo}
% Markus Kohm,
% \textit{\hologo{KOMAScript} Datei \xfile{scrlogo.dtx}},
% 2009-01-30.\\
% \CTAN{install/macros/latex/contrib/komascript.tds.zip}
%
% \end{thebibliography}
%
% \begin{History}
%   \begin{Version}{2010/04/08 v1.0}
%   \item
%     The first version.
%   \end{Version}
%   \begin{Version}{2010/04/16 v1.1}
%   \item
%     \cs{Hologo} added for support of logos at start of a sentence.
%   \item
%     \cs{hologoSetup} and \cs{hologoLogoSetup} added.
%   \item
%     Options \xoption{break}, \xoption{hyphenbreak}, \xoption{spacebreak}
%     added.
%   \item
%     Variant support added by option \xoption{variant}.
%   \end{Version}
%   \begin{Version}{2010/04/24 v1.2}
%   \item
%     \hologo{LaTeX3} added.
%   \item
%     \hologo{VTeX} added.
%   \end{Version}
%   \begin{Version}{2010/11/21 v1.3}
%   \item
%     \hologo{iniTeX}, \hologo{virTeX} added.
%   \end{Version}
%   \begin{Version}{2011/03/25 v1.4}
%   \item
%     \hologo{ConTeXt} with variants added.
%   \item
%     Option \xoption{discretionarybreak} added as refinement for
%     option \xoption{break}.
%   \end{Version}
%   \begin{Version}{2011/04/21 v1.5}
%   \item
%     Wrong TDS directory for test files fixed.
%   \end{Version}
%   \begin{Version}{2011/10/01 v1.6}
%   \item
%     Support for package \xpackage{tex4ht} added.
%   \item
%     Support for \cs{csname} added if \cs{ifincsname} is available.
%   \item
%     New logos:
%     \hologo{(La)TeX},
%     \hologo{biber},
%     \hologo{BibTeX} (\xoption{sc}, \xoption{sf}),
%     \hologo{emTeX},
%     \hologo{ExTeX},
%     \hologo{KOMAScript},
%     \hologo{La},
%     \hologo{LyX},
%     \hologo{MiKTeX},
%     \hologo{NTS},
%     \hologo{OzMF},
%     \hologo{OzMP},
%     \hologo{OzTeX},
%     \hologo{OzTtH},
%     \hologo{PCTeX},
%     \hologo{PiC},
%     \hologo{PiCTeX},
%     \hologo{METAFONT},
%     \hologo{MetaFun},
%     \hologo{METAPOST},
%     \hologo{MetaPost},
%     \hologo{SLiTeX} (\xoption{lift}, \xoption{narrow}, \xoption{simple}),
%     \hologo{SliTeX} (\xoption{narrow}, \xoption{simple}, \xoption{lift}),
%     \hologo{teTeX}.
%   \item
%     Fixes:
%     \hologo{iniTeX},
%     \hologo{pdfLaTeX},
%     \hologo{pdfTeX},
%     \hologo{virTeX}.
%   \item
%     \cs{hologoFontSetup} and \cs{hologoLogoFontSetup} added.
%   \item
%     \cs{hologoVariant} and \cs{HologoVariant} added.
%   \end{Version}
%   \begin{Version}{2011/11/22 v1.7}
%   \item
%     New logos:
%     \hologo{BibTeX8},
%     \hologo{LaTeXML},
%     \hologo{SageTeX},
%     \hologo{TeX4ht},
%     \hologo{TTH}.
%   \item
%     \hologo{Xe} and friends: Driver stuff fixed.
%   \item
%     \hologo{Xe} and friends: Support for italic added.
%   \item
%     \hologo{Xe} and friends: Package support for \xpackage{pgf}
%     and \xpackage{pstricks} added.
%   \end{Version}
%   \begin{Version}{2011/11/29 v1.8}
%   \item
%     New logos:
%     \hologo{HanTheThanh}.
%   \end{Version}
%   \begin{Version}{2011/12/21 v1.9}
%   \item
%     Patch for package \xpackage{ifxetex} added for the case that
%     \cs{newif} is undefined in \hologo{iniTeX}.
%   \item
%     Some fixes for \hologo{iniTeX}.
%   \end{Version}
%   \begin{Version}{2012/04/26 v1.10}
%   \item
%     Fix in bookmark version of logo ``\hologo{HanTheThanh}''.
%   \end{Version}
%   \begin{Version}{2016/05/12 v1.11}
%   \item
%     Update HOLOGO@IfCharExists (previously in texlive)
%   \item define pdfliteral in current luatex.
%   \end{Version}
% \end{History}
%
% \PrintIndex
%
% \Finale
\endinput
%
        \else
          \input hologo.cfg\relax
        \fi
      \else
        \@PackageInfoNoLine{hologo}{%
          Empty configuration file `hologo.cfg' ignored%
        }%
      \fi
    \fi
  }%
}
%    \end{macrocode}
%
%    \begin{macrocode}
\def\HOLOGO@temp#1#2{%
  \kv@define@key{HoLogoDriver}{#1}[]{%
    \begingroup
      \def\HOLOGO@temp{##1}%
      \ltx@onelevel@sanitize\HOLOGO@temp
      \ifx\HOLOGO@temp\ltx@empty
      \else
        \@PackageError{hologo}{%
          Value (\HOLOGO@temp) not permitted for option `#1'%
        }%
        \@ehc
      \fi
    \endgroup
    \def\hologoDriver{#2}%
  }%
}%
\def\HOLOGO@@temp#1#2{%
  \ifx\kv@value\relax
    \HOLOGO@temp{#1}{#1}%
  \else
    \HOLOGO@temp{#1}{#2}%
  \fi
}%
\kv@parse@normalized{%
  pdftex,%
  luatex=pdftex,%
  dvipdfm,%
  dvipdfmx=dvipdfm,%
  dvips,%
  dvipsone=dvips,%
  xdvi=dvips,%
  xetex,%
  vtex,%
}\HOLOGO@@temp
%    \end{macrocode}
%
%    \begin{macrocode}
\kv@define@key{HoLogoDriver}{driverfallback}{%
  \def\HOLOGO@DriverFallback{#1}%
}
%    \end{macrocode}
%
%    \begin{macro}{\HOLOGO@DriverFallback}
%    \begin{macrocode}
\def\HOLOGO@DriverFallback{dvips}
%    \end{macrocode}
%    \end{macro}
%
%    \begin{macro}{\hologoDriverSetup}
%    \begin{macrocode}
\def\hologoDriverSetup{%
  \let\hologoDriver\ltx@undefined
  \HOLOGO@DriverSetup
}
%    \end{macrocode}
%    \end{macro}
%
%    \begin{macro}{\HOLOGO@DriverSetup}
%    \begin{macrocode}
\def\HOLOGO@DriverSetup#1{%
  \kvsetkeys{HoLogoDriver}{#1}%
  \HOLOGO@CheckDriver
  \ltx@ifundefined{hologoDriver}{%
    \begingroup
    \edef\x{\endgroup
      \noexpand\kvsetkeys{HoLogoDriver}{\HOLOGO@DriverFallback}%
    }\x
  }{}%
  \@PackageInfoNoLine{hologo}{Using driver `\hologoDriver'}%
}
%    \end{macrocode}
%    \end{macro}
%
%    \begin{macro}{\HOLOGO@CheckDriver}
%    \begin{macrocode}
\def\HOLOGO@CheckDriver{%
  \ifpdf
    \def\hologoDriver{pdftex}%
    \let\HOLOGO@pdfliteral\pdfliteral
    \ifluatex
      \ifx\pdfextension\@undefined\else
        \protected\def\pdfliteral{\pdfextension literal}%
        \let\HOLOGO@pdfliteral\pdfliteral
      \fi
      \ltx@IfUndefined{HOLOGO@pdfliteral}{%
        \ifnum\luatexversion<36 %
        \else
          \begingroup
            \let\HOLOGO@temp\endgroup
            \ifcase0%
                \directlua{%
                  if tex.enableprimitives then %
                    tex.enableprimitives('HOLOGO@', {'pdfliteral'})%
                  else %
                    tex.print('1')%
                  end%
                }%
                \ifx\HOLOGO@pdfliteral\@undefined 1\fi%
                \relax%
              \endgroup
              \let\HOLOGO@temp\relax
              \global\let\HOLOGO@pdfliteral\HOLOGO@pdfliteral
            \fi%
          \HOLOGO@temp
        \fi
      }{}%
    \fi
    \ltx@IfUndefined{HOLOGO@pdfliteral}{%
      \@PackageWarningNoLine{hologo}{%
        Cannot find \string\pdfliteral
      }%
    }{}%
  \else
    \ifxetex
      \def\hologoDriver{xetex}%
    \else
      \ifvtex
        \def\hologoDriver{vtex}%
      \fi
    \fi
  \fi
}
%    \end{macrocode}
%    \end{macro}
%
%    \begin{macro}{\HOLOGO@WarningUnsupportedDriver}
%    \begin{macrocode}
\def\HOLOGO@WarningUnsupportedDriver#1{%
  \@PackageWarningNoLine{hologo}{%
    Logo `#1' needs driver specific macros,\MessageBreak
    but driver `\hologoDriver' is not supported.\MessageBreak
    Use a different driver or\MessageBreak
    load package `graphics' or `pgf'%
  }%
}
%    \end{macrocode}
%    \end{macro}
%
% \subsubsection{Reflect box macros}
%
%    Skip driver part if not needed.
%    \begin{macrocode}
\ltx@IfUndefined{reflectbox}{}{%
  \ltx@IfUndefined{rotatebox}{}{%
    \HOLOGO@AtEnd
  }%
}
\ltx@IfUndefined{pgftext}{}{%
  \HOLOGO@AtEnd
}
\ltx@IfUndefined{psscalebox}{}{%
  \HOLOGO@AtEnd
}
%    \end{macrocode}
%
%    \begin{macrocode}
\def\HOLOGO@temp{LaTeX2e}
\ifx\fmtname\HOLOGO@temp
  \RequirePackage{kvoptions}[2011/06/30]%
  \ProcessKeyvalOptions{HoLogoDriver}%
\fi
\HOLOGO@DriverSetup{}
%    \end{macrocode}
%
%    \begin{macro}{\HOLOGO@ReflectBox}
%    \begin{macrocode}
\def\HOLOGO@ReflectBox#1{%
  \begingroup
    \setbox\ltx@zero\hbox{\begingroup#1\endgroup}%
    \setbox\ltx@two\hbox{%
      \kern\wd\ltx@zero
      \csname HOLOGO@ScaleBox@\hologoDriver\endcsname{-1}{1}{%
        \hbox to 0pt{\copy\ltx@zero\hss}%
      }%
    }%
    \wd\ltx@two=\wd\ltx@zero
    \box\ltx@two
  \endgroup
}
%    \end{macrocode}
%    \end{macro}
%
%    \begin{macro}{\HOLOGO@PointReflectBox}
%    \begin{macrocode}
\def\HOLOGO@PointReflectBox#1{%
  \begingroup
    \setbox\ltx@zero\hbox{\begingroup#1\endgroup}%
    \setbox\ltx@two\hbox{%
      \kern\wd\ltx@zero
      \raise\ht\ltx@zero\hbox{%
        \csname HOLOGO@ScaleBox@\hologoDriver\endcsname{-1}{-1}{%
          \hbox to 0pt{\copy\ltx@zero\hss}%
        }%
      }%
    }%
    \wd\ltx@two=\wd\ltx@zero
    \box\ltx@two
  \endgroup
}
%    \end{macrocode}
%    \end{macro}
%
%    We must define all variants because of dynamic driver setup.
%    \begin{macrocode}
\def\HOLOGO@temp#1#2{#2}
%    \end{macrocode}
%
%    \begin{macro}{\HOLOGO@ScaleBox@pdftex}
%    \begin{macrocode}
\HOLOGO@temp{pdftex}{%
  \def\HOLOGO@ScaleBox@pdftex#1#2#3{%
    \HOLOGO@pdfliteral{%
      q #1 0 0 #2 0 0 cm%
    }%
    #3%
    \HOLOGO@pdfliteral{%
      Q%
    }%
  }%
}
%    \end{macrocode}
%    \end{macro}
%    \begin{macro}{\HOLOGO@ScaleBox@dvips}
%    \begin{macrocode}
\HOLOGO@temp{dvips}{%
  \def\HOLOGO@ScaleBox@dvips#1#2#3{%
    \special{ps:%
      gsave %
      currentpoint %
      currentpoint translate %
      #1 #2 scale %
      neg exch neg exch translate%
    }%
    #3%
    \special{ps:%
      currentpoint %
      grestore %
      moveto%
    }%
  }%
}
%    \end{macrocode}
%    \end{macro}
%    \begin{macro}{\HOLOGO@ScaleBox@dvipdfm}
%    \begin{macrocode}
\HOLOGO@temp{dvipdfm}{%
  \let\HOLOGO@ScaleBox@dvipdfm\HOLOGO@ScaleBox@dvips
}
%    \end{macrocode}
%    \end{macro}
%    Since \hologo{XeTeX} v0.6.
%    \begin{macro}{\HOLOGO@ScaleBox@xetex}
%    \begin{macrocode}
\HOLOGO@temp{xetex}{%
  \def\HOLOGO@ScaleBox@xetex#1#2#3{%
    \special{x:gsave}%
    \special{x:scale #1 #2}%
    #3%
    \special{x:grestore}%
  }%
}
%    \end{macrocode}
%    \end{macro}
%    \begin{macro}{\HOLOGO@ScaleBox@vtex}
%    \begin{macrocode}
\HOLOGO@temp{vtex}{%
  \def\HOLOGO@ScaleBox@vtex#1#2#3{%
    \special{r(#1,0,0,#2,0,0}%
    #3%
    \special{r)}%
  }%
}
%    \end{macrocode}
%    \end{macro}
%
%    \begin{macrocode}
\HOLOGO@AtEnd%
%</package>
%    \end{macrocode}
%
% \section{Test}
%
% \subsection{Catcode checks for loading}
%
%    \begin{macrocode}
%<*test1>
%    \end{macrocode}
%    \begin{macrocode}
\catcode`\{=1 %
\catcode`\}=2 %
\catcode`\#=6 %
\catcode`\@=11 %
\expandafter\ifx\csname count@\endcsname\relax
  \countdef\count@=255 %
\fi
\expandafter\ifx\csname @gobble\endcsname\relax
  \long\def\@gobble#1{}%
\fi
\expandafter\ifx\csname @firstofone\endcsname\relax
  \long\def\@firstofone#1{#1}%
\fi
\expandafter\ifx\csname loop\endcsname\relax
  \expandafter\@firstofone
\else
  \expandafter\@gobble
\fi
{%
  \def\loop#1\repeat{%
    \def\body{#1}%
    \iterate
  }%
  \def\iterate{%
    \body
      \let\next\iterate
    \else
      \let\next\relax
    \fi
    \next
  }%
  \let\repeat=\fi
}%
\def\RestoreCatcodes{}
\count@=0 %
\loop
  \edef\RestoreCatcodes{%
    \RestoreCatcodes
    \catcode\the\count@=\the\catcode\count@\relax
  }%
\ifnum\count@<255 %
  \advance\count@ 1 %
\repeat

\def\RangeCatcodeInvalid#1#2{%
  \count@=#1\relax
  \loop
    \catcode\count@=15 %
  \ifnum\count@<#2\relax
    \advance\count@ 1 %
  \repeat
}
\def\RangeCatcodeCheck#1#2#3{%
  \count@=#1\relax
  \loop
    \ifnum#3=\catcode\count@
    \else
      \errmessage{%
        Character \the\count@\space
        with wrong catcode \the\catcode\count@\space
        instead of \number#3%
      }%
    \fi
  \ifnum\count@<#2\relax
    \advance\count@ 1 %
  \repeat
}
\def\space{ }
\expandafter\ifx\csname LoadCommand\endcsname\relax
  \def\LoadCommand{\input hologo.sty\relax}%
\fi
\def\Test{%
  \RangeCatcodeInvalid{0}{47}%
  \RangeCatcodeInvalid{58}{64}%
  \RangeCatcodeInvalid{91}{96}%
  \RangeCatcodeInvalid{123}{255}%
  \catcode`\@=12 %
  \catcode`\\=0 %
  \catcode`\%=14 %
  \LoadCommand
  \RangeCatcodeCheck{0}{36}{15}%
  \RangeCatcodeCheck{37}{37}{14}%
  \RangeCatcodeCheck{38}{47}{15}%
  \RangeCatcodeCheck{48}{57}{12}%
  \RangeCatcodeCheck{58}{63}{15}%
  \RangeCatcodeCheck{64}{64}{12}%
  \RangeCatcodeCheck{65}{90}{11}%
  \RangeCatcodeCheck{91}{91}{15}%
  \RangeCatcodeCheck{92}{92}{0}%
  \RangeCatcodeCheck{93}{96}{15}%
  \RangeCatcodeCheck{97}{122}{11}%
  \RangeCatcodeCheck{123}{255}{15}%
  \RestoreCatcodes
}
\Test
\csname @@end\endcsname
\end
%    \end{macrocode}
%    \begin{macrocode}
%</test1>
%    \end{macrocode}
%
% \subsection{Spacefactor}
%
%    The space factor must be 1000 after a logo. If it is greater 1000
%    then the following space is a space after a sentence closing point.
%    If the space factor is smaller 1000 then an immediate following
%    dot is interpreted as abbreviation, not sentence closing point.
%
%    \begin{macrocode}
%<*test-spacefactor>
\NeedsTeXFormat{LaTeX2e}
\documentclass{article}
\usepackage{hologo}[2016/05/12]
\usepackage{kvsetkeys}
\usepackage{qstest}
\IncludeTests{*}
\LogTests{log}{*}{*}
\begin{document}
\begin{qstest}{spacefactor}{spacefactor}
\newcommand*{\Test}[1]{%
  \sbox0{%
    \hologo{#1}%
    \Expect*{1000 (#1)}*{\the\spacefactor\space(#1)}%
  }%
}%
\makeatletter
\def\TestList{}
\def\hologoEntry#1#2#3{%
  \edef\TestList{%
    \ifx\TestList\@empty
    \else
      \TestList,%
    \fi
    #1%
    \ifx\\#2\\%
    \else
      ={variant=#2}%
    \fi
  }%
}
\hologoList
\expandafter\kv@parse@normalized\expandafter{%
  \TestList
}{%
  \begingroup
    \let\@logo=\kv@key
    \ifx\kv@value\relax
    \else
      \expandafter\hologoLogoSetup\expandafter\@logo\expandafter{%
        \kv@value
      }%
    \fi
    \Test\@logo
  \endgroup
  \@gobbletwo
}
\end{qstest}
\end{document}
%</test-spacefactor>
%    \end{macrocode}
%
% \subsection{Complete list}
%
%    \begin{macrocode}
%<*test-list>
\NeedsTeXFormat{LaTeX2e}
\documentclass[12pt,a4paper]{article}
\usepackage{hologo}[2016/05/12]
\usepackage[T1]{fontenc}
\usepackage{lmodern}
\usepackage{parskip}
\usepackage[unicode]{hyperref}[2011/09/28]
\usepackage{bookmark}[2011/09/19]
\bookmarksetup{%
  numbered,%
  open,%
  openlevel=2,%
}
\renewcommand*{\contentsname}{List of logos}
\begin{document}
\tableofcontents
\def\TestFont#1#2#3#4#5#6{%
  \begingroup
    \usefont{#3}{#4}{#5}{#6}%
    \HologoVariant{#1}{#2}/\hologoVariant{#1}{#2}%
    \quad
    \begingroup\scriptsize\hologoVariant{#1}{#2}\endgroup
    \quad
  \endgroup
  (#3/#4/#5/#6)%
  \par
}
\makeatletter
\def\hologoEntry#1#2#3{%
  \section{%
    \HologoVariant{#1}{#2}/\hologoVariant{#1}{#2} %
    {[#1\ifx\\#2\\\else\space(#2)\fi]}% hash-ok
  }% braces around [] because of bug in tex4ht
  \begingroup
    \hypersetup{unicode=false}%
    \bookmark[%
      dest=\@currentHref,%
      rellevel=1,%
      keeplevel,%
    ]{%
      \HologoVariant{#1}{#2}/\hologoVariant{#1}{#2} %
      (PDFDocEncoding)%
    }%
  \endgroup
  \TestFont{#1}{#2}{OT1}{cmr}{m}{n}%
  \TestFont{#1}{#2}{OT1}{cmss}{m}{n}%
  \TestFont{#1}{#2}{OT1}{cmr}{b}{n}%
  \TestFont{#1}{#2}{OT1}{cmr}{m}{it}%
  \TestFont{#1}{#2}{OT1}{cmtt}{m}{n}%
  \TestFont{#1}{#2}{T1}{lmr}{m}{n}%
  \TestFont{#1}{#2}{T1}{lmss}{m}{n}%
  \TestFont{#1}{#2}{T1}{lmr}{b}{n}%
  \TestFont{#1}{#2}{T1}{lmr}{m}{it}%
  \TestFont{#1}{#2}{T1}{lmtt}{m}{n}%
  \TestFont{#1}{#2}{T1}{lmvtt}{m}{n}%
  \TestFont{#1}{#2}{T1}{qtm}{m}{n}%
  \TestFont{#1}{#2}{T1}{qhv}{m}{n}%
  \TestFont{#1}{#2}{T1}{qtm}{b}{n}%
  \TestFont{#1}{#2}{T1}{qtm}{m}{it}%
  \TestFont{#1}{#2}{T1}{qcr}{m}{n}%
  \newpage
}
\makeatother
\hologoList
\end{document}
%</test-list>
%    \end{macrocode}
%
% \section{Installation}
%
% \subsection{Download}
%
% \paragraph{Package.} This package is available on
% CTAN\footnote{\url{ftp://ftp.ctan.org/tex-archive/}}:
% \begin{description}
% \item[\CTAN{macros/latex/contrib/oberdiek/hologo.dtx}] The source file.
% \item[\CTAN{macros/latex/contrib/oberdiek/hologo.pdf}] Documentation.
% \end{description}
%
%
% \paragraph{Bundle.} All the packages of the bundle `oberdiek'
% are also available in a TDS compliant ZIP archive. There
% the packages are already unpacked and the documentation files
% are generated. The files and directories obey the TDS standard.
% \begin{description}
% \item[\CTAN{install/macros/latex/contrib/oberdiek.tds.zip}]
% \end{description}
% \emph{TDS} refers to the standard ``A Directory Structure
% for \TeX\ Files'' (\CTAN{tds/tds.pdf}). Directories
% with \xfile{texmf} in their name are usually organized this way.
%
% \subsection{Bundle installation}
%
% \paragraph{Unpacking.} Unpack the \xfile{oberdiek.tds.zip} in the
% TDS tree (also known as \xfile{texmf} tree) of your choice.
% Example (linux):
% \begin{quote}
%   |unzip oberdiek.tds.zip -d ~/texmf|
% \end{quote}
%
% \paragraph{Script installation.}
% Check the directory \xfile{TDS:scripts/oberdiek/} for
% scripts that need further installation steps.
% Package \xpackage{attachfile2} comes with the Perl script
% \xfile{pdfatfi.pl} that should be installed in such a way
% that it can be called as \texttt{pdfatfi}.
% Example (linux):
% \begin{quote}
%   |chmod +x scripts/oberdiek/pdfatfi.pl|\\
%   |cp scripts/oberdiek/pdfatfi.pl /usr/local/bin/|
% \end{quote}
%
% \subsection{Package installation}
%
% \paragraph{Unpacking.} The \xfile{.dtx} file is a self-extracting
% \docstrip\ archive. The files are extracted by running the
% \xfile{.dtx} through \plainTeX:
% \begin{quote}
%   \verb|tex hologo.dtx|
% \end{quote}
%
% \paragraph{TDS.} Now the different files must be moved into
% the different directories in your installation TDS tree
% (also known as \xfile{texmf} tree):
% \begin{quote}
% \def\t{^^A
% \begin{tabular}{@{}>{\ttfamily}l@{ $\rightarrow$ }>{\ttfamily}l@{}}
%   hologo.sty & tex/generic/oberdiek/hologo.sty\\
%   hologo.pdf & doc/latex/oberdiek/hologo.pdf\\
%   example/hologo-example.tex & doc/latex/oberdiek/example/hologo-example.tex\\
%   test/hologo-test1.tex & doc/latex/oberdiek/test/hologo-test1.tex\\
%   test/hologo-test-spacefactor.tex & doc/latex/oberdiek/test/hologo-test-spacefactor.tex\\
%   test/hologo-test-list.tex & doc/latex/oberdiek/test/hologo-test-list.tex\\
%   hologo.dtx & source/latex/oberdiek/hologo.dtx\\
% \end{tabular}^^A
% }^^A
% \sbox0{\t}^^A
% \ifdim\wd0>\linewidth
%   \begingroup
%     \advance\linewidth by\leftmargin
%     \advance\linewidth by\rightmargin
%   \edef\x{\endgroup
%     \def\noexpand\lw{\the\linewidth}^^A
%   }\x
%   \def\lwbox{^^A
%     \leavevmode
%     \hbox to \linewidth{^^A
%       \kern-\leftmargin\relax
%       \hss
%       \usebox0
%       \hss
%       \kern-\rightmargin\relax
%     }^^A
%   }^^A
%   \ifdim\wd0>\lw
%     \sbox0{\small\t}^^A
%     \ifdim\wd0>\linewidth
%       \ifdim\wd0>\lw
%         \sbox0{\footnotesize\t}^^A
%         \ifdim\wd0>\linewidth
%           \ifdim\wd0>\lw
%             \sbox0{\scriptsize\t}^^A
%             \ifdim\wd0>\linewidth
%               \ifdim\wd0>\lw
%                 \sbox0{\tiny\t}^^A
%                 \ifdim\wd0>\linewidth
%                   \lwbox
%                 \else
%                   \usebox0
%                 \fi
%               \else
%                 \lwbox
%               \fi
%             \else
%               \usebox0
%             \fi
%           \else
%             \lwbox
%           \fi
%         \else
%           \usebox0
%         \fi
%       \else
%         \lwbox
%       \fi
%     \else
%       \usebox0
%     \fi
%   \else
%     \lwbox
%   \fi
% \else
%   \usebox0
% \fi
% \end{quote}
% If you have a \xfile{docstrip.cfg} that configures and enables \docstrip's
% TDS installing feature, then some files can already be in the right
% place, see the documentation of \docstrip.
%
% \subsection{Refresh file name databases}
%
% If your \TeX~distribution
% (\teTeX, \mikTeX, \dots) relies on file name databases, you must refresh
% these. For example, \teTeX\ users run \verb|texhash| or
% \verb|mktexlsr|.
%
% \subsection{Some details for the interested}
%
% \paragraph{Attached source.}
%
% The PDF documentation on CTAN also includes the
% \xfile{.dtx} source file. It can be extracted by
% AcrobatReader 6 or higher. Another option is \textsf{pdftk},
% e.g. unpack the file into the current directory:
% \begin{quote}
%   \verb|pdftk hologo.pdf unpack_files output .|
% \end{quote}
%
% \paragraph{Unpacking with \LaTeX.}
% The \xfile{.dtx} chooses its action depending on the format:
% \begin{description}
% \item[\plainTeX:] Run \docstrip\ and extract the files.
% \item[\LaTeX:] Generate the documentation.
% \end{description}
% If you insist on using \LaTeX\ for \docstrip\ (really,
% \docstrip\ does not need \LaTeX), then inform the autodetect routine
% about your intention:
% \begin{quote}
%   \verb|latex \let\install=y% \iffalse meta-comment
%
% File: hologo.dtx
% Version: 2016/05/12 v1.11
% Info: A logo collection with bookmark support
%
% Copyright (C) 2010-2012 by
%    Heiko Oberdiek <heiko.oberdiek at googlemail.com>
%
% This work may be distributed and/or modified under the
% conditions of the LaTeX Project Public License, either
% version 1.3c of this license or (at your option) any later
% version. This version of this license is in
%    http://www.latex-project.org/lppl/lppl-1-3c.txt
% and the latest version of this license is in
%    http://www.latex-project.org/lppl.txt
% and version 1.3 or later is part of all distributions of
% LaTeX version 2005/12/01 or later.
%
% This work has the LPPL maintenance status "maintained".
%
% This Current Maintainer of this work is Heiko Oberdiek.
%
% The Base Interpreter refers to any `TeX-Format',
% because some files are installed in TDS:tex/generic//.
%
% This work consists of the main source file hologo.dtx
% and the derived files
%    hologo.sty, hologo.pdf, hologo.ins, hologo.drv, hologo-example.tex,
%    hologo-test1.tex, hologo-test-spacefactor.tex,
%    hologo-test-list.tex.
%
% Distribution:
%    CTAN:macros/latex/contrib/oberdiek/hologo.dtx
%    CTAN:macros/latex/contrib/oberdiek/hologo.pdf
%
% Unpacking:
%    (a) If hologo.ins is present:
%           tex hologo.ins
%    (b) Without hologo.ins:
%           tex hologo.dtx
%    (c) If you insist on using LaTeX
%           latex \let\install=y\input{hologo.dtx}
%        (quote the arguments according to the demands of your shell)
%
% Documentation:
%    (a) If hologo.drv is present:
%           latex hologo.drv
%    (b) Without hologo.drv:
%           latex hologo.dtx; ...
%    The class ltxdoc loads the configuration file ltxdoc.cfg
%    if available. Here you can specify further options, e.g.
%    use A4 as paper format:
%       \PassOptionsToClass{a4paper}{article}
%
%    Programm calls to get the documentation (example):
%       pdflatex hologo.dtx
%       makeindex -s gind.ist hologo.idx
%       pdflatex hologo.dtx
%       makeindex -s gind.ist hologo.idx
%       pdflatex hologo.dtx
%
% Installation:
%    TDS:tex/generic/oberdiek/hologo.sty
%    TDS:doc/latex/oberdiek/hologo.pdf
%    TDS:doc/latex/oberdiek/example/hologo-example.tex
%    TDS:doc/latex/oberdiek/test/hologo-test1.tex
%    TDS:doc/latex/oberdiek/test/hologo-test-spacefactor.tex
%    TDS:doc/latex/oberdiek/test/hologo-test-list.tex
%    TDS:source/latex/oberdiek/hologo.dtx
%
%<*ignore>
\begingroup
  \catcode123=1 %
  \catcode125=2 %
  \def\x{LaTeX2e}%
\expandafter\endgroup
\ifcase 0\ifx\install y1\fi\expandafter
         \ifx\csname processbatchFile\endcsname\relax\else1\fi
         \ifx\fmtname\x\else 1\fi\relax
\else\csname fi\endcsname
%</ignore>
%<*install>
\input docstrip.tex
\Msg{************************************************************************}
\Msg{* Installation}
\Msg{* Package: hologo 2016/05/12 v1.11 A logo collection with bookmark support (HO)}
\Msg{************************************************************************}

\keepsilent
\askforoverwritefalse

\let\MetaPrefix\relax
\preamble

This is a generated file.

Project: hologo
Version: 2016/05/12 v1.11

Copyright (C) 2010-2012 by
   Heiko Oberdiek <heiko.oberdiek at googlemail.com>

This work may be distributed and/or modified under the
conditions of the LaTeX Project Public License, either
version 1.3c of this license or (at your option) any later
version. This version of this license is in
   http://www.latex-project.org/lppl/lppl-1-3c.txt
and the latest version of this license is in
   http://www.latex-project.org/lppl.txt
and version 1.3 or later is part of all distributions of
LaTeX version 2005/12/01 or later.

This work has the LPPL maintenance status "maintained".

This Current Maintainer of this work is Heiko Oberdiek.

The Base Interpreter refers to any `TeX-Format',
because some files are installed in TDS:tex/generic//.

This work consists of the main source file hologo.dtx
and the derived files
   hologo.sty, hologo.pdf, hologo.ins, hologo.drv, hologo-example.tex,
   hologo-test1.tex, hologo-test-spacefactor.tex,
   hologo-test-list.tex.

\endpreamble
\let\MetaPrefix\DoubleperCent

\generate{%
  \file{hologo.ins}{\from{hologo.dtx}{install}}%
  \file{hologo.drv}{\from{hologo.dtx}{driver}}%
  \usedir{tex/generic/oberdiek}%
  \file{hologo.sty}{\from{hologo.dtx}{package}}%
  \usedir{doc/latex/oberdiek/example}%
  \file{hologo-example.tex}{\from{hologo.dtx}{example}}%
  \usedir{doc/latex/oberdiek/test}%
  \file{hologo-test1.tex}{\from{hologo.dtx}{test1}}%
  \file{hologo-test-spacefactor.tex}{\from{hologo.dtx}{test-spacefactor}}%
  \file{hologo-test-list.tex}{\from{hologo.dtx}{test-list}}%
  \nopreamble
  \nopostamble
  \usedir{source/latex/oberdiek/catalogue}%
  \file{hologo.xml}{\from{hologo.dtx}{catalogue}}%
}

\catcode32=13\relax% active space
\let =\space%
\Msg{************************************************************************}
\Msg{*}
\Msg{* To finish the installation you have to move the following}
\Msg{* file into a directory searched by TeX:}
\Msg{*}
\Msg{*     hologo.sty}
\Msg{*}
\Msg{* To produce the documentation run the file `hologo.drv'}
\Msg{* through LaTeX.}
\Msg{*}
\Msg{* Happy TeXing!}
\Msg{*}
\Msg{************************************************************************}

\endbatchfile
%</install>
%<*ignore>
\fi
%</ignore>
%<*driver>
\NeedsTeXFormat{LaTeX2e}
\ProvidesFile{hologo.drv}%
  [2016/05/12 v1.11 A logo collection with bookmark support (HO)]%
\documentclass{ltxdoc}
\usepackage{holtxdoc}[2011/11/22]
\usepackage{hologo}[2016/05/12]
\usepackage{longtable}
\usepackage{array}
\usepackage{paralist}
%\usepackage[T1]{fontenc}
%\usepackage{lmodern}
\begin{document}
  \DocInput{hologo.dtx}%
\end{document}
%</driver>
% \fi
%
%
% \CharacterTable
%  {Upper-case    \A\B\C\D\E\F\G\H\I\J\K\L\M\N\O\P\Q\R\S\T\U\V\W\X\Y\Z
%   Lower-case    \a\b\c\d\e\f\g\h\i\j\k\l\m\n\o\p\q\r\s\t\u\v\w\x\y\z
%   Digits        \0\1\2\3\4\5\6\7\8\9
%   Exclamation   \!     Double quote  \"     Hash (number) \#
%   Dollar        \$     Percent       \%     Ampersand     \&
%   Acute accent  \'     Left paren    \(     Right paren   \)
%   Asterisk      \*     Plus          \+     Comma         \,
%   Minus         \-     Point         \.     Solidus       \/
%   Colon         \:     Semicolon     \;     Less than     \<
%   Equals        \=     Greater than  \>     Question mark \?
%   Commercial at \@     Left bracket  \[     Backslash     \\
%   Right bracket \]     Circumflex    \^     Underscore    \_
%   Grave accent  \`     Left brace    \{     Vertical bar  \|
%   Right brace   \}     Tilde         \~}
%
% \GetFileInfo{hologo.drv}
%
% \title{The \xpackage{hologo} package}
% \date{2016/05/12 v1.11}
% \author{Heiko Oberdiek\\\xemail{heiko.oberdiek at googlemail.com}}
%
% \maketitle
%
% \begin{abstract}
% This package starts a collection of logos with support for bookmarks
% strings.
% \end{abstract}
%
% \tableofcontents
%
% \section{Documentation}
%
% \subsection{Logo macros}
%
% \begin{declcs}{hologo} \M{name}
% \end{declcs}
% Macro \cs{hologo} sets the logo with name \meta{name}.
% The following table shows the supported names.
%
% \begingroup
%   \def\hologoEntry#1#2#3{^^A
%     #1&#2&\hologoLogoSetup{#1}{variant=#2}\hologo{#1}&#3\tabularnewline
%   }
%   \begin{longtable}{>{\ttfamily}l>{\ttfamily}lll}
%     \rmfamily\bfseries{name} & \rmfamily\bfseries variant
%     & \bfseries logo & \bfseries since\\
%     \hline
%     \endhead
%     \hologoList
%   \end{longtable}
% \endgroup
%
% \begin{declcs}{Hologo} \M{name}
% \end{declcs}
% Macro \cs{Hologo} starts the logo \meta{name} with an uppercase
% letter. As an exception small greek letters are not converted
% to uppercase. Examples, see \hologo{eTeX} and \hologo{ExTeX}.
%
% \subsection{Setup macros}
%
% The package does not support package options, but the following
% setup macros can be used to set options.
%
% \begin{declcs}{hologoSetup} \M{key value list}
% \end{declcs}
% Macro \cs{hologoSetup} sets global options.
%
% \begin{declcs}{hologoLogoSetup} \M{logo} \M{key value list}
% \end{declcs}
% Some options can also be used to configure a logo.
% These settings take precedence over global option settings.
%
% \subsection{Options}\label{sec:options}
%
% There are boolean and string options:
% \begin{description}
% \item[Boolean option:]
% It takes |true| or |false|
% as value. If the value is omitted, then |true| is used.
% \item[String option:]
% A value must be given as string. (But the string might be empty.)
% \end{description}
% The following options can be used both in \cs{hologoSetup}
% and \cs{hologoLogoSetup}:
% \begin{description}
% \def\entry#1{\item[\xoption{#1}:]}
% \entry{break}
%   enables or disables line breaks inside the logo. This setting is
%   refined by options \xoption{hyphenbreak}, \xoption{spacebreak}
%   or \xoption{discretionarybreak}.
%   Default is |false|.
% \entry{hyphenbreak}
%   enables or disables the line break right after the hyphen character.
% \entry{spacebreak}
%   enables or disables line breaks at space characters.
% \entry{discretionarybreak}
%   enables or disables line breaks at hyphenation points
%   (inserted by \cs{-}).
% \end{description}
% Macro \cs{hologoLogoSetup} also knows:
% \begin{description}
% \item[\xoption{variant}:]
%   This is a string option. It specifies a variant of a logo that
%   must exist. An empty string selects the package default variant.
% \end{description}
% Example:
% \begin{quote}
%   |\hologoSetup{break=false}|\\
%   |\hologoLogoSetup{plainTeX}{variant=hyphen,hyphenbreak}|\\
%   Then ``plain-\TeX'' contains one break point after the hyphen.
% \end{quote}
%
% \subsection{Driver options}
%
% Sometimes graphical operations are needed to construct some
% glyphs (e.g.\ \hologo{XeTeX}). If package \xpackage{graphics}
% or package \xpackage{pgf} are found, then the macros are taken
% from there. Otherwise the packge defines its own operations
% and therefore needs the driver information. Many drivers are
% detected automatically (\hologo{pdfTeX}/\hologo{LuaTeX}
% in PDF mode, \hologo{XeTeX}, \hologo{VTeX}). These have precedence
% over a driver option. The driver can be given as package option
% or using \cs{hologoDriverSetup}.
% The following list contains the recognized driver options:
% \begin{itemize}
% \item \xoption{pdftex}, \xoption{luatex}
% \item \xoption{dvipdfm}, \xoption{dvipdfmx}
% \item \xoption{dvips}, \xoption{dvipsone}, \xoption{xdvi}
% \item \xoption{xetex}
% \item \xoption{vtex}
% \end{itemize}
% The left driver of a line is the driver name that is used internally.
% The following names are aliases for drivers that use the
% same method. Therefore the entry in the \xext{log} file for
% the used driver prints the internally used driver name.
% \begin{description}
% \item[\xoption{driverfallback}:]
%   This option expects a driver that is used,
%   if the driver could not be detected automatically.
% \end{description}
%
% \begin{declcs}{hologoDriverSetup} \M{driver option}
% \end{declcs}
% The driver can also be configured after package loading
% using \cs{hologoDriverSetup}, also the way for \hologo{plainTeX}
% to setup the driver.
%
% \subsection{Font setup}
%
% Some logos require a special font, but should also be usable by
% \hologo{plainTeX}. Therefore the package provides some ways
% to influence the font settings. The options below
% take font settings as values. Both font commands
% such as \cs{sffamily} and macros that take one argument
% like \cs{textsf} can be used.
%
% \begin{declcs}{hologoFontSetup} \M{key value list}
% \end{declcs}
% Macro \cs{hologoFontSetup} sets the fonts for all logos.
% Supported keys:
% \begin{description}
% \def\entry#1{\item[\xoption{#1}:]}
% \entry{general}
%   This font is used for all logos. The default is empty.
%   That means no special font is used.
% \entry{bibsf}
%   This font is used for
%   {\hologoLogoSetup{BibTeX}{variant=sf}\hologo{BibTeX}}
%   with variant \xoption{sf}.
% \entry{rm}
%   This font is a serif font. It is used for \hologo{ExTeX}.
% \entry{sc}
%   This font specifies a small caps font. It is used for
%   {\hologoLogoSetup{BibTeX}{variant=sc}\hologo{BibTeX}}
%   with variant \xoption{sc}.
% \entry{sf}
%   This font specifies a sans serif font. The default
%   is \cs{sffamily}, then \cs{sf} is tried. Otherwise
%   a warning is given. It is used by \hologo{KOMAScript}.
% \entry{sy}
%   This is the font for math symbols (e.g. cmsy).
%   It is used by \hologo{AmS}, \hologo{NTS}, \hologo{ExTeX}.
% \entry{logo}
%   \hologo{METAFONT} and \hologo{METAPOST} are using that font.
%   In \hologo{LaTeX} \cs{logofamily} is used and
%   the definitions of package \xpackage{mflogo} are used
%   if the package is not loaded.
%   Otherwise the \cs{tenlogo} is used and defined
%   if it does not already exists.
% \end{description}
%
% \begin{declcs}{hologoLogoFontSetup} \M{logo} \M{key value list}
% \end{declcs}
% Fonts can also be set for a logo or logo component separately,
% see the following list.
% The keys are the same as for \cs{hologoFontSetup}.
%
% \begin{longtable}{>{\ttfamily}l>{\sffamily}ll}
%   \meta{logo} & keys & result\\
%   \hline
%   \endhead
%   BibTeX & bibsf & {\hologoLogoSetup{BibTeX}{variant=sf}\hologo{BibTeX}}\\[.5ex]
%   BibTeX & sc & {\hologoLogoSetup{BibTeX}{variant=sc}\hologo{BibTeX}}\\[.5ex]
%   ExTeX & rm & \hologo{ExTeX}\\
%   SliTeX & rm & \hologo{SliTeX}\\[.5ex]
%   AmS & sy & \hologo{AmS}\\
%   ExTeX & sy & \hologo{ExTeX}\\
%   NTS & sy & \hologo{NTS}\\[.5ex]
%   KOMAScript & sf & \hologo{KOMAScript}\\[.5ex]
%   METAFONT & logo & \hologo{METAFONT}\\
%   METAPOST & logo & \hologo{METAPOST}\\[.5ex]
%   SliTeX & sc \hologo{SliTeX}
% \end{longtable}
%
% \subsubsection{Font order}
%
% For all logos the font \xoption{general} is applied first.
% Example:
%\begin{quote}
%|\hologoFontSetup{general=\color{red}}|
%\end{quote}
% will print red logos.
% Then if the font uses a special font \xoption{sf}, for example,
% the font is applied that is setup by \cs{hologoLogoFontSetup}.
% If this font is not setup, then the common font setup
% by \cs{hologoFontSetup} is used. Otherwise a warning is given,
% that there is no font configured.
%
% \subsection{Additional user macros}
%
% Usually a variant of a logo is configured by using
% \cs{hologoLogoSetup}, because it is bad style to mix
% different variants of the same logo in the same text.
% There the following macros are a convenience for testing.
%
% \begin{declcs}{hologoVariant} \M{name} \M{variant}\\
%   \cs{HologoVariant} \M{name} \M{variant}
% \end{declcs}
% Logo \meta{name} is set using \meta{variant} that specifies
% explicitely which variant of the macro is used. If the argument
% is empty, then the default form of the logo is used
% (configurable by \cs{hologoLogoSetup}).
%
% \cs{HologoVariant} is used if the logo is set in a context
% that needs an uppercase first letter (beginning of a sentence, \dots).
%
% \begin{declcs}{hologoList}\\
%   \cs{hologoEntry} \M{logo} \M{variant} \M{since}
% \end{declcs}
% Macro \cs{hologoList} contains all logos that are provided
% by the package including variants. The list consists of calls
% of \cs{hologoEntry} with three arguments starting with the
% logo name \meta{logo} and its variant \meta{variant}. An empty
% variant means the current default. Argument \meta{since} specifies
% with version of the package \xpackage{hologo} is needed to get
% the logo. If the logo is fixed, then the date gets updated.
% Therefore the date \meta{since} is not exactly the date of
% the first introduction, but rather the date of the latest fix.
%
% Before \cs{hologoList} can be used, macro \cs{hologoEntry} needs
% a definition. The example file in section \ref{sec:example}
% shows applications of \cs{hologoList}.
%
% \subsection{Supported contexts}
%
% Macros \cs{hologo} and friends support special contexts:
% \begin{itemize}
% \item \hologo{LaTeX}'s protection mechanism.
% \item Bookmarks of package \xpackage{hyperref}.
% \item Package \xpackage{tex4ht}.
% \item The macros can be used inside \cs{csname} constructs,
%   if \cs{ifincsname} is available (\hologo{pdfTeX}, \hologo{XeTeX},
%   \hologo{LuaTeX}).
% \end{itemize}
%
% \subsection{Example}
% \label{sec:example}
%
% The following example prints the logos in different fonts.
%    \begin{macrocode}
%<*example>
%<<verbatim
\NeedsTeXFormat{LaTeX2e}
\documentclass[a4paper]{article}
\usepackage[
  hmargin=20mm,
  vmargin=20mm,
]{geometry}
\pagestyle{empty}
\usepackage{hologo}[2016/05/12]
\usepackage{longtable}
\usepackage{array}
\setlength{\extrarowheight}{2pt}
\usepackage[T1]{fontenc}
\usepackage{lmodern}
\usepackage{pdflscape}
\usepackage[
  pdfencoding=auto,
]{hyperref}
\hypersetup{
  pdfauthor={Heiko Oberdiek},
  pdftitle={Example for package `hologo'},
  pdfsubject={Logos with fonts lmr, lmss, qtm, qpl, qhv},
}
\usepackage{bookmark}

% Print the logo list on the console

\begingroup
  \typeout{}%
  \typeout{*** Begin of logo list ***}%
  \newcommand*{\hologoEntry}[3]{%
    \typeout{#1 \ifx\\#2\\\else(#2) \fi[#3]}%
  }%
  \hologoList
  \typeout{*** End of logo list ***}%
  \typeout{}%
\endgroup

\begin{document}
\begin{landscape}

  \section{Example file for package `hologo'}

  % Table for font names

  \begin{longtable}{>{\bfseries}ll}
    \textbf{font} & \textbf{Font name}\\
    \hline
    lmr & Latin Modern Roman\\
    lmss & Latin Modern Sans\\
    qtm & \TeX\ Gyre Termes\\
    qhv & \TeX\ Gyre Heros\\
    qpl & \TeX\ Gyre Pagella\\
  \end{longtable}

  % Logo list with logos in different fonts

  \begingroup
    \newcommand*{\SetVariant}[2]{%
      \ifx\\#2\\%
      \else
        \hologoLogoSetup{#1}{variant=#2}%
      \fi
    }%
    \newcommand*{\hologoEntry}[3]{%
      \SetVariant{#1}{#2}%
      \raisebox{1em}[0pt][0pt]{\hypertarget{#1@#2}{}}%
      \bookmark[%
        dest={#1@#2},%
      ]{%
        #1\ifx\\#2\\\else\space(#2)\fi: \Hologo{#1}, \hologo{#1} %
        [Unicode]%
      }%
      \hypersetup{unicode=false}%
      \bookmark[%
        dest={#1@#2},%
      ]{%
        #1\ifx\\#2\\\else\space(#2)\fi: \Hologo{#1}, \hologo{#1} %
        [PDFDocEncoding]%
      }%
      \texttt{#1}%
      &%
      \texttt{#2}%
      &%
      \Hologo{#1}%
      &%
      \SetVariant{#1}{#2}%
      \hologo{#1}%
      &%
      \SetVariant{#1}{#2}%
      \fontfamily{qtm}\selectfont
      \hologo{#1}%
      &%
      \SetVariant{#1}{#2}%
      \fontfamily{qpl}\selectfont
      \hologo{#1}%
      &%
      \SetVariant{#1}{#2}%
      \textsf{\hologo{#1}}%
      &%
      \SetVariant{#1}{#2}%
      \fontfamily{qhv}\selectfont
      \hologo{#1}%
      \tabularnewline
    }%
    \begin{longtable}{llllllll}%
      \textbf{\textit{logo}} & \textbf{\textit{variant}} &
      \texttt{\string\Hologo} &
      \textbf{lmr} & \textbf{qtm} & \textbf{qpl} &
      \textbf{lmss} & \textbf{qhv}
      \tabularnewline
      \hline
      \endhead
      \hologoList
    \end{longtable}%
  \endgroup

\end{landscape}
\end{document}
%verbatim
%</example>
%    \end{macrocode}
%
% \StopEventually{
% }
%
% \section{Implementation}
%    \begin{macrocode}
%<*package>
%    \end{macrocode}
%    Reload check, especially if the package is not used with \LaTeX.
%    \begin{macrocode}
\begingroup\catcode61\catcode48\catcode32=10\relax%
  \catcode13=5 % ^^M
  \endlinechar=13 %
  \catcode35=6 % #
  \catcode39=12 % '
  \catcode44=12 % ,
  \catcode45=12 % -
  \catcode46=12 % .
  \catcode58=12 % :
  \catcode64=11 % @
  \catcode123=1 % {
  \catcode125=2 % }
  \expandafter\let\expandafter\x\csname ver@hologo.sty\endcsname
  \ifx\x\relax % plain-TeX, first loading
  \else
    \def\empty{}%
    \ifx\x\empty % LaTeX, first loading,
      % variable is initialized, but \ProvidesPackage not yet seen
    \else
      \expandafter\ifx\csname PackageInfo\endcsname\relax
        \def\x#1#2{%
          \immediate\write-1{Package #1 Info: #2.}%
        }%
      \else
        \def\x#1#2{\PackageInfo{#1}{#2, stopped}}%
      \fi
      \x{hologo}{The package is already loaded}%
      \aftergroup\endinput
    \fi
  \fi
\endgroup%
%    \end{macrocode}
%    Package identification:
%    \begin{macrocode}
\begingroup\catcode61\catcode48\catcode32=10\relax%
  \catcode13=5 % ^^M
  \endlinechar=13 %
  \catcode35=6 % #
  \catcode39=12 % '
  \catcode40=12 % (
  \catcode41=12 % )
  \catcode44=12 % ,
  \catcode45=12 % -
  \catcode46=12 % .
  \catcode47=12 % /
  \catcode58=12 % :
  \catcode64=11 % @
  \catcode91=12 % [
  \catcode93=12 % ]
  \catcode123=1 % {
  \catcode125=2 % }
  \expandafter\ifx\csname ProvidesPackage\endcsname\relax
    \def\x#1#2#3[#4]{\endgroup
      \immediate\write-1{Package: #3 #4}%
      \xdef#1{#4}%
    }%
  \else
    \def\x#1#2[#3]{\endgroup
      #2[{#3}]%
      \ifx#1\@undefined
        \xdef#1{#3}%
      \fi
      \ifx#1\relax
        \xdef#1{#3}%
      \fi
    }%
  \fi
\expandafter\x\csname ver@hologo.sty\endcsname
\ProvidesPackage{hologo}%
  [2016/05/12 v1.11 A logo collection with bookmark support (HO)]%
%    \end{macrocode}
%
%    \begin{macrocode}
\begingroup\catcode61\catcode48\catcode32=10\relax%
  \catcode13=5 % ^^M
  \endlinechar=13 %
  \catcode123=1 % {
  \catcode125=2 % }
  \catcode64=11 % @
  \def\x{\endgroup
    \expandafter\edef\csname HOLOGO@AtEnd\endcsname{%
      \endlinechar=\the\endlinechar\relax
      \catcode13=\the\catcode13\relax
      \catcode32=\the\catcode32\relax
      \catcode35=\the\catcode35\relax
      \catcode61=\the\catcode61\relax
      \catcode64=\the\catcode64\relax
      \catcode123=\the\catcode123\relax
      \catcode125=\the\catcode125\relax
    }%
  }%
\x\catcode61\catcode48\catcode32=10\relax%
\catcode13=5 % ^^M
\endlinechar=13 %
\catcode35=6 % #
\catcode64=11 % @
\catcode123=1 % {
\catcode125=2 % }
\def\TMP@EnsureCode#1#2{%
  \edef\HOLOGO@AtEnd{%
    \HOLOGO@AtEnd
    \catcode#1=\the\catcode#1\relax
  }%
  \catcode#1=#2\relax
}
\TMP@EnsureCode{10}{12}% ^^J
\TMP@EnsureCode{33}{12}% !
\TMP@EnsureCode{34}{12}% "
\TMP@EnsureCode{36}{3}% $
\TMP@EnsureCode{38}{4}% &
\TMP@EnsureCode{39}{12}% '
\TMP@EnsureCode{40}{12}% (
\TMP@EnsureCode{41}{12}% )
\TMP@EnsureCode{42}{12}% *
\TMP@EnsureCode{43}{12}% +
\TMP@EnsureCode{44}{12}% ,
\TMP@EnsureCode{45}{12}% -
\TMP@EnsureCode{46}{12}% .
\TMP@EnsureCode{47}{12}% /
\TMP@EnsureCode{58}{12}% :
\TMP@EnsureCode{59}{12}% ;
\TMP@EnsureCode{60}{12}% <
\TMP@EnsureCode{62}{12}% >
\TMP@EnsureCode{63}{12}% ?
\TMP@EnsureCode{91}{12}% [
\TMP@EnsureCode{93}{12}% ]
\TMP@EnsureCode{94}{7}% ^ (superscript)
\TMP@EnsureCode{95}{8}% _ (subscript)
\TMP@EnsureCode{96}{12}% `
\TMP@EnsureCode{124}{12}% |
\edef\HOLOGO@AtEnd{%
  \HOLOGO@AtEnd
  \escapechar\the\escapechar\relax
  \noexpand\endinput
}
\escapechar=92 %
%    \end{macrocode}
%
% \subsection{Logo list}
%
%    \begin{macro}{\hologoList}
%    \begin{macrocode}
\def\hologoList{%
  \hologoEntry{(La)TeX}{}{2011/10/01}%
  \hologoEntry{AmSLaTeX}{}{2010/04/16}%
  \hologoEntry{AmSTeX}{}{2010/04/16}%
  \hologoEntry{biber}{}{2011/10/01}%
  \hologoEntry{BibTeX}{}{2011/10/01}%
  \hologoEntry{BibTeX}{sf}{2011/10/01}%
  \hologoEntry{BibTeX}{sc}{2011/10/01}%
  \hologoEntry{BibTeX8}{}{2011/11/22}%
  \hologoEntry{ConTeXt}{}{2011/03/25}%
  \hologoEntry{ConTeXt}{narrow}{2011/03/25}%
  \hologoEntry{ConTeXt}{simple}{2011/03/25}%
  \hologoEntry{emTeX}{}{2010/04/26}%
  \hologoEntry{eTeX}{}{2010/04/08}%
  \hologoEntry{ExTeX}{}{2011/10/01}%
  \hologoEntry{HanTheThanh}{}{2011/11/29}%
  \hologoEntry{iniTeX}{}{2011/10/01}%
  \hologoEntry{KOMAScript}{}{2011/10/01}%
  \hologoEntry{La}{}{2010/05/08}%
  \hologoEntry{LaTeX}{}{2010/04/08}%
  \hologoEntry{LaTeX2e}{}{2010/04/08}%
  \hologoEntry{LaTeX3}{}{2010/04/24}%
  \hologoEntry{LaTeXe}{}{2010/04/08}%
  \hologoEntry{LaTeXML}{}{2011/11/22}%
  \hologoEntry{LaTeXTeX}{}{2011/10/01}%
  \hologoEntry{LuaLaTeX}{}{2010/04/08}%
  \hologoEntry{LuaTeX}{}{2010/04/08}%
  \hologoEntry{LyX}{}{2011/10/01}%
  \hologoEntry{METAFONT}{}{2011/10/01}%
  \hologoEntry{MetaFun}{}{2011/10/01}%
  \hologoEntry{METAPOST}{}{2011/10/01}%
  \hologoEntry{MetaPost}{}{2011/10/01}%
  \hologoEntry{MiKTeX}{}{2011/10/01}%
  \hologoEntry{NTS}{}{2011/10/01}%
  \hologoEntry{OzMF}{}{2011/10/01}%
  \hologoEntry{OzMP}{}{2011/10/01}%
  \hologoEntry{OzTeX}{}{2011/10/01}%
  \hologoEntry{OzTtH}{}{2011/10/01}%
  \hologoEntry{PCTeX}{}{2011/10/01}%
  \hologoEntry{pdfTeX}{}{2011/10/01}%
  \hologoEntry{pdfLaTeX}{}{2011/10/01}%
  \hologoEntry{PiC}{}{2011/10/01}%
  \hologoEntry{PiCTeX}{}{2011/10/01}%
  \hologoEntry{plainTeX}{}{2010/04/08}%
  \hologoEntry{plainTeX}{space}{2010/04/16}%
  \hologoEntry{plainTeX}{hyphen}{2010/04/16}%
  \hologoEntry{plainTeX}{runtogether}{2010/04/16}%
  \hologoEntry{SageTeX}{}{2011/11/22}%
  \hologoEntry{SLiTeX}{}{2011/10/01}%
  \hologoEntry{SLiTeX}{lift}{2011/10/01}%
  \hologoEntry{SLiTeX}{narrow}{2011/10/01}%
  \hologoEntry{SLiTeX}{simple}{2011/10/01}%
  \hologoEntry{SliTeX}{}{2011/10/01}%
  \hologoEntry{SliTeX}{narrow}{2011/10/01}%
  \hologoEntry{SliTeX}{simple}{2011/10/01}%
  \hologoEntry{SliTeX}{lift}{2011/10/01}%
  \hologoEntry{teTeX}{}{2011/10/01}%
  \hologoEntry{TeX}{}{2010/04/08}%
  \hologoEntry{TeX4ht}{}{2011/11/22}%
  \hologoEntry{TTH}{}{2011/11/22}%
  \hologoEntry{virTeX}{}{2011/10/01}%
  \hologoEntry{VTeX}{}{2010/04/24}%
  \hologoEntry{Xe}{}{2010/04/08}%
  \hologoEntry{XeLaTeX}{}{2010/04/08}%
  \hologoEntry{XeTeX}{}{2010/04/08}%
}
%    \end{macrocode}
%    \end{macro}
%
% \subsection{Load resources}
%
%    \begin{macrocode}
\begingroup\expandafter\expandafter\expandafter\endgroup
\expandafter\ifx\csname RequirePackage\endcsname\relax
  \def\TMP@RequirePackage#1[#2]{%
    \begingroup\expandafter\expandafter\expandafter\endgroup
    \expandafter\ifx\csname ver@#1.sty\endcsname\relax
      \input #1.sty\relax
    \fi
  }%
  \TMP@RequirePackage{ltxcmds}[2011/02/04]%
  \TMP@RequirePackage{infwarerr}[2010/04/08]%
  \TMP@RequirePackage{kvsetkeys}[2010/03/01]%
  \TMP@RequirePackage{kvdefinekeys}[2010/03/01]%
  \TMP@RequirePackage{pdftexcmds}[2010/04/01]%
  \TMP@RequirePackage{ifpdf}[2010/01/28]%
  \TMP@RequirePackage{ifluatex}[2010/03/01]%
  \ltx@IfUndefined{newif}{%
    \expandafter\let\csname newif\endcsname\ltx@newif
  }{}%
  \TMP@RequirePackage{ifxetex}[2009/01/23]%
  \TMP@RequirePackage{ifvtex}[2010/03/01]%
\else
  \RequirePackage{ltxcmds}[2011/02/04]%
  \RequirePackage{infwarerr}[2010/04/08]%
  \RequirePackage{kvsetkeys}[2010/03/01]%
  \RequirePackage{kvdefinekeys}[2010/03/01]%
  \RequirePackage{pdftexcmds}[2010/04/01]%
  \RequirePackage{ifpdf}[2010/01/28]%
  \RequirePackage{ifluatex}[2010/03/01]%
  \RequirePackage{ifxetex}[2009/01/23]%
  \RequirePackage{ifvtex}[2010/03/01]%
\fi
%    \end{macrocode}
%
%    \begin{macro}{\HOLOGO@IfDefined}
%    \begin{macrocode}
\def\HOLOGO@IfExists#1{%
  \ifx\@undefined#1%
    \expandafter\ltx@secondoftwo
  \else
    \ifx\relax#1%
      \expandafter\ltx@secondoftwo
    \else
      \expandafter\expandafter\expandafter\ltx@firstoftwo
    \fi
  \fi
}
%    \end{macrocode}
%    \end{macro}
%
% \subsection{Setup macros}
%
%    \begin{macro}{\hologoSetup}
%    \begin{macrocode}
\def\hologoSetup{%
  \let\HOLOGO@name\relax
  \HOLOGO@Setup
}
%    \end{macrocode}
%    \end{macro}
%
%    \begin{macro}{\hologoLogoSetup}
%    \begin{macrocode}
\def\hologoLogoSetup#1{%
  \edef\HOLOGO@name{#1}%
  \ltx@IfUndefined{HoLogo@\HOLOGO@name}{%
    \@PackageError{hologo}{%
      Unknown logo `\HOLOGO@name'%
    }\@ehc
    \ltx@gobble
  }{%
    \HOLOGO@Setup
  }%
}
%    \end{macrocode}
%    \end{macro}
%
%    \begin{macro}{\HOLOGO@Setup}
%    \begin{macrocode}
\def\HOLOGO@Setup{%
  \kvsetkeys{HoLogo}%
}
%    \end{macrocode}
%    \end{macro}
%
% \subsection{Options}
%
%    \begin{macro}{\HOLOGO@DeclareBoolOption}
%    \begin{macrocode}
\def\HOLOGO@DeclareBoolOption#1{%
  \expandafter\chardef\csname HOLOGOOPT@#1\endcsname\ltx@zero
  \kv@define@key{HoLogo}{#1}[true]{%
    \def\HOLOGO@temp{##1}%
    \ifx\HOLOGO@temp\HOLOGO@true
      \ifx\HOLOGO@name\relax
        \expandafter\chardef\csname HOLOGOOPT@#1\endcsname=\ltx@one
      \else
        \expandafter\chardef\csname
        HoLogoOpt@#1@\HOLOGO@name\endcsname\ltx@one
      \fi
      \HOLOGO@SetBreakAll{#1}%
    \else
      \ifx\HOLOGO@temp\HOLOGO@false
        \ifx\HOLOGO@name\relax
          \expandafter\chardef\csname HOLOGOOPT@#1\endcsname=\ltx@zero
        \else
          \expandafter\chardef\csname
          HoLogoOpt@#1@\HOLOGO@name\endcsname=\ltx@zero
        \fi
        \HOLOGO@SetBreakAll{#1}%
      \else
        \@PackageError{hologo}{%
          Unknown value `##1' for boolean option `#1'.\MessageBreak
          Known values are `true' and `false'%
        }\@ehc
      \fi
    \fi
  }%
}
%    \end{macrocode}
%    \end{macro}
%
%    \begin{macro}{\HOLOGO@SetBreakAll}
%    \begin{macrocode}
\def\HOLOGO@SetBreakAll#1{%
  \def\HOLOGO@temp{#1}%
  \ifx\HOLOGO@temp\HOLOGO@break
    \ifx\HOLOGO@name\relax
      \chardef\HOLOGOOPT@hyphenbreak=\HOLOGOOPT@break
      \chardef\HOLOGOOPT@spacebreak=\HOLOGOOPT@break
      \chardef\HOLOGOOPT@discretionarybreak=\HOLOGOOPT@break
    \else
      \expandafter\chardef
         \csname HoLogoOpt@hyphenbreak@\HOLOGO@name\endcsname=%
         \csname HoLogoOpt@break@\HOLOGO@name\endcsname
      \expandafter\chardef
         \csname HoLogoOpt@spacebreak@\HOLOGO@name\endcsname=%
         \csname HoLogoOpt@break@\HOLOGO@name\endcsname
      \expandafter\chardef
         \csname HoLogoOpt@discretionarybreak@\HOLOGO@name
             \endcsname=%
         \csname HoLogoOpt@break@\HOLOGO@name\endcsname
    \fi
  \fi
}
%    \end{macrocode}
%    \end{macro}
%
%    \begin{macro}{\HOLOGO@true}
%    \begin{macrocode}
\def\HOLOGO@true{true}
%    \end{macrocode}
%    \end{macro}
%    \begin{macro}{\HOLOGO@false}
%    \begin{macrocode}
\def\HOLOGO@false{false}
%    \end{macrocode}
%    \end{macro}
%    \begin{macro}{\HOLOGO@break}
%    \begin{macrocode}
\def\HOLOGO@break{break}
%    \end{macrocode}
%    \end{macro}
%
%    \begin{macrocode}
\HOLOGO@DeclareBoolOption{break}
\HOLOGO@DeclareBoolOption{hyphenbreak}
\HOLOGO@DeclareBoolOption{spacebreak}
\HOLOGO@DeclareBoolOption{discretionarybreak}
%    \end{macrocode}
%
%    \begin{macrocode}
\kv@define@key{HoLogo}{variant}{%
  \ifx\HOLOGO@name\relax
    \@PackageError{hologo}{%
      Option `variant' is not available in \string\hologoSetup,%
      \MessageBreak
      Use \string\hologoLogoSetup\space instead%
    }\@ehc
  \else
    \edef\HOLOGO@temp{#1}%
    \ifx\HOLOGO@temp\ltx@empty
      \expandafter
      \let\csname HoLogoOpt@variant@\HOLOGO@name\endcsname\@undefined
    \else
      \ltx@IfUndefined{HoLogo@\HOLOGO@name @\HOLOGO@temp}{%
        \@PackageError{hologo}{%
          Unknown variant `\HOLOGO@temp' of logo `\HOLOGO@name'%
        }\@ehc
      }{%
        \expandafter
        \let\csname HoLogoOpt@variant@\HOLOGO@name\endcsname
            \HOLOGO@temp
      }%
    \fi
  \fi
}
%    \end{macrocode}
%
%    \begin{macro}{\HOLOGO@Variant}
%    \begin{macrocode}
\def\HOLOGO@Variant#1{%
  #1%
  \ltx@ifundefined{HoLogoOpt@variant@#1}{%
  }{%
    @\csname HoLogoOpt@variant@#1\endcsname
  }%
}
%    \end{macrocode}
%    \end{macro}
%
% \subsection{Break/no-break support}
%
%    \begin{macro}{\HOLOGO@space}
%    \begin{macrocode}
\def\HOLOGO@space{%
  \ltx@ifundefined{HoLogoOpt@spacebreak@\HOLOGO@name}{%
    \ltx@ifundefined{HoLogoOpt@break@\HOLOGO@name}{%
      \chardef\HOLOGO@temp=\HOLOGOOPT@spacebreak
    }{%
      \chardef\HOLOGO@temp=%
        \csname HoLogoOpt@break@\HOLOGO@name\endcsname
    }%
  }{%
    \chardef\HOLOGO@temp=%
      \csname HoLogoOpt@spacebreak@\HOLOGO@name\endcsname
  }%
  \ifcase\HOLOGO@temp
    \penalty10000 %
  \fi
  \ltx@space
}
%    \end{macrocode}
%    \end{macro}
%
%    \begin{macro}{\HOLOGO@hyphen}
%    \begin{macrocode}
\def\HOLOGO@hyphen{%
  \ltx@ifundefined{HoLogoOpt@hyphenbreak@\HOLOGO@name}{%
    \ltx@ifundefined{HoLogoOpt@break@\HOLOGO@name}{%
      \chardef\HOLOGO@temp=\HOLOGOOPT@hyphenbreak
    }{%
      \chardef\HOLOGO@temp=%
        \csname HoLogoOpt@break@\HOLOGO@name\endcsname
    }%
  }{%
    \chardef\HOLOGO@temp=%
      \csname HoLogoOpt@hyphenbreak@\HOLOGO@name\endcsname
  }%
  \ifcase\HOLOGO@temp
    \ltx@mbox{-}%
  \else
    -%
  \fi
}
%    \end{macrocode}
%    \end{macro}
%
%    \begin{macro}{\HOLOGO@discretionary}
%    \begin{macrocode}
\def\HOLOGO@discretionary{%
  \ltx@ifundefined{HoLogoOpt@discretionarybreak@\HOLOGO@name}{%
    \ltx@ifundefined{HoLogoOpt@break@\HOLOGO@name}{%
      \chardef\HOLOGO@temp=\HOLOGOOPT@discretionarybreak
    }{%
      \chardef\HOLOGO@temp=%
        \csname HoLogoOpt@break@\HOLOGO@name\endcsname
    }%
  }{%
    \chardef\HOLOGO@temp=%
      \csname HoLogoOpt@discretionarybreak@\HOLOGO@name\endcsname
  }%
  \ifcase\HOLOGO@temp
  \else
    \-%
  \fi
}
%    \end{macrocode}
%    \end{macro}
%
%    \begin{macro}{\HOLOGO@mbox}
%    \begin{macrocode}
\def\HOLOGO@mbox#1{%
  \ltx@ifundefined{HoLogoOpt@break@\HOLOGO@name}{%
    \chardef\HOLOGO@temp=\HOLOGOOPT@hyphenbreak
  }{%
    \chardef\HOLOGO@temp=%
      \csname HoLogoOpt@break@\HOLOGO@name\endcsname
  }%
  \ifcase\HOLOGO@temp
    \ltx@mbox{#1}%
  \else
    #1%
  \fi
}
%    \end{macrocode}
%    \end{macro}
%
% \subsection{Font support}
%
%    \begin{macro}{\HoLogoFont@font}
%    \begin{tabular}{@{}ll@{}}
%    |#1|:& logo name\\
%    |#2|:& font short name\\
%    |#3|:& text
%    \end{tabular}
%    \begin{macrocode}
\def\HoLogoFont@font#1#2#3{%
  \begingroup
    \ltx@IfUndefined{HoLogoFont@logo@#1.#2}{%
      \ltx@IfUndefined{HoLogoFont@font@#2}{%
        \@PackageWarning{hologo}{%
          Missing font `#2' for logo `#1'%
        }%
        #3%
      }{%
        \csname HoLogoFont@font@#2\endcsname{#3}%
      }%
    }{%
      \csname HoLogoFont@logo@#1.#2\endcsname{#3}%
    }%
  \endgroup
}
%    \end{macrocode}
%    \end{macro}
%
%    \begin{macro}{\HoLogoFont@Def}
%    \begin{macrocode}
\def\HoLogoFont@Def#1{%
  \expandafter\def\csname HoLogoFont@font@#1\endcsname
}
%    \end{macrocode}
%    \end{macro}
%    \begin{macro}{\HoLogoFont@LogoDef}
%    \begin{macrocode}
\def\HoLogoFont@LogoDef#1#2{%
  \expandafter\def\csname HoLogoFont@logo@#1.#2\endcsname
}
%    \end{macrocode}
%    \end{macro}
%
% \subsubsection{Font defaults}
%
%    \begin{macro}{\HoLogoFont@font@general}
%    \begin{macrocode}
\HoLogoFont@Def{general}{}%
%    \end{macrocode}
%    \end{macro}
%
%    \begin{macro}{\HoLogoFont@font@rm}
%    \begin{macrocode}
\ltx@IfUndefined{rmfamily}{%
  \ltx@IfUndefined{rm}{%
  }{%
    \HoLogoFont@Def{rm}{\rm}%
  }%
}{%
  \HoLogoFont@Def{rm}{\rmfamily}%
}
%    \end{macrocode}
%    \end{macro}
%
%    \begin{macro}{\HoLogoFont@font@sf}
%    \begin{macrocode}
\ltx@IfUndefined{sffamily}{%
  \ltx@IfUndefined{sf}{%
  }{%
    \HoLogoFont@Def{sf}{\sf}%
  }%
}{%
  \HoLogoFont@Def{sf}{\sffamily}%
}
%    \end{macrocode}
%    \end{macro}
%
%    \begin{macro}{\HoLogoFont@font@bibsf}
%    In case of \hologo{plainTeX} the original small caps
%    variant is used as default. In \hologo{LaTeX}
%    the definition of package \xpackage{dtklogos} \cite{dtklogos}
%    is used.
%\begin{quote}
%\begin{verbatim}
%\DeclareRobustCommand{\BibTeX}{%
%  B%
%  \kern-.05em%
%  \hbox{%
%    $\m@th$% %% force math size calculations
%    \csname S@\f@size\endcsname
%    \fontsize\sf@size\z@
%    \math@fontsfalse
%    \selectfont
%    I%
%    \kern-.025em%
%    B
%  }%
%  \kern-.08em%
%  \-%
%  \TeX
%}
%\end{verbatim}
%\end{quote}
%    \begin{macrocode}
\ltx@IfUndefined{selectfont}{%
  \ltx@IfUndefined{tensc}{%
    \font\tensc=cmcsc10\relax
  }{}%
  \HoLogoFont@Def{bibsf}{\tensc}%
}{%
  \HoLogoFont@Def{bibsf}{%
    $\mathsurround=0pt$%
    \csname S@\f@size\endcsname
    \fontsize\sf@size{0pt}%
    \math@fontsfalse
    \selectfont
  }%
}
%    \end{macrocode}
%    \end{macro}
%
%    \begin{macro}{\HoLogoFont@font@sc}
%    \begin{macrocode}
\ltx@IfUndefined{scshape}{%
  \ltx@IfUndefined{tensc}{%
    \font\tensc=cmcsc10\relax
  }{}%
  \HoLogoFont@Def{sc}{\tensc}%
}{%
  \HoLogoFont@Def{sc}{\scshape}%
}
%    \end{macrocode}
%    \end{macro}
%
%    \begin{macro}{\HoLogoFont@font@sy}
%    \begin{macrocode}
\ltx@IfUndefined{usefont}{%
  \ltx@IfUndefined{tensy}{%
  }{%
    \HoLogoFont@Def{sy}{\tensy}%
  }%
}{%
  \HoLogoFont@Def{sy}{%
    \usefont{OMS}{cmsy}{m}{n}%
  }%
}
%    \end{macrocode}
%    \end{macro}
%
%    \begin{macro}{\HoLogoFont@font@logo}
%    \begin{macrocode}
\begingroup
  \def\x{LaTeX2e}%
\expandafter\endgroup
\ifx\fmtname\x
  \ltx@IfUndefined{logofamily}{%
    \DeclareRobustCommand\logofamily{%
      \not@math@alphabet\logofamily\relax
      \fontencoding{U}%
      \fontfamily{logo}%
      \selectfont
    }%
  }{}%
  \ltx@IfUndefined{logofamily}{%
  }{%
    \HoLogoFont@Def{logo}{\logofamily}%
  }%
\else
  \ltx@IfUndefined{tenlogo}{%
    \font\tenlogo=logo10\relax
  }{}%
  \HoLogoFont@Def{logo}{\tenlogo}%
\fi
%    \end{macrocode}
%    \end{macro}
%
% \subsubsection{Font setup}
%
%    \begin{macro}{\hologoFontSetup}
%    \begin{macrocode}
\def\hologoFontSetup{%
  \let\HOLOGO@name\relax
  \HOLOGO@FontSetup
}
%    \end{macrocode}
%    \end{macro}
%
%    \begin{macro}{\hologoLogoFontSetup}
%    \begin{macrocode}
\def\hologoLogoFontSetup#1{%
  \edef\HOLOGO@name{#1}%
  \ltx@IfUndefined{HoLogo@\HOLOGO@name}{%
    \@PackageError{hologo}{%
      Unknown logo `\HOLOGO@name'%
    }\@ehc
    \ltx@gobble
  }{%
    \HOLOGO@FontSetup
  }%
}
%    \end{macrocode}
%    \end{macro}
%
%    \begin{macro}{\HOLOGO@FontSetup}
%    \begin{macrocode}
\def\HOLOGO@FontSetup{%
  \kvsetkeys{HoLogoFont}%
}
%    \end{macrocode}
%    \end{macro}
%
%    \begin{macrocode}
\def\HOLOGO@temp#1{%
  \kv@define@key{HoLogoFont}{#1}{%
    \ifx\HOLOGO@name\relax
      \HoLogoFont@Def{#1}{##1}%
    \else
      \HoLogoFont@LogoDef\HOLOGO@name{#1}{##1}%
    \fi
  }%
}
\HOLOGO@temp{general}
\HOLOGO@temp{sf}
%    \end{macrocode}
%
% \subsection{Generic logo commands}
%
%    \begin{macrocode}
\HOLOGO@IfExists\hologo{%
  \@PackageError{hologo}{%
    \string\hologo\ltx@space is already defined.\MessageBreak
    Package loading is aborted%
  }\@ehc
  \HOLOGO@AtEnd
}%
\HOLOGO@IfExists\hologoRobust{%
  \@PackageError{hologo}{%
    \string\hologoRobust\ltx@space is already defined.\MessageBreak
    Package loading is aborted%
  }\@ehc
  \HOLOGO@AtEnd
}%
%    \end{macrocode}
%
% \subsubsection{\cs{hologo} and friends}
%
%    \begin{macrocode}
\ifluatex
  \expandafter\ltx@firstofone
\else
  \expandafter\ltx@gobble
\fi
{%
  \ltx@IfUndefined{ifincsname}{%
    \ifnum\luatexversion<36 %
      \expandafter\ltx@gobble
    \else
      \expandafter\ltx@firstofone
    \fi
    {%
      \begingroup
        \ifcase0%
            \directlua{%
              if tex.enableprimitives then %
                tex.enableprimitives('HOLOGO@', {'ifincsname'})%
              else %
                tex.print('1')%
              end%
            }%
            \ifx\HOLOGO@ifincsname\@undefined 1\fi%
            \relax
          \expandafter\ltx@firstofone
        \else
          \endgroup
          \expandafter\ltx@gobble
        \fi
        {%
          \global\let\ifincsname\HOLOGO@ifincsname
        }%
      \HOLOGO@temp
    }%
  }{}%
}
%    \end{macrocode}
%    \begin{macrocode}
\ltx@IfUndefined{ifincsname}{%
  \catcode`$=14 %
}{%
  \catcode`$=9 %
}
%    \end{macrocode}
%
%    \begin{macro}{\hologo}
%    \begin{macrocode}
\def\hologo#1{%
$ \ifincsname
$   \ltx@ifundefined{HoLogoCs@\HOLOGO@Variant{#1}}{%
$     #1%
$   }{%
$     \csname HoLogoCs@\HOLOGO@Variant{#1}\endcsname\ltx@firstoftwo
$   }%
$ \else
    \HOLOGO@IfExists\texorpdfstring\texorpdfstring\ltx@firstoftwo
    {%
      \hologoRobust{#1}%
    }{%
      \ltx@ifundefined{HoLogoBkm@\HOLOGO@Variant{#1}}{%
        \ltx@ifundefined{HoLogo@#1}{?#1?}{#1}%
      }{%
        \csname HoLogoBkm@\HOLOGO@Variant{#1}\endcsname
        \ltx@firstoftwo
      }%
    }%
$ \fi
}
%    \end{macrocode}
%    \end{macro}
%    \begin{macro}{\Hologo}
%    \begin{macrocode}
\def\Hologo#1{%
$ \ifincsname
$   \ltx@ifundefined{HoLogoCs@\HOLOGO@Variant{#1}}{%
$     #1%
$   }{%
$     \csname HoLogoCs@\HOLOGO@Variant{#1}\endcsname\ltx@secondoftwo
$   }%
$ \else
    \HOLOGO@IfExists\texorpdfstring\texorpdfstring\ltx@firstoftwo
    {%
      \HologoRobust{#1}%
    }{%
      \ltx@ifundefined{HoLogoBkm@\HOLOGO@Variant{#1}}{%
        \ltx@ifundefined{HoLogo@#1}{?#1?}{#1}%
      }{%
        \csname HoLogoBkm@\HOLOGO@Variant{#1}\endcsname
        \ltx@secondoftwo
      }%
    }%
$ \fi
}
%    \end{macrocode}
%    \end{macro}
%
%    \begin{macro}{\hologoVariant}
%    \begin{macrocode}
\def\hologoVariant#1#2{%
  \ifx\relax#2\relax
    \hologo{#1}%
  \else
$   \ifincsname
$     \ltx@ifundefined{HoLogoCs@#1@#2}{%
$       #1%
$     }{%
$       \csname HoLogoCs@#1@#2\endcsname\ltx@firstoftwo
$     }%
$   \else
      \HOLOGO@IfExists\texorpdfstring\texorpdfstring\ltx@firstoftwo
      {%
        \hologoVariantRobust{#1}{#2}%
      }{%
        \ltx@ifundefined{HoLogoBkm@#1@#2}{%
          \ltx@ifundefined{HoLogo@#1}{?#1?}{#1}%
        }{%
          \csname HoLogoBkm@#1@#2\endcsname
          \ltx@firstoftwo
        }%
      }%
$   \fi
  \fi
}
%    \end{macrocode}
%    \end{macro}
%    \begin{macro}{\HologoVariant}
%    \begin{macrocode}
\def\HologoVariant#1#2{%
  \ifx\relax#2\relax
    \Hologo{#1}%
  \else
$   \ifincsname
$     \ltx@ifundefined{HoLogoCs@#1@#2}{%
$       #1%
$     }{%
$       \csname HoLogoCs@#1@#2\endcsname\ltx@secondoftwo
$     }%
$   \else
      \HOLOGO@IfExists\texorpdfstring\texorpdfstring\ltx@firstoftwo
      {%
        \HologoVariantRobust{#1}{#2}%
      }{%
        \ltx@ifundefined{HoLogoBkm@#1@#2}{%
          \ltx@ifundefined{HoLogo@#1}{?#1?}{#1}%
        }{%
          \csname HoLogoBkm@#1@#2\endcsname
          \ltx@secondoftwo
        }%
      }%
$   \fi
  \fi
}
%    \end{macrocode}
%    \end{macro}
%
%    \begin{macrocode}
\catcode`\$=3 %
%    \end{macrocode}
%
% \subsubsection{\cs{hologoRobust} and friends}
%
%    \begin{macro}{\hologoRobust}
%    \begin{macrocode}
\ltx@IfUndefined{protected}{%
  \ltx@IfUndefined{DeclareRobustCommand}{%
    \def\hologoRobust#1%
  }{%
    \DeclareRobustCommand*\hologoRobust[1]%
  }%
}{%
  \protected\def\hologoRobust#1%
}%
{%
  \edef\HOLOGO@name{#1}%
  \ltx@IfUndefined{HoLogo@\HOLOGO@Variant\HOLOGO@name}{%
    \@PackageError{hologo}{%
      Unknown logo `\HOLOGO@name'%
    }\@ehc
    ?\HOLOGO@name?%
  }{%
    \ltx@IfUndefined{ver@tex4ht.sty}{%
      \HoLogoFont@font\HOLOGO@name{general}{%
        \csname HoLogo@\HOLOGO@Variant\HOLOGO@name\endcsname
        \ltx@firstoftwo
      }%
    }{%
      \ltx@IfUndefined{HoLogoHtml@\HOLOGO@Variant\HOLOGO@name}{%
        \HOLOGO@name
      }{%
        \csname HoLogoHtml@\HOLOGO@Variant\HOLOGO@name\endcsname
        \ltx@firstoftwo
      }%
    }%
  }%
}
%    \end{macrocode}
%    \end{macro}
%    \begin{macro}{\HologoRobust}
%    \begin{macrocode}
\ltx@IfUndefined{protected}{%
  \ltx@IfUndefined{DeclareRobustCommand}{%
    \def\HologoRobust#1%
  }{%
    \DeclareRobustCommand*\HologoRobust[1]%
  }%
}{%
  \protected\def\HologoRobust#1%
}%
{%
  \edef\HOLOGO@name{#1}%
  \ltx@IfUndefined{HoLogo@\HOLOGO@Variant\HOLOGO@name}{%
    \@PackageError{hologo}{%
      Unknown logo `\HOLOGO@name'%
    }\@ehc
    ?\HOLOGO@name?%
  }{%
    \ltx@IfUndefined{ver@tex4ht.sty}{%
      \HoLogoFont@font\HOLOGO@name{general}{%
        \csname HoLogo@\HOLOGO@Variant\HOLOGO@name\endcsname
        \ltx@secondoftwo
      }%
    }{%
      \ltx@IfUndefined{HoLogoHtml@\HOLOGO@Variant\HOLOGO@name}{%
        \expandafter\HOLOGO@Uppercase\HOLOGO@name
      }{%
        \csname HoLogoHtml@\HOLOGO@Variant\HOLOGO@name\endcsname
        \ltx@secondoftwo
      }%
    }%
  }%
}
%    \end{macrocode}
%    \end{macro}
%    \begin{macro}{\hologoVariantRobust}
%    \begin{macrocode}
\ltx@IfUndefined{protected}{%
  \ltx@IfUndefined{DeclareRobustCommand}{%
    \def\hologoVariantRobust#1#2%
  }{%
    \DeclareRobustCommand*\hologoVariantRobust[2]%
  }%
}{%
  \protected\def\hologoVariantRobust#1#2%
}%
{%
  \begingroup
    \hologoLogoSetup{#1}{variant={#2}}%
    \hologoRobust{#1}%
  \endgroup
}
%    \end{macrocode}
%    \end{macro}
%    \begin{macro}{\HologoVariantRobust}
%    \begin{macrocode}
\ltx@IfUndefined{protected}{%
  \ltx@IfUndefined{DeclareRobustCommand}{%
    \def\HologoVariantRobust#1#2%
  }{%
    \DeclareRobustCommand*\HologoVariantRobust[2]%
  }%
}{%
  \protected\def\HologoVariantRobust#1#2%
}%
{%
  \begingroup
    \hologoLogoSetup{#1}{variant={#2}}%
    \HologoRobust{#1}%
  \endgroup
}
%    \end{macrocode}
%    \end{macro}
%
%    \begin{macro}{\hologorobust}
%    Macro \cs{hologorobust} is only defined for compatibility.
%    Its use is deprecated.
%    \begin{macrocode}
\def\hologorobust{\hologoRobust}
%    \end{macrocode}
%    \end{macro}
%
% \subsection{Helpers}
%
%    \begin{macro}{\HOLOGO@Uppercase}
%    Macro \cs{HOLOGO@Uppercase} is restricted to \cs{uppercase},
%    because \hologo{plainTeX} or \hologo{iniTeX} do not provide
%    \cs{MakeUppercase}.
%    \begin{macrocode}
\def\HOLOGO@Uppercase#1{\uppercase{#1}}
%    \end{macrocode}
%    \end{macro}
%
%    \begin{macro}{\HOLOGO@PdfdocUnicode}
%    \begin{macrocode}
\def\HOLOGO@PdfdocUnicode{%
  \ifx\ifHy@unicode\iftrue
    \expandafter\ltx@secondoftwo
  \else
    \expandafter\ltx@firstoftwo
  \fi
}
%    \end{macrocode}
%    \end{macro}
%
%    \begin{macro}{\HOLOGO@Math}
%    \begin{macrocode}
\def\HOLOGO@MathSetup{%
  \mathsurround0pt\relax
  \HOLOGO@IfExists\f@series{%
    \if b\expandafter\ltx@car\f@series x\@nil
      \csname boldmath\endcsname
   \fi
  }{}%
}
%    \end{macrocode}
%    \end{macro}
%
%    \begin{macro}{\HOLOGO@TempDimen}
%    \begin{macrocode}
\dimendef\HOLOGO@TempDimen=\ltx@zero
%    \end{macrocode}
%    \end{macro}
%    \begin{macro}{\HOLOGO@NegativeKerning}
%    \begin{macrocode}
\def\HOLOGO@NegativeKerning#1{%
  \begingroup
    \HOLOGO@TempDimen=0pt\relax
    \comma@parse@normalized{#1}{%
      \ifdim\HOLOGO@TempDimen=0pt %
        \expandafter\HOLOGO@@NegativeKerning\comma@entry
      \fi
      \ltx@gobble
    }%
    \ifdim\HOLOGO@TempDimen<0pt %
      \kern\HOLOGO@TempDimen
    \fi
  \endgroup
}
%    \end{macrocode}
%    \end{macro}
%    \begin{macro}{\HOLOGO@@NegativeKerning}
%    \begin{macrocode}
\def\HOLOGO@@NegativeKerning#1#2{%
  \setbox\ltx@zero\hbox{#1#2}%
  \HOLOGO@TempDimen=\wd\ltx@zero
  \setbox\ltx@zero\hbox{#1\kern0pt#2}%
  \advance\HOLOGO@TempDimen by -\wd\ltx@zero
}
%    \end{macrocode}
%    \end{macro}
%
%    \begin{macro}{\HOLOGO@SpaceFactor}
%    \begin{macrocode}
\def\HOLOGO@SpaceFactor{%
  \spacefactor1000 %
}
%    \end{macrocode}
%    \end{macro}
%
%    \begin{macro}{\HOLOGO@Span}
%    \begin{macrocode}
\def\HOLOGO@Span#1#2{%
  \HCode{<span class="HoLogo-#1">}%
  #2%
  \HCode{</span>}%
}
%    \end{macrocode}
%    \end{macro}
%
% \subsubsection{Text subscript}
%
%    \begin{macro}{\HOLOGO@SubScript}%
%    \begin{macrocode}
\def\HOLOGO@SubScript#1{%
  \ltx@IfUndefined{textsubscript}{%
    \ltx@IfUndefined{text}{%
      \ltx@mbox{%
        \mathsurround=0pt\relax
        $%
          _{%
            \ltx@IfUndefined{sf@size}{%
              \mathrm{#1}%
            }{%
              \mbox{%
                \fontsize\sf@size{0pt}\selectfont
                #1%
              }%
            }%
          }%
        $%
      }%
    }{%
      \ltx@mbox{%
        \mathsurround=0pt\relax
        $_{\text{#1}}$%
      }%
    }%
  }{%
    \textsubscript{#1}%
  }%
}
%    \end{macrocode}
%    \end{macro}
%
% \subsection{\hologo{TeX} and friends}
%
% \subsubsection{\hologo{TeX}}
%
%    \begin{macro}{\HoLogo@TeX}
%    Source: \hologo{LaTeX} kernel.
%    \begin{macrocode}
\def\HoLogo@TeX#1{%
  T\kern-.1667em\lower.5ex\hbox{E}\kern-.125emX\HOLOGO@SpaceFactor
}
%    \end{macrocode}
%    \end{macro}
%    \begin{macro}{\HoLogoHtml@TeX}
%    \begin{macrocode}
\def\HoLogoHtml@TeX#1{%
  \HoLogoCss@TeX
  \HOLOGO@Span{TeX}{%
    T%
    \HOLOGO@Span{e}{%
      E%
    }%
    X%
  }%
}
%    \end{macrocode}
%    \end{macro}
%    \begin{macro}{\HoLogoCss@TeX}
%    \begin{macrocode}
\def\HoLogoCss@TeX{%
  \Css{%
    span.HoLogo-TeX span.HoLogo-e{%
      position:relative;%
      top:.5ex;%
      margin-left:-.1667em;%
      margin-right:-.125em;%
    }%
  }%
  \Css{%
    a span.HoLogo-TeX span.HoLogo-e{%
      text-decoration:none;%
    }%
  }%
  \global\let\HoLogoCss@TeX\relax
}
%    \end{macrocode}
%    \end{macro}
%
% \subsubsection{\hologo{plainTeX}}
%
%    \begin{macro}{\HoLogo@plainTeX@space}
%    Source: ``The \hologo{TeX}book''
%    \begin{macrocode}
\def\HoLogo@plainTeX@space#1{%
  \HOLOGO@mbox{#1{p}{P}lain}\HOLOGO@space\hologo{TeX}%
}
%    \end{macrocode}
%    \end{macro}
%    \begin{macro}{\HoLogoCs@plainTeX@space}
%    \begin{macrocode}
\def\HoLogoCs@plainTeX@space#1{#1{p}{P}lain TeX}%
%    \end{macrocode}
%    \end{macro}
%    \begin{macro}{\HoLogoBkm@plainTeX@space}
%    \begin{macrocode}
\def\HoLogoBkm@plainTeX@space#1{%
  #1{p}{P}lain \hologo{TeX}%
}
%    \end{macrocode}
%    \end{macro}
%    \begin{macro}{\HoLogoHtml@plainTeX@space}
%    \begin{macrocode}
\def\HoLogoHtml@plainTeX@space#1{%
  #1{p}{P}lain \hologo{TeX}%
}
%    \end{macrocode}
%    \end{macro}
%
%    \begin{macro}{\HoLogo@plainTeX@hyphen}
%    \begin{macrocode}
\def\HoLogo@plainTeX@hyphen#1{%
  \HOLOGO@mbox{#1{p}{P}lain}\HOLOGO@hyphen\hologo{TeX}%
}
%    \end{macrocode}
%    \end{macro}
%    \begin{macro}{\HoLogoCs@plainTeX@hyphen}
%    \begin{macrocode}
\def\HoLogoCs@plainTeX@hyphen#1{#1{p}{P}lain-TeX}
%    \end{macrocode}
%    \end{macro}
%    \begin{macro}{\HoLogoBkm@plainTeX@hyphen}
%    \begin{macrocode}
\def\HoLogoBkm@plainTeX@hyphen#1{%
  #1{p}{P}lain-\hologo{TeX}%
}
%    \end{macrocode}
%    \end{macro}
%    \begin{macro}{\HoLogoHtml@plainTeX@hyphen}
%    \begin{macrocode}
\def\HoLogoHtml@plainTeX@hyphen#1{%
  #1{p}{P}lain-\hologo{TeX}%
}
%    \end{macrocode}
%    \end{macro}
%
%    \begin{macro}{\HoLogo@plainTeX@runtogether}
%    \begin{macrocode}
\def\HoLogo@plainTeX@runtogether#1{%
  \HOLOGO@mbox{#1{p}{P}lain\hologo{TeX}}%
}
%    \end{macrocode}
%    \end{macro}
%    \begin{macro}{\HoLogoCs@plainTeX@runtogether}
%    \begin{macrocode}
\def\HoLogoCs@plainTeX@runtogether#1{#1{p}{P}lainTeX}
%    \end{macrocode}
%    \end{macro}
%    \begin{macro}{\HoLogoBkm@plainTeX@runtogether}
%    \begin{macrocode}
\def\HoLogoBkm@plainTeX@runtogether#1{%
  #1{p}{P}lain\hologo{TeX}%
}
%    \end{macrocode}
%    \end{macro}
%    \begin{macro}{\HoLogoHtml@plainTeX@runtogether}
%    \begin{macrocode}
\def\HoLogoHtml@plainTeX@runtogether#1{%
  #1{p}{P}lain\hologo{TeX}%
}
%    \end{macrocode}
%    \end{macro}
%
%    \begin{macro}{\HoLogo@plainTeX}
%    \begin{macrocode}
\def\HoLogo@plainTeX{\HoLogo@plainTeX@space}
%    \end{macrocode}
%    \end{macro}
%    \begin{macro}{\HoLogoCs@plainTeX}
%    \begin{macrocode}
\def\HoLogoCs@plainTeX{\HoLogoCs@plainTeX@space}
%    \end{macrocode}
%    \end{macro}
%    \begin{macro}{\HoLogoBkm@plainTeX}
%    \begin{macrocode}
\def\HoLogoBkm@plainTeX{\HoLogoBkm@plainTeX@space}
%    \end{macrocode}
%    \end{macro}
%    \begin{macro}{\HoLogoHtml@plainTeX}
%    \begin{macrocode}
\def\HoLogoHtml@plainTeX{\HoLogoHtml@plainTeX@space}
%    \end{macrocode}
%    \end{macro}
%
% \subsubsection{\hologo{LaTeX}}
%
%    Source: \hologo{LaTeX} kernel.
%\begin{quote}
%\begin{verbatim}
%\DeclareRobustCommand{\LaTeX}{%
%  L%
%  \kern-.36em%
%  {%
%    \sbox\z@ T%
%    \vbox to\ht\z@{%
%      \hbox{%
%        \check@mathfonts
%        \fontsize\sf@size\z@
%        \math@fontsfalse
%        \selectfont
%        A%
%      }%
%      \vss
%    }%
%  }%
%  \kern-.15em%
%  \TeX
%}
%\end{verbatim}
%\end{quote}
%
%    \begin{macro}{\HoLogo@La}
%    \begin{macrocode}
\def\HoLogo@La#1{%
  L%
  \kern-.36em%
  \begingroup
    \setbox\ltx@zero\hbox{T}%
    \vbox to\ht\ltx@zero{%
      \hbox{%
        \ltx@ifundefined{check@mathfonts}{%
          \csname sevenrm\endcsname
        }{%
          \check@mathfonts
          \fontsize\sf@size{0pt}%
          \math@fontsfalse\selectfont
        }%
        A%
      }%
      \vss
    }%
  \endgroup
}
%    \end{macrocode}
%    \end{macro}
%
%    \begin{macro}{\HoLogo@LaTeX}
%    Source: \hologo{LaTeX} kernel.
%    \begin{macrocode}
\def\HoLogo@LaTeX#1{%
  \hologo{La}%
  \kern-.15em%
  \hologo{TeX}%
}
%    \end{macrocode}
%    \end{macro}
%    \begin{macro}{\HoLogoHtml@LaTeX}
%    \begin{macrocode}
\def\HoLogoHtml@LaTeX#1{%
  \HoLogoCss@LaTeX
  \HOLOGO@Span{LaTeX}{%
    L%
    \HOLOGO@Span{a}{%
      A%
    }%
    \hologo{TeX}%
  }%
}
%    \end{macrocode}
%    \end{macro}
%    \begin{macro}{\HoLogoCss@LaTeX}
%    \begin{macrocode}
\def\HoLogoCss@LaTeX{%
  \Css{%
    span.HoLogo-LaTeX span.HoLogo-a{%
      position:relative;%
      top:-.5ex;%
      margin-left:-.36em;%
      margin-right:-.15em;%
      font-size:85\%;%
    }%
  }%
  \global\let\HoLogoCss@LaTeX\relax
}
%    \end{macrocode}
%    \end{macro}
%
% \subsubsection{\hologo{(La)TeX}}
%
%    \begin{macro}{\HoLogo@LaTeXTeX}
%    The kerning around the parentheses is taken
%    from package \xpackage{dtklogos} \cite{dtklogos}.
%\begin{quote}
%\begin{verbatim}
%\DeclareRobustCommand{\LaTeXTeX}{%
%  (%
%  \kern-.15em%
%  L%
%  \kern-.36em%
%  {%
%    \sbox\z@ T%
%    \vbox to\ht0{%
%      \hbox{%
%        $\m@th$%
%        \csname S@\f@size\endcsname
%        \fontsize\sf@size\z@
%        \math@fontsfalse
%        \selectfont
%        A%
%      }%
%      \vss
%    }%
%  }%
%  \kern-.2em%
%  )%
%  \kern-.15em%
%  \TeX
%}
%\end{verbatim}
%\end{quote}
%    \begin{macrocode}
\def\HoLogo@LaTeXTeX#1{%
  (%
  \kern-.15em%
  \hologo{La}%
  \kern-.2em%
  )%
  \kern-.15em%
  \hologo{TeX}%
}
%    \end{macrocode}
%    \end{macro}
%    \begin{macro}{\HoLogoBkm@LaTeXTeX}
%    \begin{macrocode}
\def\HoLogoBkm@LaTeXTeX#1{(La)TeX}
%    \end{macrocode}
%    \end{macro}
%
%    \begin{macro}{\HoLogo@(La)TeX}
%    \begin{macrocode}
\expandafter
\let\csname HoLogo@(La)TeX\endcsname\HoLogo@LaTeXTeX
%    \end{macrocode}
%    \end{macro}
%    \begin{macro}{\HoLogoBkm@(La)TeX}
%    \begin{macrocode}
\expandafter
\let\csname HoLogoBkm@(La)TeX\endcsname\HoLogoBkm@LaTeXTeX
%    \end{macrocode}
%    \end{macro}
%    \begin{macro}{\HoLogoHtml@LaTeXTeX}
%    \begin{macrocode}
\def\HoLogoHtml@LaTeXTeX#1{%
  \HoLogoCss@LaTeXTeX
  \HOLOGO@Span{LaTeXTeX}{%
    (%
    \HOLOGO@Span{L}{L}%
    \HOLOGO@Span{a}{A}%
    \HOLOGO@Span{ParenRight}{)}%
    \hologo{TeX}%
  }%
}
%    \end{macrocode}
%    \end{macro}
%    \begin{macro}{\HoLogoHtml@(La)TeX}
%    Kerning after opening parentheses and before closing parentheses
%    is $-0.1$\,em. The original values $-0.15$\,em
%    looked too ugly for a serif font.
%    \begin{macrocode}
\expandafter
\let\csname HoLogoHtml@(La)TeX\endcsname\HoLogoHtml@LaTeXTeX
%    \end{macrocode}
%    \end{macro}
%    \begin{macro}{\HoLogoCss@LaTeXTeX}
%    \begin{macrocode}
\def\HoLogoCss@LaTeXTeX{%
  \Css{%
    span.HoLogo-LaTeXTeX span.HoLogo-L{%
      margin-left:-.1em;%
    }%
  }%
  \Css{%
    span.HoLogo-LaTeXTeX span.HoLogo-a{%
      position:relative;%
      top:-.5ex;%
      margin-left:-.36em;%
      margin-right:-.1em;%
      font-size:85\%;%
    }%
  }%
  \Css{%
    span.HoLogo-LaTeXTeX span.HoLogo-ParenRight{%
      margin-right:-.15em;%
    }%
  }%
  \global\let\HoLogoCss@LaTeXTeX\relax
}
%    \end{macrocode}
%    \end{macro}
%
% \subsubsection{\hologo{LaTeXe}}
%
%    \begin{macro}{\HoLogo@LaTeXe}
%    Source: \hologo{LaTeX} kernel
%    \begin{macrocode}
\def\HoLogo@LaTeXe#1{%
  \hologo{LaTeX}%
  \kern.15em%
  \hbox{%
    \HOLOGO@MathSetup
    2%
    $_{\textstyle\varepsilon}$%
  }%
}
%    \end{macrocode}
%    \end{macro}
%
%    \begin{macro}{\HoLogoCs@LaTeXe}
%    \begin{macrocode}
\ifnum64=`\^^^^0040\relax % test for big chars of LuaTeX/XeTeX
  \catcode`\$=9 %
  \catcode`\&=14 %
\else
  \catcode`\$=14 %
  \catcode`\&=9 %
\fi
\def\HoLogoCs@LaTeXe#1{%
  LaTeX2%
$ \string ^^^^0395%
& e%
}%
\catcode`\$=3 %
\catcode`\&=4 %
%    \end{macrocode}
%    \end{macro}
%
%    \begin{macro}{\HoLogoBkm@LaTeXe}
%    \begin{macrocode}
\def\HoLogoBkm@LaTeXe#1{%
  \hologo{LaTeX}%
  2%
  \HOLOGO@PdfdocUnicode{e}{\textepsilon}%
}
%    \end{macrocode}
%    \end{macro}
%
%    \begin{macro}{\HoLogoHtml@LaTeXe}
%    \begin{macrocode}
\def\HoLogoHtml@LaTeXe#1{%
  \HoLogoCss@LaTeXe
  \HOLOGO@Span{LaTeX2e}{%
    \hologo{LaTeX}%
    \HOLOGO@Span{2}{2}%
    \HOLOGO@Span{e}{%
      \HOLOGO@MathSetup
      \ensuremath{\textstyle\varepsilon}%
    }%
  }%
}
%    \end{macrocode}
%    \end{macro}
%    \begin{macro}{\HoLogoCss@LaTeXe}
%    \begin{macrocode}
\def\HoLogoCss@LaTeXe{%
  \Css{%
    span.HoLogo-LaTeX2e span.HoLogo-2{%
      padding-left:.15em;%
    }%
  }%
  \Css{%
    span.HoLogo-LaTeX2e span.HoLogo-e{%
      position:relative;%
      top:.35ex;%
      text-decoration:none;%
    }%
  }%
  \global\let\HoLogoCss@LaTeXe\relax
}
%    \end{macrocode}
%    \end{macro}
%
%    \begin{macro}{\HoLogo@LaTeX2e}
%    \begin{macrocode}
\expandafter
\let\csname HoLogo@LaTeX2e\endcsname\HoLogo@LaTeXe
%    \end{macrocode}
%    \end{macro}
%    \begin{macro}{\HoLogoCs@LaTeX2e}
%    \begin{macrocode}
\expandafter
\let\csname HoLogoCs@LaTeX2e\endcsname\HoLogoCs@LaTeXe
%    \end{macrocode}
%    \end{macro}
%    \begin{macro}{\HoLogoBkm@LaTeX2e}
%    \begin{macrocode}
\expandafter
\let\csname HoLogoBkm@LaTeX2e\endcsname\HoLogoBkm@LaTeXe
%    \end{macrocode}
%    \end{macro}
%    \begin{macro}{\HoLogoHtml@LaTeX2e}
%    \begin{macrocode}
\expandafter
\let\csname HoLogoHtml@LaTeX2e\endcsname\HoLogoHtml@LaTeXe
%    \end{macrocode}
%    \end{macro}
%
% \subsubsection{\hologo{LaTeX3}}
%
%    \begin{macro}{\HoLogo@LaTeX3}
%    Source: \hologo{LaTeX} kernel
%    \begin{macrocode}
\expandafter\def\csname HoLogo@LaTeX3\endcsname#1{%
  \hologo{LaTeX}%
  3%
}
%    \end{macrocode}
%    \end{macro}
%
%    \begin{macro}{\HoLogoBkm@LaTeX3}
%    \begin{macrocode}
\expandafter\def\csname HoLogoBkm@LaTeX3\endcsname#1{%
  \hologo{LaTeX}%
  3%
}
%    \end{macrocode}
%    \end{macro}
%    \begin{macro}{\HoLogoHtml@LaTeX3}
%    \begin{macrocode}
\expandafter
\let\csname HoLogoHtml@LaTeX3\expandafter\endcsname
\csname HoLogo@LaTeX3\endcsname
%    \end{macrocode}
%    \end{macro}
%
% \subsubsection{\hologo{LaTeXML}}
%
%    \begin{macro}{\HoLogo@LaTeXML}
%    \begin{macrocode}
\def\HoLogo@LaTeXML#1{%
  \HOLOGO@mbox{%
    \hologo{La}%
    \kern-.15em%
    T%
    \kern-.1667em%
    \lower.5ex\hbox{E}%
    \kern-.125em%
    \HoLogoFont@font{LaTeXML}{sc}{xml}%
  }%
}
%    \end{macrocode}
%    \end{macro}
%    \begin{macro}{\HoLogoHtml@pdfLaTeX}
%    \begin{macrocode}
\def\HoLogoHtml@LaTeXML#1{%
  \HOLOGO@Span{LaTeXML}{%
    \HoLogoCss@LaTeX
    \HoLogoCss@TeX
    \HOLOGO@Span{LaTeX}{%
      L%
      \HOLOGO@Span{a}{%
        A%
      }%
    }%
    \HOLOGO@Span{TeX}{%
      T%
      \HOLOGO@Span{e}{%
        E%
      }%
    }%
    \HCode{<span style="font-variant: small-caps;">}%
    xml%
    \HCode{</span>}%
  }%
}
%    \end{macrocode}
%    \end{macro}
%
% \subsubsection{\hologo{eTeX}}
%
%    \begin{macro}{\HoLogo@eTeX}
%    Source: package \xpackage{etex}
%    \begin{macrocode}
\def\HoLogo@eTeX#1{%
  \ltx@mbox{%
    \HOLOGO@MathSetup
    $\varepsilon$%
    -%
    \HOLOGO@NegativeKerning{-T,T-,To}%
    \hologo{TeX}%
  }%
}
%    \end{macrocode}
%    \end{macro}
%    \begin{macro}{\HoLogoCs@eTeX}
%    \begin{macrocode}
\ifnum64=`\^^^^0040\relax % test for big chars of LuaTeX/XeTeX
  \catcode`\$=9 %
  \catcode`\&=14 %
\else
  \catcode`\$=14 %
  \catcode`\&=9 %
\fi
\def\HoLogoCs@eTeX#1{%
$ #1{\string ^^^^0395}{\string ^^^^03b5}%
& #1{e}{E}%
  TeX%
}%
\catcode`\$=3 %
\catcode`\&=4 %
%    \end{macrocode}
%    \end{macro}
%    \begin{macro}{\HoLogoBkm@eTeX}
%    \begin{macrocode}
\def\HoLogoBkm@eTeX#1{%
  \HOLOGO@PdfdocUnicode{#1{e}{E}}{\textepsilon}%
  -%
  \hologo{TeX}%
}
%    \end{macrocode}
%    \end{macro}
%    \begin{macro}{\HoLogoHtml@eTeX}
%    \begin{macrocode}
\def\HoLogoHtml@eTeX#1{%
  \ltx@mbox{%
    \HOLOGO@MathSetup
    $\varepsilon$%
    -%
    \hologo{TeX}%
  }%
}
%    \end{macrocode}
%    \end{macro}
%
% \subsubsection{\hologo{iniTeX}}
%
%    \begin{macro}{\HoLogo@iniTeX}
%    \begin{macrocode}
\def\HoLogo@iniTeX#1{%
  \HOLOGO@mbox{%
    #1{i}{I}ni\hologo{TeX}%
  }%
}
%    \end{macrocode}
%    \end{macro}
%    \begin{macro}{\HoLogoCs@iniTeX}
%    \begin{macrocode}
\def\HoLogoCs@iniTeX#1{#1{i}{I}niTeX}
%    \end{macrocode}
%    \end{macro}
%    \begin{macro}{\HoLogoBkm@iniTeX}
%    \begin{macrocode}
\def\HoLogoBkm@iniTeX#1{%
  #1{i}{I}ni\hologo{TeX}%
}
%    \end{macrocode}
%    \end{macro}
%    \begin{macro}{\HoLogoHtml@iniTeX}
%    \begin{macrocode}
\let\HoLogoHtml@iniTeX\HoLogo@iniTeX
%    \end{macrocode}
%    \end{macro}
%
% \subsubsection{\hologo{virTeX}}
%
%    \begin{macro}{\HoLogo@virTeX}
%    \begin{macrocode}
\def\HoLogo@virTeX#1{%
  \HOLOGO@mbox{%
    #1{v}{V}ir\hologo{TeX}%
  }%
}
%    \end{macrocode}
%    \end{macro}
%    \begin{macro}{\HoLogoCs@virTeX}
%    \begin{macrocode}
\def\HoLogoCs@virTeX#1{#1{v}{V}irTeX}
%    \end{macrocode}
%    \end{macro}
%    \begin{macro}{\HoLogoBkm@virTeX}
%    \begin{macrocode}
\def\HoLogoBkm@virTeX#1{%
  #1{v}{V}ir\hologo{TeX}%
}
%    \end{macrocode}
%    \end{macro}
%    \begin{macro}{\HoLogoHtml@virTeX}
%    \begin{macrocode}
\let\HoLogoHtml@virTeX\HoLogo@virTeX
%    \end{macrocode}
%    \end{macro}
%
% \subsubsection{\hologo{SliTeX}}
%
% \paragraph{Definitions of the three variants.}
%
%    \begin{macro}{\HoLogo@SLiTeX@lift}
%    \begin{macrocode}
\def\HoLogo@SLiTeX@lift#1{%
  \HoLogoFont@font{SliTeX}{rm}{%
    S%
    \kern-.06em%
    L%
    \kern-.18em%
    \raise.32ex\hbox{\HoLogoFont@font{SliTeX}{sc}{i}}%
    \HOLOGO@discretionary
    \kern-.06em%
    \hologo{TeX}%
  }%
}
%    \end{macrocode}
%    \end{macro}
%    \begin{macro}{\HoLogoBkm@SLiTeX@lift}
%    \begin{macrocode}
\def\HoLogoBkm@SLiTeX@lift#1{SLiTeX}
%    \end{macrocode}
%    \end{macro}
%    \begin{macro}{\HoLogoHtml@SLiTeX@lift}
%    \begin{macrocode}
\def\HoLogoHtml@SLiTeX@lift#1{%
  \HoLogoCss@SLiTeX@lift
  \HOLOGO@Span{SLiTeX-lift}{%
    \HoLogoFont@font{SliTeX}{rm}{%
      S%
      \HOLOGO@Span{L}{L}%
      \HOLOGO@Span{i}{i}%
      \hologo{TeX}%
    }%
  }%
}
%    \end{macrocode}
%    \end{macro}
%    \begin{macro}{\HoLogoCss@SLiTeX@lift}
%    \begin{macrocode}
\def\HoLogoCss@SLiTeX@lift{%
  \Css{%
    span.HoLogo-SLiTeX-lift span.HoLogo-L{%
      margin-left:-.06em;%
      margin-right:-.18em;%
    }%
  }%
  \Css{%
    span.HoLogo-SLiTeX-lift span.HoLogo-i{%
      position:relative;%
      top:-.32ex;%
      margin-right:-.06em;%
      font-variant:small-caps;%
    }%
  }%
  \global\let\HoLogoCss@SLiTeX@lift\relax
}
%    \end{macrocode}
%    \end{macro}
%
%    \begin{macro}{\HoLogo@SliTeX@simple}
%    \begin{macrocode}
\def\HoLogo@SliTeX@simple#1{%
  \HoLogoFont@font{SliTeX}{rm}{%
    \ltx@mbox{%
      \HoLogoFont@font{SliTeX}{sc}{Sli}%
    }%
    \HOLOGO@discretionary
    \hologo{TeX}%
  }%
}
%    \end{macrocode}
%    \end{macro}
%    \begin{macro}{\HoLogoBkm@SliTeX@simple}
%    \begin{macrocode}
\def\HoLogoBkm@SliTeX@simple#1{SliTeX}
%    \end{macrocode}
%    \end{macro}
%    \begin{macro}{\HoLogoHtml@SliTeX@simple}
%    \begin{macrocode}
\let\HoLogoHtml@SliTeX@simple\HoLogo@SliTeX@simple
%    \end{macrocode}
%    \end{macro}
%
%    \begin{macro}{\HoLogo@SliTeX@narrow}
%    \begin{macrocode}
\def\HoLogo@SliTeX@narrow#1{%
  \HoLogoFont@font{SliTeX}{rm}{%
    \ltx@mbox{%
      S%
      \kern-.06em%
      \HoLogoFont@font{SliTeX}{sc}{%
        l%
        \kern-.035em%
        i%
      }%
    }%
    \HOLOGO@discretionary
    \kern-.06em%
    \hologo{TeX}%
  }%
}
%    \end{macrocode}
%    \end{macro}
%    \begin{macro}{\HoLogoBkm@SliTeX@narrow}
%    \begin{macrocode}
\def\HoLogoBkm@SliTeX@narrow#1{SliTeX}
%    \end{macrocode}
%    \end{macro}
%    \begin{macro}{\HoLogoHtml@SliTeX@narrow}
%    \begin{macrocode}
\def\HoLogoHtml@SliTeX@narrow#1{%
  \HoLogoCss@SliTeX@narrow
  \HOLOGO@Span{SliTeX-narrow}{%
    \HoLogoFont@font{SliTeX}{rm}{%
      S%
        \HOLOGO@Span{l}{l}%
        \HOLOGO@Span{i}{i}%
      \hologo{TeX}%
    }%
  }%
}
%    \end{macrocode}
%    \end{macro}
%    \begin{macro}{\HoLogoCss@SliTeX@narrow}
%    \begin{macrocode}
\def\HoLogoCss@SliTeX@narrow{%
  \Css{%
    span.HoLogo-SliTeX-narrow span.HoLogo-l{%
      margin-left:-.06em;%
      margin-right:-.035em;%
      font-variant:small-caps;%
    }%
  }%
  \Css{%
    span.HoLogo-SliTeX-narrow span.HoLogo-i{%
      margin-right:-.06em;%
      font-variant:small-caps;%
    }%
  }%
  \global\let\HoLogoCss@SliTeX@narrow\relax
}
%    \end{macrocode}
%    \end{macro}
%
% \paragraph{Macro set completion.}
%
%    \begin{macro}{\HoLogo@SLiTeX@simple}
%    \begin{macrocode}
\def\HoLogo@SLiTeX@simple{\HoLogo@SliTeX@simple}
%    \end{macrocode}
%    \end{macro}
%    \begin{macro}{\HoLogoBkm@SLiTeX@simple}
%    \begin{macrocode}
\def\HoLogoBkm@SLiTeX@simple{\HoLogoBkm@SliTeX@simple}
%    \end{macrocode}
%    \end{macro}
%    \begin{macro}{\HoLogoHtml@SLiTeX@simple}
%    \begin{macrocode}
\def\HoLogoHtml@SLiTeX@simple{\HoLogoHtml@SliTeX@simple}
%    \end{macrocode}
%    \end{macro}
%
%    \begin{macro}{\HoLogo@SLiTeX@narrow}
%    \begin{macrocode}
\def\HoLogo@SLiTeX@narrow{\HoLogo@SliTeX@narrow}
%    \end{macrocode}
%    \end{macro}
%    \begin{macro}{\HoLogoBkm@SLiTeX@narrow}
%    \begin{macrocode}
\def\HoLogoBkm@SLiTeX@narrow{\HoLogoBkm@SliTeX@narrow}
%    \end{macrocode}
%    \end{macro}
%    \begin{macro}{\HoLogoHtml@SLiTeX@narrow}
%    \begin{macrocode}
\def\HoLogoHtml@SLiTeX@narrow{\HoLogoHtml@SliTeX@narrow}
%    \end{macrocode}
%    \end{macro}
%
%    \begin{macro}{\HoLogo@SliTeX@lift}
%    \begin{macrocode}
\def\HoLogo@SliTeX@lift{\HoLogo@SLiTeX@lift}
%    \end{macrocode}
%    \end{macro}
%    \begin{macro}{\HoLogoBkm@SliTeX@lift}
%    \begin{macrocode}
\def\HoLogoBkm@SliTeX@lift{\HoLogoBkm@SLiTeX@lift}
%    \end{macrocode}
%    \end{macro}
%    \begin{macro}{\HoLogoHtml@SliTeX@lift}
%    \begin{macrocode}
\def\HoLogoHtml@SliTeX@lift{\HoLogoHtml@SLiTeX@lift}
%    \end{macrocode}
%    \end{macro}
%
% \paragraph{Defaults.}
%
%    \begin{macro}{\HoLogo@SLiTeX}
%    \begin{macrocode}
\def\HoLogo@SLiTeX{\HoLogo@SLiTeX@lift}
%    \end{macrocode}
%    \end{macro}
%    \begin{macro}{\HoLogoBkm@SLiTeX}
%    \begin{macrocode}
\def\HoLogoBkm@SLiTeX{\HoLogoBkm@SLiTeX@lift}
%    \end{macrocode}
%    \end{macro}
%    \begin{macro}{\HoLogoHtml@SLiTeX}
%    \begin{macrocode}
\def\HoLogoHtml@SLiTeX{\HoLogoHtml@SLiTeX@lift}
%    \end{macrocode}
%    \end{macro}
%
%    \begin{macro}{\HoLogo@SliTeX}
%    \begin{macrocode}
\def\HoLogo@SliTeX{\HoLogo@SliTeX@narrow}
%    \end{macrocode}
%    \end{macro}
%    \begin{macro}{\HoLogoBkm@SliTeX}
%    \begin{macrocode}
\def\HoLogoBkm@SliTeX{\HoLogoBkm@SliTeX@narrow}
%    \end{macrocode}
%    \end{macro}
%    \begin{macro}{\HoLogoHtml@SliTeX}
%    \begin{macrocode}
\def\HoLogoHtml@SliTeX{\HoLogoHtml@SliTeX@narrow}
%    \end{macrocode}
%    \end{macro}
%
% \subsubsection{\hologo{LuaTeX}}
%
%    \begin{macro}{\HoLogo@LuaTeX}
%    The kerning is an idea of Hans Hagen, see mailing list
%    `luatex at tug dot org' in March 2010.
%    \begin{macrocode}
\def\HoLogo@LuaTeX#1{%
  \HOLOGO@mbox{%
    Lua%
    \HOLOGO@NegativeKerning{aT,oT,To}%
    \hologo{TeX}%
  }%
}
%    \end{macrocode}
%    \end{macro}
%    \begin{macro}{\HoLogoHtml@LuaTeX}
%    \begin{macrocode}
\let\HoLogoHtml@LuaTeX\HoLogo@LuaTeX
%    \end{macrocode}
%    \end{macro}
%
% \subsubsection{\hologo{LuaLaTeX}}
%
%    \begin{macro}{\HoLogo@LuaLaTeX}
%    \begin{macrocode}
\def\HoLogo@LuaLaTeX#1{%
  \HOLOGO@mbox{%
    Lua%
    \hologo{LaTeX}%
  }%
}
%    \end{macrocode}
%    \end{macro}
%    \begin{macro}{\HoLogoHtml@LuaLaTeX}
%    \begin{macrocode}
\let\HoLogoHtml@LuaLaTeX\HoLogo@LuaLaTeX
%    \end{macrocode}
%    \end{macro}
%
% \subsubsection{\hologo{XeTeX}, \hologo{XeLaTeX}}
%
%    \begin{macro}{\HOLOGO@IfCharExists}
%    \begin{macrocode}
\ifluatex
  \ifnum\luatexversion<36 %
  \else
    \def\HOLOGO@IfCharExists#1{%
      \ifnum
        \directlua{%
           if luaotfload and luaotfload.aux then
             if luaotfload.aux.font_has_glyph(%
                    font.current(), \number#1) then % 	 
	       tex.print("1") % 	 
	     end % 	 
	   elseif font and font.fonts and font.current then %
            local f = font.fonts[font.current()]%
            if f.characters and f.characters[\number#1] then %
              tex.print("1")%
            end %
          end%
        }0=\ltx@zero
        \expandafter\ltx@secondoftwo
      \else
        \expandafter\ltx@firstoftwo
      \fi
    }%
  \fi
\fi
\ltx@IfUndefined{HOLOGO@IfCharExists}{%
  \def\HOLOGO@@IfCharExists#1{%
    \begingroup
      \tracinglostchars=\ltx@zero
      \setbox\ltx@zero=\hbox{%
        \kern7sp\char#1\relax
        \ifnum\lastkern>\ltx@zero
          \expandafter\aftergroup\csname iffalse\endcsname
        \else
          \expandafter\aftergroup\csname iftrue\endcsname
        \fi
      }%
      % \if{true|false} from \aftergroup
      \endgroup
      \expandafter\ltx@firstoftwo
    \else
      \endgroup
      \expandafter\ltx@secondoftwo
    \fi
  }%
  \ifxetex
    \ltx@IfUndefined{XeTeXfonttype}{}{%
      \ltx@IfUndefined{XeTeXcharglyph}{}{%
        \def\HOLOGO@IfCharExists#1{%
          \ifnum\XeTeXfonttype\font>\ltx@zero
            \expandafter\ltx@firstofthree
          \else
            \expandafter\ltx@gobble
          \fi
          {%
            \ifnum\XeTeXcharglyph#1>\ltx@zero
              \expandafter\ltx@firstoftwo
            \else
              \expandafter\ltx@secondoftwo
            \fi
          }%
          \HOLOGO@@IfCharExists{#1}%
        }%
      }%
    }%
  \fi
}{}
\ltx@ifundefined{HOLOGO@IfCharExists}{%
  \ifnum64=`\^^^^0040\relax % test for big chars of LuaTeX/XeTeX
    \let\HOLOGO@IfCharExists\HOLOGO@@IfCharExists
  \else
    \def\HOLOGO@IfCharExists#1{%
      \ifnum#1>255 %
        \expandafter\ltx@fourthoffour
      \fi
      \HOLOGO@@IfCharExists{#1}%
    }%
  \fi
}{}
%    \end{macrocode}
%    \end{macro}
%
%    \begin{macro}{\HoLogo@Xe}
%    Source: package \xpackage{dtklogos}
%    \begin{macrocode}
\def\HoLogo@Xe#1{%
  X%
  \kern-.1em\relax
  \HOLOGO@IfCharExists{"018E}{%
    \lower.5ex\hbox{\char"018E}%
  }{%
    \chardef\HOLOGO@choice=\ltx@zero
    \ifdim\fontdimen\ltx@one\font>0pt %
      \ltx@IfUndefined{rotatebox}{%
        \ltx@IfUndefined{pgftext}{%
          \ltx@IfUndefined{psscalebox}{%
            \ltx@IfUndefined{HOLOGO@ScaleBox@\hologoDriver}{%
            }{%
              \chardef\HOLOGO@choice=4 %
            }%
          }{%
            \chardef\HOLOGO@choice=3 %
          }%
        }{%
          \chardef\HOLOGO@choice=2 %
        }%
      }{%
        \chardef\HOLOGO@choice=1 %
      }%
      \ifcase\HOLOGO@choice
        \HOLOGO@WarningUnsupportedDriver{Xe}%
        e%
      \or % 1: \rotatebox
        \begingroup
          \setbox\ltx@zero\hbox{\rotatebox{180}{E}}%
          \ltx@LocDimenA=\dp\ltx@zero
          \advance\ltx@LocDimenA by -.5ex\relax
          \raise\ltx@LocDimenA\box\ltx@zero
        \endgroup
      \or % 2: \pgftext
        \lower.5ex\hbox{%
          \pgfpicture
            \pgftext[rotate=180]{E}%
          \endpgfpicture
        }%
      \or % 3: \psscalebox
        \begingroup
          \setbox\ltx@zero\hbox{\psscalebox{-1 -1}{E}}%
          \ltx@LocDimenA=\dp\ltx@zero
          \advance\ltx@LocDimenA by -.5ex\relax
          \raise\ltx@LocDimenA\box\ltx@zero
        \endgroup
      \or % 4: \HOLOGO@PointReflectBox
        \lower.5ex\hbox{\HOLOGO@PointReflectBox{E}}%
      \else
        \@PackageError{hologo}{Internal error (choice/it}\@ehc
      \fi
    \else
      \ltx@IfUndefined{reflectbox}{%
        \ltx@IfUndefined{pgftext}{%
          \ltx@IfUndefined{psscalebox}{%
            \ltx@IfUndefined{HOLOGO@ScaleBox@\hologoDriver}{%
            }{%
              \chardef\HOLOGO@choice=4 %
            }%
          }{%
            \chardef\HOLOGO@choice=3 %
          }%
        }{%
          \chardef\HOLOGO@choice=2 %
        }%
      }{%
        \chardef\HOLOGO@choice=1 %
      }%
      \ifcase\HOLOGO@choice
        \HOLOGO@WarningUnsupportedDriver{Xe}%
        e%
      \or % 1: reflectbox
        \lower.5ex\hbox{%
          \reflectbox{E}%
        }%
      \or % 2: \pgftext
        \lower.5ex\hbox{%
          \pgfpicture
            \pgftransformxscale{-1}%
            \pgftext{E}%
          \endpgfpicture
        }%
      \or % 3: \psscalebox
        \lower.5ex\hbox{%
          \psscalebox{-1 1}{E}%
        }%
      \or % 4: \HOLOGO@Reflectbox
        \lower.5ex\hbox{%
          \HOLOGO@ReflectBox{E}%
        }%
      \else
        \@PackageError{hologo}{Internal error (choice/up)}\@ehc
      \fi
    \fi
  }%
}
%    \end{macrocode}
%    \end{macro}
%    \begin{macro}{\HoLogoHtml@Xe}
%    \begin{macrocode}
\def\HoLogoHtml@Xe#1{%
  \HoLogoCss@Xe
  \HOLOGO@Span{Xe}{%
    X%
    \HOLOGO@Span{e}{%
      \HCode{&\ltx@hashchar x018e;}%
    }%
  }%
}
%    \end{macrocode}
%    \end{macro}
%    \begin{macro}{\HoLogoCss@Xe}
%    \begin{macrocode}
\def\HoLogoCss@Xe{%
  \Css{%
    span.HoLogo-Xe span.HoLogo-e{%
      position:relative;%
      top:.5ex;%
      left-margin:-.1em;%
    }%
  }%
  \global\let\HoLogoCss@Xe\relax
}
%    \end{macrocode}
%    \end{macro}
%
%    \begin{macro}{\HoLogo@XeTeX}
%    \begin{macrocode}
\def\HoLogo@XeTeX#1{%
  \hologo{Xe}%
  \kern-.15em\relax
  \hologo{TeX}%
}
%    \end{macrocode}
%    \end{macro}
%
%    \begin{macro}{\HoLogoHtml@XeTeX}
%    \begin{macrocode}
\def\HoLogoHtml@XeTeX#1{%
  \HoLogoCss@XeTeX
  \HOLOGO@Span{XeTeX}{%
    \hologo{Xe}%
    \hologo{TeX}%
  }%
}
%    \end{macrocode}
%    \end{macro}
%    \begin{macro}{\HoLogoCss@XeTeX}
%    \begin{macrocode}
\def\HoLogoCss@XeTeX{%
  \Css{%
    span.HoLogo-XeTeX span.HoLogo-TeX{%
      margin-left:-.15em;%
    }%
  }%
  \global\let\HoLogoCss@XeTeX\relax
}
%    \end{macrocode}
%    \end{macro}
%
%    \begin{macro}{\HoLogo@XeLaTeX}
%    \begin{macrocode}
\def\HoLogo@XeLaTeX#1{%
  \hologo{Xe}%
  \kern-.13em%
  \hologo{LaTeX}%
}
%    \end{macrocode}
%    \end{macro}
%    \begin{macro}{\HoLogoHtml@XeLaTeX}
%    \begin{macrocode}
\def\HoLogoHtml@XeLaTeX#1{%
  \HoLogoCss@XeLaTeX
  \HOLOGO@Span{XeLaTeX}{%
    \hologo{Xe}%
    \hologo{LaTeX}%
  }%
}
%    \end{macrocode}
%    \end{macro}
%    \begin{macro}{\HoLogoCss@XeLaTeX}
%    \begin{macrocode}
\def\HoLogoCss@XeLaTeX{%
  \Css{%
    span.HoLogo-XeLaTeX span.HoLogo-Xe{%
      margin-right:-.13em;%
    }%
  }%
  \global\let\HoLogoCss@XeLaTeX\relax
}
%    \end{macrocode}
%    \end{macro}
%
% \subsubsection{\hologo{pdfTeX}, \hologo{pdfLaTeX}}
%
%    \begin{macro}{\HoLogo@pdfTeX}
%    \begin{macrocode}
\def\HoLogo@pdfTeX#1{%
  \HOLOGO@mbox{%
    #1{p}{P}df\hologo{TeX}%
  }%
}
%    \end{macrocode}
%    \end{macro}
%    \begin{macro}{\HoLogoCs@pdfTeX}
%    \begin{macrocode}
\def\HoLogoCs@pdfTeX#1{#1{p}{P}dfTeX}
%    \end{macrocode}
%    \end{macro}
%    \begin{macro}{\HoLogoBkm@pdfTeX}
%    \begin{macrocode}
\def\HoLogoBkm@pdfTeX#1{%
  #1{p}{P}df\hologo{TeX}%
}
%    \end{macrocode}
%    \end{macro}
%    \begin{macro}{\HoLogoHtml@pdfTeX}
%    \begin{macrocode}
\let\HoLogoHtml@pdfTeX\HoLogo@pdfTeX
%    \end{macrocode}
%    \end{macro}
%
%    \begin{macro}{\HoLogo@pdfLaTeX}
%    \begin{macrocode}
\def\HoLogo@pdfLaTeX#1{%
  \HOLOGO@mbox{%
    #1{p}{P}df\hologo{LaTeX}%
  }%
}
%    \end{macrocode}
%    \end{macro}
%    \begin{macro}{\HoLogoCs@pdfLaTeX}
%    \begin{macrocode}
\def\HoLogoCs@pdfLaTeX#1{#1{p}{P}dfLaTeX}
%    \end{macrocode}
%    \end{macro}
%    \begin{macro}{\HoLogoBkm@pdfLaTeX}
%    \begin{macrocode}
\def\HoLogoBkm@pdfLaTeX#1{%
  #1{p}{P}df\hologo{LaTeX}%
}
%    \end{macrocode}
%    \end{macro}
%    \begin{macro}{\HoLogoHtml@pdfLaTeX}
%    \begin{macrocode}
\let\HoLogoHtml@pdfLaTeX\HoLogo@pdfLaTeX
%    \end{macrocode}
%    \end{macro}
%
% \subsubsection{\hologo{VTeX}}
%
%    \begin{macro}{\HoLogo@VTeX}
%    \begin{macrocode}
\def\HoLogo@VTeX#1{%
  \HOLOGO@mbox{%
    V\hologo{TeX}%
  }%
}
%    \end{macrocode}
%    \end{macro}
%    \begin{macro}{\HoLogoHtml@VTeX}
%    \begin{macrocode}
\let\HoLogoHtml@VTeX\HoLogo@VTeX
%    \end{macrocode}
%    \end{macro}
%
% \subsubsection{\hologo{AmS}, \dots}
%
%    Source: class \xclass{amsdtx}
%
%    \begin{macro}{\HoLogo@AmS}
%    \begin{macrocode}
\def\HoLogo@AmS#1{%
  \HoLogoFont@font{AmS}{sy}{%
    A%
    \kern-.1667em%
    \lower.5ex\hbox{M}%
    \kern-.125em%
    S%
  }%
}
%    \end{macrocode}
%    \end{macro}
%    \begin{macro}{\HoLogoBkm@AmS}
%    \begin{macrocode}
\def\HoLogoBkm@AmS#1{AmS}
%    \end{macrocode}
%    \end{macro}
%    \begin{macro}{\HoLogoHtml@AmS}
%    \begin{macrocode}
\def\HoLogoHtml@AmS#1{%
  \HoLogoCss@AmS
%  \HoLogoFont@font{AmS}{sy}{%
    \HOLOGO@Span{AmS}{%
      A%
      \HOLOGO@Span{M}{M}%
      S%
    }%
%   }%
}
%    \end{macrocode}
%    \end{macro}
%    \begin{macro}{\HoLogoCss@AmS}
%    \begin{macrocode}
\def\HoLogoCss@AmS{%
  \Css{%
    span.HoLogo-AmS span.HoLogo-M{%
      position:relative;%
      top:.5ex;%
      margin-left:-.1667em;%
      margin-right:-.125em;%
      text-decoration:none;%
    }%
  }%
  \global\let\HoLogoCss@AmS\relax
}
%    \end{macrocode}
%    \end{macro}
%
%    \begin{macro}{\HoLogo@AmSTeX}
%    \begin{macrocode}
\def\HoLogo@AmSTeX#1{%
  \hologo{AmS}%
  \HOLOGO@hyphen
  \hologo{TeX}%
}
%    \end{macrocode}
%    \end{macro}
%    \begin{macro}{\HoLogoBkm@AmSTeX}
%    \begin{macrocode}
\def\HoLogoBkm@AmSTeX#1{AmS-TeX}%
%    \end{macrocode}
%    \end{macro}
%    \begin{macro}{\HoLogoHtml@AmSTeX}
%    \begin{macrocode}
\let\HoLogoHtml@AmSTeX\HoLogo@AmSTeX
%    \end{macrocode}
%    \end{macro}
%
%    \begin{macro}{\HoLogo@AmSLaTeX}
%    \begin{macrocode}
\def\HoLogo@AmSLaTeX#1{%
  \hologo{AmS}%
  \HOLOGO@hyphen
  \hologo{LaTeX}%
}
%    \end{macrocode}
%    \end{macro}
%    \begin{macro}{\HoLogoBkm@AmSLaTeX}
%    \begin{macrocode}
\def\HoLogoBkm@AmSLaTeX#1{AmS-LaTeX}%
%    \end{macrocode}
%    \end{macro}
%    \begin{macro}{\HoLogoHtml@AmSLaTeX}
%    \begin{macrocode}
\let\HoLogoHtml@AmSLaTeX\HoLogo@AmSLaTeX
%    \end{macrocode}
%    \end{macro}
%
% \subsubsection{\hologo{BibTeX}}
%
%    \begin{macro}{\HoLogo@BibTeX@sc}
%    A definition of \hologo{BibTeX} is provided in
%    the documentation source for the manual of \hologo{BibTeX}
%    \cite{btxdoc}.
%\begin{quote}
%\begin{verbatim}
%\def\BibTeX{%
%  {%
%    \rm
%    B%
%    \kern-.05em%
%    {%
%      \sc
%      i%
%      \kern-.025em %
%      b%
%    }%
%    \kern-.08em
%    T%
%    \kern-.1667em%
%    \lower.7ex\hbox{E}%
%    \kern-.125em%
%    X%
%  }%
%}
%\end{verbatim}
%\end{quote}
%    \begin{macrocode}
\def\HoLogo@BibTeX@sc#1{%
  B%
  \kern-.05em%
  \HoLogoFont@font{BibTeX}{sc}{%
    i%
    \kern-.025em%
    b%
  }%
  \HOLOGO@discretionary
  \kern-.08em%
  \hologo{TeX}%
}
%    \end{macrocode}
%    \end{macro}
%    \begin{macro}{\HoLogoHtml@BibTeX@sc}
%    \begin{macrocode}
\def\HoLogoHtml@BibTeX@sc#1{%
  \HoLogoCss@BibTeX@sc
  \HOLOGO@Span{BibTeX-sc}{%
    B%
    \HOLOGO@Span{i}{i}%
    \HOLOGO@Span{b}{b}%
    \hologo{TeX}%
  }%
}
%    \end{macrocode}
%    \end{macro}
%    \begin{macro}{\HoLogoCss@BibTeX@sc}
%    \begin{macrocode}
\def\HoLogoCss@BibTeX@sc{%
  \Css{%
    span.HoLogo-BibTeX-sc span.HoLogo-i{%
      margin-left:-.05em;%
      margin-right:-.025em;%
      font-variant:small-caps;%
    }%
  }%
  \Css{%
    span.HoLogo-BibTeX-sc span.HoLogo-b{%
      margin-right:-.08em;%
      font-variant:small-caps;%
    }%
  }%
  \global\let\HoLogoCss@BibTeX@sc\relax
}
%    \end{macrocode}
%    \end{macro}
%
%    \begin{macro}{\HoLogo@BibTeX@sf}
%    Variant \xoption{sf} avoids trouble with unavailable
%    small caps fonts (e.g., bold versions of Computer Modern or
%    Latin Modern). The definition is taken from
%    package \xpackage{dtklogos} \cite{dtklogos}.
%\begin{quote}
%\begin{verbatim}
%\DeclareRobustCommand{\BibTeX}{%
%  B%
%  \kern-.05em%
%  \hbox{%
%    $\m@th$% %% force math size calculations
%    \csname S@\f@size\endcsname
%    \fontsize\sf@size\z@
%    \math@fontsfalse
%    \selectfont
%    I%
%    \kern-.025em%
%    B
%  }%
%  \kern-.08em%
%  \-%
%  \TeX
%}
%\end{verbatim}
%\end{quote}
%    \begin{macrocode}
\def\HoLogo@BibTeX@sf#1{%
  B%
  \kern-.05em%
  \HoLogoFont@font{BibTeX}{bibsf}{%
    I%
    \kern-.025em%
    B%
  }%
  \HOLOGO@discretionary
  \kern-.08em%
  \hologo{TeX}%
}
%    \end{macrocode}
%    \end{macro}
%    \begin{macro}{\HoLogoHtml@BibTeX@sf}
%    \begin{macrocode}
\def\HoLogoHtml@BibTeX@sf#1{%
  \HoLogoCss@BibTeX@sf
  \HOLOGO@Span{BibTeX-sf}{%
    B%
    \HoLogoFont@font{BibTeX}{bibsf}{%
      \HOLOGO@Span{i}{I}%
      B%
    }%
    \hologo{TeX}%
  }%
}
%    \end{macrocode}
%    \end{macro}
%    \begin{macro}{\HoLogoCss@BibTeX@sf}
%    \begin{macrocode}
\def\HoLogoCss@BibTeX@sf{%
  \Css{%
    span.HoLogo-BibTeX-sf span.HoLogo-i{%
      margin-left:-.05em;%
      margin-right:-.025em;%
    }%
  }%
  \Css{%
    span.HoLogo-BibTeX-sf span.HoLogo-TeX{%
      margin-left:-.08em;%
    }%
  }%
  \global\let\HoLogoCss@BibTeX@sf\relax
}
%    \end{macrocode}
%    \end{macro}
%
%    \begin{macro}{\HoLogo@BibTeX}
%    \begin{macrocode}
\def\HoLogo@BibTeX{\HoLogo@BibTeX@sf}
%    \end{macrocode}
%    \end{macro}
%    \begin{macro}{\HoLogoHtml@BibTeX}
%    \begin{macrocode}
\def\HoLogoHtml@BibTeX{\HoLogoHtml@BibTeX@sf}
%    \end{macrocode}
%    \end{macro}
%
% \subsubsection{\hologo{BibTeX8}}
%
%    \begin{macro}{\HoLogo@BibTeX8}
%    \begin{macrocode}
\expandafter\def\csname HoLogo@BibTeX8\endcsname#1{%
  \hologo{BibTeX}%
  8%
}
%    \end{macrocode}
%    \end{macro}
%
%    \begin{macro}{\HoLogoBkm@BibTeX8}
%    \begin{macrocode}
\expandafter\def\csname HoLogoBkm@BibTeX8\endcsname#1{%
  \hologo{BibTeX}%
  8%
}
%    \end{macrocode}
%    \end{macro}
%    \begin{macro}{\HoLogoHtml@BibTeX8}
%    \begin{macrocode}
\expandafter
\let\csname HoLogoHtml@BibTeX8\expandafter\endcsname
\csname HoLogo@BibTeX8\endcsname
%    \end{macrocode}
%    \end{macro}
%
% \subsubsection{\hologo{ConTeXt}}
%
%    \begin{macro}{\HoLogo@ConTeXt@simple}
%    \begin{macrocode}
\def\HoLogo@ConTeXt@simple#1{%
  \HOLOGO@mbox{Con}%
  \HOLOGO@discretionary
  \HOLOGO@mbox{\hologo{TeX}t}%
}
%    \end{macrocode}
%    \end{macro}
%    \begin{macro}{\HoLogoHtml@ConTeXt@simple}
%    \begin{macrocode}
\let\HoLogoHtml@ConTeXt@simple\HoLogo@ConTeXt@simple
%    \end{macrocode}
%    \end{macro}
%
%    \begin{macro}{\HoLogo@ConTeXt@narrow}
%    This definition of logo \hologo{ConTeXt} with variant \xoption{narrow}
%    comes from TUGboat's class \xclass{ltugboat} (version 2010/11/15 v2.8).
%    \begin{macrocode}
\def\HoLogo@ConTeXt@narrow#1{%
  \HOLOGO@mbox{C\kern-.0333emon}%
  \HOLOGO@discretionary
  \kern-.0667em%
  \HOLOGO@mbox{\hologo{TeX}\kern-.0333emt}%
}
%    \end{macrocode}
%    \end{macro}
%    \begin{macro}{\HoLogoHtml@ConTeXt@narrow}
%    \begin{macrocode}
\def\HoLogoHtml@ConTeXt@narrow#1{%
  \HoLogoCss@ConTeXt@narrow
  \HOLOGO@Span{ConTeXt-narrow}{%
    \HOLOGO@Span{C}{C}%
    on%
    \hologo{TeX}%
    t%
  }%
}
%    \end{macrocode}
%    \end{macro}
%    \begin{macro}{\HoLogoCss@ConTeXt@narrow}
%    \begin{macrocode}
\def\HoLogoCss@ConTeXt@narrow{%
  \Css{%
    span.HoLogo-ConTeXt-narrow span.HoLogo-C{%
      margin-left:-.0333em;%
    }%
  }%
  \Css{%
    span.HoLogo-ConTeXt-narrow span.HoLogo-TeX{%
      margin-left:-.0667em;%
      margin-right:-.0333em;%
    }%
  }%
  \global\let\HoLogoCss@ConTeXt@narrow\relax
}
%    \end{macrocode}
%    \end{macro}
%
%    \begin{macro}{\HoLogo@ConTeXt}
%    \begin{macrocode}
\def\HoLogo@ConTeXt{\HoLogo@ConTeXt@narrow}
%    \end{macrocode}
%    \end{macro}
%    \begin{macro}{\HoLogoHtml@ConTeXt}
%    \begin{macrocode}
\def\HoLogoHtml@ConTeXt{\HoLogoHtml@ConTeXt@narrow}
%    \end{macrocode}
%    \end{macro}
%
% \subsubsection{\hologo{emTeX}}
%
%    \begin{macro}{\HoLogo@emTeX}
%    \begin{macrocode}
\def\HoLogo@emTeX#1{%
  \HOLOGO@mbox{#1{e}{E}m}%
  \HOLOGO@discretionary
  \hologo{TeX}%
}
%    \end{macrocode}
%    \end{macro}
%    \begin{macro}{\HoLogoCs@emTeX}
%    \begin{macrocode}
\def\HoLogoCs@emTeX#1{#1{e}{E}mTeX}%
%    \end{macrocode}
%    \end{macro}
%    \begin{macro}{\HoLogoBkm@emTeX}
%    \begin{macrocode}
\def\HoLogoBkm@emTeX#1{%
  #1{e}{E}m\hologo{TeX}%
}
%    \end{macrocode}
%    \end{macro}
%    \begin{macro}{\HoLogoHtml@emTeX}
%    \begin{macrocode}
\let\HoLogoHtml@emTeX\HoLogo@emTeX
%    \end{macrocode}
%    \end{macro}
%
% \subsubsection{\hologo{ExTeX}}
%
%    \begin{macro}{\HoLogo@ExTeX}
%    The definition is taken from the FAQ of the
%    project \hologo{ExTeX}
%    \cite{ExTeX-FAQ}.
%\begin{quote}
%\begin{verbatim}
%\def\ExTeX{%
%  \textrm{% Logo always with serifs
%    \ensuremath{%
%      \textstyle
%      \varepsilon_{%
%        \kern-0.15em%
%        \mathcal{X}%
%      }%
%    }%
%    \kern-.15em%
%    \TeX
%  }%
%}
%\end{verbatim}
%\end{quote}
%    \begin{macrocode}
\def\HoLogo@ExTeX#1{%
  \HoLogoFont@font{ExTeX}{rm}{%
    \ltx@mbox{%
      \HOLOGO@MathSetup
      $%
        \textstyle
        \varepsilon_{%
          \kern-0.15em%
          \HoLogoFont@font{ExTeX}{sy}{X}%
        }%
      $%
    }%
    \HOLOGO@discretionary
    \kern-.15em%
    \hologo{TeX}%
  }%
}
%    \end{macrocode}
%    \end{macro}
%    \begin{macro}{\HoLogoHtml@ExTeX}
%    \begin{macrocode}
\def\HoLogoHtml@ExTeX#1{%
  \HoLogoCss@ExTeX
  \HoLogoFont@font{ExTeX}{rm}{%
    \HOLOGO@Span{ExTeX}{%
      \ltx@mbox{%
        \HOLOGO@MathSetup
        $\textstyle\varepsilon$%
        \HOLOGO@Span{X}{$\textstyle\chi$}%
        \hologo{TeX}%
      }%
    }%
  }%
}
%    \end{macrocode}
%    \end{macro}
%    \begin{macro}{\HoLogoBkm@ExTeX}
%    \begin{macrocode}
\def\HoLogoBkm@ExTeX#1{%
  \HOLOGO@PdfdocUnicode{#1{e}{E}x}{\textepsilon\textchi}%
  \hologo{TeX}%
}
%    \end{macrocode}
%    \end{macro}
%    \begin{macro}{\HoLogoCss@ExTeX}
%    \begin{macrocode}
\def\HoLogoCss@ExTeX{%
  \Css{%
    span.HoLogo-ExTeX{%
      font-family:serif;%
    }%
  }%
  \Css{%
    span.HoLogo-ExTeX span.HoLogo-TeX{%
      margin-left:-.15em;%
    }%
  }%
  \global\let\HoLogoCss@ExTeX\relax
}
%    \end{macrocode}
%    \end{macro}
%
% \subsubsection{\hologo{MiKTeX}}
%
%    \begin{macro}{\HoLogo@MiKTeX}
%    \begin{macrocode}
\def\HoLogo@MiKTeX#1{%
  \HOLOGO@mbox{MiK}%
  \HOLOGO@discretionary
  \hologo{TeX}%
}
%    \end{macrocode}
%    \end{macro}
%    \begin{macro}{\HoLogoHtml@MiKTeX}
%    \begin{macrocode}
\let\HoLogoHtml@MiKTeX\HoLogo@MiKTeX
%    \end{macrocode}
%    \end{macro}
%
% \subsubsection{\hologo{OzTeX} and friends}
%
%    Source: \hologo{OzTeX} FAQ \cite{OzTeX}:
%    \begin{quote}
%      |\def\OzTeX{O\kern-.03em z\kern-.15em\TeX}|\\
%      (There is no kerning in OzMF, OzMP and OzTtH.)
%    \end{quote}
%
%    \begin{macro}{\HoLogo@OzTeX}
%    \begin{macrocode}
\def\HoLogo@OzTeX#1{%
  O%
  \kern-.03em %
  z%
  \kern-.15em %
  \hologo{TeX}%
}
%    \end{macrocode}
%    \end{macro}
%    \begin{macro}{\HoLogoHtml@OzTeX}
%    \begin{macrocode}
\def\HoLogoHtml@OzTeX#1{%
  \HoLogoCss@OzTeX
  \HOLOGO@Span{OzTeX}{%
    O%
    \HOLOGO@Span{z}{z}%
    \hologo{TeX}%
  }%
}
%    \end{macrocode}
%    \end{macro}
%    \begin{macro}{\HoLogoCss@OzTeX}
%    \begin{macrocode}
\def\HoLogoCss@OzTeX{%
  \Css{%
    span.HoLogo-OzTeX span.HoLogo-z{%
      margin-left:-.03em;%
      margin-right:-.15em;%
    }%
  }%
  \global\let\HoLogoCss@OzTeX\relax
}
%    \end{macrocode}
%    \end{macro}
%
%    \begin{macro}{\HoLogo@OzMF}
%    \begin{macrocode}
\def\HoLogo@OzMF#1{%
  \HOLOGO@mbox{OzMF}%
}
%    \end{macrocode}
%    \end{macro}
%    \begin{macro}{\HoLogo@OzMP}
%    \begin{macrocode}
\def\HoLogo@OzMP#1{%
  \HOLOGO@mbox{OzMP}%
}
%    \end{macrocode}
%    \end{macro}
%    \begin{macro}{\HoLogo@OzTtH}
%    \begin{macrocode}
\def\HoLogo@OzTtH#1{%
  \HOLOGO@mbox{OzTtH}%
}
%    \end{macrocode}
%    \end{macro}
%
% \subsubsection{\hologo{PCTeX}}
%
%    \begin{macro}{\HoLogo@PCTeX}
%    \begin{macrocode}
\def\HoLogo@PCTeX#1{%
  \HOLOGO@mbox{PC}%
  \hologo{TeX}%
}
%    \end{macrocode}
%    \end{macro}
%    \begin{macro}{\HoLogoHtml@PCTeX}
%    \begin{macrocode}
\let\HoLogoHtml@PCTeX\HoLogo@PCTeX
%    \end{macrocode}
%    \end{macro}
%
% \subsubsection{\hologo{PiCTeX}}
%
%    The original definitions from \xfile{pictex.tex} \cite{PiCTeX}:
%\begin{quote}
%\begin{verbatim}
%\def\PiC{%
%  P%
%  \kern-.12em%
%  \lower.5ex\hbox{I}%
%  \kern-.075em%
%  C%
%}
%\def\PiCTeX{%
%  \PiC
%  \kern-.11em%
%  \TeX
%}
%\end{verbatim}
%\end{quote}
%
%    \begin{macro}{\HoLogo@PiC}
%    \begin{macrocode}
\def\HoLogo@PiC#1{%
  P%
  \kern-.12em%
  \lower.5ex\hbox{I}%
  \kern-.075em%
  C%
  \HOLOGO@SpaceFactor
}
%    \end{macrocode}
%    \end{macro}
%    \begin{macro}{\HoLogoHtml@PiC}
%    \begin{macrocode}
\def\HoLogoHtml@PiC#1{%
  \HoLogoCss@PiC
  \HOLOGO@Span{PiC}{%
    P%
    \HOLOGO@Span{i}{I}%
    C%
  }%
}
%    \end{macrocode}
%    \end{macro}
%    \begin{macro}{\HoLogoCss@PiC}
%    \begin{macrocode}
\def\HoLogoCss@PiC{%
  \Css{%
    span.HoLogo-PiC span.HoLogo-i{%
      position:relative;%
      top:.5ex;%
      margin-left:-.12em;%
      margin-right:-.075em;%
      text-decoration:none;%
    }%
  }%
  \global\let\HoLogoCss@PiC\relax
}
%    \end{macrocode}
%    \end{macro}
%
%    \begin{macro}{\HoLogo@PiCTeX}
%    \begin{macrocode}
\def\HoLogo@PiCTeX#1{%
  \hologo{PiC}%
  \HOLOGO@discretionary
  \kern-.11em%
  \hologo{TeX}%
}
%    \end{macrocode}
%    \end{macro}
%    \begin{macro}{\HoLogoHtml@PiCTeX}
%    \begin{macrocode}
\def\HoLogoHtml@PiCTeX#1{%
  \HoLogoCss@PiCTeX
  \HOLOGO@Span{PiCTeX}{%
    \hologo{PiC}%
    \hologo{TeX}%
  }%
}
%    \end{macrocode}
%    \end{macro}
%    \begin{macro}{\HoLogoCss@PiCTeX}
%    \begin{macrocode}
\def\HoLogoCss@PiCTeX{%
  \Css{%
    span.HoLogo-PiCTeX span.HoLogo-PiC{%
      margin-right:-.11em;%
    }%
  }%
  \global\let\HoLogoCss@PiCTeX\relax
}
%    \end{macrocode}
%    \end{macro}
%
% \subsubsection{\hologo{teTeX}}
%
%    \begin{macro}{\HoLogo@teTeX}
%    \begin{macrocode}
\def\HoLogo@teTeX#1{%
  \HOLOGO@mbox{#1{t}{T}e}%
  \HOLOGO@discretionary
  \hologo{TeX}%
}
%    \end{macrocode}
%    \end{macro}
%    \begin{macro}{\HoLogoCs@teTeX}
%    \begin{macrocode}
\def\HoLogoCs@teTeX#1{#1{t}{T}dfTeX}
%    \end{macrocode}
%    \end{macro}
%    \begin{macro}{\HoLogoBkm@teTeX}
%    \begin{macrocode}
\def\HoLogoBkm@teTeX#1{%
  #1{t}{T}e\hologo{TeX}%
}
%    \end{macrocode}
%    \end{macro}
%    \begin{macro}{\HoLogoHtml@teTeX}
%    \begin{macrocode}
\let\HoLogoHtml@teTeX\HoLogo@teTeX
%    \end{macrocode}
%    \end{macro}
%
% \subsubsection{\hologo{TeX4ht}}
%
%    \begin{macro}{\HoLogo@TeX4ht}
%    \begin{macrocode}
\expandafter\def\csname HoLogo@TeX4ht\endcsname#1{%
  \HOLOGO@mbox{\hologo{TeX}4ht}%
}
%    \end{macrocode}
%    \end{macro}
%    \begin{macro}{\HoLogoHtml@TeX4ht}
%    \begin{macrocode}
\expandafter
\let\csname HoLogoHtml@TeX4ht\expandafter\endcsname
\csname HoLogo@TeX4ht\endcsname
%    \end{macrocode}
%    \end{macro}
%
%
% \subsubsection{\hologo{SageTeX}}
%
%    \begin{macro}{\HoLogo@SageTeX}
%    \begin{macrocode}
\def\HoLogo@SageTeX#1{%
  \HOLOGO@mbox{Sage}%
  \HOLOGO@discretionary
  \HOLOGO@NegativeKerning{eT,oT,To}%
  \hologo{TeX}%
}
%    \end{macrocode}
%    \end{macro}
%    \begin{macro}{\HoLogoHtml@SageTeX}
%    \begin{macrocode}
\let\HoLogoHtml@SageTeX\HoLogo@SageTeX
%    \end{macrocode}
%    \end{macro}
%
% \subsection{\hologo{METAFONT} and friends}
%
%    \begin{macro}{\HoLogo@METAFONT}
%    \begin{macrocode}
\def\HoLogo@METAFONT#1{%
  \HoLogoFont@font{METAFONT}{logo}{%
    \HOLOGO@mbox{META}%
    \HOLOGO@discretionary
    \HOLOGO@mbox{FONT}%
  }%
}
%    \end{macrocode}
%    \end{macro}
%
%    \begin{macro}{\HoLogo@METAPOST}
%    \begin{macrocode}
\def\HoLogo@METAPOST#1{%
  \HoLogoFont@font{METAPOST}{logo}{%
    \HOLOGO@mbox{META}%
    \HOLOGO@discretionary
    \HOLOGO@mbox{POST}%
  }%
}
%    \end{macrocode}
%    \end{macro}
%
%    \begin{macro}{\HoLogo@MetaFun}
%    \begin{macrocode}
\def\HoLogo@MetaFun#1{%
  \HOLOGO@mbox{Meta}%
  \HOLOGO@discretionary
  \HOLOGO@mbox{Fun}%
}
%    \end{macrocode}
%    \end{macro}
%
%    \begin{macro}{\HoLogo@MetaPost}
%    \begin{macrocode}
\def\HoLogo@MetaPost#1{%
  \HOLOGO@mbox{Meta}%
  \HOLOGO@discretionary
  \HOLOGO@mbox{Post}%
}
%    \end{macrocode}
%    \end{macro}
%
% \subsection{Others}
%
% \subsubsection{\hologo{biber}}
%
%    \begin{macro}{\HoLogo@biber}
%    \begin{macrocode}
\def\HoLogo@biber#1{%
  \HOLOGO@mbox{#1{b}{B}i}%
  \HOLOGO@discretionary
  \HOLOGO@mbox{ber}%
}
%    \end{macrocode}
%    \end{macro}
%    \begin{macro}{\HoLogoCs@biber}
%    \begin{macrocode}
\def\HoLogoCs@biber#1{#1{b}{B}iber}
%    \end{macrocode}
%    \end{macro}
%    \begin{macro}{\HoLogoBkm@biber}
%    \begin{macrocode}
\def\HoLogoBkm@biber#1{%
  #1{b}{B}iber%
}
%    \end{macrocode}
%    \end{macro}
%    \begin{macro}{\HoLogoHtml@biber}
%    \begin{macrocode}
\let\HoLogoHtml@biber\HoLogo@biber
%    \end{macrocode}
%    \end{macro}
%
% \subsubsection{\hologo{KOMAScript}}
%
%    \begin{macro}{\HoLogo@KOMAScript}
%    The definition for \hologo{KOMAScript} is taken
%    from \hologo{KOMAScript} (\xfile{scrlogo.dtx}, reformatted) \cite{scrlogo}:
%\begin{quote}
%\begin{verbatim}
%\@ifundefined{KOMAScript}{%
%  \DeclareRobustCommand{\KOMAScript}{%
%    \textsf{%
%      K\kern.05em O\kern.05emM\kern.05em A%
%      \kern.1em-\kern.1em %
%      Script%
%    }%
%  }%
%}{}
%\end{verbatim}
%\end{quote}
%    \begin{macrocode}
\def\HoLogo@KOMAScript#1{%
  \HoLogoFont@font{KOMAScript}{sf}{%
    \HOLOGO@mbox{%
      K\kern.05em%
      O\kern.05em%
      M\kern.05em%
      A%
    }%
    \kern.1em%
    \HOLOGO@hyphen
    \kern.1em%
    \HOLOGO@mbox{Script}%
  }%
}
%    \end{macrocode}
%    \end{macro}
%    \begin{macro}{\HoLogoBkm@KOMAScript}
%    \begin{macrocode}
\def\HoLogoBkm@KOMAScript#1{%
  KOMA-Script%
}
%    \end{macrocode}
%    \end{macro}
%    \begin{macro}{\HoLogoHtml@KOMAScript}
%    \begin{macrocode}
\def\HoLogoHtml@KOMAScript#1{%
  \HoLogoCss@KOMAScript
  \HoLogoFont@font{KOMAScript}{sf}{%
    \HOLOGO@Span{KOMAScript}{%
      K%
      \HOLOGO@Span{O}{O}%
      M%
      \HOLOGO@Span{A}{A}%
      \HOLOGO@Span{hyphen}{-}%
      Script%
    }%
  }%
}
%    \end{macrocode}
%    \end{macro}
%    \begin{macro}{\HoLogoCss@KOMAScript}
%    \begin{macrocode}
\def\HoLogoCss@KOMAScript{%
  \Css{%
    span.HoLogo-KOMAScript{%
      font-family:sans-serif;%
    }%
  }%
  \Css{%
    span.HoLogo-KOMAScript span.HoLogo-O{%
      padding-left:.05em;%
      padding-right:.05em;%
    }%
  }%
  \Css{%
    span.HoLogo-KOMAScript span.HoLogo-A{%
      padding-left:.05em;%
    }%
  }%
  \Css{%
    span.HoLogo-KOMAScript span.HoLogo-hyphen{%
      padding-left:.1em;%
      padding-right:.1em;%
    }%
  }%
  \global\let\HoLogoCss@KOMAScript\relax
}
%    \end{macrocode}
%    \end{macro}
%
% \subsubsection{\hologo{LyX}}
%
%    \begin{macro}{\HoLogo@LyX}
%    The definition is taken from the documentation source files
%    of \hologo{LyX}, \xfile{Intro.lyx} \cite{LyX}:
%\begin{quote}
%\begin{verbatim}
%\def\LyX{%
%  \texorpdfstring{%
%    L\kern-.1667em\lower.25em\hbox{Y}\kern-.125emX\@%
%  }{%
%    LyX%
%  }%
%}
%\end{verbatim}
%\end{quote}
%    \begin{macrocode}
\def\HoLogo@LyX#1{%
  L%
  \kern-.1667em%
  \lower.25em\hbox{Y}%
  \kern-.125em%
  X%
  \HOLOGO@SpaceFactor
}
%    \end{macrocode}
%    \end{macro}
%    \begin{macro}{\HoLogoHtml@LyX}
%    \begin{macrocode}
\def\HoLogoHtml@LyX#1{%
  \HoLogoCss@LyX
  \HOLOGO@Span{LyX}{%
    L%
    \HOLOGO@Span{y}{Y}%
    X%
  }%
}
%    \end{macrocode}
%    \end{macro}
%    \begin{macro}{\HoLogoCss@LyX}
%    \begin{macrocode}
\def\HoLogoCss@LyX{%
  \Css{%
    span.HoLogo-LyX span.HoLogo-y{%
      position:relative;%
      top:.25em;%
      margin-left:-.1667em;%
      margin-right:-.125em;%
      text-decoration:none;%
    }%
  }%
  \global\let\HoLogoCss@LyX\relax
}
%    \end{macrocode}
%    \end{macro}
%
% \subsubsection{\hologo{NTS}}
%
%    \begin{macro}{\HoLogo@NTS}
%    Definition for \hologo{NTS} can be found in
%    package \xpackage{etex\textunderscore man} for the \hologo{eTeX} manual \cite{etexman}
%    and in package \xpackage{dtklogos} \cite{dtklogos}:
%\begin{quote}
%\begin{verbatim}
%\def\NTS{%
%  \leavevmode
%  \hbox{%
%    $%
%      \cal N%
%      \kern-0.35em%
%      \lower0.5ex\hbox{$\cal T$}%
%      \kern-0.2em%
%      S%
%    $%
%  }%
%}
%\end{verbatim}
%\end{quote}
%    \begin{macrocode}
\def\HoLogo@NTS#1{%
  \HoLogoFont@font{NTS}{sy}{%
    N\/%
    \kern-.35em%
    \lower.5ex\hbox{T\/}%
    \kern-.2em%
    S\/%
  }%
  \HOLOGO@SpaceFactor
}
%    \end{macrocode}
%    \end{macro}
%
% \subsubsection{\Hologo{TTH} (\hologo{TeX} to HTML translator)}
%
%    Source: \url{http://hutchinson.belmont.ma.us/tth/}
%    In the HTML source the second `T' is printed as subscript.
%\begin{quote}
%\begin{verbatim}
%T<sub>T</sub>H
%\end{verbatim}
%\end{quote}
%    \begin{macro}{\HoLogo@TTH}
%    \begin{macrocode}
\def\HoLogo@TTH#1{%
  \ltx@mbox{%
    T\HOLOGO@SubScript{T}H%
  }%
  \HOLOGO@SpaceFactor
}
%    \end{macrocode}
%    \end{macro}
%
%    \begin{macro}{\HoLogoHtml@TTH}
%    \begin{macrocode}
\def\HoLogoHtml@TTH#1{%
  T\HCode{<sub>}T\HCode{</sub>}H%
}
%    \end{macrocode}
%    \end{macro}
%
% \subsubsection{\Hologo{HanTheThanh}}
%
%    Partial source: Package \xpackage{dtklogos}.
%    The double accent is U+1EBF (latin small letter e with circumflex
%    and acute).
%    \begin{macro}{\HoLogo@HanTheThanh}
%    \begin{macrocode}
\def\HoLogo@HanTheThanh#1{%
  \ltx@mbox{H\`an}%
  \HOLOGO@space
  \ltx@mbox{%
    Th%
    \HOLOGO@IfCharExists{"1EBF}{%
      \char"1EBF\relax
    }{%
      \^e\hbox to 0pt{\hss\raise .5ex\hbox{\'{}}}%
    }%
  }%
  \HOLOGO@space
  \ltx@mbox{Th\`anh}%
}
%    \end{macrocode}
%    \end{macro}
%    \begin{macro}{\HoLogoBkm@HanTheThanh}
%    \begin{macrocode}
\def\HoLogoBkm@HanTheThanh#1{%
  H\`an %
  Th\HOLOGO@PdfdocUnicode{\^e}{\9036\277} %
  Th\`anh%
}
%    \end{macrocode}
%    \end{macro}
%    \begin{macro}{\HoLogoHtml@HanTheThanh}
%    \begin{macrocode}
\def\HoLogoHtml@HanTheThanh#1{%
  H\`an %
  Th\HCode{&\ltx@hashchar x1ebf;} %
  Th\`anh%
}
%    \end{macrocode}
%    \end{macro}
%
% \subsection{Driver detection}
%
%    \begin{macrocode}
\HOLOGO@IfExists\InputIfFileExists{%
  \InputIfFileExists{hologo.cfg}{}{}%
}{%
  \ltx@IfUndefined{pdf@filesize}{%
    \def\HOLOGO@InputIfExists{%
      \openin\HOLOGO@temp=hologo.cfg\relax
      \ifeof\HOLOGO@temp
        \closein\HOLOGO@temp
      \else
        \closein\HOLOGO@temp
        \begingroup
          \def\x{LaTeX2e}%
        \expandafter\endgroup
        \ifx\fmtname\x
          \input{hologo.cfg}%
        \else
          \input hologo.cfg\relax
        \fi
      \fi
    }%
    \ltx@IfUndefined{newread}{%
      \chardef\HOLOGO@temp=15 %
      \def\HOLOGO@CheckRead{%
        \ifeof\HOLOGO@temp
          \HOLOGO@InputIfExists
        \else
          \ifcase\HOLOGO@temp
            \@PackageWarningNoLine{hologo}{%
              Configuration file ignored, because\MessageBreak
              a free read register could not be found%
            }%
          \else
            \begingroup
              \count\ltx@cclv=\HOLOGO@temp
              \advance\ltx@cclv by \ltx@minusone
              \edef\x{\endgroup
                \chardef\noexpand\HOLOGO@temp=\the\count\ltx@cclv
                \relax
              }%
            \x
          \fi
        \fi
      }%
    }{%
      \csname newread\endcsname\HOLOGO@temp
      \HOLOGO@InputIfExists
    }%
  }{%
    \edef\HOLOGO@temp{\pdf@filesize{hologo.cfg}}%
    \ifx\HOLOGO@temp\ltx@empty
    \else
      \ifnum\HOLOGO@temp>0 %
        \begingroup
          \def\x{LaTeX2e}%
        \expandafter\endgroup
        \ifx\fmtname\x
          \input{hologo.cfg}%
        \else
          \input hologo.cfg\relax
        \fi
      \else
        \@PackageInfoNoLine{hologo}{%
          Empty configuration file `hologo.cfg' ignored%
        }%
      \fi
    \fi
  }%
}
%    \end{macrocode}
%
%    \begin{macrocode}
\def\HOLOGO@temp#1#2{%
  \kv@define@key{HoLogoDriver}{#1}[]{%
    \begingroup
      \def\HOLOGO@temp{##1}%
      \ltx@onelevel@sanitize\HOLOGO@temp
      \ifx\HOLOGO@temp\ltx@empty
      \else
        \@PackageError{hologo}{%
          Value (\HOLOGO@temp) not permitted for option `#1'%
        }%
        \@ehc
      \fi
    \endgroup
    \def\hologoDriver{#2}%
  }%
}%
\def\HOLOGO@@temp#1#2{%
  \ifx\kv@value\relax
    \HOLOGO@temp{#1}{#1}%
  \else
    \HOLOGO@temp{#1}{#2}%
  \fi
}%
\kv@parse@normalized{%
  pdftex,%
  luatex=pdftex,%
  dvipdfm,%
  dvipdfmx=dvipdfm,%
  dvips,%
  dvipsone=dvips,%
  xdvi=dvips,%
  xetex,%
  vtex,%
}\HOLOGO@@temp
%    \end{macrocode}
%
%    \begin{macrocode}
\kv@define@key{HoLogoDriver}{driverfallback}{%
  \def\HOLOGO@DriverFallback{#1}%
}
%    \end{macrocode}
%
%    \begin{macro}{\HOLOGO@DriverFallback}
%    \begin{macrocode}
\def\HOLOGO@DriverFallback{dvips}
%    \end{macrocode}
%    \end{macro}
%
%    \begin{macro}{\hologoDriverSetup}
%    \begin{macrocode}
\def\hologoDriverSetup{%
  \let\hologoDriver\ltx@undefined
  \HOLOGO@DriverSetup
}
%    \end{macrocode}
%    \end{macro}
%
%    \begin{macro}{\HOLOGO@DriverSetup}
%    \begin{macrocode}
\def\HOLOGO@DriverSetup#1{%
  \kvsetkeys{HoLogoDriver}{#1}%
  \HOLOGO@CheckDriver
  \ltx@ifundefined{hologoDriver}{%
    \begingroup
    \edef\x{\endgroup
      \noexpand\kvsetkeys{HoLogoDriver}{\HOLOGO@DriverFallback}%
    }\x
  }{}%
  \@PackageInfoNoLine{hologo}{Using driver `\hologoDriver'}%
}
%    \end{macrocode}
%    \end{macro}
%
%    \begin{macro}{\HOLOGO@CheckDriver}
%    \begin{macrocode}
\def\HOLOGO@CheckDriver{%
  \ifpdf
    \def\hologoDriver{pdftex}%
    \let\HOLOGO@pdfliteral\pdfliteral
    \ifluatex
      \ifx\pdfextension\@undefined\else
        \protected\def\pdfliteral{\pdfextension literal}%
        \let\HOLOGO@pdfliteral\pdfliteral
      \fi
      \ltx@IfUndefined{HOLOGO@pdfliteral}{%
        \ifnum\luatexversion<36 %
        \else
          \begingroup
            \let\HOLOGO@temp\endgroup
            \ifcase0%
                \directlua{%
                  if tex.enableprimitives then %
                    tex.enableprimitives('HOLOGO@', {'pdfliteral'})%
                  else %
                    tex.print('1')%
                  end%
                }%
                \ifx\HOLOGO@pdfliteral\@undefined 1\fi%
                \relax%
              \endgroup
              \let\HOLOGO@temp\relax
              \global\let\HOLOGO@pdfliteral\HOLOGO@pdfliteral
            \fi%
          \HOLOGO@temp
        \fi
      }{}%
    \fi
    \ltx@IfUndefined{HOLOGO@pdfliteral}{%
      \@PackageWarningNoLine{hologo}{%
        Cannot find \string\pdfliteral
      }%
    }{}%
  \else
    \ifxetex
      \def\hologoDriver{xetex}%
    \else
      \ifvtex
        \def\hologoDriver{vtex}%
      \fi
    \fi
  \fi
}
%    \end{macrocode}
%    \end{macro}
%
%    \begin{macro}{\HOLOGO@WarningUnsupportedDriver}
%    \begin{macrocode}
\def\HOLOGO@WarningUnsupportedDriver#1{%
  \@PackageWarningNoLine{hologo}{%
    Logo `#1' needs driver specific macros,\MessageBreak
    but driver `\hologoDriver' is not supported.\MessageBreak
    Use a different driver or\MessageBreak
    load package `graphics' or `pgf'%
  }%
}
%    \end{macrocode}
%    \end{macro}
%
% \subsubsection{Reflect box macros}
%
%    Skip driver part if not needed.
%    \begin{macrocode}
\ltx@IfUndefined{reflectbox}{}{%
  \ltx@IfUndefined{rotatebox}{}{%
    \HOLOGO@AtEnd
  }%
}
\ltx@IfUndefined{pgftext}{}{%
  \HOLOGO@AtEnd
}
\ltx@IfUndefined{psscalebox}{}{%
  \HOLOGO@AtEnd
}
%    \end{macrocode}
%
%    \begin{macrocode}
\def\HOLOGO@temp{LaTeX2e}
\ifx\fmtname\HOLOGO@temp
  \RequirePackage{kvoptions}[2011/06/30]%
  \ProcessKeyvalOptions{HoLogoDriver}%
\fi
\HOLOGO@DriverSetup{}
%    \end{macrocode}
%
%    \begin{macro}{\HOLOGO@ReflectBox}
%    \begin{macrocode}
\def\HOLOGO@ReflectBox#1{%
  \begingroup
    \setbox\ltx@zero\hbox{\begingroup#1\endgroup}%
    \setbox\ltx@two\hbox{%
      \kern\wd\ltx@zero
      \csname HOLOGO@ScaleBox@\hologoDriver\endcsname{-1}{1}{%
        \hbox to 0pt{\copy\ltx@zero\hss}%
      }%
    }%
    \wd\ltx@two=\wd\ltx@zero
    \box\ltx@two
  \endgroup
}
%    \end{macrocode}
%    \end{macro}
%
%    \begin{macro}{\HOLOGO@PointReflectBox}
%    \begin{macrocode}
\def\HOLOGO@PointReflectBox#1{%
  \begingroup
    \setbox\ltx@zero\hbox{\begingroup#1\endgroup}%
    \setbox\ltx@two\hbox{%
      \kern\wd\ltx@zero
      \raise\ht\ltx@zero\hbox{%
        \csname HOLOGO@ScaleBox@\hologoDriver\endcsname{-1}{-1}{%
          \hbox to 0pt{\copy\ltx@zero\hss}%
        }%
      }%
    }%
    \wd\ltx@two=\wd\ltx@zero
    \box\ltx@two
  \endgroup
}
%    \end{macrocode}
%    \end{macro}
%
%    We must define all variants because of dynamic driver setup.
%    \begin{macrocode}
\def\HOLOGO@temp#1#2{#2}
%    \end{macrocode}
%
%    \begin{macro}{\HOLOGO@ScaleBox@pdftex}
%    \begin{macrocode}
\HOLOGO@temp{pdftex}{%
  \def\HOLOGO@ScaleBox@pdftex#1#2#3{%
    \HOLOGO@pdfliteral{%
      q #1 0 0 #2 0 0 cm%
    }%
    #3%
    \HOLOGO@pdfliteral{%
      Q%
    }%
  }%
}
%    \end{macrocode}
%    \end{macro}
%    \begin{macro}{\HOLOGO@ScaleBox@dvips}
%    \begin{macrocode}
\HOLOGO@temp{dvips}{%
  \def\HOLOGO@ScaleBox@dvips#1#2#3{%
    \special{ps:%
      gsave %
      currentpoint %
      currentpoint translate %
      #1 #2 scale %
      neg exch neg exch translate%
    }%
    #3%
    \special{ps:%
      currentpoint %
      grestore %
      moveto%
    }%
  }%
}
%    \end{macrocode}
%    \end{macro}
%    \begin{macro}{\HOLOGO@ScaleBox@dvipdfm}
%    \begin{macrocode}
\HOLOGO@temp{dvipdfm}{%
  \let\HOLOGO@ScaleBox@dvipdfm\HOLOGO@ScaleBox@dvips
}
%    \end{macrocode}
%    \end{macro}
%    Since \hologo{XeTeX} v0.6.
%    \begin{macro}{\HOLOGO@ScaleBox@xetex}
%    \begin{macrocode}
\HOLOGO@temp{xetex}{%
  \def\HOLOGO@ScaleBox@xetex#1#2#3{%
    \special{x:gsave}%
    \special{x:scale #1 #2}%
    #3%
    \special{x:grestore}%
  }%
}
%    \end{macrocode}
%    \end{macro}
%    \begin{macro}{\HOLOGO@ScaleBox@vtex}
%    \begin{macrocode}
\HOLOGO@temp{vtex}{%
  \def\HOLOGO@ScaleBox@vtex#1#2#3{%
    \special{r(#1,0,0,#2,0,0}%
    #3%
    \special{r)}%
  }%
}
%    \end{macrocode}
%    \end{macro}
%
%    \begin{macrocode}
\HOLOGO@AtEnd%
%</package>
%    \end{macrocode}
%
% \section{Test}
%
% \subsection{Catcode checks for loading}
%
%    \begin{macrocode}
%<*test1>
%    \end{macrocode}
%    \begin{macrocode}
\catcode`\{=1 %
\catcode`\}=2 %
\catcode`\#=6 %
\catcode`\@=11 %
\expandafter\ifx\csname count@\endcsname\relax
  \countdef\count@=255 %
\fi
\expandafter\ifx\csname @gobble\endcsname\relax
  \long\def\@gobble#1{}%
\fi
\expandafter\ifx\csname @firstofone\endcsname\relax
  \long\def\@firstofone#1{#1}%
\fi
\expandafter\ifx\csname loop\endcsname\relax
  \expandafter\@firstofone
\else
  \expandafter\@gobble
\fi
{%
  \def\loop#1\repeat{%
    \def\body{#1}%
    \iterate
  }%
  \def\iterate{%
    \body
      \let\next\iterate
    \else
      \let\next\relax
    \fi
    \next
  }%
  \let\repeat=\fi
}%
\def\RestoreCatcodes{}
\count@=0 %
\loop
  \edef\RestoreCatcodes{%
    \RestoreCatcodes
    \catcode\the\count@=\the\catcode\count@\relax
  }%
\ifnum\count@<255 %
  \advance\count@ 1 %
\repeat

\def\RangeCatcodeInvalid#1#2{%
  \count@=#1\relax
  \loop
    \catcode\count@=15 %
  \ifnum\count@<#2\relax
    \advance\count@ 1 %
  \repeat
}
\def\RangeCatcodeCheck#1#2#3{%
  \count@=#1\relax
  \loop
    \ifnum#3=\catcode\count@
    \else
      \errmessage{%
        Character \the\count@\space
        with wrong catcode \the\catcode\count@\space
        instead of \number#3%
      }%
    \fi
  \ifnum\count@<#2\relax
    \advance\count@ 1 %
  \repeat
}
\def\space{ }
\expandafter\ifx\csname LoadCommand\endcsname\relax
  \def\LoadCommand{\input hologo.sty\relax}%
\fi
\def\Test{%
  \RangeCatcodeInvalid{0}{47}%
  \RangeCatcodeInvalid{58}{64}%
  \RangeCatcodeInvalid{91}{96}%
  \RangeCatcodeInvalid{123}{255}%
  \catcode`\@=12 %
  \catcode`\\=0 %
  \catcode`\%=14 %
  \LoadCommand
  \RangeCatcodeCheck{0}{36}{15}%
  \RangeCatcodeCheck{37}{37}{14}%
  \RangeCatcodeCheck{38}{47}{15}%
  \RangeCatcodeCheck{48}{57}{12}%
  \RangeCatcodeCheck{58}{63}{15}%
  \RangeCatcodeCheck{64}{64}{12}%
  \RangeCatcodeCheck{65}{90}{11}%
  \RangeCatcodeCheck{91}{91}{15}%
  \RangeCatcodeCheck{92}{92}{0}%
  \RangeCatcodeCheck{93}{96}{15}%
  \RangeCatcodeCheck{97}{122}{11}%
  \RangeCatcodeCheck{123}{255}{15}%
  \RestoreCatcodes
}
\Test
\csname @@end\endcsname
\end
%    \end{macrocode}
%    \begin{macrocode}
%</test1>
%    \end{macrocode}
%
% \subsection{Spacefactor}
%
%    The space factor must be 1000 after a logo. If it is greater 1000
%    then the following space is a space after a sentence closing point.
%    If the space factor is smaller 1000 then an immediate following
%    dot is interpreted as abbreviation, not sentence closing point.
%
%    \begin{macrocode}
%<*test-spacefactor>
\NeedsTeXFormat{LaTeX2e}
\documentclass{article}
\usepackage{hologo}[2016/05/12]
\usepackage{kvsetkeys}
\usepackage{qstest}
\IncludeTests{*}
\LogTests{log}{*}{*}
\begin{document}
\begin{qstest}{spacefactor}{spacefactor}
\newcommand*{\Test}[1]{%
  \sbox0{%
    \hologo{#1}%
    \Expect*{1000 (#1)}*{\the\spacefactor\space(#1)}%
  }%
}%
\makeatletter
\def\TestList{}
\def\hologoEntry#1#2#3{%
  \edef\TestList{%
    \ifx\TestList\@empty
    \else
      \TestList,%
    \fi
    #1%
    \ifx\\#2\\%
    \else
      ={variant=#2}%
    \fi
  }%
}
\hologoList
\expandafter\kv@parse@normalized\expandafter{%
  \TestList
}{%
  \begingroup
    \let\@logo=\kv@key
    \ifx\kv@value\relax
    \else
      \expandafter\hologoLogoSetup\expandafter\@logo\expandafter{%
        \kv@value
      }%
    \fi
    \Test\@logo
  \endgroup
  \@gobbletwo
}
\end{qstest}
\end{document}
%</test-spacefactor>
%    \end{macrocode}
%
% \subsection{Complete list}
%
%    \begin{macrocode}
%<*test-list>
\NeedsTeXFormat{LaTeX2e}
\documentclass[12pt,a4paper]{article}
\usepackage{hologo}[2016/05/12]
\usepackage[T1]{fontenc}
\usepackage{lmodern}
\usepackage{parskip}
\usepackage[unicode]{hyperref}[2011/09/28]
\usepackage{bookmark}[2011/09/19]
\bookmarksetup{%
  numbered,%
  open,%
  openlevel=2,%
}
\renewcommand*{\contentsname}{List of logos}
\begin{document}
\tableofcontents
\def\TestFont#1#2#3#4#5#6{%
  \begingroup
    \usefont{#3}{#4}{#5}{#6}%
    \HologoVariant{#1}{#2}/\hologoVariant{#1}{#2}%
    \quad
    \begingroup\scriptsize\hologoVariant{#1}{#2}\endgroup
    \quad
  \endgroup
  (#3/#4/#5/#6)%
  \par
}
\makeatletter
\def\hologoEntry#1#2#3{%
  \section{%
    \HologoVariant{#1}{#2}/\hologoVariant{#1}{#2} %
    {[#1\ifx\\#2\\\else\space(#2)\fi]}% hash-ok
  }% braces around [] because of bug in tex4ht
  \begingroup
    \hypersetup{unicode=false}%
    \bookmark[%
      dest=\@currentHref,%
      rellevel=1,%
      keeplevel,%
    ]{%
      \HologoVariant{#1}{#2}/\hologoVariant{#1}{#2} %
      (PDFDocEncoding)%
    }%
  \endgroup
  \TestFont{#1}{#2}{OT1}{cmr}{m}{n}%
  \TestFont{#1}{#2}{OT1}{cmss}{m}{n}%
  \TestFont{#1}{#2}{OT1}{cmr}{b}{n}%
  \TestFont{#1}{#2}{OT1}{cmr}{m}{it}%
  \TestFont{#1}{#2}{OT1}{cmtt}{m}{n}%
  \TestFont{#1}{#2}{T1}{lmr}{m}{n}%
  \TestFont{#1}{#2}{T1}{lmss}{m}{n}%
  \TestFont{#1}{#2}{T1}{lmr}{b}{n}%
  \TestFont{#1}{#2}{T1}{lmr}{m}{it}%
  \TestFont{#1}{#2}{T1}{lmtt}{m}{n}%
  \TestFont{#1}{#2}{T1}{lmvtt}{m}{n}%
  \TestFont{#1}{#2}{T1}{qtm}{m}{n}%
  \TestFont{#1}{#2}{T1}{qhv}{m}{n}%
  \TestFont{#1}{#2}{T1}{qtm}{b}{n}%
  \TestFont{#1}{#2}{T1}{qtm}{m}{it}%
  \TestFont{#1}{#2}{T1}{qcr}{m}{n}%
  \newpage
}
\makeatother
\hologoList
\end{document}
%</test-list>
%    \end{macrocode}
%
% \section{Installation}
%
% \subsection{Download}
%
% \paragraph{Package.} This package is available on
% CTAN\footnote{\url{ftp://ftp.ctan.org/tex-archive/}}:
% \begin{description}
% \item[\CTAN{macros/latex/contrib/oberdiek/hologo.dtx}] The source file.
% \item[\CTAN{macros/latex/contrib/oberdiek/hologo.pdf}] Documentation.
% \end{description}
%
%
% \paragraph{Bundle.} All the packages of the bundle `oberdiek'
% are also available in a TDS compliant ZIP archive. There
% the packages are already unpacked and the documentation files
% are generated. The files and directories obey the TDS standard.
% \begin{description}
% \item[\CTAN{install/macros/latex/contrib/oberdiek.tds.zip}]
% \end{description}
% \emph{TDS} refers to the standard ``A Directory Structure
% for \TeX\ Files'' (\CTAN{tds/tds.pdf}). Directories
% with \xfile{texmf} in their name are usually organized this way.
%
% \subsection{Bundle installation}
%
% \paragraph{Unpacking.} Unpack the \xfile{oberdiek.tds.zip} in the
% TDS tree (also known as \xfile{texmf} tree) of your choice.
% Example (linux):
% \begin{quote}
%   |unzip oberdiek.tds.zip -d ~/texmf|
% \end{quote}
%
% \paragraph{Script installation.}
% Check the directory \xfile{TDS:scripts/oberdiek/} for
% scripts that need further installation steps.
% Package \xpackage{attachfile2} comes with the Perl script
% \xfile{pdfatfi.pl} that should be installed in such a way
% that it can be called as \texttt{pdfatfi}.
% Example (linux):
% \begin{quote}
%   |chmod +x scripts/oberdiek/pdfatfi.pl|\\
%   |cp scripts/oberdiek/pdfatfi.pl /usr/local/bin/|
% \end{quote}
%
% \subsection{Package installation}
%
% \paragraph{Unpacking.} The \xfile{.dtx} file is a self-extracting
% \docstrip\ archive. The files are extracted by running the
% \xfile{.dtx} through \plainTeX:
% \begin{quote}
%   \verb|tex hologo.dtx|
% \end{quote}
%
% \paragraph{TDS.} Now the different files must be moved into
% the different directories in your installation TDS tree
% (also known as \xfile{texmf} tree):
% \begin{quote}
% \def\t{^^A
% \begin{tabular}{@{}>{\ttfamily}l@{ $\rightarrow$ }>{\ttfamily}l@{}}
%   hologo.sty & tex/generic/oberdiek/hologo.sty\\
%   hologo.pdf & doc/latex/oberdiek/hologo.pdf\\
%   example/hologo-example.tex & doc/latex/oberdiek/example/hologo-example.tex\\
%   test/hologo-test1.tex & doc/latex/oberdiek/test/hologo-test1.tex\\
%   test/hologo-test-spacefactor.tex & doc/latex/oberdiek/test/hologo-test-spacefactor.tex\\
%   test/hologo-test-list.tex & doc/latex/oberdiek/test/hologo-test-list.tex\\
%   hologo.dtx & source/latex/oberdiek/hologo.dtx\\
% \end{tabular}^^A
% }^^A
% \sbox0{\t}^^A
% \ifdim\wd0>\linewidth
%   \begingroup
%     \advance\linewidth by\leftmargin
%     \advance\linewidth by\rightmargin
%   \edef\x{\endgroup
%     \def\noexpand\lw{\the\linewidth}^^A
%   }\x
%   \def\lwbox{^^A
%     \leavevmode
%     \hbox to \linewidth{^^A
%       \kern-\leftmargin\relax
%       \hss
%       \usebox0
%       \hss
%       \kern-\rightmargin\relax
%     }^^A
%   }^^A
%   \ifdim\wd0>\lw
%     \sbox0{\small\t}^^A
%     \ifdim\wd0>\linewidth
%       \ifdim\wd0>\lw
%         \sbox0{\footnotesize\t}^^A
%         \ifdim\wd0>\linewidth
%           \ifdim\wd0>\lw
%             \sbox0{\scriptsize\t}^^A
%             \ifdim\wd0>\linewidth
%               \ifdim\wd0>\lw
%                 \sbox0{\tiny\t}^^A
%                 \ifdim\wd0>\linewidth
%                   \lwbox
%                 \else
%                   \usebox0
%                 \fi
%               \else
%                 \lwbox
%               \fi
%             \else
%               \usebox0
%             \fi
%           \else
%             \lwbox
%           \fi
%         \else
%           \usebox0
%         \fi
%       \else
%         \lwbox
%       \fi
%     \else
%       \usebox0
%     \fi
%   \else
%     \lwbox
%   \fi
% \else
%   \usebox0
% \fi
% \end{quote}
% If you have a \xfile{docstrip.cfg} that configures and enables \docstrip's
% TDS installing feature, then some files can already be in the right
% place, see the documentation of \docstrip.
%
% \subsection{Refresh file name databases}
%
% If your \TeX~distribution
% (\teTeX, \mikTeX, \dots) relies on file name databases, you must refresh
% these. For example, \teTeX\ users run \verb|texhash| or
% \verb|mktexlsr|.
%
% \subsection{Some details for the interested}
%
% \paragraph{Attached source.}
%
% The PDF documentation on CTAN also includes the
% \xfile{.dtx} source file. It can be extracted by
% AcrobatReader 6 or higher. Another option is \textsf{pdftk},
% e.g. unpack the file into the current directory:
% \begin{quote}
%   \verb|pdftk hologo.pdf unpack_files output .|
% \end{quote}
%
% \paragraph{Unpacking with \LaTeX.}
% The \xfile{.dtx} chooses its action depending on the format:
% \begin{description}
% \item[\plainTeX:] Run \docstrip\ and extract the files.
% \item[\LaTeX:] Generate the documentation.
% \end{description}
% If you insist on using \LaTeX\ for \docstrip\ (really,
% \docstrip\ does not need \LaTeX), then inform the autodetect routine
% about your intention:
% \begin{quote}
%   \verb|latex \let\install=y\input{hologo.dtx}|
% \end{quote}
% Do not forget to quote the argument according to the demands
% of your shell.
%
% \paragraph{Generating the documentation.}
% You can use both the \xfile{.dtx} or the \xfile{.drv} to generate
% the documentation. The process can be configured by the
% configuration file \xfile{ltxdoc.cfg}. For instance, put this
% line into this file, if you want to have A4 as paper format:
% \begin{quote}
%   \verb|\PassOptionsToClass{a4paper}{article}|
% \end{quote}
% An example follows how to generate the
% documentation with pdf\LaTeX:
% \begin{quote}
%\begin{verbatim}
%pdflatex hologo.dtx
%makeindex -s gind.ist hologo.idx
%pdflatex hologo.dtx
%makeindex -s gind.ist hologo.idx
%pdflatex hologo.dtx
%\end{verbatim}
% \end{quote}
%
% \section{Catalogue}
%
% The following XML file can be used as source for the
% \href{http://mirror.ctan.org/help/Catalogue/catalogue.html}{\TeX\ Catalogue}.
% The elements \texttt{caption} and \texttt{description} are imported
% from the original XML file from the Catalogue.
% The name of the XML file in the Catalogue is \xfile{hologo.xml}.
%    \begin{macrocode}
%<*catalogue>
<?xml version='1.0' encoding='us-ascii'?>
<!DOCTYPE entry SYSTEM 'catalogue.dtd'>
<entry datestamp='$Date$' modifier='$Author$' id='hologo'>
  <name>hologo</name>
  <caption>A collection of logos with bookmark support.</caption>
  <authorref id='auth:oberdiek'/>
  <copyright owner='Heiko Oberdiek' year='2010-2012'/>
  <license type='lppl1.3'/>
  <version number='1.10'/>
  <description>
    The package defines a single command <tt>\hologo</tt>, whose
    argument is the usual case-confused ASCII version of the logo.
    The command is bookmark-enabled, so that every logo becomes
    available in bookmarks without further work.
    <p/>
    The package is part of the <xref refid='oberdiek'>oberdiek</xref>
    bundle.
  </description>
  <documentation details='Package documentation'
      href='ctan:/macros/latex/contrib/oberdiek/hologo.pdf'/>
  <ctan file='true' path='/macros/latex/contrib/oberdiek/hologo.dtx'/>
  <miktex location='oberdiek'/>
  <texlive location='oberdiek'/>
  <install path='/macros/latex/contrib/oberdiek/oberdiek.tds.zip'/>
</entry>
%</catalogue>
%    \end{macrocode}
%
% \begin{thebibliography}{9}
% \raggedright
%
% \bibitem{btxdoc}
% Oren Patashnik,
% \textit{\hologo{BibTeX}ing},
% 1988-02-08.\\
% \CTAN{biblio/bibtex/base/}
%
% \bibitem{dtklogos}
% Gerd Neugebauer, DANTE,
% \textit{Package \xpackage{dtklogos}},
% 2011-04-25.\\
% \CTAN{usergrps/dante/dtk/dtklogos.sty}
%
% \bibitem{etexman}
% The \hologo{NTS} Team,
% \textit{The \hologo{eTeX} manual},
% 1998-02.\\
% \CTAN{systems/e-tex/v2/doc/}
%
% \bibitem{ExTeX-FAQ}
% The \hologo{ExTeX} group,
% \textit{\hologo{ExTeX}: FAQ -- How is \hologo{ExTeX} typeset?},
% 2007-04-14.\\
% \url{http://www.extex.org/documentation/faq.html}
%
% \bibitem{LyX}
% %@MISC{ LyX,
% %  title = {{LyX 2.0.0 -- The Document Processor [Computer software and manual]}},
% %  author = {{The LyX Team}},
% %  howpublished = {Internet: http://www.lyx.org},
% %  year = {2011-05-08},
% %  note = {Retrieved May 10, 2011, from http://www.lyx.org},
% %  url = {http://www.lyx.org/}
% %}
% The \hologo{LyX} Team,
% \textit{\hologo{LyX} -- The Document Processor},
% 2011-05-08.\\
% \url{http://www.lyx.org/}
%
% \bibitem{OzTeX}
% Andrew Trevorrow,
% \hologo{OzTeX} FAQ: What is the correct way to typeset ``\hologo{OzTeX}''?,
% 2011-09-15 (visited).
% \url{http://www.trevorrow.com/oztex/ozfaq.html#oztex-logo}
%
% \bibitem{PiCTeX}
% Michael Wichura,
% \textit{The \hologo{PiCTeX} macro package},
% 1987-09-21.
% \CTAN{graphics/pictex/}
%
% \bibitem{scrlogo}
% Markus Kohm,
% \textit{\hologo{KOMAScript} Datei \xfile{scrlogo.dtx}},
% 2009-01-30.\\
% \CTAN{install/macros/latex/contrib/komascript.tds.zip}
%
% \end{thebibliography}
%
% \begin{History}
%   \begin{Version}{2010/04/08 v1.0}
%   \item
%     The first version.
%   \end{Version}
%   \begin{Version}{2010/04/16 v1.1}
%   \item
%     \cs{Hologo} added for support of logos at start of a sentence.
%   \item
%     \cs{hologoSetup} and \cs{hologoLogoSetup} added.
%   \item
%     Options \xoption{break}, \xoption{hyphenbreak}, \xoption{spacebreak}
%     added.
%   \item
%     Variant support added by option \xoption{variant}.
%   \end{Version}
%   \begin{Version}{2010/04/24 v1.2}
%   \item
%     \hologo{LaTeX3} added.
%   \item
%     \hologo{VTeX} added.
%   \end{Version}
%   \begin{Version}{2010/11/21 v1.3}
%   \item
%     \hologo{iniTeX}, \hologo{virTeX} added.
%   \end{Version}
%   \begin{Version}{2011/03/25 v1.4}
%   \item
%     \hologo{ConTeXt} with variants added.
%   \item
%     Option \xoption{discretionarybreak} added as refinement for
%     option \xoption{break}.
%   \end{Version}
%   \begin{Version}{2011/04/21 v1.5}
%   \item
%     Wrong TDS directory for test files fixed.
%   \end{Version}
%   \begin{Version}{2011/10/01 v1.6}
%   \item
%     Support for package \xpackage{tex4ht} added.
%   \item
%     Support for \cs{csname} added if \cs{ifincsname} is available.
%   \item
%     New logos:
%     \hologo{(La)TeX},
%     \hologo{biber},
%     \hologo{BibTeX} (\xoption{sc}, \xoption{sf}),
%     \hologo{emTeX},
%     \hologo{ExTeX},
%     \hologo{KOMAScript},
%     \hologo{La},
%     \hologo{LyX},
%     \hologo{MiKTeX},
%     \hologo{NTS},
%     \hologo{OzMF},
%     \hologo{OzMP},
%     \hologo{OzTeX},
%     \hologo{OzTtH},
%     \hologo{PCTeX},
%     \hologo{PiC},
%     \hologo{PiCTeX},
%     \hologo{METAFONT},
%     \hologo{MetaFun},
%     \hologo{METAPOST},
%     \hologo{MetaPost},
%     \hologo{SLiTeX} (\xoption{lift}, \xoption{narrow}, \xoption{simple}),
%     \hologo{SliTeX} (\xoption{narrow}, \xoption{simple}, \xoption{lift}),
%     \hologo{teTeX}.
%   \item
%     Fixes:
%     \hologo{iniTeX},
%     \hologo{pdfLaTeX},
%     \hologo{pdfTeX},
%     \hologo{virTeX}.
%   \item
%     \cs{hologoFontSetup} and \cs{hologoLogoFontSetup} added.
%   \item
%     \cs{hologoVariant} and \cs{HologoVariant} added.
%   \end{Version}
%   \begin{Version}{2011/11/22 v1.7}
%   \item
%     New logos:
%     \hologo{BibTeX8},
%     \hologo{LaTeXML},
%     \hologo{SageTeX},
%     \hologo{TeX4ht},
%     \hologo{TTH}.
%   \item
%     \hologo{Xe} and friends: Driver stuff fixed.
%   \item
%     \hologo{Xe} and friends: Support for italic added.
%   \item
%     \hologo{Xe} and friends: Package support for \xpackage{pgf}
%     and \xpackage{pstricks} added.
%   \end{Version}
%   \begin{Version}{2011/11/29 v1.8}
%   \item
%     New logos:
%     \hologo{HanTheThanh}.
%   \end{Version}
%   \begin{Version}{2011/12/21 v1.9}
%   \item
%     Patch for package \xpackage{ifxetex} added for the case that
%     \cs{newif} is undefined in \hologo{iniTeX}.
%   \item
%     Some fixes for \hologo{iniTeX}.
%   \end{Version}
%   \begin{Version}{2012/04/26 v1.10}
%   \item
%     Fix in bookmark version of logo ``\hologo{HanTheThanh}''.
%   \end{Version}
%   \begin{Version}{2016/05/12 v1.11}
%   \item
%     Update HOLOGO@IfCharExists (previously in texlive)
%   \item define pdfliteral in current luatex.
%   \end{Version}
% \end{History}
%
% \PrintIndex
%
% \Finale
\endinput
|
% \end{quote}
% Do not forget to quote the argument according to the demands
% of your shell.
%
% \paragraph{Generating the documentation.}
% You can use both the \xfile{.dtx} or the \xfile{.drv} to generate
% the documentation. The process can be configured by the
% configuration file \xfile{ltxdoc.cfg}. For instance, put this
% line into this file, if you want to have A4 as paper format:
% \begin{quote}
%   \verb|\PassOptionsToClass{a4paper}{article}|
% \end{quote}
% An example follows how to generate the
% documentation with pdf\LaTeX:
% \begin{quote}
%\begin{verbatim}
%pdflatex hologo.dtx
%makeindex -s gind.ist hologo.idx
%pdflatex hologo.dtx
%makeindex -s gind.ist hologo.idx
%pdflatex hologo.dtx
%\end{verbatim}
% \end{quote}
%
% \section{Catalogue}
%
% The following XML file can be used as source for the
% \href{http://mirror.ctan.org/help/Catalogue/catalogue.html}{\TeX\ Catalogue}.
% The elements \texttt{caption} and \texttt{description} are imported
% from the original XML file from the Catalogue.
% The name of the XML file in the Catalogue is \xfile{hologo.xml}.
%    \begin{macrocode}
%<*catalogue>
<?xml version='1.0' encoding='us-ascii'?>
<!DOCTYPE entry SYSTEM 'catalogue.dtd'>
<entry datestamp='$Date$' modifier='$Author$' id='hologo'>
  <name>hologo</name>
  <caption>A collection of logos with bookmark support.</caption>
  <authorref id='auth:oberdiek'/>
  <copyright owner='Heiko Oberdiek' year='2010-2012'/>
  <license type='lppl1.3'/>
  <version number='1.10'/>
  <description>
    The package defines a single command <tt>\hologo</tt>, whose
    argument is the usual case-confused ASCII version of the logo.
    The command is bookmark-enabled, so that every logo becomes
    available in bookmarks without further work.
    <p/>
    The package is part of the <xref refid='oberdiek'>oberdiek</xref>
    bundle.
  </description>
  <documentation details='Package documentation'
      href='ctan:/macros/latex/contrib/oberdiek/hologo.pdf'/>
  <ctan file='true' path='/macros/latex/contrib/oberdiek/hologo.dtx'/>
  <miktex location='oberdiek'/>
  <texlive location='oberdiek'/>
  <install path='/macros/latex/contrib/oberdiek/oberdiek.tds.zip'/>
</entry>
%</catalogue>
%    \end{macrocode}
%
% \begin{thebibliography}{9}
% \raggedright
%
% \bibitem{btxdoc}
% Oren Patashnik,
% \textit{\hologo{BibTeX}ing},
% 1988-02-08.\\
% \CTAN{biblio/bibtex/base/}
%
% \bibitem{dtklogos}
% Gerd Neugebauer, DANTE,
% \textit{Package \xpackage{dtklogos}},
% 2011-04-25.\\
% \CTAN{usergrps/dante/dtk/dtklogos.sty}
%
% \bibitem{etexman}
% The \hologo{NTS} Team,
% \textit{The \hologo{eTeX} manual},
% 1998-02.\\
% \CTAN{systems/e-tex/v2/doc/}
%
% \bibitem{ExTeX-FAQ}
% The \hologo{ExTeX} group,
% \textit{\hologo{ExTeX}: FAQ -- How is \hologo{ExTeX} typeset?},
% 2007-04-14.\\
% \url{http://www.extex.org/documentation/faq.html}
%
% \bibitem{LyX}
% %@MISC{ LyX,
% %  title = {{LyX 2.0.0 -- The Document Processor [Computer software and manual]}},
% %  author = {{The LyX Team}},
% %  howpublished = {Internet: http://www.lyx.org},
% %  year = {2011-05-08},
% %  note = {Retrieved May 10, 2011, from http://www.lyx.org},
% %  url = {http://www.lyx.org/}
% %}
% The \hologo{LyX} Team,
% \textit{\hologo{LyX} -- The Document Processor},
% 2011-05-08.\\
% \url{http://www.lyx.org/}
%
% \bibitem{OzTeX}
% Andrew Trevorrow,
% \hologo{OzTeX} FAQ: What is the correct way to typeset ``\hologo{OzTeX}''?,
% 2011-09-15 (visited).
% \url{http://www.trevorrow.com/oztex/ozfaq.html#oztex-logo}
%
% \bibitem{PiCTeX}
% Michael Wichura,
% \textit{The \hologo{PiCTeX} macro package},
% 1987-09-21.
% \CTAN{graphics/pictex/}
%
% \bibitem{scrlogo}
% Markus Kohm,
% \textit{\hologo{KOMAScript} Datei \xfile{scrlogo.dtx}},
% 2009-01-30.\\
% \CTAN{install/macros/latex/contrib/komascript.tds.zip}
%
% \end{thebibliography}
%
% \begin{History}
%   \begin{Version}{2010/04/08 v1.0}
%   \item
%     The first version.
%   \end{Version}
%   \begin{Version}{2010/04/16 v1.1}
%   \item
%     \cs{Hologo} added for support of logos at start of a sentence.
%   \item
%     \cs{hologoSetup} and \cs{hologoLogoSetup} added.
%   \item
%     Options \xoption{break}, \xoption{hyphenbreak}, \xoption{spacebreak}
%     added.
%   \item
%     Variant support added by option \xoption{variant}.
%   \end{Version}
%   \begin{Version}{2010/04/24 v1.2}
%   \item
%     \hologo{LaTeX3} added.
%   \item
%     \hologo{VTeX} added.
%   \end{Version}
%   \begin{Version}{2010/11/21 v1.3}
%   \item
%     \hologo{iniTeX}, \hologo{virTeX} added.
%   \end{Version}
%   \begin{Version}{2011/03/25 v1.4}
%   \item
%     \hologo{ConTeXt} with variants added.
%   \item
%     Option \xoption{discretionarybreak} added as refinement for
%     option \xoption{break}.
%   \end{Version}
%   \begin{Version}{2011/04/21 v1.5}
%   \item
%     Wrong TDS directory for test files fixed.
%   \end{Version}
%   \begin{Version}{2011/10/01 v1.6}
%   \item
%     Support for package \xpackage{tex4ht} added.
%   \item
%     Support for \cs{csname} added if \cs{ifincsname} is available.
%   \item
%     New logos:
%     \hologo{(La)TeX},
%     \hologo{biber},
%     \hologo{BibTeX} (\xoption{sc}, \xoption{sf}),
%     \hologo{emTeX},
%     \hologo{ExTeX},
%     \hologo{KOMAScript},
%     \hologo{La},
%     \hologo{LyX},
%     \hologo{MiKTeX},
%     \hologo{NTS},
%     \hologo{OzMF},
%     \hologo{OzMP},
%     \hologo{OzTeX},
%     \hologo{OzTtH},
%     \hologo{PCTeX},
%     \hologo{PiC},
%     \hologo{PiCTeX},
%     \hologo{METAFONT},
%     \hologo{MetaFun},
%     \hologo{METAPOST},
%     \hologo{MetaPost},
%     \hologo{SLiTeX} (\xoption{lift}, \xoption{narrow}, \xoption{simple}),
%     \hologo{SliTeX} (\xoption{narrow}, \xoption{simple}, \xoption{lift}),
%     \hologo{teTeX}.
%   \item
%     Fixes:
%     \hologo{iniTeX},
%     \hologo{pdfLaTeX},
%     \hologo{pdfTeX},
%     \hologo{virTeX}.
%   \item
%     \cs{hologoFontSetup} and \cs{hologoLogoFontSetup} added.
%   \item
%     \cs{hologoVariant} and \cs{HologoVariant} added.
%   \end{Version}
%   \begin{Version}{2011/11/22 v1.7}
%   \item
%     New logos:
%     \hologo{BibTeX8},
%     \hologo{LaTeXML},
%     \hologo{SageTeX},
%     \hologo{TeX4ht},
%     \hologo{TTH}.
%   \item
%     \hologo{Xe} and friends: Driver stuff fixed.
%   \item
%     \hologo{Xe} and friends: Support for italic added.
%   \item
%     \hologo{Xe} and friends: Package support for \xpackage{pgf}
%     and \xpackage{pstricks} added.
%   \end{Version}
%   \begin{Version}{2011/11/29 v1.8}
%   \item
%     New logos:
%     \hologo{HanTheThanh}.
%   \end{Version}
%   \begin{Version}{2011/12/21 v1.9}
%   \item
%     Patch for package \xpackage{ifxetex} added for the case that
%     \cs{newif} is undefined in \hologo{iniTeX}.
%   \item
%     Some fixes for \hologo{iniTeX}.
%   \end{Version}
%   \begin{Version}{2012/04/26 v1.10}
%   \item
%     Fix in bookmark version of logo ``\hologo{HanTheThanh}''.
%   \end{Version}
%   \begin{Version}{2016/05/12 v1.11}
%   \item
%     Update HOLOGO@IfCharExists (previously in texlive)
%   \item define pdfliteral in current luatex.
%   \end{Version}
% \end{History}
%
% \PrintIndex
%
% \Finale
\endinput
|
% \end{quote}
% Do not forget to quote the argument according to the demands
% of your shell.
%
% \paragraph{Generating the documentation.}
% You can use both the \xfile{.dtx} or the \xfile{.drv} to generate
% the documentation. The process can be configured by the
% configuration file \xfile{ltxdoc.cfg}. For instance, put this
% line into this file, if you want to have A4 as paper format:
% \begin{quote}
%   \verb|\PassOptionsToClass{a4paper}{article}|
% \end{quote}
% An example follows how to generate the
% documentation with pdf\LaTeX:
% \begin{quote}
%\begin{verbatim}
%pdflatex hologo.dtx
%makeindex -s gind.ist hologo.idx
%pdflatex hologo.dtx
%makeindex -s gind.ist hologo.idx
%pdflatex hologo.dtx
%\end{verbatim}
% \end{quote}
%
% \section{Catalogue}
%
% The following XML file can be used as source for the
% \href{http://mirror.ctan.org/help/Catalogue/catalogue.html}{\TeX\ Catalogue}.
% The elements \texttt{caption} and \texttt{description} are imported
% from the original XML file from the Catalogue.
% The name of the XML file in the Catalogue is \xfile{hologo.xml}.
%    \begin{macrocode}
%<*catalogue>
<?xml version='1.0' encoding='us-ascii'?>
<!DOCTYPE entry SYSTEM 'catalogue.dtd'>
<entry datestamp='$Date$' modifier='$Author$' id='hologo'>
  <name>hologo</name>
  <caption>A collection of logos with bookmark support.</caption>
  <authorref id='auth:oberdiek'/>
  <copyright owner='Heiko Oberdiek' year='2010-2012'/>
  <license type='lppl1.3'/>
  <version number='1.10'/>
  <description>
    The package defines a single command <tt>\hologo</tt>, whose
    argument is the usual case-confused ASCII version of the logo.
    The command is bookmark-enabled, so that every logo becomes
    available in bookmarks without further work.
    <p/>
    The package is part of the <xref refid='oberdiek'>oberdiek</xref>
    bundle.
  </description>
  <documentation details='Package documentation'
      href='ctan:/macros/latex/contrib/oberdiek/hologo.pdf'/>
  <ctan file='true' path='/macros/latex/contrib/oberdiek/hologo.dtx'/>
  <miktex location='oberdiek'/>
  <texlive location='oberdiek'/>
  <install path='/macros/latex/contrib/oberdiek/oberdiek.tds.zip'/>
</entry>
%</catalogue>
%    \end{macrocode}
%
% \begin{thebibliography}{9}
% \raggedright
%
% \bibitem{btxdoc}
% Oren Patashnik,
% \textit{\hologo{BibTeX}ing},
% 1988-02-08.\\
% \CTAN{biblio/bibtex/base/}
%
% \bibitem{dtklogos}
% Gerd Neugebauer, DANTE,
% \textit{Package \xpackage{dtklogos}},
% 2011-04-25.\\
% \CTAN{usergrps/dante/dtk/dtklogos.sty}
%
% \bibitem{etexman}
% The \hologo{NTS} Team,
% \textit{The \hologo{eTeX} manual},
% 1998-02.\\
% \CTAN{systems/e-tex/v2/doc/}
%
% \bibitem{ExTeX-FAQ}
% The \hologo{ExTeX} group,
% \textit{\hologo{ExTeX}: FAQ -- How is \hologo{ExTeX} typeset?},
% 2007-04-14.\\
% \url{http://www.extex.org/documentation/faq.html}
%
% \bibitem{LyX}
% %@MISC{ LyX,
% %  title = {{LyX 2.0.0 -- The Document Processor [Computer software and manual]}},
% %  author = {{The LyX Team}},
% %  howpublished = {Internet: http://www.lyx.org},
% %  year = {2011-05-08},
% %  note = {Retrieved May 10, 2011, from http://www.lyx.org},
% %  url = {http://www.lyx.org/}
% %}
% The \hologo{LyX} Team,
% \textit{\hologo{LyX} -- The Document Processor},
% 2011-05-08.\\
% \url{http://www.lyx.org/}
%
% \bibitem{OzTeX}
% Andrew Trevorrow,
% \hologo{OzTeX} FAQ: What is the correct way to typeset ``\hologo{OzTeX}''?,
% 2011-09-15 (visited).
% \url{http://www.trevorrow.com/oztex/ozfaq.html#oztex-logo}
%
% \bibitem{PiCTeX}
% Michael Wichura,
% \textit{The \hologo{PiCTeX} macro package},
% 1987-09-21.
% \CTAN{graphics/pictex/}
%
% \bibitem{scrlogo}
% Markus Kohm,
% \textit{\hologo{KOMAScript} Datei \xfile{scrlogo.dtx}},
% 2009-01-30.\\
% \CTAN{install/macros/latex/contrib/komascript.tds.zip}
%
% \end{thebibliography}
%
% \begin{History}
%   \begin{Version}{2010/04/08 v1.0}
%   \item
%     The first version.
%   \end{Version}
%   \begin{Version}{2010/04/16 v1.1}
%   \item
%     \cs{Hologo} added for support of logos at start of a sentence.
%   \item
%     \cs{hologoSetup} and \cs{hologoLogoSetup} added.
%   \item
%     Options \xoption{break}, \xoption{hyphenbreak}, \xoption{spacebreak}
%     added.
%   \item
%     Variant support added by option \xoption{variant}.
%   \end{Version}
%   \begin{Version}{2010/04/24 v1.2}
%   \item
%     \hologo{LaTeX3} added.
%   \item
%     \hologo{VTeX} added.
%   \end{Version}
%   \begin{Version}{2010/11/21 v1.3}
%   \item
%     \hologo{iniTeX}, \hologo{virTeX} added.
%   \end{Version}
%   \begin{Version}{2011/03/25 v1.4}
%   \item
%     \hologo{ConTeXt} with variants added.
%   \item
%     Option \xoption{discretionarybreak} added as refinement for
%     option \xoption{break}.
%   \end{Version}
%   \begin{Version}{2011/04/21 v1.5}
%   \item
%     Wrong TDS directory for test files fixed.
%   \end{Version}
%   \begin{Version}{2011/10/01 v1.6}
%   \item
%     Support for package \xpackage{tex4ht} added.
%   \item
%     Support for \cs{csname} added if \cs{ifincsname} is available.
%   \item
%     New logos:
%     \hologo{(La)TeX},
%     \hologo{biber},
%     \hologo{BibTeX} (\xoption{sc}, \xoption{sf}),
%     \hologo{emTeX},
%     \hologo{ExTeX},
%     \hologo{KOMAScript},
%     \hologo{La},
%     \hologo{LyX},
%     \hologo{MiKTeX},
%     \hologo{NTS},
%     \hologo{OzMF},
%     \hologo{OzMP},
%     \hologo{OzTeX},
%     \hologo{OzTtH},
%     \hologo{PCTeX},
%     \hologo{PiC},
%     \hologo{PiCTeX},
%     \hologo{METAFONT},
%     \hologo{MetaFun},
%     \hologo{METAPOST},
%     \hologo{MetaPost},
%     \hologo{SLiTeX} (\xoption{lift}, \xoption{narrow}, \xoption{simple}),
%     \hologo{SliTeX} (\xoption{narrow}, \xoption{simple}, \xoption{lift}),
%     \hologo{teTeX}.
%   \item
%     Fixes:
%     \hologo{iniTeX},
%     \hologo{pdfLaTeX},
%     \hologo{pdfTeX},
%     \hologo{virTeX}.
%   \item
%     \cs{hologoFontSetup} and \cs{hologoLogoFontSetup} added.
%   \item
%     \cs{hologoVariant} and \cs{HologoVariant} added.
%   \end{Version}
%   \begin{Version}{2011/11/22 v1.7}
%   \item
%     New logos:
%     \hologo{BibTeX8},
%     \hologo{LaTeXML},
%     \hologo{SageTeX},
%     \hologo{TeX4ht},
%     \hologo{TTH}.
%   \item
%     \hologo{Xe} and friends: Driver stuff fixed.
%   \item
%     \hologo{Xe} and friends: Support for italic added.
%   \item
%     \hologo{Xe} and friends: Package support for \xpackage{pgf}
%     and \xpackage{pstricks} added.
%   \end{Version}
%   \begin{Version}{2011/11/29 v1.8}
%   \item
%     New logos:
%     \hologo{HanTheThanh}.
%   \end{Version}
%   \begin{Version}{2011/12/21 v1.9}
%   \item
%     Patch for package \xpackage{ifxetex} added for the case that
%     \cs{newif} is undefined in \hologo{iniTeX}.
%   \item
%     Some fixes for \hologo{iniTeX}.
%   \end{Version}
%   \begin{Version}{2012/04/26 v1.10}
%   \item
%     Fix in bookmark version of logo ``\hologo{HanTheThanh}''.
%   \end{Version}
%   \begin{Version}{2016/05/12 v1.11}
%   \item
%     Update HOLOGO@IfCharExists (previously in texlive)
%   \item define pdfliteral in current luatex.
%   \end{Version}
% \end{History}
%
% \PrintIndex
%
% \Finale
\endinput
%
        \else
          \input hologo.cfg\relax
        \fi
      \fi
    }%
    \ltx@IfUndefined{newread}{%
      \chardef\HOLOGO@temp=15 %
      \def\HOLOGO@CheckRead{%
        \ifeof\HOLOGO@temp
          \HOLOGO@InputIfExists
        \else
          \ifcase\HOLOGO@temp
            \@PackageWarningNoLine{hologo}{%
              Configuration file ignored, because\MessageBreak
              a free read register could not be found%
            }%
          \else
            \begingroup
              \count\ltx@cclv=\HOLOGO@temp
              \advance\ltx@cclv by \ltx@minusone
              \edef\x{\endgroup
                \chardef\noexpand\HOLOGO@temp=\the\count\ltx@cclv
                \relax
              }%
            \x
          \fi
        \fi
      }%
    }{%
      \csname newread\endcsname\HOLOGO@temp
      \HOLOGO@InputIfExists
    }%
  }{%
    \edef\HOLOGO@temp{\pdf@filesize{hologo.cfg}}%
    \ifx\HOLOGO@temp\ltx@empty
    \else
      \ifnum\HOLOGO@temp>0 %
        \begingroup
          \def\x{LaTeX2e}%
        \expandafter\endgroup
        \ifx\fmtname\x
          % \iffalse meta-comment
%
% File: hologo.dtx
% Version: 2016/05/12 v1.11
% Info: A logo collection with bookmark support
%
% Copyright (C) 2010-2012 by
%    Heiko Oberdiek <heiko.oberdiek at googlemail.com>
%
% This work may be distributed and/or modified under the
% conditions of the LaTeX Project Public License, either
% version 1.3c of this license or (at your option) any later
% version. This version of this license is in
%    http://www.latex-project.org/lppl/lppl-1-3c.txt
% and the latest version of this license is in
%    http://www.latex-project.org/lppl.txt
% and version 1.3 or later is part of all distributions of
% LaTeX version 2005/12/01 or later.
%
% This work has the LPPL maintenance status "maintained".
%
% This Current Maintainer of this work is Heiko Oberdiek.
%
% The Base Interpreter refers to any `TeX-Format',
% because some files are installed in TDS:tex/generic//.
%
% This work consists of the main source file hologo.dtx
% and the derived files
%    hologo.sty, hologo.pdf, hologo.ins, hologo.drv, hologo-example.tex,
%    hologo-test1.tex, hologo-test-spacefactor.tex,
%    hologo-test-list.tex.
%
% Distribution:
%    CTAN:macros/latex/contrib/oberdiek/hologo.dtx
%    CTAN:macros/latex/contrib/oberdiek/hologo.pdf
%
% Unpacking:
%    (a) If hologo.ins is present:
%           tex hologo.ins
%    (b) Without hologo.ins:
%           tex hologo.dtx
%    (c) If you insist on using LaTeX
%           latex \let\install=y% \iffalse meta-comment
%
% File: hologo.dtx
% Version: 2016/05/12 v1.11
% Info: A logo collection with bookmark support
%
% Copyright (C) 2010-2012 by
%    Heiko Oberdiek <heiko.oberdiek at googlemail.com>
%
% This work may be distributed and/or modified under the
% conditions of the LaTeX Project Public License, either
% version 1.3c of this license or (at your option) any later
% version. This version of this license is in
%    http://www.latex-project.org/lppl/lppl-1-3c.txt
% and the latest version of this license is in
%    http://www.latex-project.org/lppl.txt
% and version 1.3 or later is part of all distributions of
% LaTeX version 2005/12/01 or later.
%
% This work has the LPPL maintenance status "maintained".
%
% This Current Maintainer of this work is Heiko Oberdiek.
%
% The Base Interpreter refers to any `TeX-Format',
% because some files are installed in TDS:tex/generic//.
%
% This work consists of the main source file hologo.dtx
% and the derived files
%    hologo.sty, hologo.pdf, hologo.ins, hologo.drv, hologo-example.tex,
%    hologo-test1.tex, hologo-test-spacefactor.tex,
%    hologo-test-list.tex.
%
% Distribution:
%    CTAN:macros/latex/contrib/oberdiek/hologo.dtx
%    CTAN:macros/latex/contrib/oberdiek/hologo.pdf
%
% Unpacking:
%    (a) If hologo.ins is present:
%           tex hologo.ins
%    (b) Without hologo.ins:
%           tex hologo.dtx
%    (c) If you insist on using LaTeX
%           latex \let\install=y% \iffalse meta-comment
%
% File: hologo.dtx
% Version: 2016/05/12 v1.11
% Info: A logo collection with bookmark support
%
% Copyright (C) 2010-2012 by
%    Heiko Oberdiek <heiko.oberdiek at googlemail.com>
%
% This work may be distributed and/or modified under the
% conditions of the LaTeX Project Public License, either
% version 1.3c of this license or (at your option) any later
% version. This version of this license is in
%    http://www.latex-project.org/lppl/lppl-1-3c.txt
% and the latest version of this license is in
%    http://www.latex-project.org/lppl.txt
% and version 1.3 or later is part of all distributions of
% LaTeX version 2005/12/01 or later.
%
% This work has the LPPL maintenance status "maintained".
%
% This Current Maintainer of this work is Heiko Oberdiek.
%
% The Base Interpreter refers to any `TeX-Format',
% because some files are installed in TDS:tex/generic//.
%
% This work consists of the main source file hologo.dtx
% and the derived files
%    hologo.sty, hologo.pdf, hologo.ins, hologo.drv, hologo-example.tex,
%    hologo-test1.tex, hologo-test-spacefactor.tex,
%    hologo-test-list.tex.
%
% Distribution:
%    CTAN:macros/latex/contrib/oberdiek/hologo.dtx
%    CTAN:macros/latex/contrib/oberdiek/hologo.pdf
%
% Unpacking:
%    (a) If hologo.ins is present:
%           tex hologo.ins
%    (b) Without hologo.ins:
%           tex hologo.dtx
%    (c) If you insist on using LaTeX
%           latex \let\install=y\input{hologo.dtx}
%        (quote the arguments according to the demands of your shell)
%
% Documentation:
%    (a) If hologo.drv is present:
%           latex hologo.drv
%    (b) Without hologo.drv:
%           latex hologo.dtx; ...
%    The class ltxdoc loads the configuration file ltxdoc.cfg
%    if available. Here you can specify further options, e.g.
%    use A4 as paper format:
%       \PassOptionsToClass{a4paper}{article}
%
%    Programm calls to get the documentation (example):
%       pdflatex hologo.dtx
%       makeindex -s gind.ist hologo.idx
%       pdflatex hologo.dtx
%       makeindex -s gind.ist hologo.idx
%       pdflatex hologo.dtx
%
% Installation:
%    TDS:tex/generic/oberdiek/hologo.sty
%    TDS:doc/latex/oberdiek/hologo.pdf
%    TDS:doc/latex/oberdiek/example/hologo-example.tex
%    TDS:doc/latex/oberdiek/test/hologo-test1.tex
%    TDS:doc/latex/oberdiek/test/hologo-test-spacefactor.tex
%    TDS:doc/latex/oberdiek/test/hologo-test-list.tex
%    TDS:source/latex/oberdiek/hologo.dtx
%
%<*ignore>
\begingroup
  \catcode123=1 %
  \catcode125=2 %
  \def\x{LaTeX2e}%
\expandafter\endgroup
\ifcase 0\ifx\install y1\fi\expandafter
         \ifx\csname processbatchFile\endcsname\relax\else1\fi
         \ifx\fmtname\x\else 1\fi\relax
\else\csname fi\endcsname
%</ignore>
%<*install>
\input docstrip.tex
\Msg{************************************************************************}
\Msg{* Installation}
\Msg{* Package: hologo 2016/05/12 v1.11 A logo collection with bookmark support (HO)}
\Msg{************************************************************************}

\keepsilent
\askforoverwritefalse

\let\MetaPrefix\relax
\preamble

This is a generated file.

Project: hologo
Version: 2016/05/12 v1.11

Copyright (C) 2010-2012 by
   Heiko Oberdiek <heiko.oberdiek at googlemail.com>

This work may be distributed and/or modified under the
conditions of the LaTeX Project Public License, either
version 1.3c of this license or (at your option) any later
version. This version of this license is in
   http://www.latex-project.org/lppl/lppl-1-3c.txt
and the latest version of this license is in
   http://www.latex-project.org/lppl.txt
and version 1.3 or later is part of all distributions of
LaTeX version 2005/12/01 or later.

This work has the LPPL maintenance status "maintained".

This Current Maintainer of this work is Heiko Oberdiek.

The Base Interpreter refers to any `TeX-Format',
because some files are installed in TDS:tex/generic//.

This work consists of the main source file hologo.dtx
and the derived files
   hologo.sty, hologo.pdf, hologo.ins, hologo.drv, hologo-example.tex,
   hologo-test1.tex, hologo-test-spacefactor.tex,
   hologo-test-list.tex.

\endpreamble
\let\MetaPrefix\DoubleperCent

\generate{%
  \file{hologo.ins}{\from{hologo.dtx}{install}}%
  \file{hologo.drv}{\from{hologo.dtx}{driver}}%
  \usedir{tex/generic/oberdiek}%
  \file{hologo.sty}{\from{hologo.dtx}{package}}%
  \usedir{doc/latex/oberdiek/example}%
  \file{hologo-example.tex}{\from{hologo.dtx}{example}}%
  \usedir{doc/latex/oberdiek/test}%
  \file{hologo-test1.tex}{\from{hologo.dtx}{test1}}%
  \file{hologo-test-spacefactor.tex}{\from{hologo.dtx}{test-spacefactor}}%
  \file{hologo-test-list.tex}{\from{hologo.dtx}{test-list}}%
  \nopreamble
  \nopostamble
  \usedir{source/latex/oberdiek/catalogue}%
  \file{hologo.xml}{\from{hologo.dtx}{catalogue}}%
}

\catcode32=13\relax% active space
\let =\space%
\Msg{************************************************************************}
\Msg{*}
\Msg{* To finish the installation you have to move the following}
\Msg{* file into a directory searched by TeX:}
\Msg{*}
\Msg{*     hologo.sty}
\Msg{*}
\Msg{* To produce the documentation run the file `hologo.drv'}
\Msg{* through LaTeX.}
\Msg{*}
\Msg{* Happy TeXing!}
\Msg{*}
\Msg{************************************************************************}

\endbatchfile
%</install>
%<*ignore>
\fi
%</ignore>
%<*driver>
\NeedsTeXFormat{LaTeX2e}
\ProvidesFile{hologo.drv}%
  [2016/05/12 v1.11 A logo collection with bookmark support (HO)]%
\documentclass{ltxdoc}
\usepackage{holtxdoc}[2011/11/22]
\usepackage{hologo}[2016/05/12]
\usepackage{longtable}
\usepackage{array}
\usepackage{paralist}
%\usepackage[T1]{fontenc}
%\usepackage{lmodern}
\begin{document}
  \DocInput{hologo.dtx}%
\end{document}
%</driver>
% \fi
%
%
% \CharacterTable
%  {Upper-case    \A\B\C\D\E\F\G\H\I\J\K\L\M\N\O\P\Q\R\S\T\U\V\W\X\Y\Z
%   Lower-case    \a\b\c\d\e\f\g\h\i\j\k\l\m\n\o\p\q\r\s\t\u\v\w\x\y\z
%   Digits        \0\1\2\3\4\5\6\7\8\9
%   Exclamation   \!     Double quote  \"     Hash (number) \#
%   Dollar        \$     Percent       \%     Ampersand     \&
%   Acute accent  \'     Left paren    \(     Right paren   \)
%   Asterisk      \*     Plus          \+     Comma         \,
%   Minus         \-     Point         \.     Solidus       \/
%   Colon         \:     Semicolon     \;     Less than     \<
%   Equals        \=     Greater than  \>     Question mark \?
%   Commercial at \@     Left bracket  \[     Backslash     \\
%   Right bracket \]     Circumflex    \^     Underscore    \_
%   Grave accent  \`     Left brace    \{     Vertical bar  \|
%   Right brace   \}     Tilde         \~}
%
% \GetFileInfo{hologo.drv}
%
% \title{The \xpackage{hologo} package}
% \date{2016/05/12 v1.11}
% \author{Heiko Oberdiek\\\xemail{heiko.oberdiek at googlemail.com}}
%
% \maketitle
%
% \begin{abstract}
% This package starts a collection of logos with support for bookmarks
% strings.
% \end{abstract}
%
% \tableofcontents
%
% \section{Documentation}
%
% \subsection{Logo macros}
%
% \begin{declcs}{hologo} \M{name}
% \end{declcs}
% Macro \cs{hologo} sets the logo with name \meta{name}.
% The following table shows the supported names.
%
% \begingroup
%   \def\hologoEntry#1#2#3{^^A
%     #1&#2&\hologoLogoSetup{#1}{variant=#2}\hologo{#1}&#3\tabularnewline
%   }
%   \begin{longtable}{>{\ttfamily}l>{\ttfamily}lll}
%     \rmfamily\bfseries{name} & \rmfamily\bfseries variant
%     & \bfseries logo & \bfseries since\\
%     \hline
%     \endhead
%     \hologoList
%   \end{longtable}
% \endgroup
%
% \begin{declcs}{Hologo} \M{name}
% \end{declcs}
% Macro \cs{Hologo} starts the logo \meta{name} with an uppercase
% letter. As an exception small greek letters are not converted
% to uppercase. Examples, see \hologo{eTeX} and \hologo{ExTeX}.
%
% \subsection{Setup macros}
%
% The package does not support package options, but the following
% setup macros can be used to set options.
%
% \begin{declcs}{hologoSetup} \M{key value list}
% \end{declcs}
% Macro \cs{hologoSetup} sets global options.
%
% \begin{declcs}{hologoLogoSetup} \M{logo} \M{key value list}
% \end{declcs}
% Some options can also be used to configure a logo.
% These settings take precedence over global option settings.
%
% \subsection{Options}\label{sec:options}
%
% There are boolean and string options:
% \begin{description}
% \item[Boolean option:]
% It takes |true| or |false|
% as value. If the value is omitted, then |true| is used.
% \item[String option:]
% A value must be given as string. (But the string might be empty.)
% \end{description}
% The following options can be used both in \cs{hologoSetup}
% and \cs{hologoLogoSetup}:
% \begin{description}
% \def\entry#1{\item[\xoption{#1}:]}
% \entry{break}
%   enables or disables line breaks inside the logo. This setting is
%   refined by options \xoption{hyphenbreak}, \xoption{spacebreak}
%   or \xoption{discretionarybreak}.
%   Default is |false|.
% \entry{hyphenbreak}
%   enables or disables the line break right after the hyphen character.
% \entry{spacebreak}
%   enables or disables line breaks at space characters.
% \entry{discretionarybreak}
%   enables or disables line breaks at hyphenation points
%   (inserted by \cs{-}).
% \end{description}
% Macro \cs{hologoLogoSetup} also knows:
% \begin{description}
% \item[\xoption{variant}:]
%   This is a string option. It specifies a variant of a logo that
%   must exist. An empty string selects the package default variant.
% \end{description}
% Example:
% \begin{quote}
%   |\hologoSetup{break=false}|\\
%   |\hologoLogoSetup{plainTeX}{variant=hyphen,hyphenbreak}|\\
%   Then ``plain-\TeX'' contains one break point after the hyphen.
% \end{quote}
%
% \subsection{Driver options}
%
% Sometimes graphical operations are needed to construct some
% glyphs (e.g.\ \hologo{XeTeX}). If package \xpackage{graphics}
% or package \xpackage{pgf} are found, then the macros are taken
% from there. Otherwise the packge defines its own operations
% and therefore needs the driver information. Many drivers are
% detected automatically (\hologo{pdfTeX}/\hologo{LuaTeX}
% in PDF mode, \hologo{XeTeX}, \hologo{VTeX}). These have precedence
% over a driver option. The driver can be given as package option
% or using \cs{hologoDriverSetup}.
% The following list contains the recognized driver options:
% \begin{itemize}
% \item \xoption{pdftex}, \xoption{luatex}
% \item \xoption{dvipdfm}, \xoption{dvipdfmx}
% \item \xoption{dvips}, \xoption{dvipsone}, \xoption{xdvi}
% \item \xoption{xetex}
% \item \xoption{vtex}
% \end{itemize}
% The left driver of a line is the driver name that is used internally.
% The following names are aliases for drivers that use the
% same method. Therefore the entry in the \xext{log} file for
% the used driver prints the internally used driver name.
% \begin{description}
% \item[\xoption{driverfallback}:]
%   This option expects a driver that is used,
%   if the driver could not be detected automatically.
% \end{description}
%
% \begin{declcs}{hologoDriverSetup} \M{driver option}
% \end{declcs}
% The driver can also be configured after package loading
% using \cs{hologoDriverSetup}, also the way for \hologo{plainTeX}
% to setup the driver.
%
% \subsection{Font setup}
%
% Some logos require a special font, but should also be usable by
% \hologo{plainTeX}. Therefore the package provides some ways
% to influence the font settings. The options below
% take font settings as values. Both font commands
% such as \cs{sffamily} and macros that take one argument
% like \cs{textsf} can be used.
%
% \begin{declcs}{hologoFontSetup} \M{key value list}
% \end{declcs}
% Macro \cs{hologoFontSetup} sets the fonts for all logos.
% Supported keys:
% \begin{description}
% \def\entry#1{\item[\xoption{#1}:]}
% \entry{general}
%   This font is used for all logos. The default is empty.
%   That means no special font is used.
% \entry{bibsf}
%   This font is used for
%   {\hologoLogoSetup{BibTeX}{variant=sf}\hologo{BibTeX}}
%   with variant \xoption{sf}.
% \entry{rm}
%   This font is a serif font. It is used for \hologo{ExTeX}.
% \entry{sc}
%   This font specifies a small caps font. It is used for
%   {\hologoLogoSetup{BibTeX}{variant=sc}\hologo{BibTeX}}
%   with variant \xoption{sc}.
% \entry{sf}
%   This font specifies a sans serif font. The default
%   is \cs{sffamily}, then \cs{sf} is tried. Otherwise
%   a warning is given. It is used by \hologo{KOMAScript}.
% \entry{sy}
%   This is the font for math symbols (e.g. cmsy).
%   It is used by \hologo{AmS}, \hologo{NTS}, \hologo{ExTeX}.
% \entry{logo}
%   \hologo{METAFONT} and \hologo{METAPOST} are using that font.
%   In \hologo{LaTeX} \cs{logofamily} is used and
%   the definitions of package \xpackage{mflogo} are used
%   if the package is not loaded.
%   Otherwise the \cs{tenlogo} is used and defined
%   if it does not already exists.
% \end{description}
%
% \begin{declcs}{hologoLogoFontSetup} \M{logo} \M{key value list}
% \end{declcs}
% Fonts can also be set for a logo or logo component separately,
% see the following list.
% The keys are the same as for \cs{hologoFontSetup}.
%
% \begin{longtable}{>{\ttfamily}l>{\sffamily}ll}
%   \meta{logo} & keys & result\\
%   \hline
%   \endhead
%   BibTeX & bibsf & {\hologoLogoSetup{BibTeX}{variant=sf}\hologo{BibTeX}}\\[.5ex]
%   BibTeX & sc & {\hologoLogoSetup{BibTeX}{variant=sc}\hologo{BibTeX}}\\[.5ex]
%   ExTeX & rm & \hologo{ExTeX}\\
%   SliTeX & rm & \hologo{SliTeX}\\[.5ex]
%   AmS & sy & \hologo{AmS}\\
%   ExTeX & sy & \hologo{ExTeX}\\
%   NTS & sy & \hologo{NTS}\\[.5ex]
%   KOMAScript & sf & \hologo{KOMAScript}\\[.5ex]
%   METAFONT & logo & \hologo{METAFONT}\\
%   METAPOST & logo & \hologo{METAPOST}\\[.5ex]
%   SliTeX & sc \hologo{SliTeX}
% \end{longtable}
%
% \subsubsection{Font order}
%
% For all logos the font \xoption{general} is applied first.
% Example:
%\begin{quote}
%|\hologoFontSetup{general=\color{red}}|
%\end{quote}
% will print red logos.
% Then if the font uses a special font \xoption{sf}, for example,
% the font is applied that is setup by \cs{hologoLogoFontSetup}.
% If this font is not setup, then the common font setup
% by \cs{hologoFontSetup} is used. Otherwise a warning is given,
% that there is no font configured.
%
% \subsection{Additional user macros}
%
% Usually a variant of a logo is configured by using
% \cs{hologoLogoSetup}, because it is bad style to mix
% different variants of the same logo in the same text.
% There the following macros are a convenience for testing.
%
% \begin{declcs}{hologoVariant} \M{name} \M{variant}\\
%   \cs{HologoVariant} \M{name} \M{variant}
% \end{declcs}
% Logo \meta{name} is set using \meta{variant} that specifies
% explicitely which variant of the macro is used. If the argument
% is empty, then the default form of the logo is used
% (configurable by \cs{hologoLogoSetup}).
%
% \cs{HologoVariant} is used if the logo is set in a context
% that needs an uppercase first letter (beginning of a sentence, \dots).
%
% \begin{declcs}{hologoList}\\
%   \cs{hologoEntry} \M{logo} \M{variant} \M{since}
% \end{declcs}
% Macro \cs{hologoList} contains all logos that are provided
% by the package including variants. The list consists of calls
% of \cs{hologoEntry} with three arguments starting with the
% logo name \meta{logo} and its variant \meta{variant}. An empty
% variant means the current default. Argument \meta{since} specifies
% with version of the package \xpackage{hologo} is needed to get
% the logo. If the logo is fixed, then the date gets updated.
% Therefore the date \meta{since} is not exactly the date of
% the first introduction, but rather the date of the latest fix.
%
% Before \cs{hologoList} can be used, macro \cs{hologoEntry} needs
% a definition. The example file in section \ref{sec:example}
% shows applications of \cs{hologoList}.
%
% \subsection{Supported contexts}
%
% Macros \cs{hologo} and friends support special contexts:
% \begin{itemize}
% \item \hologo{LaTeX}'s protection mechanism.
% \item Bookmarks of package \xpackage{hyperref}.
% \item Package \xpackage{tex4ht}.
% \item The macros can be used inside \cs{csname} constructs,
%   if \cs{ifincsname} is available (\hologo{pdfTeX}, \hologo{XeTeX},
%   \hologo{LuaTeX}).
% \end{itemize}
%
% \subsection{Example}
% \label{sec:example}
%
% The following example prints the logos in different fonts.
%    \begin{macrocode}
%<*example>
%<<verbatim
\NeedsTeXFormat{LaTeX2e}
\documentclass[a4paper]{article}
\usepackage[
  hmargin=20mm,
  vmargin=20mm,
]{geometry}
\pagestyle{empty}
\usepackage{hologo}[2016/05/12]
\usepackage{longtable}
\usepackage{array}
\setlength{\extrarowheight}{2pt}
\usepackage[T1]{fontenc}
\usepackage{lmodern}
\usepackage{pdflscape}
\usepackage[
  pdfencoding=auto,
]{hyperref}
\hypersetup{
  pdfauthor={Heiko Oberdiek},
  pdftitle={Example for package `hologo'},
  pdfsubject={Logos with fonts lmr, lmss, qtm, qpl, qhv},
}
\usepackage{bookmark}

% Print the logo list on the console

\begingroup
  \typeout{}%
  \typeout{*** Begin of logo list ***}%
  \newcommand*{\hologoEntry}[3]{%
    \typeout{#1 \ifx\\#2\\\else(#2) \fi[#3]}%
  }%
  \hologoList
  \typeout{*** End of logo list ***}%
  \typeout{}%
\endgroup

\begin{document}
\begin{landscape}

  \section{Example file for package `hologo'}

  % Table for font names

  \begin{longtable}{>{\bfseries}ll}
    \textbf{font} & \textbf{Font name}\\
    \hline
    lmr & Latin Modern Roman\\
    lmss & Latin Modern Sans\\
    qtm & \TeX\ Gyre Termes\\
    qhv & \TeX\ Gyre Heros\\
    qpl & \TeX\ Gyre Pagella\\
  \end{longtable}

  % Logo list with logos in different fonts

  \begingroup
    \newcommand*{\SetVariant}[2]{%
      \ifx\\#2\\%
      \else
        \hologoLogoSetup{#1}{variant=#2}%
      \fi
    }%
    \newcommand*{\hologoEntry}[3]{%
      \SetVariant{#1}{#2}%
      \raisebox{1em}[0pt][0pt]{\hypertarget{#1@#2}{}}%
      \bookmark[%
        dest={#1@#2},%
      ]{%
        #1\ifx\\#2\\\else\space(#2)\fi: \Hologo{#1}, \hologo{#1} %
        [Unicode]%
      }%
      \hypersetup{unicode=false}%
      \bookmark[%
        dest={#1@#2},%
      ]{%
        #1\ifx\\#2\\\else\space(#2)\fi: \Hologo{#1}, \hologo{#1} %
        [PDFDocEncoding]%
      }%
      \texttt{#1}%
      &%
      \texttt{#2}%
      &%
      \Hologo{#1}%
      &%
      \SetVariant{#1}{#2}%
      \hologo{#1}%
      &%
      \SetVariant{#1}{#2}%
      \fontfamily{qtm}\selectfont
      \hologo{#1}%
      &%
      \SetVariant{#1}{#2}%
      \fontfamily{qpl}\selectfont
      \hologo{#1}%
      &%
      \SetVariant{#1}{#2}%
      \textsf{\hologo{#1}}%
      &%
      \SetVariant{#1}{#2}%
      \fontfamily{qhv}\selectfont
      \hologo{#1}%
      \tabularnewline
    }%
    \begin{longtable}{llllllll}%
      \textbf{\textit{logo}} & \textbf{\textit{variant}} &
      \texttt{\string\Hologo} &
      \textbf{lmr} & \textbf{qtm} & \textbf{qpl} &
      \textbf{lmss} & \textbf{qhv}
      \tabularnewline
      \hline
      \endhead
      \hologoList
    \end{longtable}%
  \endgroup

\end{landscape}
\end{document}
%verbatim
%</example>
%    \end{macrocode}
%
% \StopEventually{
% }
%
% \section{Implementation}
%    \begin{macrocode}
%<*package>
%    \end{macrocode}
%    Reload check, especially if the package is not used with \LaTeX.
%    \begin{macrocode}
\begingroup\catcode61\catcode48\catcode32=10\relax%
  \catcode13=5 % ^^M
  \endlinechar=13 %
  \catcode35=6 % #
  \catcode39=12 % '
  \catcode44=12 % ,
  \catcode45=12 % -
  \catcode46=12 % .
  \catcode58=12 % :
  \catcode64=11 % @
  \catcode123=1 % {
  \catcode125=2 % }
  \expandafter\let\expandafter\x\csname ver@hologo.sty\endcsname
  \ifx\x\relax % plain-TeX, first loading
  \else
    \def\empty{}%
    \ifx\x\empty % LaTeX, first loading,
      % variable is initialized, but \ProvidesPackage not yet seen
    \else
      \expandafter\ifx\csname PackageInfo\endcsname\relax
        \def\x#1#2{%
          \immediate\write-1{Package #1 Info: #2.}%
        }%
      \else
        \def\x#1#2{\PackageInfo{#1}{#2, stopped}}%
      \fi
      \x{hologo}{The package is already loaded}%
      \aftergroup\endinput
    \fi
  \fi
\endgroup%
%    \end{macrocode}
%    Package identification:
%    \begin{macrocode}
\begingroup\catcode61\catcode48\catcode32=10\relax%
  \catcode13=5 % ^^M
  \endlinechar=13 %
  \catcode35=6 % #
  \catcode39=12 % '
  \catcode40=12 % (
  \catcode41=12 % )
  \catcode44=12 % ,
  \catcode45=12 % -
  \catcode46=12 % .
  \catcode47=12 % /
  \catcode58=12 % :
  \catcode64=11 % @
  \catcode91=12 % [
  \catcode93=12 % ]
  \catcode123=1 % {
  \catcode125=2 % }
  \expandafter\ifx\csname ProvidesPackage\endcsname\relax
    \def\x#1#2#3[#4]{\endgroup
      \immediate\write-1{Package: #3 #4}%
      \xdef#1{#4}%
    }%
  \else
    \def\x#1#2[#3]{\endgroup
      #2[{#3}]%
      \ifx#1\@undefined
        \xdef#1{#3}%
      \fi
      \ifx#1\relax
        \xdef#1{#3}%
      \fi
    }%
  \fi
\expandafter\x\csname ver@hologo.sty\endcsname
\ProvidesPackage{hologo}%
  [2016/05/12 v1.11 A logo collection with bookmark support (HO)]%
%    \end{macrocode}
%
%    \begin{macrocode}
\begingroup\catcode61\catcode48\catcode32=10\relax%
  \catcode13=5 % ^^M
  \endlinechar=13 %
  \catcode123=1 % {
  \catcode125=2 % }
  \catcode64=11 % @
  \def\x{\endgroup
    \expandafter\edef\csname HOLOGO@AtEnd\endcsname{%
      \endlinechar=\the\endlinechar\relax
      \catcode13=\the\catcode13\relax
      \catcode32=\the\catcode32\relax
      \catcode35=\the\catcode35\relax
      \catcode61=\the\catcode61\relax
      \catcode64=\the\catcode64\relax
      \catcode123=\the\catcode123\relax
      \catcode125=\the\catcode125\relax
    }%
  }%
\x\catcode61\catcode48\catcode32=10\relax%
\catcode13=5 % ^^M
\endlinechar=13 %
\catcode35=6 % #
\catcode64=11 % @
\catcode123=1 % {
\catcode125=2 % }
\def\TMP@EnsureCode#1#2{%
  \edef\HOLOGO@AtEnd{%
    \HOLOGO@AtEnd
    \catcode#1=\the\catcode#1\relax
  }%
  \catcode#1=#2\relax
}
\TMP@EnsureCode{10}{12}% ^^J
\TMP@EnsureCode{33}{12}% !
\TMP@EnsureCode{34}{12}% "
\TMP@EnsureCode{36}{3}% $
\TMP@EnsureCode{38}{4}% &
\TMP@EnsureCode{39}{12}% '
\TMP@EnsureCode{40}{12}% (
\TMP@EnsureCode{41}{12}% )
\TMP@EnsureCode{42}{12}% *
\TMP@EnsureCode{43}{12}% +
\TMP@EnsureCode{44}{12}% ,
\TMP@EnsureCode{45}{12}% -
\TMP@EnsureCode{46}{12}% .
\TMP@EnsureCode{47}{12}% /
\TMP@EnsureCode{58}{12}% :
\TMP@EnsureCode{59}{12}% ;
\TMP@EnsureCode{60}{12}% <
\TMP@EnsureCode{62}{12}% >
\TMP@EnsureCode{63}{12}% ?
\TMP@EnsureCode{91}{12}% [
\TMP@EnsureCode{93}{12}% ]
\TMP@EnsureCode{94}{7}% ^ (superscript)
\TMP@EnsureCode{95}{8}% _ (subscript)
\TMP@EnsureCode{96}{12}% `
\TMP@EnsureCode{124}{12}% |
\edef\HOLOGO@AtEnd{%
  \HOLOGO@AtEnd
  \escapechar\the\escapechar\relax
  \noexpand\endinput
}
\escapechar=92 %
%    \end{macrocode}
%
% \subsection{Logo list}
%
%    \begin{macro}{\hologoList}
%    \begin{macrocode}
\def\hologoList{%
  \hologoEntry{(La)TeX}{}{2011/10/01}%
  \hologoEntry{AmSLaTeX}{}{2010/04/16}%
  \hologoEntry{AmSTeX}{}{2010/04/16}%
  \hologoEntry{biber}{}{2011/10/01}%
  \hologoEntry{BibTeX}{}{2011/10/01}%
  \hologoEntry{BibTeX}{sf}{2011/10/01}%
  \hologoEntry{BibTeX}{sc}{2011/10/01}%
  \hologoEntry{BibTeX8}{}{2011/11/22}%
  \hologoEntry{ConTeXt}{}{2011/03/25}%
  \hologoEntry{ConTeXt}{narrow}{2011/03/25}%
  \hologoEntry{ConTeXt}{simple}{2011/03/25}%
  \hologoEntry{emTeX}{}{2010/04/26}%
  \hologoEntry{eTeX}{}{2010/04/08}%
  \hologoEntry{ExTeX}{}{2011/10/01}%
  \hologoEntry{HanTheThanh}{}{2011/11/29}%
  \hologoEntry{iniTeX}{}{2011/10/01}%
  \hologoEntry{KOMAScript}{}{2011/10/01}%
  \hologoEntry{La}{}{2010/05/08}%
  \hologoEntry{LaTeX}{}{2010/04/08}%
  \hologoEntry{LaTeX2e}{}{2010/04/08}%
  \hologoEntry{LaTeX3}{}{2010/04/24}%
  \hologoEntry{LaTeXe}{}{2010/04/08}%
  \hologoEntry{LaTeXML}{}{2011/11/22}%
  \hologoEntry{LaTeXTeX}{}{2011/10/01}%
  \hologoEntry{LuaLaTeX}{}{2010/04/08}%
  \hologoEntry{LuaTeX}{}{2010/04/08}%
  \hologoEntry{LyX}{}{2011/10/01}%
  \hologoEntry{METAFONT}{}{2011/10/01}%
  \hologoEntry{MetaFun}{}{2011/10/01}%
  \hologoEntry{METAPOST}{}{2011/10/01}%
  \hologoEntry{MetaPost}{}{2011/10/01}%
  \hologoEntry{MiKTeX}{}{2011/10/01}%
  \hologoEntry{NTS}{}{2011/10/01}%
  \hologoEntry{OzMF}{}{2011/10/01}%
  \hologoEntry{OzMP}{}{2011/10/01}%
  \hologoEntry{OzTeX}{}{2011/10/01}%
  \hologoEntry{OzTtH}{}{2011/10/01}%
  \hologoEntry{PCTeX}{}{2011/10/01}%
  \hologoEntry{pdfTeX}{}{2011/10/01}%
  \hologoEntry{pdfLaTeX}{}{2011/10/01}%
  \hologoEntry{PiC}{}{2011/10/01}%
  \hologoEntry{PiCTeX}{}{2011/10/01}%
  \hologoEntry{plainTeX}{}{2010/04/08}%
  \hologoEntry{plainTeX}{space}{2010/04/16}%
  \hologoEntry{plainTeX}{hyphen}{2010/04/16}%
  \hologoEntry{plainTeX}{runtogether}{2010/04/16}%
  \hologoEntry{SageTeX}{}{2011/11/22}%
  \hologoEntry{SLiTeX}{}{2011/10/01}%
  \hologoEntry{SLiTeX}{lift}{2011/10/01}%
  \hologoEntry{SLiTeX}{narrow}{2011/10/01}%
  \hologoEntry{SLiTeX}{simple}{2011/10/01}%
  \hologoEntry{SliTeX}{}{2011/10/01}%
  \hologoEntry{SliTeX}{narrow}{2011/10/01}%
  \hologoEntry{SliTeX}{simple}{2011/10/01}%
  \hologoEntry{SliTeX}{lift}{2011/10/01}%
  \hologoEntry{teTeX}{}{2011/10/01}%
  \hologoEntry{TeX}{}{2010/04/08}%
  \hologoEntry{TeX4ht}{}{2011/11/22}%
  \hologoEntry{TTH}{}{2011/11/22}%
  \hologoEntry{virTeX}{}{2011/10/01}%
  \hologoEntry{VTeX}{}{2010/04/24}%
  \hologoEntry{Xe}{}{2010/04/08}%
  \hologoEntry{XeLaTeX}{}{2010/04/08}%
  \hologoEntry{XeTeX}{}{2010/04/08}%
}
%    \end{macrocode}
%    \end{macro}
%
% \subsection{Load resources}
%
%    \begin{macrocode}
\begingroup\expandafter\expandafter\expandafter\endgroup
\expandafter\ifx\csname RequirePackage\endcsname\relax
  \def\TMP@RequirePackage#1[#2]{%
    \begingroup\expandafter\expandafter\expandafter\endgroup
    \expandafter\ifx\csname ver@#1.sty\endcsname\relax
      \input #1.sty\relax
    \fi
  }%
  \TMP@RequirePackage{ltxcmds}[2011/02/04]%
  \TMP@RequirePackage{infwarerr}[2010/04/08]%
  \TMP@RequirePackage{kvsetkeys}[2010/03/01]%
  \TMP@RequirePackage{kvdefinekeys}[2010/03/01]%
  \TMP@RequirePackage{pdftexcmds}[2010/04/01]%
  \TMP@RequirePackage{ifpdf}[2010/01/28]%
  \TMP@RequirePackage{ifluatex}[2010/03/01]%
  \ltx@IfUndefined{newif}{%
    \expandafter\let\csname newif\endcsname\ltx@newif
  }{}%
  \TMP@RequirePackage{ifxetex}[2009/01/23]%
  \TMP@RequirePackage{ifvtex}[2010/03/01]%
\else
  \RequirePackage{ltxcmds}[2011/02/04]%
  \RequirePackage{infwarerr}[2010/04/08]%
  \RequirePackage{kvsetkeys}[2010/03/01]%
  \RequirePackage{kvdefinekeys}[2010/03/01]%
  \RequirePackage{pdftexcmds}[2010/04/01]%
  \RequirePackage{ifpdf}[2010/01/28]%
  \RequirePackage{ifluatex}[2010/03/01]%
  \RequirePackage{ifxetex}[2009/01/23]%
  \RequirePackage{ifvtex}[2010/03/01]%
\fi
%    \end{macrocode}
%
%    \begin{macro}{\HOLOGO@IfDefined}
%    \begin{macrocode}
\def\HOLOGO@IfExists#1{%
  \ifx\@undefined#1%
    \expandafter\ltx@secondoftwo
  \else
    \ifx\relax#1%
      \expandafter\ltx@secondoftwo
    \else
      \expandafter\expandafter\expandafter\ltx@firstoftwo
    \fi
  \fi
}
%    \end{macrocode}
%    \end{macro}
%
% \subsection{Setup macros}
%
%    \begin{macro}{\hologoSetup}
%    \begin{macrocode}
\def\hologoSetup{%
  \let\HOLOGO@name\relax
  \HOLOGO@Setup
}
%    \end{macrocode}
%    \end{macro}
%
%    \begin{macro}{\hologoLogoSetup}
%    \begin{macrocode}
\def\hologoLogoSetup#1{%
  \edef\HOLOGO@name{#1}%
  \ltx@IfUndefined{HoLogo@\HOLOGO@name}{%
    \@PackageError{hologo}{%
      Unknown logo `\HOLOGO@name'%
    }\@ehc
    \ltx@gobble
  }{%
    \HOLOGO@Setup
  }%
}
%    \end{macrocode}
%    \end{macro}
%
%    \begin{macro}{\HOLOGO@Setup}
%    \begin{macrocode}
\def\HOLOGO@Setup{%
  \kvsetkeys{HoLogo}%
}
%    \end{macrocode}
%    \end{macro}
%
% \subsection{Options}
%
%    \begin{macro}{\HOLOGO@DeclareBoolOption}
%    \begin{macrocode}
\def\HOLOGO@DeclareBoolOption#1{%
  \expandafter\chardef\csname HOLOGOOPT@#1\endcsname\ltx@zero
  \kv@define@key{HoLogo}{#1}[true]{%
    \def\HOLOGO@temp{##1}%
    \ifx\HOLOGO@temp\HOLOGO@true
      \ifx\HOLOGO@name\relax
        \expandafter\chardef\csname HOLOGOOPT@#1\endcsname=\ltx@one
      \else
        \expandafter\chardef\csname
        HoLogoOpt@#1@\HOLOGO@name\endcsname\ltx@one
      \fi
      \HOLOGO@SetBreakAll{#1}%
    \else
      \ifx\HOLOGO@temp\HOLOGO@false
        \ifx\HOLOGO@name\relax
          \expandafter\chardef\csname HOLOGOOPT@#1\endcsname=\ltx@zero
        \else
          \expandafter\chardef\csname
          HoLogoOpt@#1@\HOLOGO@name\endcsname=\ltx@zero
        \fi
        \HOLOGO@SetBreakAll{#1}%
      \else
        \@PackageError{hologo}{%
          Unknown value `##1' for boolean option `#1'.\MessageBreak
          Known values are `true' and `false'%
        }\@ehc
      \fi
    \fi
  }%
}
%    \end{macrocode}
%    \end{macro}
%
%    \begin{macro}{\HOLOGO@SetBreakAll}
%    \begin{macrocode}
\def\HOLOGO@SetBreakAll#1{%
  \def\HOLOGO@temp{#1}%
  \ifx\HOLOGO@temp\HOLOGO@break
    \ifx\HOLOGO@name\relax
      \chardef\HOLOGOOPT@hyphenbreak=\HOLOGOOPT@break
      \chardef\HOLOGOOPT@spacebreak=\HOLOGOOPT@break
      \chardef\HOLOGOOPT@discretionarybreak=\HOLOGOOPT@break
    \else
      \expandafter\chardef
         \csname HoLogoOpt@hyphenbreak@\HOLOGO@name\endcsname=%
         \csname HoLogoOpt@break@\HOLOGO@name\endcsname
      \expandafter\chardef
         \csname HoLogoOpt@spacebreak@\HOLOGO@name\endcsname=%
         \csname HoLogoOpt@break@\HOLOGO@name\endcsname
      \expandafter\chardef
         \csname HoLogoOpt@discretionarybreak@\HOLOGO@name
             \endcsname=%
         \csname HoLogoOpt@break@\HOLOGO@name\endcsname
    \fi
  \fi
}
%    \end{macrocode}
%    \end{macro}
%
%    \begin{macro}{\HOLOGO@true}
%    \begin{macrocode}
\def\HOLOGO@true{true}
%    \end{macrocode}
%    \end{macro}
%    \begin{macro}{\HOLOGO@false}
%    \begin{macrocode}
\def\HOLOGO@false{false}
%    \end{macrocode}
%    \end{macro}
%    \begin{macro}{\HOLOGO@break}
%    \begin{macrocode}
\def\HOLOGO@break{break}
%    \end{macrocode}
%    \end{macro}
%
%    \begin{macrocode}
\HOLOGO@DeclareBoolOption{break}
\HOLOGO@DeclareBoolOption{hyphenbreak}
\HOLOGO@DeclareBoolOption{spacebreak}
\HOLOGO@DeclareBoolOption{discretionarybreak}
%    \end{macrocode}
%
%    \begin{macrocode}
\kv@define@key{HoLogo}{variant}{%
  \ifx\HOLOGO@name\relax
    \@PackageError{hologo}{%
      Option `variant' is not available in \string\hologoSetup,%
      \MessageBreak
      Use \string\hologoLogoSetup\space instead%
    }\@ehc
  \else
    \edef\HOLOGO@temp{#1}%
    \ifx\HOLOGO@temp\ltx@empty
      \expandafter
      \let\csname HoLogoOpt@variant@\HOLOGO@name\endcsname\@undefined
    \else
      \ltx@IfUndefined{HoLogo@\HOLOGO@name @\HOLOGO@temp}{%
        \@PackageError{hologo}{%
          Unknown variant `\HOLOGO@temp' of logo `\HOLOGO@name'%
        }\@ehc
      }{%
        \expandafter
        \let\csname HoLogoOpt@variant@\HOLOGO@name\endcsname
            \HOLOGO@temp
      }%
    \fi
  \fi
}
%    \end{macrocode}
%
%    \begin{macro}{\HOLOGO@Variant}
%    \begin{macrocode}
\def\HOLOGO@Variant#1{%
  #1%
  \ltx@ifundefined{HoLogoOpt@variant@#1}{%
  }{%
    @\csname HoLogoOpt@variant@#1\endcsname
  }%
}
%    \end{macrocode}
%    \end{macro}
%
% \subsection{Break/no-break support}
%
%    \begin{macro}{\HOLOGO@space}
%    \begin{macrocode}
\def\HOLOGO@space{%
  \ltx@ifundefined{HoLogoOpt@spacebreak@\HOLOGO@name}{%
    \ltx@ifundefined{HoLogoOpt@break@\HOLOGO@name}{%
      \chardef\HOLOGO@temp=\HOLOGOOPT@spacebreak
    }{%
      \chardef\HOLOGO@temp=%
        \csname HoLogoOpt@break@\HOLOGO@name\endcsname
    }%
  }{%
    \chardef\HOLOGO@temp=%
      \csname HoLogoOpt@spacebreak@\HOLOGO@name\endcsname
  }%
  \ifcase\HOLOGO@temp
    \penalty10000 %
  \fi
  \ltx@space
}
%    \end{macrocode}
%    \end{macro}
%
%    \begin{macro}{\HOLOGO@hyphen}
%    \begin{macrocode}
\def\HOLOGO@hyphen{%
  \ltx@ifundefined{HoLogoOpt@hyphenbreak@\HOLOGO@name}{%
    \ltx@ifundefined{HoLogoOpt@break@\HOLOGO@name}{%
      \chardef\HOLOGO@temp=\HOLOGOOPT@hyphenbreak
    }{%
      \chardef\HOLOGO@temp=%
        \csname HoLogoOpt@break@\HOLOGO@name\endcsname
    }%
  }{%
    \chardef\HOLOGO@temp=%
      \csname HoLogoOpt@hyphenbreak@\HOLOGO@name\endcsname
  }%
  \ifcase\HOLOGO@temp
    \ltx@mbox{-}%
  \else
    -%
  \fi
}
%    \end{macrocode}
%    \end{macro}
%
%    \begin{macro}{\HOLOGO@discretionary}
%    \begin{macrocode}
\def\HOLOGO@discretionary{%
  \ltx@ifundefined{HoLogoOpt@discretionarybreak@\HOLOGO@name}{%
    \ltx@ifundefined{HoLogoOpt@break@\HOLOGO@name}{%
      \chardef\HOLOGO@temp=\HOLOGOOPT@discretionarybreak
    }{%
      \chardef\HOLOGO@temp=%
        \csname HoLogoOpt@break@\HOLOGO@name\endcsname
    }%
  }{%
    \chardef\HOLOGO@temp=%
      \csname HoLogoOpt@discretionarybreak@\HOLOGO@name\endcsname
  }%
  \ifcase\HOLOGO@temp
  \else
    \-%
  \fi
}
%    \end{macrocode}
%    \end{macro}
%
%    \begin{macro}{\HOLOGO@mbox}
%    \begin{macrocode}
\def\HOLOGO@mbox#1{%
  \ltx@ifundefined{HoLogoOpt@break@\HOLOGO@name}{%
    \chardef\HOLOGO@temp=\HOLOGOOPT@hyphenbreak
  }{%
    \chardef\HOLOGO@temp=%
      \csname HoLogoOpt@break@\HOLOGO@name\endcsname
  }%
  \ifcase\HOLOGO@temp
    \ltx@mbox{#1}%
  \else
    #1%
  \fi
}
%    \end{macrocode}
%    \end{macro}
%
% \subsection{Font support}
%
%    \begin{macro}{\HoLogoFont@font}
%    \begin{tabular}{@{}ll@{}}
%    |#1|:& logo name\\
%    |#2|:& font short name\\
%    |#3|:& text
%    \end{tabular}
%    \begin{macrocode}
\def\HoLogoFont@font#1#2#3{%
  \begingroup
    \ltx@IfUndefined{HoLogoFont@logo@#1.#2}{%
      \ltx@IfUndefined{HoLogoFont@font@#2}{%
        \@PackageWarning{hologo}{%
          Missing font `#2' for logo `#1'%
        }%
        #3%
      }{%
        \csname HoLogoFont@font@#2\endcsname{#3}%
      }%
    }{%
      \csname HoLogoFont@logo@#1.#2\endcsname{#3}%
    }%
  \endgroup
}
%    \end{macrocode}
%    \end{macro}
%
%    \begin{macro}{\HoLogoFont@Def}
%    \begin{macrocode}
\def\HoLogoFont@Def#1{%
  \expandafter\def\csname HoLogoFont@font@#1\endcsname
}
%    \end{macrocode}
%    \end{macro}
%    \begin{macro}{\HoLogoFont@LogoDef}
%    \begin{macrocode}
\def\HoLogoFont@LogoDef#1#2{%
  \expandafter\def\csname HoLogoFont@logo@#1.#2\endcsname
}
%    \end{macrocode}
%    \end{macro}
%
% \subsubsection{Font defaults}
%
%    \begin{macro}{\HoLogoFont@font@general}
%    \begin{macrocode}
\HoLogoFont@Def{general}{}%
%    \end{macrocode}
%    \end{macro}
%
%    \begin{macro}{\HoLogoFont@font@rm}
%    \begin{macrocode}
\ltx@IfUndefined{rmfamily}{%
  \ltx@IfUndefined{rm}{%
  }{%
    \HoLogoFont@Def{rm}{\rm}%
  }%
}{%
  \HoLogoFont@Def{rm}{\rmfamily}%
}
%    \end{macrocode}
%    \end{macro}
%
%    \begin{macro}{\HoLogoFont@font@sf}
%    \begin{macrocode}
\ltx@IfUndefined{sffamily}{%
  \ltx@IfUndefined{sf}{%
  }{%
    \HoLogoFont@Def{sf}{\sf}%
  }%
}{%
  \HoLogoFont@Def{sf}{\sffamily}%
}
%    \end{macrocode}
%    \end{macro}
%
%    \begin{macro}{\HoLogoFont@font@bibsf}
%    In case of \hologo{plainTeX} the original small caps
%    variant is used as default. In \hologo{LaTeX}
%    the definition of package \xpackage{dtklogos} \cite{dtklogos}
%    is used.
%\begin{quote}
%\begin{verbatim}
%\DeclareRobustCommand{\BibTeX}{%
%  B%
%  \kern-.05em%
%  \hbox{%
%    $\m@th$% %% force math size calculations
%    \csname S@\f@size\endcsname
%    \fontsize\sf@size\z@
%    \math@fontsfalse
%    \selectfont
%    I%
%    \kern-.025em%
%    B
%  }%
%  \kern-.08em%
%  \-%
%  \TeX
%}
%\end{verbatim}
%\end{quote}
%    \begin{macrocode}
\ltx@IfUndefined{selectfont}{%
  \ltx@IfUndefined{tensc}{%
    \font\tensc=cmcsc10\relax
  }{}%
  \HoLogoFont@Def{bibsf}{\tensc}%
}{%
  \HoLogoFont@Def{bibsf}{%
    $\mathsurround=0pt$%
    \csname S@\f@size\endcsname
    \fontsize\sf@size{0pt}%
    \math@fontsfalse
    \selectfont
  }%
}
%    \end{macrocode}
%    \end{macro}
%
%    \begin{macro}{\HoLogoFont@font@sc}
%    \begin{macrocode}
\ltx@IfUndefined{scshape}{%
  \ltx@IfUndefined{tensc}{%
    \font\tensc=cmcsc10\relax
  }{}%
  \HoLogoFont@Def{sc}{\tensc}%
}{%
  \HoLogoFont@Def{sc}{\scshape}%
}
%    \end{macrocode}
%    \end{macro}
%
%    \begin{macro}{\HoLogoFont@font@sy}
%    \begin{macrocode}
\ltx@IfUndefined{usefont}{%
  \ltx@IfUndefined{tensy}{%
  }{%
    \HoLogoFont@Def{sy}{\tensy}%
  }%
}{%
  \HoLogoFont@Def{sy}{%
    \usefont{OMS}{cmsy}{m}{n}%
  }%
}
%    \end{macrocode}
%    \end{macro}
%
%    \begin{macro}{\HoLogoFont@font@logo}
%    \begin{macrocode}
\begingroup
  \def\x{LaTeX2e}%
\expandafter\endgroup
\ifx\fmtname\x
  \ltx@IfUndefined{logofamily}{%
    \DeclareRobustCommand\logofamily{%
      \not@math@alphabet\logofamily\relax
      \fontencoding{U}%
      \fontfamily{logo}%
      \selectfont
    }%
  }{}%
  \ltx@IfUndefined{logofamily}{%
  }{%
    \HoLogoFont@Def{logo}{\logofamily}%
  }%
\else
  \ltx@IfUndefined{tenlogo}{%
    \font\tenlogo=logo10\relax
  }{}%
  \HoLogoFont@Def{logo}{\tenlogo}%
\fi
%    \end{macrocode}
%    \end{macro}
%
% \subsubsection{Font setup}
%
%    \begin{macro}{\hologoFontSetup}
%    \begin{macrocode}
\def\hologoFontSetup{%
  \let\HOLOGO@name\relax
  \HOLOGO@FontSetup
}
%    \end{macrocode}
%    \end{macro}
%
%    \begin{macro}{\hologoLogoFontSetup}
%    \begin{macrocode}
\def\hologoLogoFontSetup#1{%
  \edef\HOLOGO@name{#1}%
  \ltx@IfUndefined{HoLogo@\HOLOGO@name}{%
    \@PackageError{hologo}{%
      Unknown logo `\HOLOGO@name'%
    }\@ehc
    \ltx@gobble
  }{%
    \HOLOGO@FontSetup
  }%
}
%    \end{macrocode}
%    \end{macro}
%
%    \begin{macro}{\HOLOGO@FontSetup}
%    \begin{macrocode}
\def\HOLOGO@FontSetup{%
  \kvsetkeys{HoLogoFont}%
}
%    \end{macrocode}
%    \end{macro}
%
%    \begin{macrocode}
\def\HOLOGO@temp#1{%
  \kv@define@key{HoLogoFont}{#1}{%
    \ifx\HOLOGO@name\relax
      \HoLogoFont@Def{#1}{##1}%
    \else
      \HoLogoFont@LogoDef\HOLOGO@name{#1}{##1}%
    \fi
  }%
}
\HOLOGO@temp{general}
\HOLOGO@temp{sf}
%    \end{macrocode}
%
% \subsection{Generic logo commands}
%
%    \begin{macrocode}
\HOLOGO@IfExists\hologo{%
  \@PackageError{hologo}{%
    \string\hologo\ltx@space is already defined.\MessageBreak
    Package loading is aborted%
  }\@ehc
  \HOLOGO@AtEnd
}%
\HOLOGO@IfExists\hologoRobust{%
  \@PackageError{hologo}{%
    \string\hologoRobust\ltx@space is already defined.\MessageBreak
    Package loading is aborted%
  }\@ehc
  \HOLOGO@AtEnd
}%
%    \end{macrocode}
%
% \subsubsection{\cs{hologo} and friends}
%
%    \begin{macrocode}
\ifluatex
  \expandafter\ltx@firstofone
\else
  \expandafter\ltx@gobble
\fi
{%
  \ltx@IfUndefined{ifincsname}{%
    \ifnum\luatexversion<36 %
      \expandafter\ltx@gobble
    \else
      \expandafter\ltx@firstofone
    \fi
    {%
      \begingroup
        \ifcase0%
            \directlua{%
              if tex.enableprimitives then %
                tex.enableprimitives('HOLOGO@', {'ifincsname'})%
              else %
                tex.print('1')%
              end%
            }%
            \ifx\HOLOGO@ifincsname\@undefined 1\fi%
            \relax
          \expandafter\ltx@firstofone
        \else
          \endgroup
          \expandafter\ltx@gobble
        \fi
        {%
          \global\let\ifincsname\HOLOGO@ifincsname
        }%
      \HOLOGO@temp
    }%
  }{}%
}
%    \end{macrocode}
%    \begin{macrocode}
\ltx@IfUndefined{ifincsname}{%
  \catcode`$=14 %
}{%
  \catcode`$=9 %
}
%    \end{macrocode}
%
%    \begin{macro}{\hologo}
%    \begin{macrocode}
\def\hologo#1{%
$ \ifincsname
$   \ltx@ifundefined{HoLogoCs@\HOLOGO@Variant{#1}}{%
$     #1%
$   }{%
$     \csname HoLogoCs@\HOLOGO@Variant{#1}\endcsname\ltx@firstoftwo
$   }%
$ \else
    \HOLOGO@IfExists\texorpdfstring\texorpdfstring\ltx@firstoftwo
    {%
      \hologoRobust{#1}%
    }{%
      \ltx@ifundefined{HoLogoBkm@\HOLOGO@Variant{#1}}{%
        \ltx@ifundefined{HoLogo@#1}{?#1?}{#1}%
      }{%
        \csname HoLogoBkm@\HOLOGO@Variant{#1}\endcsname
        \ltx@firstoftwo
      }%
    }%
$ \fi
}
%    \end{macrocode}
%    \end{macro}
%    \begin{macro}{\Hologo}
%    \begin{macrocode}
\def\Hologo#1{%
$ \ifincsname
$   \ltx@ifundefined{HoLogoCs@\HOLOGO@Variant{#1}}{%
$     #1%
$   }{%
$     \csname HoLogoCs@\HOLOGO@Variant{#1}\endcsname\ltx@secondoftwo
$   }%
$ \else
    \HOLOGO@IfExists\texorpdfstring\texorpdfstring\ltx@firstoftwo
    {%
      \HologoRobust{#1}%
    }{%
      \ltx@ifundefined{HoLogoBkm@\HOLOGO@Variant{#1}}{%
        \ltx@ifundefined{HoLogo@#1}{?#1?}{#1}%
      }{%
        \csname HoLogoBkm@\HOLOGO@Variant{#1}\endcsname
        \ltx@secondoftwo
      }%
    }%
$ \fi
}
%    \end{macrocode}
%    \end{macro}
%
%    \begin{macro}{\hologoVariant}
%    \begin{macrocode}
\def\hologoVariant#1#2{%
  \ifx\relax#2\relax
    \hologo{#1}%
  \else
$   \ifincsname
$     \ltx@ifundefined{HoLogoCs@#1@#2}{%
$       #1%
$     }{%
$       \csname HoLogoCs@#1@#2\endcsname\ltx@firstoftwo
$     }%
$   \else
      \HOLOGO@IfExists\texorpdfstring\texorpdfstring\ltx@firstoftwo
      {%
        \hologoVariantRobust{#1}{#2}%
      }{%
        \ltx@ifundefined{HoLogoBkm@#1@#2}{%
          \ltx@ifundefined{HoLogo@#1}{?#1?}{#1}%
        }{%
          \csname HoLogoBkm@#1@#2\endcsname
          \ltx@firstoftwo
        }%
      }%
$   \fi
  \fi
}
%    \end{macrocode}
%    \end{macro}
%    \begin{macro}{\HologoVariant}
%    \begin{macrocode}
\def\HologoVariant#1#2{%
  \ifx\relax#2\relax
    \Hologo{#1}%
  \else
$   \ifincsname
$     \ltx@ifundefined{HoLogoCs@#1@#2}{%
$       #1%
$     }{%
$       \csname HoLogoCs@#1@#2\endcsname\ltx@secondoftwo
$     }%
$   \else
      \HOLOGO@IfExists\texorpdfstring\texorpdfstring\ltx@firstoftwo
      {%
        \HologoVariantRobust{#1}{#2}%
      }{%
        \ltx@ifundefined{HoLogoBkm@#1@#2}{%
          \ltx@ifundefined{HoLogo@#1}{?#1?}{#1}%
        }{%
          \csname HoLogoBkm@#1@#2\endcsname
          \ltx@secondoftwo
        }%
      }%
$   \fi
  \fi
}
%    \end{macrocode}
%    \end{macro}
%
%    \begin{macrocode}
\catcode`\$=3 %
%    \end{macrocode}
%
% \subsubsection{\cs{hologoRobust} and friends}
%
%    \begin{macro}{\hologoRobust}
%    \begin{macrocode}
\ltx@IfUndefined{protected}{%
  \ltx@IfUndefined{DeclareRobustCommand}{%
    \def\hologoRobust#1%
  }{%
    \DeclareRobustCommand*\hologoRobust[1]%
  }%
}{%
  \protected\def\hologoRobust#1%
}%
{%
  \edef\HOLOGO@name{#1}%
  \ltx@IfUndefined{HoLogo@\HOLOGO@Variant\HOLOGO@name}{%
    \@PackageError{hologo}{%
      Unknown logo `\HOLOGO@name'%
    }\@ehc
    ?\HOLOGO@name?%
  }{%
    \ltx@IfUndefined{ver@tex4ht.sty}{%
      \HoLogoFont@font\HOLOGO@name{general}{%
        \csname HoLogo@\HOLOGO@Variant\HOLOGO@name\endcsname
        \ltx@firstoftwo
      }%
    }{%
      \ltx@IfUndefined{HoLogoHtml@\HOLOGO@Variant\HOLOGO@name}{%
        \HOLOGO@name
      }{%
        \csname HoLogoHtml@\HOLOGO@Variant\HOLOGO@name\endcsname
        \ltx@firstoftwo
      }%
    }%
  }%
}
%    \end{macrocode}
%    \end{macro}
%    \begin{macro}{\HologoRobust}
%    \begin{macrocode}
\ltx@IfUndefined{protected}{%
  \ltx@IfUndefined{DeclareRobustCommand}{%
    \def\HologoRobust#1%
  }{%
    \DeclareRobustCommand*\HologoRobust[1]%
  }%
}{%
  \protected\def\HologoRobust#1%
}%
{%
  \edef\HOLOGO@name{#1}%
  \ltx@IfUndefined{HoLogo@\HOLOGO@Variant\HOLOGO@name}{%
    \@PackageError{hologo}{%
      Unknown logo `\HOLOGO@name'%
    }\@ehc
    ?\HOLOGO@name?%
  }{%
    \ltx@IfUndefined{ver@tex4ht.sty}{%
      \HoLogoFont@font\HOLOGO@name{general}{%
        \csname HoLogo@\HOLOGO@Variant\HOLOGO@name\endcsname
        \ltx@secondoftwo
      }%
    }{%
      \ltx@IfUndefined{HoLogoHtml@\HOLOGO@Variant\HOLOGO@name}{%
        \expandafter\HOLOGO@Uppercase\HOLOGO@name
      }{%
        \csname HoLogoHtml@\HOLOGO@Variant\HOLOGO@name\endcsname
        \ltx@secondoftwo
      }%
    }%
  }%
}
%    \end{macrocode}
%    \end{macro}
%    \begin{macro}{\hologoVariantRobust}
%    \begin{macrocode}
\ltx@IfUndefined{protected}{%
  \ltx@IfUndefined{DeclareRobustCommand}{%
    \def\hologoVariantRobust#1#2%
  }{%
    \DeclareRobustCommand*\hologoVariantRobust[2]%
  }%
}{%
  \protected\def\hologoVariantRobust#1#2%
}%
{%
  \begingroup
    \hologoLogoSetup{#1}{variant={#2}}%
    \hologoRobust{#1}%
  \endgroup
}
%    \end{macrocode}
%    \end{macro}
%    \begin{macro}{\HologoVariantRobust}
%    \begin{macrocode}
\ltx@IfUndefined{protected}{%
  \ltx@IfUndefined{DeclareRobustCommand}{%
    \def\HologoVariantRobust#1#2%
  }{%
    \DeclareRobustCommand*\HologoVariantRobust[2]%
  }%
}{%
  \protected\def\HologoVariantRobust#1#2%
}%
{%
  \begingroup
    \hologoLogoSetup{#1}{variant={#2}}%
    \HologoRobust{#1}%
  \endgroup
}
%    \end{macrocode}
%    \end{macro}
%
%    \begin{macro}{\hologorobust}
%    Macro \cs{hologorobust} is only defined for compatibility.
%    Its use is deprecated.
%    \begin{macrocode}
\def\hologorobust{\hologoRobust}
%    \end{macrocode}
%    \end{macro}
%
% \subsection{Helpers}
%
%    \begin{macro}{\HOLOGO@Uppercase}
%    Macro \cs{HOLOGO@Uppercase} is restricted to \cs{uppercase},
%    because \hologo{plainTeX} or \hologo{iniTeX} do not provide
%    \cs{MakeUppercase}.
%    \begin{macrocode}
\def\HOLOGO@Uppercase#1{\uppercase{#1}}
%    \end{macrocode}
%    \end{macro}
%
%    \begin{macro}{\HOLOGO@PdfdocUnicode}
%    \begin{macrocode}
\def\HOLOGO@PdfdocUnicode{%
  \ifx\ifHy@unicode\iftrue
    \expandafter\ltx@secondoftwo
  \else
    \expandafter\ltx@firstoftwo
  \fi
}
%    \end{macrocode}
%    \end{macro}
%
%    \begin{macro}{\HOLOGO@Math}
%    \begin{macrocode}
\def\HOLOGO@MathSetup{%
  \mathsurround0pt\relax
  \HOLOGO@IfExists\f@series{%
    \if b\expandafter\ltx@car\f@series x\@nil
      \csname boldmath\endcsname
   \fi
  }{}%
}
%    \end{macrocode}
%    \end{macro}
%
%    \begin{macro}{\HOLOGO@TempDimen}
%    \begin{macrocode}
\dimendef\HOLOGO@TempDimen=\ltx@zero
%    \end{macrocode}
%    \end{macro}
%    \begin{macro}{\HOLOGO@NegativeKerning}
%    \begin{macrocode}
\def\HOLOGO@NegativeKerning#1{%
  \begingroup
    \HOLOGO@TempDimen=0pt\relax
    \comma@parse@normalized{#1}{%
      \ifdim\HOLOGO@TempDimen=0pt %
        \expandafter\HOLOGO@@NegativeKerning\comma@entry
      \fi
      \ltx@gobble
    }%
    \ifdim\HOLOGO@TempDimen<0pt %
      \kern\HOLOGO@TempDimen
    \fi
  \endgroup
}
%    \end{macrocode}
%    \end{macro}
%    \begin{macro}{\HOLOGO@@NegativeKerning}
%    \begin{macrocode}
\def\HOLOGO@@NegativeKerning#1#2{%
  \setbox\ltx@zero\hbox{#1#2}%
  \HOLOGO@TempDimen=\wd\ltx@zero
  \setbox\ltx@zero\hbox{#1\kern0pt#2}%
  \advance\HOLOGO@TempDimen by -\wd\ltx@zero
}
%    \end{macrocode}
%    \end{macro}
%
%    \begin{macro}{\HOLOGO@SpaceFactor}
%    \begin{macrocode}
\def\HOLOGO@SpaceFactor{%
  \spacefactor1000 %
}
%    \end{macrocode}
%    \end{macro}
%
%    \begin{macro}{\HOLOGO@Span}
%    \begin{macrocode}
\def\HOLOGO@Span#1#2{%
  \HCode{<span class="HoLogo-#1">}%
  #2%
  \HCode{</span>}%
}
%    \end{macrocode}
%    \end{macro}
%
% \subsubsection{Text subscript}
%
%    \begin{macro}{\HOLOGO@SubScript}%
%    \begin{macrocode}
\def\HOLOGO@SubScript#1{%
  \ltx@IfUndefined{textsubscript}{%
    \ltx@IfUndefined{text}{%
      \ltx@mbox{%
        \mathsurround=0pt\relax
        $%
          _{%
            \ltx@IfUndefined{sf@size}{%
              \mathrm{#1}%
            }{%
              \mbox{%
                \fontsize\sf@size{0pt}\selectfont
                #1%
              }%
            }%
          }%
        $%
      }%
    }{%
      \ltx@mbox{%
        \mathsurround=0pt\relax
        $_{\text{#1}}$%
      }%
    }%
  }{%
    \textsubscript{#1}%
  }%
}
%    \end{macrocode}
%    \end{macro}
%
% \subsection{\hologo{TeX} and friends}
%
% \subsubsection{\hologo{TeX}}
%
%    \begin{macro}{\HoLogo@TeX}
%    Source: \hologo{LaTeX} kernel.
%    \begin{macrocode}
\def\HoLogo@TeX#1{%
  T\kern-.1667em\lower.5ex\hbox{E}\kern-.125emX\HOLOGO@SpaceFactor
}
%    \end{macrocode}
%    \end{macro}
%    \begin{macro}{\HoLogoHtml@TeX}
%    \begin{macrocode}
\def\HoLogoHtml@TeX#1{%
  \HoLogoCss@TeX
  \HOLOGO@Span{TeX}{%
    T%
    \HOLOGO@Span{e}{%
      E%
    }%
    X%
  }%
}
%    \end{macrocode}
%    \end{macro}
%    \begin{macro}{\HoLogoCss@TeX}
%    \begin{macrocode}
\def\HoLogoCss@TeX{%
  \Css{%
    span.HoLogo-TeX span.HoLogo-e{%
      position:relative;%
      top:.5ex;%
      margin-left:-.1667em;%
      margin-right:-.125em;%
    }%
  }%
  \Css{%
    a span.HoLogo-TeX span.HoLogo-e{%
      text-decoration:none;%
    }%
  }%
  \global\let\HoLogoCss@TeX\relax
}
%    \end{macrocode}
%    \end{macro}
%
% \subsubsection{\hologo{plainTeX}}
%
%    \begin{macro}{\HoLogo@plainTeX@space}
%    Source: ``The \hologo{TeX}book''
%    \begin{macrocode}
\def\HoLogo@plainTeX@space#1{%
  \HOLOGO@mbox{#1{p}{P}lain}\HOLOGO@space\hologo{TeX}%
}
%    \end{macrocode}
%    \end{macro}
%    \begin{macro}{\HoLogoCs@plainTeX@space}
%    \begin{macrocode}
\def\HoLogoCs@plainTeX@space#1{#1{p}{P}lain TeX}%
%    \end{macrocode}
%    \end{macro}
%    \begin{macro}{\HoLogoBkm@plainTeX@space}
%    \begin{macrocode}
\def\HoLogoBkm@plainTeX@space#1{%
  #1{p}{P}lain \hologo{TeX}%
}
%    \end{macrocode}
%    \end{macro}
%    \begin{macro}{\HoLogoHtml@plainTeX@space}
%    \begin{macrocode}
\def\HoLogoHtml@plainTeX@space#1{%
  #1{p}{P}lain \hologo{TeX}%
}
%    \end{macrocode}
%    \end{macro}
%
%    \begin{macro}{\HoLogo@plainTeX@hyphen}
%    \begin{macrocode}
\def\HoLogo@plainTeX@hyphen#1{%
  \HOLOGO@mbox{#1{p}{P}lain}\HOLOGO@hyphen\hologo{TeX}%
}
%    \end{macrocode}
%    \end{macro}
%    \begin{macro}{\HoLogoCs@plainTeX@hyphen}
%    \begin{macrocode}
\def\HoLogoCs@plainTeX@hyphen#1{#1{p}{P}lain-TeX}
%    \end{macrocode}
%    \end{macro}
%    \begin{macro}{\HoLogoBkm@plainTeX@hyphen}
%    \begin{macrocode}
\def\HoLogoBkm@plainTeX@hyphen#1{%
  #1{p}{P}lain-\hologo{TeX}%
}
%    \end{macrocode}
%    \end{macro}
%    \begin{macro}{\HoLogoHtml@plainTeX@hyphen}
%    \begin{macrocode}
\def\HoLogoHtml@plainTeX@hyphen#1{%
  #1{p}{P}lain-\hologo{TeX}%
}
%    \end{macrocode}
%    \end{macro}
%
%    \begin{macro}{\HoLogo@plainTeX@runtogether}
%    \begin{macrocode}
\def\HoLogo@plainTeX@runtogether#1{%
  \HOLOGO@mbox{#1{p}{P}lain\hologo{TeX}}%
}
%    \end{macrocode}
%    \end{macro}
%    \begin{macro}{\HoLogoCs@plainTeX@runtogether}
%    \begin{macrocode}
\def\HoLogoCs@plainTeX@runtogether#1{#1{p}{P}lainTeX}
%    \end{macrocode}
%    \end{macro}
%    \begin{macro}{\HoLogoBkm@plainTeX@runtogether}
%    \begin{macrocode}
\def\HoLogoBkm@plainTeX@runtogether#1{%
  #1{p}{P}lain\hologo{TeX}%
}
%    \end{macrocode}
%    \end{macro}
%    \begin{macro}{\HoLogoHtml@plainTeX@runtogether}
%    \begin{macrocode}
\def\HoLogoHtml@plainTeX@runtogether#1{%
  #1{p}{P}lain\hologo{TeX}%
}
%    \end{macrocode}
%    \end{macro}
%
%    \begin{macro}{\HoLogo@plainTeX}
%    \begin{macrocode}
\def\HoLogo@plainTeX{\HoLogo@plainTeX@space}
%    \end{macrocode}
%    \end{macro}
%    \begin{macro}{\HoLogoCs@plainTeX}
%    \begin{macrocode}
\def\HoLogoCs@plainTeX{\HoLogoCs@plainTeX@space}
%    \end{macrocode}
%    \end{macro}
%    \begin{macro}{\HoLogoBkm@plainTeX}
%    \begin{macrocode}
\def\HoLogoBkm@plainTeX{\HoLogoBkm@plainTeX@space}
%    \end{macrocode}
%    \end{macro}
%    \begin{macro}{\HoLogoHtml@plainTeX}
%    \begin{macrocode}
\def\HoLogoHtml@plainTeX{\HoLogoHtml@plainTeX@space}
%    \end{macrocode}
%    \end{macro}
%
% \subsubsection{\hologo{LaTeX}}
%
%    Source: \hologo{LaTeX} kernel.
%\begin{quote}
%\begin{verbatim}
%\DeclareRobustCommand{\LaTeX}{%
%  L%
%  \kern-.36em%
%  {%
%    \sbox\z@ T%
%    \vbox to\ht\z@{%
%      \hbox{%
%        \check@mathfonts
%        \fontsize\sf@size\z@
%        \math@fontsfalse
%        \selectfont
%        A%
%      }%
%      \vss
%    }%
%  }%
%  \kern-.15em%
%  \TeX
%}
%\end{verbatim}
%\end{quote}
%
%    \begin{macro}{\HoLogo@La}
%    \begin{macrocode}
\def\HoLogo@La#1{%
  L%
  \kern-.36em%
  \begingroup
    \setbox\ltx@zero\hbox{T}%
    \vbox to\ht\ltx@zero{%
      \hbox{%
        \ltx@ifundefined{check@mathfonts}{%
          \csname sevenrm\endcsname
        }{%
          \check@mathfonts
          \fontsize\sf@size{0pt}%
          \math@fontsfalse\selectfont
        }%
        A%
      }%
      \vss
    }%
  \endgroup
}
%    \end{macrocode}
%    \end{macro}
%
%    \begin{macro}{\HoLogo@LaTeX}
%    Source: \hologo{LaTeX} kernel.
%    \begin{macrocode}
\def\HoLogo@LaTeX#1{%
  \hologo{La}%
  \kern-.15em%
  \hologo{TeX}%
}
%    \end{macrocode}
%    \end{macro}
%    \begin{macro}{\HoLogoHtml@LaTeX}
%    \begin{macrocode}
\def\HoLogoHtml@LaTeX#1{%
  \HoLogoCss@LaTeX
  \HOLOGO@Span{LaTeX}{%
    L%
    \HOLOGO@Span{a}{%
      A%
    }%
    \hologo{TeX}%
  }%
}
%    \end{macrocode}
%    \end{macro}
%    \begin{macro}{\HoLogoCss@LaTeX}
%    \begin{macrocode}
\def\HoLogoCss@LaTeX{%
  \Css{%
    span.HoLogo-LaTeX span.HoLogo-a{%
      position:relative;%
      top:-.5ex;%
      margin-left:-.36em;%
      margin-right:-.15em;%
      font-size:85\%;%
    }%
  }%
  \global\let\HoLogoCss@LaTeX\relax
}
%    \end{macrocode}
%    \end{macro}
%
% \subsubsection{\hologo{(La)TeX}}
%
%    \begin{macro}{\HoLogo@LaTeXTeX}
%    The kerning around the parentheses is taken
%    from package \xpackage{dtklogos} \cite{dtklogos}.
%\begin{quote}
%\begin{verbatim}
%\DeclareRobustCommand{\LaTeXTeX}{%
%  (%
%  \kern-.15em%
%  L%
%  \kern-.36em%
%  {%
%    \sbox\z@ T%
%    \vbox to\ht0{%
%      \hbox{%
%        $\m@th$%
%        \csname S@\f@size\endcsname
%        \fontsize\sf@size\z@
%        \math@fontsfalse
%        \selectfont
%        A%
%      }%
%      \vss
%    }%
%  }%
%  \kern-.2em%
%  )%
%  \kern-.15em%
%  \TeX
%}
%\end{verbatim}
%\end{quote}
%    \begin{macrocode}
\def\HoLogo@LaTeXTeX#1{%
  (%
  \kern-.15em%
  \hologo{La}%
  \kern-.2em%
  )%
  \kern-.15em%
  \hologo{TeX}%
}
%    \end{macrocode}
%    \end{macro}
%    \begin{macro}{\HoLogoBkm@LaTeXTeX}
%    \begin{macrocode}
\def\HoLogoBkm@LaTeXTeX#1{(La)TeX}
%    \end{macrocode}
%    \end{macro}
%
%    \begin{macro}{\HoLogo@(La)TeX}
%    \begin{macrocode}
\expandafter
\let\csname HoLogo@(La)TeX\endcsname\HoLogo@LaTeXTeX
%    \end{macrocode}
%    \end{macro}
%    \begin{macro}{\HoLogoBkm@(La)TeX}
%    \begin{macrocode}
\expandafter
\let\csname HoLogoBkm@(La)TeX\endcsname\HoLogoBkm@LaTeXTeX
%    \end{macrocode}
%    \end{macro}
%    \begin{macro}{\HoLogoHtml@LaTeXTeX}
%    \begin{macrocode}
\def\HoLogoHtml@LaTeXTeX#1{%
  \HoLogoCss@LaTeXTeX
  \HOLOGO@Span{LaTeXTeX}{%
    (%
    \HOLOGO@Span{L}{L}%
    \HOLOGO@Span{a}{A}%
    \HOLOGO@Span{ParenRight}{)}%
    \hologo{TeX}%
  }%
}
%    \end{macrocode}
%    \end{macro}
%    \begin{macro}{\HoLogoHtml@(La)TeX}
%    Kerning after opening parentheses and before closing parentheses
%    is $-0.1$\,em. The original values $-0.15$\,em
%    looked too ugly for a serif font.
%    \begin{macrocode}
\expandafter
\let\csname HoLogoHtml@(La)TeX\endcsname\HoLogoHtml@LaTeXTeX
%    \end{macrocode}
%    \end{macro}
%    \begin{macro}{\HoLogoCss@LaTeXTeX}
%    \begin{macrocode}
\def\HoLogoCss@LaTeXTeX{%
  \Css{%
    span.HoLogo-LaTeXTeX span.HoLogo-L{%
      margin-left:-.1em;%
    }%
  }%
  \Css{%
    span.HoLogo-LaTeXTeX span.HoLogo-a{%
      position:relative;%
      top:-.5ex;%
      margin-left:-.36em;%
      margin-right:-.1em;%
      font-size:85\%;%
    }%
  }%
  \Css{%
    span.HoLogo-LaTeXTeX span.HoLogo-ParenRight{%
      margin-right:-.15em;%
    }%
  }%
  \global\let\HoLogoCss@LaTeXTeX\relax
}
%    \end{macrocode}
%    \end{macro}
%
% \subsubsection{\hologo{LaTeXe}}
%
%    \begin{macro}{\HoLogo@LaTeXe}
%    Source: \hologo{LaTeX} kernel
%    \begin{macrocode}
\def\HoLogo@LaTeXe#1{%
  \hologo{LaTeX}%
  \kern.15em%
  \hbox{%
    \HOLOGO@MathSetup
    2%
    $_{\textstyle\varepsilon}$%
  }%
}
%    \end{macrocode}
%    \end{macro}
%
%    \begin{macro}{\HoLogoCs@LaTeXe}
%    \begin{macrocode}
\ifnum64=`\^^^^0040\relax % test for big chars of LuaTeX/XeTeX
  \catcode`\$=9 %
  \catcode`\&=14 %
\else
  \catcode`\$=14 %
  \catcode`\&=9 %
\fi
\def\HoLogoCs@LaTeXe#1{%
  LaTeX2%
$ \string ^^^^0395%
& e%
}%
\catcode`\$=3 %
\catcode`\&=4 %
%    \end{macrocode}
%    \end{macro}
%
%    \begin{macro}{\HoLogoBkm@LaTeXe}
%    \begin{macrocode}
\def\HoLogoBkm@LaTeXe#1{%
  \hologo{LaTeX}%
  2%
  \HOLOGO@PdfdocUnicode{e}{\textepsilon}%
}
%    \end{macrocode}
%    \end{macro}
%
%    \begin{macro}{\HoLogoHtml@LaTeXe}
%    \begin{macrocode}
\def\HoLogoHtml@LaTeXe#1{%
  \HoLogoCss@LaTeXe
  \HOLOGO@Span{LaTeX2e}{%
    \hologo{LaTeX}%
    \HOLOGO@Span{2}{2}%
    \HOLOGO@Span{e}{%
      \HOLOGO@MathSetup
      \ensuremath{\textstyle\varepsilon}%
    }%
  }%
}
%    \end{macrocode}
%    \end{macro}
%    \begin{macro}{\HoLogoCss@LaTeXe}
%    \begin{macrocode}
\def\HoLogoCss@LaTeXe{%
  \Css{%
    span.HoLogo-LaTeX2e span.HoLogo-2{%
      padding-left:.15em;%
    }%
  }%
  \Css{%
    span.HoLogo-LaTeX2e span.HoLogo-e{%
      position:relative;%
      top:.35ex;%
      text-decoration:none;%
    }%
  }%
  \global\let\HoLogoCss@LaTeXe\relax
}
%    \end{macrocode}
%    \end{macro}
%
%    \begin{macro}{\HoLogo@LaTeX2e}
%    \begin{macrocode}
\expandafter
\let\csname HoLogo@LaTeX2e\endcsname\HoLogo@LaTeXe
%    \end{macrocode}
%    \end{macro}
%    \begin{macro}{\HoLogoCs@LaTeX2e}
%    \begin{macrocode}
\expandafter
\let\csname HoLogoCs@LaTeX2e\endcsname\HoLogoCs@LaTeXe
%    \end{macrocode}
%    \end{macro}
%    \begin{macro}{\HoLogoBkm@LaTeX2e}
%    \begin{macrocode}
\expandafter
\let\csname HoLogoBkm@LaTeX2e\endcsname\HoLogoBkm@LaTeXe
%    \end{macrocode}
%    \end{macro}
%    \begin{macro}{\HoLogoHtml@LaTeX2e}
%    \begin{macrocode}
\expandafter
\let\csname HoLogoHtml@LaTeX2e\endcsname\HoLogoHtml@LaTeXe
%    \end{macrocode}
%    \end{macro}
%
% \subsubsection{\hologo{LaTeX3}}
%
%    \begin{macro}{\HoLogo@LaTeX3}
%    Source: \hologo{LaTeX} kernel
%    \begin{macrocode}
\expandafter\def\csname HoLogo@LaTeX3\endcsname#1{%
  \hologo{LaTeX}%
  3%
}
%    \end{macrocode}
%    \end{macro}
%
%    \begin{macro}{\HoLogoBkm@LaTeX3}
%    \begin{macrocode}
\expandafter\def\csname HoLogoBkm@LaTeX3\endcsname#1{%
  \hologo{LaTeX}%
  3%
}
%    \end{macrocode}
%    \end{macro}
%    \begin{macro}{\HoLogoHtml@LaTeX3}
%    \begin{macrocode}
\expandafter
\let\csname HoLogoHtml@LaTeX3\expandafter\endcsname
\csname HoLogo@LaTeX3\endcsname
%    \end{macrocode}
%    \end{macro}
%
% \subsubsection{\hologo{LaTeXML}}
%
%    \begin{macro}{\HoLogo@LaTeXML}
%    \begin{macrocode}
\def\HoLogo@LaTeXML#1{%
  \HOLOGO@mbox{%
    \hologo{La}%
    \kern-.15em%
    T%
    \kern-.1667em%
    \lower.5ex\hbox{E}%
    \kern-.125em%
    \HoLogoFont@font{LaTeXML}{sc}{xml}%
  }%
}
%    \end{macrocode}
%    \end{macro}
%    \begin{macro}{\HoLogoHtml@pdfLaTeX}
%    \begin{macrocode}
\def\HoLogoHtml@LaTeXML#1{%
  \HOLOGO@Span{LaTeXML}{%
    \HoLogoCss@LaTeX
    \HoLogoCss@TeX
    \HOLOGO@Span{LaTeX}{%
      L%
      \HOLOGO@Span{a}{%
        A%
      }%
    }%
    \HOLOGO@Span{TeX}{%
      T%
      \HOLOGO@Span{e}{%
        E%
      }%
    }%
    \HCode{<span style="font-variant: small-caps;">}%
    xml%
    \HCode{</span>}%
  }%
}
%    \end{macrocode}
%    \end{macro}
%
% \subsubsection{\hologo{eTeX}}
%
%    \begin{macro}{\HoLogo@eTeX}
%    Source: package \xpackage{etex}
%    \begin{macrocode}
\def\HoLogo@eTeX#1{%
  \ltx@mbox{%
    \HOLOGO@MathSetup
    $\varepsilon$%
    -%
    \HOLOGO@NegativeKerning{-T,T-,To}%
    \hologo{TeX}%
  }%
}
%    \end{macrocode}
%    \end{macro}
%    \begin{macro}{\HoLogoCs@eTeX}
%    \begin{macrocode}
\ifnum64=`\^^^^0040\relax % test for big chars of LuaTeX/XeTeX
  \catcode`\$=9 %
  \catcode`\&=14 %
\else
  \catcode`\$=14 %
  \catcode`\&=9 %
\fi
\def\HoLogoCs@eTeX#1{%
$ #1{\string ^^^^0395}{\string ^^^^03b5}%
& #1{e}{E}%
  TeX%
}%
\catcode`\$=3 %
\catcode`\&=4 %
%    \end{macrocode}
%    \end{macro}
%    \begin{macro}{\HoLogoBkm@eTeX}
%    \begin{macrocode}
\def\HoLogoBkm@eTeX#1{%
  \HOLOGO@PdfdocUnicode{#1{e}{E}}{\textepsilon}%
  -%
  \hologo{TeX}%
}
%    \end{macrocode}
%    \end{macro}
%    \begin{macro}{\HoLogoHtml@eTeX}
%    \begin{macrocode}
\def\HoLogoHtml@eTeX#1{%
  \ltx@mbox{%
    \HOLOGO@MathSetup
    $\varepsilon$%
    -%
    \hologo{TeX}%
  }%
}
%    \end{macrocode}
%    \end{macro}
%
% \subsubsection{\hologo{iniTeX}}
%
%    \begin{macro}{\HoLogo@iniTeX}
%    \begin{macrocode}
\def\HoLogo@iniTeX#1{%
  \HOLOGO@mbox{%
    #1{i}{I}ni\hologo{TeX}%
  }%
}
%    \end{macrocode}
%    \end{macro}
%    \begin{macro}{\HoLogoCs@iniTeX}
%    \begin{macrocode}
\def\HoLogoCs@iniTeX#1{#1{i}{I}niTeX}
%    \end{macrocode}
%    \end{macro}
%    \begin{macro}{\HoLogoBkm@iniTeX}
%    \begin{macrocode}
\def\HoLogoBkm@iniTeX#1{%
  #1{i}{I}ni\hologo{TeX}%
}
%    \end{macrocode}
%    \end{macro}
%    \begin{macro}{\HoLogoHtml@iniTeX}
%    \begin{macrocode}
\let\HoLogoHtml@iniTeX\HoLogo@iniTeX
%    \end{macrocode}
%    \end{macro}
%
% \subsubsection{\hologo{virTeX}}
%
%    \begin{macro}{\HoLogo@virTeX}
%    \begin{macrocode}
\def\HoLogo@virTeX#1{%
  \HOLOGO@mbox{%
    #1{v}{V}ir\hologo{TeX}%
  }%
}
%    \end{macrocode}
%    \end{macro}
%    \begin{macro}{\HoLogoCs@virTeX}
%    \begin{macrocode}
\def\HoLogoCs@virTeX#1{#1{v}{V}irTeX}
%    \end{macrocode}
%    \end{macro}
%    \begin{macro}{\HoLogoBkm@virTeX}
%    \begin{macrocode}
\def\HoLogoBkm@virTeX#1{%
  #1{v}{V}ir\hologo{TeX}%
}
%    \end{macrocode}
%    \end{macro}
%    \begin{macro}{\HoLogoHtml@virTeX}
%    \begin{macrocode}
\let\HoLogoHtml@virTeX\HoLogo@virTeX
%    \end{macrocode}
%    \end{macro}
%
% \subsubsection{\hologo{SliTeX}}
%
% \paragraph{Definitions of the three variants.}
%
%    \begin{macro}{\HoLogo@SLiTeX@lift}
%    \begin{macrocode}
\def\HoLogo@SLiTeX@lift#1{%
  \HoLogoFont@font{SliTeX}{rm}{%
    S%
    \kern-.06em%
    L%
    \kern-.18em%
    \raise.32ex\hbox{\HoLogoFont@font{SliTeX}{sc}{i}}%
    \HOLOGO@discretionary
    \kern-.06em%
    \hologo{TeX}%
  }%
}
%    \end{macrocode}
%    \end{macro}
%    \begin{macro}{\HoLogoBkm@SLiTeX@lift}
%    \begin{macrocode}
\def\HoLogoBkm@SLiTeX@lift#1{SLiTeX}
%    \end{macrocode}
%    \end{macro}
%    \begin{macro}{\HoLogoHtml@SLiTeX@lift}
%    \begin{macrocode}
\def\HoLogoHtml@SLiTeX@lift#1{%
  \HoLogoCss@SLiTeX@lift
  \HOLOGO@Span{SLiTeX-lift}{%
    \HoLogoFont@font{SliTeX}{rm}{%
      S%
      \HOLOGO@Span{L}{L}%
      \HOLOGO@Span{i}{i}%
      \hologo{TeX}%
    }%
  }%
}
%    \end{macrocode}
%    \end{macro}
%    \begin{macro}{\HoLogoCss@SLiTeX@lift}
%    \begin{macrocode}
\def\HoLogoCss@SLiTeX@lift{%
  \Css{%
    span.HoLogo-SLiTeX-lift span.HoLogo-L{%
      margin-left:-.06em;%
      margin-right:-.18em;%
    }%
  }%
  \Css{%
    span.HoLogo-SLiTeX-lift span.HoLogo-i{%
      position:relative;%
      top:-.32ex;%
      margin-right:-.06em;%
      font-variant:small-caps;%
    }%
  }%
  \global\let\HoLogoCss@SLiTeX@lift\relax
}
%    \end{macrocode}
%    \end{macro}
%
%    \begin{macro}{\HoLogo@SliTeX@simple}
%    \begin{macrocode}
\def\HoLogo@SliTeX@simple#1{%
  \HoLogoFont@font{SliTeX}{rm}{%
    \ltx@mbox{%
      \HoLogoFont@font{SliTeX}{sc}{Sli}%
    }%
    \HOLOGO@discretionary
    \hologo{TeX}%
  }%
}
%    \end{macrocode}
%    \end{macro}
%    \begin{macro}{\HoLogoBkm@SliTeX@simple}
%    \begin{macrocode}
\def\HoLogoBkm@SliTeX@simple#1{SliTeX}
%    \end{macrocode}
%    \end{macro}
%    \begin{macro}{\HoLogoHtml@SliTeX@simple}
%    \begin{macrocode}
\let\HoLogoHtml@SliTeX@simple\HoLogo@SliTeX@simple
%    \end{macrocode}
%    \end{macro}
%
%    \begin{macro}{\HoLogo@SliTeX@narrow}
%    \begin{macrocode}
\def\HoLogo@SliTeX@narrow#1{%
  \HoLogoFont@font{SliTeX}{rm}{%
    \ltx@mbox{%
      S%
      \kern-.06em%
      \HoLogoFont@font{SliTeX}{sc}{%
        l%
        \kern-.035em%
        i%
      }%
    }%
    \HOLOGO@discretionary
    \kern-.06em%
    \hologo{TeX}%
  }%
}
%    \end{macrocode}
%    \end{macro}
%    \begin{macro}{\HoLogoBkm@SliTeX@narrow}
%    \begin{macrocode}
\def\HoLogoBkm@SliTeX@narrow#1{SliTeX}
%    \end{macrocode}
%    \end{macro}
%    \begin{macro}{\HoLogoHtml@SliTeX@narrow}
%    \begin{macrocode}
\def\HoLogoHtml@SliTeX@narrow#1{%
  \HoLogoCss@SliTeX@narrow
  \HOLOGO@Span{SliTeX-narrow}{%
    \HoLogoFont@font{SliTeX}{rm}{%
      S%
        \HOLOGO@Span{l}{l}%
        \HOLOGO@Span{i}{i}%
      \hologo{TeX}%
    }%
  }%
}
%    \end{macrocode}
%    \end{macro}
%    \begin{macro}{\HoLogoCss@SliTeX@narrow}
%    \begin{macrocode}
\def\HoLogoCss@SliTeX@narrow{%
  \Css{%
    span.HoLogo-SliTeX-narrow span.HoLogo-l{%
      margin-left:-.06em;%
      margin-right:-.035em;%
      font-variant:small-caps;%
    }%
  }%
  \Css{%
    span.HoLogo-SliTeX-narrow span.HoLogo-i{%
      margin-right:-.06em;%
      font-variant:small-caps;%
    }%
  }%
  \global\let\HoLogoCss@SliTeX@narrow\relax
}
%    \end{macrocode}
%    \end{macro}
%
% \paragraph{Macro set completion.}
%
%    \begin{macro}{\HoLogo@SLiTeX@simple}
%    \begin{macrocode}
\def\HoLogo@SLiTeX@simple{\HoLogo@SliTeX@simple}
%    \end{macrocode}
%    \end{macro}
%    \begin{macro}{\HoLogoBkm@SLiTeX@simple}
%    \begin{macrocode}
\def\HoLogoBkm@SLiTeX@simple{\HoLogoBkm@SliTeX@simple}
%    \end{macrocode}
%    \end{macro}
%    \begin{macro}{\HoLogoHtml@SLiTeX@simple}
%    \begin{macrocode}
\def\HoLogoHtml@SLiTeX@simple{\HoLogoHtml@SliTeX@simple}
%    \end{macrocode}
%    \end{macro}
%
%    \begin{macro}{\HoLogo@SLiTeX@narrow}
%    \begin{macrocode}
\def\HoLogo@SLiTeX@narrow{\HoLogo@SliTeX@narrow}
%    \end{macrocode}
%    \end{macro}
%    \begin{macro}{\HoLogoBkm@SLiTeX@narrow}
%    \begin{macrocode}
\def\HoLogoBkm@SLiTeX@narrow{\HoLogoBkm@SliTeX@narrow}
%    \end{macrocode}
%    \end{macro}
%    \begin{macro}{\HoLogoHtml@SLiTeX@narrow}
%    \begin{macrocode}
\def\HoLogoHtml@SLiTeX@narrow{\HoLogoHtml@SliTeX@narrow}
%    \end{macrocode}
%    \end{macro}
%
%    \begin{macro}{\HoLogo@SliTeX@lift}
%    \begin{macrocode}
\def\HoLogo@SliTeX@lift{\HoLogo@SLiTeX@lift}
%    \end{macrocode}
%    \end{macro}
%    \begin{macro}{\HoLogoBkm@SliTeX@lift}
%    \begin{macrocode}
\def\HoLogoBkm@SliTeX@lift{\HoLogoBkm@SLiTeX@lift}
%    \end{macrocode}
%    \end{macro}
%    \begin{macro}{\HoLogoHtml@SliTeX@lift}
%    \begin{macrocode}
\def\HoLogoHtml@SliTeX@lift{\HoLogoHtml@SLiTeX@lift}
%    \end{macrocode}
%    \end{macro}
%
% \paragraph{Defaults.}
%
%    \begin{macro}{\HoLogo@SLiTeX}
%    \begin{macrocode}
\def\HoLogo@SLiTeX{\HoLogo@SLiTeX@lift}
%    \end{macrocode}
%    \end{macro}
%    \begin{macro}{\HoLogoBkm@SLiTeX}
%    \begin{macrocode}
\def\HoLogoBkm@SLiTeX{\HoLogoBkm@SLiTeX@lift}
%    \end{macrocode}
%    \end{macro}
%    \begin{macro}{\HoLogoHtml@SLiTeX}
%    \begin{macrocode}
\def\HoLogoHtml@SLiTeX{\HoLogoHtml@SLiTeX@lift}
%    \end{macrocode}
%    \end{macro}
%
%    \begin{macro}{\HoLogo@SliTeX}
%    \begin{macrocode}
\def\HoLogo@SliTeX{\HoLogo@SliTeX@narrow}
%    \end{macrocode}
%    \end{macro}
%    \begin{macro}{\HoLogoBkm@SliTeX}
%    \begin{macrocode}
\def\HoLogoBkm@SliTeX{\HoLogoBkm@SliTeX@narrow}
%    \end{macrocode}
%    \end{macro}
%    \begin{macro}{\HoLogoHtml@SliTeX}
%    \begin{macrocode}
\def\HoLogoHtml@SliTeX{\HoLogoHtml@SliTeX@narrow}
%    \end{macrocode}
%    \end{macro}
%
% \subsubsection{\hologo{LuaTeX}}
%
%    \begin{macro}{\HoLogo@LuaTeX}
%    The kerning is an idea of Hans Hagen, see mailing list
%    `luatex at tug dot org' in March 2010.
%    \begin{macrocode}
\def\HoLogo@LuaTeX#1{%
  \HOLOGO@mbox{%
    Lua%
    \HOLOGO@NegativeKerning{aT,oT,To}%
    \hologo{TeX}%
  }%
}
%    \end{macrocode}
%    \end{macro}
%    \begin{macro}{\HoLogoHtml@LuaTeX}
%    \begin{macrocode}
\let\HoLogoHtml@LuaTeX\HoLogo@LuaTeX
%    \end{macrocode}
%    \end{macro}
%
% \subsubsection{\hologo{LuaLaTeX}}
%
%    \begin{macro}{\HoLogo@LuaLaTeX}
%    \begin{macrocode}
\def\HoLogo@LuaLaTeX#1{%
  \HOLOGO@mbox{%
    Lua%
    \hologo{LaTeX}%
  }%
}
%    \end{macrocode}
%    \end{macro}
%    \begin{macro}{\HoLogoHtml@LuaLaTeX}
%    \begin{macrocode}
\let\HoLogoHtml@LuaLaTeX\HoLogo@LuaLaTeX
%    \end{macrocode}
%    \end{macro}
%
% \subsubsection{\hologo{XeTeX}, \hologo{XeLaTeX}}
%
%    \begin{macro}{\HOLOGO@IfCharExists}
%    \begin{macrocode}
\ifluatex
  \ifnum\luatexversion<36 %
  \else
    \def\HOLOGO@IfCharExists#1{%
      \ifnum
        \directlua{%
           if luaotfload and luaotfload.aux then
             if luaotfload.aux.font_has_glyph(%
                    font.current(), \number#1) then % 	 
	       tex.print("1") % 	 
	     end % 	 
	   elseif font and font.fonts and font.current then %
            local f = font.fonts[font.current()]%
            if f.characters and f.characters[\number#1] then %
              tex.print("1")%
            end %
          end%
        }0=\ltx@zero
        \expandafter\ltx@secondoftwo
      \else
        \expandafter\ltx@firstoftwo
      \fi
    }%
  \fi
\fi
\ltx@IfUndefined{HOLOGO@IfCharExists}{%
  \def\HOLOGO@@IfCharExists#1{%
    \begingroup
      \tracinglostchars=\ltx@zero
      \setbox\ltx@zero=\hbox{%
        \kern7sp\char#1\relax
        \ifnum\lastkern>\ltx@zero
          \expandafter\aftergroup\csname iffalse\endcsname
        \else
          \expandafter\aftergroup\csname iftrue\endcsname
        \fi
      }%
      % \if{true|false} from \aftergroup
      \endgroup
      \expandafter\ltx@firstoftwo
    \else
      \endgroup
      \expandafter\ltx@secondoftwo
    \fi
  }%
  \ifxetex
    \ltx@IfUndefined{XeTeXfonttype}{}{%
      \ltx@IfUndefined{XeTeXcharglyph}{}{%
        \def\HOLOGO@IfCharExists#1{%
          \ifnum\XeTeXfonttype\font>\ltx@zero
            \expandafter\ltx@firstofthree
          \else
            \expandafter\ltx@gobble
          \fi
          {%
            \ifnum\XeTeXcharglyph#1>\ltx@zero
              \expandafter\ltx@firstoftwo
            \else
              \expandafter\ltx@secondoftwo
            \fi
          }%
          \HOLOGO@@IfCharExists{#1}%
        }%
      }%
    }%
  \fi
}{}
\ltx@ifundefined{HOLOGO@IfCharExists}{%
  \ifnum64=`\^^^^0040\relax % test for big chars of LuaTeX/XeTeX
    \let\HOLOGO@IfCharExists\HOLOGO@@IfCharExists
  \else
    \def\HOLOGO@IfCharExists#1{%
      \ifnum#1>255 %
        \expandafter\ltx@fourthoffour
      \fi
      \HOLOGO@@IfCharExists{#1}%
    }%
  \fi
}{}
%    \end{macrocode}
%    \end{macro}
%
%    \begin{macro}{\HoLogo@Xe}
%    Source: package \xpackage{dtklogos}
%    \begin{macrocode}
\def\HoLogo@Xe#1{%
  X%
  \kern-.1em\relax
  \HOLOGO@IfCharExists{"018E}{%
    \lower.5ex\hbox{\char"018E}%
  }{%
    \chardef\HOLOGO@choice=\ltx@zero
    \ifdim\fontdimen\ltx@one\font>0pt %
      \ltx@IfUndefined{rotatebox}{%
        \ltx@IfUndefined{pgftext}{%
          \ltx@IfUndefined{psscalebox}{%
            \ltx@IfUndefined{HOLOGO@ScaleBox@\hologoDriver}{%
            }{%
              \chardef\HOLOGO@choice=4 %
            }%
          }{%
            \chardef\HOLOGO@choice=3 %
          }%
        }{%
          \chardef\HOLOGO@choice=2 %
        }%
      }{%
        \chardef\HOLOGO@choice=1 %
      }%
      \ifcase\HOLOGO@choice
        \HOLOGO@WarningUnsupportedDriver{Xe}%
        e%
      \or % 1: \rotatebox
        \begingroup
          \setbox\ltx@zero\hbox{\rotatebox{180}{E}}%
          \ltx@LocDimenA=\dp\ltx@zero
          \advance\ltx@LocDimenA by -.5ex\relax
          \raise\ltx@LocDimenA\box\ltx@zero
        \endgroup
      \or % 2: \pgftext
        \lower.5ex\hbox{%
          \pgfpicture
            \pgftext[rotate=180]{E}%
          \endpgfpicture
        }%
      \or % 3: \psscalebox
        \begingroup
          \setbox\ltx@zero\hbox{\psscalebox{-1 -1}{E}}%
          \ltx@LocDimenA=\dp\ltx@zero
          \advance\ltx@LocDimenA by -.5ex\relax
          \raise\ltx@LocDimenA\box\ltx@zero
        \endgroup
      \or % 4: \HOLOGO@PointReflectBox
        \lower.5ex\hbox{\HOLOGO@PointReflectBox{E}}%
      \else
        \@PackageError{hologo}{Internal error (choice/it}\@ehc
      \fi
    \else
      \ltx@IfUndefined{reflectbox}{%
        \ltx@IfUndefined{pgftext}{%
          \ltx@IfUndefined{psscalebox}{%
            \ltx@IfUndefined{HOLOGO@ScaleBox@\hologoDriver}{%
            }{%
              \chardef\HOLOGO@choice=4 %
            }%
          }{%
            \chardef\HOLOGO@choice=3 %
          }%
        }{%
          \chardef\HOLOGO@choice=2 %
        }%
      }{%
        \chardef\HOLOGO@choice=1 %
      }%
      \ifcase\HOLOGO@choice
        \HOLOGO@WarningUnsupportedDriver{Xe}%
        e%
      \or % 1: reflectbox
        \lower.5ex\hbox{%
          \reflectbox{E}%
        }%
      \or % 2: \pgftext
        \lower.5ex\hbox{%
          \pgfpicture
            \pgftransformxscale{-1}%
            \pgftext{E}%
          \endpgfpicture
        }%
      \or % 3: \psscalebox
        \lower.5ex\hbox{%
          \psscalebox{-1 1}{E}%
        }%
      \or % 4: \HOLOGO@Reflectbox
        \lower.5ex\hbox{%
          \HOLOGO@ReflectBox{E}%
        }%
      \else
        \@PackageError{hologo}{Internal error (choice/up)}\@ehc
      \fi
    \fi
  }%
}
%    \end{macrocode}
%    \end{macro}
%    \begin{macro}{\HoLogoHtml@Xe}
%    \begin{macrocode}
\def\HoLogoHtml@Xe#1{%
  \HoLogoCss@Xe
  \HOLOGO@Span{Xe}{%
    X%
    \HOLOGO@Span{e}{%
      \HCode{&\ltx@hashchar x018e;}%
    }%
  }%
}
%    \end{macrocode}
%    \end{macro}
%    \begin{macro}{\HoLogoCss@Xe}
%    \begin{macrocode}
\def\HoLogoCss@Xe{%
  \Css{%
    span.HoLogo-Xe span.HoLogo-e{%
      position:relative;%
      top:.5ex;%
      left-margin:-.1em;%
    }%
  }%
  \global\let\HoLogoCss@Xe\relax
}
%    \end{macrocode}
%    \end{macro}
%
%    \begin{macro}{\HoLogo@XeTeX}
%    \begin{macrocode}
\def\HoLogo@XeTeX#1{%
  \hologo{Xe}%
  \kern-.15em\relax
  \hologo{TeX}%
}
%    \end{macrocode}
%    \end{macro}
%
%    \begin{macro}{\HoLogoHtml@XeTeX}
%    \begin{macrocode}
\def\HoLogoHtml@XeTeX#1{%
  \HoLogoCss@XeTeX
  \HOLOGO@Span{XeTeX}{%
    \hologo{Xe}%
    \hologo{TeX}%
  }%
}
%    \end{macrocode}
%    \end{macro}
%    \begin{macro}{\HoLogoCss@XeTeX}
%    \begin{macrocode}
\def\HoLogoCss@XeTeX{%
  \Css{%
    span.HoLogo-XeTeX span.HoLogo-TeX{%
      margin-left:-.15em;%
    }%
  }%
  \global\let\HoLogoCss@XeTeX\relax
}
%    \end{macrocode}
%    \end{macro}
%
%    \begin{macro}{\HoLogo@XeLaTeX}
%    \begin{macrocode}
\def\HoLogo@XeLaTeX#1{%
  \hologo{Xe}%
  \kern-.13em%
  \hologo{LaTeX}%
}
%    \end{macrocode}
%    \end{macro}
%    \begin{macro}{\HoLogoHtml@XeLaTeX}
%    \begin{macrocode}
\def\HoLogoHtml@XeLaTeX#1{%
  \HoLogoCss@XeLaTeX
  \HOLOGO@Span{XeLaTeX}{%
    \hologo{Xe}%
    \hologo{LaTeX}%
  }%
}
%    \end{macrocode}
%    \end{macro}
%    \begin{macro}{\HoLogoCss@XeLaTeX}
%    \begin{macrocode}
\def\HoLogoCss@XeLaTeX{%
  \Css{%
    span.HoLogo-XeLaTeX span.HoLogo-Xe{%
      margin-right:-.13em;%
    }%
  }%
  \global\let\HoLogoCss@XeLaTeX\relax
}
%    \end{macrocode}
%    \end{macro}
%
% \subsubsection{\hologo{pdfTeX}, \hologo{pdfLaTeX}}
%
%    \begin{macro}{\HoLogo@pdfTeX}
%    \begin{macrocode}
\def\HoLogo@pdfTeX#1{%
  \HOLOGO@mbox{%
    #1{p}{P}df\hologo{TeX}%
  }%
}
%    \end{macrocode}
%    \end{macro}
%    \begin{macro}{\HoLogoCs@pdfTeX}
%    \begin{macrocode}
\def\HoLogoCs@pdfTeX#1{#1{p}{P}dfTeX}
%    \end{macrocode}
%    \end{macro}
%    \begin{macro}{\HoLogoBkm@pdfTeX}
%    \begin{macrocode}
\def\HoLogoBkm@pdfTeX#1{%
  #1{p}{P}df\hologo{TeX}%
}
%    \end{macrocode}
%    \end{macro}
%    \begin{macro}{\HoLogoHtml@pdfTeX}
%    \begin{macrocode}
\let\HoLogoHtml@pdfTeX\HoLogo@pdfTeX
%    \end{macrocode}
%    \end{macro}
%
%    \begin{macro}{\HoLogo@pdfLaTeX}
%    \begin{macrocode}
\def\HoLogo@pdfLaTeX#1{%
  \HOLOGO@mbox{%
    #1{p}{P}df\hologo{LaTeX}%
  }%
}
%    \end{macrocode}
%    \end{macro}
%    \begin{macro}{\HoLogoCs@pdfLaTeX}
%    \begin{macrocode}
\def\HoLogoCs@pdfLaTeX#1{#1{p}{P}dfLaTeX}
%    \end{macrocode}
%    \end{macro}
%    \begin{macro}{\HoLogoBkm@pdfLaTeX}
%    \begin{macrocode}
\def\HoLogoBkm@pdfLaTeX#1{%
  #1{p}{P}df\hologo{LaTeX}%
}
%    \end{macrocode}
%    \end{macro}
%    \begin{macro}{\HoLogoHtml@pdfLaTeX}
%    \begin{macrocode}
\let\HoLogoHtml@pdfLaTeX\HoLogo@pdfLaTeX
%    \end{macrocode}
%    \end{macro}
%
% \subsubsection{\hologo{VTeX}}
%
%    \begin{macro}{\HoLogo@VTeX}
%    \begin{macrocode}
\def\HoLogo@VTeX#1{%
  \HOLOGO@mbox{%
    V\hologo{TeX}%
  }%
}
%    \end{macrocode}
%    \end{macro}
%    \begin{macro}{\HoLogoHtml@VTeX}
%    \begin{macrocode}
\let\HoLogoHtml@VTeX\HoLogo@VTeX
%    \end{macrocode}
%    \end{macro}
%
% \subsubsection{\hologo{AmS}, \dots}
%
%    Source: class \xclass{amsdtx}
%
%    \begin{macro}{\HoLogo@AmS}
%    \begin{macrocode}
\def\HoLogo@AmS#1{%
  \HoLogoFont@font{AmS}{sy}{%
    A%
    \kern-.1667em%
    \lower.5ex\hbox{M}%
    \kern-.125em%
    S%
  }%
}
%    \end{macrocode}
%    \end{macro}
%    \begin{macro}{\HoLogoBkm@AmS}
%    \begin{macrocode}
\def\HoLogoBkm@AmS#1{AmS}
%    \end{macrocode}
%    \end{macro}
%    \begin{macro}{\HoLogoHtml@AmS}
%    \begin{macrocode}
\def\HoLogoHtml@AmS#1{%
  \HoLogoCss@AmS
%  \HoLogoFont@font{AmS}{sy}{%
    \HOLOGO@Span{AmS}{%
      A%
      \HOLOGO@Span{M}{M}%
      S%
    }%
%   }%
}
%    \end{macrocode}
%    \end{macro}
%    \begin{macro}{\HoLogoCss@AmS}
%    \begin{macrocode}
\def\HoLogoCss@AmS{%
  \Css{%
    span.HoLogo-AmS span.HoLogo-M{%
      position:relative;%
      top:.5ex;%
      margin-left:-.1667em;%
      margin-right:-.125em;%
      text-decoration:none;%
    }%
  }%
  \global\let\HoLogoCss@AmS\relax
}
%    \end{macrocode}
%    \end{macro}
%
%    \begin{macro}{\HoLogo@AmSTeX}
%    \begin{macrocode}
\def\HoLogo@AmSTeX#1{%
  \hologo{AmS}%
  \HOLOGO@hyphen
  \hologo{TeX}%
}
%    \end{macrocode}
%    \end{macro}
%    \begin{macro}{\HoLogoBkm@AmSTeX}
%    \begin{macrocode}
\def\HoLogoBkm@AmSTeX#1{AmS-TeX}%
%    \end{macrocode}
%    \end{macro}
%    \begin{macro}{\HoLogoHtml@AmSTeX}
%    \begin{macrocode}
\let\HoLogoHtml@AmSTeX\HoLogo@AmSTeX
%    \end{macrocode}
%    \end{macro}
%
%    \begin{macro}{\HoLogo@AmSLaTeX}
%    \begin{macrocode}
\def\HoLogo@AmSLaTeX#1{%
  \hologo{AmS}%
  \HOLOGO@hyphen
  \hologo{LaTeX}%
}
%    \end{macrocode}
%    \end{macro}
%    \begin{macro}{\HoLogoBkm@AmSLaTeX}
%    \begin{macrocode}
\def\HoLogoBkm@AmSLaTeX#1{AmS-LaTeX}%
%    \end{macrocode}
%    \end{macro}
%    \begin{macro}{\HoLogoHtml@AmSLaTeX}
%    \begin{macrocode}
\let\HoLogoHtml@AmSLaTeX\HoLogo@AmSLaTeX
%    \end{macrocode}
%    \end{macro}
%
% \subsubsection{\hologo{BibTeX}}
%
%    \begin{macro}{\HoLogo@BibTeX@sc}
%    A definition of \hologo{BibTeX} is provided in
%    the documentation source for the manual of \hologo{BibTeX}
%    \cite{btxdoc}.
%\begin{quote}
%\begin{verbatim}
%\def\BibTeX{%
%  {%
%    \rm
%    B%
%    \kern-.05em%
%    {%
%      \sc
%      i%
%      \kern-.025em %
%      b%
%    }%
%    \kern-.08em
%    T%
%    \kern-.1667em%
%    \lower.7ex\hbox{E}%
%    \kern-.125em%
%    X%
%  }%
%}
%\end{verbatim}
%\end{quote}
%    \begin{macrocode}
\def\HoLogo@BibTeX@sc#1{%
  B%
  \kern-.05em%
  \HoLogoFont@font{BibTeX}{sc}{%
    i%
    \kern-.025em%
    b%
  }%
  \HOLOGO@discretionary
  \kern-.08em%
  \hologo{TeX}%
}
%    \end{macrocode}
%    \end{macro}
%    \begin{macro}{\HoLogoHtml@BibTeX@sc}
%    \begin{macrocode}
\def\HoLogoHtml@BibTeX@sc#1{%
  \HoLogoCss@BibTeX@sc
  \HOLOGO@Span{BibTeX-sc}{%
    B%
    \HOLOGO@Span{i}{i}%
    \HOLOGO@Span{b}{b}%
    \hologo{TeX}%
  }%
}
%    \end{macrocode}
%    \end{macro}
%    \begin{macro}{\HoLogoCss@BibTeX@sc}
%    \begin{macrocode}
\def\HoLogoCss@BibTeX@sc{%
  \Css{%
    span.HoLogo-BibTeX-sc span.HoLogo-i{%
      margin-left:-.05em;%
      margin-right:-.025em;%
      font-variant:small-caps;%
    }%
  }%
  \Css{%
    span.HoLogo-BibTeX-sc span.HoLogo-b{%
      margin-right:-.08em;%
      font-variant:small-caps;%
    }%
  }%
  \global\let\HoLogoCss@BibTeX@sc\relax
}
%    \end{macrocode}
%    \end{macro}
%
%    \begin{macro}{\HoLogo@BibTeX@sf}
%    Variant \xoption{sf} avoids trouble with unavailable
%    small caps fonts (e.g., bold versions of Computer Modern or
%    Latin Modern). The definition is taken from
%    package \xpackage{dtklogos} \cite{dtklogos}.
%\begin{quote}
%\begin{verbatim}
%\DeclareRobustCommand{\BibTeX}{%
%  B%
%  \kern-.05em%
%  \hbox{%
%    $\m@th$% %% force math size calculations
%    \csname S@\f@size\endcsname
%    \fontsize\sf@size\z@
%    \math@fontsfalse
%    \selectfont
%    I%
%    \kern-.025em%
%    B
%  }%
%  \kern-.08em%
%  \-%
%  \TeX
%}
%\end{verbatim}
%\end{quote}
%    \begin{macrocode}
\def\HoLogo@BibTeX@sf#1{%
  B%
  \kern-.05em%
  \HoLogoFont@font{BibTeX}{bibsf}{%
    I%
    \kern-.025em%
    B%
  }%
  \HOLOGO@discretionary
  \kern-.08em%
  \hologo{TeX}%
}
%    \end{macrocode}
%    \end{macro}
%    \begin{macro}{\HoLogoHtml@BibTeX@sf}
%    \begin{macrocode}
\def\HoLogoHtml@BibTeX@sf#1{%
  \HoLogoCss@BibTeX@sf
  \HOLOGO@Span{BibTeX-sf}{%
    B%
    \HoLogoFont@font{BibTeX}{bibsf}{%
      \HOLOGO@Span{i}{I}%
      B%
    }%
    \hologo{TeX}%
  }%
}
%    \end{macrocode}
%    \end{macro}
%    \begin{macro}{\HoLogoCss@BibTeX@sf}
%    \begin{macrocode}
\def\HoLogoCss@BibTeX@sf{%
  \Css{%
    span.HoLogo-BibTeX-sf span.HoLogo-i{%
      margin-left:-.05em;%
      margin-right:-.025em;%
    }%
  }%
  \Css{%
    span.HoLogo-BibTeX-sf span.HoLogo-TeX{%
      margin-left:-.08em;%
    }%
  }%
  \global\let\HoLogoCss@BibTeX@sf\relax
}
%    \end{macrocode}
%    \end{macro}
%
%    \begin{macro}{\HoLogo@BibTeX}
%    \begin{macrocode}
\def\HoLogo@BibTeX{\HoLogo@BibTeX@sf}
%    \end{macrocode}
%    \end{macro}
%    \begin{macro}{\HoLogoHtml@BibTeX}
%    \begin{macrocode}
\def\HoLogoHtml@BibTeX{\HoLogoHtml@BibTeX@sf}
%    \end{macrocode}
%    \end{macro}
%
% \subsubsection{\hologo{BibTeX8}}
%
%    \begin{macro}{\HoLogo@BibTeX8}
%    \begin{macrocode}
\expandafter\def\csname HoLogo@BibTeX8\endcsname#1{%
  \hologo{BibTeX}%
  8%
}
%    \end{macrocode}
%    \end{macro}
%
%    \begin{macro}{\HoLogoBkm@BibTeX8}
%    \begin{macrocode}
\expandafter\def\csname HoLogoBkm@BibTeX8\endcsname#1{%
  \hologo{BibTeX}%
  8%
}
%    \end{macrocode}
%    \end{macro}
%    \begin{macro}{\HoLogoHtml@BibTeX8}
%    \begin{macrocode}
\expandafter
\let\csname HoLogoHtml@BibTeX8\expandafter\endcsname
\csname HoLogo@BibTeX8\endcsname
%    \end{macrocode}
%    \end{macro}
%
% \subsubsection{\hologo{ConTeXt}}
%
%    \begin{macro}{\HoLogo@ConTeXt@simple}
%    \begin{macrocode}
\def\HoLogo@ConTeXt@simple#1{%
  \HOLOGO@mbox{Con}%
  \HOLOGO@discretionary
  \HOLOGO@mbox{\hologo{TeX}t}%
}
%    \end{macrocode}
%    \end{macro}
%    \begin{macro}{\HoLogoHtml@ConTeXt@simple}
%    \begin{macrocode}
\let\HoLogoHtml@ConTeXt@simple\HoLogo@ConTeXt@simple
%    \end{macrocode}
%    \end{macro}
%
%    \begin{macro}{\HoLogo@ConTeXt@narrow}
%    This definition of logo \hologo{ConTeXt} with variant \xoption{narrow}
%    comes from TUGboat's class \xclass{ltugboat} (version 2010/11/15 v2.8).
%    \begin{macrocode}
\def\HoLogo@ConTeXt@narrow#1{%
  \HOLOGO@mbox{C\kern-.0333emon}%
  \HOLOGO@discretionary
  \kern-.0667em%
  \HOLOGO@mbox{\hologo{TeX}\kern-.0333emt}%
}
%    \end{macrocode}
%    \end{macro}
%    \begin{macro}{\HoLogoHtml@ConTeXt@narrow}
%    \begin{macrocode}
\def\HoLogoHtml@ConTeXt@narrow#1{%
  \HoLogoCss@ConTeXt@narrow
  \HOLOGO@Span{ConTeXt-narrow}{%
    \HOLOGO@Span{C}{C}%
    on%
    \hologo{TeX}%
    t%
  }%
}
%    \end{macrocode}
%    \end{macro}
%    \begin{macro}{\HoLogoCss@ConTeXt@narrow}
%    \begin{macrocode}
\def\HoLogoCss@ConTeXt@narrow{%
  \Css{%
    span.HoLogo-ConTeXt-narrow span.HoLogo-C{%
      margin-left:-.0333em;%
    }%
  }%
  \Css{%
    span.HoLogo-ConTeXt-narrow span.HoLogo-TeX{%
      margin-left:-.0667em;%
      margin-right:-.0333em;%
    }%
  }%
  \global\let\HoLogoCss@ConTeXt@narrow\relax
}
%    \end{macrocode}
%    \end{macro}
%
%    \begin{macro}{\HoLogo@ConTeXt}
%    \begin{macrocode}
\def\HoLogo@ConTeXt{\HoLogo@ConTeXt@narrow}
%    \end{macrocode}
%    \end{macro}
%    \begin{macro}{\HoLogoHtml@ConTeXt}
%    \begin{macrocode}
\def\HoLogoHtml@ConTeXt{\HoLogoHtml@ConTeXt@narrow}
%    \end{macrocode}
%    \end{macro}
%
% \subsubsection{\hologo{emTeX}}
%
%    \begin{macro}{\HoLogo@emTeX}
%    \begin{macrocode}
\def\HoLogo@emTeX#1{%
  \HOLOGO@mbox{#1{e}{E}m}%
  \HOLOGO@discretionary
  \hologo{TeX}%
}
%    \end{macrocode}
%    \end{macro}
%    \begin{macro}{\HoLogoCs@emTeX}
%    \begin{macrocode}
\def\HoLogoCs@emTeX#1{#1{e}{E}mTeX}%
%    \end{macrocode}
%    \end{macro}
%    \begin{macro}{\HoLogoBkm@emTeX}
%    \begin{macrocode}
\def\HoLogoBkm@emTeX#1{%
  #1{e}{E}m\hologo{TeX}%
}
%    \end{macrocode}
%    \end{macro}
%    \begin{macro}{\HoLogoHtml@emTeX}
%    \begin{macrocode}
\let\HoLogoHtml@emTeX\HoLogo@emTeX
%    \end{macrocode}
%    \end{macro}
%
% \subsubsection{\hologo{ExTeX}}
%
%    \begin{macro}{\HoLogo@ExTeX}
%    The definition is taken from the FAQ of the
%    project \hologo{ExTeX}
%    \cite{ExTeX-FAQ}.
%\begin{quote}
%\begin{verbatim}
%\def\ExTeX{%
%  \textrm{% Logo always with serifs
%    \ensuremath{%
%      \textstyle
%      \varepsilon_{%
%        \kern-0.15em%
%        \mathcal{X}%
%      }%
%    }%
%    \kern-.15em%
%    \TeX
%  }%
%}
%\end{verbatim}
%\end{quote}
%    \begin{macrocode}
\def\HoLogo@ExTeX#1{%
  \HoLogoFont@font{ExTeX}{rm}{%
    \ltx@mbox{%
      \HOLOGO@MathSetup
      $%
        \textstyle
        \varepsilon_{%
          \kern-0.15em%
          \HoLogoFont@font{ExTeX}{sy}{X}%
        }%
      $%
    }%
    \HOLOGO@discretionary
    \kern-.15em%
    \hologo{TeX}%
  }%
}
%    \end{macrocode}
%    \end{macro}
%    \begin{macro}{\HoLogoHtml@ExTeX}
%    \begin{macrocode}
\def\HoLogoHtml@ExTeX#1{%
  \HoLogoCss@ExTeX
  \HoLogoFont@font{ExTeX}{rm}{%
    \HOLOGO@Span{ExTeX}{%
      \ltx@mbox{%
        \HOLOGO@MathSetup
        $\textstyle\varepsilon$%
        \HOLOGO@Span{X}{$\textstyle\chi$}%
        \hologo{TeX}%
      }%
    }%
  }%
}
%    \end{macrocode}
%    \end{macro}
%    \begin{macro}{\HoLogoBkm@ExTeX}
%    \begin{macrocode}
\def\HoLogoBkm@ExTeX#1{%
  \HOLOGO@PdfdocUnicode{#1{e}{E}x}{\textepsilon\textchi}%
  \hologo{TeX}%
}
%    \end{macrocode}
%    \end{macro}
%    \begin{macro}{\HoLogoCss@ExTeX}
%    \begin{macrocode}
\def\HoLogoCss@ExTeX{%
  \Css{%
    span.HoLogo-ExTeX{%
      font-family:serif;%
    }%
  }%
  \Css{%
    span.HoLogo-ExTeX span.HoLogo-TeX{%
      margin-left:-.15em;%
    }%
  }%
  \global\let\HoLogoCss@ExTeX\relax
}
%    \end{macrocode}
%    \end{macro}
%
% \subsubsection{\hologo{MiKTeX}}
%
%    \begin{macro}{\HoLogo@MiKTeX}
%    \begin{macrocode}
\def\HoLogo@MiKTeX#1{%
  \HOLOGO@mbox{MiK}%
  \HOLOGO@discretionary
  \hologo{TeX}%
}
%    \end{macrocode}
%    \end{macro}
%    \begin{macro}{\HoLogoHtml@MiKTeX}
%    \begin{macrocode}
\let\HoLogoHtml@MiKTeX\HoLogo@MiKTeX
%    \end{macrocode}
%    \end{macro}
%
% \subsubsection{\hologo{OzTeX} and friends}
%
%    Source: \hologo{OzTeX} FAQ \cite{OzTeX}:
%    \begin{quote}
%      |\def\OzTeX{O\kern-.03em z\kern-.15em\TeX}|\\
%      (There is no kerning in OzMF, OzMP and OzTtH.)
%    \end{quote}
%
%    \begin{macro}{\HoLogo@OzTeX}
%    \begin{macrocode}
\def\HoLogo@OzTeX#1{%
  O%
  \kern-.03em %
  z%
  \kern-.15em %
  \hologo{TeX}%
}
%    \end{macrocode}
%    \end{macro}
%    \begin{macro}{\HoLogoHtml@OzTeX}
%    \begin{macrocode}
\def\HoLogoHtml@OzTeX#1{%
  \HoLogoCss@OzTeX
  \HOLOGO@Span{OzTeX}{%
    O%
    \HOLOGO@Span{z}{z}%
    \hologo{TeX}%
  }%
}
%    \end{macrocode}
%    \end{macro}
%    \begin{macro}{\HoLogoCss@OzTeX}
%    \begin{macrocode}
\def\HoLogoCss@OzTeX{%
  \Css{%
    span.HoLogo-OzTeX span.HoLogo-z{%
      margin-left:-.03em;%
      margin-right:-.15em;%
    }%
  }%
  \global\let\HoLogoCss@OzTeX\relax
}
%    \end{macrocode}
%    \end{macro}
%
%    \begin{macro}{\HoLogo@OzMF}
%    \begin{macrocode}
\def\HoLogo@OzMF#1{%
  \HOLOGO@mbox{OzMF}%
}
%    \end{macrocode}
%    \end{macro}
%    \begin{macro}{\HoLogo@OzMP}
%    \begin{macrocode}
\def\HoLogo@OzMP#1{%
  \HOLOGO@mbox{OzMP}%
}
%    \end{macrocode}
%    \end{macro}
%    \begin{macro}{\HoLogo@OzTtH}
%    \begin{macrocode}
\def\HoLogo@OzTtH#1{%
  \HOLOGO@mbox{OzTtH}%
}
%    \end{macrocode}
%    \end{macro}
%
% \subsubsection{\hologo{PCTeX}}
%
%    \begin{macro}{\HoLogo@PCTeX}
%    \begin{macrocode}
\def\HoLogo@PCTeX#1{%
  \HOLOGO@mbox{PC}%
  \hologo{TeX}%
}
%    \end{macrocode}
%    \end{macro}
%    \begin{macro}{\HoLogoHtml@PCTeX}
%    \begin{macrocode}
\let\HoLogoHtml@PCTeX\HoLogo@PCTeX
%    \end{macrocode}
%    \end{macro}
%
% \subsubsection{\hologo{PiCTeX}}
%
%    The original definitions from \xfile{pictex.tex} \cite{PiCTeX}:
%\begin{quote}
%\begin{verbatim}
%\def\PiC{%
%  P%
%  \kern-.12em%
%  \lower.5ex\hbox{I}%
%  \kern-.075em%
%  C%
%}
%\def\PiCTeX{%
%  \PiC
%  \kern-.11em%
%  \TeX
%}
%\end{verbatim}
%\end{quote}
%
%    \begin{macro}{\HoLogo@PiC}
%    \begin{macrocode}
\def\HoLogo@PiC#1{%
  P%
  \kern-.12em%
  \lower.5ex\hbox{I}%
  \kern-.075em%
  C%
  \HOLOGO@SpaceFactor
}
%    \end{macrocode}
%    \end{macro}
%    \begin{macro}{\HoLogoHtml@PiC}
%    \begin{macrocode}
\def\HoLogoHtml@PiC#1{%
  \HoLogoCss@PiC
  \HOLOGO@Span{PiC}{%
    P%
    \HOLOGO@Span{i}{I}%
    C%
  }%
}
%    \end{macrocode}
%    \end{macro}
%    \begin{macro}{\HoLogoCss@PiC}
%    \begin{macrocode}
\def\HoLogoCss@PiC{%
  \Css{%
    span.HoLogo-PiC span.HoLogo-i{%
      position:relative;%
      top:.5ex;%
      margin-left:-.12em;%
      margin-right:-.075em;%
      text-decoration:none;%
    }%
  }%
  \global\let\HoLogoCss@PiC\relax
}
%    \end{macrocode}
%    \end{macro}
%
%    \begin{macro}{\HoLogo@PiCTeX}
%    \begin{macrocode}
\def\HoLogo@PiCTeX#1{%
  \hologo{PiC}%
  \HOLOGO@discretionary
  \kern-.11em%
  \hologo{TeX}%
}
%    \end{macrocode}
%    \end{macro}
%    \begin{macro}{\HoLogoHtml@PiCTeX}
%    \begin{macrocode}
\def\HoLogoHtml@PiCTeX#1{%
  \HoLogoCss@PiCTeX
  \HOLOGO@Span{PiCTeX}{%
    \hologo{PiC}%
    \hologo{TeX}%
  }%
}
%    \end{macrocode}
%    \end{macro}
%    \begin{macro}{\HoLogoCss@PiCTeX}
%    \begin{macrocode}
\def\HoLogoCss@PiCTeX{%
  \Css{%
    span.HoLogo-PiCTeX span.HoLogo-PiC{%
      margin-right:-.11em;%
    }%
  }%
  \global\let\HoLogoCss@PiCTeX\relax
}
%    \end{macrocode}
%    \end{macro}
%
% \subsubsection{\hologo{teTeX}}
%
%    \begin{macro}{\HoLogo@teTeX}
%    \begin{macrocode}
\def\HoLogo@teTeX#1{%
  \HOLOGO@mbox{#1{t}{T}e}%
  \HOLOGO@discretionary
  \hologo{TeX}%
}
%    \end{macrocode}
%    \end{macro}
%    \begin{macro}{\HoLogoCs@teTeX}
%    \begin{macrocode}
\def\HoLogoCs@teTeX#1{#1{t}{T}dfTeX}
%    \end{macrocode}
%    \end{macro}
%    \begin{macro}{\HoLogoBkm@teTeX}
%    \begin{macrocode}
\def\HoLogoBkm@teTeX#1{%
  #1{t}{T}e\hologo{TeX}%
}
%    \end{macrocode}
%    \end{macro}
%    \begin{macro}{\HoLogoHtml@teTeX}
%    \begin{macrocode}
\let\HoLogoHtml@teTeX\HoLogo@teTeX
%    \end{macrocode}
%    \end{macro}
%
% \subsubsection{\hologo{TeX4ht}}
%
%    \begin{macro}{\HoLogo@TeX4ht}
%    \begin{macrocode}
\expandafter\def\csname HoLogo@TeX4ht\endcsname#1{%
  \HOLOGO@mbox{\hologo{TeX}4ht}%
}
%    \end{macrocode}
%    \end{macro}
%    \begin{macro}{\HoLogoHtml@TeX4ht}
%    \begin{macrocode}
\expandafter
\let\csname HoLogoHtml@TeX4ht\expandafter\endcsname
\csname HoLogo@TeX4ht\endcsname
%    \end{macrocode}
%    \end{macro}
%
%
% \subsubsection{\hologo{SageTeX}}
%
%    \begin{macro}{\HoLogo@SageTeX}
%    \begin{macrocode}
\def\HoLogo@SageTeX#1{%
  \HOLOGO@mbox{Sage}%
  \HOLOGO@discretionary
  \HOLOGO@NegativeKerning{eT,oT,To}%
  \hologo{TeX}%
}
%    \end{macrocode}
%    \end{macro}
%    \begin{macro}{\HoLogoHtml@SageTeX}
%    \begin{macrocode}
\let\HoLogoHtml@SageTeX\HoLogo@SageTeX
%    \end{macrocode}
%    \end{macro}
%
% \subsection{\hologo{METAFONT} and friends}
%
%    \begin{macro}{\HoLogo@METAFONT}
%    \begin{macrocode}
\def\HoLogo@METAFONT#1{%
  \HoLogoFont@font{METAFONT}{logo}{%
    \HOLOGO@mbox{META}%
    \HOLOGO@discretionary
    \HOLOGO@mbox{FONT}%
  }%
}
%    \end{macrocode}
%    \end{macro}
%
%    \begin{macro}{\HoLogo@METAPOST}
%    \begin{macrocode}
\def\HoLogo@METAPOST#1{%
  \HoLogoFont@font{METAPOST}{logo}{%
    \HOLOGO@mbox{META}%
    \HOLOGO@discretionary
    \HOLOGO@mbox{POST}%
  }%
}
%    \end{macrocode}
%    \end{macro}
%
%    \begin{macro}{\HoLogo@MetaFun}
%    \begin{macrocode}
\def\HoLogo@MetaFun#1{%
  \HOLOGO@mbox{Meta}%
  \HOLOGO@discretionary
  \HOLOGO@mbox{Fun}%
}
%    \end{macrocode}
%    \end{macro}
%
%    \begin{macro}{\HoLogo@MetaPost}
%    \begin{macrocode}
\def\HoLogo@MetaPost#1{%
  \HOLOGO@mbox{Meta}%
  \HOLOGO@discretionary
  \HOLOGO@mbox{Post}%
}
%    \end{macrocode}
%    \end{macro}
%
% \subsection{Others}
%
% \subsubsection{\hologo{biber}}
%
%    \begin{macro}{\HoLogo@biber}
%    \begin{macrocode}
\def\HoLogo@biber#1{%
  \HOLOGO@mbox{#1{b}{B}i}%
  \HOLOGO@discretionary
  \HOLOGO@mbox{ber}%
}
%    \end{macrocode}
%    \end{macro}
%    \begin{macro}{\HoLogoCs@biber}
%    \begin{macrocode}
\def\HoLogoCs@biber#1{#1{b}{B}iber}
%    \end{macrocode}
%    \end{macro}
%    \begin{macro}{\HoLogoBkm@biber}
%    \begin{macrocode}
\def\HoLogoBkm@biber#1{%
  #1{b}{B}iber%
}
%    \end{macrocode}
%    \end{macro}
%    \begin{macro}{\HoLogoHtml@biber}
%    \begin{macrocode}
\let\HoLogoHtml@biber\HoLogo@biber
%    \end{macrocode}
%    \end{macro}
%
% \subsubsection{\hologo{KOMAScript}}
%
%    \begin{macro}{\HoLogo@KOMAScript}
%    The definition for \hologo{KOMAScript} is taken
%    from \hologo{KOMAScript} (\xfile{scrlogo.dtx}, reformatted) \cite{scrlogo}:
%\begin{quote}
%\begin{verbatim}
%\@ifundefined{KOMAScript}{%
%  \DeclareRobustCommand{\KOMAScript}{%
%    \textsf{%
%      K\kern.05em O\kern.05emM\kern.05em A%
%      \kern.1em-\kern.1em %
%      Script%
%    }%
%  }%
%}{}
%\end{verbatim}
%\end{quote}
%    \begin{macrocode}
\def\HoLogo@KOMAScript#1{%
  \HoLogoFont@font{KOMAScript}{sf}{%
    \HOLOGO@mbox{%
      K\kern.05em%
      O\kern.05em%
      M\kern.05em%
      A%
    }%
    \kern.1em%
    \HOLOGO@hyphen
    \kern.1em%
    \HOLOGO@mbox{Script}%
  }%
}
%    \end{macrocode}
%    \end{macro}
%    \begin{macro}{\HoLogoBkm@KOMAScript}
%    \begin{macrocode}
\def\HoLogoBkm@KOMAScript#1{%
  KOMA-Script%
}
%    \end{macrocode}
%    \end{macro}
%    \begin{macro}{\HoLogoHtml@KOMAScript}
%    \begin{macrocode}
\def\HoLogoHtml@KOMAScript#1{%
  \HoLogoCss@KOMAScript
  \HoLogoFont@font{KOMAScript}{sf}{%
    \HOLOGO@Span{KOMAScript}{%
      K%
      \HOLOGO@Span{O}{O}%
      M%
      \HOLOGO@Span{A}{A}%
      \HOLOGO@Span{hyphen}{-}%
      Script%
    }%
  }%
}
%    \end{macrocode}
%    \end{macro}
%    \begin{macro}{\HoLogoCss@KOMAScript}
%    \begin{macrocode}
\def\HoLogoCss@KOMAScript{%
  \Css{%
    span.HoLogo-KOMAScript{%
      font-family:sans-serif;%
    }%
  }%
  \Css{%
    span.HoLogo-KOMAScript span.HoLogo-O{%
      padding-left:.05em;%
      padding-right:.05em;%
    }%
  }%
  \Css{%
    span.HoLogo-KOMAScript span.HoLogo-A{%
      padding-left:.05em;%
    }%
  }%
  \Css{%
    span.HoLogo-KOMAScript span.HoLogo-hyphen{%
      padding-left:.1em;%
      padding-right:.1em;%
    }%
  }%
  \global\let\HoLogoCss@KOMAScript\relax
}
%    \end{macrocode}
%    \end{macro}
%
% \subsubsection{\hologo{LyX}}
%
%    \begin{macro}{\HoLogo@LyX}
%    The definition is taken from the documentation source files
%    of \hologo{LyX}, \xfile{Intro.lyx} \cite{LyX}:
%\begin{quote}
%\begin{verbatim}
%\def\LyX{%
%  \texorpdfstring{%
%    L\kern-.1667em\lower.25em\hbox{Y}\kern-.125emX\@%
%  }{%
%    LyX%
%  }%
%}
%\end{verbatim}
%\end{quote}
%    \begin{macrocode}
\def\HoLogo@LyX#1{%
  L%
  \kern-.1667em%
  \lower.25em\hbox{Y}%
  \kern-.125em%
  X%
  \HOLOGO@SpaceFactor
}
%    \end{macrocode}
%    \end{macro}
%    \begin{macro}{\HoLogoHtml@LyX}
%    \begin{macrocode}
\def\HoLogoHtml@LyX#1{%
  \HoLogoCss@LyX
  \HOLOGO@Span{LyX}{%
    L%
    \HOLOGO@Span{y}{Y}%
    X%
  }%
}
%    \end{macrocode}
%    \end{macro}
%    \begin{macro}{\HoLogoCss@LyX}
%    \begin{macrocode}
\def\HoLogoCss@LyX{%
  \Css{%
    span.HoLogo-LyX span.HoLogo-y{%
      position:relative;%
      top:.25em;%
      margin-left:-.1667em;%
      margin-right:-.125em;%
      text-decoration:none;%
    }%
  }%
  \global\let\HoLogoCss@LyX\relax
}
%    \end{macrocode}
%    \end{macro}
%
% \subsubsection{\hologo{NTS}}
%
%    \begin{macro}{\HoLogo@NTS}
%    Definition for \hologo{NTS} can be found in
%    package \xpackage{etex\textunderscore man} for the \hologo{eTeX} manual \cite{etexman}
%    and in package \xpackage{dtklogos} \cite{dtklogos}:
%\begin{quote}
%\begin{verbatim}
%\def\NTS{%
%  \leavevmode
%  \hbox{%
%    $%
%      \cal N%
%      \kern-0.35em%
%      \lower0.5ex\hbox{$\cal T$}%
%      \kern-0.2em%
%      S%
%    $%
%  }%
%}
%\end{verbatim}
%\end{quote}
%    \begin{macrocode}
\def\HoLogo@NTS#1{%
  \HoLogoFont@font{NTS}{sy}{%
    N\/%
    \kern-.35em%
    \lower.5ex\hbox{T\/}%
    \kern-.2em%
    S\/%
  }%
  \HOLOGO@SpaceFactor
}
%    \end{macrocode}
%    \end{macro}
%
% \subsubsection{\Hologo{TTH} (\hologo{TeX} to HTML translator)}
%
%    Source: \url{http://hutchinson.belmont.ma.us/tth/}
%    In the HTML source the second `T' is printed as subscript.
%\begin{quote}
%\begin{verbatim}
%T<sub>T</sub>H
%\end{verbatim}
%\end{quote}
%    \begin{macro}{\HoLogo@TTH}
%    \begin{macrocode}
\def\HoLogo@TTH#1{%
  \ltx@mbox{%
    T\HOLOGO@SubScript{T}H%
  }%
  \HOLOGO@SpaceFactor
}
%    \end{macrocode}
%    \end{macro}
%
%    \begin{macro}{\HoLogoHtml@TTH}
%    \begin{macrocode}
\def\HoLogoHtml@TTH#1{%
  T\HCode{<sub>}T\HCode{</sub>}H%
}
%    \end{macrocode}
%    \end{macro}
%
% \subsubsection{\Hologo{HanTheThanh}}
%
%    Partial source: Package \xpackage{dtklogos}.
%    The double accent is U+1EBF (latin small letter e with circumflex
%    and acute).
%    \begin{macro}{\HoLogo@HanTheThanh}
%    \begin{macrocode}
\def\HoLogo@HanTheThanh#1{%
  \ltx@mbox{H\`an}%
  \HOLOGO@space
  \ltx@mbox{%
    Th%
    \HOLOGO@IfCharExists{"1EBF}{%
      \char"1EBF\relax
    }{%
      \^e\hbox to 0pt{\hss\raise .5ex\hbox{\'{}}}%
    }%
  }%
  \HOLOGO@space
  \ltx@mbox{Th\`anh}%
}
%    \end{macrocode}
%    \end{macro}
%    \begin{macro}{\HoLogoBkm@HanTheThanh}
%    \begin{macrocode}
\def\HoLogoBkm@HanTheThanh#1{%
  H\`an %
  Th\HOLOGO@PdfdocUnicode{\^e}{\9036\277} %
  Th\`anh%
}
%    \end{macrocode}
%    \end{macro}
%    \begin{macro}{\HoLogoHtml@HanTheThanh}
%    \begin{macrocode}
\def\HoLogoHtml@HanTheThanh#1{%
  H\`an %
  Th\HCode{&\ltx@hashchar x1ebf;} %
  Th\`anh%
}
%    \end{macrocode}
%    \end{macro}
%
% \subsection{Driver detection}
%
%    \begin{macrocode}
\HOLOGO@IfExists\InputIfFileExists{%
  \InputIfFileExists{hologo.cfg}{}{}%
}{%
  \ltx@IfUndefined{pdf@filesize}{%
    \def\HOLOGO@InputIfExists{%
      \openin\HOLOGO@temp=hologo.cfg\relax
      \ifeof\HOLOGO@temp
        \closein\HOLOGO@temp
      \else
        \closein\HOLOGO@temp
        \begingroup
          \def\x{LaTeX2e}%
        \expandafter\endgroup
        \ifx\fmtname\x
          \input{hologo.cfg}%
        \else
          \input hologo.cfg\relax
        \fi
      \fi
    }%
    \ltx@IfUndefined{newread}{%
      \chardef\HOLOGO@temp=15 %
      \def\HOLOGO@CheckRead{%
        \ifeof\HOLOGO@temp
          \HOLOGO@InputIfExists
        \else
          \ifcase\HOLOGO@temp
            \@PackageWarningNoLine{hologo}{%
              Configuration file ignored, because\MessageBreak
              a free read register could not be found%
            }%
          \else
            \begingroup
              \count\ltx@cclv=\HOLOGO@temp
              \advance\ltx@cclv by \ltx@minusone
              \edef\x{\endgroup
                \chardef\noexpand\HOLOGO@temp=\the\count\ltx@cclv
                \relax
              }%
            \x
          \fi
        \fi
      }%
    }{%
      \csname newread\endcsname\HOLOGO@temp
      \HOLOGO@InputIfExists
    }%
  }{%
    \edef\HOLOGO@temp{\pdf@filesize{hologo.cfg}}%
    \ifx\HOLOGO@temp\ltx@empty
    \else
      \ifnum\HOLOGO@temp>0 %
        \begingroup
          \def\x{LaTeX2e}%
        \expandafter\endgroup
        \ifx\fmtname\x
          \input{hologo.cfg}%
        \else
          \input hologo.cfg\relax
        \fi
      \else
        \@PackageInfoNoLine{hologo}{%
          Empty configuration file `hologo.cfg' ignored%
        }%
      \fi
    \fi
  }%
}
%    \end{macrocode}
%
%    \begin{macrocode}
\def\HOLOGO@temp#1#2{%
  \kv@define@key{HoLogoDriver}{#1}[]{%
    \begingroup
      \def\HOLOGO@temp{##1}%
      \ltx@onelevel@sanitize\HOLOGO@temp
      \ifx\HOLOGO@temp\ltx@empty
      \else
        \@PackageError{hologo}{%
          Value (\HOLOGO@temp) not permitted for option `#1'%
        }%
        \@ehc
      \fi
    \endgroup
    \def\hologoDriver{#2}%
  }%
}%
\def\HOLOGO@@temp#1#2{%
  \ifx\kv@value\relax
    \HOLOGO@temp{#1}{#1}%
  \else
    \HOLOGO@temp{#1}{#2}%
  \fi
}%
\kv@parse@normalized{%
  pdftex,%
  luatex=pdftex,%
  dvipdfm,%
  dvipdfmx=dvipdfm,%
  dvips,%
  dvipsone=dvips,%
  xdvi=dvips,%
  xetex,%
  vtex,%
}\HOLOGO@@temp
%    \end{macrocode}
%
%    \begin{macrocode}
\kv@define@key{HoLogoDriver}{driverfallback}{%
  \def\HOLOGO@DriverFallback{#1}%
}
%    \end{macrocode}
%
%    \begin{macro}{\HOLOGO@DriverFallback}
%    \begin{macrocode}
\def\HOLOGO@DriverFallback{dvips}
%    \end{macrocode}
%    \end{macro}
%
%    \begin{macro}{\hologoDriverSetup}
%    \begin{macrocode}
\def\hologoDriverSetup{%
  \let\hologoDriver\ltx@undefined
  \HOLOGO@DriverSetup
}
%    \end{macrocode}
%    \end{macro}
%
%    \begin{macro}{\HOLOGO@DriverSetup}
%    \begin{macrocode}
\def\HOLOGO@DriverSetup#1{%
  \kvsetkeys{HoLogoDriver}{#1}%
  \HOLOGO@CheckDriver
  \ltx@ifundefined{hologoDriver}{%
    \begingroup
    \edef\x{\endgroup
      \noexpand\kvsetkeys{HoLogoDriver}{\HOLOGO@DriverFallback}%
    }\x
  }{}%
  \@PackageInfoNoLine{hologo}{Using driver `\hologoDriver'}%
}
%    \end{macrocode}
%    \end{macro}
%
%    \begin{macro}{\HOLOGO@CheckDriver}
%    \begin{macrocode}
\def\HOLOGO@CheckDriver{%
  \ifpdf
    \def\hologoDriver{pdftex}%
    \let\HOLOGO@pdfliteral\pdfliteral
    \ifluatex
      \ifx\pdfextension\@undefined\else
        \protected\def\pdfliteral{\pdfextension literal}%
        \let\HOLOGO@pdfliteral\pdfliteral
      \fi
      \ltx@IfUndefined{HOLOGO@pdfliteral}{%
        \ifnum\luatexversion<36 %
        \else
          \begingroup
            \let\HOLOGO@temp\endgroup
            \ifcase0%
                \directlua{%
                  if tex.enableprimitives then %
                    tex.enableprimitives('HOLOGO@', {'pdfliteral'})%
                  else %
                    tex.print('1')%
                  end%
                }%
                \ifx\HOLOGO@pdfliteral\@undefined 1\fi%
                \relax%
              \endgroup
              \let\HOLOGO@temp\relax
              \global\let\HOLOGO@pdfliteral\HOLOGO@pdfliteral
            \fi%
          \HOLOGO@temp
        \fi
      }{}%
    \fi
    \ltx@IfUndefined{HOLOGO@pdfliteral}{%
      \@PackageWarningNoLine{hologo}{%
        Cannot find \string\pdfliteral
      }%
    }{}%
  \else
    \ifxetex
      \def\hologoDriver{xetex}%
    \else
      \ifvtex
        \def\hologoDriver{vtex}%
      \fi
    \fi
  \fi
}
%    \end{macrocode}
%    \end{macro}
%
%    \begin{macro}{\HOLOGO@WarningUnsupportedDriver}
%    \begin{macrocode}
\def\HOLOGO@WarningUnsupportedDriver#1{%
  \@PackageWarningNoLine{hologo}{%
    Logo `#1' needs driver specific macros,\MessageBreak
    but driver `\hologoDriver' is not supported.\MessageBreak
    Use a different driver or\MessageBreak
    load package `graphics' or `pgf'%
  }%
}
%    \end{macrocode}
%    \end{macro}
%
% \subsubsection{Reflect box macros}
%
%    Skip driver part if not needed.
%    \begin{macrocode}
\ltx@IfUndefined{reflectbox}{}{%
  \ltx@IfUndefined{rotatebox}{}{%
    \HOLOGO@AtEnd
  }%
}
\ltx@IfUndefined{pgftext}{}{%
  \HOLOGO@AtEnd
}
\ltx@IfUndefined{psscalebox}{}{%
  \HOLOGO@AtEnd
}
%    \end{macrocode}
%
%    \begin{macrocode}
\def\HOLOGO@temp{LaTeX2e}
\ifx\fmtname\HOLOGO@temp
  \RequirePackage{kvoptions}[2011/06/30]%
  \ProcessKeyvalOptions{HoLogoDriver}%
\fi
\HOLOGO@DriverSetup{}
%    \end{macrocode}
%
%    \begin{macro}{\HOLOGO@ReflectBox}
%    \begin{macrocode}
\def\HOLOGO@ReflectBox#1{%
  \begingroup
    \setbox\ltx@zero\hbox{\begingroup#1\endgroup}%
    \setbox\ltx@two\hbox{%
      \kern\wd\ltx@zero
      \csname HOLOGO@ScaleBox@\hologoDriver\endcsname{-1}{1}{%
        \hbox to 0pt{\copy\ltx@zero\hss}%
      }%
    }%
    \wd\ltx@two=\wd\ltx@zero
    \box\ltx@two
  \endgroup
}
%    \end{macrocode}
%    \end{macro}
%
%    \begin{macro}{\HOLOGO@PointReflectBox}
%    \begin{macrocode}
\def\HOLOGO@PointReflectBox#1{%
  \begingroup
    \setbox\ltx@zero\hbox{\begingroup#1\endgroup}%
    \setbox\ltx@two\hbox{%
      \kern\wd\ltx@zero
      \raise\ht\ltx@zero\hbox{%
        \csname HOLOGO@ScaleBox@\hologoDriver\endcsname{-1}{-1}{%
          \hbox to 0pt{\copy\ltx@zero\hss}%
        }%
      }%
    }%
    \wd\ltx@two=\wd\ltx@zero
    \box\ltx@two
  \endgroup
}
%    \end{macrocode}
%    \end{macro}
%
%    We must define all variants because of dynamic driver setup.
%    \begin{macrocode}
\def\HOLOGO@temp#1#2{#2}
%    \end{macrocode}
%
%    \begin{macro}{\HOLOGO@ScaleBox@pdftex}
%    \begin{macrocode}
\HOLOGO@temp{pdftex}{%
  \def\HOLOGO@ScaleBox@pdftex#1#2#3{%
    \HOLOGO@pdfliteral{%
      q #1 0 0 #2 0 0 cm%
    }%
    #3%
    \HOLOGO@pdfliteral{%
      Q%
    }%
  }%
}
%    \end{macrocode}
%    \end{macro}
%    \begin{macro}{\HOLOGO@ScaleBox@dvips}
%    \begin{macrocode}
\HOLOGO@temp{dvips}{%
  \def\HOLOGO@ScaleBox@dvips#1#2#3{%
    \special{ps:%
      gsave %
      currentpoint %
      currentpoint translate %
      #1 #2 scale %
      neg exch neg exch translate%
    }%
    #3%
    \special{ps:%
      currentpoint %
      grestore %
      moveto%
    }%
  }%
}
%    \end{macrocode}
%    \end{macro}
%    \begin{macro}{\HOLOGO@ScaleBox@dvipdfm}
%    \begin{macrocode}
\HOLOGO@temp{dvipdfm}{%
  \let\HOLOGO@ScaleBox@dvipdfm\HOLOGO@ScaleBox@dvips
}
%    \end{macrocode}
%    \end{macro}
%    Since \hologo{XeTeX} v0.6.
%    \begin{macro}{\HOLOGO@ScaleBox@xetex}
%    \begin{macrocode}
\HOLOGO@temp{xetex}{%
  \def\HOLOGO@ScaleBox@xetex#1#2#3{%
    \special{x:gsave}%
    \special{x:scale #1 #2}%
    #3%
    \special{x:grestore}%
  }%
}
%    \end{macrocode}
%    \end{macro}
%    \begin{macro}{\HOLOGO@ScaleBox@vtex}
%    \begin{macrocode}
\HOLOGO@temp{vtex}{%
  \def\HOLOGO@ScaleBox@vtex#1#2#3{%
    \special{r(#1,0,0,#2,0,0}%
    #3%
    \special{r)}%
  }%
}
%    \end{macrocode}
%    \end{macro}
%
%    \begin{macrocode}
\HOLOGO@AtEnd%
%</package>
%    \end{macrocode}
%
% \section{Test}
%
% \subsection{Catcode checks for loading}
%
%    \begin{macrocode}
%<*test1>
%    \end{macrocode}
%    \begin{macrocode}
\catcode`\{=1 %
\catcode`\}=2 %
\catcode`\#=6 %
\catcode`\@=11 %
\expandafter\ifx\csname count@\endcsname\relax
  \countdef\count@=255 %
\fi
\expandafter\ifx\csname @gobble\endcsname\relax
  \long\def\@gobble#1{}%
\fi
\expandafter\ifx\csname @firstofone\endcsname\relax
  \long\def\@firstofone#1{#1}%
\fi
\expandafter\ifx\csname loop\endcsname\relax
  \expandafter\@firstofone
\else
  \expandafter\@gobble
\fi
{%
  \def\loop#1\repeat{%
    \def\body{#1}%
    \iterate
  }%
  \def\iterate{%
    \body
      \let\next\iterate
    \else
      \let\next\relax
    \fi
    \next
  }%
  \let\repeat=\fi
}%
\def\RestoreCatcodes{}
\count@=0 %
\loop
  \edef\RestoreCatcodes{%
    \RestoreCatcodes
    \catcode\the\count@=\the\catcode\count@\relax
  }%
\ifnum\count@<255 %
  \advance\count@ 1 %
\repeat

\def\RangeCatcodeInvalid#1#2{%
  \count@=#1\relax
  \loop
    \catcode\count@=15 %
  \ifnum\count@<#2\relax
    \advance\count@ 1 %
  \repeat
}
\def\RangeCatcodeCheck#1#2#3{%
  \count@=#1\relax
  \loop
    \ifnum#3=\catcode\count@
    \else
      \errmessage{%
        Character \the\count@\space
        with wrong catcode \the\catcode\count@\space
        instead of \number#3%
      }%
    \fi
  \ifnum\count@<#2\relax
    \advance\count@ 1 %
  \repeat
}
\def\space{ }
\expandafter\ifx\csname LoadCommand\endcsname\relax
  \def\LoadCommand{\input hologo.sty\relax}%
\fi
\def\Test{%
  \RangeCatcodeInvalid{0}{47}%
  \RangeCatcodeInvalid{58}{64}%
  \RangeCatcodeInvalid{91}{96}%
  \RangeCatcodeInvalid{123}{255}%
  \catcode`\@=12 %
  \catcode`\\=0 %
  \catcode`\%=14 %
  \LoadCommand
  \RangeCatcodeCheck{0}{36}{15}%
  \RangeCatcodeCheck{37}{37}{14}%
  \RangeCatcodeCheck{38}{47}{15}%
  \RangeCatcodeCheck{48}{57}{12}%
  \RangeCatcodeCheck{58}{63}{15}%
  \RangeCatcodeCheck{64}{64}{12}%
  \RangeCatcodeCheck{65}{90}{11}%
  \RangeCatcodeCheck{91}{91}{15}%
  \RangeCatcodeCheck{92}{92}{0}%
  \RangeCatcodeCheck{93}{96}{15}%
  \RangeCatcodeCheck{97}{122}{11}%
  \RangeCatcodeCheck{123}{255}{15}%
  \RestoreCatcodes
}
\Test
\csname @@end\endcsname
\end
%    \end{macrocode}
%    \begin{macrocode}
%</test1>
%    \end{macrocode}
%
% \subsection{Spacefactor}
%
%    The space factor must be 1000 after a logo. If it is greater 1000
%    then the following space is a space after a sentence closing point.
%    If the space factor is smaller 1000 then an immediate following
%    dot is interpreted as abbreviation, not sentence closing point.
%
%    \begin{macrocode}
%<*test-spacefactor>
\NeedsTeXFormat{LaTeX2e}
\documentclass{article}
\usepackage{hologo}[2016/05/12]
\usepackage{kvsetkeys}
\usepackage{qstest}
\IncludeTests{*}
\LogTests{log}{*}{*}
\begin{document}
\begin{qstest}{spacefactor}{spacefactor}
\newcommand*{\Test}[1]{%
  \sbox0{%
    \hologo{#1}%
    \Expect*{1000 (#1)}*{\the\spacefactor\space(#1)}%
  }%
}%
\makeatletter
\def\TestList{}
\def\hologoEntry#1#2#3{%
  \edef\TestList{%
    \ifx\TestList\@empty
    \else
      \TestList,%
    \fi
    #1%
    \ifx\\#2\\%
    \else
      ={variant=#2}%
    \fi
  }%
}
\hologoList
\expandafter\kv@parse@normalized\expandafter{%
  \TestList
}{%
  \begingroup
    \let\@logo=\kv@key
    \ifx\kv@value\relax
    \else
      \expandafter\hologoLogoSetup\expandafter\@logo\expandafter{%
        \kv@value
      }%
    \fi
    \Test\@logo
  \endgroup
  \@gobbletwo
}
\end{qstest}
\end{document}
%</test-spacefactor>
%    \end{macrocode}
%
% \subsection{Complete list}
%
%    \begin{macrocode}
%<*test-list>
\NeedsTeXFormat{LaTeX2e}
\documentclass[12pt,a4paper]{article}
\usepackage{hologo}[2016/05/12]
\usepackage[T1]{fontenc}
\usepackage{lmodern}
\usepackage{parskip}
\usepackage[unicode]{hyperref}[2011/09/28]
\usepackage{bookmark}[2011/09/19]
\bookmarksetup{%
  numbered,%
  open,%
  openlevel=2,%
}
\renewcommand*{\contentsname}{List of logos}
\begin{document}
\tableofcontents
\def\TestFont#1#2#3#4#5#6{%
  \begingroup
    \usefont{#3}{#4}{#5}{#6}%
    \HologoVariant{#1}{#2}/\hologoVariant{#1}{#2}%
    \quad
    \begingroup\scriptsize\hologoVariant{#1}{#2}\endgroup
    \quad
  \endgroup
  (#3/#4/#5/#6)%
  \par
}
\makeatletter
\def\hologoEntry#1#2#3{%
  \section{%
    \HologoVariant{#1}{#2}/\hologoVariant{#1}{#2} %
    {[#1\ifx\\#2\\\else\space(#2)\fi]}% hash-ok
  }% braces around [] because of bug in tex4ht
  \begingroup
    \hypersetup{unicode=false}%
    \bookmark[%
      dest=\@currentHref,%
      rellevel=1,%
      keeplevel,%
    ]{%
      \HologoVariant{#1}{#2}/\hologoVariant{#1}{#2} %
      (PDFDocEncoding)%
    }%
  \endgroup
  \TestFont{#1}{#2}{OT1}{cmr}{m}{n}%
  \TestFont{#1}{#2}{OT1}{cmss}{m}{n}%
  \TestFont{#1}{#2}{OT1}{cmr}{b}{n}%
  \TestFont{#1}{#2}{OT1}{cmr}{m}{it}%
  \TestFont{#1}{#2}{OT1}{cmtt}{m}{n}%
  \TestFont{#1}{#2}{T1}{lmr}{m}{n}%
  \TestFont{#1}{#2}{T1}{lmss}{m}{n}%
  \TestFont{#1}{#2}{T1}{lmr}{b}{n}%
  \TestFont{#1}{#2}{T1}{lmr}{m}{it}%
  \TestFont{#1}{#2}{T1}{lmtt}{m}{n}%
  \TestFont{#1}{#2}{T1}{lmvtt}{m}{n}%
  \TestFont{#1}{#2}{T1}{qtm}{m}{n}%
  \TestFont{#1}{#2}{T1}{qhv}{m}{n}%
  \TestFont{#1}{#2}{T1}{qtm}{b}{n}%
  \TestFont{#1}{#2}{T1}{qtm}{m}{it}%
  \TestFont{#1}{#2}{T1}{qcr}{m}{n}%
  \newpage
}
\makeatother
\hologoList
\end{document}
%</test-list>
%    \end{macrocode}
%
% \section{Installation}
%
% \subsection{Download}
%
% \paragraph{Package.} This package is available on
% CTAN\footnote{\url{ftp://ftp.ctan.org/tex-archive/}}:
% \begin{description}
% \item[\CTAN{macros/latex/contrib/oberdiek/hologo.dtx}] The source file.
% \item[\CTAN{macros/latex/contrib/oberdiek/hologo.pdf}] Documentation.
% \end{description}
%
%
% \paragraph{Bundle.} All the packages of the bundle `oberdiek'
% are also available in a TDS compliant ZIP archive. There
% the packages are already unpacked and the documentation files
% are generated. The files and directories obey the TDS standard.
% \begin{description}
% \item[\CTAN{install/macros/latex/contrib/oberdiek.tds.zip}]
% \end{description}
% \emph{TDS} refers to the standard ``A Directory Structure
% for \TeX\ Files'' (\CTAN{tds/tds.pdf}). Directories
% with \xfile{texmf} in their name are usually organized this way.
%
% \subsection{Bundle installation}
%
% \paragraph{Unpacking.} Unpack the \xfile{oberdiek.tds.zip} in the
% TDS tree (also known as \xfile{texmf} tree) of your choice.
% Example (linux):
% \begin{quote}
%   |unzip oberdiek.tds.zip -d ~/texmf|
% \end{quote}
%
% \paragraph{Script installation.}
% Check the directory \xfile{TDS:scripts/oberdiek/} for
% scripts that need further installation steps.
% Package \xpackage{attachfile2} comes with the Perl script
% \xfile{pdfatfi.pl} that should be installed in such a way
% that it can be called as \texttt{pdfatfi}.
% Example (linux):
% \begin{quote}
%   |chmod +x scripts/oberdiek/pdfatfi.pl|\\
%   |cp scripts/oberdiek/pdfatfi.pl /usr/local/bin/|
% \end{quote}
%
% \subsection{Package installation}
%
% \paragraph{Unpacking.} The \xfile{.dtx} file is a self-extracting
% \docstrip\ archive. The files are extracted by running the
% \xfile{.dtx} through \plainTeX:
% \begin{quote}
%   \verb|tex hologo.dtx|
% \end{quote}
%
% \paragraph{TDS.} Now the different files must be moved into
% the different directories in your installation TDS tree
% (also known as \xfile{texmf} tree):
% \begin{quote}
% \def\t{^^A
% \begin{tabular}{@{}>{\ttfamily}l@{ $\rightarrow$ }>{\ttfamily}l@{}}
%   hologo.sty & tex/generic/oberdiek/hologo.sty\\
%   hologo.pdf & doc/latex/oberdiek/hologo.pdf\\
%   example/hologo-example.tex & doc/latex/oberdiek/example/hologo-example.tex\\
%   test/hologo-test1.tex & doc/latex/oberdiek/test/hologo-test1.tex\\
%   test/hologo-test-spacefactor.tex & doc/latex/oberdiek/test/hologo-test-spacefactor.tex\\
%   test/hologo-test-list.tex & doc/latex/oberdiek/test/hologo-test-list.tex\\
%   hologo.dtx & source/latex/oberdiek/hologo.dtx\\
% \end{tabular}^^A
% }^^A
% \sbox0{\t}^^A
% \ifdim\wd0>\linewidth
%   \begingroup
%     \advance\linewidth by\leftmargin
%     \advance\linewidth by\rightmargin
%   \edef\x{\endgroup
%     \def\noexpand\lw{\the\linewidth}^^A
%   }\x
%   \def\lwbox{^^A
%     \leavevmode
%     \hbox to \linewidth{^^A
%       \kern-\leftmargin\relax
%       \hss
%       \usebox0
%       \hss
%       \kern-\rightmargin\relax
%     }^^A
%   }^^A
%   \ifdim\wd0>\lw
%     \sbox0{\small\t}^^A
%     \ifdim\wd0>\linewidth
%       \ifdim\wd0>\lw
%         \sbox0{\footnotesize\t}^^A
%         \ifdim\wd0>\linewidth
%           \ifdim\wd0>\lw
%             \sbox0{\scriptsize\t}^^A
%             \ifdim\wd0>\linewidth
%               \ifdim\wd0>\lw
%                 \sbox0{\tiny\t}^^A
%                 \ifdim\wd0>\linewidth
%                   \lwbox
%                 \else
%                   \usebox0
%                 \fi
%               \else
%                 \lwbox
%               \fi
%             \else
%               \usebox0
%             \fi
%           \else
%             \lwbox
%           \fi
%         \else
%           \usebox0
%         \fi
%       \else
%         \lwbox
%       \fi
%     \else
%       \usebox0
%     \fi
%   \else
%     \lwbox
%   \fi
% \else
%   \usebox0
% \fi
% \end{quote}
% If you have a \xfile{docstrip.cfg} that configures and enables \docstrip's
% TDS installing feature, then some files can already be in the right
% place, see the documentation of \docstrip.
%
% \subsection{Refresh file name databases}
%
% If your \TeX~distribution
% (\teTeX, \mikTeX, \dots) relies on file name databases, you must refresh
% these. For example, \teTeX\ users run \verb|texhash| or
% \verb|mktexlsr|.
%
% \subsection{Some details for the interested}
%
% \paragraph{Attached source.}
%
% The PDF documentation on CTAN also includes the
% \xfile{.dtx} source file. It can be extracted by
% AcrobatReader 6 or higher. Another option is \textsf{pdftk},
% e.g. unpack the file into the current directory:
% \begin{quote}
%   \verb|pdftk hologo.pdf unpack_files output .|
% \end{quote}
%
% \paragraph{Unpacking with \LaTeX.}
% The \xfile{.dtx} chooses its action depending on the format:
% \begin{description}
% \item[\plainTeX:] Run \docstrip\ and extract the files.
% \item[\LaTeX:] Generate the documentation.
% \end{description}
% If you insist on using \LaTeX\ for \docstrip\ (really,
% \docstrip\ does not need \LaTeX), then inform the autodetect routine
% about your intention:
% \begin{quote}
%   \verb|latex \let\install=y\input{hologo.dtx}|
% \end{quote}
% Do not forget to quote the argument according to the demands
% of your shell.
%
% \paragraph{Generating the documentation.}
% You can use both the \xfile{.dtx} or the \xfile{.drv} to generate
% the documentation. The process can be configured by the
% configuration file \xfile{ltxdoc.cfg}. For instance, put this
% line into this file, if you want to have A4 as paper format:
% \begin{quote}
%   \verb|\PassOptionsToClass{a4paper}{article}|
% \end{quote}
% An example follows how to generate the
% documentation with pdf\LaTeX:
% \begin{quote}
%\begin{verbatim}
%pdflatex hologo.dtx
%makeindex -s gind.ist hologo.idx
%pdflatex hologo.dtx
%makeindex -s gind.ist hologo.idx
%pdflatex hologo.dtx
%\end{verbatim}
% \end{quote}
%
% \section{Catalogue}
%
% The following XML file can be used as source for the
% \href{http://mirror.ctan.org/help/Catalogue/catalogue.html}{\TeX\ Catalogue}.
% The elements \texttt{caption} and \texttt{description} are imported
% from the original XML file from the Catalogue.
% The name of the XML file in the Catalogue is \xfile{hologo.xml}.
%    \begin{macrocode}
%<*catalogue>
<?xml version='1.0' encoding='us-ascii'?>
<!DOCTYPE entry SYSTEM 'catalogue.dtd'>
<entry datestamp='$Date$' modifier='$Author$' id='hologo'>
  <name>hologo</name>
  <caption>A collection of logos with bookmark support.</caption>
  <authorref id='auth:oberdiek'/>
  <copyright owner='Heiko Oberdiek' year='2010-2012'/>
  <license type='lppl1.3'/>
  <version number='1.10'/>
  <description>
    The package defines a single command <tt>\hologo</tt>, whose
    argument is the usual case-confused ASCII version of the logo.
    The command is bookmark-enabled, so that every logo becomes
    available in bookmarks without further work.
    <p/>
    The package is part of the <xref refid='oberdiek'>oberdiek</xref>
    bundle.
  </description>
  <documentation details='Package documentation'
      href='ctan:/macros/latex/contrib/oberdiek/hologo.pdf'/>
  <ctan file='true' path='/macros/latex/contrib/oberdiek/hologo.dtx'/>
  <miktex location='oberdiek'/>
  <texlive location='oberdiek'/>
  <install path='/macros/latex/contrib/oberdiek/oberdiek.tds.zip'/>
</entry>
%</catalogue>
%    \end{macrocode}
%
% \begin{thebibliography}{9}
% \raggedright
%
% \bibitem{btxdoc}
% Oren Patashnik,
% \textit{\hologo{BibTeX}ing},
% 1988-02-08.\\
% \CTAN{biblio/bibtex/base/}
%
% \bibitem{dtklogos}
% Gerd Neugebauer, DANTE,
% \textit{Package \xpackage{dtklogos}},
% 2011-04-25.\\
% \CTAN{usergrps/dante/dtk/dtklogos.sty}
%
% \bibitem{etexman}
% The \hologo{NTS} Team,
% \textit{The \hologo{eTeX} manual},
% 1998-02.\\
% \CTAN{systems/e-tex/v2/doc/}
%
% \bibitem{ExTeX-FAQ}
% The \hologo{ExTeX} group,
% \textit{\hologo{ExTeX}: FAQ -- How is \hologo{ExTeX} typeset?},
% 2007-04-14.\\
% \url{http://www.extex.org/documentation/faq.html}
%
% \bibitem{LyX}
% %@MISC{ LyX,
% %  title = {{LyX 2.0.0 -- The Document Processor [Computer software and manual]}},
% %  author = {{The LyX Team}},
% %  howpublished = {Internet: http://www.lyx.org},
% %  year = {2011-05-08},
% %  note = {Retrieved May 10, 2011, from http://www.lyx.org},
% %  url = {http://www.lyx.org/}
% %}
% The \hologo{LyX} Team,
% \textit{\hologo{LyX} -- The Document Processor},
% 2011-05-08.\\
% \url{http://www.lyx.org/}
%
% \bibitem{OzTeX}
% Andrew Trevorrow,
% \hologo{OzTeX} FAQ: What is the correct way to typeset ``\hologo{OzTeX}''?,
% 2011-09-15 (visited).
% \url{http://www.trevorrow.com/oztex/ozfaq.html#oztex-logo}
%
% \bibitem{PiCTeX}
% Michael Wichura,
% \textit{The \hologo{PiCTeX} macro package},
% 1987-09-21.
% \CTAN{graphics/pictex/}
%
% \bibitem{scrlogo}
% Markus Kohm,
% \textit{\hologo{KOMAScript} Datei \xfile{scrlogo.dtx}},
% 2009-01-30.\\
% \CTAN{install/macros/latex/contrib/komascript.tds.zip}
%
% \end{thebibliography}
%
% \begin{History}
%   \begin{Version}{2010/04/08 v1.0}
%   \item
%     The first version.
%   \end{Version}
%   \begin{Version}{2010/04/16 v1.1}
%   \item
%     \cs{Hologo} added for support of logos at start of a sentence.
%   \item
%     \cs{hologoSetup} and \cs{hologoLogoSetup} added.
%   \item
%     Options \xoption{break}, \xoption{hyphenbreak}, \xoption{spacebreak}
%     added.
%   \item
%     Variant support added by option \xoption{variant}.
%   \end{Version}
%   \begin{Version}{2010/04/24 v1.2}
%   \item
%     \hologo{LaTeX3} added.
%   \item
%     \hologo{VTeX} added.
%   \end{Version}
%   \begin{Version}{2010/11/21 v1.3}
%   \item
%     \hologo{iniTeX}, \hologo{virTeX} added.
%   \end{Version}
%   \begin{Version}{2011/03/25 v1.4}
%   \item
%     \hologo{ConTeXt} with variants added.
%   \item
%     Option \xoption{discretionarybreak} added as refinement for
%     option \xoption{break}.
%   \end{Version}
%   \begin{Version}{2011/04/21 v1.5}
%   \item
%     Wrong TDS directory for test files fixed.
%   \end{Version}
%   \begin{Version}{2011/10/01 v1.6}
%   \item
%     Support for package \xpackage{tex4ht} added.
%   \item
%     Support for \cs{csname} added if \cs{ifincsname} is available.
%   \item
%     New logos:
%     \hologo{(La)TeX},
%     \hologo{biber},
%     \hologo{BibTeX} (\xoption{sc}, \xoption{sf}),
%     \hologo{emTeX},
%     \hologo{ExTeX},
%     \hologo{KOMAScript},
%     \hologo{La},
%     \hologo{LyX},
%     \hologo{MiKTeX},
%     \hologo{NTS},
%     \hologo{OzMF},
%     \hologo{OzMP},
%     \hologo{OzTeX},
%     \hologo{OzTtH},
%     \hologo{PCTeX},
%     \hologo{PiC},
%     \hologo{PiCTeX},
%     \hologo{METAFONT},
%     \hologo{MetaFun},
%     \hologo{METAPOST},
%     \hologo{MetaPost},
%     \hologo{SLiTeX} (\xoption{lift}, \xoption{narrow}, \xoption{simple}),
%     \hologo{SliTeX} (\xoption{narrow}, \xoption{simple}, \xoption{lift}),
%     \hologo{teTeX}.
%   \item
%     Fixes:
%     \hologo{iniTeX},
%     \hologo{pdfLaTeX},
%     \hologo{pdfTeX},
%     \hologo{virTeX}.
%   \item
%     \cs{hologoFontSetup} and \cs{hologoLogoFontSetup} added.
%   \item
%     \cs{hologoVariant} and \cs{HologoVariant} added.
%   \end{Version}
%   \begin{Version}{2011/11/22 v1.7}
%   \item
%     New logos:
%     \hologo{BibTeX8},
%     \hologo{LaTeXML},
%     \hologo{SageTeX},
%     \hologo{TeX4ht},
%     \hologo{TTH}.
%   \item
%     \hologo{Xe} and friends: Driver stuff fixed.
%   \item
%     \hologo{Xe} and friends: Support for italic added.
%   \item
%     \hologo{Xe} and friends: Package support for \xpackage{pgf}
%     and \xpackage{pstricks} added.
%   \end{Version}
%   \begin{Version}{2011/11/29 v1.8}
%   \item
%     New logos:
%     \hologo{HanTheThanh}.
%   \end{Version}
%   \begin{Version}{2011/12/21 v1.9}
%   \item
%     Patch for package \xpackage{ifxetex} added for the case that
%     \cs{newif} is undefined in \hologo{iniTeX}.
%   \item
%     Some fixes for \hologo{iniTeX}.
%   \end{Version}
%   \begin{Version}{2012/04/26 v1.10}
%   \item
%     Fix in bookmark version of logo ``\hologo{HanTheThanh}''.
%   \end{Version}
%   \begin{Version}{2016/05/12 v1.11}
%   \item
%     Update HOLOGO@IfCharExists (previously in texlive)
%   \item define pdfliteral in current luatex.
%   \end{Version}
% \end{History}
%
% \PrintIndex
%
% \Finale
\endinput

%        (quote the arguments according to the demands of your shell)
%
% Documentation:
%    (a) If hologo.drv is present:
%           latex hologo.drv
%    (b) Without hologo.drv:
%           latex hologo.dtx; ...
%    The class ltxdoc loads the configuration file ltxdoc.cfg
%    if available. Here you can specify further options, e.g.
%    use A4 as paper format:
%       \PassOptionsToClass{a4paper}{article}
%
%    Programm calls to get the documentation (example):
%       pdflatex hologo.dtx
%       makeindex -s gind.ist hologo.idx
%       pdflatex hologo.dtx
%       makeindex -s gind.ist hologo.idx
%       pdflatex hologo.dtx
%
% Installation:
%    TDS:tex/generic/oberdiek/hologo.sty
%    TDS:doc/latex/oberdiek/hologo.pdf
%    TDS:doc/latex/oberdiek/example/hologo-example.tex
%    TDS:doc/latex/oberdiek/test/hologo-test1.tex
%    TDS:doc/latex/oberdiek/test/hologo-test-spacefactor.tex
%    TDS:doc/latex/oberdiek/test/hologo-test-list.tex
%    TDS:source/latex/oberdiek/hologo.dtx
%
%<*ignore>
\begingroup
  \catcode123=1 %
  \catcode125=2 %
  \def\x{LaTeX2e}%
\expandafter\endgroup
\ifcase 0\ifx\install y1\fi\expandafter
         \ifx\csname processbatchFile\endcsname\relax\else1\fi
         \ifx\fmtname\x\else 1\fi\relax
\else\csname fi\endcsname
%</ignore>
%<*install>
\input docstrip.tex
\Msg{************************************************************************}
\Msg{* Installation}
\Msg{* Package: hologo 2016/05/12 v1.11 A logo collection with bookmark support (HO)}
\Msg{************************************************************************}

\keepsilent
\askforoverwritefalse

\let\MetaPrefix\relax
\preamble

This is a generated file.

Project: hologo
Version: 2016/05/12 v1.11

Copyright (C) 2010-2012 by
   Heiko Oberdiek <heiko.oberdiek at googlemail.com>

This work may be distributed and/or modified under the
conditions of the LaTeX Project Public License, either
version 1.3c of this license or (at your option) any later
version. This version of this license is in
   http://www.latex-project.org/lppl/lppl-1-3c.txt
and the latest version of this license is in
   http://www.latex-project.org/lppl.txt
and version 1.3 or later is part of all distributions of
LaTeX version 2005/12/01 or later.

This work has the LPPL maintenance status "maintained".

This Current Maintainer of this work is Heiko Oberdiek.

The Base Interpreter refers to any `TeX-Format',
because some files are installed in TDS:tex/generic//.

This work consists of the main source file hologo.dtx
and the derived files
   hologo.sty, hologo.pdf, hologo.ins, hologo.drv, hologo-example.tex,
   hologo-test1.tex, hologo-test-spacefactor.tex,
   hologo-test-list.tex.

\endpreamble
\let\MetaPrefix\DoubleperCent

\generate{%
  \file{hologo.ins}{\from{hologo.dtx}{install}}%
  \file{hologo.drv}{\from{hologo.dtx}{driver}}%
  \usedir{tex/generic/oberdiek}%
  \file{hologo.sty}{\from{hologo.dtx}{package}}%
  \usedir{doc/latex/oberdiek/example}%
  \file{hologo-example.tex}{\from{hologo.dtx}{example}}%
  \usedir{doc/latex/oberdiek/test}%
  \file{hologo-test1.tex}{\from{hologo.dtx}{test1}}%
  \file{hologo-test-spacefactor.tex}{\from{hologo.dtx}{test-spacefactor}}%
  \file{hologo-test-list.tex}{\from{hologo.dtx}{test-list}}%
  \nopreamble
  \nopostamble
  \usedir{source/latex/oberdiek/catalogue}%
  \file{hologo.xml}{\from{hologo.dtx}{catalogue}}%
}

\catcode32=13\relax% active space
\let =\space%
\Msg{************************************************************************}
\Msg{*}
\Msg{* To finish the installation you have to move the following}
\Msg{* file into a directory searched by TeX:}
\Msg{*}
\Msg{*     hologo.sty}
\Msg{*}
\Msg{* To produce the documentation run the file `hologo.drv'}
\Msg{* through LaTeX.}
\Msg{*}
\Msg{* Happy TeXing!}
\Msg{*}
\Msg{************************************************************************}

\endbatchfile
%</install>
%<*ignore>
\fi
%</ignore>
%<*driver>
\NeedsTeXFormat{LaTeX2e}
\ProvidesFile{hologo.drv}%
  [2016/05/12 v1.11 A logo collection with bookmark support (HO)]%
\documentclass{ltxdoc}
\usepackage{holtxdoc}[2011/11/22]
\usepackage{hologo}[2016/05/12]
\usepackage{longtable}
\usepackage{array}
\usepackage{paralist}
%\usepackage[T1]{fontenc}
%\usepackage{lmodern}
\begin{document}
  \DocInput{hologo.dtx}%
\end{document}
%</driver>
% \fi
%
%
% \CharacterTable
%  {Upper-case    \A\B\C\D\E\F\G\H\I\J\K\L\M\N\O\P\Q\R\S\T\U\V\W\X\Y\Z
%   Lower-case    \a\b\c\d\e\f\g\h\i\j\k\l\m\n\o\p\q\r\s\t\u\v\w\x\y\z
%   Digits        \0\1\2\3\4\5\6\7\8\9
%   Exclamation   \!     Double quote  \"     Hash (number) \#
%   Dollar        \$     Percent       \%     Ampersand     \&
%   Acute accent  \'     Left paren    \(     Right paren   \)
%   Asterisk      \*     Plus          \+     Comma         \,
%   Minus         \-     Point         \.     Solidus       \/
%   Colon         \:     Semicolon     \;     Less than     \<
%   Equals        \=     Greater than  \>     Question mark \?
%   Commercial at \@     Left bracket  \[     Backslash     \\
%   Right bracket \]     Circumflex    \^     Underscore    \_
%   Grave accent  \`     Left brace    \{     Vertical bar  \|
%   Right brace   \}     Tilde         \~}
%
% \GetFileInfo{hologo.drv}
%
% \title{The \xpackage{hologo} package}
% \date{2016/05/12 v1.11}
% \author{Heiko Oberdiek\\\xemail{heiko.oberdiek at googlemail.com}}
%
% \maketitle
%
% \begin{abstract}
% This package starts a collection of logos with support for bookmarks
% strings.
% \end{abstract}
%
% \tableofcontents
%
% \section{Documentation}
%
% \subsection{Logo macros}
%
% \begin{declcs}{hologo} \M{name}
% \end{declcs}
% Macro \cs{hologo} sets the logo with name \meta{name}.
% The following table shows the supported names.
%
% \begingroup
%   \def\hologoEntry#1#2#3{^^A
%     #1&#2&\hologoLogoSetup{#1}{variant=#2}\hologo{#1}&#3\tabularnewline
%   }
%   \begin{longtable}{>{\ttfamily}l>{\ttfamily}lll}
%     \rmfamily\bfseries{name} & \rmfamily\bfseries variant
%     & \bfseries logo & \bfseries since\\
%     \hline
%     \endhead
%     \hologoList
%   \end{longtable}
% \endgroup
%
% \begin{declcs}{Hologo} \M{name}
% \end{declcs}
% Macro \cs{Hologo} starts the logo \meta{name} with an uppercase
% letter. As an exception small greek letters are not converted
% to uppercase. Examples, see \hologo{eTeX} and \hologo{ExTeX}.
%
% \subsection{Setup macros}
%
% The package does not support package options, but the following
% setup macros can be used to set options.
%
% \begin{declcs}{hologoSetup} \M{key value list}
% \end{declcs}
% Macro \cs{hologoSetup} sets global options.
%
% \begin{declcs}{hologoLogoSetup} \M{logo} \M{key value list}
% \end{declcs}
% Some options can also be used to configure a logo.
% These settings take precedence over global option settings.
%
% \subsection{Options}\label{sec:options}
%
% There are boolean and string options:
% \begin{description}
% \item[Boolean option:]
% It takes |true| or |false|
% as value. If the value is omitted, then |true| is used.
% \item[String option:]
% A value must be given as string. (But the string might be empty.)
% \end{description}
% The following options can be used both in \cs{hologoSetup}
% and \cs{hologoLogoSetup}:
% \begin{description}
% \def\entry#1{\item[\xoption{#1}:]}
% \entry{break}
%   enables or disables line breaks inside the logo. This setting is
%   refined by options \xoption{hyphenbreak}, \xoption{spacebreak}
%   or \xoption{discretionarybreak}.
%   Default is |false|.
% \entry{hyphenbreak}
%   enables or disables the line break right after the hyphen character.
% \entry{spacebreak}
%   enables or disables line breaks at space characters.
% \entry{discretionarybreak}
%   enables or disables line breaks at hyphenation points
%   (inserted by \cs{-}).
% \end{description}
% Macro \cs{hologoLogoSetup} also knows:
% \begin{description}
% \item[\xoption{variant}:]
%   This is a string option. It specifies a variant of a logo that
%   must exist. An empty string selects the package default variant.
% \end{description}
% Example:
% \begin{quote}
%   |\hologoSetup{break=false}|\\
%   |\hologoLogoSetup{plainTeX}{variant=hyphen,hyphenbreak}|\\
%   Then ``plain-\TeX'' contains one break point after the hyphen.
% \end{quote}
%
% \subsection{Driver options}
%
% Sometimes graphical operations are needed to construct some
% glyphs (e.g.\ \hologo{XeTeX}). If package \xpackage{graphics}
% or package \xpackage{pgf} are found, then the macros are taken
% from there. Otherwise the packge defines its own operations
% and therefore needs the driver information. Many drivers are
% detected automatically (\hologo{pdfTeX}/\hologo{LuaTeX}
% in PDF mode, \hologo{XeTeX}, \hologo{VTeX}). These have precedence
% over a driver option. The driver can be given as package option
% or using \cs{hologoDriverSetup}.
% The following list contains the recognized driver options:
% \begin{itemize}
% \item \xoption{pdftex}, \xoption{luatex}
% \item \xoption{dvipdfm}, \xoption{dvipdfmx}
% \item \xoption{dvips}, \xoption{dvipsone}, \xoption{xdvi}
% \item \xoption{xetex}
% \item \xoption{vtex}
% \end{itemize}
% The left driver of a line is the driver name that is used internally.
% The following names are aliases for drivers that use the
% same method. Therefore the entry in the \xext{log} file for
% the used driver prints the internally used driver name.
% \begin{description}
% \item[\xoption{driverfallback}:]
%   This option expects a driver that is used,
%   if the driver could not be detected automatically.
% \end{description}
%
% \begin{declcs}{hologoDriverSetup} \M{driver option}
% \end{declcs}
% The driver can also be configured after package loading
% using \cs{hologoDriverSetup}, also the way for \hologo{plainTeX}
% to setup the driver.
%
% \subsection{Font setup}
%
% Some logos require a special font, but should also be usable by
% \hologo{plainTeX}. Therefore the package provides some ways
% to influence the font settings. The options below
% take font settings as values. Both font commands
% such as \cs{sffamily} and macros that take one argument
% like \cs{textsf} can be used.
%
% \begin{declcs}{hologoFontSetup} \M{key value list}
% \end{declcs}
% Macro \cs{hologoFontSetup} sets the fonts for all logos.
% Supported keys:
% \begin{description}
% \def\entry#1{\item[\xoption{#1}:]}
% \entry{general}
%   This font is used for all logos. The default is empty.
%   That means no special font is used.
% \entry{bibsf}
%   This font is used for
%   {\hologoLogoSetup{BibTeX}{variant=sf}\hologo{BibTeX}}
%   with variant \xoption{sf}.
% \entry{rm}
%   This font is a serif font. It is used for \hologo{ExTeX}.
% \entry{sc}
%   This font specifies a small caps font. It is used for
%   {\hologoLogoSetup{BibTeX}{variant=sc}\hologo{BibTeX}}
%   with variant \xoption{sc}.
% \entry{sf}
%   This font specifies a sans serif font. The default
%   is \cs{sffamily}, then \cs{sf} is tried. Otherwise
%   a warning is given. It is used by \hologo{KOMAScript}.
% \entry{sy}
%   This is the font for math symbols (e.g. cmsy).
%   It is used by \hologo{AmS}, \hologo{NTS}, \hologo{ExTeX}.
% \entry{logo}
%   \hologo{METAFONT} and \hologo{METAPOST} are using that font.
%   In \hologo{LaTeX} \cs{logofamily} is used and
%   the definitions of package \xpackage{mflogo} are used
%   if the package is not loaded.
%   Otherwise the \cs{tenlogo} is used and defined
%   if it does not already exists.
% \end{description}
%
% \begin{declcs}{hologoLogoFontSetup} \M{logo} \M{key value list}
% \end{declcs}
% Fonts can also be set for a logo or logo component separately,
% see the following list.
% The keys are the same as for \cs{hologoFontSetup}.
%
% \begin{longtable}{>{\ttfamily}l>{\sffamily}ll}
%   \meta{logo} & keys & result\\
%   \hline
%   \endhead
%   BibTeX & bibsf & {\hologoLogoSetup{BibTeX}{variant=sf}\hologo{BibTeX}}\\[.5ex]
%   BibTeX & sc & {\hologoLogoSetup{BibTeX}{variant=sc}\hologo{BibTeX}}\\[.5ex]
%   ExTeX & rm & \hologo{ExTeX}\\
%   SliTeX & rm & \hologo{SliTeX}\\[.5ex]
%   AmS & sy & \hologo{AmS}\\
%   ExTeX & sy & \hologo{ExTeX}\\
%   NTS & sy & \hologo{NTS}\\[.5ex]
%   KOMAScript & sf & \hologo{KOMAScript}\\[.5ex]
%   METAFONT & logo & \hologo{METAFONT}\\
%   METAPOST & logo & \hologo{METAPOST}\\[.5ex]
%   SliTeX & sc \hologo{SliTeX}
% \end{longtable}
%
% \subsubsection{Font order}
%
% For all logos the font \xoption{general} is applied first.
% Example:
%\begin{quote}
%|\hologoFontSetup{general=\color{red}}|
%\end{quote}
% will print red logos.
% Then if the font uses a special font \xoption{sf}, for example,
% the font is applied that is setup by \cs{hologoLogoFontSetup}.
% If this font is not setup, then the common font setup
% by \cs{hologoFontSetup} is used. Otherwise a warning is given,
% that there is no font configured.
%
% \subsection{Additional user macros}
%
% Usually a variant of a logo is configured by using
% \cs{hologoLogoSetup}, because it is bad style to mix
% different variants of the same logo in the same text.
% There the following macros are a convenience for testing.
%
% \begin{declcs}{hologoVariant} \M{name} \M{variant}\\
%   \cs{HologoVariant} \M{name} \M{variant}
% \end{declcs}
% Logo \meta{name} is set using \meta{variant} that specifies
% explicitely which variant of the macro is used. If the argument
% is empty, then the default form of the logo is used
% (configurable by \cs{hologoLogoSetup}).
%
% \cs{HologoVariant} is used if the logo is set in a context
% that needs an uppercase first letter (beginning of a sentence, \dots).
%
% \begin{declcs}{hologoList}\\
%   \cs{hologoEntry} \M{logo} \M{variant} \M{since}
% \end{declcs}
% Macro \cs{hologoList} contains all logos that are provided
% by the package including variants. The list consists of calls
% of \cs{hologoEntry} with three arguments starting with the
% logo name \meta{logo} and its variant \meta{variant}. An empty
% variant means the current default. Argument \meta{since} specifies
% with version of the package \xpackage{hologo} is needed to get
% the logo. If the logo is fixed, then the date gets updated.
% Therefore the date \meta{since} is not exactly the date of
% the first introduction, but rather the date of the latest fix.
%
% Before \cs{hologoList} can be used, macro \cs{hologoEntry} needs
% a definition. The example file in section \ref{sec:example}
% shows applications of \cs{hologoList}.
%
% \subsection{Supported contexts}
%
% Macros \cs{hologo} and friends support special contexts:
% \begin{itemize}
% \item \hologo{LaTeX}'s protection mechanism.
% \item Bookmarks of package \xpackage{hyperref}.
% \item Package \xpackage{tex4ht}.
% \item The macros can be used inside \cs{csname} constructs,
%   if \cs{ifincsname} is available (\hologo{pdfTeX}, \hologo{XeTeX},
%   \hologo{LuaTeX}).
% \end{itemize}
%
% \subsection{Example}
% \label{sec:example}
%
% The following example prints the logos in different fonts.
%    \begin{macrocode}
%<*example>
%<<verbatim
\NeedsTeXFormat{LaTeX2e}
\documentclass[a4paper]{article}
\usepackage[
  hmargin=20mm,
  vmargin=20mm,
]{geometry}
\pagestyle{empty}
\usepackage{hologo}[2016/05/12]
\usepackage{longtable}
\usepackage{array}
\setlength{\extrarowheight}{2pt}
\usepackage[T1]{fontenc}
\usepackage{lmodern}
\usepackage{pdflscape}
\usepackage[
  pdfencoding=auto,
]{hyperref}
\hypersetup{
  pdfauthor={Heiko Oberdiek},
  pdftitle={Example for package `hologo'},
  pdfsubject={Logos with fonts lmr, lmss, qtm, qpl, qhv},
}
\usepackage{bookmark}

% Print the logo list on the console

\begingroup
  \typeout{}%
  \typeout{*** Begin of logo list ***}%
  \newcommand*{\hologoEntry}[3]{%
    \typeout{#1 \ifx\\#2\\\else(#2) \fi[#3]}%
  }%
  \hologoList
  \typeout{*** End of logo list ***}%
  \typeout{}%
\endgroup

\begin{document}
\begin{landscape}

  \section{Example file for package `hologo'}

  % Table for font names

  \begin{longtable}{>{\bfseries}ll}
    \textbf{font} & \textbf{Font name}\\
    \hline
    lmr & Latin Modern Roman\\
    lmss & Latin Modern Sans\\
    qtm & \TeX\ Gyre Termes\\
    qhv & \TeX\ Gyre Heros\\
    qpl & \TeX\ Gyre Pagella\\
  \end{longtable}

  % Logo list with logos in different fonts

  \begingroup
    \newcommand*{\SetVariant}[2]{%
      \ifx\\#2\\%
      \else
        \hologoLogoSetup{#1}{variant=#2}%
      \fi
    }%
    \newcommand*{\hologoEntry}[3]{%
      \SetVariant{#1}{#2}%
      \raisebox{1em}[0pt][0pt]{\hypertarget{#1@#2}{}}%
      \bookmark[%
        dest={#1@#2},%
      ]{%
        #1\ifx\\#2\\\else\space(#2)\fi: \Hologo{#1}, \hologo{#1} %
        [Unicode]%
      }%
      \hypersetup{unicode=false}%
      \bookmark[%
        dest={#1@#2},%
      ]{%
        #1\ifx\\#2\\\else\space(#2)\fi: \Hologo{#1}, \hologo{#1} %
        [PDFDocEncoding]%
      }%
      \texttt{#1}%
      &%
      \texttt{#2}%
      &%
      \Hologo{#1}%
      &%
      \SetVariant{#1}{#2}%
      \hologo{#1}%
      &%
      \SetVariant{#1}{#2}%
      \fontfamily{qtm}\selectfont
      \hologo{#1}%
      &%
      \SetVariant{#1}{#2}%
      \fontfamily{qpl}\selectfont
      \hologo{#1}%
      &%
      \SetVariant{#1}{#2}%
      \textsf{\hologo{#1}}%
      &%
      \SetVariant{#1}{#2}%
      \fontfamily{qhv}\selectfont
      \hologo{#1}%
      \tabularnewline
    }%
    \begin{longtable}{llllllll}%
      \textbf{\textit{logo}} & \textbf{\textit{variant}} &
      \texttt{\string\Hologo} &
      \textbf{lmr} & \textbf{qtm} & \textbf{qpl} &
      \textbf{lmss} & \textbf{qhv}
      \tabularnewline
      \hline
      \endhead
      \hologoList
    \end{longtable}%
  \endgroup

\end{landscape}
\end{document}
%verbatim
%</example>
%    \end{macrocode}
%
% \StopEventually{
% }
%
% \section{Implementation}
%    \begin{macrocode}
%<*package>
%    \end{macrocode}
%    Reload check, especially if the package is not used with \LaTeX.
%    \begin{macrocode}
\begingroup\catcode61\catcode48\catcode32=10\relax%
  \catcode13=5 % ^^M
  \endlinechar=13 %
  \catcode35=6 % #
  \catcode39=12 % '
  \catcode44=12 % ,
  \catcode45=12 % -
  \catcode46=12 % .
  \catcode58=12 % :
  \catcode64=11 % @
  \catcode123=1 % {
  \catcode125=2 % }
  \expandafter\let\expandafter\x\csname ver@hologo.sty\endcsname
  \ifx\x\relax % plain-TeX, first loading
  \else
    \def\empty{}%
    \ifx\x\empty % LaTeX, first loading,
      % variable is initialized, but \ProvidesPackage not yet seen
    \else
      \expandafter\ifx\csname PackageInfo\endcsname\relax
        \def\x#1#2{%
          \immediate\write-1{Package #1 Info: #2.}%
        }%
      \else
        \def\x#1#2{\PackageInfo{#1}{#2, stopped}}%
      \fi
      \x{hologo}{The package is already loaded}%
      \aftergroup\endinput
    \fi
  \fi
\endgroup%
%    \end{macrocode}
%    Package identification:
%    \begin{macrocode}
\begingroup\catcode61\catcode48\catcode32=10\relax%
  \catcode13=5 % ^^M
  \endlinechar=13 %
  \catcode35=6 % #
  \catcode39=12 % '
  \catcode40=12 % (
  \catcode41=12 % )
  \catcode44=12 % ,
  \catcode45=12 % -
  \catcode46=12 % .
  \catcode47=12 % /
  \catcode58=12 % :
  \catcode64=11 % @
  \catcode91=12 % [
  \catcode93=12 % ]
  \catcode123=1 % {
  \catcode125=2 % }
  \expandafter\ifx\csname ProvidesPackage\endcsname\relax
    \def\x#1#2#3[#4]{\endgroup
      \immediate\write-1{Package: #3 #4}%
      \xdef#1{#4}%
    }%
  \else
    \def\x#1#2[#3]{\endgroup
      #2[{#3}]%
      \ifx#1\@undefined
        \xdef#1{#3}%
      \fi
      \ifx#1\relax
        \xdef#1{#3}%
      \fi
    }%
  \fi
\expandafter\x\csname ver@hologo.sty\endcsname
\ProvidesPackage{hologo}%
  [2016/05/12 v1.11 A logo collection with bookmark support (HO)]%
%    \end{macrocode}
%
%    \begin{macrocode}
\begingroup\catcode61\catcode48\catcode32=10\relax%
  \catcode13=5 % ^^M
  \endlinechar=13 %
  \catcode123=1 % {
  \catcode125=2 % }
  \catcode64=11 % @
  \def\x{\endgroup
    \expandafter\edef\csname HOLOGO@AtEnd\endcsname{%
      \endlinechar=\the\endlinechar\relax
      \catcode13=\the\catcode13\relax
      \catcode32=\the\catcode32\relax
      \catcode35=\the\catcode35\relax
      \catcode61=\the\catcode61\relax
      \catcode64=\the\catcode64\relax
      \catcode123=\the\catcode123\relax
      \catcode125=\the\catcode125\relax
    }%
  }%
\x\catcode61\catcode48\catcode32=10\relax%
\catcode13=5 % ^^M
\endlinechar=13 %
\catcode35=6 % #
\catcode64=11 % @
\catcode123=1 % {
\catcode125=2 % }
\def\TMP@EnsureCode#1#2{%
  \edef\HOLOGO@AtEnd{%
    \HOLOGO@AtEnd
    \catcode#1=\the\catcode#1\relax
  }%
  \catcode#1=#2\relax
}
\TMP@EnsureCode{10}{12}% ^^J
\TMP@EnsureCode{33}{12}% !
\TMP@EnsureCode{34}{12}% "
\TMP@EnsureCode{36}{3}% $
\TMP@EnsureCode{38}{4}% &
\TMP@EnsureCode{39}{12}% '
\TMP@EnsureCode{40}{12}% (
\TMP@EnsureCode{41}{12}% )
\TMP@EnsureCode{42}{12}% *
\TMP@EnsureCode{43}{12}% +
\TMP@EnsureCode{44}{12}% ,
\TMP@EnsureCode{45}{12}% -
\TMP@EnsureCode{46}{12}% .
\TMP@EnsureCode{47}{12}% /
\TMP@EnsureCode{58}{12}% :
\TMP@EnsureCode{59}{12}% ;
\TMP@EnsureCode{60}{12}% <
\TMP@EnsureCode{62}{12}% >
\TMP@EnsureCode{63}{12}% ?
\TMP@EnsureCode{91}{12}% [
\TMP@EnsureCode{93}{12}% ]
\TMP@EnsureCode{94}{7}% ^ (superscript)
\TMP@EnsureCode{95}{8}% _ (subscript)
\TMP@EnsureCode{96}{12}% `
\TMP@EnsureCode{124}{12}% |
\edef\HOLOGO@AtEnd{%
  \HOLOGO@AtEnd
  \escapechar\the\escapechar\relax
  \noexpand\endinput
}
\escapechar=92 %
%    \end{macrocode}
%
% \subsection{Logo list}
%
%    \begin{macro}{\hologoList}
%    \begin{macrocode}
\def\hologoList{%
  \hologoEntry{(La)TeX}{}{2011/10/01}%
  \hologoEntry{AmSLaTeX}{}{2010/04/16}%
  \hologoEntry{AmSTeX}{}{2010/04/16}%
  \hologoEntry{biber}{}{2011/10/01}%
  \hologoEntry{BibTeX}{}{2011/10/01}%
  \hologoEntry{BibTeX}{sf}{2011/10/01}%
  \hologoEntry{BibTeX}{sc}{2011/10/01}%
  \hologoEntry{BibTeX8}{}{2011/11/22}%
  \hologoEntry{ConTeXt}{}{2011/03/25}%
  \hologoEntry{ConTeXt}{narrow}{2011/03/25}%
  \hologoEntry{ConTeXt}{simple}{2011/03/25}%
  \hologoEntry{emTeX}{}{2010/04/26}%
  \hologoEntry{eTeX}{}{2010/04/08}%
  \hologoEntry{ExTeX}{}{2011/10/01}%
  \hologoEntry{HanTheThanh}{}{2011/11/29}%
  \hologoEntry{iniTeX}{}{2011/10/01}%
  \hologoEntry{KOMAScript}{}{2011/10/01}%
  \hologoEntry{La}{}{2010/05/08}%
  \hologoEntry{LaTeX}{}{2010/04/08}%
  \hologoEntry{LaTeX2e}{}{2010/04/08}%
  \hologoEntry{LaTeX3}{}{2010/04/24}%
  \hologoEntry{LaTeXe}{}{2010/04/08}%
  \hologoEntry{LaTeXML}{}{2011/11/22}%
  \hologoEntry{LaTeXTeX}{}{2011/10/01}%
  \hologoEntry{LuaLaTeX}{}{2010/04/08}%
  \hologoEntry{LuaTeX}{}{2010/04/08}%
  \hologoEntry{LyX}{}{2011/10/01}%
  \hologoEntry{METAFONT}{}{2011/10/01}%
  \hologoEntry{MetaFun}{}{2011/10/01}%
  \hologoEntry{METAPOST}{}{2011/10/01}%
  \hologoEntry{MetaPost}{}{2011/10/01}%
  \hologoEntry{MiKTeX}{}{2011/10/01}%
  \hologoEntry{NTS}{}{2011/10/01}%
  \hologoEntry{OzMF}{}{2011/10/01}%
  \hologoEntry{OzMP}{}{2011/10/01}%
  \hologoEntry{OzTeX}{}{2011/10/01}%
  \hologoEntry{OzTtH}{}{2011/10/01}%
  \hologoEntry{PCTeX}{}{2011/10/01}%
  \hologoEntry{pdfTeX}{}{2011/10/01}%
  \hologoEntry{pdfLaTeX}{}{2011/10/01}%
  \hologoEntry{PiC}{}{2011/10/01}%
  \hologoEntry{PiCTeX}{}{2011/10/01}%
  \hologoEntry{plainTeX}{}{2010/04/08}%
  \hologoEntry{plainTeX}{space}{2010/04/16}%
  \hologoEntry{plainTeX}{hyphen}{2010/04/16}%
  \hologoEntry{plainTeX}{runtogether}{2010/04/16}%
  \hologoEntry{SageTeX}{}{2011/11/22}%
  \hologoEntry{SLiTeX}{}{2011/10/01}%
  \hologoEntry{SLiTeX}{lift}{2011/10/01}%
  \hologoEntry{SLiTeX}{narrow}{2011/10/01}%
  \hologoEntry{SLiTeX}{simple}{2011/10/01}%
  \hologoEntry{SliTeX}{}{2011/10/01}%
  \hologoEntry{SliTeX}{narrow}{2011/10/01}%
  \hologoEntry{SliTeX}{simple}{2011/10/01}%
  \hologoEntry{SliTeX}{lift}{2011/10/01}%
  \hologoEntry{teTeX}{}{2011/10/01}%
  \hologoEntry{TeX}{}{2010/04/08}%
  \hologoEntry{TeX4ht}{}{2011/11/22}%
  \hologoEntry{TTH}{}{2011/11/22}%
  \hologoEntry{virTeX}{}{2011/10/01}%
  \hologoEntry{VTeX}{}{2010/04/24}%
  \hologoEntry{Xe}{}{2010/04/08}%
  \hologoEntry{XeLaTeX}{}{2010/04/08}%
  \hologoEntry{XeTeX}{}{2010/04/08}%
}
%    \end{macrocode}
%    \end{macro}
%
% \subsection{Load resources}
%
%    \begin{macrocode}
\begingroup\expandafter\expandafter\expandafter\endgroup
\expandafter\ifx\csname RequirePackage\endcsname\relax
  \def\TMP@RequirePackage#1[#2]{%
    \begingroup\expandafter\expandafter\expandafter\endgroup
    \expandafter\ifx\csname ver@#1.sty\endcsname\relax
      \input #1.sty\relax
    \fi
  }%
  \TMP@RequirePackage{ltxcmds}[2011/02/04]%
  \TMP@RequirePackage{infwarerr}[2010/04/08]%
  \TMP@RequirePackage{kvsetkeys}[2010/03/01]%
  \TMP@RequirePackage{kvdefinekeys}[2010/03/01]%
  \TMP@RequirePackage{pdftexcmds}[2010/04/01]%
  \TMP@RequirePackage{ifpdf}[2010/01/28]%
  \TMP@RequirePackage{ifluatex}[2010/03/01]%
  \ltx@IfUndefined{newif}{%
    \expandafter\let\csname newif\endcsname\ltx@newif
  }{}%
  \TMP@RequirePackage{ifxetex}[2009/01/23]%
  \TMP@RequirePackage{ifvtex}[2010/03/01]%
\else
  \RequirePackage{ltxcmds}[2011/02/04]%
  \RequirePackage{infwarerr}[2010/04/08]%
  \RequirePackage{kvsetkeys}[2010/03/01]%
  \RequirePackage{kvdefinekeys}[2010/03/01]%
  \RequirePackage{pdftexcmds}[2010/04/01]%
  \RequirePackage{ifpdf}[2010/01/28]%
  \RequirePackage{ifluatex}[2010/03/01]%
  \RequirePackage{ifxetex}[2009/01/23]%
  \RequirePackage{ifvtex}[2010/03/01]%
\fi
%    \end{macrocode}
%
%    \begin{macro}{\HOLOGO@IfDefined}
%    \begin{macrocode}
\def\HOLOGO@IfExists#1{%
  \ifx\@undefined#1%
    \expandafter\ltx@secondoftwo
  \else
    \ifx\relax#1%
      \expandafter\ltx@secondoftwo
    \else
      \expandafter\expandafter\expandafter\ltx@firstoftwo
    \fi
  \fi
}
%    \end{macrocode}
%    \end{macro}
%
% \subsection{Setup macros}
%
%    \begin{macro}{\hologoSetup}
%    \begin{macrocode}
\def\hologoSetup{%
  \let\HOLOGO@name\relax
  \HOLOGO@Setup
}
%    \end{macrocode}
%    \end{macro}
%
%    \begin{macro}{\hologoLogoSetup}
%    \begin{macrocode}
\def\hologoLogoSetup#1{%
  \edef\HOLOGO@name{#1}%
  \ltx@IfUndefined{HoLogo@\HOLOGO@name}{%
    \@PackageError{hologo}{%
      Unknown logo `\HOLOGO@name'%
    }\@ehc
    \ltx@gobble
  }{%
    \HOLOGO@Setup
  }%
}
%    \end{macrocode}
%    \end{macro}
%
%    \begin{macro}{\HOLOGO@Setup}
%    \begin{macrocode}
\def\HOLOGO@Setup{%
  \kvsetkeys{HoLogo}%
}
%    \end{macrocode}
%    \end{macro}
%
% \subsection{Options}
%
%    \begin{macro}{\HOLOGO@DeclareBoolOption}
%    \begin{macrocode}
\def\HOLOGO@DeclareBoolOption#1{%
  \expandafter\chardef\csname HOLOGOOPT@#1\endcsname\ltx@zero
  \kv@define@key{HoLogo}{#1}[true]{%
    \def\HOLOGO@temp{##1}%
    \ifx\HOLOGO@temp\HOLOGO@true
      \ifx\HOLOGO@name\relax
        \expandafter\chardef\csname HOLOGOOPT@#1\endcsname=\ltx@one
      \else
        \expandafter\chardef\csname
        HoLogoOpt@#1@\HOLOGO@name\endcsname\ltx@one
      \fi
      \HOLOGO@SetBreakAll{#1}%
    \else
      \ifx\HOLOGO@temp\HOLOGO@false
        \ifx\HOLOGO@name\relax
          \expandafter\chardef\csname HOLOGOOPT@#1\endcsname=\ltx@zero
        \else
          \expandafter\chardef\csname
          HoLogoOpt@#1@\HOLOGO@name\endcsname=\ltx@zero
        \fi
        \HOLOGO@SetBreakAll{#1}%
      \else
        \@PackageError{hologo}{%
          Unknown value `##1' for boolean option `#1'.\MessageBreak
          Known values are `true' and `false'%
        }\@ehc
      \fi
    \fi
  }%
}
%    \end{macrocode}
%    \end{macro}
%
%    \begin{macro}{\HOLOGO@SetBreakAll}
%    \begin{macrocode}
\def\HOLOGO@SetBreakAll#1{%
  \def\HOLOGO@temp{#1}%
  \ifx\HOLOGO@temp\HOLOGO@break
    \ifx\HOLOGO@name\relax
      \chardef\HOLOGOOPT@hyphenbreak=\HOLOGOOPT@break
      \chardef\HOLOGOOPT@spacebreak=\HOLOGOOPT@break
      \chardef\HOLOGOOPT@discretionarybreak=\HOLOGOOPT@break
    \else
      \expandafter\chardef
         \csname HoLogoOpt@hyphenbreak@\HOLOGO@name\endcsname=%
         \csname HoLogoOpt@break@\HOLOGO@name\endcsname
      \expandafter\chardef
         \csname HoLogoOpt@spacebreak@\HOLOGO@name\endcsname=%
         \csname HoLogoOpt@break@\HOLOGO@name\endcsname
      \expandafter\chardef
         \csname HoLogoOpt@discretionarybreak@\HOLOGO@name
             \endcsname=%
         \csname HoLogoOpt@break@\HOLOGO@name\endcsname
    \fi
  \fi
}
%    \end{macrocode}
%    \end{macro}
%
%    \begin{macro}{\HOLOGO@true}
%    \begin{macrocode}
\def\HOLOGO@true{true}
%    \end{macrocode}
%    \end{macro}
%    \begin{macro}{\HOLOGO@false}
%    \begin{macrocode}
\def\HOLOGO@false{false}
%    \end{macrocode}
%    \end{macro}
%    \begin{macro}{\HOLOGO@break}
%    \begin{macrocode}
\def\HOLOGO@break{break}
%    \end{macrocode}
%    \end{macro}
%
%    \begin{macrocode}
\HOLOGO@DeclareBoolOption{break}
\HOLOGO@DeclareBoolOption{hyphenbreak}
\HOLOGO@DeclareBoolOption{spacebreak}
\HOLOGO@DeclareBoolOption{discretionarybreak}
%    \end{macrocode}
%
%    \begin{macrocode}
\kv@define@key{HoLogo}{variant}{%
  \ifx\HOLOGO@name\relax
    \@PackageError{hologo}{%
      Option `variant' is not available in \string\hologoSetup,%
      \MessageBreak
      Use \string\hologoLogoSetup\space instead%
    }\@ehc
  \else
    \edef\HOLOGO@temp{#1}%
    \ifx\HOLOGO@temp\ltx@empty
      \expandafter
      \let\csname HoLogoOpt@variant@\HOLOGO@name\endcsname\@undefined
    \else
      \ltx@IfUndefined{HoLogo@\HOLOGO@name @\HOLOGO@temp}{%
        \@PackageError{hologo}{%
          Unknown variant `\HOLOGO@temp' of logo `\HOLOGO@name'%
        }\@ehc
      }{%
        \expandafter
        \let\csname HoLogoOpt@variant@\HOLOGO@name\endcsname
            \HOLOGO@temp
      }%
    \fi
  \fi
}
%    \end{macrocode}
%
%    \begin{macro}{\HOLOGO@Variant}
%    \begin{macrocode}
\def\HOLOGO@Variant#1{%
  #1%
  \ltx@ifundefined{HoLogoOpt@variant@#1}{%
  }{%
    @\csname HoLogoOpt@variant@#1\endcsname
  }%
}
%    \end{macrocode}
%    \end{macro}
%
% \subsection{Break/no-break support}
%
%    \begin{macro}{\HOLOGO@space}
%    \begin{macrocode}
\def\HOLOGO@space{%
  \ltx@ifundefined{HoLogoOpt@spacebreak@\HOLOGO@name}{%
    \ltx@ifundefined{HoLogoOpt@break@\HOLOGO@name}{%
      \chardef\HOLOGO@temp=\HOLOGOOPT@spacebreak
    }{%
      \chardef\HOLOGO@temp=%
        \csname HoLogoOpt@break@\HOLOGO@name\endcsname
    }%
  }{%
    \chardef\HOLOGO@temp=%
      \csname HoLogoOpt@spacebreak@\HOLOGO@name\endcsname
  }%
  \ifcase\HOLOGO@temp
    \penalty10000 %
  \fi
  \ltx@space
}
%    \end{macrocode}
%    \end{macro}
%
%    \begin{macro}{\HOLOGO@hyphen}
%    \begin{macrocode}
\def\HOLOGO@hyphen{%
  \ltx@ifundefined{HoLogoOpt@hyphenbreak@\HOLOGO@name}{%
    \ltx@ifundefined{HoLogoOpt@break@\HOLOGO@name}{%
      \chardef\HOLOGO@temp=\HOLOGOOPT@hyphenbreak
    }{%
      \chardef\HOLOGO@temp=%
        \csname HoLogoOpt@break@\HOLOGO@name\endcsname
    }%
  }{%
    \chardef\HOLOGO@temp=%
      \csname HoLogoOpt@hyphenbreak@\HOLOGO@name\endcsname
  }%
  \ifcase\HOLOGO@temp
    \ltx@mbox{-}%
  \else
    -%
  \fi
}
%    \end{macrocode}
%    \end{macro}
%
%    \begin{macro}{\HOLOGO@discretionary}
%    \begin{macrocode}
\def\HOLOGO@discretionary{%
  \ltx@ifundefined{HoLogoOpt@discretionarybreak@\HOLOGO@name}{%
    \ltx@ifundefined{HoLogoOpt@break@\HOLOGO@name}{%
      \chardef\HOLOGO@temp=\HOLOGOOPT@discretionarybreak
    }{%
      \chardef\HOLOGO@temp=%
        \csname HoLogoOpt@break@\HOLOGO@name\endcsname
    }%
  }{%
    \chardef\HOLOGO@temp=%
      \csname HoLogoOpt@discretionarybreak@\HOLOGO@name\endcsname
  }%
  \ifcase\HOLOGO@temp
  \else
    \-%
  \fi
}
%    \end{macrocode}
%    \end{macro}
%
%    \begin{macro}{\HOLOGO@mbox}
%    \begin{macrocode}
\def\HOLOGO@mbox#1{%
  \ltx@ifundefined{HoLogoOpt@break@\HOLOGO@name}{%
    \chardef\HOLOGO@temp=\HOLOGOOPT@hyphenbreak
  }{%
    \chardef\HOLOGO@temp=%
      \csname HoLogoOpt@break@\HOLOGO@name\endcsname
  }%
  \ifcase\HOLOGO@temp
    \ltx@mbox{#1}%
  \else
    #1%
  \fi
}
%    \end{macrocode}
%    \end{macro}
%
% \subsection{Font support}
%
%    \begin{macro}{\HoLogoFont@font}
%    \begin{tabular}{@{}ll@{}}
%    |#1|:& logo name\\
%    |#2|:& font short name\\
%    |#3|:& text
%    \end{tabular}
%    \begin{macrocode}
\def\HoLogoFont@font#1#2#3{%
  \begingroup
    \ltx@IfUndefined{HoLogoFont@logo@#1.#2}{%
      \ltx@IfUndefined{HoLogoFont@font@#2}{%
        \@PackageWarning{hologo}{%
          Missing font `#2' for logo `#1'%
        }%
        #3%
      }{%
        \csname HoLogoFont@font@#2\endcsname{#3}%
      }%
    }{%
      \csname HoLogoFont@logo@#1.#2\endcsname{#3}%
    }%
  \endgroup
}
%    \end{macrocode}
%    \end{macro}
%
%    \begin{macro}{\HoLogoFont@Def}
%    \begin{macrocode}
\def\HoLogoFont@Def#1{%
  \expandafter\def\csname HoLogoFont@font@#1\endcsname
}
%    \end{macrocode}
%    \end{macro}
%    \begin{macro}{\HoLogoFont@LogoDef}
%    \begin{macrocode}
\def\HoLogoFont@LogoDef#1#2{%
  \expandafter\def\csname HoLogoFont@logo@#1.#2\endcsname
}
%    \end{macrocode}
%    \end{macro}
%
% \subsubsection{Font defaults}
%
%    \begin{macro}{\HoLogoFont@font@general}
%    \begin{macrocode}
\HoLogoFont@Def{general}{}%
%    \end{macrocode}
%    \end{macro}
%
%    \begin{macro}{\HoLogoFont@font@rm}
%    \begin{macrocode}
\ltx@IfUndefined{rmfamily}{%
  \ltx@IfUndefined{rm}{%
  }{%
    \HoLogoFont@Def{rm}{\rm}%
  }%
}{%
  \HoLogoFont@Def{rm}{\rmfamily}%
}
%    \end{macrocode}
%    \end{macro}
%
%    \begin{macro}{\HoLogoFont@font@sf}
%    \begin{macrocode}
\ltx@IfUndefined{sffamily}{%
  \ltx@IfUndefined{sf}{%
  }{%
    \HoLogoFont@Def{sf}{\sf}%
  }%
}{%
  \HoLogoFont@Def{sf}{\sffamily}%
}
%    \end{macrocode}
%    \end{macro}
%
%    \begin{macro}{\HoLogoFont@font@bibsf}
%    In case of \hologo{plainTeX} the original small caps
%    variant is used as default. In \hologo{LaTeX}
%    the definition of package \xpackage{dtklogos} \cite{dtklogos}
%    is used.
%\begin{quote}
%\begin{verbatim}
%\DeclareRobustCommand{\BibTeX}{%
%  B%
%  \kern-.05em%
%  \hbox{%
%    $\m@th$% %% force math size calculations
%    \csname S@\f@size\endcsname
%    \fontsize\sf@size\z@
%    \math@fontsfalse
%    \selectfont
%    I%
%    \kern-.025em%
%    B
%  }%
%  \kern-.08em%
%  \-%
%  \TeX
%}
%\end{verbatim}
%\end{quote}
%    \begin{macrocode}
\ltx@IfUndefined{selectfont}{%
  \ltx@IfUndefined{tensc}{%
    \font\tensc=cmcsc10\relax
  }{}%
  \HoLogoFont@Def{bibsf}{\tensc}%
}{%
  \HoLogoFont@Def{bibsf}{%
    $\mathsurround=0pt$%
    \csname S@\f@size\endcsname
    \fontsize\sf@size{0pt}%
    \math@fontsfalse
    \selectfont
  }%
}
%    \end{macrocode}
%    \end{macro}
%
%    \begin{macro}{\HoLogoFont@font@sc}
%    \begin{macrocode}
\ltx@IfUndefined{scshape}{%
  \ltx@IfUndefined{tensc}{%
    \font\tensc=cmcsc10\relax
  }{}%
  \HoLogoFont@Def{sc}{\tensc}%
}{%
  \HoLogoFont@Def{sc}{\scshape}%
}
%    \end{macrocode}
%    \end{macro}
%
%    \begin{macro}{\HoLogoFont@font@sy}
%    \begin{macrocode}
\ltx@IfUndefined{usefont}{%
  \ltx@IfUndefined{tensy}{%
  }{%
    \HoLogoFont@Def{sy}{\tensy}%
  }%
}{%
  \HoLogoFont@Def{sy}{%
    \usefont{OMS}{cmsy}{m}{n}%
  }%
}
%    \end{macrocode}
%    \end{macro}
%
%    \begin{macro}{\HoLogoFont@font@logo}
%    \begin{macrocode}
\begingroup
  \def\x{LaTeX2e}%
\expandafter\endgroup
\ifx\fmtname\x
  \ltx@IfUndefined{logofamily}{%
    \DeclareRobustCommand\logofamily{%
      \not@math@alphabet\logofamily\relax
      \fontencoding{U}%
      \fontfamily{logo}%
      \selectfont
    }%
  }{}%
  \ltx@IfUndefined{logofamily}{%
  }{%
    \HoLogoFont@Def{logo}{\logofamily}%
  }%
\else
  \ltx@IfUndefined{tenlogo}{%
    \font\tenlogo=logo10\relax
  }{}%
  \HoLogoFont@Def{logo}{\tenlogo}%
\fi
%    \end{macrocode}
%    \end{macro}
%
% \subsubsection{Font setup}
%
%    \begin{macro}{\hologoFontSetup}
%    \begin{macrocode}
\def\hologoFontSetup{%
  \let\HOLOGO@name\relax
  \HOLOGO@FontSetup
}
%    \end{macrocode}
%    \end{macro}
%
%    \begin{macro}{\hologoLogoFontSetup}
%    \begin{macrocode}
\def\hologoLogoFontSetup#1{%
  \edef\HOLOGO@name{#1}%
  \ltx@IfUndefined{HoLogo@\HOLOGO@name}{%
    \@PackageError{hologo}{%
      Unknown logo `\HOLOGO@name'%
    }\@ehc
    \ltx@gobble
  }{%
    \HOLOGO@FontSetup
  }%
}
%    \end{macrocode}
%    \end{macro}
%
%    \begin{macro}{\HOLOGO@FontSetup}
%    \begin{macrocode}
\def\HOLOGO@FontSetup{%
  \kvsetkeys{HoLogoFont}%
}
%    \end{macrocode}
%    \end{macro}
%
%    \begin{macrocode}
\def\HOLOGO@temp#1{%
  \kv@define@key{HoLogoFont}{#1}{%
    \ifx\HOLOGO@name\relax
      \HoLogoFont@Def{#1}{##1}%
    \else
      \HoLogoFont@LogoDef\HOLOGO@name{#1}{##1}%
    \fi
  }%
}
\HOLOGO@temp{general}
\HOLOGO@temp{sf}
%    \end{macrocode}
%
% \subsection{Generic logo commands}
%
%    \begin{macrocode}
\HOLOGO@IfExists\hologo{%
  \@PackageError{hologo}{%
    \string\hologo\ltx@space is already defined.\MessageBreak
    Package loading is aborted%
  }\@ehc
  \HOLOGO@AtEnd
}%
\HOLOGO@IfExists\hologoRobust{%
  \@PackageError{hologo}{%
    \string\hologoRobust\ltx@space is already defined.\MessageBreak
    Package loading is aborted%
  }\@ehc
  \HOLOGO@AtEnd
}%
%    \end{macrocode}
%
% \subsubsection{\cs{hologo} and friends}
%
%    \begin{macrocode}
\ifluatex
  \expandafter\ltx@firstofone
\else
  \expandafter\ltx@gobble
\fi
{%
  \ltx@IfUndefined{ifincsname}{%
    \ifnum\luatexversion<36 %
      \expandafter\ltx@gobble
    \else
      \expandafter\ltx@firstofone
    \fi
    {%
      \begingroup
        \ifcase0%
            \directlua{%
              if tex.enableprimitives then %
                tex.enableprimitives('HOLOGO@', {'ifincsname'})%
              else %
                tex.print('1')%
              end%
            }%
            \ifx\HOLOGO@ifincsname\@undefined 1\fi%
            \relax
          \expandafter\ltx@firstofone
        \else
          \endgroup
          \expandafter\ltx@gobble
        \fi
        {%
          \global\let\ifincsname\HOLOGO@ifincsname
        }%
      \HOLOGO@temp
    }%
  }{}%
}
%    \end{macrocode}
%    \begin{macrocode}
\ltx@IfUndefined{ifincsname}{%
  \catcode`$=14 %
}{%
  \catcode`$=9 %
}
%    \end{macrocode}
%
%    \begin{macro}{\hologo}
%    \begin{macrocode}
\def\hologo#1{%
$ \ifincsname
$   \ltx@ifundefined{HoLogoCs@\HOLOGO@Variant{#1}}{%
$     #1%
$   }{%
$     \csname HoLogoCs@\HOLOGO@Variant{#1}\endcsname\ltx@firstoftwo
$   }%
$ \else
    \HOLOGO@IfExists\texorpdfstring\texorpdfstring\ltx@firstoftwo
    {%
      \hologoRobust{#1}%
    }{%
      \ltx@ifundefined{HoLogoBkm@\HOLOGO@Variant{#1}}{%
        \ltx@ifundefined{HoLogo@#1}{?#1?}{#1}%
      }{%
        \csname HoLogoBkm@\HOLOGO@Variant{#1}\endcsname
        \ltx@firstoftwo
      }%
    }%
$ \fi
}
%    \end{macrocode}
%    \end{macro}
%    \begin{macro}{\Hologo}
%    \begin{macrocode}
\def\Hologo#1{%
$ \ifincsname
$   \ltx@ifundefined{HoLogoCs@\HOLOGO@Variant{#1}}{%
$     #1%
$   }{%
$     \csname HoLogoCs@\HOLOGO@Variant{#1}\endcsname\ltx@secondoftwo
$   }%
$ \else
    \HOLOGO@IfExists\texorpdfstring\texorpdfstring\ltx@firstoftwo
    {%
      \HologoRobust{#1}%
    }{%
      \ltx@ifundefined{HoLogoBkm@\HOLOGO@Variant{#1}}{%
        \ltx@ifundefined{HoLogo@#1}{?#1?}{#1}%
      }{%
        \csname HoLogoBkm@\HOLOGO@Variant{#1}\endcsname
        \ltx@secondoftwo
      }%
    }%
$ \fi
}
%    \end{macrocode}
%    \end{macro}
%
%    \begin{macro}{\hologoVariant}
%    \begin{macrocode}
\def\hologoVariant#1#2{%
  \ifx\relax#2\relax
    \hologo{#1}%
  \else
$   \ifincsname
$     \ltx@ifundefined{HoLogoCs@#1@#2}{%
$       #1%
$     }{%
$       \csname HoLogoCs@#1@#2\endcsname\ltx@firstoftwo
$     }%
$   \else
      \HOLOGO@IfExists\texorpdfstring\texorpdfstring\ltx@firstoftwo
      {%
        \hologoVariantRobust{#1}{#2}%
      }{%
        \ltx@ifundefined{HoLogoBkm@#1@#2}{%
          \ltx@ifundefined{HoLogo@#1}{?#1?}{#1}%
        }{%
          \csname HoLogoBkm@#1@#2\endcsname
          \ltx@firstoftwo
        }%
      }%
$   \fi
  \fi
}
%    \end{macrocode}
%    \end{macro}
%    \begin{macro}{\HologoVariant}
%    \begin{macrocode}
\def\HologoVariant#1#2{%
  \ifx\relax#2\relax
    \Hologo{#1}%
  \else
$   \ifincsname
$     \ltx@ifundefined{HoLogoCs@#1@#2}{%
$       #1%
$     }{%
$       \csname HoLogoCs@#1@#2\endcsname\ltx@secondoftwo
$     }%
$   \else
      \HOLOGO@IfExists\texorpdfstring\texorpdfstring\ltx@firstoftwo
      {%
        \HologoVariantRobust{#1}{#2}%
      }{%
        \ltx@ifundefined{HoLogoBkm@#1@#2}{%
          \ltx@ifundefined{HoLogo@#1}{?#1?}{#1}%
        }{%
          \csname HoLogoBkm@#1@#2\endcsname
          \ltx@secondoftwo
        }%
      }%
$   \fi
  \fi
}
%    \end{macrocode}
%    \end{macro}
%
%    \begin{macrocode}
\catcode`\$=3 %
%    \end{macrocode}
%
% \subsubsection{\cs{hologoRobust} and friends}
%
%    \begin{macro}{\hologoRobust}
%    \begin{macrocode}
\ltx@IfUndefined{protected}{%
  \ltx@IfUndefined{DeclareRobustCommand}{%
    \def\hologoRobust#1%
  }{%
    \DeclareRobustCommand*\hologoRobust[1]%
  }%
}{%
  \protected\def\hologoRobust#1%
}%
{%
  \edef\HOLOGO@name{#1}%
  \ltx@IfUndefined{HoLogo@\HOLOGO@Variant\HOLOGO@name}{%
    \@PackageError{hologo}{%
      Unknown logo `\HOLOGO@name'%
    }\@ehc
    ?\HOLOGO@name?%
  }{%
    \ltx@IfUndefined{ver@tex4ht.sty}{%
      \HoLogoFont@font\HOLOGO@name{general}{%
        \csname HoLogo@\HOLOGO@Variant\HOLOGO@name\endcsname
        \ltx@firstoftwo
      }%
    }{%
      \ltx@IfUndefined{HoLogoHtml@\HOLOGO@Variant\HOLOGO@name}{%
        \HOLOGO@name
      }{%
        \csname HoLogoHtml@\HOLOGO@Variant\HOLOGO@name\endcsname
        \ltx@firstoftwo
      }%
    }%
  }%
}
%    \end{macrocode}
%    \end{macro}
%    \begin{macro}{\HologoRobust}
%    \begin{macrocode}
\ltx@IfUndefined{protected}{%
  \ltx@IfUndefined{DeclareRobustCommand}{%
    \def\HologoRobust#1%
  }{%
    \DeclareRobustCommand*\HologoRobust[1]%
  }%
}{%
  \protected\def\HologoRobust#1%
}%
{%
  \edef\HOLOGO@name{#1}%
  \ltx@IfUndefined{HoLogo@\HOLOGO@Variant\HOLOGO@name}{%
    \@PackageError{hologo}{%
      Unknown logo `\HOLOGO@name'%
    }\@ehc
    ?\HOLOGO@name?%
  }{%
    \ltx@IfUndefined{ver@tex4ht.sty}{%
      \HoLogoFont@font\HOLOGO@name{general}{%
        \csname HoLogo@\HOLOGO@Variant\HOLOGO@name\endcsname
        \ltx@secondoftwo
      }%
    }{%
      \ltx@IfUndefined{HoLogoHtml@\HOLOGO@Variant\HOLOGO@name}{%
        \expandafter\HOLOGO@Uppercase\HOLOGO@name
      }{%
        \csname HoLogoHtml@\HOLOGO@Variant\HOLOGO@name\endcsname
        \ltx@secondoftwo
      }%
    }%
  }%
}
%    \end{macrocode}
%    \end{macro}
%    \begin{macro}{\hologoVariantRobust}
%    \begin{macrocode}
\ltx@IfUndefined{protected}{%
  \ltx@IfUndefined{DeclareRobustCommand}{%
    \def\hologoVariantRobust#1#2%
  }{%
    \DeclareRobustCommand*\hologoVariantRobust[2]%
  }%
}{%
  \protected\def\hologoVariantRobust#1#2%
}%
{%
  \begingroup
    \hologoLogoSetup{#1}{variant={#2}}%
    \hologoRobust{#1}%
  \endgroup
}
%    \end{macrocode}
%    \end{macro}
%    \begin{macro}{\HologoVariantRobust}
%    \begin{macrocode}
\ltx@IfUndefined{protected}{%
  \ltx@IfUndefined{DeclareRobustCommand}{%
    \def\HologoVariantRobust#1#2%
  }{%
    \DeclareRobustCommand*\HologoVariantRobust[2]%
  }%
}{%
  \protected\def\HologoVariantRobust#1#2%
}%
{%
  \begingroup
    \hologoLogoSetup{#1}{variant={#2}}%
    \HologoRobust{#1}%
  \endgroup
}
%    \end{macrocode}
%    \end{macro}
%
%    \begin{macro}{\hologorobust}
%    Macro \cs{hologorobust} is only defined for compatibility.
%    Its use is deprecated.
%    \begin{macrocode}
\def\hologorobust{\hologoRobust}
%    \end{macrocode}
%    \end{macro}
%
% \subsection{Helpers}
%
%    \begin{macro}{\HOLOGO@Uppercase}
%    Macro \cs{HOLOGO@Uppercase} is restricted to \cs{uppercase},
%    because \hologo{plainTeX} or \hologo{iniTeX} do not provide
%    \cs{MakeUppercase}.
%    \begin{macrocode}
\def\HOLOGO@Uppercase#1{\uppercase{#1}}
%    \end{macrocode}
%    \end{macro}
%
%    \begin{macro}{\HOLOGO@PdfdocUnicode}
%    \begin{macrocode}
\def\HOLOGO@PdfdocUnicode{%
  \ifx\ifHy@unicode\iftrue
    \expandafter\ltx@secondoftwo
  \else
    \expandafter\ltx@firstoftwo
  \fi
}
%    \end{macrocode}
%    \end{macro}
%
%    \begin{macro}{\HOLOGO@Math}
%    \begin{macrocode}
\def\HOLOGO@MathSetup{%
  \mathsurround0pt\relax
  \HOLOGO@IfExists\f@series{%
    \if b\expandafter\ltx@car\f@series x\@nil
      \csname boldmath\endcsname
   \fi
  }{}%
}
%    \end{macrocode}
%    \end{macro}
%
%    \begin{macro}{\HOLOGO@TempDimen}
%    \begin{macrocode}
\dimendef\HOLOGO@TempDimen=\ltx@zero
%    \end{macrocode}
%    \end{macro}
%    \begin{macro}{\HOLOGO@NegativeKerning}
%    \begin{macrocode}
\def\HOLOGO@NegativeKerning#1{%
  \begingroup
    \HOLOGO@TempDimen=0pt\relax
    \comma@parse@normalized{#1}{%
      \ifdim\HOLOGO@TempDimen=0pt %
        \expandafter\HOLOGO@@NegativeKerning\comma@entry
      \fi
      \ltx@gobble
    }%
    \ifdim\HOLOGO@TempDimen<0pt %
      \kern\HOLOGO@TempDimen
    \fi
  \endgroup
}
%    \end{macrocode}
%    \end{macro}
%    \begin{macro}{\HOLOGO@@NegativeKerning}
%    \begin{macrocode}
\def\HOLOGO@@NegativeKerning#1#2{%
  \setbox\ltx@zero\hbox{#1#2}%
  \HOLOGO@TempDimen=\wd\ltx@zero
  \setbox\ltx@zero\hbox{#1\kern0pt#2}%
  \advance\HOLOGO@TempDimen by -\wd\ltx@zero
}
%    \end{macrocode}
%    \end{macro}
%
%    \begin{macro}{\HOLOGO@SpaceFactor}
%    \begin{macrocode}
\def\HOLOGO@SpaceFactor{%
  \spacefactor1000 %
}
%    \end{macrocode}
%    \end{macro}
%
%    \begin{macro}{\HOLOGO@Span}
%    \begin{macrocode}
\def\HOLOGO@Span#1#2{%
  \HCode{<span class="HoLogo-#1">}%
  #2%
  \HCode{</span>}%
}
%    \end{macrocode}
%    \end{macro}
%
% \subsubsection{Text subscript}
%
%    \begin{macro}{\HOLOGO@SubScript}%
%    \begin{macrocode}
\def\HOLOGO@SubScript#1{%
  \ltx@IfUndefined{textsubscript}{%
    \ltx@IfUndefined{text}{%
      \ltx@mbox{%
        \mathsurround=0pt\relax
        $%
          _{%
            \ltx@IfUndefined{sf@size}{%
              \mathrm{#1}%
            }{%
              \mbox{%
                \fontsize\sf@size{0pt}\selectfont
                #1%
              }%
            }%
          }%
        $%
      }%
    }{%
      \ltx@mbox{%
        \mathsurround=0pt\relax
        $_{\text{#1}}$%
      }%
    }%
  }{%
    \textsubscript{#1}%
  }%
}
%    \end{macrocode}
%    \end{macro}
%
% \subsection{\hologo{TeX} and friends}
%
% \subsubsection{\hologo{TeX}}
%
%    \begin{macro}{\HoLogo@TeX}
%    Source: \hologo{LaTeX} kernel.
%    \begin{macrocode}
\def\HoLogo@TeX#1{%
  T\kern-.1667em\lower.5ex\hbox{E}\kern-.125emX\HOLOGO@SpaceFactor
}
%    \end{macrocode}
%    \end{macro}
%    \begin{macro}{\HoLogoHtml@TeX}
%    \begin{macrocode}
\def\HoLogoHtml@TeX#1{%
  \HoLogoCss@TeX
  \HOLOGO@Span{TeX}{%
    T%
    \HOLOGO@Span{e}{%
      E%
    }%
    X%
  }%
}
%    \end{macrocode}
%    \end{macro}
%    \begin{macro}{\HoLogoCss@TeX}
%    \begin{macrocode}
\def\HoLogoCss@TeX{%
  \Css{%
    span.HoLogo-TeX span.HoLogo-e{%
      position:relative;%
      top:.5ex;%
      margin-left:-.1667em;%
      margin-right:-.125em;%
    }%
  }%
  \Css{%
    a span.HoLogo-TeX span.HoLogo-e{%
      text-decoration:none;%
    }%
  }%
  \global\let\HoLogoCss@TeX\relax
}
%    \end{macrocode}
%    \end{macro}
%
% \subsubsection{\hologo{plainTeX}}
%
%    \begin{macro}{\HoLogo@plainTeX@space}
%    Source: ``The \hologo{TeX}book''
%    \begin{macrocode}
\def\HoLogo@plainTeX@space#1{%
  \HOLOGO@mbox{#1{p}{P}lain}\HOLOGO@space\hologo{TeX}%
}
%    \end{macrocode}
%    \end{macro}
%    \begin{macro}{\HoLogoCs@plainTeX@space}
%    \begin{macrocode}
\def\HoLogoCs@plainTeX@space#1{#1{p}{P}lain TeX}%
%    \end{macrocode}
%    \end{macro}
%    \begin{macro}{\HoLogoBkm@plainTeX@space}
%    \begin{macrocode}
\def\HoLogoBkm@plainTeX@space#1{%
  #1{p}{P}lain \hologo{TeX}%
}
%    \end{macrocode}
%    \end{macro}
%    \begin{macro}{\HoLogoHtml@plainTeX@space}
%    \begin{macrocode}
\def\HoLogoHtml@plainTeX@space#1{%
  #1{p}{P}lain \hologo{TeX}%
}
%    \end{macrocode}
%    \end{macro}
%
%    \begin{macro}{\HoLogo@plainTeX@hyphen}
%    \begin{macrocode}
\def\HoLogo@plainTeX@hyphen#1{%
  \HOLOGO@mbox{#1{p}{P}lain}\HOLOGO@hyphen\hologo{TeX}%
}
%    \end{macrocode}
%    \end{macro}
%    \begin{macro}{\HoLogoCs@plainTeX@hyphen}
%    \begin{macrocode}
\def\HoLogoCs@plainTeX@hyphen#1{#1{p}{P}lain-TeX}
%    \end{macrocode}
%    \end{macro}
%    \begin{macro}{\HoLogoBkm@plainTeX@hyphen}
%    \begin{macrocode}
\def\HoLogoBkm@plainTeX@hyphen#1{%
  #1{p}{P}lain-\hologo{TeX}%
}
%    \end{macrocode}
%    \end{macro}
%    \begin{macro}{\HoLogoHtml@plainTeX@hyphen}
%    \begin{macrocode}
\def\HoLogoHtml@plainTeX@hyphen#1{%
  #1{p}{P}lain-\hologo{TeX}%
}
%    \end{macrocode}
%    \end{macro}
%
%    \begin{macro}{\HoLogo@plainTeX@runtogether}
%    \begin{macrocode}
\def\HoLogo@plainTeX@runtogether#1{%
  \HOLOGO@mbox{#1{p}{P}lain\hologo{TeX}}%
}
%    \end{macrocode}
%    \end{macro}
%    \begin{macro}{\HoLogoCs@plainTeX@runtogether}
%    \begin{macrocode}
\def\HoLogoCs@plainTeX@runtogether#1{#1{p}{P}lainTeX}
%    \end{macrocode}
%    \end{macro}
%    \begin{macro}{\HoLogoBkm@plainTeX@runtogether}
%    \begin{macrocode}
\def\HoLogoBkm@plainTeX@runtogether#1{%
  #1{p}{P}lain\hologo{TeX}%
}
%    \end{macrocode}
%    \end{macro}
%    \begin{macro}{\HoLogoHtml@plainTeX@runtogether}
%    \begin{macrocode}
\def\HoLogoHtml@plainTeX@runtogether#1{%
  #1{p}{P}lain\hologo{TeX}%
}
%    \end{macrocode}
%    \end{macro}
%
%    \begin{macro}{\HoLogo@plainTeX}
%    \begin{macrocode}
\def\HoLogo@plainTeX{\HoLogo@plainTeX@space}
%    \end{macrocode}
%    \end{macro}
%    \begin{macro}{\HoLogoCs@plainTeX}
%    \begin{macrocode}
\def\HoLogoCs@plainTeX{\HoLogoCs@plainTeX@space}
%    \end{macrocode}
%    \end{macro}
%    \begin{macro}{\HoLogoBkm@plainTeX}
%    \begin{macrocode}
\def\HoLogoBkm@plainTeX{\HoLogoBkm@plainTeX@space}
%    \end{macrocode}
%    \end{macro}
%    \begin{macro}{\HoLogoHtml@plainTeX}
%    \begin{macrocode}
\def\HoLogoHtml@plainTeX{\HoLogoHtml@plainTeX@space}
%    \end{macrocode}
%    \end{macro}
%
% \subsubsection{\hologo{LaTeX}}
%
%    Source: \hologo{LaTeX} kernel.
%\begin{quote}
%\begin{verbatim}
%\DeclareRobustCommand{\LaTeX}{%
%  L%
%  \kern-.36em%
%  {%
%    \sbox\z@ T%
%    \vbox to\ht\z@{%
%      \hbox{%
%        \check@mathfonts
%        \fontsize\sf@size\z@
%        \math@fontsfalse
%        \selectfont
%        A%
%      }%
%      \vss
%    }%
%  }%
%  \kern-.15em%
%  \TeX
%}
%\end{verbatim}
%\end{quote}
%
%    \begin{macro}{\HoLogo@La}
%    \begin{macrocode}
\def\HoLogo@La#1{%
  L%
  \kern-.36em%
  \begingroup
    \setbox\ltx@zero\hbox{T}%
    \vbox to\ht\ltx@zero{%
      \hbox{%
        \ltx@ifundefined{check@mathfonts}{%
          \csname sevenrm\endcsname
        }{%
          \check@mathfonts
          \fontsize\sf@size{0pt}%
          \math@fontsfalse\selectfont
        }%
        A%
      }%
      \vss
    }%
  \endgroup
}
%    \end{macrocode}
%    \end{macro}
%
%    \begin{macro}{\HoLogo@LaTeX}
%    Source: \hologo{LaTeX} kernel.
%    \begin{macrocode}
\def\HoLogo@LaTeX#1{%
  \hologo{La}%
  \kern-.15em%
  \hologo{TeX}%
}
%    \end{macrocode}
%    \end{macro}
%    \begin{macro}{\HoLogoHtml@LaTeX}
%    \begin{macrocode}
\def\HoLogoHtml@LaTeX#1{%
  \HoLogoCss@LaTeX
  \HOLOGO@Span{LaTeX}{%
    L%
    \HOLOGO@Span{a}{%
      A%
    }%
    \hologo{TeX}%
  }%
}
%    \end{macrocode}
%    \end{macro}
%    \begin{macro}{\HoLogoCss@LaTeX}
%    \begin{macrocode}
\def\HoLogoCss@LaTeX{%
  \Css{%
    span.HoLogo-LaTeX span.HoLogo-a{%
      position:relative;%
      top:-.5ex;%
      margin-left:-.36em;%
      margin-right:-.15em;%
      font-size:85\%;%
    }%
  }%
  \global\let\HoLogoCss@LaTeX\relax
}
%    \end{macrocode}
%    \end{macro}
%
% \subsubsection{\hologo{(La)TeX}}
%
%    \begin{macro}{\HoLogo@LaTeXTeX}
%    The kerning around the parentheses is taken
%    from package \xpackage{dtklogos} \cite{dtklogos}.
%\begin{quote}
%\begin{verbatim}
%\DeclareRobustCommand{\LaTeXTeX}{%
%  (%
%  \kern-.15em%
%  L%
%  \kern-.36em%
%  {%
%    \sbox\z@ T%
%    \vbox to\ht0{%
%      \hbox{%
%        $\m@th$%
%        \csname S@\f@size\endcsname
%        \fontsize\sf@size\z@
%        \math@fontsfalse
%        \selectfont
%        A%
%      }%
%      \vss
%    }%
%  }%
%  \kern-.2em%
%  )%
%  \kern-.15em%
%  \TeX
%}
%\end{verbatim}
%\end{quote}
%    \begin{macrocode}
\def\HoLogo@LaTeXTeX#1{%
  (%
  \kern-.15em%
  \hologo{La}%
  \kern-.2em%
  )%
  \kern-.15em%
  \hologo{TeX}%
}
%    \end{macrocode}
%    \end{macro}
%    \begin{macro}{\HoLogoBkm@LaTeXTeX}
%    \begin{macrocode}
\def\HoLogoBkm@LaTeXTeX#1{(La)TeX}
%    \end{macrocode}
%    \end{macro}
%
%    \begin{macro}{\HoLogo@(La)TeX}
%    \begin{macrocode}
\expandafter
\let\csname HoLogo@(La)TeX\endcsname\HoLogo@LaTeXTeX
%    \end{macrocode}
%    \end{macro}
%    \begin{macro}{\HoLogoBkm@(La)TeX}
%    \begin{macrocode}
\expandafter
\let\csname HoLogoBkm@(La)TeX\endcsname\HoLogoBkm@LaTeXTeX
%    \end{macrocode}
%    \end{macro}
%    \begin{macro}{\HoLogoHtml@LaTeXTeX}
%    \begin{macrocode}
\def\HoLogoHtml@LaTeXTeX#1{%
  \HoLogoCss@LaTeXTeX
  \HOLOGO@Span{LaTeXTeX}{%
    (%
    \HOLOGO@Span{L}{L}%
    \HOLOGO@Span{a}{A}%
    \HOLOGO@Span{ParenRight}{)}%
    \hologo{TeX}%
  }%
}
%    \end{macrocode}
%    \end{macro}
%    \begin{macro}{\HoLogoHtml@(La)TeX}
%    Kerning after opening parentheses and before closing parentheses
%    is $-0.1$\,em. The original values $-0.15$\,em
%    looked too ugly for a serif font.
%    \begin{macrocode}
\expandafter
\let\csname HoLogoHtml@(La)TeX\endcsname\HoLogoHtml@LaTeXTeX
%    \end{macrocode}
%    \end{macro}
%    \begin{macro}{\HoLogoCss@LaTeXTeX}
%    \begin{macrocode}
\def\HoLogoCss@LaTeXTeX{%
  \Css{%
    span.HoLogo-LaTeXTeX span.HoLogo-L{%
      margin-left:-.1em;%
    }%
  }%
  \Css{%
    span.HoLogo-LaTeXTeX span.HoLogo-a{%
      position:relative;%
      top:-.5ex;%
      margin-left:-.36em;%
      margin-right:-.1em;%
      font-size:85\%;%
    }%
  }%
  \Css{%
    span.HoLogo-LaTeXTeX span.HoLogo-ParenRight{%
      margin-right:-.15em;%
    }%
  }%
  \global\let\HoLogoCss@LaTeXTeX\relax
}
%    \end{macrocode}
%    \end{macro}
%
% \subsubsection{\hologo{LaTeXe}}
%
%    \begin{macro}{\HoLogo@LaTeXe}
%    Source: \hologo{LaTeX} kernel
%    \begin{macrocode}
\def\HoLogo@LaTeXe#1{%
  \hologo{LaTeX}%
  \kern.15em%
  \hbox{%
    \HOLOGO@MathSetup
    2%
    $_{\textstyle\varepsilon}$%
  }%
}
%    \end{macrocode}
%    \end{macro}
%
%    \begin{macro}{\HoLogoCs@LaTeXe}
%    \begin{macrocode}
\ifnum64=`\^^^^0040\relax % test for big chars of LuaTeX/XeTeX
  \catcode`\$=9 %
  \catcode`\&=14 %
\else
  \catcode`\$=14 %
  \catcode`\&=9 %
\fi
\def\HoLogoCs@LaTeXe#1{%
  LaTeX2%
$ \string ^^^^0395%
& e%
}%
\catcode`\$=3 %
\catcode`\&=4 %
%    \end{macrocode}
%    \end{macro}
%
%    \begin{macro}{\HoLogoBkm@LaTeXe}
%    \begin{macrocode}
\def\HoLogoBkm@LaTeXe#1{%
  \hologo{LaTeX}%
  2%
  \HOLOGO@PdfdocUnicode{e}{\textepsilon}%
}
%    \end{macrocode}
%    \end{macro}
%
%    \begin{macro}{\HoLogoHtml@LaTeXe}
%    \begin{macrocode}
\def\HoLogoHtml@LaTeXe#1{%
  \HoLogoCss@LaTeXe
  \HOLOGO@Span{LaTeX2e}{%
    \hologo{LaTeX}%
    \HOLOGO@Span{2}{2}%
    \HOLOGO@Span{e}{%
      \HOLOGO@MathSetup
      \ensuremath{\textstyle\varepsilon}%
    }%
  }%
}
%    \end{macrocode}
%    \end{macro}
%    \begin{macro}{\HoLogoCss@LaTeXe}
%    \begin{macrocode}
\def\HoLogoCss@LaTeXe{%
  \Css{%
    span.HoLogo-LaTeX2e span.HoLogo-2{%
      padding-left:.15em;%
    }%
  }%
  \Css{%
    span.HoLogo-LaTeX2e span.HoLogo-e{%
      position:relative;%
      top:.35ex;%
      text-decoration:none;%
    }%
  }%
  \global\let\HoLogoCss@LaTeXe\relax
}
%    \end{macrocode}
%    \end{macro}
%
%    \begin{macro}{\HoLogo@LaTeX2e}
%    \begin{macrocode}
\expandafter
\let\csname HoLogo@LaTeX2e\endcsname\HoLogo@LaTeXe
%    \end{macrocode}
%    \end{macro}
%    \begin{macro}{\HoLogoCs@LaTeX2e}
%    \begin{macrocode}
\expandafter
\let\csname HoLogoCs@LaTeX2e\endcsname\HoLogoCs@LaTeXe
%    \end{macrocode}
%    \end{macro}
%    \begin{macro}{\HoLogoBkm@LaTeX2e}
%    \begin{macrocode}
\expandafter
\let\csname HoLogoBkm@LaTeX2e\endcsname\HoLogoBkm@LaTeXe
%    \end{macrocode}
%    \end{macro}
%    \begin{macro}{\HoLogoHtml@LaTeX2e}
%    \begin{macrocode}
\expandafter
\let\csname HoLogoHtml@LaTeX2e\endcsname\HoLogoHtml@LaTeXe
%    \end{macrocode}
%    \end{macro}
%
% \subsubsection{\hologo{LaTeX3}}
%
%    \begin{macro}{\HoLogo@LaTeX3}
%    Source: \hologo{LaTeX} kernel
%    \begin{macrocode}
\expandafter\def\csname HoLogo@LaTeX3\endcsname#1{%
  \hologo{LaTeX}%
  3%
}
%    \end{macrocode}
%    \end{macro}
%
%    \begin{macro}{\HoLogoBkm@LaTeX3}
%    \begin{macrocode}
\expandafter\def\csname HoLogoBkm@LaTeX3\endcsname#1{%
  \hologo{LaTeX}%
  3%
}
%    \end{macrocode}
%    \end{macro}
%    \begin{macro}{\HoLogoHtml@LaTeX3}
%    \begin{macrocode}
\expandafter
\let\csname HoLogoHtml@LaTeX3\expandafter\endcsname
\csname HoLogo@LaTeX3\endcsname
%    \end{macrocode}
%    \end{macro}
%
% \subsubsection{\hologo{LaTeXML}}
%
%    \begin{macro}{\HoLogo@LaTeXML}
%    \begin{macrocode}
\def\HoLogo@LaTeXML#1{%
  \HOLOGO@mbox{%
    \hologo{La}%
    \kern-.15em%
    T%
    \kern-.1667em%
    \lower.5ex\hbox{E}%
    \kern-.125em%
    \HoLogoFont@font{LaTeXML}{sc}{xml}%
  }%
}
%    \end{macrocode}
%    \end{macro}
%    \begin{macro}{\HoLogoHtml@pdfLaTeX}
%    \begin{macrocode}
\def\HoLogoHtml@LaTeXML#1{%
  \HOLOGO@Span{LaTeXML}{%
    \HoLogoCss@LaTeX
    \HoLogoCss@TeX
    \HOLOGO@Span{LaTeX}{%
      L%
      \HOLOGO@Span{a}{%
        A%
      }%
    }%
    \HOLOGO@Span{TeX}{%
      T%
      \HOLOGO@Span{e}{%
        E%
      }%
    }%
    \HCode{<span style="font-variant: small-caps;">}%
    xml%
    \HCode{</span>}%
  }%
}
%    \end{macrocode}
%    \end{macro}
%
% \subsubsection{\hologo{eTeX}}
%
%    \begin{macro}{\HoLogo@eTeX}
%    Source: package \xpackage{etex}
%    \begin{macrocode}
\def\HoLogo@eTeX#1{%
  \ltx@mbox{%
    \HOLOGO@MathSetup
    $\varepsilon$%
    -%
    \HOLOGO@NegativeKerning{-T,T-,To}%
    \hologo{TeX}%
  }%
}
%    \end{macrocode}
%    \end{macro}
%    \begin{macro}{\HoLogoCs@eTeX}
%    \begin{macrocode}
\ifnum64=`\^^^^0040\relax % test for big chars of LuaTeX/XeTeX
  \catcode`\$=9 %
  \catcode`\&=14 %
\else
  \catcode`\$=14 %
  \catcode`\&=9 %
\fi
\def\HoLogoCs@eTeX#1{%
$ #1{\string ^^^^0395}{\string ^^^^03b5}%
& #1{e}{E}%
  TeX%
}%
\catcode`\$=3 %
\catcode`\&=4 %
%    \end{macrocode}
%    \end{macro}
%    \begin{macro}{\HoLogoBkm@eTeX}
%    \begin{macrocode}
\def\HoLogoBkm@eTeX#1{%
  \HOLOGO@PdfdocUnicode{#1{e}{E}}{\textepsilon}%
  -%
  \hologo{TeX}%
}
%    \end{macrocode}
%    \end{macro}
%    \begin{macro}{\HoLogoHtml@eTeX}
%    \begin{macrocode}
\def\HoLogoHtml@eTeX#1{%
  \ltx@mbox{%
    \HOLOGO@MathSetup
    $\varepsilon$%
    -%
    \hologo{TeX}%
  }%
}
%    \end{macrocode}
%    \end{macro}
%
% \subsubsection{\hologo{iniTeX}}
%
%    \begin{macro}{\HoLogo@iniTeX}
%    \begin{macrocode}
\def\HoLogo@iniTeX#1{%
  \HOLOGO@mbox{%
    #1{i}{I}ni\hologo{TeX}%
  }%
}
%    \end{macrocode}
%    \end{macro}
%    \begin{macro}{\HoLogoCs@iniTeX}
%    \begin{macrocode}
\def\HoLogoCs@iniTeX#1{#1{i}{I}niTeX}
%    \end{macrocode}
%    \end{macro}
%    \begin{macro}{\HoLogoBkm@iniTeX}
%    \begin{macrocode}
\def\HoLogoBkm@iniTeX#1{%
  #1{i}{I}ni\hologo{TeX}%
}
%    \end{macrocode}
%    \end{macro}
%    \begin{macro}{\HoLogoHtml@iniTeX}
%    \begin{macrocode}
\let\HoLogoHtml@iniTeX\HoLogo@iniTeX
%    \end{macrocode}
%    \end{macro}
%
% \subsubsection{\hologo{virTeX}}
%
%    \begin{macro}{\HoLogo@virTeX}
%    \begin{macrocode}
\def\HoLogo@virTeX#1{%
  \HOLOGO@mbox{%
    #1{v}{V}ir\hologo{TeX}%
  }%
}
%    \end{macrocode}
%    \end{macro}
%    \begin{macro}{\HoLogoCs@virTeX}
%    \begin{macrocode}
\def\HoLogoCs@virTeX#1{#1{v}{V}irTeX}
%    \end{macrocode}
%    \end{macro}
%    \begin{macro}{\HoLogoBkm@virTeX}
%    \begin{macrocode}
\def\HoLogoBkm@virTeX#1{%
  #1{v}{V}ir\hologo{TeX}%
}
%    \end{macrocode}
%    \end{macro}
%    \begin{macro}{\HoLogoHtml@virTeX}
%    \begin{macrocode}
\let\HoLogoHtml@virTeX\HoLogo@virTeX
%    \end{macrocode}
%    \end{macro}
%
% \subsubsection{\hologo{SliTeX}}
%
% \paragraph{Definitions of the three variants.}
%
%    \begin{macro}{\HoLogo@SLiTeX@lift}
%    \begin{macrocode}
\def\HoLogo@SLiTeX@lift#1{%
  \HoLogoFont@font{SliTeX}{rm}{%
    S%
    \kern-.06em%
    L%
    \kern-.18em%
    \raise.32ex\hbox{\HoLogoFont@font{SliTeX}{sc}{i}}%
    \HOLOGO@discretionary
    \kern-.06em%
    \hologo{TeX}%
  }%
}
%    \end{macrocode}
%    \end{macro}
%    \begin{macro}{\HoLogoBkm@SLiTeX@lift}
%    \begin{macrocode}
\def\HoLogoBkm@SLiTeX@lift#1{SLiTeX}
%    \end{macrocode}
%    \end{macro}
%    \begin{macro}{\HoLogoHtml@SLiTeX@lift}
%    \begin{macrocode}
\def\HoLogoHtml@SLiTeX@lift#1{%
  \HoLogoCss@SLiTeX@lift
  \HOLOGO@Span{SLiTeX-lift}{%
    \HoLogoFont@font{SliTeX}{rm}{%
      S%
      \HOLOGO@Span{L}{L}%
      \HOLOGO@Span{i}{i}%
      \hologo{TeX}%
    }%
  }%
}
%    \end{macrocode}
%    \end{macro}
%    \begin{macro}{\HoLogoCss@SLiTeX@lift}
%    \begin{macrocode}
\def\HoLogoCss@SLiTeX@lift{%
  \Css{%
    span.HoLogo-SLiTeX-lift span.HoLogo-L{%
      margin-left:-.06em;%
      margin-right:-.18em;%
    }%
  }%
  \Css{%
    span.HoLogo-SLiTeX-lift span.HoLogo-i{%
      position:relative;%
      top:-.32ex;%
      margin-right:-.06em;%
      font-variant:small-caps;%
    }%
  }%
  \global\let\HoLogoCss@SLiTeX@lift\relax
}
%    \end{macrocode}
%    \end{macro}
%
%    \begin{macro}{\HoLogo@SliTeX@simple}
%    \begin{macrocode}
\def\HoLogo@SliTeX@simple#1{%
  \HoLogoFont@font{SliTeX}{rm}{%
    \ltx@mbox{%
      \HoLogoFont@font{SliTeX}{sc}{Sli}%
    }%
    \HOLOGO@discretionary
    \hologo{TeX}%
  }%
}
%    \end{macrocode}
%    \end{macro}
%    \begin{macro}{\HoLogoBkm@SliTeX@simple}
%    \begin{macrocode}
\def\HoLogoBkm@SliTeX@simple#1{SliTeX}
%    \end{macrocode}
%    \end{macro}
%    \begin{macro}{\HoLogoHtml@SliTeX@simple}
%    \begin{macrocode}
\let\HoLogoHtml@SliTeX@simple\HoLogo@SliTeX@simple
%    \end{macrocode}
%    \end{macro}
%
%    \begin{macro}{\HoLogo@SliTeX@narrow}
%    \begin{macrocode}
\def\HoLogo@SliTeX@narrow#1{%
  \HoLogoFont@font{SliTeX}{rm}{%
    \ltx@mbox{%
      S%
      \kern-.06em%
      \HoLogoFont@font{SliTeX}{sc}{%
        l%
        \kern-.035em%
        i%
      }%
    }%
    \HOLOGO@discretionary
    \kern-.06em%
    \hologo{TeX}%
  }%
}
%    \end{macrocode}
%    \end{macro}
%    \begin{macro}{\HoLogoBkm@SliTeX@narrow}
%    \begin{macrocode}
\def\HoLogoBkm@SliTeX@narrow#1{SliTeX}
%    \end{macrocode}
%    \end{macro}
%    \begin{macro}{\HoLogoHtml@SliTeX@narrow}
%    \begin{macrocode}
\def\HoLogoHtml@SliTeX@narrow#1{%
  \HoLogoCss@SliTeX@narrow
  \HOLOGO@Span{SliTeX-narrow}{%
    \HoLogoFont@font{SliTeX}{rm}{%
      S%
        \HOLOGO@Span{l}{l}%
        \HOLOGO@Span{i}{i}%
      \hologo{TeX}%
    }%
  }%
}
%    \end{macrocode}
%    \end{macro}
%    \begin{macro}{\HoLogoCss@SliTeX@narrow}
%    \begin{macrocode}
\def\HoLogoCss@SliTeX@narrow{%
  \Css{%
    span.HoLogo-SliTeX-narrow span.HoLogo-l{%
      margin-left:-.06em;%
      margin-right:-.035em;%
      font-variant:small-caps;%
    }%
  }%
  \Css{%
    span.HoLogo-SliTeX-narrow span.HoLogo-i{%
      margin-right:-.06em;%
      font-variant:small-caps;%
    }%
  }%
  \global\let\HoLogoCss@SliTeX@narrow\relax
}
%    \end{macrocode}
%    \end{macro}
%
% \paragraph{Macro set completion.}
%
%    \begin{macro}{\HoLogo@SLiTeX@simple}
%    \begin{macrocode}
\def\HoLogo@SLiTeX@simple{\HoLogo@SliTeX@simple}
%    \end{macrocode}
%    \end{macro}
%    \begin{macro}{\HoLogoBkm@SLiTeX@simple}
%    \begin{macrocode}
\def\HoLogoBkm@SLiTeX@simple{\HoLogoBkm@SliTeX@simple}
%    \end{macrocode}
%    \end{macro}
%    \begin{macro}{\HoLogoHtml@SLiTeX@simple}
%    \begin{macrocode}
\def\HoLogoHtml@SLiTeX@simple{\HoLogoHtml@SliTeX@simple}
%    \end{macrocode}
%    \end{macro}
%
%    \begin{macro}{\HoLogo@SLiTeX@narrow}
%    \begin{macrocode}
\def\HoLogo@SLiTeX@narrow{\HoLogo@SliTeX@narrow}
%    \end{macrocode}
%    \end{macro}
%    \begin{macro}{\HoLogoBkm@SLiTeX@narrow}
%    \begin{macrocode}
\def\HoLogoBkm@SLiTeX@narrow{\HoLogoBkm@SliTeX@narrow}
%    \end{macrocode}
%    \end{macro}
%    \begin{macro}{\HoLogoHtml@SLiTeX@narrow}
%    \begin{macrocode}
\def\HoLogoHtml@SLiTeX@narrow{\HoLogoHtml@SliTeX@narrow}
%    \end{macrocode}
%    \end{macro}
%
%    \begin{macro}{\HoLogo@SliTeX@lift}
%    \begin{macrocode}
\def\HoLogo@SliTeX@lift{\HoLogo@SLiTeX@lift}
%    \end{macrocode}
%    \end{macro}
%    \begin{macro}{\HoLogoBkm@SliTeX@lift}
%    \begin{macrocode}
\def\HoLogoBkm@SliTeX@lift{\HoLogoBkm@SLiTeX@lift}
%    \end{macrocode}
%    \end{macro}
%    \begin{macro}{\HoLogoHtml@SliTeX@lift}
%    \begin{macrocode}
\def\HoLogoHtml@SliTeX@lift{\HoLogoHtml@SLiTeX@lift}
%    \end{macrocode}
%    \end{macro}
%
% \paragraph{Defaults.}
%
%    \begin{macro}{\HoLogo@SLiTeX}
%    \begin{macrocode}
\def\HoLogo@SLiTeX{\HoLogo@SLiTeX@lift}
%    \end{macrocode}
%    \end{macro}
%    \begin{macro}{\HoLogoBkm@SLiTeX}
%    \begin{macrocode}
\def\HoLogoBkm@SLiTeX{\HoLogoBkm@SLiTeX@lift}
%    \end{macrocode}
%    \end{macro}
%    \begin{macro}{\HoLogoHtml@SLiTeX}
%    \begin{macrocode}
\def\HoLogoHtml@SLiTeX{\HoLogoHtml@SLiTeX@lift}
%    \end{macrocode}
%    \end{macro}
%
%    \begin{macro}{\HoLogo@SliTeX}
%    \begin{macrocode}
\def\HoLogo@SliTeX{\HoLogo@SliTeX@narrow}
%    \end{macrocode}
%    \end{macro}
%    \begin{macro}{\HoLogoBkm@SliTeX}
%    \begin{macrocode}
\def\HoLogoBkm@SliTeX{\HoLogoBkm@SliTeX@narrow}
%    \end{macrocode}
%    \end{macro}
%    \begin{macro}{\HoLogoHtml@SliTeX}
%    \begin{macrocode}
\def\HoLogoHtml@SliTeX{\HoLogoHtml@SliTeX@narrow}
%    \end{macrocode}
%    \end{macro}
%
% \subsubsection{\hologo{LuaTeX}}
%
%    \begin{macro}{\HoLogo@LuaTeX}
%    The kerning is an idea of Hans Hagen, see mailing list
%    `luatex at tug dot org' in March 2010.
%    \begin{macrocode}
\def\HoLogo@LuaTeX#1{%
  \HOLOGO@mbox{%
    Lua%
    \HOLOGO@NegativeKerning{aT,oT,To}%
    \hologo{TeX}%
  }%
}
%    \end{macrocode}
%    \end{macro}
%    \begin{macro}{\HoLogoHtml@LuaTeX}
%    \begin{macrocode}
\let\HoLogoHtml@LuaTeX\HoLogo@LuaTeX
%    \end{macrocode}
%    \end{macro}
%
% \subsubsection{\hologo{LuaLaTeX}}
%
%    \begin{macro}{\HoLogo@LuaLaTeX}
%    \begin{macrocode}
\def\HoLogo@LuaLaTeX#1{%
  \HOLOGO@mbox{%
    Lua%
    \hologo{LaTeX}%
  }%
}
%    \end{macrocode}
%    \end{macro}
%    \begin{macro}{\HoLogoHtml@LuaLaTeX}
%    \begin{macrocode}
\let\HoLogoHtml@LuaLaTeX\HoLogo@LuaLaTeX
%    \end{macrocode}
%    \end{macro}
%
% \subsubsection{\hologo{XeTeX}, \hologo{XeLaTeX}}
%
%    \begin{macro}{\HOLOGO@IfCharExists}
%    \begin{macrocode}
\ifluatex
  \ifnum\luatexversion<36 %
  \else
    \def\HOLOGO@IfCharExists#1{%
      \ifnum
        \directlua{%
           if luaotfload and luaotfload.aux then
             if luaotfload.aux.font_has_glyph(%
                    font.current(), \number#1) then % 	 
	       tex.print("1") % 	 
	     end % 	 
	   elseif font and font.fonts and font.current then %
            local f = font.fonts[font.current()]%
            if f.characters and f.characters[\number#1] then %
              tex.print("1")%
            end %
          end%
        }0=\ltx@zero
        \expandafter\ltx@secondoftwo
      \else
        \expandafter\ltx@firstoftwo
      \fi
    }%
  \fi
\fi
\ltx@IfUndefined{HOLOGO@IfCharExists}{%
  \def\HOLOGO@@IfCharExists#1{%
    \begingroup
      \tracinglostchars=\ltx@zero
      \setbox\ltx@zero=\hbox{%
        \kern7sp\char#1\relax
        \ifnum\lastkern>\ltx@zero
          \expandafter\aftergroup\csname iffalse\endcsname
        \else
          \expandafter\aftergroup\csname iftrue\endcsname
        \fi
      }%
      % \if{true|false} from \aftergroup
      \endgroup
      \expandafter\ltx@firstoftwo
    \else
      \endgroup
      \expandafter\ltx@secondoftwo
    \fi
  }%
  \ifxetex
    \ltx@IfUndefined{XeTeXfonttype}{}{%
      \ltx@IfUndefined{XeTeXcharglyph}{}{%
        \def\HOLOGO@IfCharExists#1{%
          \ifnum\XeTeXfonttype\font>\ltx@zero
            \expandafter\ltx@firstofthree
          \else
            \expandafter\ltx@gobble
          \fi
          {%
            \ifnum\XeTeXcharglyph#1>\ltx@zero
              \expandafter\ltx@firstoftwo
            \else
              \expandafter\ltx@secondoftwo
            \fi
          }%
          \HOLOGO@@IfCharExists{#1}%
        }%
      }%
    }%
  \fi
}{}
\ltx@ifundefined{HOLOGO@IfCharExists}{%
  \ifnum64=`\^^^^0040\relax % test for big chars of LuaTeX/XeTeX
    \let\HOLOGO@IfCharExists\HOLOGO@@IfCharExists
  \else
    \def\HOLOGO@IfCharExists#1{%
      \ifnum#1>255 %
        \expandafter\ltx@fourthoffour
      \fi
      \HOLOGO@@IfCharExists{#1}%
    }%
  \fi
}{}
%    \end{macrocode}
%    \end{macro}
%
%    \begin{macro}{\HoLogo@Xe}
%    Source: package \xpackage{dtklogos}
%    \begin{macrocode}
\def\HoLogo@Xe#1{%
  X%
  \kern-.1em\relax
  \HOLOGO@IfCharExists{"018E}{%
    \lower.5ex\hbox{\char"018E}%
  }{%
    \chardef\HOLOGO@choice=\ltx@zero
    \ifdim\fontdimen\ltx@one\font>0pt %
      \ltx@IfUndefined{rotatebox}{%
        \ltx@IfUndefined{pgftext}{%
          \ltx@IfUndefined{psscalebox}{%
            \ltx@IfUndefined{HOLOGO@ScaleBox@\hologoDriver}{%
            }{%
              \chardef\HOLOGO@choice=4 %
            }%
          }{%
            \chardef\HOLOGO@choice=3 %
          }%
        }{%
          \chardef\HOLOGO@choice=2 %
        }%
      }{%
        \chardef\HOLOGO@choice=1 %
      }%
      \ifcase\HOLOGO@choice
        \HOLOGO@WarningUnsupportedDriver{Xe}%
        e%
      \or % 1: \rotatebox
        \begingroup
          \setbox\ltx@zero\hbox{\rotatebox{180}{E}}%
          \ltx@LocDimenA=\dp\ltx@zero
          \advance\ltx@LocDimenA by -.5ex\relax
          \raise\ltx@LocDimenA\box\ltx@zero
        \endgroup
      \or % 2: \pgftext
        \lower.5ex\hbox{%
          \pgfpicture
            \pgftext[rotate=180]{E}%
          \endpgfpicture
        }%
      \or % 3: \psscalebox
        \begingroup
          \setbox\ltx@zero\hbox{\psscalebox{-1 -1}{E}}%
          \ltx@LocDimenA=\dp\ltx@zero
          \advance\ltx@LocDimenA by -.5ex\relax
          \raise\ltx@LocDimenA\box\ltx@zero
        \endgroup
      \or % 4: \HOLOGO@PointReflectBox
        \lower.5ex\hbox{\HOLOGO@PointReflectBox{E}}%
      \else
        \@PackageError{hologo}{Internal error (choice/it}\@ehc
      \fi
    \else
      \ltx@IfUndefined{reflectbox}{%
        \ltx@IfUndefined{pgftext}{%
          \ltx@IfUndefined{psscalebox}{%
            \ltx@IfUndefined{HOLOGO@ScaleBox@\hologoDriver}{%
            }{%
              \chardef\HOLOGO@choice=4 %
            }%
          }{%
            \chardef\HOLOGO@choice=3 %
          }%
        }{%
          \chardef\HOLOGO@choice=2 %
        }%
      }{%
        \chardef\HOLOGO@choice=1 %
      }%
      \ifcase\HOLOGO@choice
        \HOLOGO@WarningUnsupportedDriver{Xe}%
        e%
      \or % 1: reflectbox
        \lower.5ex\hbox{%
          \reflectbox{E}%
        }%
      \or % 2: \pgftext
        \lower.5ex\hbox{%
          \pgfpicture
            \pgftransformxscale{-1}%
            \pgftext{E}%
          \endpgfpicture
        }%
      \or % 3: \psscalebox
        \lower.5ex\hbox{%
          \psscalebox{-1 1}{E}%
        }%
      \or % 4: \HOLOGO@Reflectbox
        \lower.5ex\hbox{%
          \HOLOGO@ReflectBox{E}%
        }%
      \else
        \@PackageError{hologo}{Internal error (choice/up)}\@ehc
      \fi
    \fi
  }%
}
%    \end{macrocode}
%    \end{macro}
%    \begin{macro}{\HoLogoHtml@Xe}
%    \begin{macrocode}
\def\HoLogoHtml@Xe#1{%
  \HoLogoCss@Xe
  \HOLOGO@Span{Xe}{%
    X%
    \HOLOGO@Span{e}{%
      \HCode{&\ltx@hashchar x018e;}%
    }%
  }%
}
%    \end{macrocode}
%    \end{macro}
%    \begin{macro}{\HoLogoCss@Xe}
%    \begin{macrocode}
\def\HoLogoCss@Xe{%
  \Css{%
    span.HoLogo-Xe span.HoLogo-e{%
      position:relative;%
      top:.5ex;%
      left-margin:-.1em;%
    }%
  }%
  \global\let\HoLogoCss@Xe\relax
}
%    \end{macrocode}
%    \end{macro}
%
%    \begin{macro}{\HoLogo@XeTeX}
%    \begin{macrocode}
\def\HoLogo@XeTeX#1{%
  \hologo{Xe}%
  \kern-.15em\relax
  \hologo{TeX}%
}
%    \end{macrocode}
%    \end{macro}
%
%    \begin{macro}{\HoLogoHtml@XeTeX}
%    \begin{macrocode}
\def\HoLogoHtml@XeTeX#1{%
  \HoLogoCss@XeTeX
  \HOLOGO@Span{XeTeX}{%
    \hologo{Xe}%
    \hologo{TeX}%
  }%
}
%    \end{macrocode}
%    \end{macro}
%    \begin{macro}{\HoLogoCss@XeTeX}
%    \begin{macrocode}
\def\HoLogoCss@XeTeX{%
  \Css{%
    span.HoLogo-XeTeX span.HoLogo-TeX{%
      margin-left:-.15em;%
    }%
  }%
  \global\let\HoLogoCss@XeTeX\relax
}
%    \end{macrocode}
%    \end{macro}
%
%    \begin{macro}{\HoLogo@XeLaTeX}
%    \begin{macrocode}
\def\HoLogo@XeLaTeX#1{%
  \hologo{Xe}%
  \kern-.13em%
  \hologo{LaTeX}%
}
%    \end{macrocode}
%    \end{macro}
%    \begin{macro}{\HoLogoHtml@XeLaTeX}
%    \begin{macrocode}
\def\HoLogoHtml@XeLaTeX#1{%
  \HoLogoCss@XeLaTeX
  \HOLOGO@Span{XeLaTeX}{%
    \hologo{Xe}%
    \hologo{LaTeX}%
  }%
}
%    \end{macrocode}
%    \end{macro}
%    \begin{macro}{\HoLogoCss@XeLaTeX}
%    \begin{macrocode}
\def\HoLogoCss@XeLaTeX{%
  \Css{%
    span.HoLogo-XeLaTeX span.HoLogo-Xe{%
      margin-right:-.13em;%
    }%
  }%
  \global\let\HoLogoCss@XeLaTeX\relax
}
%    \end{macrocode}
%    \end{macro}
%
% \subsubsection{\hologo{pdfTeX}, \hologo{pdfLaTeX}}
%
%    \begin{macro}{\HoLogo@pdfTeX}
%    \begin{macrocode}
\def\HoLogo@pdfTeX#1{%
  \HOLOGO@mbox{%
    #1{p}{P}df\hologo{TeX}%
  }%
}
%    \end{macrocode}
%    \end{macro}
%    \begin{macro}{\HoLogoCs@pdfTeX}
%    \begin{macrocode}
\def\HoLogoCs@pdfTeX#1{#1{p}{P}dfTeX}
%    \end{macrocode}
%    \end{macro}
%    \begin{macro}{\HoLogoBkm@pdfTeX}
%    \begin{macrocode}
\def\HoLogoBkm@pdfTeX#1{%
  #1{p}{P}df\hologo{TeX}%
}
%    \end{macrocode}
%    \end{macro}
%    \begin{macro}{\HoLogoHtml@pdfTeX}
%    \begin{macrocode}
\let\HoLogoHtml@pdfTeX\HoLogo@pdfTeX
%    \end{macrocode}
%    \end{macro}
%
%    \begin{macro}{\HoLogo@pdfLaTeX}
%    \begin{macrocode}
\def\HoLogo@pdfLaTeX#1{%
  \HOLOGO@mbox{%
    #1{p}{P}df\hologo{LaTeX}%
  }%
}
%    \end{macrocode}
%    \end{macro}
%    \begin{macro}{\HoLogoCs@pdfLaTeX}
%    \begin{macrocode}
\def\HoLogoCs@pdfLaTeX#1{#1{p}{P}dfLaTeX}
%    \end{macrocode}
%    \end{macro}
%    \begin{macro}{\HoLogoBkm@pdfLaTeX}
%    \begin{macrocode}
\def\HoLogoBkm@pdfLaTeX#1{%
  #1{p}{P}df\hologo{LaTeX}%
}
%    \end{macrocode}
%    \end{macro}
%    \begin{macro}{\HoLogoHtml@pdfLaTeX}
%    \begin{macrocode}
\let\HoLogoHtml@pdfLaTeX\HoLogo@pdfLaTeX
%    \end{macrocode}
%    \end{macro}
%
% \subsubsection{\hologo{VTeX}}
%
%    \begin{macro}{\HoLogo@VTeX}
%    \begin{macrocode}
\def\HoLogo@VTeX#1{%
  \HOLOGO@mbox{%
    V\hologo{TeX}%
  }%
}
%    \end{macrocode}
%    \end{macro}
%    \begin{macro}{\HoLogoHtml@VTeX}
%    \begin{macrocode}
\let\HoLogoHtml@VTeX\HoLogo@VTeX
%    \end{macrocode}
%    \end{macro}
%
% \subsubsection{\hologo{AmS}, \dots}
%
%    Source: class \xclass{amsdtx}
%
%    \begin{macro}{\HoLogo@AmS}
%    \begin{macrocode}
\def\HoLogo@AmS#1{%
  \HoLogoFont@font{AmS}{sy}{%
    A%
    \kern-.1667em%
    \lower.5ex\hbox{M}%
    \kern-.125em%
    S%
  }%
}
%    \end{macrocode}
%    \end{macro}
%    \begin{macro}{\HoLogoBkm@AmS}
%    \begin{macrocode}
\def\HoLogoBkm@AmS#1{AmS}
%    \end{macrocode}
%    \end{macro}
%    \begin{macro}{\HoLogoHtml@AmS}
%    \begin{macrocode}
\def\HoLogoHtml@AmS#1{%
  \HoLogoCss@AmS
%  \HoLogoFont@font{AmS}{sy}{%
    \HOLOGO@Span{AmS}{%
      A%
      \HOLOGO@Span{M}{M}%
      S%
    }%
%   }%
}
%    \end{macrocode}
%    \end{macro}
%    \begin{macro}{\HoLogoCss@AmS}
%    \begin{macrocode}
\def\HoLogoCss@AmS{%
  \Css{%
    span.HoLogo-AmS span.HoLogo-M{%
      position:relative;%
      top:.5ex;%
      margin-left:-.1667em;%
      margin-right:-.125em;%
      text-decoration:none;%
    }%
  }%
  \global\let\HoLogoCss@AmS\relax
}
%    \end{macrocode}
%    \end{macro}
%
%    \begin{macro}{\HoLogo@AmSTeX}
%    \begin{macrocode}
\def\HoLogo@AmSTeX#1{%
  \hologo{AmS}%
  \HOLOGO@hyphen
  \hologo{TeX}%
}
%    \end{macrocode}
%    \end{macro}
%    \begin{macro}{\HoLogoBkm@AmSTeX}
%    \begin{macrocode}
\def\HoLogoBkm@AmSTeX#1{AmS-TeX}%
%    \end{macrocode}
%    \end{macro}
%    \begin{macro}{\HoLogoHtml@AmSTeX}
%    \begin{macrocode}
\let\HoLogoHtml@AmSTeX\HoLogo@AmSTeX
%    \end{macrocode}
%    \end{macro}
%
%    \begin{macro}{\HoLogo@AmSLaTeX}
%    \begin{macrocode}
\def\HoLogo@AmSLaTeX#1{%
  \hologo{AmS}%
  \HOLOGO@hyphen
  \hologo{LaTeX}%
}
%    \end{macrocode}
%    \end{macro}
%    \begin{macro}{\HoLogoBkm@AmSLaTeX}
%    \begin{macrocode}
\def\HoLogoBkm@AmSLaTeX#1{AmS-LaTeX}%
%    \end{macrocode}
%    \end{macro}
%    \begin{macro}{\HoLogoHtml@AmSLaTeX}
%    \begin{macrocode}
\let\HoLogoHtml@AmSLaTeX\HoLogo@AmSLaTeX
%    \end{macrocode}
%    \end{macro}
%
% \subsubsection{\hologo{BibTeX}}
%
%    \begin{macro}{\HoLogo@BibTeX@sc}
%    A definition of \hologo{BibTeX} is provided in
%    the documentation source for the manual of \hologo{BibTeX}
%    \cite{btxdoc}.
%\begin{quote}
%\begin{verbatim}
%\def\BibTeX{%
%  {%
%    \rm
%    B%
%    \kern-.05em%
%    {%
%      \sc
%      i%
%      \kern-.025em %
%      b%
%    }%
%    \kern-.08em
%    T%
%    \kern-.1667em%
%    \lower.7ex\hbox{E}%
%    \kern-.125em%
%    X%
%  }%
%}
%\end{verbatim}
%\end{quote}
%    \begin{macrocode}
\def\HoLogo@BibTeX@sc#1{%
  B%
  \kern-.05em%
  \HoLogoFont@font{BibTeX}{sc}{%
    i%
    \kern-.025em%
    b%
  }%
  \HOLOGO@discretionary
  \kern-.08em%
  \hologo{TeX}%
}
%    \end{macrocode}
%    \end{macro}
%    \begin{macro}{\HoLogoHtml@BibTeX@sc}
%    \begin{macrocode}
\def\HoLogoHtml@BibTeX@sc#1{%
  \HoLogoCss@BibTeX@sc
  \HOLOGO@Span{BibTeX-sc}{%
    B%
    \HOLOGO@Span{i}{i}%
    \HOLOGO@Span{b}{b}%
    \hologo{TeX}%
  }%
}
%    \end{macrocode}
%    \end{macro}
%    \begin{macro}{\HoLogoCss@BibTeX@sc}
%    \begin{macrocode}
\def\HoLogoCss@BibTeX@sc{%
  \Css{%
    span.HoLogo-BibTeX-sc span.HoLogo-i{%
      margin-left:-.05em;%
      margin-right:-.025em;%
      font-variant:small-caps;%
    }%
  }%
  \Css{%
    span.HoLogo-BibTeX-sc span.HoLogo-b{%
      margin-right:-.08em;%
      font-variant:small-caps;%
    }%
  }%
  \global\let\HoLogoCss@BibTeX@sc\relax
}
%    \end{macrocode}
%    \end{macro}
%
%    \begin{macro}{\HoLogo@BibTeX@sf}
%    Variant \xoption{sf} avoids trouble with unavailable
%    small caps fonts (e.g., bold versions of Computer Modern or
%    Latin Modern). The definition is taken from
%    package \xpackage{dtklogos} \cite{dtklogos}.
%\begin{quote}
%\begin{verbatim}
%\DeclareRobustCommand{\BibTeX}{%
%  B%
%  \kern-.05em%
%  \hbox{%
%    $\m@th$% %% force math size calculations
%    \csname S@\f@size\endcsname
%    \fontsize\sf@size\z@
%    \math@fontsfalse
%    \selectfont
%    I%
%    \kern-.025em%
%    B
%  }%
%  \kern-.08em%
%  \-%
%  \TeX
%}
%\end{verbatim}
%\end{quote}
%    \begin{macrocode}
\def\HoLogo@BibTeX@sf#1{%
  B%
  \kern-.05em%
  \HoLogoFont@font{BibTeX}{bibsf}{%
    I%
    \kern-.025em%
    B%
  }%
  \HOLOGO@discretionary
  \kern-.08em%
  \hologo{TeX}%
}
%    \end{macrocode}
%    \end{macro}
%    \begin{macro}{\HoLogoHtml@BibTeX@sf}
%    \begin{macrocode}
\def\HoLogoHtml@BibTeX@sf#1{%
  \HoLogoCss@BibTeX@sf
  \HOLOGO@Span{BibTeX-sf}{%
    B%
    \HoLogoFont@font{BibTeX}{bibsf}{%
      \HOLOGO@Span{i}{I}%
      B%
    }%
    \hologo{TeX}%
  }%
}
%    \end{macrocode}
%    \end{macro}
%    \begin{macro}{\HoLogoCss@BibTeX@sf}
%    \begin{macrocode}
\def\HoLogoCss@BibTeX@sf{%
  \Css{%
    span.HoLogo-BibTeX-sf span.HoLogo-i{%
      margin-left:-.05em;%
      margin-right:-.025em;%
    }%
  }%
  \Css{%
    span.HoLogo-BibTeX-sf span.HoLogo-TeX{%
      margin-left:-.08em;%
    }%
  }%
  \global\let\HoLogoCss@BibTeX@sf\relax
}
%    \end{macrocode}
%    \end{macro}
%
%    \begin{macro}{\HoLogo@BibTeX}
%    \begin{macrocode}
\def\HoLogo@BibTeX{\HoLogo@BibTeX@sf}
%    \end{macrocode}
%    \end{macro}
%    \begin{macro}{\HoLogoHtml@BibTeX}
%    \begin{macrocode}
\def\HoLogoHtml@BibTeX{\HoLogoHtml@BibTeX@sf}
%    \end{macrocode}
%    \end{macro}
%
% \subsubsection{\hologo{BibTeX8}}
%
%    \begin{macro}{\HoLogo@BibTeX8}
%    \begin{macrocode}
\expandafter\def\csname HoLogo@BibTeX8\endcsname#1{%
  \hologo{BibTeX}%
  8%
}
%    \end{macrocode}
%    \end{macro}
%
%    \begin{macro}{\HoLogoBkm@BibTeX8}
%    \begin{macrocode}
\expandafter\def\csname HoLogoBkm@BibTeX8\endcsname#1{%
  \hologo{BibTeX}%
  8%
}
%    \end{macrocode}
%    \end{macro}
%    \begin{macro}{\HoLogoHtml@BibTeX8}
%    \begin{macrocode}
\expandafter
\let\csname HoLogoHtml@BibTeX8\expandafter\endcsname
\csname HoLogo@BibTeX8\endcsname
%    \end{macrocode}
%    \end{macro}
%
% \subsubsection{\hologo{ConTeXt}}
%
%    \begin{macro}{\HoLogo@ConTeXt@simple}
%    \begin{macrocode}
\def\HoLogo@ConTeXt@simple#1{%
  \HOLOGO@mbox{Con}%
  \HOLOGO@discretionary
  \HOLOGO@mbox{\hologo{TeX}t}%
}
%    \end{macrocode}
%    \end{macro}
%    \begin{macro}{\HoLogoHtml@ConTeXt@simple}
%    \begin{macrocode}
\let\HoLogoHtml@ConTeXt@simple\HoLogo@ConTeXt@simple
%    \end{macrocode}
%    \end{macro}
%
%    \begin{macro}{\HoLogo@ConTeXt@narrow}
%    This definition of logo \hologo{ConTeXt} with variant \xoption{narrow}
%    comes from TUGboat's class \xclass{ltugboat} (version 2010/11/15 v2.8).
%    \begin{macrocode}
\def\HoLogo@ConTeXt@narrow#1{%
  \HOLOGO@mbox{C\kern-.0333emon}%
  \HOLOGO@discretionary
  \kern-.0667em%
  \HOLOGO@mbox{\hologo{TeX}\kern-.0333emt}%
}
%    \end{macrocode}
%    \end{macro}
%    \begin{macro}{\HoLogoHtml@ConTeXt@narrow}
%    \begin{macrocode}
\def\HoLogoHtml@ConTeXt@narrow#1{%
  \HoLogoCss@ConTeXt@narrow
  \HOLOGO@Span{ConTeXt-narrow}{%
    \HOLOGO@Span{C}{C}%
    on%
    \hologo{TeX}%
    t%
  }%
}
%    \end{macrocode}
%    \end{macro}
%    \begin{macro}{\HoLogoCss@ConTeXt@narrow}
%    \begin{macrocode}
\def\HoLogoCss@ConTeXt@narrow{%
  \Css{%
    span.HoLogo-ConTeXt-narrow span.HoLogo-C{%
      margin-left:-.0333em;%
    }%
  }%
  \Css{%
    span.HoLogo-ConTeXt-narrow span.HoLogo-TeX{%
      margin-left:-.0667em;%
      margin-right:-.0333em;%
    }%
  }%
  \global\let\HoLogoCss@ConTeXt@narrow\relax
}
%    \end{macrocode}
%    \end{macro}
%
%    \begin{macro}{\HoLogo@ConTeXt}
%    \begin{macrocode}
\def\HoLogo@ConTeXt{\HoLogo@ConTeXt@narrow}
%    \end{macrocode}
%    \end{macro}
%    \begin{macro}{\HoLogoHtml@ConTeXt}
%    \begin{macrocode}
\def\HoLogoHtml@ConTeXt{\HoLogoHtml@ConTeXt@narrow}
%    \end{macrocode}
%    \end{macro}
%
% \subsubsection{\hologo{emTeX}}
%
%    \begin{macro}{\HoLogo@emTeX}
%    \begin{macrocode}
\def\HoLogo@emTeX#1{%
  \HOLOGO@mbox{#1{e}{E}m}%
  \HOLOGO@discretionary
  \hologo{TeX}%
}
%    \end{macrocode}
%    \end{macro}
%    \begin{macro}{\HoLogoCs@emTeX}
%    \begin{macrocode}
\def\HoLogoCs@emTeX#1{#1{e}{E}mTeX}%
%    \end{macrocode}
%    \end{macro}
%    \begin{macro}{\HoLogoBkm@emTeX}
%    \begin{macrocode}
\def\HoLogoBkm@emTeX#1{%
  #1{e}{E}m\hologo{TeX}%
}
%    \end{macrocode}
%    \end{macro}
%    \begin{macro}{\HoLogoHtml@emTeX}
%    \begin{macrocode}
\let\HoLogoHtml@emTeX\HoLogo@emTeX
%    \end{macrocode}
%    \end{macro}
%
% \subsubsection{\hologo{ExTeX}}
%
%    \begin{macro}{\HoLogo@ExTeX}
%    The definition is taken from the FAQ of the
%    project \hologo{ExTeX}
%    \cite{ExTeX-FAQ}.
%\begin{quote}
%\begin{verbatim}
%\def\ExTeX{%
%  \textrm{% Logo always with serifs
%    \ensuremath{%
%      \textstyle
%      \varepsilon_{%
%        \kern-0.15em%
%        \mathcal{X}%
%      }%
%    }%
%    \kern-.15em%
%    \TeX
%  }%
%}
%\end{verbatim}
%\end{quote}
%    \begin{macrocode}
\def\HoLogo@ExTeX#1{%
  \HoLogoFont@font{ExTeX}{rm}{%
    \ltx@mbox{%
      \HOLOGO@MathSetup
      $%
        \textstyle
        \varepsilon_{%
          \kern-0.15em%
          \HoLogoFont@font{ExTeX}{sy}{X}%
        }%
      $%
    }%
    \HOLOGO@discretionary
    \kern-.15em%
    \hologo{TeX}%
  }%
}
%    \end{macrocode}
%    \end{macro}
%    \begin{macro}{\HoLogoHtml@ExTeX}
%    \begin{macrocode}
\def\HoLogoHtml@ExTeX#1{%
  \HoLogoCss@ExTeX
  \HoLogoFont@font{ExTeX}{rm}{%
    \HOLOGO@Span{ExTeX}{%
      \ltx@mbox{%
        \HOLOGO@MathSetup
        $\textstyle\varepsilon$%
        \HOLOGO@Span{X}{$\textstyle\chi$}%
        \hologo{TeX}%
      }%
    }%
  }%
}
%    \end{macrocode}
%    \end{macro}
%    \begin{macro}{\HoLogoBkm@ExTeX}
%    \begin{macrocode}
\def\HoLogoBkm@ExTeX#1{%
  \HOLOGO@PdfdocUnicode{#1{e}{E}x}{\textepsilon\textchi}%
  \hologo{TeX}%
}
%    \end{macrocode}
%    \end{macro}
%    \begin{macro}{\HoLogoCss@ExTeX}
%    \begin{macrocode}
\def\HoLogoCss@ExTeX{%
  \Css{%
    span.HoLogo-ExTeX{%
      font-family:serif;%
    }%
  }%
  \Css{%
    span.HoLogo-ExTeX span.HoLogo-TeX{%
      margin-left:-.15em;%
    }%
  }%
  \global\let\HoLogoCss@ExTeX\relax
}
%    \end{macrocode}
%    \end{macro}
%
% \subsubsection{\hologo{MiKTeX}}
%
%    \begin{macro}{\HoLogo@MiKTeX}
%    \begin{macrocode}
\def\HoLogo@MiKTeX#1{%
  \HOLOGO@mbox{MiK}%
  \HOLOGO@discretionary
  \hologo{TeX}%
}
%    \end{macrocode}
%    \end{macro}
%    \begin{macro}{\HoLogoHtml@MiKTeX}
%    \begin{macrocode}
\let\HoLogoHtml@MiKTeX\HoLogo@MiKTeX
%    \end{macrocode}
%    \end{macro}
%
% \subsubsection{\hologo{OzTeX} and friends}
%
%    Source: \hologo{OzTeX} FAQ \cite{OzTeX}:
%    \begin{quote}
%      |\def\OzTeX{O\kern-.03em z\kern-.15em\TeX}|\\
%      (There is no kerning in OzMF, OzMP and OzTtH.)
%    \end{quote}
%
%    \begin{macro}{\HoLogo@OzTeX}
%    \begin{macrocode}
\def\HoLogo@OzTeX#1{%
  O%
  \kern-.03em %
  z%
  \kern-.15em %
  \hologo{TeX}%
}
%    \end{macrocode}
%    \end{macro}
%    \begin{macro}{\HoLogoHtml@OzTeX}
%    \begin{macrocode}
\def\HoLogoHtml@OzTeX#1{%
  \HoLogoCss@OzTeX
  \HOLOGO@Span{OzTeX}{%
    O%
    \HOLOGO@Span{z}{z}%
    \hologo{TeX}%
  }%
}
%    \end{macrocode}
%    \end{macro}
%    \begin{macro}{\HoLogoCss@OzTeX}
%    \begin{macrocode}
\def\HoLogoCss@OzTeX{%
  \Css{%
    span.HoLogo-OzTeX span.HoLogo-z{%
      margin-left:-.03em;%
      margin-right:-.15em;%
    }%
  }%
  \global\let\HoLogoCss@OzTeX\relax
}
%    \end{macrocode}
%    \end{macro}
%
%    \begin{macro}{\HoLogo@OzMF}
%    \begin{macrocode}
\def\HoLogo@OzMF#1{%
  \HOLOGO@mbox{OzMF}%
}
%    \end{macrocode}
%    \end{macro}
%    \begin{macro}{\HoLogo@OzMP}
%    \begin{macrocode}
\def\HoLogo@OzMP#1{%
  \HOLOGO@mbox{OzMP}%
}
%    \end{macrocode}
%    \end{macro}
%    \begin{macro}{\HoLogo@OzTtH}
%    \begin{macrocode}
\def\HoLogo@OzTtH#1{%
  \HOLOGO@mbox{OzTtH}%
}
%    \end{macrocode}
%    \end{macro}
%
% \subsubsection{\hologo{PCTeX}}
%
%    \begin{macro}{\HoLogo@PCTeX}
%    \begin{macrocode}
\def\HoLogo@PCTeX#1{%
  \HOLOGO@mbox{PC}%
  \hologo{TeX}%
}
%    \end{macrocode}
%    \end{macro}
%    \begin{macro}{\HoLogoHtml@PCTeX}
%    \begin{macrocode}
\let\HoLogoHtml@PCTeX\HoLogo@PCTeX
%    \end{macrocode}
%    \end{macro}
%
% \subsubsection{\hologo{PiCTeX}}
%
%    The original definitions from \xfile{pictex.tex} \cite{PiCTeX}:
%\begin{quote}
%\begin{verbatim}
%\def\PiC{%
%  P%
%  \kern-.12em%
%  \lower.5ex\hbox{I}%
%  \kern-.075em%
%  C%
%}
%\def\PiCTeX{%
%  \PiC
%  \kern-.11em%
%  \TeX
%}
%\end{verbatim}
%\end{quote}
%
%    \begin{macro}{\HoLogo@PiC}
%    \begin{macrocode}
\def\HoLogo@PiC#1{%
  P%
  \kern-.12em%
  \lower.5ex\hbox{I}%
  \kern-.075em%
  C%
  \HOLOGO@SpaceFactor
}
%    \end{macrocode}
%    \end{macro}
%    \begin{macro}{\HoLogoHtml@PiC}
%    \begin{macrocode}
\def\HoLogoHtml@PiC#1{%
  \HoLogoCss@PiC
  \HOLOGO@Span{PiC}{%
    P%
    \HOLOGO@Span{i}{I}%
    C%
  }%
}
%    \end{macrocode}
%    \end{macro}
%    \begin{macro}{\HoLogoCss@PiC}
%    \begin{macrocode}
\def\HoLogoCss@PiC{%
  \Css{%
    span.HoLogo-PiC span.HoLogo-i{%
      position:relative;%
      top:.5ex;%
      margin-left:-.12em;%
      margin-right:-.075em;%
      text-decoration:none;%
    }%
  }%
  \global\let\HoLogoCss@PiC\relax
}
%    \end{macrocode}
%    \end{macro}
%
%    \begin{macro}{\HoLogo@PiCTeX}
%    \begin{macrocode}
\def\HoLogo@PiCTeX#1{%
  \hologo{PiC}%
  \HOLOGO@discretionary
  \kern-.11em%
  \hologo{TeX}%
}
%    \end{macrocode}
%    \end{macro}
%    \begin{macro}{\HoLogoHtml@PiCTeX}
%    \begin{macrocode}
\def\HoLogoHtml@PiCTeX#1{%
  \HoLogoCss@PiCTeX
  \HOLOGO@Span{PiCTeX}{%
    \hologo{PiC}%
    \hologo{TeX}%
  }%
}
%    \end{macrocode}
%    \end{macro}
%    \begin{macro}{\HoLogoCss@PiCTeX}
%    \begin{macrocode}
\def\HoLogoCss@PiCTeX{%
  \Css{%
    span.HoLogo-PiCTeX span.HoLogo-PiC{%
      margin-right:-.11em;%
    }%
  }%
  \global\let\HoLogoCss@PiCTeX\relax
}
%    \end{macrocode}
%    \end{macro}
%
% \subsubsection{\hologo{teTeX}}
%
%    \begin{macro}{\HoLogo@teTeX}
%    \begin{macrocode}
\def\HoLogo@teTeX#1{%
  \HOLOGO@mbox{#1{t}{T}e}%
  \HOLOGO@discretionary
  \hologo{TeX}%
}
%    \end{macrocode}
%    \end{macro}
%    \begin{macro}{\HoLogoCs@teTeX}
%    \begin{macrocode}
\def\HoLogoCs@teTeX#1{#1{t}{T}dfTeX}
%    \end{macrocode}
%    \end{macro}
%    \begin{macro}{\HoLogoBkm@teTeX}
%    \begin{macrocode}
\def\HoLogoBkm@teTeX#1{%
  #1{t}{T}e\hologo{TeX}%
}
%    \end{macrocode}
%    \end{macro}
%    \begin{macro}{\HoLogoHtml@teTeX}
%    \begin{macrocode}
\let\HoLogoHtml@teTeX\HoLogo@teTeX
%    \end{macrocode}
%    \end{macro}
%
% \subsubsection{\hologo{TeX4ht}}
%
%    \begin{macro}{\HoLogo@TeX4ht}
%    \begin{macrocode}
\expandafter\def\csname HoLogo@TeX4ht\endcsname#1{%
  \HOLOGO@mbox{\hologo{TeX}4ht}%
}
%    \end{macrocode}
%    \end{macro}
%    \begin{macro}{\HoLogoHtml@TeX4ht}
%    \begin{macrocode}
\expandafter
\let\csname HoLogoHtml@TeX4ht\expandafter\endcsname
\csname HoLogo@TeX4ht\endcsname
%    \end{macrocode}
%    \end{macro}
%
%
% \subsubsection{\hologo{SageTeX}}
%
%    \begin{macro}{\HoLogo@SageTeX}
%    \begin{macrocode}
\def\HoLogo@SageTeX#1{%
  \HOLOGO@mbox{Sage}%
  \HOLOGO@discretionary
  \HOLOGO@NegativeKerning{eT,oT,To}%
  \hologo{TeX}%
}
%    \end{macrocode}
%    \end{macro}
%    \begin{macro}{\HoLogoHtml@SageTeX}
%    \begin{macrocode}
\let\HoLogoHtml@SageTeX\HoLogo@SageTeX
%    \end{macrocode}
%    \end{macro}
%
% \subsection{\hologo{METAFONT} and friends}
%
%    \begin{macro}{\HoLogo@METAFONT}
%    \begin{macrocode}
\def\HoLogo@METAFONT#1{%
  \HoLogoFont@font{METAFONT}{logo}{%
    \HOLOGO@mbox{META}%
    \HOLOGO@discretionary
    \HOLOGO@mbox{FONT}%
  }%
}
%    \end{macrocode}
%    \end{macro}
%
%    \begin{macro}{\HoLogo@METAPOST}
%    \begin{macrocode}
\def\HoLogo@METAPOST#1{%
  \HoLogoFont@font{METAPOST}{logo}{%
    \HOLOGO@mbox{META}%
    \HOLOGO@discretionary
    \HOLOGO@mbox{POST}%
  }%
}
%    \end{macrocode}
%    \end{macro}
%
%    \begin{macro}{\HoLogo@MetaFun}
%    \begin{macrocode}
\def\HoLogo@MetaFun#1{%
  \HOLOGO@mbox{Meta}%
  \HOLOGO@discretionary
  \HOLOGO@mbox{Fun}%
}
%    \end{macrocode}
%    \end{macro}
%
%    \begin{macro}{\HoLogo@MetaPost}
%    \begin{macrocode}
\def\HoLogo@MetaPost#1{%
  \HOLOGO@mbox{Meta}%
  \HOLOGO@discretionary
  \HOLOGO@mbox{Post}%
}
%    \end{macrocode}
%    \end{macro}
%
% \subsection{Others}
%
% \subsubsection{\hologo{biber}}
%
%    \begin{macro}{\HoLogo@biber}
%    \begin{macrocode}
\def\HoLogo@biber#1{%
  \HOLOGO@mbox{#1{b}{B}i}%
  \HOLOGO@discretionary
  \HOLOGO@mbox{ber}%
}
%    \end{macrocode}
%    \end{macro}
%    \begin{macro}{\HoLogoCs@biber}
%    \begin{macrocode}
\def\HoLogoCs@biber#1{#1{b}{B}iber}
%    \end{macrocode}
%    \end{macro}
%    \begin{macro}{\HoLogoBkm@biber}
%    \begin{macrocode}
\def\HoLogoBkm@biber#1{%
  #1{b}{B}iber%
}
%    \end{macrocode}
%    \end{macro}
%    \begin{macro}{\HoLogoHtml@biber}
%    \begin{macrocode}
\let\HoLogoHtml@biber\HoLogo@biber
%    \end{macrocode}
%    \end{macro}
%
% \subsubsection{\hologo{KOMAScript}}
%
%    \begin{macro}{\HoLogo@KOMAScript}
%    The definition for \hologo{KOMAScript} is taken
%    from \hologo{KOMAScript} (\xfile{scrlogo.dtx}, reformatted) \cite{scrlogo}:
%\begin{quote}
%\begin{verbatim}
%\@ifundefined{KOMAScript}{%
%  \DeclareRobustCommand{\KOMAScript}{%
%    \textsf{%
%      K\kern.05em O\kern.05emM\kern.05em A%
%      \kern.1em-\kern.1em %
%      Script%
%    }%
%  }%
%}{}
%\end{verbatim}
%\end{quote}
%    \begin{macrocode}
\def\HoLogo@KOMAScript#1{%
  \HoLogoFont@font{KOMAScript}{sf}{%
    \HOLOGO@mbox{%
      K\kern.05em%
      O\kern.05em%
      M\kern.05em%
      A%
    }%
    \kern.1em%
    \HOLOGO@hyphen
    \kern.1em%
    \HOLOGO@mbox{Script}%
  }%
}
%    \end{macrocode}
%    \end{macro}
%    \begin{macro}{\HoLogoBkm@KOMAScript}
%    \begin{macrocode}
\def\HoLogoBkm@KOMAScript#1{%
  KOMA-Script%
}
%    \end{macrocode}
%    \end{macro}
%    \begin{macro}{\HoLogoHtml@KOMAScript}
%    \begin{macrocode}
\def\HoLogoHtml@KOMAScript#1{%
  \HoLogoCss@KOMAScript
  \HoLogoFont@font{KOMAScript}{sf}{%
    \HOLOGO@Span{KOMAScript}{%
      K%
      \HOLOGO@Span{O}{O}%
      M%
      \HOLOGO@Span{A}{A}%
      \HOLOGO@Span{hyphen}{-}%
      Script%
    }%
  }%
}
%    \end{macrocode}
%    \end{macro}
%    \begin{macro}{\HoLogoCss@KOMAScript}
%    \begin{macrocode}
\def\HoLogoCss@KOMAScript{%
  \Css{%
    span.HoLogo-KOMAScript{%
      font-family:sans-serif;%
    }%
  }%
  \Css{%
    span.HoLogo-KOMAScript span.HoLogo-O{%
      padding-left:.05em;%
      padding-right:.05em;%
    }%
  }%
  \Css{%
    span.HoLogo-KOMAScript span.HoLogo-A{%
      padding-left:.05em;%
    }%
  }%
  \Css{%
    span.HoLogo-KOMAScript span.HoLogo-hyphen{%
      padding-left:.1em;%
      padding-right:.1em;%
    }%
  }%
  \global\let\HoLogoCss@KOMAScript\relax
}
%    \end{macrocode}
%    \end{macro}
%
% \subsubsection{\hologo{LyX}}
%
%    \begin{macro}{\HoLogo@LyX}
%    The definition is taken from the documentation source files
%    of \hologo{LyX}, \xfile{Intro.lyx} \cite{LyX}:
%\begin{quote}
%\begin{verbatim}
%\def\LyX{%
%  \texorpdfstring{%
%    L\kern-.1667em\lower.25em\hbox{Y}\kern-.125emX\@%
%  }{%
%    LyX%
%  }%
%}
%\end{verbatim}
%\end{quote}
%    \begin{macrocode}
\def\HoLogo@LyX#1{%
  L%
  \kern-.1667em%
  \lower.25em\hbox{Y}%
  \kern-.125em%
  X%
  \HOLOGO@SpaceFactor
}
%    \end{macrocode}
%    \end{macro}
%    \begin{macro}{\HoLogoHtml@LyX}
%    \begin{macrocode}
\def\HoLogoHtml@LyX#1{%
  \HoLogoCss@LyX
  \HOLOGO@Span{LyX}{%
    L%
    \HOLOGO@Span{y}{Y}%
    X%
  }%
}
%    \end{macrocode}
%    \end{macro}
%    \begin{macro}{\HoLogoCss@LyX}
%    \begin{macrocode}
\def\HoLogoCss@LyX{%
  \Css{%
    span.HoLogo-LyX span.HoLogo-y{%
      position:relative;%
      top:.25em;%
      margin-left:-.1667em;%
      margin-right:-.125em;%
      text-decoration:none;%
    }%
  }%
  \global\let\HoLogoCss@LyX\relax
}
%    \end{macrocode}
%    \end{macro}
%
% \subsubsection{\hologo{NTS}}
%
%    \begin{macro}{\HoLogo@NTS}
%    Definition for \hologo{NTS} can be found in
%    package \xpackage{etex\textunderscore man} for the \hologo{eTeX} manual \cite{etexman}
%    and in package \xpackage{dtklogos} \cite{dtklogos}:
%\begin{quote}
%\begin{verbatim}
%\def\NTS{%
%  \leavevmode
%  \hbox{%
%    $%
%      \cal N%
%      \kern-0.35em%
%      \lower0.5ex\hbox{$\cal T$}%
%      \kern-0.2em%
%      S%
%    $%
%  }%
%}
%\end{verbatim}
%\end{quote}
%    \begin{macrocode}
\def\HoLogo@NTS#1{%
  \HoLogoFont@font{NTS}{sy}{%
    N\/%
    \kern-.35em%
    \lower.5ex\hbox{T\/}%
    \kern-.2em%
    S\/%
  }%
  \HOLOGO@SpaceFactor
}
%    \end{macrocode}
%    \end{macro}
%
% \subsubsection{\Hologo{TTH} (\hologo{TeX} to HTML translator)}
%
%    Source: \url{http://hutchinson.belmont.ma.us/tth/}
%    In the HTML source the second `T' is printed as subscript.
%\begin{quote}
%\begin{verbatim}
%T<sub>T</sub>H
%\end{verbatim}
%\end{quote}
%    \begin{macro}{\HoLogo@TTH}
%    \begin{macrocode}
\def\HoLogo@TTH#1{%
  \ltx@mbox{%
    T\HOLOGO@SubScript{T}H%
  }%
  \HOLOGO@SpaceFactor
}
%    \end{macrocode}
%    \end{macro}
%
%    \begin{macro}{\HoLogoHtml@TTH}
%    \begin{macrocode}
\def\HoLogoHtml@TTH#1{%
  T\HCode{<sub>}T\HCode{</sub>}H%
}
%    \end{macrocode}
%    \end{macro}
%
% \subsubsection{\Hologo{HanTheThanh}}
%
%    Partial source: Package \xpackage{dtklogos}.
%    The double accent is U+1EBF (latin small letter e with circumflex
%    and acute).
%    \begin{macro}{\HoLogo@HanTheThanh}
%    \begin{macrocode}
\def\HoLogo@HanTheThanh#1{%
  \ltx@mbox{H\`an}%
  \HOLOGO@space
  \ltx@mbox{%
    Th%
    \HOLOGO@IfCharExists{"1EBF}{%
      \char"1EBF\relax
    }{%
      \^e\hbox to 0pt{\hss\raise .5ex\hbox{\'{}}}%
    }%
  }%
  \HOLOGO@space
  \ltx@mbox{Th\`anh}%
}
%    \end{macrocode}
%    \end{macro}
%    \begin{macro}{\HoLogoBkm@HanTheThanh}
%    \begin{macrocode}
\def\HoLogoBkm@HanTheThanh#1{%
  H\`an %
  Th\HOLOGO@PdfdocUnicode{\^e}{\9036\277} %
  Th\`anh%
}
%    \end{macrocode}
%    \end{macro}
%    \begin{macro}{\HoLogoHtml@HanTheThanh}
%    \begin{macrocode}
\def\HoLogoHtml@HanTheThanh#1{%
  H\`an %
  Th\HCode{&\ltx@hashchar x1ebf;} %
  Th\`anh%
}
%    \end{macrocode}
%    \end{macro}
%
% \subsection{Driver detection}
%
%    \begin{macrocode}
\HOLOGO@IfExists\InputIfFileExists{%
  \InputIfFileExists{hologo.cfg}{}{}%
}{%
  \ltx@IfUndefined{pdf@filesize}{%
    \def\HOLOGO@InputIfExists{%
      \openin\HOLOGO@temp=hologo.cfg\relax
      \ifeof\HOLOGO@temp
        \closein\HOLOGO@temp
      \else
        \closein\HOLOGO@temp
        \begingroup
          \def\x{LaTeX2e}%
        \expandafter\endgroup
        \ifx\fmtname\x
          % \iffalse meta-comment
%
% File: hologo.dtx
% Version: 2016/05/12 v1.11
% Info: A logo collection with bookmark support
%
% Copyright (C) 2010-2012 by
%    Heiko Oberdiek <heiko.oberdiek at googlemail.com>
%
% This work may be distributed and/or modified under the
% conditions of the LaTeX Project Public License, either
% version 1.3c of this license or (at your option) any later
% version. This version of this license is in
%    http://www.latex-project.org/lppl/lppl-1-3c.txt
% and the latest version of this license is in
%    http://www.latex-project.org/lppl.txt
% and version 1.3 or later is part of all distributions of
% LaTeX version 2005/12/01 or later.
%
% This work has the LPPL maintenance status "maintained".
%
% This Current Maintainer of this work is Heiko Oberdiek.
%
% The Base Interpreter refers to any `TeX-Format',
% because some files are installed in TDS:tex/generic//.
%
% This work consists of the main source file hologo.dtx
% and the derived files
%    hologo.sty, hologo.pdf, hologo.ins, hologo.drv, hologo-example.tex,
%    hologo-test1.tex, hologo-test-spacefactor.tex,
%    hologo-test-list.tex.
%
% Distribution:
%    CTAN:macros/latex/contrib/oberdiek/hologo.dtx
%    CTAN:macros/latex/contrib/oberdiek/hologo.pdf
%
% Unpacking:
%    (a) If hologo.ins is present:
%           tex hologo.ins
%    (b) Without hologo.ins:
%           tex hologo.dtx
%    (c) If you insist on using LaTeX
%           latex \let\install=y\input{hologo.dtx}
%        (quote the arguments according to the demands of your shell)
%
% Documentation:
%    (a) If hologo.drv is present:
%           latex hologo.drv
%    (b) Without hologo.drv:
%           latex hologo.dtx; ...
%    The class ltxdoc loads the configuration file ltxdoc.cfg
%    if available. Here you can specify further options, e.g.
%    use A4 as paper format:
%       \PassOptionsToClass{a4paper}{article}
%
%    Programm calls to get the documentation (example):
%       pdflatex hologo.dtx
%       makeindex -s gind.ist hologo.idx
%       pdflatex hologo.dtx
%       makeindex -s gind.ist hologo.idx
%       pdflatex hologo.dtx
%
% Installation:
%    TDS:tex/generic/oberdiek/hologo.sty
%    TDS:doc/latex/oberdiek/hologo.pdf
%    TDS:doc/latex/oberdiek/example/hologo-example.tex
%    TDS:doc/latex/oberdiek/test/hologo-test1.tex
%    TDS:doc/latex/oberdiek/test/hologo-test-spacefactor.tex
%    TDS:doc/latex/oberdiek/test/hologo-test-list.tex
%    TDS:source/latex/oberdiek/hologo.dtx
%
%<*ignore>
\begingroup
  \catcode123=1 %
  \catcode125=2 %
  \def\x{LaTeX2e}%
\expandafter\endgroup
\ifcase 0\ifx\install y1\fi\expandafter
         \ifx\csname processbatchFile\endcsname\relax\else1\fi
         \ifx\fmtname\x\else 1\fi\relax
\else\csname fi\endcsname
%</ignore>
%<*install>
\input docstrip.tex
\Msg{************************************************************************}
\Msg{* Installation}
\Msg{* Package: hologo 2016/05/12 v1.11 A logo collection with bookmark support (HO)}
\Msg{************************************************************************}

\keepsilent
\askforoverwritefalse

\let\MetaPrefix\relax
\preamble

This is a generated file.

Project: hologo
Version: 2016/05/12 v1.11

Copyright (C) 2010-2012 by
   Heiko Oberdiek <heiko.oberdiek at googlemail.com>

This work may be distributed and/or modified under the
conditions of the LaTeX Project Public License, either
version 1.3c of this license or (at your option) any later
version. This version of this license is in
   http://www.latex-project.org/lppl/lppl-1-3c.txt
and the latest version of this license is in
   http://www.latex-project.org/lppl.txt
and version 1.3 or later is part of all distributions of
LaTeX version 2005/12/01 or later.

This work has the LPPL maintenance status "maintained".

This Current Maintainer of this work is Heiko Oberdiek.

The Base Interpreter refers to any `TeX-Format',
because some files are installed in TDS:tex/generic//.

This work consists of the main source file hologo.dtx
and the derived files
   hologo.sty, hologo.pdf, hologo.ins, hologo.drv, hologo-example.tex,
   hologo-test1.tex, hologo-test-spacefactor.tex,
   hologo-test-list.tex.

\endpreamble
\let\MetaPrefix\DoubleperCent

\generate{%
  \file{hologo.ins}{\from{hologo.dtx}{install}}%
  \file{hologo.drv}{\from{hologo.dtx}{driver}}%
  \usedir{tex/generic/oberdiek}%
  \file{hologo.sty}{\from{hologo.dtx}{package}}%
  \usedir{doc/latex/oberdiek/example}%
  \file{hologo-example.tex}{\from{hologo.dtx}{example}}%
  \usedir{doc/latex/oberdiek/test}%
  \file{hologo-test1.tex}{\from{hologo.dtx}{test1}}%
  \file{hologo-test-spacefactor.tex}{\from{hologo.dtx}{test-spacefactor}}%
  \file{hologo-test-list.tex}{\from{hologo.dtx}{test-list}}%
  \nopreamble
  \nopostamble
  \usedir{source/latex/oberdiek/catalogue}%
  \file{hologo.xml}{\from{hologo.dtx}{catalogue}}%
}

\catcode32=13\relax% active space
\let =\space%
\Msg{************************************************************************}
\Msg{*}
\Msg{* To finish the installation you have to move the following}
\Msg{* file into a directory searched by TeX:}
\Msg{*}
\Msg{*     hologo.sty}
\Msg{*}
\Msg{* To produce the documentation run the file `hologo.drv'}
\Msg{* through LaTeX.}
\Msg{*}
\Msg{* Happy TeXing!}
\Msg{*}
\Msg{************************************************************************}

\endbatchfile
%</install>
%<*ignore>
\fi
%</ignore>
%<*driver>
\NeedsTeXFormat{LaTeX2e}
\ProvidesFile{hologo.drv}%
  [2016/05/12 v1.11 A logo collection with bookmark support (HO)]%
\documentclass{ltxdoc}
\usepackage{holtxdoc}[2011/11/22]
\usepackage{hologo}[2016/05/12]
\usepackage{longtable}
\usepackage{array}
\usepackage{paralist}
%\usepackage[T1]{fontenc}
%\usepackage{lmodern}
\begin{document}
  \DocInput{hologo.dtx}%
\end{document}
%</driver>
% \fi
%
%
% \CharacterTable
%  {Upper-case    \A\B\C\D\E\F\G\H\I\J\K\L\M\N\O\P\Q\R\S\T\U\V\W\X\Y\Z
%   Lower-case    \a\b\c\d\e\f\g\h\i\j\k\l\m\n\o\p\q\r\s\t\u\v\w\x\y\z
%   Digits        \0\1\2\3\4\5\6\7\8\9
%   Exclamation   \!     Double quote  \"     Hash (number) \#
%   Dollar        \$     Percent       \%     Ampersand     \&
%   Acute accent  \'     Left paren    \(     Right paren   \)
%   Asterisk      \*     Plus          \+     Comma         \,
%   Minus         \-     Point         \.     Solidus       \/
%   Colon         \:     Semicolon     \;     Less than     \<
%   Equals        \=     Greater than  \>     Question mark \?
%   Commercial at \@     Left bracket  \[     Backslash     \\
%   Right bracket \]     Circumflex    \^     Underscore    \_
%   Grave accent  \`     Left brace    \{     Vertical bar  \|
%   Right brace   \}     Tilde         \~}
%
% \GetFileInfo{hologo.drv}
%
% \title{The \xpackage{hologo} package}
% \date{2016/05/12 v1.11}
% \author{Heiko Oberdiek\\\xemail{heiko.oberdiek at googlemail.com}}
%
% \maketitle
%
% \begin{abstract}
% This package starts a collection of logos with support for bookmarks
% strings.
% \end{abstract}
%
% \tableofcontents
%
% \section{Documentation}
%
% \subsection{Logo macros}
%
% \begin{declcs}{hologo} \M{name}
% \end{declcs}
% Macro \cs{hologo} sets the logo with name \meta{name}.
% The following table shows the supported names.
%
% \begingroup
%   \def\hologoEntry#1#2#3{^^A
%     #1&#2&\hologoLogoSetup{#1}{variant=#2}\hologo{#1}&#3\tabularnewline
%   }
%   \begin{longtable}{>{\ttfamily}l>{\ttfamily}lll}
%     \rmfamily\bfseries{name} & \rmfamily\bfseries variant
%     & \bfseries logo & \bfseries since\\
%     \hline
%     \endhead
%     \hologoList
%   \end{longtable}
% \endgroup
%
% \begin{declcs}{Hologo} \M{name}
% \end{declcs}
% Macro \cs{Hologo} starts the logo \meta{name} with an uppercase
% letter. As an exception small greek letters are not converted
% to uppercase. Examples, see \hologo{eTeX} and \hologo{ExTeX}.
%
% \subsection{Setup macros}
%
% The package does not support package options, but the following
% setup macros can be used to set options.
%
% \begin{declcs}{hologoSetup} \M{key value list}
% \end{declcs}
% Macro \cs{hologoSetup} sets global options.
%
% \begin{declcs}{hologoLogoSetup} \M{logo} \M{key value list}
% \end{declcs}
% Some options can also be used to configure a logo.
% These settings take precedence over global option settings.
%
% \subsection{Options}\label{sec:options}
%
% There are boolean and string options:
% \begin{description}
% \item[Boolean option:]
% It takes |true| or |false|
% as value. If the value is omitted, then |true| is used.
% \item[String option:]
% A value must be given as string. (But the string might be empty.)
% \end{description}
% The following options can be used both in \cs{hologoSetup}
% and \cs{hologoLogoSetup}:
% \begin{description}
% \def\entry#1{\item[\xoption{#1}:]}
% \entry{break}
%   enables or disables line breaks inside the logo. This setting is
%   refined by options \xoption{hyphenbreak}, \xoption{spacebreak}
%   or \xoption{discretionarybreak}.
%   Default is |false|.
% \entry{hyphenbreak}
%   enables or disables the line break right after the hyphen character.
% \entry{spacebreak}
%   enables or disables line breaks at space characters.
% \entry{discretionarybreak}
%   enables or disables line breaks at hyphenation points
%   (inserted by \cs{-}).
% \end{description}
% Macro \cs{hologoLogoSetup} also knows:
% \begin{description}
% \item[\xoption{variant}:]
%   This is a string option. It specifies a variant of a logo that
%   must exist. An empty string selects the package default variant.
% \end{description}
% Example:
% \begin{quote}
%   |\hologoSetup{break=false}|\\
%   |\hologoLogoSetup{plainTeX}{variant=hyphen,hyphenbreak}|\\
%   Then ``plain-\TeX'' contains one break point after the hyphen.
% \end{quote}
%
% \subsection{Driver options}
%
% Sometimes graphical operations are needed to construct some
% glyphs (e.g.\ \hologo{XeTeX}). If package \xpackage{graphics}
% or package \xpackage{pgf} are found, then the macros are taken
% from there. Otherwise the packge defines its own operations
% and therefore needs the driver information. Many drivers are
% detected automatically (\hologo{pdfTeX}/\hologo{LuaTeX}
% in PDF mode, \hologo{XeTeX}, \hologo{VTeX}). These have precedence
% over a driver option. The driver can be given as package option
% or using \cs{hologoDriverSetup}.
% The following list contains the recognized driver options:
% \begin{itemize}
% \item \xoption{pdftex}, \xoption{luatex}
% \item \xoption{dvipdfm}, \xoption{dvipdfmx}
% \item \xoption{dvips}, \xoption{dvipsone}, \xoption{xdvi}
% \item \xoption{xetex}
% \item \xoption{vtex}
% \end{itemize}
% The left driver of a line is the driver name that is used internally.
% The following names are aliases for drivers that use the
% same method. Therefore the entry in the \xext{log} file for
% the used driver prints the internally used driver name.
% \begin{description}
% \item[\xoption{driverfallback}:]
%   This option expects a driver that is used,
%   if the driver could not be detected automatically.
% \end{description}
%
% \begin{declcs}{hologoDriverSetup} \M{driver option}
% \end{declcs}
% The driver can also be configured after package loading
% using \cs{hologoDriverSetup}, also the way for \hologo{plainTeX}
% to setup the driver.
%
% \subsection{Font setup}
%
% Some logos require a special font, but should also be usable by
% \hologo{plainTeX}. Therefore the package provides some ways
% to influence the font settings. The options below
% take font settings as values. Both font commands
% such as \cs{sffamily} and macros that take one argument
% like \cs{textsf} can be used.
%
% \begin{declcs}{hologoFontSetup} \M{key value list}
% \end{declcs}
% Macro \cs{hologoFontSetup} sets the fonts for all logos.
% Supported keys:
% \begin{description}
% \def\entry#1{\item[\xoption{#1}:]}
% \entry{general}
%   This font is used for all logos. The default is empty.
%   That means no special font is used.
% \entry{bibsf}
%   This font is used for
%   {\hologoLogoSetup{BibTeX}{variant=sf}\hologo{BibTeX}}
%   with variant \xoption{sf}.
% \entry{rm}
%   This font is a serif font. It is used for \hologo{ExTeX}.
% \entry{sc}
%   This font specifies a small caps font. It is used for
%   {\hologoLogoSetup{BibTeX}{variant=sc}\hologo{BibTeX}}
%   with variant \xoption{sc}.
% \entry{sf}
%   This font specifies a sans serif font. The default
%   is \cs{sffamily}, then \cs{sf} is tried. Otherwise
%   a warning is given. It is used by \hologo{KOMAScript}.
% \entry{sy}
%   This is the font for math symbols (e.g. cmsy).
%   It is used by \hologo{AmS}, \hologo{NTS}, \hologo{ExTeX}.
% \entry{logo}
%   \hologo{METAFONT} and \hologo{METAPOST} are using that font.
%   In \hologo{LaTeX} \cs{logofamily} is used and
%   the definitions of package \xpackage{mflogo} are used
%   if the package is not loaded.
%   Otherwise the \cs{tenlogo} is used and defined
%   if it does not already exists.
% \end{description}
%
% \begin{declcs}{hologoLogoFontSetup} \M{logo} \M{key value list}
% \end{declcs}
% Fonts can also be set for a logo or logo component separately,
% see the following list.
% The keys are the same as for \cs{hologoFontSetup}.
%
% \begin{longtable}{>{\ttfamily}l>{\sffamily}ll}
%   \meta{logo} & keys & result\\
%   \hline
%   \endhead
%   BibTeX & bibsf & {\hologoLogoSetup{BibTeX}{variant=sf}\hologo{BibTeX}}\\[.5ex]
%   BibTeX & sc & {\hologoLogoSetup{BibTeX}{variant=sc}\hologo{BibTeX}}\\[.5ex]
%   ExTeX & rm & \hologo{ExTeX}\\
%   SliTeX & rm & \hologo{SliTeX}\\[.5ex]
%   AmS & sy & \hologo{AmS}\\
%   ExTeX & sy & \hologo{ExTeX}\\
%   NTS & sy & \hologo{NTS}\\[.5ex]
%   KOMAScript & sf & \hologo{KOMAScript}\\[.5ex]
%   METAFONT & logo & \hologo{METAFONT}\\
%   METAPOST & logo & \hologo{METAPOST}\\[.5ex]
%   SliTeX & sc \hologo{SliTeX}
% \end{longtable}
%
% \subsubsection{Font order}
%
% For all logos the font \xoption{general} is applied first.
% Example:
%\begin{quote}
%|\hologoFontSetup{general=\color{red}}|
%\end{quote}
% will print red logos.
% Then if the font uses a special font \xoption{sf}, for example,
% the font is applied that is setup by \cs{hologoLogoFontSetup}.
% If this font is not setup, then the common font setup
% by \cs{hologoFontSetup} is used. Otherwise a warning is given,
% that there is no font configured.
%
% \subsection{Additional user macros}
%
% Usually a variant of a logo is configured by using
% \cs{hologoLogoSetup}, because it is bad style to mix
% different variants of the same logo in the same text.
% There the following macros are a convenience for testing.
%
% \begin{declcs}{hologoVariant} \M{name} \M{variant}\\
%   \cs{HologoVariant} \M{name} \M{variant}
% \end{declcs}
% Logo \meta{name} is set using \meta{variant} that specifies
% explicitely which variant of the macro is used. If the argument
% is empty, then the default form of the logo is used
% (configurable by \cs{hologoLogoSetup}).
%
% \cs{HologoVariant} is used if the logo is set in a context
% that needs an uppercase first letter (beginning of a sentence, \dots).
%
% \begin{declcs}{hologoList}\\
%   \cs{hologoEntry} \M{logo} \M{variant} \M{since}
% \end{declcs}
% Macro \cs{hologoList} contains all logos that are provided
% by the package including variants. The list consists of calls
% of \cs{hologoEntry} with three arguments starting with the
% logo name \meta{logo} and its variant \meta{variant}. An empty
% variant means the current default. Argument \meta{since} specifies
% with version of the package \xpackage{hologo} is needed to get
% the logo. If the logo is fixed, then the date gets updated.
% Therefore the date \meta{since} is not exactly the date of
% the first introduction, but rather the date of the latest fix.
%
% Before \cs{hologoList} can be used, macro \cs{hologoEntry} needs
% a definition. The example file in section \ref{sec:example}
% shows applications of \cs{hologoList}.
%
% \subsection{Supported contexts}
%
% Macros \cs{hologo} and friends support special contexts:
% \begin{itemize}
% \item \hologo{LaTeX}'s protection mechanism.
% \item Bookmarks of package \xpackage{hyperref}.
% \item Package \xpackage{tex4ht}.
% \item The macros can be used inside \cs{csname} constructs,
%   if \cs{ifincsname} is available (\hologo{pdfTeX}, \hologo{XeTeX},
%   \hologo{LuaTeX}).
% \end{itemize}
%
% \subsection{Example}
% \label{sec:example}
%
% The following example prints the logos in different fonts.
%    \begin{macrocode}
%<*example>
%<<verbatim
\NeedsTeXFormat{LaTeX2e}
\documentclass[a4paper]{article}
\usepackage[
  hmargin=20mm,
  vmargin=20mm,
]{geometry}
\pagestyle{empty}
\usepackage{hologo}[2016/05/12]
\usepackage{longtable}
\usepackage{array}
\setlength{\extrarowheight}{2pt}
\usepackage[T1]{fontenc}
\usepackage{lmodern}
\usepackage{pdflscape}
\usepackage[
  pdfencoding=auto,
]{hyperref}
\hypersetup{
  pdfauthor={Heiko Oberdiek},
  pdftitle={Example for package `hologo'},
  pdfsubject={Logos with fonts lmr, lmss, qtm, qpl, qhv},
}
\usepackage{bookmark}

% Print the logo list on the console

\begingroup
  \typeout{}%
  \typeout{*** Begin of logo list ***}%
  \newcommand*{\hologoEntry}[3]{%
    \typeout{#1 \ifx\\#2\\\else(#2) \fi[#3]}%
  }%
  \hologoList
  \typeout{*** End of logo list ***}%
  \typeout{}%
\endgroup

\begin{document}
\begin{landscape}

  \section{Example file for package `hologo'}

  % Table for font names

  \begin{longtable}{>{\bfseries}ll}
    \textbf{font} & \textbf{Font name}\\
    \hline
    lmr & Latin Modern Roman\\
    lmss & Latin Modern Sans\\
    qtm & \TeX\ Gyre Termes\\
    qhv & \TeX\ Gyre Heros\\
    qpl & \TeX\ Gyre Pagella\\
  \end{longtable}

  % Logo list with logos in different fonts

  \begingroup
    \newcommand*{\SetVariant}[2]{%
      \ifx\\#2\\%
      \else
        \hologoLogoSetup{#1}{variant=#2}%
      \fi
    }%
    \newcommand*{\hologoEntry}[3]{%
      \SetVariant{#1}{#2}%
      \raisebox{1em}[0pt][0pt]{\hypertarget{#1@#2}{}}%
      \bookmark[%
        dest={#1@#2},%
      ]{%
        #1\ifx\\#2\\\else\space(#2)\fi: \Hologo{#1}, \hologo{#1} %
        [Unicode]%
      }%
      \hypersetup{unicode=false}%
      \bookmark[%
        dest={#1@#2},%
      ]{%
        #1\ifx\\#2\\\else\space(#2)\fi: \Hologo{#1}, \hologo{#1} %
        [PDFDocEncoding]%
      }%
      \texttt{#1}%
      &%
      \texttt{#2}%
      &%
      \Hologo{#1}%
      &%
      \SetVariant{#1}{#2}%
      \hologo{#1}%
      &%
      \SetVariant{#1}{#2}%
      \fontfamily{qtm}\selectfont
      \hologo{#1}%
      &%
      \SetVariant{#1}{#2}%
      \fontfamily{qpl}\selectfont
      \hologo{#1}%
      &%
      \SetVariant{#1}{#2}%
      \textsf{\hologo{#1}}%
      &%
      \SetVariant{#1}{#2}%
      \fontfamily{qhv}\selectfont
      \hologo{#1}%
      \tabularnewline
    }%
    \begin{longtable}{llllllll}%
      \textbf{\textit{logo}} & \textbf{\textit{variant}} &
      \texttt{\string\Hologo} &
      \textbf{lmr} & \textbf{qtm} & \textbf{qpl} &
      \textbf{lmss} & \textbf{qhv}
      \tabularnewline
      \hline
      \endhead
      \hologoList
    \end{longtable}%
  \endgroup

\end{landscape}
\end{document}
%verbatim
%</example>
%    \end{macrocode}
%
% \StopEventually{
% }
%
% \section{Implementation}
%    \begin{macrocode}
%<*package>
%    \end{macrocode}
%    Reload check, especially if the package is not used with \LaTeX.
%    \begin{macrocode}
\begingroup\catcode61\catcode48\catcode32=10\relax%
  \catcode13=5 % ^^M
  \endlinechar=13 %
  \catcode35=6 % #
  \catcode39=12 % '
  \catcode44=12 % ,
  \catcode45=12 % -
  \catcode46=12 % .
  \catcode58=12 % :
  \catcode64=11 % @
  \catcode123=1 % {
  \catcode125=2 % }
  \expandafter\let\expandafter\x\csname ver@hologo.sty\endcsname
  \ifx\x\relax % plain-TeX, first loading
  \else
    \def\empty{}%
    \ifx\x\empty % LaTeX, first loading,
      % variable is initialized, but \ProvidesPackage not yet seen
    \else
      \expandafter\ifx\csname PackageInfo\endcsname\relax
        \def\x#1#2{%
          \immediate\write-1{Package #1 Info: #2.}%
        }%
      \else
        \def\x#1#2{\PackageInfo{#1}{#2, stopped}}%
      \fi
      \x{hologo}{The package is already loaded}%
      \aftergroup\endinput
    \fi
  \fi
\endgroup%
%    \end{macrocode}
%    Package identification:
%    \begin{macrocode}
\begingroup\catcode61\catcode48\catcode32=10\relax%
  \catcode13=5 % ^^M
  \endlinechar=13 %
  \catcode35=6 % #
  \catcode39=12 % '
  \catcode40=12 % (
  \catcode41=12 % )
  \catcode44=12 % ,
  \catcode45=12 % -
  \catcode46=12 % .
  \catcode47=12 % /
  \catcode58=12 % :
  \catcode64=11 % @
  \catcode91=12 % [
  \catcode93=12 % ]
  \catcode123=1 % {
  \catcode125=2 % }
  \expandafter\ifx\csname ProvidesPackage\endcsname\relax
    \def\x#1#2#3[#4]{\endgroup
      \immediate\write-1{Package: #3 #4}%
      \xdef#1{#4}%
    }%
  \else
    \def\x#1#2[#3]{\endgroup
      #2[{#3}]%
      \ifx#1\@undefined
        \xdef#1{#3}%
      \fi
      \ifx#1\relax
        \xdef#1{#3}%
      \fi
    }%
  \fi
\expandafter\x\csname ver@hologo.sty\endcsname
\ProvidesPackage{hologo}%
  [2016/05/12 v1.11 A logo collection with bookmark support (HO)]%
%    \end{macrocode}
%
%    \begin{macrocode}
\begingroup\catcode61\catcode48\catcode32=10\relax%
  \catcode13=5 % ^^M
  \endlinechar=13 %
  \catcode123=1 % {
  \catcode125=2 % }
  \catcode64=11 % @
  \def\x{\endgroup
    \expandafter\edef\csname HOLOGO@AtEnd\endcsname{%
      \endlinechar=\the\endlinechar\relax
      \catcode13=\the\catcode13\relax
      \catcode32=\the\catcode32\relax
      \catcode35=\the\catcode35\relax
      \catcode61=\the\catcode61\relax
      \catcode64=\the\catcode64\relax
      \catcode123=\the\catcode123\relax
      \catcode125=\the\catcode125\relax
    }%
  }%
\x\catcode61\catcode48\catcode32=10\relax%
\catcode13=5 % ^^M
\endlinechar=13 %
\catcode35=6 % #
\catcode64=11 % @
\catcode123=1 % {
\catcode125=2 % }
\def\TMP@EnsureCode#1#2{%
  \edef\HOLOGO@AtEnd{%
    \HOLOGO@AtEnd
    \catcode#1=\the\catcode#1\relax
  }%
  \catcode#1=#2\relax
}
\TMP@EnsureCode{10}{12}% ^^J
\TMP@EnsureCode{33}{12}% !
\TMP@EnsureCode{34}{12}% "
\TMP@EnsureCode{36}{3}% $
\TMP@EnsureCode{38}{4}% &
\TMP@EnsureCode{39}{12}% '
\TMP@EnsureCode{40}{12}% (
\TMP@EnsureCode{41}{12}% )
\TMP@EnsureCode{42}{12}% *
\TMP@EnsureCode{43}{12}% +
\TMP@EnsureCode{44}{12}% ,
\TMP@EnsureCode{45}{12}% -
\TMP@EnsureCode{46}{12}% .
\TMP@EnsureCode{47}{12}% /
\TMP@EnsureCode{58}{12}% :
\TMP@EnsureCode{59}{12}% ;
\TMP@EnsureCode{60}{12}% <
\TMP@EnsureCode{62}{12}% >
\TMP@EnsureCode{63}{12}% ?
\TMP@EnsureCode{91}{12}% [
\TMP@EnsureCode{93}{12}% ]
\TMP@EnsureCode{94}{7}% ^ (superscript)
\TMP@EnsureCode{95}{8}% _ (subscript)
\TMP@EnsureCode{96}{12}% `
\TMP@EnsureCode{124}{12}% |
\edef\HOLOGO@AtEnd{%
  \HOLOGO@AtEnd
  \escapechar\the\escapechar\relax
  \noexpand\endinput
}
\escapechar=92 %
%    \end{macrocode}
%
% \subsection{Logo list}
%
%    \begin{macro}{\hologoList}
%    \begin{macrocode}
\def\hologoList{%
  \hologoEntry{(La)TeX}{}{2011/10/01}%
  \hologoEntry{AmSLaTeX}{}{2010/04/16}%
  \hologoEntry{AmSTeX}{}{2010/04/16}%
  \hologoEntry{biber}{}{2011/10/01}%
  \hologoEntry{BibTeX}{}{2011/10/01}%
  \hologoEntry{BibTeX}{sf}{2011/10/01}%
  \hologoEntry{BibTeX}{sc}{2011/10/01}%
  \hologoEntry{BibTeX8}{}{2011/11/22}%
  \hologoEntry{ConTeXt}{}{2011/03/25}%
  \hologoEntry{ConTeXt}{narrow}{2011/03/25}%
  \hologoEntry{ConTeXt}{simple}{2011/03/25}%
  \hologoEntry{emTeX}{}{2010/04/26}%
  \hologoEntry{eTeX}{}{2010/04/08}%
  \hologoEntry{ExTeX}{}{2011/10/01}%
  \hologoEntry{HanTheThanh}{}{2011/11/29}%
  \hologoEntry{iniTeX}{}{2011/10/01}%
  \hologoEntry{KOMAScript}{}{2011/10/01}%
  \hologoEntry{La}{}{2010/05/08}%
  \hologoEntry{LaTeX}{}{2010/04/08}%
  \hologoEntry{LaTeX2e}{}{2010/04/08}%
  \hologoEntry{LaTeX3}{}{2010/04/24}%
  \hologoEntry{LaTeXe}{}{2010/04/08}%
  \hologoEntry{LaTeXML}{}{2011/11/22}%
  \hologoEntry{LaTeXTeX}{}{2011/10/01}%
  \hologoEntry{LuaLaTeX}{}{2010/04/08}%
  \hologoEntry{LuaTeX}{}{2010/04/08}%
  \hologoEntry{LyX}{}{2011/10/01}%
  \hologoEntry{METAFONT}{}{2011/10/01}%
  \hologoEntry{MetaFun}{}{2011/10/01}%
  \hologoEntry{METAPOST}{}{2011/10/01}%
  \hologoEntry{MetaPost}{}{2011/10/01}%
  \hologoEntry{MiKTeX}{}{2011/10/01}%
  \hologoEntry{NTS}{}{2011/10/01}%
  \hologoEntry{OzMF}{}{2011/10/01}%
  \hologoEntry{OzMP}{}{2011/10/01}%
  \hologoEntry{OzTeX}{}{2011/10/01}%
  \hologoEntry{OzTtH}{}{2011/10/01}%
  \hologoEntry{PCTeX}{}{2011/10/01}%
  \hologoEntry{pdfTeX}{}{2011/10/01}%
  \hologoEntry{pdfLaTeX}{}{2011/10/01}%
  \hologoEntry{PiC}{}{2011/10/01}%
  \hologoEntry{PiCTeX}{}{2011/10/01}%
  \hologoEntry{plainTeX}{}{2010/04/08}%
  \hologoEntry{plainTeX}{space}{2010/04/16}%
  \hologoEntry{plainTeX}{hyphen}{2010/04/16}%
  \hologoEntry{plainTeX}{runtogether}{2010/04/16}%
  \hologoEntry{SageTeX}{}{2011/11/22}%
  \hologoEntry{SLiTeX}{}{2011/10/01}%
  \hologoEntry{SLiTeX}{lift}{2011/10/01}%
  \hologoEntry{SLiTeX}{narrow}{2011/10/01}%
  \hologoEntry{SLiTeX}{simple}{2011/10/01}%
  \hologoEntry{SliTeX}{}{2011/10/01}%
  \hologoEntry{SliTeX}{narrow}{2011/10/01}%
  \hologoEntry{SliTeX}{simple}{2011/10/01}%
  \hologoEntry{SliTeX}{lift}{2011/10/01}%
  \hologoEntry{teTeX}{}{2011/10/01}%
  \hologoEntry{TeX}{}{2010/04/08}%
  \hologoEntry{TeX4ht}{}{2011/11/22}%
  \hologoEntry{TTH}{}{2011/11/22}%
  \hologoEntry{virTeX}{}{2011/10/01}%
  \hologoEntry{VTeX}{}{2010/04/24}%
  \hologoEntry{Xe}{}{2010/04/08}%
  \hologoEntry{XeLaTeX}{}{2010/04/08}%
  \hologoEntry{XeTeX}{}{2010/04/08}%
}
%    \end{macrocode}
%    \end{macro}
%
% \subsection{Load resources}
%
%    \begin{macrocode}
\begingroup\expandafter\expandafter\expandafter\endgroup
\expandafter\ifx\csname RequirePackage\endcsname\relax
  \def\TMP@RequirePackage#1[#2]{%
    \begingroup\expandafter\expandafter\expandafter\endgroup
    \expandafter\ifx\csname ver@#1.sty\endcsname\relax
      \input #1.sty\relax
    \fi
  }%
  \TMP@RequirePackage{ltxcmds}[2011/02/04]%
  \TMP@RequirePackage{infwarerr}[2010/04/08]%
  \TMP@RequirePackage{kvsetkeys}[2010/03/01]%
  \TMP@RequirePackage{kvdefinekeys}[2010/03/01]%
  \TMP@RequirePackage{pdftexcmds}[2010/04/01]%
  \TMP@RequirePackage{ifpdf}[2010/01/28]%
  \TMP@RequirePackage{ifluatex}[2010/03/01]%
  \ltx@IfUndefined{newif}{%
    \expandafter\let\csname newif\endcsname\ltx@newif
  }{}%
  \TMP@RequirePackage{ifxetex}[2009/01/23]%
  \TMP@RequirePackage{ifvtex}[2010/03/01]%
\else
  \RequirePackage{ltxcmds}[2011/02/04]%
  \RequirePackage{infwarerr}[2010/04/08]%
  \RequirePackage{kvsetkeys}[2010/03/01]%
  \RequirePackage{kvdefinekeys}[2010/03/01]%
  \RequirePackage{pdftexcmds}[2010/04/01]%
  \RequirePackage{ifpdf}[2010/01/28]%
  \RequirePackage{ifluatex}[2010/03/01]%
  \RequirePackage{ifxetex}[2009/01/23]%
  \RequirePackage{ifvtex}[2010/03/01]%
\fi
%    \end{macrocode}
%
%    \begin{macro}{\HOLOGO@IfDefined}
%    \begin{macrocode}
\def\HOLOGO@IfExists#1{%
  \ifx\@undefined#1%
    \expandafter\ltx@secondoftwo
  \else
    \ifx\relax#1%
      \expandafter\ltx@secondoftwo
    \else
      \expandafter\expandafter\expandafter\ltx@firstoftwo
    \fi
  \fi
}
%    \end{macrocode}
%    \end{macro}
%
% \subsection{Setup macros}
%
%    \begin{macro}{\hologoSetup}
%    \begin{macrocode}
\def\hologoSetup{%
  \let\HOLOGO@name\relax
  \HOLOGO@Setup
}
%    \end{macrocode}
%    \end{macro}
%
%    \begin{macro}{\hologoLogoSetup}
%    \begin{macrocode}
\def\hologoLogoSetup#1{%
  \edef\HOLOGO@name{#1}%
  \ltx@IfUndefined{HoLogo@\HOLOGO@name}{%
    \@PackageError{hologo}{%
      Unknown logo `\HOLOGO@name'%
    }\@ehc
    \ltx@gobble
  }{%
    \HOLOGO@Setup
  }%
}
%    \end{macrocode}
%    \end{macro}
%
%    \begin{macro}{\HOLOGO@Setup}
%    \begin{macrocode}
\def\HOLOGO@Setup{%
  \kvsetkeys{HoLogo}%
}
%    \end{macrocode}
%    \end{macro}
%
% \subsection{Options}
%
%    \begin{macro}{\HOLOGO@DeclareBoolOption}
%    \begin{macrocode}
\def\HOLOGO@DeclareBoolOption#1{%
  \expandafter\chardef\csname HOLOGOOPT@#1\endcsname\ltx@zero
  \kv@define@key{HoLogo}{#1}[true]{%
    \def\HOLOGO@temp{##1}%
    \ifx\HOLOGO@temp\HOLOGO@true
      \ifx\HOLOGO@name\relax
        \expandafter\chardef\csname HOLOGOOPT@#1\endcsname=\ltx@one
      \else
        \expandafter\chardef\csname
        HoLogoOpt@#1@\HOLOGO@name\endcsname\ltx@one
      \fi
      \HOLOGO@SetBreakAll{#1}%
    \else
      \ifx\HOLOGO@temp\HOLOGO@false
        \ifx\HOLOGO@name\relax
          \expandafter\chardef\csname HOLOGOOPT@#1\endcsname=\ltx@zero
        \else
          \expandafter\chardef\csname
          HoLogoOpt@#1@\HOLOGO@name\endcsname=\ltx@zero
        \fi
        \HOLOGO@SetBreakAll{#1}%
      \else
        \@PackageError{hologo}{%
          Unknown value `##1' for boolean option `#1'.\MessageBreak
          Known values are `true' and `false'%
        }\@ehc
      \fi
    \fi
  }%
}
%    \end{macrocode}
%    \end{macro}
%
%    \begin{macro}{\HOLOGO@SetBreakAll}
%    \begin{macrocode}
\def\HOLOGO@SetBreakAll#1{%
  \def\HOLOGO@temp{#1}%
  \ifx\HOLOGO@temp\HOLOGO@break
    \ifx\HOLOGO@name\relax
      \chardef\HOLOGOOPT@hyphenbreak=\HOLOGOOPT@break
      \chardef\HOLOGOOPT@spacebreak=\HOLOGOOPT@break
      \chardef\HOLOGOOPT@discretionarybreak=\HOLOGOOPT@break
    \else
      \expandafter\chardef
         \csname HoLogoOpt@hyphenbreak@\HOLOGO@name\endcsname=%
         \csname HoLogoOpt@break@\HOLOGO@name\endcsname
      \expandafter\chardef
         \csname HoLogoOpt@spacebreak@\HOLOGO@name\endcsname=%
         \csname HoLogoOpt@break@\HOLOGO@name\endcsname
      \expandafter\chardef
         \csname HoLogoOpt@discretionarybreak@\HOLOGO@name
             \endcsname=%
         \csname HoLogoOpt@break@\HOLOGO@name\endcsname
    \fi
  \fi
}
%    \end{macrocode}
%    \end{macro}
%
%    \begin{macro}{\HOLOGO@true}
%    \begin{macrocode}
\def\HOLOGO@true{true}
%    \end{macrocode}
%    \end{macro}
%    \begin{macro}{\HOLOGO@false}
%    \begin{macrocode}
\def\HOLOGO@false{false}
%    \end{macrocode}
%    \end{macro}
%    \begin{macro}{\HOLOGO@break}
%    \begin{macrocode}
\def\HOLOGO@break{break}
%    \end{macrocode}
%    \end{macro}
%
%    \begin{macrocode}
\HOLOGO@DeclareBoolOption{break}
\HOLOGO@DeclareBoolOption{hyphenbreak}
\HOLOGO@DeclareBoolOption{spacebreak}
\HOLOGO@DeclareBoolOption{discretionarybreak}
%    \end{macrocode}
%
%    \begin{macrocode}
\kv@define@key{HoLogo}{variant}{%
  \ifx\HOLOGO@name\relax
    \@PackageError{hologo}{%
      Option `variant' is not available in \string\hologoSetup,%
      \MessageBreak
      Use \string\hologoLogoSetup\space instead%
    }\@ehc
  \else
    \edef\HOLOGO@temp{#1}%
    \ifx\HOLOGO@temp\ltx@empty
      \expandafter
      \let\csname HoLogoOpt@variant@\HOLOGO@name\endcsname\@undefined
    \else
      \ltx@IfUndefined{HoLogo@\HOLOGO@name @\HOLOGO@temp}{%
        \@PackageError{hologo}{%
          Unknown variant `\HOLOGO@temp' of logo `\HOLOGO@name'%
        }\@ehc
      }{%
        \expandafter
        \let\csname HoLogoOpt@variant@\HOLOGO@name\endcsname
            \HOLOGO@temp
      }%
    \fi
  \fi
}
%    \end{macrocode}
%
%    \begin{macro}{\HOLOGO@Variant}
%    \begin{macrocode}
\def\HOLOGO@Variant#1{%
  #1%
  \ltx@ifundefined{HoLogoOpt@variant@#1}{%
  }{%
    @\csname HoLogoOpt@variant@#1\endcsname
  }%
}
%    \end{macrocode}
%    \end{macro}
%
% \subsection{Break/no-break support}
%
%    \begin{macro}{\HOLOGO@space}
%    \begin{macrocode}
\def\HOLOGO@space{%
  \ltx@ifundefined{HoLogoOpt@spacebreak@\HOLOGO@name}{%
    \ltx@ifundefined{HoLogoOpt@break@\HOLOGO@name}{%
      \chardef\HOLOGO@temp=\HOLOGOOPT@spacebreak
    }{%
      \chardef\HOLOGO@temp=%
        \csname HoLogoOpt@break@\HOLOGO@name\endcsname
    }%
  }{%
    \chardef\HOLOGO@temp=%
      \csname HoLogoOpt@spacebreak@\HOLOGO@name\endcsname
  }%
  \ifcase\HOLOGO@temp
    \penalty10000 %
  \fi
  \ltx@space
}
%    \end{macrocode}
%    \end{macro}
%
%    \begin{macro}{\HOLOGO@hyphen}
%    \begin{macrocode}
\def\HOLOGO@hyphen{%
  \ltx@ifundefined{HoLogoOpt@hyphenbreak@\HOLOGO@name}{%
    \ltx@ifundefined{HoLogoOpt@break@\HOLOGO@name}{%
      \chardef\HOLOGO@temp=\HOLOGOOPT@hyphenbreak
    }{%
      \chardef\HOLOGO@temp=%
        \csname HoLogoOpt@break@\HOLOGO@name\endcsname
    }%
  }{%
    \chardef\HOLOGO@temp=%
      \csname HoLogoOpt@hyphenbreak@\HOLOGO@name\endcsname
  }%
  \ifcase\HOLOGO@temp
    \ltx@mbox{-}%
  \else
    -%
  \fi
}
%    \end{macrocode}
%    \end{macro}
%
%    \begin{macro}{\HOLOGO@discretionary}
%    \begin{macrocode}
\def\HOLOGO@discretionary{%
  \ltx@ifundefined{HoLogoOpt@discretionarybreak@\HOLOGO@name}{%
    \ltx@ifundefined{HoLogoOpt@break@\HOLOGO@name}{%
      \chardef\HOLOGO@temp=\HOLOGOOPT@discretionarybreak
    }{%
      \chardef\HOLOGO@temp=%
        \csname HoLogoOpt@break@\HOLOGO@name\endcsname
    }%
  }{%
    \chardef\HOLOGO@temp=%
      \csname HoLogoOpt@discretionarybreak@\HOLOGO@name\endcsname
  }%
  \ifcase\HOLOGO@temp
  \else
    \-%
  \fi
}
%    \end{macrocode}
%    \end{macro}
%
%    \begin{macro}{\HOLOGO@mbox}
%    \begin{macrocode}
\def\HOLOGO@mbox#1{%
  \ltx@ifundefined{HoLogoOpt@break@\HOLOGO@name}{%
    \chardef\HOLOGO@temp=\HOLOGOOPT@hyphenbreak
  }{%
    \chardef\HOLOGO@temp=%
      \csname HoLogoOpt@break@\HOLOGO@name\endcsname
  }%
  \ifcase\HOLOGO@temp
    \ltx@mbox{#1}%
  \else
    #1%
  \fi
}
%    \end{macrocode}
%    \end{macro}
%
% \subsection{Font support}
%
%    \begin{macro}{\HoLogoFont@font}
%    \begin{tabular}{@{}ll@{}}
%    |#1|:& logo name\\
%    |#2|:& font short name\\
%    |#3|:& text
%    \end{tabular}
%    \begin{macrocode}
\def\HoLogoFont@font#1#2#3{%
  \begingroup
    \ltx@IfUndefined{HoLogoFont@logo@#1.#2}{%
      \ltx@IfUndefined{HoLogoFont@font@#2}{%
        \@PackageWarning{hologo}{%
          Missing font `#2' for logo `#1'%
        }%
        #3%
      }{%
        \csname HoLogoFont@font@#2\endcsname{#3}%
      }%
    }{%
      \csname HoLogoFont@logo@#1.#2\endcsname{#3}%
    }%
  \endgroup
}
%    \end{macrocode}
%    \end{macro}
%
%    \begin{macro}{\HoLogoFont@Def}
%    \begin{macrocode}
\def\HoLogoFont@Def#1{%
  \expandafter\def\csname HoLogoFont@font@#1\endcsname
}
%    \end{macrocode}
%    \end{macro}
%    \begin{macro}{\HoLogoFont@LogoDef}
%    \begin{macrocode}
\def\HoLogoFont@LogoDef#1#2{%
  \expandafter\def\csname HoLogoFont@logo@#1.#2\endcsname
}
%    \end{macrocode}
%    \end{macro}
%
% \subsubsection{Font defaults}
%
%    \begin{macro}{\HoLogoFont@font@general}
%    \begin{macrocode}
\HoLogoFont@Def{general}{}%
%    \end{macrocode}
%    \end{macro}
%
%    \begin{macro}{\HoLogoFont@font@rm}
%    \begin{macrocode}
\ltx@IfUndefined{rmfamily}{%
  \ltx@IfUndefined{rm}{%
  }{%
    \HoLogoFont@Def{rm}{\rm}%
  }%
}{%
  \HoLogoFont@Def{rm}{\rmfamily}%
}
%    \end{macrocode}
%    \end{macro}
%
%    \begin{macro}{\HoLogoFont@font@sf}
%    \begin{macrocode}
\ltx@IfUndefined{sffamily}{%
  \ltx@IfUndefined{sf}{%
  }{%
    \HoLogoFont@Def{sf}{\sf}%
  }%
}{%
  \HoLogoFont@Def{sf}{\sffamily}%
}
%    \end{macrocode}
%    \end{macro}
%
%    \begin{macro}{\HoLogoFont@font@bibsf}
%    In case of \hologo{plainTeX} the original small caps
%    variant is used as default. In \hologo{LaTeX}
%    the definition of package \xpackage{dtklogos} \cite{dtklogos}
%    is used.
%\begin{quote}
%\begin{verbatim}
%\DeclareRobustCommand{\BibTeX}{%
%  B%
%  \kern-.05em%
%  \hbox{%
%    $\m@th$% %% force math size calculations
%    \csname S@\f@size\endcsname
%    \fontsize\sf@size\z@
%    \math@fontsfalse
%    \selectfont
%    I%
%    \kern-.025em%
%    B
%  }%
%  \kern-.08em%
%  \-%
%  \TeX
%}
%\end{verbatim}
%\end{quote}
%    \begin{macrocode}
\ltx@IfUndefined{selectfont}{%
  \ltx@IfUndefined{tensc}{%
    \font\tensc=cmcsc10\relax
  }{}%
  \HoLogoFont@Def{bibsf}{\tensc}%
}{%
  \HoLogoFont@Def{bibsf}{%
    $\mathsurround=0pt$%
    \csname S@\f@size\endcsname
    \fontsize\sf@size{0pt}%
    \math@fontsfalse
    \selectfont
  }%
}
%    \end{macrocode}
%    \end{macro}
%
%    \begin{macro}{\HoLogoFont@font@sc}
%    \begin{macrocode}
\ltx@IfUndefined{scshape}{%
  \ltx@IfUndefined{tensc}{%
    \font\tensc=cmcsc10\relax
  }{}%
  \HoLogoFont@Def{sc}{\tensc}%
}{%
  \HoLogoFont@Def{sc}{\scshape}%
}
%    \end{macrocode}
%    \end{macro}
%
%    \begin{macro}{\HoLogoFont@font@sy}
%    \begin{macrocode}
\ltx@IfUndefined{usefont}{%
  \ltx@IfUndefined{tensy}{%
  }{%
    \HoLogoFont@Def{sy}{\tensy}%
  }%
}{%
  \HoLogoFont@Def{sy}{%
    \usefont{OMS}{cmsy}{m}{n}%
  }%
}
%    \end{macrocode}
%    \end{macro}
%
%    \begin{macro}{\HoLogoFont@font@logo}
%    \begin{macrocode}
\begingroup
  \def\x{LaTeX2e}%
\expandafter\endgroup
\ifx\fmtname\x
  \ltx@IfUndefined{logofamily}{%
    \DeclareRobustCommand\logofamily{%
      \not@math@alphabet\logofamily\relax
      \fontencoding{U}%
      \fontfamily{logo}%
      \selectfont
    }%
  }{}%
  \ltx@IfUndefined{logofamily}{%
  }{%
    \HoLogoFont@Def{logo}{\logofamily}%
  }%
\else
  \ltx@IfUndefined{tenlogo}{%
    \font\tenlogo=logo10\relax
  }{}%
  \HoLogoFont@Def{logo}{\tenlogo}%
\fi
%    \end{macrocode}
%    \end{macro}
%
% \subsubsection{Font setup}
%
%    \begin{macro}{\hologoFontSetup}
%    \begin{macrocode}
\def\hologoFontSetup{%
  \let\HOLOGO@name\relax
  \HOLOGO@FontSetup
}
%    \end{macrocode}
%    \end{macro}
%
%    \begin{macro}{\hologoLogoFontSetup}
%    \begin{macrocode}
\def\hologoLogoFontSetup#1{%
  \edef\HOLOGO@name{#1}%
  \ltx@IfUndefined{HoLogo@\HOLOGO@name}{%
    \@PackageError{hologo}{%
      Unknown logo `\HOLOGO@name'%
    }\@ehc
    \ltx@gobble
  }{%
    \HOLOGO@FontSetup
  }%
}
%    \end{macrocode}
%    \end{macro}
%
%    \begin{macro}{\HOLOGO@FontSetup}
%    \begin{macrocode}
\def\HOLOGO@FontSetup{%
  \kvsetkeys{HoLogoFont}%
}
%    \end{macrocode}
%    \end{macro}
%
%    \begin{macrocode}
\def\HOLOGO@temp#1{%
  \kv@define@key{HoLogoFont}{#1}{%
    \ifx\HOLOGO@name\relax
      \HoLogoFont@Def{#1}{##1}%
    \else
      \HoLogoFont@LogoDef\HOLOGO@name{#1}{##1}%
    \fi
  }%
}
\HOLOGO@temp{general}
\HOLOGO@temp{sf}
%    \end{macrocode}
%
% \subsection{Generic logo commands}
%
%    \begin{macrocode}
\HOLOGO@IfExists\hologo{%
  \@PackageError{hologo}{%
    \string\hologo\ltx@space is already defined.\MessageBreak
    Package loading is aborted%
  }\@ehc
  \HOLOGO@AtEnd
}%
\HOLOGO@IfExists\hologoRobust{%
  \@PackageError{hologo}{%
    \string\hologoRobust\ltx@space is already defined.\MessageBreak
    Package loading is aborted%
  }\@ehc
  \HOLOGO@AtEnd
}%
%    \end{macrocode}
%
% \subsubsection{\cs{hologo} and friends}
%
%    \begin{macrocode}
\ifluatex
  \expandafter\ltx@firstofone
\else
  \expandafter\ltx@gobble
\fi
{%
  \ltx@IfUndefined{ifincsname}{%
    \ifnum\luatexversion<36 %
      \expandafter\ltx@gobble
    \else
      \expandafter\ltx@firstofone
    \fi
    {%
      \begingroup
        \ifcase0%
            \directlua{%
              if tex.enableprimitives then %
                tex.enableprimitives('HOLOGO@', {'ifincsname'})%
              else %
                tex.print('1')%
              end%
            }%
            \ifx\HOLOGO@ifincsname\@undefined 1\fi%
            \relax
          \expandafter\ltx@firstofone
        \else
          \endgroup
          \expandafter\ltx@gobble
        \fi
        {%
          \global\let\ifincsname\HOLOGO@ifincsname
        }%
      \HOLOGO@temp
    }%
  }{}%
}
%    \end{macrocode}
%    \begin{macrocode}
\ltx@IfUndefined{ifincsname}{%
  \catcode`$=14 %
}{%
  \catcode`$=9 %
}
%    \end{macrocode}
%
%    \begin{macro}{\hologo}
%    \begin{macrocode}
\def\hologo#1{%
$ \ifincsname
$   \ltx@ifundefined{HoLogoCs@\HOLOGO@Variant{#1}}{%
$     #1%
$   }{%
$     \csname HoLogoCs@\HOLOGO@Variant{#1}\endcsname\ltx@firstoftwo
$   }%
$ \else
    \HOLOGO@IfExists\texorpdfstring\texorpdfstring\ltx@firstoftwo
    {%
      \hologoRobust{#1}%
    }{%
      \ltx@ifundefined{HoLogoBkm@\HOLOGO@Variant{#1}}{%
        \ltx@ifundefined{HoLogo@#1}{?#1?}{#1}%
      }{%
        \csname HoLogoBkm@\HOLOGO@Variant{#1}\endcsname
        \ltx@firstoftwo
      }%
    }%
$ \fi
}
%    \end{macrocode}
%    \end{macro}
%    \begin{macro}{\Hologo}
%    \begin{macrocode}
\def\Hologo#1{%
$ \ifincsname
$   \ltx@ifundefined{HoLogoCs@\HOLOGO@Variant{#1}}{%
$     #1%
$   }{%
$     \csname HoLogoCs@\HOLOGO@Variant{#1}\endcsname\ltx@secondoftwo
$   }%
$ \else
    \HOLOGO@IfExists\texorpdfstring\texorpdfstring\ltx@firstoftwo
    {%
      \HologoRobust{#1}%
    }{%
      \ltx@ifundefined{HoLogoBkm@\HOLOGO@Variant{#1}}{%
        \ltx@ifundefined{HoLogo@#1}{?#1?}{#1}%
      }{%
        \csname HoLogoBkm@\HOLOGO@Variant{#1}\endcsname
        \ltx@secondoftwo
      }%
    }%
$ \fi
}
%    \end{macrocode}
%    \end{macro}
%
%    \begin{macro}{\hologoVariant}
%    \begin{macrocode}
\def\hologoVariant#1#2{%
  \ifx\relax#2\relax
    \hologo{#1}%
  \else
$   \ifincsname
$     \ltx@ifundefined{HoLogoCs@#1@#2}{%
$       #1%
$     }{%
$       \csname HoLogoCs@#1@#2\endcsname\ltx@firstoftwo
$     }%
$   \else
      \HOLOGO@IfExists\texorpdfstring\texorpdfstring\ltx@firstoftwo
      {%
        \hologoVariantRobust{#1}{#2}%
      }{%
        \ltx@ifundefined{HoLogoBkm@#1@#2}{%
          \ltx@ifundefined{HoLogo@#1}{?#1?}{#1}%
        }{%
          \csname HoLogoBkm@#1@#2\endcsname
          \ltx@firstoftwo
        }%
      }%
$   \fi
  \fi
}
%    \end{macrocode}
%    \end{macro}
%    \begin{macro}{\HologoVariant}
%    \begin{macrocode}
\def\HologoVariant#1#2{%
  \ifx\relax#2\relax
    \Hologo{#1}%
  \else
$   \ifincsname
$     \ltx@ifundefined{HoLogoCs@#1@#2}{%
$       #1%
$     }{%
$       \csname HoLogoCs@#1@#2\endcsname\ltx@secondoftwo
$     }%
$   \else
      \HOLOGO@IfExists\texorpdfstring\texorpdfstring\ltx@firstoftwo
      {%
        \HologoVariantRobust{#1}{#2}%
      }{%
        \ltx@ifundefined{HoLogoBkm@#1@#2}{%
          \ltx@ifundefined{HoLogo@#1}{?#1?}{#1}%
        }{%
          \csname HoLogoBkm@#1@#2\endcsname
          \ltx@secondoftwo
        }%
      }%
$   \fi
  \fi
}
%    \end{macrocode}
%    \end{macro}
%
%    \begin{macrocode}
\catcode`\$=3 %
%    \end{macrocode}
%
% \subsubsection{\cs{hologoRobust} and friends}
%
%    \begin{macro}{\hologoRobust}
%    \begin{macrocode}
\ltx@IfUndefined{protected}{%
  \ltx@IfUndefined{DeclareRobustCommand}{%
    \def\hologoRobust#1%
  }{%
    \DeclareRobustCommand*\hologoRobust[1]%
  }%
}{%
  \protected\def\hologoRobust#1%
}%
{%
  \edef\HOLOGO@name{#1}%
  \ltx@IfUndefined{HoLogo@\HOLOGO@Variant\HOLOGO@name}{%
    \@PackageError{hologo}{%
      Unknown logo `\HOLOGO@name'%
    }\@ehc
    ?\HOLOGO@name?%
  }{%
    \ltx@IfUndefined{ver@tex4ht.sty}{%
      \HoLogoFont@font\HOLOGO@name{general}{%
        \csname HoLogo@\HOLOGO@Variant\HOLOGO@name\endcsname
        \ltx@firstoftwo
      }%
    }{%
      \ltx@IfUndefined{HoLogoHtml@\HOLOGO@Variant\HOLOGO@name}{%
        \HOLOGO@name
      }{%
        \csname HoLogoHtml@\HOLOGO@Variant\HOLOGO@name\endcsname
        \ltx@firstoftwo
      }%
    }%
  }%
}
%    \end{macrocode}
%    \end{macro}
%    \begin{macro}{\HologoRobust}
%    \begin{macrocode}
\ltx@IfUndefined{protected}{%
  \ltx@IfUndefined{DeclareRobustCommand}{%
    \def\HologoRobust#1%
  }{%
    \DeclareRobustCommand*\HologoRobust[1]%
  }%
}{%
  \protected\def\HologoRobust#1%
}%
{%
  \edef\HOLOGO@name{#1}%
  \ltx@IfUndefined{HoLogo@\HOLOGO@Variant\HOLOGO@name}{%
    \@PackageError{hologo}{%
      Unknown logo `\HOLOGO@name'%
    }\@ehc
    ?\HOLOGO@name?%
  }{%
    \ltx@IfUndefined{ver@tex4ht.sty}{%
      \HoLogoFont@font\HOLOGO@name{general}{%
        \csname HoLogo@\HOLOGO@Variant\HOLOGO@name\endcsname
        \ltx@secondoftwo
      }%
    }{%
      \ltx@IfUndefined{HoLogoHtml@\HOLOGO@Variant\HOLOGO@name}{%
        \expandafter\HOLOGO@Uppercase\HOLOGO@name
      }{%
        \csname HoLogoHtml@\HOLOGO@Variant\HOLOGO@name\endcsname
        \ltx@secondoftwo
      }%
    }%
  }%
}
%    \end{macrocode}
%    \end{macro}
%    \begin{macro}{\hologoVariantRobust}
%    \begin{macrocode}
\ltx@IfUndefined{protected}{%
  \ltx@IfUndefined{DeclareRobustCommand}{%
    \def\hologoVariantRobust#1#2%
  }{%
    \DeclareRobustCommand*\hologoVariantRobust[2]%
  }%
}{%
  \protected\def\hologoVariantRobust#1#2%
}%
{%
  \begingroup
    \hologoLogoSetup{#1}{variant={#2}}%
    \hologoRobust{#1}%
  \endgroup
}
%    \end{macrocode}
%    \end{macro}
%    \begin{macro}{\HologoVariantRobust}
%    \begin{macrocode}
\ltx@IfUndefined{protected}{%
  \ltx@IfUndefined{DeclareRobustCommand}{%
    \def\HologoVariantRobust#1#2%
  }{%
    \DeclareRobustCommand*\HologoVariantRobust[2]%
  }%
}{%
  \protected\def\HologoVariantRobust#1#2%
}%
{%
  \begingroup
    \hologoLogoSetup{#1}{variant={#2}}%
    \HologoRobust{#1}%
  \endgroup
}
%    \end{macrocode}
%    \end{macro}
%
%    \begin{macro}{\hologorobust}
%    Macro \cs{hologorobust} is only defined for compatibility.
%    Its use is deprecated.
%    \begin{macrocode}
\def\hologorobust{\hologoRobust}
%    \end{macrocode}
%    \end{macro}
%
% \subsection{Helpers}
%
%    \begin{macro}{\HOLOGO@Uppercase}
%    Macro \cs{HOLOGO@Uppercase} is restricted to \cs{uppercase},
%    because \hologo{plainTeX} or \hologo{iniTeX} do not provide
%    \cs{MakeUppercase}.
%    \begin{macrocode}
\def\HOLOGO@Uppercase#1{\uppercase{#1}}
%    \end{macrocode}
%    \end{macro}
%
%    \begin{macro}{\HOLOGO@PdfdocUnicode}
%    \begin{macrocode}
\def\HOLOGO@PdfdocUnicode{%
  \ifx\ifHy@unicode\iftrue
    \expandafter\ltx@secondoftwo
  \else
    \expandafter\ltx@firstoftwo
  \fi
}
%    \end{macrocode}
%    \end{macro}
%
%    \begin{macro}{\HOLOGO@Math}
%    \begin{macrocode}
\def\HOLOGO@MathSetup{%
  \mathsurround0pt\relax
  \HOLOGO@IfExists\f@series{%
    \if b\expandafter\ltx@car\f@series x\@nil
      \csname boldmath\endcsname
   \fi
  }{}%
}
%    \end{macrocode}
%    \end{macro}
%
%    \begin{macro}{\HOLOGO@TempDimen}
%    \begin{macrocode}
\dimendef\HOLOGO@TempDimen=\ltx@zero
%    \end{macrocode}
%    \end{macro}
%    \begin{macro}{\HOLOGO@NegativeKerning}
%    \begin{macrocode}
\def\HOLOGO@NegativeKerning#1{%
  \begingroup
    \HOLOGO@TempDimen=0pt\relax
    \comma@parse@normalized{#1}{%
      \ifdim\HOLOGO@TempDimen=0pt %
        \expandafter\HOLOGO@@NegativeKerning\comma@entry
      \fi
      \ltx@gobble
    }%
    \ifdim\HOLOGO@TempDimen<0pt %
      \kern\HOLOGO@TempDimen
    \fi
  \endgroup
}
%    \end{macrocode}
%    \end{macro}
%    \begin{macro}{\HOLOGO@@NegativeKerning}
%    \begin{macrocode}
\def\HOLOGO@@NegativeKerning#1#2{%
  \setbox\ltx@zero\hbox{#1#2}%
  \HOLOGO@TempDimen=\wd\ltx@zero
  \setbox\ltx@zero\hbox{#1\kern0pt#2}%
  \advance\HOLOGO@TempDimen by -\wd\ltx@zero
}
%    \end{macrocode}
%    \end{macro}
%
%    \begin{macro}{\HOLOGO@SpaceFactor}
%    \begin{macrocode}
\def\HOLOGO@SpaceFactor{%
  \spacefactor1000 %
}
%    \end{macrocode}
%    \end{macro}
%
%    \begin{macro}{\HOLOGO@Span}
%    \begin{macrocode}
\def\HOLOGO@Span#1#2{%
  \HCode{<span class="HoLogo-#1">}%
  #2%
  \HCode{</span>}%
}
%    \end{macrocode}
%    \end{macro}
%
% \subsubsection{Text subscript}
%
%    \begin{macro}{\HOLOGO@SubScript}%
%    \begin{macrocode}
\def\HOLOGO@SubScript#1{%
  \ltx@IfUndefined{textsubscript}{%
    \ltx@IfUndefined{text}{%
      \ltx@mbox{%
        \mathsurround=0pt\relax
        $%
          _{%
            \ltx@IfUndefined{sf@size}{%
              \mathrm{#1}%
            }{%
              \mbox{%
                \fontsize\sf@size{0pt}\selectfont
                #1%
              }%
            }%
          }%
        $%
      }%
    }{%
      \ltx@mbox{%
        \mathsurround=0pt\relax
        $_{\text{#1}}$%
      }%
    }%
  }{%
    \textsubscript{#1}%
  }%
}
%    \end{macrocode}
%    \end{macro}
%
% \subsection{\hologo{TeX} and friends}
%
% \subsubsection{\hologo{TeX}}
%
%    \begin{macro}{\HoLogo@TeX}
%    Source: \hologo{LaTeX} kernel.
%    \begin{macrocode}
\def\HoLogo@TeX#1{%
  T\kern-.1667em\lower.5ex\hbox{E}\kern-.125emX\HOLOGO@SpaceFactor
}
%    \end{macrocode}
%    \end{macro}
%    \begin{macro}{\HoLogoHtml@TeX}
%    \begin{macrocode}
\def\HoLogoHtml@TeX#1{%
  \HoLogoCss@TeX
  \HOLOGO@Span{TeX}{%
    T%
    \HOLOGO@Span{e}{%
      E%
    }%
    X%
  }%
}
%    \end{macrocode}
%    \end{macro}
%    \begin{macro}{\HoLogoCss@TeX}
%    \begin{macrocode}
\def\HoLogoCss@TeX{%
  \Css{%
    span.HoLogo-TeX span.HoLogo-e{%
      position:relative;%
      top:.5ex;%
      margin-left:-.1667em;%
      margin-right:-.125em;%
    }%
  }%
  \Css{%
    a span.HoLogo-TeX span.HoLogo-e{%
      text-decoration:none;%
    }%
  }%
  \global\let\HoLogoCss@TeX\relax
}
%    \end{macrocode}
%    \end{macro}
%
% \subsubsection{\hologo{plainTeX}}
%
%    \begin{macro}{\HoLogo@plainTeX@space}
%    Source: ``The \hologo{TeX}book''
%    \begin{macrocode}
\def\HoLogo@plainTeX@space#1{%
  \HOLOGO@mbox{#1{p}{P}lain}\HOLOGO@space\hologo{TeX}%
}
%    \end{macrocode}
%    \end{macro}
%    \begin{macro}{\HoLogoCs@plainTeX@space}
%    \begin{macrocode}
\def\HoLogoCs@plainTeX@space#1{#1{p}{P}lain TeX}%
%    \end{macrocode}
%    \end{macro}
%    \begin{macro}{\HoLogoBkm@plainTeX@space}
%    \begin{macrocode}
\def\HoLogoBkm@plainTeX@space#1{%
  #1{p}{P}lain \hologo{TeX}%
}
%    \end{macrocode}
%    \end{macro}
%    \begin{macro}{\HoLogoHtml@plainTeX@space}
%    \begin{macrocode}
\def\HoLogoHtml@plainTeX@space#1{%
  #1{p}{P}lain \hologo{TeX}%
}
%    \end{macrocode}
%    \end{macro}
%
%    \begin{macro}{\HoLogo@plainTeX@hyphen}
%    \begin{macrocode}
\def\HoLogo@plainTeX@hyphen#1{%
  \HOLOGO@mbox{#1{p}{P}lain}\HOLOGO@hyphen\hologo{TeX}%
}
%    \end{macrocode}
%    \end{macro}
%    \begin{macro}{\HoLogoCs@plainTeX@hyphen}
%    \begin{macrocode}
\def\HoLogoCs@plainTeX@hyphen#1{#1{p}{P}lain-TeX}
%    \end{macrocode}
%    \end{macro}
%    \begin{macro}{\HoLogoBkm@plainTeX@hyphen}
%    \begin{macrocode}
\def\HoLogoBkm@plainTeX@hyphen#1{%
  #1{p}{P}lain-\hologo{TeX}%
}
%    \end{macrocode}
%    \end{macro}
%    \begin{macro}{\HoLogoHtml@plainTeX@hyphen}
%    \begin{macrocode}
\def\HoLogoHtml@plainTeX@hyphen#1{%
  #1{p}{P}lain-\hologo{TeX}%
}
%    \end{macrocode}
%    \end{macro}
%
%    \begin{macro}{\HoLogo@plainTeX@runtogether}
%    \begin{macrocode}
\def\HoLogo@plainTeX@runtogether#1{%
  \HOLOGO@mbox{#1{p}{P}lain\hologo{TeX}}%
}
%    \end{macrocode}
%    \end{macro}
%    \begin{macro}{\HoLogoCs@plainTeX@runtogether}
%    \begin{macrocode}
\def\HoLogoCs@plainTeX@runtogether#1{#1{p}{P}lainTeX}
%    \end{macrocode}
%    \end{macro}
%    \begin{macro}{\HoLogoBkm@plainTeX@runtogether}
%    \begin{macrocode}
\def\HoLogoBkm@plainTeX@runtogether#1{%
  #1{p}{P}lain\hologo{TeX}%
}
%    \end{macrocode}
%    \end{macro}
%    \begin{macro}{\HoLogoHtml@plainTeX@runtogether}
%    \begin{macrocode}
\def\HoLogoHtml@plainTeX@runtogether#1{%
  #1{p}{P}lain\hologo{TeX}%
}
%    \end{macrocode}
%    \end{macro}
%
%    \begin{macro}{\HoLogo@plainTeX}
%    \begin{macrocode}
\def\HoLogo@plainTeX{\HoLogo@plainTeX@space}
%    \end{macrocode}
%    \end{macro}
%    \begin{macro}{\HoLogoCs@plainTeX}
%    \begin{macrocode}
\def\HoLogoCs@plainTeX{\HoLogoCs@plainTeX@space}
%    \end{macrocode}
%    \end{macro}
%    \begin{macro}{\HoLogoBkm@plainTeX}
%    \begin{macrocode}
\def\HoLogoBkm@plainTeX{\HoLogoBkm@plainTeX@space}
%    \end{macrocode}
%    \end{macro}
%    \begin{macro}{\HoLogoHtml@plainTeX}
%    \begin{macrocode}
\def\HoLogoHtml@plainTeX{\HoLogoHtml@plainTeX@space}
%    \end{macrocode}
%    \end{macro}
%
% \subsubsection{\hologo{LaTeX}}
%
%    Source: \hologo{LaTeX} kernel.
%\begin{quote}
%\begin{verbatim}
%\DeclareRobustCommand{\LaTeX}{%
%  L%
%  \kern-.36em%
%  {%
%    \sbox\z@ T%
%    \vbox to\ht\z@{%
%      \hbox{%
%        \check@mathfonts
%        \fontsize\sf@size\z@
%        \math@fontsfalse
%        \selectfont
%        A%
%      }%
%      \vss
%    }%
%  }%
%  \kern-.15em%
%  \TeX
%}
%\end{verbatim}
%\end{quote}
%
%    \begin{macro}{\HoLogo@La}
%    \begin{macrocode}
\def\HoLogo@La#1{%
  L%
  \kern-.36em%
  \begingroup
    \setbox\ltx@zero\hbox{T}%
    \vbox to\ht\ltx@zero{%
      \hbox{%
        \ltx@ifundefined{check@mathfonts}{%
          \csname sevenrm\endcsname
        }{%
          \check@mathfonts
          \fontsize\sf@size{0pt}%
          \math@fontsfalse\selectfont
        }%
        A%
      }%
      \vss
    }%
  \endgroup
}
%    \end{macrocode}
%    \end{macro}
%
%    \begin{macro}{\HoLogo@LaTeX}
%    Source: \hologo{LaTeX} kernel.
%    \begin{macrocode}
\def\HoLogo@LaTeX#1{%
  \hologo{La}%
  \kern-.15em%
  \hologo{TeX}%
}
%    \end{macrocode}
%    \end{macro}
%    \begin{macro}{\HoLogoHtml@LaTeX}
%    \begin{macrocode}
\def\HoLogoHtml@LaTeX#1{%
  \HoLogoCss@LaTeX
  \HOLOGO@Span{LaTeX}{%
    L%
    \HOLOGO@Span{a}{%
      A%
    }%
    \hologo{TeX}%
  }%
}
%    \end{macrocode}
%    \end{macro}
%    \begin{macro}{\HoLogoCss@LaTeX}
%    \begin{macrocode}
\def\HoLogoCss@LaTeX{%
  \Css{%
    span.HoLogo-LaTeX span.HoLogo-a{%
      position:relative;%
      top:-.5ex;%
      margin-left:-.36em;%
      margin-right:-.15em;%
      font-size:85\%;%
    }%
  }%
  \global\let\HoLogoCss@LaTeX\relax
}
%    \end{macrocode}
%    \end{macro}
%
% \subsubsection{\hologo{(La)TeX}}
%
%    \begin{macro}{\HoLogo@LaTeXTeX}
%    The kerning around the parentheses is taken
%    from package \xpackage{dtklogos} \cite{dtklogos}.
%\begin{quote}
%\begin{verbatim}
%\DeclareRobustCommand{\LaTeXTeX}{%
%  (%
%  \kern-.15em%
%  L%
%  \kern-.36em%
%  {%
%    \sbox\z@ T%
%    \vbox to\ht0{%
%      \hbox{%
%        $\m@th$%
%        \csname S@\f@size\endcsname
%        \fontsize\sf@size\z@
%        \math@fontsfalse
%        \selectfont
%        A%
%      }%
%      \vss
%    }%
%  }%
%  \kern-.2em%
%  )%
%  \kern-.15em%
%  \TeX
%}
%\end{verbatim}
%\end{quote}
%    \begin{macrocode}
\def\HoLogo@LaTeXTeX#1{%
  (%
  \kern-.15em%
  \hologo{La}%
  \kern-.2em%
  )%
  \kern-.15em%
  \hologo{TeX}%
}
%    \end{macrocode}
%    \end{macro}
%    \begin{macro}{\HoLogoBkm@LaTeXTeX}
%    \begin{macrocode}
\def\HoLogoBkm@LaTeXTeX#1{(La)TeX}
%    \end{macrocode}
%    \end{macro}
%
%    \begin{macro}{\HoLogo@(La)TeX}
%    \begin{macrocode}
\expandafter
\let\csname HoLogo@(La)TeX\endcsname\HoLogo@LaTeXTeX
%    \end{macrocode}
%    \end{macro}
%    \begin{macro}{\HoLogoBkm@(La)TeX}
%    \begin{macrocode}
\expandafter
\let\csname HoLogoBkm@(La)TeX\endcsname\HoLogoBkm@LaTeXTeX
%    \end{macrocode}
%    \end{macro}
%    \begin{macro}{\HoLogoHtml@LaTeXTeX}
%    \begin{macrocode}
\def\HoLogoHtml@LaTeXTeX#1{%
  \HoLogoCss@LaTeXTeX
  \HOLOGO@Span{LaTeXTeX}{%
    (%
    \HOLOGO@Span{L}{L}%
    \HOLOGO@Span{a}{A}%
    \HOLOGO@Span{ParenRight}{)}%
    \hologo{TeX}%
  }%
}
%    \end{macrocode}
%    \end{macro}
%    \begin{macro}{\HoLogoHtml@(La)TeX}
%    Kerning after opening parentheses and before closing parentheses
%    is $-0.1$\,em. The original values $-0.15$\,em
%    looked too ugly for a serif font.
%    \begin{macrocode}
\expandafter
\let\csname HoLogoHtml@(La)TeX\endcsname\HoLogoHtml@LaTeXTeX
%    \end{macrocode}
%    \end{macro}
%    \begin{macro}{\HoLogoCss@LaTeXTeX}
%    \begin{macrocode}
\def\HoLogoCss@LaTeXTeX{%
  \Css{%
    span.HoLogo-LaTeXTeX span.HoLogo-L{%
      margin-left:-.1em;%
    }%
  }%
  \Css{%
    span.HoLogo-LaTeXTeX span.HoLogo-a{%
      position:relative;%
      top:-.5ex;%
      margin-left:-.36em;%
      margin-right:-.1em;%
      font-size:85\%;%
    }%
  }%
  \Css{%
    span.HoLogo-LaTeXTeX span.HoLogo-ParenRight{%
      margin-right:-.15em;%
    }%
  }%
  \global\let\HoLogoCss@LaTeXTeX\relax
}
%    \end{macrocode}
%    \end{macro}
%
% \subsubsection{\hologo{LaTeXe}}
%
%    \begin{macro}{\HoLogo@LaTeXe}
%    Source: \hologo{LaTeX} kernel
%    \begin{macrocode}
\def\HoLogo@LaTeXe#1{%
  \hologo{LaTeX}%
  \kern.15em%
  \hbox{%
    \HOLOGO@MathSetup
    2%
    $_{\textstyle\varepsilon}$%
  }%
}
%    \end{macrocode}
%    \end{macro}
%
%    \begin{macro}{\HoLogoCs@LaTeXe}
%    \begin{macrocode}
\ifnum64=`\^^^^0040\relax % test for big chars of LuaTeX/XeTeX
  \catcode`\$=9 %
  \catcode`\&=14 %
\else
  \catcode`\$=14 %
  \catcode`\&=9 %
\fi
\def\HoLogoCs@LaTeXe#1{%
  LaTeX2%
$ \string ^^^^0395%
& e%
}%
\catcode`\$=3 %
\catcode`\&=4 %
%    \end{macrocode}
%    \end{macro}
%
%    \begin{macro}{\HoLogoBkm@LaTeXe}
%    \begin{macrocode}
\def\HoLogoBkm@LaTeXe#1{%
  \hologo{LaTeX}%
  2%
  \HOLOGO@PdfdocUnicode{e}{\textepsilon}%
}
%    \end{macrocode}
%    \end{macro}
%
%    \begin{macro}{\HoLogoHtml@LaTeXe}
%    \begin{macrocode}
\def\HoLogoHtml@LaTeXe#1{%
  \HoLogoCss@LaTeXe
  \HOLOGO@Span{LaTeX2e}{%
    \hologo{LaTeX}%
    \HOLOGO@Span{2}{2}%
    \HOLOGO@Span{e}{%
      \HOLOGO@MathSetup
      \ensuremath{\textstyle\varepsilon}%
    }%
  }%
}
%    \end{macrocode}
%    \end{macro}
%    \begin{macro}{\HoLogoCss@LaTeXe}
%    \begin{macrocode}
\def\HoLogoCss@LaTeXe{%
  \Css{%
    span.HoLogo-LaTeX2e span.HoLogo-2{%
      padding-left:.15em;%
    }%
  }%
  \Css{%
    span.HoLogo-LaTeX2e span.HoLogo-e{%
      position:relative;%
      top:.35ex;%
      text-decoration:none;%
    }%
  }%
  \global\let\HoLogoCss@LaTeXe\relax
}
%    \end{macrocode}
%    \end{macro}
%
%    \begin{macro}{\HoLogo@LaTeX2e}
%    \begin{macrocode}
\expandafter
\let\csname HoLogo@LaTeX2e\endcsname\HoLogo@LaTeXe
%    \end{macrocode}
%    \end{macro}
%    \begin{macro}{\HoLogoCs@LaTeX2e}
%    \begin{macrocode}
\expandafter
\let\csname HoLogoCs@LaTeX2e\endcsname\HoLogoCs@LaTeXe
%    \end{macrocode}
%    \end{macro}
%    \begin{macro}{\HoLogoBkm@LaTeX2e}
%    \begin{macrocode}
\expandafter
\let\csname HoLogoBkm@LaTeX2e\endcsname\HoLogoBkm@LaTeXe
%    \end{macrocode}
%    \end{macro}
%    \begin{macro}{\HoLogoHtml@LaTeX2e}
%    \begin{macrocode}
\expandafter
\let\csname HoLogoHtml@LaTeX2e\endcsname\HoLogoHtml@LaTeXe
%    \end{macrocode}
%    \end{macro}
%
% \subsubsection{\hologo{LaTeX3}}
%
%    \begin{macro}{\HoLogo@LaTeX3}
%    Source: \hologo{LaTeX} kernel
%    \begin{macrocode}
\expandafter\def\csname HoLogo@LaTeX3\endcsname#1{%
  \hologo{LaTeX}%
  3%
}
%    \end{macrocode}
%    \end{macro}
%
%    \begin{macro}{\HoLogoBkm@LaTeX3}
%    \begin{macrocode}
\expandafter\def\csname HoLogoBkm@LaTeX3\endcsname#1{%
  \hologo{LaTeX}%
  3%
}
%    \end{macrocode}
%    \end{macro}
%    \begin{macro}{\HoLogoHtml@LaTeX3}
%    \begin{macrocode}
\expandafter
\let\csname HoLogoHtml@LaTeX3\expandafter\endcsname
\csname HoLogo@LaTeX3\endcsname
%    \end{macrocode}
%    \end{macro}
%
% \subsubsection{\hologo{LaTeXML}}
%
%    \begin{macro}{\HoLogo@LaTeXML}
%    \begin{macrocode}
\def\HoLogo@LaTeXML#1{%
  \HOLOGO@mbox{%
    \hologo{La}%
    \kern-.15em%
    T%
    \kern-.1667em%
    \lower.5ex\hbox{E}%
    \kern-.125em%
    \HoLogoFont@font{LaTeXML}{sc}{xml}%
  }%
}
%    \end{macrocode}
%    \end{macro}
%    \begin{macro}{\HoLogoHtml@pdfLaTeX}
%    \begin{macrocode}
\def\HoLogoHtml@LaTeXML#1{%
  \HOLOGO@Span{LaTeXML}{%
    \HoLogoCss@LaTeX
    \HoLogoCss@TeX
    \HOLOGO@Span{LaTeX}{%
      L%
      \HOLOGO@Span{a}{%
        A%
      }%
    }%
    \HOLOGO@Span{TeX}{%
      T%
      \HOLOGO@Span{e}{%
        E%
      }%
    }%
    \HCode{<span style="font-variant: small-caps;">}%
    xml%
    \HCode{</span>}%
  }%
}
%    \end{macrocode}
%    \end{macro}
%
% \subsubsection{\hologo{eTeX}}
%
%    \begin{macro}{\HoLogo@eTeX}
%    Source: package \xpackage{etex}
%    \begin{macrocode}
\def\HoLogo@eTeX#1{%
  \ltx@mbox{%
    \HOLOGO@MathSetup
    $\varepsilon$%
    -%
    \HOLOGO@NegativeKerning{-T,T-,To}%
    \hologo{TeX}%
  }%
}
%    \end{macrocode}
%    \end{macro}
%    \begin{macro}{\HoLogoCs@eTeX}
%    \begin{macrocode}
\ifnum64=`\^^^^0040\relax % test for big chars of LuaTeX/XeTeX
  \catcode`\$=9 %
  \catcode`\&=14 %
\else
  \catcode`\$=14 %
  \catcode`\&=9 %
\fi
\def\HoLogoCs@eTeX#1{%
$ #1{\string ^^^^0395}{\string ^^^^03b5}%
& #1{e}{E}%
  TeX%
}%
\catcode`\$=3 %
\catcode`\&=4 %
%    \end{macrocode}
%    \end{macro}
%    \begin{macro}{\HoLogoBkm@eTeX}
%    \begin{macrocode}
\def\HoLogoBkm@eTeX#1{%
  \HOLOGO@PdfdocUnicode{#1{e}{E}}{\textepsilon}%
  -%
  \hologo{TeX}%
}
%    \end{macrocode}
%    \end{macro}
%    \begin{macro}{\HoLogoHtml@eTeX}
%    \begin{macrocode}
\def\HoLogoHtml@eTeX#1{%
  \ltx@mbox{%
    \HOLOGO@MathSetup
    $\varepsilon$%
    -%
    \hologo{TeX}%
  }%
}
%    \end{macrocode}
%    \end{macro}
%
% \subsubsection{\hologo{iniTeX}}
%
%    \begin{macro}{\HoLogo@iniTeX}
%    \begin{macrocode}
\def\HoLogo@iniTeX#1{%
  \HOLOGO@mbox{%
    #1{i}{I}ni\hologo{TeX}%
  }%
}
%    \end{macrocode}
%    \end{macro}
%    \begin{macro}{\HoLogoCs@iniTeX}
%    \begin{macrocode}
\def\HoLogoCs@iniTeX#1{#1{i}{I}niTeX}
%    \end{macrocode}
%    \end{macro}
%    \begin{macro}{\HoLogoBkm@iniTeX}
%    \begin{macrocode}
\def\HoLogoBkm@iniTeX#1{%
  #1{i}{I}ni\hologo{TeX}%
}
%    \end{macrocode}
%    \end{macro}
%    \begin{macro}{\HoLogoHtml@iniTeX}
%    \begin{macrocode}
\let\HoLogoHtml@iniTeX\HoLogo@iniTeX
%    \end{macrocode}
%    \end{macro}
%
% \subsubsection{\hologo{virTeX}}
%
%    \begin{macro}{\HoLogo@virTeX}
%    \begin{macrocode}
\def\HoLogo@virTeX#1{%
  \HOLOGO@mbox{%
    #1{v}{V}ir\hologo{TeX}%
  }%
}
%    \end{macrocode}
%    \end{macro}
%    \begin{macro}{\HoLogoCs@virTeX}
%    \begin{macrocode}
\def\HoLogoCs@virTeX#1{#1{v}{V}irTeX}
%    \end{macrocode}
%    \end{macro}
%    \begin{macro}{\HoLogoBkm@virTeX}
%    \begin{macrocode}
\def\HoLogoBkm@virTeX#1{%
  #1{v}{V}ir\hologo{TeX}%
}
%    \end{macrocode}
%    \end{macro}
%    \begin{macro}{\HoLogoHtml@virTeX}
%    \begin{macrocode}
\let\HoLogoHtml@virTeX\HoLogo@virTeX
%    \end{macrocode}
%    \end{macro}
%
% \subsubsection{\hologo{SliTeX}}
%
% \paragraph{Definitions of the three variants.}
%
%    \begin{macro}{\HoLogo@SLiTeX@lift}
%    \begin{macrocode}
\def\HoLogo@SLiTeX@lift#1{%
  \HoLogoFont@font{SliTeX}{rm}{%
    S%
    \kern-.06em%
    L%
    \kern-.18em%
    \raise.32ex\hbox{\HoLogoFont@font{SliTeX}{sc}{i}}%
    \HOLOGO@discretionary
    \kern-.06em%
    \hologo{TeX}%
  }%
}
%    \end{macrocode}
%    \end{macro}
%    \begin{macro}{\HoLogoBkm@SLiTeX@lift}
%    \begin{macrocode}
\def\HoLogoBkm@SLiTeX@lift#1{SLiTeX}
%    \end{macrocode}
%    \end{macro}
%    \begin{macro}{\HoLogoHtml@SLiTeX@lift}
%    \begin{macrocode}
\def\HoLogoHtml@SLiTeX@lift#1{%
  \HoLogoCss@SLiTeX@lift
  \HOLOGO@Span{SLiTeX-lift}{%
    \HoLogoFont@font{SliTeX}{rm}{%
      S%
      \HOLOGO@Span{L}{L}%
      \HOLOGO@Span{i}{i}%
      \hologo{TeX}%
    }%
  }%
}
%    \end{macrocode}
%    \end{macro}
%    \begin{macro}{\HoLogoCss@SLiTeX@lift}
%    \begin{macrocode}
\def\HoLogoCss@SLiTeX@lift{%
  \Css{%
    span.HoLogo-SLiTeX-lift span.HoLogo-L{%
      margin-left:-.06em;%
      margin-right:-.18em;%
    }%
  }%
  \Css{%
    span.HoLogo-SLiTeX-lift span.HoLogo-i{%
      position:relative;%
      top:-.32ex;%
      margin-right:-.06em;%
      font-variant:small-caps;%
    }%
  }%
  \global\let\HoLogoCss@SLiTeX@lift\relax
}
%    \end{macrocode}
%    \end{macro}
%
%    \begin{macro}{\HoLogo@SliTeX@simple}
%    \begin{macrocode}
\def\HoLogo@SliTeX@simple#1{%
  \HoLogoFont@font{SliTeX}{rm}{%
    \ltx@mbox{%
      \HoLogoFont@font{SliTeX}{sc}{Sli}%
    }%
    \HOLOGO@discretionary
    \hologo{TeX}%
  }%
}
%    \end{macrocode}
%    \end{macro}
%    \begin{macro}{\HoLogoBkm@SliTeX@simple}
%    \begin{macrocode}
\def\HoLogoBkm@SliTeX@simple#1{SliTeX}
%    \end{macrocode}
%    \end{macro}
%    \begin{macro}{\HoLogoHtml@SliTeX@simple}
%    \begin{macrocode}
\let\HoLogoHtml@SliTeX@simple\HoLogo@SliTeX@simple
%    \end{macrocode}
%    \end{macro}
%
%    \begin{macro}{\HoLogo@SliTeX@narrow}
%    \begin{macrocode}
\def\HoLogo@SliTeX@narrow#1{%
  \HoLogoFont@font{SliTeX}{rm}{%
    \ltx@mbox{%
      S%
      \kern-.06em%
      \HoLogoFont@font{SliTeX}{sc}{%
        l%
        \kern-.035em%
        i%
      }%
    }%
    \HOLOGO@discretionary
    \kern-.06em%
    \hologo{TeX}%
  }%
}
%    \end{macrocode}
%    \end{macro}
%    \begin{macro}{\HoLogoBkm@SliTeX@narrow}
%    \begin{macrocode}
\def\HoLogoBkm@SliTeX@narrow#1{SliTeX}
%    \end{macrocode}
%    \end{macro}
%    \begin{macro}{\HoLogoHtml@SliTeX@narrow}
%    \begin{macrocode}
\def\HoLogoHtml@SliTeX@narrow#1{%
  \HoLogoCss@SliTeX@narrow
  \HOLOGO@Span{SliTeX-narrow}{%
    \HoLogoFont@font{SliTeX}{rm}{%
      S%
        \HOLOGO@Span{l}{l}%
        \HOLOGO@Span{i}{i}%
      \hologo{TeX}%
    }%
  }%
}
%    \end{macrocode}
%    \end{macro}
%    \begin{macro}{\HoLogoCss@SliTeX@narrow}
%    \begin{macrocode}
\def\HoLogoCss@SliTeX@narrow{%
  \Css{%
    span.HoLogo-SliTeX-narrow span.HoLogo-l{%
      margin-left:-.06em;%
      margin-right:-.035em;%
      font-variant:small-caps;%
    }%
  }%
  \Css{%
    span.HoLogo-SliTeX-narrow span.HoLogo-i{%
      margin-right:-.06em;%
      font-variant:small-caps;%
    }%
  }%
  \global\let\HoLogoCss@SliTeX@narrow\relax
}
%    \end{macrocode}
%    \end{macro}
%
% \paragraph{Macro set completion.}
%
%    \begin{macro}{\HoLogo@SLiTeX@simple}
%    \begin{macrocode}
\def\HoLogo@SLiTeX@simple{\HoLogo@SliTeX@simple}
%    \end{macrocode}
%    \end{macro}
%    \begin{macro}{\HoLogoBkm@SLiTeX@simple}
%    \begin{macrocode}
\def\HoLogoBkm@SLiTeX@simple{\HoLogoBkm@SliTeX@simple}
%    \end{macrocode}
%    \end{macro}
%    \begin{macro}{\HoLogoHtml@SLiTeX@simple}
%    \begin{macrocode}
\def\HoLogoHtml@SLiTeX@simple{\HoLogoHtml@SliTeX@simple}
%    \end{macrocode}
%    \end{macro}
%
%    \begin{macro}{\HoLogo@SLiTeX@narrow}
%    \begin{macrocode}
\def\HoLogo@SLiTeX@narrow{\HoLogo@SliTeX@narrow}
%    \end{macrocode}
%    \end{macro}
%    \begin{macro}{\HoLogoBkm@SLiTeX@narrow}
%    \begin{macrocode}
\def\HoLogoBkm@SLiTeX@narrow{\HoLogoBkm@SliTeX@narrow}
%    \end{macrocode}
%    \end{macro}
%    \begin{macro}{\HoLogoHtml@SLiTeX@narrow}
%    \begin{macrocode}
\def\HoLogoHtml@SLiTeX@narrow{\HoLogoHtml@SliTeX@narrow}
%    \end{macrocode}
%    \end{macro}
%
%    \begin{macro}{\HoLogo@SliTeX@lift}
%    \begin{macrocode}
\def\HoLogo@SliTeX@lift{\HoLogo@SLiTeX@lift}
%    \end{macrocode}
%    \end{macro}
%    \begin{macro}{\HoLogoBkm@SliTeX@lift}
%    \begin{macrocode}
\def\HoLogoBkm@SliTeX@lift{\HoLogoBkm@SLiTeX@lift}
%    \end{macrocode}
%    \end{macro}
%    \begin{macro}{\HoLogoHtml@SliTeX@lift}
%    \begin{macrocode}
\def\HoLogoHtml@SliTeX@lift{\HoLogoHtml@SLiTeX@lift}
%    \end{macrocode}
%    \end{macro}
%
% \paragraph{Defaults.}
%
%    \begin{macro}{\HoLogo@SLiTeX}
%    \begin{macrocode}
\def\HoLogo@SLiTeX{\HoLogo@SLiTeX@lift}
%    \end{macrocode}
%    \end{macro}
%    \begin{macro}{\HoLogoBkm@SLiTeX}
%    \begin{macrocode}
\def\HoLogoBkm@SLiTeX{\HoLogoBkm@SLiTeX@lift}
%    \end{macrocode}
%    \end{macro}
%    \begin{macro}{\HoLogoHtml@SLiTeX}
%    \begin{macrocode}
\def\HoLogoHtml@SLiTeX{\HoLogoHtml@SLiTeX@lift}
%    \end{macrocode}
%    \end{macro}
%
%    \begin{macro}{\HoLogo@SliTeX}
%    \begin{macrocode}
\def\HoLogo@SliTeX{\HoLogo@SliTeX@narrow}
%    \end{macrocode}
%    \end{macro}
%    \begin{macro}{\HoLogoBkm@SliTeX}
%    \begin{macrocode}
\def\HoLogoBkm@SliTeX{\HoLogoBkm@SliTeX@narrow}
%    \end{macrocode}
%    \end{macro}
%    \begin{macro}{\HoLogoHtml@SliTeX}
%    \begin{macrocode}
\def\HoLogoHtml@SliTeX{\HoLogoHtml@SliTeX@narrow}
%    \end{macrocode}
%    \end{macro}
%
% \subsubsection{\hologo{LuaTeX}}
%
%    \begin{macro}{\HoLogo@LuaTeX}
%    The kerning is an idea of Hans Hagen, see mailing list
%    `luatex at tug dot org' in March 2010.
%    \begin{macrocode}
\def\HoLogo@LuaTeX#1{%
  \HOLOGO@mbox{%
    Lua%
    \HOLOGO@NegativeKerning{aT,oT,To}%
    \hologo{TeX}%
  }%
}
%    \end{macrocode}
%    \end{macro}
%    \begin{macro}{\HoLogoHtml@LuaTeX}
%    \begin{macrocode}
\let\HoLogoHtml@LuaTeX\HoLogo@LuaTeX
%    \end{macrocode}
%    \end{macro}
%
% \subsubsection{\hologo{LuaLaTeX}}
%
%    \begin{macro}{\HoLogo@LuaLaTeX}
%    \begin{macrocode}
\def\HoLogo@LuaLaTeX#1{%
  \HOLOGO@mbox{%
    Lua%
    \hologo{LaTeX}%
  }%
}
%    \end{macrocode}
%    \end{macro}
%    \begin{macro}{\HoLogoHtml@LuaLaTeX}
%    \begin{macrocode}
\let\HoLogoHtml@LuaLaTeX\HoLogo@LuaLaTeX
%    \end{macrocode}
%    \end{macro}
%
% \subsubsection{\hologo{XeTeX}, \hologo{XeLaTeX}}
%
%    \begin{macro}{\HOLOGO@IfCharExists}
%    \begin{macrocode}
\ifluatex
  \ifnum\luatexversion<36 %
  \else
    \def\HOLOGO@IfCharExists#1{%
      \ifnum
        \directlua{%
           if luaotfload and luaotfload.aux then
             if luaotfload.aux.font_has_glyph(%
                    font.current(), \number#1) then % 	 
	       tex.print("1") % 	 
	     end % 	 
	   elseif font and font.fonts and font.current then %
            local f = font.fonts[font.current()]%
            if f.characters and f.characters[\number#1] then %
              tex.print("1")%
            end %
          end%
        }0=\ltx@zero
        \expandafter\ltx@secondoftwo
      \else
        \expandafter\ltx@firstoftwo
      \fi
    }%
  \fi
\fi
\ltx@IfUndefined{HOLOGO@IfCharExists}{%
  \def\HOLOGO@@IfCharExists#1{%
    \begingroup
      \tracinglostchars=\ltx@zero
      \setbox\ltx@zero=\hbox{%
        \kern7sp\char#1\relax
        \ifnum\lastkern>\ltx@zero
          \expandafter\aftergroup\csname iffalse\endcsname
        \else
          \expandafter\aftergroup\csname iftrue\endcsname
        \fi
      }%
      % \if{true|false} from \aftergroup
      \endgroup
      \expandafter\ltx@firstoftwo
    \else
      \endgroup
      \expandafter\ltx@secondoftwo
    \fi
  }%
  \ifxetex
    \ltx@IfUndefined{XeTeXfonttype}{}{%
      \ltx@IfUndefined{XeTeXcharglyph}{}{%
        \def\HOLOGO@IfCharExists#1{%
          \ifnum\XeTeXfonttype\font>\ltx@zero
            \expandafter\ltx@firstofthree
          \else
            \expandafter\ltx@gobble
          \fi
          {%
            \ifnum\XeTeXcharglyph#1>\ltx@zero
              \expandafter\ltx@firstoftwo
            \else
              \expandafter\ltx@secondoftwo
            \fi
          }%
          \HOLOGO@@IfCharExists{#1}%
        }%
      }%
    }%
  \fi
}{}
\ltx@ifundefined{HOLOGO@IfCharExists}{%
  \ifnum64=`\^^^^0040\relax % test for big chars of LuaTeX/XeTeX
    \let\HOLOGO@IfCharExists\HOLOGO@@IfCharExists
  \else
    \def\HOLOGO@IfCharExists#1{%
      \ifnum#1>255 %
        \expandafter\ltx@fourthoffour
      \fi
      \HOLOGO@@IfCharExists{#1}%
    }%
  \fi
}{}
%    \end{macrocode}
%    \end{macro}
%
%    \begin{macro}{\HoLogo@Xe}
%    Source: package \xpackage{dtklogos}
%    \begin{macrocode}
\def\HoLogo@Xe#1{%
  X%
  \kern-.1em\relax
  \HOLOGO@IfCharExists{"018E}{%
    \lower.5ex\hbox{\char"018E}%
  }{%
    \chardef\HOLOGO@choice=\ltx@zero
    \ifdim\fontdimen\ltx@one\font>0pt %
      \ltx@IfUndefined{rotatebox}{%
        \ltx@IfUndefined{pgftext}{%
          \ltx@IfUndefined{psscalebox}{%
            \ltx@IfUndefined{HOLOGO@ScaleBox@\hologoDriver}{%
            }{%
              \chardef\HOLOGO@choice=4 %
            }%
          }{%
            \chardef\HOLOGO@choice=3 %
          }%
        }{%
          \chardef\HOLOGO@choice=2 %
        }%
      }{%
        \chardef\HOLOGO@choice=1 %
      }%
      \ifcase\HOLOGO@choice
        \HOLOGO@WarningUnsupportedDriver{Xe}%
        e%
      \or % 1: \rotatebox
        \begingroup
          \setbox\ltx@zero\hbox{\rotatebox{180}{E}}%
          \ltx@LocDimenA=\dp\ltx@zero
          \advance\ltx@LocDimenA by -.5ex\relax
          \raise\ltx@LocDimenA\box\ltx@zero
        \endgroup
      \or % 2: \pgftext
        \lower.5ex\hbox{%
          \pgfpicture
            \pgftext[rotate=180]{E}%
          \endpgfpicture
        }%
      \or % 3: \psscalebox
        \begingroup
          \setbox\ltx@zero\hbox{\psscalebox{-1 -1}{E}}%
          \ltx@LocDimenA=\dp\ltx@zero
          \advance\ltx@LocDimenA by -.5ex\relax
          \raise\ltx@LocDimenA\box\ltx@zero
        \endgroup
      \or % 4: \HOLOGO@PointReflectBox
        \lower.5ex\hbox{\HOLOGO@PointReflectBox{E}}%
      \else
        \@PackageError{hologo}{Internal error (choice/it}\@ehc
      \fi
    \else
      \ltx@IfUndefined{reflectbox}{%
        \ltx@IfUndefined{pgftext}{%
          \ltx@IfUndefined{psscalebox}{%
            \ltx@IfUndefined{HOLOGO@ScaleBox@\hologoDriver}{%
            }{%
              \chardef\HOLOGO@choice=4 %
            }%
          }{%
            \chardef\HOLOGO@choice=3 %
          }%
        }{%
          \chardef\HOLOGO@choice=2 %
        }%
      }{%
        \chardef\HOLOGO@choice=1 %
      }%
      \ifcase\HOLOGO@choice
        \HOLOGO@WarningUnsupportedDriver{Xe}%
        e%
      \or % 1: reflectbox
        \lower.5ex\hbox{%
          \reflectbox{E}%
        }%
      \or % 2: \pgftext
        \lower.5ex\hbox{%
          \pgfpicture
            \pgftransformxscale{-1}%
            \pgftext{E}%
          \endpgfpicture
        }%
      \or % 3: \psscalebox
        \lower.5ex\hbox{%
          \psscalebox{-1 1}{E}%
        }%
      \or % 4: \HOLOGO@Reflectbox
        \lower.5ex\hbox{%
          \HOLOGO@ReflectBox{E}%
        }%
      \else
        \@PackageError{hologo}{Internal error (choice/up)}\@ehc
      \fi
    \fi
  }%
}
%    \end{macrocode}
%    \end{macro}
%    \begin{macro}{\HoLogoHtml@Xe}
%    \begin{macrocode}
\def\HoLogoHtml@Xe#1{%
  \HoLogoCss@Xe
  \HOLOGO@Span{Xe}{%
    X%
    \HOLOGO@Span{e}{%
      \HCode{&\ltx@hashchar x018e;}%
    }%
  }%
}
%    \end{macrocode}
%    \end{macro}
%    \begin{macro}{\HoLogoCss@Xe}
%    \begin{macrocode}
\def\HoLogoCss@Xe{%
  \Css{%
    span.HoLogo-Xe span.HoLogo-e{%
      position:relative;%
      top:.5ex;%
      left-margin:-.1em;%
    }%
  }%
  \global\let\HoLogoCss@Xe\relax
}
%    \end{macrocode}
%    \end{macro}
%
%    \begin{macro}{\HoLogo@XeTeX}
%    \begin{macrocode}
\def\HoLogo@XeTeX#1{%
  \hologo{Xe}%
  \kern-.15em\relax
  \hologo{TeX}%
}
%    \end{macrocode}
%    \end{macro}
%
%    \begin{macro}{\HoLogoHtml@XeTeX}
%    \begin{macrocode}
\def\HoLogoHtml@XeTeX#1{%
  \HoLogoCss@XeTeX
  \HOLOGO@Span{XeTeX}{%
    \hologo{Xe}%
    \hologo{TeX}%
  }%
}
%    \end{macrocode}
%    \end{macro}
%    \begin{macro}{\HoLogoCss@XeTeX}
%    \begin{macrocode}
\def\HoLogoCss@XeTeX{%
  \Css{%
    span.HoLogo-XeTeX span.HoLogo-TeX{%
      margin-left:-.15em;%
    }%
  }%
  \global\let\HoLogoCss@XeTeX\relax
}
%    \end{macrocode}
%    \end{macro}
%
%    \begin{macro}{\HoLogo@XeLaTeX}
%    \begin{macrocode}
\def\HoLogo@XeLaTeX#1{%
  \hologo{Xe}%
  \kern-.13em%
  \hologo{LaTeX}%
}
%    \end{macrocode}
%    \end{macro}
%    \begin{macro}{\HoLogoHtml@XeLaTeX}
%    \begin{macrocode}
\def\HoLogoHtml@XeLaTeX#1{%
  \HoLogoCss@XeLaTeX
  \HOLOGO@Span{XeLaTeX}{%
    \hologo{Xe}%
    \hologo{LaTeX}%
  }%
}
%    \end{macrocode}
%    \end{macro}
%    \begin{macro}{\HoLogoCss@XeLaTeX}
%    \begin{macrocode}
\def\HoLogoCss@XeLaTeX{%
  \Css{%
    span.HoLogo-XeLaTeX span.HoLogo-Xe{%
      margin-right:-.13em;%
    }%
  }%
  \global\let\HoLogoCss@XeLaTeX\relax
}
%    \end{macrocode}
%    \end{macro}
%
% \subsubsection{\hologo{pdfTeX}, \hologo{pdfLaTeX}}
%
%    \begin{macro}{\HoLogo@pdfTeX}
%    \begin{macrocode}
\def\HoLogo@pdfTeX#1{%
  \HOLOGO@mbox{%
    #1{p}{P}df\hologo{TeX}%
  }%
}
%    \end{macrocode}
%    \end{macro}
%    \begin{macro}{\HoLogoCs@pdfTeX}
%    \begin{macrocode}
\def\HoLogoCs@pdfTeX#1{#1{p}{P}dfTeX}
%    \end{macrocode}
%    \end{macro}
%    \begin{macro}{\HoLogoBkm@pdfTeX}
%    \begin{macrocode}
\def\HoLogoBkm@pdfTeX#1{%
  #1{p}{P}df\hologo{TeX}%
}
%    \end{macrocode}
%    \end{macro}
%    \begin{macro}{\HoLogoHtml@pdfTeX}
%    \begin{macrocode}
\let\HoLogoHtml@pdfTeX\HoLogo@pdfTeX
%    \end{macrocode}
%    \end{macro}
%
%    \begin{macro}{\HoLogo@pdfLaTeX}
%    \begin{macrocode}
\def\HoLogo@pdfLaTeX#1{%
  \HOLOGO@mbox{%
    #1{p}{P}df\hologo{LaTeX}%
  }%
}
%    \end{macrocode}
%    \end{macro}
%    \begin{macro}{\HoLogoCs@pdfLaTeX}
%    \begin{macrocode}
\def\HoLogoCs@pdfLaTeX#1{#1{p}{P}dfLaTeX}
%    \end{macrocode}
%    \end{macro}
%    \begin{macro}{\HoLogoBkm@pdfLaTeX}
%    \begin{macrocode}
\def\HoLogoBkm@pdfLaTeX#1{%
  #1{p}{P}df\hologo{LaTeX}%
}
%    \end{macrocode}
%    \end{macro}
%    \begin{macro}{\HoLogoHtml@pdfLaTeX}
%    \begin{macrocode}
\let\HoLogoHtml@pdfLaTeX\HoLogo@pdfLaTeX
%    \end{macrocode}
%    \end{macro}
%
% \subsubsection{\hologo{VTeX}}
%
%    \begin{macro}{\HoLogo@VTeX}
%    \begin{macrocode}
\def\HoLogo@VTeX#1{%
  \HOLOGO@mbox{%
    V\hologo{TeX}%
  }%
}
%    \end{macrocode}
%    \end{macro}
%    \begin{macro}{\HoLogoHtml@VTeX}
%    \begin{macrocode}
\let\HoLogoHtml@VTeX\HoLogo@VTeX
%    \end{macrocode}
%    \end{macro}
%
% \subsubsection{\hologo{AmS}, \dots}
%
%    Source: class \xclass{amsdtx}
%
%    \begin{macro}{\HoLogo@AmS}
%    \begin{macrocode}
\def\HoLogo@AmS#1{%
  \HoLogoFont@font{AmS}{sy}{%
    A%
    \kern-.1667em%
    \lower.5ex\hbox{M}%
    \kern-.125em%
    S%
  }%
}
%    \end{macrocode}
%    \end{macro}
%    \begin{macro}{\HoLogoBkm@AmS}
%    \begin{macrocode}
\def\HoLogoBkm@AmS#1{AmS}
%    \end{macrocode}
%    \end{macro}
%    \begin{macro}{\HoLogoHtml@AmS}
%    \begin{macrocode}
\def\HoLogoHtml@AmS#1{%
  \HoLogoCss@AmS
%  \HoLogoFont@font{AmS}{sy}{%
    \HOLOGO@Span{AmS}{%
      A%
      \HOLOGO@Span{M}{M}%
      S%
    }%
%   }%
}
%    \end{macrocode}
%    \end{macro}
%    \begin{macro}{\HoLogoCss@AmS}
%    \begin{macrocode}
\def\HoLogoCss@AmS{%
  \Css{%
    span.HoLogo-AmS span.HoLogo-M{%
      position:relative;%
      top:.5ex;%
      margin-left:-.1667em;%
      margin-right:-.125em;%
      text-decoration:none;%
    }%
  }%
  \global\let\HoLogoCss@AmS\relax
}
%    \end{macrocode}
%    \end{macro}
%
%    \begin{macro}{\HoLogo@AmSTeX}
%    \begin{macrocode}
\def\HoLogo@AmSTeX#1{%
  \hologo{AmS}%
  \HOLOGO@hyphen
  \hologo{TeX}%
}
%    \end{macrocode}
%    \end{macro}
%    \begin{macro}{\HoLogoBkm@AmSTeX}
%    \begin{macrocode}
\def\HoLogoBkm@AmSTeX#1{AmS-TeX}%
%    \end{macrocode}
%    \end{macro}
%    \begin{macro}{\HoLogoHtml@AmSTeX}
%    \begin{macrocode}
\let\HoLogoHtml@AmSTeX\HoLogo@AmSTeX
%    \end{macrocode}
%    \end{macro}
%
%    \begin{macro}{\HoLogo@AmSLaTeX}
%    \begin{macrocode}
\def\HoLogo@AmSLaTeX#1{%
  \hologo{AmS}%
  \HOLOGO@hyphen
  \hologo{LaTeX}%
}
%    \end{macrocode}
%    \end{macro}
%    \begin{macro}{\HoLogoBkm@AmSLaTeX}
%    \begin{macrocode}
\def\HoLogoBkm@AmSLaTeX#1{AmS-LaTeX}%
%    \end{macrocode}
%    \end{macro}
%    \begin{macro}{\HoLogoHtml@AmSLaTeX}
%    \begin{macrocode}
\let\HoLogoHtml@AmSLaTeX\HoLogo@AmSLaTeX
%    \end{macrocode}
%    \end{macro}
%
% \subsubsection{\hologo{BibTeX}}
%
%    \begin{macro}{\HoLogo@BibTeX@sc}
%    A definition of \hologo{BibTeX} is provided in
%    the documentation source for the manual of \hologo{BibTeX}
%    \cite{btxdoc}.
%\begin{quote}
%\begin{verbatim}
%\def\BibTeX{%
%  {%
%    \rm
%    B%
%    \kern-.05em%
%    {%
%      \sc
%      i%
%      \kern-.025em %
%      b%
%    }%
%    \kern-.08em
%    T%
%    \kern-.1667em%
%    \lower.7ex\hbox{E}%
%    \kern-.125em%
%    X%
%  }%
%}
%\end{verbatim}
%\end{quote}
%    \begin{macrocode}
\def\HoLogo@BibTeX@sc#1{%
  B%
  \kern-.05em%
  \HoLogoFont@font{BibTeX}{sc}{%
    i%
    \kern-.025em%
    b%
  }%
  \HOLOGO@discretionary
  \kern-.08em%
  \hologo{TeX}%
}
%    \end{macrocode}
%    \end{macro}
%    \begin{macro}{\HoLogoHtml@BibTeX@sc}
%    \begin{macrocode}
\def\HoLogoHtml@BibTeX@sc#1{%
  \HoLogoCss@BibTeX@sc
  \HOLOGO@Span{BibTeX-sc}{%
    B%
    \HOLOGO@Span{i}{i}%
    \HOLOGO@Span{b}{b}%
    \hologo{TeX}%
  }%
}
%    \end{macrocode}
%    \end{macro}
%    \begin{macro}{\HoLogoCss@BibTeX@sc}
%    \begin{macrocode}
\def\HoLogoCss@BibTeX@sc{%
  \Css{%
    span.HoLogo-BibTeX-sc span.HoLogo-i{%
      margin-left:-.05em;%
      margin-right:-.025em;%
      font-variant:small-caps;%
    }%
  }%
  \Css{%
    span.HoLogo-BibTeX-sc span.HoLogo-b{%
      margin-right:-.08em;%
      font-variant:small-caps;%
    }%
  }%
  \global\let\HoLogoCss@BibTeX@sc\relax
}
%    \end{macrocode}
%    \end{macro}
%
%    \begin{macro}{\HoLogo@BibTeX@sf}
%    Variant \xoption{sf} avoids trouble with unavailable
%    small caps fonts (e.g., bold versions of Computer Modern or
%    Latin Modern). The definition is taken from
%    package \xpackage{dtklogos} \cite{dtklogos}.
%\begin{quote}
%\begin{verbatim}
%\DeclareRobustCommand{\BibTeX}{%
%  B%
%  \kern-.05em%
%  \hbox{%
%    $\m@th$% %% force math size calculations
%    \csname S@\f@size\endcsname
%    \fontsize\sf@size\z@
%    \math@fontsfalse
%    \selectfont
%    I%
%    \kern-.025em%
%    B
%  }%
%  \kern-.08em%
%  \-%
%  \TeX
%}
%\end{verbatim}
%\end{quote}
%    \begin{macrocode}
\def\HoLogo@BibTeX@sf#1{%
  B%
  \kern-.05em%
  \HoLogoFont@font{BibTeX}{bibsf}{%
    I%
    \kern-.025em%
    B%
  }%
  \HOLOGO@discretionary
  \kern-.08em%
  \hologo{TeX}%
}
%    \end{macrocode}
%    \end{macro}
%    \begin{macro}{\HoLogoHtml@BibTeX@sf}
%    \begin{macrocode}
\def\HoLogoHtml@BibTeX@sf#1{%
  \HoLogoCss@BibTeX@sf
  \HOLOGO@Span{BibTeX-sf}{%
    B%
    \HoLogoFont@font{BibTeX}{bibsf}{%
      \HOLOGO@Span{i}{I}%
      B%
    }%
    \hologo{TeX}%
  }%
}
%    \end{macrocode}
%    \end{macro}
%    \begin{macro}{\HoLogoCss@BibTeX@sf}
%    \begin{macrocode}
\def\HoLogoCss@BibTeX@sf{%
  \Css{%
    span.HoLogo-BibTeX-sf span.HoLogo-i{%
      margin-left:-.05em;%
      margin-right:-.025em;%
    }%
  }%
  \Css{%
    span.HoLogo-BibTeX-sf span.HoLogo-TeX{%
      margin-left:-.08em;%
    }%
  }%
  \global\let\HoLogoCss@BibTeX@sf\relax
}
%    \end{macrocode}
%    \end{macro}
%
%    \begin{macro}{\HoLogo@BibTeX}
%    \begin{macrocode}
\def\HoLogo@BibTeX{\HoLogo@BibTeX@sf}
%    \end{macrocode}
%    \end{macro}
%    \begin{macro}{\HoLogoHtml@BibTeX}
%    \begin{macrocode}
\def\HoLogoHtml@BibTeX{\HoLogoHtml@BibTeX@sf}
%    \end{macrocode}
%    \end{macro}
%
% \subsubsection{\hologo{BibTeX8}}
%
%    \begin{macro}{\HoLogo@BibTeX8}
%    \begin{macrocode}
\expandafter\def\csname HoLogo@BibTeX8\endcsname#1{%
  \hologo{BibTeX}%
  8%
}
%    \end{macrocode}
%    \end{macro}
%
%    \begin{macro}{\HoLogoBkm@BibTeX8}
%    \begin{macrocode}
\expandafter\def\csname HoLogoBkm@BibTeX8\endcsname#1{%
  \hologo{BibTeX}%
  8%
}
%    \end{macrocode}
%    \end{macro}
%    \begin{macro}{\HoLogoHtml@BibTeX8}
%    \begin{macrocode}
\expandafter
\let\csname HoLogoHtml@BibTeX8\expandafter\endcsname
\csname HoLogo@BibTeX8\endcsname
%    \end{macrocode}
%    \end{macro}
%
% \subsubsection{\hologo{ConTeXt}}
%
%    \begin{macro}{\HoLogo@ConTeXt@simple}
%    \begin{macrocode}
\def\HoLogo@ConTeXt@simple#1{%
  \HOLOGO@mbox{Con}%
  \HOLOGO@discretionary
  \HOLOGO@mbox{\hologo{TeX}t}%
}
%    \end{macrocode}
%    \end{macro}
%    \begin{macro}{\HoLogoHtml@ConTeXt@simple}
%    \begin{macrocode}
\let\HoLogoHtml@ConTeXt@simple\HoLogo@ConTeXt@simple
%    \end{macrocode}
%    \end{macro}
%
%    \begin{macro}{\HoLogo@ConTeXt@narrow}
%    This definition of logo \hologo{ConTeXt} with variant \xoption{narrow}
%    comes from TUGboat's class \xclass{ltugboat} (version 2010/11/15 v2.8).
%    \begin{macrocode}
\def\HoLogo@ConTeXt@narrow#1{%
  \HOLOGO@mbox{C\kern-.0333emon}%
  \HOLOGO@discretionary
  \kern-.0667em%
  \HOLOGO@mbox{\hologo{TeX}\kern-.0333emt}%
}
%    \end{macrocode}
%    \end{macro}
%    \begin{macro}{\HoLogoHtml@ConTeXt@narrow}
%    \begin{macrocode}
\def\HoLogoHtml@ConTeXt@narrow#1{%
  \HoLogoCss@ConTeXt@narrow
  \HOLOGO@Span{ConTeXt-narrow}{%
    \HOLOGO@Span{C}{C}%
    on%
    \hologo{TeX}%
    t%
  }%
}
%    \end{macrocode}
%    \end{macro}
%    \begin{macro}{\HoLogoCss@ConTeXt@narrow}
%    \begin{macrocode}
\def\HoLogoCss@ConTeXt@narrow{%
  \Css{%
    span.HoLogo-ConTeXt-narrow span.HoLogo-C{%
      margin-left:-.0333em;%
    }%
  }%
  \Css{%
    span.HoLogo-ConTeXt-narrow span.HoLogo-TeX{%
      margin-left:-.0667em;%
      margin-right:-.0333em;%
    }%
  }%
  \global\let\HoLogoCss@ConTeXt@narrow\relax
}
%    \end{macrocode}
%    \end{macro}
%
%    \begin{macro}{\HoLogo@ConTeXt}
%    \begin{macrocode}
\def\HoLogo@ConTeXt{\HoLogo@ConTeXt@narrow}
%    \end{macrocode}
%    \end{macro}
%    \begin{macro}{\HoLogoHtml@ConTeXt}
%    \begin{macrocode}
\def\HoLogoHtml@ConTeXt{\HoLogoHtml@ConTeXt@narrow}
%    \end{macrocode}
%    \end{macro}
%
% \subsubsection{\hologo{emTeX}}
%
%    \begin{macro}{\HoLogo@emTeX}
%    \begin{macrocode}
\def\HoLogo@emTeX#1{%
  \HOLOGO@mbox{#1{e}{E}m}%
  \HOLOGO@discretionary
  \hologo{TeX}%
}
%    \end{macrocode}
%    \end{macro}
%    \begin{macro}{\HoLogoCs@emTeX}
%    \begin{macrocode}
\def\HoLogoCs@emTeX#1{#1{e}{E}mTeX}%
%    \end{macrocode}
%    \end{macro}
%    \begin{macro}{\HoLogoBkm@emTeX}
%    \begin{macrocode}
\def\HoLogoBkm@emTeX#1{%
  #1{e}{E}m\hologo{TeX}%
}
%    \end{macrocode}
%    \end{macro}
%    \begin{macro}{\HoLogoHtml@emTeX}
%    \begin{macrocode}
\let\HoLogoHtml@emTeX\HoLogo@emTeX
%    \end{macrocode}
%    \end{macro}
%
% \subsubsection{\hologo{ExTeX}}
%
%    \begin{macro}{\HoLogo@ExTeX}
%    The definition is taken from the FAQ of the
%    project \hologo{ExTeX}
%    \cite{ExTeX-FAQ}.
%\begin{quote}
%\begin{verbatim}
%\def\ExTeX{%
%  \textrm{% Logo always with serifs
%    \ensuremath{%
%      \textstyle
%      \varepsilon_{%
%        \kern-0.15em%
%        \mathcal{X}%
%      }%
%    }%
%    \kern-.15em%
%    \TeX
%  }%
%}
%\end{verbatim}
%\end{quote}
%    \begin{macrocode}
\def\HoLogo@ExTeX#1{%
  \HoLogoFont@font{ExTeX}{rm}{%
    \ltx@mbox{%
      \HOLOGO@MathSetup
      $%
        \textstyle
        \varepsilon_{%
          \kern-0.15em%
          \HoLogoFont@font{ExTeX}{sy}{X}%
        }%
      $%
    }%
    \HOLOGO@discretionary
    \kern-.15em%
    \hologo{TeX}%
  }%
}
%    \end{macrocode}
%    \end{macro}
%    \begin{macro}{\HoLogoHtml@ExTeX}
%    \begin{macrocode}
\def\HoLogoHtml@ExTeX#1{%
  \HoLogoCss@ExTeX
  \HoLogoFont@font{ExTeX}{rm}{%
    \HOLOGO@Span{ExTeX}{%
      \ltx@mbox{%
        \HOLOGO@MathSetup
        $\textstyle\varepsilon$%
        \HOLOGO@Span{X}{$\textstyle\chi$}%
        \hologo{TeX}%
      }%
    }%
  }%
}
%    \end{macrocode}
%    \end{macro}
%    \begin{macro}{\HoLogoBkm@ExTeX}
%    \begin{macrocode}
\def\HoLogoBkm@ExTeX#1{%
  \HOLOGO@PdfdocUnicode{#1{e}{E}x}{\textepsilon\textchi}%
  \hologo{TeX}%
}
%    \end{macrocode}
%    \end{macro}
%    \begin{macro}{\HoLogoCss@ExTeX}
%    \begin{macrocode}
\def\HoLogoCss@ExTeX{%
  \Css{%
    span.HoLogo-ExTeX{%
      font-family:serif;%
    }%
  }%
  \Css{%
    span.HoLogo-ExTeX span.HoLogo-TeX{%
      margin-left:-.15em;%
    }%
  }%
  \global\let\HoLogoCss@ExTeX\relax
}
%    \end{macrocode}
%    \end{macro}
%
% \subsubsection{\hologo{MiKTeX}}
%
%    \begin{macro}{\HoLogo@MiKTeX}
%    \begin{macrocode}
\def\HoLogo@MiKTeX#1{%
  \HOLOGO@mbox{MiK}%
  \HOLOGO@discretionary
  \hologo{TeX}%
}
%    \end{macrocode}
%    \end{macro}
%    \begin{macro}{\HoLogoHtml@MiKTeX}
%    \begin{macrocode}
\let\HoLogoHtml@MiKTeX\HoLogo@MiKTeX
%    \end{macrocode}
%    \end{macro}
%
% \subsubsection{\hologo{OzTeX} and friends}
%
%    Source: \hologo{OzTeX} FAQ \cite{OzTeX}:
%    \begin{quote}
%      |\def\OzTeX{O\kern-.03em z\kern-.15em\TeX}|\\
%      (There is no kerning in OzMF, OzMP and OzTtH.)
%    \end{quote}
%
%    \begin{macro}{\HoLogo@OzTeX}
%    \begin{macrocode}
\def\HoLogo@OzTeX#1{%
  O%
  \kern-.03em %
  z%
  \kern-.15em %
  \hologo{TeX}%
}
%    \end{macrocode}
%    \end{macro}
%    \begin{macro}{\HoLogoHtml@OzTeX}
%    \begin{macrocode}
\def\HoLogoHtml@OzTeX#1{%
  \HoLogoCss@OzTeX
  \HOLOGO@Span{OzTeX}{%
    O%
    \HOLOGO@Span{z}{z}%
    \hologo{TeX}%
  }%
}
%    \end{macrocode}
%    \end{macro}
%    \begin{macro}{\HoLogoCss@OzTeX}
%    \begin{macrocode}
\def\HoLogoCss@OzTeX{%
  \Css{%
    span.HoLogo-OzTeX span.HoLogo-z{%
      margin-left:-.03em;%
      margin-right:-.15em;%
    }%
  }%
  \global\let\HoLogoCss@OzTeX\relax
}
%    \end{macrocode}
%    \end{macro}
%
%    \begin{macro}{\HoLogo@OzMF}
%    \begin{macrocode}
\def\HoLogo@OzMF#1{%
  \HOLOGO@mbox{OzMF}%
}
%    \end{macrocode}
%    \end{macro}
%    \begin{macro}{\HoLogo@OzMP}
%    \begin{macrocode}
\def\HoLogo@OzMP#1{%
  \HOLOGO@mbox{OzMP}%
}
%    \end{macrocode}
%    \end{macro}
%    \begin{macro}{\HoLogo@OzTtH}
%    \begin{macrocode}
\def\HoLogo@OzTtH#1{%
  \HOLOGO@mbox{OzTtH}%
}
%    \end{macrocode}
%    \end{macro}
%
% \subsubsection{\hologo{PCTeX}}
%
%    \begin{macro}{\HoLogo@PCTeX}
%    \begin{macrocode}
\def\HoLogo@PCTeX#1{%
  \HOLOGO@mbox{PC}%
  \hologo{TeX}%
}
%    \end{macrocode}
%    \end{macro}
%    \begin{macro}{\HoLogoHtml@PCTeX}
%    \begin{macrocode}
\let\HoLogoHtml@PCTeX\HoLogo@PCTeX
%    \end{macrocode}
%    \end{macro}
%
% \subsubsection{\hologo{PiCTeX}}
%
%    The original definitions from \xfile{pictex.tex} \cite{PiCTeX}:
%\begin{quote}
%\begin{verbatim}
%\def\PiC{%
%  P%
%  \kern-.12em%
%  \lower.5ex\hbox{I}%
%  \kern-.075em%
%  C%
%}
%\def\PiCTeX{%
%  \PiC
%  \kern-.11em%
%  \TeX
%}
%\end{verbatim}
%\end{quote}
%
%    \begin{macro}{\HoLogo@PiC}
%    \begin{macrocode}
\def\HoLogo@PiC#1{%
  P%
  \kern-.12em%
  \lower.5ex\hbox{I}%
  \kern-.075em%
  C%
  \HOLOGO@SpaceFactor
}
%    \end{macrocode}
%    \end{macro}
%    \begin{macro}{\HoLogoHtml@PiC}
%    \begin{macrocode}
\def\HoLogoHtml@PiC#1{%
  \HoLogoCss@PiC
  \HOLOGO@Span{PiC}{%
    P%
    \HOLOGO@Span{i}{I}%
    C%
  }%
}
%    \end{macrocode}
%    \end{macro}
%    \begin{macro}{\HoLogoCss@PiC}
%    \begin{macrocode}
\def\HoLogoCss@PiC{%
  \Css{%
    span.HoLogo-PiC span.HoLogo-i{%
      position:relative;%
      top:.5ex;%
      margin-left:-.12em;%
      margin-right:-.075em;%
      text-decoration:none;%
    }%
  }%
  \global\let\HoLogoCss@PiC\relax
}
%    \end{macrocode}
%    \end{macro}
%
%    \begin{macro}{\HoLogo@PiCTeX}
%    \begin{macrocode}
\def\HoLogo@PiCTeX#1{%
  \hologo{PiC}%
  \HOLOGO@discretionary
  \kern-.11em%
  \hologo{TeX}%
}
%    \end{macrocode}
%    \end{macro}
%    \begin{macro}{\HoLogoHtml@PiCTeX}
%    \begin{macrocode}
\def\HoLogoHtml@PiCTeX#1{%
  \HoLogoCss@PiCTeX
  \HOLOGO@Span{PiCTeX}{%
    \hologo{PiC}%
    \hologo{TeX}%
  }%
}
%    \end{macrocode}
%    \end{macro}
%    \begin{macro}{\HoLogoCss@PiCTeX}
%    \begin{macrocode}
\def\HoLogoCss@PiCTeX{%
  \Css{%
    span.HoLogo-PiCTeX span.HoLogo-PiC{%
      margin-right:-.11em;%
    }%
  }%
  \global\let\HoLogoCss@PiCTeX\relax
}
%    \end{macrocode}
%    \end{macro}
%
% \subsubsection{\hologo{teTeX}}
%
%    \begin{macro}{\HoLogo@teTeX}
%    \begin{macrocode}
\def\HoLogo@teTeX#1{%
  \HOLOGO@mbox{#1{t}{T}e}%
  \HOLOGO@discretionary
  \hologo{TeX}%
}
%    \end{macrocode}
%    \end{macro}
%    \begin{macro}{\HoLogoCs@teTeX}
%    \begin{macrocode}
\def\HoLogoCs@teTeX#1{#1{t}{T}dfTeX}
%    \end{macrocode}
%    \end{macro}
%    \begin{macro}{\HoLogoBkm@teTeX}
%    \begin{macrocode}
\def\HoLogoBkm@teTeX#1{%
  #1{t}{T}e\hologo{TeX}%
}
%    \end{macrocode}
%    \end{macro}
%    \begin{macro}{\HoLogoHtml@teTeX}
%    \begin{macrocode}
\let\HoLogoHtml@teTeX\HoLogo@teTeX
%    \end{macrocode}
%    \end{macro}
%
% \subsubsection{\hologo{TeX4ht}}
%
%    \begin{macro}{\HoLogo@TeX4ht}
%    \begin{macrocode}
\expandafter\def\csname HoLogo@TeX4ht\endcsname#1{%
  \HOLOGO@mbox{\hologo{TeX}4ht}%
}
%    \end{macrocode}
%    \end{macro}
%    \begin{macro}{\HoLogoHtml@TeX4ht}
%    \begin{macrocode}
\expandafter
\let\csname HoLogoHtml@TeX4ht\expandafter\endcsname
\csname HoLogo@TeX4ht\endcsname
%    \end{macrocode}
%    \end{macro}
%
%
% \subsubsection{\hologo{SageTeX}}
%
%    \begin{macro}{\HoLogo@SageTeX}
%    \begin{macrocode}
\def\HoLogo@SageTeX#1{%
  \HOLOGO@mbox{Sage}%
  \HOLOGO@discretionary
  \HOLOGO@NegativeKerning{eT,oT,To}%
  \hologo{TeX}%
}
%    \end{macrocode}
%    \end{macro}
%    \begin{macro}{\HoLogoHtml@SageTeX}
%    \begin{macrocode}
\let\HoLogoHtml@SageTeX\HoLogo@SageTeX
%    \end{macrocode}
%    \end{macro}
%
% \subsection{\hologo{METAFONT} and friends}
%
%    \begin{macro}{\HoLogo@METAFONT}
%    \begin{macrocode}
\def\HoLogo@METAFONT#1{%
  \HoLogoFont@font{METAFONT}{logo}{%
    \HOLOGO@mbox{META}%
    \HOLOGO@discretionary
    \HOLOGO@mbox{FONT}%
  }%
}
%    \end{macrocode}
%    \end{macro}
%
%    \begin{macro}{\HoLogo@METAPOST}
%    \begin{macrocode}
\def\HoLogo@METAPOST#1{%
  \HoLogoFont@font{METAPOST}{logo}{%
    \HOLOGO@mbox{META}%
    \HOLOGO@discretionary
    \HOLOGO@mbox{POST}%
  }%
}
%    \end{macrocode}
%    \end{macro}
%
%    \begin{macro}{\HoLogo@MetaFun}
%    \begin{macrocode}
\def\HoLogo@MetaFun#1{%
  \HOLOGO@mbox{Meta}%
  \HOLOGO@discretionary
  \HOLOGO@mbox{Fun}%
}
%    \end{macrocode}
%    \end{macro}
%
%    \begin{macro}{\HoLogo@MetaPost}
%    \begin{macrocode}
\def\HoLogo@MetaPost#1{%
  \HOLOGO@mbox{Meta}%
  \HOLOGO@discretionary
  \HOLOGO@mbox{Post}%
}
%    \end{macrocode}
%    \end{macro}
%
% \subsection{Others}
%
% \subsubsection{\hologo{biber}}
%
%    \begin{macro}{\HoLogo@biber}
%    \begin{macrocode}
\def\HoLogo@biber#1{%
  \HOLOGO@mbox{#1{b}{B}i}%
  \HOLOGO@discretionary
  \HOLOGO@mbox{ber}%
}
%    \end{macrocode}
%    \end{macro}
%    \begin{macro}{\HoLogoCs@biber}
%    \begin{macrocode}
\def\HoLogoCs@biber#1{#1{b}{B}iber}
%    \end{macrocode}
%    \end{macro}
%    \begin{macro}{\HoLogoBkm@biber}
%    \begin{macrocode}
\def\HoLogoBkm@biber#1{%
  #1{b}{B}iber%
}
%    \end{macrocode}
%    \end{macro}
%    \begin{macro}{\HoLogoHtml@biber}
%    \begin{macrocode}
\let\HoLogoHtml@biber\HoLogo@biber
%    \end{macrocode}
%    \end{macro}
%
% \subsubsection{\hologo{KOMAScript}}
%
%    \begin{macro}{\HoLogo@KOMAScript}
%    The definition for \hologo{KOMAScript} is taken
%    from \hologo{KOMAScript} (\xfile{scrlogo.dtx}, reformatted) \cite{scrlogo}:
%\begin{quote}
%\begin{verbatim}
%\@ifundefined{KOMAScript}{%
%  \DeclareRobustCommand{\KOMAScript}{%
%    \textsf{%
%      K\kern.05em O\kern.05emM\kern.05em A%
%      \kern.1em-\kern.1em %
%      Script%
%    }%
%  }%
%}{}
%\end{verbatim}
%\end{quote}
%    \begin{macrocode}
\def\HoLogo@KOMAScript#1{%
  \HoLogoFont@font{KOMAScript}{sf}{%
    \HOLOGO@mbox{%
      K\kern.05em%
      O\kern.05em%
      M\kern.05em%
      A%
    }%
    \kern.1em%
    \HOLOGO@hyphen
    \kern.1em%
    \HOLOGO@mbox{Script}%
  }%
}
%    \end{macrocode}
%    \end{macro}
%    \begin{macro}{\HoLogoBkm@KOMAScript}
%    \begin{macrocode}
\def\HoLogoBkm@KOMAScript#1{%
  KOMA-Script%
}
%    \end{macrocode}
%    \end{macro}
%    \begin{macro}{\HoLogoHtml@KOMAScript}
%    \begin{macrocode}
\def\HoLogoHtml@KOMAScript#1{%
  \HoLogoCss@KOMAScript
  \HoLogoFont@font{KOMAScript}{sf}{%
    \HOLOGO@Span{KOMAScript}{%
      K%
      \HOLOGO@Span{O}{O}%
      M%
      \HOLOGO@Span{A}{A}%
      \HOLOGO@Span{hyphen}{-}%
      Script%
    }%
  }%
}
%    \end{macrocode}
%    \end{macro}
%    \begin{macro}{\HoLogoCss@KOMAScript}
%    \begin{macrocode}
\def\HoLogoCss@KOMAScript{%
  \Css{%
    span.HoLogo-KOMAScript{%
      font-family:sans-serif;%
    }%
  }%
  \Css{%
    span.HoLogo-KOMAScript span.HoLogo-O{%
      padding-left:.05em;%
      padding-right:.05em;%
    }%
  }%
  \Css{%
    span.HoLogo-KOMAScript span.HoLogo-A{%
      padding-left:.05em;%
    }%
  }%
  \Css{%
    span.HoLogo-KOMAScript span.HoLogo-hyphen{%
      padding-left:.1em;%
      padding-right:.1em;%
    }%
  }%
  \global\let\HoLogoCss@KOMAScript\relax
}
%    \end{macrocode}
%    \end{macro}
%
% \subsubsection{\hologo{LyX}}
%
%    \begin{macro}{\HoLogo@LyX}
%    The definition is taken from the documentation source files
%    of \hologo{LyX}, \xfile{Intro.lyx} \cite{LyX}:
%\begin{quote}
%\begin{verbatim}
%\def\LyX{%
%  \texorpdfstring{%
%    L\kern-.1667em\lower.25em\hbox{Y}\kern-.125emX\@%
%  }{%
%    LyX%
%  }%
%}
%\end{verbatim}
%\end{quote}
%    \begin{macrocode}
\def\HoLogo@LyX#1{%
  L%
  \kern-.1667em%
  \lower.25em\hbox{Y}%
  \kern-.125em%
  X%
  \HOLOGO@SpaceFactor
}
%    \end{macrocode}
%    \end{macro}
%    \begin{macro}{\HoLogoHtml@LyX}
%    \begin{macrocode}
\def\HoLogoHtml@LyX#1{%
  \HoLogoCss@LyX
  \HOLOGO@Span{LyX}{%
    L%
    \HOLOGO@Span{y}{Y}%
    X%
  }%
}
%    \end{macrocode}
%    \end{macro}
%    \begin{macro}{\HoLogoCss@LyX}
%    \begin{macrocode}
\def\HoLogoCss@LyX{%
  \Css{%
    span.HoLogo-LyX span.HoLogo-y{%
      position:relative;%
      top:.25em;%
      margin-left:-.1667em;%
      margin-right:-.125em;%
      text-decoration:none;%
    }%
  }%
  \global\let\HoLogoCss@LyX\relax
}
%    \end{macrocode}
%    \end{macro}
%
% \subsubsection{\hologo{NTS}}
%
%    \begin{macro}{\HoLogo@NTS}
%    Definition for \hologo{NTS} can be found in
%    package \xpackage{etex\textunderscore man} for the \hologo{eTeX} manual \cite{etexman}
%    and in package \xpackage{dtklogos} \cite{dtklogos}:
%\begin{quote}
%\begin{verbatim}
%\def\NTS{%
%  \leavevmode
%  \hbox{%
%    $%
%      \cal N%
%      \kern-0.35em%
%      \lower0.5ex\hbox{$\cal T$}%
%      \kern-0.2em%
%      S%
%    $%
%  }%
%}
%\end{verbatim}
%\end{quote}
%    \begin{macrocode}
\def\HoLogo@NTS#1{%
  \HoLogoFont@font{NTS}{sy}{%
    N\/%
    \kern-.35em%
    \lower.5ex\hbox{T\/}%
    \kern-.2em%
    S\/%
  }%
  \HOLOGO@SpaceFactor
}
%    \end{macrocode}
%    \end{macro}
%
% \subsubsection{\Hologo{TTH} (\hologo{TeX} to HTML translator)}
%
%    Source: \url{http://hutchinson.belmont.ma.us/tth/}
%    In the HTML source the second `T' is printed as subscript.
%\begin{quote}
%\begin{verbatim}
%T<sub>T</sub>H
%\end{verbatim}
%\end{quote}
%    \begin{macro}{\HoLogo@TTH}
%    \begin{macrocode}
\def\HoLogo@TTH#1{%
  \ltx@mbox{%
    T\HOLOGO@SubScript{T}H%
  }%
  \HOLOGO@SpaceFactor
}
%    \end{macrocode}
%    \end{macro}
%
%    \begin{macro}{\HoLogoHtml@TTH}
%    \begin{macrocode}
\def\HoLogoHtml@TTH#1{%
  T\HCode{<sub>}T\HCode{</sub>}H%
}
%    \end{macrocode}
%    \end{macro}
%
% \subsubsection{\Hologo{HanTheThanh}}
%
%    Partial source: Package \xpackage{dtklogos}.
%    The double accent is U+1EBF (latin small letter e with circumflex
%    and acute).
%    \begin{macro}{\HoLogo@HanTheThanh}
%    \begin{macrocode}
\def\HoLogo@HanTheThanh#1{%
  \ltx@mbox{H\`an}%
  \HOLOGO@space
  \ltx@mbox{%
    Th%
    \HOLOGO@IfCharExists{"1EBF}{%
      \char"1EBF\relax
    }{%
      \^e\hbox to 0pt{\hss\raise .5ex\hbox{\'{}}}%
    }%
  }%
  \HOLOGO@space
  \ltx@mbox{Th\`anh}%
}
%    \end{macrocode}
%    \end{macro}
%    \begin{macro}{\HoLogoBkm@HanTheThanh}
%    \begin{macrocode}
\def\HoLogoBkm@HanTheThanh#1{%
  H\`an %
  Th\HOLOGO@PdfdocUnicode{\^e}{\9036\277} %
  Th\`anh%
}
%    \end{macrocode}
%    \end{macro}
%    \begin{macro}{\HoLogoHtml@HanTheThanh}
%    \begin{macrocode}
\def\HoLogoHtml@HanTheThanh#1{%
  H\`an %
  Th\HCode{&\ltx@hashchar x1ebf;} %
  Th\`anh%
}
%    \end{macrocode}
%    \end{macro}
%
% \subsection{Driver detection}
%
%    \begin{macrocode}
\HOLOGO@IfExists\InputIfFileExists{%
  \InputIfFileExists{hologo.cfg}{}{}%
}{%
  \ltx@IfUndefined{pdf@filesize}{%
    \def\HOLOGO@InputIfExists{%
      \openin\HOLOGO@temp=hologo.cfg\relax
      \ifeof\HOLOGO@temp
        \closein\HOLOGO@temp
      \else
        \closein\HOLOGO@temp
        \begingroup
          \def\x{LaTeX2e}%
        \expandafter\endgroup
        \ifx\fmtname\x
          \input{hologo.cfg}%
        \else
          \input hologo.cfg\relax
        \fi
      \fi
    }%
    \ltx@IfUndefined{newread}{%
      \chardef\HOLOGO@temp=15 %
      \def\HOLOGO@CheckRead{%
        \ifeof\HOLOGO@temp
          \HOLOGO@InputIfExists
        \else
          \ifcase\HOLOGO@temp
            \@PackageWarningNoLine{hologo}{%
              Configuration file ignored, because\MessageBreak
              a free read register could not be found%
            }%
          \else
            \begingroup
              \count\ltx@cclv=\HOLOGO@temp
              \advance\ltx@cclv by \ltx@minusone
              \edef\x{\endgroup
                \chardef\noexpand\HOLOGO@temp=\the\count\ltx@cclv
                \relax
              }%
            \x
          \fi
        \fi
      }%
    }{%
      \csname newread\endcsname\HOLOGO@temp
      \HOLOGO@InputIfExists
    }%
  }{%
    \edef\HOLOGO@temp{\pdf@filesize{hologo.cfg}}%
    \ifx\HOLOGO@temp\ltx@empty
    \else
      \ifnum\HOLOGO@temp>0 %
        \begingroup
          \def\x{LaTeX2e}%
        \expandafter\endgroup
        \ifx\fmtname\x
          \input{hologo.cfg}%
        \else
          \input hologo.cfg\relax
        \fi
      \else
        \@PackageInfoNoLine{hologo}{%
          Empty configuration file `hologo.cfg' ignored%
        }%
      \fi
    \fi
  }%
}
%    \end{macrocode}
%
%    \begin{macrocode}
\def\HOLOGO@temp#1#2{%
  \kv@define@key{HoLogoDriver}{#1}[]{%
    \begingroup
      \def\HOLOGO@temp{##1}%
      \ltx@onelevel@sanitize\HOLOGO@temp
      \ifx\HOLOGO@temp\ltx@empty
      \else
        \@PackageError{hologo}{%
          Value (\HOLOGO@temp) not permitted for option `#1'%
        }%
        \@ehc
      \fi
    \endgroup
    \def\hologoDriver{#2}%
  }%
}%
\def\HOLOGO@@temp#1#2{%
  \ifx\kv@value\relax
    \HOLOGO@temp{#1}{#1}%
  \else
    \HOLOGO@temp{#1}{#2}%
  \fi
}%
\kv@parse@normalized{%
  pdftex,%
  luatex=pdftex,%
  dvipdfm,%
  dvipdfmx=dvipdfm,%
  dvips,%
  dvipsone=dvips,%
  xdvi=dvips,%
  xetex,%
  vtex,%
}\HOLOGO@@temp
%    \end{macrocode}
%
%    \begin{macrocode}
\kv@define@key{HoLogoDriver}{driverfallback}{%
  \def\HOLOGO@DriverFallback{#1}%
}
%    \end{macrocode}
%
%    \begin{macro}{\HOLOGO@DriverFallback}
%    \begin{macrocode}
\def\HOLOGO@DriverFallback{dvips}
%    \end{macrocode}
%    \end{macro}
%
%    \begin{macro}{\hologoDriverSetup}
%    \begin{macrocode}
\def\hologoDriverSetup{%
  \let\hologoDriver\ltx@undefined
  \HOLOGO@DriverSetup
}
%    \end{macrocode}
%    \end{macro}
%
%    \begin{macro}{\HOLOGO@DriverSetup}
%    \begin{macrocode}
\def\HOLOGO@DriverSetup#1{%
  \kvsetkeys{HoLogoDriver}{#1}%
  \HOLOGO@CheckDriver
  \ltx@ifundefined{hologoDriver}{%
    \begingroup
    \edef\x{\endgroup
      \noexpand\kvsetkeys{HoLogoDriver}{\HOLOGO@DriverFallback}%
    }\x
  }{}%
  \@PackageInfoNoLine{hologo}{Using driver `\hologoDriver'}%
}
%    \end{macrocode}
%    \end{macro}
%
%    \begin{macro}{\HOLOGO@CheckDriver}
%    \begin{macrocode}
\def\HOLOGO@CheckDriver{%
  \ifpdf
    \def\hologoDriver{pdftex}%
    \let\HOLOGO@pdfliteral\pdfliteral
    \ifluatex
      \ifx\pdfextension\@undefined\else
        \protected\def\pdfliteral{\pdfextension literal}%
        \let\HOLOGO@pdfliteral\pdfliteral
      \fi
      \ltx@IfUndefined{HOLOGO@pdfliteral}{%
        \ifnum\luatexversion<36 %
        \else
          \begingroup
            \let\HOLOGO@temp\endgroup
            \ifcase0%
                \directlua{%
                  if tex.enableprimitives then %
                    tex.enableprimitives('HOLOGO@', {'pdfliteral'})%
                  else %
                    tex.print('1')%
                  end%
                }%
                \ifx\HOLOGO@pdfliteral\@undefined 1\fi%
                \relax%
              \endgroup
              \let\HOLOGO@temp\relax
              \global\let\HOLOGO@pdfliteral\HOLOGO@pdfliteral
            \fi%
          \HOLOGO@temp
        \fi
      }{}%
    \fi
    \ltx@IfUndefined{HOLOGO@pdfliteral}{%
      \@PackageWarningNoLine{hologo}{%
        Cannot find \string\pdfliteral
      }%
    }{}%
  \else
    \ifxetex
      \def\hologoDriver{xetex}%
    \else
      \ifvtex
        \def\hologoDriver{vtex}%
      \fi
    \fi
  \fi
}
%    \end{macrocode}
%    \end{macro}
%
%    \begin{macro}{\HOLOGO@WarningUnsupportedDriver}
%    \begin{macrocode}
\def\HOLOGO@WarningUnsupportedDriver#1{%
  \@PackageWarningNoLine{hologo}{%
    Logo `#1' needs driver specific macros,\MessageBreak
    but driver `\hologoDriver' is not supported.\MessageBreak
    Use a different driver or\MessageBreak
    load package `graphics' or `pgf'%
  }%
}
%    \end{macrocode}
%    \end{macro}
%
% \subsubsection{Reflect box macros}
%
%    Skip driver part if not needed.
%    \begin{macrocode}
\ltx@IfUndefined{reflectbox}{}{%
  \ltx@IfUndefined{rotatebox}{}{%
    \HOLOGO@AtEnd
  }%
}
\ltx@IfUndefined{pgftext}{}{%
  \HOLOGO@AtEnd
}
\ltx@IfUndefined{psscalebox}{}{%
  \HOLOGO@AtEnd
}
%    \end{macrocode}
%
%    \begin{macrocode}
\def\HOLOGO@temp{LaTeX2e}
\ifx\fmtname\HOLOGO@temp
  \RequirePackage{kvoptions}[2011/06/30]%
  \ProcessKeyvalOptions{HoLogoDriver}%
\fi
\HOLOGO@DriverSetup{}
%    \end{macrocode}
%
%    \begin{macro}{\HOLOGO@ReflectBox}
%    \begin{macrocode}
\def\HOLOGO@ReflectBox#1{%
  \begingroup
    \setbox\ltx@zero\hbox{\begingroup#1\endgroup}%
    \setbox\ltx@two\hbox{%
      \kern\wd\ltx@zero
      \csname HOLOGO@ScaleBox@\hologoDriver\endcsname{-1}{1}{%
        \hbox to 0pt{\copy\ltx@zero\hss}%
      }%
    }%
    \wd\ltx@two=\wd\ltx@zero
    \box\ltx@two
  \endgroup
}
%    \end{macrocode}
%    \end{macro}
%
%    \begin{macro}{\HOLOGO@PointReflectBox}
%    \begin{macrocode}
\def\HOLOGO@PointReflectBox#1{%
  \begingroup
    \setbox\ltx@zero\hbox{\begingroup#1\endgroup}%
    \setbox\ltx@two\hbox{%
      \kern\wd\ltx@zero
      \raise\ht\ltx@zero\hbox{%
        \csname HOLOGO@ScaleBox@\hologoDriver\endcsname{-1}{-1}{%
          \hbox to 0pt{\copy\ltx@zero\hss}%
        }%
      }%
    }%
    \wd\ltx@two=\wd\ltx@zero
    \box\ltx@two
  \endgroup
}
%    \end{macrocode}
%    \end{macro}
%
%    We must define all variants because of dynamic driver setup.
%    \begin{macrocode}
\def\HOLOGO@temp#1#2{#2}
%    \end{macrocode}
%
%    \begin{macro}{\HOLOGO@ScaleBox@pdftex}
%    \begin{macrocode}
\HOLOGO@temp{pdftex}{%
  \def\HOLOGO@ScaleBox@pdftex#1#2#3{%
    \HOLOGO@pdfliteral{%
      q #1 0 0 #2 0 0 cm%
    }%
    #3%
    \HOLOGO@pdfliteral{%
      Q%
    }%
  }%
}
%    \end{macrocode}
%    \end{macro}
%    \begin{macro}{\HOLOGO@ScaleBox@dvips}
%    \begin{macrocode}
\HOLOGO@temp{dvips}{%
  \def\HOLOGO@ScaleBox@dvips#1#2#3{%
    \special{ps:%
      gsave %
      currentpoint %
      currentpoint translate %
      #1 #2 scale %
      neg exch neg exch translate%
    }%
    #3%
    \special{ps:%
      currentpoint %
      grestore %
      moveto%
    }%
  }%
}
%    \end{macrocode}
%    \end{macro}
%    \begin{macro}{\HOLOGO@ScaleBox@dvipdfm}
%    \begin{macrocode}
\HOLOGO@temp{dvipdfm}{%
  \let\HOLOGO@ScaleBox@dvipdfm\HOLOGO@ScaleBox@dvips
}
%    \end{macrocode}
%    \end{macro}
%    Since \hologo{XeTeX} v0.6.
%    \begin{macro}{\HOLOGO@ScaleBox@xetex}
%    \begin{macrocode}
\HOLOGO@temp{xetex}{%
  \def\HOLOGO@ScaleBox@xetex#1#2#3{%
    \special{x:gsave}%
    \special{x:scale #1 #2}%
    #3%
    \special{x:grestore}%
  }%
}
%    \end{macrocode}
%    \end{macro}
%    \begin{macro}{\HOLOGO@ScaleBox@vtex}
%    \begin{macrocode}
\HOLOGO@temp{vtex}{%
  \def\HOLOGO@ScaleBox@vtex#1#2#3{%
    \special{r(#1,0,0,#2,0,0}%
    #3%
    \special{r)}%
  }%
}
%    \end{macrocode}
%    \end{macro}
%
%    \begin{macrocode}
\HOLOGO@AtEnd%
%</package>
%    \end{macrocode}
%
% \section{Test}
%
% \subsection{Catcode checks for loading}
%
%    \begin{macrocode}
%<*test1>
%    \end{macrocode}
%    \begin{macrocode}
\catcode`\{=1 %
\catcode`\}=2 %
\catcode`\#=6 %
\catcode`\@=11 %
\expandafter\ifx\csname count@\endcsname\relax
  \countdef\count@=255 %
\fi
\expandafter\ifx\csname @gobble\endcsname\relax
  \long\def\@gobble#1{}%
\fi
\expandafter\ifx\csname @firstofone\endcsname\relax
  \long\def\@firstofone#1{#1}%
\fi
\expandafter\ifx\csname loop\endcsname\relax
  \expandafter\@firstofone
\else
  \expandafter\@gobble
\fi
{%
  \def\loop#1\repeat{%
    \def\body{#1}%
    \iterate
  }%
  \def\iterate{%
    \body
      \let\next\iterate
    \else
      \let\next\relax
    \fi
    \next
  }%
  \let\repeat=\fi
}%
\def\RestoreCatcodes{}
\count@=0 %
\loop
  \edef\RestoreCatcodes{%
    \RestoreCatcodes
    \catcode\the\count@=\the\catcode\count@\relax
  }%
\ifnum\count@<255 %
  \advance\count@ 1 %
\repeat

\def\RangeCatcodeInvalid#1#2{%
  \count@=#1\relax
  \loop
    \catcode\count@=15 %
  \ifnum\count@<#2\relax
    \advance\count@ 1 %
  \repeat
}
\def\RangeCatcodeCheck#1#2#3{%
  \count@=#1\relax
  \loop
    \ifnum#3=\catcode\count@
    \else
      \errmessage{%
        Character \the\count@\space
        with wrong catcode \the\catcode\count@\space
        instead of \number#3%
      }%
    \fi
  \ifnum\count@<#2\relax
    \advance\count@ 1 %
  \repeat
}
\def\space{ }
\expandafter\ifx\csname LoadCommand\endcsname\relax
  \def\LoadCommand{\input hologo.sty\relax}%
\fi
\def\Test{%
  \RangeCatcodeInvalid{0}{47}%
  \RangeCatcodeInvalid{58}{64}%
  \RangeCatcodeInvalid{91}{96}%
  \RangeCatcodeInvalid{123}{255}%
  \catcode`\@=12 %
  \catcode`\\=0 %
  \catcode`\%=14 %
  \LoadCommand
  \RangeCatcodeCheck{0}{36}{15}%
  \RangeCatcodeCheck{37}{37}{14}%
  \RangeCatcodeCheck{38}{47}{15}%
  \RangeCatcodeCheck{48}{57}{12}%
  \RangeCatcodeCheck{58}{63}{15}%
  \RangeCatcodeCheck{64}{64}{12}%
  \RangeCatcodeCheck{65}{90}{11}%
  \RangeCatcodeCheck{91}{91}{15}%
  \RangeCatcodeCheck{92}{92}{0}%
  \RangeCatcodeCheck{93}{96}{15}%
  \RangeCatcodeCheck{97}{122}{11}%
  \RangeCatcodeCheck{123}{255}{15}%
  \RestoreCatcodes
}
\Test
\csname @@end\endcsname
\end
%    \end{macrocode}
%    \begin{macrocode}
%</test1>
%    \end{macrocode}
%
% \subsection{Spacefactor}
%
%    The space factor must be 1000 after a logo. If it is greater 1000
%    then the following space is a space after a sentence closing point.
%    If the space factor is smaller 1000 then an immediate following
%    dot is interpreted as abbreviation, not sentence closing point.
%
%    \begin{macrocode}
%<*test-spacefactor>
\NeedsTeXFormat{LaTeX2e}
\documentclass{article}
\usepackage{hologo}[2016/05/12]
\usepackage{kvsetkeys}
\usepackage{qstest}
\IncludeTests{*}
\LogTests{log}{*}{*}
\begin{document}
\begin{qstest}{spacefactor}{spacefactor}
\newcommand*{\Test}[1]{%
  \sbox0{%
    \hologo{#1}%
    \Expect*{1000 (#1)}*{\the\spacefactor\space(#1)}%
  }%
}%
\makeatletter
\def\TestList{}
\def\hologoEntry#1#2#3{%
  \edef\TestList{%
    \ifx\TestList\@empty
    \else
      \TestList,%
    \fi
    #1%
    \ifx\\#2\\%
    \else
      ={variant=#2}%
    \fi
  }%
}
\hologoList
\expandafter\kv@parse@normalized\expandafter{%
  \TestList
}{%
  \begingroup
    \let\@logo=\kv@key
    \ifx\kv@value\relax
    \else
      \expandafter\hologoLogoSetup\expandafter\@logo\expandafter{%
        \kv@value
      }%
    \fi
    \Test\@logo
  \endgroup
  \@gobbletwo
}
\end{qstest}
\end{document}
%</test-spacefactor>
%    \end{macrocode}
%
% \subsection{Complete list}
%
%    \begin{macrocode}
%<*test-list>
\NeedsTeXFormat{LaTeX2e}
\documentclass[12pt,a4paper]{article}
\usepackage{hologo}[2016/05/12]
\usepackage[T1]{fontenc}
\usepackage{lmodern}
\usepackage{parskip}
\usepackage[unicode]{hyperref}[2011/09/28]
\usepackage{bookmark}[2011/09/19]
\bookmarksetup{%
  numbered,%
  open,%
  openlevel=2,%
}
\renewcommand*{\contentsname}{List of logos}
\begin{document}
\tableofcontents
\def\TestFont#1#2#3#4#5#6{%
  \begingroup
    \usefont{#3}{#4}{#5}{#6}%
    \HologoVariant{#1}{#2}/\hologoVariant{#1}{#2}%
    \quad
    \begingroup\scriptsize\hologoVariant{#1}{#2}\endgroup
    \quad
  \endgroup
  (#3/#4/#5/#6)%
  \par
}
\makeatletter
\def\hologoEntry#1#2#3{%
  \section{%
    \HologoVariant{#1}{#2}/\hologoVariant{#1}{#2} %
    {[#1\ifx\\#2\\\else\space(#2)\fi]}% hash-ok
  }% braces around [] because of bug in tex4ht
  \begingroup
    \hypersetup{unicode=false}%
    \bookmark[%
      dest=\@currentHref,%
      rellevel=1,%
      keeplevel,%
    ]{%
      \HologoVariant{#1}{#2}/\hologoVariant{#1}{#2} %
      (PDFDocEncoding)%
    }%
  \endgroup
  \TestFont{#1}{#2}{OT1}{cmr}{m}{n}%
  \TestFont{#1}{#2}{OT1}{cmss}{m}{n}%
  \TestFont{#1}{#2}{OT1}{cmr}{b}{n}%
  \TestFont{#1}{#2}{OT1}{cmr}{m}{it}%
  \TestFont{#1}{#2}{OT1}{cmtt}{m}{n}%
  \TestFont{#1}{#2}{T1}{lmr}{m}{n}%
  \TestFont{#1}{#2}{T1}{lmss}{m}{n}%
  \TestFont{#1}{#2}{T1}{lmr}{b}{n}%
  \TestFont{#1}{#2}{T1}{lmr}{m}{it}%
  \TestFont{#1}{#2}{T1}{lmtt}{m}{n}%
  \TestFont{#1}{#2}{T1}{lmvtt}{m}{n}%
  \TestFont{#1}{#2}{T1}{qtm}{m}{n}%
  \TestFont{#1}{#2}{T1}{qhv}{m}{n}%
  \TestFont{#1}{#2}{T1}{qtm}{b}{n}%
  \TestFont{#1}{#2}{T1}{qtm}{m}{it}%
  \TestFont{#1}{#2}{T1}{qcr}{m}{n}%
  \newpage
}
\makeatother
\hologoList
\end{document}
%</test-list>
%    \end{macrocode}
%
% \section{Installation}
%
% \subsection{Download}
%
% \paragraph{Package.} This package is available on
% CTAN\footnote{\url{ftp://ftp.ctan.org/tex-archive/}}:
% \begin{description}
% \item[\CTAN{macros/latex/contrib/oberdiek/hologo.dtx}] The source file.
% \item[\CTAN{macros/latex/contrib/oberdiek/hologo.pdf}] Documentation.
% \end{description}
%
%
% \paragraph{Bundle.} All the packages of the bundle `oberdiek'
% are also available in a TDS compliant ZIP archive. There
% the packages are already unpacked and the documentation files
% are generated. The files and directories obey the TDS standard.
% \begin{description}
% \item[\CTAN{install/macros/latex/contrib/oberdiek.tds.zip}]
% \end{description}
% \emph{TDS} refers to the standard ``A Directory Structure
% for \TeX\ Files'' (\CTAN{tds/tds.pdf}). Directories
% with \xfile{texmf} in their name are usually organized this way.
%
% \subsection{Bundle installation}
%
% \paragraph{Unpacking.} Unpack the \xfile{oberdiek.tds.zip} in the
% TDS tree (also known as \xfile{texmf} tree) of your choice.
% Example (linux):
% \begin{quote}
%   |unzip oberdiek.tds.zip -d ~/texmf|
% \end{quote}
%
% \paragraph{Script installation.}
% Check the directory \xfile{TDS:scripts/oberdiek/} for
% scripts that need further installation steps.
% Package \xpackage{attachfile2} comes with the Perl script
% \xfile{pdfatfi.pl} that should be installed in such a way
% that it can be called as \texttt{pdfatfi}.
% Example (linux):
% \begin{quote}
%   |chmod +x scripts/oberdiek/pdfatfi.pl|\\
%   |cp scripts/oberdiek/pdfatfi.pl /usr/local/bin/|
% \end{quote}
%
% \subsection{Package installation}
%
% \paragraph{Unpacking.} The \xfile{.dtx} file is a self-extracting
% \docstrip\ archive. The files are extracted by running the
% \xfile{.dtx} through \plainTeX:
% \begin{quote}
%   \verb|tex hologo.dtx|
% \end{quote}
%
% \paragraph{TDS.} Now the different files must be moved into
% the different directories in your installation TDS tree
% (also known as \xfile{texmf} tree):
% \begin{quote}
% \def\t{^^A
% \begin{tabular}{@{}>{\ttfamily}l@{ $\rightarrow$ }>{\ttfamily}l@{}}
%   hologo.sty & tex/generic/oberdiek/hologo.sty\\
%   hologo.pdf & doc/latex/oberdiek/hologo.pdf\\
%   example/hologo-example.tex & doc/latex/oberdiek/example/hologo-example.tex\\
%   test/hologo-test1.tex & doc/latex/oberdiek/test/hologo-test1.tex\\
%   test/hologo-test-spacefactor.tex & doc/latex/oberdiek/test/hologo-test-spacefactor.tex\\
%   test/hologo-test-list.tex & doc/latex/oberdiek/test/hologo-test-list.tex\\
%   hologo.dtx & source/latex/oberdiek/hologo.dtx\\
% \end{tabular}^^A
% }^^A
% \sbox0{\t}^^A
% \ifdim\wd0>\linewidth
%   \begingroup
%     \advance\linewidth by\leftmargin
%     \advance\linewidth by\rightmargin
%   \edef\x{\endgroup
%     \def\noexpand\lw{\the\linewidth}^^A
%   }\x
%   \def\lwbox{^^A
%     \leavevmode
%     \hbox to \linewidth{^^A
%       \kern-\leftmargin\relax
%       \hss
%       \usebox0
%       \hss
%       \kern-\rightmargin\relax
%     }^^A
%   }^^A
%   \ifdim\wd0>\lw
%     \sbox0{\small\t}^^A
%     \ifdim\wd0>\linewidth
%       \ifdim\wd0>\lw
%         \sbox0{\footnotesize\t}^^A
%         \ifdim\wd0>\linewidth
%           \ifdim\wd0>\lw
%             \sbox0{\scriptsize\t}^^A
%             \ifdim\wd0>\linewidth
%               \ifdim\wd0>\lw
%                 \sbox0{\tiny\t}^^A
%                 \ifdim\wd0>\linewidth
%                   \lwbox
%                 \else
%                   \usebox0
%                 \fi
%               \else
%                 \lwbox
%               \fi
%             \else
%               \usebox0
%             \fi
%           \else
%             \lwbox
%           \fi
%         \else
%           \usebox0
%         \fi
%       \else
%         \lwbox
%       \fi
%     \else
%       \usebox0
%     \fi
%   \else
%     \lwbox
%   \fi
% \else
%   \usebox0
% \fi
% \end{quote}
% If you have a \xfile{docstrip.cfg} that configures and enables \docstrip's
% TDS installing feature, then some files can already be in the right
% place, see the documentation of \docstrip.
%
% \subsection{Refresh file name databases}
%
% If your \TeX~distribution
% (\teTeX, \mikTeX, \dots) relies on file name databases, you must refresh
% these. For example, \teTeX\ users run \verb|texhash| or
% \verb|mktexlsr|.
%
% \subsection{Some details for the interested}
%
% \paragraph{Attached source.}
%
% The PDF documentation on CTAN also includes the
% \xfile{.dtx} source file. It can be extracted by
% AcrobatReader 6 or higher. Another option is \textsf{pdftk},
% e.g. unpack the file into the current directory:
% \begin{quote}
%   \verb|pdftk hologo.pdf unpack_files output .|
% \end{quote}
%
% \paragraph{Unpacking with \LaTeX.}
% The \xfile{.dtx} chooses its action depending on the format:
% \begin{description}
% \item[\plainTeX:] Run \docstrip\ and extract the files.
% \item[\LaTeX:] Generate the documentation.
% \end{description}
% If you insist on using \LaTeX\ for \docstrip\ (really,
% \docstrip\ does not need \LaTeX), then inform the autodetect routine
% about your intention:
% \begin{quote}
%   \verb|latex \let\install=y\input{hologo.dtx}|
% \end{quote}
% Do not forget to quote the argument according to the demands
% of your shell.
%
% \paragraph{Generating the documentation.}
% You can use both the \xfile{.dtx} or the \xfile{.drv} to generate
% the documentation. The process can be configured by the
% configuration file \xfile{ltxdoc.cfg}. For instance, put this
% line into this file, if you want to have A4 as paper format:
% \begin{quote}
%   \verb|\PassOptionsToClass{a4paper}{article}|
% \end{quote}
% An example follows how to generate the
% documentation with pdf\LaTeX:
% \begin{quote}
%\begin{verbatim}
%pdflatex hologo.dtx
%makeindex -s gind.ist hologo.idx
%pdflatex hologo.dtx
%makeindex -s gind.ist hologo.idx
%pdflatex hologo.dtx
%\end{verbatim}
% \end{quote}
%
% \section{Catalogue}
%
% The following XML file can be used as source for the
% \href{http://mirror.ctan.org/help/Catalogue/catalogue.html}{\TeX\ Catalogue}.
% The elements \texttt{caption} and \texttt{description} are imported
% from the original XML file from the Catalogue.
% The name of the XML file in the Catalogue is \xfile{hologo.xml}.
%    \begin{macrocode}
%<*catalogue>
<?xml version='1.0' encoding='us-ascii'?>
<!DOCTYPE entry SYSTEM 'catalogue.dtd'>
<entry datestamp='$Date$' modifier='$Author$' id='hologo'>
  <name>hologo</name>
  <caption>A collection of logos with bookmark support.</caption>
  <authorref id='auth:oberdiek'/>
  <copyright owner='Heiko Oberdiek' year='2010-2012'/>
  <license type='lppl1.3'/>
  <version number='1.10'/>
  <description>
    The package defines a single command <tt>\hologo</tt>, whose
    argument is the usual case-confused ASCII version of the logo.
    The command is bookmark-enabled, so that every logo becomes
    available in bookmarks without further work.
    <p/>
    The package is part of the <xref refid='oberdiek'>oberdiek</xref>
    bundle.
  </description>
  <documentation details='Package documentation'
      href='ctan:/macros/latex/contrib/oberdiek/hologo.pdf'/>
  <ctan file='true' path='/macros/latex/contrib/oberdiek/hologo.dtx'/>
  <miktex location='oberdiek'/>
  <texlive location='oberdiek'/>
  <install path='/macros/latex/contrib/oberdiek/oberdiek.tds.zip'/>
</entry>
%</catalogue>
%    \end{macrocode}
%
% \begin{thebibliography}{9}
% \raggedright
%
% \bibitem{btxdoc}
% Oren Patashnik,
% \textit{\hologo{BibTeX}ing},
% 1988-02-08.\\
% \CTAN{biblio/bibtex/base/}
%
% \bibitem{dtklogos}
% Gerd Neugebauer, DANTE,
% \textit{Package \xpackage{dtklogos}},
% 2011-04-25.\\
% \CTAN{usergrps/dante/dtk/dtklogos.sty}
%
% \bibitem{etexman}
% The \hologo{NTS} Team,
% \textit{The \hologo{eTeX} manual},
% 1998-02.\\
% \CTAN{systems/e-tex/v2/doc/}
%
% \bibitem{ExTeX-FAQ}
% The \hologo{ExTeX} group,
% \textit{\hologo{ExTeX}: FAQ -- How is \hologo{ExTeX} typeset?},
% 2007-04-14.\\
% \url{http://www.extex.org/documentation/faq.html}
%
% \bibitem{LyX}
% %@MISC{ LyX,
% %  title = {{LyX 2.0.0 -- The Document Processor [Computer software and manual]}},
% %  author = {{The LyX Team}},
% %  howpublished = {Internet: http://www.lyx.org},
% %  year = {2011-05-08},
% %  note = {Retrieved May 10, 2011, from http://www.lyx.org},
% %  url = {http://www.lyx.org/}
% %}
% The \hologo{LyX} Team,
% \textit{\hologo{LyX} -- The Document Processor},
% 2011-05-08.\\
% \url{http://www.lyx.org/}
%
% \bibitem{OzTeX}
% Andrew Trevorrow,
% \hologo{OzTeX} FAQ: What is the correct way to typeset ``\hologo{OzTeX}''?,
% 2011-09-15 (visited).
% \url{http://www.trevorrow.com/oztex/ozfaq.html#oztex-logo}
%
% \bibitem{PiCTeX}
% Michael Wichura,
% \textit{The \hologo{PiCTeX} macro package},
% 1987-09-21.
% \CTAN{graphics/pictex/}
%
% \bibitem{scrlogo}
% Markus Kohm,
% \textit{\hologo{KOMAScript} Datei \xfile{scrlogo.dtx}},
% 2009-01-30.\\
% \CTAN{install/macros/latex/contrib/komascript.tds.zip}
%
% \end{thebibliography}
%
% \begin{History}
%   \begin{Version}{2010/04/08 v1.0}
%   \item
%     The first version.
%   \end{Version}
%   \begin{Version}{2010/04/16 v1.1}
%   \item
%     \cs{Hologo} added for support of logos at start of a sentence.
%   \item
%     \cs{hologoSetup} and \cs{hologoLogoSetup} added.
%   \item
%     Options \xoption{break}, \xoption{hyphenbreak}, \xoption{spacebreak}
%     added.
%   \item
%     Variant support added by option \xoption{variant}.
%   \end{Version}
%   \begin{Version}{2010/04/24 v1.2}
%   \item
%     \hologo{LaTeX3} added.
%   \item
%     \hologo{VTeX} added.
%   \end{Version}
%   \begin{Version}{2010/11/21 v1.3}
%   \item
%     \hologo{iniTeX}, \hologo{virTeX} added.
%   \end{Version}
%   \begin{Version}{2011/03/25 v1.4}
%   \item
%     \hologo{ConTeXt} with variants added.
%   \item
%     Option \xoption{discretionarybreak} added as refinement for
%     option \xoption{break}.
%   \end{Version}
%   \begin{Version}{2011/04/21 v1.5}
%   \item
%     Wrong TDS directory for test files fixed.
%   \end{Version}
%   \begin{Version}{2011/10/01 v1.6}
%   \item
%     Support for package \xpackage{tex4ht} added.
%   \item
%     Support for \cs{csname} added if \cs{ifincsname} is available.
%   \item
%     New logos:
%     \hologo{(La)TeX},
%     \hologo{biber},
%     \hologo{BibTeX} (\xoption{sc}, \xoption{sf}),
%     \hologo{emTeX},
%     \hologo{ExTeX},
%     \hologo{KOMAScript},
%     \hologo{La},
%     \hologo{LyX},
%     \hologo{MiKTeX},
%     \hologo{NTS},
%     \hologo{OzMF},
%     \hologo{OzMP},
%     \hologo{OzTeX},
%     \hologo{OzTtH},
%     \hologo{PCTeX},
%     \hologo{PiC},
%     \hologo{PiCTeX},
%     \hologo{METAFONT},
%     \hologo{MetaFun},
%     \hologo{METAPOST},
%     \hologo{MetaPost},
%     \hologo{SLiTeX} (\xoption{lift}, \xoption{narrow}, \xoption{simple}),
%     \hologo{SliTeX} (\xoption{narrow}, \xoption{simple}, \xoption{lift}),
%     \hologo{teTeX}.
%   \item
%     Fixes:
%     \hologo{iniTeX},
%     \hologo{pdfLaTeX},
%     \hologo{pdfTeX},
%     \hologo{virTeX}.
%   \item
%     \cs{hologoFontSetup} and \cs{hologoLogoFontSetup} added.
%   \item
%     \cs{hologoVariant} and \cs{HologoVariant} added.
%   \end{Version}
%   \begin{Version}{2011/11/22 v1.7}
%   \item
%     New logos:
%     \hologo{BibTeX8},
%     \hologo{LaTeXML},
%     \hologo{SageTeX},
%     \hologo{TeX4ht},
%     \hologo{TTH}.
%   \item
%     \hologo{Xe} and friends: Driver stuff fixed.
%   \item
%     \hologo{Xe} and friends: Support for italic added.
%   \item
%     \hologo{Xe} and friends: Package support for \xpackage{pgf}
%     and \xpackage{pstricks} added.
%   \end{Version}
%   \begin{Version}{2011/11/29 v1.8}
%   \item
%     New logos:
%     \hologo{HanTheThanh}.
%   \end{Version}
%   \begin{Version}{2011/12/21 v1.9}
%   \item
%     Patch for package \xpackage{ifxetex} added for the case that
%     \cs{newif} is undefined in \hologo{iniTeX}.
%   \item
%     Some fixes for \hologo{iniTeX}.
%   \end{Version}
%   \begin{Version}{2012/04/26 v1.10}
%   \item
%     Fix in bookmark version of logo ``\hologo{HanTheThanh}''.
%   \end{Version}
%   \begin{Version}{2016/05/12 v1.11}
%   \item
%     Update HOLOGO@IfCharExists (previously in texlive)
%   \item define pdfliteral in current luatex.
%   \end{Version}
% \end{History}
%
% \PrintIndex
%
% \Finale
\endinput
%
        \else
          \input hologo.cfg\relax
        \fi
      \fi
    }%
    \ltx@IfUndefined{newread}{%
      \chardef\HOLOGO@temp=15 %
      \def\HOLOGO@CheckRead{%
        \ifeof\HOLOGO@temp
          \HOLOGO@InputIfExists
        \else
          \ifcase\HOLOGO@temp
            \@PackageWarningNoLine{hologo}{%
              Configuration file ignored, because\MessageBreak
              a free read register could not be found%
            }%
          \else
            \begingroup
              \count\ltx@cclv=\HOLOGO@temp
              \advance\ltx@cclv by \ltx@minusone
              \edef\x{\endgroup
                \chardef\noexpand\HOLOGO@temp=\the\count\ltx@cclv
                \relax
              }%
            \x
          \fi
        \fi
      }%
    }{%
      \csname newread\endcsname\HOLOGO@temp
      \HOLOGO@InputIfExists
    }%
  }{%
    \edef\HOLOGO@temp{\pdf@filesize{hologo.cfg}}%
    \ifx\HOLOGO@temp\ltx@empty
    \else
      \ifnum\HOLOGO@temp>0 %
        \begingroup
          \def\x{LaTeX2e}%
        \expandafter\endgroup
        \ifx\fmtname\x
          % \iffalse meta-comment
%
% File: hologo.dtx
% Version: 2016/05/12 v1.11
% Info: A logo collection with bookmark support
%
% Copyright (C) 2010-2012 by
%    Heiko Oberdiek <heiko.oberdiek at googlemail.com>
%
% This work may be distributed and/or modified under the
% conditions of the LaTeX Project Public License, either
% version 1.3c of this license or (at your option) any later
% version. This version of this license is in
%    http://www.latex-project.org/lppl/lppl-1-3c.txt
% and the latest version of this license is in
%    http://www.latex-project.org/lppl.txt
% and version 1.3 or later is part of all distributions of
% LaTeX version 2005/12/01 or later.
%
% This work has the LPPL maintenance status "maintained".
%
% This Current Maintainer of this work is Heiko Oberdiek.
%
% The Base Interpreter refers to any `TeX-Format',
% because some files are installed in TDS:tex/generic//.
%
% This work consists of the main source file hologo.dtx
% and the derived files
%    hologo.sty, hologo.pdf, hologo.ins, hologo.drv, hologo-example.tex,
%    hologo-test1.tex, hologo-test-spacefactor.tex,
%    hologo-test-list.tex.
%
% Distribution:
%    CTAN:macros/latex/contrib/oberdiek/hologo.dtx
%    CTAN:macros/latex/contrib/oberdiek/hologo.pdf
%
% Unpacking:
%    (a) If hologo.ins is present:
%           tex hologo.ins
%    (b) Without hologo.ins:
%           tex hologo.dtx
%    (c) If you insist on using LaTeX
%           latex \let\install=y\input{hologo.dtx}
%        (quote the arguments according to the demands of your shell)
%
% Documentation:
%    (a) If hologo.drv is present:
%           latex hologo.drv
%    (b) Without hologo.drv:
%           latex hologo.dtx; ...
%    The class ltxdoc loads the configuration file ltxdoc.cfg
%    if available. Here you can specify further options, e.g.
%    use A4 as paper format:
%       \PassOptionsToClass{a4paper}{article}
%
%    Programm calls to get the documentation (example):
%       pdflatex hologo.dtx
%       makeindex -s gind.ist hologo.idx
%       pdflatex hologo.dtx
%       makeindex -s gind.ist hologo.idx
%       pdflatex hologo.dtx
%
% Installation:
%    TDS:tex/generic/oberdiek/hologo.sty
%    TDS:doc/latex/oberdiek/hologo.pdf
%    TDS:doc/latex/oberdiek/example/hologo-example.tex
%    TDS:doc/latex/oberdiek/test/hologo-test1.tex
%    TDS:doc/latex/oberdiek/test/hologo-test-spacefactor.tex
%    TDS:doc/latex/oberdiek/test/hologo-test-list.tex
%    TDS:source/latex/oberdiek/hologo.dtx
%
%<*ignore>
\begingroup
  \catcode123=1 %
  \catcode125=2 %
  \def\x{LaTeX2e}%
\expandafter\endgroup
\ifcase 0\ifx\install y1\fi\expandafter
         \ifx\csname processbatchFile\endcsname\relax\else1\fi
         \ifx\fmtname\x\else 1\fi\relax
\else\csname fi\endcsname
%</ignore>
%<*install>
\input docstrip.tex
\Msg{************************************************************************}
\Msg{* Installation}
\Msg{* Package: hologo 2016/05/12 v1.11 A logo collection with bookmark support (HO)}
\Msg{************************************************************************}

\keepsilent
\askforoverwritefalse

\let\MetaPrefix\relax
\preamble

This is a generated file.

Project: hologo
Version: 2016/05/12 v1.11

Copyright (C) 2010-2012 by
   Heiko Oberdiek <heiko.oberdiek at googlemail.com>

This work may be distributed and/or modified under the
conditions of the LaTeX Project Public License, either
version 1.3c of this license or (at your option) any later
version. This version of this license is in
   http://www.latex-project.org/lppl/lppl-1-3c.txt
and the latest version of this license is in
   http://www.latex-project.org/lppl.txt
and version 1.3 or later is part of all distributions of
LaTeX version 2005/12/01 or later.

This work has the LPPL maintenance status "maintained".

This Current Maintainer of this work is Heiko Oberdiek.

The Base Interpreter refers to any `TeX-Format',
because some files are installed in TDS:tex/generic//.

This work consists of the main source file hologo.dtx
and the derived files
   hologo.sty, hologo.pdf, hologo.ins, hologo.drv, hologo-example.tex,
   hologo-test1.tex, hologo-test-spacefactor.tex,
   hologo-test-list.tex.

\endpreamble
\let\MetaPrefix\DoubleperCent

\generate{%
  \file{hologo.ins}{\from{hologo.dtx}{install}}%
  \file{hologo.drv}{\from{hologo.dtx}{driver}}%
  \usedir{tex/generic/oberdiek}%
  \file{hologo.sty}{\from{hologo.dtx}{package}}%
  \usedir{doc/latex/oberdiek/example}%
  \file{hologo-example.tex}{\from{hologo.dtx}{example}}%
  \usedir{doc/latex/oberdiek/test}%
  \file{hologo-test1.tex}{\from{hologo.dtx}{test1}}%
  \file{hologo-test-spacefactor.tex}{\from{hologo.dtx}{test-spacefactor}}%
  \file{hologo-test-list.tex}{\from{hologo.dtx}{test-list}}%
  \nopreamble
  \nopostamble
  \usedir{source/latex/oberdiek/catalogue}%
  \file{hologo.xml}{\from{hologo.dtx}{catalogue}}%
}

\catcode32=13\relax% active space
\let =\space%
\Msg{************************************************************************}
\Msg{*}
\Msg{* To finish the installation you have to move the following}
\Msg{* file into a directory searched by TeX:}
\Msg{*}
\Msg{*     hologo.sty}
\Msg{*}
\Msg{* To produce the documentation run the file `hologo.drv'}
\Msg{* through LaTeX.}
\Msg{*}
\Msg{* Happy TeXing!}
\Msg{*}
\Msg{************************************************************************}

\endbatchfile
%</install>
%<*ignore>
\fi
%</ignore>
%<*driver>
\NeedsTeXFormat{LaTeX2e}
\ProvidesFile{hologo.drv}%
  [2016/05/12 v1.11 A logo collection with bookmark support (HO)]%
\documentclass{ltxdoc}
\usepackage{holtxdoc}[2011/11/22]
\usepackage{hologo}[2016/05/12]
\usepackage{longtable}
\usepackage{array}
\usepackage{paralist}
%\usepackage[T1]{fontenc}
%\usepackage{lmodern}
\begin{document}
  \DocInput{hologo.dtx}%
\end{document}
%</driver>
% \fi
%
%
% \CharacterTable
%  {Upper-case    \A\B\C\D\E\F\G\H\I\J\K\L\M\N\O\P\Q\R\S\T\U\V\W\X\Y\Z
%   Lower-case    \a\b\c\d\e\f\g\h\i\j\k\l\m\n\o\p\q\r\s\t\u\v\w\x\y\z
%   Digits        \0\1\2\3\4\5\6\7\8\9
%   Exclamation   \!     Double quote  \"     Hash (number) \#
%   Dollar        \$     Percent       \%     Ampersand     \&
%   Acute accent  \'     Left paren    \(     Right paren   \)
%   Asterisk      \*     Plus          \+     Comma         \,
%   Minus         \-     Point         \.     Solidus       \/
%   Colon         \:     Semicolon     \;     Less than     \<
%   Equals        \=     Greater than  \>     Question mark \?
%   Commercial at \@     Left bracket  \[     Backslash     \\
%   Right bracket \]     Circumflex    \^     Underscore    \_
%   Grave accent  \`     Left brace    \{     Vertical bar  \|
%   Right brace   \}     Tilde         \~}
%
% \GetFileInfo{hologo.drv}
%
% \title{The \xpackage{hologo} package}
% \date{2016/05/12 v1.11}
% \author{Heiko Oberdiek\\\xemail{heiko.oberdiek at googlemail.com}}
%
% \maketitle
%
% \begin{abstract}
% This package starts a collection of logos with support for bookmarks
% strings.
% \end{abstract}
%
% \tableofcontents
%
% \section{Documentation}
%
% \subsection{Logo macros}
%
% \begin{declcs}{hologo} \M{name}
% \end{declcs}
% Macro \cs{hologo} sets the logo with name \meta{name}.
% The following table shows the supported names.
%
% \begingroup
%   \def\hologoEntry#1#2#3{^^A
%     #1&#2&\hologoLogoSetup{#1}{variant=#2}\hologo{#1}&#3\tabularnewline
%   }
%   \begin{longtable}{>{\ttfamily}l>{\ttfamily}lll}
%     \rmfamily\bfseries{name} & \rmfamily\bfseries variant
%     & \bfseries logo & \bfseries since\\
%     \hline
%     \endhead
%     \hologoList
%   \end{longtable}
% \endgroup
%
% \begin{declcs}{Hologo} \M{name}
% \end{declcs}
% Macro \cs{Hologo} starts the logo \meta{name} with an uppercase
% letter. As an exception small greek letters are not converted
% to uppercase. Examples, see \hologo{eTeX} and \hologo{ExTeX}.
%
% \subsection{Setup macros}
%
% The package does not support package options, but the following
% setup macros can be used to set options.
%
% \begin{declcs}{hologoSetup} \M{key value list}
% \end{declcs}
% Macro \cs{hologoSetup} sets global options.
%
% \begin{declcs}{hologoLogoSetup} \M{logo} \M{key value list}
% \end{declcs}
% Some options can also be used to configure a logo.
% These settings take precedence over global option settings.
%
% \subsection{Options}\label{sec:options}
%
% There are boolean and string options:
% \begin{description}
% \item[Boolean option:]
% It takes |true| or |false|
% as value. If the value is omitted, then |true| is used.
% \item[String option:]
% A value must be given as string. (But the string might be empty.)
% \end{description}
% The following options can be used both in \cs{hologoSetup}
% and \cs{hologoLogoSetup}:
% \begin{description}
% \def\entry#1{\item[\xoption{#1}:]}
% \entry{break}
%   enables or disables line breaks inside the logo. This setting is
%   refined by options \xoption{hyphenbreak}, \xoption{spacebreak}
%   or \xoption{discretionarybreak}.
%   Default is |false|.
% \entry{hyphenbreak}
%   enables or disables the line break right after the hyphen character.
% \entry{spacebreak}
%   enables or disables line breaks at space characters.
% \entry{discretionarybreak}
%   enables or disables line breaks at hyphenation points
%   (inserted by \cs{-}).
% \end{description}
% Macro \cs{hologoLogoSetup} also knows:
% \begin{description}
% \item[\xoption{variant}:]
%   This is a string option. It specifies a variant of a logo that
%   must exist. An empty string selects the package default variant.
% \end{description}
% Example:
% \begin{quote}
%   |\hologoSetup{break=false}|\\
%   |\hologoLogoSetup{plainTeX}{variant=hyphen,hyphenbreak}|\\
%   Then ``plain-\TeX'' contains one break point after the hyphen.
% \end{quote}
%
% \subsection{Driver options}
%
% Sometimes graphical operations are needed to construct some
% glyphs (e.g.\ \hologo{XeTeX}). If package \xpackage{graphics}
% or package \xpackage{pgf} are found, then the macros are taken
% from there. Otherwise the packge defines its own operations
% and therefore needs the driver information. Many drivers are
% detected automatically (\hologo{pdfTeX}/\hologo{LuaTeX}
% in PDF mode, \hologo{XeTeX}, \hologo{VTeX}). These have precedence
% over a driver option. The driver can be given as package option
% or using \cs{hologoDriverSetup}.
% The following list contains the recognized driver options:
% \begin{itemize}
% \item \xoption{pdftex}, \xoption{luatex}
% \item \xoption{dvipdfm}, \xoption{dvipdfmx}
% \item \xoption{dvips}, \xoption{dvipsone}, \xoption{xdvi}
% \item \xoption{xetex}
% \item \xoption{vtex}
% \end{itemize}
% The left driver of a line is the driver name that is used internally.
% The following names are aliases for drivers that use the
% same method. Therefore the entry in the \xext{log} file for
% the used driver prints the internally used driver name.
% \begin{description}
% \item[\xoption{driverfallback}:]
%   This option expects a driver that is used,
%   if the driver could not be detected automatically.
% \end{description}
%
% \begin{declcs}{hologoDriverSetup} \M{driver option}
% \end{declcs}
% The driver can also be configured after package loading
% using \cs{hologoDriverSetup}, also the way for \hologo{plainTeX}
% to setup the driver.
%
% \subsection{Font setup}
%
% Some logos require a special font, but should also be usable by
% \hologo{plainTeX}. Therefore the package provides some ways
% to influence the font settings. The options below
% take font settings as values. Both font commands
% such as \cs{sffamily} and macros that take one argument
% like \cs{textsf} can be used.
%
% \begin{declcs}{hologoFontSetup} \M{key value list}
% \end{declcs}
% Macro \cs{hologoFontSetup} sets the fonts for all logos.
% Supported keys:
% \begin{description}
% \def\entry#1{\item[\xoption{#1}:]}
% \entry{general}
%   This font is used for all logos. The default is empty.
%   That means no special font is used.
% \entry{bibsf}
%   This font is used for
%   {\hologoLogoSetup{BibTeX}{variant=sf}\hologo{BibTeX}}
%   with variant \xoption{sf}.
% \entry{rm}
%   This font is a serif font. It is used for \hologo{ExTeX}.
% \entry{sc}
%   This font specifies a small caps font. It is used for
%   {\hologoLogoSetup{BibTeX}{variant=sc}\hologo{BibTeX}}
%   with variant \xoption{sc}.
% \entry{sf}
%   This font specifies a sans serif font. The default
%   is \cs{sffamily}, then \cs{sf} is tried. Otherwise
%   a warning is given. It is used by \hologo{KOMAScript}.
% \entry{sy}
%   This is the font for math symbols (e.g. cmsy).
%   It is used by \hologo{AmS}, \hologo{NTS}, \hologo{ExTeX}.
% \entry{logo}
%   \hologo{METAFONT} and \hologo{METAPOST} are using that font.
%   In \hologo{LaTeX} \cs{logofamily} is used and
%   the definitions of package \xpackage{mflogo} are used
%   if the package is not loaded.
%   Otherwise the \cs{tenlogo} is used and defined
%   if it does not already exists.
% \end{description}
%
% \begin{declcs}{hologoLogoFontSetup} \M{logo} \M{key value list}
% \end{declcs}
% Fonts can also be set for a logo or logo component separately,
% see the following list.
% The keys are the same as for \cs{hologoFontSetup}.
%
% \begin{longtable}{>{\ttfamily}l>{\sffamily}ll}
%   \meta{logo} & keys & result\\
%   \hline
%   \endhead
%   BibTeX & bibsf & {\hologoLogoSetup{BibTeX}{variant=sf}\hologo{BibTeX}}\\[.5ex]
%   BibTeX & sc & {\hologoLogoSetup{BibTeX}{variant=sc}\hologo{BibTeX}}\\[.5ex]
%   ExTeX & rm & \hologo{ExTeX}\\
%   SliTeX & rm & \hologo{SliTeX}\\[.5ex]
%   AmS & sy & \hologo{AmS}\\
%   ExTeX & sy & \hologo{ExTeX}\\
%   NTS & sy & \hologo{NTS}\\[.5ex]
%   KOMAScript & sf & \hologo{KOMAScript}\\[.5ex]
%   METAFONT & logo & \hologo{METAFONT}\\
%   METAPOST & logo & \hologo{METAPOST}\\[.5ex]
%   SliTeX & sc \hologo{SliTeX}
% \end{longtable}
%
% \subsubsection{Font order}
%
% For all logos the font \xoption{general} is applied first.
% Example:
%\begin{quote}
%|\hologoFontSetup{general=\color{red}}|
%\end{quote}
% will print red logos.
% Then if the font uses a special font \xoption{sf}, for example,
% the font is applied that is setup by \cs{hologoLogoFontSetup}.
% If this font is not setup, then the common font setup
% by \cs{hologoFontSetup} is used. Otherwise a warning is given,
% that there is no font configured.
%
% \subsection{Additional user macros}
%
% Usually a variant of a logo is configured by using
% \cs{hologoLogoSetup}, because it is bad style to mix
% different variants of the same logo in the same text.
% There the following macros are a convenience for testing.
%
% \begin{declcs}{hologoVariant} \M{name} \M{variant}\\
%   \cs{HologoVariant} \M{name} \M{variant}
% \end{declcs}
% Logo \meta{name} is set using \meta{variant} that specifies
% explicitely which variant of the macro is used. If the argument
% is empty, then the default form of the logo is used
% (configurable by \cs{hologoLogoSetup}).
%
% \cs{HologoVariant} is used if the logo is set in a context
% that needs an uppercase first letter (beginning of a sentence, \dots).
%
% \begin{declcs}{hologoList}\\
%   \cs{hologoEntry} \M{logo} \M{variant} \M{since}
% \end{declcs}
% Macro \cs{hologoList} contains all logos that are provided
% by the package including variants. The list consists of calls
% of \cs{hologoEntry} with three arguments starting with the
% logo name \meta{logo} and its variant \meta{variant}. An empty
% variant means the current default. Argument \meta{since} specifies
% with version of the package \xpackage{hologo} is needed to get
% the logo. If the logo is fixed, then the date gets updated.
% Therefore the date \meta{since} is not exactly the date of
% the first introduction, but rather the date of the latest fix.
%
% Before \cs{hologoList} can be used, macro \cs{hologoEntry} needs
% a definition. The example file in section \ref{sec:example}
% shows applications of \cs{hologoList}.
%
% \subsection{Supported contexts}
%
% Macros \cs{hologo} and friends support special contexts:
% \begin{itemize}
% \item \hologo{LaTeX}'s protection mechanism.
% \item Bookmarks of package \xpackage{hyperref}.
% \item Package \xpackage{tex4ht}.
% \item The macros can be used inside \cs{csname} constructs,
%   if \cs{ifincsname} is available (\hologo{pdfTeX}, \hologo{XeTeX},
%   \hologo{LuaTeX}).
% \end{itemize}
%
% \subsection{Example}
% \label{sec:example}
%
% The following example prints the logos in different fonts.
%    \begin{macrocode}
%<*example>
%<<verbatim
\NeedsTeXFormat{LaTeX2e}
\documentclass[a4paper]{article}
\usepackage[
  hmargin=20mm,
  vmargin=20mm,
]{geometry}
\pagestyle{empty}
\usepackage{hologo}[2016/05/12]
\usepackage{longtable}
\usepackage{array}
\setlength{\extrarowheight}{2pt}
\usepackage[T1]{fontenc}
\usepackage{lmodern}
\usepackage{pdflscape}
\usepackage[
  pdfencoding=auto,
]{hyperref}
\hypersetup{
  pdfauthor={Heiko Oberdiek},
  pdftitle={Example for package `hologo'},
  pdfsubject={Logos with fonts lmr, lmss, qtm, qpl, qhv},
}
\usepackage{bookmark}

% Print the logo list on the console

\begingroup
  \typeout{}%
  \typeout{*** Begin of logo list ***}%
  \newcommand*{\hologoEntry}[3]{%
    \typeout{#1 \ifx\\#2\\\else(#2) \fi[#3]}%
  }%
  \hologoList
  \typeout{*** End of logo list ***}%
  \typeout{}%
\endgroup

\begin{document}
\begin{landscape}

  \section{Example file for package `hologo'}

  % Table for font names

  \begin{longtable}{>{\bfseries}ll}
    \textbf{font} & \textbf{Font name}\\
    \hline
    lmr & Latin Modern Roman\\
    lmss & Latin Modern Sans\\
    qtm & \TeX\ Gyre Termes\\
    qhv & \TeX\ Gyre Heros\\
    qpl & \TeX\ Gyre Pagella\\
  \end{longtable}

  % Logo list with logos in different fonts

  \begingroup
    \newcommand*{\SetVariant}[2]{%
      \ifx\\#2\\%
      \else
        \hologoLogoSetup{#1}{variant=#2}%
      \fi
    }%
    \newcommand*{\hologoEntry}[3]{%
      \SetVariant{#1}{#2}%
      \raisebox{1em}[0pt][0pt]{\hypertarget{#1@#2}{}}%
      \bookmark[%
        dest={#1@#2},%
      ]{%
        #1\ifx\\#2\\\else\space(#2)\fi: \Hologo{#1}, \hologo{#1} %
        [Unicode]%
      }%
      \hypersetup{unicode=false}%
      \bookmark[%
        dest={#1@#2},%
      ]{%
        #1\ifx\\#2\\\else\space(#2)\fi: \Hologo{#1}, \hologo{#1} %
        [PDFDocEncoding]%
      }%
      \texttt{#1}%
      &%
      \texttt{#2}%
      &%
      \Hologo{#1}%
      &%
      \SetVariant{#1}{#2}%
      \hologo{#1}%
      &%
      \SetVariant{#1}{#2}%
      \fontfamily{qtm}\selectfont
      \hologo{#1}%
      &%
      \SetVariant{#1}{#2}%
      \fontfamily{qpl}\selectfont
      \hologo{#1}%
      &%
      \SetVariant{#1}{#2}%
      \textsf{\hologo{#1}}%
      &%
      \SetVariant{#1}{#2}%
      \fontfamily{qhv}\selectfont
      \hologo{#1}%
      \tabularnewline
    }%
    \begin{longtable}{llllllll}%
      \textbf{\textit{logo}} & \textbf{\textit{variant}} &
      \texttt{\string\Hologo} &
      \textbf{lmr} & \textbf{qtm} & \textbf{qpl} &
      \textbf{lmss} & \textbf{qhv}
      \tabularnewline
      \hline
      \endhead
      \hologoList
    \end{longtable}%
  \endgroup

\end{landscape}
\end{document}
%verbatim
%</example>
%    \end{macrocode}
%
% \StopEventually{
% }
%
% \section{Implementation}
%    \begin{macrocode}
%<*package>
%    \end{macrocode}
%    Reload check, especially if the package is not used with \LaTeX.
%    \begin{macrocode}
\begingroup\catcode61\catcode48\catcode32=10\relax%
  \catcode13=5 % ^^M
  \endlinechar=13 %
  \catcode35=6 % #
  \catcode39=12 % '
  \catcode44=12 % ,
  \catcode45=12 % -
  \catcode46=12 % .
  \catcode58=12 % :
  \catcode64=11 % @
  \catcode123=1 % {
  \catcode125=2 % }
  \expandafter\let\expandafter\x\csname ver@hologo.sty\endcsname
  \ifx\x\relax % plain-TeX, first loading
  \else
    \def\empty{}%
    \ifx\x\empty % LaTeX, first loading,
      % variable is initialized, but \ProvidesPackage not yet seen
    \else
      \expandafter\ifx\csname PackageInfo\endcsname\relax
        \def\x#1#2{%
          \immediate\write-1{Package #1 Info: #2.}%
        }%
      \else
        \def\x#1#2{\PackageInfo{#1}{#2, stopped}}%
      \fi
      \x{hologo}{The package is already loaded}%
      \aftergroup\endinput
    \fi
  \fi
\endgroup%
%    \end{macrocode}
%    Package identification:
%    \begin{macrocode}
\begingroup\catcode61\catcode48\catcode32=10\relax%
  \catcode13=5 % ^^M
  \endlinechar=13 %
  \catcode35=6 % #
  \catcode39=12 % '
  \catcode40=12 % (
  \catcode41=12 % )
  \catcode44=12 % ,
  \catcode45=12 % -
  \catcode46=12 % .
  \catcode47=12 % /
  \catcode58=12 % :
  \catcode64=11 % @
  \catcode91=12 % [
  \catcode93=12 % ]
  \catcode123=1 % {
  \catcode125=2 % }
  \expandafter\ifx\csname ProvidesPackage\endcsname\relax
    \def\x#1#2#3[#4]{\endgroup
      \immediate\write-1{Package: #3 #4}%
      \xdef#1{#4}%
    }%
  \else
    \def\x#1#2[#3]{\endgroup
      #2[{#3}]%
      \ifx#1\@undefined
        \xdef#1{#3}%
      \fi
      \ifx#1\relax
        \xdef#1{#3}%
      \fi
    }%
  \fi
\expandafter\x\csname ver@hologo.sty\endcsname
\ProvidesPackage{hologo}%
  [2016/05/12 v1.11 A logo collection with bookmark support (HO)]%
%    \end{macrocode}
%
%    \begin{macrocode}
\begingroup\catcode61\catcode48\catcode32=10\relax%
  \catcode13=5 % ^^M
  \endlinechar=13 %
  \catcode123=1 % {
  \catcode125=2 % }
  \catcode64=11 % @
  \def\x{\endgroup
    \expandafter\edef\csname HOLOGO@AtEnd\endcsname{%
      \endlinechar=\the\endlinechar\relax
      \catcode13=\the\catcode13\relax
      \catcode32=\the\catcode32\relax
      \catcode35=\the\catcode35\relax
      \catcode61=\the\catcode61\relax
      \catcode64=\the\catcode64\relax
      \catcode123=\the\catcode123\relax
      \catcode125=\the\catcode125\relax
    }%
  }%
\x\catcode61\catcode48\catcode32=10\relax%
\catcode13=5 % ^^M
\endlinechar=13 %
\catcode35=6 % #
\catcode64=11 % @
\catcode123=1 % {
\catcode125=2 % }
\def\TMP@EnsureCode#1#2{%
  \edef\HOLOGO@AtEnd{%
    \HOLOGO@AtEnd
    \catcode#1=\the\catcode#1\relax
  }%
  \catcode#1=#2\relax
}
\TMP@EnsureCode{10}{12}% ^^J
\TMP@EnsureCode{33}{12}% !
\TMP@EnsureCode{34}{12}% "
\TMP@EnsureCode{36}{3}% $
\TMP@EnsureCode{38}{4}% &
\TMP@EnsureCode{39}{12}% '
\TMP@EnsureCode{40}{12}% (
\TMP@EnsureCode{41}{12}% )
\TMP@EnsureCode{42}{12}% *
\TMP@EnsureCode{43}{12}% +
\TMP@EnsureCode{44}{12}% ,
\TMP@EnsureCode{45}{12}% -
\TMP@EnsureCode{46}{12}% .
\TMP@EnsureCode{47}{12}% /
\TMP@EnsureCode{58}{12}% :
\TMP@EnsureCode{59}{12}% ;
\TMP@EnsureCode{60}{12}% <
\TMP@EnsureCode{62}{12}% >
\TMP@EnsureCode{63}{12}% ?
\TMP@EnsureCode{91}{12}% [
\TMP@EnsureCode{93}{12}% ]
\TMP@EnsureCode{94}{7}% ^ (superscript)
\TMP@EnsureCode{95}{8}% _ (subscript)
\TMP@EnsureCode{96}{12}% `
\TMP@EnsureCode{124}{12}% |
\edef\HOLOGO@AtEnd{%
  \HOLOGO@AtEnd
  \escapechar\the\escapechar\relax
  \noexpand\endinput
}
\escapechar=92 %
%    \end{macrocode}
%
% \subsection{Logo list}
%
%    \begin{macro}{\hologoList}
%    \begin{macrocode}
\def\hologoList{%
  \hologoEntry{(La)TeX}{}{2011/10/01}%
  \hologoEntry{AmSLaTeX}{}{2010/04/16}%
  \hologoEntry{AmSTeX}{}{2010/04/16}%
  \hologoEntry{biber}{}{2011/10/01}%
  \hologoEntry{BibTeX}{}{2011/10/01}%
  \hologoEntry{BibTeX}{sf}{2011/10/01}%
  \hologoEntry{BibTeX}{sc}{2011/10/01}%
  \hologoEntry{BibTeX8}{}{2011/11/22}%
  \hologoEntry{ConTeXt}{}{2011/03/25}%
  \hologoEntry{ConTeXt}{narrow}{2011/03/25}%
  \hologoEntry{ConTeXt}{simple}{2011/03/25}%
  \hologoEntry{emTeX}{}{2010/04/26}%
  \hologoEntry{eTeX}{}{2010/04/08}%
  \hologoEntry{ExTeX}{}{2011/10/01}%
  \hologoEntry{HanTheThanh}{}{2011/11/29}%
  \hologoEntry{iniTeX}{}{2011/10/01}%
  \hologoEntry{KOMAScript}{}{2011/10/01}%
  \hologoEntry{La}{}{2010/05/08}%
  \hologoEntry{LaTeX}{}{2010/04/08}%
  \hologoEntry{LaTeX2e}{}{2010/04/08}%
  \hologoEntry{LaTeX3}{}{2010/04/24}%
  \hologoEntry{LaTeXe}{}{2010/04/08}%
  \hologoEntry{LaTeXML}{}{2011/11/22}%
  \hologoEntry{LaTeXTeX}{}{2011/10/01}%
  \hologoEntry{LuaLaTeX}{}{2010/04/08}%
  \hologoEntry{LuaTeX}{}{2010/04/08}%
  \hologoEntry{LyX}{}{2011/10/01}%
  \hologoEntry{METAFONT}{}{2011/10/01}%
  \hologoEntry{MetaFun}{}{2011/10/01}%
  \hologoEntry{METAPOST}{}{2011/10/01}%
  \hologoEntry{MetaPost}{}{2011/10/01}%
  \hologoEntry{MiKTeX}{}{2011/10/01}%
  \hologoEntry{NTS}{}{2011/10/01}%
  \hologoEntry{OzMF}{}{2011/10/01}%
  \hologoEntry{OzMP}{}{2011/10/01}%
  \hologoEntry{OzTeX}{}{2011/10/01}%
  \hologoEntry{OzTtH}{}{2011/10/01}%
  \hologoEntry{PCTeX}{}{2011/10/01}%
  \hologoEntry{pdfTeX}{}{2011/10/01}%
  \hologoEntry{pdfLaTeX}{}{2011/10/01}%
  \hologoEntry{PiC}{}{2011/10/01}%
  \hologoEntry{PiCTeX}{}{2011/10/01}%
  \hologoEntry{plainTeX}{}{2010/04/08}%
  \hologoEntry{plainTeX}{space}{2010/04/16}%
  \hologoEntry{plainTeX}{hyphen}{2010/04/16}%
  \hologoEntry{plainTeX}{runtogether}{2010/04/16}%
  \hologoEntry{SageTeX}{}{2011/11/22}%
  \hologoEntry{SLiTeX}{}{2011/10/01}%
  \hologoEntry{SLiTeX}{lift}{2011/10/01}%
  \hologoEntry{SLiTeX}{narrow}{2011/10/01}%
  \hologoEntry{SLiTeX}{simple}{2011/10/01}%
  \hologoEntry{SliTeX}{}{2011/10/01}%
  \hologoEntry{SliTeX}{narrow}{2011/10/01}%
  \hologoEntry{SliTeX}{simple}{2011/10/01}%
  \hologoEntry{SliTeX}{lift}{2011/10/01}%
  \hologoEntry{teTeX}{}{2011/10/01}%
  \hologoEntry{TeX}{}{2010/04/08}%
  \hologoEntry{TeX4ht}{}{2011/11/22}%
  \hologoEntry{TTH}{}{2011/11/22}%
  \hologoEntry{virTeX}{}{2011/10/01}%
  \hologoEntry{VTeX}{}{2010/04/24}%
  \hologoEntry{Xe}{}{2010/04/08}%
  \hologoEntry{XeLaTeX}{}{2010/04/08}%
  \hologoEntry{XeTeX}{}{2010/04/08}%
}
%    \end{macrocode}
%    \end{macro}
%
% \subsection{Load resources}
%
%    \begin{macrocode}
\begingroup\expandafter\expandafter\expandafter\endgroup
\expandafter\ifx\csname RequirePackage\endcsname\relax
  \def\TMP@RequirePackage#1[#2]{%
    \begingroup\expandafter\expandafter\expandafter\endgroup
    \expandafter\ifx\csname ver@#1.sty\endcsname\relax
      \input #1.sty\relax
    \fi
  }%
  \TMP@RequirePackage{ltxcmds}[2011/02/04]%
  \TMP@RequirePackage{infwarerr}[2010/04/08]%
  \TMP@RequirePackage{kvsetkeys}[2010/03/01]%
  \TMP@RequirePackage{kvdefinekeys}[2010/03/01]%
  \TMP@RequirePackage{pdftexcmds}[2010/04/01]%
  \TMP@RequirePackage{ifpdf}[2010/01/28]%
  \TMP@RequirePackage{ifluatex}[2010/03/01]%
  \ltx@IfUndefined{newif}{%
    \expandafter\let\csname newif\endcsname\ltx@newif
  }{}%
  \TMP@RequirePackage{ifxetex}[2009/01/23]%
  \TMP@RequirePackage{ifvtex}[2010/03/01]%
\else
  \RequirePackage{ltxcmds}[2011/02/04]%
  \RequirePackage{infwarerr}[2010/04/08]%
  \RequirePackage{kvsetkeys}[2010/03/01]%
  \RequirePackage{kvdefinekeys}[2010/03/01]%
  \RequirePackage{pdftexcmds}[2010/04/01]%
  \RequirePackage{ifpdf}[2010/01/28]%
  \RequirePackage{ifluatex}[2010/03/01]%
  \RequirePackage{ifxetex}[2009/01/23]%
  \RequirePackage{ifvtex}[2010/03/01]%
\fi
%    \end{macrocode}
%
%    \begin{macro}{\HOLOGO@IfDefined}
%    \begin{macrocode}
\def\HOLOGO@IfExists#1{%
  \ifx\@undefined#1%
    \expandafter\ltx@secondoftwo
  \else
    \ifx\relax#1%
      \expandafter\ltx@secondoftwo
    \else
      \expandafter\expandafter\expandafter\ltx@firstoftwo
    \fi
  \fi
}
%    \end{macrocode}
%    \end{macro}
%
% \subsection{Setup macros}
%
%    \begin{macro}{\hologoSetup}
%    \begin{macrocode}
\def\hologoSetup{%
  \let\HOLOGO@name\relax
  \HOLOGO@Setup
}
%    \end{macrocode}
%    \end{macro}
%
%    \begin{macro}{\hologoLogoSetup}
%    \begin{macrocode}
\def\hologoLogoSetup#1{%
  \edef\HOLOGO@name{#1}%
  \ltx@IfUndefined{HoLogo@\HOLOGO@name}{%
    \@PackageError{hologo}{%
      Unknown logo `\HOLOGO@name'%
    }\@ehc
    \ltx@gobble
  }{%
    \HOLOGO@Setup
  }%
}
%    \end{macrocode}
%    \end{macro}
%
%    \begin{macro}{\HOLOGO@Setup}
%    \begin{macrocode}
\def\HOLOGO@Setup{%
  \kvsetkeys{HoLogo}%
}
%    \end{macrocode}
%    \end{macro}
%
% \subsection{Options}
%
%    \begin{macro}{\HOLOGO@DeclareBoolOption}
%    \begin{macrocode}
\def\HOLOGO@DeclareBoolOption#1{%
  \expandafter\chardef\csname HOLOGOOPT@#1\endcsname\ltx@zero
  \kv@define@key{HoLogo}{#1}[true]{%
    \def\HOLOGO@temp{##1}%
    \ifx\HOLOGO@temp\HOLOGO@true
      \ifx\HOLOGO@name\relax
        \expandafter\chardef\csname HOLOGOOPT@#1\endcsname=\ltx@one
      \else
        \expandafter\chardef\csname
        HoLogoOpt@#1@\HOLOGO@name\endcsname\ltx@one
      \fi
      \HOLOGO@SetBreakAll{#1}%
    \else
      \ifx\HOLOGO@temp\HOLOGO@false
        \ifx\HOLOGO@name\relax
          \expandafter\chardef\csname HOLOGOOPT@#1\endcsname=\ltx@zero
        \else
          \expandafter\chardef\csname
          HoLogoOpt@#1@\HOLOGO@name\endcsname=\ltx@zero
        \fi
        \HOLOGO@SetBreakAll{#1}%
      \else
        \@PackageError{hologo}{%
          Unknown value `##1' for boolean option `#1'.\MessageBreak
          Known values are `true' and `false'%
        }\@ehc
      \fi
    \fi
  }%
}
%    \end{macrocode}
%    \end{macro}
%
%    \begin{macro}{\HOLOGO@SetBreakAll}
%    \begin{macrocode}
\def\HOLOGO@SetBreakAll#1{%
  \def\HOLOGO@temp{#1}%
  \ifx\HOLOGO@temp\HOLOGO@break
    \ifx\HOLOGO@name\relax
      \chardef\HOLOGOOPT@hyphenbreak=\HOLOGOOPT@break
      \chardef\HOLOGOOPT@spacebreak=\HOLOGOOPT@break
      \chardef\HOLOGOOPT@discretionarybreak=\HOLOGOOPT@break
    \else
      \expandafter\chardef
         \csname HoLogoOpt@hyphenbreak@\HOLOGO@name\endcsname=%
         \csname HoLogoOpt@break@\HOLOGO@name\endcsname
      \expandafter\chardef
         \csname HoLogoOpt@spacebreak@\HOLOGO@name\endcsname=%
         \csname HoLogoOpt@break@\HOLOGO@name\endcsname
      \expandafter\chardef
         \csname HoLogoOpt@discretionarybreak@\HOLOGO@name
             \endcsname=%
         \csname HoLogoOpt@break@\HOLOGO@name\endcsname
    \fi
  \fi
}
%    \end{macrocode}
%    \end{macro}
%
%    \begin{macro}{\HOLOGO@true}
%    \begin{macrocode}
\def\HOLOGO@true{true}
%    \end{macrocode}
%    \end{macro}
%    \begin{macro}{\HOLOGO@false}
%    \begin{macrocode}
\def\HOLOGO@false{false}
%    \end{macrocode}
%    \end{macro}
%    \begin{macro}{\HOLOGO@break}
%    \begin{macrocode}
\def\HOLOGO@break{break}
%    \end{macrocode}
%    \end{macro}
%
%    \begin{macrocode}
\HOLOGO@DeclareBoolOption{break}
\HOLOGO@DeclareBoolOption{hyphenbreak}
\HOLOGO@DeclareBoolOption{spacebreak}
\HOLOGO@DeclareBoolOption{discretionarybreak}
%    \end{macrocode}
%
%    \begin{macrocode}
\kv@define@key{HoLogo}{variant}{%
  \ifx\HOLOGO@name\relax
    \@PackageError{hologo}{%
      Option `variant' is not available in \string\hologoSetup,%
      \MessageBreak
      Use \string\hologoLogoSetup\space instead%
    }\@ehc
  \else
    \edef\HOLOGO@temp{#1}%
    \ifx\HOLOGO@temp\ltx@empty
      \expandafter
      \let\csname HoLogoOpt@variant@\HOLOGO@name\endcsname\@undefined
    \else
      \ltx@IfUndefined{HoLogo@\HOLOGO@name @\HOLOGO@temp}{%
        \@PackageError{hologo}{%
          Unknown variant `\HOLOGO@temp' of logo `\HOLOGO@name'%
        }\@ehc
      }{%
        \expandafter
        \let\csname HoLogoOpt@variant@\HOLOGO@name\endcsname
            \HOLOGO@temp
      }%
    \fi
  \fi
}
%    \end{macrocode}
%
%    \begin{macro}{\HOLOGO@Variant}
%    \begin{macrocode}
\def\HOLOGO@Variant#1{%
  #1%
  \ltx@ifundefined{HoLogoOpt@variant@#1}{%
  }{%
    @\csname HoLogoOpt@variant@#1\endcsname
  }%
}
%    \end{macrocode}
%    \end{macro}
%
% \subsection{Break/no-break support}
%
%    \begin{macro}{\HOLOGO@space}
%    \begin{macrocode}
\def\HOLOGO@space{%
  \ltx@ifundefined{HoLogoOpt@spacebreak@\HOLOGO@name}{%
    \ltx@ifundefined{HoLogoOpt@break@\HOLOGO@name}{%
      \chardef\HOLOGO@temp=\HOLOGOOPT@spacebreak
    }{%
      \chardef\HOLOGO@temp=%
        \csname HoLogoOpt@break@\HOLOGO@name\endcsname
    }%
  }{%
    \chardef\HOLOGO@temp=%
      \csname HoLogoOpt@spacebreak@\HOLOGO@name\endcsname
  }%
  \ifcase\HOLOGO@temp
    \penalty10000 %
  \fi
  \ltx@space
}
%    \end{macrocode}
%    \end{macro}
%
%    \begin{macro}{\HOLOGO@hyphen}
%    \begin{macrocode}
\def\HOLOGO@hyphen{%
  \ltx@ifundefined{HoLogoOpt@hyphenbreak@\HOLOGO@name}{%
    \ltx@ifundefined{HoLogoOpt@break@\HOLOGO@name}{%
      \chardef\HOLOGO@temp=\HOLOGOOPT@hyphenbreak
    }{%
      \chardef\HOLOGO@temp=%
        \csname HoLogoOpt@break@\HOLOGO@name\endcsname
    }%
  }{%
    \chardef\HOLOGO@temp=%
      \csname HoLogoOpt@hyphenbreak@\HOLOGO@name\endcsname
  }%
  \ifcase\HOLOGO@temp
    \ltx@mbox{-}%
  \else
    -%
  \fi
}
%    \end{macrocode}
%    \end{macro}
%
%    \begin{macro}{\HOLOGO@discretionary}
%    \begin{macrocode}
\def\HOLOGO@discretionary{%
  \ltx@ifundefined{HoLogoOpt@discretionarybreak@\HOLOGO@name}{%
    \ltx@ifundefined{HoLogoOpt@break@\HOLOGO@name}{%
      \chardef\HOLOGO@temp=\HOLOGOOPT@discretionarybreak
    }{%
      \chardef\HOLOGO@temp=%
        \csname HoLogoOpt@break@\HOLOGO@name\endcsname
    }%
  }{%
    \chardef\HOLOGO@temp=%
      \csname HoLogoOpt@discretionarybreak@\HOLOGO@name\endcsname
  }%
  \ifcase\HOLOGO@temp
  \else
    \-%
  \fi
}
%    \end{macrocode}
%    \end{macro}
%
%    \begin{macro}{\HOLOGO@mbox}
%    \begin{macrocode}
\def\HOLOGO@mbox#1{%
  \ltx@ifundefined{HoLogoOpt@break@\HOLOGO@name}{%
    \chardef\HOLOGO@temp=\HOLOGOOPT@hyphenbreak
  }{%
    \chardef\HOLOGO@temp=%
      \csname HoLogoOpt@break@\HOLOGO@name\endcsname
  }%
  \ifcase\HOLOGO@temp
    \ltx@mbox{#1}%
  \else
    #1%
  \fi
}
%    \end{macrocode}
%    \end{macro}
%
% \subsection{Font support}
%
%    \begin{macro}{\HoLogoFont@font}
%    \begin{tabular}{@{}ll@{}}
%    |#1|:& logo name\\
%    |#2|:& font short name\\
%    |#3|:& text
%    \end{tabular}
%    \begin{macrocode}
\def\HoLogoFont@font#1#2#3{%
  \begingroup
    \ltx@IfUndefined{HoLogoFont@logo@#1.#2}{%
      \ltx@IfUndefined{HoLogoFont@font@#2}{%
        \@PackageWarning{hologo}{%
          Missing font `#2' for logo `#1'%
        }%
        #3%
      }{%
        \csname HoLogoFont@font@#2\endcsname{#3}%
      }%
    }{%
      \csname HoLogoFont@logo@#1.#2\endcsname{#3}%
    }%
  \endgroup
}
%    \end{macrocode}
%    \end{macro}
%
%    \begin{macro}{\HoLogoFont@Def}
%    \begin{macrocode}
\def\HoLogoFont@Def#1{%
  \expandafter\def\csname HoLogoFont@font@#1\endcsname
}
%    \end{macrocode}
%    \end{macro}
%    \begin{macro}{\HoLogoFont@LogoDef}
%    \begin{macrocode}
\def\HoLogoFont@LogoDef#1#2{%
  \expandafter\def\csname HoLogoFont@logo@#1.#2\endcsname
}
%    \end{macrocode}
%    \end{macro}
%
% \subsubsection{Font defaults}
%
%    \begin{macro}{\HoLogoFont@font@general}
%    \begin{macrocode}
\HoLogoFont@Def{general}{}%
%    \end{macrocode}
%    \end{macro}
%
%    \begin{macro}{\HoLogoFont@font@rm}
%    \begin{macrocode}
\ltx@IfUndefined{rmfamily}{%
  \ltx@IfUndefined{rm}{%
  }{%
    \HoLogoFont@Def{rm}{\rm}%
  }%
}{%
  \HoLogoFont@Def{rm}{\rmfamily}%
}
%    \end{macrocode}
%    \end{macro}
%
%    \begin{macro}{\HoLogoFont@font@sf}
%    \begin{macrocode}
\ltx@IfUndefined{sffamily}{%
  \ltx@IfUndefined{sf}{%
  }{%
    \HoLogoFont@Def{sf}{\sf}%
  }%
}{%
  \HoLogoFont@Def{sf}{\sffamily}%
}
%    \end{macrocode}
%    \end{macro}
%
%    \begin{macro}{\HoLogoFont@font@bibsf}
%    In case of \hologo{plainTeX} the original small caps
%    variant is used as default. In \hologo{LaTeX}
%    the definition of package \xpackage{dtklogos} \cite{dtklogos}
%    is used.
%\begin{quote}
%\begin{verbatim}
%\DeclareRobustCommand{\BibTeX}{%
%  B%
%  \kern-.05em%
%  \hbox{%
%    $\m@th$% %% force math size calculations
%    \csname S@\f@size\endcsname
%    \fontsize\sf@size\z@
%    \math@fontsfalse
%    \selectfont
%    I%
%    \kern-.025em%
%    B
%  }%
%  \kern-.08em%
%  \-%
%  \TeX
%}
%\end{verbatim}
%\end{quote}
%    \begin{macrocode}
\ltx@IfUndefined{selectfont}{%
  \ltx@IfUndefined{tensc}{%
    \font\tensc=cmcsc10\relax
  }{}%
  \HoLogoFont@Def{bibsf}{\tensc}%
}{%
  \HoLogoFont@Def{bibsf}{%
    $\mathsurround=0pt$%
    \csname S@\f@size\endcsname
    \fontsize\sf@size{0pt}%
    \math@fontsfalse
    \selectfont
  }%
}
%    \end{macrocode}
%    \end{macro}
%
%    \begin{macro}{\HoLogoFont@font@sc}
%    \begin{macrocode}
\ltx@IfUndefined{scshape}{%
  \ltx@IfUndefined{tensc}{%
    \font\tensc=cmcsc10\relax
  }{}%
  \HoLogoFont@Def{sc}{\tensc}%
}{%
  \HoLogoFont@Def{sc}{\scshape}%
}
%    \end{macrocode}
%    \end{macro}
%
%    \begin{macro}{\HoLogoFont@font@sy}
%    \begin{macrocode}
\ltx@IfUndefined{usefont}{%
  \ltx@IfUndefined{tensy}{%
  }{%
    \HoLogoFont@Def{sy}{\tensy}%
  }%
}{%
  \HoLogoFont@Def{sy}{%
    \usefont{OMS}{cmsy}{m}{n}%
  }%
}
%    \end{macrocode}
%    \end{macro}
%
%    \begin{macro}{\HoLogoFont@font@logo}
%    \begin{macrocode}
\begingroup
  \def\x{LaTeX2e}%
\expandafter\endgroup
\ifx\fmtname\x
  \ltx@IfUndefined{logofamily}{%
    \DeclareRobustCommand\logofamily{%
      \not@math@alphabet\logofamily\relax
      \fontencoding{U}%
      \fontfamily{logo}%
      \selectfont
    }%
  }{}%
  \ltx@IfUndefined{logofamily}{%
  }{%
    \HoLogoFont@Def{logo}{\logofamily}%
  }%
\else
  \ltx@IfUndefined{tenlogo}{%
    \font\tenlogo=logo10\relax
  }{}%
  \HoLogoFont@Def{logo}{\tenlogo}%
\fi
%    \end{macrocode}
%    \end{macro}
%
% \subsubsection{Font setup}
%
%    \begin{macro}{\hologoFontSetup}
%    \begin{macrocode}
\def\hologoFontSetup{%
  \let\HOLOGO@name\relax
  \HOLOGO@FontSetup
}
%    \end{macrocode}
%    \end{macro}
%
%    \begin{macro}{\hologoLogoFontSetup}
%    \begin{macrocode}
\def\hologoLogoFontSetup#1{%
  \edef\HOLOGO@name{#1}%
  \ltx@IfUndefined{HoLogo@\HOLOGO@name}{%
    \@PackageError{hologo}{%
      Unknown logo `\HOLOGO@name'%
    }\@ehc
    \ltx@gobble
  }{%
    \HOLOGO@FontSetup
  }%
}
%    \end{macrocode}
%    \end{macro}
%
%    \begin{macro}{\HOLOGO@FontSetup}
%    \begin{macrocode}
\def\HOLOGO@FontSetup{%
  \kvsetkeys{HoLogoFont}%
}
%    \end{macrocode}
%    \end{macro}
%
%    \begin{macrocode}
\def\HOLOGO@temp#1{%
  \kv@define@key{HoLogoFont}{#1}{%
    \ifx\HOLOGO@name\relax
      \HoLogoFont@Def{#1}{##1}%
    \else
      \HoLogoFont@LogoDef\HOLOGO@name{#1}{##1}%
    \fi
  }%
}
\HOLOGO@temp{general}
\HOLOGO@temp{sf}
%    \end{macrocode}
%
% \subsection{Generic logo commands}
%
%    \begin{macrocode}
\HOLOGO@IfExists\hologo{%
  \@PackageError{hologo}{%
    \string\hologo\ltx@space is already defined.\MessageBreak
    Package loading is aborted%
  }\@ehc
  \HOLOGO@AtEnd
}%
\HOLOGO@IfExists\hologoRobust{%
  \@PackageError{hologo}{%
    \string\hologoRobust\ltx@space is already defined.\MessageBreak
    Package loading is aborted%
  }\@ehc
  \HOLOGO@AtEnd
}%
%    \end{macrocode}
%
% \subsubsection{\cs{hologo} and friends}
%
%    \begin{macrocode}
\ifluatex
  \expandafter\ltx@firstofone
\else
  \expandafter\ltx@gobble
\fi
{%
  \ltx@IfUndefined{ifincsname}{%
    \ifnum\luatexversion<36 %
      \expandafter\ltx@gobble
    \else
      \expandafter\ltx@firstofone
    \fi
    {%
      \begingroup
        \ifcase0%
            \directlua{%
              if tex.enableprimitives then %
                tex.enableprimitives('HOLOGO@', {'ifincsname'})%
              else %
                tex.print('1')%
              end%
            }%
            \ifx\HOLOGO@ifincsname\@undefined 1\fi%
            \relax
          \expandafter\ltx@firstofone
        \else
          \endgroup
          \expandafter\ltx@gobble
        \fi
        {%
          \global\let\ifincsname\HOLOGO@ifincsname
        }%
      \HOLOGO@temp
    }%
  }{}%
}
%    \end{macrocode}
%    \begin{macrocode}
\ltx@IfUndefined{ifincsname}{%
  \catcode`$=14 %
}{%
  \catcode`$=9 %
}
%    \end{macrocode}
%
%    \begin{macro}{\hologo}
%    \begin{macrocode}
\def\hologo#1{%
$ \ifincsname
$   \ltx@ifundefined{HoLogoCs@\HOLOGO@Variant{#1}}{%
$     #1%
$   }{%
$     \csname HoLogoCs@\HOLOGO@Variant{#1}\endcsname\ltx@firstoftwo
$   }%
$ \else
    \HOLOGO@IfExists\texorpdfstring\texorpdfstring\ltx@firstoftwo
    {%
      \hologoRobust{#1}%
    }{%
      \ltx@ifundefined{HoLogoBkm@\HOLOGO@Variant{#1}}{%
        \ltx@ifundefined{HoLogo@#1}{?#1?}{#1}%
      }{%
        \csname HoLogoBkm@\HOLOGO@Variant{#1}\endcsname
        \ltx@firstoftwo
      }%
    }%
$ \fi
}
%    \end{macrocode}
%    \end{macro}
%    \begin{macro}{\Hologo}
%    \begin{macrocode}
\def\Hologo#1{%
$ \ifincsname
$   \ltx@ifundefined{HoLogoCs@\HOLOGO@Variant{#1}}{%
$     #1%
$   }{%
$     \csname HoLogoCs@\HOLOGO@Variant{#1}\endcsname\ltx@secondoftwo
$   }%
$ \else
    \HOLOGO@IfExists\texorpdfstring\texorpdfstring\ltx@firstoftwo
    {%
      \HologoRobust{#1}%
    }{%
      \ltx@ifundefined{HoLogoBkm@\HOLOGO@Variant{#1}}{%
        \ltx@ifundefined{HoLogo@#1}{?#1?}{#1}%
      }{%
        \csname HoLogoBkm@\HOLOGO@Variant{#1}\endcsname
        \ltx@secondoftwo
      }%
    }%
$ \fi
}
%    \end{macrocode}
%    \end{macro}
%
%    \begin{macro}{\hologoVariant}
%    \begin{macrocode}
\def\hologoVariant#1#2{%
  \ifx\relax#2\relax
    \hologo{#1}%
  \else
$   \ifincsname
$     \ltx@ifundefined{HoLogoCs@#1@#2}{%
$       #1%
$     }{%
$       \csname HoLogoCs@#1@#2\endcsname\ltx@firstoftwo
$     }%
$   \else
      \HOLOGO@IfExists\texorpdfstring\texorpdfstring\ltx@firstoftwo
      {%
        \hologoVariantRobust{#1}{#2}%
      }{%
        \ltx@ifundefined{HoLogoBkm@#1@#2}{%
          \ltx@ifundefined{HoLogo@#1}{?#1?}{#1}%
        }{%
          \csname HoLogoBkm@#1@#2\endcsname
          \ltx@firstoftwo
        }%
      }%
$   \fi
  \fi
}
%    \end{macrocode}
%    \end{macro}
%    \begin{macro}{\HologoVariant}
%    \begin{macrocode}
\def\HologoVariant#1#2{%
  \ifx\relax#2\relax
    \Hologo{#1}%
  \else
$   \ifincsname
$     \ltx@ifundefined{HoLogoCs@#1@#2}{%
$       #1%
$     }{%
$       \csname HoLogoCs@#1@#2\endcsname\ltx@secondoftwo
$     }%
$   \else
      \HOLOGO@IfExists\texorpdfstring\texorpdfstring\ltx@firstoftwo
      {%
        \HologoVariantRobust{#1}{#2}%
      }{%
        \ltx@ifundefined{HoLogoBkm@#1@#2}{%
          \ltx@ifundefined{HoLogo@#1}{?#1?}{#1}%
        }{%
          \csname HoLogoBkm@#1@#2\endcsname
          \ltx@secondoftwo
        }%
      }%
$   \fi
  \fi
}
%    \end{macrocode}
%    \end{macro}
%
%    \begin{macrocode}
\catcode`\$=3 %
%    \end{macrocode}
%
% \subsubsection{\cs{hologoRobust} and friends}
%
%    \begin{macro}{\hologoRobust}
%    \begin{macrocode}
\ltx@IfUndefined{protected}{%
  \ltx@IfUndefined{DeclareRobustCommand}{%
    \def\hologoRobust#1%
  }{%
    \DeclareRobustCommand*\hologoRobust[1]%
  }%
}{%
  \protected\def\hologoRobust#1%
}%
{%
  \edef\HOLOGO@name{#1}%
  \ltx@IfUndefined{HoLogo@\HOLOGO@Variant\HOLOGO@name}{%
    \@PackageError{hologo}{%
      Unknown logo `\HOLOGO@name'%
    }\@ehc
    ?\HOLOGO@name?%
  }{%
    \ltx@IfUndefined{ver@tex4ht.sty}{%
      \HoLogoFont@font\HOLOGO@name{general}{%
        \csname HoLogo@\HOLOGO@Variant\HOLOGO@name\endcsname
        \ltx@firstoftwo
      }%
    }{%
      \ltx@IfUndefined{HoLogoHtml@\HOLOGO@Variant\HOLOGO@name}{%
        \HOLOGO@name
      }{%
        \csname HoLogoHtml@\HOLOGO@Variant\HOLOGO@name\endcsname
        \ltx@firstoftwo
      }%
    }%
  }%
}
%    \end{macrocode}
%    \end{macro}
%    \begin{macro}{\HologoRobust}
%    \begin{macrocode}
\ltx@IfUndefined{protected}{%
  \ltx@IfUndefined{DeclareRobustCommand}{%
    \def\HologoRobust#1%
  }{%
    \DeclareRobustCommand*\HologoRobust[1]%
  }%
}{%
  \protected\def\HologoRobust#1%
}%
{%
  \edef\HOLOGO@name{#1}%
  \ltx@IfUndefined{HoLogo@\HOLOGO@Variant\HOLOGO@name}{%
    \@PackageError{hologo}{%
      Unknown logo `\HOLOGO@name'%
    }\@ehc
    ?\HOLOGO@name?%
  }{%
    \ltx@IfUndefined{ver@tex4ht.sty}{%
      \HoLogoFont@font\HOLOGO@name{general}{%
        \csname HoLogo@\HOLOGO@Variant\HOLOGO@name\endcsname
        \ltx@secondoftwo
      }%
    }{%
      \ltx@IfUndefined{HoLogoHtml@\HOLOGO@Variant\HOLOGO@name}{%
        \expandafter\HOLOGO@Uppercase\HOLOGO@name
      }{%
        \csname HoLogoHtml@\HOLOGO@Variant\HOLOGO@name\endcsname
        \ltx@secondoftwo
      }%
    }%
  }%
}
%    \end{macrocode}
%    \end{macro}
%    \begin{macro}{\hologoVariantRobust}
%    \begin{macrocode}
\ltx@IfUndefined{protected}{%
  \ltx@IfUndefined{DeclareRobustCommand}{%
    \def\hologoVariantRobust#1#2%
  }{%
    \DeclareRobustCommand*\hologoVariantRobust[2]%
  }%
}{%
  \protected\def\hologoVariantRobust#1#2%
}%
{%
  \begingroup
    \hologoLogoSetup{#1}{variant={#2}}%
    \hologoRobust{#1}%
  \endgroup
}
%    \end{macrocode}
%    \end{macro}
%    \begin{macro}{\HologoVariantRobust}
%    \begin{macrocode}
\ltx@IfUndefined{protected}{%
  \ltx@IfUndefined{DeclareRobustCommand}{%
    \def\HologoVariantRobust#1#2%
  }{%
    \DeclareRobustCommand*\HologoVariantRobust[2]%
  }%
}{%
  \protected\def\HologoVariantRobust#1#2%
}%
{%
  \begingroup
    \hologoLogoSetup{#1}{variant={#2}}%
    \HologoRobust{#1}%
  \endgroup
}
%    \end{macrocode}
%    \end{macro}
%
%    \begin{macro}{\hologorobust}
%    Macro \cs{hologorobust} is only defined for compatibility.
%    Its use is deprecated.
%    \begin{macrocode}
\def\hologorobust{\hologoRobust}
%    \end{macrocode}
%    \end{macro}
%
% \subsection{Helpers}
%
%    \begin{macro}{\HOLOGO@Uppercase}
%    Macro \cs{HOLOGO@Uppercase} is restricted to \cs{uppercase},
%    because \hologo{plainTeX} or \hologo{iniTeX} do not provide
%    \cs{MakeUppercase}.
%    \begin{macrocode}
\def\HOLOGO@Uppercase#1{\uppercase{#1}}
%    \end{macrocode}
%    \end{macro}
%
%    \begin{macro}{\HOLOGO@PdfdocUnicode}
%    \begin{macrocode}
\def\HOLOGO@PdfdocUnicode{%
  \ifx\ifHy@unicode\iftrue
    \expandafter\ltx@secondoftwo
  \else
    \expandafter\ltx@firstoftwo
  \fi
}
%    \end{macrocode}
%    \end{macro}
%
%    \begin{macro}{\HOLOGO@Math}
%    \begin{macrocode}
\def\HOLOGO@MathSetup{%
  \mathsurround0pt\relax
  \HOLOGO@IfExists\f@series{%
    \if b\expandafter\ltx@car\f@series x\@nil
      \csname boldmath\endcsname
   \fi
  }{}%
}
%    \end{macrocode}
%    \end{macro}
%
%    \begin{macro}{\HOLOGO@TempDimen}
%    \begin{macrocode}
\dimendef\HOLOGO@TempDimen=\ltx@zero
%    \end{macrocode}
%    \end{macro}
%    \begin{macro}{\HOLOGO@NegativeKerning}
%    \begin{macrocode}
\def\HOLOGO@NegativeKerning#1{%
  \begingroup
    \HOLOGO@TempDimen=0pt\relax
    \comma@parse@normalized{#1}{%
      \ifdim\HOLOGO@TempDimen=0pt %
        \expandafter\HOLOGO@@NegativeKerning\comma@entry
      \fi
      \ltx@gobble
    }%
    \ifdim\HOLOGO@TempDimen<0pt %
      \kern\HOLOGO@TempDimen
    \fi
  \endgroup
}
%    \end{macrocode}
%    \end{macro}
%    \begin{macro}{\HOLOGO@@NegativeKerning}
%    \begin{macrocode}
\def\HOLOGO@@NegativeKerning#1#2{%
  \setbox\ltx@zero\hbox{#1#2}%
  \HOLOGO@TempDimen=\wd\ltx@zero
  \setbox\ltx@zero\hbox{#1\kern0pt#2}%
  \advance\HOLOGO@TempDimen by -\wd\ltx@zero
}
%    \end{macrocode}
%    \end{macro}
%
%    \begin{macro}{\HOLOGO@SpaceFactor}
%    \begin{macrocode}
\def\HOLOGO@SpaceFactor{%
  \spacefactor1000 %
}
%    \end{macrocode}
%    \end{macro}
%
%    \begin{macro}{\HOLOGO@Span}
%    \begin{macrocode}
\def\HOLOGO@Span#1#2{%
  \HCode{<span class="HoLogo-#1">}%
  #2%
  \HCode{</span>}%
}
%    \end{macrocode}
%    \end{macro}
%
% \subsubsection{Text subscript}
%
%    \begin{macro}{\HOLOGO@SubScript}%
%    \begin{macrocode}
\def\HOLOGO@SubScript#1{%
  \ltx@IfUndefined{textsubscript}{%
    \ltx@IfUndefined{text}{%
      \ltx@mbox{%
        \mathsurround=0pt\relax
        $%
          _{%
            \ltx@IfUndefined{sf@size}{%
              \mathrm{#1}%
            }{%
              \mbox{%
                \fontsize\sf@size{0pt}\selectfont
                #1%
              }%
            }%
          }%
        $%
      }%
    }{%
      \ltx@mbox{%
        \mathsurround=0pt\relax
        $_{\text{#1}}$%
      }%
    }%
  }{%
    \textsubscript{#1}%
  }%
}
%    \end{macrocode}
%    \end{macro}
%
% \subsection{\hologo{TeX} and friends}
%
% \subsubsection{\hologo{TeX}}
%
%    \begin{macro}{\HoLogo@TeX}
%    Source: \hologo{LaTeX} kernel.
%    \begin{macrocode}
\def\HoLogo@TeX#1{%
  T\kern-.1667em\lower.5ex\hbox{E}\kern-.125emX\HOLOGO@SpaceFactor
}
%    \end{macrocode}
%    \end{macro}
%    \begin{macro}{\HoLogoHtml@TeX}
%    \begin{macrocode}
\def\HoLogoHtml@TeX#1{%
  \HoLogoCss@TeX
  \HOLOGO@Span{TeX}{%
    T%
    \HOLOGO@Span{e}{%
      E%
    }%
    X%
  }%
}
%    \end{macrocode}
%    \end{macro}
%    \begin{macro}{\HoLogoCss@TeX}
%    \begin{macrocode}
\def\HoLogoCss@TeX{%
  \Css{%
    span.HoLogo-TeX span.HoLogo-e{%
      position:relative;%
      top:.5ex;%
      margin-left:-.1667em;%
      margin-right:-.125em;%
    }%
  }%
  \Css{%
    a span.HoLogo-TeX span.HoLogo-e{%
      text-decoration:none;%
    }%
  }%
  \global\let\HoLogoCss@TeX\relax
}
%    \end{macrocode}
%    \end{macro}
%
% \subsubsection{\hologo{plainTeX}}
%
%    \begin{macro}{\HoLogo@plainTeX@space}
%    Source: ``The \hologo{TeX}book''
%    \begin{macrocode}
\def\HoLogo@plainTeX@space#1{%
  \HOLOGO@mbox{#1{p}{P}lain}\HOLOGO@space\hologo{TeX}%
}
%    \end{macrocode}
%    \end{macro}
%    \begin{macro}{\HoLogoCs@plainTeX@space}
%    \begin{macrocode}
\def\HoLogoCs@plainTeX@space#1{#1{p}{P}lain TeX}%
%    \end{macrocode}
%    \end{macro}
%    \begin{macro}{\HoLogoBkm@plainTeX@space}
%    \begin{macrocode}
\def\HoLogoBkm@plainTeX@space#1{%
  #1{p}{P}lain \hologo{TeX}%
}
%    \end{macrocode}
%    \end{macro}
%    \begin{macro}{\HoLogoHtml@plainTeX@space}
%    \begin{macrocode}
\def\HoLogoHtml@plainTeX@space#1{%
  #1{p}{P}lain \hologo{TeX}%
}
%    \end{macrocode}
%    \end{macro}
%
%    \begin{macro}{\HoLogo@plainTeX@hyphen}
%    \begin{macrocode}
\def\HoLogo@plainTeX@hyphen#1{%
  \HOLOGO@mbox{#1{p}{P}lain}\HOLOGO@hyphen\hologo{TeX}%
}
%    \end{macrocode}
%    \end{macro}
%    \begin{macro}{\HoLogoCs@plainTeX@hyphen}
%    \begin{macrocode}
\def\HoLogoCs@plainTeX@hyphen#1{#1{p}{P}lain-TeX}
%    \end{macrocode}
%    \end{macro}
%    \begin{macro}{\HoLogoBkm@plainTeX@hyphen}
%    \begin{macrocode}
\def\HoLogoBkm@plainTeX@hyphen#1{%
  #1{p}{P}lain-\hologo{TeX}%
}
%    \end{macrocode}
%    \end{macro}
%    \begin{macro}{\HoLogoHtml@plainTeX@hyphen}
%    \begin{macrocode}
\def\HoLogoHtml@plainTeX@hyphen#1{%
  #1{p}{P}lain-\hologo{TeX}%
}
%    \end{macrocode}
%    \end{macro}
%
%    \begin{macro}{\HoLogo@plainTeX@runtogether}
%    \begin{macrocode}
\def\HoLogo@plainTeX@runtogether#1{%
  \HOLOGO@mbox{#1{p}{P}lain\hologo{TeX}}%
}
%    \end{macrocode}
%    \end{macro}
%    \begin{macro}{\HoLogoCs@plainTeX@runtogether}
%    \begin{macrocode}
\def\HoLogoCs@plainTeX@runtogether#1{#1{p}{P}lainTeX}
%    \end{macrocode}
%    \end{macro}
%    \begin{macro}{\HoLogoBkm@plainTeX@runtogether}
%    \begin{macrocode}
\def\HoLogoBkm@plainTeX@runtogether#1{%
  #1{p}{P}lain\hologo{TeX}%
}
%    \end{macrocode}
%    \end{macro}
%    \begin{macro}{\HoLogoHtml@plainTeX@runtogether}
%    \begin{macrocode}
\def\HoLogoHtml@plainTeX@runtogether#1{%
  #1{p}{P}lain\hologo{TeX}%
}
%    \end{macrocode}
%    \end{macro}
%
%    \begin{macro}{\HoLogo@plainTeX}
%    \begin{macrocode}
\def\HoLogo@plainTeX{\HoLogo@plainTeX@space}
%    \end{macrocode}
%    \end{macro}
%    \begin{macro}{\HoLogoCs@plainTeX}
%    \begin{macrocode}
\def\HoLogoCs@plainTeX{\HoLogoCs@plainTeX@space}
%    \end{macrocode}
%    \end{macro}
%    \begin{macro}{\HoLogoBkm@plainTeX}
%    \begin{macrocode}
\def\HoLogoBkm@plainTeX{\HoLogoBkm@plainTeX@space}
%    \end{macrocode}
%    \end{macro}
%    \begin{macro}{\HoLogoHtml@plainTeX}
%    \begin{macrocode}
\def\HoLogoHtml@plainTeX{\HoLogoHtml@plainTeX@space}
%    \end{macrocode}
%    \end{macro}
%
% \subsubsection{\hologo{LaTeX}}
%
%    Source: \hologo{LaTeX} kernel.
%\begin{quote}
%\begin{verbatim}
%\DeclareRobustCommand{\LaTeX}{%
%  L%
%  \kern-.36em%
%  {%
%    \sbox\z@ T%
%    \vbox to\ht\z@{%
%      \hbox{%
%        \check@mathfonts
%        \fontsize\sf@size\z@
%        \math@fontsfalse
%        \selectfont
%        A%
%      }%
%      \vss
%    }%
%  }%
%  \kern-.15em%
%  \TeX
%}
%\end{verbatim}
%\end{quote}
%
%    \begin{macro}{\HoLogo@La}
%    \begin{macrocode}
\def\HoLogo@La#1{%
  L%
  \kern-.36em%
  \begingroup
    \setbox\ltx@zero\hbox{T}%
    \vbox to\ht\ltx@zero{%
      \hbox{%
        \ltx@ifundefined{check@mathfonts}{%
          \csname sevenrm\endcsname
        }{%
          \check@mathfonts
          \fontsize\sf@size{0pt}%
          \math@fontsfalse\selectfont
        }%
        A%
      }%
      \vss
    }%
  \endgroup
}
%    \end{macrocode}
%    \end{macro}
%
%    \begin{macro}{\HoLogo@LaTeX}
%    Source: \hologo{LaTeX} kernel.
%    \begin{macrocode}
\def\HoLogo@LaTeX#1{%
  \hologo{La}%
  \kern-.15em%
  \hologo{TeX}%
}
%    \end{macrocode}
%    \end{macro}
%    \begin{macro}{\HoLogoHtml@LaTeX}
%    \begin{macrocode}
\def\HoLogoHtml@LaTeX#1{%
  \HoLogoCss@LaTeX
  \HOLOGO@Span{LaTeX}{%
    L%
    \HOLOGO@Span{a}{%
      A%
    }%
    \hologo{TeX}%
  }%
}
%    \end{macrocode}
%    \end{macro}
%    \begin{macro}{\HoLogoCss@LaTeX}
%    \begin{macrocode}
\def\HoLogoCss@LaTeX{%
  \Css{%
    span.HoLogo-LaTeX span.HoLogo-a{%
      position:relative;%
      top:-.5ex;%
      margin-left:-.36em;%
      margin-right:-.15em;%
      font-size:85\%;%
    }%
  }%
  \global\let\HoLogoCss@LaTeX\relax
}
%    \end{macrocode}
%    \end{macro}
%
% \subsubsection{\hologo{(La)TeX}}
%
%    \begin{macro}{\HoLogo@LaTeXTeX}
%    The kerning around the parentheses is taken
%    from package \xpackage{dtklogos} \cite{dtklogos}.
%\begin{quote}
%\begin{verbatim}
%\DeclareRobustCommand{\LaTeXTeX}{%
%  (%
%  \kern-.15em%
%  L%
%  \kern-.36em%
%  {%
%    \sbox\z@ T%
%    \vbox to\ht0{%
%      \hbox{%
%        $\m@th$%
%        \csname S@\f@size\endcsname
%        \fontsize\sf@size\z@
%        \math@fontsfalse
%        \selectfont
%        A%
%      }%
%      \vss
%    }%
%  }%
%  \kern-.2em%
%  )%
%  \kern-.15em%
%  \TeX
%}
%\end{verbatim}
%\end{quote}
%    \begin{macrocode}
\def\HoLogo@LaTeXTeX#1{%
  (%
  \kern-.15em%
  \hologo{La}%
  \kern-.2em%
  )%
  \kern-.15em%
  \hologo{TeX}%
}
%    \end{macrocode}
%    \end{macro}
%    \begin{macro}{\HoLogoBkm@LaTeXTeX}
%    \begin{macrocode}
\def\HoLogoBkm@LaTeXTeX#1{(La)TeX}
%    \end{macrocode}
%    \end{macro}
%
%    \begin{macro}{\HoLogo@(La)TeX}
%    \begin{macrocode}
\expandafter
\let\csname HoLogo@(La)TeX\endcsname\HoLogo@LaTeXTeX
%    \end{macrocode}
%    \end{macro}
%    \begin{macro}{\HoLogoBkm@(La)TeX}
%    \begin{macrocode}
\expandafter
\let\csname HoLogoBkm@(La)TeX\endcsname\HoLogoBkm@LaTeXTeX
%    \end{macrocode}
%    \end{macro}
%    \begin{macro}{\HoLogoHtml@LaTeXTeX}
%    \begin{macrocode}
\def\HoLogoHtml@LaTeXTeX#1{%
  \HoLogoCss@LaTeXTeX
  \HOLOGO@Span{LaTeXTeX}{%
    (%
    \HOLOGO@Span{L}{L}%
    \HOLOGO@Span{a}{A}%
    \HOLOGO@Span{ParenRight}{)}%
    \hologo{TeX}%
  }%
}
%    \end{macrocode}
%    \end{macro}
%    \begin{macro}{\HoLogoHtml@(La)TeX}
%    Kerning after opening parentheses and before closing parentheses
%    is $-0.1$\,em. The original values $-0.15$\,em
%    looked too ugly for a serif font.
%    \begin{macrocode}
\expandafter
\let\csname HoLogoHtml@(La)TeX\endcsname\HoLogoHtml@LaTeXTeX
%    \end{macrocode}
%    \end{macro}
%    \begin{macro}{\HoLogoCss@LaTeXTeX}
%    \begin{macrocode}
\def\HoLogoCss@LaTeXTeX{%
  \Css{%
    span.HoLogo-LaTeXTeX span.HoLogo-L{%
      margin-left:-.1em;%
    }%
  }%
  \Css{%
    span.HoLogo-LaTeXTeX span.HoLogo-a{%
      position:relative;%
      top:-.5ex;%
      margin-left:-.36em;%
      margin-right:-.1em;%
      font-size:85\%;%
    }%
  }%
  \Css{%
    span.HoLogo-LaTeXTeX span.HoLogo-ParenRight{%
      margin-right:-.15em;%
    }%
  }%
  \global\let\HoLogoCss@LaTeXTeX\relax
}
%    \end{macrocode}
%    \end{macro}
%
% \subsubsection{\hologo{LaTeXe}}
%
%    \begin{macro}{\HoLogo@LaTeXe}
%    Source: \hologo{LaTeX} kernel
%    \begin{macrocode}
\def\HoLogo@LaTeXe#1{%
  \hologo{LaTeX}%
  \kern.15em%
  \hbox{%
    \HOLOGO@MathSetup
    2%
    $_{\textstyle\varepsilon}$%
  }%
}
%    \end{macrocode}
%    \end{macro}
%
%    \begin{macro}{\HoLogoCs@LaTeXe}
%    \begin{macrocode}
\ifnum64=`\^^^^0040\relax % test for big chars of LuaTeX/XeTeX
  \catcode`\$=9 %
  \catcode`\&=14 %
\else
  \catcode`\$=14 %
  \catcode`\&=9 %
\fi
\def\HoLogoCs@LaTeXe#1{%
  LaTeX2%
$ \string ^^^^0395%
& e%
}%
\catcode`\$=3 %
\catcode`\&=4 %
%    \end{macrocode}
%    \end{macro}
%
%    \begin{macro}{\HoLogoBkm@LaTeXe}
%    \begin{macrocode}
\def\HoLogoBkm@LaTeXe#1{%
  \hologo{LaTeX}%
  2%
  \HOLOGO@PdfdocUnicode{e}{\textepsilon}%
}
%    \end{macrocode}
%    \end{macro}
%
%    \begin{macro}{\HoLogoHtml@LaTeXe}
%    \begin{macrocode}
\def\HoLogoHtml@LaTeXe#1{%
  \HoLogoCss@LaTeXe
  \HOLOGO@Span{LaTeX2e}{%
    \hologo{LaTeX}%
    \HOLOGO@Span{2}{2}%
    \HOLOGO@Span{e}{%
      \HOLOGO@MathSetup
      \ensuremath{\textstyle\varepsilon}%
    }%
  }%
}
%    \end{macrocode}
%    \end{macro}
%    \begin{macro}{\HoLogoCss@LaTeXe}
%    \begin{macrocode}
\def\HoLogoCss@LaTeXe{%
  \Css{%
    span.HoLogo-LaTeX2e span.HoLogo-2{%
      padding-left:.15em;%
    }%
  }%
  \Css{%
    span.HoLogo-LaTeX2e span.HoLogo-e{%
      position:relative;%
      top:.35ex;%
      text-decoration:none;%
    }%
  }%
  \global\let\HoLogoCss@LaTeXe\relax
}
%    \end{macrocode}
%    \end{macro}
%
%    \begin{macro}{\HoLogo@LaTeX2e}
%    \begin{macrocode}
\expandafter
\let\csname HoLogo@LaTeX2e\endcsname\HoLogo@LaTeXe
%    \end{macrocode}
%    \end{macro}
%    \begin{macro}{\HoLogoCs@LaTeX2e}
%    \begin{macrocode}
\expandafter
\let\csname HoLogoCs@LaTeX2e\endcsname\HoLogoCs@LaTeXe
%    \end{macrocode}
%    \end{macro}
%    \begin{macro}{\HoLogoBkm@LaTeX2e}
%    \begin{macrocode}
\expandafter
\let\csname HoLogoBkm@LaTeX2e\endcsname\HoLogoBkm@LaTeXe
%    \end{macrocode}
%    \end{macro}
%    \begin{macro}{\HoLogoHtml@LaTeX2e}
%    \begin{macrocode}
\expandafter
\let\csname HoLogoHtml@LaTeX2e\endcsname\HoLogoHtml@LaTeXe
%    \end{macrocode}
%    \end{macro}
%
% \subsubsection{\hologo{LaTeX3}}
%
%    \begin{macro}{\HoLogo@LaTeX3}
%    Source: \hologo{LaTeX} kernel
%    \begin{macrocode}
\expandafter\def\csname HoLogo@LaTeX3\endcsname#1{%
  \hologo{LaTeX}%
  3%
}
%    \end{macrocode}
%    \end{macro}
%
%    \begin{macro}{\HoLogoBkm@LaTeX3}
%    \begin{macrocode}
\expandafter\def\csname HoLogoBkm@LaTeX3\endcsname#1{%
  \hologo{LaTeX}%
  3%
}
%    \end{macrocode}
%    \end{macro}
%    \begin{macro}{\HoLogoHtml@LaTeX3}
%    \begin{macrocode}
\expandafter
\let\csname HoLogoHtml@LaTeX3\expandafter\endcsname
\csname HoLogo@LaTeX3\endcsname
%    \end{macrocode}
%    \end{macro}
%
% \subsubsection{\hologo{LaTeXML}}
%
%    \begin{macro}{\HoLogo@LaTeXML}
%    \begin{macrocode}
\def\HoLogo@LaTeXML#1{%
  \HOLOGO@mbox{%
    \hologo{La}%
    \kern-.15em%
    T%
    \kern-.1667em%
    \lower.5ex\hbox{E}%
    \kern-.125em%
    \HoLogoFont@font{LaTeXML}{sc}{xml}%
  }%
}
%    \end{macrocode}
%    \end{macro}
%    \begin{macro}{\HoLogoHtml@pdfLaTeX}
%    \begin{macrocode}
\def\HoLogoHtml@LaTeXML#1{%
  \HOLOGO@Span{LaTeXML}{%
    \HoLogoCss@LaTeX
    \HoLogoCss@TeX
    \HOLOGO@Span{LaTeX}{%
      L%
      \HOLOGO@Span{a}{%
        A%
      }%
    }%
    \HOLOGO@Span{TeX}{%
      T%
      \HOLOGO@Span{e}{%
        E%
      }%
    }%
    \HCode{<span style="font-variant: small-caps;">}%
    xml%
    \HCode{</span>}%
  }%
}
%    \end{macrocode}
%    \end{macro}
%
% \subsubsection{\hologo{eTeX}}
%
%    \begin{macro}{\HoLogo@eTeX}
%    Source: package \xpackage{etex}
%    \begin{macrocode}
\def\HoLogo@eTeX#1{%
  \ltx@mbox{%
    \HOLOGO@MathSetup
    $\varepsilon$%
    -%
    \HOLOGO@NegativeKerning{-T,T-,To}%
    \hologo{TeX}%
  }%
}
%    \end{macrocode}
%    \end{macro}
%    \begin{macro}{\HoLogoCs@eTeX}
%    \begin{macrocode}
\ifnum64=`\^^^^0040\relax % test for big chars of LuaTeX/XeTeX
  \catcode`\$=9 %
  \catcode`\&=14 %
\else
  \catcode`\$=14 %
  \catcode`\&=9 %
\fi
\def\HoLogoCs@eTeX#1{%
$ #1{\string ^^^^0395}{\string ^^^^03b5}%
& #1{e}{E}%
  TeX%
}%
\catcode`\$=3 %
\catcode`\&=4 %
%    \end{macrocode}
%    \end{macro}
%    \begin{macro}{\HoLogoBkm@eTeX}
%    \begin{macrocode}
\def\HoLogoBkm@eTeX#1{%
  \HOLOGO@PdfdocUnicode{#1{e}{E}}{\textepsilon}%
  -%
  \hologo{TeX}%
}
%    \end{macrocode}
%    \end{macro}
%    \begin{macro}{\HoLogoHtml@eTeX}
%    \begin{macrocode}
\def\HoLogoHtml@eTeX#1{%
  \ltx@mbox{%
    \HOLOGO@MathSetup
    $\varepsilon$%
    -%
    \hologo{TeX}%
  }%
}
%    \end{macrocode}
%    \end{macro}
%
% \subsubsection{\hologo{iniTeX}}
%
%    \begin{macro}{\HoLogo@iniTeX}
%    \begin{macrocode}
\def\HoLogo@iniTeX#1{%
  \HOLOGO@mbox{%
    #1{i}{I}ni\hologo{TeX}%
  }%
}
%    \end{macrocode}
%    \end{macro}
%    \begin{macro}{\HoLogoCs@iniTeX}
%    \begin{macrocode}
\def\HoLogoCs@iniTeX#1{#1{i}{I}niTeX}
%    \end{macrocode}
%    \end{macro}
%    \begin{macro}{\HoLogoBkm@iniTeX}
%    \begin{macrocode}
\def\HoLogoBkm@iniTeX#1{%
  #1{i}{I}ni\hologo{TeX}%
}
%    \end{macrocode}
%    \end{macro}
%    \begin{macro}{\HoLogoHtml@iniTeX}
%    \begin{macrocode}
\let\HoLogoHtml@iniTeX\HoLogo@iniTeX
%    \end{macrocode}
%    \end{macro}
%
% \subsubsection{\hologo{virTeX}}
%
%    \begin{macro}{\HoLogo@virTeX}
%    \begin{macrocode}
\def\HoLogo@virTeX#1{%
  \HOLOGO@mbox{%
    #1{v}{V}ir\hologo{TeX}%
  }%
}
%    \end{macrocode}
%    \end{macro}
%    \begin{macro}{\HoLogoCs@virTeX}
%    \begin{macrocode}
\def\HoLogoCs@virTeX#1{#1{v}{V}irTeX}
%    \end{macrocode}
%    \end{macro}
%    \begin{macro}{\HoLogoBkm@virTeX}
%    \begin{macrocode}
\def\HoLogoBkm@virTeX#1{%
  #1{v}{V}ir\hologo{TeX}%
}
%    \end{macrocode}
%    \end{macro}
%    \begin{macro}{\HoLogoHtml@virTeX}
%    \begin{macrocode}
\let\HoLogoHtml@virTeX\HoLogo@virTeX
%    \end{macrocode}
%    \end{macro}
%
% \subsubsection{\hologo{SliTeX}}
%
% \paragraph{Definitions of the three variants.}
%
%    \begin{macro}{\HoLogo@SLiTeX@lift}
%    \begin{macrocode}
\def\HoLogo@SLiTeX@lift#1{%
  \HoLogoFont@font{SliTeX}{rm}{%
    S%
    \kern-.06em%
    L%
    \kern-.18em%
    \raise.32ex\hbox{\HoLogoFont@font{SliTeX}{sc}{i}}%
    \HOLOGO@discretionary
    \kern-.06em%
    \hologo{TeX}%
  }%
}
%    \end{macrocode}
%    \end{macro}
%    \begin{macro}{\HoLogoBkm@SLiTeX@lift}
%    \begin{macrocode}
\def\HoLogoBkm@SLiTeX@lift#1{SLiTeX}
%    \end{macrocode}
%    \end{macro}
%    \begin{macro}{\HoLogoHtml@SLiTeX@lift}
%    \begin{macrocode}
\def\HoLogoHtml@SLiTeX@lift#1{%
  \HoLogoCss@SLiTeX@lift
  \HOLOGO@Span{SLiTeX-lift}{%
    \HoLogoFont@font{SliTeX}{rm}{%
      S%
      \HOLOGO@Span{L}{L}%
      \HOLOGO@Span{i}{i}%
      \hologo{TeX}%
    }%
  }%
}
%    \end{macrocode}
%    \end{macro}
%    \begin{macro}{\HoLogoCss@SLiTeX@lift}
%    \begin{macrocode}
\def\HoLogoCss@SLiTeX@lift{%
  \Css{%
    span.HoLogo-SLiTeX-lift span.HoLogo-L{%
      margin-left:-.06em;%
      margin-right:-.18em;%
    }%
  }%
  \Css{%
    span.HoLogo-SLiTeX-lift span.HoLogo-i{%
      position:relative;%
      top:-.32ex;%
      margin-right:-.06em;%
      font-variant:small-caps;%
    }%
  }%
  \global\let\HoLogoCss@SLiTeX@lift\relax
}
%    \end{macrocode}
%    \end{macro}
%
%    \begin{macro}{\HoLogo@SliTeX@simple}
%    \begin{macrocode}
\def\HoLogo@SliTeX@simple#1{%
  \HoLogoFont@font{SliTeX}{rm}{%
    \ltx@mbox{%
      \HoLogoFont@font{SliTeX}{sc}{Sli}%
    }%
    \HOLOGO@discretionary
    \hologo{TeX}%
  }%
}
%    \end{macrocode}
%    \end{macro}
%    \begin{macro}{\HoLogoBkm@SliTeX@simple}
%    \begin{macrocode}
\def\HoLogoBkm@SliTeX@simple#1{SliTeX}
%    \end{macrocode}
%    \end{macro}
%    \begin{macro}{\HoLogoHtml@SliTeX@simple}
%    \begin{macrocode}
\let\HoLogoHtml@SliTeX@simple\HoLogo@SliTeX@simple
%    \end{macrocode}
%    \end{macro}
%
%    \begin{macro}{\HoLogo@SliTeX@narrow}
%    \begin{macrocode}
\def\HoLogo@SliTeX@narrow#1{%
  \HoLogoFont@font{SliTeX}{rm}{%
    \ltx@mbox{%
      S%
      \kern-.06em%
      \HoLogoFont@font{SliTeX}{sc}{%
        l%
        \kern-.035em%
        i%
      }%
    }%
    \HOLOGO@discretionary
    \kern-.06em%
    \hologo{TeX}%
  }%
}
%    \end{macrocode}
%    \end{macro}
%    \begin{macro}{\HoLogoBkm@SliTeX@narrow}
%    \begin{macrocode}
\def\HoLogoBkm@SliTeX@narrow#1{SliTeX}
%    \end{macrocode}
%    \end{macro}
%    \begin{macro}{\HoLogoHtml@SliTeX@narrow}
%    \begin{macrocode}
\def\HoLogoHtml@SliTeX@narrow#1{%
  \HoLogoCss@SliTeX@narrow
  \HOLOGO@Span{SliTeX-narrow}{%
    \HoLogoFont@font{SliTeX}{rm}{%
      S%
        \HOLOGO@Span{l}{l}%
        \HOLOGO@Span{i}{i}%
      \hologo{TeX}%
    }%
  }%
}
%    \end{macrocode}
%    \end{macro}
%    \begin{macro}{\HoLogoCss@SliTeX@narrow}
%    \begin{macrocode}
\def\HoLogoCss@SliTeX@narrow{%
  \Css{%
    span.HoLogo-SliTeX-narrow span.HoLogo-l{%
      margin-left:-.06em;%
      margin-right:-.035em;%
      font-variant:small-caps;%
    }%
  }%
  \Css{%
    span.HoLogo-SliTeX-narrow span.HoLogo-i{%
      margin-right:-.06em;%
      font-variant:small-caps;%
    }%
  }%
  \global\let\HoLogoCss@SliTeX@narrow\relax
}
%    \end{macrocode}
%    \end{macro}
%
% \paragraph{Macro set completion.}
%
%    \begin{macro}{\HoLogo@SLiTeX@simple}
%    \begin{macrocode}
\def\HoLogo@SLiTeX@simple{\HoLogo@SliTeX@simple}
%    \end{macrocode}
%    \end{macro}
%    \begin{macro}{\HoLogoBkm@SLiTeX@simple}
%    \begin{macrocode}
\def\HoLogoBkm@SLiTeX@simple{\HoLogoBkm@SliTeX@simple}
%    \end{macrocode}
%    \end{macro}
%    \begin{macro}{\HoLogoHtml@SLiTeX@simple}
%    \begin{macrocode}
\def\HoLogoHtml@SLiTeX@simple{\HoLogoHtml@SliTeX@simple}
%    \end{macrocode}
%    \end{macro}
%
%    \begin{macro}{\HoLogo@SLiTeX@narrow}
%    \begin{macrocode}
\def\HoLogo@SLiTeX@narrow{\HoLogo@SliTeX@narrow}
%    \end{macrocode}
%    \end{macro}
%    \begin{macro}{\HoLogoBkm@SLiTeX@narrow}
%    \begin{macrocode}
\def\HoLogoBkm@SLiTeX@narrow{\HoLogoBkm@SliTeX@narrow}
%    \end{macrocode}
%    \end{macro}
%    \begin{macro}{\HoLogoHtml@SLiTeX@narrow}
%    \begin{macrocode}
\def\HoLogoHtml@SLiTeX@narrow{\HoLogoHtml@SliTeX@narrow}
%    \end{macrocode}
%    \end{macro}
%
%    \begin{macro}{\HoLogo@SliTeX@lift}
%    \begin{macrocode}
\def\HoLogo@SliTeX@lift{\HoLogo@SLiTeX@lift}
%    \end{macrocode}
%    \end{macro}
%    \begin{macro}{\HoLogoBkm@SliTeX@lift}
%    \begin{macrocode}
\def\HoLogoBkm@SliTeX@lift{\HoLogoBkm@SLiTeX@lift}
%    \end{macrocode}
%    \end{macro}
%    \begin{macro}{\HoLogoHtml@SliTeX@lift}
%    \begin{macrocode}
\def\HoLogoHtml@SliTeX@lift{\HoLogoHtml@SLiTeX@lift}
%    \end{macrocode}
%    \end{macro}
%
% \paragraph{Defaults.}
%
%    \begin{macro}{\HoLogo@SLiTeX}
%    \begin{macrocode}
\def\HoLogo@SLiTeX{\HoLogo@SLiTeX@lift}
%    \end{macrocode}
%    \end{macro}
%    \begin{macro}{\HoLogoBkm@SLiTeX}
%    \begin{macrocode}
\def\HoLogoBkm@SLiTeX{\HoLogoBkm@SLiTeX@lift}
%    \end{macrocode}
%    \end{macro}
%    \begin{macro}{\HoLogoHtml@SLiTeX}
%    \begin{macrocode}
\def\HoLogoHtml@SLiTeX{\HoLogoHtml@SLiTeX@lift}
%    \end{macrocode}
%    \end{macro}
%
%    \begin{macro}{\HoLogo@SliTeX}
%    \begin{macrocode}
\def\HoLogo@SliTeX{\HoLogo@SliTeX@narrow}
%    \end{macrocode}
%    \end{macro}
%    \begin{macro}{\HoLogoBkm@SliTeX}
%    \begin{macrocode}
\def\HoLogoBkm@SliTeX{\HoLogoBkm@SliTeX@narrow}
%    \end{macrocode}
%    \end{macro}
%    \begin{macro}{\HoLogoHtml@SliTeX}
%    \begin{macrocode}
\def\HoLogoHtml@SliTeX{\HoLogoHtml@SliTeX@narrow}
%    \end{macrocode}
%    \end{macro}
%
% \subsubsection{\hologo{LuaTeX}}
%
%    \begin{macro}{\HoLogo@LuaTeX}
%    The kerning is an idea of Hans Hagen, see mailing list
%    `luatex at tug dot org' in March 2010.
%    \begin{macrocode}
\def\HoLogo@LuaTeX#1{%
  \HOLOGO@mbox{%
    Lua%
    \HOLOGO@NegativeKerning{aT,oT,To}%
    \hologo{TeX}%
  }%
}
%    \end{macrocode}
%    \end{macro}
%    \begin{macro}{\HoLogoHtml@LuaTeX}
%    \begin{macrocode}
\let\HoLogoHtml@LuaTeX\HoLogo@LuaTeX
%    \end{macrocode}
%    \end{macro}
%
% \subsubsection{\hologo{LuaLaTeX}}
%
%    \begin{macro}{\HoLogo@LuaLaTeX}
%    \begin{macrocode}
\def\HoLogo@LuaLaTeX#1{%
  \HOLOGO@mbox{%
    Lua%
    \hologo{LaTeX}%
  }%
}
%    \end{macrocode}
%    \end{macro}
%    \begin{macro}{\HoLogoHtml@LuaLaTeX}
%    \begin{macrocode}
\let\HoLogoHtml@LuaLaTeX\HoLogo@LuaLaTeX
%    \end{macrocode}
%    \end{macro}
%
% \subsubsection{\hologo{XeTeX}, \hologo{XeLaTeX}}
%
%    \begin{macro}{\HOLOGO@IfCharExists}
%    \begin{macrocode}
\ifluatex
  \ifnum\luatexversion<36 %
  \else
    \def\HOLOGO@IfCharExists#1{%
      \ifnum
        \directlua{%
           if luaotfload and luaotfload.aux then
             if luaotfload.aux.font_has_glyph(%
                    font.current(), \number#1) then % 	 
	       tex.print("1") % 	 
	     end % 	 
	   elseif font and font.fonts and font.current then %
            local f = font.fonts[font.current()]%
            if f.characters and f.characters[\number#1] then %
              tex.print("1")%
            end %
          end%
        }0=\ltx@zero
        \expandafter\ltx@secondoftwo
      \else
        \expandafter\ltx@firstoftwo
      \fi
    }%
  \fi
\fi
\ltx@IfUndefined{HOLOGO@IfCharExists}{%
  \def\HOLOGO@@IfCharExists#1{%
    \begingroup
      \tracinglostchars=\ltx@zero
      \setbox\ltx@zero=\hbox{%
        \kern7sp\char#1\relax
        \ifnum\lastkern>\ltx@zero
          \expandafter\aftergroup\csname iffalse\endcsname
        \else
          \expandafter\aftergroup\csname iftrue\endcsname
        \fi
      }%
      % \if{true|false} from \aftergroup
      \endgroup
      \expandafter\ltx@firstoftwo
    \else
      \endgroup
      \expandafter\ltx@secondoftwo
    \fi
  }%
  \ifxetex
    \ltx@IfUndefined{XeTeXfonttype}{}{%
      \ltx@IfUndefined{XeTeXcharglyph}{}{%
        \def\HOLOGO@IfCharExists#1{%
          \ifnum\XeTeXfonttype\font>\ltx@zero
            \expandafter\ltx@firstofthree
          \else
            \expandafter\ltx@gobble
          \fi
          {%
            \ifnum\XeTeXcharglyph#1>\ltx@zero
              \expandafter\ltx@firstoftwo
            \else
              \expandafter\ltx@secondoftwo
            \fi
          }%
          \HOLOGO@@IfCharExists{#1}%
        }%
      }%
    }%
  \fi
}{}
\ltx@ifundefined{HOLOGO@IfCharExists}{%
  \ifnum64=`\^^^^0040\relax % test for big chars of LuaTeX/XeTeX
    \let\HOLOGO@IfCharExists\HOLOGO@@IfCharExists
  \else
    \def\HOLOGO@IfCharExists#1{%
      \ifnum#1>255 %
        \expandafter\ltx@fourthoffour
      \fi
      \HOLOGO@@IfCharExists{#1}%
    }%
  \fi
}{}
%    \end{macrocode}
%    \end{macro}
%
%    \begin{macro}{\HoLogo@Xe}
%    Source: package \xpackage{dtklogos}
%    \begin{macrocode}
\def\HoLogo@Xe#1{%
  X%
  \kern-.1em\relax
  \HOLOGO@IfCharExists{"018E}{%
    \lower.5ex\hbox{\char"018E}%
  }{%
    \chardef\HOLOGO@choice=\ltx@zero
    \ifdim\fontdimen\ltx@one\font>0pt %
      \ltx@IfUndefined{rotatebox}{%
        \ltx@IfUndefined{pgftext}{%
          \ltx@IfUndefined{psscalebox}{%
            \ltx@IfUndefined{HOLOGO@ScaleBox@\hologoDriver}{%
            }{%
              \chardef\HOLOGO@choice=4 %
            }%
          }{%
            \chardef\HOLOGO@choice=3 %
          }%
        }{%
          \chardef\HOLOGO@choice=2 %
        }%
      }{%
        \chardef\HOLOGO@choice=1 %
      }%
      \ifcase\HOLOGO@choice
        \HOLOGO@WarningUnsupportedDriver{Xe}%
        e%
      \or % 1: \rotatebox
        \begingroup
          \setbox\ltx@zero\hbox{\rotatebox{180}{E}}%
          \ltx@LocDimenA=\dp\ltx@zero
          \advance\ltx@LocDimenA by -.5ex\relax
          \raise\ltx@LocDimenA\box\ltx@zero
        \endgroup
      \or % 2: \pgftext
        \lower.5ex\hbox{%
          \pgfpicture
            \pgftext[rotate=180]{E}%
          \endpgfpicture
        }%
      \or % 3: \psscalebox
        \begingroup
          \setbox\ltx@zero\hbox{\psscalebox{-1 -1}{E}}%
          \ltx@LocDimenA=\dp\ltx@zero
          \advance\ltx@LocDimenA by -.5ex\relax
          \raise\ltx@LocDimenA\box\ltx@zero
        \endgroup
      \or % 4: \HOLOGO@PointReflectBox
        \lower.5ex\hbox{\HOLOGO@PointReflectBox{E}}%
      \else
        \@PackageError{hologo}{Internal error (choice/it}\@ehc
      \fi
    \else
      \ltx@IfUndefined{reflectbox}{%
        \ltx@IfUndefined{pgftext}{%
          \ltx@IfUndefined{psscalebox}{%
            \ltx@IfUndefined{HOLOGO@ScaleBox@\hologoDriver}{%
            }{%
              \chardef\HOLOGO@choice=4 %
            }%
          }{%
            \chardef\HOLOGO@choice=3 %
          }%
        }{%
          \chardef\HOLOGO@choice=2 %
        }%
      }{%
        \chardef\HOLOGO@choice=1 %
      }%
      \ifcase\HOLOGO@choice
        \HOLOGO@WarningUnsupportedDriver{Xe}%
        e%
      \or % 1: reflectbox
        \lower.5ex\hbox{%
          \reflectbox{E}%
        }%
      \or % 2: \pgftext
        \lower.5ex\hbox{%
          \pgfpicture
            \pgftransformxscale{-1}%
            \pgftext{E}%
          \endpgfpicture
        }%
      \or % 3: \psscalebox
        \lower.5ex\hbox{%
          \psscalebox{-1 1}{E}%
        }%
      \or % 4: \HOLOGO@Reflectbox
        \lower.5ex\hbox{%
          \HOLOGO@ReflectBox{E}%
        }%
      \else
        \@PackageError{hologo}{Internal error (choice/up)}\@ehc
      \fi
    \fi
  }%
}
%    \end{macrocode}
%    \end{macro}
%    \begin{macro}{\HoLogoHtml@Xe}
%    \begin{macrocode}
\def\HoLogoHtml@Xe#1{%
  \HoLogoCss@Xe
  \HOLOGO@Span{Xe}{%
    X%
    \HOLOGO@Span{e}{%
      \HCode{&\ltx@hashchar x018e;}%
    }%
  }%
}
%    \end{macrocode}
%    \end{macro}
%    \begin{macro}{\HoLogoCss@Xe}
%    \begin{macrocode}
\def\HoLogoCss@Xe{%
  \Css{%
    span.HoLogo-Xe span.HoLogo-e{%
      position:relative;%
      top:.5ex;%
      left-margin:-.1em;%
    }%
  }%
  \global\let\HoLogoCss@Xe\relax
}
%    \end{macrocode}
%    \end{macro}
%
%    \begin{macro}{\HoLogo@XeTeX}
%    \begin{macrocode}
\def\HoLogo@XeTeX#1{%
  \hologo{Xe}%
  \kern-.15em\relax
  \hologo{TeX}%
}
%    \end{macrocode}
%    \end{macro}
%
%    \begin{macro}{\HoLogoHtml@XeTeX}
%    \begin{macrocode}
\def\HoLogoHtml@XeTeX#1{%
  \HoLogoCss@XeTeX
  \HOLOGO@Span{XeTeX}{%
    \hologo{Xe}%
    \hologo{TeX}%
  }%
}
%    \end{macrocode}
%    \end{macro}
%    \begin{macro}{\HoLogoCss@XeTeX}
%    \begin{macrocode}
\def\HoLogoCss@XeTeX{%
  \Css{%
    span.HoLogo-XeTeX span.HoLogo-TeX{%
      margin-left:-.15em;%
    }%
  }%
  \global\let\HoLogoCss@XeTeX\relax
}
%    \end{macrocode}
%    \end{macro}
%
%    \begin{macro}{\HoLogo@XeLaTeX}
%    \begin{macrocode}
\def\HoLogo@XeLaTeX#1{%
  \hologo{Xe}%
  \kern-.13em%
  \hologo{LaTeX}%
}
%    \end{macrocode}
%    \end{macro}
%    \begin{macro}{\HoLogoHtml@XeLaTeX}
%    \begin{macrocode}
\def\HoLogoHtml@XeLaTeX#1{%
  \HoLogoCss@XeLaTeX
  \HOLOGO@Span{XeLaTeX}{%
    \hologo{Xe}%
    \hologo{LaTeX}%
  }%
}
%    \end{macrocode}
%    \end{macro}
%    \begin{macro}{\HoLogoCss@XeLaTeX}
%    \begin{macrocode}
\def\HoLogoCss@XeLaTeX{%
  \Css{%
    span.HoLogo-XeLaTeX span.HoLogo-Xe{%
      margin-right:-.13em;%
    }%
  }%
  \global\let\HoLogoCss@XeLaTeX\relax
}
%    \end{macrocode}
%    \end{macro}
%
% \subsubsection{\hologo{pdfTeX}, \hologo{pdfLaTeX}}
%
%    \begin{macro}{\HoLogo@pdfTeX}
%    \begin{macrocode}
\def\HoLogo@pdfTeX#1{%
  \HOLOGO@mbox{%
    #1{p}{P}df\hologo{TeX}%
  }%
}
%    \end{macrocode}
%    \end{macro}
%    \begin{macro}{\HoLogoCs@pdfTeX}
%    \begin{macrocode}
\def\HoLogoCs@pdfTeX#1{#1{p}{P}dfTeX}
%    \end{macrocode}
%    \end{macro}
%    \begin{macro}{\HoLogoBkm@pdfTeX}
%    \begin{macrocode}
\def\HoLogoBkm@pdfTeX#1{%
  #1{p}{P}df\hologo{TeX}%
}
%    \end{macrocode}
%    \end{macro}
%    \begin{macro}{\HoLogoHtml@pdfTeX}
%    \begin{macrocode}
\let\HoLogoHtml@pdfTeX\HoLogo@pdfTeX
%    \end{macrocode}
%    \end{macro}
%
%    \begin{macro}{\HoLogo@pdfLaTeX}
%    \begin{macrocode}
\def\HoLogo@pdfLaTeX#1{%
  \HOLOGO@mbox{%
    #1{p}{P}df\hologo{LaTeX}%
  }%
}
%    \end{macrocode}
%    \end{macro}
%    \begin{macro}{\HoLogoCs@pdfLaTeX}
%    \begin{macrocode}
\def\HoLogoCs@pdfLaTeX#1{#1{p}{P}dfLaTeX}
%    \end{macrocode}
%    \end{macro}
%    \begin{macro}{\HoLogoBkm@pdfLaTeX}
%    \begin{macrocode}
\def\HoLogoBkm@pdfLaTeX#1{%
  #1{p}{P}df\hologo{LaTeX}%
}
%    \end{macrocode}
%    \end{macro}
%    \begin{macro}{\HoLogoHtml@pdfLaTeX}
%    \begin{macrocode}
\let\HoLogoHtml@pdfLaTeX\HoLogo@pdfLaTeX
%    \end{macrocode}
%    \end{macro}
%
% \subsubsection{\hologo{VTeX}}
%
%    \begin{macro}{\HoLogo@VTeX}
%    \begin{macrocode}
\def\HoLogo@VTeX#1{%
  \HOLOGO@mbox{%
    V\hologo{TeX}%
  }%
}
%    \end{macrocode}
%    \end{macro}
%    \begin{macro}{\HoLogoHtml@VTeX}
%    \begin{macrocode}
\let\HoLogoHtml@VTeX\HoLogo@VTeX
%    \end{macrocode}
%    \end{macro}
%
% \subsubsection{\hologo{AmS}, \dots}
%
%    Source: class \xclass{amsdtx}
%
%    \begin{macro}{\HoLogo@AmS}
%    \begin{macrocode}
\def\HoLogo@AmS#1{%
  \HoLogoFont@font{AmS}{sy}{%
    A%
    \kern-.1667em%
    \lower.5ex\hbox{M}%
    \kern-.125em%
    S%
  }%
}
%    \end{macrocode}
%    \end{macro}
%    \begin{macro}{\HoLogoBkm@AmS}
%    \begin{macrocode}
\def\HoLogoBkm@AmS#1{AmS}
%    \end{macrocode}
%    \end{macro}
%    \begin{macro}{\HoLogoHtml@AmS}
%    \begin{macrocode}
\def\HoLogoHtml@AmS#1{%
  \HoLogoCss@AmS
%  \HoLogoFont@font{AmS}{sy}{%
    \HOLOGO@Span{AmS}{%
      A%
      \HOLOGO@Span{M}{M}%
      S%
    }%
%   }%
}
%    \end{macrocode}
%    \end{macro}
%    \begin{macro}{\HoLogoCss@AmS}
%    \begin{macrocode}
\def\HoLogoCss@AmS{%
  \Css{%
    span.HoLogo-AmS span.HoLogo-M{%
      position:relative;%
      top:.5ex;%
      margin-left:-.1667em;%
      margin-right:-.125em;%
      text-decoration:none;%
    }%
  }%
  \global\let\HoLogoCss@AmS\relax
}
%    \end{macrocode}
%    \end{macro}
%
%    \begin{macro}{\HoLogo@AmSTeX}
%    \begin{macrocode}
\def\HoLogo@AmSTeX#1{%
  \hologo{AmS}%
  \HOLOGO@hyphen
  \hologo{TeX}%
}
%    \end{macrocode}
%    \end{macro}
%    \begin{macro}{\HoLogoBkm@AmSTeX}
%    \begin{macrocode}
\def\HoLogoBkm@AmSTeX#1{AmS-TeX}%
%    \end{macrocode}
%    \end{macro}
%    \begin{macro}{\HoLogoHtml@AmSTeX}
%    \begin{macrocode}
\let\HoLogoHtml@AmSTeX\HoLogo@AmSTeX
%    \end{macrocode}
%    \end{macro}
%
%    \begin{macro}{\HoLogo@AmSLaTeX}
%    \begin{macrocode}
\def\HoLogo@AmSLaTeX#1{%
  \hologo{AmS}%
  \HOLOGO@hyphen
  \hologo{LaTeX}%
}
%    \end{macrocode}
%    \end{macro}
%    \begin{macro}{\HoLogoBkm@AmSLaTeX}
%    \begin{macrocode}
\def\HoLogoBkm@AmSLaTeX#1{AmS-LaTeX}%
%    \end{macrocode}
%    \end{macro}
%    \begin{macro}{\HoLogoHtml@AmSLaTeX}
%    \begin{macrocode}
\let\HoLogoHtml@AmSLaTeX\HoLogo@AmSLaTeX
%    \end{macrocode}
%    \end{macro}
%
% \subsubsection{\hologo{BibTeX}}
%
%    \begin{macro}{\HoLogo@BibTeX@sc}
%    A definition of \hologo{BibTeX} is provided in
%    the documentation source for the manual of \hologo{BibTeX}
%    \cite{btxdoc}.
%\begin{quote}
%\begin{verbatim}
%\def\BibTeX{%
%  {%
%    \rm
%    B%
%    \kern-.05em%
%    {%
%      \sc
%      i%
%      \kern-.025em %
%      b%
%    }%
%    \kern-.08em
%    T%
%    \kern-.1667em%
%    \lower.7ex\hbox{E}%
%    \kern-.125em%
%    X%
%  }%
%}
%\end{verbatim}
%\end{quote}
%    \begin{macrocode}
\def\HoLogo@BibTeX@sc#1{%
  B%
  \kern-.05em%
  \HoLogoFont@font{BibTeX}{sc}{%
    i%
    \kern-.025em%
    b%
  }%
  \HOLOGO@discretionary
  \kern-.08em%
  \hologo{TeX}%
}
%    \end{macrocode}
%    \end{macro}
%    \begin{macro}{\HoLogoHtml@BibTeX@sc}
%    \begin{macrocode}
\def\HoLogoHtml@BibTeX@sc#1{%
  \HoLogoCss@BibTeX@sc
  \HOLOGO@Span{BibTeX-sc}{%
    B%
    \HOLOGO@Span{i}{i}%
    \HOLOGO@Span{b}{b}%
    \hologo{TeX}%
  }%
}
%    \end{macrocode}
%    \end{macro}
%    \begin{macro}{\HoLogoCss@BibTeX@sc}
%    \begin{macrocode}
\def\HoLogoCss@BibTeX@sc{%
  \Css{%
    span.HoLogo-BibTeX-sc span.HoLogo-i{%
      margin-left:-.05em;%
      margin-right:-.025em;%
      font-variant:small-caps;%
    }%
  }%
  \Css{%
    span.HoLogo-BibTeX-sc span.HoLogo-b{%
      margin-right:-.08em;%
      font-variant:small-caps;%
    }%
  }%
  \global\let\HoLogoCss@BibTeX@sc\relax
}
%    \end{macrocode}
%    \end{macro}
%
%    \begin{macro}{\HoLogo@BibTeX@sf}
%    Variant \xoption{sf} avoids trouble with unavailable
%    small caps fonts (e.g., bold versions of Computer Modern or
%    Latin Modern). The definition is taken from
%    package \xpackage{dtklogos} \cite{dtklogos}.
%\begin{quote}
%\begin{verbatim}
%\DeclareRobustCommand{\BibTeX}{%
%  B%
%  \kern-.05em%
%  \hbox{%
%    $\m@th$% %% force math size calculations
%    \csname S@\f@size\endcsname
%    \fontsize\sf@size\z@
%    \math@fontsfalse
%    \selectfont
%    I%
%    \kern-.025em%
%    B
%  }%
%  \kern-.08em%
%  \-%
%  \TeX
%}
%\end{verbatim}
%\end{quote}
%    \begin{macrocode}
\def\HoLogo@BibTeX@sf#1{%
  B%
  \kern-.05em%
  \HoLogoFont@font{BibTeX}{bibsf}{%
    I%
    \kern-.025em%
    B%
  }%
  \HOLOGO@discretionary
  \kern-.08em%
  \hologo{TeX}%
}
%    \end{macrocode}
%    \end{macro}
%    \begin{macro}{\HoLogoHtml@BibTeX@sf}
%    \begin{macrocode}
\def\HoLogoHtml@BibTeX@sf#1{%
  \HoLogoCss@BibTeX@sf
  \HOLOGO@Span{BibTeX-sf}{%
    B%
    \HoLogoFont@font{BibTeX}{bibsf}{%
      \HOLOGO@Span{i}{I}%
      B%
    }%
    \hologo{TeX}%
  }%
}
%    \end{macrocode}
%    \end{macro}
%    \begin{macro}{\HoLogoCss@BibTeX@sf}
%    \begin{macrocode}
\def\HoLogoCss@BibTeX@sf{%
  \Css{%
    span.HoLogo-BibTeX-sf span.HoLogo-i{%
      margin-left:-.05em;%
      margin-right:-.025em;%
    }%
  }%
  \Css{%
    span.HoLogo-BibTeX-sf span.HoLogo-TeX{%
      margin-left:-.08em;%
    }%
  }%
  \global\let\HoLogoCss@BibTeX@sf\relax
}
%    \end{macrocode}
%    \end{macro}
%
%    \begin{macro}{\HoLogo@BibTeX}
%    \begin{macrocode}
\def\HoLogo@BibTeX{\HoLogo@BibTeX@sf}
%    \end{macrocode}
%    \end{macro}
%    \begin{macro}{\HoLogoHtml@BibTeX}
%    \begin{macrocode}
\def\HoLogoHtml@BibTeX{\HoLogoHtml@BibTeX@sf}
%    \end{macrocode}
%    \end{macro}
%
% \subsubsection{\hologo{BibTeX8}}
%
%    \begin{macro}{\HoLogo@BibTeX8}
%    \begin{macrocode}
\expandafter\def\csname HoLogo@BibTeX8\endcsname#1{%
  \hologo{BibTeX}%
  8%
}
%    \end{macrocode}
%    \end{macro}
%
%    \begin{macro}{\HoLogoBkm@BibTeX8}
%    \begin{macrocode}
\expandafter\def\csname HoLogoBkm@BibTeX8\endcsname#1{%
  \hologo{BibTeX}%
  8%
}
%    \end{macrocode}
%    \end{macro}
%    \begin{macro}{\HoLogoHtml@BibTeX8}
%    \begin{macrocode}
\expandafter
\let\csname HoLogoHtml@BibTeX8\expandafter\endcsname
\csname HoLogo@BibTeX8\endcsname
%    \end{macrocode}
%    \end{macro}
%
% \subsubsection{\hologo{ConTeXt}}
%
%    \begin{macro}{\HoLogo@ConTeXt@simple}
%    \begin{macrocode}
\def\HoLogo@ConTeXt@simple#1{%
  \HOLOGO@mbox{Con}%
  \HOLOGO@discretionary
  \HOLOGO@mbox{\hologo{TeX}t}%
}
%    \end{macrocode}
%    \end{macro}
%    \begin{macro}{\HoLogoHtml@ConTeXt@simple}
%    \begin{macrocode}
\let\HoLogoHtml@ConTeXt@simple\HoLogo@ConTeXt@simple
%    \end{macrocode}
%    \end{macro}
%
%    \begin{macro}{\HoLogo@ConTeXt@narrow}
%    This definition of logo \hologo{ConTeXt} with variant \xoption{narrow}
%    comes from TUGboat's class \xclass{ltugboat} (version 2010/11/15 v2.8).
%    \begin{macrocode}
\def\HoLogo@ConTeXt@narrow#1{%
  \HOLOGO@mbox{C\kern-.0333emon}%
  \HOLOGO@discretionary
  \kern-.0667em%
  \HOLOGO@mbox{\hologo{TeX}\kern-.0333emt}%
}
%    \end{macrocode}
%    \end{macro}
%    \begin{macro}{\HoLogoHtml@ConTeXt@narrow}
%    \begin{macrocode}
\def\HoLogoHtml@ConTeXt@narrow#1{%
  \HoLogoCss@ConTeXt@narrow
  \HOLOGO@Span{ConTeXt-narrow}{%
    \HOLOGO@Span{C}{C}%
    on%
    \hologo{TeX}%
    t%
  }%
}
%    \end{macrocode}
%    \end{macro}
%    \begin{macro}{\HoLogoCss@ConTeXt@narrow}
%    \begin{macrocode}
\def\HoLogoCss@ConTeXt@narrow{%
  \Css{%
    span.HoLogo-ConTeXt-narrow span.HoLogo-C{%
      margin-left:-.0333em;%
    }%
  }%
  \Css{%
    span.HoLogo-ConTeXt-narrow span.HoLogo-TeX{%
      margin-left:-.0667em;%
      margin-right:-.0333em;%
    }%
  }%
  \global\let\HoLogoCss@ConTeXt@narrow\relax
}
%    \end{macrocode}
%    \end{macro}
%
%    \begin{macro}{\HoLogo@ConTeXt}
%    \begin{macrocode}
\def\HoLogo@ConTeXt{\HoLogo@ConTeXt@narrow}
%    \end{macrocode}
%    \end{macro}
%    \begin{macro}{\HoLogoHtml@ConTeXt}
%    \begin{macrocode}
\def\HoLogoHtml@ConTeXt{\HoLogoHtml@ConTeXt@narrow}
%    \end{macrocode}
%    \end{macro}
%
% \subsubsection{\hologo{emTeX}}
%
%    \begin{macro}{\HoLogo@emTeX}
%    \begin{macrocode}
\def\HoLogo@emTeX#1{%
  \HOLOGO@mbox{#1{e}{E}m}%
  \HOLOGO@discretionary
  \hologo{TeX}%
}
%    \end{macrocode}
%    \end{macro}
%    \begin{macro}{\HoLogoCs@emTeX}
%    \begin{macrocode}
\def\HoLogoCs@emTeX#1{#1{e}{E}mTeX}%
%    \end{macrocode}
%    \end{macro}
%    \begin{macro}{\HoLogoBkm@emTeX}
%    \begin{macrocode}
\def\HoLogoBkm@emTeX#1{%
  #1{e}{E}m\hologo{TeX}%
}
%    \end{macrocode}
%    \end{macro}
%    \begin{macro}{\HoLogoHtml@emTeX}
%    \begin{macrocode}
\let\HoLogoHtml@emTeX\HoLogo@emTeX
%    \end{macrocode}
%    \end{macro}
%
% \subsubsection{\hologo{ExTeX}}
%
%    \begin{macro}{\HoLogo@ExTeX}
%    The definition is taken from the FAQ of the
%    project \hologo{ExTeX}
%    \cite{ExTeX-FAQ}.
%\begin{quote}
%\begin{verbatim}
%\def\ExTeX{%
%  \textrm{% Logo always with serifs
%    \ensuremath{%
%      \textstyle
%      \varepsilon_{%
%        \kern-0.15em%
%        \mathcal{X}%
%      }%
%    }%
%    \kern-.15em%
%    \TeX
%  }%
%}
%\end{verbatim}
%\end{quote}
%    \begin{macrocode}
\def\HoLogo@ExTeX#1{%
  \HoLogoFont@font{ExTeX}{rm}{%
    \ltx@mbox{%
      \HOLOGO@MathSetup
      $%
        \textstyle
        \varepsilon_{%
          \kern-0.15em%
          \HoLogoFont@font{ExTeX}{sy}{X}%
        }%
      $%
    }%
    \HOLOGO@discretionary
    \kern-.15em%
    \hologo{TeX}%
  }%
}
%    \end{macrocode}
%    \end{macro}
%    \begin{macro}{\HoLogoHtml@ExTeX}
%    \begin{macrocode}
\def\HoLogoHtml@ExTeX#1{%
  \HoLogoCss@ExTeX
  \HoLogoFont@font{ExTeX}{rm}{%
    \HOLOGO@Span{ExTeX}{%
      \ltx@mbox{%
        \HOLOGO@MathSetup
        $\textstyle\varepsilon$%
        \HOLOGO@Span{X}{$\textstyle\chi$}%
        \hologo{TeX}%
      }%
    }%
  }%
}
%    \end{macrocode}
%    \end{macro}
%    \begin{macro}{\HoLogoBkm@ExTeX}
%    \begin{macrocode}
\def\HoLogoBkm@ExTeX#1{%
  \HOLOGO@PdfdocUnicode{#1{e}{E}x}{\textepsilon\textchi}%
  \hologo{TeX}%
}
%    \end{macrocode}
%    \end{macro}
%    \begin{macro}{\HoLogoCss@ExTeX}
%    \begin{macrocode}
\def\HoLogoCss@ExTeX{%
  \Css{%
    span.HoLogo-ExTeX{%
      font-family:serif;%
    }%
  }%
  \Css{%
    span.HoLogo-ExTeX span.HoLogo-TeX{%
      margin-left:-.15em;%
    }%
  }%
  \global\let\HoLogoCss@ExTeX\relax
}
%    \end{macrocode}
%    \end{macro}
%
% \subsubsection{\hologo{MiKTeX}}
%
%    \begin{macro}{\HoLogo@MiKTeX}
%    \begin{macrocode}
\def\HoLogo@MiKTeX#1{%
  \HOLOGO@mbox{MiK}%
  \HOLOGO@discretionary
  \hologo{TeX}%
}
%    \end{macrocode}
%    \end{macro}
%    \begin{macro}{\HoLogoHtml@MiKTeX}
%    \begin{macrocode}
\let\HoLogoHtml@MiKTeX\HoLogo@MiKTeX
%    \end{macrocode}
%    \end{macro}
%
% \subsubsection{\hologo{OzTeX} and friends}
%
%    Source: \hologo{OzTeX} FAQ \cite{OzTeX}:
%    \begin{quote}
%      |\def\OzTeX{O\kern-.03em z\kern-.15em\TeX}|\\
%      (There is no kerning in OzMF, OzMP and OzTtH.)
%    \end{quote}
%
%    \begin{macro}{\HoLogo@OzTeX}
%    \begin{macrocode}
\def\HoLogo@OzTeX#1{%
  O%
  \kern-.03em %
  z%
  \kern-.15em %
  \hologo{TeX}%
}
%    \end{macrocode}
%    \end{macro}
%    \begin{macro}{\HoLogoHtml@OzTeX}
%    \begin{macrocode}
\def\HoLogoHtml@OzTeX#1{%
  \HoLogoCss@OzTeX
  \HOLOGO@Span{OzTeX}{%
    O%
    \HOLOGO@Span{z}{z}%
    \hologo{TeX}%
  }%
}
%    \end{macrocode}
%    \end{macro}
%    \begin{macro}{\HoLogoCss@OzTeX}
%    \begin{macrocode}
\def\HoLogoCss@OzTeX{%
  \Css{%
    span.HoLogo-OzTeX span.HoLogo-z{%
      margin-left:-.03em;%
      margin-right:-.15em;%
    }%
  }%
  \global\let\HoLogoCss@OzTeX\relax
}
%    \end{macrocode}
%    \end{macro}
%
%    \begin{macro}{\HoLogo@OzMF}
%    \begin{macrocode}
\def\HoLogo@OzMF#1{%
  \HOLOGO@mbox{OzMF}%
}
%    \end{macrocode}
%    \end{macro}
%    \begin{macro}{\HoLogo@OzMP}
%    \begin{macrocode}
\def\HoLogo@OzMP#1{%
  \HOLOGO@mbox{OzMP}%
}
%    \end{macrocode}
%    \end{macro}
%    \begin{macro}{\HoLogo@OzTtH}
%    \begin{macrocode}
\def\HoLogo@OzTtH#1{%
  \HOLOGO@mbox{OzTtH}%
}
%    \end{macrocode}
%    \end{macro}
%
% \subsubsection{\hologo{PCTeX}}
%
%    \begin{macro}{\HoLogo@PCTeX}
%    \begin{macrocode}
\def\HoLogo@PCTeX#1{%
  \HOLOGO@mbox{PC}%
  \hologo{TeX}%
}
%    \end{macrocode}
%    \end{macro}
%    \begin{macro}{\HoLogoHtml@PCTeX}
%    \begin{macrocode}
\let\HoLogoHtml@PCTeX\HoLogo@PCTeX
%    \end{macrocode}
%    \end{macro}
%
% \subsubsection{\hologo{PiCTeX}}
%
%    The original definitions from \xfile{pictex.tex} \cite{PiCTeX}:
%\begin{quote}
%\begin{verbatim}
%\def\PiC{%
%  P%
%  \kern-.12em%
%  \lower.5ex\hbox{I}%
%  \kern-.075em%
%  C%
%}
%\def\PiCTeX{%
%  \PiC
%  \kern-.11em%
%  \TeX
%}
%\end{verbatim}
%\end{quote}
%
%    \begin{macro}{\HoLogo@PiC}
%    \begin{macrocode}
\def\HoLogo@PiC#1{%
  P%
  \kern-.12em%
  \lower.5ex\hbox{I}%
  \kern-.075em%
  C%
  \HOLOGO@SpaceFactor
}
%    \end{macrocode}
%    \end{macro}
%    \begin{macro}{\HoLogoHtml@PiC}
%    \begin{macrocode}
\def\HoLogoHtml@PiC#1{%
  \HoLogoCss@PiC
  \HOLOGO@Span{PiC}{%
    P%
    \HOLOGO@Span{i}{I}%
    C%
  }%
}
%    \end{macrocode}
%    \end{macro}
%    \begin{macro}{\HoLogoCss@PiC}
%    \begin{macrocode}
\def\HoLogoCss@PiC{%
  \Css{%
    span.HoLogo-PiC span.HoLogo-i{%
      position:relative;%
      top:.5ex;%
      margin-left:-.12em;%
      margin-right:-.075em;%
      text-decoration:none;%
    }%
  }%
  \global\let\HoLogoCss@PiC\relax
}
%    \end{macrocode}
%    \end{macro}
%
%    \begin{macro}{\HoLogo@PiCTeX}
%    \begin{macrocode}
\def\HoLogo@PiCTeX#1{%
  \hologo{PiC}%
  \HOLOGO@discretionary
  \kern-.11em%
  \hologo{TeX}%
}
%    \end{macrocode}
%    \end{macro}
%    \begin{macro}{\HoLogoHtml@PiCTeX}
%    \begin{macrocode}
\def\HoLogoHtml@PiCTeX#1{%
  \HoLogoCss@PiCTeX
  \HOLOGO@Span{PiCTeX}{%
    \hologo{PiC}%
    \hologo{TeX}%
  }%
}
%    \end{macrocode}
%    \end{macro}
%    \begin{macro}{\HoLogoCss@PiCTeX}
%    \begin{macrocode}
\def\HoLogoCss@PiCTeX{%
  \Css{%
    span.HoLogo-PiCTeX span.HoLogo-PiC{%
      margin-right:-.11em;%
    }%
  }%
  \global\let\HoLogoCss@PiCTeX\relax
}
%    \end{macrocode}
%    \end{macro}
%
% \subsubsection{\hologo{teTeX}}
%
%    \begin{macro}{\HoLogo@teTeX}
%    \begin{macrocode}
\def\HoLogo@teTeX#1{%
  \HOLOGO@mbox{#1{t}{T}e}%
  \HOLOGO@discretionary
  \hologo{TeX}%
}
%    \end{macrocode}
%    \end{macro}
%    \begin{macro}{\HoLogoCs@teTeX}
%    \begin{macrocode}
\def\HoLogoCs@teTeX#1{#1{t}{T}dfTeX}
%    \end{macrocode}
%    \end{macro}
%    \begin{macro}{\HoLogoBkm@teTeX}
%    \begin{macrocode}
\def\HoLogoBkm@teTeX#1{%
  #1{t}{T}e\hologo{TeX}%
}
%    \end{macrocode}
%    \end{macro}
%    \begin{macro}{\HoLogoHtml@teTeX}
%    \begin{macrocode}
\let\HoLogoHtml@teTeX\HoLogo@teTeX
%    \end{macrocode}
%    \end{macro}
%
% \subsubsection{\hologo{TeX4ht}}
%
%    \begin{macro}{\HoLogo@TeX4ht}
%    \begin{macrocode}
\expandafter\def\csname HoLogo@TeX4ht\endcsname#1{%
  \HOLOGO@mbox{\hologo{TeX}4ht}%
}
%    \end{macrocode}
%    \end{macro}
%    \begin{macro}{\HoLogoHtml@TeX4ht}
%    \begin{macrocode}
\expandafter
\let\csname HoLogoHtml@TeX4ht\expandafter\endcsname
\csname HoLogo@TeX4ht\endcsname
%    \end{macrocode}
%    \end{macro}
%
%
% \subsubsection{\hologo{SageTeX}}
%
%    \begin{macro}{\HoLogo@SageTeX}
%    \begin{macrocode}
\def\HoLogo@SageTeX#1{%
  \HOLOGO@mbox{Sage}%
  \HOLOGO@discretionary
  \HOLOGO@NegativeKerning{eT,oT,To}%
  \hologo{TeX}%
}
%    \end{macrocode}
%    \end{macro}
%    \begin{macro}{\HoLogoHtml@SageTeX}
%    \begin{macrocode}
\let\HoLogoHtml@SageTeX\HoLogo@SageTeX
%    \end{macrocode}
%    \end{macro}
%
% \subsection{\hologo{METAFONT} and friends}
%
%    \begin{macro}{\HoLogo@METAFONT}
%    \begin{macrocode}
\def\HoLogo@METAFONT#1{%
  \HoLogoFont@font{METAFONT}{logo}{%
    \HOLOGO@mbox{META}%
    \HOLOGO@discretionary
    \HOLOGO@mbox{FONT}%
  }%
}
%    \end{macrocode}
%    \end{macro}
%
%    \begin{macro}{\HoLogo@METAPOST}
%    \begin{macrocode}
\def\HoLogo@METAPOST#1{%
  \HoLogoFont@font{METAPOST}{logo}{%
    \HOLOGO@mbox{META}%
    \HOLOGO@discretionary
    \HOLOGO@mbox{POST}%
  }%
}
%    \end{macrocode}
%    \end{macro}
%
%    \begin{macro}{\HoLogo@MetaFun}
%    \begin{macrocode}
\def\HoLogo@MetaFun#1{%
  \HOLOGO@mbox{Meta}%
  \HOLOGO@discretionary
  \HOLOGO@mbox{Fun}%
}
%    \end{macrocode}
%    \end{macro}
%
%    \begin{macro}{\HoLogo@MetaPost}
%    \begin{macrocode}
\def\HoLogo@MetaPost#1{%
  \HOLOGO@mbox{Meta}%
  \HOLOGO@discretionary
  \HOLOGO@mbox{Post}%
}
%    \end{macrocode}
%    \end{macro}
%
% \subsection{Others}
%
% \subsubsection{\hologo{biber}}
%
%    \begin{macro}{\HoLogo@biber}
%    \begin{macrocode}
\def\HoLogo@biber#1{%
  \HOLOGO@mbox{#1{b}{B}i}%
  \HOLOGO@discretionary
  \HOLOGO@mbox{ber}%
}
%    \end{macrocode}
%    \end{macro}
%    \begin{macro}{\HoLogoCs@biber}
%    \begin{macrocode}
\def\HoLogoCs@biber#1{#1{b}{B}iber}
%    \end{macrocode}
%    \end{macro}
%    \begin{macro}{\HoLogoBkm@biber}
%    \begin{macrocode}
\def\HoLogoBkm@biber#1{%
  #1{b}{B}iber%
}
%    \end{macrocode}
%    \end{macro}
%    \begin{macro}{\HoLogoHtml@biber}
%    \begin{macrocode}
\let\HoLogoHtml@biber\HoLogo@biber
%    \end{macrocode}
%    \end{macro}
%
% \subsubsection{\hologo{KOMAScript}}
%
%    \begin{macro}{\HoLogo@KOMAScript}
%    The definition for \hologo{KOMAScript} is taken
%    from \hologo{KOMAScript} (\xfile{scrlogo.dtx}, reformatted) \cite{scrlogo}:
%\begin{quote}
%\begin{verbatim}
%\@ifundefined{KOMAScript}{%
%  \DeclareRobustCommand{\KOMAScript}{%
%    \textsf{%
%      K\kern.05em O\kern.05emM\kern.05em A%
%      \kern.1em-\kern.1em %
%      Script%
%    }%
%  }%
%}{}
%\end{verbatim}
%\end{quote}
%    \begin{macrocode}
\def\HoLogo@KOMAScript#1{%
  \HoLogoFont@font{KOMAScript}{sf}{%
    \HOLOGO@mbox{%
      K\kern.05em%
      O\kern.05em%
      M\kern.05em%
      A%
    }%
    \kern.1em%
    \HOLOGO@hyphen
    \kern.1em%
    \HOLOGO@mbox{Script}%
  }%
}
%    \end{macrocode}
%    \end{macro}
%    \begin{macro}{\HoLogoBkm@KOMAScript}
%    \begin{macrocode}
\def\HoLogoBkm@KOMAScript#1{%
  KOMA-Script%
}
%    \end{macrocode}
%    \end{macro}
%    \begin{macro}{\HoLogoHtml@KOMAScript}
%    \begin{macrocode}
\def\HoLogoHtml@KOMAScript#1{%
  \HoLogoCss@KOMAScript
  \HoLogoFont@font{KOMAScript}{sf}{%
    \HOLOGO@Span{KOMAScript}{%
      K%
      \HOLOGO@Span{O}{O}%
      M%
      \HOLOGO@Span{A}{A}%
      \HOLOGO@Span{hyphen}{-}%
      Script%
    }%
  }%
}
%    \end{macrocode}
%    \end{macro}
%    \begin{macro}{\HoLogoCss@KOMAScript}
%    \begin{macrocode}
\def\HoLogoCss@KOMAScript{%
  \Css{%
    span.HoLogo-KOMAScript{%
      font-family:sans-serif;%
    }%
  }%
  \Css{%
    span.HoLogo-KOMAScript span.HoLogo-O{%
      padding-left:.05em;%
      padding-right:.05em;%
    }%
  }%
  \Css{%
    span.HoLogo-KOMAScript span.HoLogo-A{%
      padding-left:.05em;%
    }%
  }%
  \Css{%
    span.HoLogo-KOMAScript span.HoLogo-hyphen{%
      padding-left:.1em;%
      padding-right:.1em;%
    }%
  }%
  \global\let\HoLogoCss@KOMAScript\relax
}
%    \end{macrocode}
%    \end{macro}
%
% \subsubsection{\hologo{LyX}}
%
%    \begin{macro}{\HoLogo@LyX}
%    The definition is taken from the documentation source files
%    of \hologo{LyX}, \xfile{Intro.lyx} \cite{LyX}:
%\begin{quote}
%\begin{verbatim}
%\def\LyX{%
%  \texorpdfstring{%
%    L\kern-.1667em\lower.25em\hbox{Y}\kern-.125emX\@%
%  }{%
%    LyX%
%  }%
%}
%\end{verbatim}
%\end{quote}
%    \begin{macrocode}
\def\HoLogo@LyX#1{%
  L%
  \kern-.1667em%
  \lower.25em\hbox{Y}%
  \kern-.125em%
  X%
  \HOLOGO@SpaceFactor
}
%    \end{macrocode}
%    \end{macro}
%    \begin{macro}{\HoLogoHtml@LyX}
%    \begin{macrocode}
\def\HoLogoHtml@LyX#1{%
  \HoLogoCss@LyX
  \HOLOGO@Span{LyX}{%
    L%
    \HOLOGO@Span{y}{Y}%
    X%
  }%
}
%    \end{macrocode}
%    \end{macro}
%    \begin{macro}{\HoLogoCss@LyX}
%    \begin{macrocode}
\def\HoLogoCss@LyX{%
  \Css{%
    span.HoLogo-LyX span.HoLogo-y{%
      position:relative;%
      top:.25em;%
      margin-left:-.1667em;%
      margin-right:-.125em;%
      text-decoration:none;%
    }%
  }%
  \global\let\HoLogoCss@LyX\relax
}
%    \end{macrocode}
%    \end{macro}
%
% \subsubsection{\hologo{NTS}}
%
%    \begin{macro}{\HoLogo@NTS}
%    Definition for \hologo{NTS} can be found in
%    package \xpackage{etex\textunderscore man} for the \hologo{eTeX} manual \cite{etexman}
%    and in package \xpackage{dtklogos} \cite{dtklogos}:
%\begin{quote}
%\begin{verbatim}
%\def\NTS{%
%  \leavevmode
%  \hbox{%
%    $%
%      \cal N%
%      \kern-0.35em%
%      \lower0.5ex\hbox{$\cal T$}%
%      \kern-0.2em%
%      S%
%    $%
%  }%
%}
%\end{verbatim}
%\end{quote}
%    \begin{macrocode}
\def\HoLogo@NTS#1{%
  \HoLogoFont@font{NTS}{sy}{%
    N\/%
    \kern-.35em%
    \lower.5ex\hbox{T\/}%
    \kern-.2em%
    S\/%
  }%
  \HOLOGO@SpaceFactor
}
%    \end{macrocode}
%    \end{macro}
%
% \subsubsection{\Hologo{TTH} (\hologo{TeX} to HTML translator)}
%
%    Source: \url{http://hutchinson.belmont.ma.us/tth/}
%    In the HTML source the second `T' is printed as subscript.
%\begin{quote}
%\begin{verbatim}
%T<sub>T</sub>H
%\end{verbatim}
%\end{quote}
%    \begin{macro}{\HoLogo@TTH}
%    \begin{macrocode}
\def\HoLogo@TTH#1{%
  \ltx@mbox{%
    T\HOLOGO@SubScript{T}H%
  }%
  \HOLOGO@SpaceFactor
}
%    \end{macrocode}
%    \end{macro}
%
%    \begin{macro}{\HoLogoHtml@TTH}
%    \begin{macrocode}
\def\HoLogoHtml@TTH#1{%
  T\HCode{<sub>}T\HCode{</sub>}H%
}
%    \end{macrocode}
%    \end{macro}
%
% \subsubsection{\Hologo{HanTheThanh}}
%
%    Partial source: Package \xpackage{dtklogos}.
%    The double accent is U+1EBF (latin small letter e with circumflex
%    and acute).
%    \begin{macro}{\HoLogo@HanTheThanh}
%    \begin{macrocode}
\def\HoLogo@HanTheThanh#1{%
  \ltx@mbox{H\`an}%
  \HOLOGO@space
  \ltx@mbox{%
    Th%
    \HOLOGO@IfCharExists{"1EBF}{%
      \char"1EBF\relax
    }{%
      \^e\hbox to 0pt{\hss\raise .5ex\hbox{\'{}}}%
    }%
  }%
  \HOLOGO@space
  \ltx@mbox{Th\`anh}%
}
%    \end{macrocode}
%    \end{macro}
%    \begin{macro}{\HoLogoBkm@HanTheThanh}
%    \begin{macrocode}
\def\HoLogoBkm@HanTheThanh#1{%
  H\`an %
  Th\HOLOGO@PdfdocUnicode{\^e}{\9036\277} %
  Th\`anh%
}
%    \end{macrocode}
%    \end{macro}
%    \begin{macro}{\HoLogoHtml@HanTheThanh}
%    \begin{macrocode}
\def\HoLogoHtml@HanTheThanh#1{%
  H\`an %
  Th\HCode{&\ltx@hashchar x1ebf;} %
  Th\`anh%
}
%    \end{macrocode}
%    \end{macro}
%
% \subsection{Driver detection}
%
%    \begin{macrocode}
\HOLOGO@IfExists\InputIfFileExists{%
  \InputIfFileExists{hologo.cfg}{}{}%
}{%
  \ltx@IfUndefined{pdf@filesize}{%
    \def\HOLOGO@InputIfExists{%
      \openin\HOLOGO@temp=hologo.cfg\relax
      \ifeof\HOLOGO@temp
        \closein\HOLOGO@temp
      \else
        \closein\HOLOGO@temp
        \begingroup
          \def\x{LaTeX2e}%
        \expandafter\endgroup
        \ifx\fmtname\x
          \input{hologo.cfg}%
        \else
          \input hologo.cfg\relax
        \fi
      \fi
    }%
    \ltx@IfUndefined{newread}{%
      \chardef\HOLOGO@temp=15 %
      \def\HOLOGO@CheckRead{%
        \ifeof\HOLOGO@temp
          \HOLOGO@InputIfExists
        \else
          \ifcase\HOLOGO@temp
            \@PackageWarningNoLine{hologo}{%
              Configuration file ignored, because\MessageBreak
              a free read register could not be found%
            }%
          \else
            \begingroup
              \count\ltx@cclv=\HOLOGO@temp
              \advance\ltx@cclv by \ltx@minusone
              \edef\x{\endgroup
                \chardef\noexpand\HOLOGO@temp=\the\count\ltx@cclv
                \relax
              }%
            \x
          \fi
        \fi
      }%
    }{%
      \csname newread\endcsname\HOLOGO@temp
      \HOLOGO@InputIfExists
    }%
  }{%
    \edef\HOLOGO@temp{\pdf@filesize{hologo.cfg}}%
    \ifx\HOLOGO@temp\ltx@empty
    \else
      \ifnum\HOLOGO@temp>0 %
        \begingroup
          \def\x{LaTeX2e}%
        \expandafter\endgroup
        \ifx\fmtname\x
          \input{hologo.cfg}%
        \else
          \input hologo.cfg\relax
        \fi
      \else
        \@PackageInfoNoLine{hologo}{%
          Empty configuration file `hologo.cfg' ignored%
        }%
      \fi
    \fi
  }%
}
%    \end{macrocode}
%
%    \begin{macrocode}
\def\HOLOGO@temp#1#2{%
  \kv@define@key{HoLogoDriver}{#1}[]{%
    \begingroup
      \def\HOLOGO@temp{##1}%
      \ltx@onelevel@sanitize\HOLOGO@temp
      \ifx\HOLOGO@temp\ltx@empty
      \else
        \@PackageError{hologo}{%
          Value (\HOLOGO@temp) not permitted for option `#1'%
        }%
        \@ehc
      \fi
    \endgroup
    \def\hologoDriver{#2}%
  }%
}%
\def\HOLOGO@@temp#1#2{%
  \ifx\kv@value\relax
    \HOLOGO@temp{#1}{#1}%
  \else
    \HOLOGO@temp{#1}{#2}%
  \fi
}%
\kv@parse@normalized{%
  pdftex,%
  luatex=pdftex,%
  dvipdfm,%
  dvipdfmx=dvipdfm,%
  dvips,%
  dvipsone=dvips,%
  xdvi=dvips,%
  xetex,%
  vtex,%
}\HOLOGO@@temp
%    \end{macrocode}
%
%    \begin{macrocode}
\kv@define@key{HoLogoDriver}{driverfallback}{%
  \def\HOLOGO@DriverFallback{#1}%
}
%    \end{macrocode}
%
%    \begin{macro}{\HOLOGO@DriverFallback}
%    \begin{macrocode}
\def\HOLOGO@DriverFallback{dvips}
%    \end{macrocode}
%    \end{macro}
%
%    \begin{macro}{\hologoDriverSetup}
%    \begin{macrocode}
\def\hologoDriverSetup{%
  \let\hologoDriver\ltx@undefined
  \HOLOGO@DriverSetup
}
%    \end{macrocode}
%    \end{macro}
%
%    \begin{macro}{\HOLOGO@DriverSetup}
%    \begin{macrocode}
\def\HOLOGO@DriverSetup#1{%
  \kvsetkeys{HoLogoDriver}{#1}%
  \HOLOGO@CheckDriver
  \ltx@ifundefined{hologoDriver}{%
    \begingroup
    \edef\x{\endgroup
      \noexpand\kvsetkeys{HoLogoDriver}{\HOLOGO@DriverFallback}%
    }\x
  }{}%
  \@PackageInfoNoLine{hologo}{Using driver `\hologoDriver'}%
}
%    \end{macrocode}
%    \end{macro}
%
%    \begin{macro}{\HOLOGO@CheckDriver}
%    \begin{macrocode}
\def\HOLOGO@CheckDriver{%
  \ifpdf
    \def\hologoDriver{pdftex}%
    \let\HOLOGO@pdfliteral\pdfliteral
    \ifluatex
      \ifx\pdfextension\@undefined\else
        \protected\def\pdfliteral{\pdfextension literal}%
        \let\HOLOGO@pdfliteral\pdfliteral
      \fi
      \ltx@IfUndefined{HOLOGO@pdfliteral}{%
        \ifnum\luatexversion<36 %
        \else
          \begingroup
            \let\HOLOGO@temp\endgroup
            \ifcase0%
                \directlua{%
                  if tex.enableprimitives then %
                    tex.enableprimitives('HOLOGO@', {'pdfliteral'})%
                  else %
                    tex.print('1')%
                  end%
                }%
                \ifx\HOLOGO@pdfliteral\@undefined 1\fi%
                \relax%
              \endgroup
              \let\HOLOGO@temp\relax
              \global\let\HOLOGO@pdfliteral\HOLOGO@pdfliteral
            \fi%
          \HOLOGO@temp
        \fi
      }{}%
    \fi
    \ltx@IfUndefined{HOLOGO@pdfliteral}{%
      \@PackageWarningNoLine{hologo}{%
        Cannot find \string\pdfliteral
      }%
    }{}%
  \else
    \ifxetex
      \def\hologoDriver{xetex}%
    \else
      \ifvtex
        \def\hologoDriver{vtex}%
      \fi
    \fi
  \fi
}
%    \end{macrocode}
%    \end{macro}
%
%    \begin{macro}{\HOLOGO@WarningUnsupportedDriver}
%    \begin{macrocode}
\def\HOLOGO@WarningUnsupportedDriver#1{%
  \@PackageWarningNoLine{hologo}{%
    Logo `#1' needs driver specific macros,\MessageBreak
    but driver `\hologoDriver' is not supported.\MessageBreak
    Use a different driver or\MessageBreak
    load package `graphics' or `pgf'%
  }%
}
%    \end{macrocode}
%    \end{macro}
%
% \subsubsection{Reflect box macros}
%
%    Skip driver part if not needed.
%    \begin{macrocode}
\ltx@IfUndefined{reflectbox}{}{%
  \ltx@IfUndefined{rotatebox}{}{%
    \HOLOGO@AtEnd
  }%
}
\ltx@IfUndefined{pgftext}{}{%
  \HOLOGO@AtEnd
}
\ltx@IfUndefined{psscalebox}{}{%
  \HOLOGO@AtEnd
}
%    \end{macrocode}
%
%    \begin{macrocode}
\def\HOLOGO@temp{LaTeX2e}
\ifx\fmtname\HOLOGO@temp
  \RequirePackage{kvoptions}[2011/06/30]%
  \ProcessKeyvalOptions{HoLogoDriver}%
\fi
\HOLOGO@DriverSetup{}
%    \end{macrocode}
%
%    \begin{macro}{\HOLOGO@ReflectBox}
%    \begin{macrocode}
\def\HOLOGO@ReflectBox#1{%
  \begingroup
    \setbox\ltx@zero\hbox{\begingroup#1\endgroup}%
    \setbox\ltx@two\hbox{%
      \kern\wd\ltx@zero
      \csname HOLOGO@ScaleBox@\hologoDriver\endcsname{-1}{1}{%
        \hbox to 0pt{\copy\ltx@zero\hss}%
      }%
    }%
    \wd\ltx@two=\wd\ltx@zero
    \box\ltx@two
  \endgroup
}
%    \end{macrocode}
%    \end{macro}
%
%    \begin{macro}{\HOLOGO@PointReflectBox}
%    \begin{macrocode}
\def\HOLOGO@PointReflectBox#1{%
  \begingroup
    \setbox\ltx@zero\hbox{\begingroup#1\endgroup}%
    \setbox\ltx@two\hbox{%
      \kern\wd\ltx@zero
      \raise\ht\ltx@zero\hbox{%
        \csname HOLOGO@ScaleBox@\hologoDriver\endcsname{-1}{-1}{%
          \hbox to 0pt{\copy\ltx@zero\hss}%
        }%
      }%
    }%
    \wd\ltx@two=\wd\ltx@zero
    \box\ltx@two
  \endgroup
}
%    \end{macrocode}
%    \end{macro}
%
%    We must define all variants because of dynamic driver setup.
%    \begin{macrocode}
\def\HOLOGO@temp#1#2{#2}
%    \end{macrocode}
%
%    \begin{macro}{\HOLOGO@ScaleBox@pdftex}
%    \begin{macrocode}
\HOLOGO@temp{pdftex}{%
  \def\HOLOGO@ScaleBox@pdftex#1#2#3{%
    \HOLOGO@pdfliteral{%
      q #1 0 0 #2 0 0 cm%
    }%
    #3%
    \HOLOGO@pdfliteral{%
      Q%
    }%
  }%
}
%    \end{macrocode}
%    \end{macro}
%    \begin{macro}{\HOLOGO@ScaleBox@dvips}
%    \begin{macrocode}
\HOLOGO@temp{dvips}{%
  \def\HOLOGO@ScaleBox@dvips#1#2#3{%
    \special{ps:%
      gsave %
      currentpoint %
      currentpoint translate %
      #1 #2 scale %
      neg exch neg exch translate%
    }%
    #3%
    \special{ps:%
      currentpoint %
      grestore %
      moveto%
    }%
  }%
}
%    \end{macrocode}
%    \end{macro}
%    \begin{macro}{\HOLOGO@ScaleBox@dvipdfm}
%    \begin{macrocode}
\HOLOGO@temp{dvipdfm}{%
  \let\HOLOGO@ScaleBox@dvipdfm\HOLOGO@ScaleBox@dvips
}
%    \end{macrocode}
%    \end{macro}
%    Since \hologo{XeTeX} v0.6.
%    \begin{macro}{\HOLOGO@ScaleBox@xetex}
%    \begin{macrocode}
\HOLOGO@temp{xetex}{%
  \def\HOLOGO@ScaleBox@xetex#1#2#3{%
    \special{x:gsave}%
    \special{x:scale #1 #2}%
    #3%
    \special{x:grestore}%
  }%
}
%    \end{macrocode}
%    \end{macro}
%    \begin{macro}{\HOLOGO@ScaleBox@vtex}
%    \begin{macrocode}
\HOLOGO@temp{vtex}{%
  \def\HOLOGO@ScaleBox@vtex#1#2#3{%
    \special{r(#1,0,0,#2,0,0}%
    #3%
    \special{r)}%
  }%
}
%    \end{macrocode}
%    \end{macro}
%
%    \begin{macrocode}
\HOLOGO@AtEnd%
%</package>
%    \end{macrocode}
%
% \section{Test}
%
% \subsection{Catcode checks for loading}
%
%    \begin{macrocode}
%<*test1>
%    \end{macrocode}
%    \begin{macrocode}
\catcode`\{=1 %
\catcode`\}=2 %
\catcode`\#=6 %
\catcode`\@=11 %
\expandafter\ifx\csname count@\endcsname\relax
  \countdef\count@=255 %
\fi
\expandafter\ifx\csname @gobble\endcsname\relax
  \long\def\@gobble#1{}%
\fi
\expandafter\ifx\csname @firstofone\endcsname\relax
  \long\def\@firstofone#1{#1}%
\fi
\expandafter\ifx\csname loop\endcsname\relax
  \expandafter\@firstofone
\else
  \expandafter\@gobble
\fi
{%
  \def\loop#1\repeat{%
    \def\body{#1}%
    \iterate
  }%
  \def\iterate{%
    \body
      \let\next\iterate
    \else
      \let\next\relax
    \fi
    \next
  }%
  \let\repeat=\fi
}%
\def\RestoreCatcodes{}
\count@=0 %
\loop
  \edef\RestoreCatcodes{%
    \RestoreCatcodes
    \catcode\the\count@=\the\catcode\count@\relax
  }%
\ifnum\count@<255 %
  \advance\count@ 1 %
\repeat

\def\RangeCatcodeInvalid#1#2{%
  \count@=#1\relax
  \loop
    \catcode\count@=15 %
  \ifnum\count@<#2\relax
    \advance\count@ 1 %
  \repeat
}
\def\RangeCatcodeCheck#1#2#3{%
  \count@=#1\relax
  \loop
    \ifnum#3=\catcode\count@
    \else
      \errmessage{%
        Character \the\count@\space
        with wrong catcode \the\catcode\count@\space
        instead of \number#3%
      }%
    \fi
  \ifnum\count@<#2\relax
    \advance\count@ 1 %
  \repeat
}
\def\space{ }
\expandafter\ifx\csname LoadCommand\endcsname\relax
  \def\LoadCommand{\input hologo.sty\relax}%
\fi
\def\Test{%
  \RangeCatcodeInvalid{0}{47}%
  \RangeCatcodeInvalid{58}{64}%
  \RangeCatcodeInvalid{91}{96}%
  \RangeCatcodeInvalid{123}{255}%
  \catcode`\@=12 %
  \catcode`\\=0 %
  \catcode`\%=14 %
  \LoadCommand
  \RangeCatcodeCheck{0}{36}{15}%
  \RangeCatcodeCheck{37}{37}{14}%
  \RangeCatcodeCheck{38}{47}{15}%
  \RangeCatcodeCheck{48}{57}{12}%
  \RangeCatcodeCheck{58}{63}{15}%
  \RangeCatcodeCheck{64}{64}{12}%
  \RangeCatcodeCheck{65}{90}{11}%
  \RangeCatcodeCheck{91}{91}{15}%
  \RangeCatcodeCheck{92}{92}{0}%
  \RangeCatcodeCheck{93}{96}{15}%
  \RangeCatcodeCheck{97}{122}{11}%
  \RangeCatcodeCheck{123}{255}{15}%
  \RestoreCatcodes
}
\Test
\csname @@end\endcsname
\end
%    \end{macrocode}
%    \begin{macrocode}
%</test1>
%    \end{macrocode}
%
% \subsection{Spacefactor}
%
%    The space factor must be 1000 after a logo. If it is greater 1000
%    then the following space is a space after a sentence closing point.
%    If the space factor is smaller 1000 then an immediate following
%    dot is interpreted as abbreviation, not sentence closing point.
%
%    \begin{macrocode}
%<*test-spacefactor>
\NeedsTeXFormat{LaTeX2e}
\documentclass{article}
\usepackage{hologo}[2016/05/12]
\usepackage{kvsetkeys}
\usepackage{qstest}
\IncludeTests{*}
\LogTests{log}{*}{*}
\begin{document}
\begin{qstest}{spacefactor}{spacefactor}
\newcommand*{\Test}[1]{%
  \sbox0{%
    \hologo{#1}%
    \Expect*{1000 (#1)}*{\the\spacefactor\space(#1)}%
  }%
}%
\makeatletter
\def\TestList{}
\def\hologoEntry#1#2#3{%
  \edef\TestList{%
    \ifx\TestList\@empty
    \else
      \TestList,%
    \fi
    #1%
    \ifx\\#2\\%
    \else
      ={variant=#2}%
    \fi
  }%
}
\hologoList
\expandafter\kv@parse@normalized\expandafter{%
  \TestList
}{%
  \begingroup
    \let\@logo=\kv@key
    \ifx\kv@value\relax
    \else
      \expandafter\hologoLogoSetup\expandafter\@logo\expandafter{%
        \kv@value
      }%
    \fi
    \Test\@logo
  \endgroup
  \@gobbletwo
}
\end{qstest}
\end{document}
%</test-spacefactor>
%    \end{macrocode}
%
% \subsection{Complete list}
%
%    \begin{macrocode}
%<*test-list>
\NeedsTeXFormat{LaTeX2e}
\documentclass[12pt,a4paper]{article}
\usepackage{hologo}[2016/05/12]
\usepackage[T1]{fontenc}
\usepackage{lmodern}
\usepackage{parskip}
\usepackage[unicode]{hyperref}[2011/09/28]
\usepackage{bookmark}[2011/09/19]
\bookmarksetup{%
  numbered,%
  open,%
  openlevel=2,%
}
\renewcommand*{\contentsname}{List of logos}
\begin{document}
\tableofcontents
\def\TestFont#1#2#3#4#5#6{%
  \begingroup
    \usefont{#3}{#4}{#5}{#6}%
    \HologoVariant{#1}{#2}/\hologoVariant{#1}{#2}%
    \quad
    \begingroup\scriptsize\hologoVariant{#1}{#2}\endgroup
    \quad
  \endgroup
  (#3/#4/#5/#6)%
  \par
}
\makeatletter
\def\hologoEntry#1#2#3{%
  \section{%
    \HologoVariant{#1}{#2}/\hologoVariant{#1}{#2} %
    {[#1\ifx\\#2\\\else\space(#2)\fi]}% hash-ok
  }% braces around [] because of bug in tex4ht
  \begingroup
    \hypersetup{unicode=false}%
    \bookmark[%
      dest=\@currentHref,%
      rellevel=1,%
      keeplevel,%
    ]{%
      \HologoVariant{#1}{#2}/\hologoVariant{#1}{#2} %
      (PDFDocEncoding)%
    }%
  \endgroup
  \TestFont{#1}{#2}{OT1}{cmr}{m}{n}%
  \TestFont{#1}{#2}{OT1}{cmss}{m}{n}%
  \TestFont{#1}{#2}{OT1}{cmr}{b}{n}%
  \TestFont{#1}{#2}{OT1}{cmr}{m}{it}%
  \TestFont{#1}{#2}{OT1}{cmtt}{m}{n}%
  \TestFont{#1}{#2}{T1}{lmr}{m}{n}%
  \TestFont{#1}{#2}{T1}{lmss}{m}{n}%
  \TestFont{#1}{#2}{T1}{lmr}{b}{n}%
  \TestFont{#1}{#2}{T1}{lmr}{m}{it}%
  \TestFont{#1}{#2}{T1}{lmtt}{m}{n}%
  \TestFont{#1}{#2}{T1}{lmvtt}{m}{n}%
  \TestFont{#1}{#2}{T1}{qtm}{m}{n}%
  \TestFont{#1}{#2}{T1}{qhv}{m}{n}%
  \TestFont{#1}{#2}{T1}{qtm}{b}{n}%
  \TestFont{#1}{#2}{T1}{qtm}{m}{it}%
  \TestFont{#1}{#2}{T1}{qcr}{m}{n}%
  \newpage
}
\makeatother
\hologoList
\end{document}
%</test-list>
%    \end{macrocode}
%
% \section{Installation}
%
% \subsection{Download}
%
% \paragraph{Package.} This package is available on
% CTAN\footnote{\url{ftp://ftp.ctan.org/tex-archive/}}:
% \begin{description}
% \item[\CTAN{macros/latex/contrib/oberdiek/hologo.dtx}] The source file.
% \item[\CTAN{macros/latex/contrib/oberdiek/hologo.pdf}] Documentation.
% \end{description}
%
%
% \paragraph{Bundle.} All the packages of the bundle `oberdiek'
% are also available in a TDS compliant ZIP archive. There
% the packages are already unpacked and the documentation files
% are generated. The files and directories obey the TDS standard.
% \begin{description}
% \item[\CTAN{install/macros/latex/contrib/oberdiek.tds.zip}]
% \end{description}
% \emph{TDS} refers to the standard ``A Directory Structure
% for \TeX\ Files'' (\CTAN{tds/tds.pdf}). Directories
% with \xfile{texmf} in their name are usually organized this way.
%
% \subsection{Bundle installation}
%
% \paragraph{Unpacking.} Unpack the \xfile{oberdiek.tds.zip} in the
% TDS tree (also known as \xfile{texmf} tree) of your choice.
% Example (linux):
% \begin{quote}
%   |unzip oberdiek.tds.zip -d ~/texmf|
% \end{quote}
%
% \paragraph{Script installation.}
% Check the directory \xfile{TDS:scripts/oberdiek/} for
% scripts that need further installation steps.
% Package \xpackage{attachfile2} comes with the Perl script
% \xfile{pdfatfi.pl} that should be installed in such a way
% that it can be called as \texttt{pdfatfi}.
% Example (linux):
% \begin{quote}
%   |chmod +x scripts/oberdiek/pdfatfi.pl|\\
%   |cp scripts/oberdiek/pdfatfi.pl /usr/local/bin/|
% \end{quote}
%
% \subsection{Package installation}
%
% \paragraph{Unpacking.} The \xfile{.dtx} file is a self-extracting
% \docstrip\ archive. The files are extracted by running the
% \xfile{.dtx} through \plainTeX:
% \begin{quote}
%   \verb|tex hologo.dtx|
% \end{quote}
%
% \paragraph{TDS.} Now the different files must be moved into
% the different directories in your installation TDS tree
% (also known as \xfile{texmf} tree):
% \begin{quote}
% \def\t{^^A
% \begin{tabular}{@{}>{\ttfamily}l@{ $\rightarrow$ }>{\ttfamily}l@{}}
%   hologo.sty & tex/generic/oberdiek/hologo.sty\\
%   hologo.pdf & doc/latex/oberdiek/hologo.pdf\\
%   example/hologo-example.tex & doc/latex/oberdiek/example/hologo-example.tex\\
%   test/hologo-test1.tex & doc/latex/oberdiek/test/hologo-test1.tex\\
%   test/hologo-test-spacefactor.tex & doc/latex/oberdiek/test/hologo-test-spacefactor.tex\\
%   test/hologo-test-list.tex & doc/latex/oberdiek/test/hologo-test-list.tex\\
%   hologo.dtx & source/latex/oberdiek/hologo.dtx\\
% \end{tabular}^^A
% }^^A
% \sbox0{\t}^^A
% \ifdim\wd0>\linewidth
%   \begingroup
%     \advance\linewidth by\leftmargin
%     \advance\linewidth by\rightmargin
%   \edef\x{\endgroup
%     \def\noexpand\lw{\the\linewidth}^^A
%   }\x
%   \def\lwbox{^^A
%     \leavevmode
%     \hbox to \linewidth{^^A
%       \kern-\leftmargin\relax
%       \hss
%       \usebox0
%       \hss
%       \kern-\rightmargin\relax
%     }^^A
%   }^^A
%   \ifdim\wd0>\lw
%     \sbox0{\small\t}^^A
%     \ifdim\wd0>\linewidth
%       \ifdim\wd0>\lw
%         \sbox0{\footnotesize\t}^^A
%         \ifdim\wd0>\linewidth
%           \ifdim\wd0>\lw
%             \sbox0{\scriptsize\t}^^A
%             \ifdim\wd0>\linewidth
%               \ifdim\wd0>\lw
%                 \sbox0{\tiny\t}^^A
%                 \ifdim\wd0>\linewidth
%                   \lwbox
%                 \else
%                   \usebox0
%                 \fi
%               \else
%                 \lwbox
%               \fi
%             \else
%               \usebox0
%             \fi
%           \else
%             \lwbox
%           \fi
%         \else
%           \usebox0
%         \fi
%       \else
%         \lwbox
%       \fi
%     \else
%       \usebox0
%     \fi
%   \else
%     \lwbox
%   \fi
% \else
%   \usebox0
% \fi
% \end{quote}
% If you have a \xfile{docstrip.cfg} that configures and enables \docstrip's
% TDS installing feature, then some files can already be in the right
% place, see the documentation of \docstrip.
%
% \subsection{Refresh file name databases}
%
% If your \TeX~distribution
% (\teTeX, \mikTeX, \dots) relies on file name databases, you must refresh
% these. For example, \teTeX\ users run \verb|texhash| or
% \verb|mktexlsr|.
%
% \subsection{Some details for the interested}
%
% \paragraph{Attached source.}
%
% The PDF documentation on CTAN also includes the
% \xfile{.dtx} source file. It can be extracted by
% AcrobatReader 6 or higher. Another option is \textsf{pdftk},
% e.g. unpack the file into the current directory:
% \begin{quote}
%   \verb|pdftk hologo.pdf unpack_files output .|
% \end{quote}
%
% \paragraph{Unpacking with \LaTeX.}
% The \xfile{.dtx} chooses its action depending on the format:
% \begin{description}
% \item[\plainTeX:] Run \docstrip\ and extract the files.
% \item[\LaTeX:] Generate the documentation.
% \end{description}
% If you insist on using \LaTeX\ for \docstrip\ (really,
% \docstrip\ does not need \LaTeX), then inform the autodetect routine
% about your intention:
% \begin{quote}
%   \verb|latex \let\install=y\input{hologo.dtx}|
% \end{quote}
% Do not forget to quote the argument according to the demands
% of your shell.
%
% \paragraph{Generating the documentation.}
% You can use both the \xfile{.dtx} or the \xfile{.drv} to generate
% the documentation. The process can be configured by the
% configuration file \xfile{ltxdoc.cfg}. For instance, put this
% line into this file, if you want to have A4 as paper format:
% \begin{quote}
%   \verb|\PassOptionsToClass{a4paper}{article}|
% \end{quote}
% An example follows how to generate the
% documentation with pdf\LaTeX:
% \begin{quote}
%\begin{verbatim}
%pdflatex hologo.dtx
%makeindex -s gind.ist hologo.idx
%pdflatex hologo.dtx
%makeindex -s gind.ist hologo.idx
%pdflatex hologo.dtx
%\end{verbatim}
% \end{quote}
%
% \section{Catalogue}
%
% The following XML file can be used as source for the
% \href{http://mirror.ctan.org/help/Catalogue/catalogue.html}{\TeX\ Catalogue}.
% The elements \texttt{caption} and \texttt{description} are imported
% from the original XML file from the Catalogue.
% The name of the XML file in the Catalogue is \xfile{hologo.xml}.
%    \begin{macrocode}
%<*catalogue>
<?xml version='1.0' encoding='us-ascii'?>
<!DOCTYPE entry SYSTEM 'catalogue.dtd'>
<entry datestamp='$Date$' modifier='$Author$' id='hologo'>
  <name>hologo</name>
  <caption>A collection of logos with bookmark support.</caption>
  <authorref id='auth:oberdiek'/>
  <copyright owner='Heiko Oberdiek' year='2010-2012'/>
  <license type='lppl1.3'/>
  <version number='1.10'/>
  <description>
    The package defines a single command <tt>\hologo</tt>, whose
    argument is the usual case-confused ASCII version of the logo.
    The command is bookmark-enabled, so that every logo becomes
    available in bookmarks without further work.
    <p/>
    The package is part of the <xref refid='oberdiek'>oberdiek</xref>
    bundle.
  </description>
  <documentation details='Package documentation'
      href='ctan:/macros/latex/contrib/oberdiek/hologo.pdf'/>
  <ctan file='true' path='/macros/latex/contrib/oberdiek/hologo.dtx'/>
  <miktex location='oberdiek'/>
  <texlive location='oberdiek'/>
  <install path='/macros/latex/contrib/oberdiek/oberdiek.tds.zip'/>
</entry>
%</catalogue>
%    \end{macrocode}
%
% \begin{thebibliography}{9}
% \raggedright
%
% \bibitem{btxdoc}
% Oren Patashnik,
% \textit{\hologo{BibTeX}ing},
% 1988-02-08.\\
% \CTAN{biblio/bibtex/base/}
%
% \bibitem{dtklogos}
% Gerd Neugebauer, DANTE,
% \textit{Package \xpackage{dtklogos}},
% 2011-04-25.\\
% \CTAN{usergrps/dante/dtk/dtklogos.sty}
%
% \bibitem{etexman}
% The \hologo{NTS} Team,
% \textit{The \hologo{eTeX} manual},
% 1998-02.\\
% \CTAN{systems/e-tex/v2/doc/}
%
% \bibitem{ExTeX-FAQ}
% The \hologo{ExTeX} group,
% \textit{\hologo{ExTeX}: FAQ -- How is \hologo{ExTeX} typeset?},
% 2007-04-14.\\
% \url{http://www.extex.org/documentation/faq.html}
%
% \bibitem{LyX}
% %@MISC{ LyX,
% %  title = {{LyX 2.0.0 -- The Document Processor [Computer software and manual]}},
% %  author = {{The LyX Team}},
% %  howpublished = {Internet: http://www.lyx.org},
% %  year = {2011-05-08},
% %  note = {Retrieved May 10, 2011, from http://www.lyx.org},
% %  url = {http://www.lyx.org/}
% %}
% The \hologo{LyX} Team,
% \textit{\hologo{LyX} -- The Document Processor},
% 2011-05-08.\\
% \url{http://www.lyx.org/}
%
% \bibitem{OzTeX}
% Andrew Trevorrow,
% \hologo{OzTeX} FAQ: What is the correct way to typeset ``\hologo{OzTeX}''?,
% 2011-09-15 (visited).
% \url{http://www.trevorrow.com/oztex/ozfaq.html#oztex-logo}
%
% \bibitem{PiCTeX}
% Michael Wichura,
% \textit{The \hologo{PiCTeX} macro package},
% 1987-09-21.
% \CTAN{graphics/pictex/}
%
% \bibitem{scrlogo}
% Markus Kohm,
% \textit{\hologo{KOMAScript} Datei \xfile{scrlogo.dtx}},
% 2009-01-30.\\
% \CTAN{install/macros/latex/contrib/komascript.tds.zip}
%
% \end{thebibliography}
%
% \begin{History}
%   \begin{Version}{2010/04/08 v1.0}
%   \item
%     The first version.
%   \end{Version}
%   \begin{Version}{2010/04/16 v1.1}
%   \item
%     \cs{Hologo} added for support of logos at start of a sentence.
%   \item
%     \cs{hologoSetup} and \cs{hologoLogoSetup} added.
%   \item
%     Options \xoption{break}, \xoption{hyphenbreak}, \xoption{spacebreak}
%     added.
%   \item
%     Variant support added by option \xoption{variant}.
%   \end{Version}
%   \begin{Version}{2010/04/24 v1.2}
%   \item
%     \hologo{LaTeX3} added.
%   \item
%     \hologo{VTeX} added.
%   \end{Version}
%   \begin{Version}{2010/11/21 v1.3}
%   \item
%     \hologo{iniTeX}, \hologo{virTeX} added.
%   \end{Version}
%   \begin{Version}{2011/03/25 v1.4}
%   \item
%     \hologo{ConTeXt} with variants added.
%   \item
%     Option \xoption{discretionarybreak} added as refinement for
%     option \xoption{break}.
%   \end{Version}
%   \begin{Version}{2011/04/21 v1.5}
%   \item
%     Wrong TDS directory for test files fixed.
%   \end{Version}
%   \begin{Version}{2011/10/01 v1.6}
%   \item
%     Support for package \xpackage{tex4ht} added.
%   \item
%     Support for \cs{csname} added if \cs{ifincsname} is available.
%   \item
%     New logos:
%     \hologo{(La)TeX},
%     \hologo{biber},
%     \hologo{BibTeX} (\xoption{sc}, \xoption{sf}),
%     \hologo{emTeX},
%     \hologo{ExTeX},
%     \hologo{KOMAScript},
%     \hologo{La},
%     \hologo{LyX},
%     \hologo{MiKTeX},
%     \hologo{NTS},
%     \hologo{OzMF},
%     \hologo{OzMP},
%     \hologo{OzTeX},
%     \hologo{OzTtH},
%     \hologo{PCTeX},
%     \hologo{PiC},
%     \hologo{PiCTeX},
%     \hologo{METAFONT},
%     \hologo{MetaFun},
%     \hologo{METAPOST},
%     \hologo{MetaPost},
%     \hologo{SLiTeX} (\xoption{lift}, \xoption{narrow}, \xoption{simple}),
%     \hologo{SliTeX} (\xoption{narrow}, \xoption{simple}, \xoption{lift}),
%     \hologo{teTeX}.
%   \item
%     Fixes:
%     \hologo{iniTeX},
%     \hologo{pdfLaTeX},
%     \hologo{pdfTeX},
%     \hologo{virTeX}.
%   \item
%     \cs{hologoFontSetup} and \cs{hologoLogoFontSetup} added.
%   \item
%     \cs{hologoVariant} and \cs{HologoVariant} added.
%   \end{Version}
%   \begin{Version}{2011/11/22 v1.7}
%   \item
%     New logos:
%     \hologo{BibTeX8},
%     \hologo{LaTeXML},
%     \hologo{SageTeX},
%     \hologo{TeX4ht},
%     \hologo{TTH}.
%   \item
%     \hologo{Xe} and friends: Driver stuff fixed.
%   \item
%     \hologo{Xe} and friends: Support for italic added.
%   \item
%     \hologo{Xe} and friends: Package support for \xpackage{pgf}
%     and \xpackage{pstricks} added.
%   \end{Version}
%   \begin{Version}{2011/11/29 v1.8}
%   \item
%     New logos:
%     \hologo{HanTheThanh}.
%   \end{Version}
%   \begin{Version}{2011/12/21 v1.9}
%   \item
%     Patch for package \xpackage{ifxetex} added for the case that
%     \cs{newif} is undefined in \hologo{iniTeX}.
%   \item
%     Some fixes for \hologo{iniTeX}.
%   \end{Version}
%   \begin{Version}{2012/04/26 v1.10}
%   \item
%     Fix in bookmark version of logo ``\hologo{HanTheThanh}''.
%   \end{Version}
%   \begin{Version}{2016/05/12 v1.11}
%   \item
%     Update HOLOGO@IfCharExists (previously in texlive)
%   \item define pdfliteral in current luatex.
%   \end{Version}
% \end{History}
%
% \PrintIndex
%
% \Finale
\endinput
%
        \else
          \input hologo.cfg\relax
        \fi
      \else
        \@PackageInfoNoLine{hologo}{%
          Empty configuration file `hologo.cfg' ignored%
        }%
      \fi
    \fi
  }%
}
%    \end{macrocode}
%
%    \begin{macrocode}
\def\HOLOGO@temp#1#2{%
  \kv@define@key{HoLogoDriver}{#1}[]{%
    \begingroup
      \def\HOLOGO@temp{##1}%
      \ltx@onelevel@sanitize\HOLOGO@temp
      \ifx\HOLOGO@temp\ltx@empty
      \else
        \@PackageError{hologo}{%
          Value (\HOLOGO@temp) not permitted for option `#1'%
        }%
        \@ehc
      \fi
    \endgroup
    \def\hologoDriver{#2}%
  }%
}%
\def\HOLOGO@@temp#1#2{%
  \ifx\kv@value\relax
    \HOLOGO@temp{#1}{#1}%
  \else
    \HOLOGO@temp{#1}{#2}%
  \fi
}%
\kv@parse@normalized{%
  pdftex,%
  luatex=pdftex,%
  dvipdfm,%
  dvipdfmx=dvipdfm,%
  dvips,%
  dvipsone=dvips,%
  xdvi=dvips,%
  xetex,%
  vtex,%
}\HOLOGO@@temp
%    \end{macrocode}
%
%    \begin{macrocode}
\kv@define@key{HoLogoDriver}{driverfallback}{%
  \def\HOLOGO@DriverFallback{#1}%
}
%    \end{macrocode}
%
%    \begin{macro}{\HOLOGO@DriverFallback}
%    \begin{macrocode}
\def\HOLOGO@DriverFallback{dvips}
%    \end{macrocode}
%    \end{macro}
%
%    \begin{macro}{\hologoDriverSetup}
%    \begin{macrocode}
\def\hologoDriverSetup{%
  \let\hologoDriver\ltx@undefined
  \HOLOGO@DriverSetup
}
%    \end{macrocode}
%    \end{macro}
%
%    \begin{macro}{\HOLOGO@DriverSetup}
%    \begin{macrocode}
\def\HOLOGO@DriverSetup#1{%
  \kvsetkeys{HoLogoDriver}{#1}%
  \HOLOGO@CheckDriver
  \ltx@ifundefined{hologoDriver}{%
    \begingroup
    \edef\x{\endgroup
      \noexpand\kvsetkeys{HoLogoDriver}{\HOLOGO@DriverFallback}%
    }\x
  }{}%
  \@PackageInfoNoLine{hologo}{Using driver `\hologoDriver'}%
}
%    \end{macrocode}
%    \end{macro}
%
%    \begin{macro}{\HOLOGO@CheckDriver}
%    \begin{macrocode}
\def\HOLOGO@CheckDriver{%
  \ifpdf
    \def\hologoDriver{pdftex}%
    \let\HOLOGO@pdfliteral\pdfliteral
    \ifluatex
      \ifx\pdfextension\@undefined\else
        \protected\def\pdfliteral{\pdfextension literal}%
        \let\HOLOGO@pdfliteral\pdfliteral
      \fi
      \ltx@IfUndefined{HOLOGO@pdfliteral}{%
        \ifnum\luatexversion<36 %
        \else
          \begingroup
            \let\HOLOGO@temp\endgroup
            \ifcase0%
                \directlua{%
                  if tex.enableprimitives then %
                    tex.enableprimitives('HOLOGO@', {'pdfliteral'})%
                  else %
                    tex.print('1')%
                  end%
                }%
                \ifx\HOLOGO@pdfliteral\@undefined 1\fi%
                \relax%
              \endgroup
              \let\HOLOGO@temp\relax
              \global\let\HOLOGO@pdfliteral\HOLOGO@pdfliteral
            \fi%
          \HOLOGO@temp
        \fi
      }{}%
    \fi
    \ltx@IfUndefined{HOLOGO@pdfliteral}{%
      \@PackageWarningNoLine{hologo}{%
        Cannot find \string\pdfliteral
      }%
    }{}%
  \else
    \ifxetex
      \def\hologoDriver{xetex}%
    \else
      \ifvtex
        \def\hologoDriver{vtex}%
      \fi
    \fi
  \fi
}
%    \end{macrocode}
%    \end{macro}
%
%    \begin{macro}{\HOLOGO@WarningUnsupportedDriver}
%    \begin{macrocode}
\def\HOLOGO@WarningUnsupportedDriver#1{%
  \@PackageWarningNoLine{hologo}{%
    Logo `#1' needs driver specific macros,\MessageBreak
    but driver `\hologoDriver' is not supported.\MessageBreak
    Use a different driver or\MessageBreak
    load package `graphics' or `pgf'%
  }%
}
%    \end{macrocode}
%    \end{macro}
%
% \subsubsection{Reflect box macros}
%
%    Skip driver part if not needed.
%    \begin{macrocode}
\ltx@IfUndefined{reflectbox}{}{%
  \ltx@IfUndefined{rotatebox}{}{%
    \HOLOGO@AtEnd
  }%
}
\ltx@IfUndefined{pgftext}{}{%
  \HOLOGO@AtEnd
}
\ltx@IfUndefined{psscalebox}{}{%
  \HOLOGO@AtEnd
}
%    \end{macrocode}
%
%    \begin{macrocode}
\def\HOLOGO@temp{LaTeX2e}
\ifx\fmtname\HOLOGO@temp
  \RequirePackage{kvoptions}[2011/06/30]%
  \ProcessKeyvalOptions{HoLogoDriver}%
\fi
\HOLOGO@DriverSetup{}
%    \end{macrocode}
%
%    \begin{macro}{\HOLOGO@ReflectBox}
%    \begin{macrocode}
\def\HOLOGO@ReflectBox#1{%
  \begingroup
    \setbox\ltx@zero\hbox{\begingroup#1\endgroup}%
    \setbox\ltx@two\hbox{%
      \kern\wd\ltx@zero
      \csname HOLOGO@ScaleBox@\hologoDriver\endcsname{-1}{1}{%
        \hbox to 0pt{\copy\ltx@zero\hss}%
      }%
    }%
    \wd\ltx@two=\wd\ltx@zero
    \box\ltx@two
  \endgroup
}
%    \end{macrocode}
%    \end{macro}
%
%    \begin{macro}{\HOLOGO@PointReflectBox}
%    \begin{macrocode}
\def\HOLOGO@PointReflectBox#1{%
  \begingroup
    \setbox\ltx@zero\hbox{\begingroup#1\endgroup}%
    \setbox\ltx@two\hbox{%
      \kern\wd\ltx@zero
      \raise\ht\ltx@zero\hbox{%
        \csname HOLOGO@ScaleBox@\hologoDriver\endcsname{-1}{-1}{%
          \hbox to 0pt{\copy\ltx@zero\hss}%
        }%
      }%
    }%
    \wd\ltx@two=\wd\ltx@zero
    \box\ltx@two
  \endgroup
}
%    \end{macrocode}
%    \end{macro}
%
%    We must define all variants because of dynamic driver setup.
%    \begin{macrocode}
\def\HOLOGO@temp#1#2{#2}
%    \end{macrocode}
%
%    \begin{macro}{\HOLOGO@ScaleBox@pdftex}
%    \begin{macrocode}
\HOLOGO@temp{pdftex}{%
  \def\HOLOGO@ScaleBox@pdftex#1#2#3{%
    \HOLOGO@pdfliteral{%
      q #1 0 0 #2 0 0 cm%
    }%
    #3%
    \HOLOGO@pdfliteral{%
      Q%
    }%
  }%
}
%    \end{macrocode}
%    \end{macro}
%    \begin{macro}{\HOLOGO@ScaleBox@dvips}
%    \begin{macrocode}
\HOLOGO@temp{dvips}{%
  \def\HOLOGO@ScaleBox@dvips#1#2#3{%
    \special{ps:%
      gsave %
      currentpoint %
      currentpoint translate %
      #1 #2 scale %
      neg exch neg exch translate%
    }%
    #3%
    \special{ps:%
      currentpoint %
      grestore %
      moveto%
    }%
  }%
}
%    \end{macrocode}
%    \end{macro}
%    \begin{macro}{\HOLOGO@ScaleBox@dvipdfm}
%    \begin{macrocode}
\HOLOGO@temp{dvipdfm}{%
  \let\HOLOGO@ScaleBox@dvipdfm\HOLOGO@ScaleBox@dvips
}
%    \end{macrocode}
%    \end{macro}
%    Since \hologo{XeTeX} v0.6.
%    \begin{macro}{\HOLOGO@ScaleBox@xetex}
%    \begin{macrocode}
\HOLOGO@temp{xetex}{%
  \def\HOLOGO@ScaleBox@xetex#1#2#3{%
    \special{x:gsave}%
    \special{x:scale #1 #2}%
    #3%
    \special{x:grestore}%
  }%
}
%    \end{macrocode}
%    \end{macro}
%    \begin{macro}{\HOLOGO@ScaleBox@vtex}
%    \begin{macrocode}
\HOLOGO@temp{vtex}{%
  \def\HOLOGO@ScaleBox@vtex#1#2#3{%
    \special{r(#1,0,0,#2,0,0}%
    #3%
    \special{r)}%
  }%
}
%    \end{macrocode}
%    \end{macro}
%
%    \begin{macrocode}
\HOLOGO@AtEnd%
%</package>
%    \end{macrocode}
%
% \section{Test}
%
% \subsection{Catcode checks for loading}
%
%    \begin{macrocode}
%<*test1>
%    \end{macrocode}
%    \begin{macrocode}
\catcode`\{=1 %
\catcode`\}=2 %
\catcode`\#=6 %
\catcode`\@=11 %
\expandafter\ifx\csname count@\endcsname\relax
  \countdef\count@=255 %
\fi
\expandafter\ifx\csname @gobble\endcsname\relax
  \long\def\@gobble#1{}%
\fi
\expandafter\ifx\csname @firstofone\endcsname\relax
  \long\def\@firstofone#1{#1}%
\fi
\expandafter\ifx\csname loop\endcsname\relax
  \expandafter\@firstofone
\else
  \expandafter\@gobble
\fi
{%
  \def\loop#1\repeat{%
    \def\body{#1}%
    \iterate
  }%
  \def\iterate{%
    \body
      \let\next\iterate
    \else
      \let\next\relax
    \fi
    \next
  }%
  \let\repeat=\fi
}%
\def\RestoreCatcodes{}
\count@=0 %
\loop
  \edef\RestoreCatcodes{%
    \RestoreCatcodes
    \catcode\the\count@=\the\catcode\count@\relax
  }%
\ifnum\count@<255 %
  \advance\count@ 1 %
\repeat

\def\RangeCatcodeInvalid#1#2{%
  \count@=#1\relax
  \loop
    \catcode\count@=15 %
  \ifnum\count@<#2\relax
    \advance\count@ 1 %
  \repeat
}
\def\RangeCatcodeCheck#1#2#3{%
  \count@=#1\relax
  \loop
    \ifnum#3=\catcode\count@
    \else
      \errmessage{%
        Character \the\count@\space
        with wrong catcode \the\catcode\count@\space
        instead of \number#3%
      }%
    \fi
  \ifnum\count@<#2\relax
    \advance\count@ 1 %
  \repeat
}
\def\space{ }
\expandafter\ifx\csname LoadCommand\endcsname\relax
  \def\LoadCommand{\input hologo.sty\relax}%
\fi
\def\Test{%
  \RangeCatcodeInvalid{0}{47}%
  \RangeCatcodeInvalid{58}{64}%
  \RangeCatcodeInvalid{91}{96}%
  \RangeCatcodeInvalid{123}{255}%
  \catcode`\@=12 %
  \catcode`\\=0 %
  \catcode`\%=14 %
  \LoadCommand
  \RangeCatcodeCheck{0}{36}{15}%
  \RangeCatcodeCheck{37}{37}{14}%
  \RangeCatcodeCheck{38}{47}{15}%
  \RangeCatcodeCheck{48}{57}{12}%
  \RangeCatcodeCheck{58}{63}{15}%
  \RangeCatcodeCheck{64}{64}{12}%
  \RangeCatcodeCheck{65}{90}{11}%
  \RangeCatcodeCheck{91}{91}{15}%
  \RangeCatcodeCheck{92}{92}{0}%
  \RangeCatcodeCheck{93}{96}{15}%
  \RangeCatcodeCheck{97}{122}{11}%
  \RangeCatcodeCheck{123}{255}{15}%
  \RestoreCatcodes
}
\Test
\csname @@end\endcsname
\end
%    \end{macrocode}
%    \begin{macrocode}
%</test1>
%    \end{macrocode}
%
% \subsection{Spacefactor}
%
%    The space factor must be 1000 after a logo. If it is greater 1000
%    then the following space is a space after a sentence closing point.
%    If the space factor is smaller 1000 then an immediate following
%    dot is interpreted as abbreviation, not sentence closing point.
%
%    \begin{macrocode}
%<*test-spacefactor>
\NeedsTeXFormat{LaTeX2e}
\documentclass{article}
\usepackage{hologo}[2016/05/12]
\usepackage{kvsetkeys}
\usepackage{qstest}
\IncludeTests{*}
\LogTests{log}{*}{*}
\begin{document}
\begin{qstest}{spacefactor}{spacefactor}
\newcommand*{\Test}[1]{%
  \sbox0{%
    \hologo{#1}%
    \Expect*{1000 (#1)}*{\the\spacefactor\space(#1)}%
  }%
}%
\makeatletter
\def\TestList{}
\def\hologoEntry#1#2#3{%
  \edef\TestList{%
    \ifx\TestList\@empty
    \else
      \TestList,%
    \fi
    #1%
    \ifx\\#2\\%
    \else
      ={variant=#2}%
    \fi
  }%
}
\hologoList
\expandafter\kv@parse@normalized\expandafter{%
  \TestList
}{%
  \begingroup
    \let\@logo=\kv@key
    \ifx\kv@value\relax
    \else
      \expandafter\hologoLogoSetup\expandafter\@logo\expandafter{%
        \kv@value
      }%
    \fi
    \Test\@logo
  \endgroup
  \@gobbletwo
}
\end{qstest}
\end{document}
%</test-spacefactor>
%    \end{macrocode}
%
% \subsection{Complete list}
%
%    \begin{macrocode}
%<*test-list>
\NeedsTeXFormat{LaTeX2e}
\documentclass[12pt,a4paper]{article}
\usepackage{hologo}[2016/05/12]
\usepackage[T1]{fontenc}
\usepackage{lmodern}
\usepackage{parskip}
\usepackage[unicode]{hyperref}[2011/09/28]
\usepackage{bookmark}[2011/09/19]
\bookmarksetup{%
  numbered,%
  open,%
  openlevel=2,%
}
\renewcommand*{\contentsname}{List of logos}
\begin{document}
\tableofcontents
\def\TestFont#1#2#3#4#5#6{%
  \begingroup
    \usefont{#3}{#4}{#5}{#6}%
    \HologoVariant{#1}{#2}/\hologoVariant{#1}{#2}%
    \quad
    \begingroup\scriptsize\hologoVariant{#1}{#2}\endgroup
    \quad
  \endgroup
  (#3/#4/#5/#6)%
  \par
}
\makeatletter
\def\hologoEntry#1#2#3{%
  \section{%
    \HologoVariant{#1}{#2}/\hologoVariant{#1}{#2} %
    {[#1\ifx\\#2\\\else\space(#2)\fi]}% hash-ok
  }% braces around [] because of bug in tex4ht
  \begingroup
    \hypersetup{unicode=false}%
    \bookmark[%
      dest=\@currentHref,%
      rellevel=1,%
      keeplevel,%
    ]{%
      \HologoVariant{#1}{#2}/\hologoVariant{#1}{#2} %
      (PDFDocEncoding)%
    }%
  \endgroup
  \TestFont{#1}{#2}{OT1}{cmr}{m}{n}%
  \TestFont{#1}{#2}{OT1}{cmss}{m}{n}%
  \TestFont{#1}{#2}{OT1}{cmr}{b}{n}%
  \TestFont{#1}{#2}{OT1}{cmr}{m}{it}%
  \TestFont{#1}{#2}{OT1}{cmtt}{m}{n}%
  \TestFont{#1}{#2}{T1}{lmr}{m}{n}%
  \TestFont{#1}{#2}{T1}{lmss}{m}{n}%
  \TestFont{#1}{#2}{T1}{lmr}{b}{n}%
  \TestFont{#1}{#2}{T1}{lmr}{m}{it}%
  \TestFont{#1}{#2}{T1}{lmtt}{m}{n}%
  \TestFont{#1}{#2}{T1}{lmvtt}{m}{n}%
  \TestFont{#1}{#2}{T1}{qtm}{m}{n}%
  \TestFont{#1}{#2}{T1}{qhv}{m}{n}%
  \TestFont{#1}{#2}{T1}{qtm}{b}{n}%
  \TestFont{#1}{#2}{T1}{qtm}{m}{it}%
  \TestFont{#1}{#2}{T1}{qcr}{m}{n}%
  \newpage
}
\makeatother
\hologoList
\end{document}
%</test-list>
%    \end{macrocode}
%
% \section{Installation}
%
% \subsection{Download}
%
% \paragraph{Package.} This package is available on
% CTAN\footnote{\url{ftp://ftp.ctan.org/tex-archive/}}:
% \begin{description}
% \item[\CTAN{macros/latex/contrib/oberdiek/hologo.dtx}] The source file.
% \item[\CTAN{macros/latex/contrib/oberdiek/hologo.pdf}] Documentation.
% \end{description}
%
%
% \paragraph{Bundle.} All the packages of the bundle `oberdiek'
% are also available in a TDS compliant ZIP archive. There
% the packages are already unpacked and the documentation files
% are generated. The files and directories obey the TDS standard.
% \begin{description}
% \item[\CTAN{install/macros/latex/contrib/oberdiek.tds.zip}]
% \end{description}
% \emph{TDS} refers to the standard ``A Directory Structure
% for \TeX\ Files'' (\CTAN{tds/tds.pdf}). Directories
% with \xfile{texmf} in their name are usually organized this way.
%
% \subsection{Bundle installation}
%
% \paragraph{Unpacking.} Unpack the \xfile{oberdiek.tds.zip} in the
% TDS tree (also known as \xfile{texmf} tree) of your choice.
% Example (linux):
% \begin{quote}
%   |unzip oberdiek.tds.zip -d ~/texmf|
% \end{quote}
%
% \paragraph{Script installation.}
% Check the directory \xfile{TDS:scripts/oberdiek/} for
% scripts that need further installation steps.
% Package \xpackage{attachfile2} comes with the Perl script
% \xfile{pdfatfi.pl} that should be installed in such a way
% that it can be called as \texttt{pdfatfi}.
% Example (linux):
% \begin{quote}
%   |chmod +x scripts/oberdiek/pdfatfi.pl|\\
%   |cp scripts/oberdiek/pdfatfi.pl /usr/local/bin/|
% \end{quote}
%
% \subsection{Package installation}
%
% \paragraph{Unpacking.} The \xfile{.dtx} file is a self-extracting
% \docstrip\ archive. The files are extracted by running the
% \xfile{.dtx} through \plainTeX:
% \begin{quote}
%   \verb|tex hologo.dtx|
% \end{quote}
%
% \paragraph{TDS.} Now the different files must be moved into
% the different directories in your installation TDS tree
% (also known as \xfile{texmf} tree):
% \begin{quote}
% \def\t{^^A
% \begin{tabular}{@{}>{\ttfamily}l@{ $\rightarrow$ }>{\ttfamily}l@{}}
%   hologo.sty & tex/generic/oberdiek/hologo.sty\\
%   hologo.pdf & doc/latex/oberdiek/hologo.pdf\\
%   example/hologo-example.tex & doc/latex/oberdiek/example/hologo-example.tex\\
%   test/hologo-test1.tex & doc/latex/oberdiek/test/hologo-test1.tex\\
%   test/hologo-test-spacefactor.tex & doc/latex/oberdiek/test/hologo-test-spacefactor.tex\\
%   test/hologo-test-list.tex & doc/latex/oberdiek/test/hologo-test-list.tex\\
%   hologo.dtx & source/latex/oberdiek/hologo.dtx\\
% \end{tabular}^^A
% }^^A
% \sbox0{\t}^^A
% \ifdim\wd0>\linewidth
%   \begingroup
%     \advance\linewidth by\leftmargin
%     \advance\linewidth by\rightmargin
%   \edef\x{\endgroup
%     \def\noexpand\lw{\the\linewidth}^^A
%   }\x
%   \def\lwbox{^^A
%     \leavevmode
%     \hbox to \linewidth{^^A
%       \kern-\leftmargin\relax
%       \hss
%       \usebox0
%       \hss
%       \kern-\rightmargin\relax
%     }^^A
%   }^^A
%   \ifdim\wd0>\lw
%     \sbox0{\small\t}^^A
%     \ifdim\wd0>\linewidth
%       \ifdim\wd0>\lw
%         \sbox0{\footnotesize\t}^^A
%         \ifdim\wd0>\linewidth
%           \ifdim\wd0>\lw
%             \sbox0{\scriptsize\t}^^A
%             \ifdim\wd0>\linewidth
%               \ifdim\wd0>\lw
%                 \sbox0{\tiny\t}^^A
%                 \ifdim\wd0>\linewidth
%                   \lwbox
%                 \else
%                   \usebox0
%                 \fi
%               \else
%                 \lwbox
%               \fi
%             \else
%               \usebox0
%             \fi
%           \else
%             \lwbox
%           \fi
%         \else
%           \usebox0
%         \fi
%       \else
%         \lwbox
%       \fi
%     \else
%       \usebox0
%     \fi
%   \else
%     \lwbox
%   \fi
% \else
%   \usebox0
% \fi
% \end{quote}
% If you have a \xfile{docstrip.cfg} that configures and enables \docstrip's
% TDS installing feature, then some files can already be in the right
% place, see the documentation of \docstrip.
%
% \subsection{Refresh file name databases}
%
% If your \TeX~distribution
% (\teTeX, \mikTeX, \dots) relies on file name databases, you must refresh
% these. For example, \teTeX\ users run \verb|texhash| or
% \verb|mktexlsr|.
%
% \subsection{Some details for the interested}
%
% \paragraph{Attached source.}
%
% The PDF documentation on CTAN also includes the
% \xfile{.dtx} source file. It can be extracted by
% AcrobatReader 6 or higher. Another option is \textsf{pdftk},
% e.g. unpack the file into the current directory:
% \begin{quote}
%   \verb|pdftk hologo.pdf unpack_files output .|
% \end{quote}
%
% \paragraph{Unpacking with \LaTeX.}
% The \xfile{.dtx} chooses its action depending on the format:
% \begin{description}
% \item[\plainTeX:] Run \docstrip\ and extract the files.
% \item[\LaTeX:] Generate the documentation.
% \end{description}
% If you insist on using \LaTeX\ for \docstrip\ (really,
% \docstrip\ does not need \LaTeX), then inform the autodetect routine
% about your intention:
% \begin{quote}
%   \verb|latex \let\install=y% \iffalse meta-comment
%
% File: hologo.dtx
% Version: 2016/05/12 v1.11
% Info: A logo collection with bookmark support
%
% Copyright (C) 2010-2012 by
%    Heiko Oberdiek <heiko.oberdiek at googlemail.com>
%
% This work may be distributed and/or modified under the
% conditions of the LaTeX Project Public License, either
% version 1.3c of this license or (at your option) any later
% version. This version of this license is in
%    http://www.latex-project.org/lppl/lppl-1-3c.txt
% and the latest version of this license is in
%    http://www.latex-project.org/lppl.txt
% and version 1.3 or later is part of all distributions of
% LaTeX version 2005/12/01 or later.
%
% This work has the LPPL maintenance status "maintained".
%
% This Current Maintainer of this work is Heiko Oberdiek.
%
% The Base Interpreter refers to any `TeX-Format',
% because some files are installed in TDS:tex/generic//.
%
% This work consists of the main source file hologo.dtx
% and the derived files
%    hologo.sty, hologo.pdf, hologo.ins, hologo.drv, hologo-example.tex,
%    hologo-test1.tex, hologo-test-spacefactor.tex,
%    hologo-test-list.tex.
%
% Distribution:
%    CTAN:macros/latex/contrib/oberdiek/hologo.dtx
%    CTAN:macros/latex/contrib/oberdiek/hologo.pdf
%
% Unpacking:
%    (a) If hologo.ins is present:
%           tex hologo.ins
%    (b) Without hologo.ins:
%           tex hologo.dtx
%    (c) If you insist on using LaTeX
%           latex \let\install=y\input{hologo.dtx}
%        (quote the arguments according to the demands of your shell)
%
% Documentation:
%    (a) If hologo.drv is present:
%           latex hologo.drv
%    (b) Without hologo.drv:
%           latex hologo.dtx; ...
%    The class ltxdoc loads the configuration file ltxdoc.cfg
%    if available. Here you can specify further options, e.g.
%    use A4 as paper format:
%       \PassOptionsToClass{a4paper}{article}
%
%    Programm calls to get the documentation (example):
%       pdflatex hologo.dtx
%       makeindex -s gind.ist hologo.idx
%       pdflatex hologo.dtx
%       makeindex -s gind.ist hologo.idx
%       pdflatex hologo.dtx
%
% Installation:
%    TDS:tex/generic/oberdiek/hologo.sty
%    TDS:doc/latex/oberdiek/hologo.pdf
%    TDS:doc/latex/oberdiek/example/hologo-example.tex
%    TDS:doc/latex/oberdiek/test/hologo-test1.tex
%    TDS:doc/latex/oberdiek/test/hologo-test-spacefactor.tex
%    TDS:doc/latex/oberdiek/test/hologo-test-list.tex
%    TDS:source/latex/oberdiek/hologo.dtx
%
%<*ignore>
\begingroup
  \catcode123=1 %
  \catcode125=2 %
  \def\x{LaTeX2e}%
\expandafter\endgroup
\ifcase 0\ifx\install y1\fi\expandafter
         \ifx\csname processbatchFile\endcsname\relax\else1\fi
         \ifx\fmtname\x\else 1\fi\relax
\else\csname fi\endcsname
%</ignore>
%<*install>
\input docstrip.tex
\Msg{************************************************************************}
\Msg{* Installation}
\Msg{* Package: hologo 2016/05/12 v1.11 A logo collection with bookmark support (HO)}
\Msg{************************************************************************}

\keepsilent
\askforoverwritefalse

\let\MetaPrefix\relax
\preamble

This is a generated file.

Project: hologo
Version: 2016/05/12 v1.11

Copyright (C) 2010-2012 by
   Heiko Oberdiek <heiko.oberdiek at googlemail.com>

This work may be distributed and/or modified under the
conditions of the LaTeX Project Public License, either
version 1.3c of this license or (at your option) any later
version. This version of this license is in
   http://www.latex-project.org/lppl/lppl-1-3c.txt
and the latest version of this license is in
   http://www.latex-project.org/lppl.txt
and version 1.3 or later is part of all distributions of
LaTeX version 2005/12/01 or later.

This work has the LPPL maintenance status "maintained".

This Current Maintainer of this work is Heiko Oberdiek.

The Base Interpreter refers to any `TeX-Format',
because some files are installed in TDS:tex/generic//.

This work consists of the main source file hologo.dtx
and the derived files
   hologo.sty, hologo.pdf, hologo.ins, hologo.drv, hologo-example.tex,
   hologo-test1.tex, hologo-test-spacefactor.tex,
   hologo-test-list.tex.

\endpreamble
\let\MetaPrefix\DoubleperCent

\generate{%
  \file{hologo.ins}{\from{hologo.dtx}{install}}%
  \file{hologo.drv}{\from{hologo.dtx}{driver}}%
  \usedir{tex/generic/oberdiek}%
  \file{hologo.sty}{\from{hologo.dtx}{package}}%
  \usedir{doc/latex/oberdiek/example}%
  \file{hologo-example.tex}{\from{hologo.dtx}{example}}%
  \usedir{doc/latex/oberdiek/test}%
  \file{hologo-test1.tex}{\from{hologo.dtx}{test1}}%
  \file{hologo-test-spacefactor.tex}{\from{hologo.dtx}{test-spacefactor}}%
  \file{hologo-test-list.tex}{\from{hologo.dtx}{test-list}}%
  \nopreamble
  \nopostamble
  \usedir{source/latex/oberdiek/catalogue}%
  \file{hologo.xml}{\from{hologo.dtx}{catalogue}}%
}

\catcode32=13\relax% active space
\let =\space%
\Msg{************************************************************************}
\Msg{*}
\Msg{* To finish the installation you have to move the following}
\Msg{* file into a directory searched by TeX:}
\Msg{*}
\Msg{*     hologo.sty}
\Msg{*}
\Msg{* To produce the documentation run the file `hologo.drv'}
\Msg{* through LaTeX.}
\Msg{*}
\Msg{* Happy TeXing!}
\Msg{*}
\Msg{************************************************************************}

\endbatchfile
%</install>
%<*ignore>
\fi
%</ignore>
%<*driver>
\NeedsTeXFormat{LaTeX2e}
\ProvidesFile{hologo.drv}%
  [2016/05/12 v1.11 A logo collection with bookmark support (HO)]%
\documentclass{ltxdoc}
\usepackage{holtxdoc}[2011/11/22]
\usepackage{hologo}[2016/05/12]
\usepackage{longtable}
\usepackage{array}
\usepackage{paralist}
%\usepackage[T1]{fontenc}
%\usepackage{lmodern}
\begin{document}
  \DocInput{hologo.dtx}%
\end{document}
%</driver>
% \fi
%
%
% \CharacterTable
%  {Upper-case    \A\B\C\D\E\F\G\H\I\J\K\L\M\N\O\P\Q\R\S\T\U\V\W\X\Y\Z
%   Lower-case    \a\b\c\d\e\f\g\h\i\j\k\l\m\n\o\p\q\r\s\t\u\v\w\x\y\z
%   Digits        \0\1\2\3\4\5\6\7\8\9
%   Exclamation   \!     Double quote  \"     Hash (number) \#
%   Dollar        \$     Percent       \%     Ampersand     \&
%   Acute accent  \'     Left paren    \(     Right paren   \)
%   Asterisk      \*     Plus          \+     Comma         \,
%   Minus         \-     Point         \.     Solidus       \/
%   Colon         \:     Semicolon     \;     Less than     \<
%   Equals        \=     Greater than  \>     Question mark \?
%   Commercial at \@     Left bracket  \[     Backslash     \\
%   Right bracket \]     Circumflex    \^     Underscore    \_
%   Grave accent  \`     Left brace    \{     Vertical bar  \|
%   Right brace   \}     Tilde         \~}
%
% \GetFileInfo{hologo.drv}
%
% \title{The \xpackage{hologo} package}
% \date{2016/05/12 v1.11}
% \author{Heiko Oberdiek\\\xemail{heiko.oberdiek at googlemail.com}}
%
% \maketitle
%
% \begin{abstract}
% This package starts a collection of logos with support for bookmarks
% strings.
% \end{abstract}
%
% \tableofcontents
%
% \section{Documentation}
%
% \subsection{Logo macros}
%
% \begin{declcs}{hologo} \M{name}
% \end{declcs}
% Macro \cs{hologo} sets the logo with name \meta{name}.
% The following table shows the supported names.
%
% \begingroup
%   \def\hologoEntry#1#2#3{^^A
%     #1&#2&\hologoLogoSetup{#1}{variant=#2}\hologo{#1}&#3\tabularnewline
%   }
%   \begin{longtable}{>{\ttfamily}l>{\ttfamily}lll}
%     \rmfamily\bfseries{name} & \rmfamily\bfseries variant
%     & \bfseries logo & \bfseries since\\
%     \hline
%     \endhead
%     \hologoList
%   \end{longtable}
% \endgroup
%
% \begin{declcs}{Hologo} \M{name}
% \end{declcs}
% Macro \cs{Hologo} starts the logo \meta{name} with an uppercase
% letter. As an exception small greek letters are not converted
% to uppercase. Examples, see \hologo{eTeX} and \hologo{ExTeX}.
%
% \subsection{Setup macros}
%
% The package does not support package options, but the following
% setup macros can be used to set options.
%
% \begin{declcs}{hologoSetup} \M{key value list}
% \end{declcs}
% Macro \cs{hologoSetup} sets global options.
%
% \begin{declcs}{hologoLogoSetup} \M{logo} \M{key value list}
% \end{declcs}
% Some options can also be used to configure a logo.
% These settings take precedence over global option settings.
%
% \subsection{Options}\label{sec:options}
%
% There are boolean and string options:
% \begin{description}
% \item[Boolean option:]
% It takes |true| or |false|
% as value. If the value is omitted, then |true| is used.
% \item[String option:]
% A value must be given as string. (But the string might be empty.)
% \end{description}
% The following options can be used both in \cs{hologoSetup}
% and \cs{hologoLogoSetup}:
% \begin{description}
% \def\entry#1{\item[\xoption{#1}:]}
% \entry{break}
%   enables or disables line breaks inside the logo. This setting is
%   refined by options \xoption{hyphenbreak}, \xoption{spacebreak}
%   or \xoption{discretionarybreak}.
%   Default is |false|.
% \entry{hyphenbreak}
%   enables or disables the line break right after the hyphen character.
% \entry{spacebreak}
%   enables or disables line breaks at space characters.
% \entry{discretionarybreak}
%   enables or disables line breaks at hyphenation points
%   (inserted by \cs{-}).
% \end{description}
% Macro \cs{hologoLogoSetup} also knows:
% \begin{description}
% \item[\xoption{variant}:]
%   This is a string option. It specifies a variant of a logo that
%   must exist. An empty string selects the package default variant.
% \end{description}
% Example:
% \begin{quote}
%   |\hologoSetup{break=false}|\\
%   |\hologoLogoSetup{plainTeX}{variant=hyphen,hyphenbreak}|\\
%   Then ``plain-\TeX'' contains one break point after the hyphen.
% \end{quote}
%
% \subsection{Driver options}
%
% Sometimes graphical operations are needed to construct some
% glyphs (e.g.\ \hologo{XeTeX}). If package \xpackage{graphics}
% or package \xpackage{pgf} are found, then the macros are taken
% from there. Otherwise the packge defines its own operations
% and therefore needs the driver information. Many drivers are
% detected automatically (\hologo{pdfTeX}/\hologo{LuaTeX}
% in PDF mode, \hologo{XeTeX}, \hologo{VTeX}). These have precedence
% over a driver option. The driver can be given as package option
% or using \cs{hologoDriverSetup}.
% The following list contains the recognized driver options:
% \begin{itemize}
% \item \xoption{pdftex}, \xoption{luatex}
% \item \xoption{dvipdfm}, \xoption{dvipdfmx}
% \item \xoption{dvips}, \xoption{dvipsone}, \xoption{xdvi}
% \item \xoption{xetex}
% \item \xoption{vtex}
% \end{itemize}
% The left driver of a line is the driver name that is used internally.
% The following names are aliases for drivers that use the
% same method. Therefore the entry in the \xext{log} file for
% the used driver prints the internally used driver name.
% \begin{description}
% \item[\xoption{driverfallback}:]
%   This option expects a driver that is used,
%   if the driver could not be detected automatically.
% \end{description}
%
% \begin{declcs}{hologoDriverSetup} \M{driver option}
% \end{declcs}
% The driver can also be configured after package loading
% using \cs{hologoDriverSetup}, also the way for \hologo{plainTeX}
% to setup the driver.
%
% \subsection{Font setup}
%
% Some logos require a special font, but should also be usable by
% \hologo{plainTeX}. Therefore the package provides some ways
% to influence the font settings. The options below
% take font settings as values. Both font commands
% such as \cs{sffamily} and macros that take one argument
% like \cs{textsf} can be used.
%
% \begin{declcs}{hologoFontSetup} \M{key value list}
% \end{declcs}
% Macro \cs{hologoFontSetup} sets the fonts for all logos.
% Supported keys:
% \begin{description}
% \def\entry#1{\item[\xoption{#1}:]}
% \entry{general}
%   This font is used for all logos. The default is empty.
%   That means no special font is used.
% \entry{bibsf}
%   This font is used for
%   {\hologoLogoSetup{BibTeX}{variant=sf}\hologo{BibTeX}}
%   with variant \xoption{sf}.
% \entry{rm}
%   This font is a serif font. It is used for \hologo{ExTeX}.
% \entry{sc}
%   This font specifies a small caps font. It is used for
%   {\hologoLogoSetup{BibTeX}{variant=sc}\hologo{BibTeX}}
%   with variant \xoption{sc}.
% \entry{sf}
%   This font specifies a sans serif font. The default
%   is \cs{sffamily}, then \cs{sf} is tried. Otherwise
%   a warning is given. It is used by \hologo{KOMAScript}.
% \entry{sy}
%   This is the font for math symbols (e.g. cmsy).
%   It is used by \hologo{AmS}, \hologo{NTS}, \hologo{ExTeX}.
% \entry{logo}
%   \hologo{METAFONT} and \hologo{METAPOST} are using that font.
%   In \hologo{LaTeX} \cs{logofamily} is used and
%   the definitions of package \xpackage{mflogo} are used
%   if the package is not loaded.
%   Otherwise the \cs{tenlogo} is used and defined
%   if it does not already exists.
% \end{description}
%
% \begin{declcs}{hologoLogoFontSetup} \M{logo} \M{key value list}
% \end{declcs}
% Fonts can also be set for a logo or logo component separately,
% see the following list.
% The keys are the same as for \cs{hologoFontSetup}.
%
% \begin{longtable}{>{\ttfamily}l>{\sffamily}ll}
%   \meta{logo} & keys & result\\
%   \hline
%   \endhead
%   BibTeX & bibsf & {\hologoLogoSetup{BibTeX}{variant=sf}\hologo{BibTeX}}\\[.5ex]
%   BibTeX & sc & {\hologoLogoSetup{BibTeX}{variant=sc}\hologo{BibTeX}}\\[.5ex]
%   ExTeX & rm & \hologo{ExTeX}\\
%   SliTeX & rm & \hologo{SliTeX}\\[.5ex]
%   AmS & sy & \hologo{AmS}\\
%   ExTeX & sy & \hologo{ExTeX}\\
%   NTS & sy & \hologo{NTS}\\[.5ex]
%   KOMAScript & sf & \hologo{KOMAScript}\\[.5ex]
%   METAFONT & logo & \hologo{METAFONT}\\
%   METAPOST & logo & \hologo{METAPOST}\\[.5ex]
%   SliTeX & sc \hologo{SliTeX}
% \end{longtable}
%
% \subsubsection{Font order}
%
% For all logos the font \xoption{general} is applied first.
% Example:
%\begin{quote}
%|\hologoFontSetup{general=\color{red}}|
%\end{quote}
% will print red logos.
% Then if the font uses a special font \xoption{sf}, for example,
% the font is applied that is setup by \cs{hologoLogoFontSetup}.
% If this font is not setup, then the common font setup
% by \cs{hologoFontSetup} is used. Otherwise a warning is given,
% that there is no font configured.
%
% \subsection{Additional user macros}
%
% Usually a variant of a logo is configured by using
% \cs{hologoLogoSetup}, because it is bad style to mix
% different variants of the same logo in the same text.
% There the following macros are a convenience for testing.
%
% \begin{declcs}{hologoVariant} \M{name} \M{variant}\\
%   \cs{HologoVariant} \M{name} \M{variant}
% \end{declcs}
% Logo \meta{name} is set using \meta{variant} that specifies
% explicitely which variant of the macro is used. If the argument
% is empty, then the default form of the logo is used
% (configurable by \cs{hologoLogoSetup}).
%
% \cs{HologoVariant} is used if the logo is set in a context
% that needs an uppercase first letter (beginning of a sentence, \dots).
%
% \begin{declcs}{hologoList}\\
%   \cs{hologoEntry} \M{logo} \M{variant} \M{since}
% \end{declcs}
% Macro \cs{hologoList} contains all logos that are provided
% by the package including variants. The list consists of calls
% of \cs{hologoEntry} with three arguments starting with the
% logo name \meta{logo} and its variant \meta{variant}. An empty
% variant means the current default. Argument \meta{since} specifies
% with version of the package \xpackage{hologo} is needed to get
% the logo. If the logo is fixed, then the date gets updated.
% Therefore the date \meta{since} is not exactly the date of
% the first introduction, but rather the date of the latest fix.
%
% Before \cs{hologoList} can be used, macro \cs{hologoEntry} needs
% a definition. The example file in section \ref{sec:example}
% shows applications of \cs{hologoList}.
%
% \subsection{Supported contexts}
%
% Macros \cs{hologo} and friends support special contexts:
% \begin{itemize}
% \item \hologo{LaTeX}'s protection mechanism.
% \item Bookmarks of package \xpackage{hyperref}.
% \item Package \xpackage{tex4ht}.
% \item The macros can be used inside \cs{csname} constructs,
%   if \cs{ifincsname} is available (\hologo{pdfTeX}, \hologo{XeTeX},
%   \hologo{LuaTeX}).
% \end{itemize}
%
% \subsection{Example}
% \label{sec:example}
%
% The following example prints the logos in different fonts.
%    \begin{macrocode}
%<*example>
%<<verbatim
\NeedsTeXFormat{LaTeX2e}
\documentclass[a4paper]{article}
\usepackage[
  hmargin=20mm,
  vmargin=20mm,
]{geometry}
\pagestyle{empty}
\usepackage{hologo}[2016/05/12]
\usepackage{longtable}
\usepackage{array}
\setlength{\extrarowheight}{2pt}
\usepackage[T1]{fontenc}
\usepackage{lmodern}
\usepackage{pdflscape}
\usepackage[
  pdfencoding=auto,
]{hyperref}
\hypersetup{
  pdfauthor={Heiko Oberdiek},
  pdftitle={Example for package `hologo'},
  pdfsubject={Logos with fonts lmr, lmss, qtm, qpl, qhv},
}
\usepackage{bookmark}

% Print the logo list on the console

\begingroup
  \typeout{}%
  \typeout{*** Begin of logo list ***}%
  \newcommand*{\hologoEntry}[3]{%
    \typeout{#1 \ifx\\#2\\\else(#2) \fi[#3]}%
  }%
  \hologoList
  \typeout{*** End of logo list ***}%
  \typeout{}%
\endgroup

\begin{document}
\begin{landscape}

  \section{Example file for package `hologo'}

  % Table for font names

  \begin{longtable}{>{\bfseries}ll}
    \textbf{font} & \textbf{Font name}\\
    \hline
    lmr & Latin Modern Roman\\
    lmss & Latin Modern Sans\\
    qtm & \TeX\ Gyre Termes\\
    qhv & \TeX\ Gyre Heros\\
    qpl & \TeX\ Gyre Pagella\\
  \end{longtable}

  % Logo list with logos in different fonts

  \begingroup
    \newcommand*{\SetVariant}[2]{%
      \ifx\\#2\\%
      \else
        \hologoLogoSetup{#1}{variant=#2}%
      \fi
    }%
    \newcommand*{\hologoEntry}[3]{%
      \SetVariant{#1}{#2}%
      \raisebox{1em}[0pt][0pt]{\hypertarget{#1@#2}{}}%
      \bookmark[%
        dest={#1@#2},%
      ]{%
        #1\ifx\\#2\\\else\space(#2)\fi: \Hologo{#1}, \hologo{#1} %
        [Unicode]%
      }%
      \hypersetup{unicode=false}%
      \bookmark[%
        dest={#1@#2},%
      ]{%
        #1\ifx\\#2\\\else\space(#2)\fi: \Hologo{#1}, \hologo{#1} %
        [PDFDocEncoding]%
      }%
      \texttt{#1}%
      &%
      \texttt{#2}%
      &%
      \Hologo{#1}%
      &%
      \SetVariant{#1}{#2}%
      \hologo{#1}%
      &%
      \SetVariant{#1}{#2}%
      \fontfamily{qtm}\selectfont
      \hologo{#1}%
      &%
      \SetVariant{#1}{#2}%
      \fontfamily{qpl}\selectfont
      \hologo{#1}%
      &%
      \SetVariant{#1}{#2}%
      \textsf{\hologo{#1}}%
      &%
      \SetVariant{#1}{#2}%
      \fontfamily{qhv}\selectfont
      \hologo{#1}%
      \tabularnewline
    }%
    \begin{longtable}{llllllll}%
      \textbf{\textit{logo}} & \textbf{\textit{variant}} &
      \texttt{\string\Hologo} &
      \textbf{lmr} & \textbf{qtm} & \textbf{qpl} &
      \textbf{lmss} & \textbf{qhv}
      \tabularnewline
      \hline
      \endhead
      \hologoList
    \end{longtable}%
  \endgroup

\end{landscape}
\end{document}
%verbatim
%</example>
%    \end{macrocode}
%
% \StopEventually{
% }
%
% \section{Implementation}
%    \begin{macrocode}
%<*package>
%    \end{macrocode}
%    Reload check, especially if the package is not used with \LaTeX.
%    \begin{macrocode}
\begingroup\catcode61\catcode48\catcode32=10\relax%
  \catcode13=5 % ^^M
  \endlinechar=13 %
  \catcode35=6 % #
  \catcode39=12 % '
  \catcode44=12 % ,
  \catcode45=12 % -
  \catcode46=12 % .
  \catcode58=12 % :
  \catcode64=11 % @
  \catcode123=1 % {
  \catcode125=2 % }
  \expandafter\let\expandafter\x\csname ver@hologo.sty\endcsname
  \ifx\x\relax % plain-TeX, first loading
  \else
    \def\empty{}%
    \ifx\x\empty % LaTeX, first loading,
      % variable is initialized, but \ProvidesPackage not yet seen
    \else
      \expandafter\ifx\csname PackageInfo\endcsname\relax
        \def\x#1#2{%
          \immediate\write-1{Package #1 Info: #2.}%
        }%
      \else
        \def\x#1#2{\PackageInfo{#1}{#2, stopped}}%
      \fi
      \x{hologo}{The package is already loaded}%
      \aftergroup\endinput
    \fi
  \fi
\endgroup%
%    \end{macrocode}
%    Package identification:
%    \begin{macrocode}
\begingroup\catcode61\catcode48\catcode32=10\relax%
  \catcode13=5 % ^^M
  \endlinechar=13 %
  \catcode35=6 % #
  \catcode39=12 % '
  \catcode40=12 % (
  \catcode41=12 % )
  \catcode44=12 % ,
  \catcode45=12 % -
  \catcode46=12 % .
  \catcode47=12 % /
  \catcode58=12 % :
  \catcode64=11 % @
  \catcode91=12 % [
  \catcode93=12 % ]
  \catcode123=1 % {
  \catcode125=2 % }
  \expandafter\ifx\csname ProvidesPackage\endcsname\relax
    \def\x#1#2#3[#4]{\endgroup
      \immediate\write-1{Package: #3 #4}%
      \xdef#1{#4}%
    }%
  \else
    \def\x#1#2[#3]{\endgroup
      #2[{#3}]%
      \ifx#1\@undefined
        \xdef#1{#3}%
      \fi
      \ifx#1\relax
        \xdef#1{#3}%
      \fi
    }%
  \fi
\expandafter\x\csname ver@hologo.sty\endcsname
\ProvidesPackage{hologo}%
  [2016/05/12 v1.11 A logo collection with bookmark support (HO)]%
%    \end{macrocode}
%
%    \begin{macrocode}
\begingroup\catcode61\catcode48\catcode32=10\relax%
  \catcode13=5 % ^^M
  \endlinechar=13 %
  \catcode123=1 % {
  \catcode125=2 % }
  \catcode64=11 % @
  \def\x{\endgroup
    \expandafter\edef\csname HOLOGO@AtEnd\endcsname{%
      \endlinechar=\the\endlinechar\relax
      \catcode13=\the\catcode13\relax
      \catcode32=\the\catcode32\relax
      \catcode35=\the\catcode35\relax
      \catcode61=\the\catcode61\relax
      \catcode64=\the\catcode64\relax
      \catcode123=\the\catcode123\relax
      \catcode125=\the\catcode125\relax
    }%
  }%
\x\catcode61\catcode48\catcode32=10\relax%
\catcode13=5 % ^^M
\endlinechar=13 %
\catcode35=6 % #
\catcode64=11 % @
\catcode123=1 % {
\catcode125=2 % }
\def\TMP@EnsureCode#1#2{%
  \edef\HOLOGO@AtEnd{%
    \HOLOGO@AtEnd
    \catcode#1=\the\catcode#1\relax
  }%
  \catcode#1=#2\relax
}
\TMP@EnsureCode{10}{12}% ^^J
\TMP@EnsureCode{33}{12}% !
\TMP@EnsureCode{34}{12}% "
\TMP@EnsureCode{36}{3}% $
\TMP@EnsureCode{38}{4}% &
\TMP@EnsureCode{39}{12}% '
\TMP@EnsureCode{40}{12}% (
\TMP@EnsureCode{41}{12}% )
\TMP@EnsureCode{42}{12}% *
\TMP@EnsureCode{43}{12}% +
\TMP@EnsureCode{44}{12}% ,
\TMP@EnsureCode{45}{12}% -
\TMP@EnsureCode{46}{12}% .
\TMP@EnsureCode{47}{12}% /
\TMP@EnsureCode{58}{12}% :
\TMP@EnsureCode{59}{12}% ;
\TMP@EnsureCode{60}{12}% <
\TMP@EnsureCode{62}{12}% >
\TMP@EnsureCode{63}{12}% ?
\TMP@EnsureCode{91}{12}% [
\TMP@EnsureCode{93}{12}% ]
\TMP@EnsureCode{94}{7}% ^ (superscript)
\TMP@EnsureCode{95}{8}% _ (subscript)
\TMP@EnsureCode{96}{12}% `
\TMP@EnsureCode{124}{12}% |
\edef\HOLOGO@AtEnd{%
  \HOLOGO@AtEnd
  \escapechar\the\escapechar\relax
  \noexpand\endinput
}
\escapechar=92 %
%    \end{macrocode}
%
% \subsection{Logo list}
%
%    \begin{macro}{\hologoList}
%    \begin{macrocode}
\def\hologoList{%
  \hologoEntry{(La)TeX}{}{2011/10/01}%
  \hologoEntry{AmSLaTeX}{}{2010/04/16}%
  \hologoEntry{AmSTeX}{}{2010/04/16}%
  \hologoEntry{biber}{}{2011/10/01}%
  \hologoEntry{BibTeX}{}{2011/10/01}%
  \hologoEntry{BibTeX}{sf}{2011/10/01}%
  \hologoEntry{BibTeX}{sc}{2011/10/01}%
  \hologoEntry{BibTeX8}{}{2011/11/22}%
  \hologoEntry{ConTeXt}{}{2011/03/25}%
  \hologoEntry{ConTeXt}{narrow}{2011/03/25}%
  \hologoEntry{ConTeXt}{simple}{2011/03/25}%
  \hologoEntry{emTeX}{}{2010/04/26}%
  \hologoEntry{eTeX}{}{2010/04/08}%
  \hologoEntry{ExTeX}{}{2011/10/01}%
  \hologoEntry{HanTheThanh}{}{2011/11/29}%
  \hologoEntry{iniTeX}{}{2011/10/01}%
  \hologoEntry{KOMAScript}{}{2011/10/01}%
  \hologoEntry{La}{}{2010/05/08}%
  \hologoEntry{LaTeX}{}{2010/04/08}%
  \hologoEntry{LaTeX2e}{}{2010/04/08}%
  \hologoEntry{LaTeX3}{}{2010/04/24}%
  \hologoEntry{LaTeXe}{}{2010/04/08}%
  \hologoEntry{LaTeXML}{}{2011/11/22}%
  \hologoEntry{LaTeXTeX}{}{2011/10/01}%
  \hologoEntry{LuaLaTeX}{}{2010/04/08}%
  \hologoEntry{LuaTeX}{}{2010/04/08}%
  \hologoEntry{LyX}{}{2011/10/01}%
  \hologoEntry{METAFONT}{}{2011/10/01}%
  \hologoEntry{MetaFun}{}{2011/10/01}%
  \hologoEntry{METAPOST}{}{2011/10/01}%
  \hologoEntry{MetaPost}{}{2011/10/01}%
  \hologoEntry{MiKTeX}{}{2011/10/01}%
  \hologoEntry{NTS}{}{2011/10/01}%
  \hologoEntry{OzMF}{}{2011/10/01}%
  \hologoEntry{OzMP}{}{2011/10/01}%
  \hologoEntry{OzTeX}{}{2011/10/01}%
  \hologoEntry{OzTtH}{}{2011/10/01}%
  \hologoEntry{PCTeX}{}{2011/10/01}%
  \hologoEntry{pdfTeX}{}{2011/10/01}%
  \hologoEntry{pdfLaTeX}{}{2011/10/01}%
  \hologoEntry{PiC}{}{2011/10/01}%
  \hologoEntry{PiCTeX}{}{2011/10/01}%
  \hologoEntry{plainTeX}{}{2010/04/08}%
  \hologoEntry{plainTeX}{space}{2010/04/16}%
  \hologoEntry{plainTeX}{hyphen}{2010/04/16}%
  \hologoEntry{plainTeX}{runtogether}{2010/04/16}%
  \hologoEntry{SageTeX}{}{2011/11/22}%
  \hologoEntry{SLiTeX}{}{2011/10/01}%
  \hologoEntry{SLiTeX}{lift}{2011/10/01}%
  \hologoEntry{SLiTeX}{narrow}{2011/10/01}%
  \hologoEntry{SLiTeX}{simple}{2011/10/01}%
  \hologoEntry{SliTeX}{}{2011/10/01}%
  \hologoEntry{SliTeX}{narrow}{2011/10/01}%
  \hologoEntry{SliTeX}{simple}{2011/10/01}%
  \hologoEntry{SliTeX}{lift}{2011/10/01}%
  \hologoEntry{teTeX}{}{2011/10/01}%
  \hologoEntry{TeX}{}{2010/04/08}%
  \hologoEntry{TeX4ht}{}{2011/11/22}%
  \hologoEntry{TTH}{}{2011/11/22}%
  \hologoEntry{virTeX}{}{2011/10/01}%
  \hologoEntry{VTeX}{}{2010/04/24}%
  \hologoEntry{Xe}{}{2010/04/08}%
  \hologoEntry{XeLaTeX}{}{2010/04/08}%
  \hologoEntry{XeTeX}{}{2010/04/08}%
}
%    \end{macrocode}
%    \end{macro}
%
% \subsection{Load resources}
%
%    \begin{macrocode}
\begingroup\expandafter\expandafter\expandafter\endgroup
\expandafter\ifx\csname RequirePackage\endcsname\relax
  \def\TMP@RequirePackage#1[#2]{%
    \begingroup\expandafter\expandafter\expandafter\endgroup
    \expandafter\ifx\csname ver@#1.sty\endcsname\relax
      \input #1.sty\relax
    \fi
  }%
  \TMP@RequirePackage{ltxcmds}[2011/02/04]%
  \TMP@RequirePackage{infwarerr}[2010/04/08]%
  \TMP@RequirePackage{kvsetkeys}[2010/03/01]%
  \TMP@RequirePackage{kvdefinekeys}[2010/03/01]%
  \TMP@RequirePackage{pdftexcmds}[2010/04/01]%
  \TMP@RequirePackage{ifpdf}[2010/01/28]%
  \TMP@RequirePackage{ifluatex}[2010/03/01]%
  \ltx@IfUndefined{newif}{%
    \expandafter\let\csname newif\endcsname\ltx@newif
  }{}%
  \TMP@RequirePackage{ifxetex}[2009/01/23]%
  \TMP@RequirePackage{ifvtex}[2010/03/01]%
\else
  \RequirePackage{ltxcmds}[2011/02/04]%
  \RequirePackage{infwarerr}[2010/04/08]%
  \RequirePackage{kvsetkeys}[2010/03/01]%
  \RequirePackage{kvdefinekeys}[2010/03/01]%
  \RequirePackage{pdftexcmds}[2010/04/01]%
  \RequirePackage{ifpdf}[2010/01/28]%
  \RequirePackage{ifluatex}[2010/03/01]%
  \RequirePackage{ifxetex}[2009/01/23]%
  \RequirePackage{ifvtex}[2010/03/01]%
\fi
%    \end{macrocode}
%
%    \begin{macro}{\HOLOGO@IfDefined}
%    \begin{macrocode}
\def\HOLOGO@IfExists#1{%
  \ifx\@undefined#1%
    \expandafter\ltx@secondoftwo
  \else
    \ifx\relax#1%
      \expandafter\ltx@secondoftwo
    \else
      \expandafter\expandafter\expandafter\ltx@firstoftwo
    \fi
  \fi
}
%    \end{macrocode}
%    \end{macro}
%
% \subsection{Setup macros}
%
%    \begin{macro}{\hologoSetup}
%    \begin{macrocode}
\def\hologoSetup{%
  \let\HOLOGO@name\relax
  \HOLOGO@Setup
}
%    \end{macrocode}
%    \end{macro}
%
%    \begin{macro}{\hologoLogoSetup}
%    \begin{macrocode}
\def\hologoLogoSetup#1{%
  \edef\HOLOGO@name{#1}%
  \ltx@IfUndefined{HoLogo@\HOLOGO@name}{%
    \@PackageError{hologo}{%
      Unknown logo `\HOLOGO@name'%
    }\@ehc
    \ltx@gobble
  }{%
    \HOLOGO@Setup
  }%
}
%    \end{macrocode}
%    \end{macro}
%
%    \begin{macro}{\HOLOGO@Setup}
%    \begin{macrocode}
\def\HOLOGO@Setup{%
  \kvsetkeys{HoLogo}%
}
%    \end{macrocode}
%    \end{macro}
%
% \subsection{Options}
%
%    \begin{macro}{\HOLOGO@DeclareBoolOption}
%    \begin{macrocode}
\def\HOLOGO@DeclareBoolOption#1{%
  \expandafter\chardef\csname HOLOGOOPT@#1\endcsname\ltx@zero
  \kv@define@key{HoLogo}{#1}[true]{%
    \def\HOLOGO@temp{##1}%
    \ifx\HOLOGO@temp\HOLOGO@true
      \ifx\HOLOGO@name\relax
        \expandafter\chardef\csname HOLOGOOPT@#1\endcsname=\ltx@one
      \else
        \expandafter\chardef\csname
        HoLogoOpt@#1@\HOLOGO@name\endcsname\ltx@one
      \fi
      \HOLOGO@SetBreakAll{#1}%
    \else
      \ifx\HOLOGO@temp\HOLOGO@false
        \ifx\HOLOGO@name\relax
          \expandafter\chardef\csname HOLOGOOPT@#1\endcsname=\ltx@zero
        \else
          \expandafter\chardef\csname
          HoLogoOpt@#1@\HOLOGO@name\endcsname=\ltx@zero
        \fi
        \HOLOGO@SetBreakAll{#1}%
      \else
        \@PackageError{hologo}{%
          Unknown value `##1' for boolean option `#1'.\MessageBreak
          Known values are `true' and `false'%
        }\@ehc
      \fi
    \fi
  }%
}
%    \end{macrocode}
%    \end{macro}
%
%    \begin{macro}{\HOLOGO@SetBreakAll}
%    \begin{macrocode}
\def\HOLOGO@SetBreakAll#1{%
  \def\HOLOGO@temp{#1}%
  \ifx\HOLOGO@temp\HOLOGO@break
    \ifx\HOLOGO@name\relax
      \chardef\HOLOGOOPT@hyphenbreak=\HOLOGOOPT@break
      \chardef\HOLOGOOPT@spacebreak=\HOLOGOOPT@break
      \chardef\HOLOGOOPT@discretionarybreak=\HOLOGOOPT@break
    \else
      \expandafter\chardef
         \csname HoLogoOpt@hyphenbreak@\HOLOGO@name\endcsname=%
         \csname HoLogoOpt@break@\HOLOGO@name\endcsname
      \expandafter\chardef
         \csname HoLogoOpt@spacebreak@\HOLOGO@name\endcsname=%
         \csname HoLogoOpt@break@\HOLOGO@name\endcsname
      \expandafter\chardef
         \csname HoLogoOpt@discretionarybreak@\HOLOGO@name
             \endcsname=%
         \csname HoLogoOpt@break@\HOLOGO@name\endcsname
    \fi
  \fi
}
%    \end{macrocode}
%    \end{macro}
%
%    \begin{macro}{\HOLOGO@true}
%    \begin{macrocode}
\def\HOLOGO@true{true}
%    \end{macrocode}
%    \end{macro}
%    \begin{macro}{\HOLOGO@false}
%    \begin{macrocode}
\def\HOLOGO@false{false}
%    \end{macrocode}
%    \end{macro}
%    \begin{macro}{\HOLOGO@break}
%    \begin{macrocode}
\def\HOLOGO@break{break}
%    \end{macrocode}
%    \end{macro}
%
%    \begin{macrocode}
\HOLOGO@DeclareBoolOption{break}
\HOLOGO@DeclareBoolOption{hyphenbreak}
\HOLOGO@DeclareBoolOption{spacebreak}
\HOLOGO@DeclareBoolOption{discretionarybreak}
%    \end{macrocode}
%
%    \begin{macrocode}
\kv@define@key{HoLogo}{variant}{%
  \ifx\HOLOGO@name\relax
    \@PackageError{hologo}{%
      Option `variant' is not available in \string\hologoSetup,%
      \MessageBreak
      Use \string\hologoLogoSetup\space instead%
    }\@ehc
  \else
    \edef\HOLOGO@temp{#1}%
    \ifx\HOLOGO@temp\ltx@empty
      \expandafter
      \let\csname HoLogoOpt@variant@\HOLOGO@name\endcsname\@undefined
    \else
      \ltx@IfUndefined{HoLogo@\HOLOGO@name @\HOLOGO@temp}{%
        \@PackageError{hologo}{%
          Unknown variant `\HOLOGO@temp' of logo `\HOLOGO@name'%
        }\@ehc
      }{%
        \expandafter
        \let\csname HoLogoOpt@variant@\HOLOGO@name\endcsname
            \HOLOGO@temp
      }%
    \fi
  \fi
}
%    \end{macrocode}
%
%    \begin{macro}{\HOLOGO@Variant}
%    \begin{macrocode}
\def\HOLOGO@Variant#1{%
  #1%
  \ltx@ifundefined{HoLogoOpt@variant@#1}{%
  }{%
    @\csname HoLogoOpt@variant@#1\endcsname
  }%
}
%    \end{macrocode}
%    \end{macro}
%
% \subsection{Break/no-break support}
%
%    \begin{macro}{\HOLOGO@space}
%    \begin{macrocode}
\def\HOLOGO@space{%
  \ltx@ifundefined{HoLogoOpt@spacebreak@\HOLOGO@name}{%
    \ltx@ifundefined{HoLogoOpt@break@\HOLOGO@name}{%
      \chardef\HOLOGO@temp=\HOLOGOOPT@spacebreak
    }{%
      \chardef\HOLOGO@temp=%
        \csname HoLogoOpt@break@\HOLOGO@name\endcsname
    }%
  }{%
    \chardef\HOLOGO@temp=%
      \csname HoLogoOpt@spacebreak@\HOLOGO@name\endcsname
  }%
  \ifcase\HOLOGO@temp
    \penalty10000 %
  \fi
  \ltx@space
}
%    \end{macrocode}
%    \end{macro}
%
%    \begin{macro}{\HOLOGO@hyphen}
%    \begin{macrocode}
\def\HOLOGO@hyphen{%
  \ltx@ifundefined{HoLogoOpt@hyphenbreak@\HOLOGO@name}{%
    \ltx@ifundefined{HoLogoOpt@break@\HOLOGO@name}{%
      \chardef\HOLOGO@temp=\HOLOGOOPT@hyphenbreak
    }{%
      \chardef\HOLOGO@temp=%
        \csname HoLogoOpt@break@\HOLOGO@name\endcsname
    }%
  }{%
    \chardef\HOLOGO@temp=%
      \csname HoLogoOpt@hyphenbreak@\HOLOGO@name\endcsname
  }%
  \ifcase\HOLOGO@temp
    \ltx@mbox{-}%
  \else
    -%
  \fi
}
%    \end{macrocode}
%    \end{macro}
%
%    \begin{macro}{\HOLOGO@discretionary}
%    \begin{macrocode}
\def\HOLOGO@discretionary{%
  \ltx@ifundefined{HoLogoOpt@discretionarybreak@\HOLOGO@name}{%
    \ltx@ifundefined{HoLogoOpt@break@\HOLOGO@name}{%
      \chardef\HOLOGO@temp=\HOLOGOOPT@discretionarybreak
    }{%
      \chardef\HOLOGO@temp=%
        \csname HoLogoOpt@break@\HOLOGO@name\endcsname
    }%
  }{%
    \chardef\HOLOGO@temp=%
      \csname HoLogoOpt@discretionarybreak@\HOLOGO@name\endcsname
  }%
  \ifcase\HOLOGO@temp
  \else
    \-%
  \fi
}
%    \end{macrocode}
%    \end{macro}
%
%    \begin{macro}{\HOLOGO@mbox}
%    \begin{macrocode}
\def\HOLOGO@mbox#1{%
  \ltx@ifundefined{HoLogoOpt@break@\HOLOGO@name}{%
    \chardef\HOLOGO@temp=\HOLOGOOPT@hyphenbreak
  }{%
    \chardef\HOLOGO@temp=%
      \csname HoLogoOpt@break@\HOLOGO@name\endcsname
  }%
  \ifcase\HOLOGO@temp
    \ltx@mbox{#1}%
  \else
    #1%
  \fi
}
%    \end{macrocode}
%    \end{macro}
%
% \subsection{Font support}
%
%    \begin{macro}{\HoLogoFont@font}
%    \begin{tabular}{@{}ll@{}}
%    |#1|:& logo name\\
%    |#2|:& font short name\\
%    |#3|:& text
%    \end{tabular}
%    \begin{macrocode}
\def\HoLogoFont@font#1#2#3{%
  \begingroup
    \ltx@IfUndefined{HoLogoFont@logo@#1.#2}{%
      \ltx@IfUndefined{HoLogoFont@font@#2}{%
        \@PackageWarning{hologo}{%
          Missing font `#2' for logo `#1'%
        }%
        #3%
      }{%
        \csname HoLogoFont@font@#2\endcsname{#3}%
      }%
    }{%
      \csname HoLogoFont@logo@#1.#2\endcsname{#3}%
    }%
  \endgroup
}
%    \end{macrocode}
%    \end{macro}
%
%    \begin{macro}{\HoLogoFont@Def}
%    \begin{macrocode}
\def\HoLogoFont@Def#1{%
  \expandafter\def\csname HoLogoFont@font@#1\endcsname
}
%    \end{macrocode}
%    \end{macro}
%    \begin{macro}{\HoLogoFont@LogoDef}
%    \begin{macrocode}
\def\HoLogoFont@LogoDef#1#2{%
  \expandafter\def\csname HoLogoFont@logo@#1.#2\endcsname
}
%    \end{macrocode}
%    \end{macro}
%
% \subsubsection{Font defaults}
%
%    \begin{macro}{\HoLogoFont@font@general}
%    \begin{macrocode}
\HoLogoFont@Def{general}{}%
%    \end{macrocode}
%    \end{macro}
%
%    \begin{macro}{\HoLogoFont@font@rm}
%    \begin{macrocode}
\ltx@IfUndefined{rmfamily}{%
  \ltx@IfUndefined{rm}{%
  }{%
    \HoLogoFont@Def{rm}{\rm}%
  }%
}{%
  \HoLogoFont@Def{rm}{\rmfamily}%
}
%    \end{macrocode}
%    \end{macro}
%
%    \begin{macro}{\HoLogoFont@font@sf}
%    \begin{macrocode}
\ltx@IfUndefined{sffamily}{%
  \ltx@IfUndefined{sf}{%
  }{%
    \HoLogoFont@Def{sf}{\sf}%
  }%
}{%
  \HoLogoFont@Def{sf}{\sffamily}%
}
%    \end{macrocode}
%    \end{macro}
%
%    \begin{macro}{\HoLogoFont@font@bibsf}
%    In case of \hologo{plainTeX} the original small caps
%    variant is used as default. In \hologo{LaTeX}
%    the definition of package \xpackage{dtklogos} \cite{dtklogos}
%    is used.
%\begin{quote}
%\begin{verbatim}
%\DeclareRobustCommand{\BibTeX}{%
%  B%
%  \kern-.05em%
%  \hbox{%
%    $\m@th$% %% force math size calculations
%    \csname S@\f@size\endcsname
%    \fontsize\sf@size\z@
%    \math@fontsfalse
%    \selectfont
%    I%
%    \kern-.025em%
%    B
%  }%
%  \kern-.08em%
%  \-%
%  \TeX
%}
%\end{verbatim}
%\end{quote}
%    \begin{macrocode}
\ltx@IfUndefined{selectfont}{%
  \ltx@IfUndefined{tensc}{%
    \font\tensc=cmcsc10\relax
  }{}%
  \HoLogoFont@Def{bibsf}{\tensc}%
}{%
  \HoLogoFont@Def{bibsf}{%
    $\mathsurround=0pt$%
    \csname S@\f@size\endcsname
    \fontsize\sf@size{0pt}%
    \math@fontsfalse
    \selectfont
  }%
}
%    \end{macrocode}
%    \end{macro}
%
%    \begin{macro}{\HoLogoFont@font@sc}
%    \begin{macrocode}
\ltx@IfUndefined{scshape}{%
  \ltx@IfUndefined{tensc}{%
    \font\tensc=cmcsc10\relax
  }{}%
  \HoLogoFont@Def{sc}{\tensc}%
}{%
  \HoLogoFont@Def{sc}{\scshape}%
}
%    \end{macrocode}
%    \end{macro}
%
%    \begin{macro}{\HoLogoFont@font@sy}
%    \begin{macrocode}
\ltx@IfUndefined{usefont}{%
  \ltx@IfUndefined{tensy}{%
  }{%
    \HoLogoFont@Def{sy}{\tensy}%
  }%
}{%
  \HoLogoFont@Def{sy}{%
    \usefont{OMS}{cmsy}{m}{n}%
  }%
}
%    \end{macrocode}
%    \end{macro}
%
%    \begin{macro}{\HoLogoFont@font@logo}
%    \begin{macrocode}
\begingroup
  \def\x{LaTeX2e}%
\expandafter\endgroup
\ifx\fmtname\x
  \ltx@IfUndefined{logofamily}{%
    \DeclareRobustCommand\logofamily{%
      \not@math@alphabet\logofamily\relax
      \fontencoding{U}%
      \fontfamily{logo}%
      \selectfont
    }%
  }{}%
  \ltx@IfUndefined{logofamily}{%
  }{%
    \HoLogoFont@Def{logo}{\logofamily}%
  }%
\else
  \ltx@IfUndefined{tenlogo}{%
    \font\tenlogo=logo10\relax
  }{}%
  \HoLogoFont@Def{logo}{\tenlogo}%
\fi
%    \end{macrocode}
%    \end{macro}
%
% \subsubsection{Font setup}
%
%    \begin{macro}{\hologoFontSetup}
%    \begin{macrocode}
\def\hologoFontSetup{%
  \let\HOLOGO@name\relax
  \HOLOGO@FontSetup
}
%    \end{macrocode}
%    \end{macro}
%
%    \begin{macro}{\hologoLogoFontSetup}
%    \begin{macrocode}
\def\hologoLogoFontSetup#1{%
  \edef\HOLOGO@name{#1}%
  \ltx@IfUndefined{HoLogo@\HOLOGO@name}{%
    \@PackageError{hologo}{%
      Unknown logo `\HOLOGO@name'%
    }\@ehc
    \ltx@gobble
  }{%
    \HOLOGO@FontSetup
  }%
}
%    \end{macrocode}
%    \end{macro}
%
%    \begin{macro}{\HOLOGO@FontSetup}
%    \begin{macrocode}
\def\HOLOGO@FontSetup{%
  \kvsetkeys{HoLogoFont}%
}
%    \end{macrocode}
%    \end{macro}
%
%    \begin{macrocode}
\def\HOLOGO@temp#1{%
  \kv@define@key{HoLogoFont}{#1}{%
    \ifx\HOLOGO@name\relax
      \HoLogoFont@Def{#1}{##1}%
    \else
      \HoLogoFont@LogoDef\HOLOGO@name{#1}{##1}%
    \fi
  }%
}
\HOLOGO@temp{general}
\HOLOGO@temp{sf}
%    \end{macrocode}
%
% \subsection{Generic logo commands}
%
%    \begin{macrocode}
\HOLOGO@IfExists\hologo{%
  \@PackageError{hologo}{%
    \string\hologo\ltx@space is already defined.\MessageBreak
    Package loading is aborted%
  }\@ehc
  \HOLOGO@AtEnd
}%
\HOLOGO@IfExists\hologoRobust{%
  \@PackageError{hologo}{%
    \string\hologoRobust\ltx@space is already defined.\MessageBreak
    Package loading is aborted%
  }\@ehc
  \HOLOGO@AtEnd
}%
%    \end{macrocode}
%
% \subsubsection{\cs{hologo} and friends}
%
%    \begin{macrocode}
\ifluatex
  \expandafter\ltx@firstofone
\else
  \expandafter\ltx@gobble
\fi
{%
  \ltx@IfUndefined{ifincsname}{%
    \ifnum\luatexversion<36 %
      \expandafter\ltx@gobble
    \else
      \expandafter\ltx@firstofone
    \fi
    {%
      \begingroup
        \ifcase0%
            \directlua{%
              if tex.enableprimitives then %
                tex.enableprimitives('HOLOGO@', {'ifincsname'})%
              else %
                tex.print('1')%
              end%
            }%
            \ifx\HOLOGO@ifincsname\@undefined 1\fi%
            \relax
          \expandafter\ltx@firstofone
        \else
          \endgroup
          \expandafter\ltx@gobble
        \fi
        {%
          \global\let\ifincsname\HOLOGO@ifincsname
        }%
      \HOLOGO@temp
    }%
  }{}%
}
%    \end{macrocode}
%    \begin{macrocode}
\ltx@IfUndefined{ifincsname}{%
  \catcode`$=14 %
}{%
  \catcode`$=9 %
}
%    \end{macrocode}
%
%    \begin{macro}{\hologo}
%    \begin{macrocode}
\def\hologo#1{%
$ \ifincsname
$   \ltx@ifundefined{HoLogoCs@\HOLOGO@Variant{#1}}{%
$     #1%
$   }{%
$     \csname HoLogoCs@\HOLOGO@Variant{#1}\endcsname\ltx@firstoftwo
$   }%
$ \else
    \HOLOGO@IfExists\texorpdfstring\texorpdfstring\ltx@firstoftwo
    {%
      \hologoRobust{#1}%
    }{%
      \ltx@ifundefined{HoLogoBkm@\HOLOGO@Variant{#1}}{%
        \ltx@ifundefined{HoLogo@#1}{?#1?}{#1}%
      }{%
        \csname HoLogoBkm@\HOLOGO@Variant{#1}\endcsname
        \ltx@firstoftwo
      }%
    }%
$ \fi
}
%    \end{macrocode}
%    \end{macro}
%    \begin{macro}{\Hologo}
%    \begin{macrocode}
\def\Hologo#1{%
$ \ifincsname
$   \ltx@ifundefined{HoLogoCs@\HOLOGO@Variant{#1}}{%
$     #1%
$   }{%
$     \csname HoLogoCs@\HOLOGO@Variant{#1}\endcsname\ltx@secondoftwo
$   }%
$ \else
    \HOLOGO@IfExists\texorpdfstring\texorpdfstring\ltx@firstoftwo
    {%
      \HologoRobust{#1}%
    }{%
      \ltx@ifundefined{HoLogoBkm@\HOLOGO@Variant{#1}}{%
        \ltx@ifundefined{HoLogo@#1}{?#1?}{#1}%
      }{%
        \csname HoLogoBkm@\HOLOGO@Variant{#1}\endcsname
        \ltx@secondoftwo
      }%
    }%
$ \fi
}
%    \end{macrocode}
%    \end{macro}
%
%    \begin{macro}{\hologoVariant}
%    \begin{macrocode}
\def\hologoVariant#1#2{%
  \ifx\relax#2\relax
    \hologo{#1}%
  \else
$   \ifincsname
$     \ltx@ifundefined{HoLogoCs@#1@#2}{%
$       #1%
$     }{%
$       \csname HoLogoCs@#1@#2\endcsname\ltx@firstoftwo
$     }%
$   \else
      \HOLOGO@IfExists\texorpdfstring\texorpdfstring\ltx@firstoftwo
      {%
        \hologoVariantRobust{#1}{#2}%
      }{%
        \ltx@ifundefined{HoLogoBkm@#1@#2}{%
          \ltx@ifundefined{HoLogo@#1}{?#1?}{#1}%
        }{%
          \csname HoLogoBkm@#1@#2\endcsname
          \ltx@firstoftwo
        }%
      }%
$   \fi
  \fi
}
%    \end{macrocode}
%    \end{macro}
%    \begin{macro}{\HologoVariant}
%    \begin{macrocode}
\def\HologoVariant#1#2{%
  \ifx\relax#2\relax
    \Hologo{#1}%
  \else
$   \ifincsname
$     \ltx@ifundefined{HoLogoCs@#1@#2}{%
$       #1%
$     }{%
$       \csname HoLogoCs@#1@#2\endcsname\ltx@secondoftwo
$     }%
$   \else
      \HOLOGO@IfExists\texorpdfstring\texorpdfstring\ltx@firstoftwo
      {%
        \HologoVariantRobust{#1}{#2}%
      }{%
        \ltx@ifundefined{HoLogoBkm@#1@#2}{%
          \ltx@ifundefined{HoLogo@#1}{?#1?}{#1}%
        }{%
          \csname HoLogoBkm@#1@#2\endcsname
          \ltx@secondoftwo
        }%
      }%
$   \fi
  \fi
}
%    \end{macrocode}
%    \end{macro}
%
%    \begin{macrocode}
\catcode`\$=3 %
%    \end{macrocode}
%
% \subsubsection{\cs{hologoRobust} and friends}
%
%    \begin{macro}{\hologoRobust}
%    \begin{macrocode}
\ltx@IfUndefined{protected}{%
  \ltx@IfUndefined{DeclareRobustCommand}{%
    \def\hologoRobust#1%
  }{%
    \DeclareRobustCommand*\hologoRobust[1]%
  }%
}{%
  \protected\def\hologoRobust#1%
}%
{%
  \edef\HOLOGO@name{#1}%
  \ltx@IfUndefined{HoLogo@\HOLOGO@Variant\HOLOGO@name}{%
    \@PackageError{hologo}{%
      Unknown logo `\HOLOGO@name'%
    }\@ehc
    ?\HOLOGO@name?%
  }{%
    \ltx@IfUndefined{ver@tex4ht.sty}{%
      \HoLogoFont@font\HOLOGO@name{general}{%
        \csname HoLogo@\HOLOGO@Variant\HOLOGO@name\endcsname
        \ltx@firstoftwo
      }%
    }{%
      \ltx@IfUndefined{HoLogoHtml@\HOLOGO@Variant\HOLOGO@name}{%
        \HOLOGO@name
      }{%
        \csname HoLogoHtml@\HOLOGO@Variant\HOLOGO@name\endcsname
        \ltx@firstoftwo
      }%
    }%
  }%
}
%    \end{macrocode}
%    \end{macro}
%    \begin{macro}{\HologoRobust}
%    \begin{macrocode}
\ltx@IfUndefined{protected}{%
  \ltx@IfUndefined{DeclareRobustCommand}{%
    \def\HologoRobust#1%
  }{%
    \DeclareRobustCommand*\HologoRobust[1]%
  }%
}{%
  \protected\def\HologoRobust#1%
}%
{%
  \edef\HOLOGO@name{#1}%
  \ltx@IfUndefined{HoLogo@\HOLOGO@Variant\HOLOGO@name}{%
    \@PackageError{hologo}{%
      Unknown logo `\HOLOGO@name'%
    }\@ehc
    ?\HOLOGO@name?%
  }{%
    \ltx@IfUndefined{ver@tex4ht.sty}{%
      \HoLogoFont@font\HOLOGO@name{general}{%
        \csname HoLogo@\HOLOGO@Variant\HOLOGO@name\endcsname
        \ltx@secondoftwo
      }%
    }{%
      \ltx@IfUndefined{HoLogoHtml@\HOLOGO@Variant\HOLOGO@name}{%
        \expandafter\HOLOGO@Uppercase\HOLOGO@name
      }{%
        \csname HoLogoHtml@\HOLOGO@Variant\HOLOGO@name\endcsname
        \ltx@secondoftwo
      }%
    }%
  }%
}
%    \end{macrocode}
%    \end{macro}
%    \begin{macro}{\hologoVariantRobust}
%    \begin{macrocode}
\ltx@IfUndefined{protected}{%
  \ltx@IfUndefined{DeclareRobustCommand}{%
    \def\hologoVariantRobust#1#2%
  }{%
    \DeclareRobustCommand*\hologoVariantRobust[2]%
  }%
}{%
  \protected\def\hologoVariantRobust#1#2%
}%
{%
  \begingroup
    \hologoLogoSetup{#1}{variant={#2}}%
    \hologoRobust{#1}%
  \endgroup
}
%    \end{macrocode}
%    \end{macro}
%    \begin{macro}{\HologoVariantRobust}
%    \begin{macrocode}
\ltx@IfUndefined{protected}{%
  \ltx@IfUndefined{DeclareRobustCommand}{%
    \def\HologoVariantRobust#1#2%
  }{%
    \DeclareRobustCommand*\HologoVariantRobust[2]%
  }%
}{%
  \protected\def\HologoVariantRobust#1#2%
}%
{%
  \begingroup
    \hologoLogoSetup{#1}{variant={#2}}%
    \HologoRobust{#1}%
  \endgroup
}
%    \end{macrocode}
%    \end{macro}
%
%    \begin{macro}{\hologorobust}
%    Macro \cs{hologorobust} is only defined for compatibility.
%    Its use is deprecated.
%    \begin{macrocode}
\def\hologorobust{\hologoRobust}
%    \end{macrocode}
%    \end{macro}
%
% \subsection{Helpers}
%
%    \begin{macro}{\HOLOGO@Uppercase}
%    Macro \cs{HOLOGO@Uppercase} is restricted to \cs{uppercase},
%    because \hologo{plainTeX} or \hologo{iniTeX} do not provide
%    \cs{MakeUppercase}.
%    \begin{macrocode}
\def\HOLOGO@Uppercase#1{\uppercase{#1}}
%    \end{macrocode}
%    \end{macro}
%
%    \begin{macro}{\HOLOGO@PdfdocUnicode}
%    \begin{macrocode}
\def\HOLOGO@PdfdocUnicode{%
  \ifx\ifHy@unicode\iftrue
    \expandafter\ltx@secondoftwo
  \else
    \expandafter\ltx@firstoftwo
  \fi
}
%    \end{macrocode}
%    \end{macro}
%
%    \begin{macro}{\HOLOGO@Math}
%    \begin{macrocode}
\def\HOLOGO@MathSetup{%
  \mathsurround0pt\relax
  \HOLOGO@IfExists\f@series{%
    \if b\expandafter\ltx@car\f@series x\@nil
      \csname boldmath\endcsname
   \fi
  }{}%
}
%    \end{macrocode}
%    \end{macro}
%
%    \begin{macro}{\HOLOGO@TempDimen}
%    \begin{macrocode}
\dimendef\HOLOGO@TempDimen=\ltx@zero
%    \end{macrocode}
%    \end{macro}
%    \begin{macro}{\HOLOGO@NegativeKerning}
%    \begin{macrocode}
\def\HOLOGO@NegativeKerning#1{%
  \begingroup
    \HOLOGO@TempDimen=0pt\relax
    \comma@parse@normalized{#1}{%
      \ifdim\HOLOGO@TempDimen=0pt %
        \expandafter\HOLOGO@@NegativeKerning\comma@entry
      \fi
      \ltx@gobble
    }%
    \ifdim\HOLOGO@TempDimen<0pt %
      \kern\HOLOGO@TempDimen
    \fi
  \endgroup
}
%    \end{macrocode}
%    \end{macro}
%    \begin{macro}{\HOLOGO@@NegativeKerning}
%    \begin{macrocode}
\def\HOLOGO@@NegativeKerning#1#2{%
  \setbox\ltx@zero\hbox{#1#2}%
  \HOLOGO@TempDimen=\wd\ltx@zero
  \setbox\ltx@zero\hbox{#1\kern0pt#2}%
  \advance\HOLOGO@TempDimen by -\wd\ltx@zero
}
%    \end{macrocode}
%    \end{macro}
%
%    \begin{macro}{\HOLOGO@SpaceFactor}
%    \begin{macrocode}
\def\HOLOGO@SpaceFactor{%
  \spacefactor1000 %
}
%    \end{macrocode}
%    \end{macro}
%
%    \begin{macro}{\HOLOGO@Span}
%    \begin{macrocode}
\def\HOLOGO@Span#1#2{%
  \HCode{<span class="HoLogo-#1">}%
  #2%
  \HCode{</span>}%
}
%    \end{macrocode}
%    \end{macro}
%
% \subsubsection{Text subscript}
%
%    \begin{macro}{\HOLOGO@SubScript}%
%    \begin{macrocode}
\def\HOLOGO@SubScript#1{%
  \ltx@IfUndefined{textsubscript}{%
    \ltx@IfUndefined{text}{%
      \ltx@mbox{%
        \mathsurround=0pt\relax
        $%
          _{%
            \ltx@IfUndefined{sf@size}{%
              \mathrm{#1}%
            }{%
              \mbox{%
                \fontsize\sf@size{0pt}\selectfont
                #1%
              }%
            }%
          }%
        $%
      }%
    }{%
      \ltx@mbox{%
        \mathsurround=0pt\relax
        $_{\text{#1}}$%
      }%
    }%
  }{%
    \textsubscript{#1}%
  }%
}
%    \end{macrocode}
%    \end{macro}
%
% \subsection{\hologo{TeX} and friends}
%
% \subsubsection{\hologo{TeX}}
%
%    \begin{macro}{\HoLogo@TeX}
%    Source: \hologo{LaTeX} kernel.
%    \begin{macrocode}
\def\HoLogo@TeX#1{%
  T\kern-.1667em\lower.5ex\hbox{E}\kern-.125emX\HOLOGO@SpaceFactor
}
%    \end{macrocode}
%    \end{macro}
%    \begin{macro}{\HoLogoHtml@TeX}
%    \begin{macrocode}
\def\HoLogoHtml@TeX#1{%
  \HoLogoCss@TeX
  \HOLOGO@Span{TeX}{%
    T%
    \HOLOGO@Span{e}{%
      E%
    }%
    X%
  }%
}
%    \end{macrocode}
%    \end{macro}
%    \begin{macro}{\HoLogoCss@TeX}
%    \begin{macrocode}
\def\HoLogoCss@TeX{%
  \Css{%
    span.HoLogo-TeX span.HoLogo-e{%
      position:relative;%
      top:.5ex;%
      margin-left:-.1667em;%
      margin-right:-.125em;%
    }%
  }%
  \Css{%
    a span.HoLogo-TeX span.HoLogo-e{%
      text-decoration:none;%
    }%
  }%
  \global\let\HoLogoCss@TeX\relax
}
%    \end{macrocode}
%    \end{macro}
%
% \subsubsection{\hologo{plainTeX}}
%
%    \begin{macro}{\HoLogo@plainTeX@space}
%    Source: ``The \hologo{TeX}book''
%    \begin{macrocode}
\def\HoLogo@plainTeX@space#1{%
  \HOLOGO@mbox{#1{p}{P}lain}\HOLOGO@space\hologo{TeX}%
}
%    \end{macrocode}
%    \end{macro}
%    \begin{macro}{\HoLogoCs@plainTeX@space}
%    \begin{macrocode}
\def\HoLogoCs@plainTeX@space#1{#1{p}{P}lain TeX}%
%    \end{macrocode}
%    \end{macro}
%    \begin{macro}{\HoLogoBkm@plainTeX@space}
%    \begin{macrocode}
\def\HoLogoBkm@plainTeX@space#1{%
  #1{p}{P}lain \hologo{TeX}%
}
%    \end{macrocode}
%    \end{macro}
%    \begin{macro}{\HoLogoHtml@plainTeX@space}
%    \begin{macrocode}
\def\HoLogoHtml@plainTeX@space#1{%
  #1{p}{P}lain \hologo{TeX}%
}
%    \end{macrocode}
%    \end{macro}
%
%    \begin{macro}{\HoLogo@plainTeX@hyphen}
%    \begin{macrocode}
\def\HoLogo@plainTeX@hyphen#1{%
  \HOLOGO@mbox{#1{p}{P}lain}\HOLOGO@hyphen\hologo{TeX}%
}
%    \end{macrocode}
%    \end{macro}
%    \begin{macro}{\HoLogoCs@plainTeX@hyphen}
%    \begin{macrocode}
\def\HoLogoCs@plainTeX@hyphen#1{#1{p}{P}lain-TeX}
%    \end{macrocode}
%    \end{macro}
%    \begin{macro}{\HoLogoBkm@plainTeX@hyphen}
%    \begin{macrocode}
\def\HoLogoBkm@plainTeX@hyphen#1{%
  #1{p}{P}lain-\hologo{TeX}%
}
%    \end{macrocode}
%    \end{macro}
%    \begin{macro}{\HoLogoHtml@plainTeX@hyphen}
%    \begin{macrocode}
\def\HoLogoHtml@plainTeX@hyphen#1{%
  #1{p}{P}lain-\hologo{TeX}%
}
%    \end{macrocode}
%    \end{macro}
%
%    \begin{macro}{\HoLogo@plainTeX@runtogether}
%    \begin{macrocode}
\def\HoLogo@plainTeX@runtogether#1{%
  \HOLOGO@mbox{#1{p}{P}lain\hologo{TeX}}%
}
%    \end{macrocode}
%    \end{macro}
%    \begin{macro}{\HoLogoCs@plainTeX@runtogether}
%    \begin{macrocode}
\def\HoLogoCs@plainTeX@runtogether#1{#1{p}{P}lainTeX}
%    \end{macrocode}
%    \end{macro}
%    \begin{macro}{\HoLogoBkm@plainTeX@runtogether}
%    \begin{macrocode}
\def\HoLogoBkm@plainTeX@runtogether#1{%
  #1{p}{P}lain\hologo{TeX}%
}
%    \end{macrocode}
%    \end{macro}
%    \begin{macro}{\HoLogoHtml@plainTeX@runtogether}
%    \begin{macrocode}
\def\HoLogoHtml@plainTeX@runtogether#1{%
  #1{p}{P}lain\hologo{TeX}%
}
%    \end{macrocode}
%    \end{macro}
%
%    \begin{macro}{\HoLogo@plainTeX}
%    \begin{macrocode}
\def\HoLogo@plainTeX{\HoLogo@plainTeX@space}
%    \end{macrocode}
%    \end{macro}
%    \begin{macro}{\HoLogoCs@plainTeX}
%    \begin{macrocode}
\def\HoLogoCs@plainTeX{\HoLogoCs@plainTeX@space}
%    \end{macrocode}
%    \end{macro}
%    \begin{macro}{\HoLogoBkm@plainTeX}
%    \begin{macrocode}
\def\HoLogoBkm@plainTeX{\HoLogoBkm@plainTeX@space}
%    \end{macrocode}
%    \end{macro}
%    \begin{macro}{\HoLogoHtml@plainTeX}
%    \begin{macrocode}
\def\HoLogoHtml@plainTeX{\HoLogoHtml@plainTeX@space}
%    \end{macrocode}
%    \end{macro}
%
% \subsubsection{\hologo{LaTeX}}
%
%    Source: \hologo{LaTeX} kernel.
%\begin{quote}
%\begin{verbatim}
%\DeclareRobustCommand{\LaTeX}{%
%  L%
%  \kern-.36em%
%  {%
%    \sbox\z@ T%
%    \vbox to\ht\z@{%
%      \hbox{%
%        \check@mathfonts
%        \fontsize\sf@size\z@
%        \math@fontsfalse
%        \selectfont
%        A%
%      }%
%      \vss
%    }%
%  }%
%  \kern-.15em%
%  \TeX
%}
%\end{verbatim}
%\end{quote}
%
%    \begin{macro}{\HoLogo@La}
%    \begin{macrocode}
\def\HoLogo@La#1{%
  L%
  \kern-.36em%
  \begingroup
    \setbox\ltx@zero\hbox{T}%
    \vbox to\ht\ltx@zero{%
      \hbox{%
        \ltx@ifundefined{check@mathfonts}{%
          \csname sevenrm\endcsname
        }{%
          \check@mathfonts
          \fontsize\sf@size{0pt}%
          \math@fontsfalse\selectfont
        }%
        A%
      }%
      \vss
    }%
  \endgroup
}
%    \end{macrocode}
%    \end{macro}
%
%    \begin{macro}{\HoLogo@LaTeX}
%    Source: \hologo{LaTeX} kernel.
%    \begin{macrocode}
\def\HoLogo@LaTeX#1{%
  \hologo{La}%
  \kern-.15em%
  \hologo{TeX}%
}
%    \end{macrocode}
%    \end{macro}
%    \begin{macro}{\HoLogoHtml@LaTeX}
%    \begin{macrocode}
\def\HoLogoHtml@LaTeX#1{%
  \HoLogoCss@LaTeX
  \HOLOGO@Span{LaTeX}{%
    L%
    \HOLOGO@Span{a}{%
      A%
    }%
    \hologo{TeX}%
  }%
}
%    \end{macrocode}
%    \end{macro}
%    \begin{macro}{\HoLogoCss@LaTeX}
%    \begin{macrocode}
\def\HoLogoCss@LaTeX{%
  \Css{%
    span.HoLogo-LaTeX span.HoLogo-a{%
      position:relative;%
      top:-.5ex;%
      margin-left:-.36em;%
      margin-right:-.15em;%
      font-size:85\%;%
    }%
  }%
  \global\let\HoLogoCss@LaTeX\relax
}
%    \end{macrocode}
%    \end{macro}
%
% \subsubsection{\hologo{(La)TeX}}
%
%    \begin{macro}{\HoLogo@LaTeXTeX}
%    The kerning around the parentheses is taken
%    from package \xpackage{dtklogos} \cite{dtklogos}.
%\begin{quote}
%\begin{verbatim}
%\DeclareRobustCommand{\LaTeXTeX}{%
%  (%
%  \kern-.15em%
%  L%
%  \kern-.36em%
%  {%
%    \sbox\z@ T%
%    \vbox to\ht0{%
%      \hbox{%
%        $\m@th$%
%        \csname S@\f@size\endcsname
%        \fontsize\sf@size\z@
%        \math@fontsfalse
%        \selectfont
%        A%
%      }%
%      \vss
%    }%
%  }%
%  \kern-.2em%
%  )%
%  \kern-.15em%
%  \TeX
%}
%\end{verbatim}
%\end{quote}
%    \begin{macrocode}
\def\HoLogo@LaTeXTeX#1{%
  (%
  \kern-.15em%
  \hologo{La}%
  \kern-.2em%
  )%
  \kern-.15em%
  \hologo{TeX}%
}
%    \end{macrocode}
%    \end{macro}
%    \begin{macro}{\HoLogoBkm@LaTeXTeX}
%    \begin{macrocode}
\def\HoLogoBkm@LaTeXTeX#1{(La)TeX}
%    \end{macrocode}
%    \end{macro}
%
%    \begin{macro}{\HoLogo@(La)TeX}
%    \begin{macrocode}
\expandafter
\let\csname HoLogo@(La)TeX\endcsname\HoLogo@LaTeXTeX
%    \end{macrocode}
%    \end{macro}
%    \begin{macro}{\HoLogoBkm@(La)TeX}
%    \begin{macrocode}
\expandafter
\let\csname HoLogoBkm@(La)TeX\endcsname\HoLogoBkm@LaTeXTeX
%    \end{macrocode}
%    \end{macro}
%    \begin{macro}{\HoLogoHtml@LaTeXTeX}
%    \begin{macrocode}
\def\HoLogoHtml@LaTeXTeX#1{%
  \HoLogoCss@LaTeXTeX
  \HOLOGO@Span{LaTeXTeX}{%
    (%
    \HOLOGO@Span{L}{L}%
    \HOLOGO@Span{a}{A}%
    \HOLOGO@Span{ParenRight}{)}%
    \hologo{TeX}%
  }%
}
%    \end{macrocode}
%    \end{macro}
%    \begin{macro}{\HoLogoHtml@(La)TeX}
%    Kerning after opening parentheses and before closing parentheses
%    is $-0.1$\,em. The original values $-0.15$\,em
%    looked too ugly for a serif font.
%    \begin{macrocode}
\expandafter
\let\csname HoLogoHtml@(La)TeX\endcsname\HoLogoHtml@LaTeXTeX
%    \end{macrocode}
%    \end{macro}
%    \begin{macro}{\HoLogoCss@LaTeXTeX}
%    \begin{macrocode}
\def\HoLogoCss@LaTeXTeX{%
  \Css{%
    span.HoLogo-LaTeXTeX span.HoLogo-L{%
      margin-left:-.1em;%
    }%
  }%
  \Css{%
    span.HoLogo-LaTeXTeX span.HoLogo-a{%
      position:relative;%
      top:-.5ex;%
      margin-left:-.36em;%
      margin-right:-.1em;%
      font-size:85\%;%
    }%
  }%
  \Css{%
    span.HoLogo-LaTeXTeX span.HoLogo-ParenRight{%
      margin-right:-.15em;%
    }%
  }%
  \global\let\HoLogoCss@LaTeXTeX\relax
}
%    \end{macrocode}
%    \end{macro}
%
% \subsubsection{\hologo{LaTeXe}}
%
%    \begin{macro}{\HoLogo@LaTeXe}
%    Source: \hologo{LaTeX} kernel
%    \begin{macrocode}
\def\HoLogo@LaTeXe#1{%
  \hologo{LaTeX}%
  \kern.15em%
  \hbox{%
    \HOLOGO@MathSetup
    2%
    $_{\textstyle\varepsilon}$%
  }%
}
%    \end{macrocode}
%    \end{macro}
%
%    \begin{macro}{\HoLogoCs@LaTeXe}
%    \begin{macrocode}
\ifnum64=`\^^^^0040\relax % test for big chars of LuaTeX/XeTeX
  \catcode`\$=9 %
  \catcode`\&=14 %
\else
  \catcode`\$=14 %
  \catcode`\&=9 %
\fi
\def\HoLogoCs@LaTeXe#1{%
  LaTeX2%
$ \string ^^^^0395%
& e%
}%
\catcode`\$=3 %
\catcode`\&=4 %
%    \end{macrocode}
%    \end{macro}
%
%    \begin{macro}{\HoLogoBkm@LaTeXe}
%    \begin{macrocode}
\def\HoLogoBkm@LaTeXe#1{%
  \hologo{LaTeX}%
  2%
  \HOLOGO@PdfdocUnicode{e}{\textepsilon}%
}
%    \end{macrocode}
%    \end{macro}
%
%    \begin{macro}{\HoLogoHtml@LaTeXe}
%    \begin{macrocode}
\def\HoLogoHtml@LaTeXe#1{%
  \HoLogoCss@LaTeXe
  \HOLOGO@Span{LaTeX2e}{%
    \hologo{LaTeX}%
    \HOLOGO@Span{2}{2}%
    \HOLOGO@Span{e}{%
      \HOLOGO@MathSetup
      \ensuremath{\textstyle\varepsilon}%
    }%
  }%
}
%    \end{macrocode}
%    \end{macro}
%    \begin{macro}{\HoLogoCss@LaTeXe}
%    \begin{macrocode}
\def\HoLogoCss@LaTeXe{%
  \Css{%
    span.HoLogo-LaTeX2e span.HoLogo-2{%
      padding-left:.15em;%
    }%
  }%
  \Css{%
    span.HoLogo-LaTeX2e span.HoLogo-e{%
      position:relative;%
      top:.35ex;%
      text-decoration:none;%
    }%
  }%
  \global\let\HoLogoCss@LaTeXe\relax
}
%    \end{macrocode}
%    \end{macro}
%
%    \begin{macro}{\HoLogo@LaTeX2e}
%    \begin{macrocode}
\expandafter
\let\csname HoLogo@LaTeX2e\endcsname\HoLogo@LaTeXe
%    \end{macrocode}
%    \end{macro}
%    \begin{macro}{\HoLogoCs@LaTeX2e}
%    \begin{macrocode}
\expandafter
\let\csname HoLogoCs@LaTeX2e\endcsname\HoLogoCs@LaTeXe
%    \end{macrocode}
%    \end{macro}
%    \begin{macro}{\HoLogoBkm@LaTeX2e}
%    \begin{macrocode}
\expandafter
\let\csname HoLogoBkm@LaTeX2e\endcsname\HoLogoBkm@LaTeXe
%    \end{macrocode}
%    \end{macro}
%    \begin{macro}{\HoLogoHtml@LaTeX2e}
%    \begin{macrocode}
\expandafter
\let\csname HoLogoHtml@LaTeX2e\endcsname\HoLogoHtml@LaTeXe
%    \end{macrocode}
%    \end{macro}
%
% \subsubsection{\hologo{LaTeX3}}
%
%    \begin{macro}{\HoLogo@LaTeX3}
%    Source: \hologo{LaTeX} kernel
%    \begin{macrocode}
\expandafter\def\csname HoLogo@LaTeX3\endcsname#1{%
  \hologo{LaTeX}%
  3%
}
%    \end{macrocode}
%    \end{macro}
%
%    \begin{macro}{\HoLogoBkm@LaTeX3}
%    \begin{macrocode}
\expandafter\def\csname HoLogoBkm@LaTeX3\endcsname#1{%
  \hologo{LaTeX}%
  3%
}
%    \end{macrocode}
%    \end{macro}
%    \begin{macro}{\HoLogoHtml@LaTeX3}
%    \begin{macrocode}
\expandafter
\let\csname HoLogoHtml@LaTeX3\expandafter\endcsname
\csname HoLogo@LaTeX3\endcsname
%    \end{macrocode}
%    \end{macro}
%
% \subsubsection{\hologo{LaTeXML}}
%
%    \begin{macro}{\HoLogo@LaTeXML}
%    \begin{macrocode}
\def\HoLogo@LaTeXML#1{%
  \HOLOGO@mbox{%
    \hologo{La}%
    \kern-.15em%
    T%
    \kern-.1667em%
    \lower.5ex\hbox{E}%
    \kern-.125em%
    \HoLogoFont@font{LaTeXML}{sc}{xml}%
  }%
}
%    \end{macrocode}
%    \end{macro}
%    \begin{macro}{\HoLogoHtml@pdfLaTeX}
%    \begin{macrocode}
\def\HoLogoHtml@LaTeXML#1{%
  \HOLOGO@Span{LaTeXML}{%
    \HoLogoCss@LaTeX
    \HoLogoCss@TeX
    \HOLOGO@Span{LaTeX}{%
      L%
      \HOLOGO@Span{a}{%
        A%
      }%
    }%
    \HOLOGO@Span{TeX}{%
      T%
      \HOLOGO@Span{e}{%
        E%
      }%
    }%
    \HCode{<span style="font-variant: small-caps;">}%
    xml%
    \HCode{</span>}%
  }%
}
%    \end{macrocode}
%    \end{macro}
%
% \subsubsection{\hologo{eTeX}}
%
%    \begin{macro}{\HoLogo@eTeX}
%    Source: package \xpackage{etex}
%    \begin{macrocode}
\def\HoLogo@eTeX#1{%
  \ltx@mbox{%
    \HOLOGO@MathSetup
    $\varepsilon$%
    -%
    \HOLOGO@NegativeKerning{-T,T-,To}%
    \hologo{TeX}%
  }%
}
%    \end{macrocode}
%    \end{macro}
%    \begin{macro}{\HoLogoCs@eTeX}
%    \begin{macrocode}
\ifnum64=`\^^^^0040\relax % test for big chars of LuaTeX/XeTeX
  \catcode`\$=9 %
  \catcode`\&=14 %
\else
  \catcode`\$=14 %
  \catcode`\&=9 %
\fi
\def\HoLogoCs@eTeX#1{%
$ #1{\string ^^^^0395}{\string ^^^^03b5}%
& #1{e}{E}%
  TeX%
}%
\catcode`\$=3 %
\catcode`\&=4 %
%    \end{macrocode}
%    \end{macro}
%    \begin{macro}{\HoLogoBkm@eTeX}
%    \begin{macrocode}
\def\HoLogoBkm@eTeX#1{%
  \HOLOGO@PdfdocUnicode{#1{e}{E}}{\textepsilon}%
  -%
  \hologo{TeX}%
}
%    \end{macrocode}
%    \end{macro}
%    \begin{macro}{\HoLogoHtml@eTeX}
%    \begin{macrocode}
\def\HoLogoHtml@eTeX#1{%
  \ltx@mbox{%
    \HOLOGO@MathSetup
    $\varepsilon$%
    -%
    \hologo{TeX}%
  }%
}
%    \end{macrocode}
%    \end{macro}
%
% \subsubsection{\hologo{iniTeX}}
%
%    \begin{macro}{\HoLogo@iniTeX}
%    \begin{macrocode}
\def\HoLogo@iniTeX#1{%
  \HOLOGO@mbox{%
    #1{i}{I}ni\hologo{TeX}%
  }%
}
%    \end{macrocode}
%    \end{macro}
%    \begin{macro}{\HoLogoCs@iniTeX}
%    \begin{macrocode}
\def\HoLogoCs@iniTeX#1{#1{i}{I}niTeX}
%    \end{macrocode}
%    \end{macro}
%    \begin{macro}{\HoLogoBkm@iniTeX}
%    \begin{macrocode}
\def\HoLogoBkm@iniTeX#1{%
  #1{i}{I}ni\hologo{TeX}%
}
%    \end{macrocode}
%    \end{macro}
%    \begin{macro}{\HoLogoHtml@iniTeX}
%    \begin{macrocode}
\let\HoLogoHtml@iniTeX\HoLogo@iniTeX
%    \end{macrocode}
%    \end{macro}
%
% \subsubsection{\hologo{virTeX}}
%
%    \begin{macro}{\HoLogo@virTeX}
%    \begin{macrocode}
\def\HoLogo@virTeX#1{%
  \HOLOGO@mbox{%
    #1{v}{V}ir\hologo{TeX}%
  }%
}
%    \end{macrocode}
%    \end{macro}
%    \begin{macro}{\HoLogoCs@virTeX}
%    \begin{macrocode}
\def\HoLogoCs@virTeX#1{#1{v}{V}irTeX}
%    \end{macrocode}
%    \end{macro}
%    \begin{macro}{\HoLogoBkm@virTeX}
%    \begin{macrocode}
\def\HoLogoBkm@virTeX#1{%
  #1{v}{V}ir\hologo{TeX}%
}
%    \end{macrocode}
%    \end{macro}
%    \begin{macro}{\HoLogoHtml@virTeX}
%    \begin{macrocode}
\let\HoLogoHtml@virTeX\HoLogo@virTeX
%    \end{macrocode}
%    \end{macro}
%
% \subsubsection{\hologo{SliTeX}}
%
% \paragraph{Definitions of the three variants.}
%
%    \begin{macro}{\HoLogo@SLiTeX@lift}
%    \begin{macrocode}
\def\HoLogo@SLiTeX@lift#1{%
  \HoLogoFont@font{SliTeX}{rm}{%
    S%
    \kern-.06em%
    L%
    \kern-.18em%
    \raise.32ex\hbox{\HoLogoFont@font{SliTeX}{sc}{i}}%
    \HOLOGO@discretionary
    \kern-.06em%
    \hologo{TeX}%
  }%
}
%    \end{macrocode}
%    \end{macro}
%    \begin{macro}{\HoLogoBkm@SLiTeX@lift}
%    \begin{macrocode}
\def\HoLogoBkm@SLiTeX@lift#1{SLiTeX}
%    \end{macrocode}
%    \end{macro}
%    \begin{macro}{\HoLogoHtml@SLiTeX@lift}
%    \begin{macrocode}
\def\HoLogoHtml@SLiTeX@lift#1{%
  \HoLogoCss@SLiTeX@lift
  \HOLOGO@Span{SLiTeX-lift}{%
    \HoLogoFont@font{SliTeX}{rm}{%
      S%
      \HOLOGO@Span{L}{L}%
      \HOLOGO@Span{i}{i}%
      \hologo{TeX}%
    }%
  }%
}
%    \end{macrocode}
%    \end{macro}
%    \begin{macro}{\HoLogoCss@SLiTeX@lift}
%    \begin{macrocode}
\def\HoLogoCss@SLiTeX@lift{%
  \Css{%
    span.HoLogo-SLiTeX-lift span.HoLogo-L{%
      margin-left:-.06em;%
      margin-right:-.18em;%
    }%
  }%
  \Css{%
    span.HoLogo-SLiTeX-lift span.HoLogo-i{%
      position:relative;%
      top:-.32ex;%
      margin-right:-.06em;%
      font-variant:small-caps;%
    }%
  }%
  \global\let\HoLogoCss@SLiTeX@lift\relax
}
%    \end{macrocode}
%    \end{macro}
%
%    \begin{macro}{\HoLogo@SliTeX@simple}
%    \begin{macrocode}
\def\HoLogo@SliTeX@simple#1{%
  \HoLogoFont@font{SliTeX}{rm}{%
    \ltx@mbox{%
      \HoLogoFont@font{SliTeX}{sc}{Sli}%
    }%
    \HOLOGO@discretionary
    \hologo{TeX}%
  }%
}
%    \end{macrocode}
%    \end{macro}
%    \begin{macro}{\HoLogoBkm@SliTeX@simple}
%    \begin{macrocode}
\def\HoLogoBkm@SliTeX@simple#1{SliTeX}
%    \end{macrocode}
%    \end{macro}
%    \begin{macro}{\HoLogoHtml@SliTeX@simple}
%    \begin{macrocode}
\let\HoLogoHtml@SliTeX@simple\HoLogo@SliTeX@simple
%    \end{macrocode}
%    \end{macro}
%
%    \begin{macro}{\HoLogo@SliTeX@narrow}
%    \begin{macrocode}
\def\HoLogo@SliTeX@narrow#1{%
  \HoLogoFont@font{SliTeX}{rm}{%
    \ltx@mbox{%
      S%
      \kern-.06em%
      \HoLogoFont@font{SliTeX}{sc}{%
        l%
        \kern-.035em%
        i%
      }%
    }%
    \HOLOGO@discretionary
    \kern-.06em%
    \hologo{TeX}%
  }%
}
%    \end{macrocode}
%    \end{macro}
%    \begin{macro}{\HoLogoBkm@SliTeX@narrow}
%    \begin{macrocode}
\def\HoLogoBkm@SliTeX@narrow#1{SliTeX}
%    \end{macrocode}
%    \end{macro}
%    \begin{macro}{\HoLogoHtml@SliTeX@narrow}
%    \begin{macrocode}
\def\HoLogoHtml@SliTeX@narrow#1{%
  \HoLogoCss@SliTeX@narrow
  \HOLOGO@Span{SliTeX-narrow}{%
    \HoLogoFont@font{SliTeX}{rm}{%
      S%
        \HOLOGO@Span{l}{l}%
        \HOLOGO@Span{i}{i}%
      \hologo{TeX}%
    }%
  }%
}
%    \end{macrocode}
%    \end{macro}
%    \begin{macro}{\HoLogoCss@SliTeX@narrow}
%    \begin{macrocode}
\def\HoLogoCss@SliTeX@narrow{%
  \Css{%
    span.HoLogo-SliTeX-narrow span.HoLogo-l{%
      margin-left:-.06em;%
      margin-right:-.035em;%
      font-variant:small-caps;%
    }%
  }%
  \Css{%
    span.HoLogo-SliTeX-narrow span.HoLogo-i{%
      margin-right:-.06em;%
      font-variant:small-caps;%
    }%
  }%
  \global\let\HoLogoCss@SliTeX@narrow\relax
}
%    \end{macrocode}
%    \end{macro}
%
% \paragraph{Macro set completion.}
%
%    \begin{macro}{\HoLogo@SLiTeX@simple}
%    \begin{macrocode}
\def\HoLogo@SLiTeX@simple{\HoLogo@SliTeX@simple}
%    \end{macrocode}
%    \end{macro}
%    \begin{macro}{\HoLogoBkm@SLiTeX@simple}
%    \begin{macrocode}
\def\HoLogoBkm@SLiTeX@simple{\HoLogoBkm@SliTeX@simple}
%    \end{macrocode}
%    \end{macro}
%    \begin{macro}{\HoLogoHtml@SLiTeX@simple}
%    \begin{macrocode}
\def\HoLogoHtml@SLiTeX@simple{\HoLogoHtml@SliTeX@simple}
%    \end{macrocode}
%    \end{macro}
%
%    \begin{macro}{\HoLogo@SLiTeX@narrow}
%    \begin{macrocode}
\def\HoLogo@SLiTeX@narrow{\HoLogo@SliTeX@narrow}
%    \end{macrocode}
%    \end{macro}
%    \begin{macro}{\HoLogoBkm@SLiTeX@narrow}
%    \begin{macrocode}
\def\HoLogoBkm@SLiTeX@narrow{\HoLogoBkm@SliTeX@narrow}
%    \end{macrocode}
%    \end{macro}
%    \begin{macro}{\HoLogoHtml@SLiTeX@narrow}
%    \begin{macrocode}
\def\HoLogoHtml@SLiTeX@narrow{\HoLogoHtml@SliTeX@narrow}
%    \end{macrocode}
%    \end{macro}
%
%    \begin{macro}{\HoLogo@SliTeX@lift}
%    \begin{macrocode}
\def\HoLogo@SliTeX@lift{\HoLogo@SLiTeX@lift}
%    \end{macrocode}
%    \end{macro}
%    \begin{macro}{\HoLogoBkm@SliTeX@lift}
%    \begin{macrocode}
\def\HoLogoBkm@SliTeX@lift{\HoLogoBkm@SLiTeX@lift}
%    \end{macrocode}
%    \end{macro}
%    \begin{macro}{\HoLogoHtml@SliTeX@lift}
%    \begin{macrocode}
\def\HoLogoHtml@SliTeX@lift{\HoLogoHtml@SLiTeX@lift}
%    \end{macrocode}
%    \end{macro}
%
% \paragraph{Defaults.}
%
%    \begin{macro}{\HoLogo@SLiTeX}
%    \begin{macrocode}
\def\HoLogo@SLiTeX{\HoLogo@SLiTeX@lift}
%    \end{macrocode}
%    \end{macro}
%    \begin{macro}{\HoLogoBkm@SLiTeX}
%    \begin{macrocode}
\def\HoLogoBkm@SLiTeX{\HoLogoBkm@SLiTeX@lift}
%    \end{macrocode}
%    \end{macro}
%    \begin{macro}{\HoLogoHtml@SLiTeX}
%    \begin{macrocode}
\def\HoLogoHtml@SLiTeX{\HoLogoHtml@SLiTeX@lift}
%    \end{macrocode}
%    \end{macro}
%
%    \begin{macro}{\HoLogo@SliTeX}
%    \begin{macrocode}
\def\HoLogo@SliTeX{\HoLogo@SliTeX@narrow}
%    \end{macrocode}
%    \end{macro}
%    \begin{macro}{\HoLogoBkm@SliTeX}
%    \begin{macrocode}
\def\HoLogoBkm@SliTeX{\HoLogoBkm@SliTeX@narrow}
%    \end{macrocode}
%    \end{macro}
%    \begin{macro}{\HoLogoHtml@SliTeX}
%    \begin{macrocode}
\def\HoLogoHtml@SliTeX{\HoLogoHtml@SliTeX@narrow}
%    \end{macrocode}
%    \end{macro}
%
% \subsubsection{\hologo{LuaTeX}}
%
%    \begin{macro}{\HoLogo@LuaTeX}
%    The kerning is an idea of Hans Hagen, see mailing list
%    `luatex at tug dot org' in March 2010.
%    \begin{macrocode}
\def\HoLogo@LuaTeX#1{%
  \HOLOGO@mbox{%
    Lua%
    \HOLOGO@NegativeKerning{aT,oT,To}%
    \hologo{TeX}%
  }%
}
%    \end{macrocode}
%    \end{macro}
%    \begin{macro}{\HoLogoHtml@LuaTeX}
%    \begin{macrocode}
\let\HoLogoHtml@LuaTeX\HoLogo@LuaTeX
%    \end{macrocode}
%    \end{macro}
%
% \subsubsection{\hologo{LuaLaTeX}}
%
%    \begin{macro}{\HoLogo@LuaLaTeX}
%    \begin{macrocode}
\def\HoLogo@LuaLaTeX#1{%
  \HOLOGO@mbox{%
    Lua%
    \hologo{LaTeX}%
  }%
}
%    \end{macrocode}
%    \end{macro}
%    \begin{macro}{\HoLogoHtml@LuaLaTeX}
%    \begin{macrocode}
\let\HoLogoHtml@LuaLaTeX\HoLogo@LuaLaTeX
%    \end{macrocode}
%    \end{macro}
%
% \subsubsection{\hologo{XeTeX}, \hologo{XeLaTeX}}
%
%    \begin{macro}{\HOLOGO@IfCharExists}
%    \begin{macrocode}
\ifluatex
  \ifnum\luatexversion<36 %
  \else
    \def\HOLOGO@IfCharExists#1{%
      \ifnum
        \directlua{%
           if luaotfload and luaotfload.aux then
             if luaotfload.aux.font_has_glyph(%
                    font.current(), \number#1) then % 	 
	       tex.print("1") % 	 
	     end % 	 
	   elseif font and font.fonts and font.current then %
            local f = font.fonts[font.current()]%
            if f.characters and f.characters[\number#1] then %
              tex.print("1")%
            end %
          end%
        }0=\ltx@zero
        \expandafter\ltx@secondoftwo
      \else
        \expandafter\ltx@firstoftwo
      \fi
    }%
  \fi
\fi
\ltx@IfUndefined{HOLOGO@IfCharExists}{%
  \def\HOLOGO@@IfCharExists#1{%
    \begingroup
      \tracinglostchars=\ltx@zero
      \setbox\ltx@zero=\hbox{%
        \kern7sp\char#1\relax
        \ifnum\lastkern>\ltx@zero
          \expandafter\aftergroup\csname iffalse\endcsname
        \else
          \expandafter\aftergroup\csname iftrue\endcsname
        \fi
      }%
      % \if{true|false} from \aftergroup
      \endgroup
      \expandafter\ltx@firstoftwo
    \else
      \endgroup
      \expandafter\ltx@secondoftwo
    \fi
  }%
  \ifxetex
    \ltx@IfUndefined{XeTeXfonttype}{}{%
      \ltx@IfUndefined{XeTeXcharglyph}{}{%
        \def\HOLOGO@IfCharExists#1{%
          \ifnum\XeTeXfonttype\font>\ltx@zero
            \expandafter\ltx@firstofthree
          \else
            \expandafter\ltx@gobble
          \fi
          {%
            \ifnum\XeTeXcharglyph#1>\ltx@zero
              \expandafter\ltx@firstoftwo
            \else
              \expandafter\ltx@secondoftwo
            \fi
          }%
          \HOLOGO@@IfCharExists{#1}%
        }%
      }%
    }%
  \fi
}{}
\ltx@ifundefined{HOLOGO@IfCharExists}{%
  \ifnum64=`\^^^^0040\relax % test for big chars of LuaTeX/XeTeX
    \let\HOLOGO@IfCharExists\HOLOGO@@IfCharExists
  \else
    \def\HOLOGO@IfCharExists#1{%
      \ifnum#1>255 %
        \expandafter\ltx@fourthoffour
      \fi
      \HOLOGO@@IfCharExists{#1}%
    }%
  \fi
}{}
%    \end{macrocode}
%    \end{macro}
%
%    \begin{macro}{\HoLogo@Xe}
%    Source: package \xpackage{dtklogos}
%    \begin{macrocode}
\def\HoLogo@Xe#1{%
  X%
  \kern-.1em\relax
  \HOLOGO@IfCharExists{"018E}{%
    \lower.5ex\hbox{\char"018E}%
  }{%
    \chardef\HOLOGO@choice=\ltx@zero
    \ifdim\fontdimen\ltx@one\font>0pt %
      \ltx@IfUndefined{rotatebox}{%
        \ltx@IfUndefined{pgftext}{%
          \ltx@IfUndefined{psscalebox}{%
            \ltx@IfUndefined{HOLOGO@ScaleBox@\hologoDriver}{%
            }{%
              \chardef\HOLOGO@choice=4 %
            }%
          }{%
            \chardef\HOLOGO@choice=3 %
          }%
        }{%
          \chardef\HOLOGO@choice=2 %
        }%
      }{%
        \chardef\HOLOGO@choice=1 %
      }%
      \ifcase\HOLOGO@choice
        \HOLOGO@WarningUnsupportedDriver{Xe}%
        e%
      \or % 1: \rotatebox
        \begingroup
          \setbox\ltx@zero\hbox{\rotatebox{180}{E}}%
          \ltx@LocDimenA=\dp\ltx@zero
          \advance\ltx@LocDimenA by -.5ex\relax
          \raise\ltx@LocDimenA\box\ltx@zero
        \endgroup
      \or % 2: \pgftext
        \lower.5ex\hbox{%
          \pgfpicture
            \pgftext[rotate=180]{E}%
          \endpgfpicture
        }%
      \or % 3: \psscalebox
        \begingroup
          \setbox\ltx@zero\hbox{\psscalebox{-1 -1}{E}}%
          \ltx@LocDimenA=\dp\ltx@zero
          \advance\ltx@LocDimenA by -.5ex\relax
          \raise\ltx@LocDimenA\box\ltx@zero
        \endgroup
      \or % 4: \HOLOGO@PointReflectBox
        \lower.5ex\hbox{\HOLOGO@PointReflectBox{E}}%
      \else
        \@PackageError{hologo}{Internal error (choice/it}\@ehc
      \fi
    \else
      \ltx@IfUndefined{reflectbox}{%
        \ltx@IfUndefined{pgftext}{%
          \ltx@IfUndefined{psscalebox}{%
            \ltx@IfUndefined{HOLOGO@ScaleBox@\hologoDriver}{%
            }{%
              \chardef\HOLOGO@choice=4 %
            }%
          }{%
            \chardef\HOLOGO@choice=3 %
          }%
        }{%
          \chardef\HOLOGO@choice=2 %
        }%
      }{%
        \chardef\HOLOGO@choice=1 %
      }%
      \ifcase\HOLOGO@choice
        \HOLOGO@WarningUnsupportedDriver{Xe}%
        e%
      \or % 1: reflectbox
        \lower.5ex\hbox{%
          \reflectbox{E}%
        }%
      \or % 2: \pgftext
        \lower.5ex\hbox{%
          \pgfpicture
            \pgftransformxscale{-1}%
            \pgftext{E}%
          \endpgfpicture
        }%
      \or % 3: \psscalebox
        \lower.5ex\hbox{%
          \psscalebox{-1 1}{E}%
        }%
      \or % 4: \HOLOGO@Reflectbox
        \lower.5ex\hbox{%
          \HOLOGO@ReflectBox{E}%
        }%
      \else
        \@PackageError{hologo}{Internal error (choice/up)}\@ehc
      \fi
    \fi
  }%
}
%    \end{macrocode}
%    \end{macro}
%    \begin{macro}{\HoLogoHtml@Xe}
%    \begin{macrocode}
\def\HoLogoHtml@Xe#1{%
  \HoLogoCss@Xe
  \HOLOGO@Span{Xe}{%
    X%
    \HOLOGO@Span{e}{%
      \HCode{&\ltx@hashchar x018e;}%
    }%
  }%
}
%    \end{macrocode}
%    \end{macro}
%    \begin{macro}{\HoLogoCss@Xe}
%    \begin{macrocode}
\def\HoLogoCss@Xe{%
  \Css{%
    span.HoLogo-Xe span.HoLogo-e{%
      position:relative;%
      top:.5ex;%
      left-margin:-.1em;%
    }%
  }%
  \global\let\HoLogoCss@Xe\relax
}
%    \end{macrocode}
%    \end{macro}
%
%    \begin{macro}{\HoLogo@XeTeX}
%    \begin{macrocode}
\def\HoLogo@XeTeX#1{%
  \hologo{Xe}%
  \kern-.15em\relax
  \hologo{TeX}%
}
%    \end{macrocode}
%    \end{macro}
%
%    \begin{macro}{\HoLogoHtml@XeTeX}
%    \begin{macrocode}
\def\HoLogoHtml@XeTeX#1{%
  \HoLogoCss@XeTeX
  \HOLOGO@Span{XeTeX}{%
    \hologo{Xe}%
    \hologo{TeX}%
  }%
}
%    \end{macrocode}
%    \end{macro}
%    \begin{macro}{\HoLogoCss@XeTeX}
%    \begin{macrocode}
\def\HoLogoCss@XeTeX{%
  \Css{%
    span.HoLogo-XeTeX span.HoLogo-TeX{%
      margin-left:-.15em;%
    }%
  }%
  \global\let\HoLogoCss@XeTeX\relax
}
%    \end{macrocode}
%    \end{macro}
%
%    \begin{macro}{\HoLogo@XeLaTeX}
%    \begin{macrocode}
\def\HoLogo@XeLaTeX#1{%
  \hologo{Xe}%
  \kern-.13em%
  \hologo{LaTeX}%
}
%    \end{macrocode}
%    \end{macro}
%    \begin{macro}{\HoLogoHtml@XeLaTeX}
%    \begin{macrocode}
\def\HoLogoHtml@XeLaTeX#1{%
  \HoLogoCss@XeLaTeX
  \HOLOGO@Span{XeLaTeX}{%
    \hologo{Xe}%
    \hologo{LaTeX}%
  }%
}
%    \end{macrocode}
%    \end{macro}
%    \begin{macro}{\HoLogoCss@XeLaTeX}
%    \begin{macrocode}
\def\HoLogoCss@XeLaTeX{%
  \Css{%
    span.HoLogo-XeLaTeX span.HoLogo-Xe{%
      margin-right:-.13em;%
    }%
  }%
  \global\let\HoLogoCss@XeLaTeX\relax
}
%    \end{macrocode}
%    \end{macro}
%
% \subsubsection{\hologo{pdfTeX}, \hologo{pdfLaTeX}}
%
%    \begin{macro}{\HoLogo@pdfTeX}
%    \begin{macrocode}
\def\HoLogo@pdfTeX#1{%
  \HOLOGO@mbox{%
    #1{p}{P}df\hologo{TeX}%
  }%
}
%    \end{macrocode}
%    \end{macro}
%    \begin{macro}{\HoLogoCs@pdfTeX}
%    \begin{macrocode}
\def\HoLogoCs@pdfTeX#1{#1{p}{P}dfTeX}
%    \end{macrocode}
%    \end{macro}
%    \begin{macro}{\HoLogoBkm@pdfTeX}
%    \begin{macrocode}
\def\HoLogoBkm@pdfTeX#1{%
  #1{p}{P}df\hologo{TeX}%
}
%    \end{macrocode}
%    \end{macro}
%    \begin{macro}{\HoLogoHtml@pdfTeX}
%    \begin{macrocode}
\let\HoLogoHtml@pdfTeX\HoLogo@pdfTeX
%    \end{macrocode}
%    \end{macro}
%
%    \begin{macro}{\HoLogo@pdfLaTeX}
%    \begin{macrocode}
\def\HoLogo@pdfLaTeX#1{%
  \HOLOGO@mbox{%
    #1{p}{P}df\hologo{LaTeX}%
  }%
}
%    \end{macrocode}
%    \end{macro}
%    \begin{macro}{\HoLogoCs@pdfLaTeX}
%    \begin{macrocode}
\def\HoLogoCs@pdfLaTeX#1{#1{p}{P}dfLaTeX}
%    \end{macrocode}
%    \end{macro}
%    \begin{macro}{\HoLogoBkm@pdfLaTeX}
%    \begin{macrocode}
\def\HoLogoBkm@pdfLaTeX#1{%
  #1{p}{P}df\hologo{LaTeX}%
}
%    \end{macrocode}
%    \end{macro}
%    \begin{macro}{\HoLogoHtml@pdfLaTeX}
%    \begin{macrocode}
\let\HoLogoHtml@pdfLaTeX\HoLogo@pdfLaTeX
%    \end{macrocode}
%    \end{macro}
%
% \subsubsection{\hologo{VTeX}}
%
%    \begin{macro}{\HoLogo@VTeX}
%    \begin{macrocode}
\def\HoLogo@VTeX#1{%
  \HOLOGO@mbox{%
    V\hologo{TeX}%
  }%
}
%    \end{macrocode}
%    \end{macro}
%    \begin{macro}{\HoLogoHtml@VTeX}
%    \begin{macrocode}
\let\HoLogoHtml@VTeX\HoLogo@VTeX
%    \end{macrocode}
%    \end{macro}
%
% \subsubsection{\hologo{AmS}, \dots}
%
%    Source: class \xclass{amsdtx}
%
%    \begin{macro}{\HoLogo@AmS}
%    \begin{macrocode}
\def\HoLogo@AmS#1{%
  \HoLogoFont@font{AmS}{sy}{%
    A%
    \kern-.1667em%
    \lower.5ex\hbox{M}%
    \kern-.125em%
    S%
  }%
}
%    \end{macrocode}
%    \end{macro}
%    \begin{macro}{\HoLogoBkm@AmS}
%    \begin{macrocode}
\def\HoLogoBkm@AmS#1{AmS}
%    \end{macrocode}
%    \end{macro}
%    \begin{macro}{\HoLogoHtml@AmS}
%    \begin{macrocode}
\def\HoLogoHtml@AmS#1{%
  \HoLogoCss@AmS
%  \HoLogoFont@font{AmS}{sy}{%
    \HOLOGO@Span{AmS}{%
      A%
      \HOLOGO@Span{M}{M}%
      S%
    }%
%   }%
}
%    \end{macrocode}
%    \end{macro}
%    \begin{macro}{\HoLogoCss@AmS}
%    \begin{macrocode}
\def\HoLogoCss@AmS{%
  \Css{%
    span.HoLogo-AmS span.HoLogo-M{%
      position:relative;%
      top:.5ex;%
      margin-left:-.1667em;%
      margin-right:-.125em;%
      text-decoration:none;%
    }%
  }%
  \global\let\HoLogoCss@AmS\relax
}
%    \end{macrocode}
%    \end{macro}
%
%    \begin{macro}{\HoLogo@AmSTeX}
%    \begin{macrocode}
\def\HoLogo@AmSTeX#1{%
  \hologo{AmS}%
  \HOLOGO@hyphen
  \hologo{TeX}%
}
%    \end{macrocode}
%    \end{macro}
%    \begin{macro}{\HoLogoBkm@AmSTeX}
%    \begin{macrocode}
\def\HoLogoBkm@AmSTeX#1{AmS-TeX}%
%    \end{macrocode}
%    \end{macro}
%    \begin{macro}{\HoLogoHtml@AmSTeX}
%    \begin{macrocode}
\let\HoLogoHtml@AmSTeX\HoLogo@AmSTeX
%    \end{macrocode}
%    \end{macro}
%
%    \begin{macro}{\HoLogo@AmSLaTeX}
%    \begin{macrocode}
\def\HoLogo@AmSLaTeX#1{%
  \hologo{AmS}%
  \HOLOGO@hyphen
  \hologo{LaTeX}%
}
%    \end{macrocode}
%    \end{macro}
%    \begin{macro}{\HoLogoBkm@AmSLaTeX}
%    \begin{macrocode}
\def\HoLogoBkm@AmSLaTeX#1{AmS-LaTeX}%
%    \end{macrocode}
%    \end{macro}
%    \begin{macro}{\HoLogoHtml@AmSLaTeX}
%    \begin{macrocode}
\let\HoLogoHtml@AmSLaTeX\HoLogo@AmSLaTeX
%    \end{macrocode}
%    \end{macro}
%
% \subsubsection{\hologo{BibTeX}}
%
%    \begin{macro}{\HoLogo@BibTeX@sc}
%    A definition of \hologo{BibTeX} is provided in
%    the documentation source for the manual of \hologo{BibTeX}
%    \cite{btxdoc}.
%\begin{quote}
%\begin{verbatim}
%\def\BibTeX{%
%  {%
%    \rm
%    B%
%    \kern-.05em%
%    {%
%      \sc
%      i%
%      \kern-.025em %
%      b%
%    }%
%    \kern-.08em
%    T%
%    \kern-.1667em%
%    \lower.7ex\hbox{E}%
%    \kern-.125em%
%    X%
%  }%
%}
%\end{verbatim}
%\end{quote}
%    \begin{macrocode}
\def\HoLogo@BibTeX@sc#1{%
  B%
  \kern-.05em%
  \HoLogoFont@font{BibTeX}{sc}{%
    i%
    \kern-.025em%
    b%
  }%
  \HOLOGO@discretionary
  \kern-.08em%
  \hologo{TeX}%
}
%    \end{macrocode}
%    \end{macro}
%    \begin{macro}{\HoLogoHtml@BibTeX@sc}
%    \begin{macrocode}
\def\HoLogoHtml@BibTeX@sc#1{%
  \HoLogoCss@BibTeX@sc
  \HOLOGO@Span{BibTeX-sc}{%
    B%
    \HOLOGO@Span{i}{i}%
    \HOLOGO@Span{b}{b}%
    \hologo{TeX}%
  }%
}
%    \end{macrocode}
%    \end{macro}
%    \begin{macro}{\HoLogoCss@BibTeX@sc}
%    \begin{macrocode}
\def\HoLogoCss@BibTeX@sc{%
  \Css{%
    span.HoLogo-BibTeX-sc span.HoLogo-i{%
      margin-left:-.05em;%
      margin-right:-.025em;%
      font-variant:small-caps;%
    }%
  }%
  \Css{%
    span.HoLogo-BibTeX-sc span.HoLogo-b{%
      margin-right:-.08em;%
      font-variant:small-caps;%
    }%
  }%
  \global\let\HoLogoCss@BibTeX@sc\relax
}
%    \end{macrocode}
%    \end{macro}
%
%    \begin{macro}{\HoLogo@BibTeX@sf}
%    Variant \xoption{sf} avoids trouble with unavailable
%    small caps fonts (e.g., bold versions of Computer Modern or
%    Latin Modern). The definition is taken from
%    package \xpackage{dtklogos} \cite{dtklogos}.
%\begin{quote}
%\begin{verbatim}
%\DeclareRobustCommand{\BibTeX}{%
%  B%
%  \kern-.05em%
%  \hbox{%
%    $\m@th$% %% force math size calculations
%    \csname S@\f@size\endcsname
%    \fontsize\sf@size\z@
%    \math@fontsfalse
%    \selectfont
%    I%
%    \kern-.025em%
%    B
%  }%
%  \kern-.08em%
%  \-%
%  \TeX
%}
%\end{verbatim}
%\end{quote}
%    \begin{macrocode}
\def\HoLogo@BibTeX@sf#1{%
  B%
  \kern-.05em%
  \HoLogoFont@font{BibTeX}{bibsf}{%
    I%
    \kern-.025em%
    B%
  }%
  \HOLOGO@discretionary
  \kern-.08em%
  \hologo{TeX}%
}
%    \end{macrocode}
%    \end{macro}
%    \begin{macro}{\HoLogoHtml@BibTeX@sf}
%    \begin{macrocode}
\def\HoLogoHtml@BibTeX@sf#1{%
  \HoLogoCss@BibTeX@sf
  \HOLOGO@Span{BibTeX-sf}{%
    B%
    \HoLogoFont@font{BibTeX}{bibsf}{%
      \HOLOGO@Span{i}{I}%
      B%
    }%
    \hologo{TeX}%
  }%
}
%    \end{macrocode}
%    \end{macro}
%    \begin{macro}{\HoLogoCss@BibTeX@sf}
%    \begin{macrocode}
\def\HoLogoCss@BibTeX@sf{%
  \Css{%
    span.HoLogo-BibTeX-sf span.HoLogo-i{%
      margin-left:-.05em;%
      margin-right:-.025em;%
    }%
  }%
  \Css{%
    span.HoLogo-BibTeX-sf span.HoLogo-TeX{%
      margin-left:-.08em;%
    }%
  }%
  \global\let\HoLogoCss@BibTeX@sf\relax
}
%    \end{macrocode}
%    \end{macro}
%
%    \begin{macro}{\HoLogo@BibTeX}
%    \begin{macrocode}
\def\HoLogo@BibTeX{\HoLogo@BibTeX@sf}
%    \end{macrocode}
%    \end{macro}
%    \begin{macro}{\HoLogoHtml@BibTeX}
%    \begin{macrocode}
\def\HoLogoHtml@BibTeX{\HoLogoHtml@BibTeX@sf}
%    \end{macrocode}
%    \end{macro}
%
% \subsubsection{\hologo{BibTeX8}}
%
%    \begin{macro}{\HoLogo@BibTeX8}
%    \begin{macrocode}
\expandafter\def\csname HoLogo@BibTeX8\endcsname#1{%
  \hologo{BibTeX}%
  8%
}
%    \end{macrocode}
%    \end{macro}
%
%    \begin{macro}{\HoLogoBkm@BibTeX8}
%    \begin{macrocode}
\expandafter\def\csname HoLogoBkm@BibTeX8\endcsname#1{%
  \hologo{BibTeX}%
  8%
}
%    \end{macrocode}
%    \end{macro}
%    \begin{macro}{\HoLogoHtml@BibTeX8}
%    \begin{macrocode}
\expandafter
\let\csname HoLogoHtml@BibTeX8\expandafter\endcsname
\csname HoLogo@BibTeX8\endcsname
%    \end{macrocode}
%    \end{macro}
%
% \subsubsection{\hologo{ConTeXt}}
%
%    \begin{macro}{\HoLogo@ConTeXt@simple}
%    \begin{macrocode}
\def\HoLogo@ConTeXt@simple#1{%
  \HOLOGO@mbox{Con}%
  \HOLOGO@discretionary
  \HOLOGO@mbox{\hologo{TeX}t}%
}
%    \end{macrocode}
%    \end{macro}
%    \begin{macro}{\HoLogoHtml@ConTeXt@simple}
%    \begin{macrocode}
\let\HoLogoHtml@ConTeXt@simple\HoLogo@ConTeXt@simple
%    \end{macrocode}
%    \end{macro}
%
%    \begin{macro}{\HoLogo@ConTeXt@narrow}
%    This definition of logo \hologo{ConTeXt} with variant \xoption{narrow}
%    comes from TUGboat's class \xclass{ltugboat} (version 2010/11/15 v2.8).
%    \begin{macrocode}
\def\HoLogo@ConTeXt@narrow#1{%
  \HOLOGO@mbox{C\kern-.0333emon}%
  \HOLOGO@discretionary
  \kern-.0667em%
  \HOLOGO@mbox{\hologo{TeX}\kern-.0333emt}%
}
%    \end{macrocode}
%    \end{macro}
%    \begin{macro}{\HoLogoHtml@ConTeXt@narrow}
%    \begin{macrocode}
\def\HoLogoHtml@ConTeXt@narrow#1{%
  \HoLogoCss@ConTeXt@narrow
  \HOLOGO@Span{ConTeXt-narrow}{%
    \HOLOGO@Span{C}{C}%
    on%
    \hologo{TeX}%
    t%
  }%
}
%    \end{macrocode}
%    \end{macro}
%    \begin{macro}{\HoLogoCss@ConTeXt@narrow}
%    \begin{macrocode}
\def\HoLogoCss@ConTeXt@narrow{%
  \Css{%
    span.HoLogo-ConTeXt-narrow span.HoLogo-C{%
      margin-left:-.0333em;%
    }%
  }%
  \Css{%
    span.HoLogo-ConTeXt-narrow span.HoLogo-TeX{%
      margin-left:-.0667em;%
      margin-right:-.0333em;%
    }%
  }%
  \global\let\HoLogoCss@ConTeXt@narrow\relax
}
%    \end{macrocode}
%    \end{macro}
%
%    \begin{macro}{\HoLogo@ConTeXt}
%    \begin{macrocode}
\def\HoLogo@ConTeXt{\HoLogo@ConTeXt@narrow}
%    \end{macrocode}
%    \end{macro}
%    \begin{macro}{\HoLogoHtml@ConTeXt}
%    \begin{macrocode}
\def\HoLogoHtml@ConTeXt{\HoLogoHtml@ConTeXt@narrow}
%    \end{macrocode}
%    \end{macro}
%
% \subsubsection{\hologo{emTeX}}
%
%    \begin{macro}{\HoLogo@emTeX}
%    \begin{macrocode}
\def\HoLogo@emTeX#1{%
  \HOLOGO@mbox{#1{e}{E}m}%
  \HOLOGO@discretionary
  \hologo{TeX}%
}
%    \end{macrocode}
%    \end{macro}
%    \begin{macro}{\HoLogoCs@emTeX}
%    \begin{macrocode}
\def\HoLogoCs@emTeX#1{#1{e}{E}mTeX}%
%    \end{macrocode}
%    \end{macro}
%    \begin{macro}{\HoLogoBkm@emTeX}
%    \begin{macrocode}
\def\HoLogoBkm@emTeX#1{%
  #1{e}{E}m\hologo{TeX}%
}
%    \end{macrocode}
%    \end{macro}
%    \begin{macro}{\HoLogoHtml@emTeX}
%    \begin{macrocode}
\let\HoLogoHtml@emTeX\HoLogo@emTeX
%    \end{macrocode}
%    \end{macro}
%
% \subsubsection{\hologo{ExTeX}}
%
%    \begin{macro}{\HoLogo@ExTeX}
%    The definition is taken from the FAQ of the
%    project \hologo{ExTeX}
%    \cite{ExTeX-FAQ}.
%\begin{quote}
%\begin{verbatim}
%\def\ExTeX{%
%  \textrm{% Logo always with serifs
%    \ensuremath{%
%      \textstyle
%      \varepsilon_{%
%        \kern-0.15em%
%        \mathcal{X}%
%      }%
%    }%
%    \kern-.15em%
%    \TeX
%  }%
%}
%\end{verbatim}
%\end{quote}
%    \begin{macrocode}
\def\HoLogo@ExTeX#1{%
  \HoLogoFont@font{ExTeX}{rm}{%
    \ltx@mbox{%
      \HOLOGO@MathSetup
      $%
        \textstyle
        \varepsilon_{%
          \kern-0.15em%
          \HoLogoFont@font{ExTeX}{sy}{X}%
        }%
      $%
    }%
    \HOLOGO@discretionary
    \kern-.15em%
    \hologo{TeX}%
  }%
}
%    \end{macrocode}
%    \end{macro}
%    \begin{macro}{\HoLogoHtml@ExTeX}
%    \begin{macrocode}
\def\HoLogoHtml@ExTeX#1{%
  \HoLogoCss@ExTeX
  \HoLogoFont@font{ExTeX}{rm}{%
    \HOLOGO@Span{ExTeX}{%
      \ltx@mbox{%
        \HOLOGO@MathSetup
        $\textstyle\varepsilon$%
        \HOLOGO@Span{X}{$\textstyle\chi$}%
        \hologo{TeX}%
      }%
    }%
  }%
}
%    \end{macrocode}
%    \end{macro}
%    \begin{macro}{\HoLogoBkm@ExTeX}
%    \begin{macrocode}
\def\HoLogoBkm@ExTeX#1{%
  \HOLOGO@PdfdocUnicode{#1{e}{E}x}{\textepsilon\textchi}%
  \hologo{TeX}%
}
%    \end{macrocode}
%    \end{macro}
%    \begin{macro}{\HoLogoCss@ExTeX}
%    \begin{macrocode}
\def\HoLogoCss@ExTeX{%
  \Css{%
    span.HoLogo-ExTeX{%
      font-family:serif;%
    }%
  }%
  \Css{%
    span.HoLogo-ExTeX span.HoLogo-TeX{%
      margin-left:-.15em;%
    }%
  }%
  \global\let\HoLogoCss@ExTeX\relax
}
%    \end{macrocode}
%    \end{macro}
%
% \subsubsection{\hologo{MiKTeX}}
%
%    \begin{macro}{\HoLogo@MiKTeX}
%    \begin{macrocode}
\def\HoLogo@MiKTeX#1{%
  \HOLOGO@mbox{MiK}%
  \HOLOGO@discretionary
  \hologo{TeX}%
}
%    \end{macrocode}
%    \end{macro}
%    \begin{macro}{\HoLogoHtml@MiKTeX}
%    \begin{macrocode}
\let\HoLogoHtml@MiKTeX\HoLogo@MiKTeX
%    \end{macrocode}
%    \end{macro}
%
% \subsubsection{\hologo{OzTeX} and friends}
%
%    Source: \hologo{OzTeX} FAQ \cite{OzTeX}:
%    \begin{quote}
%      |\def\OzTeX{O\kern-.03em z\kern-.15em\TeX}|\\
%      (There is no kerning in OzMF, OzMP and OzTtH.)
%    \end{quote}
%
%    \begin{macro}{\HoLogo@OzTeX}
%    \begin{macrocode}
\def\HoLogo@OzTeX#1{%
  O%
  \kern-.03em %
  z%
  \kern-.15em %
  \hologo{TeX}%
}
%    \end{macrocode}
%    \end{macro}
%    \begin{macro}{\HoLogoHtml@OzTeX}
%    \begin{macrocode}
\def\HoLogoHtml@OzTeX#1{%
  \HoLogoCss@OzTeX
  \HOLOGO@Span{OzTeX}{%
    O%
    \HOLOGO@Span{z}{z}%
    \hologo{TeX}%
  }%
}
%    \end{macrocode}
%    \end{macro}
%    \begin{macro}{\HoLogoCss@OzTeX}
%    \begin{macrocode}
\def\HoLogoCss@OzTeX{%
  \Css{%
    span.HoLogo-OzTeX span.HoLogo-z{%
      margin-left:-.03em;%
      margin-right:-.15em;%
    }%
  }%
  \global\let\HoLogoCss@OzTeX\relax
}
%    \end{macrocode}
%    \end{macro}
%
%    \begin{macro}{\HoLogo@OzMF}
%    \begin{macrocode}
\def\HoLogo@OzMF#1{%
  \HOLOGO@mbox{OzMF}%
}
%    \end{macrocode}
%    \end{macro}
%    \begin{macro}{\HoLogo@OzMP}
%    \begin{macrocode}
\def\HoLogo@OzMP#1{%
  \HOLOGO@mbox{OzMP}%
}
%    \end{macrocode}
%    \end{macro}
%    \begin{macro}{\HoLogo@OzTtH}
%    \begin{macrocode}
\def\HoLogo@OzTtH#1{%
  \HOLOGO@mbox{OzTtH}%
}
%    \end{macrocode}
%    \end{macro}
%
% \subsubsection{\hologo{PCTeX}}
%
%    \begin{macro}{\HoLogo@PCTeX}
%    \begin{macrocode}
\def\HoLogo@PCTeX#1{%
  \HOLOGO@mbox{PC}%
  \hologo{TeX}%
}
%    \end{macrocode}
%    \end{macro}
%    \begin{macro}{\HoLogoHtml@PCTeX}
%    \begin{macrocode}
\let\HoLogoHtml@PCTeX\HoLogo@PCTeX
%    \end{macrocode}
%    \end{macro}
%
% \subsubsection{\hologo{PiCTeX}}
%
%    The original definitions from \xfile{pictex.tex} \cite{PiCTeX}:
%\begin{quote}
%\begin{verbatim}
%\def\PiC{%
%  P%
%  \kern-.12em%
%  \lower.5ex\hbox{I}%
%  \kern-.075em%
%  C%
%}
%\def\PiCTeX{%
%  \PiC
%  \kern-.11em%
%  \TeX
%}
%\end{verbatim}
%\end{quote}
%
%    \begin{macro}{\HoLogo@PiC}
%    \begin{macrocode}
\def\HoLogo@PiC#1{%
  P%
  \kern-.12em%
  \lower.5ex\hbox{I}%
  \kern-.075em%
  C%
  \HOLOGO@SpaceFactor
}
%    \end{macrocode}
%    \end{macro}
%    \begin{macro}{\HoLogoHtml@PiC}
%    \begin{macrocode}
\def\HoLogoHtml@PiC#1{%
  \HoLogoCss@PiC
  \HOLOGO@Span{PiC}{%
    P%
    \HOLOGO@Span{i}{I}%
    C%
  }%
}
%    \end{macrocode}
%    \end{macro}
%    \begin{macro}{\HoLogoCss@PiC}
%    \begin{macrocode}
\def\HoLogoCss@PiC{%
  \Css{%
    span.HoLogo-PiC span.HoLogo-i{%
      position:relative;%
      top:.5ex;%
      margin-left:-.12em;%
      margin-right:-.075em;%
      text-decoration:none;%
    }%
  }%
  \global\let\HoLogoCss@PiC\relax
}
%    \end{macrocode}
%    \end{macro}
%
%    \begin{macro}{\HoLogo@PiCTeX}
%    \begin{macrocode}
\def\HoLogo@PiCTeX#1{%
  \hologo{PiC}%
  \HOLOGO@discretionary
  \kern-.11em%
  \hologo{TeX}%
}
%    \end{macrocode}
%    \end{macro}
%    \begin{macro}{\HoLogoHtml@PiCTeX}
%    \begin{macrocode}
\def\HoLogoHtml@PiCTeX#1{%
  \HoLogoCss@PiCTeX
  \HOLOGO@Span{PiCTeX}{%
    \hologo{PiC}%
    \hologo{TeX}%
  }%
}
%    \end{macrocode}
%    \end{macro}
%    \begin{macro}{\HoLogoCss@PiCTeX}
%    \begin{macrocode}
\def\HoLogoCss@PiCTeX{%
  \Css{%
    span.HoLogo-PiCTeX span.HoLogo-PiC{%
      margin-right:-.11em;%
    }%
  }%
  \global\let\HoLogoCss@PiCTeX\relax
}
%    \end{macrocode}
%    \end{macro}
%
% \subsubsection{\hologo{teTeX}}
%
%    \begin{macro}{\HoLogo@teTeX}
%    \begin{macrocode}
\def\HoLogo@teTeX#1{%
  \HOLOGO@mbox{#1{t}{T}e}%
  \HOLOGO@discretionary
  \hologo{TeX}%
}
%    \end{macrocode}
%    \end{macro}
%    \begin{macro}{\HoLogoCs@teTeX}
%    \begin{macrocode}
\def\HoLogoCs@teTeX#1{#1{t}{T}dfTeX}
%    \end{macrocode}
%    \end{macro}
%    \begin{macro}{\HoLogoBkm@teTeX}
%    \begin{macrocode}
\def\HoLogoBkm@teTeX#1{%
  #1{t}{T}e\hologo{TeX}%
}
%    \end{macrocode}
%    \end{macro}
%    \begin{macro}{\HoLogoHtml@teTeX}
%    \begin{macrocode}
\let\HoLogoHtml@teTeX\HoLogo@teTeX
%    \end{macrocode}
%    \end{macro}
%
% \subsubsection{\hologo{TeX4ht}}
%
%    \begin{macro}{\HoLogo@TeX4ht}
%    \begin{macrocode}
\expandafter\def\csname HoLogo@TeX4ht\endcsname#1{%
  \HOLOGO@mbox{\hologo{TeX}4ht}%
}
%    \end{macrocode}
%    \end{macro}
%    \begin{macro}{\HoLogoHtml@TeX4ht}
%    \begin{macrocode}
\expandafter
\let\csname HoLogoHtml@TeX4ht\expandafter\endcsname
\csname HoLogo@TeX4ht\endcsname
%    \end{macrocode}
%    \end{macro}
%
%
% \subsubsection{\hologo{SageTeX}}
%
%    \begin{macro}{\HoLogo@SageTeX}
%    \begin{macrocode}
\def\HoLogo@SageTeX#1{%
  \HOLOGO@mbox{Sage}%
  \HOLOGO@discretionary
  \HOLOGO@NegativeKerning{eT,oT,To}%
  \hologo{TeX}%
}
%    \end{macrocode}
%    \end{macro}
%    \begin{macro}{\HoLogoHtml@SageTeX}
%    \begin{macrocode}
\let\HoLogoHtml@SageTeX\HoLogo@SageTeX
%    \end{macrocode}
%    \end{macro}
%
% \subsection{\hologo{METAFONT} and friends}
%
%    \begin{macro}{\HoLogo@METAFONT}
%    \begin{macrocode}
\def\HoLogo@METAFONT#1{%
  \HoLogoFont@font{METAFONT}{logo}{%
    \HOLOGO@mbox{META}%
    \HOLOGO@discretionary
    \HOLOGO@mbox{FONT}%
  }%
}
%    \end{macrocode}
%    \end{macro}
%
%    \begin{macro}{\HoLogo@METAPOST}
%    \begin{macrocode}
\def\HoLogo@METAPOST#1{%
  \HoLogoFont@font{METAPOST}{logo}{%
    \HOLOGO@mbox{META}%
    \HOLOGO@discretionary
    \HOLOGO@mbox{POST}%
  }%
}
%    \end{macrocode}
%    \end{macro}
%
%    \begin{macro}{\HoLogo@MetaFun}
%    \begin{macrocode}
\def\HoLogo@MetaFun#1{%
  \HOLOGO@mbox{Meta}%
  \HOLOGO@discretionary
  \HOLOGO@mbox{Fun}%
}
%    \end{macrocode}
%    \end{macro}
%
%    \begin{macro}{\HoLogo@MetaPost}
%    \begin{macrocode}
\def\HoLogo@MetaPost#1{%
  \HOLOGO@mbox{Meta}%
  \HOLOGO@discretionary
  \HOLOGO@mbox{Post}%
}
%    \end{macrocode}
%    \end{macro}
%
% \subsection{Others}
%
% \subsubsection{\hologo{biber}}
%
%    \begin{macro}{\HoLogo@biber}
%    \begin{macrocode}
\def\HoLogo@biber#1{%
  \HOLOGO@mbox{#1{b}{B}i}%
  \HOLOGO@discretionary
  \HOLOGO@mbox{ber}%
}
%    \end{macrocode}
%    \end{macro}
%    \begin{macro}{\HoLogoCs@biber}
%    \begin{macrocode}
\def\HoLogoCs@biber#1{#1{b}{B}iber}
%    \end{macrocode}
%    \end{macro}
%    \begin{macro}{\HoLogoBkm@biber}
%    \begin{macrocode}
\def\HoLogoBkm@biber#1{%
  #1{b}{B}iber%
}
%    \end{macrocode}
%    \end{macro}
%    \begin{macro}{\HoLogoHtml@biber}
%    \begin{macrocode}
\let\HoLogoHtml@biber\HoLogo@biber
%    \end{macrocode}
%    \end{macro}
%
% \subsubsection{\hologo{KOMAScript}}
%
%    \begin{macro}{\HoLogo@KOMAScript}
%    The definition for \hologo{KOMAScript} is taken
%    from \hologo{KOMAScript} (\xfile{scrlogo.dtx}, reformatted) \cite{scrlogo}:
%\begin{quote}
%\begin{verbatim}
%\@ifundefined{KOMAScript}{%
%  \DeclareRobustCommand{\KOMAScript}{%
%    \textsf{%
%      K\kern.05em O\kern.05emM\kern.05em A%
%      \kern.1em-\kern.1em %
%      Script%
%    }%
%  }%
%}{}
%\end{verbatim}
%\end{quote}
%    \begin{macrocode}
\def\HoLogo@KOMAScript#1{%
  \HoLogoFont@font{KOMAScript}{sf}{%
    \HOLOGO@mbox{%
      K\kern.05em%
      O\kern.05em%
      M\kern.05em%
      A%
    }%
    \kern.1em%
    \HOLOGO@hyphen
    \kern.1em%
    \HOLOGO@mbox{Script}%
  }%
}
%    \end{macrocode}
%    \end{macro}
%    \begin{macro}{\HoLogoBkm@KOMAScript}
%    \begin{macrocode}
\def\HoLogoBkm@KOMAScript#1{%
  KOMA-Script%
}
%    \end{macrocode}
%    \end{macro}
%    \begin{macro}{\HoLogoHtml@KOMAScript}
%    \begin{macrocode}
\def\HoLogoHtml@KOMAScript#1{%
  \HoLogoCss@KOMAScript
  \HoLogoFont@font{KOMAScript}{sf}{%
    \HOLOGO@Span{KOMAScript}{%
      K%
      \HOLOGO@Span{O}{O}%
      M%
      \HOLOGO@Span{A}{A}%
      \HOLOGO@Span{hyphen}{-}%
      Script%
    }%
  }%
}
%    \end{macrocode}
%    \end{macro}
%    \begin{macro}{\HoLogoCss@KOMAScript}
%    \begin{macrocode}
\def\HoLogoCss@KOMAScript{%
  \Css{%
    span.HoLogo-KOMAScript{%
      font-family:sans-serif;%
    }%
  }%
  \Css{%
    span.HoLogo-KOMAScript span.HoLogo-O{%
      padding-left:.05em;%
      padding-right:.05em;%
    }%
  }%
  \Css{%
    span.HoLogo-KOMAScript span.HoLogo-A{%
      padding-left:.05em;%
    }%
  }%
  \Css{%
    span.HoLogo-KOMAScript span.HoLogo-hyphen{%
      padding-left:.1em;%
      padding-right:.1em;%
    }%
  }%
  \global\let\HoLogoCss@KOMAScript\relax
}
%    \end{macrocode}
%    \end{macro}
%
% \subsubsection{\hologo{LyX}}
%
%    \begin{macro}{\HoLogo@LyX}
%    The definition is taken from the documentation source files
%    of \hologo{LyX}, \xfile{Intro.lyx} \cite{LyX}:
%\begin{quote}
%\begin{verbatim}
%\def\LyX{%
%  \texorpdfstring{%
%    L\kern-.1667em\lower.25em\hbox{Y}\kern-.125emX\@%
%  }{%
%    LyX%
%  }%
%}
%\end{verbatim}
%\end{quote}
%    \begin{macrocode}
\def\HoLogo@LyX#1{%
  L%
  \kern-.1667em%
  \lower.25em\hbox{Y}%
  \kern-.125em%
  X%
  \HOLOGO@SpaceFactor
}
%    \end{macrocode}
%    \end{macro}
%    \begin{macro}{\HoLogoHtml@LyX}
%    \begin{macrocode}
\def\HoLogoHtml@LyX#1{%
  \HoLogoCss@LyX
  \HOLOGO@Span{LyX}{%
    L%
    \HOLOGO@Span{y}{Y}%
    X%
  }%
}
%    \end{macrocode}
%    \end{macro}
%    \begin{macro}{\HoLogoCss@LyX}
%    \begin{macrocode}
\def\HoLogoCss@LyX{%
  \Css{%
    span.HoLogo-LyX span.HoLogo-y{%
      position:relative;%
      top:.25em;%
      margin-left:-.1667em;%
      margin-right:-.125em;%
      text-decoration:none;%
    }%
  }%
  \global\let\HoLogoCss@LyX\relax
}
%    \end{macrocode}
%    \end{macro}
%
% \subsubsection{\hologo{NTS}}
%
%    \begin{macro}{\HoLogo@NTS}
%    Definition for \hologo{NTS} can be found in
%    package \xpackage{etex\textunderscore man} for the \hologo{eTeX} manual \cite{etexman}
%    and in package \xpackage{dtklogos} \cite{dtklogos}:
%\begin{quote}
%\begin{verbatim}
%\def\NTS{%
%  \leavevmode
%  \hbox{%
%    $%
%      \cal N%
%      \kern-0.35em%
%      \lower0.5ex\hbox{$\cal T$}%
%      \kern-0.2em%
%      S%
%    $%
%  }%
%}
%\end{verbatim}
%\end{quote}
%    \begin{macrocode}
\def\HoLogo@NTS#1{%
  \HoLogoFont@font{NTS}{sy}{%
    N\/%
    \kern-.35em%
    \lower.5ex\hbox{T\/}%
    \kern-.2em%
    S\/%
  }%
  \HOLOGO@SpaceFactor
}
%    \end{macrocode}
%    \end{macro}
%
% \subsubsection{\Hologo{TTH} (\hologo{TeX} to HTML translator)}
%
%    Source: \url{http://hutchinson.belmont.ma.us/tth/}
%    In the HTML source the second `T' is printed as subscript.
%\begin{quote}
%\begin{verbatim}
%T<sub>T</sub>H
%\end{verbatim}
%\end{quote}
%    \begin{macro}{\HoLogo@TTH}
%    \begin{macrocode}
\def\HoLogo@TTH#1{%
  \ltx@mbox{%
    T\HOLOGO@SubScript{T}H%
  }%
  \HOLOGO@SpaceFactor
}
%    \end{macrocode}
%    \end{macro}
%
%    \begin{macro}{\HoLogoHtml@TTH}
%    \begin{macrocode}
\def\HoLogoHtml@TTH#1{%
  T\HCode{<sub>}T\HCode{</sub>}H%
}
%    \end{macrocode}
%    \end{macro}
%
% \subsubsection{\Hologo{HanTheThanh}}
%
%    Partial source: Package \xpackage{dtklogos}.
%    The double accent is U+1EBF (latin small letter e with circumflex
%    and acute).
%    \begin{macro}{\HoLogo@HanTheThanh}
%    \begin{macrocode}
\def\HoLogo@HanTheThanh#1{%
  \ltx@mbox{H\`an}%
  \HOLOGO@space
  \ltx@mbox{%
    Th%
    \HOLOGO@IfCharExists{"1EBF}{%
      \char"1EBF\relax
    }{%
      \^e\hbox to 0pt{\hss\raise .5ex\hbox{\'{}}}%
    }%
  }%
  \HOLOGO@space
  \ltx@mbox{Th\`anh}%
}
%    \end{macrocode}
%    \end{macro}
%    \begin{macro}{\HoLogoBkm@HanTheThanh}
%    \begin{macrocode}
\def\HoLogoBkm@HanTheThanh#1{%
  H\`an %
  Th\HOLOGO@PdfdocUnicode{\^e}{\9036\277} %
  Th\`anh%
}
%    \end{macrocode}
%    \end{macro}
%    \begin{macro}{\HoLogoHtml@HanTheThanh}
%    \begin{macrocode}
\def\HoLogoHtml@HanTheThanh#1{%
  H\`an %
  Th\HCode{&\ltx@hashchar x1ebf;} %
  Th\`anh%
}
%    \end{macrocode}
%    \end{macro}
%
% \subsection{Driver detection}
%
%    \begin{macrocode}
\HOLOGO@IfExists\InputIfFileExists{%
  \InputIfFileExists{hologo.cfg}{}{}%
}{%
  \ltx@IfUndefined{pdf@filesize}{%
    \def\HOLOGO@InputIfExists{%
      \openin\HOLOGO@temp=hologo.cfg\relax
      \ifeof\HOLOGO@temp
        \closein\HOLOGO@temp
      \else
        \closein\HOLOGO@temp
        \begingroup
          \def\x{LaTeX2e}%
        \expandafter\endgroup
        \ifx\fmtname\x
          \input{hologo.cfg}%
        \else
          \input hologo.cfg\relax
        \fi
      \fi
    }%
    \ltx@IfUndefined{newread}{%
      \chardef\HOLOGO@temp=15 %
      \def\HOLOGO@CheckRead{%
        \ifeof\HOLOGO@temp
          \HOLOGO@InputIfExists
        \else
          \ifcase\HOLOGO@temp
            \@PackageWarningNoLine{hologo}{%
              Configuration file ignored, because\MessageBreak
              a free read register could not be found%
            }%
          \else
            \begingroup
              \count\ltx@cclv=\HOLOGO@temp
              \advance\ltx@cclv by \ltx@minusone
              \edef\x{\endgroup
                \chardef\noexpand\HOLOGO@temp=\the\count\ltx@cclv
                \relax
              }%
            \x
          \fi
        \fi
      }%
    }{%
      \csname newread\endcsname\HOLOGO@temp
      \HOLOGO@InputIfExists
    }%
  }{%
    \edef\HOLOGO@temp{\pdf@filesize{hologo.cfg}}%
    \ifx\HOLOGO@temp\ltx@empty
    \else
      \ifnum\HOLOGO@temp>0 %
        \begingroup
          \def\x{LaTeX2e}%
        \expandafter\endgroup
        \ifx\fmtname\x
          \input{hologo.cfg}%
        \else
          \input hologo.cfg\relax
        \fi
      \else
        \@PackageInfoNoLine{hologo}{%
          Empty configuration file `hologo.cfg' ignored%
        }%
      \fi
    \fi
  }%
}
%    \end{macrocode}
%
%    \begin{macrocode}
\def\HOLOGO@temp#1#2{%
  \kv@define@key{HoLogoDriver}{#1}[]{%
    \begingroup
      \def\HOLOGO@temp{##1}%
      \ltx@onelevel@sanitize\HOLOGO@temp
      \ifx\HOLOGO@temp\ltx@empty
      \else
        \@PackageError{hologo}{%
          Value (\HOLOGO@temp) not permitted for option `#1'%
        }%
        \@ehc
      \fi
    \endgroup
    \def\hologoDriver{#2}%
  }%
}%
\def\HOLOGO@@temp#1#2{%
  \ifx\kv@value\relax
    \HOLOGO@temp{#1}{#1}%
  \else
    \HOLOGO@temp{#1}{#2}%
  \fi
}%
\kv@parse@normalized{%
  pdftex,%
  luatex=pdftex,%
  dvipdfm,%
  dvipdfmx=dvipdfm,%
  dvips,%
  dvipsone=dvips,%
  xdvi=dvips,%
  xetex,%
  vtex,%
}\HOLOGO@@temp
%    \end{macrocode}
%
%    \begin{macrocode}
\kv@define@key{HoLogoDriver}{driverfallback}{%
  \def\HOLOGO@DriverFallback{#1}%
}
%    \end{macrocode}
%
%    \begin{macro}{\HOLOGO@DriverFallback}
%    \begin{macrocode}
\def\HOLOGO@DriverFallback{dvips}
%    \end{macrocode}
%    \end{macro}
%
%    \begin{macro}{\hologoDriverSetup}
%    \begin{macrocode}
\def\hologoDriverSetup{%
  \let\hologoDriver\ltx@undefined
  \HOLOGO@DriverSetup
}
%    \end{macrocode}
%    \end{macro}
%
%    \begin{macro}{\HOLOGO@DriverSetup}
%    \begin{macrocode}
\def\HOLOGO@DriverSetup#1{%
  \kvsetkeys{HoLogoDriver}{#1}%
  \HOLOGO@CheckDriver
  \ltx@ifundefined{hologoDriver}{%
    \begingroup
    \edef\x{\endgroup
      \noexpand\kvsetkeys{HoLogoDriver}{\HOLOGO@DriverFallback}%
    }\x
  }{}%
  \@PackageInfoNoLine{hologo}{Using driver `\hologoDriver'}%
}
%    \end{macrocode}
%    \end{macro}
%
%    \begin{macro}{\HOLOGO@CheckDriver}
%    \begin{macrocode}
\def\HOLOGO@CheckDriver{%
  \ifpdf
    \def\hologoDriver{pdftex}%
    \let\HOLOGO@pdfliteral\pdfliteral
    \ifluatex
      \ifx\pdfextension\@undefined\else
        \protected\def\pdfliteral{\pdfextension literal}%
        \let\HOLOGO@pdfliteral\pdfliteral
      \fi
      \ltx@IfUndefined{HOLOGO@pdfliteral}{%
        \ifnum\luatexversion<36 %
        \else
          \begingroup
            \let\HOLOGO@temp\endgroup
            \ifcase0%
                \directlua{%
                  if tex.enableprimitives then %
                    tex.enableprimitives('HOLOGO@', {'pdfliteral'})%
                  else %
                    tex.print('1')%
                  end%
                }%
                \ifx\HOLOGO@pdfliteral\@undefined 1\fi%
                \relax%
              \endgroup
              \let\HOLOGO@temp\relax
              \global\let\HOLOGO@pdfliteral\HOLOGO@pdfliteral
            \fi%
          \HOLOGO@temp
        \fi
      }{}%
    \fi
    \ltx@IfUndefined{HOLOGO@pdfliteral}{%
      \@PackageWarningNoLine{hologo}{%
        Cannot find \string\pdfliteral
      }%
    }{}%
  \else
    \ifxetex
      \def\hologoDriver{xetex}%
    \else
      \ifvtex
        \def\hologoDriver{vtex}%
      \fi
    \fi
  \fi
}
%    \end{macrocode}
%    \end{macro}
%
%    \begin{macro}{\HOLOGO@WarningUnsupportedDriver}
%    \begin{macrocode}
\def\HOLOGO@WarningUnsupportedDriver#1{%
  \@PackageWarningNoLine{hologo}{%
    Logo `#1' needs driver specific macros,\MessageBreak
    but driver `\hologoDriver' is not supported.\MessageBreak
    Use a different driver or\MessageBreak
    load package `graphics' or `pgf'%
  }%
}
%    \end{macrocode}
%    \end{macro}
%
% \subsubsection{Reflect box macros}
%
%    Skip driver part if not needed.
%    \begin{macrocode}
\ltx@IfUndefined{reflectbox}{}{%
  \ltx@IfUndefined{rotatebox}{}{%
    \HOLOGO@AtEnd
  }%
}
\ltx@IfUndefined{pgftext}{}{%
  \HOLOGO@AtEnd
}
\ltx@IfUndefined{psscalebox}{}{%
  \HOLOGO@AtEnd
}
%    \end{macrocode}
%
%    \begin{macrocode}
\def\HOLOGO@temp{LaTeX2e}
\ifx\fmtname\HOLOGO@temp
  \RequirePackage{kvoptions}[2011/06/30]%
  \ProcessKeyvalOptions{HoLogoDriver}%
\fi
\HOLOGO@DriverSetup{}
%    \end{macrocode}
%
%    \begin{macro}{\HOLOGO@ReflectBox}
%    \begin{macrocode}
\def\HOLOGO@ReflectBox#1{%
  \begingroup
    \setbox\ltx@zero\hbox{\begingroup#1\endgroup}%
    \setbox\ltx@two\hbox{%
      \kern\wd\ltx@zero
      \csname HOLOGO@ScaleBox@\hologoDriver\endcsname{-1}{1}{%
        \hbox to 0pt{\copy\ltx@zero\hss}%
      }%
    }%
    \wd\ltx@two=\wd\ltx@zero
    \box\ltx@two
  \endgroup
}
%    \end{macrocode}
%    \end{macro}
%
%    \begin{macro}{\HOLOGO@PointReflectBox}
%    \begin{macrocode}
\def\HOLOGO@PointReflectBox#1{%
  \begingroup
    \setbox\ltx@zero\hbox{\begingroup#1\endgroup}%
    \setbox\ltx@two\hbox{%
      \kern\wd\ltx@zero
      \raise\ht\ltx@zero\hbox{%
        \csname HOLOGO@ScaleBox@\hologoDriver\endcsname{-1}{-1}{%
          \hbox to 0pt{\copy\ltx@zero\hss}%
        }%
      }%
    }%
    \wd\ltx@two=\wd\ltx@zero
    \box\ltx@two
  \endgroup
}
%    \end{macrocode}
%    \end{macro}
%
%    We must define all variants because of dynamic driver setup.
%    \begin{macrocode}
\def\HOLOGO@temp#1#2{#2}
%    \end{macrocode}
%
%    \begin{macro}{\HOLOGO@ScaleBox@pdftex}
%    \begin{macrocode}
\HOLOGO@temp{pdftex}{%
  \def\HOLOGO@ScaleBox@pdftex#1#2#3{%
    \HOLOGO@pdfliteral{%
      q #1 0 0 #2 0 0 cm%
    }%
    #3%
    \HOLOGO@pdfliteral{%
      Q%
    }%
  }%
}
%    \end{macrocode}
%    \end{macro}
%    \begin{macro}{\HOLOGO@ScaleBox@dvips}
%    \begin{macrocode}
\HOLOGO@temp{dvips}{%
  \def\HOLOGO@ScaleBox@dvips#1#2#3{%
    \special{ps:%
      gsave %
      currentpoint %
      currentpoint translate %
      #1 #2 scale %
      neg exch neg exch translate%
    }%
    #3%
    \special{ps:%
      currentpoint %
      grestore %
      moveto%
    }%
  }%
}
%    \end{macrocode}
%    \end{macro}
%    \begin{macro}{\HOLOGO@ScaleBox@dvipdfm}
%    \begin{macrocode}
\HOLOGO@temp{dvipdfm}{%
  \let\HOLOGO@ScaleBox@dvipdfm\HOLOGO@ScaleBox@dvips
}
%    \end{macrocode}
%    \end{macro}
%    Since \hologo{XeTeX} v0.6.
%    \begin{macro}{\HOLOGO@ScaleBox@xetex}
%    \begin{macrocode}
\HOLOGO@temp{xetex}{%
  \def\HOLOGO@ScaleBox@xetex#1#2#3{%
    \special{x:gsave}%
    \special{x:scale #1 #2}%
    #3%
    \special{x:grestore}%
  }%
}
%    \end{macrocode}
%    \end{macro}
%    \begin{macro}{\HOLOGO@ScaleBox@vtex}
%    \begin{macrocode}
\HOLOGO@temp{vtex}{%
  \def\HOLOGO@ScaleBox@vtex#1#2#3{%
    \special{r(#1,0,0,#2,0,0}%
    #3%
    \special{r)}%
  }%
}
%    \end{macrocode}
%    \end{macro}
%
%    \begin{macrocode}
\HOLOGO@AtEnd%
%</package>
%    \end{macrocode}
%
% \section{Test}
%
% \subsection{Catcode checks for loading}
%
%    \begin{macrocode}
%<*test1>
%    \end{macrocode}
%    \begin{macrocode}
\catcode`\{=1 %
\catcode`\}=2 %
\catcode`\#=6 %
\catcode`\@=11 %
\expandafter\ifx\csname count@\endcsname\relax
  \countdef\count@=255 %
\fi
\expandafter\ifx\csname @gobble\endcsname\relax
  \long\def\@gobble#1{}%
\fi
\expandafter\ifx\csname @firstofone\endcsname\relax
  \long\def\@firstofone#1{#1}%
\fi
\expandafter\ifx\csname loop\endcsname\relax
  \expandafter\@firstofone
\else
  \expandafter\@gobble
\fi
{%
  \def\loop#1\repeat{%
    \def\body{#1}%
    \iterate
  }%
  \def\iterate{%
    \body
      \let\next\iterate
    \else
      \let\next\relax
    \fi
    \next
  }%
  \let\repeat=\fi
}%
\def\RestoreCatcodes{}
\count@=0 %
\loop
  \edef\RestoreCatcodes{%
    \RestoreCatcodes
    \catcode\the\count@=\the\catcode\count@\relax
  }%
\ifnum\count@<255 %
  \advance\count@ 1 %
\repeat

\def\RangeCatcodeInvalid#1#2{%
  \count@=#1\relax
  \loop
    \catcode\count@=15 %
  \ifnum\count@<#2\relax
    \advance\count@ 1 %
  \repeat
}
\def\RangeCatcodeCheck#1#2#3{%
  \count@=#1\relax
  \loop
    \ifnum#3=\catcode\count@
    \else
      \errmessage{%
        Character \the\count@\space
        with wrong catcode \the\catcode\count@\space
        instead of \number#3%
      }%
    \fi
  \ifnum\count@<#2\relax
    \advance\count@ 1 %
  \repeat
}
\def\space{ }
\expandafter\ifx\csname LoadCommand\endcsname\relax
  \def\LoadCommand{\input hologo.sty\relax}%
\fi
\def\Test{%
  \RangeCatcodeInvalid{0}{47}%
  \RangeCatcodeInvalid{58}{64}%
  \RangeCatcodeInvalid{91}{96}%
  \RangeCatcodeInvalid{123}{255}%
  \catcode`\@=12 %
  \catcode`\\=0 %
  \catcode`\%=14 %
  \LoadCommand
  \RangeCatcodeCheck{0}{36}{15}%
  \RangeCatcodeCheck{37}{37}{14}%
  \RangeCatcodeCheck{38}{47}{15}%
  \RangeCatcodeCheck{48}{57}{12}%
  \RangeCatcodeCheck{58}{63}{15}%
  \RangeCatcodeCheck{64}{64}{12}%
  \RangeCatcodeCheck{65}{90}{11}%
  \RangeCatcodeCheck{91}{91}{15}%
  \RangeCatcodeCheck{92}{92}{0}%
  \RangeCatcodeCheck{93}{96}{15}%
  \RangeCatcodeCheck{97}{122}{11}%
  \RangeCatcodeCheck{123}{255}{15}%
  \RestoreCatcodes
}
\Test
\csname @@end\endcsname
\end
%    \end{macrocode}
%    \begin{macrocode}
%</test1>
%    \end{macrocode}
%
% \subsection{Spacefactor}
%
%    The space factor must be 1000 after a logo. If it is greater 1000
%    then the following space is a space after a sentence closing point.
%    If the space factor is smaller 1000 then an immediate following
%    dot is interpreted as abbreviation, not sentence closing point.
%
%    \begin{macrocode}
%<*test-spacefactor>
\NeedsTeXFormat{LaTeX2e}
\documentclass{article}
\usepackage{hologo}[2016/05/12]
\usepackage{kvsetkeys}
\usepackage{qstest}
\IncludeTests{*}
\LogTests{log}{*}{*}
\begin{document}
\begin{qstest}{spacefactor}{spacefactor}
\newcommand*{\Test}[1]{%
  \sbox0{%
    \hologo{#1}%
    \Expect*{1000 (#1)}*{\the\spacefactor\space(#1)}%
  }%
}%
\makeatletter
\def\TestList{}
\def\hologoEntry#1#2#3{%
  \edef\TestList{%
    \ifx\TestList\@empty
    \else
      \TestList,%
    \fi
    #1%
    \ifx\\#2\\%
    \else
      ={variant=#2}%
    \fi
  }%
}
\hologoList
\expandafter\kv@parse@normalized\expandafter{%
  \TestList
}{%
  \begingroup
    \let\@logo=\kv@key
    \ifx\kv@value\relax
    \else
      \expandafter\hologoLogoSetup\expandafter\@logo\expandafter{%
        \kv@value
      }%
    \fi
    \Test\@logo
  \endgroup
  \@gobbletwo
}
\end{qstest}
\end{document}
%</test-spacefactor>
%    \end{macrocode}
%
% \subsection{Complete list}
%
%    \begin{macrocode}
%<*test-list>
\NeedsTeXFormat{LaTeX2e}
\documentclass[12pt,a4paper]{article}
\usepackage{hologo}[2016/05/12]
\usepackage[T1]{fontenc}
\usepackage{lmodern}
\usepackage{parskip}
\usepackage[unicode]{hyperref}[2011/09/28]
\usepackage{bookmark}[2011/09/19]
\bookmarksetup{%
  numbered,%
  open,%
  openlevel=2,%
}
\renewcommand*{\contentsname}{List of logos}
\begin{document}
\tableofcontents
\def\TestFont#1#2#3#4#5#6{%
  \begingroup
    \usefont{#3}{#4}{#5}{#6}%
    \HologoVariant{#1}{#2}/\hologoVariant{#1}{#2}%
    \quad
    \begingroup\scriptsize\hologoVariant{#1}{#2}\endgroup
    \quad
  \endgroup
  (#3/#4/#5/#6)%
  \par
}
\makeatletter
\def\hologoEntry#1#2#3{%
  \section{%
    \HologoVariant{#1}{#2}/\hologoVariant{#1}{#2} %
    {[#1\ifx\\#2\\\else\space(#2)\fi]}% hash-ok
  }% braces around [] because of bug in tex4ht
  \begingroup
    \hypersetup{unicode=false}%
    \bookmark[%
      dest=\@currentHref,%
      rellevel=1,%
      keeplevel,%
    ]{%
      \HologoVariant{#1}{#2}/\hologoVariant{#1}{#2} %
      (PDFDocEncoding)%
    }%
  \endgroup
  \TestFont{#1}{#2}{OT1}{cmr}{m}{n}%
  \TestFont{#1}{#2}{OT1}{cmss}{m}{n}%
  \TestFont{#1}{#2}{OT1}{cmr}{b}{n}%
  \TestFont{#1}{#2}{OT1}{cmr}{m}{it}%
  \TestFont{#1}{#2}{OT1}{cmtt}{m}{n}%
  \TestFont{#1}{#2}{T1}{lmr}{m}{n}%
  \TestFont{#1}{#2}{T1}{lmss}{m}{n}%
  \TestFont{#1}{#2}{T1}{lmr}{b}{n}%
  \TestFont{#1}{#2}{T1}{lmr}{m}{it}%
  \TestFont{#1}{#2}{T1}{lmtt}{m}{n}%
  \TestFont{#1}{#2}{T1}{lmvtt}{m}{n}%
  \TestFont{#1}{#2}{T1}{qtm}{m}{n}%
  \TestFont{#1}{#2}{T1}{qhv}{m}{n}%
  \TestFont{#1}{#2}{T1}{qtm}{b}{n}%
  \TestFont{#1}{#2}{T1}{qtm}{m}{it}%
  \TestFont{#1}{#2}{T1}{qcr}{m}{n}%
  \newpage
}
\makeatother
\hologoList
\end{document}
%</test-list>
%    \end{macrocode}
%
% \section{Installation}
%
% \subsection{Download}
%
% \paragraph{Package.} This package is available on
% CTAN\footnote{\url{ftp://ftp.ctan.org/tex-archive/}}:
% \begin{description}
% \item[\CTAN{macros/latex/contrib/oberdiek/hologo.dtx}] The source file.
% \item[\CTAN{macros/latex/contrib/oberdiek/hologo.pdf}] Documentation.
% \end{description}
%
%
% \paragraph{Bundle.} All the packages of the bundle `oberdiek'
% are also available in a TDS compliant ZIP archive. There
% the packages are already unpacked and the documentation files
% are generated. The files and directories obey the TDS standard.
% \begin{description}
% \item[\CTAN{install/macros/latex/contrib/oberdiek.tds.zip}]
% \end{description}
% \emph{TDS} refers to the standard ``A Directory Structure
% for \TeX\ Files'' (\CTAN{tds/tds.pdf}). Directories
% with \xfile{texmf} in their name are usually organized this way.
%
% \subsection{Bundle installation}
%
% \paragraph{Unpacking.} Unpack the \xfile{oberdiek.tds.zip} in the
% TDS tree (also known as \xfile{texmf} tree) of your choice.
% Example (linux):
% \begin{quote}
%   |unzip oberdiek.tds.zip -d ~/texmf|
% \end{quote}
%
% \paragraph{Script installation.}
% Check the directory \xfile{TDS:scripts/oberdiek/} for
% scripts that need further installation steps.
% Package \xpackage{attachfile2} comes with the Perl script
% \xfile{pdfatfi.pl} that should be installed in such a way
% that it can be called as \texttt{pdfatfi}.
% Example (linux):
% \begin{quote}
%   |chmod +x scripts/oberdiek/pdfatfi.pl|\\
%   |cp scripts/oberdiek/pdfatfi.pl /usr/local/bin/|
% \end{quote}
%
% \subsection{Package installation}
%
% \paragraph{Unpacking.} The \xfile{.dtx} file is a self-extracting
% \docstrip\ archive. The files are extracted by running the
% \xfile{.dtx} through \plainTeX:
% \begin{quote}
%   \verb|tex hologo.dtx|
% \end{quote}
%
% \paragraph{TDS.} Now the different files must be moved into
% the different directories in your installation TDS tree
% (also known as \xfile{texmf} tree):
% \begin{quote}
% \def\t{^^A
% \begin{tabular}{@{}>{\ttfamily}l@{ $\rightarrow$ }>{\ttfamily}l@{}}
%   hologo.sty & tex/generic/oberdiek/hologo.sty\\
%   hologo.pdf & doc/latex/oberdiek/hologo.pdf\\
%   example/hologo-example.tex & doc/latex/oberdiek/example/hologo-example.tex\\
%   test/hologo-test1.tex & doc/latex/oberdiek/test/hologo-test1.tex\\
%   test/hologo-test-spacefactor.tex & doc/latex/oberdiek/test/hologo-test-spacefactor.tex\\
%   test/hologo-test-list.tex & doc/latex/oberdiek/test/hologo-test-list.tex\\
%   hologo.dtx & source/latex/oberdiek/hologo.dtx\\
% \end{tabular}^^A
% }^^A
% \sbox0{\t}^^A
% \ifdim\wd0>\linewidth
%   \begingroup
%     \advance\linewidth by\leftmargin
%     \advance\linewidth by\rightmargin
%   \edef\x{\endgroup
%     \def\noexpand\lw{\the\linewidth}^^A
%   }\x
%   \def\lwbox{^^A
%     \leavevmode
%     \hbox to \linewidth{^^A
%       \kern-\leftmargin\relax
%       \hss
%       \usebox0
%       \hss
%       \kern-\rightmargin\relax
%     }^^A
%   }^^A
%   \ifdim\wd0>\lw
%     \sbox0{\small\t}^^A
%     \ifdim\wd0>\linewidth
%       \ifdim\wd0>\lw
%         \sbox0{\footnotesize\t}^^A
%         \ifdim\wd0>\linewidth
%           \ifdim\wd0>\lw
%             \sbox0{\scriptsize\t}^^A
%             \ifdim\wd0>\linewidth
%               \ifdim\wd0>\lw
%                 \sbox0{\tiny\t}^^A
%                 \ifdim\wd0>\linewidth
%                   \lwbox
%                 \else
%                   \usebox0
%                 \fi
%               \else
%                 \lwbox
%               \fi
%             \else
%               \usebox0
%             \fi
%           \else
%             \lwbox
%           \fi
%         \else
%           \usebox0
%         \fi
%       \else
%         \lwbox
%       \fi
%     \else
%       \usebox0
%     \fi
%   \else
%     \lwbox
%   \fi
% \else
%   \usebox0
% \fi
% \end{quote}
% If you have a \xfile{docstrip.cfg} that configures and enables \docstrip's
% TDS installing feature, then some files can already be in the right
% place, see the documentation of \docstrip.
%
% \subsection{Refresh file name databases}
%
% If your \TeX~distribution
% (\teTeX, \mikTeX, \dots) relies on file name databases, you must refresh
% these. For example, \teTeX\ users run \verb|texhash| or
% \verb|mktexlsr|.
%
% \subsection{Some details for the interested}
%
% \paragraph{Attached source.}
%
% The PDF documentation on CTAN also includes the
% \xfile{.dtx} source file. It can be extracted by
% AcrobatReader 6 or higher. Another option is \textsf{pdftk},
% e.g. unpack the file into the current directory:
% \begin{quote}
%   \verb|pdftk hologo.pdf unpack_files output .|
% \end{quote}
%
% \paragraph{Unpacking with \LaTeX.}
% The \xfile{.dtx} chooses its action depending on the format:
% \begin{description}
% \item[\plainTeX:] Run \docstrip\ and extract the files.
% \item[\LaTeX:] Generate the documentation.
% \end{description}
% If you insist on using \LaTeX\ for \docstrip\ (really,
% \docstrip\ does not need \LaTeX), then inform the autodetect routine
% about your intention:
% \begin{quote}
%   \verb|latex \let\install=y\input{hologo.dtx}|
% \end{quote}
% Do not forget to quote the argument according to the demands
% of your shell.
%
% \paragraph{Generating the documentation.}
% You can use both the \xfile{.dtx} or the \xfile{.drv} to generate
% the documentation. The process can be configured by the
% configuration file \xfile{ltxdoc.cfg}. For instance, put this
% line into this file, if you want to have A4 as paper format:
% \begin{quote}
%   \verb|\PassOptionsToClass{a4paper}{article}|
% \end{quote}
% An example follows how to generate the
% documentation with pdf\LaTeX:
% \begin{quote}
%\begin{verbatim}
%pdflatex hologo.dtx
%makeindex -s gind.ist hologo.idx
%pdflatex hologo.dtx
%makeindex -s gind.ist hologo.idx
%pdflatex hologo.dtx
%\end{verbatim}
% \end{quote}
%
% \section{Catalogue}
%
% The following XML file can be used as source for the
% \href{http://mirror.ctan.org/help/Catalogue/catalogue.html}{\TeX\ Catalogue}.
% The elements \texttt{caption} and \texttt{description} are imported
% from the original XML file from the Catalogue.
% The name of the XML file in the Catalogue is \xfile{hologo.xml}.
%    \begin{macrocode}
%<*catalogue>
<?xml version='1.0' encoding='us-ascii'?>
<!DOCTYPE entry SYSTEM 'catalogue.dtd'>
<entry datestamp='$Date$' modifier='$Author$' id='hologo'>
  <name>hologo</name>
  <caption>A collection of logos with bookmark support.</caption>
  <authorref id='auth:oberdiek'/>
  <copyright owner='Heiko Oberdiek' year='2010-2012'/>
  <license type='lppl1.3'/>
  <version number='1.10'/>
  <description>
    The package defines a single command <tt>\hologo</tt>, whose
    argument is the usual case-confused ASCII version of the logo.
    The command is bookmark-enabled, so that every logo becomes
    available in bookmarks without further work.
    <p/>
    The package is part of the <xref refid='oberdiek'>oberdiek</xref>
    bundle.
  </description>
  <documentation details='Package documentation'
      href='ctan:/macros/latex/contrib/oberdiek/hologo.pdf'/>
  <ctan file='true' path='/macros/latex/contrib/oberdiek/hologo.dtx'/>
  <miktex location='oberdiek'/>
  <texlive location='oberdiek'/>
  <install path='/macros/latex/contrib/oberdiek/oberdiek.tds.zip'/>
</entry>
%</catalogue>
%    \end{macrocode}
%
% \begin{thebibliography}{9}
% \raggedright
%
% \bibitem{btxdoc}
% Oren Patashnik,
% \textit{\hologo{BibTeX}ing},
% 1988-02-08.\\
% \CTAN{biblio/bibtex/base/}
%
% \bibitem{dtklogos}
% Gerd Neugebauer, DANTE,
% \textit{Package \xpackage{dtklogos}},
% 2011-04-25.\\
% \CTAN{usergrps/dante/dtk/dtklogos.sty}
%
% \bibitem{etexman}
% The \hologo{NTS} Team,
% \textit{The \hologo{eTeX} manual},
% 1998-02.\\
% \CTAN{systems/e-tex/v2/doc/}
%
% \bibitem{ExTeX-FAQ}
% The \hologo{ExTeX} group,
% \textit{\hologo{ExTeX}: FAQ -- How is \hologo{ExTeX} typeset?},
% 2007-04-14.\\
% \url{http://www.extex.org/documentation/faq.html}
%
% \bibitem{LyX}
% %@MISC{ LyX,
% %  title = {{LyX 2.0.0 -- The Document Processor [Computer software and manual]}},
% %  author = {{The LyX Team}},
% %  howpublished = {Internet: http://www.lyx.org},
% %  year = {2011-05-08},
% %  note = {Retrieved May 10, 2011, from http://www.lyx.org},
% %  url = {http://www.lyx.org/}
% %}
% The \hologo{LyX} Team,
% \textit{\hologo{LyX} -- The Document Processor},
% 2011-05-08.\\
% \url{http://www.lyx.org/}
%
% \bibitem{OzTeX}
% Andrew Trevorrow,
% \hologo{OzTeX} FAQ: What is the correct way to typeset ``\hologo{OzTeX}''?,
% 2011-09-15 (visited).
% \url{http://www.trevorrow.com/oztex/ozfaq.html#oztex-logo}
%
% \bibitem{PiCTeX}
% Michael Wichura,
% \textit{The \hologo{PiCTeX} macro package},
% 1987-09-21.
% \CTAN{graphics/pictex/}
%
% \bibitem{scrlogo}
% Markus Kohm,
% \textit{\hologo{KOMAScript} Datei \xfile{scrlogo.dtx}},
% 2009-01-30.\\
% \CTAN{install/macros/latex/contrib/komascript.tds.zip}
%
% \end{thebibliography}
%
% \begin{History}
%   \begin{Version}{2010/04/08 v1.0}
%   \item
%     The first version.
%   \end{Version}
%   \begin{Version}{2010/04/16 v1.1}
%   \item
%     \cs{Hologo} added for support of logos at start of a sentence.
%   \item
%     \cs{hologoSetup} and \cs{hologoLogoSetup} added.
%   \item
%     Options \xoption{break}, \xoption{hyphenbreak}, \xoption{spacebreak}
%     added.
%   \item
%     Variant support added by option \xoption{variant}.
%   \end{Version}
%   \begin{Version}{2010/04/24 v1.2}
%   \item
%     \hologo{LaTeX3} added.
%   \item
%     \hologo{VTeX} added.
%   \end{Version}
%   \begin{Version}{2010/11/21 v1.3}
%   \item
%     \hologo{iniTeX}, \hologo{virTeX} added.
%   \end{Version}
%   \begin{Version}{2011/03/25 v1.4}
%   \item
%     \hologo{ConTeXt} with variants added.
%   \item
%     Option \xoption{discretionarybreak} added as refinement for
%     option \xoption{break}.
%   \end{Version}
%   \begin{Version}{2011/04/21 v1.5}
%   \item
%     Wrong TDS directory for test files fixed.
%   \end{Version}
%   \begin{Version}{2011/10/01 v1.6}
%   \item
%     Support for package \xpackage{tex4ht} added.
%   \item
%     Support for \cs{csname} added if \cs{ifincsname} is available.
%   \item
%     New logos:
%     \hologo{(La)TeX},
%     \hologo{biber},
%     \hologo{BibTeX} (\xoption{sc}, \xoption{sf}),
%     \hologo{emTeX},
%     \hologo{ExTeX},
%     \hologo{KOMAScript},
%     \hologo{La},
%     \hologo{LyX},
%     \hologo{MiKTeX},
%     \hologo{NTS},
%     \hologo{OzMF},
%     \hologo{OzMP},
%     \hologo{OzTeX},
%     \hologo{OzTtH},
%     \hologo{PCTeX},
%     \hologo{PiC},
%     \hologo{PiCTeX},
%     \hologo{METAFONT},
%     \hologo{MetaFun},
%     \hologo{METAPOST},
%     \hologo{MetaPost},
%     \hologo{SLiTeX} (\xoption{lift}, \xoption{narrow}, \xoption{simple}),
%     \hologo{SliTeX} (\xoption{narrow}, \xoption{simple}, \xoption{lift}),
%     \hologo{teTeX}.
%   \item
%     Fixes:
%     \hologo{iniTeX},
%     \hologo{pdfLaTeX},
%     \hologo{pdfTeX},
%     \hologo{virTeX}.
%   \item
%     \cs{hologoFontSetup} and \cs{hologoLogoFontSetup} added.
%   \item
%     \cs{hologoVariant} and \cs{HologoVariant} added.
%   \end{Version}
%   \begin{Version}{2011/11/22 v1.7}
%   \item
%     New logos:
%     \hologo{BibTeX8},
%     \hologo{LaTeXML},
%     \hologo{SageTeX},
%     \hologo{TeX4ht},
%     \hologo{TTH}.
%   \item
%     \hologo{Xe} and friends: Driver stuff fixed.
%   \item
%     \hologo{Xe} and friends: Support for italic added.
%   \item
%     \hologo{Xe} and friends: Package support for \xpackage{pgf}
%     and \xpackage{pstricks} added.
%   \end{Version}
%   \begin{Version}{2011/11/29 v1.8}
%   \item
%     New logos:
%     \hologo{HanTheThanh}.
%   \end{Version}
%   \begin{Version}{2011/12/21 v1.9}
%   \item
%     Patch for package \xpackage{ifxetex} added for the case that
%     \cs{newif} is undefined in \hologo{iniTeX}.
%   \item
%     Some fixes for \hologo{iniTeX}.
%   \end{Version}
%   \begin{Version}{2012/04/26 v1.10}
%   \item
%     Fix in bookmark version of logo ``\hologo{HanTheThanh}''.
%   \end{Version}
%   \begin{Version}{2016/05/12 v1.11}
%   \item
%     Update HOLOGO@IfCharExists (previously in texlive)
%   \item define pdfliteral in current luatex.
%   \end{Version}
% \end{History}
%
% \PrintIndex
%
% \Finale
\endinput
|
% \end{quote}
% Do not forget to quote the argument according to the demands
% of your shell.
%
% \paragraph{Generating the documentation.}
% You can use both the \xfile{.dtx} or the \xfile{.drv} to generate
% the documentation. The process can be configured by the
% configuration file \xfile{ltxdoc.cfg}. For instance, put this
% line into this file, if you want to have A4 as paper format:
% \begin{quote}
%   \verb|\PassOptionsToClass{a4paper}{article}|
% \end{quote}
% An example follows how to generate the
% documentation with pdf\LaTeX:
% \begin{quote}
%\begin{verbatim}
%pdflatex hologo.dtx
%makeindex -s gind.ist hologo.idx
%pdflatex hologo.dtx
%makeindex -s gind.ist hologo.idx
%pdflatex hologo.dtx
%\end{verbatim}
% \end{quote}
%
% \section{Catalogue}
%
% The following XML file can be used as source for the
% \href{http://mirror.ctan.org/help/Catalogue/catalogue.html}{\TeX\ Catalogue}.
% The elements \texttt{caption} and \texttt{description} are imported
% from the original XML file from the Catalogue.
% The name of the XML file in the Catalogue is \xfile{hologo.xml}.
%    \begin{macrocode}
%<*catalogue>
<?xml version='1.0' encoding='us-ascii'?>
<!DOCTYPE entry SYSTEM 'catalogue.dtd'>
<entry datestamp='$Date$' modifier='$Author$' id='hologo'>
  <name>hologo</name>
  <caption>A collection of logos with bookmark support.</caption>
  <authorref id='auth:oberdiek'/>
  <copyright owner='Heiko Oberdiek' year='2010-2012'/>
  <license type='lppl1.3'/>
  <version number='1.10'/>
  <description>
    The package defines a single command <tt>\hologo</tt>, whose
    argument is the usual case-confused ASCII version of the logo.
    The command is bookmark-enabled, so that every logo becomes
    available in bookmarks without further work.
    <p/>
    The package is part of the <xref refid='oberdiek'>oberdiek</xref>
    bundle.
  </description>
  <documentation details='Package documentation'
      href='ctan:/macros/latex/contrib/oberdiek/hologo.pdf'/>
  <ctan file='true' path='/macros/latex/contrib/oberdiek/hologo.dtx'/>
  <miktex location='oberdiek'/>
  <texlive location='oberdiek'/>
  <install path='/macros/latex/contrib/oberdiek/oberdiek.tds.zip'/>
</entry>
%</catalogue>
%    \end{macrocode}
%
% \begin{thebibliography}{9}
% \raggedright
%
% \bibitem{btxdoc}
% Oren Patashnik,
% \textit{\hologo{BibTeX}ing},
% 1988-02-08.\\
% \CTAN{biblio/bibtex/base/}
%
% \bibitem{dtklogos}
% Gerd Neugebauer, DANTE,
% \textit{Package \xpackage{dtklogos}},
% 2011-04-25.\\
% \CTAN{usergrps/dante/dtk/dtklogos.sty}
%
% \bibitem{etexman}
% The \hologo{NTS} Team,
% \textit{The \hologo{eTeX} manual},
% 1998-02.\\
% \CTAN{systems/e-tex/v2/doc/}
%
% \bibitem{ExTeX-FAQ}
% The \hologo{ExTeX} group,
% \textit{\hologo{ExTeX}: FAQ -- How is \hologo{ExTeX} typeset?},
% 2007-04-14.\\
% \url{http://www.extex.org/documentation/faq.html}
%
% \bibitem{LyX}
% %@MISC{ LyX,
% %  title = {{LyX 2.0.0 -- The Document Processor [Computer software and manual]}},
% %  author = {{The LyX Team}},
% %  howpublished = {Internet: http://www.lyx.org},
% %  year = {2011-05-08},
% %  note = {Retrieved May 10, 2011, from http://www.lyx.org},
% %  url = {http://www.lyx.org/}
% %}
% The \hologo{LyX} Team,
% \textit{\hologo{LyX} -- The Document Processor},
% 2011-05-08.\\
% \url{http://www.lyx.org/}
%
% \bibitem{OzTeX}
% Andrew Trevorrow,
% \hologo{OzTeX} FAQ: What is the correct way to typeset ``\hologo{OzTeX}''?,
% 2011-09-15 (visited).
% \url{http://www.trevorrow.com/oztex/ozfaq.html#oztex-logo}
%
% \bibitem{PiCTeX}
% Michael Wichura,
% \textit{The \hologo{PiCTeX} macro package},
% 1987-09-21.
% \CTAN{graphics/pictex/}
%
% \bibitem{scrlogo}
% Markus Kohm,
% \textit{\hologo{KOMAScript} Datei \xfile{scrlogo.dtx}},
% 2009-01-30.\\
% \CTAN{install/macros/latex/contrib/komascript.tds.zip}
%
% \end{thebibliography}
%
% \begin{History}
%   \begin{Version}{2010/04/08 v1.0}
%   \item
%     The first version.
%   \end{Version}
%   \begin{Version}{2010/04/16 v1.1}
%   \item
%     \cs{Hologo} added for support of logos at start of a sentence.
%   \item
%     \cs{hologoSetup} and \cs{hologoLogoSetup} added.
%   \item
%     Options \xoption{break}, \xoption{hyphenbreak}, \xoption{spacebreak}
%     added.
%   \item
%     Variant support added by option \xoption{variant}.
%   \end{Version}
%   \begin{Version}{2010/04/24 v1.2}
%   \item
%     \hologo{LaTeX3} added.
%   \item
%     \hologo{VTeX} added.
%   \end{Version}
%   \begin{Version}{2010/11/21 v1.3}
%   \item
%     \hologo{iniTeX}, \hologo{virTeX} added.
%   \end{Version}
%   \begin{Version}{2011/03/25 v1.4}
%   \item
%     \hologo{ConTeXt} with variants added.
%   \item
%     Option \xoption{discretionarybreak} added as refinement for
%     option \xoption{break}.
%   \end{Version}
%   \begin{Version}{2011/04/21 v1.5}
%   \item
%     Wrong TDS directory for test files fixed.
%   \end{Version}
%   \begin{Version}{2011/10/01 v1.6}
%   \item
%     Support for package \xpackage{tex4ht} added.
%   \item
%     Support for \cs{csname} added if \cs{ifincsname} is available.
%   \item
%     New logos:
%     \hologo{(La)TeX},
%     \hologo{biber},
%     \hologo{BibTeX} (\xoption{sc}, \xoption{sf}),
%     \hologo{emTeX},
%     \hologo{ExTeX},
%     \hologo{KOMAScript},
%     \hologo{La},
%     \hologo{LyX},
%     \hologo{MiKTeX},
%     \hologo{NTS},
%     \hologo{OzMF},
%     \hologo{OzMP},
%     \hologo{OzTeX},
%     \hologo{OzTtH},
%     \hologo{PCTeX},
%     \hologo{PiC},
%     \hologo{PiCTeX},
%     \hologo{METAFONT},
%     \hologo{MetaFun},
%     \hologo{METAPOST},
%     \hologo{MetaPost},
%     \hologo{SLiTeX} (\xoption{lift}, \xoption{narrow}, \xoption{simple}),
%     \hologo{SliTeX} (\xoption{narrow}, \xoption{simple}, \xoption{lift}),
%     \hologo{teTeX}.
%   \item
%     Fixes:
%     \hologo{iniTeX},
%     \hologo{pdfLaTeX},
%     \hologo{pdfTeX},
%     \hologo{virTeX}.
%   \item
%     \cs{hologoFontSetup} and \cs{hologoLogoFontSetup} added.
%   \item
%     \cs{hologoVariant} and \cs{HologoVariant} added.
%   \end{Version}
%   \begin{Version}{2011/11/22 v1.7}
%   \item
%     New logos:
%     \hologo{BibTeX8},
%     \hologo{LaTeXML},
%     \hologo{SageTeX},
%     \hologo{TeX4ht},
%     \hologo{TTH}.
%   \item
%     \hologo{Xe} and friends: Driver stuff fixed.
%   \item
%     \hologo{Xe} and friends: Support for italic added.
%   \item
%     \hologo{Xe} and friends: Package support for \xpackage{pgf}
%     and \xpackage{pstricks} added.
%   \end{Version}
%   \begin{Version}{2011/11/29 v1.8}
%   \item
%     New logos:
%     \hologo{HanTheThanh}.
%   \end{Version}
%   \begin{Version}{2011/12/21 v1.9}
%   \item
%     Patch for package \xpackage{ifxetex} added for the case that
%     \cs{newif} is undefined in \hologo{iniTeX}.
%   \item
%     Some fixes for \hologo{iniTeX}.
%   \end{Version}
%   \begin{Version}{2012/04/26 v1.10}
%   \item
%     Fix in bookmark version of logo ``\hologo{HanTheThanh}''.
%   \end{Version}
%   \begin{Version}{2016/05/12 v1.11}
%   \item
%     Update HOLOGO@IfCharExists (previously in texlive)
%   \item define pdfliteral in current luatex.
%   \end{Version}
% \end{History}
%
% \PrintIndex
%
% \Finale
\endinput

%        (quote the arguments according to the demands of your shell)
%
% Documentation:
%    (a) If hologo.drv is present:
%           latex hologo.drv
%    (b) Without hologo.drv:
%           latex hologo.dtx; ...
%    The class ltxdoc loads the configuration file ltxdoc.cfg
%    if available. Here you can specify further options, e.g.
%    use A4 as paper format:
%       \PassOptionsToClass{a4paper}{article}
%
%    Programm calls to get the documentation (example):
%       pdflatex hologo.dtx
%       makeindex -s gind.ist hologo.idx
%       pdflatex hologo.dtx
%       makeindex -s gind.ist hologo.idx
%       pdflatex hologo.dtx
%
% Installation:
%    TDS:tex/generic/oberdiek/hologo.sty
%    TDS:doc/latex/oberdiek/hologo.pdf
%    TDS:doc/latex/oberdiek/example/hologo-example.tex
%    TDS:doc/latex/oberdiek/test/hologo-test1.tex
%    TDS:doc/latex/oberdiek/test/hologo-test-spacefactor.tex
%    TDS:doc/latex/oberdiek/test/hologo-test-list.tex
%    TDS:source/latex/oberdiek/hologo.dtx
%
%<*ignore>
\begingroup
  \catcode123=1 %
  \catcode125=2 %
  \def\x{LaTeX2e}%
\expandafter\endgroup
\ifcase 0\ifx\install y1\fi\expandafter
         \ifx\csname processbatchFile\endcsname\relax\else1\fi
         \ifx\fmtname\x\else 1\fi\relax
\else\csname fi\endcsname
%</ignore>
%<*install>
\input docstrip.tex
\Msg{************************************************************************}
\Msg{* Installation}
\Msg{* Package: hologo 2016/05/12 v1.11 A logo collection with bookmark support (HO)}
\Msg{************************************************************************}

\keepsilent
\askforoverwritefalse

\let\MetaPrefix\relax
\preamble

This is a generated file.

Project: hologo
Version: 2016/05/12 v1.11

Copyright (C) 2010-2012 by
   Heiko Oberdiek <heiko.oberdiek at googlemail.com>

This work may be distributed and/or modified under the
conditions of the LaTeX Project Public License, either
version 1.3c of this license or (at your option) any later
version. This version of this license is in
   http://www.latex-project.org/lppl/lppl-1-3c.txt
and the latest version of this license is in
   http://www.latex-project.org/lppl.txt
and version 1.3 or later is part of all distributions of
LaTeX version 2005/12/01 or later.

This work has the LPPL maintenance status "maintained".

This Current Maintainer of this work is Heiko Oberdiek.

The Base Interpreter refers to any `TeX-Format',
because some files are installed in TDS:tex/generic//.

This work consists of the main source file hologo.dtx
and the derived files
   hologo.sty, hologo.pdf, hologo.ins, hologo.drv, hologo-example.tex,
   hologo-test1.tex, hologo-test-spacefactor.tex,
   hologo-test-list.tex.

\endpreamble
\let\MetaPrefix\DoubleperCent

\generate{%
  \file{hologo.ins}{\from{hologo.dtx}{install}}%
  \file{hologo.drv}{\from{hologo.dtx}{driver}}%
  \usedir{tex/generic/oberdiek}%
  \file{hologo.sty}{\from{hologo.dtx}{package}}%
  \usedir{doc/latex/oberdiek/example}%
  \file{hologo-example.tex}{\from{hologo.dtx}{example}}%
  \usedir{doc/latex/oberdiek/test}%
  \file{hologo-test1.tex}{\from{hologo.dtx}{test1}}%
  \file{hologo-test-spacefactor.tex}{\from{hologo.dtx}{test-spacefactor}}%
  \file{hologo-test-list.tex}{\from{hologo.dtx}{test-list}}%
  \nopreamble
  \nopostamble
  \usedir{source/latex/oberdiek/catalogue}%
  \file{hologo.xml}{\from{hologo.dtx}{catalogue}}%
}

\catcode32=13\relax% active space
\let =\space%
\Msg{************************************************************************}
\Msg{*}
\Msg{* To finish the installation you have to move the following}
\Msg{* file into a directory searched by TeX:}
\Msg{*}
\Msg{*     hologo.sty}
\Msg{*}
\Msg{* To produce the documentation run the file `hologo.drv'}
\Msg{* through LaTeX.}
\Msg{*}
\Msg{* Happy TeXing!}
\Msg{*}
\Msg{************************************************************************}

\endbatchfile
%</install>
%<*ignore>
\fi
%</ignore>
%<*driver>
\NeedsTeXFormat{LaTeX2e}
\ProvidesFile{hologo.drv}%
  [2016/05/12 v1.11 A logo collection with bookmark support (HO)]%
\documentclass{ltxdoc}
\usepackage{holtxdoc}[2011/11/22]
\usepackage{hologo}[2016/05/12]
\usepackage{longtable}
\usepackage{array}
\usepackage{paralist}
%\usepackage[T1]{fontenc}
%\usepackage{lmodern}
\begin{document}
  \DocInput{hologo.dtx}%
\end{document}
%</driver>
% \fi
%
%
% \CharacterTable
%  {Upper-case    \A\B\C\D\E\F\G\H\I\J\K\L\M\N\O\P\Q\R\S\T\U\V\W\X\Y\Z
%   Lower-case    \a\b\c\d\e\f\g\h\i\j\k\l\m\n\o\p\q\r\s\t\u\v\w\x\y\z
%   Digits        \0\1\2\3\4\5\6\7\8\9
%   Exclamation   \!     Double quote  \"     Hash (number) \#
%   Dollar        \$     Percent       \%     Ampersand     \&
%   Acute accent  \'     Left paren    \(     Right paren   \)
%   Asterisk      \*     Plus          \+     Comma         \,
%   Minus         \-     Point         \.     Solidus       \/
%   Colon         \:     Semicolon     \;     Less than     \<
%   Equals        \=     Greater than  \>     Question mark \?
%   Commercial at \@     Left bracket  \[     Backslash     \\
%   Right bracket \]     Circumflex    \^     Underscore    \_
%   Grave accent  \`     Left brace    \{     Vertical bar  \|
%   Right brace   \}     Tilde         \~}
%
% \GetFileInfo{hologo.drv}
%
% \title{The \xpackage{hologo} package}
% \date{2016/05/12 v1.11}
% \author{Heiko Oberdiek\\\xemail{heiko.oberdiek at googlemail.com}}
%
% \maketitle
%
% \begin{abstract}
% This package starts a collection of logos with support for bookmarks
% strings.
% \end{abstract}
%
% \tableofcontents
%
% \section{Documentation}
%
% \subsection{Logo macros}
%
% \begin{declcs}{hologo} \M{name}
% \end{declcs}
% Macro \cs{hologo} sets the logo with name \meta{name}.
% The following table shows the supported names.
%
% \begingroup
%   \def\hologoEntry#1#2#3{^^A
%     #1&#2&\hologoLogoSetup{#1}{variant=#2}\hologo{#1}&#3\tabularnewline
%   }
%   \begin{longtable}{>{\ttfamily}l>{\ttfamily}lll}
%     \rmfamily\bfseries{name} & \rmfamily\bfseries variant
%     & \bfseries logo & \bfseries since\\
%     \hline
%     \endhead
%     \hologoList
%   \end{longtable}
% \endgroup
%
% \begin{declcs}{Hologo} \M{name}
% \end{declcs}
% Macro \cs{Hologo} starts the logo \meta{name} with an uppercase
% letter. As an exception small greek letters are not converted
% to uppercase. Examples, see \hologo{eTeX} and \hologo{ExTeX}.
%
% \subsection{Setup macros}
%
% The package does not support package options, but the following
% setup macros can be used to set options.
%
% \begin{declcs}{hologoSetup} \M{key value list}
% \end{declcs}
% Macro \cs{hologoSetup} sets global options.
%
% \begin{declcs}{hologoLogoSetup} \M{logo} \M{key value list}
% \end{declcs}
% Some options can also be used to configure a logo.
% These settings take precedence over global option settings.
%
% \subsection{Options}\label{sec:options}
%
% There are boolean and string options:
% \begin{description}
% \item[Boolean option:]
% It takes |true| or |false|
% as value. If the value is omitted, then |true| is used.
% \item[String option:]
% A value must be given as string. (But the string might be empty.)
% \end{description}
% The following options can be used both in \cs{hologoSetup}
% and \cs{hologoLogoSetup}:
% \begin{description}
% \def\entry#1{\item[\xoption{#1}:]}
% \entry{break}
%   enables or disables line breaks inside the logo. This setting is
%   refined by options \xoption{hyphenbreak}, \xoption{spacebreak}
%   or \xoption{discretionarybreak}.
%   Default is |false|.
% \entry{hyphenbreak}
%   enables or disables the line break right after the hyphen character.
% \entry{spacebreak}
%   enables or disables line breaks at space characters.
% \entry{discretionarybreak}
%   enables or disables line breaks at hyphenation points
%   (inserted by \cs{-}).
% \end{description}
% Macro \cs{hologoLogoSetup} also knows:
% \begin{description}
% \item[\xoption{variant}:]
%   This is a string option. It specifies a variant of a logo that
%   must exist. An empty string selects the package default variant.
% \end{description}
% Example:
% \begin{quote}
%   |\hologoSetup{break=false}|\\
%   |\hologoLogoSetup{plainTeX}{variant=hyphen,hyphenbreak}|\\
%   Then ``plain-\TeX'' contains one break point after the hyphen.
% \end{quote}
%
% \subsection{Driver options}
%
% Sometimes graphical operations are needed to construct some
% glyphs (e.g.\ \hologo{XeTeX}). If package \xpackage{graphics}
% or package \xpackage{pgf} are found, then the macros are taken
% from there. Otherwise the packge defines its own operations
% and therefore needs the driver information. Many drivers are
% detected automatically (\hologo{pdfTeX}/\hologo{LuaTeX}
% in PDF mode, \hologo{XeTeX}, \hologo{VTeX}). These have precedence
% over a driver option. The driver can be given as package option
% or using \cs{hologoDriverSetup}.
% The following list contains the recognized driver options:
% \begin{itemize}
% \item \xoption{pdftex}, \xoption{luatex}
% \item \xoption{dvipdfm}, \xoption{dvipdfmx}
% \item \xoption{dvips}, \xoption{dvipsone}, \xoption{xdvi}
% \item \xoption{xetex}
% \item \xoption{vtex}
% \end{itemize}
% The left driver of a line is the driver name that is used internally.
% The following names are aliases for drivers that use the
% same method. Therefore the entry in the \xext{log} file for
% the used driver prints the internally used driver name.
% \begin{description}
% \item[\xoption{driverfallback}:]
%   This option expects a driver that is used,
%   if the driver could not be detected automatically.
% \end{description}
%
% \begin{declcs}{hologoDriverSetup} \M{driver option}
% \end{declcs}
% The driver can also be configured after package loading
% using \cs{hologoDriverSetup}, also the way for \hologo{plainTeX}
% to setup the driver.
%
% \subsection{Font setup}
%
% Some logos require a special font, but should also be usable by
% \hologo{plainTeX}. Therefore the package provides some ways
% to influence the font settings. The options below
% take font settings as values. Both font commands
% such as \cs{sffamily} and macros that take one argument
% like \cs{textsf} can be used.
%
% \begin{declcs}{hologoFontSetup} \M{key value list}
% \end{declcs}
% Macro \cs{hologoFontSetup} sets the fonts for all logos.
% Supported keys:
% \begin{description}
% \def\entry#1{\item[\xoption{#1}:]}
% \entry{general}
%   This font is used for all logos. The default is empty.
%   That means no special font is used.
% \entry{bibsf}
%   This font is used for
%   {\hologoLogoSetup{BibTeX}{variant=sf}\hologo{BibTeX}}
%   with variant \xoption{sf}.
% \entry{rm}
%   This font is a serif font. It is used for \hologo{ExTeX}.
% \entry{sc}
%   This font specifies a small caps font. It is used for
%   {\hologoLogoSetup{BibTeX}{variant=sc}\hologo{BibTeX}}
%   with variant \xoption{sc}.
% \entry{sf}
%   This font specifies a sans serif font. The default
%   is \cs{sffamily}, then \cs{sf} is tried. Otherwise
%   a warning is given. It is used by \hologo{KOMAScript}.
% \entry{sy}
%   This is the font for math symbols (e.g. cmsy).
%   It is used by \hologo{AmS}, \hologo{NTS}, \hologo{ExTeX}.
% \entry{logo}
%   \hologo{METAFONT} and \hologo{METAPOST} are using that font.
%   In \hologo{LaTeX} \cs{logofamily} is used and
%   the definitions of package \xpackage{mflogo} are used
%   if the package is not loaded.
%   Otherwise the \cs{tenlogo} is used and defined
%   if it does not already exists.
% \end{description}
%
% \begin{declcs}{hologoLogoFontSetup} \M{logo} \M{key value list}
% \end{declcs}
% Fonts can also be set for a logo or logo component separately,
% see the following list.
% The keys are the same as for \cs{hologoFontSetup}.
%
% \begin{longtable}{>{\ttfamily}l>{\sffamily}ll}
%   \meta{logo} & keys & result\\
%   \hline
%   \endhead
%   BibTeX & bibsf & {\hologoLogoSetup{BibTeX}{variant=sf}\hologo{BibTeX}}\\[.5ex]
%   BibTeX & sc & {\hologoLogoSetup{BibTeX}{variant=sc}\hologo{BibTeX}}\\[.5ex]
%   ExTeX & rm & \hologo{ExTeX}\\
%   SliTeX & rm & \hologo{SliTeX}\\[.5ex]
%   AmS & sy & \hologo{AmS}\\
%   ExTeX & sy & \hologo{ExTeX}\\
%   NTS & sy & \hologo{NTS}\\[.5ex]
%   KOMAScript & sf & \hologo{KOMAScript}\\[.5ex]
%   METAFONT & logo & \hologo{METAFONT}\\
%   METAPOST & logo & \hologo{METAPOST}\\[.5ex]
%   SliTeX & sc \hologo{SliTeX}
% \end{longtable}
%
% \subsubsection{Font order}
%
% For all logos the font \xoption{general} is applied first.
% Example:
%\begin{quote}
%|\hologoFontSetup{general=\color{red}}|
%\end{quote}
% will print red logos.
% Then if the font uses a special font \xoption{sf}, for example,
% the font is applied that is setup by \cs{hologoLogoFontSetup}.
% If this font is not setup, then the common font setup
% by \cs{hologoFontSetup} is used. Otherwise a warning is given,
% that there is no font configured.
%
% \subsection{Additional user macros}
%
% Usually a variant of a logo is configured by using
% \cs{hologoLogoSetup}, because it is bad style to mix
% different variants of the same logo in the same text.
% There the following macros are a convenience for testing.
%
% \begin{declcs}{hologoVariant} \M{name} \M{variant}\\
%   \cs{HologoVariant} \M{name} \M{variant}
% \end{declcs}
% Logo \meta{name} is set using \meta{variant} that specifies
% explicitely which variant of the macro is used. If the argument
% is empty, then the default form of the logo is used
% (configurable by \cs{hologoLogoSetup}).
%
% \cs{HologoVariant} is used if the logo is set in a context
% that needs an uppercase first letter (beginning of a sentence, \dots).
%
% \begin{declcs}{hologoList}\\
%   \cs{hologoEntry} \M{logo} \M{variant} \M{since}
% \end{declcs}
% Macro \cs{hologoList} contains all logos that are provided
% by the package including variants. The list consists of calls
% of \cs{hologoEntry} with three arguments starting with the
% logo name \meta{logo} and its variant \meta{variant}. An empty
% variant means the current default. Argument \meta{since} specifies
% with version of the package \xpackage{hologo} is needed to get
% the logo. If the logo is fixed, then the date gets updated.
% Therefore the date \meta{since} is not exactly the date of
% the first introduction, but rather the date of the latest fix.
%
% Before \cs{hologoList} can be used, macro \cs{hologoEntry} needs
% a definition. The example file in section \ref{sec:example}
% shows applications of \cs{hologoList}.
%
% \subsection{Supported contexts}
%
% Macros \cs{hologo} and friends support special contexts:
% \begin{itemize}
% \item \hologo{LaTeX}'s protection mechanism.
% \item Bookmarks of package \xpackage{hyperref}.
% \item Package \xpackage{tex4ht}.
% \item The macros can be used inside \cs{csname} constructs,
%   if \cs{ifincsname} is available (\hologo{pdfTeX}, \hologo{XeTeX},
%   \hologo{LuaTeX}).
% \end{itemize}
%
% \subsection{Example}
% \label{sec:example}
%
% The following example prints the logos in different fonts.
%    \begin{macrocode}
%<*example>
%<<verbatim
\NeedsTeXFormat{LaTeX2e}
\documentclass[a4paper]{article}
\usepackage[
  hmargin=20mm,
  vmargin=20mm,
]{geometry}
\pagestyle{empty}
\usepackage{hologo}[2016/05/12]
\usepackage{longtable}
\usepackage{array}
\setlength{\extrarowheight}{2pt}
\usepackage[T1]{fontenc}
\usepackage{lmodern}
\usepackage{pdflscape}
\usepackage[
  pdfencoding=auto,
]{hyperref}
\hypersetup{
  pdfauthor={Heiko Oberdiek},
  pdftitle={Example for package `hologo'},
  pdfsubject={Logos with fonts lmr, lmss, qtm, qpl, qhv},
}
\usepackage{bookmark}

% Print the logo list on the console

\begingroup
  \typeout{}%
  \typeout{*** Begin of logo list ***}%
  \newcommand*{\hologoEntry}[3]{%
    \typeout{#1 \ifx\\#2\\\else(#2) \fi[#3]}%
  }%
  \hologoList
  \typeout{*** End of logo list ***}%
  \typeout{}%
\endgroup

\begin{document}
\begin{landscape}

  \section{Example file for package `hologo'}

  % Table for font names

  \begin{longtable}{>{\bfseries}ll}
    \textbf{font} & \textbf{Font name}\\
    \hline
    lmr & Latin Modern Roman\\
    lmss & Latin Modern Sans\\
    qtm & \TeX\ Gyre Termes\\
    qhv & \TeX\ Gyre Heros\\
    qpl & \TeX\ Gyre Pagella\\
  \end{longtable}

  % Logo list with logos in different fonts

  \begingroup
    \newcommand*{\SetVariant}[2]{%
      \ifx\\#2\\%
      \else
        \hologoLogoSetup{#1}{variant=#2}%
      \fi
    }%
    \newcommand*{\hologoEntry}[3]{%
      \SetVariant{#1}{#2}%
      \raisebox{1em}[0pt][0pt]{\hypertarget{#1@#2}{}}%
      \bookmark[%
        dest={#1@#2},%
      ]{%
        #1\ifx\\#2\\\else\space(#2)\fi: \Hologo{#1}, \hologo{#1} %
        [Unicode]%
      }%
      \hypersetup{unicode=false}%
      \bookmark[%
        dest={#1@#2},%
      ]{%
        #1\ifx\\#2\\\else\space(#2)\fi: \Hologo{#1}, \hologo{#1} %
        [PDFDocEncoding]%
      }%
      \texttt{#1}%
      &%
      \texttt{#2}%
      &%
      \Hologo{#1}%
      &%
      \SetVariant{#1}{#2}%
      \hologo{#1}%
      &%
      \SetVariant{#1}{#2}%
      \fontfamily{qtm}\selectfont
      \hologo{#1}%
      &%
      \SetVariant{#1}{#2}%
      \fontfamily{qpl}\selectfont
      \hologo{#1}%
      &%
      \SetVariant{#1}{#2}%
      \textsf{\hologo{#1}}%
      &%
      \SetVariant{#1}{#2}%
      \fontfamily{qhv}\selectfont
      \hologo{#1}%
      \tabularnewline
    }%
    \begin{longtable}{llllllll}%
      \textbf{\textit{logo}} & \textbf{\textit{variant}} &
      \texttt{\string\Hologo} &
      \textbf{lmr} & \textbf{qtm} & \textbf{qpl} &
      \textbf{lmss} & \textbf{qhv}
      \tabularnewline
      \hline
      \endhead
      \hologoList
    \end{longtable}%
  \endgroup

\end{landscape}
\end{document}
%verbatim
%</example>
%    \end{macrocode}
%
% \StopEventually{
% }
%
% \section{Implementation}
%    \begin{macrocode}
%<*package>
%    \end{macrocode}
%    Reload check, especially if the package is not used with \LaTeX.
%    \begin{macrocode}
\begingroup\catcode61\catcode48\catcode32=10\relax%
  \catcode13=5 % ^^M
  \endlinechar=13 %
  \catcode35=6 % #
  \catcode39=12 % '
  \catcode44=12 % ,
  \catcode45=12 % -
  \catcode46=12 % .
  \catcode58=12 % :
  \catcode64=11 % @
  \catcode123=1 % {
  \catcode125=2 % }
  \expandafter\let\expandafter\x\csname ver@hologo.sty\endcsname
  \ifx\x\relax % plain-TeX, first loading
  \else
    \def\empty{}%
    \ifx\x\empty % LaTeX, first loading,
      % variable is initialized, but \ProvidesPackage not yet seen
    \else
      \expandafter\ifx\csname PackageInfo\endcsname\relax
        \def\x#1#2{%
          \immediate\write-1{Package #1 Info: #2.}%
        }%
      \else
        \def\x#1#2{\PackageInfo{#1}{#2, stopped}}%
      \fi
      \x{hologo}{The package is already loaded}%
      \aftergroup\endinput
    \fi
  \fi
\endgroup%
%    \end{macrocode}
%    Package identification:
%    \begin{macrocode}
\begingroup\catcode61\catcode48\catcode32=10\relax%
  \catcode13=5 % ^^M
  \endlinechar=13 %
  \catcode35=6 % #
  \catcode39=12 % '
  \catcode40=12 % (
  \catcode41=12 % )
  \catcode44=12 % ,
  \catcode45=12 % -
  \catcode46=12 % .
  \catcode47=12 % /
  \catcode58=12 % :
  \catcode64=11 % @
  \catcode91=12 % [
  \catcode93=12 % ]
  \catcode123=1 % {
  \catcode125=2 % }
  \expandafter\ifx\csname ProvidesPackage\endcsname\relax
    \def\x#1#2#3[#4]{\endgroup
      \immediate\write-1{Package: #3 #4}%
      \xdef#1{#4}%
    }%
  \else
    \def\x#1#2[#3]{\endgroup
      #2[{#3}]%
      \ifx#1\@undefined
        \xdef#1{#3}%
      \fi
      \ifx#1\relax
        \xdef#1{#3}%
      \fi
    }%
  \fi
\expandafter\x\csname ver@hologo.sty\endcsname
\ProvidesPackage{hologo}%
  [2016/05/12 v1.11 A logo collection with bookmark support (HO)]%
%    \end{macrocode}
%
%    \begin{macrocode}
\begingroup\catcode61\catcode48\catcode32=10\relax%
  \catcode13=5 % ^^M
  \endlinechar=13 %
  \catcode123=1 % {
  \catcode125=2 % }
  \catcode64=11 % @
  \def\x{\endgroup
    \expandafter\edef\csname HOLOGO@AtEnd\endcsname{%
      \endlinechar=\the\endlinechar\relax
      \catcode13=\the\catcode13\relax
      \catcode32=\the\catcode32\relax
      \catcode35=\the\catcode35\relax
      \catcode61=\the\catcode61\relax
      \catcode64=\the\catcode64\relax
      \catcode123=\the\catcode123\relax
      \catcode125=\the\catcode125\relax
    }%
  }%
\x\catcode61\catcode48\catcode32=10\relax%
\catcode13=5 % ^^M
\endlinechar=13 %
\catcode35=6 % #
\catcode64=11 % @
\catcode123=1 % {
\catcode125=2 % }
\def\TMP@EnsureCode#1#2{%
  \edef\HOLOGO@AtEnd{%
    \HOLOGO@AtEnd
    \catcode#1=\the\catcode#1\relax
  }%
  \catcode#1=#2\relax
}
\TMP@EnsureCode{10}{12}% ^^J
\TMP@EnsureCode{33}{12}% !
\TMP@EnsureCode{34}{12}% "
\TMP@EnsureCode{36}{3}% $
\TMP@EnsureCode{38}{4}% &
\TMP@EnsureCode{39}{12}% '
\TMP@EnsureCode{40}{12}% (
\TMP@EnsureCode{41}{12}% )
\TMP@EnsureCode{42}{12}% *
\TMP@EnsureCode{43}{12}% +
\TMP@EnsureCode{44}{12}% ,
\TMP@EnsureCode{45}{12}% -
\TMP@EnsureCode{46}{12}% .
\TMP@EnsureCode{47}{12}% /
\TMP@EnsureCode{58}{12}% :
\TMP@EnsureCode{59}{12}% ;
\TMP@EnsureCode{60}{12}% <
\TMP@EnsureCode{62}{12}% >
\TMP@EnsureCode{63}{12}% ?
\TMP@EnsureCode{91}{12}% [
\TMP@EnsureCode{93}{12}% ]
\TMP@EnsureCode{94}{7}% ^ (superscript)
\TMP@EnsureCode{95}{8}% _ (subscript)
\TMP@EnsureCode{96}{12}% `
\TMP@EnsureCode{124}{12}% |
\edef\HOLOGO@AtEnd{%
  \HOLOGO@AtEnd
  \escapechar\the\escapechar\relax
  \noexpand\endinput
}
\escapechar=92 %
%    \end{macrocode}
%
% \subsection{Logo list}
%
%    \begin{macro}{\hologoList}
%    \begin{macrocode}
\def\hologoList{%
  \hologoEntry{(La)TeX}{}{2011/10/01}%
  \hologoEntry{AmSLaTeX}{}{2010/04/16}%
  \hologoEntry{AmSTeX}{}{2010/04/16}%
  \hologoEntry{biber}{}{2011/10/01}%
  \hologoEntry{BibTeX}{}{2011/10/01}%
  \hologoEntry{BibTeX}{sf}{2011/10/01}%
  \hologoEntry{BibTeX}{sc}{2011/10/01}%
  \hologoEntry{BibTeX8}{}{2011/11/22}%
  \hologoEntry{ConTeXt}{}{2011/03/25}%
  \hologoEntry{ConTeXt}{narrow}{2011/03/25}%
  \hologoEntry{ConTeXt}{simple}{2011/03/25}%
  \hologoEntry{emTeX}{}{2010/04/26}%
  \hologoEntry{eTeX}{}{2010/04/08}%
  \hologoEntry{ExTeX}{}{2011/10/01}%
  \hologoEntry{HanTheThanh}{}{2011/11/29}%
  \hologoEntry{iniTeX}{}{2011/10/01}%
  \hologoEntry{KOMAScript}{}{2011/10/01}%
  \hologoEntry{La}{}{2010/05/08}%
  \hologoEntry{LaTeX}{}{2010/04/08}%
  \hologoEntry{LaTeX2e}{}{2010/04/08}%
  \hologoEntry{LaTeX3}{}{2010/04/24}%
  \hologoEntry{LaTeXe}{}{2010/04/08}%
  \hologoEntry{LaTeXML}{}{2011/11/22}%
  \hologoEntry{LaTeXTeX}{}{2011/10/01}%
  \hologoEntry{LuaLaTeX}{}{2010/04/08}%
  \hologoEntry{LuaTeX}{}{2010/04/08}%
  \hologoEntry{LyX}{}{2011/10/01}%
  \hologoEntry{METAFONT}{}{2011/10/01}%
  \hologoEntry{MetaFun}{}{2011/10/01}%
  \hologoEntry{METAPOST}{}{2011/10/01}%
  \hologoEntry{MetaPost}{}{2011/10/01}%
  \hologoEntry{MiKTeX}{}{2011/10/01}%
  \hologoEntry{NTS}{}{2011/10/01}%
  \hologoEntry{OzMF}{}{2011/10/01}%
  \hologoEntry{OzMP}{}{2011/10/01}%
  \hologoEntry{OzTeX}{}{2011/10/01}%
  \hologoEntry{OzTtH}{}{2011/10/01}%
  \hologoEntry{PCTeX}{}{2011/10/01}%
  \hologoEntry{pdfTeX}{}{2011/10/01}%
  \hologoEntry{pdfLaTeX}{}{2011/10/01}%
  \hologoEntry{PiC}{}{2011/10/01}%
  \hologoEntry{PiCTeX}{}{2011/10/01}%
  \hologoEntry{plainTeX}{}{2010/04/08}%
  \hologoEntry{plainTeX}{space}{2010/04/16}%
  \hologoEntry{plainTeX}{hyphen}{2010/04/16}%
  \hologoEntry{plainTeX}{runtogether}{2010/04/16}%
  \hologoEntry{SageTeX}{}{2011/11/22}%
  \hologoEntry{SLiTeX}{}{2011/10/01}%
  \hologoEntry{SLiTeX}{lift}{2011/10/01}%
  \hologoEntry{SLiTeX}{narrow}{2011/10/01}%
  \hologoEntry{SLiTeX}{simple}{2011/10/01}%
  \hologoEntry{SliTeX}{}{2011/10/01}%
  \hologoEntry{SliTeX}{narrow}{2011/10/01}%
  \hologoEntry{SliTeX}{simple}{2011/10/01}%
  \hologoEntry{SliTeX}{lift}{2011/10/01}%
  \hologoEntry{teTeX}{}{2011/10/01}%
  \hologoEntry{TeX}{}{2010/04/08}%
  \hologoEntry{TeX4ht}{}{2011/11/22}%
  \hologoEntry{TTH}{}{2011/11/22}%
  \hologoEntry{virTeX}{}{2011/10/01}%
  \hologoEntry{VTeX}{}{2010/04/24}%
  \hologoEntry{Xe}{}{2010/04/08}%
  \hologoEntry{XeLaTeX}{}{2010/04/08}%
  \hologoEntry{XeTeX}{}{2010/04/08}%
}
%    \end{macrocode}
%    \end{macro}
%
% \subsection{Load resources}
%
%    \begin{macrocode}
\begingroup\expandafter\expandafter\expandafter\endgroup
\expandafter\ifx\csname RequirePackage\endcsname\relax
  \def\TMP@RequirePackage#1[#2]{%
    \begingroup\expandafter\expandafter\expandafter\endgroup
    \expandafter\ifx\csname ver@#1.sty\endcsname\relax
      \input #1.sty\relax
    \fi
  }%
  \TMP@RequirePackage{ltxcmds}[2011/02/04]%
  \TMP@RequirePackage{infwarerr}[2010/04/08]%
  \TMP@RequirePackage{kvsetkeys}[2010/03/01]%
  \TMP@RequirePackage{kvdefinekeys}[2010/03/01]%
  \TMP@RequirePackage{pdftexcmds}[2010/04/01]%
  \TMP@RequirePackage{ifpdf}[2010/01/28]%
  \TMP@RequirePackage{ifluatex}[2010/03/01]%
  \ltx@IfUndefined{newif}{%
    \expandafter\let\csname newif\endcsname\ltx@newif
  }{}%
  \TMP@RequirePackage{ifxetex}[2009/01/23]%
  \TMP@RequirePackage{ifvtex}[2010/03/01]%
\else
  \RequirePackage{ltxcmds}[2011/02/04]%
  \RequirePackage{infwarerr}[2010/04/08]%
  \RequirePackage{kvsetkeys}[2010/03/01]%
  \RequirePackage{kvdefinekeys}[2010/03/01]%
  \RequirePackage{pdftexcmds}[2010/04/01]%
  \RequirePackage{ifpdf}[2010/01/28]%
  \RequirePackage{ifluatex}[2010/03/01]%
  \RequirePackage{ifxetex}[2009/01/23]%
  \RequirePackage{ifvtex}[2010/03/01]%
\fi
%    \end{macrocode}
%
%    \begin{macro}{\HOLOGO@IfDefined}
%    \begin{macrocode}
\def\HOLOGO@IfExists#1{%
  \ifx\@undefined#1%
    \expandafter\ltx@secondoftwo
  \else
    \ifx\relax#1%
      \expandafter\ltx@secondoftwo
    \else
      \expandafter\expandafter\expandafter\ltx@firstoftwo
    \fi
  \fi
}
%    \end{macrocode}
%    \end{macro}
%
% \subsection{Setup macros}
%
%    \begin{macro}{\hologoSetup}
%    \begin{macrocode}
\def\hologoSetup{%
  \let\HOLOGO@name\relax
  \HOLOGO@Setup
}
%    \end{macrocode}
%    \end{macro}
%
%    \begin{macro}{\hologoLogoSetup}
%    \begin{macrocode}
\def\hologoLogoSetup#1{%
  \edef\HOLOGO@name{#1}%
  \ltx@IfUndefined{HoLogo@\HOLOGO@name}{%
    \@PackageError{hologo}{%
      Unknown logo `\HOLOGO@name'%
    }\@ehc
    \ltx@gobble
  }{%
    \HOLOGO@Setup
  }%
}
%    \end{macrocode}
%    \end{macro}
%
%    \begin{macro}{\HOLOGO@Setup}
%    \begin{macrocode}
\def\HOLOGO@Setup{%
  \kvsetkeys{HoLogo}%
}
%    \end{macrocode}
%    \end{macro}
%
% \subsection{Options}
%
%    \begin{macro}{\HOLOGO@DeclareBoolOption}
%    \begin{macrocode}
\def\HOLOGO@DeclareBoolOption#1{%
  \expandafter\chardef\csname HOLOGOOPT@#1\endcsname\ltx@zero
  \kv@define@key{HoLogo}{#1}[true]{%
    \def\HOLOGO@temp{##1}%
    \ifx\HOLOGO@temp\HOLOGO@true
      \ifx\HOLOGO@name\relax
        \expandafter\chardef\csname HOLOGOOPT@#1\endcsname=\ltx@one
      \else
        \expandafter\chardef\csname
        HoLogoOpt@#1@\HOLOGO@name\endcsname\ltx@one
      \fi
      \HOLOGO@SetBreakAll{#1}%
    \else
      \ifx\HOLOGO@temp\HOLOGO@false
        \ifx\HOLOGO@name\relax
          \expandafter\chardef\csname HOLOGOOPT@#1\endcsname=\ltx@zero
        \else
          \expandafter\chardef\csname
          HoLogoOpt@#1@\HOLOGO@name\endcsname=\ltx@zero
        \fi
        \HOLOGO@SetBreakAll{#1}%
      \else
        \@PackageError{hologo}{%
          Unknown value `##1' for boolean option `#1'.\MessageBreak
          Known values are `true' and `false'%
        }\@ehc
      \fi
    \fi
  }%
}
%    \end{macrocode}
%    \end{macro}
%
%    \begin{macro}{\HOLOGO@SetBreakAll}
%    \begin{macrocode}
\def\HOLOGO@SetBreakAll#1{%
  \def\HOLOGO@temp{#1}%
  \ifx\HOLOGO@temp\HOLOGO@break
    \ifx\HOLOGO@name\relax
      \chardef\HOLOGOOPT@hyphenbreak=\HOLOGOOPT@break
      \chardef\HOLOGOOPT@spacebreak=\HOLOGOOPT@break
      \chardef\HOLOGOOPT@discretionarybreak=\HOLOGOOPT@break
    \else
      \expandafter\chardef
         \csname HoLogoOpt@hyphenbreak@\HOLOGO@name\endcsname=%
         \csname HoLogoOpt@break@\HOLOGO@name\endcsname
      \expandafter\chardef
         \csname HoLogoOpt@spacebreak@\HOLOGO@name\endcsname=%
         \csname HoLogoOpt@break@\HOLOGO@name\endcsname
      \expandafter\chardef
         \csname HoLogoOpt@discretionarybreak@\HOLOGO@name
             \endcsname=%
         \csname HoLogoOpt@break@\HOLOGO@name\endcsname
    \fi
  \fi
}
%    \end{macrocode}
%    \end{macro}
%
%    \begin{macro}{\HOLOGO@true}
%    \begin{macrocode}
\def\HOLOGO@true{true}
%    \end{macrocode}
%    \end{macro}
%    \begin{macro}{\HOLOGO@false}
%    \begin{macrocode}
\def\HOLOGO@false{false}
%    \end{macrocode}
%    \end{macro}
%    \begin{macro}{\HOLOGO@break}
%    \begin{macrocode}
\def\HOLOGO@break{break}
%    \end{macrocode}
%    \end{macro}
%
%    \begin{macrocode}
\HOLOGO@DeclareBoolOption{break}
\HOLOGO@DeclareBoolOption{hyphenbreak}
\HOLOGO@DeclareBoolOption{spacebreak}
\HOLOGO@DeclareBoolOption{discretionarybreak}
%    \end{macrocode}
%
%    \begin{macrocode}
\kv@define@key{HoLogo}{variant}{%
  \ifx\HOLOGO@name\relax
    \@PackageError{hologo}{%
      Option `variant' is not available in \string\hologoSetup,%
      \MessageBreak
      Use \string\hologoLogoSetup\space instead%
    }\@ehc
  \else
    \edef\HOLOGO@temp{#1}%
    \ifx\HOLOGO@temp\ltx@empty
      \expandafter
      \let\csname HoLogoOpt@variant@\HOLOGO@name\endcsname\@undefined
    \else
      \ltx@IfUndefined{HoLogo@\HOLOGO@name @\HOLOGO@temp}{%
        \@PackageError{hologo}{%
          Unknown variant `\HOLOGO@temp' of logo `\HOLOGO@name'%
        }\@ehc
      }{%
        \expandafter
        \let\csname HoLogoOpt@variant@\HOLOGO@name\endcsname
            \HOLOGO@temp
      }%
    \fi
  \fi
}
%    \end{macrocode}
%
%    \begin{macro}{\HOLOGO@Variant}
%    \begin{macrocode}
\def\HOLOGO@Variant#1{%
  #1%
  \ltx@ifundefined{HoLogoOpt@variant@#1}{%
  }{%
    @\csname HoLogoOpt@variant@#1\endcsname
  }%
}
%    \end{macrocode}
%    \end{macro}
%
% \subsection{Break/no-break support}
%
%    \begin{macro}{\HOLOGO@space}
%    \begin{macrocode}
\def\HOLOGO@space{%
  \ltx@ifundefined{HoLogoOpt@spacebreak@\HOLOGO@name}{%
    \ltx@ifundefined{HoLogoOpt@break@\HOLOGO@name}{%
      \chardef\HOLOGO@temp=\HOLOGOOPT@spacebreak
    }{%
      \chardef\HOLOGO@temp=%
        \csname HoLogoOpt@break@\HOLOGO@name\endcsname
    }%
  }{%
    \chardef\HOLOGO@temp=%
      \csname HoLogoOpt@spacebreak@\HOLOGO@name\endcsname
  }%
  \ifcase\HOLOGO@temp
    \penalty10000 %
  \fi
  \ltx@space
}
%    \end{macrocode}
%    \end{macro}
%
%    \begin{macro}{\HOLOGO@hyphen}
%    \begin{macrocode}
\def\HOLOGO@hyphen{%
  \ltx@ifundefined{HoLogoOpt@hyphenbreak@\HOLOGO@name}{%
    \ltx@ifundefined{HoLogoOpt@break@\HOLOGO@name}{%
      \chardef\HOLOGO@temp=\HOLOGOOPT@hyphenbreak
    }{%
      \chardef\HOLOGO@temp=%
        \csname HoLogoOpt@break@\HOLOGO@name\endcsname
    }%
  }{%
    \chardef\HOLOGO@temp=%
      \csname HoLogoOpt@hyphenbreak@\HOLOGO@name\endcsname
  }%
  \ifcase\HOLOGO@temp
    \ltx@mbox{-}%
  \else
    -%
  \fi
}
%    \end{macrocode}
%    \end{macro}
%
%    \begin{macro}{\HOLOGO@discretionary}
%    \begin{macrocode}
\def\HOLOGO@discretionary{%
  \ltx@ifundefined{HoLogoOpt@discretionarybreak@\HOLOGO@name}{%
    \ltx@ifundefined{HoLogoOpt@break@\HOLOGO@name}{%
      \chardef\HOLOGO@temp=\HOLOGOOPT@discretionarybreak
    }{%
      \chardef\HOLOGO@temp=%
        \csname HoLogoOpt@break@\HOLOGO@name\endcsname
    }%
  }{%
    \chardef\HOLOGO@temp=%
      \csname HoLogoOpt@discretionarybreak@\HOLOGO@name\endcsname
  }%
  \ifcase\HOLOGO@temp
  \else
    \-%
  \fi
}
%    \end{macrocode}
%    \end{macro}
%
%    \begin{macro}{\HOLOGO@mbox}
%    \begin{macrocode}
\def\HOLOGO@mbox#1{%
  \ltx@ifundefined{HoLogoOpt@break@\HOLOGO@name}{%
    \chardef\HOLOGO@temp=\HOLOGOOPT@hyphenbreak
  }{%
    \chardef\HOLOGO@temp=%
      \csname HoLogoOpt@break@\HOLOGO@name\endcsname
  }%
  \ifcase\HOLOGO@temp
    \ltx@mbox{#1}%
  \else
    #1%
  \fi
}
%    \end{macrocode}
%    \end{macro}
%
% \subsection{Font support}
%
%    \begin{macro}{\HoLogoFont@font}
%    \begin{tabular}{@{}ll@{}}
%    |#1|:& logo name\\
%    |#2|:& font short name\\
%    |#3|:& text
%    \end{tabular}
%    \begin{macrocode}
\def\HoLogoFont@font#1#2#3{%
  \begingroup
    \ltx@IfUndefined{HoLogoFont@logo@#1.#2}{%
      \ltx@IfUndefined{HoLogoFont@font@#2}{%
        \@PackageWarning{hologo}{%
          Missing font `#2' for logo `#1'%
        }%
        #3%
      }{%
        \csname HoLogoFont@font@#2\endcsname{#3}%
      }%
    }{%
      \csname HoLogoFont@logo@#1.#2\endcsname{#3}%
    }%
  \endgroup
}
%    \end{macrocode}
%    \end{macro}
%
%    \begin{macro}{\HoLogoFont@Def}
%    \begin{macrocode}
\def\HoLogoFont@Def#1{%
  \expandafter\def\csname HoLogoFont@font@#1\endcsname
}
%    \end{macrocode}
%    \end{macro}
%    \begin{macro}{\HoLogoFont@LogoDef}
%    \begin{macrocode}
\def\HoLogoFont@LogoDef#1#2{%
  \expandafter\def\csname HoLogoFont@logo@#1.#2\endcsname
}
%    \end{macrocode}
%    \end{macro}
%
% \subsubsection{Font defaults}
%
%    \begin{macro}{\HoLogoFont@font@general}
%    \begin{macrocode}
\HoLogoFont@Def{general}{}%
%    \end{macrocode}
%    \end{macro}
%
%    \begin{macro}{\HoLogoFont@font@rm}
%    \begin{macrocode}
\ltx@IfUndefined{rmfamily}{%
  \ltx@IfUndefined{rm}{%
  }{%
    \HoLogoFont@Def{rm}{\rm}%
  }%
}{%
  \HoLogoFont@Def{rm}{\rmfamily}%
}
%    \end{macrocode}
%    \end{macro}
%
%    \begin{macro}{\HoLogoFont@font@sf}
%    \begin{macrocode}
\ltx@IfUndefined{sffamily}{%
  \ltx@IfUndefined{sf}{%
  }{%
    \HoLogoFont@Def{sf}{\sf}%
  }%
}{%
  \HoLogoFont@Def{sf}{\sffamily}%
}
%    \end{macrocode}
%    \end{macro}
%
%    \begin{macro}{\HoLogoFont@font@bibsf}
%    In case of \hologo{plainTeX} the original small caps
%    variant is used as default. In \hologo{LaTeX}
%    the definition of package \xpackage{dtklogos} \cite{dtklogos}
%    is used.
%\begin{quote}
%\begin{verbatim}
%\DeclareRobustCommand{\BibTeX}{%
%  B%
%  \kern-.05em%
%  \hbox{%
%    $\m@th$% %% force math size calculations
%    \csname S@\f@size\endcsname
%    \fontsize\sf@size\z@
%    \math@fontsfalse
%    \selectfont
%    I%
%    \kern-.025em%
%    B
%  }%
%  \kern-.08em%
%  \-%
%  \TeX
%}
%\end{verbatim}
%\end{quote}
%    \begin{macrocode}
\ltx@IfUndefined{selectfont}{%
  \ltx@IfUndefined{tensc}{%
    \font\tensc=cmcsc10\relax
  }{}%
  \HoLogoFont@Def{bibsf}{\tensc}%
}{%
  \HoLogoFont@Def{bibsf}{%
    $\mathsurround=0pt$%
    \csname S@\f@size\endcsname
    \fontsize\sf@size{0pt}%
    \math@fontsfalse
    \selectfont
  }%
}
%    \end{macrocode}
%    \end{macro}
%
%    \begin{macro}{\HoLogoFont@font@sc}
%    \begin{macrocode}
\ltx@IfUndefined{scshape}{%
  \ltx@IfUndefined{tensc}{%
    \font\tensc=cmcsc10\relax
  }{}%
  \HoLogoFont@Def{sc}{\tensc}%
}{%
  \HoLogoFont@Def{sc}{\scshape}%
}
%    \end{macrocode}
%    \end{macro}
%
%    \begin{macro}{\HoLogoFont@font@sy}
%    \begin{macrocode}
\ltx@IfUndefined{usefont}{%
  \ltx@IfUndefined{tensy}{%
  }{%
    \HoLogoFont@Def{sy}{\tensy}%
  }%
}{%
  \HoLogoFont@Def{sy}{%
    \usefont{OMS}{cmsy}{m}{n}%
  }%
}
%    \end{macrocode}
%    \end{macro}
%
%    \begin{macro}{\HoLogoFont@font@logo}
%    \begin{macrocode}
\begingroup
  \def\x{LaTeX2e}%
\expandafter\endgroup
\ifx\fmtname\x
  \ltx@IfUndefined{logofamily}{%
    \DeclareRobustCommand\logofamily{%
      \not@math@alphabet\logofamily\relax
      \fontencoding{U}%
      \fontfamily{logo}%
      \selectfont
    }%
  }{}%
  \ltx@IfUndefined{logofamily}{%
  }{%
    \HoLogoFont@Def{logo}{\logofamily}%
  }%
\else
  \ltx@IfUndefined{tenlogo}{%
    \font\tenlogo=logo10\relax
  }{}%
  \HoLogoFont@Def{logo}{\tenlogo}%
\fi
%    \end{macrocode}
%    \end{macro}
%
% \subsubsection{Font setup}
%
%    \begin{macro}{\hologoFontSetup}
%    \begin{macrocode}
\def\hologoFontSetup{%
  \let\HOLOGO@name\relax
  \HOLOGO@FontSetup
}
%    \end{macrocode}
%    \end{macro}
%
%    \begin{macro}{\hologoLogoFontSetup}
%    \begin{macrocode}
\def\hologoLogoFontSetup#1{%
  \edef\HOLOGO@name{#1}%
  \ltx@IfUndefined{HoLogo@\HOLOGO@name}{%
    \@PackageError{hologo}{%
      Unknown logo `\HOLOGO@name'%
    }\@ehc
    \ltx@gobble
  }{%
    \HOLOGO@FontSetup
  }%
}
%    \end{macrocode}
%    \end{macro}
%
%    \begin{macro}{\HOLOGO@FontSetup}
%    \begin{macrocode}
\def\HOLOGO@FontSetup{%
  \kvsetkeys{HoLogoFont}%
}
%    \end{macrocode}
%    \end{macro}
%
%    \begin{macrocode}
\def\HOLOGO@temp#1{%
  \kv@define@key{HoLogoFont}{#1}{%
    \ifx\HOLOGO@name\relax
      \HoLogoFont@Def{#1}{##1}%
    \else
      \HoLogoFont@LogoDef\HOLOGO@name{#1}{##1}%
    \fi
  }%
}
\HOLOGO@temp{general}
\HOLOGO@temp{sf}
%    \end{macrocode}
%
% \subsection{Generic logo commands}
%
%    \begin{macrocode}
\HOLOGO@IfExists\hologo{%
  \@PackageError{hologo}{%
    \string\hologo\ltx@space is already defined.\MessageBreak
    Package loading is aborted%
  }\@ehc
  \HOLOGO@AtEnd
}%
\HOLOGO@IfExists\hologoRobust{%
  \@PackageError{hologo}{%
    \string\hologoRobust\ltx@space is already defined.\MessageBreak
    Package loading is aborted%
  }\@ehc
  \HOLOGO@AtEnd
}%
%    \end{macrocode}
%
% \subsubsection{\cs{hologo} and friends}
%
%    \begin{macrocode}
\ifluatex
  \expandafter\ltx@firstofone
\else
  \expandafter\ltx@gobble
\fi
{%
  \ltx@IfUndefined{ifincsname}{%
    \ifnum\luatexversion<36 %
      \expandafter\ltx@gobble
    \else
      \expandafter\ltx@firstofone
    \fi
    {%
      \begingroup
        \ifcase0%
            \directlua{%
              if tex.enableprimitives then %
                tex.enableprimitives('HOLOGO@', {'ifincsname'})%
              else %
                tex.print('1')%
              end%
            }%
            \ifx\HOLOGO@ifincsname\@undefined 1\fi%
            \relax
          \expandafter\ltx@firstofone
        \else
          \endgroup
          \expandafter\ltx@gobble
        \fi
        {%
          \global\let\ifincsname\HOLOGO@ifincsname
        }%
      \HOLOGO@temp
    }%
  }{}%
}
%    \end{macrocode}
%    \begin{macrocode}
\ltx@IfUndefined{ifincsname}{%
  \catcode`$=14 %
}{%
  \catcode`$=9 %
}
%    \end{macrocode}
%
%    \begin{macro}{\hologo}
%    \begin{macrocode}
\def\hologo#1{%
$ \ifincsname
$   \ltx@ifundefined{HoLogoCs@\HOLOGO@Variant{#1}}{%
$     #1%
$   }{%
$     \csname HoLogoCs@\HOLOGO@Variant{#1}\endcsname\ltx@firstoftwo
$   }%
$ \else
    \HOLOGO@IfExists\texorpdfstring\texorpdfstring\ltx@firstoftwo
    {%
      \hologoRobust{#1}%
    }{%
      \ltx@ifundefined{HoLogoBkm@\HOLOGO@Variant{#1}}{%
        \ltx@ifundefined{HoLogo@#1}{?#1?}{#1}%
      }{%
        \csname HoLogoBkm@\HOLOGO@Variant{#1}\endcsname
        \ltx@firstoftwo
      }%
    }%
$ \fi
}
%    \end{macrocode}
%    \end{macro}
%    \begin{macro}{\Hologo}
%    \begin{macrocode}
\def\Hologo#1{%
$ \ifincsname
$   \ltx@ifundefined{HoLogoCs@\HOLOGO@Variant{#1}}{%
$     #1%
$   }{%
$     \csname HoLogoCs@\HOLOGO@Variant{#1}\endcsname\ltx@secondoftwo
$   }%
$ \else
    \HOLOGO@IfExists\texorpdfstring\texorpdfstring\ltx@firstoftwo
    {%
      \HologoRobust{#1}%
    }{%
      \ltx@ifundefined{HoLogoBkm@\HOLOGO@Variant{#1}}{%
        \ltx@ifundefined{HoLogo@#1}{?#1?}{#1}%
      }{%
        \csname HoLogoBkm@\HOLOGO@Variant{#1}\endcsname
        \ltx@secondoftwo
      }%
    }%
$ \fi
}
%    \end{macrocode}
%    \end{macro}
%
%    \begin{macro}{\hologoVariant}
%    \begin{macrocode}
\def\hologoVariant#1#2{%
  \ifx\relax#2\relax
    \hologo{#1}%
  \else
$   \ifincsname
$     \ltx@ifundefined{HoLogoCs@#1@#2}{%
$       #1%
$     }{%
$       \csname HoLogoCs@#1@#2\endcsname\ltx@firstoftwo
$     }%
$   \else
      \HOLOGO@IfExists\texorpdfstring\texorpdfstring\ltx@firstoftwo
      {%
        \hologoVariantRobust{#1}{#2}%
      }{%
        \ltx@ifundefined{HoLogoBkm@#1@#2}{%
          \ltx@ifundefined{HoLogo@#1}{?#1?}{#1}%
        }{%
          \csname HoLogoBkm@#1@#2\endcsname
          \ltx@firstoftwo
        }%
      }%
$   \fi
  \fi
}
%    \end{macrocode}
%    \end{macro}
%    \begin{macro}{\HologoVariant}
%    \begin{macrocode}
\def\HologoVariant#1#2{%
  \ifx\relax#2\relax
    \Hologo{#1}%
  \else
$   \ifincsname
$     \ltx@ifundefined{HoLogoCs@#1@#2}{%
$       #1%
$     }{%
$       \csname HoLogoCs@#1@#2\endcsname\ltx@secondoftwo
$     }%
$   \else
      \HOLOGO@IfExists\texorpdfstring\texorpdfstring\ltx@firstoftwo
      {%
        \HologoVariantRobust{#1}{#2}%
      }{%
        \ltx@ifundefined{HoLogoBkm@#1@#2}{%
          \ltx@ifundefined{HoLogo@#1}{?#1?}{#1}%
        }{%
          \csname HoLogoBkm@#1@#2\endcsname
          \ltx@secondoftwo
        }%
      }%
$   \fi
  \fi
}
%    \end{macrocode}
%    \end{macro}
%
%    \begin{macrocode}
\catcode`\$=3 %
%    \end{macrocode}
%
% \subsubsection{\cs{hologoRobust} and friends}
%
%    \begin{macro}{\hologoRobust}
%    \begin{macrocode}
\ltx@IfUndefined{protected}{%
  \ltx@IfUndefined{DeclareRobustCommand}{%
    \def\hologoRobust#1%
  }{%
    \DeclareRobustCommand*\hologoRobust[1]%
  }%
}{%
  \protected\def\hologoRobust#1%
}%
{%
  \edef\HOLOGO@name{#1}%
  \ltx@IfUndefined{HoLogo@\HOLOGO@Variant\HOLOGO@name}{%
    \@PackageError{hologo}{%
      Unknown logo `\HOLOGO@name'%
    }\@ehc
    ?\HOLOGO@name?%
  }{%
    \ltx@IfUndefined{ver@tex4ht.sty}{%
      \HoLogoFont@font\HOLOGO@name{general}{%
        \csname HoLogo@\HOLOGO@Variant\HOLOGO@name\endcsname
        \ltx@firstoftwo
      }%
    }{%
      \ltx@IfUndefined{HoLogoHtml@\HOLOGO@Variant\HOLOGO@name}{%
        \HOLOGO@name
      }{%
        \csname HoLogoHtml@\HOLOGO@Variant\HOLOGO@name\endcsname
        \ltx@firstoftwo
      }%
    }%
  }%
}
%    \end{macrocode}
%    \end{macro}
%    \begin{macro}{\HologoRobust}
%    \begin{macrocode}
\ltx@IfUndefined{protected}{%
  \ltx@IfUndefined{DeclareRobustCommand}{%
    \def\HologoRobust#1%
  }{%
    \DeclareRobustCommand*\HologoRobust[1]%
  }%
}{%
  \protected\def\HologoRobust#1%
}%
{%
  \edef\HOLOGO@name{#1}%
  \ltx@IfUndefined{HoLogo@\HOLOGO@Variant\HOLOGO@name}{%
    \@PackageError{hologo}{%
      Unknown logo `\HOLOGO@name'%
    }\@ehc
    ?\HOLOGO@name?%
  }{%
    \ltx@IfUndefined{ver@tex4ht.sty}{%
      \HoLogoFont@font\HOLOGO@name{general}{%
        \csname HoLogo@\HOLOGO@Variant\HOLOGO@name\endcsname
        \ltx@secondoftwo
      }%
    }{%
      \ltx@IfUndefined{HoLogoHtml@\HOLOGO@Variant\HOLOGO@name}{%
        \expandafter\HOLOGO@Uppercase\HOLOGO@name
      }{%
        \csname HoLogoHtml@\HOLOGO@Variant\HOLOGO@name\endcsname
        \ltx@secondoftwo
      }%
    }%
  }%
}
%    \end{macrocode}
%    \end{macro}
%    \begin{macro}{\hologoVariantRobust}
%    \begin{macrocode}
\ltx@IfUndefined{protected}{%
  \ltx@IfUndefined{DeclareRobustCommand}{%
    \def\hologoVariantRobust#1#2%
  }{%
    \DeclareRobustCommand*\hologoVariantRobust[2]%
  }%
}{%
  \protected\def\hologoVariantRobust#1#2%
}%
{%
  \begingroup
    \hologoLogoSetup{#1}{variant={#2}}%
    \hologoRobust{#1}%
  \endgroup
}
%    \end{macrocode}
%    \end{macro}
%    \begin{macro}{\HologoVariantRobust}
%    \begin{macrocode}
\ltx@IfUndefined{protected}{%
  \ltx@IfUndefined{DeclareRobustCommand}{%
    \def\HologoVariantRobust#1#2%
  }{%
    \DeclareRobustCommand*\HologoVariantRobust[2]%
  }%
}{%
  \protected\def\HologoVariantRobust#1#2%
}%
{%
  \begingroup
    \hologoLogoSetup{#1}{variant={#2}}%
    \HologoRobust{#1}%
  \endgroup
}
%    \end{macrocode}
%    \end{macro}
%
%    \begin{macro}{\hologorobust}
%    Macro \cs{hologorobust} is only defined for compatibility.
%    Its use is deprecated.
%    \begin{macrocode}
\def\hologorobust{\hologoRobust}
%    \end{macrocode}
%    \end{macro}
%
% \subsection{Helpers}
%
%    \begin{macro}{\HOLOGO@Uppercase}
%    Macro \cs{HOLOGO@Uppercase} is restricted to \cs{uppercase},
%    because \hologo{plainTeX} or \hologo{iniTeX} do not provide
%    \cs{MakeUppercase}.
%    \begin{macrocode}
\def\HOLOGO@Uppercase#1{\uppercase{#1}}
%    \end{macrocode}
%    \end{macro}
%
%    \begin{macro}{\HOLOGO@PdfdocUnicode}
%    \begin{macrocode}
\def\HOLOGO@PdfdocUnicode{%
  \ifx\ifHy@unicode\iftrue
    \expandafter\ltx@secondoftwo
  \else
    \expandafter\ltx@firstoftwo
  \fi
}
%    \end{macrocode}
%    \end{macro}
%
%    \begin{macro}{\HOLOGO@Math}
%    \begin{macrocode}
\def\HOLOGO@MathSetup{%
  \mathsurround0pt\relax
  \HOLOGO@IfExists\f@series{%
    \if b\expandafter\ltx@car\f@series x\@nil
      \csname boldmath\endcsname
   \fi
  }{}%
}
%    \end{macrocode}
%    \end{macro}
%
%    \begin{macro}{\HOLOGO@TempDimen}
%    \begin{macrocode}
\dimendef\HOLOGO@TempDimen=\ltx@zero
%    \end{macrocode}
%    \end{macro}
%    \begin{macro}{\HOLOGO@NegativeKerning}
%    \begin{macrocode}
\def\HOLOGO@NegativeKerning#1{%
  \begingroup
    \HOLOGO@TempDimen=0pt\relax
    \comma@parse@normalized{#1}{%
      \ifdim\HOLOGO@TempDimen=0pt %
        \expandafter\HOLOGO@@NegativeKerning\comma@entry
      \fi
      \ltx@gobble
    }%
    \ifdim\HOLOGO@TempDimen<0pt %
      \kern\HOLOGO@TempDimen
    \fi
  \endgroup
}
%    \end{macrocode}
%    \end{macro}
%    \begin{macro}{\HOLOGO@@NegativeKerning}
%    \begin{macrocode}
\def\HOLOGO@@NegativeKerning#1#2{%
  \setbox\ltx@zero\hbox{#1#2}%
  \HOLOGO@TempDimen=\wd\ltx@zero
  \setbox\ltx@zero\hbox{#1\kern0pt#2}%
  \advance\HOLOGO@TempDimen by -\wd\ltx@zero
}
%    \end{macrocode}
%    \end{macro}
%
%    \begin{macro}{\HOLOGO@SpaceFactor}
%    \begin{macrocode}
\def\HOLOGO@SpaceFactor{%
  \spacefactor1000 %
}
%    \end{macrocode}
%    \end{macro}
%
%    \begin{macro}{\HOLOGO@Span}
%    \begin{macrocode}
\def\HOLOGO@Span#1#2{%
  \HCode{<span class="HoLogo-#1">}%
  #2%
  \HCode{</span>}%
}
%    \end{macrocode}
%    \end{macro}
%
% \subsubsection{Text subscript}
%
%    \begin{macro}{\HOLOGO@SubScript}%
%    \begin{macrocode}
\def\HOLOGO@SubScript#1{%
  \ltx@IfUndefined{textsubscript}{%
    \ltx@IfUndefined{text}{%
      \ltx@mbox{%
        \mathsurround=0pt\relax
        $%
          _{%
            \ltx@IfUndefined{sf@size}{%
              \mathrm{#1}%
            }{%
              \mbox{%
                \fontsize\sf@size{0pt}\selectfont
                #1%
              }%
            }%
          }%
        $%
      }%
    }{%
      \ltx@mbox{%
        \mathsurround=0pt\relax
        $_{\text{#1}}$%
      }%
    }%
  }{%
    \textsubscript{#1}%
  }%
}
%    \end{macrocode}
%    \end{macro}
%
% \subsection{\hologo{TeX} and friends}
%
% \subsubsection{\hologo{TeX}}
%
%    \begin{macro}{\HoLogo@TeX}
%    Source: \hologo{LaTeX} kernel.
%    \begin{macrocode}
\def\HoLogo@TeX#1{%
  T\kern-.1667em\lower.5ex\hbox{E}\kern-.125emX\HOLOGO@SpaceFactor
}
%    \end{macrocode}
%    \end{macro}
%    \begin{macro}{\HoLogoHtml@TeX}
%    \begin{macrocode}
\def\HoLogoHtml@TeX#1{%
  \HoLogoCss@TeX
  \HOLOGO@Span{TeX}{%
    T%
    \HOLOGO@Span{e}{%
      E%
    }%
    X%
  }%
}
%    \end{macrocode}
%    \end{macro}
%    \begin{macro}{\HoLogoCss@TeX}
%    \begin{macrocode}
\def\HoLogoCss@TeX{%
  \Css{%
    span.HoLogo-TeX span.HoLogo-e{%
      position:relative;%
      top:.5ex;%
      margin-left:-.1667em;%
      margin-right:-.125em;%
    }%
  }%
  \Css{%
    a span.HoLogo-TeX span.HoLogo-e{%
      text-decoration:none;%
    }%
  }%
  \global\let\HoLogoCss@TeX\relax
}
%    \end{macrocode}
%    \end{macro}
%
% \subsubsection{\hologo{plainTeX}}
%
%    \begin{macro}{\HoLogo@plainTeX@space}
%    Source: ``The \hologo{TeX}book''
%    \begin{macrocode}
\def\HoLogo@plainTeX@space#1{%
  \HOLOGO@mbox{#1{p}{P}lain}\HOLOGO@space\hologo{TeX}%
}
%    \end{macrocode}
%    \end{macro}
%    \begin{macro}{\HoLogoCs@plainTeX@space}
%    \begin{macrocode}
\def\HoLogoCs@plainTeX@space#1{#1{p}{P}lain TeX}%
%    \end{macrocode}
%    \end{macro}
%    \begin{macro}{\HoLogoBkm@plainTeX@space}
%    \begin{macrocode}
\def\HoLogoBkm@plainTeX@space#1{%
  #1{p}{P}lain \hologo{TeX}%
}
%    \end{macrocode}
%    \end{macro}
%    \begin{macro}{\HoLogoHtml@plainTeX@space}
%    \begin{macrocode}
\def\HoLogoHtml@plainTeX@space#1{%
  #1{p}{P}lain \hologo{TeX}%
}
%    \end{macrocode}
%    \end{macro}
%
%    \begin{macro}{\HoLogo@plainTeX@hyphen}
%    \begin{macrocode}
\def\HoLogo@plainTeX@hyphen#1{%
  \HOLOGO@mbox{#1{p}{P}lain}\HOLOGO@hyphen\hologo{TeX}%
}
%    \end{macrocode}
%    \end{macro}
%    \begin{macro}{\HoLogoCs@plainTeX@hyphen}
%    \begin{macrocode}
\def\HoLogoCs@plainTeX@hyphen#1{#1{p}{P}lain-TeX}
%    \end{macrocode}
%    \end{macro}
%    \begin{macro}{\HoLogoBkm@plainTeX@hyphen}
%    \begin{macrocode}
\def\HoLogoBkm@plainTeX@hyphen#1{%
  #1{p}{P}lain-\hologo{TeX}%
}
%    \end{macrocode}
%    \end{macro}
%    \begin{macro}{\HoLogoHtml@plainTeX@hyphen}
%    \begin{macrocode}
\def\HoLogoHtml@plainTeX@hyphen#1{%
  #1{p}{P}lain-\hologo{TeX}%
}
%    \end{macrocode}
%    \end{macro}
%
%    \begin{macro}{\HoLogo@plainTeX@runtogether}
%    \begin{macrocode}
\def\HoLogo@plainTeX@runtogether#1{%
  \HOLOGO@mbox{#1{p}{P}lain\hologo{TeX}}%
}
%    \end{macrocode}
%    \end{macro}
%    \begin{macro}{\HoLogoCs@plainTeX@runtogether}
%    \begin{macrocode}
\def\HoLogoCs@plainTeX@runtogether#1{#1{p}{P}lainTeX}
%    \end{macrocode}
%    \end{macro}
%    \begin{macro}{\HoLogoBkm@plainTeX@runtogether}
%    \begin{macrocode}
\def\HoLogoBkm@plainTeX@runtogether#1{%
  #1{p}{P}lain\hologo{TeX}%
}
%    \end{macrocode}
%    \end{macro}
%    \begin{macro}{\HoLogoHtml@plainTeX@runtogether}
%    \begin{macrocode}
\def\HoLogoHtml@plainTeX@runtogether#1{%
  #1{p}{P}lain\hologo{TeX}%
}
%    \end{macrocode}
%    \end{macro}
%
%    \begin{macro}{\HoLogo@plainTeX}
%    \begin{macrocode}
\def\HoLogo@plainTeX{\HoLogo@plainTeX@space}
%    \end{macrocode}
%    \end{macro}
%    \begin{macro}{\HoLogoCs@plainTeX}
%    \begin{macrocode}
\def\HoLogoCs@plainTeX{\HoLogoCs@plainTeX@space}
%    \end{macrocode}
%    \end{macro}
%    \begin{macro}{\HoLogoBkm@plainTeX}
%    \begin{macrocode}
\def\HoLogoBkm@plainTeX{\HoLogoBkm@plainTeX@space}
%    \end{macrocode}
%    \end{macro}
%    \begin{macro}{\HoLogoHtml@plainTeX}
%    \begin{macrocode}
\def\HoLogoHtml@plainTeX{\HoLogoHtml@plainTeX@space}
%    \end{macrocode}
%    \end{macro}
%
% \subsubsection{\hologo{LaTeX}}
%
%    Source: \hologo{LaTeX} kernel.
%\begin{quote}
%\begin{verbatim}
%\DeclareRobustCommand{\LaTeX}{%
%  L%
%  \kern-.36em%
%  {%
%    \sbox\z@ T%
%    \vbox to\ht\z@{%
%      \hbox{%
%        \check@mathfonts
%        \fontsize\sf@size\z@
%        \math@fontsfalse
%        \selectfont
%        A%
%      }%
%      \vss
%    }%
%  }%
%  \kern-.15em%
%  \TeX
%}
%\end{verbatim}
%\end{quote}
%
%    \begin{macro}{\HoLogo@La}
%    \begin{macrocode}
\def\HoLogo@La#1{%
  L%
  \kern-.36em%
  \begingroup
    \setbox\ltx@zero\hbox{T}%
    \vbox to\ht\ltx@zero{%
      \hbox{%
        \ltx@ifundefined{check@mathfonts}{%
          \csname sevenrm\endcsname
        }{%
          \check@mathfonts
          \fontsize\sf@size{0pt}%
          \math@fontsfalse\selectfont
        }%
        A%
      }%
      \vss
    }%
  \endgroup
}
%    \end{macrocode}
%    \end{macro}
%
%    \begin{macro}{\HoLogo@LaTeX}
%    Source: \hologo{LaTeX} kernel.
%    \begin{macrocode}
\def\HoLogo@LaTeX#1{%
  \hologo{La}%
  \kern-.15em%
  \hologo{TeX}%
}
%    \end{macrocode}
%    \end{macro}
%    \begin{macro}{\HoLogoHtml@LaTeX}
%    \begin{macrocode}
\def\HoLogoHtml@LaTeX#1{%
  \HoLogoCss@LaTeX
  \HOLOGO@Span{LaTeX}{%
    L%
    \HOLOGO@Span{a}{%
      A%
    }%
    \hologo{TeX}%
  }%
}
%    \end{macrocode}
%    \end{macro}
%    \begin{macro}{\HoLogoCss@LaTeX}
%    \begin{macrocode}
\def\HoLogoCss@LaTeX{%
  \Css{%
    span.HoLogo-LaTeX span.HoLogo-a{%
      position:relative;%
      top:-.5ex;%
      margin-left:-.36em;%
      margin-right:-.15em;%
      font-size:85\%;%
    }%
  }%
  \global\let\HoLogoCss@LaTeX\relax
}
%    \end{macrocode}
%    \end{macro}
%
% \subsubsection{\hologo{(La)TeX}}
%
%    \begin{macro}{\HoLogo@LaTeXTeX}
%    The kerning around the parentheses is taken
%    from package \xpackage{dtklogos} \cite{dtklogos}.
%\begin{quote}
%\begin{verbatim}
%\DeclareRobustCommand{\LaTeXTeX}{%
%  (%
%  \kern-.15em%
%  L%
%  \kern-.36em%
%  {%
%    \sbox\z@ T%
%    \vbox to\ht0{%
%      \hbox{%
%        $\m@th$%
%        \csname S@\f@size\endcsname
%        \fontsize\sf@size\z@
%        \math@fontsfalse
%        \selectfont
%        A%
%      }%
%      \vss
%    }%
%  }%
%  \kern-.2em%
%  )%
%  \kern-.15em%
%  \TeX
%}
%\end{verbatim}
%\end{quote}
%    \begin{macrocode}
\def\HoLogo@LaTeXTeX#1{%
  (%
  \kern-.15em%
  \hologo{La}%
  \kern-.2em%
  )%
  \kern-.15em%
  \hologo{TeX}%
}
%    \end{macrocode}
%    \end{macro}
%    \begin{macro}{\HoLogoBkm@LaTeXTeX}
%    \begin{macrocode}
\def\HoLogoBkm@LaTeXTeX#1{(La)TeX}
%    \end{macrocode}
%    \end{macro}
%
%    \begin{macro}{\HoLogo@(La)TeX}
%    \begin{macrocode}
\expandafter
\let\csname HoLogo@(La)TeX\endcsname\HoLogo@LaTeXTeX
%    \end{macrocode}
%    \end{macro}
%    \begin{macro}{\HoLogoBkm@(La)TeX}
%    \begin{macrocode}
\expandafter
\let\csname HoLogoBkm@(La)TeX\endcsname\HoLogoBkm@LaTeXTeX
%    \end{macrocode}
%    \end{macro}
%    \begin{macro}{\HoLogoHtml@LaTeXTeX}
%    \begin{macrocode}
\def\HoLogoHtml@LaTeXTeX#1{%
  \HoLogoCss@LaTeXTeX
  \HOLOGO@Span{LaTeXTeX}{%
    (%
    \HOLOGO@Span{L}{L}%
    \HOLOGO@Span{a}{A}%
    \HOLOGO@Span{ParenRight}{)}%
    \hologo{TeX}%
  }%
}
%    \end{macrocode}
%    \end{macro}
%    \begin{macro}{\HoLogoHtml@(La)TeX}
%    Kerning after opening parentheses and before closing parentheses
%    is $-0.1$\,em. The original values $-0.15$\,em
%    looked too ugly for a serif font.
%    \begin{macrocode}
\expandafter
\let\csname HoLogoHtml@(La)TeX\endcsname\HoLogoHtml@LaTeXTeX
%    \end{macrocode}
%    \end{macro}
%    \begin{macro}{\HoLogoCss@LaTeXTeX}
%    \begin{macrocode}
\def\HoLogoCss@LaTeXTeX{%
  \Css{%
    span.HoLogo-LaTeXTeX span.HoLogo-L{%
      margin-left:-.1em;%
    }%
  }%
  \Css{%
    span.HoLogo-LaTeXTeX span.HoLogo-a{%
      position:relative;%
      top:-.5ex;%
      margin-left:-.36em;%
      margin-right:-.1em;%
      font-size:85\%;%
    }%
  }%
  \Css{%
    span.HoLogo-LaTeXTeX span.HoLogo-ParenRight{%
      margin-right:-.15em;%
    }%
  }%
  \global\let\HoLogoCss@LaTeXTeX\relax
}
%    \end{macrocode}
%    \end{macro}
%
% \subsubsection{\hologo{LaTeXe}}
%
%    \begin{macro}{\HoLogo@LaTeXe}
%    Source: \hologo{LaTeX} kernel
%    \begin{macrocode}
\def\HoLogo@LaTeXe#1{%
  \hologo{LaTeX}%
  \kern.15em%
  \hbox{%
    \HOLOGO@MathSetup
    2%
    $_{\textstyle\varepsilon}$%
  }%
}
%    \end{macrocode}
%    \end{macro}
%
%    \begin{macro}{\HoLogoCs@LaTeXe}
%    \begin{macrocode}
\ifnum64=`\^^^^0040\relax % test for big chars of LuaTeX/XeTeX
  \catcode`\$=9 %
  \catcode`\&=14 %
\else
  \catcode`\$=14 %
  \catcode`\&=9 %
\fi
\def\HoLogoCs@LaTeXe#1{%
  LaTeX2%
$ \string ^^^^0395%
& e%
}%
\catcode`\$=3 %
\catcode`\&=4 %
%    \end{macrocode}
%    \end{macro}
%
%    \begin{macro}{\HoLogoBkm@LaTeXe}
%    \begin{macrocode}
\def\HoLogoBkm@LaTeXe#1{%
  \hologo{LaTeX}%
  2%
  \HOLOGO@PdfdocUnicode{e}{\textepsilon}%
}
%    \end{macrocode}
%    \end{macro}
%
%    \begin{macro}{\HoLogoHtml@LaTeXe}
%    \begin{macrocode}
\def\HoLogoHtml@LaTeXe#1{%
  \HoLogoCss@LaTeXe
  \HOLOGO@Span{LaTeX2e}{%
    \hologo{LaTeX}%
    \HOLOGO@Span{2}{2}%
    \HOLOGO@Span{e}{%
      \HOLOGO@MathSetup
      \ensuremath{\textstyle\varepsilon}%
    }%
  }%
}
%    \end{macrocode}
%    \end{macro}
%    \begin{macro}{\HoLogoCss@LaTeXe}
%    \begin{macrocode}
\def\HoLogoCss@LaTeXe{%
  \Css{%
    span.HoLogo-LaTeX2e span.HoLogo-2{%
      padding-left:.15em;%
    }%
  }%
  \Css{%
    span.HoLogo-LaTeX2e span.HoLogo-e{%
      position:relative;%
      top:.35ex;%
      text-decoration:none;%
    }%
  }%
  \global\let\HoLogoCss@LaTeXe\relax
}
%    \end{macrocode}
%    \end{macro}
%
%    \begin{macro}{\HoLogo@LaTeX2e}
%    \begin{macrocode}
\expandafter
\let\csname HoLogo@LaTeX2e\endcsname\HoLogo@LaTeXe
%    \end{macrocode}
%    \end{macro}
%    \begin{macro}{\HoLogoCs@LaTeX2e}
%    \begin{macrocode}
\expandafter
\let\csname HoLogoCs@LaTeX2e\endcsname\HoLogoCs@LaTeXe
%    \end{macrocode}
%    \end{macro}
%    \begin{macro}{\HoLogoBkm@LaTeX2e}
%    \begin{macrocode}
\expandafter
\let\csname HoLogoBkm@LaTeX2e\endcsname\HoLogoBkm@LaTeXe
%    \end{macrocode}
%    \end{macro}
%    \begin{macro}{\HoLogoHtml@LaTeX2e}
%    \begin{macrocode}
\expandafter
\let\csname HoLogoHtml@LaTeX2e\endcsname\HoLogoHtml@LaTeXe
%    \end{macrocode}
%    \end{macro}
%
% \subsubsection{\hologo{LaTeX3}}
%
%    \begin{macro}{\HoLogo@LaTeX3}
%    Source: \hologo{LaTeX} kernel
%    \begin{macrocode}
\expandafter\def\csname HoLogo@LaTeX3\endcsname#1{%
  \hologo{LaTeX}%
  3%
}
%    \end{macrocode}
%    \end{macro}
%
%    \begin{macro}{\HoLogoBkm@LaTeX3}
%    \begin{macrocode}
\expandafter\def\csname HoLogoBkm@LaTeX3\endcsname#1{%
  \hologo{LaTeX}%
  3%
}
%    \end{macrocode}
%    \end{macro}
%    \begin{macro}{\HoLogoHtml@LaTeX3}
%    \begin{macrocode}
\expandafter
\let\csname HoLogoHtml@LaTeX3\expandafter\endcsname
\csname HoLogo@LaTeX3\endcsname
%    \end{macrocode}
%    \end{macro}
%
% \subsubsection{\hologo{LaTeXML}}
%
%    \begin{macro}{\HoLogo@LaTeXML}
%    \begin{macrocode}
\def\HoLogo@LaTeXML#1{%
  \HOLOGO@mbox{%
    \hologo{La}%
    \kern-.15em%
    T%
    \kern-.1667em%
    \lower.5ex\hbox{E}%
    \kern-.125em%
    \HoLogoFont@font{LaTeXML}{sc}{xml}%
  }%
}
%    \end{macrocode}
%    \end{macro}
%    \begin{macro}{\HoLogoHtml@pdfLaTeX}
%    \begin{macrocode}
\def\HoLogoHtml@LaTeXML#1{%
  \HOLOGO@Span{LaTeXML}{%
    \HoLogoCss@LaTeX
    \HoLogoCss@TeX
    \HOLOGO@Span{LaTeX}{%
      L%
      \HOLOGO@Span{a}{%
        A%
      }%
    }%
    \HOLOGO@Span{TeX}{%
      T%
      \HOLOGO@Span{e}{%
        E%
      }%
    }%
    \HCode{<span style="font-variant: small-caps;">}%
    xml%
    \HCode{</span>}%
  }%
}
%    \end{macrocode}
%    \end{macro}
%
% \subsubsection{\hologo{eTeX}}
%
%    \begin{macro}{\HoLogo@eTeX}
%    Source: package \xpackage{etex}
%    \begin{macrocode}
\def\HoLogo@eTeX#1{%
  \ltx@mbox{%
    \HOLOGO@MathSetup
    $\varepsilon$%
    -%
    \HOLOGO@NegativeKerning{-T,T-,To}%
    \hologo{TeX}%
  }%
}
%    \end{macrocode}
%    \end{macro}
%    \begin{macro}{\HoLogoCs@eTeX}
%    \begin{macrocode}
\ifnum64=`\^^^^0040\relax % test for big chars of LuaTeX/XeTeX
  \catcode`\$=9 %
  \catcode`\&=14 %
\else
  \catcode`\$=14 %
  \catcode`\&=9 %
\fi
\def\HoLogoCs@eTeX#1{%
$ #1{\string ^^^^0395}{\string ^^^^03b5}%
& #1{e}{E}%
  TeX%
}%
\catcode`\$=3 %
\catcode`\&=4 %
%    \end{macrocode}
%    \end{macro}
%    \begin{macro}{\HoLogoBkm@eTeX}
%    \begin{macrocode}
\def\HoLogoBkm@eTeX#1{%
  \HOLOGO@PdfdocUnicode{#1{e}{E}}{\textepsilon}%
  -%
  \hologo{TeX}%
}
%    \end{macrocode}
%    \end{macro}
%    \begin{macro}{\HoLogoHtml@eTeX}
%    \begin{macrocode}
\def\HoLogoHtml@eTeX#1{%
  \ltx@mbox{%
    \HOLOGO@MathSetup
    $\varepsilon$%
    -%
    \hologo{TeX}%
  }%
}
%    \end{macrocode}
%    \end{macro}
%
% \subsubsection{\hologo{iniTeX}}
%
%    \begin{macro}{\HoLogo@iniTeX}
%    \begin{macrocode}
\def\HoLogo@iniTeX#1{%
  \HOLOGO@mbox{%
    #1{i}{I}ni\hologo{TeX}%
  }%
}
%    \end{macrocode}
%    \end{macro}
%    \begin{macro}{\HoLogoCs@iniTeX}
%    \begin{macrocode}
\def\HoLogoCs@iniTeX#1{#1{i}{I}niTeX}
%    \end{macrocode}
%    \end{macro}
%    \begin{macro}{\HoLogoBkm@iniTeX}
%    \begin{macrocode}
\def\HoLogoBkm@iniTeX#1{%
  #1{i}{I}ni\hologo{TeX}%
}
%    \end{macrocode}
%    \end{macro}
%    \begin{macro}{\HoLogoHtml@iniTeX}
%    \begin{macrocode}
\let\HoLogoHtml@iniTeX\HoLogo@iniTeX
%    \end{macrocode}
%    \end{macro}
%
% \subsubsection{\hologo{virTeX}}
%
%    \begin{macro}{\HoLogo@virTeX}
%    \begin{macrocode}
\def\HoLogo@virTeX#1{%
  \HOLOGO@mbox{%
    #1{v}{V}ir\hologo{TeX}%
  }%
}
%    \end{macrocode}
%    \end{macro}
%    \begin{macro}{\HoLogoCs@virTeX}
%    \begin{macrocode}
\def\HoLogoCs@virTeX#1{#1{v}{V}irTeX}
%    \end{macrocode}
%    \end{macro}
%    \begin{macro}{\HoLogoBkm@virTeX}
%    \begin{macrocode}
\def\HoLogoBkm@virTeX#1{%
  #1{v}{V}ir\hologo{TeX}%
}
%    \end{macrocode}
%    \end{macro}
%    \begin{macro}{\HoLogoHtml@virTeX}
%    \begin{macrocode}
\let\HoLogoHtml@virTeX\HoLogo@virTeX
%    \end{macrocode}
%    \end{macro}
%
% \subsubsection{\hologo{SliTeX}}
%
% \paragraph{Definitions of the three variants.}
%
%    \begin{macro}{\HoLogo@SLiTeX@lift}
%    \begin{macrocode}
\def\HoLogo@SLiTeX@lift#1{%
  \HoLogoFont@font{SliTeX}{rm}{%
    S%
    \kern-.06em%
    L%
    \kern-.18em%
    \raise.32ex\hbox{\HoLogoFont@font{SliTeX}{sc}{i}}%
    \HOLOGO@discretionary
    \kern-.06em%
    \hologo{TeX}%
  }%
}
%    \end{macrocode}
%    \end{macro}
%    \begin{macro}{\HoLogoBkm@SLiTeX@lift}
%    \begin{macrocode}
\def\HoLogoBkm@SLiTeX@lift#1{SLiTeX}
%    \end{macrocode}
%    \end{macro}
%    \begin{macro}{\HoLogoHtml@SLiTeX@lift}
%    \begin{macrocode}
\def\HoLogoHtml@SLiTeX@lift#1{%
  \HoLogoCss@SLiTeX@lift
  \HOLOGO@Span{SLiTeX-lift}{%
    \HoLogoFont@font{SliTeX}{rm}{%
      S%
      \HOLOGO@Span{L}{L}%
      \HOLOGO@Span{i}{i}%
      \hologo{TeX}%
    }%
  }%
}
%    \end{macrocode}
%    \end{macro}
%    \begin{macro}{\HoLogoCss@SLiTeX@lift}
%    \begin{macrocode}
\def\HoLogoCss@SLiTeX@lift{%
  \Css{%
    span.HoLogo-SLiTeX-lift span.HoLogo-L{%
      margin-left:-.06em;%
      margin-right:-.18em;%
    }%
  }%
  \Css{%
    span.HoLogo-SLiTeX-lift span.HoLogo-i{%
      position:relative;%
      top:-.32ex;%
      margin-right:-.06em;%
      font-variant:small-caps;%
    }%
  }%
  \global\let\HoLogoCss@SLiTeX@lift\relax
}
%    \end{macrocode}
%    \end{macro}
%
%    \begin{macro}{\HoLogo@SliTeX@simple}
%    \begin{macrocode}
\def\HoLogo@SliTeX@simple#1{%
  \HoLogoFont@font{SliTeX}{rm}{%
    \ltx@mbox{%
      \HoLogoFont@font{SliTeX}{sc}{Sli}%
    }%
    \HOLOGO@discretionary
    \hologo{TeX}%
  }%
}
%    \end{macrocode}
%    \end{macro}
%    \begin{macro}{\HoLogoBkm@SliTeX@simple}
%    \begin{macrocode}
\def\HoLogoBkm@SliTeX@simple#1{SliTeX}
%    \end{macrocode}
%    \end{macro}
%    \begin{macro}{\HoLogoHtml@SliTeX@simple}
%    \begin{macrocode}
\let\HoLogoHtml@SliTeX@simple\HoLogo@SliTeX@simple
%    \end{macrocode}
%    \end{macro}
%
%    \begin{macro}{\HoLogo@SliTeX@narrow}
%    \begin{macrocode}
\def\HoLogo@SliTeX@narrow#1{%
  \HoLogoFont@font{SliTeX}{rm}{%
    \ltx@mbox{%
      S%
      \kern-.06em%
      \HoLogoFont@font{SliTeX}{sc}{%
        l%
        \kern-.035em%
        i%
      }%
    }%
    \HOLOGO@discretionary
    \kern-.06em%
    \hologo{TeX}%
  }%
}
%    \end{macrocode}
%    \end{macro}
%    \begin{macro}{\HoLogoBkm@SliTeX@narrow}
%    \begin{macrocode}
\def\HoLogoBkm@SliTeX@narrow#1{SliTeX}
%    \end{macrocode}
%    \end{macro}
%    \begin{macro}{\HoLogoHtml@SliTeX@narrow}
%    \begin{macrocode}
\def\HoLogoHtml@SliTeX@narrow#1{%
  \HoLogoCss@SliTeX@narrow
  \HOLOGO@Span{SliTeX-narrow}{%
    \HoLogoFont@font{SliTeX}{rm}{%
      S%
        \HOLOGO@Span{l}{l}%
        \HOLOGO@Span{i}{i}%
      \hologo{TeX}%
    }%
  }%
}
%    \end{macrocode}
%    \end{macro}
%    \begin{macro}{\HoLogoCss@SliTeX@narrow}
%    \begin{macrocode}
\def\HoLogoCss@SliTeX@narrow{%
  \Css{%
    span.HoLogo-SliTeX-narrow span.HoLogo-l{%
      margin-left:-.06em;%
      margin-right:-.035em;%
      font-variant:small-caps;%
    }%
  }%
  \Css{%
    span.HoLogo-SliTeX-narrow span.HoLogo-i{%
      margin-right:-.06em;%
      font-variant:small-caps;%
    }%
  }%
  \global\let\HoLogoCss@SliTeX@narrow\relax
}
%    \end{macrocode}
%    \end{macro}
%
% \paragraph{Macro set completion.}
%
%    \begin{macro}{\HoLogo@SLiTeX@simple}
%    \begin{macrocode}
\def\HoLogo@SLiTeX@simple{\HoLogo@SliTeX@simple}
%    \end{macrocode}
%    \end{macro}
%    \begin{macro}{\HoLogoBkm@SLiTeX@simple}
%    \begin{macrocode}
\def\HoLogoBkm@SLiTeX@simple{\HoLogoBkm@SliTeX@simple}
%    \end{macrocode}
%    \end{macro}
%    \begin{macro}{\HoLogoHtml@SLiTeX@simple}
%    \begin{macrocode}
\def\HoLogoHtml@SLiTeX@simple{\HoLogoHtml@SliTeX@simple}
%    \end{macrocode}
%    \end{macro}
%
%    \begin{macro}{\HoLogo@SLiTeX@narrow}
%    \begin{macrocode}
\def\HoLogo@SLiTeX@narrow{\HoLogo@SliTeX@narrow}
%    \end{macrocode}
%    \end{macro}
%    \begin{macro}{\HoLogoBkm@SLiTeX@narrow}
%    \begin{macrocode}
\def\HoLogoBkm@SLiTeX@narrow{\HoLogoBkm@SliTeX@narrow}
%    \end{macrocode}
%    \end{macro}
%    \begin{macro}{\HoLogoHtml@SLiTeX@narrow}
%    \begin{macrocode}
\def\HoLogoHtml@SLiTeX@narrow{\HoLogoHtml@SliTeX@narrow}
%    \end{macrocode}
%    \end{macro}
%
%    \begin{macro}{\HoLogo@SliTeX@lift}
%    \begin{macrocode}
\def\HoLogo@SliTeX@lift{\HoLogo@SLiTeX@lift}
%    \end{macrocode}
%    \end{macro}
%    \begin{macro}{\HoLogoBkm@SliTeX@lift}
%    \begin{macrocode}
\def\HoLogoBkm@SliTeX@lift{\HoLogoBkm@SLiTeX@lift}
%    \end{macrocode}
%    \end{macro}
%    \begin{macro}{\HoLogoHtml@SliTeX@lift}
%    \begin{macrocode}
\def\HoLogoHtml@SliTeX@lift{\HoLogoHtml@SLiTeX@lift}
%    \end{macrocode}
%    \end{macro}
%
% \paragraph{Defaults.}
%
%    \begin{macro}{\HoLogo@SLiTeX}
%    \begin{macrocode}
\def\HoLogo@SLiTeX{\HoLogo@SLiTeX@lift}
%    \end{macrocode}
%    \end{macro}
%    \begin{macro}{\HoLogoBkm@SLiTeX}
%    \begin{macrocode}
\def\HoLogoBkm@SLiTeX{\HoLogoBkm@SLiTeX@lift}
%    \end{macrocode}
%    \end{macro}
%    \begin{macro}{\HoLogoHtml@SLiTeX}
%    \begin{macrocode}
\def\HoLogoHtml@SLiTeX{\HoLogoHtml@SLiTeX@lift}
%    \end{macrocode}
%    \end{macro}
%
%    \begin{macro}{\HoLogo@SliTeX}
%    \begin{macrocode}
\def\HoLogo@SliTeX{\HoLogo@SliTeX@narrow}
%    \end{macrocode}
%    \end{macro}
%    \begin{macro}{\HoLogoBkm@SliTeX}
%    \begin{macrocode}
\def\HoLogoBkm@SliTeX{\HoLogoBkm@SliTeX@narrow}
%    \end{macrocode}
%    \end{macro}
%    \begin{macro}{\HoLogoHtml@SliTeX}
%    \begin{macrocode}
\def\HoLogoHtml@SliTeX{\HoLogoHtml@SliTeX@narrow}
%    \end{macrocode}
%    \end{macro}
%
% \subsubsection{\hologo{LuaTeX}}
%
%    \begin{macro}{\HoLogo@LuaTeX}
%    The kerning is an idea of Hans Hagen, see mailing list
%    `luatex at tug dot org' in March 2010.
%    \begin{macrocode}
\def\HoLogo@LuaTeX#1{%
  \HOLOGO@mbox{%
    Lua%
    \HOLOGO@NegativeKerning{aT,oT,To}%
    \hologo{TeX}%
  }%
}
%    \end{macrocode}
%    \end{macro}
%    \begin{macro}{\HoLogoHtml@LuaTeX}
%    \begin{macrocode}
\let\HoLogoHtml@LuaTeX\HoLogo@LuaTeX
%    \end{macrocode}
%    \end{macro}
%
% \subsubsection{\hologo{LuaLaTeX}}
%
%    \begin{macro}{\HoLogo@LuaLaTeX}
%    \begin{macrocode}
\def\HoLogo@LuaLaTeX#1{%
  \HOLOGO@mbox{%
    Lua%
    \hologo{LaTeX}%
  }%
}
%    \end{macrocode}
%    \end{macro}
%    \begin{macro}{\HoLogoHtml@LuaLaTeX}
%    \begin{macrocode}
\let\HoLogoHtml@LuaLaTeX\HoLogo@LuaLaTeX
%    \end{macrocode}
%    \end{macro}
%
% \subsubsection{\hologo{XeTeX}, \hologo{XeLaTeX}}
%
%    \begin{macro}{\HOLOGO@IfCharExists}
%    \begin{macrocode}
\ifluatex
  \ifnum\luatexversion<36 %
  \else
    \def\HOLOGO@IfCharExists#1{%
      \ifnum
        \directlua{%
           if luaotfload and luaotfload.aux then
             if luaotfload.aux.font_has_glyph(%
                    font.current(), \number#1) then % 	 
	       tex.print("1") % 	 
	     end % 	 
	   elseif font and font.fonts and font.current then %
            local f = font.fonts[font.current()]%
            if f.characters and f.characters[\number#1] then %
              tex.print("1")%
            end %
          end%
        }0=\ltx@zero
        \expandafter\ltx@secondoftwo
      \else
        \expandafter\ltx@firstoftwo
      \fi
    }%
  \fi
\fi
\ltx@IfUndefined{HOLOGO@IfCharExists}{%
  \def\HOLOGO@@IfCharExists#1{%
    \begingroup
      \tracinglostchars=\ltx@zero
      \setbox\ltx@zero=\hbox{%
        \kern7sp\char#1\relax
        \ifnum\lastkern>\ltx@zero
          \expandafter\aftergroup\csname iffalse\endcsname
        \else
          \expandafter\aftergroup\csname iftrue\endcsname
        \fi
      }%
      % \if{true|false} from \aftergroup
      \endgroup
      \expandafter\ltx@firstoftwo
    \else
      \endgroup
      \expandafter\ltx@secondoftwo
    \fi
  }%
  \ifxetex
    \ltx@IfUndefined{XeTeXfonttype}{}{%
      \ltx@IfUndefined{XeTeXcharglyph}{}{%
        \def\HOLOGO@IfCharExists#1{%
          \ifnum\XeTeXfonttype\font>\ltx@zero
            \expandafter\ltx@firstofthree
          \else
            \expandafter\ltx@gobble
          \fi
          {%
            \ifnum\XeTeXcharglyph#1>\ltx@zero
              \expandafter\ltx@firstoftwo
            \else
              \expandafter\ltx@secondoftwo
            \fi
          }%
          \HOLOGO@@IfCharExists{#1}%
        }%
      }%
    }%
  \fi
}{}
\ltx@ifundefined{HOLOGO@IfCharExists}{%
  \ifnum64=`\^^^^0040\relax % test for big chars of LuaTeX/XeTeX
    \let\HOLOGO@IfCharExists\HOLOGO@@IfCharExists
  \else
    \def\HOLOGO@IfCharExists#1{%
      \ifnum#1>255 %
        \expandafter\ltx@fourthoffour
      \fi
      \HOLOGO@@IfCharExists{#1}%
    }%
  \fi
}{}
%    \end{macrocode}
%    \end{macro}
%
%    \begin{macro}{\HoLogo@Xe}
%    Source: package \xpackage{dtklogos}
%    \begin{macrocode}
\def\HoLogo@Xe#1{%
  X%
  \kern-.1em\relax
  \HOLOGO@IfCharExists{"018E}{%
    \lower.5ex\hbox{\char"018E}%
  }{%
    \chardef\HOLOGO@choice=\ltx@zero
    \ifdim\fontdimen\ltx@one\font>0pt %
      \ltx@IfUndefined{rotatebox}{%
        \ltx@IfUndefined{pgftext}{%
          \ltx@IfUndefined{psscalebox}{%
            \ltx@IfUndefined{HOLOGO@ScaleBox@\hologoDriver}{%
            }{%
              \chardef\HOLOGO@choice=4 %
            }%
          }{%
            \chardef\HOLOGO@choice=3 %
          }%
        }{%
          \chardef\HOLOGO@choice=2 %
        }%
      }{%
        \chardef\HOLOGO@choice=1 %
      }%
      \ifcase\HOLOGO@choice
        \HOLOGO@WarningUnsupportedDriver{Xe}%
        e%
      \or % 1: \rotatebox
        \begingroup
          \setbox\ltx@zero\hbox{\rotatebox{180}{E}}%
          \ltx@LocDimenA=\dp\ltx@zero
          \advance\ltx@LocDimenA by -.5ex\relax
          \raise\ltx@LocDimenA\box\ltx@zero
        \endgroup
      \or % 2: \pgftext
        \lower.5ex\hbox{%
          \pgfpicture
            \pgftext[rotate=180]{E}%
          \endpgfpicture
        }%
      \or % 3: \psscalebox
        \begingroup
          \setbox\ltx@zero\hbox{\psscalebox{-1 -1}{E}}%
          \ltx@LocDimenA=\dp\ltx@zero
          \advance\ltx@LocDimenA by -.5ex\relax
          \raise\ltx@LocDimenA\box\ltx@zero
        \endgroup
      \or % 4: \HOLOGO@PointReflectBox
        \lower.5ex\hbox{\HOLOGO@PointReflectBox{E}}%
      \else
        \@PackageError{hologo}{Internal error (choice/it}\@ehc
      \fi
    \else
      \ltx@IfUndefined{reflectbox}{%
        \ltx@IfUndefined{pgftext}{%
          \ltx@IfUndefined{psscalebox}{%
            \ltx@IfUndefined{HOLOGO@ScaleBox@\hologoDriver}{%
            }{%
              \chardef\HOLOGO@choice=4 %
            }%
          }{%
            \chardef\HOLOGO@choice=3 %
          }%
        }{%
          \chardef\HOLOGO@choice=2 %
        }%
      }{%
        \chardef\HOLOGO@choice=1 %
      }%
      \ifcase\HOLOGO@choice
        \HOLOGO@WarningUnsupportedDriver{Xe}%
        e%
      \or % 1: reflectbox
        \lower.5ex\hbox{%
          \reflectbox{E}%
        }%
      \or % 2: \pgftext
        \lower.5ex\hbox{%
          \pgfpicture
            \pgftransformxscale{-1}%
            \pgftext{E}%
          \endpgfpicture
        }%
      \or % 3: \psscalebox
        \lower.5ex\hbox{%
          \psscalebox{-1 1}{E}%
        }%
      \or % 4: \HOLOGO@Reflectbox
        \lower.5ex\hbox{%
          \HOLOGO@ReflectBox{E}%
        }%
      \else
        \@PackageError{hologo}{Internal error (choice/up)}\@ehc
      \fi
    \fi
  }%
}
%    \end{macrocode}
%    \end{macro}
%    \begin{macro}{\HoLogoHtml@Xe}
%    \begin{macrocode}
\def\HoLogoHtml@Xe#1{%
  \HoLogoCss@Xe
  \HOLOGO@Span{Xe}{%
    X%
    \HOLOGO@Span{e}{%
      \HCode{&\ltx@hashchar x018e;}%
    }%
  }%
}
%    \end{macrocode}
%    \end{macro}
%    \begin{macro}{\HoLogoCss@Xe}
%    \begin{macrocode}
\def\HoLogoCss@Xe{%
  \Css{%
    span.HoLogo-Xe span.HoLogo-e{%
      position:relative;%
      top:.5ex;%
      left-margin:-.1em;%
    }%
  }%
  \global\let\HoLogoCss@Xe\relax
}
%    \end{macrocode}
%    \end{macro}
%
%    \begin{macro}{\HoLogo@XeTeX}
%    \begin{macrocode}
\def\HoLogo@XeTeX#1{%
  \hologo{Xe}%
  \kern-.15em\relax
  \hologo{TeX}%
}
%    \end{macrocode}
%    \end{macro}
%
%    \begin{macro}{\HoLogoHtml@XeTeX}
%    \begin{macrocode}
\def\HoLogoHtml@XeTeX#1{%
  \HoLogoCss@XeTeX
  \HOLOGO@Span{XeTeX}{%
    \hologo{Xe}%
    \hologo{TeX}%
  }%
}
%    \end{macrocode}
%    \end{macro}
%    \begin{macro}{\HoLogoCss@XeTeX}
%    \begin{macrocode}
\def\HoLogoCss@XeTeX{%
  \Css{%
    span.HoLogo-XeTeX span.HoLogo-TeX{%
      margin-left:-.15em;%
    }%
  }%
  \global\let\HoLogoCss@XeTeX\relax
}
%    \end{macrocode}
%    \end{macro}
%
%    \begin{macro}{\HoLogo@XeLaTeX}
%    \begin{macrocode}
\def\HoLogo@XeLaTeX#1{%
  \hologo{Xe}%
  \kern-.13em%
  \hologo{LaTeX}%
}
%    \end{macrocode}
%    \end{macro}
%    \begin{macro}{\HoLogoHtml@XeLaTeX}
%    \begin{macrocode}
\def\HoLogoHtml@XeLaTeX#1{%
  \HoLogoCss@XeLaTeX
  \HOLOGO@Span{XeLaTeX}{%
    \hologo{Xe}%
    \hologo{LaTeX}%
  }%
}
%    \end{macrocode}
%    \end{macro}
%    \begin{macro}{\HoLogoCss@XeLaTeX}
%    \begin{macrocode}
\def\HoLogoCss@XeLaTeX{%
  \Css{%
    span.HoLogo-XeLaTeX span.HoLogo-Xe{%
      margin-right:-.13em;%
    }%
  }%
  \global\let\HoLogoCss@XeLaTeX\relax
}
%    \end{macrocode}
%    \end{macro}
%
% \subsubsection{\hologo{pdfTeX}, \hologo{pdfLaTeX}}
%
%    \begin{macro}{\HoLogo@pdfTeX}
%    \begin{macrocode}
\def\HoLogo@pdfTeX#1{%
  \HOLOGO@mbox{%
    #1{p}{P}df\hologo{TeX}%
  }%
}
%    \end{macrocode}
%    \end{macro}
%    \begin{macro}{\HoLogoCs@pdfTeX}
%    \begin{macrocode}
\def\HoLogoCs@pdfTeX#1{#1{p}{P}dfTeX}
%    \end{macrocode}
%    \end{macro}
%    \begin{macro}{\HoLogoBkm@pdfTeX}
%    \begin{macrocode}
\def\HoLogoBkm@pdfTeX#1{%
  #1{p}{P}df\hologo{TeX}%
}
%    \end{macrocode}
%    \end{macro}
%    \begin{macro}{\HoLogoHtml@pdfTeX}
%    \begin{macrocode}
\let\HoLogoHtml@pdfTeX\HoLogo@pdfTeX
%    \end{macrocode}
%    \end{macro}
%
%    \begin{macro}{\HoLogo@pdfLaTeX}
%    \begin{macrocode}
\def\HoLogo@pdfLaTeX#1{%
  \HOLOGO@mbox{%
    #1{p}{P}df\hologo{LaTeX}%
  }%
}
%    \end{macrocode}
%    \end{macro}
%    \begin{macro}{\HoLogoCs@pdfLaTeX}
%    \begin{macrocode}
\def\HoLogoCs@pdfLaTeX#1{#1{p}{P}dfLaTeX}
%    \end{macrocode}
%    \end{macro}
%    \begin{macro}{\HoLogoBkm@pdfLaTeX}
%    \begin{macrocode}
\def\HoLogoBkm@pdfLaTeX#1{%
  #1{p}{P}df\hologo{LaTeX}%
}
%    \end{macrocode}
%    \end{macro}
%    \begin{macro}{\HoLogoHtml@pdfLaTeX}
%    \begin{macrocode}
\let\HoLogoHtml@pdfLaTeX\HoLogo@pdfLaTeX
%    \end{macrocode}
%    \end{macro}
%
% \subsubsection{\hologo{VTeX}}
%
%    \begin{macro}{\HoLogo@VTeX}
%    \begin{macrocode}
\def\HoLogo@VTeX#1{%
  \HOLOGO@mbox{%
    V\hologo{TeX}%
  }%
}
%    \end{macrocode}
%    \end{macro}
%    \begin{macro}{\HoLogoHtml@VTeX}
%    \begin{macrocode}
\let\HoLogoHtml@VTeX\HoLogo@VTeX
%    \end{macrocode}
%    \end{macro}
%
% \subsubsection{\hologo{AmS}, \dots}
%
%    Source: class \xclass{amsdtx}
%
%    \begin{macro}{\HoLogo@AmS}
%    \begin{macrocode}
\def\HoLogo@AmS#1{%
  \HoLogoFont@font{AmS}{sy}{%
    A%
    \kern-.1667em%
    \lower.5ex\hbox{M}%
    \kern-.125em%
    S%
  }%
}
%    \end{macrocode}
%    \end{macro}
%    \begin{macro}{\HoLogoBkm@AmS}
%    \begin{macrocode}
\def\HoLogoBkm@AmS#1{AmS}
%    \end{macrocode}
%    \end{macro}
%    \begin{macro}{\HoLogoHtml@AmS}
%    \begin{macrocode}
\def\HoLogoHtml@AmS#1{%
  \HoLogoCss@AmS
%  \HoLogoFont@font{AmS}{sy}{%
    \HOLOGO@Span{AmS}{%
      A%
      \HOLOGO@Span{M}{M}%
      S%
    }%
%   }%
}
%    \end{macrocode}
%    \end{macro}
%    \begin{macro}{\HoLogoCss@AmS}
%    \begin{macrocode}
\def\HoLogoCss@AmS{%
  \Css{%
    span.HoLogo-AmS span.HoLogo-M{%
      position:relative;%
      top:.5ex;%
      margin-left:-.1667em;%
      margin-right:-.125em;%
      text-decoration:none;%
    }%
  }%
  \global\let\HoLogoCss@AmS\relax
}
%    \end{macrocode}
%    \end{macro}
%
%    \begin{macro}{\HoLogo@AmSTeX}
%    \begin{macrocode}
\def\HoLogo@AmSTeX#1{%
  \hologo{AmS}%
  \HOLOGO@hyphen
  \hologo{TeX}%
}
%    \end{macrocode}
%    \end{macro}
%    \begin{macro}{\HoLogoBkm@AmSTeX}
%    \begin{macrocode}
\def\HoLogoBkm@AmSTeX#1{AmS-TeX}%
%    \end{macrocode}
%    \end{macro}
%    \begin{macro}{\HoLogoHtml@AmSTeX}
%    \begin{macrocode}
\let\HoLogoHtml@AmSTeX\HoLogo@AmSTeX
%    \end{macrocode}
%    \end{macro}
%
%    \begin{macro}{\HoLogo@AmSLaTeX}
%    \begin{macrocode}
\def\HoLogo@AmSLaTeX#1{%
  \hologo{AmS}%
  \HOLOGO@hyphen
  \hologo{LaTeX}%
}
%    \end{macrocode}
%    \end{macro}
%    \begin{macro}{\HoLogoBkm@AmSLaTeX}
%    \begin{macrocode}
\def\HoLogoBkm@AmSLaTeX#1{AmS-LaTeX}%
%    \end{macrocode}
%    \end{macro}
%    \begin{macro}{\HoLogoHtml@AmSLaTeX}
%    \begin{macrocode}
\let\HoLogoHtml@AmSLaTeX\HoLogo@AmSLaTeX
%    \end{macrocode}
%    \end{macro}
%
% \subsubsection{\hologo{BibTeX}}
%
%    \begin{macro}{\HoLogo@BibTeX@sc}
%    A definition of \hologo{BibTeX} is provided in
%    the documentation source for the manual of \hologo{BibTeX}
%    \cite{btxdoc}.
%\begin{quote}
%\begin{verbatim}
%\def\BibTeX{%
%  {%
%    \rm
%    B%
%    \kern-.05em%
%    {%
%      \sc
%      i%
%      \kern-.025em %
%      b%
%    }%
%    \kern-.08em
%    T%
%    \kern-.1667em%
%    \lower.7ex\hbox{E}%
%    \kern-.125em%
%    X%
%  }%
%}
%\end{verbatim}
%\end{quote}
%    \begin{macrocode}
\def\HoLogo@BibTeX@sc#1{%
  B%
  \kern-.05em%
  \HoLogoFont@font{BibTeX}{sc}{%
    i%
    \kern-.025em%
    b%
  }%
  \HOLOGO@discretionary
  \kern-.08em%
  \hologo{TeX}%
}
%    \end{macrocode}
%    \end{macro}
%    \begin{macro}{\HoLogoHtml@BibTeX@sc}
%    \begin{macrocode}
\def\HoLogoHtml@BibTeX@sc#1{%
  \HoLogoCss@BibTeX@sc
  \HOLOGO@Span{BibTeX-sc}{%
    B%
    \HOLOGO@Span{i}{i}%
    \HOLOGO@Span{b}{b}%
    \hologo{TeX}%
  }%
}
%    \end{macrocode}
%    \end{macro}
%    \begin{macro}{\HoLogoCss@BibTeX@sc}
%    \begin{macrocode}
\def\HoLogoCss@BibTeX@sc{%
  \Css{%
    span.HoLogo-BibTeX-sc span.HoLogo-i{%
      margin-left:-.05em;%
      margin-right:-.025em;%
      font-variant:small-caps;%
    }%
  }%
  \Css{%
    span.HoLogo-BibTeX-sc span.HoLogo-b{%
      margin-right:-.08em;%
      font-variant:small-caps;%
    }%
  }%
  \global\let\HoLogoCss@BibTeX@sc\relax
}
%    \end{macrocode}
%    \end{macro}
%
%    \begin{macro}{\HoLogo@BibTeX@sf}
%    Variant \xoption{sf} avoids trouble with unavailable
%    small caps fonts (e.g., bold versions of Computer Modern or
%    Latin Modern). The definition is taken from
%    package \xpackage{dtklogos} \cite{dtklogos}.
%\begin{quote}
%\begin{verbatim}
%\DeclareRobustCommand{\BibTeX}{%
%  B%
%  \kern-.05em%
%  \hbox{%
%    $\m@th$% %% force math size calculations
%    \csname S@\f@size\endcsname
%    \fontsize\sf@size\z@
%    \math@fontsfalse
%    \selectfont
%    I%
%    \kern-.025em%
%    B
%  }%
%  \kern-.08em%
%  \-%
%  \TeX
%}
%\end{verbatim}
%\end{quote}
%    \begin{macrocode}
\def\HoLogo@BibTeX@sf#1{%
  B%
  \kern-.05em%
  \HoLogoFont@font{BibTeX}{bibsf}{%
    I%
    \kern-.025em%
    B%
  }%
  \HOLOGO@discretionary
  \kern-.08em%
  \hologo{TeX}%
}
%    \end{macrocode}
%    \end{macro}
%    \begin{macro}{\HoLogoHtml@BibTeX@sf}
%    \begin{macrocode}
\def\HoLogoHtml@BibTeX@sf#1{%
  \HoLogoCss@BibTeX@sf
  \HOLOGO@Span{BibTeX-sf}{%
    B%
    \HoLogoFont@font{BibTeX}{bibsf}{%
      \HOLOGO@Span{i}{I}%
      B%
    }%
    \hologo{TeX}%
  }%
}
%    \end{macrocode}
%    \end{macro}
%    \begin{macro}{\HoLogoCss@BibTeX@sf}
%    \begin{macrocode}
\def\HoLogoCss@BibTeX@sf{%
  \Css{%
    span.HoLogo-BibTeX-sf span.HoLogo-i{%
      margin-left:-.05em;%
      margin-right:-.025em;%
    }%
  }%
  \Css{%
    span.HoLogo-BibTeX-sf span.HoLogo-TeX{%
      margin-left:-.08em;%
    }%
  }%
  \global\let\HoLogoCss@BibTeX@sf\relax
}
%    \end{macrocode}
%    \end{macro}
%
%    \begin{macro}{\HoLogo@BibTeX}
%    \begin{macrocode}
\def\HoLogo@BibTeX{\HoLogo@BibTeX@sf}
%    \end{macrocode}
%    \end{macro}
%    \begin{macro}{\HoLogoHtml@BibTeX}
%    \begin{macrocode}
\def\HoLogoHtml@BibTeX{\HoLogoHtml@BibTeX@sf}
%    \end{macrocode}
%    \end{macro}
%
% \subsubsection{\hologo{BibTeX8}}
%
%    \begin{macro}{\HoLogo@BibTeX8}
%    \begin{macrocode}
\expandafter\def\csname HoLogo@BibTeX8\endcsname#1{%
  \hologo{BibTeX}%
  8%
}
%    \end{macrocode}
%    \end{macro}
%
%    \begin{macro}{\HoLogoBkm@BibTeX8}
%    \begin{macrocode}
\expandafter\def\csname HoLogoBkm@BibTeX8\endcsname#1{%
  \hologo{BibTeX}%
  8%
}
%    \end{macrocode}
%    \end{macro}
%    \begin{macro}{\HoLogoHtml@BibTeX8}
%    \begin{macrocode}
\expandafter
\let\csname HoLogoHtml@BibTeX8\expandafter\endcsname
\csname HoLogo@BibTeX8\endcsname
%    \end{macrocode}
%    \end{macro}
%
% \subsubsection{\hologo{ConTeXt}}
%
%    \begin{macro}{\HoLogo@ConTeXt@simple}
%    \begin{macrocode}
\def\HoLogo@ConTeXt@simple#1{%
  \HOLOGO@mbox{Con}%
  \HOLOGO@discretionary
  \HOLOGO@mbox{\hologo{TeX}t}%
}
%    \end{macrocode}
%    \end{macro}
%    \begin{macro}{\HoLogoHtml@ConTeXt@simple}
%    \begin{macrocode}
\let\HoLogoHtml@ConTeXt@simple\HoLogo@ConTeXt@simple
%    \end{macrocode}
%    \end{macro}
%
%    \begin{macro}{\HoLogo@ConTeXt@narrow}
%    This definition of logo \hologo{ConTeXt} with variant \xoption{narrow}
%    comes from TUGboat's class \xclass{ltugboat} (version 2010/11/15 v2.8).
%    \begin{macrocode}
\def\HoLogo@ConTeXt@narrow#1{%
  \HOLOGO@mbox{C\kern-.0333emon}%
  \HOLOGO@discretionary
  \kern-.0667em%
  \HOLOGO@mbox{\hologo{TeX}\kern-.0333emt}%
}
%    \end{macrocode}
%    \end{macro}
%    \begin{macro}{\HoLogoHtml@ConTeXt@narrow}
%    \begin{macrocode}
\def\HoLogoHtml@ConTeXt@narrow#1{%
  \HoLogoCss@ConTeXt@narrow
  \HOLOGO@Span{ConTeXt-narrow}{%
    \HOLOGO@Span{C}{C}%
    on%
    \hologo{TeX}%
    t%
  }%
}
%    \end{macrocode}
%    \end{macro}
%    \begin{macro}{\HoLogoCss@ConTeXt@narrow}
%    \begin{macrocode}
\def\HoLogoCss@ConTeXt@narrow{%
  \Css{%
    span.HoLogo-ConTeXt-narrow span.HoLogo-C{%
      margin-left:-.0333em;%
    }%
  }%
  \Css{%
    span.HoLogo-ConTeXt-narrow span.HoLogo-TeX{%
      margin-left:-.0667em;%
      margin-right:-.0333em;%
    }%
  }%
  \global\let\HoLogoCss@ConTeXt@narrow\relax
}
%    \end{macrocode}
%    \end{macro}
%
%    \begin{macro}{\HoLogo@ConTeXt}
%    \begin{macrocode}
\def\HoLogo@ConTeXt{\HoLogo@ConTeXt@narrow}
%    \end{macrocode}
%    \end{macro}
%    \begin{macro}{\HoLogoHtml@ConTeXt}
%    \begin{macrocode}
\def\HoLogoHtml@ConTeXt{\HoLogoHtml@ConTeXt@narrow}
%    \end{macrocode}
%    \end{macro}
%
% \subsubsection{\hologo{emTeX}}
%
%    \begin{macro}{\HoLogo@emTeX}
%    \begin{macrocode}
\def\HoLogo@emTeX#1{%
  \HOLOGO@mbox{#1{e}{E}m}%
  \HOLOGO@discretionary
  \hologo{TeX}%
}
%    \end{macrocode}
%    \end{macro}
%    \begin{macro}{\HoLogoCs@emTeX}
%    \begin{macrocode}
\def\HoLogoCs@emTeX#1{#1{e}{E}mTeX}%
%    \end{macrocode}
%    \end{macro}
%    \begin{macro}{\HoLogoBkm@emTeX}
%    \begin{macrocode}
\def\HoLogoBkm@emTeX#1{%
  #1{e}{E}m\hologo{TeX}%
}
%    \end{macrocode}
%    \end{macro}
%    \begin{macro}{\HoLogoHtml@emTeX}
%    \begin{macrocode}
\let\HoLogoHtml@emTeX\HoLogo@emTeX
%    \end{macrocode}
%    \end{macro}
%
% \subsubsection{\hologo{ExTeX}}
%
%    \begin{macro}{\HoLogo@ExTeX}
%    The definition is taken from the FAQ of the
%    project \hologo{ExTeX}
%    \cite{ExTeX-FAQ}.
%\begin{quote}
%\begin{verbatim}
%\def\ExTeX{%
%  \textrm{% Logo always with serifs
%    \ensuremath{%
%      \textstyle
%      \varepsilon_{%
%        \kern-0.15em%
%        \mathcal{X}%
%      }%
%    }%
%    \kern-.15em%
%    \TeX
%  }%
%}
%\end{verbatim}
%\end{quote}
%    \begin{macrocode}
\def\HoLogo@ExTeX#1{%
  \HoLogoFont@font{ExTeX}{rm}{%
    \ltx@mbox{%
      \HOLOGO@MathSetup
      $%
        \textstyle
        \varepsilon_{%
          \kern-0.15em%
          \HoLogoFont@font{ExTeX}{sy}{X}%
        }%
      $%
    }%
    \HOLOGO@discretionary
    \kern-.15em%
    \hologo{TeX}%
  }%
}
%    \end{macrocode}
%    \end{macro}
%    \begin{macro}{\HoLogoHtml@ExTeX}
%    \begin{macrocode}
\def\HoLogoHtml@ExTeX#1{%
  \HoLogoCss@ExTeX
  \HoLogoFont@font{ExTeX}{rm}{%
    \HOLOGO@Span{ExTeX}{%
      \ltx@mbox{%
        \HOLOGO@MathSetup
        $\textstyle\varepsilon$%
        \HOLOGO@Span{X}{$\textstyle\chi$}%
        \hologo{TeX}%
      }%
    }%
  }%
}
%    \end{macrocode}
%    \end{macro}
%    \begin{macro}{\HoLogoBkm@ExTeX}
%    \begin{macrocode}
\def\HoLogoBkm@ExTeX#1{%
  \HOLOGO@PdfdocUnicode{#1{e}{E}x}{\textepsilon\textchi}%
  \hologo{TeX}%
}
%    \end{macrocode}
%    \end{macro}
%    \begin{macro}{\HoLogoCss@ExTeX}
%    \begin{macrocode}
\def\HoLogoCss@ExTeX{%
  \Css{%
    span.HoLogo-ExTeX{%
      font-family:serif;%
    }%
  }%
  \Css{%
    span.HoLogo-ExTeX span.HoLogo-TeX{%
      margin-left:-.15em;%
    }%
  }%
  \global\let\HoLogoCss@ExTeX\relax
}
%    \end{macrocode}
%    \end{macro}
%
% \subsubsection{\hologo{MiKTeX}}
%
%    \begin{macro}{\HoLogo@MiKTeX}
%    \begin{macrocode}
\def\HoLogo@MiKTeX#1{%
  \HOLOGO@mbox{MiK}%
  \HOLOGO@discretionary
  \hologo{TeX}%
}
%    \end{macrocode}
%    \end{macro}
%    \begin{macro}{\HoLogoHtml@MiKTeX}
%    \begin{macrocode}
\let\HoLogoHtml@MiKTeX\HoLogo@MiKTeX
%    \end{macrocode}
%    \end{macro}
%
% \subsubsection{\hologo{OzTeX} and friends}
%
%    Source: \hologo{OzTeX} FAQ \cite{OzTeX}:
%    \begin{quote}
%      |\def\OzTeX{O\kern-.03em z\kern-.15em\TeX}|\\
%      (There is no kerning in OzMF, OzMP and OzTtH.)
%    \end{quote}
%
%    \begin{macro}{\HoLogo@OzTeX}
%    \begin{macrocode}
\def\HoLogo@OzTeX#1{%
  O%
  \kern-.03em %
  z%
  \kern-.15em %
  \hologo{TeX}%
}
%    \end{macrocode}
%    \end{macro}
%    \begin{macro}{\HoLogoHtml@OzTeX}
%    \begin{macrocode}
\def\HoLogoHtml@OzTeX#1{%
  \HoLogoCss@OzTeX
  \HOLOGO@Span{OzTeX}{%
    O%
    \HOLOGO@Span{z}{z}%
    \hologo{TeX}%
  }%
}
%    \end{macrocode}
%    \end{macro}
%    \begin{macro}{\HoLogoCss@OzTeX}
%    \begin{macrocode}
\def\HoLogoCss@OzTeX{%
  \Css{%
    span.HoLogo-OzTeX span.HoLogo-z{%
      margin-left:-.03em;%
      margin-right:-.15em;%
    }%
  }%
  \global\let\HoLogoCss@OzTeX\relax
}
%    \end{macrocode}
%    \end{macro}
%
%    \begin{macro}{\HoLogo@OzMF}
%    \begin{macrocode}
\def\HoLogo@OzMF#1{%
  \HOLOGO@mbox{OzMF}%
}
%    \end{macrocode}
%    \end{macro}
%    \begin{macro}{\HoLogo@OzMP}
%    \begin{macrocode}
\def\HoLogo@OzMP#1{%
  \HOLOGO@mbox{OzMP}%
}
%    \end{macrocode}
%    \end{macro}
%    \begin{macro}{\HoLogo@OzTtH}
%    \begin{macrocode}
\def\HoLogo@OzTtH#1{%
  \HOLOGO@mbox{OzTtH}%
}
%    \end{macrocode}
%    \end{macro}
%
% \subsubsection{\hologo{PCTeX}}
%
%    \begin{macro}{\HoLogo@PCTeX}
%    \begin{macrocode}
\def\HoLogo@PCTeX#1{%
  \HOLOGO@mbox{PC}%
  \hologo{TeX}%
}
%    \end{macrocode}
%    \end{macro}
%    \begin{macro}{\HoLogoHtml@PCTeX}
%    \begin{macrocode}
\let\HoLogoHtml@PCTeX\HoLogo@PCTeX
%    \end{macrocode}
%    \end{macro}
%
% \subsubsection{\hologo{PiCTeX}}
%
%    The original definitions from \xfile{pictex.tex} \cite{PiCTeX}:
%\begin{quote}
%\begin{verbatim}
%\def\PiC{%
%  P%
%  \kern-.12em%
%  \lower.5ex\hbox{I}%
%  \kern-.075em%
%  C%
%}
%\def\PiCTeX{%
%  \PiC
%  \kern-.11em%
%  \TeX
%}
%\end{verbatim}
%\end{quote}
%
%    \begin{macro}{\HoLogo@PiC}
%    \begin{macrocode}
\def\HoLogo@PiC#1{%
  P%
  \kern-.12em%
  \lower.5ex\hbox{I}%
  \kern-.075em%
  C%
  \HOLOGO@SpaceFactor
}
%    \end{macrocode}
%    \end{macro}
%    \begin{macro}{\HoLogoHtml@PiC}
%    \begin{macrocode}
\def\HoLogoHtml@PiC#1{%
  \HoLogoCss@PiC
  \HOLOGO@Span{PiC}{%
    P%
    \HOLOGO@Span{i}{I}%
    C%
  }%
}
%    \end{macrocode}
%    \end{macro}
%    \begin{macro}{\HoLogoCss@PiC}
%    \begin{macrocode}
\def\HoLogoCss@PiC{%
  \Css{%
    span.HoLogo-PiC span.HoLogo-i{%
      position:relative;%
      top:.5ex;%
      margin-left:-.12em;%
      margin-right:-.075em;%
      text-decoration:none;%
    }%
  }%
  \global\let\HoLogoCss@PiC\relax
}
%    \end{macrocode}
%    \end{macro}
%
%    \begin{macro}{\HoLogo@PiCTeX}
%    \begin{macrocode}
\def\HoLogo@PiCTeX#1{%
  \hologo{PiC}%
  \HOLOGO@discretionary
  \kern-.11em%
  \hologo{TeX}%
}
%    \end{macrocode}
%    \end{macro}
%    \begin{macro}{\HoLogoHtml@PiCTeX}
%    \begin{macrocode}
\def\HoLogoHtml@PiCTeX#1{%
  \HoLogoCss@PiCTeX
  \HOLOGO@Span{PiCTeX}{%
    \hologo{PiC}%
    \hologo{TeX}%
  }%
}
%    \end{macrocode}
%    \end{macro}
%    \begin{macro}{\HoLogoCss@PiCTeX}
%    \begin{macrocode}
\def\HoLogoCss@PiCTeX{%
  \Css{%
    span.HoLogo-PiCTeX span.HoLogo-PiC{%
      margin-right:-.11em;%
    }%
  }%
  \global\let\HoLogoCss@PiCTeX\relax
}
%    \end{macrocode}
%    \end{macro}
%
% \subsubsection{\hologo{teTeX}}
%
%    \begin{macro}{\HoLogo@teTeX}
%    \begin{macrocode}
\def\HoLogo@teTeX#1{%
  \HOLOGO@mbox{#1{t}{T}e}%
  \HOLOGO@discretionary
  \hologo{TeX}%
}
%    \end{macrocode}
%    \end{macro}
%    \begin{macro}{\HoLogoCs@teTeX}
%    \begin{macrocode}
\def\HoLogoCs@teTeX#1{#1{t}{T}dfTeX}
%    \end{macrocode}
%    \end{macro}
%    \begin{macro}{\HoLogoBkm@teTeX}
%    \begin{macrocode}
\def\HoLogoBkm@teTeX#1{%
  #1{t}{T}e\hologo{TeX}%
}
%    \end{macrocode}
%    \end{macro}
%    \begin{macro}{\HoLogoHtml@teTeX}
%    \begin{macrocode}
\let\HoLogoHtml@teTeX\HoLogo@teTeX
%    \end{macrocode}
%    \end{macro}
%
% \subsubsection{\hologo{TeX4ht}}
%
%    \begin{macro}{\HoLogo@TeX4ht}
%    \begin{macrocode}
\expandafter\def\csname HoLogo@TeX4ht\endcsname#1{%
  \HOLOGO@mbox{\hologo{TeX}4ht}%
}
%    \end{macrocode}
%    \end{macro}
%    \begin{macro}{\HoLogoHtml@TeX4ht}
%    \begin{macrocode}
\expandafter
\let\csname HoLogoHtml@TeX4ht\expandafter\endcsname
\csname HoLogo@TeX4ht\endcsname
%    \end{macrocode}
%    \end{macro}
%
%
% \subsubsection{\hologo{SageTeX}}
%
%    \begin{macro}{\HoLogo@SageTeX}
%    \begin{macrocode}
\def\HoLogo@SageTeX#1{%
  \HOLOGO@mbox{Sage}%
  \HOLOGO@discretionary
  \HOLOGO@NegativeKerning{eT,oT,To}%
  \hologo{TeX}%
}
%    \end{macrocode}
%    \end{macro}
%    \begin{macro}{\HoLogoHtml@SageTeX}
%    \begin{macrocode}
\let\HoLogoHtml@SageTeX\HoLogo@SageTeX
%    \end{macrocode}
%    \end{macro}
%
% \subsection{\hologo{METAFONT} and friends}
%
%    \begin{macro}{\HoLogo@METAFONT}
%    \begin{macrocode}
\def\HoLogo@METAFONT#1{%
  \HoLogoFont@font{METAFONT}{logo}{%
    \HOLOGO@mbox{META}%
    \HOLOGO@discretionary
    \HOLOGO@mbox{FONT}%
  }%
}
%    \end{macrocode}
%    \end{macro}
%
%    \begin{macro}{\HoLogo@METAPOST}
%    \begin{macrocode}
\def\HoLogo@METAPOST#1{%
  \HoLogoFont@font{METAPOST}{logo}{%
    \HOLOGO@mbox{META}%
    \HOLOGO@discretionary
    \HOLOGO@mbox{POST}%
  }%
}
%    \end{macrocode}
%    \end{macro}
%
%    \begin{macro}{\HoLogo@MetaFun}
%    \begin{macrocode}
\def\HoLogo@MetaFun#1{%
  \HOLOGO@mbox{Meta}%
  \HOLOGO@discretionary
  \HOLOGO@mbox{Fun}%
}
%    \end{macrocode}
%    \end{macro}
%
%    \begin{macro}{\HoLogo@MetaPost}
%    \begin{macrocode}
\def\HoLogo@MetaPost#1{%
  \HOLOGO@mbox{Meta}%
  \HOLOGO@discretionary
  \HOLOGO@mbox{Post}%
}
%    \end{macrocode}
%    \end{macro}
%
% \subsection{Others}
%
% \subsubsection{\hologo{biber}}
%
%    \begin{macro}{\HoLogo@biber}
%    \begin{macrocode}
\def\HoLogo@biber#1{%
  \HOLOGO@mbox{#1{b}{B}i}%
  \HOLOGO@discretionary
  \HOLOGO@mbox{ber}%
}
%    \end{macrocode}
%    \end{macro}
%    \begin{macro}{\HoLogoCs@biber}
%    \begin{macrocode}
\def\HoLogoCs@biber#1{#1{b}{B}iber}
%    \end{macrocode}
%    \end{macro}
%    \begin{macro}{\HoLogoBkm@biber}
%    \begin{macrocode}
\def\HoLogoBkm@biber#1{%
  #1{b}{B}iber%
}
%    \end{macrocode}
%    \end{macro}
%    \begin{macro}{\HoLogoHtml@biber}
%    \begin{macrocode}
\let\HoLogoHtml@biber\HoLogo@biber
%    \end{macrocode}
%    \end{macro}
%
% \subsubsection{\hologo{KOMAScript}}
%
%    \begin{macro}{\HoLogo@KOMAScript}
%    The definition for \hologo{KOMAScript} is taken
%    from \hologo{KOMAScript} (\xfile{scrlogo.dtx}, reformatted) \cite{scrlogo}:
%\begin{quote}
%\begin{verbatim}
%\@ifundefined{KOMAScript}{%
%  \DeclareRobustCommand{\KOMAScript}{%
%    \textsf{%
%      K\kern.05em O\kern.05emM\kern.05em A%
%      \kern.1em-\kern.1em %
%      Script%
%    }%
%  }%
%}{}
%\end{verbatim}
%\end{quote}
%    \begin{macrocode}
\def\HoLogo@KOMAScript#1{%
  \HoLogoFont@font{KOMAScript}{sf}{%
    \HOLOGO@mbox{%
      K\kern.05em%
      O\kern.05em%
      M\kern.05em%
      A%
    }%
    \kern.1em%
    \HOLOGO@hyphen
    \kern.1em%
    \HOLOGO@mbox{Script}%
  }%
}
%    \end{macrocode}
%    \end{macro}
%    \begin{macro}{\HoLogoBkm@KOMAScript}
%    \begin{macrocode}
\def\HoLogoBkm@KOMAScript#1{%
  KOMA-Script%
}
%    \end{macrocode}
%    \end{macro}
%    \begin{macro}{\HoLogoHtml@KOMAScript}
%    \begin{macrocode}
\def\HoLogoHtml@KOMAScript#1{%
  \HoLogoCss@KOMAScript
  \HoLogoFont@font{KOMAScript}{sf}{%
    \HOLOGO@Span{KOMAScript}{%
      K%
      \HOLOGO@Span{O}{O}%
      M%
      \HOLOGO@Span{A}{A}%
      \HOLOGO@Span{hyphen}{-}%
      Script%
    }%
  }%
}
%    \end{macrocode}
%    \end{macro}
%    \begin{macro}{\HoLogoCss@KOMAScript}
%    \begin{macrocode}
\def\HoLogoCss@KOMAScript{%
  \Css{%
    span.HoLogo-KOMAScript{%
      font-family:sans-serif;%
    }%
  }%
  \Css{%
    span.HoLogo-KOMAScript span.HoLogo-O{%
      padding-left:.05em;%
      padding-right:.05em;%
    }%
  }%
  \Css{%
    span.HoLogo-KOMAScript span.HoLogo-A{%
      padding-left:.05em;%
    }%
  }%
  \Css{%
    span.HoLogo-KOMAScript span.HoLogo-hyphen{%
      padding-left:.1em;%
      padding-right:.1em;%
    }%
  }%
  \global\let\HoLogoCss@KOMAScript\relax
}
%    \end{macrocode}
%    \end{macro}
%
% \subsubsection{\hologo{LyX}}
%
%    \begin{macro}{\HoLogo@LyX}
%    The definition is taken from the documentation source files
%    of \hologo{LyX}, \xfile{Intro.lyx} \cite{LyX}:
%\begin{quote}
%\begin{verbatim}
%\def\LyX{%
%  \texorpdfstring{%
%    L\kern-.1667em\lower.25em\hbox{Y}\kern-.125emX\@%
%  }{%
%    LyX%
%  }%
%}
%\end{verbatim}
%\end{quote}
%    \begin{macrocode}
\def\HoLogo@LyX#1{%
  L%
  \kern-.1667em%
  \lower.25em\hbox{Y}%
  \kern-.125em%
  X%
  \HOLOGO@SpaceFactor
}
%    \end{macrocode}
%    \end{macro}
%    \begin{macro}{\HoLogoHtml@LyX}
%    \begin{macrocode}
\def\HoLogoHtml@LyX#1{%
  \HoLogoCss@LyX
  \HOLOGO@Span{LyX}{%
    L%
    \HOLOGO@Span{y}{Y}%
    X%
  }%
}
%    \end{macrocode}
%    \end{macro}
%    \begin{macro}{\HoLogoCss@LyX}
%    \begin{macrocode}
\def\HoLogoCss@LyX{%
  \Css{%
    span.HoLogo-LyX span.HoLogo-y{%
      position:relative;%
      top:.25em;%
      margin-left:-.1667em;%
      margin-right:-.125em;%
      text-decoration:none;%
    }%
  }%
  \global\let\HoLogoCss@LyX\relax
}
%    \end{macrocode}
%    \end{macro}
%
% \subsubsection{\hologo{NTS}}
%
%    \begin{macro}{\HoLogo@NTS}
%    Definition for \hologo{NTS} can be found in
%    package \xpackage{etex\textunderscore man} for the \hologo{eTeX} manual \cite{etexman}
%    and in package \xpackage{dtklogos} \cite{dtklogos}:
%\begin{quote}
%\begin{verbatim}
%\def\NTS{%
%  \leavevmode
%  \hbox{%
%    $%
%      \cal N%
%      \kern-0.35em%
%      \lower0.5ex\hbox{$\cal T$}%
%      \kern-0.2em%
%      S%
%    $%
%  }%
%}
%\end{verbatim}
%\end{quote}
%    \begin{macrocode}
\def\HoLogo@NTS#1{%
  \HoLogoFont@font{NTS}{sy}{%
    N\/%
    \kern-.35em%
    \lower.5ex\hbox{T\/}%
    \kern-.2em%
    S\/%
  }%
  \HOLOGO@SpaceFactor
}
%    \end{macrocode}
%    \end{macro}
%
% \subsubsection{\Hologo{TTH} (\hologo{TeX} to HTML translator)}
%
%    Source: \url{http://hutchinson.belmont.ma.us/tth/}
%    In the HTML source the second `T' is printed as subscript.
%\begin{quote}
%\begin{verbatim}
%T<sub>T</sub>H
%\end{verbatim}
%\end{quote}
%    \begin{macro}{\HoLogo@TTH}
%    \begin{macrocode}
\def\HoLogo@TTH#1{%
  \ltx@mbox{%
    T\HOLOGO@SubScript{T}H%
  }%
  \HOLOGO@SpaceFactor
}
%    \end{macrocode}
%    \end{macro}
%
%    \begin{macro}{\HoLogoHtml@TTH}
%    \begin{macrocode}
\def\HoLogoHtml@TTH#1{%
  T\HCode{<sub>}T\HCode{</sub>}H%
}
%    \end{macrocode}
%    \end{macro}
%
% \subsubsection{\Hologo{HanTheThanh}}
%
%    Partial source: Package \xpackage{dtklogos}.
%    The double accent is U+1EBF (latin small letter e with circumflex
%    and acute).
%    \begin{macro}{\HoLogo@HanTheThanh}
%    \begin{macrocode}
\def\HoLogo@HanTheThanh#1{%
  \ltx@mbox{H\`an}%
  \HOLOGO@space
  \ltx@mbox{%
    Th%
    \HOLOGO@IfCharExists{"1EBF}{%
      \char"1EBF\relax
    }{%
      \^e\hbox to 0pt{\hss\raise .5ex\hbox{\'{}}}%
    }%
  }%
  \HOLOGO@space
  \ltx@mbox{Th\`anh}%
}
%    \end{macrocode}
%    \end{macro}
%    \begin{macro}{\HoLogoBkm@HanTheThanh}
%    \begin{macrocode}
\def\HoLogoBkm@HanTheThanh#1{%
  H\`an %
  Th\HOLOGO@PdfdocUnicode{\^e}{\9036\277} %
  Th\`anh%
}
%    \end{macrocode}
%    \end{macro}
%    \begin{macro}{\HoLogoHtml@HanTheThanh}
%    \begin{macrocode}
\def\HoLogoHtml@HanTheThanh#1{%
  H\`an %
  Th\HCode{&\ltx@hashchar x1ebf;} %
  Th\`anh%
}
%    \end{macrocode}
%    \end{macro}
%
% \subsection{Driver detection}
%
%    \begin{macrocode}
\HOLOGO@IfExists\InputIfFileExists{%
  \InputIfFileExists{hologo.cfg}{}{}%
}{%
  \ltx@IfUndefined{pdf@filesize}{%
    \def\HOLOGO@InputIfExists{%
      \openin\HOLOGO@temp=hologo.cfg\relax
      \ifeof\HOLOGO@temp
        \closein\HOLOGO@temp
      \else
        \closein\HOLOGO@temp
        \begingroup
          \def\x{LaTeX2e}%
        \expandafter\endgroup
        \ifx\fmtname\x
          % \iffalse meta-comment
%
% File: hologo.dtx
% Version: 2016/05/12 v1.11
% Info: A logo collection with bookmark support
%
% Copyright (C) 2010-2012 by
%    Heiko Oberdiek <heiko.oberdiek at googlemail.com>
%
% This work may be distributed and/or modified under the
% conditions of the LaTeX Project Public License, either
% version 1.3c of this license or (at your option) any later
% version. This version of this license is in
%    http://www.latex-project.org/lppl/lppl-1-3c.txt
% and the latest version of this license is in
%    http://www.latex-project.org/lppl.txt
% and version 1.3 or later is part of all distributions of
% LaTeX version 2005/12/01 or later.
%
% This work has the LPPL maintenance status "maintained".
%
% This Current Maintainer of this work is Heiko Oberdiek.
%
% The Base Interpreter refers to any `TeX-Format',
% because some files are installed in TDS:tex/generic//.
%
% This work consists of the main source file hologo.dtx
% and the derived files
%    hologo.sty, hologo.pdf, hologo.ins, hologo.drv, hologo-example.tex,
%    hologo-test1.tex, hologo-test-spacefactor.tex,
%    hologo-test-list.tex.
%
% Distribution:
%    CTAN:macros/latex/contrib/oberdiek/hologo.dtx
%    CTAN:macros/latex/contrib/oberdiek/hologo.pdf
%
% Unpacking:
%    (a) If hologo.ins is present:
%           tex hologo.ins
%    (b) Without hologo.ins:
%           tex hologo.dtx
%    (c) If you insist on using LaTeX
%           latex \let\install=y% \iffalse meta-comment
%
% File: hologo.dtx
% Version: 2016/05/12 v1.11
% Info: A logo collection with bookmark support
%
% Copyright (C) 2010-2012 by
%    Heiko Oberdiek <heiko.oberdiek at googlemail.com>
%
% This work may be distributed and/or modified under the
% conditions of the LaTeX Project Public License, either
% version 1.3c of this license or (at your option) any later
% version. This version of this license is in
%    http://www.latex-project.org/lppl/lppl-1-3c.txt
% and the latest version of this license is in
%    http://www.latex-project.org/lppl.txt
% and version 1.3 or later is part of all distributions of
% LaTeX version 2005/12/01 or later.
%
% This work has the LPPL maintenance status "maintained".
%
% This Current Maintainer of this work is Heiko Oberdiek.
%
% The Base Interpreter refers to any `TeX-Format',
% because some files are installed in TDS:tex/generic//.
%
% This work consists of the main source file hologo.dtx
% and the derived files
%    hologo.sty, hologo.pdf, hologo.ins, hologo.drv, hologo-example.tex,
%    hologo-test1.tex, hologo-test-spacefactor.tex,
%    hologo-test-list.tex.
%
% Distribution:
%    CTAN:macros/latex/contrib/oberdiek/hologo.dtx
%    CTAN:macros/latex/contrib/oberdiek/hologo.pdf
%
% Unpacking:
%    (a) If hologo.ins is present:
%           tex hologo.ins
%    (b) Without hologo.ins:
%           tex hologo.dtx
%    (c) If you insist on using LaTeX
%           latex \let\install=y\input{hologo.dtx}
%        (quote the arguments according to the demands of your shell)
%
% Documentation:
%    (a) If hologo.drv is present:
%           latex hologo.drv
%    (b) Without hologo.drv:
%           latex hologo.dtx; ...
%    The class ltxdoc loads the configuration file ltxdoc.cfg
%    if available. Here you can specify further options, e.g.
%    use A4 as paper format:
%       \PassOptionsToClass{a4paper}{article}
%
%    Programm calls to get the documentation (example):
%       pdflatex hologo.dtx
%       makeindex -s gind.ist hologo.idx
%       pdflatex hologo.dtx
%       makeindex -s gind.ist hologo.idx
%       pdflatex hologo.dtx
%
% Installation:
%    TDS:tex/generic/oberdiek/hologo.sty
%    TDS:doc/latex/oberdiek/hologo.pdf
%    TDS:doc/latex/oberdiek/example/hologo-example.tex
%    TDS:doc/latex/oberdiek/test/hologo-test1.tex
%    TDS:doc/latex/oberdiek/test/hologo-test-spacefactor.tex
%    TDS:doc/latex/oberdiek/test/hologo-test-list.tex
%    TDS:source/latex/oberdiek/hologo.dtx
%
%<*ignore>
\begingroup
  \catcode123=1 %
  \catcode125=2 %
  \def\x{LaTeX2e}%
\expandafter\endgroup
\ifcase 0\ifx\install y1\fi\expandafter
         \ifx\csname processbatchFile\endcsname\relax\else1\fi
         \ifx\fmtname\x\else 1\fi\relax
\else\csname fi\endcsname
%</ignore>
%<*install>
\input docstrip.tex
\Msg{************************************************************************}
\Msg{* Installation}
\Msg{* Package: hologo 2016/05/12 v1.11 A logo collection with bookmark support (HO)}
\Msg{************************************************************************}

\keepsilent
\askforoverwritefalse

\let\MetaPrefix\relax
\preamble

This is a generated file.

Project: hologo
Version: 2016/05/12 v1.11

Copyright (C) 2010-2012 by
   Heiko Oberdiek <heiko.oberdiek at googlemail.com>

This work may be distributed and/or modified under the
conditions of the LaTeX Project Public License, either
version 1.3c of this license or (at your option) any later
version. This version of this license is in
   http://www.latex-project.org/lppl/lppl-1-3c.txt
and the latest version of this license is in
   http://www.latex-project.org/lppl.txt
and version 1.3 or later is part of all distributions of
LaTeX version 2005/12/01 or later.

This work has the LPPL maintenance status "maintained".

This Current Maintainer of this work is Heiko Oberdiek.

The Base Interpreter refers to any `TeX-Format',
because some files are installed in TDS:tex/generic//.

This work consists of the main source file hologo.dtx
and the derived files
   hologo.sty, hologo.pdf, hologo.ins, hologo.drv, hologo-example.tex,
   hologo-test1.tex, hologo-test-spacefactor.tex,
   hologo-test-list.tex.

\endpreamble
\let\MetaPrefix\DoubleperCent

\generate{%
  \file{hologo.ins}{\from{hologo.dtx}{install}}%
  \file{hologo.drv}{\from{hologo.dtx}{driver}}%
  \usedir{tex/generic/oberdiek}%
  \file{hologo.sty}{\from{hologo.dtx}{package}}%
  \usedir{doc/latex/oberdiek/example}%
  \file{hologo-example.tex}{\from{hologo.dtx}{example}}%
  \usedir{doc/latex/oberdiek/test}%
  \file{hologo-test1.tex}{\from{hologo.dtx}{test1}}%
  \file{hologo-test-spacefactor.tex}{\from{hologo.dtx}{test-spacefactor}}%
  \file{hologo-test-list.tex}{\from{hologo.dtx}{test-list}}%
  \nopreamble
  \nopostamble
  \usedir{source/latex/oberdiek/catalogue}%
  \file{hologo.xml}{\from{hologo.dtx}{catalogue}}%
}

\catcode32=13\relax% active space
\let =\space%
\Msg{************************************************************************}
\Msg{*}
\Msg{* To finish the installation you have to move the following}
\Msg{* file into a directory searched by TeX:}
\Msg{*}
\Msg{*     hologo.sty}
\Msg{*}
\Msg{* To produce the documentation run the file `hologo.drv'}
\Msg{* through LaTeX.}
\Msg{*}
\Msg{* Happy TeXing!}
\Msg{*}
\Msg{************************************************************************}

\endbatchfile
%</install>
%<*ignore>
\fi
%</ignore>
%<*driver>
\NeedsTeXFormat{LaTeX2e}
\ProvidesFile{hologo.drv}%
  [2016/05/12 v1.11 A logo collection with bookmark support (HO)]%
\documentclass{ltxdoc}
\usepackage{holtxdoc}[2011/11/22]
\usepackage{hologo}[2016/05/12]
\usepackage{longtable}
\usepackage{array}
\usepackage{paralist}
%\usepackage[T1]{fontenc}
%\usepackage{lmodern}
\begin{document}
  \DocInput{hologo.dtx}%
\end{document}
%</driver>
% \fi
%
%
% \CharacterTable
%  {Upper-case    \A\B\C\D\E\F\G\H\I\J\K\L\M\N\O\P\Q\R\S\T\U\V\W\X\Y\Z
%   Lower-case    \a\b\c\d\e\f\g\h\i\j\k\l\m\n\o\p\q\r\s\t\u\v\w\x\y\z
%   Digits        \0\1\2\3\4\5\6\7\8\9
%   Exclamation   \!     Double quote  \"     Hash (number) \#
%   Dollar        \$     Percent       \%     Ampersand     \&
%   Acute accent  \'     Left paren    \(     Right paren   \)
%   Asterisk      \*     Plus          \+     Comma         \,
%   Minus         \-     Point         \.     Solidus       \/
%   Colon         \:     Semicolon     \;     Less than     \<
%   Equals        \=     Greater than  \>     Question mark \?
%   Commercial at \@     Left bracket  \[     Backslash     \\
%   Right bracket \]     Circumflex    \^     Underscore    \_
%   Grave accent  \`     Left brace    \{     Vertical bar  \|
%   Right brace   \}     Tilde         \~}
%
% \GetFileInfo{hologo.drv}
%
% \title{The \xpackage{hologo} package}
% \date{2016/05/12 v1.11}
% \author{Heiko Oberdiek\\\xemail{heiko.oberdiek at googlemail.com}}
%
% \maketitle
%
% \begin{abstract}
% This package starts a collection of logos with support for bookmarks
% strings.
% \end{abstract}
%
% \tableofcontents
%
% \section{Documentation}
%
% \subsection{Logo macros}
%
% \begin{declcs}{hologo} \M{name}
% \end{declcs}
% Macro \cs{hologo} sets the logo with name \meta{name}.
% The following table shows the supported names.
%
% \begingroup
%   \def\hologoEntry#1#2#3{^^A
%     #1&#2&\hologoLogoSetup{#1}{variant=#2}\hologo{#1}&#3\tabularnewline
%   }
%   \begin{longtable}{>{\ttfamily}l>{\ttfamily}lll}
%     \rmfamily\bfseries{name} & \rmfamily\bfseries variant
%     & \bfseries logo & \bfseries since\\
%     \hline
%     \endhead
%     \hologoList
%   \end{longtable}
% \endgroup
%
% \begin{declcs}{Hologo} \M{name}
% \end{declcs}
% Macro \cs{Hologo} starts the logo \meta{name} with an uppercase
% letter. As an exception small greek letters are not converted
% to uppercase. Examples, see \hologo{eTeX} and \hologo{ExTeX}.
%
% \subsection{Setup macros}
%
% The package does not support package options, but the following
% setup macros can be used to set options.
%
% \begin{declcs}{hologoSetup} \M{key value list}
% \end{declcs}
% Macro \cs{hologoSetup} sets global options.
%
% \begin{declcs}{hologoLogoSetup} \M{logo} \M{key value list}
% \end{declcs}
% Some options can also be used to configure a logo.
% These settings take precedence over global option settings.
%
% \subsection{Options}\label{sec:options}
%
% There are boolean and string options:
% \begin{description}
% \item[Boolean option:]
% It takes |true| or |false|
% as value. If the value is omitted, then |true| is used.
% \item[String option:]
% A value must be given as string. (But the string might be empty.)
% \end{description}
% The following options can be used both in \cs{hologoSetup}
% and \cs{hologoLogoSetup}:
% \begin{description}
% \def\entry#1{\item[\xoption{#1}:]}
% \entry{break}
%   enables or disables line breaks inside the logo. This setting is
%   refined by options \xoption{hyphenbreak}, \xoption{spacebreak}
%   or \xoption{discretionarybreak}.
%   Default is |false|.
% \entry{hyphenbreak}
%   enables or disables the line break right after the hyphen character.
% \entry{spacebreak}
%   enables or disables line breaks at space characters.
% \entry{discretionarybreak}
%   enables or disables line breaks at hyphenation points
%   (inserted by \cs{-}).
% \end{description}
% Macro \cs{hologoLogoSetup} also knows:
% \begin{description}
% \item[\xoption{variant}:]
%   This is a string option. It specifies a variant of a logo that
%   must exist. An empty string selects the package default variant.
% \end{description}
% Example:
% \begin{quote}
%   |\hologoSetup{break=false}|\\
%   |\hologoLogoSetup{plainTeX}{variant=hyphen,hyphenbreak}|\\
%   Then ``plain-\TeX'' contains one break point after the hyphen.
% \end{quote}
%
% \subsection{Driver options}
%
% Sometimes graphical operations are needed to construct some
% glyphs (e.g.\ \hologo{XeTeX}). If package \xpackage{graphics}
% or package \xpackage{pgf} are found, then the macros are taken
% from there. Otherwise the packge defines its own operations
% and therefore needs the driver information. Many drivers are
% detected automatically (\hologo{pdfTeX}/\hologo{LuaTeX}
% in PDF mode, \hologo{XeTeX}, \hologo{VTeX}). These have precedence
% over a driver option. The driver can be given as package option
% or using \cs{hologoDriverSetup}.
% The following list contains the recognized driver options:
% \begin{itemize}
% \item \xoption{pdftex}, \xoption{luatex}
% \item \xoption{dvipdfm}, \xoption{dvipdfmx}
% \item \xoption{dvips}, \xoption{dvipsone}, \xoption{xdvi}
% \item \xoption{xetex}
% \item \xoption{vtex}
% \end{itemize}
% The left driver of a line is the driver name that is used internally.
% The following names are aliases for drivers that use the
% same method. Therefore the entry in the \xext{log} file for
% the used driver prints the internally used driver name.
% \begin{description}
% \item[\xoption{driverfallback}:]
%   This option expects a driver that is used,
%   if the driver could not be detected automatically.
% \end{description}
%
% \begin{declcs}{hologoDriverSetup} \M{driver option}
% \end{declcs}
% The driver can also be configured after package loading
% using \cs{hologoDriverSetup}, also the way for \hologo{plainTeX}
% to setup the driver.
%
% \subsection{Font setup}
%
% Some logos require a special font, but should also be usable by
% \hologo{plainTeX}. Therefore the package provides some ways
% to influence the font settings. The options below
% take font settings as values. Both font commands
% such as \cs{sffamily} and macros that take one argument
% like \cs{textsf} can be used.
%
% \begin{declcs}{hologoFontSetup} \M{key value list}
% \end{declcs}
% Macro \cs{hologoFontSetup} sets the fonts for all logos.
% Supported keys:
% \begin{description}
% \def\entry#1{\item[\xoption{#1}:]}
% \entry{general}
%   This font is used for all logos. The default is empty.
%   That means no special font is used.
% \entry{bibsf}
%   This font is used for
%   {\hologoLogoSetup{BibTeX}{variant=sf}\hologo{BibTeX}}
%   with variant \xoption{sf}.
% \entry{rm}
%   This font is a serif font. It is used for \hologo{ExTeX}.
% \entry{sc}
%   This font specifies a small caps font. It is used for
%   {\hologoLogoSetup{BibTeX}{variant=sc}\hologo{BibTeX}}
%   with variant \xoption{sc}.
% \entry{sf}
%   This font specifies a sans serif font. The default
%   is \cs{sffamily}, then \cs{sf} is tried. Otherwise
%   a warning is given. It is used by \hologo{KOMAScript}.
% \entry{sy}
%   This is the font for math symbols (e.g. cmsy).
%   It is used by \hologo{AmS}, \hologo{NTS}, \hologo{ExTeX}.
% \entry{logo}
%   \hologo{METAFONT} and \hologo{METAPOST} are using that font.
%   In \hologo{LaTeX} \cs{logofamily} is used and
%   the definitions of package \xpackage{mflogo} are used
%   if the package is not loaded.
%   Otherwise the \cs{tenlogo} is used and defined
%   if it does not already exists.
% \end{description}
%
% \begin{declcs}{hologoLogoFontSetup} \M{logo} \M{key value list}
% \end{declcs}
% Fonts can also be set for a logo or logo component separately,
% see the following list.
% The keys are the same as for \cs{hologoFontSetup}.
%
% \begin{longtable}{>{\ttfamily}l>{\sffamily}ll}
%   \meta{logo} & keys & result\\
%   \hline
%   \endhead
%   BibTeX & bibsf & {\hologoLogoSetup{BibTeX}{variant=sf}\hologo{BibTeX}}\\[.5ex]
%   BibTeX & sc & {\hologoLogoSetup{BibTeX}{variant=sc}\hologo{BibTeX}}\\[.5ex]
%   ExTeX & rm & \hologo{ExTeX}\\
%   SliTeX & rm & \hologo{SliTeX}\\[.5ex]
%   AmS & sy & \hologo{AmS}\\
%   ExTeX & sy & \hologo{ExTeX}\\
%   NTS & sy & \hologo{NTS}\\[.5ex]
%   KOMAScript & sf & \hologo{KOMAScript}\\[.5ex]
%   METAFONT & logo & \hologo{METAFONT}\\
%   METAPOST & logo & \hologo{METAPOST}\\[.5ex]
%   SliTeX & sc \hologo{SliTeX}
% \end{longtable}
%
% \subsubsection{Font order}
%
% For all logos the font \xoption{general} is applied first.
% Example:
%\begin{quote}
%|\hologoFontSetup{general=\color{red}}|
%\end{quote}
% will print red logos.
% Then if the font uses a special font \xoption{sf}, for example,
% the font is applied that is setup by \cs{hologoLogoFontSetup}.
% If this font is not setup, then the common font setup
% by \cs{hologoFontSetup} is used. Otherwise a warning is given,
% that there is no font configured.
%
% \subsection{Additional user macros}
%
% Usually a variant of a logo is configured by using
% \cs{hologoLogoSetup}, because it is bad style to mix
% different variants of the same logo in the same text.
% There the following macros are a convenience for testing.
%
% \begin{declcs}{hologoVariant} \M{name} \M{variant}\\
%   \cs{HologoVariant} \M{name} \M{variant}
% \end{declcs}
% Logo \meta{name} is set using \meta{variant} that specifies
% explicitely which variant of the macro is used. If the argument
% is empty, then the default form of the logo is used
% (configurable by \cs{hologoLogoSetup}).
%
% \cs{HologoVariant} is used if the logo is set in a context
% that needs an uppercase first letter (beginning of a sentence, \dots).
%
% \begin{declcs}{hologoList}\\
%   \cs{hologoEntry} \M{logo} \M{variant} \M{since}
% \end{declcs}
% Macro \cs{hologoList} contains all logos that are provided
% by the package including variants. The list consists of calls
% of \cs{hologoEntry} with three arguments starting with the
% logo name \meta{logo} and its variant \meta{variant}. An empty
% variant means the current default. Argument \meta{since} specifies
% with version of the package \xpackage{hologo} is needed to get
% the logo. If the logo is fixed, then the date gets updated.
% Therefore the date \meta{since} is not exactly the date of
% the first introduction, but rather the date of the latest fix.
%
% Before \cs{hologoList} can be used, macro \cs{hologoEntry} needs
% a definition. The example file in section \ref{sec:example}
% shows applications of \cs{hologoList}.
%
% \subsection{Supported contexts}
%
% Macros \cs{hologo} and friends support special contexts:
% \begin{itemize}
% \item \hologo{LaTeX}'s protection mechanism.
% \item Bookmarks of package \xpackage{hyperref}.
% \item Package \xpackage{tex4ht}.
% \item The macros can be used inside \cs{csname} constructs,
%   if \cs{ifincsname} is available (\hologo{pdfTeX}, \hologo{XeTeX},
%   \hologo{LuaTeX}).
% \end{itemize}
%
% \subsection{Example}
% \label{sec:example}
%
% The following example prints the logos in different fonts.
%    \begin{macrocode}
%<*example>
%<<verbatim
\NeedsTeXFormat{LaTeX2e}
\documentclass[a4paper]{article}
\usepackage[
  hmargin=20mm,
  vmargin=20mm,
]{geometry}
\pagestyle{empty}
\usepackage{hologo}[2016/05/12]
\usepackage{longtable}
\usepackage{array}
\setlength{\extrarowheight}{2pt}
\usepackage[T1]{fontenc}
\usepackage{lmodern}
\usepackage{pdflscape}
\usepackage[
  pdfencoding=auto,
]{hyperref}
\hypersetup{
  pdfauthor={Heiko Oberdiek},
  pdftitle={Example for package `hologo'},
  pdfsubject={Logos with fonts lmr, lmss, qtm, qpl, qhv},
}
\usepackage{bookmark}

% Print the logo list on the console

\begingroup
  \typeout{}%
  \typeout{*** Begin of logo list ***}%
  \newcommand*{\hologoEntry}[3]{%
    \typeout{#1 \ifx\\#2\\\else(#2) \fi[#3]}%
  }%
  \hologoList
  \typeout{*** End of logo list ***}%
  \typeout{}%
\endgroup

\begin{document}
\begin{landscape}

  \section{Example file for package `hologo'}

  % Table for font names

  \begin{longtable}{>{\bfseries}ll}
    \textbf{font} & \textbf{Font name}\\
    \hline
    lmr & Latin Modern Roman\\
    lmss & Latin Modern Sans\\
    qtm & \TeX\ Gyre Termes\\
    qhv & \TeX\ Gyre Heros\\
    qpl & \TeX\ Gyre Pagella\\
  \end{longtable}

  % Logo list with logos in different fonts

  \begingroup
    \newcommand*{\SetVariant}[2]{%
      \ifx\\#2\\%
      \else
        \hologoLogoSetup{#1}{variant=#2}%
      \fi
    }%
    \newcommand*{\hologoEntry}[3]{%
      \SetVariant{#1}{#2}%
      \raisebox{1em}[0pt][0pt]{\hypertarget{#1@#2}{}}%
      \bookmark[%
        dest={#1@#2},%
      ]{%
        #1\ifx\\#2\\\else\space(#2)\fi: \Hologo{#1}, \hologo{#1} %
        [Unicode]%
      }%
      \hypersetup{unicode=false}%
      \bookmark[%
        dest={#1@#2},%
      ]{%
        #1\ifx\\#2\\\else\space(#2)\fi: \Hologo{#1}, \hologo{#1} %
        [PDFDocEncoding]%
      }%
      \texttt{#1}%
      &%
      \texttt{#2}%
      &%
      \Hologo{#1}%
      &%
      \SetVariant{#1}{#2}%
      \hologo{#1}%
      &%
      \SetVariant{#1}{#2}%
      \fontfamily{qtm}\selectfont
      \hologo{#1}%
      &%
      \SetVariant{#1}{#2}%
      \fontfamily{qpl}\selectfont
      \hologo{#1}%
      &%
      \SetVariant{#1}{#2}%
      \textsf{\hologo{#1}}%
      &%
      \SetVariant{#1}{#2}%
      \fontfamily{qhv}\selectfont
      \hologo{#1}%
      \tabularnewline
    }%
    \begin{longtable}{llllllll}%
      \textbf{\textit{logo}} & \textbf{\textit{variant}} &
      \texttt{\string\Hologo} &
      \textbf{lmr} & \textbf{qtm} & \textbf{qpl} &
      \textbf{lmss} & \textbf{qhv}
      \tabularnewline
      \hline
      \endhead
      \hologoList
    \end{longtable}%
  \endgroup

\end{landscape}
\end{document}
%verbatim
%</example>
%    \end{macrocode}
%
% \StopEventually{
% }
%
% \section{Implementation}
%    \begin{macrocode}
%<*package>
%    \end{macrocode}
%    Reload check, especially if the package is not used with \LaTeX.
%    \begin{macrocode}
\begingroup\catcode61\catcode48\catcode32=10\relax%
  \catcode13=5 % ^^M
  \endlinechar=13 %
  \catcode35=6 % #
  \catcode39=12 % '
  \catcode44=12 % ,
  \catcode45=12 % -
  \catcode46=12 % .
  \catcode58=12 % :
  \catcode64=11 % @
  \catcode123=1 % {
  \catcode125=2 % }
  \expandafter\let\expandafter\x\csname ver@hologo.sty\endcsname
  \ifx\x\relax % plain-TeX, first loading
  \else
    \def\empty{}%
    \ifx\x\empty % LaTeX, first loading,
      % variable is initialized, but \ProvidesPackage not yet seen
    \else
      \expandafter\ifx\csname PackageInfo\endcsname\relax
        \def\x#1#2{%
          \immediate\write-1{Package #1 Info: #2.}%
        }%
      \else
        \def\x#1#2{\PackageInfo{#1}{#2, stopped}}%
      \fi
      \x{hologo}{The package is already loaded}%
      \aftergroup\endinput
    \fi
  \fi
\endgroup%
%    \end{macrocode}
%    Package identification:
%    \begin{macrocode}
\begingroup\catcode61\catcode48\catcode32=10\relax%
  \catcode13=5 % ^^M
  \endlinechar=13 %
  \catcode35=6 % #
  \catcode39=12 % '
  \catcode40=12 % (
  \catcode41=12 % )
  \catcode44=12 % ,
  \catcode45=12 % -
  \catcode46=12 % .
  \catcode47=12 % /
  \catcode58=12 % :
  \catcode64=11 % @
  \catcode91=12 % [
  \catcode93=12 % ]
  \catcode123=1 % {
  \catcode125=2 % }
  \expandafter\ifx\csname ProvidesPackage\endcsname\relax
    \def\x#1#2#3[#4]{\endgroup
      \immediate\write-1{Package: #3 #4}%
      \xdef#1{#4}%
    }%
  \else
    \def\x#1#2[#3]{\endgroup
      #2[{#3}]%
      \ifx#1\@undefined
        \xdef#1{#3}%
      \fi
      \ifx#1\relax
        \xdef#1{#3}%
      \fi
    }%
  \fi
\expandafter\x\csname ver@hologo.sty\endcsname
\ProvidesPackage{hologo}%
  [2016/05/12 v1.11 A logo collection with bookmark support (HO)]%
%    \end{macrocode}
%
%    \begin{macrocode}
\begingroup\catcode61\catcode48\catcode32=10\relax%
  \catcode13=5 % ^^M
  \endlinechar=13 %
  \catcode123=1 % {
  \catcode125=2 % }
  \catcode64=11 % @
  \def\x{\endgroup
    \expandafter\edef\csname HOLOGO@AtEnd\endcsname{%
      \endlinechar=\the\endlinechar\relax
      \catcode13=\the\catcode13\relax
      \catcode32=\the\catcode32\relax
      \catcode35=\the\catcode35\relax
      \catcode61=\the\catcode61\relax
      \catcode64=\the\catcode64\relax
      \catcode123=\the\catcode123\relax
      \catcode125=\the\catcode125\relax
    }%
  }%
\x\catcode61\catcode48\catcode32=10\relax%
\catcode13=5 % ^^M
\endlinechar=13 %
\catcode35=6 % #
\catcode64=11 % @
\catcode123=1 % {
\catcode125=2 % }
\def\TMP@EnsureCode#1#2{%
  \edef\HOLOGO@AtEnd{%
    \HOLOGO@AtEnd
    \catcode#1=\the\catcode#1\relax
  }%
  \catcode#1=#2\relax
}
\TMP@EnsureCode{10}{12}% ^^J
\TMP@EnsureCode{33}{12}% !
\TMP@EnsureCode{34}{12}% "
\TMP@EnsureCode{36}{3}% $
\TMP@EnsureCode{38}{4}% &
\TMP@EnsureCode{39}{12}% '
\TMP@EnsureCode{40}{12}% (
\TMP@EnsureCode{41}{12}% )
\TMP@EnsureCode{42}{12}% *
\TMP@EnsureCode{43}{12}% +
\TMP@EnsureCode{44}{12}% ,
\TMP@EnsureCode{45}{12}% -
\TMP@EnsureCode{46}{12}% .
\TMP@EnsureCode{47}{12}% /
\TMP@EnsureCode{58}{12}% :
\TMP@EnsureCode{59}{12}% ;
\TMP@EnsureCode{60}{12}% <
\TMP@EnsureCode{62}{12}% >
\TMP@EnsureCode{63}{12}% ?
\TMP@EnsureCode{91}{12}% [
\TMP@EnsureCode{93}{12}% ]
\TMP@EnsureCode{94}{7}% ^ (superscript)
\TMP@EnsureCode{95}{8}% _ (subscript)
\TMP@EnsureCode{96}{12}% `
\TMP@EnsureCode{124}{12}% |
\edef\HOLOGO@AtEnd{%
  \HOLOGO@AtEnd
  \escapechar\the\escapechar\relax
  \noexpand\endinput
}
\escapechar=92 %
%    \end{macrocode}
%
% \subsection{Logo list}
%
%    \begin{macro}{\hologoList}
%    \begin{macrocode}
\def\hologoList{%
  \hologoEntry{(La)TeX}{}{2011/10/01}%
  \hologoEntry{AmSLaTeX}{}{2010/04/16}%
  \hologoEntry{AmSTeX}{}{2010/04/16}%
  \hologoEntry{biber}{}{2011/10/01}%
  \hologoEntry{BibTeX}{}{2011/10/01}%
  \hologoEntry{BibTeX}{sf}{2011/10/01}%
  \hologoEntry{BibTeX}{sc}{2011/10/01}%
  \hologoEntry{BibTeX8}{}{2011/11/22}%
  \hologoEntry{ConTeXt}{}{2011/03/25}%
  \hologoEntry{ConTeXt}{narrow}{2011/03/25}%
  \hologoEntry{ConTeXt}{simple}{2011/03/25}%
  \hologoEntry{emTeX}{}{2010/04/26}%
  \hologoEntry{eTeX}{}{2010/04/08}%
  \hologoEntry{ExTeX}{}{2011/10/01}%
  \hologoEntry{HanTheThanh}{}{2011/11/29}%
  \hologoEntry{iniTeX}{}{2011/10/01}%
  \hologoEntry{KOMAScript}{}{2011/10/01}%
  \hologoEntry{La}{}{2010/05/08}%
  \hologoEntry{LaTeX}{}{2010/04/08}%
  \hologoEntry{LaTeX2e}{}{2010/04/08}%
  \hologoEntry{LaTeX3}{}{2010/04/24}%
  \hologoEntry{LaTeXe}{}{2010/04/08}%
  \hologoEntry{LaTeXML}{}{2011/11/22}%
  \hologoEntry{LaTeXTeX}{}{2011/10/01}%
  \hologoEntry{LuaLaTeX}{}{2010/04/08}%
  \hologoEntry{LuaTeX}{}{2010/04/08}%
  \hologoEntry{LyX}{}{2011/10/01}%
  \hologoEntry{METAFONT}{}{2011/10/01}%
  \hologoEntry{MetaFun}{}{2011/10/01}%
  \hologoEntry{METAPOST}{}{2011/10/01}%
  \hologoEntry{MetaPost}{}{2011/10/01}%
  \hologoEntry{MiKTeX}{}{2011/10/01}%
  \hologoEntry{NTS}{}{2011/10/01}%
  \hologoEntry{OzMF}{}{2011/10/01}%
  \hologoEntry{OzMP}{}{2011/10/01}%
  \hologoEntry{OzTeX}{}{2011/10/01}%
  \hologoEntry{OzTtH}{}{2011/10/01}%
  \hologoEntry{PCTeX}{}{2011/10/01}%
  \hologoEntry{pdfTeX}{}{2011/10/01}%
  \hologoEntry{pdfLaTeX}{}{2011/10/01}%
  \hologoEntry{PiC}{}{2011/10/01}%
  \hologoEntry{PiCTeX}{}{2011/10/01}%
  \hologoEntry{plainTeX}{}{2010/04/08}%
  \hologoEntry{plainTeX}{space}{2010/04/16}%
  \hologoEntry{plainTeX}{hyphen}{2010/04/16}%
  \hologoEntry{plainTeX}{runtogether}{2010/04/16}%
  \hologoEntry{SageTeX}{}{2011/11/22}%
  \hologoEntry{SLiTeX}{}{2011/10/01}%
  \hologoEntry{SLiTeX}{lift}{2011/10/01}%
  \hologoEntry{SLiTeX}{narrow}{2011/10/01}%
  \hologoEntry{SLiTeX}{simple}{2011/10/01}%
  \hologoEntry{SliTeX}{}{2011/10/01}%
  \hologoEntry{SliTeX}{narrow}{2011/10/01}%
  \hologoEntry{SliTeX}{simple}{2011/10/01}%
  \hologoEntry{SliTeX}{lift}{2011/10/01}%
  \hologoEntry{teTeX}{}{2011/10/01}%
  \hologoEntry{TeX}{}{2010/04/08}%
  \hologoEntry{TeX4ht}{}{2011/11/22}%
  \hologoEntry{TTH}{}{2011/11/22}%
  \hologoEntry{virTeX}{}{2011/10/01}%
  \hologoEntry{VTeX}{}{2010/04/24}%
  \hologoEntry{Xe}{}{2010/04/08}%
  \hologoEntry{XeLaTeX}{}{2010/04/08}%
  \hologoEntry{XeTeX}{}{2010/04/08}%
}
%    \end{macrocode}
%    \end{macro}
%
% \subsection{Load resources}
%
%    \begin{macrocode}
\begingroup\expandafter\expandafter\expandafter\endgroup
\expandafter\ifx\csname RequirePackage\endcsname\relax
  \def\TMP@RequirePackage#1[#2]{%
    \begingroup\expandafter\expandafter\expandafter\endgroup
    \expandafter\ifx\csname ver@#1.sty\endcsname\relax
      \input #1.sty\relax
    \fi
  }%
  \TMP@RequirePackage{ltxcmds}[2011/02/04]%
  \TMP@RequirePackage{infwarerr}[2010/04/08]%
  \TMP@RequirePackage{kvsetkeys}[2010/03/01]%
  \TMP@RequirePackage{kvdefinekeys}[2010/03/01]%
  \TMP@RequirePackage{pdftexcmds}[2010/04/01]%
  \TMP@RequirePackage{ifpdf}[2010/01/28]%
  \TMP@RequirePackage{ifluatex}[2010/03/01]%
  \ltx@IfUndefined{newif}{%
    \expandafter\let\csname newif\endcsname\ltx@newif
  }{}%
  \TMP@RequirePackage{ifxetex}[2009/01/23]%
  \TMP@RequirePackage{ifvtex}[2010/03/01]%
\else
  \RequirePackage{ltxcmds}[2011/02/04]%
  \RequirePackage{infwarerr}[2010/04/08]%
  \RequirePackage{kvsetkeys}[2010/03/01]%
  \RequirePackage{kvdefinekeys}[2010/03/01]%
  \RequirePackage{pdftexcmds}[2010/04/01]%
  \RequirePackage{ifpdf}[2010/01/28]%
  \RequirePackage{ifluatex}[2010/03/01]%
  \RequirePackage{ifxetex}[2009/01/23]%
  \RequirePackage{ifvtex}[2010/03/01]%
\fi
%    \end{macrocode}
%
%    \begin{macro}{\HOLOGO@IfDefined}
%    \begin{macrocode}
\def\HOLOGO@IfExists#1{%
  \ifx\@undefined#1%
    \expandafter\ltx@secondoftwo
  \else
    \ifx\relax#1%
      \expandafter\ltx@secondoftwo
    \else
      \expandafter\expandafter\expandafter\ltx@firstoftwo
    \fi
  \fi
}
%    \end{macrocode}
%    \end{macro}
%
% \subsection{Setup macros}
%
%    \begin{macro}{\hologoSetup}
%    \begin{macrocode}
\def\hologoSetup{%
  \let\HOLOGO@name\relax
  \HOLOGO@Setup
}
%    \end{macrocode}
%    \end{macro}
%
%    \begin{macro}{\hologoLogoSetup}
%    \begin{macrocode}
\def\hologoLogoSetup#1{%
  \edef\HOLOGO@name{#1}%
  \ltx@IfUndefined{HoLogo@\HOLOGO@name}{%
    \@PackageError{hologo}{%
      Unknown logo `\HOLOGO@name'%
    }\@ehc
    \ltx@gobble
  }{%
    \HOLOGO@Setup
  }%
}
%    \end{macrocode}
%    \end{macro}
%
%    \begin{macro}{\HOLOGO@Setup}
%    \begin{macrocode}
\def\HOLOGO@Setup{%
  \kvsetkeys{HoLogo}%
}
%    \end{macrocode}
%    \end{macro}
%
% \subsection{Options}
%
%    \begin{macro}{\HOLOGO@DeclareBoolOption}
%    \begin{macrocode}
\def\HOLOGO@DeclareBoolOption#1{%
  \expandafter\chardef\csname HOLOGOOPT@#1\endcsname\ltx@zero
  \kv@define@key{HoLogo}{#1}[true]{%
    \def\HOLOGO@temp{##1}%
    \ifx\HOLOGO@temp\HOLOGO@true
      \ifx\HOLOGO@name\relax
        \expandafter\chardef\csname HOLOGOOPT@#1\endcsname=\ltx@one
      \else
        \expandafter\chardef\csname
        HoLogoOpt@#1@\HOLOGO@name\endcsname\ltx@one
      \fi
      \HOLOGO@SetBreakAll{#1}%
    \else
      \ifx\HOLOGO@temp\HOLOGO@false
        \ifx\HOLOGO@name\relax
          \expandafter\chardef\csname HOLOGOOPT@#1\endcsname=\ltx@zero
        \else
          \expandafter\chardef\csname
          HoLogoOpt@#1@\HOLOGO@name\endcsname=\ltx@zero
        \fi
        \HOLOGO@SetBreakAll{#1}%
      \else
        \@PackageError{hologo}{%
          Unknown value `##1' for boolean option `#1'.\MessageBreak
          Known values are `true' and `false'%
        }\@ehc
      \fi
    \fi
  }%
}
%    \end{macrocode}
%    \end{macro}
%
%    \begin{macro}{\HOLOGO@SetBreakAll}
%    \begin{macrocode}
\def\HOLOGO@SetBreakAll#1{%
  \def\HOLOGO@temp{#1}%
  \ifx\HOLOGO@temp\HOLOGO@break
    \ifx\HOLOGO@name\relax
      \chardef\HOLOGOOPT@hyphenbreak=\HOLOGOOPT@break
      \chardef\HOLOGOOPT@spacebreak=\HOLOGOOPT@break
      \chardef\HOLOGOOPT@discretionarybreak=\HOLOGOOPT@break
    \else
      \expandafter\chardef
         \csname HoLogoOpt@hyphenbreak@\HOLOGO@name\endcsname=%
         \csname HoLogoOpt@break@\HOLOGO@name\endcsname
      \expandafter\chardef
         \csname HoLogoOpt@spacebreak@\HOLOGO@name\endcsname=%
         \csname HoLogoOpt@break@\HOLOGO@name\endcsname
      \expandafter\chardef
         \csname HoLogoOpt@discretionarybreak@\HOLOGO@name
             \endcsname=%
         \csname HoLogoOpt@break@\HOLOGO@name\endcsname
    \fi
  \fi
}
%    \end{macrocode}
%    \end{macro}
%
%    \begin{macro}{\HOLOGO@true}
%    \begin{macrocode}
\def\HOLOGO@true{true}
%    \end{macrocode}
%    \end{macro}
%    \begin{macro}{\HOLOGO@false}
%    \begin{macrocode}
\def\HOLOGO@false{false}
%    \end{macrocode}
%    \end{macro}
%    \begin{macro}{\HOLOGO@break}
%    \begin{macrocode}
\def\HOLOGO@break{break}
%    \end{macrocode}
%    \end{macro}
%
%    \begin{macrocode}
\HOLOGO@DeclareBoolOption{break}
\HOLOGO@DeclareBoolOption{hyphenbreak}
\HOLOGO@DeclareBoolOption{spacebreak}
\HOLOGO@DeclareBoolOption{discretionarybreak}
%    \end{macrocode}
%
%    \begin{macrocode}
\kv@define@key{HoLogo}{variant}{%
  \ifx\HOLOGO@name\relax
    \@PackageError{hologo}{%
      Option `variant' is not available in \string\hologoSetup,%
      \MessageBreak
      Use \string\hologoLogoSetup\space instead%
    }\@ehc
  \else
    \edef\HOLOGO@temp{#1}%
    \ifx\HOLOGO@temp\ltx@empty
      \expandafter
      \let\csname HoLogoOpt@variant@\HOLOGO@name\endcsname\@undefined
    \else
      \ltx@IfUndefined{HoLogo@\HOLOGO@name @\HOLOGO@temp}{%
        \@PackageError{hologo}{%
          Unknown variant `\HOLOGO@temp' of logo `\HOLOGO@name'%
        }\@ehc
      }{%
        \expandafter
        \let\csname HoLogoOpt@variant@\HOLOGO@name\endcsname
            \HOLOGO@temp
      }%
    \fi
  \fi
}
%    \end{macrocode}
%
%    \begin{macro}{\HOLOGO@Variant}
%    \begin{macrocode}
\def\HOLOGO@Variant#1{%
  #1%
  \ltx@ifundefined{HoLogoOpt@variant@#1}{%
  }{%
    @\csname HoLogoOpt@variant@#1\endcsname
  }%
}
%    \end{macrocode}
%    \end{macro}
%
% \subsection{Break/no-break support}
%
%    \begin{macro}{\HOLOGO@space}
%    \begin{macrocode}
\def\HOLOGO@space{%
  \ltx@ifundefined{HoLogoOpt@spacebreak@\HOLOGO@name}{%
    \ltx@ifundefined{HoLogoOpt@break@\HOLOGO@name}{%
      \chardef\HOLOGO@temp=\HOLOGOOPT@spacebreak
    }{%
      \chardef\HOLOGO@temp=%
        \csname HoLogoOpt@break@\HOLOGO@name\endcsname
    }%
  }{%
    \chardef\HOLOGO@temp=%
      \csname HoLogoOpt@spacebreak@\HOLOGO@name\endcsname
  }%
  \ifcase\HOLOGO@temp
    \penalty10000 %
  \fi
  \ltx@space
}
%    \end{macrocode}
%    \end{macro}
%
%    \begin{macro}{\HOLOGO@hyphen}
%    \begin{macrocode}
\def\HOLOGO@hyphen{%
  \ltx@ifundefined{HoLogoOpt@hyphenbreak@\HOLOGO@name}{%
    \ltx@ifundefined{HoLogoOpt@break@\HOLOGO@name}{%
      \chardef\HOLOGO@temp=\HOLOGOOPT@hyphenbreak
    }{%
      \chardef\HOLOGO@temp=%
        \csname HoLogoOpt@break@\HOLOGO@name\endcsname
    }%
  }{%
    \chardef\HOLOGO@temp=%
      \csname HoLogoOpt@hyphenbreak@\HOLOGO@name\endcsname
  }%
  \ifcase\HOLOGO@temp
    \ltx@mbox{-}%
  \else
    -%
  \fi
}
%    \end{macrocode}
%    \end{macro}
%
%    \begin{macro}{\HOLOGO@discretionary}
%    \begin{macrocode}
\def\HOLOGO@discretionary{%
  \ltx@ifundefined{HoLogoOpt@discretionarybreak@\HOLOGO@name}{%
    \ltx@ifundefined{HoLogoOpt@break@\HOLOGO@name}{%
      \chardef\HOLOGO@temp=\HOLOGOOPT@discretionarybreak
    }{%
      \chardef\HOLOGO@temp=%
        \csname HoLogoOpt@break@\HOLOGO@name\endcsname
    }%
  }{%
    \chardef\HOLOGO@temp=%
      \csname HoLogoOpt@discretionarybreak@\HOLOGO@name\endcsname
  }%
  \ifcase\HOLOGO@temp
  \else
    \-%
  \fi
}
%    \end{macrocode}
%    \end{macro}
%
%    \begin{macro}{\HOLOGO@mbox}
%    \begin{macrocode}
\def\HOLOGO@mbox#1{%
  \ltx@ifundefined{HoLogoOpt@break@\HOLOGO@name}{%
    \chardef\HOLOGO@temp=\HOLOGOOPT@hyphenbreak
  }{%
    \chardef\HOLOGO@temp=%
      \csname HoLogoOpt@break@\HOLOGO@name\endcsname
  }%
  \ifcase\HOLOGO@temp
    \ltx@mbox{#1}%
  \else
    #1%
  \fi
}
%    \end{macrocode}
%    \end{macro}
%
% \subsection{Font support}
%
%    \begin{macro}{\HoLogoFont@font}
%    \begin{tabular}{@{}ll@{}}
%    |#1|:& logo name\\
%    |#2|:& font short name\\
%    |#3|:& text
%    \end{tabular}
%    \begin{macrocode}
\def\HoLogoFont@font#1#2#3{%
  \begingroup
    \ltx@IfUndefined{HoLogoFont@logo@#1.#2}{%
      \ltx@IfUndefined{HoLogoFont@font@#2}{%
        \@PackageWarning{hologo}{%
          Missing font `#2' for logo `#1'%
        }%
        #3%
      }{%
        \csname HoLogoFont@font@#2\endcsname{#3}%
      }%
    }{%
      \csname HoLogoFont@logo@#1.#2\endcsname{#3}%
    }%
  \endgroup
}
%    \end{macrocode}
%    \end{macro}
%
%    \begin{macro}{\HoLogoFont@Def}
%    \begin{macrocode}
\def\HoLogoFont@Def#1{%
  \expandafter\def\csname HoLogoFont@font@#1\endcsname
}
%    \end{macrocode}
%    \end{macro}
%    \begin{macro}{\HoLogoFont@LogoDef}
%    \begin{macrocode}
\def\HoLogoFont@LogoDef#1#2{%
  \expandafter\def\csname HoLogoFont@logo@#1.#2\endcsname
}
%    \end{macrocode}
%    \end{macro}
%
% \subsubsection{Font defaults}
%
%    \begin{macro}{\HoLogoFont@font@general}
%    \begin{macrocode}
\HoLogoFont@Def{general}{}%
%    \end{macrocode}
%    \end{macro}
%
%    \begin{macro}{\HoLogoFont@font@rm}
%    \begin{macrocode}
\ltx@IfUndefined{rmfamily}{%
  \ltx@IfUndefined{rm}{%
  }{%
    \HoLogoFont@Def{rm}{\rm}%
  }%
}{%
  \HoLogoFont@Def{rm}{\rmfamily}%
}
%    \end{macrocode}
%    \end{macro}
%
%    \begin{macro}{\HoLogoFont@font@sf}
%    \begin{macrocode}
\ltx@IfUndefined{sffamily}{%
  \ltx@IfUndefined{sf}{%
  }{%
    \HoLogoFont@Def{sf}{\sf}%
  }%
}{%
  \HoLogoFont@Def{sf}{\sffamily}%
}
%    \end{macrocode}
%    \end{macro}
%
%    \begin{macro}{\HoLogoFont@font@bibsf}
%    In case of \hologo{plainTeX} the original small caps
%    variant is used as default. In \hologo{LaTeX}
%    the definition of package \xpackage{dtklogos} \cite{dtklogos}
%    is used.
%\begin{quote}
%\begin{verbatim}
%\DeclareRobustCommand{\BibTeX}{%
%  B%
%  \kern-.05em%
%  \hbox{%
%    $\m@th$% %% force math size calculations
%    \csname S@\f@size\endcsname
%    \fontsize\sf@size\z@
%    \math@fontsfalse
%    \selectfont
%    I%
%    \kern-.025em%
%    B
%  }%
%  \kern-.08em%
%  \-%
%  \TeX
%}
%\end{verbatim}
%\end{quote}
%    \begin{macrocode}
\ltx@IfUndefined{selectfont}{%
  \ltx@IfUndefined{tensc}{%
    \font\tensc=cmcsc10\relax
  }{}%
  \HoLogoFont@Def{bibsf}{\tensc}%
}{%
  \HoLogoFont@Def{bibsf}{%
    $\mathsurround=0pt$%
    \csname S@\f@size\endcsname
    \fontsize\sf@size{0pt}%
    \math@fontsfalse
    \selectfont
  }%
}
%    \end{macrocode}
%    \end{macro}
%
%    \begin{macro}{\HoLogoFont@font@sc}
%    \begin{macrocode}
\ltx@IfUndefined{scshape}{%
  \ltx@IfUndefined{tensc}{%
    \font\tensc=cmcsc10\relax
  }{}%
  \HoLogoFont@Def{sc}{\tensc}%
}{%
  \HoLogoFont@Def{sc}{\scshape}%
}
%    \end{macrocode}
%    \end{macro}
%
%    \begin{macro}{\HoLogoFont@font@sy}
%    \begin{macrocode}
\ltx@IfUndefined{usefont}{%
  \ltx@IfUndefined{tensy}{%
  }{%
    \HoLogoFont@Def{sy}{\tensy}%
  }%
}{%
  \HoLogoFont@Def{sy}{%
    \usefont{OMS}{cmsy}{m}{n}%
  }%
}
%    \end{macrocode}
%    \end{macro}
%
%    \begin{macro}{\HoLogoFont@font@logo}
%    \begin{macrocode}
\begingroup
  \def\x{LaTeX2e}%
\expandafter\endgroup
\ifx\fmtname\x
  \ltx@IfUndefined{logofamily}{%
    \DeclareRobustCommand\logofamily{%
      \not@math@alphabet\logofamily\relax
      \fontencoding{U}%
      \fontfamily{logo}%
      \selectfont
    }%
  }{}%
  \ltx@IfUndefined{logofamily}{%
  }{%
    \HoLogoFont@Def{logo}{\logofamily}%
  }%
\else
  \ltx@IfUndefined{tenlogo}{%
    \font\tenlogo=logo10\relax
  }{}%
  \HoLogoFont@Def{logo}{\tenlogo}%
\fi
%    \end{macrocode}
%    \end{macro}
%
% \subsubsection{Font setup}
%
%    \begin{macro}{\hologoFontSetup}
%    \begin{macrocode}
\def\hologoFontSetup{%
  \let\HOLOGO@name\relax
  \HOLOGO@FontSetup
}
%    \end{macrocode}
%    \end{macro}
%
%    \begin{macro}{\hologoLogoFontSetup}
%    \begin{macrocode}
\def\hologoLogoFontSetup#1{%
  \edef\HOLOGO@name{#1}%
  \ltx@IfUndefined{HoLogo@\HOLOGO@name}{%
    \@PackageError{hologo}{%
      Unknown logo `\HOLOGO@name'%
    }\@ehc
    \ltx@gobble
  }{%
    \HOLOGO@FontSetup
  }%
}
%    \end{macrocode}
%    \end{macro}
%
%    \begin{macro}{\HOLOGO@FontSetup}
%    \begin{macrocode}
\def\HOLOGO@FontSetup{%
  \kvsetkeys{HoLogoFont}%
}
%    \end{macrocode}
%    \end{macro}
%
%    \begin{macrocode}
\def\HOLOGO@temp#1{%
  \kv@define@key{HoLogoFont}{#1}{%
    \ifx\HOLOGO@name\relax
      \HoLogoFont@Def{#1}{##1}%
    \else
      \HoLogoFont@LogoDef\HOLOGO@name{#1}{##1}%
    \fi
  }%
}
\HOLOGO@temp{general}
\HOLOGO@temp{sf}
%    \end{macrocode}
%
% \subsection{Generic logo commands}
%
%    \begin{macrocode}
\HOLOGO@IfExists\hologo{%
  \@PackageError{hologo}{%
    \string\hologo\ltx@space is already defined.\MessageBreak
    Package loading is aborted%
  }\@ehc
  \HOLOGO@AtEnd
}%
\HOLOGO@IfExists\hologoRobust{%
  \@PackageError{hologo}{%
    \string\hologoRobust\ltx@space is already defined.\MessageBreak
    Package loading is aborted%
  }\@ehc
  \HOLOGO@AtEnd
}%
%    \end{macrocode}
%
% \subsubsection{\cs{hologo} and friends}
%
%    \begin{macrocode}
\ifluatex
  \expandafter\ltx@firstofone
\else
  \expandafter\ltx@gobble
\fi
{%
  \ltx@IfUndefined{ifincsname}{%
    \ifnum\luatexversion<36 %
      \expandafter\ltx@gobble
    \else
      \expandafter\ltx@firstofone
    \fi
    {%
      \begingroup
        \ifcase0%
            \directlua{%
              if tex.enableprimitives then %
                tex.enableprimitives('HOLOGO@', {'ifincsname'})%
              else %
                tex.print('1')%
              end%
            }%
            \ifx\HOLOGO@ifincsname\@undefined 1\fi%
            \relax
          \expandafter\ltx@firstofone
        \else
          \endgroup
          \expandafter\ltx@gobble
        \fi
        {%
          \global\let\ifincsname\HOLOGO@ifincsname
        }%
      \HOLOGO@temp
    }%
  }{}%
}
%    \end{macrocode}
%    \begin{macrocode}
\ltx@IfUndefined{ifincsname}{%
  \catcode`$=14 %
}{%
  \catcode`$=9 %
}
%    \end{macrocode}
%
%    \begin{macro}{\hologo}
%    \begin{macrocode}
\def\hologo#1{%
$ \ifincsname
$   \ltx@ifundefined{HoLogoCs@\HOLOGO@Variant{#1}}{%
$     #1%
$   }{%
$     \csname HoLogoCs@\HOLOGO@Variant{#1}\endcsname\ltx@firstoftwo
$   }%
$ \else
    \HOLOGO@IfExists\texorpdfstring\texorpdfstring\ltx@firstoftwo
    {%
      \hologoRobust{#1}%
    }{%
      \ltx@ifundefined{HoLogoBkm@\HOLOGO@Variant{#1}}{%
        \ltx@ifundefined{HoLogo@#1}{?#1?}{#1}%
      }{%
        \csname HoLogoBkm@\HOLOGO@Variant{#1}\endcsname
        \ltx@firstoftwo
      }%
    }%
$ \fi
}
%    \end{macrocode}
%    \end{macro}
%    \begin{macro}{\Hologo}
%    \begin{macrocode}
\def\Hologo#1{%
$ \ifincsname
$   \ltx@ifundefined{HoLogoCs@\HOLOGO@Variant{#1}}{%
$     #1%
$   }{%
$     \csname HoLogoCs@\HOLOGO@Variant{#1}\endcsname\ltx@secondoftwo
$   }%
$ \else
    \HOLOGO@IfExists\texorpdfstring\texorpdfstring\ltx@firstoftwo
    {%
      \HologoRobust{#1}%
    }{%
      \ltx@ifundefined{HoLogoBkm@\HOLOGO@Variant{#1}}{%
        \ltx@ifundefined{HoLogo@#1}{?#1?}{#1}%
      }{%
        \csname HoLogoBkm@\HOLOGO@Variant{#1}\endcsname
        \ltx@secondoftwo
      }%
    }%
$ \fi
}
%    \end{macrocode}
%    \end{macro}
%
%    \begin{macro}{\hologoVariant}
%    \begin{macrocode}
\def\hologoVariant#1#2{%
  \ifx\relax#2\relax
    \hologo{#1}%
  \else
$   \ifincsname
$     \ltx@ifundefined{HoLogoCs@#1@#2}{%
$       #1%
$     }{%
$       \csname HoLogoCs@#1@#2\endcsname\ltx@firstoftwo
$     }%
$   \else
      \HOLOGO@IfExists\texorpdfstring\texorpdfstring\ltx@firstoftwo
      {%
        \hologoVariantRobust{#1}{#2}%
      }{%
        \ltx@ifundefined{HoLogoBkm@#1@#2}{%
          \ltx@ifundefined{HoLogo@#1}{?#1?}{#1}%
        }{%
          \csname HoLogoBkm@#1@#2\endcsname
          \ltx@firstoftwo
        }%
      }%
$   \fi
  \fi
}
%    \end{macrocode}
%    \end{macro}
%    \begin{macro}{\HologoVariant}
%    \begin{macrocode}
\def\HologoVariant#1#2{%
  \ifx\relax#2\relax
    \Hologo{#1}%
  \else
$   \ifincsname
$     \ltx@ifundefined{HoLogoCs@#1@#2}{%
$       #1%
$     }{%
$       \csname HoLogoCs@#1@#2\endcsname\ltx@secondoftwo
$     }%
$   \else
      \HOLOGO@IfExists\texorpdfstring\texorpdfstring\ltx@firstoftwo
      {%
        \HologoVariantRobust{#1}{#2}%
      }{%
        \ltx@ifundefined{HoLogoBkm@#1@#2}{%
          \ltx@ifundefined{HoLogo@#1}{?#1?}{#1}%
        }{%
          \csname HoLogoBkm@#1@#2\endcsname
          \ltx@secondoftwo
        }%
      }%
$   \fi
  \fi
}
%    \end{macrocode}
%    \end{macro}
%
%    \begin{macrocode}
\catcode`\$=3 %
%    \end{macrocode}
%
% \subsubsection{\cs{hologoRobust} and friends}
%
%    \begin{macro}{\hologoRobust}
%    \begin{macrocode}
\ltx@IfUndefined{protected}{%
  \ltx@IfUndefined{DeclareRobustCommand}{%
    \def\hologoRobust#1%
  }{%
    \DeclareRobustCommand*\hologoRobust[1]%
  }%
}{%
  \protected\def\hologoRobust#1%
}%
{%
  \edef\HOLOGO@name{#1}%
  \ltx@IfUndefined{HoLogo@\HOLOGO@Variant\HOLOGO@name}{%
    \@PackageError{hologo}{%
      Unknown logo `\HOLOGO@name'%
    }\@ehc
    ?\HOLOGO@name?%
  }{%
    \ltx@IfUndefined{ver@tex4ht.sty}{%
      \HoLogoFont@font\HOLOGO@name{general}{%
        \csname HoLogo@\HOLOGO@Variant\HOLOGO@name\endcsname
        \ltx@firstoftwo
      }%
    }{%
      \ltx@IfUndefined{HoLogoHtml@\HOLOGO@Variant\HOLOGO@name}{%
        \HOLOGO@name
      }{%
        \csname HoLogoHtml@\HOLOGO@Variant\HOLOGO@name\endcsname
        \ltx@firstoftwo
      }%
    }%
  }%
}
%    \end{macrocode}
%    \end{macro}
%    \begin{macro}{\HologoRobust}
%    \begin{macrocode}
\ltx@IfUndefined{protected}{%
  \ltx@IfUndefined{DeclareRobustCommand}{%
    \def\HologoRobust#1%
  }{%
    \DeclareRobustCommand*\HologoRobust[1]%
  }%
}{%
  \protected\def\HologoRobust#1%
}%
{%
  \edef\HOLOGO@name{#1}%
  \ltx@IfUndefined{HoLogo@\HOLOGO@Variant\HOLOGO@name}{%
    \@PackageError{hologo}{%
      Unknown logo `\HOLOGO@name'%
    }\@ehc
    ?\HOLOGO@name?%
  }{%
    \ltx@IfUndefined{ver@tex4ht.sty}{%
      \HoLogoFont@font\HOLOGO@name{general}{%
        \csname HoLogo@\HOLOGO@Variant\HOLOGO@name\endcsname
        \ltx@secondoftwo
      }%
    }{%
      \ltx@IfUndefined{HoLogoHtml@\HOLOGO@Variant\HOLOGO@name}{%
        \expandafter\HOLOGO@Uppercase\HOLOGO@name
      }{%
        \csname HoLogoHtml@\HOLOGO@Variant\HOLOGO@name\endcsname
        \ltx@secondoftwo
      }%
    }%
  }%
}
%    \end{macrocode}
%    \end{macro}
%    \begin{macro}{\hologoVariantRobust}
%    \begin{macrocode}
\ltx@IfUndefined{protected}{%
  \ltx@IfUndefined{DeclareRobustCommand}{%
    \def\hologoVariantRobust#1#2%
  }{%
    \DeclareRobustCommand*\hologoVariantRobust[2]%
  }%
}{%
  \protected\def\hologoVariantRobust#1#2%
}%
{%
  \begingroup
    \hologoLogoSetup{#1}{variant={#2}}%
    \hologoRobust{#1}%
  \endgroup
}
%    \end{macrocode}
%    \end{macro}
%    \begin{macro}{\HologoVariantRobust}
%    \begin{macrocode}
\ltx@IfUndefined{protected}{%
  \ltx@IfUndefined{DeclareRobustCommand}{%
    \def\HologoVariantRobust#1#2%
  }{%
    \DeclareRobustCommand*\HologoVariantRobust[2]%
  }%
}{%
  \protected\def\HologoVariantRobust#1#2%
}%
{%
  \begingroup
    \hologoLogoSetup{#1}{variant={#2}}%
    \HologoRobust{#1}%
  \endgroup
}
%    \end{macrocode}
%    \end{macro}
%
%    \begin{macro}{\hologorobust}
%    Macro \cs{hologorobust} is only defined for compatibility.
%    Its use is deprecated.
%    \begin{macrocode}
\def\hologorobust{\hologoRobust}
%    \end{macrocode}
%    \end{macro}
%
% \subsection{Helpers}
%
%    \begin{macro}{\HOLOGO@Uppercase}
%    Macro \cs{HOLOGO@Uppercase} is restricted to \cs{uppercase},
%    because \hologo{plainTeX} or \hologo{iniTeX} do not provide
%    \cs{MakeUppercase}.
%    \begin{macrocode}
\def\HOLOGO@Uppercase#1{\uppercase{#1}}
%    \end{macrocode}
%    \end{macro}
%
%    \begin{macro}{\HOLOGO@PdfdocUnicode}
%    \begin{macrocode}
\def\HOLOGO@PdfdocUnicode{%
  \ifx\ifHy@unicode\iftrue
    \expandafter\ltx@secondoftwo
  \else
    \expandafter\ltx@firstoftwo
  \fi
}
%    \end{macrocode}
%    \end{macro}
%
%    \begin{macro}{\HOLOGO@Math}
%    \begin{macrocode}
\def\HOLOGO@MathSetup{%
  \mathsurround0pt\relax
  \HOLOGO@IfExists\f@series{%
    \if b\expandafter\ltx@car\f@series x\@nil
      \csname boldmath\endcsname
   \fi
  }{}%
}
%    \end{macrocode}
%    \end{macro}
%
%    \begin{macro}{\HOLOGO@TempDimen}
%    \begin{macrocode}
\dimendef\HOLOGO@TempDimen=\ltx@zero
%    \end{macrocode}
%    \end{macro}
%    \begin{macro}{\HOLOGO@NegativeKerning}
%    \begin{macrocode}
\def\HOLOGO@NegativeKerning#1{%
  \begingroup
    \HOLOGO@TempDimen=0pt\relax
    \comma@parse@normalized{#1}{%
      \ifdim\HOLOGO@TempDimen=0pt %
        \expandafter\HOLOGO@@NegativeKerning\comma@entry
      \fi
      \ltx@gobble
    }%
    \ifdim\HOLOGO@TempDimen<0pt %
      \kern\HOLOGO@TempDimen
    \fi
  \endgroup
}
%    \end{macrocode}
%    \end{macro}
%    \begin{macro}{\HOLOGO@@NegativeKerning}
%    \begin{macrocode}
\def\HOLOGO@@NegativeKerning#1#2{%
  \setbox\ltx@zero\hbox{#1#2}%
  \HOLOGO@TempDimen=\wd\ltx@zero
  \setbox\ltx@zero\hbox{#1\kern0pt#2}%
  \advance\HOLOGO@TempDimen by -\wd\ltx@zero
}
%    \end{macrocode}
%    \end{macro}
%
%    \begin{macro}{\HOLOGO@SpaceFactor}
%    \begin{macrocode}
\def\HOLOGO@SpaceFactor{%
  \spacefactor1000 %
}
%    \end{macrocode}
%    \end{macro}
%
%    \begin{macro}{\HOLOGO@Span}
%    \begin{macrocode}
\def\HOLOGO@Span#1#2{%
  \HCode{<span class="HoLogo-#1">}%
  #2%
  \HCode{</span>}%
}
%    \end{macrocode}
%    \end{macro}
%
% \subsubsection{Text subscript}
%
%    \begin{macro}{\HOLOGO@SubScript}%
%    \begin{macrocode}
\def\HOLOGO@SubScript#1{%
  \ltx@IfUndefined{textsubscript}{%
    \ltx@IfUndefined{text}{%
      \ltx@mbox{%
        \mathsurround=0pt\relax
        $%
          _{%
            \ltx@IfUndefined{sf@size}{%
              \mathrm{#1}%
            }{%
              \mbox{%
                \fontsize\sf@size{0pt}\selectfont
                #1%
              }%
            }%
          }%
        $%
      }%
    }{%
      \ltx@mbox{%
        \mathsurround=0pt\relax
        $_{\text{#1}}$%
      }%
    }%
  }{%
    \textsubscript{#1}%
  }%
}
%    \end{macrocode}
%    \end{macro}
%
% \subsection{\hologo{TeX} and friends}
%
% \subsubsection{\hologo{TeX}}
%
%    \begin{macro}{\HoLogo@TeX}
%    Source: \hologo{LaTeX} kernel.
%    \begin{macrocode}
\def\HoLogo@TeX#1{%
  T\kern-.1667em\lower.5ex\hbox{E}\kern-.125emX\HOLOGO@SpaceFactor
}
%    \end{macrocode}
%    \end{macro}
%    \begin{macro}{\HoLogoHtml@TeX}
%    \begin{macrocode}
\def\HoLogoHtml@TeX#1{%
  \HoLogoCss@TeX
  \HOLOGO@Span{TeX}{%
    T%
    \HOLOGO@Span{e}{%
      E%
    }%
    X%
  }%
}
%    \end{macrocode}
%    \end{macro}
%    \begin{macro}{\HoLogoCss@TeX}
%    \begin{macrocode}
\def\HoLogoCss@TeX{%
  \Css{%
    span.HoLogo-TeX span.HoLogo-e{%
      position:relative;%
      top:.5ex;%
      margin-left:-.1667em;%
      margin-right:-.125em;%
    }%
  }%
  \Css{%
    a span.HoLogo-TeX span.HoLogo-e{%
      text-decoration:none;%
    }%
  }%
  \global\let\HoLogoCss@TeX\relax
}
%    \end{macrocode}
%    \end{macro}
%
% \subsubsection{\hologo{plainTeX}}
%
%    \begin{macro}{\HoLogo@plainTeX@space}
%    Source: ``The \hologo{TeX}book''
%    \begin{macrocode}
\def\HoLogo@plainTeX@space#1{%
  \HOLOGO@mbox{#1{p}{P}lain}\HOLOGO@space\hologo{TeX}%
}
%    \end{macrocode}
%    \end{macro}
%    \begin{macro}{\HoLogoCs@plainTeX@space}
%    \begin{macrocode}
\def\HoLogoCs@plainTeX@space#1{#1{p}{P}lain TeX}%
%    \end{macrocode}
%    \end{macro}
%    \begin{macro}{\HoLogoBkm@plainTeX@space}
%    \begin{macrocode}
\def\HoLogoBkm@plainTeX@space#1{%
  #1{p}{P}lain \hologo{TeX}%
}
%    \end{macrocode}
%    \end{macro}
%    \begin{macro}{\HoLogoHtml@plainTeX@space}
%    \begin{macrocode}
\def\HoLogoHtml@plainTeX@space#1{%
  #1{p}{P}lain \hologo{TeX}%
}
%    \end{macrocode}
%    \end{macro}
%
%    \begin{macro}{\HoLogo@plainTeX@hyphen}
%    \begin{macrocode}
\def\HoLogo@plainTeX@hyphen#1{%
  \HOLOGO@mbox{#1{p}{P}lain}\HOLOGO@hyphen\hologo{TeX}%
}
%    \end{macrocode}
%    \end{macro}
%    \begin{macro}{\HoLogoCs@plainTeX@hyphen}
%    \begin{macrocode}
\def\HoLogoCs@plainTeX@hyphen#1{#1{p}{P}lain-TeX}
%    \end{macrocode}
%    \end{macro}
%    \begin{macro}{\HoLogoBkm@plainTeX@hyphen}
%    \begin{macrocode}
\def\HoLogoBkm@plainTeX@hyphen#1{%
  #1{p}{P}lain-\hologo{TeX}%
}
%    \end{macrocode}
%    \end{macro}
%    \begin{macro}{\HoLogoHtml@plainTeX@hyphen}
%    \begin{macrocode}
\def\HoLogoHtml@plainTeX@hyphen#1{%
  #1{p}{P}lain-\hologo{TeX}%
}
%    \end{macrocode}
%    \end{macro}
%
%    \begin{macro}{\HoLogo@plainTeX@runtogether}
%    \begin{macrocode}
\def\HoLogo@plainTeX@runtogether#1{%
  \HOLOGO@mbox{#1{p}{P}lain\hologo{TeX}}%
}
%    \end{macrocode}
%    \end{macro}
%    \begin{macro}{\HoLogoCs@plainTeX@runtogether}
%    \begin{macrocode}
\def\HoLogoCs@plainTeX@runtogether#1{#1{p}{P}lainTeX}
%    \end{macrocode}
%    \end{macro}
%    \begin{macro}{\HoLogoBkm@plainTeX@runtogether}
%    \begin{macrocode}
\def\HoLogoBkm@plainTeX@runtogether#1{%
  #1{p}{P}lain\hologo{TeX}%
}
%    \end{macrocode}
%    \end{macro}
%    \begin{macro}{\HoLogoHtml@plainTeX@runtogether}
%    \begin{macrocode}
\def\HoLogoHtml@plainTeX@runtogether#1{%
  #1{p}{P}lain\hologo{TeX}%
}
%    \end{macrocode}
%    \end{macro}
%
%    \begin{macro}{\HoLogo@plainTeX}
%    \begin{macrocode}
\def\HoLogo@plainTeX{\HoLogo@plainTeX@space}
%    \end{macrocode}
%    \end{macro}
%    \begin{macro}{\HoLogoCs@plainTeX}
%    \begin{macrocode}
\def\HoLogoCs@plainTeX{\HoLogoCs@plainTeX@space}
%    \end{macrocode}
%    \end{macro}
%    \begin{macro}{\HoLogoBkm@plainTeX}
%    \begin{macrocode}
\def\HoLogoBkm@plainTeX{\HoLogoBkm@plainTeX@space}
%    \end{macrocode}
%    \end{macro}
%    \begin{macro}{\HoLogoHtml@plainTeX}
%    \begin{macrocode}
\def\HoLogoHtml@plainTeX{\HoLogoHtml@plainTeX@space}
%    \end{macrocode}
%    \end{macro}
%
% \subsubsection{\hologo{LaTeX}}
%
%    Source: \hologo{LaTeX} kernel.
%\begin{quote}
%\begin{verbatim}
%\DeclareRobustCommand{\LaTeX}{%
%  L%
%  \kern-.36em%
%  {%
%    \sbox\z@ T%
%    \vbox to\ht\z@{%
%      \hbox{%
%        \check@mathfonts
%        \fontsize\sf@size\z@
%        \math@fontsfalse
%        \selectfont
%        A%
%      }%
%      \vss
%    }%
%  }%
%  \kern-.15em%
%  \TeX
%}
%\end{verbatim}
%\end{quote}
%
%    \begin{macro}{\HoLogo@La}
%    \begin{macrocode}
\def\HoLogo@La#1{%
  L%
  \kern-.36em%
  \begingroup
    \setbox\ltx@zero\hbox{T}%
    \vbox to\ht\ltx@zero{%
      \hbox{%
        \ltx@ifundefined{check@mathfonts}{%
          \csname sevenrm\endcsname
        }{%
          \check@mathfonts
          \fontsize\sf@size{0pt}%
          \math@fontsfalse\selectfont
        }%
        A%
      }%
      \vss
    }%
  \endgroup
}
%    \end{macrocode}
%    \end{macro}
%
%    \begin{macro}{\HoLogo@LaTeX}
%    Source: \hologo{LaTeX} kernel.
%    \begin{macrocode}
\def\HoLogo@LaTeX#1{%
  \hologo{La}%
  \kern-.15em%
  \hologo{TeX}%
}
%    \end{macrocode}
%    \end{macro}
%    \begin{macro}{\HoLogoHtml@LaTeX}
%    \begin{macrocode}
\def\HoLogoHtml@LaTeX#1{%
  \HoLogoCss@LaTeX
  \HOLOGO@Span{LaTeX}{%
    L%
    \HOLOGO@Span{a}{%
      A%
    }%
    \hologo{TeX}%
  }%
}
%    \end{macrocode}
%    \end{macro}
%    \begin{macro}{\HoLogoCss@LaTeX}
%    \begin{macrocode}
\def\HoLogoCss@LaTeX{%
  \Css{%
    span.HoLogo-LaTeX span.HoLogo-a{%
      position:relative;%
      top:-.5ex;%
      margin-left:-.36em;%
      margin-right:-.15em;%
      font-size:85\%;%
    }%
  }%
  \global\let\HoLogoCss@LaTeX\relax
}
%    \end{macrocode}
%    \end{macro}
%
% \subsubsection{\hologo{(La)TeX}}
%
%    \begin{macro}{\HoLogo@LaTeXTeX}
%    The kerning around the parentheses is taken
%    from package \xpackage{dtklogos} \cite{dtklogos}.
%\begin{quote}
%\begin{verbatim}
%\DeclareRobustCommand{\LaTeXTeX}{%
%  (%
%  \kern-.15em%
%  L%
%  \kern-.36em%
%  {%
%    \sbox\z@ T%
%    \vbox to\ht0{%
%      \hbox{%
%        $\m@th$%
%        \csname S@\f@size\endcsname
%        \fontsize\sf@size\z@
%        \math@fontsfalse
%        \selectfont
%        A%
%      }%
%      \vss
%    }%
%  }%
%  \kern-.2em%
%  )%
%  \kern-.15em%
%  \TeX
%}
%\end{verbatim}
%\end{quote}
%    \begin{macrocode}
\def\HoLogo@LaTeXTeX#1{%
  (%
  \kern-.15em%
  \hologo{La}%
  \kern-.2em%
  )%
  \kern-.15em%
  \hologo{TeX}%
}
%    \end{macrocode}
%    \end{macro}
%    \begin{macro}{\HoLogoBkm@LaTeXTeX}
%    \begin{macrocode}
\def\HoLogoBkm@LaTeXTeX#1{(La)TeX}
%    \end{macrocode}
%    \end{macro}
%
%    \begin{macro}{\HoLogo@(La)TeX}
%    \begin{macrocode}
\expandafter
\let\csname HoLogo@(La)TeX\endcsname\HoLogo@LaTeXTeX
%    \end{macrocode}
%    \end{macro}
%    \begin{macro}{\HoLogoBkm@(La)TeX}
%    \begin{macrocode}
\expandafter
\let\csname HoLogoBkm@(La)TeX\endcsname\HoLogoBkm@LaTeXTeX
%    \end{macrocode}
%    \end{macro}
%    \begin{macro}{\HoLogoHtml@LaTeXTeX}
%    \begin{macrocode}
\def\HoLogoHtml@LaTeXTeX#1{%
  \HoLogoCss@LaTeXTeX
  \HOLOGO@Span{LaTeXTeX}{%
    (%
    \HOLOGO@Span{L}{L}%
    \HOLOGO@Span{a}{A}%
    \HOLOGO@Span{ParenRight}{)}%
    \hologo{TeX}%
  }%
}
%    \end{macrocode}
%    \end{macro}
%    \begin{macro}{\HoLogoHtml@(La)TeX}
%    Kerning after opening parentheses and before closing parentheses
%    is $-0.1$\,em. The original values $-0.15$\,em
%    looked too ugly for a serif font.
%    \begin{macrocode}
\expandafter
\let\csname HoLogoHtml@(La)TeX\endcsname\HoLogoHtml@LaTeXTeX
%    \end{macrocode}
%    \end{macro}
%    \begin{macro}{\HoLogoCss@LaTeXTeX}
%    \begin{macrocode}
\def\HoLogoCss@LaTeXTeX{%
  \Css{%
    span.HoLogo-LaTeXTeX span.HoLogo-L{%
      margin-left:-.1em;%
    }%
  }%
  \Css{%
    span.HoLogo-LaTeXTeX span.HoLogo-a{%
      position:relative;%
      top:-.5ex;%
      margin-left:-.36em;%
      margin-right:-.1em;%
      font-size:85\%;%
    }%
  }%
  \Css{%
    span.HoLogo-LaTeXTeX span.HoLogo-ParenRight{%
      margin-right:-.15em;%
    }%
  }%
  \global\let\HoLogoCss@LaTeXTeX\relax
}
%    \end{macrocode}
%    \end{macro}
%
% \subsubsection{\hologo{LaTeXe}}
%
%    \begin{macro}{\HoLogo@LaTeXe}
%    Source: \hologo{LaTeX} kernel
%    \begin{macrocode}
\def\HoLogo@LaTeXe#1{%
  \hologo{LaTeX}%
  \kern.15em%
  \hbox{%
    \HOLOGO@MathSetup
    2%
    $_{\textstyle\varepsilon}$%
  }%
}
%    \end{macrocode}
%    \end{macro}
%
%    \begin{macro}{\HoLogoCs@LaTeXe}
%    \begin{macrocode}
\ifnum64=`\^^^^0040\relax % test for big chars of LuaTeX/XeTeX
  \catcode`\$=9 %
  \catcode`\&=14 %
\else
  \catcode`\$=14 %
  \catcode`\&=9 %
\fi
\def\HoLogoCs@LaTeXe#1{%
  LaTeX2%
$ \string ^^^^0395%
& e%
}%
\catcode`\$=3 %
\catcode`\&=4 %
%    \end{macrocode}
%    \end{macro}
%
%    \begin{macro}{\HoLogoBkm@LaTeXe}
%    \begin{macrocode}
\def\HoLogoBkm@LaTeXe#1{%
  \hologo{LaTeX}%
  2%
  \HOLOGO@PdfdocUnicode{e}{\textepsilon}%
}
%    \end{macrocode}
%    \end{macro}
%
%    \begin{macro}{\HoLogoHtml@LaTeXe}
%    \begin{macrocode}
\def\HoLogoHtml@LaTeXe#1{%
  \HoLogoCss@LaTeXe
  \HOLOGO@Span{LaTeX2e}{%
    \hologo{LaTeX}%
    \HOLOGO@Span{2}{2}%
    \HOLOGO@Span{e}{%
      \HOLOGO@MathSetup
      \ensuremath{\textstyle\varepsilon}%
    }%
  }%
}
%    \end{macrocode}
%    \end{macro}
%    \begin{macro}{\HoLogoCss@LaTeXe}
%    \begin{macrocode}
\def\HoLogoCss@LaTeXe{%
  \Css{%
    span.HoLogo-LaTeX2e span.HoLogo-2{%
      padding-left:.15em;%
    }%
  }%
  \Css{%
    span.HoLogo-LaTeX2e span.HoLogo-e{%
      position:relative;%
      top:.35ex;%
      text-decoration:none;%
    }%
  }%
  \global\let\HoLogoCss@LaTeXe\relax
}
%    \end{macrocode}
%    \end{macro}
%
%    \begin{macro}{\HoLogo@LaTeX2e}
%    \begin{macrocode}
\expandafter
\let\csname HoLogo@LaTeX2e\endcsname\HoLogo@LaTeXe
%    \end{macrocode}
%    \end{macro}
%    \begin{macro}{\HoLogoCs@LaTeX2e}
%    \begin{macrocode}
\expandafter
\let\csname HoLogoCs@LaTeX2e\endcsname\HoLogoCs@LaTeXe
%    \end{macrocode}
%    \end{macro}
%    \begin{macro}{\HoLogoBkm@LaTeX2e}
%    \begin{macrocode}
\expandafter
\let\csname HoLogoBkm@LaTeX2e\endcsname\HoLogoBkm@LaTeXe
%    \end{macrocode}
%    \end{macro}
%    \begin{macro}{\HoLogoHtml@LaTeX2e}
%    \begin{macrocode}
\expandafter
\let\csname HoLogoHtml@LaTeX2e\endcsname\HoLogoHtml@LaTeXe
%    \end{macrocode}
%    \end{macro}
%
% \subsubsection{\hologo{LaTeX3}}
%
%    \begin{macro}{\HoLogo@LaTeX3}
%    Source: \hologo{LaTeX} kernel
%    \begin{macrocode}
\expandafter\def\csname HoLogo@LaTeX3\endcsname#1{%
  \hologo{LaTeX}%
  3%
}
%    \end{macrocode}
%    \end{macro}
%
%    \begin{macro}{\HoLogoBkm@LaTeX3}
%    \begin{macrocode}
\expandafter\def\csname HoLogoBkm@LaTeX3\endcsname#1{%
  \hologo{LaTeX}%
  3%
}
%    \end{macrocode}
%    \end{macro}
%    \begin{macro}{\HoLogoHtml@LaTeX3}
%    \begin{macrocode}
\expandafter
\let\csname HoLogoHtml@LaTeX3\expandafter\endcsname
\csname HoLogo@LaTeX3\endcsname
%    \end{macrocode}
%    \end{macro}
%
% \subsubsection{\hologo{LaTeXML}}
%
%    \begin{macro}{\HoLogo@LaTeXML}
%    \begin{macrocode}
\def\HoLogo@LaTeXML#1{%
  \HOLOGO@mbox{%
    \hologo{La}%
    \kern-.15em%
    T%
    \kern-.1667em%
    \lower.5ex\hbox{E}%
    \kern-.125em%
    \HoLogoFont@font{LaTeXML}{sc}{xml}%
  }%
}
%    \end{macrocode}
%    \end{macro}
%    \begin{macro}{\HoLogoHtml@pdfLaTeX}
%    \begin{macrocode}
\def\HoLogoHtml@LaTeXML#1{%
  \HOLOGO@Span{LaTeXML}{%
    \HoLogoCss@LaTeX
    \HoLogoCss@TeX
    \HOLOGO@Span{LaTeX}{%
      L%
      \HOLOGO@Span{a}{%
        A%
      }%
    }%
    \HOLOGO@Span{TeX}{%
      T%
      \HOLOGO@Span{e}{%
        E%
      }%
    }%
    \HCode{<span style="font-variant: small-caps;">}%
    xml%
    \HCode{</span>}%
  }%
}
%    \end{macrocode}
%    \end{macro}
%
% \subsubsection{\hologo{eTeX}}
%
%    \begin{macro}{\HoLogo@eTeX}
%    Source: package \xpackage{etex}
%    \begin{macrocode}
\def\HoLogo@eTeX#1{%
  \ltx@mbox{%
    \HOLOGO@MathSetup
    $\varepsilon$%
    -%
    \HOLOGO@NegativeKerning{-T,T-,To}%
    \hologo{TeX}%
  }%
}
%    \end{macrocode}
%    \end{macro}
%    \begin{macro}{\HoLogoCs@eTeX}
%    \begin{macrocode}
\ifnum64=`\^^^^0040\relax % test for big chars of LuaTeX/XeTeX
  \catcode`\$=9 %
  \catcode`\&=14 %
\else
  \catcode`\$=14 %
  \catcode`\&=9 %
\fi
\def\HoLogoCs@eTeX#1{%
$ #1{\string ^^^^0395}{\string ^^^^03b5}%
& #1{e}{E}%
  TeX%
}%
\catcode`\$=3 %
\catcode`\&=4 %
%    \end{macrocode}
%    \end{macro}
%    \begin{macro}{\HoLogoBkm@eTeX}
%    \begin{macrocode}
\def\HoLogoBkm@eTeX#1{%
  \HOLOGO@PdfdocUnicode{#1{e}{E}}{\textepsilon}%
  -%
  \hologo{TeX}%
}
%    \end{macrocode}
%    \end{macro}
%    \begin{macro}{\HoLogoHtml@eTeX}
%    \begin{macrocode}
\def\HoLogoHtml@eTeX#1{%
  \ltx@mbox{%
    \HOLOGO@MathSetup
    $\varepsilon$%
    -%
    \hologo{TeX}%
  }%
}
%    \end{macrocode}
%    \end{macro}
%
% \subsubsection{\hologo{iniTeX}}
%
%    \begin{macro}{\HoLogo@iniTeX}
%    \begin{macrocode}
\def\HoLogo@iniTeX#1{%
  \HOLOGO@mbox{%
    #1{i}{I}ni\hologo{TeX}%
  }%
}
%    \end{macrocode}
%    \end{macro}
%    \begin{macro}{\HoLogoCs@iniTeX}
%    \begin{macrocode}
\def\HoLogoCs@iniTeX#1{#1{i}{I}niTeX}
%    \end{macrocode}
%    \end{macro}
%    \begin{macro}{\HoLogoBkm@iniTeX}
%    \begin{macrocode}
\def\HoLogoBkm@iniTeX#1{%
  #1{i}{I}ni\hologo{TeX}%
}
%    \end{macrocode}
%    \end{macro}
%    \begin{macro}{\HoLogoHtml@iniTeX}
%    \begin{macrocode}
\let\HoLogoHtml@iniTeX\HoLogo@iniTeX
%    \end{macrocode}
%    \end{macro}
%
% \subsubsection{\hologo{virTeX}}
%
%    \begin{macro}{\HoLogo@virTeX}
%    \begin{macrocode}
\def\HoLogo@virTeX#1{%
  \HOLOGO@mbox{%
    #1{v}{V}ir\hologo{TeX}%
  }%
}
%    \end{macrocode}
%    \end{macro}
%    \begin{macro}{\HoLogoCs@virTeX}
%    \begin{macrocode}
\def\HoLogoCs@virTeX#1{#1{v}{V}irTeX}
%    \end{macrocode}
%    \end{macro}
%    \begin{macro}{\HoLogoBkm@virTeX}
%    \begin{macrocode}
\def\HoLogoBkm@virTeX#1{%
  #1{v}{V}ir\hologo{TeX}%
}
%    \end{macrocode}
%    \end{macro}
%    \begin{macro}{\HoLogoHtml@virTeX}
%    \begin{macrocode}
\let\HoLogoHtml@virTeX\HoLogo@virTeX
%    \end{macrocode}
%    \end{macro}
%
% \subsubsection{\hologo{SliTeX}}
%
% \paragraph{Definitions of the three variants.}
%
%    \begin{macro}{\HoLogo@SLiTeX@lift}
%    \begin{macrocode}
\def\HoLogo@SLiTeX@lift#1{%
  \HoLogoFont@font{SliTeX}{rm}{%
    S%
    \kern-.06em%
    L%
    \kern-.18em%
    \raise.32ex\hbox{\HoLogoFont@font{SliTeX}{sc}{i}}%
    \HOLOGO@discretionary
    \kern-.06em%
    \hologo{TeX}%
  }%
}
%    \end{macrocode}
%    \end{macro}
%    \begin{macro}{\HoLogoBkm@SLiTeX@lift}
%    \begin{macrocode}
\def\HoLogoBkm@SLiTeX@lift#1{SLiTeX}
%    \end{macrocode}
%    \end{macro}
%    \begin{macro}{\HoLogoHtml@SLiTeX@lift}
%    \begin{macrocode}
\def\HoLogoHtml@SLiTeX@lift#1{%
  \HoLogoCss@SLiTeX@lift
  \HOLOGO@Span{SLiTeX-lift}{%
    \HoLogoFont@font{SliTeX}{rm}{%
      S%
      \HOLOGO@Span{L}{L}%
      \HOLOGO@Span{i}{i}%
      \hologo{TeX}%
    }%
  }%
}
%    \end{macrocode}
%    \end{macro}
%    \begin{macro}{\HoLogoCss@SLiTeX@lift}
%    \begin{macrocode}
\def\HoLogoCss@SLiTeX@lift{%
  \Css{%
    span.HoLogo-SLiTeX-lift span.HoLogo-L{%
      margin-left:-.06em;%
      margin-right:-.18em;%
    }%
  }%
  \Css{%
    span.HoLogo-SLiTeX-lift span.HoLogo-i{%
      position:relative;%
      top:-.32ex;%
      margin-right:-.06em;%
      font-variant:small-caps;%
    }%
  }%
  \global\let\HoLogoCss@SLiTeX@lift\relax
}
%    \end{macrocode}
%    \end{macro}
%
%    \begin{macro}{\HoLogo@SliTeX@simple}
%    \begin{macrocode}
\def\HoLogo@SliTeX@simple#1{%
  \HoLogoFont@font{SliTeX}{rm}{%
    \ltx@mbox{%
      \HoLogoFont@font{SliTeX}{sc}{Sli}%
    }%
    \HOLOGO@discretionary
    \hologo{TeX}%
  }%
}
%    \end{macrocode}
%    \end{macro}
%    \begin{macro}{\HoLogoBkm@SliTeX@simple}
%    \begin{macrocode}
\def\HoLogoBkm@SliTeX@simple#1{SliTeX}
%    \end{macrocode}
%    \end{macro}
%    \begin{macro}{\HoLogoHtml@SliTeX@simple}
%    \begin{macrocode}
\let\HoLogoHtml@SliTeX@simple\HoLogo@SliTeX@simple
%    \end{macrocode}
%    \end{macro}
%
%    \begin{macro}{\HoLogo@SliTeX@narrow}
%    \begin{macrocode}
\def\HoLogo@SliTeX@narrow#1{%
  \HoLogoFont@font{SliTeX}{rm}{%
    \ltx@mbox{%
      S%
      \kern-.06em%
      \HoLogoFont@font{SliTeX}{sc}{%
        l%
        \kern-.035em%
        i%
      }%
    }%
    \HOLOGO@discretionary
    \kern-.06em%
    \hologo{TeX}%
  }%
}
%    \end{macrocode}
%    \end{macro}
%    \begin{macro}{\HoLogoBkm@SliTeX@narrow}
%    \begin{macrocode}
\def\HoLogoBkm@SliTeX@narrow#1{SliTeX}
%    \end{macrocode}
%    \end{macro}
%    \begin{macro}{\HoLogoHtml@SliTeX@narrow}
%    \begin{macrocode}
\def\HoLogoHtml@SliTeX@narrow#1{%
  \HoLogoCss@SliTeX@narrow
  \HOLOGO@Span{SliTeX-narrow}{%
    \HoLogoFont@font{SliTeX}{rm}{%
      S%
        \HOLOGO@Span{l}{l}%
        \HOLOGO@Span{i}{i}%
      \hologo{TeX}%
    }%
  }%
}
%    \end{macrocode}
%    \end{macro}
%    \begin{macro}{\HoLogoCss@SliTeX@narrow}
%    \begin{macrocode}
\def\HoLogoCss@SliTeX@narrow{%
  \Css{%
    span.HoLogo-SliTeX-narrow span.HoLogo-l{%
      margin-left:-.06em;%
      margin-right:-.035em;%
      font-variant:small-caps;%
    }%
  }%
  \Css{%
    span.HoLogo-SliTeX-narrow span.HoLogo-i{%
      margin-right:-.06em;%
      font-variant:small-caps;%
    }%
  }%
  \global\let\HoLogoCss@SliTeX@narrow\relax
}
%    \end{macrocode}
%    \end{macro}
%
% \paragraph{Macro set completion.}
%
%    \begin{macro}{\HoLogo@SLiTeX@simple}
%    \begin{macrocode}
\def\HoLogo@SLiTeX@simple{\HoLogo@SliTeX@simple}
%    \end{macrocode}
%    \end{macro}
%    \begin{macro}{\HoLogoBkm@SLiTeX@simple}
%    \begin{macrocode}
\def\HoLogoBkm@SLiTeX@simple{\HoLogoBkm@SliTeX@simple}
%    \end{macrocode}
%    \end{macro}
%    \begin{macro}{\HoLogoHtml@SLiTeX@simple}
%    \begin{macrocode}
\def\HoLogoHtml@SLiTeX@simple{\HoLogoHtml@SliTeX@simple}
%    \end{macrocode}
%    \end{macro}
%
%    \begin{macro}{\HoLogo@SLiTeX@narrow}
%    \begin{macrocode}
\def\HoLogo@SLiTeX@narrow{\HoLogo@SliTeX@narrow}
%    \end{macrocode}
%    \end{macro}
%    \begin{macro}{\HoLogoBkm@SLiTeX@narrow}
%    \begin{macrocode}
\def\HoLogoBkm@SLiTeX@narrow{\HoLogoBkm@SliTeX@narrow}
%    \end{macrocode}
%    \end{macro}
%    \begin{macro}{\HoLogoHtml@SLiTeX@narrow}
%    \begin{macrocode}
\def\HoLogoHtml@SLiTeX@narrow{\HoLogoHtml@SliTeX@narrow}
%    \end{macrocode}
%    \end{macro}
%
%    \begin{macro}{\HoLogo@SliTeX@lift}
%    \begin{macrocode}
\def\HoLogo@SliTeX@lift{\HoLogo@SLiTeX@lift}
%    \end{macrocode}
%    \end{macro}
%    \begin{macro}{\HoLogoBkm@SliTeX@lift}
%    \begin{macrocode}
\def\HoLogoBkm@SliTeX@lift{\HoLogoBkm@SLiTeX@lift}
%    \end{macrocode}
%    \end{macro}
%    \begin{macro}{\HoLogoHtml@SliTeX@lift}
%    \begin{macrocode}
\def\HoLogoHtml@SliTeX@lift{\HoLogoHtml@SLiTeX@lift}
%    \end{macrocode}
%    \end{macro}
%
% \paragraph{Defaults.}
%
%    \begin{macro}{\HoLogo@SLiTeX}
%    \begin{macrocode}
\def\HoLogo@SLiTeX{\HoLogo@SLiTeX@lift}
%    \end{macrocode}
%    \end{macro}
%    \begin{macro}{\HoLogoBkm@SLiTeX}
%    \begin{macrocode}
\def\HoLogoBkm@SLiTeX{\HoLogoBkm@SLiTeX@lift}
%    \end{macrocode}
%    \end{macro}
%    \begin{macro}{\HoLogoHtml@SLiTeX}
%    \begin{macrocode}
\def\HoLogoHtml@SLiTeX{\HoLogoHtml@SLiTeX@lift}
%    \end{macrocode}
%    \end{macro}
%
%    \begin{macro}{\HoLogo@SliTeX}
%    \begin{macrocode}
\def\HoLogo@SliTeX{\HoLogo@SliTeX@narrow}
%    \end{macrocode}
%    \end{macro}
%    \begin{macro}{\HoLogoBkm@SliTeX}
%    \begin{macrocode}
\def\HoLogoBkm@SliTeX{\HoLogoBkm@SliTeX@narrow}
%    \end{macrocode}
%    \end{macro}
%    \begin{macro}{\HoLogoHtml@SliTeX}
%    \begin{macrocode}
\def\HoLogoHtml@SliTeX{\HoLogoHtml@SliTeX@narrow}
%    \end{macrocode}
%    \end{macro}
%
% \subsubsection{\hologo{LuaTeX}}
%
%    \begin{macro}{\HoLogo@LuaTeX}
%    The kerning is an idea of Hans Hagen, see mailing list
%    `luatex at tug dot org' in March 2010.
%    \begin{macrocode}
\def\HoLogo@LuaTeX#1{%
  \HOLOGO@mbox{%
    Lua%
    \HOLOGO@NegativeKerning{aT,oT,To}%
    \hologo{TeX}%
  }%
}
%    \end{macrocode}
%    \end{macro}
%    \begin{macro}{\HoLogoHtml@LuaTeX}
%    \begin{macrocode}
\let\HoLogoHtml@LuaTeX\HoLogo@LuaTeX
%    \end{macrocode}
%    \end{macro}
%
% \subsubsection{\hologo{LuaLaTeX}}
%
%    \begin{macro}{\HoLogo@LuaLaTeX}
%    \begin{macrocode}
\def\HoLogo@LuaLaTeX#1{%
  \HOLOGO@mbox{%
    Lua%
    \hologo{LaTeX}%
  }%
}
%    \end{macrocode}
%    \end{macro}
%    \begin{macro}{\HoLogoHtml@LuaLaTeX}
%    \begin{macrocode}
\let\HoLogoHtml@LuaLaTeX\HoLogo@LuaLaTeX
%    \end{macrocode}
%    \end{macro}
%
% \subsubsection{\hologo{XeTeX}, \hologo{XeLaTeX}}
%
%    \begin{macro}{\HOLOGO@IfCharExists}
%    \begin{macrocode}
\ifluatex
  \ifnum\luatexversion<36 %
  \else
    \def\HOLOGO@IfCharExists#1{%
      \ifnum
        \directlua{%
           if luaotfload and luaotfload.aux then
             if luaotfload.aux.font_has_glyph(%
                    font.current(), \number#1) then % 	 
	       tex.print("1") % 	 
	     end % 	 
	   elseif font and font.fonts and font.current then %
            local f = font.fonts[font.current()]%
            if f.characters and f.characters[\number#1] then %
              tex.print("1")%
            end %
          end%
        }0=\ltx@zero
        \expandafter\ltx@secondoftwo
      \else
        \expandafter\ltx@firstoftwo
      \fi
    }%
  \fi
\fi
\ltx@IfUndefined{HOLOGO@IfCharExists}{%
  \def\HOLOGO@@IfCharExists#1{%
    \begingroup
      \tracinglostchars=\ltx@zero
      \setbox\ltx@zero=\hbox{%
        \kern7sp\char#1\relax
        \ifnum\lastkern>\ltx@zero
          \expandafter\aftergroup\csname iffalse\endcsname
        \else
          \expandafter\aftergroup\csname iftrue\endcsname
        \fi
      }%
      % \if{true|false} from \aftergroup
      \endgroup
      \expandafter\ltx@firstoftwo
    \else
      \endgroup
      \expandafter\ltx@secondoftwo
    \fi
  }%
  \ifxetex
    \ltx@IfUndefined{XeTeXfonttype}{}{%
      \ltx@IfUndefined{XeTeXcharglyph}{}{%
        \def\HOLOGO@IfCharExists#1{%
          \ifnum\XeTeXfonttype\font>\ltx@zero
            \expandafter\ltx@firstofthree
          \else
            \expandafter\ltx@gobble
          \fi
          {%
            \ifnum\XeTeXcharglyph#1>\ltx@zero
              \expandafter\ltx@firstoftwo
            \else
              \expandafter\ltx@secondoftwo
            \fi
          }%
          \HOLOGO@@IfCharExists{#1}%
        }%
      }%
    }%
  \fi
}{}
\ltx@ifundefined{HOLOGO@IfCharExists}{%
  \ifnum64=`\^^^^0040\relax % test for big chars of LuaTeX/XeTeX
    \let\HOLOGO@IfCharExists\HOLOGO@@IfCharExists
  \else
    \def\HOLOGO@IfCharExists#1{%
      \ifnum#1>255 %
        \expandafter\ltx@fourthoffour
      \fi
      \HOLOGO@@IfCharExists{#1}%
    }%
  \fi
}{}
%    \end{macrocode}
%    \end{macro}
%
%    \begin{macro}{\HoLogo@Xe}
%    Source: package \xpackage{dtklogos}
%    \begin{macrocode}
\def\HoLogo@Xe#1{%
  X%
  \kern-.1em\relax
  \HOLOGO@IfCharExists{"018E}{%
    \lower.5ex\hbox{\char"018E}%
  }{%
    \chardef\HOLOGO@choice=\ltx@zero
    \ifdim\fontdimen\ltx@one\font>0pt %
      \ltx@IfUndefined{rotatebox}{%
        \ltx@IfUndefined{pgftext}{%
          \ltx@IfUndefined{psscalebox}{%
            \ltx@IfUndefined{HOLOGO@ScaleBox@\hologoDriver}{%
            }{%
              \chardef\HOLOGO@choice=4 %
            }%
          }{%
            \chardef\HOLOGO@choice=3 %
          }%
        }{%
          \chardef\HOLOGO@choice=2 %
        }%
      }{%
        \chardef\HOLOGO@choice=1 %
      }%
      \ifcase\HOLOGO@choice
        \HOLOGO@WarningUnsupportedDriver{Xe}%
        e%
      \or % 1: \rotatebox
        \begingroup
          \setbox\ltx@zero\hbox{\rotatebox{180}{E}}%
          \ltx@LocDimenA=\dp\ltx@zero
          \advance\ltx@LocDimenA by -.5ex\relax
          \raise\ltx@LocDimenA\box\ltx@zero
        \endgroup
      \or % 2: \pgftext
        \lower.5ex\hbox{%
          \pgfpicture
            \pgftext[rotate=180]{E}%
          \endpgfpicture
        }%
      \or % 3: \psscalebox
        \begingroup
          \setbox\ltx@zero\hbox{\psscalebox{-1 -1}{E}}%
          \ltx@LocDimenA=\dp\ltx@zero
          \advance\ltx@LocDimenA by -.5ex\relax
          \raise\ltx@LocDimenA\box\ltx@zero
        \endgroup
      \or % 4: \HOLOGO@PointReflectBox
        \lower.5ex\hbox{\HOLOGO@PointReflectBox{E}}%
      \else
        \@PackageError{hologo}{Internal error (choice/it}\@ehc
      \fi
    \else
      \ltx@IfUndefined{reflectbox}{%
        \ltx@IfUndefined{pgftext}{%
          \ltx@IfUndefined{psscalebox}{%
            \ltx@IfUndefined{HOLOGO@ScaleBox@\hologoDriver}{%
            }{%
              \chardef\HOLOGO@choice=4 %
            }%
          }{%
            \chardef\HOLOGO@choice=3 %
          }%
        }{%
          \chardef\HOLOGO@choice=2 %
        }%
      }{%
        \chardef\HOLOGO@choice=1 %
      }%
      \ifcase\HOLOGO@choice
        \HOLOGO@WarningUnsupportedDriver{Xe}%
        e%
      \or % 1: reflectbox
        \lower.5ex\hbox{%
          \reflectbox{E}%
        }%
      \or % 2: \pgftext
        \lower.5ex\hbox{%
          \pgfpicture
            \pgftransformxscale{-1}%
            \pgftext{E}%
          \endpgfpicture
        }%
      \or % 3: \psscalebox
        \lower.5ex\hbox{%
          \psscalebox{-1 1}{E}%
        }%
      \or % 4: \HOLOGO@Reflectbox
        \lower.5ex\hbox{%
          \HOLOGO@ReflectBox{E}%
        }%
      \else
        \@PackageError{hologo}{Internal error (choice/up)}\@ehc
      \fi
    \fi
  }%
}
%    \end{macrocode}
%    \end{macro}
%    \begin{macro}{\HoLogoHtml@Xe}
%    \begin{macrocode}
\def\HoLogoHtml@Xe#1{%
  \HoLogoCss@Xe
  \HOLOGO@Span{Xe}{%
    X%
    \HOLOGO@Span{e}{%
      \HCode{&\ltx@hashchar x018e;}%
    }%
  }%
}
%    \end{macrocode}
%    \end{macro}
%    \begin{macro}{\HoLogoCss@Xe}
%    \begin{macrocode}
\def\HoLogoCss@Xe{%
  \Css{%
    span.HoLogo-Xe span.HoLogo-e{%
      position:relative;%
      top:.5ex;%
      left-margin:-.1em;%
    }%
  }%
  \global\let\HoLogoCss@Xe\relax
}
%    \end{macrocode}
%    \end{macro}
%
%    \begin{macro}{\HoLogo@XeTeX}
%    \begin{macrocode}
\def\HoLogo@XeTeX#1{%
  \hologo{Xe}%
  \kern-.15em\relax
  \hologo{TeX}%
}
%    \end{macrocode}
%    \end{macro}
%
%    \begin{macro}{\HoLogoHtml@XeTeX}
%    \begin{macrocode}
\def\HoLogoHtml@XeTeX#1{%
  \HoLogoCss@XeTeX
  \HOLOGO@Span{XeTeX}{%
    \hologo{Xe}%
    \hologo{TeX}%
  }%
}
%    \end{macrocode}
%    \end{macro}
%    \begin{macro}{\HoLogoCss@XeTeX}
%    \begin{macrocode}
\def\HoLogoCss@XeTeX{%
  \Css{%
    span.HoLogo-XeTeX span.HoLogo-TeX{%
      margin-left:-.15em;%
    }%
  }%
  \global\let\HoLogoCss@XeTeX\relax
}
%    \end{macrocode}
%    \end{macro}
%
%    \begin{macro}{\HoLogo@XeLaTeX}
%    \begin{macrocode}
\def\HoLogo@XeLaTeX#1{%
  \hologo{Xe}%
  \kern-.13em%
  \hologo{LaTeX}%
}
%    \end{macrocode}
%    \end{macro}
%    \begin{macro}{\HoLogoHtml@XeLaTeX}
%    \begin{macrocode}
\def\HoLogoHtml@XeLaTeX#1{%
  \HoLogoCss@XeLaTeX
  \HOLOGO@Span{XeLaTeX}{%
    \hologo{Xe}%
    \hologo{LaTeX}%
  }%
}
%    \end{macrocode}
%    \end{macro}
%    \begin{macro}{\HoLogoCss@XeLaTeX}
%    \begin{macrocode}
\def\HoLogoCss@XeLaTeX{%
  \Css{%
    span.HoLogo-XeLaTeX span.HoLogo-Xe{%
      margin-right:-.13em;%
    }%
  }%
  \global\let\HoLogoCss@XeLaTeX\relax
}
%    \end{macrocode}
%    \end{macro}
%
% \subsubsection{\hologo{pdfTeX}, \hologo{pdfLaTeX}}
%
%    \begin{macro}{\HoLogo@pdfTeX}
%    \begin{macrocode}
\def\HoLogo@pdfTeX#1{%
  \HOLOGO@mbox{%
    #1{p}{P}df\hologo{TeX}%
  }%
}
%    \end{macrocode}
%    \end{macro}
%    \begin{macro}{\HoLogoCs@pdfTeX}
%    \begin{macrocode}
\def\HoLogoCs@pdfTeX#1{#1{p}{P}dfTeX}
%    \end{macrocode}
%    \end{macro}
%    \begin{macro}{\HoLogoBkm@pdfTeX}
%    \begin{macrocode}
\def\HoLogoBkm@pdfTeX#1{%
  #1{p}{P}df\hologo{TeX}%
}
%    \end{macrocode}
%    \end{macro}
%    \begin{macro}{\HoLogoHtml@pdfTeX}
%    \begin{macrocode}
\let\HoLogoHtml@pdfTeX\HoLogo@pdfTeX
%    \end{macrocode}
%    \end{macro}
%
%    \begin{macro}{\HoLogo@pdfLaTeX}
%    \begin{macrocode}
\def\HoLogo@pdfLaTeX#1{%
  \HOLOGO@mbox{%
    #1{p}{P}df\hologo{LaTeX}%
  }%
}
%    \end{macrocode}
%    \end{macro}
%    \begin{macro}{\HoLogoCs@pdfLaTeX}
%    \begin{macrocode}
\def\HoLogoCs@pdfLaTeX#1{#1{p}{P}dfLaTeX}
%    \end{macrocode}
%    \end{macro}
%    \begin{macro}{\HoLogoBkm@pdfLaTeX}
%    \begin{macrocode}
\def\HoLogoBkm@pdfLaTeX#1{%
  #1{p}{P}df\hologo{LaTeX}%
}
%    \end{macrocode}
%    \end{macro}
%    \begin{macro}{\HoLogoHtml@pdfLaTeX}
%    \begin{macrocode}
\let\HoLogoHtml@pdfLaTeX\HoLogo@pdfLaTeX
%    \end{macrocode}
%    \end{macro}
%
% \subsubsection{\hologo{VTeX}}
%
%    \begin{macro}{\HoLogo@VTeX}
%    \begin{macrocode}
\def\HoLogo@VTeX#1{%
  \HOLOGO@mbox{%
    V\hologo{TeX}%
  }%
}
%    \end{macrocode}
%    \end{macro}
%    \begin{macro}{\HoLogoHtml@VTeX}
%    \begin{macrocode}
\let\HoLogoHtml@VTeX\HoLogo@VTeX
%    \end{macrocode}
%    \end{macro}
%
% \subsubsection{\hologo{AmS}, \dots}
%
%    Source: class \xclass{amsdtx}
%
%    \begin{macro}{\HoLogo@AmS}
%    \begin{macrocode}
\def\HoLogo@AmS#1{%
  \HoLogoFont@font{AmS}{sy}{%
    A%
    \kern-.1667em%
    \lower.5ex\hbox{M}%
    \kern-.125em%
    S%
  }%
}
%    \end{macrocode}
%    \end{macro}
%    \begin{macro}{\HoLogoBkm@AmS}
%    \begin{macrocode}
\def\HoLogoBkm@AmS#1{AmS}
%    \end{macrocode}
%    \end{macro}
%    \begin{macro}{\HoLogoHtml@AmS}
%    \begin{macrocode}
\def\HoLogoHtml@AmS#1{%
  \HoLogoCss@AmS
%  \HoLogoFont@font{AmS}{sy}{%
    \HOLOGO@Span{AmS}{%
      A%
      \HOLOGO@Span{M}{M}%
      S%
    }%
%   }%
}
%    \end{macrocode}
%    \end{macro}
%    \begin{macro}{\HoLogoCss@AmS}
%    \begin{macrocode}
\def\HoLogoCss@AmS{%
  \Css{%
    span.HoLogo-AmS span.HoLogo-M{%
      position:relative;%
      top:.5ex;%
      margin-left:-.1667em;%
      margin-right:-.125em;%
      text-decoration:none;%
    }%
  }%
  \global\let\HoLogoCss@AmS\relax
}
%    \end{macrocode}
%    \end{macro}
%
%    \begin{macro}{\HoLogo@AmSTeX}
%    \begin{macrocode}
\def\HoLogo@AmSTeX#1{%
  \hologo{AmS}%
  \HOLOGO@hyphen
  \hologo{TeX}%
}
%    \end{macrocode}
%    \end{macro}
%    \begin{macro}{\HoLogoBkm@AmSTeX}
%    \begin{macrocode}
\def\HoLogoBkm@AmSTeX#1{AmS-TeX}%
%    \end{macrocode}
%    \end{macro}
%    \begin{macro}{\HoLogoHtml@AmSTeX}
%    \begin{macrocode}
\let\HoLogoHtml@AmSTeX\HoLogo@AmSTeX
%    \end{macrocode}
%    \end{macro}
%
%    \begin{macro}{\HoLogo@AmSLaTeX}
%    \begin{macrocode}
\def\HoLogo@AmSLaTeX#1{%
  \hologo{AmS}%
  \HOLOGO@hyphen
  \hologo{LaTeX}%
}
%    \end{macrocode}
%    \end{macro}
%    \begin{macro}{\HoLogoBkm@AmSLaTeX}
%    \begin{macrocode}
\def\HoLogoBkm@AmSLaTeX#1{AmS-LaTeX}%
%    \end{macrocode}
%    \end{macro}
%    \begin{macro}{\HoLogoHtml@AmSLaTeX}
%    \begin{macrocode}
\let\HoLogoHtml@AmSLaTeX\HoLogo@AmSLaTeX
%    \end{macrocode}
%    \end{macro}
%
% \subsubsection{\hologo{BibTeX}}
%
%    \begin{macro}{\HoLogo@BibTeX@sc}
%    A definition of \hologo{BibTeX} is provided in
%    the documentation source for the manual of \hologo{BibTeX}
%    \cite{btxdoc}.
%\begin{quote}
%\begin{verbatim}
%\def\BibTeX{%
%  {%
%    \rm
%    B%
%    \kern-.05em%
%    {%
%      \sc
%      i%
%      \kern-.025em %
%      b%
%    }%
%    \kern-.08em
%    T%
%    \kern-.1667em%
%    \lower.7ex\hbox{E}%
%    \kern-.125em%
%    X%
%  }%
%}
%\end{verbatim}
%\end{quote}
%    \begin{macrocode}
\def\HoLogo@BibTeX@sc#1{%
  B%
  \kern-.05em%
  \HoLogoFont@font{BibTeX}{sc}{%
    i%
    \kern-.025em%
    b%
  }%
  \HOLOGO@discretionary
  \kern-.08em%
  \hologo{TeX}%
}
%    \end{macrocode}
%    \end{macro}
%    \begin{macro}{\HoLogoHtml@BibTeX@sc}
%    \begin{macrocode}
\def\HoLogoHtml@BibTeX@sc#1{%
  \HoLogoCss@BibTeX@sc
  \HOLOGO@Span{BibTeX-sc}{%
    B%
    \HOLOGO@Span{i}{i}%
    \HOLOGO@Span{b}{b}%
    \hologo{TeX}%
  }%
}
%    \end{macrocode}
%    \end{macro}
%    \begin{macro}{\HoLogoCss@BibTeX@sc}
%    \begin{macrocode}
\def\HoLogoCss@BibTeX@sc{%
  \Css{%
    span.HoLogo-BibTeX-sc span.HoLogo-i{%
      margin-left:-.05em;%
      margin-right:-.025em;%
      font-variant:small-caps;%
    }%
  }%
  \Css{%
    span.HoLogo-BibTeX-sc span.HoLogo-b{%
      margin-right:-.08em;%
      font-variant:small-caps;%
    }%
  }%
  \global\let\HoLogoCss@BibTeX@sc\relax
}
%    \end{macrocode}
%    \end{macro}
%
%    \begin{macro}{\HoLogo@BibTeX@sf}
%    Variant \xoption{sf} avoids trouble with unavailable
%    small caps fonts (e.g., bold versions of Computer Modern or
%    Latin Modern). The definition is taken from
%    package \xpackage{dtklogos} \cite{dtklogos}.
%\begin{quote}
%\begin{verbatim}
%\DeclareRobustCommand{\BibTeX}{%
%  B%
%  \kern-.05em%
%  \hbox{%
%    $\m@th$% %% force math size calculations
%    \csname S@\f@size\endcsname
%    \fontsize\sf@size\z@
%    \math@fontsfalse
%    \selectfont
%    I%
%    \kern-.025em%
%    B
%  }%
%  \kern-.08em%
%  \-%
%  \TeX
%}
%\end{verbatim}
%\end{quote}
%    \begin{macrocode}
\def\HoLogo@BibTeX@sf#1{%
  B%
  \kern-.05em%
  \HoLogoFont@font{BibTeX}{bibsf}{%
    I%
    \kern-.025em%
    B%
  }%
  \HOLOGO@discretionary
  \kern-.08em%
  \hologo{TeX}%
}
%    \end{macrocode}
%    \end{macro}
%    \begin{macro}{\HoLogoHtml@BibTeX@sf}
%    \begin{macrocode}
\def\HoLogoHtml@BibTeX@sf#1{%
  \HoLogoCss@BibTeX@sf
  \HOLOGO@Span{BibTeX-sf}{%
    B%
    \HoLogoFont@font{BibTeX}{bibsf}{%
      \HOLOGO@Span{i}{I}%
      B%
    }%
    \hologo{TeX}%
  }%
}
%    \end{macrocode}
%    \end{macro}
%    \begin{macro}{\HoLogoCss@BibTeX@sf}
%    \begin{macrocode}
\def\HoLogoCss@BibTeX@sf{%
  \Css{%
    span.HoLogo-BibTeX-sf span.HoLogo-i{%
      margin-left:-.05em;%
      margin-right:-.025em;%
    }%
  }%
  \Css{%
    span.HoLogo-BibTeX-sf span.HoLogo-TeX{%
      margin-left:-.08em;%
    }%
  }%
  \global\let\HoLogoCss@BibTeX@sf\relax
}
%    \end{macrocode}
%    \end{macro}
%
%    \begin{macro}{\HoLogo@BibTeX}
%    \begin{macrocode}
\def\HoLogo@BibTeX{\HoLogo@BibTeX@sf}
%    \end{macrocode}
%    \end{macro}
%    \begin{macro}{\HoLogoHtml@BibTeX}
%    \begin{macrocode}
\def\HoLogoHtml@BibTeX{\HoLogoHtml@BibTeX@sf}
%    \end{macrocode}
%    \end{macro}
%
% \subsubsection{\hologo{BibTeX8}}
%
%    \begin{macro}{\HoLogo@BibTeX8}
%    \begin{macrocode}
\expandafter\def\csname HoLogo@BibTeX8\endcsname#1{%
  \hologo{BibTeX}%
  8%
}
%    \end{macrocode}
%    \end{macro}
%
%    \begin{macro}{\HoLogoBkm@BibTeX8}
%    \begin{macrocode}
\expandafter\def\csname HoLogoBkm@BibTeX8\endcsname#1{%
  \hologo{BibTeX}%
  8%
}
%    \end{macrocode}
%    \end{macro}
%    \begin{macro}{\HoLogoHtml@BibTeX8}
%    \begin{macrocode}
\expandafter
\let\csname HoLogoHtml@BibTeX8\expandafter\endcsname
\csname HoLogo@BibTeX8\endcsname
%    \end{macrocode}
%    \end{macro}
%
% \subsubsection{\hologo{ConTeXt}}
%
%    \begin{macro}{\HoLogo@ConTeXt@simple}
%    \begin{macrocode}
\def\HoLogo@ConTeXt@simple#1{%
  \HOLOGO@mbox{Con}%
  \HOLOGO@discretionary
  \HOLOGO@mbox{\hologo{TeX}t}%
}
%    \end{macrocode}
%    \end{macro}
%    \begin{macro}{\HoLogoHtml@ConTeXt@simple}
%    \begin{macrocode}
\let\HoLogoHtml@ConTeXt@simple\HoLogo@ConTeXt@simple
%    \end{macrocode}
%    \end{macro}
%
%    \begin{macro}{\HoLogo@ConTeXt@narrow}
%    This definition of logo \hologo{ConTeXt} with variant \xoption{narrow}
%    comes from TUGboat's class \xclass{ltugboat} (version 2010/11/15 v2.8).
%    \begin{macrocode}
\def\HoLogo@ConTeXt@narrow#1{%
  \HOLOGO@mbox{C\kern-.0333emon}%
  \HOLOGO@discretionary
  \kern-.0667em%
  \HOLOGO@mbox{\hologo{TeX}\kern-.0333emt}%
}
%    \end{macrocode}
%    \end{macro}
%    \begin{macro}{\HoLogoHtml@ConTeXt@narrow}
%    \begin{macrocode}
\def\HoLogoHtml@ConTeXt@narrow#1{%
  \HoLogoCss@ConTeXt@narrow
  \HOLOGO@Span{ConTeXt-narrow}{%
    \HOLOGO@Span{C}{C}%
    on%
    \hologo{TeX}%
    t%
  }%
}
%    \end{macrocode}
%    \end{macro}
%    \begin{macro}{\HoLogoCss@ConTeXt@narrow}
%    \begin{macrocode}
\def\HoLogoCss@ConTeXt@narrow{%
  \Css{%
    span.HoLogo-ConTeXt-narrow span.HoLogo-C{%
      margin-left:-.0333em;%
    }%
  }%
  \Css{%
    span.HoLogo-ConTeXt-narrow span.HoLogo-TeX{%
      margin-left:-.0667em;%
      margin-right:-.0333em;%
    }%
  }%
  \global\let\HoLogoCss@ConTeXt@narrow\relax
}
%    \end{macrocode}
%    \end{macro}
%
%    \begin{macro}{\HoLogo@ConTeXt}
%    \begin{macrocode}
\def\HoLogo@ConTeXt{\HoLogo@ConTeXt@narrow}
%    \end{macrocode}
%    \end{macro}
%    \begin{macro}{\HoLogoHtml@ConTeXt}
%    \begin{macrocode}
\def\HoLogoHtml@ConTeXt{\HoLogoHtml@ConTeXt@narrow}
%    \end{macrocode}
%    \end{macro}
%
% \subsubsection{\hologo{emTeX}}
%
%    \begin{macro}{\HoLogo@emTeX}
%    \begin{macrocode}
\def\HoLogo@emTeX#1{%
  \HOLOGO@mbox{#1{e}{E}m}%
  \HOLOGO@discretionary
  \hologo{TeX}%
}
%    \end{macrocode}
%    \end{macro}
%    \begin{macro}{\HoLogoCs@emTeX}
%    \begin{macrocode}
\def\HoLogoCs@emTeX#1{#1{e}{E}mTeX}%
%    \end{macrocode}
%    \end{macro}
%    \begin{macro}{\HoLogoBkm@emTeX}
%    \begin{macrocode}
\def\HoLogoBkm@emTeX#1{%
  #1{e}{E}m\hologo{TeX}%
}
%    \end{macrocode}
%    \end{macro}
%    \begin{macro}{\HoLogoHtml@emTeX}
%    \begin{macrocode}
\let\HoLogoHtml@emTeX\HoLogo@emTeX
%    \end{macrocode}
%    \end{macro}
%
% \subsubsection{\hologo{ExTeX}}
%
%    \begin{macro}{\HoLogo@ExTeX}
%    The definition is taken from the FAQ of the
%    project \hologo{ExTeX}
%    \cite{ExTeX-FAQ}.
%\begin{quote}
%\begin{verbatim}
%\def\ExTeX{%
%  \textrm{% Logo always with serifs
%    \ensuremath{%
%      \textstyle
%      \varepsilon_{%
%        \kern-0.15em%
%        \mathcal{X}%
%      }%
%    }%
%    \kern-.15em%
%    \TeX
%  }%
%}
%\end{verbatim}
%\end{quote}
%    \begin{macrocode}
\def\HoLogo@ExTeX#1{%
  \HoLogoFont@font{ExTeX}{rm}{%
    \ltx@mbox{%
      \HOLOGO@MathSetup
      $%
        \textstyle
        \varepsilon_{%
          \kern-0.15em%
          \HoLogoFont@font{ExTeX}{sy}{X}%
        }%
      $%
    }%
    \HOLOGO@discretionary
    \kern-.15em%
    \hologo{TeX}%
  }%
}
%    \end{macrocode}
%    \end{macro}
%    \begin{macro}{\HoLogoHtml@ExTeX}
%    \begin{macrocode}
\def\HoLogoHtml@ExTeX#1{%
  \HoLogoCss@ExTeX
  \HoLogoFont@font{ExTeX}{rm}{%
    \HOLOGO@Span{ExTeX}{%
      \ltx@mbox{%
        \HOLOGO@MathSetup
        $\textstyle\varepsilon$%
        \HOLOGO@Span{X}{$\textstyle\chi$}%
        \hologo{TeX}%
      }%
    }%
  }%
}
%    \end{macrocode}
%    \end{macro}
%    \begin{macro}{\HoLogoBkm@ExTeX}
%    \begin{macrocode}
\def\HoLogoBkm@ExTeX#1{%
  \HOLOGO@PdfdocUnicode{#1{e}{E}x}{\textepsilon\textchi}%
  \hologo{TeX}%
}
%    \end{macrocode}
%    \end{macro}
%    \begin{macro}{\HoLogoCss@ExTeX}
%    \begin{macrocode}
\def\HoLogoCss@ExTeX{%
  \Css{%
    span.HoLogo-ExTeX{%
      font-family:serif;%
    }%
  }%
  \Css{%
    span.HoLogo-ExTeX span.HoLogo-TeX{%
      margin-left:-.15em;%
    }%
  }%
  \global\let\HoLogoCss@ExTeX\relax
}
%    \end{macrocode}
%    \end{macro}
%
% \subsubsection{\hologo{MiKTeX}}
%
%    \begin{macro}{\HoLogo@MiKTeX}
%    \begin{macrocode}
\def\HoLogo@MiKTeX#1{%
  \HOLOGO@mbox{MiK}%
  \HOLOGO@discretionary
  \hologo{TeX}%
}
%    \end{macrocode}
%    \end{macro}
%    \begin{macro}{\HoLogoHtml@MiKTeX}
%    \begin{macrocode}
\let\HoLogoHtml@MiKTeX\HoLogo@MiKTeX
%    \end{macrocode}
%    \end{macro}
%
% \subsubsection{\hologo{OzTeX} and friends}
%
%    Source: \hologo{OzTeX} FAQ \cite{OzTeX}:
%    \begin{quote}
%      |\def\OzTeX{O\kern-.03em z\kern-.15em\TeX}|\\
%      (There is no kerning in OzMF, OzMP and OzTtH.)
%    \end{quote}
%
%    \begin{macro}{\HoLogo@OzTeX}
%    \begin{macrocode}
\def\HoLogo@OzTeX#1{%
  O%
  \kern-.03em %
  z%
  \kern-.15em %
  \hologo{TeX}%
}
%    \end{macrocode}
%    \end{macro}
%    \begin{macro}{\HoLogoHtml@OzTeX}
%    \begin{macrocode}
\def\HoLogoHtml@OzTeX#1{%
  \HoLogoCss@OzTeX
  \HOLOGO@Span{OzTeX}{%
    O%
    \HOLOGO@Span{z}{z}%
    \hologo{TeX}%
  }%
}
%    \end{macrocode}
%    \end{macro}
%    \begin{macro}{\HoLogoCss@OzTeX}
%    \begin{macrocode}
\def\HoLogoCss@OzTeX{%
  \Css{%
    span.HoLogo-OzTeX span.HoLogo-z{%
      margin-left:-.03em;%
      margin-right:-.15em;%
    }%
  }%
  \global\let\HoLogoCss@OzTeX\relax
}
%    \end{macrocode}
%    \end{macro}
%
%    \begin{macro}{\HoLogo@OzMF}
%    \begin{macrocode}
\def\HoLogo@OzMF#1{%
  \HOLOGO@mbox{OzMF}%
}
%    \end{macrocode}
%    \end{macro}
%    \begin{macro}{\HoLogo@OzMP}
%    \begin{macrocode}
\def\HoLogo@OzMP#1{%
  \HOLOGO@mbox{OzMP}%
}
%    \end{macrocode}
%    \end{macro}
%    \begin{macro}{\HoLogo@OzTtH}
%    \begin{macrocode}
\def\HoLogo@OzTtH#1{%
  \HOLOGO@mbox{OzTtH}%
}
%    \end{macrocode}
%    \end{macro}
%
% \subsubsection{\hologo{PCTeX}}
%
%    \begin{macro}{\HoLogo@PCTeX}
%    \begin{macrocode}
\def\HoLogo@PCTeX#1{%
  \HOLOGO@mbox{PC}%
  \hologo{TeX}%
}
%    \end{macrocode}
%    \end{macro}
%    \begin{macro}{\HoLogoHtml@PCTeX}
%    \begin{macrocode}
\let\HoLogoHtml@PCTeX\HoLogo@PCTeX
%    \end{macrocode}
%    \end{macro}
%
% \subsubsection{\hologo{PiCTeX}}
%
%    The original definitions from \xfile{pictex.tex} \cite{PiCTeX}:
%\begin{quote}
%\begin{verbatim}
%\def\PiC{%
%  P%
%  \kern-.12em%
%  \lower.5ex\hbox{I}%
%  \kern-.075em%
%  C%
%}
%\def\PiCTeX{%
%  \PiC
%  \kern-.11em%
%  \TeX
%}
%\end{verbatim}
%\end{quote}
%
%    \begin{macro}{\HoLogo@PiC}
%    \begin{macrocode}
\def\HoLogo@PiC#1{%
  P%
  \kern-.12em%
  \lower.5ex\hbox{I}%
  \kern-.075em%
  C%
  \HOLOGO@SpaceFactor
}
%    \end{macrocode}
%    \end{macro}
%    \begin{macro}{\HoLogoHtml@PiC}
%    \begin{macrocode}
\def\HoLogoHtml@PiC#1{%
  \HoLogoCss@PiC
  \HOLOGO@Span{PiC}{%
    P%
    \HOLOGO@Span{i}{I}%
    C%
  }%
}
%    \end{macrocode}
%    \end{macro}
%    \begin{macro}{\HoLogoCss@PiC}
%    \begin{macrocode}
\def\HoLogoCss@PiC{%
  \Css{%
    span.HoLogo-PiC span.HoLogo-i{%
      position:relative;%
      top:.5ex;%
      margin-left:-.12em;%
      margin-right:-.075em;%
      text-decoration:none;%
    }%
  }%
  \global\let\HoLogoCss@PiC\relax
}
%    \end{macrocode}
%    \end{macro}
%
%    \begin{macro}{\HoLogo@PiCTeX}
%    \begin{macrocode}
\def\HoLogo@PiCTeX#1{%
  \hologo{PiC}%
  \HOLOGO@discretionary
  \kern-.11em%
  \hologo{TeX}%
}
%    \end{macrocode}
%    \end{macro}
%    \begin{macro}{\HoLogoHtml@PiCTeX}
%    \begin{macrocode}
\def\HoLogoHtml@PiCTeX#1{%
  \HoLogoCss@PiCTeX
  \HOLOGO@Span{PiCTeX}{%
    \hologo{PiC}%
    \hologo{TeX}%
  }%
}
%    \end{macrocode}
%    \end{macro}
%    \begin{macro}{\HoLogoCss@PiCTeX}
%    \begin{macrocode}
\def\HoLogoCss@PiCTeX{%
  \Css{%
    span.HoLogo-PiCTeX span.HoLogo-PiC{%
      margin-right:-.11em;%
    }%
  }%
  \global\let\HoLogoCss@PiCTeX\relax
}
%    \end{macrocode}
%    \end{macro}
%
% \subsubsection{\hologo{teTeX}}
%
%    \begin{macro}{\HoLogo@teTeX}
%    \begin{macrocode}
\def\HoLogo@teTeX#1{%
  \HOLOGO@mbox{#1{t}{T}e}%
  \HOLOGO@discretionary
  \hologo{TeX}%
}
%    \end{macrocode}
%    \end{macro}
%    \begin{macro}{\HoLogoCs@teTeX}
%    \begin{macrocode}
\def\HoLogoCs@teTeX#1{#1{t}{T}dfTeX}
%    \end{macrocode}
%    \end{macro}
%    \begin{macro}{\HoLogoBkm@teTeX}
%    \begin{macrocode}
\def\HoLogoBkm@teTeX#1{%
  #1{t}{T}e\hologo{TeX}%
}
%    \end{macrocode}
%    \end{macro}
%    \begin{macro}{\HoLogoHtml@teTeX}
%    \begin{macrocode}
\let\HoLogoHtml@teTeX\HoLogo@teTeX
%    \end{macrocode}
%    \end{macro}
%
% \subsubsection{\hologo{TeX4ht}}
%
%    \begin{macro}{\HoLogo@TeX4ht}
%    \begin{macrocode}
\expandafter\def\csname HoLogo@TeX4ht\endcsname#1{%
  \HOLOGO@mbox{\hologo{TeX}4ht}%
}
%    \end{macrocode}
%    \end{macro}
%    \begin{macro}{\HoLogoHtml@TeX4ht}
%    \begin{macrocode}
\expandafter
\let\csname HoLogoHtml@TeX4ht\expandafter\endcsname
\csname HoLogo@TeX4ht\endcsname
%    \end{macrocode}
%    \end{macro}
%
%
% \subsubsection{\hologo{SageTeX}}
%
%    \begin{macro}{\HoLogo@SageTeX}
%    \begin{macrocode}
\def\HoLogo@SageTeX#1{%
  \HOLOGO@mbox{Sage}%
  \HOLOGO@discretionary
  \HOLOGO@NegativeKerning{eT,oT,To}%
  \hologo{TeX}%
}
%    \end{macrocode}
%    \end{macro}
%    \begin{macro}{\HoLogoHtml@SageTeX}
%    \begin{macrocode}
\let\HoLogoHtml@SageTeX\HoLogo@SageTeX
%    \end{macrocode}
%    \end{macro}
%
% \subsection{\hologo{METAFONT} and friends}
%
%    \begin{macro}{\HoLogo@METAFONT}
%    \begin{macrocode}
\def\HoLogo@METAFONT#1{%
  \HoLogoFont@font{METAFONT}{logo}{%
    \HOLOGO@mbox{META}%
    \HOLOGO@discretionary
    \HOLOGO@mbox{FONT}%
  }%
}
%    \end{macrocode}
%    \end{macro}
%
%    \begin{macro}{\HoLogo@METAPOST}
%    \begin{macrocode}
\def\HoLogo@METAPOST#1{%
  \HoLogoFont@font{METAPOST}{logo}{%
    \HOLOGO@mbox{META}%
    \HOLOGO@discretionary
    \HOLOGO@mbox{POST}%
  }%
}
%    \end{macrocode}
%    \end{macro}
%
%    \begin{macro}{\HoLogo@MetaFun}
%    \begin{macrocode}
\def\HoLogo@MetaFun#1{%
  \HOLOGO@mbox{Meta}%
  \HOLOGO@discretionary
  \HOLOGO@mbox{Fun}%
}
%    \end{macrocode}
%    \end{macro}
%
%    \begin{macro}{\HoLogo@MetaPost}
%    \begin{macrocode}
\def\HoLogo@MetaPost#1{%
  \HOLOGO@mbox{Meta}%
  \HOLOGO@discretionary
  \HOLOGO@mbox{Post}%
}
%    \end{macrocode}
%    \end{macro}
%
% \subsection{Others}
%
% \subsubsection{\hologo{biber}}
%
%    \begin{macro}{\HoLogo@biber}
%    \begin{macrocode}
\def\HoLogo@biber#1{%
  \HOLOGO@mbox{#1{b}{B}i}%
  \HOLOGO@discretionary
  \HOLOGO@mbox{ber}%
}
%    \end{macrocode}
%    \end{macro}
%    \begin{macro}{\HoLogoCs@biber}
%    \begin{macrocode}
\def\HoLogoCs@biber#1{#1{b}{B}iber}
%    \end{macrocode}
%    \end{macro}
%    \begin{macro}{\HoLogoBkm@biber}
%    \begin{macrocode}
\def\HoLogoBkm@biber#1{%
  #1{b}{B}iber%
}
%    \end{macrocode}
%    \end{macro}
%    \begin{macro}{\HoLogoHtml@biber}
%    \begin{macrocode}
\let\HoLogoHtml@biber\HoLogo@biber
%    \end{macrocode}
%    \end{macro}
%
% \subsubsection{\hologo{KOMAScript}}
%
%    \begin{macro}{\HoLogo@KOMAScript}
%    The definition for \hologo{KOMAScript} is taken
%    from \hologo{KOMAScript} (\xfile{scrlogo.dtx}, reformatted) \cite{scrlogo}:
%\begin{quote}
%\begin{verbatim}
%\@ifundefined{KOMAScript}{%
%  \DeclareRobustCommand{\KOMAScript}{%
%    \textsf{%
%      K\kern.05em O\kern.05emM\kern.05em A%
%      \kern.1em-\kern.1em %
%      Script%
%    }%
%  }%
%}{}
%\end{verbatim}
%\end{quote}
%    \begin{macrocode}
\def\HoLogo@KOMAScript#1{%
  \HoLogoFont@font{KOMAScript}{sf}{%
    \HOLOGO@mbox{%
      K\kern.05em%
      O\kern.05em%
      M\kern.05em%
      A%
    }%
    \kern.1em%
    \HOLOGO@hyphen
    \kern.1em%
    \HOLOGO@mbox{Script}%
  }%
}
%    \end{macrocode}
%    \end{macro}
%    \begin{macro}{\HoLogoBkm@KOMAScript}
%    \begin{macrocode}
\def\HoLogoBkm@KOMAScript#1{%
  KOMA-Script%
}
%    \end{macrocode}
%    \end{macro}
%    \begin{macro}{\HoLogoHtml@KOMAScript}
%    \begin{macrocode}
\def\HoLogoHtml@KOMAScript#1{%
  \HoLogoCss@KOMAScript
  \HoLogoFont@font{KOMAScript}{sf}{%
    \HOLOGO@Span{KOMAScript}{%
      K%
      \HOLOGO@Span{O}{O}%
      M%
      \HOLOGO@Span{A}{A}%
      \HOLOGO@Span{hyphen}{-}%
      Script%
    }%
  }%
}
%    \end{macrocode}
%    \end{macro}
%    \begin{macro}{\HoLogoCss@KOMAScript}
%    \begin{macrocode}
\def\HoLogoCss@KOMAScript{%
  \Css{%
    span.HoLogo-KOMAScript{%
      font-family:sans-serif;%
    }%
  }%
  \Css{%
    span.HoLogo-KOMAScript span.HoLogo-O{%
      padding-left:.05em;%
      padding-right:.05em;%
    }%
  }%
  \Css{%
    span.HoLogo-KOMAScript span.HoLogo-A{%
      padding-left:.05em;%
    }%
  }%
  \Css{%
    span.HoLogo-KOMAScript span.HoLogo-hyphen{%
      padding-left:.1em;%
      padding-right:.1em;%
    }%
  }%
  \global\let\HoLogoCss@KOMAScript\relax
}
%    \end{macrocode}
%    \end{macro}
%
% \subsubsection{\hologo{LyX}}
%
%    \begin{macro}{\HoLogo@LyX}
%    The definition is taken from the documentation source files
%    of \hologo{LyX}, \xfile{Intro.lyx} \cite{LyX}:
%\begin{quote}
%\begin{verbatim}
%\def\LyX{%
%  \texorpdfstring{%
%    L\kern-.1667em\lower.25em\hbox{Y}\kern-.125emX\@%
%  }{%
%    LyX%
%  }%
%}
%\end{verbatim}
%\end{quote}
%    \begin{macrocode}
\def\HoLogo@LyX#1{%
  L%
  \kern-.1667em%
  \lower.25em\hbox{Y}%
  \kern-.125em%
  X%
  \HOLOGO@SpaceFactor
}
%    \end{macrocode}
%    \end{macro}
%    \begin{macro}{\HoLogoHtml@LyX}
%    \begin{macrocode}
\def\HoLogoHtml@LyX#1{%
  \HoLogoCss@LyX
  \HOLOGO@Span{LyX}{%
    L%
    \HOLOGO@Span{y}{Y}%
    X%
  }%
}
%    \end{macrocode}
%    \end{macro}
%    \begin{macro}{\HoLogoCss@LyX}
%    \begin{macrocode}
\def\HoLogoCss@LyX{%
  \Css{%
    span.HoLogo-LyX span.HoLogo-y{%
      position:relative;%
      top:.25em;%
      margin-left:-.1667em;%
      margin-right:-.125em;%
      text-decoration:none;%
    }%
  }%
  \global\let\HoLogoCss@LyX\relax
}
%    \end{macrocode}
%    \end{macro}
%
% \subsubsection{\hologo{NTS}}
%
%    \begin{macro}{\HoLogo@NTS}
%    Definition for \hologo{NTS} can be found in
%    package \xpackage{etex\textunderscore man} for the \hologo{eTeX} manual \cite{etexman}
%    and in package \xpackage{dtklogos} \cite{dtklogos}:
%\begin{quote}
%\begin{verbatim}
%\def\NTS{%
%  \leavevmode
%  \hbox{%
%    $%
%      \cal N%
%      \kern-0.35em%
%      \lower0.5ex\hbox{$\cal T$}%
%      \kern-0.2em%
%      S%
%    $%
%  }%
%}
%\end{verbatim}
%\end{quote}
%    \begin{macrocode}
\def\HoLogo@NTS#1{%
  \HoLogoFont@font{NTS}{sy}{%
    N\/%
    \kern-.35em%
    \lower.5ex\hbox{T\/}%
    \kern-.2em%
    S\/%
  }%
  \HOLOGO@SpaceFactor
}
%    \end{macrocode}
%    \end{macro}
%
% \subsubsection{\Hologo{TTH} (\hologo{TeX} to HTML translator)}
%
%    Source: \url{http://hutchinson.belmont.ma.us/tth/}
%    In the HTML source the second `T' is printed as subscript.
%\begin{quote}
%\begin{verbatim}
%T<sub>T</sub>H
%\end{verbatim}
%\end{quote}
%    \begin{macro}{\HoLogo@TTH}
%    \begin{macrocode}
\def\HoLogo@TTH#1{%
  \ltx@mbox{%
    T\HOLOGO@SubScript{T}H%
  }%
  \HOLOGO@SpaceFactor
}
%    \end{macrocode}
%    \end{macro}
%
%    \begin{macro}{\HoLogoHtml@TTH}
%    \begin{macrocode}
\def\HoLogoHtml@TTH#1{%
  T\HCode{<sub>}T\HCode{</sub>}H%
}
%    \end{macrocode}
%    \end{macro}
%
% \subsubsection{\Hologo{HanTheThanh}}
%
%    Partial source: Package \xpackage{dtklogos}.
%    The double accent is U+1EBF (latin small letter e with circumflex
%    and acute).
%    \begin{macro}{\HoLogo@HanTheThanh}
%    \begin{macrocode}
\def\HoLogo@HanTheThanh#1{%
  \ltx@mbox{H\`an}%
  \HOLOGO@space
  \ltx@mbox{%
    Th%
    \HOLOGO@IfCharExists{"1EBF}{%
      \char"1EBF\relax
    }{%
      \^e\hbox to 0pt{\hss\raise .5ex\hbox{\'{}}}%
    }%
  }%
  \HOLOGO@space
  \ltx@mbox{Th\`anh}%
}
%    \end{macrocode}
%    \end{macro}
%    \begin{macro}{\HoLogoBkm@HanTheThanh}
%    \begin{macrocode}
\def\HoLogoBkm@HanTheThanh#1{%
  H\`an %
  Th\HOLOGO@PdfdocUnicode{\^e}{\9036\277} %
  Th\`anh%
}
%    \end{macrocode}
%    \end{macro}
%    \begin{macro}{\HoLogoHtml@HanTheThanh}
%    \begin{macrocode}
\def\HoLogoHtml@HanTheThanh#1{%
  H\`an %
  Th\HCode{&\ltx@hashchar x1ebf;} %
  Th\`anh%
}
%    \end{macrocode}
%    \end{macro}
%
% \subsection{Driver detection}
%
%    \begin{macrocode}
\HOLOGO@IfExists\InputIfFileExists{%
  \InputIfFileExists{hologo.cfg}{}{}%
}{%
  \ltx@IfUndefined{pdf@filesize}{%
    \def\HOLOGO@InputIfExists{%
      \openin\HOLOGO@temp=hologo.cfg\relax
      \ifeof\HOLOGO@temp
        \closein\HOLOGO@temp
      \else
        \closein\HOLOGO@temp
        \begingroup
          \def\x{LaTeX2e}%
        \expandafter\endgroup
        \ifx\fmtname\x
          \input{hologo.cfg}%
        \else
          \input hologo.cfg\relax
        \fi
      \fi
    }%
    \ltx@IfUndefined{newread}{%
      \chardef\HOLOGO@temp=15 %
      \def\HOLOGO@CheckRead{%
        \ifeof\HOLOGO@temp
          \HOLOGO@InputIfExists
        \else
          \ifcase\HOLOGO@temp
            \@PackageWarningNoLine{hologo}{%
              Configuration file ignored, because\MessageBreak
              a free read register could not be found%
            }%
          \else
            \begingroup
              \count\ltx@cclv=\HOLOGO@temp
              \advance\ltx@cclv by \ltx@minusone
              \edef\x{\endgroup
                \chardef\noexpand\HOLOGO@temp=\the\count\ltx@cclv
                \relax
              }%
            \x
          \fi
        \fi
      }%
    }{%
      \csname newread\endcsname\HOLOGO@temp
      \HOLOGO@InputIfExists
    }%
  }{%
    \edef\HOLOGO@temp{\pdf@filesize{hologo.cfg}}%
    \ifx\HOLOGO@temp\ltx@empty
    \else
      \ifnum\HOLOGO@temp>0 %
        \begingroup
          \def\x{LaTeX2e}%
        \expandafter\endgroup
        \ifx\fmtname\x
          \input{hologo.cfg}%
        \else
          \input hologo.cfg\relax
        \fi
      \else
        \@PackageInfoNoLine{hologo}{%
          Empty configuration file `hologo.cfg' ignored%
        }%
      \fi
    \fi
  }%
}
%    \end{macrocode}
%
%    \begin{macrocode}
\def\HOLOGO@temp#1#2{%
  \kv@define@key{HoLogoDriver}{#1}[]{%
    \begingroup
      \def\HOLOGO@temp{##1}%
      \ltx@onelevel@sanitize\HOLOGO@temp
      \ifx\HOLOGO@temp\ltx@empty
      \else
        \@PackageError{hologo}{%
          Value (\HOLOGO@temp) not permitted for option `#1'%
        }%
        \@ehc
      \fi
    \endgroup
    \def\hologoDriver{#2}%
  }%
}%
\def\HOLOGO@@temp#1#2{%
  \ifx\kv@value\relax
    \HOLOGO@temp{#1}{#1}%
  \else
    \HOLOGO@temp{#1}{#2}%
  \fi
}%
\kv@parse@normalized{%
  pdftex,%
  luatex=pdftex,%
  dvipdfm,%
  dvipdfmx=dvipdfm,%
  dvips,%
  dvipsone=dvips,%
  xdvi=dvips,%
  xetex,%
  vtex,%
}\HOLOGO@@temp
%    \end{macrocode}
%
%    \begin{macrocode}
\kv@define@key{HoLogoDriver}{driverfallback}{%
  \def\HOLOGO@DriverFallback{#1}%
}
%    \end{macrocode}
%
%    \begin{macro}{\HOLOGO@DriverFallback}
%    \begin{macrocode}
\def\HOLOGO@DriverFallback{dvips}
%    \end{macrocode}
%    \end{macro}
%
%    \begin{macro}{\hologoDriverSetup}
%    \begin{macrocode}
\def\hologoDriverSetup{%
  \let\hologoDriver\ltx@undefined
  \HOLOGO@DriverSetup
}
%    \end{macrocode}
%    \end{macro}
%
%    \begin{macro}{\HOLOGO@DriverSetup}
%    \begin{macrocode}
\def\HOLOGO@DriverSetup#1{%
  \kvsetkeys{HoLogoDriver}{#1}%
  \HOLOGO@CheckDriver
  \ltx@ifundefined{hologoDriver}{%
    \begingroup
    \edef\x{\endgroup
      \noexpand\kvsetkeys{HoLogoDriver}{\HOLOGO@DriverFallback}%
    }\x
  }{}%
  \@PackageInfoNoLine{hologo}{Using driver `\hologoDriver'}%
}
%    \end{macrocode}
%    \end{macro}
%
%    \begin{macro}{\HOLOGO@CheckDriver}
%    \begin{macrocode}
\def\HOLOGO@CheckDriver{%
  \ifpdf
    \def\hologoDriver{pdftex}%
    \let\HOLOGO@pdfliteral\pdfliteral
    \ifluatex
      \ifx\pdfextension\@undefined\else
        \protected\def\pdfliteral{\pdfextension literal}%
        \let\HOLOGO@pdfliteral\pdfliteral
      \fi
      \ltx@IfUndefined{HOLOGO@pdfliteral}{%
        \ifnum\luatexversion<36 %
        \else
          \begingroup
            \let\HOLOGO@temp\endgroup
            \ifcase0%
                \directlua{%
                  if tex.enableprimitives then %
                    tex.enableprimitives('HOLOGO@', {'pdfliteral'})%
                  else %
                    tex.print('1')%
                  end%
                }%
                \ifx\HOLOGO@pdfliteral\@undefined 1\fi%
                \relax%
              \endgroup
              \let\HOLOGO@temp\relax
              \global\let\HOLOGO@pdfliteral\HOLOGO@pdfliteral
            \fi%
          \HOLOGO@temp
        \fi
      }{}%
    \fi
    \ltx@IfUndefined{HOLOGO@pdfliteral}{%
      \@PackageWarningNoLine{hologo}{%
        Cannot find \string\pdfliteral
      }%
    }{}%
  \else
    \ifxetex
      \def\hologoDriver{xetex}%
    \else
      \ifvtex
        \def\hologoDriver{vtex}%
      \fi
    \fi
  \fi
}
%    \end{macrocode}
%    \end{macro}
%
%    \begin{macro}{\HOLOGO@WarningUnsupportedDriver}
%    \begin{macrocode}
\def\HOLOGO@WarningUnsupportedDriver#1{%
  \@PackageWarningNoLine{hologo}{%
    Logo `#1' needs driver specific macros,\MessageBreak
    but driver `\hologoDriver' is not supported.\MessageBreak
    Use a different driver or\MessageBreak
    load package `graphics' or `pgf'%
  }%
}
%    \end{macrocode}
%    \end{macro}
%
% \subsubsection{Reflect box macros}
%
%    Skip driver part if not needed.
%    \begin{macrocode}
\ltx@IfUndefined{reflectbox}{}{%
  \ltx@IfUndefined{rotatebox}{}{%
    \HOLOGO@AtEnd
  }%
}
\ltx@IfUndefined{pgftext}{}{%
  \HOLOGO@AtEnd
}
\ltx@IfUndefined{psscalebox}{}{%
  \HOLOGO@AtEnd
}
%    \end{macrocode}
%
%    \begin{macrocode}
\def\HOLOGO@temp{LaTeX2e}
\ifx\fmtname\HOLOGO@temp
  \RequirePackage{kvoptions}[2011/06/30]%
  \ProcessKeyvalOptions{HoLogoDriver}%
\fi
\HOLOGO@DriverSetup{}
%    \end{macrocode}
%
%    \begin{macro}{\HOLOGO@ReflectBox}
%    \begin{macrocode}
\def\HOLOGO@ReflectBox#1{%
  \begingroup
    \setbox\ltx@zero\hbox{\begingroup#1\endgroup}%
    \setbox\ltx@two\hbox{%
      \kern\wd\ltx@zero
      \csname HOLOGO@ScaleBox@\hologoDriver\endcsname{-1}{1}{%
        \hbox to 0pt{\copy\ltx@zero\hss}%
      }%
    }%
    \wd\ltx@two=\wd\ltx@zero
    \box\ltx@two
  \endgroup
}
%    \end{macrocode}
%    \end{macro}
%
%    \begin{macro}{\HOLOGO@PointReflectBox}
%    \begin{macrocode}
\def\HOLOGO@PointReflectBox#1{%
  \begingroup
    \setbox\ltx@zero\hbox{\begingroup#1\endgroup}%
    \setbox\ltx@two\hbox{%
      \kern\wd\ltx@zero
      \raise\ht\ltx@zero\hbox{%
        \csname HOLOGO@ScaleBox@\hologoDriver\endcsname{-1}{-1}{%
          \hbox to 0pt{\copy\ltx@zero\hss}%
        }%
      }%
    }%
    \wd\ltx@two=\wd\ltx@zero
    \box\ltx@two
  \endgroup
}
%    \end{macrocode}
%    \end{macro}
%
%    We must define all variants because of dynamic driver setup.
%    \begin{macrocode}
\def\HOLOGO@temp#1#2{#2}
%    \end{macrocode}
%
%    \begin{macro}{\HOLOGO@ScaleBox@pdftex}
%    \begin{macrocode}
\HOLOGO@temp{pdftex}{%
  \def\HOLOGO@ScaleBox@pdftex#1#2#3{%
    \HOLOGO@pdfliteral{%
      q #1 0 0 #2 0 0 cm%
    }%
    #3%
    \HOLOGO@pdfliteral{%
      Q%
    }%
  }%
}
%    \end{macrocode}
%    \end{macro}
%    \begin{macro}{\HOLOGO@ScaleBox@dvips}
%    \begin{macrocode}
\HOLOGO@temp{dvips}{%
  \def\HOLOGO@ScaleBox@dvips#1#2#3{%
    \special{ps:%
      gsave %
      currentpoint %
      currentpoint translate %
      #1 #2 scale %
      neg exch neg exch translate%
    }%
    #3%
    \special{ps:%
      currentpoint %
      grestore %
      moveto%
    }%
  }%
}
%    \end{macrocode}
%    \end{macro}
%    \begin{macro}{\HOLOGO@ScaleBox@dvipdfm}
%    \begin{macrocode}
\HOLOGO@temp{dvipdfm}{%
  \let\HOLOGO@ScaleBox@dvipdfm\HOLOGO@ScaleBox@dvips
}
%    \end{macrocode}
%    \end{macro}
%    Since \hologo{XeTeX} v0.6.
%    \begin{macro}{\HOLOGO@ScaleBox@xetex}
%    \begin{macrocode}
\HOLOGO@temp{xetex}{%
  \def\HOLOGO@ScaleBox@xetex#1#2#3{%
    \special{x:gsave}%
    \special{x:scale #1 #2}%
    #3%
    \special{x:grestore}%
  }%
}
%    \end{macrocode}
%    \end{macro}
%    \begin{macro}{\HOLOGO@ScaleBox@vtex}
%    \begin{macrocode}
\HOLOGO@temp{vtex}{%
  \def\HOLOGO@ScaleBox@vtex#1#2#3{%
    \special{r(#1,0,0,#2,0,0}%
    #3%
    \special{r)}%
  }%
}
%    \end{macrocode}
%    \end{macro}
%
%    \begin{macrocode}
\HOLOGO@AtEnd%
%</package>
%    \end{macrocode}
%
% \section{Test}
%
% \subsection{Catcode checks for loading}
%
%    \begin{macrocode}
%<*test1>
%    \end{macrocode}
%    \begin{macrocode}
\catcode`\{=1 %
\catcode`\}=2 %
\catcode`\#=6 %
\catcode`\@=11 %
\expandafter\ifx\csname count@\endcsname\relax
  \countdef\count@=255 %
\fi
\expandafter\ifx\csname @gobble\endcsname\relax
  \long\def\@gobble#1{}%
\fi
\expandafter\ifx\csname @firstofone\endcsname\relax
  \long\def\@firstofone#1{#1}%
\fi
\expandafter\ifx\csname loop\endcsname\relax
  \expandafter\@firstofone
\else
  \expandafter\@gobble
\fi
{%
  \def\loop#1\repeat{%
    \def\body{#1}%
    \iterate
  }%
  \def\iterate{%
    \body
      \let\next\iterate
    \else
      \let\next\relax
    \fi
    \next
  }%
  \let\repeat=\fi
}%
\def\RestoreCatcodes{}
\count@=0 %
\loop
  \edef\RestoreCatcodes{%
    \RestoreCatcodes
    \catcode\the\count@=\the\catcode\count@\relax
  }%
\ifnum\count@<255 %
  \advance\count@ 1 %
\repeat

\def\RangeCatcodeInvalid#1#2{%
  \count@=#1\relax
  \loop
    \catcode\count@=15 %
  \ifnum\count@<#2\relax
    \advance\count@ 1 %
  \repeat
}
\def\RangeCatcodeCheck#1#2#3{%
  \count@=#1\relax
  \loop
    \ifnum#3=\catcode\count@
    \else
      \errmessage{%
        Character \the\count@\space
        with wrong catcode \the\catcode\count@\space
        instead of \number#3%
      }%
    \fi
  \ifnum\count@<#2\relax
    \advance\count@ 1 %
  \repeat
}
\def\space{ }
\expandafter\ifx\csname LoadCommand\endcsname\relax
  \def\LoadCommand{\input hologo.sty\relax}%
\fi
\def\Test{%
  \RangeCatcodeInvalid{0}{47}%
  \RangeCatcodeInvalid{58}{64}%
  \RangeCatcodeInvalid{91}{96}%
  \RangeCatcodeInvalid{123}{255}%
  \catcode`\@=12 %
  \catcode`\\=0 %
  \catcode`\%=14 %
  \LoadCommand
  \RangeCatcodeCheck{0}{36}{15}%
  \RangeCatcodeCheck{37}{37}{14}%
  \RangeCatcodeCheck{38}{47}{15}%
  \RangeCatcodeCheck{48}{57}{12}%
  \RangeCatcodeCheck{58}{63}{15}%
  \RangeCatcodeCheck{64}{64}{12}%
  \RangeCatcodeCheck{65}{90}{11}%
  \RangeCatcodeCheck{91}{91}{15}%
  \RangeCatcodeCheck{92}{92}{0}%
  \RangeCatcodeCheck{93}{96}{15}%
  \RangeCatcodeCheck{97}{122}{11}%
  \RangeCatcodeCheck{123}{255}{15}%
  \RestoreCatcodes
}
\Test
\csname @@end\endcsname
\end
%    \end{macrocode}
%    \begin{macrocode}
%</test1>
%    \end{macrocode}
%
% \subsection{Spacefactor}
%
%    The space factor must be 1000 after a logo. If it is greater 1000
%    then the following space is a space after a sentence closing point.
%    If the space factor is smaller 1000 then an immediate following
%    dot is interpreted as abbreviation, not sentence closing point.
%
%    \begin{macrocode}
%<*test-spacefactor>
\NeedsTeXFormat{LaTeX2e}
\documentclass{article}
\usepackage{hologo}[2016/05/12]
\usepackage{kvsetkeys}
\usepackage{qstest}
\IncludeTests{*}
\LogTests{log}{*}{*}
\begin{document}
\begin{qstest}{spacefactor}{spacefactor}
\newcommand*{\Test}[1]{%
  \sbox0{%
    \hologo{#1}%
    \Expect*{1000 (#1)}*{\the\spacefactor\space(#1)}%
  }%
}%
\makeatletter
\def\TestList{}
\def\hologoEntry#1#2#3{%
  \edef\TestList{%
    \ifx\TestList\@empty
    \else
      \TestList,%
    \fi
    #1%
    \ifx\\#2\\%
    \else
      ={variant=#2}%
    \fi
  }%
}
\hologoList
\expandafter\kv@parse@normalized\expandafter{%
  \TestList
}{%
  \begingroup
    \let\@logo=\kv@key
    \ifx\kv@value\relax
    \else
      \expandafter\hologoLogoSetup\expandafter\@logo\expandafter{%
        \kv@value
      }%
    \fi
    \Test\@logo
  \endgroup
  \@gobbletwo
}
\end{qstest}
\end{document}
%</test-spacefactor>
%    \end{macrocode}
%
% \subsection{Complete list}
%
%    \begin{macrocode}
%<*test-list>
\NeedsTeXFormat{LaTeX2e}
\documentclass[12pt,a4paper]{article}
\usepackage{hologo}[2016/05/12]
\usepackage[T1]{fontenc}
\usepackage{lmodern}
\usepackage{parskip}
\usepackage[unicode]{hyperref}[2011/09/28]
\usepackage{bookmark}[2011/09/19]
\bookmarksetup{%
  numbered,%
  open,%
  openlevel=2,%
}
\renewcommand*{\contentsname}{List of logos}
\begin{document}
\tableofcontents
\def\TestFont#1#2#3#4#5#6{%
  \begingroup
    \usefont{#3}{#4}{#5}{#6}%
    \HologoVariant{#1}{#2}/\hologoVariant{#1}{#2}%
    \quad
    \begingroup\scriptsize\hologoVariant{#1}{#2}\endgroup
    \quad
  \endgroup
  (#3/#4/#5/#6)%
  \par
}
\makeatletter
\def\hologoEntry#1#2#3{%
  \section{%
    \HologoVariant{#1}{#2}/\hologoVariant{#1}{#2} %
    {[#1\ifx\\#2\\\else\space(#2)\fi]}% hash-ok
  }% braces around [] because of bug in tex4ht
  \begingroup
    \hypersetup{unicode=false}%
    \bookmark[%
      dest=\@currentHref,%
      rellevel=1,%
      keeplevel,%
    ]{%
      \HologoVariant{#1}{#2}/\hologoVariant{#1}{#2} %
      (PDFDocEncoding)%
    }%
  \endgroup
  \TestFont{#1}{#2}{OT1}{cmr}{m}{n}%
  \TestFont{#1}{#2}{OT1}{cmss}{m}{n}%
  \TestFont{#1}{#2}{OT1}{cmr}{b}{n}%
  \TestFont{#1}{#2}{OT1}{cmr}{m}{it}%
  \TestFont{#1}{#2}{OT1}{cmtt}{m}{n}%
  \TestFont{#1}{#2}{T1}{lmr}{m}{n}%
  \TestFont{#1}{#2}{T1}{lmss}{m}{n}%
  \TestFont{#1}{#2}{T1}{lmr}{b}{n}%
  \TestFont{#1}{#2}{T1}{lmr}{m}{it}%
  \TestFont{#1}{#2}{T1}{lmtt}{m}{n}%
  \TestFont{#1}{#2}{T1}{lmvtt}{m}{n}%
  \TestFont{#1}{#2}{T1}{qtm}{m}{n}%
  \TestFont{#1}{#2}{T1}{qhv}{m}{n}%
  \TestFont{#1}{#2}{T1}{qtm}{b}{n}%
  \TestFont{#1}{#2}{T1}{qtm}{m}{it}%
  \TestFont{#1}{#2}{T1}{qcr}{m}{n}%
  \newpage
}
\makeatother
\hologoList
\end{document}
%</test-list>
%    \end{macrocode}
%
% \section{Installation}
%
% \subsection{Download}
%
% \paragraph{Package.} This package is available on
% CTAN\footnote{\url{ftp://ftp.ctan.org/tex-archive/}}:
% \begin{description}
% \item[\CTAN{macros/latex/contrib/oberdiek/hologo.dtx}] The source file.
% \item[\CTAN{macros/latex/contrib/oberdiek/hologo.pdf}] Documentation.
% \end{description}
%
%
% \paragraph{Bundle.} All the packages of the bundle `oberdiek'
% are also available in a TDS compliant ZIP archive. There
% the packages are already unpacked and the documentation files
% are generated. The files and directories obey the TDS standard.
% \begin{description}
% \item[\CTAN{install/macros/latex/contrib/oberdiek.tds.zip}]
% \end{description}
% \emph{TDS} refers to the standard ``A Directory Structure
% for \TeX\ Files'' (\CTAN{tds/tds.pdf}). Directories
% with \xfile{texmf} in their name are usually organized this way.
%
% \subsection{Bundle installation}
%
% \paragraph{Unpacking.} Unpack the \xfile{oberdiek.tds.zip} in the
% TDS tree (also known as \xfile{texmf} tree) of your choice.
% Example (linux):
% \begin{quote}
%   |unzip oberdiek.tds.zip -d ~/texmf|
% \end{quote}
%
% \paragraph{Script installation.}
% Check the directory \xfile{TDS:scripts/oberdiek/} for
% scripts that need further installation steps.
% Package \xpackage{attachfile2} comes with the Perl script
% \xfile{pdfatfi.pl} that should be installed in such a way
% that it can be called as \texttt{pdfatfi}.
% Example (linux):
% \begin{quote}
%   |chmod +x scripts/oberdiek/pdfatfi.pl|\\
%   |cp scripts/oberdiek/pdfatfi.pl /usr/local/bin/|
% \end{quote}
%
% \subsection{Package installation}
%
% \paragraph{Unpacking.} The \xfile{.dtx} file is a self-extracting
% \docstrip\ archive. The files are extracted by running the
% \xfile{.dtx} through \plainTeX:
% \begin{quote}
%   \verb|tex hologo.dtx|
% \end{quote}
%
% \paragraph{TDS.} Now the different files must be moved into
% the different directories in your installation TDS tree
% (also known as \xfile{texmf} tree):
% \begin{quote}
% \def\t{^^A
% \begin{tabular}{@{}>{\ttfamily}l@{ $\rightarrow$ }>{\ttfamily}l@{}}
%   hologo.sty & tex/generic/oberdiek/hologo.sty\\
%   hologo.pdf & doc/latex/oberdiek/hologo.pdf\\
%   example/hologo-example.tex & doc/latex/oberdiek/example/hologo-example.tex\\
%   test/hologo-test1.tex & doc/latex/oberdiek/test/hologo-test1.tex\\
%   test/hologo-test-spacefactor.tex & doc/latex/oberdiek/test/hologo-test-spacefactor.tex\\
%   test/hologo-test-list.tex & doc/latex/oberdiek/test/hologo-test-list.tex\\
%   hologo.dtx & source/latex/oberdiek/hologo.dtx\\
% \end{tabular}^^A
% }^^A
% \sbox0{\t}^^A
% \ifdim\wd0>\linewidth
%   \begingroup
%     \advance\linewidth by\leftmargin
%     \advance\linewidth by\rightmargin
%   \edef\x{\endgroup
%     \def\noexpand\lw{\the\linewidth}^^A
%   }\x
%   \def\lwbox{^^A
%     \leavevmode
%     \hbox to \linewidth{^^A
%       \kern-\leftmargin\relax
%       \hss
%       \usebox0
%       \hss
%       \kern-\rightmargin\relax
%     }^^A
%   }^^A
%   \ifdim\wd0>\lw
%     \sbox0{\small\t}^^A
%     \ifdim\wd0>\linewidth
%       \ifdim\wd0>\lw
%         \sbox0{\footnotesize\t}^^A
%         \ifdim\wd0>\linewidth
%           \ifdim\wd0>\lw
%             \sbox0{\scriptsize\t}^^A
%             \ifdim\wd0>\linewidth
%               \ifdim\wd0>\lw
%                 \sbox0{\tiny\t}^^A
%                 \ifdim\wd0>\linewidth
%                   \lwbox
%                 \else
%                   \usebox0
%                 \fi
%               \else
%                 \lwbox
%               \fi
%             \else
%               \usebox0
%             \fi
%           \else
%             \lwbox
%           \fi
%         \else
%           \usebox0
%         \fi
%       \else
%         \lwbox
%       \fi
%     \else
%       \usebox0
%     \fi
%   \else
%     \lwbox
%   \fi
% \else
%   \usebox0
% \fi
% \end{quote}
% If you have a \xfile{docstrip.cfg} that configures and enables \docstrip's
% TDS installing feature, then some files can already be in the right
% place, see the documentation of \docstrip.
%
% \subsection{Refresh file name databases}
%
% If your \TeX~distribution
% (\teTeX, \mikTeX, \dots) relies on file name databases, you must refresh
% these. For example, \teTeX\ users run \verb|texhash| or
% \verb|mktexlsr|.
%
% \subsection{Some details for the interested}
%
% \paragraph{Attached source.}
%
% The PDF documentation on CTAN also includes the
% \xfile{.dtx} source file. It can be extracted by
% AcrobatReader 6 or higher. Another option is \textsf{pdftk},
% e.g. unpack the file into the current directory:
% \begin{quote}
%   \verb|pdftk hologo.pdf unpack_files output .|
% \end{quote}
%
% \paragraph{Unpacking with \LaTeX.}
% The \xfile{.dtx} chooses its action depending on the format:
% \begin{description}
% \item[\plainTeX:] Run \docstrip\ and extract the files.
% \item[\LaTeX:] Generate the documentation.
% \end{description}
% If you insist on using \LaTeX\ for \docstrip\ (really,
% \docstrip\ does not need \LaTeX), then inform the autodetect routine
% about your intention:
% \begin{quote}
%   \verb|latex \let\install=y\input{hologo.dtx}|
% \end{quote}
% Do not forget to quote the argument according to the demands
% of your shell.
%
% \paragraph{Generating the documentation.}
% You can use both the \xfile{.dtx} or the \xfile{.drv} to generate
% the documentation. The process can be configured by the
% configuration file \xfile{ltxdoc.cfg}. For instance, put this
% line into this file, if you want to have A4 as paper format:
% \begin{quote}
%   \verb|\PassOptionsToClass{a4paper}{article}|
% \end{quote}
% An example follows how to generate the
% documentation with pdf\LaTeX:
% \begin{quote}
%\begin{verbatim}
%pdflatex hologo.dtx
%makeindex -s gind.ist hologo.idx
%pdflatex hologo.dtx
%makeindex -s gind.ist hologo.idx
%pdflatex hologo.dtx
%\end{verbatim}
% \end{quote}
%
% \section{Catalogue}
%
% The following XML file can be used as source for the
% \href{http://mirror.ctan.org/help/Catalogue/catalogue.html}{\TeX\ Catalogue}.
% The elements \texttt{caption} and \texttt{description} are imported
% from the original XML file from the Catalogue.
% The name of the XML file in the Catalogue is \xfile{hologo.xml}.
%    \begin{macrocode}
%<*catalogue>
<?xml version='1.0' encoding='us-ascii'?>
<!DOCTYPE entry SYSTEM 'catalogue.dtd'>
<entry datestamp='$Date$' modifier='$Author$' id='hologo'>
  <name>hologo</name>
  <caption>A collection of logos with bookmark support.</caption>
  <authorref id='auth:oberdiek'/>
  <copyright owner='Heiko Oberdiek' year='2010-2012'/>
  <license type='lppl1.3'/>
  <version number='1.10'/>
  <description>
    The package defines a single command <tt>\hologo</tt>, whose
    argument is the usual case-confused ASCII version of the logo.
    The command is bookmark-enabled, so that every logo becomes
    available in bookmarks without further work.
    <p/>
    The package is part of the <xref refid='oberdiek'>oberdiek</xref>
    bundle.
  </description>
  <documentation details='Package documentation'
      href='ctan:/macros/latex/contrib/oberdiek/hologo.pdf'/>
  <ctan file='true' path='/macros/latex/contrib/oberdiek/hologo.dtx'/>
  <miktex location='oberdiek'/>
  <texlive location='oberdiek'/>
  <install path='/macros/latex/contrib/oberdiek/oberdiek.tds.zip'/>
</entry>
%</catalogue>
%    \end{macrocode}
%
% \begin{thebibliography}{9}
% \raggedright
%
% \bibitem{btxdoc}
% Oren Patashnik,
% \textit{\hologo{BibTeX}ing},
% 1988-02-08.\\
% \CTAN{biblio/bibtex/base/}
%
% \bibitem{dtklogos}
% Gerd Neugebauer, DANTE,
% \textit{Package \xpackage{dtklogos}},
% 2011-04-25.\\
% \CTAN{usergrps/dante/dtk/dtklogos.sty}
%
% \bibitem{etexman}
% The \hologo{NTS} Team,
% \textit{The \hologo{eTeX} manual},
% 1998-02.\\
% \CTAN{systems/e-tex/v2/doc/}
%
% \bibitem{ExTeX-FAQ}
% The \hologo{ExTeX} group,
% \textit{\hologo{ExTeX}: FAQ -- How is \hologo{ExTeX} typeset?},
% 2007-04-14.\\
% \url{http://www.extex.org/documentation/faq.html}
%
% \bibitem{LyX}
% %@MISC{ LyX,
% %  title = {{LyX 2.0.0 -- The Document Processor [Computer software and manual]}},
% %  author = {{The LyX Team}},
% %  howpublished = {Internet: http://www.lyx.org},
% %  year = {2011-05-08},
% %  note = {Retrieved May 10, 2011, from http://www.lyx.org},
% %  url = {http://www.lyx.org/}
% %}
% The \hologo{LyX} Team,
% \textit{\hologo{LyX} -- The Document Processor},
% 2011-05-08.\\
% \url{http://www.lyx.org/}
%
% \bibitem{OzTeX}
% Andrew Trevorrow,
% \hologo{OzTeX} FAQ: What is the correct way to typeset ``\hologo{OzTeX}''?,
% 2011-09-15 (visited).
% \url{http://www.trevorrow.com/oztex/ozfaq.html#oztex-logo}
%
% \bibitem{PiCTeX}
% Michael Wichura,
% \textit{The \hologo{PiCTeX} macro package},
% 1987-09-21.
% \CTAN{graphics/pictex/}
%
% \bibitem{scrlogo}
% Markus Kohm,
% \textit{\hologo{KOMAScript} Datei \xfile{scrlogo.dtx}},
% 2009-01-30.\\
% \CTAN{install/macros/latex/contrib/komascript.tds.zip}
%
% \end{thebibliography}
%
% \begin{History}
%   \begin{Version}{2010/04/08 v1.0}
%   \item
%     The first version.
%   \end{Version}
%   \begin{Version}{2010/04/16 v1.1}
%   \item
%     \cs{Hologo} added for support of logos at start of a sentence.
%   \item
%     \cs{hologoSetup} and \cs{hologoLogoSetup} added.
%   \item
%     Options \xoption{break}, \xoption{hyphenbreak}, \xoption{spacebreak}
%     added.
%   \item
%     Variant support added by option \xoption{variant}.
%   \end{Version}
%   \begin{Version}{2010/04/24 v1.2}
%   \item
%     \hologo{LaTeX3} added.
%   \item
%     \hologo{VTeX} added.
%   \end{Version}
%   \begin{Version}{2010/11/21 v1.3}
%   \item
%     \hologo{iniTeX}, \hologo{virTeX} added.
%   \end{Version}
%   \begin{Version}{2011/03/25 v1.4}
%   \item
%     \hologo{ConTeXt} with variants added.
%   \item
%     Option \xoption{discretionarybreak} added as refinement for
%     option \xoption{break}.
%   \end{Version}
%   \begin{Version}{2011/04/21 v1.5}
%   \item
%     Wrong TDS directory for test files fixed.
%   \end{Version}
%   \begin{Version}{2011/10/01 v1.6}
%   \item
%     Support for package \xpackage{tex4ht} added.
%   \item
%     Support for \cs{csname} added if \cs{ifincsname} is available.
%   \item
%     New logos:
%     \hologo{(La)TeX},
%     \hologo{biber},
%     \hologo{BibTeX} (\xoption{sc}, \xoption{sf}),
%     \hologo{emTeX},
%     \hologo{ExTeX},
%     \hologo{KOMAScript},
%     \hologo{La},
%     \hologo{LyX},
%     \hologo{MiKTeX},
%     \hologo{NTS},
%     \hologo{OzMF},
%     \hologo{OzMP},
%     \hologo{OzTeX},
%     \hologo{OzTtH},
%     \hologo{PCTeX},
%     \hologo{PiC},
%     \hologo{PiCTeX},
%     \hologo{METAFONT},
%     \hologo{MetaFun},
%     \hologo{METAPOST},
%     \hologo{MetaPost},
%     \hologo{SLiTeX} (\xoption{lift}, \xoption{narrow}, \xoption{simple}),
%     \hologo{SliTeX} (\xoption{narrow}, \xoption{simple}, \xoption{lift}),
%     \hologo{teTeX}.
%   \item
%     Fixes:
%     \hologo{iniTeX},
%     \hologo{pdfLaTeX},
%     \hologo{pdfTeX},
%     \hologo{virTeX}.
%   \item
%     \cs{hologoFontSetup} and \cs{hologoLogoFontSetup} added.
%   \item
%     \cs{hologoVariant} and \cs{HologoVariant} added.
%   \end{Version}
%   \begin{Version}{2011/11/22 v1.7}
%   \item
%     New logos:
%     \hologo{BibTeX8},
%     \hologo{LaTeXML},
%     \hologo{SageTeX},
%     \hologo{TeX4ht},
%     \hologo{TTH}.
%   \item
%     \hologo{Xe} and friends: Driver stuff fixed.
%   \item
%     \hologo{Xe} and friends: Support for italic added.
%   \item
%     \hologo{Xe} and friends: Package support for \xpackage{pgf}
%     and \xpackage{pstricks} added.
%   \end{Version}
%   \begin{Version}{2011/11/29 v1.8}
%   \item
%     New logos:
%     \hologo{HanTheThanh}.
%   \end{Version}
%   \begin{Version}{2011/12/21 v1.9}
%   \item
%     Patch for package \xpackage{ifxetex} added for the case that
%     \cs{newif} is undefined in \hologo{iniTeX}.
%   \item
%     Some fixes for \hologo{iniTeX}.
%   \end{Version}
%   \begin{Version}{2012/04/26 v1.10}
%   \item
%     Fix in bookmark version of logo ``\hologo{HanTheThanh}''.
%   \end{Version}
%   \begin{Version}{2016/05/12 v1.11}
%   \item
%     Update HOLOGO@IfCharExists (previously in texlive)
%   \item define pdfliteral in current luatex.
%   \end{Version}
% \end{History}
%
% \PrintIndex
%
% \Finale
\endinput

%        (quote the arguments according to the demands of your shell)
%
% Documentation:
%    (a) If hologo.drv is present:
%           latex hologo.drv
%    (b) Without hologo.drv:
%           latex hologo.dtx; ...
%    The class ltxdoc loads the configuration file ltxdoc.cfg
%    if available. Here you can specify further options, e.g.
%    use A4 as paper format:
%       \PassOptionsToClass{a4paper}{article}
%
%    Programm calls to get the documentation (example):
%       pdflatex hologo.dtx
%       makeindex -s gind.ist hologo.idx
%       pdflatex hologo.dtx
%       makeindex -s gind.ist hologo.idx
%       pdflatex hologo.dtx
%
% Installation:
%    TDS:tex/generic/oberdiek/hologo.sty
%    TDS:doc/latex/oberdiek/hologo.pdf
%    TDS:doc/latex/oberdiek/example/hologo-example.tex
%    TDS:doc/latex/oberdiek/test/hologo-test1.tex
%    TDS:doc/latex/oberdiek/test/hologo-test-spacefactor.tex
%    TDS:doc/latex/oberdiek/test/hologo-test-list.tex
%    TDS:source/latex/oberdiek/hologo.dtx
%
%<*ignore>
\begingroup
  \catcode123=1 %
  \catcode125=2 %
  \def\x{LaTeX2e}%
\expandafter\endgroup
\ifcase 0\ifx\install y1\fi\expandafter
         \ifx\csname processbatchFile\endcsname\relax\else1\fi
         \ifx\fmtname\x\else 1\fi\relax
\else\csname fi\endcsname
%</ignore>
%<*install>
\input docstrip.tex
\Msg{************************************************************************}
\Msg{* Installation}
\Msg{* Package: hologo 2016/05/12 v1.11 A logo collection with bookmark support (HO)}
\Msg{************************************************************************}

\keepsilent
\askforoverwritefalse

\let\MetaPrefix\relax
\preamble

This is a generated file.

Project: hologo
Version: 2016/05/12 v1.11

Copyright (C) 2010-2012 by
   Heiko Oberdiek <heiko.oberdiek at googlemail.com>

This work may be distributed and/or modified under the
conditions of the LaTeX Project Public License, either
version 1.3c of this license or (at your option) any later
version. This version of this license is in
   http://www.latex-project.org/lppl/lppl-1-3c.txt
and the latest version of this license is in
   http://www.latex-project.org/lppl.txt
and version 1.3 or later is part of all distributions of
LaTeX version 2005/12/01 or later.

This work has the LPPL maintenance status "maintained".

This Current Maintainer of this work is Heiko Oberdiek.

The Base Interpreter refers to any `TeX-Format',
because some files are installed in TDS:tex/generic//.

This work consists of the main source file hologo.dtx
and the derived files
   hologo.sty, hologo.pdf, hologo.ins, hologo.drv, hologo-example.tex,
   hologo-test1.tex, hologo-test-spacefactor.tex,
   hologo-test-list.tex.

\endpreamble
\let\MetaPrefix\DoubleperCent

\generate{%
  \file{hologo.ins}{\from{hologo.dtx}{install}}%
  \file{hologo.drv}{\from{hologo.dtx}{driver}}%
  \usedir{tex/generic/oberdiek}%
  \file{hologo.sty}{\from{hologo.dtx}{package}}%
  \usedir{doc/latex/oberdiek/example}%
  \file{hologo-example.tex}{\from{hologo.dtx}{example}}%
  \usedir{doc/latex/oberdiek/test}%
  \file{hologo-test1.tex}{\from{hologo.dtx}{test1}}%
  \file{hologo-test-spacefactor.tex}{\from{hologo.dtx}{test-spacefactor}}%
  \file{hologo-test-list.tex}{\from{hologo.dtx}{test-list}}%
  \nopreamble
  \nopostamble
  \usedir{source/latex/oberdiek/catalogue}%
  \file{hologo.xml}{\from{hologo.dtx}{catalogue}}%
}

\catcode32=13\relax% active space
\let =\space%
\Msg{************************************************************************}
\Msg{*}
\Msg{* To finish the installation you have to move the following}
\Msg{* file into a directory searched by TeX:}
\Msg{*}
\Msg{*     hologo.sty}
\Msg{*}
\Msg{* To produce the documentation run the file `hologo.drv'}
\Msg{* through LaTeX.}
\Msg{*}
\Msg{* Happy TeXing!}
\Msg{*}
\Msg{************************************************************************}

\endbatchfile
%</install>
%<*ignore>
\fi
%</ignore>
%<*driver>
\NeedsTeXFormat{LaTeX2e}
\ProvidesFile{hologo.drv}%
  [2016/05/12 v1.11 A logo collection with bookmark support (HO)]%
\documentclass{ltxdoc}
\usepackage{holtxdoc}[2011/11/22]
\usepackage{hologo}[2016/05/12]
\usepackage{longtable}
\usepackage{array}
\usepackage{paralist}
%\usepackage[T1]{fontenc}
%\usepackage{lmodern}
\begin{document}
  \DocInput{hologo.dtx}%
\end{document}
%</driver>
% \fi
%
%
% \CharacterTable
%  {Upper-case    \A\B\C\D\E\F\G\H\I\J\K\L\M\N\O\P\Q\R\S\T\U\V\W\X\Y\Z
%   Lower-case    \a\b\c\d\e\f\g\h\i\j\k\l\m\n\o\p\q\r\s\t\u\v\w\x\y\z
%   Digits        \0\1\2\3\4\5\6\7\8\9
%   Exclamation   \!     Double quote  \"     Hash (number) \#
%   Dollar        \$     Percent       \%     Ampersand     \&
%   Acute accent  \'     Left paren    \(     Right paren   \)
%   Asterisk      \*     Plus          \+     Comma         \,
%   Minus         \-     Point         \.     Solidus       \/
%   Colon         \:     Semicolon     \;     Less than     \<
%   Equals        \=     Greater than  \>     Question mark \?
%   Commercial at \@     Left bracket  \[     Backslash     \\
%   Right bracket \]     Circumflex    \^     Underscore    \_
%   Grave accent  \`     Left brace    \{     Vertical bar  \|
%   Right brace   \}     Tilde         \~}
%
% \GetFileInfo{hologo.drv}
%
% \title{The \xpackage{hologo} package}
% \date{2016/05/12 v1.11}
% \author{Heiko Oberdiek\\\xemail{heiko.oberdiek at googlemail.com}}
%
% \maketitle
%
% \begin{abstract}
% This package starts a collection of logos with support for bookmarks
% strings.
% \end{abstract}
%
% \tableofcontents
%
% \section{Documentation}
%
% \subsection{Logo macros}
%
% \begin{declcs}{hologo} \M{name}
% \end{declcs}
% Macro \cs{hologo} sets the logo with name \meta{name}.
% The following table shows the supported names.
%
% \begingroup
%   \def\hologoEntry#1#2#3{^^A
%     #1&#2&\hologoLogoSetup{#1}{variant=#2}\hologo{#1}&#3\tabularnewline
%   }
%   \begin{longtable}{>{\ttfamily}l>{\ttfamily}lll}
%     \rmfamily\bfseries{name} & \rmfamily\bfseries variant
%     & \bfseries logo & \bfseries since\\
%     \hline
%     \endhead
%     \hologoList
%   \end{longtable}
% \endgroup
%
% \begin{declcs}{Hologo} \M{name}
% \end{declcs}
% Macro \cs{Hologo} starts the logo \meta{name} with an uppercase
% letter. As an exception small greek letters are not converted
% to uppercase. Examples, see \hologo{eTeX} and \hologo{ExTeX}.
%
% \subsection{Setup macros}
%
% The package does not support package options, but the following
% setup macros can be used to set options.
%
% \begin{declcs}{hologoSetup} \M{key value list}
% \end{declcs}
% Macro \cs{hologoSetup} sets global options.
%
% \begin{declcs}{hologoLogoSetup} \M{logo} \M{key value list}
% \end{declcs}
% Some options can also be used to configure a logo.
% These settings take precedence over global option settings.
%
% \subsection{Options}\label{sec:options}
%
% There are boolean and string options:
% \begin{description}
% \item[Boolean option:]
% It takes |true| or |false|
% as value. If the value is omitted, then |true| is used.
% \item[String option:]
% A value must be given as string. (But the string might be empty.)
% \end{description}
% The following options can be used both in \cs{hologoSetup}
% and \cs{hologoLogoSetup}:
% \begin{description}
% \def\entry#1{\item[\xoption{#1}:]}
% \entry{break}
%   enables or disables line breaks inside the logo. This setting is
%   refined by options \xoption{hyphenbreak}, \xoption{spacebreak}
%   or \xoption{discretionarybreak}.
%   Default is |false|.
% \entry{hyphenbreak}
%   enables or disables the line break right after the hyphen character.
% \entry{spacebreak}
%   enables or disables line breaks at space characters.
% \entry{discretionarybreak}
%   enables or disables line breaks at hyphenation points
%   (inserted by \cs{-}).
% \end{description}
% Macro \cs{hologoLogoSetup} also knows:
% \begin{description}
% \item[\xoption{variant}:]
%   This is a string option. It specifies a variant of a logo that
%   must exist. An empty string selects the package default variant.
% \end{description}
% Example:
% \begin{quote}
%   |\hologoSetup{break=false}|\\
%   |\hologoLogoSetup{plainTeX}{variant=hyphen,hyphenbreak}|\\
%   Then ``plain-\TeX'' contains one break point after the hyphen.
% \end{quote}
%
% \subsection{Driver options}
%
% Sometimes graphical operations are needed to construct some
% glyphs (e.g.\ \hologo{XeTeX}). If package \xpackage{graphics}
% or package \xpackage{pgf} are found, then the macros are taken
% from there. Otherwise the packge defines its own operations
% and therefore needs the driver information. Many drivers are
% detected automatically (\hologo{pdfTeX}/\hologo{LuaTeX}
% in PDF mode, \hologo{XeTeX}, \hologo{VTeX}). These have precedence
% over a driver option. The driver can be given as package option
% or using \cs{hologoDriverSetup}.
% The following list contains the recognized driver options:
% \begin{itemize}
% \item \xoption{pdftex}, \xoption{luatex}
% \item \xoption{dvipdfm}, \xoption{dvipdfmx}
% \item \xoption{dvips}, \xoption{dvipsone}, \xoption{xdvi}
% \item \xoption{xetex}
% \item \xoption{vtex}
% \end{itemize}
% The left driver of a line is the driver name that is used internally.
% The following names are aliases for drivers that use the
% same method. Therefore the entry in the \xext{log} file for
% the used driver prints the internally used driver name.
% \begin{description}
% \item[\xoption{driverfallback}:]
%   This option expects a driver that is used,
%   if the driver could not be detected automatically.
% \end{description}
%
% \begin{declcs}{hologoDriverSetup} \M{driver option}
% \end{declcs}
% The driver can also be configured after package loading
% using \cs{hologoDriverSetup}, also the way for \hologo{plainTeX}
% to setup the driver.
%
% \subsection{Font setup}
%
% Some logos require a special font, but should also be usable by
% \hologo{plainTeX}. Therefore the package provides some ways
% to influence the font settings. The options below
% take font settings as values. Both font commands
% such as \cs{sffamily} and macros that take one argument
% like \cs{textsf} can be used.
%
% \begin{declcs}{hologoFontSetup} \M{key value list}
% \end{declcs}
% Macro \cs{hologoFontSetup} sets the fonts for all logos.
% Supported keys:
% \begin{description}
% \def\entry#1{\item[\xoption{#1}:]}
% \entry{general}
%   This font is used for all logos. The default is empty.
%   That means no special font is used.
% \entry{bibsf}
%   This font is used for
%   {\hologoLogoSetup{BibTeX}{variant=sf}\hologo{BibTeX}}
%   with variant \xoption{sf}.
% \entry{rm}
%   This font is a serif font. It is used for \hologo{ExTeX}.
% \entry{sc}
%   This font specifies a small caps font. It is used for
%   {\hologoLogoSetup{BibTeX}{variant=sc}\hologo{BibTeX}}
%   with variant \xoption{sc}.
% \entry{sf}
%   This font specifies a sans serif font. The default
%   is \cs{sffamily}, then \cs{sf} is tried. Otherwise
%   a warning is given. It is used by \hologo{KOMAScript}.
% \entry{sy}
%   This is the font for math symbols (e.g. cmsy).
%   It is used by \hologo{AmS}, \hologo{NTS}, \hologo{ExTeX}.
% \entry{logo}
%   \hologo{METAFONT} and \hologo{METAPOST} are using that font.
%   In \hologo{LaTeX} \cs{logofamily} is used and
%   the definitions of package \xpackage{mflogo} are used
%   if the package is not loaded.
%   Otherwise the \cs{tenlogo} is used and defined
%   if it does not already exists.
% \end{description}
%
% \begin{declcs}{hologoLogoFontSetup} \M{logo} \M{key value list}
% \end{declcs}
% Fonts can also be set for a logo or logo component separately,
% see the following list.
% The keys are the same as for \cs{hologoFontSetup}.
%
% \begin{longtable}{>{\ttfamily}l>{\sffamily}ll}
%   \meta{logo} & keys & result\\
%   \hline
%   \endhead
%   BibTeX & bibsf & {\hologoLogoSetup{BibTeX}{variant=sf}\hologo{BibTeX}}\\[.5ex]
%   BibTeX & sc & {\hologoLogoSetup{BibTeX}{variant=sc}\hologo{BibTeX}}\\[.5ex]
%   ExTeX & rm & \hologo{ExTeX}\\
%   SliTeX & rm & \hologo{SliTeX}\\[.5ex]
%   AmS & sy & \hologo{AmS}\\
%   ExTeX & sy & \hologo{ExTeX}\\
%   NTS & sy & \hologo{NTS}\\[.5ex]
%   KOMAScript & sf & \hologo{KOMAScript}\\[.5ex]
%   METAFONT & logo & \hologo{METAFONT}\\
%   METAPOST & logo & \hologo{METAPOST}\\[.5ex]
%   SliTeX & sc \hologo{SliTeX}
% \end{longtable}
%
% \subsubsection{Font order}
%
% For all logos the font \xoption{general} is applied first.
% Example:
%\begin{quote}
%|\hologoFontSetup{general=\color{red}}|
%\end{quote}
% will print red logos.
% Then if the font uses a special font \xoption{sf}, for example,
% the font is applied that is setup by \cs{hologoLogoFontSetup}.
% If this font is not setup, then the common font setup
% by \cs{hologoFontSetup} is used. Otherwise a warning is given,
% that there is no font configured.
%
% \subsection{Additional user macros}
%
% Usually a variant of a logo is configured by using
% \cs{hologoLogoSetup}, because it is bad style to mix
% different variants of the same logo in the same text.
% There the following macros are a convenience for testing.
%
% \begin{declcs}{hologoVariant} \M{name} \M{variant}\\
%   \cs{HologoVariant} \M{name} \M{variant}
% \end{declcs}
% Logo \meta{name} is set using \meta{variant} that specifies
% explicitely which variant of the macro is used. If the argument
% is empty, then the default form of the logo is used
% (configurable by \cs{hologoLogoSetup}).
%
% \cs{HologoVariant} is used if the logo is set in a context
% that needs an uppercase first letter (beginning of a sentence, \dots).
%
% \begin{declcs}{hologoList}\\
%   \cs{hologoEntry} \M{logo} \M{variant} \M{since}
% \end{declcs}
% Macro \cs{hologoList} contains all logos that are provided
% by the package including variants. The list consists of calls
% of \cs{hologoEntry} with three arguments starting with the
% logo name \meta{logo} and its variant \meta{variant}. An empty
% variant means the current default. Argument \meta{since} specifies
% with version of the package \xpackage{hologo} is needed to get
% the logo. If the logo is fixed, then the date gets updated.
% Therefore the date \meta{since} is not exactly the date of
% the first introduction, but rather the date of the latest fix.
%
% Before \cs{hologoList} can be used, macro \cs{hologoEntry} needs
% a definition. The example file in section \ref{sec:example}
% shows applications of \cs{hologoList}.
%
% \subsection{Supported contexts}
%
% Macros \cs{hologo} and friends support special contexts:
% \begin{itemize}
% \item \hologo{LaTeX}'s protection mechanism.
% \item Bookmarks of package \xpackage{hyperref}.
% \item Package \xpackage{tex4ht}.
% \item The macros can be used inside \cs{csname} constructs,
%   if \cs{ifincsname} is available (\hologo{pdfTeX}, \hologo{XeTeX},
%   \hologo{LuaTeX}).
% \end{itemize}
%
% \subsection{Example}
% \label{sec:example}
%
% The following example prints the logos in different fonts.
%    \begin{macrocode}
%<*example>
%<<verbatim
\NeedsTeXFormat{LaTeX2e}
\documentclass[a4paper]{article}
\usepackage[
  hmargin=20mm,
  vmargin=20mm,
]{geometry}
\pagestyle{empty}
\usepackage{hologo}[2016/05/12]
\usepackage{longtable}
\usepackage{array}
\setlength{\extrarowheight}{2pt}
\usepackage[T1]{fontenc}
\usepackage{lmodern}
\usepackage{pdflscape}
\usepackage[
  pdfencoding=auto,
]{hyperref}
\hypersetup{
  pdfauthor={Heiko Oberdiek},
  pdftitle={Example for package `hologo'},
  pdfsubject={Logos with fonts lmr, lmss, qtm, qpl, qhv},
}
\usepackage{bookmark}

% Print the logo list on the console

\begingroup
  \typeout{}%
  \typeout{*** Begin of logo list ***}%
  \newcommand*{\hologoEntry}[3]{%
    \typeout{#1 \ifx\\#2\\\else(#2) \fi[#3]}%
  }%
  \hologoList
  \typeout{*** End of logo list ***}%
  \typeout{}%
\endgroup

\begin{document}
\begin{landscape}

  \section{Example file for package `hologo'}

  % Table for font names

  \begin{longtable}{>{\bfseries}ll}
    \textbf{font} & \textbf{Font name}\\
    \hline
    lmr & Latin Modern Roman\\
    lmss & Latin Modern Sans\\
    qtm & \TeX\ Gyre Termes\\
    qhv & \TeX\ Gyre Heros\\
    qpl & \TeX\ Gyre Pagella\\
  \end{longtable}

  % Logo list with logos in different fonts

  \begingroup
    \newcommand*{\SetVariant}[2]{%
      \ifx\\#2\\%
      \else
        \hologoLogoSetup{#1}{variant=#2}%
      \fi
    }%
    \newcommand*{\hologoEntry}[3]{%
      \SetVariant{#1}{#2}%
      \raisebox{1em}[0pt][0pt]{\hypertarget{#1@#2}{}}%
      \bookmark[%
        dest={#1@#2},%
      ]{%
        #1\ifx\\#2\\\else\space(#2)\fi: \Hologo{#1}, \hologo{#1} %
        [Unicode]%
      }%
      \hypersetup{unicode=false}%
      \bookmark[%
        dest={#1@#2},%
      ]{%
        #1\ifx\\#2\\\else\space(#2)\fi: \Hologo{#1}, \hologo{#1} %
        [PDFDocEncoding]%
      }%
      \texttt{#1}%
      &%
      \texttt{#2}%
      &%
      \Hologo{#1}%
      &%
      \SetVariant{#1}{#2}%
      \hologo{#1}%
      &%
      \SetVariant{#1}{#2}%
      \fontfamily{qtm}\selectfont
      \hologo{#1}%
      &%
      \SetVariant{#1}{#2}%
      \fontfamily{qpl}\selectfont
      \hologo{#1}%
      &%
      \SetVariant{#1}{#2}%
      \textsf{\hologo{#1}}%
      &%
      \SetVariant{#1}{#2}%
      \fontfamily{qhv}\selectfont
      \hologo{#1}%
      \tabularnewline
    }%
    \begin{longtable}{llllllll}%
      \textbf{\textit{logo}} & \textbf{\textit{variant}} &
      \texttt{\string\Hologo} &
      \textbf{lmr} & \textbf{qtm} & \textbf{qpl} &
      \textbf{lmss} & \textbf{qhv}
      \tabularnewline
      \hline
      \endhead
      \hologoList
    \end{longtable}%
  \endgroup

\end{landscape}
\end{document}
%verbatim
%</example>
%    \end{macrocode}
%
% \StopEventually{
% }
%
% \section{Implementation}
%    \begin{macrocode}
%<*package>
%    \end{macrocode}
%    Reload check, especially if the package is not used with \LaTeX.
%    \begin{macrocode}
\begingroup\catcode61\catcode48\catcode32=10\relax%
  \catcode13=5 % ^^M
  \endlinechar=13 %
  \catcode35=6 % #
  \catcode39=12 % '
  \catcode44=12 % ,
  \catcode45=12 % -
  \catcode46=12 % .
  \catcode58=12 % :
  \catcode64=11 % @
  \catcode123=1 % {
  \catcode125=2 % }
  \expandafter\let\expandafter\x\csname ver@hologo.sty\endcsname
  \ifx\x\relax % plain-TeX, first loading
  \else
    \def\empty{}%
    \ifx\x\empty % LaTeX, first loading,
      % variable is initialized, but \ProvidesPackage not yet seen
    \else
      \expandafter\ifx\csname PackageInfo\endcsname\relax
        \def\x#1#2{%
          \immediate\write-1{Package #1 Info: #2.}%
        }%
      \else
        \def\x#1#2{\PackageInfo{#1}{#2, stopped}}%
      \fi
      \x{hologo}{The package is already loaded}%
      \aftergroup\endinput
    \fi
  \fi
\endgroup%
%    \end{macrocode}
%    Package identification:
%    \begin{macrocode}
\begingroup\catcode61\catcode48\catcode32=10\relax%
  \catcode13=5 % ^^M
  \endlinechar=13 %
  \catcode35=6 % #
  \catcode39=12 % '
  \catcode40=12 % (
  \catcode41=12 % )
  \catcode44=12 % ,
  \catcode45=12 % -
  \catcode46=12 % .
  \catcode47=12 % /
  \catcode58=12 % :
  \catcode64=11 % @
  \catcode91=12 % [
  \catcode93=12 % ]
  \catcode123=1 % {
  \catcode125=2 % }
  \expandafter\ifx\csname ProvidesPackage\endcsname\relax
    \def\x#1#2#3[#4]{\endgroup
      \immediate\write-1{Package: #3 #4}%
      \xdef#1{#4}%
    }%
  \else
    \def\x#1#2[#3]{\endgroup
      #2[{#3}]%
      \ifx#1\@undefined
        \xdef#1{#3}%
      \fi
      \ifx#1\relax
        \xdef#1{#3}%
      \fi
    }%
  \fi
\expandafter\x\csname ver@hologo.sty\endcsname
\ProvidesPackage{hologo}%
  [2016/05/12 v1.11 A logo collection with bookmark support (HO)]%
%    \end{macrocode}
%
%    \begin{macrocode}
\begingroup\catcode61\catcode48\catcode32=10\relax%
  \catcode13=5 % ^^M
  \endlinechar=13 %
  \catcode123=1 % {
  \catcode125=2 % }
  \catcode64=11 % @
  \def\x{\endgroup
    \expandafter\edef\csname HOLOGO@AtEnd\endcsname{%
      \endlinechar=\the\endlinechar\relax
      \catcode13=\the\catcode13\relax
      \catcode32=\the\catcode32\relax
      \catcode35=\the\catcode35\relax
      \catcode61=\the\catcode61\relax
      \catcode64=\the\catcode64\relax
      \catcode123=\the\catcode123\relax
      \catcode125=\the\catcode125\relax
    }%
  }%
\x\catcode61\catcode48\catcode32=10\relax%
\catcode13=5 % ^^M
\endlinechar=13 %
\catcode35=6 % #
\catcode64=11 % @
\catcode123=1 % {
\catcode125=2 % }
\def\TMP@EnsureCode#1#2{%
  \edef\HOLOGO@AtEnd{%
    \HOLOGO@AtEnd
    \catcode#1=\the\catcode#1\relax
  }%
  \catcode#1=#2\relax
}
\TMP@EnsureCode{10}{12}% ^^J
\TMP@EnsureCode{33}{12}% !
\TMP@EnsureCode{34}{12}% "
\TMP@EnsureCode{36}{3}% $
\TMP@EnsureCode{38}{4}% &
\TMP@EnsureCode{39}{12}% '
\TMP@EnsureCode{40}{12}% (
\TMP@EnsureCode{41}{12}% )
\TMP@EnsureCode{42}{12}% *
\TMP@EnsureCode{43}{12}% +
\TMP@EnsureCode{44}{12}% ,
\TMP@EnsureCode{45}{12}% -
\TMP@EnsureCode{46}{12}% .
\TMP@EnsureCode{47}{12}% /
\TMP@EnsureCode{58}{12}% :
\TMP@EnsureCode{59}{12}% ;
\TMP@EnsureCode{60}{12}% <
\TMP@EnsureCode{62}{12}% >
\TMP@EnsureCode{63}{12}% ?
\TMP@EnsureCode{91}{12}% [
\TMP@EnsureCode{93}{12}% ]
\TMP@EnsureCode{94}{7}% ^ (superscript)
\TMP@EnsureCode{95}{8}% _ (subscript)
\TMP@EnsureCode{96}{12}% `
\TMP@EnsureCode{124}{12}% |
\edef\HOLOGO@AtEnd{%
  \HOLOGO@AtEnd
  \escapechar\the\escapechar\relax
  \noexpand\endinput
}
\escapechar=92 %
%    \end{macrocode}
%
% \subsection{Logo list}
%
%    \begin{macro}{\hologoList}
%    \begin{macrocode}
\def\hologoList{%
  \hologoEntry{(La)TeX}{}{2011/10/01}%
  \hologoEntry{AmSLaTeX}{}{2010/04/16}%
  \hologoEntry{AmSTeX}{}{2010/04/16}%
  \hologoEntry{biber}{}{2011/10/01}%
  \hologoEntry{BibTeX}{}{2011/10/01}%
  \hologoEntry{BibTeX}{sf}{2011/10/01}%
  \hologoEntry{BibTeX}{sc}{2011/10/01}%
  \hologoEntry{BibTeX8}{}{2011/11/22}%
  \hologoEntry{ConTeXt}{}{2011/03/25}%
  \hologoEntry{ConTeXt}{narrow}{2011/03/25}%
  \hologoEntry{ConTeXt}{simple}{2011/03/25}%
  \hologoEntry{emTeX}{}{2010/04/26}%
  \hologoEntry{eTeX}{}{2010/04/08}%
  \hologoEntry{ExTeX}{}{2011/10/01}%
  \hologoEntry{HanTheThanh}{}{2011/11/29}%
  \hologoEntry{iniTeX}{}{2011/10/01}%
  \hologoEntry{KOMAScript}{}{2011/10/01}%
  \hologoEntry{La}{}{2010/05/08}%
  \hologoEntry{LaTeX}{}{2010/04/08}%
  \hologoEntry{LaTeX2e}{}{2010/04/08}%
  \hologoEntry{LaTeX3}{}{2010/04/24}%
  \hologoEntry{LaTeXe}{}{2010/04/08}%
  \hologoEntry{LaTeXML}{}{2011/11/22}%
  \hologoEntry{LaTeXTeX}{}{2011/10/01}%
  \hologoEntry{LuaLaTeX}{}{2010/04/08}%
  \hologoEntry{LuaTeX}{}{2010/04/08}%
  \hologoEntry{LyX}{}{2011/10/01}%
  \hologoEntry{METAFONT}{}{2011/10/01}%
  \hologoEntry{MetaFun}{}{2011/10/01}%
  \hologoEntry{METAPOST}{}{2011/10/01}%
  \hologoEntry{MetaPost}{}{2011/10/01}%
  \hologoEntry{MiKTeX}{}{2011/10/01}%
  \hologoEntry{NTS}{}{2011/10/01}%
  \hologoEntry{OzMF}{}{2011/10/01}%
  \hologoEntry{OzMP}{}{2011/10/01}%
  \hologoEntry{OzTeX}{}{2011/10/01}%
  \hologoEntry{OzTtH}{}{2011/10/01}%
  \hologoEntry{PCTeX}{}{2011/10/01}%
  \hologoEntry{pdfTeX}{}{2011/10/01}%
  \hologoEntry{pdfLaTeX}{}{2011/10/01}%
  \hologoEntry{PiC}{}{2011/10/01}%
  \hologoEntry{PiCTeX}{}{2011/10/01}%
  \hologoEntry{plainTeX}{}{2010/04/08}%
  \hologoEntry{plainTeX}{space}{2010/04/16}%
  \hologoEntry{plainTeX}{hyphen}{2010/04/16}%
  \hologoEntry{plainTeX}{runtogether}{2010/04/16}%
  \hologoEntry{SageTeX}{}{2011/11/22}%
  \hologoEntry{SLiTeX}{}{2011/10/01}%
  \hologoEntry{SLiTeX}{lift}{2011/10/01}%
  \hologoEntry{SLiTeX}{narrow}{2011/10/01}%
  \hologoEntry{SLiTeX}{simple}{2011/10/01}%
  \hologoEntry{SliTeX}{}{2011/10/01}%
  \hologoEntry{SliTeX}{narrow}{2011/10/01}%
  \hologoEntry{SliTeX}{simple}{2011/10/01}%
  \hologoEntry{SliTeX}{lift}{2011/10/01}%
  \hologoEntry{teTeX}{}{2011/10/01}%
  \hologoEntry{TeX}{}{2010/04/08}%
  \hologoEntry{TeX4ht}{}{2011/11/22}%
  \hologoEntry{TTH}{}{2011/11/22}%
  \hologoEntry{virTeX}{}{2011/10/01}%
  \hologoEntry{VTeX}{}{2010/04/24}%
  \hologoEntry{Xe}{}{2010/04/08}%
  \hologoEntry{XeLaTeX}{}{2010/04/08}%
  \hologoEntry{XeTeX}{}{2010/04/08}%
}
%    \end{macrocode}
%    \end{macro}
%
% \subsection{Load resources}
%
%    \begin{macrocode}
\begingroup\expandafter\expandafter\expandafter\endgroup
\expandafter\ifx\csname RequirePackage\endcsname\relax
  \def\TMP@RequirePackage#1[#2]{%
    \begingroup\expandafter\expandafter\expandafter\endgroup
    \expandafter\ifx\csname ver@#1.sty\endcsname\relax
      \input #1.sty\relax
    \fi
  }%
  \TMP@RequirePackage{ltxcmds}[2011/02/04]%
  \TMP@RequirePackage{infwarerr}[2010/04/08]%
  \TMP@RequirePackage{kvsetkeys}[2010/03/01]%
  \TMP@RequirePackage{kvdefinekeys}[2010/03/01]%
  \TMP@RequirePackage{pdftexcmds}[2010/04/01]%
  \TMP@RequirePackage{ifpdf}[2010/01/28]%
  \TMP@RequirePackage{ifluatex}[2010/03/01]%
  \ltx@IfUndefined{newif}{%
    \expandafter\let\csname newif\endcsname\ltx@newif
  }{}%
  \TMP@RequirePackage{ifxetex}[2009/01/23]%
  \TMP@RequirePackage{ifvtex}[2010/03/01]%
\else
  \RequirePackage{ltxcmds}[2011/02/04]%
  \RequirePackage{infwarerr}[2010/04/08]%
  \RequirePackage{kvsetkeys}[2010/03/01]%
  \RequirePackage{kvdefinekeys}[2010/03/01]%
  \RequirePackage{pdftexcmds}[2010/04/01]%
  \RequirePackage{ifpdf}[2010/01/28]%
  \RequirePackage{ifluatex}[2010/03/01]%
  \RequirePackage{ifxetex}[2009/01/23]%
  \RequirePackage{ifvtex}[2010/03/01]%
\fi
%    \end{macrocode}
%
%    \begin{macro}{\HOLOGO@IfDefined}
%    \begin{macrocode}
\def\HOLOGO@IfExists#1{%
  \ifx\@undefined#1%
    \expandafter\ltx@secondoftwo
  \else
    \ifx\relax#1%
      \expandafter\ltx@secondoftwo
    \else
      \expandafter\expandafter\expandafter\ltx@firstoftwo
    \fi
  \fi
}
%    \end{macrocode}
%    \end{macro}
%
% \subsection{Setup macros}
%
%    \begin{macro}{\hologoSetup}
%    \begin{macrocode}
\def\hologoSetup{%
  \let\HOLOGO@name\relax
  \HOLOGO@Setup
}
%    \end{macrocode}
%    \end{macro}
%
%    \begin{macro}{\hologoLogoSetup}
%    \begin{macrocode}
\def\hologoLogoSetup#1{%
  \edef\HOLOGO@name{#1}%
  \ltx@IfUndefined{HoLogo@\HOLOGO@name}{%
    \@PackageError{hologo}{%
      Unknown logo `\HOLOGO@name'%
    }\@ehc
    \ltx@gobble
  }{%
    \HOLOGO@Setup
  }%
}
%    \end{macrocode}
%    \end{macro}
%
%    \begin{macro}{\HOLOGO@Setup}
%    \begin{macrocode}
\def\HOLOGO@Setup{%
  \kvsetkeys{HoLogo}%
}
%    \end{macrocode}
%    \end{macro}
%
% \subsection{Options}
%
%    \begin{macro}{\HOLOGO@DeclareBoolOption}
%    \begin{macrocode}
\def\HOLOGO@DeclareBoolOption#1{%
  \expandafter\chardef\csname HOLOGOOPT@#1\endcsname\ltx@zero
  \kv@define@key{HoLogo}{#1}[true]{%
    \def\HOLOGO@temp{##1}%
    \ifx\HOLOGO@temp\HOLOGO@true
      \ifx\HOLOGO@name\relax
        \expandafter\chardef\csname HOLOGOOPT@#1\endcsname=\ltx@one
      \else
        \expandafter\chardef\csname
        HoLogoOpt@#1@\HOLOGO@name\endcsname\ltx@one
      \fi
      \HOLOGO@SetBreakAll{#1}%
    \else
      \ifx\HOLOGO@temp\HOLOGO@false
        \ifx\HOLOGO@name\relax
          \expandafter\chardef\csname HOLOGOOPT@#1\endcsname=\ltx@zero
        \else
          \expandafter\chardef\csname
          HoLogoOpt@#1@\HOLOGO@name\endcsname=\ltx@zero
        \fi
        \HOLOGO@SetBreakAll{#1}%
      \else
        \@PackageError{hologo}{%
          Unknown value `##1' for boolean option `#1'.\MessageBreak
          Known values are `true' and `false'%
        }\@ehc
      \fi
    \fi
  }%
}
%    \end{macrocode}
%    \end{macro}
%
%    \begin{macro}{\HOLOGO@SetBreakAll}
%    \begin{macrocode}
\def\HOLOGO@SetBreakAll#1{%
  \def\HOLOGO@temp{#1}%
  \ifx\HOLOGO@temp\HOLOGO@break
    \ifx\HOLOGO@name\relax
      \chardef\HOLOGOOPT@hyphenbreak=\HOLOGOOPT@break
      \chardef\HOLOGOOPT@spacebreak=\HOLOGOOPT@break
      \chardef\HOLOGOOPT@discretionarybreak=\HOLOGOOPT@break
    \else
      \expandafter\chardef
         \csname HoLogoOpt@hyphenbreak@\HOLOGO@name\endcsname=%
         \csname HoLogoOpt@break@\HOLOGO@name\endcsname
      \expandafter\chardef
         \csname HoLogoOpt@spacebreak@\HOLOGO@name\endcsname=%
         \csname HoLogoOpt@break@\HOLOGO@name\endcsname
      \expandafter\chardef
         \csname HoLogoOpt@discretionarybreak@\HOLOGO@name
             \endcsname=%
         \csname HoLogoOpt@break@\HOLOGO@name\endcsname
    \fi
  \fi
}
%    \end{macrocode}
%    \end{macro}
%
%    \begin{macro}{\HOLOGO@true}
%    \begin{macrocode}
\def\HOLOGO@true{true}
%    \end{macrocode}
%    \end{macro}
%    \begin{macro}{\HOLOGO@false}
%    \begin{macrocode}
\def\HOLOGO@false{false}
%    \end{macrocode}
%    \end{macro}
%    \begin{macro}{\HOLOGO@break}
%    \begin{macrocode}
\def\HOLOGO@break{break}
%    \end{macrocode}
%    \end{macro}
%
%    \begin{macrocode}
\HOLOGO@DeclareBoolOption{break}
\HOLOGO@DeclareBoolOption{hyphenbreak}
\HOLOGO@DeclareBoolOption{spacebreak}
\HOLOGO@DeclareBoolOption{discretionarybreak}
%    \end{macrocode}
%
%    \begin{macrocode}
\kv@define@key{HoLogo}{variant}{%
  \ifx\HOLOGO@name\relax
    \@PackageError{hologo}{%
      Option `variant' is not available in \string\hologoSetup,%
      \MessageBreak
      Use \string\hologoLogoSetup\space instead%
    }\@ehc
  \else
    \edef\HOLOGO@temp{#1}%
    \ifx\HOLOGO@temp\ltx@empty
      \expandafter
      \let\csname HoLogoOpt@variant@\HOLOGO@name\endcsname\@undefined
    \else
      \ltx@IfUndefined{HoLogo@\HOLOGO@name @\HOLOGO@temp}{%
        \@PackageError{hologo}{%
          Unknown variant `\HOLOGO@temp' of logo `\HOLOGO@name'%
        }\@ehc
      }{%
        \expandafter
        \let\csname HoLogoOpt@variant@\HOLOGO@name\endcsname
            \HOLOGO@temp
      }%
    \fi
  \fi
}
%    \end{macrocode}
%
%    \begin{macro}{\HOLOGO@Variant}
%    \begin{macrocode}
\def\HOLOGO@Variant#1{%
  #1%
  \ltx@ifundefined{HoLogoOpt@variant@#1}{%
  }{%
    @\csname HoLogoOpt@variant@#1\endcsname
  }%
}
%    \end{macrocode}
%    \end{macro}
%
% \subsection{Break/no-break support}
%
%    \begin{macro}{\HOLOGO@space}
%    \begin{macrocode}
\def\HOLOGO@space{%
  \ltx@ifundefined{HoLogoOpt@spacebreak@\HOLOGO@name}{%
    \ltx@ifundefined{HoLogoOpt@break@\HOLOGO@name}{%
      \chardef\HOLOGO@temp=\HOLOGOOPT@spacebreak
    }{%
      \chardef\HOLOGO@temp=%
        \csname HoLogoOpt@break@\HOLOGO@name\endcsname
    }%
  }{%
    \chardef\HOLOGO@temp=%
      \csname HoLogoOpt@spacebreak@\HOLOGO@name\endcsname
  }%
  \ifcase\HOLOGO@temp
    \penalty10000 %
  \fi
  \ltx@space
}
%    \end{macrocode}
%    \end{macro}
%
%    \begin{macro}{\HOLOGO@hyphen}
%    \begin{macrocode}
\def\HOLOGO@hyphen{%
  \ltx@ifundefined{HoLogoOpt@hyphenbreak@\HOLOGO@name}{%
    \ltx@ifundefined{HoLogoOpt@break@\HOLOGO@name}{%
      \chardef\HOLOGO@temp=\HOLOGOOPT@hyphenbreak
    }{%
      \chardef\HOLOGO@temp=%
        \csname HoLogoOpt@break@\HOLOGO@name\endcsname
    }%
  }{%
    \chardef\HOLOGO@temp=%
      \csname HoLogoOpt@hyphenbreak@\HOLOGO@name\endcsname
  }%
  \ifcase\HOLOGO@temp
    \ltx@mbox{-}%
  \else
    -%
  \fi
}
%    \end{macrocode}
%    \end{macro}
%
%    \begin{macro}{\HOLOGO@discretionary}
%    \begin{macrocode}
\def\HOLOGO@discretionary{%
  \ltx@ifundefined{HoLogoOpt@discretionarybreak@\HOLOGO@name}{%
    \ltx@ifundefined{HoLogoOpt@break@\HOLOGO@name}{%
      \chardef\HOLOGO@temp=\HOLOGOOPT@discretionarybreak
    }{%
      \chardef\HOLOGO@temp=%
        \csname HoLogoOpt@break@\HOLOGO@name\endcsname
    }%
  }{%
    \chardef\HOLOGO@temp=%
      \csname HoLogoOpt@discretionarybreak@\HOLOGO@name\endcsname
  }%
  \ifcase\HOLOGO@temp
  \else
    \-%
  \fi
}
%    \end{macrocode}
%    \end{macro}
%
%    \begin{macro}{\HOLOGO@mbox}
%    \begin{macrocode}
\def\HOLOGO@mbox#1{%
  \ltx@ifundefined{HoLogoOpt@break@\HOLOGO@name}{%
    \chardef\HOLOGO@temp=\HOLOGOOPT@hyphenbreak
  }{%
    \chardef\HOLOGO@temp=%
      \csname HoLogoOpt@break@\HOLOGO@name\endcsname
  }%
  \ifcase\HOLOGO@temp
    \ltx@mbox{#1}%
  \else
    #1%
  \fi
}
%    \end{macrocode}
%    \end{macro}
%
% \subsection{Font support}
%
%    \begin{macro}{\HoLogoFont@font}
%    \begin{tabular}{@{}ll@{}}
%    |#1|:& logo name\\
%    |#2|:& font short name\\
%    |#3|:& text
%    \end{tabular}
%    \begin{macrocode}
\def\HoLogoFont@font#1#2#3{%
  \begingroup
    \ltx@IfUndefined{HoLogoFont@logo@#1.#2}{%
      \ltx@IfUndefined{HoLogoFont@font@#2}{%
        \@PackageWarning{hologo}{%
          Missing font `#2' for logo `#1'%
        }%
        #3%
      }{%
        \csname HoLogoFont@font@#2\endcsname{#3}%
      }%
    }{%
      \csname HoLogoFont@logo@#1.#2\endcsname{#3}%
    }%
  \endgroup
}
%    \end{macrocode}
%    \end{macro}
%
%    \begin{macro}{\HoLogoFont@Def}
%    \begin{macrocode}
\def\HoLogoFont@Def#1{%
  \expandafter\def\csname HoLogoFont@font@#1\endcsname
}
%    \end{macrocode}
%    \end{macro}
%    \begin{macro}{\HoLogoFont@LogoDef}
%    \begin{macrocode}
\def\HoLogoFont@LogoDef#1#2{%
  \expandafter\def\csname HoLogoFont@logo@#1.#2\endcsname
}
%    \end{macrocode}
%    \end{macro}
%
% \subsubsection{Font defaults}
%
%    \begin{macro}{\HoLogoFont@font@general}
%    \begin{macrocode}
\HoLogoFont@Def{general}{}%
%    \end{macrocode}
%    \end{macro}
%
%    \begin{macro}{\HoLogoFont@font@rm}
%    \begin{macrocode}
\ltx@IfUndefined{rmfamily}{%
  \ltx@IfUndefined{rm}{%
  }{%
    \HoLogoFont@Def{rm}{\rm}%
  }%
}{%
  \HoLogoFont@Def{rm}{\rmfamily}%
}
%    \end{macrocode}
%    \end{macro}
%
%    \begin{macro}{\HoLogoFont@font@sf}
%    \begin{macrocode}
\ltx@IfUndefined{sffamily}{%
  \ltx@IfUndefined{sf}{%
  }{%
    \HoLogoFont@Def{sf}{\sf}%
  }%
}{%
  \HoLogoFont@Def{sf}{\sffamily}%
}
%    \end{macrocode}
%    \end{macro}
%
%    \begin{macro}{\HoLogoFont@font@bibsf}
%    In case of \hologo{plainTeX} the original small caps
%    variant is used as default. In \hologo{LaTeX}
%    the definition of package \xpackage{dtklogos} \cite{dtklogos}
%    is used.
%\begin{quote}
%\begin{verbatim}
%\DeclareRobustCommand{\BibTeX}{%
%  B%
%  \kern-.05em%
%  \hbox{%
%    $\m@th$% %% force math size calculations
%    \csname S@\f@size\endcsname
%    \fontsize\sf@size\z@
%    \math@fontsfalse
%    \selectfont
%    I%
%    \kern-.025em%
%    B
%  }%
%  \kern-.08em%
%  \-%
%  \TeX
%}
%\end{verbatim}
%\end{quote}
%    \begin{macrocode}
\ltx@IfUndefined{selectfont}{%
  \ltx@IfUndefined{tensc}{%
    \font\tensc=cmcsc10\relax
  }{}%
  \HoLogoFont@Def{bibsf}{\tensc}%
}{%
  \HoLogoFont@Def{bibsf}{%
    $\mathsurround=0pt$%
    \csname S@\f@size\endcsname
    \fontsize\sf@size{0pt}%
    \math@fontsfalse
    \selectfont
  }%
}
%    \end{macrocode}
%    \end{macro}
%
%    \begin{macro}{\HoLogoFont@font@sc}
%    \begin{macrocode}
\ltx@IfUndefined{scshape}{%
  \ltx@IfUndefined{tensc}{%
    \font\tensc=cmcsc10\relax
  }{}%
  \HoLogoFont@Def{sc}{\tensc}%
}{%
  \HoLogoFont@Def{sc}{\scshape}%
}
%    \end{macrocode}
%    \end{macro}
%
%    \begin{macro}{\HoLogoFont@font@sy}
%    \begin{macrocode}
\ltx@IfUndefined{usefont}{%
  \ltx@IfUndefined{tensy}{%
  }{%
    \HoLogoFont@Def{sy}{\tensy}%
  }%
}{%
  \HoLogoFont@Def{sy}{%
    \usefont{OMS}{cmsy}{m}{n}%
  }%
}
%    \end{macrocode}
%    \end{macro}
%
%    \begin{macro}{\HoLogoFont@font@logo}
%    \begin{macrocode}
\begingroup
  \def\x{LaTeX2e}%
\expandafter\endgroup
\ifx\fmtname\x
  \ltx@IfUndefined{logofamily}{%
    \DeclareRobustCommand\logofamily{%
      \not@math@alphabet\logofamily\relax
      \fontencoding{U}%
      \fontfamily{logo}%
      \selectfont
    }%
  }{}%
  \ltx@IfUndefined{logofamily}{%
  }{%
    \HoLogoFont@Def{logo}{\logofamily}%
  }%
\else
  \ltx@IfUndefined{tenlogo}{%
    \font\tenlogo=logo10\relax
  }{}%
  \HoLogoFont@Def{logo}{\tenlogo}%
\fi
%    \end{macrocode}
%    \end{macro}
%
% \subsubsection{Font setup}
%
%    \begin{macro}{\hologoFontSetup}
%    \begin{macrocode}
\def\hologoFontSetup{%
  \let\HOLOGO@name\relax
  \HOLOGO@FontSetup
}
%    \end{macrocode}
%    \end{macro}
%
%    \begin{macro}{\hologoLogoFontSetup}
%    \begin{macrocode}
\def\hologoLogoFontSetup#1{%
  \edef\HOLOGO@name{#1}%
  \ltx@IfUndefined{HoLogo@\HOLOGO@name}{%
    \@PackageError{hologo}{%
      Unknown logo `\HOLOGO@name'%
    }\@ehc
    \ltx@gobble
  }{%
    \HOLOGO@FontSetup
  }%
}
%    \end{macrocode}
%    \end{macro}
%
%    \begin{macro}{\HOLOGO@FontSetup}
%    \begin{macrocode}
\def\HOLOGO@FontSetup{%
  \kvsetkeys{HoLogoFont}%
}
%    \end{macrocode}
%    \end{macro}
%
%    \begin{macrocode}
\def\HOLOGO@temp#1{%
  \kv@define@key{HoLogoFont}{#1}{%
    \ifx\HOLOGO@name\relax
      \HoLogoFont@Def{#1}{##1}%
    \else
      \HoLogoFont@LogoDef\HOLOGO@name{#1}{##1}%
    \fi
  }%
}
\HOLOGO@temp{general}
\HOLOGO@temp{sf}
%    \end{macrocode}
%
% \subsection{Generic logo commands}
%
%    \begin{macrocode}
\HOLOGO@IfExists\hologo{%
  \@PackageError{hologo}{%
    \string\hologo\ltx@space is already defined.\MessageBreak
    Package loading is aborted%
  }\@ehc
  \HOLOGO@AtEnd
}%
\HOLOGO@IfExists\hologoRobust{%
  \@PackageError{hologo}{%
    \string\hologoRobust\ltx@space is already defined.\MessageBreak
    Package loading is aborted%
  }\@ehc
  \HOLOGO@AtEnd
}%
%    \end{macrocode}
%
% \subsubsection{\cs{hologo} and friends}
%
%    \begin{macrocode}
\ifluatex
  \expandafter\ltx@firstofone
\else
  \expandafter\ltx@gobble
\fi
{%
  \ltx@IfUndefined{ifincsname}{%
    \ifnum\luatexversion<36 %
      \expandafter\ltx@gobble
    \else
      \expandafter\ltx@firstofone
    \fi
    {%
      \begingroup
        \ifcase0%
            \directlua{%
              if tex.enableprimitives then %
                tex.enableprimitives('HOLOGO@', {'ifincsname'})%
              else %
                tex.print('1')%
              end%
            }%
            \ifx\HOLOGO@ifincsname\@undefined 1\fi%
            \relax
          \expandafter\ltx@firstofone
        \else
          \endgroup
          \expandafter\ltx@gobble
        \fi
        {%
          \global\let\ifincsname\HOLOGO@ifincsname
        }%
      \HOLOGO@temp
    }%
  }{}%
}
%    \end{macrocode}
%    \begin{macrocode}
\ltx@IfUndefined{ifincsname}{%
  \catcode`$=14 %
}{%
  \catcode`$=9 %
}
%    \end{macrocode}
%
%    \begin{macro}{\hologo}
%    \begin{macrocode}
\def\hologo#1{%
$ \ifincsname
$   \ltx@ifundefined{HoLogoCs@\HOLOGO@Variant{#1}}{%
$     #1%
$   }{%
$     \csname HoLogoCs@\HOLOGO@Variant{#1}\endcsname\ltx@firstoftwo
$   }%
$ \else
    \HOLOGO@IfExists\texorpdfstring\texorpdfstring\ltx@firstoftwo
    {%
      \hologoRobust{#1}%
    }{%
      \ltx@ifundefined{HoLogoBkm@\HOLOGO@Variant{#1}}{%
        \ltx@ifundefined{HoLogo@#1}{?#1?}{#1}%
      }{%
        \csname HoLogoBkm@\HOLOGO@Variant{#1}\endcsname
        \ltx@firstoftwo
      }%
    }%
$ \fi
}
%    \end{macrocode}
%    \end{macro}
%    \begin{macro}{\Hologo}
%    \begin{macrocode}
\def\Hologo#1{%
$ \ifincsname
$   \ltx@ifundefined{HoLogoCs@\HOLOGO@Variant{#1}}{%
$     #1%
$   }{%
$     \csname HoLogoCs@\HOLOGO@Variant{#1}\endcsname\ltx@secondoftwo
$   }%
$ \else
    \HOLOGO@IfExists\texorpdfstring\texorpdfstring\ltx@firstoftwo
    {%
      \HologoRobust{#1}%
    }{%
      \ltx@ifundefined{HoLogoBkm@\HOLOGO@Variant{#1}}{%
        \ltx@ifundefined{HoLogo@#1}{?#1?}{#1}%
      }{%
        \csname HoLogoBkm@\HOLOGO@Variant{#1}\endcsname
        \ltx@secondoftwo
      }%
    }%
$ \fi
}
%    \end{macrocode}
%    \end{macro}
%
%    \begin{macro}{\hologoVariant}
%    \begin{macrocode}
\def\hologoVariant#1#2{%
  \ifx\relax#2\relax
    \hologo{#1}%
  \else
$   \ifincsname
$     \ltx@ifundefined{HoLogoCs@#1@#2}{%
$       #1%
$     }{%
$       \csname HoLogoCs@#1@#2\endcsname\ltx@firstoftwo
$     }%
$   \else
      \HOLOGO@IfExists\texorpdfstring\texorpdfstring\ltx@firstoftwo
      {%
        \hologoVariantRobust{#1}{#2}%
      }{%
        \ltx@ifundefined{HoLogoBkm@#1@#2}{%
          \ltx@ifundefined{HoLogo@#1}{?#1?}{#1}%
        }{%
          \csname HoLogoBkm@#1@#2\endcsname
          \ltx@firstoftwo
        }%
      }%
$   \fi
  \fi
}
%    \end{macrocode}
%    \end{macro}
%    \begin{macro}{\HologoVariant}
%    \begin{macrocode}
\def\HologoVariant#1#2{%
  \ifx\relax#2\relax
    \Hologo{#1}%
  \else
$   \ifincsname
$     \ltx@ifundefined{HoLogoCs@#1@#2}{%
$       #1%
$     }{%
$       \csname HoLogoCs@#1@#2\endcsname\ltx@secondoftwo
$     }%
$   \else
      \HOLOGO@IfExists\texorpdfstring\texorpdfstring\ltx@firstoftwo
      {%
        \HologoVariantRobust{#1}{#2}%
      }{%
        \ltx@ifundefined{HoLogoBkm@#1@#2}{%
          \ltx@ifundefined{HoLogo@#1}{?#1?}{#1}%
        }{%
          \csname HoLogoBkm@#1@#2\endcsname
          \ltx@secondoftwo
        }%
      }%
$   \fi
  \fi
}
%    \end{macrocode}
%    \end{macro}
%
%    \begin{macrocode}
\catcode`\$=3 %
%    \end{macrocode}
%
% \subsubsection{\cs{hologoRobust} and friends}
%
%    \begin{macro}{\hologoRobust}
%    \begin{macrocode}
\ltx@IfUndefined{protected}{%
  \ltx@IfUndefined{DeclareRobustCommand}{%
    \def\hologoRobust#1%
  }{%
    \DeclareRobustCommand*\hologoRobust[1]%
  }%
}{%
  \protected\def\hologoRobust#1%
}%
{%
  \edef\HOLOGO@name{#1}%
  \ltx@IfUndefined{HoLogo@\HOLOGO@Variant\HOLOGO@name}{%
    \@PackageError{hologo}{%
      Unknown logo `\HOLOGO@name'%
    }\@ehc
    ?\HOLOGO@name?%
  }{%
    \ltx@IfUndefined{ver@tex4ht.sty}{%
      \HoLogoFont@font\HOLOGO@name{general}{%
        \csname HoLogo@\HOLOGO@Variant\HOLOGO@name\endcsname
        \ltx@firstoftwo
      }%
    }{%
      \ltx@IfUndefined{HoLogoHtml@\HOLOGO@Variant\HOLOGO@name}{%
        \HOLOGO@name
      }{%
        \csname HoLogoHtml@\HOLOGO@Variant\HOLOGO@name\endcsname
        \ltx@firstoftwo
      }%
    }%
  }%
}
%    \end{macrocode}
%    \end{macro}
%    \begin{macro}{\HologoRobust}
%    \begin{macrocode}
\ltx@IfUndefined{protected}{%
  \ltx@IfUndefined{DeclareRobustCommand}{%
    \def\HologoRobust#1%
  }{%
    \DeclareRobustCommand*\HologoRobust[1]%
  }%
}{%
  \protected\def\HologoRobust#1%
}%
{%
  \edef\HOLOGO@name{#1}%
  \ltx@IfUndefined{HoLogo@\HOLOGO@Variant\HOLOGO@name}{%
    \@PackageError{hologo}{%
      Unknown logo `\HOLOGO@name'%
    }\@ehc
    ?\HOLOGO@name?%
  }{%
    \ltx@IfUndefined{ver@tex4ht.sty}{%
      \HoLogoFont@font\HOLOGO@name{general}{%
        \csname HoLogo@\HOLOGO@Variant\HOLOGO@name\endcsname
        \ltx@secondoftwo
      }%
    }{%
      \ltx@IfUndefined{HoLogoHtml@\HOLOGO@Variant\HOLOGO@name}{%
        \expandafter\HOLOGO@Uppercase\HOLOGO@name
      }{%
        \csname HoLogoHtml@\HOLOGO@Variant\HOLOGO@name\endcsname
        \ltx@secondoftwo
      }%
    }%
  }%
}
%    \end{macrocode}
%    \end{macro}
%    \begin{macro}{\hologoVariantRobust}
%    \begin{macrocode}
\ltx@IfUndefined{protected}{%
  \ltx@IfUndefined{DeclareRobustCommand}{%
    \def\hologoVariantRobust#1#2%
  }{%
    \DeclareRobustCommand*\hologoVariantRobust[2]%
  }%
}{%
  \protected\def\hologoVariantRobust#1#2%
}%
{%
  \begingroup
    \hologoLogoSetup{#1}{variant={#2}}%
    \hologoRobust{#1}%
  \endgroup
}
%    \end{macrocode}
%    \end{macro}
%    \begin{macro}{\HologoVariantRobust}
%    \begin{macrocode}
\ltx@IfUndefined{protected}{%
  \ltx@IfUndefined{DeclareRobustCommand}{%
    \def\HologoVariantRobust#1#2%
  }{%
    \DeclareRobustCommand*\HologoVariantRobust[2]%
  }%
}{%
  \protected\def\HologoVariantRobust#1#2%
}%
{%
  \begingroup
    \hologoLogoSetup{#1}{variant={#2}}%
    \HologoRobust{#1}%
  \endgroup
}
%    \end{macrocode}
%    \end{macro}
%
%    \begin{macro}{\hologorobust}
%    Macro \cs{hologorobust} is only defined for compatibility.
%    Its use is deprecated.
%    \begin{macrocode}
\def\hologorobust{\hologoRobust}
%    \end{macrocode}
%    \end{macro}
%
% \subsection{Helpers}
%
%    \begin{macro}{\HOLOGO@Uppercase}
%    Macro \cs{HOLOGO@Uppercase} is restricted to \cs{uppercase},
%    because \hologo{plainTeX} or \hologo{iniTeX} do not provide
%    \cs{MakeUppercase}.
%    \begin{macrocode}
\def\HOLOGO@Uppercase#1{\uppercase{#1}}
%    \end{macrocode}
%    \end{macro}
%
%    \begin{macro}{\HOLOGO@PdfdocUnicode}
%    \begin{macrocode}
\def\HOLOGO@PdfdocUnicode{%
  \ifx\ifHy@unicode\iftrue
    \expandafter\ltx@secondoftwo
  \else
    \expandafter\ltx@firstoftwo
  \fi
}
%    \end{macrocode}
%    \end{macro}
%
%    \begin{macro}{\HOLOGO@Math}
%    \begin{macrocode}
\def\HOLOGO@MathSetup{%
  \mathsurround0pt\relax
  \HOLOGO@IfExists\f@series{%
    \if b\expandafter\ltx@car\f@series x\@nil
      \csname boldmath\endcsname
   \fi
  }{}%
}
%    \end{macrocode}
%    \end{macro}
%
%    \begin{macro}{\HOLOGO@TempDimen}
%    \begin{macrocode}
\dimendef\HOLOGO@TempDimen=\ltx@zero
%    \end{macrocode}
%    \end{macro}
%    \begin{macro}{\HOLOGO@NegativeKerning}
%    \begin{macrocode}
\def\HOLOGO@NegativeKerning#1{%
  \begingroup
    \HOLOGO@TempDimen=0pt\relax
    \comma@parse@normalized{#1}{%
      \ifdim\HOLOGO@TempDimen=0pt %
        \expandafter\HOLOGO@@NegativeKerning\comma@entry
      \fi
      \ltx@gobble
    }%
    \ifdim\HOLOGO@TempDimen<0pt %
      \kern\HOLOGO@TempDimen
    \fi
  \endgroup
}
%    \end{macrocode}
%    \end{macro}
%    \begin{macro}{\HOLOGO@@NegativeKerning}
%    \begin{macrocode}
\def\HOLOGO@@NegativeKerning#1#2{%
  \setbox\ltx@zero\hbox{#1#2}%
  \HOLOGO@TempDimen=\wd\ltx@zero
  \setbox\ltx@zero\hbox{#1\kern0pt#2}%
  \advance\HOLOGO@TempDimen by -\wd\ltx@zero
}
%    \end{macrocode}
%    \end{macro}
%
%    \begin{macro}{\HOLOGO@SpaceFactor}
%    \begin{macrocode}
\def\HOLOGO@SpaceFactor{%
  \spacefactor1000 %
}
%    \end{macrocode}
%    \end{macro}
%
%    \begin{macro}{\HOLOGO@Span}
%    \begin{macrocode}
\def\HOLOGO@Span#1#2{%
  \HCode{<span class="HoLogo-#1">}%
  #2%
  \HCode{</span>}%
}
%    \end{macrocode}
%    \end{macro}
%
% \subsubsection{Text subscript}
%
%    \begin{macro}{\HOLOGO@SubScript}%
%    \begin{macrocode}
\def\HOLOGO@SubScript#1{%
  \ltx@IfUndefined{textsubscript}{%
    \ltx@IfUndefined{text}{%
      \ltx@mbox{%
        \mathsurround=0pt\relax
        $%
          _{%
            \ltx@IfUndefined{sf@size}{%
              \mathrm{#1}%
            }{%
              \mbox{%
                \fontsize\sf@size{0pt}\selectfont
                #1%
              }%
            }%
          }%
        $%
      }%
    }{%
      \ltx@mbox{%
        \mathsurround=0pt\relax
        $_{\text{#1}}$%
      }%
    }%
  }{%
    \textsubscript{#1}%
  }%
}
%    \end{macrocode}
%    \end{macro}
%
% \subsection{\hologo{TeX} and friends}
%
% \subsubsection{\hologo{TeX}}
%
%    \begin{macro}{\HoLogo@TeX}
%    Source: \hologo{LaTeX} kernel.
%    \begin{macrocode}
\def\HoLogo@TeX#1{%
  T\kern-.1667em\lower.5ex\hbox{E}\kern-.125emX\HOLOGO@SpaceFactor
}
%    \end{macrocode}
%    \end{macro}
%    \begin{macro}{\HoLogoHtml@TeX}
%    \begin{macrocode}
\def\HoLogoHtml@TeX#1{%
  \HoLogoCss@TeX
  \HOLOGO@Span{TeX}{%
    T%
    \HOLOGO@Span{e}{%
      E%
    }%
    X%
  }%
}
%    \end{macrocode}
%    \end{macro}
%    \begin{macro}{\HoLogoCss@TeX}
%    \begin{macrocode}
\def\HoLogoCss@TeX{%
  \Css{%
    span.HoLogo-TeX span.HoLogo-e{%
      position:relative;%
      top:.5ex;%
      margin-left:-.1667em;%
      margin-right:-.125em;%
    }%
  }%
  \Css{%
    a span.HoLogo-TeX span.HoLogo-e{%
      text-decoration:none;%
    }%
  }%
  \global\let\HoLogoCss@TeX\relax
}
%    \end{macrocode}
%    \end{macro}
%
% \subsubsection{\hologo{plainTeX}}
%
%    \begin{macro}{\HoLogo@plainTeX@space}
%    Source: ``The \hologo{TeX}book''
%    \begin{macrocode}
\def\HoLogo@plainTeX@space#1{%
  \HOLOGO@mbox{#1{p}{P}lain}\HOLOGO@space\hologo{TeX}%
}
%    \end{macrocode}
%    \end{macro}
%    \begin{macro}{\HoLogoCs@plainTeX@space}
%    \begin{macrocode}
\def\HoLogoCs@plainTeX@space#1{#1{p}{P}lain TeX}%
%    \end{macrocode}
%    \end{macro}
%    \begin{macro}{\HoLogoBkm@plainTeX@space}
%    \begin{macrocode}
\def\HoLogoBkm@plainTeX@space#1{%
  #1{p}{P}lain \hologo{TeX}%
}
%    \end{macrocode}
%    \end{macro}
%    \begin{macro}{\HoLogoHtml@plainTeX@space}
%    \begin{macrocode}
\def\HoLogoHtml@plainTeX@space#1{%
  #1{p}{P}lain \hologo{TeX}%
}
%    \end{macrocode}
%    \end{macro}
%
%    \begin{macro}{\HoLogo@plainTeX@hyphen}
%    \begin{macrocode}
\def\HoLogo@plainTeX@hyphen#1{%
  \HOLOGO@mbox{#1{p}{P}lain}\HOLOGO@hyphen\hologo{TeX}%
}
%    \end{macrocode}
%    \end{macro}
%    \begin{macro}{\HoLogoCs@plainTeX@hyphen}
%    \begin{macrocode}
\def\HoLogoCs@plainTeX@hyphen#1{#1{p}{P}lain-TeX}
%    \end{macrocode}
%    \end{macro}
%    \begin{macro}{\HoLogoBkm@plainTeX@hyphen}
%    \begin{macrocode}
\def\HoLogoBkm@plainTeX@hyphen#1{%
  #1{p}{P}lain-\hologo{TeX}%
}
%    \end{macrocode}
%    \end{macro}
%    \begin{macro}{\HoLogoHtml@plainTeX@hyphen}
%    \begin{macrocode}
\def\HoLogoHtml@plainTeX@hyphen#1{%
  #1{p}{P}lain-\hologo{TeX}%
}
%    \end{macrocode}
%    \end{macro}
%
%    \begin{macro}{\HoLogo@plainTeX@runtogether}
%    \begin{macrocode}
\def\HoLogo@plainTeX@runtogether#1{%
  \HOLOGO@mbox{#1{p}{P}lain\hologo{TeX}}%
}
%    \end{macrocode}
%    \end{macro}
%    \begin{macro}{\HoLogoCs@plainTeX@runtogether}
%    \begin{macrocode}
\def\HoLogoCs@plainTeX@runtogether#1{#1{p}{P}lainTeX}
%    \end{macrocode}
%    \end{macro}
%    \begin{macro}{\HoLogoBkm@plainTeX@runtogether}
%    \begin{macrocode}
\def\HoLogoBkm@plainTeX@runtogether#1{%
  #1{p}{P}lain\hologo{TeX}%
}
%    \end{macrocode}
%    \end{macro}
%    \begin{macro}{\HoLogoHtml@plainTeX@runtogether}
%    \begin{macrocode}
\def\HoLogoHtml@plainTeX@runtogether#1{%
  #1{p}{P}lain\hologo{TeX}%
}
%    \end{macrocode}
%    \end{macro}
%
%    \begin{macro}{\HoLogo@plainTeX}
%    \begin{macrocode}
\def\HoLogo@plainTeX{\HoLogo@plainTeX@space}
%    \end{macrocode}
%    \end{macro}
%    \begin{macro}{\HoLogoCs@plainTeX}
%    \begin{macrocode}
\def\HoLogoCs@plainTeX{\HoLogoCs@plainTeX@space}
%    \end{macrocode}
%    \end{macro}
%    \begin{macro}{\HoLogoBkm@plainTeX}
%    \begin{macrocode}
\def\HoLogoBkm@plainTeX{\HoLogoBkm@plainTeX@space}
%    \end{macrocode}
%    \end{macro}
%    \begin{macro}{\HoLogoHtml@plainTeX}
%    \begin{macrocode}
\def\HoLogoHtml@plainTeX{\HoLogoHtml@plainTeX@space}
%    \end{macrocode}
%    \end{macro}
%
% \subsubsection{\hologo{LaTeX}}
%
%    Source: \hologo{LaTeX} kernel.
%\begin{quote}
%\begin{verbatim}
%\DeclareRobustCommand{\LaTeX}{%
%  L%
%  \kern-.36em%
%  {%
%    \sbox\z@ T%
%    \vbox to\ht\z@{%
%      \hbox{%
%        \check@mathfonts
%        \fontsize\sf@size\z@
%        \math@fontsfalse
%        \selectfont
%        A%
%      }%
%      \vss
%    }%
%  }%
%  \kern-.15em%
%  \TeX
%}
%\end{verbatim}
%\end{quote}
%
%    \begin{macro}{\HoLogo@La}
%    \begin{macrocode}
\def\HoLogo@La#1{%
  L%
  \kern-.36em%
  \begingroup
    \setbox\ltx@zero\hbox{T}%
    \vbox to\ht\ltx@zero{%
      \hbox{%
        \ltx@ifundefined{check@mathfonts}{%
          \csname sevenrm\endcsname
        }{%
          \check@mathfonts
          \fontsize\sf@size{0pt}%
          \math@fontsfalse\selectfont
        }%
        A%
      }%
      \vss
    }%
  \endgroup
}
%    \end{macrocode}
%    \end{macro}
%
%    \begin{macro}{\HoLogo@LaTeX}
%    Source: \hologo{LaTeX} kernel.
%    \begin{macrocode}
\def\HoLogo@LaTeX#1{%
  \hologo{La}%
  \kern-.15em%
  \hologo{TeX}%
}
%    \end{macrocode}
%    \end{macro}
%    \begin{macro}{\HoLogoHtml@LaTeX}
%    \begin{macrocode}
\def\HoLogoHtml@LaTeX#1{%
  \HoLogoCss@LaTeX
  \HOLOGO@Span{LaTeX}{%
    L%
    \HOLOGO@Span{a}{%
      A%
    }%
    \hologo{TeX}%
  }%
}
%    \end{macrocode}
%    \end{macro}
%    \begin{macro}{\HoLogoCss@LaTeX}
%    \begin{macrocode}
\def\HoLogoCss@LaTeX{%
  \Css{%
    span.HoLogo-LaTeX span.HoLogo-a{%
      position:relative;%
      top:-.5ex;%
      margin-left:-.36em;%
      margin-right:-.15em;%
      font-size:85\%;%
    }%
  }%
  \global\let\HoLogoCss@LaTeX\relax
}
%    \end{macrocode}
%    \end{macro}
%
% \subsubsection{\hologo{(La)TeX}}
%
%    \begin{macro}{\HoLogo@LaTeXTeX}
%    The kerning around the parentheses is taken
%    from package \xpackage{dtklogos} \cite{dtklogos}.
%\begin{quote}
%\begin{verbatim}
%\DeclareRobustCommand{\LaTeXTeX}{%
%  (%
%  \kern-.15em%
%  L%
%  \kern-.36em%
%  {%
%    \sbox\z@ T%
%    \vbox to\ht0{%
%      \hbox{%
%        $\m@th$%
%        \csname S@\f@size\endcsname
%        \fontsize\sf@size\z@
%        \math@fontsfalse
%        \selectfont
%        A%
%      }%
%      \vss
%    }%
%  }%
%  \kern-.2em%
%  )%
%  \kern-.15em%
%  \TeX
%}
%\end{verbatim}
%\end{quote}
%    \begin{macrocode}
\def\HoLogo@LaTeXTeX#1{%
  (%
  \kern-.15em%
  \hologo{La}%
  \kern-.2em%
  )%
  \kern-.15em%
  \hologo{TeX}%
}
%    \end{macrocode}
%    \end{macro}
%    \begin{macro}{\HoLogoBkm@LaTeXTeX}
%    \begin{macrocode}
\def\HoLogoBkm@LaTeXTeX#1{(La)TeX}
%    \end{macrocode}
%    \end{macro}
%
%    \begin{macro}{\HoLogo@(La)TeX}
%    \begin{macrocode}
\expandafter
\let\csname HoLogo@(La)TeX\endcsname\HoLogo@LaTeXTeX
%    \end{macrocode}
%    \end{macro}
%    \begin{macro}{\HoLogoBkm@(La)TeX}
%    \begin{macrocode}
\expandafter
\let\csname HoLogoBkm@(La)TeX\endcsname\HoLogoBkm@LaTeXTeX
%    \end{macrocode}
%    \end{macro}
%    \begin{macro}{\HoLogoHtml@LaTeXTeX}
%    \begin{macrocode}
\def\HoLogoHtml@LaTeXTeX#1{%
  \HoLogoCss@LaTeXTeX
  \HOLOGO@Span{LaTeXTeX}{%
    (%
    \HOLOGO@Span{L}{L}%
    \HOLOGO@Span{a}{A}%
    \HOLOGO@Span{ParenRight}{)}%
    \hologo{TeX}%
  }%
}
%    \end{macrocode}
%    \end{macro}
%    \begin{macro}{\HoLogoHtml@(La)TeX}
%    Kerning after opening parentheses and before closing parentheses
%    is $-0.1$\,em. The original values $-0.15$\,em
%    looked too ugly for a serif font.
%    \begin{macrocode}
\expandafter
\let\csname HoLogoHtml@(La)TeX\endcsname\HoLogoHtml@LaTeXTeX
%    \end{macrocode}
%    \end{macro}
%    \begin{macro}{\HoLogoCss@LaTeXTeX}
%    \begin{macrocode}
\def\HoLogoCss@LaTeXTeX{%
  \Css{%
    span.HoLogo-LaTeXTeX span.HoLogo-L{%
      margin-left:-.1em;%
    }%
  }%
  \Css{%
    span.HoLogo-LaTeXTeX span.HoLogo-a{%
      position:relative;%
      top:-.5ex;%
      margin-left:-.36em;%
      margin-right:-.1em;%
      font-size:85\%;%
    }%
  }%
  \Css{%
    span.HoLogo-LaTeXTeX span.HoLogo-ParenRight{%
      margin-right:-.15em;%
    }%
  }%
  \global\let\HoLogoCss@LaTeXTeX\relax
}
%    \end{macrocode}
%    \end{macro}
%
% \subsubsection{\hologo{LaTeXe}}
%
%    \begin{macro}{\HoLogo@LaTeXe}
%    Source: \hologo{LaTeX} kernel
%    \begin{macrocode}
\def\HoLogo@LaTeXe#1{%
  \hologo{LaTeX}%
  \kern.15em%
  \hbox{%
    \HOLOGO@MathSetup
    2%
    $_{\textstyle\varepsilon}$%
  }%
}
%    \end{macrocode}
%    \end{macro}
%
%    \begin{macro}{\HoLogoCs@LaTeXe}
%    \begin{macrocode}
\ifnum64=`\^^^^0040\relax % test for big chars of LuaTeX/XeTeX
  \catcode`\$=9 %
  \catcode`\&=14 %
\else
  \catcode`\$=14 %
  \catcode`\&=9 %
\fi
\def\HoLogoCs@LaTeXe#1{%
  LaTeX2%
$ \string ^^^^0395%
& e%
}%
\catcode`\$=3 %
\catcode`\&=4 %
%    \end{macrocode}
%    \end{macro}
%
%    \begin{macro}{\HoLogoBkm@LaTeXe}
%    \begin{macrocode}
\def\HoLogoBkm@LaTeXe#1{%
  \hologo{LaTeX}%
  2%
  \HOLOGO@PdfdocUnicode{e}{\textepsilon}%
}
%    \end{macrocode}
%    \end{macro}
%
%    \begin{macro}{\HoLogoHtml@LaTeXe}
%    \begin{macrocode}
\def\HoLogoHtml@LaTeXe#1{%
  \HoLogoCss@LaTeXe
  \HOLOGO@Span{LaTeX2e}{%
    \hologo{LaTeX}%
    \HOLOGO@Span{2}{2}%
    \HOLOGO@Span{e}{%
      \HOLOGO@MathSetup
      \ensuremath{\textstyle\varepsilon}%
    }%
  }%
}
%    \end{macrocode}
%    \end{macro}
%    \begin{macro}{\HoLogoCss@LaTeXe}
%    \begin{macrocode}
\def\HoLogoCss@LaTeXe{%
  \Css{%
    span.HoLogo-LaTeX2e span.HoLogo-2{%
      padding-left:.15em;%
    }%
  }%
  \Css{%
    span.HoLogo-LaTeX2e span.HoLogo-e{%
      position:relative;%
      top:.35ex;%
      text-decoration:none;%
    }%
  }%
  \global\let\HoLogoCss@LaTeXe\relax
}
%    \end{macrocode}
%    \end{macro}
%
%    \begin{macro}{\HoLogo@LaTeX2e}
%    \begin{macrocode}
\expandafter
\let\csname HoLogo@LaTeX2e\endcsname\HoLogo@LaTeXe
%    \end{macrocode}
%    \end{macro}
%    \begin{macro}{\HoLogoCs@LaTeX2e}
%    \begin{macrocode}
\expandafter
\let\csname HoLogoCs@LaTeX2e\endcsname\HoLogoCs@LaTeXe
%    \end{macrocode}
%    \end{macro}
%    \begin{macro}{\HoLogoBkm@LaTeX2e}
%    \begin{macrocode}
\expandafter
\let\csname HoLogoBkm@LaTeX2e\endcsname\HoLogoBkm@LaTeXe
%    \end{macrocode}
%    \end{macro}
%    \begin{macro}{\HoLogoHtml@LaTeX2e}
%    \begin{macrocode}
\expandafter
\let\csname HoLogoHtml@LaTeX2e\endcsname\HoLogoHtml@LaTeXe
%    \end{macrocode}
%    \end{macro}
%
% \subsubsection{\hologo{LaTeX3}}
%
%    \begin{macro}{\HoLogo@LaTeX3}
%    Source: \hologo{LaTeX} kernel
%    \begin{macrocode}
\expandafter\def\csname HoLogo@LaTeX3\endcsname#1{%
  \hologo{LaTeX}%
  3%
}
%    \end{macrocode}
%    \end{macro}
%
%    \begin{macro}{\HoLogoBkm@LaTeX3}
%    \begin{macrocode}
\expandafter\def\csname HoLogoBkm@LaTeX3\endcsname#1{%
  \hologo{LaTeX}%
  3%
}
%    \end{macrocode}
%    \end{macro}
%    \begin{macro}{\HoLogoHtml@LaTeX3}
%    \begin{macrocode}
\expandafter
\let\csname HoLogoHtml@LaTeX3\expandafter\endcsname
\csname HoLogo@LaTeX3\endcsname
%    \end{macrocode}
%    \end{macro}
%
% \subsubsection{\hologo{LaTeXML}}
%
%    \begin{macro}{\HoLogo@LaTeXML}
%    \begin{macrocode}
\def\HoLogo@LaTeXML#1{%
  \HOLOGO@mbox{%
    \hologo{La}%
    \kern-.15em%
    T%
    \kern-.1667em%
    \lower.5ex\hbox{E}%
    \kern-.125em%
    \HoLogoFont@font{LaTeXML}{sc}{xml}%
  }%
}
%    \end{macrocode}
%    \end{macro}
%    \begin{macro}{\HoLogoHtml@pdfLaTeX}
%    \begin{macrocode}
\def\HoLogoHtml@LaTeXML#1{%
  \HOLOGO@Span{LaTeXML}{%
    \HoLogoCss@LaTeX
    \HoLogoCss@TeX
    \HOLOGO@Span{LaTeX}{%
      L%
      \HOLOGO@Span{a}{%
        A%
      }%
    }%
    \HOLOGO@Span{TeX}{%
      T%
      \HOLOGO@Span{e}{%
        E%
      }%
    }%
    \HCode{<span style="font-variant: small-caps;">}%
    xml%
    \HCode{</span>}%
  }%
}
%    \end{macrocode}
%    \end{macro}
%
% \subsubsection{\hologo{eTeX}}
%
%    \begin{macro}{\HoLogo@eTeX}
%    Source: package \xpackage{etex}
%    \begin{macrocode}
\def\HoLogo@eTeX#1{%
  \ltx@mbox{%
    \HOLOGO@MathSetup
    $\varepsilon$%
    -%
    \HOLOGO@NegativeKerning{-T,T-,To}%
    \hologo{TeX}%
  }%
}
%    \end{macrocode}
%    \end{macro}
%    \begin{macro}{\HoLogoCs@eTeX}
%    \begin{macrocode}
\ifnum64=`\^^^^0040\relax % test for big chars of LuaTeX/XeTeX
  \catcode`\$=9 %
  \catcode`\&=14 %
\else
  \catcode`\$=14 %
  \catcode`\&=9 %
\fi
\def\HoLogoCs@eTeX#1{%
$ #1{\string ^^^^0395}{\string ^^^^03b5}%
& #1{e}{E}%
  TeX%
}%
\catcode`\$=3 %
\catcode`\&=4 %
%    \end{macrocode}
%    \end{macro}
%    \begin{macro}{\HoLogoBkm@eTeX}
%    \begin{macrocode}
\def\HoLogoBkm@eTeX#1{%
  \HOLOGO@PdfdocUnicode{#1{e}{E}}{\textepsilon}%
  -%
  \hologo{TeX}%
}
%    \end{macrocode}
%    \end{macro}
%    \begin{macro}{\HoLogoHtml@eTeX}
%    \begin{macrocode}
\def\HoLogoHtml@eTeX#1{%
  \ltx@mbox{%
    \HOLOGO@MathSetup
    $\varepsilon$%
    -%
    \hologo{TeX}%
  }%
}
%    \end{macrocode}
%    \end{macro}
%
% \subsubsection{\hologo{iniTeX}}
%
%    \begin{macro}{\HoLogo@iniTeX}
%    \begin{macrocode}
\def\HoLogo@iniTeX#1{%
  \HOLOGO@mbox{%
    #1{i}{I}ni\hologo{TeX}%
  }%
}
%    \end{macrocode}
%    \end{macro}
%    \begin{macro}{\HoLogoCs@iniTeX}
%    \begin{macrocode}
\def\HoLogoCs@iniTeX#1{#1{i}{I}niTeX}
%    \end{macrocode}
%    \end{macro}
%    \begin{macro}{\HoLogoBkm@iniTeX}
%    \begin{macrocode}
\def\HoLogoBkm@iniTeX#1{%
  #1{i}{I}ni\hologo{TeX}%
}
%    \end{macrocode}
%    \end{macro}
%    \begin{macro}{\HoLogoHtml@iniTeX}
%    \begin{macrocode}
\let\HoLogoHtml@iniTeX\HoLogo@iniTeX
%    \end{macrocode}
%    \end{macro}
%
% \subsubsection{\hologo{virTeX}}
%
%    \begin{macro}{\HoLogo@virTeX}
%    \begin{macrocode}
\def\HoLogo@virTeX#1{%
  \HOLOGO@mbox{%
    #1{v}{V}ir\hologo{TeX}%
  }%
}
%    \end{macrocode}
%    \end{macro}
%    \begin{macro}{\HoLogoCs@virTeX}
%    \begin{macrocode}
\def\HoLogoCs@virTeX#1{#1{v}{V}irTeX}
%    \end{macrocode}
%    \end{macro}
%    \begin{macro}{\HoLogoBkm@virTeX}
%    \begin{macrocode}
\def\HoLogoBkm@virTeX#1{%
  #1{v}{V}ir\hologo{TeX}%
}
%    \end{macrocode}
%    \end{macro}
%    \begin{macro}{\HoLogoHtml@virTeX}
%    \begin{macrocode}
\let\HoLogoHtml@virTeX\HoLogo@virTeX
%    \end{macrocode}
%    \end{macro}
%
% \subsubsection{\hologo{SliTeX}}
%
% \paragraph{Definitions of the three variants.}
%
%    \begin{macro}{\HoLogo@SLiTeX@lift}
%    \begin{macrocode}
\def\HoLogo@SLiTeX@lift#1{%
  \HoLogoFont@font{SliTeX}{rm}{%
    S%
    \kern-.06em%
    L%
    \kern-.18em%
    \raise.32ex\hbox{\HoLogoFont@font{SliTeX}{sc}{i}}%
    \HOLOGO@discretionary
    \kern-.06em%
    \hologo{TeX}%
  }%
}
%    \end{macrocode}
%    \end{macro}
%    \begin{macro}{\HoLogoBkm@SLiTeX@lift}
%    \begin{macrocode}
\def\HoLogoBkm@SLiTeX@lift#1{SLiTeX}
%    \end{macrocode}
%    \end{macro}
%    \begin{macro}{\HoLogoHtml@SLiTeX@lift}
%    \begin{macrocode}
\def\HoLogoHtml@SLiTeX@lift#1{%
  \HoLogoCss@SLiTeX@lift
  \HOLOGO@Span{SLiTeX-lift}{%
    \HoLogoFont@font{SliTeX}{rm}{%
      S%
      \HOLOGO@Span{L}{L}%
      \HOLOGO@Span{i}{i}%
      \hologo{TeX}%
    }%
  }%
}
%    \end{macrocode}
%    \end{macro}
%    \begin{macro}{\HoLogoCss@SLiTeX@lift}
%    \begin{macrocode}
\def\HoLogoCss@SLiTeX@lift{%
  \Css{%
    span.HoLogo-SLiTeX-lift span.HoLogo-L{%
      margin-left:-.06em;%
      margin-right:-.18em;%
    }%
  }%
  \Css{%
    span.HoLogo-SLiTeX-lift span.HoLogo-i{%
      position:relative;%
      top:-.32ex;%
      margin-right:-.06em;%
      font-variant:small-caps;%
    }%
  }%
  \global\let\HoLogoCss@SLiTeX@lift\relax
}
%    \end{macrocode}
%    \end{macro}
%
%    \begin{macro}{\HoLogo@SliTeX@simple}
%    \begin{macrocode}
\def\HoLogo@SliTeX@simple#1{%
  \HoLogoFont@font{SliTeX}{rm}{%
    \ltx@mbox{%
      \HoLogoFont@font{SliTeX}{sc}{Sli}%
    }%
    \HOLOGO@discretionary
    \hologo{TeX}%
  }%
}
%    \end{macrocode}
%    \end{macro}
%    \begin{macro}{\HoLogoBkm@SliTeX@simple}
%    \begin{macrocode}
\def\HoLogoBkm@SliTeX@simple#1{SliTeX}
%    \end{macrocode}
%    \end{macro}
%    \begin{macro}{\HoLogoHtml@SliTeX@simple}
%    \begin{macrocode}
\let\HoLogoHtml@SliTeX@simple\HoLogo@SliTeX@simple
%    \end{macrocode}
%    \end{macro}
%
%    \begin{macro}{\HoLogo@SliTeX@narrow}
%    \begin{macrocode}
\def\HoLogo@SliTeX@narrow#1{%
  \HoLogoFont@font{SliTeX}{rm}{%
    \ltx@mbox{%
      S%
      \kern-.06em%
      \HoLogoFont@font{SliTeX}{sc}{%
        l%
        \kern-.035em%
        i%
      }%
    }%
    \HOLOGO@discretionary
    \kern-.06em%
    \hologo{TeX}%
  }%
}
%    \end{macrocode}
%    \end{macro}
%    \begin{macro}{\HoLogoBkm@SliTeX@narrow}
%    \begin{macrocode}
\def\HoLogoBkm@SliTeX@narrow#1{SliTeX}
%    \end{macrocode}
%    \end{macro}
%    \begin{macro}{\HoLogoHtml@SliTeX@narrow}
%    \begin{macrocode}
\def\HoLogoHtml@SliTeX@narrow#1{%
  \HoLogoCss@SliTeX@narrow
  \HOLOGO@Span{SliTeX-narrow}{%
    \HoLogoFont@font{SliTeX}{rm}{%
      S%
        \HOLOGO@Span{l}{l}%
        \HOLOGO@Span{i}{i}%
      \hologo{TeX}%
    }%
  }%
}
%    \end{macrocode}
%    \end{macro}
%    \begin{macro}{\HoLogoCss@SliTeX@narrow}
%    \begin{macrocode}
\def\HoLogoCss@SliTeX@narrow{%
  \Css{%
    span.HoLogo-SliTeX-narrow span.HoLogo-l{%
      margin-left:-.06em;%
      margin-right:-.035em;%
      font-variant:small-caps;%
    }%
  }%
  \Css{%
    span.HoLogo-SliTeX-narrow span.HoLogo-i{%
      margin-right:-.06em;%
      font-variant:small-caps;%
    }%
  }%
  \global\let\HoLogoCss@SliTeX@narrow\relax
}
%    \end{macrocode}
%    \end{macro}
%
% \paragraph{Macro set completion.}
%
%    \begin{macro}{\HoLogo@SLiTeX@simple}
%    \begin{macrocode}
\def\HoLogo@SLiTeX@simple{\HoLogo@SliTeX@simple}
%    \end{macrocode}
%    \end{macro}
%    \begin{macro}{\HoLogoBkm@SLiTeX@simple}
%    \begin{macrocode}
\def\HoLogoBkm@SLiTeX@simple{\HoLogoBkm@SliTeX@simple}
%    \end{macrocode}
%    \end{macro}
%    \begin{macro}{\HoLogoHtml@SLiTeX@simple}
%    \begin{macrocode}
\def\HoLogoHtml@SLiTeX@simple{\HoLogoHtml@SliTeX@simple}
%    \end{macrocode}
%    \end{macro}
%
%    \begin{macro}{\HoLogo@SLiTeX@narrow}
%    \begin{macrocode}
\def\HoLogo@SLiTeX@narrow{\HoLogo@SliTeX@narrow}
%    \end{macrocode}
%    \end{macro}
%    \begin{macro}{\HoLogoBkm@SLiTeX@narrow}
%    \begin{macrocode}
\def\HoLogoBkm@SLiTeX@narrow{\HoLogoBkm@SliTeX@narrow}
%    \end{macrocode}
%    \end{macro}
%    \begin{macro}{\HoLogoHtml@SLiTeX@narrow}
%    \begin{macrocode}
\def\HoLogoHtml@SLiTeX@narrow{\HoLogoHtml@SliTeX@narrow}
%    \end{macrocode}
%    \end{macro}
%
%    \begin{macro}{\HoLogo@SliTeX@lift}
%    \begin{macrocode}
\def\HoLogo@SliTeX@lift{\HoLogo@SLiTeX@lift}
%    \end{macrocode}
%    \end{macro}
%    \begin{macro}{\HoLogoBkm@SliTeX@lift}
%    \begin{macrocode}
\def\HoLogoBkm@SliTeX@lift{\HoLogoBkm@SLiTeX@lift}
%    \end{macrocode}
%    \end{macro}
%    \begin{macro}{\HoLogoHtml@SliTeX@lift}
%    \begin{macrocode}
\def\HoLogoHtml@SliTeX@lift{\HoLogoHtml@SLiTeX@lift}
%    \end{macrocode}
%    \end{macro}
%
% \paragraph{Defaults.}
%
%    \begin{macro}{\HoLogo@SLiTeX}
%    \begin{macrocode}
\def\HoLogo@SLiTeX{\HoLogo@SLiTeX@lift}
%    \end{macrocode}
%    \end{macro}
%    \begin{macro}{\HoLogoBkm@SLiTeX}
%    \begin{macrocode}
\def\HoLogoBkm@SLiTeX{\HoLogoBkm@SLiTeX@lift}
%    \end{macrocode}
%    \end{macro}
%    \begin{macro}{\HoLogoHtml@SLiTeX}
%    \begin{macrocode}
\def\HoLogoHtml@SLiTeX{\HoLogoHtml@SLiTeX@lift}
%    \end{macrocode}
%    \end{macro}
%
%    \begin{macro}{\HoLogo@SliTeX}
%    \begin{macrocode}
\def\HoLogo@SliTeX{\HoLogo@SliTeX@narrow}
%    \end{macrocode}
%    \end{macro}
%    \begin{macro}{\HoLogoBkm@SliTeX}
%    \begin{macrocode}
\def\HoLogoBkm@SliTeX{\HoLogoBkm@SliTeX@narrow}
%    \end{macrocode}
%    \end{macro}
%    \begin{macro}{\HoLogoHtml@SliTeX}
%    \begin{macrocode}
\def\HoLogoHtml@SliTeX{\HoLogoHtml@SliTeX@narrow}
%    \end{macrocode}
%    \end{macro}
%
% \subsubsection{\hologo{LuaTeX}}
%
%    \begin{macro}{\HoLogo@LuaTeX}
%    The kerning is an idea of Hans Hagen, see mailing list
%    `luatex at tug dot org' in March 2010.
%    \begin{macrocode}
\def\HoLogo@LuaTeX#1{%
  \HOLOGO@mbox{%
    Lua%
    \HOLOGO@NegativeKerning{aT,oT,To}%
    \hologo{TeX}%
  }%
}
%    \end{macrocode}
%    \end{macro}
%    \begin{macro}{\HoLogoHtml@LuaTeX}
%    \begin{macrocode}
\let\HoLogoHtml@LuaTeX\HoLogo@LuaTeX
%    \end{macrocode}
%    \end{macro}
%
% \subsubsection{\hologo{LuaLaTeX}}
%
%    \begin{macro}{\HoLogo@LuaLaTeX}
%    \begin{macrocode}
\def\HoLogo@LuaLaTeX#1{%
  \HOLOGO@mbox{%
    Lua%
    \hologo{LaTeX}%
  }%
}
%    \end{macrocode}
%    \end{macro}
%    \begin{macro}{\HoLogoHtml@LuaLaTeX}
%    \begin{macrocode}
\let\HoLogoHtml@LuaLaTeX\HoLogo@LuaLaTeX
%    \end{macrocode}
%    \end{macro}
%
% \subsubsection{\hologo{XeTeX}, \hologo{XeLaTeX}}
%
%    \begin{macro}{\HOLOGO@IfCharExists}
%    \begin{macrocode}
\ifluatex
  \ifnum\luatexversion<36 %
  \else
    \def\HOLOGO@IfCharExists#1{%
      \ifnum
        \directlua{%
           if luaotfload and luaotfload.aux then
             if luaotfload.aux.font_has_glyph(%
                    font.current(), \number#1) then % 	 
	       tex.print("1") % 	 
	     end % 	 
	   elseif font and font.fonts and font.current then %
            local f = font.fonts[font.current()]%
            if f.characters and f.characters[\number#1] then %
              tex.print("1")%
            end %
          end%
        }0=\ltx@zero
        \expandafter\ltx@secondoftwo
      \else
        \expandafter\ltx@firstoftwo
      \fi
    }%
  \fi
\fi
\ltx@IfUndefined{HOLOGO@IfCharExists}{%
  \def\HOLOGO@@IfCharExists#1{%
    \begingroup
      \tracinglostchars=\ltx@zero
      \setbox\ltx@zero=\hbox{%
        \kern7sp\char#1\relax
        \ifnum\lastkern>\ltx@zero
          \expandafter\aftergroup\csname iffalse\endcsname
        \else
          \expandafter\aftergroup\csname iftrue\endcsname
        \fi
      }%
      % \if{true|false} from \aftergroup
      \endgroup
      \expandafter\ltx@firstoftwo
    \else
      \endgroup
      \expandafter\ltx@secondoftwo
    \fi
  }%
  \ifxetex
    \ltx@IfUndefined{XeTeXfonttype}{}{%
      \ltx@IfUndefined{XeTeXcharglyph}{}{%
        \def\HOLOGO@IfCharExists#1{%
          \ifnum\XeTeXfonttype\font>\ltx@zero
            \expandafter\ltx@firstofthree
          \else
            \expandafter\ltx@gobble
          \fi
          {%
            \ifnum\XeTeXcharglyph#1>\ltx@zero
              \expandafter\ltx@firstoftwo
            \else
              \expandafter\ltx@secondoftwo
            \fi
          }%
          \HOLOGO@@IfCharExists{#1}%
        }%
      }%
    }%
  \fi
}{}
\ltx@ifundefined{HOLOGO@IfCharExists}{%
  \ifnum64=`\^^^^0040\relax % test for big chars of LuaTeX/XeTeX
    \let\HOLOGO@IfCharExists\HOLOGO@@IfCharExists
  \else
    \def\HOLOGO@IfCharExists#1{%
      \ifnum#1>255 %
        \expandafter\ltx@fourthoffour
      \fi
      \HOLOGO@@IfCharExists{#1}%
    }%
  \fi
}{}
%    \end{macrocode}
%    \end{macro}
%
%    \begin{macro}{\HoLogo@Xe}
%    Source: package \xpackage{dtklogos}
%    \begin{macrocode}
\def\HoLogo@Xe#1{%
  X%
  \kern-.1em\relax
  \HOLOGO@IfCharExists{"018E}{%
    \lower.5ex\hbox{\char"018E}%
  }{%
    \chardef\HOLOGO@choice=\ltx@zero
    \ifdim\fontdimen\ltx@one\font>0pt %
      \ltx@IfUndefined{rotatebox}{%
        \ltx@IfUndefined{pgftext}{%
          \ltx@IfUndefined{psscalebox}{%
            \ltx@IfUndefined{HOLOGO@ScaleBox@\hologoDriver}{%
            }{%
              \chardef\HOLOGO@choice=4 %
            }%
          }{%
            \chardef\HOLOGO@choice=3 %
          }%
        }{%
          \chardef\HOLOGO@choice=2 %
        }%
      }{%
        \chardef\HOLOGO@choice=1 %
      }%
      \ifcase\HOLOGO@choice
        \HOLOGO@WarningUnsupportedDriver{Xe}%
        e%
      \or % 1: \rotatebox
        \begingroup
          \setbox\ltx@zero\hbox{\rotatebox{180}{E}}%
          \ltx@LocDimenA=\dp\ltx@zero
          \advance\ltx@LocDimenA by -.5ex\relax
          \raise\ltx@LocDimenA\box\ltx@zero
        \endgroup
      \or % 2: \pgftext
        \lower.5ex\hbox{%
          \pgfpicture
            \pgftext[rotate=180]{E}%
          \endpgfpicture
        }%
      \or % 3: \psscalebox
        \begingroup
          \setbox\ltx@zero\hbox{\psscalebox{-1 -1}{E}}%
          \ltx@LocDimenA=\dp\ltx@zero
          \advance\ltx@LocDimenA by -.5ex\relax
          \raise\ltx@LocDimenA\box\ltx@zero
        \endgroup
      \or % 4: \HOLOGO@PointReflectBox
        \lower.5ex\hbox{\HOLOGO@PointReflectBox{E}}%
      \else
        \@PackageError{hologo}{Internal error (choice/it}\@ehc
      \fi
    \else
      \ltx@IfUndefined{reflectbox}{%
        \ltx@IfUndefined{pgftext}{%
          \ltx@IfUndefined{psscalebox}{%
            \ltx@IfUndefined{HOLOGO@ScaleBox@\hologoDriver}{%
            }{%
              \chardef\HOLOGO@choice=4 %
            }%
          }{%
            \chardef\HOLOGO@choice=3 %
          }%
        }{%
          \chardef\HOLOGO@choice=2 %
        }%
      }{%
        \chardef\HOLOGO@choice=1 %
      }%
      \ifcase\HOLOGO@choice
        \HOLOGO@WarningUnsupportedDriver{Xe}%
        e%
      \or % 1: reflectbox
        \lower.5ex\hbox{%
          \reflectbox{E}%
        }%
      \or % 2: \pgftext
        \lower.5ex\hbox{%
          \pgfpicture
            \pgftransformxscale{-1}%
            \pgftext{E}%
          \endpgfpicture
        }%
      \or % 3: \psscalebox
        \lower.5ex\hbox{%
          \psscalebox{-1 1}{E}%
        }%
      \or % 4: \HOLOGO@Reflectbox
        \lower.5ex\hbox{%
          \HOLOGO@ReflectBox{E}%
        }%
      \else
        \@PackageError{hologo}{Internal error (choice/up)}\@ehc
      \fi
    \fi
  }%
}
%    \end{macrocode}
%    \end{macro}
%    \begin{macro}{\HoLogoHtml@Xe}
%    \begin{macrocode}
\def\HoLogoHtml@Xe#1{%
  \HoLogoCss@Xe
  \HOLOGO@Span{Xe}{%
    X%
    \HOLOGO@Span{e}{%
      \HCode{&\ltx@hashchar x018e;}%
    }%
  }%
}
%    \end{macrocode}
%    \end{macro}
%    \begin{macro}{\HoLogoCss@Xe}
%    \begin{macrocode}
\def\HoLogoCss@Xe{%
  \Css{%
    span.HoLogo-Xe span.HoLogo-e{%
      position:relative;%
      top:.5ex;%
      left-margin:-.1em;%
    }%
  }%
  \global\let\HoLogoCss@Xe\relax
}
%    \end{macrocode}
%    \end{macro}
%
%    \begin{macro}{\HoLogo@XeTeX}
%    \begin{macrocode}
\def\HoLogo@XeTeX#1{%
  \hologo{Xe}%
  \kern-.15em\relax
  \hologo{TeX}%
}
%    \end{macrocode}
%    \end{macro}
%
%    \begin{macro}{\HoLogoHtml@XeTeX}
%    \begin{macrocode}
\def\HoLogoHtml@XeTeX#1{%
  \HoLogoCss@XeTeX
  \HOLOGO@Span{XeTeX}{%
    \hologo{Xe}%
    \hologo{TeX}%
  }%
}
%    \end{macrocode}
%    \end{macro}
%    \begin{macro}{\HoLogoCss@XeTeX}
%    \begin{macrocode}
\def\HoLogoCss@XeTeX{%
  \Css{%
    span.HoLogo-XeTeX span.HoLogo-TeX{%
      margin-left:-.15em;%
    }%
  }%
  \global\let\HoLogoCss@XeTeX\relax
}
%    \end{macrocode}
%    \end{macro}
%
%    \begin{macro}{\HoLogo@XeLaTeX}
%    \begin{macrocode}
\def\HoLogo@XeLaTeX#1{%
  \hologo{Xe}%
  \kern-.13em%
  \hologo{LaTeX}%
}
%    \end{macrocode}
%    \end{macro}
%    \begin{macro}{\HoLogoHtml@XeLaTeX}
%    \begin{macrocode}
\def\HoLogoHtml@XeLaTeX#1{%
  \HoLogoCss@XeLaTeX
  \HOLOGO@Span{XeLaTeX}{%
    \hologo{Xe}%
    \hologo{LaTeX}%
  }%
}
%    \end{macrocode}
%    \end{macro}
%    \begin{macro}{\HoLogoCss@XeLaTeX}
%    \begin{macrocode}
\def\HoLogoCss@XeLaTeX{%
  \Css{%
    span.HoLogo-XeLaTeX span.HoLogo-Xe{%
      margin-right:-.13em;%
    }%
  }%
  \global\let\HoLogoCss@XeLaTeX\relax
}
%    \end{macrocode}
%    \end{macro}
%
% \subsubsection{\hologo{pdfTeX}, \hologo{pdfLaTeX}}
%
%    \begin{macro}{\HoLogo@pdfTeX}
%    \begin{macrocode}
\def\HoLogo@pdfTeX#1{%
  \HOLOGO@mbox{%
    #1{p}{P}df\hologo{TeX}%
  }%
}
%    \end{macrocode}
%    \end{macro}
%    \begin{macro}{\HoLogoCs@pdfTeX}
%    \begin{macrocode}
\def\HoLogoCs@pdfTeX#1{#1{p}{P}dfTeX}
%    \end{macrocode}
%    \end{macro}
%    \begin{macro}{\HoLogoBkm@pdfTeX}
%    \begin{macrocode}
\def\HoLogoBkm@pdfTeX#1{%
  #1{p}{P}df\hologo{TeX}%
}
%    \end{macrocode}
%    \end{macro}
%    \begin{macro}{\HoLogoHtml@pdfTeX}
%    \begin{macrocode}
\let\HoLogoHtml@pdfTeX\HoLogo@pdfTeX
%    \end{macrocode}
%    \end{macro}
%
%    \begin{macro}{\HoLogo@pdfLaTeX}
%    \begin{macrocode}
\def\HoLogo@pdfLaTeX#1{%
  \HOLOGO@mbox{%
    #1{p}{P}df\hologo{LaTeX}%
  }%
}
%    \end{macrocode}
%    \end{macro}
%    \begin{macro}{\HoLogoCs@pdfLaTeX}
%    \begin{macrocode}
\def\HoLogoCs@pdfLaTeX#1{#1{p}{P}dfLaTeX}
%    \end{macrocode}
%    \end{macro}
%    \begin{macro}{\HoLogoBkm@pdfLaTeX}
%    \begin{macrocode}
\def\HoLogoBkm@pdfLaTeX#1{%
  #1{p}{P}df\hologo{LaTeX}%
}
%    \end{macrocode}
%    \end{macro}
%    \begin{macro}{\HoLogoHtml@pdfLaTeX}
%    \begin{macrocode}
\let\HoLogoHtml@pdfLaTeX\HoLogo@pdfLaTeX
%    \end{macrocode}
%    \end{macro}
%
% \subsubsection{\hologo{VTeX}}
%
%    \begin{macro}{\HoLogo@VTeX}
%    \begin{macrocode}
\def\HoLogo@VTeX#1{%
  \HOLOGO@mbox{%
    V\hologo{TeX}%
  }%
}
%    \end{macrocode}
%    \end{macro}
%    \begin{macro}{\HoLogoHtml@VTeX}
%    \begin{macrocode}
\let\HoLogoHtml@VTeX\HoLogo@VTeX
%    \end{macrocode}
%    \end{macro}
%
% \subsubsection{\hologo{AmS}, \dots}
%
%    Source: class \xclass{amsdtx}
%
%    \begin{macro}{\HoLogo@AmS}
%    \begin{macrocode}
\def\HoLogo@AmS#1{%
  \HoLogoFont@font{AmS}{sy}{%
    A%
    \kern-.1667em%
    \lower.5ex\hbox{M}%
    \kern-.125em%
    S%
  }%
}
%    \end{macrocode}
%    \end{macro}
%    \begin{macro}{\HoLogoBkm@AmS}
%    \begin{macrocode}
\def\HoLogoBkm@AmS#1{AmS}
%    \end{macrocode}
%    \end{macro}
%    \begin{macro}{\HoLogoHtml@AmS}
%    \begin{macrocode}
\def\HoLogoHtml@AmS#1{%
  \HoLogoCss@AmS
%  \HoLogoFont@font{AmS}{sy}{%
    \HOLOGO@Span{AmS}{%
      A%
      \HOLOGO@Span{M}{M}%
      S%
    }%
%   }%
}
%    \end{macrocode}
%    \end{macro}
%    \begin{macro}{\HoLogoCss@AmS}
%    \begin{macrocode}
\def\HoLogoCss@AmS{%
  \Css{%
    span.HoLogo-AmS span.HoLogo-M{%
      position:relative;%
      top:.5ex;%
      margin-left:-.1667em;%
      margin-right:-.125em;%
      text-decoration:none;%
    }%
  }%
  \global\let\HoLogoCss@AmS\relax
}
%    \end{macrocode}
%    \end{macro}
%
%    \begin{macro}{\HoLogo@AmSTeX}
%    \begin{macrocode}
\def\HoLogo@AmSTeX#1{%
  \hologo{AmS}%
  \HOLOGO@hyphen
  \hologo{TeX}%
}
%    \end{macrocode}
%    \end{macro}
%    \begin{macro}{\HoLogoBkm@AmSTeX}
%    \begin{macrocode}
\def\HoLogoBkm@AmSTeX#1{AmS-TeX}%
%    \end{macrocode}
%    \end{macro}
%    \begin{macro}{\HoLogoHtml@AmSTeX}
%    \begin{macrocode}
\let\HoLogoHtml@AmSTeX\HoLogo@AmSTeX
%    \end{macrocode}
%    \end{macro}
%
%    \begin{macro}{\HoLogo@AmSLaTeX}
%    \begin{macrocode}
\def\HoLogo@AmSLaTeX#1{%
  \hologo{AmS}%
  \HOLOGO@hyphen
  \hologo{LaTeX}%
}
%    \end{macrocode}
%    \end{macro}
%    \begin{macro}{\HoLogoBkm@AmSLaTeX}
%    \begin{macrocode}
\def\HoLogoBkm@AmSLaTeX#1{AmS-LaTeX}%
%    \end{macrocode}
%    \end{macro}
%    \begin{macro}{\HoLogoHtml@AmSLaTeX}
%    \begin{macrocode}
\let\HoLogoHtml@AmSLaTeX\HoLogo@AmSLaTeX
%    \end{macrocode}
%    \end{macro}
%
% \subsubsection{\hologo{BibTeX}}
%
%    \begin{macro}{\HoLogo@BibTeX@sc}
%    A definition of \hologo{BibTeX} is provided in
%    the documentation source for the manual of \hologo{BibTeX}
%    \cite{btxdoc}.
%\begin{quote}
%\begin{verbatim}
%\def\BibTeX{%
%  {%
%    \rm
%    B%
%    \kern-.05em%
%    {%
%      \sc
%      i%
%      \kern-.025em %
%      b%
%    }%
%    \kern-.08em
%    T%
%    \kern-.1667em%
%    \lower.7ex\hbox{E}%
%    \kern-.125em%
%    X%
%  }%
%}
%\end{verbatim}
%\end{quote}
%    \begin{macrocode}
\def\HoLogo@BibTeX@sc#1{%
  B%
  \kern-.05em%
  \HoLogoFont@font{BibTeX}{sc}{%
    i%
    \kern-.025em%
    b%
  }%
  \HOLOGO@discretionary
  \kern-.08em%
  \hologo{TeX}%
}
%    \end{macrocode}
%    \end{macro}
%    \begin{macro}{\HoLogoHtml@BibTeX@sc}
%    \begin{macrocode}
\def\HoLogoHtml@BibTeX@sc#1{%
  \HoLogoCss@BibTeX@sc
  \HOLOGO@Span{BibTeX-sc}{%
    B%
    \HOLOGO@Span{i}{i}%
    \HOLOGO@Span{b}{b}%
    \hologo{TeX}%
  }%
}
%    \end{macrocode}
%    \end{macro}
%    \begin{macro}{\HoLogoCss@BibTeX@sc}
%    \begin{macrocode}
\def\HoLogoCss@BibTeX@sc{%
  \Css{%
    span.HoLogo-BibTeX-sc span.HoLogo-i{%
      margin-left:-.05em;%
      margin-right:-.025em;%
      font-variant:small-caps;%
    }%
  }%
  \Css{%
    span.HoLogo-BibTeX-sc span.HoLogo-b{%
      margin-right:-.08em;%
      font-variant:small-caps;%
    }%
  }%
  \global\let\HoLogoCss@BibTeX@sc\relax
}
%    \end{macrocode}
%    \end{macro}
%
%    \begin{macro}{\HoLogo@BibTeX@sf}
%    Variant \xoption{sf} avoids trouble with unavailable
%    small caps fonts (e.g., bold versions of Computer Modern or
%    Latin Modern). The definition is taken from
%    package \xpackage{dtklogos} \cite{dtklogos}.
%\begin{quote}
%\begin{verbatim}
%\DeclareRobustCommand{\BibTeX}{%
%  B%
%  \kern-.05em%
%  \hbox{%
%    $\m@th$% %% force math size calculations
%    \csname S@\f@size\endcsname
%    \fontsize\sf@size\z@
%    \math@fontsfalse
%    \selectfont
%    I%
%    \kern-.025em%
%    B
%  }%
%  \kern-.08em%
%  \-%
%  \TeX
%}
%\end{verbatim}
%\end{quote}
%    \begin{macrocode}
\def\HoLogo@BibTeX@sf#1{%
  B%
  \kern-.05em%
  \HoLogoFont@font{BibTeX}{bibsf}{%
    I%
    \kern-.025em%
    B%
  }%
  \HOLOGO@discretionary
  \kern-.08em%
  \hologo{TeX}%
}
%    \end{macrocode}
%    \end{macro}
%    \begin{macro}{\HoLogoHtml@BibTeX@sf}
%    \begin{macrocode}
\def\HoLogoHtml@BibTeX@sf#1{%
  \HoLogoCss@BibTeX@sf
  \HOLOGO@Span{BibTeX-sf}{%
    B%
    \HoLogoFont@font{BibTeX}{bibsf}{%
      \HOLOGO@Span{i}{I}%
      B%
    }%
    \hologo{TeX}%
  }%
}
%    \end{macrocode}
%    \end{macro}
%    \begin{macro}{\HoLogoCss@BibTeX@sf}
%    \begin{macrocode}
\def\HoLogoCss@BibTeX@sf{%
  \Css{%
    span.HoLogo-BibTeX-sf span.HoLogo-i{%
      margin-left:-.05em;%
      margin-right:-.025em;%
    }%
  }%
  \Css{%
    span.HoLogo-BibTeX-sf span.HoLogo-TeX{%
      margin-left:-.08em;%
    }%
  }%
  \global\let\HoLogoCss@BibTeX@sf\relax
}
%    \end{macrocode}
%    \end{macro}
%
%    \begin{macro}{\HoLogo@BibTeX}
%    \begin{macrocode}
\def\HoLogo@BibTeX{\HoLogo@BibTeX@sf}
%    \end{macrocode}
%    \end{macro}
%    \begin{macro}{\HoLogoHtml@BibTeX}
%    \begin{macrocode}
\def\HoLogoHtml@BibTeX{\HoLogoHtml@BibTeX@sf}
%    \end{macrocode}
%    \end{macro}
%
% \subsubsection{\hologo{BibTeX8}}
%
%    \begin{macro}{\HoLogo@BibTeX8}
%    \begin{macrocode}
\expandafter\def\csname HoLogo@BibTeX8\endcsname#1{%
  \hologo{BibTeX}%
  8%
}
%    \end{macrocode}
%    \end{macro}
%
%    \begin{macro}{\HoLogoBkm@BibTeX8}
%    \begin{macrocode}
\expandafter\def\csname HoLogoBkm@BibTeX8\endcsname#1{%
  \hologo{BibTeX}%
  8%
}
%    \end{macrocode}
%    \end{macro}
%    \begin{macro}{\HoLogoHtml@BibTeX8}
%    \begin{macrocode}
\expandafter
\let\csname HoLogoHtml@BibTeX8\expandafter\endcsname
\csname HoLogo@BibTeX8\endcsname
%    \end{macrocode}
%    \end{macro}
%
% \subsubsection{\hologo{ConTeXt}}
%
%    \begin{macro}{\HoLogo@ConTeXt@simple}
%    \begin{macrocode}
\def\HoLogo@ConTeXt@simple#1{%
  \HOLOGO@mbox{Con}%
  \HOLOGO@discretionary
  \HOLOGO@mbox{\hologo{TeX}t}%
}
%    \end{macrocode}
%    \end{macro}
%    \begin{macro}{\HoLogoHtml@ConTeXt@simple}
%    \begin{macrocode}
\let\HoLogoHtml@ConTeXt@simple\HoLogo@ConTeXt@simple
%    \end{macrocode}
%    \end{macro}
%
%    \begin{macro}{\HoLogo@ConTeXt@narrow}
%    This definition of logo \hologo{ConTeXt} with variant \xoption{narrow}
%    comes from TUGboat's class \xclass{ltugboat} (version 2010/11/15 v2.8).
%    \begin{macrocode}
\def\HoLogo@ConTeXt@narrow#1{%
  \HOLOGO@mbox{C\kern-.0333emon}%
  \HOLOGO@discretionary
  \kern-.0667em%
  \HOLOGO@mbox{\hologo{TeX}\kern-.0333emt}%
}
%    \end{macrocode}
%    \end{macro}
%    \begin{macro}{\HoLogoHtml@ConTeXt@narrow}
%    \begin{macrocode}
\def\HoLogoHtml@ConTeXt@narrow#1{%
  \HoLogoCss@ConTeXt@narrow
  \HOLOGO@Span{ConTeXt-narrow}{%
    \HOLOGO@Span{C}{C}%
    on%
    \hologo{TeX}%
    t%
  }%
}
%    \end{macrocode}
%    \end{macro}
%    \begin{macro}{\HoLogoCss@ConTeXt@narrow}
%    \begin{macrocode}
\def\HoLogoCss@ConTeXt@narrow{%
  \Css{%
    span.HoLogo-ConTeXt-narrow span.HoLogo-C{%
      margin-left:-.0333em;%
    }%
  }%
  \Css{%
    span.HoLogo-ConTeXt-narrow span.HoLogo-TeX{%
      margin-left:-.0667em;%
      margin-right:-.0333em;%
    }%
  }%
  \global\let\HoLogoCss@ConTeXt@narrow\relax
}
%    \end{macrocode}
%    \end{macro}
%
%    \begin{macro}{\HoLogo@ConTeXt}
%    \begin{macrocode}
\def\HoLogo@ConTeXt{\HoLogo@ConTeXt@narrow}
%    \end{macrocode}
%    \end{macro}
%    \begin{macro}{\HoLogoHtml@ConTeXt}
%    \begin{macrocode}
\def\HoLogoHtml@ConTeXt{\HoLogoHtml@ConTeXt@narrow}
%    \end{macrocode}
%    \end{macro}
%
% \subsubsection{\hologo{emTeX}}
%
%    \begin{macro}{\HoLogo@emTeX}
%    \begin{macrocode}
\def\HoLogo@emTeX#1{%
  \HOLOGO@mbox{#1{e}{E}m}%
  \HOLOGO@discretionary
  \hologo{TeX}%
}
%    \end{macrocode}
%    \end{macro}
%    \begin{macro}{\HoLogoCs@emTeX}
%    \begin{macrocode}
\def\HoLogoCs@emTeX#1{#1{e}{E}mTeX}%
%    \end{macrocode}
%    \end{macro}
%    \begin{macro}{\HoLogoBkm@emTeX}
%    \begin{macrocode}
\def\HoLogoBkm@emTeX#1{%
  #1{e}{E}m\hologo{TeX}%
}
%    \end{macrocode}
%    \end{macro}
%    \begin{macro}{\HoLogoHtml@emTeX}
%    \begin{macrocode}
\let\HoLogoHtml@emTeX\HoLogo@emTeX
%    \end{macrocode}
%    \end{macro}
%
% \subsubsection{\hologo{ExTeX}}
%
%    \begin{macro}{\HoLogo@ExTeX}
%    The definition is taken from the FAQ of the
%    project \hologo{ExTeX}
%    \cite{ExTeX-FAQ}.
%\begin{quote}
%\begin{verbatim}
%\def\ExTeX{%
%  \textrm{% Logo always with serifs
%    \ensuremath{%
%      \textstyle
%      \varepsilon_{%
%        \kern-0.15em%
%        \mathcal{X}%
%      }%
%    }%
%    \kern-.15em%
%    \TeX
%  }%
%}
%\end{verbatim}
%\end{quote}
%    \begin{macrocode}
\def\HoLogo@ExTeX#1{%
  \HoLogoFont@font{ExTeX}{rm}{%
    \ltx@mbox{%
      \HOLOGO@MathSetup
      $%
        \textstyle
        \varepsilon_{%
          \kern-0.15em%
          \HoLogoFont@font{ExTeX}{sy}{X}%
        }%
      $%
    }%
    \HOLOGO@discretionary
    \kern-.15em%
    \hologo{TeX}%
  }%
}
%    \end{macrocode}
%    \end{macro}
%    \begin{macro}{\HoLogoHtml@ExTeX}
%    \begin{macrocode}
\def\HoLogoHtml@ExTeX#1{%
  \HoLogoCss@ExTeX
  \HoLogoFont@font{ExTeX}{rm}{%
    \HOLOGO@Span{ExTeX}{%
      \ltx@mbox{%
        \HOLOGO@MathSetup
        $\textstyle\varepsilon$%
        \HOLOGO@Span{X}{$\textstyle\chi$}%
        \hologo{TeX}%
      }%
    }%
  }%
}
%    \end{macrocode}
%    \end{macro}
%    \begin{macro}{\HoLogoBkm@ExTeX}
%    \begin{macrocode}
\def\HoLogoBkm@ExTeX#1{%
  \HOLOGO@PdfdocUnicode{#1{e}{E}x}{\textepsilon\textchi}%
  \hologo{TeX}%
}
%    \end{macrocode}
%    \end{macro}
%    \begin{macro}{\HoLogoCss@ExTeX}
%    \begin{macrocode}
\def\HoLogoCss@ExTeX{%
  \Css{%
    span.HoLogo-ExTeX{%
      font-family:serif;%
    }%
  }%
  \Css{%
    span.HoLogo-ExTeX span.HoLogo-TeX{%
      margin-left:-.15em;%
    }%
  }%
  \global\let\HoLogoCss@ExTeX\relax
}
%    \end{macrocode}
%    \end{macro}
%
% \subsubsection{\hologo{MiKTeX}}
%
%    \begin{macro}{\HoLogo@MiKTeX}
%    \begin{macrocode}
\def\HoLogo@MiKTeX#1{%
  \HOLOGO@mbox{MiK}%
  \HOLOGO@discretionary
  \hologo{TeX}%
}
%    \end{macrocode}
%    \end{macro}
%    \begin{macro}{\HoLogoHtml@MiKTeX}
%    \begin{macrocode}
\let\HoLogoHtml@MiKTeX\HoLogo@MiKTeX
%    \end{macrocode}
%    \end{macro}
%
% \subsubsection{\hologo{OzTeX} and friends}
%
%    Source: \hologo{OzTeX} FAQ \cite{OzTeX}:
%    \begin{quote}
%      |\def\OzTeX{O\kern-.03em z\kern-.15em\TeX}|\\
%      (There is no kerning in OzMF, OzMP and OzTtH.)
%    \end{quote}
%
%    \begin{macro}{\HoLogo@OzTeX}
%    \begin{macrocode}
\def\HoLogo@OzTeX#1{%
  O%
  \kern-.03em %
  z%
  \kern-.15em %
  \hologo{TeX}%
}
%    \end{macrocode}
%    \end{macro}
%    \begin{macro}{\HoLogoHtml@OzTeX}
%    \begin{macrocode}
\def\HoLogoHtml@OzTeX#1{%
  \HoLogoCss@OzTeX
  \HOLOGO@Span{OzTeX}{%
    O%
    \HOLOGO@Span{z}{z}%
    \hologo{TeX}%
  }%
}
%    \end{macrocode}
%    \end{macro}
%    \begin{macro}{\HoLogoCss@OzTeX}
%    \begin{macrocode}
\def\HoLogoCss@OzTeX{%
  \Css{%
    span.HoLogo-OzTeX span.HoLogo-z{%
      margin-left:-.03em;%
      margin-right:-.15em;%
    }%
  }%
  \global\let\HoLogoCss@OzTeX\relax
}
%    \end{macrocode}
%    \end{macro}
%
%    \begin{macro}{\HoLogo@OzMF}
%    \begin{macrocode}
\def\HoLogo@OzMF#1{%
  \HOLOGO@mbox{OzMF}%
}
%    \end{macrocode}
%    \end{macro}
%    \begin{macro}{\HoLogo@OzMP}
%    \begin{macrocode}
\def\HoLogo@OzMP#1{%
  \HOLOGO@mbox{OzMP}%
}
%    \end{macrocode}
%    \end{macro}
%    \begin{macro}{\HoLogo@OzTtH}
%    \begin{macrocode}
\def\HoLogo@OzTtH#1{%
  \HOLOGO@mbox{OzTtH}%
}
%    \end{macrocode}
%    \end{macro}
%
% \subsubsection{\hologo{PCTeX}}
%
%    \begin{macro}{\HoLogo@PCTeX}
%    \begin{macrocode}
\def\HoLogo@PCTeX#1{%
  \HOLOGO@mbox{PC}%
  \hologo{TeX}%
}
%    \end{macrocode}
%    \end{macro}
%    \begin{macro}{\HoLogoHtml@PCTeX}
%    \begin{macrocode}
\let\HoLogoHtml@PCTeX\HoLogo@PCTeX
%    \end{macrocode}
%    \end{macro}
%
% \subsubsection{\hologo{PiCTeX}}
%
%    The original definitions from \xfile{pictex.tex} \cite{PiCTeX}:
%\begin{quote}
%\begin{verbatim}
%\def\PiC{%
%  P%
%  \kern-.12em%
%  \lower.5ex\hbox{I}%
%  \kern-.075em%
%  C%
%}
%\def\PiCTeX{%
%  \PiC
%  \kern-.11em%
%  \TeX
%}
%\end{verbatim}
%\end{quote}
%
%    \begin{macro}{\HoLogo@PiC}
%    \begin{macrocode}
\def\HoLogo@PiC#1{%
  P%
  \kern-.12em%
  \lower.5ex\hbox{I}%
  \kern-.075em%
  C%
  \HOLOGO@SpaceFactor
}
%    \end{macrocode}
%    \end{macro}
%    \begin{macro}{\HoLogoHtml@PiC}
%    \begin{macrocode}
\def\HoLogoHtml@PiC#1{%
  \HoLogoCss@PiC
  \HOLOGO@Span{PiC}{%
    P%
    \HOLOGO@Span{i}{I}%
    C%
  }%
}
%    \end{macrocode}
%    \end{macro}
%    \begin{macro}{\HoLogoCss@PiC}
%    \begin{macrocode}
\def\HoLogoCss@PiC{%
  \Css{%
    span.HoLogo-PiC span.HoLogo-i{%
      position:relative;%
      top:.5ex;%
      margin-left:-.12em;%
      margin-right:-.075em;%
      text-decoration:none;%
    }%
  }%
  \global\let\HoLogoCss@PiC\relax
}
%    \end{macrocode}
%    \end{macro}
%
%    \begin{macro}{\HoLogo@PiCTeX}
%    \begin{macrocode}
\def\HoLogo@PiCTeX#1{%
  \hologo{PiC}%
  \HOLOGO@discretionary
  \kern-.11em%
  \hologo{TeX}%
}
%    \end{macrocode}
%    \end{macro}
%    \begin{macro}{\HoLogoHtml@PiCTeX}
%    \begin{macrocode}
\def\HoLogoHtml@PiCTeX#1{%
  \HoLogoCss@PiCTeX
  \HOLOGO@Span{PiCTeX}{%
    \hologo{PiC}%
    \hologo{TeX}%
  }%
}
%    \end{macrocode}
%    \end{macro}
%    \begin{macro}{\HoLogoCss@PiCTeX}
%    \begin{macrocode}
\def\HoLogoCss@PiCTeX{%
  \Css{%
    span.HoLogo-PiCTeX span.HoLogo-PiC{%
      margin-right:-.11em;%
    }%
  }%
  \global\let\HoLogoCss@PiCTeX\relax
}
%    \end{macrocode}
%    \end{macro}
%
% \subsubsection{\hologo{teTeX}}
%
%    \begin{macro}{\HoLogo@teTeX}
%    \begin{macrocode}
\def\HoLogo@teTeX#1{%
  \HOLOGO@mbox{#1{t}{T}e}%
  \HOLOGO@discretionary
  \hologo{TeX}%
}
%    \end{macrocode}
%    \end{macro}
%    \begin{macro}{\HoLogoCs@teTeX}
%    \begin{macrocode}
\def\HoLogoCs@teTeX#1{#1{t}{T}dfTeX}
%    \end{macrocode}
%    \end{macro}
%    \begin{macro}{\HoLogoBkm@teTeX}
%    \begin{macrocode}
\def\HoLogoBkm@teTeX#1{%
  #1{t}{T}e\hologo{TeX}%
}
%    \end{macrocode}
%    \end{macro}
%    \begin{macro}{\HoLogoHtml@teTeX}
%    \begin{macrocode}
\let\HoLogoHtml@teTeX\HoLogo@teTeX
%    \end{macrocode}
%    \end{macro}
%
% \subsubsection{\hologo{TeX4ht}}
%
%    \begin{macro}{\HoLogo@TeX4ht}
%    \begin{macrocode}
\expandafter\def\csname HoLogo@TeX4ht\endcsname#1{%
  \HOLOGO@mbox{\hologo{TeX}4ht}%
}
%    \end{macrocode}
%    \end{macro}
%    \begin{macro}{\HoLogoHtml@TeX4ht}
%    \begin{macrocode}
\expandafter
\let\csname HoLogoHtml@TeX4ht\expandafter\endcsname
\csname HoLogo@TeX4ht\endcsname
%    \end{macrocode}
%    \end{macro}
%
%
% \subsubsection{\hologo{SageTeX}}
%
%    \begin{macro}{\HoLogo@SageTeX}
%    \begin{macrocode}
\def\HoLogo@SageTeX#1{%
  \HOLOGO@mbox{Sage}%
  \HOLOGO@discretionary
  \HOLOGO@NegativeKerning{eT,oT,To}%
  \hologo{TeX}%
}
%    \end{macrocode}
%    \end{macro}
%    \begin{macro}{\HoLogoHtml@SageTeX}
%    \begin{macrocode}
\let\HoLogoHtml@SageTeX\HoLogo@SageTeX
%    \end{macrocode}
%    \end{macro}
%
% \subsection{\hologo{METAFONT} and friends}
%
%    \begin{macro}{\HoLogo@METAFONT}
%    \begin{macrocode}
\def\HoLogo@METAFONT#1{%
  \HoLogoFont@font{METAFONT}{logo}{%
    \HOLOGO@mbox{META}%
    \HOLOGO@discretionary
    \HOLOGO@mbox{FONT}%
  }%
}
%    \end{macrocode}
%    \end{macro}
%
%    \begin{macro}{\HoLogo@METAPOST}
%    \begin{macrocode}
\def\HoLogo@METAPOST#1{%
  \HoLogoFont@font{METAPOST}{logo}{%
    \HOLOGO@mbox{META}%
    \HOLOGO@discretionary
    \HOLOGO@mbox{POST}%
  }%
}
%    \end{macrocode}
%    \end{macro}
%
%    \begin{macro}{\HoLogo@MetaFun}
%    \begin{macrocode}
\def\HoLogo@MetaFun#1{%
  \HOLOGO@mbox{Meta}%
  \HOLOGO@discretionary
  \HOLOGO@mbox{Fun}%
}
%    \end{macrocode}
%    \end{macro}
%
%    \begin{macro}{\HoLogo@MetaPost}
%    \begin{macrocode}
\def\HoLogo@MetaPost#1{%
  \HOLOGO@mbox{Meta}%
  \HOLOGO@discretionary
  \HOLOGO@mbox{Post}%
}
%    \end{macrocode}
%    \end{macro}
%
% \subsection{Others}
%
% \subsubsection{\hologo{biber}}
%
%    \begin{macro}{\HoLogo@biber}
%    \begin{macrocode}
\def\HoLogo@biber#1{%
  \HOLOGO@mbox{#1{b}{B}i}%
  \HOLOGO@discretionary
  \HOLOGO@mbox{ber}%
}
%    \end{macrocode}
%    \end{macro}
%    \begin{macro}{\HoLogoCs@biber}
%    \begin{macrocode}
\def\HoLogoCs@biber#1{#1{b}{B}iber}
%    \end{macrocode}
%    \end{macro}
%    \begin{macro}{\HoLogoBkm@biber}
%    \begin{macrocode}
\def\HoLogoBkm@biber#1{%
  #1{b}{B}iber%
}
%    \end{macrocode}
%    \end{macro}
%    \begin{macro}{\HoLogoHtml@biber}
%    \begin{macrocode}
\let\HoLogoHtml@biber\HoLogo@biber
%    \end{macrocode}
%    \end{macro}
%
% \subsubsection{\hologo{KOMAScript}}
%
%    \begin{macro}{\HoLogo@KOMAScript}
%    The definition for \hologo{KOMAScript} is taken
%    from \hologo{KOMAScript} (\xfile{scrlogo.dtx}, reformatted) \cite{scrlogo}:
%\begin{quote}
%\begin{verbatim}
%\@ifundefined{KOMAScript}{%
%  \DeclareRobustCommand{\KOMAScript}{%
%    \textsf{%
%      K\kern.05em O\kern.05emM\kern.05em A%
%      \kern.1em-\kern.1em %
%      Script%
%    }%
%  }%
%}{}
%\end{verbatim}
%\end{quote}
%    \begin{macrocode}
\def\HoLogo@KOMAScript#1{%
  \HoLogoFont@font{KOMAScript}{sf}{%
    \HOLOGO@mbox{%
      K\kern.05em%
      O\kern.05em%
      M\kern.05em%
      A%
    }%
    \kern.1em%
    \HOLOGO@hyphen
    \kern.1em%
    \HOLOGO@mbox{Script}%
  }%
}
%    \end{macrocode}
%    \end{macro}
%    \begin{macro}{\HoLogoBkm@KOMAScript}
%    \begin{macrocode}
\def\HoLogoBkm@KOMAScript#1{%
  KOMA-Script%
}
%    \end{macrocode}
%    \end{macro}
%    \begin{macro}{\HoLogoHtml@KOMAScript}
%    \begin{macrocode}
\def\HoLogoHtml@KOMAScript#1{%
  \HoLogoCss@KOMAScript
  \HoLogoFont@font{KOMAScript}{sf}{%
    \HOLOGO@Span{KOMAScript}{%
      K%
      \HOLOGO@Span{O}{O}%
      M%
      \HOLOGO@Span{A}{A}%
      \HOLOGO@Span{hyphen}{-}%
      Script%
    }%
  }%
}
%    \end{macrocode}
%    \end{macro}
%    \begin{macro}{\HoLogoCss@KOMAScript}
%    \begin{macrocode}
\def\HoLogoCss@KOMAScript{%
  \Css{%
    span.HoLogo-KOMAScript{%
      font-family:sans-serif;%
    }%
  }%
  \Css{%
    span.HoLogo-KOMAScript span.HoLogo-O{%
      padding-left:.05em;%
      padding-right:.05em;%
    }%
  }%
  \Css{%
    span.HoLogo-KOMAScript span.HoLogo-A{%
      padding-left:.05em;%
    }%
  }%
  \Css{%
    span.HoLogo-KOMAScript span.HoLogo-hyphen{%
      padding-left:.1em;%
      padding-right:.1em;%
    }%
  }%
  \global\let\HoLogoCss@KOMAScript\relax
}
%    \end{macrocode}
%    \end{macro}
%
% \subsubsection{\hologo{LyX}}
%
%    \begin{macro}{\HoLogo@LyX}
%    The definition is taken from the documentation source files
%    of \hologo{LyX}, \xfile{Intro.lyx} \cite{LyX}:
%\begin{quote}
%\begin{verbatim}
%\def\LyX{%
%  \texorpdfstring{%
%    L\kern-.1667em\lower.25em\hbox{Y}\kern-.125emX\@%
%  }{%
%    LyX%
%  }%
%}
%\end{verbatim}
%\end{quote}
%    \begin{macrocode}
\def\HoLogo@LyX#1{%
  L%
  \kern-.1667em%
  \lower.25em\hbox{Y}%
  \kern-.125em%
  X%
  \HOLOGO@SpaceFactor
}
%    \end{macrocode}
%    \end{macro}
%    \begin{macro}{\HoLogoHtml@LyX}
%    \begin{macrocode}
\def\HoLogoHtml@LyX#1{%
  \HoLogoCss@LyX
  \HOLOGO@Span{LyX}{%
    L%
    \HOLOGO@Span{y}{Y}%
    X%
  }%
}
%    \end{macrocode}
%    \end{macro}
%    \begin{macro}{\HoLogoCss@LyX}
%    \begin{macrocode}
\def\HoLogoCss@LyX{%
  \Css{%
    span.HoLogo-LyX span.HoLogo-y{%
      position:relative;%
      top:.25em;%
      margin-left:-.1667em;%
      margin-right:-.125em;%
      text-decoration:none;%
    }%
  }%
  \global\let\HoLogoCss@LyX\relax
}
%    \end{macrocode}
%    \end{macro}
%
% \subsubsection{\hologo{NTS}}
%
%    \begin{macro}{\HoLogo@NTS}
%    Definition for \hologo{NTS} can be found in
%    package \xpackage{etex\textunderscore man} for the \hologo{eTeX} manual \cite{etexman}
%    and in package \xpackage{dtklogos} \cite{dtklogos}:
%\begin{quote}
%\begin{verbatim}
%\def\NTS{%
%  \leavevmode
%  \hbox{%
%    $%
%      \cal N%
%      \kern-0.35em%
%      \lower0.5ex\hbox{$\cal T$}%
%      \kern-0.2em%
%      S%
%    $%
%  }%
%}
%\end{verbatim}
%\end{quote}
%    \begin{macrocode}
\def\HoLogo@NTS#1{%
  \HoLogoFont@font{NTS}{sy}{%
    N\/%
    \kern-.35em%
    \lower.5ex\hbox{T\/}%
    \kern-.2em%
    S\/%
  }%
  \HOLOGO@SpaceFactor
}
%    \end{macrocode}
%    \end{macro}
%
% \subsubsection{\Hologo{TTH} (\hologo{TeX} to HTML translator)}
%
%    Source: \url{http://hutchinson.belmont.ma.us/tth/}
%    In the HTML source the second `T' is printed as subscript.
%\begin{quote}
%\begin{verbatim}
%T<sub>T</sub>H
%\end{verbatim}
%\end{quote}
%    \begin{macro}{\HoLogo@TTH}
%    \begin{macrocode}
\def\HoLogo@TTH#1{%
  \ltx@mbox{%
    T\HOLOGO@SubScript{T}H%
  }%
  \HOLOGO@SpaceFactor
}
%    \end{macrocode}
%    \end{macro}
%
%    \begin{macro}{\HoLogoHtml@TTH}
%    \begin{macrocode}
\def\HoLogoHtml@TTH#1{%
  T\HCode{<sub>}T\HCode{</sub>}H%
}
%    \end{macrocode}
%    \end{macro}
%
% \subsubsection{\Hologo{HanTheThanh}}
%
%    Partial source: Package \xpackage{dtklogos}.
%    The double accent is U+1EBF (latin small letter e with circumflex
%    and acute).
%    \begin{macro}{\HoLogo@HanTheThanh}
%    \begin{macrocode}
\def\HoLogo@HanTheThanh#1{%
  \ltx@mbox{H\`an}%
  \HOLOGO@space
  \ltx@mbox{%
    Th%
    \HOLOGO@IfCharExists{"1EBF}{%
      \char"1EBF\relax
    }{%
      \^e\hbox to 0pt{\hss\raise .5ex\hbox{\'{}}}%
    }%
  }%
  \HOLOGO@space
  \ltx@mbox{Th\`anh}%
}
%    \end{macrocode}
%    \end{macro}
%    \begin{macro}{\HoLogoBkm@HanTheThanh}
%    \begin{macrocode}
\def\HoLogoBkm@HanTheThanh#1{%
  H\`an %
  Th\HOLOGO@PdfdocUnicode{\^e}{\9036\277} %
  Th\`anh%
}
%    \end{macrocode}
%    \end{macro}
%    \begin{macro}{\HoLogoHtml@HanTheThanh}
%    \begin{macrocode}
\def\HoLogoHtml@HanTheThanh#1{%
  H\`an %
  Th\HCode{&\ltx@hashchar x1ebf;} %
  Th\`anh%
}
%    \end{macrocode}
%    \end{macro}
%
% \subsection{Driver detection}
%
%    \begin{macrocode}
\HOLOGO@IfExists\InputIfFileExists{%
  \InputIfFileExists{hologo.cfg}{}{}%
}{%
  \ltx@IfUndefined{pdf@filesize}{%
    \def\HOLOGO@InputIfExists{%
      \openin\HOLOGO@temp=hologo.cfg\relax
      \ifeof\HOLOGO@temp
        \closein\HOLOGO@temp
      \else
        \closein\HOLOGO@temp
        \begingroup
          \def\x{LaTeX2e}%
        \expandafter\endgroup
        \ifx\fmtname\x
          % \iffalse meta-comment
%
% File: hologo.dtx
% Version: 2016/05/12 v1.11
% Info: A logo collection with bookmark support
%
% Copyright (C) 2010-2012 by
%    Heiko Oberdiek <heiko.oberdiek at googlemail.com>
%
% This work may be distributed and/or modified under the
% conditions of the LaTeX Project Public License, either
% version 1.3c of this license or (at your option) any later
% version. This version of this license is in
%    http://www.latex-project.org/lppl/lppl-1-3c.txt
% and the latest version of this license is in
%    http://www.latex-project.org/lppl.txt
% and version 1.3 or later is part of all distributions of
% LaTeX version 2005/12/01 or later.
%
% This work has the LPPL maintenance status "maintained".
%
% This Current Maintainer of this work is Heiko Oberdiek.
%
% The Base Interpreter refers to any `TeX-Format',
% because some files are installed in TDS:tex/generic//.
%
% This work consists of the main source file hologo.dtx
% and the derived files
%    hologo.sty, hologo.pdf, hologo.ins, hologo.drv, hologo-example.tex,
%    hologo-test1.tex, hologo-test-spacefactor.tex,
%    hologo-test-list.tex.
%
% Distribution:
%    CTAN:macros/latex/contrib/oberdiek/hologo.dtx
%    CTAN:macros/latex/contrib/oberdiek/hologo.pdf
%
% Unpacking:
%    (a) If hologo.ins is present:
%           tex hologo.ins
%    (b) Without hologo.ins:
%           tex hologo.dtx
%    (c) If you insist on using LaTeX
%           latex \let\install=y\input{hologo.dtx}
%        (quote the arguments according to the demands of your shell)
%
% Documentation:
%    (a) If hologo.drv is present:
%           latex hologo.drv
%    (b) Without hologo.drv:
%           latex hologo.dtx; ...
%    The class ltxdoc loads the configuration file ltxdoc.cfg
%    if available. Here you can specify further options, e.g.
%    use A4 as paper format:
%       \PassOptionsToClass{a4paper}{article}
%
%    Programm calls to get the documentation (example):
%       pdflatex hologo.dtx
%       makeindex -s gind.ist hologo.idx
%       pdflatex hologo.dtx
%       makeindex -s gind.ist hologo.idx
%       pdflatex hologo.dtx
%
% Installation:
%    TDS:tex/generic/oberdiek/hologo.sty
%    TDS:doc/latex/oberdiek/hologo.pdf
%    TDS:doc/latex/oberdiek/example/hologo-example.tex
%    TDS:doc/latex/oberdiek/test/hologo-test1.tex
%    TDS:doc/latex/oberdiek/test/hologo-test-spacefactor.tex
%    TDS:doc/latex/oberdiek/test/hologo-test-list.tex
%    TDS:source/latex/oberdiek/hologo.dtx
%
%<*ignore>
\begingroup
  \catcode123=1 %
  \catcode125=2 %
  \def\x{LaTeX2e}%
\expandafter\endgroup
\ifcase 0\ifx\install y1\fi\expandafter
         \ifx\csname processbatchFile\endcsname\relax\else1\fi
         \ifx\fmtname\x\else 1\fi\relax
\else\csname fi\endcsname
%</ignore>
%<*install>
\input docstrip.tex
\Msg{************************************************************************}
\Msg{* Installation}
\Msg{* Package: hologo 2016/05/12 v1.11 A logo collection with bookmark support (HO)}
\Msg{************************************************************************}

\keepsilent
\askforoverwritefalse

\let\MetaPrefix\relax
\preamble

This is a generated file.

Project: hologo
Version: 2016/05/12 v1.11

Copyright (C) 2010-2012 by
   Heiko Oberdiek <heiko.oberdiek at googlemail.com>

This work may be distributed and/or modified under the
conditions of the LaTeX Project Public License, either
version 1.3c of this license or (at your option) any later
version. This version of this license is in
   http://www.latex-project.org/lppl/lppl-1-3c.txt
and the latest version of this license is in
   http://www.latex-project.org/lppl.txt
and version 1.3 or later is part of all distributions of
LaTeX version 2005/12/01 or later.

This work has the LPPL maintenance status "maintained".

This Current Maintainer of this work is Heiko Oberdiek.

The Base Interpreter refers to any `TeX-Format',
because some files are installed in TDS:tex/generic//.

This work consists of the main source file hologo.dtx
and the derived files
   hologo.sty, hologo.pdf, hologo.ins, hologo.drv, hologo-example.tex,
   hologo-test1.tex, hologo-test-spacefactor.tex,
   hologo-test-list.tex.

\endpreamble
\let\MetaPrefix\DoubleperCent

\generate{%
  \file{hologo.ins}{\from{hologo.dtx}{install}}%
  \file{hologo.drv}{\from{hologo.dtx}{driver}}%
  \usedir{tex/generic/oberdiek}%
  \file{hologo.sty}{\from{hologo.dtx}{package}}%
  \usedir{doc/latex/oberdiek/example}%
  \file{hologo-example.tex}{\from{hologo.dtx}{example}}%
  \usedir{doc/latex/oberdiek/test}%
  \file{hologo-test1.tex}{\from{hologo.dtx}{test1}}%
  \file{hologo-test-spacefactor.tex}{\from{hologo.dtx}{test-spacefactor}}%
  \file{hologo-test-list.tex}{\from{hologo.dtx}{test-list}}%
  \nopreamble
  \nopostamble
  \usedir{source/latex/oberdiek/catalogue}%
  \file{hologo.xml}{\from{hologo.dtx}{catalogue}}%
}

\catcode32=13\relax% active space
\let =\space%
\Msg{************************************************************************}
\Msg{*}
\Msg{* To finish the installation you have to move the following}
\Msg{* file into a directory searched by TeX:}
\Msg{*}
\Msg{*     hologo.sty}
\Msg{*}
\Msg{* To produce the documentation run the file `hologo.drv'}
\Msg{* through LaTeX.}
\Msg{*}
\Msg{* Happy TeXing!}
\Msg{*}
\Msg{************************************************************************}

\endbatchfile
%</install>
%<*ignore>
\fi
%</ignore>
%<*driver>
\NeedsTeXFormat{LaTeX2e}
\ProvidesFile{hologo.drv}%
  [2016/05/12 v1.11 A logo collection with bookmark support (HO)]%
\documentclass{ltxdoc}
\usepackage{holtxdoc}[2011/11/22]
\usepackage{hologo}[2016/05/12]
\usepackage{longtable}
\usepackage{array}
\usepackage{paralist}
%\usepackage[T1]{fontenc}
%\usepackage{lmodern}
\begin{document}
  \DocInput{hologo.dtx}%
\end{document}
%</driver>
% \fi
%
%
% \CharacterTable
%  {Upper-case    \A\B\C\D\E\F\G\H\I\J\K\L\M\N\O\P\Q\R\S\T\U\V\W\X\Y\Z
%   Lower-case    \a\b\c\d\e\f\g\h\i\j\k\l\m\n\o\p\q\r\s\t\u\v\w\x\y\z
%   Digits        \0\1\2\3\4\5\6\7\8\9
%   Exclamation   \!     Double quote  \"     Hash (number) \#
%   Dollar        \$     Percent       \%     Ampersand     \&
%   Acute accent  \'     Left paren    \(     Right paren   \)
%   Asterisk      \*     Plus          \+     Comma         \,
%   Minus         \-     Point         \.     Solidus       \/
%   Colon         \:     Semicolon     \;     Less than     \<
%   Equals        \=     Greater than  \>     Question mark \?
%   Commercial at \@     Left bracket  \[     Backslash     \\
%   Right bracket \]     Circumflex    \^     Underscore    \_
%   Grave accent  \`     Left brace    \{     Vertical bar  \|
%   Right brace   \}     Tilde         \~}
%
% \GetFileInfo{hologo.drv}
%
% \title{The \xpackage{hologo} package}
% \date{2016/05/12 v1.11}
% \author{Heiko Oberdiek\\\xemail{heiko.oberdiek at googlemail.com}}
%
% \maketitle
%
% \begin{abstract}
% This package starts a collection of logos with support for bookmarks
% strings.
% \end{abstract}
%
% \tableofcontents
%
% \section{Documentation}
%
% \subsection{Logo macros}
%
% \begin{declcs}{hologo} \M{name}
% \end{declcs}
% Macro \cs{hologo} sets the logo with name \meta{name}.
% The following table shows the supported names.
%
% \begingroup
%   \def\hologoEntry#1#2#3{^^A
%     #1&#2&\hologoLogoSetup{#1}{variant=#2}\hologo{#1}&#3\tabularnewline
%   }
%   \begin{longtable}{>{\ttfamily}l>{\ttfamily}lll}
%     \rmfamily\bfseries{name} & \rmfamily\bfseries variant
%     & \bfseries logo & \bfseries since\\
%     \hline
%     \endhead
%     \hologoList
%   \end{longtable}
% \endgroup
%
% \begin{declcs}{Hologo} \M{name}
% \end{declcs}
% Macro \cs{Hologo} starts the logo \meta{name} with an uppercase
% letter. As an exception small greek letters are not converted
% to uppercase. Examples, see \hologo{eTeX} and \hologo{ExTeX}.
%
% \subsection{Setup macros}
%
% The package does not support package options, but the following
% setup macros can be used to set options.
%
% \begin{declcs}{hologoSetup} \M{key value list}
% \end{declcs}
% Macro \cs{hologoSetup} sets global options.
%
% \begin{declcs}{hologoLogoSetup} \M{logo} \M{key value list}
% \end{declcs}
% Some options can also be used to configure a logo.
% These settings take precedence over global option settings.
%
% \subsection{Options}\label{sec:options}
%
% There are boolean and string options:
% \begin{description}
% \item[Boolean option:]
% It takes |true| or |false|
% as value. If the value is omitted, then |true| is used.
% \item[String option:]
% A value must be given as string. (But the string might be empty.)
% \end{description}
% The following options can be used both in \cs{hologoSetup}
% and \cs{hologoLogoSetup}:
% \begin{description}
% \def\entry#1{\item[\xoption{#1}:]}
% \entry{break}
%   enables or disables line breaks inside the logo. This setting is
%   refined by options \xoption{hyphenbreak}, \xoption{spacebreak}
%   or \xoption{discretionarybreak}.
%   Default is |false|.
% \entry{hyphenbreak}
%   enables or disables the line break right after the hyphen character.
% \entry{spacebreak}
%   enables or disables line breaks at space characters.
% \entry{discretionarybreak}
%   enables or disables line breaks at hyphenation points
%   (inserted by \cs{-}).
% \end{description}
% Macro \cs{hologoLogoSetup} also knows:
% \begin{description}
% \item[\xoption{variant}:]
%   This is a string option. It specifies a variant of a logo that
%   must exist. An empty string selects the package default variant.
% \end{description}
% Example:
% \begin{quote}
%   |\hologoSetup{break=false}|\\
%   |\hologoLogoSetup{plainTeX}{variant=hyphen,hyphenbreak}|\\
%   Then ``plain-\TeX'' contains one break point after the hyphen.
% \end{quote}
%
% \subsection{Driver options}
%
% Sometimes graphical operations are needed to construct some
% glyphs (e.g.\ \hologo{XeTeX}). If package \xpackage{graphics}
% or package \xpackage{pgf} are found, then the macros are taken
% from there. Otherwise the packge defines its own operations
% and therefore needs the driver information. Many drivers are
% detected automatically (\hologo{pdfTeX}/\hologo{LuaTeX}
% in PDF mode, \hologo{XeTeX}, \hologo{VTeX}). These have precedence
% over a driver option. The driver can be given as package option
% or using \cs{hologoDriverSetup}.
% The following list contains the recognized driver options:
% \begin{itemize}
% \item \xoption{pdftex}, \xoption{luatex}
% \item \xoption{dvipdfm}, \xoption{dvipdfmx}
% \item \xoption{dvips}, \xoption{dvipsone}, \xoption{xdvi}
% \item \xoption{xetex}
% \item \xoption{vtex}
% \end{itemize}
% The left driver of a line is the driver name that is used internally.
% The following names are aliases for drivers that use the
% same method. Therefore the entry in the \xext{log} file for
% the used driver prints the internally used driver name.
% \begin{description}
% \item[\xoption{driverfallback}:]
%   This option expects a driver that is used,
%   if the driver could not be detected automatically.
% \end{description}
%
% \begin{declcs}{hologoDriverSetup} \M{driver option}
% \end{declcs}
% The driver can also be configured after package loading
% using \cs{hologoDriverSetup}, also the way for \hologo{plainTeX}
% to setup the driver.
%
% \subsection{Font setup}
%
% Some logos require a special font, but should also be usable by
% \hologo{plainTeX}. Therefore the package provides some ways
% to influence the font settings. The options below
% take font settings as values. Both font commands
% such as \cs{sffamily} and macros that take one argument
% like \cs{textsf} can be used.
%
% \begin{declcs}{hologoFontSetup} \M{key value list}
% \end{declcs}
% Macro \cs{hologoFontSetup} sets the fonts for all logos.
% Supported keys:
% \begin{description}
% \def\entry#1{\item[\xoption{#1}:]}
% \entry{general}
%   This font is used for all logos. The default is empty.
%   That means no special font is used.
% \entry{bibsf}
%   This font is used for
%   {\hologoLogoSetup{BibTeX}{variant=sf}\hologo{BibTeX}}
%   with variant \xoption{sf}.
% \entry{rm}
%   This font is a serif font. It is used for \hologo{ExTeX}.
% \entry{sc}
%   This font specifies a small caps font. It is used for
%   {\hologoLogoSetup{BibTeX}{variant=sc}\hologo{BibTeX}}
%   with variant \xoption{sc}.
% \entry{sf}
%   This font specifies a sans serif font. The default
%   is \cs{sffamily}, then \cs{sf} is tried. Otherwise
%   a warning is given. It is used by \hologo{KOMAScript}.
% \entry{sy}
%   This is the font for math symbols (e.g. cmsy).
%   It is used by \hologo{AmS}, \hologo{NTS}, \hologo{ExTeX}.
% \entry{logo}
%   \hologo{METAFONT} and \hologo{METAPOST} are using that font.
%   In \hologo{LaTeX} \cs{logofamily} is used and
%   the definitions of package \xpackage{mflogo} are used
%   if the package is not loaded.
%   Otherwise the \cs{tenlogo} is used and defined
%   if it does not already exists.
% \end{description}
%
% \begin{declcs}{hologoLogoFontSetup} \M{logo} \M{key value list}
% \end{declcs}
% Fonts can also be set for a logo or logo component separately,
% see the following list.
% The keys are the same as for \cs{hologoFontSetup}.
%
% \begin{longtable}{>{\ttfamily}l>{\sffamily}ll}
%   \meta{logo} & keys & result\\
%   \hline
%   \endhead
%   BibTeX & bibsf & {\hologoLogoSetup{BibTeX}{variant=sf}\hologo{BibTeX}}\\[.5ex]
%   BibTeX & sc & {\hologoLogoSetup{BibTeX}{variant=sc}\hologo{BibTeX}}\\[.5ex]
%   ExTeX & rm & \hologo{ExTeX}\\
%   SliTeX & rm & \hologo{SliTeX}\\[.5ex]
%   AmS & sy & \hologo{AmS}\\
%   ExTeX & sy & \hologo{ExTeX}\\
%   NTS & sy & \hologo{NTS}\\[.5ex]
%   KOMAScript & sf & \hologo{KOMAScript}\\[.5ex]
%   METAFONT & logo & \hologo{METAFONT}\\
%   METAPOST & logo & \hologo{METAPOST}\\[.5ex]
%   SliTeX & sc \hologo{SliTeX}
% \end{longtable}
%
% \subsubsection{Font order}
%
% For all logos the font \xoption{general} is applied first.
% Example:
%\begin{quote}
%|\hologoFontSetup{general=\color{red}}|
%\end{quote}
% will print red logos.
% Then if the font uses a special font \xoption{sf}, for example,
% the font is applied that is setup by \cs{hologoLogoFontSetup}.
% If this font is not setup, then the common font setup
% by \cs{hologoFontSetup} is used. Otherwise a warning is given,
% that there is no font configured.
%
% \subsection{Additional user macros}
%
% Usually a variant of a logo is configured by using
% \cs{hologoLogoSetup}, because it is bad style to mix
% different variants of the same logo in the same text.
% There the following macros are a convenience for testing.
%
% \begin{declcs}{hologoVariant} \M{name} \M{variant}\\
%   \cs{HologoVariant} \M{name} \M{variant}
% \end{declcs}
% Logo \meta{name} is set using \meta{variant} that specifies
% explicitely which variant of the macro is used. If the argument
% is empty, then the default form of the logo is used
% (configurable by \cs{hologoLogoSetup}).
%
% \cs{HologoVariant} is used if the logo is set in a context
% that needs an uppercase first letter (beginning of a sentence, \dots).
%
% \begin{declcs}{hologoList}\\
%   \cs{hologoEntry} \M{logo} \M{variant} \M{since}
% \end{declcs}
% Macro \cs{hologoList} contains all logos that are provided
% by the package including variants. The list consists of calls
% of \cs{hologoEntry} with three arguments starting with the
% logo name \meta{logo} and its variant \meta{variant}. An empty
% variant means the current default. Argument \meta{since} specifies
% with version of the package \xpackage{hologo} is needed to get
% the logo. If the logo is fixed, then the date gets updated.
% Therefore the date \meta{since} is not exactly the date of
% the first introduction, but rather the date of the latest fix.
%
% Before \cs{hologoList} can be used, macro \cs{hologoEntry} needs
% a definition. The example file in section \ref{sec:example}
% shows applications of \cs{hologoList}.
%
% \subsection{Supported contexts}
%
% Macros \cs{hologo} and friends support special contexts:
% \begin{itemize}
% \item \hologo{LaTeX}'s protection mechanism.
% \item Bookmarks of package \xpackage{hyperref}.
% \item Package \xpackage{tex4ht}.
% \item The macros can be used inside \cs{csname} constructs,
%   if \cs{ifincsname} is available (\hologo{pdfTeX}, \hologo{XeTeX},
%   \hologo{LuaTeX}).
% \end{itemize}
%
% \subsection{Example}
% \label{sec:example}
%
% The following example prints the logos in different fonts.
%    \begin{macrocode}
%<*example>
%<<verbatim
\NeedsTeXFormat{LaTeX2e}
\documentclass[a4paper]{article}
\usepackage[
  hmargin=20mm,
  vmargin=20mm,
]{geometry}
\pagestyle{empty}
\usepackage{hologo}[2016/05/12]
\usepackage{longtable}
\usepackage{array}
\setlength{\extrarowheight}{2pt}
\usepackage[T1]{fontenc}
\usepackage{lmodern}
\usepackage{pdflscape}
\usepackage[
  pdfencoding=auto,
]{hyperref}
\hypersetup{
  pdfauthor={Heiko Oberdiek},
  pdftitle={Example for package `hologo'},
  pdfsubject={Logos with fonts lmr, lmss, qtm, qpl, qhv},
}
\usepackage{bookmark}

% Print the logo list on the console

\begingroup
  \typeout{}%
  \typeout{*** Begin of logo list ***}%
  \newcommand*{\hologoEntry}[3]{%
    \typeout{#1 \ifx\\#2\\\else(#2) \fi[#3]}%
  }%
  \hologoList
  \typeout{*** End of logo list ***}%
  \typeout{}%
\endgroup

\begin{document}
\begin{landscape}

  \section{Example file for package `hologo'}

  % Table for font names

  \begin{longtable}{>{\bfseries}ll}
    \textbf{font} & \textbf{Font name}\\
    \hline
    lmr & Latin Modern Roman\\
    lmss & Latin Modern Sans\\
    qtm & \TeX\ Gyre Termes\\
    qhv & \TeX\ Gyre Heros\\
    qpl & \TeX\ Gyre Pagella\\
  \end{longtable}

  % Logo list with logos in different fonts

  \begingroup
    \newcommand*{\SetVariant}[2]{%
      \ifx\\#2\\%
      \else
        \hologoLogoSetup{#1}{variant=#2}%
      \fi
    }%
    \newcommand*{\hologoEntry}[3]{%
      \SetVariant{#1}{#2}%
      \raisebox{1em}[0pt][0pt]{\hypertarget{#1@#2}{}}%
      \bookmark[%
        dest={#1@#2},%
      ]{%
        #1\ifx\\#2\\\else\space(#2)\fi: \Hologo{#1}, \hologo{#1} %
        [Unicode]%
      }%
      \hypersetup{unicode=false}%
      \bookmark[%
        dest={#1@#2},%
      ]{%
        #1\ifx\\#2\\\else\space(#2)\fi: \Hologo{#1}, \hologo{#1} %
        [PDFDocEncoding]%
      }%
      \texttt{#1}%
      &%
      \texttt{#2}%
      &%
      \Hologo{#1}%
      &%
      \SetVariant{#1}{#2}%
      \hologo{#1}%
      &%
      \SetVariant{#1}{#2}%
      \fontfamily{qtm}\selectfont
      \hologo{#1}%
      &%
      \SetVariant{#1}{#2}%
      \fontfamily{qpl}\selectfont
      \hologo{#1}%
      &%
      \SetVariant{#1}{#2}%
      \textsf{\hologo{#1}}%
      &%
      \SetVariant{#1}{#2}%
      \fontfamily{qhv}\selectfont
      \hologo{#1}%
      \tabularnewline
    }%
    \begin{longtable}{llllllll}%
      \textbf{\textit{logo}} & \textbf{\textit{variant}} &
      \texttt{\string\Hologo} &
      \textbf{lmr} & \textbf{qtm} & \textbf{qpl} &
      \textbf{lmss} & \textbf{qhv}
      \tabularnewline
      \hline
      \endhead
      \hologoList
    \end{longtable}%
  \endgroup

\end{landscape}
\end{document}
%verbatim
%</example>
%    \end{macrocode}
%
% \StopEventually{
% }
%
% \section{Implementation}
%    \begin{macrocode}
%<*package>
%    \end{macrocode}
%    Reload check, especially if the package is not used with \LaTeX.
%    \begin{macrocode}
\begingroup\catcode61\catcode48\catcode32=10\relax%
  \catcode13=5 % ^^M
  \endlinechar=13 %
  \catcode35=6 % #
  \catcode39=12 % '
  \catcode44=12 % ,
  \catcode45=12 % -
  \catcode46=12 % .
  \catcode58=12 % :
  \catcode64=11 % @
  \catcode123=1 % {
  \catcode125=2 % }
  \expandafter\let\expandafter\x\csname ver@hologo.sty\endcsname
  \ifx\x\relax % plain-TeX, first loading
  \else
    \def\empty{}%
    \ifx\x\empty % LaTeX, first loading,
      % variable is initialized, but \ProvidesPackage not yet seen
    \else
      \expandafter\ifx\csname PackageInfo\endcsname\relax
        \def\x#1#2{%
          \immediate\write-1{Package #1 Info: #2.}%
        }%
      \else
        \def\x#1#2{\PackageInfo{#1}{#2, stopped}}%
      \fi
      \x{hologo}{The package is already loaded}%
      \aftergroup\endinput
    \fi
  \fi
\endgroup%
%    \end{macrocode}
%    Package identification:
%    \begin{macrocode}
\begingroup\catcode61\catcode48\catcode32=10\relax%
  \catcode13=5 % ^^M
  \endlinechar=13 %
  \catcode35=6 % #
  \catcode39=12 % '
  \catcode40=12 % (
  \catcode41=12 % )
  \catcode44=12 % ,
  \catcode45=12 % -
  \catcode46=12 % .
  \catcode47=12 % /
  \catcode58=12 % :
  \catcode64=11 % @
  \catcode91=12 % [
  \catcode93=12 % ]
  \catcode123=1 % {
  \catcode125=2 % }
  \expandafter\ifx\csname ProvidesPackage\endcsname\relax
    \def\x#1#2#3[#4]{\endgroup
      \immediate\write-1{Package: #3 #4}%
      \xdef#1{#4}%
    }%
  \else
    \def\x#1#2[#3]{\endgroup
      #2[{#3}]%
      \ifx#1\@undefined
        \xdef#1{#3}%
      \fi
      \ifx#1\relax
        \xdef#1{#3}%
      \fi
    }%
  \fi
\expandafter\x\csname ver@hologo.sty\endcsname
\ProvidesPackage{hologo}%
  [2016/05/12 v1.11 A logo collection with bookmark support (HO)]%
%    \end{macrocode}
%
%    \begin{macrocode}
\begingroup\catcode61\catcode48\catcode32=10\relax%
  \catcode13=5 % ^^M
  \endlinechar=13 %
  \catcode123=1 % {
  \catcode125=2 % }
  \catcode64=11 % @
  \def\x{\endgroup
    \expandafter\edef\csname HOLOGO@AtEnd\endcsname{%
      \endlinechar=\the\endlinechar\relax
      \catcode13=\the\catcode13\relax
      \catcode32=\the\catcode32\relax
      \catcode35=\the\catcode35\relax
      \catcode61=\the\catcode61\relax
      \catcode64=\the\catcode64\relax
      \catcode123=\the\catcode123\relax
      \catcode125=\the\catcode125\relax
    }%
  }%
\x\catcode61\catcode48\catcode32=10\relax%
\catcode13=5 % ^^M
\endlinechar=13 %
\catcode35=6 % #
\catcode64=11 % @
\catcode123=1 % {
\catcode125=2 % }
\def\TMP@EnsureCode#1#2{%
  \edef\HOLOGO@AtEnd{%
    \HOLOGO@AtEnd
    \catcode#1=\the\catcode#1\relax
  }%
  \catcode#1=#2\relax
}
\TMP@EnsureCode{10}{12}% ^^J
\TMP@EnsureCode{33}{12}% !
\TMP@EnsureCode{34}{12}% "
\TMP@EnsureCode{36}{3}% $
\TMP@EnsureCode{38}{4}% &
\TMP@EnsureCode{39}{12}% '
\TMP@EnsureCode{40}{12}% (
\TMP@EnsureCode{41}{12}% )
\TMP@EnsureCode{42}{12}% *
\TMP@EnsureCode{43}{12}% +
\TMP@EnsureCode{44}{12}% ,
\TMP@EnsureCode{45}{12}% -
\TMP@EnsureCode{46}{12}% .
\TMP@EnsureCode{47}{12}% /
\TMP@EnsureCode{58}{12}% :
\TMP@EnsureCode{59}{12}% ;
\TMP@EnsureCode{60}{12}% <
\TMP@EnsureCode{62}{12}% >
\TMP@EnsureCode{63}{12}% ?
\TMP@EnsureCode{91}{12}% [
\TMP@EnsureCode{93}{12}% ]
\TMP@EnsureCode{94}{7}% ^ (superscript)
\TMP@EnsureCode{95}{8}% _ (subscript)
\TMP@EnsureCode{96}{12}% `
\TMP@EnsureCode{124}{12}% |
\edef\HOLOGO@AtEnd{%
  \HOLOGO@AtEnd
  \escapechar\the\escapechar\relax
  \noexpand\endinput
}
\escapechar=92 %
%    \end{macrocode}
%
% \subsection{Logo list}
%
%    \begin{macro}{\hologoList}
%    \begin{macrocode}
\def\hologoList{%
  \hologoEntry{(La)TeX}{}{2011/10/01}%
  \hologoEntry{AmSLaTeX}{}{2010/04/16}%
  \hologoEntry{AmSTeX}{}{2010/04/16}%
  \hologoEntry{biber}{}{2011/10/01}%
  \hologoEntry{BibTeX}{}{2011/10/01}%
  \hologoEntry{BibTeX}{sf}{2011/10/01}%
  \hologoEntry{BibTeX}{sc}{2011/10/01}%
  \hologoEntry{BibTeX8}{}{2011/11/22}%
  \hologoEntry{ConTeXt}{}{2011/03/25}%
  \hologoEntry{ConTeXt}{narrow}{2011/03/25}%
  \hologoEntry{ConTeXt}{simple}{2011/03/25}%
  \hologoEntry{emTeX}{}{2010/04/26}%
  \hologoEntry{eTeX}{}{2010/04/08}%
  \hologoEntry{ExTeX}{}{2011/10/01}%
  \hologoEntry{HanTheThanh}{}{2011/11/29}%
  \hologoEntry{iniTeX}{}{2011/10/01}%
  \hologoEntry{KOMAScript}{}{2011/10/01}%
  \hologoEntry{La}{}{2010/05/08}%
  \hologoEntry{LaTeX}{}{2010/04/08}%
  \hologoEntry{LaTeX2e}{}{2010/04/08}%
  \hologoEntry{LaTeX3}{}{2010/04/24}%
  \hologoEntry{LaTeXe}{}{2010/04/08}%
  \hologoEntry{LaTeXML}{}{2011/11/22}%
  \hologoEntry{LaTeXTeX}{}{2011/10/01}%
  \hologoEntry{LuaLaTeX}{}{2010/04/08}%
  \hologoEntry{LuaTeX}{}{2010/04/08}%
  \hologoEntry{LyX}{}{2011/10/01}%
  \hologoEntry{METAFONT}{}{2011/10/01}%
  \hologoEntry{MetaFun}{}{2011/10/01}%
  \hologoEntry{METAPOST}{}{2011/10/01}%
  \hologoEntry{MetaPost}{}{2011/10/01}%
  \hologoEntry{MiKTeX}{}{2011/10/01}%
  \hologoEntry{NTS}{}{2011/10/01}%
  \hologoEntry{OzMF}{}{2011/10/01}%
  \hologoEntry{OzMP}{}{2011/10/01}%
  \hologoEntry{OzTeX}{}{2011/10/01}%
  \hologoEntry{OzTtH}{}{2011/10/01}%
  \hologoEntry{PCTeX}{}{2011/10/01}%
  \hologoEntry{pdfTeX}{}{2011/10/01}%
  \hologoEntry{pdfLaTeX}{}{2011/10/01}%
  \hologoEntry{PiC}{}{2011/10/01}%
  \hologoEntry{PiCTeX}{}{2011/10/01}%
  \hologoEntry{plainTeX}{}{2010/04/08}%
  \hologoEntry{plainTeX}{space}{2010/04/16}%
  \hologoEntry{plainTeX}{hyphen}{2010/04/16}%
  \hologoEntry{plainTeX}{runtogether}{2010/04/16}%
  \hologoEntry{SageTeX}{}{2011/11/22}%
  \hologoEntry{SLiTeX}{}{2011/10/01}%
  \hologoEntry{SLiTeX}{lift}{2011/10/01}%
  \hologoEntry{SLiTeX}{narrow}{2011/10/01}%
  \hologoEntry{SLiTeX}{simple}{2011/10/01}%
  \hologoEntry{SliTeX}{}{2011/10/01}%
  \hologoEntry{SliTeX}{narrow}{2011/10/01}%
  \hologoEntry{SliTeX}{simple}{2011/10/01}%
  \hologoEntry{SliTeX}{lift}{2011/10/01}%
  \hologoEntry{teTeX}{}{2011/10/01}%
  \hologoEntry{TeX}{}{2010/04/08}%
  \hologoEntry{TeX4ht}{}{2011/11/22}%
  \hologoEntry{TTH}{}{2011/11/22}%
  \hologoEntry{virTeX}{}{2011/10/01}%
  \hologoEntry{VTeX}{}{2010/04/24}%
  \hologoEntry{Xe}{}{2010/04/08}%
  \hologoEntry{XeLaTeX}{}{2010/04/08}%
  \hologoEntry{XeTeX}{}{2010/04/08}%
}
%    \end{macrocode}
%    \end{macro}
%
% \subsection{Load resources}
%
%    \begin{macrocode}
\begingroup\expandafter\expandafter\expandafter\endgroup
\expandafter\ifx\csname RequirePackage\endcsname\relax
  \def\TMP@RequirePackage#1[#2]{%
    \begingroup\expandafter\expandafter\expandafter\endgroup
    \expandafter\ifx\csname ver@#1.sty\endcsname\relax
      \input #1.sty\relax
    \fi
  }%
  \TMP@RequirePackage{ltxcmds}[2011/02/04]%
  \TMP@RequirePackage{infwarerr}[2010/04/08]%
  \TMP@RequirePackage{kvsetkeys}[2010/03/01]%
  \TMP@RequirePackage{kvdefinekeys}[2010/03/01]%
  \TMP@RequirePackage{pdftexcmds}[2010/04/01]%
  \TMP@RequirePackage{ifpdf}[2010/01/28]%
  \TMP@RequirePackage{ifluatex}[2010/03/01]%
  \ltx@IfUndefined{newif}{%
    \expandafter\let\csname newif\endcsname\ltx@newif
  }{}%
  \TMP@RequirePackage{ifxetex}[2009/01/23]%
  \TMP@RequirePackage{ifvtex}[2010/03/01]%
\else
  \RequirePackage{ltxcmds}[2011/02/04]%
  \RequirePackage{infwarerr}[2010/04/08]%
  \RequirePackage{kvsetkeys}[2010/03/01]%
  \RequirePackage{kvdefinekeys}[2010/03/01]%
  \RequirePackage{pdftexcmds}[2010/04/01]%
  \RequirePackage{ifpdf}[2010/01/28]%
  \RequirePackage{ifluatex}[2010/03/01]%
  \RequirePackage{ifxetex}[2009/01/23]%
  \RequirePackage{ifvtex}[2010/03/01]%
\fi
%    \end{macrocode}
%
%    \begin{macro}{\HOLOGO@IfDefined}
%    \begin{macrocode}
\def\HOLOGO@IfExists#1{%
  \ifx\@undefined#1%
    \expandafter\ltx@secondoftwo
  \else
    \ifx\relax#1%
      \expandafter\ltx@secondoftwo
    \else
      \expandafter\expandafter\expandafter\ltx@firstoftwo
    \fi
  \fi
}
%    \end{macrocode}
%    \end{macro}
%
% \subsection{Setup macros}
%
%    \begin{macro}{\hologoSetup}
%    \begin{macrocode}
\def\hologoSetup{%
  \let\HOLOGO@name\relax
  \HOLOGO@Setup
}
%    \end{macrocode}
%    \end{macro}
%
%    \begin{macro}{\hologoLogoSetup}
%    \begin{macrocode}
\def\hologoLogoSetup#1{%
  \edef\HOLOGO@name{#1}%
  \ltx@IfUndefined{HoLogo@\HOLOGO@name}{%
    \@PackageError{hologo}{%
      Unknown logo `\HOLOGO@name'%
    }\@ehc
    \ltx@gobble
  }{%
    \HOLOGO@Setup
  }%
}
%    \end{macrocode}
%    \end{macro}
%
%    \begin{macro}{\HOLOGO@Setup}
%    \begin{macrocode}
\def\HOLOGO@Setup{%
  \kvsetkeys{HoLogo}%
}
%    \end{macrocode}
%    \end{macro}
%
% \subsection{Options}
%
%    \begin{macro}{\HOLOGO@DeclareBoolOption}
%    \begin{macrocode}
\def\HOLOGO@DeclareBoolOption#1{%
  \expandafter\chardef\csname HOLOGOOPT@#1\endcsname\ltx@zero
  \kv@define@key{HoLogo}{#1}[true]{%
    \def\HOLOGO@temp{##1}%
    \ifx\HOLOGO@temp\HOLOGO@true
      \ifx\HOLOGO@name\relax
        \expandafter\chardef\csname HOLOGOOPT@#1\endcsname=\ltx@one
      \else
        \expandafter\chardef\csname
        HoLogoOpt@#1@\HOLOGO@name\endcsname\ltx@one
      \fi
      \HOLOGO@SetBreakAll{#1}%
    \else
      \ifx\HOLOGO@temp\HOLOGO@false
        \ifx\HOLOGO@name\relax
          \expandafter\chardef\csname HOLOGOOPT@#1\endcsname=\ltx@zero
        \else
          \expandafter\chardef\csname
          HoLogoOpt@#1@\HOLOGO@name\endcsname=\ltx@zero
        \fi
        \HOLOGO@SetBreakAll{#1}%
      \else
        \@PackageError{hologo}{%
          Unknown value `##1' for boolean option `#1'.\MessageBreak
          Known values are `true' and `false'%
        }\@ehc
      \fi
    \fi
  }%
}
%    \end{macrocode}
%    \end{macro}
%
%    \begin{macro}{\HOLOGO@SetBreakAll}
%    \begin{macrocode}
\def\HOLOGO@SetBreakAll#1{%
  \def\HOLOGO@temp{#1}%
  \ifx\HOLOGO@temp\HOLOGO@break
    \ifx\HOLOGO@name\relax
      \chardef\HOLOGOOPT@hyphenbreak=\HOLOGOOPT@break
      \chardef\HOLOGOOPT@spacebreak=\HOLOGOOPT@break
      \chardef\HOLOGOOPT@discretionarybreak=\HOLOGOOPT@break
    \else
      \expandafter\chardef
         \csname HoLogoOpt@hyphenbreak@\HOLOGO@name\endcsname=%
         \csname HoLogoOpt@break@\HOLOGO@name\endcsname
      \expandafter\chardef
         \csname HoLogoOpt@spacebreak@\HOLOGO@name\endcsname=%
         \csname HoLogoOpt@break@\HOLOGO@name\endcsname
      \expandafter\chardef
         \csname HoLogoOpt@discretionarybreak@\HOLOGO@name
             \endcsname=%
         \csname HoLogoOpt@break@\HOLOGO@name\endcsname
    \fi
  \fi
}
%    \end{macrocode}
%    \end{macro}
%
%    \begin{macro}{\HOLOGO@true}
%    \begin{macrocode}
\def\HOLOGO@true{true}
%    \end{macrocode}
%    \end{macro}
%    \begin{macro}{\HOLOGO@false}
%    \begin{macrocode}
\def\HOLOGO@false{false}
%    \end{macrocode}
%    \end{macro}
%    \begin{macro}{\HOLOGO@break}
%    \begin{macrocode}
\def\HOLOGO@break{break}
%    \end{macrocode}
%    \end{macro}
%
%    \begin{macrocode}
\HOLOGO@DeclareBoolOption{break}
\HOLOGO@DeclareBoolOption{hyphenbreak}
\HOLOGO@DeclareBoolOption{spacebreak}
\HOLOGO@DeclareBoolOption{discretionarybreak}
%    \end{macrocode}
%
%    \begin{macrocode}
\kv@define@key{HoLogo}{variant}{%
  \ifx\HOLOGO@name\relax
    \@PackageError{hologo}{%
      Option `variant' is not available in \string\hologoSetup,%
      \MessageBreak
      Use \string\hologoLogoSetup\space instead%
    }\@ehc
  \else
    \edef\HOLOGO@temp{#1}%
    \ifx\HOLOGO@temp\ltx@empty
      \expandafter
      \let\csname HoLogoOpt@variant@\HOLOGO@name\endcsname\@undefined
    \else
      \ltx@IfUndefined{HoLogo@\HOLOGO@name @\HOLOGO@temp}{%
        \@PackageError{hologo}{%
          Unknown variant `\HOLOGO@temp' of logo `\HOLOGO@name'%
        }\@ehc
      }{%
        \expandafter
        \let\csname HoLogoOpt@variant@\HOLOGO@name\endcsname
            \HOLOGO@temp
      }%
    \fi
  \fi
}
%    \end{macrocode}
%
%    \begin{macro}{\HOLOGO@Variant}
%    \begin{macrocode}
\def\HOLOGO@Variant#1{%
  #1%
  \ltx@ifundefined{HoLogoOpt@variant@#1}{%
  }{%
    @\csname HoLogoOpt@variant@#1\endcsname
  }%
}
%    \end{macrocode}
%    \end{macro}
%
% \subsection{Break/no-break support}
%
%    \begin{macro}{\HOLOGO@space}
%    \begin{macrocode}
\def\HOLOGO@space{%
  \ltx@ifundefined{HoLogoOpt@spacebreak@\HOLOGO@name}{%
    \ltx@ifundefined{HoLogoOpt@break@\HOLOGO@name}{%
      \chardef\HOLOGO@temp=\HOLOGOOPT@spacebreak
    }{%
      \chardef\HOLOGO@temp=%
        \csname HoLogoOpt@break@\HOLOGO@name\endcsname
    }%
  }{%
    \chardef\HOLOGO@temp=%
      \csname HoLogoOpt@spacebreak@\HOLOGO@name\endcsname
  }%
  \ifcase\HOLOGO@temp
    \penalty10000 %
  \fi
  \ltx@space
}
%    \end{macrocode}
%    \end{macro}
%
%    \begin{macro}{\HOLOGO@hyphen}
%    \begin{macrocode}
\def\HOLOGO@hyphen{%
  \ltx@ifundefined{HoLogoOpt@hyphenbreak@\HOLOGO@name}{%
    \ltx@ifundefined{HoLogoOpt@break@\HOLOGO@name}{%
      \chardef\HOLOGO@temp=\HOLOGOOPT@hyphenbreak
    }{%
      \chardef\HOLOGO@temp=%
        \csname HoLogoOpt@break@\HOLOGO@name\endcsname
    }%
  }{%
    \chardef\HOLOGO@temp=%
      \csname HoLogoOpt@hyphenbreak@\HOLOGO@name\endcsname
  }%
  \ifcase\HOLOGO@temp
    \ltx@mbox{-}%
  \else
    -%
  \fi
}
%    \end{macrocode}
%    \end{macro}
%
%    \begin{macro}{\HOLOGO@discretionary}
%    \begin{macrocode}
\def\HOLOGO@discretionary{%
  \ltx@ifundefined{HoLogoOpt@discretionarybreak@\HOLOGO@name}{%
    \ltx@ifundefined{HoLogoOpt@break@\HOLOGO@name}{%
      \chardef\HOLOGO@temp=\HOLOGOOPT@discretionarybreak
    }{%
      \chardef\HOLOGO@temp=%
        \csname HoLogoOpt@break@\HOLOGO@name\endcsname
    }%
  }{%
    \chardef\HOLOGO@temp=%
      \csname HoLogoOpt@discretionarybreak@\HOLOGO@name\endcsname
  }%
  \ifcase\HOLOGO@temp
  \else
    \-%
  \fi
}
%    \end{macrocode}
%    \end{macro}
%
%    \begin{macro}{\HOLOGO@mbox}
%    \begin{macrocode}
\def\HOLOGO@mbox#1{%
  \ltx@ifundefined{HoLogoOpt@break@\HOLOGO@name}{%
    \chardef\HOLOGO@temp=\HOLOGOOPT@hyphenbreak
  }{%
    \chardef\HOLOGO@temp=%
      \csname HoLogoOpt@break@\HOLOGO@name\endcsname
  }%
  \ifcase\HOLOGO@temp
    \ltx@mbox{#1}%
  \else
    #1%
  \fi
}
%    \end{macrocode}
%    \end{macro}
%
% \subsection{Font support}
%
%    \begin{macro}{\HoLogoFont@font}
%    \begin{tabular}{@{}ll@{}}
%    |#1|:& logo name\\
%    |#2|:& font short name\\
%    |#3|:& text
%    \end{tabular}
%    \begin{macrocode}
\def\HoLogoFont@font#1#2#3{%
  \begingroup
    \ltx@IfUndefined{HoLogoFont@logo@#1.#2}{%
      \ltx@IfUndefined{HoLogoFont@font@#2}{%
        \@PackageWarning{hologo}{%
          Missing font `#2' for logo `#1'%
        }%
        #3%
      }{%
        \csname HoLogoFont@font@#2\endcsname{#3}%
      }%
    }{%
      \csname HoLogoFont@logo@#1.#2\endcsname{#3}%
    }%
  \endgroup
}
%    \end{macrocode}
%    \end{macro}
%
%    \begin{macro}{\HoLogoFont@Def}
%    \begin{macrocode}
\def\HoLogoFont@Def#1{%
  \expandafter\def\csname HoLogoFont@font@#1\endcsname
}
%    \end{macrocode}
%    \end{macro}
%    \begin{macro}{\HoLogoFont@LogoDef}
%    \begin{macrocode}
\def\HoLogoFont@LogoDef#1#2{%
  \expandafter\def\csname HoLogoFont@logo@#1.#2\endcsname
}
%    \end{macrocode}
%    \end{macro}
%
% \subsubsection{Font defaults}
%
%    \begin{macro}{\HoLogoFont@font@general}
%    \begin{macrocode}
\HoLogoFont@Def{general}{}%
%    \end{macrocode}
%    \end{macro}
%
%    \begin{macro}{\HoLogoFont@font@rm}
%    \begin{macrocode}
\ltx@IfUndefined{rmfamily}{%
  \ltx@IfUndefined{rm}{%
  }{%
    \HoLogoFont@Def{rm}{\rm}%
  }%
}{%
  \HoLogoFont@Def{rm}{\rmfamily}%
}
%    \end{macrocode}
%    \end{macro}
%
%    \begin{macro}{\HoLogoFont@font@sf}
%    \begin{macrocode}
\ltx@IfUndefined{sffamily}{%
  \ltx@IfUndefined{sf}{%
  }{%
    \HoLogoFont@Def{sf}{\sf}%
  }%
}{%
  \HoLogoFont@Def{sf}{\sffamily}%
}
%    \end{macrocode}
%    \end{macro}
%
%    \begin{macro}{\HoLogoFont@font@bibsf}
%    In case of \hologo{plainTeX} the original small caps
%    variant is used as default. In \hologo{LaTeX}
%    the definition of package \xpackage{dtklogos} \cite{dtklogos}
%    is used.
%\begin{quote}
%\begin{verbatim}
%\DeclareRobustCommand{\BibTeX}{%
%  B%
%  \kern-.05em%
%  \hbox{%
%    $\m@th$% %% force math size calculations
%    \csname S@\f@size\endcsname
%    \fontsize\sf@size\z@
%    \math@fontsfalse
%    \selectfont
%    I%
%    \kern-.025em%
%    B
%  }%
%  \kern-.08em%
%  \-%
%  \TeX
%}
%\end{verbatim}
%\end{quote}
%    \begin{macrocode}
\ltx@IfUndefined{selectfont}{%
  \ltx@IfUndefined{tensc}{%
    \font\tensc=cmcsc10\relax
  }{}%
  \HoLogoFont@Def{bibsf}{\tensc}%
}{%
  \HoLogoFont@Def{bibsf}{%
    $\mathsurround=0pt$%
    \csname S@\f@size\endcsname
    \fontsize\sf@size{0pt}%
    \math@fontsfalse
    \selectfont
  }%
}
%    \end{macrocode}
%    \end{macro}
%
%    \begin{macro}{\HoLogoFont@font@sc}
%    \begin{macrocode}
\ltx@IfUndefined{scshape}{%
  \ltx@IfUndefined{tensc}{%
    \font\tensc=cmcsc10\relax
  }{}%
  \HoLogoFont@Def{sc}{\tensc}%
}{%
  \HoLogoFont@Def{sc}{\scshape}%
}
%    \end{macrocode}
%    \end{macro}
%
%    \begin{macro}{\HoLogoFont@font@sy}
%    \begin{macrocode}
\ltx@IfUndefined{usefont}{%
  \ltx@IfUndefined{tensy}{%
  }{%
    \HoLogoFont@Def{sy}{\tensy}%
  }%
}{%
  \HoLogoFont@Def{sy}{%
    \usefont{OMS}{cmsy}{m}{n}%
  }%
}
%    \end{macrocode}
%    \end{macro}
%
%    \begin{macro}{\HoLogoFont@font@logo}
%    \begin{macrocode}
\begingroup
  \def\x{LaTeX2e}%
\expandafter\endgroup
\ifx\fmtname\x
  \ltx@IfUndefined{logofamily}{%
    \DeclareRobustCommand\logofamily{%
      \not@math@alphabet\logofamily\relax
      \fontencoding{U}%
      \fontfamily{logo}%
      \selectfont
    }%
  }{}%
  \ltx@IfUndefined{logofamily}{%
  }{%
    \HoLogoFont@Def{logo}{\logofamily}%
  }%
\else
  \ltx@IfUndefined{tenlogo}{%
    \font\tenlogo=logo10\relax
  }{}%
  \HoLogoFont@Def{logo}{\tenlogo}%
\fi
%    \end{macrocode}
%    \end{macro}
%
% \subsubsection{Font setup}
%
%    \begin{macro}{\hologoFontSetup}
%    \begin{macrocode}
\def\hologoFontSetup{%
  \let\HOLOGO@name\relax
  \HOLOGO@FontSetup
}
%    \end{macrocode}
%    \end{macro}
%
%    \begin{macro}{\hologoLogoFontSetup}
%    \begin{macrocode}
\def\hologoLogoFontSetup#1{%
  \edef\HOLOGO@name{#1}%
  \ltx@IfUndefined{HoLogo@\HOLOGO@name}{%
    \@PackageError{hologo}{%
      Unknown logo `\HOLOGO@name'%
    }\@ehc
    \ltx@gobble
  }{%
    \HOLOGO@FontSetup
  }%
}
%    \end{macrocode}
%    \end{macro}
%
%    \begin{macro}{\HOLOGO@FontSetup}
%    \begin{macrocode}
\def\HOLOGO@FontSetup{%
  \kvsetkeys{HoLogoFont}%
}
%    \end{macrocode}
%    \end{macro}
%
%    \begin{macrocode}
\def\HOLOGO@temp#1{%
  \kv@define@key{HoLogoFont}{#1}{%
    \ifx\HOLOGO@name\relax
      \HoLogoFont@Def{#1}{##1}%
    \else
      \HoLogoFont@LogoDef\HOLOGO@name{#1}{##1}%
    \fi
  }%
}
\HOLOGO@temp{general}
\HOLOGO@temp{sf}
%    \end{macrocode}
%
% \subsection{Generic logo commands}
%
%    \begin{macrocode}
\HOLOGO@IfExists\hologo{%
  \@PackageError{hologo}{%
    \string\hologo\ltx@space is already defined.\MessageBreak
    Package loading is aborted%
  }\@ehc
  \HOLOGO@AtEnd
}%
\HOLOGO@IfExists\hologoRobust{%
  \@PackageError{hologo}{%
    \string\hologoRobust\ltx@space is already defined.\MessageBreak
    Package loading is aborted%
  }\@ehc
  \HOLOGO@AtEnd
}%
%    \end{macrocode}
%
% \subsubsection{\cs{hologo} and friends}
%
%    \begin{macrocode}
\ifluatex
  \expandafter\ltx@firstofone
\else
  \expandafter\ltx@gobble
\fi
{%
  \ltx@IfUndefined{ifincsname}{%
    \ifnum\luatexversion<36 %
      \expandafter\ltx@gobble
    \else
      \expandafter\ltx@firstofone
    \fi
    {%
      \begingroup
        \ifcase0%
            \directlua{%
              if tex.enableprimitives then %
                tex.enableprimitives('HOLOGO@', {'ifincsname'})%
              else %
                tex.print('1')%
              end%
            }%
            \ifx\HOLOGO@ifincsname\@undefined 1\fi%
            \relax
          \expandafter\ltx@firstofone
        \else
          \endgroup
          \expandafter\ltx@gobble
        \fi
        {%
          \global\let\ifincsname\HOLOGO@ifincsname
        }%
      \HOLOGO@temp
    }%
  }{}%
}
%    \end{macrocode}
%    \begin{macrocode}
\ltx@IfUndefined{ifincsname}{%
  \catcode`$=14 %
}{%
  \catcode`$=9 %
}
%    \end{macrocode}
%
%    \begin{macro}{\hologo}
%    \begin{macrocode}
\def\hologo#1{%
$ \ifincsname
$   \ltx@ifundefined{HoLogoCs@\HOLOGO@Variant{#1}}{%
$     #1%
$   }{%
$     \csname HoLogoCs@\HOLOGO@Variant{#1}\endcsname\ltx@firstoftwo
$   }%
$ \else
    \HOLOGO@IfExists\texorpdfstring\texorpdfstring\ltx@firstoftwo
    {%
      \hologoRobust{#1}%
    }{%
      \ltx@ifundefined{HoLogoBkm@\HOLOGO@Variant{#1}}{%
        \ltx@ifundefined{HoLogo@#1}{?#1?}{#1}%
      }{%
        \csname HoLogoBkm@\HOLOGO@Variant{#1}\endcsname
        \ltx@firstoftwo
      }%
    }%
$ \fi
}
%    \end{macrocode}
%    \end{macro}
%    \begin{macro}{\Hologo}
%    \begin{macrocode}
\def\Hologo#1{%
$ \ifincsname
$   \ltx@ifundefined{HoLogoCs@\HOLOGO@Variant{#1}}{%
$     #1%
$   }{%
$     \csname HoLogoCs@\HOLOGO@Variant{#1}\endcsname\ltx@secondoftwo
$   }%
$ \else
    \HOLOGO@IfExists\texorpdfstring\texorpdfstring\ltx@firstoftwo
    {%
      \HologoRobust{#1}%
    }{%
      \ltx@ifundefined{HoLogoBkm@\HOLOGO@Variant{#1}}{%
        \ltx@ifundefined{HoLogo@#1}{?#1?}{#1}%
      }{%
        \csname HoLogoBkm@\HOLOGO@Variant{#1}\endcsname
        \ltx@secondoftwo
      }%
    }%
$ \fi
}
%    \end{macrocode}
%    \end{macro}
%
%    \begin{macro}{\hologoVariant}
%    \begin{macrocode}
\def\hologoVariant#1#2{%
  \ifx\relax#2\relax
    \hologo{#1}%
  \else
$   \ifincsname
$     \ltx@ifundefined{HoLogoCs@#1@#2}{%
$       #1%
$     }{%
$       \csname HoLogoCs@#1@#2\endcsname\ltx@firstoftwo
$     }%
$   \else
      \HOLOGO@IfExists\texorpdfstring\texorpdfstring\ltx@firstoftwo
      {%
        \hologoVariantRobust{#1}{#2}%
      }{%
        \ltx@ifundefined{HoLogoBkm@#1@#2}{%
          \ltx@ifundefined{HoLogo@#1}{?#1?}{#1}%
        }{%
          \csname HoLogoBkm@#1@#2\endcsname
          \ltx@firstoftwo
        }%
      }%
$   \fi
  \fi
}
%    \end{macrocode}
%    \end{macro}
%    \begin{macro}{\HologoVariant}
%    \begin{macrocode}
\def\HologoVariant#1#2{%
  \ifx\relax#2\relax
    \Hologo{#1}%
  \else
$   \ifincsname
$     \ltx@ifundefined{HoLogoCs@#1@#2}{%
$       #1%
$     }{%
$       \csname HoLogoCs@#1@#2\endcsname\ltx@secondoftwo
$     }%
$   \else
      \HOLOGO@IfExists\texorpdfstring\texorpdfstring\ltx@firstoftwo
      {%
        \HologoVariantRobust{#1}{#2}%
      }{%
        \ltx@ifundefined{HoLogoBkm@#1@#2}{%
          \ltx@ifundefined{HoLogo@#1}{?#1?}{#1}%
        }{%
          \csname HoLogoBkm@#1@#2\endcsname
          \ltx@secondoftwo
        }%
      }%
$   \fi
  \fi
}
%    \end{macrocode}
%    \end{macro}
%
%    \begin{macrocode}
\catcode`\$=3 %
%    \end{macrocode}
%
% \subsubsection{\cs{hologoRobust} and friends}
%
%    \begin{macro}{\hologoRobust}
%    \begin{macrocode}
\ltx@IfUndefined{protected}{%
  \ltx@IfUndefined{DeclareRobustCommand}{%
    \def\hologoRobust#1%
  }{%
    \DeclareRobustCommand*\hologoRobust[1]%
  }%
}{%
  \protected\def\hologoRobust#1%
}%
{%
  \edef\HOLOGO@name{#1}%
  \ltx@IfUndefined{HoLogo@\HOLOGO@Variant\HOLOGO@name}{%
    \@PackageError{hologo}{%
      Unknown logo `\HOLOGO@name'%
    }\@ehc
    ?\HOLOGO@name?%
  }{%
    \ltx@IfUndefined{ver@tex4ht.sty}{%
      \HoLogoFont@font\HOLOGO@name{general}{%
        \csname HoLogo@\HOLOGO@Variant\HOLOGO@name\endcsname
        \ltx@firstoftwo
      }%
    }{%
      \ltx@IfUndefined{HoLogoHtml@\HOLOGO@Variant\HOLOGO@name}{%
        \HOLOGO@name
      }{%
        \csname HoLogoHtml@\HOLOGO@Variant\HOLOGO@name\endcsname
        \ltx@firstoftwo
      }%
    }%
  }%
}
%    \end{macrocode}
%    \end{macro}
%    \begin{macro}{\HologoRobust}
%    \begin{macrocode}
\ltx@IfUndefined{protected}{%
  \ltx@IfUndefined{DeclareRobustCommand}{%
    \def\HologoRobust#1%
  }{%
    \DeclareRobustCommand*\HologoRobust[1]%
  }%
}{%
  \protected\def\HologoRobust#1%
}%
{%
  \edef\HOLOGO@name{#1}%
  \ltx@IfUndefined{HoLogo@\HOLOGO@Variant\HOLOGO@name}{%
    \@PackageError{hologo}{%
      Unknown logo `\HOLOGO@name'%
    }\@ehc
    ?\HOLOGO@name?%
  }{%
    \ltx@IfUndefined{ver@tex4ht.sty}{%
      \HoLogoFont@font\HOLOGO@name{general}{%
        \csname HoLogo@\HOLOGO@Variant\HOLOGO@name\endcsname
        \ltx@secondoftwo
      }%
    }{%
      \ltx@IfUndefined{HoLogoHtml@\HOLOGO@Variant\HOLOGO@name}{%
        \expandafter\HOLOGO@Uppercase\HOLOGO@name
      }{%
        \csname HoLogoHtml@\HOLOGO@Variant\HOLOGO@name\endcsname
        \ltx@secondoftwo
      }%
    }%
  }%
}
%    \end{macrocode}
%    \end{macro}
%    \begin{macro}{\hologoVariantRobust}
%    \begin{macrocode}
\ltx@IfUndefined{protected}{%
  \ltx@IfUndefined{DeclareRobustCommand}{%
    \def\hologoVariantRobust#1#2%
  }{%
    \DeclareRobustCommand*\hologoVariantRobust[2]%
  }%
}{%
  \protected\def\hologoVariantRobust#1#2%
}%
{%
  \begingroup
    \hologoLogoSetup{#1}{variant={#2}}%
    \hologoRobust{#1}%
  \endgroup
}
%    \end{macrocode}
%    \end{macro}
%    \begin{macro}{\HologoVariantRobust}
%    \begin{macrocode}
\ltx@IfUndefined{protected}{%
  \ltx@IfUndefined{DeclareRobustCommand}{%
    \def\HologoVariantRobust#1#2%
  }{%
    \DeclareRobustCommand*\HologoVariantRobust[2]%
  }%
}{%
  \protected\def\HologoVariantRobust#1#2%
}%
{%
  \begingroup
    \hologoLogoSetup{#1}{variant={#2}}%
    \HologoRobust{#1}%
  \endgroup
}
%    \end{macrocode}
%    \end{macro}
%
%    \begin{macro}{\hologorobust}
%    Macro \cs{hologorobust} is only defined for compatibility.
%    Its use is deprecated.
%    \begin{macrocode}
\def\hologorobust{\hologoRobust}
%    \end{macrocode}
%    \end{macro}
%
% \subsection{Helpers}
%
%    \begin{macro}{\HOLOGO@Uppercase}
%    Macro \cs{HOLOGO@Uppercase} is restricted to \cs{uppercase},
%    because \hologo{plainTeX} or \hologo{iniTeX} do not provide
%    \cs{MakeUppercase}.
%    \begin{macrocode}
\def\HOLOGO@Uppercase#1{\uppercase{#1}}
%    \end{macrocode}
%    \end{macro}
%
%    \begin{macro}{\HOLOGO@PdfdocUnicode}
%    \begin{macrocode}
\def\HOLOGO@PdfdocUnicode{%
  \ifx\ifHy@unicode\iftrue
    \expandafter\ltx@secondoftwo
  \else
    \expandafter\ltx@firstoftwo
  \fi
}
%    \end{macrocode}
%    \end{macro}
%
%    \begin{macro}{\HOLOGO@Math}
%    \begin{macrocode}
\def\HOLOGO@MathSetup{%
  \mathsurround0pt\relax
  \HOLOGO@IfExists\f@series{%
    \if b\expandafter\ltx@car\f@series x\@nil
      \csname boldmath\endcsname
   \fi
  }{}%
}
%    \end{macrocode}
%    \end{macro}
%
%    \begin{macro}{\HOLOGO@TempDimen}
%    \begin{macrocode}
\dimendef\HOLOGO@TempDimen=\ltx@zero
%    \end{macrocode}
%    \end{macro}
%    \begin{macro}{\HOLOGO@NegativeKerning}
%    \begin{macrocode}
\def\HOLOGO@NegativeKerning#1{%
  \begingroup
    \HOLOGO@TempDimen=0pt\relax
    \comma@parse@normalized{#1}{%
      \ifdim\HOLOGO@TempDimen=0pt %
        \expandafter\HOLOGO@@NegativeKerning\comma@entry
      \fi
      \ltx@gobble
    }%
    \ifdim\HOLOGO@TempDimen<0pt %
      \kern\HOLOGO@TempDimen
    \fi
  \endgroup
}
%    \end{macrocode}
%    \end{macro}
%    \begin{macro}{\HOLOGO@@NegativeKerning}
%    \begin{macrocode}
\def\HOLOGO@@NegativeKerning#1#2{%
  \setbox\ltx@zero\hbox{#1#2}%
  \HOLOGO@TempDimen=\wd\ltx@zero
  \setbox\ltx@zero\hbox{#1\kern0pt#2}%
  \advance\HOLOGO@TempDimen by -\wd\ltx@zero
}
%    \end{macrocode}
%    \end{macro}
%
%    \begin{macro}{\HOLOGO@SpaceFactor}
%    \begin{macrocode}
\def\HOLOGO@SpaceFactor{%
  \spacefactor1000 %
}
%    \end{macrocode}
%    \end{macro}
%
%    \begin{macro}{\HOLOGO@Span}
%    \begin{macrocode}
\def\HOLOGO@Span#1#2{%
  \HCode{<span class="HoLogo-#1">}%
  #2%
  \HCode{</span>}%
}
%    \end{macrocode}
%    \end{macro}
%
% \subsubsection{Text subscript}
%
%    \begin{macro}{\HOLOGO@SubScript}%
%    \begin{macrocode}
\def\HOLOGO@SubScript#1{%
  \ltx@IfUndefined{textsubscript}{%
    \ltx@IfUndefined{text}{%
      \ltx@mbox{%
        \mathsurround=0pt\relax
        $%
          _{%
            \ltx@IfUndefined{sf@size}{%
              \mathrm{#1}%
            }{%
              \mbox{%
                \fontsize\sf@size{0pt}\selectfont
                #1%
              }%
            }%
          }%
        $%
      }%
    }{%
      \ltx@mbox{%
        \mathsurround=0pt\relax
        $_{\text{#1}}$%
      }%
    }%
  }{%
    \textsubscript{#1}%
  }%
}
%    \end{macrocode}
%    \end{macro}
%
% \subsection{\hologo{TeX} and friends}
%
% \subsubsection{\hologo{TeX}}
%
%    \begin{macro}{\HoLogo@TeX}
%    Source: \hologo{LaTeX} kernel.
%    \begin{macrocode}
\def\HoLogo@TeX#1{%
  T\kern-.1667em\lower.5ex\hbox{E}\kern-.125emX\HOLOGO@SpaceFactor
}
%    \end{macrocode}
%    \end{macro}
%    \begin{macro}{\HoLogoHtml@TeX}
%    \begin{macrocode}
\def\HoLogoHtml@TeX#1{%
  \HoLogoCss@TeX
  \HOLOGO@Span{TeX}{%
    T%
    \HOLOGO@Span{e}{%
      E%
    }%
    X%
  }%
}
%    \end{macrocode}
%    \end{macro}
%    \begin{macro}{\HoLogoCss@TeX}
%    \begin{macrocode}
\def\HoLogoCss@TeX{%
  \Css{%
    span.HoLogo-TeX span.HoLogo-e{%
      position:relative;%
      top:.5ex;%
      margin-left:-.1667em;%
      margin-right:-.125em;%
    }%
  }%
  \Css{%
    a span.HoLogo-TeX span.HoLogo-e{%
      text-decoration:none;%
    }%
  }%
  \global\let\HoLogoCss@TeX\relax
}
%    \end{macrocode}
%    \end{macro}
%
% \subsubsection{\hologo{plainTeX}}
%
%    \begin{macro}{\HoLogo@plainTeX@space}
%    Source: ``The \hologo{TeX}book''
%    \begin{macrocode}
\def\HoLogo@plainTeX@space#1{%
  \HOLOGO@mbox{#1{p}{P}lain}\HOLOGO@space\hologo{TeX}%
}
%    \end{macrocode}
%    \end{macro}
%    \begin{macro}{\HoLogoCs@plainTeX@space}
%    \begin{macrocode}
\def\HoLogoCs@plainTeX@space#1{#1{p}{P}lain TeX}%
%    \end{macrocode}
%    \end{macro}
%    \begin{macro}{\HoLogoBkm@plainTeX@space}
%    \begin{macrocode}
\def\HoLogoBkm@plainTeX@space#1{%
  #1{p}{P}lain \hologo{TeX}%
}
%    \end{macrocode}
%    \end{macro}
%    \begin{macro}{\HoLogoHtml@plainTeX@space}
%    \begin{macrocode}
\def\HoLogoHtml@plainTeX@space#1{%
  #1{p}{P}lain \hologo{TeX}%
}
%    \end{macrocode}
%    \end{macro}
%
%    \begin{macro}{\HoLogo@plainTeX@hyphen}
%    \begin{macrocode}
\def\HoLogo@plainTeX@hyphen#1{%
  \HOLOGO@mbox{#1{p}{P}lain}\HOLOGO@hyphen\hologo{TeX}%
}
%    \end{macrocode}
%    \end{macro}
%    \begin{macro}{\HoLogoCs@plainTeX@hyphen}
%    \begin{macrocode}
\def\HoLogoCs@plainTeX@hyphen#1{#1{p}{P}lain-TeX}
%    \end{macrocode}
%    \end{macro}
%    \begin{macro}{\HoLogoBkm@plainTeX@hyphen}
%    \begin{macrocode}
\def\HoLogoBkm@plainTeX@hyphen#1{%
  #1{p}{P}lain-\hologo{TeX}%
}
%    \end{macrocode}
%    \end{macro}
%    \begin{macro}{\HoLogoHtml@plainTeX@hyphen}
%    \begin{macrocode}
\def\HoLogoHtml@plainTeX@hyphen#1{%
  #1{p}{P}lain-\hologo{TeX}%
}
%    \end{macrocode}
%    \end{macro}
%
%    \begin{macro}{\HoLogo@plainTeX@runtogether}
%    \begin{macrocode}
\def\HoLogo@plainTeX@runtogether#1{%
  \HOLOGO@mbox{#1{p}{P}lain\hologo{TeX}}%
}
%    \end{macrocode}
%    \end{macro}
%    \begin{macro}{\HoLogoCs@plainTeX@runtogether}
%    \begin{macrocode}
\def\HoLogoCs@plainTeX@runtogether#1{#1{p}{P}lainTeX}
%    \end{macrocode}
%    \end{macro}
%    \begin{macro}{\HoLogoBkm@plainTeX@runtogether}
%    \begin{macrocode}
\def\HoLogoBkm@plainTeX@runtogether#1{%
  #1{p}{P}lain\hologo{TeX}%
}
%    \end{macrocode}
%    \end{macro}
%    \begin{macro}{\HoLogoHtml@plainTeX@runtogether}
%    \begin{macrocode}
\def\HoLogoHtml@plainTeX@runtogether#1{%
  #1{p}{P}lain\hologo{TeX}%
}
%    \end{macrocode}
%    \end{macro}
%
%    \begin{macro}{\HoLogo@plainTeX}
%    \begin{macrocode}
\def\HoLogo@plainTeX{\HoLogo@plainTeX@space}
%    \end{macrocode}
%    \end{macro}
%    \begin{macro}{\HoLogoCs@plainTeX}
%    \begin{macrocode}
\def\HoLogoCs@plainTeX{\HoLogoCs@plainTeX@space}
%    \end{macrocode}
%    \end{macro}
%    \begin{macro}{\HoLogoBkm@plainTeX}
%    \begin{macrocode}
\def\HoLogoBkm@plainTeX{\HoLogoBkm@plainTeX@space}
%    \end{macrocode}
%    \end{macro}
%    \begin{macro}{\HoLogoHtml@plainTeX}
%    \begin{macrocode}
\def\HoLogoHtml@plainTeX{\HoLogoHtml@plainTeX@space}
%    \end{macrocode}
%    \end{macro}
%
% \subsubsection{\hologo{LaTeX}}
%
%    Source: \hologo{LaTeX} kernel.
%\begin{quote}
%\begin{verbatim}
%\DeclareRobustCommand{\LaTeX}{%
%  L%
%  \kern-.36em%
%  {%
%    \sbox\z@ T%
%    \vbox to\ht\z@{%
%      \hbox{%
%        \check@mathfonts
%        \fontsize\sf@size\z@
%        \math@fontsfalse
%        \selectfont
%        A%
%      }%
%      \vss
%    }%
%  }%
%  \kern-.15em%
%  \TeX
%}
%\end{verbatim}
%\end{quote}
%
%    \begin{macro}{\HoLogo@La}
%    \begin{macrocode}
\def\HoLogo@La#1{%
  L%
  \kern-.36em%
  \begingroup
    \setbox\ltx@zero\hbox{T}%
    \vbox to\ht\ltx@zero{%
      \hbox{%
        \ltx@ifundefined{check@mathfonts}{%
          \csname sevenrm\endcsname
        }{%
          \check@mathfonts
          \fontsize\sf@size{0pt}%
          \math@fontsfalse\selectfont
        }%
        A%
      }%
      \vss
    }%
  \endgroup
}
%    \end{macrocode}
%    \end{macro}
%
%    \begin{macro}{\HoLogo@LaTeX}
%    Source: \hologo{LaTeX} kernel.
%    \begin{macrocode}
\def\HoLogo@LaTeX#1{%
  \hologo{La}%
  \kern-.15em%
  \hologo{TeX}%
}
%    \end{macrocode}
%    \end{macro}
%    \begin{macro}{\HoLogoHtml@LaTeX}
%    \begin{macrocode}
\def\HoLogoHtml@LaTeX#1{%
  \HoLogoCss@LaTeX
  \HOLOGO@Span{LaTeX}{%
    L%
    \HOLOGO@Span{a}{%
      A%
    }%
    \hologo{TeX}%
  }%
}
%    \end{macrocode}
%    \end{macro}
%    \begin{macro}{\HoLogoCss@LaTeX}
%    \begin{macrocode}
\def\HoLogoCss@LaTeX{%
  \Css{%
    span.HoLogo-LaTeX span.HoLogo-a{%
      position:relative;%
      top:-.5ex;%
      margin-left:-.36em;%
      margin-right:-.15em;%
      font-size:85\%;%
    }%
  }%
  \global\let\HoLogoCss@LaTeX\relax
}
%    \end{macrocode}
%    \end{macro}
%
% \subsubsection{\hologo{(La)TeX}}
%
%    \begin{macro}{\HoLogo@LaTeXTeX}
%    The kerning around the parentheses is taken
%    from package \xpackage{dtklogos} \cite{dtklogos}.
%\begin{quote}
%\begin{verbatim}
%\DeclareRobustCommand{\LaTeXTeX}{%
%  (%
%  \kern-.15em%
%  L%
%  \kern-.36em%
%  {%
%    \sbox\z@ T%
%    \vbox to\ht0{%
%      \hbox{%
%        $\m@th$%
%        \csname S@\f@size\endcsname
%        \fontsize\sf@size\z@
%        \math@fontsfalse
%        \selectfont
%        A%
%      }%
%      \vss
%    }%
%  }%
%  \kern-.2em%
%  )%
%  \kern-.15em%
%  \TeX
%}
%\end{verbatim}
%\end{quote}
%    \begin{macrocode}
\def\HoLogo@LaTeXTeX#1{%
  (%
  \kern-.15em%
  \hologo{La}%
  \kern-.2em%
  )%
  \kern-.15em%
  \hologo{TeX}%
}
%    \end{macrocode}
%    \end{macro}
%    \begin{macro}{\HoLogoBkm@LaTeXTeX}
%    \begin{macrocode}
\def\HoLogoBkm@LaTeXTeX#1{(La)TeX}
%    \end{macrocode}
%    \end{macro}
%
%    \begin{macro}{\HoLogo@(La)TeX}
%    \begin{macrocode}
\expandafter
\let\csname HoLogo@(La)TeX\endcsname\HoLogo@LaTeXTeX
%    \end{macrocode}
%    \end{macro}
%    \begin{macro}{\HoLogoBkm@(La)TeX}
%    \begin{macrocode}
\expandafter
\let\csname HoLogoBkm@(La)TeX\endcsname\HoLogoBkm@LaTeXTeX
%    \end{macrocode}
%    \end{macro}
%    \begin{macro}{\HoLogoHtml@LaTeXTeX}
%    \begin{macrocode}
\def\HoLogoHtml@LaTeXTeX#1{%
  \HoLogoCss@LaTeXTeX
  \HOLOGO@Span{LaTeXTeX}{%
    (%
    \HOLOGO@Span{L}{L}%
    \HOLOGO@Span{a}{A}%
    \HOLOGO@Span{ParenRight}{)}%
    \hologo{TeX}%
  }%
}
%    \end{macrocode}
%    \end{macro}
%    \begin{macro}{\HoLogoHtml@(La)TeX}
%    Kerning after opening parentheses and before closing parentheses
%    is $-0.1$\,em. The original values $-0.15$\,em
%    looked too ugly for a serif font.
%    \begin{macrocode}
\expandafter
\let\csname HoLogoHtml@(La)TeX\endcsname\HoLogoHtml@LaTeXTeX
%    \end{macrocode}
%    \end{macro}
%    \begin{macro}{\HoLogoCss@LaTeXTeX}
%    \begin{macrocode}
\def\HoLogoCss@LaTeXTeX{%
  \Css{%
    span.HoLogo-LaTeXTeX span.HoLogo-L{%
      margin-left:-.1em;%
    }%
  }%
  \Css{%
    span.HoLogo-LaTeXTeX span.HoLogo-a{%
      position:relative;%
      top:-.5ex;%
      margin-left:-.36em;%
      margin-right:-.1em;%
      font-size:85\%;%
    }%
  }%
  \Css{%
    span.HoLogo-LaTeXTeX span.HoLogo-ParenRight{%
      margin-right:-.15em;%
    }%
  }%
  \global\let\HoLogoCss@LaTeXTeX\relax
}
%    \end{macrocode}
%    \end{macro}
%
% \subsubsection{\hologo{LaTeXe}}
%
%    \begin{macro}{\HoLogo@LaTeXe}
%    Source: \hologo{LaTeX} kernel
%    \begin{macrocode}
\def\HoLogo@LaTeXe#1{%
  \hologo{LaTeX}%
  \kern.15em%
  \hbox{%
    \HOLOGO@MathSetup
    2%
    $_{\textstyle\varepsilon}$%
  }%
}
%    \end{macrocode}
%    \end{macro}
%
%    \begin{macro}{\HoLogoCs@LaTeXe}
%    \begin{macrocode}
\ifnum64=`\^^^^0040\relax % test for big chars of LuaTeX/XeTeX
  \catcode`\$=9 %
  \catcode`\&=14 %
\else
  \catcode`\$=14 %
  \catcode`\&=9 %
\fi
\def\HoLogoCs@LaTeXe#1{%
  LaTeX2%
$ \string ^^^^0395%
& e%
}%
\catcode`\$=3 %
\catcode`\&=4 %
%    \end{macrocode}
%    \end{macro}
%
%    \begin{macro}{\HoLogoBkm@LaTeXe}
%    \begin{macrocode}
\def\HoLogoBkm@LaTeXe#1{%
  \hologo{LaTeX}%
  2%
  \HOLOGO@PdfdocUnicode{e}{\textepsilon}%
}
%    \end{macrocode}
%    \end{macro}
%
%    \begin{macro}{\HoLogoHtml@LaTeXe}
%    \begin{macrocode}
\def\HoLogoHtml@LaTeXe#1{%
  \HoLogoCss@LaTeXe
  \HOLOGO@Span{LaTeX2e}{%
    \hologo{LaTeX}%
    \HOLOGO@Span{2}{2}%
    \HOLOGO@Span{e}{%
      \HOLOGO@MathSetup
      \ensuremath{\textstyle\varepsilon}%
    }%
  }%
}
%    \end{macrocode}
%    \end{macro}
%    \begin{macro}{\HoLogoCss@LaTeXe}
%    \begin{macrocode}
\def\HoLogoCss@LaTeXe{%
  \Css{%
    span.HoLogo-LaTeX2e span.HoLogo-2{%
      padding-left:.15em;%
    }%
  }%
  \Css{%
    span.HoLogo-LaTeX2e span.HoLogo-e{%
      position:relative;%
      top:.35ex;%
      text-decoration:none;%
    }%
  }%
  \global\let\HoLogoCss@LaTeXe\relax
}
%    \end{macrocode}
%    \end{macro}
%
%    \begin{macro}{\HoLogo@LaTeX2e}
%    \begin{macrocode}
\expandafter
\let\csname HoLogo@LaTeX2e\endcsname\HoLogo@LaTeXe
%    \end{macrocode}
%    \end{macro}
%    \begin{macro}{\HoLogoCs@LaTeX2e}
%    \begin{macrocode}
\expandafter
\let\csname HoLogoCs@LaTeX2e\endcsname\HoLogoCs@LaTeXe
%    \end{macrocode}
%    \end{macro}
%    \begin{macro}{\HoLogoBkm@LaTeX2e}
%    \begin{macrocode}
\expandafter
\let\csname HoLogoBkm@LaTeX2e\endcsname\HoLogoBkm@LaTeXe
%    \end{macrocode}
%    \end{macro}
%    \begin{macro}{\HoLogoHtml@LaTeX2e}
%    \begin{macrocode}
\expandafter
\let\csname HoLogoHtml@LaTeX2e\endcsname\HoLogoHtml@LaTeXe
%    \end{macrocode}
%    \end{macro}
%
% \subsubsection{\hologo{LaTeX3}}
%
%    \begin{macro}{\HoLogo@LaTeX3}
%    Source: \hologo{LaTeX} kernel
%    \begin{macrocode}
\expandafter\def\csname HoLogo@LaTeX3\endcsname#1{%
  \hologo{LaTeX}%
  3%
}
%    \end{macrocode}
%    \end{macro}
%
%    \begin{macro}{\HoLogoBkm@LaTeX3}
%    \begin{macrocode}
\expandafter\def\csname HoLogoBkm@LaTeX3\endcsname#1{%
  \hologo{LaTeX}%
  3%
}
%    \end{macrocode}
%    \end{macro}
%    \begin{macro}{\HoLogoHtml@LaTeX3}
%    \begin{macrocode}
\expandafter
\let\csname HoLogoHtml@LaTeX3\expandafter\endcsname
\csname HoLogo@LaTeX3\endcsname
%    \end{macrocode}
%    \end{macro}
%
% \subsubsection{\hologo{LaTeXML}}
%
%    \begin{macro}{\HoLogo@LaTeXML}
%    \begin{macrocode}
\def\HoLogo@LaTeXML#1{%
  \HOLOGO@mbox{%
    \hologo{La}%
    \kern-.15em%
    T%
    \kern-.1667em%
    \lower.5ex\hbox{E}%
    \kern-.125em%
    \HoLogoFont@font{LaTeXML}{sc}{xml}%
  }%
}
%    \end{macrocode}
%    \end{macro}
%    \begin{macro}{\HoLogoHtml@pdfLaTeX}
%    \begin{macrocode}
\def\HoLogoHtml@LaTeXML#1{%
  \HOLOGO@Span{LaTeXML}{%
    \HoLogoCss@LaTeX
    \HoLogoCss@TeX
    \HOLOGO@Span{LaTeX}{%
      L%
      \HOLOGO@Span{a}{%
        A%
      }%
    }%
    \HOLOGO@Span{TeX}{%
      T%
      \HOLOGO@Span{e}{%
        E%
      }%
    }%
    \HCode{<span style="font-variant: small-caps;">}%
    xml%
    \HCode{</span>}%
  }%
}
%    \end{macrocode}
%    \end{macro}
%
% \subsubsection{\hologo{eTeX}}
%
%    \begin{macro}{\HoLogo@eTeX}
%    Source: package \xpackage{etex}
%    \begin{macrocode}
\def\HoLogo@eTeX#1{%
  \ltx@mbox{%
    \HOLOGO@MathSetup
    $\varepsilon$%
    -%
    \HOLOGO@NegativeKerning{-T,T-,To}%
    \hologo{TeX}%
  }%
}
%    \end{macrocode}
%    \end{macro}
%    \begin{macro}{\HoLogoCs@eTeX}
%    \begin{macrocode}
\ifnum64=`\^^^^0040\relax % test for big chars of LuaTeX/XeTeX
  \catcode`\$=9 %
  \catcode`\&=14 %
\else
  \catcode`\$=14 %
  \catcode`\&=9 %
\fi
\def\HoLogoCs@eTeX#1{%
$ #1{\string ^^^^0395}{\string ^^^^03b5}%
& #1{e}{E}%
  TeX%
}%
\catcode`\$=3 %
\catcode`\&=4 %
%    \end{macrocode}
%    \end{macro}
%    \begin{macro}{\HoLogoBkm@eTeX}
%    \begin{macrocode}
\def\HoLogoBkm@eTeX#1{%
  \HOLOGO@PdfdocUnicode{#1{e}{E}}{\textepsilon}%
  -%
  \hologo{TeX}%
}
%    \end{macrocode}
%    \end{macro}
%    \begin{macro}{\HoLogoHtml@eTeX}
%    \begin{macrocode}
\def\HoLogoHtml@eTeX#1{%
  \ltx@mbox{%
    \HOLOGO@MathSetup
    $\varepsilon$%
    -%
    \hologo{TeX}%
  }%
}
%    \end{macrocode}
%    \end{macro}
%
% \subsubsection{\hologo{iniTeX}}
%
%    \begin{macro}{\HoLogo@iniTeX}
%    \begin{macrocode}
\def\HoLogo@iniTeX#1{%
  \HOLOGO@mbox{%
    #1{i}{I}ni\hologo{TeX}%
  }%
}
%    \end{macrocode}
%    \end{macro}
%    \begin{macro}{\HoLogoCs@iniTeX}
%    \begin{macrocode}
\def\HoLogoCs@iniTeX#1{#1{i}{I}niTeX}
%    \end{macrocode}
%    \end{macro}
%    \begin{macro}{\HoLogoBkm@iniTeX}
%    \begin{macrocode}
\def\HoLogoBkm@iniTeX#1{%
  #1{i}{I}ni\hologo{TeX}%
}
%    \end{macrocode}
%    \end{macro}
%    \begin{macro}{\HoLogoHtml@iniTeX}
%    \begin{macrocode}
\let\HoLogoHtml@iniTeX\HoLogo@iniTeX
%    \end{macrocode}
%    \end{macro}
%
% \subsubsection{\hologo{virTeX}}
%
%    \begin{macro}{\HoLogo@virTeX}
%    \begin{macrocode}
\def\HoLogo@virTeX#1{%
  \HOLOGO@mbox{%
    #1{v}{V}ir\hologo{TeX}%
  }%
}
%    \end{macrocode}
%    \end{macro}
%    \begin{macro}{\HoLogoCs@virTeX}
%    \begin{macrocode}
\def\HoLogoCs@virTeX#1{#1{v}{V}irTeX}
%    \end{macrocode}
%    \end{macro}
%    \begin{macro}{\HoLogoBkm@virTeX}
%    \begin{macrocode}
\def\HoLogoBkm@virTeX#1{%
  #1{v}{V}ir\hologo{TeX}%
}
%    \end{macrocode}
%    \end{macro}
%    \begin{macro}{\HoLogoHtml@virTeX}
%    \begin{macrocode}
\let\HoLogoHtml@virTeX\HoLogo@virTeX
%    \end{macrocode}
%    \end{macro}
%
% \subsubsection{\hologo{SliTeX}}
%
% \paragraph{Definitions of the three variants.}
%
%    \begin{macro}{\HoLogo@SLiTeX@lift}
%    \begin{macrocode}
\def\HoLogo@SLiTeX@lift#1{%
  \HoLogoFont@font{SliTeX}{rm}{%
    S%
    \kern-.06em%
    L%
    \kern-.18em%
    \raise.32ex\hbox{\HoLogoFont@font{SliTeX}{sc}{i}}%
    \HOLOGO@discretionary
    \kern-.06em%
    \hologo{TeX}%
  }%
}
%    \end{macrocode}
%    \end{macro}
%    \begin{macro}{\HoLogoBkm@SLiTeX@lift}
%    \begin{macrocode}
\def\HoLogoBkm@SLiTeX@lift#1{SLiTeX}
%    \end{macrocode}
%    \end{macro}
%    \begin{macro}{\HoLogoHtml@SLiTeX@lift}
%    \begin{macrocode}
\def\HoLogoHtml@SLiTeX@lift#1{%
  \HoLogoCss@SLiTeX@lift
  \HOLOGO@Span{SLiTeX-lift}{%
    \HoLogoFont@font{SliTeX}{rm}{%
      S%
      \HOLOGO@Span{L}{L}%
      \HOLOGO@Span{i}{i}%
      \hologo{TeX}%
    }%
  }%
}
%    \end{macrocode}
%    \end{macro}
%    \begin{macro}{\HoLogoCss@SLiTeX@lift}
%    \begin{macrocode}
\def\HoLogoCss@SLiTeX@lift{%
  \Css{%
    span.HoLogo-SLiTeX-lift span.HoLogo-L{%
      margin-left:-.06em;%
      margin-right:-.18em;%
    }%
  }%
  \Css{%
    span.HoLogo-SLiTeX-lift span.HoLogo-i{%
      position:relative;%
      top:-.32ex;%
      margin-right:-.06em;%
      font-variant:small-caps;%
    }%
  }%
  \global\let\HoLogoCss@SLiTeX@lift\relax
}
%    \end{macrocode}
%    \end{macro}
%
%    \begin{macro}{\HoLogo@SliTeX@simple}
%    \begin{macrocode}
\def\HoLogo@SliTeX@simple#1{%
  \HoLogoFont@font{SliTeX}{rm}{%
    \ltx@mbox{%
      \HoLogoFont@font{SliTeX}{sc}{Sli}%
    }%
    \HOLOGO@discretionary
    \hologo{TeX}%
  }%
}
%    \end{macrocode}
%    \end{macro}
%    \begin{macro}{\HoLogoBkm@SliTeX@simple}
%    \begin{macrocode}
\def\HoLogoBkm@SliTeX@simple#1{SliTeX}
%    \end{macrocode}
%    \end{macro}
%    \begin{macro}{\HoLogoHtml@SliTeX@simple}
%    \begin{macrocode}
\let\HoLogoHtml@SliTeX@simple\HoLogo@SliTeX@simple
%    \end{macrocode}
%    \end{macro}
%
%    \begin{macro}{\HoLogo@SliTeX@narrow}
%    \begin{macrocode}
\def\HoLogo@SliTeX@narrow#1{%
  \HoLogoFont@font{SliTeX}{rm}{%
    \ltx@mbox{%
      S%
      \kern-.06em%
      \HoLogoFont@font{SliTeX}{sc}{%
        l%
        \kern-.035em%
        i%
      }%
    }%
    \HOLOGO@discretionary
    \kern-.06em%
    \hologo{TeX}%
  }%
}
%    \end{macrocode}
%    \end{macro}
%    \begin{macro}{\HoLogoBkm@SliTeX@narrow}
%    \begin{macrocode}
\def\HoLogoBkm@SliTeX@narrow#1{SliTeX}
%    \end{macrocode}
%    \end{macro}
%    \begin{macro}{\HoLogoHtml@SliTeX@narrow}
%    \begin{macrocode}
\def\HoLogoHtml@SliTeX@narrow#1{%
  \HoLogoCss@SliTeX@narrow
  \HOLOGO@Span{SliTeX-narrow}{%
    \HoLogoFont@font{SliTeX}{rm}{%
      S%
        \HOLOGO@Span{l}{l}%
        \HOLOGO@Span{i}{i}%
      \hologo{TeX}%
    }%
  }%
}
%    \end{macrocode}
%    \end{macro}
%    \begin{macro}{\HoLogoCss@SliTeX@narrow}
%    \begin{macrocode}
\def\HoLogoCss@SliTeX@narrow{%
  \Css{%
    span.HoLogo-SliTeX-narrow span.HoLogo-l{%
      margin-left:-.06em;%
      margin-right:-.035em;%
      font-variant:small-caps;%
    }%
  }%
  \Css{%
    span.HoLogo-SliTeX-narrow span.HoLogo-i{%
      margin-right:-.06em;%
      font-variant:small-caps;%
    }%
  }%
  \global\let\HoLogoCss@SliTeX@narrow\relax
}
%    \end{macrocode}
%    \end{macro}
%
% \paragraph{Macro set completion.}
%
%    \begin{macro}{\HoLogo@SLiTeX@simple}
%    \begin{macrocode}
\def\HoLogo@SLiTeX@simple{\HoLogo@SliTeX@simple}
%    \end{macrocode}
%    \end{macro}
%    \begin{macro}{\HoLogoBkm@SLiTeX@simple}
%    \begin{macrocode}
\def\HoLogoBkm@SLiTeX@simple{\HoLogoBkm@SliTeX@simple}
%    \end{macrocode}
%    \end{macro}
%    \begin{macro}{\HoLogoHtml@SLiTeX@simple}
%    \begin{macrocode}
\def\HoLogoHtml@SLiTeX@simple{\HoLogoHtml@SliTeX@simple}
%    \end{macrocode}
%    \end{macro}
%
%    \begin{macro}{\HoLogo@SLiTeX@narrow}
%    \begin{macrocode}
\def\HoLogo@SLiTeX@narrow{\HoLogo@SliTeX@narrow}
%    \end{macrocode}
%    \end{macro}
%    \begin{macro}{\HoLogoBkm@SLiTeX@narrow}
%    \begin{macrocode}
\def\HoLogoBkm@SLiTeX@narrow{\HoLogoBkm@SliTeX@narrow}
%    \end{macrocode}
%    \end{macro}
%    \begin{macro}{\HoLogoHtml@SLiTeX@narrow}
%    \begin{macrocode}
\def\HoLogoHtml@SLiTeX@narrow{\HoLogoHtml@SliTeX@narrow}
%    \end{macrocode}
%    \end{macro}
%
%    \begin{macro}{\HoLogo@SliTeX@lift}
%    \begin{macrocode}
\def\HoLogo@SliTeX@lift{\HoLogo@SLiTeX@lift}
%    \end{macrocode}
%    \end{macro}
%    \begin{macro}{\HoLogoBkm@SliTeX@lift}
%    \begin{macrocode}
\def\HoLogoBkm@SliTeX@lift{\HoLogoBkm@SLiTeX@lift}
%    \end{macrocode}
%    \end{macro}
%    \begin{macro}{\HoLogoHtml@SliTeX@lift}
%    \begin{macrocode}
\def\HoLogoHtml@SliTeX@lift{\HoLogoHtml@SLiTeX@lift}
%    \end{macrocode}
%    \end{macro}
%
% \paragraph{Defaults.}
%
%    \begin{macro}{\HoLogo@SLiTeX}
%    \begin{macrocode}
\def\HoLogo@SLiTeX{\HoLogo@SLiTeX@lift}
%    \end{macrocode}
%    \end{macro}
%    \begin{macro}{\HoLogoBkm@SLiTeX}
%    \begin{macrocode}
\def\HoLogoBkm@SLiTeX{\HoLogoBkm@SLiTeX@lift}
%    \end{macrocode}
%    \end{macro}
%    \begin{macro}{\HoLogoHtml@SLiTeX}
%    \begin{macrocode}
\def\HoLogoHtml@SLiTeX{\HoLogoHtml@SLiTeX@lift}
%    \end{macrocode}
%    \end{macro}
%
%    \begin{macro}{\HoLogo@SliTeX}
%    \begin{macrocode}
\def\HoLogo@SliTeX{\HoLogo@SliTeX@narrow}
%    \end{macrocode}
%    \end{macro}
%    \begin{macro}{\HoLogoBkm@SliTeX}
%    \begin{macrocode}
\def\HoLogoBkm@SliTeX{\HoLogoBkm@SliTeX@narrow}
%    \end{macrocode}
%    \end{macro}
%    \begin{macro}{\HoLogoHtml@SliTeX}
%    \begin{macrocode}
\def\HoLogoHtml@SliTeX{\HoLogoHtml@SliTeX@narrow}
%    \end{macrocode}
%    \end{macro}
%
% \subsubsection{\hologo{LuaTeX}}
%
%    \begin{macro}{\HoLogo@LuaTeX}
%    The kerning is an idea of Hans Hagen, see mailing list
%    `luatex at tug dot org' in March 2010.
%    \begin{macrocode}
\def\HoLogo@LuaTeX#1{%
  \HOLOGO@mbox{%
    Lua%
    \HOLOGO@NegativeKerning{aT,oT,To}%
    \hologo{TeX}%
  }%
}
%    \end{macrocode}
%    \end{macro}
%    \begin{macro}{\HoLogoHtml@LuaTeX}
%    \begin{macrocode}
\let\HoLogoHtml@LuaTeX\HoLogo@LuaTeX
%    \end{macrocode}
%    \end{macro}
%
% \subsubsection{\hologo{LuaLaTeX}}
%
%    \begin{macro}{\HoLogo@LuaLaTeX}
%    \begin{macrocode}
\def\HoLogo@LuaLaTeX#1{%
  \HOLOGO@mbox{%
    Lua%
    \hologo{LaTeX}%
  }%
}
%    \end{macrocode}
%    \end{macro}
%    \begin{macro}{\HoLogoHtml@LuaLaTeX}
%    \begin{macrocode}
\let\HoLogoHtml@LuaLaTeX\HoLogo@LuaLaTeX
%    \end{macrocode}
%    \end{macro}
%
% \subsubsection{\hologo{XeTeX}, \hologo{XeLaTeX}}
%
%    \begin{macro}{\HOLOGO@IfCharExists}
%    \begin{macrocode}
\ifluatex
  \ifnum\luatexversion<36 %
  \else
    \def\HOLOGO@IfCharExists#1{%
      \ifnum
        \directlua{%
           if luaotfload and luaotfload.aux then
             if luaotfload.aux.font_has_glyph(%
                    font.current(), \number#1) then % 	 
	       tex.print("1") % 	 
	     end % 	 
	   elseif font and font.fonts and font.current then %
            local f = font.fonts[font.current()]%
            if f.characters and f.characters[\number#1] then %
              tex.print("1")%
            end %
          end%
        }0=\ltx@zero
        \expandafter\ltx@secondoftwo
      \else
        \expandafter\ltx@firstoftwo
      \fi
    }%
  \fi
\fi
\ltx@IfUndefined{HOLOGO@IfCharExists}{%
  \def\HOLOGO@@IfCharExists#1{%
    \begingroup
      \tracinglostchars=\ltx@zero
      \setbox\ltx@zero=\hbox{%
        \kern7sp\char#1\relax
        \ifnum\lastkern>\ltx@zero
          \expandafter\aftergroup\csname iffalse\endcsname
        \else
          \expandafter\aftergroup\csname iftrue\endcsname
        \fi
      }%
      % \if{true|false} from \aftergroup
      \endgroup
      \expandafter\ltx@firstoftwo
    \else
      \endgroup
      \expandafter\ltx@secondoftwo
    \fi
  }%
  \ifxetex
    \ltx@IfUndefined{XeTeXfonttype}{}{%
      \ltx@IfUndefined{XeTeXcharglyph}{}{%
        \def\HOLOGO@IfCharExists#1{%
          \ifnum\XeTeXfonttype\font>\ltx@zero
            \expandafter\ltx@firstofthree
          \else
            \expandafter\ltx@gobble
          \fi
          {%
            \ifnum\XeTeXcharglyph#1>\ltx@zero
              \expandafter\ltx@firstoftwo
            \else
              \expandafter\ltx@secondoftwo
            \fi
          }%
          \HOLOGO@@IfCharExists{#1}%
        }%
      }%
    }%
  \fi
}{}
\ltx@ifundefined{HOLOGO@IfCharExists}{%
  \ifnum64=`\^^^^0040\relax % test for big chars of LuaTeX/XeTeX
    \let\HOLOGO@IfCharExists\HOLOGO@@IfCharExists
  \else
    \def\HOLOGO@IfCharExists#1{%
      \ifnum#1>255 %
        \expandafter\ltx@fourthoffour
      \fi
      \HOLOGO@@IfCharExists{#1}%
    }%
  \fi
}{}
%    \end{macrocode}
%    \end{macro}
%
%    \begin{macro}{\HoLogo@Xe}
%    Source: package \xpackage{dtklogos}
%    \begin{macrocode}
\def\HoLogo@Xe#1{%
  X%
  \kern-.1em\relax
  \HOLOGO@IfCharExists{"018E}{%
    \lower.5ex\hbox{\char"018E}%
  }{%
    \chardef\HOLOGO@choice=\ltx@zero
    \ifdim\fontdimen\ltx@one\font>0pt %
      \ltx@IfUndefined{rotatebox}{%
        \ltx@IfUndefined{pgftext}{%
          \ltx@IfUndefined{psscalebox}{%
            \ltx@IfUndefined{HOLOGO@ScaleBox@\hologoDriver}{%
            }{%
              \chardef\HOLOGO@choice=4 %
            }%
          }{%
            \chardef\HOLOGO@choice=3 %
          }%
        }{%
          \chardef\HOLOGO@choice=2 %
        }%
      }{%
        \chardef\HOLOGO@choice=1 %
      }%
      \ifcase\HOLOGO@choice
        \HOLOGO@WarningUnsupportedDriver{Xe}%
        e%
      \or % 1: \rotatebox
        \begingroup
          \setbox\ltx@zero\hbox{\rotatebox{180}{E}}%
          \ltx@LocDimenA=\dp\ltx@zero
          \advance\ltx@LocDimenA by -.5ex\relax
          \raise\ltx@LocDimenA\box\ltx@zero
        \endgroup
      \or % 2: \pgftext
        \lower.5ex\hbox{%
          \pgfpicture
            \pgftext[rotate=180]{E}%
          \endpgfpicture
        }%
      \or % 3: \psscalebox
        \begingroup
          \setbox\ltx@zero\hbox{\psscalebox{-1 -1}{E}}%
          \ltx@LocDimenA=\dp\ltx@zero
          \advance\ltx@LocDimenA by -.5ex\relax
          \raise\ltx@LocDimenA\box\ltx@zero
        \endgroup
      \or % 4: \HOLOGO@PointReflectBox
        \lower.5ex\hbox{\HOLOGO@PointReflectBox{E}}%
      \else
        \@PackageError{hologo}{Internal error (choice/it}\@ehc
      \fi
    \else
      \ltx@IfUndefined{reflectbox}{%
        \ltx@IfUndefined{pgftext}{%
          \ltx@IfUndefined{psscalebox}{%
            \ltx@IfUndefined{HOLOGO@ScaleBox@\hologoDriver}{%
            }{%
              \chardef\HOLOGO@choice=4 %
            }%
          }{%
            \chardef\HOLOGO@choice=3 %
          }%
        }{%
          \chardef\HOLOGO@choice=2 %
        }%
      }{%
        \chardef\HOLOGO@choice=1 %
      }%
      \ifcase\HOLOGO@choice
        \HOLOGO@WarningUnsupportedDriver{Xe}%
        e%
      \or % 1: reflectbox
        \lower.5ex\hbox{%
          \reflectbox{E}%
        }%
      \or % 2: \pgftext
        \lower.5ex\hbox{%
          \pgfpicture
            \pgftransformxscale{-1}%
            \pgftext{E}%
          \endpgfpicture
        }%
      \or % 3: \psscalebox
        \lower.5ex\hbox{%
          \psscalebox{-1 1}{E}%
        }%
      \or % 4: \HOLOGO@Reflectbox
        \lower.5ex\hbox{%
          \HOLOGO@ReflectBox{E}%
        }%
      \else
        \@PackageError{hologo}{Internal error (choice/up)}\@ehc
      \fi
    \fi
  }%
}
%    \end{macrocode}
%    \end{macro}
%    \begin{macro}{\HoLogoHtml@Xe}
%    \begin{macrocode}
\def\HoLogoHtml@Xe#1{%
  \HoLogoCss@Xe
  \HOLOGO@Span{Xe}{%
    X%
    \HOLOGO@Span{e}{%
      \HCode{&\ltx@hashchar x018e;}%
    }%
  }%
}
%    \end{macrocode}
%    \end{macro}
%    \begin{macro}{\HoLogoCss@Xe}
%    \begin{macrocode}
\def\HoLogoCss@Xe{%
  \Css{%
    span.HoLogo-Xe span.HoLogo-e{%
      position:relative;%
      top:.5ex;%
      left-margin:-.1em;%
    }%
  }%
  \global\let\HoLogoCss@Xe\relax
}
%    \end{macrocode}
%    \end{macro}
%
%    \begin{macro}{\HoLogo@XeTeX}
%    \begin{macrocode}
\def\HoLogo@XeTeX#1{%
  \hologo{Xe}%
  \kern-.15em\relax
  \hologo{TeX}%
}
%    \end{macrocode}
%    \end{macro}
%
%    \begin{macro}{\HoLogoHtml@XeTeX}
%    \begin{macrocode}
\def\HoLogoHtml@XeTeX#1{%
  \HoLogoCss@XeTeX
  \HOLOGO@Span{XeTeX}{%
    \hologo{Xe}%
    \hologo{TeX}%
  }%
}
%    \end{macrocode}
%    \end{macro}
%    \begin{macro}{\HoLogoCss@XeTeX}
%    \begin{macrocode}
\def\HoLogoCss@XeTeX{%
  \Css{%
    span.HoLogo-XeTeX span.HoLogo-TeX{%
      margin-left:-.15em;%
    }%
  }%
  \global\let\HoLogoCss@XeTeX\relax
}
%    \end{macrocode}
%    \end{macro}
%
%    \begin{macro}{\HoLogo@XeLaTeX}
%    \begin{macrocode}
\def\HoLogo@XeLaTeX#1{%
  \hologo{Xe}%
  \kern-.13em%
  \hologo{LaTeX}%
}
%    \end{macrocode}
%    \end{macro}
%    \begin{macro}{\HoLogoHtml@XeLaTeX}
%    \begin{macrocode}
\def\HoLogoHtml@XeLaTeX#1{%
  \HoLogoCss@XeLaTeX
  \HOLOGO@Span{XeLaTeX}{%
    \hologo{Xe}%
    \hologo{LaTeX}%
  }%
}
%    \end{macrocode}
%    \end{macro}
%    \begin{macro}{\HoLogoCss@XeLaTeX}
%    \begin{macrocode}
\def\HoLogoCss@XeLaTeX{%
  \Css{%
    span.HoLogo-XeLaTeX span.HoLogo-Xe{%
      margin-right:-.13em;%
    }%
  }%
  \global\let\HoLogoCss@XeLaTeX\relax
}
%    \end{macrocode}
%    \end{macro}
%
% \subsubsection{\hologo{pdfTeX}, \hologo{pdfLaTeX}}
%
%    \begin{macro}{\HoLogo@pdfTeX}
%    \begin{macrocode}
\def\HoLogo@pdfTeX#1{%
  \HOLOGO@mbox{%
    #1{p}{P}df\hologo{TeX}%
  }%
}
%    \end{macrocode}
%    \end{macro}
%    \begin{macro}{\HoLogoCs@pdfTeX}
%    \begin{macrocode}
\def\HoLogoCs@pdfTeX#1{#1{p}{P}dfTeX}
%    \end{macrocode}
%    \end{macro}
%    \begin{macro}{\HoLogoBkm@pdfTeX}
%    \begin{macrocode}
\def\HoLogoBkm@pdfTeX#1{%
  #1{p}{P}df\hologo{TeX}%
}
%    \end{macrocode}
%    \end{macro}
%    \begin{macro}{\HoLogoHtml@pdfTeX}
%    \begin{macrocode}
\let\HoLogoHtml@pdfTeX\HoLogo@pdfTeX
%    \end{macrocode}
%    \end{macro}
%
%    \begin{macro}{\HoLogo@pdfLaTeX}
%    \begin{macrocode}
\def\HoLogo@pdfLaTeX#1{%
  \HOLOGO@mbox{%
    #1{p}{P}df\hologo{LaTeX}%
  }%
}
%    \end{macrocode}
%    \end{macro}
%    \begin{macro}{\HoLogoCs@pdfLaTeX}
%    \begin{macrocode}
\def\HoLogoCs@pdfLaTeX#1{#1{p}{P}dfLaTeX}
%    \end{macrocode}
%    \end{macro}
%    \begin{macro}{\HoLogoBkm@pdfLaTeX}
%    \begin{macrocode}
\def\HoLogoBkm@pdfLaTeX#1{%
  #1{p}{P}df\hologo{LaTeX}%
}
%    \end{macrocode}
%    \end{macro}
%    \begin{macro}{\HoLogoHtml@pdfLaTeX}
%    \begin{macrocode}
\let\HoLogoHtml@pdfLaTeX\HoLogo@pdfLaTeX
%    \end{macrocode}
%    \end{macro}
%
% \subsubsection{\hologo{VTeX}}
%
%    \begin{macro}{\HoLogo@VTeX}
%    \begin{macrocode}
\def\HoLogo@VTeX#1{%
  \HOLOGO@mbox{%
    V\hologo{TeX}%
  }%
}
%    \end{macrocode}
%    \end{macro}
%    \begin{macro}{\HoLogoHtml@VTeX}
%    \begin{macrocode}
\let\HoLogoHtml@VTeX\HoLogo@VTeX
%    \end{macrocode}
%    \end{macro}
%
% \subsubsection{\hologo{AmS}, \dots}
%
%    Source: class \xclass{amsdtx}
%
%    \begin{macro}{\HoLogo@AmS}
%    \begin{macrocode}
\def\HoLogo@AmS#1{%
  \HoLogoFont@font{AmS}{sy}{%
    A%
    \kern-.1667em%
    \lower.5ex\hbox{M}%
    \kern-.125em%
    S%
  }%
}
%    \end{macrocode}
%    \end{macro}
%    \begin{macro}{\HoLogoBkm@AmS}
%    \begin{macrocode}
\def\HoLogoBkm@AmS#1{AmS}
%    \end{macrocode}
%    \end{macro}
%    \begin{macro}{\HoLogoHtml@AmS}
%    \begin{macrocode}
\def\HoLogoHtml@AmS#1{%
  \HoLogoCss@AmS
%  \HoLogoFont@font{AmS}{sy}{%
    \HOLOGO@Span{AmS}{%
      A%
      \HOLOGO@Span{M}{M}%
      S%
    }%
%   }%
}
%    \end{macrocode}
%    \end{macro}
%    \begin{macro}{\HoLogoCss@AmS}
%    \begin{macrocode}
\def\HoLogoCss@AmS{%
  \Css{%
    span.HoLogo-AmS span.HoLogo-M{%
      position:relative;%
      top:.5ex;%
      margin-left:-.1667em;%
      margin-right:-.125em;%
      text-decoration:none;%
    }%
  }%
  \global\let\HoLogoCss@AmS\relax
}
%    \end{macrocode}
%    \end{macro}
%
%    \begin{macro}{\HoLogo@AmSTeX}
%    \begin{macrocode}
\def\HoLogo@AmSTeX#1{%
  \hologo{AmS}%
  \HOLOGO@hyphen
  \hologo{TeX}%
}
%    \end{macrocode}
%    \end{macro}
%    \begin{macro}{\HoLogoBkm@AmSTeX}
%    \begin{macrocode}
\def\HoLogoBkm@AmSTeX#1{AmS-TeX}%
%    \end{macrocode}
%    \end{macro}
%    \begin{macro}{\HoLogoHtml@AmSTeX}
%    \begin{macrocode}
\let\HoLogoHtml@AmSTeX\HoLogo@AmSTeX
%    \end{macrocode}
%    \end{macro}
%
%    \begin{macro}{\HoLogo@AmSLaTeX}
%    \begin{macrocode}
\def\HoLogo@AmSLaTeX#1{%
  \hologo{AmS}%
  \HOLOGO@hyphen
  \hologo{LaTeX}%
}
%    \end{macrocode}
%    \end{macro}
%    \begin{macro}{\HoLogoBkm@AmSLaTeX}
%    \begin{macrocode}
\def\HoLogoBkm@AmSLaTeX#1{AmS-LaTeX}%
%    \end{macrocode}
%    \end{macro}
%    \begin{macro}{\HoLogoHtml@AmSLaTeX}
%    \begin{macrocode}
\let\HoLogoHtml@AmSLaTeX\HoLogo@AmSLaTeX
%    \end{macrocode}
%    \end{macro}
%
% \subsubsection{\hologo{BibTeX}}
%
%    \begin{macro}{\HoLogo@BibTeX@sc}
%    A definition of \hologo{BibTeX} is provided in
%    the documentation source for the manual of \hologo{BibTeX}
%    \cite{btxdoc}.
%\begin{quote}
%\begin{verbatim}
%\def\BibTeX{%
%  {%
%    \rm
%    B%
%    \kern-.05em%
%    {%
%      \sc
%      i%
%      \kern-.025em %
%      b%
%    }%
%    \kern-.08em
%    T%
%    \kern-.1667em%
%    \lower.7ex\hbox{E}%
%    \kern-.125em%
%    X%
%  }%
%}
%\end{verbatim}
%\end{quote}
%    \begin{macrocode}
\def\HoLogo@BibTeX@sc#1{%
  B%
  \kern-.05em%
  \HoLogoFont@font{BibTeX}{sc}{%
    i%
    \kern-.025em%
    b%
  }%
  \HOLOGO@discretionary
  \kern-.08em%
  \hologo{TeX}%
}
%    \end{macrocode}
%    \end{macro}
%    \begin{macro}{\HoLogoHtml@BibTeX@sc}
%    \begin{macrocode}
\def\HoLogoHtml@BibTeX@sc#1{%
  \HoLogoCss@BibTeX@sc
  \HOLOGO@Span{BibTeX-sc}{%
    B%
    \HOLOGO@Span{i}{i}%
    \HOLOGO@Span{b}{b}%
    \hologo{TeX}%
  }%
}
%    \end{macrocode}
%    \end{macro}
%    \begin{macro}{\HoLogoCss@BibTeX@sc}
%    \begin{macrocode}
\def\HoLogoCss@BibTeX@sc{%
  \Css{%
    span.HoLogo-BibTeX-sc span.HoLogo-i{%
      margin-left:-.05em;%
      margin-right:-.025em;%
      font-variant:small-caps;%
    }%
  }%
  \Css{%
    span.HoLogo-BibTeX-sc span.HoLogo-b{%
      margin-right:-.08em;%
      font-variant:small-caps;%
    }%
  }%
  \global\let\HoLogoCss@BibTeX@sc\relax
}
%    \end{macrocode}
%    \end{macro}
%
%    \begin{macro}{\HoLogo@BibTeX@sf}
%    Variant \xoption{sf} avoids trouble with unavailable
%    small caps fonts (e.g., bold versions of Computer Modern or
%    Latin Modern). The definition is taken from
%    package \xpackage{dtklogos} \cite{dtklogos}.
%\begin{quote}
%\begin{verbatim}
%\DeclareRobustCommand{\BibTeX}{%
%  B%
%  \kern-.05em%
%  \hbox{%
%    $\m@th$% %% force math size calculations
%    \csname S@\f@size\endcsname
%    \fontsize\sf@size\z@
%    \math@fontsfalse
%    \selectfont
%    I%
%    \kern-.025em%
%    B
%  }%
%  \kern-.08em%
%  \-%
%  \TeX
%}
%\end{verbatim}
%\end{quote}
%    \begin{macrocode}
\def\HoLogo@BibTeX@sf#1{%
  B%
  \kern-.05em%
  \HoLogoFont@font{BibTeX}{bibsf}{%
    I%
    \kern-.025em%
    B%
  }%
  \HOLOGO@discretionary
  \kern-.08em%
  \hologo{TeX}%
}
%    \end{macrocode}
%    \end{macro}
%    \begin{macro}{\HoLogoHtml@BibTeX@sf}
%    \begin{macrocode}
\def\HoLogoHtml@BibTeX@sf#1{%
  \HoLogoCss@BibTeX@sf
  \HOLOGO@Span{BibTeX-sf}{%
    B%
    \HoLogoFont@font{BibTeX}{bibsf}{%
      \HOLOGO@Span{i}{I}%
      B%
    }%
    \hologo{TeX}%
  }%
}
%    \end{macrocode}
%    \end{macro}
%    \begin{macro}{\HoLogoCss@BibTeX@sf}
%    \begin{macrocode}
\def\HoLogoCss@BibTeX@sf{%
  \Css{%
    span.HoLogo-BibTeX-sf span.HoLogo-i{%
      margin-left:-.05em;%
      margin-right:-.025em;%
    }%
  }%
  \Css{%
    span.HoLogo-BibTeX-sf span.HoLogo-TeX{%
      margin-left:-.08em;%
    }%
  }%
  \global\let\HoLogoCss@BibTeX@sf\relax
}
%    \end{macrocode}
%    \end{macro}
%
%    \begin{macro}{\HoLogo@BibTeX}
%    \begin{macrocode}
\def\HoLogo@BibTeX{\HoLogo@BibTeX@sf}
%    \end{macrocode}
%    \end{macro}
%    \begin{macro}{\HoLogoHtml@BibTeX}
%    \begin{macrocode}
\def\HoLogoHtml@BibTeX{\HoLogoHtml@BibTeX@sf}
%    \end{macrocode}
%    \end{macro}
%
% \subsubsection{\hologo{BibTeX8}}
%
%    \begin{macro}{\HoLogo@BibTeX8}
%    \begin{macrocode}
\expandafter\def\csname HoLogo@BibTeX8\endcsname#1{%
  \hologo{BibTeX}%
  8%
}
%    \end{macrocode}
%    \end{macro}
%
%    \begin{macro}{\HoLogoBkm@BibTeX8}
%    \begin{macrocode}
\expandafter\def\csname HoLogoBkm@BibTeX8\endcsname#1{%
  \hologo{BibTeX}%
  8%
}
%    \end{macrocode}
%    \end{macro}
%    \begin{macro}{\HoLogoHtml@BibTeX8}
%    \begin{macrocode}
\expandafter
\let\csname HoLogoHtml@BibTeX8\expandafter\endcsname
\csname HoLogo@BibTeX8\endcsname
%    \end{macrocode}
%    \end{macro}
%
% \subsubsection{\hologo{ConTeXt}}
%
%    \begin{macro}{\HoLogo@ConTeXt@simple}
%    \begin{macrocode}
\def\HoLogo@ConTeXt@simple#1{%
  \HOLOGO@mbox{Con}%
  \HOLOGO@discretionary
  \HOLOGO@mbox{\hologo{TeX}t}%
}
%    \end{macrocode}
%    \end{macro}
%    \begin{macro}{\HoLogoHtml@ConTeXt@simple}
%    \begin{macrocode}
\let\HoLogoHtml@ConTeXt@simple\HoLogo@ConTeXt@simple
%    \end{macrocode}
%    \end{macro}
%
%    \begin{macro}{\HoLogo@ConTeXt@narrow}
%    This definition of logo \hologo{ConTeXt} with variant \xoption{narrow}
%    comes from TUGboat's class \xclass{ltugboat} (version 2010/11/15 v2.8).
%    \begin{macrocode}
\def\HoLogo@ConTeXt@narrow#1{%
  \HOLOGO@mbox{C\kern-.0333emon}%
  \HOLOGO@discretionary
  \kern-.0667em%
  \HOLOGO@mbox{\hologo{TeX}\kern-.0333emt}%
}
%    \end{macrocode}
%    \end{macro}
%    \begin{macro}{\HoLogoHtml@ConTeXt@narrow}
%    \begin{macrocode}
\def\HoLogoHtml@ConTeXt@narrow#1{%
  \HoLogoCss@ConTeXt@narrow
  \HOLOGO@Span{ConTeXt-narrow}{%
    \HOLOGO@Span{C}{C}%
    on%
    \hologo{TeX}%
    t%
  }%
}
%    \end{macrocode}
%    \end{macro}
%    \begin{macro}{\HoLogoCss@ConTeXt@narrow}
%    \begin{macrocode}
\def\HoLogoCss@ConTeXt@narrow{%
  \Css{%
    span.HoLogo-ConTeXt-narrow span.HoLogo-C{%
      margin-left:-.0333em;%
    }%
  }%
  \Css{%
    span.HoLogo-ConTeXt-narrow span.HoLogo-TeX{%
      margin-left:-.0667em;%
      margin-right:-.0333em;%
    }%
  }%
  \global\let\HoLogoCss@ConTeXt@narrow\relax
}
%    \end{macrocode}
%    \end{macro}
%
%    \begin{macro}{\HoLogo@ConTeXt}
%    \begin{macrocode}
\def\HoLogo@ConTeXt{\HoLogo@ConTeXt@narrow}
%    \end{macrocode}
%    \end{macro}
%    \begin{macro}{\HoLogoHtml@ConTeXt}
%    \begin{macrocode}
\def\HoLogoHtml@ConTeXt{\HoLogoHtml@ConTeXt@narrow}
%    \end{macrocode}
%    \end{macro}
%
% \subsubsection{\hologo{emTeX}}
%
%    \begin{macro}{\HoLogo@emTeX}
%    \begin{macrocode}
\def\HoLogo@emTeX#1{%
  \HOLOGO@mbox{#1{e}{E}m}%
  \HOLOGO@discretionary
  \hologo{TeX}%
}
%    \end{macrocode}
%    \end{macro}
%    \begin{macro}{\HoLogoCs@emTeX}
%    \begin{macrocode}
\def\HoLogoCs@emTeX#1{#1{e}{E}mTeX}%
%    \end{macrocode}
%    \end{macro}
%    \begin{macro}{\HoLogoBkm@emTeX}
%    \begin{macrocode}
\def\HoLogoBkm@emTeX#1{%
  #1{e}{E}m\hologo{TeX}%
}
%    \end{macrocode}
%    \end{macro}
%    \begin{macro}{\HoLogoHtml@emTeX}
%    \begin{macrocode}
\let\HoLogoHtml@emTeX\HoLogo@emTeX
%    \end{macrocode}
%    \end{macro}
%
% \subsubsection{\hologo{ExTeX}}
%
%    \begin{macro}{\HoLogo@ExTeX}
%    The definition is taken from the FAQ of the
%    project \hologo{ExTeX}
%    \cite{ExTeX-FAQ}.
%\begin{quote}
%\begin{verbatim}
%\def\ExTeX{%
%  \textrm{% Logo always with serifs
%    \ensuremath{%
%      \textstyle
%      \varepsilon_{%
%        \kern-0.15em%
%        \mathcal{X}%
%      }%
%    }%
%    \kern-.15em%
%    \TeX
%  }%
%}
%\end{verbatim}
%\end{quote}
%    \begin{macrocode}
\def\HoLogo@ExTeX#1{%
  \HoLogoFont@font{ExTeX}{rm}{%
    \ltx@mbox{%
      \HOLOGO@MathSetup
      $%
        \textstyle
        \varepsilon_{%
          \kern-0.15em%
          \HoLogoFont@font{ExTeX}{sy}{X}%
        }%
      $%
    }%
    \HOLOGO@discretionary
    \kern-.15em%
    \hologo{TeX}%
  }%
}
%    \end{macrocode}
%    \end{macro}
%    \begin{macro}{\HoLogoHtml@ExTeX}
%    \begin{macrocode}
\def\HoLogoHtml@ExTeX#1{%
  \HoLogoCss@ExTeX
  \HoLogoFont@font{ExTeX}{rm}{%
    \HOLOGO@Span{ExTeX}{%
      \ltx@mbox{%
        \HOLOGO@MathSetup
        $\textstyle\varepsilon$%
        \HOLOGO@Span{X}{$\textstyle\chi$}%
        \hologo{TeX}%
      }%
    }%
  }%
}
%    \end{macrocode}
%    \end{macro}
%    \begin{macro}{\HoLogoBkm@ExTeX}
%    \begin{macrocode}
\def\HoLogoBkm@ExTeX#1{%
  \HOLOGO@PdfdocUnicode{#1{e}{E}x}{\textepsilon\textchi}%
  \hologo{TeX}%
}
%    \end{macrocode}
%    \end{macro}
%    \begin{macro}{\HoLogoCss@ExTeX}
%    \begin{macrocode}
\def\HoLogoCss@ExTeX{%
  \Css{%
    span.HoLogo-ExTeX{%
      font-family:serif;%
    }%
  }%
  \Css{%
    span.HoLogo-ExTeX span.HoLogo-TeX{%
      margin-left:-.15em;%
    }%
  }%
  \global\let\HoLogoCss@ExTeX\relax
}
%    \end{macrocode}
%    \end{macro}
%
% \subsubsection{\hologo{MiKTeX}}
%
%    \begin{macro}{\HoLogo@MiKTeX}
%    \begin{macrocode}
\def\HoLogo@MiKTeX#1{%
  \HOLOGO@mbox{MiK}%
  \HOLOGO@discretionary
  \hologo{TeX}%
}
%    \end{macrocode}
%    \end{macro}
%    \begin{macro}{\HoLogoHtml@MiKTeX}
%    \begin{macrocode}
\let\HoLogoHtml@MiKTeX\HoLogo@MiKTeX
%    \end{macrocode}
%    \end{macro}
%
% \subsubsection{\hologo{OzTeX} and friends}
%
%    Source: \hologo{OzTeX} FAQ \cite{OzTeX}:
%    \begin{quote}
%      |\def\OzTeX{O\kern-.03em z\kern-.15em\TeX}|\\
%      (There is no kerning in OzMF, OzMP and OzTtH.)
%    \end{quote}
%
%    \begin{macro}{\HoLogo@OzTeX}
%    \begin{macrocode}
\def\HoLogo@OzTeX#1{%
  O%
  \kern-.03em %
  z%
  \kern-.15em %
  \hologo{TeX}%
}
%    \end{macrocode}
%    \end{macro}
%    \begin{macro}{\HoLogoHtml@OzTeX}
%    \begin{macrocode}
\def\HoLogoHtml@OzTeX#1{%
  \HoLogoCss@OzTeX
  \HOLOGO@Span{OzTeX}{%
    O%
    \HOLOGO@Span{z}{z}%
    \hologo{TeX}%
  }%
}
%    \end{macrocode}
%    \end{macro}
%    \begin{macro}{\HoLogoCss@OzTeX}
%    \begin{macrocode}
\def\HoLogoCss@OzTeX{%
  \Css{%
    span.HoLogo-OzTeX span.HoLogo-z{%
      margin-left:-.03em;%
      margin-right:-.15em;%
    }%
  }%
  \global\let\HoLogoCss@OzTeX\relax
}
%    \end{macrocode}
%    \end{macro}
%
%    \begin{macro}{\HoLogo@OzMF}
%    \begin{macrocode}
\def\HoLogo@OzMF#1{%
  \HOLOGO@mbox{OzMF}%
}
%    \end{macrocode}
%    \end{macro}
%    \begin{macro}{\HoLogo@OzMP}
%    \begin{macrocode}
\def\HoLogo@OzMP#1{%
  \HOLOGO@mbox{OzMP}%
}
%    \end{macrocode}
%    \end{macro}
%    \begin{macro}{\HoLogo@OzTtH}
%    \begin{macrocode}
\def\HoLogo@OzTtH#1{%
  \HOLOGO@mbox{OzTtH}%
}
%    \end{macrocode}
%    \end{macro}
%
% \subsubsection{\hologo{PCTeX}}
%
%    \begin{macro}{\HoLogo@PCTeX}
%    \begin{macrocode}
\def\HoLogo@PCTeX#1{%
  \HOLOGO@mbox{PC}%
  \hologo{TeX}%
}
%    \end{macrocode}
%    \end{macro}
%    \begin{macro}{\HoLogoHtml@PCTeX}
%    \begin{macrocode}
\let\HoLogoHtml@PCTeX\HoLogo@PCTeX
%    \end{macrocode}
%    \end{macro}
%
% \subsubsection{\hologo{PiCTeX}}
%
%    The original definitions from \xfile{pictex.tex} \cite{PiCTeX}:
%\begin{quote}
%\begin{verbatim}
%\def\PiC{%
%  P%
%  \kern-.12em%
%  \lower.5ex\hbox{I}%
%  \kern-.075em%
%  C%
%}
%\def\PiCTeX{%
%  \PiC
%  \kern-.11em%
%  \TeX
%}
%\end{verbatim}
%\end{quote}
%
%    \begin{macro}{\HoLogo@PiC}
%    \begin{macrocode}
\def\HoLogo@PiC#1{%
  P%
  \kern-.12em%
  \lower.5ex\hbox{I}%
  \kern-.075em%
  C%
  \HOLOGO@SpaceFactor
}
%    \end{macrocode}
%    \end{macro}
%    \begin{macro}{\HoLogoHtml@PiC}
%    \begin{macrocode}
\def\HoLogoHtml@PiC#1{%
  \HoLogoCss@PiC
  \HOLOGO@Span{PiC}{%
    P%
    \HOLOGO@Span{i}{I}%
    C%
  }%
}
%    \end{macrocode}
%    \end{macro}
%    \begin{macro}{\HoLogoCss@PiC}
%    \begin{macrocode}
\def\HoLogoCss@PiC{%
  \Css{%
    span.HoLogo-PiC span.HoLogo-i{%
      position:relative;%
      top:.5ex;%
      margin-left:-.12em;%
      margin-right:-.075em;%
      text-decoration:none;%
    }%
  }%
  \global\let\HoLogoCss@PiC\relax
}
%    \end{macrocode}
%    \end{macro}
%
%    \begin{macro}{\HoLogo@PiCTeX}
%    \begin{macrocode}
\def\HoLogo@PiCTeX#1{%
  \hologo{PiC}%
  \HOLOGO@discretionary
  \kern-.11em%
  \hologo{TeX}%
}
%    \end{macrocode}
%    \end{macro}
%    \begin{macro}{\HoLogoHtml@PiCTeX}
%    \begin{macrocode}
\def\HoLogoHtml@PiCTeX#1{%
  \HoLogoCss@PiCTeX
  \HOLOGO@Span{PiCTeX}{%
    \hologo{PiC}%
    \hologo{TeX}%
  }%
}
%    \end{macrocode}
%    \end{macro}
%    \begin{macro}{\HoLogoCss@PiCTeX}
%    \begin{macrocode}
\def\HoLogoCss@PiCTeX{%
  \Css{%
    span.HoLogo-PiCTeX span.HoLogo-PiC{%
      margin-right:-.11em;%
    }%
  }%
  \global\let\HoLogoCss@PiCTeX\relax
}
%    \end{macrocode}
%    \end{macro}
%
% \subsubsection{\hologo{teTeX}}
%
%    \begin{macro}{\HoLogo@teTeX}
%    \begin{macrocode}
\def\HoLogo@teTeX#1{%
  \HOLOGO@mbox{#1{t}{T}e}%
  \HOLOGO@discretionary
  \hologo{TeX}%
}
%    \end{macrocode}
%    \end{macro}
%    \begin{macro}{\HoLogoCs@teTeX}
%    \begin{macrocode}
\def\HoLogoCs@teTeX#1{#1{t}{T}dfTeX}
%    \end{macrocode}
%    \end{macro}
%    \begin{macro}{\HoLogoBkm@teTeX}
%    \begin{macrocode}
\def\HoLogoBkm@teTeX#1{%
  #1{t}{T}e\hologo{TeX}%
}
%    \end{macrocode}
%    \end{macro}
%    \begin{macro}{\HoLogoHtml@teTeX}
%    \begin{macrocode}
\let\HoLogoHtml@teTeX\HoLogo@teTeX
%    \end{macrocode}
%    \end{macro}
%
% \subsubsection{\hologo{TeX4ht}}
%
%    \begin{macro}{\HoLogo@TeX4ht}
%    \begin{macrocode}
\expandafter\def\csname HoLogo@TeX4ht\endcsname#1{%
  \HOLOGO@mbox{\hologo{TeX}4ht}%
}
%    \end{macrocode}
%    \end{macro}
%    \begin{macro}{\HoLogoHtml@TeX4ht}
%    \begin{macrocode}
\expandafter
\let\csname HoLogoHtml@TeX4ht\expandafter\endcsname
\csname HoLogo@TeX4ht\endcsname
%    \end{macrocode}
%    \end{macro}
%
%
% \subsubsection{\hologo{SageTeX}}
%
%    \begin{macro}{\HoLogo@SageTeX}
%    \begin{macrocode}
\def\HoLogo@SageTeX#1{%
  \HOLOGO@mbox{Sage}%
  \HOLOGO@discretionary
  \HOLOGO@NegativeKerning{eT,oT,To}%
  \hologo{TeX}%
}
%    \end{macrocode}
%    \end{macro}
%    \begin{macro}{\HoLogoHtml@SageTeX}
%    \begin{macrocode}
\let\HoLogoHtml@SageTeX\HoLogo@SageTeX
%    \end{macrocode}
%    \end{macro}
%
% \subsection{\hologo{METAFONT} and friends}
%
%    \begin{macro}{\HoLogo@METAFONT}
%    \begin{macrocode}
\def\HoLogo@METAFONT#1{%
  \HoLogoFont@font{METAFONT}{logo}{%
    \HOLOGO@mbox{META}%
    \HOLOGO@discretionary
    \HOLOGO@mbox{FONT}%
  }%
}
%    \end{macrocode}
%    \end{macro}
%
%    \begin{macro}{\HoLogo@METAPOST}
%    \begin{macrocode}
\def\HoLogo@METAPOST#1{%
  \HoLogoFont@font{METAPOST}{logo}{%
    \HOLOGO@mbox{META}%
    \HOLOGO@discretionary
    \HOLOGO@mbox{POST}%
  }%
}
%    \end{macrocode}
%    \end{macro}
%
%    \begin{macro}{\HoLogo@MetaFun}
%    \begin{macrocode}
\def\HoLogo@MetaFun#1{%
  \HOLOGO@mbox{Meta}%
  \HOLOGO@discretionary
  \HOLOGO@mbox{Fun}%
}
%    \end{macrocode}
%    \end{macro}
%
%    \begin{macro}{\HoLogo@MetaPost}
%    \begin{macrocode}
\def\HoLogo@MetaPost#1{%
  \HOLOGO@mbox{Meta}%
  \HOLOGO@discretionary
  \HOLOGO@mbox{Post}%
}
%    \end{macrocode}
%    \end{macro}
%
% \subsection{Others}
%
% \subsubsection{\hologo{biber}}
%
%    \begin{macro}{\HoLogo@biber}
%    \begin{macrocode}
\def\HoLogo@biber#1{%
  \HOLOGO@mbox{#1{b}{B}i}%
  \HOLOGO@discretionary
  \HOLOGO@mbox{ber}%
}
%    \end{macrocode}
%    \end{macro}
%    \begin{macro}{\HoLogoCs@biber}
%    \begin{macrocode}
\def\HoLogoCs@biber#1{#1{b}{B}iber}
%    \end{macrocode}
%    \end{macro}
%    \begin{macro}{\HoLogoBkm@biber}
%    \begin{macrocode}
\def\HoLogoBkm@biber#1{%
  #1{b}{B}iber%
}
%    \end{macrocode}
%    \end{macro}
%    \begin{macro}{\HoLogoHtml@biber}
%    \begin{macrocode}
\let\HoLogoHtml@biber\HoLogo@biber
%    \end{macrocode}
%    \end{macro}
%
% \subsubsection{\hologo{KOMAScript}}
%
%    \begin{macro}{\HoLogo@KOMAScript}
%    The definition for \hologo{KOMAScript} is taken
%    from \hologo{KOMAScript} (\xfile{scrlogo.dtx}, reformatted) \cite{scrlogo}:
%\begin{quote}
%\begin{verbatim}
%\@ifundefined{KOMAScript}{%
%  \DeclareRobustCommand{\KOMAScript}{%
%    \textsf{%
%      K\kern.05em O\kern.05emM\kern.05em A%
%      \kern.1em-\kern.1em %
%      Script%
%    }%
%  }%
%}{}
%\end{verbatim}
%\end{quote}
%    \begin{macrocode}
\def\HoLogo@KOMAScript#1{%
  \HoLogoFont@font{KOMAScript}{sf}{%
    \HOLOGO@mbox{%
      K\kern.05em%
      O\kern.05em%
      M\kern.05em%
      A%
    }%
    \kern.1em%
    \HOLOGO@hyphen
    \kern.1em%
    \HOLOGO@mbox{Script}%
  }%
}
%    \end{macrocode}
%    \end{macro}
%    \begin{macro}{\HoLogoBkm@KOMAScript}
%    \begin{macrocode}
\def\HoLogoBkm@KOMAScript#1{%
  KOMA-Script%
}
%    \end{macrocode}
%    \end{macro}
%    \begin{macro}{\HoLogoHtml@KOMAScript}
%    \begin{macrocode}
\def\HoLogoHtml@KOMAScript#1{%
  \HoLogoCss@KOMAScript
  \HoLogoFont@font{KOMAScript}{sf}{%
    \HOLOGO@Span{KOMAScript}{%
      K%
      \HOLOGO@Span{O}{O}%
      M%
      \HOLOGO@Span{A}{A}%
      \HOLOGO@Span{hyphen}{-}%
      Script%
    }%
  }%
}
%    \end{macrocode}
%    \end{macro}
%    \begin{macro}{\HoLogoCss@KOMAScript}
%    \begin{macrocode}
\def\HoLogoCss@KOMAScript{%
  \Css{%
    span.HoLogo-KOMAScript{%
      font-family:sans-serif;%
    }%
  }%
  \Css{%
    span.HoLogo-KOMAScript span.HoLogo-O{%
      padding-left:.05em;%
      padding-right:.05em;%
    }%
  }%
  \Css{%
    span.HoLogo-KOMAScript span.HoLogo-A{%
      padding-left:.05em;%
    }%
  }%
  \Css{%
    span.HoLogo-KOMAScript span.HoLogo-hyphen{%
      padding-left:.1em;%
      padding-right:.1em;%
    }%
  }%
  \global\let\HoLogoCss@KOMAScript\relax
}
%    \end{macrocode}
%    \end{macro}
%
% \subsubsection{\hologo{LyX}}
%
%    \begin{macro}{\HoLogo@LyX}
%    The definition is taken from the documentation source files
%    of \hologo{LyX}, \xfile{Intro.lyx} \cite{LyX}:
%\begin{quote}
%\begin{verbatim}
%\def\LyX{%
%  \texorpdfstring{%
%    L\kern-.1667em\lower.25em\hbox{Y}\kern-.125emX\@%
%  }{%
%    LyX%
%  }%
%}
%\end{verbatim}
%\end{quote}
%    \begin{macrocode}
\def\HoLogo@LyX#1{%
  L%
  \kern-.1667em%
  \lower.25em\hbox{Y}%
  \kern-.125em%
  X%
  \HOLOGO@SpaceFactor
}
%    \end{macrocode}
%    \end{macro}
%    \begin{macro}{\HoLogoHtml@LyX}
%    \begin{macrocode}
\def\HoLogoHtml@LyX#1{%
  \HoLogoCss@LyX
  \HOLOGO@Span{LyX}{%
    L%
    \HOLOGO@Span{y}{Y}%
    X%
  }%
}
%    \end{macrocode}
%    \end{macro}
%    \begin{macro}{\HoLogoCss@LyX}
%    \begin{macrocode}
\def\HoLogoCss@LyX{%
  \Css{%
    span.HoLogo-LyX span.HoLogo-y{%
      position:relative;%
      top:.25em;%
      margin-left:-.1667em;%
      margin-right:-.125em;%
      text-decoration:none;%
    }%
  }%
  \global\let\HoLogoCss@LyX\relax
}
%    \end{macrocode}
%    \end{macro}
%
% \subsubsection{\hologo{NTS}}
%
%    \begin{macro}{\HoLogo@NTS}
%    Definition for \hologo{NTS} can be found in
%    package \xpackage{etex\textunderscore man} for the \hologo{eTeX} manual \cite{etexman}
%    and in package \xpackage{dtklogos} \cite{dtklogos}:
%\begin{quote}
%\begin{verbatim}
%\def\NTS{%
%  \leavevmode
%  \hbox{%
%    $%
%      \cal N%
%      \kern-0.35em%
%      \lower0.5ex\hbox{$\cal T$}%
%      \kern-0.2em%
%      S%
%    $%
%  }%
%}
%\end{verbatim}
%\end{quote}
%    \begin{macrocode}
\def\HoLogo@NTS#1{%
  \HoLogoFont@font{NTS}{sy}{%
    N\/%
    \kern-.35em%
    \lower.5ex\hbox{T\/}%
    \kern-.2em%
    S\/%
  }%
  \HOLOGO@SpaceFactor
}
%    \end{macrocode}
%    \end{macro}
%
% \subsubsection{\Hologo{TTH} (\hologo{TeX} to HTML translator)}
%
%    Source: \url{http://hutchinson.belmont.ma.us/tth/}
%    In the HTML source the second `T' is printed as subscript.
%\begin{quote}
%\begin{verbatim}
%T<sub>T</sub>H
%\end{verbatim}
%\end{quote}
%    \begin{macro}{\HoLogo@TTH}
%    \begin{macrocode}
\def\HoLogo@TTH#1{%
  \ltx@mbox{%
    T\HOLOGO@SubScript{T}H%
  }%
  \HOLOGO@SpaceFactor
}
%    \end{macrocode}
%    \end{macro}
%
%    \begin{macro}{\HoLogoHtml@TTH}
%    \begin{macrocode}
\def\HoLogoHtml@TTH#1{%
  T\HCode{<sub>}T\HCode{</sub>}H%
}
%    \end{macrocode}
%    \end{macro}
%
% \subsubsection{\Hologo{HanTheThanh}}
%
%    Partial source: Package \xpackage{dtklogos}.
%    The double accent is U+1EBF (latin small letter e with circumflex
%    and acute).
%    \begin{macro}{\HoLogo@HanTheThanh}
%    \begin{macrocode}
\def\HoLogo@HanTheThanh#1{%
  \ltx@mbox{H\`an}%
  \HOLOGO@space
  \ltx@mbox{%
    Th%
    \HOLOGO@IfCharExists{"1EBF}{%
      \char"1EBF\relax
    }{%
      \^e\hbox to 0pt{\hss\raise .5ex\hbox{\'{}}}%
    }%
  }%
  \HOLOGO@space
  \ltx@mbox{Th\`anh}%
}
%    \end{macrocode}
%    \end{macro}
%    \begin{macro}{\HoLogoBkm@HanTheThanh}
%    \begin{macrocode}
\def\HoLogoBkm@HanTheThanh#1{%
  H\`an %
  Th\HOLOGO@PdfdocUnicode{\^e}{\9036\277} %
  Th\`anh%
}
%    \end{macrocode}
%    \end{macro}
%    \begin{macro}{\HoLogoHtml@HanTheThanh}
%    \begin{macrocode}
\def\HoLogoHtml@HanTheThanh#1{%
  H\`an %
  Th\HCode{&\ltx@hashchar x1ebf;} %
  Th\`anh%
}
%    \end{macrocode}
%    \end{macro}
%
% \subsection{Driver detection}
%
%    \begin{macrocode}
\HOLOGO@IfExists\InputIfFileExists{%
  \InputIfFileExists{hologo.cfg}{}{}%
}{%
  \ltx@IfUndefined{pdf@filesize}{%
    \def\HOLOGO@InputIfExists{%
      \openin\HOLOGO@temp=hologo.cfg\relax
      \ifeof\HOLOGO@temp
        \closein\HOLOGO@temp
      \else
        \closein\HOLOGO@temp
        \begingroup
          \def\x{LaTeX2e}%
        \expandafter\endgroup
        \ifx\fmtname\x
          \input{hologo.cfg}%
        \else
          \input hologo.cfg\relax
        \fi
      \fi
    }%
    \ltx@IfUndefined{newread}{%
      \chardef\HOLOGO@temp=15 %
      \def\HOLOGO@CheckRead{%
        \ifeof\HOLOGO@temp
          \HOLOGO@InputIfExists
        \else
          \ifcase\HOLOGO@temp
            \@PackageWarningNoLine{hologo}{%
              Configuration file ignored, because\MessageBreak
              a free read register could not be found%
            }%
          \else
            \begingroup
              \count\ltx@cclv=\HOLOGO@temp
              \advance\ltx@cclv by \ltx@minusone
              \edef\x{\endgroup
                \chardef\noexpand\HOLOGO@temp=\the\count\ltx@cclv
                \relax
              }%
            \x
          \fi
        \fi
      }%
    }{%
      \csname newread\endcsname\HOLOGO@temp
      \HOLOGO@InputIfExists
    }%
  }{%
    \edef\HOLOGO@temp{\pdf@filesize{hologo.cfg}}%
    \ifx\HOLOGO@temp\ltx@empty
    \else
      \ifnum\HOLOGO@temp>0 %
        \begingroup
          \def\x{LaTeX2e}%
        \expandafter\endgroup
        \ifx\fmtname\x
          \input{hologo.cfg}%
        \else
          \input hologo.cfg\relax
        \fi
      \else
        \@PackageInfoNoLine{hologo}{%
          Empty configuration file `hologo.cfg' ignored%
        }%
      \fi
    \fi
  }%
}
%    \end{macrocode}
%
%    \begin{macrocode}
\def\HOLOGO@temp#1#2{%
  \kv@define@key{HoLogoDriver}{#1}[]{%
    \begingroup
      \def\HOLOGO@temp{##1}%
      \ltx@onelevel@sanitize\HOLOGO@temp
      \ifx\HOLOGO@temp\ltx@empty
      \else
        \@PackageError{hologo}{%
          Value (\HOLOGO@temp) not permitted for option `#1'%
        }%
        \@ehc
      \fi
    \endgroup
    \def\hologoDriver{#2}%
  }%
}%
\def\HOLOGO@@temp#1#2{%
  \ifx\kv@value\relax
    \HOLOGO@temp{#1}{#1}%
  \else
    \HOLOGO@temp{#1}{#2}%
  \fi
}%
\kv@parse@normalized{%
  pdftex,%
  luatex=pdftex,%
  dvipdfm,%
  dvipdfmx=dvipdfm,%
  dvips,%
  dvipsone=dvips,%
  xdvi=dvips,%
  xetex,%
  vtex,%
}\HOLOGO@@temp
%    \end{macrocode}
%
%    \begin{macrocode}
\kv@define@key{HoLogoDriver}{driverfallback}{%
  \def\HOLOGO@DriverFallback{#1}%
}
%    \end{macrocode}
%
%    \begin{macro}{\HOLOGO@DriverFallback}
%    \begin{macrocode}
\def\HOLOGO@DriverFallback{dvips}
%    \end{macrocode}
%    \end{macro}
%
%    \begin{macro}{\hologoDriverSetup}
%    \begin{macrocode}
\def\hologoDriverSetup{%
  \let\hologoDriver\ltx@undefined
  \HOLOGO@DriverSetup
}
%    \end{macrocode}
%    \end{macro}
%
%    \begin{macro}{\HOLOGO@DriverSetup}
%    \begin{macrocode}
\def\HOLOGO@DriverSetup#1{%
  \kvsetkeys{HoLogoDriver}{#1}%
  \HOLOGO@CheckDriver
  \ltx@ifundefined{hologoDriver}{%
    \begingroup
    \edef\x{\endgroup
      \noexpand\kvsetkeys{HoLogoDriver}{\HOLOGO@DriverFallback}%
    }\x
  }{}%
  \@PackageInfoNoLine{hologo}{Using driver `\hologoDriver'}%
}
%    \end{macrocode}
%    \end{macro}
%
%    \begin{macro}{\HOLOGO@CheckDriver}
%    \begin{macrocode}
\def\HOLOGO@CheckDriver{%
  \ifpdf
    \def\hologoDriver{pdftex}%
    \let\HOLOGO@pdfliteral\pdfliteral
    \ifluatex
      \ifx\pdfextension\@undefined\else
        \protected\def\pdfliteral{\pdfextension literal}%
        \let\HOLOGO@pdfliteral\pdfliteral
      \fi
      \ltx@IfUndefined{HOLOGO@pdfliteral}{%
        \ifnum\luatexversion<36 %
        \else
          \begingroup
            \let\HOLOGO@temp\endgroup
            \ifcase0%
                \directlua{%
                  if tex.enableprimitives then %
                    tex.enableprimitives('HOLOGO@', {'pdfliteral'})%
                  else %
                    tex.print('1')%
                  end%
                }%
                \ifx\HOLOGO@pdfliteral\@undefined 1\fi%
                \relax%
              \endgroup
              \let\HOLOGO@temp\relax
              \global\let\HOLOGO@pdfliteral\HOLOGO@pdfliteral
            \fi%
          \HOLOGO@temp
        \fi
      }{}%
    \fi
    \ltx@IfUndefined{HOLOGO@pdfliteral}{%
      \@PackageWarningNoLine{hologo}{%
        Cannot find \string\pdfliteral
      }%
    }{}%
  \else
    \ifxetex
      \def\hologoDriver{xetex}%
    \else
      \ifvtex
        \def\hologoDriver{vtex}%
      \fi
    \fi
  \fi
}
%    \end{macrocode}
%    \end{macro}
%
%    \begin{macro}{\HOLOGO@WarningUnsupportedDriver}
%    \begin{macrocode}
\def\HOLOGO@WarningUnsupportedDriver#1{%
  \@PackageWarningNoLine{hologo}{%
    Logo `#1' needs driver specific macros,\MessageBreak
    but driver `\hologoDriver' is not supported.\MessageBreak
    Use a different driver or\MessageBreak
    load package `graphics' or `pgf'%
  }%
}
%    \end{macrocode}
%    \end{macro}
%
% \subsubsection{Reflect box macros}
%
%    Skip driver part if not needed.
%    \begin{macrocode}
\ltx@IfUndefined{reflectbox}{}{%
  \ltx@IfUndefined{rotatebox}{}{%
    \HOLOGO@AtEnd
  }%
}
\ltx@IfUndefined{pgftext}{}{%
  \HOLOGO@AtEnd
}
\ltx@IfUndefined{psscalebox}{}{%
  \HOLOGO@AtEnd
}
%    \end{macrocode}
%
%    \begin{macrocode}
\def\HOLOGO@temp{LaTeX2e}
\ifx\fmtname\HOLOGO@temp
  \RequirePackage{kvoptions}[2011/06/30]%
  \ProcessKeyvalOptions{HoLogoDriver}%
\fi
\HOLOGO@DriverSetup{}
%    \end{macrocode}
%
%    \begin{macro}{\HOLOGO@ReflectBox}
%    \begin{macrocode}
\def\HOLOGO@ReflectBox#1{%
  \begingroup
    \setbox\ltx@zero\hbox{\begingroup#1\endgroup}%
    \setbox\ltx@two\hbox{%
      \kern\wd\ltx@zero
      \csname HOLOGO@ScaleBox@\hologoDriver\endcsname{-1}{1}{%
        \hbox to 0pt{\copy\ltx@zero\hss}%
      }%
    }%
    \wd\ltx@two=\wd\ltx@zero
    \box\ltx@two
  \endgroup
}
%    \end{macrocode}
%    \end{macro}
%
%    \begin{macro}{\HOLOGO@PointReflectBox}
%    \begin{macrocode}
\def\HOLOGO@PointReflectBox#1{%
  \begingroup
    \setbox\ltx@zero\hbox{\begingroup#1\endgroup}%
    \setbox\ltx@two\hbox{%
      \kern\wd\ltx@zero
      \raise\ht\ltx@zero\hbox{%
        \csname HOLOGO@ScaleBox@\hologoDriver\endcsname{-1}{-1}{%
          \hbox to 0pt{\copy\ltx@zero\hss}%
        }%
      }%
    }%
    \wd\ltx@two=\wd\ltx@zero
    \box\ltx@two
  \endgroup
}
%    \end{macrocode}
%    \end{macro}
%
%    We must define all variants because of dynamic driver setup.
%    \begin{macrocode}
\def\HOLOGO@temp#1#2{#2}
%    \end{macrocode}
%
%    \begin{macro}{\HOLOGO@ScaleBox@pdftex}
%    \begin{macrocode}
\HOLOGO@temp{pdftex}{%
  \def\HOLOGO@ScaleBox@pdftex#1#2#3{%
    \HOLOGO@pdfliteral{%
      q #1 0 0 #2 0 0 cm%
    }%
    #3%
    \HOLOGO@pdfliteral{%
      Q%
    }%
  }%
}
%    \end{macrocode}
%    \end{macro}
%    \begin{macro}{\HOLOGO@ScaleBox@dvips}
%    \begin{macrocode}
\HOLOGO@temp{dvips}{%
  \def\HOLOGO@ScaleBox@dvips#1#2#3{%
    \special{ps:%
      gsave %
      currentpoint %
      currentpoint translate %
      #1 #2 scale %
      neg exch neg exch translate%
    }%
    #3%
    \special{ps:%
      currentpoint %
      grestore %
      moveto%
    }%
  }%
}
%    \end{macrocode}
%    \end{macro}
%    \begin{macro}{\HOLOGO@ScaleBox@dvipdfm}
%    \begin{macrocode}
\HOLOGO@temp{dvipdfm}{%
  \let\HOLOGO@ScaleBox@dvipdfm\HOLOGO@ScaleBox@dvips
}
%    \end{macrocode}
%    \end{macro}
%    Since \hologo{XeTeX} v0.6.
%    \begin{macro}{\HOLOGO@ScaleBox@xetex}
%    \begin{macrocode}
\HOLOGO@temp{xetex}{%
  \def\HOLOGO@ScaleBox@xetex#1#2#3{%
    \special{x:gsave}%
    \special{x:scale #1 #2}%
    #3%
    \special{x:grestore}%
  }%
}
%    \end{macrocode}
%    \end{macro}
%    \begin{macro}{\HOLOGO@ScaleBox@vtex}
%    \begin{macrocode}
\HOLOGO@temp{vtex}{%
  \def\HOLOGO@ScaleBox@vtex#1#2#3{%
    \special{r(#1,0,0,#2,0,0}%
    #3%
    \special{r)}%
  }%
}
%    \end{macrocode}
%    \end{macro}
%
%    \begin{macrocode}
\HOLOGO@AtEnd%
%</package>
%    \end{macrocode}
%
% \section{Test}
%
% \subsection{Catcode checks for loading}
%
%    \begin{macrocode}
%<*test1>
%    \end{macrocode}
%    \begin{macrocode}
\catcode`\{=1 %
\catcode`\}=2 %
\catcode`\#=6 %
\catcode`\@=11 %
\expandafter\ifx\csname count@\endcsname\relax
  \countdef\count@=255 %
\fi
\expandafter\ifx\csname @gobble\endcsname\relax
  \long\def\@gobble#1{}%
\fi
\expandafter\ifx\csname @firstofone\endcsname\relax
  \long\def\@firstofone#1{#1}%
\fi
\expandafter\ifx\csname loop\endcsname\relax
  \expandafter\@firstofone
\else
  \expandafter\@gobble
\fi
{%
  \def\loop#1\repeat{%
    \def\body{#1}%
    \iterate
  }%
  \def\iterate{%
    \body
      \let\next\iterate
    \else
      \let\next\relax
    \fi
    \next
  }%
  \let\repeat=\fi
}%
\def\RestoreCatcodes{}
\count@=0 %
\loop
  \edef\RestoreCatcodes{%
    \RestoreCatcodes
    \catcode\the\count@=\the\catcode\count@\relax
  }%
\ifnum\count@<255 %
  \advance\count@ 1 %
\repeat

\def\RangeCatcodeInvalid#1#2{%
  \count@=#1\relax
  \loop
    \catcode\count@=15 %
  \ifnum\count@<#2\relax
    \advance\count@ 1 %
  \repeat
}
\def\RangeCatcodeCheck#1#2#3{%
  \count@=#1\relax
  \loop
    \ifnum#3=\catcode\count@
    \else
      \errmessage{%
        Character \the\count@\space
        with wrong catcode \the\catcode\count@\space
        instead of \number#3%
      }%
    \fi
  \ifnum\count@<#2\relax
    \advance\count@ 1 %
  \repeat
}
\def\space{ }
\expandafter\ifx\csname LoadCommand\endcsname\relax
  \def\LoadCommand{\input hologo.sty\relax}%
\fi
\def\Test{%
  \RangeCatcodeInvalid{0}{47}%
  \RangeCatcodeInvalid{58}{64}%
  \RangeCatcodeInvalid{91}{96}%
  \RangeCatcodeInvalid{123}{255}%
  \catcode`\@=12 %
  \catcode`\\=0 %
  \catcode`\%=14 %
  \LoadCommand
  \RangeCatcodeCheck{0}{36}{15}%
  \RangeCatcodeCheck{37}{37}{14}%
  \RangeCatcodeCheck{38}{47}{15}%
  \RangeCatcodeCheck{48}{57}{12}%
  \RangeCatcodeCheck{58}{63}{15}%
  \RangeCatcodeCheck{64}{64}{12}%
  \RangeCatcodeCheck{65}{90}{11}%
  \RangeCatcodeCheck{91}{91}{15}%
  \RangeCatcodeCheck{92}{92}{0}%
  \RangeCatcodeCheck{93}{96}{15}%
  \RangeCatcodeCheck{97}{122}{11}%
  \RangeCatcodeCheck{123}{255}{15}%
  \RestoreCatcodes
}
\Test
\csname @@end\endcsname
\end
%    \end{macrocode}
%    \begin{macrocode}
%</test1>
%    \end{macrocode}
%
% \subsection{Spacefactor}
%
%    The space factor must be 1000 after a logo. If it is greater 1000
%    then the following space is a space after a sentence closing point.
%    If the space factor is smaller 1000 then an immediate following
%    dot is interpreted as abbreviation, not sentence closing point.
%
%    \begin{macrocode}
%<*test-spacefactor>
\NeedsTeXFormat{LaTeX2e}
\documentclass{article}
\usepackage{hologo}[2016/05/12]
\usepackage{kvsetkeys}
\usepackage{qstest}
\IncludeTests{*}
\LogTests{log}{*}{*}
\begin{document}
\begin{qstest}{spacefactor}{spacefactor}
\newcommand*{\Test}[1]{%
  \sbox0{%
    \hologo{#1}%
    \Expect*{1000 (#1)}*{\the\spacefactor\space(#1)}%
  }%
}%
\makeatletter
\def\TestList{}
\def\hologoEntry#1#2#3{%
  \edef\TestList{%
    \ifx\TestList\@empty
    \else
      \TestList,%
    \fi
    #1%
    \ifx\\#2\\%
    \else
      ={variant=#2}%
    \fi
  }%
}
\hologoList
\expandafter\kv@parse@normalized\expandafter{%
  \TestList
}{%
  \begingroup
    \let\@logo=\kv@key
    \ifx\kv@value\relax
    \else
      \expandafter\hologoLogoSetup\expandafter\@logo\expandafter{%
        \kv@value
      }%
    \fi
    \Test\@logo
  \endgroup
  \@gobbletwo
}
\end{qstest}
\end{document}
%</test-spacefactor>
%    \end{macrocode}
%
% \subsection{Complete list}
%
%    \begin{macrocode}
%<*test-list>
\NeedsTeXFormat{LaTeX2e}
\documentclass[12pt,a4paper]{article}
\usepackage{hologo}[2016/05/12]
\usepackage[T1]{fontenc}
\usepackage{lmodern}
\usepackage{parskip}
\usepackage[unicode]{hyperref}[2011/09/28]
\usepackage{bookmark}[2011/09/19]
\bookmarksetup{%
  numbered,%
  open,%
  openlevel=2,%
}
\renewcommand*{\contentsname}{List of logos}
\begin{document}
\tableofcontents
\def\TestFont#1#2#3#4#5#6{%
  \begingroup
    \usefont{#3}{#4}{#5}{#6}%
    \HologoVariant{#1}{#2}/\hologoVariant{#1}{#2}%
    \quad
    \begingroup\scriptsize\hologoVariant{#1}{#2}\endgroup
    \quad
  \endgroup
  (#3/#4/#5/#6)%
  \par
}
\makeatletter
\def\hologoEntry#1#2#3{%
  \section{%
    \HologoVariant{#1}{#2}/\hologoVariant{#1}{#2} %
    {[#1\ifx\\#2\\\else\space(#2)\fi]}% hash-ok
  }% braces around [] because of bug in tex4ht
  \begingroup
    \hypersetup{unicode=false}%
    \bookmark[%
      dest=\@currentHref,%
      rellevel=1,%
      keeplevel,%
    ]{%
      \HologoVariant{#1}{#2}/\hologoVariant{#1}{#2} %
      (PDFDocEncoding)%
    }%
  \endgroup
  \TestFont{#1}{#2}{OT1}{cmr}{m}{n}%
  \TestFont{#1}{#2}{OT1}{cmss}{m}{n}%
  \TestFont{#1}{#2}{OT1}{cmr}{b}{n}%
  \TestFont{#1}{#2}{OT1}{cmr}{m}{it}%
  \TestFont{#1}{#2}{OT1}{cmtt}{m}{n}%
  \TestFont{#1}{#2}{T1}{lmr}{m}{n}%
  \TestFont{#1}{#2}{T1}{lmss}{m}{n}%
  \TestFont{#1}{#2}{T1}{lmr}{b}{n}%
  \TestFont{#1}{#2}{T1}{lmr}{m}{it}%
  \TestFont{#1}{#2}{T1}{lmtt}{m}{n}%
  \TestFont{#1}{#2}{T1}{lmvtt}{m}{n}%
  \TestFont{#1}{#2}{T1}{qtm}{m}{n}%
  \TestFont{#1}{#2}{T1}{qhv}{m}{n}%
  \TestFont{#1}{#2}{T1}{qtm}{b}{n}%
  \TestFont{#1}{#2}{T1}{qtm}{m}{it}%
  \TestFont{#1}{#2}{T1}{qcr}{m}{n}%
  \newpage
}
\makeatother
\hologoList
\end{document}
%</test-list>
%    \end{macrocode}
%
% \section{Installation}
%
% \subsection{Download}
%
% \paragraph{Package.} This package is available on
% CTAN\footnote{\url{ftp://ftp.ctan.org/tex-archive/}}:
% \begin{description}
% \item[\CTAN{macros/latex/contrib/oberdiek/hologo.dtx}] The source file.
% \item[\CTAN{macros/latex/contrib/oberdiek/hologo.pdf}] Documentation.
% \end{description}
%
%
% \paragraph{Bundle.} All the packages of the bundle `oberdiek'
% are also available in a TDS compliant ZIP archive. There
% the packages are already unpacked and the documentation files
% are generated. The files and directories obey the TDS standard.
% \begin{description}
% \item[\CTAN{install/macros/latex/contrib/oberdiek.tds.zip}]
% \end{description}
% \emph{TDS} refers to the standard ``A Directory Structure
% for \TeX\ Files'' (\CTAN{tds/tds.pdf}). Directories
% with \xfile{texmf} in their name are usually organized this way.
%
% \subsection{Bundle installation}
%
% \paragraph{Unpacking.} Unpack the \xfile{oberdiek.tds.zip} in the
% TDS tree (also known as \xfile{texmf} tree) of your choice.
% Example (linux):
% \begin{quote}
%   |unzip oberdiek.tds.zip -d ~/texmf|
% \end{quote}
%
% \paragraph{Script installation.}
% Check the directory \xfile{TDS:scripts/oberdiek/} for
% scripts that need further installation steps.
% Package \xpackage{attachfile2} comes with the Perl script
% \xfile{pdfatfi.pl} that should be installed in such a way
% that it can be called as \texttt{pdfatfi}.
% Example (linux):
% \begin{quote}
%   |chmod +x scripts/oberdiek/pdfatfi.pl|\\
%   |cp scripts/oberdiek/pdfatfi.pl /usr/local/bin/|
% \end{quote}
%
% \subsection{Package installation}
%
% \paragraph{Unpacking.} The \xfile{.dtx} file is a self-extracting
% \docstrip\ archive. The files are extracted by running the
% \xfile{.dtx} through \plainTeX:
% \begin{quote}
%   \verb|tex hologo.dtx|
% \end{quote}
%
% \paragraph{TDS.} Now the different files must be moved into
% the different directories in your installation TDS tree
% (also known as \xfile{texmf} tree):
% \begin{quote}
% \def\t{^^A
% \begin{tabular}{@{}>{\ttfamily}l@{ $\rightarrow$ }>{\ttfamily}l@{}}
%   hologo.sty & tex/generic/oberdiek/hologo.sty\\
%   hologo.pdf & doc/latex/oberdiek/hologo.pdf\\
%   example/hologo-example.tex & doc/latex/oberdiek/example/hologo-example.tex\\
%   test/hologo-test1.tex & doc/latex/oberdiek/test/hologo-test1.tex\\
%   test/hologo-test-spacefactor.tex & doc/latex/oberdiek/test/hologo-test-spacefactor.tex\\
%   test/hologo-test-list.tex & doc/latex/oberdiek/test/hologo-test-list.tex\\
%   hologo.dtx & source/latex/oberdiek/hologo.dtx\\
% \end{tabular}^^A
% }^^A
% \sbox0{\t}^^A
% \ifdim\wd0>\linewidth
%   \begingroup
%     \advance\linewidth by\leftmargin
%     \advance\linewidth by\rightmargin
%   \edef\x{\endgroup
%     \def\noexpand\lw{\the\linewidth}^^A
%   }\x
%   \def\lwbox{^^A
%     \leavevmode
%     \hbox to \linewidth{^^A
%       \kern-\leftmargin\relax
%       \hss
%       \usebox0
%       \hss
%       \kern-\rightmargin\relax
%     }^^A
%   }^^A
%   \ifdim\wd0>\lw
%     \sbox0{\small\t}^^A
%     \ifdim\wd0>\linewidth
%       \ifdim\wd0>\lw
%         \sbox0{\footnotesize\t}^^A
%         \ifdim\wd0>\linewidth
%           \ifdim\wd0>\lw
%             \sbox0{\scriptsize\t}^^A
%             \ifdim\wd0>\linewidth
%               \ifdim\wd0>\lw
%                 \sbox0{\tiny\t}^^A
%                 \ifdim\wd0>\linewidth
%                   \lwbox
%                 \else
%                   \usebox0
%                 \fi
%               \else
%                 \lwbox
%               \fi
%             \else
%               \usebox0
%             \fi
%           \else
%             \lwbox
%           \fi
%         \else
%           \usebox0
%         \fi
%       \else
%         \lwbox
%       \fi
%     \else
%       \usebox0
%     \fi
%   \else
%     \lwbox
%   \fi
% \else
%   \usebox0
% \fi
% \end{quote}
% If you have a \xfile{docstrip.cfg} that configures and enables \docstrip's
% TDS installing feature, then some files can already be in the right
% place, see the documentation of \docstrip.
%
% \subsection{Refresh file name databases}
%
% If your \TeX~distribution
% (\teTeX, \mikTeX, \dots) relies on file name databases, you must refresh
% these. For example, \teTeX\ users run \verb|texhash| or
% \verb|mktexlsr|.
%
% \subsection{Some details for the interested}
%
% \paragraph{Attached source.}
%
% The PDF documentation on CTAN also includes the
% \xfile{.dtx} source file. It can be extracted by
% AcrobatReader 6 or higher. Another option is \textsf{pdftk},
% e.g. unpack the file into the current directory:
% \begin{quote}
%   \verb|pdftk hologo.pdf unpack_files output .|
% \end{quote}
%
% \paragraph{Unpacking with \LaTeX.}
% The \xfile{.dtx} chooses its action depending on the format:
% \begin{description}
% \item[\plainTeX:] Run \docstrip\ and extract the files.
% \item[\LaTeX:] Generate the documentation.
% \end{description}
% If you insist on using \LaTeX\ for \docstrip\ (really,
% \docstrip\ does not need \LaTeX), then inform the autodetect routine
% about your intention:
% \begin{quote}
%   \verb|latex \let\install=y\input{hologo.dtx}|
% \end{quote}
% Do not forget to quote the argument according to the demands
% of your shell.
%
% \paragraph{Generating the documentation.}
% You can use both the \xfile{.dtx} or the \xfile{.drv} to generate
% the documentation. The process can be configured by the
% configuration file \xfile{ltxdoc.cfg}. For instance, put this
% line into this file, if you want to have A4 as paper format:
% \begin{quote}
%   \verb|\PassOptionsToClass{a4paper}{article}|
% \end{quote}
% An example follows how to generate the
% documentation with pdf\LaTeX:
% \begin{quote}
%\begin{verbatim}
%pdflatex hologo.dtx
%makeindex -s gind.ist hologo.idx
%pdflatex hologo.dtx
%makeindex -s gind.ist hologo.idx
%pdflatex hologo.dtx
%\end{verbatim}
% \end{quote}
%
% \section{Catalogue}
%
% The following XML file can be used as source for the
% \href{http://mirror.ctan.org/help/Catalogue/catalogue.html}{\TeX\ Catalogue}.
% The elements \texttt{caption} and \texttt{description} are imported
% from the original XML file from the Catalogue.
% The name of the XML file in the Catalogue is \xfile{hologo.xml}.
%    \begin{macrocode}
%<*catalogue>
<?xml version='1.0' encoding='us-ascii'?>
<!DOCTYPE entry SYSTEM 'catalogue.dtd'>
<entry datestamp='$Date$' modifier='$Author$' id='hologo'>
  <name>hologo</name>
  <caption>A collection of logos with bookmark support.</caption>
  <authorref id='auth:oberdiek'/>
  <copyright owner='Heiko Oberdiek' year='2010-2012'/>
  <license type='lppl1.3'/>
  <version number='1.10'/>
  <description>
    The package defines a single command <tt>\hologo</tt>, whose
    argument is the usual case-confused ASCII version of the logo.
    The command is bookmark-enabled, so that every logo becomes
    available in bookmarks without further work.
    <p/>
    The package is part of the <xref refid='oberdiek'>oberdiek</xref>
    bundle.
  </description>
  <documentation details='Package documentation'
      href='ctan:/macros/latex/contrib/oberdiek/hologo.pdf'/>
  <ctan file='true' path='/macros/latex/contrib/oberdiek/hologo.dtx'/>
  <miktex location='oberdiek'/>
  <texlive location='oberdiek'/>
  <install path='/macros/latex/contrib/oberdiek/oberdiek.tds.zip'/>
</entry>
%</catalogue>
%    \end{macrocode}
%
% \begin{thebibliography}{9}
% \raggedright
%
% \bibitem{btxdoc}
% Oren Patashnik,
% \textit{\hologo{BibTeX}ing},
% 1988-02-08.\\
% \CTAN{biblio/bibtex/base/}
%
% \bibitem{dtklogos}
% Gerd Neugebauer, DANTE,
% \textit{Package \xpackage{dtklogos}},
% 2011-04-25.\\
% \CTAN{usergrps/dante/dtk/dtklogos.sty}
%
% \bibitem{etexman}
% The \hologo{NTS} Team,
% \textit{The \hologo{eTeX} manual},
% 1998-02.\\
% \CTAN{systems/e-tex/v2/doc/}
%
% \bibitem{ExTeX-FAQ}
% The \hologo{ExTeX} group,
% \textit{\hologo{ExTeX}: FAQ -- How is \hologo{ExTeX} typeset?},
% 2007-04-14.\\
% \url{http://www.extex.org/documentation/faq.html}
%
% \bibitem{LyX}
% %@MISC{ LyX,
% %  title = {{LyX 2.0.0 -- The Document Processor [Computer software and manual]}},
% %  author = {{The LyX Team}},
% %  howpublished = {Internet: http://www.lyx.org},
% %  year = {2011-05-08},
% %  note = {Retrieved May 10, 2011, from http://www.lyx.org},
% %  url = {http://www.lyx.org/}
% %}
% The \hologo{LyX} Team,
% \textit{\hologo{LyX} -- The Document Processor},
% 2011-05-08.\\
% \url{http://www.lyx.org/}
%
% \bibitem{OzTeX}
% Andrew Trevorrow,
% \hologo{OzTeX} FAQ: What is the correct way to typeset ``\hologo{OzTeX}''?,
% 2011-09-15 (visited).
% \url{http://www.trevorrow.com/oztex/ozfaq.html#oztex-logo}
%
% \bibitem{PiCTeX}
% Michael Wichura,
% \textit{The \hologo{PiCTeX} macro package},
% 1987-09-21.
% \CTAN{graphics/pictex/}
%
% \bibitem{scrlogo}
% Markus Kohm,
% \textit{\hologo{KOMAScript} Datei \xfile{scrlogo.dtx}},
% 2009-01-30.\\
% \CTAN{install/macros/latex/contrib/komascript.tds.zip}
%
% \end{thebibliography}
%
% \begin{History}
%   \begin{Version}{2010/04/08 v1.0}
%   \item
%     The first version.
%   \end{Version}
%   \begin{Version}{2010/04/16 v1.1}
%   \item
%     \cs{Hologo} added for support of logos at start of a sentence.
%   \item
%     \cs{hologoSetup} and \cs{hologoLogoSetup} added.
%   \item
%     Options \xoption{break}, \xoption{hyphenbreak}, \xoption{spacebreak}
%     added.
%   \item
%     Variant support added by option \xoption{variant}.
%   \end{Version}
%   \begin{Version}{2010/04/24 v1.2}
%   \item
%     \hologo{LaTeX3} added.
%   \item
%     \hologo{VTeX} added.
%   \end{Version}
%   \begin{Version}{2010/11/21 v1.3}
%   \item
%     \hologo{iniTeX}, \hologo{virTeX} added.
%   \end{Version}
%   \begin{Version}{2011/03/25 v1.4}
%   \item
%     \hologo{ConTeXt} with variants added.
%   \item
%     Option \xoption{discretionarybreak} added as refinement for
%     option \xoption{break}.
%   \end{Version}
%   \begin{Version}{2011/04/21 v1.5}
%   \item
%     Wrong TDS directory for test files fixed.
%   \end{Version}
%   \begin{Version}{2011/10/01 v1.6}
%   \item
%     Support for package \xpackage{tex4ht} added.
%   \item
%     Support for \cs{csname} added if \cs{ifincsname} is available.
%   \item
%     New logos:
%     \hologo{(La)TeX},
%     \hologo{biber},
%     \hologo{BibTeX} (\xoption{sc}, \xoption{sf}),
%     \hologo{emTeX},
%     \hologo{ExTeX},
%     \hologo{KOMAScript},
%     \hologo{La},
%     \hologo{LyX},
%     \hologo{MiKTeX},
%     \hologo{NTS},
%     \hologo{OzMF},
%     \hologo{OzMP},
%     \hologo{OzTeX},
%     \hologo{OzTtH},
%     \hologo{PCTeX},
%     \hologo{PiC},
%     \hologo{PiCTeX},
%     \hologo{METAFONT},
%     \hologo{MetaFun},
%     \hologo{METAPOST},
%     \hologo{MetaPost},
%     \hologo{SLiTeX} (\xoption{lift}, \xoption{narrow}, \xoption{simple}),
%     \hologo{SliTeX} (\xoption{narrow}, \xoption{simple}, \xoption{lift}),
%     \hologo{teTeX}.
%   \item
%     Fixes:
%     \hologo{iniTeX},
%     \hologo{pdfLaTeX},
%     \hologo{pdfTeX},
%     \hologo{virTeX}.
%   \item
%     \cs{hologoFontSetup} and \cs{hologoLogoFontSetup} added.
%   \item
%     \cs{hologoVariant} and \cs{HologoVariant} added.
%   \end{Version}
%   \begin{Version}{2011/11/22 v1.7}
%   \item
%     New logos:
%     \hologo{BibTeX8},
%     \hologo{LaTeXML},
%     \hologo{SageTeX},
%     \hologo{TeX4ht},
%     \hologo{TTH}.
%   \item
%     \hologo{Xe} and friends: Driver stuff fixed.
%   \item
%     \hologo{Xe} and friends: Support for italic added.
%   \item
%     \hologo{Xe} and friends: Package support for \xpackage{pgf}
%     and \xpackage{pstricks} added.
%   \end{Version}
%   \begin{Version}{2011/11/29 v1.8}
%   \item
%     New logos:
%     \hologo{HanTheThanh}.
%   \end{Version}
%   \begin{Version}{2011/12/21 v1.9}
%   \item
%     Patch for package \xpackage{ifxetex} added for the case that
%     \cs{newif} is undefined in \hologo{iniTeX}.
%   \item
%     Some fixes for \hologo{iniTeX}.
%   \end{Version}
%   \begin{Version}{2012/04/26 v1.10}
%   \item
%     Fix in bookmark version of logo ``\hologo{HanTheThanh}''.
%   \end{Version}
%   \begin{Version}{2016/05/12 v1.11}
%   \item
%     Update HOLOGO@IfCharExists (previously in texlive)
%   \item define pdfliteral in current luatex.
%   \end{Version}
% \end{History}
%
% \PrintIndex
%
% \Finale
\endinput
%
        \else
          \input hologo.cfg\relax
        \fi
      \fi
    }%
    \ltx@IfUndefined{newread}{%
      \chardef\HOLOGO@temp=15 %
      \def\HOLOGO@CheckRead{%
        \ifeof\HOLOGO@temp
          \HOLOGO@InputIfExists
        \else
          \ifcase\HOLOGO@temp
            \@PackageWarningNoLine{hologo}{%
              Configuration file ignored, because\MessageBreak
              a free read register could not be found%
            }%
          \else
            \begingroup
              \count\ltx@cclv=\HOLOGO@temp
              \advance\ltx@cclv by \ltx@minusone
              \edef\x{\endgroup
                \chardef\noexpand\HOLOGO@temp=\the\count\ltx@cclv
                \relax
              }%
            \x
          \fi
        \fi
      }%
    }{%
      \csname newread\endcsname\HOLOGO@temp
      \HOLOGO@InputIfExists
    }%
  }{%
    \edef\HOLOGO@temp{\pdf@filesize{hologo.cfg}}%
    \ifx\HOLOGO@temp\ltx@empty
    \else
      \ifnum\HOLOGO@temp>0 %
        \begingroup
          \def\x{LaTeX2e}%
        \expandafter\endgroup
        \ifx\fmtname\x
          % \iffalse meta-comment
%
% File: hologo.dtx
% Version: 2016/05/12 v1.11
% Info: A logo collection with bookmark support
%
% Copyright (C) 2010-2012 by
%    Heiko Oberdiek <heiko.oberdiek at googlemail.com>
%
% This work may be distributed and/or modified under the
% conditions of the LaTeX Project Public License, either
% version 1.3c of this license or (at your option) any later
% version. This version of this license is in
%    http://www.latex-project.org/lppl/lppl-1-3c.txt
% and the latest version of this license is in
%    http://www.latex-project.org/lppl.txt
% and version 1.3 or later is part of all distributions of
% LaTeX version 2005/12/01 or later.
%
% This work has the LPPL maintenance status "maintained".
%
% This Current Maintainer of this work is Heiko Oberdiek.
%
% The Base Interpreter refers to any `TeX-Format',
% because some files are installed in TDS:tex/generic//.
%
% This work consists of the main source file hologo.dtx
% and the derived files
%    hologo.sty, hologo.pdf, hologo.ins, hologo.drv, hologo-example.tex,
%    hologo-test1.tex, hologo-test-spacefactor.tex,
%    hologo-test-list.tex.
%
% Distribution:
%    CTAN:macros/latex/contrib/oberdiek/hologo.dtx
%    CTAN:macros/latex/contrib/oberdiek/hologo.pdf
%
% Unpacking:
%    (a) If hologo.ins is present:
%           tex hologo.ins
%    (b) Without hologo.ins:
%           tex hologo.dtx
%    (c) If you insist on using LaTeX
%           latex \let\install=y\input{hologo.dtx}
%        (quote the arguments according to the demands of your shell)
%
% Documentation:
%    (a) If hologo.drv is present:
%           latex hologo.drv
%    (b) Without hologo.drv:
%           latex hologo.dtx; ...
%    The class ltxdoc loads the configuration file ltxdoc.cfg
%    if available. Here you can specify further options, e.g.
%    use A4 as paper format:
%       \PassOptionsToClass{a4paper}{article}
%
%    Programm calls to get the documentation (example):
%       pdflatex hologo.dtx
%       makeindex -s gind.ist hologo.idx
%       pdflatex hologo.dtx
%       makeindex -s gind.ist hologo.idx
%       pdflatex hologo.dtx
%
% Installation:
%    TDS:tex/generic/oberdiek/hologo.sty
%    TDS:doc/latex/oberdiek/hologo.pdf
%    TDS:doc/latex/oberdiek/example/hologo-example.tex
%    TDS:doc/latex/oberdiek/test/hologo-test1.tex
%    TDS:doc/latex/oberdiek/test/hologo-test-spacefactor.tex
%    TDS:doc/latex/oberdiek/test/hologo-test-list.tex
%    TDS:source/latex/oberdiek/hologo.dtx
%
%<*ignore>
\begingroup
  \catcode123=1 %
  \catcode125=2 %
  \def\x{LaTeX2e}%
\expandafter\endgroup
\ifcase 0\ifx\install y1\fi\expandafter
         \ifx\csname processbatchFile\endcsname\relax\else1\fi
         \ifx\fmtname\x\else 1\fi\relax
\else\csname fi\endcsname
%</ignore>
%<*install>
\input docstrip.tex
\Msg{************************************************************************}
\Msg{* Installation}
\Msg{* Package: hologo 2016/05/12 v1.11 A logo collection with bookmark support (HO)}
\Msg{************************************************************************}

\keepsilent
\askforoverwritefalse

\let\MetaPrefix\relax
\preamble

This is a generated file.

Project: hologo
Version: 2016/05/12 v1.11

Copyright (C) 2010-2012 by
   Heiko Oberdiek <heiko.oberdiek at googlemail.com>

This work may be distributed and/or modified under the
conditions of the LaTeX Project Public License, either
version 1.3c of this license or (at your option) any later
version. This version of this license is in
   http://www.latex-project.org/lppl/lppl-1-3c.txt
and the latest version of this license is in
   http://www.latex-project.org/lppl.txt
and version 1.3 or later is part of all distributions of
LaTeX version 2005/12/01 or later.

This work has the LPPL maintenance status "maintained".

This Current Maintainer of this work is Heiko Oberdiek.

The Base Interpreter refers to any `TeX-Format',
because some files are installed in TDS:tex/generic//.

This work consists of the main source file hologo.dtx
and the derived files
   hologo.sty, hologo.pdf, hologo.ins, hologo.drv, hologo-example.tex,
   hologo-test1.tex, hologo-test-spacefactor.tex,
   hologo-test-list.tex.

\endpreamble
\let\MetaPrefix\DoubleperCent

\generate{%
  \file{hologo.ins}{\from{hologo.dtx}{install}}%
  \file{hologo.drv}{\from{hologo.dtx}{driver}}%
  \usedir{tex/generic/oberdiek}%
  \file{hologo.sty}{\from{hologo.dtx}{package}}%
  \usedir{doc/latex/oberdiek/example}%
  \file{hologo-example.tex}{\from{hologo.dtx}{example}}%
  \usedir{doc/latex/oberdiek/test}%
  \file{hologo-test1.tex}{\from{hologo.dtx}{test1}}%
  \file{hologo-test-spacefactor.tex}{\from{hologo.dtx}{test-spacefactor}}%
  \file{hologo-test-list.tex}{\from{hologo.dtx}{test-list}}%
  \nopreamble
  \nopostamble
  \usedir{source/latex/oberdiek/catalogue}%
  \file{hologo.xml}{\from{hologo.dtx}{catalogue}}%
}

\catcode32=13\relax% active space
\let =\space%
\Msg{************************************************************************}
\Msg{*}
\Msg{* To finish the installation you have to move the following}
\Msg{* file into a directory searched by TeX:}
\Msg{*}
\Msg{*     hologo.sty}
\Msg{*}
\Msg{* To produce the documentation run the file `hologo.drv'}
\Msg{* through LaTeX.}
\Msg{*}
\Msg{* Happy TeXing!}
\Msg{*}
\Msg{************************************************************************}

\endbatchfile
%</install>
%<*ignore>
\fi
%</ignore>
%<*driver>
\NeedsTeXFormat{LaTeX2e}
\ProvidesFile{hologo.drv}%
  [2016/05/12 v1.11 A logo collection with bookmark support (HO)]%
\documentclass{ltxdoc}
\usepackage{holtxdoc}[2011/11/22]
\usepackage{hologo}[2016/05/12]
\usepackage{longtable}
\usepackage{array}
\usepackage{paralist}
%\usepackage[T1]{fontenc}
%\usepackage{lmodern}
\begin{document}
  \DocInput{hologo.dtx}%
\end{document}
%</driver>
% \fi
%
%
% \CharacterTable
%  {Upper-case    \A\B\C\D\E\F\G\H\I\J\K\L\M\N\O\P\Q\R\S\T\U\V\W\X\Y\Z
%   Lower-case    \a\b\c\d\e\f\g\h\i\j\k\l\m\n\o\p\q\r\s\t\u\v\w\x\y\z
%   Digits        \0\1\2\3\4\5\6\7\8\9
%   Exclamation   \!     Double quote  \"     Hash (number) \#
%   Dollar        \$     Percent       \%     Ampersand     \&
%   Acute accent  \'     Left paren    \(     Right paren   \)
%   Asterisk      \*     Plus          \+     Comma         \,
%   Minus         \-     Point         \.     Solidus       \/
%   Colon         \:     Semicolon     \;     Less than     \<
%   Equals        \=     Greater than  \>     Question mark \?
%   Commercial at \@     Left bracket  \[     Backslash     \\
%   Right bracket \]     Circumflex    \^     Underscore    \_
%   Grave accent  \`     Left brace    \{     Vertical bar  \|
%   Right brace   \}     Tilde         \~}
%
% \GetFileInfo{hologo.drv}
%
% \title{The \xpackage{hologo} package}
% \date{2016/05/12 v1.11}
% \author{Heiko Oberdiek\\\xemail{heiko.oberdiek at googlemail.com}}
%
% \maketitle
%
% \begin{abstract}
% This package starts a collection of logos with support for bookmarks
% strings.
% \end{abstract}
%
% \tableofcontents
%
% \section{Documentation}
%
% \subsection{Logo macros}
%
% \begin{declcs}{hologo} \M{name}
% \end{declcs}
% Macro \cs{hologo} sets the logo with name \meta{name}.
% The following table shows the supported names.
%
% \begingroup
%   \def\hologoEntry#1#2#3{^^A
%     #1&#2&\hologoLogoSetup{#1}{variant=#2}\hologo{#1}&#3\tabularnewline
%   }
%   \begin{longtable}{>{\ttfamily}l>{\ttfamily}lll}
%     \rmfamily\bfseries{name} & \rmfamily\bfseries variant
%     & \bfseries logo & \bfseries since\\
%     \hline
%     \endhead
%     \hologoList
%   \end{longtable}
% \endgroup
%
% \begin{declcs}{Hologo} \M{name}
% \end{declcs}
% Macro \cs{Hologo} starts the logo \meta{name} with an uppercase
% letter. As an exception small greek letters are not converted
% to uppercase. Examples, see \hologo{eTeX} and \hologo{ExTeX}.
%
% \subsection{Setup macros}
%
% The package does not support package options, but the following
% setup macros can be used to set options.
%
% \begin{declcs}{hologoSetup} \M{key value list}
% \end{declcs}
% Macro \cs{hologoSetup} sets global options.
%
% \begin{declcs}{hologoLogoSetup} \M{logo} \M{key value list}
% \end{declcs}
% Some options can also be used to configure a logo.
% These settings take precedence over global option settings.
%
% \subsection{Options}\label{sec:options}
%
% There are boolean and string options:
% \begin{description}
% \item[Boolean option:]
% It takes |true| or |false|
% as value. If the value is omitted, then |true| is used.
% \item[String option:]
% A value must be given as string. (But the string might be empty.)
% \end{description}
% The following options can be used both in \cs{hologoSetup}
% and \cs{hologoLogoSetup}:
% \begin{description}
% \def\entry#1{\item[\xoption{#1}:]}
% \entry{break}
%   enables or disables line breaks inside the logo. This setting is
%   refined by options \xoption{hyphenbreak}, \xoption{spacebreak}
%   or \xoption{discretionarybreak}.
%   Default is |false|.
% \entry{hyphenbreak}
%   enables or disables the line break right after the hyphen character.
% \entry{spacebreak}
%   enables or disables line breaks at space characters.
% \entry{discretionarybreak}
%   enables or disables line breaks at hyphenation points
%   (inserted by \cs{-}).
% \end{description}
% Macro \cs{hologoLogoSetup} also knows:
% \begin{description}
% \item[\xoption{variant}:]
%   This is a string option. It specifies a variant of a logo that
%   must exist. An empty string selects the package default variant.
% \end{description}
% Example:
% \begin{quote}
%   |\hologoSetup{break=false}|\\
%   |\hologoLogoSetup{plainTeX}{variant=hyphen,hyphenbreak}|\\
%   Then ``plain-\TeX'' contains one break point after the hyphen.
% \end{quote}
%
% \subsection{Driver options}
%
% Sometimes graphical operations are needed to construct some
% glyphs (e.g.\ \hologo{XeTeX}). If package \xpackage{graphics}
% or package \xpackage{pgf} are found, then the macros are taken
% from there. Otherwise the packge defines its own operations
% and therefore needs the driver information. Many drivers are
% detected automatically (\hologo{pdfTeX}/\hologo{LuaTeX}
% in PDF mode, \hologo{XeTeX}, \hologo{VTeX}). These have precedence
% over a driver option. The driver can be given as package option
% or using \cs{hologoDriverSetup}.
% The following list contains the recognized driver options:
% \begin{itemize}
% \item \xoption{pdftex}, \xoption{luatex}
% \item \xoption{dvipdfm}, \xoption{dvipdfmx}
% \item \xoption{dvips}, \xoption{dvipsone}, \xoption{xdvi}
% \item \xoption{xetex}
% \item \xoption{vtex}
% \end{itemize}
% The left driver of a line is the driver name that is used internally.
% The following names are aliases for drivers that use the
% same method. Therefore the entry in the \xext{log} file for
% the used driver prints the internally used driver name.
% \begin{description}
% \item[\xoption{driverfallback}:]
%   This option expects a driver that is used,
%   if the driver could not be detected automatically.
% \end{description}
%
% \begin{declcs}{hologoDriverSetup} \M{driver option}
% \end{declcs}
% The driver can also be configured after package loading
% using \cs{hologoDriverSetup}, also the way for \hologo{plainTeX}
% to setup the driver.
%
% \subsection{Font setup}
%
% Some logos require a special font, but should also be usable by
% \hologo{plainTeX}. Therefore the package provides some ways
% to influence the font settings. The options below
% take font settings as values. Both font commands
% such as \cs{sffamily} and macros that take one argument
% like \cs{textsf} can be used.
%
% \begin{declcs}{hologoFontSetup} \M{key value list}
% \end{declcs}
% Macro \cs{hologoFontSetup} sets the fonts for all logos.
% Supported keys:
% \begin{description}
% \def\entry#1{\item[\xoption{#1}:]}
% \entry{general}
%   This font is used for all logos. The default is empty.
%   That means no special font is used.
% \entry{bibsf}
%   This font is used for
%   {\hologoLogoSetup{BibTeX}{variant=sf}\hologo{BibTeX}}
%   with variant \xoption{sf}.
% \entry{rm}
%   This font is a serif font. It is used for \hologo{ExTeX}.
% \entry{sc}
%   This font specifies a small caps font. It is used for
%   {\hologoLogoSetup{BibTeX}{variant=sc}\hologo{BibTeX}}
%   with variant \xoption{sc}.
% \entry{sf}
%   This font specifies a sans serif font. The default
%   is \cs{sffamily}, then \cs{sf} is tried. Otherwise
%   a warning is given. It is used by \hologo{KOMAScript}.
% \entry{sy}
%   This is the font for math symbols (e.g. cmsy).
%   It is used by \hologo{AmS}, \hologo{NTS}, \hologo{ExTeX}.
% \entry{logo}
%   \hologo{METAFONT} and \hologo{METAPOST} are using that font.
%   In \hologo{LaTeX} \cs{logofamily} is used and
%   the definitions of package \xpackage{mflogo} are used
%   if the package is not loaded.
%   Otherwise the \cs{tenlogo} is used and defined
%   if it does not already exists.
% \end{description}
%
% \begin{declcs}{hologoLogoFontSetup} \M{logo} \M{key value list}
% \end{declcs}
% Fonts can also be set for a logo or logo component separately,
% see the following list.
% The keys are the same as for \cs{hologoFontSetup}.
%
% \begin{longtable}{>{\ttfamily}l>{\sffamily}ll}
%   \meta{logo} & keys & result\\
%   \hline
%   \endhead
%   BibTeX & bibsf & {\hologoLogoSetup{BibTeX}{variant=sf}\hologo{BibTeX}}\\[.5ex]
%   BibTeX & sc & {\hologoLogoSetup{BibTeX}{variant=sc}\hologo{BibTeX}}\\[.5ex]
%   ExTeX & rm & \hologo{ExTeX}\\
%   SliTeX & rm & \hologo{SliTeX}\\[.5ex]
%   AmS & sy & \hologo{AmS}\\
%   ExTeX & sy & \hologo{ExTeX}\\
%   NTS & sy & \hologo{NTS}\\[.5ex]
%   KOMAScript & sf & \hologo{KOMAScript}\\[.5ex]
%   METAFONT & logo & \hologo{METAFONT}\\
%   METAPOST & logo & \hologo{METAPOST}\\[.5ex]
%   SliTeX & sc \hologo{SliTeX}
% \end{longtable}
%
% \subsubsection{Font order}
%
% For all logos the font \xoption{general} is applied first.
% Example:
%\begin{quote}
%|\hologoFontSetup{general=\color{red}}|
%\end{quote}
% will print red logos.
% Then if the font uses a special font \xoption{sf}, for example,
% the font is applied that is setup by \cs{hologoLogoFontSetup}.
% If this font is not setup, then the common font setup
% by \cs{hologoFontSetup} is used. Otherwise a warning is given,
% that there is no font configured.
%
% \subsection{Additional user macros}
%
% Usually a variant of a logo is configured by using
% \cs{hologoLogoSetup}, because it is bad style to mix
% different variants of the same logo in the same text.
% There the following macros are a convenience for testing.
%
% \begin{declcs}{hologoVariant} \M{name} \M{variant}\\
%   \cs{HologoVariant} \M{name} \M{variant}
% \end{declcs}
% Logo \meta{name} is set using \meta{variant} that specifies
% explicitely which variant of the macro is used. If the argument
% is empty, then the default form of the logo is used
% (configurable by \cs{hologoLogoSetup}).
%
% \cs{HologoVariant} is used if the logo is set in a context
% that needs an uppercase first letter (beginning of a sentence, \dots).
%
% \begin{declcs}{hologoList}\\
%   \cs{hologoEntry} \M{logo} \M{variant} \M{since}
% \end{declcs}
% Macro \cs{hologoList} contains all logos that are provided
% by the package including variants. The list consists of calls
% of \cs{hologoEntry} with three arguments starting with the
% logo name \meta{logo} and its variant \meta{variant}. An empty
% variant means the current default. Argument \meta{since} specifies
% with version of the package \xpackage{hologo} is needed to get
% the logo. If the logo is fixed, then the date gets updated.
% Therefore the date \meta{since} is not exactly the date of
% the first introduction, but rather the date of the latest fix.
%
% Before \cs{hologoList} can be used, macro \cs{hologoEntry} needs
% a definition. The example file in section \ref{sec:example}
% shows applications of \cs{hologoList}.
%
% \subsection{Supported contexts}
%
% Macros \cs{hologo} and friends support special contexts:
% \begin{itemize}
% \item \hologo{LaTeX}'s protection mechanism.
% \item Bookmarks of package \xpackage{hyperref}.
% \item Package \xpackage{tex4ht}.
% \item The macros can be used inside \cs{csname} constructs,
%   if \cs{ifincsname} is available (\hologo{pdfTeX}, \hologo{XeTeX},
%   \hologo{LuaTeX}).
% \end{itemize}
%
% \subsection{Example}
% \label{sec:example}
%
% The following example prints the logos in different fonts.
%    \begin{macrocode}
%<*example>
%<<verbatim
\NeedsTeXFormat{LaTeX2e}
\documentclass[a4paper]{article}
\usepackage[
  hmargin=20mm,
  vmargin=20mm,
]{geometry}
\pagestyle{empty}
\usepackage{hologo}[2016/05/12]
\usepackage{longtable}
\usepackage{array}
\setlength{\extrarowheight}{2pt}
\usepackage[T1]{fontenc}
\usepackage{lmodern}
\usepackage{pdflscape}
\usepackage[
  pdfencoding=auto,
]{hyperref}
\hypersetup{
  pdfauthor={Heiko Oberdiek},
  pdftitle={Example for package `hologo'},
  pdfsubject={Logos with fonts lmr, lmss, qtm, qpl, qhv},
}
\usepackage{bookmark}

% Print the logo list on the console

\begingroup
  \typeout{}%
  \typeout{*** Begin of logo list ***}%
  \newcommand*{\hologoEntry}[3]{%
    \typeout{#1 \ifx\\#2\\\else(#2) \fi[#3]}%
  }%
  \hologoList
  \typeout{*** End of logo list ***}%
  \typeout{}%
\endgroup

\begin{document}
\begin{landscape}

  \section{Example file for package `hologo'}

  % Table for font names

  \begin{longtable}{>{\bfseries}ll}
    \textbf{font} & \textbf{Font name}\\
    \hline
    lmr & Latin Modern Roman\\
    lmss & Latin Modern Sans\\
    qtm & \TeX\ Gyre Termes\\
    qhv & \TeX\ Gyre Heros\\
    qpl & \TeX\ Gyre Pagella\\
  \end{longtable}

  % Logo list with logos in different fonts

  \begingroup
    \newcommand*{\SetVariant}[2]{%
      \ifx\\#2\\%
      \else
        \hologoLogoSetup{#1}{variant=#2}%
      \fi
    }%
    \newcommand*{\hologoEntry}[3]{%
      \SetVariant{#1}{#2}%
      \raisebox{1em}[0pt][0pt]{\hypertarget{#1@#2}{}}%
      \bookmark[%
        dest={#1@#2},%
      ]{%
        #1\ifx\\#2\\\else\space(#2)\fi: \Hologo{#1}, \hologo{#1} %
        [Unicode]%
      }%
      \hypersetup{unicode=false}%
      \bookmark[%
        dest={#1@#2},%
      ]{%
        #1\ifx\\#2\\\else\space(#2)\fi: \Hologo{#1}, \hologo{#1} %
        [PDFDocEncoding]%
      }%
      \texttt{#1}%
      &%
      \texttt{#2}%
      &%
      \Hologo{#1}%
      &%
      \SetVariant{#1}{#2}%
      \hologo{#1}%
      &%
      \SetVariant{#1}{#2}%
      \fontfamily{qtm}\selectfont
      \hologo{#1}%
      &%
      \SetVariant{#1}{#2}%
      \fontfamily{qpl}\selectfont
      \hologo{#1}%
      &%
      \SetVariant{#1}{#2}%
      \textsf{\hologo{#1}}%
      &%
      \SetVariant{#1}{#2}%
      \fontfamily{qhv}\selectfont
      \hologo{#1}%
      \tabularnewline
    }%
    \begin{longtable}{llllllll}%
      \textbf{\textit{logo}} & \textbf{\textit{variant}} &
      \texttt{\string\Hologo} &
      \textbf{lmr} & \textbf{qtm} & \textbf{qpl} &
      \textbf{lmss} & \textbf{qhv}
      \tabularnewline
      \hline
      \endhead
      \hologoList
    \end{longtable}%
  \endgroup

\end{landscape}
\end{document}
%verbatim
%</example>
%    \end{macrocode}
%
% \StopEventually{
% }
%
% \section{Implementation}
%    \begin{macrocode}
%<*package>
%    \end{macrocode}
%    Reload check, especially if the package is not used with \LaTeX.
%    \begin{macrocode}
\begingroup\catcode61\catcode48\catcode32=10\relax%
  \catcode13=5 % ^^M
  \endlinechar=13 %
  \catcode35=6 % #
  \catcode39=12 % '
  \catcode44=12 % ,
  \catcode45=12 % -
  \catcode46=12 % .
  \catcode58=12 % :
  \catcode64=11 % @
  \catcode123=1 % {
  \catcode125=2 % }
  \expandafter\let\expandafter\x\csname ver@hologo.sty\endcsname
  \ifx\x\relax % plain-TeX, first loading
  \else
    \def\empty{}%
    \ifx\x\empty % LaTeX, first loading,
      % variable is initialized, but \ProvidesPackage not yet seen
    \else
      \expandafter\ifx\csname PackageInfo\endcsname\relax
        \def\x#1#2{%
          \immediate\write-1{Package #1 Info: #2.}%
        }%
      \else
        \def\x#1#2{\PackageInfo{#1}{#2, stopped}}%
      \fi
      \x{hologo}{The package is already loaded}%
      \aftergroup\endinput
    \fi
  \fi
\endgroup%
%    \end{macrocode}
%    Package identification:
%    \begin{macrocode}
\begingroup\catcode61\catcode48\catcode32=10\relax%
  \catcode13=5 % ^^M
  \endlinechar=13 %
  \catcode35=6 % #
  \catcode39=12 % '
  \catcode40=12 % (
  \catcode41=12 % )
  \catcode44=12 % ,
  \catcode45=12 % -
  \catcode46=12 % .
  \catcode47=12 % /
  \catcode58=12 % :
  \catcode64=11 % @
  \catcode91=12 % [
  \catcode93=12 % ]
  \catcode123=1 % {
  \catcode125=2 % }
  \expandafter\ifx\csname ProvidesPackage\endcsname\relax
    \def\x#1#2#3[#4]{\endgroup
      \immediate\write-1{Package: #3 #4}%
      \xdef#1{#4}%
    }%
  \else
    \def\x#1#2[#3]{\endgroup
      #2[{#3}]%
      \ifx#1\@undefined
        \xdef#1{#3}%
      \fi
      \ifx#1\relax
        \xdef#1{#3}%
      \fi
    }%
  \fi
\expandafter\x\csname ver@hologo.sty\endcsname
\ProvidesPackage{hologo}%
  [2016/05/12 v1.11 A logo collection with bookmark support (HO)]%
%    \end{macrocode}
%
%    \begin{macrocode}
\begingroup\catcode61\catcode48\catcode32=10\relax%
  \catcode13=5 % ^^M
  \endlinechar=13 %
  \catcode123=1 % {
  \catcode125=2 % }
  \catcode64=11 % @
  \def\x{\endgroup
    \expandafter\edef\csname HOLOGO@AtEnd\endcsname{%
      \endlinechar=\the\endlinechar\relax
      \catcode13=\the\catcode13\relax
      \catcode32=\the\catcode32\relax
      \catcode35=\the\catcode35\relax
      \catcode61=\the\catcode61\relax
      \catcode64=\the\catcode64\relax
      \catcode123=\the\catcode123\relax
      \catcode125=\the\catcode125\relax
    }%
  }%
\x\catcode61\catcode48\catcode32=10\relax%
\catcode13=5 % ^^M
\endlinechar=13 %
\catcode35=6 % #
\catcode64=11 % @
\catcode123=1 % {
\catcode125=2 % }
\def\TMP@EnsureCode#1#2{%
  \edef\HOLOGO@AtEnd{%
    \HOLOGO@AtEnd
    \catcode#1=\the\catcode#1\relax
  }%
  \catcode#1=#2\relax
}
\TMP@EnsureCode{10}{12}% ^^J
\TMP@EnsureCode{33}{12}% !
\TMP@EnsureCode{34}{12}% "
\TMP@EnsureCode{36}{3}% $
\TMP@EnsureCode{38}{4}% &
\TMP@EnsureCode{39}{12}% '
\TMP@EnsureCode{40}{12}% (
\TMP@EnsureCode{41}{12}% )
\TMP@EnsureCode{42}{12}% *
\TMP@EnsureCode{43}{12}% +
\TMP@EnsureCode{44}{12}% ,
\TMP@EnsureCode{45}{12}% -
\TMP@EnsureCode{46}{12}% .
\TMP@EnsureCode{47}{12}% /
\TMP@EnsureCode{58}{12}% :
\TMP@EnsureCode{59}{12}% ;
\TMP@EnsureCode{60}{12}% <
\TMP@EnsureCode{62}{12}% >
\TMP@EnsureCode{63}{12}% ?
\TMP@EnsureCode{91}{12}% [
\TMP@EnsureCode{93}{12}% ]
\TMP@EnsureCode{94}{7}% ^ (superscript)
\TMP@EnsureCode{95}{8}% _ (subscript)
\TMP@EnsureCode{96}{12}% `
\TMP@EnsureCode{124}{12}% |
\edef\HOLOGO@AtEnd{%
  \HOLOGO@AtEnd
  \escapechar\the\escapechar\relax
  \noexpand\endinput
}
\escapechar=92 %
%    \end{macrocode}
%
% \subsection{Logo list}
%
%    \begin{macro}{\hologoList}
%    \begin{macrocode}
\def\hologoList{%
  \hologoEntry{(La)TeX}{}{2011/10/01}%
  \hologoEntry{AmSLaTeX}{}{2010/04/16}%
  \hologoEntry{AmSTeX}{}{2010/04/16}%
  \hologoEntry{biber}{}{2011/10/01}%
  \hologoEntry{BibTeX}{}{2011/10/01}%
  \hologoEntry{BibTeX}{sf}{2011/10/01}%
  \hologoEntry{BibTeX}{sc}{2011/10/01}%
  \hologoEntry{BibTeX8}{}{2011/11/22}%
  \hologoEntry{ConTeXt}{}{2011/03/25}%
  \hologoEntry{ConTeXt}{narrow}{2011/03/25}%
  \hologoEntry{ConTeXt}{simple}{2011/03/25}%
  \hologoEntry{emTeX}{}{2010/04/26}%
  \hologoEntry{eTeX}{}{2010/04/08}%
  \hologoEntry{ExTeX}{}{2011/10/01}%
  \hologoEntry{HanTheThanh}{}{2011/11/29}%
  \hologoEntry{iniTeX}{}{2011/10/01}%
  \hologoEntry{KOMAScript}{}{2011/10/01}%
  \hologoEntry{La}{}{2010/05/08}%
  \hologoEntry{LaTeX}{}{2010/04/08}%
  \hologoEntry{LaTeX2e}{}{2010/04/08}%
  \hologoEntry{LaTeX3}{}{2010/04/24}%
  \hologoEntry{LaTeXe}{}{2010/04/08}%
  \hologoEntry{LaTeXML}{}{2011/11/22}%
  \hologoEntry{LaTeXTeX}{}{2011/10/01}%
  \hologoEntry{LuaLaTeX}{}{2010/04/08}%
  \hologoEntry{LuaTeX}{}{2010/04/08}%
  \hologoEntry{LyX}{}{2011/10/01}%
  \hologoEntry{METAFONT}{}{2011/10/01}%
  \hologoEntry{MetaFun}{}{2011/10/01}%
  \hologoEntry{METAPOST}{}{2011/10/01}%
  \hologoEntry{MetaPost}{}{2011/10/01}%
  \hologoEntry{MiKTeX}{}{2011/10/01}%
  \hologoEntry{NTS}{}{2011/10/01}%
  \hologoEntry{OzMF}{}{2011/10/01}%
  \hologoEntry{OzMP}{}{2011/10/01}%
  \hologoEntry{OzTeX}{}{2011/10/01}%
  \hologoEntry{OzTtH}{}{2011/10/01}%
  \hologoEntry{PCTeX}{}{2011/10/01}%
  \hologoEntry{pdfTeX}{}{2011/10/01}%
  \hologoEntry{pdfLaTeX}{}{2011/10/01}%
  \hologoEntry{PiC}{}{2011/10/01}%
  \hologoEntry{PiCTeX}{}{2011/10/01}%
  \hologoEntry{plainTeX}{}{2010/04/08}%
  \hologoEntry{plainTeX}{space}{2010/04/16}%
  \hologoEntry{plainTeX}{hyphen}{2010/04/16}%
  \hologoEntry{plainTeX}{runtogether}{2010/04/16}%
  \hologoEntry{SageTeX}{}{2011/11/22}%
  \hologoEntry{SLiTeX}{}{2011/10/01}%
  \hologoEntry{SLiTeX}{lift}{2011/10/01}%
  \hologoEntry{SLiTeX}{narrow}{2011/10/01}%
  \hologoEntry{SLiTeX}{simple}{2011/10/01}%
  \hologoEntry{SliTeX}{}{2011/10/01}%
  \hologoEntry{SliTeX}{narrow}{2011/10/01}%
  \hologoEntry{SliTeX}{simple}{2011/10/01}%
  \hologoEntry{SliTeX}{lift}{2011/10/01}%
  \hologoEntry{teTeX}{}{2011/10/01}%
  \hologoEntry{TeX}{}{2010/04/08}%
  \hologoEntry{TeX4ht}{}{2011/11/22}%
  \hologoEntry{TTH}{}{2011/11/22}%
  \hologoEntry{virTeX}{}{2011/10/01}%
  \hologoEntry{VTeX}{}{2010/04/24}%
  \hologoEntry{Xe}{}{2010/04/08}%
  \hologoEntry{XeLaTeX}{}{2010/04/08}%
  \hologoEntry{XeTeX}{}{2010/04/08}%
}
%    \end{macrocode}
%    \end{macro}
%
% \subsection{Load resources}
%
%    \begin{macrocode}
\begingroup\expandafter\expandafter\expandafter\endgroup
\expandafter\ifx\csname RequirePackage\endcsname\relax
  \def\TMP@RequirePackage#1[#2]{%
    \begingroup\expandafter\expandafter\expandafter\endgroup
    \expandafter\ifx\csname ver@#1.sty\endcsname\relax
      \input #1.sty\relax
    \fi
  }%
  \TMP@RequirePackage{ltxcmds}[2011/02/04]%
  \TMP@RequirePackage{infwarerr}[2010/04/08]%
  \TMP@RequirePackage{kvsetkeys}[2010/03/01]%
  \TMP@RequirePackage{kvdefinekeys}[2010/03/01]%
  \TMP@RequirePackage{pdftexcmds}[2010/04/01]%
  \TMP@RequirePackage{ifpdf}[2010/01/28]%
  \TMP@RequirePackage{ifluatex}[2010/03/01]%
  \ltx@IfUndefined{newif}{%
    \expandafter\let\csname newif\endcsname\ltx@newif
  }{}%
  \TMP@RequirePackage{ifxetex}[2009/01/23]%
  \TMP@RequirePackage{ifvtex}[2010/03/01]%
\else
  \RequirePackage{ltxcmds}[2011/02/04]%
  \RequirePackage{infwarerr}[2010/04/08]%
  \RequirePackage{kvsetkeys}[2010/03/01]%
  \RequirePackage{kvdefinekeys}[2010/03/01]%
  \RequirePackage{pdftexcmds}[2010/04/01]%
  \RequirePackage{ifpdf}[2010/01/28]%
  \RequirePackage{ifluatex}[2010/03/01]%
  \RequirePackage{ifxetex}[2009/01/23]%
  \RequirePackage{ifvtex}[2010/03/01]%
\fi
%    \end{macrocode}
%
%    \begin{macro}{\HOLOGO@IfDefined}
%    \begin{macrocode}
\def\HOLOGO@IfExists#1{%
  \ifx\@undefined#1%
    \expandafter\ltx@secondoftwo
  \else
    \ifx\relax#1%
      \expandafter\ltx@secondoftwo
    \else
      \expandafter\expandafter\expandafter\ltx@firstoftwo
    \fi
  \fi
}
%    \end{macrocode}
%    \end{macro}
%
% \subsection{Setup macros}
%
%    \begin{macro}{\hologoSetup}
%    \begin{macrocode}
\def\hologoSetup{%
  \let\HOLOGO@name\relax
  \HOLOGO@Setup
}
%    \end{macrocode}
%    \end{macro}
%
%    \begin{macro}{\hologoLogoSetup}
%    \begin{macrocode}
\def\hologoLogoSetup#1{%
  \edef\HOLOGO@name{#1}%
  \ltx@IfUndefined{HoLogo@\HOLOGO@name}{%
    \@PackageError{hologo}{%
      Unknown logo `\HOLOGO@name'%
    }\@ehc
    \ltx@gobble
  }{%
    \HOLOGO@Setup
  }%
}
%    \end{macrocode}
%    \end{macro}
%
%    \begin{macro}{\HOLOGO@Setup}
%    \begin{macrocode}
\def\HOLOGO@Setup{%
  \kvsetkeys{HoLogo}%
}
%    \end{macrocode}
%    \end{macro}
%
% \subsection{Options}
%
%    \begin{macro}{\HOLOGO@DeclareBoolOption}
%    \begin{macrocode}
\def\HOLOGO@DeclareBoolOption#1{%
  \expandafter\chardef\csname HOLOGOOPT@#1\endcsname\ltx@zero
  \kv@define@key{HoLogo}{#1}[true]{%
    \def\HOLOGO@temp{##1}%
    \ifx\HOLOGO@temp\HOLOGO@true
      \ifx\HOLOGO@name\relax
        \expandafter\chardef\csname HOLOGOOPT@#1\endcsname=\ltx@one
      \else
        \expandafter\chardef\csname
        HoLogoOpt@#1@\HOLOGO@name\endcsname\ltx@one
      \fi
      \HOLOGO@SetBreakAll{#1}%
    \else
      \ifx\HOLOGO@temp\HOLOGO@false
        \ifx\HOLOGO@name\relax
          \expandafter\chardef\csname HOLOGOOPT@#1\endcsname=\ltx@zero
        \else
          \expandafter\chardef\csname
          HoLogoOpt@#1@\HOLOGO@name\endcsname=\ltx@zero
        \fi
        \HOLOGO@SetBreakAll{#1}%
      \else
        \@PackageError{hologo}{%
          Unknown value `##1' for boolean option `#1'.\MessageBreak
          Known values are `true' and `false'%
        }\@ehc
      \fi
    \fi
  }%
}
%    \end{macrocode}
%    \end{macro}
%
%    \begin{macro}{\HOLOGO@SetBreakAll}
%    \begin{macrocode}
\def\HOLOGO@SetBreakAll#1{%
  \def\HOLOGO@temp{#1}%
  \ifx\HOLOGO@temp\HOLOGO@break
    \ifx\HOLOGO@name\relax
      \chardef\HOLOGOOPT@hyphenbreak=\HOLOGOOPT@break
      \chardef\HOLOGOOPT@spacebreak=\HOLOGOOPT@break
      \chardef\HOLOGOOPT@discretionarybreak=\HOLOGOOPT@break
    \else
      \expandafter\chardef
         \csname HoLogoOpt@hyphenbreak@\HOLOGO@name\endcsname=%
         \csname HoLogoOpt@break@\HOLOGO@name\endcsname
      \expandafter\chardef
         \csname HoLogoOpt@spacebreak@\HOLOGO@name\endcsname=%
         \csname HoLogoOpt@break@\HOLOGO@name\endcsname
      \expandafter\chardef
         \csname HoLogoOpt@discretionarybreak@\HOLOGO@name
             \endcsname=%
         \csname HoLogoOpt@break@\HOLOGO@name\endcsname
    \fi
  \fi
}
%    \end{macrocode}
%    \end{macro}
%
%    \begin{macro}{\HOLOGO@true}
%    \begin{macrocode}
\def\HOLOGO@true{true}
%    \end{macrocode}
%    \end{macro}
%    \begin{macro}{\HOLOGO@false}
%    \begin{macrocode}
\def\HOLOGO@false{false}
%    \end{macrocode}
%    \end{macro}
%    \begin{macro}{\HOLOGO@break}
%    \begin{macrocode}
\def\HOLOGO@break{break}
%    \end{macrocode}
%    \end{macro}
%
%    \begin{macrocode}
\HOLOGO@DeclareBoolOption{break}
\HOLOGO@DeclareBoolOption{hyphenbreak}
\HOLOGO@DeclareBoolOption{spacebreak}
\HOLOGO@DeclareBoolOption{discretionarybreak}
%    \end{macrocode}
%
%    \begin{macrocode}
\kv@define@key{HoLogo}{variant}{%
  \ifx\HOLOGO@name\relax
    \@PackageError{hologo}{%
      Option `variant' is not available in \string\hologoSetup,%
      \MessageBreak
      Use \string\hologoLogoSetup\space instead%
    }\@ehc
  \else
    \edef\HOLOGO@temp{#1}%
    \ifx\HOLOGO@temp\ltx@empty
      \expandafter
      \let\csname HoLogoOpt@variant@\HOLOGO@name\endcsname\@undefined
    \else
      \ltx@IfUndefined{HoLogo@\HOLOGO@name @\HOLOGO@temp}{%
        \@PackageError{hologo}{%
          Unknown variant `\HOLOGO@temp' of logo `\HOLOGO@name'%
        }\@ehc
      }{%
        \expandafter
        \let\csname HoLogoOpt@variant@\HOLOGO@name\endcsname
            \HOLOGO@temp
      }%
    \fi
  \fi
}
%    \end{macrocode}
%
%    \begin{macro}{\HOLOGO@Variant}
%    \begin{macrocode}
\def\HOLOGO@Variant#1{%
  #1%
  \ltx@ifundefined{HoLogoOpt@variant@#1}{%
  }{%
    @\csname HoLogoOpt@variant@#1\endcsname
  }%
}
%    \end{macrocode}
%    \end{macro}
%
% \subsection{Break/no-break support}
%
%    \begin{macro}{\HOLOGO@space}
%    \begin{macrocode}
\def\HOLOGO@space{%
  \ltx@ifundefined{HoLogoOpt@spacebreak@\HOLOGO@name}{%
    \ltx@ifundefined{HoLogoOpt@break@\HOLOGO@name}{%
      \chardef\HOLOGO@temp=\HOLOGOOPT@spacebreak
    }{%
      \chardef\HOLOGO@temp=%
        \csname HoLogoOpt@break@\HOLOGO@name\endcsname
    }%
  }{%
    \chardef\HOLOGO@temp=%
      \csname HoLogoOpt@spacebreak@\HOLOGO@name\endcsname
  }%
  \ifcase\HOLOGO@temp
    \penalty10000 %
  \fi
  \ltx@space
}
%    \end{macrocode}
%    \end{macro}
%
%    \begin{macro}{\HOLOGO@hyphen}
%    \begin{macrocode}
\def\HOLOGO@hyphen{%
  \ltx@ifundefined{HoLogoOpt@hyphenbreak@\HOLOGO@name}{%
    \ltx@ifundefined{HoLogoOpt@break@\HOLOGO@name}{%
      \chardef\HOLOGO@temp=\HOLOGOOPT@hyphenbreak
    }{%
      \chardef\HOLOGO@temp=%
        \csname HoLogoOpt@break@\HOLOGO@name\endcsname
    }%
  }{%
    \chardef\HOLOGO@temp=%
      \csname HoLogoOpt@hyphenbreak@\HOLOGO@name\endcsname
  }%
  \ifcase\HOLOGO@temp
    \ltx@mbox{-}%
  \else
    -%
  \fi
}
%    \end{macrocode}
%    \end{macro}
%
%    \begin{macro}{\HOLOGO@discretionary}
%    \begin{macrocode}
\def\HOLOGO@discretionary{%
  \ltx@ifundefined{HoLogoOpt@discretionarybreak@\HOLOGO@name}{%
    \ltx@ifundefined{HoLogoOpt@break@\HOLOGO@name}{%
      \chardef\HOLOGO@temp=\HOLOGOOPT@discretionarybreak
    }{%
      \chardef\HOLOGO@temp=%
        \csname HoLogoOpt@break@\HOLOGO@name\endcsname
    }%
  }{%
    \chardef\HOLOGO@temp=%
      \csname HoLogoOpt@discretionarybreak@\HOLOGO@name\endcsname
  }%
  \ifcase\HOLOGO@temp
  \else
    \-%
  \fi
}
%    \end{macrocode}
%    \end{macro}
%
%    \begin{macro}{\HOLOGO@mbox}
%    \begin{macrocode}
\def\HOLOGO@mbox#1{%
  \ltx@ifundefined{HoLogoOpt@break@\HOLOGO@name}{%
    \chardef\HOLOGO@temp=\HOLOGOOPT@hyphenbreak
  }{%
    \chardef\HOLOGO@temp=%
      \csname HoLogoOpt@break@\HOLOGO@name\endcsname
  }%
  \ifcase\HOLOGO@temp
    \ltx@mbox{#1}%
  \else
    #1%
  \fi
}
%    \end{macrocode}
%    \end{macro}
%
% \subsection{Font support}
%
%    \begin{macro}{\HoLogoFont@font}
%    \begin{tabular}{@{}ll@{}}
%    |#1|:& logo name\\
%    |#2|:& font short name\\
%    |#3|:& text
%    \end{tabular}
%    \begin{macrocode}
\def\HoLogoFont@font#1#2#3{%
  \begingroup
    \ltx@IfUndefined{HoLogoFont@logo@#1.#2}{%
      \ltx@IfUndefined{HoLogoFont@font@#2}{%
        \@PackageWarning{hologo}{%
          Missing font `#2' for logo `#1'%
        }%
        #3%
      }{%
        \csname HoLogoFont@font@#2\endcsname{#3}%
      }%
    }{%
      \csname HoLogoFont@logo@#1.#2\endcsname{#3}%
    }%
  \endgroup
}
%    \end{macrocode}
%    \end{macro}
%
%    \begin{macro}{\HoLogoFont@Def}
%    \begin{macrocode}
\def\HoLogoFont@Def#1{%
  \expandafter\def\csname HoLogoFont@font@#1\endcsname
}
%    \end{macrocode}
%    \end{macro}
%    \begin{macro}{\HoLogoFont@LogoDef}
%    \begin{macrocode}
\def\HoLogoFont@LogoDef#1#2{%
  \expandafter\def\csname HoLogoFont@logo@#1.#2\endcsname
}
%    \end{macrocode}
%    \end{macro}
%
% \subsubsection{Font defaults}
%
%    \begin{macro}{\HoLogoFont@font@general}
%    \begin{macrocode}
\HoLogoFont@Def{general}{}%
%    \end{macrocode}
%    \end{macro}
%
%    \begin{macro}{\HoLogoFont@font@rm}
%    \begin{macrocode}
\ltx@IfUndefined{rmfamily}{%
  \ltx@IfUndefined{rm}{%
  }{%
    \HoLogoFont@Def{rm}{\rm}%
  }%
}{%
  \HoLogoFont@Def{rm}{\rmfamily}%
}
%    \end{macrocode}
%    \end{macro}
%
%    \begin{macro}{\HoLogoFont@font@sf}
%    \begin{macrocode}
\ltx@IfUndefined{sffamily}{%
  \ltx@IfUndefined{sf}{%
  }{%
    \HoLogoFont@Def{sf}{\sf}%
  }%
}{%
  \HoLogoFont@Def{sf}{\sffamily}%
}
%    \end{macrocode}
%    \end{macro}
%
%    \begin{macro}{\HoLogoFont@font@bibsf}
%    In case of \hologo{plainTeX} the original small caps
%    variant is used as default. In \hologo{LaTeX}
%    the definition of package \xpackage{dtklogos} \cite{dtklogos}
%    is used.
%\begin{quote}
%\begin{verbatim}
%\DeclareRobustCommand{\BibTeX}{%
%  B%
%  \kern-.05em%
%  \hbox{%
%    $\m@th$% %% force math size calculations
%    \csname S@\f@size\endcsname
%    \fontsize\sf@size\z@
%    \math@fontsfalse
%    \selectfont
%    I%
%    \kern-.025em%
%    B
%  }%
%  \kern-.08em%
%  \-%
%  \TeX
%}
%\end{verbatim}
%\end{quote}
%    \begin{macrocode}
\ltx@IfUndefined{selectfont}{%
  \ltx@IfUndefined{tensc}{%
    \font\tensc=cmcsc10\relax
  }{}%
  \HoLogoFont@Def{bibsf}{\tensc}%
}{%
  \HoLogoFont@Def{bibsf}{%
    $\mathsurround=0pt$%
    \csname S@\f@size\endcsname
    \fontsize\sf@size{0pt}%
    \math@fontsfalse
    \selectfont
  }%
}
%    \end{macrocode}
%    \end{macro}
%
%    \begin{macro}{\HoLogoFont@font@sc}
%    \begin{macrocode}
\ltx@IfUndefined{scshape}{%
  \ltx@IfUndefined{tensc}{%
    \font\tensc=cmcsc10\relax
  }{}%
  \HoLogoFont@Def{sc}{\tensc}%
}{%
  \HoLogoFont@Def{sc}{\scshape}%
}
%    \end{macrocode}
%    \end{macro}
%
%    \begin{macro}{\HoLogoFont@font@sy}
%    \begin{macrocode}
\ltx@IfUndefined{usefont}{%
  \ltx@IfUndefined{tensy}{%
  }{%
    \HoLogoFont@Def{sy}{\tensy}%
  }%
}{%
  \HoLogoFont@Def{sy}{%
    \usefont{OMS}{cmsy}{m}{n}%
  }%
}
%    \end{macrocode}
%    \end{macro}
%
%    \begin{macro}{\HoLogoFont@font@logo}
%    \begin{macrocode}
\begingroup
  \def\x{LaTeX2e}%
\expandafter\endgroup
\ifx\fmtname\x
  \ltx@IfUndefined{logofamily}{%
    \DeclareRobustCommand\logofamily{%
      \not@math@alphabet\logofamily\relax
      \fontencoding{U}%
      \fontfamily{logo}%
      \selectfont
    }%
  }{}%
  \ltx@IfUndefined{logofamily}{%
  }{%
    \HoLogoFont@Def{logo}{\logofamily}%
  }%
\else
  \ltx@IfUndefined{tenlogo}{%
    \font\tenlogo=logo10\relax
  }{}%
  \HoLogoFont@Def{logo}{\tenlogo}%
\fi
%    \end{macrocode}
%    \end{macro}
%
% \subsubsection{Font setup}
%
%    \begin{macro}{\hologoFontSetup}
%    \begin{macrocode}
\def\hologoFontSetup{%
  \let\HOLOGO@name\relax
  \HOLOGO@FontSetup
}
%    \end{macrocode}
%    \end{macro}
%
%    \begin{macro}{\hologoLogoFontSetup}
%    \begin{macrocode}
\def\hologoLogoFontSetup#1{%
  \edef\HOLOGO@name{#1}%
  \ltx@IfUndefined{HoLogo@\HOLOGO@name}{%
    \@PackageError{hologo}{%
      Unknown logo `\HOLOGO@name'%
    }\@ehc
    \ltx@gobble
  }{%
    \HOLOGO@FontSetup
  }%
}
%    \end{macrocode}
%    \end{macro}
%
%    \begin{macro}{\HOLOGO@FontSetup}
%    \begin{macrocode}
\def\HOLOGO@FontSetup{%
  \kvsetkeys{HoLogoFont}%
}
%    \end{macrocode}
%    \end{macro}
%
%    \begin{macrocode}
\def\HOLOGO@temp#1{%
  \kv@define@key{HoLogoFont}{#1}{%
    \ifx\HOLOGO@name\relax
      \HoLogoFont@Def{#1}{##1}%
    \else
      \HoLogoFont@LogoDef\HOLOGO@name{#1}{##1}%
    \fi
  }%
}
\HOLOGO@temp{general}
\HOLOGO@temp{sf}
%    \end{macrocode}
%
% \subsection{Generic logo commands}
%
%    \begin{macrocode}
\HOLOGO@IfExists\hologo{%
  \@PackageError{hologo}{%
    \string\hologo\ltx@space is already defined.\MessageBreak
    Package loading is aborted%
  }\@ehc
  \HOLOGO@AtEnd
}%
\HOLOGO@IfExists\hologoRobust{%
  \@PackageError{hologo}{%
    \string\hologoRobust\ltx@space is already defined.\MessageBreak
    Package loading is aborted%
  }\@ehc
  \HOLOGO@AtEnd
}%
%    \end{macrocode}
%
% \subsubsection{\cs{hologo} and friends}
%
%    \begin{macrocode}
\ifluatex
  \expandafter\ltx@firstofone
\else
  \expandafter\ltx@gobble
\fi
{%
  \ltx@IfUndefined{ifincsname}{%
    \ifnum\luatexversion<36 %
      \expandafter\ltx@gobble
    \else
      \expandafter\ltx@firstofone
    \fi
    {%
      \begingroup
        \ifcase0%
            \directlua{%
              if tex.enableprimitives then %
                tex.enableprimitives('HOLOGO@', {'ifincsname'})%
              else %
                tex.print('1')%
              end%
            }%
            \ifx\HOLOGO@ifincsname\@undefined 1\fi%
            \relax
          \expandafter\ltx@firstofone
        \else
          \endgroup
          \expandafter\ltx@gobble
        \fi
        {%
          \global\let\ifincsname\HOLOGO@ifincsname
        }%
      \HOLOGO@temp
    }%
  }{}%
}
%    \end{macrocode}
%    \begin{macrocode}
\ltx@IfUndefined{ifincsname}{%
  \catcode`$=14 %
}{%
  \catcode`$=9 %
}
%    \end{macrocode}
%
%    \begin{macro}{\hologo}
%    \begin{macrocode}
\def\hologo#1{%
$ \ifincsname
$   \ltx@ifundefined{HoLogoCs@\HOLOGO@Variant{#1}}{%
$     #1%
$   }{%
$     \csname HoLogoCs@\HOLOGO@Variant{#1}\endcsname\ltx@firstoftwo
$   }%
$ \else
    \HOLOGO@IfExists\texorpdfstring\texorpdfstring\ltx@firstoftwo
    {%
      \hologoRobust{#1}%
    }{%
      \ltx@ifundefined{HoLogoBkm@\HOLOGO@Variant{#1}}{%
        \ltx@ifundefined{HoLogo@#1}{?#1?}{#1}%
      }{%
        \csname HoLogoBkm@\HOLOGO@Variant{#1}\endcsname
        \ltx@firstoftwo
      }%
    }%
$ \fi
}
%    \end{macrocode}
%    \end{macro}
%    \begin{macro}{\Hologo}
%    \begin{macrocode}
\def\Hologo#1{%
$ \ifincsname
$   \ltx@ifundefined{HoLogoCs@\HOLOGO@Variant{#1}}{%
$     #1%
$   }{%
$     \csname HoLogoCs@\HOLOGO@Variant{#1}\endcsname\ltx@secondoftwo
$   }%
$ \else
    \HOLOGO@IfExists\texorpdfstring\texorpdfstring\ltx@firstoftwo
    {%
      \HologoRobust{#1}%
    }{%
      \ltx@ifundefined{HoLogoBkm@\HOLOGO@Variant{#1}}{%
        \ltx@ifundefined{HoLogo@#1}{?#1?}{#1}%
      }{%
        \csname HoLogoBkm@\HOLOGO@Variant{#1}\endcsname
        \ltx@secondoftwo
      }%
    }%
$ \fi
}
%    \end{macrocode}
%    \end{macro}
%
%    \begin{macro}{\hologoVariant}
%    \begin{macrocode}
\def\hologoVariant#1#2{%
  \ifx\relax#2\relax
    \hologo{#1}%
  \else
$   \ifincsname
$     \ltx@ifundefined{HoLogoCs@#1@#2}{%
$       #1%
$     }{%
$       \csname HoLogoCs@#1@#2\endcsname\ltx@firstoftwo
$     }%
$   \else
      \HOLOGO@IfExists\texorpdfstring\texorpdfstring\ltx@firstoftwo
      {%
        \hologoVariantRobust{#1}{#2}%
      }{%
        \ltx@ifundefined{HoLogoBkm@#1@#2}{%
          \ltx@ifundefined{HoLogo@#1}{?#1?}{#1}%
        }{%
          \csname HoLogoBkm@#1@#2\endcsname
          \ltx@firstoftwo
        }%
      }%
$   \fi
  \fi
}
%    \end{macrocode}
%    \end{macro}
%    \begin{macro}{\HologoVariant}
%    \begin{macrocode}
\def\HologoVariant#1#2{%
  \ifx\relax#2\relax
    \Hologo{#1}%
  \else
$   \ifincsname
$     \ltx@ifundefined{HoLogoCs@#1@#2}{%
$       #1%
$     }{%
$       \csname HoLogoCs@#1@#2\endcsname\ltx@secondoftwo
$     }%
$   \else
      \HOLOGO@IfExists\texorpdfstring\texorpdfstring\ltx@firstoftwo
      {%
        \HologoVariantRobust{#1}{#2}%
      }{%
        \ltx@ifundefined{HoLogoBkm@#1@#2}{%
          \ltx@ifundefined{HoLogo@#1}{?#1?}{#1}%
        }{%
          \csname HoLogoBkm@#1@#2\endcsname
          \ltx@secondoftwo
        }%
      }%
$   \fi
  \fi
}
%    \end{macrocode}
%    \end{macro}
%
%    \begin{macrocode}
\catcode`\$=3 %
%    \end{macrocode}
%
% \subsubsection{\cs{hologoRobust} and friends}
%
%    \begin{macro}{\hologoRobust}
%    \begin{macrocode}
\ltx@IfUndefined{protected}{%
  \ltx@IfUndefined{DeclareRobustCommand}{%
    \def\hologoRobust#1%
  }{%
    \DeclareRobustCommand*\hologoRobust[1]%
  }%
}{%
  \protected\def\hologoRobust#1%
}%
{%
  \edef\HOLOGO@name{#1}%
  \ltx@IfUndefined{HoLogo@\HOLOGO@Variant\HOLOGO@name}{%
    \@PackageError{hologo}{%
      Unknown logo `\HOLOGO@name'%
    }\@ehc
    ?\HOLOGO@name?%
  }{%
    \ltx@IfUndefined{ver@tex4ht.sty}{%
      \HoLogoFont@font\HOLOGO@name{general}{%
        \csname HoLogo@\HOLOGO@Variant\HOLOGO@name\endcsname
        \ltx@firstoftwo
      }%
    }{%
      \ltx@IfUndefined{HoLogoHtml@\HOLOGO@Variant\HOLOGO@name}{%
        \HOLOGO@name
      }{%
        \csname HoLogoHtml@\HOLOGO@Variant\HOLOGO@name\endcsname
        \ltx@firstoftwo
      }%
    }%
  }%
}
%    \end{macrocode}
%    \end{macro}
%    \begin{macro}{\HologoRobust}
%    \begin{macrocode}
\ltx@IfUndefined{protected}{%
  \ltx@IfUndefined{DeclareRobustCommand}{%
    \def\HologoRobust#1%
  }{%
    \DeclareRobustCommand*\HologoRobust[1]%
  }%
}{%
  \protected\def\HologoRobust#1%
}%
{%
  \edef\HOLOGO@name{#1}%
  \ltx@IfUndefined{HoLogo@\HOLOGO@Variant\HOLOGO@name}{%
    \@PackageError{hologo}{%
      Unknown logo `\HOLOGO@name'%
    }\@ehc
    ?\HOLOGO@name?%
  }{%
    \ltx@IfUndefined{ver@tex4ht.sty}{%
      \HoLogoFont@font\HOLOGO@name{general}{%
        \csname HoLogo@\HOLOGO@Variant\HOLOGO@name\endcsname
        \ltx@secondoftwo
      }%
    }{%
      \ltx@IfUndefined{HoLogoHtml@\HOLOGO@Variant\HOLOGO@name}{%
        \expandafter\HOLOGO@Uppercase\HOLOGO@name
      }{%
        \csname HoLogoHtml@\HOLOGO@Variant\HOLOGO@name\endcsname
        \ltx@secondoftwo
      }%
    }%
  }%
}
%    \end{macrocode}
%    \end{macro}
%    \begin{macro}{\hologoVariantRobust}
%    \begin{macrocode}
\ltx@IfUndefined{protected}{%
  \ltx@IfUndefined{DeclareRobustCommand}{%
    \def\hologoVariantRobust#1#2%
  }{%
    \DeclareRobustCommand*\hologoVariantRobust[2]%
  }%
}{%
  \protected\def\hologoVariantRobust#1#2%
}%
{%
  \begingroup
    \hologoLogoSetup{#1}{variant={#2}}%
    \hologoRobust{#1}%
  \endgroup
}
%    \end{macrocode}
%    \end{macro}
%    \begin{macro}{\HologoVariantRobust}
%    \begin{macrocode}
\ltx@IfUndefined{protected}{%
  \ltx@IfUndefined{DeclareRobustCommand}{%
    \def\HologoVariantRobust#1#2%
  }{%
    \DeclareRobustCommand*\HologoVariantRobust[2]%
  }%
}{%
  \protected\def\HologoVariantRobust#1#2%
}%
{%
  \begingroup
    \hologoLogoSetup{#1}{variant={#2}}%
    \HologoRobust{#1}%
  \endgroup
}
%    \end{macrocode}
%    \end{macro}
%
%    \begin{macro}{\hologorobust}
%    Macro \cs{hologorobust} is only defined for compatibility.
%    Its use is deprecated.
%    \begin{macrocode}
\def\hologorobust{\hologoRobust}
%    \end{macrocode}
%    \end{macro}
%
% \subsection{Helpers}
%
%    \begin{macro}{\HOLOGO@Uppercase}
%    Macro \cs{HOLOGO@Uppercase} is restricted to \cs{uppercase},
%    because \hologo{plainTeX} or \hologo{iniTeX} do not provide
%    \cs{MakeUppercase}.
%    \begin{macrocode}
\def\HOLOGO@Uppercase#1{\uppercase{#1}}
%    \end{macrocode}
%    \end{macro}
%
%    \begin{macro}{\HOLOGO@PdfdocUnicode}
%    \begin{macrocode}
\def\HOLOGO@PdfdocUnicode{%
  \ifx\ifHy@unicode\iftrue
    \expandafter\ltx@secondoftwo
  \else
    \expandafter\ltx@firstoftwo
  \fi
}
%    \end{macrocode}
%    \end{macro}
%
%    \begin{macro}{\HOLOGO@Math}
%    \begin{macrocode}
\def\HOLOGO@MathSetup{%
  \mathsurround0pt\relax
  \HOLOGO@IfExists\f@series{%
    \if b\expandafter\ltx@car\f@series x\@nil
      \csname boldmath\endcsname
   \fi
  }{}%
}
%    \end{macrocode}
%    \end{macro}
%
%    \begin{macro}{\HOLOGO@TempDimen}
%    \begin{macrocode}
\dimendef\HOLOGO@TempDimen=\ltx@zero
%    \end{macrocode}
%    \end{macro}
%    \begin{macro}{\HOLOGO@NegativeKerning}
%    \begin{macrocode}
\def\HOLOGO@NegativeKerning#1{%
  \begingroup
    \HOLOGO@TempDimen=0pt\relax
    \comma@parse@normalized{#1}{%
      \ifdim\HOLOGO@TempDimen=0pt %
        \expandafter\HOLOGO@@NegativeKerning\comma@entry
      \fi
      \ltx@gobble
    }%
    \ifdim\HOLOGO@TempDimen<0pt %
      \kern\HOLOGO@TempDimen
    \fi
  \endgroup
}
%    \end{macrocode}
%    \end{macro}
%    \begin{macro}{\HOLOGO@@NegativeKerning}
%    \begin{macrocode}
\def\HOLOGO@@NegativeKerning#1#2{%
  \setbox\ltx@zero\hbox{#1#2}%
  \HOLOGO@TempDimen=\wd\ltx@zero
  \setbox\ltx@zero\hbox{#1\kern0pt#2}%
  \advance\HOLOGO@TempDimen by -\wd\ltx@zero
}
%    \end{macrocode}
%    \end{macro}
%
%    \begin{macro}{\HOLOGO@SpaceFactor}
%    \begin{macrocode}
\def\HOLOGO@SpaceFactor{%
  \spacefactor1000 %
}
%    \end{macrocode}
%    \end{macro}
%
%    \begin{macro}{\HOLOGO@Span}
%    \begin{macrocode}
\def\HOLOGO@Span#1#2{%
  \HCode{<span class="HoLogo-#1">}%
  #2%
  \HCode{</span>}%
}
%    \end{macrocode}
%    \end{macro}
%
% \subsubsection{Text subscript}
%
%    \begin{macro}{\HOLOGO@SubScript}%
%    \begin{macrocode}
\def\HOLOGO@SubScript#1{%
  \ltx@IfUndefined{textsubscript}{%
    \ltx@IfUndefined{text}{%
      \ltx@mbox{%
        \mathsurround=0pt\relax
        $%
          _{%
            \ltx@IfUndefined{sf@size}{%
              \mathrm{#1}%
            }{%
              \mbox{%
                \fontsize\sf@size{0pt}\selectfont
                #1%
              }%
            }%
          }%
        $%
      }%
    }{%
      \ltx@mbox{%
        \mathsurround=0pt\relax
        $_{\text{#1}}$%
      }%
    }%
  }{%
    \textsubscript{#1}%
  }%
}
%    \end{macrocode}
%    \end{macro}
%
% \subsection{\hologo{TeX} and friends}
%
% \subsubsection{\hologo{TeX}}
%
%    \begin{macro}{\HoLogo@TeX}
%    Source: \hologo{LaTeX} kernel.
%    \begin{macrocode}
\def\HoLogo@TeX#1{%
  T\kern-.1667em\lower.5ex\hbox{E}\kern-.125emX\HOLOGO@SpaceFactor
}
%    \end{macrocode}
%    \end{macro}
%    \begin{macro}{\HoLogoHtml@TeX}
%    \begin{macrocode}
\def\HoLogoHtml@TeX#1{%
  \HoLogoCss@TeX
  \HOLOGO@Span{TeX}{%
    T%
    \HOLOGO@Span{e}{%
      E%
    }%
    X%
  }%
}
%    \end{macrocode}
%    \end{macro}
%    \begin{macro}{\HoLogoCss@TeX}
%    \begin{macrocode}
\def\HoLogoCss@TeX{%
  \Css{%
    span.HoLogo-TeX span.HoLogo-e{%
      position:relative;%
      top:.5ex;%
      margin-left:-.1667em;%
      margin-right:-.125em;%
    }%
  }%
  \Css{%
    a span.HoLogo-TeX span.HoLogo-e{%
      text-decoration:none;%
    }%
  }%
  \global\let\HoLogoCss@TeX\relax
}
%    \end{macrocode}
%    \end{macro}
%
% \subsubsection{\hologo{plainTeX}}
%
%    \begin{macro}{\HoLogo@plainTeX@space}
%    Source: ``The \hologo{TeX}book''
%    \begin{macrocode}
\def\HoLogo@plainTeX@space#1{%
  \HOLOGO@mbox{#1{p}{P}lain}\HOLOGO@space\hologo{TeX}%
}
%    \end{macrocode}
%    \end{macro}
%    \begin{macro}{\HoLogoCs@plainTeX@space}
%    \begin{macrocode}
\def\HoLogoCs@plainTeX@space#1{#1{p}{P}lain TeX}%
%    \end{macrocode}
%    \end{macro}
%    \begin{macro}{\HoLogoBkm@plainTeX@space}
%    \begin{macrocode}
\def\HoLogoBkm@plainTeX@space#1{%
  #1{p}{P}lain \hologo{TeX}%
}
%    \end{macrocode}
%    \end{macro}
%    \begin{macro}{\HoLogoHtml@plainTeX@space}
%    \begin{macrocode}
\def\HoLogoHtml@plainTeX@space#1{%
  #1{p}{P}lain \hologo{TeX}%
}
%    \end{macrocode}
%    \end{macro}
%
%    \begin{macro}{\HoLogo@plainTeX@hyphen}
%    \begin{macrocode}
\def\HoLogo@plainTeX@hyphen#1{%
  \HOLOGO@mbox{#1{p}{P}lain}\HOLOGO@hyphen\hologo{TeX}%
}
%    \end{macrocode}
%    \end{macro}
%    \begin{macro}{\HoLogoCs@plainTeX@hyphen}
%    \begin{macrocode}
\def\HoLogoCs@plainTeX@hyphen#1{#1{p}{P}lain-TeX}
%    \end{macrocode}
%    \end{macro}
%    \begin{macro}{\HoLogoBkm@plainTeX@hyphen}
%    \begin{macrocode}
\def\HoLogoBkm@plainTeX@hyphen#1{%
  #1{p}{P}lain-\hologo{TeX}%
}
%    \end{macrocode}
%    \end{macro}
%    \begin{macro}{\HoLogoHtml@plainTeX@hyphen}
%    \begin{macrocode}
\def\HoLogoHtml@plainTeX@hyphen#1{%
  #1{p}{P}lain-\hologo{TeX}%
}
%    \end{macrocode}
%    \end{macro}
%
%    \begin{macro}{\HoLogo@plainTeX@runtogether}
%    \begin{macrocode}
\def\HoLogo@plainTeX@runtogether#1{%
  \HOLOGO@mbox{#1{p}{P}lain\hologo{TeX}}%
}
%    \end{macrocode}
%    \end{macro}
%    \begin{macro}{\HoLogoCs@plainTeX@runtogether}
%    \begin{macrocode}
\def\HoLogoCs@plainTeX@runtogether#1{#1{p}{P}lainTeX}
%    \end{macrocode}
%    \end{macro}
%    \begin{macro}{\HoLogoBkm@plainTeX@runtogether}
%    \begin{macrocode}
\def\HoLogoBkm@plainTeX@runtogether#1{%
  #1{p}{P}lain\hologo{TeX}%
}
%    \end{macrocode}
%    \end{macro}
%    \begin{macro}{\HoLogoHtml@plainTeX@runtogether}
%    \begin{macrocode}
\def\HoLogoHtml@plainTeX@runtogether#1{%
  #1{p}{P}lain\hologo{TeX}%
}
%    \end{macrocode}
%    \end{macro}
%
%    \begin{macro}{\HoLogo@plainTeX}
%    \begin{macrocode}
\def\HoLogo@plainTeX{\HoLogo@plainTeX@space}
%    \end{macrocode}
%    \end{macro}
%    \begin{macro}{\HoLogoCs@plainTeX}
%    \begin{macrocode}
\def\HoLogoCs@plainTeX{\HoLogoCs@plainTeX@space}
%    \end{macrocode}
%    \end{macro}
%    \begin{macro}{\HoLogoBkm@plainTeX}
%    \begin{macrocode}
\def\HoLogoBkm@plainTeX{\HoLogoBkm@plainTeX@space}
%    \end{macrocode}
%    \end{macro}
%    \begin{macro}{\HoLogoHtml@plainTeX}
%    \begin{macrocode}
\def\HoLogoHtml@plainTeX{\HoLogoHtml@plainTeX@space}
%    \end{macrocode}
%    \end{macro}
%
% \subsubsection{\hologo{LaTeX}}
%
%    Source: \hologo{LaTeX} kernel.
%\begin{quote}
%\begin{verbatim}
%\DeclareRobustCommand{\LaTeX}{%
%  L%
%  \kern-.36em%
%  {%
%    \sbox\z@ T%
%    \vbox to\ht\z@{%
%      \hbox{%
%        \check@mathfonts
%        \fontsize\sf@size\z@
%        \math@fontsfalse
%        \selectfont
%        A%
%      }%
%      \vss
%    }%
%  }%
%  \kern-.15em%
%  \TeX
%}
%\end{verbatim}
%\end{quote}
%
%    \begin{macro}{\HoLogo@La}
%    \begin{macrocode}
\def\HoLogo@La#1{%
  L%
  \kern-.36em%
  \begingroup
    \setbox\ltx@zero\hbox{T}%
    \vbox to\ht\ltx@zero{%
      \hbox{%
        \ltx@ifundefined{check@mathfonts}{%
          \csname sevenrm\endcsname
        }{%
          \check@mathfonts
          \fontsize\sf@size{0pt}%
          \math@fontsfalse\selectfont
        }%
        A%
      }%
      \vss
    }%
  \endgroup
}
%    \end{macrocode}
%    \end{macro}
%
%    \begin{macro}{\HoLogo@LaTeX}
%    Source: \hologo{LaTeX} kernel.
%    \begin{macrocode}
\def\HoLogo@LaTeX#1{%
  \hologo{La}%
  \kern-.15em%
  \hologo{TeX}%
}
%    \end{macrocode}
%    \end{macro}
%    \begin{macro}{\HoLogoHtml@LaTeX}
%    \begin{macrocode}
\def\HoLogoHtml@LaTeX#1{%
  \HoLogoCss@LaTeX
  \HOLOGO@Span{LaTeX}{%
    L%
    \HOLOGO@Span{a}{%
      A%
    }%
    \hologo{TeX}%
  }%
}
%    \end{macrocode}
%    \end{macro}
%    \begin{macro}{\HoLogoCss@LaTeX}
%    \begin{macrocode}
\def\HoLogoCss@LaTeX{%
  \Css{%
    span.HoLogo-LaTeX span.HoLogo-a{%
      position:relative;%
      top:-.5ex;%
      margin-left:-.36em;%
      margin-right:-.15em;%
      font-size:85\%;%
    }%
  }%
  \global\let\HoLogoCss@LaTeX\relax
}
%    \end{macrocode}
%    \end{macro}
%
% \subsubsection{\hologo{(La)TeX}}
%
%    \begin{macro}{\HoLogo@LaTeXTeX}
%    The kerning around the parentheses is taken
%    from package \xpackage{dtklogos} \cite{dtklogos}.
%\begin{quote}
%\begin{verbatim}
%\DeclareRobustCommand{\LaTeXTeX}{%
%  (%
%  \kern-.15em%
%  L%
%  \kern-.36em%
%  {%
%    \sbox\z@ T%
%    \vbox to\ht0{%
%      \hbox{%
%        $\m@th$%
%        \csname S@\f@size\endcsname
%        \fontsize\sf@size\z@
%        \math@fontsfalse
%        \selectfont
%        A%
%      }%
%      \vss
%    }%
%  }%
%  \kern-.2em%
%  )%
%  \kern-.15em%
%  \TeX
%}
%\end{verbatim}
%\end{quote}
%    \begin{macrocode}
\def\HoLogo@LaTeXTeX#1{%
  (%
  \kern-.15em%
  \hologo{La}%
  \kern-.2em%
  )%
  \kern-.15em%
  \hologo{TeX}%
}
%    \end{macrocode}
%    \end{macro}
%    \begin{macro}{\HoLogoBkm@LaTeXTeX}
%    \begin{macrocode}
\def\HoLogoBkm@LaTeXTeX#1{(La)TeX}
%    \end{macrocode}
%    \end{macro}
%
%    \begin{macro}{\HoLogo@(La)TeX}
%    \begin{macrocode}
\expandafter
\let\csname HoLogo@(La)TeX\endcsname\HoLogo@LaTeXTeX
%    \end{macrocode}
%    \end{macro}
%    \begin{macro}{\HoLogoBkm@(La)TeX}
%    \begin{macrocode}
\expandafter
\let\csname HoLogoBkm@(La)TeX\endcsname\HoLogoBkm@LaTeXTeX
%    \end{macrocode}
%    \end{macro}
%    \begin{macro}{\HoLogoHtml@LaTeXTeX}
%    \begin{macrocode}
\def\HoLogoHtml@LaTeXTeX#1{%
  \HoLogoCss@LaTeXTeX
  \HOLOGO@Span{LaTeXTeX}{%
    (%
    \HOLOGO@Span{L}{L}%
    \HOLOGO@Span{a}{A}%
    \HOLOGO@Span{ParenRight}{)}%
    \hologo{TeX}%
  }%
}
%    \end{macrocode}
%    \end{macro}
%    \begin{macro}{\HoLogoHtml@(La)TeX}
%    Kerning after opening parentheses and before closing parentheses
%    is $-0.1$\,em. The original values $-0.15$\,em
%    looked too ugly for a serif font.
%    \begin{macrocode}
\expandafter
\let\csname HoLogoHtml@(La)TeX\endcsname\HoLogoHtml@LaTeXTeX
%    \end{macrocode}
%    \end{macro}
%    \begin{macro}{\HoLogoCss@LaTeXTeX}
%    \begin{macrocode}
\def\HoLogoCss@LaTeXTeX{%
  \Css{%
    span.HoLogo-LaTeXTeX span.HoLogo-L{%
      margin-left:-.1em;%
    }%
  }%
  \Css{%
    span.HoLogo-LaTeXTeX span.HoLogo-a{%
      position:relative;%
      top:-.5ex;%
      margin-left:-.36em;%
      margin-right:-.1em;%
      font-size:85\%;%
    }%
  }%
  \Css{%
    span.HoLogo-LaTeXTeX span.HoLogo-ParenRight{%
      margin-right:-.15em;%
    }%
  }%
  \global\let\HoLogoCss@LaTeXTeX\relax
}
%    \end{macrocode}
%    \end{macro}
%
% \subsubsection{\hologo{LaTeXe}}
%
%    \begin{macro}{\HoLogo@LaTeXe}
%    Source: \hologo{LaTeX} kernel
%    \begin{macrocode}
\def\HoLogo@LaTeXe#1{%
  \hologo{LaTeX}%
  \kern.15em%
  \hbox{%
    \HOLOGO@MathSetup
    2%
    $_{\textstyle\varepsilon}$%
  }%
}
%    \end{macrocode}
%    \end{macro}
%
%    \begin{macro}{\HoLogoCs@LaTeXe}
%    \begin{macrocode}
\ifnum64=`\^^^^0040\relax % test for big chars of LuaTeX/XeTeX
  \catcode`\$=9 %
  \catcode`\&=14 %
\else
  \catcode`\$=14 %
  \catcode`\&=9 %
\fi
\def\HoLogoCs@LaTeXe#1{%
  LaTeX2%
$ \string ^^^^0395%
& e%
}%
\catcode`\$=3 %
\catcode`\&=4 %
%    \end{macrocode}
%    \end{macro}
%
%    \begin{macro}{\HoLogoBkm@LaTeXe}
%    \begin{macrocode}
\def\HoLogoBkm@LaTeXe#1{%
  \hologo{LaTeX}%
  2%
  \HOLOGO@PdfdocUnicode{e}{\textepsilon}%
}
%    \end{macrocode}
%    \end{macro}
%
%    \begin{macro}{\HoLogoHtml@LaTeXe}
%    \begin{macrocode}
\def\HoLogoHtml@LaTeXe#1{%
  \HoLogoCss@LaTeXe
  \HOLOGO@Span{LaTeX2e}{%
    \hologo{LaTeX}%
    \HOLOGO@Span{2}{2}%
    \HOLOGO@Span{e}{%
      \HOLOGO@MathSetup
      \ensuremath{\textstyle\varepsilon}%
    }%
  }%
}
%    \end{macrocode}
%    \end{macro}
%    \begin{macro}{\HoLogoCss@LaTeXe}
%    \begin{macrocode}
\def\HoLogoCss@LaTeXe{%
  \Css{%
    span.HoLogo-LaTeX2e span.HoLogo-2{%
      padding-left:.15em;%
    }%
  }%
  \Css{%
    span.HoLogo-LaTeX2e span.HoLogo-e{%
      position:relative;%
      top:.35ex;%
      text-decoration:none;%
    }%
  }%
  \global\let\HoLogoCss@LaTeXe\relax
}
%    \end{macrocode}
%    \end{macro}
%
%    \begin{macro}{\HoLogo@LaTeX2e}
%    \begin{macrocode}
\expandafter
\let\csname HoLogo@LaTeX2e\endcsname\HoLogo@LaTeXe
%    \end{macrocode}
%    \end{macro}
%    \begin{macro}{\HoLogoCs@LaTeX2e}
%    \begin{macrocode}
\expandafter
\let\csname HoLogoCs@LaTeX2e\endcsname\HoLogoCs@LaTeXe
%    \end{macrocode}
%    \end{macro}
%    \begin{macro}{\HoLogoBkm@LaTeX2e}
%    \begin{macrocode}
\expandafter
\let\csname HoLogoBkm@LaTeX2e\endcsname\HoLogoBkm@LaTeXe
%    \end{macrocode}
%    \end{macro}
%    \begin{macro}{\HoLogoHtml@LaTeX2e}
%    \begin{macrocode}
\expandafter
\let\csname HoLogoHtml@LaTeX2e\endcsname\HoLogoHtml@LaTeXe
%    \end{macrocode}
%    \end{macro}
%
% \subsubsection{\hologo{LaTeX3}}
%
%    \begin{macro}{\HoLogo@LaTeX3}
%    Source: \hologo{LaTeX} kernel
%    \begin{macrocode}
\expandafter\def\csname HoLogo@LaTeX3\endcsname#1{%
  \hologo{LaTeX}%
  3%
}
%    \end{macrocode}
%    \end{macro}
%
%    \begin{macro}{\HoLogoBkm@LaTeX3}
%    \begin{macrocode}
\expandafter\def\csname HoLogoBkm@LaTeX3\endcsname#1{%
  \hologo{LaTeX}%
  3%
}
%    \end{macrocode}
%    \end{macro}
%    \begin{macro}{\HoLogoHtml@LaTeX3}
%    \begin{macrocode}
\expandafter
\let\csname HoLogoHtml@LaTeX3\expandafter\endcsname
\csname HoLogo@LaTeX3\endcsname
%    \end{macrocode}
%    \end{macro}
%
% \subsubsection{\hologo{LaTeXML}}
%
%    \begin{macro}{\HoLogo@LaTeXML}
%    \begin{macrocode}
\def\HoLogo@LaTeXML#1{%
  \HOLOGO@mbox{%
    \hologo{La}%
    \kern-.15em%
    T%
    \kern-.1667em%
    \lower.5ex\hbox{E}%
    \kern-.125em%
    \HoLogoFont@font{LaTeXML}{sc}{xml}%
  }%
}
%    \end{macrocode}
%    \end{macro}
%    \begin{macro}{\HoLogoHtml@pdfLaTeX}
%    \begin{macrocode}
\def\HoLogoHtml@LaTeXML#1{%
  \HOLOGO@Span{LaTeXML}{%
    \HoLogoCss@LaTeX
    \HoLogoCss@TeX
    \HOLOGO@Span{LaTeX}{%
      L%
      \HOLOGO@Span{a}{%
        A%
      }%
    }%
    \HOLOGO@Span{TeX}{%
      T%
      \HOLOGO@Span{e}{%
        E%
      }%
    }%
    \HCode{<span style="font-variant: small-caps;">}%
    xml%
    \HCode{</span>}%
  }%
}
%    \end{macrocode}
%    \end{macro}
%
% \subsubsection{\hologo{eTeX}}
%
%    \begin{macro}{\HoLogo@eTeX}
%    Source: package \xpackage{etex}
%    \begin{macrocode}
\def\HoLogo@eTeX#1{%
  \ltx@mbox{%
    \HOLOGO@MathSetup
    $\varepsilon$%
    -%
    \HOLOGO@NegativeKerning{-T,T-,To}%
    \hologo{TeX}%
  }%
}
%    \end{macrocode}
%    \end{macro}
%    \begin{macro}{\HoLogoCs@eTeX}
%    \begin{macrocode}
\ifnum64=`\^^^^0040\relax % test for big chars of LuaTeX/XeTeX
  \catcode`\$=9 %
  \catcode`\&=14 %
\else
  \catcode`\$=14 %
  \catcode`\&=9 %
\fi
\def\HoLogoCs@eTeX#1{%
$ #1{\string ^^^^0395}{\string ^^^^03b5}%
& #1{e}{E}%
  TeX%
}%
\catcode`\$=3 %
\catcode`\&=4 %
%    \end{macrocode}
%    \end{macro}
%    \begin{macro}{\HoLogoBkm@eTeX}
%    \begin{macrocode}
\def\HoLogoBkm@eTeX#1{%
  \HOLOGO@PdfdocUnicode{#1{e}{E}}{\textepsilon}%
  -%
  \hologo{TeX}%
}
%    \end{macrocode}
%    \end{macro}
%    \begin{macro}{\HoLogoHtml@eTeX}
%    \begin{macrocode}
\def\HoLogoHtml@eTeX#1{%
  \ltx@mbox{%
    \HOLOGO@MathSetup
    $\varepsilon$%
    -%
    \hologo{TeX}%
  }%
}
%    \end{macrocode}
%    \end{macro}
%
% \subsubsection{\hologo{iniTeX}}
%
%    \begin{macro}{\HoLogo@iniTeX}
%    \begin{macrocode}
\def\HoLogo@iniTeX#1{%
  \HOLOGO@mbox{%
    #1{i}{I}ni\hologo{TeX}%
  }%
}
%    \end{macrocode}
%    \end{macro}
%    \begin{macro}{\HoLogoCs@iniTeX}
%    \begin{macrocode}
\def\HoLogoCs@iniTeX#1{#1{i}{I}niTeX}
%    \end{macrocode}
%    \end{macro}
%    \begin{macro}{\HoLogoBkm@iniTeX}
%    \begin{macrocode}
\def\HoLogoBkm@iniTeX#1{%
  #1{i}{I}ni\hologo{TeX}%
}
%    \end{macrocode}
%    \end{macro}
%    \begin{macro}{\HoLogoHtml@iniTeX}
%    \begin{macrocode}
\let\HoLogoHtml@iniTeX\HoLogo@iniTeX
%    \end{macrocode}
%    \end{macro}
%
% \subsubsection{\hologo{virTeX}}
%
%    \begin{macro}{\HoLogo@virTeX}
%    \begin{macrocode}
\def\HoLogo@virTeX#1{%
  \HOLOGO@mbox{%
    #1{v}{V}ir\hologo{TeX}%
  }%
}
%    \end{macrocode}
%    \end{macro}
%    \begin{macro}{\HoLogoCs@virTeX}
%    \begin{macrocode}
\def\HoLogoCs@virTeX#1{#1{v}{V}irTeX}
%    \end{macrocode}
%    \end{macro}
%    \begin{macro}{\HoLogoBkm@virTeX}
%    \begin{macrocode}
\def\HoLogoBkm@virTeX#1{%
  #1{v}{V}ir\hologo{TeX}%
}
%    \end{macrocode}
%    \end{macro}
%    \begin{macro}{\HoLogoHtml@virTeX}
%    \begin{macrocode}
\let\HoLogoHtml@virTeX\HoLogo@virTeX
%    \end{macrocode}
%    \end{macro}
%
% \subsubsection{\hologo{SliTeX}}
%
% \paragraph{Definitions of the three variants.}
%
%    \begin{macro}{\HoLogo@SLiTeX@lift}
%    \begin{macrocode}
\def\HoLogo@SLiTeX@lift#1{%
  \HoLogoFont@font{SliTeX}{rm}{%
    S%
    \kern-.06em%
    L%
    \kern-.18em%
    \raise.32ex\hbox{\HoLogoFont@font{SliTeX}{sc}{i}}%
    \HOLOGO@discretionary
    \kern-.06em%
    \hologo{TeX}%
  }%
}
%    \end{macrocode}
%    \end{macro}
%    \begin{macro}{\HoLogoBkm@SLiTeX@lift}
%    \begin{macrocode}
\def\HoLogoBkm@SLiTeX@lift#1{SLiTeX}
%    \end{macrocode}
%    \end{macro}
%    \begin{macro}{\HoLogoHtml@SLiTeX@lift}
%    \begin{macrocode}
\def\HoLogoHtml@SLiTeX@lift#1{%
  \HoLogoCss@SLiTeX@lift
  \HOLOGO@Span{SLiTeX-lift}{%
    \HoLogoFont@font{SliTeX}{rm}{%
      S%
      \HOLOGO@Span{L}{L}%
      \HOLOGO@Span{i}{i}%
      \hologo{TeX}%
    }%
  }%
}
%    \end{macrocode}
%    \end{macro}
%    \begin{macro}{\HoLogoCss@SLiTeX@lift}
%    \begin{macrocode}
\def\HoLogoCss@SLiTeX@lift{%
  \Css{%
    span.HoLogo-SLiTeX-lift span.HoLogo-L{%
      margin-left:-.06em;%
      margin-right:-.18em;%
    }%
  }%
  \Css{%
    span.HoLogo-SLiTeX-lift span.HoLogo-i{%
      position:relative;%
      top:-.32ex;%
      margin-right:-.06em;%
      font-variant:small-caps;%
    }%
  }%
  \global\let\HoLogoCss@SLiTeX@lift\relax
}
%    \end{macrocode}
%    \end{macro}
%
%    \begin{macro}{\HoLogo@SliTeX@simple}
%    \begin{macrocode}
\def\HoLogo@SliTeX@simple#1{%
  \HoLogoFont@font{SliTeX}{rm}{%
    \ltx@mbox{%
      \HoLogoFont@font{SliTeX}{sc}{Sli}%
    }%
    \HOLOGO@discretionary
    \hologo{TeX}%
  }%
}
%    \end{macrocode}
%    \end{macro}
%    \begin{macro}{\HoLogoBkm@SliTeX@simple}
%    \begin{macrocode}
\def\HoLogoBkm@SliTeX@simple#1{SliTeX}
%    \end{macrocode}
%    \end{macro}
%    \begin{macro}{\HoLogoHtml@SliTeX@simple}
%    \begin{macrocode}
\let\HoLogoHtml@SliTeX@simple\HoLogo@SliTeX@simple
%    \end{macrocode}
%    \end{macro}
%
%    \begin{macro}{\HoLogo@SliTeX@narrow}
%    \begin{macrocode}
\def\HoLogo@SliTeX@narrow#1{%
  \HoLogoFont@font{SliTeX}{rm}{%
    \ltx@mbox{%
      S%
      \kern-.06em%
      \HoLogoFont@font{SliTeX}{sc}{%
        l%
        \kern-.035em%
        i%
      }%
    }%
    \HOLOGO@discretionary
    \kern-.06em%
    \hologo{TeX}%
  }%
}
%    \end{macrocode}
%    \end{macro}
%    \begin{macro}{\HoLogoBkm@SliTeX@narrow}
%    \begin{macrocode}
\def\HoLogoBkm@SliTeX@narrow#1{SliTeX}
%    \end{macrocode}
%    \end{macro}
%    \begin{macro}{\HoLogoHtml@SliTeX@narrow}
%    \begin{macrocode}
\def\HoLogoHtml@SliTeX@narrow#1{%
  \HoLogoCss@SliTeX@narrow
  \HOLOGO@Span{SliTeX-narrow}{%
    \HoLogoFont@font{SliTeX}{rm}{%
      S%
        \HOLOGO@Span{l}{l}%
        \HOLOGO@Span{i}{i}%
      \hologo{TeX}%
    }%
  }%
}
%    \end{macrocode}
%    \end{macro}
%    \begin{macro}{\HoLogoCss@SliTeX@narrow}
%    \begin{macrocode}
\def\HoLogoCss@SliTeX@narrow{%
  \Css{%
    span.HoLogo-SliTeX-narrow span.HoLogo-l{%
      margin-left:-.06em;%
      margin-right:-.035em;%
      font-variant:small-caps;%
    }%
  }%
  \Css{%
    span.HoLogo-SliTeX-narrow span.HoLogo-i{%
      margin-right:-.06em;%
      font-variant:small-caps;%
    }%
  }%
  \global\let\HoLogoCss@SliTeX@narrow\relax
}
%    \end{macrocode}
%    \end{macro}
%
% \paragraph{Macro set completion.}
%
%    \begin{macro}{\HoLogo@SLiTeX@simple}
%    \begin{macrocode}
\def\HoLogo@SLiTeX@simple{\HoLogo@SliTeX@simple}
%    \end{macrocode}
%    \end{macro}
%    \begin{macro}{\HoLogoBkm@SLiTeX@simple}
%    \begin{macrocode}
\def\HoLogoBkm@SLiTeX@simple{\HoLogoBkm@SliTeX@simple}
%    \end{macrocode}
%    \end{macro}
%    \begin{macro}{\HoLogoHtml@SLiTeX@simple}
%    \begin{macrocode}
\def\HoLogoHtml@SLiTeX@simple{\HoLogoHtml@SliTeX@simple}
%    \end{macrocode}
%    \end{macro}
%
%    \begin{macro}{\HoLogo@SLiTeX@narrow}
%    \begin{macrocode}
\def\HoLogo@SLiTeX@narrow{\HoLogo@SliTeX@narrow}
%    \end{macrocode}
%    \end{macro}
%    \begin{macro}{\HoLogoBkm@SLiTeX@narrow}
%    \begin{macrocode}
\def\HoLogoBkm@SLiTeX@narrow{\HoLogoBkm@SliTeX@narrow}
%    \end{macrocode}
%    \end{macro}
%    \begin{macro}{\HoLogoHtml@SLiTeX@narrow}
%    \begin{macrocode}
\def\HoLogoHtml@SLiTeX@narrow{\HoLogoHtml@SliTeX@narrow}
%    \end{macrocode}
%    \end{macro}
%
%    \begin{macro}{\HoLogo@SliTeX@lift}
%    \begin{macrocode}
\def\HoLogo@SliTeX@lift{\HoLogo@SLiTeX@lift}
%    \end{macrocode}
%    \end{macro}
%    \begin{macro}{\HoLogoBkm@SliTeX@lift}
%    \begin{macrocode}
\def\HoLogoBkm@SliTeX@lift{\HoLogoBkm@SLiTeX@lift}
%    \end{macrocode}
%    \end{macro}
%    \begin{macro}{\HoLogoHtml@SliTeX@lift}
%    \begin{macrocode}
\def\HoLogoHtml@SliTeX@lift{\HoLogoHtml@SLiTeX@lift}
%    \end{macrocode}
%    \end{macro}
%
% \paragraph{Defaults.}
%
%    \begin{macro}{\HoLogo@SLiTeX}
%    \begin{macrocode}
\def\HoLogo@SLiTeX{\HoLogo@SLiTeX@lift}
%    \end{macrocode}
%    \end{macro}
%    \begin{macro}{\HoLogoBkm@SLiTeX}
%    \begin{macrocode}
\def\HoLogoBkm@SLiTeX{\HoLogoBkm@SLiTeX@lift}
%    \end{macrocode}
%    \end{macro}
%    \begin{macro}{\HoLogoHtml@SLiTeX}
%    \begin{macrocode}
\def\HoLogoHtml@SLiTeX{\HoLogoHtml@SLiTeX@lift}
%    \end{macrocode}
%    \end{macro}
%
%    \begin{macro}{\HoLogo@SliTeX}
%    \begin{macrocode}
\def\HoLogo@SliTeX{\HoLogo@SliTeX@narrow}
%    \end{macrocode}
%    \end{macro}
%    \begin{macro}{\HoLogoBkm@SliTeX}
%    \begin{macrocode}
\def\HoLogoBkm@SliTeX{\HoLogoBkm@SliTeX@narrow}
%    \end{macrocode}
%    \end{macro}
%    \begin{macro}{\HoLogoHtml@SliTeX}
%    \begin{macrocode}
\def\HoLogoHtml@SliTeX{\HoLogoHtml@SliTeX@narrow}
%    \end{macrocode}
%    \end{macro}
%
% \subsubsection{\hologo{LuaTeX}}
%
%    \begin{macro}{\HoLogo@LuaTeX}
%    The kerning is an idea of Hans Hagen, see mailing list
%    `luatex at tug dot org' in March 2010.
%    \begin{macrocode}
\def\HoLogo@LuaTeX#1{%
  \HOLOGO@mbox{%
    Lua%
    \HOLOGO@NegativeKerning{aT,oT,To}%
    \hologo{TeX}%
  }%
}
%    \end{macrocode}
%    \end{macro}
%    \begin{macro}{\HoLogoHtml@LuaTeX}
%    \begin{macrocode}
\let\HoLogoHtml@LuaTeX\HoLogo@LuaTeX
%    \end{macrocode}
%    \end{macro}
%
% \subsubsection{\hologo{LuaLaTeX}}
%
%    \begin{macro}{\HoLogo@LuaLaTeX}
%    \begin{macrocode}
\def\HoLogo@LuaLaTeX#1{%
  \HOLOGO@mbox{%
    Lua%
    \hologo{LaTeX}%
  }%
}
%    \end{macrocode}
%    \end{macro}
%    \begin{macro}{\HoLogoHtml@LuaLaTeX}
%    \begin{macrocode}
\let\HoLogoHtml@LuaLaTeX\HoLogo@LuaLaTeX
%    \end{macrocode}
%    \end{macro}
%
% \subsubsection{\hologo{XeTeX}, \hologo{XeLaTeX}}
%
%    \begin{macro}{\HOLOGO@IfCharExists}
%    \begin{macrocode}
\ifluatex
  \ifnum\luatexversion<36 %
  \else
    \def\HOLOGO@IfCharExists#1{%
      \ifnum
        \directlua{%
           if luaotfload and luaotfload.aux then
             if luaotfload.aux.font_has_glyph(%
                    font.current(), \number#1) then % 	 
	       tex.print("1") % 	 
	     end % 	 
	   elseif font and font.fonts and font.current then %
            local f = font.fonts[font.current()]%
            if f.characters and f.characters[\number#1] then %
              tex.print("1")%
            end %
          end%
        }0=\ltx@zero
        \expandafter\ltx@secondoftwo
      \else
        \expandafter\ltx@firstoftwo
      \fi
    }%
  \fi
\fi
\ltx@IfUndefined{HOLOGO@IfCharExists}{%
  \def\HOLOGO@@IfCharExists#1{%
    \begingroup
      \tracinglostchars=\ltx@zero
      \setbox\ltx@zero=\hbox{%
        \kern7sp\char#1\relax
        \ifnum\lastkern>\ltx@zero
          \expandafter\aftergroup\csname iffalse\endcsname
        \else
          \expandafter\aftergroup\csname iftrue\endcsname
        \fi
      }%
      % \if{true|false} from \aftergroup
      \endgroup
      \expandafter\ltx@firstoftwo
    \else
      \endgroup
      \expandafter\ltx@secondoftwo
    \fi
  }%
  \ifxetex
    \ltx@IfUndefined{XeTeXfonttype}{}{%
      \ltx@IfUndefined{XeTeXcharglyph}{}{%
        \def\HOLOGO@IfCharExists#1{%
          \ifnum\XeTeXfonttype\font>\ltx@zero
            \expandafter\ltx@firstofthree
          \else
            \expandafter\ltx@gobble
          \fi
          {%
            \ifnum\XeTeXcharglyph#1>\ltx@zero
              \expandafter\ltx@firstoftwo
            \else
              \expandafter\ltx@secondoftwo
            \fi
          }%
          \HOLOGO@@IfCharExists{#1}%
        }%
      }%
    }%
  \fi
}{}
\ltx@ifundefined{HOLOGO@IfCharExists}{%
  \ifnum64=`\^^^^0040\relax % test for big chars of LuaTeX/XeTeX
    \let\HOLOGO@IfCharExists\HOLOGO@@IfCharExists
  \else
    \def\HOLOGO@IfCharExists#1{%
      \ifnum#1>255 %
        \expandafter\ltx@fourthoffour
      \fi
      \HOLOGO@@IfCharExists{#1}%
    }%
  \fi
}{}
%    \end{macrocode}
%    \end{macro}
%
%    \begin{macro}{\HoLogo@Xe}
%    Source: package \xpackage{dtklogos}
%    \begin{macrocode}
\def\HoLogo@Xe#1{%
  X%
  \kern-.1em\relax
  \HOLOGO@IfCharExists{"018E}{%
    \lower.5ex\hbox{\char"018E}%
  }{%
    \chardef\HOLOGO@choice=\ltx@zero
    \ifdim\fontdimen\ltx@one\font>0pt %
      \ltx@IfUndefined{rotatebox}{%
        \ltx@IfUndefined{pgftext}{%
          \ltx@IfUndefined{psscalebox}{%
            \ltx@IfUndefined{HOLOGO@ScaleBox@\hologoDriver}{%
            }{%
              \chardef\HOLOGO@choice=4 %
            }%
          }{%
            \chardef\HOLOGO@choice=3 %
          }%
        }{%
          \chardef\HOLOGO@choice=2 %
        }%
      }{%
        \chardef\HOLOGO@choice=1 %
      }%
      \ifcase\HOLOGO@choice
        \HOLOGO@WarningUnsupportedDriver{Xe}%
        e%
      \or % 1: \rotatebox
        \begingroup
          \setbox\ltx@zero\hbox{\rotatebox{180}{E}}%
          \ltx@LocDimenA=\dp\ltx@zero
          \advance\ltx@LocDimenA by -.5ex\relax
          \raise\ltx@LocDimenA\box\ltx@zero
        \endgroup
      \or % 2: \pgftext
        \lower.5ex\hbox{%
          \pgfpicture
            \pgftext[rotate=180]{E}%
          \endpgfpicture
        }%
      \or % 3: \psscalebox
        \begingroup
          \setbox\ltx@zero\hbox{\psscalebox{-1 -1}{E}}%
          \ltx@LocDimenA=\dp\ltx@zero
          \advance\ltx@LocDimenA by -.5ex\relax
          \raise\ltx@LocDimenA\box\ltx@zero
        \endgroup
      \or % 4: \HOLOGO@PointReflectBox
        \lower.5ex\hbox{\HOLOGO@PointReflectBox{E}}%
      \else
        \@PackageError{hologo}{Internal error (choice/it}\@ehc
      \fi
    \else
      \ltx@IfUndefined{reflectbox}{%
        \ltx@IfUndefined{pgftext}{%
          \ltx@IfUndefined{psscalebox}{%
            \ltx@IfUndefined{HOLOGO@ScaleBox@\hologoDriver}{%
            }{%
              \chardef\HOLOGO@choice=4 %
            }%
          }{%
            \chardef\HOLOGO@choice=3 %
          }%
        }{%
          \chardef\HOLOGO@choice=2 %
        }%
      }{%
        \chardef\HOLOGO@choice=1 %
      }%
      \ifcase\HOLOGO@choice
        \HOLOGO@WarningUnsupportedDriver{Xe}%
        e%
      \or % 1: reflectbox
        \lower.5ex\hbox{%
          \reflectbox{E}%
        }%
      \or % 2: \pgftext
        \lower.5ex\hbox{%
          \pgfpicture
            \pgftransformxscale{-1}%
            \pgftext{E}%
          \endpgfpicture
        }%
      \or % 3: \psscalebox
        \lower.5ex\hbox{%
          \psscalebox{-1 1}{E}%
        }%
      \or % 4: \HOLOGO@Reflectbox
        \lower.5ex\hbox{%
          \HOLOGO@ReflectBox{E}%
        }%
      \else
        \@PackageError{hologo}{Internal error (choice/up)}\@ehc
      \fi
    \fi
  }%
}
%    \end{macrocode}
%    \end{macro}
%    \begin{macro}{\HoLogoHtml@Xe}
%    \begin{macrocode}
\def\HoLogoHtml@Xe#1{%
  \HoLogoCss@Xe
  \HOLOGO@Span{Xe}{%
    X%
    \HOLOGO@Span{e}{%
      \HCode{&\ltx@hashchar x018e;}%
    }%
  }%
}
%    \end{macrocode}
%    \end{macro}
%    \begin{macro}{\HoLogoCss@Xe}
%    \begin{macrocode}
\def\HoLogoCss@Xe{%
  \Css{%
    span.HoLogo-Xe span.HoLogo-e{%
      position:relative;%
      top:.5ex;%
      left-margin:-.1em;%
    }%
  }%
  \global\let\HoLogoCss@Xe\relax
}
%    \end{macrocode}
%    \end{macro}
%
%    \begin{macro}{\HoLogo@XeTeX}
%    \begin{macrocode}
\def\HoLogo@XeTeX#1{%
  \hologo{Xe}%
  \kern-.15em\relax
  \hologo{TeX}%
}
%    \end{macrocode}
%    \end{macro}
%
%    \begin{macro}{\HoLogoHtml@XeTeX}
%    \begin{macrocode}
\def\HoLogoHtml@XeTeX#1{%
  \HoLogoCss@XeTeX
  \HOLOGO@Span{XeTeX}{%
    \hologo{Xe}%
    \hologo{TeX}%
  }%
}
%    \end{macrocode}
%    \end{macro}
%    \begin{macro}{\HoLogoCss@XeTeX}
%    \begin{macrocode}
\def\HoLogoCss@XeTeX{%
  \Css{%
    span.HoLogo-XeTeX span.HoLogo-TeX{%
      margin-left:-.15em;%
    }%
  }%
  \global\let\HoLogoCss@XeTeX\relax
}
%    \end{macrocode}
%    \end{macro}
%
%    \begin{macro}{\HoLogo@XeLaTeX}
%    \begin{macrocode}
\def\HoLogo@XeLaTeX#1{%
  \hologo{Xe}%
  \kern-.13em%
  \hologo{LaTeX}%
}
%    \end{macrocode}
%    \end{macro}
%    \begin{macro}{\HoLogoHtml@XeLaTeX}
%    \begin{macrocode}
\def\HoLogoHtml@XeLaTeX#1{%
  \HoLogoCss@XeLaTeX
  \HOLOGO@Span{XeLaTeX}{%
    \hologo{Xe}%
    \hologo{LaTeX}%
  }%
}
%    \end{macrocode}
%    \end{macro}
%    \begin{macro}{\HoLogoCss@XeLaTeX}
%    \begin{macrocode}
\def\HoLogoCss@XeLaTeX{%
  \Css{%
    span.HoLogo-XeLaTeX span.HoLogo-Xe{%
      margin-right:-.13em;%
    }%
  }%
  \global\let\HoLogoCss@XeLaTeX\relax
}
%    \end{macrocode}
%    \end{macro}
%
% \subsubsection{\hologo{pdfTeX}, \hologo{pdfLaTeX}}
%
%    \begin{macro}{\HoLogo@pdfTeX}
%    \begin{macrocode}
\def\HoLogo@pdfTeX#1{%
  \HOLOGO@mbox{%
    #1{p}{P}df\hologo{TeX}%
  }%
}
%    \end{macrocode}
%    \end{macro}
%    \begin{macro}{\HoLogoCs@pdfTeX}
%    \begin{macrocode}
\def\HoLogoCs@pdfTeX#1{#1{p}{P}dfTeX}
%    \end{macrocode}
%    \end{macro}
%    \begin{macro}{\HoLogoBkm@pdfTeX}
%    \begin{macrocode}
\def\HoLogoBkm@pdfTeX#1{%
  #1{p}{P}df\hologo{TeX}%
}
%    \end{macrocode}
%    \end{macro}
%    \begin{macro}{\HoLogoHtml@pdfTeX}
%    \begin{macrocode}
\let\HoLogoHtml@pdfTeX\HoLogo@pdfTeX
%    \end{macrocode}
%    \end{macro}
%
%    \begin{macro}{\HoLogo@pdfLaTeX}
%    \begin{macrocode}
\def\HoLogo@pdfLaTeX#1{%
  \HOLOGO@mbox{%
    #1{p}{P}df\hologo{LaTeX}%
  }%
}
%    \end{macrocode}
%    \end{macro}
%    \begin{macro}{\HoLogoCs@pdfLaTeX}
%    \begin{macrocode}
\def\HoLogoCs@pdfLaTeX#1{#1{p}{P}dfLaTeX}
%    \end{macrocode}
%    \end{macro}
%    \begin{macro}{\HoLogoBkm@pdfLaTeX}
%    \begin{macrocode}
\def\HoLogoBkm@pdfLaTeX#1{%
  #1{p}{P}df\hologo{LaTeX}%
}
%    \end{macrocode}
%    \end{macro}
%    \begin{macro}{\HoLogoHtml@pdfLaTeX}
%    \begin{macrocode}
\let\HoLogoHtml@pdfLaTeX\HoLogo@pdfLaTeX
%    \end{macrocode}
%    \end{macro}
%
% \subsubsection{\hologo{VTeX}}
%
%    \begin{macro}{\HoLogo@VTeX}
%    \begin{macrocode}
\def\HoLogo@VTeX#1{%
  \HOLOGO@mbox{%
    V\hologo{TeX}%
  }%
}
%    \end{macrocode}
%    \end{macro}
%    \begin{macro}{\HoLogoHtml@VTeX}
%    \begin{macrocode}
\let\HoLogoHtml@VTeX\HoLogo@VTeX
%    \end{macrocode}
%    \end{macro}
%
% \subsubsection{\hologo{AmS}, \dots}
%
%    Source: class \xclass{amsdtx}
%
%    \begin{macro}{\HoLogo@AmS}
%    \begin{macrocode}
\def\HoLogo@AmS#1{%
  \HoLogoFont@font{AmS}{sy}{%
    A%
    \kern-.1667em%
    \lower.5ex\hbox{M}%
    \kern-.125em%
    S%
  }%
}
%    \end{macrocode}
%    \end{macro}
%    \begin{macro}{\HoLogoBkm@AmS}
%    \begin{macrocode}
\def\HoLogoBkm@AmS#1{AmS}
%    \end{macrocode}
%    \end{macro}
%    \begin{macro}{\HoLogoHtml@AmS}
%    \begin{macrocode}
\def\HoLogoHtml@AmS#1{%
  \HoLogoCss@AmS
%  \HoLogoFont@font{AmS}{sy}{%
    \HOLOGO@Span{AmS}{%
      A%
      \HOLOGO@Span{M}{M}%
      S%
    }%
%   }%
}
%    \end{macrocode}
%    \end{macro}
%    \begin{macro}{\HoLogoCss@AmS}
%    \begin{macrocode}
\def\HoLogoCss@AmS{%
  \Css{%
    span.HoLogo-AmS span.HoLogo-M{%
      position:relative;%
      top:.5ex;%
      margin-left:-.1667em;%
      margin-right:-.125em;%
      text-decoration:none;%
    }%
  }%
  \global\let\HoLogoCss@AmS\relax
}
%    \end{macrocode}
%    \end{macro}
%
%    \begin{macro}{\HoLogo@AmSTeX}
%    \begin{macrocode}
\def\HoLogo@AmSTeX#1{%
  \hologo{AmS}%
  \HOLOGO@hyphen
  \hologo{TeX}%
}
%    \end{macrocode}
%    \end{macro}
%    \begin{macro}{\HoLogoBkm@AmSTeX}
%    \begin{macrocode}
\def\HoLogoBkm@AmSTeX#1{AmS-TeX}%
%    \end{macrocode}
%    \end{macro}
%    \begin{macro}{\HoLogoHtml@AmSTeX}
%    \begin{macrocode}
\let\HoLogoHtml@AmSTeX\HoLogo@AmSTeX
%    \end{macrocode}
%    \end{macro}
%
%    \begin{macro}{\HoLogo@AmSLaTeX}
%    \begin{macrocode}
\def\HoLogo@AmSLaTeX#1{%
  \hologo{AmS}%
  \HOLOGO@hyphen
  \hologo{LaTeX}%
}
%    \end{macrocode}
%    \end{macro}
%    \begin{macro}{\HoLogoBkm@AmSLaTeX}
%    \begin{macrocode}
\def\HoLogoBkm@AmSLaTeX#1{AmS-LaTeX}%
%    \end{macrocode}
%    \end{macro}
%    \begin{macro}{\HoLogoHtml@AmSLaTeX}
%    \begin{macrocode}
\let\HoLogoHtml@AmSLaTeX\HoLogo@AmSLaTeX
%    \end{macrocode}
%    \end{macro}
%
% \subsubsection{\hologo{BibTeX}}
%
%    \begin{macro}{\HoLogo@BibTeX@sc}
%    A definition of \hologo{BibTeX} is provided in
%    the documentation source for the manual of \hologo{BibTeX}
%    \cite{btxdoc}.
%\begin{quote}
%\begin{verbatim}
%\def\BibTeX{%
%  {%
%    \rm
%    B%
%    \kern-.05em%
%    {%
%      \sc
%      i%
%      \kern-.025em %
%      b%
%    }%
%    \kern-.08em
%    T%
%    \kern-.1667em%
%    \lower.7ex\hbox{E}%
%    \kern-.125em%
%    X%
%  }%
%}
%\end{verbatim}
%\end{quote}
%    \begin{macrocode}
\def\HoLogo@BibTeX@sc#1{%
  B%
  \kern-.05em%
  \HoLogoFont@font{BibTeX}{sc}{%
    i%
    \kern-.025em%
    b%
  }%
  \HOLOGO@discretionary
  \kern-.08em%
  \hologo{TeX}%
}
%    \end{macrocode}
%    \end{macro}
%    \begin{macro}{\HoLogoHtml@BibTeX@sc}
%    \begin{macrocode}
\def\HoLogoHtml@BibTeX@sc#1{%
  \HoLogoCss@BibTeX@sc
  \HOLOGO@Span{BibTeX-sc}{%
    B%
    \HOLOGO@Span{i}{i}%
    \HOLOGO@Span{b}{b}%
    \hologo{TeX}%
  }%
}
%    \end{macrocode}
%    \end{macro}
%    \begin{macro}{\HoLogoCss@BibTeX@sc}
%    \begin{macrocode}
\def\HoLogoCss@BibTeX@sc{%
  \Css{%
    span.HoLogo-BibTeX-sc span.HoLogo-i{%
      margin-left:-.05em;%
      margin-right:-.025em;%
      font-variant:small-caps;%
    }%
  }%
  \Css{%
    span.HoLogo-BibTeX-sc span.HoLogo-b{%
      margin-right:-.08em;%
      font-variant:small-caps;%
    }%
  }%
  \global\let\HoLogoCss@BibTeX@sc\relax
}
%    \end{macrocode}
%    \end{macro}
%
%    \begin{macro}{\HoLogo@BibTeX@sf}
%    Variant \xoption{sf} avoids trouble with unavailable
%    small caps fonts (e.g., bold versions of Computer Modern or
%    Latin Modern). The definition is taken from
%    package \xpackage{dtklogos} \cite{dtklogos}.
%\begin{quote}
%\begin{verbatim}
%\DeclareRobustCommand{\BibTeX}{%
%  B%
%  \kern-.05em%
%  \hbox{%
%    $\m@th$% %% force math size calculations
%    \csname S@\f@size\endcsname
%    \fontsize\sf@size\z@
%    \math@fontsfalse
%    \selectfont
%    I%
%    \kern-.025em%
%    B
%  }%
%  \kern-.08em%
%  \-%
%  \TeX
%}
%\end{verbatim}
%\end{quote}
%    \begin{macrocode}
\def\HoLogo@BibTeX@sf#1{%
  B%
  \kern-.05em%
  \HoLogoFont@font{BibTeX}{bibsf}{%
    I%
    \kern-.025em%
    B%
  }%
  \HOLOGO@discretionary
  \kern-.08em%
  \hologo{TeX}%
}
%    \end{macrocode}
%    \end{macro}
%    \begin{macro}{\HoLogoHtml@BibTeX@sf}
%    \begin{macrocode}
\def\HoLogoHtml@BibTeX@sf#1{%
  \HoLogoCss@BibTeX@sf
  \HOLOGO@Span{BibTeX-sf}{%
    B%
    \HoLogoFont@font{BibTeX}{bibsf}{%
      \HOLOGO@Span{i}{I}%
      B%
    }%
    \hologo{TeX}%
  }%
}
%    \end{macrocode}
%    \end{macro}
%    \begin{macro}{\HoLogoCss@BibTeX@sf}
%    \begin{macrocode}
\def\HoLogoCss@BibTeX@sf{%
  \Css{%
    span.HoLogo-BibTeX-sf span.HoLogo-i{%
      margin-left:-.05em;%
      margin-right:-.025em;%
    }%
  }%
  \Css{%
    span.HoLogo-BibTeX-sf span.HoLogo-TeX{%
      margin-left:-.08em;%
    }%
  }%
  \global\let\HoLogoCss@BibTeX@sf\relax
}
%    \end{macrocode}
%    \end{macro}
%
%    \begin{macro}{\HoLogo@BibTeX}
%    \begin{macrocode}
\def\HoLogo@BibTeX{\HoLogo@BibTeX@sf}
%    \end{macrocode}
%    \end{macro}
%    \begin{macro}{\HoLogoHtml@BibTeX}
%    \begin{macrocode}
\def\HoLogoHtml@BibTeX{\HoLogoHtml@BibTeX@sf}
%    \end{macrocode}
%    \end{macro}
%
% \subsubsection{\hologo{BibTeX8}}
%
%    \begin{macro}{\HoLogo@BibTeX8}
%    \begin{macrocode}
\expandafter\def\csname HoLogo@BibTeX8\endcsname#1{%
  \hologo{BibTeX}%
  8%
}
%    \end{macrocode}
%    \end{macro}
%
%    \begin{macro}{\HoLogoBkm@BibTeX8}
%    \begin{macrocode}
\expandafter\def\csname HoLogoBkm@BibTeX8\endcsname#1{%
  \hologo{BibTeX}%
  8%
}
%    \end{macrocode}
%    \end{macro}
%    \begin{macro}{\HoLogoHtml@BibTeX8}
%    \begin{macrocode}
\expandafter
\let\csname HoLogoHtml@BibTeX8\expandafter\endcsname
\csname HoLogo@BibTeX8\endcsname
%    \end{macrocode}
%    \end{macro}
%
% \subsubsection{\hologo{ConTeXt}}
%
%    \begin{macro}{\HoLogo@ConTeXt@simple}
%    \begin{macrocode}
\def\HoLogo@ConTeXt@simple#1{%
  \HOLOGO@mbox{Con}%
  \HOLOGO@discretionary
  \HOLOGO@mbox{\hologo{TeX}t}%
}
%    \end{macrocode}
%    \end{macro}
%    \begin{macro}{\HoLogoHtml@ConTeXt@simple}
%    \begin{macrocode}
\let\HoLogoHtml@ConTeXt@simple\HoLogo@ConTeXt@simple
%    \end{macrocode}
%    \end{macro}
%
%    \begin{macro}{\HoLogo@ConTeXt@narrow}
%    This definition of logo \hologo{ConTeXt} with variant \xoption{narrow}
%    comes from TUGboat's class \xclass{ltugboat} (version 2010/11/15 v2.8).
%    \begin{macrocode}
\def\HoLogo@ConTeXt@narrow#1{%
  \HOLOGO@mbox{C\kern-.0333emon}%
  \HOLOGO@discretionary
  \kern-.0667em%
  \HOLOGO@mbox{\hologo{TeX}\kern-.0333emt}%
}
%    \end{macrocode}
%    \end{macro}
%    \begin{macro}{\HoLogoHtml@ConTeXt@narrow}
%    \begin{macrocode}
\def\HoLogoHtml@ConTeXt@narrow#1{%
  \HoLogoCss@ConTeXt@narrow
  \HOLOGO@Span{ConTeXt-narrow}{%
    \HOLOGO@Span{C}{C}%
    on%
    \hologo{TeX}%
    t%
  }%
}
%    \end{macrocode}
%    \end{macro}
%    \begin{macro}{\HoLogoCss@ConTeXt@narrow}
%    \begin{macrocode}
\def\HoLogoCss@ConTeXt@narrow{%
  \Css{%
    span.HoLogo-ConTeXt-narrow span.HoLogo-C{%
      margin-left:-.0333em;%
    }%
  }%
  \Css{%
    span.HoLogo-ConTeXt-narrow span.HoLogo-TeX{%
      margin-left:-.0667em;%
      margin-right:-.0333em;%
    }%
  }%
  \global\let\HoLogoCss@ConTeXt@narrow\relax
}
%    \end{macrocode}
%    \end{macro}
%
%    \begin{macro}{\HoLogo@ConTeXt}
%    \begin{macrocode}
\def\HoLogo@ConTeXt{\HoLogo@ConTeXt@narrow}
%    \end{macrocode}
%    \end{macro}
%    \begin{macro}{\HoLogoHtml@ConTeXt}
%    \begin{macrocode}
\def\HoLogoHtml@ConTeXt{\HoLogoHtml@ConTeXt@narrow}
%    \end{macrocode}
%    \end{macro}
%
% \subsubsection{\hologo{emTeX}}
%
%    \begin{macro}{\HoLogo@emTeX}
%    \begin{macrocode}
\def\HoLogo@emTeX#1{%
  \HOLOGO@mbox{#1{e}{E}m}%
  \HOLOGO@discretionary
  \hologo{TeX}%
}
%    \end{macrocode}
%    \end{macro}
%    \begin{macro}{\HoLogoCs@emTeX}
%    \begin{macrocode}
\def\HoLogoCs@emTeX#1{#1{e}{E}mTeX}%
%    \end{macrocode}
%    \end{macro}
%    \begin{macro}{\HoLogoBkm@emTeX}
%    \begin{macrocode}
\def\HoLogoBkm@emTeX#1{%
  #1{e}{E}m\hologo{TeX}%
}
%    \end{macrocode}
%    \end{macro}
%    \begin{macro}{\HoLogoHtml@emTeX}
%    \begin{macrocode}
\let\HoLogoHtml@emTeX\HoLogo@emTeX
%    \end{macrocode}
%    \end{macro}
%
% \subsubsection{\hologo{ExTeX}}
%
%    \begin{macro}{\HoLogo@ExTeX}
%    The definition is taken from the FAQ of the
%    project \hologo{ExTeX}
%    \cite{ExTeX-FAQ}.
%\begin{quote}
%\begin{verbatim}
%\def\ExTeX{%
%  \textrm{% Logo always with serifs
%    \ensuremath{%
%      \textstyle
%      \varepsilon_{%
%        \kern-0.15em%
%        \mathcal{X}%
%      }%
%    }%
%    \kern-.15em%
%    \TeX
%  }%
%}
%\end{verbatim}
%\end{quote}
%    \begin{macrocode}
\def\HoLogo@ExTeX#1{%
  \HoLogoFont@font{ExTeX}{rm}{%
    \ltx@mbox{%
      \HOLOGO@MathSetup
      $%
        \textstyle
        \varepsilon_{%
          \kern-0.15em%
          \HoLogoFont@font{ExTeX}{sy}{X}%
        }%
      $%
    }%
    \HOLOGO@discretionary
    \kern-.15em%
    \hologo{TeX}%
  }%
}
%    \end{macrocode}
%    \end{macro}
%    \begin{macro}{\HoLogoHtml@ExTeX}
%    \begin{macrocode}
\def\HoLogoHtml@ExTeX#1{%
  \HoLogoCss@ExTeX
  \HoLogoFont@font{ExTeX}{rm}{%
    \HOLOGO@Span{ExTeX}{%
      \ltx@mbox{%
        \HOLOGO@MathSetup
        $\textstyle\varepsilon$%
        \HOLOGO@Span{X}{$\textstyle\chi$}%
        \hologo{TeX}%
      }%
    }%
  }%
}
%    \end{macrocode}
%    \end{macro}
%    \begin{macro}{\HoLogoBkm@ExTeX}
%    \begin{macrocode}
\def\HoLogoBkm@ExTeX#1{%
  \HOLOGO@PdfdocUnicode{#1{e}{E}x}{\textepsilon\textchi}%
  \hologo{TeX}%
}
%    \end{macrocode}
%    \end{macro}
%    \begin{macro}{\HoLogoCss@ExTeX}
%    \begin{macrocode}
\def\HoLogoCss@ExTeX{%
  \Css{%
    span.HoLogo-ExTeX{%
      font-family:serif;%
    }%
  }%
  \Css{%
    span.HoLogo-ExTeX span.HoLogo-TeX{%
      margin-left:-.15em;%
    }%
  }%
  \global\let\HoLogoCss@ExTeX\relax
}
%    \end{macrocode}
%    \end{macro}
%
% \subsubsection{\hologo{MiKTeX}}
%
%    \begin{macro}{\HoLogo@MiKTeX}
%    \begin{macrocode}
\def\HoLogo@MiKTeX#1{%
  \HOLOGO@mbox{MiK}%
  \HOLOGO@discretionary
  \hologo{TeX}%
}
%    \end{macrocode}
%    \end{macro}
%    \begin{macro}{\HoLogoHtml@MiKTeX}
%    \begin{macrocode}
\let\HoLogoHtml@MiKTeX\HoLogo@MiKTeX
%    \end{macrocode}
%    \end{macro}
%
% \subsubsection{\hologo{OzTeX} and friends}
%
%    Source: \hologo{OzTeX} FAQ \cite{OzTeX}:
%    \begin{quote}
%      |\def\OzTeX{O\kern-.03em z\kern-.15em\TeX}|\\
%      (There is no kerning in OzMF, OzMP and OzTtH.)
%    \end{quote}
%
%    \begin{macro}{\HoLogo@OzTeX}
%    \begin{macrocode}
\def\HoLogo@OzTeX#1{%
  O%
  \kern-.03em %
  z%
  \kern-.15em %
  \hologo{TeX}%
}
%    \end{macrocode}
%    \end{macro}
%    \begin{macro}{\HoLogoHtml@OzTeX}
%    \begin{macrocode}
\def\HoLogoHtml@OzTeX#1{%
  \HoLogoCss@OzTeX
  \HOLOGO@Span{OzTeX}{%
    O%
    \HOLOGO@Span{z}{z}%
    \hologo{TeX}%
  }%
}
%    \end{macrocode}
%    \end{macro}
%    \begin{macro}{\HoLogoCss@OzTeX}
%    \begin{macrocode}
\def\HoLogoCss@OzTeX{%
  \Css{%
    span.HoLogo-OzTeX span.HoLogo-z{%
      margin-left:-.03em;%
      margin-right:-.15em;%
    }%
  }%
  \global\let\HoLogoCss@OzTeX\relax
}
%    \end{macrocode}
%    \end{macro}
%
%    \begin{macro}{\HoLogo@OzMF}
%    \begin{macrocode}
\def\HoLogo@OzMF#1{%
  \HOLOGO@mbox{OzMF}%
}
%    \end{macrocode}
%    \end{macro}
%    \begin{macro}{\HoLogo@OzMP}
%    \begin{macrocode}
\def\HoLogo@OzMP#1{%
  \HOLOGO@mbox{OzMP}%
}
%    \end{macrocode}
%    \end{macro}
%    \begin{macro}{\HoLogo@OzTtH}
%    \begin{macrocode}
\def\HoLogo@OzTtH#1{%
  \HOLOGO@mbox{OzTtH}%
}
%    \end{macrocode}
%    \end{macro}
%
% \subsubsection{\hologo{PCTeX}}
%
%    \begin{macro}{\HoLogo@PCTeX}
%    \begin{macrocode}
\def\HoLogo@PCTeX#1{%
  \HOLOGO@mbox{PC}%
  \hologo{TeX}%
}
%    \end{macrocode}
%    \end{macro}
%    \begin{macro}{\HoLogoHtml@PCTeX}
%    \begin{macrocode}
\let\HoLogoHtml@PCTeX\HoLogo@PCTeX
%    \end{macrocode}
%    \end{macro}
%
% \subsubsection{\hologo{PiCTeX}}
%
%    The original definitions from \xfile{pictex.tex} \cite{PiCTeX}:
%\begin{quote}
%\begin{verbatim}
%\def\PiC{%
%  P%
%  \kern-.12em%
%  \lower.5ex\hbox{I}%
%  \kern-.075em%
%  C%
%}
%\def\PiCTeX{%
%  \PiC
%  \kern-.11em%
%  \TeX
%}
%\end{verbatim}
%\end{quote}
%
%    \begin{macro}{\HoLogo@PiC}
%    \begin{macrocode}
\def\HoLogo@PiC#1{%
  P%
  \kern-.12em%
  \lower.5ex\hbox{I}%
  \kern-.075em%
  C%
  \HOLOGO@SpaceFactor
}
%    \end{macrocode}
%    \end{macro}
%    \begin{macro}{\HoLogoHtml@PiC}
%    \begin{macrocode}
\def\HoLogoHtml@PiC#1{%
  \HoLogoCss@PiC
  \HOLOGO@Span{PiC}{%
    P%
    \HOLOGO@Span{i}{I}%
    C%
  }%
}
%    \end{macrocode}
%    \end{macro}
%    \begin{macro}{\HoLogoCss@PiC}
%    \begin{macrocode}
\def\HoLogoCss@PiC{%
  \Css{%
    span.HoLogo-PiC span.HoLogo-i{%
      position:relative;%
      top:.5ex;%
      margin-left:-.12em;%
      margin-right:-.075em;%
      text-decoration:none;%
    }%
  }%
  \global\let\HoLogoCss@PiC\relax
}
%    \end{macrocode}
%    \end{macro}
%
%    \begin{macro}{\HoLogo@PiCTeX}
%    \begin{macrocode}
\def\HoLogo@PiCTeX#1{%
  \hologo{PiC}%
  \HOLOGO@discretionary
  \kern-.11em%
  \hologo{TeX}%
}
%    \end{macrocode}
%    \end{macro}
%    \begin{macro}{\HoLogoHtml@PiCTeX}
%    \begin{macrocode}
\def\HoLogoHtml@PiCTeX#1{%
  \HoLogoCss@PiCTeX
  \HOLOGO@Span{PiCTeX}{%
    \hologo{PiC}%
    \hologo{TeX}%
  }%
}
%    \end{macrocode}
%    \end{macro}
%    \begin{macro}{\HoLogoCss@PiCTeX}
%    \begin{macrocode}
\def\HoLogoCss@PiCTeX{%
  \Css{%
    span.HoLogo-PiCTeX span.HoLogo-PiC{%
      margin-right:-.11em;%
    }%
  }%
  \global\let\HoLogoCss@PiCTeX\relax
}
%    \end{macrocode}
%    \end{macro}
%
% \subsubsection{\hologo{teTeX}}
%
%    \begin{macro}{\HoLogo@teTeX}
%    \begin{macrocode}
\def\HoLogo@teTeX#1{%
  \HOLOGO@mbox{#1{t}{T}e}%
  \HOLOGO@discretionary
  \hologo{TeX}%
}
%    \end{macrocode}
%    \end{macro}
%    \begin{macro}{\HoLogoCs@teTeX}
%    \begin{macrocode}
\def\HoLogoCs@teTeX#1{#1{t}{T}dfTeX}
%    \end{macrocode}
%    \end{macro}
%    \begin{macro}{\HoLogoBkm@teTeX}
%    \begin{macrocode}
\def\HoLogoBkm@teTeX#1{%
  #1{t}{T}e\hologo{TeX}%
}
%    \end{macrocode}
%    \end{macro}
%    \begin{macro}{\HoLogoHtml@teTeX}
%    \begin{macrocode}
\let\HoLogoHtml@teTeX\HoLogo@teTeX
%    \end{macrocode}
%    \end{macro}
%
% \subsubsection{\hologo{TeX4ht}}
%
%    \begin{macro}{\HoLogo@TeX4ht}
%    \begin{macrocode}
\expandafter\def\csname HoLogo@TeX4ht\endcsname#1{%
  \HOLOGO@mbox{\hologo{TeX}4ht}%
}
%    \end{macrocode}
%    \end{macro}
%    \begin{macro}{\HoLogoHtml@TeX4ht}
%    \begin{macrocode}
\expandafter
\let\csname HoLogoHtml@TeX4ht\expandafter\endcsname
\csname HoLogo@TeX4ht\endcsname
%    \end{macrocode}
%    \end{macro}
%
%
% \subsubsection{\hologo{SageTeX}}
%
%    \begin{macro}{\HoLogo@SageTeX}
%    \begin{macrocode}
\def\HoLogo@SageTeX#1{%
  \HOLOGO@mbox{Sage}%
  \HOLOGO@discretionary
  \HOLOGO@NegativeKerning{eT,oT,To}%
  \hologo{TeX}%
}
%    \end{macrocode}
%    \end{macro}
%    \begin{macro}{\HoLogoHtml@SageTeX}
%    \begin{macrocode}
\let\HoLogoHtml@SageTeX\HoLogo@SageTeX
%    \end{macrocode}
%    \end{macro}
%
% \subsection{\hologo{METAFONT} and friends}
%
%    \begin{macro}{\HoLogo@METAFONT}
%    \begin{macrocode}
\def\HoLogo@METAFONT#1{%
  \HoLogoFont@font{METAFONT}{logo}{%
    \HOLOGO@mbox{META}%
    \HOLOGO@discretionary
    \HOLOGO@mbox{FONT}%
  }%
}
%    \end{macrocode}
%    \end{macro}
%
%    \begin{macro}{\HoLogo@METAPOST}
%    \begin{macrocode}
\def\HoLogo@METAPOST#1{%
  \HoLogoFont@font{METAPOST}{logo}{%
    \HOLOGO@mbox{META}%
    \HOLOGO@discretionary
    \HOLOGO@mbox{POST}%
  }%
}
%    \end{macrocode}
%    \end{macro}
%
%    \begin{macro}{\HoLogo@MetaFun}
%    \begin{macrocode}
\def\HoLogo@MetaFun#1{%
  \HOLOGO@mbox{Meta}%
  \HOLOGO@discretionary
  \HOLOGO@mbox{Fun}%
}
%    \end{macrocode}
%    \end{macro}
%
%    \begin{macro}{\HoLogo@MetaPost}
%    \begin{macrocode}
\def\HoLogo@MetaPost#1{%
  \HOLOGO@mbox{Meta}%
  \HOLOGO@discretionary
  \HOLOGO@mbox{Post}%
}
%    \end{macrocode}
%    \end{macro}
%
% \subsection{Others}
%
% \subsubsection{\hologo{biber}}
%
%    \begin{macro}{\HoLogo@biber}
%    \begin{macrocode}
\def\HoLogo@biber#1{%
  \HOLOGO@mbox{#1{b}{B}i}%
  \HOLOGO@discretionary
  \HOLOGO@mbox{ber}%
}
%    \end{macrocode}
%    \end{macro}
%    \begin{macro}{\HoLogoCs@biber}
%    \begin{macrocode}
\def\HoLogoCs@biber#1{#1{b}{B}iber}
%    \end{macrocode}
%    \end{macro}
%    \begin{macro}{\HoLogoBkm@biber}
%    \begin{macrocode}
\def\HoLogoBkm@biber#1{%
  #1{b}{B}iber%
}
%    \end{macrocode}
%    \end{macro}
%    \begin{macro}{\HoLogoHtml@biber}
%    \begin{macrocode}
\let\HoLogoHtml@biber\HoLogo@biber
%    \end{macrocode}
%    \end{macro}
%
% \subsubsection{\hologo{KOMAScript}}
%
%    \begin{macro}{\HoLogo@KOMAScript}
%    The definition for \hologo{KOMAScript} is taken
%    from \hologo{KOMAScript} (\xfile{scrlogo.dtx}, reformatted) \cite{scrlogo}:
%\begin{quote}
%\begin{verbatim}
%\@ifundefined{KOMAScript}{%
%  \DeclareRobustCommand{\KOMAScript}{%
%    \textsf{%
%      K\kern.05em O\kern.05emM\kern.05em A%
%      \kern.1em-\kern.1em %
%      Script%
%    }%
%  }%
%}{}
%\end{verbatim}
%\end{quote}
%    \begin{macrocode}
\def\HoLogo@KOMAScript#1{%
  \HoLogoFont@font{KOMAScript}{sf}{%
    \HOLOGO@mbox{%
      K\kern.05em%
      O\kern.05em%
      M\kern.05em%
      A%
    }%
    \kern.1em%
    \HOLOGO@hyphen
    \kern.1em%
    \HOLOGO@mbox{Script}%
  }%
}
%    \end{macrocode}
%    \end{macro}
%    \begin{macro}{\HoLogoBkm@KOMAScript}
%    \begin{macrocode}
\def\HoLogoBkm@KOMAScript#1{%
  KOMA-Script%
}
%    \end{macrocode}
%    \end{macro}
%    \begin{macro}{\HoLogoHtml@KOMAScript}
%    \begin{macrocode}
\def\HoLogoHtml@KOMAScript#1{%
  \HoLogoCss@KOMAScript
  \HoLogoFont@font{KOMAScript}{sf}{%
    \HOLOGO@Span{KOMAScript}{%
      K%
      \HOLOGO@Span{O}{O}%
      M%
      \HOLOGO@Span{A}{A}%
      \HOLOGO@Span{hyphen}{-}%
      Script%
    }%
  }%
}
%    \end{macrocode}
%    \end{macro}
%    \begin{macro}{\HoLogoCss@KOMAScript}
%    \begin{macrocode}
\def\HoLogoCss@KOMAScript{%
  \Css{%
    span.HoLogo-KOMAScript{%
      font-family:sans-serif;%
    }%
  }%
  \Css{%
    span.HoLogo-KOMAScript span.HoLogo-O{%
      padding-left:.05em;%
      padding-right:.05em;%
    }%
  }%
  \Css{%
    span.HoLogo-KOMAScript span.HoLogo-A{%
      padding-left:.05em;%
    }%
  }%
  \Css{%
    span.HoLogo-KOMAScript span.HoLogo-hyphen{%
      padding-left:.1em;%
      padding-right:.1em;%
    }%
  }%
  \global\let\HoLogoCss@KOMAScript\relax
}
%    \end{macrocode}
%    \end{macro}
%
% \subsubsection{\hologo{LyX}}
%
%    \begin{macro}{\HoLogo@LyX}
%    The definition is taken from the documentation source files
%    of \hologo{LyX}, \xfile{Intro.lyx} \cite{LyX}:
%\begin{quote}
%\begin{verbatim}
%\def\LyX{%
%  \texorpdfstring{%
%    L\kern-.1667em\lower.25em\hbox{Y}\kern-.125emX\@%
%  }{%
%    LyX%
%  }%
%}
%\end{verbatim}
%\end{quote}
%    \begin{macrocode}
\def\HoLogo@LyX#1{%
  L%
  \kern-.1667em%
  \lower.25em\hbox{Y}%
  \kern-.125em%
  X%
  \HOLOGO@SpaceFactor
}
%    \end{macrocode}
%    \end{macro}
%    \begin{macro}{\HoLogoHtml@LyX}
%    \begin{macrocode}
\def\HoLogoHtml@LyX#1{%
  \HoLogoCss@LyX
  \HOLOGO@Span{LyX}{%
    L%
    \HOLOGO@Span{y}{Y}%
    X%
  }%
}
%    \end{macrocode}
%    \end{macro}
%    \begin{macro}{\HoLogoCss@LyX}
%    \begin{macrocode}
\def\HoLogoCss@LyX{%
  \Css{%
    span.HoLogo-LyX span.HoLogo-y{%
      position:relative;%
      top:.25em;%
      margin-left:-.1667em;%
      margin-right:-.125em;%
      text-decoration:none;%
    }%
  }%
  \global\let\HoLogoCss@LyX\relax
}
%    \end{macrocode}
%    \end{macro}
%
% \subsubsection{\hologo{NTS}}
%
%    \begin{macro}{\HoLogo@NTS}
%    Definition for \hologo{NTS} can be found in
%    package \xpackage{etex\textunderscore man} for the \hologo{eTeX} manual \cite{etexman}
%    and in package \xpackage{dtklogos} \cite{dtklogos}:
%\begin{quote}
%\begin{verbatim}
%\def\NTS{%
%  \leavevmode
%  \hbox{%
%    $%
%      \cal N%
%      \kern-0.35em%
%      \lower0.5ex\hbox{$\cal T$}%
%      \kern-0.2em%
%      S%
%    $%
%  }%
%}
%\end{verbatim}
%\end{quote}
%    \begin{macrocode}
\def\HoLogo@NTS#1{%
  \HoLogoFont@font{NTS}{sy}{%
    N\/%
    \kern-.35em%
    \lower.5ex\hbox{T\/}%
    \kern-.2em%
    S\/%
  }%
  \HOLOGO@SpaceFactor
}
%    \end{macrocode}
%    \end{macro}
%
% \subsubsection{\Hologo{TTH} (\hologo{TeX} to HTML translator)}
%
%    Source: \url{http://hutchinson.belmont.ma.us/tth/}
%    In the HTML source the second `T' is printed as subscript.
%\begin{quote}
%\begin{verbatim}
%T<sub>T</sub>H
%\end{verbatim}
%\end{quote}
%    \begin{macro}{\HoLogo@TTH}
%    \begin{macrocode}
\def\HoLogo@TTH#1{%
  \ltx@mbox{%
    T\HOLOGO@SubScript{T}H%
  }%
  \HOLOGO@SpaceFactor
}
%    \end{macrocode}
%    \end{macro}
%
%    \begin{macro}{\HoLogoHtml@TTH}
%    \begin{macrocode}
\def\HoLogoHtml@TTH#1{%
  T\HCode{<sub>}T\HCode{</sub>}H%
}
%    \end{macrocode}
%    \end{macro}
%
% \subsubsection{\Hologo{HanTheThanh}}
%
%    Partial source: Package \xpackage{dtklogos}.
%    The double accent is U+1EBF (latin small letter e with circumflex
%    and acute).
%    \begin{macro}{\HoLogo@HanTheThanh}
%    \begin{macrocode}
\def\HoLogo@HanTheThanh#1{%
  \ltx@mbox{H\`an}%
  \HOLOGO@space
  \ltx@mbox{%
    Th%
    \HOLOGO@IfCharExists{"1EBF}{%
      \char"1EBF\relax
    }{%
      \^e\hbox to 0pt{\hss\raise .5ex\hbox{\'{}}}%
    }%
  }%
  \HOLOGO@space
  \ltx@mbox{Th\`anh}%
}
%    \end{macrocode}
%    \end{macro}
%    \begin{macro}{\HoLogoBkm@HanTheThanh}
%    \begin{macrocode}
\def\HoLogoBkm@HanTheThanh#1{%
  H\`an %
  Th\HOLOGO@PdfdocUnicode{\^e}{\9036\277} %
  Th\`anh%
}
%    \end{macrocode}
%    \end{macro}
%    \begin{macro}{\HoLogoHtml@HanTheThanh}
%    \begin{macrocode}
\def\HoLogoHtml@HanTheThanh#1{%
  H\`an %
  Th\HCode{&\ltx@hashchar x1ebf;} %
  Th\`anh%
}
%    \end{macrocode}
%    \end{macro}
%
% \subsection{Driver detection}
%
%    \begin{macrocode}
\HOLOGO@IfExists\InputIfFileExists{%
  \InputIfFileExists{hologo.cfg}{}{}%
}{%
  \ltx@IfUndefined{pdf@filesize}{%
    \def\HOLOGO@InputIfExists{%
      \openin\HOLOGO@temp=hologo.cfg\relax
      \ifeof\HOLOGO@temp
        \closein\HOLOGO@temp
      \else
        \closein\HOLOGO@temp
        \begingroup
          \def\x{LaTeX2e}%
        \expandafter\endgroup
        \ifx\fmtname\x
          \input{hologo.cfg}%
        \else
          \input hologo.cfg\relax
        \fi
      \fi
    }%
    \ltx@IfUndefined{newread}{%
      \chardef\HOLOGO@temp=15 %
      \def\HOLOGO@CheckRead{%
        \ifeof\HOLOGO@temp
          \HOLOGO@InputIfExists
        \else
          \ifcase\HOLOGO@temp
            \@PackageWarningNoLine{hologo}{%
              Configuration file ignored, because\MessageBreak
              a free read register could not be found%
            }%
          \else
            \begingroup
              \count\ltx@cclv=\HOLOGO@temp
              \advance\ltx@cclv by \ltx@minusone
              \edef\x{\endgroup
                \chardef\noexpand\HOLOGO@temp=\the\count\ltx@cclv
                \relax
              }%
            \x
          \fi
        \fi
      }%
    }{%
      \csname newread\endcsname\HOLOGO@temp
      \HOLOGO@InputIfExists
    }%
  }{%
    \edef\HOLOGO@temp{\pdf@filesize{hologo.cfg}}%
    \ifx\HOLOGO@temp\ltx@empty
    \else
      \ifnum\HOLOGO@temp>0 %
        \begingroup
          \def\x{LaTeX2e}%
        \expandafter\endgroup
        \ifx\fmtname\x
          \input{hologo.cfg}%
        \else
          \input hologo.cfg\relax
        \fi
      \else
        \@PackageInfoNoLine{hologo}{%
          Empty configuration file `hologo.cfg' ignored%
        }%
      \fi
    \fi
  }%
}
%    \end{macrocode}
%
%    \begin{macrocode}
\def\HOLOGO@temp#1#2{%
  \kv@define@key{HoLogoDriver}{#1}[]{%
    \begingroup
      \def\HOLOGO@temp{##1}%
      \ltx@onelevel@sanitize\HOLOGO@temp
      \ifx\HOLOGO@temp\ltx@empty
      \else
        \@PackageError{hologo}{%
          Value (\HOLOGO@temp) not permitted for option `#1'%
        }%
        \@ehc
      \fi
    \endgroup
    \def\hologoDriver{#2}%
  }%
}%
\def\HOLOGO@@temp#1#2{%
  \ifx\kv@value\relax
    \HOLOGO@temp{#1}{#1}%
  \else
    \HOLOGO@temp{#1}{#2}%
  \fi
}%
\kv@parse@normalized{%
  pdftex,%
  luatex=pdftex,%
  dvipdfm,%
  dvipdfmx=dvipdfm,%
  dvips,%
  dvipsone=dvips,%
  xdvi=dvips,%
  xetex,%
  vtex,%
}\HOLOGO@@temp
%    \end{macrocode}
%
%    \begin{macrocode}
\kv@define@key{HoLogoDriver}{driverfallback}{%
  \def\HOLOGO@DriverFallback{#1}%
}
%    \end{macrocode}
%
%    \begin{macro}{\HOLOGO@DriverFallback}
%    \begin{macrocode}
\def\HOLOGO@DriverFallback{dvips}
%    \end{macrocode}
%    \end{macro}
%
%    \begin{macro}{\hologoDriverSetup}
%    \begin{macrocode}
\def\hologoDriverSetup{%
  \let\hologoDriver\ltx@undefined
  \HOLOGO@DriverSetup
}
%    \end{macrocode}
%    \end{macro}
%
%    \begin{macro}{\HOLOGO@DriverSetup}
%    \begin{macrocode}
\def\HOLOGO@DriverSetup#1{%
  \kvsetkeys{HoLogoDriver}{#1}%
  \HOLOGO@CheckDriver
  \ltx@ifundefined{hologoDriver}{%
    \begingroup
    \edef\x{\endgroup
      \noexpand\kvsetkeys{HoLogoDriver}{\HOLOGO@DriverFallback}%
    }\x
  }{}%
  \@PackageInfoNoLine{hologo}{Using driver `\hologoDriver'}%
}
%    \end{macrocode}
%    \end{macro}
%
%    \begin{macro}{\HOLOGO@CheckDriver}
%    \begin{macrocode}
\def\HOLOGO@CheckDriver{%
  \ifpdf
    \def\hologoDriver{pdftex}%
    \let\HOLOGO@pdfliteral\pdfliteral
    \ifluatex
      \ifx\pdfextension\@undefined\else
        \protected\def\pdfliteral{\pdfextension literal}%
        \let\HOLOGO@pdfliteral\pdfliteral
      \fi
      \ltx@IfUndefined{HOLOGO@pdfliteral}{%
        \ifnum\luatexversion<36 %
        \else
          \begingroup
            \let\HOLOGO@temp\endgroup
            \ifcase0%
                \directlua{%
                  if tex.enableprimitives then %
                    tex.enableprimitives('HOLOGO@', {'pdfliteral'})%
                  else %
                    tex.print('1')%
                  end%
                }%
                \ifx\HOLOGO@pdfliteral\@undefined 1\fi%
                \relax%
              \endgroup
              \let\HOLOGO@temp\relax
              \global\let\HOLOGO@pdfliteral\HOLOGO@pdfliteral
            \fi%
          \HOLOGO@temp
        \fi
      }{}%
    \fi
    \ltx@IfUndefined{HOLOGO@pdfliteral}{%
      \@PackageWarningNoLine{hologo}{%
        Cannot find \string\pdfliteral
      }%
    }{}%
  \else
    \ifxetex
      \def\hologoDriver{xetex}%
    \else
      \ifvtex
        \def\hologoDriver{vtex}%
      \fi
    \fi
  \fi
}
%    \end{macrocode}
%    \end{macro}
%
%    \begin{macro}{\HOLOGO@WarningUnsupportedDriver}
%    \begin{macrocode}
\def\HOLOGO@WarningUnsupportedDriver#1{%
  \@PackageWarningNoLine{hologo}{%
    Logo `#1' needs driver specific macros,\MessageBreak
    but driver `\hologoDriver' is not supported.\MessageBreak
    Use a different driver or\MessageBreak
    load package `graphics' or `pgf'%
  }%
}
%    \end{macrocode}
%    \end{macro}
%
% \subsubsection{Reflect box macros}
%
%    Skip driver part if not needed.
%    \begin{macrocode}
\ltx@IfUndefined{reflectbox}{}{%
  \ltx@IfUndefined{rotatebox}{}{%
    \HOLOGO@AtEnd
  }%
}
\ltx@IfUndefined{pgftext}{}{%
  \HOLOGO@AtEnd
}
\ltx@IfUndefined{psscalebox}{}{%
  \HOLOGO@AtEnd
}
%    \end{macrocode}
%
%    \begin{macrocode}
\def\HOLOGO@temp{LaTeX2e}
\ifx\fmtname\HOLOGO@temp
  \RequirePackage{kvoptions}[2011/06/30]%
  \ProcessKeyvalOptions{HoLogoDriver}%
\fi
\HOLOGO@DriverSetup{}
%    \end{macrocode}
%
%    \begin{macro}{\HOLOGO@ReflectBox}
%    \begin{macrocode}
\def\HOLOGO@ReflectBox#1{%
  \begingroup
    \setbox\ltx@zero\hbox{\begingroup#1\endgroup}%
    \setbox\ltx@two\hbox{%
      \kern\wd\ltx@zero
      \csname HOLOGO@ScaleBox@\hologoDriver\endcsname{-1}{1}{%
        \hbox to 0pt{\copy\ltx@zero\hss}%
      }%
    }%
    \wd\ltx@two=\wd\ltx@zero
    \box\ltx@two
  \endgroup
}
%    \end{macrocode}
%    \end{macro}
%
%    \begin{macro}{\HOLOGO@PointReflectBox}
%    \begin{macrocode}
\def\HOLOGO@PointReflectBox#1{%
  \begingroup
    \setbox\ltx@zero\hbox{\begingroup#1\endgroup}%
    \setbox\ltx@two\hbox{%
      \kern\wd\ltx@zero
      \raise\ht\ltx@zero\hbox{%
        \csname HOLOGO@ScaleBox@\hologoDriver\endcsname{-1}{-1}{%
          \hbox to 0pt{\copy\ltx@zero\hss}%
        }%
      }%
    }%
    \wd\ltx@two=\wd\ltx@zero
    \box\ltx@two
  \endgroup
}
%    \end{macrocode}
%    \end{macro}
%
%    We must define all variants because of dynamic driver setup.
%    \begin{macrocode}
\def\HOLOGO@temp#1#2{#2}
%    \end{macrocode}
%
%    \begin{macro}{\HOLOGO@ScaleBox@pdftex}
%    \begin{macrocode}
\HOLOGO@temp{pdftex}{%
  \def\HOLOGO@ScaleBox@pdftex#1#2#3{%
    \HOLOGO@pdfliteral{%
      q #1 0 0 #2 0 0 cm%
    }%
    #3%
    \HOLOGO@pdfliteral{%
      Q%
    }%
  }%
}
%    \end{macrocode}
%    \end{macro}
%    \begin{macro}{\HOLOGO@ScaleBox@dvips}
%    \begin{macrocode}
\HOLOGO@temp{dvips}{%
  \def\HOLOGO@ScaleBox@dvips#1#2#3{%
    \special{ps:%
      gsave %
      currentpoint %
      currentpoint translate %
      #1 #2 scale %
      neg exch neg exch translate%
    }%
    #3%
    \special{ps:%
      currentpoint %
      grestore %
      moveto%
    }%
  }%
}
%    \end{macrocode}
%    \end{macro}
%    \begin{macro}{\HOLOGO@ScaleBox@dvipdfm}
%    \begin{macrocode}
\HOLOGO@temp{dvipdfm}{%
  \let\HOLOGO@ScaleBox@dvipdfm\HOLOGO@ScaleBox@dvips
}
%    \end{macrocode}
%    \end{macro}
%    Since \hologo{XeTeX} v0.6.
%    \begin{macro}{\HOLOGO@ScaleBox@xetex}
%    \begin{macrocode}
\HOLOGO@temp{xetex}{%
  \def\HOLOGO@ScaleBox@xetex#1#2#3{%
    \special{x:gsave}%
    \special{x:scale #1 #2}%
    #3%
    \special{x:grestore}%
  }%
}
%    \end{macrocode}
%    \end{macro}
%    \begin{macro}{\HOLOGO@ScaleBox@vtex}
%    \begin{macrocode}
\HOLOGO@temp{vtex}{%
  \def\HOLOGO@ScaleBox@vtex#1#2#3{%
    \special{r(#1,0,0,#2,0,0}%
    #3%
    \special{r)}%
  }%
}
%    \end{macrocode}
%    \end{macro}
%
%    \begin{macrocode}
\HOLOGO@AtEnd%
%</package>
%    \end{macrocode}
%
% \section{Test}
%
% \subsection{Catcode checks for loading}
%
%    \begin{macrocode}
%<*test1>
%    \end{macrocode}
%    \begin{macrocode}
\catcode`\{=1 %
\catcode`\}=2 %
\catcode`\#=6 %
\catcode`\@=11 %
\expandafter\ifx\csname count@\endcsname\relax
  \countdef\count@=255 %
\fi
\expandafter\ifx\csname @gobble\endcsname\relax
  \long\def\@gobble#1{}%
\fi
\expandafter\ifx\csname @firstofone\endcsname\relax
  \long\def\@firstofone#1{#1}%
\fi
\expandafter\ifx\csname loop\endcsname\relax
  \expandafter\@firstofone
\else
  \expandafter\@gobble
\fi
{%
  \def\loop#1\repeat{%
    \def\body{#1}%
    \iterate
  }%
  \def\iterate{%
    \body
      \let\next\iterate
    \else
      \let\next\relax
    \fi
    \next
  }%
  \let\repeat=\fi
}%
\def\RestoreCatcodes{}
\count@=0 %
\loop
  \edef\RestoreCatcodes{%
    \RestoreCatcodes
    \catcode\the\count@=\the\catcode\count@\relax
  }%
\ifnum\count@<255 %
  \advance\count@ 1 %
\repeat

\def\RangeCatcodeInvalid#1#2{%
  \count@=#1\relax
  \loop
    \catcode\count@=15 %
  \ifnum\count@<#2\relax
    \advance\count@ 1 %
  \repeat
}
\def\RangeCatcodeCheck#1#2#3{%
  \count@=#1\relax
  \loop
    \ifnum#3=\catcode\count@
    \else
      \errmessage{%
        Character \the\count@\space
        with wrong catcode \the\catcode\count@\space
        instead of \number#3%
      }%
    \fi
  \ifnum\count@<#2\relax
    \advance\count@ 1 %
  \repeat
}
\def\space{ }
\expandafter\ifx\csname LoadCommand\endcsname\relax
  \def\LoadCommand{\input hologo.sty\relax}%
\fi
\def\Test{%
  \RangeCatcodeInvalid{0}{47}%
  \RangeCatcodeInvalid{58}{64}%
  \RangeCatcodeInvalid{91}{96}%
  \RangeCatcodeInvalid{123}{255}%
  \catcode`\@=12 %
  \catcode`\\=0 %
  \catcode`\%=14 %
  \LoadCommand
  \RangeCatcodeCheck{0}{36}{15}%
  \RangeCatcodeCheck{37}{37}{14}%
  \RangeCatcodeCheck{38}{47}{15}%
  \RangeCatcodeCheck{48}{57}{12}%
  \RangeCatcodeCheck{58}{63}{15}%
  \RangeCatcodeCheck{64}{64}{12}%
  \RangeCatcodeCheck{65}{90}{11}%
  \RangeCatcodeCheck{91}{91}{15}%
  \RangeCatcodeCheck{92}{92}{0}%
  \RangeCatcodeCheck{93}{96}{15}%
  \RangeCatcodeCheck{97}{122}{11}%
  \RangeCatcodeCheck{123}{255}{15}%
  \RestoreCatcodes
}
\Test
\csname @@end\endcsname
\end
%    \end{macrocode}
%    \begin{macrocode}
%</test1>
%    \end{macrocode}
%
% \subsection{Spacefactor}
%
%    The space factor must be 1000 after a logo. If it is greater 1000
%    then the following space is a space after a sentence closing point.
%    If the space factor is smaller 1000 then an immediate following
%    dot is interpreted as abbreviation, not sentence closing point.
%
%    \begin{macrocode}
%<*test-spacefactor>
\NeedsTeXFormat{LaTeX2e}
\documentclass{article}
\usepackage{hologo}[2016/05/12]
\usepackage{kvsetkeys}
\usepackage{qstest}
\IncludeTests{*}
\LogTests{log}{*}{*}
\begin{document}
\begin{qstest}{spacefactor}{spacefactor}
\newcommand*{\Test}[1]{%
  \sbox0{%
    \hologo{#1}%
    \Expect*{1000 (#1)}*{\the\spacefactor\space(#1)}%
  }%
}%
\makeatletter
\def\TestList{}
\def\hologoEntry#1#2#3{%
  \edef\TestList{%
    \ifx\TestList\@empty
    \else
      \TestList,%
    \fi
    #1%
    \ifx\\#2\\%
    \else
      ={variant=#2}%
    \fi
  }%
}
\hologoList
\expandafter\kv@parse@normalized\expandafter{%
  \TestList
}{%
  \begingroup
    \let\@logo=\kv@key
    \ifx\kv@value\relax
    \else
      \expandafter\hologoLogoSetup\expandafter\@logo\expandafter{%
        \kv@value
      }%
    \fi
    \Test\@logo
  \endgroup
  \@gobbletwo
}
\end{qstest}
\end{document}
%</test-spacefactor>
%    \end{macrocode}
%
% \subsection{Complete list}
%
%    \begin{macrocode}
%<*test-list>
\NeedsTeXFormat{LaTeX2e}
\documentclass[12pt,a4paper]{article}
\usepackage{hologo}[2016/05/12]
\usepackage[T1]{fontenc}
\usepackage{lmodern}
\usepackage{parskip}
\usepackage[unicode]{hyperref}[2011/09/28]
\usepackage{bookmark}[2011/09/19]
\bookmarksetup{%
  numbered,%
  open,%
  openlevel=2,%
}
\renewcommand*{\contentsname}{List of logos}
\begin{document}
\tableofcontents
\def\TestFont#1#2#3#4#5#6{%
  \begingroup
    \usefont{#3}{#4}{#5}{#6}%
    \HologoVariant{#1}{#2}/\hologoVariant{#1}{#2}%
    \quad
    \begingroup\scriptsize\hologoVariant{#1}{#2}\endgroup
    \quad
  \endgroup
  (#3/#4/#5/#6)%
  \par
}
\makeatletter
\def\hologoEntry#1#2#3{%
  \section{%
    \HologoVariant{#1}{#2}/\hologoVariant{#1}{#2} %
    {[#1\ifx\\#2\\\else\space(#2)\fi]}% hash-ok
  }% braces around [] because of bug in tex4ht
  \begingroup
    \hypersetup{unicode=false}%
    \bookmark[%
      dest=\@currentHref,%
      rellevel=1,%
      keeplevel,%
    ]{%
      \HologoVariant{#1}{#2}/\hologoVariant{#1}{#2} %
      (PDFDocEncoding)%
    }%
  \endgroup
  \TestFont{#1}{#2}{OT1}{cmr}{m}{n}%
  \TestFont{#1}{#2}{OT1}{cmss}{m}{n}%
  \TestFont{#1}{#2}{OT1}{cmr}{b}{n}%
  \TestFont{#1}{#2}{OT1}{cmr}{m}{it}%
  \TestFont{#1}{#2}{OT1}{cmtt}{m}{n}%
  \TestFont{#1}{#2}{T1}{lmr}{m}{n}%
  \TestFont{#1}{#2}{T1}{lmss}{m}{n}%
  \TestFont{#1}{#2}{T1}{lmr}{b}{n}%
  \TestFont{#1}{#2}{T1}{lmr}{m}{it}%
  \TestFont{#1}{#2}{T1}{lmtt}{m}{n}%
  \TestFont{#1}{#2}{T1}{lmvtt}{m}{n}%
  \TestFont{#1}{#2}{T1}{qtm}{m}{n}%
  \TestFont{#1}{#2}{T1}{qhv}{m}{n}%
  \TestFont{#1}{#2}{T1}{qtm}{b}{n}%
  \TestFont{#1}{#2}{T1}{qtm}{m}{it}%
  \TestFont{#1}{#2}{T1}{qcr}{m}{n}%
  \newpage
}
\makeatother
\hologoList
\end{document}
%</test-list>
%    \end{macrocode}
%
% \section{Installation}
%
% \subsection{Download}
%
% \paragraph{Package.} This package is available on
% CTAN\footnote{\url{ftp://ftp.ctan.org/tex-archive/}}:
% \begin{description}
% \item[\CTAN{macros/latex/contrib/oberdiek/hologo.dtx}] The source file.
% \item[\CTAN{macros/latex/contrib/oberdiek/hologo.pdf}] Documentation.
% \end{description}
%
%
% \paragraph{Bundle.} All the packages of the bundle `oberdiek'
% are also available in a TDS compliant ZIP archive. There
% the packages are already unpacked and the documentation files
% are generated. The files and directories obey the TDS standard.
% \begin{description}
% \item[\CTAN{install/macros/latex/contrib/oberdiek.tds.zip}]
% \end{description}
% \emph{TDS} refers to the standard ``A Directory Structure
% for \TeX\ Files'' (\CTAN{tds/tds.pdf}). Directories
% with \xfile{texmf} in their name are usually organized this way.
%
% \subsection{Bundle installation}
%
% \paragraph{Unpacking.} Unpack the \xfile{oberdiek.tds.zip} in the
% TDS tree (also known as \xfile{texmf} tree) of your choice.
% Example (linux):
% \begin{quote}
%   |unzip oberdiek.tds.zip -d ~/texmf|
% \end{quote}
%
% \paragraph{Script installation.}
% Check the directory \xfile{TDS:scripts/oberdiek/} for
% scripts that need further installation steps.
% Package \xpackage{attachfile2} comes with the Perl script
% \xfile{pdfatfi.pl} that should be installed in such a way
% that it can be called as \texttt{pdfatfi}.
% Example (linux):
% \begin{quote}
%   |chmod +x scripts/oberdiek/pdfatfi.pl|\\
%   |cp scripts/oberdiek/pdfatfi.pl /usr/local/bin/|
% \end{quote}
%
% \subsection{Package installation}
%
% \paragraph{Unpacking.} The \xfile{.dtx} file is a self-extracting
% \docstrip\ archive. The files are extracted by running the
% \xfile{.dtx} through \plainTeX:
% \begin{quote}
%   \verb|tex hologo.dtx|
% \end{quote}
%
% \paragraph{TDS.} Now the different files must be moved into
% the different directories in your installation TDS tree
% (also known as \xfile{texmf} tree):
% \begin{quote}
% \def\t{^^A
% \begin{tabular}{@{}>{\ttfamily}l@{ $\rightarrow$ }>{\ttfamily}l@{}}
%   hologo.sty & tex/generic/oberdiek/hologo.sty\\
%   hologo.pdf & doc/latex/oberdiek/hologo.pdf\\
%   example/hologo-example.tex & doc/latex/oberdiek/example/hologo-example.tex\\
%   test/hologo-test1.tex & doc/latex/oberdiek/test/hologo-test1.tex\\
%   test/hologo-test-spacefactor.tex & doc/latex/oberdiek/test/hologo-test-spacefactor.tex\\
%   test/hologo-test-list.tex & doc/latex/oberdiek/test/hologo-test-list.tex\\
%   hologo.dtx & source/latex/oberdiek/hologo.dtx\\
% \end{tabular}^^A
% }^^A
% \sbox0{\t}^^A
% \ifdim\wd0>\linewidth
%   \begingroup
%     \advance\linewidth by\leftmargin
%     \advance\linewidth by\rightmargin
%   \edef\x{\endgroup
%     \def\noexpand\lw{\the\linewidth}^^A
%   }\x
%   \def\lwbox{^^A
%     \leavevmode
%     \hbox to \linewidth{^^A
%       \kern-\leftmargin\relax
%       \hss
%       \usebox0
%       \hss
%       \kern-\rightmargin\relax
%     }^^A
%   }^^A
%   \ifdim\wd0>\lw
%     \sbox0{\small\t}^^A
%     \ifdim\wd0>\linewidth
%       \ifdim\wd0>\lw
%         \sbox0{\footnotesize\t}^^A
%         \ifdim\wd0>\linewidth
%           \ifdim\wd0>\lw
%             \sbox0{\scriptsize\t}^^A
%             \ifdim\wd0>\linewidth
%               \ifdim\wd0>\lw
%                 \sbox0{\tiny\t}^^A
%                 \ifdim\wd0>\linewidth
%                   \lwbox
%                 \else
%                   \usebox0
%                 \fi
%               \else
%                 \lwbox
%               \fi
%             \else
%               \usebox0
%             \fi
%           \else
%             \lwbox
%           \fi
%         \else
%           \usebox0
%         \fi
%       \else
%         \lwbox
%       \fi
%     \else
%       \usebox0
%     \fi
%   \else
%     \lwbox
%   \fi
% \else
%   \usebox0
% \fi
% \end{quote}
% If you have a \xfile{docstrip.cfg} that configures and enables \docstrip's
% TDS installing feature, then some files can already be in the right
% place, see the documentation of \docstrip.
%
% \subsection{Refresh file name databases}
%
% If your \TeX~distribution
% (\teTeX, \mikTeX, \dots) relies on file name databases, you must refresh
% these. For example, \teTeX\ users run \verb|texhash| or
% \verb|mktexlsr|.
%
% \subsection{Some details for the interested}
%
% \paragraph{Attached source.}
%
% The PDF documentation on CTAN also includes the
% \xfile{.dtx} source file. It can be extracted by
% AcrobatReader 6 or higher. Another option is \textsf{pdftk},
% e.g. unpack the file into the current directory:
% \begin{quote}
%   \verb|pdftk hologo.pdf unpack_files output .|
% \end{quote}
%
% \paragraph{Unpacking with \LaTeX.}
% The \xfile{.dtx} chooses its action depending on the format:
% \begin{description}
% \item[\plainTeX:] Run \docstrip\ and extract the files.
% \item[\LaTeX:] Generate the documentation.
% \end{description}
% If you insist on using \LaTeX\ for \docstrip\ (really,
% \docstrip\ does not need \LaTeX), then inform the autodetect routine
% about your intention:
% \begin{quote}
%   \verb|latex \let\install=y\input{hologo.dtx}|
% \end{quote}
% Do not forget to quote the argument according to the demands
% of your shell.
%
% \paragraph{Generating the documentation.}
% You can use both the \xfile{.dtx} or the \xfile{.drv} to generate
% the documentation. The process can be configured by the
% configuration file \xfile{ltxdoc.cfg}. For instance, put this
% line into this file, if you want to have A4 as paper format:
% \begin{quote}
%   \verb|\PassOptionsToClass{a4paper}{article}|
% \end{quote}
% An example follows how to generate the
% documentation with pdf\LaTeX:
% \begin{quote}
%\begin{verbatim}
%pdflatex hologo.dtx
%makeindex -s gind.ist hologo.idx
%pdflatex hologo.dtx
%makeindex -s gind.ist hologo.idx
%pdflatex hologo.dtx
%\end{verbatim}
% \end{quote}
%
% \section{Catalogue}
%
% The following XML file can be used as source for the
% \href{http://mirror.ctan.org/help/Catalogue/catalogue.html}{\TeX\ Catalogue}.
% The elements \texttt{caption} and \texttt{description} are imported
% from the original XML file from the Catalogue.
% The name of the XML file in the Catalogue is \xfile{hologo.xml}.
%    \begin{macrocode}
%<*catalogue>
<?xml version='1.0' encoding='us-ascii'?>
<!DOCTYPE entry SYSTEM 'catalogue.dtd'>
<entry datestamp='$Date$' modifier='$Author$' id='hologo'>
  <name>hologo</name>
  <caption>A collection of logos with bookmark support.</caption>
  <authorref id='auth:oberdiek'/>
  <copyright owner='Heiko Oberdiek' year='2010-2012'/>
  <license type='lppl1.3'/>
  <version number='1.10'/>
  <description>
    The package defines a single command <tt>\hologo</tt>, whose
    argument is the usual case-confused ASCII version of the logo.
    The command is bookmark-enabled, so that every logo becomes
    available in bookmarks without further work.
    <p/>
    The package is part of the <xref refid='oberdiek'>oberdiek</xref>
    bundle.
  </description>
  <documentation details='Package documentation'
      href='ctan:/macros/latex/contrib/oberdiek/hologo.pdf'/>
  <ctan file='true' path='/macros/latex/contrib/oberdiek/hologo.dtx'/>
  <miktex location='oberdiek'/>
  <texlive location='oberdiek'/>
  <install path='/macros/latex/contrib/oberdiek/oberdiek.tds.zip'/>
</entry>
%</catalogue>
%    \end{macrocode}
%
% \begin{thebibliography}{9}
% \raggedright
%
% \bibitem{btxdoc}
% Oren Patashnik,
% \textit{\hologo{BibTeX}ing},
% 1988-02-08.\\
% \CTAN{biblio/bibtex/base/}
%
% \bibitem{dtklogos}
% Gerd Neugebauer, DANTE,
% \textit{Package \xpackage{dtklogos}},
% 2011-04-25.\\
% \CTAN{usergrps/dante/dtk/dtklogos.sty}
%
% \bibitem{etexman}
% The \hologo{NTS} Team,
% \textit{The \hologo{eTeX} manual},
% 1998-02.\\
% \CTAN{systems/e-tex/v2/doc/}
%
% \bibitem{ExTeX-FAQ}
% The \hologo{ExTeX} group,
% \textit{\hologo{ExTeX}: FAQ -- How is \hologo{ExTeX} typeset?},
% 2007-04-14.\\
% \url{http://www.extex.org/documentation/faq.html}
%
% \bibitem{LyX}
% %@MISC{ LyX,
% %  title = {{LyX 2.0.0 -- The Document Processor [Computer software and manual]}},
% %  author = {{The LyX Team}},
% %  howpublished = {Internet: http://www.lyx.org},
% %  year = {2011-05-08},
% %  note = {Retrieved May 10, 2011, from http://www.lyx.org},
% %  url = {http://www.lyx.org/}
% %}
% The \hologo{LyX} Team,
% \textit{\hologo{LyX} -- The Document Processor},
% 2011-05-08.\\
% \url{http://www.lyx.org/}
%
% \bibitem{OzTeX}
% Andrew Trevorrow,
% \hologo{OzTeX} FAQ: What is the correct way to typeset ``\hologo{OzTeX}''?,
% 2011-09-15 (visited).
% \url{http://www.trevorrow.com/oztex/ozfaq.html#oztex-logo}
%
% \bibitem{PiCTeX}
% Michael Wichura,
% \textit{The \hologo{PiCTeX} macro package},
% 1987-09-21.
% \CTAN{graphics/pictex/}
%
% \bibitem{scrlogo}
% Markus Kohm,
% \textit{\hologo{KOMAScript} Datei \xfile{scrlogo.dtx}},
% 2009-01-30.\\
% \CTAN{install/macros/latex/contrib/komascript.tds.zip}
%
% \end{thebibliography}
%
% \begin{History}
%   \begin{Version}{2010/04/08 v1.0}
%   \item
%     The first version.
%   \end{Version}
%   \begin{Version}{2010/04/16 v1.1}
%   \item
%     \cs{Hologo} added for support of logos at start of a sentence.
%   \item
%     \cs{hologoSetup} and \cs{hologoLogoSetup} added.
%   \item
%     Options \xoption{break}, \xoption{hyphenbreak}, \xoption{spacebreak}
%     added.
%   \item
%     Variant support added by option \xoption{variant}.
%   \end{Version}
%   \begin{Version}{2010/04/24 v1.2}
%   \item
%     \hologo{LaTeX3} added.
%   \item
%     \hologo{VTeX} added.
%   \end{Version}
%   \begin{Version}{2010/11/21 v1.3}
%   \item
%     \hologo{iniTeX}, \hologo{virTeX} added.
%   \end{Version}
%   \begin{Version}{2011/03/25 v1.4}
%   \item
%     \hologo{ConTeXt} with variants added.
%   \item
%     Option \xoption{discretionarybreak} added as refinement for
%     option \xoption{break}.
%   \end{Version}
%   \begin{Version}{2011/04/21 v1.5}
%   \item
%     Wrong TDS directory for test files fixed.
%   \end{Version}
%   \begin{Version}{2011/10/01 v1.6}
%   \item
%     Support for package \xpackage{tex4ht} added.
%   \item
%     Support for \cs{csname} added if \cs{ifincsname} is available.
%   \item
%     New logos:
%     \hologo{(La)TeX},
%     \hologo{biber},
%     \hologo{BibTeX} (\xoption{sc}, \xoption{sf}),
%     \hologo{emTeX},
%     \hologo{ExTeX},
%     \hologo{KOMAScript},
%     \hologo{La},
%     \hologo{LyX},
%     \hologo{MiKTeX},
%     \hologo{NTS},
%     \hologo{OzMF},
%     \hologo{OzMP},
%     \hologo{OzTeX},
%     \hologo{OzTtH},
%     \hologo{PCTeX},
%     \hologo{PiC},
%     \hologo{PiCTeX},
%     \hologo{METAFONT},
%     \hologo{MetaFun},
%     \hologo{METAPOST},
%     \hologo{MetaPost},
%     \hologo{SLiTeX} (\xoption{lift}, \xoption{narrow}, \xoption{simple}),
%     \hologo{SliTeX} (\xoption{narrow}, \xoption{simple}, \xoption{lift}),
%     \hologo{teTeX}.
%   \item
%     Fixes:
%     \hologo{iniTeX},
%     \hologo{pdfLaTeX},
%     \hologo{pdfTeX},
%     \hologo{virTeX}.
%   \item
%     \cs{hologoFontSetup} and \cs{hologoLogoFontSetup} added.
%   \item
%     \cs{hologoVariant} and \cs{HologoVariant} added.
%   \end{Version}
%   \begin{Version}{2011/11/22 v1.7}
%   \item
%     New logos:
%     \hologo{BibTeX8},
%     \hologo{LaTeXML},
%     \hologo{SageTeX},
%     \hologo{TeX4ht},
%     \hologo{TTH}.
%   \item
%     \hologo{Xe} and friends: Driver stuff fixed.
%   \item
%     \hologo{Xe} and friends: Support for italic added.
%   \item
%     \hologo{Xe} and friends: Package support for \xpackage{pgf}
%     and \xpackage{pstricks} added.
%   \end{Version}
%   \begin{Version}{2011/11/29 v1.8}
%   \item
%     New logos:
%     \hologo{HanTheThanh}.
%   \end{Version}
%   \begin{Version}{2011/12/21 v1.9}
%   \item
%     Patch for package \xpackage{ifxetex} added for the case that
%     \cs{newif} is undefined in \hologo{iniTeX}.
%   \item
%     Some fixes for \hologo{iniTeX}.
%   \end{Version}
%   \begin{Version}{2012/04/26 v1.10}
%   \item
%     Fix in bookmark version of logo ``\hologo{HanTheThanh}''.
%   \end{Version}
%   \begin{Version}{2016/05/12 v1.11}
%   \item
%     Update HOLOGO@IfCharExists (previously in texlive)
%   \item define pdfliteral in current luatex.
%   \end{Version}
% \end{History}
%
% \PrintIndex
%
% \Finale
\endinput
%
        \else
          \input hologo.cfg\relax
        \fi
      \else
        \@PackageInfoNoLine{hologo}{%
          Empty configuration file `hologo.cfg' ignored%
        }%
      \fi
    \fi
  }%
}
%    \end{macrocode}
%
%    \begin{macrocode}
\def\HOLOGO@temp#1#2{%
  \kv@define@key{HoLogoDriver}{#1}[]{%
    \begingroup
      \def\HOLOGO@temp{##1}%
      \ltx@onelevel@sanitize\HOLOGO@temp
      \ifx\HOLOGO@temp\ltx@empty
      \else
        \@PackageError{hologo}{%
          Value (\HOLOGO@temp) not permitted for option `#1'%
        }%
        \@ehc
      \fi
    \endgroup
    \def\hologoDriver{#2}%
  }%
}%
\def\HOLOGO@@temp#1#2{%
  \ifx\kv@value\relax
    \HOLOGO@temp{#1}{#1}%
  \else
    \HOLOGO@temp{#1}{#2}%
  \fi
}%
\kv@parse@normalized{%
  pdftex,%
  luatex=pdftex,%
  dvipdfm,%
  dvipdfmx=dvipdfm,%
  dvips,%
  dvipsone=dvips,%
  xdvi=dvips,%
  xetex,%
  vtex,%
}\HOLOGO@@temp
%    \end{macrocode}
%
%    \begin{macrocode}
\kv@define@key{HoLogoDriver}{driverfallback}{%
  \def\HOLOGO@DriverFallback{#1}%
}
%    \end{macrocode}
%
%    \begin{macro}{\HOLOGO@DriverFallback}
%    \begin{macrocode}
\def\HOLOGO@DriverFallback{dvips}
%    \end{macrocode}
%    \end{macro}
%
%    \begin{macro}{\hologoDriverSetup}
%    \begin{macrocode}
\def\hologoDriverSetup{%
  \let\hologoDriver\ltx@undefined
  \HOLOGO@DriverSetup
}
%    \end{macrocode}
%    \end{macro}
%
%    \begin{macro}{\HOLOGO@DriverSetup}
%    \begin{macrocode}
\def\HOLOGO@DriverSetup#1{%
  \kvsetkeys{HoLogoDriver}{#1}%
  \HOLOGO@CheckDriver
  \ltx@ifundefined{hologoDriver}{%
    \begingroup
    \edef\x{\endgroup
      \noexpand\kvsetkeys{HoLogoDriver}{\HOLOGO@DriverFallback}%
    }\x
  }{}%
  \@PackageInfoNoLine{hologo}{Using driver `\hologoDriver'}%
}
%    \end{macrocode}
%    \end{macro}
%
%    \begin{macro}{\HOLOGO@CheckDriver}
%    \begin{macrocode}
\def\HOLOGO@CheckDriver{%
  \ifpdf
    \def\hologoDriver{pdftex}%
    \let\HOLOGO@pdfliteral\pdfliteral
    \ifluatex
      \ifx\pdfextension\@undefined\else
        \protected\def\pdfliteral{\pdfextension literal}%
        \let\HOLOGO@pdfliteral\pdfliteral
      \fi
      \ltx@IfUndefined{HOLOGO@pdfliteral}{%
        \ifnum\luatexversion<36 %
        \else
          \begingroup
            \let\HOLOGO@temp\endgroup
            \ifcase0%
                \directlua{%
                  if tex.enableprimitives then %
                    tex.enableprimitives('HOLOGO@', {'pdfliteral'})%
                  else %
                    tex.print('1')%
                  end%
                }%
                \ifx\HOLOGO@pdfliteral\@undefined 1\fi%
                \relax%
              \endgroup
              \let\HOLOGO@temp\relax
              \global\let\HOLOGO@pdfliteral\HOLOGO@pdfliteral
            \fi%
          \HOLOGO@temp
        \fi
      }{}%
    \fi
    \ltx@IfUndefined{HOLOGO@pdfliteral}{%
      \@PackageWarningNoLine{hologo}{%
        Cannot find \string\pdfliteral
      }%
    }{}%
  \else
    \ifxetex
      \def\hologoDriver{xetex}%
    \else
      \ifvtex
        \def\hologoDriver{vtex}%
      \fi
    \fi
  \fi
}
%    \end{macrocode}
%    \end{macro}
%
%    \begin{macro}{\HOLOGO@WarningUnsupportedDriver}
%    \begin{macrocode}
\def\HOLOGO@WarningUnsupportedDriver#1{%
  \@PackageWarningNoLine{hologo}{%
    Logo `#1' needs driver specific macros,\MessageBreak
    but driver `\hologoDriver' is not supported.\MessageBreak
    Use a different driver or\MessageBreak
    load package `graphics' or `pgf'%
  }%
}
%    \end{macrocode}
%    \end{macro}
%
% \subsubsection{Reflect box macros}
%
%    Skip driver part if not needed.
%    \begin{macrocode}
\ltx@IfUndefined{reflectbox}{}{%
  \ltx@IfUndefined{rotatebox}{}{%
    \HOLOGO@AtEnd
  }%
}
\ltx@IfUndefined{pgftext}{}{%
  \HOLOGO@AtEnd
}
\ltx@IfUndefined{psscalebox}{}{%
  \HOLOGO@AtEnd
}
%    \end{macrocode}
%
%    \begin{macrocode}
\def\HOLOGO@temp{LaTeX2e}
\ifx\fmtname\HOLOGO@temp
  \RequirePackage{kvoptions}[2011/06/30]%
  \ProcessKeyvalOptions{HoLogoDriver}%
\fi
\HOLOGO@DriverSetup{}
%    \end{macrocode}
%
%    \begin{macro}{\HOLOGO@ReflectBox}
%    \begin{macrocode}
\def\HOLOGO@ReflectBox#1{%
  \begingroup
    \setbox\ltx@zero\hbox{\begingroup#1\endgroup}%
    \setbox\ltx@two\hbox{%
      \kern\wd\ltx@zero
      \csname HOLOGO@ScaleBox@\hologoDriver\endcsname{-1}{1}{%
        \hbox to 0pt{\copy\ltx@zero\hss}%
      }%
    }%
    \wd\ltx@two=\wd\ltx@zero
    \box\ltx@two
  \endgroup
}
%    \end{macrocode}
%    \end{macro}
%
%    \begin{macro}{\HOLOGO@PointReflectBox}
%    \begin{macrocode}
\def\HOLOGO@PointReflectBox#1{%
  \begingroup
    \setbox\ltx@zero\hbox{\begingroup#1\endgroup}%
    \setbox\ltx@two\hbox{%
      \kern\wd\ltx@zero
      \raise\ht\ltx@zero\hbox{%
        \csname HOLOGO@ScaleBox@\hologoDriver\endcsname{-1}{-1}{%
          \hbox to 0pt{\copy\ltx@zero\hss}%
        }%
      }%
    }%
    \wd\ltx@two=\wd\ltx@zero
    \box\ltx@two
  \endgroup
}
%    \end{macrocode}
%    \end{macro}
%
%    We must define all variants because of dynamic driver setup.
%    \begin{macrocode}
\def\HOLOGO@temp#1#2{#2}
%    \end{macrocode}
%
%    \begin{macro}{\HOLOGO@ScaleBox@pdftex}
%    \begin{macrocode}
\HOLOGO@temp{pdftex}{%
  \def\HOLOGO@ScaleBox@pdftex#1#2#3{%
    \HOLOGO@pdfliteral{%
      q #1 0 0 #2 0 0 cm%
    }%
    #3%
    \HOLOGO@pdfliteral{%
      Q%
    }%
  }%
}
%    \end{macrocode}
%    \end{macro}
%    \begin{macro}{\HOLOGO@ScaleBox@dvips}
%    \begin{macrocode}
\HOLOGO@temp{dvips}{%
  \def\HOLOGO@ScaleBox@dvips#1#2#3{%
    \special{ps:%
      gsave %
      currentpoint %
      currentpoint translate %
      #1 #2 scale %
      neg exch neg exch translate%
    }%
    #3%
    \special{ps:%
      currentpoint %
      grestore %
      moveto%
    }%
  }%
}
%    \end{macrocode}
%    \end{macro}
%    \begin{macro}{\HOLOGO@ScaleBox@dvipdfm}
%    \begin{macrocode}
\HOLOGO@temp{dvipdfm}{%
  \let\HOLOGO@ScaleBox@dvipdfm\HOLOGO@ScaleBox@dvips
}
%    \end{macrocode}
%    \end{macro}
%    Since \hologo{XeTeX} v0.6.
%    \begin{macro}{\HOLOGO@ScaleBox@xetex}
%    \begin{macrocode}
\HOLOGO@temp{xetex}{%
  \def\HOLOGO@ScaleBox@xetex#1#2#3{%
    \special{x:gsave}%
    \special{x:scale #1 #2}%
    #3%
    \special{x:grestore}%
  }%
}
%    \end{macrocode}
%    \end{macro}
%    \begin{macro}{\HOLOGO@ScaleBox@vtex}
%    \begin{macrocode}
\HOLOGO@temp{vtex}{%
  \def\HOLOGO@ScaleBox@vtex#1#2#3{%
    \special{r(#1,0,0,#2,0,0}%
    #3%
    \special{r)}%
  }%
}
%    \end{macrocode}
%    \end{macro}
%
%    \begin{macrocode}
\HOLOGO@AtEnd%
%</package>
%    \end{macrocode}
%
% \section{Test}
%
% \subsection{Catcode checks for loading}
%
%    \begin{macrocode}
%<*test1>
%    \end{macrocode}
%    \begin{macrocode}
\catcode`\{=1 %
\catcode`\}=2 %
\catcode`\#=6 %
\catcode`\@=11 %
\expandafter\ifx\csname count@\endcsname\relax
  \countdef\count@=255 %
\fi
\expandafter\ifx\csname @gobble\endcsname\relax
  \long\def\@gobble#1{}%
\fi
\expandafter\ifx\csname @firstofone\endcsname\relax
  \long\def\@firstofone#1{#1}%
\fi
\expandafter\ifx\csname loop\endcsname\relax
  \expandafter\@firstofone
\else
  \expandafter\@gobble
\fi
{%
  \def\loop#1\repeat{%
    \def\body{#1}%
    \iterate
  }%
  \def\iterate{%
    \body
      \let\next\iterate
    \else
      \let\next\relax
    \fi
    \next
  }%
  \let\repeat=\fi
}%
\def\RestoreCatcodes{}
\count@=0 %
\loop
  \edef\RestoreCatcodes{%
    \RestoreCatcodes
    \catcode\the\count@=\the\catcode\count@\relax
  }%
\ifnum\count@<255 %
  \advance\count@ 1 %
\repeat

\def\RangeCatcodeInvalid#1#2{%
  \count@=#1\relax
  \loop
    \catcode\count@=15 %
  \ifnum\count@<#2\relax
    \advance\count@ 1 %
  \repeat
}
\def\RangeCatcodeCheck#1#2#3{%
  \count@=#1\relax
  \loop
    \ifnum#3=\catcode\count@
    \else
      \errmessage{%
        Character \the\count@\space
        with wrong catcode \the\catcode\count@\space
        instead of \number#3%
      }%
    \fi
  \ifnum\count@<#2\relax
    \advance\count@ 1 %
  \repeat
}
\def\space{ }
\expandafter\ifx\csname LoadCommand\endcsname\relax
  \def\LoadCommand{\input hologo.sty\relax}%
\fi
\def\Test{%
  \RangeCatcodeInvalid{0}{47}%
  \RangeCatcodeInvalid{58}{64}%
  \RangeCatcodeInvalid{91}{96}%
  \RangeCatcodeInvalid{123}{255}%
  \catcode`\@=12 %
  \catcode`\\=0 %
  \catcode`\%=14 %
  \LoadCommand
  \RangeCatcodeCheck{0}{36}{15}%
  \RangeCatcodeCheck{37}{37}{14}%
  \RangeCatcodeCheck{38}{47}{15}%
  \RangeCatcodeCheck{48}{57}{12}%
  \RangeCatcodeCheck{58}{63}{15}%
  \RangeCatcodeCheck{64}{64}{12}%
  \RangeCatcodeCheck{65}{90}{11}%
  \RangeCatcodeCheck{91}{91}{15}%
  \RangeCatcodeCheck{92}{92}{0}%
  \RangeCatcodeCheck{93}{96}{15}%
  \RangeCatcodeCheck{97}{122}{11}%
  \RangeCatcodeCheck{123}{255}{15}%
  \RestoreCatcodes
}
\Test
\csname @@end\endcsname
\end
%    \end{macrocode}
%    \begin{macrocode}
%</test1>
%    \end{macrocode}
%
% \subsection{Spacefactor}
%
%    The space factor must be 1000 after a logo. If it is greater 1000
%    then the following space is a space after a sentence closing point.
%    If the space factor is smaller 1000 then an immediate following
%    dot is interpreted as abbreviation, not sentence closing point.
%
%    \begin{macrocode}
%<*test-spacefactor>
\NeedsTeXFormat{LaTeX2e}
\documentclass{article}
\usepackage{hologo}[2016/05/12]
\usepackage{kvsetkeys}
\usepackage{qstest}
\IncludeTests{*}
\LogTests{log}{*}{*}
\begin{document}
\begin{qstest}{spacefactor}{spacefactor}
\newcommand*{\Test}[1]{%
  \sbox0{%
    \hologo{#1}%
    \Expect*{1000 (#1)}*{\the\spacefactor\space(#1)}%
  }%
}%
\makeatletter
\def\TestList{}
\def\hologoEntry#1#2#3{%
  \edef\TestList{%
    \ifx\TestList\@empty
    \else
      \TestList,%
    \fi
    #1%
    \ifx\\#2\\%
    \else
      ={variant=#2}%
    \fi
  }%
}
\hologoList
\expandafter\kv@parse@normalized\expandafter{%
  \TestList
}{%
  \begingroup
    \let\@logo=\kv@key
    \ifx\kv@value\relax
    \else
      \expandafter\hologoLogoSetup\expandafter\@logo\expandafter{%
        \kv@value
      }%
    \fi
    \Test\@logo
  \endgroup
  \@gobbletwo
}
\end{qstest}
\end{document}
%</test-spacefactor>
%    \end{macrocode}
%
% \subsection{Complete list}
%
%    \begin{macrocode}
%<*test-list>
\NeedsTeXFormat{LaTeX2e}
\documentclass[12pt,a4paper]{article}
\usepackage{hologo}[2016/05/12]
\usepackage[T1]{fontenc}
\usepackage{lmodern}
\usepackage{parskip}
\usepackage[unicode]{hyperref}[2011/09/28]
\usepackage{bookmark}[2011/09/19]
\bookmarksetup{%
  numbered,%
  open,%
  openlevel=2,%
}
\renewcommand*{\contentsname}{List of logos}
\begin{document}
\tableofcontents
\def\TestFont#1#2#3#4#5#6{%
  \begingroup
    \usefont{#3}{#4}{#5}{#6}%
    \HologoVariant{#1}{#2}/\hologoVariant{#1}{#2}%
    \quad
    \begingroup\scriptsize\hologoVariant{#1}{#2}\endgroup
    \quad
  \endgroup
  (#3/#4/#5/#6)%
  \par
}
\makeatletter
\def\hologoEntry#1#2#3{%
  \section{%
    \HologoVariant{#1}{#2}/\hologoVariant{#1}{#2} %
    {[#1\ifx\\#2\\\else\space(#2)\fi]}% hash-ok
  }% braces around [] because of bug in tex4ht
  \begingroup
    \hypersetup{unicode=false}%
    \bookmark[%
      dest=\@currentHref,%
      rellevel=1,%
      keeplevel,%
    ]{%
      \HologoVariant{#1}{#2}/\hologoVariant{#1}{#2} %
      (PDFDocEncoding)%
    }%
  \endgroup
  \TestFont{#1}{#2}{OT1}{cmr}{m}{n}%
  \TestFont{#1}{#2}{OT1}{cmss}{m}{n}%
  \TestFont{#1}{#2}{OT1}{cmr}{b}{n}%
  \TestFont{#1}{#2}{OT1}{cmr}{m}{it}%
  \TestFont{#1}{#2}{OT1}{cmtt}{m}{n}%
  \TestFont{#1}{#2}{T1}{lmr}{m}{n}%
  \TestFont{#1}{#2}{T1}{lmss}{m}{n}%
  \TestFont{#1}{#2}{T1}{lmr}{b}{n}%
  \TestFont{#1}{#2}{T1}{lmr}{m}{it}%
  \TestFont{#1}{#2}{T1}{lmtt}{m}{n}%
  \TestFont{#1}{#2}{T1}{lmvtt}{m}{n}%
  \TestFont{#1}{#2}{T1}{qtm}{m}{n}%
  \TestFont{#1}{#2}{T1}{qhv}{m}{n}%
  \TestFont{#1}{#2}{T1}{qtm}{b}{n}%
  \TestFont{#1}{#2}{T1}{qtm}{m}{it}%
  \TestFont{#1}{#2}{T1}{qcr}{m}{n}%
  \newpage
}
\makeatother
\hologoList
\end{document}
%</test-list>
%    \end{macrocode}
%
% \section{Installation}
%
% \subsection{Download}
%
% \paragraph{Package.} This package is available on
% CTAN\footnote{\url{ftp://ftp.ctan.org/tex-archive/}}:
% \begin{description}
% \item[\CTAN{macros/latex/contrib/oberdiek/hologo.dtx}] The source file.
% \item[\CTAN{macros/latex/contrib/oberdiek/hologo.pdf}] Documentation.
% \end{description}
%
%
% \paragraph{Bundle.} All the packages of the bundle `oberdiek'
% are also available in a TDS compliant ZIP archive. There
% the packages are already unpacked and the documentation files
% are generated. The files and directories obey the TDS standard.
% \begin{description}
% \item[\CTAN{install/macros/latex/contrib/oberdiek.tds.zip}]
% \end{description}
% \emph{TDS} refers to the standard ``A Directory Structure
% for \TeX\ Files'' (\CTAN{tds/tds.pdf}). Directories
% with \xfile{texmf} in their name are usually organized this way.
%
% \subsection{Bundle installation}
%
% \paragraph{Unpacking.} Unpack the \xfile{oberdiek.tds.zip} in the
% TDS tree (also known as \xfile{texmf} tree) of your choice.
% Example (linux):
% \begin{quote}
%   |unzip oberdiek.tds.zip -d ~/texmf|
% \end{quote}
%
% \paragraph{Script installation.}
% Check the directory \xfile{TDS:scripts/oberdiek/} for
% scripts that need further installation steps.
% Package \xpackage{attachfile2} comes with the Perl script
% \xfile{pdfatfi.pl} that should be installed in such a way
% that it can be called as \texttt{pdfatfi}.
% Example (linux):
% \begin{quote}
%   |chmod +x scripts/oberdiek/pdfatfi.pl|\\
%   |cp scripts/oberdiek/pdfatfi.pl /usr/local/bin/|
% \end{quote}
%
% \subsection{Package installation}
%
% \paragraph{Unpacking.} The \xfile{.dtx} file is a self-extracting
% \docstrip\ archive. The files are extracted by running the
% \xfile{.dtx} through \plainTeX:
% \begin{quote}
%   \verb|tex hologo.dtx|
% \end{quote}
%
% \paragraph{TDS.} Now the different files must be moved into
% the different directories in your installation TDS tree
% (also known as \xfile{texmf} tree):
% \begin{quote}
% \def\t{^^A
% \begin{tabular}{@{}>{\ttfamily}l@{ $\rightarrow$ }>{\ttfamily}l@{}}
%   hologo.sty & tex/generic/oberdiek/hologo.sty\\
%   hologo.pdf & doc/latex/oberdiek/hologo.pdf\\
%   example/hologo-example.tex & doc/latex/oberdiek/example/hologo-example.tex\\
%   test/hologo-test1.tex & doc/latex/oberdiek/test/hologo-test1.tex\\
%   test/hologo-test-spacefactor.tex & doc/latex/oberdiek/test/hologo-test-spacefactor.tex\\
%   test/hologo-test-list.tex & doc/latex/oberdiek/test/hologo-test-list.tex\\
%   hologo.dtx & source/latex/oberdiek/hologo.dtx\\
% \end{tabular}^^A
% }^^A
% \sbox0{\t}^^A
% \ifdim\wd0>\linewidth
%   \begingroup
%     \advance\linewidth by\leftmargin
%     \advance\linewidth by\rightmargin
%   \edef\x{\endgroup
%     \def\noexpand\lw{\the\linewidth}^^A
%   }\x
%   \def\lwbox{^^A
%     \leavevmode
%     \hbox to \linewidth{^^A
%       \kern-\leftmargin\relax
%       \hss
%       \usebox0
%       \hss
%       \kern-\rightmargin\relax
%     }^^A
%   }^^A
%   \ifdim\wd0>\lw
%     \sbox0{\small\t}^^A
%     \ifdim\wd0>\linewidth
%       \ifdim\wd0>\lw
%         \sbox0{\footnotesize\t}^^A
%         \ifdim\wd0>\linewidth
%           \ifdim\wd0>\lw
%             \sbox0{\scriptsize\t}^^A
%             \ifdim\wd0>\linewidth
%               \ifdim\wd0>\lw
%                 \sbox0{\tiny\t}^^A
%                 \ifdim\wd0>\linewidth
%                   \lwbox
%                 \else
%                   \usebox0
%                 \fi
%               \else
%                 \lwbox
%               \fi
%             \else
%               \usebox0
%             \fi
%           \else
%             \lwbox
%           \fi
%         \else
%           \usebox0
%         \fi
%       \else
%         \lwbox
%       \fi
%     \else
%       \usebox0
%     \fi
%   \else
%     \lwbox
%   \fi
% \else
%   \usebox0
% \fi
% \end{quote}
% If you have a \xfile{docstrip.cfg} that configures and enables \docstrip's
% TDS installing feature, then some files can already be in the right
% place, see the documentation of \docstrip.
%
% \subsection{Refresh file name databases}
%
% If your \TeX~distribution
% (\teTeX, \mikTeX, \dots) relies on file name databases, you must refresh
% these. For example, \teTeX\ users run \verb|texhash| or
% \verb|mktexlsr|.
%
% \subsection{Some details for the interested}
%
% \paragraph{Attached source.}
%
% The PDF documentation on CTAN also includes the
% \xfile{.dtx} source file. It can be extracted by
% AcrobatReader 6 or higher. Another option is \textsf{pdftk},
% e.g. unpack the file into the current directory:
% \begin{quote}
%   \verb|pdftk hologo.pdf unpack_files output .|
% \end{quote}
%
% \paragraph{Unpacking with \LaTeX.}
% The \xfile{.dtx} chooses its action depending on the format:
% \begin{description}
% \item[\plainTeX:] Run \docstrip\ and extract the files.
% \item[\LaTeX:] Generate the documentation.
% \end{description}
% If you insist on using \LaTeX\ for \docstrip\ (really,
% \docstrip\ does not need \LaTeX), then inform the autodetect routine
% about your intention:
% \begin{quote}
%   \verb|latex \let\install=y% \iffalse meta-comment
%
% File: hologo.dtx
% Version: 2016/05/12 v1.11
% Info: A logo collection with bookmark support
%
% Copyright (C) 2010-2012 by
%    Heiko Oberdiek <heiko.oberdiek at googlemail.com>
%
% This work may be distributed and/or modified under the
% conditions of the LaTeX Project Public License, either
% version 1.3c of this license or (at your option) any later
% version. This version of this license is in
%    http://www.latex-project.org/lppl/lppl-1-3c.txt
% and the latest version of this license is in
%    http://www.latex-project.org/lppl.txt
% and version 1.3 or later is part of all distributions of
% LaTeX version 2005/12/01 or later.
%
% This work has the LPPL maintenance status "maintained".
%
% This Current Maintainer of this work is Heiko Oberdiek.
%
% The Base Interpreter refers to any `TeX-Format',
% because some files are installed in TDS:tex/generic//.
%
% This work consists of the main source file hologo.dtx
% and the derived files
%    hologo.sty, hologo.pdf, hologo.ins, hologo.drv, hologo-example.tex,
%    hologo-test1.tex, hologo-test-spacefactor.tex,
%    hologo-test-list.tex.
%
% Distribution:
%    CTAN:macros/latex/contrib/oberdiek/hologo.dtx
%    CTAN:macros/latex/contrib/oberdiek/hologo.pdf
%
% Unpacking:
%    (a) If hologo.ins is present:
%           tex hologo.ins
%    (b) Without hologo.ins:
%           tex hologo.dtx
%    (c) If you insist on using LaTeX
%           latex \let\install=y\input{hologo.dtx}
%        (quote the arguments according to the demands of your shell)
%
% Documentation:
%    (a) If hologo.drv is present:
%           latex hologo.drv
%    (b) Without hologo.drv:
%           latex hologo.dtx; ...
%    The class ltxdoc loads the configuration file ltxdoc.cfg
%    if available. Here you can specify further options, e.g.
%    use A4 as paper format:
%       \PassOptionsToClass{a4paper}{article}
%
%    Programm calls to get the documentation (example):
%       pdflatex hologo.dtx
%       makeindex -s gind.ist hologo.idx
%       pdflatex hologo.dtx
%       makeindex -s gind.ist hologo.idx
%       pdflatex hologo.dtx
%
% Installation:
%    TDS:tex/generic/oberdiek/hologo.sty
%    TDS:doc/latex/oberdiek/hologo.pdf
%    TDS:doc/latex/oberdiek/example/hologo-example.tex
%    TDS:doc/latex/oberdiek/test/hologo-test1.tex
%    TDS:doc/latex/oberdiek/test/hologo-test-spacefactor.tex
%    TDS:doc/latex/oberdiek/test/hologo-test-list.tex
%    TDS:source/latex/oberdiek/hologo.dtx
%
%<*ignore>
\begingroup
  \catcode123=1 %
  \catcode125=2 %
  \def\x{LaTeX2e}%
\expandafter\endgroup
\ifcase 0\ifx\install y1\fi\expandafter
         \ifx\csname processbatchFile\endcsname\relax\else1\fi
         \ifx\fmtname\x\else 1\fi\relax
\else\csname fi\endcsname
%</ignore>
%<*install>
\input docstrip.tex
\Msg{************************************************************************}
\Msg{* Installation}
\Msg{* Package: hologo 2016/05/12 v1.11 A logo collection with bookmark support (HO)}
\Msg{************************************************************************}

\keepsilent
\askforoverwritefalse

\let\MetaPrefix\relax
\preamble

This is a generated file.

Project: hologo
Version: 2016/05/12 v1.11

Copyright (C) 2010-2012 by
   Heiko Oberdiek <heiko.oberdiek at googlemail.com>

This work may be distributed and/or modified under the
conditions of the LaTeX Project Public License, either
version 1.3c of this license or (at your option) any later
version. This version of this license is in
   http://www.latex-project.org/lppl/lppl-1-3c.txt
and the latest version of this license is in
   http://www.latex-project.org/lppl.txt
and version 1.3 or later is part of all distributions of
LaTeX version 2005/12/01 or later.

This work has the LPPL maintenance status "maintained".

This Current Maintainer of this work is Heiko Oberdiek.

The Base Interpreter refers to any `TeX-Format',
because some files are installed in TDS:tex/generic//.

This work consists of the main source file hologo.dtx
and the derived files
   hologo.sty, hologo.pdf, hologo.ins, hologo.drv, hologo-example.tex,
   hologo-test1.tex, hologo-test-spacefactor.tex,
   hologo-test-list.tex.

\endpreamble
\let\MetaPrefix\DoubleperCent

\generate{%
  \file{hologo.ins}{\from{hologo.dtx}{install}}%
  \file{hologo.drv}{\from{hologo.dtx}{driver}}%
  \usedir{tex/generic/oberdiek}%
  \file{hologo.sty}{\from{hologo.dtx}{package}}%
  \usedir{doc/latex/oberdiek/example}%
  \file{hologo-example.tex}{\from{hologo.dtx}{example}}%
  \usedir{doc/latex/oberdiek/test}%
  \file{hologo-test1.tex}{\from{hologo.dtx}{test1}}%
  \file{hologo-test-spacefactor.tex}{\from{hologo.dtx}{test-spacefactor}}%
  \file{hologo-test-list.tex}{\from{hologo.dtx}{test-list}}%
  \nopreamble
  \nopostamble
  \usedir{source/latex/oberdiek/catalogue}%
  \file{hologo.xml}{\from{hologo.dtx}{catalogue}}%
}

\catcode32=13\relax% active space
\let =\space%
\Msg{************************************************************************}
\Msg{*}
\Msg{* To finish the installation you have to move the following}
\Msg{* file into a directory searched by TeX:}
\Msg{*}
\Msg{*     hologo.sty}
\Msg{*}
\Msg{* To produce the documentation run the file `hologo.drv'}
\Msg{* through LaTeX.}
\Msg{*}
\Msg{* Happy TeXing!}
\Msg{*}
\Msg{************************************************************************}

\endbatchfile
%</install>
%<*ignore>
\fi
%</ignore>
%<*driver>
\NeedsTeXFormat{LaTeX2e}
\ProvidesFile{hologo.drv}%
  [2016/05/12 v1.11 A logo collection with bookmark support (HO)]%
\documentclass{ltxdoc}
\usepackage{holtxdoc}[2011/11/22]
\usepackage{hologo}[2016/05/12]
\usepackage{longtable}
\usepackage{array}
\usepackage{paralist}
%\usepackage[T1]{fontenc}
%\usepackage{lmodern}
\begin{document}
  \DocInput{hologo.dtx}%
\end{document}
%</driver>
% \fi
%
%
% \CharacterTable
%  {Upper-case    \A\B\C\D\E\F\G\H\I\J\K\L\M\N\O\P\Q\R\S\T\U\V\W\X\Y\Z
%   Lower-case    \a\b\c\d\e\f\g\h\i\j\k\l\m\n\o\p\q\r\s\t\u\v\w\x\y\z
%   Digits        \0\1\2\3\4\5\6\7\8\9
%   Exclamation   \!     Double quote  \"     Hash (number) \#
%   Dollar        \$     Percent       \%     Ampersand     \&
%   Acute accent  \'     Left paren    \(     Right paren   \)
%   Asterisk      \*     Plus          \+     Comma         \,
%   Minus         \-     Point         \.     Solidus       \/
%   Colon         \:     Semicolon     \;     Less than     \<
%   Equals        \=     Greater than  \>     Question mark \?
%   Commercial at \@     Left bracket  \[     Backslash     \\
%   Right bracket \]     Circumflex    \^     Underscore    \_
%   Grave accent  \`     Left brace    \{     Vertical bar  \|
%   Right brace   \}     Tilde         \~}
%
% \GetFileInfo{hologo.drv}
%
% \title{The \xpackage{hologo} package}
% \date{2016/05/12 v1.11}
% \author{Heiko Oberdiek\\\xemail{heiko.oberdiek at googlemail.com}}
%
% \maketitle
%
% \begin{abstract}
% This package starts a collection of logos with support for bookmarks
% strings.
% \end{abstract}
%
% \tableofcontents
%
% \section{Documentation}
%
% \subsection{Logo macros}
%
% \begin{declcs}{hologo} \M{name}
% \end{declcs}
% Macro \cs{hologo} sets the logo with name \meta{name}.
% The following table shows the supported names.
%
% \begingroup
%   \def\hologoEntry#1#2#3{^^A
%     #1&#2&\hologoLogoSetup{#1}{variant=#2}\hologo{#1}&#3\tabularnewline
%   }
%   \begin{longtable}{>{\ttfamily}l>{\ttfamily}lll}
%     \rmfamily\bfseries{name} & \rmfamily\bfseries variant
%     & \bfseries logo & \bfseries since\\
%     \hline
%     \endhead
%     \hologoList
%   \end{longtable}
% \endgroup
%
% \begin{declcs}{Hologo} \M{name}
% \end{declcs}
% Macro \cs{Hologo} starts the logo \meta{name} with an uppercase
% letter. As an exception small greek letters are not converted
% to uppercase. Examples, see \hologo{eTeX} and \hologo{ExTeX}.
%
% \subsection{Setup macros}
%
% The package does not support package options, but the following
% setup macros can be used to set options.
%
% \begin{declcs}{hologoSetup} \M{key value list}
% \end{declcs}
% Macro \cs{hologoSetup} sets global options.
%
% \begin{declcs}{hologoLogoSetup} \M{logo} \M{key value list}
% \end{declcs}
% Some options can also be used to configure a logo.
% These settings take precedence over global option settings.
%
% \subsection{Options}\label{sec:options}
%
% There are boolean and string options:
% \begin{description}
% \item[Boolean option:]
% It takes |true| or |false|
% as value. If the value is omitted, then |true| is used.
% \item[String option:]
% A value must be given as string. (But the string might be empty.)
% \end{description}
% The following options can be used both in \cs{hologoSetup}
% and \cs{hologoLogoSetup}:
% \begin{description}
% \def\entry#1{\item[\xoption{#1}:]}
% \entry{break}
%   enables or disables line breaks inside the logo. This setting is
%   refined by options \xoption{hyphenbreak}, \xoption{spacebreak}
%   or \xoption{discretionarybreak}.
%   Default is |false|.
% \entry{hyphenbreak}
%   enables or disables the line break right after the hyphen character.
% \entry{spacebreak}
%   enables or disables line breaks at space characters.
% \entry{discretionarybreak}
%   enables or disables line breaks at hyphenation points
%   (inserted by \cs{-}).
% \end{description}
% Macro \cs{hologoLogoSetup} also knows:
% \begin{description}
% \item[\xoption{variant}:]
%   This is a string option. It specifies a variant of a logo that
%   must exist. An empty string selects the package default variant.
% \end{description}
% Example:
% \begin{quote}
%   |\hologoSetup{break=false}|\\
%   |\hologoLogoSetup{plainTeX}{variant=hyphen,hyphenbreak}|\\
%   Then ``plain-\TeX'' contains one break point after the hyphen.
% \end{quote}
%
% \subsection{Driver options}
%
% Sometimes graphical operations are needed to construct some
% glyphs (e.g.\ \hologo{XeTeX}). If package \xpackage{graphics}
% or package \xpackage{pgf} are found, then the macros are taken
% from there. Otherwise the packge defines its own operations
% and therefore needs the driver information. Many drivers are
% detected automatically (\hologo{pdfTeX}/\hologo{LuaTeX}
% in PDF mode, \hologo{XeTeX}, \hologo{VTeX}). These have precedence
% over a driver option. The driver can be given as package option
% or using \cs{hologoDriverSetup}.
% The following list contains the recognized driver options:
% \begin{itemize}
% \item \xoption{pdftex}, \xoption{luatex}
% \item \xoption{dvipdfm}, \xoption{dvipdfmx}
% \item \xoption{dvips}, \xoption{dvipsone}, \xoption{xdvi}
% \item \xoption{xetex}
% \item \xoption{vtex}
% \end{itemize}
% The left driver of a line is the driver name that is used internally.
% The following names are aliases for drivers that use the
% same method. Therefore the entry in the \xext{log} file for
% the used driver prints the internally used driver name.
% \begin{description}
% \item[\xoption{driverfallback}:]
%   This option expects a driver that is used,
%   if the driver could not be detected automatically.
% \end{description}
%
% \begin{declcs}{hologoDriverSetup} \M{driver option}
% \end{declcs}
% The driver can also be configured after package loading
% using \cs{hologoDriverSetup}, also the way for \hologo{plainTeX}
% to setup the driver.
%
% \subsection{Font setup}
%
% Some logos require a special font, but should also be usable by
% \hologo{plainTeX}. Therefore the package provides some ways
% to influence the font settings. The options below
% take font settings as values. Both font commands
% such as \cs{sffamily} and macros that take one argument
% like \cs{textsf} can be used.
%
% \begin{declcs}{hologoFontSetup} \M{key value list}
% \end{declcs}
% Macro \cs{hologoFontSetup} sets the fonts for all logos.
% Supported keys:
% \begin{description}
% \def\entry#1{\item[\xoption{#1}:]}
% \entry{general}
%   This font is used for all logos. The default is empty.
%   That means no special font is used.
% \entry{bibsf}
%   This font is used for
%   {\hologoLogoSetup{BibTeX}{variant=sf}\hologo{BibTeX}}
%   with variant \xoption{sf}.
% \entry{rm}
%   This font is a serif font. It is used for \hologo{ExTeX}.
% \entry{sc}
%   This font specifies a small caps font. It is used for
%   {\hologoLogoSetup{BibTeX}{variant=sc}\hologo{BibTeX}}
%   with variant \xoption{sc}.
% \entry{sf}
%   This font specifies a sans serif font. The default
%   is \cs{sffamily}, then \cs{sf} is tried. Otherwise
%   a warning is given. It is used by \hologo{KOMAScript}.
% \entry{sy}
%   This is the font for math symbols (e.g. cmsy).
%   It is used by \hologo{AmS}, \hologo{NTS}, \hologo{ExTeX}.
% \entry{logo}
%   \hologo{METAFONT} and \hologo{METAPOST} are using that font.
%   In \hologo{LaTeX} \cs{logofamily} is used and
%   the definitions of package \xpackage{mflogo} are used
%   if the package is not loaded.
%   Otherwise the \cs{tenlogo} is used and defined
%   if it does not already exists.
% \end{description}
%
% \begin{declcs}{hologoLogoFontSetup} \M{logo} \M{key value list}
% \end{declcs}
% Fonts can also be set for a logo or logo component separately,
% see the following list.
% The keys are the same as for \cs{hologoFontSetup}.
%
% \begin{longtable}{>{\ttfamily}l>{\sffamily}ll}
%   \meta{logo} & keys & result\\
%   \hline
%   \endhead
%   BibTeX & bibsf & {\hologoLogoSetup{BibTeX}{variant=sf}\hologo{BibTeX}}\\[.5ex]
%   BibTeX & sc & {\hologoLogoSetup{BibTeX}{variant=sc}\hologo{BibTeX}}\\[.5ex]
%   ExTeX & rm & \hologo{ExTeX}\\
%   SliTeX & rm & \hologo{SliTeX}\\[.5ex]
%   AmS & sy & \hologo{AmS}\\
%   ExTeX & sy & \hologo{ExTeX}\\
%   NTS & sy & \hologo{NTS}\\[.5ex]
%   KOMAScript & sf & \hologo{KOMAScript}\\[.5ex]
%   METAFONT & logo & \hologo{METAFONT}\\
%   METAPOST & logo & \hologo{METAPOST}\\[.5ex]
%   SliTeX & sc \hologo{SliTeX}
% \end{longtable}
%
% \subsubsection{Font order}
%
% For all logos the font \xoption{general} is applied first.
% Example:
%\begin{quote}
%|\hologoFontSetup{general=\color{red}}|
%\end{quote}
% will print red logos.
% Then if the font uses a special font \xoption{sf}, for example,
% the font is applied that is setup by \cs{hologoLogoFontSetup}.
% If this font is not setup, then the common font setup
% by \cs{hologoFontSetup} is used. Otherwise a warning is given,
% that there is no font configured.
%
% \subsection{Additional user macros}
%
% Usually a variant of a logo is configured by using
% \cs{hologoLogoSetup}, because it is bad style to mix
% different variants of the same logo in the same text.
% There the following macros are a convenience for testing.
%
% \begin{declcs}{hologoVariant} \M{name} \M{variant}\\
%   \cs{HologoVariant} \M{name} \M{variant}
% \end{declcs}
% Logo \meta{name} is set using \meta{variant} that specifies
% explicitely which variant of the macro is used. If the argument
% is empty, then the default form of the logo is used
% (configurable by \cs{hologoLogoSetup}).
%
% \cs{HologoVariant} is used if the logo is set in a context
% that needs an uppercase first letter (beginning of a sentence, \dots).
%
% \begin{declcs}{hologoList}\\
%   \cs{hologoEntry} \M{logo} \M{variant} \M{since}
% \end{declcs}
% Macro \cs{hologoList} contains all logos that are provided
% by the package including variants. The list consists of calls
% of \cs{hologoEntry} with three arguments starting with the
% logo name \meta{logo} and its variant \meta{variant}. An empty
% variant means the current default. Argument \meta{since} specifies
% with version of the package \xpackage{hologo} is needed to get
% the logo. If the logo is fixed, then the date gets updated.
% Therefore the date \meta{since} is not exactly the date of
% the first introduction, but rather the date of the latest fix.
%
% Before \cs{hologoList} can be used, macro \cs{hologoEntry} needs
% a definition. The example file in section \ref{sec:example}
% shows applications of \cs{hologoList}.
%
% \subsection{Supported contexts}
%
% Macros \cs{hologo} and friends support special contexts:
% \begin{itemize}
% \item \hologo{LaTeX}'s protection mechanism.
% \item Bookmarks of package \xpackage{hyperref}.
% \item Package \xpackage{tex4ht}.
% \item The macros can be used inside \cs{csname} constructs,
%   if \cs{ifincsname} is available (\hologo{pdfTeX}, \hologo{XeTeX},
%   \hologo{LuaTeX}).
% \end{itemize}
%
% \subsection{Example}
% \label{sec:example}
%
% The following example prints the logos in different fonts.
%    \begin{macrocode}
%<*example>
%<<verbatim
\NeedsTeXFormat{LaTeX2e}
\documentclass[a4paper]{article}
\usepackage[
  hmargin=20mm,
  vmargin=20mm,
]{geometry}
\pagestyle{empty}
\usepackage{hologo}[2016/05/12]
\usepackage{longtable}
\usepackage{array}
\setlength{\extrarowheight}{2pt}
\usepackage[T1]{fontenc}
\usepackage{lmodern}
\usepackage{pdflscape}
\usepackage[
  pdfencoding=auto,
]{hyperref}
\hypersetup{
  pdfauthor={Heiko Oberdiek},
  pdftitle={Example for package `hologo'},
  pdfsubject={Logos with fonts lmr, lmss, qtm, qpl, qhv},
}
\usepackage{bookmark}

% Print the logo list on the console

\begingroup
  \typeout{}%
  \typeout{*** Begin of logo list ***}%
  \newcommand*{\hologoEntry}[3]{%
    \typeout{#1 \ifx\\#2\\\else(#2) \fi[#3]}%
  }%
  \hologoList
  \typeout{*** End of logo list ***}%
  \typeout{}%
\endgroup

\begin{document}
\begin{landscape}

  \section{Example file for package `hologo'}

  % Table for font names

  \begin{longtable}{>{\bfseries}ll}
    \textbf{font} & \textbf{Font name}\\
    \hline
    lmr & Latin Modern Roman\\
    lmss & Latin Modern Sans\\
    qtm & \TeX\ Gyre Termes\\
    qhv & \TeX\ Gyre Heros\\
    qpl & \TeX\ Gyre Pagella\\
  \end{longtable}

  % Logo list with logos in different fonts

  \begingroup
    \newcommand*{\SetVariant}[2]{%
      \ifx\\#2\\%
      \else
        \hologoLogoSetup{#1}{variant=#2}%
      \fi
    }%
    \newcommand*{\hologoEntry}[3]{%
      \SetVariant{#1}{#2}%
      \raisebox{1em}[0pt][0pt]{\hypertarget{#1@#2}{}}%
      \bookmark[%
        dest={#1@#2},%
      ]{%
        #1\ifx\\#2\\\else\space(#2)\fi: \Hologo{#1}, \hologo{#1} %
        [Unicode]%
      }%
      \hypersetup{unicode=false}%
      \bookmark[%
        dest={#1@#2},%
      ]{%
        #1\ifx\\#2\\\else\space(#2)\fi: \Hologo{#1}, \hologo{#1} %
        [PDFDocEncoding]%
      }%
      \texttt{#1}%
      &%
      \texttt{#2}%
      &%
      \Hologo{#1}%
      &%
      \SetVariant{#1}{#2}%
      \hologo{#1}%
      &%
      \SetVariant{#1}{#2}%
      \fontfamily{qtm}\selectfont
      \hologo{#1}%
      &%
      \SetVariant{#1}{#2}%
      \fontfamily{qpl}\selectfont
      \hologo{#1}%
      &%
      \SetVariant{#1}{#2}%
      \textsf{\hologo{#1}}%
      &%
      \SetVariant{#1}{#2}%
      \fontfamily{qhv}\selectfont
      \hologo{#1}%
      \tabularnewline
    }%
    \begin{longtable}{llllllll}%
      \textbf{\textit{logo}} & \textbf{\textit{variant}} &
      \texttt{\string\Hologo} &
      \textbf{lmr} & \textbf{qtm} & \textbf{qpl} &
      \textbf{lmss} & \textbf{qhv}
      \tabularnewline
      \hline
      \endhead
      \hologoList
    \end{longtable}%
  \endgroup

\end{landscape}
\end{document}
%verbatim
%</example>
%    \end{macrocode}
%
% \StopEventually{
% }
%
% \section{Implementation}
%    \begin{macrocode}
%<*package>
%    \end{macrocode}
%    Reload check, especially if the package is not used with \LaTeX.
%    \begin{macrocode}
\begingroup\catcode61\catcode48\catcode32=10\relax%
  \catcode13=5 % ^^M
  \endlinechar=13 %
  \catcode35=6 % #
  \catcode39=12 % '
  \catcode44=12 % ,
  \catcode45=12 % -
  \catcode46=12 % .
  \catcode58=12 % :
  \catcode64=11 % @
  \catcode123=1 % {
  \catcode125=2 % }
  \expandafter\let\expandafter\x\csname ver@hologo.sty\endcsname
  \ifx\x\relax % plain-TeX, first loading
  \else
    \def\empty{}%
    \ifx\x\empty % LaTeX, first loading,
      % variable is initialized, but \ProvidesPackage not yet seen
    \else
      \expandafter\ifx\csname PackageInfo\endcsname\relax
        \def\x#1#2{%
          \immediate\write-1{Package #1 Info: #2.}%
        }%
      \else
        \def\x#1#2{\PackageInfo{#1}{#2, stopped}}%
      \fi
      \x{hologo}{The package is already loaded}%
      \aftergroup\endinput
    \fi
  \fi
\endgroup%
%    \end{macrocode}
%    Package identification:
%    \begin{macrocode}
\begingroup\catcode61\catcode48\catcode32=10\relax%
  \catcode13=5 % ^^M
  \endlinechar=13 %
  \catcode35=6 % #
  \catcode39=12 % '
  \catcode40=12 % (
  \catcode41=12 % )
  \catcode44=12 % ,
  \catcode45=12 % -
  \catcode46=12 % .
  \catcode47=12 % /
  \catcode58=12 % :
  \catcode64=11 % @
  \catcode91=12 % [
  \catcode93=12 % ]
  \catcode123=1 % {
  \catcode125=2 % }
  \expandafter\ifx\csname ProvidesPackage\endcsname\relax
    \def\x#1#2#3[#4]{\endgroup
      \immediate\write-1{Package: #3 #4}%
      \xdef#1{#4}%
    }%
  \else
    \def\x#1#2[#3]{\endgroup
      #2[{#3}]%
      \ifx#1\@undefined
        \xdef#1{#3}%
      \fi
      \ifx#1\relax
        \xdef#1{#3}%
      \fi
    }%
  \fi
\expandafter\x\csname ver@hologo.sty\endcsname
\ProvidesPackage{hologo}%
  [2016/05/12 v1.11 A logo collection with bookmark support (HO)]%
%    \end{macrocode}
%
%    \begin{macrocode}
\begingroup\catcode61\catcode48\catcode32=10\relax%
  \catcode13=5 % ^^M
  \endlinechar=13 %
  \catcode123=1 % {
  \catcode125=2 % }
  \catcode64=11 % @
  \def\x{\endgroup
    \expandafter\edef\csname HOLOGO@AtEnd\endcsname{%
      \endlinechar=\the\endlinechar\relax
      \catcode13=\the\catcode13\relax
      \catcode32=\the\catcode32\relax
      \catcode35=\the\catcode35\relax
      \catcode61=\the\catcode61\relax
      \catcode64=\the\catcode64\relax
      \catcode123=\the\catcode123\relax
      \catcode125=\the\catcode125\relax
    }%
  }%
\x\catcode61\catcode48\catcode32=10\relax%
\catcode13=5 % ^^M
\endlinechar=13 %
\catcode35=6 % #
\catcode64=11 % @
\catcode123=1 % {
\catcode125=2 % }
\def\TMP@EnsureCode#1#2{%
  \edef\HOLOGO@AtEnd{%
    \HOLOGO@AtEnd
    \catcode#1=\the\catcode#1\relax
  }%
  \catcode#1=#2\relax
}
\TMP@EnsureCode{10}{12}% ^^J
\TMP@EnsureCode{33}{12}% !
\TMP@EnsureCode{34}{12}% "
\TMP@EnsureCode{36}{3}% $
\TMP@EnsureCode{38}{4}% &
\TMP@EnsureCode{39}{12}% '
\TMP@EnsureCode{40}{12}% (
\TMP@EnsureCode{41}{12}% )
\TMP@EnsureCode{42}{12}% *
\TMP@EnsureCode{43}{12}% +
\TMP@EnsureCode{44}{12}% ,
\TMP@EnsureCode{45}{12}% -
\TMP@EnsureCode{46}{12}% .
\TMP@EnsureCode{47}{12}% /
\TMP@EnsureCode{58}{12}% :
\TMP@EnsureCode{59}{12}% ;
\TMP@EnsureCode{60}{12}% <
\TMP@EnsureCode{62}{12}% >
\TMP@EnsureCode{63}{12}% ?
\TMP@EnsureCode{91}{12}% [
\TMP@EnsureCode{93}{12}% ]
\TMP@EnsureCode{94}{7}% ^ (superscript)
\TMP@EnsureCode{95}{8}% _ (subscript)
\TMP@EnsureCode{96}{12}% `
\TMP@EnsureCode{124}{12}% |
\edef\HOLOGO@AtEnd{%
  \HOLOGO@AtEnd
  \escapechar\the\escapechar\relax
  \noexpand\endinput
}
\escapechar=92 %
%    \end{macrocode}
%
% \subsection{Logo list}
%
%    \begin{macro}{\hologoList}
%    \begin{macrocode}
\def\hologoList{%
  \hologoEntry{(La)TeX}{}{2011/10/01}%
  \hologoEntry{AmSLaTeX}{}{2010/04/16}%
  \hologoEntry{AmSTeX}{}{2010/04/16}%
  \hologoEntry{biber}{}{2011/10/01}%
  \hologoEntry{BibTeX}{}{2011/10/01}%
  \hologoEntry{BibTeX}{sf}{2011/10/01}%
  \hologoEntry{BibTeX}{sc}{2011/10/01}%
  \hologoEntry{BibTeX8}{}{2011/11/22}%
  \hologoEntry{ConTeXt}{}{2011/03/25}%
  \hologoEntry{ConTeXt}{narrow}{2011/03/25}%
  \hologoEntry{ConTeXt}{simple}{2011/03/25}%
  \hologoEntry{emTeX}{}{2010/04/26}%
  \hologoEntry{eTeX}{}{2010/04/08}%
  \hologoEntry{ExTeX}{}{2011/10/01}%
  \hologoEntry{HanTheThanh}{}{2011/11/29}%
  \hologoEntry{iniTeX}{}{2011/10/01}%
  \hologoEntry{KOMAScript}{}{2011/10/01}%
  \hologoEntry{La}{}{2010/05/08}%
  \hologoEntry{LaTeX}{}{2010/04/08}%
  \hologoEntry{LaTeX2e}{}{2010/04/08}%
  \hologoEntry{LaTeX3}{}{2010/04/24}%
  \hologoEntry{LaTeXe}{}{2010/04/08}%
  \hologoEntry{LaTeXML}{}{2011/11/22}%
  \hologoEntry{LaTeXTeX}{}{2011/10/01}%
  \hologoEntry{LuaLaTeX}{}{2010/04/08}%
  \hologoEntry{LuaTeX}{}{2010/04/08}%
  \hologoEntry{LyX}{}{2011/10/01}%
  \hologoEntry{METAFONT}{}{2011/10/01}%
  \hologoEntry{MetaFun}{}{2011/10/01}%
  \hologoEntry{METAPOST}{}{2011/10/01}%
  \hologoEntry{MetaPost}{}{2011/10/01}%
  \hologoEntry{MiKTeX}{}{2011/10/01}%
  \hologoEntry{NTS}{}{2011/10/01}%
  \hologoEntry{OzMF}{}{2011/10/01}%
  \hologoEntry{OzMP}{}{2011/10/01}%
  \hologoEntry{OzTeX}{}{2011/10/01}%
  \hologoEntry{OzTtH}{}{2011/10/01}%
  \hologoEntry{PCTeX}{}{2011/10/01}%
  \hologoEntry{pdfTeX}{}{2011/10/01}%
  \hologoEntry{pdfLaTeX}{}{2011/10/01}%
  \hologoEntry{PiC}{}{2011/10/01}%
  \hologoEntry{PiCTeX}{}{2011/10/01}%
  \hologoEntry{plainTeX}{}{2010/04/08}%
  \hologoEntry{plainTeX}{space}{2010/04/16}%
  \hologoEntry{plainTeX}{hyphen}{2010/04/16}%
  \hologoEntry{plainTeX}{runtogether}{2010/04/16}%
  \hologoEntry{SageTeX}{}{2011/11/22}%
  \hologoEntry{SLiTeX}{}{2011/10/01}%
  \hologoEntry{SLiTeX}{lift}{2011/10/01}%
  \hologoEntry{SLiTeX}{narrow}{2011/10/01}%
  \hologoEntry{SLiTeX}{simple}{2011/10/01}%
  \hologoEntry{SliTeX}{}{2011/10/01}%
  \hologoEntry{SliTeX}{narrow}{2011/10/01}%
  \hologoEntry{SliTeX}{simple}{2011/10/01}%
  \hologoEntry{SliTeX}{lift}{2011/10/01}%
  \hologoEntry{teTeX}{}{2011/10/01}%
  \hologoEntry{TeX}{}{2010/04/08}%
  \hologoEntry{TeX4ht}{}{2011/11/22}%
  \hologoEntry{TTH}{}{2011/11/22}%
  \hologoEntry{virTeX}{}{2011/10/01}%
  \hologoEntry{VTeX}{}{2010/04/24}%
  \hologoEntry{Xe}{}{2010/04/08}%
  \hologoEntry{XeLaTeX}{}{2010/04/08}%
  \hologoEntry{XeTeX}{}{2010/04/08}%
}
%    \end{macrocode}
%    \end{macro}
%
% \subsection{Load resources}
%
%    \begin{macrocode}
\begingroup\expandafter\expandafter\expandafter\endgroup
\expandafter\ifx\csname RequirePackage\endcsname\relax
  \def\TMP@RequirePackage#1[#2]{%
    \begingroup\expandafter\expandafter\expandafter\endgroup
    \expandafter\ifx\csname ver@#1.sty\endcsname\relax
      \input #1.sty\relax
    \fi
  }%
  \TMP@RequirePackage{ltxcmds}[2011/02/04]%
  \TMP@RequirePackage{infwarerr}[2010/04/08]%
  \TMP@RequirePackage{kvsetkeys}[2010/03/01]%
  \TMP@RequirePackage{kvdefinekeys}[2010/03/01]%
  \TMP@RequirePackage{pdftexcmds}[2010/04/01]%
  \TMP@RequirePackage{ifpdf}[2010/01/28]%
  \TMP@RequirePackage{ifluatex}[2010/03/01]%
  \ltx@IfUndefined{newif}{%
    \expandafter\let\csname newif\endcsname\ltx@newif
  }{}%
  \TMP@RequirePackage{ifxetex}[2009/01/23]%
  \TMP@RequirePackage{ifvtex}[2010/03/01]%
\else
  \RequirePackage{ltxcmds}[2011/02/04]%
  \RequirePackage{infwarerr}[2010/04/08]%
  \RequirePackage{kvsetkeys}[2010/03/01]%
  \RequirePackage{kvdefinekeys}[2010/03/01]%
  \RequirePackage{pdftexcmds}[2010/04/01]%
  \RequirePackage{ifpdf}[2010/01/28]%
  \RequirePackage{ifluatex}[2010/03/01]%
  \RequirePackage{ifxetex}[2009/01/23]%
  \RequirePackage{ifvtex}[2010/03/01]%
\fi
%    \end{macrocode}
%
%    \begin{macro}{\HOLOGO@IfDefined}
%    \begin{macrocode}
\def\HOLOGO@IfExists#1{%
  \ifx\@undefined#1%
    \expandafter\ltx@secondoftwo
  \else
    \ifx\relax#1%
      \expandafter\ltx@secondoftwo
    \else
      \expandafter\expandafter\expandafter\ltx@firstoftwo
    \fi
  \fi
}
%    \end{macrocode}
%    \end{macro}
%
% \subsection{Setup macros}
%
%    \begin{macro}{\hologoSetup}
%    \begin{macrocode}
\def\hologoSetup{%
  \let\HOLOGO@name\relax
  \HOLOGO@Setup
}
%    \end{macrocode}
%    \end{macro}
%
%    \begin{macro}{\hologoLogoSetup}
%    \begin{macrocode}
\def\hologoLogoSetup#1{%
  \edef\HOLOGO@name{#1}%
  \ltx@IfUndefined{HoLogo@\HOLOGO@name}{%
    \@PackageError{hologo}{%
      Unknown logo `\HOLOGO@name'%
    }\@ehc
    \ltx@gobble
  }{%
    \HOLOGO@Setup
  }%
}
%    \end{macrocode}
%    \end{macro}
%
%    \begin{macro}{\HOLOGO@Setup}
%    \begin{macrocode}
\def\HOLOGO@Setup{%
  \kvsetkeys{HoLogo}%
}
%    \end{macrocode}
%    \end{macro}
%
% \subsection{Options}
%
%    \begin{macro}{\HOLOGO@DeclareBoolOption}
%    \begin{macrocode}
\def\HOLOGO@DeclareBoolOption#1{%
  \expandafter\chardef\csname HOLOGOOPT@#1\endcsname\ltx@zero
  \kv@define@key{HoLogo}{#1}[true]{%
    \def\HOLOGO@temp{##1}%
    \ifx\HOLOGO@temp\HOLOGO@true
      \ifx\HOLOGO@name\relax
        \expandafter\chardef\csname HOLOGOOPT@#1\endcsname=\ltx@one
      \else
        \expandafter\chardef\csname
        HoLogoOpt@#1@\HOLOGO@name\endcsname\ltx@one
      \fi
      \HOLOGO@SetBreakAll{#1}%
    \else
      \ifx\HOLOGO@temp\HOLOGO@false
        \ifx\HOLOGO@name\relax
          \expandafter\chardef\csname HOLOGOOPT@#1\endcsname=\ltx@zero
        \else
          \expandafter\chardef\csname
          HoLogoOpt@#1@\HOLOGO@name\endcsname=\ltx@zero
        \fi
        \HOLOGO@SetBreakAll{#1}%
      \else
        \@PackageError{hologo}{%
          Unknown value `##1' for boolean option `#1'.\MessageBreak
          Known values are `true' and `false'%
        }\@ehc
      \fi
    \fi
  }%
}
%    \end{macrocode}
%    \end{macro}
%
%    \begin{macro}{\HOLOGO@SetBreakAll}
%    \begin{macrocode}
\def\HOLOGO@SetBreakAll#1{%
  \def\HOLOGO@temp{#1}%
  \ifx\HOLOGO@temp\HOLOGO@break
    \ifx\HOLOGO@name\relax
      \chardef\HOLOGOOPT@hyphenbreak=\HOLOGOOPT@break
      \chardef\HOLOGOOPT@spacebreak=\HOLOGOOPT@break
      \chardef\HOLOGOOPT@discretionarybreak=\HOLOGOOPT@break
    \else
      \expandafter\chardef
         \csname HoLogoOpt@hyphenbreak@\HOLOGO@name\endcsname=%
         \csname HoLogoOpt@break@\HOLOGO@name\endcsname
      \expandafter\chardef
         \csname HoLogoOpt@spacebreak@\HOLOGO@name\endcsname=%
         \csname HoLogoOpt@break@\HOLOGO@name\endcsname
      \expandafter\chardef
         \csname HoLogoOpt@discretionarybreak@\HOLOGO@name
             \endcsname=%
         \csname HoLogoOpt@break@\HOLOGO@name\endcsname
    \fi
  \fi
}
%    \end{macrocode}
%    \end{macro}
%
%    \begin{macro}{\HOLOGO@true}
%    \begin{macrocode}
\def\HOLOGO@true{true}
%    \end{macrocode}
%    \end{macro}
%    \begin{macro}{\HOLOGO@false}
%    \begin{macrocode}
\def\HOLOGO@false{false}
%    \end{macrocode}
%    \end{macro}
%    \begin{macro}{\HOLOGO@break}
%    \begin{macrocode}
\def\HOLOGO@break{break}
%    \end{macrocode}
%    \end{macro}
%
%    \begin{macrocode}
\HOLOGO@DeclareBoolOption{break}
\HOLOGO@DeclareBoolOption{hyphenbreak}
\HOLOGO@DeclareBoolOption{spacebreak}
\HOLOGO@DeclareBoolOption{discretionarybreak}
%    \end{macrocode}
%
%    \begin{macrocode}
\kv@define@key{HoLogo}{variant}{%
  \ifx\HOLOGO@name\relax
    \@PackageError{hologo}{%
      Option `variant' is not available in \string\hologoSetup,%
      \MessageBreak
      Use \string\hologoLogoSetup\space instead%
    }\@ehc
  \else
    \edef\HOLOGO@temp{#1}%
    \ifx\HOLOGO@temp\ltx@empty
      \expandafter
      \let\csname HoLogoOpt@variant@\HOLOGO@name\endcsname\@undefined
    \else
      \ltx@IfUndefined{HoLogo@\HOLOGO@name @\HOLOGO@temp}{%
        \@PackageError{hologo}{%
          Unknown variant `\HOLOGO@temp' of logo `\HOLOGO@name'%
        }\@ehc
      }{%
        \expandafter
        \let\csname HoLogoOpt@variant@\HOLOGO@name\endcsname
            \HOLOGO@temp
      }%
    \fi
  \fi
}
%    \end{macrocode}
%
%    \begin{macro}{\HOLOGO@Variant}
%    \begin{macrocode}
\def\HOLOGO@Variant#1{%
  #1%
  \ltx@ifundefined{HoLogoOpt@variant@#1}{%
  }{%
    @\csname HoLogoOpt@variant@#1\endcsname
  }%
}
%    \end{macrocode}
%    \end{macro}
%
% \subsection{Break/no-break support}
%
%    \begin{macro}{\HOLOGO@space}
%    \begin{macrocode}
\def\HOLOGO@space{%
  \ltx@ifundefined{HoLogoOpt@spacebreak@\HOLOGO@name}{%
    \ltx@ifundefined{HoLogoOpt@break@\HOLOGO@name}{%
      \chardef\HOLOGO@temp=\HOLOGOOPT@spacebreak
    }{%
      \chardef\HOLOGO@temp=%
        \csname HoLogoOpt@break@\HOLOGO@name\endcsname
    }%
  }{%
    \chardef\HOLOGO@temp=%
      \csname HoLogoOpt@spacebreak@\HOLOGO@name\endcsname
  }%
  \ifcase\HOLOGO@temp
    \penalty10000 %
  \fi
  \ltx@space
}
%    \end{macrocode}
%    \end{macro}
%
%    \begin{macro}{\HOLOGO@hyphen}
%    \begin{macrocode}
\def\HOLOGO@hyphen{%
  \ltx@ifundefined{HoLogoOpt@hyphenbreak@\HOLOGO@name}{%
    \ltx@ifundefined{HoLogoOpt@break@\HOLOGO@name}{%
      \chardef\HOLOGO@temp=\HOLOGOOPT@hyphenbreak
    }{%
      \chardef\HOLOGO@temp=%
        \csname HoLogoOpt@break@\HOLOGO@name\endcsname
    }%
  }{%
    \chardef\HOLOGO@temp=%
      \csname HoLogoOpt@hyphenbreak@\HOLOGO@name\endcsname
  }%
  \ifcase\HOLOGO@temp
    \ltx@mbox{-}%
  \else
    -%
  \fi
}
%    \end{macrocode}
%    \end{macro}
%
%    \begin{macro}{\HOLOGO@discretionary}
%    \begin{macrocode}
\def\HOLOGO@discretionary{%
  \ltx@ifundefined{HoLogoOpt@discretionarybreak@\HOLOGO@name}{%
    \ltx@ifundefined{HoLogoOpt@break@\HOLOGO@name}{%
      \chardef\HOLOGO@temp=\HOLOGOOPT@discretionarybreak
    }{%
      \chardef\HOLOGO@temp=%
        \csname HoLogoOpt@break@\HOLOGO@name\endcsname
    }%
  }{%
    \chardef\HOLOGO@temp=%
      \csname HoLogoOpt@discretionarybreak@\HOLOGO@name\endcsname
  }%
  \ifcase\HOLOGO@temp
  \else
    \-%
  \fi
}
%    \end{macrocode}
%    \end{macro}
%
%    \begin{macro}{\HOLOGO@mbox}
%    \begin{macrocode}
\def\HOLOGO@mbox#1{%
  \ltx@ifundefined{HoLogoOpt@break@\HOLOGO@name}{%
    \chardef\HOLOGO@temp=\HOLOGOOPT@hyphenbreak
  }{%
    \chardef\HOLOGO@temp=%
      \csname HoLogoOpt@break@\HOLOGO@name\endcsname
  }%
  \ifcase\HOLOGO@temp
    \ltx@mbox{#1}%
  \else
    #1%
  \fi
}
%    \end{macrocode}
%    \end{macro}
%
% \subsection{Font support}
%
%    \begin{macro}{\HoLogoFont@font}
%    \begin{tabular}{@{}ll@{}}
%    |#1|:& logo name\\
%    |#2|:& font short name\\
%    |#3|:& text
%    \end{tabular}
%    \begin{macrocode}
\def\HoLogoFont@font#1#2#3{%
  \begingroup
    \ltx@IfUndefined{HoLogoFont@logo@#1.#2}{%
      \ltx@IfUndefined{HoLogoFont@font@#2}{%
        \@PackageWarning{hologo}{%
          Missing font `#2' for logo `#1'%
        }%
        #3%
      }{%
        \csname HoLogoFont@font@#2\endcsname{#3}%
      }%
    }{%
      \csname HoLogoFont@logo@#1.#2\endcsname{#3}%
    }%
  \endgroup
}
%    \end{macrocode}
%    \end{macro}
%
%    \begin{macro}{\HoLogoFont@Def}
%    \begin{macrocode}
\def\HoLogoFont@Def#1{%
  \expandafter\def\csname HoLogoFont@font@#1\endcsname
}
%    \end{macrocode}
%    \end{macro}
%    \begin{macro}{\HoLogoFont@LogoDef}
%    \begin{macrocode}
\def\HoLogoFont@LogoDef#1#2{%
  \expandafter\def\csname HoLogoFont@logo@#1.#2\endcsname
}
%    \end{macrocode}
%    \end{macro}
%
% \subsubsection{Font defaults}
%
%    \begin{macro}{\HoLogoFont@font@general}
%    \begin{macrocode}
\HoLogoFont@Def{general}{}%
%    \end{macrocode}
%    \end{macro}
%
%    \begin{macro}{\HoLogoFont@font@rm}
%    \begin{macrocode}
\ltx@IfUndefined{rmfamily}{%
  \ltx@IfUndefined{rm}{%
  }{%
    \HoLogoFont@Def{rm}{\rm}%
  }%
}{%
  \HoLogoFont@Def{rm}{\rmfamily}%
}
%    \end{macrocode}
%    \end{macro}
%
%    \begin{macro}{\HoLogoFont@font@sf}
%    \begin{macrocode}
\ltx@IfUndefined{sffamily}{%
  \ltx@IfUndefined{sf}{%
  }{%
    \HoLogoFont@Def{sf}{\sf}%
  }%
}{%
  \HoLogoFont@Def{sf}{\sffamily}%
}
%    \end{macrocode}
%    \end{macro}
%
%    \begin{macro}{\HoLogoFont@font@bibsf}
%    In case of \hologo{plainTeX} the original small caps
%    variant is used as default. In \hologo{LaTeX}
%    the definition of package \xpackage{dtklogos} \cite{dtklogos}
%    is used.
%\begin{quote}
%\begin{verbatim}
%\DeclareRobustCommand{\BibTeX}{%
%  B%
%  \kern-.05em%
%  \hbox{%
%    $\m@th$% %% force math size calculations
%    \csname S@\f@size\endcsname
%    \fontsize\sf@size\z@
%    \math@fontsfalse
%    \selectfont
%    I%
%    \kern-.025em%
%    B
%  }%
%  \kern-.08em%
%  \-%
%  \TeX
%}
%\end{verbatim}
%\end{quote}
%    \begin{macrocode}
\ltx@IfUndefined{selectfont}{%
  \ltx@IfUndefined{tensc}{%
    \font\tensc=cmcsc10\relax
  }{}%
  \HoLogoFont@Def{bibsf}{\tensc}%
}{%
  \HoLogoFont@Def{bibsf}{%
    $\mathsurround=0pt$%
    \csname S@\f@size\endcsname
    \fontsize\sf@size{0pt}%
    \math@fontsfalse
    \selectfont
  }%
}
%    \end{macrocode}
%    \end{macro}
%
%    \begin{macro}{\HoLogoFont@font@sc}
%    \begin{macrocode}
\ltx@IfUndefined{scshape}{%
  \ltx@IfUndefined{tensc}{%
    \font\tensc=cmcsc10\relax
  }{}%
  \HoLogoFont@Def{sc}{\tensc}%
}{%
  \HoLogoFont@Def{sc}{\scshape}%
}
%    \end{macrocode}
%    \end{macro}
%
%    \begin{macro}{\HoLogoFont@font@sy}
%    \begin{macrocode}
\ltx@IfUndefined{usefont}{%
  \ltx@IfUndefined{tensy}{%
  }{%
    \HoLogoFont@Def{sy}{\tensy}%
  }%
}{%
  \HoLogoFont@Def{sy}{%
    \usefont{OMS}{cmsy}{m}{n}%
  }%
}
%    \end{macrocode}
%    \end{macro}
%
%    \begin{macro}{\HoLogoFont@font@logo}
%    \begin{macrocode}
\begingroup
  \def\x{LaTeX2e}%
\expandafter\endgroup
\ifx\fmtname\x
  \ltx@IfUndefined{logofamily}{%
    \DeclareRobustCommand\logofamily{%
      \not@math@alphabet\logofamily\relax
      \fontencoding{U}%
      \fontfamily{logo}%
      \selectfont
    }%
  }{}%
  \ltx@IfUndefined{logofamily}{%
  }{%
    \HoLogoFont@Def{logo}{\logofamily}%
  }%
\else
  \ltx@IfUndefined{tenlogo}{%
    \font\tenlogo=logo10\relax
  }{}%
  \HoLogoFont@Def{logo}{\tenlogo}%
\fi
%    \end{macrocode}
%    \end{macro}
%
% \subsubsection{Font setup}
%
%    \begin{macro}{\hologoFontSetup}
%    \begin{macrocode}
\def\hologoFontSetup{%
  \let\HOLOGO@name\relax
  \HOLOGO@FontSetup
}
%    \end{macrocode}
%    \end{macro}
%
%    \begin{macro}{\hologoLogoFontSetup}
%    \begin{macrocode}
\def\hologoLogoFontSetup#1{%
  \edef\HOLOGO@name{#1}%
  \ltx@IfUndefined{HoLogo@\HOLOGO@name}{%
    \@PackageError{hologo}{%
      Unknown logo `\HOLOGO@name'%
    }\@ehc
    \ltx@gobble
  }{%
    \HOLOGO@FontSetup
  }%
}
%    \end{macrocode}
%    \end{macro}
%
%    \begin{macro}{\HOLOGO@FontSetup}
%    \begin{macrocode}
\def\HOLOGO@FontSetup{%
  \kvsetkeys{HoLogoFont}%
}
%    \end{macrocode}
%    \end{macro}
%
%    \begin{macrocode}
\def\HOLOGO@temp#1{%
  \kv@define@key{HoLogoFont}{#1}{%
    \ifx\HOLOGO@name\relax
      \HoLogoFont@Def{#1}{##1}%
    \else
      \HoLogoFont@LogoDef\HOLOGO@name{#1}{##1}%
    \fi
  }%
}
\HOLOGO@temp{general}
\HOLOGO@temp{sf}
%    \end{macrocode}
%
% \subsection{Generic logo commands}
%
%    \begin{macrocode}
\HOLOGO@IfExists\hologo{%
  \@PackageError{hologo}{%
    \string\hologo\ltx@space is already defined.\MessageBreak
    Package loading is aborted%
  }\@ehc
  \HOLOGO@AtEnd
}%
\HOLOGO@IfExists\hologoRobust{%
  \@PackageError{hologo}{%
    \string\hologoRobust\ltx@space is already defined.\MessageBreak
    Package loading is aborted%
  }\@ehc
  \HOLOGO@AtEnd
}%
%    \end{macrocode}
%
% \subsubsection{\cs{hologo} and friends}
%
%    \begin{macrocode}
\ifluatex
  \expandafter\ltx@firstofone
\else
  \expandafter\ltx@gobble
\fi
{%
  \ltx@IfUndefined{ifincsname}{%
    \ifnum\luatexversion<36 %
      \expandafter\ltx@gobble
    \else
      \expandafter\ltx@firstofone
    \fi
    {%
      \begingroup
        \ifcase0%
            \directlua{%
              if tex.enableprimitives then %
                tex.enableprimitives('HOLOGO@', {'ifincsname'})%
              else %
                tex.print('1')%
              end%
            }%
            \ifx\HOLOGO@ifincsname\@undefined 1\fi%
            \relax
          \expandafter\ltx@firstofone
        \else
          \endgroup
          \expandafter\ltx@gobble
        \fi
        {%
          \global\let\ifincsname\HOLOGO@ifincsname
        }%
      \HOLOGO@temp
    }%
  }{}%
}
%    \end{macrocode}
%    \begin{macrocode}
\ltx@IfUndefined{ifincsname}{%
  \catcode`$=14 %
}{%
  \catcode`$=9 %
}
%    \end{macrocode}
%
%    \begin{macro}{\hologo}
%    \begin{macrocode}
\def\hologo#1{%
$ \ifincsname
$   \ltx@ifundefined{HoLogoCs@\HOLOGO@Variant{#1}}{%
$     #1%
$   }{%
$     \csname HoLogoCs@\HOLOGO@Variant{#1}\endcsname\ltx@firstoftwo
$   }%
$ \else
    \HOLOGO@IfExists\texorpdfstring\texorpdfstring\ltx@firstoftwo
    {%
      \hologoRobust{#1}%
    }{%
      \ltx@ifundefined{HoLogoBkm@\HOLOGO@Variant{#1}}{%
        \ltx@ifundefined{HoLogo@#1}{?#1?}{#1}%
      }{%
        \csname HoLogoBkm@\HOLOGO@Variant{#1}\endcsname
        \ltx@firstoftwo
      }%
    }%
$ \fi
}
%    \end{macrocode}
%    \end{macro}
%    \begin{macro}{\Hologo}
%    \begin{macrocode}
\def\Hologo#1{%
$ \ifincsname
$   \ltx@ifundefined{HoLogoCs@\HOLOGO@Variant{#1}}{%
$     #1%
$   }{%
$     \csname HoLogoCs@\HOLOGO@Variant{#1}\endcsname\ltx@secondoftwo
$   }%
$ \else
    \HOLOGO@IfExists\texorpdfstring\texorpdfstring\ltx@firstoftwo
    {%
      \HologoRobust{#1}%
    }{%
      \ltx@ifundefined{HoLogoBkm@\HOLOGO@Variant{#1}}{%
        \ltx@ifundefined{HoLogo@#1}{?#1?}{#1}%
      }{%
        \csname HoLogoBkm@\HOLOGO@Variant{#1}\endcsname
        \ltx@secondoftwo
      }%
    }%
$ \fi
}
%    \end{macrocode}
%    \end{macro}
%
%    \begin{macro}{\hologoVariant}
%    \begin{macrocode}
\def\hologoVariant#1#2{%
  \ifx\relax#2\relax
    \hologo{#1}%
  \else
$   \ifincsname
$     \ltx@ifundefined{HoLogoCs@#1@#2}{%
$       #1%
$     }{%
$       \csname HoLogoCs@#1@#2\endcsname\ltx@firstoftwo
$     }%
$   \else
      \HOLOGO@IfExists\texorpdfstring\texorpdfstring\ltx@firstoftwo
      {%
        \hologoVariantRobust{#1}{#2}%
      }{%
        \ltx@ifundefined{HoLogoBkm@#1@#2}{%
          \ltx@ifundefined{HoLogo@#1}{?#1?}{#1}%
        }{%
          \csname HoLogoBkm@#1@#2\endcsname
          \ltx@firstoftwo
        }%
      }%
$   \fi
  \fi
}
%    \end{macrocode}
%    \end{macro}
%    \begin{macro}{\HologoVariant}
%    \begin{macrocode}
\def\HologoVariant#1#2{%
  \ifx\relax#2\relax
    \Hologo{#1}%
  \else
$   \ifincsname
$     \ltx@ifundefined{HoLogoCs@#1@#2}{%
$       #1%
$     }{%
$       \csname HoLogoCs@#1@#2\endcsname\ltx@secondoftwo
$     }%
$   \else
      \HOLOGO@IfExists\texorpdfstring\texorpdfstring\ltx@firstoftwo
      {%
        \HologoVariantRobust{#1}{#2}%
      }{%
        \ltx@ifundefined{HoLogoBkm@#1@#2}{%
          \ltx@ifundefined{HoLogo@#1}{?#1?}{#1}%
        }{%
          \csname HoLogoBkm@#1@#2\endcsname
          \ltx@secondoftwo
        }%
      }%
$   \fi
  \fi
}
%    \end{macrocode}
%    \end{macro}
%
%    \begin{macrocode}
\catcode`\$=3 %
%    \end{macrocode}
%
% \subsubsection{\cs{hologoRobust} and friends}
%
%    \begin{macro}{\hologoRobust}
%    \begin{macrocode}
\ltx@IfUndefined{protected}{%
  \ltx@IfUndefined{DeclareRobustCommand}{%
    \def\hologoRobust#1%
  }{%
    \DeclareRobustCommand*\hologoRobust[1]%
  }%
}{%
  \protected\def\hologoRobust#1%
}%
{%
  \edef\HOLOGO@name{#1}%
  \ltx@IfUndefined{HoLogo@\HOLOGO@Variant\HOLOGO@name}{%
    \@PackageError{hologo}{%
      Unknown logo `\HOLOGO@name'%
    }\@ehc
    ?\HOLOGO@name?%
  }{%
    \ltx@IfUndefined{ver@tex4ht.sty}{%
      \HoLogoFont@font\HOLOGO@name{general}{%
        \csname HoLogo@\HOLOGO@Variant\HOLOGO@name\endcsname
        \ltx@firstoftwo
      }%
    }{%
      \ltx@IfUndefined{HoLogoHtml@\HOLOGO@Variant\HOLOGO@name}{%
        \HOLOGO@name
      }{%
        \csname HoLogoHtml@\HOLOGO@Variant\HOLOGO@name\endcsname
        \ltx@firstoftwo
      }%
    }%
  }%
}
%    \end{macrocode}
%    \end{macro}
%    \begin{macro}{\HologoRobust}
%    \begin{macrocode}
\ltx@IfUndefined{protected}{%
  \ltx@IfUndefined{DeclareRobustCommand}{%
    \def\HologoRobust#1%
  }{%
    \DeclareRobustCommand*\HologoRobust[1]%
  }%
}{%
  \protected\def\HologoRobust#1%
}%
{%
  \edef\HOLOGO@name{#1}%
  \ltx@IfUndefined{HoLogo@\HOLOGO@Variant\HOLOGO@name}{%
    \@PackageError{hologo}{%
      Unknown logo `\HOLOGO@name'%
    }\@ehc
    ?\HOLOGO@name?%
  }{%
    \ltx@IfUndefined{ver@tex4ht.sty}{%
      \HoLogoFont@font\HOLOGO@name{general}{%
        \csname HoLogo@\HOLOGO@Variant\HOLOGO@name\endcsname
        \ltx@secondoftwo
      }%
    }{%
      \ltx@IfUndefined{HoLogoHtml@\HOLOGO@Variant\HOLOGO@name}{%
        \expandafter\HOLOGO@Uppercase\HOLOGO@name
      }{%
        \csname HoLogoHtml@\HOLOGO@Variant\HOLOGO@name\endcsname
        \ltx@secondoftwo
      }%
    }%
  }%
}
%    \end{macrocode}
%    \end{macro}
%    \begin{macro}{\hologoVariantRobust}
%    \begin{macrocode}
\ltx@IfUndefined{protected}{%
  \ltx@IfUndefined{DeclareRobustCommand}{%
    \def\hologoVariantRobust#1#2%
  }{%
    \DeclareRobustCommand*\hologoVariantRobust[2]%
  }%
}{%
  \protected\def\hologoVariantRobust#1#2%
}%
{%
  \begingroup
    \hologoLogoSetup{#1}{variant={#2}}%
    \hologoRobust{#1}%
  \endgroup
}
%    \end{macrocode}
%    \end{macro}
%    \begin{macro}{\HologoVariantRobust}
%    \begin{macrocode}
\ltx@IfUndefined{protected}{%
  \ltx@IfUndefined{DeclareRobustCommand}{%
    \def\HologoVariantRobust#1#2%
  }{%
    \DeclareRobustCommand*\HologoVariantRobust[2]%
  }%
}{%
  \protected\def\HologoVariantRobust#1#2%
}%
{%
  \begingroup
    \hologoLogoSetup{#1}{variant={#2}}%
    \HologoRobust{#1}%
  \endgroup
}
%    \end{macrocode}
%    \end{macro}
%
%    \begin{macro}{\hologorobust}
%    Macro \cs{hologorobust} is only defined for compatibility.
%    Its use is deprecated.
%    \begin{macrocode}
\def\hologorobust{\hologoRobust}
%    \end{macrocode}
%    \end{macro}
%
% \subsection{Helpers}
%
%    \begin{macro}{\HOLOGO@Uppercase}
%    Macro \cs{HOLOGO@Uppercase} is restricted to \cs{uppercase},
%    because \hologo{plainTeX} or \hologo{iniTeX} do not provide
%    \cs{MakeUppercase}.
%    \begin{macrocode}
\def\HOLOGO@Uppercase#1{\uppercase{#1}}
%    \end{macrocode}
%    \end{macro}
%
%    \begin{macro}{\HOLOGO@PdfdocUnicode}
%    \begin{macrocode}
\def\HOLOGO@PdfdocUnicode{%
  \ifx\ifHy@unicode\iftrue
    \expandafter\ltx@secondoftwo
  \else
    \expandafter\ltx@firstoftwo
  \fi
}
%    \end{macrocode}
%    \end{macro}
%
%    \begin{macro}{\HOLOGO@Math}
%    \begin{macrocode}
\def\HOLOGO@MathSetup{%
  \mathsurround0pt\relax
  \HOLOGO@IfExists\f@series{%
    \if b\expandafter\ltx@car\f@series x\@nil
      \csname boldmath\endcsname
   \fi
  }{}%
}
%    \end{macrocode}
%    \end{macro}
%
%    \begin{macro}{\HOLOGO@TempDimen}
%    \begin{macrocode}
\dimendef\HOLOGO@TempDimen=\ltx@zero
%    \end{macrocode}
%    \end{macro}
%    \begin{macro}{\HOLOGO@NegativeKerning}
%    \begin{macrocode}
\def\HOLOGO@NegativeKerning#1{%
  \begingroup
    \HOLOGO@TempDimen=0pt\relax
    \comma@parse@normalized{#1}{%
      \ifdim\HOLOGO@TempDimen=0pt %
        \expandafter\HOLOGO@@NegativeKerning\comma@entry
      \fi
      \ltx@gobble
    }%
    \ifdim\HOLOGO@TempDimen<0pt %
      \kern\HOLOGO@TempDimen
    \fi
  \endgroup
}
%    \end{macrocode}
%    \end{macro}
%    \begin{macro}{\HOLOGO@@NegativeKerning}
%    \begin{macrocode}
\def\HOLOGO@@NegativeKerning#1#2{%
  \setbox\ltx@zero\hbox{#1#2}%
  \HOLOGO@TempDimen=\wd\ltx@zero
  \setbox\ltx@zero\hbox{#1\kern0pt#2}%
  \advance\HOLOGO@TempDimen by -\wd\ltx@zero
}
%    \end{macrocode}
%    \end{macro}
%
%    \begin{macro}{\HOLOGO@SpaceFactor}
%    \begin{macrocode}
\def\HOLOGO@SpaceFactor{%
  \spacefactor1000 %
}
%    \end{macrocode}
%    \end{macro}
%
%    \begin{macro}{\HOLOGO@Span}
%    \begin{macrocode}
\def\HOLOGO@Span#1#2{%
  \HCode{<span class="HoLogo-#1">}%
  #2%
  \HCode{</span>}%
}
%    \end{macrocode}
%    \end{macro}
%
% \subsubsection{Text subscript}
%
%    \begin{macro}{\HOLOGO@SubScript}%
%    \begin{macrocode}
\def\HOLOGO@SubScript#1{%
  \ltx@IfUndefined{textsubscript}{%
    \ltx@IfUndefined{text}{%
      \ltx@mbox{%
        \mathsurround=0pt\relax
        $%
          _{%
            \ltx@IfUndefined{sf@size}{%
              \mathrm{#1}%
            }{%
              \mbox{%
                \fontsize\sf@size{0pt}\selectfont
                #1%
              }%
            }%
          }%
        $%
      }%
    }{%
      \ltx@mbox{%
        \mathsurround=0pt\relax
        $_{\text{#1}}$%
      }%
    }%
  }{%
    \textsubscript{#1}%
  }%
}
%    \end{macrocode}
%    \end{macro}
%
% \subsection{\hologo{TeX} and friends}
%
% \subsubsection{\hologo{TeX}}
%
%    \begin{macro}{\HoLogo@TeX}
%    Source: \hologo{LaTeX} kernel.
%    \begin{macrocode}
\def\HoLogo@TeX#1{%
  T\kern-.1667em\lower.5ex\hbox{E}\kern-.125emX\HOLOGO@SpaceFactor
}
%    \end{macrocode}
%    \end{macro}
%    \begin{macro}{\HoLogoHtml@TeX}
%    \begin{macrocode}
\def\HoLogoHtml@TeX#1{%
  \HoLogoCss@TeX
  \HOLOGO@Span{TeX}{%
    T%
    \HOLOGO@Span{e}{%
      E%
    }%
    X%
  }%
}
%    \end{macrocode}
%    \end{macro}
%    \begin{macro}{\HoLogoCss@TeX}
%    \begin{macrocode}
\def\HoLogoCss@TeX{%
  \Css{%
    span.HoLogo-TeX span.HoLogo-e{%
      position:relative;%
      top:.5ex;%
      margin-left:-.1667em;%
      margin-right:-.125em;%
    }%
  }%
  \Css{%
    a span.HoLogo-TeX span.HoLogo-e{%
      text-decoration:none;%
    }%
  }%
  \global\let\HoLogoCss@TeX\relax
}
%    \end{macrocode}
%    \end{macro}
%
% \subsubsection{\hologo{plainTeX}}
%
%    \begin{macro}{\HoLogo@plainTeX@space}
%    Source: ``The \hologo{TeX}book''
%    \begin{macrocode}
\def\HoLogo@plainTeX@space#1{%
  \HOLOGO@mbox{#1{p}{P}lain}\HOLOGO@space\hologo{TeX}%
}
%    \end{macrocode}
%    \end{macro}
%    \begin{macro}{\HoLogoCs@plainTeX@space}
%    \begin{macrocode}
\def\HoLogoCs@plainTeX@space#1{#1{p}{P}lain TeX}%
%    \end{macrocode}
%    \end{macro}
%    \begin{macro}{\HoLogoBkm@plainTeX@space}
%    \begin{macrocode}
\def\HoLogoBkm@plainTeX@space#1{%
  #1{p}{P}lain \hologo{TeX}%
}
%    \end{macrocode}
%    \end{macro}
%    \begin{macro}{\HoLogoHtml@plainTeX@space}
%    \begin{macrocode}
\def\HoLogoHtml@plainTeX@space#1{%
  #1{p}{P}lain \hologo{TeX}%
}
%    \end{macrocode}
%    \end{macro}
%
%    \begin{macro}{\HoLogo@plainTeX@hyphen}
%    \begin{macrocode}
\def\HoLogo@plainTeX@hyphen#1{%
  \HOLOGO@mbox{#1{p}{P}lain}\HOLOGO@hyphen\hologo{TeX}%
}
%    \end{macrocode}
%    \end{macro}
%    \begin{macro}{\HoLogoCs@plainTeX@hyphen}
%    \begin{macrocode}
\def\HoLogoCs@plainTeX@hyphen#1{#1{p}{P}lain-TeX}
%    \end{macrocode}
%    \end{macro}
%    \begin{macro}{\HoLogoBkm@plainTeX@hyphen}
%    \begin{macrocode}
\def\HoLogoBkm@plainTeX@hyphen#1{%
  #1{p}{P}lain-\hologo{TeX}%
}
%    \end{macrocode}
%    \end{macro}
%    \begin{macro}{\HoLogoHtml@plainTeX@hyphen}
%    \begin{macrocode}
\def\HoLogoHtml@plainTeX@hyphen#1{%
  #1{p}{P}lain-\hologo{TeX}%
}
%    \end{macrocode}
%    \end{macro}
%
%    \begin{macro}{\HoLogo@plainTeX@runtogether}
%    \begin{macrocode}
\def\HoLogo@plainTeX@runtogether#1{%
  \HOLOGO@mbox{#1{p}{P}lain\hologo{TeX}}%
}
%    \end{macrocode}
%    \end{macro}
%    \begin{macro}{\HoLogoCs@plainTeX@runtogether}
%    \begin{macrocode}
\def\HoLogoCs@plainTeX@runtogether#1{#1{p}{P}lainTeX}
%    \end{macrocode}
%    \end{macro}
%    \begin{macro}{\HoLogoBkm@plainTeX@runtogether}
%    \begin{macrocode}
\def\HoLogoBkm@plainTeX@runtogether#1{%
  #1{p}{P}lain\hologo{TeX}%
}
%    \end{macrocode}
%    \end{macro}
%    \begin{macro}{\HoLogoHtml@plainTeX@runtogether}
%    \begin{macrocode}
\def\HoLogoHtml@plainTeX@runtogether#1{%
  #1{p}{P}lain\hologo{TeX}%
}
%    \end{macrocode}
%    \end{macro}
%
%    \begin{macro}{\HoLogo@plainTeX}
%    \begin{macrocode}
\def\HoLogo@plainTeX{\HoLogo@plainTeX@space}
%    \end{macrocode}
%    \end{macro}
%    \begin{macro}{\HoLogoCs@plainTeX}
%    \begin{macrocode}
\def\HoLogoCs@plainTeX{\HoLogoCs@plainTeX@space}
%    \end{macrocode}
%    \end{macro}
%    \begin{macro}{\HoLogoBkm@plainTeX}
%    \begin{macrocode}
\def\HoLogoBkm@plainTeX{\HoLogoBkm@plainTeX@space}
%    \end{macrocode}
%    \end{macro}
%    \begin{macro}{\HoLogoHtml@plainTeX}
%    \begin{macrocode}
\def\HoLogoHtml@plainTeX{\HoLogoHtml@plainTeX@space}
%    \end{macrocode}
%    \end{macro}
%
% \subsubsection{\hologo{LaTeX}}
%
%    Source: \hologo{LaTeX} kernel.
%\begin{quote}
%\begin{verbatim}
%\DeclareRobustCommand{\LaTeX}{%
%  L%
%  \kern-.36em%
%  {%
%    \sbox\z@ T%
%    \vbox to\ht\z@{%
%      \hbox{%
%        \check@mathfonts
%        \fontsize\sf@size\z@
%        \math@fontsfalse
%        \selectfont
%        A%
%      }%
%      \vss
%    }%
%  }%
%  \kern-.15em%
%  \TeX
%}
%\end{verbatim}
%\end{quote}
%
%    \begin{macro}{\HoLogo@La}
%    \begin{macrocode}
\def\HoLogo@La#1{%
  L%
  \kern-.36em%
  \begingroup
    \setbox\ltx@zero\hbox{T}%
    \vbox to\ht\ltx@zero{%
      \hbox{%
        \ltx@ifundefined{check@mathfonts}{%
          \csname sevenrm\endcsname
        }{%
          \check@mathfonts
          \fontsize\sf@size{0pt}%
          \math@fontsfalse\selectfont
        }%
        A%
      }%
      \vss
    }%
  \endgroup
}
%    \end{macrocode}
%    \end{macro}
%
%    \begin{macro}{\HoLogo@LaTeX}
%    Source: \hologo{LaTeX} kernel.
%    \begin{macrocode}
\def\HoLogo@LaTeX#1{%
  \hologo{La}%
  \kern-.15em%
  \hologo{TeX}%
}
%    \end{macrocode}
%    \end{macro}
%    \begin{macro}{\HoLogoHtml@LaTeX}
%    \begin{macrocode}
\def\HoLogoHtml@LaTeX#1{%
  \HoLogoCss@LaTeX
  \HOLOGO@Span{LaTeX}{%
    L%
    \HOLOGO@Span{a}{%
      A%
    }%
    \hologo{TeX}%
  }%
}
%    \end{macrocode}
%    \end{macro}
%    \begin{macro}{\HoLogoCss@LaTeX}
%    \begin{macrocode}
\def\HoLogoCss@LaTeX{%
  \Css{%
    span.HoLogo-LaTeX span.HoLogo-a{%
      position:relative;%
      top:-.5ex;%
      margin-left:-.36em;%
      margin-right:-.15em;%
      font-size:85\%;%
    }%
  }%
  \global\let\HoLogoCss@LaTeX\relax
}
%    \end{macrocode}
%    \end{macro}
%
% \subsubsection{\hologo{(La)TeX}}
%
%    \begin{macro}{\HoLogo@LaTeXTeX}
%    The kerning around the parentheses is taken
%    from package \xpackage{dtklogos} \cite{dtklogos}.
%\begin{quote}
%\begin{verbatim}
%\DeclareRobustCommand{\LaTeXTeX}{%
%  (%
%  \kern-.15em%
%  L%
%  \kern-.36em%
%  {%
%    \sbox\z@ T%
%    \vbox to\ht0{%
%      \hbox{%
%        $\m@th$%
%        \csname S@\f@size\endcsname
%        \fontsize\sf@size\z@
%        \math@fontsfalse
%        \selectfont
%        A%
%      }%
%      \vss
%    }%
%  }%
%  \kern-.2em%
%  )%
%  \kern-.15em%
%  \TeX
%}
%\end{verbatim}
%\end{quote}
%    \begin{macrocode}
\def\HoLogo@LaTeXTeX#1{%
  (%
  \kern-.15em%
  \hologo{La}%
  \kern-.2em%
  )%
  \kern-.15em%
  \hologo{TeX}%
}
%    \end{macrocode}
%    \end{macro}
%    \begin{macro}{\HoLogoBkm@LaTeXTeX}
%    \begin{macrocode}
\def\HoLogoBkm@LaTeXTeX#1{(La)TeX}
%    \end{macrocode}
%    \end{macro}
%
%    \begin{macro}{\HoLogo@(La)TeX}
%    \begin{macrocode}
\expandafter
\let\csname HoLogo@(La)TeX\endcsname\HoLogo@LaTeXTeX
%    \end{macrocode}
%    \end{macro}
%    \begin{macro}{\HoLogoBkm@(La)TeX}
%    \begin{macrocode}
\expandafter
\let\csname HoLogoBkm@(La)TeX\endcsname\HoLogoBkm@LaTeXTeX
%    \end{macrocode}
%    \end{macro}
%    \begin{macro}{\HoLogoHtml@LaTeXTeX}
%    \begin{macrocode}
\def\HoLogoHtml@LaTeXTeX#1{%
  \HoLogoCss@LaTeXTeX
  \HOLOGO@Span{LaTeXTeX}{%
    (%
    \HOLOGO@Span{L}{L}%
    \HOLOGO@Span{a}{A}%
    \HOLOGO@Span{ParenRight}{)}%
    \hologo{TeX}%
  }%
}
%    \end{macrocode}
%    \end{macro}
%    \begin{macro}{\HoLogoHtml@(La)TeX}
%    Kerning after opening parentheses and before closing parentheses
%    is $-0.1$\,em. The original values $-0.15$\,em
%    looked too ugly for a serif font.
%    \begin{macrocode}
\expandafter
\let\csname HoLogoHtml@(La)TeX\endcsname\HoLogoHtml@LaTeXTeX
%    \end{macrocode}
%    \end{macro}
%    \begin{macro}{\HoLogoCss@LaTeXTeX}
%    \begin{macrocode}
\def\HoLogoCss@LaTeXTeX{%
  \Css{%
    span.HoLogo-LaTeXTeX span.HoLogo-L{%
      margin-left:-.1em;%
    }%
  }%
  \Css{%
    span.HoLogo-LaTeXTeX span.HoLogo-a{%
      position:relative;%
      top:-.5ex;%
      margin-left:-.36em;%
      margin-right:-.1em;%
      font-size:85\%;%
    }%
  }%
  \Css{%
    span.HoLogo-LaTeXTeX span.HoLogo-ParenRight{%
      margin-right:-.15em;%
    }%
  }%
  \global\let\HoLogoCss@LaTeXTeX\relax
}
%    \end{macrocode}
%    \end{macro}
%
% \subsubsection{\hologo{LaTeXe}}
%
%    \begin{macro}{\HoLogo@LaTeXe}
%    Source: \hologo{LaTeX} kernel
%    \begin{macrocode}
\def\HoLogo@LaTeXe#1{%
  \hologo{LaTeX}%
  \kern.15em%
  \hbox{%
    \HOLOGO@MathSetup
    2%
    $_{\textstyle\varepsilon}$%
  }%
}
%    \end{macrocode}
%    \end{macro}
%
%    \begin{macro}{\HoLogoCs@LaTeXe}
%    \begin{macrocode}
\ifnum64=`\^^^^0040\relax % test for big chars of LuaTeX/XeTeX
  \catcode`\$=9 %
  \catcode`\&=14 %
\else
  \catcode`\$=14 %
  \catcode`\&=9 %
\fi
\def\HoLogoCs@LaTeXe#1{%
  LaTeX2%
$ \string ^^^^0395%
& e%
}%
\catcode`\$=3 %
\catcode`\&=4 %
%    \end{macrocode}
%    \end{macro}
%
%    \begin{macro}{\HoLogoBkm@LaTeXe}
%    \begin{macrocode}
\def\HoLogoBkm@LaTeXe#1{%
  \hologo{LaTeX}%
  2%
  \HOLOGO@PdfdocUnicode{e}{\textepsilon}%
}
%    \end{macrocode}
%    \end{macro}
%
%    \begin{macro}{\HoLogoHtml@LaTeXe}
%    \begin{macrocode}
\def\HoLogoHtml@LaTeXe#1{%
  \HoLogoCss@LaTeXe
  \HOLOGO@Span{LaTeX2e}{%
    \hologo{LaTeX}%
    \HOLOGO@Span{2}{2}%
    \HOLOGO@Span{e}{%
      \HOLOGO@MathSetup
      \ensuremath{\textstyle\varepsilon}%
    }%
  }%
}
%    \end{macrocode}
%    \end{macro}
%    \begin{macro}{\HoLogoCss@LaTeXe}
%    \begin{macrocode}
\def\HoLogoCss@LaTeXe{%
  \Css{%
    span.HoLogo-LaTeX2e span.HoLogo-2{%
      padding-left:.15em;%
    }%
  }%
  \Css{%
    span.HoLogo-LaTeX2e span.HoLogo-e{%
      position:relative;%
      top:.35ex;%
      text-decoration:none;%
    }%
  }%
  \global\let\HoLogoCss@LaTeXe\relax
}
%    \end{macrocode}
%    \end{macro}
%
%    \begin{macro}{\HoLogo@LaTeX2e}
%    \begin{macrocode}
\expandafter
\let\csname HoLogo@LaTeX2e\endcsname\HoLogo@LaTeXe
%    \end{macrocode}
%    \end{macro}
%    \begin{macro}{\HoLogoCs@LaTeX2e}
%    \begin{macrocode}
\expandafter
\let\csname HoLogoCs@LaTeX2e\endcsname\HoLogoCs@LaTeXe
%    \end{macrocode}
%    \end{macro}
%    \begin{macro}{\HoLogoBkm@LaTeX2e}
%    \begin{macrocode}
\expandafter
\let\csname HoLogoBkm@LaTeX2e\endcsname\HoLogoBkm@LaTeXe
%    \end{macrocode}
%    \end{macro}
%    \begin{macro}{\HoLogoHtml@LaTeX2e}
%    \begin{macrocode}
\expandafter
\let\csname HoLogoHtml@LaTeX2e\endcsname\HoLogoHtml@LaTeXe
%    \end{macrocode}
%    \end{macro}
%
% \subsubsection{\hologo{LaTeX3}}
%
%    \begin{macro}{\HoLogo@LaTeX3}
%    Source: \hologo{LaTeX} kernel
%    \begin{macrocode}
\expandafter\def\csname HoLogo@LaTeX3\endcsname#1{%
  \hologo{LaTeX}%
  3%
}
%    \end{macrocode}
%    \end{macro}
%
%    \begin{macro}{\HoLogoBkm@LaTeX3}
%    \begin{macrocode}
\expandafter\def\csname HoLogoBkm@LaTeX3\endcsname#1{%
  \hologo{LaTeX}%
  3%
}
%    \end{macrocode}
%    \end{macro}
%    \begin{macro}{\HoLogoHtml@LaTeX3}
%    \begin{macrocode}
\expandafter
\let\csname HoLogoHtml@LaTeX3\expandafter\endcsname
\csname HoLogo@LaTeX3\endcsname
%    \end{macrocode}
%    \end{macro}
%
% \subsubsection{\hologo{LaTeXML}}
%
%    \begin{macro}{\HoLogo@LaTeXML}
%    \begin{macrocode}
\def\HoLogo@LaTeXML#1{%
  \HOLOGO@mbox{%
    \hologo{La}%
    \kern-.15em%
    T%
    \kern-.1667em%
    \lower.5ex\hbox{E}%
    \kern-.125em%
    \HoLogoFont@font{LaTeXML}{sc}{xml}%
  }%
}
%    \end{macrocode}
%    \end{macro}
%    \begin{macro}{\HoLogoHtml@pdfLaTeX}
%    \begin{macrocode}
\def\HoLogoHtml@LaTeXML#1{%
  \HOLOGO@Span{LaTeXML}{%
    \HoLogoCss@LaTeX
    \HoLogoCss@TeX
    \HOLOGO@Span{LaTeX}{%
      L%
      \HOLOGO@Span{a}{%
        A%
      }%
    }%
    \HOLOGO@Span{TeX}{%
      T%
      \HOLOGO@Span{e}{%
        E%
      }%
    }%
    \HCode{<span style="font-variant: small-caps;">}%
    xml%
    \HCode{</span>}%
  }%
}
%    \end{macrocode}
%    \end{macro}
%
% \subsubsection{\hologo{eTeX}}
%
%    \begin{macro}{\HoLogo@eTeX}
%    Source: package \xpackage{etex}
%    \begin{macrocode}
\def\HoLogo@eTeX#1{%
  \ltx@mbox{%
    \HOLOGO@MathSetup
    $\varepsilon$%
    -%
    \HOLOGO@NegativeKerning{-T,T-,To}%
    \hologo{TeX}%
  }%
}
%    \end{macrocode}
%    \end{macro}
%    \begin{macro}{\HoLogoCs@eTeX}
%    \begin{macrocode}
\ifnum64=`\^^^^0040\relax % test for big chars of LuaTeX/XeTeX
  \catcode`\$=9 %
  \catcode`\&=14 %
\else
  \catcode`\$=14 %
  \catcode`\&=9 %
\fi
\def\HoLogoCs@eTeX#1{%
$ #1{\string ^^^^0395}{\string ^^^^03b5}%
& #1{e}{E}%
  TeX%
}%
\catcode`\$=3 %
\catcode`\&=4 %
%    \end{macrocode}
%    \end{macro}
%    \begin{macro}{\HoLogoBkm@eTeX}
%    \begin{macrocode}
\def\HoLogoBkm@eTeX#1{%
  \HOLOGO@PdfdocUnicode{#1{e}{E}}{\textepsilon}%
  -%
  \hologo{TeX}%
}
%    \end{macrocode}
%    \end{macro}
%    \begin{macro}{\HoLogoHtml@eTeX}
%    \begin{macrocode}
\def\HoLogoHtml@eTeX#1{%
  \ltx@mbox{%
    \HOLOGO@MathSetup
    $\varepsilon$%
    -%
    \hologo{TeX}%
  }%
}
%    \end{macrocode}
%    \end{macro}
%
% \subsubsection{\hologo{iniTeX}}
%
%    \begin{macro}{\HoLogo@iniTeX}
%    \begin{macrocode}
\def\HoLogo@iniTeX#1{%
  \HOLOGO@mbox{%
    #1{i}{I}ni\hologo{TeX}%
  }%
}
%    \end{macrocode}
%    \end{macro}
%    \begin{macro}{\HoLogoCs@iniTeX}
%    \begin{macrocode}
\def\HoLogoCs@iniTeX#1{#1{i}{I}niTeX}
%    \end{macrocode}
%    \end{macro}
%    \begin{macro}{\HoLogoBkm@iniTeX}
%    \begin{macrocode}
\def\HoLogoBkm@iniTeX#1{%
  #1{i}{I}ni\hologo{TeX}%
}
%    \end{macrocode}
%    \end{macro}
%    \begin{macro}{\HoLogoHtml@iniTeX}
%    \begin{macrocode}
\let\HoLogoHtml@iniTeX\HoLogo@iniTeX
%    \end{macrocode}
%    \end{macro}
%
% \subsubsection{\hologo{virTeX}}
%
%    \begin{macro}{\HoLogo@virTeX}
%    \begin{macrocode}
\def\HoLogo@virTeX#1{%
  \HOLOGO@mbox{%
    #1{v}{V}ir\hologo{TeX}%
  }%
}
%    \end{macrocode}
%    \end{macro}
%    \begin{macro}{\HoLogoCs@virTeX}
%    \begin{macrocode}
\def\HoLogoCs@virTeX#1{#1{v}{V}irTeX}
%    \end{macrocode}
%    \end{macro}
%    \begin{macro}{\HoLogoBkm@virTeX}
%    \begin{macrocode}
\def\HoLogoBkm@virTeX#1{%
  #1{v}{V}ir\hologo{TeX}%
}
%    \end{macrocode}
%    \end{macro}
%    \begin{macro}{\HoLogoHtml@virTeX}
%    \begin{macrocode}
\let\HoLogoHtml@virTeX\HoLogo@virTeX
%    \end{macrocode}
%    \end{macro}
%
% \subsubsection{\hologo{SliTeX}}
%
% \paragraph{Definitions of the three variants.}
%
%    \begin{macro}{\HoLogo@SLiTeX@lift}
%    \begin{macrocode}
\def\HoLogo@SLiTeX@lift#1{%
  \HoLogoFont@font{SliTeX}{rm}{%
    S%
    \kern-.06em%
    L%
    \kern-.18em%
    \raise.32ex\hbox{\HoLogoFont@font{SliTeX}{sc}{i}}%
    \HOLOGO@discretionary
    \kern-.06em%
    \hologo{TeX}%
  }%
}
%    \end{macrocode}
%    \end{macro}
%    \begin{macro}{\HoLogoBkm@SLiTeX@lift}
%    \begin{macrocode}
\def\HoLogoBkm@SLiTeX@lift#1{SLiTeX}
%    \end{macrocode}
%    \end{macro}
%    \begin{macro}{\HoLogoHtml@SLiTeX@lift}
%    \begin{macrocode}
\def\HoLogoHtml@SLiTeX@lift#1{%
  \HoLogoCss@SLiTeX@lift
  \HOLOGO@Span{SLiTeX-lift}{%
    \HoLogoFont@font{SliTeX}{rm}{%
      S%
      \HOLOGO@Span{L}{L}%
      \HOLOGO@Span{i}{i}%
      \hologo{TeX}%
    }%
  }%
}
%    \end{macrocode}
%    \end{macro}
%    \begin{macro}{\HoLogoCss@SLiTeX@lift}
%    \begin{macrocode}
\def\HoLogoCss@SLiTeX@lift{%
  \Css{%
    span.HoLogo-SLiTeX-lift span.HoLogo-L{%
      margin-left:-.06em;%
      margin-right:-.18em;%
    }%
  }%
  \Css{%
    span.HoLogo-SLiTeX-lift span.HoLogo-i{%
      position:relative;%
      top:-.32ex;%
      margin-right:-.06em;%
      font-variant:small-caps;%
    }%
  }%
  \global\let\HoLogoCss@SLiTeX@lift\relax
}
%    \end{macrocode}
%    \end{macro}
%
%    \begin{macro}{\HoLogo@SliTeX@simple}
%    \begin{macrocode}
\def\HoLogo@SliTeX@simple#1{%
  \HoLogoFont@font{SliTeX}{rm}{%
    \ltx@mbox{%
      \HoLogoFont@font{SliTeX}{sc}{Sli}%
    }%
    \HOLOGO@discretionary
    \hologo{TeX}%
  }%
}
%    \end{macrocode}
%    \end{macro}
%    \begin{macro}{\HoLogoBkm@SliTeX@simple}
%    \begin{macrocode}
\def\HoLogoBkm@SliTeX@simple#1{SliTeX}
%    \end{macrocode}
%    \end{macro}
%    \begin{macro}{\HoLogoHtml@SliTeX@simple}
%    \begin{macrocode}
\let\HoLogoHtml@SliTeX@simple\HoLogo@SliTeX@simple
%    \end{macrocode}
%    \end{macro}
%
%    \begin{macro}{\HoLogo@SliTeX@narrow}
%    \begin{macrocode}
\def\HoLogo@SliTeX@narrow#1{%
  \HoLogoFont@font{SliTeX}{rm}{%
    \ltx@mbox{%
      S%
      \kern-.06em%
      \HoLogoFont@font{SliTeX}{sc}{%
        l%
        \kern-.035em%
        i%
      }%
    }%
    \HOLOGO@discretionary
    \kern-.06em%
    \hologo{TeX}%
  }%
}
%    \end{macrocode}
%    \end{macro}
%    \begin{macro}{\HoLogoBkm@SliTeX@narrow}
%    \begin{macrocode}
\def\HoLogoBkm@SliTeX@narrow#1{SliTeX}
%    \end{macrocode}
%    \end{macro}
%    \begin{macro}{\HoLogoHtml@SliTeX@narrow}
%    \begin{macrocode}
\def\HoLogoHtml@SliTeX@narrow#1{%
  \HoLogoCss@SliTeX@narrow
  \HOLOGO@Span{SliTeX-narrow}{%
    \HoLogoFont@font{SliTeX}{rm}{%
      S%
        \HOLOGO@Span{l}{l}%
        \HOLOGO@Span{i}{i}%
      \hologo{TeX}%
    }%
  }%
}
%    \end{macrocode}
%    \end{macro}
%    \begin{macro}{\HoLogoCss@SliTeX@narrow}
%    \begin{macrocode}
\def\HoLogoCss@SliTeX@narrow{%
  \Css{%
    span.HoLogo-SliTeX-narrow span.HoLogo-l{%
      margin-left:-.06em;%
      margin-right:-.035em;%
      font-variant:small-caps;%
    }%
  }%
  \Css{%
    span.HoLogo-SliTeX-narrow span.HoLogo-i{%
      margin-right:-.06em;%
      font-variant:small-caps;%
    }%
  }%
  \global\let\HoLogoCss@SliTeX@narrow\relax
}
%    \end{macrocode}
%    \end{macro}
%
% \paragraph{Macro set completion.}
%
%    \begin{macro}{\HoLogo@SLiTeX@simple}
%    \begin{macrocode}
\def\HoLogo@SLiTeX@simple{\HoLogo@SliTeX@simple}
%    \end{macrocode}
%    \end{macro}
%    \begin{macro}{\HoLogoBkm@SLiTeX@simple}
%    \begin{macrocode}
\def\HoLogoBkm@SLiTeX@simple{\HoLogoBkm@SliTeX@simple}
%    \end{macrocode}
%    \end{macro}
%    \begin{macro}{\HoLogoHtml@SLiTeX@simple}
%    \begin{macrocode}
\def\HoLogoHtml@SLiTeX@simple{\HoLogoHtml@SliTeX@simple}
%    \end{macrocode}
%    \end{macro}
%
%    \begin{macro}{\HoLogo@SLiTeX@narrow}
%    \begin{macrocode}
\def\HoLogo@SLiTeX@narrow{\HoLogo@SliTeX@narrow}
%    \end{macrocode}
%    \end{macro}
%    \begin{macro}{\HoLogoBkm@SLiTeX@narrow}
%    \begin{macrocode}
\def\HoLogoBkm@SLiTeX@narrow{\HoLogoBkm@SliTeX@narrow}
%    \end{macrocode}
%    \end{macro}
%    \begin{macro}{\HoLogoHtml@SLiTeX@narrow}
%    \begin{macrocode}
\def\HoLogoHtml@SLiTeX@narrow{\HoLogoHtml@SliTeX@narrow}
%    \end{macrocode}
%    \end{macro}
%
%    \begin{macro}{\HoLogo@SliTeX@lift}
%    \begin{macrocode}
\def\HoLogo@SliTeX@lift{\HoLogo@SLiTeX@lift}
%    \end{macrocode}
%    \end{macro}
%    \begin{macro}{\HoLogoBkm@SliTeX@lift}
%    \begin{macrocode}
\def\HoLogoBkm@SliTeX@lift{\HoLogoBkm@SLiTeX@lift}
%    \end{macrocode}
%    \end{macro}
%    \begin{macro}{\HoLogoHtml@SliTeX@lift}
%    \begin{macrocode}
\def\HoLogoHtml@SliTeX@lift{\HoLogoHtml@SLiTeX@lift}
%    \end{macrocode}
%    \end{macro}
%
% \paragraph{Defaults.}
%
%    \begin{macro}{\HoLogo@SLiTeX}
%    \begin{macrocode}
\def\HoLogo@SLiTeX{\HoLogo@SLiTeX@lift}
%    \end{macrocode}
%    \end{macro}
%    \begin{macro}{\HoLogoBkm@SLiTeX}
%    \begin{macrocode}
\def\HoLogoBkm@SLiTeX{\HoLogoBkm@SLiTeX@lift}
%    \end{macrocode}
%    \end{macro}
%    \begin{macro}{\HoLogoHtml@SLiTeX}
%    \begin{macrocode}
\def\HoLogoHtml@SLiTeX{\HoLogoHtml@SLiTeX@lift}
%    \end{macrocode}
%    \end{macro}
%
%    \begin{macro}{\HoLogo@SliTeX}
%    \begin{macrocode}
\def\HoLogo@SliTeX{\HoLogo@SliTeX@narrow}
%    \end{macrocode}
%    \end{macro}
%    \begin{macro}{\HoLogoBkm@SliTeX}
%    \begin{macrocode}
\def\HoLogoBkm@SliTeX{\HoLogoBkm@SliTeX@narrow}
%    \end{macrocode}
%    \end{macro}
%    \begin{macro}{\HoLogoHtml@SliTeX}
%    \begin{macrocode}
\def\HoLogoHtml@SliTeX{\HoLogoHtml@SliTeX@narrow}
%    \end{macrocode}
%    \end{macro}
%
% \subsubsection{\hologo{LuaTeX}}
%
%    \begin{macro}{\HoLogo@LuaTeX}
%    The kerning is an idea of Hans Hagen, see mailing list
%    `luatex at tug dot org' in March 2010.
%    \begin{macrocode}
\def\HoLogo@LuaTeX#1{%
  \HOLOGO@mbox{%
    Lua%
    \HOLOGO@NegativeKerning{aT,oT,To}%
    \hologo{TeX}%
  }%
}
%    \end{macrocode}
%    \end{macro}
%    \begin{macro}{\HoLogoHtml@LuaTeX}
%    \begin{macrocode}
\let\HoLogoHtml@LuaTeX\HoLogo@LuaTeX
%    \end{macrocode}
%    \end{macro}
%
% \subsubsection{\hologo{LuaLaTeX}}
%
%    \begin{macro}{\HoLogo@LuaLaTeX}
%    \begin{macrocode}
\def\HoLogo@LuaLaTeX#1{%
  \HOLOGO@mbox{%
    Lua%
    \hologo{LaTeX}%
  }%
}
%    \end{macrocode}
%    \end{macro}
%    \begin{macro}{\HoLogoHtml@LuaLaTeX}
%    \begin{macrocode}
\let\HoLogoHtml@LuaLaTeX\HoLogo@LuaLaTeX
%    \end{macrocode}
%    \end{macro}
%
% \subsubsection{\hologo{XeTeX}, \hologo{XeLaTeX}}
%
%    \begin{macro}{\HOLOGO@IfCharExists}
%    \begin{macrocode}
\ifluatex
  \ifnum\luatexversion<36 %
  \else
    \def\HOLOGO@IfCharExists#1{%
      \ifnum
        \directlua{%
           if luaotfload and luaotfload.aux then
             if luaotfload.aux.font_has_glyph(%
                    font.current(), \number#1) then % 	 
	       tex.print("1") % 	 
	     end % 	 
	   elseif font and font.fonts and font.current then %
            local f = font.fonts[font.current()]%
            if f.characters and f.characters[\number#1] then %
              tex.print("1")%
            end %
          end%
        }0=\ltx@zero
        \expandafter\ltx@secondoftwo
      \else
        \expandafter\ltx@firstoftwo
      \fi
    }%
  \fi
\fi
\ltx@IfUndefined{HOLOGO@IfCharExists}{%
  \def\HOLOGO@@IfCharExists#1{%
    \begingroup
      \tracinglostchars=\ltx@zero
      \setbox\ltx@zero=\hbox{%
        \kern7sp\char#1\relax
        \ifnum\lastkern>\ltx@zero
          \expandafter\aftergroup\csname iffalse\endcsname
        \else
          \expandafter\aftergroup\csname iftrue\endcsname
        \fi
      }%
      % \if{true|false} from \aftergroup
      \endgroup
      \expandafter\ltx@firstoftwo
    \else
      \endgroup
      \expandafter\ltx@secondoftwo
    \fi
  }%
  \ifxetex
    \ltx@IfUndefined{XeTeXfonttype}{}{%
      \ltx@IfUndefined{XeTeXcharglyph}{}{%
        \def\HOLOGO@IfCharExists#1{%
          \ifnum\XeTeXfonttype\font>\ltx@zero
            \expandafter\ltx@firstofthree
          \else
            \expandafter\ltx@gobble
          \fi
          {%
            \ifnum\XeTeXcharglyph#1>\ltx@zero
              \expandafter\ltx@firstoftwo
            \else
              \expandafter\ltx@secondoftwo
            \fi
          }%
          \HOLOGO@@IfCharExists{#1}%
        }%
      }%
    }%
  \fi
}{}
\ltx@ifundefined{HOLOGO@IfCharExists}{%
  \ifnum64=`\^^^^0040\relax % test for big chars of LuaTeX/XeTeX
    \let\HOLOGO@IfCharExists\HOLOGO@@IfCharExists
  \else
    \def\HOLOGO@IfCharExists#1{%
      \ifnum#1>255 %
        \expandafter\ltx@fourthoffour
      \fi
      \HOLOGO@@IfCharExists{#1}%
    }%
  \fi
}{}
%    \end{macrocode}
%    \end{macro}
%
%    \begin{macro}{\HoLogo@Xe}
%    Source: package \xpackage{dtklogos}
%    \begin{macrocode}
\def\HoLogo@Xe#1{%
  X%
  \kern-.1em\relax
  \HOLOGO@IfCharExists{"018E}{%
    \lower.5ex\hbox{\char"018E}%
  }{%
    \chardef\HOLOGO@choice=\ltx@zero
    \ifdim\fontdimen\ltx@one\font>0pt %
      \ltx@IfUndefined{rotatebox}{%
        \ltx@IfUndefined{pgftext}{%
          \ltx@IfUndefined{psscalebox}{%
            \ltx@IfUndefined{HOLOGO@ScaleBox@\hologoDriver}{%
            }{%
              \chardef\HOLOGO@choice=4 %
            }%
          }{%
            \chardef\HOLOGO@choice=3 %
          }%
        }{%
          \chardef\HOLOGO@choice=2 %
        }%
      }{%
        \chardef\HOLOGO@choice=1 %
      }%
      \ifcase\HOLOGO@choice
        \HOLOGO@WarningUnsupportedDriver{Xe}%
        e%
      \or % 1: \rotatebox
        \begingroup
          \setbox\ltx@zero\hbox{\rotatebox{180}{E}}%
          \ltx@LocDimenA=\dp\ltx@zero
          \advance\ltx@LocDimenA by -.5ex\relax
          \raise\ltx@LocDimenA\box\ltx@zero
        \endgroup
      \or % 2: \pgftext
        \lower.5ex\hbox{%
          \pgfpicture
            \pgftext[rotate=180]{E}%
          \endpgfpicture
        }%
      \or % 3: \psscalebox
        \begingroup
          \setbox\ltx@zero\hbox{\psscalebox{-1 -1}{E}}%
          \ltx@LocDimenA=\dp\ltx@zero
          \advance\ltx@LocDimenA by -.5ex\relax
          \raise\ltx@LocDimenA\box\ltx@zero
        \endgroup
      \or % 4: \HOLOGO@PointReflectBox
        \lower.5ex\hbox{\HOLOGO@PointReflectBox{E}}%
      \else
        \@PackageError{hologo}{Internal error (choice/it}\@ehc
      \fi
    \else
      \ltx@IfUndefined{reflectbox}{%
        \ltx@IfUndefined{pgftext}{%
          \ltx@IfUndefined{psscalebox}{%
            \ltx@IfUndefined{HOLOGO@ScaleBox@\hologoDriver}{%
            }{%
              \chardef\HOLOGO@choice=4 %
            }%
          }{%
            \chardef\HOLOGO@choice=3 %
          }%
        }{%
          \chardef\HOLOGO@choice=2 %
        }%
      }{%
        \chardef\HOLOGO@choice=1 %
      }%
      \ifcase\HOLOGO@choice
        \HOLOGO@WarningUnsupportedDriver{Xe}%
        e%
      \or % 1: reflectbox
        \lower.5ex\hbox{%
          \reflectbox{E}%
        }%
      \or % 2: \pgftext
        \lower.5ex\hbox{%
          \pgfpicture
            \pgftransformxscale{-1}%
            \pgftext{E}%
          \endpgfpicture
        }%
      \or % 3: \psscalebox
        \lower.5ex\hbox{%
          \psscalebox{-1 1}{E}%
        }%
      \or % 4: \HOLOGO@Reflectbox
        \lower.5ex\hbox{%
          \HOLOGO@ReflectBox{E}%
        }%
      \else
        \@PackageError{hologo}{Internal error (choice/up)}\@ehc
      \fi
    \fi
  }%
}
%    \end{macrocode}
%    \end{macro}
%    \begin{macro}{\HoLogoHtml@Xe}
%    \begin{macrocode}
\def\HoLogoHtml@Xe#1{%
  \HoLogoCss@Xe
  \HOLOGO@Span{Xe}{%
    X%
    \HOLOGO@Span{e}{%
      \HCode{&\ltx@hashchar x018e;}%
    }%
  }%
}
%    \end{macrocode}
%    \end{macro}
%    \begin{macro}{\HoLogoCss@Xe}
%    \begin{macrocode}
\def\HoLogoCss@Xe{%
  \Css{%
    span.HoLogo-Xe span.HoLogo-e{%
      position:relative;%
      top:.5ex;%
      left-margin:-.1em;%
    }%
  }%
  \global\let\HoLogoCss@Xe\relax
}
%    \end{macrocode}
%    \end{macro}
%
%    \begin{macro}{\HoLogo@XeTeX}
%    \begin{macrocode}
\def\HoLogo@XeTeX#1{%
  \hologo{Xe}%
  \kern-.15em\relax
  \hologo{TeX}%
}
%    \end{macrocode}
%    \end{macro}
%
%    \begin{macro}{\HoLogoHtml@XeTeX}
%    \begin{macrocode}
\def\HoLogoHtml@XeTeX#1{%
  \HoLogoCss@XeTeX
  \HOLOGO@Span{XeTeX}{%
    \hologo{Xe}%
    \hologo{TeX}%
  }%
}
%    \end{macrocode}
%    \end{macro}
%    \begin{macro}{\HoLogoCss@XeTeX}
%    \begin{macrocode}
\def\HoLogoCss@XeTeX{%
  \Css{%
    span.HoLogo-XeTeX span.HoLogo-TeX{%
      margin-left:-.15em;%
    }%
  }%
  \global\let\HoLogoCss@XeTeX\relax
}
%    \end{macrocode}
%    \end{macro}
%
%    \begin{macro}{\HoLogo@XeLaTeX}
%    \begin{macrocode}
\def\HoLogo@XeLaTeX#1{%
  \hologo{Xe}%
  \kern-.13em%
  \hologo{LaTeX}%
}
%    \end{macrocode}
%    \end{macro}
%    \begin{macro}{\HoLogoHtml@XeLaTeX}
%    \begin{macrocode}
\def\HoLogoHtml@XeLaTeX#1{%
  \HoLogoCss@XeLaTeX
  \HOLOGO@Span{XeLaTeX}{%
    \hologo{Xe}%
    \hologo{LaTeX}%
  }%
}
%    \end{macrocode}
%    \end{macro}
%    \begin{macro}{\HoLogoCss@XeLaTeX}
%    \begin{macrocode}
\def\HoLogoCss@XeLaTeX{%
  \Css{%
    span.HoLogo-XeLaTeX span.HoLogo-Xe{%
      margin-right:-.13em;%
    }%
  }%
  \global\let\HoLogoCss@XeLaTeX\relax
}
%    \end{macrocode}
%    \end{macro}
%
% \subsubsection{\hologo{pdfTeX}, \hologo{pdfLaTeX}}
%
%    \begin{macro}{\HoLogo@pdfTeX}
%    \begin{macrocode}
\def\HoLogo@pdfTeX#1{%
  \HOLOGO@mbox{%
    #1{p}{P}df\hologo{TeX}%
  }%
}
%    \end{macrocode}
%    \end{macro}
%    \begin{macro}{\HoLogoCs@pdfTeX}
%    \begin{macrocode}
\def\HoLogoCs@pdfTeX#1{#1{p}{P}dfTeX}
%    \end{macrocode}
%    \end{macro}
%    \begin{macro}{\HoLogoBkm@pdfTeX}
%    \begin{macrocode}
\def\HoLogoBkm@pdfTeX#1{%
  #1{p}{P}df\hologo{TeX}%
}
%    \end{macrocode}
%    \end{macro}
%    \begin{macro}{\HoLogoHtml@pdfTeX}
%    \begin{macrocode}
\let\HoLogoHtml@pdfTeX\HoLogo@pdfTeX
%    \end{macrocode}
%    \end{macro}
%
%    \begin{macro}{\HoLogo@pdfLaTeX}
%    \begin{macrocode}
\def\HoLogo@pdfLaTeX#1{%
  \HOLOGO@mbox{%
    #1{p}{P}df\hologo{LaTeX}%
  }%
}
%    \end{macrocode}
%    \end{macro}
%    \begin{macro}{\HoLogoCs@pdfLaTeX}
%    \begin{macrocode}
\def\HoLogoCs@pdfLaTeX#1{#1{p}{P}dfLaTeX}
%    \end{macrocode}
%    \end{macro}
%    \begin{macro}{\HoLogoBkm@pdfLaTeX}
%    \begin{macrocode}
\def\HoLogoBkm@pdfLaTeX#1{%
  #1{p}{P}df\hologo{LaTeX}%
}
%    \end{macrocode}
%    \end{macro}
%    \begin{macro}{\HoLogoHtml@pdfLaTeX}
%    \begin{macrocode}
\let\HoLogoHtml@pdfLaTeX\HoLogo@pdfLaTeX
%    \end{macrocode}
%    \end{macro}
%
% \subsubsection{\hologo{VTeX}}
%
%    \begin{macro}{\HoLogo@VTeX}
%    \begin{macrocode}
\def\HoLogo@VTeX#1{%
  \HOLOGO@mbox{%
    V\hologo{TeX}%
  }%
}
%    \end{macrocode}
%    \end{macro}
%    \begin{macro}{\HoLogoHtml@VTeX}
%    \begin{macrocode}
\let\HoLogoHtml@VTeX\HoLogo@VTeX
%    \end{macrocode}
%    \end{macro}
%
% \subsubsection{\hologo{AmS}, \dots}
%
%    Source: class \xclass{amsdtx}
%
%    \begin{macro}{\HoLogo@AmS}
%    \begin{macrocode}
\def\HoLogo@AmS#1{%
  \HoLogoFont@font{AmS}{sy}{%
    A%
    \kern-.1667em%
    \lower.5ex\hbox{M}%
    \kern-.125em%
    S%
  }%
}
%    \end{macrocode}
%    \end{macro}
%    \begin{macro}{\HoLogoBkm@AmS}
%    \begin{macrocode}
\def\HoLogoBkm@AmS#1{AmS}
%    \end{macrocode}
%    \end{macro}
%    \begin{macro}{\HoLogoHtml@AmS}
%    \begin{macrocode}
\def\HoLogoHtml@AmS#1{%
  \HoLogoCss@AmS
%  \HoLogoFont@font{AmS}{sy}{%
    \HOLOGO@Span{AmS}{%
      A%
      \HOLOGO@Span{M}{M}%
      S%
    }%
%   }%
}
%    \end{macrocode}
%    \end{macro}
%    \begin{macro}{\HoLogoCss@AmS}
%    \begin{macrocode}
\def\HoLogoCss@AmS{%
  \Css{%
    span.HoLogo-AmS span.HoLogo-M{%
      position:relative;%
      top:.5ex;%
      margin-left:-.1667em;%
      margin-right:-.125em;%
      text-decoration:none;%
    }%
  }%
  \global\let\HoLogoCss@AmS\relax
}
%    \end{macrocode}
%    \end{macro}
%
%    \begin{macro}{\HoLogo@AmSTeX}
%    \begin{macrocode}
\def\HoLogo@AmSTeX#1{%
  \hologo{AmS}%
  \HOLOGO@hyphen
  \hologo{TeX}%
}
%    \end{macrocode}
%    \end{macro}
%    \begin{macro}{\HoLogoBkm@AmSTeX}
%    \begin{macrocode}
\def\HoLogoBkm@AmSTeX#1{AmS-TeX}%
%    \end{macrocode}
%    \end{macro}
%    \begin{macro}{\HoLogoHtml@AmSTeX}
%    \begin{macrocode}
\let\HoLogoHtml@AmSTeX\HoLogo@AmSTeX
%    \end{macrocode}
%    \end{macro}
%
%    \begin{macro}{\HoLogo@AmSLaTeX}
%    \begin{macrocode}
\def\HoLogo@AmSLaTeX#1{%
  \hologo{AmS}%
  \HOLOGO@hyphen
  \hologo{LaTeX}%
}
%    \end{macrocode}
%    \end{macro}
%    \begin{macro}{\HoLogoBkm@AmSLaTeX}
%    \begin{macrocode}
\def\HoLogoBkm@AmSLaTeX#1{AmS-LaTeX}%
%    \end{macrocode}
%    \end{macro}
%    \begin{macro}{\HoLogoHtml@AmSLaTeX}
%    \begin{macrocode}
\let\HoLogoHtml@AmSLaTeX\HoLogo@AmSLaTeX
%    \end{macrocode}
%    \end{macro}
%
% \subsubsection{\hologo{BibTeX}}
%
%    \begin{macro}{\HoLogo@BibTeX@sc}
%    A definition of \hologo{BibTeX} is provided in
%    the documentation source for the manual of \hologo{BibTeX}
%    \cite{btxdoc}.
%\begin{quote}
%\begin{verbatim}
%\def\BibTeX{%
%  {%
%    \rm
%    B%
%    \kern-.05em%
%    {%
%      \sc
%      i%
%      \kern-.025em %
%      b%
%    }%
%    \kern-.08em
%    T%
%    \kern-.1667em%
%    \lower.7ex\hbox{E}%
%    \kern-.125em%
%    X%
%  }%
%}
%\end{verbatim}
%\end{quote}
%    \begin{macrocode}
\def\HoLogo@BibTeX@sc#1{%
  B%
  \kern-.05em%
  \HoLogoFont@font{BibTeX}{sc}{%
    i%
    \kern-.025em%
    b%
  }%
  \HOLOGO@discretionary
  \kern-.08em%
  \hologo{TeX}%
}
%    \end{macrocode}
%    \end{macro}
%    \begin{macro}{\HoLogoHtml@BibTeX@sc}
%    \begin{macrocode}
\def\HoLogoHtml@BibTeX@sc#1{%
  \HoLogoCss@BibTeX@sc
  \HOLOGO@Span{BibTeX-sc}{%
    B%
    \HOLOGO@Span{i}{i}%
    \HOLOGO@Span{b}{b}%
    \hologo{TeX}%
  }%
}
%    \end{macrocode}
%    \end{macro}
%    \begin{macro}{\HoLogoCss@BibTeX@sc}
%    \begin{macrocode}
\def\HoLogoCss@BibTeX@sc{%
  \Css{%
    span.HoLogo-BibTeX-sc span.HoLogo-i{%
      margin-left:-.05em;%
      margin-right:-.025em;%
      font-variant:small-caps;%
    }%
  }%
  \Css{%
    span.HoLogo-BibTeX-sc span.HoLogo-b{%
      margin-right:-.08em;%
      font-variant:small-caps;%
    }%
  }%
  \global\let\HoLogoCss@BibTeX@sc\relax
}
%    \end{macrocode}
%    \end{macro}
%
%    \begin{macro}{\HoLogo@BibTeX@sf}
%    Variant \xoption{sf} avoids trouble with unavailable
%    small caps fonts (e.g., bold versions of Computer Modern or
%    Latin Modern). The definition is taken from
%    package \xpackage{dtklogos} \cite{dtklogos}.
%\begin{quote}
%\begin{verbatim}
%\DeclareRobustCommand{\BibTeX}{%
%  B%
%  \kern-.05em%
%  \hbox{%
%    $\m@th$% %% force math size calculations
%    \csname S@\f@size\endcsname
%    \fontsize\sf@size\z@
%    \math@fontsfalse
%    \selectfont
%    I%
%    \kern-.025em%
%    B
%  }%
%  \kern-.08em%
%  \-%
%  \TeX
%}
%\end{verbatim}
%\end{quote}
%    \begin{macrocode}
\def\HoLogo@BibTeX@sf#1{%
  B%
  \kern-.05em%
  \HoLogoFont@font{BibTeX}{bibsf}{%
    I%
    \kern-.025em%
    B%
  }%
  \HOLOGO@discretionary
  \kern-.08em%
  \hologo{TeX}%
}
%    \end{macrocode}
%    \end{macro}
%    \begin{macro}{\HoLogoHtml@BibTeX@sf}
%    \begin{macrocode}
\def\HoLogoHtml@BibTeX@sf#1{%
  \HoLogoCss@BibTeX@sf
  \HOLOGO@Span{BibTeX-sf}{%
    B%
    \HoLogoFont@font{BibTeX}{bibsf}{%
      \HOLOGO@Span{i}{I}%
      B%
    }%
    \hologo{TeX}%
  }%
}
%    \end{macrocode}
%    \end{macro}
%    \begin{macro}{\HoLogoCss@BibTeX@sf}
%    \begin{macrocode}
\def\HoLogoCss@BibTeX@sf{%
  \Css{%
    span.HoLogo-BibTeX-sf span.HoLogo-i{%
      margin-left:-.05em;%
      margin-right:-.025em;%
    }%
  }%
  \Css{%
    span.HoLogo-BibTeX-sf span.HoLogo-TeX{%
      margin-left:-.08em;%
    }%
  }%
  \global\let\HoLogoCss@BibTeX@sf\relax
}
%    \end{macrocode}
%    \end{macro}
%
%    \begin{macro}{\HoLogo@BibTeX}
%    \begin{macrocode}
\def\HoLogo@BibTeX{\HoLogo@BibTeX@sf}
%    \end{macrocode}
%    \end{macro}
%    \begin{macro}{\HoLogoHtml@BibTeX}
%    \begin{macrocode}
\def\HoLogoHtml@BibTeX{\HoLogoHtml@BibTeX@sf}
%    \end{macrocode}
%    \end{macro}
%
% \subsubsection{\hologo{BibTeX8}}
%
%    \begin{macro}{\HoLogo@BibTeX8}
%    \begin{macrocode}
\expandafter\def\csname HoLogo@BibTeX8\endcsname#1{%
  \hologo{BibTeX}%
  8%
}
%    \end{macrocode}
%    \end{macro}
%
%    \begin{macro}{\HoLogoBkm@BibTeX8}
%    \begin{macrocode}
\expandafter\def\csname HoLogoBkm@BibTeX8\endcsname#1{%
  \hologo{BibTeX}%
  8%
}
%    \end{macrocode}
%    \end{macro}
%    \begin{macro}{\HoLogoHtml@BibTeX8}
%    \begin{macrocode}
\expandafter
\let\csname HoLogoHtml@BibTeX8\expandafter\endcsname
\csname HoLogo@BibTeX8\endcsname
%    \end{macrocode}
%    \end{macro}
%
% \subsubsection{\hologo{ConTeXt}}
%
%    \begin{macro}{\HoLogo@ConTeXt@simple}
%    \begin{macrocode}
\def\HoLogo@ConTeXt@simple#1{%
  \HOLOGO@mbox{Con}%
  \HOLOGO@discretionary
  \HOLOGO@mbox{\hologo{TeX}t}%
}
%    \end{macrocode}
%    \end{macro}
%    \begin{macro}{\HoLogoHtml@ConTeXt@simple}
%    \begin{macrocode}
\let\HoLogoHtml@ConTeXt@simple\HoLogo@ConTeXt@simple
%    \end{macrocode}
%    \end{macro}
%
%    \begin{macro}{\HoLogo@ConTeXt@narrow}
%    This definition of logo \hologo{ConTeXt} with variant \xoption{narrow}
%    comes from TUGboat's class \xclass{ltugboat} (version 2010/11/15 v2.8).
%    \begin{macrocode}
\def\HoLogo@ConTeXt@narrow#1{%
  \HOLOGO@mbox{C\kern-.0333emon}%
  \HOLOGO@discretionary
  \kern-.0667em%
  \HOLOGO@mbox{\hologo{TeX}\kern-.0333emt}%
}
%    \end{macrocode}
%    \end{macro}
%    \begin{macro}{\HoLogoHtml@ConTeXt@narrow}
%    \begin{macrocode}
\def\HoLogoHtml@ConTeXt@narrow#1{%
  \HoLogoCss@ConTeXt@narrow
  \HOLOGO@Span{ConTeXt-narrow}{%
    \HOLOGO@Span{C}{C}%
    on%
    \hologo{TeX}%
    t%
  }%
}
%    \end{macrocode}
%    \end{macro}
%    \begin{macro}{\HoLogoCss@ConTeXt@narrow}
%    \begin{macrocode}
\def\HoLogoCss@ConTeXt@narrow{%
  \Css{%
    span.HoLogo-ConTeXt-narrow span.HoLogo-C{%
      margin-left:-.0333em;%
    }%
  }%
  \Css{%
    span.HoLogo-ConTeXt-narrow span.HoLogo-TeX{%
      margin-left:-.0667em;%
      margin-right:-.0333em;%
    }%
  }%
  \global\let\HoLogoCss@ConTeXt@narrow\relax
}
%    \end{macrocode}
%    \end{macro}
%
%    \begin{macro}{\HoLogo@ConTeXt}
%    \begin{macrocode}
\def\HoLogo@ConTeXt{\HoLogo@ConTeXt@narrow}
%    \end{macrocode}
%    \end{macro}
%    \begin{macro}{\HoLogoHtml@ConTeXt}
%    \begin{macrocode}
\def\HoLogoHtml@ConTeXt{\HoLogoHtml@ConTeXt@narrow}
%    \end{macrocode}
%    \end{macro}
%
% \subsubsection{\hologo{emTeX}}
%
%    \begin{macro}{\HoLogo@emTeX}
%    \begin{macrocode}
\def\HoLogo@emTeX#1{%
  \HOLOGO@mbox{#1{e}{E}m}%
  \HOLOGO@discretionary
  \hologo{TeX}%
}
%    \end{macrocode}
%    \end{macro}
%    \begin{macro}{\HoLogoCs@emTeX}
%    \begin{macrocode}
\def\HoLogoCs@emTeX#1{#1{e}{E}mTeX}%
%    \end{macrocode}
%    \end{macro}
%    \begin{macro}{\HoLogoBkm@emTeX}
%    \begin{macrocode}
\def\HoLogoBkm@emTeX#1{%
  #1{e}{E}m\hologo{TeX}%
}
%    \end{macrocode}
%    \end{macro}
%    \begin{macro}{\HoLogoHtml@emTeX}
%    \begin{macrocode}
\let\HoLogoHtml@emTeX\HoLogo@emTeX
%    \end{macrocode}
%    \end{macro}
%
% \subsubsection{\hologo{ExTeX}}
%
%    \begin{macro}{\HoLogo@ExTeX}
%    The definition is taken from the FAQ of the
%    project \hologo{ExTeX}
%    \cite{ExTeX-FAQ}.
%\begin{quote}
%\begin{verbatim}
%\def\ExTeX{%
%  \textrm{% Logo always with serifs
%    \ensuremath{%
%      \textstyle
%      \varepsilon_{%
%        \kern-0.15em%
%        \mathcal{X}%
%      }%
%    }%
%    \kern-.15em%
%    \TeX
%  }%
%}
%\end{verbatim}
%\end{quote}
%    \begin{macrocode}
\def\HoLogo@ExTeX#1{%
  \HoLogoFont@font{ExTeX}{rm}{%
    \ltx@mbox{%
      \HOLOGO@MathSetup
      $%
        \textstyle
        \varepsilon_{%
          \kern-0.15em%
          \HoLogoFont@font{ExTeX}{sy}{X}%
        }%
      $%
    }%
    \HOLOGO@discretionary
    \kern-.15em%
    \hologo{TeX}%
  }%
}
%    \end{macrocode}
%    \end{macro}
%    \begin{macro}{\HoLogoHtml@ExTeX}
%    \begin{macrocode}
\def\HoLogoHtml@ExTeX#1{%
  \HoLogoCss@ExTeX
  \HoLogoFont@font{ExTeX}{rm}{%
    \HOLOGO@Span{ExTeX}{%
      \ltx@mbox{%
        \HOLOGO@MathSetup
        $\textstyle\varepsilon$%
        \HOLOGO@Span{X}{$\textstyle\chi$}%
        \hologo{TeX}%
      }%
    }%
  }%
}
%    \end{macrocode}
%    \end{macro}
%    \begin{macro}{\HoLogoBkm@ExTeX}
%    \begin{macrocode}
\def\HoLogoBkm@ExTeX#1{%
  \HOLOGO@PdfdocUnicode{#1{e}{E}x}{\textepsilon\textchi}%
  \hologo{TeX}%
}
%    \end{macrocode}
%    \end{macro}
%    \begin{macro}{\HoLogoCss@ExTeX}
%    \begin{macrocode}
\def\HoLogoCss@ExTeX{%
  \Css{%
    span.HoLogo-ExTeX{%
      font-family:serif;%
    }%
  }%
  \Css{%
    span.HoLogo-ExTeX span.HoLogo-TeX{%
      margin-left:-.15em;%
    }%
  }%
  \global\let\HoLogoCss@ExTeX\relax
}
%    \end{macrocode}
%    \end{macro}
%
% \subsubsection{\hologo{MiKTeX}}
%
%    \begin{macro}{\HoLogo@MiKTeX}
%    \begin{macrocode}
\def\HoLogo@MiKTeX#1{%
  \HOLOGO@mbox{MiK}%
  \HOLOGO@discretionary
  \hologo{TeX}%
}
%    \end{macrocode}
%    \end{macro}
%    \begin{macro}{\HoLogoHtml@MiKTeX}
%    \begin{macrocode}
\let\HoLogoHtml@MiKTeX\HoLogo@MiKTeX
%    \end{macrocode}
%    \end{macro}
%
% \subsubsection{\hologo{OzTeX} and friends}
%
%    Source: \hologo{OzTeX} FAQ \cite{OzTeX}:
%    \begin{quote}
%      |\def\OzTeX{O\kern-.03em z\kern-.15em\TeX}|\\
%      (There is no kerning in OzMF, OzMP and OzTtH.)
%    \end{quote}
%
%    \begin{macro}{\HoLogo@OzTeX}
%    \begin{macrocode}
\def\HoLogo@OzTeX#1{%
  O%
  \kern-.03em %
  z%
  \kern-.15em %
  \hologo{TeX}%
}
%    \end{macrocode}
%    \end{macro}
%    \begin{macro}{\HoLogoHtml@OzTeX}
%    \begin{macrocode}
\def\HoLogoHtml@OzTeX#1{%
  \HoLogoCss@OzTeX
  \HOLOGO@Span{OzTeX}{%
    O%
    \HOLOGO@Span{z}{z}%
    \hologo{TeX}%
  }%
}
%    \end{macrocode}
%    \end{macro}
%    \begin{macro}{\HoLogoCss@OzTeX}
%    \begin{macrocode}
\def\HoLogoCss@OzTeX{%
  \Css{%
    span.HoLogo-OzTeX span.HoLogo-z{%
      margin-left:-.03em;%
      margin-right:-.15em;%
    }%
  }%
  \global\let\HoLogoCss@OzTeX\relax
}
%    \end{macrocode}
%    \end{macro}
%
%    \begin{macro}{\HoLogo@OzMF}
%    \begin{macrocode}
\def\HoLogo@OzMF#1{%
  \HOLOGO@mbox{OzMF}%
}
%    \end{macrocode}
%    \end{macro}
%    \begin{macro}{\HoLogo@OzMP}
%    \begin{macrocode}
\def\HoLogo@OzMP#1{%
  \HOLOGO@mbox{OzMP}%
}
%    \end{macrocode}
%    \end{macro}
%    \begin{macro}{\HoLogo@OzTtH}
%    \begin{macrocode}
\def\HoLogo@OzTtH#1{%
  \HOLOGO@mbox{OzTtH}%
}
%    \end{macrocode}
%    \end{macro}
%
% \subsubsection{\hologo{PCTeX}}
%
%    \begin{macro}{\HoLogo@PCTeX}
%    \begin{macrocode}
\def\HoLogo@PCTeX#1{%
  \HOLOGO@mbox{PC}%
  \hologo{TeX}%
}
%    \end{macrocode}
%    \end{macro}
%    \begin{macro}{\HoLogoHtml@PCTeX}
%    \begin{macrocode}
\let\HoLogoHtml@PCTeX\HoLogo@PCTeX
%    \end{macrocode}
%    \end{macro}
%
% \subsubsection{\hologo{PiCTeX}}
%
%    The original definitions from \xfile{pictex.tex} \cite{PiCTeX}:
%\begin{quote}
%\begin{verbatim}
%\def\PiC{%
%  P%
%  \kern-.12em%
%  \lower.5ex\hbox{I}%
%  \kern-.075em%
%  C%
%}
%\def\PiCTeX{%
%  \PiC
%  \kern-.11em%
%  \TeX
%}
%\end{verbatim}
%\end{quote}
%
%    \begin{macro}{\HoLogo@PiC}
%    \begin{macrocode}
\def\HoLogo@PiC#1{%
  P%
  \kern-.12em%
  \lower.5ex\hbox{I}%
  \kern-.075em%
  C%
  \HOLOGO@SpaceFactor
}
%    \end{macrocode}
%    \end{macro}
%    \begin{macro}{\HoLogoHtml@PiC}
%    \begin{macrocode}
\def\HoLogoHtml@PiC#1{%
  \HoLogoCss@PiC
  \HOLOGO@Span{PiC}{%
    P%
    \HOLOGO@Span{i}{I}%
    C%
  }%
}
%    \end{macrocode}
%    \end{macro}
%    \begin{macro}{\HoLogoCss@PiC}
%    \begin{macrocode}
\def\HoLogoCss@PiC{%
  \Css{%
    span.HoLogo-PiC span.HoLogo-i{%
      position:relative;%
      top:.5ex;%
      margin-left:-.12em;%
      margin-right:-.075em;%
      text-decoration:none;%
    }%
  }%
  \global\let\HoLogoCss@PiC\relax
}
%    \end{macrocode}
%    \end{macro}
%
%    \begin{macro}{\HoLogo@PiCTeX}
%    \begin{macrocode}
\def\HoLogo@PiCTeX#1{%
  \hologo{PiC}%
  \HOLOGO@discretionary
  \kern-.11em%
  \hologo{TeX}%
}
%    \end{macrocode}
%    \end{macro}
%    \begin{macro}{\HoLogoHtml@PiCTeX}
%    \begin{macrocode}
\def\HoLogoHtml@PiCTeX#1{%
  \HoLogoCss@PiCTeX
  \HOLOGO@Span{PiCTeX}{%
    \hologo{PiC}%
    \hologo{TeX}%
  }%
}
%    \end{macrocode}
%    \end{macro}
%    \begin{macro}{\HoLogoCss@PiCTeX}
%    \begin{macrocode}
\def\HoLogoCss@PiCTeX{%
  \Css{%
    span.HoLogo-PiCTeX span.HoLogo-PiC{%
      margin-right:-.11em;%
    }%
  }%
  \global\let\HoLogoCss@PiCTeX\relax
}
%    \end{macrocode}
%    \end{macro}
%
% \subsubsection{\hologo{teTeX}}
%
%    \begin{macro}{\HoLogo@teTeX}
%    \begin{macrocode}
\def\HoLogo@teTeX#1{%
  \HOLOGO@mbox{#1{t}{T}e}%
  \HOLOGO@discretionary
  \hologo{TeX}%
}
%    \end{macrocode}
%    \end{macro}
%    \begin{macro}{\HoLogoCs@teTeX}
%    \begin{macrocode}
\def\HoLogoCs@teTeX#1{#1{t}{T}dfTeX}
%    \end{macrocode}
%    \end{macro}
%    \begin{macro}{\HoLogoBkm@teTeX}
%    \begin{macrocode}
\def\HoLogoBkm@teTeX#1{%
  #1{t}{T}e\hologo{TeX}%
}
%    \end{macrocode}
%    \end{macro}
%    \begin{macro}{\HoLogoHtml@teTeX}
%    \begin{macrocode}
\let\HoLogoHtml@teTeX\HoLogo@teTeX
%    \end{macrocode}
%    \end{macro}
%
% \subsubsection{\hologo{TeX4ht}}
%
%    \begin{macro}{\HoLogo@TeX4ht}
%    \begin{macrocode}
\expandafter\def\csname HoLogo@TeX4ht\endcsname#1{%
  \HOLOGO@mbox{\hologo{TeX}4ht}%
}
%    \end{macrocode}
%    \end{macro}
%    \begin{macro}{\HoLogoHtml@TeX4ht}
%    \begin{macrocode}
\expandafter
\let\csname HoLogoHtml@TeX4ht\expandafter\endcsname
\csname HoLogo@TeX4ht\endcsname
%    \end{macrocode}
%    \end{macro}
%
%
% \subsubsection{\hologo{SageTeX}}
%
%    \begin{macro}{\HoLogo@SageTeX}
%    \begin{macrocode}
\def\HoLogo@SageTeX#1{%
  \HOLOGO@mbox{Sage}%
  \HOLOGO@discretionary
  \HOLOGO@NegativeKerning{eT,oT,To}%
  \hologo{TeX}%
}
%    \end{macrocode}
%    \end{macro}
%    \begin{macro}{\HoLogoHtml@SageTeX}
%    \begin{macrocode}
\let\HoLogoHtml@SageTeX\HoLogo@SageTeX
%    \end{macrocode}
%    \end{macro}
%
% \subsection{\hologo{METAFONT} and friends}
%
%    \begin{macro}{\HoLogo@METAFONT}
%    \begin{macrocode}
\def\HoLogo@METAFONT#1{%
  \HoLogoFont@font{METAFONT}{logo}{%
    \HOLOGO@mbox{META}%
    \HOLOGO@discretionary
    \HOLOGO@mbox{FONT}%
  }%
}
%    \end{macrocode}
%    \end{macro}
%
%    \begin{macro}{\HoLogo@METAPOST}
%    \begin{macrocode}
\def\HoLogo@METAPOST#1{%
  \HoLogoFont@font{METAPOST}{logo}{%
    \HOLOGO@mbox{META}%
    \HOLOGO@discretionary
    \HOLOGO@mbox{POST}%
  }%
}
%    \end{macrocode}
%    \end{macro}
%
%    \begin{macro}{\HoLogo@MetaFun}
%    \begin{macrocode}
\def\HoLogo@MetaFun#1{%
  \HOLOGO@mbox{Meta}%
  \HOLOGO@discretionary
  \HOLOGO@mbox{Fun}%
}
%    \end{macrocode}
%    \end{macro}
%
%    \begin{macro}{\HoLogo@MetaPost}
%    \begin{macrocode}
\def\HoLogo@MetaPost#1{%
  \HOLOGO@mbox{Meta}%
  \HOLOGO@discretionary
  \HOLOGO@mbox{Post}%
}
%    \end{macrocode}
%    \end{macro}
%
% \subsection{Others}
%
% \subsubsection{\hologo{biber}}
%
%    \begin{macro}{\HoLogo@biber}
%    \begin{macrocode}
\def\HoLogo@biber#1{%
  \HOLOGO@mbox{#1{b}{B}i}%
  \HOLOGO@discretionary
  \HOLOGO@mbox{ber}%
}
%    \end{macrocode}
%    \end{macro}
%    \begin{macro}{\HoLogoCs@biber}
%    \begin{macrocode}
\def\HoLogoCs@biber#1{#1{b}{B}iber}
%    \end{macrocode}
%    \end{macro}
%    \begin{macro}{\HoLogoBkm@biber}
%    \begin{macrocode}
\def\HoLogoBkm@biber#1{%
  #1{b}{B}iber%
}
%    \end{macrocode}
%    \end{macro}
%    \begin{macro}{\HoLogoHtml@biber}
%    \begin{macrocode}
\let\HoLogoHtml@biber\HoLogo@biber
%    \end{macrocode}
%    \end{macro}
%
% \subsubsection{\hologo{KOMAScript}}
%
%    \begin{macro}{\HoLogo@KOMAScript}
%    The definition for \hologo{KOMAScript} is taken
%    from \hologo{KOMAScript} (\xfile{scrlogo.dtx}, reformatted) \cite{scrlogo}:
%\begin{quote}
%\begin{verbatim}
%\@ifundefined{KOMAScript}{%
%  \DeclareRobustCommand{\KOMAScript}{%
%    \textsf{%
%      K\kern.05em O\kern.05emM\kern.05em A%
%      \kern.1em-\kern.1em %
%      Script%
%    }%
%  }%
%}{}
%\end{verbatim}
%\end{quote}
%    \begin{macrocode}
\def\HoLogo@KOMAScript#1{%
  \HoLogoFont@font{KOMAScript}{sf}{%
    \HOLOGO@mbox{%
      K\kern.05em%
      O\kern.05em%
      M\kern.05em%
      A%
    }%
    \kern.1em%
    \HOLOGO@hyphen
    \kern.1em%
    \HOLOGO@mbox{Script}%
  }%
}
%    \end{macrocode}
%    \end{macro}
%    \begin{macro}{\HoLogoBkm@KOMAScript}
%    \begin{macrocode}
\def\HoLogoBkm@KOMAScript#1{%
  KOMA-Script%
}
%    \end{macrocode}
%    \end{macro}
%    \begin{macro}{\HoLogoHtml@KOMAScript}
%    \begin{macrocode}
\def\HoLogoHtml@KOMAScript#1{%
  \HoLogoCss@KOMAScript
  \HoLogoFont@font{KOMAScript}{sf}{%
    \HOLOGO@Span{KOMAScript}{%
      K%
      \HOLOGO@Span{O}{O}%
      M%
      \HOLOGO@Span{A}{A}%
      \HOLOGO@Span{hyphen}{-}%
      Script%
    }%
  }%
}
%    \end{macrocode}
%    \end{macro}
%    \begin{macro}{\HoLogoCss@KOMAScript}
%    \begin{macrocode}
\def\HoLogoCss@KOMAScript{%
  \Css{%
    span.HoLogo-KOMAScript{%
      font-family:sans-serif;%
    }%
  }%
  \Css{%
    span.HoLogo-KOMAScript span.HoLogo-O{%
      padding-left:.05em;%
      padding-right:.05em;%
    }%
  }%
  \Css{%
    span.HoLogo-KOMAScript span.HoLogo-A{%
      padding-left:.05em;%
    }%
  }%
  \Css{%
    span.HoLogo-KOMAScript span.HoLogo-hyphen{%
      padding-left:.1em;%
      padding-right:.1em;%
    }%
  }%
  \global\let\HoLogoCss@KOMAScript\relax
}
%    \end{macrocode}
%    \end{macro}
%
% \subsubsection{\hologo{LyX}}
%
%    \begin{macro}{\HoLogo@LyX}
%    The definition is taken from the documentation source files
%    of \hologo{LyX}, \xfile{Intro.lyx} \cite{LyX}:
%\begin{quote}
%\begin{verbatim}
%\def\LyX{%
%  \texorpdfstring{%
%    L\kern-.1667em\lower.25em\hbox{Y}\kern-.125emX\@%
%  }{%
%    LyX%
%  }%
%}
%\end{verbatim}
%\end{quote}
%    \begin{macrocode}
\def\HoLogo@LyX#1{%
  L%
  \kern-.1667em%
  \lower.25em\hbox{Y}%
  \kern-.125em%
  X%
  \HOLOGO@SpaceFactor
}
%    \end{macrocode}
%    \end{macro}
%    \begin{macro}{\HoLogoHtml@LyX}
%    \begin{macrocode}
\def\HoLogoHtml@LyX#1{%
  \HoLogoCss@LyX
  \HOLOGO@Span{LyX}{%
    L%
    \HOLOGO@Span{y}{Y}%
    X%
  }%
}
%    \end{macrocode}
%    \end{macro}
%    \begin{macro}{\HoLogoCss@LyX}
%    \begin{macrocode}
\def\HoLogoCss@LyX{%
  \Css{%
    span.HoLogo-LyX span.HoLogo-y{%
      position:relative;%
      top:.25em;%
      margin-left:-.1667em;%
      margin-right:-.125em;%
      text-decoration:none;%
    }%
  }%
  \global\let\HoLogoCss@LyX\relax
}
%    \end{macrocode}
%    \end{macro}
%
% \subsubsection{\hologo{NTS}}
%
%    \begin{macro}{\HoLogo@NTS}
%    Definition for \hologo{NTS} can be found in
%    package \xpackage{etex\textunderscore man} for the \hologo{eTeX} manual \cite{etexman}
%    and in package \xpackage{dtklogos} \cite{dtklogos}:
%\begin{quote}
%\begin{verbatim}
%\def\NTS{%
%  \leavevmode
%  \hbox{%
%    $%
%      \cal N%
%      \kern-0.35em%
%      \lower0.5ex\hbox{$\cal T$}%
%      \kern-0.2em%
%      S%
%    $%
%  }%
%}
%\end{verbatim}
%\end{quote}
%    \begin{macrocode}
\def\HoLogo@NTS#1{%
  \HoLogoFont@font{NTS}{sy}{%
    N\/%
    \kern-.35em%
    \lower.5ex\hbox{T\/}%
    \kern-.2em%
    S\/%
  }%
  \HOLOGO@SpaceFactor
}
%    \end{macrocode}
%    \end{macro}
%
% \subsubsection{\Hologo{TTH} (\hologo{TeX} to HTML translator)}
%
%    Source: \url{http://hutchinson.belmont.ma.us/tth/}
%    In the HTML source the second `T' is printed as subscript.
%\begin{quote}
%\begin{verbatim}
%T<sub>T</sub>H
%\end{verbatim}
%\end{quote}
%    \begin{macro}{\HoLogo@TTH}
%    \begin{macrocode}
\def\HoLogo@TTH#1{%
  \ltx@mbox{%
    T\HOLOGO@SubScript{T}H%
  }%
  \HOLOGO@SpaceFactor
}
%    \end{macrocode}
%    \end{macro}
%
%    \begin{macro}{\HoLogoHtml@TTH}
%    \begin{macrocode}
\def\HoLogoHtml@TTH#1{%
  T\HCode{<sub>}T\HCode{</sub>}H%
}
%    \end{macrocode}
%    \end{macro}
%
% \subsubsection{\Hologo{HanTheThanh}}
%
%    Partial source: Package \xpackage{dtklogos}.
%    The double accent is U+1EBF (latin small letter e with circumflex
%    and acute).
%    \begin{macro}{\HoLogo@HanTheThanh}
%    \begin{macrocode}
\def\HoLogo@HanTheThanh#1{%
  \ltx@mbox{H\`an}%
  \HOLOGO@space
  \ltx@mbox{%
    Th%
    \HOLOGO@IfCharExists{"1EBF}{%
      \char"1EBF\relax
    }{%
      \^e\hbox to 0pt{\hss\raise .5ex\hbox{\'{}}}%
    }%
  }%
  \HOLOGO@space
  \ltx@mbox{Th\`anh}%
}
%    \end{macrocode}
%    \end{macro}
%    \begin{macro}{\HoLogoBkm@HanTheThanh}
%    \begin{macrocode}
\def\HoLogoBkm@HanTheThanh#1{%
  H\`an %
  Th\HOLOGO@PdfdocUnicode{\^e}{\9036\277} %
  Th\`anh%
}
%    \end{macrocode}
%    \end{macro}
%    \begin{macro}{\HoLogoHtml@HanTheThanh}
%    \begin{macrocode}
\def\HoLogoHtml@HanTheThanh#1{%
  H\`an %
  Th\HCode{&\ltx@hashchar x1ebf;} %
  Th\`anh%
}
%    \end{macrocode}
%    \end{macro}
%
% \subsection{Driver detection}
%
%    \begin{macrocode}
\HOLOGO@IfExists\InputIfFileExists{%
  \InputIfFileExists{hologo.cfg}{}{}%
}{%
  \ltx@IfUndefined{pdf@filesize}{%
    \def\HOLOGO@InputIfExists{%
      \openin\HOLOGO@temp=hologo.cfg\relax
      \ifeof\HOLOGO@temp
        \closein\HOLOGO@temp
      \else
        \closein\HOLOGO@temp
        \begingroup
          \def\x{LaTeX2e}%
        \expandafter\endgroup
        \ifx\fmtname\x
          \input{hologo.cfg}%
        \else
          \input hologo.cfg\relax
        \fi
      \fi
    }%
    \ltx@IfUndefined{newread}{%
      \chardef\HOLOGO@temp=15 %
      \def\HOLOGO@CheckRead{%
        \ifeof\HOLOGO@temp
          \HOLOGO@InputIfExists
        \else
          \ifcase\HOLOGO@temp
            \@PackageWarningNoLine{hologo}{%
              Configuration file ignored, because\MessageBreak
              a free read register could not be found%
            }%
          \else
            \begingroup
              \count\ltx@cclv=\HOLOGO@temp
              \advance\ltx@cclv by \ltx@minusone
              \edef\x{\endgroup
                \chardef\noexpand\HOLOGO@temp=\the\count\ltx@cclv
                \relax
              }%
            \x
          \fi
        \fi
      }%
    }{%
      \csname newread\endcsname\HOLOGO@temp
      \HOLOGO@InputIfExists
    }%
  }{%
    \edef\HOLOGO@temp{\pdf@filesize{hologo.cfg}}%
    \ifx\HOLOGO@temp\ltx@empty
    \else
      \ifnum\HOLOGO@temp>0 %
        \begingroup
          \def\x{LaTeX2e}%
        \expandafter\endgroup
        \ifx\fmtname\x
          \input{hologo.cfg}%
        \else
          \input hologo.cfg\relax
        \fi
      \else
        \@PackageInfoNoLine{hologo}{%
          Empty configuration file `hologo.cfg' ignored%
        }%
      \fi
    \fi
  }%
}
%    \end{macrocode}
%
%    \begin{macrocode}
\def\HOLOGO@temp#1#2{%
  \kv@define@key{HoLogoDriver}{#1}[]{%
    \begingroup
      \def\HOLOGO@temp{##1}%
      \ltx@onelevel@sanitize\HOLOGO@temp
      \ifx\HOLOGO@temp\ltx@empty
      \else
        \@PackageError{hologo}{%
          Value (\HOLOGO@temp) not permitted for option `#1'%
        }%
        \@ehc
      \fi
    \endgroup
    \def\hologoDriver{#2}%
  }%
}%
\def\HOLOGO@@temp#1#2{%
  \ifx\kv@value\relax
    \HOLOGO@temp{#1}{#1}%
  \else
    \HOLOGO@temp{#1}{#2}%
  \fi
}%
\kv@parse@normalized{%
  pdftex,%
  luatex=pdftex,%
  dvipdfm,%
  dvipdfmx=dvipdfm,%
  dvips,%
  dvipsone=dvips,%
  xdvi=dvips,%
  xetex,%
  vtex,%
}\HOLOGO@@temp
%    \end{macrocode}
%
%    \begin{macrocode}
\kv@define@key{HoLogoDriver}{driverfallback}{%
  \def\HOLOGO@DriverFallback{#1}%
}
%    \end{macrocode}
%
%    \begin{macro}{\HOLOGO@DriverFallback}
%    \begin{macrocode}
\def\HOLOGO@DriverFallback{dvips}
%    \end{macrocode}
%    \end{macro}
%
%    \begin{macro}{\hologoDriverSetup}
%    \begin{macrocode}
\def\hologoDriverSetup{%
  \let\hologoDriver\ltx@undefined
  \HOLOGO@DriverSetup
}
%    \end{macrocode}
%    \end{macro}
%
%    \begin{macro}{\HOLOGO@DriverSetup}
%    \begin{macrocode}
\def\HOLOGO@DriverSetup#1{%
  \kvsetkeys{HoLogoDriver}{#1}%
  \HOLOGO@CheckDriver
  \ltx@ifundefined{hologoDriver}{%
    \begingroup
    \edef\x{\endgroup
      \noexpand\kvsetkeys{HoLogoDriver}{\HOLOGO@DriverFallback}%
    }\x
  }{}%
  \@PackageInfoNoLine{hologo}{Using driver `\hologoDriver'}%
}
%    \end{macrocode}
%    \end{macro}
%
%    \begin{macro}{\HOLOGO@CheckDriver}
%    \begin{macrocode}
\def\HOLOGO@CheckDriver{%
  \ifpdf
    \def\hologoDriver{pdftex}%
    \let\HOLOGO@pdfliteral\pdfliteral
    \ifluatex
      \ifx\pdfextension\@undefined\else
        \protected\def\pdfliteral{\pdfextension literal}%
        \let\HOLOGO@pdfliteral\pdfliteral
      \fi
      \ltx@IfUndefined{HOLOGO@pdfliteral}{%
        \ifnum\luatexversion<36 %
        \else
          \begingroup
            \let\HOLOGO@temp\endgroup
            \ifcase0%
                \directlua{%
                  if tex.enableprimitives then %
                    tex.enableprimitives('HOLOGO@', {'pdfliteral'})%
                  else %
                    tex.print('1')%
                  end%
                }%
                \ifx\HOLOGO@pdfliteral\@undefined 1\fi%
                \relax%
              \endgroup
              \let\HOLOGO@temp\relax
              \global\let\HOLOGO@pdfliteral\HOLOGO@pdfliteral
            \fi%
          \HOLOGO@temp
        \fi
      }{}%
    \fi
    \ltx@IfUndefined{HOLOGO@pdfliteral}{%
      \@PackageWarningNoLine{hologo}{%
        Cannot find \string\pdfliteral
      }%
    }{}%
  \else
    \ifxetex
      \def\hologoDriver{xetex}%
    \else
      \ifvtex
        \def\hologoDriver{vtex}%
      \fi
    \fi
  \fi
}
%    \end{macrocode}
%    \end{macro}
%
%    \begin{macro}{\HOLOGO@WarningUnsupportedDriver}
%    \begin{macrocode}
\def\HOLOGO@WarningUnsupportedDriver#1{%
  \@PackageWarningNoLine{hologo}{%
    Logo `#1' needs driver specific macros,\MessageBreak
    but driver `\hologoDriver' is not supported.\MessageBreak
    Use a different driver or\MessageBreak
    load package `graphics' or `pgf'%
  }%
}
%    \end{macrocode}
%    \end{macro}
%
% \subsubsection{Reflect box macros}
%
%    Skip driver part if not needed.
%    \begin{macrocode}
\ltx@IfUndefined{reflectbox}{}{%
  \ltx@IfUndefined{rotatebox}{}{%
    \HOLOGO@AtEnd
  }%
}
\ltx@IfUndefined{pgftext}{}{%
  \HOLOGO@AtEnd
}
\ltx@IfUndefined{psscalebox}{}{%
  \HOLOGO@AtEnd
}
%    \end{macrocode}
%
%    \begin{macrocode}
\def\HOLOGO@temp{LaTeX2e}
\ifx\fmtname\HOLOGO@temp
  \RequirePackage{kvoptions}[2011/06/30]%
  \ProcessKeyvalOptions{HoLogoDriver}%
\fi
\HOLOGO@DriverSetup{}
%    \end{macrocode}
%
%    \begin{macro}{\HOLOGO@ReflectBox}
%    \begin{macrocode}
\def\HOLOGO@ReflectBox#1{%
  \begingroup
    \setbox\ltx@zero\hbox{\begingroup#1\endgroup}%
    \setbox\ltx@two\hbox{%
      \kern\wd\ltx@zero
      \csname HOLOGO@ScaleBox@\hologoDriver\endcsname{-1}{1}{%
        \hbox to 0pt{\copy\ltx@zero\hss}%
      }%
    }%
    \wd\ltx@two=\wd\ltx@zero
    \box\ltx@two
  \endgroup
}
%    \end{macrocode}
%    \end{macro}
%
%    \begin{macro}{\HOLOGO@PointReflectBox}
%    \begin{macrocode}
\def\HOLOGO@PointReflectBox#1{%
  \begingroup
    \setbox\ltx@zero\hbox{\begingroup#1\endgroup}%
    \setbox\ltx@two\hbox{%
      \kern\wd\ltx@zero
      \raise\ht\ltx@zero\hbox{%
        \csname HOLOGO@ScaleBox@\hologoDriver\endcsname{-1}{-1}{%
          \hbox to 0pt{\copy\ltx@zero\hss}%
        }%
      }%
    }%
    \wd\ltx@two=\wd\ltx@zero
    \box\ltx@two
  \endgroup
}
%    \end{macrocode}
%    \end{macro}
%
%    We must define all variants because of dynamic driver setup.
%    \begin{macrocode}
\def\HOLOGO@temp#1#2{#2}
%    \end{macrocode}
%
%    \begin{macro}{\HOLOGO@ScaleBox@pdftex}
%    \begin{macrocode}
\HOLOGO@temp{pdftex}{%
  \def\HOLOGO@ScaleBox@pdftex#1#2#3{%
    \HOLOGO@pdfliteral{%
      q #1 0 0 #2 0 0 cm%
    }%
    #3%
    \HOLOGO@pdfliteral{%
      Q%
    }%
  }%
}
%    \end{macrocode}
%    \end{macro}
%    \begin{macro}{\HOLOGO@ScaleBox@dvips}
%    \begin{macrocode}
\HOLOGO@temp{dvips}{%
  \def\HOLOGO@ScaleBox@dvips#1#2#3{%
    \special{ps:%
      gsave %
      currentpoint %
      currentpoint translate %
      #1 #2 scale %
      neg exch neg exch translate%
    }%
    #3%
    \special{ps:%
      currentpoint %
      grestore %
      moveto%
    }%
  }%
}
%    \end{macrocode}
%    \end{macro}
%    \begin{macro}{\HOLOGO@ScaleBox@dvipdfm}
%    \begin{macrocode}
\HOLOGO@temp{dvipdfm}{%
  \let\HOLOGO@ScaleBox@dvipdfm\HOLOGO@ScaleBox@dvips
}
%    \end{macrocode}
%    \end{macro}
%    Since \hologo{XeTeX} v0.6.
%    \begin{macro}{\HOLOGO@ScaleBox@xetex}
%    \begin{macrocode}
\HOLOGO@temp{xetex}{%
  \def\HOLOGO@ScaleBox@xetex#1#2#3{%
    \special{x:gsave}%
    \special{x:scale #1 #2}%
    #3%
    \special{x:grestore}%
  }%
}
%    \end{macrocode}
%    \end{macro}
%    \begin{macro}{\HOLOGO@ScaleBox@vtex}
%    \begin{macrocode}
\HOLOGO@temp{vtex}{%
  \def\HOLOGO@ScaleBox@vtex#1#2#3{%
    \special{r(#1,0,0,#2,0,0}%
    #3%
    \special{r)}%
  }%
}
%    \end{macrocode}
%    \end{macro}
%
%    \begin{macrocode}
\HOLOGO@AtEnd%
%</package>
%    \end{macrocode}
%
% \section{Test}
%
% \subsection{Catcode checks for loading}
%
%    \begin{macrocode}
%<*test1>
%    \end{macrocode}
%    \begin{macrocode}
\catcode`\{=1 %
\catcode`\}=2 %
\catcode`\#=6 %
\catcode`\@=11 %
\expandafter\ifx\csname count@\endcsname\relax
  \countdef\count@=255 %
\fi
\expandafter\ifx\csname @gobble\endcsname\relax
  \long\def\@gobble#1{}%
\fi
\expandafter\ifx\csname @firstofone\endcsname\relax
  \long\def\@firstofone#1{#1}%
\fi
\expandafter\ifx\csname loop\endcsname\relax
  \expandafter\@firstofone
\else
  \expandafter\@gobble
\fi
{%
  \def\loop#1\repeat{%
    \def\body{#1}%
    \iterate
  }%
  \def\iterate{%
    \body
      \let\next\iterate
    \else
      \let\next\relax
    \fi
    \next
  }%
  \let\repeat=\fi
}%
\def\RestoreCatcodes{}
\count@=0 %
\loop
  \edef\RestoreCatcodes{%
    \RestoreCatcodes
    \catcode\the\count@=\the\catcode\count@\relax
  }%
\ifnum\count@<255 %
  \advance\count@ 1 %
\repeat

\def\RangeCatcodeInvalid#1#2{%
  \count@=#1\relax
  \loop
    \catcode\count@=15 %
  \ifnum\count@<#2\relax
    \advance\count@ 1 %
  \repeat
}
\def\RangeCatcodeCheck#1#2#3{%
  \count@=#1\relax
  \loop
    \ifnum#3=\catcode\count@
    \else
      \errmessage{%
        Character \the\count@\space
        with wrong catcode \the\catcode\count@\space
        instead of \number#3%
      }%
    \fi
  \ifnum\count@<#2\relax
    \advance\count@ 1 %
  \repeat
}
\def\space{ }
\expandafter\ifx\csname LoadCommand\endcsname\relax
  \def\LoadCommand{\input hologo.sty\relax}%
\fi
\def\Test{%
  \RangeCatcodeInvalid{0}{47}%
  \RangeCatcodeInvalid{58}{64}%
  \RangeCatcodeInvalid{91}{96}%
  \RangeCatcodeInvalid{123}{255}%
  \catcode`\@=12 %
  \catcode`\\=0 %
  \catcode`\%=14 %
  \LoadCommand
  \RangeCatcodeCheck{0}{36}{15}%
  \RangeCatcodeCheck{37}{37}{14}%
  \RangeCatcodeCheck{38}{47}{15}%
  \RangeCatcodeCheck{48}{57}{12}%
  \RangeCatcodeCheck{58}{63}{15}%
  \RangeCatcodeCheck{64}{64}{12}%
  \RangeCatcodeCheck{65}{90}{11}%
  \RangeCatcodeCheck{91}{91}{15}%
  \RangeCatcodeCheck{92}{92}{0}%
  \RangeCatcodeCheck{93}{96}{15}%
  \RangeCatcodeCheck{97}{122}{11}%
  \RangeCatcodeCheck{123}{255}{15}%
  \RestoreCatcodes
}
\Test
\csname @@end\endcsname
\end
%    \end{macrocode}
%    \begin{macrocode}
%</test1>
%    \end{macrocode}
%
% \subsection{Spacefactor}
%
%    The space factor must be 1000 after a logo. If it is greater 1000
%    then the following space is a space after a sentence closing point.
%    If the space factor is smaller 1000 then an immediate following
%    dot is interpreted as abbreviation, not sentence closing point.
%
%    \begin{macrocode}
%<*test-spacefactor>
\NeedsTeXFormat{LaTeX2e}
\documentclass{article}
\usepackage{hologo}[2016/05/12]
\usepackage{kvsetkeys}
\usepackage{qstest}
\IncludeTests{*}
\LogTests{log}{*}{*}
\begin{document}
\begin{qstest}{spacefactor}{spacefactor}
\newcommand*{\Test}[1]{%
  \sbox0{%
    \hologo{#1}%
    \Expect*{1000 (#1)}*{\the\spacefactor\space(#1)}%
  }%
}%
\makeatletter
\def\TestList{}
\def\hologoEntry#1#2#3{%
  \edef\TestList{%
    \ifx\TestList\@empty
    \else
      \TestList,%
    \fi
    #1%
    \ifx\\#2\\%
    \else
      ={variant=#2}%
    \fi
  }%
}
\hologoList
\expandafter\kv@parse@normalized\expandafter{%
  \TestList
}{%
  \begingroup
    \let\@logo=\kv@key
    \ifx\kv@value\relax
    \else
      \expandafter\hologoLogoSetup\expandafter\@logo\expandafter{%
        \kv@value
      }%
    \fi
    \Test\@logo
  \endgroup
  \@gobbletwo
}
\end{qstest}
\end{document}
%</test-spacefactor>
%    \end{macrocode}
%
% \subsection{Complete list}
%
%    \begin{macrocode}
%<*test-list>
\NeedsTeXFormat{LaTeX2e}
\documentclass[12pt,a4paper]{article}
\usepackage{hologo}[2016/05/12]
\usepackage[T1]{fontenc}
\usepackage{lmodern}
\usepackage{parskip}
\usepackage[unicode]{hyperref}[2011/09/28]
\usepackage{bookmark}[2011/09/19]
\bookmarksetup{%
  numbered,%
  open,%
  openlevel=2,%
}
\renewcommand*{\contentsname}{List of logos}
\begin{document}
\tableofcontents
\def\TestFont#1#2#3#4#5#6{%
  \begingroup
    \usefont{#3}{#4}{#5}{#6}%
    \HologoVariant{#1}{#2}/\hologoVariant{#1}{#2}%
    \quad
    \begingroup\scriptsize\hologoVariant{#1}{#2}\endgroup
    \quad
  \endgroup
  (#3/#4/#5/#6)%
  \par
}
\makeatletter
\def\hologoEntry#1#2#3{%
  \section{%
    \HologoVariant{#1}{#2}/\hologoVariant{#1}{#2} %
    {[#1\ifx\\#2\\\else\space(#2)\fi]}% hash-ok
  }% braces around [] because of bug in tex4ht
  \begingroup
    \hypersetup{unicode=false}%
    \bookmark[%
      dest=\@currentHref,%
      rellevel=1,%
      keeplevel,%
    ]{%
      \HologoVariant{#1}{#2}/\hologoVariant{#1}{#2} %
      (PDFDocEncoding)%
    }%
  \endgroup
  \TestFont{#1}{#2}{OT1}{cmr}{m}{n}%
  \TestFont{#1}{#2}{OT1}{cmss}{m}{n}%
  \TestFont{#1}{#2}{OT1}{cmr}{b}{n}%
  \TestFont{#1}{#2}{OT1}{cmr}{m}{it}%
  \TestFont{#1}{#2}{OT1}{cmtt}{m}{n}%
  \TestFont{#1}{#2}{T1}{lmr}{m}{n}%
  \TestFont{#1}{#2}{T1}{lmss}{m}{n}%
  \TestFont{#1}{#2}{T1}{lmr}{b}{n}%
  \TestFont{#1}{#2}{T1}{lmr}{m}{it}%
  \TestFont{#1}{#2}{T1}{lmtt}{m}{n}%
  \TestFont{#1}{#2}{T1}{lmvtt}{m}{n}%
  \TestFont{#1}{#2}{T1}{qtm}{m}{n}%
  \TestFont{#1}{#2}{T1}{qhv}{m}{n}%
  \TestFont{#1}{#2}{T1}{qtm}{b}{n}%
  \TestFont{#1}{#2}{T1}{qtm}{m}{it}%
  \TestFont{#1}{#2}{T1}{qcr}{m}{n}%
  \newpage
}
\makeatother
\hologoList
\end{document}
%</test-list>
%    \end{macrocode}
%
% \section{Installation}
%
% \subsection{Download}
%
% \paragraph{Package.} This package is available on
% CTAN\footnote{\url{ftp://ftp.ctan.org/tex-archive/}}:
% \begin{description}
% \item[\CTAN{macros/latex/contrib/oberdiek/hologo.dtx}] The source file.
% \item[\CTAN{macros/latex/contrib/oberdiek/hologo.pdf}] Documentation.
% \end{description}
%
%
% \paragraph{Bundle.} All the packages of the bundle `oberdiek'
% are also available in a TDS compliant ZIP archive. There
% the packages are already unpacked and the documentation files
% are generated. The files and directories obey the TDS standard.
% \begin{description}
% \item[\CTAN{install/macros/latex/contrib/oberdiek.tds.zip}]
% \end{description}
% \emph{TDS} refers to the standard ``A Directory Structure
% for \TeX\ Files'' (\CTAN{tds/tds.pdf}). Directories
% with \xfile{texmf} in their name are usually organized this way.
%
% \subsection{Bundle installation}
%
% \paragraph{Unpacking.} Unpack the \xfile{oberdiek.tds.zip} in the
% TDS tree (also known as \xfile{texmf} tree) of your choice.
% Example (linux):
% \begin{quote}
%   |unzip oberdiek.tds.zip -d ~/texmf|
% \end{quote}
%
% \paragraph{Script installation.}
% Check the directory \xfile{TDS:scripts/oberdiek/} for
% scripts that need further installation steps.
% Package \xpackage{attachfile2} comes with the Perl script
% \xfile{pdfatfi.pl} that should be installed in such a way
% that it can be called as \texttt{pdfatfi}.
% Example (linux):
% \begin{quote}
%   |chmod +x scripts/oberdiek/pdfatfi.pl|\\
%   |cp scripts/oberdiek/pdfatfi.pl /usr/local/bin/|
% \end{quote}
%
% \subsection{Package installation}
%
% \paragraph{Unpacking.} The \xfile{.dtx} file is a self-extracting
% \docstrip\ archive. The files are extracted by running the
% \xfile{.dtx} through \plainTeX:
% \begin{quote}
%   \verb|tex hologo.dtx|
% \end{quote}
%
% \paragraph{TDS.} Now the different files must be moved into
% the different directories in your installation TDS tree
% (also known as \xfile{texmf} tree):
% \begin{quote}
% \def\t{^^A
% \begin{tabular}{@{}>{\ttfamily}l@{ $\rightarrow$ }>{\ttfamily}l@{}}
%   hologo.sty & tex/generic/oberdiek/hologo.sty\\
%   hologo.pdf & doc/latex/oberdiek/hologo.pdf\\
%   example/hologo-example.tex & doc/latex/oberdiek/example/hologo-example.tex\\
%   test/hologo-test1.tex & doc/latex/oberdiek/test/hologo-test1.tex\\
%   test/hologo-test-spacefactor.tex & doc/latex/oberdiek/test/hologo-test-spacefactor.tex\\
%   test/hologo-test-list.tex & doc/latex/oberdiek/test/hologo-test-list.tex\\
%   hologo.dtx & source/latex/oberdiek/hologo.dtx\\
% \end{tabular}^^A
% }^^A
% \sbox0{\t}^^A
% \ifdim\wd0>\linewidth
%   \begingroup
%     \advance\linewidth by\leftmargin
%     \advance\linewidth by\rightmargin
%   \edef\x{\endgroup
%     \def\noexpand\lw{\the\linewidth}^^A
%   }\x
%   \def\lwbox{^^A
%     \leavevmode
%     \hbox to \linewidth{^^A
%       \kern-\leftmargin\relax
%       \hss
%       \usebox0
%       \hss
%       \kern-\rightmargin\relax
%     }^^A
%   }^^A
%   \ifdim\wd0>\lw
%     \sbox0{\small\t}^^A
%     \ifdim\wd0>\linewidth
%       \ifdim\wd0>\lw
%         \sbox0{\footnotesize\t}^^A
%         \ifdim\wd0>\linewidth
%           \ifdim\wd0>\lw
%             \sbox0{\scriptsize\t}^^A
%             \ifdim\wd0>\linewidth
%               \ifdim\wd0>\lw
%                 \sbox0{\tiny\t}^^A
%                 \ifdim\wd0>\linewidth
%                   \lwbox
%                 \else
%                   \usebox0
%                 \fi
%               \else
%                 \lwbox
%               \fi
%             \else
%               \usebox0
%             \fi
%           \else
%             \lwbox
%           \fi
%         \else
%           \usebox0
%         \fi
%       \else
%         \lwbox
%       \fi
%     \else
%       \usebox0
%     \fi
%   \else
%     \lwbox
%   \fi
% \else
%   \usebox0
% \fi
% \end{quote}
% If you have a \xfile{docstrip.cfg} that configures and enables \docstrip's
% TDS installing feature, then some files can already be in the right
% place, see the documentation of \docstrip.
%
% \subsection{Refresh file name databases}
%
% If your \TeX~distribution
% (\teTeX, \mikTeX, \dots) relies on file name databases, you must refresh
% these. For example, \teTeX\ users run \verb|texhash| or
% \verb|mktexlsr|.
%
% \subsection{Some details for the interested}
%
% \paragraph{Attached source.}
%
% The PDF documentation on CTAN also includes the
% \xfile{.dtx} source file. It can be extracted by
% AcrobatReader 6 or higher. Another option is \textsf{pdftk},
% e.g. unpack the file into the current directory:
% \begin{quote}
%   \verb|pdftk hologo.pdf unpack_files output .|
% \end{quote}
%
% \paragraph{Unpacking with \LaTeX.}
% The \xfile{.dtx} chooses its action depending on the format:
% \begin{description}
% \item[\plainTeX:] Run \docstrip\ and extract the files.
% \item[\LaTeX:] Generate the documentation.
% \end{description}
% If you insist on using \LaTeX\ for \docstrip\ (really,
% \docstrip\ does not need \LaTeX), then inform the autodetect routine
% about your intention:
% \begin{quote}
%   \verb|latex \let\install=y\input{hologo.dtx}|
% \end{quote}
% Do not forget to quote the argument according to the demands
% of your shell.
%
% \paragraph{Generating the documentation.}
% You can use both the \xfile{.dtx} or the \xfile{.drv} to generate
% the documentation. The process can be configured by the
% configuration file \xfile{ltxdoc.cfg}. For instance, put this
% line into this file, if you want to have A4 as paper format:
% \begin{quote}
%   \verb|\PassOptionsToClass{a4paper}{article}|
% \end{quote}
% An example follows how to generate the
% documentation with pdf\LaTeX:
% \begin{quote}
%\begin{verbatim}
%pdflatex hologo.dtx
%makeindex -s gind.ist hologo.idx
%pdflatex hologo.dtx
%makeindex -s gind.ist hologo.idx
%pdflatex hologo.dtx
%\end{verbatim}
% \end{quote}
%
% \section{Catalogue}
%
% The following XML file can be used as source for the
% \href{http://mirror.ctan.org/help/Catalogue/catalogue.html}{\TeX\ Catalogue}.
% The elements \texttt{caption} and \texttt{description} are imported
% from the original XML file from the Catalogue.
% The name of the XML file in the Catalogue is \xfile{hologo.xml}.
%    \begin{macrocode}
%<*catalogue>
<?xml version='1.0' encoding='us-ascii'?>
<!DOCTYPE entry SYSTEM 'catalogue.dtd'>
<entry datestamp='$Date$' modifier='$Author$' id='hologo'>
  <name>hologo</name>
  <caption>A collection of logos with bookmark support.</caption>
  <authorref id='auth:oberdiek'/>
  <copyright owner='Heiko Oberdiek' year='2010-2012'/>
  <license type='lppl1.3'/>
  <version number='1.10'/>
  <description>
    The package defines a single command <tt>\hologo</tt>, whose
    argument is the usual case-confused ASCII version of the logo.
    The command is bookmark-enabled, so that every logo becomes
    available in bookmarks without further work.
    <p/>
    The package is part of the <xref refid='oberdiek'>oberdiek</xref>
    bundle.
  </description>
  <documentation details='Package documentation'
      href='ctan:/macros/latex/contrib/oberdiek/hologo.pdf'/>
  <ctan file='true' path='/macros/latex/contrib/oberdiek/hologo.dtx'/>
  <miktex location='oberdiek'/>
  <texlive location='oberdiek'/>
  <install path='/macros/latex/contrib/oberdiek/oberdiek.tds.zip'/>
</entry>
%</catalogue>
%    \end{macrocode}
%
% \begin{thebibliography}{9}
% \raggedright
%
% \bibitem{btxdoc}
% Oren Patashnik,
% \textit{\hologo{BibTeX}ing},
% 1988-02-08.\\
% \CTAN{biblio/bibtex/base/}
%
% \bibitem{dtklogos}
% Gerd Neugebauer, DANTE,
% \textit{Package \xpackage{dtklogos}},
% 2011-04-25.\\
% \CTAN{usergrps/dante/dtk/dtklogos.sty}
%
% \bibitem{etexman}
% The \hologo{NTS} Team,
% \textit{The \hologo{eTeX} manual},
% 1998-02.\\
% \CTAN{systems/e-tex/v2/doc/}
%
% \bibitem{ExTeX-FAQ}
% The \hologo{ExTeX} group,
% \textit{\hologo{ExTeX}: FAQ -- How is \hologo{ExTeX} typeset?},
% 2007-04-14.\\
% \url{http://www.extex.org/documentation/faq.html}
%
% \bibitem{LyX}
% %@MISC{ LyX,
% %  title = {{LyX 2.0.0 -- The Document Processor [Computer software and manual]}},
% %  author = {{The LyX Team}},
% %  howpublished = {Internet: http://www.lyx.org},
% %  year = {2011-05-08},
% %  note = {Retrieved May 10, 2011, from http://www.lyx.org},
% %  url = {http://www.lyx.org/}
% %}
% The \hologo{LyX} Team,
% \textit{\hologo{LyX} -- The Document Processor},
% 2011-05-08.\\
% \url{http://www.lyx.org/}
%
% \bibitem{OzTeX}
% Andrew Trevorrow,
% \hologo{OzTeX} FAQ: What is the correct way to typeset ``\hologo{OzTeX}''?,
% 2011-09-15 (visited).
% \url{http://www.trevorrow.com/oztex/ozfaq.html#oztex-logo}
%
% \bibitem{PiCTeX}
% Michael Wichura,
% \textit{The \hologo{PiCTeX} macro package},
% 1987-09-21.
% \CTAN{graphics/pictex/}
%
% \bibitem{scrlogo}
% Markus Kohm,
% \textit{\hologo{KOMAScript} Datei \xfile{scrlogo.dtx}},
% 2009-01-30.\\
% \CTAN{install/macros/latex/contrib/komascript.tds.zip}
%
% \end{thebibliography}
%
% \begin{History}
%   \begin{Version}{2010/04/08 v1.0}
%   \item
%     The first version.
%   \end{Version}
%   \begin{Version}{2010/04/16 v1.1}
%   \item
%     \cs{Hologo} added for support of logos at start of a sentence.
%   \item
%     \cs{hologoSetup} and \cs{hologoLogoSetup} added.
%   \item
%     Options \xoption{break}, \xoption{hyphenbreak}, \xoption{spacebreak}
%     added.
%   \item
%     Variant support added by option \xoption{variant}.
%   \end{Version}
%   \begin{Version}{2010/04/24 v1.2}
%   \item
%     \hologo{LaTeX3} added.
%   \item
%     \hologo{VTeX} added.
%   \end{Version}
%   \begin{Version}{2010/11/21 v1.3}
%   \item
%     \hologo{iniTeX}, \hologo{virTeX} added.
%   \end{Version}
%   \begin{Version}{2011/03/25 v1.4}
%   \item
%     \hologo{ConTeXt} with variants added.
%   \item
%     Option \xoption{discretionarybreak} added as refinement for
%     option \xoption{break}.
%   \end{Version}
%   \begin{Version}{2011/04/21 v1.5}
%   \item
%     Wrong TDS directory for test files fixed.
%   \end{Version}
%   \begin{Version}{2011/10/01 v1.6}
%   \item
%     Support for package \xpackage{tex4ht} added.
%   \item
%     Support for \cs{csname} added if \cs{ifincsname} is available.
%   \item
%     New logos:
%     \hologo{(La)TeX},
%     \hologo{biber},
%     \hologo{BibTeX} (\xoption{sc}, \xoption{sf}),
%     \hologo{emTeX},
%     \hologo{ExTeX},
%     \hologo{KOMAScript},
%     \hologo{La},
%     \hologo{LyX},
%     \hologo{MiKTeX},
%     \hologo{NTS},
%     \hologo{OzMF},
%     \hologo{OzMP},
%     \hologo{OzTeX},
%     \hologo{OzTtH},
%     \hologo{PCTeX},
%     \hologo{PiC},
%     \hologo{PiCTeX},
%     \hologo{METAFONT},
%     \hologo{MetaFun},
%     \hologo{METAPOST},
%     \hologo{MetaPost},
%     \hologo{SLiTeX} (\xoption{lift}, \xoption{narrow}, \xoption{simple}),
%     \hologo{SliTeX} (\xoption{narrow}, \xoption{simple}, \xoption{lift}),
%     \hologo{teTeX}.
%   \item
%     Fixes:
%     \hologo{iniTeX},
%     \hologo{pdfLaTeX},
%     \hologo{pdfTeX},
%     \hologo{virTeX}.
%   \item
%     \cs{hologoFontSetup} and \cs{hologoLogoFontSetup} added.
%   \item
%     \cs{hologoVariant} and \cs{HologoVariant} added.
%   \end{Version}
%   \begin{Version}{2011/11/22 v1.7}
%   \item
%     New logos:
%     \hologo{BibTeX8},
%     \hologo{LaTeXML},
%     \hologo{SageTeX},
%     \hologo{TeX4ht},
%     \hologo{TTH}.
%   \item
%     \hologo{Xe} and friends: Driver stuff fixed.
%   \item
%     \hologo{Xe} and friends: Support for italic added.
%   \item
%     \hologo{Xe} and friends: Package support for \xpackage{pgf}
%     and \xpackage{pstricks} added.
%   \end{Version}
%   \begin{Version}{2011/11/29 v1.8}
%   \item
%     New logos:
%     \hologo{HanTheThanh}.
%   \end{Version}
%   \begin{Version}{2011/12/21 v1.9}
%   \item
%     Patch for package \xpackage{ifxetex} added for the case that
%     \cs{newif} is undefined in \hologo{iniTeX}.
%   \item
%     Some fixes for \hologo{iniTeX}.
%   \end{Version}
%   \begin{Version}{2012/04/26 v1.10}
%   \item
%     Fix in bookmark version of logo ``\hologo{HanTheThanh}''.
%   \end{Version}
%   \begin{Version}{2016/05/12 v1.11}
%   \item
%     Update HOLOGO@IfCharExists (previously in texlive)
%   \item define pdfliteral in current luatex.
%   \end{Version}
% \end{History}
%
% \PrintIndex
%
% \Finale
\endinput
|
% \end{quote}
% Do not forget to quote the argument according to the demands
% of your shell.
%
% \paragraph{Generating the documentation.}
% You can use both the \xfile{.dtx} or the \xfile{.drv} to generate
% the documentation. The process can be configured by the
% configuration file \xfile{ltxdoc.cfg}. For instance, put this
% line into this file, if you want to have A4 as paper format:
% \begin{quote}
%   \verb|\PassOptionsToClass{a4paper}{article}|
% \end{quote}
% An example follows how to generate the
% documentation with pdf\LaTeX:
% \begin{quote}
%\begin{verbatim}
%pdflatex hologo.dtx
%makeindex -s gind.ist hologo.idx
%pdflatex hologo.dtx
%makeindex -s gind.ist hologo.idx
%pdflatex hologo.dtx
%\end{verbatim}
% \end{quote}
%
% \section{Catalogue}
%
% The following XML file can be used as source for the
% \href{http://mirror.ctan.org/help/Catalogue/catalogue.html}{\TeX\ Catalogue}.
% The elements \texttt{caption} and \texttt{description} are imported
% from the original XML file from the Catalogue.
% The name of the XML file in the Catalogue is \xfile{hologo.xml}.
%    \begin{macrocode}
%<*catalogue>
<?xml version='1.0' encoding='us-ascii'?>
<!DOCTYPE entry SYSTEM 'catalogue.dtd'>
<entry datestamp='$Date$' modifier='$Author$' id='hologo'>
  <name>hologo</name>
  <caption>A collection of logos with bookmark support.</caption>
  <authorref id='auth:oberdiek'/>
  <copyright owner='Heiko Oberdiek' year='2010-2012'/>
  <license type='lppl1.3'/>
  <version number='1.10'/>
  <description>
    The package defines a single command <tt>\hologo</tt>, whose
    argument is the usual case-confused ASCII version of the logo.
    The command is bookmark-enabled, so that every logo becomes
    available in bookmarks without further work.
    <p/>
    The package is part of the <xref refid='oberdiek'>oberdiek</xref>
    bundle.
  </description>
  <documentation details='Package documentation'
      href='ctan:/macros/latex/contrib/oberdiek/hologo.pdf'/>
  <ctan file='true' path='/macros/latex/contrib/oberdiek/hologo.dtx'/>
  <miktex location='oberdiek'/>
  <texlive location='oberdiek'/>
  <install path='/macros/latex/contrib/oberdiek/oberdiek.tds.zip'/>
</entry>
%</catalogue>
%    \end{macrocode}
%
% \begin{thebibliography}{9}
% \raggedright
%
% \bibitem{btxdoc}
% Oren Patashnik,
% \textit{\hologo{BibTeX}ing},
% 1988-02-08.\\
% \CTAN{biblio/bibtex/base/}
%
% \bibitem{dtklogos}
% Gerd Neugebauer, DANTE,
% \textit{Package \xpackage{dtklogos}},
% 2011-04-25.\\
% \CTAN{usergrps/dante/dtk/dtklogos.sty}
%
% \bibitem{etexman}
% The \hologo{NTS} Team,
% \textit{The \hologo{eTeX} manual},
% 1998-02.\\
% \CTAN{systems/e-tex/v2/doc/}
%
% \bibitem{ExTeX-FAQ}
% The \hologo{ExTeX} group,
% \textit{\hologo{ExTeX}: FAQ -- How is \hologo{ExTeX} typeset?},
% 2007-04-14.\\
% \url{http://www.extex.org/documentation/faq.html}
%
% \bibitem{LyX}
% %@MISC{ LyX,
% %  title = {{LyX 2.0.0 -- The Document Processor [Computer software and manual]}},
% %  author = {{The LyX Team}},
% %  howpublished = {Internet: http://www.lyx.org},
% %  year = {2011-05-08},
% %  note = {Retrieved May 10, 2011, from http://www.lyx.org},
% %  url = {http://www.lyx.org/}
% %}
% The \hologo{LyX} Team,
% \textit{\hologo{LyX} -- The Document Processor},
% 2011-05-08.\\
% \url{http://www.lyx.org/}
%
% \bibitem{OzTeX}
% Andrew Trevorrow,
% \hologo{OzTeX} FAQ: What is the correct way to typeset ``\hologo{OzTeX}''?,
% 2011-09-15 (visited).
% \url{http://www.trevorrow.com/oztex/ozfaq.html#oztex-logo}
%
% \bibitem{PiCTeX}
% Michael Wichura,
% \textit{The \hologo{PiCTeX} macro package},
% 1987-09-21.
% \CTAN{graphics/pictex/}
%
% \bibitem{scrlogo}
% Markus Kohm,
% \textit{\hologo{KOMAScript} Datei \xfile{scrlogo.dtx}},
% 2009-01-30.\\
% \CTAN{install/macros/latex/contrib/komascript.tds.zip}
%
% \end{thebibliography}
%
% \begin{History}
%   \begin{Version}{2010/04/08 v1.0}
%   \item
%     The first version.
%   \end{Version}
%   \begin{Version}{2010/04/16 v1.1}
%   \item
%     \cs{Hologo} added for support of logos at start of a sentence.
%   \item
%     \cs{hologoSetup} and \cs{hologoLogoSetup} added.
%   \item
%     Options \xoption{break}, \xoption{hyphenbreak}, \xoption{spacebreak}
%     added.
%   \item
%     Variant support added by option \xoption{variant}.
%   \end{Version}
%   \begin{Version}{2010/04/24 v1.2}
%   \item
%     \hologo{LaTeX3} added.
%   \item
%     \hologo{VTeX} added.
%   \end{Version}
%   \begin{Version}{2010/11/21 v1.3}
%   \item
%     \hologo{iniTeX}, \hologo{virTeX} added.
%   \end{Version}
%   \begin{Version}{2011/03/25 v1.4}
%   \item
%     \hologo{ConTeXt} with variants added.
%   \item
%     Option \xoption{discretionarybreak} added as refinement for
%     option \xoption{break}.
%   \end{Version}
%   \begin{Version}{2011/04/21 v1.5}
%   \item
%     Wrong TDS directory for test files fixed.
%   \end{Version}
%   \begin{Version}{2011/10/01 v1.6}
%   \item
%     Support for package \xpackage{tex4ht} added.
%   \item
%     Support for \cs{csname} added if \cs{ifincsname} is available.
%   \item
%     New logos:
%     \hologo{(La)TeX},
%     \hologo{biber},
%     \hologo{BibTeX} (\xoption{sc}, \xoption{sf}),
%     \hologo{emTeX},
%     \hologo{ExTeX},
%     \hologo{KOMAScript},
%     \hologo{La},
%     \hologo{LyX},
%     \hologo{MiKTeX},
%     \hologo{NTS},
%     \hologo{OzMF},
%     \hologo{OzMP},
%     \hologo{OzTeX},
%     \hologo{OzTtH},
%     \hologo{PCTeX},
%     \hologo{PiC},
%     \hologo{PiCTeX},
%     \hologo{METAFONT},
%     \hologo{MetaFun},
%     \hologo{METAPOST},
%     \hologo{MetaPost},
%     \hologo{SLiTeX} (\xoption{lift}, \xoption{narrow}, \xoption{simple}),
%     \hologo{SliTeX} (\xoption{narrow}, \xoption{simple}, \xoption{lift}),
%     \hologo{teTeX}.
%   \item
%     Fixes:
%     \hologo{iniTeX},
%     \hologo{pdfLaTeX},
%     \hologo{pdfTeX},
%     \hologo{virTeX}.
%   \item
%     \cs{hologoFontSetup} and \cs{hologoLogoFontSetup} added.
%   \item
%     \cs{hologoVariant} and \cs{HologoVariant} added.
%   \end{Version}
%   \begin{Version}{2011/11/22 v1.7}
%   \item
%     New logos:
%     \hologo{BibTeX8},
%     \hologo{LaTeXML},
%     \hologo{SageTeX},
%     \hologo{TeX4ht},
%     \hologo{TTH}.
%   \item
%     \hologo{Xe} and friends: Driver stuff fixed.
%   \item
%     \hologo{Xe} and friends: Support for italic added.
%   \item
%     \hologo{Xe} and friends: Package support for \xpackage{pgf}
%     and \xpackage{pstricks} added.
%   \end{Version}
%   \begin{Version}{2011/11/29 v1.8}
%   \item
%     New logos:
%     \hologo{HanTheThanh}.
%   \end{Version}
%   \begin{Version}{2011/12/21 v1.9}
%   \item
%     Patch for package \xpackage{ifxetex} added for the case that
%     \cs{newif} is undefined in \hologo{iniTeX}.
%   \item
%     Some fixes for \hologo{iniTeX}.
%   \end{Version}
%   \begin{Version}{2012/04/26 v1.10}
%   \item
%     Fix in bookmark version of logo ``\hologo{HanTheThanh}''.
%   \end{Version}
%   \begin{Version}{2016/05/12 v1.11}
%   \item
%     Update HOLOGO@IfCharExists (previously in texlive)
%   \item define pdfliteral in current luatex.
%   \end{Version}
% \end{History}
%
% \PrintIndex
%
% \Finale
\endinput
%
        \else
          \input hologo.cfg\relax
        \fi
      \fi
    }%
    \ltx@IfUndefined{newread}{%
      \chardef\HOLOGO@temp=15 %
      \def\HOLOGO@CheckRead{%
        \ifeof\HOLOGO@temp
          \HOLOGO@InputIfExists
        \else
          \ifcase\HOLOGO@temp
            \@PackageWarningNoLine{hologo}{%
              Configuration file ignored, because\MessageBreak
              a free read register could not be found%
            }%
          \else
            \begingroup
              \count\ltx@cclv=\HOLOGO@temp
              \advance\ltx@cclv by \ltx@minusone
              \edef\x{\endgroup
                \chardef\noexpand\HOLOGO@temp=\the\count\ltx@cclv
                \relax
              }%
            \x
          \fi
        \fi
      }%
    }{%
      \csname newread\endcsname\HOLOGO@temp
      \HOLOGO@InputIfExists
    }%
  }{%
    \edef\HOLOGO@temp{\pdf@filesize{hologo.cfg}}%
    \ifx\HOLOGO@temp\ltx@empty
    \else
      \ifnum\HOLOGO@temp>0 %
        \begingroup
          \def\x{LaTeX2e}%
        \expandafter\endgroup
        \ifx\fmtname\x
          % \iffalse meta-comment
%
% File: hologo.dtx
% Version: 2016/05/12 v1.11
% Info: A logo collection with bookmark support
%
% Copyright (C) 2010-2012 by
%    Heiko Oberdiek <heiko.oberdiek at googlemail.com>
%
% This work may be distributed and/or modified under the
% conditions of the LaTeX Project Public License, either
% version 1.3c of this license or (at your option) any later
% version. This version of this license is in
%    http://www.latex-project.org/lppl/lppl-1-3c.txt
% and the latest version of this license is in
%    http://www.latex-project.org/lppl.txt
% and version 1.3 or later is part of all distributions of
% LaTeX version 2005/12/01 or later.
%
% This work has the LPPL maintenance status "maintained".
%
% This Current Maintainer of this work is Heiko Oberdiek.
%
% The Base Interpreter refers to any `TeX-Format',
% because some files are installed in TDS:tex/generic//.
%
% This work consists of the main source file hologo.dtx
% and the derived files
%    hologo.sty, hologo.pdf, hologo.ins, hologo.drv, hologo-example.tex,
%    hologo-test1.tex, hologo-test-spacefactor.tex,
%    hologo-test-list.tex.
%
% Distribution:
%    CTAN:macros/latex/contrib/oberdiek/hologo.dtx
%    CTAN:macros/latex/contrib/oberdiek/hologo.pdf
%
% Unpacking:
%    (a) If hologo.ins is present:
%           tex hologo.ins
%    (b) Without hologo.ins:
%           tex hologo.dtx
%    (c) If you insist on using LaTeX
%           latex \let\install=y% \iffalse meta-comment
%
% File: hologo.dtx
% Version: 2016/05/12 v1.11
% Info: A logo collection with bookmark support
%
% Copyright (C) 2010-2012 by
%    Heiko Oberdiek <heiko.oberdiek at googlemail.com>
%
% This work may be distributed and/or modified under the
% conditions of the LaTeX Project Public License, either
% version 1.3c of this license or (at your option) any later
% version. This version of this license is in
%    http://www.latex-project.org/lppl/lppl-1-3c.txt
% and the latest version of this license is in
%    http://www.latex-project.org/lppl.txt
% and version 1.3 or later is part of all distributions of
% LaTeX version 2005/12/01 or later.
%
% This work has the LPPL maintenance status "maintained".
%
% This Current Maintainer of this work is Heiko Oberdiek.
%
% The Base Interpreter refers to any `TeX-Format',
% because some files are installed in TDS:tex/generic//.
%
% This work consists of the main source file hologo.dtx
% and the derived files
%    hologo.sty, hologo.pdf, hologo.ins, hologo.drv, hologo-example.tex,
%    hologo-test1.tex, hologo-test-spacefactor.tex,
%    hologo-test-list.tex.
%
% Distribution:
%    CTAN:macros/latex/contrib/oberdiek/hologo.dtx
%    CTAN:macros/latex/contrib/oberdiek/hologo.pdf
%
% Unpacking:
%    (a) If hologo.ins is present:
%           tex hologo.ins
%    (b) Without hologo.ins:
%           tex hologo.dtx
%    (c) If you insist on using LaTeX
%           latex \let\install=y\input{hologo.dtx}
%        (quote the arguments according to the demands of your shell)
%
% Documentation:
%    (a) If hologo.drv is present:
%           latex hologo.drv
%    (b) Without hologo.drv:
%           latex hologo.dtx; ...
%    The class ltxdoc loads the configuration file ltxdoc.cfg
%    if available. Here you can specify further options, e.g.
%    use A4 as paper format:
%       \PassOptionsToClass{a4paper}{article}
%
%    Programm calls to get the documentation (example):
%       pdflatex hologo.dtx
%       makeindex -s gind.ist hologo.idx
%       pdflatex hologo.dtx
%       makeindex -s gind.ist hologo.idx
%       pdflatex hologo.dtx
%
% Installation:
%    TDS:tex/generic/oberdiek/hologo.sty
%    TDS:doc/latex/oberdiek/hologo.pdf
%    TDS:doc/latex/oberdiek/example/hologo-example.tex
%    TDS:doc/latex/oberdiek/test/hologo-test1.tex
%    TDS:doc/latex/oberdiek/test/hologo-test-spacefactor.tex
%    TDS:doc/latex/oberdiek/test/hologo-test-list.tex
%    TDS:source/latex/oberdiek/hologo.dtx
%
%<*ignore>
\begingroup
  \catcode123=1 %
  \catcode125=2 %
  \def\x{LaTeX2e}%
\expandafter\endgroup
\ifcase 0\ifx\install y1\fi\expandafter
         \ifx\csname processbatchFile\endcsname\relax\else1\fi
         \ifx\fmtname\x\else 1\fi\relax
\else\csname fi\endcsname
%</ignore>
%<*install>
\input docstrip.tex
\Msg{************************************************************************}
\Msg{* Installation}
\Msg{* Package: hologo 2016/05/12 v1.11 A logo collection with bookmark support (HO)}
\Msg{************************************************************************}

\keepsilent
\askforoverwritefalse

\let\MetaPrefix\relax
\preamble

This is a generated file.

Project: hologo
Version: 2016/05/12 v1.11

Copyright (C) 2010-2012 by
   Heiko Oberdiek <heiko.oberdiek at googlemail.com>

This work may be distributed and/or modified under the
conditions of the LaTeX Project Public License, either
version 1.3c of this license or (at your option) any later
version. This version of this license is in
   http://www.latex-project.org/lppl/lppl-1-3c.txt
and the latest version of this license is in
   http://www.latex-project.org/lppl.txt
and version 1.3 or later is part of all distributions of
LaTeX version 2005/12/01 or later.

This work has the LPPL maintenance status "maintained".

This Current Maintainer of this work is Heiko Oberdiek.

The Base Interpreter refers to any `TeX-Format',
because some files are installed in TDS:tex/generic//.

This work consists of the main source file hologo.dtx
and the derived files
   hologo.sty, hologo.pdf, hologo.ins, hologo.drv, hologo-example.tex,
   hologo-test1.tex, hologo-test-spacefactor.tex,
   hologo-test-list.tex.

\endpreamble
\let\MetaPrefix\DoubleperCent

\generate{%
  \file{hologo.ins}{\from{hologo.dtx}{install}}%
  \file{hologo.drv}{\from{hologo.dtx}{driver}}%
  \usedir{tex/generic/oberdiek}%
  \file{hologo.sty}{\from{hologo.dtx}{package}}%
  \usedir{doc/latex/oberdiek/example}%
  \file{hologo-example.tex}{\from{hologo.dtx}{example}}%
  \usedir{doc/latex/oberdiek/test}%
  \file{hologo-test1.tex}{\from{hologo.dtx}{test1}}%
  \file{hologo-test-spacefactor.tex}{\from{hologo.dtx}{test-spacefactor}}%
  \file{hologo-test-list.tex}{\from{hologo.dtx}{test-list}}%
  \nopreamble
  \nopostamble
  \usedir{source/latex/oberdiek/catalogue}%
  \file{hologo.xml}{\from{hologo.dtx}{catalogue}}%
}

\catcode32=13\relax% active space
\let =\space%
\Msg{************************************************************************}
\Msg{*}
\Msg{* To finish the installation you have to move the following}
\Msg{* file into a directory searched by TeX:}
\Msg{*}
\Msg{*     hologo.sty}
\Msg{*}
\Msg{* To produce the documentation run the file `hologo.drv'}
\Msg{* through LaTeX.}
\Msg{*}
\Msg{* Happy TeXing!}
\Msg{*}
\Msg{************************************************************************}

\endbatchfile
%</install>
%<*ignore>
\fi
%</ignore>
%<*driver>
\NeedsTeXFormat{LaTeX2e}
\ProvidesFile{hologo.drv}%
  [2016/05/12 v1.11 A logo collection with bookmark support (HO)]%
\documentclass{ltxdoc}
\usepackage{holtxdoc}[2011/11/22]
\usepackage{hologo}[2016/05/12]
\usepackage{longtable}
\usepackage{array}
\usepackage{paralist}
%\usepackage[T1]{fontenc}
%\usepackage{lmodern}
\begin{document}
  \DocInput{hologo.dtx}%
\end{document}
%</driver>
% \fi
%
%
% \CharacterTable
%  {Upper-case    \A\B\C\D\E\F\G\H\I\J\K\L\M\N\O\P\Q\R\S\T\U\V\W\X\Y\Z
%   Lower-case    \a\b\c\d\e\f\g\h\i\j\k\l\m\n\o\p\q\r\s\t\u\v\w\x\y\z
%   Digits        \0\1\2\3\4\5\6\7\8\9
%   Exclamation   \!     Double quote  \"     Hash (number) \#
%   Dollar        \$     Percent       \%     Ampersand     \&
%   Acute accent  \'     Left paren    \(     Right paren   \)
%   Asterisk      \*     Plus          \+     Comma         \,
%   Minus         \-     Point         \.     Solidus       \/
%   Colon         \:     Semicolon     \;     Less than     \<
%   Equals        \=     Greater than  \>     Question mark \?
%   Commercial at \@     Left bracket  \[     Backslash     \\
%   Right bracket \]     Circumflex    \^     Underscore    \_
%   Grave accent  \`     Left brace    \{     Vertical bar  \|
%   Right brace   \}     Tilde         \~}
%
% \GetFileInfo{hologo.drv}
%
% \title{The \xpackage{hologo} package}
% \date{2016/05/12 v1.11}
% \author{Heiko Oberdiek\\\xemail{heiko.oberdiek at googlemail.com}}
%
% \maketitle
%
% \begin{abstract}
% This package starts a collection of logos with support for bookmarks
% strings.
% \end{abstract}
%
% \tableofcontents
%
% \section{Documentation}
%
% \subsection{Logo macros}
%
% \begin{declcs}{hologo} \M{name}
% \end{declcs}
% Macro \cs{hologo} sets the logo with name \meta{name}.
% The following table shows the supported names.
%
% \begingroup
%   \def\hologoEntry#1#2#3{^^A
%     #1&#2&\hologoLogoSetup{#1}{variant=#2}\hologo{#1}&#3\tabularnewline
%   }
%   \begin{longtable}{>{\ttfamily}l>{\ttfamily}lll}
%     \rmfamily\bfseries{name} & \rmfamily\bfseries variant
%     & \bfseries logo & \bfseries since\\
%     \hline
%     \endhead
%     \hologoList
%   \end{longtable}
% \endgroup
%
% \begin{declcs}{Hologo} \M{name}
% \end{declcs}
% Macro \cs{Hologo} starts the logo \meta{name} with an uppercase
% letter. As an exception small greek letters are not converted
% to uppercase. Examples, see \hologo{eTeX} and \hologo{ExTeX}.
%
% \subsection{Setup macros}
%
% The package does not support package options, but the following
% setup macros can be used to set options.
%
% \begin{declcs}{hologoSetup} \M{key value list}
% \end{declcs}
% Macro \cs{hologoSetup} sets global options.
%
% \begin{declcs}{hologoLogoSetup} \M{logo} \M{key value list}
% \end{declcs}
% Some options can also be used to configure a logo.
% These settings take precedence over global option settings.
%
% \subsection{Options}\label{sec:options}
%
% There are boolean and string options:
% \begin{description}
% \item[Boolean option:]
% It takes |true| or |false|
% as value. If the value is omitted, then |true| is used.
% \item[String option:]
% A value must be given as string. (But the string might be empty.)
% \end{description}
% The following options can be used both in \cs{hologoSetup}
% and \cs{hologoLogoSetup}:
% \begin{description}
% \def\entry#1{\item[\xoption{#1}:]}
% \entry{break}
%   enables or disables line breaks inside the logo. This setting is
%   refined by options \xoption{hyphenbreak}, \xoption{spacebreak}
%   or \xoption{discretionarybreak}.
%   Default is |false|.
% \entry{hyphenbreak}
%   enables or disables the line break right after the hyphen character.
% \entry{spacebreak}
%   enables or disables line breaks at space characters.
% \entry{discretionarybreak}
%   enables or disables line breaks at hyphenation points
%   (inserted by \cs{-}).
% \end{description}
% Macro \cs{hologoLogoSetup} also knows:
% \begin{description}
% \item[\xoption{variant}:]
%   This is a string option. It specifies a variant of a logo that
%   must exist. An empty string selects the package default variant.
% \end{description}
% Example:
% \begin{quote}
%   |\hologoSetup{break=false}|\\
%   |\hologoLogoSetup{plainTeX}{variant=hyphen,hyphenbreak}|\\
%   Then ``plain-\TeX'' contains one break point after the hyphen.
% \end{quote}
%
% \subsection{Driver options}
%
% Sometimes graphical operations are needed to construct some
% glyphs (e.g.\ \hologo{XeTeX}). If package \xpackage{graphics}
% or package \xpackage{pgf} are found, then the macros are taken
% from there. Otherwise the packge defines its own operations
% and therefore needs the driver information. Many drivers are
% detected automatically (\hologo{pdfTeX}/\hologo{LuaTeX}
% in PDF mode, \hologo{XeTeX}, \hologo{VTeX}). These have precedence
% over a driver option. The driver can be given as package option
% or using \cs{hologoDriverSetup}.
% The following list contains the recognized driver options:
% \begin{itemize}
% \item \xoption{pdftex}, \xoption{luatex}
% \item \xoption{dvipdfm}, \xoption{dvipdfmx}
% \item \xoption{dvips}, \xoption{dvipsone}, \xoption{xdvi}
% \item \xoption{xetex}
% \item \xoption{vtex}
% \end{itemize}
% The left driver of a line is the driver name that is used internally.
% The following names are aliases for drivers that use the
% same method. Therefore the entry in the \xext{log} file for
% the used driver prints the internally used driver name.
% \begin{description}
% \item[\xoption{driverfallback}:]
%   This option expects a driver that is used,
%   if the driver could not be detected automatically.
% \end{description}
%
% \begin{declcs}{hologoDriverSetup} \M{driver option}
% \end{declcs}
% The driver can also be configured after package loading
% using \cs{hologoDriverSetup}, also the way for \hologo{plainTeX}
% to setup the driver.
%
% \subsection{Font setup}
%
% Some logos require a special font, but should also be usable by
% \hologo{plainTeX}. Therefore the package provides some ways
% to influence the font settings. The options below
% take font settings as values. Both font commands
% such as \cs{sffamily} and macros that take one argument
% like \cs{textsf} can be used.
%
% \begin{declcs}{hologoFontSetup} \M{key value list}
% \end{declcs}
% Macro \cs{hologoFontSetup} sets the fonts for all logos.
% Supported keys:
% \begin{description}
% \def\entry#1{\item[\xoption{#1}:]}
% \entry{general}
%   This font is used for all logos. The default is empty.
%   That means no special font is used.
% \entry{bibsf}
%   This font is used for
%   {\hologoLogoSetup{BibTeX}{variant=sf}\hologo{BibTeX}}
%   with variant \xoption{sf}.
% \entry{rm}
%   This font is a serif font. It is used for \hologo{ExTeX}.
% \entry{sc}
%   This font specifies a small caps font. It is used for
%   {\hologoLogoSetup{BibTeX}{variant=sc}\hologo{BibTeX}}
%   with variant \xoption{sc}.
% \entry{sf}
%   This font specifies a sans serif font. The default
%   is \cs{sffamily}, then \cs{sf} is tried. Otherwise
%   a warning is given. It is used by \hologo{KOMAScript}.
% \entry{sy}
%   This is the font for math symbols (e.g. cmsy).
%   It is used by \hologo{AmS}, \hologo{NTS}, \hologo{ExTeX}.
% \entry{logo}
%   \hologo{METAFONT} and \hologo{METAPOST} are using that font.
%   In \hologo{LaTeX} \cs{logofamily} is used and
%   the definitions of package \xpackage{mflogo} are used
%   if the package is not loaded.
%   Otherwise the \cs{tenlogo} is used and defined
%   if it does not already exists.
% \end{description}
%
% \begin{declcs}{hologoLogoFontSetup} \M{logo} \M{key value list}
% \end{declcs}
% Fonts can also be set for a logo or logo component separately,
% see the following list.
% The keys are the same as for \cs{hologoFontSetup}.
%
% \begin{longtable}{>{\ttfamily}l>{\sffamily}ll}
%   \meta{logo} & keys & result\\
%   \hline
%   \endhead
%   BibTeX & bibsf & {\hologoLogoSetup{BibTeX}{variant=sf}\hologo{BibTeX}}\\[.5ex]
%   BibTeX & sc & {\hologoLogoSetup{BibTeX}{variant=sc}\hologo{BibTeX}}\\[.5ex]
%   ExTeX & rm & \hologo{ExTeX}\\
%   SliTeX & rm & \hologo{SliTeX}\\[.5ex]
%   AmS & sy & \hologo{AmS}\\
%   ExTeX & sy & \hologo{ExTeX}\\
%   NTS & sy & \hologo{NTS}\\[.5ex]
%   KOMAScript & sf & \hologo{KOMAScript}\\[.5ex]
%   METAFONT & logo & \hologo{METAFONT}\\
%   METAPOST & logo & \hologo{METAPOST}\\[.5ex]
%   SliTeX & sc \hologo{SliTeX}
% \end{longtable}
%
% \subsubsection{Font order}
%
% For all logos the font \xoption{general} is applied first.
% Example:
%\begin{quote}
%|\hologoFontSetup{general=\color{red}}|
%\end{quote}
% will print red logos.
% Then if the font uses a special font \xoption{sf}, for example,
% the font is applied that is setup by \cs{hologoLogoFontSetup}.
% If this font is not setup, then the common font setup
% by \cs{hologoFontSetup} is used. Otherwise a warning is given,
% that there is no font configured.
%
% \subsection{Additional user macros}
%
% Usually a variant of a logo is configured by using
% \cs{hologoLogoSetup}, because it is bad style to mix
% different variants of the same logo in the same text.
% There the following macros are a convenience for testing.
%
% \begin{declcs}{hologoVariant} \M{name} \M{variant}\\
%   \cs{HologoVariant} \M{name} \M{variant}
% \end{declcs}
% Logo \meta{name} is set using \meta{variant} that specifies
% explicitely which variant of the macro is used. If the argument
% is empty, then the default form of the logo is used
% (configurable by \cs{hologoLogoSetup}).
%
% \cs{HologoVariant} is used if the logo is set in a context
% that needs an uppercase first letter (beginning of a sentence, \dots).
%
% \begin{declcs}{hologoList}\\
%   \cs{hologoEntry} \M{logo} \M{variant} \M{since}
% \end{declcs}
% Macro \cs{hologoList} contains all logos that are provided
% by the package including variants. The list consists of calls
% of \cs{hologoEntry} with three arguments starting with the
% logo name \meta{logo} and its variant \meta{variant}. An empty
% variant means the current default. Argument \meta{since} specifies
% with version of the package \xpackage{hologo} is needed to get
% the logo. If the logo is fixed, then the date gets updated.
% Therefore the date \meta{since} is not exactly the date of
% the first introduction, but rather the date of the latest fix.
%
% Before \cs{hologoList} can be used, macro \cs{hologoEntry} needs
% a definition. The example file in section \ref{sec:example}
% shows applications of \cs{hologoList}.
%
% \subsection{Supported contexts}
%
% Macros \cs{hologo} and friends support special contexts:
% \begin{itemize}
% \item \hologo{LaTeX}'s protection mechanism.
% \item Bookmarks of package \xpackage{hyperref}.
% \item Package \xpackage{tex4ht}.
% \item The macros can be used inside \cs{csname} constructs,
%   if \cs{ifincsname} is available (\hologo{pdfTeX}, \hologo{XeTeX},
%   \hologo{LuaTeX}).
% \end{itemize}
%
% \subsection{Example}
% \label{sec:example}
%
% The following example prints the logos in different fonts.
%    \begin{macrocode}
%<*example>
%<<verbatim
\NeedsTeXFormat{LaTeX2e}
\documentclass[a4paper]{article}
\usepackage[
  hmargin=20mm,
  vmargin=20mm,
]{geometry}
\pagestyle{empty}
\usepackage{hologo}[2016/05/12]
\usepackage{longtable}
\usepackage{array}
\setlength{\extrarowheight}{2pt}
\usepackage[T1]{fontenc}
\usepackage{lmodern}
\usepackage{pdflscape}
\usepackage[
  pdfencoding=auto,
]{hyperref}
\hypersetup{
  pdfauthor={Heiko Oberdiek},
  pdftitle={Example for package `hologo'},
  pdfsubject={Logos with fonts lmr, lmss, qtm, qpl, qhv},
}
\usepackage{bookmark}

% Print the logo list on the console

\begingroup
  \typeout{}%
  \typeout{*** Begin of logo list ***}%
  \newcommand*{\hologoEntry}[3]{%
    \typeout{#1 \ifx\\#2\\\else(#2) \fi[#3]}%
  }%
  \hologoList
  \typeout{*** End of logo list ***}%
  \typeout{}%
\endgroup

\begin{document}
\begin{landscape}

  \section{Example file for package `hologo'}

  % Table for font names

  \begin{longtable}{>{\bfseries}ll}
    \textbf{font} & \textbf{Font name}\\
    \hline
    lmr & Latin Modern Roman\\
    lmss & Latin Modern Sans\\
    qtm & \TeX\ Gyre Termes\\
    qhv & \TeX\ Gyre Heros\\
    qpl & \TeX\ Gyre Pagella\\
  \end{longtable}

  % Logo list with logos in different fonts

  \begingroup
    \newcommand*{\SetVariant}[2]{%
      \ifx\\#2\\%
      \else
        \hologoLogoSetup{#1}{variant=#2}%
      \fi
    }%
    \newcommand*{\hologoEntry}[3]{%
      \SetVariant{#1}{#2}%
      \raisebox{1em}[0pt][0pt]{\hypertarget{#1@#2}{}}%
      \bookmark[%
        dest={#1@#2},%
      ]{%
        #1\ifx\\#2\\\else\space(#2)\fi: \Hologo{#1}, \hologo{#1} %
        [Unicode]%
      }%
      \hypersetup{unicode=false}%
      \bookmark[%
        dest={#1@#2},%
      ]{%
        #1\ifx\\#2\\\else\space(#2)\fi: \Hologo{#1}, \hologo{#1} %
        [PDFDocEncoding]%
      }%
      \texttt{#1}%
      &%
      \texttt{#2}%
      &%
      \Hologo{#1}%
      &%
      \SetVariant{#1}{#2}%
      \hologo{#1}%
      &%
      \SetVariant{#1}{#2}%
      \fontfamily{qtm}\selectfont
      \hologo{#1}%
      &%
      \SetVariant{#1}{#2}%
      \fontfamily{qpl}\selectfont
      \hologo{#1}%
      &%
      \SetVariant{#1}{#2}%
      \textsf{\hologo{#1}}%
      &%
      \SetVariant{#1}{#2}%
      \fontfamily{qhv}\selectfont
      \hologo{#1}%
      \tabularnewline
    }%
    \begin{longtable}{llllllll}%
      \textbf{\textit{logo}} & \textbf{\textit{variant}} &
      \texttt{\string\Hologo} &
      \textbf{lmr} & \textbf{qtm} & \textbf{qpl} &
      \textbf{lmss} & \textbf{qhv}
      \tabularnewline
      \hline
      \endhead
      \hologoList
    \end{longtable}%
  \endgroup

\end{landscape}
\end{document}
%verbatim
%</example>
%    \end{macrocode}
%
% \StopEventually{
% }
%
% \section{Implementation}
%    \begin{macrocode}
%<*package>
%    \end{macrocode}
%    Reload check, especially if the package is not used with \LaTeX.
%    \begin{macrocode}
\begingroup\catcode61\catcode48\catcode32=10\relax%
  \catcode13=5 % ^^M
  \endlinechar=13 %
  \catcode35=6 % #
  \catcode39=12 % '
  \catcode44=12 % ,
  \catcode45=12 % -
  \catcode46=12 % .
  \catcode58=12 % :
  \catcode64=11 % @
  \catcode123=1 % {
  \catcode125=2 % }
  \expandafter\let\expandafter\x\csname ver@hologo.sty\endcsname
  \ifx\x\relax % plain-TeX, first loading
  \else
    \def\empty{}%
    \ifx\x\empty % LaTeX, first loading,
      % variable is initialized, but \ProvidesPackage not yet seen
    \else
      \expandafter\ifx\csname PackageInfo\endcsname\relax
        \def\x#1#2{%
          \immediate\write-1{Package #1 Info: #2.}%
        }%
      \else
        \def\x#1#2{\PackageInfo{#1}{#2, stopped}}%
      \fi
      \x{hologo}{The package is already loaded}%
      \aftergroup\endinput
    \fi
  \fi
\endgroup%
%    \end{macrocode}
%    Package identification:
%    \begin{macrocode}
\begingroup\catcode61\catcode48\catcode32=10\relax%
  \catcode13=5 % ^^M
  \endlinechar=13 %
  \catcode35=6 % #
  \catcode39=12 % '
  \catcode40=12 % (
  \catcode41=12 % )
  \catcode44=12 % ,
  \catcode45=12 % -
  \catcode46=12 % .
  \catcode47=12 % /
  \catcode58=12 % :
  \catcode64=11 % @
  \catcode91=12 % [
  \catcode93=12 % ]
  \catcode123=1 % {
  \catcode125=2 % }
  \expandafter\ifx\csname ProvidesPackage\endcsname\relax
    \def\x#1#2#3[#4]{\endgroup
      \immediate\write-1{Package: #3 #4}%
      \xdef#1{#4}%
    }%
  \else
    \def\x#1#2[#3]{\endgroup
      #2[{#3}]%
      \ifx#1\@undefined
        \xdef#1{#3}%
      \fi
      \ifx#1\relax
        \xdef#1{#3}%
      \fi
    }%
  \fi
\expandafter\x\csname ver@hologo.sty\endcsname
\ProvidesPackage{hologo}%
  [2016/05/12 v1.11 A logo collection with bookmark support (HO)]%
%    \end{macrocode}
%
%    \begin{macrocode}
\begingroup\catcode61\catcode48\catcode32=10\relax%
  \catcode13=5 % ^^M
  \endlinechar=13 %
  \catcode123=1 % {
  \catcode125=2 % }
  \catcode64=11 % @
  \def\x{\endgroup
    \expandafter\edef\csname HOLOGO@AtEnd\endcsname{%
      \endlinechar=\the\endlinechar\relax
      \catcode13=\the\catcode13\relax
      \catcode32=\the\catcode32\relax
      \catcode35=\the\catcode35\relax
      \catcode61=\the\catcode61\relax
      \catcode64=\the\catcode64\relax
      \catcode123=\the\catcode123\relax
      \catcode125=\the\catcode125\relax
    }%
  }%
\x\catcode61\catcode48\catcode32=10\relax%
\catcode13=5 % ^^M
\endlinechar=13 %
\catcode35=6 % #
\catcode64=11 % @
\catcode123=1 % {
\catcode125=2 % }
\def\TMP@EnsureCode#1#2{%
  \edef\HOLOGO@AtEnd{%
    \HOLOGO@AtEnd
    \catcode#1=\the\catcode#1\relax
  }%
  \catcode#1=#2\relax
}
\TMP@EnsureCode{10}{12}% ^^J
\TMP@EnsureCode{33}{12}% !
\TMP@EnsureCode{34}{12}% "
\TMP@EnsureCode{36}{3}% $
\TMP@EnsureCode{38}{4}% &
\TMP@EnsureCode{39}{12}% '
\TMP@EnsureCode{40}{12}% (
\TMP@EnsureCode{41}{12}% )
\TMP@EnsureCode{42}{12}% *
\TMP@EnsureCode{43}{12}% +
\TMP@EnsureCode{44}{12}% ,
\TMP@EnsureCode{45}{12}% -
\TMP@EnsureCode{46}{12}% .
\TMP@EnsureCode{47}{12}% /
\TMP@EnsureCode{58}{12}% :
\TMP@EnsureCode{59}{12}% ;
\TMP@EnsureCode{60}{12}% <
\TMP@EnsureCode{62}{12}% >
\TMP@EnsureCode{63}{12}% ?
\TMP@EnsureCode{91}{12}% [
\TMP@EnsureCode{93}{12}% ]
\TMP@EnsureCode{94}{7}% ^ (superscript)
\TMP@EnsureCode{95}{8}% _ (subscript)
\TMP@EnsureCode{96}{12}% `
\TMP@EnsureCode{124}{12}% |
\edef\HOLOGO@AtEnd{%
  \HOLOGO@AtEnd
  \escapechar\the\escapechar\relax
  \noexpand\endinput
}
\escapechar=92 %
%    \end{macrocode}
%
% \subsection{Logo list}
%
%    \begin{macro}{\hologoList}
%    \begin{macrocode}
\def\hologoList{%
  \hologoEntry{(La)TeX}{}{2011/10/01}%
  \hologoEntry{AmSLaTeX}{}{2010/04/16}%
  \hologoEntry{AmSTeX}{}{2010/04/16}%
  \hologoEntry{biber}{}{2011/10/01}%
  \hologoEntry{BibTeX}{}{2011/10/01}%
  \hologoEntry{BibTeX}{sf}{2011/10/01}%
  \hologoEntry{BibTeX}{sc}{2011/10/01}%
  \hologoEntry{BibTeX8}{}{2011/11/22}%
  \hologoEntry{ConTeXt}{}{2011/03/25}%
  \hologoEntry{ConTeXt}{narrow}{2011/03/25}%
  \hologoEntry{ConTeXt}{simple}{2011/03/25}%
  \hologoEntry{emTeX}{}{2010/04/26}%
  \hologoEntry{eTeX}{}{2010/04/08}%
  \hologoEntry{ExTeX}{}{2011/10/01}%
  \hologoEntry{HanTheThanh}{}{2011/11/29}%
  \hologoEntry{iniTeX}{}{2011/10/01}%
  \hologoEntry{KOMAScript}{}{2011/10/01}%
  \hologoEntry{La}{}{2010/05/08}%
  \hologoEntry{LaTeX}{}{2010/04/08}%
  \hologoEntry{LaTeX2e}{}{2010/04/08}%
  \hologoEntry{LaTeX3}{}{2010/04/24}%
  \hologoEntry{LaTeXe}{}{2010/04/08}%
  \hologoEntry{LaTeXML}{}{2011/11/22}%
  \hologoEntry{LaTeXTeX}{}{2011/10/01}%
  \hologoEntry{LuaLaTeX}{}{2010/04/08}%
  \hologoEntry{LuaTeX}{}{2010/04/08}%
  \hologoEntry{LyX}{}{2011/10/01}%
  \hologoEntry{METAFONT}{}{2011/10/01}%
  \hologoEntry{MetaFun}{}{2011/10/01}%
  \hologoEntry{METAPOST}{}{2011/10/01}%
  \hologoEntry{MetaPost}{}{2011/10/01}%
  \hologoEntry{MiKTeX}{}{2011/10/01}%
  \hologoEntry{NTS}{}{2011/10/01}%
  \hologoEntry{OzMF}{}{2011/10/01}%
  \hologoEntry{OzMP}{}{2011/10/01}%
  \hologoEntry{OzTeX}{}{2011/10/01}%
  \hologoEntry{OzTtH}{}{2011/10/01}%
  \hologoEntry{PCTeX}{}{2011/10/01}%
  \hologoEntry{pdfTeX}{}{2011/10/01}%
  \hologoEntry{pdfLaTeX}{}{2011/10/01}%
  \hologoEntry{PiC}{}{2011/10/01}%
  \hologoEntry{PiCTeX}{}{2011/10/01}%
  \hologoEntry{plainTeX}{}{2010/04/08}%
  \hologoEntry{plainTeX}{space}{2010/04/16}%
  \hologoEntry{plainTeX}{hyphen}{2010/04/16}%
  \hologoEntry{plainTeX}{runtogether}{2010/04/16}%
  \hologoEntry{SageTeX}{}{2011/11/22}%
  \hologoEntry{SLiTeX}{}{2011/10/01}%
  \hologoEntry{SLiTeX}{lift}{2011/10/01}%
  \hologoEntry{SLiTeX}{narrow}{2011/10/01}%
  \hologoEntry{SLiTeX}{simple}{2011/10/01}%
  \hologoEntry{SliTeX}{}{2011/10/01}%
  \hologoEntry{SliTeX}{narrow}{2011/10/01}%
  \hologoEntry{SliTeX}{simple}{2011/10/01}%
  \hologoEntry{SliTeX}{lift}{2011/10/01}%
  \hologoEntry{teTeX}{}{2011/10/01}%
  \hologoEntry{TeX}{}{2010/04/08}%
  \hologoEntry{TeX4ht}{}{2011/11/22}%
  \hologoEntry{TTH}{}{2011/11/22}%
  \hologoEntry{virTeX}{}{2011/10/01}%
  \hologoEntry{VTeX}{}{2010/04/24}%
  \hologoEntry{Xe}{}{2010/04/08}%
  \hologoEntry{XeLaTeX}{}{2010/04/08}%
  \hologoEntry{XeTeX}{}{2010/04/08}%
}
%    \end{macrocode}
%    \end{macro}
%
% \subsection{Load resources}
%
%    \begin{macrocode}
\begingroup\expandafter\expandafter\expandafter\endgroup
\expandafter\ifx\csname RequirePackage\endcsname\relax
  \def\TMP@RequirePackage#1[#2]{%
    \begingroup\expandafter\expandafter\expandafter\endgroup
    \expandafter\ifx\csname ver@#1.sty\endcsname\relax
      \input #1.sty\relax
    \fi
  }%
  \TMP@RequirePackage{ltxcmds}[2011/02/04]%
  \TMP@RequirePackage{infwarerr}[2010/04/08]%
  \TMP@RequirePackage{kvsetkeys}[2010/03/01]%
  \TMP@RequirePackage{kvdefinekeys}[2010/03/01]%
  \TMP@RequirePackage{pdftexcmds}[2010/04/01]%
  \TMP@RequirePackage{ifpdf}[2010/01/28]%
  \TMP@RequirePackage{ifluatex}[2010/03/01]%
  \ltx@IfUndefined{newif}{%
    \expandafter\let\csname newif\endcsname\ltx@newif
  }{}%
  \TMP@RequirePackage{ifxetex}[2009/01/23]%
  \TMP@RequirePackage{ifvtex}[2010/03/01]%
\else
  \RequirePackage{ltxcmds}[2011/02/04]%
  \RequirePackage{infwarerr}[2010/04/08]%
  \RequirePackage{kvsetkeys}[2010/03/01]%
  \RequirePackage{kvdefinekeys}[2010/03/01]%
  \RequirePackage{pdftexcmds}[2010/04/01]%
  \RequirePackage{ifpdf}[2010/01/28]%
  \RequirePackage{ifluatex}[2010/03/01]%
  \RequirePackage{ifxetex}[2009/01/23]%
  \RequirePackage{ifvtex}[2010/03/01]%
\fi
%    \end{macrocode}
%
%    \begin{macro}{\HOLOGO@IfDefined}
%    \begin{macrocode}
\def\HOLOGO@IfExists#1{%
  \ifx\@undefined#1%
    \expandafter\ltx@secondoftwo
  \else
    \ifx\relax#1%
      \expandafter\ltx@secondoftwo
    \else
      \expandafter\expandafter\expandafter\ltx@firstoftwo
    \fi
  \fi
}
%    \end{macrocode}
%    \end{macro}
%
% \subsection{Setup macros}
%
%    \begin{macro}{\hologoSetup}
%    \begin{macrocode}
\def\hologoSetup{%
  \let\HOLOGO@name\relax
  \HOLOGO@Setup
}
%    \end{macrocode}
%    \end{macro}
%
%    \begin{macro}{\hologoLogoSetup}
%    \begin{macrocode}
\def\hologoLogoSetup#1{%
  \edef\HOLOGO@name{#1}%
  \ltx@IfUndefined{HoLogo@\HOLOGO@name}{%
    \@PackageError{hologo}{%
      Unknown logo `\HOLOGO@name'%
    }\@ehc
    \ltx@gobble
  }{%
    \HOLOGO@Setup
  }%
}
%    \end{macrocode}
%    \end{macro}
%
%    \begin{macro}{\HOLOGO@Setup}
%    \begin{macrocode}
\def\HOLOGO@Setup{%
  \kvsetkeys{HoLogo}%
}
%    \end{macrocode}
%    \end{macro}
%
% \subsection{Options}
%
%    \begin{macro}{\HOLOGO@DeclareBoolOption}
%    \begin{macrocode}
\def\HOLOGO@DeclareBoolOption#1{%
  \expandafter\chardef\csname HOLOGOOPT@#1\endcsname\ltx@zero
  \kv@define@key{HoLogo}{#1}[true]{%
    \def\HOLOGO@temp{##1}%
    \ifx\HOLOGO@temp\HOLOGO@true
      \ifx\HOLOGO@name\relax
        \expandafter\chardef\csname HOLOGOOPT@#1\endcsname=\ltx@one
      \else
        \expandafter\chardef\csname
        HoLogoOpt@#1@\HOLOGO@name\endcsname\ltx@one
      \fi
      \HOLOGO@SetBreakAll{#1}%
    \else
      \ifx\HOLOGO@temp\HOLOGO@false
        \ifx\HOLOGO@name\relax
          \expandafter\chardef\csname HOLOGOOPT@#1\endcsname=\ltx@zero
        \else
          \expandafter\chardef\csname
          HoLogoOpt@#1@\HOLOGO@name\endcsname=\ltx@zero
        \fi
        \HOLOGO@SetBreakAll{#1}%
      \else
        \@PackageError{hologo}{%
          Unknown value `##1' for boolean option `#1'.\MessageBreak
          Known values are `true' and `false'%
        }\@ehc
      \fi
    \fi
  }%
}
%    \end{macrocode}
%    \end{macro}
%
%    \begin{macro}{\HOLOGO@SetBreakAll}
%    \begin{macrocode}
\def\HOLOGO@SetBreakAll#1{%
  \def\HOLOGO@temp{#1}%
  \ifx\HOLOGO@temp\HOLOGO@break
    \ifx\HOLOGO@name\relax
      \chardef\HOLOGOOPT@hyphenbreak=\HOLOGOOPT@break
      \chardef\HOLOGOOPT@spacebreak=\HOLOGOOPT@break
      \chardef\HOLOGOOPT@discretionarybreak=\HOLOGOOPT@break
    \else
      \expandafter\chardef
         \csname HoLogoOpt@hyphenbreak@\HOLOGO@name\endcsname=%
         \csname HoLogoOpt@break@\HOLOGO@name\endcsname
      \expandafter\chardef
         \csname HoLogoOpt@spacebreak@\HOLOGO@name\endcsname=%
         \csname HoLogoOpt@break@\HOLOGO@name\endcsname
      \expandafter\chardef
         \csname HoLogoOpt@discretionarybreak@\HOLOGO@name
             \endcsname=%
         \csname HoLogoOpt@break@\HOLOGO@name\endcsname
    \fi
  \fi
}
%    \end{macrocode}
%    \end{macro}
%
%    \begin{macro}{\HOLOGO@true}
%    \begin{macrocode}
\def\HOLOGO@true{true}
%    \end{macrocode}
%    \end{macro}
%    \begin{macro}{\HOLOGO@false}
%    \begin{macrocode}
\def\HOLOGO@false{false}
%    \end{macrocode}
%    \end{macro}
%    \begin{macro}{\HOLOGO@break}
%    \begin{macrocode}
\def\HOLOGO@break{break}
%    \end{macrocode}
%    \end{macro}
%
%    \begin{macrocode}
\HOLOGO@DeclareBoolOption{break}
\HOLOGO@DeclareBoolOption{hyphenbreak}
\HOLOGO@DeclareBoolOption{spacebreak}
\HOLOGO@DeclareBoolOption{discretionarybreak}
%    \end{macrocode}
%
%    \begin{macrocode}
\kv@define@key{HoLogo}{variant}{%
  \ifx\HOLOGO@name\relax
    \@PackageError{hologo}{%
      Option `variant' is not available in \string\hologoSetup,%
      \MessageBreak
      Use \string\hologoLogoSetup\space instead%
    }\@ehc
  \else
    \edef\HOLOGO@temp{#1}%
    \ifx\HOLOGO@temp\ltx@empty
      \expandafter
      \let\csname HoLogoOpt@variant@\HOLOGO@name\endcsname\@undefined
    \else
      \ltx@IfUndefined{HoLogo@\HOLOGO@name @\HOLOGO@temp}{%
        \@PackageError{hologo}{%
          Unknown variant `\HOLOGO@temp' of logo `\HOLOGO@name'%
        }\@ehc
      }{%
        \expandafter
        \let\csname HoLogoOpt@variant@\HOLOGO@name\endcsname
            \HOLOGO@temp
      }%
    \fi
  \fi
}
%    \end{macrocode}
%
%    \begin{macro}{\HOLOGO@Variant}
%    \begin{macrocode}
\def\HOLOGO@Variant#1{%
  #1%
  \ltx@ifundefined{HoLogoOpt@variant@#1}{%
  }{%
    @\csname HoLogoOpt@variant@#1\endcsname
  }%
}
%    \end{macrocode}
%    \end{macro}
%
% \subsection{Break/no-break support}
%
%    \begin{macro}{\HOLOGO@space}
%    \begin{macrocode}
\def\HOLOGO@space{%
  \ltx@ifundefined{HoLogoOpt@spacebreak@\HOLOGO@name}{%
    \ltx@ifundefined{HoLogoOpt@break@\HOLOGO@name}{%
      \chardef\HOLOGO@temp=\HOLOGOOPT@spacebreak
    }{%
      \chardef\HOLOGO@temp=%
        \csname HoLogoOpt@break@\HOLOGO@name\endcsname
    }%
  }{%
    \chardef\HOLOGO@temp=%
      \csname HoLogoOpt@spacebreak@\HOLOGO@name\endcsname
  }%
  \ifcase\HOLOGO@temp
    \penalty10000 %
  \fi
  \ltx@space
}
%    \end{macrocode}
%    \end{macro}
%
%    \begin{macro}{\HOLOGO@hyphen}
%    \begin{macrocode}
\def\HOLOGO@hyphen{%
  \ltx@ifundefined{HoLogoOpt@hyphenbreak@\HOLOGO@name}{%
    \ltx@ifundefined{HoLogoOpt@break@\HOLOGO@name}{%
      \chardef\HOLOGO@temp=\HOLOGOOPT@hyphenbreak
    }{%
      \chardef\HOLOGO@temp=%
        \csname HoLogoOpt@break@\HOLOGO@name\endcsname
    }%
  }{%
    \chardef\HOLOGO@temp=%
      \csname HoLogoOpt@hyphenbreak@\HOLOGO@name\endcsname
  }%
  \ifcase\HOLOGO@temp
    \ltx@mbox{-}%
  \else
    -%
  \fi
}
%    \end{macrocode}
%    \end{macro}
%
%    \begin{macro}{\HOLOGO@discretionary}
%    \begin{macrocode}
\def\HOLOGO@discretionary{%
  \ltx@ifundefined{HoLogoOpt@discretionarybreak@\HOLOGO@name}{%
    \ltx@ifundefined{HoLogoOpt@break@\HOLOGO@name}{%
      \chardef\HOLOGO@temp=\HOLOGOOPT@discretionarybreak
    }{%
      \chardef\HOLOGO@temp=%
        \csname HoLogoOpt@break@\HOLOGO@name\endcsname
    }%
  }{%
    \chardef\HOLOGO@temp=%
      \csname HoLogoOpt@discretionarybreak@\HOLOGO@name\endcsname
  }%
  \ifcase\HOLOGO@temp
  \else
    \-%
  \fi
}
%    \end{macrocode}
%    \end{macro}
%
%    \begin{macro}{\HOLOGO@mbox}
%    \begin{macrocode}
\def\HOLOGO@mbox#1{%
  \ltx@ifundefined{HoLogoOpt@break@\HOLOGO@name}{%
    \chardef\HOLOGO@temp=\HOLOGOOPT@hyphenbreak
  }{%
    \chardef\HOLOGO@temp=%
      \csname HoLogoOpt@break@\HOLOGO@name\endcsname
  }%
  \ifcase\HOLOGO@temp
    \ltx@mbox{#1}%
  \else
    #1%
  \fi
}
%    \end{macrocode}
%    \end{macro}
%
% \subsection{Font support}
%
%    \begin{macro}{\HoLogoFont@font}
%    \begin{tabular}{@{}ll@{}}
%    |#1|:& logo name\\
%    |#2|:& font short name\\
%    |#3|:& text
%    \end{tabular}
%    \begin{macrocode}
\def\HoLogoFont@font#1#2#3{%
  \begingroup
    \ltx@IfUndefined{HoLogoFont@logo@#1.#2}{%
      \ltx@IfUndefined{HoLogoFont@font@#2}{%
        \@PackageWarning{hologo}{%
          Missing font `#2' for logo `#1'%
        }%
        #3%
      }{%
        \csname HoLogoFont@font@#2\endcsname{#3}%
      }%
    }{%
      \csname HoLogoFont@logo@#1.#2\endcsname{#3}%
    }%
  \endgroup
}
%    \end{macrocode}
%    \end{macro}
%
%    \begin{macro}{\HoLogoFont@Def}
%    \begin{macrocode}
\def\HoLogoFont@Def#1{%
  \expandafter\def\csname HoLogoFont@font@#1\endcsname
}
%    \end{macrocode}
%    \end{macro}
%    \begin{macro}{\HoLogoFont@LogoDef}
%    \begin{macrocode}
\def\HoLogoFont@LogoDef#1#2{%
  \expandafter\def\csname HoLogoFont@logo@#1.#2\endcsname
}
%    \end{macrocode}
%    \end{macro}
%
% \subsubsection{Font defaults}
%
%    \begin{macro}{\HoLogoFont@font@general}
%    \begin{macrocode}
\HoLogoFont@Def{general}{}%
%    \end{macrocode}
%    \end{macro}
%
%    \begin{macro}{\HoLogoFont@font@rm}
%    \begin{macrocode}
\ltx@IfUndefined{rmfamily}{%
  \ltx@IfUndefined{rm}{%
  }{%
    \HoLogoFont@Def{rm}{\rm}%
  }%
}{%
  \HoLogoFont@Def{rm}{\rmfamily}%
}
%    \end{macrocode}
%    \end{macro}
%
%    \begin{macro}{\HoLogoFont@font@sf}
%    \begin{macrocode}
\ltx@IfUndefined{sffamily}{%
  \ltx@IfUndefined{sf}{%
  }{%
    \HoLogoFont@Def{sf}{\sf}%
  }%
}{%
  \HoLogoFont@Def{sf}{\sffamily}%
}
%    \end{macrocode}
%    \end{macro}
%
%    \begin{macro}{\HoLogoFont@font@bibsf}
%    In case of \hologo{plainTeX} the original small caps
%    variant is used as default. In \hologo{LaTeX}
%    the definition of package \xpackage{dtklogos} \cite{dtklogos}
%    is used.
%\begin{quote}
%\begin{verbatim}
%\DeclareRobustCommand{\BibTeX}{%
%  B%
%  \kern-.05em%
%  \hbox{%
%    $\m@th$% %% force math size calculations
%    \csname S@\f@size\endcsname
%    \fontsize\sf@size\z@
%    \math@fontsfalse
%    \selectfont
%    I%
%    \kern-.025em%
%    B
%  }%
%  \kern-.08em%
%  \-%
%  \TeX
%}
%\end{verbatim}
%\end{quote}
%    \begin{macrocode}
\ltx@IfUndefined{selectfont}{%
  \ltx@IfUndefined{tensc}{%
    \font\tensc=cmcsc10\relax
  }{}%
  \HoLogoFont@Def{bibsf}{\tensc}%
}{%
  \HoLogoFont@Def{bibsf}{%
    $\mathsurround=0pt$%
    \csname S@\f@size\endcsname
    \fontsize\sf@size{0pt}%
    \math@fontsfalse
    \selectfont
  }%
}
%    \end{macrocode}
%    \end{macro}
%
%    \begin{macro}{\HoLogoFont@font@sc}
%    \begin{macrocode}
\ltx@IfUndefined{scshape}{%
  \ltx@IfUndefined{tensc}{%
    \font\tensc=cmcsc10\relax
  }{}%
  \HoLogoFont@Def{sc}{\tensc}%
}{%
  \HoLogoFont@Def{sc}{\scshape}%
}
%    \end{macrocode}
%    \end{macro}
%
%    \begin{macro}{\HoLogoFont@font@sy}
%    \begin{macrocode}
\ltx@IfUndefined{usefont}{%
  \ltx@IfUndefined{tensy}{%
  }{%
    \HoLogoFont@Def{sy}{\tensy}%
  }%
}{%
  \HoLogoFont@Def{sy}{%
    \usefont{OMS}{cmsy}{m}{n}%
  }%
}
%    \end{macrocode}
%    \end{macro}
%
%    \begin{macro}{\HoLogoFont@font@logo}
%    \begin{macrocode}
\begingroup
  \def\x{LaTeX2e}%
\expandafter\endgroup
\ifx\fmtname\x
  \ltx@IfUndefined{logofamily}{%
    \DeclareRobustCommand\logofamily{%
      \not@math@alphabet\logofamily\relax
      \fontencoding{U}%
      \fontfamily{logo}%
      \selectfont
    }%
  }{}%
  \ltx@IfUndefined{logofamily}{%
  }{%
    \HoLogoFont@Def{logo}{\logofamily}%
  }%
\else
  \ltx@IfUndefined{tenlogo}{%
    \font\tenlogo=logo10\relax
  }{}%
  \HoLogoFont@Def{logo}{\tenlogo}%
\fi
%    \end{macrocode}
%    \end{macro}
%
% \subsubsection{Font setup}
%
%    \begin{macro}{\hologoFontSetup}
%    \begin{macrocode}
\def\hologoFontSetup{%
  \let\HOLOGO@name\relax
  \HOLOGO@FontSetup
}
%    \end{macrocode}
%    \end{macro}
%
%    \begin{macro}{\hologoLogoFontSetup}
%    \begin{macrocode}
\def\hologoLogoFontSetup#1{%
  \edef\HOLOGO@name{#1}%
  \ltx@IfUndefined{HoLogo@\HOLOGO@name}{%
    \@PackageError{hologo}{%
      Unknown logo `\HOLOGO@name'%
    }\@ehc
    \ltx@gobble
  }{%
    \HOLOGO@FontSetup
  }%
}
%    \end{macrocode}
%    \end{macro}
%
%    \begin{macro}{\HOLOGO@FontSetup}
%    \begin{macrocode}
\def\HOLOGO@FontSetup{%
  \kvsetkeys{HoLogoFont}%
}
%    \end{macrocode}
%    \end{macro}
%
%    \begin{macrocode}
\def\HOLOGO@temp#1{%
  \kv@define@key{HoLogoFont}{#1}{%
    \ifx\HOLOGO@name\relax
      \HoLogoFont@Def{#1}{##1}%
    \else
      \HoLogoFont@LogoDef\HOLOGO@name{#1}{##1}%
    \fi
  }%
}
\HOLOGO@temp{general}
\HOLOGO@temp{sf}
%    \end{macrocode}
%
% \subsection{Generic logo commands}
%
%    \begin{macrocode}
\HOLOGO@IfExists\hologo{%
  \@PackageError{hologo}{%
    \string\hologo\ltx@space is already defined.\MessageBreak
    Package loading is aborted%
  }\@ehc
  \HOLOGO@AtEnd
}%
\HOLOGO@IfExists\hologoRobust{%
  \@PackageError{hologo}{%
    \string\hologoRobust\ltx@space is already defined.\MessageBreak
    Package loading is aborted%
  }\@ehc
  \HOLOGO@AtEnd
}%
%    \end{macrocode}
%
% \subsubsection{\cs{hologo} and friends}
%
%    \begin{macrocode}
\ifluatex
  \expandafter\ltx@firstofone
\else
  \expandafter\ltx@gobble
\fi
{%
  \ltx@IfUndefined{ifincsname}{%
    \ifnum\luatexversion<36 %
      \expandafter\ltx@gobble
    \else
      \expandafter\ltx@firstofone
    \fi
    {%
      \begingroup
        \ifcase0%
            \directlua{%
              if tex.enableprimitives then %
                tex.enableprimitives('HOLOGO@', {'ifincsname'})%
              else %
                tex.print('1')%
              end%
            }%
            \ifx\HOLOGO@ifincsname\@undefined 1\fi%
            \relax
          \expandafter\ltx@firstofone
        \else
          \endgroup
          \expandafter\ltx@gobble
        \fi
        {%
          \global\let\ifincsname\HOLOGO@ifincsname
        }%
      \HOLOGO@temp
    }%
  }{}%
}
%    \end{macrocode}
%    \begin{macrocode}
\ltx@IfUndefined{ifincsname}{%
  \catcode`$=14 %
}{%
  \catcode`$=9 %
}
%    \end{macrocode}
%
%    \begin{macro}{\hologo}
%    \begin{macrocode}
\def\hologo#1{%
$ \ifincsname
$   \ltx@ifundefined{HoLogoCs@\HOLOGO@Variant{#1}}{%
$     #1%
$   }{%
$     \csname HoLogoCs@\HOLOGO@Variant{#1}\endcsname\ltx@firstoftwo
$   }%
$ \else
    \HOLOGO@IfExists\texorpdfstring\texorpdfstring\ltx@firstoftwo
    {%
      \hologoRobust{#1}%
    }{%
      \ltx@ifundefined{HoLogoBkm@\HOLOGO@Variant{#1}}{%
        \ltx@ifundefined{HoLogo@#1}{?#1?}{#1}%
      }{%
        \csname HoLogoBkm@\HOLOGO@Variant{#1}\endcsname
        \ltx@firstoftwo
      }%
    }%
$ \fi
}
%    \end{macrocode}
%    \end{macro}
%    \begin{macro}{\Hologo}
%    \begin{macrocode}
\def\Hologo#1{%
$ \ifincsname
$   \ltx@ifundefined{HoLogoCs@\HOLOGO@Variant{#1}}{%
$     #1%
$   }{%
$     \csname HoLogoCs@\HOLOGO@Variant{#1}\endcsname\ltx@secondoftwo
$   }%
$ \else
    \HOLOGO@IfExists\texorpdfstring\texorpdfstring\ltx@firstoftwo
    {%
      \HologoRobust{#1}%
    }{%
      \ltx@ifundefined{HoLogoBkm@\HOLOGO@Variant{#1}}{%
        \ltx@ifundefined{HoLogo@#1}{?#1?}{#1}%
      }{%
        \csname HoLogoBkm@\HOLOGO@Variant{#1}\endcsname
        \ltx@secondoftwo
      }%
    }%
$ \fi
}
%    \end{macrocode}
%    \end{macro}
%
%    \begin{macro}{\hologoVariant}
%    \begin{macrocode}
\def\hologoVariant#1#2{%
  \ifx\relax#2\relax
    \hologo{#1}%
  \else
$   \ifincsname
$     \ltx@ifundefined{HoLogoCs@#1@#2}{%
$       #1%
$     }{%
$       \csname HoLogoCs@#1@#2\endcsname\ltx@firstoftwo
$     }%
$   \else
      \HOLOGO@IfExists\texorpdfstring\texorpdfstring\ltx@firstoftwo
      {%
        \hologoVariantRobust{#1}{#2}%
      }{%
        \ltx@ifundefined{HoLogoBkm@#1@#2}{%
          \ltx@ifundefined{HoLogo@#1}{?#1?}{#1}%
        }{%
          \csname HoLogoBkm@#1@#2\endcsname
          \ltx@firstoftwo
        }%
      }%
$   \fi
  \fi
}
%    \end{macrocode}
%    \end{macro}
%    \begin{macro}{\HologoVariant}
%    \begin{macrocode}
\def\HologoVariant#1#2{%
  \ifx\relax#2\relax
    \Hologo{#1}%
  \else
$   \ifincsname
$     \ltx@ifundefined{HoLogoCs@#1@#2}{%
$       #1%
$     }{%
$       \csname HoLogoCs@#1@#2\endcsname\ltx@secondoftwo
$     }%
$   \else
      \HOLOGO@IfExists\texorpdfstring\texorpdfstring\ltx@firstoftwo
      {%
        \HologoVariantRobust{#1}{#2}%
      }{%
        \ltx@ifundefined{HoLogoBkm@#1@#2}{%
          \ltx@ifundefined{HoLogo@#1}{?#1?}{#1}%
        }{%
          \csname HoLogoBkm@#1@#2\endcsname
          \ltx@secondoftwo
        }%
      }%
$   \fi
  \fi
}
%    \end{macrocode}
%    \end{macro}
%
%    \begin{macrocode}
\catcode`\$=3 %
%    \end{macrocode}
%
% \subsubsection{\cs{hologoRobust} and friends}
%
%    \begin{macro}{\hologoRobust}
%    \begin{macrocode}
\ltx@IfUndefined{protected}{%
  \ltx@IfUndefined{DeclareRobustCommand}{%
    \def\hologoRobust#1%
  }{%
    \DeclareRobustCommand*\hologoRobust[1]%
  }%
}{%
  \protected\def\hologoRobust#1%
}%
{%
  \edef\HOLOGO@name{#1}%
  \ltx@IfUndefined{HoLogo@\HOLOGO@Variant\HOLOGO@name}{%
    \@PackageError{hologo}{%
      Unknown logo `\HOLOGO@name'%
    }\@ehc
    ?\HOLOGO@name?%
  }{%
    \ltx@IfUndefined{ver@tex4ht.sty}{%
      \HoLogoFont@font\HOLOGO@name{general}{%
        \csname HoLogo@\HOLOGO@Variant\HOLOGO@name\endcsname
        \ltx@firstoftwo
      }%
    }{%
      \ltx@IfUndefined{HoLogoHtml@\HOLOGO@Variant\HOLOGO@name}{%
        \HOLOGO@name
      }{%
        \csname HoLogoHtml@\HOLOGO@Variant\HOLOGO@name\endcsname
        \ltx@firstoftwo
      }%
    }%
  }%
}
%    \end{macrocode}
%    \end{macro}
%    \begin{macro}{\HologoRobust}
%    \begin{macrocode}
\ltx@IfUndefined{protected}{%
  \ltx@IfUndefined{DeclareRobustCommand}{%
    \def\HologoRobust#1%
  }{%
    \DeclareRobustCommand*\HologoRobust[1]%
  }%
}{%
  \protected\def\HologoRobust#1%
}%
{%
  \edef\HOLOGO@name{#1}%
  \ltx@IfUndefined{HoLogo@\HOLOGO@Variant\HOLOGO@name}{%
    \@PackageError{hologo}{%
      Unknown logo `\HOLOGO@name'%
    }\@ehc
    ?\HOLOGO@name?%
  }{%
    \ltx@IfUndefined{ver@tex4ht.sty}{%
      \HoLogoFont@font\HOLOGO@name{general}{%
        \csname HoLogo@\HOLOGO@Variant\HOLOGO@name\endcsname
        \ltx@secondoftwo
      }%
    }{%
      \ltx@IfUndefined{HoLogoHtml@\HOLOGO@Variant\HOLOGO@name}{%
        \expandafter\HOLOGO@Uppercase\HOLOGO@name
      }{%
        \csname HoLogoHtml@\HOLOGO@Variant\HOLOGO@name\endcsname
        \ltx@secondoftwo
      }%
    }%
  }%
}
%    \end{macrocode}
%    \end{macro}
%    \begin{macro}{\hologoVariantRobust}
%    \begin{macrocode}
\ltx@IfUndefined{protected}{%
  \ltx@IfUndefined{DeclareRobustCommand}{%
    \def\hologoVariantRobust#1#2%
  }{%
    \DeclareRobustCommand*\hologoVariantRobust[2]%
  }%
}{%
  \protected\def\hologoVariantRobust#1#2%
}%
{%
  \begingroup
    \hologoLogoSetup{#1}{variant={#2}}%
    \hologoRobust{#1}%
  \endgroup
}
%    \end{macrocode}
%    \end{macro}
%    \begin{macro}{\HologoVariantRobust}
%    \begin{macrocode}
\ltx@IfUndefined{protected}{%
  \ltx@IfUndefined{DeclareRobustCommand}{%
    \def\HologoVariantRobust#1#2%
  }{%
    \DeclareRobustCommand*\HologoVariantRobust[2]%
  }%
}{%
  \protected\def\HologoVariantRobust#1#2%
}%
{%
  \begingroup
    \hologoLogoSetup{#1}{variant={#2}}%
    \HologoRobust{#1}%
  \endgroup
}
%    \end{macrocode}
%    \end{macro}
%
%    \begin{macro}{\hologorobust}
%    Macro \cs{hologorobust} is only defined for compatibility.
%    Its use is deprecated.
%    \begin{macrocode}
\def\hologorobust{\hologoRobust}
%    \end{macrocode}
%    \end{macro}
%
% \subsection{Helpers}
%
%    \begin{macro}{\HOLOGO@Uppercase}
%    Macro \cs{HOLOGO@Uppercase} is restricted to \cs{uppercase},
%    because \hologo{plainTeX} or \hologo{iniTeX} do not provide
%    \cs{MakeUppercase}.
%    \begin{macrocode}
\def\HOLOGO@Uppercase#1{\uppercase{#1}}
%    \end{macrocode}
%    \end{macro}
%
%    \begin{macro}{\HOLOGO@PdfdocUnicode}
%    \begin{macrocode}
\def\HOLOGO@PdfdocUnicode{%
  \ifx\ifHy@unicode\iftrue
    \expandafter\ltx@secondoftwo
  \else
    \expandafter\ltx@firstoftwo
  \fi
}
%    \end{macrocode}
%    \end{macro}
%
%    \begin{macro}{\HOLOGO@Math}
%    \begin{macrocode}
\def\HOLOGO@MathSetup{%
  \mathsurround0pt\relax
  \HOLOGO@IfExists\f@series{%
    \if b\expandafter\ltx@car\f@series x\@nil
      \csname boldmath\endcsname
   \fi
  }{}%
}
%    \end{macrocode}
%    \end{macro}
%
%    \begin{macro}{\HOLOGO@TempDimen}
%    \begin{macrocode}
\dimendef\HOLOGO@TempDimen=\ltx@zero
%    \end{macrocode}
%    \end{macro}
%    \begin{macro}{\HOLOGO@NegativeKerning}
%    \begin{macrocode}
\def\HOLOGO@NegativeKerning#1{%
  \begingroup
    \HOLOGO@TempDimen=0pt\relax
    \comma@parse@normalized{#1}{%
      \ifdim\HOLOGO@TempDimen=0pt %
        \expandafter\HOLOGO@@NegativeKerning\comma@entry
      \fi
      \ltx@gobble
    }%
    \ifdim\HOLOGO@TempDimen<0pt %
      \kern\HOLOGO@TempDimen
    \fi
  \endgroup
}
%    \end{macrocode}
%    \end{macro}
%    \begin{macro}{\HOLOGO@@NegativeKerning}
%    \begin{macrocode}
\def\HOLOGO@@NegativeKerning#1#2{%
  \setbox\ltx@zero\hbox{#1#2}%
  \HOLOGO@TempDimen=\wd\ltx@zero
  \setbox\ltx@zero\hbox{#1\kern0pt#2}%
  \advance\HOLOGO@TempDimen by -\wd\ltx@zero
}
%    \end{macrocode}
%    \end{macro}
%
%    \begin{macro}{\HOLOGO@SpaceFactor}
%    \begin{macrocode}
\def\HOLOGO@SpaceFactor{%
  \spacefactor1000 %
}
%    \end{macrocode}
%    \end{macro}
%
%    \begin{macro}{\HOLOGO@Span}
%    \begin{macrocode}
\def\HOLOGO@Span#1#2{%
  \HCode{<span class="HoLogo-#1">}%
  #2%
  \HCode{</span>}%
}
%    \end{macrocode}
%    \end{macro}
%
% \subsubsection{Text subscript}
%
%    \begin{macro}{\HOLOGO@SubScript}%
%    \begin{macrocode}
\def\HOLOGO@SubScript#1{%
  \ltx@IfUndefined{textsubscript}{%
    \ltx@IfUndefined{text}{%
      \ltx@mbox{%
        \mathsurround=0pt\relax
        $%
          _{%
            \ltx@IfUndefined{sf@size}{%
              \mathrm{#1}%
            }{%
              \mbox{%
                \fontsize\sf@size{0pt}\selectfont
                #1%
              }%
            }%
          }%
        $%
      }%
    }{%
      \ltx@mbox{%
        \mathsurround=0pt\relax
        $_{\text{#1}}$%
      }%
    }%
  }{%
    \textsubscript{#1}%
  }%
}
%    \end{macrocode}
%    \end{macro}
%
% \subsection{\hologo{TeX} and friends}
%
% \subsubsection{\hologo{TeX}}
%
%    \begin{macro}{\HoLogo@TeX}
%    Source: \hologo{LaTeX} kernel.
%    \begin{macrocode}
\def\HoLogo@TeX#1{%
  T\kern-.1667em\lower.5ex\hbox{E}\kern-.125emX\HOLOGO@SpaceFactor
}
%    \end{macrocode}
%    \end{macro}
%    \begin{macro}{\HoLogoHtml@TeX}
%    \begin{macrocode}
\def\HoLogoHtml@TeX#1{%
  \HoLogoCss@TeX
  \HOLOGO@Span{TeX}{%
    T%
    \HOLOGO@Span{e}{%
      E%
    }%
    X%
  }%
}
%    \end{macrocode}
%    \end{macro}
%    \begin{macro}{\HoLogoCss@TeX}
%    \begin{macrocode}
\def\HoLogoCss@TeX{%
  \Css{%
    span.HoLogo-TeX span.HoLogo-e{%
      position:relative;%
      top:.5ex;%
      margin-left:-.1667em;%
      margin-right:-.125em;%
    }%
  }%
  \Css{%
    a span.HoLogo-TeX span.HoLogo-e{%
      text-decoration:none;%
    }%
  }%
  \global\let\HoLogoCss@TeX\relax
}
%    \end{macrocode}
%    \end{macro}
%
% \subsubsection{\hologo{plainTeX}}
%
%    \begin{macro}{\HoLogo@plainTeX@space}
%    Source: ``The \hologo{TeX}book''
%    \begin{macrocode}
\def\HoLogo@plainTeX@space#1{%
  \HOLOGO@mbox{#1{p}{P}lain}\HOLOGO@space\hologo{TeX}%
}
%    \end{macrocode}
%    \end{macro}
%    \begin{macro}{\HoLogoCs@plainTeX@space}
%    \begin{macrocode}
\def\HoLogoCs@plainTeX@space#1{#1{p}{P}lain TeX}%
%    \end{macrocode}
%    \end{macro}
%    \begin{macro}{\HoLogoBkm@plainTeX@space}
%    \begin{macrocode}
\def\HoLogoBkm@plainTeX@space#1{%
  #1{p}{P}lain \hologo{TeX}%
}
%    \end{macrocode}
%    \end{macro}
%    \begin{macro}{\HoLogoHtml@plainTeX@space}
%    \begin{macrocode}
\def\HoLogoHtml@plainTeX@space#1{%
  #1{p}{P}lain \hologo{TeX}%
}
%    \end{macrocode}
%    \end{macro}
%
%    \begin{macro}{\HoLogo@plainTeX@hyphen}
%    \begin{macrocode}
\def\HoLogo@plainTeX@hyphen#1{%
  \HOLOGO@mbox{#1{p}{P}lain}\HOLOGO@hyphen\hologo{TeX}%
}
%    \end{macrocode}
%    \end{macro}
%    \begin{macro}{\HoLogoCs@plainTeX@hyphen}
%    \begin{macrocode}
\def\HoLogoCs@plainTeX@hyphen#1{#1{p}{P}lain-TeX}
%    \end{macrocode}
%    \end{macro}
%    \begin{macro}{\HoLogoBkm@plainTeX@hyphen}
%    \begin{macrocode}
\def\HoLogoBkm@plainTeX@hyphen#1{%
  #1{p}{P}lain-\hologo{TeX}%
}
%    \end{macrocode}
%    \end{macro}
%    \begin{macro}{\HoLogoHtml@plainTeX@hyphen}
%    \begin{macrocode}
\def\HoLogoHtml@plainTeX@hyphen#1{%
  #1{p}{P}lain-\hologo{TeX}%
}
%    \end{macrocode}
%    \end{macro}
%
%    \begin{macro}{\HoLogo@plainTeX@runtogether}
%    \begin{macrocode}
\def\HoLogo@plainTeX@runtogether#1{%
  \HOLOGO@mbox{#1{p}{P}lain\hologo{TeX}}%
}
%    \end{macrocode}
%    \end{macro}
%    \begin{macro}{\HoLogoCs@plainTeX@runtogether}
%    \begin{macrocode}
\def\HoLogoCs@plainTeX@runtogether#1{#1{p}{P}lainTeX}
%    \end{macrocode}
%    \end{macro}
%    \begin{macro}{\HoLogoBkm@plainTeX@runtogether}
%    \begin{macrocode}
\def\HoLogoBkm@plainTeX@runtogether#1{%
  #1{p}{P}lain\hologo{TeX}%
}
%    \end{macrocode}
%    \end{macro}
%    \begin{macro}{\HoLogoHtml@plainTeX@runtogether}
%    \begin{macrocode}
\def\HoLogoHtml@plainTeX@runtogether#1{%
  #1{p}{P}lain\hologo{TeX}%
}
%    \end{macrocode}
%    \end{macro}
%
%    \begin{macro}{\HoLogo@plainTeX}
%    \begin{macrocode}
\def\HoLogo@plainTeX{\HoLogo@plainTeX@space}
%    \end{macrocode}
%    \end{macro}
%    \begin{macro}{\HoLogoCs@plainTeX}
%    \begin{macrocode}
\def\HoLogoCs@plainTeX{\HoLogoCs@plainTeX@space}
%    \end{macrocode}
%    \end{macro}
%    \begin{macro}{\HoLogoBkm@plainTeX}
%    \begin{macrocode}
\def\HoLogoBkm@plainTeX{\HoLogoBkm@plainTeX@space}
%    \end{macrocode}
%    \end{macro}
%    \begin{macro}{\HoLogoHtml@plainTeX}
%    \begin{macrocode}
\def\HoLogoHtml@plainTeX{\HoLogoHtml@plainTeX@space}
%    \end{macrocode}
%    \end{macro}
%
% \subsubsection{\hologo{LaTeX}}
%
%    Source: \hologo{LaTeX} kernel.
%\begin{quote}
%\begin{verbatim}
%\DeclareRobustCommand{\LaTeX}{%
%  L%
%  \kern-.36em%
%  {%
%    \sbox\z@ T%
%    \vbox to\ht\z@{%
%      \hbox{%
%        \check@mathfonts
%        \fontsize\sf@size\z@
%        \math@fontsfalse
%        \selectfont
%        A%
%      }%
%      \vss
%    }%
%  }%
%  \kern-.15em%
%  \TeX
%}
%\end{verbatim}
%\end{quote}
%
%    \begin{macro}{\HoLogo@La}
%    \begin{macrocode}
\def\HoLogo@La#1{%
  L%
  \kern-.36em%
  \begingroup
    \setbox\ltx@zero\hbox{T}%
    \vbox to\ht\ltx@zero{%
      \hbox{%
        \ltx@ifundefined{check@mathfonts}{%
          \csname sevenrm\endcsname
        }{%
          \check@mathfonts
          \fontsize\sf@size{0pt}%
          \math@fontsfalse\selectfont
        }%
        A%
      }%
      \vss
    }%
  \endgroup
}
%    \end{macrocode}
%    \end{macro}
%
%    \begin{macro}{\HoLogo@LaTeX}
%    Source: \hologo{LaTeX} kernel.
%    \begin{macrocode}
\def\HoLogo@LaTeX#1{%
  \hologo{La}%
  \kern-.15em%
  \hologo{TeX}%
}
%    \end{macrocode}
%    \end{macro}
%    \begin{macro}{\HoLogoHtml@LaTeX}
%    \begin{macrocode}
\def\HoLogoHtml@LaTeX#1{%
  \HoLogoCss@LaTeX
  \HOLOGO@Span{LaTeX}{%
    L%
    \HOLOGO@Span{a}{%
      A%
    }%
    \hologo{TeX}%
  }%
}
%    \end{macrocode}
%    \end{macro}
%    \begin{macro}{\HoLogoCss@LaTeX}
%    \begin{macrocode}
\def\HoLogoCss@LaTeX{%
  \Css{%
    span.HoLogo-LaTeX span.HoLogo-a{%
      position:relative;%
      top:-.5ex;%
      margin-left:-.36em;%
      margin-right:-.15em;%
      font-size:85\%;%
    }%
  }%
  \global\let\HoLogoCss@LaTeX\relax
}
%    \end{macrocode}
%    \end{macro}
%
% \subsubsection{\hologo{(La)TeX}}
%
%    \begin{macro}{\HoLogo@LaTeXTeX}
%    The kerning around the parentheses is taken
%    from package \xpackage{dtklogos} \cite{dtklogos}.
%\begin{quote}
%\begin{verbatim}
%\DeclareRobustCommand{\LaTeXTeX}{%
%  (%
%  \kern-.15em%
%  L%
%  \kern-.36em%
%  {%
%    \sbox\z@ T%
%    \vbox to\ht0{%
%      \hbox{%
%        $\m@th$%
%        \csname S@\f@size\endcsname
%        \fontsize\sf@size\z@
%        \math@fontsfalse
%        \selectfont
%        A%
%      }%
%      \vss
%    }%
%  }%
%  \kern-.2em%
%  )%
%  \kern-.15em%
%  \TeX
%}
%\end{verbatim}
%\end{quote}
%    \begin{macrocode}
\def\HoLogo@LaTeXTeX#1{%
  (%
  \kern-.15em%
  \hologo{La}%
  \kern-.2em%
  )%
  \kern-.15em%
  \hologo{TeX}%
}
%    \end{macrocode}
%    \end{macro}
%    \begin{macro}{\HoLogoBkm@LaTeXTeX}
%    \begin{macrocode}
\def\HoLogoBkm@LaTeXTeX#1{(La)TeX}
%    \end{macrocode}
%    \end{macro}
%
%    \begin{macro}{\HoLogo@(La)TeX}
%    \begin{macrocode}
\expandafter
\let\csname HoLogo@(La)TeX\endcsname\HoLogo@LaTeXTeX
%    \end{macrocode}
%    \end{macro}
%    \begin{macro}{\HoLogoBkm@(La)TeX}
%    \begin{macrocode}
\expandafter
\let\csname HoLogoBkm@(La)TeX\endcsname\HoLogoBkm@LaTeXTeX
%    \end{macrocode}
%    \end{macro}
%    \begin{macro}{\HoLogoHtml@LaTeXTeX}
%    \begin{macrocode}
\def\HoLogoHtml@LaTeXTeX#1{%
  \HoLogoCss@LaTeXTeX
  \HOLOGO@Span{LaTeXTeX}{%
    (%
    \HOLOGO@Span{L}{L}%
    \HOLOGO@Span{a}{A}%
    \HOLOGO@Span{ParenRight}{)}%
    \hologo{TeX}%
  }%
}
%    \end{macrocode}
%    \end{macro}
%    \begin{macro}{\HoLogoHtml@(La)TeX}
%    Kerning after opening parentheses and before closing parentheses
%    is $-0.1$\,em. The original values $-0.15$\,em
%    looked too ugly for a serif font.
%    \begin{macrocode}
\expandafter
\let\csname HoLogoHtml@(La)TeX\endcsname\HoLogoHtml@LaTeXTeX
%    \end{macrocode}
%    \end{macro}
%    \begin{macro}{\HoLogoCss@LaTeXTeX}
%    \begin{macrocode}
\def\HoLogoCss@LaTeXTeX{%
  \Css{%
    span.HoLogo-LaTeXTeX span.HoLogo-L{%
      margin-left:-.1em;%
    }%
  }%
  \Css{%
    span.HoLogo-LaTeXTeX span.HoLogo-a{%
      position:relative;%
      top:-.5ex;%
      margin-left:-.36em;%
      margin-right:-.1em;%
      font-size:85\%;%
    }%
  }%
  \Css{%
    span.HoLogo-LaTeXTeX span.HoLogo-ParenRight{%
      margin-right:-.15em;%
    }%
  }%
  \global\let\HoLogoCss@LaTeXTeX\relax
}
%    \end{macrocode}
%    \end{macro}
%
% \subsubsection{\hologo{LaTeXe}}
%
%    \begin{macro}{\HoLogo@LaTeXe}
%    Source: \hologo{LaTeX} kernel
%    \begin{macrocode}
\def\HoLogo@LaTeXe#1{%
  \hologo{LaTeX}%
  \kern.15em%
  \hbox{%
    \HOLOGO@MathSetup
    2%
    $_{\textstyle\varepsilon}$%
  }%
}
%    \end{macrocode}
%    \end{macro}
%
%    \begin{macro}{\HoLogoCs@LaTeXe}
%    \begin{macrocode}
\ifnum64=`\^^^^0040\relax % test for big chars of LuaTeX/XeTeX
  \catcode`\$=9 %
  \catcode`\&=14 %
\else
  \catcode`\$=14 %
  \catcode`\&=9 %
\fi
\def\HoLogoCs@LaTeXe#1{%
  LaTeX2%
$ \string ^^^^0395%
& e%
}%
\catcode`\$=3 %
\catcode`\&=4 %
%    \end{macrocode}
%    \end{macro}
%
%    \begin{macro}{\HoLogoBkm@LaTeXe}
%    \begin{macrocode}
\def\HoLogoBkm@LaTeXe#1{%
  \hologo{LaTeX}%
  2%
  \HOLOGO@PdfdocUnicode{e}{\textepsilon}%
}
%    \end{macrocode}
%    \end{macro}
%
%    \begin{macro}{\HoLogoHtml@LaTeXe}
%    \begin{macrocode}
\def\HoLogoHtml@LaTeXe#1{%
  \HoLogoCss@LaTeXe
  \HOLOGO@Span{LaTeX2e}{%
    \hologo{LaTeX}%
    \HOLOGO@Span{2}{2}%
    \HOLOGO@Span{e}{%
      \HOLOGO@MathSetup
      \ensuremath{\textstyle\varepsilon}%
    }%
  }%
}
%    \end{macrocode}
%    \end{macro}
%    \begin{macro}{\HoLogoCss@LaTeXe}
%    \begin{macrocode}
\def\HoLogoCss@LaTeXe{%
  \Css{%
    span.HoLogo-LaTeX2e span.HoLogo-2{%
      padding-left:.15em;%
    }%
  }%
  \Css{%
    span.HoLogo-LaTeX2e span.HoLogo-e{%
      position:relative;%
      top:.35ex;%
      text-decoration:none;%
    }%
  }%
  \global\let\HoLogoCss@LaTeXe\relax
}
%    \end{macrocode}
%    \end{macro}
%
%    \begin{macro}{\HoLogo@LaTeX2e}
%    \begin{macrocode}
\expandafter
\let\csname HoLogo@LaTeX2e\endcsname\HoLogo@LaTeXe
%    \end{macrocode}
%    \end{macro}
%    \begin{macro}{\HoLogoCs@LaTeX2e}
%    \begin{macrocode}
\expandafter
\let\csname HoLogoCs@LaTeX2e\endcsname\HoLogoCs@LaTeXe
%    \end{macrocode}
%    \end{macro}
%    \begin{macro}{\HoLogoBkm@LaTeX2e}
%    \begin{macrocode}
\expandafter
\let\csname HoLogoBkm@LaTeX2e\endcsname\HoLogoBkm@LaTeXe
%    \end{macrocode}
%    \end{macro}
%    \begin{macro}{\HoLogoHtml@LaTeX2e}
%    \begin{macrocode}
\expandafter
\let\csname HoLogoHtml@LaTeX2e\endcsname\HoLogoHtml@LaTeXe
%    \end{macrocode}
%    \end{macro}
%
% \subsubsection{\hologo{LaTeX3}}
%
%    \begin{macro}{\HoLogo@LaTeX3}
%    Source: \hologo{LaTeX} kernel
%    \begin{macrocode}
\expandafter\def\csname HoLogo@LaTeX3\endcsname#1{%
  \hologo{LaTeX}%
  3%
}
%    \end{macrocode}
%    \end{macro}
%
%    \begin{macro}{\HoLogoBkm@LaTeX3}
%    \begin{macrocode}
\expandafter\def\csname HoLogoBkm@LaTeX3\endcsname#1{%
  \hologo{LaTeX}%
  3%
}
%    \end{macrocode}
%    \end{macro}
%    \begin{macro}{\HoLogoHtml@LaTeX3}
%    \begin{macrocode}
\expandafter
\let\csname HoLogoHtml@LaTeX3\expandafter\endcsname
\csname HoLogo@LaTeX3\endcsname
%    \end{macrocode}
%    \end{macro}
%
% \subsubsection{\hologo{LaTeXML}}
%
%    \begin{macro}{\HoLogo@LaTeXML}
%    \begin{macrocode}
\def\HoLogo@LaTeXML#1{%
  \HOLOGO@mbox{%
    \hologo{La}%
    \kern-.15em%
    T%
    \kern-.1667em%
    \lower.5ex\hbox{E}%
    \kern-.125em%
    \HoLogoFont@font{LaTeXML}{sc}{xml}%
  }%
}
%    \end{macrocode}
%    \end{macro}
%    \begin{macro}{\HoLogoHtml@pdfLaTeX}
%    \begin{macrocode}
\def\HoLogoHtml@LaTeXML#1{%
  \HOLOGO@Span{LaTeXML}{%
    \HoLogoCss@LaTeX
    \HoLogoCss@TeX
    \HOLOGO@Span{LaTeX}{%
      L%
      \HOLOGO@Span{a}{%
        A%
      }%
    }%
    \HOLOGO@Span{TeX}{%
      T%
      \HOLOGO@Span{e}{%
        E%
      }%
    }%
    \HCode{<span style="font-variant: small-caps;">}%
    xml%
    \HCode{</span>}%
  }%
}
%    \end{macrocode}
%    \end{macro}
%
% \subsubsection{\hologo{eTeX}}
%
%    \begin{macro}{\HoLogo@eTeX}
%    Source: package \xpackage{etex}
%    \begin{macrocode}
\def\HoLogo@eTeX#1{%
  \ltx@mbox{%
    \HOLOGO@MathSetup
    $\varepsilon$%
    -%
    \HOLOGO@NegativeKerning{-T,T-,To}%
    \hologo{TeX}%
  }%
}
%    \end{macrocode}
%    \end{macro}
%    \begin{macro}{\HoLogoCs@eTeX}
%    \begin{macrocode}
\ifnum64=`\^^^^0040\relax % test for big chars of LuaTeX/XeTeX
  \catcode`\$=9 %
  \catcode`\&=14 %
\else
  \catcode`\$=14 %
  \catcode`\&=9 %
\fi
\def\HoLogoCs@eTeX#1{%
$ #1{\string ^^^^0395}{\string ^^^^03b5}%
& #1{e}{E}%
  TeX%
}%
\catcode`\$=3 %
\catcode`\&=4 %
%    \end{macrocode}
%    \end{macro}
%    \begin{macro}{\HoLogoBkm@eTeX}
%    \begin{macrocode}
\def\HoLogoBkm@eTeX#1{%
  \HOLOGO@PdfdocUnicode{#1{e}{E}}{\textepsilon}%
  -%
  \hologo{TeX}%
}
%    \end{macrocode}
%    \end{macro}
%    \begin{macro}{\HoLogoHtml@eTeX}
%    \begin{macrocode}
\def\HoLogoHtml@eTeX#1{%
  \ltx@mbox{%
    \HOLOGO@MathSetup
    $\varepsilon$%
    -%
    \hologo{TeX}%
  }%
}
%    \end{macrocode}
%    \end{macro}
%
% \subsubsection{\hologo{iniTeX}}
%
%    \begin{macro}{\HoLogo@iniTeX}
%    \begin{macrocode}
\def\HoLogo@iniTeX#1{%
  \HOLOGO@mbox{%
    #1{i}{I}ni\hologo{TeX}%
  }%
}
%    \end{macrocode}
%    \end{macro}
%    \begin{macro}{\HoLogoCs@iniTeX}
%    \begin{macrocode}
\def\HoLogoCs@iniTeX#1{#1{i}{I}niTeX}
%    \end{macrocode}
%    \end{macro}
%    \begin{macro}{\HoLogoBkm@iniTeX}
%    \begin{macrocode}
\def\HoLogoBkm@iniTeX#1{%
  #1{i}{I}ni\hologo{TeX}%
}
%    \end{macrocode}
%    \end{macro}
%    \begin{macro}{\HoLogoHtml@iniTeX}
%    \begin{macrocode}
\let\HoLogoHtml@iniTeX\HoLogo@iniTeX
%    \end{macrocode}
%    \end{macro}
%
% \subsubsection{\hologo{virTeX}}
%
%    \begin{macro}{\HoLogo@virTeX}
%    \begin{macrocode}
\def\HoLogo@virTeX#1{%
  \HOLOGO@mbox{%
    #1{v}{V}ir\hologo{TeX}%
  }%
}
%    \end{macrocode}
%    \end{macro}
%    \begin{macro}{\HoLogoCs@virTeX}
%    \begin{macrocode}
\def\HoLogoCs@virTeX#1{#1{v}{V}irTeX}
%    \end{macrocode}
%    \end{macro}
%    \begin{macro}{\HoLogoBkm@virTeX}
%    \begin{macrocode}
\def\HoLogoBkm@virTeX#1{%
  #1{v}{V}ir\hologo{TeX}%
}
%    \end{macrocode}
%    \end{macro}
%    \begin{macro}{\HoLogoHtml@virTeX}
%    \begin{macrocode}
\let\HoLogoHtml@virTeX\HoLogo@virTeX
%    \end{macrocode}
%    \end{macro}
%
% \subsubsection{\hologo{SliTeX}}
%
% \paragraph{Definitions of the three variants.}
%
%    \begin{macro}{\HoLogo@SLiTeX@lift}
%    \begin{macrocode}
\def\HoLogo@SLiTeX@lift#1{%
  \HoLogoFont@font{SliTeX}{rm}{%
    S%
    \kern-.06em%
    L%
    \kern-.18em%
    \raise.32ex\hbox{\HoLogoFont@font{SliTeX}{sc}{i}}%
    \HOLOGO@discretionary
    \kern-.06em%
    \hologo{TeX}%
  }%
}
%    \end{macrocode}
%    \end{macro}
%    \begin{macro}{\HoLogoBkm@SLiTeX@lift}
%    \begin{macrocode}
\def\HoLogoBkm@SLiTeX@lift#1{SLiTeX}
%    \end{macrocode}
%    \end{macro}
%    \begin{macro}{\HoLogoHtml@SLiTeX@lift}
%    \begin{macrocode}
\def\HoLogoHtml@SLiTeX@lift#1{%
  \HoLogoCss@SLiTeX@lift
  \HOLOGO@Span{SLiTeX-lift}{%
    \HoLogoFont@font{SliTeX}{rm}{%
      S%
      \HOLOGO@Span{L}{L}%
      \HOLOGO@Span{i}{i}%
      \hologo{TeX}%
    }%
  }%
}
%    \end{macrocode}
%    \end{macro}
%    \begin{macro}{\HoLogoCss@SLiTeX@lift}
%    \begin{macrocode}
\def\HoLogoCss@SLiTeX@lift{%
  \Css{%
    span.HoLogo-SLiTeX-lift span.HoLogo-L{%
      margin-left:-.06em;%
      margin-right:-.18em;%
    }%
  }%
  \Css{%
    span.HoLogo-SLiTeX-lift span.HoLogo-i{%
      position:relative;%
      top:-.32ex;%
      margin-right:-.06em;%
      font-variant:small-caps;%
    }%
  }%
  \global\let\HoLogoCss@SLiTeX@lift\relax
}
%    \end{macrocode}
%    \end{macro}
%
%    \begin{macro}{\HoLogo@SliTeX@simple}
%    \begin{macrocode}
\def\HoLogo@SliTeX@simple#1{%
  \HoLogoFont@font{SliTeX}{rm}{%
    \ltx@mbox{%
      \HoLogoFont@font{SliTeX}{sc}{Sli}%
    }%
    \HOLOGO@discretionary
    \hologo{TeX}%
  }%
}
%    \end{macrocode}
%    \end{macro}
%    \begin{macro}{\HoLogoBkm@SliTeX@simple}
%    \begin{macrocode}
\def\HoLogoBkm@SliTeX@simple#1{SliTeX}
%    \end{macrocode}
%    \end{macro}
%    \begin{macro}{\HoLogoHtml@SliTeX@simple}
%    \begin{macrocode}
\let\HoLogoHtml@SliTeX@simple\HoLogo@SliTeX@simple
%    \end{macrocode}
%    \end{macro}
%
%    \begin{macro}{\HoLogo@SliTeX@narrow}
%    \begin{macrocode}
\def\HoLogo@SliTeX@narrow#1{%
  \HoLogoFont@font{SliTeX}{rm}{%
    \ltx@mbox{%
      S%
      \kern-.06em%
      \HoLogoFont@font{SliTeX}{sc}{%
        l%
        \kern-.035em%
        i%
      }%
    }%
    \HOLOGO@discretionary
    \kern-.06em%
    \hologo{TeX}%
  }%
}
%    \end{macrocode}
%    \end{macro}
%    \begin{macro}{\HoLogoBkm@SliTeX@narrow}
%    \begin{macrocode}
\def\HoLogoBkm@SliTeX@narrow#1{SliTeX}
%    \end{macrocode}
%    \end{macro}
%    \begin{macro}{\HoLogoHtml@SliTeX@narrow}
%    \begin{macrocode}
\def\HoLogoHtml@SliTeX@narrow#1{%
  \HoLogoCss@SliTeX@narrow
  \HOLOGO@Span{SliTeX-narrow}{%
    \HoLogoFont@font{SliTeX}{rm}{%
      S%
        \HOLOGO@Span{l}{l}%
        \HOLOGO@Span{i}{i}%
      \hologo{TeX}%
    }%
  }%
}
%    \end{macrocode}
%    \end{macro}
%    \begin{macro}{\HoLogoCss@SliTeX@narrow}
%    \begin{macrocode}
\def\HoLogoCss@SliTeX@narrow{%
  \Css{%
    span.HoLogo-SliTeX-narrow span.HoLogo-l{%
      margin-left:-.06em;%
      margin-right:-.035em;%
      font-variant:small-caps;%
    }%
  }%
  \Css{%
    span.HoLogo-SliTeX-narrow span.HoLogo-i{%
      margin-right:-.06em;%
      font-variant:small-caps;%
    }%
  }%
  \global\let\HoLogoCss@SliTeX@narrow\relax
}
%    \end{macrocode}
%    \end{macro}
%
% \paragraph{Macro set completion.}
%
%    \begin{macro}{\HoLogo@SLiTeX@simple}
%    \begin{macrocode}
\def\HoLogo@SLiTeX@simple{\HoLogo@SliTeX@simple}
%    \end{macrocode}
%    \end{macro}
%    \begin{macro}{\HoLogoBkm@SLiTeX@simple}
%    \begin{macrocode}
\def\HoLogoBkm@SLiTeX@simple{\HoLogoBkm@SliTeX@simple}
%    \end{macrocode}
%    \end{macro}
%    \begin{macro}{\HoLogoHtml@SLiTeX@simple}
%    \begin{macrocode}
\def\HoLogoHtml@SLiTeX@simple{\HoLogoHtml@SliTeX@simple}
%    \end{macrocode}
%    \end{macro}
%
%    \begin{macro}{\HoLogo@SLiTeX@narrow}
%    \begin{macrocode}
\def\HoLogo@SLiTeX@narrow{\HoLogo@SliTeX@narrow}
%    \end{macrocode}
%    \end{macro}
%    \begin{macro}{\HoLogoBkm@SLiTeX@narrow}
%    \begin{macrocode}
\def\HoLogoBkm@SLiTeX@narrow{\HoLogoBkm@SliTeX@narrow}
%    \end{macrocode}
%    \end{macro}
%    \begin{macro}{\HoLogoHtml@SLiTeX@narrow}
%    \begin{macrocode}
\def\HoLogoHtml@SLiTeX@narrow{\HoLogoHtml@SliTeX@narrow}
%    \end{macrocode}
%    \end{macro}
%
%    \begin{macro}{\HoLogo@SliTeX@lift}
%    \begin{macrocode}
\def\HoLogo@SliTeX@lift{\HoLogo@SLiTeX@lift}
%    \end{macrocode}
%    \end{macro}
%    \begin{macro}{\HoLogoBkm@SliTeX@lift}
%    \begin{macrocode}
\def\HoLogoBkm@SliTeX@lift{\HoLogoBkm@SLiTeX@lift}
%    \end{macrocode}
%    \end{macro}
%    \begin{macro}{\HoLogoHtml@SliTeX@lift}
%    \begin{macrocode}
\def\HoLogoHtml@SliTeX@lift{\HoLogoHtml@SLiTeX@lift}
%    \end{macrocode}
%    \end{macro}
%
% \paragraph{Defaults.}
%
%    \begin{macro}{\HoLogo@SLiTeX}
%    \begin{macrocode}
\def\HoLogo@SLiTeX{\HoLogo@SLiTeX@lift}
%    \end{macrocode}
%    \end{macro}
%    \begin{macro}{\HoLogoBkm@SLiTeX}
%    \begin{macrocode}
\def\HoLogoBkm@SLiTeX{\HoLogoBkm@SLiTeX@lift}
%    \end{macrocode}
%    \end{macro}
%    \begin{macro}{\HoLogoHtml@SLiTeX}
%    \begin{macrocode}
\def\HoLogoHtml@SLiTeX{\HoLogoHtml@SLiTeX@lift}
%    \end{macrocode}
%    \end{macro}
%
%    \begin{macro}{\HoLogo@SliTeX}
%    \begin{macrocode}
\def\HoLogo@SliTeX{\HoLogo@SliTeX@narrow}
%    \end{macrocode}
%    \end{macro}
%    \begin{macro}{\HoLogoBkm@SliTeX}
%    \begin{macrocode}
\def\HoLogoBkm@SliTeX{\HoLogoBkm@SliTeX@narrow}
%    \end{macrocode}
%    \end{macro}
%    \begin{macro}{\HoLogoHtml@SliTeX}
%    \begin{macrocode}
\def\HoLogoHtml@SliTeX{\HoLogoHtml@SliTeX@narrow}
%    \end{macrocode}
%    \end{macro}
%
% \subsubsection{\hologo{LuaTeX}}
%
%    \begin{macro}{\HoLogo@LuaTeX}
%    The kerning is an idea of Hans Hagen, see mailing list
%    `luatex at tug dot org' in March 2010.
%    \begin{macrocode}
\def\HoLogo@LuaTeX#1{%
  \HOLOGO@mbox{%
    Lua%
    \HOLOGO@NegativeKerning{aT,oT,To}%
    \hologo{TeX}%
  }%
}
%    \end{macrocode}
%    \end{macro}
%    \begin{macro}{\HoLogoHtml@LuaTeX}
%    \begin{macrocode}
\let\HoLogoHtml@LuaTeX\HoLogo@LuaTeX
%    \end{macrocode}
%    \end{macro}
%
% \subsubsection{\hologo{LuaLaTeX}}
%
%    \begin{macro}{\HoLogo@LuaLaTeX}
%    \begin{macrocode}
\def\HoLogo@LuaLaTeX#1{%
  \HOLOGO@mbox{%
    Lua%
    \hologo{LaTeX}%
  }%
}
%    \end{macrocode}
%    \end{macro}
%    \begin{macro}{\HoLogoHtml@LuaLaTeX}
%    \begin{macrocode}
\let\HoLogoHtml@LuaLaTeX\HoLogo@LuaLaTeX
%    \end{macrocode}
%    \end{macro}
%
% \subsubsection{\hologo{XeTeX}, \hologo{XeLaTeX}}
%
%    \begin{macro}{\HOLOGO@IfCharExists}
%    \begin{macrocode}
\ifluatex
  \ifnum\luatexversion<36 %
  \else
    \def\HOLOGO@IfCharExists#1{%
      \ifnum
        \directlua{%
           if luaotfload and luaotfload.aux then
             if luaotfload.aux.font_has_glyph(%
                    font.current(), \number#1) then % 	 
	       tex.print("1") % 	 
	     end % 	 
	   elseif font and font.fonts and font.current then %
            local f = font.fonts[font.current()]%
            if f.characters and f.characters[\number#1] then %
              tex.print("1")%
            end %
          end%
        }0=\ltx@zero
        \expandafter\ltx@secondoftwo
      \else
        \expandafter\ltx@firstoftwo
      \fi
    }%
  \fi
\fi
\ltx@IfUndefined{HOLOGO@IfCharExists}{%
  \def\HOLOGO@@IfCharExists#1{%
    \begingroup
      \tracinglostchars=\ltx@zero
      \setbox\ltx@zero=\hbox{%
        \kern7sp\char#1\relax
        \ifnum\lastkern>\ltx@zero
          \expandafter\aftergroup\csname iffalse\endcsname
        \else
          \expandafter\aftergroup\csname iftrue\endcsname
        \fi
      }%
      % \if{true|false} from \aftergroup
      \endgroup
      \expandafter\ltx@firstoftwo
    \else
      \endgroup
      \expandafter\ltx@secondoftwo
    \fi
  }%
  \ifxetex
    \ltx@IfUndefined{XeTeXfonttype}{}{%
      \ltx@IfUndefined{XeTeXcharglyph}{}{%
        \def\HOLOGO@IfCharExists#1{%
          \ifnum\XeTeXfonttype\font>\ltx@zero
            \expandafter\ltx@firstofthree
          \else
            \expandafter\ltx@gobble
          \fi
          {%
            \ifnum\XeTeXcharglyph#1>\ltx@zero
              \expandafter\ltx@firstoftwo
            \else
              \expandafter\ltx@secondoftwo
            \fi
          }%
          \HOLOGO@@IfCharExists{#1}%
        }%
      }%
    }%
  \fi
}{}
\ltx@ifundefined{HOLOGO@IfCharExists}{%
  \ifnum64=`\^^^^0040\relax % test for big chars of LuaTeX/XeTeX
    \let\HOLOGO@IfCharExists\HOLOGO@@IfCharExists
  \else
    \def\HOLOGO@IfCharExists#1{%
      \ifnum#1>255 %
        \expandafter\ltx@fourthoffour
      \fi
      \HOLOGO@@IfCharExists{#1}%
    }%
  \fi
}{}
%    \end{macrocode}
%    \end{macro}
%
%    \begin{macro}{\HoLogo@Xe}
%    Source: package \xpackage{dtklogos}
%    \begin{macrocode}
\def\HoLogo@Xe#1{%
  X%
  \kern-.1em\relax
  \HOLOGO@IfCharExists{"018E}{%
    \lower.5ex\hbox{\char"018E}%
  }{%
    \chardef\HOLOGO@choice=\ltx@zero
    \ifdim\fontdimen\ltx@one\font>0pt %
      \ltx@IfUndefined{rotatebox}{%
        \ltx@IfUndefined{pgftext}{%
          \ltx@IfUndefined{psscalebox}{%
            \ltx@IfUndefined{HOLOGO@ScaleBox@\hologoDriver}{%
            }{%
              \chardef\HOLOGO@choice=4 %
            }%
          }{%
            \chardef\HOLOGO@choice=3 %
          }%
        }{%
          \chardef\HOLOGO@choice=2 %
        }%
      }{%
        \chardef\HOLOGO@choice=1 %
      }%
      \ifcase\HOLOGO@choice
        \HOLOGO@WarningUnsupportedDriver{Xe}%
        e%
      \or % 1: \rotatebox
        \begingroup
          \setbox\ltx@zero\hbox{\rotatebox{180}{E}}%
          \ltx@LocDimenA=\dp\ltx@zero
          \advance\ltx@LocDimenA by -.5ex\relax
          \raise\ltx@LocDimenA\box\ltx@zero
        \endgroup
      \or % 2: \pgftext
        \lower.5ex\hbox{%
          \pgfpicture
            \pgftext[rotate=180]{E}%
          \endpgfpicture
        }%
      \or % 3: \psscalebox
        \begingroup
          \setbox\ltx@zero\hbox{\psscalebox{-1 -1}{E}}%
          \ltx@LocDimenA=\dp\ltx@zero
          \advance\ltx@LocDimenA by -.5ex\relax
          \raise\ltx@LocDimenA\box\ltx@zero
        \endgroup
      \or % 4: \HOLOGO@PointReflectBox
        \lower.5ex\hbox{\HOLOGO@PointReflectBox{E}}%
      \else
        \@PackageError{hologo}{Internal error (choice/it}\@ehc
      \fi
    \else
      \ltx@IfUndefined{reflectbox}{%
        \ltx@IfUndefined{pgftext}{%
          \ltx@IfUndefined{psscalebox}{%
            \ltx@IfUndefined{HOLOGO@ScaleBox@\hologoDriver}{%
            }{%
              \chardef\HOLOGO@choice=4 %
            }%
          }{%
            \chardef\HOLOGO@choice=3 %
          }%
        }{%
          \chardef\HOLOGO@choice=2 %
        }%
      }{%
        \chardef\HOLOGO@choice=1 %
      }%
      \ifcase\HOLOGO@choice
        \HOLOGO@WarningUnsupportedDriver{Xe}%
        e%
      \or % 1: reflectbox
        \lower.5ex\hbox{%
          \reflectbox{E}%
        }%
      \or % 2: \pgftext
        \lower.5ex\hbox{%
          \pgfpicture
            \pgftransformxscale{-1}%
            \pgftext{E}%
          \endpgfpicture
        }%
      \or % 3: \psscalebox
        \lower.5ex\hbox{%
          \psscalebox{-1 1}{E}%
        }%
      \or % 4: \HOLOGO@Reflectbox
        \lower.5ex\hbox{%
          \HOLOGO@ReflectBox{E}%
        }%
      \else
        \@PackageError{hologo}{Internal error (choice/up)}\@ehc
      \fi
    \fi
  }%
}
%    \end{macrocode}
%    \end{macro}
%    \begin{macro}{\HoLogoHtml@Xe}
%    \begin{macrocode}
\def\HoLogoHtml@Xe#1{%
  \HoLogoCss@Xe
  \HOLOGO@Span{Xe}{%
    X%
    \HOLOGO@Span{e}{%
      \HCode{&\ltx@hashchar x018e;}%
    }%
  }%
}
%    \end{macrocode}
%    \end{macro}
%    \begin{macro}{\HoLogoCss@Xe}
%    \begin{macrocode}
\def\HoLogoCss@Xe{%
  \Css{%
    span.HoLogo-Xe span.HoLogo-e{%
      position:relative;%
      top:.5ex;%
      left-margin:-.1em;%
    }%
  }%
  \global\let\HoLogoCss@Xe\relax
}
%    \end{macrocode}
%    \end{macro}
%
%    \begin{macro}{\HoLogo@XeTeX}
%    \begin{macrocode}
\def\HoLogo@XeTeX#1{%
  \hologo{Xe}%
  \kern-.15em\relax
  \hologo{TeX}%
}
%    \end{macrocode}
%    \end{macro}
%
%    \begin{macro}{\HoLogoHtml@XeTeX}
%    \begin{macrocode}
\def\HoLogoHtml@XeTeX#1{%
  \HoLogoCss@XeTeX
  \HOLOGO@Span{XeTeX}{%
    \hologo{Xe}%
    \hologo{TeX}%
  }%
}
%    \end{macrocode}
%    \end{macro}
%    \begin{macro}{\HoLogoCss@XeTeX}
%    \begin{macrocode}
\def\HoLogoCss@XeTeX{%
  \Css{%
    span.HoLogo-XeTeX span.HoLogo-TeX{%
      margin-left:-.15em;%
    }%
  }%
  \global\let\HoLogoCss@XeTeX\relax
}
%    \end{macrocode}
%    \end{macro}
%
%    \begin{macro}{\HoLogo@XeLaTeX}
%    \begin{macrocode}
\def\HoLogo@XeLaTeX#1{%
  \hologo{Xe}%
  \kern-.13em%
  \hologo{LaTeX}%
}
%    \end{macrocode}
%    \end{macro}
%    \begin{macro}{\HoLogoHtml@XeLaTeX}
%    \begin{macrocode}
\def\HoLogoHtml@XeLaTeX#1{%
  \HoLogoCss@XeLaTeX
  \HOLOGO@Span{XeLaTeX}{%
    \hologo{Xe}%
    \hologo{LaTeX}%
  }%
}
%    \end{macrocode}
%    \end{macro}
%    \begin{macro}{\HoLogoCss@XeLaTeX}
%    \begin{macrocode}
\def\HoLogoCss@XeLaTeX{%
  \Css{%
    span.HoLogo-XeLaTeX span.HoLogo-Xe{%
      margin-right:-.13em;%
    }%
  }%
  \global\let\HoLogoCss@XeLaTeX\relax
}
%    \end{macrocode}
%    \end{macro}
%
% \subsubsection{\hologo{pdfTeX}, \hologo{pdfLaTeX}}
%
%    \begin{macro}{\HoLogo@pdfTeX}
%    \begin{macrocode}
\def\HoLogo@pdfTeX#1{%
  \HOLOGO@mbox{%
    #1{p}{P}df\hologo{TeX}%
  }%
}
%    \end{macrocode}
%    \end{macro}
%    \begin{macro}{\HoLogoCs@pdfTeX}
%    \begin{macrocode}
\def\HoLogoCs@pdfTeX#1{#1{p}{P}dfTeX}
%    \end{macrocode}
%    \end{macro}
%    \begin{macro}{\HoLogoBkm@pdfTeX}
%    \begin{macrocode}
\def\HoLogoBkm@pdfTeX#1{%
  #1{p}{P}df\hologo{TeX}%
}
%    \end{macrocode}
%    \end{macro}
%    \begin{macro}{\HoLogoHtml@pdfTeX}
%    \begin{macrocode}
\let\HoLogoHtml@pdfTeX\HoLogo@pdfTeX
%    \end{macrocode}
%    \end{macro}
%
%    \begin{macro}{\HoLogo@pdfLaTeX}
%    \begin{macrocode}
\def\HoLogo@pdfLaTeX#1{%
  \HOLOGO@mbox{%
    #1{p}{P}df\hologo{LaTeX}%
  }%
}
%    \end{macrocode}
%    \end{macro}
%    \begin{macro}{\HoLogoCs@pdfLaTeX}
%    \begin{macrocode}
\def\HoLogoCs@pdfLaTeX#1{#1{p}{P}dfLaTeX}
%    \end{macrocode}
%    \end{macro}
%    \begin{macro}{\HoLogoBkm@pdfLaTeX}
%    \begin{macrocode}
\def\HoLogoBkm@pdfLaTeX#1{%
  #1{p}{P}df\hologo{LaTeX}%
}
%    \end{macrocode}
%    \end{macro}
%    \begin{macro}{\HoLogoHtml@pdfLaTeX}
%    \begin{macrocode}
\let\HoLogoHtml@pdfLaTeX\HoLogo@pdfLaTeX
%    \end{macrocode}
%    \end{macro}
%
% \subsubsection{\hologo{VTeX}}
%
%    \begin{macro}{\HoLogo@VTeX}
%    \begin{macrocode}
\def\HoLogo@VTeX#1{%
  \HOLOGO@mbox{%
    V\hologo{TeX}%
  }%
}
%    \end{macrocode}
%    \end{macro}
%    \begin{macro}{\HoLogoHtml@VTeX}
%    \begin{macrocode}
\let\HoLogoHtml@VTeX\HoLogo@VTeX
%    \end{macrocode}
%    \end{macro}
%
% \subsubsection{\hologo{AmS}, \dots}
%
%    Source: class \xclass{amsdtx}
%
%    \begin{macro}{\HoLogo@AmS}
%    \begin{macrocode}
\def\HoLogo@AmS#1{%
  \HoLogoFont@font{AmS}{sy}{%
    A%
    \kern-.1667em%
    \lower.5ex\hbox{M}%
    \kern-.125em%
    S%
  }%
}
%    \end{macrocode}
%    \end{macro}
%    \begin{macro}{\HoLogoBkm@AmS}
%    \begin{macrocode}
\def\HoLogoBkm@AmS#1{AmS}
%    \end{macrocode}
%    \end{macro}
%    \begin{macro}{\HoLogoHtml@AmS}
%    \begin{macrocode}
\def\HoLogoHtml@AmS#1{%
  \HoLogoCss@AmS
%  \HoLogoFont@font{AmS}{sy}{%
    \HOLOGO@Span{AmS}{%
      A%
      \HOLOGO@Span{M}{M}%
      S%
    }%
%   }%
}
%    \end{macrocode}
%    \end{macro}
%    \begin{macro}{\HoLogoCss@AmS}
%    \begin{macrocode}
\def\HoLogoCss@AmS{%
  \Css{%
    span.HoLogo-AmS span.HoLogo-M{%
      position:relative;%
      top:.5ex;%
      margin-left:-.1667em;%
      margin-right:-.125em;%
      text-decoration:none;%
    }%
  }%
  \global\let\HoLogoCss@AmS\relax
}
%    \end{macrocode}
%    \end{macro}
%
%    \begin{macro}{\HoLogo@AmSTeX}
%    \begin{macrocode}
\def\HoLogo@AmSTeX#1{%
  \hologo{AmS}%
  \HOLOGO@hyphen
  \hologo{TeX}%
}
%    \end{macrocode}
%    \end{macro}
%    \begin{macro}{\HoLogoBkm@AmSTeX}
%    \begin{macrocode}
\def\HoLogoBkm@AmSTeX#1{AmS-TeX}%
%    \end{macrocode}
%    \end{macro}
%    \begin{macro}{\HoLogoHtml@AmSTeX}
%    \begin{macrocode}
\let\HoLogoHtml@AmSTeX\HoLogo@AmSTeX
%    \end{macrocode}
%    \end{macro}
%
%    \begin{macro}{\HoLogo@AmSLaTeX}
%    \begin{macrocode}
\def\HoLogo@AmSLaTeX#1{%
  \hologo{AmS}%
  \HOLOGO@hyphen
  \hologo{LaTeX}%
}
%    \end{macrocode}
%    \end{macro}
%    \begin{macro}{\HoLogoBkm@AmSLaTeX}
%    \begin{macrocode}
\def\HoLogoBkm@AmSLaTeX#1{AmS-LaTeX}%
%    \end{macrocode}
%    \end{macro}
%    \begin{macro}{\HoLogoHtml@AmSLaTeX}
%    \begin{macrocode}
\let\HoLogoHtml@AmSLaTeX\HoLogo@AmSLaTeX
%    \end{macrocode}
%    \end{macro}
%
% \subsubsection{\hologo{BibTeX}}
%
%    \begin{macro}{\HoLogo@BibTeX@sc}
%    A definition of \hologo{BibTeX} is provided in
%    the documentation source for the manual of \hologo{BibTeX}
%    \cite{btxdoc}.
%\begin{quote}
%\begin{verbatim}
%\def\BibTeX{%
%  {%
%    \rm
%    B%
%    \kern-.05em%
%    {%
%      \sc
%      i%
%      \kern-.025em %
%      b%
%    }%
%    \kern-.08em
%    T%
%    \kern-.1667em%
%    \lower.7ex\hbox{E}%
%    \kern-.125em%
%    X%
%  }%
%}
%\end{verbatim}
%\end{quote}
%    \begin{macrocode}
\def\HoLogo@BibTeX@sc#1{%
  B%
  \kern-.05em%
  \HoLogoFont@font{BibTeX}{sc}{%
    i%
    \kern-.025em%
    b%
  }%
  \HOLOGO@discretionary
  \kern-.08em%
  \hologo{TeX}%
}
%    \end{macrocode}
%    \end{macro}
%    \begin{macro}{\HoLogoHtml@BibTeX@sc}
%    \begin{macrocode}
\def\HoLogoHtml@BibTeX@sc#1{%
  \HoLogoCss@BibTeX@sc
  \HOLOGO@Span{BibTeX-sc}{%
    B%
    \HOLOGO@Span{i}{i}%
    \HOLOGO@Span{b}{b}%
    \hologo{TeX}%
  }%
}
%    \end{macrocode}
%    \end{macro}
%    \begin{macro}{\HoLogoCss@BibTeX@sc}
%    \begin{macrocode}
\def\HoLogoCss@BibTeX@sc{%
  \Css{%
    span.HoLogo-BibTeX-sc span.HoLogo-i{%
      margin-left:-.05em;%
      margin-right:-.025em;%
      font-variant:small-caps;%
    }%
  }%
  \Css{%
    span.HoLogo-BibTeX-sc span.HoLogo-b{%
      margin-right:-.08em;%
      font-variant:small-caps;%
    }%
  }%
  \global\let\HoLogoCss@BibTeX@sc\relax
}
%    \end{macrocode}
%    \end{macro}
%
%    \begin{macro}{\HoLogo@BibTeX@sf}
%    Variant \xoption{sf} avoids trouble with unavailable
%    small caps fonts (e.g., bold versions of Computer Modern or
%    Latin Modern). The definition is taken from
%    package \xpackage{dtklogos} \cite{dtklogos}.
%\begin{quote}
%\begin{verbatim}
%\DeclareRobustCommand{\BibTeX}{%
%  B%
%  \kern-.05em%
%  \hbox{%
%    $\m@th$% %% force math size calculations
%    \csname S@\f@size\endcsname
%    \fontsize\sf@size\z@
%    \math@fontsfalse
%    \selectfont
%    I%
%    \kern-.025em%
%    B
%  }%
%  \kern-.08em%
%  \-%
%  \TeX
%}
%\end{verbatim}
%\end{quote}
%    \begin{macrocode}
\def\HoLogo@BibTeX@sf#1{%
  B%
  \kern-.05em%
  \HoLogoFont@font{BibTeX}{bibsf}{%
    I%
    \kern-.025em%
    B%
  }%
  \HOLOGO@discretionary
  \kern-.08em%
  \hologo{TeX}%
}
%    \end{macrocode}
%    \end{macro}
%    \begin{macro}{\HoLogoHtml@BibTeX@sf}
%    \begin{macrocode}
\def\HoLogoHtml@BibTeX@sf#1{%
  \HoLogoCss@BibTeX@sf
  \HOLOGO@Span{BibTeX-sf}{%
    B%
    \HoLogoFont@font{BibTeX}{bibsf}{%
      \HOLOGO@Span{i}{I}%
      B%
    }%
    \hologo{TeX}%
  }%
}
%    \end{macrocode}
%    \end{macro}
%    \begin{macro}{\HoLogoCss@BibTeX@sf}
%    \begin{macrocode}
\def\HoLogoCss@BibTeX@sf{%
  \Css{%
    span.HoLogo-BibTeX-sf span.HoLogo-i{%
      margin-left:-.05em;%
      margin-right:-.025em;%
    }%
  }%
  \Css{%
    span.HoLogo-BibTeX-sf span.HoLogo-TeX{%
      margin-left:-.08em;%
    }%
  }%
  \global\let\HoLogoCss@BibTeX@sf\relax
}
%    \end{macrocode}
%    \end{macro}
%
%    \begin{macro}{\HoLogo@BibTeX}
%    \begin{macrocode}
\def\HoLogo@BibTeX{\HoLogo@BibTeX@sf}
%    \end{macrocode}
%    \end{macro}
%    \begin{macro}{\HoLogoHtml@BibTeX}
%    \begin{macrocode}
\def\HoLogoHtml@BibTeX{\HoLogoHtml@BibTeX@sf}
%    \end{macrocode}
%    \end{macro}
%
% \subsubsection{\hologo{BibTeX8}}
%
%    \begin{macro}{\HoLogo@BibTeX8}
%    \begin{macrocode}
\expandafter\def\csname HoLogo@BibTeX8\endcsname#1{%
  \hologo{BibTeX}%
  8%
}
%    \end{macrocode}
%    \end{macro}
%
%    \begin{macro}{\HoLogoBkm@BibTeX8}
%    \begin{macrocode}
\expandafter\def\csname HoLogoBkm@BibTeX8\endcsname#1{%
  \hologo{BibTeX}%
  8%
}
%    \end{macrocode}
%    \end{macro}
%    \begin{macro}{\HoLogoHtml@BibTeX8}
%    \begin{macrocode}
\expandafter
\let\csname HoLogoHtml@BibTeX8\expandafter\endcsname
\csname HoLogo@BibTeX8\endcsname
%    \end{macrocode}
%    \end{macro}
%
% \subsubsection{\hologo{ConTeXt}}
%
%    \begin{macro}{\HoLogo@ConTeXt@simple}
%    \begin{macrocode}
\def\HoLogo@ConTeXt@simple#1{%
  \HOLOGO@mbox{Con}%
  \HOLOGO@discretionary
  \HOLOGO@mbox{\hologo{TeX}t}%
}
%    \end{macrocode}
%    \end{macro}
%    \begin{macro}{\HoLogoHtml@ConTeXt@simple}
%    \begin{macrocode}
\let\HoLogoHtml@ConTeXt@simple\HoLogo@ConTeXt@simple
%    \end{macrocode}
%    \end{macro}
%
%    \begin{macro}{\HoLogo@ConTeXt@narrow}
%    This definition of logo \hologo{ConTeXt} with variant \xoption{narrow}
%    comes from TUGboat's class \xclass{ltugboat} (version 2010/11/15 v2.8).
%    \begin{macrocode}
\def\HoLogo@ConTeXt@narrow#1{%
  \HOLOGO@mbox{C\kern-.0333emon}%
  \HOLOGO@discretionary
  \kern-.0667em%
  \HOLOGO@mbox{\hologo{TeX}\kern-.0333emt}%
}
%    \end{macrocode}
%    \end{macro}
%    \begin{macro}{\HoLogoHtml@ConTeXt@narrow}
%    \begin{macrocode}
\def\HoLogoHtml@ConTeXt@narrow#1{%
  \HoLogoCss@ConTeXt@narrow
  \HOLOGO@Span{ConTeXt-narrow}{%
    \HOLOGO@Span{C}{C}%
    on%
    \hologo{TeX}%
    t%
  }%
}
%    \end{macrocode}
%    \end{macro}
%    \begin{macro}{\HoLogoCss@ConTeXt@narrow}
%    \begin{macrocode}
\def\HoLogoCss@ConTeXt@narrow{%
  \Css{%
    span.HoLogo-ConTeXt-narrow span.HoLogo-C{%
      margin-left:-.0333em;%
    }%
  }%
  \Css{%
    span.HoLogo-ConTeXt-narrow span.HoLogo-TeX{%
      margin-left:-.0667em;%
      margin-right:-.0333em;%
    }%
  }%
  \global\let\HoLogoCss@ConTeXt@narrow\relax
}
%    \end{macrocode}
%    \end{macro}
%
%    \begin{macro}{\HoLogo@ConTeXt}
%    \begin{macrocode}
\def\HoLogo@ConTeXt{\HoLogo@ConTeXt@narrow}
%    \end{macrocode}
%    \end{macro}
%    \begin{macro}{\HoLogoHtml@ConTeXt}
%    \begin{macrocode}
\def\HoLogoHtml@ConTeXt{\HoLogoHtml@ConTeXt@narrow}
%    \end{macrocode}
%    \end{macro}
%
% \subsubsection{\hologo{emTeX}}
%
%    \begin{macro}{\HoLogo@emTeX}
%    \begin{macrocode}
\def\HoLogo@emTeX#1{%
  \HOLOGO@mbox{#1{e}{E}m}%
  \HOLOGO@discretionary
  \hologo{TeX}%
}
%    \end{macrocode}
%    \end{macro}
%    \begin{macro}{\HoLogoCs@emTeX}
%    \begin{macrocode}
\def\HoLogoCs@emTeX#1{#1{e}{E}mTeX}%
%    \end{macrocode}
%    \end{macro}
%    \begin{macro}{\HoLogoBkm@emTeX}
%    \begin{macrocode}
\def\HoLogoBkm@emTeX#1{%
  #1{e}{E}m\hologo{TeX}%
}
%    \end{macrocode}
%    \end{macro}
%    \begin{macro}{\HoLogoHtml@emTeX}
%    \begin{macrocode}
\let\HoLogoHtml@emTeX\HoLogo@emTeX
%    \end{macrocode}
%    \end{macro}
%
% \subsubsection{\hologo{ExTeX}}
%
%    \begin{macro}{\HoLogo@ExTeX}
%    The definition is taken from the FAQ of the
%    project \hologo{ExTeX}
%    \cite{ExTeX-FAQ}.
%\begin{quote}
%\begin{verbatim}
%\def\ExTeX{%
%  \textrm{% Logo always with serifs
%    \ensuremath{%
%      \textstyle
%      \varepsilon_{%
%        \kern-0.15em%
%        \mathcal{X}%
%      }%
%    }%
%    \kern-.15em%
%    \TeX
%  }%
%}
%\end{verbatim}
%\end{quote}
%    \begin{macrocode}
\def\HoLogo@ExTeX#1{%
  \HoLogoFont@font{ExTeX}{rm}{%
    \ltx@mbox{%
      \HOLOGO@MathSetup
      $%
        \textstyle
        \varepsilon_{%
          \kern-0.15em%
          \HoLogoFont@font{ExTeX}{sy}{X}%
        }%
      $%
    }%
    \HOLOGO@discretionary
    \kern-.15em%
    \hologo{TeX}%
  }%
}
%    \end{macrocode}
%    \end{macro}
%    \begin{macro}{\HoLogoHtml@ExTeX}
%    \begin{macrocode}
\def\HoLogoHtml@ExTeX#1{%
  \HoLogoCss@ExTeX
  \HoLogoFont@font{ExTeX}{rm}{%
    \HOLOGO@Span{ExTeX}{%
      \ltx@mbox{%
        \HOLOGO@MathSetup
        $\textstyle\varepsilon$%
        \HOLOGO@Span{X}{$\textstyle\chi$}%
        \hologo{TeX}%
      }%
    }%
  }%
}
%    \end{macrocode}
%    \end{macro}
%    \begin{macro}{\HoLogoBkm@ExTeX}
%    \begin{macrocode}
\def\HoLogoBkm@ExTeX#1{%
  \HOLOGO@PdfdocUnicode{#1{e}{E}x}{\textepsilon\textchi}%
  \hologo{TeX}%
}
%    \end{macrocode}
%    \end{macro}
%    \begin{macro}{\HoLogoCss@ExTeX}
%    \begin{macrocode}
\def\HoLogoCss@ExTeX{%
  \Css{%
    span.HoLogo-ExTeX{%
      font-family:serif;%
    }%
  }%
  \Css{%
    span.HoLogo-ExTeX span.HoLogo-TeX{%
      margin-left:-.15em;%
    }%
  }%
  \global\let\HoLogoCss@ExTeX\relax
}
%    \end{macrocode}
%    \end{macro}
%
% \subsubsection{\hologo{MiKTeX}}
%
%    \begin{macro}{\HoLogo@MiKTeX}
%    \begin{macrocode}
\def\HoLogo@MiKTeX#1{%
  \HOLOGO@mbox{MiK}%
  \HOLOGO@discretionary
  \hologo{TeX}%
}
%    \end{macrocode}
%    \end{macro}
%    \begin{macro}{\HoLogoHtml@MiKTeX}
%    \begin{macrocode}
\let\HoLogoHtml@MiKTeX\HoLogo@MiKTeX
%    \end{macrocode}
%    \end{macro}
%
% \subsubsection{\hologo{OzTeX} and friends}
%
%    Source: \hologo{OzTeX} FAQ \cite{OzTeX}:
%    \begin{quote}
%      |\def\OzTeX{O\kern-.03em z\kern-.15em\TeX}|\\
%      (There is no kerning in OzMF, OzMP and OzTtH.)
%    \end{quote}
%
%    \begin{macro}{\HoLogo@OzTeX}
%    \begin{macrocode}
\def\HoLogo@OzTeX#1{%
  O%
  \kern-.03em %
  z%
  \kern-.15em %
  \hologo{TeX}%
}
%    \end{macrocode}
%    \end{macro}
%    \begin{macro}{\HoLogoHtml@OzTeX}
%    \begin{macrocode}
\def\HoLogoHtml@OzTeX#1{%
  \HoLogoCss@OzTeX
  \HOLOGO@Span{OzTeX}{%
    O%
    \HOLOGO@Span{z}{z}%
    \hologo{TeX}%
  }%
}
%    \end{macrocode}
%    \end{macro}
%    \begin{macro}{\HoLogoCss@OzTeX}
%    \begin{macrocode}
\def\HoLogoCss@OzTeX{%
  \Css{%
    span.HoLogo-OzTeX span.HoLogo-z{%
      margin-left:-.03em;%
      margin-right:-.15em;%
    }%
  }%
  \global\let\HoLogoCss@OzTeX\relax
}
%    \end{macrocode}
%    \end{macro}
%
%    \begin{macro}{\HoLogo@OzMF}
%    \begin{macrocode}
\def\HoLogo@OzMF#1{%
  \HOLOGO@mbox{OzMF}%
}
%    \end{macrocode}
%    \end{macro}
%    \begin{macro}{\HoLogo@OzMP}
%    \begin{macrocode}
\def\HoLogo@OzMP#1{%
  \HOLOGO@mbox{OzMP}%
}
%    \end{macrocode}
%    \end{macro}
%    \begin{macro}{\HoLogo@OzTtH}
%    \begin{macrocode}
\def\HoLogo@OzTtH#1{%
  \HOLOGO@mbox{OzTtH}%
}
%    \end{macrocode}
%    \end{macro}
%
% \subsubsection{\hologo{PCTeX}}
%
%    \begin{macro}{\HoLogo@PCTeX}
%    \begin{macrocode}
\def\HoLogo@PCTeX#1{%
  \HOLOGO@mbox{PC}%
  \hologo{TeX}%
}
%    \end{macrocode}
%    \end{macro}
%    \begin{macro}{\HoLogoHtml@PCTeX}
%    \begin{macrocode}
\let\HoLogoHtml@PCTeX\HoLogo@PCTeX
%    \end{macrocode}
%    \end{macro}
%
% \subsubsection{\hologo{PiCTeX}}
%
%    The original definitions from \xfile{pictex.tex} \cite{PiCTeX}:
%\begin{quote}
%\begin{verbatim}
%\def\PiC{%
%  P%
%  \kern-.12em%
%  \lower.5ex\hbox{I}%
%  \kern-.075em%
%  C%
%}
%\def\PiCTeX{%
%  \PiC
%  \kern-.11em%
%  \TeX
%}
%\end{verbatim}
%\end{quote}
%
%    \begin{macro}{\HoLogo@PiC}
%    \begin{macrocode}
\def\HoLogo@PiC#1{%
  P%
  \kern-.12em%
  \lower.5ex\hbox{I}%
  \kern-.075em%
  C%
  \HOLOGO@SpaceFactor
}
%    \end{macrocode}
%    \end{macro}
%    \begin{macro}{\HoLogoHtml@PiC}
%    \begin{macrocode}
\def\HoLogoHtml@PiC#1{%
  \HoLogoCss@PiC
  \HOLOGO@Span{PiC}{%
    P%
    \HOLOGO@Span{i}{I}%
    C%
  }%
}
%    \end{macrocode}
%    \end{macro}
%    \begin{macro}{\HoLogoCss@PiC}
%    \begin{macrocode}
\def\HoLogoCss@PiC{%
  \Css{%
    span.HoLogo-PiC span.HoLogo-i{%
      position:relative;%
      top:.5ex;%
      margin-left:-.12em;%
      margin-right:-.075em;%
      text-decoration:none;%
    }%
  }%
  \global\let\HoLogoCss@PiC\relax
}
%    \end{macrocode}
%    \end{macro}
%
%    \begin{macro}{\HoLogo@PiCTeX}
%    \begin{macrocode}
\def\HoLogo@PiCTeX#1{%
  \hologo{PiC}%
  \HOLOGO@discretionary
  \kern-.11em%
  \hologo{TeX}%
}
%    \end{macrocode}
%    \end{macro}
%    \begin{macro}{\HoLogoHtml@PiCTeX}
%    \begin{macrocode}
\def\HoLogoHtml@PiCTeX#1{%
  \HoLogoCss@PiCTeX
  \HOLOGO@Span{PiCTeX}{%
    \hologo{PiC}%
    \hologo{TeX}%
  }%
}
%    \end{macrocode}
%    \end{macro}
%    \begin{macro}{\HoLogoCss@PiCTeX}
%    \begin{macrocode}
\def\HoLogoCss@PiCTeX{%
  \Css{%
    span.HoLogo-PiCTeX span.HoLogo-PiC{%
      margin-right:-.11em;%
    }%
  }%
  \global\let\HoLogoCss@PiCTeX\relax
}
%    \end{macrocode}
%    \end{macro}
%
% \subsubsection{\hologo{teTeX}}
%
%    \begin{macro}{\HoLogo@teTeX}
%    \begin{macrocode}
\def\HoLogo@teTeX#1{%
  \HOLOGO@mbox{#1{t}{T}e}%
  \HOLOGO@discretionary
  \hologo{TeX}%
}
%    \end{macrocode}
%    \end{macro}
%    \begin{macro}{\HoLogoCs@teTeX}
%    \begin{macrocode}
\def\HoLogoCs@teTeX#1{#1{t}{T}dfTeX}
%    \end{macrocode}
%    \end{macro}
%    \begin{macro}{\HoLogoBkm@teTeX}
%    \begin{macrocode}
\def\HoLogoBkm@teTeX#1{%
  #1{t}{T}e\hologo{TeX}%
}
%    \end{macrocode}
%    \end{macro}
%    \begin{macro}{\HoLogoHtml@teTeX}
%    \begin{macrocode}
\let\HoLogoHtml@teTeX\HoLogo@teTeX
%    \end{macrocode}
%    \end{macro}
%
% \subsubsection{\hologo{TeX4ht}}
%
%    \begin{macro}{\HoLogo@TeX4ht}
%    \begin{macrocode}
\expandafter\def\csname HoLogo@TeX4ht\endcsname#1{%
  \HOLOGO@mbox{\hologo{TeX}4ht}%
}
%    \end{macrocode}
%    \end{macro}
%    \begin{macro}{\HoLogoHtml@TeX4ht}
%    \begin{macrocode}
\expandafter
\let\csname HoLogoHtml@TeX4ht\expandafter\endcsname
\csname HoLogo@TeX4ht\endcsname
%    \end{macrocode}
%    \end{macro}
%
%
% \subsubsection{\hologo{SageTeX}}
%
%    \begin{macro}{\HoLogo@SageTeX}
%    \begin{macrocode}
\def\HoLogo@SageTeX#1{%
  \HOLOGO@mbox{Sage}%
  \HOLOGO@discretionary
  \HOLOGO@NegativeKerning{eT,oT,To}%
  \hologo{TeX}%
}
%    \end{macrocode}
%    \end{macro}
%    \begin{macro}{\HoLogoHtml@SageTeX}
%    \begin{macrocode}
\let\HoLogoHtml@SageTeX\HoLogo@SageTeX
%    \end{macrocode}
%    \end{macro}
%
% \subsection{\hologo{METAFONT} and friends}
%
%    \begin{macro}{\HoLogo@METAFONT}
%    \begin{macrocode}
\def\HoLogo@METAFONT#1{%
  \HoLogoFont@font{METAFONT}{logo}{%
    \HOLOGO@mbox{META}%
    \HOLOGO@discretionary
    \HOLOGO@mbox{FONT}%
  }%
}
%    \end{macrocode}
%    \end{macro}
%
%    \begin{macro}{\HoLogo@METAPOST}
%    \begin{macrocode}
\def\HoLogo@METAPOST#1{%
  \HoLogoFont@font{METAPOST}{logo}{%
    \HOLOGO@mbox{META}%
    \HOLOGO@discretionary
    \HOLOGO@mbox{POST}%
  }%
}
%    \end{macrocode}
%    \end{macro}
%
%    \begin{macro}{\HoLogo@MetaFun}
%    \begin{macrocode}
\def\HoLogo@MetaFun#1{%
  \HOLOGO@mbox{Meta}%
  \HOLOGO@discretionary
  \HOLOGO@mbox{Fun}%
}
%    \end{macrocode}
%    \end{macro}
%
%    \begin{macro}{\HoLogo@MetaPost}
%    \begin{macrocode}
\def\HoLogo@MetaPost#1{%
  \HOLOGO@mbox{Meta}%
  \HOLOGO@discretionary
  \HOLOGO@mbox{Post}%
}
%    \end{macrocode}
%    \end{macro}
%
% \subsection{Others}
%
% \subsubsection{\hologo{biber}}
%
%    \begin{macro}{\HoLogo@biber}
%    \begin{macrocode}
\def\HoLogo@biber#1{%
  \HOLOGO@mbox{#1{b}{B}i}%
  \HOLOGO@discretionary
  \HOLOGO@mbox{ber}%
}
%    \end{macrocode}
%    \end{macro}
%    \begin{macro}{\HoLogoCs@biber}
%    \begin{macrocode}
\def\HoLogoCs@biber#1{#1{b}{B}iber}
%    \end{macrocode}
%    \end{macro}
%    \begin{macro}{\HoLogoBkm@biber}
%    \begin{macrocode}
\def\HoLogoBkm@biber#1{%
  #1{b}{B}iber%
}
%    \end{macrocode}
%    \end{macro}
%    \begin{macro}{\HoLogoHtml@biber}
%    \begin{macrocode}
\let\HoLogoHtml@biber\HoLogo@biber
%    \end{macrocode}
%    \end{macro}
%
% \subsubsection{\hologo{KOMAScript}}
%
%    \begin{macro}{\HoLogo@KOMAScript}
%    The definition for \hologo{KOMAScript} is taken
%    from \hologo{KOMAScript} (\xfile{scrlogo.dtx}, reformatted) \cite{scrlogo}:
%\begin{quote}
%\begin{verbatim}
%\@ifundefined{KOMAScript}{%
%  \DeclareRobustCommand{\KOMAScript}{%
%    \textsf{%
%      K\kern.05em O\kern.05emM\kern.05em A%
%      \kern.1em-\kern.1em %
%      Script%
%    }%
%  }%
%}{}
%\end{verbatim}
%\end{quote}
%    \begin{macrocode}
\def\HoLogo@KOMAScript#1{%
  \HoLogoFont@font{KOMAScript}{sf}{%
    \HOLOGO@mbox{%
      K\kern.05em%
      O\kern.05em%
      M\kern.05em%
      A%
    }%
    \kern.1em%
    \HOLOGO@hyphen
    \kern.1em%
    \HOLOGO@mbox{Script}%
  }%
}
%    \end{macrocode}
%    \end{macro}
%    \begin{macro}{\HoLogoBkm@KOMAScript}
%    \begin{macrocode}
\def\HoLogoBkm@KOMAScript#1{%
  KOMA-Script%
}
%    \end{macrocode}
%    \end{macro}
%    \begin{macro}{\HoLogoHtml@KOMAScript}
%    \begin{macrocode}
\def\HoLogoHtml@KOMAScript#1{%
  \HoLogoCss@KOMAScript
  \HoLogoFont@font{KOMAScript}{sf}{%
    \HOLOGO@Span{KOMAScript}{%
      K%
      \HOLOGO@Span{O}{O}%
      M%
      \HOLOGO@Span{A}{A}%
      \HOLOGO@Span{hyphen}{-}%
      Script%
    }%
  }%
}
%    \end{macrocode}
%    \end{macro}
%    \begin{macro}{\HoLogoCss@KOMAScript}
%    \begin{macrocode}
\def\HoLogoCss@KOMAScript{%
  \Css{%
    span.HoLogo-KOMAScript{%
      font-family:sans-serif;%
    }%
  }%
  \Css{%
    span.HoLogo-KOMAScript span.HoLogo-O{%
      padding-left:.05em;%
      padding-right:.05em;%
    }%
  }%
  \Css{%
    span.HoLogo-KOMAScript span.HoLogo-A{%
      padding-left:.05em;%
    }%
  }%
  \Css{%
    span.HoLogo-KOMAScript span.HoLogo-hyphen{%
      padding-left:.1em;%
      padding-right:.1em;%
    }%
  }%
  \global\let\HoLogoCss@KOMAScript\relax
}
%    \end{macrocode}
%    \end{macro}
%
% \subsubsection{\hologo{LyX}}
%
%    \begin{macro}{\HoLogo@LyX}
%    The definition is taken from the documentation source files
%    of \hologo{LyX}, \xfile{Intro.lyx} \cite{LyX}:
%\begin{quote}
%\begin{verbatim}
%\def\LyX{%
%  \texorpdfstring{%
%    L\kern-.1667em\lower.25em\hbox{Y}\kern-.125emX\@%
%  }{%
%    LyX%
%  }%
%}
%\end{verbatim}
%\end{quote}
%    \begin{macrocode}
\def\HoLogo@LyX#1{%
  L%
  \kern-.1667em%
  \lower.25em\hbox{Y}%
  \kern-.125em%
  X%
  \HOLOGO@SpaceFactor
}
%    \end{macrocode}
%    \end{macro}
%    \begin{macro}{\HoLogoHtml@LyX}
%    \begin{macrocode}
\def\HoLogoHtml@LyX#1{%
  \HoLogoCss@LyX
  \HOLOGO@Span{LyX}{%
    L%
    \HOLOGO@Span{y}{Y}%
    X%
  }%
}
%    \end{macrocode}
%    \end{macro}
%    \begin{macro}{\HoLogoCss@LyX}
%    \begin{macrocode}
\def\HoLogoCss@LyX{%
  \Css{%
    span.HoLogo-LyX span.HoLogo-y{%
      position:relative;%
      top:.25em;%
      margin-left:-.1667em;%
      margin-right:-.125em;%
      text-decoration:none;%
    }%
  }%
  \global\let\HoLogoCss@LyX\relax
}
%    \end{macrocode}
%    \end{macro}
%
% \subsubsection{\hologo{NTS}}
%
%    \begin{macro}{\HoLogo@NTS}
%    Definition for \hologo{NTS} can be found in
%    package \xpackage{etex\textunderscore man} for the \hologo{eTeX} manual \cite{etexman}
%    and in package \xpackage{dtklogos} \cite{dtklogos}:
%\begin{quote}
%\begin{verbatim}
%\def\NTS{%
%  \leavevmode
%  \hbox{%
%    $%
%      \cal N%
%      \kern-0.35em%
%      \lower0.5ex\hbox{$\cal T$}%
%      \kern-0.2em%
%      S%
%    $%
%  }%
%}
%\end{verbatim}
%\end{quote}
%    \begin{macrocode}
\def\HoLogo@NTS#1{%
  \HoLogoFont@font{NTS}{sy}{%
    N\/%
    \kern-.35em%
    \lower.5ex\hbox{T\/}%
    \kern-.2em%
    S\/%
  }%
  \HOLOGO@SpaceFactor
}
%    \end{macrocode}
%    \end{macro}
%
% \subsubsection{\Hologo{TTH} (\hologo{TeX} to HTML translator)}
%
%    Source: \url{http://hutchinson.belmont.ma.us/tth/}
%    In the HTML source the second `T' is printed as subscript.
%\begin{quote}
%\begin{verbatim}
%T<sub>T</sub>H
%\end{verbatim}
%\end{quote}
%    \begin{macro}{\HoLogo@TTH}
%    \begin{macrocode}
\def\HoLogo@TTH#1{%
  \ltx@mbox{%
    T\HOLOGO@SubScript{T}H%
  }%
  \HOLOGO@SpaceFactor
}
%    \end{macrocode}
%    \end{macro}
%
%    \begin{macro}{\HoLogoHtml@TTH}
%    \begin{macrocode}
\def\HoLogoHtml@TTH#1{%
  T\HCode{<sub>}T\HCode{</sub>}H%
}
%    \end{macrocode}
%    \end{macro}
%
% \subsubsection{\Hologo{HanTheThanh}}
%
%    Partial source: Package \xpackage{dtklogos}.
%    The double accent is U+1EBF (latin small letter e with circumflex
%    and acute).
%    \begin{macro}{\HoLogo@HanTheThanh}
%    \begin{macrocode}
\def\HoLogo@HanTheThanh#1{%
  \ltx@mbox{H\`an}%
  \HOLOGO@space
  \ltx@mbox{%
    Th%
    \HOLOGO@IfCharExists{"1EBF}{%
      \char"1EBF\relax
    }{%
      \^e\hbox to 0pt{\hss\raise .5ex\hbox{\'{}}}%
    }%
  }%
  \HOLOGO@space
  \ltx@mbox{Th\`anh}%
}
%    \end{macrocode}
%    \end{macro}
%    \begin{macro}{\HoLogoBkm@HanTheThanh}
%    \begin{macrocode}
\def\HoLogoBkm@HanTheThanh#1{%
  H\`an %
  Th\HOLOGO@PdfdocUnicode{\^e}{\9036\277} %
  Th\`anh%
}
%    \end{macrocode}
%    \end{macro}
%    \begin{macro}{\HoLogoHtml@HanTheThanh}
%    \begin{macrocode}
\def\HoLogoHtml@HanTheThanh#1{%
  H\`an %
  Th\HCode{&\ltx@hashchar x1ebf;} %
  Th\`anh%
}
%    \end{macrocode}
%    \end{macro}
%
% \subsection{Driver detection}
%
%    \begin{macrocode}
\HOLOGO@IfExists\InputIfFileExists{%
  \InputIfFileExists{hologo.cfg}{}{}%
}{%
  \ltx@IfUndefined{pdf@filesize}{%
    \def\HOLOGO@InputIfExists{%
      \openin\HOLOGO@temp=hologo.cfg\relax
      \ifeof\HOLOGO@temp
        \closein\HOLOGO@temp
      \else
        \closein\HOLOGO@temp
        \begingroup
          \def\x{LaTeX2e}%
        \expandafter\endgroup
        \ifx\fmtname\x
          \input{hologo.cfg}%
        \else
          \input hologo.cfg\relax
        \fi
      \fi
    }%
    \ltx@IfUndefined{newread}{%
      \chardef\HOLOGO@temp=15 %
      \def\HOLOGO@CheckRead{%
        \ifeof\HOLOGO@temp
          \HOLOGO@InputIfExists
        \else
          \ifcase\HOLOGO@temp
            \@PackageWarningNoLine{hologo}{%
              Configuration file ignored, because\MessageBreak
              a free read register could not be found%
            }%
          \else
            \begingroup
              \count\ltx@cclv=\HOLOGO@temp
              \advance\ltx@cclv by \ltx@minusone
              \edef\x{\endgroup
                \chardef\noexpand\HOLOGO@temp=\the\count\ltx@cclv
                \relax
              }%
            \x
          \fi
        \fi
      }%
    }{%
      \csname newread\endcsname\HOLOGO@temp
      \HOLOGO@InputIfExists
    }%
  }{%
    \edef\HOLOGO@temp{\pdf@filesize{hologo.cfg}}%
    \ifx\HOLOGO@temp\ltx@empty
    \else
      \ifnum\HOLOGO@temp>0 %
        \begingroup
          \def\x{LaTeX2e}%
        \expandafter\endgroup
        \ifx\fmtname\x
          \input{hologo.cfg}%
        \else
          \input hologo.cfg\relax
        \fi
      \else
        \@PackageInfoNoLine{hologo}{%
          Empty configuration file `hologo.cfg' ignored%
        }%
      \fi
    \fi
  }%
}
%    \end{macrocode}
%
%    \begin{macrocode}
\def\HOLOGO@temp#1#2{%
  \kv@define@key{HoLogoDriver}{#1}[]{%
    \begingroup
      \def\HOLOGO@temp{##1}%
      \ltx@onelevel@sanitize\HOLOGO@temp
      \ifx\HOLOGO@temp\ltx@empty
      \else
        \@PackageError{hologo}{%
          Value (\HOLOGO@temp) not permitted for option `#1'%
        }%
        \@ehc
      \fi
    \endgroup
    \def\hologoDriver{#2}%
  }%
}%
\def\HOLOGO@@temp#1#2{%
  \ifx\kv@value\relax
    \HOLOGO@temp{#1}{#1}%
  \else
    \HOLOGO@temp{#1}{#2}%
  \fi
}%
\kv@parse@normalized{%
  pdftex,%
  luatex=pdftex,%
  dvipdfm,%
  dvipdfmx=dvipdfm,%
  dvips,%
  dvipsone=dvips,%
  xdvi=dvips,%
  xetex,%
  vtex,%
}\HOLOGO@@temp
%    \end{macrocode}
%
%    \begin{macrocode}
\kv@define@key{HoLogoDriver}{driverfallback}{%
  \def\HOLOGO@DriverFallback{#1}%
}
%    \end{macrocode}
%
%    \begin{macro}{\HOLOGO@DriverFallback}
%    \begin{macrocode}
\def\HOLOGO@DriverFallback{dvips}
%    \end{macrocode}
%    \end{macro}
%
%    \begin{macro}{\hologoDriverSetup}
%    \begin{macrocode}
\def\hologoDriverSetup{%
  \let\hologoDriver\ltx@undefined
  \HOLOGO@DriverSetup
}
%    \end{macrocode}
%    \end{macro}
%
%    \begin{macro}{\HOLOGO@DriverSetup}
%    \begin{macrocode}
\def\HOLOGO@DriverSetup#1{%
  \kvsetkeys{HoLogoDriver}{#1}%
  \HOLOGO@CheckDriver
  \ltx@ifundefined{hologoDriver}{%
    \begingroup
    \edef\x{\endgroup
      \noexpand\kvsetkeys{HoLogoDriver}{\HOLOGO@DriverFallback}%
    }\x
  }{}%
  \@PackageInfoNoLine{hologo}{Using driver `\hologoDriver'}%
}
%    \end{macrocode}
%    \end{macro}
%
%    \begin{macro}{\HOLOGO@CheckDriver}
%    \begin{macrocode}
\def\HOLOGO@CheckDriver{%
  \ifpdf
    \def\hologoDriver{pdftex}%
    \let\HOLOGO@pdfliteral\pdfliteral
    \ifluatex
      \ifx\pdfextension\@undefined\else
        \protected\def\pdfliteral{\pdfextension literal}%
        \let\HOLOGO@pdfliteral\pdfliteral
      \fi
      \ltx@IfUndefined{HOLOGO@pdfliteral}{%
        \ifnum\luatexversion<36 %
        \else
          \begingroup
            \let\HOLOGO@temp\endgroup
            \ifcase0%
                \directlua{%
                  if tex.enableprimitives then %
                    tex.enableprimitives('HOLOGO@', {'pdfliteral'})%
                  else %
                    tex.print('1')%
                  end%
                }%
                \ifx\HOLOGO@pdfliteral\@undefined 1\fi%
                \relax%
              \endgroup
              \let\HOLOGO@temp\relax
              \global\let\HOLOGO@pdfliteral\HOLOGO@pdfliteral
            \fi%
          \HOLOGO@temp
        \fi
      }{}%
    \fi
    \ltx@IfUndefined{HOLOGO@pdfliteral}{%
      \@PackageWarningNoLine{hologo}{%
        Cannot find \string\pdfliteral
      }%
    }{}%
  \else
    \ifxetex
      \def\hologoDriver{xetex}%
    \else
      \ifvtex
        \def\hologoDriver{vtex}%
      \fi
    \fi
  \fi
}
%    \end{macrocode}
%    \end{macro}
%
%    \begin{macro}{\HOLOGO@WarningUnsupportedDriver}
%    \begin{macrocode}
\def\HOLOGO@WarningUnsupportedDriver#1{%
  \@PackageWarningNoLine{hologo}{%
    Logo `#1' needs driver specific macros,\MessageBreak
    but driver `\hologoDriver' is not supported.\MessageBreak
    Use a different driver or\MessageBreak
    load package `graphics' or `pgf'%
  }%
}
%    \end{macrocode}
%    \end{macro}
%
% \subsubsection{Reflect box macros}
%
%    Skip driver part if not needed.
%    \begin{macrocode}
\ltx@IfUndefined{reflectbox}{}{%
  \ltx@IfUndefined{rotatebox}{}{%
    \HOLOGO@AtEnd
  }%
}
\ltx@IfUndefined{pgftext}{}{%
  \HOLOGO@AtEnd
}
\ltx@IfUndefined{psscalebox}{}{%
  \HOLOGO@AtEnd
}
%    \end{macrocode}
%
%    \begin{macrocode}
\def\HOLOGO@temp{LaTeX2e}
\ifx\fmtname\HOLOGO@temp
  \RequirePackage{kvoptions}[2011/06/30]%
  \ProcessKeyvalOptions{HoLogoDriver}%
\fi
\HOLOGO@DriverSetup{}
%    \end{macrocode}
%
%    \begin{macro}{\HOLOGO@ReflectBox}
%    \begin{macrocode}
\def\HOLOGO@ReflectBox#1{%
  \begingroup
    \setbox\ltx@zero\hbox{\begingroup#1\endgroup}%
    \setbox\ltx@two\hbox{%
      \kern\wd\ltx@zero
      \csname HOLOGO@ScaleBox@\hologoDriver\endcsname{-1}{1}{%
        \hbox to 0pt{\copy\ltx@zero\hss}%
      }%
    }%
    \wd\ltx@two=\wd\ltx@zero
    \box\ltx@two
  \endgroup
}
%    \end{macrocode}
%    \end{macro}
%
%    \begin{macro}{\HOLOGO@PointReflectBox}
%    \begin{macrocode}
\def\HOLOGO@PointReflectBox#1{%
  \begingroup
    \setbox\ltx@zero\hbox{\begingroup#1\endgroup}%
    \setbox\ltx@two\hbox{%
      \kern\wd\ltx@zero
      \raise\ht\ltx@zero\hbox{%
        \csname HOLOGO@ScaleBox@\hologoDriver\endcsname{-1}{-1}{%
          \hbox to 0pt{\copy\ltx@zero\hss}%
        }%
      }%
    }%
    \wd\ltx@two=\wd\ltx@zero
    \box\ltx@two
  \endgroup
}
%    \end{macrocode}
%    \end{macro}
%
%    We must define all variants because of dynamic driver setup.
%    \begin{macrocode}
\def\HOLOGO@temp#1#2{#2}
%    \end{macrocode}
%
%    \begin{macro}{\HOLOGO@ScaleBox@pdftex}
%    \begin{macrocode}
\HOLOGO@temp{pdftex}{%
  \def\HOLOGO@ScaleBox@pdftex#1#2#3{%
    \HOLOGO@pdfliteral{%
      q #1 0 0 #2 0 0 cm%
    }%
    #3%
    \HOLOGO@pdfliteral{%
      Q%
    }%
  }%
}
%    \end{macrocode}
%    \end{macro}
%    \begin{macro}{\HOLOGO@ScaleBox@dvips}
%    \begin{macrocode}
\HOLOGO@temp{dvips}{%
  \def\HOLOGO@ScaleBox@dvips#1#2#3{%
    \special{ps:%
      gsave %
      currentpoint %
      currentpoint translate %
      #1 #2 scale %
      neg exch neg exch translate%
    }%
    #3%
    \special{ps:%
      currentpoint %
      grestore %
      moveto%
    }%
  }%
}
%    \end{macrocode}
%    \end{macro}
%    \begin{macro}{\HOLOGO@ScaleBox@dvipdfm}
%    \begin{macrocode}
\HOLOGO@temp{dvipdfm}{%
  \let\HOLOGO@ScaleBox@dvipdfm\HOLOGO@ScaleBox@dvips
}
%    \end{macrocode}
%    \end{macro}
%    Since \hologo{XeTeX} v0.6.
%    \begin{macro}{\HOLOGO@ScaleBox@xetex}
%    \begin{macrocode}
\HOLOGO@temp{xetex}{%
  \def\HOLOGO@ScaleBox@xetex#1#2#3{%
    \special{x:gsave}%
    \special{x:scale #1 #2}%
    #3%
    \special{x:grestore}%
  }%
}
%    \end{macrocode}
%    \end{macro}
%    \begin{macro}{\HOLOGO@ScaleBox@vtex}
%    \begin{macrocode}
\HOLOGO@temp{vtex}{%
  \def\HOLOGO@ScaleBox@vtex#1#2#3{%
    \special{r(#1,0,0,#2,0,0}%
    #3%
    \special{r)}%
  }%
}
%    \end{macrocode}
%    \end{macro}
%
%    \begin{macrocode}
\HOLOGO@AtEnd%
%</package>
%    \end{macrocode}
%
% \section{Test}
%
% \subsection{Catcode checks for loading}
%
%    \begin{macrocode}
%<*test1>
%    \end{macrocode}
%    \begin{macrocode}
\catcode`\{=1 %
\catcode`\}=2 %
\catcode`\#=6 %
\catcode`\@=11 %
\expandafter\ifx\csname count@\endcsname\relax
  \countdef\count@=255 %
\fi
\expandafter\ifx\csname @gobble\endcsname\relax
  \long\def\@gobble#1{}%
\fi
\expandafter\ifx\csname @firstofone\endcsname\relax
  \long\def\@firstofone#1{#1}%
\fi
\expandafter\ifx\csname loop\endcsname\relax
  \expandafter\@firstofone
\else
  \expandafter\@gobble
\fi
{%
  \def\loop#1\repeat{%
    \def\body{#1}%
    \iterate
  }%
  \def\iterate{%
    \body
      \let\next\iterate
    \else
      \let\next\relax
    \fi
    \next
  }%
  \let\repeat=\fi
}%
\def\RestoreCatcodes{}
\count@=0 %
\loop
  \edef\RestoreCatcodes{%
    \RestoreCatcodes
    \catcode\the\count@=\the\catcode\count@\relax
  }%
\ifnum\count@<255 %
  \advance\count@ 1 %
\repeat

\def\RangeCatcodeInvalid#1#2{%
  \count@=#1\relax
  \loop
    \catcode\count@=15 %
  \ifnum\count@<#2\relax
    \advance\count@ 1 %
  \repeat
}
\def\RangeCatcodeCheck#1#2#3{%
  \count@=#1\relax
  \loop
    \ifnum#3=\catcode\count@
    \else
      \errmessage{%
        Character \the\count@\space
        with wrong catcode \the\catcode\count@\space
        instead of \number#3%
      }%
    \fi
  \ifnum\count@<#2\relax
    \advance\count@ 1 %
  \repeat
}
\def\space{ }
\expandafter\ifx\csname LoadCommand\endcsname\relax
  \def\LoadCommand{\input hologo.sty\relax}%
\fi
\def\Test{%
  \RangeCatcodeInvalid{0}{47}%
  \RangeCatcodeInvalid{58}{64}%
  \RangeCatcodeInvalid{91}{96}%
  \RangeCatcodeInvalid{123}{255}%
  \catcode`\@=12 %
  \catcode`\\=0 %
  \catcode`\%=14 %
  \LoadCommand
  \RangeCatcodeCheck{0}{36}{15}%
  \RangeCatcodeCheck{37}{37}{14}%
  \RangeCatcodeCheck{38}{47}{15}%
  \RangeCatcodeCheck{48}{57}{12}%
  \RangeCatcodeCheck{58}{63}{15}%
  \RangeCatcodeCheck{64}{64}{12}%
  \RangeCatcodeCheck{65}{90}{11}%
  \RangeCatcodeCheck{91}{91}{15}%
  \RangeCatcodeCheck{92}{92}{0}%
  \RangeCatcodeCheck{93}{96}{15}%
  \RangeCatcodeCheck{97}{122}{11}%
  \RangeCatcodeCheck{123}{255}{15}%
  \RestoreCatcodes
}
\Test
\csname @@end\endcsname
\end
%    \end{macrocode}
%    \begin{macrocode}
%</test1>
%    \end{macrocode}
%
% \subsection{Spacefactor}
%
%    The space factor must be 1000 after a logo. If it is greater 1000
%    then the following space is a space after a sentence closing point.
%    If the space factor is smaller 1000 then an immediate following
%    dot is interpreted as abbreviation, not sentence closing point.
%
%    \begin{macrocode}
%<*test-spacefactor>
\NeedsTeXFormat{LaTeX2e}
\documentclass{article}
\usepackage{hologo}[2016/05/12]
\usepackage{kvsetkeys}
\usepackage{qstest}
\IncludeTests{*}
\LogTests{log}{*}{*}
\begin{document}
\begin{qstest}{spacefactor}{spacefactor}
\newcommand*{\Test}[1]{%
  \sbox0{%
    \hologo{#1}%
    \Expect*{1000 (#1)}*{\the\spacefactor\space(#1)}%
  }%
}%
\makeatletter
\def\TestList{}
\def\hologoEntry#1#2#3{%
  \edef\TestList{%
    \ifx\TestList\@empty
    \else
      \TestList,%
    \fi
    #1%
    \ifx\\#2\\%
    \else
      ={variant=#2}%
    \fi
  }%
}
\hologoList
\expandafter\kv@parse@normalized\expandafter{%
  \TestList
}{%
  \begingroup
    \let\@logo=\kv@key
    \ifx\kv@value\relax
    \else
      \expandafter\hologoLogoSetup\expandafter\@logo\expandafter{%
        \kv@value
      }%
    \fi
    \Test\@logo
  \endgroup
  \@gobbletwo
}
\end{qstest}
\end{document}
%</test-spacefactor>
%    \end{macrocode}
%
% \subsection{Complete list}
%
%    \begin{macrocode}
%<*test-list>
\NeedsTeXFormat{LaTeX2e}
\documentclass[12pt,a4paper]{article}
\usepackage{hologo}[2016/05/12]
\usepackage[T1]{fontenc}
\usepackage{lmodern}
\usepackage{parskip}
\usepackage[unicode]{hyperref}[2011/09/28]
\usepackage{bookmark}[2011/09/19]
\bookmarksetup{%
  numbered,%
  open,%
  openlevel=2,%
}
\renewcommand*{\contentsname}{List of logos}
\begin{document}
\tableofcontents
\def\TestFont#1#2#3#4#5#6{%
  \begingroup
    \usefont{#3}{#4}{#5}{#6}%
    \HologoVariant{#1}{#2}/\hologoVariant{#1}{#2}%
    \quad
    \begingroup\scriptsize\hologoVariant{#1}{#2}\endgroup
    \quad
  \endgroup
  (#3/#4/#5/#6)%
  \par
}
\makeatletter
\def\hologoEntry#1#2#3{%
  \section{%
    \HologoVariant{#1}{#2}/\hologoVariant{#1}{#2} %
    {[#1\ifx\\#2\\\else\space(#2)\fi]}% hash-ok
  }% braces around [] because of bug in tex4ht
  \begingroup
    \hypersetup{unicode=false}%
    \bookmark[%
      dest=\@currentHref,%
      rellevel=1,%
      keeplevel,%
    ]{%
      \HologoVariant{#1}{#2}/\hologoVariant{#1}{#2} %
      (PDFDocEncoding)%
    }%
  \endgroup
  \TestFont{#1}{#2}{OT1}{cmr}{m}{n}%
  \TestFont{#1}{#2}{OT1}{cmss}{m}{n}%
  \TestFont{#1}{#2}{OT1}{cmr}{b}{n}%
  \TestFont{#1}{#2}{OT1}{cmr}{m}{it}%
  \TestFont{#1}{#2}{OT1}{cmtt}{m}{n}%
  \TestFont{#1}{#2}{T1}{lmr}{m}{n}%
  \TestFont{#1}{#2}{T1}{lmss}{m}{n}%
  \TestFont{#1}{#2}{T1}{lmr}{b}{n}%
  \TestFont{#1}{#2}{T1}{lmr}{m}{it}%
  \TestFont{#1}{#2}{T1}{lmtt}{m}{n}%
  \TestFont{#1}{#2}{T1}{lmvtt}{m}{n}%
  \TestFont{#1}{#2}{T1}{qtm}{m}{n}%
  \TestFont{#1}{#2}{T1}{qhv}{m}{n}%
  \TestFont{#1}{#2}{T1}{qtm}{b}{n}%
  \TestFont{#1}{#2}{T1}{qtm}{m}{it}%
  \TestFont{#1}{#2}{T1}{qcr}{m}{n}%
  \newpage
}
\makeatother
\hologoList
\end{document}
%</test-list>
%    \end{macrocode}
%
% \section{Installation}
%
% \subsection{Download}
%
% \paragraph{Package.} This package is available on
% CTAN\footnote{\url{ftp://ftp.ctan.org/tex-archive/}}:
% \begin{description}
% \item[\CTAN{macros/latex/contrib/oberdiek/hologo.dtx}] The source file.
% \item[\CTAN{macros/latex/contrib/oberdiek/hologo.pdf}] Documentation.
% \end{description}
%
%
% \paragraph{Bundle.} All the packages of the bundle `oberdiek'
% are also available in a TDS compliant ZIP archive. There
% the packages are already unpacked and the documentation files
% are generated. The files and directories obey the TDS standard.
% \begin{description}
% \item[\CTAN{install/macros/latex/contrib/oberdiek.tds.zip}]
% \end{description}
% \emph{TDS} refers to the standard ``A Directory Structure
% for \TeX\ Files'' (\CTAN{tds/tds.pdf}). Directories
% with \xfile{texmf} in their name are usually organized this way.
%
% \subsection{Bundle installation}
%
% \paragraph{Unpacking.} Unpack the \xfile{oberdiek.tds.zip} in the
% TDS tree (also known as \xfile{texmf} tree) of your choice.
% Example (linux):
% \begin{quote}
%   |unzip oberdiek.tds.zip -d ~/texmf|
% \end{quote}
%
% \paragraph{Script installation.}
% Check the directory \xfile{TDS:scripts/oberdiek/} for
% scripts that need further installation steps.
% Package \xpackage{attachfile2} comes with the Perl script
% \xfile{pdfatfi.pl} that should be installed in such a way
% that it can be called as \texttt{pdfatfi}.
% Example (linux):
% \begin{quote}
%   |chmod +x scripts/oberdiek/pdfatfi.pl|\\
%   |cp scripts/oberdiek/pdfatfi.pl /usr/local/bin/|
% \end{quote}
%
% \subsection{Package installation}
%
% \paragraph{Unpacking.} The \xfile{.dtx} file is a self-extracting
% \docstrip\ archive. The files are extracted by running the
% \xfile{.dtx} through \plainTeX:
% \begin{quote}
%   \verb|tex hologo.dtx|
% \end{quote}
%
% \paragraph{TDS.} Now the different files must be moved into
% the different directories in your installation TDS tree
% (also known as \xfile{texmf} tree):
% \begin{quote}
% \def\t{^^A
% \begin{tabular}{@{}>{\ttfamily}l@{ $\rightarrow$ }>{\ttfamily}l@{}}
%   hologo.sty & tex/generic/oberdiek/hologo.sty\\
%   hologo.pdf & doc/latex/oberdiek/hologo.pdf\\
%   example/hologo-example.tex & doc/latex/oberdiek/example/hologo-example.tex\\
%   test/hologo-test1.tex & doc/latex/oberdiek/test/hologo-test1.tex\\
%   test/hologo-test-spacefactor.tex & doc/latex/oberdiek/test/hologo-test-spacefactor.tex\\
%   test/hologo-test-list.tex & doc/latex/oberdiek/test/hologo-test-list.tex\\
%   hologo.dtx & source/latex/oberdiek/hologo.dtx\\
% \end{tabular}^^A
% }^^A
% \sbox0{\t}^^A
% \ifdim\wd0>\linewidth
%   \begingroup
%     \advance\linewidth by\leftmargin
%     \advance\linewidth by\rightmargin
%   \edef\x{\endgroup
%     \def\noexpand\lw{\the\linewidth}^^A
%   }\x
%   \def\lwbox{^^A
%     \leavevmode
%     \hbox to \linewidth{^^A
%       \kern-\leftmargin\relax
%       \hss
%       \usebox0
%       \hss
%       \kern-\rightmargin\relax
%     }^^A
%   }^^A
%   \ifdim\wd0>\lw
%     \sbox0{\small\t}^^A
%     \ifdim\wd0>\linewidth
%       \ifdim\wd0>\lw
%         \sbox0{\footnotesize\t}^^A
%         \ifdim\wd0>\linewidth
%           \ifdim\wd0>\lw
%             \sbox0{\scriptsize\t}^^A
%             \ifdim\wd0>\linewidth
%               \ifdim\wd0>\lw
%                 \sbox0{\tiny\t}^^A
%                 \ifdim\wd0>\linewidth
%                   \lwbox
%                 \else
%                   \usebox0
%                 \fi
%               \else
%                 \lwbox
%               \fi
%             \else
%               \usebox0
%             \fi
%           \else
%             \lwbox
%           \fi
%         \else
%           \usebox0
%         \fi
%       \else
%         \lwbox
%       \fi
%     \else
%       \usebox0
%     \fi
%   \else
%     \lwbox
%   \fi
% \else
%   \usebox0
% \fi
% \end{quote}
% If you have a \xfile{docstrip.cfg} that configures and enables \docstrip's
% TDS installing feature, then some files can already be in the right
% place, see the documentation of \docstrip.
%
% \subsection{Refresh file name databases}
%
% If your \TeX~distribution
% (\teTeX, \mikTeX, \dots) relies on file name databases, you must refresh
% these. For example, \teTeX\ users run \verb|texhash| or
% \verb|mktexlsr|.
%
% \subsection{Some details for the interested}
%
% \paragraph{Attached source.}
%
% The PDF documentation on CTAN also includes the
% \xfile{.dtx} source file. It can be extracted by
% AcrobatReader 6 or higher. Another option is \textsf{pdftk},
% e.g. unpack the file into the current directory:
% \begin{quote}
%   \verb|pdftk hologo.pdf unpack_files output .|
% \end{quote}
%
% \paragraph{Unpacking with \LaTeX.}
% The \xfile{.dtx} chooses its action depending on the format:
% \begin{description}
% \item[\plainTeX:] Run \docstrip\ and extract the files.
% \item[\LaTeX:] Generate the documentation.
% \end{description}
% If you insist on using \LaTeX\ for \docstrip\ (really,
% \docstrip\ does not need \LaTeX), then inform the autodetect routine
% about your intention:
% \begin{quote}
%   \verb|latex \let\install=y\input{hologo.dtx}|
% \end{quote}
% Do not forget to quote the argument according to the demands
% of your shell.
%
% \paragraph{Generating the documentation.}
% You can use both the \xfile{.dtx} or the \xfile{.drv} to generate
% the documentation. The process can be configured by the
% configuration file \xfile{ltxdoc.cfg}. For instance, put this
% line into this file, if you want to have A4 as paper format:
% \begin{quote}
%   \verb|\PassOptionsToClass{a4paper}{article}|
% \end{quote}
% An example follows how to generate the
% documentation with pdf\LaTeX:
% \begin{quote}
%\begin{verbatim}
%pdflatex hologo.dtx
%makeindex -s gind.ist hologo.idx
%pdflatex hologo.dtx
%makeindex -s gind.ist hologo.idx
%pdflatex hologo.dtx
%\end{verbatim}
% \end{quote}
%
% \section{Catalogue}
%
% The following XML file can be used as source for the
% \href{http://mirror.ctan.org/help/Catalogue/catalogue.html}{\TeX\ Catalogue}.
% The elements \texttt{caption} and \texttt{description} are imported
% from the original XML file from the Catalogue.
% The name of the XML file in the Catalogue is \xfile{hologo.xml}.
%    \begin{macrocode}
%<*catalogue>
<?xml version='1.0' encoding='us-ascii'?>
<!DOCTYPE entry SYSTEM 'catalogue.dtd'>
<entry datestamp='$Date$' modifier='$Author$' id='hologo'>
  <name>hologo</name>
  <caption>A collection of logos with bookmark support.</caption>
  <authorref id='auth:oberdiek'/>
  <copyright owner='Heiko Oberdiek' year='2010-2012'/>
  <license type='lppl1.3'/>
  <version number='1.10'/>
  <description>
    The package defines a single command <tt>\hologo</tt>, whose
    argument is the usual case-confused ASCII version of the logo.
    The command is bookmark-enabled, so that every logo becomes
    available in bookmarks without further work.
    <p/>
    The package is part of the <xref refid='oberdiek'>oberdiek</xref>
    bundle.
  </description>
  <documentation details='Package documentation'
      href='ctan:/macros/latex/contrib/oberdiek/hologo.pdf'/>
  <ctan file='true' path='/macros/latex/contrib/oberdiek/hologo.dtx'/>
  <miktex location='oberdiek'/>
  <texlive location='oberdiek'/>
  <install path='/macros/latex/contrib/oberdiek/oberdiek.tds.zip'/>
</entry>
%</catalogue>
%    \end{macrocode}
%
% \begin{thebibliography}{9}
% \raggedright
%
% \bibitem{btxdoc}
% Oren Patashnik,
% \textit{\hologo{BibTeX}ing},
% 1988-02-08.\\
% \CTAN{biblio/bibtex/base/}
%
% \bibitem{dtklogos}
% Gerd Neugebauer, DANTE,
% \textit{Package \xpackage{dtklogos}},
% 2011-04-25.\\
% \CTAN{usergrps/dante/dtk/dtklogos.sty}
%
% \bibitem{etexman}
% The \hologo{NTS} Team,
% \textit{The \hologo{eTeX} manual},
% 1998-02.\\
% \CTAN{systems/e-tex/v2/doc/}
%
% \bibitem{ExTeX-FAQ}
% The \hologo{ExTeX} group,
% \textit{\hologo{ExTeX}: FAQ -- How is \hologo{ExTeX} typeset?},
% 2007-04-14.\\
% \url{http://www.extex.org/documentation/faq.html}
%
% \bibitem{LyX}
% %@MISC{ LyX,
% %  title = {{LyX 2.0.0 -- The Document Processor [Computer software and manual]}},
% %  author = {{The LyX Team}},
% %  howpublished = {Internet: http://www.lyx.org},
% %  year = {2011-05-08},
% %  note = {Retrieved May 10, 2011, from http://www.lyx.org},
% %  url = {http://www.lyx.org/}
% %}
% The \hologo{LyX} Team,
% \textit{\hologo{LyX} -- The Document Processor},
% 2011-05-08.\\
% \url{http://www.lyx.org/}
%
% \bibitem{OzTeX}
% Andrew Trevorrow,
% \hologo{OzTeX} FAQ: What is the correct way to typeset ``\hologo{OzTeX}''?,
% 2011-09-15 (visited).
% \url{http://www.trevorrow.com/oztex/ozfaq.html#oztex-logo}
%
% \bibitem{PiCTeX}
% Michael Wichura,
% \textit{The \hologo{PiCTeX} macro package},
% 1987-09-21.
% \CTAN{graphics/pictex/}
%
% \bibitem{scrlogo}
% Markus Kohm,
% \textit{\hologo{KOMAScript} Datei \xfile{scrlogo.dtx}},
% 2009-01-30.\\
% \CTAN{install/macros/latex/contrib/komascript.tds.zip}
%
% \end{thebibliography}
%
% \begin{History}
%   \begin{Version}{2010/04/08 v1.0}
%   \item
%     The first version.
%   \end{Version}
%   \begin{Version}{2010/04/16 v1.1}
%   \item
%     \cs{Hologo} added for support of logos at start of a sentence.
%   \item
%     \cs{hologoSetup} and \cs{hologoLogoSetup} added.
%   \item
%     Options \xoption{break}, \xoption{hyphenbreak}, \xoption{spacebreak}
%     added.
%   \item
%     Variant support added by option \xoption{variant}.
%   \end{Version}
%   \begin{Version}{2010/04/24 v1.2}
%   \item
%     \hologo{LaTeX3} added.
%   \item
%     \hologo{VTeX} added.
%   \end{Version}
%   \begin{Version}{2010/11/21 v1.3}
%   \item
%     \hologo{iniTeX}, \hologo{virTeX} added.
%   \end{Version}
%   \begin{Version}{2011/03/25 v1.4}
%   \item
%     \hologo{ConTeXt} with variants added.
%   \item
%     Option \xoption{discretionarybreak} added as refinement for
%     option \xoption{break}.
%   \end{Version}
%   \begin{Version}{2011/04/21 v1.5}
%   \item
%     Wrong TDS directory for test files fixed.
%   \end{Version}
%   \begin{Version}{2011/10/01 v1.6}
%   \item
%     Support for package \xpackage{tex4ht} added.
%   \item
%     Support for \cs{csname} added if \cs{ifincsname} is available.
%   \item
%     New logos:
%     \hologo{(La)TeX},
%     \hologo{biber},
%     \hologo{BibTeX} (\xoption{sc}, \xoption{sf}),
%     \hologo{emTeX},
%     \hologo{ExTeX},
%     \hologo{KOMAScript},
%     \hologo{La},
%     \hologo{LyX},
%     \hologo{MiKTeX},
%     \hologo{NTS},
%     \hologo{OzMF},
%     \hologo{OzMP},
%     \hologo{OzTeX},
%     \hologo{OzTtH},
%     \hologo{PCTeX},
%     \hologo{PiC},
%     \hologo{PiCTeX},
%     \hologo{METAFONT},
%     \hologo{MetaFun},
%     \hologo{METAPOST},
%     \hologo{MetaPost},
%     \hologo{SLiTeX} (\xoption{lift}, \xoption{narrow}, \xoption{simple}),
%     \hologo{SliTeX} (\xoption{narrow}, \xoption{simple}, \xoption{lift}),
%     \hologo{teTeX}.
%   \item
%     Fixes:
%     \hologo{iniTeX},
%     \hologo{pdfLaTeX},
%     \hologo{pdfTeX},
%     \hologo{virTeX}.
%   \item
%     \cs{hologoFontSetup} and \cs{hologoLogoFontSetup} added.
%   \item
%     \cs{hologoVariant} and \cs{HologoVariant} added.
%   \end{Version}
%   \begin{Version}{2011/11/22 v1.7}
%   \item
%     New logos:
%     \hologo{BibTeX8},
%     \hologo{LaTeXML},
%     \hologo{SageTeX},
%     \hologo{TeX4ht},
%     \hologo{TTH}.
%   \item
%     \hologo{Xe} and friends: Driver stuff fixed.
%   \item
%     \hologo{Xe} and friends: Support for italic added.
%   \item
%     \hologo{Xe} and friends: Package support for \xpackage{pgf}
%     and \xpackage{pstricks} added.
%   \end{Version}
%   \begin{Version}{2011/11/29 v1.8}
%   \item
%     New logos:
%     \hologo{HanTheThanh}.
%   \end{Version}
%   \begin{Version}{2011/12/21 v1.9}
%   \item
%     Patch for package \xpackage{ifxetex} added for the case that
%     \cs{newif} is undefined in \hologo{iniTeX}.
%   \item
%     Some fixes for \hologo{iniTeX}.
%   \end{Version}
%   \begin{Version}{2012/04/26 v1.10}
%   \item
%     Fix in bookmark version of logo ``\hologo{HanTheThanh}''.
%   \end{Version}
%   \begin{Version}{2016/05/12 v1.11}
%   \item
%     Update HOLOGO@IfCharExists (previously in texlive)
%   \item define pdfliteral in current luatex.
%   \end{Version}
% \end{History}
%
% \PrintIndex
%
% \Finale
\endinput

%        (quote the arguments according to the demands of your shell)
%
% Documentation:
%    (a) If hologo.drv is present:
%           latex hologo.drv
%    (b) Without hologo.drv:
%           latex hologo.dtx; ...
%    The class ltxdoc loads the configuration file ltxdoc.cfg
%    if available. Here you can specify further options, e.g.
%    use A4 as paper format:
%       \PassOptionsToClass{a4paper}{article}
%
%    Programm calls to get the documentation (example):
%       pdflatex hologo.dtx
%       makeindex -s gind.ist hologo.idx
%       pdflatex hologo.dtx
%       makeindex -s gind.ist hologo.idx
%       pdflatex hologo.dtx
%
% Installation:
%    TDS:tex/generic/oberdiek/hologo.sty
%    TDS:doc/latex/oberdiek/hologo.pdf
%    TDS:doc/latex/oberdiek/example/hologo-example.tex
%    TDS:doc/latex/oberdiek/test/hologo-test1.tex
%    TDS:doc/latex/oberdiek/test/hologo-test-spacefactor.tex
%    TDS:doc/latex/oberdiek/test/hologo-test-list.tex
%    TDS:source/latex/oberdiek/hologo.dtx
%
%<*ignore>
\begingroup
  \catcode123=1 %
  \catcode125=2 %
  \def\x{LaTeX2e}%
\expandafter\endgroup
\ifcase 0\ifx\install y1\fi\expandafter
         \ifx\csname processbatchFile\endcsname\relax\else1\fi
         \ifx\fmtname\x\else 1\fi\relax
\else\csname fi\endcsname
%</ignore>
%<*install>
\input docstrip.tex
\Msg{************************************************************************}
\Msg{* Installation}
\Msg{* Package: hologo 2016/05/12 v1.11 A logo collection with bookmark support (HO)}
\Msg{************************************************************************}

\keepsilent
\askforoverwritefalse

\let\MetaPrefix\relax
\preamble

This is a generated file.

Project: hologo
Version: 2016/05/12 v1.11

Copyright (C) 2010-2012 by
   Heiko Oberdiek <heiko.oberdiek at googlemail.com>

This work may be distributed and/or modified under the
conditions of the LaTeX Project Public License, either
version 1.3c of this license or (at your option) any later
version. This version of this license is in
   http://www.latex-project.org/lppl/lppl-1-3c.txt
and the latest version of this license is in
   http://www.latex-project.org/lppl.txt
and version 1.3 or later is part of all distributions of
LaTeX version 2005/12/01 or later.

This work has the LPPL maintenance status "maintained".

This Current Maintainer of this work is Heiko Oberdiek.

The Base Interpreter refers to any `TeX-Format',
because some files are installed in TDS:tex/generic//.

This work consists of the main source file hologo.dtx
and the derived files
   hologo.sty, hologo.pdf, hologo.ins, hologo.drv, hologo-example.tex,
   hologo-test1.tex, hologo-test-spacefactor.tex,
   hologo-test-list.tex.

\endpreamble
\let\MetaPrefix\DoubleperCent

\generate{%
  \file{hologo.ins}{\from{hologo.dtx}{install}}%
  \file{hologo.drv}{\from{hologo.dtx}{driver}}%
  \usedir{tex/generic/oberdiek}%
  \file{hologo.sty}{\from{hologo.dtx}{package}}%
  \usedir{doc/latex/oberdiek/example}%
  \file{hologo-example.tex}{\from{hologo.dtx}{example}}%
  \usedir{doc/latex/oberdiek/test}%
  \file{hologo-test1.tex}{\from{hologo.dtx}{test1}}%
  \file{hologo-test-spacefactor.tex}{\from{hologo.dtx}{test-spacefactor}}%
  \file{hologo-test-list.tex}{\from{hologo.dtx}{test-list}}%
  \nopreamble
  \nopostamble
  \usedir{source/latex/oberdiek/catalogue}%
  \file{hologo.xml}{\from{hologo.dtx}{catalogue}}%
}

\catcode32=13\relax% active space
\let =\space%
\Msg{************************************************************************}
\Msg{*}
\Msg{* To finish the installation you have to move the following}
\Msg{* file into a directory searched by TeX:}
\Msg{*}
\Msg{*     hologo.sty}
\Msg{*}
\Msg{* To produce the documentation run the file `hologo.drv'}
\Msg{* through LaTeX.}
\Msg{*}
\Msg{* Happy TeXing!}
\Msg{*}
\Msg{************************************************************************}

\endbatchfile
%</install>
%<*ignore>
\fi
%</ignore>
%<*driver>
\NeedsTeXFormat{LaTeX2e}
\ProvidesFile{hologo.drv}%
  [2016/05/12 v1.11 A logo collection with bookmark support (HO)]%
\documentclass{ltxdoc}
\usepackage{holtxdoc}[2011/11/22]
\usepackage{hologo}[2016/05/12]
\usepackage{longtable}
\usepackage{array}
\usepackage{paralist}
%\usepackage[T1]{fontenc}
%\usepackage{lmodern}
\begin{document}
  \DocInput{hologo.dtx}%
\end{document}
%</driver>
% \fi
%
%
% \CharacterTable
%  {Upper-case    \A\B\C\D\E\F\G\H\I\J\K\L\M\N\O\P\Q\R\S\T\U\V\W\X\Y\Z
%   Lower-case    \a\b\c\d\e\f\g\h\i\j\k\l\m\n\o\p\q\r\s\t\u\v\w\x\y\z
%   Digits        \0\1\2\3\4\5\6\7\8\9
%   Exclamation   \!     Double quote  \"     Hash (number) \#
%   Dollar        \$     Percent       \%     Ampersand     \&
%   Acute accent  \'     Left paren    \(     Right paren   \)
%   Asterisk      \*     Plus          \+     Comma         \,
%   Minus         \-     Point         \.     Solidus       \/
%   Colon         \:     Semicolon     \;     Less than     \<
%   Equals        \=     Greater than  \>     Question mark \?
%   Commercial at \@     Left bracket  \[     Backslash     \\
%   Right bracket \]     Circumflex    \^     Underscore    \_
%   Grave accent  \`     Left brace    \{     Vertical bar  \|
%   Right brace   \}     Tilde         \~}
%
% \GetFileInfo{hologo.drv}
%
% \title{The \xpackage{hologo} package}
% \date{2016/05/12 v1.11}
% \author{Heiko Oberdiek\\\xemail{heiko.oberdiek at googlemail.com}}
%
% \maketitle
%
% \begin{abstract}
% This package starts a collection of logos with support for bookmarks
% strings.
% \end{abstract}
%
% \tableofcontents
%
% \section{Documentation}
%
% \subsection{Logo macros}
%
% \begin{declcs}{hologo} \M{name}
% \end{declcs}
% Macro \cs{hologo} sets the logo with name \meta{name}.
% The following table shows the supported names.
%
% \begingroup
%   \def\hologoEntry#1#2#3{^^A
%     #1&#2&\hologoLogoSetup{#1}{variant=#2}\hologo{#1}&#3\tabularnewline
%   }
%   \begin{longtable}{>{\ttfamily}l>{\ttfamily}lll}
%     \rmfamily\bfseries{name} & \rmfamily\bfseries variant
%     & \bfseries logo & \bfseries since\\
%     \hline
%     \endhead
%     \hologoList
%   \end{longtable}
% \endgroup
%
% \begin{declcs}{Hologo} \M{name}
% \end{declcs}
% Macro \cs{Hologo} starts the logo \meta{name} with an uppercase
% letter. As an exception small greek letters are not converted
% to uppercase. Examples, see \hologo{eTeX} and \hologo{ExTeX}.
%
% \subsection{Setup macros}
%
% The package does not support package options, but the following
% setup macros can be used to set options.
%
% \begin{declcs}{hologoSetup} \M{key value list}
% \end{declcs}
% Macro \cs{hologoSetup} sets global options.
%
% \begin{declcs}{hologoLogoSetup} \M{logo} \M{key value list}
% \end{declcs}
% Some options can also be used to configure a logo.
% These settings take precedence over global option settings.
%
% \subsection{Options}\label{sec:options}
%
% There are boolean and string options:
% \begin{description}
% \item[Boolean option:]
% It takes |true| or |false|
% as value. If the value is omitted, then |true| is used.
% \item[String option:]
% A value must be given as string. (But the string might be empty.)
% \end{description}
% The following options can be used both in \cs{hologoSetup}
% and \cs{hologoLogoSetup}:
% \begin{description}
% \def\entry#1{\item[\xoption{#1}:]}
% \entry{break}
%   enables or disables line breaks inside the logo. This setting is
%   refined by options \xoption{hyphenbreak}, \xoption{spacebreak}
%   or \xoption{discretionarybreak}.
%   Default is |false|.
% \entry{hyphenbreak}
%   enables or disables the line break right after the hyphen character.
% \entry{spacebreak}
%   enables or disables line breaks at space characters.
% \entry{discretionarybreak}
%   enables or disables line breaks at hyphenation points
%   (inserted by \cs{-}).
% \end{description}
% Macro \cs{hologoLogoSetup} also knows:
% \begin{description}
% \item[\xoption{variant}:]
%   This is a string option. It specifies a variant of a logo that
%   must exist. An empty string selects the package default variant.
% \end{description}
% Example:
% \begin{quote}
%   |\hologoSetup{break=false}|\\
%   |\hologoLogoSetup{plainTeX}{variant=hyphen,hyphenbreak}|\\
%   Then ``plain-\TeX'' contains one break point after the hyphen.
% \end{quote}
%
% \subsection{Driver options}
%
% Sometimes graphical operations are needed to construct some
% glyphs (e.g.\ \hologo{XeTeX}). If package \xpackage{graphics}
% or package \xpackage{pgf} are found, then the macros are taken
% from there. Otherwise the packge defines its own operations
% and therefore needs the driver information. Many drivers are
% detected automatically (\hologo{pdfTeX}/\hologo{LuaTeX}
% in PDF mode, \hologo{XeTeX}, \hologo{VTeX}). These have precedence
% over a driver option. The driver can be given as package option
% or using \cs{hologoDriverSetup}.
% The following list contains the recognized driver options:
% \begin{itemize}
% \item \xoption{pdftex}, \xoption{luatex}
% \item \xoption{dvipdfm}, \xoption{dvipdfmx}
% \item \xoption{dvips}, \xoption{dvipsone}, \xoption{xdvi}
% \item \xoption{xetex}
% \item \xoption{vtex}
% \end{itemize}
% The left driver of a line is the driver name that is used internally.
% The following names are aliases for drivers that use the
% same method. Therefore the entry in the \xext{log} file for
% the used driver prints the internally used driver name.
% \begin{description}
% \item[\xoption{driverfallback}:]
%   This option expects a driver that is used,
%   if the driver could not be detected automatically.
% \end{description}
%
% \begin{declcs}{hologoDriverSetup} \M{driver option}
% \end{declcs}
% The driver can also be configured after package loading
% using \cs{hologoDriverSetup}, also the way for \hologo{plainTeX}
% to setup the driver.
%
% \subsection{Font setup}
%
% Some logos require a special font, but should also be usable by
% \hologo{plainTeX}. Therefore the package provides some ways
% to influence the font settings. The options below
% take font settings as values. Both font commands
% such as \cs{sffamily} and macros that take one argument
% like \cs{textsf} can be used.
%
% \begin{declcs}{hologoFontSetup} \M{key value list}
% \end{declcs}
% Macro \cs{hologoFontSetup} sets the fonts for all logos.
% Supported keys:
% \begin{description}
% \def\entry#1{\item[\xoption{#1}:]}
% \entry{general}
%   This font is used for all logos. The default is empty.
%   That means no special font is used.
% \entry{bibsf}
%   This font is used for
%   {\hologoLogoSetup{BibTeX}{variant=sf}\hologo{BibTeX}}
%   with variant \xoption{sf}.
% \entry{rm}
%   This font is a serif font. It is used for \hologo{ExTeX}.
% \entry{sc}
%   This font specifies a small caps font. It is used for
%   {\hologoLogoSetup{BibTeX}{variant=sc}\hologo{BibTeX}}
%   with variant \xoption{sc}.
% \entry{sf}
%   This font specifies a sans serif font. The default
%   is \cs{sffamily}, then \cs{sf} is tried. Otherwise
%   a warning is given. It is used by \hologo{KOMAScript}.
% \entry{sy}
%   This is the font for math symbols (e.g. cmsy).
%   It is used by \hologo{AmS}, \hologo{NTS}, \hologo{ExTeX}.
% \entry{logo}
%   \hologo{METAFONT} and \hologo{METAPOST} are using that font.
%   In \hologo{LaTeX} \cs{logofamily} is used and
%   the definitions of package \xpackage{mflogo} are used
%   if the package is not loaded.
%   Otherwise the \cs{tenlogo} is used and defined
%   if it does not already exists.
% \end{description}
%
% \begin{declcs}{hologoLogoFontSetup} \M{logo} \M{key value list}
% \end{declcs}
% Fonts can also be set for a logo or logo component separately,
% see the following list.
% The keys are the same as for \cs{hologoFontSetup}.
%
% \begin{longtable}{>{\ttfamily}l>{\sffamily}ll}
%   \meta{logo} & keys & result\\
%   \hline
%   \endhead
%   BibTeX & bibsf & {\hologoLogoSetup{BibTeX}{variant=sf}\hologo{BibTeX}}\\[.5ex]
%   BibTeX & sc & {\hologoLogoSetup{BibTeX}{variant=sc}\hologo{BibTeX}}\\[.5ex]
%   ExTeX & rm & \hologo{ExTeX}\\
%   SliTeX & rm & \hologo{SliTeX}\\[.5ex]
%   AmS & sy & \hologo{AmS}\\
%   ExTeX & sy & \hologo{ExTeX}\\
%   NTS & sy & \hologo{NTS}\\[.5ex]
%   KOMAScript & sf & \hologo{KOMAScript}\\[.5ex]
%   METAFONT & logo & \hologo{METAFONT}\\
%   METAPOST & logo & \hologo{METAPOST}\\[.5ex]
%   SliTeX & sc \hologo{SliTeX}
% \end{longtable}
%
% \subsubsection{Font order}
%
% For all logos the font \xoption{general} is applied first.
% Example:
%\begin{quote}
%|\hologoFontSetup{general=\color{red}}|
%\end{quote}
% will print red logos.
% Then if the font uses a special font \xoption{sf}, for example,
% the font is applied that is setup by \cs{hologoLogoFontSetup}.
% If this font is not setup, then the common font setup
% by \cs{hologoFontSetup} is used. Otherwise a warning is given,
% that there is no font configured.
%
% \subsection{Additional user macros}
%
% Usually a variant of a logo is configured by using
% \cs{hologoLogoSetup}, because it is bad style to mix
% different variants of the same logo in the same text.
% There the following macros are a convenience for testing.
%
% \begin{declcs}{hologoVariant} \M{name} \M{variant}\\
%   \cs{HologoVariant} \M{name} \M{variant}
% \end{declcs}
% Logo \meta{name} is set using \meta{variant} that specifies
% explicitely which variant of the macro is used. If the argument
% is empty, then the default form of the logo is used
% (configurable by \cs{hologoLogoSetup}).
%
% \cs{HologoVariant} is used if the logo is set in a context
% that needs an uppercase first letter (beginning of a sentence, \dots).
%
% \begin{declcs}{hologoList}\\
%   \cs{hologoEntry} \M{logo} \M{variant} \M{since}
% \end{declcs}
% Macro \cs{hologoList} contains all logos that are provided
% by the package including variants. The list consists of calls
% of \cs{hologoEntry} with three arguments starting with the
% logo name \meta{logo} and its variant \meta{variant}. An empty
% variant means the current default. Argument \meta{since} specifies
% with version of the package \xpackage{hologo} is needed to get
% the logo. If the logo is fixed, then the date gets updated.
% Therefore the date \meta{since} is not exactly the date of
% the first introduction, but rather the date of the latest fix.
%
% Before \cs{hologoList} can be used, macro \cs{hologoEntry} needs
% a definition. The example file in section \ref{sec:example}
% shows applications of \cs{hologoList}.
%
% \subsection{Supported contexts}
%
% Macros \cs{hologo} and friends support special contexts:
% \begin{itemize}
% \item \hologo{LaTeX}'s protection mechanism.
% \item Bookmarks of package \xpackage{hyperref}.
% \item Package \xpackage{tex4ht}.
% \item The macros can be used inside \cs{csname} constructs,
%   if \cs{ifincsname} is available (\hologo{pdfTeX}, \hologo{XeTeX},
%   \hologo{LuaTeX}).
% \end{itemize}
%
% \subsection{Example}
% \label{sec:example}
%
% The following example prints the logos in different fonts.
%    \begin{macrocode}
%<*example>
%<<verbatim
\NeedsTeXFormat{LaTeX2e}
\documentclass[a4paper]{article}
\usepackage[
  hmargin=20mm,
  vmargin=20mm,
]{geometry}
\pagestyle{empty}
\usepackage{hologo}[2016/05/12]
\usepackage{longtable}
\usepackage{array}
\setlength{\extrarowheight}{2pt}
\usepackage[T1]{fontenc}
\usepackage{lmodern}
\usepackage{pdflscape}
\usepackage[
  pdfencoding=auto,
]{hyperref}
\hypersetup{
  pdfauthor={Heiko Oberdiek},
  pdftitle={Example for package `hologo'},
  pdfsubject={Logos with fonts lmr, lmss, qtm, qpl, qhv},
}
\usepackage{bookmark}

% Print the logo list on the console

\begingroup
  \typeout{}%
  \typeout{*** Begin of logo list ***}%
  \newcommand*{\hologoEntry}[3]{%
    \typeout{#1 \ifx\\#2\\\else(#2) \fi[#3]}%
  }%
  \hologoList
  \typeout{*** End of logo list ***}%
  \typeout{}%
\endgroup

\begin{document}
\begin{landscape}

  \section{Example file for package `hologo'}

  % Table for font names

  \begin{longtable}{>{\bfseries}ll}
    \textbf{font} & \textbf{Font name}\\
    \hline
    lmr & Latin Modern Roman\\
    lmss & Latin Modern Sans\\
    qtm & \TeX\ Gyre Termes\\
    qhv & \TeX\ Gyre Heros\\
    qpl & \TeX\ Gyre Pagella\\
  \end{longtable}

  % Logo list with logos in different fonts

  \begingroup
    \newcommand*{\SetVariant}[2]{%
      \ifx\\#2\\%
      \else
        \hologoLogoSetup{#1}{variant=#2}%
      \fi
    }%
    \newcommand*{\hologoEntry}[3]{%
      \SetVariant{#1}{#2}%
      \raisebox{1em}[0pt][0pt]{\hypertarget{#1@#2}{}}%
      \bookmark[%
        dest={#1@#2},%
      ]{%
        #1\ifx\\#2\\\else\space(#2)\fi: \Hologo{#1}, \hologo{#1} %
        [Unicode]%
      }%
      \hypersetup{unicode=false}%
      \bookmark[%
        dest={#1@#2},%
      ]{%
        #1\ifx\\#2\\\else\space(#2)\fi: \Hologo{#1}, \hologo{#1} %
        [PDFDocEncoding]%
      }%
      \texttt{#1}%
      &%
      \texttt{#2}%
      &%
      \Hologo{#1}%
      &%
      \SetVariant{#1}{#2}%
      \hologo{#1}%
      &%
      \SetVariant{#1}{#2}%
      \fontfamily{qtm}\selectfont
      \hologo{#1}%
      &%
      \SetVariant{#1}{#2}%
      \fontfamily{qpl}\selectfont
      \hologo{#1}%
      &%
      \SetVariant{#1}{#2}%
      \textsf{\hologo{#1}}%
      &%
      \SetVariant{#1}{#2}%
      \fontfamily{qhv}\selectfont
      \hologo{#1}%
      \tabularnewline
    }%
    \begin{longtable}{llllllll}%
      \textbf{\textit{logo}} & \textbf{\textit{variant}} &
      \texttt{\string\Hologo} &
      \textbf{lmr} & \textbf{qtm} & \textbf{qpl} &
      \textbf{lmss} & \textbf{qhv}
      \tabularnewline
      \hline
      \endhead
      \hologoList
    \end{longtable}%
  \endgroup

\end{landscape}
\end{document}
%verbatim
%</example>
%    \end{macrocode}
%
% \StopEventually{
% }
%
% \section{Implementation}
%    \begin{macrocode}
%<*package>
%    \end{macrocode}
%    Reload check, especially if the package is not used with \LaTeX.
%    \begin{macrocode}
\begingroup\catcode61\catcode48\catcode32=10\relax%
  \catcode13=5 % ^^M
  \endlinechar=13 %
  \catcode35=6 % #
  \catcode39=12 % '
  \catcode44=12 % ,
  \catcode45=12 % -
  \catcode46=12 % .
  \catcode58=12 % :
  \catcode64=11 % @
  \catcode123=1 % {
  \catcode125=2 % }
  \expandafter\let\expandafter\x\csname ver@hologo.sty\endcsname
  \ifx\x\relax % plain-TeX, first loading
  \else
    \def\empty{}%
    \ifx\x\empty % LaTeX, first loading,
      % variable is initialized, but \ProvidesPackage not yet seen
    \else
      \expandafter\ifx\csname PackageInfo\endcsname\relax
        \def\x#1#2{%
          \immediate\write-1{Package #1 Info: #2.}%
        }%
      \else
        \def\x#1#2{\PackageInfo{#1}{#2, stopped}}%
      \fi
      \x{hologo}{The package is already loaded}%
      \aftergroup\endinput
    \fi
  \fi
\endgroup%
%    \end{macrocode}
%    Package identification:
%    \begin{macrocode}
\begingroup\catcode61\catcode48\catcode32=10\relax%
  \catcode13=5 % ^^M
  \endlinechar=13 %
  \catcode35=6 % #
  \catcode39=12 % '
  \catcode40=12 % (
  \catcode41=12 % )
  \catcode44=12 % ,
  \catcode45=12 % -
  \catcode46=12 % .
  \catcode47=12 % /
  \catcode58=12 % :
  \catcode64=11 % @
  \catcode91=12 % [
  \catcode93=12 % ]
  \catcode123=1 % {
  \catcode125=2 % }
  \expandafter\ifx\csname ProvidesPackage\endcsname\relax
    \def\x#1#2#3[#4]{\endgroup
      \immediate\write-1{Package: #3 #4}%
      \xdef#1{#4}%
    }%
  \else
    \def\x#1#2[#3]{\endgroup
      #2[{#3}]%
      \ifx#1\@undefined
        \xdef#1{#3}%
      \fi
      \ifx#1\relax
        \xdef#1{#3}%
      \fi
    }%
  \fi
\expandafter\x\csname ver@hologo.sty\endcsname
\ProvidesPackage{hologo}%
  [2016/05/12 v1.11 A logo collection with bookmark support (HO)]%
%    \end{macrocode}
%
%    \begin{macrocode}
\begingroup\catcode61\catcode48\catcode32=10\relax%
  \catcode13=5 % ^^M
  \endlinechar=13 %
  \catcode123=1 % {
  \catcode125=2 % }
  \catcode64=11 % @
  \def\x{\endgroup
    \expandafter\edef\csname HOLOGO@AtEnd\endcsname{%
      \endlinechar=\the\endlinechar\relax
      \catcode13=\the\catcode13\relax
      \catcode32=\the\catcode32\relax
      \catcode35=\the\catcode35\relax
      \catcode61=\the\catcode61\relax
      \catcode64=\the\catcode64\relax
      \catcode123=\the\catcode123\relax
      \catcode125=\the\catcode125\relax
    }%
  }%
\x\catcode61\catcode48\catcode32=10\relax%
\catcode13=5 % ^^M
\endlinechar=13 %
\catcode35=6 % #
\catcode64=11 % @
\catcode123=1 % {
\catcode125=2 % }
\def\TMP@EnsureCode#1#2{%
  \edef\HOLOGO@AtEnd{%
    \HOLOGO@AtEnd
    \catcode#1=\the\catcode#1\relax
  }%
  \catcode#1=#2\relax
}
\TMP@EnsureCode{10}{12}% ^^J
\TMP@EnsureCode{33}{12}% !
\TMP@EnsureCode{34}{12}% "
\TMP@EnsureCode{36}{3}% $
\TMP@EnsureCode{38}{4}% &
\TMP@EnsureCode{39}{12}% '
\TMP@EnsureCode{40}{12}% (
\TMP@EnsureCode{41}{12}% )
\TMP@EnsureCode{42}{12}% *
\TMP@EnsureCode{43}{12}% +
\TMP@EnsureCode{44}{12}% ,
\TMP@EnsureCode{45}{12}% -
\TMP@EnsureCode{46}{12}% .
\TMP@EnsureCode{47}{12}% /
\TMP@EnsureCode{58}{12}% :
\TMP@EnsureCode{59}{12}% ;
\TMP@EnsureCode{60}{12}% <
\TMP@EnsureCode{62}{12}% >
\TMP@EnsureCode{63}{12}% ?
\TMP@EnsureCode{91}{12}% [
\TMP@EnsureCode{93}{12}% ]
\TMP@EnsureCode{94}{7}% ^ (superscript)
\TMP@EnsureCode{95}{8}% _ (subscript)
\TMP@EnsureCode{96}{12}% `
\TMP@EnsureCode{124}{12}% |
\edef\HOLOGO@AtEnd{%
  \HOLOGO@AtEnd
  \escapechar\the\escapechar\relax
  \noexpand\endinput
}
\escapechar=92 %
%    \end{macrocode}
%
% \subsection{Logo list}
%
%    \begin{macro}{\hologoList}
%    \begin{macrocode}
\def\hologoList{%
  \hologoEntry{(La)TeX}{}{2011/10/01}%
  \hologoEntry{AmSLaTeX}{}{2010/04/16}%
  \hologoEntry{AmSTeX}{}{2010/04/16}%
  \hologoEntry{biber}{}{2011/10/01}%
  \hologoEntry{BibTeX}{}{2011/10/01}%
  \hologoEntry{BibTeX}{sf}{2011/10/01}%
  \hologoEntry{BibTeX}{sc}{2011/10/01}%
  \hologoEntry{BibTeX8}{}{2011/11/22}%
  \hologoEntry{ConTeXt}{}{2011/03/25}%
  \hologoEntry{ConTeXt}{narrow}{2011/03/25}%
  \hologoEntry{ConTeXt}{simple}{2011/03/25}%
  \hologoEntry{emTeX}{}{2010/04/26}%
  \hologoEntry{eTeX}{}{2010/04/08}%
  \hologoEntry{ExTeX}{}{2011/10/01}%
  \hologoEntry{HanTheThanh}{}{2011/11/29}%
  \hologoEntry{iniTeX}{}{2011/10/01}%
  \hologoEntry{KOMAScript}{}{2011/10/01}%
  \hologoEntry{La}{}{2010/05/08}%
  \hologoEntry{LaTeX}{}{2010/04/08}%
  \hologoEntry{LaTeX2e}{}{2010/04/08}%
  \hologoEntry{LaTeX3}{}{2010/04/24}%
  \hologoEntry{LaTeXe}{}{2010/04/08}%
  \hologoEntry{LaTeXML}{}{2011/11/22}%
  \hologoEntry{LaTeXTeX}{}{2011/10/01}%
  \hologoEntry{LuaLaTeX}{}{2010/04/08}%
  \hologoEntry{LuaTeX}{}{2010/04/08}%
  \hologoEntry{LyX}{}{2011/10/01}%
  \hologoEntry{METAFONT}{}{2011/10/01}%
  \hologoEntry{MetaFun}{}{2011/10/01}%
  \hologoEntry{METAPOST}{}{2011/10/01}%
  \hologoEntry{MetaPost}{}{2011/10/01}%
  \hologoEntry{MiKTeX}{}{2011/10/01}%
  \hologoEntry{NTS}{}{2011/10/01}%
  \hologoEntry{OzMF}{}{2011/10/01}%
  \hologoEntry{OzMP}{}{2011/10/01}%
  \hologoEntry{OzTeX}{}{2011/10/01}%
  \hologoEntry{OzTtH}{}{2011/10/01}%
  \hologoEntry{PCTeX}{}{2011/10/01}%
  \hologoEntry{pdfTeX}{}{2011/10/01}%
  \hologoEntry{pdfLaTeX}{}{2011/10/01}%
  \hologoEntry{PiC}{}{2011/10/01}%
  \hologoEntry{PiCTeX}{}{2011/10/01}%
  \hologoEntry{plainTeX}{}{2010/04/08}%
  \hologoEntry{plainTeX}{space}{2010/04/16}%
  \hologoEntry{plainTeX}{hyphen}{2010/04/16}%
  \hologoEntry{plainTeX}{runtogether}{2010/04/16}%
  \hologoEntry{SageTeX}{}{2011/11/22}%
  \hologoEntry{SLiTeX}{}{2011/10/01}%
  \hologoEntry{SLiTeX}{lift}{2011/10/01}%
  \hologoEntry{SLiTeX}{narrow}{2011/10/01}%
  \hologoEntry{SLiTeX}{simple}{2011/10/01}%
  \hologoEntry{SliTeX}{}{2011/10/01}%
  \hologoEntry{SliTeX}{narrow}{2011/10/01}%
  \hologoEntry{SliTeX}{simple}{2011/10/01}%
  \hologoEntry{SliTeX}{lift}{2011/10/01}%
  \hologoEntry{teTeX}{}{2011/10/01}%
  \hologoEntry{TeX}{}{2010/04/08}%
  \hologoEntry{TeX4ht}{}{2011/11/22}%
  \hologoEntry{TTH}{}{2011/11/22}%
  \hologoEntry{virTeX}{}{2011/10/01}%
  \hologoEntry{VTeX}{}{2010/04/24}%
  \hologoEntry{Xe}{}{2010/04/08}%
  \hologoEntry{XeLaTeX}{}{2010/04/08}%
  \hologoEntry{XeTeX}{}{2010/04/08}%
}
%    \end{macrocode}
%    \end{macro}
%
% \subsection{Load resources}
%
%    \begin{macrocode}
\begingroup\expandafter\expandafter\expandafter\endgroup
\expandafter\ifx\csname RequirePackage\endcsname\relax
  \def\TMP@RequirePackage#1[#2]{%
    \begingroup\expandafter\expandafter\expandafter\endgroup
    \expandafter\ifx\csname ver@#1.sty\endcsname\relax
      \input #1.sty\relax
    \fi
  }%
  \TMP@RequirePackage{ltxcmds}[2011/02/04]%
  \TMP@RequirePackage{infwarerr}[2010/04/08]%
  \TMP@RequirePackage{kvsetkeys}[2010/03/01]%
  \TMP@RequirePackage{kvdefinekeys}[2010/03/01]%
  \TMP@RequirePackage{pdftexcmds}[2010/04/01]%
  \TMP@RequirePackage{ifpdf}[2010/01/28]%
  \TMP@RequirePackage{ifluatex}[2010/03/01]%
  \ltx@IfUndefined{newif}{%
    \expandafter\let\csname newif\endcsname\ltx@newif
  }{}%
  \TMP@RequirePackage{ifxetex}[2009/01/23]%
  \TMP@RequirePackage{ifvtex}[2010/03/01]%
\else
  \RequirePackage{ltxcmds}[2011/02/04]%
  \RequirePackage{infwarerr}[2010/04/08]%
  \RequirePackage{kvsetkeys}[2010/03/01]%
  \RequirePackage{kvdefinekeys}[2010/03/01]%
  \RequirePackage{pdftexcmds}[2010/04/01]%
  \RequirePackage{ifpdf}[2010/01/28]%
  \RequirePackage{ifluatex}[2010/03/01]%
  \RequirePackage{ifxetex}[2009/01/23]%
  \RequirePackage{ifvtex}[2010/03/01]%
\fi
%    \end{macrocode}
%
%    \begin{macro}{\HOLOGO@IfDefined}
%    \begin{macrocode}
\def\HOLOGO@IfExists#1{%
  \ifx\@undefined#1%
    \expandafter\ltx@secondoftwo
  \else
    \ifx\relax#1%
      \expandafter\ltx@secondoftwo
    \else
      \expandafter\expandafter\expandafter\ltx@firstoftwo
    \fi
  \fi
}
%    \end{macrocode}
%    \end{macro}
%
% \subsection{Setup macros}
%
%    \begin{macro}{\hologoSetup}
%    \begin{macrocode}
\def\hologoSetup{%
  \let\HOLOGO@name\relax
  \HOLOGO@Setup
}
%    \end{macrocode}
%    \end{macro}
%
%    \begin{macro}{\hologoLogoSetup}
%    \begin{macrocode}
\def\hologoLogoSetup#1{%
  \edef\HOLOGO@name{#1}%
  \ltx@IfUndefined{HoLogo@\HOLOGO@name}{%
    \@PackageError{hologo}{%
      Unknown logo `\HOLOGO@name'%
    }\@ehc
    \ltx@gobble
  }{%
    \HOLOGO@Setup
  }%
}
%    \end{macrocode}
%    \end{macro}
%
%    \begin{macro}{\HOLOGO@Setup}
%    \begin{macrocode}
\def\HOLOGO@Setup{%
  \kvsetkeys{HoLogo}%
}
%    \end{macrocode}
%    \end{macro}
%
% \subsection{Options}
%
%    \begin{macro}{\HOLOGO@DeclareBoolOption}
%    \begin{macrocode}
\def\HOLOGO@DeclareBoolOption#1{%
  \expandafter\chardef\csname HOLOGOOPT@#1\endcsname\ltx@zero
  \kv@define@key{HoLogo}{#1}[true]{%
    \def\HOLOGO@temp{##1}%
    \ifx\HOLOGO@temp\HOLOGO@true
      \ifx\HOLOGO@name\relax
        \expandafter\chardef\csname HOLOGOOPT@#1\endcsname=\ltx@one
      \else
        \expandafter\chardef\csname
        HoLogoOpt@#1@\HOLOGO@name\endcsname\ltx@one
      \fi
      \HOLOGO@SetBreakAll{#1}%
    \else
      \ifx\HOLOGO@temp\HOLOGO@false
        \ifx\HOLOGO@name\relax
          \expandafter\chardef\csname HOLOGOOPT@#1\endcsname=\ltx@zero
        \else
          \expandafter\chardef\csname
          HoLogoOpt@#1@\HOLOGO@name\endcsname=\ltx@zero
        \fi
        \HOLOGO@SetBreakAll{#1}%
      \else
        \@PackageError{hologo}{%
          Unknown value `##1' for boolean option `#1'.\MessageBreak
          Known values are `true' and `false'%
        }\@ehc
      \fi
    \fi
  }%
}
%    \end{macrocode}
%    \end{macro}
%
%    \begin{macro}{\HOLOGO@SetBreakAll}
%    \begin{macrocode}
\def\HOLOGO@SetBreakAll#1{%
  \def\HOLOGO@temp{#1}%
  \ifx\HOLOGO@temp\HOLOGO@break
    \ifx\HOLOGO@name\relax
      \chardef\HOLOGOOPT@hyphenbreak=\HOLOGOOPT@break
      \chardef\HOLOGOOPT@spacebreak=\HOLOGOOPT@break
      \chardef\HOLOGOOPT@discretionarybreak=\HOLOGOOPT@break
    \else
      \expandafter\chardef
         \csname HoLogoOpt@hyphenbreak@\HOLOGO@name\endcsname=%
         \csname HoLogoOpt@break@\HOLOGO@name\endcsname
      \expandafter\chardef
         \csname HoLogoOpt@spacebreak@\HOLOGO@name\endcsname=%
         \csname HoLogoOpt@break@\HOLOGO@name\endcsname
      \expandafter\chardef
         \csname HoLogoOpt@discretionarybreak@\HOLOGO@name
             \endcsname=%
         \csname HoLogoOpt@break@\HOLOGO@name\endcsname
    \fi
  \fi
}
%    \end{macrocode}
%    \end{macro}
%
%    \begin{macro}{\HOLOGO@true}
%    \begin{macrocode}
\def\HOLOGO@true{true}
%    \end{macrocode}
%    \end{macro}
%    \begin{macro}{\HOLOGO@false}
%    \begin{macrocode}
\def\HOLOGO@false{false}
%    \end{macrocode}
%    \end{macro}
%    \begin{macro}{\HOLOGO@break}
%    \begin{macrocode}
\def\HOLOGO@break{break}
%    \end{macrocode}
%    \end{macro}
%
%    \begin{macrocode}
\HOLOGO@DeclareBoolOption{break}
\HOLOGO@DeclareBoolOption{hyphenbreak}
\HOLOGO@DeclareBoolOption{spacebreak}
\HOLOGO@DeclareBoolOption{discretionarybreak}
%    \end{macrocode}
%
%    \begin{macrocode}
\kv@define@key{HoLogo}{variant}{%
  \ifx\HOLOGO@name\relax
    \@PackageError{hologo}{%
      Option `variant' is not available in \string\hologoSetup,%
      \MessageBreak
      Use \string\hologoLogoSetup\space instead%
    }\@ehc
  \else
    \edef\HOLOGO@temp{#1}%
    \ifx\HOLOGO@temp\ltx@empty
      \expandafter
      \let\csname HoLogoOpt@variant@\HOLOGO@name\endcsname\@undefined
    \else
      \ltx@IfUndefined{HoLogo@\HOLOGO@name @\HOLOGO@temp}{%
        \@PackageError{hologo}{%
          Unknown variant `\HOLOGO@temp' of logo `\HOLOGO@name'%
        }\@ehc
      }{%
        \expandafter
        \let\csname HoLogoOpt@variant@\HOLOGO@name\endcsname
            \HOLOGO@temp
      }%
    \fi
  \fi
}
%    \end{macrocode}
%
%    \begin{macro}{\HOLOGO@Variant}
%    \begin{macrocode}
\def\HOLOGO@Variant#1{%
  #1%
  \ltx@ifundefined{HoLogoOpt@variant@#1}{%
  }{%
    @\csname HoLogoOpt@variant@#1\endcsname
  }%
}
%    \end{macrocode}
%    \end{macro}
%
% \subsection{Break/no-break support}
%
%    \begin{macro}{\HOLOGO@space}
%    \begin{macrocode}
\def\HOLOGO@space{%
  \ltx@ifundefined{HoLogoOpt@spacebreak@\HOLOGO@name}{%
    \ltx@ifundefined{HoLogoOpt@break@\HOLOGO@name}{%
      \chardef\HOLOGO@temp=\HOLOGOOPT@spacebreak
    }{%
      \chardef\HOLOGO@temp=%
        \csname HoLogoOpt@break@\HOLOGO@name\endcsname
    }%
  }{%
    \chardef\HOLOGO@temp=%
      \csname HoLogoOpt@spacebreak@\HOLOGO@name\endcsname
  }%
  \ifcase\HOLOGO@temp
    \penalty10000 %
  \fi
  \ltx@space
}
%    \end{macrocode}
%    \end{macro}
%
%    \begin{macro}{\HOLOGO@hyphen}
%    \begin{macrocode}
\def\HOLOGO@hyphen{%
  \ltx@ifundefined{HoLogoOpt@hyphenbreak@\HOLOGO@name}{%
    \ltx@ifundefined{HoLogoOpt@break@\HOLOGO@name}{%
      \chardef\HOLOGO@temp=\HOLOGOOPT@hyphenbreak
    }{%
      \chardef\HOLOGO@temp=%
        \csname HoLogoOpt@break@\HOLOGO@name\endcsname
    }%
  }{%
    \chardef\HOLOGO@temp=%
      \csname HoLogoOpt@hyphenbreak@\HOLOGO@name\endcsname
  }%
  \ifcase\HOLOGO@temp
    \ltx@mbox{-}%
  \else
    -%
  \fi
}
%    \end{macrocode}
%    \end{macro}
%
%    \begin{macro}{\HOLOGO@discretionary}
%    \begin{macrocode}
\def\HOLOGO@discretionary{%
  \ltx@ifundefined{HoLogoOpt@discretionarybreak@\HOLOGO@name}{%
    \ltx@ifundefined{HoLogoOpt@break@\HOLOGO@name}{%
      \chardef\HOLOGO@temp=\HOLOGOOPT@discretionarybreak
    }{%
      \chardef\HOLOGO@temp=%
        \csname HoLogoOpt@break@\HOLOGO@name\endcsname
    }%
  }{%
    \chardef\HOLOGO@temp=%
      \csname HoLogoOpt@discretionarybreak@\HOLOGO@name\endcsname
  }%
  \ifcase\HOLOGO@temp
  \else
    \-%
  \fi
}
%    \end{macrocode}
%    \end{macro}
%
%    \begin{macro}{\HOLOGO@mbox}
%    \begin{macrocode}
\def\HOLOGO@mbox#1{%
  \ltx@ifundefined{HoLogoOpt@break@\HOLOGO@name}{%
    \chardef\HOLOGO@temp=\HOLOGOOPT@hyphenbreak
  }{%
    \chardef\HOLOGO@temp=%
      \csname HoLogoOpt@break@\HOLOGO@name\endcsname
  }%
  \ifcase\HOLOGO@temp
    \ltx@mbox{#1}%
  \else
    #1%
  \fi
}
%    \end{macrocode}
%    \end{macro}
%
% \subsection{Font support}
%
%    \begin{macro}{\HoLogoFont@font}
%    \begin{tabular}{@{}ll@{}}
%    |#1|:& logo name\\
%    |#2|:& font short name\\
%    |#3|:& text
%    \end{tabular}
%    \begin{macrocode}
\def\HoLogoFont@font#1#2#3{%
  \begingroup
    \ltx@IfUndefined{HoLogoFont@logo@#1.#2}{%
      \ltx@IfUndefined{HoLogoFont@font@#2}{%
        \@PackageWarning{hologo}{%
          Missing font `#2' for logo `#1'%
        }%
        #3%
      }{%
        \csname HoLogoFont@font@#2\endcsname{#3}%
      }%
    }{%
      \csname HoLogoFont@logo@#1.#2\endcsname{#3}%
    }%
  \endgroup
}
%    \end{macrocode}
%    \end{macro}
%
%    \begin{macro}{\HoLogoFont@Def}
%    \begin{macrocode}
\def\HoLogoFont@Def#1{%
  \expandafter\def\csname HoLogoFont@font@#1\endcsname
}
%    \end{macrocode}
%    \end{macro}
%    \begin{macro}{\HoLogoFont@LogoDef}
%    \begin{macrocode}
\def\HoLogoFont@LogoDef#1#2{%
  \expandafter\def\csname HoLogoFont@logo@#1.#2\endcsname
}
%    \end{macrocode}
%    \end{macro}
%
% \subsubsection{Font defaults}
%
%    \begin{macro}{\HoLogoFont@font@general}
%    \begin{macrocode}
\HoLogoFont@Def{general}{}%
%    \end{macrocode}
%    \end{macro}
%
%    \begin{macro}{\HoLogoFont@font@rm}
%    \begin{macrocode}
\ltx@IfUndefined{rmfamily}{%
  \ltx@IfUndefined{rm}{%
  }{%
    \HoLogoFont@Def{rm}{\rm}%
  }%
}{%
  \HoLogoFont@Def{rm}{\rmfamily}%
}
%    \end{macrocode}
%    \end{macro}
%
%    \begin{macro}{\HoLogoFont@font@sf}
%    \begin{macrocode}
\ltx@IfUndefined{sffamily}{%
  \ltx@IfUndefined{sf}{%
  }{%
    \HoLogoFont@Def{sf}{\sf}%
  }%
}{%
  \HoLogoFont@Def{sf}{\sffamily}%
}
%    \end{macrocode}
%    \end{macro}
%
%    \begin{macro}{\HoLogoFont@font@bibsf}
%    In case of \hologo{plainTeX} the original small caps
%    variant is used as default. In \hologo{LaTeX}
%    the definition of package \xpackage{dtklogos} \cite{dtklogos}
%    is used.
%\begin{quote}
%\begin{verbatim}
%\DeclareRobustCommand{\BibTeX}{%
%  B%
%  \kern-.05em%
%  \hbox{%
%    $\m@th$% %% force math size calculations
%    \csname S@\f@size\endcsname
%    \fontsize\sf@size\z@
%    \math@fontsfalse
%    \selectfont
%    I%
%    \kern-.025em%
%    B
%  }%
%  \kern-.08em%
%  \-%
%  \TeX
%}
%\end{verbatim}
%\end{quote}
%    \begin{macrocode}
\ltx@IfUndefined{selectfont}{%
  \ltx@IfUndefined{tensc}{%
    \font\tensc=cmcsc10\relax
  }{}%
  \HoLogoFont@Def{bibsf}{\tensc}%
}{%
  \HoLogoFont@Def{bibsf}{%
    $\mathsurround=0pt$%
    \csname S@\f@size\endcsname
    \fontsize\sf@size{0pt}%
    \math@fontsfalse
    \selectfont
  }%
}
%    \end{macrocode}
%    \end{macro}
%
%    \begin{macro}{\HoLogoFont@font@sc}
%    \begin{macrocode}
\ltx@IfUndefined{scshape}{%
  \ltx@IfUndefined{tensc}{%
    \font\tensc=cmcsc10\relax
  }{}%
  \HoLogoFont@Def{sc}{\tensc}%
}{%
  \HoLogoFont@Def{sc}{\scshape}%
}
%    \end{macrocode}
%    \end{macro}
%
%    \begin{macro}{\HoLogoFont@font@sy}
%    \begin{macrocode}
\ltx@IfUndefined{usefont}{%
  \ltx@IfUndefined{tensy}{%
  }{%
    \HoLogoFont@Def{sy}{\tensy}%
  }%
}{%
  \HoLogoFont@Def{sy}{%
    \usefont{OMS}{cmsy}{m}{n}%
  }%
}
%    \end{macrocode}
%    \end{macro}
%
%    \begin{macro}{\HoLogoFont@font@logo}
%    \begin{macrocode}
\begingroup
  \def\x{LaTeX2e}%
\expandafter\endgroup
\ifx\fmtname\x
  \ltx@IfUndefined{logofamily}{%
    \DeclareRobustCommand\logofamily{%
      \not@math@alphabet\logofamily\relax
      \fontencoding{U}%
      \fontfamily{logo}%
      \selectfont
    }%
  }{}%
  \ltx@IfUndefined{logofamily}{%
  }{%
    \HoLogoFont@Def{logo}{\logofamily}%
  }%
\else
  \ltx@IfUndefined{tenlogo}{%
    \font\tenlogo=logo10\relax
  }{}%
  \HoLogoFont@Def{logo}{\tenlogo}%
\fi
%    \end{macrocode}
%    \end{macro}
%
% \subsubsection{Font setup}
%
%    \begin{macro}{\hologoFontSetup}
%    \begin{macrocode}
\def\hologoFontSetup{%
  \let\HOLOGO@name\relax
  \HOLOGO@FontSetup
}
%    \end{macrocode}
%    \end{macro}
%
%    \begin{macro}{\hologoLogoFontSetup}
%    \begin{macrocode}
\def\hologoLogoFontSetup#1{%
  \edef\HOLOGO@name{#1}%
  \ltx@IfUndefined{HoLogo@\HOLOGO@name}{%
    \@PackageError{hologo}{%
      Unknown logo `\HOLOGO@name'%
    }\@ehc
    \ltx@gobble
  }{%
    \HOLOGO@FontSetup
  }%
}
%    \end{macrocode}
%    \end{macro}
%
%    \begin{macro}{\HOLOGO@FontSetup}
%    \begin{macrocode}
\def\HOLOGO@FontSetup{%
  \kvsetkeys{HoLogoFont}%
}
%    \end{macrocode}
%    \end{macro}
%
%    \begin{macrocode}
\def\HOLOGO@temp#1{%
  \kv@define@key{HoLogoFont}{#1}{%
    \ifx\HOLOGO@name\relax
      \HoLogoFont@Def{#1}{##1}%
    \else
      \HoLogoFont@LogoDef\HOLOGO@name{#1}{##1}%
    \fi
  }%
}
\HOLOGO@temp{general}
\HOLOGO@temp{sf}
%    \end{macrocode}
%
% \subsection{Generic logo commands}
%
%    \begin{macrocode}
\HOLOGO@IfExists\hologo{%
  \@PackageError{hologo}{%
    \string\hologo\ltx@space is already defined.\MessageBreak
    Package loading is aborted%
  }\@ehc
  \HOLOGO@AtEnd
}%
\HOLOGO@IfExists\hologoRobust{%
  \@PackageError{hologo}{%
    \string\hologoRobust\ltx@space is already defined.\MessageBreak
    Package loading is aborted%
  }\@ehc
  \HOLOGO@AtEnd
}%
%    \end{macrocode}
%
% \subsubsection{\cs{hologo} and friends}
%
%    \begin{macrocode}
\ifluatex
  \expandafter\ltx@firstofone
\else
  \expandafter\ltx@gobble
\fi
{%
  \ltx@IfUndefined{ifincsname}{%
    \ifnum\luatexversion<36 %
      \expandafter\ltx@gobble
    \else
      \expandafter\ltx@firstofone
    \fi
    {%
      \begingroup
        \ifcase0%
            \directlua{%
              if tex.enableprimitives then %
                tex.enableprimitives('HOLOGO@', {'ifincsname'})%
              else %
                tex.print('1')%
              end%
            }%
            \ifx\HOLOGO@ifincsname\@undefined 1\fi%
            \relax
          \expandafter\ltx@firstofone
        \else
          \endgroup
          \expandafter\ltx@gobble
        \fi
        {%
          \global\let\ifincsname\HOLOGO@ifincsname
        }%
      \HOLOGO@temp
    }%
  }{}%
}
%    \end{macrocode}
%    \begin{macrocode}
\ltx@IfUndefined{ifincsname}{%
  \catcode`$=14 %
}{%
  \catcode`$=9 %
}
%    \end{macrocode}
%
%    \begin{macro}{\hologo}
%    \begin{macrocode}
\def\hologo#1{%
$ \ifincsname
$   \ltx@ifundefined{HoLogoCs@\HOLOGO@Variant{#1}}{%
$     #1%
$   }{%
$     \csname HoLogoCs@\HOLOGO@Variant{#1}\endcsname\ltx@firstoftwo
$   }%
$ \else
    \HOLOGO@IfExists\texorpdfstring\texorpdfstring\ltx@firstoftwo
    {%
      \hologoRobust{#1}%
    }{%
      \ltx@ifundefined{HoLogoBkm@\HOLOGO@Variant{#1}}{%
        \ltx@ifundefined{HoLogo@#1}{?#1?}{#1}%
      }{%
        \csname HoLogoBkm@\HOLOGO@Variant{#1}\endcsname
        \ltx@firstoftwo
      }%
    }%
$ \fi
}
%    \end{macrocode}
%    \end{macro}
%    \begin{macro}{\Hologo}
%    \begin{macrocode}
\def\Hologo#1{%
$ \ifincsname
$   \ltx@ifundefined{HoLogoCs@\HOLOGO@Variant{#1}}{%
$     #1%
$   }{%
$     \csname HoLogoCs@\HOLOGO@Variant{#1}\endcsname\ltx@secondoftwo
$   }%
$ \else
    \HOLOGO@IfExists\texorpdfstring\texorpdfstring\ltx@firstoftwo
    {%
      \HologoRobust{#1}%
    }{%
      \ltx@ifundefined{HoLogoBkm@\HOLOGO@Variant{#1}}{%
        \ltx@ifundefined{HoLogo@#1}{?#1?}{#1}%
      }{%
        \csname HoLogoBkm@\HOLOGO@Variant{#1}\endcsname
        \ltx@secondoftwo
      }%
    }%
$ \fi
}
%    \end{macrocode}
%    \end{macro}
%
%    \begin{macro}{\hologoVariant}
%    \begin{macrocode}
\def\hologoVariant#1#2{%
  \ifx\relax#2\relax
    \hologo{#1}%
  \else
$   \ifincsname
$     \ltx@ifundefined{HoLogoCs@#1@#2}{%
$       #1%
$     }{%
$       \csname HoLogoCs@#1@#2\endcsname\ltx@firstoftwo
$     }%
$   \else
      \HOLOGO@IfExists\texorpdfstring\texorpdfstring\ltx@firstoftwo
      {%
        \hologoVariantRobust{#1}{#2}%
      }{%
        \ltx@ifundefined{HoLogoBkm@#1@#2}{%
          \ltx@ifundefined{HoLogo@#1}{?#1?}{#1}%
        }{%
          \csname HoLogoBkm@#1@#2\endcsname
          \ltx@firstoftwo
        }%
      }%
$   \fi
  \fi
}
%    \end{macrocode}
%    \end{macro}
%    \begin{macro}{\HologoVariant}
%    \begin{macrocode}
\def\HologoVariant#1#2{%
  \ifx\relax#2\relax
    \Hologo{#1}%
  \else
$   \ifincsname
$     \ltx@ifundefined{HoLogoCs@#1@#2}{%
$       #1%
$     }{%
$       \csname HoLogoCs@#1@#2\endcsname\ltx@secondoftwo
$     }%
$   \else
      \HOLOGO@IfExists\texorpdfstring\texorpdfstring\ltx@firstoftwo
      {%
        \HologoVariantRobust{#1}{#2}%
      }{%
        \ltx@ifundefined{HoLogoBkm@#1@#2}{%
          \ltx@ifundefined{HoLogo@#1}{?#1?}{#1}%
        }{%
          \csname HoLogoBkm@#1@#2\endcsname
          \ltx@secondoftwo
        }%
      }%
$   \fi
  \fi
}
%    \end{macrocode}
%    \end{macro}
%
%    \begin{macrocode}
\catcode`\$=3 %
%    \end{macrocode}
%
% \subsubsection{\cs{hologoRobust} and friends}
%
%    \begin{macro}{\hologoRobust}
%    \begin{macrocode}
\ltx@IfUndefined{protected}{%
  \ltx@IfUndefined{DeclareRobustCommand}{%
    \def\hologoRobust#1%
  }{%
    \DeclareRobustCommand*\hologoRobust[1]%
  }%
}{%
  \protected\def\hologoRobust#1%
}%
{%
  \edef\HOLOGO@name{#1}%
  \ltx@IfUndefined{HoLogo@\HOLOGO@Variant\HOLOGO@name}{%
    \@PackageError{hologo}{%
      Unknown logo `\HOLOGO@name'%
    }\@ehc
    ?\HOLOGO@name?%
  }{%
    \ltx@IfUndefined{ver@tex4ht.sty}{%
      \HoLogoFont@font\HOLOGO@name{general}{%
        \csname HoLogo@\HOLOGO@Variant\HOLOGO@name\endcsname
        \ltx@firstoftwo
      }%
    }{%
      \ltx@IfUndefined{HoLogoHtml@\HOLOGO@Variant\HOLOGO@name}{%
        \HOLOGO@name
      }{%
        \csname HoLogoHtml@\HOLOGO@Variant\HOLOGO@name\endcsname
        \ltx@firstoftwo
      }%
    }%
  }%
}
%    \end{macrocode}
%    \end{macro}
%    \begin{macro}{\HologoRobust}
%    \begin{macrocode}
\ltx@IfUndefined{protected}{%
  \ltx@IfUndefined{DeclareRobustCommand}{%
    \def\HologoRobust#1%
  }{%
    \DeclareRobustCommand*\HologoRobust[1]%
  }%
}{%
  \protected\def\HologoRobust#1%
}%
{%
  \edef\HOLOGO@name{#1}%
  \ltx@IfUndefined{HoLogo@\HOLOGO@Variant\HOLOGO@name}{%
    \@PackageError{hologo}{%
      Unknown logo `\HOLOGO@name'%
    }\@ehc
    ?\HOLOGO@name?%
  }{%
    \ltx@IfUndefined{ver@tex4ht.sty}{%
      \HoLogoFont@font\HOLOGO@name{general}{%
        \csname HoLogo@\HOLOGO@Variant\HOLOGO@name\endcsname
        \ltx@secondoftwo
      }%
    }{%
      \ltx@IfUndefined{HoLogoHtml@\HOLOGO@Variant\HOLOGO@name}{%
        \expandafter\HOLOGO@Uppercase\HOLOGO@name
      }{%
        \csname HoLogoHtml@\HOLOGO@Variant\HOLOGO@name\endcsname
        \ltx@secondoftwo
      }%
    }%
  }%
}
%    \end{macrocode}
%    \end{macro}
%    \begin{macro}{\hologoVariantRobust}
%    \begin{macrocode}
\ltx@IfUndefined{protected}{%
  \ltx@IfUndefined{DeclareRobustCommand}{%
    \def\hologoVariantRobust#1#2%
  }{%
    \DeclareRobustCommand*\hologoVariantRobust[2]%
  }%
}{%
  \protected\def\hologoVariantRobust#1#2%
}%
{%
  \begingroup
    \hologoLogoSetup{#1}{variant={#2}}%
    \hologoRobust{#1}%
  \endgroup
}
%    \end{macrocode}
%    \end{macro}
%    \begin{macro}{\HologoVariantRobust}
%    \begin{macrocode}
\ltx@IfUndefined{protected}{%
  \ltx@IfUndefined{DeclareRobustCommand}{%
    \def\HologoVariantRobust#1#2%
  }{%
    \DeclareRobustCommand*\HologoVariantRobust[2]%
  }%
}{%
  \protected\def\HologoVariantRobust#1#2%
}%
{%
  \begingroup
    \hologoLogoSetup{#1}{variant={#2}}%
    \HologoRobust{#1}%
  \endgroup
}
%    \end{macrocode}
%    \end{macro}
%
%    \begin{macro}{\hologorobust}
%    Macro \cs{hologorobust} is only defined for compatibility.
%    Its use is deprecated.
%    \begin{macrocode}
\def\hologorobust{\hologoRobust}
%    \end{macrocode}
%    \end{macro}
%
% \subsection{Helpers}
%
%    \begin{macro}{\HOLOGO@Uppercase}
%    Macro \cs{HOLOGO@Uppercase} is restricted to \cs{uppercase},
%    because \hologo{plainTeX} or \hologo{iniTeX} do not provide
%    \cs{MakeUppercase}.
%    \begin{macrocode}
\def\HOLOGO@Uppercase#1{\uppercase{#1}}
%    \end{macrocode}
%    \end{macro}
%
%    \begin{macro}{\HOLOGO@PdfdocUnicode}
%    \begin{macrocode}
\def\HOLOGO@PdfdocUnicode{%
  \ifx\ifHy@unicode\iftrue
    \expandafter\ltx@secondoftwo
  \else
    \expandafter\ltx@firstoftwo
  \fi
}
%    \end{macrocode}
%    \end{macro}
%
%    \begin{macro}{\HOLOGO@Math}
%    \begin{macrocode}
\def\HOLOGO@MathSetup{%
  \mathsurround0pt\relax
  \HOLOGO@IfExists\f@series{%
    \if b\expandafter\ltx@car\f@series x\@nil
      \csname boldmath\endcsname
   \fi
  }{}%
}
%    \end{macrocode}
%    \end{macro}
%
%    \begin{macro}{\HOLOGO@TempDimen}
%    \begin{macrocode}
\dimendef\HOLOGO@TempDimen=\ltx@zero
%    \end{macrocode}
%    \end{macro}
%    \begin{macro}{\HOLOGO@NegativeKerning}
%    \begin{macrocode}
\def\HOLOGO@NegativeKerning#1{%
  \begingroup
    \HOLOGO@TempDimen=0pt\relax
    \comma@parse@normalized{#1}{%
      \ifdim\HOLOGO@TempDimen=0pt %
        \expandafter\HOLOGO@@NegativeKerning\comma@entry
      \fi
      \ltx@gobble
    }%
    \ifdim\HOLOGO@TempDimen<0pt %
      \kern\HOLOGO@TempDimen
    \fi
  \endgroup
}
%    \end{macrocode}
%    \end{macro}
%    \begin{macro}{\HOLOGO@@NegativeKerning}
%    \begin{macrocode}
\def\HOLOGO@@NegativeKerning#1#2{%
  \setbox\ltx@zero\hbox{#1#2}%
  \HOLOGO@TempDimen=\wd\ltx@zero
  \setbox\ltx@zero\hbox{#1\kern0pt#2}%
  \advance\HOLOGO@TempDimen by -\wd\ltx@zero
}
%    \end{macrocode}
%    \end{macro}
%
%    \begin{macro}{\HOLOGO@SpaceFactor}
%    \begin{macrocode}
\def\HOLOGO@SpaceFactor{%
  \spacefactor1000 %
}
%    \end{macrocode}
%    \end{macro}
%
%    \begin{macro}{\HOLOGO@Span}
%    \begin{macrocode}
\def\HOLOGO@Span#1#2{%
  \HCode{<span class="HoLogo-#1">}%
  #2%
  \HCode{</span>}%
}
%    \end{macrocode}
%    \end{macro}
%
% \subsubsection{Text subscript}
%
%    \begin{macro}{\HOLOGO@SubScript}%
%    \begin{macrocode}
\def\HOLOGO@SubScript#1{%
  \ltx@IfUndefined{textsubscript}{%
    \ltx@IfUndefined{text}{%
      \ltx@mbox{%
        \mathsurround=0pt\relax
        $%
          _{%
            \ltx@IfUndefined{sf@size}{%
              \mathrm{#1}%
            }{%
              \mbox{%
                \fontsize\sf@size{0pt}\selectfont
                #1%
              }%
            }%
          }%
        $%
      }%
    }{%
      \ltx@mbox{%
        \mathsurround=0pt\relax
        $_{\text{#1}}$%
      }%
    }%
  }{%
    \textsubscript{#1}%
  }%
}
%    \end{macrocode}
%    \end{macro}
%
% \subsection{\hologo{TeX} and friends}
%
% \subsubsection{\hologo{TeX}}
%
%    \begin{macro}{\HoLogo@TeX}
%    Source: \hologo{LaTeX} kernel.
%    \begin{macrocode}
\def\HoLogo@TeX#1{%
  T\kern-.1667em\lower.5ex\hbox{E}\kern-.125emX\HOLOGO@SpaceFactor
}
%    \end{macrocode}
%    \end{macro}
%    \begin{macro}{\HoLogoHtml@TeX}
%    \begin{macrocode}
\def\HoLogoHtml@TeX#1{%
  \HoLogoCss@TeX
  \HOLOGO@Span{TeX}{%
    T%
    \HOLOGO@Span{e}{%
      E%
    }%
    X%
  }%
}
%    \end{macrocode}
%    \end{macro}
%    \begin{macro}{\HoLogoCss@TeX}
%    \begin{macrocode}
\def\HoLogoCss@TeX{%
  \Css{%
    span.HoLogo-TeX span.HoLogo-e{%
      position:relative;%
      top:.5ex;%
      margin-left:-.1667em;%
      margin-right:-.125em;%
    }%
  }%
  \Css{%
    a span.HoLogo-TeX span.HoLogo-e{%
      text-decoration:none;%
    }%
  }%
  \global\let\HoLogoCss@TeX\relax
}
%    \end{macrocode}
%    \end{macro}
%
% \subsubsection{\hologo{plainTeX}}
%
%    \begin{macro}{\HoLogo@plainTeX@space}
%    Source: ``The \hologo{TeX}book''
%    \begin{macrocode}
\def\HoLogo@plainTeX@space#1{%
  \HOLOGO@mbox{#1{p}{P}lain}\HOLOGO@space\hologo{TeX}%
}
%    \end{macrocode}
%    \end{macro}
%    \begin{macro}{\HoLogoCs@plainTeX@space}
%    \begin{macrocode}
\def\HoLogoCs@plainTeX@space#1{#1{p}{P}lain TeX}%
%    \end{macrocode}
%    \end{macro}
%    \begin{macro}{\HoLogoBkm@plainTeX@space}
%    \begin{macrocode}
\def\HoLogoBkm@plainTeX@space#1{%
  #1{p}{P}lain \hologo{TeX}%
}
%    \end{macrocode}
%    \end{macro}
%    \begin{macro}{\HoLogoHtml@plainTeX@space}
%    \begin{macrocode}
\def\HoLogoHtml@plainTeX@space#1{%
  #1{p}{P}lain \hologo{TeX}%
}
%    \end{macrocode}
%    \end{macro}
%
%    \begin{macro}{\HoLogo@plainTeX@hyphen}
%    \begin{macrocode}
\def\HoLogo@plainTeX@hyphen#1{%
  \HOLOGO@mbox{#1{p}{P}lain}\HOLOGO@hyphen\hologo{TeX}%
}
%    \end{macrocode}
%    \end{macro}
%    \begin{macro}{\HoLogoCs@plainTeX@hyphen}
%    \begin{macrocode}
\def\HoLogoCs@plainTeX@hyphen#1{#1{p}{P}lain-TeX}
%    \end{macrocode}
%    \end{macro}
%    \begin{macro}{\HoLogoBkm@plainTeX@hyphen}
%    \begin{macrocode}
\def\HoLogoBkm@plainTeX@hyphen#1{%
  #1{p}{P}lain-\hologo{TeX}%
}
%    \end{macrocode}
%    \end{macro}
%    \begin{macro}{\HoLogoHtml@plainTeX@hyphen}
%    \begin{macrocode}
\def\HoLogoHtml@plainTeX@hyphen#1{%
  #1{p}{P}lain-\hologo{TeX}%
}
%    \end{macrocode}
%    \end{macro}
%
%    \begin{macro}{\HoLogo@plainTeX@runtogether}
%    \begin{macrocode}
\def\HoLogo@plainTeX@runtogether#1{%
  \HOLOGO@mbox{#1{p}{P}lain\hologo{TeX}}%
}
%    \end{macrocode}
%    \end{macro}
%    \begin{macro}{\HoLogoCs@plainTeX@runtogether}
%    \begin{macrocode}
\def\HoLogoCs@plainTeX@runtogether#1{#1{p}{P}lainTeX}
%    \end{macrocode}
%    \end{macro}
%    \begin{macro}{\HoLogoBkm@plainTeX@runtogether}
%    \begin{macrocode}
\def\HoLogoBkm@plainTeX@runtogether#1{%
  #1{p}{P}lain\hologo{TeX}%
}
%    \end{macrocode}
%    \end{macro}
%    \begin{macro}{\HoLogoHtml@plainTeX@runtogether}
%    \begin{macrocode}
\def\HoLogoHtml@plainTeX@runtogether#1{%
  #1{p}{P}lain\hologo{TeX}%
}
%    \end{macrocode}
%    \end{macro}
%
%    \begin{macro}{\HoLogo@plainTeX}
%    \begin{macrocode}
\def\HoLogo@plainTeX{\HoLogo@plainTeX@space}
%    \end{macrocode}
%    \end{macro}
%    \begin{macro}{\HoLogoCs@plainTeX}
%    \begin{macrocode}
\def\HoLogoCs@plainTeX{\HoLogoCs@plainTeX@space}
%    \end{macrocode}
%    \end{macro}
%    \begin{macro}{\HoLogoBkm@plainTeX}
%    \begin{macrocode}
\def\HoLogoBkm@plainTeX{\HoLogoBkm@plainTeX@space}
%    \end{macrocode}
%    \end{macro}
%    \begin{macro}{\HoLogoHtml@plainTeX}
%    \begin{macrocode}
\def\HoLogoHtml@plainTeX{\HoLogoHtml@plainTeX@space}
%    \end{macrocode}
%    \end{macro}
%
% \subsubsection{\hologo{LaTeX}}
%
%    Source: \hologo{LaTeX} kernel.
%\begin{quote}
%\begin{verbatim}
%\DeclareRobustCommand{\LaTeX}{%
%  L%
%  \kern-.36em%
%  {%
%    \sbox\z@ T%
%    \vbox to\ht\z@{%
%      \hbox{%
%        \check@mathfonts
%        \fontsize\sf@size\z@
%        \math@fontsfalse
%        \selectfont
%        A%
%      }%
%      \vss
%    }%
%  }%
%  \kern-.15em%
%  \TeX
%}
%\end{verbatim}
%\end{quote}
%
%    \begin{macro}{\HoLogo@La}
%    \begin{macrocode}
\def\HoLogo@La#1{%
  L%
  \kern-.36em%
  \begingroup
    \setbox\ltx@zero\hbox{T}%
    \vbox to\ht\ltx@zero{%
      \hbox{%
        \ltx@ifundefined{check@mathfonts}{%
          \csname sevenrm\endcsname
        }{%
          \check@mathfonts
          \fontsize\sf@size{0pt}%
          \math@fontsfalse\selectfont
        }%
        A%
      }%
      \vss
    }%
  \endgroup
}
%    \end{macrocode}
%    \end{macro}
%
%    \begin{macro}{\HoLogo@LaTeX}
%    Source: \hologo{LaTeX} kernel.
%    \begin{macrocode}
\def\HoLogo@LaTeX#1{%
  \hologo{La}%
  \kern-.15em%
  \hologo{TeX}%
}
%    \end{macrocode}
%    \end{macro}
%    \begin{macro}{\HoLogoHtml@LaTeX}
%    \begin{macrocode}
\def\HoLogoHtml@LaTeX#1{%
  \HoLogoCss@LaTeX
  \HOLOGO@Span{LaTeX}{%
    L%
    \HOLOGO@Span{a}{%
      A%
    }%
    \hologo{TeX}%
  }%
}
%    \end{macrocode}
%    \end{macro}
%    \begin{macro}{\HoLogoCss@LaTeX}
%    \begin{macrocode}
\def\HoLogoCss@LaTeX{%
  \Css{%
    span.HoLogo-LaTeX span.HoLogo-a{%
      position:relative;%
      top:-.5ex;%
      margin-left:-.36em;%
      margin-right:-.15em;%
      font-size:85\%;%
    }%
  }%
  \global\let\HoLogoCss@LaTeX\relax
}
%    \end{macrocode}
%    \end{macro}
%
% \subsubsection{\hologo{(La)TeX}}
%
%    \begin{macro}{\HoLogo@LaTeXTeX}
%    The kerning around the parentheses is taken
%    from package \xpackage{dtklogos} \cite{dtklogos}.
%\begin{quote}
%\begin{verbatim}
%\DeclareRobustCommand{\LaTeXTeX}{%
%  (%
%  \kern-.15em%
%  L%
%  \kern-.36em%
%  {%
%    \sbox\z@ T%
%    \vbox to\ht0{%
%      \hbox{%
%        $\m@th$%
%        \csname S@\f@size\endcsname
%        \fontsize\sf@size\z@
%        \math@fontsfalse
%        \selectfont
%        A%
%      }%
%      \vss
%    }%
%  }%
%  \kern-.2em%
%  )%
%  \kern-.15em%
%  \TeX
%}
%\end{verbatim}
%\end{quote}
%    \begin{macrocode}
\def\HoLogo@LaTeXTeX#1{%
  (%
  \kern-.15em%
  \hologo{La}%
  \kern-.2em%
  )%
  \kern-.15em%
  \hologo{TeX}%
}
%    \end{macrocode}
%    \end{macro}
%    \begin{macro}{\HoLogoBkm@LaTeXTeX}
%    \begin{macrocode}
\def\HoLogoBkm@LaTeXTeX#1{(La)TeX}
%    \end{macrocode}
%    \end{macro}
%
%    \begin{macro}{\HoLogo@(La)TeX}
%    \begin{macrocode}
\expandafter
\let\csname HoLogo@(La)TeX\endcsname\HoLogo@LaTeXTeX
%    \end{macrocode}
%    \end{macro}
%    \begin{macro}{\HoLogoBkm@(La)TeX}
%    \begin{macrocode}
\expandafter
\let\csname HoLogoBkm@(La)TeX\endcsname\HoLogoBkm@LaTeXTeX
%    \end{macrocode}
%    \end{macro}
%    \begin{macro}{\HoLogoHtml@LaTeXTeX}
%    \begin{macrocode}
\def\HoLogoHtml@LaTeXTeX#1{%
  \HoLogoCss@LaTeXTeX
  \HOLOGO@Span{LaTeXTeX}{%
    (%
    \HOLOGO@Span{L}{L}%
    \HOLOGO@Span{a}{A}%
    \HOLOGO@Span{ParenRight}{)}%
    \hologo{TeX}%
  }%
}
%    \end{macrocode}
%    \end{macro}
%    \begin{macro}{\HoLogoHtml@(La)TeX}
%    Kerning after opening parentheses and before closing parentheses
%    is $-0.1$\,em. The original values $-0.15$\,em
%    looked too ugly for a serif font.
%    \begin{macrocode}
\expandafter
\let\csname HoLogoHtml@(La)TeX\endcsname\HoLogoHtml@LaTeXTeX
%    \end{macrocode}
%    \end{macro}
%    \begin{macro}{\HoLogoCss@LaTeXTeX}
%    \begin{macrocode}
\def\HoLogoCss@LaTeXTeX{%
  \Css{%
    span.HoLogo-LaTeXTeX span.HoLogo-L{%
      margin-left:-.1em;%
    }%
  }%
  \Css{%
    span.HoLogo-LaTeXTeX span.HoLogo-a{%
      position:relative;%
      top:-.5ex;%
      margin-left:-.36em;%
      margin-right:-.1em;%
      font-size:85\%;%
    }%
  }%
  \Css{%
    span.HoLogo-LaTeXTeX span.HoLogo-ParenRight{%
      margin-right:-.15em;%
    }%
  }%
  \global\let\HoLogoCss@LaTeXTeX\relax
}
%    \end{macrocode}
%    \end{macro}
%
% \subsubsection{\hologo{LaTeXe}}
%
%    \begin{macro}{\HoLogo@LaTeXe}
%    Source: \hologo{LaTeX} kernel
%    \begin{macrocode}
\def\HoLogo@LaTeXe#1{%
  \hologo{LaTeX}%
  \kern.15em%
  \hbox{%
    \HOLOGO@MathSetup
    2%
    $_{\textstyle\varepsilon}$%
  }%
}
%    \end{macrocode}
%    \end{macro}
%
%    \begin{macro}{\HoLogoCs@LaTeXe}
%    \begin{macrocode}
\ifnum64=`\^^^^0040\relax % test for big chars of LuaTeX/XeTeX
  \catcode`\$=9 %
  \catcode`\&=14 %
\else
  \catcode`\$=14 %
  \catcode`\&=9 %
\fi
\def\HoLogoCs@LaTeXe#1{%
  LaTeX2%
$ \string ^^^^0395%
& e%
}%
\catcode`\$=3 %
\catcode`\&=4 %
%    \end{macrocode}
%    \end{macro}
%
%    \begin{macro}{\HoLogoBkm@LaTeXe}
%    \begin{macrocode}
\def\HoLogoBkm@LaTeXe#1{%
  \hologo{LaTeX}%
  2%
  \HOLOGO@PdfdocUnicode{e}{\textepsilon}%
}
%    \end{macrocode}
%    \end{macro}
%
%    \begin{macro}{\HoLogoHtml@LaTeXe}
%    \begin{macrocode}
\def\HoLogoHtml@LaTeXe#1{%
  \HoLogoCss@LaTeXe
  \HOLOGO@Span{LaTeX2e}{%
    \hologo{LaTeX}%
    \HOLOGO@Span{2}{2}%
    \HOLOGO@Span{e}{%
      \HOLOGO@MathSetup
      \ensuremath{\textstyle\varepsilon}%
    }%
  }%
}
%    \end{macrocode}
%    \end{macro}
%    \begin{macro}{\HoLogoCss@LaTeXe}
%    \begin{macrocode}
\def\HoLogoCss@LaTeXe{%
  \Css{%
    span.HoLogo-LaTeX2e span.HoLogo-2{%
      padding-left:.15em;%
    }%
  }%
  \Css{%
    span.HoLogo-LaTeX2e span.HoLogo-e{%
      position:relative;%
      top:.35ex;%
      text-decoration:none;%
    }%
  }%
  \global\let\HoLogoCss@LaTeXe\relax
}
%    \end{macrocode}
%    \end{macro}
%
%    \begin{macro}{\HoLogo@LaTeX2e}
%    \begin{macrocode}
\expandafter
\let\csname HoLogo@LaTeX2e\endcsname\HoLogo@LaTeXe
%    \end{macrocode}
%    \end{macro}
%    \begin{macro}{\HoLogoCs@LaTeX2e}
%    \begin{macrocode}
\expandafter
\let\csname HoLogoCs@LaTeX2e\endcsname\HoLogoCs@LaTeXe
%    \end{macrocode}
%    \end{macro}
%    \begin{macro}{\HoLogoBkm@LaTeX2e}
%    \begin{macrocode}
\expandafter
\let\csname HoLogoBkm@LaTeX2e\endcsname\HoLogoBkm@LaTeXe
%    \end{macrocode}
%    \end{macro}
%    \begin{macro}{\HoLogoHtml@LaTeX2e}
%    \begin{macrocode}
\expandafter
\let\csname HoLogoHtml@LaTeX2e\endcsname\HoLogoHtml@LaTeXe
%    \end{macrocode}
%    \end{macro}
%
% \subsubsection{\hologo{LaTeX3}}
%
%    \begin{macro}{\HoLogo@LaTeX3}
%    Source: \hologo{LaTeX} kernel
%    \begin{macrocode}
\expandafter\def\csname HoLogo@LaTeX3\endcsname#1{%
  \hologo{LaTeX}%
  3%
}
%    \end{macrocode}
%    \end{macro}
%
%    \begin{macro}{\HoLogoBkm@LaTeX3}
%    \begin{macrocode}
\expandafter\def\csname HoLogoBkm@LaTeX3\endcsname#1{%
  \hologo{LaTeX}%
  3%
}
%    \end{macrocode}
%    \end{macro}
%    \begin{macro}{\HoLogoHtml@LaTeX3}
%    \begin{macrocode}
\expandafter
\let\csname HoLogoHtml@LaTeX3\expandafter\endcsname
\csname HoLogo@LaTeX3\endcsname
%    \end{macrocode}
%    \end{macro}
%
% \subsubsection{\hologo{LaTeXML}}
%
%    \begin{macro}{\HoLogo@LaTeXML}
%    \begin{macrocode}
\def\HoLogo@LaTeXML#1{%
  \HOLOGO@mbox{%
    \hologo{La}%
    \kern-.15em%
    T%
    \kern-.1667em%
    \lower.5ex\hbox{E}%
    \kern-.125em%
    \HoLogoFont@font{LaTeXML}{sc}{xml}%
  }%
}
%    \end{macrocode}
%    \end{macro}
%    \begin{macro}{\HoLogoHtml@pdfLaTeX}
%    \begin{macrocode}
\def\HoLogoHtml@LaTeXML#1{%
  \HOLOGO@Span{LaTeXML}{%
    \HoLogoCss@LaTeX
    \HoLogoCss@TeX
    \HOLOGO@Span{LaTeX}{%
      L%
      \HOLOGO@Span{a}{%
        A%
      }%
    }%
    \HOLOGO@Span{TeX}{%
      T%
      \HOLOGO@Span{e}{%
        E%
      }%
    }%
    \HCode{<span style="font-variant: small-caps;">}%
    xml%
    \HCode{</span>}%
  }%
}
%    \end{macrocode}
%    \end{macro}
%
% \subsubsection{\hologo{eTeX}}
%
%    \begin{macro}{\HoLogo@eTeX}
%    Source: package \xpackage{etex}
%    \begin{macrocode}
\def\HoLogo@eTeX#1{%
  \ltx@mbox{%
    \HOLOGO@MathSetup
    $\varepsilon$%
    -%
    \HOLOGO@NegativeKerning{-T,T-,To}%
    \hologo{TeX}%
  }%
}
%    \end{macrocode}
%    \end{macro}
%    \begin{macro}{\HoLogoCs@eTeX}
%    \begin{macrocode}
\ifnum64=`\^^^^0040\relax % test for big chars of LuaTeX/XeTeX
  \catcode`\$=9 %
  \catcode`\&=14 %
\else
  \catcode`\$=14 %
  \catcode`\&=9 %
\fi
\def\HoLogoCs@eTeX#1{%
$ #1{\string ^^^^0395}{\string ^^^^03b5}%
& #1{e}{E}%
  TeX%
}%
\catcode`\$=3 %
\catcode`\&=4 %
%    \end{macrocode}
%    \end{macro}
%    \begin{macro}{\HoLogoBkm@eTeX}
%    \begin{macrocode}
\def\HoLogoBkm@eTeX#1{%
  \HOLOGO@PdfdocUnicode{#1{e}{E}}{\textepsilon}%
  -%
  \hologo{TeX}%
}
%    \end{macrocode}
%    \end{macro}
%    \begin{macro}{\HoLogoHtml@eTeX}
%    \begin{macrocode}
\def\HoLogoHtml@eTeX#1{%
  \ltx@mbox{%
    \HOLOGO@MathSetup
    $\varepsilon$%
    -%
    \hologo{TeX}%
  }%
}
%    \end{macrocode}
%    \end{macro}
%
% \subsubsection{\hologo{iniTeX}}
%
%    \begin{macro}{\HoLogo@iniTeX}
%    \begin{macrocode}
\def\HoLogo@iniTeX#1{%
  \HOLOGO@mbox{%
    #1{i}{I}ni\hologo{TeX}%
  }%
}
%    \end{macrocode}
%    \end{macro}
%    \begin{macro}{\HoLogoCs@iniTeX}
%    \begin{macrocode}
\def\HoLogoCs@iniTeX#1{#1{i}{I}niTeX}
%    \end{macrocode}
%    \end{macro}
%    \begin{macro}{\HoLogoBkm@iniTeX}
%    \begin{macrocode}
\def\HoLogoBkm@iniTeX#1{%
  #1{i}{I}ni\hologo{TeX}%
}
%    \end{macrocode}
%    \end{macro}
%    \begin{macro}{\HoLogoHtml@iniTeX}
%    \begin{macrocode}
\let\HoLogoHtml@iniTeX\HoLogo@iniTeX
%    \end{macrocode}
%    \end{macro}
%
% \subsubsection{\hologo{virTeX}}
%
%    \begin{macro}{\HoLogo@virTeX}
%    \begin{macrocode}
\def\HoLogo@virTeX#1{%
  \HOLOGO@mbox{%
    #1{v}{V}ir\hologo{TeX}%
  }%
}
%    \end{macrocode}
%    \end{macro}
%    \begin{macro}{\HoLogoCs@virTeX}
%    \begin{macrocode}
\def\HoLogoCs@virTeX#1{#1{v}{V}irTeX}
%    \end{macrocode}
%    \end{macro}
%    \begin{macro}{\HoLogoBkm@virTeX}
%    \begin{macrocode}
\def\HoLogoBkm@virTeX#1{%
  #1{v}{V}ir\hologo{TeX}%
}
%    \end{macrocode}
%    \end{macro}
%    \begin{macro}{\HoLogoHtml@virTeX}
%    \begin{macrocode}
\let\HoLogoHtml@virTeX\HoLogo@virTeX
%    \end{macrocode}
%    \end{macro}
%
% \subsubsection{\hologo{SliTeX}}
%
% \paragraph{Definitions of the three variants.}
%
%    \begin{macro}{\HoLogo@SLiTeX@lift}
%    \begin{macrocode}
\def\HoLogo@SLiTeX@lift#1{%
  \HoLogoFont@font{SliTeX}{rm}{%
    S%
    \kern-.06em%
    L%
    \kern-.18em%
    \raise.32ex\hbox{\HoLogoFont@font{SliTeX}{sc}{i}}%
    \HOLOGO@discretionary
    \kern-.06em%
    \hologo{TeX}%
  }%
}
%    \end{macrocode}
%    \end{macro}
%    \begin{macro}{\HoLogoBkm@SLiTeX@lift}
%    \begin{macrocode}
\def\HoLogoBkm@SLiTeX@lift#1{SLiTeX}
%    \end{macrocode}
%    \end{macro}
%    \begin{macro}{\HoLogoHtml@SLiTeX@lift}
%    \begin{macrocode}
\def\HoLogoHtml@SLiTeX@lift#1{%
  \HoLogoCss@SLiTeX@lift
  \HOLOGO@Span{SLiTeX-lift}{%
    \HoLogoFont@font{SliTeX}{rm}{%
      S%
      \HOLOGO@Span{L}{L}%
      \HOLOGO@Span{i}{i}%
      \hologo{TeX}%
    }%
  }%
}
%    \end{macrocode}
%    \end{macro}
%    \begin{macro}{\HoLogoCss@SLiTeX@lift}
%    \begin{macrocode}
\def\HoLogoCss@SLiTeX@lift{%
  \Css{%
    span.HoLogo-SLiTeX-lift span.HoLogo-L{%
      margin-left:-.06em;%
      margin-right:-.18em;%
    }%
  }%
  \Css{%
    span.HoLogo-SLiTeX-lift span.HoLogo-i{%
      position:relative;%
      top:-.32ex;%
      margin-right:-.06em;%
      font-variant:small-caps;%
    }%
  }%
  \global\let\HoLogoCss@SLiTeX@lift\relax
}
%    \end{macrocode}
%    \end{macro}
%
%    \begin{macro}{\HoLogo@SliTeX@simple}
%    \begin{macrocode}
\def\HoLogo@SliTeX@simple#1{%
  \HoLogoFont@font{SliTeX}{rm}{%
    \ltx@mbox{%
      \HoLogoFont@font{SliTeX}{sc}{Sli}%
    }%
    \HOLOGO@discretionary
    \hologo{TeX}%
  }%
}
%    \end{macrocode}
%    \end{macro}
%    \begin{macro}{\HoLogoBkm@SliTeX@simple}
%    \begin{macrocode}
\def\HoLogoBkm@SliTeX@simple#1{SliTeX}
%    \end{macrocode}
%    \end{macro}
%    \begin{macro}{\HoLogoHtml@SliTeX@simple}
%    \begin{macrocode}
\let\HoLogoHtml@SliTeX@simple\HoLogo@SliTeX@simple
%    \end{macrocode}
%    \end{macro}
%
%    \begin{macro}{\HoLogo@SliTeX@narrow}
%    \begin{macrocode}
\def\HoLogo@SliTeX@narrow#1{%
  \HoLogoFont@font{SliTeX}{rm}{%
    \ltx@mbox{%
      S%
      \kern-.06em%
      \HoLogoFont@font{SliTeX}{sc}{%
        l%
        \kern-.035em%
        i%
      }%
    }%
    \HOLOGO@discretionary
    \kern-.06em%
    \hologo{TeX}%
  }%
}
%    \end{macrocode}
%    \end{macro}
%    \begin{macro}{\HoLogoBkm@SliTeX@narrow}
%    \begin{macrocode}
\def\HoLogoBkm@SliTeX@narrow#1{SliTeX}
%    \end{macrocode}
%    \end{macro}
%    \begin{macro}{\HoLogoHtml@SliTeX@narrow}
%    \begin{macrocode}
\def\HoLogoHtml@SliTeX@narrow#1{%
  \HoLogoCss@SliTeX@narrow
  \HOLOGO@Span{SliTeX-narrow}{%
    \HoLogoFont@font{SliTeX}{rm}{%
      S%
        \HOLOGO@Span{l}{l}%
        \HOLOGO@Span{i}{i}%
      \hologo{TeX}%
    }%
  }%
}
%    \end{macrocode}
%    \end{macro}
%    \begin{macro}{\HoLogoCss@SliTeX@narrow}
%    \begin{macrocode}
\def\HoLogoCss@SliTeX@narrow{%
  \Css{%
    span.HoLogo-SliTeX-narrow span.HoLogo-l{%
      margin-left:-.06em;%
      margin-right:-.035em;%
      font-variant:small-caps;%
    }%
  }%
  \Css{%
    span.HoLogo-SliTeX-narrow span.HoLogo-i{%
      margin-right:-.06em;%
      font-variant:small-caps;%
    }%
  }%
  \global\let\HoLogoCss@SliTeX@narrow\relax
}
%    \end{macrocode}
%    \end{macro}
%
% \paragraph{Macro set completion.}
%
%    \begin{macro}{\HoLogo@SLiTeX@simple}
%    \begin{macrocode}
\def\HoLogo@SLiTeX@simple{\HoLogo@SliTeX@simple}
%    \end{macrocode}
%    \end{macro}
%    \begin{macro}{\HoLogoBkm@SLiTeX@simple}
%    \begin{macrocode}
\def\HoLogoBkm@SLiTeX@simple{\HoLogoBkm@SliTeX@simple}
%    \end{macrocode}
%    \end{macro}
%    \begin{macro}{\HoLogoHtml@SLiTeX@simple}
%    \begin{macrocode}
\def\HoLogoHtml@SLiTeX@simple{\HoLogoHtml@SliTeX@simple}
%    \end{macrocode}
%    \end{macro}
%
%    \begin{macro}{\HoLogo@SLiTeX@narrow}
%    \begin{macrocode}
\def\HoLogo@SLiTeX@narrow{\HoLogo@SliTeX@narrow}
%    \end{macrocode}
%    \end{macro}
%    \begin{macro}{\HoLogoBkm@SLiTeX@narrow}
%    \begin{macrocode}
\def\HoLogoBkm@SLiTeX@narrow{\HoLogoBkm@SliTeX@narrow}
%    \end{macrocode}
%    \end{macro}
%    \begin{macro}{\HoLogoHtml@SLiTeX@narrow}
%    \begin{macrocode}
\def\HoLogoHtml@SLiTeX@narrow{\HoLogoHtml@SliTeX@narrow}
%    \end{macrocode}
%    \end{macro}
%
%    \begin{macro}{\HoLogo@SliTeX@lift}
%    \begin{macrocode}
\def\HoLogo@SliTeX@lift{\HoLogo@SLiTeX@lift}
%    \end{macrocode}
%    \end{macro}
%    \begin{macro}{\HoLogoBkm@SliTeX@lift}
%    \begin{macrocode}
\def\HoLogoBkm@SliTeX@lift{\HoLogoBkm@SLiTeX@lift}
%    \end{macrocode}
%    \end{macro}
%    \begin{macro}{\HoLogoHtml@SliTeX@lift}
%    \begin{macrocode}
\def\HoLogoHtml@SliTeX@lift{\HoLogoHtml@SLiTeX@lift}
%    \end{macrocode}
%    \end{macro}
%
% \paragraph{Defaults.}
%
%    \begin{macro}{\HoLogo@SLiTeX}
%    \begin{macrocode}
\def\HoLogo@SLiTeX{\HoLogo@SLiTeX@lift}
%    \end{macrocode}
%    \end{macro}
%    \begin{macro}{\HoLogoBkm@SLiTeX}
%    \begin{macrocode}
\def\HoLogoBkm@SLiTeX{\HoLogoBkm@SLiTeX@lift}
%    \end{macrocode}
%    \end{macro}
%    \begin{macro}{\HoLogoHtml@SLiTeX}
%    \begin{macrocode}
\def\HoLogoHtml@SLiTeX{\HoLogoHtml@SLiTeX@lift}
%    \end{macrocode}
%    \end{macro}
%
%    \begin{macro}{\HoLogo@SliTeX}
%    \begin{macrocode}
\def\HoLogo@SliTeX{\HoLogo@SliTeX@narrow}
%    \end{macrocode}
%    \end{macro}
%    \begin{macro}{\HoLogoBkm@SliTeX}
%    \begin{macrocode}
\def\HoLogoBkm@SliTeX{\HoLogoBkm@SliTeX@narrow}
%    \end{macrocode}
%    \end{macro}
%    \begin{macro}{\HoLogoHtml@SliTeX}
%    \begin{macrocode}
\def\HoLogoHtml@SliTeX{\HoLogoHtml@SliTeX@narrow}
%    \end{macrocode}
%    \end{macro}
%
% \subsubsection{\hologo{LuaTeX}}
%
%    \begin{macro}{\HoLogo@LuaTeX}
%    The kerning is an idea of Hans Hagen, see mailing list
%    `luatex at tug dot org' in March 2010.
%    \begin{macrocode}
\def\HoLogo@LuaTeX#1{%
  \HOLOGO@mbox{%
    Lua%
    \HOLOGO@NegativeKerning{aT,oT,To}%
    \hologo{TeX}%
  }%
}
%    \end{macrocode}
%    \end{macro}
%    \begin{macro}{\HoLogoHtml@LuaTeX}
%    \begin{macrocode}
\let\HoLogoHtml@LuaTeX\HoLogo@LuaTeX
%    \end{macrocode}
%    \end{macro}
%
% \subsubsection{\hologo{LuaLaTeX}}
%
%    \begin{macro}{\HoLogo@LuaLaTeX}
%    \begin{macrocode}
\def\HoLogo@LuaLaTeX#1{%
  \HOLOGO@mbox{%
    Lua%
    \hologo{LaTeX}%
  }%
}
%    \end{macrocode}
%    \end{macro}
%    \begin{macro}{\HoLogoHtml@LuaLaTeX}
%    \begin{macrocode}
\let\HoLogoHtml@LuaLaTeX\HoLogo@LuaLaTeX
%    \end{macrocode}
%    \end{macro}
%
% \subsubsection{\hologo{XeTeX}, \hologo{XeLaTeX}}
%
%    \begin{macro}{\HOLOGO@IfCharExists}
%    \begin{macrocode}
\ifluatex
  \ifnum\luatexversion<36 %
  \else
    \def\HOLOGO@IfCharExists#1{%
      \ifnum
        \directlua{%
           if luaotfload and luaotfload.aux then
             if luaotfload.aux.font_has_glyph(%
                    font.current(), \number#1) then % 	 
	       tex.print("1") % 	 
	     end % 	 
	   elseif font and font.fonts and font.current then %
            local f = font.fonts[font.current()]%
            if f.characters and f.characters[\number#1] then %
              tex.print("1")%
            end %
          end%
        }0=\ltx@zero
        \expandafter\ltx@secondoftwo
      \else
        \expandafter\ltx@firstoftwo
      \fi
    }%
  \fi
\fi
\ltx@IfUndefined{HOLOGO@IfCharExists}{%
  \def\HOLOGO@@IfCharExists#1{%
    \begingroup
      \tracinglostchars=\ltx@zero
      \setbox\ltx@zero=\hbox{%
        \kern7sp\char#1\relax
        \ifnum\lastkern>\ltx@zero
          \expandafter\aftergroup\csname iffalse\endcsname
        \else
          \expandafter\aftergroup\csname iftrue\endcsname
        \fi
      }%
      % \if{true|false} from \aftergroup
      \endgroup
      \expandafter\ltx@firstoftwo
    \else
      \endgroup
      \expandafter\ltx@secondoftwo
    \fi
  }%
  \ifxetex
    \ltx@IfUndefined{XeTeXfonttype}{}{%
      \ltx@IfUndefined{XeTeXcharglyph}{}{%
        \def\HOLOGO@IfCharExists#1{%
          \ifnum\XeTeXfonttype\font>\ltx@zero
            \expandafter\ltx@firstofthree
          \else
            \expandafter\ltx@gobble
          \fi
          {%
            \ifnum\XeTeXcharglyph#1>\ltx@zero
              \expandafter\ltx@firstoftwo
            \else
              \expandafter\ltx@secondoftwo
            \fi
          }%
          \HOLOGO@@IfCharExists{#1}%
        }%
      }%
    }%
  \fi
}{}
\ltx@ifundefined{HOLOGO@IfCharExists}{%
  \ifnum64=`\^^^^0040\relax % test for big chars of LuaTeX/XeTeX
    \let\HOLOGO@IfCharExists\HOLOGO@@IfCharExists
  \else
    \def\HOLOGO@IfCharExists#1{%
      \ifnum#1>255 %
        \expandafter\ltx@fourthoffour
      \fi
      \HOLOGO@@IfCharExists{#1}%
    }%
  \fi
}{}
%    \end{macrocode}
%    \end{macro}
%
%    \begin{macro}{\HoLogo@Xe}
%    Source: package \xpackage{dtklogos}
%    \begin{macrocode}
\def\HoLogo@Xe#1{%
  X%
  \kern-.1em\relax
  \HOLOGO@IfCharExists{"018E}{%
    \lower.5ex\hbox{\char"018E}%
  }{%
    \chardef\HOLOGO@choice=\ltx@zero
    \ifdim\fontdimen\ltx@one\font>0pt %
      \ltx@IfUndefined{rotatebox}{%
        \ltx@IfUndefined{pgftext}{%
          \ltx@IfUndefined{psscalebox}{%
            \ltx@IfUndefined{HOLOGO@ScaleBox@\hologoDriver}{%
            }{%
              \chardef\HOLOGO@choice=4 %
            }%
          }{%
            \chardef\HOLOGO@choice=3 %
          }%
        }{%
          \chardef\HOLOGO@choice=2 %
        }%
      }{%
        \chardef\HOLOGO@choice=1 %
      }%
      \ifcase\HOLOGO@choice
        \HOLOGO@WarningUnsupportedDriver{Xe}%
        e%
      \or % 1: \rotatebox
        \begingroup
          \setbox\ltx@zero\hbox{\rotatebox{180}{E}}%
          \ltx@LocDimenA=\dp\ltx@zero
          \advance\ltx@LocDimenA by -.5ex\relax
          \raise\ltx@LocDimenA\box\ltx@zero
        \endgroup
      \or % 2: \pgftext
        \lower.5ex\hbox{%
          \pgfpicture
            \pgftext[rotate=180]{E}%
          \endpgfpicture
        }%
      \or % 3: \psscalebox
        \begingroup
          \setbox\ltx@zero\hbox{\psscalebox{-1 -1}{E}}%
          \ltx@LocDimenA=\dp\ltx@zero
          \advance\ltx@LocDimenA by -.5ex\relax
          \raise\ltx@LocDimenA\box\ltx@zero
        \endgroup
      \or % 4: \HOLOGO@PointReflectBox
        \lower.5ex\hbox{\HOLOGO@PointReflectBox{E}}%
      \else
        \@PackageError{hologo}{Internal error (choice/it}\@ehc
      \fi
    \else
      \ltx@IfUndefined{reflectbox}{%
        \ltx@IfUndefined{pgftext}{%
          \ltx@IfUndefined{psscalebox}{%
            \ltx@IfUndefined{HOLOGO@ScaleBox@\hologoDriver}{%
            }{%
              \chardef\HOLOGO@choice=4 %
            }%
          }{%
            \chardef\HOLOGO@choice=3 %
          }%
        }{%
          \chardef\HOLOGO@choice=2 %
        }%
      }{%
        \chardef\HOLOGO@choice=1 %
      }%
      \ifcase\HOLOGO@choice
        \HOLOGO@WarningUnsupportedDriver{Xe}%
        e%
      \or % 1: reflectbox
        \lower.5ex\hbox{%
          \reflectbox{E}%
        }%
      \or % 2: \pgftext
        \lower.5ex\hbox{%
          \pgfpicture
            \pgftransformxscale{-1}%
            \pgftext{E}%
          \endpgfpicture
        }%
      \or % 3: \psscalebox
        \lower.5ex\hbox{%
          \psscalebox{-1 1}{E}%
        }%
      \or % 4: \HOLOGO@Reflectbox
        \lower.5ex\hbox{%
          \HOLOGO@ReflectBox{E}%
        }%
      \else
        \@PackageError{hologo}{Internal error (choice/up)}\@ehc
      \fi
    \fi
  }%
}
%    \end{macrocode}
%    \end{macro}
%    \begin{macro}{\HoLogoHtml@Xe}
%    \begin{macrocode}
\def\HoLogoHtml@Xe#1{%
  \HoLogoCss@Xe
  \HOLOGO@Span{Xe}{%
    X%
    \HOLOGO@Span{e}{%
      \HCode{&\ltx@hashchar x018e;}%
    }%
  }%
}
%    \end{macrocode}
%    \end{macro}
%    \begin{macro}{\HoLogoCss@Xe}
%    \begin{macrocode}
\def\HoLogoCss@Xe{%
  \Css{%
    span.HoLogo-Xe span.HoLogo-e{%
      position:relative;%
      top:.5ex;%
      left-margin:-.1em;%
    }%
  }%
  \global\let\HoLogoCss@Xe\relax
}
%    \end{macrocode}
%    \end{macro}
%
%    \begin{macro}{\HoLogo@XeTeX}
%    \begin{macrocode}
\def\HoLogo@XeTeX#1{%
  \hologo{Xe}%
  \kern-.15em\relax
  \hologo{TeX}%
}
%    \end{macrocode}
%    \end{macro}
%
%    \begin{macro}{\HoLogoHtml@XeTeX}
%    \begin{macrocode}
\def\HoLogoHtml@XeTeX#1{%
  \HoLogoCss@XeTeX
  \HOLOGO@Span{XeTeX}{%
    \hologo{Xe}%
    \hologo{TeX}%
  }%
}
%    \end{macrocode}
%    \end{macro}
%    \begin{macro}{\HoLogoCss@XeTeX}
%    \begin{macrocode}
\def\HoLogoCss@XeTeX{%
  \Css{%
    span.HoLogo-XeTeX span.HoLogo-TeX{%
      margin-left:-.15em;%
    }%
  }%
  \global\let\HoLogoCss@XeTeX\relax
}
%    \end{macrocode}
%    \end{macro}
%
%    \begin{macro}{\HoLogo@XeLaTeX}
%    \begin{macrocode}
\def\HoLogo@XeLaTeX#1{%
  \hologo{Xe}%
  \kern-.13em%
  \hologo{LaTeX}%
}
%    \end{macrocode}
%    \end{macro}
%    \begin{macro}{\HoLogoHtml@XeLaTeX}
%    \begin{macrocode}
\def\HoLogoHtml@XeLaTeX#1{%
  \HoLogoCss@XeLaTeX
  \HOLOGO@Span{XeLaTeX}{%
    \hologo{Xe}%
    \hologo{LaTeX}%
  }%
}
%    \end{macrocode}
%    \end{macro}
%    \begin{macro}{\HoLogoCss@XeLaTeX}
%    \begin{macrocode}
\def\HoLogoCss@XeLaTeX{%
  \Css{%
    span.HoLogo-XeLaTeX span.HoLogo-Xe{%
      margin-right:-.13em;%
    }%
  }%
  \global\let\HoLogoCss@XeLaTeX\relax
}
%    \end{macrocode}
%    \end{macro}
%
% \subsubsection{\hologo{pdfTeX}, \hologo{pdfLaTeX}}
%
%    \begin{macro}{\HoLogo@pdfTeX}
%    \begin{macrocode}
\def\HoLogo@pdfTeX#1{%
  \HOLOGO@mbox{%
    #1{p}{P}df\hologo{TeX}%
  }%
}
%    \end{macrocode}
%    \end{macro}
%    \begin{macro}{\HoLogoCs@pdfTeX}
%    \begin{macrocode}
\def\HoLogoCs@pdfTeX#1{#1{p}{P}dfTeX}
%    \end{macrocode}
%    \end{macro}
%    \begin{macro}{\HoLogoBkm@pdfTeX}
%    \begin{macrocode}
\def\HoLogoBkm@pdfTeX#1{%
  #1{p}{P}df\hologo{TeX}%
}
%    \end{macrocode}
%    \end{macro}
%    \begin{macro}{\HoLogoHtml@pdfTeX}
%    \begin{macrocode}
\let\HoLogoHtml@pdfTeX\HoLogo@pdfTeX
%    \end{macrocode}
%    \end{macro}
%
%    \begin{macro}{\HoLogo@pdfLaTeX}
%    \begin{macrocode}
\def\HoLogo@pdfLaTeX#1{%
  \HOLOGO@mbox{%
    #1{p}{P}df\hologo{LaTeX}%
  }%
}
%    \end{macrocode}
%    \end{macro}
%    \begin{macro}{\HoLogoCs@pdfLaTeX}
%    \begin{macrocode}
\def\HoLogoCs@pdfLaTeX#1{#1{p}{P}dfLaTeX}
%    \end{macrocode}
%    \end{macro}
%    \begin{macro}{\HoLogoBkm@pdfLaTeX}
%    \begin{macrocode}
\def\HoLogoBkm@pdfLaTeX#1{%
  #1{p}{P}df\hologo{LaTeX}%
}
%    \end{macrocode}
%    \end{macro}
%    \begin{macro}{\HoLogoHtml@pdfLaTeX}
%    \begin{macrocode}
\let\HoLogoHtml@pdfLaTeX\HoLogo@pdfLaTeX
%    \end{macrocode}
%    \end{macro}
%
% \subsubsection{\hologo{VTeX}}
%
%    \begin{macro}{\HoLogo@VTeX}
%    \begin{macrocode}
\def\HoLogo@VTeX#1{%
  \HOLOGO@mbox{%
    V\hologo{TeX}%
  }%
}
%    \end{macrocode}
%    \end{macro}
%    \begin{macro}{\HoLogoHtml@VTeX}
%    \begin{macrocode}
\let\HoLogoHtml@VTeX\HoLogo@VTeX
%    \end{macrocode}
%    \end{macro}
%
% \subsubsection{\hologo{AmS}, \dots}
%
%    Source: class \xclass{amsdtx}
%
%    \begin{macro}{\HoLogo@AmS}
%    \begin{macrocode}
\def\HoLogo@AmS#1{%
  \HoLogoFont@font{AmS}{sy}{%
    A%
    \kern-.1667em%
    \lower.5ex\hbox{M}%
    \kern-.125em%
    S%
  }%
}
%    \end{macrocode}
%    \end{macro}
%    \begin{macro}{\HoLogoBkm@AmS}
%    \begin{macrocode}
\def\HoLogoBkm@AmS#1{AmS}
%    \end{macrocode}
%    \end{macro}
%    \begin{macro}{\HoLogoHtml@AmS}
%    \begin{macrocode}
\def\HoLogoHtml@AmS#1{%
  \HoLogoCss@AmS
%  \HoLogoFont@font{AmS}{sy}{%
    \HOLOGO@Span{AmS}{%
      A%
      \HOLOGO@Span{M}{M}%
      S%
    }%
%   }%
}
%    \end{macrocode}
%    \end{macro}
%    \begin{macro}{\HoLogoCss@AmS}
%    \begin{macrocode}
\def\HoLogoCss@AmS{%
  \Css{%
    span.HoLogo-AmS span.HoLogo-M{%
      position:relative;%
      top:.5ex;%
      margin-left:-.1667em;%
      margin-right:-.125em;%
      text-decoration:none;%
    }%
  }%
  \global\let\HoLogoCss@AmS\relax
}
%    \end{macrocode}
%    \end{macro}
%
%    \begin{macro}{\HoLogo@AmSTeX}
%    \begin{macrocode}
\def\HoLogo@AmSTeX#1{%
  \hologo{AmS}%
  \HOLOGO@hyphen
  \hologo{TeX}%
}
%    \end{macrocode}
%    \end{macro}
%    \begin{macro}{\HoLogoBkm@AmSTeX}
%    \begin{macrocode}
\def\HoLogoBkm@AmSTeX#1{AmS-TeX}%
%    \end{macrocode}
%    \end{macro}
%    \begin{macro}{\HoLogoHtml@AmSTeX}
%    \begin{macrocode}
\let\HoLogoHtml@AmSTeX\HoLogo@AmSTeX
%    \end{macrocode}
%    \end{macro}
%
%    \begin{macro}{\HoLogo@AmSLaTeX}
%    \begin{macrocode}
\def\HoLogo@AmSLaTeX#1{%
  \hologo{AmS}%
  \HOLOGO@hyphen
  \hologo{LaTeX}%
}
%    \end{macrocode}
%    \end{macro}
%    \begin{macro}{\HoLogoBkm@AmSLaTeX}
%    \begin{macrocode}
\def\HoLogoBkm@AmSLaTeX#1{AmS-LaTeX}%
%    \end{macrocode}
%    \end{macro}
%    \begin{macro}{\HoLogoHtml@AmSLaTeX}
%    \begin{macrocode}
\let\HoLogoHtml@AmSLaTeX\HoLogo@AmSLaTeX
%    \end{macrocode}
%    \end{macro}
%
% \subsubsection{\hologo{BibTeX}}
%
%    \begin{macro}{\HoLogo@BibTeX@sc}
%    A definition of \hologo{BibTeX} is provided in
%    the documentation source for the manual of \hologo{BibTeX}
%    \cite{btxdoc}.
%\begin{quote}
%\begin{verbatim}
%\def\BibTeX{%
%  {%
%    \rm
%    B%
%    \kern-.05em%
%    {%
%      \sc
%      i%
%      \kern-.025em %
%      b%
%    }%
%    \kern-.08em
%    T%
%    \kern-.1667em%
%    \lower.7ex\hbox{E}%
%    \kern-.125em%
%    X%
%  }%
%}
%\end{verbatim}
%\end{quote}
%    \begin{macrocode}
\def\HoLogo@BibTeX@sc#1{%
  B%
  \kern-.05em%
  \HoLogoFont@font{BibTeX}{sc}{%
    i%
    \kern-.025em%
    b%
  }%
  \HOLOGO@discretionary
  \kern-.08em%
  \hologo{TeX}%
}
%    \end{macrocode}
%    \end{macro}
%    \begin{macro}{\HoLogoHtml@BibTeX@sc}
%    \begin{macrocode}
\def\HoLogoHtml@BibTeX@sc#1{%
  \HoLogoCss@BibTeX@sc
  \HOLOGO@Span{BibTeX-sc}{%
    B%
    \HOLOGO@Span{i}{i}%
    \HOLOGO@Span{b}{b}%
    \hologo{TeX}%
  }%
}
%    \end{macrocode}
%    \end{macro}
%    \begin{macro}{\HoLogoCss@BibTeX@sc}
%    \begin{macrocode}
\def\HoLogoCss@BibTeX@sc{%
  \Css{%
    span.HoLogo-BibTeX-sc span.HoLogo-i{%
      margin-left:-.05em;%
      margin-right:-.025em;%
      font-variant:small-caps;%
    }%
  }%
  \Css{%
    span.HoLogo-BibTeX-sc span.HoLogo-b{%
      margin-right:-.08em;%
      font-variant:small-caps;%
    }%
  }%
  \global\let\HoLogoCss@BibTeX@sc\relax
}
%    \end{macrocode}
%    \end{macro}
%
%    \begin{macro}{\HoLogo@BibTeX@sf}
%    Variant \xoption{sf} avoids trouble with unavailable
%    small caps fonts (e.g., bold versions of Computer Modern or
%    Latin Modern). The definition is taken from
%    package \xpackage{dtklogos} \cite{dtklogos}.
%\begin{quote}
%\begin{verbatim}
%\DeclareRobustCommand{\BibTeX}{%
%  B%
%  \kern-.05em%
%  \hbox{%
%    $\m@th$% %% force math size calculations
%    \csname S@\f@size\endcsname
%    \fontsize\sf@size\z@
%    \math@fontsfalse
%    \selectfont
%    I%
%    \kern-.025em%
%    B
%  }%
%  \kern-.08em%
%  \-%
%  \TeX
%}
%\end{verbatim}
%\end{quote}
%    \begin{macrocode}
\def\HoLogo@BibTeX@sf#1{%
  B%
  \kern-.05em%
  \HoLogoFont@font{BibTeX}{bibsf}{%
    I%
    \kern-.025em%
    B%
  }%
  \HOLOGO@discretionary
  \kern-.08em%
  \hologo{TeX}%
}
%    \end{macrocode}
%    \end{macro}
%    \begin{macro}{\HoLogoHtml@BibTeX@sf}
%    \begin{macrocode}
\def\HoLogoHtml@BibTeX@sf#1{%
  \HoLogoCss@BibTeX@sf
  \HOLOGO@Span{BibTeX-sf}{%
    B%
    \HoLogoFont@font{BibTeX}{bibsf}{%
      \HOLOGO@Span{i}{I}%
      B%
    }%
    \hologo{TeX}%
  }%
}
%    \end{macrocode}
%    \end{macro}
%    \begin{macro}{\HoLogoCss@BibTeX@sf}
%    \begin{macrocode}
\def\HoLogoCss@BibTeX@sf{%
  \Css{%
    span.HoLogo-BibTeX-sf span.HoLogo-i{%
      margin-left:-.05em;%
      margin-right:-.025em;%
    }%
  }%
  \Css{%
    span.HoLogo-BibTeX-sf span.HoLogo-TeX{%
      margin-left:-.08em;%
    }%
  }%
  \global\let\HoLogoCss@BibTeX@sf\relax
}
%    \end{macrocode}
%    \end{macro}
%
%    \begin{macro}{\HoLogo@BibTeX}
%    \begin{macrocode}
\def\HoLogo@BibTeX{\HoLogo@BibTeX@sf}
%    \end{macrocode}
%    \end{macro}
%    \begin{macro}{\HoLogoHtml@BibTeX}
%    \begin{macrocode}
\def\HoLogoHtml@BibTeX{\HoLogoHtml@BibTeX@sf}
%    \end{macrocode}
%    \end{macro}
%
% \subsubsection{\hologo{BibTeX8}}
%
%    \begin{macro}{\HoLogo@BibTeX8}
%    \begin{macrocode}
\expandafter\def\csname HoLogo@BibTeX8\endcsname#1{%
  \hologo{BibTeX}%
  8%
}
%    \end{macrocode}
%    \end{macro}
%
%    \begin{macro}{\HoLogoBkm@BibTeX8}
%    \begin{macrocode}
\expandafter\def\csname HoLogoBkm@BibTeX8\endcsname#1{%
  \hologo{BibTeX}%
  8%
}
%    \end{macrocode}
%    \end{macro}
%    \begin{macro}{\HoLogoHtml@BibTeX8}
%    \begin{macrocode}
\expandafter
\let\csname HoLogoHtml@BibTeX8\expandafter\endcsname
\csname HoLogo@BibTeX8\endcsname
%    \end{macrocode}
%    \end{macro}
%
% \subsubsection{\hologo{ConTeXt}}
%
%    \begin{macro}{\HoLogo@ConTeXt@simple}
%    \begin{macrocode}
\def\HoLogo@ConTeXt@simple#1{%
  \HOLOGO@mbox{Con}%
  \HOLOGO@discretionary
  \HOLOGO@mbox{\hologo{TeX}t}%
}
%    \end{macrocode}
%    \end{macro}
%    \begin{macro}{\HoLogoHtml@ConTeXt@simple}
%    \begin{macrocode}
\let\HoLogoHtml@ConTeXt@simple\HoLogo@ConTeXt@simple
%    \end{macrocode}
%    \end{macro}
%
%    \begin{macro}{\HoLogo@ConTeXt@narrow}
%    This definition of logo \hologo{ConTeXt} with variant \xoption{narrow}
%    comes from TUGboat's class \xclass{ltugboat} (version 2010/11/15 v2.8).
%    \begin{macrocode}
\def\HoLogo@ConTeXt@narrow#1{%
  \HOLOGO@mbox{C\kern-.0333emon}%
  \HOLOGO@discretionary
  \kern-.0667em%
  \HOLOGO@mbox{\hologo{TeX}\kern-.0333emt}%
}
%    \end{macrocode}
%    \end{macro}
%    \begin{macro}{\HoLogoHtml@ConTeXt@narrow}
%    \begin{macrocode}
\def\HoLogoHtml@ConTeXt@narrow#1{%
  \HoLogoCss@ConTeXt@narrow
  \HOLOGO@Span{ConTeXt-narrow}{%
    \HOLOGO@Span{C}{C}%
    on%
    \hologo{TeX}%
    t%
  }%
}
%    \end{macrocode}
%    \end{macro}
%    \begin{macro}{\HoLogoCss@ConTeXt@narrow}
%    \begin{macrocode}
\def\HoLogoCss@ConTeXt@narrow{%
  \Css{%
    span.HoLogo-ConTeXt-narrow span.HoLogo-C{%
      margin-left:-.0333em;%
    }%
  }%
  \Css{%
    span.HoLogo-ConTeXt-narrow span.HoLogo-TeX{%
      margin-left:-.0667em;%
      margin-right:-.0333em;%
    }%
  }%
  \global\let\HoLogoCss@ConTeXt@narrow\relax
}
%    \end{macrocode}
%    \end{macro}
%
%    \begin{macro}{\HoLogo@ConTeXt}
%    \begin{macrocode}
\def\HoLogo@ConTeXt{\HoLogo@ConTeXt@narrow}
%    \end{macrocode}
%    \end{macro}
%    \begin{macro}{\HoLogoHtml@ConTeXt}
%    \begin{macrocode}
\def\HoLogoHtml@ConTeXt{\HoLogoHtml@ConTeXt@narrow}
%    \end{macrocode}
%    \end{macro}
%
% \subsubsection{\hologo{emTeX}}
%
%    \begin{macro}{\HoLogo@emTeX}
%    \begin{macrocode}
\def\HoLogo@emTeX#1{%
  \HOLOGO@mbox{#1{e}{E}m}%
  \HOLOGO@discretionary
  \hologo{TeX}%
}
%    \end{macrocode}
%    \end{macro}
%    \begin{macro}{\HoLogoCs@emTeX}
%    \begin{macrocode}
\def\HoLogoCs@emTeX#1{#1{e}{E}mTeX}%
%    \end{macrocode}
%    \end{macro}
%    \begin{macro}{\HoLogoBkm@emTeX}
%    \begin{macrocode}
\def\HoLogoBkm@emTeX#1{%
  #1{e}{E}m\hologo{TeX}%
}
%    \end{macrocode}
%    \end{macro}
%    \begin{macro}{\HoLogoHtml@emTeX}
%    \begin{macrocode}
\let\HoLogoHtml@emTeX\HoLogo@emTeX
%    \end{macrocode}
%    \end{macro}
%
% \subsubsection{\hologo{ExTeX}}
%
%    \begin{macro}{\HoLogo@ExTeX}
%    The definition is taken from the FAQ of the
%    project \hologo{ExTeX}
%    \cite{ExTeX-FAQ}.
%\begin{quote}
%\begin{verbatim}
%\def\ExTeX{%
%  \textrm{% Logo always with serifs
%    \ensuremath{%
%      \textstyle
%      \varepsilon_{%
%        \kern-0.15em%
%        \mathcal{X}%
%      }%
%    }%
%    \kern-.15em%
%    \TeX
%  }%
%}
%\end{verbatim}
%\end{quote}
%    \begin{macrocode}
\def\HoLogo@ExTeX#1{%
  \HoLogoFont@font{ExTeX}{rm}{%
    \ltx@mbox{%
      \HOLOGO@MathSetup
      $%
        \textstyle
        \varepsilon_{%
          \kern-0.15em%
          \HoLogoFont@font{ExTeX}{sy}{X}%
        }%
      $%
    }%
    \HOLOGO@discretionary
    \kern-.15em%
    \hologo{TeX}%
  }%
}
%    \end{macrocode}
%    \end{macro}
%    \begin{macro}{\HoLogoHtml@ExTeX}
%    \begin{macrocode}
\def\HoLogoHtml@ExTeX#1{%
  \HoLogoCss@ExTeX
  \HoLogoFont@font{ExTeX}{rm}{%
    \HOLOGO@Span{ExTeX}{%
      \ltx@mbox{%
        \HOLOGO@MathSetup
        $\textstyle\varepsilon$%
        \HOLOGO@Span{X}{$\textstyle\chi$}%
        \hologo{TeX}%
      }%
    }%
  }%
}
%    \end{macrocode}
%    \end{macro}
%    \begin{macro}{\HoLogoBkm@ExTeX}
%    \begin{macrocode}
\def\HoLogoBkm@ExTeX#1{%
  \HOLOGO@PdfdocUnicode{#1{e}{E}x}{\textepsilon\textchi}%
  \hologo{TeX}%
}
%    \end{macrocode}
%    \end{macro}
%    \begin{macro}{\HoLogoCss@ExTeX}
%    \begin{macrocode}
\def\HoLogoCss@ExTeX{%
  \Css{%
    span.HoLogo-ExTeX{%
      font-family:serif;%
    }%
  }%
  \Css{%
    span.HoLogo-ExTeX span.HoLogo-TeX{%
      margin-left:-.15em;%
    }%
  }%
  \global\let\HoLogoCss@ExTeX\relax
}
%    \end{macrocode}
%    \end{macro}
%
% \subsubsection{\hologo{MiKTeX}}
%
%    \begin{macro}{\HoLogo@MiKTeX}
%    \begin{macrocode}
\def\HoLogo@MiKTeX#1{%
  \HOLOGO@mbox{MiK}%
  \HOLOGO@discretionary
  \hologo{TeX}%
}
%    \end{macrocode}
%    \end{macro}
%    \begin{macro}{\HoLogoHtml@MiKTeX}
%    \begin{macrocode}
\let\HoLogoHtml@MiKTeX\HoLogo@MiKTeX
%    \end{macrocode}
%    \end{macro}
%
% \subsubsection{\hologo{OzTeX} and friends}
%
%    Source: \hologo{OzTeX} FAQ \cite{OzTeX}:
%    \begin{quote}
%      |\def\OzTeX{O\kern-.03em z\kern-.15em\TeX}|\\
%      (There is no kerning in OzMF, OzMP and OzTtH.)
%    \end{quote}
%
%    \begin{macro}{\HoLogo@OzTeX}
%    \begin{macrocode}
\def\HoLogo@OzTeX#1{%
  O%
  \kern-.03em %
  z%
  \kern-.15em %
  \hologo{TeX}%
}
%    \end{macrocode}
%    \end{macro}
%    \begin{macro}{\HoLogoHtml@OzTeX}
%    \begin{macrocode}
\def\HoLogoHtml@OzTeX#1{%
  \HoLogoCss@OzTeX
  \HOLOGO@Span{OzTeX}{%
    O%
    \HOLOGO@Span{z}{z}%
    \hologo{TeX}%
  }%
}
%    \end{macrocode}
%    \end{macro}
%    \begin{macro}{\HoLogoCss@OzTeX}
%    \begin{macrocode}
\def\HoLogoCss@OzTeX{%
  \Css{%
    span.HoLogo-OzTeX span.HoLogo-z{%
      margin-left:-.03em;%
      margin-right:-.15em;%
    }%
  }%
  \global\let\HoLogoCss@OzTeX\relax
}
%    \end{macrocode}
%    \end{macro}
%
%    \begin{macro}{\HoLogo@OzMF}
%    \begin{macrocode}
\def\HoLogo@OzMF#1{%
  \HOLOGO@mbox{OzMF}%
}
%    \end{macrocode}
%    \end{macro}
%    \begin{macro}{\HoLogo@OzMP}
%    \begin{macrocode}
\def\HoLogo@OzMP#1{%
  \HOLOGO@mbox{OzMP}%
}
%    \end{macrocode}
%    \end{macro}
%    \begin{macro}{\HoLogo@OzTtH}
%    \begin{macrocode}
\def\HoLogo@OzTtH#1{%
  \HOLOGO@mbox{OzTtH}%
}
%    \end{macrocode}
%    \end{macro}
%
% \subsubsection{\hologo{PCTeX}}
%
%    \begin{macro}{\HoLogo@PCTeX}
%    \begin{macrocode}
\def\HoLogo@PCTeX#1{%
  \HOLOGO@mbox{PC}%
  \hologo{TeX}%
}
%    \end{macrocode}
%    \end{macro}
%    \begin{macro}{\HoLogoHtml@PCTeX}
%    \begin{macrocode}
\let\HoLogoHtml@PCTeX\HoLogo@PCTeX
%    \end{macrocode}
%    \end{macro}
%
% \subsubsection{\hologo{PiCTeX}}
%
%    The original definitions from \xfile{pictex.tex} \cite{PiCTeX}:
%\begin{quote}
%\begin{verbatim}
%\def\PiC{%
%  P%
%  \kern-.12em%
%  \lower.5ex\hbox{I}%
%  \kern-.075em%
%  C%
%}
%\def\PiCTeX{%
%  \PiC
%  \kern-.11em%
%  \TeX
%}
%\end{verbatim}
%\end{quote}
%
%    \begin{macro}{\HoLogo@PiC}
%    \begin{macrocode}
\def\HoLogo@PiC#1{%
  P%
  \kern-.12em%
  \lower.5ex\hbox{I}%
  \kern-.075em%
  C%
  \HOLOGO@SpaceFactor
}
%    \end{macrocode}
%    \end{macro}
%    \begin{macro}{\HoLogoHtml@PiC}
%    \begin{macrocode}
\def\HoLogoHtml@PiC#1{%
  \HoLogoCss@PiC
  \HOLOGO@Span{PiC}{%
    P%
    \HOLOGO@Span{i}{I}%
    C%
  }%
}
%    \end{macrocode}
%    \end{macro}
%    \begin{macro}{\HoLogoCss@PiC}
%    \begin{macrocode}
\def\HoLogoCss@PiC{%
  \Css{%
    span.HoLogo-PiC span.HoLogo-i{%
      position:relative;%
      top:.5ex;%
      margin-left:-.12em;%
      margin-right:-.075em;%
      text-decoration:none;%
    }%
  }%
  \global\let\HoLogoCss@PiC\relax
}
%    \end{macrocode}
%    \end{macro}
%
%    \begin{macro}{\HoLogo@PiCTeX}
%    \begin{macrocode}
\def\HoLogo@PiCTeX#1{%
  \hologo{PiC}%
  \HOLOGO@discretionary
  \kern-.11em%
  \hologo{TeX}%
}
%    \end{macrocode}
%    \end{macro}
%    \begin{macro}{\HoLogoHtml@PiCTeX}
%    \begin{macrocode}
\def\HoLogoHtml@PiCTeX#1{%
  \HoLogoCss@PiCTeX
  \HOLOGO@Span{PiCTeX}{%
    \hologo{PiC}%
    \hologo{TeX}%
  }%
}
%    \end{macrocode}
%    \end{macro}
%    \begin{macro}{\HoLogoCss@PiCTeX}
%    \begin{macrocode}
\def\HoLogoCss@PiCTeX{%
  \Css{%
    span.HoLogo-PiCTeX span.HoLogo-PiC{%
      margin-right:-.11em;%
    }%
  }%
  \global\let\HoLogoCss@PiCTeX\relax
}
%    \end{macrocode}
%    \end{macro}
%
% \subsubsection{\hologo{teTeX}}
%
%    \begin{macro}{\HoLogo@teTeX}
%    \begin{macrocode}
\def\HoLogo@teTeX#1{%
  \HOLOGO@mbox{#1{t}{T}e}%
  \HOLOGO@discretionary
  \hologo{TeX}%
}
%    \end{macrocode}
%    \end{macro}
%    \begin{macro}{\HoLogoCs@teTeX}
%    \begin{macrocode}
\def\HoLogoCs@teTeX#1{#1{t}{T}dfTeX}
%    \end{macrocode}
%    \end{macro}
%    \begin{macro}{\HoLogoBkm@teTeX}
%    \begin{macrocode}
\def\HoLogoBkm@teTeX#1{%
  #1{t}{T}e\hologo{TeX}%
}
%    \end{macrocode}
%    \end{macro}
%    \begin{macro}{\HoLogoHtml@teTeX}
%    \begin{macrocode}
\let\HoLogoHtml@teTeX\HoLogo@teTeX
%    \end{macrocode}
%    \end{macro}
%
% \subsubsection{\hologo{TeX4ht}}
%
%    \begin{macro}{\HoLogo@TeX4ht}
%    \begin{macrocode}
\expandafter\def\csname HoLogo@TeX4ht\endcsname#1{%
  \HOLOGO@mbox{\hologo{TeX}4ht}%
}
%    \end{macrocode}
%    \end{macro}
%    \begin{macro}{\HoLogoHtml@TeX4ht}
%    \begin{macrocode}
\expandafter
\let\csname HoLogoHtml@TeX4ht\expandafter\endcsname
\csname HoLogo@TeX4ht\endcsname
%    \end{macrocode}
%    \end{macro}
%
%
% \subsubsection{\hologo{SageTeX}}
%
%    \begin{macro}{\HoLogo@SageTeX}
%    \begin{macrocode}
\def\HoLogo@SageTeX#1{%
  \HOLOGO@mbox{Sage}%
  \HOLOGO@discretionary
  \HOLOGO@NegativeKerning{eT,oT,To}%
  \hologo{TeX}%
}
%    \end{macrocode}
%    \end{macro}
%    \begin{macro}{\HoLogoHtml@SageTeX}
%    \begin{macrocode}
\let\HoLogoHtml@SageTeX\HoLogo@SageTeX
%    \end{macrocode}
%    \end{macro}
%
% \subsection{\hologo{METAFONT} and friends}
%
%    \begin{macro}{\HoLogo@METAFONT}
%    \begin{macrocode}
\def\HoLogo@METAFONT#1{%
  \HoLogoFont@font{METAFONT}{logo}{%
    \HOLOGO@mbox{META}%
    \HOLOGO@discretionary
    \HOLOGO@mbox{FONT}%
  }%
}
%    \end{macrocode}
%    \end{macro}
%
%    \begin{macro}{\HoLogo@METAPOST}
%    \begin{macrocode}
\def\HoLogo@METAPOST#1{%
  \HoLogoFont@font{METAPOST}{logo}{%
    \HOLOGO@mbox{META}%
    \HOLOGO@discretionary
    \HOLOGO@mbox{POST}%
  }%
}
%    \end{macrocode}
%    \end{macro}
%
%    \begin{macro}{\HoLogo@MetaFun}
%    \begin{macrocode}
\def\HoLogo@MetaFun#1{%
  \HOLOGO@mbox{Meta}%
  \HOLOGO@discretionary
  \HOLOGO@mbox{Fun}%
}
%    \end{macrocode}
%    \end{macro}
%
%    \begin{macro}{\HoLogo@MetaPost}
%    \begin{macrocode}
\def\HoLogo@MetaPost#1{%
  \HOLOGO@mbox{Meta}%
  \HOLOGO@discretionary
  \HOLOGO@mbox{Post}%
}
%    \end{macrocode}
%    \end{macro}
%
% \subsection{Others}
%
% \subsubsection{\hologo{biber}}
%
%    \begin{macro}{\HoLogo@biber}
%    \begin{macrocode}
\def\HoLogo@biber#1{%
  \HOLOGO@mbox{#1{b}{B}i}%
  \HOLOGO@discretionary
  \HOLOGO@mbox{ber}%
}
%    \end{macrocode}
%    \end{macro}
%    \begin{macro}{\HoLogoCs@biber}
%    \begin{macrocode}
\def\HoLogoCs@biber#1{#1{b}{B}iber}
%    \end{macrocode}
%    \end{macro}
%    \begin{macro}{\HoLogoBkm@biber}
%    \begin{macrocode}
\def\HoLogoBkm@biber#1{%
  #1{b}{B}iber%
}
%    \end{macrocode}
%    \end{macro}
%    \begin{macro}{\HoLogoHtml@biber}
%    \begin{macrocode}
\let\HoLogoHtml@biber\HoLogo@biber
%    \end{macrocode}
%    \end{macro}
%
% \subsubsection{\hologo{KOMAScript}}
%
%    \begin{macro}{\HoLogo@KOMAScript}
%    The definition for \hologo{KOMAScript} is taken
%    from \hologo{KOMAScript} (\xfile{scrlogo.dtx}, reformatted) \cite{scrlogo}:
%\begin{quote}
%\begin{verbatim}
%\@ifundefined{KOMAScript}{%
%  \DeclareRobustCommand{\KOMAScript}{%
%    \textsf{%
%      K\kern.05em O\kern.05emM\kern.05em A%
%      \kern.1em-\kern.1em %
%      Script%
%    }%
%  }%
%}{}
%\end{verbatim}
%\end{quote}
%    \begin{macrocode}
\def\HoLogo@KOMAScript#1{%
  \HoLogoFont@font{KOMAScript}{sf}{%
    \HOLOGO@mbox{%
      K\kern.05em%
      O\kern.05em%
      M\kern.05em%
      A%
    }%
    \kern.1em%
    \HOLOGO@hyphen
    \kern.1em%
    \HOLOGO@mbox{Script}%
  }%
}
%    \end{macrocode}
%    \end{macro}
%    \begin{macro}{\HoLogoBkm@KOMAScript}
%    \begin{macrocode}
\def\HoLogoBkm@KOMAScript#1{%
  KOMA-Script%
}
%    \end{macrocode}
%    \end{macro}
%    \begin{macro}{\HoLogoHtml@KOMAScript}
%    \begin{macrocode}
\def\HoLogoHtml@KOMAScript#1{%
  \HoLogoCss@KOMAScript
  \HoLogoFont@font{KOMAScript}{sf}{%
    \HOLOGO@Span{KOMAScript}{%
      K%
      \HOLOGO@Span{O}{O}%
      M%
      \HOLOGO@Span{A}{A}%
      \HOLOGO@Span{hyphen}{-}%
      Script%
    }%
  }%
}
%    \end{macrocode}
%    \end{macro}
%    \begin{macro}{\HoLogoCss@KOMAScript}
%    \begin{macrocode}
\def\HoLogoCss@KOMAScript{%
  \Css{%
    span.HoLogo-KOMAScript{%
      font-family:sans-serif;%
    }%
  }%
  \Css{%
    span.HoLogo-KOMAScript span.HoLogo-O{%
      padding-left:.05em;%
      padding-right:.05em;%
    }%
  }%
  \Css{%
    span.HoLogo-KOMAScript span.HoLogo-A{%
      padding-left:.05em;%
    }%
  }%
  \Css{%
    span.HoLogo-KOMAScript span.HoLogo-hyphen{%
      padding-left:.1em;%
      padding-right:.1em;%
    }%
  }%
  \global\let\HoLogoCss@KOMAScript\relax
}
%    \end{macrocode}
%    \end{macro}
%
% \subsubsection{\hologo{LyX}}
%
%    \begin{macro}{\HoLogo@LyX}
%    The definition is taken from the documentation source files
%    of \hologo{LyX}, \xfile{Intro.lyx} \cite{LyX}:
%\begin{quote}
%\begin{verbatim}
%\def\LyX{%
%  \texorpdfstring{%
%    L\kern-.1667em\lower.25em\hbox{Y}\kern-.125emX\@%
%  }{%
%    LyX%
%  }%
%}
%\end{verbatim}
%\end{quote}
%    \begin{macrocode}
\def\HoLogo@LyX#1{%
  L%
  \kern-.1667em%
  \lower.25em\hbox{Y}%
  \kern-.125em%
  X%
  \HOLOGO@SpaceFactor
}
%    \end{macrocode}
%    \end{macro}
%    \begin{macro}{\HoLogoHtml@LyX}
%    \begin{macrocode}
\def\HoLogoHtml@LyX#1{%
  \HoLogoCss@LyX
  \HOLOGO@Span{LyX}{%
    L%
    \HOLOGO@Span{y}{Y}%
    X%
  }%
}
%    \end{macrocode}
%    \end{macro}
%    \begin{macro}{\HoLogoCss@LyX}
%    \begin{macrocode}
\def\HoLogoCss@LyX{%
  \Css{%
    span.HoLogo-LyX span.HoLogo-y{%
      position:relative;%
      top:.25em;%
      margin-left:-.1667em;%
      margin-right:-.125em;%
      text-decoration:none;%
    }%
  }%
  \global\let\HoLogoCss@LyX\relax
}
%    \end{macrocode}
%    \end{macro}
%
% \subsubsection{\hologo{NTS}}
%
%    \begin{macro}{\HoLogo@NTS}
%    Definition for \hologo{NTS} can be found in
%    package \xpackage{etex\textunderscore man} for the \hologo{eTeX} manual \cite{etexman}
%    and in package \xpackage{dtklogos} \cite{dtklogos}:
%\begin{quote}
%\begin{verbatim}
%\def\NTS{%
%  \leavevmode
%  \hbox{%
%    $%
%      \cal N%
%      \kern-0.35em%
%      \lower0.5ex\hbox{$\cal T$}%
%      \kern-0.2em%
%      S%
%    $%
%  }%
%}
%\end{verbatim}
%\end{quote}
%    \begin{macrocode}
\def\HoLogo@NTS#1{%
  \HoLogoFont@font{NTS}{sy}{%
    N\/%
    \kern-.35em%
    \lower.5ex\hbox{T\/}%
    \kern-.2em%
    S\/%
  }%
  \HOLOGO@SpaceFactor
}
%    \end{macrocode}
%    \end{macro}
%
% \subsubsection{\Hologo{TTH} (\hologo{TeX} to HTML translator)}
%
%    Source: \url{http://hutchinson.belmont.ma.us/tth/}
%    In the HTML source the second `T' is printed as subscript.
%\begin{quote}
%\begin{verbatim}
%T<sub>T</sub>H
%\end{verbatim}
%\end{quote}
%    \begin{macro}{\HoLogo@TTH}
%    \begin{macrocode}
\def\HoLogo@TTH#1{%
  \ltx@mbox{%
    T\HOLOGO@SubScript{T}H%
  }%
  \HOLOGO@SpaceFactor
}
%    \end{macrocode}
%    \end{macro}
%
%    \begin{macro}{\HoLogoHtml@TTH}
%    \begin{macrocode}
\def\HoLogoHtml@TTH#1{%
  T\HCode{<sub>}T\HCode{</sub>}H%
}
%    \end{macrocode}
%    \end{macro}
%
% \subsubsection{\Hologo{HanTheThanh}}
%
%    Partial source: Package \xpackage{dtklogos}.
%    The double accent is U+1EBF (latin small letter e with circumflex
%    and acute).
%    \begin{macro}{\HoLogo@HanTheThanh}
%    \begin{macrocode}
\def\HoLogo@HanTheThanh#1{%
  \ltx@mbox{H\`an}%
  \HOLOGO@space
  \ltx@mbox{%
    Th%
    \HOLOGO@IfCharExists{"1EBF}{%
      \char"1EBF\relax
    }{%
      \^e\hbox to 0pt{\hss\raise .5ex\hbox{\'{}}}%
    }%
  }%
  \HOLOGO@space
  \ltx@mbox{Th\`anh}%
}
%    \end{macrocode}
%    \end{macro}
%    \begin{macro}{\HoLogoBkm@HanTheThanh}
%    \begin{macrocode}
\def\HoLogoBkm@HanTheThanh#1{%
  H\`an %
  Th\HOLOGO@PdfdocUnicode{\^e}{\9036\277} %
  Th\`anh%
}
%    \end{macrocode}
%    \end{macro}
%    \begin{macro}{\HoLogoHtml@HanTheThanh}
%    \begin{macrocode}
\def\HoLogoHtml@HanTheThanh#1{%
  H\`an %
  Th\HCode{&\ltx@hashchar x1ebf;} %
  Th\`anh%
}
%    \end{macrocode}
%    \end{macro}
%
% \subsection{Driver detection}
%
%    \begin{macrocode}
\HOLOGO@IfExists\InputIfFileExists{%
  \InputIfFileExists{hologo.cfg}{}{}%
}{%
  \ltx@IfUndefined{pdf@filesize}{%
    \def\HOLOGO@InputIfExists{%
      \openin\HOLOGO@temp=hologo.cfg\relax
      \ifeof\HOLOGO@temp
        \closein\HOLOGO@temp
      \else
        \closein\HOLOGO@temp
        \begingroup
          \def\x{LaTeX2e}%
        \expandafter\endgroup
        \ifx\fmtname\x
          % \iffalse meta-comment
%
% File: hologo.dtx
% Version: 2016/05/12 v1.11
% Info: A logo collection with bookmark support
%
% Copyright (C) 2010-2012 by
%    Heiko Oberdiek <heiko.oberdiek at googlemail.com>
%
% This work may be distributed and/or modified under the
% conditions of the LaTeX Project Public License, either
% version 1.3c of this license or (at your option) any later
% version. This version of this license is in
%    http://www.latex-project.org/lppl/lppl-1-3c.txt
% and the latest version of this license is in
%    http://www.latex-project.org/lppl.txt
% and version 1.3 or later is part of all distributions of
% LaTeX version 2005/12/01 or later.
%
% This work has the LPPL maintenance status "maintained".
%
% This Current Maintainer of this work is Heiko Oberdiek.
%
% The Base Interpreter refers to any `TeX-Format',
% because some files are installed in TDS:tex/generic//.
%
% This work consists of the main source file hologo.dtx
% and the derived files
%    hologo.sty, hologo.pdf, hologo.ins, hologo.drv, hologo-example.tex,
%    hologo-test1.tex, hologo-test-spacefactor.tex,
%    hologo-test-list.tex.
%
% Distribution:
%    CTAN:macros/latex/contrib/oberdiek/hologo.dtx
%    CTAN:macros/latex/contrib/oberdiek/hologo.pdf
%
% Unpacking:
%    (a) If hologo.ins is present:
%           tex hologo.ins
%    (b) Without hologo.ins:
%           tex hologo.dtx
%    (c) If you insist on using LaTeX
%           latex \let\install=y\input{hologo.dtx}
%        (quote the arguments according to the demands of your shell)
%
% Documentation:
%    (a) If hologo.drv is present:
%           latex hologo.drv
%    (b) Without hologo.drv:
%           latex hologo.dtx; ...
%    The class ltxdoc loads the configuration file ltxdoc.cfg
%    if available. Here you can specify further options, e.g.
%    use A4 as paper format:
%       \PassOptionsToClass{a4paper}{article}
%
%    Programm calls to get the documentation (example):
%       pdflatex hologo.dtx
%       makeindex -s gind.ist hologo.idx
%       pdflatex hologo.dtx
%       makeindex -s gind.ist hologo.idx
%       pdflatex hologo.dtx
%
% Installation:
%    TDS:tex/generic/oberdiek/hologo.sty
%    TDS:doc/latex/oberdiek/hologo.pdf
%    TDS:doc/latex/oberdiek/example/hologo-example.tex
%    TDS:doc/latex/oberdiek/test/hologo-test1.tex
%    TDS:doc/latex/oberdiek/test/hologo-test-spacefactor.tex
%    TDS:doc/latex/oberdiek/test/hologo-test-list.tex
%    TDS:source/latex/oberdiek/hologo.dtx
%
%<*ignore>
\begingroup
  \catcode123=1 %
  \catcode125=2 %
  \def\x{LaTeX2e}%
\expandafter\endgroup
\ifcase 0\ifx\install y1\fi\expandafter
         \ifx\csname processbatchFile\endcsname\relax\else1\fi
         \ifx\fmtname\x\else 1\fi\relax
\else\csname fi\endcsname
%</ignore>
%<*install>
\input docstrip.tex
\Msg{************************************************************************}
\Msg{* Installation}
\Msg{* Package: hologo 2016/05/12 v1.11 A logo collection with bookmark support (HO)}
\Msg{************************************************************************}

\keepsilent
\askforoverwritefalse

\let\MetaPrefix\relax
\preamble

This is a generated file.

Project: hologo
Version: 2016/05/12 v1.11

Copyright (C) 2010-2012 by
   Heiko Oberdiek <heiko.oberdiek at googlemail.com>

This work may be distributed and/or modified under the
conditions of the LaTeX Project Public License, either
version 1.3c of this license or (at your option) any later
version. This version of this license is in
   http://www.latex-project.org/lppl/lppl-1-3c.txt
and the latest version of this license is in
   http://www.latex-project.org/lppl.txt
and version 1.3 or later is part of all distributions of
LaTeX version 2005/12/01 or later.

This work has the LPPL maintenance status "maintained".

This Current Maintainer of this work is Heiko Oberdiek.

The Base Interpreter refers to any `TeX-Format',
because some files are installed in TDS:tex/generic//.

This work consists of the main source file hologo.dtx
and the derived files
   hologo.sty, hologo.pdf, hologo.ins, hologo.drv, hologo-example.tex,
   hologo-test1.tex, hologo-test-spacefactor.tex,
   hologo-test-list.tex.

\endpreamble
\let\MetaPrefix\DoubleperCent

\generate{%
  \file{hologo.ins}{\from{hologo.dtx}{install}}%
  \file{hologo.drv}{\from{hologo.dtx}{driver}}%
  \usedir{tex/generic/oberdiek}%
  \file{hologo.sty}{\from{hologo.dtx}{package}}%
  \usedir{doc/latex/oberdiek/example}%
  \file{hologo-example.tex}{\from{hologo.dtx}{example}}%
  \usedir{doc/latex/oberdiek/test}%
  \file{hologo-test1.tex}{\from{hologo.dtx}{test1}}%
  \file{hologo-test-spacefactor.tex}{\from{hologo.dtx}{test-spacefactor}}%
  \file{hologo-test-list.tex}{\from{hologo.dtx}{test-list}}%
  \nopreamble
  \nopostamble
  \usedir{source/latex/oberdiek/catalogue}%
  \file{hologo.xml}{\from{hologo.dtx}{catalogue}}%
}

\catcode32=13\relax% active space
\let =\space%
\Msg{************************************************************************}
\Msg{*}
\Msg{* To finish the installation you have to move the following}
\Msg{* file into a directory searched by TeX:}
\Msg{*}
\Msg{*     hologo.sty}
\Msg{*}
\Msg{* To produce the documentation run the file `hologo.drv'}
\Msg{* through LaTeX.}
\Msg{*}
\Msg{* Happy TeXing!}
\Msg{*}
\Msg{************************************************************************}

\endbatchfile
%</install>
%<*ignore>
\fi
%</ignore>
%<*driver>
\NeedsTeXFormat{LaTeX2e}
\ProvidesFile{hologo.drv}%
  [2016/05/12 v1.11 A logo collection with bookmark support (HO)]%
\documentclass{ltxdoc}
\usepackage{holtxdoc}[2011/11/22]
\usepackage{hologo}[2016/05/12]
\usepackage{longtable}
\usepackage{array}
\usepackage{paralist}
%\usepackage[T1]{fontenc}
%\usepackage{lmodern}
\begin{document}
  \DocInput{hologo.dtx}%
\end{document}
%</driver>
% \fi
%
%
% \CharacterTable
%  {Upper-case    \A\B\C\D\E\F\G\H\I\J\K\L\M\N\O\P\Q\R\S\T\U\V\W\X\Y\Z
%   Lower-case    \a\b\c\d\e\f\g\h\i\j\k\l\m\n\o\p\q\r\s\t\u\v\w\x\y\z
%   Digits        \0\1\2\3\4\5\6\7\8\9
%   Exclamation   \!     Double quote  \"     Hash (number) \#
%   Dollar        \$     Percent       \%     Ampersand     \&
%   Acute accent  \'     Left paren    \(     Right paren   \)
%   Asterisk      \*     Plus          \+     Comma         \,
%   Minus         \-     Point         \.     Solidus       \/
%   Colon         \:     Semicolon     \;     Less than     \<
%   Equals        \=     Greater than  \>     Question mark \?
%   Commercial at \@     Left bracket  \[     Backslash     \\
%   Right bracket \]     Circumflex    \^     Underscore    \_
%   Grave accent  \`     Left brace    \{     Vertical bar  \|
%   Right brace   \}     Tilde         \~}
%
% \GetFileInfo{hologo.drv}
%
% \title{The \xpackage{hologo} package}
% \date{2016/05/12 v1.11}
% \author{Heiko Oberdiek\\\xemail{heiko.oberdiek at googlemail.com}}
%
% \maketitle
%
% \begin{abstract}
% This package starts a collection of logos with support for bookmarks
% strings.
% \end{abstract}
%
% \tableofcontents
%
% \section{Documentation}
%
% \subsection{Logo macros}
%
% \begin{declcs}{hologo} \M{name}
% \end{declcs}
% Macro \cs{hologo} sets the logo with name \meta{name}.
% The following table shows the supported names.
%
% \begingroup
%   \def\hologoEntry#1#2#3{^^A
%     #1&#2&\hologoLogoSetup{#1}{variant=#2}\hologo{#1}&#3\tabularnewline
%   }
%   \begin{longtable}{>{\ttfamily}l>{\ttfamily}lll}
%     \rmfamily\bfseries{name} & \rmfamily\bfseries variant
%     & \bfseries logo & \bfseries since\\
%     \hline
%     \endhead
%     \hologoList
%   \end{longtable}
% \endgroup
%
% \begin{declcs}{Hologo} \M{name}
% \end{declcs}
% Macro \cs{Hologo} starts the logo \meta{name} with an uppercase
% letter. As an exception small greek letters are not converted
% to uppercase. Examples, see \hologo{eTeX} and \hologo{ExTeX}.
%
% \subsection{Setup macros}
%
% The package does not support package options, but the following
% setup macros can be used to set options.
%
% \begin{declcs}{hologoSetup} \M{key value list}
% \end{declcs}
% Macro \cs{hologoSetup} sets global options.
%
% \begin{declcs}{hologoLogoSetup} \M{logo} \M{key value list}
% \end{declcs}
% Some options can also be used to configure a logo.
% These settings take precedence over global option settings.
%
% \subsection{Options}\label{sec:options}
%
% There are boolean and string options:
% \begin{description}
% \item[Boolean option:]
% It takes |true| or |false|
% as value. If the value is omitted, then |true| is used.
% \item[String option:]
% A value must be given as string. (But the string might be empty.)
% \end{description}
% The following options can be used both in \cs{hologoSetup}
% and \cs{hologoLogoSetup}:
% \begin{description}
% \def\entry#1{\item[\xoption{#1}:]}
% \entry{break}
%   enables or disables line breaks inside the logo. This setting is
%   refined by options \xoption{hyphenbreak}, \xoption{spacebreak}
%   or \xoption{discretionarybreak}.
%   Default is |false|.
% \entry{hyphenbreak}
%   enables or disables the line break right after the hyphen character.
% \entry{spacebreak}
%   enables or disables line breaks at space characters.
% \entry{discretionarybreak}
%   enables or disables line breaks at hyphenation points
%   (inserted by \cs{-}).
% \end{description}
% Macro \cs{hologoLogoSetup} also knows:
% \begin{description}
% \item[\xoption{variant}:]
%   This is a string option. It specifies a variant of a logo that
%   must exist. An empty string selects the package default variant.
% \end{description}
% Example:
% \begin{quote}
%   |\hologoSetup{break=false}|\\
%   |\hologoLogoSetup{plainTeX}{variant=hyphen,hyphenbreak}|\\
%   Then ``plain-\TeX'' contains one break point after the hyphen.
% \end{quote}
%
% \subsection{Driver options}
%
% Sometimes graphical operations are needed to construct some
% glyphs (e.g.\ \hologo{XeTeX}). If package \xpackage{graphics}
% or package \xpackage{pgf} are found, then the macros are taken
% from there. Otherwise the packge defines its own operations
% and therefore needs the driver information. Many drivers are
% detected automatically (\hologo{pdfTeX}/\hologo{LuaTeX}
% in PDF mode, \hologo{XeTeX}, \hologo{VTeX}). These have precedence
% over a driver option. The driver can be given as package option
% or using \cs{hologoDriverSetup}.
% The following list contains the recognized driver options:
% \begin{itemize}
% \item \xoption{pdftex}, \xoption{luatex}
% \item \xoption{dvipdfm}, \xoption{dvipdfmx}
% \item \xoption{dvips}, \xoption{dvipsone}, \xoption{xdvi}
% \item \xoption{xetex}
% \item \xoption{vtex}
% \end{itemize}
% The left driver of a line is the driver name that is used internally.
% The following names are aliases for drivers that use the
% same method. Therefore the entry in the \xext{log} file for
% the used driver prints the internally used driver name.
% \begin{description}
% \item[\xoption{driverfallback}:]
%   This option expects a driver that is used,
%   if the driver could not be detected automatically.
% \end{description}
%
% \begin{declcs}{hologoDriverSetup} \M{driver option}
% \end{declcs}
% The driver can also be configured after package loading
% using \cs{hologoDriverSetup}, also the way for \hologo{plainTeX}
% to setup the driver.
%
% \subsection{Font setup}
%
% Some logos require a special font, but should also be usable by
% \hologo{plainTeX}. Therefore the package provides some ways
% to influence the font settings. The options below
% take font settings as values. Both font commands
% such as \cs{sffamily} and macros that take one argument
% like \cs{textsf} can be used.
%
% \begin{declcs}{hologoFontSetup} \M{key value list}
% \end{declcs}
% Macro \cs{hologoFontSetup} sets the fonts for all logos.
% Supported keys:
% \begin{description}
% \def\entry#1{\item[\xoption{#1}:]}
% \entry{general}
%   This font is used for all logos. The default is empty.
%   That means no special font is used.
% \entry{bibsf}
%   This font is used for
%   {\hologoLogoSetup{BibTeX}{variant=sf}\hologo{BibTeX}}
%   with variant \xoption{sf}.
% \entry{rm}
%   This font is a serif font. It is used for \hologo{ExTeX}.
% \entry{sc}
%   This font specifies a small caps font. It is used for
%   {\hologoLogoSetup{BibTeX}{variant=sc}\hologo{BibTeX}}
%   with variant \xoption{sc}.
% \entry{sf}
%   This font specifies a sans serif font. The default
%   is \cs{sffamily}, then \cs{sf} is tried. Otherwise
%   a warning is given. It is used by \hologo{KOMAScript}.
% \entry{sy}
%   This is the font for math symbols (e.g. cmsy).
%   It is used by \hologo{AmS}, \hologo{NTS}, \hologo{ExTeX}.
% \entry{logo}
%   \hologo{METAFONT} and \hologo{METAPOST} are using that font.
%   In \hologo{LaTeX} \cs{logofamily} is used and
%   the definitions of package \xpackage{mflogo} are used
%   if the package is not loaded.
%   Otherwise the \cs{tenlogo} is used and defined
%   if it does not already exists.
% \end{description}
%
% \begin{declcs}{hologoLogoFontSetup} \M{logo} \M{key value list}
% \end{declcs}
% Fonts can also be set for a logo or logo component separately,
% see the following list.
% The keys are the same as for \cs{hologoFontSetup}.
%
% \begin{longtable}{>{\ttfamily}l>{\sffamily}ll}
%   \meta{logo} & keys & result\\
%   \hline
%   \endhead
%   BibTeX & bibsf & {\hologoLogoSetup{BibTeX}{variant=sf}\hologo{BibTeX}}\\[.5ex]
%   BibTeX & sc & {\hologoLogoSetup{BibTeX}{variant=sc}\hologo{BibTeX}}\\[.5ex]
%   ExTeX & rm & \hologo{ExTeX}\\
%   SliTeX & rm & \hologo{SliTeX}\\[.5ex]
%   AmS & sy & \hologo{AmS}\\
%   ExTeX & sy & \hologo{ExTeX}\\
%   NTS & sy & \hologo{NTS}\\[.5ex]
%   KOMAScript & sf & \hologo{KOMAScript}\\[.5ex]
%   METAFONT & logo & \hologo{METAFONT}\\
%   METAPOST & logo & \hologo{METAPOST}\\[.5ex]
%   SliTeX & sc \hologo{SliTeX}
% \end{longtable}
%
% \subsubsection{Font order}
%
% For all logos the font \xoption{general} is applied first.
% Example:
%\begin{quote}
%|\hologoFontSetup{general=\color{red}}|
%\end{quote}
% will print red logos.
% Then if the font uses a special font \xoption{sf}, for example,
% the font is applied that is setup by \cs{hologoLogoFontSetup}.
% If this font is not setup, then the common font setup
% by \cs{hologoFontSetup} is used. Otherwise a warning is given,
% that there is no font configured.
%
% \subsection{Additional user macros}
%
% Usually a variant of a logo is configured by using
% \cs{hologoLogoSetup}, because it is bad style to mix
% different variants of the same logo in the same text.
% There the following macros are a convenience for testing.
%
% \begin{declcs}{hologoVariant} \M{name} \M{variant}\\
%   \cs{HologoVariant} \M{name} \M{variant}
% \end{declcs}
% Logo \meta{name} is set using \meta{variant} that specifies
% explicitely which variant of the macro is used. If the argument
% is empty, then the default form of the logo is used
% (configurable by \cs{hologoLogoSetup}).
%
% \cs{HologoVariant} is used if the logo is set in a context
% that needs an uppercase first letter (beginning of a sentence, \dots).
%
% \begin{declcs}{hologoList}\\
%   \cs{hologoEntry} \M{logo} \M{variant} \M{since}
% \end{declcs}
% Macro \cs{hologoList} contains all logos that are provided
% by the package including variants. The list consists of calls
% of \cs{hologoEntry} with three arguments starting with the
% logo name \meta{logo} and its variant \meta{variant}. An empty
% variant means the current default. Argument \meta{since} specifies
% with version of the package \xpackage{hologo} is needed to get
% the logo. If the logo is fixed, then the date gets updated.
% Therefore the date \meta{since} is not exactly the date of
% the first introduction, but rather the date of the latest fix.
%
% Before \cs{hologoList} can be used, macro \cs{hologoEntry} needs
% a definition. The example file in section \ref{sec:example}
% shows applications of \cs{hologoList}.
%
% \subsection{Supported contexts}
%
% Macros \cs{hologo} and friends support special contexts:
% \begin{itemize}
% \item \hologo{LaTeX}'s protection mechanism.
% \item Bookmarks of package \xpackage{hyperref}.
% \item Package \xpackage{tex4ht}.
% \item The macros can be used inside \cs{csname} constructs,
%   if \cs{ifincsname} is available (\hologo{pdfTeX}, \hologo{XeTeX},
%   \hologo{LuaTeX}).
% \end{itemize}
%
% \subsection{Example}
% \label{sec:example}
%
% The following example prints the logos in different fonts.
%    \begin{macrocode}
%<*example>
%<<verbatim
\NeedsTeXFormat{LaTeX2e}
\documentclass[a4paper]{article}
\usepackage[
  hmargin=20mm,
  vmargin=20mm,
]{geometry}
\pagestyle{empty}
\usepackage{hologo}[2016/05/12]
\usepackage{longtable}
\usepackage{array}
\setlength{\extrarowheight}{2pt}
\usepackage[T1]{fontenc}
\usepackage{lmodern}
\usepackage{pdflscape}
\usepackage[
  pdfencoding=auto,
]{hyperref}
\hypersetup{
  pdfauthor={Heiko Oberdiek},
  pdftitle={Example for package `hologo'},
  pdfsubject={Logos with fonts lmr, lmss, qtm, qpl, qhv},
}
\usepackage{bookmark}

% Print the logo list on the console

\begingroup
  \typeout{}%
  \typeout{*** Begin of logo list ***}%
  \newcommand*{\hologoEntry}[3]{%
    \typeout{#1 \ifx\\#2\\\else(#2) \fi[#3]}%
  }%
  \hologoList
  \typeout{*** End of logo list ***}%
  \typeout{}%
\endgroup

\begin{document}
\begin{landscape}

  \section{Example file for package `hologo'}

  % Table for font names

  \begin{longtable}{>{\bfseries}ll}
    \textbf{font} & \textbf{Font name}\\
    \hline
    lmr & Latin Modern Roman\\
    lmss & Latin Modern Sans\\
    qtm & \TeX\ Gyre Termes\\
    qhv & \TeX\ Gyre Heros\\
    qpl & \TeX\ Gyre Pagella\\
  \end{longtable}

  % Logo list with logos in different fonts

  \begingroup
    \newcommand*{\SetVariant}[2]{%
      \ifx\\#2\\%
      \else
        \hologoLogoSetup{#1}{variant=#2}%
      \fi
    }%
    \newcommand*{\hologoEntry}[3]{%
      \SetVariant{#1}{#2}%
      \raisebox{1em}[0pt][0pt]{\hypertarget{#1@#2}{}}%
      \bookmark[%
        dest={#1@#2},%
      ]{%
        #1\ifx\\#2\\\else\space(#2)\fi: \Hologo{#1}, \hologo{#1} %
        [Unicode]%
      }%
      \hypersetup{unicode=false}%
      \bookmark[%
        dest={#1@#2},%
      ]{%
        #1\ifx\\#2\\\else\space(#2)\fi: \Hologo{#1}, \hologo{#1} %
        [PDFDocEncoding]%
      }%
      \texttt{#1}%
      &%
      \texttt{#2}%
      &%
      \Hologo{#1}%
      &%
      \SetVariant{#1}{#2}%
      \hologo{#1}%
      &%
      \SetVariant{#1}{#2}%
      \fontfamily{qtm}\selectfont
      \hologo{#1}%
      &%
      \SetVariant{#1}{#2}%
      \fontfamily{qpl}\selectfont
      \hologo{#1}%
      &%
      \SetVariant{#1}{#2}%
      \textsf{\hologo{#1}}%
      &%
      \SetVariant{#1}{#2}%
      \fontfamily{qhv}\selectfont
      \hologo{#1}%
      \tabularnewline
    }%
    \begin{longtable}{llllllll}%
      \textbf{\textit{logo}} & \textbf{\textit{variant}} &
      \texttt{\string\Hologo} &
      \textbf{lmr} & \textbf{qtm} & \textbf{qpl} &
      \textbf{lmss} & \textbf{qhv}
      \tabularnewline
      \hline
      \endhead
      \hologoList
    \end{longtable}%
  \endgroup

\end{landscape}
\end{document}
%verbatim
%</example>
%    \end{macrocode}
%
% \StopEventually{
% }
%
% \section{Implementation}
%    \begin{macrocode}
%<*package>
%    \end{macrocode}
%    Reload check, especially if the package is not used with \LaTeX.
%    \begin{macrocode}
\begingroup\catcode61\catcode48\catcode32=10\relax%
  \catcode13=5 % ^^M
  \endlinechar=13 %
  \catcode35=6 % #
  \catcode39=12 % '
  \catcode44=12 % ,
  \catcode45=12 % -
  \catcode46=12 % .
  \catcode58=12 % :
  \catcode64=11 % @
  \catcode123=1 % {
  \catcode125=2 % }
  \expandafter\let\expandafter\x\csname ver@hologo.sty\endcsname
  \ifx\x\relax % plain-TeX, first loading
  \else
    \def\empty{}%
    \ifx\x\empty % LaTeX, first loading,
      % variable is initialized, but \ProvidesPackage not yet seen
    \else
      \expandafter\ifx\csname PackageInfo\endcsname\relax
        \def\x#1#2{%
          \immediate\write-1{Package #1 Info: #2.}%
        }%
      \else
        \def\x#1#2{\PackageInfo{#1}{#2, stopped}}%
      \fi
      \x{hologo}{The package is already loaded}%
      \aftergroup\endinput
    \fi
  \fi
\endgroup%
%    \end{macrocode}
%    Package identification:
%    \begin{macrocode}
\begingroup\catcode61\catcode48\catcode32=10\relax%
  \catcode13=5 % ^^M
  \endlinechar=13 %
  \catcode35=6 % #
  \catcode39=12 % '
  \catcode40=12 % (
  \catcode41=12 % )
  \catcode44=12 % ,
  \catcode45=12 % -
  \catcode46=12 % .
  \catcode47=12 % /
  \catcode58=12 % :
  \catcode64=11 % @
  \catcode91=12 % [
  \catcode93=12 % ]
  \catcode123=1 % {
  \catcode125=2 % }
  \expandafter\ifx\csname ProvidesPackage\endcsname\relax
    \def\x#1#2#3[#4]{\endgroup
      \immediate\write-1{Package: #3 #4}%
      \xdef#1{#4}%
    }%
  \else
    \def\x#1#2[#3]{\endgroup
      #2[{#3}]%
      \ifx#1\@undefined
        \xdef#1{#3}%
      \fi
      \ifx#1\relax
        \xdef#1{#3}%
      \fi
    }%
  \fi
\expandafter\x\csname ver@hologo.sty\endcsname
\ProvidesPackage{hologo}%
  [2016/05/12 v1.11 A logo collection with bookmark support (HO)]%
%    \end{macrocode}
%
%    \begin{macrocode}
\begingroup\catcode61\catcode48\catcode32=10\relax%
  \catcode13=5 % ^^M
  \endlinechar=13 %
  \catcode123=1 % {
  \catcode125=2 % }
  \catcode64=11 % @
  \def\x{\endgroup
    \expandafter\edef\csname HOLOGO@AtEnd\endcsname{%
      \endlinechar=\the\endlinechar\relax
      \catcode13=\the\catcode13\relax
      \catcode32=\the\catcode32\relax
      \catcode35=\the\catcode35\relax
      \catcode61=\the\catcode61\relax
      \catcode64=\the\catcode64\relax
      \catcode123=\the\catcode123\relax
      \catcode125=\the\catcode125\relax
    }%
  }%
\x\catcode61\catcode48\catcode32=10\relax%
\catcode13=5 % ^^M
\endlinechar=13 %
\catcode35=6 % #
\catcode64=11 % @
\catcode123=1 % {
\catcode125=2 % }
\def\TMP@EnsureCode#1#2{%
  \edef\HOLOGO@AtEnd{%
    \HOLOGO@AtEnd
    \catcode#1=\the\catcode#1\relax
  }%
  \catcode#1=#2\relax
}
\TMP@EnsureCode{10}{12}% ^^J
\TMP@EnsureCode{33}{12}% !
\TMP@EnsureCode{34}{12}% "
\TMP@EnsureCode{36}{3}% $
\TMP@EnsureCode{38}{4}% &
\TMP@EnsureCode{39}{12}% '
\TMP@EnsureCode{40}{12}% (
\TMP@EnsureCode{41}{12}% )
\TMP@EnsureCode{42}{12}% *
\TMP@EnsureCode{43}{12}% +
\TMP@EnsureCode{44}{12}% ,
\TMP@EnsureCode{45}{12}% -
\TMP@EnsureCode{46}{12}% .
\TMP@EnsureCode{47}{12}% /
\TMP@EnsureCode{58}{12}% :
\TMP@EnsureCode{59}{12}% ;
\TMP@EnsureCode{60}{12}% <
\TMP@EnsureCode{62}{12}% >
\TMP@EnsureCode{63}{12}% ?
\TMP@EnsureCode{91}{12}% [
\TMP@EnsureCode{93}{12}% ]
\TMP@EnsureCode{94}{7}% ^ (superscript)
\TMP@EnsureCode{95}{8}% _ (subscript)
\TMP@EnsureCode{96}{12}% `
\TMP@EnsureCode{124}{12}% |
\edef\HOLOGO@AtEnd{%
  \HOLOGO@AtEnd
  \escapechar\the\escapechar\relax
  \noexpand\endinput
}
\escapechar=92 %
%    \end{macrocode}
%
% \subsection{Logo list}
%
%    \begin{macro}{\hologoList}
%    \begin{macrocode}
\def\hologoList{%
  \hologoEntry{(La)TeX}{}{2011/10/01}%
  \hologoEntry{AmSLaTeX}{}{2010/04/16}%
  \hologoEntry{AmSTeX}{}{2010/04/16}%
  \hologoEntry{biber}{}{2011/10/01}%
  \hologoEntry{BibTeX}{}{2011/10/01}%
  \hologoEntry{BibTeX}{sf}{2011/10/01}%
  \hologoEntry{BibTeX}{sc}{2011/10/01}%
  \hologoEntry{BibTeX8}{}{2011/11/22}%
  \hologoEntry{ConTeXt}{}{2011/03/25}%
  \hologoEntry{ConTeXt}{narrow}{2011/03/25}%
  \hologoEntry{ConTeXt}{simple}{2011/03/25}%
  \hologoEntry{emTeX}{}{2010/04/26}%
  \hologoEntry{eTeX}{}{2010/04/08}%
  \hologoEntry{ExTeX}{}{2011/10/01}%
  \hologoEntry{HanTheThanh}{}{2011/11/29}%
  \hologoEntry{iniTeX}{}{2011/10/01}%
  \hologoEntry{KOMAScript}{}{2011/10/01}%
  \hologoEntry{La}{}{2010/05/08}%
  \hologoEntry{LaTeX}{}{2010/04/08}%
  \hologoEntry{LaTeX2e}{}{2010/04/08}%
  \hologoEntry{LaTeX3}{}{2010/04/24}%
  \hologoEntry{LaTeXe}{}{2010/04/08}%
  \hologoEntry{LaTeXML}{}{2011/11/22}%
  \hologoEntry{LaTeXTeX}{}{2011/10/01}%
  \hologoEntry{LuaLaTeX}{}{2010/04/08}%
  \hologoEntry{LuaTeX}{}{2010/04/08}%
  \hologoEntry{LyX}{}{2011/10/01}%
  \hologoEntry{METAFONT}{}{2011/10/01}%
  \hologoEntry{MetaFun}{}{2011/10/01}%
  \hologoEntry{METAPOST}{}{2011/10/01}%
  \hologoEntry{MetaPost}{}{2011/10/01}%
  \hologoEntry{MiKTeX}{}{2011/10/01}%
  \hologoEntry{NTS}{}{2011/10/01}%
  \hologoEntry{OzMF}{}{2011/10/01}%
  \hologoEntry{OzMP}{}{2011/10/01}%
  \hologoEntry{OzTeX}{}{2011/10/01}%
  \hologoEntry{OzTtH}{}{2011/10/01}%
  \hologoEntry{PCTeX}{}{2011/10/01}%
  \hologoEntry{pdfTeX}{}{2011/10/01}%
  \hologoEntry{pdfLaTeX}{}{2011/10/01}%
  \hologoEntry{PiC}{}{2011/10/01}%
  \hologoEntry{PiCTeX}{}{2011/10/01}%
  \hologoEntry{plainTeX}{}{2010/04/08}%
  \hologoEntry{plainTeX}{space}{2010/04/16}%
  \hologoEntry{plainTeX}{hyphen}{2010/04/16}%
  \hologoEntry{plainTeX}{runtogether}{2010/04/16}%
  \hologoEntry{SageTeX}{}{2011/11/22}%
  \hologoEntry{SLiTeX}{}{2011/10/01}%
  \hologoEntry{SLiTeX}{lift}{2011/10/01}%
  \hologoEntry{SLiTeX}{narrow}{2011/10/01}%
  \hologoEntry{SLiTeX}{simple}{2011/10/01}%
  \hologoEntry{SliTeX}{}{2011/10/01}%
  \hologoEntry{SliTeX}{narrow}{2011/10/01}%
  \hologoEntry{SliTeX}{simple}{2011/10/01}%
  \hologoEntry{SliTeX}{lift}{2011/10/01}%
  \hologoEntry{teTeX}{}{2011/10/01}%
  \hologoEntry{TeX}{}{2010/04/08}%
  \hologoEntry{TeX4ht}{}{2011/11/22}%
  \hologoEntry{TTH}{}{2011/11/22}%
  \hologoEntry{virTeX}{}{2011/10/01}%
  \hologoEntry{VTeX}{}{2010/04/24}%
  \hologoEntry{Xe}{}{2010/04/08}%
  \hologoEntry{XeLaTeX}{}{2010/04/08}%
  \hologoEntry{XeTeX}{}{2010/04/08}%
}
%    \end{macrocode}
%    \end{macro}
%
% \subsection{Load resources}
%
%    \begin{macrocode}
\begingroup\expandafter\expandafter\expandafter\endgroup
\expandafter\ifx\csname RequirePackage\endcsname\relax
  \def\TMP@RequirePackage#1[#2]{%
    \begingroup\expandafter\expandafter\expandafter\endgroup
    \expandafter\ifx\csname ver@#1.sty\endcsname\relax
      \input #1.sty\relax
    \fi
  }%
  \TMP@RequirePackage{ltxcmds}[2011/02/04]%
  \TMP@RequirePackage{infwarerr}[2010/04/08]%
  \TMP@RequirePackage{kvsetkeys}[2010/03/01]%
  \TMP@RequirePackage{kvdefinekeys}[2010/03/01]%
  \TMP@RequirePackage{pdftexcmds}[2010/04/01]%
  \TMP@RequirePackage{ifpdf}[2010/01/28]%
  \TMP@RequirePackage{ifluatex}[2010/03/01]%
  \ltx@IfUndefined{newif}{%
    \expandafter\let\csname newif\endcsname\ltx@newif
  }{}%
  \TMP@RequirePackage{ifxetex}[2009/01/23]%
  \TMP@RequirePackage{ifvtex}[2010/03/01]%
\else
  \RequirePackage{ltxcmds}[2011/02/04]%
  \RequirePackage{infwarerr}[2010/04/08]%
  \RequirePackage{kvsetkeys}[2010/03/01]%
  \RequirePackage{kvdefinekeys}[2010/03/01]%
  \RequirePackage{pdftexcmds}[2010/04/01]%
  \RequirePackage{ifpdf}[2010/01/28]%
  \RequirePackage{ifluatex}[2010/03/01]%
  \RequirePackage{ifxetex}[2009/01/23]%
  \RequirePackage{ifvtex}[2010/03/01]%
\fi
%    \end{macrocode}
%
%    \begin{macro}{\HOLOGO@IfDefined}
%    \begin{macrocode}
\def\HOLOGO@IfExists#1{%
  \ifx\@undefined#1%
    \expandafter\ltx@secondoftwo
  \else
    \ifx\relax#1%
      \expandafter\ltx@secondoftwo
    \else
      \expandafter\expandafter\expandafter\ltx@firstoftwo
    \fi
  \fi
}
%    \end{macrocode}
%    \end{macro}
%
% \subsection{Setup macros}
%
%    \begin{macro}{\hologoSetup}
%    \begin{macrocode}
\def\hologoSetup{%
  \let\HOLOGO@name\relax
  \HOLOGO@Setup
}
%    \end{macrocode}
%    \end{macro}
%
%    \begin{macro}{\hologoLogoSetup}
%    \begin{macrocode}
\def\hologoLogoSetup#1{%
  \edef\HOLOGO@name{#1}%
  \ltx@IfUndefined{HoLogo@\HOLOGO@name}{%
    \@PackageError{hologo}{%
      Unknown logo `\HOLOGO@name'%
    }\@ehc
    \ltx@gobble
  }{%
    \HOLOGO@Setup
  }%
}
%    \end{macrocode}
%    \end{macro}
%
%    \begin{macro}{\HOLOGO@Setup}
%    \begin{macrocode}
\def\HOLOGO@Setup{%
  \kvsetkeys{HoLogo}%
}
%    \end{macrocode}
%    \end{macro}
%
% \subsection{Options}
%
%    \begin{macro}{\HOLOGO@DeclareBoolOption}
%    \begin{macrocode}
\def\HOLOGO@DeclareBoolOption#1{%
  \expandafter\chardef\csname HOLOGOOPT@#1\endcsname\ltx@zero
  \kv@define@key{HoLogo}{#1}[true]{%
    \def\HOLOGO@temp{##1}%
    \ifx\HOLOGO@temp\HOLOGO@true
      \ifx\HOLOGO@name\relax
        \expandafter\chardef\csname HOLOGOOPT@#1\endcsname=\ltx@one
      \else
        \expandafter\chardef\csname
        HoLogoOpt@#1@\HOLOGO@name\endcsname\ltx@one
      \fi
      \HOLOGO@SetBreakAll{#1}%
    \else
      \ifx\HOLOGO@temp\HOLOGO@false
        \ifx\HOLOGO@name\relax
          \expandafter\chardef\csname HOLOGOOPT@#1\endcsname=\ltx@zero
        \else
          \expandafter\chardef\csname
          HoLogoOpt@#1@\HOLOGO@name\endcsname=\ltx@zero
        \fi
        \HOLOGO@SetBreakAll{#1}%
      \else
        \@PackageError{hologo}{%
          Unknown value `##1' for boolean option `#1'.\MessageBreak
          Known values are `true' and `false'%
        }\@ehc
      \fi
    \fi
  }%
}
%    \end{macrocode}
%    \end{macro}
%
%    \begin{macro}{\HOLOGO@SetBreakAll}
%    \begin{macrocode}
\def\HOLOGO@SetBreakAll#1{%
  \def\HOLOGO@temp{#1}%
  \ifx\HOLOGO@temp\HOLOGO@break
    \ifx\HOLOGO@name\relax
      \chardef\HOLOGOOPT@hyphenbreak=\HOLOGOOPT@break
      \chardef\HOLOGOOPT@spacebreak=\HOLOGOOPT@break
      \chardef\HOLOGOOPT@discretionarybreak=\HOLOGOOPT@break
    \else
      \expandafter\chardef
         \csname HoLogoOpt@hyphenbreak@\HOLOGO@name\endcsname=%
         \csname HoLogoOpt@break@\HOLOGO@name\endcsname
      \expandafter\chardef
         \csname HoLogoOpt@spacebreak@\HOLOGO@name\endcsname=%
         \csname HoLogoOpt@break@\HOLOGO@name\endcsname
      \expandafter\chardef
         \csname HoLogoOpt@discretionarybreak@\HOLOGO@name
             \endcsname=%
         \csname HoLogoOpt@break@\HOLOGO@name\endcsname
    \fi
  \fi
}
%    \end{macrocode}
%    \end{macro}
%
%    \begin{macro}{\HOLOGO@true}
%    \begin{macrocode}
\def\HOLOGO@true{true}
%    \end{macrocode}
%    \end{macro}
%    \begin{macro}{\HOLOGO@false}
%    \begin{macrocode}
\def\HOLOGO@false{false}
%    \end{macrocode}
%    \end{macro}
%    \begin{macro}{\HOLOGO@break}
%    \begin{macrocode}
\def\HOLOGO@break{break}
%    \end{macrocode}
%    \end{macro}
%
%    \begin{macrocode}
\HOLOGO@DeclareBoolOption{break}
\HOLOGO@DeclareBoolOption{hyphenbreak}
\HOLOGO@DeclareBoolOption{spacebreak}
\HOLOGO@DeclareBoolOption{discretionarybreak}
%    \end{macrocode}
%
%    \begin{macrocode}
\kv@define@key{HoLogo}{variant}{%
  \ifx\HOLOGO@name\relax
    \@PackageError{hologo}{%
      Option `variant' is not available in \string\hologoSetup,%
      \MessageBreak
      Use \string\hologoLogoSetup\space instead%
    }\@ehc
  \else
    \edef\HOLOGO@temp{#1}%
    \ifx\HOLOGO@temp\ltx@empty
      \expandafter
      \let\csname HoLogoOpt@variant@\HOLOGO@name\endcsname\@undefined
    \else
      \ltx@IfUndefined{HoLogo@\HOLOGO@name @\HOLOGO@temp}{%
        \@PackageError{hologo}{%
          Unknown variant `\HOLOGO@temp' of logo `\HOLOGO@name'%
        }\@ehc
      }{%
        \expandafter
        \let\csname HoLogoOpt@variant@\HOLOGO@name\endcsname
            \HOLOGO@temp
      }%
    \fi
  \fi
}
%    \end{macrocode}
%
%    \begin{macro}{\HOLOGO@Variant}
%    \begin{macrocode}
\def\HOLOGO@Variant#1{%
  #1%
  \ltx@ifundefined{HoLogoOpt@variant@#1}{%
  }{%
    @\csname HoLogoOpt@variant@#1\endcsname
  }%
}
%    \end{macrocode}
%    \end{macro}
%
% \subsection{Break/no-break support}
%
%    \begin{macro}{\HOLOGO@space}
%    \begin{macrocode}
\def\HOLOGO@space{%
  \ltx@ifundefined{HoLogoOpt@spacebreak@\HOLOGO@name}{%
    \ltx@ifundefined{HoLogoOpt@break@\HOLOGO@name}{%
      \chardef\HOLOGO@temp=\HOLOGOOPT@spacebreak
    }{%
      \chardef\HOLOGO@temp=%
        \csname HoLogoOpt@break@\HOLOGO@name\endcsname
    }%
  }{%
    \chardef\HOLOGO@temp=%
      \csname HoLogoOpt@spacebreak@\HOLOGO@name\endcsname
  }%
  \ifcase\HOLOGO@temp
    \penalty10000 %
  \fi
  \ltx@space
}
%    \end{macrocode}
%    \end{macro}
%
%    \begin{macro}{\HOLOGO@hyphen}
%    \begin{macrocode}
\def\HOLOGO@hyphen{%
  \ltx@ifundefined{HoLogoOpt@hyphenbreak@\HOLOGO@name}{%
    \ltx@ifundefined{HoLogoOpt@break@\HOLOGO@name}{%
      \chardef\HOLOGO@temp=\HOLOGOOPT@hyphenbreak
    }{%
      \chardef\HOLOGO@temp=%
        \csname HoLogoOpt@break@\HOLOGO@name\endcsname
    }%
  }{%
    \chardef\HOLOGO@temp=%
      \csname HoLogoOpt@hyphenbreak@\HOLOGO@name\endcsname
  }%
  \ifcase\HOLOGO@temp
    \ltx@mbox{-}%
  \else
    -%
  \fi
}
%    \end{macrocode}
%    \end{macro}
%
%    \begin{macro}{\HOLOGO@discretionary}
%    \begin{macrocode}
\def\HOLOGO@discretionary{%
  \ltx@ifundefined{HoLogoOpt@discretionarybreak@\HOLOGO@name}{%
    \ltx@ifundefined{HoLogoOpt@break@\HOLOGO@name}{%
      \chardef\HOLOGO@temp=\HOLOGOOPT@discretionarybreak
    }{%
      \chardef\HOLOGO@temp=%
        \csname HoLogoOpt@break@\HOLOGO@name\endcsname
    }%
  }{%
    \chardef\HOLOGO@temp=%
      \csname HoLogoOpt@discretionarybreak@\HOLOGO@name\endcsname
  }%
  \ifcase\HOLOGO@temp
  \else
    \-%
  \fi
}
%    \end{macrocode}
%    \end{macro}
%
%    \begin{macro}{\HOLOGO@mbox}
%    \begin{macrocode}
\def\HOLOGO@mbox#1{%
  \ltx@ifundefined{HoLogoOpt@break@\HOLOGO@name}{%
    \chardef\HOLOGO@temp=\HOLOGOOPT@hyphenbreak
  }{%
    \chardef\HOLOGO@temp=%
      \csname HoLogoOpt@break@\HOLOGO@name\endcsname
  }%
  \ifcase\HOLOGO@temp
    \ltx@mbox{#1}%
  \else
    #1%
  \fi
}
%    \end{macrocode}
%    \end{macro}
%
% \subsection{Font support}
%
%    \begin{macro}{\HoLogoFont@font}
%    \begin{tabular}{@{}ll@{}}
%    |#1|:& logo name\\
%    |#2|:& font short name\\
%    |#3|:& text
%    \end{tabular}
%    \begin{macrocode}
\def\HoLogoFont@font#1#2#3{%
  \begingroup
    \ltx@IfUndefined{HoLogoFont@logo@#1.#2}{%
      \ltx@IfUndefined{HoLogoFont@font@#2}{%
        \@PackageWarning{hologo}{%
          Missing font `#2' for logo `#1'%
        }%
        #3%
      }{%
        \csname HoLogoFont@font@#2\endcsname{#3}%
      }%
    }{%
      \csname HoLogoFont@logo@#1.#2\endcsname{#3}%
    }%
  \endgroup
}
%    \end{macrocode}
%    \end{macro}
%
%    \begin{macro}{\HoLogoFont@Def}
%    \begin{macrocode}
\def\HoLogoFont@Def#1{%
  \expandafter\def\csname HoLogoFont@font@#1\endcsname
}
%    \end{macrocode}
%    \end{macro}
%    \begin{macro}{\HoLogoFont@LogoDef}
%    \begin{macrocode}
\def\HoLogoFont@LogoDef#1#2{%
  \expandafter\def\csname HoLogoFont@logo@#1.#2\endcsname
}
%    \end{macrocode}
%    \end{macro}
%
% \subsubsection{Font defaults}
%
%    \begin{macro}{\HoLogoFont@font@general}
%    \begin{macrocode}
\HoLogoFont@Def{general}{}%
%    \end{macrocode}
%    \end{macro}
%
%    \begin{macro}{\HoLogoFont@font@rm}
%    \begin{macrocode}
\ltx@IfUndefined{rmfamily}{%
  \ltx@IfUndefined{rm}{%
  }{%
    \HoLogoFont@Def{rm}{\rm}%
  }%
}{%
  \HoLogoFont@Def{rm}{\rmfamily}%
}
%    \end{macrocode}
%    \end{macro}
%
%    \begin{macro}{\HoLogoFont@font@sf}
%    \begin{macrocode}
\ltx@IfUndefined{sffamily}{%
  \ltx@IfUndefined{sf}{%
  }{%
    \HoLogoFont@Def{sf}{\sf}%
  }%
}{%
  \HoLogoFont@Def{sf}{\sffamily}%
}
%    \end{macrocode}
%    \end{macro}
%
%    \begin{macro}{\HoLogoFont@font@bibsf}
%    In case of \hologo{plainTeX} the original small caps
%    variant is used as default. In \hologo{LaTeX}
%    the definition of package \xpackage{dtklogos} \cite{dtklogos}
%    is used.
%\begin{quote}
%\begin{verbatim}
%\DeclareRobustCommand{\BibTeX}{%
%  B%
%  \kern-.05em%
%  \hbox{%
%    $\m@th$% %% force math size calculations
%    \csname S@\f@size\endcsname
%    \fontsize\sf@size\z@
%    \math@fontsfalse
%    \selectfont
%    I%
%    \kern-.025em%
%    B
%  }%
%  \kern-.08em%
%  \-%
%  \TeX
%}
%\end{verbatim}
%\end{quote}
%    \begin{macrocode}
\ltx@IfUndefined{selectfont}{%
  \ltx@IfUndefined{tensc}{%
    \font\tensc=cmcsc10\relax
  }{}%
  \HoLogoFont@Def{bibsf}{\tensc}%
}{%
  \HoLogoFont@Def{bibsf}{%
    $\mathsurround=0pt$%
    \csname S@\f@size\endcsname
    \fontsize\sf@size{0pt}%
    \math@fontsfalse
    \selectfont
  }%
}
%    \end{macrocode}
%    \end{macro}
%
%    \begin{macro}{\HoLogoFont@font@sc}
%    \begin{macrocode}
\ltx@IfUndefined{scshape}{%
  \ltx@IfUndefined{tensc}{%
    \font\tensc=cmcsc10\relax
  }{}%
  \HoLogoFont@Def{sc}{\tensc}%
}{%
  \HoLogoFont@Def{sc}{\scshape}%
}
%    \end{macrocode}
%    \end{macro}
%
%    \begin{macro}{\HoLogoFont@font@sy}
%    \begin{macrocode}
\ltx@IfUndefined{usefont}{%
  \ltx@IfUndefined{tensy}{%
  }{%
    \HoLogoFont@Def{sy}{\tensy}%
  }%
}{%
  \HoLogoFont@Def{sy}{%
    \usefont{OMS}{cmsy}{m}{n}%
  }%
}
%    \end{macrocode}
%    \end{macro}
%
%    \begin{macro}{\HoLogoFont@font@logo}
%    \begin{macrocode}
\begingroup
  \def\x{LaTeX2e}%
\expandafter\endgroup
\ifx\fmtname\x
  \ltx@IfUndefined{logofamily}{%
    \DeclareRobustCommand\logofamily{%
      \not@math@alphabet\logofamily\relax
      \fontencoding{U}%
      \fontfamily{logo}%
      \selectfont
    }%
  }{}%
  \ltx@IfUndefined{logofamily}{%
  }{%
    \HoLogoFont@Def{logo}{\logofamily}%
  }%
\else
  \ltx@IfUndefined{tenlogo}{%
    \font\tenlogo=logo10\relax
  }{}%
  \HoLogoFont@Def{logo}{\tenlogo}%
\fi
%    \end{macrocode}
%    \end{macro}
%
% \subsubsection{Font setup}
%
%    \begin{macro}{\hologoFontSetup}
%    \begin{macrocode}
\def\hologoFontSetup{%
  \let\HOLOGO@name\relax
  \HOLOGO@FontSetup
}
%    \end{macrocode}
%    \end{macro}
%
%    \begin{macro}{\hologoLogoFontSetup}
%    \begin{macrocode}
\def\hologoLogoFontSetup#1{%
  \edef\HOLOGO@name{#1}%
  \ltx@IfUndefined{HoLogo@\HOLOGO@name}{%
    \@PackageError{hologo}{%
      Unknown logo `\HOLOGO@name'%
    }\@ehc
    \ltx@gobble
  }{%
    \HOLOGO@FontSetup
  }%
}
%    \end{macrocode}
%    \end{macro}
%
%    \begin{macro}{\HOLOGO@FontSetup}
%    \begin{macrocode}
\def\HOLOGO@FontSetup{%
  \kvsetkeys{HoLogoFont}%
}
%    \end{macrocode}
%    \end{macro}
%
%    \begin{macrocode}
\def\HOLOGO@temp#1{%
  \kv@define@key{HoLogoFont}{#1}{%
    \ifx\HOLOGO@name\relax
      \HoLogoFont@Def{#1}{##1}%
    \else
      \HoLogoFont@LogoDef\HOLOGO@name{#1}{##1}%
    \fi
  }%
}
\HOLOGO@temp{general}
\HOLOGO@temp{sf}
%    \end{macrocode}
%
% \subsection{Generic logo commands}
%
%    \begin{macrocode}
\HOLOGO@IfExists\hologo{%
  \@PackageError{hologo}{%
    \string\hologo\ltx@space is already defined.\MessageBreak
    Package loading is aborted%
  }\@ehc
  \HOLOGO@AtEnd
}%
\HOLOGO@IfExists\hologoRobust{%
  \@PackageError{hologo}{%
    \string\hologoRobust\ltx@space is already defined.\MessageBreak
    Package loading is aborted%
  }\@ehc
  \HOLOGO@AtEnd
}%
%    \end{macrocode}
%
% \subsubsection{\cs{hologo} and friends}
%
%    \begin{macrocode}
\ifluatex
  \expandafter\ltx@firstofone
\else
  \expandafter\ltx@gobble
\fi
{%
  \ltx@IfUndefined{ifincsname}{%
    \ifnum\luatexversion<36 %
      \expandafter\ltx@gobble
    \else
      \expandafter\ltx@firstofone
    \fi
    {%
      \begingroup
        \ifcase0%
            \directlua{%
              if tex.enableprimitives then %
                tex.enableprimitives('HOLOGO@', {'ifincsname'})%
              else %
                tex.print('1')%
              end%
            }%
            \ifx\HOLOGO@ifincsname\@undefined 1\fi%
            \relax
          \expandafter\ltx@firstofone
        \else
          \endgroup
          \expandafter\ltx@gobble
        \fi
        {%
          \global\let\ifincsname\HOLOGO@ifincsname
        }%
      \HOLOGO@temp
    }%
  }{}%
}
%    \end{macrocode}
%    \begin{macrocode}
\ltx@IfUndefined{ifincsname}{%
  \catcode`$=14 %
}{%
  \catcode`$=9 %
}
%    \end{macrocode}
%
%    \begin{macro}{\hologo}
%    \begin{macrocode}
\def\hologo#1{%
$ \ifincsname
$   \ltx@ifundefined{HoLogoCs@\HOLOGO@Variant{#1}}{%
$     #1%
$   }{%
$     \csname HoLogoCs@\HOLOGO@Variant{#1}\endcsname\ltx@firstoftwo
$   }%
$ \else
    \HOLOGO@IfExists\texorpdfstring\texorpdfstring\ltx@firstoftwo
    {%
      \hologoRobust{#1}%
    }{%
      \ltx@ifundefined{HoLogoBkm@\HOLOGO@Variant{#1}}{%
        \ltx@ifundefined{HoLogo@#1}{?#1?}{#1}%
      }{%
        \csname HoLogoBkm@\HOLOGO@Variant{#1}\endcsname
        \ltx@firstoftwo
      }%
    }%
$ \fi
}
%    \end{macrocode}
%    \end{macro}
%    \begin{macro}{\Hologo}
%    \begin{macrocode}
\def\Hologo#1{%
$ \ifincsname
$   \ltx@ifundefined{HoLogoCs@\HOLOGO@Variant{#1}}{%
$     #1%
$   }{%
$     \csname HoLogoCs@\HOLOGO@Variant{#1}\endcsname\ltx@secondoftwo
$   }%
$ \else
    \HOLOGO@IfExists\texorpdfstring\texorpdfstring\ltx@firstoftwo
    {%
      \HologoRobust{#1}%
    }{%
      \ltx@ifundefined{HoLogoBkm@\HOLOGO@Variant{#1}}{%
        \ltx@ifundefined{HoLogo@#1}{?#1?}{#1}%
      }{%
        \csname HoLogoBkm@\HOLOGO@Variant{#1}\endcsname
        \ltx@secondoftwo
      }%
    }%
$ \fi
}
%    \end{macrocode}
%    \end{macro}
%
%    \begin{macro}{\hologoVariant}
%    \begin{macrocode}
\def\hologoVariant#1#2{%
  \ifx\relax#2\relax
    \hologo{#1}%
  \else
$   \ifincsname
$     \ltx@ifundefined{HoLogoCs@#1@#2}{%
$       #1%
$     }{%
$       \csname HoLogoCs@#1@#2\endcsname\ltx@firstoftwo
$     }%
$   \else
      \HOLOGO@IfExists\texorpdfstring\texorpdfstring\ltx@firstoftwo
      {%
        \hologoVariantRobust{#1}{#2}%
      }{%
        \ltx@ifundefined{HoLogoBkm@#1@#2}{%
          \ltx@ifundefined{HoLogo@#1}{?#1?}{#1}%
        }{%
          \csname HoLogoBkm@#1@#2\endcsname
          \ltx@firstoftwo
        }%
      }%
$   \fi
  \fi
}
%    \end{macrocode}
%    \end{macro}
%    \begin{macro}{\HologoVariant}
%    \begin{macrocode}
\def\HologoVariant#1#2{%
  \ifx\relax#2\relax
    \Hologo{#1}%
  \else
$   \ifincsname
$     \ltx@ifundefined{HoLogoCs@#1@#2}{%
$       #1%
$     }{%
$       \csname HoLogoCs@#1@#2\endcsname\ltx@secondoftwo
$     }%
$   \else
      \HOLOGO@IfExists\texorpdfstring\texorpdfstring\ltx@firstoftwo
      {%
        \HologoVariantRobust{#1}{#2}%
      }{%
        \ltx@ifundefined{HoLogoBkm@#1@#2}{%
          \ltx@ifundefined{HoLogo@#1}{?#1?}{#1}%
        }{%
          \csname HoLogoBkm@#1@#2\endcsname
          \ltx@secondoftwo
        }%
      }%
$   \fi
  \fi
}
%    \end{macrocode}
%    \end{macro}
%
%    \begin{macrocode}
\catcode`\$=3 %
%    \end{macrocode}
%
% \subsubsection{\cs{hologoRobust} and friends}
%
%    \begin{macro}{\hologoRobust}
%    \begin{macrocode}
\ltx@IfUndefined{protected}{%
  \ltx@IfUndefined{DeclareRobustCommand}{%
    \def\hologoRobust#1%
  }{%
    \DeclareRobustCommand*\hologoRobust[1]%
  }%
}{%
  \protected\def\hologoRobust#1%
}%
{%
  \edef\HOLOGO@name{#1}%
  \ltx@IfUndefined{HoLogo@\HOLOGO@Variant\HOLOGO@name}{%
    \@PackageError{hologo}{%
      Unknown logo `\HOLOGO@name'%
    }\@ehc
    ?\HOLOGO@name?%
  }{%
    \ltx@IfUndefined{ver@tex4ht.sty}{%
      \HoLogoFont@font\HOLOGO@name{general}{%
        \csname HoLogo@\HOLOGO@Variant\HOLOGO@name\endcsname
        \ltx@firstoftwo
      }%
    }{%
      \ltx@IfUndefined{HoLogoHtml@\HOLOGO@Variant\HOLOGO@name}{%
        \HOLOGO@name
      }{%
        \csname HoLogoHtml@\HOLOGO@Variant\HOLOGO@name\endcsname
        \ltx@firstoftwo
      }%
    }%
  }%
}
%    \end{macrocode}
%    \end{macro}
%    \begin{macro}{\HologoRobust}
%    \begin{macrocode}
\ltx@IfUndefined{protected}{%
  \ltx@IfUndefined{DeclareRobustCommand}{%
    \def\HologoRobust#1%
  }{%
    \DeclareRobustCommand*\HologoRobust[1]%
  }%
}{%
  \protected\def\HologoRobust#1%
}%
{%
  \edef\HOLOGO@name{#1}%
  \ltx@IfUndefined{HoLogo@\HOLOGO@Variant\HOLOGO@name}{%
    \@PackageError{hologo}{%
      Unknown logo `\HOLOGO@name'%
    }\@ehc
    ?\HOLOGO@name?%
  }{%
    \ltx@IfUndefined{ver@tex4ht.sty}{%
      \HoLogoFont@font\HOLOGO@name{general}{%
        \csname HoLogo@\HOLOGO@Variant\HOLOGO@name\endcsname
        \ltx@secondoftwo
      }%
    }{%
      \ltx@IfUndefined{HoLogoHtml@\HOLOGO@Variant\HOLOGO@name}{%
        \expandafter\HOLOGO@Uppercase\HOLOGO@name
      }{%
        \csname HoLogoHtml@\HOLOGO@Variant\HOLOGO@name\endcsname
        \ltx@secondoftwo
      }%
    }%
  }%
}
%    \end{macrocode}
%    \end{macro}
%    \begin{macro}{\hologoVariantRobust}
%    \begin{macrocode}
\ltx@IfUndefined{protected}{%
  \ltx@IfUndefined{DeclareRobustCommand}{%
    \def\hologoVariantRobust#1#2%
  }{%
    \DeclareRobustCommand*\hologoVariantRobust[2]%
  }%
}{%
  \protected\def\hologoVariantRobust#1#2%
}%
{%
  \begingroup
    \hologoLogoSetup{#1}{variant={#2}}%
    \hologoRobust{#1}%
  \endgroup
}
%    \end{macrocode}
%    \end{macro}
%    \begin{macro}{\HologoVariantRobust}
%    \begin{macrocode}
\ltx@IfUndefined{protected}{%
  \ltx@IfUndefined{DeclareRobustCommand}{%
    \def\HologoVariantRobust#1#2%
  }{%
    \DeclareRobustCommand*\HologoVariantRobust[2]%
  }%
}{%
  \protected\def\HologoVariantRobust#1#2%
}%
{%
  \begingroup
    \hologoLogoSetup{#1}{variant={#2}}%
    \HologoRobust{#1}%
  \endgroup
}
%    \end{macrocode}
%    \end{macro}
%
%    \begin{macro}{\hologorobust}
%    Macro \cs{hologorobust} is only defined for compatibility.
%    Its use is deprecated.
%    \begin{macrocode}
\def\hologorobust{\hologoRobust}
%    \end{macrocode}
%    \end{macro}
%
% \subsection{Helpers}
%
%    \begin{macro}{\HOLOGO@Uppercase}
%    Macro \cs{HOLOGO@Uppercase} is restricted to \cs{uppercase},
%    because \hologo{plainTeX} or \hologo{iniTeX} do not provide
%    \cs{MakeUppercase}.
%    \begin{macrocode}
\def\HOLOGO@Uppercase#1{\uppercase{#1}}
%    \end{macrocode}
%    \end{macro}
%
%    \begin{macro}{\HOLOGO@PdfdocUnicode}
%    \begin{macrocode}
\def\HOLOGO@PdfdocUnicode{%
  \ifx\ifHy@unicode\iftrue
    \expandafter\ltx@secondoftwo
  \else
    \expandafter\ltx@firstoftwo
  \fi
}
%    \end{macrocode}
%    \end{macro}
%
%    \begin{macro}{\HOLOGO@Math}
%    \begin{macrocode}
\def\HOLOGO@MathSetup{%
  \mathsurround0pt\relax
  \HOLOGO@IfExists\f@series{%
    \if b\expandafter\ltx@car\f@series x\@nil
      \csname boldmath\endcsname
   \fi
  }{}%
}
%    \end{macrocode}
%    \end{macro}
%
%    \begin{macro}{\HOLOGO@TempDimen}
%    \begin{macrocode}
\dimendef\HOLOGO@TempDimen=\ltx@zero
%    \end{macrocode}
%    \end{macro}
%    \begin{macro}{\HOLOGO@NegativeKerning}
%    \begin{macrocode}
\def\HOLOGO@NegativeKerning#1{%
  \begingroup
    \HOLOGO@TempDimen=0pt\relax
    \comma@parse@normalized{#1}{%
      \ifdim\HOLOGO@TempDimen=0pt %
        \expandafter\HOLOGO@@NegativeKerning\comma@entry
      \fi
      \ltx@gobble
    }%
    \ifdim\HOLOGO@TempDimen<0pt %
      \kern\HOLOGO@TempDimen
    \fi
  \endgroup
}
%    \end{macrocode}
%    \end{macro}
%    \begin{macro}{\HOLOGO@@NegativeKerning}
%    \begin{macrocode}
\def\HOLOGO@@NegativeKerning#1#2{%
  \setbox\ltx@zero\hbox{#1#2}%
  \HOLOGO@TempDimen=\wd\ltx@zero
  \setbox\ltx@zero\hbox{#1\kern0pt#2}%
  \advance\HOLOGO@TempDimen by -\wd\ltx@zero
}
%    \end{macrocode}
%    \end{macro}
%
%    \begin{macro}{\HOLOGO@SpaceFactor}
%    \begin{macrocode}
\def\HOLOGO@SpaceFactor{%
  \spacefactor1000 %
}
%    \end{macrocode}
%    \end{macro}
%
%    \begin{macro}{\HOLOGO@Span}
%    \begin{macrocode}
\def\HOLOGO@Span#1#2{%
  \HCode{<span class="HoLogo-#1">}%
  #2%
  \HCode{</span>}%
}
%    \end{macrocode}
%    \end{macro}
%
% \subsubsection{Text subscript}
%
%    \begin{macro}{\HOLOGO@SubScript}%
%    \begin{macrocode}
\def\HOLOGO@SubScript#1{%
  \ltx@IfUndefined{textsubscript}{%
    \ltx@IfUndefined{text}{%
      \ltx@mbox{%
        \mathsurround=0pt\relax
        $%
          _{%
            \ltx@IfUndefined{sf@size}{%
              \mathrm{#1}%
            }{%
              \mbox{%
                \fontsize\sf@size{0pt}\selectfont
                #1%
              }%
            }%
          }%
        $%
      }%
    }{%
      \ltx@mbox{%
        \mathsurround=0pt\relax
        $_{\text{#1}}$%
      }%
    }%
  }{%
    \textsubscript{#1}%
  }%
}
%    \end{macrocode}
%    \end{macro}
%
% \subsection{\hologo{TeX} and friends}
%
% \subsubsection{\hologo{TeX}}
%
%    \begin{macro}{\HoLogo@TeX}
%    Source: \hologo{LaTeX} kernel.
%    \begin{macrocode}
\def\HoLogo@TeX#1{%
  T\kern-.1667em\lower.5ex\hbox{E}\kern-.125emX\HOLOGO@SpaceFactor
}
%    \end{macrocode}
%    \end{macro}
%    \begin{macro}{\HoLogoHtml@TeX}
%    \begin{macrocode}
\def\HoLogoHtml@TeX#1{%
  \HoLogoCss@TeX
  \HOLOGO@Span{TeX}{%
    T%
    \HOLOGO@Span{e}{%
      E%
    }%
    X%
  }%
}
%    \end{macrocode}
%    \end{macro}
%    \begin{macro}{\HoLogoCss@TeX}
%    \begin{macrocode}
\def\HoLogoCss@TeX{%
  \Css{%
    span.HoLogo-TeX span.HoLogo-e{%
      position:relative;%
      top:.5ex;%
      margin-left:-.1667em;%
      margin-right:-.125em;%
    }%
  }%
  \Css{%
    a span.HoLogo-TeX span.HoLogo-e{%
      text-decoration:none;%
    }%
  }%
  \global\let\HoLogoCss@TeX\relax
}
%    \end{macrocode}
%    \end{macro}
%
% \subsubsection{\hologo{plainTeX}}
%
%    \begin{macro}{\HoLogo@plainTeX@space}
%    Source: ``The \hologo{TeX}book''
%    \begin{macrocode}
\def\HoLogo@plainTeX@space#1{%
  \HOLOGO@mbox{#1{p}{P}lain}\HOLOGO@space\hologo{TeX}%
}
%    \end{macrocode}
%    \end{macro}
%    \begin{macro}{\HoLogoCs@plainTeX@space}
%    \begin{macrocode}
\def\HoLogoCs@plainTeX@space#1{#1{p}{P}lain TeX}%
%    \end{macrocode}
%    \end{macro}
%    \begin{macro}{\HoLogoBkm@plainTeX@space}
%    \begin{macrocode}
\def\HoLogoBkm@plainTeX@space#1{%
  #1{p}{P}lain \hologo{TeX}%
}
%    \end{macrocode}
%    \end{macro}
%    \begin{macro}{\HoLogoHtml@plainTeX@space}
%    \begin{macrocode}
\def\HoLogoHtml@plainTeX@space#1{%
  #1{p}{P}lain \hologo{TeX}%
}
%    \end{macrocode}
%    \end{macro}
%
%    \begin{macro}{\HoLogo@plainTeX@hyphen}
%    \begin{macrocode}
\def\HoLogo@plainTeX@hyphen#1{%
  \HOLOGO@mbox{#1{p}{P}lain}\HOLOGO@hyphen\hologo{TeX}%
}
%    \end{macrocode}
%    \end{macro}
%    \begin{macro}{\HoLogoCs@plainTeX@hyphen}
%    \begin{macrocode}
\def\HoLogoCs@plainTeX@hyphen#1{#1{p}{P}lain-TeX}
%    \end{macrocode}
%    \end{macro}
%    \begin{macro}{\HoLogoBkm@plainTeX@hyphen}
%    \begin{macrocode}
\def\HoLogoBkm@plainTeX@hyphen#1{%
  #1{p}{P}lain-\hologo{TeX}%
}
%    \end{macrocode}
%    \end{macro}
%    \begin{macro}{\HoLogoHtml@plainTeX@hyphen}
%    \begin{macrocode}
\def\HoLogoHtml@plainTeX@hyphen#1{%
  #1{p}{P}lain-\hologo{TeX}%
}
%    \end{macrocode}
%    \end{macro}
%
%    \begin{macro}{\HoLogo@plainTeX@runtogether}
%    \begin{macrocode}
\def\HoLogo@plainTeX@runtogether#1{%
  \HOLOGO@mbox{#1{p}{P}lain\hologo{TeX}}%
}
%    \end{macrocode}
%    \end{macro}
%    \begin{macro}{\HoLogoCs@plainTeX@runtogether}
%    \begin{macrocode}
\def\HoLogoCs@plainTeX@runtogether#1{#1{p}{P}lainTeX}
%    \end{macrocode}
%    \end{macro}
%    \begin{macro}{\HoLogoBkm@plainTeX@runtogether}
%    \begin{macrocode}
\def\HoLogoBkm@plainTeX@runtogether#1{%
  #1{p}{P}lain\hologo{TeX}%
}
%    \end{macrocode}
%    \end{macro}
%    \begin{macro}{\HoLogoHtml@plainTeX@runtogether}
%    \begin{macrocode}
\def\HoLogoHtml@plainTeX@runtogether#1{%
  #1{p}{P}lain\hologo{TeX}%
}
%    \end{macrocode}
%    \end{macro}
%
%    \begin{macro}{\HoLogo@plainTeX}
%    \begin{macrocode}
\def\HoLogo@plainTeX{\HoLogo@plainTeX@space}
%    \end{macrocode}
%    \end{macro}
%    \begin{macro}{\HoLogoCs@plainTeX}
%    \begin{macrocode}
\def\HoLogoCs@plainTeX{\HoLogoCs@plainTeX@space}
%    \end{macrocode}
%    \end{macro}
%    \begin{macro}{\HoLogoBkm@plainTeX}
%    \begin{macrocode}
\def\HoLogoBkm@plainTeX{\HoLogoBkm@plainTeX@space}
%    \end{macrocode}
%    \end{macro}
%    \begin{macro}{\HoLogoHtml@plainTeX}
%    \begin{macrocode}
\def\HoLogoHtml@plainTeX{\HoLogoHtml@plainTeX@space}
%    \end{macrocode}
%    \end{macro}
%
% \subsubsection{\hologo{LaTeX}}
%
%    Source: \hologo{LaTeX} kernel.
%\begin{quote}
%\begin{verbatim}
%\DeclareRobustCommand{\LaTeX}{%
%  L%
%  \kern-.36em%
%  {%
%    \sbox\z@ T%
%    \vbox to\ht\z@{%
%      \hbox{%
%        \check@mathfonts
%        \fontsize\sf@size\z@
%        \math@fontsfalse
%        \selectfont
%        A%
%      }%
%      \vss
%    }%
%  }%
%  \kern-.15em%
%  \TeX
%}
%\end{verbatim}
%\end{quote}
%
%    \begin{macro}{\HoLogo@La}
%    \begin{macrocode}
\def\HoLogo@La#1{%
  L%
  \kern-.36em%
  \begingroup
    \setbox\ltx@zero\hbox{T}%
    \vbox to\ht\ltx@zero{%
      \hbox{%
        \ltx@ifundefined{check@mathfonts}{%
          \csname sevenrm\endcsname
        }{%
          \check@mathfonts
          \fontsize\sf@size{0pt}%
          \math@fontsfalse\selectfont
        }%
        A%
      }%
      \vss
    }%
  \endgroup
}
%    \end{macrocode}
%    \end{macro}
%
%    \begin{macro}{\HoLogo@LaTeX}
%    Source: \hologo{LaTeX} kernel.
%    \begin{macrocode}
\def\HoLogo@LaTeX#1{%
  \hologo{La}%
  \kern-.15em%
  \hologo{TeX}%
}
%    \end{macrocode}
%    \end{macro}
%    \begin{macro}{\HoLogoHtml@LaTeX}
%    \begin{macrocode}
\def\HoLogoHtml@LaTeX#1{%
  \HoLogoCss@LaTeX
  \HOLOGO@Span{LaTeX}{%
    L%
    \HOLOGO@Span{a}{%
      A%
    }%
    \hologo{TeX}%
  }%
}
%    \end{macrocode}
%    \end{macro}
%    \begin{macro}{\HoLogoCss@LaTeX}
%    \begin{macrocode}
\def\HoLogoCss@LaTeX{%
  \Css{%
    span.HoLogo-LaTeX span.HoLogo-a{%
      position:relative;%
      top:-.5ex;%
      margin-left:-.36em;%
      margin-right:-.15em;%
      font-size:85\%;%
    }%
  }%
  \global\let\HoLogoCss@LaTeX\relax
}
%    \end{macrocode}
%    \end{macro}
%
% \subsubsection{\hologo{(La)TeX}}
%
%    \begin{macro}{\HoLogo@LaTeXTeX}
%    The kerning around the parentheses is taken
%    from package \xpackage{dtklogos} \cite{dtklogos}.
%\begin{quote}
%\begin{verbatim}
%\DeclareRobustCommand{\LaTeXTeX}{%
%  (%
%  \kern-.15em%
%  L%
%  \kern-.36em%
%  {%
%    \sbox\z@ T%
%    \vbox to\ht0{%
%      \hbox{%
%        $\m@th$%
%        \csname S@\f@size\endcsname
%        \fontsize\sf@size\z@
%        \math@fontsfalse
%        \selectfont
%        A%
%      }%
%      \vss
%    }%
%  }%
%  \kern-.2em%
%  )%
%  \kern-.15em%
%  \TeX
%}
%\end{verbatim}
%\end{quote}
%    \begin{macrocode}
\def\HoLogo@LaTeXTeX#1{%
  (%
  \kern-.15em%
  \hologo{La}%
  \kern-.2em%
  )%
  \kern-.15em%
  \hologo{TeX}%
}
%    \end{macrocode}
%    \end{macro}
%    \begin{macro}{\HoLogoBkm@LaTeXTeX}
%    \begin{macrocode}
\def\HoLogoBkm@LaTeXTeX#1{(La)TeX}
%    \end{macrocode}
%    \end{macro}
%
%    \begin{macro}{\HoLogo@(La)TeX}
%    \begin{macrocode}
\expandafter
\let\csname HoLogo@(La)TeX\endcsname\HoLogo@LaTeXTeX
%    \end{macrocode}
%    \end{macro}
%    \begin{macro}{\HoLogoBkm@(La)TeX}
%    \begin{macrocode}
\expandafter
\let\csname HoLogoBkm@(La)TeX\endcsname\HoLogoBkm@LaTeXTeX
%    \end{macrocode}
%    \end{macro}
%    \begin{macro}{\HoLogoHtml@LaTeXTeX}
%    \begin{macrocode}
\def\HoLogoHtml@LaTeXTeX#1{%
  \HoLogoCss@LaTeXTeX
  \HOLOGO@Span{LaTeXTeX}{%
    (%
    \HOLOGO@Span{L}{L}%
    \HOLOGO@Span{a}{A}%
    \HOLOGO@Span{ParenRight}{)}%
    \hologo{TeX}%
  }%
}
%    \end{macrocode}
%    \end{macro}
%    \begin{macro}{\HoLogoHtml@(La)TeX}
%    Kerning after opening parentheses and before closing parentheses
%    is $-0.1$\,em. The original values $-0.15$\,em
%    looked too ugly for a serif font.
%    \begin{macrocode}
\expandafter
\let\csname HoLogoHtml@(La)TeX\endcsname\HoLogoHtml@LaTeXTeX
%    \end{macrocode}
%    \end{macro}
%    \begin{macro}{\HoLogoCss@LaTeXTeX}
%    \begin{macrocode}
\def\HoLogoCss@LaTeXTeX{%
  \Css{%
    span.HoLogo-LaTeXTeX span.HoLogo-L{%
      margin-left:-.1em;%
    }%
  }%
  \Css{%
    span.HoLogo-LaTeXTeX span.HoLogo-a{%
      position:relative;%
      top:-.5ex;%
      margin-left:-.36em;%
      margin-right:-.1em;%
      font-size:85\%;%
    }%
  }%
  \Css{%
    span.HoLogo-LaTeXTeX span.HoLogo-ParenRight{%
      margin-right:-.15em;%
    }%
  }%
  \global\let\HoLogoCss@LaTeXTeX\relax
}
%    \end{macrocode}
%    \end{macro}
%
% \subsubsection{\hologo{LaTeXe}}
%
%    \begin{macro}{\HoLogo@LaTeXe}
%    Source: \hologo{LaTeX} kernel
%    \begin{macrocode}
\def\HoLogo@LaTeXe#1{%
  \hologo{LaTeX}%
  \kern.15em%
  \hbox{%
    \HOLOGO@MathSetup
    2%
    $_{\textstyle\varepsilon}$%
  }%
}
%    \end{macrocode}
%    \end{macro}
%
%    \begin{macro}{\HoLogoCs@LaTeXe}
%    \begin{macrocode}
\ifnum64=`\^^^^0040\relax % test for big chars of LuaTeX/XeTeX
  \catcode`\$=9 %
  \catcode`\&=14 %
\else
  \catcode`\$=14 %
  \catcode`\&=9 %
\fi
\def\HoLogoCs@LaTeXe#1{%
  LaTeX2%
$ \string ^^^^0395%
& e%
}%
\catcode`\$=3 %
\catcode`\&=4 %
%    \end{macrocode}
%    \end{macro}
%
%    \begin{macro}{\HoLogoBkm@LaTeXe}
%    \begin{macrocode}
\def\HoLogoBkm@LaTeXe#1{%
  \hologo{LaTeX}%
  2%
  \HOLOGO@PdfdocUnicode{e}{\textepsilon}%
}
%    \end{macrocode}
%    \end{macro}
%
%    \begin{macro}{\HoLogoHtml@LaTeXe}
%    \begin{macrocode}
\def\HoLogoHtml@LaTeXe#1{%
  \HoLogoCss@LaTeXe
  \HOLOGO@Span{LaTeX2e}{%
    \hologo{LaTeX}%
    \HOLOGO@Span{2}{2}%
    \HOLOGO@Span{e}{%
      \HOLOGO@MathSetup
      \ensuremath{\textstyle\varepsilon}%
    }%
  }%
}
%    \end{macrocode}
%    \end{macro}
%    \begin{macro}{\HoLogoCss@LaTeXe}
%    \begin{macrocode}
\def\HoLogoCss@LaTeXe{%
  \Css{%
    span.HoLogo-LaTeX2e span.HoLogo-2{%
      padding-left:.15em;%
    }%
  }%
  \Css{%
    span.HoLogo-LaTeX2e span.HoLogo-e{%
      position:relative;%
      top:.35ex;%
      text-decoration:none;%
    }%
  }%
  \global\let\HoLogoCss@LaTeXe\relax
}
%    \end{macrocode}
%    \end{macro}
%
%    \begin{macro}{\HoLogo@LaTeX2e}
%    \begin{macrocode}
\expandafter
\let\csname HoLogo@LaTeX2e\endcsname\HoLogo@LaTeXe
%    \end{macrocode}
%    \end{macro}
%    \begin{macro}{\HoLogoCs@LaTeX2e}
%    \begin{macrocode}
\expandafter
\let\csname HoLogoCs@LaTeX2e\endcsname\HoLogoCs@LaTeXe
%    \end{macrocode}
%    \end{macro}
%    \begin{macro}{\HoLogoBkm@LaTeX2e}
%    \begin{macrocode}
\expandafter
\let\csname HoLogoBkm@LaTeX2e\endcsname\HoLogoBkm@LaTeXe
%    \end{macrocode}
%    \end{macro}
%    \begin{macro}{\HoLogoHtml@LaTeX2e}
%    \begin{macrocode}
\expandafter
\let\csname HoLogoHtml@LaTeX2e\endcsname\HoLogoHtml@LaTeXe
%    \end{macrocode}
%    \end{macro}
%
% \subsubsection{\hologo{LaTeX3}}
%
%    \begin{macro}{\HoLogo@LaTeX3}
%    Source: \hologo{LaTeX} kernel
%    \begin{macrocode}
\expandafter\def\csname HoLogo@LaTeX3\endcsname#1{%
  \hologo{LaTeX}%
  3%
}
%    \end{macrocode}
%    \end{macro}
%
%    \begin{macro}{\HoLogoBkm@LaTeX3}
%    \begin{macrocode}
\expandafter\def\csname HoLogoBkm@LaTeX3\endcsname#1{%
  \hologo{LaTeX}%
  3%
}
%    \end{macrocode}
%    \end{macro}
%    \begin{macro}{\HoLogoHtml@LaTeX3}
%    \begin{macrocode}
\expandafter
\let\csname HoLogoHtml@LaTeX3\expandafter\endcsname
\csname HoLogo@LaTeX3\endcsname
%    \end{macrocode}
%    \end{macro}
%
% \subsubsection{\hologo{LaTeXML}}
%
%    \begin{macro}{\HoLogo@LaTeXML}
%    \begin{macrocode}
\def\HoLogo@LaTeXML#1{%
  \HOLOGO@mbox{%
    \hologo{La}%
    \kern-.15em%
    T%
    \kern-.1667em%
    \lower.5ex\hbox{E}%
    \kern-.125em%
    \HoLogoFont@font{LaTeXML}{sc}{xml}%
  }%
}
%    \end{macrocode}
%    \end{macro}
%    \begin{macro}{\HoLogoHtml@pdfLaTeX}
%    \begin{macrocode}
\def\HoLogoHtml@LaTeXML#1{%
  \HOLOGO@Span{LaTeXML}{%
    \HoLogoCss@LaTeX
    \HoLogoCss@TeX
    \HOLOGO@Span{LaTeX}{%
      L%
      \HOLOGO@Span{a}{%
        A%
      }%
    }%
    \HOLOGO@Span{TeX}{%
      T%
      \HOLOGO@Span{e}{%
        E%
      }%
    }%
    \HCode{<span style="font-variant: small-caps;">}%
    xml%
    \HCode{</span>}%
  }%
}
%    \end{macrocode}
%    \end{macro}
%
% \subsubsection{\hologo{eTeX}}
%
%    \begin{macro}{\HoLogo@eTeX}
%    Source: package \xpackage{etex}
%    \begin{macrocode}
\def\HoLogo@eTeX#1{%
  \ltx@mbox{%
    \HOLOGO@MathSetup
    $\varepsilon$%
    -%
    \HOLOGO@NegativeKerning{-T,T-,To}%
    \hologo{TeX}%
  }%
}
%    \end{macrocode}
%    \end{macro}
%    \begin{macro}{\HoLogoCs@eTeX}
%    \begin{macrocode}
\ifnum64=`\^^^^0040\relax % test for big chars of LuaTeX/XeTeX
  \catcode`\$=9 %
  \catcode`\&=14 %
\else
  \catcode`\$=14 %
  \catcode`\&=9 %
\fi
\def\HoLogoCs@eTeX#1{%
$ #1{\string ^^^^0395}{\string ^^^^03b5}%
& #1{e}{E}%
  TeX%
}%
\catcode`\$=3 %
\catcode`\&=4 %
%    \end{macrocode}
%    \end{macro}
%    \begin{macro}{\HoLogoBkm@eTeX}
%    \begin{macrocode}
\def\HoLogoBkm@eTeX#1{%
  \HOLOGO@PdfdocUnicode{#1{e}{E}}{\textepsilon}%
  -%
  \hologo{TeX}%
}
%    \end{macrocode}
%    \end{macro}
%    \begin{macro}{\HoLogoHtml@eTeX}
%    \begin{macrocode}
\def\HoLogoHtml@eTeX#1{%
  \ltx@mbox{%
    \HOLOGO@MathSetup
    $\varepsilon$%
    -%
    \hologo{TeX}%
  }%
}
%    \end{macrocode}
%    \end{macro}
%
% \subsubsection{\hologo{iniTeX}}
%
%    \begin{macro}{\HoLogo@iniTeX}
%    \begin{macrocode}
\def\HoLogo@iniTeX#1{%
  \HOLOGO@mbox{%
    #1{i}{I}ni\hologo{TeX}%
  }%
}
%    \end{macrocode}
%    \end{macro}
%    \begin{macro}{\HoLogoCs@iniTeX}
%    \begin{macrocode}
\def\HoLogoCs@iniTeX#1{#1{i}{I}niTeX}
%    \end{macrocode}
%    \end{macro}
%    \begin{macro}{\HoLogoBkm@iniTeX}
%    \begin{macrocode}
\def\HoLogoBkm@iniTeX#1{%
  #1{i}{I}ni\hologo{TeX}%
}
%    \end{macrocode}
%    \end{macro}
%    \begin{macro}{\HoLogoHtml@iniTeX}
%    \begin{macrocode}
\let\HoLogoHtml@iniTeX\HoLogo@iniTeX
%    \end{macrocode}
%    \end{macro}
%
% \subsubsection{\hologo{virTeX}}
%
%    \begin{macro}{\HoLogo@virTeX}
%    \begin{macrocode}
\def\HoLogo@virTeX#1{%
  \HOLOGO@mbox{%
    #1{v}{V}ir\hologo{TeX}%
  }%
}
%    \end{macrocode}
%    \end{macro}
%    \begin{macro}{\HoLogoCs@virTeX}
%    \begin{macrocode}
\def\HoLogoCs@virTeX#1{#1{v}{V}irTeX}
%    \end{macrocode}
%    \end{macro}
%    \begin{macro}{\HoLogoBkm@virTeX}
%    \begin{macrocode}
\def\HoLogoBkm@virTeX#1{%
  #1{v}{V}ir\hologo{TeX}%
}
%    \end{macrocode}
%    \end{macro}
%    \begin{macro}{\HoLogoHtml@virTeX}
%    \begin{macrocode}
\let\HoLogoHtml@virTeX\HoLogo@virTeX
%    \end{macrocode}
%    \end{macro}
%
% \subsubsection{\hologo{SliTeX}}
%
% \paragraph{Definitions of the three variants.}
%
%    \begin{macro}{\HoLogo@SLiTeX@lift}
%    \begin{macrocode}
\def\HoLogo@SLiTeX@lift#1{%
  \HoLogoFont@font{SliTeX}{rm}{%
    S%
    \kern-.06em%
    L%
    \kern-.18em%
    \raise.32ex\hbox{\HoLogoFont@font{SliTeX}{sc}{i}}%
    \HOLOGO@discretionary
    \kern-.06em%
    \hologo{TeX}%
  }%
}
%    \end{macrocode}
%    \end{macro}
%    \begin{macro}{\HoLogoBkm@SLiTeX@lift}
%    \begin{macrocode}
\def\HoLogoBkm@SLiTeX@lift#1{SLiTeX}
%    \end{macrocode}
%    \end{macro}
%    \begin{macro}{\HoLogoHtml@SLiTeX@lift}
%    \begin{macrocode}
\def\HoLogoHtml@SLiTeX@lift#1{%
  \HoLogoCss@SLiTeX@lift
  \HOLOGO@Span{SLiTeX-lift}{%
    \HoLogoFont@font{SliTeX}{rm}{%
      S%
      \HOLOGO@Span{L}{L}%
      \HOLOGO@Span{i}{i}%
      \hologo{TeX}%
    }%
  }%
}
%    \end{macrocode}
%    \end{macro}
%    \begin{macro}{\HoLogoCss@SLiTeX@lift}
%    \begin{macrocode}
\def\HoLogoCss@SLiTeX@lift{%
  \Css{%
    span.HoLogo-SLiTeX-lift span.HoLogo-L{%
      margin-left:-.06em;%
      margin-right:-.18em;%
    }%
  }%
  \Css{%
    span.HoLogo-SLiTeX-lift span.HoLogo-i{%
      position:relative;%
      top:-.32ex;%
      margin-right:-.06em;%
      font-variant:small-caps;%
    }%
  }%
  \global\let\HoLogoCss@SLiTeX@lift\relax
}
%    \end{macrocode}
%    \end{macro}
%
%    \begin{macro}{\HoLogo@SliTeX@simple}
%    \begin{macrocode}
\def\HoLogo@SliTeX@simple#1{%
  \HoLogoFont@font{SliTeX}{rm}{%
    \ltx@mbox{%
      \HoLogoFont@font{SliTeX}{sc}{Sli}%
    }%
    \HOLOGO@discretionary
    \hologo{TeX}%
  }%
}
%    \end{macrocode}
%    \end{macro}
%    \begin{macro}{\HoLogoBkm@SliTeX@simple}
%    \begin{macrocode}
\def\HoLogoBkm@SliTeX@simple#1{SliTeX}
%    \end{macrocode}
%    \end{macro}
%    \begin{macro}{\HoLogoHtml@SliTeX@simple}
%    \begin{macrocode}
\let\HoLogoHtml@SliTeX@simple\HoLogo@SliTeX@simple
%    \end{macrocode}
%    \end{macro}
%
%    \begin{macro}{\HoLogo@SliTeX@narrow}
%    \begin{macrocode}
\def\HoLogo@SliTeX@narrow#1{%
  \HoLogoFont@font{SliTeX}{rm}{%
    \ltx@mbox{%
      S%
      \kern-.06em%
      \HoLogoFont@font{SliTeX}{sc}{%
        l%
        \kern-.035em%
        i%
      }%
    }%
    \HOLOGO@discretionary
    \kern-.06em%
    \hologo{TeX}%
  }%
}
%    \end{macrocode}
%    \end{macro}
%    \begin{macro}{\HoLogoBkm@SliTeX@narrow}
%    \begin{macrocode}
\def\HoLogoBkm@SliTeX@narrow#1{SliTeX}
%    \end{macrocode}
%    \end{macro}
%    \begin{macro}{\HoLogoHtml@SliTeX@narrow}
%    \begin{macrocode}
\def\HoLogoHtml@SliTeX@narrow#1{%
  \HoLogoCss@SliTeX@narrow
  \HOLOGO@Span{SliTeX-narrow}{%
    \HoLogoFont@font{SliTeX}{rm}{%
      S%
        \HOLOGO@Span{l}{l}%
        \HOLOGO@Span{i}{i}%
      \hologo{TeX}%
    }%
  }%
}
%    \end{macrocode}
%    \end{macro}
%    \begin{macro}{\HoLogoCss@SliTeX@narrow}
%    \begin{macrocode}
\def\HoLogoCss@SliTeX@narrow{%
  \Css{%
    span.HoLogo-SliTeX-narrow span.HoLogo-l{%
      margin-left:-.06em;%
      margin-right:-.035em;%
      font-variant:small-caps;%
    }%
  }%
  \Css{%
    span.HoLogo-SliTeX-narrow span.HoLogo-i{%
      margin-right:-.06em;%
      font-variant:small-caps;%
    }%
  }%
  \global\let\HoLogoCss@SliTeX@narrow\relax
}
%    \end{macrocode}
%    \end{macro}
%
% \paragraph{Macro set completion.}
%
%    \begin{macro}{\HoLogo@SLiTeX@simple}
%    \begin{macrocode}
\def\HoLogo@SLiTeX@simple{\HoLogo@SliTeX@simple}
%    \end{macrocode}
%    \end{macro}
%    \begin{macro}{\HoLogoBkm@SLiTeX@simple}
%    \begin{macrocode}
\def\HoLogoBkm@SLiTeX@simple{\HoLogoBkm@SliTeX@simple}
%    \end{macrocode}
%    \end{macro}
%    \begin{macro}{\HoLogoHtml@SLiTeX@simple}
%    \begin{macrocode}
\def\HoLogoHtml@SLiTeX@simple{\HoLogoHtml@SliTeX@simple}
%    \end{macrocode}
%    \end{macro}
%
%    \begin{macro}{\HoLogo@SLiTeX@narrow}
%    \begin{macrocode}
\def\HoLogo@SLiTeX@narrow{\HoLogo@SliTeX@narrow}
%    \end{macrocode}
%    \end{macro}
%    \begin{macro}{\HoLogoBkm@SLiTeX@narrow}
%    \begin{macrocode}
\def\HoLogoBkm@SLiTeX@narrow{\HoLogoBkm@SliTeX@narrow}
%    \end{macrocode}
%    \end{macro}
%    \begin{macro}{\HoLogoHtml@SLiTeX@narrow}
%    \begin{macrocode}
\def\HoLogoHtml@SLiTeX@narrow{\HoLogoHtml@SliTeX@narrow}
%    \end{macrocode}
%    \end{macro}
%
%    \begin{macro}{\HoLogo@SliTeX@lift}
%    \begin{macrocode}
\def\HoLogo@SliTeX@lift{\HoLogo@SLiTeX@lift}
%    \end{macrocode}
%    \end{macro}
%    \begin{macro}{\HoLogoBkm@SliTeX@lift}
%    \begin{macrocode}
\def\HoLogoBkm@SliTeX@lift{\HoLogoBkm@SLiTeX@lift}
%    \end{macrocode}
%    \end{macro}
%    \begin{macro}{\HoLogoHtml@SliTeX@lift}
%    \begin{macrocode}
\def\HoLogoHtml@SliTeX@lift{\HoLogoHtml@SLiTeX@lift}
%    \end{macrocode}
%    \end{macro}
%
% \paragraph{Defaults.}
%
%    \begin{macro}{\HoLogo@SLiTeX}
%    \begin{macrocode}
\def\HoLogo@SLiTeX{\HoLogo@SLiTeX@lift}
%    \end{macrocode}
%    \end{macro}
%    \begin{macro}{\HoLogoBkm@SLiTeX}
%    \begin{macrocode}
\def\HoLogoBkm@SLiTeX{\HoLogoBkm@SLiTeX@lift}
%    \end{macrocode}
%    \end{macro}
%    \begin{macro}{\HoLogoHtml@SLiTeX}
%    \begin{macrocode}
\def\HoLogoHtml@SLiTeX{\HoLogoHtml@SLiTeX@lift}
%    \end{macrocode}
%    \end{macro}
%
%    \begin{macro}{\HoLogo@SliTeX}
%    \begin{macrocode}
\def\HoLogo@SliTeX{\HoLogo@SliTeX@narrow}
%    \end{macrocode}
%    \end{macro}
%    \begin{macro}{\HoLogoBkm@SliTeX}
%    \begin{macrocode}
\def\HoLogoBkm@SliTeX{\HoLogoBkm@SliTeX@narrow}
%    \end{macrocode}
%    \end{macro}
%    \begin{macro}{\HoLogoHtml@SliTeX}
%    \begin{macrocode}
\def\HoLogoHtml@SliTeX{\HoLogoHtml@SliTeX@narrow}
%    \end{macrocode}
%    \end{macro}
%
% \subsubsection{\hologo{LuaTeX}}
%
%    \begin{macro}{\HoLogo@LuaTeX}
%    The kerning is an idea of Hans Hagen, see mailing list
%    `luatex at tug dot org' in March 2010.
%    \begin{macrocode}
\def\HoLogo@LuaTeX#1{%
  \HOLOGO@mbox{%
    Lua%
    \HOLOGO@NegativeKerning{aT,oT,To}%
    \hologo{TeX}%
  }%
}
%    \end{macrocode}
%    \end{macro}
%    \begin{macro}{\HoLogoHtml@LuaTeX}
%    \begin{macrocode}
\let\HoLogoHtml@LuaTeX\HoLogo@LuaTeX
%    \end{macrocode}
%    \end{macro}
%
% \subsubsection{\hologo{LuaLaTeX}}
%
%    \begin{macro}{\HoLogo@LuaLaTeX}
%    \begin{macrocode}
\def\HoLogo@LuaLaTeX#1{%
  \HOLOGO@mbox{%
    Lua%
    \hologo{LaTeX}%
  }%
}
%    \end{macrocode}
%    \end{macro}
%    \begin{macro}{\HoLogoHtml@LuaLaTeX}
%    \begin{macrocode}
\let\HoLogoHtml@LuaLaTeX\HoLogo@LuaLaTeX
%    \end{macrocode}
%    \end{macro}
%
% \subsubsection{\hologo{XeTeX}, \hologo{XeLaTeX}}
%
%    \begin{macro}{\HOLOGO@IfCharExists}
%    \begin{macrocode}
\ifluatex
  \ifnum\luatexversion<36 %
  \else
    \def\HOLOGO@IfCharExists#1{%
      \ifnum
        \directlua{%
           if luaotfload and luaotfload.aux then
             if luaotfload.aux.font_has_glyph(%
                    font.current(), \number#1) then % 	 
	       tex.print("1") % 	 
	     end % 	 
	   elseif font and font.fonts and font.current then %
            local f = font.fonts[font.current()]%
            if f.characters and f.characters[\number#1] then %
              tex.print("1")%
            end %
          end%
        }0=\ltx@zero
        \expandafter\ltx@secondoftwo
      \else
        \expandafter\ltx@firstoftwo
      \fi
    }%
  \fi
\fi
\ltx@IfUndefined{HOLOGO@IfCharExists}{%
  \def\HOLOGO@@IfCharExists#1{%
    \begingroup
      \tracinglostchars=\ltx@zero
      \setbox\ltx@zero=\hbox{%
        \kern7sp\char#1\relax
        \ifnum\lastkern>\ltx@zero
          \expandafter\aftergroup\csname iffalse\endcsname
        \else
          \expandafter\aftergroup\csname iftrue\endcsname
        \fi
      }%
      % \if{true|false} from \aftergroup
      \endgroup
      \expandafter\ltx@firstoftwo
    \else
      \endgroup
      \expandafter\ltx@secondoftwo
    \fi
  }%
  \ifxetex
    \ltx@IfUndefined{XeTeXfonttype}{}{%
      \ltx@IfUndefined{XeTeXcharglyph}{}{%
        \def\HOLOGO@IfCharExists#1{%
          \ifnum\XeTeXfonttype\font>\ltx@zero
            \expandafter\ltx@firstofthree
          \else
            \expandafter\ltx@gobble
          \fi
          {%
            \ifnum\XeTeXcharglyph#1>\ltx@zero
              \expandafter\ltx@firstoftwo
            \else
              \expandafter\ltx@secondoftwo
            \fi
          }%
          \HOLOGO@@IfCharExists{#1}%
        }%
      }%
    }%
  \fi
}{}
\ltx@ifundefined{HOLOGO@IfCharExists}{%
  \ifnum64=`\^^^^0040\relax % test for big chars of LuaTeX/XeTeX
    \let\HOLOGO@IfCharExists\HOLOGO@@IfCharExists
  \else
    \def\HOLOGO@IfCharExists#1{%
      \ifnum#1>255 %
        \expandafter\ltx@fourthoffour
      \fi
      \HOLOGO@@IfCharExists{#1}%
    }%
  \fi
}{}
%    \end{macrocode}
%    \end{macro}
%
%    \begin{macro}{\HoLogo@Xe}
%    Source: package \xpackage{dtklogos}
%    \begin{macrocode}
\def\HoLogo@Xe#1{%
  X%
  \kern-.1em\relax
  \HOLOGO@IfCharExists{"018E}{%
    \lower.5ex\hbox{\char"018E}%
  }{%
    \chardef\HOLOGO@choice=\ltx@zero
    \ifdim\fontdimen\ltx@one\font>0pt %
      \ltx@IfUndefined{rotatebox}{%
        \ltx@IfUndefined{pgftext}{%
          \ltx@IfUndefined{psscalebox}{%
            \ltx@IfUndefined{HOLOGO@ScaleBox@\hologoDriver}{%
            }{%
              \chardef\HOLOGO@choice=4 %
            }%
          }{%
            \chardef\HOLOGO@choice=3 %
          }%
        }{%
          \chardef\HOLOGO@choice=2 %
        }%
      }{%
        \chardef\HOLOGO@choice=1 %
      }%
      \ifcase\HOLOGO@choice
        \HOLOGO@WarningUnsupportedDriver{Xe}%
        e%
      \or % 1: \rotatebox
        \begingroup
          \setbox\ltx@zero\hbox{\rotatebox{180}{E}}%
          \ltx@LocDimenA=\dp\ltx@zero
          \advance\ltx@LocDimenA by -.5ex\relax
          \raise\ltx@LocDimenA\box\ltx@zero
        \endgroup
      \or % 2: \pgftext
        \lower.5ex\hbox{%
          \pgfpicture
            \pgftext[rotate=180]{E}%
          \endpgfpicture
        }%
      \or % 3: \psscalebox
        \begingroup
          \setbox\ltx@zero\hbox{\psscalebox{-1 -1}{E}}%
          \ltx@LocDimenA=\dp\ltx@zero
          \advance\ltx@LocDimenA by -.5ex\relax
          \raise\ltx@LocDimenA\box\ltx@zero
        \endgroup
      \or % 4: \HOLOGO@PointReflectBox
        \lower.5ex\hbox{\HOLOGO@PointReflectBox{E}}%
      \else
        \@PackageError{hologo}{Internal error (choice/it}\@ehc
      \fi
    \else
      \ltx@IfUndefined{reflectbox}{%
        \ltx@IfUndefined{pgftext}{%
          \ltx@IfUndefined{psscalebox}{%
            \ltx@IfUndefined{HOLOGO@ScaleBox@\hologoDriver}{%
            }{%
              \chardef\HOLOGO@choice=4 %
            }%
          }{%
            \chardef\HOLOGO@choice=3 %
          }%
        }{%
          \chardef\HOLOGO@choice=2 %
        }%
      }{%
        \chardef\HOLOGO@choice=1 %
      }%
      \ifcase\HOLOGO@choice
        \HOLOGO@WarningUnsupportedDriver{Xe}%
        e%
      \or % 1: reflectbox
        \lower.5ex\hbox{%
          \reflectbox{E}%
        }%
      \or % 2: \pgftext
        \lower.5ex\hbox{%
          \pgfpicture
            \pgftransformxscale{-1}%
            \pgftext{E}%
          \endpgfpicture
        }%
      \or % 3: \psscalebox
        \lower.5ex\hbox{%
          \psscalebox{-1 1}{E}%
        }%
      \or % 4: \HOLOGO@Reflectbox
        \lower.5ex\hbox{%
          \HOLOGO@ReflectBox{E}%
        }%
      \else
        \@PackageError{hologo}{Internal error (choice/up)}\@ehc
      \fi
    \fi
  }%
}
%    \end{macrocode}
%    \end{macro}
%    \begin{macro}{\HoLogoHtml@Xe}
%    \begin{macrocode}
\def\HoLogoHtml@Xe#1{%
  \HoLogoCss@Xe
  \HOLOGO@Span{Xe}{%
    X%
    \HOLOGO@Span{e}{%
      \HCode{&\ltx@hashchar x018e;}%
    }%
  }%
}
%    \end{macrocode}
%    \end{macro}
%    \begin{macro}{\HoLogoCss@Xe}
%    \begin{macrocode}
\def\HoLogoCss@Xe{%
  \Css{%
    span.HoLogo-Xe span.HoLogo-e{%
      position:relative;%
      top:.5ex;%
      left-margin:-.1em;%
    }%
  }%
  \global\let\HoLogoCss@Xe\relax
}
%    \end{macrocode}
%    \end{macro}
%
%    \begin{macro}{\HoLogo@XeTeX}
%    \begin{macrocode}
\def\HoLogo@XeTeX#1{%
  \hologo{Xe}%
  \kern-.15em\relax
  \hologo{TeX}%
}
%    \end{macrocode}
%    \end{macro}
%
%    \begin{macro}{\HoLogoHtml@XeTeX}
%    \begin{macrocode}
\def\HoLogoHtml@XeTeX#1{%
  \HoLogoCss@XeTeX
  \HOLOGO@Span{XeTeX}{%
    \hologo{Xe}%
    \hologo{TeX}%
  }%
}
%    \end{macrocode}
%    \end{macro}
%    \begin{macro}{\HoLogoCss@XeTeX}
%    \begin{macrocode}
\def\HoLogoCss@XeTeX{%
  \Css{%
    span.HoLogo-XeTeX span.HoLogo-TeX{%
      margin-left:-.15em;%
    }%
  }%
  \global\let\HoLogoCss@XeTeX\relax
}
%    \end{macrocode}
%    \end{macro}
%
%    \begin{macro}{\HoLogo@XeLaTeX}
%    \begin{macrocode}
\def\HoLogo@XeLaTeX#1{%
  \hologo{Xe}%
  \kern-.13em%
  \hologo{LaTeX}%
}
%    \end{macrocode}
%    \end{macro}
%    \begin{macro}{\HoLogoHtml@XeLaTeX}
%    \begin{macrocode}
\def\HoLogoHtml@XeLaTeX#1{%
  \HoLogoCss@XeLaTeX
  \HOLOGO@Span{XeLaTeX}{%
    \hologo{Xe}%
    \hologo{LaTeX}%
  }%
}
%    \end{macrocode}
%    \end{macro}
%    \begin{macro}{\HoLogoCss@XeLaTeX}
%    \begin{macrocode}
\def\HoLogoCss@XeLaTeX{%
  \Css{%
    span.HoLogo-XeLaTeX span.HoLogo-Xe{%
      margin-right:-.13em;%
    }%
  }%
  \global\let\HoLogoCss@XeLaTeX\relax
}
%    \end{macrocode}
%    \end{macro}
%
% \subsubsection{\hologo{pdfTeX}, \hologo{pdfLaTeX}}
%
%    \begin{macro}{\HoLogo@pdfTeX}
%    \begin{macrocode}
\def\HoLogo@pdfTeX#1{%
  \HOLOGO@mbox{%
    #1{p}{P}df\hologo{TeX}%
  }%
}
%    \end{macrocode}
%    \end{macro}
%    \begin{macro}{\HoLogoCs@pdfTeX}
%    \begin{macrocode}
\def\HoLogoCs@pdfTeX#1{#1{p}{P}dfTeX}
%    \end{macrocode}
%    \end{macro}
%    \begin{macro}{\HoLogoBkm@pdfTeX}
%    \begin{macrocode}
\def\HoLogoBkm@pdfTeX#1{%
  #1{p}{P}df\hologo{TeX}%
}
%    \end{macrocode}
%    \end{macro}
%    \begin{macro}{\HoLogoHtml@pdfTeX}
%    \begin{macrocode}
\let\HoLogoHtml@pdfTeX\HoLogo@pdfTeX
%    \end{macrocode}
%    \end{macro}
%
%    \begin{macro}{\HoLogo@pdfLaTeX}
%    \begin{macrocode}
\def\HoLogo@pdfLaTeX#1{%
  \HOLOGO@mbox{%
    #1{p}{P}df\hologo{LaTeX}%
  }%
}
%    \end{macrocode}
%    \end{macro}
%    \begin{macro}{\HoLogoCs@pdfLaTeX}
%    \begin{macrocode}
\def\HoLogoCs@pdfLaTeX#1{#1{p}{P}dfLaTeX}
%    \end{macrocode}
%    \end{macro}
%    \begin{macro}{\HoLogoBkm@pdfLaTeX}
%    \begin{macrocode}
\def\HoLogoBkm@pdfLaTeX#1{%
  #1{p}{P}df\hologo{LaTeX}%
}
%    \end{macrocode}
%    \end{macro}
%    \begin{macro}{\HoLogoHtml@pdfLaTeX}
%    \begin{macrocode}
\let\HoLogoHtml@pdfLaTeX\HoLogo@pdfLaTeX
%    \end{macrocode}
%    \end{macro}
%
% \subsubsection{\hologo{VTeX}}
%
%    \begin{macro}{\HoLogo@VTeX}
%    \begin{macrocode}
\def\HoLogo@VTeX#1{%
  \HOLOGO@mbox{%
    V\hologo{TeX}%
  }%
}
%    \end{macrocode}
%    \end{macro}
%    \begin{macro}{\HoLogoHtml@VTeX}
%    \begin{macrocode}
\let\HoLogoHtml@VTeX\HoLogo@VTeX
%    \end{macrocode}
%    \end{macro}
%
% \subsubsection{\hologo{AmS}, \dots}
%
%    Source: class \xclass{amsdtx}
%
%    \begin{macro}{\HoLogo@AmS}
%    \begin{macrocode}
\def\HoLogo@AmS#1{%
  \HoLogoFont@font{AmS}{sy}{%
    A%
    \kern-.1667em%
    \lower.5ex\hbox{M}%
    \kern-.125em%
    S%
  }%
}
%    \end{macrocode}
%    \end{macro}
%    \begin{macro}{\HoLogoBkm@AmS}
%    \begin{macrocode}
\def\HoLogoBkm@AmS#1{AmS}
%    \end{macrocode}
%    \end{macro}
%    \begin{macro}{\HoLogoHtml@AmS}
%    \begin{macrocode}
\def\HoLogoHtml@AmS#1{%
  \HoLogoCss@AmS
%  \HoLogoFont@font{AmS}{sy}{%
    \HOLOGO@Span{AmS}{%
      A%
      \HOLOGO@Span{M}{M}%
      S%
    }%
%   }%
}
%    \end{macrocode}
%    \end{macro}
%    \begin{macro}{\HoLogoCss@AmS}
%    \begin{macrocode}
\def\HoLogoCss@AmS{%
  \Css{%
    span.HoLogo-AmS span.HoLogo-M{%
      position:relative;%
      top:.5ex;%
      margin-left:-.1667em;%
      margin-right:-.125em;%
      text-decoration:none;%
    }%
  }%
  \global\let\HoLogoCss@AmS\relax
}
%    \end{macrocode}
%    \end{macro}
%
%    \begin{macro}{\HoLogo@AmSTeX}
%    \begin{macrocode}
\def\HoLogo@AmSTeX#1{%
  \hologo{AmS}%
  \HOLOGO@hyphen
  \hologo{TeX}%
}
%    \end{macrocode}
%    \end{macro}
%    \begin{macro}{\HoLogoBkm@AmSTeX}
%    \begin{macrocode}
\def\HoLogoBkm@AmSTeX#1{AmS-TeX}%
%    \end{macrocode}
%    \end{macro}
%    \begin{macro}{\HoLogoHtml@AmSTeX}
%    \begin{macrocode}
\let\HoLogoHtml@AmSTeX\HoLogo@AmSTeX
%    \end{macrocode}
%    \end{macro}
%
%    \begin{macro}{\HoLogo@AmSLaTeX}
%    \begin{macrocode}
\def\HoLogo@AmSLaTeX#1{%
  \hologo{AmS}%
  \HOLOGO@hyphen
  \hologo{LaTeX}%
}
%    \end{macrocode}
%    \end{macro}
%    \begin{macro}{\HoLogoBkm@AmSLaTeX}
%    \begin{macrocode}
\def\HoLogoBkm@AmSLaTeX#1{AmS-LaTeX}%
%    \end{macrocode}
%    \end{macro}
%    \begin{macro}{\HoLogoHtml@AmSLaTeX}
%    \begin{macrocode}
\let\HoLogoHtml@AmSLaTeX\HoLogo@AmSLaTeX
%    \end{macrocode}
%    \end{macro}
%
% \subsubsection{\hologo{BibTeX}}
%
%    \begin{macro}{\HoLogo@BibTeX@sc}
%    A definition of \hologo{BibTeX} is provided in
%    the documentation source for the manual of \hologo{BibTeX}
%    \cite{btxdoc}.
%\begin{quote}
%\begin{verbatim}
%\def\BibTeX{%
%  {%
%    \rm
%    B%
%    \kern-.05em%
%    {%
%      \sc
%      i%
%      \kern-.025em %
%      b%
%    }%
%    \kern-.08em
%    T%
%    \kern-.1667em%
%    \lower.7ex\hbox{E}%
%    \kern-.125em%
%    X%
%  }%
%}
%\end{verbatim}
%\end{quote}
%    \begin{macrocode}
\def\HoLogo@BibTeX@sc#1{%
  B%
  \kern-.05em%
  \HoLogoFont@font{BibTeX}{sc}{%
    i%
    \kern-.025em%
    b%
  }%
  \HOLOGO@discretionary
  \kern-.08em%
  \hologo{TeX}%
}
%    \end{macrocode}
%    \end{macro}
%    \begin{macro}{\HoLogoHtml@BibTeX@sc}
%    \begin{macrocode}
\def\HoLogoHtml@BibTeX@sc#1{%
  \HoLogoCss@BibTeX@sc
  \HOLOGO@Span{BibTeX-sc}{%
    B%
    \HOLOGO@Span{i}{i}%
    \HOLOGO@Span{b}{b}%
    \hologo{TeX}%
  }%
}
%    \end{macrocode}
%    \end{macro}
%    \begin{macro}{\HoLogoCss@BibTeX@sc}
%    \begin{macrocode}
\def\HoLogoCss@BibTeX@sc{%
  \Css{%
    span.HoLogo-BibTeX-sc span.HoLogo-i{%
      margin-left:-.05em;%
      margin-right:-.025em;%
      font-variant:small-caps;%
    }%
  }%
  \Css{%
    span.HoLogo-BibTeX-sc span.HoLogo-b{%
      margin-right:-.08em;%
      font-variant:small-caps;%
    }%
  }%
  \global\let\HoLogoCss@BibTeX@sc\relax
}
%    \end{macrocode}
%    \end{macro}
%
%    \begin{macro}{\HoLogo@BibTeX@sf}
%    Variant \xoption{sf} avoids trouble with unavailable
%    small caps fonts (e.g., bold versions of Computer Modern or
%    Latin Modern). The definition is taken from
%    package \xpackage{dtklogos} \cite{dtklogos}.
%\begin{quote}
%\begin{verbatim}
%\DeclareRobustCommand{\BibTeX}{%
%  B%
%  \kern-.05em%
%  \hbox{%
%    $\m@th$% %% force math size calculations
%    \csname S@\f@size\endcsname
%    \fontsize\sf@size\z@
%    \math@fontsfalse
%    \selectfont
%    I%
%    \kern-.025em%
%    B
%  }%
%  \kern-.08em%
%  \-%
%  \TeX
%}
%\end{verbatim}
%\end{quote}
%    \begin{macrocode}
\def\HoLogo@BibTeX@sf#1{%
  B%
  \kern-.05em%
  \HoLogoFont@font{BibTeX}{bibsf}{%
    I%
    \kern-.025em%
    B%
  }%
  \HOLOGO@discretionary
  \kern-.08em%
  \hologo{TeX}%
}
%    \end{macrocode}
%    \end{macro}
%    \begin{macro}{\HoLogoHtml@BibTeX@sf}
%    \begin{macrocode}
\def\HoLogoHtml@BibTeX@sf#1{%
  \HoLogoCss@BibTeX@sf
  \HOLOGO@Span{BibTeX-sf}{%
    B%
    \HoLogoFont@font{BibTeX}{bibsf}{%
      \HOLOGO@Span{i}{I}%
      B%
    }%
    \hologo{TeX}%
  }%
}
%    \end{macrocode}
%    \end{macro}
%    \begin{macro}{\HoLogoCss@BibTeX@sf}
%    \begin{macrocode}
\def\HoLogoCss@BibTeX@sf{%
  \Css{%
    span.HoLogo-BibTeX-sf span.HoLogo-i{%
      margin-left:-.05em;%
      margin-right:-.025em;%
    }%
  }%
  \Css{%
    span.HoLogo-BibTeX-sf span.HoLogo-TeX{%
      margin-left:-.08em;%
    }%
  }%
  \global\let\HoLogoCss@BibTeX@sf\relax
}
%    \end{macrocode}
%    \end{macro}
%
%    \begin{macro}{\HoLogo@BibTeX}
%    \begin{macrocode}
\def\HoLogo@BibTeX{\HoLogo@BibTeX@sf}
%    \end{macrocode}
%    \end{macro}
%    \begin{macro}{\HoLogoHtml@BibTeX}
%    \begin{macrocode}
\def\HoLogoHtml@BibTeX{\HoLogoHtml@BibTeX@sf}
%    \end{macrocode}
%    \end{macro}
%
% \subsubsection{\hologo{BibTeX8}}
%
%    \begin{macro}{\HoLogo@BibTeX8}
%    \begin{macrocode}
\expandafter\def\csname HoLogo@BibTeX8\endcsname#1{%
  \hologo{BibTeX}%
  8%
}
%    \end{macrocode}
%    \end{macro}
%
%    \begin{macro}{\HoLogoBkm@BibTeX8}
%    \begin{macrocode}
\expandafter\def\csname HoLogoBkm@BibTeX8\endcsname#1{%
  \hologo{BibTeX}%
  8%
}
%    \end{macrocode}
%    \end{macro}
%    \begin{macro}{\HoLogoHtml@BibTeX8}
%    \begin{macrocode}
\expandafter
\let\csname HoLogoHtml@BibTeX8\expandafter\endcsname
\csname HoLogo@BibTeX8\endcsname
%    \end{macrocode}
%    \end{macro}
%
% \subsubsection{\hologo{ConTeXt}}
%
%    \begin{macro}{\HoLogo@ConTeXt@simple}
%    \begin{macrocode}
\def\HoLogo@ConTeXt@simple#1{%
  \HOLOGO@mbox{Con}%
  \HOLOGO@discretionary
  \HOLOGO@mbox{\hologo{TeX}t}%
}
%    \end{macrocode}
%    \end{macro}
%    \begin{macro}{\HoLogoHtml@ConTeXt@simple}
%    \begin{macrocode}
\let\HoLogoHtml@ConTeXt@simple\HoLogo@ConTeXt@simple
%    \end{macrocode}
%    \end{macro}
%
%    \begin{macro}{\HoLogo@ConTeXt@narrow}
%    This definition of logo \hologo{ConTeXt} with variant \xoption{narrow}
%    comes from TUGboat's class \xclass{ltugboat} (version 2010/11/15 v2.8).
%    \begin{macrocode}
\def\HoLogo@ConTeXt@narrow#1{%
  \HOLOGO@mbox{C\kern-.0333emon}%
  \HOLOGO@discretionary
  \kern-.0667em%
  \HOLOGO@mbox{\hologo{TeX}\kern-.0333emt}%
}
%    \end{macrocode}
%    \end{macro}
%    \begin{macro}{\HoLogoHtml@ConTeXt@narrow}
%    \begin{macrocode}
\def\HoLogoHtml@ConTeXt@narrow#1{%
  \HoLogoCss@ConTeXt@narrow
  \HOLOGO@Span{ConTeXt-narrow}{%
    \HOLOGO@Span{C}{C}%
    on%
    \hologo{TeX}%
    t%
  }%
}
%    \end{macrocode}
%    \end{macro}
%    \begin{macro}{\HoLogoCss@ConTeXt@narrow}
%    \begin{macrocode}
\def\HoLogoCss@ConTeXt@narrow{%
  \Css{%
    span.HoLogo-ConTeXt-narrow span.HoLogo-C{%
      margin-left:-.0333em;%
    }%
  }%
  \Css{%
    span.HoLogo-ConTeXt-narrow span.HoLogo-TeX{%
      margin-left:-.0667em;%
      margin-right:-.0333em;%
    }%
  }%
  \global\let\HoLogoCss@ConTeXt@narrow\relax
}
%    \end{macrocode}
%    \end{macro}
%
%    \begin{macro}{\HoLogo@ConTeXt}
%    \begin{macrocode}
\def\HoLogo@ConTeXt{\HoLogo@ConTeXt@narrow}
%    \end{macrocode}
%    \end{macro}
%    \begin{macro}{\HoLogoHtml@ConTeXt}
%    \begin{macrocode}
\def\HoLogoHtml@ConTeXt{\HoLogoHtml@ConTeXt@narrow}
%    \end{macrocode}
%    \end{macro}
%
% \subsubsection{\hologo{emTeX}}
%
%    \begin{macro}{\HoLogo@emTeX}
%    \begin{macrocode}
\def\HoLogo@emTeX#1{%
  \HOLOGO@mbox{#1{e}{E}m}%
  \HOLOGO@discretionary
  \hologo{TeX}%
}
%    \end{macrocode}
%    \end{macro}
%    \begin{macro}{\HoLogoCs@emTeX}
%    \begin{macrocode}
\def\HoLogoCs@emTeX#1{#1{e}{E}mTeX}%
%    \end{macrocode}
%    \end{macro}
%    \begin{macro}{\HoLogoBkm@emTeX}
%    \begin{macrocode}
\def\HoLogoBkm@emTeX#1{%
  #1{e}{E}m\hologo{TeX}%
}
%    \end{macrocode}
%    \end{macro}
%    \begin{macro}{\HoLogoHtml@emTeX}
%    \begin{macrocode}
\let\HoLogoHtml@emTeX\HoLogo@emTeX
%    \end{macrocode}
%    \end{macro}
%
% \subsubsection{\hologo{ExTeX}}
%
%    \begin{macro}{\HoLogo@ExTeX}
%    The definition is taken from the FAQ of the
%    project \hologo{ExTeX}
%    \cite{ExTeX-FAQ}.
%\begin{quote}
%\begin{verbatim}
%\def\ExTeX{%
%  \textrm{% Logo always with serifs
%    \ensuremath{%
%      \textstyle
%      \varepsilon_{%
%        \kern-0.15em%
%        \mathcal{X}%
%      }%
%    }%
%    \kern-.15em%
%    \TeX
%  }%
%}
%\end{verbatim}
%\end{quote}
%    \begin{macrocode}
\def\HoLogo@ExTeX#1{%
  \HoLogoFont@font{ExTeX}{rm}{%
    \ltx@mbox{%
      \HOLOGO@MathSetup
      $%
        \textstyle
        \varepsilon_{%
          \kern-0.15em%
          \HoLogoFont@font{ExTeX}{sy}{X}%
        }%
      $%
    }%
    \HOLOGO@discretionary
    \kern-.15em%
    \hologo{TeX}%
  }%
}
%    \end{macrocode}
%    \end{macro}
%    \begin{macro}{\HoLogoHtml@ExTeX}
%    \begin{macrocode}
\def\HoLogoHtml@ExTeX#1{%
  \HoLogoCss@ExTeX
  \HoLogoFont@font{ExTeX}{rm}{%
    \HOLOGO@Span{ExTeX}{%
      \ltx@mbox{%
        \HOLOGO@MathSetup
        $\textstyle\varepsilon$%
        \HOLOGO@Span{X}{$\textstyle\chi$}%
        \hologo{TeX}%
      }%
    }%
  }%
}
%    \end{macrocode}
%    \end{macro}
%    \begin{macro}{\HoLogoBkm@ExTeX}
%    \begin{macrocode}
\def\HoLogoBkm@ExTeX#1{%
  \HOLOGO@PdfdocUnicode{#1{e}{E}x}{\textepsilon\textchi}%
  \hologo{TeX}%
}
%    \end{macrocode}
%    \end{macro}
%    \begin{macro}{\HoLogoCss@ExTeX}
%    \begin{macrocode}
\def\HoLogoCss@ExTeX{%
  \Css{%
    span.HoLogo-ExTeX{%
      font-family:serif;%
    }%
  }%
  \Css{%
    span.HoLogo-ExTeX span.HoLogo-TeX{%
      margin-left:-.15em;%
    }%
  }%
  \global\let\HoLogoCss@ExTeX\relax
}
%    \end{macrocode}
%    \end{macro}
%
% \subsubsection{\hologo{MiKTeX}}
%
%    \begin{macro}{\HoLogo@MiKTeX}
%    \begin{macrocode}
\def\HoLogo@MiKTeX#1{%
  \HOLOGO@mbox{MiK}%
  \HOLOGO@discretionary
  \hologo{TeX}%
}
%    \end{macrocode}
%    \end{macro}
%    \begin{macro}{\HoLogoHtml@MiKTeX}
%    \begin{macrocode}
\let\HoLogoHtml@MiKTeX\HoLogo@MiKTeX
%    \end{macrocode}
%    \end{macro}
%
% \subsubsection{\hologo{OzTeX} and friends}
%
%    Source: \hologo{OzTeX} FAQ \cite{OzTeX}:
%    \begin{quote}
%      |\def\OzTeX{O\kern-.03em z\kern-.15em\TeX}|\\
%      (There is no kerning in OzMF, OzMP and OzTtH.)
%    \end{quote}
%
%    \begin{macro}{\HoLogo@OzTeX}
%    \begin{macrocode}
\def\HoLogo@OzTeX#1{%
  O%
  \kern-.03em %
  z%
  \kern-.15em %
  \hologo{TeX}%
}
%    \end{macrocode}
%    \end{macro}
%    \begin{macro}{\HoLogoHtml@OzTeX}
%    \begin{macrocode}
\def\HoLogoHtml@OzTeX#1{%
  \HoLogoCss@OzTeX
  \HOLOGO@Span{OzTeX}{%
    O%
    \HOLOGO@Span{z}{z}%
    \hologo{TeX}%
  }%
}
%    \end{macrocode}
%    \end{macro}
%    \begin{macro}{\HoLogoCss@OzTeX}
%    \begin{macrocode}
\def\HoLogoCss@OzTeX{%
  \Css{%
    span.HoLogo-OzTeX span.HoLogo-z{%
      margin-left:-.03em;%
      margin-right:-.15em;%
    }%
  }%
  \global\let\HoLogoCss@OzTeX\relax
}
%    \end{macrocode}
%    \end{macro}
%
%    \begin{macro}{\HoLogo@OzMF}
%    \begin{macrocode}
\def\HoLogo@OzMF#1{%
  \HOLOGO@mbox{OzMF}%
}
%    \end{macrocode}
%    \end{macro}
%    \begin{macro}{\HoLogo@OzMP}
%    \begin{macrocode}
\def\HoLogo@OzMP#1{%
  \HOLOGO@mbox{OzMP}%
}
%    \end{macrocode}
%    \end{macro}
%    \begin{macro}{\HoLogo@OzTtH}
%    \begin{macrocode}
\def\HoLogo@OzTtH#1{%
  \HOLOGO@mbox{OzTtH}%
}
%    \end{macrocode}
%    \end{macro}
%
% \subsubsection{\hologo{PCTeX}}
%
%    \begin{macro}{\HoLogo@PCTeX}
%    \begin{macrocode}
\def\HoLogo@PCTeX#1{%
  \HOLOGO@mbox{PC}%
  \hologo{TeX}%
}
%    \end{macrocode}
%    \end{macro}
%    \begin{macro}{\HoLogoHtml@PCTeX}
%    \begin{macrocode}
\let\HoLogoHtml@PCTeX\HoLogo@PCTeX
%    \end{macrocode}
%    \end{macro}
%
% \subsubsection{\hologo{PiCTeX}}
%
%    The original definitions from \xfile{pictex.tex} \cite{PiCTeX}:
%\begin{quote}
%\begin{verbatim}
%\def\PiC{%
%  P%
%  \kern-.12em%
%  \lower.5ex\hbox{I}%
%  \kern-.075em%
%  C%
%}
%\def\PiCTeX{%
%  \PiC
%  \kern-.11em%
%  \TeX
%}
%\end{verbatim}
%\end{quote}
%
%    \begin{macro}{\HoLogo@PiC}
%    \begin{macrocode}
\def\HoLogo@PiC#1{%
  P%
  \kern-.12em%
  \lower.5ex\hbox{I}%
  \kern-.075em%
  C%
  \HOLOGO@SpaceFactor
}
%    \end{macrocode}
%    \end{macro}
%    \begin{macro}{\HoLogoHtml@PiC}
%    \begin{macrocode}
\def\HoLogoHtml@PiC#1{%
  \HoLogoCss@PiC
  \HOLOGO@Span{PiC}{%
    P%
    \HOLOGO@Span{i}{I}%
    C%
  }%
}
%    \end{macrocode}
%    \end{macro}
%    \begin{macro}{\HoLogoCss@PiC}
%    \begin{macrocode}
\def\HoLogoCss@PiC{%
  \Css{%
    span.HoLogo-PiC span.HoLogo-i{%
      position:relative;%
      top:.5ex;%
      margin-left:-.12em;%
      margin-right:-.075em;%
      text-decoration:none;%
    }%
  }%
  \global\let\HoLogoCss@PiC\relax
}
%    \end{macrocode}
%    \end{macro}
%
%    \begin{macro}{\HoLogo@PiCTeX}
%    \begin{macrocode}
\def\HoLogo@PiCTeX#1{%
  \hologo{PiC}%
  \HOLOGO@discretionary
  \kern-.11em%
  \hologo{TeX}%
}
%    \end{macrocode}
%    \end{macro}
%    \begin{macro}{\HoLogoHtml@PiCTeX}
%    \begin{macrocode}
\def\HoLogoHtml@PiCTeX#1{%
  \HoLogoCss@PiCTeX
  \HOLOGO@Span{PiCTeX}{%
    \hologo{PiC}%
    \hologo{TeX}%
  }%
}
%    \end{macrocode}
%    \end{macro}
%    \begin{macro}{\HoLogoCss@PiCTeX}
%    \begin{macrocode}
\def\HoLogoCss@PiCTeX{%
  \Css{%
    span.HoLogo-PiCTeX span.HoLogo-PiC{%
      margin-right:-.11em;%
    }%
  }%
  \global\let\HoLogoCss@PiCTeX\relax
}
%    \end{macrocode}
%    \end{macro}
%
% \subsubsection{\hologo{teTeX}}
%
%    \begin{macro}{\HoLogo@teTeX}
%    \begin{macrocode}
\def\HoLogo@teTeX#1{%
  \HOLOGO@mbox{#1{t}{T}e}%
  \HOLOGO@discretionary
  \hologo{TeX}%
}
%    \end{macrocode}
%    \end{macro}
%    \begin{macro}{\HoLogoCs@teTeX}
%    \begin{macrocode}
\def\HoLogoCs@teTeX#1{#1{t}{T}dfTeX}
%    \end{macrocode}
%    \end{macro}
%    \begin{macro}{\HoLogoBkm@teTeX}
%    \begin{macrocode}
\def\HoLogoBkm@teTeX#1{%
  #1{t}{T}e\hologo{TeX}%
}
%    \end{macrocode}
%    \end{macro}
%    \begin{macro}{\HoLogoHtml@teTeX}
%    \begin{macrocode}
\let\HoLogoHtml@teTeX\HoLogo@teTeX
%    \end{macrocode}
%    \end{macro}
%
% \subsubsection{\hologo{TeX4ht}}
%
%    \begin{macro}{\HoLogo@TeX4ht}
%    \begin{macrocode}
\expandafter\def\csname HoLogo@TeX4ht\endcsname#1{%
  \HOLOGO@mbox{\hologo{TeX}4ht}%
}
%    \end{macrocode}
%    \end{macro}
%    \begin{macro}{\HoLogoHtml@TeX4ht}
%    \begin{macrocode}
\expandafter
\let\csname HoLogoHtml@TeX4ht\expandafter\endcsname
\csname HoLogo@TeX4ht\endcsname
%    \end{macrocode}
%    \end{macro}
%
%
% \subsubsection{\hologo{SageTeX}}
%
%    \begin{macro}{\HoLogo@SageTeX}
%    \begin{macrocode}
\def\HoLogo@SageTeX#1{%
  \HOLOGO@mbox{Sage}%
  \HOLOGO@discretionary
  \HOLOGO@NegativeKerning{eT,oT,To}%
  \hologo{TeX}%
}
%    \end{macrocode}
%    \end{macro}
%    \begin{macro}{\HoLogoHtml@SageTeX}
%    \begin{macrocode}
\let\HoLogoHtml@SageTeX\HoLogo@SageTeX
%    \end{macrocode}
%    \end{macro}
%
% \subsection{\hologo{METAFONT} and friends}
%
%    \begin{macro}{\HoLogo@METAFONT}
%    \begin{macrocode}
\def\HoLogo@METAFONT#1{%
  \HoLogoFont@font{METAFONT}{logo}{%
    \HOLOGO@mbox{META}%
    \HOLOGO@discretionary
    \HOLOGO@mbox{FONT}%
  }%
}
%    \end{macrocode}
%    \end{macro}
%
%    \begin{macro}{\HoLogo@METAPOST}
%    \begin{macrocode}
\def\HoLogo@METAPOST#1{%
  \HoLogoFont@font{METAPOST}{logo}{%
    \HOLOGO@mbox{META}%
    \HOLOGO@discretionary
    \HOLOGO@mbox{POST}%
  }%
}
%    \end{macrocode}
%    \end{macro}
%
%    \begin{macro}{\HoLogo@MetaFun}
%    \begin{macrocode}
\def\HoLogo@MetaFun#1{%
  \HOLOGO@mbox{Meta}%
  \HOLOGO@discretionary
  \HOLOGO@mbox{Fun}%
}
%    \end{macrocode}
%    \end{macro}
%
%    \begin{macro}{\HoLogo@MetaPost}
%    \begin{macrocode}
\def\HoLogo@MetaPost#1{%
  \HOLOGO@mbox{Meta}%
  \HOLOGO@discretionary
  \HOLOGO@mbox{Post}%
}
%    \end{macrocode}
%    \end{macro}
%
% \subsection{Others}
%
% \subsubsection{\hologo{biber}}
%
%    \begin{macro}{\HoLogo@biber}
%    \begin{macrocode}
\def\HoLogo@biber#1{%
  \HOLOGO@mbox{#1{b}{B}i}%
  \HOLOGO@discretionary
  \HOLOGO@mbox{ber}%
}
%    \end{macrocode}
%    \end{macro}
%    \begin{macro}{\HoLogoCs@biber}
%    \begin{macrocode}
\def\HoLogoCs@biber#1{#1{b}{B}iber}
%    \end{macrocode}
%    \end{macro}
%    \begin{macro}{\HoLogoBkm@biber}
%    \begin{macrocode}
\def\HoLogoBkm@biber#1{%
  #1{b}{B}iber%
}
%    \end{macrocode}
%    \end{macro}
%    \begin{macro}{\HoLogoHtml@biber}
%    \begin{macrocode}
\let\HoLogoHtml@biber\HoLogo@biber
%    \end{macrocode}
%    \end{macro}
%
% \subsubsection{\hologo{KOMAScript}}
%
%    \begin{macro}{\HoLogo@KOMAScript}
%    The definition for \hologo{KOMAScript} is taken
%    from \hologo{KOMAScript} (\xfile{scrlogo.dtx}, reformatted) \cite{scrlogo}:
%\begin{quote}
%\begin{verbatim}
%\@ifundefined{KOMAScript}{%
%  \DeclareRobustCommand{\KOMAScript}{%
%    \textsf{%
%      K\kern.05em O\kern.05emM\kern.05em A%
%      \kern.1em-\kern.1em %
%      Script%
%    }%
%  }%
%}{}
%\end{verbatim}
%\end{quote}
%    \begin{macrocode}
\def\HoLogo@KOMAScript#1{%
  \HoLogoFont@font{KOMAScript}{sf}{%
    \HOLOGO@mbox{%
      K\kern.05em%
      O\kern.05em%
      M\kern.05em%
      A%
    }%
    \kern.1em%
    \HOLOGO@hyphen
    \kern.1em%
    \HOLOGO@mbox{Script}%
  }%
}
%    \end{macrocode}
%    \end{macro}
%    \begin{macro}{\HoLogoBkm@KOMAScript}
%    \begin{macrocode}
\def\HoLogoBkm@KOMAScript#1{%
  KOMA-Script%
}
%    \end{macrocode}
%    \end{macro}
%    \begin{macro}{\HoLogoHtml@KOMAScript}
%    \begin{macrocode}
\def\HoLogoHtml@KOMAScript#1{%
  \HoLogoCss@KOMAScript
  \HoLogoFont@font{KOMAScript}{sf}{%
    \HOLOGO@Span{KOMAScript}{%
      K%
      \HOLOGO@Span{O}{O}%
      M%
      \HOLOGO@Span{A}{A}%
      \HOLOGO@Span{hyphen}{-}%
      Script%
    }%
  }%
}
%    \end{macrocode}
%    \end{macro}
%    \begin{macro}{\HoLogoCss@KOMAScript}
%    \begin{macrocode}
\def\HoLogoCss@KOMAScript{%
  \Css{%
    span.HoLogo-KOMAScript{%
      font-family:sans-serif;%
    }%
  }%
  \Css{%
    span.HoLogo-KOMAScript span.HoLogo-O{%
      padding-left:.05em;%
      padding-right:.05em;%
    }%
  }%
  \Css{%
    span.HoLogo-KOMAScript span.HoLogo-A{%
      padding-left:.05em;%
    }%
  }%
  \Css{%
    span.HoLogo-KOMAScript span.HoLogo-hyphen{%
      padding-left:.1em;%
      padding-right:.1em;%
    }%
  }%
  \global\let\HoLogoCss@KOMAScript\relax
}
%    \end{macrocode}
%    \end{macro}
%
% \subsubsection{\hologo{LyX}}
%
%    \begin{macro}{\HoLogo@LyX}
%    The definition is taken from the documentation source files
%    of \hologo{LyX}, \xfile{Intro.lyx} \cite{LyX}:
%\begin{quote}
%\begin{verbatim}
%\def\LyX{%
%  \texorpdfstring{%
%    L\kern-.1667em\lower.25em\hbox{Y}\kern-.125emX\@%
%  }{%
%    LyX%
%  }%
%}
%\end{verbatim}
%\end{quote}
%    \begin{macrocode}
\def\HoLogo@LyX#1{%
  L%
  \kern-.1667em%
  \lower.25em\hbox{Y}%
  \kern-.125em%
  X%
  \HOLOGO@SpaceFactor
}
%    \end{macrocode}
%    \end{macro}
%    \begin{macro}{\HoLogoHtml@LyX}
%    \begin{macrocode}
\def\HoLogoHtml@LyX#1{%
  \HoLogoCss@LyX
  \HOLOGO@Span{LyX}{%
    L%
    \HOLOGO@Span{y}{Y}%
    X%
  }%
}
%    \end{macrocode}
%    \end{macro}
%    \begin{macro}{\HoLogoCss@LyX}
%    \begin{macrocode}
\def\HoLogoCss@LyX{%
  \Css{%
    span.HoLogo-LyX span.HoLogo-y{%
      position:relative;%
      top:.25em;%
      margin-left:-.1667em;%
      margin-right:-.125em;%
      text-decoration:none;%
    }%
  }%
  \global\let\HoLogoCss@LyX\relax
}
%    \end{macrocode}
%    \end{macro}
%
% \subsubsection{\hologo{NTS}}
%
%    \begin{macro}{\HoLogo@NTS}
%    Definition for \hologo{NTS} can be found in
%    package \xpackage{etex\textunderscore man} for the \hologo{eTeX} manual \cite{etexman}
%    and in package \xpackage{dtklogos} \cite{dtklogos}:
%\begin{quote}
%\begin{verbatim}
%\def\NTS{%
%  \leavevmode
%  \hbox{%
%    $%
%      \cal N%
%      \kern-0.35em%
%      \lower0.5ex\hbox{$\cal T$}%
%      \kern-0.2em%
%      S%
%    $%
%  }%
%}
%\end{verbatim}
%\end{quote}
%    \begin{macrocode}
\def\HoLogo@NTS#1{%
  \HoLogoFont@font{NTS}{sy}{%
    N\/%
    \kern-.35em%
    \lower.5ex\hbox{T\/}%
    \kern-.2em%
    S\/%
  }%
  \HOLOGO@SpaceFactor
}
%    \end{macrocode}
%    \end{macro}
%
% \subsubsection{\Hologo{TTH} (\hologo{TeX} to HTML translator)}
%
%    Source: \url{http://hutchinson.belmont.ma.us/tth/}
%    In the HTML source the second `T' is printed as subscript.
%\begin{quote}
%\begin{verbatim}
%T<sub>T</sub>H
%\end{verbatim}
%\end{quote}
%    \begin{macro}{\HoLogo@TTH}
%    \begin{macrocode}
\def\HoLogo@TTH#1{%
  \ltx@mbox{%
    T\HOLOGO@SubScript{T}H%
  }%
  \HOLOGO@SpaceFactor
}
%    \end{macrocode}
%    \end{macro}
%
%    \begin{macro}{\HoLogoHtml@TTH}
%    \begin{macrocode}
\def\HoLogoHtml@TTH#1{%
  T\HCode{<sub>}T\HCode{</sub>}H%
}
%    \end{macrocode}
%    \end{macro}
%
% \subsubsection{\Hologo{HanTheThanh}}
%
%    Partial source: Package \xpackage{dtklogos}.
%    The double accent is U+1EBF (latin small letter e with circumflex
%    and acute).
%    \begin{macro}{\HoLogo@HanTheThanh}
%    \begin{macrocode}
\def\HoLogo@HanTheThanh#1{%
  \ltx@mbox{H\`an}%
  \HOLOGO@space
  \ltx@mbox{%
    Th%
    \HOLOGO@IfCharExists{"1EBF}{%
      \char"1EBF\relax
    }{%
      \^e\hbox to 0pt{\hss\raise .5ex\hbox{\'{}}}%
    }%
  }%
  \HOLOGO@space
  \ltx@mbox{Th\`anh}%
}
%    \end{macrocode}
%    \end{macro}
%    \begin{macro}{\HoLogoBkm@HanTheThanh}
%    \begin{macrocode}
\def\HoLogoBkm@HanTheThanh#1{%
  H\`an %
  Th\HOLOGO@PdfdocUnicode{\^e}{\9036\277} %
  Th\`anh%
}
%    \end{macrocode}
%    \end{macro}
%    \begin{macro}{\HoLogoHtml@HanTheThanh}
%    \begin{macrocode}
\def\HoLogoHtml@HanTheThanh#1{%
  H\`an %
  Th\HCode{&\ltx@hashchar x1ebf;} %
  Th\`anh%
}
%    \end{macrocode}
%    \end{macro}
%
% \subsection{Driver detection}
%
%    \begin{macrocode}
\HOLOGO@IfExists\InputIfFileExists{%
  \InputIfFileExists{hologo.cfg}{}{}%
}{%
  \ltx@IfUndefined{pdf@filesize}{%
    \def\HOLOGO@InputIfExists{%
      \openin\HOLOGO@temp=hologo.cfg\relax
      \ifeof\HOLOGO@temp
        \closein\HOLOGO@temp
      \else
        \closein\HOLOGO@temp
        \begingroup
          \def\x{LaTeX2e}%
        \expandafter\endgroup
        \ifx\fmtname\x
          \input{hologo.cfg}%
        \else
          \input hologo.cfg\relax
        \fi
      \fi
    }%
    \ltx@IfUndefined{newread}{%
      \chardef\HOLOGO@temp=15 %
      \def\HOLOGO@CheckRead{%
        \ifeof\HOLOGO@temp
          \HOLOGO@InputIfExists
        \else
          \ifcase\HOLOGO@temp
            \@PackageWarningNoLine{hologo}{%
              Configuration file ignored, because\MessageBreak
              a free read register could not be found%
            }%
          \else
            \begingroup
              \count\ltx@cclv=\HOLOGO@temp
              \advance\ltx@cclv by \ltx@minusone
              \edef\x{\endgroup
                \chardef\noexpand\HOLOGO@temp=\the\count\ltx@cclv
                \relax
              }%
            \x
          \fi
        \fi
      }%
    }{%
      \csname newread\endcsname\HOLOGO@temp
      \HOLOGO@InputIfExists
    }%
  }{%
    \edef\HOLOGO@temp{\pdf@filesize{hologo.cfg}}%
    \ifx\HOLOGO@temp\ltx@empty
    \else
      \ifnum\HOLOGO@temp>0 %
        \begingroup
          \def\x{LaTeX2e}%
        \expandafter\endgroup
        \ifx\fmtname\x
          \input{hologo.cfg}%
        \else
          \input hologo.cfg\relax
        \fi
      \else
        \@PackageInfoNoLine{hologo}{%
          Empty configuration file `hologo.cfg' ignored%
        }%
      \fi
    \fi
  }%
}
%    \end{macrocode}
%
%    \begin{macrocode}
\def\HOLOGO@temp#1#2{%
  \kv@define@key{HoLogoDriver}{#1}[]{%
    \begingroup
      \def\HOLOGO@temp{##1}%
      \ltx@onelevel@sanitize\HOLOGO@temp
      \ifx\HOLOGO@temp\ltx@empty
      \else
        \@PackageError{hologo}{%
          Value (\HOLOGO@temp) not permitted for option `#1'%
        }%
        \@ehc
      \fi
    \endgroup
    \def\hologoDriver{#2}%
  }%
}%
\def\HOLOGO@@temp#1#2{%
  \ifx\kv@value\relax
    \HOLOGO@temp{#1}{#1}%
  \else
    \HOLOGO@temp{#1}{#2}%
  \fi
}%
\kv@parse@normalized{%
  pdftex,%
  luatex=pdftex,%
  dvipdfm,%
  dvipdfmx=dvipdfm,%
  dvips,%
  dvipsone=dvips,%
  xdvi=dvips,%
  xetex,%
  vtex,%
}\HOLOGO@@temp
%    \end{macrocode}
%
%    \begin{macrocode}
\kv@define@key{HoLogoDriver}{driverfallback}{%
  \def\HOLOGO@DriverFallback{#1}%
}
%    \end{macrocode}
%
%    \begin{macro}{\HOLOGO@DriverFallback}
%    \begin{macrocode}
\def\HOLOGO@DriverFallback{dvips}
%    \end{macrocode}
%    \end{macro}
%
%    \begin{macro}{\hologoDriverSetup}
%    \begin{macrocode}
\def\hologoDriverSetup{%
  \let\hologoDriver\ltx@undefined
  \HOLOGO@DriverSetup
}
%    \end{macrocode}
%    \end{macro}
%
%    \begin{macro}{\HOLOGO@DriverSetup}
%    \begin{macrocode}
\def\HOLOGO@DriverSetup#1{%
  \kvsetkeys{HoLogoDriver}{#1}%
  \HOLOGO@CheckDriver
  \ltx@ifundefined{hologoDriver}{%
    \begingroup
    \edef\x{\endgroup
      \noexpand\kvsetkeys{HoLogoDriver}{\HOLOGO@DriverFallback}%
    }\x
  }{}%
  \@PackageInfoNoLine{hologo}{Using driver `\hologoDriver'}%
}
%    \end{macrocode}
%    \end{macro}
%
%    \begin{macro}{\HOLOGO@CheckDriver}
%    \begin{macrocode}
\def\HOLOGO@CheckDriver{%
  \ifpdf
    \def\hologoDriver{pdftex}%
    \let\HOLOGO@pdfliteral\pdfliteral
    \ifluatex
      \ifx\pdfextension\@undefined\else
        \protected\def\pdfliteral{\pdfextension literal}%
        \let\HOLOGO@pdfliteral\pdfliteral
      \fi
      \ltx@IfUndefined{HOLOGO@pdfliteral}{%
        \ifnum\luatexversion<36 %
        \else
          \begingroup
            \let\HOLOGO@temp\endgroup
            \ifcase0%
                \directlua{%
                  if tex.enableprimitives then %
                    tex.enableprimitives('HOLOGO@', {'pdfliteral'})%
                  else %
                    tex.print('1')%
                  end%
                }%
                \ifx\HOLOGO@pdfliteral\@undefined 1\fi%
                \relax%
              \endgroup
              \let\HOLOGO@temp\relax
              \global\let\HOLOGO@pdfliteral\HOLOGO@pdfliteral
            \fi%
          \HOLOGO@temp
        \fi
      }{}%
    \fi
    \ltx@IfUndefined{HOLOGO@pdfliteral}{%
      \@PackageWarningNoLine{hologo}{%
        Cannot find \string\pdfliteral
      }%
    }{}%
  \else
    \ifxetex
      \def\hologoDriver{xetex}%
    \else
      \ifvtex
        \def\hologoDriver{vtex}%
      \fi
    \fi
  \fi
}
%    \end{macrocode}
%    \end{macro}
%
%    \begin{macro}{\HOLOGO@WarningUnsupportedDriver}
%    \begin{macrocode}
\def\HOLOGO@WarningUnsupportedDriver#1{%
  \@PackageWarningNoLine{hologo}{%
    Logo `#1' needs driver specific macros,\MessageBreak
    but driver `\hologoDriver' is not supported.\MessageBreak
    Use a different driver or\MessageBreak
    load package `graphics' or `pgf'%
  }%
}
%    \end{macrocode}
%    \end{macro}
%
% \subsubsection{Reflect box macros}
%
%    Skip driver part if not needed.
%    \begin{macrocode}
\ltx@IfUndefined{reflectbox}{}{%
  \ltx@IfUndefined{rotatebox}{}{%
    \HOLOGO@AtEnd
  }%
}
\ltx@IfUndefined{pgftext}{}{%
  \HOLOGO@AtEnd
}
\ltx@IfUndefined{psscalebox}{}{%
  \HOLOGO@AtEnd
}
%    \end{macrocode}
%
%    \begin{macrocode}
\def\HOLOGO@temp{LaTeX2e}
\ifx\fmtname\HOLOGO@temp
  \RequirePackage{kvoptions}[2011/06/30]%
  \ProcessKeyvalOptions{HoLogoDriver}%
\fi
\HOLOGO@DriverSetup{}
%    \end{macrocode}
%
%    \begin{macro}{\HOLOGO@ReflectBox}
%    \begin{macrocode}
\def\HOLOGO@ReflectBox#1{%
  \begingroup
    \setbox\ltx@zero\hbox{\begingroup#1\endgroup}%
    \setbox\ltx@two\hbox{%
      \kern\wd\ltx@zero
      \csname HOLOGO@ScaleBox@\hologoDriver\endcsname{-1}{1}{%
        \hbox to 0pt{\copy\ltx@zero\hss}%
      }%
    }%
    \wd\ltx@two=\wd\ltx@zero
    \box\ltx@two
  \endgroup
}
%    \end{macrocode}
%    \end{macro}
%
%    \begin{macro}{\HOLOGO@PointReflectBox}
%    \begin{macrocode}
\def\HOLOGO@PointReflectBox#1{%
  \begingroup
    \setbox\ltx@zero\hbox{\begingroup#1\endgroup}%
    \setbox\ltx@two\hbox{%
      \kern\wd\ltx@zero
      \raise\ht\ltx@zero\hbox{%
        \csname HOLOGO@ScaleBox@\hologoDriver\endcsname{-1}{-1}{%
          \hbox to 0pt{\copy\ltx@zero\hss}%
        }%
      }%
    }%
    \wd\ltx@two=\wd\ltx@zero
    \box\ltx@two
  \endgroup
}
%    \end{macrocode}
%    \end{macro}
%
%    We must define all variants because of dynamic driver setup.
%    \begin{macrocode}
\def\HOLOGO@temp#1#2{#2}
%    \end{macrocode}
%
%    \begin{macro}{\HOLOGO@ScaleBox@pdftex}
%    \begin{macrocode}
\HOLOGO@temp{pdftex}{%
  \def\HOLOGO@ScaleBox@pdftex#1#2#3{%
    \HOLOGO@pdfliteral{%
      q #1 0 0 #2 0 0 cm%
    }%
    #3%
    \HOLOGO@pdfliteral{%
      Q%
    }%
  }%
}
%    \end{macrocode}
%    \end{macro}
%    \begin{macro}{\HOLOGO@ScaleBox@dvips}
%    \begin{macrocode}
\HOLOGO@temp{dvips}{%
  \def\HOLOGO@ScaleBox@dvips#1#2#3{%
    \special{ps:%
      gsave %
      currentpoint %
      currentpoint translate %
      #1 #2 scale %
      neg exch neg exch translate%
    }%
    #3%
    \special{ps:%
      currentpoint %
      grestore %
      moveto%
    }%
  }%
}
%    \end{macrocode}
%    \end{macro}
%    \begin{macro}{\HOLOGO@ScaleBox@dvipdfm}
%    \begin{macrocode}
\HOLOGO@temp{dvipdfm}{%
  \let\HOLOGO@ScaleBox@dvipdfm\HOLOGO@ScaleBox@dvips
}
%    \end{macrocode}
%    \end{macro}
%    Since \hologo{XeTeX} v0.6.
%    \begin{macro}{\HOLOGO@ScaleBox@xetex}
%    \begin{macrocode}
\HOLOGO@temp{xetex}{%
  \def\HOLOGO@ScaleBox@xetex#1#2#3{%
    \special{x:gsave}%
    \special{x:scale #1 #2}%
    #3%
    \special{x:grestore}%
  }%
}
%    \end{macrocode}
%    \end{macro}
%    \begin{macro}{\HOLOGO@ScaleBox@vtex}
%    \begin{macrocode}
\HOLOGO@temp{vtex}{%
  \def\HOLOGO@ScaleBox@vtex#1#2#3{%
    \special{r(#1,0,0,#2,0,0}%
    #3%
    \special{r)}%
  }%
}
%    \end{macrocode}
%    \end{macro}
%
%    \begin{macrocode}
\HOLOGO@AtEnd%
%</package>
%    \end{macrocode}
%
% \section{Test}
%
% \subsection{Catcode checks for loading}
%
%    \begin{macrocode}
%<*test1>
%    \end{macrocode}
%    \begin{macrocode}
\catcode`\{=1 %
\catcode`\}=2 %
\catcode`\#=6 %
\catcode`\@=11 %
\expandafter\ifx\csname count@\endcsname\relax
  \countdef\count@=255 %
\fi
\expandafter\ifx\csname @gobble\endcsname\relax
  \long\def\@gobble#1{}%
\fi
\expandafter\ifx\csname @firstofone\endcsname\relax
  \long\def\@firstofone#1{#1}%
\fi
\expandafter\ifx\csname loop\endcsname\relax
  \expandafter\@firstofone
\else
  \expandafter\@gobble
\fi
{%
  \def\loop#1\repeat{%
    \def\body{#1}%
    \iterate
  }%
  \def\iterate{%
    \body
      \let\next\iterate
    \else
      \let\next\relax
    \fi
    \next
  }%
  \let\repeat=\fi
}%
\def\RestoreCatcodes{}
\count@=0 %
\loop
  \edef\RestoreCatcodes{%
    \RestoreCatcodes
    \catcode\the\count@=\the\catcode\count@\relax
  }%
\ifnum\count@<255 %
  \advance\count@ 1 %
\repeat

\def\RangeCatcodeInvalid#1#2{%
  \count@=#1\relax
  \loop
    \catcode\count@=15 %
  \ifnum\count@<#2\relax
    \advance\count@ 1 %
  \repeat
}
\def\RangeCatcodeCheck#1#2#3{%
  \count@=#1\relax
  \loop
    \ifnum#3=\catcode\count@
    \else
      \errmessage{%
        Character \the\count@\space
        with wrong catcode \the\catcode\count@\space
        instead of \number#3%
      }%
    \fi
  \ifnum\count@<#2\relax
    \advance\count@ 1 %
  \repeat
}
\def\space{ }
\expandafter\ifx\csname LoadCommand\endcsname\relax
  \def\LoadCommand{\input hologo.sty\relax}%
\fi
\def\Test{%
  \RangeCatcodeInvalid{0}{47}%
  \RangeCatcodeInvalid{58}{64}%
  \RangeCatcodeInvalid{91}{96}%
  \RangeCatcodeInvalid{123}{255}%
  \catcode`\@=12 %
  \catcode`\\=0 %
  \catcode`\%=14 %
  \LoadCommand
  \RangeCatcodeCheck{0}{36}{15}%
  \RangeCatcodeCheck{37}{37}{14}%
  \RangeCatcodeCheck{38}{47}{15}%
  \RangeCatcodeCheck{48}{57}{12}%
  \RangeCatcodeCheck{58}{63}{15}%
  \RangeCatcodeCheck{64}{64}{12}%
  \RangeCatcodeCheck{65}{90}{11}%
  \RangeCatcodeCheck{91}{91}{15}%
  \RangeCatcodeCheck{92}{92}{0}%
  \RangeCatcodeCheck{93}{96}{15}%
  \RangeCatcodeCheck{97}{122}{11}%
  \RangeCatcodeCheck{123}{255}{15}%
  \RestoreCatcodes
}
\Test
\csname @@end\endcsname
\end
%    \end{macrocode}
%    \begin{macrocode}
%</test1>
%    \end{macrocode}
%
% \subsection{Spacefactor}
%
%    The space factor must be 1000 after a logo. If it is greater 1000
%    then the following space is a space after a sentence closing point.
%    If the space factor is smaller 1000 then an immediate following
%    dot is interpreted as abbreviation, not sentence closing point.
%
%    \begin{macrocode}
%<*test-spacefactor>
\NeedsTeXFormat{LaTeX2e}
\documentclass{article}
\usepackage{hologo}[2016/05/12]
\usepackage{kvsetkeys}
\usepackage{qstest}
\IncludeTests{*}
\LogTests{log}{*}{*}
\begin{document}
\begin{qstest}{spacefactor}{spacefactor}
\newcommand*{\Test}[1]{%
  \sbox0{%
    \hologo{#1}%
    \Expect*{1000 (#1)}*{\the\spacefactor\space(#1)}%
  }%
}%
\makeatletter
\def\TestList{}
\def\hologoEntry#1#2#3{%
  \edef\TestList{%
    \ifx\TestList\@empty
    \else
      \TestList,%
    \fi
    #1%
    \ifx\\#2\\%
    \else
      ={variant=#2}%
    \fi
  }%
}
\hologoList
\expandafter\kv@parse@normalized\expandafter{%
  \TestList
}{%
  \begingroup
    \let\@logo=\kv@key
    \ifx\kv@value\relax
    \else
      \expandafter\hologoLogoSetup\expandafter\@logo\expandafter{%
        \kv@value
      }%
    \fi
    \Test\@logo
  \endgroup
  \@gobbletwo
}
\end{qstest}
\end{document}
%</test-spacefactor>
%    \end{macrocode}
%
% \subsection{Complete list}
%
%    \begin{macrocode}
%<*test-list>
\NeedsTeXFormat{LaTeX2e}
\documentclass[12pt,a4paper]{article}
\usepackage{hologo}[2016/05/12]
\usepackage[T1]{fontenc}
\usepackage{lmodern}
\usepackage{parskip}
\usepackage[unicode]{hyperref}[2011/09/28]
\usepackage{bookmark}[2011/09/19]
\bookmarksetup{%
  numbered,%
  open,%
  openlevel=2,%
}
\renewcommand*{\contentsname}{List of logos}
\begin{document}
\tableofcontents
\def\TestFont#1#2#3#4#5#6{%
  \begingroup
    \usefont{#3}{#4}{#5}{#6}%
    \HologoVariant{#1}{#2}/\hologoVariant{#1}{#2}%
    \quad
    \begingroup\scriptsize\hologoVariant{#1}{#2}\endgroup
    \quad
  \endgroup
  (#3/#4/#5/#6)%
  \par
}
\makeatletter
\def\hologoEntry#1#2#3{%
  \section{%
    \HologoVariant{#1}{#2}/\hologoVariant{#1}{#2} %
    {[#1\ifx\\#2\\\else\space(#2)\fi]}% hash-ok
  }% braces around [] because of bug in tex4ht
  \begingroup
    \hypersetup{unicode=false}%
    \bookmark[%
      dest=\@currentHref,%
      rellevel=1,%
      keeplevel,%
    ]{%
      \HologoVariant{#1}{#2}/\hologoVariant{#1}{#2} %
      (PDFDocEncoding)%
    }%
  \endgroup
  \TestFont{#1}{#2}{OT1}{cmr}{m}{n}%
  \TestFont{#1}{#2}{OT1}{cmss}{m}{n}%
  \TestFont{#1}{#2}{OT1}{cmr}{b}{n}%
  \TestFont{#1}{#2}{OT1}{cmr}{m}{it}%
  \TestFont{#1}{#2}{OT1}{cmtt}{m}{n}%
  \TestFont{#1}{#2}{T1}{lmr}{m}{n}%
  \TestFont{#1}{#2}{T1}{lmss}{m}{n}%
  \TestFont{#1}{#2}{T1}{lmr}{b}{n}%
  \TestFont{#1}{#2}{T1}{lmr}{m}{it}%
  \TestFont{#1}{#2}{T1}{lmtt}{m}{n}%
  \TestFont{#1}{#2}{T1}{lmvtt}{m}{n}%
  \TestFont{#1}{#2}{T1}{qtm}{m}{n}%
  \TestFont{#1}{#2}{T1}{qhv}{m}{n}%
  \TestFont{#1}{#2}{T1}{qtm}{b}{n}%
  \TestFont{#1}{#2}{T1}{qtm}{m}{it}%
  \TestFont{#1}{#2}{T1}{qcr}{m}{n}%
  \newpage
}
\makeatother
\hologoList
\end{document}
%</test-list>
%    \end{macrocode}
%
% \section{Installation}
%
% \subsection{Download}
%
% \paragraph{Package.} This package is available on
% CTAN\footnote{\url{ftp://ftp.ctan.org/tex-archive/}}:
% \begin{description}
% \item[\CTAN{macros/latex/contrib/oberdiek/hologo.dtx}] The source file.
% \item[\CTAN{macros/latex/contrib/oberdiek/hologo.pdf}] Documentation.
% \end{description}
%
%
% \paragraph{Bundle.} All the packages of the bundle `oberdiek'
% are also available in a TDS compliant ZIP archive. There
% the packages are already unpacked and the documentation files
% are generated. The files and directories obey the TDS standard.
% \begin{description}
% \item[\CTAN{install/macros/latex/contrib/oberdiek.tds.zip}]
% \end{description}
% \emph{TDS} refers to the standard ``A Directory Structure
% for \TeX\ Files'' (\CTAN{tds/tds.pdf}). Directories
% with \xfile{texmf} in their name are usually organized this way.
%
% \subsection{Bundle installation}
%
% \paragraph{Unpacking.} Unpack the \xfile{oberdiek.tds.zip} in the
% TDS tree (also known as \xfile{texmf} tree) of your choice.
% Example (linux):
% \begin{quote}
%   |unzip oberdiek.tds.zip -d ~/texmf|
% \end{quote}
%
% \paragraph{Script installation.}
% Check the directory \xfile{TDS:scripts/oberdiek/} for
% scripts that need further installation steps.
% Package \xpackage{attachfile2} comes with the Perl script
% \xfile{pdfatfi.pl} that should be installed in such a way
% that it can be called as \texttt{pdfatfi}.
% Example (linux):
% \begin{quote}
%   |chmod +x scripts/oberdiek/pdfatfi.pl|\\
%   |cp scripts/oberdiek/pdfatfi.pl /usr/local/bin/|
% \end{quote}
%
% \subsection{Package installation}
%
% \paragraph{Unpacking.} The \xfile{.dtx} file is a self-extracting
% \docstrip\ archive. The files are extracted by running the
% \xfile{.dtx} through \plainTeX:
% \begin{quote}
%   \verb|tex hologo.dtx|
% \end{quote}
%
% \paragraph{TDS.} Now the different files must be moved into
% the different directories in your installation TDS tree
% (also known as \xfile{texmf} tree):
% \begin{quote}
% \def\t{^^A
% \begin{tabular}{@{}>{\ttfamily}l@{ $\rightarrow$ }>{\ttfamily}l@{}}
%   hologo.sty & tex/generic/oberdiek/hologo.sty\\
%   hologo.pdf & doc/latex/oberdiek/hologo.pdf\\
%   example/hologo-example.tex & doc/latex/oberdiek/example/hologo-example.tex\\
%   test/hologo-test1.tex & doc/latex/oberdiek/test/hologo-test1.tex\\
%   test/hologo-test-spacefactor.tex & doc/latex/oberdiek/test/hologo-test-spacefactor.tex\\
%   test/hologo-test-list.tex & doc/latex/oberdiek/test/hologo-test-list.tex\\
%   hologo.dtx & source/latex/oberdiek/hologo.dtx\\
% \end{tabular}^^A
% }^^A
% \sbox0{\t}^^A
% \ifdim\wd0>\linewidth
%   \begingroup
%     \advance\linewidth by\leftmargin
%     \advance\linewidth by\rightmargin
%   \edef\x{\endgroup
%     \def\noexpand\lw{\the\linewidth}^^A
%   }\x
%   \def\lwbox{^^A
%     \leavevmode
%     \hbox to \linewidth{^^A
%       \kern-\leftmargin\relax
%       \hss
%       \usebox0
%       \hss
%       \kern-\rightmargin\relax
%     }^^A
%   }^^A
%   \ifdim\wd0>\lw
%     \sbox0{\small\t}^^A
%     \ifdim\wd0>\linewidth
%       \ifdim\wd0>\lw
%         \sbox0{\footnotesize\t}^^A
%         \ifdim\wd0>\linewidth
%           \ifdim\wd0>\lw
%             \sbox0{\scriptsize\t}^^A
%             \ifdim\wd0>\linewidth
%               \ifdim\wd0>\lw
%                 \sbox0{\tiny\t}^^A
%                 \ifdim\wd0>\linewidth
%                   \lwbox
%                 \else
%                   \usebox0
%                 \fi
%               \else
%                 \lwbox
%               \fi
%             \else
%               \usebox0
%             \fi
%           \else
%             \lwbox
%           \fi
%         \else
%           \usebox0
%         \fi
%       \else
%         \lwbox
%       \fi
%     \else
%       \usebox0
%     \fi
%   \else
%     \lwbox
%   \fi
% \else
%   \usebox0
% \fi
% \end{quote}
% If you have a \xfile{docstrip.cfg} that configures and enables \docstrip's
% TDS installing feature, then some files can already be in the right
% place, see the documentation of \docstrip.
%
% \subsection{Refresh file name databases}
%
% If your \TeX~distribution
% (\teTeX, \mikTeX, \dots) relies on file name databases, you must refresh
% these. For example, \teTeX\ users run \verb|texhash| or
% \verb|mktexlsr|.
%
% \subsection{Some details for the interested}
%
% \paragraph{Attached source.}
%
% The PDF documentation on CTAN also includes the
% \xfile{.dtx} source file. It can be extracted by
% AcrobatReader 6 or higher. Another option is \textsf{pdftk},
% e.g. unpack the file into the current directory:
% \begin{quote}
%   \verb|pdftk hologo.pdf unpack_files output .|
% \end{quote}
%
% \paragraph{Unpacking with \LaTeX.}
% The \xfile{.dtx} chooses its action depending on the format:
% \begin{description}
% \item[\plainTeX:] Run \docstrip\ and extract the files.
% \item[\LaTeX:] Generate the documentation.
% \end{description}
% If you insist on using \LaTeX\ for \docstrip\ (really,
% \docstrip\ does not need \LaTeX), then inform the autodetect routine
% about your intention:
% \begin{quote}
%   \verb|latex \let\install=y\input{hologo.dtx}|
% \end{quote}
% Do not forget to quote the argument according to the demands
% of your shell.
%
% \paragraph{Generating the documentation.}
% You can use both the \xfile{.dtx} or the \xfile{.drv} to generate
% the documentation. The process can be configured by the
% configuration file \xfile{ltxdoc.cfg}. For instance, put this
% line into this file, if you want to have A4 as paper format:
% \begin{quote}
%   \verb|\PassOptionsToClass{a4paper}{article}|
% \end{quote}
% An example follows how to generate the
% documentation with pdf\LaTeX:
% \begin{quote}
%\begin{verbatim}
%pdflatex hologo.dtx
%makeindex -s gind.ist hologo.idx
%pdflatex hologo.dtx
%makeindex -s gind.ist hologo.idx
%pdflatex hologo.dtx
%\end{verbatim}
% \end{quote}
%
% \section{Catalogue}
%
% The following XML file can be used as source for the
% \href{http://mirror.ctan.org/help/Catalogue/catalogue.html}{\TeX\ Catalogue}.
% The elements \texttt{caption} and \texttt{description} are imported
% from the original XML file from the Catalogue.
% The name of the XML file in the Catalogue is \xfile{hologo.xml}.
%    \begin{macrocode}
%<*catalogue>
<?xml version='1.0' encoding='us-ascii'?>
<!DOCTYPE entry SYSTEM 'catalogue.dtd'>
<entry datestamp='$Date$' modifier='$Author$' id='hologo'>
  <name>hologo</name>
  <caption>A collection of logos with bookmark support.</caption>
  <authorref id='auth:oberdiek'/>
  <copyright owner='Heiko Oberdiek' year='2010-2012'/>
  <license type='lppl1.3'/>
  <version number='1.10'/>
  <description>
    The package defines a single command <tt>\hologo</tt>, whose
    argument is the usual case-confused ASCII version of the logo.
    The command is bookmark-enabled, so that every logo becomes
    available in bookmarks without further work.
    <p/>
    The package is part of the <xref refid='oberdiek'>oberdiek</xref>
    bundle.
  </description>
  <documentation details='Package documentation'
      href='ctan:/macros/latex/contrib/oberdiek/hologo.pdf'/>
  <ctan file='true' path='/macros/latex/contrib/oberdiek/hologo.dtx'/>
  <miktex location='oberdiek'/>
  <texlive location='oberdiek'/>
  <install path='/macros/latex/contrib/oberdiek/oberdiek.tds.zip'/>
</entry>
%</catalogue>
%    \end{macrocode}
%
% \begin{thebibliography}{9}
% \raggedright
%
% \bibitem{btxdoc}
% Oren Patashnik,
% \textit{\hologo{BibTeX}ing},
% 1988-02-08.\\
% \CTAN{biblio/bibtex/base/}
%
% \bibitem{dtklogos}
% Gerd Neugebauer, DANTE,
% \textit{Package \xpackage{dtklogos}},
% 2011-04-25.\\
% \CTAN{usergrps/dante/dtk/dtklogos.sty}
%
% \bibitem{etexman}
% The \hologo{NTS} Team,
% \textit{The \hologo{eTeX} manual},
% 1998-02.\\
% \CTAN{systems/e-tex/v2/doc/}
%
% \bibitem{ExTeX-FAQ}
% The \hologo{ExTeX} group,
% \textit{\hologo{ExTeX}: FAQ -- How is \hologo{ExTeX} typeset?},
% 2007-04-14.\\
% \url{http://www.extex.org/documentation/faq.html}
%
% \bibitem{LyX}
% %@MISC{ LyX,
% %  title = {{LyX 2.0.0 -- The Document Processor [Computer software and manual]}},
% %  author = {{The LyX Team}},
% %  howpublished = {Internet: http://www.lyx.org},
% %  year = {2011-05-08},
% %  note = {Retrieved May 10, 2011, from http://www.lyx.org},
% %  url = {http://www.lyx.org/}
% %}
% The \hologo{LyX} Team,
% \textit{\hologo{LyX} -- The Document Processor},
% 2011-05-08.\\
% \url{http://www.lyx.org/}
%
% \bibitem{OzTeX}
% Andrew Trevorrow,
% \hologo{OzTeX} FAQ: What is the correct way to typeset ``\hologo{OzTeX}''?,
% 2011-09-15 (visited).
% \url{http://www.trevorrow.com/oztex/ozfaq.html#oztex-logo}
%
% \bibitem{PiCTeX}
% Michael Wichura,
% \textit{The \hologo{PiCTeX} macro package},
% 1987-09-21.
% \CTAN{graphics/pictex/}
%
% \bibitem{scrlogo}
% Markus Kohm,
% \textit{\hologo{KOMAScript} Datei \xfile{scrlogo.dtx}},
% 2009-01-30.\\
% \CTAN{install/macros/latex/contrib/komascript.tds.zip}
%
% \end{thebibliography}
%
% \begin{History}
%   \begin{Version}{2010/04/08 v1.0}
%   \item
%     The first version.
%   \end{Version}
%   \begin{Version}{2010/04/16 v1.1}
%   \item
%     \cs{Hologo} added for support of logos at start of a sentence.
%   \item
%     \cs{hologoSetup} and \cs{hologoLogoSetup} added.
%   \item
%     Options \xoption{break}, \xoption{hyphenbreak}, \xoption{spacebreak}
%     added.
%   \item
%     Variant support added by option \xoption{variant}.
%   \end{Version}
%   \begin{Version}{2010/04/24 v1.2}
%   \item
%     \hologo{LaTeX3} added.
%   \item
%     \hologo{VTeX} added.
%   \end{Version}
%   \begin{Version}{2010/11/21 v1.3}
%   \item
%     \hologo{iniTeX}, \hologo{virTeX} added.
%   \end{Version}
%   \begin{Version}{2011/03/25 v1.4}
%   \item
%     \hologo{ConTeXt} with variants added.
%   \item
%     Option \xoption{discretionarybreak} added as refinement for
%     option \xoption{break}.
%   \end{Version}
%   \begin{Version}{2011/04/21 v1.5}
%   \item
%     Wrong TDS directory for test files fixed.
%   \end{Version}
%   \begin{Version}{2011/10/01 v1.6}
%   \item
%     Support for package \xpackage{tex4ht} added.
%   \item
%     Support for \cs{csname} added if \cs{ifincsname} is available.
%   \item
%     New logos:
%     \hologo{(La)TeX},
%     \hologo{biber},
%     \hologo{BibTeX} (\xoption{sc}, \xoption{sf}),
%     \hologo{emTeX},
%     \hologo{ExTeX},
%     \hologo{KOMAScript},
%     \hologo{La},
%     \hologo{LyX},
%     \hologo{MiKTeX},
%     \hologo{NTS},
%     \hologo{OzMF},
%     \hologo{OzMP},
%     \hologo{OzTeX},
%     \hologo{OzTtH},
%     \hologo{PCTeX},
%     \hologo{PiC},
%     \hologo{PiCTeX},
%     \hologo{METAFONT},
%     \hologo{MetaFun},
%     \hologo{METAPOST},
%     \hologo{MetaPost},
%     \hologo{SLiTeX} (\xoption{lift}, \xoption{narrow}, \xoption{simple}),
%     \hologo{SliTeX} (\xoption{narrow}, \xoption{simple}, \xoption{lift}),
%     \hologo{teTeX}.
%   \item
%     Fixes:
%     \hologo{iniTeX},
%     \hologo{pdfLaTeX},
%     \hologo{pdfTeX},
%     \hologo{virTeX}.
%   \item
%     \cs{hologoFontSetup} and \cs{hologoLogoFontSetup} added.
%   \item
%     \cs{hologoVariant} and \cs{HologoVariant} added.
%   \end{Version}
%   \begin{Version}{2011/11/22 v1.7}
%   \item
%     New logos:
%     \hologo{BibTeX8},
%     \hologo{LaTeXML},
%     \hologo{SageTeX},
%     \hologo{TeX4ht},
%     \hologo{TTH}.
%   \item
%     \hologo{Xe} and friends: Driver stuff fixed.
%   \item
%     \hologo{Xe} and friends: Support for italic added.
%   \item
%     \hologo{Xe} and friends: Package support for \xpackage{pgf}
%     and \xpackage{pstricks} added.
%   \end{Version}
%   \begin{Version}{2011/11/29 v1.8}
%   \item
%     New logos:
%     \hologo{HanTheThanh}.
%   \end{Version}
%   \begin{Version}{2011/12/21 v1.9}
%   \item
%     Patch for package \xpackage{ifxetex} added for the case that
%     \cs{newif} is undefined in \hologo{iniTeX}.
%   \item
%     Some fixes for \hologo{iniTeX}.
%   \end{Version}
%   \begin{Version}{2012/04/26 v1.10}
%   \item
%     Fix in bookmark version of logo ``\hologo{HanTheThanh}''.
%   \end{Version}
%   \begin{Version}{2016/05/12 v1.11}
%   \item
%     Update HOLOGO@IfCharExists (previously in texlive)
%   \item define pdfliteral in current luatex.
%   \end{Version}
% \end{History}
%
% \PrintIndex
%
% \Finale
\endinput
%
        \else
          \input hologo.cfg\relax
        \fi
      \fi
    }%
    \ltx@IfUndefined{newread}{%
      \chardef\HOLOGO@temp=15 %
      \def\HOLOGO@CheckRead{%
        \ifeof\HOLOGO@temp
          \HOLOGO@InputIfExists
        \else
          \ifcase\HOLOGO@temp
            \@PackageWarningNoLine{hologo}{%
              Configuration file ignored, because\MessageBreak
              a free read register could not be found%
            }%
          \else
            \begingroup
              \count\ltx@cclv=\HOLOGO@temp
              \advance\ltx@cclv by \ltx@minusone
              \edef\x{\endgroup
                \chardef\noexpand\HOLOGO@temp=\the\count\ltx@cclv
                \relax
              }%
            \x
          \fi
        \fi
      }%
    }{%
      \csname newread\endcsname\HOLOGO@temp
      \HOLOGO@InputIfExists
    }%
  }{%
    \edef\HOLOGO@temp{\pdf@filesize{hologo.cfg}}%
    \ifx\HOLOGO@temp\ltx@empty
    \else
      \ifnum\HOLOGO@temp>0 %
        \begingroup
          \def\x{LaTeX2e}%
        \expandafter\endgroup
        \ifx\fmtname\x
          % \iffalse meta-comment
%
% File: hologo.dtx
% Version: 2016/05/12 v1.11
% Info: A logo collection with bookmark support
%
% Copyright (C) 2010-2012 by
%    Heiko Oberdiek <heiko.oberdiek at googlemail.com>
%
% This work may be distributed and/or modified under the
% conditions of the LaTeX Project Public License, either
% version 1.3c of this license or (at your option) any later
% version. This version of this license is in
%    http://www.latex-project.org/lppl/lppl-1-3c.txt
% and the latest version of this license is in
%    http://www.latex-project.org/lppl.txt
% and version 1.3 or later is part of all distributions of
% LaTeX version 2005/12/01 or later.
%
% This work has the LPPL maintenance status "maintained".
%
% This Current Maintainer of this work is Heiko Oberdiek.
%
% The Base Interpreter refers to any `TeX-Format',
% because some files are installed in TDS:tex/generic//.
%
% This work consists of the main source file hologo.dtx
% and the derived files
%    hologo.sty, hologo.pdf, hologo.ins, hologo.drv, hologo-example.tex,
%    hologo-test1.tex, hologo-test-spacefactor.tex,
%    hologo-test-list.tex.
%
% Distribution:
%    CTAN:macros/latex/contrib/oberdiek/hologo.dtx
%    CTAN:macros/latex/contrib/oberdiek/hologo.pdf
%
% Unpacking:
%    (a) If hologo.ins is present:
%           tex hologo.ins
%    (b) Without hologo.ins:
%           tex hologo.dtx
%    (c) If you insist on using LaTeX
%           latex \let\install=y\input{hologo.dtx}
%        (quote the arguments according to the demands of your shell)
%
% Documentation:
%    (a) If hologo.drv is present:
%           latex hologo.drv
%    (b) Without hologo.drv:
%           latex hologo.dtx; ...
%    The class ltxdoc loads the configuration file ltxdoc.cfg
%    if available. Here you can specify further options, e.g.
%    use A4 as paper format:
%       \PassOptionsToClass{a4paper}{article}
%
%    Programm calls to get the documentation (example):
%       pdflatex hologo.dtx
%       makeindex -s gind.ist hologo.idx
%       pdflatex hologo.dtx
%       makeindex -s gind.ist hologo.idx
%       pdflatex hologo.dtx
%
% Installation:
%    TDS:tex/generic/oberdiek/hologo.sty
%    TDS:doc/latex/oberdiek/hologo.pdf
%    TDS:doc/latex/oberdiek/example/hologo-example.tex
%    TDS:doc/latex/oberdiek/test/hologo-test1.tex
%    TDS:doc/latex/oberdiek/test/hologo-test-spacefactor.tex
%    TDS:doc/latex/oberdiek/test/hologo-test-list.tex
%    TDS:source/latex/oberdiek/hologo.dtx
%
%<*ignore>
\begingroup
  \catcode123=1 %
  \catcode125=2 %
  \def\x{LaTeX2e}%
\expandafter\endgroup
\ifcase 0\ifx\install y1\fi\expandafter
         \ifx\csname processbatchFile\endcsname\relax\else1\fi
         \ifx\fmtname\x\else 1\fi\relax
\else\csname fi\endcsname
%</ignore>
%<*install>
\input docstrip.tex
\Msg{************************************************************************}
\Msg{* Installation}
\Msg{* Package: hologo 2016/05/12 v1.11 A logo collection with bookmark support (HO)}
\Msg{************************************************************************}

\keepsilent
\askforoverwritefalse

\let\MetaPrefix\relax
\preamble

This is a generated file.

Project: hologo
Version: 2016/05/12 v1.11

Copyright (C) 2010-2012 by
   Heiko Oberdiek <heiko.oberdiek at googlemail.com>

This work may be distributed and/or modified under the
conditions of the LaTeX Project Public License, either
version 1.3c of this license or (at your option) any later
version. This version of this license is in
   http://www.latex-project.org/lppl/lppl-1-3c.txt
and the latest version of this license is in
   http://www.latex-project.org/lppl.txt
and version 1.3 or later is part of all distributions of
LaTeX version 2005/12/01 or later.

This work has the LPPL maintenance status "maintained".

This Current Maintainer of this work is Heiko Oberdiek.

The Base Interpreter refers to any `TeX-Format',
because some files are installed in TDS:tex/generic//.

This work consists of the main source file hologo.dtx
and the derived files
   hologo.sty, hologo.pdf, hologo.ins, hologo.drv, hologo-example.tex,
   hologo-test1.tex, hologo-test-spacefactor.tex,
   hologo-test-list.tex.

\endpreamble
\let\MetaPrefix\DoubleperCent

\generate{%
  \file{hologo.ins}{\from{hologo.dtx}{install}}%
  \file{hologo.drv}{\from{hologo.dtx}{driver}}%
  \usedir{tex/generic/oberdiek}%
  \file{hologo.sty}{\from{hologo.dtx}{package}}%
  \usedir{doc/latex/oberdiek/example}%
  \file{hologo-example.tex}{\from{hologo.dtx}{example}}%
  \usedir{doc/latex/oberdiek/test}%
  \file{hologo-test1.tex}{\from{hologo.dtx}{test1}}%
  \file{hologo-test-spacefactor.tex}{\from{hologo.dtx}{test-spacefactor}}%
  \file{hologo-test-list.tex}{\from{hologo.dtx}{test-list}}%
  \nopreamble
  \nopostamble
  \usedir{source/latex/oberdiek/catalogue}%
  \file{hologo.xml}{\from{hologo.dtx}{catalogue}}%
}

\catcode32=13\relax% active space
\let =\space%
\Msg{************************************************************************}
\Msg{*}
\Msg{* To finish the installation you have to move the following}
\Msg{* file into a directory searched by TeX:}
\Msg{*}
\Msg{*     hologo.sty}
\Msg{*}
\Msg{* To produce the documentation run the file `hologo.drv'}
\Msg{* through LaTeX.}
\Msg{*}
\Msg{* Happy TeXing!}
\Msg{*}
\Msg{************************************************************************}

\endbatchfile
%</install>
%<*ignore>
\fi
%</ignore>
%<*driver>
\NeedsTeXFormat{LaTeX2e}
\ProvidesFile{hologo.drv}%
  [2016/05/12 v1.11 A logo collection with bookmark support (HO)]%
\documentclass{ltxdoc}
\usepackage{holtxdoc}[2011/11/22]
\usepackage{hologo}[2016/05/12]
\usepackage{longtable}
\usepackage{array}
\usepackage{paralist}
%\usepackage[T1]{fontenc}
%\usepackage{lmodern}
\begin{document}
  \DocInput{hologo.dtx}%
\end{document}
%</driver>
% \fi
%
%
% \CharacterTable
%  {Upper-case    \A\B\C\D\E\F\G\H\I\J\K\L\M\N\O\P\Q\R\S\T\U\V\W\X\Y\Z
%   Lower-case    \a\b\c\d\e\f\g\h\i\j\k\l\m\n\o\p\q\r\s\t\u\v\w\x\y\z
%   Digits        \0\1\2\3\4\5\6\7\8\9
%   Exclamation   \!     Double quote  \"     Hash (number) \#
%   Dollar        \$     Percent       \%     Ampersand     \&
%   Acute accent  \'     Left paren    \(     Right paren   \)
%   Asterisk      \*     Plus          \+     Comma         \,
%   Minus         \-     Point         \.     Solidus       \/
%   Colon         \:     Semicolon     \;     Less than     \<
%   Equals        \=     Greater than  \>     Question mark \?
%   Commercial at \@     Left bracket  \[     Backslash     \\
%   Right bracket \]     Circumflex    \^     Underscore    \_
%   Grave accent  \`     Left brace    \{     Vertical bar  \|
%   Right brace   \}     Tilde         \~}
%
% \GetFileInfo{hologo.drv}
%
% \title{The \xpackage{hologo} package}
% \date{2016/05/12 v1.11}
% \author{Heiko Oberdiek\\\xemail{heiko.oberdiek at googlemail.com}}
%
% \maketitle
%
% \begin{abstract}
% This package starts a collection of logos with support for bookmarks
% strings.
% \end{abstract}
%
% \tableofcontents
%
% \section{Documentation}
%
% \subsection{Logo macros}
%
% \begin{declcs}{hologo} \M{name}
% \end{declcs}
% Macro \cs{hologo} sets the logo with name \meta{name}.
% The following table shows the supported names.
%
% \begingroup
%   \def\hologoEntry#1#2#3{^^A
%     #1&#2&\hologoLogoSetup{#1}{variant=#2}\hologo{#1}&#3\tabularnewline
%   }
%   \begin{longtable}{>{\ttfamily}l>{\ttfamily}lll}
%     \rmfamily\bfseries{name} & \rmfamily\bfseries variant
%     & \bfseries logo & \bfseries since\\
%     \hline
%     \endhead
%     \hologoList
%   \end{longtable}
% \endgroup
%
% \begin{declcs}{Hologo} \M{name}
% \end{declcs}
% Macro \cs{Hologo} starts the logo \meta{name} with an uppercase
% letter. As an exception small greek letters are not converted
% to uppercase. Examples, see \hologo{eTeX} and \hologo{ExTeX}.
%
% \subsection{Setup macros}
%
% The package does not support package options, but the following
% setup macros can be used to set options.
%
% \begin{declcs}{hologoSetup} \M{key value list}
% \end{declcs}
% Macro \cs{hologoSetup} sets global options.
%
% \begin{declcs}{hologoLogoSetup} \M{logo} \M{key value list}
% \end{declcs}
% Some options can also be used to configure a logo.
% These settings take precedence over global option settings.
%
% \subsection{Options}\label{sec:options}
%
% There are boolean and string options:
% \begin{description}
% \item[Boolean option:]
% It takes |true| or |false|
% as value. If the value is omitted, then |true| is used.
% \item[String option:]
% A value must be given as string. (But the string might be empty.)
% \end{description}
% The following options can be used both in \cs{hologoSetup}
% and \cs{hologoLogoSetup}:
% \begin{description}
% \def\entry#1{\item[\xoption{#1}:]}
% \entry{break}
%   enables or disables line breaks inside the logo. This setting is
%   refined by options \xoption{hyphenbreak}, \xoption{spacebreak}
%   or \xoption{discretionarybreak}.
%   Default is |false|.
% \entry{hyphenbreak}
%   enables or disables the line break right after the hyphen character.
% \entry{spacebreak}
%   enables or disables line breaks at space characters.
% \entry{discretionarybreak}
%   enables or disables line breaks at hyphenation points
%   (inserted by \cs{-}).
% \end{description}
% Macro \cs{hologoLogoSetup} also knows:
% \begin{description}
% \item[\xoption{variant}:]
%   This is a string option. It specifies a variant of a logo that
%   must exist. An empty string selects the package default variant.
% \end{description}
% Example:
% \begin{quote}
%   |\hologoSetup{break=false}|\\
%   |\hologoLogoSetup{plainTeX}{variant=hyphen,hyphenbreak}|\\
%   Then ``plain-\TeX'' contains one break point after the hyphen.
% \end{quote}
%
% \subsection{Driver options}
%
% Sometimes graphical operations are needed to construct some
% glyphs (e.g.\ \hologo{XeTeX}). If package \xpackage{graphics}
% or package \xpackage{pgf} are found, then the macros are taken
% from there. Otherwise the packge defines its own operations
% and therefore needs the driver information. Many drivers are
% detected automatically (\hologo{pdfTeX}/\hologo{LuaTeX}
% in PDF mode, \hologo{XeTeX}, \hologo{VTeX}). These have precedence
% over a driver option. The driver can be given as package option
% or using \cs{hologoDriverSetup}.
% The following list contains the recognized driver options:
% \begin{itemize}
% \item \xoption{pdftex}, \xoption{luatex}
% \item \xoption{dvipdfm}, \xoption{dvipdfmx}
% \item \xoption{dvips}, \xoption{dvipsone}, \xoption{xdvi}
% \item \xoption{xetex}
% \item \xoption{vtex}
% \end{itemize}
% The left driver of a line is the driver name that is used internally.
% The following names are aliases for drivers that use the
% same method. Therefore the entry in the \xext{log} file for
% the used driver prints the internally used driver name.
% \begin{description}
% \item[\xoption{driverfallback}:]
%   This option expects a driver that is used,
%   if the driver could not be detected automatically.
% \end{description}
%
% \begin{declcs}{hologoDriverSetup} \M{driver option}
% \end{declcs}
% The driver can also be configured after package loading
% using \cs{hologoDriverSetup}, also the way for \hologo{plainTeX}
% to setup the driver.
%
% \subsection{Font setup}
%
% Some logos require a special font, but should also be usable by
% \hologo{plainTeX}. Therefore the package provides some ways
% to influence the font settings. The options below
% take font settings as values. Both font commands
% such as \cs{sffamily} and macros that take one argument
% like \cs{textsf} can be used.
%
% \begin{declcs}{hologoFontSetup} \M{key value list}
% \end{declcs}
% Macro \cs{hologoFontSetup} sets the fonts for all logos.
% Supported keys:
% \begin{description}
% \def\entry#1{\item[\xoption{#1}:]}
% \entry{general}
%   This font is used for all logos. The default is empty.
%   That means no special font is used.
% \entry{bibsf}
%   This font is used for
%   {\hologoLogoSetup{BibTeX}{variant=sf}\hologo{BibTeX}}
%   with variant \xoption{sf}.
% \entry{rm}
%   This font is a serif font. It is used for \hologo{ExTeX}.
% \entry{sc}
%   This font specifies a small caps font. It is used for
%   {\hologoLogoSetup{BibTeX}{variant=sc}\hologo{BibTeX}}
%   with variant \xoption{sc}.
% \entry{sf}
%   This font specifies a sans serif font. The default
%   is \cs{sffamily}, then \cs{sf} is tried. Otherwise
%   a warning is given. It is used by \hologo{KOMAScript}.
% \entry{sy}
%   This is the font for math symbols (e.g. cmsy).
%   It is used by \hologo{AmS}, \hologo{NTS}, \hologo{ExTeX}.
% \entry{logo}
%   \hologo{METAFONT} and \hologo{METAPOST} are using that font.
%   In \hologo{LaTeX} \cs{logofamily} is used and
%   the definitions of package \xpackage{mflogo} are used
%   if the package is not loaded.
%   Otherwise the \cs{tenlogo} is used and defined
%   if it does not already exists.
% \end{description}
%
% \begin{declcs}{hologoLogoFontSetup} \M{logo} \M{key value list}
% \end{declcs}
% Fonts can also be set for a logo or logo component separately,
% see the following list.
% The keys are the same as for \cs{hologoFontSetup}.
%
% \begin{longtable}{>{\ttfamily}l>{\sffamily}ll}
%   \meta{logo} & keys & result\\
%   \hline
%   \endhead
%   BibTeX & bibsf & {\hologoLogoSetup{BibTeX}{variant=sf}\hologo{BibTeX}}\\[.5ex]
%   BibTeX & sc & {\hologoLogoSetup{BibTeX}{variant=sc}\hologo{BibTeX}}\\[.5ex]
%   ExTeX & rm & \hologo{ExTeX}\\
%   SliTeX & rm & \hologo{SliTeX}\\[.5ex]
%   AmS & sy & \hologo{AmS}\\
%   ExTeX & sy & \hologo{ExTeX}\\
%   NTS & sy & \hologo{NTS}\\[.5ex]
%   KOMAScript & sf & \hologo{KOMAScript}\\[.5ex]
%   METAFONT & logo & \hologo{METAFONT}\\
%   METAPOST & logo & \hologo{METAPOST}\\[.5ex]
%   SliTeX & sc \hologo{SliTeX}
% \end{longtable}
%
% \subsubsection{Font order}
%
% For all logos the font \xoption{general} is applied first.
% Example:
%\begin{quote}
%|\hologoFontSetup{general=\color{red}}|
%\end{quote}
% will print red logos.
% Then if the font uses a special font \xoption{sf}, for example,
% the font is applied that is setup by \cs{hologoLogoFontSetup}.
% If this font is not setup, then the common font setup
% by \cs{hologoFontSetup} is used. Otherwise a warning is given,
% that there is no font configured.
%
% \subsection{Additional user macros}
%
% Usually a variant of a logo is configured by using
% \cs{hologoLogoSetup}, because it is bad style to mix
% different variants of the same logo in the same text.
% There the following macros are a convenience for testing.
%
% \begin{declcs}{hologoVariant} \M{name} \M{variant}\\
%   \cs{HologoVariant} \M{name} \M{variant}
% \end{declcs}
% Logo \meta{name} is set using \meta{variant} that specifies
% explicitely which variant of the macro is used. If the argument
% is empty, then the default form of the logo is used
% (configurable by \cs{hologoLogoSetup}).
%
% \cs{HologoVariant} is used if the logo is set in a context
% that needs an uppercase first letter (beginning of a sentence, \dots).
%
% \begin{declcs}{hologoList}\\
%   \cs{hologoEntry} \M{logo} \M{variant} \M{since}
% \end{declcs}
% Macro \cs{hologoList} contains all logos that are provided
% by the package including variants. The list consists of calls
% of \cs{hologoEntry} with three arguments starting with the
% logo name \meta{logo} and its variant \meta{variant}. An empty
% variant means the current default. Argument \meta{since} specifies
% with version of the package \xpackage{hologo} is needed to get
% the logo. If the logo is fixed, then the date gets updated.
% Therefore the date \meta{since} is not exactly the date of
% the first introduction, but rather the date of the latest fix.
%
% Before \cs{hologoList} can be used, macro \cs{hologoEntry} needs
% a definition. The example file in section \ref{sec:example}
% shows applications of \cs{hologoList}.
%
% \subsection{Supported contexts}
%
% Macros \cs{hologo} and friends support special contexts:
% \begin{itemize}
% \item \hologo{LaTeX}'s protection mechanism.
% \item Bookmarks of package \xpackage{hyperref}.
% \item Package \xpackage{tex4ht}.
% \item The macros can be used inside \cs{csname} constructs,
%   if \cs{ifincsname} is available (\hologo{pdfTeX}, \hologo{XeTeX},
%   \hologo{LuaTeX}).
% \end{itemize}
%
% \subsection{Example}
% \label{sec:example}
%
% The following example prints the logos in different fonts.
%    \begin{macrocode}
%<*example>
%<<verbatim
\NeedsTeXFormat{LaTeX2e}
\documentclass[a4paper]{article}
\usepackage[
  hmargin=20mm,
  vmargin=20mm,
]{geometry}
\pagestyle{empty}
\usepackage{hologo}[2016/05/12]
\usepackage{longtable}
\usepackage{array}
\setlength{\extrarowheight}{2pt}
\usepackage[T1]{fontenc}
\usepackage{lmodern}
\usepackage{pdflscape}
\usepackage[
  pdfencoding=auto,
]{hyperref}
\hypersetup{
  pdfauthor={Heiko Oberdiek},
  pdftitle={Example for package `hologo'},
  pdfsubject={Logos with fonts lmr, lmss, qtm, qpl, qhv},
}
\usepackage{bookmark}

% Print the logo list on the console

\begingroup
  \typeout{}%
  \typeout{*** Begin of logo list ***}%
  \newcommand*{\hologoEntry}[3]{%
    \typeout{#1 \ifx\\#2\\\else(#2) \fi[#3]}%
  }%
  \hologoList
  \typeout{*** End of logo list ***}%
  \typeout{}%
\endgroup

\begin{document}
\begin{landscape}

  \section{Example file for package `hologo'}

  % Table for font names

  \begin{longtable}{>{\bfseries}ll}
    \textbf{font} & \textbf{Font name}\\
    \hline
    lmr & Latin Modern Roman\\
    lmss & Latin Modern Sans\\
    qtm & \TeX\ Gyre Termes\\
    qhv & \TeX\ Gyre Heros\\
    qpl & \TeX\ Gyre Pagella\\
  \end{longtable}

  % Logo list with logos in different fonts

  \begingroup
    \newcommand*{\SetVariant}[2]{%
      \ifx\\#2\\%
      \else
        \hologoLogoSetup{#1}{variant=#2}%
      \fi
    }%
    \newcommand*{\hologoEntry}[3]{%
      \SetVariant{#1}{#2}%
      \raisebox{1em}[0pt][0pt]{\hypertarget{#1@#2}{}}%
      \bookmark[%
        dest={#1@#2},%
      ]{%
        #1\ifx\\#2\\\else\space(#2)\fi: \Hologo{#1}, \hologo{#1} %
        [Unicode]%
      }%
      \hypersetup{unicode=false}%
      \bookmark[%
        dest={#1@#2},%
      ]{%
        #1\ifx\\#2\\\else\space(#2)\fi: \Hologo{#1}, \hologo{#1} %
        [PDFDocEncoding]%
      }%
      \texttt{#1}%
      &%
      \texttt{#2}%
      &%
      \Hologo{#1}%
      &%
      \SetVariant{#1}{#2}%
      \hologo{#1}%
      &%
      \SetVariant{#1}{#2}%
      \fontfamily{qtm}\selectfont
      \hologo{#1}%
      &%
      \SetVariant{#1}{#2}%
      \fontfamily{qpl}\selectfont
      \hologo{#1}%
      &%
      \SetVariant{#1}{#2}%
      \textsf{\hologo{#1}}%
      &%
      \SetVariant{#1}{#2}%
      \fontfamily{qhv}\selectfont
      \hologo{#1}%
      \tabularnewline
    }%
    \begin{longtable}{llllllll}%
      \textbf{\textit{logo}} & \textbf{\textit{variant}} &
      \texttt{\string\Hologo} &
      \textbf{lmr} & \textbf{qtm} & \textbf{qpl} &
      \textbf{lmss} & \textbf{qhv}
      \tabularnewline
      \hline
      \endhead
      \hologoList
    \end{longtable}%
  \endgroup

\end{landscape}
\end{document}
%verbatim
%</example>
%    \end{macrocode}
%
% \StopEventually{
% }
%
% \section{Implementation}
%    \begin{macrocode}
%<*package>
%    \end{macrocode}
%    Reload check, especially if the package is not used with \LaTeX.
%    \begin{macrocode}
\begingroup\catcode61\catcode48\catcode32=10\relax%
  \catcode13=5 % ^^M
  \endlinechar=13 %
  \catcode35=6 % #
  \catcode39=12 % '
  \catcode44=12 % ,
  \catcode45=12 % -
  \catcode46=12 % .
  \catcode58=12 % :
  \catcode64=11 % @
  \catcode123=1 % {
  \catcode125=2 % }
  \expandafter\let\expandafter\x\csname ver@hologo.sty\endcsname
  \ifx\x\relax % plain-TeX, first loading
  \else
    \def\empty{}%
    \ifx\x\empty % LaTeX, first loading,
      % variable is initialized, but \ProvidesPackage not yet seen
    \else
      \expandafter\ifx\csname PackageInfo\endcsname\relax
        \def\x#1#2{%
          \immediate\write-1{Package #1 Info: #2.}%
        }%
      \else
        \def\x#1#2{\PackageInfo{#1}{#2, stopped}}%
      \fi
      \x{hologo}{The package is already loaded}%
      \aftergroup\endinput
    \fi
  \fi
\endgroup%
%    \end{macrocode}
%    Package identification:
%    \begin{macrocode}
\begingroup\catcode61\catcode48\catcode32=10\relax%
  \catcode13=5 % ^^M
  \endlinechar=13 %
  \catcode35=6 % #
  \catcode39=12 % '
  \catcode40=12 % (
  \catcode41=12 % )
  \catcode44=12 % ,
  \catcode45=12 % -
  \catcode46=12 % .
  \catcode47=12 % /
  \catcode58=12 % :
  \catcode64=11 % @
  \catcode91=12 % [
  \catcode93=12 % ]
  \catcode123=1 % {
  \catcode125=2 % }
  \expandafter\ifx\csname ProvidesPackage\endcsname\relax
    \def\x#1#2#3[#4]{\endgroup
      \immediate\write-1{Package: #3 #4}%
      \xdef#1{#4}%
    }%
  \else
    \def\x#1#2[#3]{\endgroup
      #2[{#3}]%
      \ifx#1\@undefined
        \xdef#1{#3}%
      \fi
      \ifx#1\relax
        \xdef#1{#3}%
      \fi
    }%
  \fi
\expandafter\x\csname ver@hologo.sty\endcsname
\ProvidesPackage{hologo}%
  [2016/05/12 v1.11 A logo collection with bookmark support (HO)]%
%    \end{macrocode}
%
%    \begin{macrocode}
\begingroup\catcode61\catcode48\catcode32=10\relax%
  \catcode13=5 % ^^M
  \endlinechar=13 %
  \catcode123=1 % {
  \catcode125=2 % }
  \catcode64=11 % @
  \def\x{\endgroup
    \expandafter\edef\csname HOLOGO@AtEnd\endcsname{%
      \endlinechar=\the\endlinechar\relax
      \catcode13=\the\catcode13\relax
      \catcode32=\the\catcode32\relax
      \catcode35=\the\catcode35\relax
      \catcode61=\the\catcode61\relax
      \catcode64=\the\catcode64\relax
      \catcode123=\the\catcode123\relax
      \catcode125=\the\catcode125\relax
    }%
  }%
\x\catcode61\catcode48\catcode32=10\relax%
\catcode13=5 % ^^M
\endlinechar=13 %
\catcode35=6 % #
\catcode64=11 % @
\catcode123=1 % {
\catcode125=2 % }
\def\TMP@EnsureCode#1#2{%
  \edef\HOLOGO@AtEnd{%
    \HOLOGO@AtEnd
    \catcode#1=\the\catcode#1\relax
  }%
  \catcode#1=#2\relax
}
\TMP@EnsureCode{10}{12}% ^^J
\TMP@EnsureCode{33}{12}% !
\TMP@EnsureCode{34}{12}% "
\TMP@EnsureCode{36}{3}% $
\TMP@EnsureCode{38}{4}% &
\TMP@EnsureCode{39}{12}% '
\TMP@EnsureCode{40}{12}% (
\TMP@EnsureCode{41}{12}% )
\TMP@EnsureCode{42}{12}% *
\TMP@EnsureCode{43}{12}% +
\TMP@EnsureCode{44}{12}% ,
\TMP@EnsureCode{45}{12}% -
\TMP@EnsureCode{46}{12}% .
\TMP@EnsureCode{47}{12}% /
\TMP@EnsureCode{58}{12}% :
\TMP@EnsureCode{59}{12}% ;
\TMP@EnsureCode{60}{12}% <
\TMP@EnsureCode{62}{12}% >
\TMP@EnsureCode{63}{12}% ?
\TMP@EnsureCode{91}{12}% [
\TMP@EnsureCode{93}{12}% ]
\TMP@EnsureCode{94}{7}% ^ (superscript)
\TMP@EnsureCode{95}{8}% _ (subscript)
\TMP@EnsureCode{96}{12}% `
\TMP@EnsureCode{124}{12}% |
\edef\HOLOGO@AtEnd{%
  \HOLOGO@AtEnd
  \escapechar\the\escapechar\relax
  \noexpand\endinput
}
\escapechar=92 %
%    \end{macrocode}
%
% \subsection{Logo list}
%
%    \begin{macro}{\hologoList}
%    \begin{macrocode}
\def\hologoList{%
  \hologoEntry{(La)TeX}{}{2011/10/01}%
  \hologoEntry{AmSLaTeX}{}{2010/04/16}%
  \hologoEntry{AmSTeX}{}{2010/04/16}%
  \hologoEntry{biber}{}{2011/10/01}%
  \hologoEntry{BibTeX}{}{2011/10/01}%
  \hologoEntry{BibTeX}{sf}{2011/10/01}%
  \hologoEntry{BibTeX}{sc}{2011/10/01}%
  \hologoEntry{BibTeX8}{}{2011/11/22}%
  \hologoEntry{ConTeXt}{}{2011/03/25}%
  \hologoEntry{ConTeXt}{narrow}{2011/03/25}%
  \hologoEntry{ConTeXt}{simple}{2011/03/25}%
  \hologoEntry{emTeX}{}{2010/04/26}%
  \hologoEntry{eTeX}{}{2010/04/08}%
  \hologoEntry{ExTeX}{}{2011/10/01}%
  \hologoEntry{HanTheThanh}{}{2011/11/29}%
  \hologoEntry{iniTeX}{}{2011/10/01}%
  \hologoEntry{KOMAScript}{}{2011/10/01}%
  \hologoEntry{La}{}{2010/05/08}%
  \hologoEntry{LaTeX}{}{2010/04/08}%
  \hologoEntry{LaTeX2e}{}{2010/04/08}%
  \hologoEntry{LaTeX3}{}{2010/04/24}%
  \hologoEntry{LaTeXe}{}{2010/04/08}%
  \hologoEntry{LaTeXML}{}{2011/11/22}%
  \hologoEntry{LaTeXTeX}{}{2011/10/01}%
  \hologoEntry{LuaLaTeX}{}{2010/04/08}%
  \hologoEntry{LuaTeX}{}{2010/04/08}%
  \hologoEntry{LyX}{}{2011/10/01}%
  \hologoEntry{METAFONT}{}{2011/10/01}%
  \hologoEntry{MetaFun}{}{2011/10/01}%
  \hologoEntry{METAPOST}{}{2011/10/01}%
  \hologoEntry{MetaPost}{}{2011/10/01}%
  \hologoEntry{MiKTeX}{}{2011/10/01}%
  \hologoEntry{NTS}{}{2011/10/01}%
  \hologoEntry{OzMF}{}{2011/10/01}%
  \hologoEntry{OzMP}{}{2011/10/01}%
  \hologoEntry{OzTeX}{}{2011/10/01}%
  \hologoEntry{OzTtH}{}{2011/10/01}%
  \hologoEntry{PCTeX}{}{2011/10/01}%
  \hologoEntry{pdfTeX}{}{2011/10/01}%
  \hologoEntry{pdfLaTeX}{}{2011/10/01}%
  \hologoEntry{PiC}{}{2011/10/01}%
  \hologoEntry{PiCTeX}{}{2011/10/01}%
  \hologoEntry{plainTeX}{}{2010/04/08}%
  \hologoEntry{plainTeX}{space}{2010/04/16}%
  \hologoEntry{plainTeX}{hyphen}{2010/04/16}%
  \hologoEntry{plainTeX}{runtogether}{2010/04/16}%
  \hologoEntry{SageTeX}{}{2011/11/22}%
  \hologoEntry{SLiTeX}{}{2011/10/01}%
  \hologoEntry{SLiTeX}{lift}{2011/10/01}%
  \hologoEntry{SLiTeX}{narrow}{2011/10/01}%
  \hologoEntry{SLiTeX}{simple}{2011/10/01}%
  \hologoEntry{SliTeX}{}{2011/10/01}%
  \hologoEntry{SliTeX}{narrow}{2011/10/01}%
  \hologoEntry{SliTeX}{simple}{2011/10/01}%
  \hologoEntry{SliTeX}{lift}{2011/10/01}%
  \hologoEntry{teTeX}{}{2011/10/01}%
  \hologoEntry{TeX}{}{2010/04/08}%
  \hologoEntry{TeX4ht}{}{2011/11/22}%
  \hologoEntry{TTH}{}{2011/11/22}%
  \hologoEntry{virTeX}{}{2011/10/01}%
  \hologoEntry{VTeX}{}{2010/04/24}%
  \hologoEntry{Xe}{}{2010/04/08}%
  \hologoEntry{XeLaTeX}{}{2010/04/08}%
  \hologoEntry{XeTeX}{}{2010/04/08}%
}
%    \end{macrocode}
%    \end{macro}
%
% \subsection{Load resources}
%
%    \begin{macrocode}
\begingroup\expandafter\expandafter\expandafter\endgroup
\expandafter\ifx\csname RequirePackage\endcsname\relax
  \def\TMP@RequirePackage#1[#2]{%
    \begingroup\expandafter\expandafter\expandafter\endgroup
    \expandafter\ifx\csname ver@#1.sty\endcsname\relax
      \input #1.sty\relax
    \fi
  }%
  \TMP@RequirePackage{ltxcmds}[2011/02/04]%
  \TMP@RequirePackage{infwarerr}[2010/04/08]%
  \TMP@RequirePackage{kvsetkeys}[2010/03/01]%
  \TMP@RequirePackage{kvdefinekeys}[2010/03/01]%
  \TMP@RequirePackage{pdftexcmds}[2010/04/01]%
  \TMP@RequirePackage{ifpdf}[2010/01/28]%
  \TMP@RequirePackage{ifluatex}[2010/03/01]%
  \ltx@IfUndefined{newif}{%
    \expandafter\let\csname newif\endcsname\ltx@newif
  }{}%
  \TMP@RequirePackage{ifxetex}[2009/01/23]%
  \TMP@RequirePackage{ifvtex}[2010/03/01]%
\else
  \RequirePackage{ltxcmds}[2011/02/04]%
  \RequirePackage{infwarerr}[2010/04/08]%
  \RequirePackage{kvsetkeys}[2010/03/01]%
  \RequirePackage{kvdefinekeys}[2010/03/01]%
  \RequirePackage{pdftexcmds}[2010/04/01]%
  \RequirePackage{ifpdf}[2010/01/28]%
  \RequirePackage{ifluatex}[2010/03/01]%
  \RequirePackage{ifxetex}[2009/01/23]%
  \RequirePackage{ifvtex}[2010/03/01]%
\fi
%    \end{macrocode}
%
%    \begin{macro}{\HOLOGO@IfDefined}
%    \begin{macrocode}
\def\HOLOGO@IfExists#1{%
  \ifx\@undefined#1%
    \expandafter\ltx@secondoftwo
  \else
    \ifx\relax#1%
      \expandafter\ltx@secondoftwo
    \else
      \expandafter\expandafter\expandafter\ltx@firstoftwo
    \fi
  \fi
}
%    \end{macrocode}
%    \end{macro}
%
% \subsection{Setup macros}
%
%    \begin{macro}{\hologoSetup}
%    \begin{macrocode}
\def\hologoSetup{%
  \let\HOLOGO@name\relax
  \HOLOGO@Setup
}
%    \end{macrocode}
%    \end{macro}
%
%    \begin{macro}{\hologoLogoSetup}
%    \begin{macrocode}
\def\hologoLogoSetup#1{%
  \edef\HOLOGO@name{#1}%
  \ltx@IfUndefined{HoLogo@\HOLOGO@name}{%
    \@PackageError{hologo}{%
      Unknown logo `\HOLOGO@name'%
    }\@ehc
    \ltx@gobble
  }{%
    \HOLOGO@Setup
  }%
}
%    \end{macrocode}
%    \end{macro}
%
%    \begin{macro}{\HOLOGO@Setup}
%    \begin{macrocode}
\def\HOLOGO@Setup{%
  \kvsetkeys{HoLogo}%
}
%    \end{macrocode}
%    \end{macro}
%
% \subsection{Options}
%
%    \begin{macro}{\HOLOGO@DeclareBoolOption}
%    \begin{macrocode}
\def\HOLOGO@DeclareBoolOption#1{%
  \expandafter\chardef\csname HOLOGOOPT@#1\endcsname\ltx@zero
  \kv@define@key{HoLogo}{#1}[true]{%
    \def\HOLOGO@temp{##1}%
    \ifx\HOLOGO@temp\HOLOGO@true
      \ifx\HOLOGO@name\relax
        \expandafter\chardef\csname HOLOGOOPT@#1\endcsname=\ltx@one
      \else
        \expandafter\chardef\csname
        HoLogoOpt@#1@\HOLOGO@name\endcsname\ltx@one
      \fi
      \HOLOGO@SetBreakAll{#1}%
    \else
      \ifx\HOLOGO@temp\HOLOGO@false
        \ifx\HOLOGO@name\relax
          \expandafter\chardef\csname HOLOGOOPT@#1\endcsname=\ltx@zero
        \else
          \expandafter\chardef\csname
          HoLogoOpt@#1@\HOLOGO@name\endcsname=\ltx@zero
        \fi
        \HOLOGO@SetBreakAll{#1}%
      \else
        \@PackageError{hologo}{%
          Unknown value `##1' for boolean option `#1'.\MessageBreak
          Known values are `true' and `false'%
        }\@ehc
      \fi
    \fi
  }%
}
%    \end{macrocode}
%    \end{macro}
%
%    \begin{macro}{\HOLOGO@SetBreakAll}
%    \begin{macrocode}
\def\HOLOGO@SetBreakAll#1{%
  \def\HOLOGO@temp{#1}%
  \ifx\HOLOGO@temp\HOLOGO@break
    \ifx\HOLOGO@name\relax
      \chardef\HOLOGOOPT@hyphenbreak=\HOLOGOOPT@break
      \chardef\HOLOGOOPT@spacebreak=\HOLOGOOPT@break
      \chardef\HOLOGOOPT@discretionarybreak=\HOLOGOOPT@break
    \else
      \expandafter\chardef
         \csname HoLogoOpt@hyphenbreak@\HOLOGO@name\endcsname=%
         \csname HoLogoOpt@break@\HOLOGO@name\endcsname
      \expandafter\chardef
         \csname HoLogoOpt@spacebreak@\HOLOGO@name\endcsname=%
         \csname HoLogoOpt@break@\HOLOGO@name\endcsname
      \expandafter\chardef
         \csname HoLogoOpt@discretionarybreak@\HOLOGO@name
             \endcsname=%
         \csname HoLogoOpt@break@\HOLOGO@name\endcsname
    \fi
  \fi
}
%    \end{macrocode}
%    \end{macro}
%
%    \begin{macro}{\HOLOGO@true}
%    \begin{macrocode}
\def\HOLOGO@true{true}
%    \end{macrocode}
%    \end{macro}
%    \begin{macro}{\HOLOGO@false}
%    \begin{macrocode}
\def\HOLOGO@false{false}
%    \end{macrocode}
%    \end{macro}
%    \begin{macro}{\HOLOGO@break}
%    \begin{macrocode}
\def\HOLOGO@break{break}
%    \end{macrocode}
%    \end{macro}
%
%    \begin{macrocode}
\HOLOGO@DeclareBoolOption{break}
\HOLOGO@DeclareBoolOption{hyphenbreak}
\HOLOGO@DeclareBoolOption{spacebreak}
\HOLOGO@DeclareBoolOption{discretionarybreak}
%    \end{macrocode}
%
%    \begin{macrocode}
\kv@define@key{HoLogo}{variant}{%
  \ifx\HOLOGO@name\relax
    \@PackageError{hologo}{%
      Option `variant' is not available in \string\hologoSetup,%
      \MessageBreak
      Use \string\hologoLogoSetup\space instead%
    }\@ehc
  \else
    \edef\HOLOGO@temp{#1}%
    \ifx\HOLOGO@temp\ltx@empty
      \expandafter
      \let\csname HoLogoOpt@variant@\HOLOGO@name\endcsname\@undefined
    \else
      \ltx@IfUndefined{HoLogo@\HOLOGO@name @\HOLOGO@temp}{%
        \@PackageError{hologo}{%
          Unknown variant `\HOLOGO@temp' of logo `\HOLOGO@name'%
        }\@ehc
      }{%
        \expandafter
        \let\csname HoLogoOpt@variant@\HOLOGO@name\endcsname
            \HOLOGO@temp
      }%
    \fi
  \fi
}
%    \end{macrocode}
%
%    \begin{macro}{\HOLOGO@Variant}
%    \begin{macrocode}
\def\HOLOGO@Variant#1{%
  #1%
  \ltx@ifundefined{HoLogoOpt@variant@#1}{%
  }{%
    @\csname HoLogoOpt@variant@#1\endcsname
  }%
}
%    \end{macrocode}
%    \end{macro}
%
% \subsection{Break/no-break support}
%
%    \begin{macro}{\HOLOGO@space}
%    \begin{macrocode}
\def\HOLOGO@space{%
  \ltx@ifundefined{HoLogoOpt@spacebreak@\HOLOGO@name}{%
    \ltx@ifundefined{HoLogoOpt@break@\HOLOGO@name}{%
      \chardef\HOLOGO@temp=\HOLOGOOPT@spacebreak
    }{%
      \chardef\HOLOGO@temp=%
        \csname HoLogoOpt@break@\HOLOGO@name\endcsname
    }%
  }{%
    \chardef\HOLOGO@temp=%
      \csname HoLogoOpt@spacebreak@\HOLOGO@name\endcsname
  }%
  \ifcase\HOLOGO@temp
    \penalty10000 %
  \fi
  \ltx@space
}
%    \end{macrocode}
%    \end{macro}
%
%    \begin{macro}{\HOLOGO@hyphen}
%    \begin{macrocode}
\def\HOLOGO@hyphen{%
  \ltx@ifundefined{HoLogoOpt@hyphenbreak@\HOLOGO@name}{%
    \ltx@ifundefined{HoLogoOpt@break@\HOLOGO@name}{%
      \chardef\HOLOGO@temp=\HOLOGOOPT@hyphenbreak
    }{%
      \chardef\HOLOGO@temp=%
        \csname HoLogoOpt@break@\HOLOGO@name\endcsname
    }%
  }{%
    \chardef\HOLOGO@temp=%
      \csname HoLogoOpt@hyphenbreak@\HOLOGO@name\endcsname
  }%
  \ifcase\HOLOGO@temp
    \ltx@mbox{-}%
  \else
    -%
  \fi
}
%    \end{macrocode}
%    \end{macro}
%
%    \begin{macro}{\HOLOGO@discretionary}
%    \begin{macrocode}
\def\HOLOGO@discretionary{%
  \ltx@ifundefined{HoLogoOpt@discretionarybreak@\HOLOGO@name}{%
    \ltx@ifundefined{HoLogoOpt@break@\HOLOGO@name}{%
      \chardef\HOLOGO@temp=\HOLOGOOPT@discretionarybreak
    }{%
      \chardef\HOLOGO@temp=%
        \csname HoLogoOpt@break@\HOLOGO@name\endcsname
    }%
  }{%
    \chardef\HOLOGO@temp=%
      \csname HoLogoOpt@discretionarybreak@\HOLOGO@name\endcsname
  }%
  \ifcase\HOLOGO@temp
  \else
    \-%
  \fi
}
%    \end{macrocode}
%    \end{macro}
%
%    \begin{macro}{\HOLOGO@mbox}
%    \begin{macrocode}
\def\HOLOGO@mbox#1{%
  \ltx@ifundefined{HoLogoOpt@break@\HOLOGO@name}{%
    \chardef\HOLOGO@temp=\HOLOGOOPT@hyphenbreak
  }{%
    \chardef\HOLOGO@temp=%
      \csname HoLogoOpt@break@\HOLOGO@name\endcsname
  }%
  \ifcase\HOLOGO@temp
    \ltx@mbox{#1}%
  \else
    #1%
  \fi
}
%    \end{macrocode}
%    \end{macro}
%
% \subsection{Font support}
%
%    \begin{macro}{\HoLogoFont@font}
%    \begin{tabular}{@{}ll@{}}
%    |#1|:& logo name\\
%    |#2|:& font short name\\
%    |#3|:& text
%    \end{tabular}
%    \begin{macrocode}
\def\HoLogoFont@font#1#2#3{%
  \begingroup
    \ltx@IfUndefined{HoLogoFont@logo@#1.#2}{%
      \ltx@IfUndefined{HoLogoFont@font@#2}{%
        \@PackageWarning{hologo}{%
          Missing font `#2' for logo `#1'%
        }%
        #3%
      }{%
        \csname HoLogoFont@font@#2\endcsname{#3}%
      }%
    }{%
      \csname HoLogoFont@logo@#1.#2\endcsname{#3}%
    }%
  \endgroup
}
%    \end{macrocode}
%    \end{macro}
%
%    \begin{macro}{\HoLogoFont@Def}
%    \begin{macrocode}
\def\HoLogoFont@Def#1{%
  \expandafter\def\csname HoLogoFont@font@#1\endcsname
}
%    \end{macrocode}
%    \end{macro}
%    \begin{macro}{\HoLogoFont@LogoDef}
%    \begin{macrocode}
\def\HoLogoFont@LogoDef#1#2{%
  \expandafter\def\csname HoLogoFont@logo@#1.#2\endcsname
}
%    \end{macrocode}
%    \end{macro}
%
% \subsubsection{Font defaults}
%
%    \begin{macro}{\HoLogoFont@font@general}
%    \begin{macrocode}
\HoLogoFont@Def{general}{}%
%    \end{macrocode}
%    \end{macro}
%
%    \begin{macro}{\HoLogoFont@font@rm}
%    \begin{macrocode}
\ltx@IfUndefined{rmfamily}{%
  \ltx@IfUndefined{rm}{%
  }{%
    \HoLogoFont@Def{rm}{\rm}%
  }%
}{%
  \HoLogoFont@Def{rm}{\rmfamily}%
}
%    \end{macrocode}
%    \end{macro}
%
%    \begin{macro}{\HoLogoFont@font@sf}
%    \begin{macrocode}
\ltx@IfUndefined{sffamily}{%
  \ltx@IfUndefined{sf}{%
  }{%
    \HoLogoFont@Def{sf}{\sf}%
  }%
}{%
  \HoLogoFont@Def{sf}{\sffamily}%
}
%    \end{macrocode}
%    \end{macro}
%
%    \begin{macro}{\HoLogoFont@font@bibsf}
%    In case of \hologo{plainTeX} the original small caps
%    variant is used as default. In \hologo{LaTeX}
%    the definition of package \xpackage{dtklogos} \cite{dtklogos}
%    is used.
%\begin{quote}
%\begin{verbatim}
%\DeclareRobustCommand{\BibTeX}{%
%  B%
%  \kern-.05em%
%  \hbox{%
%    $\m@th$% %% force math size calculations
%    \csname S@\f@size\endcsname
%    \fontsize\sf@size\z@
%    \math@fontsfalse
%    \selectfont
%    I%
%    \kern-.025em%
%    B
%  }%
%  \kern-.08em%
%  \-%
%  \TeX
%}
%\end{verbatim}
%\end{quote}
%    \begin{macrocode}
\ltx@IfUndefined{selectfont}{%
  \ltx@IfUndefined{tensc}{%
    \font\tensc=cmcsc10\relax
  }{}%
  \HoLogoFont@Def{bibsf}{\tensc}%
}{%
  \HoLogoFont@Def{bibsf}{%
    $\mathsurround=0pt$%
    \csname S@\f@size\endcsname
    \fontsize\sf@size{0pt}%
    \math@fontsfalse
    \selectfont
  }%
}
%    \end{macrocode}
%    \end{macro}
%
%    \begin{macro}{\HoLogoFont@font@sc}
%    \begin{macrocode}
\ltx@IfUndefined{scshape}{%
  \ltx@IfUndefined{tensc}{%
    \font\tensc=cmcsc10\relax
  }{}%
  \HoLogoFont@Def{sc}{\tensc}%
}{%
  \HoLogoFont@Def{sc}{\scshape}%
}
%    \end{macrocode}
%    \end{macro}
%
%    \begin{macro}{\HoLogoFont@font@sy}
%    \begin{macrocode}
\ltx@IfUndefined{usefont}{%
  \ltx@IfUndefined{tensy}{%
  }{%
    \HoLogoFont@Def{sy}{\tensy}%
  }%
}{%
  \HoLogoFont@Def{sy}{%
    \usefont{OMS}{cmsy}{m}{n}%
  }%
}
%    \end{macrocode}
%    \end{macro}
%
%    \begin{macro}{\HoLogoFont@font@logo}
%    \begin{macrocode}
\begingroup
  \def\x{LaTeX2e}%
\expandafter\endgroup
\ifx\fmtname\x
  \ltx@IfUndefined{logofamily}{%
    \DeclareRobustCommand\logofamily{%
      \not@math@alphabet\logofamily\relax
      \fontencoding{U}%
      \fontfamily{logo}%
      \selectfont
    }%
  }{}%
  \ltx@IfUndefined{logofamily}{%
  }{%
    \HoLogoFont@Def{logo}{\logofamily}%
  }%
\else
  \ltx@IfUndefined{tenlogo}{%
    \font\tenlogo=logo10\relax
  }{}%
  \HoLogoFont@Def{logo}{\tenlogo}%
\fi
%    \end{macrocode}
%    \end{macro}
%
% \subsubsection{Font setup}
%
%    \begin{macro}{\hologoFontSetup}
%    \begin{macrocode}
\def\hologoFontSetup{%
  \let\HOLOGO@name\relax
  \HOLOGO@FontSetup
}
%    \end{macrocode}
%    \end{macro}
%
%    \begin{macro}{\hologoLogoFontSetup}
%    \begin{macrocode}
\def\hologoLogoFontSetup#1{%
  \edef\HOLOGO@name{#1}%
  \ltx@IfUndefined{HoLogo@\HOLOGO@name}{%
    \@PackageError{hologo}{%
      Unknown logo `\HOLOGO@name'%
    }\@ehc
    \ltx@gobble
  }{%
    \HOLOGO@FontSetup
  }%
}
%    \end{macrocode}
%    \end{macro}
%
%    \begin{macro}{\HOLOGO@FontSetup}
%    \begin{macrocode}
\def\HOLOGO@FontSetup{%
  \kvsetkeys{HoLogoFont}%
}
%    \end{macrocode}
%    \end{macro}
%
%    \begin{macrocode}
\def\HOLOGO@temp#1{%
  \kv@define@key{HoLogoFont}{#1}{%
    \ifx\HOLOGO@name\relax
      \HoLogoFont@Def{#1}{##1}%
    \else
      \HoLogoFont@LogoDef\HOLOGO@name{#1}{##1}%
    \fi
  }%
}
\HOLOGO@temp{general}
\HOLOGO@temp{sf}
%    \end{macrocode}
%
% \subsection{Generic logo commands}
%
%    \begin{macrocode}
\HOLOGO@IfExists\hologo{%
  \@PackageError{hologo}{%
    \string\hologo\ltx@space is already defined.\MessageBreak
    Package loading is aborted%
  }\@ehc
  \HOLOGO@AtEnd
}%
\HOLOGO@IfExists\hologoRobust{%
  \@PackageError{hologo}{%
    \string\hologoRobust\ltx@space is already defined.\MessageBreak
    Package loading is aborted%
  }\@ehc
  \HOLOGO@AtEnd
}%
%    \end{macrocode}
%
% \subsubsection{\cs{hologo} and friends}
%
%    \begin{macrocode}
\ifluatex
  \expandafter\ltx@firstofone
\else
  \expandafter\ltx@gobble
\fi
{%
  \ltx@IfUndefined{ifincsname}{%
    \ifnum\luatexversion<36 %
      \expandafter\ltx@gobble
    \else
      \expandafter\ltx@firstofone
    \fi
    {%
      \begingroup
        \ifcase0%
            \directlua{%
              if tex.enableprimitives then %
                tex.enableprimitives('HOLOGO@', {'ifincsname'})%
              else %
                tex.print('1')%
              end%
            }%
            \ifx\HOLOGO@ifincsname\@undefined 1\fi%
            \relax
          \expandafter\ltx@firstofone
        \else
          \endgroup
          \expandafter\ltx@gobble
        \fi
        {%
          \global\let\ifincsname\HOLOGO@ifincsname
        }%
      \HOLOGO@temp
    }%
  }{}%
}
%    \end{macrocode}
%    \begin{macrocode}
\ltx@IfUndefined{ifincsname}{%
  \catcode`$=14 %
}{%
  \catcode`$=9 %
}
%    \end{macrocode}
%
%    \begin{macro}{\hologo}
%    \begin{macrocode}
\def\hologo#1{%
$ \ifincsname
$   \ltx@ifundefined{HoLogoCs@\HOLOGO@Variant{#1}}{%
$     #1%
$   }{%
$     \csname HoLogoCs@\HOLOGO@Variant{#1}\endcsname\ltx@firstoftwo
$   }%
$ \else
    \HOLOGO@IfExists\texorpdfstring\texorpdfstring\ltx@firstoftwo
    {%
      \hologoRobust{#1}%
    }{%
      \ltx@ifundefined{HoLogoBkm@\HOLOGO@Variant{#1}}{%
        \ltx@ifundefined{HoLogo@#1}{?#1?}{#1}%
      }{%
        \csname HoLogoBkm@\HOLOGO@Variant{#1}\endcsname
        \ltx@firstoftwo
      }%
    }%
$ \fi
}
%    \end{macrocode}
%    \end{macro}
%    \begin{macro}{\Hologo}
%    \begin{macrocode}
\def\Hologo#1{%
$ \ifincsname
$   \ltx@ifundefined{HoLogoCs@\HOLOGO@Variant{#1}}{%
$     #1%
$   }{%
$     \csname HoLogoCs@\HOLOGO@Variant{#1}\endcsname\ltx@secondoftwo
$   }%
$ \else
    \HOLOGO@IfExists\texorpdfstring\texorpdfstring\ltx@firstoftwo
    {%
      \HologoRobust{#1}%
    }{%
      \ltx@ifundefined{HoLogoBkm@\HOLOGO@Variant{#1}}{%
        \ltx@ifundefined{HoLogo@#1}{?#1?}{#1}%
      }{%
        \csname HoLogoBkm@\HOLOGO@Variant{#1}\endcsname
        \ltx@secondoftwo
      }%
    }%
$ \fi
}
%    \end{macrocode}
%    \end{macro}
%
%    \begin{macro}{\hologoVariant}
%    \begin{macrocode}
\def\hologoVariant#1#2{%
  \ifx\relax#2\relax
    \hologo{#1}%
  \else
$   \ifincsname
$     \ltx@ifundefined{HoLogoCs@#1@#2}{%
$       #1%
$     }{%
$       \csname HoLogoCs@#1@#2\endcsname\ltx@firstoftwo
$     }%
$   \else
      \HOLOGO@IfExists\texorpdfstring\texorpdfstring\ltx@firstoftwo
      {%
        \hologoVariantRobust{#1}{#2}%
      }{%
        \ltx@ifundefined{HoLogoBkm@#1@#2}{%
          \ltx@ifundefined{HoLogo@#1}{?#1?}{#1}%
        }{%
          \csname HoLogoBkm@#1@#2\endcsname
          \ltx@firstoftwo
        }%
      }%
$   \fi
  \fi
}
%    \end{macrocode}
%    \end{macro}
%    \begin{macro}{\HologoVariant}
%    \begin{macrocode}
\def\HologoVariant#1#2{%
  \ifx\relax#2\relax
    \Hologo{#1}%
  \else
$   \ifincsname
$     \ltx@ifundefined{HoLogoCs@#1@#2}{%
$       #1%
$     }{%
$       \csname HoLogoCs@#1@#2\endcsname\ltx@secondoftwo
$     }%
$   \else
      \HOLOGO@IfExists\texorpdfstring\texorpdfstring\ltx@firstoftwo
      {%
        \HologoVariantRobust{#1}{#2}%
      }{%
        \ltx@ifundefined{HoLogoBkm@#1@#2}{%
          \ltx@ifundefined{HoLogo@#1}{?#1?}{#1}%
        }{%
          \csname HoLogoBkm@#1@#2\endcsname
          \ltx@secondoftwo
        }%
      }%
$   \fi
  \fi
}
%    \end{macrocode}
%    \end{macro}
%
%    \begin{macrocode}
\catcode`\$=3 %
%    \end{macrocode}
%
% \subsubsection{\cs{hologoRobust} and friends}
%
%    \begin{macro}{\hologoRobust}
%    \begin{macrocode}
\ltx@IfUndefined{protected}{%
  \ltx@IfUndefined{DeclareRobustCommand}{%
    \def\hologoRobust#1%
  }{%
    \DeclareRobustCommand*\hologoRobust[1]%
  }%
}{%
  \protected\def\hologoRobust#1%
}%
{%
  \edef\HOLOGO@name{#1}%
  \ltx@IfUndefined{HoLogo@\HOLOGO@Variant\HOLOGO@name}{%
    \@PackageError{hologo}{%
      Unknown logo `\HOLOGO@name'%
    }\@ehc
    ?\HOLOGO@name?%
  }{%
    \ltx@IfUndefined{ver@tex4ht.sty}{%
      \HoLogoFont@font\HOLOGO@name{general}{%
        \csname HoLogo@\HOLOGO@Variant\HOLOGO@name\endcsname
        \ltx@firstoftwo
      }%
    }{%
      \ltx@IfUndefined{HoLogoHtml@\HOLOGO@Variant\HOLOGO@name}{%
        \HOLOGO@name
      }{%
        \csname HoLogoHtml@\HOLOGO@Variant\HOLOGO@name\endcsname
        \ltx@firstoftwo
      }%
    }%
  }%
}
%    \end{macrocode}
%    \end{macro}
%    \begin{macro}{\HologoRobust}
%    \begin{macrocode}
\ltx@IfUndefined{protected}{%
  \ltx@IfUndefined{DeclareRobustCommand}{%
    \def\HologoRobust#1%
  }{%
    \DeclareRobustCommand*\HologoRobust[1]%
  }%
}{%
  \protected\def\HologoRobust#1%
}%
{%
  \edef\HOLOGO@name{#1}%
  \ltx@IfUndefined{HoLogo@\HOLOGO@Variant\HOLOGO@name}{%
    \@PackageError{hologo}{%
      Unknown logo `\HOLOGO@name'%
    }\@ehc
    ?\HOLOGO@name?%
  }{%
    \ltx@IfUndefined{ver@tex4ht.sty}{%
      \HoLogoFont@font\HOLOGO@name{general}{%
        \csname HoLogo@\HOLOGO@Variant\HOLOGO@name\endcsname
        \ltx@secondoftwo
      }%
    }{%
      \ltx@IfUndefined{HoLogoHtml@\HOLOGO@Variant\HOLOGO@name}{%
        \expandafter\HOLOGO@Uppercase\HOLOGO@name
      }{%
        \csname HoLogoHtml@\HOLOGO@Variant\HOLOGO@name\endcsname
        \ltx@secondoftwo
      }%
    }%
  }%
}
%    \end{macrocode}
%    \end{macro}
%    \begin{macro}{\hologoVariantRobust}
%    \begin{macrocode}
\ltx@IfUndefined{protected}{%
  \ltx@IfUndefined{DeclareRobustCommand}{%
    \def\hologoVariantRobust#1#2%
  }{%
    \DeclareRobustCommand*\hologoVariantRobust[2]%
  }%
}{%
  \protected\def\hologoVariantRobust#1#2%
}%
{%
  \begingroup
    \hologoLogoSetup{#1}{variant={#2}}%
    \hologoRobust{#1}%
  \endgroup
}
%    \end{macrocode}
%    \end{macro}
%    \begin{macro}{\HologoVariantRobust}
%    \begin{macrocode}
\ltx@IfUndefined{protected}{%
  \ltx@IfUndefined{DeclareRobustCommand}{%
    \def\HologoVariantRobust#1#2%
  }{%
    \DeclareRobustCommand*\HologoVariantRobust[2]%
  }%
}{%
  \protected\def\HologoVariantRobust#1#2%
}%
{%
  \begingroup
    \hologoLogoSetup{#1}{variant={#2}}%
    \HologoRobust{#1}%
  \endgroup
}
%    \end{macrocode}
%    \end{macro}
%
%    \begin{macro}{\hologorobust}
%    Macro \cs{hologorobust} is only defined for compatibility.
%    Its use is deprecated.
%    \begin{macrocode}
\def\hologorobust{\hologoRobust}
%    \end{macrocode}
%    \end{macro}
%
% \subsection{Helpers}
%
%    \begin{macro}{\HOLOGO@Uppercase}
%    Macro \cs{HOLOGO@Uppercase} is restricted to \cs{uppercase},
%    because \hologo{plainTeX} or \hologo{iniTeX} do not provide
%    \cs{MakeUppercase}.
%    \begin{macrocode}
\def\HOLOGO@Uppercase#1{\uppercase{#1}}
%    \end{macrocode}
%    \end{macro}
%
%    \begin{macro}{\HOLOGO@PdfdocUnicode}
%    \begin{macrocode}
\def\HOLOGO@PdfdocUnicode{%
  \ifx\ifHy@unicode\iftrue
    \expandafter\ltx@secondoftwo
  \else
    \expandafter\ltx@firstoftwo
  \fi
}
%    \end{macrocode}
%    \end{macro}
%
%    \begin{macro}{\HOLOGO@Math}
%    \begin{macrocode}
\def\HOLOGO@MathSetup{%
  \mathsurround0pt\relax
  \HOLOGO@IfExists\f@series{%
    \if b\expandafter\ltx@car\f@series x\@nil
      \csname boldmath\endcsname
   \fi
  }{}%
}
%    \end{macrocode}
%    \end{macro}
%
%    \begin{macro}{\HOLOGO@TempDimen}
%    \begin{macrocode}
\dimendef\HOLOGO@TempDimen=\ltx@zero
%    \end{macrocode}
%    \end{macro}
%    \begin{macro}{\HOLOGO@NegativeKerning}
%    \begin{macrocode}
\def\HOLOGO@NegativeKerning#1{%
  \begingroup
    \HOLOGO@TempDimen=0pt\relax
    \comma@parse@normalized{#1}{%
      \ifdim\HOLOGO@TempDimen=0pt %
        \expandafter\HOLOGO@@NegativeKerning\comma@entry
      \fi
      \ltx@gobble
    }%
    \ifdim\HOLOGO@TempDimen<0pt %
      \kern\HOLOGO@TempDimen
    \fi
  \endgroup
}
%    \end{macrocode}
%    \end{macro}
%    \begin{macro}{\HOLOGO@@NegativeKerning}
%    \begin{macrocode}
\def\HOLOGO@@NegativeKerning#1#2{%
  \setbox\ltx@zero\hbox{#1#2}%
  \HOLOGO@TempDimen=\wd\ltx@zero
  \setbox\ltx@zero\hbox{#1\kern0pt#2}%
  \advance\HOLOGO@TempDimen by -\wd\ltx@zero
}
%    \end{macrocode}
%    \end{macro}
%
%    \begin{macro}{\HOLOGO@SpaceFactor}
%    \begin{macrocode}
\def\HOLOGO@SpaceFactor{%
  \spacefactor1000 %
}
%    \end{macrocode}
%    \end{macro}
%
%    \begin{macro}{\HOLOGO@Span}
%    \begin{macrocode}
\def\HOLOGO@Span#1#2{%
  \HCode{<span class="HoLogo-#1">}%
  #2%
  \HCode{</span>}%
}
%    \end{macrocode}
%    \end{macro}
%
% \subsubsection{Text subscript}
%
%    \begin{macro}{\HOLOGO@SubScript}%
%    \begin{macrocode}
\def\HOLOGO@SubScript#1{%
  \ltx@IfUndefined{textsubscript}{%
    \ltx@IfUndefined{text}{%
      \ltx@mbox{%
        \mathsurround=0pt\relax
        $%
          _{%
            \ltx@IfUndefined{sf@size}{%
              \mathrm{#1}%
            }{%
              \mbox{%
                \fontsize\sf@size{0pt}\selectfont
                #1%
              }%
            }%
          }%
        $%
      }%
    }{%
      \ltx@mbox{%
        \mathsurround=0pt\relax
        $_{\text{#1}}$%
      }%
    }%
  }{%
    \textsubscript{#1}%
  }%
}
%    \end{macrocode}
%    \end{macro}
%
% \subsection{\hologo{TeX} and friends}
%
% \subsubsection{\hologo{TeX}}
%
%    \begin{macro}{\HoLogo@TeX}
%    Source: \hologo{LaTeX} kernel.
%    \begin{macrocode}
\def\HoLogo@TeX#1{%
  T\kern-.1667em\lower.5ex\hbox{E}\kern-.125emX\HOLOGO@SpaceFactor
}
%    \end{macrocode}
%    \end{macro}
%    \begin{macro}{\HoLogoHtml@TeX}
%    \begin{macrocode}
\def\HoLogoHtml@TeX#1{%
  \HoLogoCss@TeX
  \HOLOGO@Span{TeX}{%
    T%
    \HOLOGO@Span{e}{%
      E%
    }%
    X%
  }%
}
%    \end{macrocode}
%    \end{macro}
%    \begin{macro}{\HoLogoCss@TeX}
%    \begin{macrocode}
\def\HoLogoCss@TeX{%
  \Css{%
    span.HoLogo-TeX span.HoLogo-e{%
      position:relative;%
      top:.5ex;%
      margin-left:-.1667em;%
      margin-right:-.125em;%
    }%
  }%
  \Css{%
    a span.HoLogo-TeX span.HoLogo-e{%
      text-decoration:none;%
    }%
  }%
  \global\let\HoLogoCss@TeX\relax
}
%    \end{macrocode}
%    \end{macro}
%
% \subsubsection{\hologo{plainTeX}}
%
%    \begin{macro}{\HoLogo@plainTeX@space}
%    Source: ``The \hologo{TeX}book''
%    \begin{macrocode}
\def\HoLogo@plainTeX@space#1{%
  \HOLOGO@mbox{#1{p}{P}lain}\HOLOGO@space\hologo{TeX}%
}
%    \end{macrocode}
%    \end{macro}
%    \begin{macro}{\HoLogoCs@plainTeX@space}
%    \begin{macrocode}
\def\HoLogoCs@plainTeX@space#1{#1{p}{P}lain TeX}%
%    \end{macrocode}
%    \end{macro}
%    \begin{macro}{\HoLogoBkm@plainTeX@space}
%    \begin{macrocode}
\def\HoLogoBkm@plainTeX@space#1{%
  #1{p}{P}lain \hologo{TeX}%
}
%    \end{macrocode}
%    \end{macro}
%    \begin{macro}{\HoLogoHtml@plainTeX@space}
%    \begin{macrocode}
\def\HoLogoHtml@plainTeX@space#1{%
  #1{p}{P}lain \hologo{TeX}%
}
%    \end{macrocode}
%    \end{macro}
%
%    \begin{macro}{\HoLogo@plainTeX@hyphen}
%    \begin{macrocode}
\def\HoLogo@plainTeX@hyphen#1{%
  \HOLOGO@mbox{#1{p}{P}lain}\HOLOGO@hyphen\hologo{TeX}%
}
%    \end{macrocode}
%    \end{macro}
%    \begin{macro}{\HoLogoCs@plainTeX@hyphen}
%    \begin{macrocode}
\def\HoLogoCs@plainTeX@hyphen#1{#1{p}{P}lain-TeX}
%    \end{macrocode}
%    \end{macro}
%    \begin{macro}{\HoLogoBkm@plainTeX@hyphen}
%    \begin{macrocode}
\def\HoLogoBkm@plainTeX@hyphen#1{%
  #1{p}{P}lain-\hologo{TeX}%
}
%    \end{macrocode}
%    \end{macro}
%    \begin{macro}{\HoLogoHtml@plainTeX@hyphen}
%    \begin{macrocode}
\def\HoLogoHtml@plainTeX@hyphen#1{%
  #1{p}{P}lain-\hologo{TeX}%
}
%    \end{macrocode}
%    \end{macro}
%
%    \begin{macro}{\HoLogo@plainTeX@runtogether}
%    \begin{macrocode}
\def\HoLogo@plainTeX@runtogether#1{%
  \HOLOGO@mbox{#1{p}{P}lain\hologo{TeX}}%
}
%    \end{macrocode}
%    \end{macro}
%    \begin{macro}{\HoLogoCs@plainTeX@runtogether}
%    \begin{macrocode}
\def\HoLogoCs@plainTeX@runtogether#1{#1{p}{P}lainTeX}
%    \end{macrocode}
%    \end{macro}
%    \begin{macro}{\HoLogoBkm@plainTeX@runtogether}
%    \begin{macrocode}
\def\HoLogoBkm@plainTeX@runtogether#1{%
  #1{p}{P}lain\hologo{TeX}%
}
%    \end{macrocode}
%    \end{macro}
%    \begin{macro}{\HoLogoHtml@plainTeX@runtogether}
%    \begin{macrocode}
\def\HoLogoHtml@plainTeX@runtogether#1{%
  #1{p}{P}lain\hologo{TeX}%
}
%    \end{macrocode}
%    \end{macro}
%
%    \begin{macro}{\HoLogo@plainTeX}
%    \begin{macrocode}
\def\HoLogo@plainTeX{\HoLogo@plainTeX@space}
%    \end{macrocode}
%    \end{macro}
%    \begin{macro}{\HoLogoCs@plainTeX}
%    \begin{macrocode}
\def\HoLogoCs@plainTeX{\HoLogoCs@plainTeX@space}
%    \end{macrocode}
%    \end{macro}
%    \begin{macro}{\HoLogoBkm@plainTeX}
%    \begin{macrocode}
\def\HoLogoBkm@plainTeX{\HoLogoBkm@plainTeX@space}
%    \end{macrocode}
%    \end{macro}
%    \begin{macro}{\HoLogoHtml@plainTeX}
%    \begin{macrocode}
\def\HoLogoHtml@plainTeX{\HoLogoHtml@plainTeX@space}
%    \end{macrocode}
%    \end{macro}
%
% \subsubsection{\hologo{LaTeX}}
%
%    Source: \hologo{LaTeX} kernel.
%\begin{quote}
%\begin{verbatim}
%\DeclareRobustCommand{\LaTeX}{%
%  L%
%  \kern-.36em%
%  {%
%    \sbox\z@ T%
%    \vbox to\ht\z@{%
%      \hbox{%
%        \check@mathfonts
%        \fontsize\sf@size\z@
%        \math@fontsfalse
%        \selectfont
%        A%
%      }%
%      \vss
%    }%
%  }%
%  \kern-.15em%
%  \TeX
%}
%\end{verbatim}
%\end{quote}
%
%    \begin{macro}{\HoLogo@La}
%    \begin{macrocode}
\def\HoLogo@La#1{%
  L%
  \kern-.36em%
  \begingroup
    \setbox\ltx@zero\hbox{T}%
    \vbox to\ht\ltx@zero{%
      \hbox{%
        \ltx@ifundefined{check@mathfonts}{%
          \csname sevenrm\endcsname
        }{%
          \check@mathfonts
          \fontsize\sf@size{0pt}%
          \math@fontsfalse\selectfont
        }%
        A%
      }%
      \vss
    }%
  \endgroup
}
%    \end{macrocode}
%    \end{macro}
%
%    \begin{macro}{\HoLogo@LaTeX}
%    Source: \hologo{LaTeX} kernel.
%    \begin{macrocode}
\def\HoLogo@LaTeX#1{%
  \hologo{La}%
  \kern-.15em%
  \hologo{TeX}%
}
%    \end{macrocode}
%    \end{macro}
%    \begin{macro}{\HoLogoHtml@LaTeX}
%    \begin{macrocode}
\def\HoLogoHtml@LaTeX#1{%
  \HoLogoCss@LaTeX
  \HOLOGO@Span{LaTeX}{%
    L%
    \HOLOGO@Span{a}{%
      A%
    }%
    \hologo{TeX}%
  }%
}
%    \end{macrocode}
%    \end{macro}
%    \begin{macro}{\HoLogoCss@LaTeX}
%    \begin{macrocode}
\def\HoLogoCss@LaTeX{%
  \Css{%
    span.HoLogo-LaTeX span.HoLogo-a{%
      position:relative;%
      top:-.5ex;%
      margin-left:-.36em;%
      margin-right:-.15em;%
      font-size:85\%;%
    }%
  }%
  \global\let\HoLogoCss@LaTeX\relax
}
%    \end{macrocode}
%    \end{macro}
%
% \subsubsection{\hologo{(La)TeX}}
%
%    \begin{macro}{\HoLogo@LaTeXTeX}
%    The kerning around the parentheses is taken
%    from package \xpackage{dtklogos} \cite{dtklogos}.
%\begin{quote}
%\begin{verbatim}
%\DeclareRobustCommand{\LaTeXTeX}{%
%  (%
%  \kern-.15em%
%  L%
%  \kern-.36em%
%  {%
%    \sbox\z@ T%
%    \vbox to\ht0{%
%      \hbox{%
%        $\m@th$%
%        \csname S@\f@size\endcsname
%        \fontsize\sf@size\z@
%        \math@fontsfalse
%        \selectfont
%        A%
%      }%
%      \vss
%    }%
%  }%
%  \kern-.2em%
%  )%
%  \kern-.15em%
%  \TeX
%}
%\end{verbatim}
%\end{quote}
%    \begin{macrocode}
\def\HoLogo@LaTeXTeX#1{%
  (%
  \kern-.15em%
  \hologo{La}%
  \kern-.2em%
  )%
  \kern-.15em%
  \hologo{TeX}%
}
%    \end{macrocode}
%    \end{macro}
%    \begin{macro}{\HoLogoBkm@LaTeXTeX}
%    \begin{macrocode}
\def\HoLogoBkm@LaTeXTeX#1{(La)TeX}
%    \end{macrocode}
%    \end{macro}
%
%    \begin{macro}{\HoLogo@(La)TeX}
%    \begin{macrocode}
\expandafter
\let\csname HoLogo@(La)TeX\endcsname\HoLogo@LaTeXTeX
%    \end{macrocode}
%    \end{macro}
%    \begin{macro}{\HoLogoBkm@(La)TeX}
%    \begin{macrocode}
\expandafter
\let\csname HoLogoBkm@(La)TeX\endcsname\HoLogoBkm@LaTeXTeX
%    \end{macrocode}
%    \end{macro}
%    \begin{macro}{\HoLogoHtml@LaTeXTeX}
%    \begin{macrocode}
\def\HoLogoHtml@LaTeXTeX#1{%
  \HoLogoCss@LaTeXTeX
  \HOLOGO@Span{LaTeXTeX}{%
    (%
    \HOLOGO@Span{L}{L}%
    \HOLOGO@Span{a}{A}%
    \HOLOGO@Span{ParenRight}{)}%
    \hologo{TeX}%
  }%
}
%    \end{macrocode}
%    \end{macro}
%    \begin{macro}{\HoLogoHtml@(La)TeX}
%    Kerning after opening parentheses and before closing parentheses
%    is $-0.1$\,em. The original values $-0.15$\,em
%    looked too ugly for a serif font.
%    \begin{macrocode}
\expandafter
\let\csname HoLogoHtml@(La)TeX\endcsname\HoLogoHtml@LaTeXTeX
%    \end{macrocode}
%    \end{macro}
%    \begin{macro}{\HoLogoCss@LaTeXTeX}
%    \begin{macrocode}
\def\HoLogoCss@LaTeXTeX{%
  \Css{%
    span.HoLogo-LaTeXTeX span.HoLogo-L{%
      margin-left:-.1em;%
    }%
  }%
  \Css{%
    span.HoLogo-LaTeXTeX span.HoLogo-a{%
      position:relative;%
      top:-.5ex;%
      margin-left:-.36em;%
      margin-right:-.1em;%
      font-size:85\%;%
    }%
  }%
  \Css{%
    span.HoLogo-LaTeXTeX span.HoLogo-ParenRight{%
      margin-right:-.15em;%
    }%
  }%
  \global\let\HoLogoCss@LaTeXTeX\relax
}
%    \end{macrocode}
%    \end{macro}
%
% \subsubsection{\hologo{LaTeXe}}
%
%    \begin{macro}{\HoLogo@LaTeXe}
%    Source: \hologo{LaTeX} kernel
%    \begin{macrocode}
\def\HoLogo@LaTeXe#1{%
  \hologo{LaTeX}%
  \kern.15em%
  \hbox{%
    \HOLOGO@MathSetup
    2%
    $_{\textstyle\varepsilon}$%
  }%
}
%    \end{macrocode}
%    \end{macro}
%
%    \begin{macro}{\HoLogoCs@LaTeXe}
%    \begin{macrocode}
\ifnum64=`\^^^^0040\relax % test for big chars of LuaTeX/XeTeX
  \catcode`\$=9 %
  \catcode`\&=14 %
\else
  \catcode`\$=14 %
  \catcode`\&=9 %
\fi
\def\HoLogoCs@LaTeXe#1{%
  LaTeX2%
$ \string ^^^^0395%
& e%
}%
\catcode`\$=3 %
\catcode`\&=4 %
%    \end{macrocode}
%    \end{macro}
%
%    \begin{macro}{\HoLogoBkm@LaTeXe}
%    \begin{macrocode}
\def\HoLogoBkm@LaTeXe#1{%
  \hologo{LaTeX}%
  2%
  \HOLOGO@PdfdocUnicode{e}{\textepsilon}%
}
%    \end{macrocode}
%    \end{macro}
%
%    \begin{macro}{\HoLogoHtml@LaTeXe}
%    \begin{macrocode}
\def\HoLogoHtml@LaTeXe#1{%
  \HoLogoCss@LaTeXe
  \HOLOGO@Span{LaTeX2e}{%
    \hologo{LaTeX}%
    \HOLOGO@Span{2}{2}%
    \HOLOGO@Span{e}{%
      \HOLOGO@MathSetup
      \ensuremath{\textstyle\varepsilon}%
    }%
  }%
}
%    \end{macrocode}
%    \end{macro}
%    \begin{macro}{\HoLogoCss@LaTeXe}
%    \begin{macrocode}
\def\HoLogoCss@LaTeXe{%
  \Css{%
    span.HoLogo-LaTeX2e span.HoLogo-2{%
      padding-left:.15em;%
    }%
  }%
  \Css{%
    span.HoLogo-LaTeX2e span.HoLogo-e{%
      position:relative;%
      top:.35ex;%
      text-decoration:none;%
    }%
  }%
  \global\let\HoLogoCss@LaTeXe\relax
}
%    \end{macrocode}
%    \end{macro}
%
%    \begin{macro}{\HoLogo@LaTeX2e}
%    \begin{macrocode}
\expandafter
\let\csname HoLogo@LaTeX2e\endcsname\HoLogo@LaTeXe
%    \end{macrocode}
%    \end{macro}
%    \begin{macro}{\HoLogoCs@LaTeX2e}
%    \begin{macrocode}
\expandafter
\let\csname HoLogoCs@LaTeX2e\endcsname\HoLogoCs@LaTeXe
%    \end{macrocode}
%    \end{macro}
%    \begin{macro}{\HoLogoBkm@LaTeX2e}
%    \begin{macrocode}
\expandafter
\let\csname HoLogoBkm@LaTeX2e\endcsname\HoLogoBkm@LaTeXe
%    \end{macrocode}
%    \end{macro}
%    \begin{macro}{\HoLogoHtml@LaTeX2e}
%    \begin{macrocode}
\expandafter
\let\csname HoLogoHtml@LaTeX2e\endcsname\HoLogoHtml@LaTeXe
%    \end{macrocode}
%    \end{macro}
%
% \subsubsection{\hologo{LaTeX3}}
%
%    \begin{macro}{\HoLogo@LaTeX3}
%    Source: \hologo{LaTeX} kernel
%    \begin{macrocode}
\expandafter\def\csname HoLogo@LaTeX3\endcsname#1{%
  \hologo{LaTeX}%
  3%
}
%    \end{macrocode}
%    \end{macro}
%
%    \begin{macro}{\HoLogoBkm@LaTeX3}
%    \begin{macrocode}
\expandafter\def\csname HoLogoBkm@LaTeX3\endcsname#1{%
  \hologo{LaTeX}%
  3%
}
%    \end{macrocode}
%    \end{macro}
%    \begin{macro}{\HoLogoHtml@LaTeX3}
%    \begin{macrocode}
\expandafter
\let\csname HoLogoHtml@LaTeX3\expandafter\endcsname
\csname HoLogo@LaTeX3\endcsname
%    \end{macrocode}
%    \end{macro}
%
% \subsubsection{\hologo{LaTeXML}}
%
%    \begin{macro}{\HoLogo@LaTeXML}
%    \begin{macrocode}
\def\HoLogo@LaTeXML#1{%
  \HOLOGO@mbox{%
    \hologo{La}%
    \kern-.15em%
    T%
    \kern-.1667em%
    \lower.5ex\hbox{E}%
    \kern-.125em%
    \HoLogoFont@font{LaTeXML}{sc}{xml}%
  }%
}
%    \end{macrocode}
%    \end{macro}
%    \begin{macro}{\HoLogoHtml@pdfLaTeX}
%    \begin{macrocode}
\def\HoLogoHtml@LaTeXML#1{%
  \HOLOGO@Span{LaTeXML}{%
    \HoLogoCss@LaTeX
    \HoLogoCss@TeX
    \HOLOGO@Span{LaTeX}{%
      L%
      \HOLOGO@Span{a}{%
        A%
      }%
    }%
    \HOLOGO@Span{TeX}{%
      T%
      \HOLOGO@Span{e}{%
        E%
      }%
    }%
    \HCode{<span style="font-variant: small-caps;">}%
    xml%
    \HCode{</span>}%
  }%
}
%    \end{macrocode}
%    \end{macro}
%
% \subsubsection{\hologo{eTeX}}
%
%    \begin{macro}{\HoLogo@eTeX}
%    Source: package \xpackage{etex}
%    \begin{macrocode}
\def\HoLogo@eTeX#1{%
  \ltx@mbox{%
    \HOLOGO@MathSetup
    $\varepsilon$%
    -%
    \HOLOGO@NegativeKerning{-T,T-,To}%
    \hologo{TeX}%
  }%
}
%    \end{macrocode}
%    \end{macro}
%    \begin{macro}{\HoLogoCs@eTeX}
%    \begin{macrocode}
\ifnum64=`\^^^^0040\relax % test for big chars of LuaTeX/XeTeX
  \catcode`\$=9 %
  \catcode`\&=14 %
\else
  \catcode`\$=14 %
  \catcode`\&=9 %
\fi
\def\HoLogoCs@eTeX#1{%
$ #1{\string ^^^^0395}{\string ^^^^03b5}%
& #1{e}{E}%
  TeX%
}%
\catcode`\$=3 %
\catcode`\&=4 %
%    \end{macrocode}
%    \end{macro}
%    \begin{macro}{\HoLogoBkm@eTeX}
%    \begin{macrocode}
\def\HoLogoBkm@eTeX#1{%
  \HOLOGO@PdfdocUnicode{#1{e}{E}}{\textepsilon}%
  -%
  \hologo{TeX}%
}
%    \end{macrocode}
%    \end{macro}
%    \begin{macro}{\HoLogoHtml@eTeX}
%    \begin{macrocode}
\def\HoLogoHtml@eTeX#1{%
  \ltx@mbox{%
    \HOLOGO@MathSetup
    $\varepsilon$%
    -%
    \hologo{TeX}%
  }%
}
%    \end{macrocode}
%    \end{macro}
%
% \subsubsection{\hologo{iniTeX}}
%
%    \begin{macro}{\HoLogo@iniTeX}
%    \begin{macrocode}
\def\HoLogo@iniTeX#1{%
  \HOLOGO@mbox{%
    #1{i}{I}ni\hologo{TeX}%
  }%
}
%    \end{macrocode}
%    \end{macro}
%    \begin{macro}{\HoLogoCs@iniTeX}
%    \begin{macrocode}
\def\HoLogoCs@iniTeX#1{#1{i}{I}niTeX}
%    \end{macrocode}
%    \end{macro}
%    \begin{macro}{\HoLogoBkm@iniTeX}
%    \begin{macrocode}
\def\HoLogoBkm@iniTeX#1{%
  #1{i}{I}ni\hologo{TeX}%
}
%    \end{macrocode}
%    \end{macro}
%    \begin{macro}{\HoLogoHtml@iniTeX}
%    \begin{macrocode}
\let\HoLogoHtml@iniTeX\HoLogo@iniTeX
%    \end{macrocode}
%    \end{macro}
%
% \subsubsection{\hologo{virTeX}}
%
%    \begin{macro}{\HoLogo@virTeX}
%    \begin{macrocode}
\def\HoLogo@virTeX#1{%
  \HOLOGO@mbox{%
    #1{v}{V}ir\hologo{TeX}%
  }%
}
%    \end{macrocode}
%    \end{macro}
%    \begin{macro}{\HoLogoCs@virTeX}
%    \begin{macrocode}
\def\HoLogoCs@virTeX#1{#1{v}{V}irTeX}
%    \end{macrocode}
%    \end{macro}
%    \begin{macro}{\HoLogoBkm@virTeX}
%    \begin{macrocode}
\def\HoLogoBkm@virTeX#1{%
  #1{v}{V}ir\hologo{TeX}%
}
%    \end{macrocode}
%    \end{macro}
%    \begin{macro}{\HoLogoHtml@virTeX}
%    \begin{macrocode}
\let\HoLogoHtml@virTeX\HoLogo@virTeX
%    \end{macrocode}
%    \end{macro}
%
% \subsubsection{\hologo{SliTeX}}
%
% \paragraph{Definitions of the three variants.}
%
%    \begin{macro}{\HoLogo@SLiTeX@lift}
%    \begin{macrocode}
\def\HoLogo@SLiTeX@lift#1{%
  \HoLogoFont@font{SliTeX}{rm}{%
    S%
    \kern-.06em%
    L%
    \kern-.18em%
    \raise.32ex\hbox{\HoLogoFont@font{SliTeX}{sc}{i}}%
    \HOLOGO@discretionary
    \kern-.06em%
    \hologo{TeX}%
  }%
}
%    \end{macrocode}
%    \end{macro}
%    \begin{macro}{\HoLogoBkm@SLiTeX@lift}
%    \begin{macrocode}
\def\HoLogoBkm@SLiTeX@lift#1{SLiTeX}
%    \end{macrocode}
%    \end{macro}
%    \begin{macro}{\HoLogoHtml@SLiTeX@lift}
%    \begin{macrocode}
\def\HoLogoHtml@SLiTeX@lift#1{%
  \HoLogoCss@SLiTeX@lift
  \HOLOGO@Span{SLiTeX-lift}{%
    \HoLogoFont@font{SliTeX}{rm}{%
      S%
      \HOLOGO@Span{L}{L}%
      \HOLOGO@Span{i}{i}%
      \hologo{TeX}%
    }%
  }%
}
%    \end{macrocode}
%    \end{macro}
%    \begin{macro}{\HoLogoCss@SLiTeX@lift}
%    \begin{macrocode}
\def\HoLogoCss@SLiTeX@lift{%
  \Css{%
    span.HoLogo-SLiTeX-lift span.HoLogo-L{%
      margin-left:-.06em;%
      margin-right:-.18em;%
    }%
  }%
  \Css{%
    span.HoLogo-SLiTeX-lift span.HoLogo-i{%
      position:relative;%
      top:-.32ex;%
      margin-right:-.06em;%
      font-variant:small-caps;%
    }%
  }%
  \global\let\HoLogoCss@SLiTeX@lift\relax
}
%    \end{macrocode}
%    \end{macro}
%
%    \begin{macro}{\HoLogo@SliTeX@simple}
%    \begin{macrocode}
\def\HoLogo@SliTeX@simple#1{%
  \HoLogoFont@font{SliTeX}{rm}{%
    \ltx@mbox{%
      \HoLogoFont@font{SliTeX}{sc}{Sli}%
    }%
    \HOLOGO@discretionary
    \hologo{TeX}%
  }%
}
%    \end{macrocode}
%    \end{macro}
%    \begin{macro}{\HoLogoBkm@SliTeX@simple}
%    \begin{macrocode}
\def\HoLogoBkm@SliTeX@simple#1{SliTeX}
%    \end{macrocode}
%    \end{macro}
%    \begin{macro}{\HoLogoHtml@SliTeX@simple}
%    \begin{macrocode}
\let\HoLogoHtml@SliTeX@simple\HoLogo@SliTeX@simple
%    \end{macrocode}
%    \end{macro}
%
%    \begin{macro}{\HoLogo@SliTeX@narrow}
%    \begin{macrocode}
\def\HoLogo@SliTeX@narrow#1{%
  \HoLogoFont@font{SliTeX}{rm}{%
    \ltx@mbox{%
      S%
      \kern-.06em%
      \HoLogoFont@font{SliTeX}{sc}{%
        l%
        \kern-.035em%
        i%
      }%
    }%
    \HOLOGO@discretionary
    \kern-.06em%
    \hologo{TeX}%
  }%
}
%    \end{macrocode}
%    \end{macro}
%    \begin{macro}{\HoLogoBkm@SliTeX@narrow}
%    \begin{macrocode}
\def\HoLogoBkm@SliTeX@narrow#1{SliTeX}
%    \end{macrocode}
%    \end{macro}
%    \begin{macro}{\HoLogoHtml@SliTeX@narrow}
%    \begin{macrocode}
\def\HoLogoHtml@SliTeX@narrow#1{%
  \HoLogoCss@SliTeX@narrow
  \HOLOGO@Span{SliTeX-narrow}{%
    \HoLogoFont@font{SliTeX}{rm}{%
      S%
        \HOLOGO@Span{l}{l}%
        \HOLOGO@Span{i}{i}%
      \hologo{TeX}%
    }%
  }%
}
%    \end{macrocode}
%    \end{macro}
%    \begin{macro}{\HoLogoCss@SliTeX@narrow}
%    \begin{macrocode}
\def\HoLogoCss@SliTeX@narrow{%
  \Css{%
    span.HoLogo-SliTeX-narrow span.HoLogo-l{%
      margin-left:-.06em;%
      margin-right:-.035em;%
      font-variant:small-caps;%
    }%
  }%
  \Css{%
    span.HoLogo-SliTeX-narrow span.HoLogo-i{%
      margin-right:-.06em;%
      font-variant:small-caps;%
    }%
  }%
  \global\let\HoLogoCss@SliTeX@narrow\relax
}
%    \end{macrocode}
%    \end{macro}
%
% \paragraph{Macro set completion.}
%
%    \begin{macro}{\HoLogo@SLiTeX@simple}
%    \begin{macrocode}
\def\HoLogo@SLiTeX@simple{\HoLogo@SliTeX@simple}
%    \end{macrocode}
%    \end{macro}
%    \begin{macro}{\HoLogoBkm@SLiTeX@simple}
%    \begin{macrocode}
\def\HoLogoBkm@SLiTeX@simple{\HoLogoBkm@SliTeX@simple}
%    \end{macrocode}
%    \end{macro}
%    \begin{macro}{\HoLogoHtml@SLiTeX@simple}
%    \begin{macrocode}
\def\HoLogoHtml@SLiTeX@simple{\HoLogoHtml@SliTeX@simple}
%    \end{macrocode}
%    \end{macro}
%
%    \begin{macro}{\HoLogo@SLiTeX@narrow}
%    \begin{macrocode}
\def\HoLogo@SLiTeX@narrow{\HoLogo@SliTeX@narrow}
%    \end{macrocode}
%    \end{macro}
%    \begin{macro}{\HoLogoBkm@SLiTeX@narrow}
%    \begin{macrocode}
\def\HoLogoBkm@SLiTeX@narrow{\HoLogoBkm@SliTeX@narrow}
%    \end{macrocode}
%    \end{macro}
%    \begin{macro}{\HoLogoHtml@SLiTeX@narrow}
%    \begin{macrocode}
\def\HoLogoHtml@SLiTeX@narrow{\HoLogoHtml@SliTeX@narrow}
%    \end{macrocode}
%    \end{macro}
%
%    \begin{macro}{\HoLogo@SliTeX@lift}
%    \begin{macrocode}
\def\HoLogo@SliTeX@lift{\HoLogo@SLiTeX@lift}
%    \end{macrocode}
%    \end{macro}
%    \begin{macro}{\HoLogoBkm@SliTeX@lift}
%    \begin{macrocode}
\def\HoLogoBkm@SliTeX@lift{\HoLogoBkm@SLiTeX@lift}
%    \end{macrocode}
%    \end{macro}
%    \begin{macro}{\HoLogoHtml@SliTeX@lift}
%    \begin{macrocode}
\def\HoLogoHtml@SliTeX@lift{\HoLogoHtml@SLiTeX@lift}
%    \end{macrocode}
%    \end{macro}
%
% \paragraph{Defaults.}
%
%    \begin{macro}{\HoLogo@SLiTeX}
%    \begin{macrocode}
\def\HoLogo@SLiTeX{\HoLogo@SLiTeX@lift}
%    \end{macrocode}
%    \end{macro}
%    \begin{macro}{\HoLogoBkm@SLiTeX}
%    \begin{macrocode}
\def\HoLogoBkm@SLiTeX{\HoLogoBkm@SLiTeX@lift}
%    \end{macrocode}
%    \end{macro}
%    \begin{macro}{\HoLogoHtml@SLiTeX}
%    \begin{macrocode}
\def\HoLogoHtml@SLiTeX{\HoLogoHtml@SLiTeX@lift}
%    \end{macrocode}
%    \end{macro}
%
%    \begin{macro}{\HoLogo@SliTeX}
%    \begin{macrocode}
\def\HoLogo@SliTeX{\HoLogo@SliTeX@narrow}
%    \end{macrocode}
%    \end{macro}
%    \begin{macro}{\HoLogoBkm@SliTeX}
%    \begin{macrocode}
\def\HoLogoBkm@SliTeX{\HoLogoBkm@SliTeX@narrow}
%    \end{macrocode}
%    \end{macro}
%    \begin{macro}{\HoLogoHtml@SliTeX}
%    \begin{macrocode}
\def\HoLogoHtml@SliTeX{\HoLogoHtml@SliTeX@narrow}
%    \end{macrocode}
%    \end{macro}
%
% \subsubsection{\hologo{LuaTeX}}
%
%    \begin{macro}{\HoLogo@LuaTeX}
%    The kerning is an idea of Hans Hagen, see mailing list
%    `luatex at tug dot org' in March 2010.
%    \begin{macrocode}
\def\HoLogo@LuaTeX#1{%
  \HOLOGO@mbox{%
    Lua%
    \HOLOGO@NegativeKerning{aT,oT,To}%
    \hologo{TeX}%
  }%
}
%    \end{macrocode}
%    \end{macro}
%    \begin{macro}{\HoLogoHtml@LuaTeX}
%    \begin{macrocode}
\let\HoLogoHtml@LuaTeX\HoLogo@LuaTeX
%    \end{macrocode}
%    \end{macro}
%
% \subsubsection{\hologo{LuaLaTeX}}
%
%    \begin{macro}{\HoLogo@LuaLaTeX}
%    \begin{macrocode}
\def\HoLogo@LuaLaTeX#1{%
  \HOLOGO@mbox{%
    Lua%
    \hologo{LaTeX}%
  }%
}
%    \end{macrocode}
%    \end{macro}
%    \begin{macro}{\HoLogoHtml@LuaLaTeX}
%    \begin{macrocode}
\let\HoLogoHtml@LuaLaTeX\HoLogo@LuaLaTeX
%    \end{macrocode}
%    \end{macro}
%
% \subsubsection{\hologo{XeTeX}, \hologo{XeLaTeX}}
%
%    \begin{macro}{\HOLOGO@IfCharExists}
%    \begin{macrocode}
\ifluatex
  \ifnum\luatexversion<36 %
  \else
    \def\HOLOGO@IfCharExists#1{%
      \ifnum
        \directlua{%
           if luaotfload and luaotfload.aux then
             if luaotfload.aux.font_has_glyph(%
                    font.current(), \number#1) then % 	 
	       tex.print("1") % 	 
	     end % 	 
	   elseif font and font.fonts and font.current then %
            local f = font.fonts[font.current()]%
            if f.characters and f.characters[\number#1] then %
              tex.print("1")%
            end %
          end%
        }0=\ltx@zero
        \expandafter\ltx@secondoftwo
      \else
        \expandafter\ltx@firstoftwo
      \fi
    }%
  \fi
\fi
\ltx@IfUndefined{HOLOGO@IfCharExists}{%
  \def\HOLOGO@@IfCharExists#1{%
    \begingroup
      \tracinglostchars=\ltx@zero
      \setbox\ltx@zero=\hbox{%
        \kern7sp\char#1\relax
        \ifnum\lastkern>\ltx@zero
          \expandafter\aftergroup\csname iffalse\endcsname
        \else
          \expandafter\aftergroup\csname iftrue\endcsname
        \fi
      }%
      % \if{true|false} from \aftergroup
      \endgroup
      \expandafter\ltx@firstoftwo
    \else
      \endgroup
      \expandafter\ltx@secondoftwo
    \fi
  }%
  \ifxetex
    \ltx@IfUndefined{XeTeXfonttype}{}{%
      \ltx@IfUndefined{XeTeXcharglyph}{}{%
        \def\HOLOGO@IfCharExists#1{%
          \ifnum\XeTeXfonttype\font>\ltx@zero
            \expandafter\ltx@firstofthree
          \else
            \expandafter\ltx@gobble
          \fi
          {%
            \ifnum\XeTeXcharglyph#1>\ltx@zero
              \expandafter\ltx@firstoftwo
            \else
              \expandafter\ltx@secondoftwo
            \fi
          }%
          \HOLOGO@@IfCharExists{#1}%
        }%
      }%
    }%
  \fi
}{}
\ltx@ifundefined{HOLOGO@IfCharExists}{%
  \ifnum64=`\^^^^0040\relax % test for big chars of LuaTeX/XeTeX
    \let\HOLOGO@IfCharExists\HOLOGO@@IfCharExists
  \else
    \def\HOLOGO@IfCharExists#1{%
      \ifnum#1>255 %
        \expandafter\ltx@fourthoffour
      \fi
      \HOLOGO@@IfCharExists{#1}%
    }%
  \fi
}{}
%    \end{macrocode}
%    \end{macro}
%
%    \begin{macro}{\HoLogo@Xe}
%    Source: package \xpackage{dtklogos}
%    \begin{macrocode}
\def\HoLogo@Xe#1{%
  X%
  \kern-.1em\relax
  \HOLOGO@IfCharExists{"018E}{%
    \lower.5ex\hbox{\char"018E}%
  }{%
    \chardef\HOLOGO@choice=\ltx@zero
    \ifdim\fontdimen\ltx@one\font>0pt %
      \ltx@IfUndefined{rotatebox}{%
        \ltx@IfUndefined{pgftext}{%
          \ltx@IfUndefined{psscalebox}{%
            \ltx@IfUndefined{HOLOGO@ScaleBox@\hologoDriver}{%
            }{%
              \chardef\HOLOGO@choice=4 %
            }%
          }{%
            \chardef\HOLOGO@choice=3 %
          }%
        }{%
          \chardef\HOLOGO@choice=2 %
        }%
      }{%
        \chardef\HOLOGO@choice=1 %
      }%
      \ifcase\HOLOGO@choice
        \HOLOGO@WarningUnsupportedDriver{Xe}%
        e%
      \or % 1: \rotatebox
        \begingroup
          \setbox\ltx@zero\hbox{\rotatebox{180}{E}}%
          \ltx@LocDimenA=\dp\ltx@zero
          \advance\ltx@LocDimenA by -.5ex\relax
          \raise\ltx@LocDimenA\box\ltx@zero
        \endgroup
      \or % 2: \pgftext
        \lower.5ex\hbox{%
          \pgfpicture
            \pgftext[rotate=180]{E}%
          \endpgfpicture
        }%
      \or % 3: \psscalebox
        \begingroup
          \setbox\ltx@zero\hbox{\psscalebox{-1 -1}{E}}%
          \ltx@LocDimenA=\dp\ltx@zero
          \advance\ltx@LocDimenA by -.5ex\relax
          \raise\ltx@LocDimenA\box\ltx@zero
        \endgroup
      \or % 4: \HOLOGO@PointReflectBox
        \lower.5ex\hbox{\HOLOGO@PointReflectBox{E}}%
      \else
        \@PackageError{hologo}{Internal error (choice/it}\@ehc
      \fi
    \else
      \ltx@IfUndefined{reflectbox}{%
        \ltx@IfUndefined{pgftext}{%
          \ltx@IfUndefined{psscalebox}{%
            \ltx@IfUndefined{HOLOGO@ScaleBox@\hologoDriver}{%
            }{%
              \chardef\HOLOGO@choice=4 %
            }%
          }{%
            \chardef\HOLOGO@choice=3 %
          }%
        }{%
          \chardef\HOLOGO@choice=2 %
        }%
      }{%
        \chardef\HOLOGO@choice=1 %
      }%
      \ifcase\HOLOGO@choice
        \HOLOGO@WarningUnsupportedDriver{Xe}%
        e%
      \or % 1: reflectbox
        \lower.5ex\hbox{%
          \reflectbox{E}%
        }%
      \or % 2: \pgftext
        \lower.5ex\hbox{%
          \pgfpicture
            \pgftransformxscale{-1}%
            \pgftext{E}%
          \endpgfpicture
        }%
      \or % 3: \psscalebox
        \lower.5ex\hbox{%
          \psscalebox{-1 1}{E}%
        }%
      \or % 4: \HOLOGO@Reflectbox
        \lower.5ex\hbox{%
          \HOLOGO@ReflectBox{E}%
        }%
      \else
        \@PackageError{hologo}{Internal error (choice/up)}\@ehc
      \fi
    \fi
  }%
}
%    \end{macrocode}
%    \end{macro}
%    \begin{macro}{\HoLogoHtml@Xe}
%    \begin{macrocode}
\def\HoLogoHtml@Xe#1{%
  \HoLogoCss@Xe
  \HOLOGO@Span{Xe}{%
    X%
    \HOLOGO@Span{e}{%
      \HCode{&\ltx@hashchar x018e;}%
    }%
  }%
}
%    \end{macrocode}
%    \end{macro}
%    \begin{macro}{\HoLogoCss@Xe}
%    \begin{macrocode}
\def\HoLogoCss@Xe{%
  \Css{%
    span.HoLogo-Xe span.HoLogo-e{%
      position:relative;%
      top:.5ex;%
      left-margin:-.1em;%
    }%
  }%
  \global\let\HoLogoCss@Xe\relax
}
%    \end{macrocode}
%    \end{macro}
%
%    \begin{macro}{\HoLogo@XeTeX}
%    \begin{macrocode}
\def\HoLogo@XeTeX#1{%
  \hologo{Xe}%
  \kern-.15em\relax
  \hologo{TeX}%
}
%    \end{macrocode}
%    \end{macro}
%
%    \begin{macro}{\HoLogoHtml@XeTeX}
%    \begin{macrocode}
\def\HoLogoHtml@XeTeX#1{%
  \HoLogoCss@XeTeX
  \HOLOGO@Span{XeTeX}{%
    \hologo{Xe}%
    \hologo{TeX}%
  }%
}
%    \end{macrocode}
%    \end{macro}
%    \begin{macro}{\HoLogoCss@XeTeX}
%    \begin{macrocode}
\def\HoLogoCss@XeTeX{%
  \Css{%
    span.HoLogo-XeTeX span.HoLogo-TeX{%
      margin-left:-.15em;%
    }%
  }%
  \global\let\HoLogoCss@XeTeX\relax
}
%    \end{macrocode}
%    \end{macro}
%
%    \begin{macro}{\HoLogo@XeLaTeX}
%    \begin{macrocode}
\def\HoLogo@XeLaTeX#1{%
  \hologo{Xe}%
  \kern-.13em%
  \hologo{LaTeX}%
}
%    \end{macrocode}
%    \end{macro}
%    \begin{macro}{\HoLogoHtml@XeLaTeX}
%    \begin{macrocode}
\def\HoLogoHtml@XeLaTeX#1{%
  \HoLogoCss@XeLaTeX
  \HOLOGO@Span{XeLaTeX}{%
    \hologo{Xe}%
    \hologo{LaTeX}%
  }%
}
%    \end{macrocode}
%    \end{macro}
%    \begin{macro}{\HoLogoCss@XeLaTeX}
%    \begin{macrocode}
\def\HoLogoCss@XeLaTeX{%
  \Css{%
    span.HoLogo-XeLaTeX span.HoLogo-Xe{%
      margin-right:-.13em;%
    }%
  }%
  \global\let\HoLogoCss@XeLaTeX\relax
}
%    \end{macrocode}
%    \end{macro}
%
% \subsubsection{\hologo{pdfTeX}, \hologo{pdfLaTeX}}
%
%    \begin{macro}{\HoLogo@pdfTeX}
%    \begin{macrocode}
\def\HoLogo@pdfTeX#1{%
  \HOLOGO@mbox{%
    #1{p}{P}df\hologo{TeX}%
  }%
}
%    \end{macrocode}
%    \end{macro}
%    \begin{macro}{\HoLogoCs@pdfTeX}
%    \begin{macrocode}
\def\HoLogoCs@pdfTeX#1{#1{p}{P}dfTeX}
%    \end{macrocode}
%    \end{macro}
%    \begin{macro}{\HoLogoBkm@pdfTeX}
%    \begin{macrocode}
\def\HoLogoBkm@pdfTeX#1{%
  #1{p}{P}df\hologo{TeX}%
}
%    \end{macrocode}
%    \end{macro}
%    \begin{macro}{\HoLogoHtml@pdfTeX}
%    \begin{macrocode}
\let\HoLogoHtml@pdfTeX\HoLogo@pdfTeX
%    \end{macrocode}
%    \end{macro}
%
%    \begin{macro}{\HoLogo@pdfLaTeX}
%    \begin{macrocode}
\def\HoLogo@pdfLaTeX#1{%
  \HOLOGO@mbox{%
    #1{p}{P}df\hologo{LaTeX}%
  }%
}
%    \end{macrocode}
%    \end{macro}
%    \begin{macro}{\HoLogoCs@pdfLaTeX}
%    \begin{macrocode}
\def\HoLogoCs@pdfLaTeX#1{#1{p}{P}dfLaTeX}
%    \end{macrocode}
%    \end{macro}
%    \begin{macro}{\HoLogoBkm@pdfLaTeX}
%    \begin{macrocode}
\def\HoLogoBkm@pdfLaTeX#1{%
  #1{p}{P}df\hologo{LaTeX}%
}
%    \end{macrocode}
%    \end{macro}
%    \begin{macro}{\HoLogoHtml@pdfLaTeX}
%    \begin{macrocode}
\let\HoLogoHtml@pdfLaTeX\HoLogo@pdfLaTeX
%    \end{macrocode}
%    \end{macro}
%
% \subsubsection{\hologo{VTeX}}
%
%    \begin{macro}{\HoLogo@VTeX}
%    \begin{macrocode}
\def\HoLogo@VTeX#1{%
  \HOLOGO@mbox{%
    V\hologo{TeX}%
  }%
}
%    \end{macrocode}
%    \end{macro}
%    \begin{macro}{\HoLogoHtml@VTeX}
%    \begin{macrocode}
\let\HoLogoHtml@VTeX\HoLogo@VTeX
%    \end{macrocode}
%    \end{macro}
%
% \subsubsection{\hologo{AmS}, \dots}
%
%    Source: class \xclass{amsdtx}
%
%    \begin{macro}{\HoLogo@AmS}
%    \begin{macrocode}
\def\HoLogo@AmS#1{%
  \HoLogoFont@font{AmS}{sy}{%
    A%
    \kern-.1667em%
    \lower.5ex\hbox{M}%
    \kern-.125em%
    S%
  }%
}
%    \end{macrocode}
%    \end{macro}
%    \begin{macro}{\HoLogoBkm@AmS}
%    \begin{macrocode}
\def\HoLogoBkm@AmS#1{AmS}
%    \end{macrocode}
%    \end{macro}
%    \begin{macro}{\HoLogoHtml@AmS}
%    \begin{macrocode}
\def\HoLogoHtml@AmS#1{%
  \HoLogoCss@AmS
%  \HoLogoFont@font{AmS}{sy}{%
    \HOLOGO@Span{AmS}{%
      A%
      \HOLOGO@Span{M}{M}%
      S%
    }%
%   }%
}
%    \end{macrocode}
%    \end{macro}
%    \begin{macro}{\HoLogoCss@AmS}
%    \begin{macrocode}
\def\HoLogoCss@AmS{%
  \Css{%
    span.HoLogo-AmS span.HoLogo-M{%
      position:relative;%
      top:.5ex;%
      margin-left:-.1667em;%
      margin-right:-.125em;%
      text-decoration:none;%
    }%
  }%
  \global\let\HoLogoCss@AmS\relax
}
%    \end{macrocode}
%    \end{macro}
%
%    \begin{macro}{\HoLogo@AmSTeX}
%    \begin{macrocode}
\def\HoLogo@AmSTeX#1{%
  \hologo{AmS}%
  \HOLOGO@hyphen
  \hologo{TeX}%
}
%    \end{macrocode}
%    \end{macro}
%    \begin{macro}{\HoLogoBkm@AmSTeX}
%    \begin{macrocode}
\def\HoLogoBkm@AmSTeX#1{AmS-TeX}%
%    \end{macrocode}
%    \end{macro}
%    \begin{macro}{\HoLogoHtml@AmSTeX}
%    \begin{macrocode}
\let\HoLogoHtml@AmSTeX\HoLogo@AmSTeX
%    \end{macrocode}
%    \end{macro}
%
%    \begin{macro}{\HoLogo@AmSLaTeX}
%    \begin{macrocode}
\def\HoLogo@AmSLaTeX#1{%
  \hologo{AmS}%
  \HOLOGO@hyphen
  \hologo{LaTeX}%
}
%    \end{macrocode}
%    \end{macro}
%    \begin{macro}{\HoLogoBkm@AmSLaTeX}
%    \begin{macrocode}
\def\HoLogoBkm@AmSLaTeX#1{AmS-LaTeX}%
%    \end{macrocode}
%    \end{macro}
%    \begin{macro}{\HoLogoHtml@AmSLaTeX}
%    \begin{macrocode}
\let\HoLogoHtml@AmSLaTeX\HoLogo@AmSLaTeX
%    \end{macrocode}
%    \end{macro}
%
% \subsubsection{\hologo{BibTeX}}
%
%    \begin{macro}{\HoLogo@BibTeX@sc}
%    A definition of \hologo{BibTeX} is provided in
%    the documentation source for the manual of \hologo{BibTeX}
%    \cite{btxdoc}.
%\begin{quote}
%\begin{verbatim}
%\def\BibTeX{%
%  {%
%    \rm
%    B%
%    \kern-.05em%
%    {%
%      \sc
%      i%
%      \kern-.025em %
%      b%
%    }%
%    \kern-.08em
%    T%
%    \kern-.1667em%
%    \lower.7ex\hbox{E}%
%    \kern-.125em%
%    X%
%  }%
%}
%\end{verbatim}
%\end{quote}
%    \begin{macrocode}
\def\HoLogo@BibTeX@sc#1{%
  B%
  \kern-.05em%
  \HoLogoFont@font{BibTeX}{sc}{%
    i%
    \kern-.025em%
    b%
  }%
  \HOLOGO@discretionary
  \kern-.08em%
  \hologo{TeX}%
}
%    \end{macrocode}
%    \end{macro}
%    \begin{macro}{\HoLogoHtml@BibTeX@sc}
%    \begin{macrocode}
\def\HoLogoHtml@BibTeX@sc#1{%
  \HoLogoCss@BibTeX@sc
  \HOLOGO@Span{BibTeX-sc}{%
    B%
    \HOLOGO@Span{i}{i}%
    \HOLOGO@Span{b}{b}%
    \hologo{TeX}%
  }%
}
%    \end{macrocode}
%    \end{macro}
%    \begin{macro}{\HoLogoCss@BibTeX@sc}
%    \begin{macrocode}
\def\HoLogoCss@BibTeX@sc{%
  \Css{%
    span.HoLogo-BibTeX-sc span.HoLogo-i{%
      margin-left:-.05em;%
      margin-right:-.025em;%
      font-variant:small-caps;%
    }%
  }%
  \Css{%
    span.HoLogo-BibTeX-sc span.HoLogo-b{%
      margin-right:-.08em;%
      font-variant:small-caps;%
    }%
  }%
  \global\let\HoLogoCss@BibTeX@sc\relax
}
%    \end{macrocode}
%    \end{macro}
%
%    \begin{macro}{\HoLogo@BibTeX@sf}
%    Variant \xoption{sf} avoids trouble with unavailable
%    small caps fonts (e.g., bold versions of Computer Modern or
%    Latin Modern). The definition is taken from
%    package \xpackage{dtklogos} \cite{dtklogos}.
%\begin{quote}
%\begin{verbatim}
%\DeclareRobustCommand{\BibTeX}{%
%  B%
%  \kern-.05em%
%  \hbox{%
%    $\m@th$% %% force math size calculations
%    \csname S@\f@size\endcsname
%    \fontsize\sf@size\z@
%    \math@fontsfalse
%    \selectfont
%    I%
%    \kern-.025em%
%    B
%  }%
%  \kern-.08em%
%  \-%
%  \TeX
%}
%\end{verbatim}
%\end{quote}
%    \begin{macrocode}
\def\HoLogo@BibTeX@sf#1{%
  B%
  \kern-.05em%
  \HoLogoFont@font{BibTeX}{bibsf}{%
    I%
    \kern-.025em%
    B%
  }%
  \HOLOGO@discretionary
  \kern-.08em%
  \hologo{TeX}%
}
%    \end{macrocode}
%    \end{macro}
%    \begin{macro}{\HoLogoHtml@BibTeX@sf}
%    \begin{macrocode}
\def\HoLogoHtml@BibTeX@sf#1{%
  \HoLogoCss@BibTeX@sf
  \HOLOGO@Span{BibTeX-sf}{%
    B%
    \HoLogoFont@font{BibTeX}{bibsf}{%
      \HOLOGO@Span{i}{I}%
      B%
    }%
    \hologo{TeX}%
  }%
}
%    \end{macrocode}
%    \end{macro}
%    \begin{macro}{\HoLogoCss@BibTeX@sf}
%    \begin{macrocode}
\def\HoLogoCss@BibTeX@sf{%
  \Css{%
    span.HoLogo-BibTeX-sf span.HoLogo-i{%
      margin-left:-.05em;%
      margin-right:-.025em;%
    }%
  }%
  \Css{%
    span.HoLogo-BibTeX-sf span.HoLogo-TeX{%
      margin-left:-.08em;%
    }%
  }%
  \global\let\HoLogoCss@BibTeX@sf\relax
}
%    \end{macrocode}
%    \end{macro}
%
%    \begin{macro}{\HoLogo@BibTeX}
%    \begin{macrocode}
\def\HoLogo@BibTeX{\HoLogo@BibTeX@sf}
%    \end{macrocode}
%    \end{macro}
%    \begin{macro}{\HoLogoHtml@BibTeX}
%    \begin{macrocode}
\def\HoLogoHtml@BibTeX{\HoLogoHtml@BibTeX@sf}
%    \end{macrocode}
%    \end{macro}
%
% \subsubsection{\hologo{BibTeX8}}
%
%    \begin{macro}{\HoLogo@BibTeX8}
%    \begin{macrocode}
\expandafter\def\csname HoLogo@BibTeX8\endcsname#1{%
  \hologo{BibTeX}%
  8%
}
%    \end{macrocode}
%    \end{macro}
%
%    \begin{macro}{\HoLogoBkm@BibTeX8}
%    \begin{macrocode}
\expandafter\def\csname HoLogoBkm@BibTeX8\endcsname#1{%
  \hologo{BibTeX}%
  8%
}
%    \end{macrocode}
%    \end{macro}
%    \begin{macro}{\HoLogoHtml@BibTeX8}
%    \begin{macrocode}
\expandafter
\let\csname HoLogoHtml@BibTeX8\expandafter\endcsname
\csname HoLogo@BibTeX8\endcsname
%    \end{macrocode}
%    \end{macro}
%
% \subsubsection{\hologo{ConTeXt}}
%
%    \begin{macro}{\HoLogo@ConTeXt@simple}
%    \begin{macrocode}
\def\HoLogo@ConTeXt@simple#1{%
  \HOLOGO@mbox{Con}%
  \HOLOGO@discretionary
  \HOLOGO@mbox{\hologo{TeX}t}%
}
%    \end{macrocode}
%    \end{macro}
%    \begin{macro}{\HoLogoHtml@ConTeXt@simple}
%    \begin{macrocode}
\let\HoLogoHtml@ConTeXt@simple\HoLogo@ConTeXt@simple
%    \end{macrocode}
%    \end{macro}
%
%    \begin{macro}{\HoLogo@ConTeXt@narrow}
%    This definition of logo \hologo{ConTeXt} with variant \xoption{narrow}
%    comes from TUGboat's class \xclass{ltugboat} (version 2010/11/15 v2.8).
%    \begin{macrocode}
\def\HoLogo@ConTeXt@narrow#1{%
  \HOLOGO@mbox{C\kern-.0333emon}%
  \HOLOGO@discretionary
  \kern-.0667em%
  \HOLOGO@mbox{\hologo{TeX}\kern-.0333emt}%
}
%    \end{macrocode}
%    \end{macro}
%    \begin{macro}{\HoLogoHtml@ConTeXt@narrow}
%    \begin{macrocode}
\def\HoLogoHtml@ConTeXt@narrow#1{%
  \HoLogoCss@ConTeXt@narrow
  \HOLOGO@Span{ConTeXt-narrow}{%
    \HOLOGO@Span{C}{C}%
    on%
    \hologo{TeX}%
    t%
  }%
}
%    \end{macrocode}
%    \end{macro}
%    \begin{macro}{\HoLogoCss@ConTeXt@narrow}
%    \begin{macrocode}
\def\HoLogoCss@ConTeXt@narrow{%
  \Css{%
    span.HoLogo-ConTeXt-narrow span.HoLogo-C{%
      margin-left:-.0333em;%
    }%
  }%
  \Css{%
    span.HoLogo-ConTeXt-narrow span.HoLogo-TeX{%
      margin-left:-.0667em;%
      margin-right:-.0333em;%
    }%
  }%
  \global\let\HoLogoCss@ConTeXt@narrow\relax
}
%    \end{macrocode}
%    \end{macro}
%
%    \begin{macro}{\HoLogo@ConTeXt}
%    \begin{macrocode}
\def\HoLogo@ConTeXt{\HoLogo@ConTeXt@narrow}
%    \end{macrocode}
%    \end{macro}
%    \begin{macro}{\HoLogoHtml@ConTeXt}
%    \begin{macrocode}
\def\HoLogoHtml@ConTeXt{\HoLogoHtml@ConTeXt@narrow}
%    \end{macrocode}
%    \end{macro}
%
% \subsubsection{\hologo{emTeX}}
%
%    \begin{macro}{\HoLogo@emTeX}
%    \begin{macrocode}
\def\HoLogo@emTeX#1{%
  \HOLOGO@mbox{#1{e}{E}m}%
  \HOLOGO@discretionary
  \hologo{TeX}%
}
%    \end{macrocode}
%    \end{macro}
%    \begin{macro}{\HoLogoCs@emTeX}
%    \begin{macrocode}
\def\HoLogoCs@emTeX#1{#1{e}{E}mTeX}%
%    \end{macrocode}
%    \end{macro}
%    \begin{macro}{\HoLogoBkm@emTeX}
%    \begin{macrocode}
\def\HoLogoBkm@emTeX#1{%
  #1{e}{E}m\hologo{TeX}%
}
%    \end{macrocode}
%    \end{macro}
%    \begin{macro}{\HoLogoHtml@emTeX}
%    \begin{macrocode}
\let\HoLogoHtml@emTeX\HoLogo@emTeX
%    \end{macrocode}
%    \end{macro}
%
% \subsubsection{\hologo{ExTeX}}
%
%    \begin{macro}{\HoLogo@ExTeX}
%    The definition is taken from the FAQ of the
%    project \hologo{ExTeX}
%    \cite{ExTeX-FAQ}.
%\begin{quote}
%\begin{verbatim}
%\def\ExTeX{%
%  \textrm{% Logo always with serifs
%    \ensuremath{%
%      \textstyle
%      \varepsilon_{%
%        \kern-0.15em%
%        \mathcal{X}%
%      }%
%    }%
%    \kern-.15em%
%    \TeX
%  }%
%}
%\end{verbatim}
%\end{quote}
%    \begin{macrocode}
\def\HoLogo@ExTeX#1{%
  \HoLogoFont@font{ExTeX}{rm}{%
    \ltx@mbox{%
      \HOLOGO@MathSetup
      $%
        \textstyle
        \varepsilon_{%
          \kern-0.15em%
          \HoLogoFont@font{ExTeX}{sy}{X}%
        }%
      $%
    }%
    \HOLOGO@discretionary
    \kern-.15em%
    \hologo{TeX}%
  }%
}
%    \end{macrocode}
%    \end{macro}
%    \begin{macro}{\HoLogoHtml@ExTeX}
%    \begin{macrocode}
\def\HoLogoHtml@ExTeX#1{%
  \HoLogoCss@ExTeX
  \HoLogoFont@font{ExTeX}{rm}{%
    \HOLOGO@Span{ExTeX}{%
      \ltx@mbox{%
        \HOLOGO@MathSetup
        $\textstyle\varepsilon$%
        \HOLOGO@Span{X}{$\textstyle\chi$}%
        \hologo{TeX}%
      }%
    }%
  }%
}
%    \end{macrocode}
%    \end{macro}
%    \begin{macro}{\HoLogoBkm@ExTeX}
%    \begin{macrocode}
\def\HoLogoBkm@ExTeX#1{%
  \HOLOGO@PdfdocUnicode{#1{e}{E}x}{\textepsilon\textchi}%
  \hologo{TeX}%
}
%    \end{macrocode}
%    \end{macro}
%    \begin{macro}{\HoLogoCss@ExTeX}
%    \begin{macrocode}
\def\HoLogoCss@ExTeX{%
  \Css{%
    span.HoLogo-ExTeX{%
      font-family:serif;%
    }%
  }%
  \Css{%
    span.HoLogo-ExTeX span.HoLogo-TeX{%
      margin-left:-.15em;%
    }%
  }%
  \global\let\HoLogoCss@ExTeX\relax
}
%    \end{macrocode}
%    \end{macro}
%
% \subsubsection{\hologo{MiKTeX}}
%
%    \begin{macro}{\HoLogo@MiKTeX}
%    \begin{macrocode}
\def\HoLogo@MiKTeX#1{%
  \HOLOGO@mbox{MiK}%
  \HOLOGO@discretionary
  \hologo{TeX}%
}
%    \end{macrocode}
%    \end{macro}
%    \begin{macro}{\HoLogoHtml@MiKTeX}
%    \begin{macrocode}
\let\HoLogoHtml@MiKTeX\HoLogo@MiKTeX
%    \end{macrocode}
%    \end{macro}
%
% \subsubsection{\hologo{OzTeX} and friends}
%
%    Source: \hologo{OzTeX} FAQ \cite{OzTeX}:
%    \begin{quote}
%      |\def\OzTeX{O\kern-.03em z\kern-.15em\TeX}|\\
%      (There is no kerning in OzMF, OzMP and OzTtH.)
%    \end{quote}
%
%    \begin{macro}{\HoLogo@OzTeX}
%    \begin{macrocode}
\def\HoLogo@OzTeX#1{%
  O%
  \kern-.03em %
  z%
  \kern-.15em %
  \hologo{TeX}%
}
%    \end{macrocode}
%    \end{macro}
%    \begin{macro}{\HoLogoHtml@OzTeX}
%    \begin{macrocode}
\def\HoLogoHtml@OzTeX#1{%
  \HoLogoCss@OzTeX
  \HOLOGO@Span{OzTeX}{%
    O%
    \HOLOGO@Span{z}{z}%
    \hologo{TeX}%
  }%
}
%    \end{macrocode}
%    \end{macro}
%    \begin{macro}{\HoLogoCss@OzTeX}
%    \begin{macrocode}
\def\HoLogoCss@OzTeX{%
  \Css{%
    span.HoLogo-OzTeX span.HoLogo-z{%
      margin-left:-.03em;%
      margin-right:-.15em;%
    }%
  }%
  \global\let\HoLogoCss@OzTeX\relax
}
%    \end{macrocode}
%    \end{macro}
%
%    \begin{macro}{\HoLogo@OzMF}
%    \begin{macrocode}
\def\HoLogo@OzMF#1{%
  \HOLOGO@mbox{OzMF}%
}
%    \end{macrocode}
%    \end{macro}
%    \begin{macro}{\HoLogo@OzMP}
%    \begin{macrocode}
\def\HoLogo@OzMP#1{%
  \HOLOGO@mbox{OzMP}%
}
%    \end{macrocode}
%    \end{macro}
%    \begin{macro}{\HoLogo@OzTtH}
%    \begin{macrocode}
\def\HoLogo@OzTtH#1{%
  \HOLOGO@mbox{OzTtH}%
}
%    \end{macrocode}
%    \end{macro}
%
% \subsubsection{\hologo{PCTeX}}
%
%    \begin{macro}{\HoLogo@PCTeX}
%    \begin{macrocode}
\def\HoLogo@PCTeX#1{%
  \HOLOGO@mbox{PC}%
  \hologo{TeX}%
}
%    \end{macrocode}
%    \end{macro}
%    \begin{macro}{\HoLogoHtml@PCTeX}
%    \begin{macrocode}
\let\HoLogoHtml@PCTeX\HoLogo@PCTeX
%    \end{macrocode}
%    \end{macro}
%
% \subsubsection{\hologo{PiCTeX}}
%
%    The original definitions from \xfile{pictex.tex} \cite{PiCTeX}:
%\begin{quote}
%\begin{verbatim}
%\def\PiC{%
%  P%
%  \kern-.12em%
%  \lower.5ex\hbox{I}%
%  \kern-.075em%
%  C%
%}
%\def\PiCTeX{%
%  \PiC
%  \kern-.11em%
%  \TeX
%}
%\end{verbatim}
%\end{quote}
%
%    \begin{macro}{\HoLogo@PiC}
%    \begin{macrocode}
\def\HoLogo@PiC#1{%
  P%
  \kern-.12em%
  \lower.5ex\hbox{I}%
  \kern-.075em%
  C%
  \HOLOGO@SpaceFactor
}
%    \end{macrocode}
%    \end{macro}
%    \begin{macro}{\HoLogoHtml@PiC}
%    \begin{macrocode}
\def\HoLogoHtml@PiC#1{%
  \HoLogoCss@PiC
  \HOLOGO@Span{PiC}{%
    P%
    \HOLOGO@Span{i}{I}%
    C%
  }%
}
%    \end{macrocode}
%    \end{macro}
%    \begin{macro}{\HoLogoCss@PiC}
%    \begin{macrocode}
\def\HoLogoCss@PiC{%
  \Css{%
    span.HoLogo-PiC span.HoLogo-i{%
      position:relative;%
      top:.5ex;%
      margin-left:-.12em;%
      margin-right:-.075em;%
      text-decoration:none;%
    }%
  }%
  \global\let\HoLogoCss@PiC\relax
}
%    \end{macrocode}
%    \end{macro}
%
%    \begin{macro}{\HoLogo@PiCTeX}
%    \begin{macrocode}
\def\HoLogo@PiCTeX#1{%
  \hologo{PiC}%
  \HOLOGO@discretionary
  \kern-.11em%
  \hologo{TeX}%
}
%    \end{macrocode}
%    \end{macro}
%    \begin{macro}{\HoLogoHtml@PiCTeX}
%    \begin{macrocode}
\def\HoLogoHtml@PiCTeX#1{%
  \HoLogoCss@PiCTeX
  \HOLOGO@Span{PiCTeX}{%
    \hologo{PiC}%
    \hologo{TeX}%
  }%
}
%    \end{macrocode}
%    \end{macro}
%    \begin{macro}{\HoLogoCss@PiCTeX}
%    \begin{macrocode}
\def\HoLogoCss@PiCTeX{%
  \Css{%
    span.HoLogo-PiCTeX span.HoLogo-PiC{%
      margin-right:-.11em;%
    }%
  }%
  \global\let\HoLogoCss@PiCTeX\relax
}
%    \end{macrocode}
%    \end{macro}
%
% \subsubsection{\hologo{teTeX}}
%
%    \begin{macro}{\HoLogo@teTeX}
%    \begin{macrocode}
\def\HoLogo@teTeX#1{%
  \HOLOGO@mbox{#1{t}{T}e}%
  \HOLOGO@discretionary
  \hologo{TeX}%
}
%    \end{macrocode}
%    \end{macro}
%    \begin{macro}{\HoLogoCs@teTeX}
%    \begin{macrocode}
\def\HoLogoCs@teTeX#1{#1{t}{T}dfTeX}
%    \end{macrocode}
%    \end{macro}
%    \begin{macro}{\HoLogoBkm@teTeX}
%    \begin{macrocode}
\def\HoLogoBkm@teTeX#1{%
  #1{t}{T}e\hologo{TeX}%
}
%    \end{macrocode}
%    \end{macro}
%    \begin{macro}{\HoLogoHtml@teTeX}
%    \begin{macrocode}
\let\HoLogoHtml@teTeX\HoLogo@teTeX
%    \end{macrocode}
%    \end{macro}
%
% \subsubsection{\hologo{TeX4ht}}
%
%    \begin{macro}{\HoLogo@TeX4ht}
%    \begin{macrocode}
\expandafter\def\csname HoLogo@TeX4ht\endcsname#1{%
  \HOLOGO@mbox{\hologo{TeX}4ht}%
}
%    \end{macrocode}
%    \end{macro}
%    \begin{macro}{\HoLogoHtml@TeX4ht}
%    \begin{macrocode}
\expandafter
\let\csname HoLogoHtml@TeX4ht\expandafter\endcsname
\csname HoLogo@TeX4ht\endcsname
%    \end{macrocode}
%    \end{macro}
%
%
% \subsubsection{\hologo{SageTeX}}
%
%    \begin{macro}{\HoLogo@SageTeX}
%    \begin{macrocode}
\def\HoLogo@SageTeX#1{%
  \HOLOGO@mbox{Sage}%
  \HOLOGO@discretionary
  \HOLOGO@NegativeKerning{eT,oT,To}%
  \hologo{TeX}%
}
%    \end{macrocode}
%    \end{macro}
%    \begin{macro}{\HoLogoHtml@SageTeX}
%    \begin{macrocode}
\let\HoLogoHtml@SageTeX\HoLogo@SageTeX
%    \end{macrocode}
%    \end{macro}
%
% \subsection{\hologo{METAFONT} and friends}
%
%    \begin{macro}{\HoLogo@METAFONT}
%    \begin{macrocode}
\def\HoLogo@METAFONT#1{%
  \HoLogoFont@font{METAFONT}{logo}{%
    \HOLOGO@mbox{META}%
    \HOLOGO@discretionary
    \HOLOGO@mbox{FONT}%
  }%
}
%    \end{macrocode}
%    \end{macro}
%
%    \begin{macro}{\HoLogo@METAPOST}
%    \begin{macrocode}
\def\HoLogo@METAPOST#1{%
  \HoLogoFont@font{METAPOST}{logo}{%
    \HOLOGO@mbox{META}%
    \HOLOGO@discretionary
    \HOLOGO@mbox{POST}%
  }%
}
%    \end{macrocode}
%    \end{macro}
%
%    \begin{macro}{\HoLogo@MetaFun}
%    \begin{macrocode}
\def\HoLogo@MetaFun#1{%
  \HOLOGO@mbox{Meta}%
  \HOLOGO@discretionary
  \HOLOGO@mbox{Fun}%
}
%    \end{macrocode}
%    \end{macro}
%
%    \begin{macro}{\HoLogo@MetaPost}
%    \begin{macrocode}
\def\HoLogo@MetaPost#1{%
  \HOLOGO@mbox{Meta}%
  \HOLOGO@discretionary
  \HOLOGO@mbox{Post}%
}
%    \end{macrocode}
%    \end{macro}
%
% \subsection{Others}
%
% \subsubsection{\hologo{biber}}
%
%    \begin{macro}{\HoLogo@biber}
%    \begin{macrocode}
\def\HoLogo@biber#1{%
  \HOLOGO@mbox{#1{b}{B}i}%
  \HOLOGO@discretionary
  \HOLOGO@mbox{ber}%
}
%    \end{macrocode}
%    \end{macro}
%    \begin{macro}{\HoLogoCs@biber}
%    \begin{macrocode}
\def\HoLogoCs@biber#1{#1{b}{B}iber}
%    \end{macrocode}
%    \end{macro}
%    \begin{macro}{\HoLogoBkm@biber}
%    \begin{macrocode}
\def\HoLogoBkm@biber#1{%
  #1{b}{B}iber%
}
%    \end{macrocode}
%    \end{macro}
%    \begin{macro}{\HoLogoHtml@biber}
%    \begin{macrocode}
\let\HoLogoHtml@biber\HoLogo@biber
%    \end{macrocode}
%    \end{macro}
%
% \subsubsection{\hologo{KOMAScript}}
%
%    \begin{macro}{\HoLogo@KOMAScript}
%    The definition for \hologo{KOMAScript} is taken
%    from \hologo{KOMAScript} (\xfile{scrlogo.dtx}, reformatted) \cite{scrlogo}:
%\begin{quote}
%\begin{verbatim}
%\@ifundefined{KOMAScript}{%
%  \DeclareRobustCommand{\KOMAScript}{%
%    \textsf{%
%      K\kern.05em O\kern.05emM\kern.05em A%
%      \kern.1em-\kern.1em %
%      Script%
%    }%
%  }%
%}{}
%\end{verbatim}
%\end{quote}
%    \begin{macrocode}
\def\HoLogo@KOMAScript#1{%
  \HoLogoFont@font{KOMAScript}{sf}{%
    \HOLOGO@mbox{%
      K\kern.05em%
      O\kern.05em%
      M\kern.05em%
      A%
    }%
    \kern.1em%
    \HOLOGO@hyphen
    \kern.1em%
    \HOLOGO@mbox{Script}%
  }%
}
%    \end{macrocode}
%    \end{macro}
%    \begin{macro}{\HoLogoBkm@KOMAScript}
%    \begin{macrocode}
\def\HoLogoBkm@KOMAScript#1{%
  KOMA-Script%
}
%    \end{macrocode}
%    \end{macro}
%    \begin{macro}{\HoLogoHtml@KOMAScript}
%    \begin{macrocode}
\def\HoLogoHtml@KOMAScript#1{%
  \HoLogoCss@KOMAScript
  \HoLogoFont@font{KOMAScript}{sf}{%
    \HOLOGO@Span{KOMAScript}{%
      K%
      \HOLOGO@Span{O}{O}%
      M%
      \HOLOGO@Span{A}{A}%
      \HOLOGO@Span{hyphen}{-}%
      Script%
    }%
  }%
}
%    \end{macrocode}
%    \end{macro}
%    \begin{macro}{\HoLogoCss@KOMAScript}
%    \begin{macrocode}
\def\HoLogoCss@KOMAScript{%
  \Css{%
    span.HoLogo-KOMAScript{%
      font-family:sans-serif;%
    }%
  }%
  \Css{%
    span.HoLogo-KOMAScript span.HoLogo-O{%
      padding-left:.05em;%
      padding-right:.05em;%
    }%
  }%
  \Css{%
    span.HoLogo-KOMAScript span.HoLogo-A{%
      padding-left:.05em;%
    }%
  }%
  \Css{%
    span.HoLogo-KOMAScript span.HoLogo-hyphen{%
      padding-left:.1em;%
      padding-right:.1em;%
    }%
  }%
  \global\let\HoLogoCss@KOMAScript\relax
}
%    \end{macrocode}
%    \end{macro}
%
% \subsubsection{\hologo{LyX}}
%
%    \begin{macro}{\HoLogo@LyX}
%    The definition is taken from the documentation source files
%    of \hologo{LyX}, \xfile{Intro.lyx} \cite{LyX}:
%\begin{quote}
%\begin{verbatim}
%\def\LyX{%
%  \texorpdfstring{%
%    L\kern-.1667em\lower.25em\hbox{Y}\kern-.125emX\@%
%  }{%
%    LyX%
%  }%
%}
%\end{verbatim}
%\end{quote}
%    \begin{macrocode}
\def\HoLogo@LyX#1{%
  L%
  \kern-.1667em%
  \lower.25em\hbox{Y}%
  \kern-.125em%
  X%
  \HOLOGO@SpaceFactor
}
%    \end{macrocode}
%    \end{macro}
%    \begin{macro}{\HoLogoHtml@LyX}
%    \begin{macrocode}
\def\HoLogoHtml@LyX#1{%
  \HoLogoCss@LyX
  \HOLOGO@Span{LyX}{%
    L%
    \HOLOGO@Span{y}{Y}%
    X%
  }%
}
%    \end{macrocode}
%    \end{macro}
%    \begin{macro}{\HoLogoCss@LyX}
%    \begin{macrocode}
\def\HoLogoCss@LyX{%
  \Css{%
    span.HoLogo-LyX span.HoLogo-y{%
      position:relative;%
      top:.25em;%
      margin-left:-.1667em;%
      margin-right:-.125em;%
      text-decoration:none;%
    }%
  }%
  \global\let\HoLogoCss@LyX\relax
}
%    \end{macrocode}
%    \end{macro}
%
% \subsubsection{\hologo{NTS}}
%
%    \begin{macro}{\HoLogo@NTS}
%    Definition for \hologo{NTS} can be found in
%    package \xpackage{etex\textunderscore man} for the \hologo{eTeX} manual \cite{etexman}
%    and in package \xpackage{dtklogos} \cite{dtklogos}:
%\begin{quote}
%\begin{verbatim}
%\def\NTS{%
%  \leavevmode
%  \hbox{%
%    $%
%      \cal N%
%      \kern-0.35em%
%      \lower0.5ex\hbox{$\cal T$}%
%      \kern-0.2em%
%      S%
%    $%
%  }%
%}
%\end{verbatim}
%\end{quote}
%    \begin{macrocode}
\def\HoLogo@NTS#1{%
  \HoLogoFont@font{NTS}{sy}{%
    N\/%
    \kern-.35em%
    \lower.5ex\hbox{T\/}%
    \kern-.2em%
    S\/%
  }%
  \HOLOGO@SpaceFactor
}
%    \end{macrocode}
%    \end{macro}
%
% \subsubsection{\Hologo{TTH} (\hologo{TeX} to HTML translator)}
%
%    Source: \url{http://hutchinson.belmont.ma.us/tth/}
%    In the HTML source the second `T' is printed as subscript.
%\begin{quote}
%\begin{verbatim}
%T<sub>T</sub>H
%\end{verbatim}
%\end{quote}
%    \begin{macro}{\HoLogo@TTH}
%    \begin{macrocode}
\def\HoLogo@TTH#1{%
  \ltx@mbox{%
    T\HOLOGO@SubScript{T}H%
  }%
  \HOLOGO@SpaceFactor
}
%    \end{macrocode}
%    \end{macro}
%
%    \begin{macro}{\HoLogoHtml@TTH}
%    \begin{macrocode}
\def\HoLogoHtml@TTH#1{%
  T\HCode{<sub>}T\HCode{</sub>}H%
}
%    \end{macrocode}
%    \end{macro}
%
% \subsubsection{\Hologo{HanTheThanh}}
%
%    Partial source: Package \xpackage{dtklogos}.
%    The double accent is U+1EBF (latin small letter e with circumflex
%    and acute).
%    \begin{macro}{\HoLogo@HanTheThanh}
%    \begin{macrocode}
\def\HoLogo@HanTheThanh#1{%
  \ltx@mbox{H\`an}%
  \HOLOGO@space
  \ltx@mbox{%
    Th%
    \HOLOGO@IfCharExists{"1EBF}{%
      \char"1EBF\relax
    }{%
      \^e\hbox to 0pt{\hss\raise .5ex\hbox{\'{}}}%
    }%
  }%
  \HOLOGO@space
  \ltx@mbox{Th\`anh}%
}
%    \end{macrocode}
%    \end{macro}
%    \begin{macro}{\HoLogoBkm@HanTheThanh}
%    \begin{macrocode}
\def\HoLogoBkm@HanTheThanh#1{%
  H\`an %
  Th\HOLOGO@PdfdocUnicode{\^e}{\9036\277} %
  Th\`anh%
}
%    \end{macrocode}
%    \end{macro}
%    \begin{macro}{\HoLogoHtml@HanTheThanh}
%    \begin{macrocode}
\def\HoLogoHtml@HanTheThanh#1{%
  H\`an %
  Th\HCode{&\ltx@hashchar x1ebf;} %
  Th\`anh%
}
%    \end{macrocode}
%    \end{macro}
%
% \subsection{Driver detection}
%
%    \begin{macrocode}
\HOLOGO@IfExists\InputIfFileExists{%
  \InputIfFileExists{hologo.cfg}{}{}%
}{%
  \ltx@IfUndefined{pdf@filesize}{%
    \def\HOLOGO@InputIfExists{%
      \openin\HOLOGO@temp=hologo.cfg\relax
      \ifeof\HOLOGO@temp
        \closein\HOLOGO@temp
      \else
        \closein\HOLOGO@temp
        \begingroup
          \def\x{LaTeX2e}%
        \expandafter\endgroup
        \ifx\fmtname\x
          \input{hologo.cfg}%
        \else
          \input hologo.cfg\relax
        \fi
      \fi
    }%
    \ltx@IfUndefined{newread}{%
      \chardef\HOLOGO@temp=15 %
      \def\HOLOGO@CheckRead{%
        \ifeof\HOLOGO@temp
          \HOLOGO@InputIfExists
        \else
          \ifcase\HOLOGO@temp
            \@PackageWarningNoLine{hologo}{%
              Configuration file ignored, because\MessageBreak
              a free read register could not be found%
            }%
          \else
            \begingroup
              \count\ltx@cclv=\HOLOGO@temp
              \advance\ltx@cclv by \ltx@minusone
              \edef\x{\endgroup
                \chardef\noexpand\HOLOGO@temp=\the\count\ltx@cclv
                \relax
              }%
            \x
          \fi
        \fi
      }%
    }{%
      \csname newread\endcsname\HOLOGO@temp
      \HOLOGO@InputIfExists
    }%
  }{%
    \edef\HOLOGO@temp{\pdf@filesize{hologo.cfg}}%
    \ifx\HOLOGO@temp\ltx@empty
    \else
      \ifnum\HOLOGO@temp>0 %
        \begingroup
          \def\x{LaTeX2e}%
        \expandafter\endgroup
        \ifx\fmtname\x
          \input{hologo.cfg}%
        \else
          \input hologo.cfg\relax
        \fi
      \else
        \@PackageInfoNoLine{hologo}{%
          Empty configuration file `hologo.cfg' ignored%
        }%
      \fi
    \fi
  }%
}
%    \end{macrocode}
%
%    \begin{macrocode}
\def\HOLOGO@temp#1#2{%
  \kv@define@key{HoLogoDriver}{#1}[]{%
    \begingroup
      \def\HOLOGO@temp{##1}%
      \ltx@onelevel@sanitize\HOLOGO@temp
      \ifx\HOLOGO@temp\ltx@empty
      \else
        \@PackageError{hologo}{%
          Value (\HOLOGO@temp) not permitted for option `#1'%
        }%
        \@ehc
      \fi
    \endgroup
    \def\hologoDriver{#2}%
  }%
}%
\def\HOLOGO@@temp#1#2{%
  \ifx\kv@value\relax
    \HOLOGO@temp{#1}{#1}%
  \else
    \HOLOGO@temp{#1}{#2}%
  \fi
}%
\kv@parse@normalized{%
  pdftex,%
  luatex=pdftex,%
  dvipdfm,%
  dvipdfmx=dvipdfm,%
  dvips,%
  dvipsone=dvips,%
  xdvi=dvips,%
  xetex,%
  vtex,%
}\HOLOGO@@temp
%    \end{macrocode}
%
%    \begin{macrocode}
\kv@define@key{HoLogoDriver}{driverfallback}{%
  \def\HOLOGO@DriverFallback{#1}%
}
%    \end{macrocode}
%
%    \begin{macro}{\HOLOGO@DriverFallback}
%    \begin{macrocode}
\def\HOLOGO@DriverFallback{dvips}
%    \end{macrocode}
%    \end{macro}
%
%    \begin{macro}{\hologoDriverSetup}
%    \begin{macrocode}
\def\hologoDriverSetup{%
  \let\hologoDriver\ltx@undefined
  \HOLOGO@DriverSetup
}
%    \end{macrocode}
%    \end{macro}
%
%    \begin{macro}{\HOLOGO@DriverSetup}
%    \begin{macrocode}
\def\HOLOGO@DriverSetup#1{%
  \kvsetkeys{HoLogoDriver}{#1}%
  \HOLOGO@CheckDriver
  \ltx@ifundefined{hologoDriver}{%
    \begingroup
    \edef\x{\endgroup
      \noexpand\kvsetkeys{HoLogoDriver}{\HOLOGO@DriverFallback}%
    }\x
  }{}%
  \@PackageInfoNoLine{hologo}{Using driver `\hologoDriver'}%
}
%    \end{macrocode}
%    \end{macro}
%
%    \begin{macro}{\HOLOGO@CheckDriver}
%    \begin{macrocode}
\def\HOLOGO@CheckDriver{%
  \ifpdf
    \def\hologoDriver{pdftex}%
    \let\HOLOGO@pdfliteral\pdfliteral
    \ifluatex
      \ifx\pdfextension\@undefined\else
        \protected\def\pdfliteral{\pdfextension literal}%
        \let\HOLOGO@pdfliteral\pdfliteral
      \fi
      \ltx@IfUndefined{HOLOGO@pdfliteral}{%
        \ifnum\luatexversion<36 %
        \else
          \begingroup
            \let\HOLOGO@temp\endgroup
            \ifcase0%
                \directlua{%
                  if tex.enableprimitives then %
                    tex.enableprimitives('HOLOGO@', {'pdfliteral'})%
                  else %
                    tex.print('1')%
                  end%
                }%
                \ifx\HOLOGO@pdfliteral\@undefined 1\fi%
                \relax%
              \endgroup
              \let\HOLOGO@temp\relax
              \global\let\HOLOGO@pdfliteral\HOLOGO@pdfliteral
            \fi%
          \HOLOGO@temp
        \fi
      }{}%
    \fi
    \ltx@IfUndefined{HOLOGO@pdfliteral}{%
      \@PackageWarningNoLine{hologo}{%
        Cannot find \string\pdfliteral
      }%
    }{}%
  \else
    \ifxetex
      \def\hologoDriver{xetex}%
    \else
      \ifvtex
        \def\hologoDriver{vtex}%
      \fi
    \fi
  \fi
}
%    \end{macrocode}
%    \end{macro}
%
%    \begin{macro}{\HOLOGO@WarningUnsupportedDriver}
%    \begin{macrocode}
\def\HOLOGO@WarningUnsupportedDriver#1{%
  \@PackageWarningNoLine{hologo}{%
    Logo `#1' needs driver specific macros,\MessageBreak
    but driver `\hologoDriver' is not supported.\MessageBreak
    Use a different driver or\MessageBreak
    load package `graphics' or `pgf'%
  }%
}
%    \end{macrocode}
%    \end{macro}
%
% \subsubsection{Reflect box macros}
%
%    Skip driver part if not needed.
%    \begin{macrocode}
\ltx@IfUndefined{reflectbox}{}{%
  \ltx@IfUndefined{rotatebox}{}{%
    \HOLOGO@AtEnd
  }%
}
\ltx@IfUndefined{pgftext}{}{%
  \HOLOGO@AtEnd
}
\ltx@IfUndefined{psscalebox}{}{%
  \HOLOGO@AtEnd
}
%    \end{macrocode}
%
%    \begin{macrocode}
\def\HOLOGO@temp{LaTeX2e}
\ifx\fmtname\HOLOGO@temp
  \RequirePackage{kvoptions}[2011/06/30]%
  \ProcessKeyvalOptions{HoLogoDriver}%
\fi
\HOLOGO@DriverSetup{}
%    \end{macrocode}
%
%    \begin{macro}{\HOLOGO@ReflectBox}
%    \begin{macrocode}
\def\HOLOGO@ReflectBox#1{%
  \begingroup
    \setbox\ltx@zero\hbox{\begingroup#1\endgroup}%
    \setbox\ltx@two\hbox{%
      \kern\wd\ltx@zero
      \csname HOLOGO@ScaleBox@\hologoDriver\endcsname{-1}{1}{%
        \hbox to 0pt{\copy\ltx@zero\hss}%
      }%
    }%
    \wd\ltx@two=\wd\ltx@zero
    \box\ltx@two
  \endgroup
}
%    \end{macrocode}
%    \end{macro}
%
%    \begin{macro}{\HOLOGO@PointReflectBox}
%    \begin{macrocode}
\def\HOLOGO@PointReflectBox#1{%
  \begingroup
    \setbox\ltx@zero\hbox{\begingroup#1\endgroup}%
    \setbox\ltx@two\hbox{%
      \kern\wd\ltx@zero
      \raise\ht\ltx@zero\hbox{%
        \csname HOLOGO@ScaleBox@\hologoDriver\endcsname{-1}{-1}{%
          \hbox to 0pt{\copy\ltx@zero\hss}%
        }%
      }%
    }%
    \wd\ltx@two=\wd\ltx@zero
    \box\ltx@two
  \endgroup
}
%    \end{macrocode}
%    \end{macro}
%
%    We must define all variants because of dynamic driver setup.
%    \begin{macrocode}
\def\HOLOGO@temp#1#2{#2}
%    \end{macrocode}
%
%    \begin{macro}{\HOLOGO@ScaleBox@pdftex}
%    \begin{macrocode}
\HOLOGO@temp{pdftex}{%
  \def\HOLOGO@ScaleBox@pdftex#1#2#3{%
    \HOLOGO@pdfliteral{%
      q #1 0 0 #2 0 0 cm%
    }%
    #3%
    \HOLOGO@pdfliteral{%
      Q%
    }%
  }%
}
%    \end{macrocode}
%    \end{macro}
%    \begin{macro}{\HOLOGO@ScaleBox@dvips}
%    \begin{macrocode}
\HOLOGO@temp{dvips}{%
  \def\HOLOGO@ScaleBox@dvips#1#2#3{%
    \special{ps:%
      gsave %
      currentpoint %
      currentpoint translate %
      #1 #2 scale %
      neg exch neg exch translate%
    }%
    #3%
    \special{ps:%
      currentpoint %
      grestore %
      moveto%
    }%
  }%
}
%    \end{macrocode}
%    \end{macro}
%    \begin{macro}{\HOLOGO@ScaleBox@dvipdfm}
%    \begin{macrocode}
\HOLOGO@temp{dvipdfm}{%
  \let\HOLOGO@ScaleBox@dvipdfm\HOLOGO@ScaleBox@dvips
}
%    \end{macrocode}
%    \end{macro}
%    Since \hologo{XeTeX} v0.6.
%    \begin{macro}{\HOLOGO@ScaleBox@xetex}
%    \begin{macrocode}
\HOLOGO@temp{xetex}{%
  \def\HOLOGO@ScaleBox@xetex#1#2#3{%
    \special{x:gsave}%
    \special{x:scale #1 #2}%
    #3%
    \special{x:grestore}%
  }%
}
%    \end{macrocode}
%    \end{macro}
%    \begin{macro}{\HOLOGO@ScaleBox@vtex}
%    \begin{macrocode}
\HOLOGO@temp{vtex}{%
  \def\HOLOGO@ScaleBox@vtex#1#2#3{%
    \special{r(#1,0,0,#2,0,0}%
    #3%
    \special{r)}%
  }%
}
%    \end{macrocode}
%    \end{macro}
%
%    \begin{macrocode}
\HOLOGO@AtEnd%
%</package>
%    \end{macrocode}
%
% \section{Test}
%
% \subsection{Catcode checks for loading}
%
%    \begin{macrocode}
%<*test1>
%    \end{macrocode}
%    \begin{macrocode}
\catcode`\{=1 %
\catcode`\}=2 %
\catcode`\#=6 %
\catcode`\@=11 %
\expandafter\ifx\csname count@\endcsname\relax
  \countdef\count@=255 %
\fi
\expandafter\ifx\csname @gobble\endcsname\relax
  \long\def\@gobble#1{}%
\fi
\expandafter\ifx\csname @firstofone\endcsname\relax
  \long\def\@firstofone#1{#1}%
\fi
\expandafter\ifx\csname loop\endcsname\relax
  \expandafter\@firstofone
\else
  \expandafter\@gobble
\fi
{%
  \def\loop#1\repeat{%
    \def\body{#1}%
    \iterate
  }%
  \def\iterate{%
    \body
      \let\next\iterate
    \else
      \let\next\relax
    \fi
    \next
  }%
  \let\repeat=\fi
}%
\def\RestoreCatcodes{}
\count@=0 %
\loop
  \edef\RestoreCatcodes{%
    \RestoreCatcodes
    \catcode\the\count@=\the\catcode\count@\relax
  }%
\ifnum\count@<255 %
  \advance\count@ 1 %
\repeat

\def\RangeCatcodeInvalid#1#2{%
  \count@=#1\relax
  \loop
    \catcode\count@=15 %
  \ifnum\count@<#2\relax
    \advance\count@ 1 %
  \repeat
}
\def\RangeCatcodeCheck#1#2#3{%
  \count@=#1\relax
  \loop
    \ifnum#3=\catcode\count@
    \else
      \errmessage{%
        Character \the\count@\space
        with wrong catcode \the\catcode\count@\space
        instead of \number#3%
      }%
    \fi
  \ifnum\count@<#2\relax
    \advance\count@ 1 %
  \repeat
}
\def\space{ }
\expandafter\ifx\csname LoadCommand\endcsname\relax
  \def\LoadCommand{\input hologo.sty\relax}%
\fi
\def\Test{%
  \RangeCatcodeInvalid{0}{47}%
  \RangeCatcodeInvalid{58}{64}%
  \RangeCatcodeInvalid{91}{96}%
  \RangeCatcodeInvalid{123}{255}%
  \catcode`\@=12 %
  \catcode`\\=0 %
  \catcode`\%=14 %
  \LoadCommand
  \RangeCatcodeCheck{0}{36}{15}%
  \RangeCatcodeCheck{37}{37}{14}%
  \RangeCatcodeCheck{38}{47}{15}%
  \RangeCatcodeCheck{48}{57}{12}%
  \RangeCatcodeCheck{58}{63}{15}%
  \RangeCatcodeCheck{64}{64}{12}%
  \RangeCatcodeCheck{65}{90}{11}%
  \RangeCatcodeCheck{91}{91}{15}%
  \RangeCatcodeCheck{92}{92}{0}%
  \RangeCatcodeCheck{93}{96}{15}%
  \RangeCatcodeCheck{97}{122}{11}%
  \RangeCatcodeCheck{123}{255}{15}%
  \RestoreCatcodes
}
\Test
\csname @@end\endcsname
\end
%    \end{macrocode}
%    \begin{macrocode}
%</test1>
%    \end{macrocode}
%
% \subsection{Spacefactor}
%
%    The space factor must be 1000 after a logo. If it is greater 1000
%    then the following space is a space after a sentence closing point.
%    If the space factor is smaller 1000 then an immediate following
%    dot is interpreted as abbreviation, not sentence closing point.
%
%    \begin{macrocode}
%<*test-spacefactor>
\NeedsTeXFormat{LaTeX2e}
\documentclass{article}
\usepackage{hologo}[2016/05/12]
\usepackage{kvsetkeys}
\usepackage{qstest}
\IncludeTests{*}
\LogTests{log}{*}{*}
\begin{document}
\begin{qstest}{spacefactor}{spacefactor}
\newcommand*{\Test}[1]{%
  \sbox0{%
    \hologo{#1}%
    \Expect*{1000 (#1)}*{\the\spacefactor\space(#1)}%
  }%
}%
\makeatletter
\def\TestList{}
\def\hologoEntry#1#2#3{%
  \edef\TestList{%
    \ifx\TestList\@empty
    \else
      \TestList,%
    \fi
    #1%
    \ifx\\#2\\%
    \else
      ={variant=#2}%
    \fi
  }%
}
\hologoList
\expandafter\kv@parse@normalized\expandafter{%
  \TestList
}{%
  \begingroup
    \let\@logo=\kv@key
    \ifx\kv@value\relax
    \else
      \expandafter\hologoLogoSetup\expandafter\@logo\expandafter{%
        \kv@value
      }%
    \fi
    \Test\@logo
  \endgroup
  \@gobbletwo
}
\end{qstest}
\end{document}
%</test-spacefactor>
%    \end{macrocode}
%
% \subsection{Complete list}
%
%    \begin{macrocode}
%<*test-list>
\NeedsTeXFormat{LaTeX2e}
\documentclass[12pt,a4paper]{article}
\usepackage{hologo}[2016/05/12]
\usepackage[T1]{fontenc}
\usepackage{lmodern}
\usepackage{parskip}
\usepackage[unicode]{hyperref}[2011/09/28]
\usepackage{bookmark}[2011/09/19]
\bookmarksetup{%
  numbered,%
  open,%
  openlevel=2,%
}
\renewcommand*{\contentsname}{List of logos}
\begin{document}
\tableofcontents
\def\TestFont#1#2#3#4#5#6{%
  \begingroup
    \usefont{#3}{#4}{#5}{#6}%
    \HologoVariant{#1}{#2}/\hologoVariant{#1}{#2}%
    \quad
    \begingroup\scriptsize\hologoVariant{#1}{#2}\endgroup
    \quad
  \endgroup
  (#3/#4/#5/#6)%
  \par
}
\makeatletter
\def\hologoEntry#1#2#3{%
  \section{%
    \HologoVariant{#1}{#2}/\hologoVariant{#1}{#2} %
    {[#1\ifx\\#2\\\else\space(#2)\fi]}% hash-ok
  }% braces around [] because of bug in tex4ht
  \begingroup
    \hypersetup{unicode=false}%
    \bookmark[%
      dest=\@currentHref,%
      rellevel=1,%
      keeplevel,%
    ]{%
      \HologoVariant{#1}{#2}/\hologoVariant{#1}{#2} %
      (PDFDocEncoding)%
    }%
  \endgroup
  \TestFont{#1}{#2}{OT1}{cmr}{m}{n}%
  \TestFont{#1}{#2}{OT1}{cmss}{m}{n}%
  \TestFont{#1}{#2}{OT1}{cmr}{b}{n}%
  \TestFont{#1}{#2}{OT1}{cmr}{m}{it}%
  \TestFont{#1}{#2}{OT1}{cmtt}{m}{n}%
  \TestFont{#1}{#2}{T1}{lmr}{m}{n}%
  \TestFont{#1}{#2}{T1}{lmss}{m}{n}%
  \TestFont{#1}{#2}{T1}{lmr}{b}{n}%
  \TestFont{#1}{#2}{T1}{lmr}{m}{it}%
  \TestFont{#1}{#2}{T1}{lmtt}{m}{n}%
  \TestFont{#1}{#2}{T1}{lmvtt}{m}{n}%
  \TestFont{#1}{#2}{T1}{qtm}{m}{n}%
  \TestFont{#1}{#2}{T1}{qhv}{m}{n}%
  \TestFont{#1}{#2}{T1}{qtm}{b}{n}%
  \TestFont{#1}{#2}{T1}{qtm}{m}{it}%
  \TestFont{#1}{#2}{T1}{qcr}{m}{n}%
  \newpage
}
\makeatother
\hologoList
\end{document}
%</test-list>
%    \end{macrocode}
%
% \section{Installation}
%
% \subsection{Download}
%
% \paragraph{Package.} This package is available on
% CTAN\footnote{\url{ftp://ftp.ctan.org/tex-archive/}}:
% \begin{description}
% \item[\CTAN{macros/latex/contrib/oberdiek/hologo.dtx}] The source file.
% \item[\CTAN{macros/latex/contrib/oberdiek/hologo.pdf}] Documentation.
% \end{description}
%
%
% \paragraph{Bundle.} All the packages of the bundle `oberdiek'
% are also available in a TDS compliant ZIP archive. There
% the packages are already unpacked and the documentation files
% are generated. The files and directories obey the TDS standard.
% \begin{description}
% \item[\CTAN{install/macros/latex/contrib/oberdiek.tds.zip}]
% \end{description}
% \emph{TDS} refers to the standard ``A Directory Structure
% for \TeX\ Files'' (\CTAN{tds/tds.pdf}). Directories
% with \xfile{texmf} in their name are usually organized this way.
%
% \subsection{Bundle installation}
%
% \paragraph{Unpacking.} Unpack the \xfile{oberdiek.tds.zip} in the
% TDS tree (also known as \xfile{texmf} tree) of your choice.
% Example (linux):
% \begin{quote}
%   |unzip oberdiek.tds.zip -d ~/texmf|
% \end{quote}
%
% \paragraph{Script installation.}
% Check the directory \xfile{TDS:scripts/oberdiek/} for
% scripts that need further installation steps.
% Package \xpackage{attachfile2} comes with the Perl script
% \xfile{pdfatfi.pl} that should be installed in such a way
% that it can be called as \texttt{pdfatfi}.
% Example (linux):
% \begin{quote}
%   |chmod +x scripts/oberdiek/pdfatfi.pl|\\
%   |cp scripts/oberdiek/pdfatfi.pl /usr/local/bin/|
% \end{quote}
%
% \subsection{Package installation}
%
% \paragraph{Unpacking.} The \xfile{.dtx} file is a self-extracting
% \docstrip\ archive. The files are extracted by running the
% \xfile{.dtx} through \plainTeX:
% \begin{quote}
%   \verb|tex hologo.dtx|
% \end{quote}
%
% \paragraph{TDS.} Now the different files must be moved into
% the different directories in your installation TDS tree
% (also known as \xfile{texmf} tree):
% \begin{quote}
% \def\t{^^A
% \begin{tabular}{@{}>{\ttfamily}l@{ $\rightarrow$ }>{\ttfamily}l@{}}
%   hologo.sty & tex/generic/oberdiek/hologo.sty\\
%   hologo.pdf & doc/latex/oberdiek/hologo.pdf\\
%   example/hologo-example.tex & doc/latex/oberdiek/example/hologo-example.tex\\
%   test/hologo-test1.tex & doc/latex/oberdiek/test/hologo-test1.tex\\
%   test/hologo-test-spacefactor.tex & doc/latex/oberdiek/test/hologo-test-spacefactor.tex\\
%   test/hologo-test-list.tex & doc/latex/oberdiek/test/hologo-test-list.tex\\
%   hologo.dtx & source/latex/oberdiek/hologo.dtx\\
% \end{tabular}^^A
% }^^A
% \sbox0{\t}^^A
% \ifdim\wd0>\linewidth
%   \begingroup
%     \advance\linewidth by\leftmargin
%     \advance\linewidth by\rightmargin
%   \edef\x{\endgroup
%     \def\noexpand\lw{\the\linewidth}^^A
%   }\x
%   \def\lwbox{^^A
%     \leavevmode
%     \hbox to \linewidth{^^A
%       \kern-\leftmargin\relax
%       \hss
%       \usebox0
%       \hss
%       \kern-\rightmargin\relax
%     }^^A
%   }^^A
%   \ifdim\wd0>\lw
%     \sbox0{\small\t}^^A
%     \ifdim\wd0>\linewidth
%       \ifdim\wd0>\lw
%         \sbox0{\footnotesize\t}^^A
%         \ifdim\wd0>\linewidth
%           \ifdim\wd0>\lw
%             \sbox0{\scriptsize\t}^^A
%             \ifdim\wd0>\linewidth
%               \ifdim\wd0>\lw
%                 \sbox0{\tiny\t}^^A
%                 \ifdim\wd0>\linewidth
%                   \lwbox
%                 \else
%                   \usebox0
%                 \fi
%               \else
%                 \lwbox
%               \fi
%             \else
%               \usebox0
%             \fi
%           \else
%             \lwbox
%           \fi
%         \else
%           \usebox0
%         \fi
%       \else
%         \lwbox
%       \fi
%     \else
%       \usebox0
%     \fi
%   \else
%     \lwbox
%   \fi
% \else
%   \usebox0
% \fi
% \end{quote}
% If you have a \xfile{docstrip.cfg} that configures and enables \docstrip's
% TDS installing feature, then some files can already be in the right
% place, see the documentation of \docstrip.
%
% \subsection{Refresh file name databases}
%
% If your \TeX~distribution
% (\teTeX, \mikTeX, \dots) relies on file name databases, you must refresh
% these. For example, \teTeX\ users run \verb|texhash| or
% \verb|mktexlsr|.
%
% \subsection{Some details for the interested}
%
% \paragraph{Attached source.}
%
% The PDF documentation on CTAN also includes the
% \xfile{.dtx} source file. It can be extracted by
% AcrobatReader 6 or higher. Another option is \textsf{pdftk},
% e.g. unpack the file into the current directory:
% \begin{quote}
%   \verb|pdftk hologo.pdf unpack_files output .|
% \end{quote}
%
% \paragraph{Unpacking with \LaTeX.}
% The \xfile{.dtx} chooses its action depending on the format:
% \begin{description}
% \item[\plainTeX:] Run \docstrip\ and extract the files.
% \item[\LaTeX:] Generate the documentation.
% \end{description}
% If you insist on using \LaTeX\ for \docstrip\ (really,
% \docstrip\ does not need \LaTeX), then inform the autodetect routine
% about your intention:
% \begin{quote}
%   \verb|latex \let\install=y\input{hologo.dtx}|
% \end{quote}
% Do not forget to quote the argument according to the demands
% of your shell.
%
% \paragraph{Generating the documentation.}
% You can use both the \xfile{.dtx} or the \xfile{.drv} to generate
% the documentation. The process can be configured by the
% configuration file \xfile{ltxdoc.cfg}. For instance, put this
% line into this file, if you want to have A4 as paper format:
% \begin{quote}
%   \verb|\PassOptionsToClass{a4paper}{article}|
% \end{quote}
% An example follows how to generate the
% documentation with pdf\LaTeX:
% \begin{quote}
%\begin{verbatim}
%pdflatex hologo.dtx
%makeindex -s gind.ist hologo.idx
%pdflatex hologo.dtx
%makeindex -s gind.ist hologo.idx
%pdflatex hologo.dtx
%\end{verbatim}
% \end{quote}
%
% \section{Catalogue}
%
% The following XML file can be used as source for the
% \href{http://mirror.ctan.org/help/Catalogue/catalogue.html}{\TeX\ Catalogue}.
% The elements \texttt{caption} and \texttt{description} are imported
% from the original XML file from the Catalogue.
% The name of the XML file in the Catalogue is \xfile{hologo.xml}.
%    \begin{macrocode}
%<*catalogue>
<?xml version='1.0' encoding='us-ascii'?>
<!DOCTYPE entry SYSTEM 'catalogue.dtd'>
<entry datestamp='$Date$' modifier='$Author$' id='hologo'>
  <name>hologo</name>
  <caption>A collection of logos with bookmark support.</caption>
  <authorref id='auth:oberdiek'/>
  <copyright owner='Heiko Oberdiek' year='2010-2012'/>
  <license type='lppl1.3'/>
  <version number='1.10'/>
  <description>
    The package defines a single command <tt>\hologo</tt>, whose
    argument is the usual case-confused ASCII version of the logo.
    The command is bookmark-enabled, so that every logo becomes
    available in bookmarks without further work.
    <p/>
    The package is part of the <xref refid='oberdiek'>oberdiek</xref>
    bundle.
  </description>
  <documentation details='Package documentation'
      href='ctan:/macros/latex/contrib/oberdiek/hologo.pdf'/>
  <ctan file='true' path='/macros/latex/contrib/oberdiek/hologo.dtx'/>
  <miktex location='oberdiek'/>
  <texlive location='oberdiek'/>
  <install path='/macros/latex/contrib/oberdiek/oberdiek.tds.zip'/>
</entry>
%</catalogue>
%    \end{macrocode}
%
% \begin{thebibliography}{9}
% \raggedright
%
% \bibitem{btxdoc}
% Oren Patashnik,
% \textit{\hologo{BibTeX}ing},
% 1988-02-08.\\
% \CTAN{biblio/bibtex/base/}
%
% \bibitem{dtklogos}
% Gerd Neugebauer, DANTE,
% \textit{Package \xpackage{dtklogos}},
% 2011-04-25.\\
% \CTAN{usergrps/dante/dtk/dtklogos.sty}
%
% \bibitem{etexman}
% The \hologo{NTS} Team,
% \textit{The \hologo{eTeX} manual},
% 1998-02.\\
% \CTAN{systems/e-tex/v2/doc/}
%
% \bibitem{ExTeX-FAQ}
% The \hologo{ExTeX} group,
% \textit{\hologo{ExTeX}: FAQ -- How is \hologo{ExTeX} typeset?},
% 2007-04-14.\\
% \url{http://www.extex.org/documentation/faq.html}
%
% \bibitem{LyX}
% %@MISC{ LyX,
% %  title = {{LyX 2.0.0 -- The Document Processor [Computer software and manual]}},
% %  author = {{The LyX Team}},
% %  howpublished = {Internet: http://www.lyx.org},
% %  year = {2011-05-08},
% %  note = {Retrieved May 10, 2011, from http://www.lyx.org},
% %  url = {http://www.lyx.org/}
% %}
% The \hologo{LyX} Team,
% \textit{\hologo{LyX} -- The Document Processor},
% 2011-05-08.\\
% \url{http://www.lyx.org/}
%
% \bibitem{OzTeX}
% Andrew Trevorrow,
% \hologo{OzTeX} FAQ: What is the correct way to typeset ``\hologo{OzTeX}''?,
% 2011-09-15 (visited).
% \url{http://www.trevorrow.com/oztex/ozfaq.html#oztex-logo}
%
% \bibitem{PiCTeX}
% Michael Wichura,
% \textit{The \hologo{PiCTeX} macro package},
% 1987-09-21.
% \CTAN{graphics/pictex/}
%
% \bibitem{scrlogo}
% Markus Kohm,
% \textit{\hologo{KOMAScript} Datei \xfile{scrlogo.dtx}},
% 2009-01-30.\\
% \CTAN{install/macros/latex/contrib/komascript.tds.zip}
%
% \end{thebibliography}
%
% \begin{History}
%   \begin{Version}{2010/04/08 v1.0}
%   \item
%     The first version.
%   \end{Version}
%   \begin{Version}{2010/04/16 v1.1}
%   \item
%     \cs{Hologo} added for support of logos at start of a sentence.
%   \item
%     \cs{hologoSetup} and \cs{hologoLogoSetup} added.
%   \item
%     Options \xoption{break}, \xoption{hyphenbreak}, \xoption{spacebreak}
%     added.
%   \item
%     Variant support added by option \xoption{variant}.
%   \end{Version}
%   \begin{Version}{2010/04/24 v1.2}
%   \item
%     \hologo{LaTeX3} added.
%   \item
%     \hologo{VTeX} added.
%   \end{Version}
%   \begin{Version}{2010/11/21 v1.3}
%   \item
%     \hologo{iniTeX}, \hologo{virTeX} added.
%   \end{Version}
%   \begin{Version}{2011/03/25 v1.4}
%   \item
%     \hologo{ConTeXt} with variants added.
%   \item
%     Option \xoption{discretionarybreak} added as refinement for
%     option \xoption{break}.
%   \end{Version}
%   \begin{Version}{2011/04/21 v1.5}
%   \item
%     Wrong TDS directory for test files fixed.
%   \end{Version}
%   \begin{Version}{2011/10/01 v1.6}
%   \item
%     Support for package \xpackage{tex4ht} added.
%   \item
%     Support for \cs{csname} added if \cs{ifincsname} is available.
%   \item
%     New logos:
%     \hologo{(La)TeX},
%     \hologo{biber},
%     \hologo{BibTeX} (\xoption{sc}, \xoption{sf}),
%     \hologo{emTeX},
%     \hologo{ExTeX},
%     \hologo{KOMAScript},
%     \hologo{La},
%     \hologo{LyX},
%     \hologo{MiKTeX},
%     \hologo{NTS},
%     \hologo{OzMF},
%     \hologo{OzMP},
%     \hologo{OzTeX},
%     \hologo{OzTtH},
%     \hologo{PCTeX},
%     \hologo{PiC},
%     \hologo{PiCTeX},
%     \hologo{METAFONT},
%     \hologo{MetaFun},
%     \hologo{METAPOST},
%     \hologo{MetaPost},
%     \hologo{SLiTeX} (\xoption{lift}, \xoption{narrow}, \xoption{simple}),
%     \hologo{SliTeX} (\xoption{narrow}, \xoption{simple}, \xoption{lift}),
%     \hologo{teTeX}.
%   \item
%     Fixes:
%     \hologo{iniTeX},
%     \hologo{pdfLaTeX},
%     \hologo{pdfTeX},
%     \hologo{virTeX}.
%   \item
%     \cs{hologoFontSetup} and \cs{hologoLogoFontSetup} added.
%   \item
%     \cs{hologoVariant} and \cs{HologoVariant} added.
%   \end{Version}
%   \begin{Version}{2011/11/22 v1.7}
%   \item
%     New logos:
%     \hologo{BibTeX8},
%     \hologo{LaTeXML},
%     \hologo{SageTeX},
%     \hologo{TeX4ht},
%     \hologo{TTH}.
%   \item
%     \hologo{Xe} and friends: Driver stuff fixed.
%   \item
%     \hologo{Xe} and friends: Support for italic added.
%   \item
%     \hologo{Xe} and friends: Package support for \xpackage{pgf}
%     and \xpackage{pstricks} added.
%   \end{Version}
%   \begin{Version}{2011/11/29 v1.8}
%   \item
%     New logos:
%     \hologo{HanTheThanh}.
%   \end{Version}
%   \begin{Version}{2011/12/21 v1.9}
%   \item
%     Patch for package \xpackage{ifxetex} added for the case that
%     \cs{newif} is undefined in \hologo{iniTeX}.
%   \item
%     Some fixes for \hologo{iniTeX}.
%   \end{Version}
%   \begin{Version}{2012/04/26 v1.10}
%   \item
%     Fix in bookmark version of logo ``\hologo{HanTheThanh}''.
%   \end{Version}
%   \begin{Version}{2016/05/12 v1.11}
%   \item
%     Update HOLOGO@IfCharExists (previously in texlive)
%   \item define pdfliteral in current luatex.
%   \end{Version}
% \end{History}
%
% \PrintIndex
%
% \Finale
\endinput
%
        \else
          \input hologo.cfg\relax
        \fi
      \else
        \@PackageInfoNoLine{hologo}{%
          Empty configuration file `hologo.cfg' ignored%
        }%
      \fi
    \fi
  }%
}
%    \end{macrocode}
%
%    \begin{macrocode}
\def\HOLOGO@temp#1#2{%
  \kv@define@key{HoLogoDriver}{#1}[]{%
    \begingroup
      \def\HOLOGO@temp{##1}%
      \ltx@onelevel@sanitize\HOLOGO@temp
      \ifx\HOLOGO@temp\ltx@empty
      \else
        \@PackageError{hologo}{%
          Value (\HOLOGO@temp) not permitted for option `#1'%
        }%
        \@ehc
      \fi
    \endgroup
    \def\hologoDriver{#2}%
  }%
}%
\def\HOLOGO@@temp#1#2{%
  \ifx\kv@value\relax
    \HOLOGO@temp{#1}{#1}%
  \else
    \HOLOGO@temp{#1}{#2}%
  \fi
}%
\kv@parse@normalized{%
  pdftex,%
  luatex=pdftex,%
  dvipdfm,%
  dvipdfmx=dvipdfm,%
  dvips,%
  dvipsone=dvips,%
  xdvi=dvips,%
  xetex,%
  vtex,%
}\HOLOGO@@temp
%    \end{macrocode}
%
%    \begin{macrocode}
\kv@define@key{HoLogoDriver}{driverfallback}{%
  \def\HOLOGO@DriverFallback{#1}%
}
%    \end{macrocode}
%
%    \begin{macro}{\HOLOGO@DriverFallback}
%    \begin{macrocode}
\def\HOLOGO@DriverFallback{dvips}
%    \end{macrocode}
%    \end{macro}
%
%    \begin{macro}{\hologoDriverSetup}
%    \begin{macrocode}
\def\hologoDriverSetup{%
  \let\hologoDriver\ltx@undefined
  \HOLOGO@DriverSetup
}
%    \end{macrocode}
%    \end{macro}
%
%    \begin{macro}{\HOLOGO@DriverSetup}
%    \begin{macrocode}
\def\HOLOGO@DriverSetup#1{%
  \kvsetkeys{HoLogoDriver}{#1}%
  \HOLOGO@CheckDriver
  \ltx@ifundefined{hologoDriver}{%
    \begingroup
    \edef\x{\endgroup
      \noexpand\kvsetkeys{HoLogoDriver}{\HOLOGO@DriverFallback}%
    }\x
  }{}%
  \@PackageInfoNoLine{hologo}{Using driver `\hologoDriver'}%
}
%    \end{macrocode}
%    \end{macro}
%
%    \begin{macro}{\HOLOGO@CheckDriver}
%    \begin{macrocode}
\def\HOLOGO@CheckDriver{%
  \ifpdf
    \def\hologoDriver{pdftex}%
    \let\HOLOGO@pdfliteral\pdfliteral
    \ifluatex
      \ifx\pdfextension\@undefined\else
        \protected\def\pdfliteral{\pdfextension literal}%
        \let\HOLOGO@pdfliteral\pdfliteral
      \fi
      \ltx@IfUndefined{HOLOGO@pdfliteral}{%
        \ifnum\luatexversion<36 %
        \else
          \begingroup
            \let\HOLOGO@temp\endgroup
            \ifcase0%
                \directlua{%
                  if tex.enableprimitives then %
                    tex.enableprimitives('HOLOGO@', {'pdfliteral'})%
                  else %
                    tex.print('1')%
                  end%
                }%
                \ifx\HOLOGO@pdfliteral\@undefined 1\fi%
                \relax%
              \endgroup
              \let\HOLOGO@temp\relax
              \global\let\HOLOGO@pdfliteral\HOLOGO@pdfliteral
            \fi%
          \HOLOGO@temp
        \fi
      }{}%
    \fi
    \ltx@IfUndefined{HOLOGO@pdfliteral}{%
      \@PackageWarningNoLine{hologo}{%
        Cannot find \string\pdfliteral
      }%
    }{}%
  \else
    \ifxetex
      \def\hologoDriver{xetex}%
    \else
      \ifvtex
        \def\hologoDriver{vtex}%
      \fi
    \fi
  \fi
}
%    \end{macrocode}
%    \end{macro}
%
%    \begin{macro}{\HOLOGO@WarningUnsupportedDriver}
%    \begin{macrocode}
\def\HOLOGO@WarningUnsupportedDriver#1{%
  \@PackageWarningNoLine{hologo}{%
    Logo `#1' needs driver specific macros,\MessageBreak
    but driver `\hologoDriver' is not supported.\MessageBreak
    Use a different driver or\MessageBreak
    load package `graphics' or `pgf'%
  }%
}
%    \end{macrocode}
%    \end{macro}
%
% \subsubsection{Reflect box macros}
%
%    Skip driver part if not needed.
%    \begin{macrocode}
\ltx@IfUndefined{reflectbox}{}{%
  \ltx@IfUndefined{rotatebox}{}{%
    \HOLOGO@AtEnd
  }%
}
\ltx@IfUndefined{pgftext}{}{%
  \HOLOGO@AtEnd
}
\ltx@IfUndefined{psscalebox}{}{%
  \HOLOGO@AtEnd
}
%    \end{macrocode}
%
%    \begin{macrocode}
\def\HOLOGO@temp{LaTeX2e}
\ifx\fmtname\HOLOGO@temp
  \RequirePackage{kvoptions}[2011/06/30]%
  \ProcessKeyvalOptions{HoLogoDriver}%
\fi
\HOLOGO@DriverSetup{}
%    \end{macrocode}
%
%    \begin{macro}{\HOLOGO@ReflectBox}
%    \begin{macrocode}
\def\HOLOGO@ReflectBox#1{%
  \begingroup
    \setbox\ltx@zero\hbox{\begingroup#1\endgroup}%
    \setbox\ltx@two\hbox{%
      \kern\wd\ltx@zero
      \csname HOLOGO@ScaleBox@\hologoDriver\endcsname{-1}{1}{%
        \hbox to 0pt{\copy\ltx@zero\hss}%
      }%
    }%
    \wd\ltx@two=\wd\ltx@zero
    \box\ltx@two
  \endgroup
}
%    \end{macrocode}
%    \end{macro}
%
%    \begin{macro}{\HOLOGO@PointReflectBox}
%    \begin{macrocode}
\def\HOLOGO@PointReflectBox#1{%
  \begingroup
    \setbox\ltx@zero\hbox{\begingroup#1\endgroup}%
    \setbox\ltx@two\hbox{%
      \kern\wd\ltx@zero
      \raise\ht\ltx@zero\hbox{%
        \csname HOLOGO@ScaleBox@\hologoDriver\endcsname{-1}{-1}{%
          \hbox to 0pt{\copy\ltx@zero\hss}%
        }%
      }%
    }%
    \wd\ltx@two=\wd\ltx@zero
    \box\ltx@two
  \endgroup
}
%    \end{macrocode}
%    \end{macro}
%
%    We must define all variants because of dynamic driver setup.
%    \begin{macrocode}
\def\HOLOGO@temp#1#2{#2}
%    \end{macrocode}
%
%    \begin{macro}{\HOLOGO@ScaleBox@pdftex}
%    \begin{macrocode}
\HOLOGO@temp{pdftex}{%
  \def\HOLOGO@ScaleBox@pdftex#1#2#3{%
    \HOLOGO@pdfliteral{%
      q #1 0 0 #2 0 0 cm%
    }%
    #3%
    \HOLOGO@pdfliteral{%
      Q%
    }%
  }%
}
%    \end{macrocode}
%    \end{macro}
%    \begin{macro}{\HOLOGO@ScaleBox@dvips}
%    \begin{macrocode}
\HOLOGO@temp{dvips}{%
  \def\HOLOGO@ScaleBox@dvips#1#2#3{%
    \special{ps:%
      gsave %
      currentpoint %
      currentpoint translate %
      #1 #2 scale %
      neg exch neg exch translate%
    }%
    #3%
    \special{ps:%
      currentpoint %
      grestore %
      moveto%
    }%
  }%
}
%    \end{macrocode}
%    \end{macro}
%    \begin{macro}{\HOLOGO@ScaleBox@dvipdfm}
%    \begin{macrocode}
\HOLOGO@temp{dvipdfm}{%
  \let\HOLOGO@ScaleBox@dvipdfm\HOLOGO@ScaleBox@dvips
}
%    \end{macrocode}
%    \end{macro}
%    Since \hologo{XeTeX} v0.6.
%    \begin{macro}{\HOLOGO@ScaleBox@xetex}
%    \begin{macrocode}
\HOLOGO@temp{xetex}{%
  \def\HOLOGO@ScaleBox@xetex#1#2#3{%
    \special{x:gsave}%
    \special{x:scale #1 #2}%
    #3%
    \special{x:grestore}%
  }%
}
%    \end{macrocode}
%    \end{macro}
%    \begin{macro}{\HOLOGO@ScaleBox@vtex}
%    \begin{macrocode}
\HOLOGO@temp{vtex}{%
  \def\HOLOGO@ScaleBox@vtex#1#2#3{%
    \special{r(#1,0,0,#2,0,0}%
    #3%
    \special{r)}%
  }%
}
%    \end{macrocode}
%    \end{macro}
%
%    \begin{macrocode}
\HOLOGO@AtEnd%
%</package>
%    \end{macrocode}
%
% \section{Test}
%
% \subsection{Catcode checks for loading}
%
%    \begin{macrocode}
%<*test1>
%    \end{macrocode}
%    \begin{macrocode}
\catcode`\{=1 %
\catcode`\}=2 %
\catcode`\#=6 %
\catcode`\@=11 %
\expandafter\ifx\csname count@\endcsname\relax
  \countdef\count@=255 %
\fi
\expandafter\ifx\csname @gobble\endcsname\relax
  \long\def\@gobble#1{}%
\fi
\expandafter\ifx\csname @firstofone\endcsname\relax
  \long\def\@firstofone#1{#1}%
\fi
\expandafter\ifx\csname loop\endcsname\relax
  \expandafter\@firstofone
\else
  \expandafter\@gobble
\fi
{%
  \def\loop#1\repeat{%
    \def\body{#1}%
    \iterate
  }%
  \def\iterate{%
    \body
      \let\next\iterate
    \else
      \let\next\relax
    \fi
    \next
  }%
  \let\repeat=\fi
}%
\def\RestoreCatcodes{}
\count@=0 %
\loop
  \edef\RestoreCatcodes{%
    \RestoreCatcodes
    \catcode\the\count@=\the\catcode\count@\relax
  }%
\ifnum\count@<255 %
  \advance\count@ 1 %
\repeat

\def\RangeCatcodeInvalid#1#2{%
  \count@=#1\relax
  \loop
    \catcode\count@=15 %
  \ifnum\count@<#2\relax
    \advance\count@ 1 %
  \repeat
}
\def\RangeCatcodeCheck#1#2#3{%
  \count@=#1\relax
  \loop
    \ifnum#3=\catcode\count@
    \else
      \errmessage{%
        Character \the\count@\space
        with wrong catcode \the\catcode\count@\space
        instead of \number#3%
      }%
    \fi
  \ifnum\count@<#2\relax
    \advance\count@ 1 %
  \repeat
}
\def\space{ }
\expandafter\ifx\csname LoadCommand\endcsname\relax
  \def\LoadCommand{\input hologo.sty\relax}%
\fi
\def\Test{%
  \RangeCatcodeInvalid{0}{47}%
  \RangeCatcodeInvalid{58}{64}%
  \RangeCatcodeInvalid{91}{96}%
  \RangeCatcodeInvalid{123}{255}%
  \catcode`\@=12 %
  \catcode`\\=0 %
  \catcode`\%=14 %
  \LoadCommand
  \RangeCatcodeCheck{0}{36}{15}%
  \RangeCatcodeCheck{37}{37}{14}%
  \RangeCatcodeCheck{38}{47}{15}%
  \RangeCatcodeCheck{48}{57}{12}%
  \RangeCatcodeCheck{58}{63}{15}%
  \RangeCatcodeCheck{64}{64}{12}%
  \RangeCatcodeCheck{65}{90}{11}%
  \RangeCatcodeCheck{91}{91}{15}%
  \RangeCatcodeCheck{92}{92}{0}%
  \RangeCatcodeCheck{93}{96}{15}%
  \RangeCatcodeCheck{97}{122}{11}%
  \RangeCatcodeCheck{123}{255}{15}%
  \RestoreCatcodes
}
\Test
\csname @@end\endcsname
\end
%    \end{macrocode}
%    \begin{macrocode}
%</test1>
%    \end{macrocode}
%
% \subsection{Spacefactor}
%
%    The space factor must be 1000 after a logo. If it is greater 1000
%    then the following space is a space after a sentence closing point.
%    If the space factor is smaller 1000 then an immediate following
%    dot is interpreted as abbreviation, not sentence closing point.
%
%    \begin{macrocode}
%<*test-spacefactor>
\NeedsTeXFormat{LaTeX2e}
\documentclass{article}
\usepackage{hologo}[2016/05/12]
\usepackage{kvsetkeys}
\usepackage{qstest}
\IncludeTests{*}
\LogTests{log}{*}{*}
\begin{document}
\begin{qstest}{spacefactor}{spacefactor}
\newcommand*{\Test}[1]{%
  \sbox0{%
    \hologo{#1}%
    \Expect*{1000 (#1)}*{\the\spacefactor\space(#1)}%
  }%
}%
\makeatletter
\def\TestList{}
\def\hologoEntry#1#2#3{%
  \edef\TestList{%
    \ifx\TestList\@empty
    \else
      \TestList,%
    \fi
    #1%
    \ifx\\#2\\%
    \else
      ={variant=#2}%
    \fi
  }%
}
\hologoList
\expandafter\kv@parse@normalized\expandafter{%
  \TestList
}{%
  \begingroup
    \let\@logo=\kv@key
    \ifx\kv@value\relax
    \else
      \expandafter\hologoLogoSetup\expandafter\@logo\expandafter{%
        \kv@value
      }%
    \fi
    \Test\@logo
  \endgroup
  \@gobbletwo
}
\end{qstest}
\end{document}
%</test-spacefactor>
%    \end{macrocode}
%
% \subsection{Complete list}
%
%    \begin{macrocode}
%<*test-list>
\NeedsTeXFormat{LaTeX2e}
\documentclass[12pt,a4paper]{article}
\usepackage{hologo}[2016/05/12]
\usepackage[T1]{fontenc}
\usepackage{lmodern}
\usepackage{parskip}
\usepackage[unicode]{hyperref}[2011/09/28]
\usepackage{bookmark}[2011/09/19]
\bookmarksetup{%
  numbered,%
  open,%
  openlevel=2,%
}
\renewcommand*{\contentsname}{List of logos}
\begin{document}
\tableofcontents
\def\TestFont#1#2#3#4#5#6{%
  \begingroup
    \usefont{#3}{#4}{#5}{#6}%
    \HologoVariant{#1}{#2}/\hologoVariant{#1}{#2}%
    \quad
    \begingroup\scriptsize\hologoVariant{#1}{#2}\endgroup
    \quad
  \endgroup
  (#3/#4/#5/#6)%
  \par
}
\makeatletter
\def\hologoEntry#1#2#3{%
  \section{%
    \HologoVariant{#1}{#2}/\hologoVariant{#1}{#2} %
    {[#1\ifx\\#2\\\else\space(#2)\fi]}% hash-ok
  }% braces around [] because of bug in tex4ht
  \begingroup
    \hypersetup{unicode=false}%
    \bookmark[%
      dest=\@currentHref,%
      rellevel=1,%
      keeplevel,%
    ]{%
      \HologoVariant{#1}{#2}/\hologoVariant{#1}{#2} %
      (PDFDocEncoding)%
    }%
  \endgroup
  \TestFont{#1}{#2}{OT1}{cmr}{m}{n}%
  \TestFont{#1}{#2}{OT1}{cmss}{m}{n}%
  \TestFont{#1}{#2}{OT1}{cmr}{b}{n}%
  \TestFont{#1}{#2}{OT1}{cmr}{m}{it}%
  \TestFont{#1}{#2}{OT1}{cmtt}{m}{n}%
  \TestFont{#1}{#2}{T1}{lmr}{m}{n}%
  \TestFont{#1}{#2}{T1}{lmss}{m}{n}%
  \TestFont{#1}{#2}{T1}{lmr}{b}{n}%
  \TestFont{#1}{#2}{T1}{lmr}{m}{it}%
  \TestFont{#1}{#2}{T1}{lmtt}{m}{n}%
  \TestFont{#1}{#2}{T1}{lmvtt}{m}{n}%
  \TestFont{#1}{#2}{T1}{qtm}{m}{n}%
  \TestFont{#1}{#2}{T1}{qhv}{m}{n}%
  \TestFont{#1}{#2}{T1}{qtm}{b}{n}%
  \TestFont{#1}{#2}{T1}{qtm}{m}{it}%
  \TestFont{#1}{#2}{T1}{qcr}{m}{n}%
  \newpage
}
\makeatother
\hologoList
\end{document}
%</test-list>
%    \end{macrocode}
%
% \section{Installation}
%
% \subsection{Download}
%
% \paragraph{Package.} This package is available on
% CTAN\footnote{\url{ftp://ftp.ctan.org/tex-archive/}}:
% \begin{description}
% \item[\CTAN{macros/latex/contrib/oberdiek/hologo.dtx}] The source file.
% \item[\CTAN{macros/latex/contrib/oberdiek/hologo.pdf}] Documentation.
% \end{description}
%
%
% \paragraph{Bundle.} All the packages of the bundle `oberdiek'
% are also available in a TDS compliant ZIP archive. There
% the packages are already unpacked and the documentation files
% are generated. The files and directories obey the TDS standard.
% \begin{description}
% \item[\CTAN{install/macros/latex/contrib/oberdiek.tds.zip}]
% \end{description}
% \emph{TDS} refers to the standard ``A Directory Structure
% for \TeX\ Files'' (\CTAN{tds/tds.pdf}). Directories
% with \xfile{texmf} in their name are usually organized this way.
%
% \subsection{Bundle installation}
%
% \paragraph{Unpacking.} Unpack the \xfile{oberdiek.tds.zip} in the
% TDS tree (also known as \xfile{texmf} tree) of your choice.
% Example (linux):
% \begin{quote}
%   |unzip oberdiek.tds.zip -d ~/texmf|
% \end{quote}
%
% \paragraph{Script installation.}
% Check the directory \xfile{TDS:scripts/oberdiek/} for
% scripts that need further installation steps.
% Package \xpackage{attachfile2} comes with the Perl script
% \xfile{pdfatfi.pl} that should be installed in such a way
% that it can be called as \texttt{pdfatfi}.
% Example (linux):
% \begin{quote}
%   |chmod +x scripts/oberdiek/pdfatfi.pl|\\
%   |cp scripts/oberdiek/pdfatfi.pl /usr/local/bin/|
% \end{quote}
%
% \subsection{Package installation}
%
% \paragraph{Unpacking.} The \xfile{.dtx} file is a self-extracting
% \docstrip\ archive. The files are extracted by running the
% \xfile{.dtx} through \plainTeX:
% \begin{quote}
%   \verb|tex hologo.dtx|
% \end{quote}
%
% \paragraph{TDS.} Now the different files must be moved into
% the different directories in your installation TDS tree
% (also known as \xfile{texmf} tree):
% \begin{quote}
% \def\t{^^A
% \begin{tabular}{@{}>{\ttfamily}l@{ $\rightarrow$ }>{\ttfamily}l@{}}
%   hologo.sty & tex/generic/oberdiek/hologo.sty\\
%   hologo.pdf & doc/latex/oberdiek/hologo.pdf\\
%   example/hologo-example.tex & doc/latex/oberdiek/example/hologo-example.tex\\
%   test/hologo-test1.tex & doc/latex/oberdiek/test/hologo-test1.tex\\
%   test/hologo-test-spacefactor.tex & doc/latex/oberdiek/test/hologo-test-spacefactor.tex\\
%   test/hologo-test-list.tex & doc/latex/oberdiek/test/hologo-test-list.tex\\
%   hologo.dtx & source/latex/oberdiek/hologo.dtx\\
% \end{tabular}^^A
% }^^A
% \sbox0{\t}^^A
% \ifdim\wd0>\linewidth
%   \begingroup
%     \advance\linewidth by\leftmargin
%     \advance\linewidth by\rightmargin
%   \edef\x{\endgroup
%     \def\noexpand\lw{\the\linewidth}^^A
%   }\x
%   \def\lwbox{^^A
%     \leavevmode
%     \hbox to \linewidth{^^A
%       \kern-\leftmargin\relax
%       \hss
%       \usebox0
%       \hss
%       \kern-\rightmargin\relax
%     }^^A
%   }^^A
%   \ifdim\wd0>\lw
%     \sbox0{\small\t}^^A
%     \ifdim\wd0>\linewidth
%       \ifdim\wd0>\lw
%         \sbox0{\footnotesize\t}^^A
%         \ifdim\wd0>\linewidth
%           \ifdim\wd0>\lw
%             \sbox0{\scriptsize\t}^^A
%             \ifdim\wd0>\linewidth
%               \ifdim\wd0>\lw
%                 \sbox0{\tiny\t}^^A
%                 \ifdim\wd0>\linewidth
%                   \lwbox
%                 \else
%                   \usebox0
%                 \fi
%               \else
%                 \lwbox
%               \fi
%             \else
%               \usebox0
%             \fi
%           \else
%             \lwbox
%           \fi
%         \else
%           \usebox0
%         \fi
%       \else
%         \lwbox
%       \fi
%     \else
%       \usebox0
%     \fi
%   \else
%     \lwbox
%   \fi
% \else
%   \usebox0
% \fi
% \end{quote}
% If you have a \xfile{docstrip.cfg} that configures and enables \docstrip's
% TDS installing feature, then some files can already be in the right
% place, see the documentation of \docstrip.
%
% \subsection{Refresh file name databases}
%
% If your \TeX~distribution
% (\teTeX, \mikTeX, \dots) relies on file name databases, you must refresh
% these. For example, \teTeX\ users run \verb|texhash| or
% \verb|mktexlsr|.
%
% \subsection{Some details for the interested}
%
% \paragraph{Attached source.}
%
% The PDF documentation on CTAN also includes the
% \xfile{.dtx} source file. It can be extracted by
% AcrobatReader 6 or higher. Another option is \textsf{pdftk},
% e.g. unpack the file into the current directory:
% \begin{quote}
%   \verb|pdftk hologo.pdf unpack_files output .|
% \end{quote}
%
% \paragraph{Unpacking with \LaTeX.}
% The \xfile{.dtx} chooses its action depending on the format:
% \begin{description}
% \item[\plainTeX:] Run \docstrip\ and extract the files.
% \item[\LaTeX:] Generate the documentation.
% \end{description}
% If you insist on using \LaTeX\ for \docstrip\ (really,
% \docstrip\ does not need \LaTeX), then inform the autodetect routine
% about your intention:
% \begin{quote}
%   \verb|latex \let\install=y% \iffalse meta-comment
%
% File: hologo.dtx
% Version: 2016/05/12 v1.11
% Info: A logo collection with bookmark support
%
% Copyright (C) 2010-2012 by
%    Heiko Oberdiek <heiko.oberdiek at googlemail.com>
%
% This work may be distributed and/or modified under the
% conditions of the LaTeX Project Public License, either
% version 1.3c of this license or (at your option) any later
% version. This version of this license is in
%    http://www.latex-project.org/lppl/lppl-1-3c.txt
% and the latest version of this license is in
%    http://www.latex-project.org/lppl.txt
% and version 1.3 or later is part of all distributions of
% LaTeX version 2005/12/01 or later.
%
% This work has the LPPL maintenance status "maintained".
%
% This Current Maintainer of this work is Heiko Oberdiek.
%
% The Base Interpreter refers to any `TeX-Format',
% because some files are installed in TDS:tex/generic//.
%
% This work consists of the main source file hologo.dtx
% and the derived files
%    hologo.sty, hologo.pdf, hologo.ins, hologo.drv, hologo-example.tex,
%    hologo-test1.tex, hologo-test-spacefactor.tex,
%    hologo-test-list.tex.
%
% Distribution:
%    CTAN:macros/latex/contrib/oberdiek/hologo.dtx
%    CTAN:macros/latex/contrib/oberdiek/hologo.pdf
%
% Unpacking:
%    (a) If hologo.ins is present:
%           tex hologo.ins
%    (b) Without hologo.ins:
%           tex hologo.dtx
%    (c) If you insist on using LaTeX
%           latex \let\install=y\input{hologo.dtx}
%        (quote the arguments according to the demands of your shell)
%
% Documentation:
%    (a) If hologo.drv is present:
%           latex hologo.drv
%    (b) Without hologo.drv:
%           latex hologo.dtx; ...
%    The class ltxdoc loads the configuration file ltxdoc.cfg
%    if available. Here you can specify further options, e.g.
%    use A4 as paper format:
%       \PassOptionsToClass{a4paper}{article}
%
%    Programm calls to get the documentation (example):
%       pdflatex hologo.dtx
%       makeindex -s gind.ist hologo.idx
%       pdflatex hologo.dtx
%       makeindex -s gind.ist hologo.idx
%       pdflatex hologo.dtx
%
% Installation:
%    TDS:tex/generic/oberdiek/hologo.sty
%    TDS:doc/latex/oberdiek/hologo.pdf
%    TDS:doc/latex/oberdiek/example/hologo-example.tex
%    TDS:doc/latex/oberdiek/test/hologo-test1.tex
%    TDS:doc/latex/oberdiek/test/hologo-test-spacefactor.tex
%    TDS:doc/latex/oberdiek/test/hologo-test-list.tex
%    TDS:source/latex/oberdiek/hologo.dtx
%
%<*ignore>
\begingroup
  \catcode123=1 %
  \catcode125=2 %
  \def\x{LaTeX2e}%
\expandafter\endgroup
\ifcase 0\ifx\install y1\fi\expandafter
         \ifx\csname processbatchFile\endcsname\relax\else1\fi
         \ifx\fmtname\x\else 1\fi\relax
\else\csname fi\endcsname
%</ignore>
%<*install>
\input docstrip.tex
\Msg{************************************************************************}
\Msg{* Installation}
\Msg{* Package: hologo 2016/05/12 v1.11 A logo collection with bookmark support (HO)}
\Msg{************************************************************************}

\keepsilent
\askforoverwritefalse

\let\MetaPrefix\relax
\preamble

This is a generated file.

Project: hologo
Version: 2016/05/12 v1.11

Copyright (C) 2010-2012 by
   Heiko Oberdiek <heiko.oberdiek at googlemail.com>

This work may be distributed and/or modified under the
conditions of the LaTeX Project Public License, either
version 1.3c of this license or (at your option) any later
version. This version of this license is in
   http://www.latex-project.org/lppl/lppl-1-3c.txt
and the latest version of this license is in
   http://www.latex-project.org/lppl.txt
and version 1.3 or later is part of all distributions of
LaTeX version 2005/12/01 or later.

This work has the LPPL maintenance status "maintained".

This Current Maintainer of this work is Heiko Oberdiek.

The Base Interpreter refers to any `TeX-Format',
because some files are installed in TDS:tex/generic//.

This work consists of the main source file hologo.dtx
and the derived files
   hologo.sty, hologo.pdf, hologo.ins, hologo.drv, hologo-example.tex,
   hologo-test1.tex, hologo-test-spacefactor.tex,
   hologo-test-list.tex.

\endpreamble
\let\MetaPrefix\DoubleperCent

\generate{%
  \file{hologo.ins}{\from{hologo.dtx}{install}}%
  \file{hologo.drv}{\from{hologo.dtx}{driver}}%
  \usedir{tex/generic/oberdiek}%
  \file{hologo.sty}{\from{hologo.dtx}{package}}%
  \usedir{doc/latex/oberdiek/example}%
  \file{hologo-example.tex}{\from{hologo.dtx}{example}}%
  \usedir{doc/latex/oberdiek/test}%
  \file{hologo-test1.tex}{\from{hologo.dtx}{test1}}%
  \file{hologo-test-spacefactor.tex}{\from{hologo.dtx}{test-spacefactor}}%
  \file{hologo-test-list.tex}{\from{hologo.dtx}{test-list}}%
  \nopreamble
  \nopostamble
  \usedir{source/latex/oberdiek/catalogue}%
  \file{hologo.xml}{\from{hologo.dtx}{catalogue}}%
}

\catcode32=13\relax% active space
\let =\space%
\Msg{************************************************************************}
\Msg{*}
\Msg{* To finish the installation you have to move the following}
\Msg{* file into a directory searched by TeX:}
\Msg{*}
\Msg{*     hologo.sty}
\Msg{*}
\Msg{* To produce the documentation run the file `hologo.drv'}
\Msg{* through LaTeX.}
\Msg{*}
\Msg{* Happy TeXing!}
\Msg{*}
\Msg{************************************************************************}

\endbatchfile
%</install>
%<*ignore>
\fi
%</ignore>
%<*driver>
\NeedsTeXFormat{LaTeX2e}
\ProvidesFile{hologo.drv}%
  [2016/05/12 v1.11 A logo collection with bookmark support (HO)]%
\documentclass{ltxdoc}
\usepackage{holtxdoc}[2011/11/22]
\usepackage{hologo}[2016/05/12]
\usepackage{longtable}
\usepackage{array}
\usepackage{paralist}
%\usepackage[T1]{fontenc}
%\usepackage{lmodern}
\begin{document}
  \DocInput{hologo.dtx}%
\end{document}
%</driver>
% \fi
%
%
% \CharacterTable
%  {Upper-case    \A\B\C\D\E\F\G\H\I\J\K\L\M\N\O\P\Q\R\S\T\U\V\W\X\Y\Z
%   Lower-case    \a\b\c\d\e\f\g\h\i\j\k\l\m\n\o\p\q\r\s\t\u\v\w\x\y\z
%   Digits        \0\1\2\3\4\5\6\7\8\9
%   Exclamation   \!     Double quote  \"     Hash (number) \#
%   Dollar        \$     Percent       \%     Ampersand     \&
%   Acute accent  \'     Left paren    \(     Right paren   \)
%   Asterisk      \*     Plus          \+     Comma         \,
%   Minus         \-     Point         \.     Solidus       \/
%   Colon         \:     Semicolon     \;     Less than     \<
%   Equals        \=     Greater than  \>     Question mark \?
%   Commercial at \@     Left bracket  \[     Backslash     \\
%   Right bracket \]     Circumflex    \^     Underscore    \_
%   Grave accent  \`     Left brace    \{     Vertical bar  \|
%   Right brace   \}     Tilde         \~}
%
% \GetFileInfo{hologo.drv}
%
% \title{The \xpackage{hologo} package}
% \date{2016/05/12 v1.11}
% \author{Heiko Oberdiek\\\xemail{heiko.oberdiek at googlemail.com}}
%
% \maketitle
%
% \begin{abstract}
% This package starts a collection of logos with support for bookmarks
% strings.
% \end{abstract}
%
% \tableofcontents
%
% \section{Documentation}
%
% \subsection{Logo macros}
%
% \begin{declcs}{hologo} \M{name}
% \end{declcs}
% Macro \cs{hologo} sets the logo with name \meta{name}.
% The following table shows the supported names.
%
% \begingroup
%   \def\hologoEntry#1#2#3{^^A
%     #1&#2&\hologoLogoSetup{#1}{variant=#2}\hologo{#1}&#3\tabularnewline
%   }
%   \begin{longtable}{>{\ttfamily}l>{\ttfamily}lll}
%     \rmfamily\bfseries{name} & \rmfamily\bfseries variant
%     & \bfseries logo & \bfseries since\\
%     \hline
%     \endhead
%     \hologoList
%   \end{longtable}
% \endgroup
%
% \begin{declcs}{Hologo} \M{name}
% \end{declcs}
% Macro \cs{Hologo} starts the logo \meta{name} with an uppercase
% letter. As an exception small greek letters are not converted
% to uppercase. Examples, see \hologo{eTeX} and \hologo{ExTeX}.
%
% \subsection{Setup macros}
%
% The package does not support package options, but the following
% setup macros can be used to set options.
%
% \begin{declcs}{hologoSetup} \M{key value list}
% \end{declcs}
% Macro \cs{hologoSetup} sets global options.
%
% \begin{declcs}{hologoLogoSetup} \M{logo} \M{key value list}
% \end{declcs}
% Some options can also be used to configure a logo.
% These settings take precedence over global option settings.
%
% \subsection{Options}\label{sec:options}
%
% There are boolean and string options:
% \begin{description}
% \item[Boolean option:]
% It takes |true| or |false|
% as value. If the value is omitted, then |true| is used.
% \item[String option:]
% A value must be given as string. (But the string might be empty.)
% \end{description}
% The following options can be used both in \cs{hologoSetup}
% and \cs{hologoLogoSetup}:
% \begin{description}
% \def\entry#1{\item[\xoption{#1}:]}
% \entry{break}
%   enables or disables line breaks inside the logo. This setting is
%   refined by options \xoption{hyphenbreak}, \xoption{spacebreak}
%   or \xoption{discretionarybreak}.
%   Default is |false|.
% \entry{hyphenbreak}
%   enables or disables the line break right after the hyphen character.
% \entry{spacebreak}
%   enables or disables line breaks at space characters.
% \entry{discretionarybreak}
%   enables or disables line breaks at hyphenation points
%   (inserted by \cs{-}).
% \end{description}
% Macro \cs{hologoLogoSetup} also knows:
% \begin{description}
% \item[\xoption{variant}:]
%   This is a string option. It specifies a variant of a logo that
%   must exist. An empty string selects the package default variant.
% \end{description}
% Example:
% \begin{quote}
%   |\hologoSetup{break=false}|\\
%   |\hologoLogoSetup{plainTeX}{variant=hyphen,hyphenbreak}|\\
%   Then ``plain-\TeX'' contains one break point after the hyphen.
% \end{quote}
%
% \subsection{Driver options}
%
% Sometimes graphical operations are needed to construct some
% glyphs (e.g.\ \hologo{XeTeX}). If package \xpackage{graphics}
% or package \xpackage{pgf} are found, then the macros are taken
% from there. Otherwise the packge defines its own operations
% and therefore needs the driver information. Many drivers are
% detected automatically (\hologo{pdfTeX}/\hologo{LuaTeX}
% in PDF mode, \hologo{XeTeX}, \hologo{VTeX}). These have precedence
% over a driver option. The driver can be given as package option
% or using \cs{hologoDriverSetup}.
% The following list contains the recognized driver options:
% \begin{itemize}
% \item \xoption{pdftex}, \xoption{luatex}
% \item \xoption{dvipdfm}, \xoption{dvipdfmx}
% \item \xoption{dvips}, \xoption{dvipsone}, \xoption{xdvi}
% \item \xoption{xetex}
% \item \xoption{vtex}
% \end{itemize}
% The left driver of a line is the driver name that is used internally.
% The following names are aliases for drivers that use the
% same method. Therefore the entry in the \xext{log} file for
% the used driver prints the internally used driver name.
% \begin{description}
% \item[\xoption{driverfallback}:]
%   This option expects a driver that is used,
%   if the driver could not be detected automatically.
% \end{description}
%
% \begin{declcs}{hologoDriverSetup} \M{driver option}
% \end{declcs}
% The driver can also be configured after package loading
% using \cs{hologoDriverSetup}, also the way for \hologo{plainTeX}
% to setup the driver.
%
% \subsection{Font setup}
%
% Some logos require a special font, but should also be usable by
% \hologo{plainTeX}. Therefore the package provides some ways
% to influence the font settings. The options below
% take font settings as values. Both font commands
% such as \cs{sffamily} and macros that take one argument
% like \cs{textsf} can be used.
%
% \begin{declcs}{hologoFontSetup} \M{key value list}
% \end{declcs}
% Macro \cs{hologoFontSetup} sets the fonts for all logos.
% Supported keys:
% \begin{description}
% \def\entry#1{\item[\xoption{#1}:]}
% \entry{general}
%   This font is used for all logos. The default is empty.
%   That means no special font is used.
% \entry{bibsf}
%   This font is used for
%   {\hologoLogoSetup{BibTeX}{variant=sf}\hologo{BibTeX}}
%   with variant \xoption{sf}.
% \entry{rm}
%   This font is a serif font. It is used for \hologo{ExTeX}.
% \entry{sc}
%   This font specifies a small caps font. It is used for
%   {\hologoLogoSetup{BibTeX}{variant=sc}\hologo{BibTeX}}
%   with variant \xoption{sc}.
% \entry{sf}
%   This font specifies a sans serif font. The default
%   is \cs{sffamily}, then \cs{sf} is tried. Otherwise
%   a warning is given. It is used by \hologo{KOMAScript}.
% \entry{sy}
%   This is the font for math symbols (e.g. cmsy).
%   It is used by \hologo{AmS}, \hologo{NTS}, \hologo{ExTeX}.
% \entry{logo}
%   \hologo{METAFONT} and \hologo{METAPOST} are using that font.
%   In \hologo{LaTeX} \cs{logofamily} is used and
%   the definitions of package \xpackage{mflogo} are used
%   if the package is not loaded.
%   Otherwise the \cs{tenlogo} is used and defined
%   if it does not already exists.
% \end{description}
%
% \begin{declcs}{hologoLogoFontSetup} \M{logo} \M{key value list}
% \end{declcs}
% Fonts can also be set for a logo or logo component separately,
% see the following list.
% The keys are the same as for \cs{hologoFontSetup}.
%
% \begin{longtable}{>{\ttfamily}l>{\sffamily}ll}
%   \meta{logo} & keys & result\\
%   \hline
%   \endhead
%   BibTeX & bibsf & {\hologoLogoSetup{BibTeX}{variant=sf}\hologo{BibTeX}}\\[.5ex]
%   BibTeX & sc & {\hologoLogoSetup{BibTeX}{variant=sc}\hologo{BibTeX}}\\[.5ex]
%   ExTeX & rm & \hologo{ExTeX}\\
%   SliTeX & rm & \hologo{SliTeX}\\[.5ex]
%   AmS & sy & \hologo{AmS}\\
%   ExTeX & sy & \hologo{ExTeX}\\
%   NTS & sy & \hologo{NTS}\\[.5ex]
%   KOMAScript & sf & \hologo{KOMAScript}\\[.5ex]
%   METAFONT & logo & \hologo{METAFONT}\\
%   METAPOST & logo & \hologo{METAPOST}\\[.5ex]
%   SliTeX & sc \hologo{SliTeX}
% \end{longtable}
%
% \subsubsection{Font order}
%
% For all logos the font \xoption{general} is applied first.
% Example:
%\begin{quote}
%|\hologoFontSetup{general=\color{red}}|
%\end{quote}
% will print red logos.
% Then if the font uses a special font \xoption{sf}, for example,
% the font is applied that is setup by \cs{hologoLogoFontSetup}.
% If this font is not setup, then the common font setup
% by \cs{hologoFontSetup} is used. Otherwise a warning is given,
% that there is no font configured.
%
% \subsection{Additional user macros}
%
% Usually a variant of a logo is configured by using
% \cs{hologoLogoSetup}, because it is bad style to mix
% different variants of the same logo in the same text.
% There the following macros are a convenience for testing.
%
% \begin{declcs}{hologoVariant} \M{name} \M{variant}\\
%   \cs{HologoVariant} \M{name} \M{variant}
% \end{declcs}
% Logo \meta{name} is set using \meta{variant} that specifies
% explicitely which variant of the macro is used. If the argument
% is empty, then the default form of the logo is used
% (configurable by \cs{hologoLogoSetup}).
%
% \cs{HologoVariant} is used if the logo is set in a context
% that needs an uppercase first letter (beginning of a sentence, \dots).
%
% \begin{declcs}{hologoList}\\
%   \cs{hologoEntry} \M{logo} \M{variant} \M{since}
% \end{declcs}
% Macro \cs{hologoList} contains all logos that are provided
% by the package including variants. The list consists of calls
% of \cs{hologoEntry} with three arguments starting with the
% logo name \meta{logo} and its variant \meta{variant}. An empty
% variant means the current default. Argument \meta{since} specifies
% with version of the package \xpackage{hologo} is needed to get
% the logo. If the logo is fixed, then the date gets updated.
% Therefore the date \meta{since} is not exactly the date of
% the first introduction, but rather the date of the latest fix.
%
% Before \cs{hologoList} can be used, macro \cs{hologoEntry} needs
% a definition. The example file in section \ref{sec:example}
% shows applications of \cs{hologoList}.
%
% \subsection{Supported contexts}
%
% Macros \cs{hologo} and friends support special contexts:
% \begin{itemize}
% \item \hologo{LaTeX}'s protection mechanism.
% \item Bookmarks of package \xpackage{hyperref}.
% \item Package \xpackage{tex4ht}.
% \item The macros can be used inside \cs{csname} constructs,
%   if \cs{ifincsname} is available (\hologo{pdfTeX}, \hologo{XeTeX},
%   \hologo{LuaTeX}).
% \end{itemize}
%
% \subsection{Example}
% \label{sec:example}
%
% The following example prints the logos in different fonts.
%    \begin{macrocode}
%<*example>
%<<verbatim
\NeedsTeXFormat{LaTeX2e}
\documentclass[a4paper]{article}
\usepackage[
  hmargin=20mm,
  vmargin=20mm,
]{geometry}
\pagestyle{empty}
\usepackage{hologo}[2016/05/12]
\usepackage{longtable}
\usepackage{array}
\setlength{\extrarowheight}{2pt}
\usepackage[T1]{fontenc}
\usepackage{lmodern}
\usepackage{pdflscape}
\usepackage[
  pdfencoding=auto,
]{hyperref}
\hypersetup{
  pdfauthor={Heiko Oberdiek},
  pdftitle={Example for package `hologo'},
  pdfsubject={Logos with fonts lmr, lmss, qtm, qpl, qhv},
}
\usepackage{bookmark}

% Print the logo list on the console

\begingroup
  \typeout{}%
  \typeout{*** Begin of logo list ***}%
  \newcommand*{\hologoEntry}[3]{%
    \typeout{#1 \ifx\\#2\\\else(#2) \fi[#3]}%
  }%
  \hologoList
  \typeout{*** End of logo list ***}%
  \typeout{}%
\endgroup

\begin{document}
\begin{landscape}

  \section{Example file for package `hologo'}

  % Table for font names

  \begin{longtable}{>{\bfseries}ll}
    \textbf{font} & \textbf{Font name}\\
    \hline
    lmr & Latin Modern Roman\\
    lmss & Latin Modern Sans\\
    qtm & \TeX\ Gyre Termes\\
    qhv & \TeX\ Gyre Heros\\
    qpl & \TeX\ Gyre Pagella\\
  \end{longtable}

  % Logo list with logos in different fonts

  \begingroup
    \newcommand*{\SetVariant}[2]{%
      \ifx\\#2\\%
      \else
        \hologoLogoSetup{#1}{variant=#2}%
      \fi
    }%
    \newcommand*{\hologoEntry}[3]{%
      \SetVariant{#1}{#2}%
      \raisebox{1em}[0pt][0pt]{\hypertarget{#1@#2}{}}%
      \bookmark[%
        dest={#1@#2},%
      ]{%
        #1\ifx\\#2\\\else\space(#2)\fi: \Hologo{#1}, \hologo{#1} %
        [Unicode]%
      }%
      \hypersetup{unicode=false}%
      \bookmark[%
        dest={#1@#2},%
      ]{%
        #1\ifx\\#2\\\else\space(#2)\fi: \Hologo{#1}, \hologo{#1} %
        [PDFDocEncoding]%
      }%
      \texttt{#1}%
      &%
      \texttt{#2}%
      &%
      \Hologo{#1}%
      &%
      \SetVariant{#1}{#2}%
      \hologo{#1}%
      &%
      \SetVariant{#1}{#2}%
      \fontfamily{qtm}\selectfont
      \hologo{#1}%
      &%
      \SetVariant{#1}{#2}%
      \fontfamily{qpl}\selectfont
      \hologo{#1}%
      &%
      \SetVariant{#1}{#2}%
      \textsf{\hologo{#1}}%
      &%
      \SetVariant{#1}{#2}%
      \fontfamily{qhv}\selectfont
      \hologo{#1}%
      \tabularnewline
    }%
    \begin{longtable}{llllllll}%
      \textbf{\textit{logo}} & \textbf{\textit{variant}} &
      \texttt{\string\Hologo} &
      \textbf{lmr} & \textbf{qtm} & \textbf{qpl} &
      \textbf{lmss} & \textbf{qhv}
      \tabularnewline
      \hline
      \endhead
      \hologoList
    \end{longtable}%
  \endgroup

\end{landscape}
\end{document}
%verbatim
%</example>
%    \end{macrocode}
%
% \StopEventually{
% }
%
% \section{Implementation}
%    \begin{macrocode}
%<*package>
%    \end{macrocode}
%    Reload check, especially if the package is not used with \LaTeX.
%    \begin{macrocode}
\begingroup\catcode61\catcode48\catcode32=10\relax%
  \catcode13=5 % ^^M
  \endlinechar=13 %
  \catcode35=6 % #
  \catcode39=12 % '
  \catcode44=12 % ,
  \catcode45=12 % -
  \catcode46=12 % .
  \catcode58=12 % :
  \catcode64=11 % @
  \catcode123=1 % {
  \catcode125=2 % }
  \expandafter\let\expandafter\x\csname ver@hologo.sty\endcsname
  \ifx\x\relax % plain-TeX, first loading
  \else
    \def\empty{}%
    \ifx\x\empty % LaTeX, first loading,
      % variable is initialized, but \ProvidesPackage not yet seen
    \else
      \expandafter\ifx\csname PackageInfo\endcsname\relax
        \def\x#1#2{%
          \immediate\write-1{Package #1 Info: #2.}%
        }%
      \else
        \def\x#1#2{\PackageInfo{#1}{#2, stopped}}%
      \fi
      \x{hologo}{The package is already loaded}%
      \aftergroup\endinput
    \fi
  \fi
\endgroup%
%    \end{macrocode}
%    Package identification:
%    \begin{macrocode}
\begingroup\catcode61\catcode48\catcode32=10\relax%
  \catcode13=5 % ^^M
  \endlinechar=13 %
  \catcode35=6 % #
  \catcode39=12 % '
  \catcode40=12 % (
  \catcode41=12 % )
  \catcode44=12 % ,
  \catcode45=12 % -
  \catcode46=12 % .
  \catcode47=12 % /
  \catcode58=12 % :
  \catcode64=11 % @
  \catcode91=12 % [
  \catcode93=12 % ]
  \catcode123=1 % {
  \catcode125=2 % }
  \expandafter\ifx\csname ProvidesPackage\endcsname\relax
    \def\x#1#2#3[#4]{\endgroup
      \immediate\write-1{Package: #3 #4}%
      \xdef#1{#4}%
    }%
  \else
    \def\x#1#2[#3]{\endgroup
      #2[{#3}]%
      \ifx#1\@undefined
        \xdef#1{#3}%
      \fi
      \ifx#1\relax
        \xdef#1{#3}%
      \fi
    }%
  \fi
\expandafter\x\csname ver@hologo.sty\endcsname
\ProvidesPackage{hologo}%
  [2016/05/12 v1.11 A logo collection with bookmark support (HO)]%
%    \end{macrocode}
%
%    \begin{macrocode}
\begingroup\catcode61\catcode48\catcode32=10\relax%
  \catcode13=5 % ^^M
  \endlinechar=13 %
  \catcode123=1 % {
  \catcode125=2 % }
  \catcode64=11 % @
  \def\x{\endgroup
    \expandafter\edef\csname HOLOGO@AtEnd\endcsname{%
      \endlinechar=\the\endlinechar\relax
      \catcode13=\the\catcode13\relax
      \catcode32=\the\catcode32\relax
      \catcode35=\the\catcode35\relax
      \catcode61=\the\catcode61\relax
      \catcode64=\the\catcode64\relax
      \catcode123=\the\catcode123\relax
      \catcode125=\the\catcode125\relax
    }%
  }%
\x\catcode61\catcode48\catcode32=10\relax%
\catcode13=5 % ^^M
\endlinechar=13 %
\catcode35=6 % #
\catcode64=11 % @
\catcode123=1 % {
\catcode125=2 % }
\def\TMP@EnsureCode#1#2{%
  \edef\HOLOGO@AtEnd{%
    \HOLOGO@AtEnd
    \catcode#1=\the\catcode#1\relax
  }%
  \catcode#1=#2\relax
}
\TMP@EnsureCode{10}{12}% ^^J
\TMP@EnsureCode{33}{12}% !
\TMP@EnsureCode{34}{12}% "
\TMP@EnsureCode{36}{3}% $
\TMP@EnsureCode{38}{4}% &
\TMP@EnsureCode{39}{12}% '
\TMP@EnsureCode{40}{12}% (
\TMP@EnsureCode{41}{12}% )
\TMP@EnsureCode{42}{12}% *
\TMP@EnsureCode{43}{12}% +
\TMP@EnsureCode{44}{12}% ,
\TMP@EnsureCode{45}{12}% -
\TMP@EnsureCode{46}{12}% .
\TMP@EnsureCode{47}{12}% /
\TMP@EnsureCode{58}{12}% :
\TMP@EnsureCode{59}{12}% ;
\TMP@EnsureCode{60}{12}% <
\TMP@EnsureCode{62}{12}% >
\TMP@EnsureCode{63}{12}% ?
\TMP@EnsureCode{91}{12}% [
\TMP@EnsureCode{93}{12}% ]
\TMP@EnsureCode{94}{7}% ^ (superscript)
\TMP@EnsureCode{95}{8}% _ (subscript)
\TMP@EnsureCode{96}{12}% `
\TMP@EnsureCode{124}{12}% |
\edef\HOLOGO@AtEnd{%
  \HOLOGO@AtEnd
  \escapechar\the\escapechar\relax
  \noexpand\endinput
}
\escapechar=92 %
%    \end{macrocode}
%
% \subsection{Logo list}
%
%    \begin{macro}{\hologoList}
%    \begin{macrocode}
\def\hologoList{%
  \hologoEntry{(La)TeX}{}{2011/10/01}%
  \hologoEntry{AmSLaTeX}{}{2010/04/16}%
  \hologoEntry{AmSTeX}{}{2010/04/16}%
  \hologoEntry{biber}{}{2011/10/01}%
  \hologoEntry{BibTeX}{}{2011/10/01}%
  \hologoEntry{BibTeX}{sf}{2011/10/01}%
  \hologoEntry{BibTeX}{sc}{2011/10/01}%
  \hologoEntry{BibTeX8}{}{2011/11/22}%
  \hologoEntry{ConTeXt}{}{2011/03/25}%
  \hologoEntry{ConTeXt}{narrow}{2011/03/25}%
  \hologoEntry{ConTeXt}{simple}{2011/03/25}%
  \hologoEntry{emTeX}{}{2010/04/26}%
  \hologoEntry{eTeX}{}{2010/04/08}%
  \hologoEntry{ExTeX}{}{2011/10/01}%
  \hologoEntry{HanTheThanh}{}{2011/11/29}%
  \hologoEntry{iniTeX}{}{2011/10/01}%
  \hologoEntry{KOMAScript}{}{2011/10/01}%
  \hologoEntry{La}{}{2010/05/08}%
  \hologoEntry{LaTeX}{}{2010/04/08}%
  \hologoEntry{LaTeX2e}{}{2010/04/08}%
  \hologoEntry{LaTeX3}{}{2010/04/24}%
  \hologoEntry{LaTeXe}{}{2010/04/08}%
  \hologoEntry{LaTeXML}{}{2011/11/22}%
  \hologoEntry{LaTeXTeX}{}{2011/10/01}%
  \hologoEntry{LuaLaTeX}{}{2010/04/08}%
  \hologoEntry{LuaTeX}{}{2010/04/08}%
  \hologoEntry{LyX}{}{2011/10/01}%
  \hologoEntry{METAFONT}{}{2011/10/01}%
  \hologoEntry{MetaFun}{}{2011/10/01}%
  \hologoEntry{METAPOST}{}{2011/10/01}%
  \hologoEntry{MetaPost}{}{2011/10/01}%
  \hologoEntry{MiKTeX}{}{2011/10/01}%
  \hologoEntry{NTS}{}{2011/10/01}%
  \hologoEntry{OzMF}{}{2011/10/01}%
  \hologoEntry{OzMP}{}{2011/10/01}%
  \hologoEntry{OzTeX}{}{2011/10/01}%
  \hologoEntry{OzTtH}{}{2011/10/01}%
  \hologoEntry{PCTeX}{}{2011/10/01}%
  \hologoEntry{pdfTeX}{}{2011/10/01}%
  \hologoEntry{pdfLaTeX}{}{2011/10/01}%
  \hologoEntry{PiC}{}{2011/10/01}%
  \hologoEntry{PiCTeX}{}{2011/10/01}%
  \hologoEntry{plainTeX}{}{2010/04/08}%
  \hologoEntry{plainTeX}{space}{2010/04/16}%
  \hologoEntry{plainTeX}{hyphen}{2010/04/16}%
  \hologoEntry{plainTeX}{runtogether}{2010/04/16}%
  \hologoEntry{SageTeX}{}{2011/11/22}%
  \hologoEntry{SLiTeX}{}{2011/10/01}%
  \hologoEntry{SLiTeX}{lift}{2011/10/01}%
  \hologoEntry{SLiTeX}{narrow}{2011/10/01}%
  \hologoEntry{SLiTeX}{simple}{2011/10/01}%
  \hologoEntry{SliTeX}{}{2011/10/01}%
  \hologoEntry{SliTeX}{narrow}{2011/10/01}%
  \hologoEntry{SliTeX}{simple}{2011/10/01}%
  \hologoEntry{SliTeX}{lift}{2011/10/01}%
  \hologoEntry{teTeX}{}{2011/10/01}%
  \hologoEntry{TeX}{}{2010/04/08}%
  \hologoEntry{TeX4ht}{}{2011/11/22}%
  \hologoEntry{TTH}{}{2011/11/22}%
  \hologoEntry{virTeX}{}{2011/10/01}%
  \hologoEntry{VTeX}{}{2010/04/24}%
  \hologoEntry{Xe}{}{2010/04/08}%
  \hologoEntry{XeLaTeX}{}{2010/04/08}%
  \hologoEntry{XeTeX}{}{2010/04/08}%
}
%    \end{macrocode}
%    \end{macro}
%
% \subsection{Load resources}
%
%    \begin{macrocode}
\begingroup\expandafter\expandafter\expandafter\endgroup
\expandafter\ifx\csname RequirePackage\endcsname\relax
  \def\TMP@RequirePackage#1[#2]{%
    \begingroup\expandafter\expandafter\expandafter\endgroup
    \expandafter\ifx\csname ver@#1.sty\endcsname\relax
      \input #1.sty\relax
    \fi
  }%
  \TMP@RequirePackage{ltxcmds}[2011/02/04]%
  \TMP@RequirePackage{infwarerr}[2010/04/08]%
  \TMP@RequirePackage{kvsetkeys}[2010/03/01]%
  \TMP@RequirePackage{kvdefinekeys}[2010/03/01]%
  \TMP@RequirePackage{pdftexcmds}[2010/04/01]%
  \TMP@RequirePackage{ifpdf}[2010/01/28]%
  \TMP@RequirePackage{ifluatex}[2010/03/01]%
  \ltx@IfUndefined{newif}{%
    \expandafter\let\csname newif\endcsname\ltx@newif
  }{}%
  \TMP@RequirePackage{ifxetex}[2009/01/23]%
  \TMP@RequirePackage{ifvtex}[2010/03/01]%
\else
  \RequirePackage{ltxcmds}[2011/02/04]%
  \RequirePackage{infwarerr}[2010/04/08]%
  \RequirePackage{kvsetkeys}[2010/03/01]%
  \RequirePackage{kvdefinekeys}[2010/03/01]%
  \RequirePackage{pdftexcmds}[2010/04/01]%
  \RequirePackage{ifpdf}[2010/01/28]%
  \RequirePackage{ifluatex}[2010/03/01]%
  \RequirePackage{ifxetex}[2009/01/23]%
  \RequirePackage{ifvtex}[2010/03/01]%
\fi
%    \end{macrocode}
%
%    \begin{macro}{\HOLOGO@IfDefined}
%    \begin{macrocode}
\def\HOLOGO@IfExists#1{%
  \ifx\@undefined#1%
    \expandafter\ltx@secondoftwo
  \else
    \ifx\relax#1%
      \expandafter\ltx@secondoftwo
    \else
      \expandafter\expandafter\expandafter\ltx@firstoftwo
    \fi
  \fi
}
%    \end{macrocode}
%    \end{macro}
%
% \subsection{Setup macros}
%
%    \begin{macro}{\hologoSetup}
%    \begin{macrocode}
\def\hologoSetup{%
  \let\HOLOGO@name\relax
  \HOLOGO@Setup
}
%    \end{macrocode}
%    \end{macro}
%
%    \begin{macro}{\hologoLogoSetup}
%    \begin{macrocode}
\def\hologoLogoSetup#1{%
  \edef\HOLOGO@name{#1}%
  \ltx@IfUndefined{HoLogo@\HOLOGO@name}{%
    \@PackageError{hologo}{%
      Unknown logo `\HOLOGO@name'%
    }\@ehc
    \ltx@gobble
  }{%
    \HOLOGO@Setup
  }%
}
%    \end{macrocode}
%    \end{macro}
%
%    \begin{macro}{\HOLOGO@Setup}
%    \begin{macrocode}
\def\HOLOGO@Setup{%
  \kvsetkeys{HoLogo}%
}
%    \end{macrocode}
%    \end{macro}
%
% \subsection{Options}
%
%    \begin{macro}{\HOLOGO@DeclareBoolOption}
%    \begin{macrocode}
\def\HOLOGO@DeclareBoolOption#1{%
  \expandafter\chardef\csname HOLOGOOPT@#1\endcsname\ltx@zero
  \kv@define@key{HoLogo}{#1}[true]{%
    \def\HOLOGO@temp{##1}%
    \ifx\HOLOGO@temp\HOLOGO@true
      \ifx\HOLOGO@name\relax
        \expandafter\chardef\csname HOLOGOOPT@#1\endcsname=\ltx@one
      \else
        \expandafter\chardef\csname
        HoLogoOpt@#1@\HOLOGO@name\endcsname\ltx@one
      \fi
      \HOLOGO@SetBreakAll{#1}%
    \else
      \ifx\HOLOGO@temp\HOLOGO@false
        \ifx\HOLOGO@name\relax
          \expandafter\chardef\csname HOLOGOOPT@#1\endcsname=\ltx@zero
        \else
          \expandafter\chardef\csname
          HoLogoOpt@#1@\HOLOGO@name\endcsname=\ltx@zero
        \fi
        \HOLOGO@SetBreakAll{#1}%
      \else
        \@PackageError{hologo}{%
          Unknown value `##1' for boolean option `#1'.\MessageBreak
          Known values are `true' and `false'%
        }\@ehc
      \fi
    \fi
  }%
}
%    \end{macrocode}
%    \end{macro}
%
%    \begin{macro}{\HOLOGO@SetBreakAll}
%    \begin{macrocode}
\def\HOLOGO@SetBreakAll#1{%
  \def\HOLOGO@temp{#1}%
  \ifx\HOLOGO@temp\HOLOGO@break
    \ifx\HOLOGO@name\relax
      \chardef\HOLOGOOPT@hyphenbreak=\HOLOGOOPT@break
      \chardef\HOLOGOOPT@spacebreak=\HOLOGOOPT@break
      \chardef\HOLOGOOPT@discretionarybreak=\HOLOGOOPT@break
    \else
      \expandafter\chardef
         \csname HoLogoOpt@hyphenbreak@\HOLOGO@name\endcsname=%
         \csname HoLogoOpt@break@\HOLOGO@name\endcsname
      \expandafter\chardef
         \csname HoLogoOpt@spacebreak@\HOLOGO@name\endcsname=%
         \csname HoLogoOpt@break@\HOLOGO@name\endcsname
      \expandafter\chardef
         \csname HoLogoOpt@discretionarybreak@\HOLOGO@name
             \endcsname=%
         \csname HoLogoOpt@break@\HOLOGO@name\endcsname
    \fi
  \fi
}
%    \end{macrocode}
%    \end{macro}
%
%    \begin{macro}{\HOLOGO@true}
%    \begin{macrocode}
\def\HOLOGO@true{true}
%    \end{macrocode}
%    \end{macro}
%    \begin{macro}{\HOLOGO@false}
%    \begin{macrocode}
\def\HOLOGO@false{false}
%    \end{macrocode}
%    \end{macro}
%    \begin{macro}{\HOLOGO@break}
%    \begin{macrocode}
\def\HOLOGO@break{break}
%    \end{macrocode}
%    \end{macro}
%
%    \begin{macrocode}
\HOLOGO@DeclareBoolOption{break}
\HOLOGO@DeclareBoolOption{hyphenbreak}
\HOLOGO@DeclareBoolOption{spacebreak}
\HOLOGO@DeclareBoolOption{discretionarybreak}
%    \end{macrocode}
%
%    \begin{macrocode}
\kv@define@key{HoLogo}{variant}{%
  \ifx\HOLOGO@name\relax
    \@PackageError{hologo}{%
      Option `variant' is not available in \string\hologoSetup,%
      \MessageBreak
      Use \string\hologoLogoSetup\space instead%
    }\@ehc
  \else
    \edef\HOLOGO@temp{#1}%
    \ifx\HOLOGO@temp\ltx@empty
      \expandafter
      \let\csname HoLogoOpt@variant@\HOLOGO@name\endcsname\@undefined
    \else
      \ltx@IfUndefined{HoLogo@\HOLOGO@name @\HOLOGO@temp}{%
        \@PackageError{hologo}{%
          Unknown variant `\HOLOGO@temp' of logo `\HOLOGO@name'%
        }\@ehc
      }{%
        \expandafter
        \let\csname HoLogoOpt@variant@\HOLOGO@name\endcsname
            \HOLOGO@temp
      }%
    \fi
  \fi
}
%    \end{macrocode}
%
%    \begin{macro}{\HOLOGO@Variant}
%    \begin{macrocode}
\def\HOLOGO@Variant#1{%
  #1%
  \ltx@ifundefined{HoLogoOpt@variant@#1}{%
  }{%
    @\csname HoLogoOpt@variant@#1\endcsname
  }%
}
%    \end{macrocode}
%    \end{macro}
%
% \subsection{Break/no-break support}
%
%    \begin{macro}{\HOLOGO@space}
%    \begin{macrocode}
\def\HOLOGO@space{%
  \ltx@ifundefined{HoLogoOpt@spacebreak@\HOLOGO@name}{%
    \ltx@ifundefined{HoLogoOpt@break@\HOLOGO@name}{%
      \chardef\HOLOGO@temp=\HOLOGOOPT@spacebreak
    }{%
      \chardef\HOLOGO@temp=%
        \csname HoLogoOpt@break@\HOLOGO@name\endcsname
    }%
  }{%
    \chardef\HOLOGO@temp=%
      \csname HoLogoOpt@spacebreak@\HOLOGO@name\endcsname
  }%
  \ifcase\HOLOGO@temp
    \penalty10000 %
  \fi
  \ltx@space
}
%    \end{macrocode}
%    \end{macro}
%
%    \begin{macro}{\HOLOGO@hyphen}
%    \begin{macrocode}
\def\HOLOGO@hyphen{%
  \ltx@ifundefined{HoLogoOpt@hyphenbreak@\HOLOGO@name}{%
    \ltx@ifundefined{HoLogoOpt@break@\HOLOGO@name}{%
      \chardef\HOLOGO@temp=\HOLOGOOPT@hyphenbreak
    }{%
      \chardef\HOLOGO@temp=%
        \csname HoLogoOpt@break@\HOLOGO@name\endcsname
    }%
  }{%
    \chardef\HOLOGO@temp=%
      \csname HoLogoOpt@hyphenbreak@\HOLOGO@name\endcsname
  }%
  \ifcase\HOLOGO@temp
    \ltx@mbox{-}%
  \else
    -%
  \fi
}
%    \end{macrocode}
%    \end{macro}
%
%    \begin{macro}{\HOLOGO@discretionary}
%    \begin{macrocode}
\def\HOLOGO@discretionary{%
  \ltx@ifundefined{HoLogoOpt@discretionarybreak@\HOLOGO@name}{%
    \ltx@ifundefined{HoLogoOpt@break@\HOLOGO@name}{%
      \chardef\HOLOGO@temp=\HOLOGOOPT@discretionarybreak
    }{%
      \chardef\HOLOGO@temp=%
        \csname HoLogoOpt@break@\HOLOGO@name\endcsname
    }%
  }{%
    \chardef\HOLOGO@temp=%
      \csname HoLogoOpt@discretionarybreak@\HOLOGO@name\endcsname
  }%
  \ifcase\HOLOGO@temp
  \else
    \-%
  \fi
}
%    \end{macrocode}
%    \end{macro}
%
%    \begin{macro}{\HOLOGO@mbox}
%    \begin{macrocode}
\def\HOLOGO@mbox#1{%
  \ltx@ifundefined{HoLogoOpt@break@\HOLOGO@name}{%
    \chardef\HOLOGO@temp=\HOLOGOOPT@hyphenbreak
  }{%
    \chardef\HOLOGO@temp=%
      \csname HoLogoOpt@break@\HOLOGO@name\endcsname
  }%
  \ifcase\HOLOGO@temp
    \ltx@mbox{#1}%
  \else
    #1%
  \fi
}
%    \end{macrocode}
%    \end{macro}
%
% \subsection{Font support}
%
%    \begin{macro}{\HoLogoFont@font}
%    \begin{tabular}{@{}ll@{}}
%    |#1|:& logo name\\
%    |#2|:& font short name\\
%    |#3|:& text
%    \end{tabular}
%    \begin{macrocode}
\def\HoLogoFont@font#1#2#3{%
  \begingroup
    \ltx@IfUndefined{HoLogoFont@logo@#1.#2}{%
      \ltx@IfUndefined{HoLogoFont@font@#2}{%
        \@PackageWarning{hologo}{%
          Missing font `#2' for logo `#1'%
        }%
        #3%
      }{%
        \csname HoLogoFont@font@#2\endcsname{#3}%
      }%
    }{%
      \csname HoLogoFont@logo@#1.#2\endcsname{#3}%
    }%
  \endgroup
}
%    \end{macrocode}
%    \end{macro}
%
%    \begin{macro}{\HoLogoFont@Def}
%    \begin{macrocode}
\def\HoLogoFont@Def#1{%
  \expandafter\def\csname HoLogoFont@font@#1\endcsname
}
%    \end{macrocode}
%    \end{macro}
%    \begin{macro}{\HoLogoFont@LogoDef}
%    \begin{macrocode}
\def\HoLogoFont@LogoDef#1#2{%
  \expandafter\def\csname HoLogoFont@logo@#1.#2\endcsname
}
%    \end{macrocode}
%    \end{macro}
%
% \subsubsection{Font defaults}
%
%    \begin{macro}{\HoLogoFont@font@general}
%    \begin{macrocode}
\HoLogoFont@Def{general}{}%
%    \end{macrocode}
%    \end{macro}
%
%    \begin{macro}{\HoLogoFont@font@rm}
%    \begin{macrocode}
\ltx@IfUndefined{rmfamily}{%
  \ltx@IfUndefined{rm}{%
  }{%
    \HoLogoFont@Def{rm}{\rm}%
  }%
}{%
  \HoLogoFont@Def{rm}{\rmfamily}%
}
%    \end{macrocode}
%    \end{macro}
%
%    \begin{macro}{\HoLogoFont@font@sf}
%    \begin{macrocode}
\ltx@IfUndefined{sffamily}{%
  \ltx@IfUndefined{sf}{%
  }{%
    \HoLogoFont@Def{sf}{\sf}%
  }%
}{%
  \HoLogoFont@Def{sf}{\sffamily}%
}
%    \end{macrocode}
%    \end{macro}
%
%    \begin{macro}{\HoLogoFont@font@bibsf}
%    In case of \hologo{plainTeX} the original small caps
%    variant is used as default. In \hologo{LaTeX}
%    the definition of package \xpackage{dtklogos} \cite{dtklogos}
%    is used.
%\begin{quote}
%\begin{verbatim}
%\DeclareRobustCommand{\BibTeX}{%
%  B%
%  \kern-.05em%
%  \hbox{%
%    $\m@th$% %% force math size calculations
%    \csname S@\f@size\endcsname
%    \fontsize\sf@size\z@
%    \math@fontsfalse
%    \selectfont
%    I%
%    \kern-.025em%
%    B
%  }%
%  \kern-.08em%
%  \-%
%  \TeX
%}
%\end{verbatim}
%\end{quote}
%    \begin{macrocode}
\ltx@IfUndefined{selectfont}{%
  \ltx@IfUndefined{tensc}{%
    \font\tensc=cmcsc10\relax
  }{}%
  \HoLogoFont@Def{bibsf}{\tensc}%
}{%
  \HoLogoFont@Def{bibsf}{%
    $\mathsurround=0pt$%
    \csname S@\f@size\endcsname
    \fontsize\sf@size{0pt}%
    \math@fontsfalse
    \selectfont
  }%
}
%    \end{macrocode}
%    \end{macro}
%
%    \begin{macro}{\HoLogoFont@font@sc}
%    \begin{macrocode}
\ltx@IfUndefined{scshape}{%
  \ltx@IfUndefined{tensc}{%
    \font\tensc=cmcsc10\relax
  }{}%
  \HoLogoFont@Def{sc}{\tensc}%
}{%
  \HoLogoFont@Def{sc}{\scshape}%
}
%    \end{macrocode}
%    \end{macro}
%
%    \begin{macro}{\HoLogoFont@font@sy}
%    \begin{macrocode}
\ltx@IfUndefined{usefont}{%
  \ltx@IfUndefined{tensy}{%
  }{%
    \HoLogoFont@Def{sy}{\tensy}%
  }%
}{%
  \HoLogoFont@Def{sy}{%
    \usefont{OMS}{cmsy}{m}{n}%
  }%
}
%    \end{macrocode}
%    \end{macro}
%
%    \begin{macro}{\HoLogoFont@font@logo}
%    \begin{macrocode}
\begingroup
  \def\x{LaTeX2e}%
\expandafter\endgroup
\ifx\fmtname\x
  \ltx@IfUndefined{logofamily}{%
    \DeclareRobustCommand\logofamily{%
      \not@math@alphabet\logofamily\relax
      \fontencoding{U}%
      \fontfamily{logo}%
      \selectfont
    }%
  }{}%
  \ltx@IfUndefined{logofamily}{%
  }{%
    \HoLogoFont@Def{logo}{\logofamily}%
  }%
\else
  \ltx@IfUndefined{tenlogo}{%
    \font\tenlogo=logo10\relax
  }{}%
  \HoLogoFont@Def{logo}{\tenlogo}%
\fi
%    \end{macrocode}
%    \end{macro}
%
% \subsubsection{Font setup}
%
%    \begin{macro}{\hologoFontSetup}
%    \begin{macrocode}
\def\hologoFontSetup{%
  \let\HOLOGO@name\relax
  \HOLOGO@FontSetup
}
%    \end{macrocode}
%    \end{macro}
%
%    \begin{macro}{\hologoLogoFontSetup}
%    \begin{macrocode}
\def\hologoLogoFontSetup#1{%
  \edef\HOLOGO@name{#1}%
  \ltx@IfUndefined{HoLogo@\HOLOGO@name}{%
    \@PackageError{hologo}{%
      Unknown logo `\HOLOGO@name'%
    }\@ehc
    \ltx@gobble
  }{%
    \HOLOGO@FontSetup
  }%
}
%    \end{macrocode}
%    \end{macro}
%
%    \begin{macro}{\HOLOGO@FontSetup}
%    \begin{macrocode}
\def\HOLOGO@FontSetup{%
  \kvsetkeys{HoLogoFont}%
}
%    \end{macrocode}
%    \end{macro}
%
%    \begin{macrocode}
\def\HOLOGO@temp#1{%
  \kv@define@key{HoLogoFont}{#1}{%
    \ifx\HOLOGO@name\relax
      \HoLogoFont@Def{#1}{##1}%
    \else
      \HoLogoFont@LogoDef\HOLOGO@name{#1}{##1}%
    \fi
  }%
}
\HOLOGO@temp{general}
\HOLOGO@temp{sf}
%    \end{macrocode}
%
% \subsection{Generic logo commands}
%
%    \begin{macrocode}
\HOLOGO@IfExists\hologo{%
  \@PackageError{hologo}{%
    \string\hologo\ltx@space is already defined.\MessageBreak
    Package loading is aborted%
  }\@ehc
  \HOLOGO@AtEnd
}%
\HOLOGO@IfExists\hologoRobust{%
  \@PackageError{hologo}{%
    \string\hologoRobust\ltx@space is already defined.\MessageBreak
    Package loading is aborted%
  }\@ehc
  \HOLOGO@AtEnd
}%
%    \end{macrocode}
%
% \subsubsection{\cs{hologo} and friends}
%
%    \begin{macrocode}
\ifluatex
  \expandafter\ltx@firstofone
\else
  \expandafter\ltx@gobble
\fi
{%
  \ltx@IfUndefined{ifincsname}{%
    \ifnum\luatexversion<36 %
      \expandafter\ltx@gobble
    \else
      \expandafter\ltx@firstofone
    \fi
    {%
      \begingroup
        \ifcase0%
            \directlua{%
              if tex.enableprimitives then %
                tex.enableprimitives('HOLOGO@', {'ifincsname'})%
              else %
                tex.print('1')%
              end%
            }%
            \ifx\HOLOGO@ifincsname\@undefined 1\fi%
            \relax
          \expandafter\ltx@firstofone
        \else
          \endgroup
          \expandafter\ltx@gobble
        \fi
        {%
          \global\let\ifincsname\HOLOGO@ifincsname
        }%
      \HOLOGO@temp
    }%
  }{}%
}
%    \end{macrocode}
%    \begin{macrocode}
\ltx@IfUndefined{ifincsname}{%
  \catcode`$=14 %
}{%
  \catcode`$=9 %
}
%    \end{macrocode}
%
%    \begin{macro}{\hologo}
%    \begin{macrocode}
\def\hologo#1{%
$ \ifincsname
$   \ltx@ifundefined{HoLogoCs@\HOLOGO@Variant{#1}}{%
$     #1%
$   }{%
$     \csname HoLogoCs@\HOLOGO@Variant{#1}\endcsname\ltx@firstoftwo
$   }%
$ \else
    \HOLOGO@IfExists\texorpdfstring\texorpdfstring\ltx@firstoftwo
    {%
      \hologoRobust{#1}%
    }{%
      \ltx@ifundefined{HoLogoBkm@\HOLOGO@Variant{#1}}{%
        \ltx@ifundefined{HoLogo@#1}{?#1?}{#1}%
      }{%
        \csname HoLogoBkm@\HOLOGO@Variant{#1}\endcsname
        \ltx@firstoftwo
      }%
    }%
$ \fi
}
%    \end{macrocode}
%    \end{macro}
%    \begin{macro}{\Hologo}
%    \begin{macrocode}
\def\Hologo#1{%
$ \ifincsname
$   \ltx@ifundefined{HoLogoCs@\HOLOGO@Variant{#1}}{%
$     #1%
$   }{%
$     \csname HoLogoCs@\HOLOGO@Variant{#1}\endcsname\ltx@secondoftwo
$   }%
$ \else
    \HOLOGO@IfExists\texorpdfstring\texorpdfstring\ltx@firstoftwo
    {%
      \HologoRobust{#1}%
    }{%
      \ltx@ifundefined{HoLogoBkm@\HOLOGO@Variant{#1}}{%
        \ltx@ifundefined{HoLogo@#1}{?#1?}{#1}%
      }{%
        \csname HoLogoBkm@\HOLOGO@Variant{#1}\endcsname
        \ltx@secondoftwo
      }%
    }%
$ \fi
}
%    \end{macrocode}
%    \end{macro}
%
%    \begin{macro}{\hologoVariant}
%    \begin{macrocode}
\def\hologoVariant#1#2{%
  \ifx\relax#2\relax
    \hologo{#1}%
  \else
$   \ifincsname
$     \ltx@ifundefined{HoLogoCs@#1@#2}{%
$       #1%
$     }{%
$       \csname HoLogoCs@#1@#2\endcsname\ltx@firstoftwo
$     }%
$   \else
      \HOLOGO@IfExists\texorpdfstring\texorpdfstring\ltx@firstoftwo
      {%
        \hologoVariantRobust{#1}{#2}%
      }{%
        \ltx@ifundefined{HoLogoBkm@#1@#2}{%
          \ltx@ifundefined{HoLogo@#1}{?#1?}{#1}%
        }{%
          \csname HoLogoBkm@#1@#2\endcsname
          \ltx@firstoftwo
        }%
      }%
$   \fi
  \fi
}
%    \end{macrocode}
%    \end{macro}
%    \begin{macro}{\HologoVariant}
%    \begin{macrocode}
\def\HologoVariant#1#2{%
  \ifx\relax#2\relax
    \Hologo{#1}%
  \else
$   \ifincsname
$     \ltx@ifundefined{HoLogoCs@#1@#2}{%
$       #1%
$     }{%
$       \csname HoLogoCs@#1@#2\endcsname\ltx@secondoftwo
$     }%
$   \else
      \HOLOGO@IfExists\texorpdfstring\texorpdfstring\ltx@firstoftwo
      {%
        \HologoVariantRobust{#1}{#2}%
      }{%
        \ltx@ifundefined{HoLogoBkm@#1@#2}{%
          \ltx@ifundefined{HoLogo@#1}{?#1?}{#1}%
        }{%
          \csname HoLogoBkm@#1@#2\endcsname
          \ltx@secondoftwo
        }%
      }%
$   \fi
  \fi
}
%    \end{macrocode}
%    \end{macro}
%
%    \begin{macrocode}
\catcode`\$=3 %
%    \end{macrocode}
%
% \subsubsection{\cs{hologoRobust} and friends}
%
%    \begin{macro}{\hologoRobust}
%    \begin{macrocode}
\ltx@IfUndefined{protected}{%
  \ltx@IfUndefined{DeclareRobustCommand}{%
    \def\hologoRobust#1%
  }{%
    \DeclareRobustCommand*\hologoRobust[1]%
  }%
}{%
  \protected\def\hologoRobust#1%
}%
{%
  \edef\HOLOGO@name{#1}%
  \ltx@IfUndefined{HoLogo@\HOLOGO@Variant\HOLOGO@name}{%
    \@PackageError{hologo}{%
      Unknown logo `\HOLOGO@name'%
    }\@ehc
    ?\HOLOGO@name?%
  }{%
    \ltx@IfUndefined{ver@tex4ht.sty}{%
      \HoLogoFont@font\HOLOGO@name{general}{%
        \csname HoLogo@\HOLOGO@Variant\HOLOGO@name\endcsname
        \ltx@firstoftwo
      }%
    }{%
      \ltx@IfUndefined{HoLogoHtml@\HOLOGO@Variant\HOLOGO@name}{%
        \HOLOGO@name
      }{%
        \csname HoLogoHtml@\HOLOGO@Variant\HOLOGO@name\endcsname
        \ltx@firstoftwo
      }%
    }%
  }%
}
%    \end{macrocode}
%    \end{macro}
%    \begin{macro}{\HologoRobust}
%    \begin{macrocode}
\ltx@IfUndefined{protected}{%
  \ltx@IfUndefined{DeclareRobustCommand}{%
    \def\HologoRobust#1%
  }{%
    \DeclareRobustCommand*\HologoRobust[1]%
  }%
}{%
  \protected\def\HologoRobust#1%
}%
{%
  \edef\HOLOGO@name{#1}%
  \ltx@IfUndefined{HoLogo@\HOLOGO@Variant\HOLOGO@name}{%
    \@PackageError{hologo}{%
      Unknown logo `\HOLOGO@name'%
    }\@ehc
    ?\HOLOGO@name?%
  }{%
    \ltx@IfUndefined{ver@tex4ht.sty}{%
      \HoLogoFont@font\HOLOGO@name{general}{%
        \csname HoLogo@\HOLOGO@Variant\HOLOGO@name\endcsname
        \ltx@secondoftwo
      }%
    }{%
      \ltx@IfUndefined{HoLogoHtml@\HOLOGO@Variant\HOLOGO@name}{%
        \expandafter\HOLOGO@Uppercase\HOLOGO@name
      }{%
        \csname HoLogoHtml@\HOLOGO@Variant\HOLOGO@name\endcsname
        \ltx@secondoftwo
      }%
    }%
  }%
}
%    \end{macrocode}
%    \end{macro}
%    \begin{macro}{\hologoVariantRobust}
%    \begin{macrocode}
\ltx@IfUndefined{protected}{%
  \ltx@IfUndefined{DeclareRobustCommand}{%
    \def\hologoVariantRobust#1#2%
  }{%
    \DeclareRobustCommand*\hologoVariantRobust[2]%
  }%
}{%
  \protected\def\hologoVariantRobust#1#2%
}%
{%
  \begingroup
    \hologoLogoSetup{#1}{variant={#2}}%
    \hologoRobust{#1}%
  \endgroup
}
%    \end{macrocode}
%    \end{macro}
%    \begin{macro}{\HologoVariantRobust}
%    \begin{macrocode}
\ltx@IfUndefined{protected}{%
  \ltx@IfUndefined{DeclareRobustCommand}{%
    \def\HologoVariantRobust#1#2%
  }{%
    \DeclareRobustCommand*\HologoVariantRobust[2]%
  }%
}{%
  \protected\def\HologoVariantRobust#1#2%
}%
{%
  \begingroup
    \hologoLogoSetup{#1}{variant={#2}}%
    \HologoRobust{#1}%
  \endgroup
}
%    \end{macrocode}
%    \end{macro}
%
%    \begin{macro}{\hologorobust}
%    Macro \cs{hologorobust} is only defined for compatibility.
%    Its use is deprecated.
%    \begin{macrocode}
\def\hologorobust{\hologoRobust}
%    \end{macrocode}
%    \end{macro}
%
% \subsection{Helpers}
%
%    \begin{macro}{\HOLOGO@Uppercase}
%    Macro \cs{HOLOGO@Uppercase} is restricted to \cs{uppercase},
%    because \hologo{plainTeX} or \hologo{iniTeX} do not provide
%    \cs{MakeUppercase}.
%    \begin{macrocode}
\def\HOLOGO@Uppercase#1{\uppercase{#1}}
%    \end{macrocode}
%    \end{macro}
%
%    \begin{macro}{\HOLOGO@PdfdocUnicode}
%    \begin{macrocode}
\def\HOLOGO@PdfdocUnicode{%
  \ifx\ifHy@unicode\iftrue
    \expandafter\ltx@secondoftwo
  \else
    \expandafter\ltx@firstoftwo
  \fi
}
%    \end{macrocode}
%    \end{macro}
%
%    \begin{macro}{\HOLOGO@Math}
%    \begin{macrocode}
\def\HOLOGO@MathSetup{%
  \mathsurround0pt\relax
  \HOLOGO@IfExists\f@series{%
    \if b\expandafter\ltx@car\f@series x\@nil
      \csname boldmath\endcsname
   \fi
  }{}%
}
%    \end{macrocode}
%    \end{macro}
%
%    \begin{macro}{\HOLOGO@TempDimen}
%    \begin{macrocode}
\dimendef\HOLOGO@TempDimen=\ltx@zero
%    \end{macrocode}
%    \end{macro}
%    \begin{macro}{\HOLOGO@NegativeKerning}
%    \begin{macrocode}
\def\HOLOGO@NegativeKerning#1{%
  \begingroup
    \HOLOGO@TempDimen=0pt\relax
    \comma@parse@normalized{#1}{%
      \ifdim\HOLOGO@TempDimen=0pt %
        \expandafter\HOLOGO@@NegativeKerning\comma@entry
      \fi
      \ltx@gobble
    }%
    \ifdim\HOLOGO@TempDimen<0pt %
      \kern\HOLOGO@TempDimen
    \fi
  \endgroup
}
%    \end{macrocode}
%    \end{macro}
%    \begin{macro}{\HOLOGO@@NegativeKerning}
%    \begin{macrocode}
\def\HOLOGO@@NegativeKerning#1#2{%
  \setbox\ltx@zero\hbox{#1#2}%
  \HOLOGO@TempDimen=\wd\ltx@zero
  \setbox\ltx@zero\hbox{#1\kern0pt#2}%
  \advance\HOLOGO@TempDimen by -\wd\ltx@zero
}
%    \end{macrocode}
%    \end{macro}
%
%    \begin{macro}{\HOLOGO@SpaceFactor}
%    \begin{macrocode}
\def\HOLOGO@SpaceFactor{%
  \spacefactor1000 %
}
%    \end{macrocode}
%    \end{macro}
%
%    \begin{macro}{\HOLOGO@Span}
%    \begin{macrocode}
\def\HOLOGO@Span#1#2{%
  \HCode{<span class="HoLogo-#1">}%
  #2%
  \HCode{</span>}%
}
%    \end{macrocode}
%    \end{macro}
%
% \subsubsection{Text subscript}
%
%    \begin{macro}{\HOLOGO@SubScript}%
%    \begin{macrocode}
\def\HOLOGO@SubScript#1{%
  \ltx@IfUndefined{textsubscript}{%
    \ltx@IfUndefined{text}{%
      \ltx@mbox{%
        \mathsurround=0pt\relax
        $%
          _{%
            \ltx@IfUndefined{sf@size}{%
              \mathrm{#1}%
            }{%
              \mbox{%
                \fontsize\sf@size{0pt}\selectfont
                #1%
              }%
            }%
          }%
        $%
      }%
    }{%
      \ltx@mbox{%
        \mathsurround=0pt\relax
        $_{\text{#1}}$%
      }%
    }%
  }{%
    \textsubscript{#1}%
  }%
}
%    \end{macrocode}
%    \end{macro}
%
% \subsection{\hologo{TeX} and friends}
%
% \subsubsection{\hologo{TeX}}
%
%    \begin{macro}{\HoLogo@TeX}
%    Source: \hologo{LaTeX} kernel.
%    \begin{macrocode}
\def\HoLogo@TeX#1{%
  T\kern-.1667em\lower.5ex\hbox{E}\kern-.125emX\HOLOGO@SpaceFactor
}
%    \end{macrocode}
%    \end{macro}
%    \begin{macro}{\HoLogoHtml@TeX}
%    \begin{macrocode}
\def\HoLogoHtml@TeX#1{%
  \HoLogoCss@TeX
  \HOLOGO@Span{TeX}{%
    T%
    \HOLOGO@Span{e}{%
      E%
    }%
    X%
  }%
}
%    \end{macrocode}
%    \end{macro}
%    \begin{macro}{\HoLogoCss@TeX}
%    \begin{macrocode}
\def\HoLogoCss@TeX{%
  \Css{%
    span.HoLogo-TeX span.HoLogo-e{%
      position:relative;%
      top:.5ex;%
      margin-left:-.1667em;%
      margin-right:-.125em;%
    }%
  }%
  \Css{%
    a span.HoLogo-TeX span.HoLogo-e{%
      text-decoration:none;%
    }%
  }%
  \global\let\HoLogoCss@TeX\relax
}
%    \end{macrocode}
%    \end{macro}
%
% \subsubsection{\hologo{plainTeX}}
%
%    \begin{macro}{\HoLogo@plainTeX@space}
%    Source: ``The \hologo{TeX}book''
%    \begin{macrocode}
\def\HoLogo@plainTeX@space#1{%
  \HOLOGO@mbox{#1{p}{P}lain}\HOLOGO@space\hologo{TeX}%
}
%    \end{macrocode}
%    \end{macro}
%    \begin{macro}{\HoLogoCs@plainTeX@space}
%    \begin{macrocode}
\def\HoLogoCs@plainTeX@space#1{#1{p}{P}lain TeX}%
%    \end{macrocode}
%    \end{macro}
%    \begin{macro}{\HoLogoBkm@plainTeX@space}
%    \begin{macrocode}
\def\HoLogoBkm@plainTeX@space#1{%
  #1{p}{P}lain \hologo{TeX}%
}
%    \end{macrocode}
%    \end{macro}
%    \begin{macro}{\HoLogoHtml@plainTeX@space}
%    \begin{macrocode}
\def\HoLogoHtml@plainTeX@space#1{%
  #1{p}{P}lain \hologo{TeX}%
}
%    \end{macrocode}
%    \end{macro}
%
%    \begin{macro}{\HoLogo@plainTeX@hyphen}
%    \begin{macrocode}
\def\HoLogo@plainTeX@hyphen#1{%
  \HOLOGO@mbox{#1{p}{P}lain}\HOLOGO@hyphen\hologo{TeX}%
}
%    \end{macrocode}
%    \end{macro}
%    \begin{macro}{\HoLogoCs@plainTeX@hyphen}
%    \begin{macrocode}
\def\HoLogoCs@plainTeX@hyphen#1{#1{p}{P}lain-TeX}
%    \end{macrocode}
%    \end{macro}
%    \begin{macro}{\HoLogoBkm@plainTeX@hyphen}
%    \begin{macrocode}
\def\HoLogoBkm@plainTeX@hyphen#1{%
  #1{p}{P}lain-\hologo{TeX}%
}
%    \end{macrocode}
%    \end{macro}
%    \begin{macro}{\HoLogoHtml@plainTeX@hyphen}
%    \begin{macrocode}
\def\HoLogoHtml@plainTeX@hyphen#1{%
  #1{p}{P}lain-\hologo{TeX}%
}
%    \end{macrocode}
%    \end{macro}
%
%    \begin{macro}{\HoLogo@plainTeX@runtogether}
%    \begin{macrocode}
\def\HoLogo@plainTeX@runtogether#1{%
  \HOLOGO@mbox{#1{p}{P}lain\hologo{TeX}}%
}
%    \end{macrocode}
%    \end{macro}
%    \begin{macro}{\HoLogoCs@plainTeX@runtogether}
%    \begin{macrocode}
\def\HoLogoCs@plainTeX@runtogether#1{#1{p}{P}lainTeX}
%    \end{macrocode}
%    \end{macro}
%    \begin{macro}{\HoLogoBkm@plainTeX@runtogether}
%    \begin{macrocode}
\def\HoLogoBkm@plainTeX@runtogether#1{%
  #1{p}{P}lain\hologo{TeX}%
}
%    \end{macrocode}
%    \end{macro}
%    \begin{macro}{\HoLogoHtml@plainTeX@runtogether}
%    \begin{macrocode}
\def\HoLogoHtml@plainTeX@runtogether#1{%
  #1{p}{P}lain\hologo{TeX}%
}
%    \end{macrocode}
%    \end{macro}
%
%    \begin{macro}{\HoLogo@plainTeX}
%    \begin{macrocode}
\def\HoLogo@plainTeX{\HoLogo@plainTeX@space}
%    \end{macrocode}
%    \end{macro}
%    \begin{macro}{\HoLogoCs@plainTeX}
%    \begin{macrocode}
\def\HoLogoCs@plainTeX{\HoLogoCs@plainTeX@space}
%    \end{macrocode}
%    \end{macro}
%    \begin{macro}{\HoLogoBkm@plainTeX}
%    \begin{macrocode}
\def\HoLogoBkm@plainTeX{\HoLogoBkm@plainTeX@space}
%    \end{macrocode}
%    \end{macro}
%    \begin{macro}{\HoLogoHtml@plainTeX}
%    \begin{macrocode}
\def\HoLogoHtml@plainTeX{\HoLogoHtml@plainTeX@space}
%    \end{macrocode}
%    \end{macro}
%
% \subsubsection{\hologo{LaTeX}}
%
%    Source: \hologo{LaTeX} kernel.
%\begin{quote}
%\begin{verbatim}
%\DeclareRobustCommand{\LaTeX}{%
%  L%
%  \kern-.36em%
%  {%
%    \sbox\z@ T%
%    \vbox to\ht\z@{%
%      \hbox{%
%        \check@mathfonts
%        \fontsize\sf@size\z@
%        \math@fontsfalse
%        \selectfont
%        A%
%      }%
%      \vss
%    }%
%  }%
%  \kern-.15em%
%  \TeX
%}
%\end{verbatim}
%\end{quote}
%
%    \begin{macro}{\HoLogo@La}
%    \begin{macrocode}
\def\HoLogo@La#1{%
  L%
  \kern-.36em%
  \begingroup
    \setbox\ltx@zero\hbox{T}%
    \vbox to\ht\ltx@zero{%
      \hbox{%
        \ltx@ifundefined{check@mathfonts}{%
          \csname sevenrm\endcsname
        }{%
          \check@mathfonts
          \fontsize\sf@size{0pt}%
          \math@fontsfalse\selectfont
        }%
        A%
      }%
      \vss
    }%
  \endgroup
}
%    \end{macrocode}
%    \end{macro}
%
%    \begin{macro}{\HoLogo@LaTeX}
%    Source: \hologo{LaTeX} kernel.
%    \begin{macrocode}
\def\HoLogo@LaTeX#1{%
  \hologo{La}%
  \kern-.15em%
  \hologo{TeX}%
}
%    \end{macrocode}
%    \end{macro}
%    \begin{macro}{\HoLogoHtml@LaTeX}
%    \begin{macrocode}
\def\HoLogoHtml@LaTeX#1{%
  \HoLogoCss@LaTeX
  \HOLOGO@Span{LaTeX}{%
    L%
    \HOLOGO@Span{a}{%
      A%
    }%
    \hologo{TeX}%
  }%
}
%    \end{macrocode}
%    \end{macro}
%    \begin{macro}{\HoLogoCss@LaTeX}
%    \begin{macrocode}
\def\HoLogoCss@LaTeX{%
  \Css{%
    span.HoLogo-LaTeX span.HoLogo-a{%
      position:relative;%
      top:-.5ex;%
      margin-left:-.36em;%
      margin-right:-.15em;%
      font-size:85\%;%
    }%
  }%
  \global\let\HoLogoCss@LaTeX\relax
}
%    \end{macrocode}
%    \end{macro}
%
% \subsubsection{\hologo{(La)TeX}}
%
%    \begin{macro}{\HoLogo@LaTeXTeX}
%    The kerning around the parentheses is taken
%    from package \xpackage{dtklogos} \cite{dtklogos}.
%\begin{quote}
%\begin{verbatim}
%\DeclareRobustCommand{\LaTeXTeX}{%
%  (%
%  \kern-.15em%
%  L%
%  \kern-.36em%
%  {%
%    \sbox\z@ T%
%    \vbox to\ht0{%
%      \hbox{%
%        $\m@th$%
%        \csname S@\f@size\endcsname
%        \fontsize\sf@size\z@
%        \math@fontsfalse
%        \selectfont
%        A%
%      }%
%      \vss
%    }%
%  }%
%  \kern-.2em%
%  )%
%  \kern-.15em%
%  \TeX
%}
%\end{verbatim}
%\end{quote}
%    \begin{macrocode}
\def\HoLogo@LaTeXTeX#1{%
  (%
  \kern-.15em%
  \hologo{La}%
  \kern-.2em%
  )%
  \kern-.15em%
  \hologo{TeX}%
}
%    \end{macrocode}
%    \end{macro}
%    \begin{macro}{\HoLogoBkm@LaTeXTeX}
%    \begin{macrocode}
\def\HoLogoBkm@LaTeXTeX#1{(La)TeX}
%    \end{macrocode}
%    \end{macro}
%
%    \begin{macro}{\HoLogo@(La)TeX}
%    \begin{macrocode}
\expandafter
\let\csname HoLogo@(La)TeX\endcsname\HoLogo@LaTeXTeX
%    \end{macrocode}
%    \end{macro}
%    \begin{macro}{\HoLogoBkm@(La)TeX}
%    \begin{macrocode}
\expandafter
\let\csname HoLogoBkm@(La)TeX\endcsname\HoLogoBkm@LaTeXTeX
%    \end{macrocode}
%    \end{macro}
%    \begin{macro}{\HoLogoHtml@LaTeXTeX}
%    \begin{macrocode}
\def\HoLogoHtml@LaTeXTeX#1{%
  \HoLogoCss@LaTeXTeX
  \HOLOGO@Span{LaTeXTeX}{%
    (%
    \HOLOGO@Span{L}{L}%
    \HOLOGO@Span{a}{A}%
    \HOLOGO@Span{ParenRight}{)}%
    \hologo{TeX}%
  }%
}
%    \end{macrocode}
%    \end{macro}
%    \begin{macro}{\HoLogoHtml@(La)TeX}
%    Kerning after opening parentheses and before closing parentheses
%    is $-0.1$\,em. The original values $-0.15$\,em
%    looked too ugly for a serif font.
%    \begin{macrocode}
\expandafter
\let\csname HoLogoHtml@(La)TeX\endcsname\HoLogoHtml@LaTeXTeX
%    \end{macrocode}
%    \end{macro}
%    \begin{macro}{\HoLogoCss@LaTeXTeX}
%    \begin{macrocode}
\def\HoLogoCss@LaTeXTeX{%
  \Css{%
    span.HoLogo-LaTeXTeX span.HoLogo-L{%
      margin-left:-.1em;%
    }%
  }%
  \Css{%
    span.HoLogo-LaTeXTeX span.HoLogo-a{%
      position:relative;%
      top:-.5ex;%
      margin-left:-.36em;%
      margin-right:-.1em;%
      font-size:85\%;%
    }%
  }%
  \Css{%
    span.HoLogo-LaTeXTeX span.HoLogo-ParenRight{%
      margin-right:-.15em;%
    }%
  }%
  \global\let\HoLogoCss@LaTeXTeX\relax
}
%    \end{macrocode}
%    \end{macro}
%
% \subsubsection{\hologo{LaTeXe}}
%
%    \begin{macro}{\HoLogo@LaTeXe}
%    Source: \hologo{LaTeX} kernel
%    \begin{macrocode}
\def\HoLogo@LaTeXe#1{%
  \hologo{LaTeX}%
  \kern.15em%
  \hbox{%
    \HOLOGO@MathSetup
    2%
    $_{\textstyle\varepsilon}$%
  }%
}
%    \end{macrocode}
%    \end{macro}
%
%    \begin{macro}{\HoLogoCs@LaTeXe}
%    \begin{macrocode}
\ifnum64=`\^^^^0040\relax % test for big chars of LuaTeX/XeTeX
  \catcode`\$=9 %
  \catcode`\&=14 %
\else
  \catcode`\$=14 %
  \catcode`\&=9 %
\fi
\def\HoLogoCs@LaTeXe#1{%
  LaTeX2%
$ \string ^^^^0395%
& e%
}%
\catcode`\$=3 %
\catcode`\&=4 %
%    \end{macrocode}
%    \end{macro}
%
%    \begin{macro}{\HoLogoBkm@LaTeXe}
%    \begin{macrocode}
\def\HoLogoBkm@LaTeXe#1{%
  \hologo{LaTeX}%
  2%
  \HOLOGO@PdfdocUnicode{e}{\textepsilon}%
}
%    \end{macrocode}
%    \end{macro}
%
%    \begin{macro}{\HoLogoHtml@LaTeXe}
%    \begin{macrocode}
\def\HoLogoHtml@LaTeXe#1{%
  \HoLogoCss@LaTeXe
  \HOLOGO@Span{LaTeX2e}{%
    \hologo{LaTeX}%
    \HOLOGO@Span{2}{2}%
    \HOLOGO@Span{e}{%
      \HOLOGO@MathSetup
      \ensuremath{\textstyle\varepsilon}%
    }%
  }%
}
%    \end{macrocode}
%    \end{macro}
%    \begin{macro}{\HoLogoCss@LaTeXe}
%    \begin{macrocode}
\def\HoLogoCss@LaTeXe{%
  \Css{%
    span.HoLogo-LaTeX2e span.HoLogo-2{%
      padding-left:.15em;%
    }%
  }%
  \Css{%
    span.HoLogo-LaTeX2e span.HoLogo-e{%
      position:relative;%
      top:.35ex;%
      text-decoration:none;%
    }%
  }%
  \global\let\HoLogoCss@LaTeXe\relax
}
%    \end{macrocode}
%    \end{macro}
%
%    \begin{macro}{\HoLogo@LaTeX2e}
%    \begin{macrocode}
\expandafter
\let\csname HoLogo@LaTeX2e\endcsname\HoLogo@LaTeXe
%    \end{macrocode}
%    \end{macro}
%    \begin{macro}{\HoLogoCs@LaTeX2e}
%    \begin{macrocode}
\expandafter
\let\csname HoLogoCs@LaTeX2e\endcsname\HoLogoCs@LaTeXe
%    \end{macrocode}
%    \end{macro}
%    \begin{macro}{\HoLogoBkm@LaTeX2e}
%    \begin{macrocode}
\expandafter
\let\csname HoLogoBkm@LaTeX2e\endcsname\HoLogoBkm@LaTeXe
%    \end{macrocode}
%    \end{macro}
%    \begin{macro}{\HoLogoHtml@LaTeX2e}
%    \begin{macrocode}
\expandafter
\let\csname HoLogoHtml@LaTeX2e\endcsname\HoLogoHtml@LaTeXe
%    \end{macrocode}
%    \end{macro}
%
% \subsubsection{\hologo{LaTeX3}}
%
%    \begin{macro}{\HoLogo@LaTeX3}
%    Source: \hologo{LaTeX} kernel
%    \begin{macrocode}
\expandafter\def\csname HoLogo@LaTeX3\endcsname#1{%
  \hologo{LaTeX}%
  3%
}
%    \end{macrocode}
%    \end{macro}
%
%    \begin{macro}{\HoLogoBkm@LaTeX3}
%    \begin{macrocode}
\expandafter\def\csname HoLogoBkm@LaTeX3\endcsname#1{%
  \hologo{LaTeX}%
  3%
}
%    \end{macrocode}
%    \end{macro}
%    \begin{macro}{\HoLogoHtml@LaTeX3}
%    \begin{macrocode}
\expandafter
\let\csname HoLogoHtml@LaTeX3\expandafter\endcsname
\csname HoLogo@LaTeX3\endcsname
%    \end{macrocode}
%    \end{macro}
%
% \subsubsection{\hologo{LaTeXML}}
%
%    \begin{macro}{\HoLogo@LaTeXML}
%    \begin{macrocode}
\def\HoLogo@LaTeXML#1{%
  \HOLOGO@mbox{%
    \hologo{La}%
    \kern-.15em%
    T%
    \kern-.1667em%
    \lower.5ex\hbox{E}%
    \kern-.125em%
    \HoLogoFont@font{LaTeXML}{sc}{xml}%
  }%
}
%    \end{macrocode}
%    \end{macro}
%    \begin{macro}{\HoLogoHtml@pdfLaTeX}
%    \begin{macrocode}
\def\HoLogoHtml@LaTeXML#1{%
  \HOLOGO@Span{LaTeXML}{%
    \HoLogoCss@LaTeX
    \HoLogoCss@TeX
    \HOLOGO@Span{LaTeX}{%
      L%
      \HOLOGO@Span{a}{%
        A%
      }%
    }%
    \HOLOGO@Span{TeX}{%
      T%
      \HOLOGO@Span{e}{%
        E%
      }%
    }%
    \HCode{<span style="font-variant: small-caps;">}%
    xml%
    \HCode{</span>}%
  }%
}
%    \end{macrocode}
%    \end{macro}
%
% \subsubsection{\hologo{eTeX}}
%
%    \begin{macro}{\HoLogo@eTeX}
%    Source: package \xpackage{etex}
%    \begin{macrocode}
\def\HoLogo@eTeX#1{%
  \ltx@mbox{%
    \HOLOGO@MathSetup
    $\varepsilon$%
    -%
    \HOLOGO@NegativeKerning{-T,T-,To}%
    \hologo{TeX}%
  }%
}
%    \end{macrocode}
%    \end{macro}
%    \begin{macro}{\HoLogoCs@eTeX}
%    \begin{macrocode}
\ifnum64=`\^^^^0040\relax % test for big chars of LuaTeX/XeTeX
  \catcode`\$=9 %
  \catcode`\&=14 %
\else
  \catcode`\$=14 %
  \catcode`\&=9 %
\fi
\def\HoLogoCs@eTeX#1{%
$ #1{\string ^^^^0395}{\string ^^^^03b5}%
& #1{e}{E}%
  TeX%
}%
\catcode`\$=3 %
\catcode`\&=4 %
%    \end{macrocode}
%    \end{macro}
%    \begin{macro}{\HoLogoBkm@eTeX}
%    \begin{macrocode}
\def\HoLogoBkm@eTeX#1{%
  \HOLOGO@PdfdocUnicode{#1{e}{E}}{\textepsilon}%
  -%
  \hologo{TeX}%
}
%    \end{macrocode}
%    \end{macro}
%    \begin{macro}{\HoLogoHtml@eTeX}
%    \begin{macrocode}
\def\HoLogoHtml@eTeX#1{%
  \ltx@mbox{%
    \HOLOGO@MathSetup
    $\varepsilon$%
    -%
    \hologo{TeX}%
  }%
}
%    \end{macrocode}
%    \end{macro}
%
% \subsubsection{\hologo{iniTeX}}
%
%    \begin{macro}{\HoLogo@iniTeX}
%    \begin{macrocode}
\def\HoLogo@iniTeX#1{%
  \HOLOGO@mbox{%
    #1{i}{I}ni\hologo{TeX}%
  }%
}
%    \end{macrocode}
%    \end{macro}
%    \begin{macro}{\HoLogoCs@iniTeX}
%    \begin{macrocode}
\def\HoLogoCs@iniTeX#1{#1{i}{I}niTeX}
%    \end{macrocode}
%    \end{macro}
%    \begin{macro}{\HoLogoBkm@iniTeX}
%    \begin{macrocode}
\def\HoLogoBkm@iniTeX#1{%
  #1{i}{I}ni\hologo{TeX}%
}
%    \end{macrocode}
%    \end{macro}
%    \begin{macro}{\HoLogoHtml@iniTeX}
%    \begin{macrocode}
\let\HoLogoHtml@iniTeX\HoLogo@iniTeX
%    \end{macrocode}
%    \end{macro}
%
% \subsubsection{\hologo{virTeX}}
%
%    \begin{macro}{\HoLogo@virTeX}
%    \begin{macrocode}
\def\HoLogo@virTeX#1{%
  \HOLOGO@mbox{%
    #1{v}{V}ir\hologo{TeX}%
  }%
}
%    \end{macrocode}
%    \end{macro}
%    \begin{macro}{\HoLogoCs@virTeX}
%    \begin{macrocode}
\def\HoLogoCs@virTeX#1{#1{v}{V}irTeX}
%    \end{macrocode}
%    \end{macro}
%    \begin{macro}{\HoLogoBkm@virTeX}
%    \begin{macrocode}
\def\HoLogoBkm@virTeX#1{%
  #1{v}{V}ir\hologo{TeX}%
}
%    \end{macrocode}
%    \end{macro}
%    \begin{macro}{\HoLogoHtml@virTeX}
%    \begin{macrocode}
\let\HoLogoHtml@virTeX\HoLogo@virTeX
%    \end{macrocode}
%    \end{macro}
%
% \subsubsection{\hologo{SliTeX}}
%
% \paragraph{Definitions of the three variants.}
%
%    \begin{macro}{\HoLogo@SLiTeX@lift}
%    \begin{macrocode}
\def\HoLogo@SLiTeX@lift#1{%
  \HoLogoFont@font{SliTeX}{rm}{%
    S%
    \kern-.06em%
    L%
    \kern-.18em%
    \raise.32ex\hbox{\HoLogoFont@font{SliTeX}{sc}{i}}%
    \HOLOGO@discretionary
    \kern-.06em%
    \hologo{TeX}%
  }%
}
%    \end{macrocode}
%    \end{macro}
%    \begin{macro}{\HoLogoBkm@SLiTeX@lift}
%    \begin{macrocode}
\def\HoLogoBkm@SLiTeX@lift#1{SLiTeX}
%    \end{macrocode}
%    \end{macro}
%    \begin{macro}{\HoLogoHtml@SLiTeX@lift}
%    \begin{macrocode}
\def\HoLogoHtml@SLiTeX@lift#1{%
  \HoLogoCss@SLiTeX@lift
  \HOLOGO@Span{SLiTeX-lift}{%
    \HoLogoFont@font{SliTeX}{rm}{%
      S%
      \HOLOGO@Span{L}{L}%
      \HOLOGO@Span{i}{i}%
      \hologo{TeX}%
    }%
  }%
}
%    \end{macrocode}
%    \end{macro}
%    \begin{macro}{\HoLogoCss@SLiTeX@lift}
%    \begin{macrocode}
\def\HoLogoCss@SLiTeX@lift{%
  \Css{%
    span.HoLogo-SLiTeX-lift span.HoLogo-L{%
      margin-left:-.06em;%
      margin-right:-.18em;%
    }%
  }%
  \Css{%
    span.HoLogo-SLiTeX-lift span.HoLogo-i{%
      position:relative;%
      top:-.32ex;%
      margin-right:-.06em;%
      font-variant:small-caps;%
    }%
  }%
  \global\let\HoLogoCss@SLiTeX@lift\relax
}
%    \end{macrocode}
%    \end{macro}
%
%    \begin{macro}{\HoLogo@SliTeX@simple}
%    \begin{macrocode}
\def\HoLogo@SliTeX@simple#1{%
  \HoLogoFont@font{SliTeX}{rm}{%
    \ltx@mbox{%
      \HoLogoFont@font{SliTeX}{sc}{Sli}%
    }%
    \HOLOGO@discretionary
    \hologo{TeX}%
  }%
}
%    \end{macrocode}
%    \end{macro}
%    \begin{macro}{\HoLogoBkm@SliTeX@simple}
%    \begin{macrocode}
\def\HoLogoBkm@SliTeX@simple#1{SliTeX}
%    \end{macrocode}
%    \end{macro}
%    \begin{macro}{\HoLogoHtml@SliTeX@simple}
%    \begin{macrocode}
\let\HoLogoHtml@SliTeX@simple\HoLogo@SliTeX@simple
%    \end{macrocode}
%    \end{macro}
%
%    \begin{macro}{\HoLogo@SliTeX@narrow}
%    \begin{macrocode}
\def\HoLogo@SliTeX@narrow#1{%
  \HoLogoFont@font{SliTeX}{rm}{%
    \ltx@mbox{%
      S%
      \kern-.06em%
      \HoLogoFont@font{SliTeX}{sc}{%
        l%
        \kern-.035em%
        i%
      }%
    }%
    \HOLOGO@discretionary
    \kern-.06em%
    \hologo{TeX}%
  }%
}
%    \end{macrocode}
%    \end{macro}
%    \begin{macro}{\HoLogoBkm@SliTeX@narrow}
%    \begin{macrocode}
\def\HoLogoBkm@SliTeX@narrow#1{SliTeX}
%    \end{macrocode}
%    \end{macro}
%    \begin{macro}{\HoLogoHtml@SliTeX@narrow}
%    \begin{macrocode}
\def\HoLogoHtml@SliTeX@narrow#1{%
  \HoLogoCss@SliTeX@narrow
  \HOLOGO@Span{SliTeX-narrow}{%
    \HoLogoFont@font{SliTeX}{rm}{%
      S%
        \HOLOGO@Span{l}{l}%
        \HOLOGO@Span{i}{i}%
      \hologo{TeX}%
    }%
  }%
}
%    \end{macrocode}
%    \end{macro}
%    \begin{macro}{\HoLogoCss@SliTeX@narrow}
%    \begin{macrocode}
\def\HoLogoCss@SliTeX@narrow{%
  \Css{%
    span.HoLogo-SliTeX-narrow span.HoLogo-l{%
      margin-left:-.06em;%
      margin-right:-.035em;%
      font-variant:small-caps;%
    }%
  }%
  \Css{%
    span.HoLogo-SliTeX-narrow span.HoLogo-i{%
      margin-right:-.06em;%
      font-variant:small-caps;%
    }%
  }%
  \global\let\HoLogoCss@SliTeX@narrow\relax
}
%    \end{macrocode}
%    \end{macro}
%
% \paragraph{Macro set completion.}
%
%    \begin{macro}{\HoLogo@SLiTeX@simple}
%    \begin{macrocode}
\def\HoLogo@SLiTeX@simple{\HoLogo@SliTeX@simple}
%    \end{macrocode}
%    \end{macro}
%    \begin{macro}{\HoLogoBkm@SLiTeX@simple}
%    \begin{macrocode}
\def\HoLogoBkm@SLiTeX@simple{\HoLogoBkm@SliTeX@simple}
%    \end{macrocode}
%    \end{macro}
%    \begin{macro}{\HoLogoHtml@SLiTeX@simple}
%    \begin{macrocode}
\def\HoLogoHtml@SLiTeX@simple{\HoLogoHtml@SliTeX@simple}
%    \end{macrocode}
%    \end{macro}
%
%    \begin{macro}{\HoLogo@SLiTeX@narrow}
%    \begin{macrocode}
\def\HoLogo@SLiTeX@narrow{\HoLogo@SliTeX@narrow}
%    \end{macrocode}
%    \end{macro}
%    \begin{macro}{\HoLogoBkm@SLiTeX@narrow}
%    \begin{macrocode}
\def\HoLogoBkm@SLiTeX@narrow{\HoLogoBkm@SliTeX@narrow}
%    \end{macrocode}
%    \end{macro}
%    \begin{macro}{\HoLogoHtml@SLiTeX@narrow}
%    \begin{macrocode}
\def\HoLogoHtml@SLiTeX@narrow{\HoLogoHtml@SliTeX@narrow}
%    \end{macrocode}
%    \end{macro}
%
%    \begin{macro}{\HoLogo@SliTeX@lift}
%    \begin{macrocode}
\def\HoLogo@SliTeX@lift{\HoLogo@SLiTeX@lift}
%    \end{macrocode}
%    \end{macro}
%    \begin{macro}{\HoLogoBkm@SliTeX@lift}
%    \begin{macrocode}
\def\HoLogoBkm@SliTeX@lift{\HoLogoBkm@SLiTeX@lift}
%    \end{macrocode}
%    \end{macro}
%    \begin{macro}{\HoLogoHtml@SliTeX@lift}
%    \begin{macrocode}
\def\HoLogoHtml@SliTeX@lift{\HoLogoHtml@SLiTeX@lift}
%    \end{macrocode}
%    \end{macro}
%
% \paragraph{Defaults.}
%
%    \begin{macro}{\HoLogo@SLiTeX}
%    \begin{macrocode}
\def\HoLogo@SLiTeX{\HoLogo@SLiTeX@lift}
%    \end{macrocode}
%    \end{macro}
%    \begin{macro}{\HoLogoBkm@SLiTeX}
%    \begin{macrocode}
\def\HoLogoBkm@SLiTeX{\HoLogoBkm@SLiTeX@lift}
%    \end{macrocode}
%    \end{macro}
%    \begin{macro}{\HoLogoHtml@SLiTeX}
%    \begin{macrocode}
\def\HoLogoHtml@SLiTeX{\HoLogoHtml@SLiTeX@lift}
%    \end{macrocode}
%    \end{macro}
%
%    \begin{macro}{\HoLogo@SliTeX}
%    \begin{macrocode}
\def\HoLogo@SliTeX{\HoLogo@SliTeX@narrow}
%    \end{macrocode}
%    \end{macro}
%    \begin{macro}{\HoLogoBkm@SliTeX}
%    \begin{macrocode}
\def\HoLogoBkm@SliTeX{\HoLogoBkm@SliTeX@narrow}
%    \end{macrocode}
%    \end{macro}
%    \begin{macro}{\HoLogoHtml@SliTeX}
%    \begin{macrocode}
\def\HoLogoHtml@SliTeX{\HoLogoHtml@SliTeX@narrow}
%    \end{macrocode}
%    \end{macro}
%
% \subsubsection{\hologo{LuaTeX}}
%
%    \begin{macro}{\HoLogo@LuaTeX}
%    The kerning is an idea of Hans Hagen, see mailing list
%    `luatex at tug dot org' in March 2010.
%    \begin{macrocode}
\def\HoLogo@LuaTeX#1{%
  \HOLOGO@mbox{%
    Lua%
    \HOLOGO@NegativeKerning{aT,oT,To}%
    \hologo{TeX}%
  }%
}
%    \end{macrocode}
%    \end{macro}
%    \begin{macro}{\HoLogoHtml@LuaTeX}
%    \begin{macrocode}
\let\HoLogoHtml@LuaTeX\HoLogo@LuaTeX
%    \end{macrocode}
%    \end{macro}
%
% \subsubsection{\hologo{LuaLaTeX}}
%
%    \begin{macro}{\HoLogo@LuaLaTeX}
%    \begin{macrocode}
\def\HoLogo@LuaLaTeX#1{%
  \HOLOGO@mbox{%
    Lua%
    \hologo{LaTeX}%
  }%
}
%    \end{macrocode}
%    \end{macro}
%    \begin{macro}{\HoLogoHtml@LuaLaTeX}
%    \begin{macrocode}
\let\HoLogoHtml@LuaLaTeX\HoLogo@LuaLaTeX
%    \end{macrocode}
%    \end{macro}
%
% \subsubsection{\hologo{XeTeX}, \hologo{XeLaTeX}}
%
%    \begin{macro}{\HOLOGO@IfCharExists}
%    \begin{macrocode}
\ifluatex
  \ifnum\luatexversion<36 %
  \else
    \def\HOLOGO@IfCharExists#1{%
      \ifnum
        \directlua{%
           if luaotfload and luaotfload.aux then
             if luaotfload.aux.font_has_glyph(%
                    font.current(), \number#1) then % 	 
	       tex.print("1") % 	 
	     end % 	 
	   elseif font and font.fonts and font.current then %
            local f = font.fonts[font.current()]%
            if f.characters and f.characters[\number#1] then %
              tex.print("1")%
            end %
          end%
        }0=\ltx@zero
        \expandafter\ltx@secondoftwo
      \else
        \expandafter\ltx@firstoftwo
      \fi
    }%
  \fi
\fi
\ltx@IfUndefined{HOLOGO@IfCharExists}{%
  \def\HOLOGO@@IfCharExists#1{%
    \begingroup
      \tracinglostchars=\ltx@zero
      \setbox\ltx@zero=\hbox{%
        \kern7sp\char#1\relax
        \ifnum\lastkern>\ltx@zero
          \expandafter\aftergroup\csname iffalse\endcsname
        \else
          \expandafter\aftergroup\csname iftrue\endcsname
        \fi
      }%
      % \if{true|false} from \aftergroup
      \endgroup
      \expandafter\ltx@firstoftwo
    \else
      \endgroup
      \expandafter\ltx@secondoftwo
    \fi
  }%
  \ifxetex
    \ltx@IfUndefined{XeTeXfonttype}{}{%
      \ltx@IfUndefined{XeTeXcharglyph}{}{%
        \def\HOLOGO@IfCharExists#1{%
          \ifnum\XeTeXfonttype\font>\ltx@zero
            \expandafter\ltx@firstofthree
          \else
            \expandafter\ltx@gobble
          \fi
          {%
            \ifnum\XeTeXcharglyph#1>\ltx@zero
              \expandafter\ltx@firstoftwo
            \else
              \expandafter\ltx@secondoftwo
            \fi
          }%
          \HOLOGO@@IfCharExists{#1}%
        }%
      }%
    }%
  \fi
}{}
\ltx@ifundefined{HOLOGO@IfCharExists}{%
  \ifnum64=`\^^^^0040\relax % test for big chars of LuaTeX/XeTeX
    \let\HOLOGO@IfCharExists\HOLOGO@@IfCharExists
  \else
    \def\HOLOGO@IfCharExists#1{%
      \ifnum#1>255 %
        \expandafter\ltx@fourthoffour
      \fi
      \HOLOGO@@IfCharExists{#1}%
    }%
  \fi
}{}
%    \end{macrocode}
%    \end{macro}
%
%    \begin{macro}{\HoLogo@Xe}
%    Source: package \xpackage{dtklogos}
%    \begin{macrocode}
\def\HoLogo@Xe#1{%
  X%
  \kern-.1em\relax
  \HOLOGO@IfCharExists{"018E}{%
    \lower.5ex\hbox{\char"018E}%
  }{%
    \chardef\HOLOGO@choice=\ltx@zero
    \ifdim\fontdimen\ltx@one\font>0pt %
      \ltx@IfUndefined{rotatebox}{%
        \ltx@IfUndefined{pgftext}{%
          \ltx@IfUndefined{psscalebox}{%
            \ltx@IfUndefined{HOLOGO@ScaleBox@\hologoDriver}{%
            }{%
              \chardef\HOLOGO@choice=4 %
            }%
          }{%
            \chardef\HOLOGO@choice=3 %
          }%
        }{%
          \chardef\HOLOGO@choice=2 %
        }%
      }{%
        \chardef\HOLOGO@choice=1 %
      }%
      \ifcase\HOLOGO@choice
        \HOLOGO@WarningUnsupportedDriver{Xe}%
        e%
      \or % 1: \rotatebox
        \begingroup
          \setbox\ltx@zero\hbox{\rotatebox{180}{E}}%
          \ltx@LocDimenA=\dp\ltx@zero
          \advance\ltx@LocDimenA by -.5ex\relax
          \raise\ltx@LocDimenA\box\ltx@zero
        \endgroup
      \or % 2: \pgftext
        \lower.5ex\hbox{%
          \pgfpicture
            \pgftext[rotate=180]{E}%
          \endpgfpicture
        }%
      \or % 3: \psscalebox
        \begingroup
          \setbox\ltx@zero\hbox{\psscalebox{-1 -1}{E}}%
          \ltx@LocDimenA=\dp\ltx@zero
          \advance\ltx@LocDimenA by -.5ex\relax
          \raise\ltx@LocDimenA\box\ltx@zero
        \endgroup
      \or % 4: \HOLOGO@PointReflectBox
        \lower.5ex\hbox{\HOLOGO@PointReflectBox{E}}%
      \else
        \@PackageError{hologo}{Internal error (choice/it}\@ehc
      \fi
    \else
      \ltx@IfUndefined{reflectbox}{%
        \ltx@IfUndefined{pgftext}{%
          \ltx@IfUndefined{psscalebox}{%
            \ltx@IfUndefined{HOLOGO@ScaleBox@\hologoDriver}{%
            }{%
              \chardef\HOLOGO@choice=4 %
            }%
          }{%
            \chardef\HOLOGO@choice=3 %
          }%
        }{%
          \chardef\HOLOGO@choice=2 %
        }%
      }{%
        \chardef\HOLOGO@choice=1 %
      }%
      \ifcase\HOLOGO@choice
        \HOLOGO@WarningUnsupportedDriver{Xe}%
        e%
      \or % 1: reflectbox
        \lower.5ex\hbox{%
          \reflectbox{E}%
        }%
      \or % 2: \pgftext
        \lower.5ex\hbox{%
          \pgfpicture
            \pgftransformxscale{-1}%
            \pgftext{E}%
          \endpgfpicture
        }%
      \or % 3: \psscalebox
        \lower.5ex\hbox{%
          \psscalebox{-1 1}{E}%
        }%
      \or % 4: \HOLOGO@Reflectbox
        \lower.5ex\hbox{%
          \HOLOGO@ReflectBox{E}%
        }%
      \else
        \@PackageError{hologo}{Internal error (choice/up)}\@ehc
      \fi
    \fi
  }%
}
%    \end{macrocode}
%    \end{macro}
%    \begin{macro}{\HoLogoHtml@Xe}
%    \begin{macrocode}
\def\HoLogoHtml@Xe#1{%
  \HoLogoCss@Xe
  \HOLOGO@Span{Xe}{%
    X%
    \HOLOGO@Span{e}{%
      \HCode{&\ltx@hashchar x018e;}%
    }%
  }%
}
%    \end{macrocode}
%    \end{macro}
%    \begin{macro}{\HoLogoCss@Xe}
%    \begin{macrocode}
\def\HoLogoCss@Xe{%
  \Css{%
    span.HoLogo-Xe span.HoLogo-e{%
      position:relative;%
      top:.5ex;%
      left-margin:-.1em;%
    }%
  }%
  \global\let\HoLogoCss@Xe\relax
}
%    \end{macrocode}
%    \end{macro}
%
%    \begin{macro}{\HoLogo@XeTeX}
%    \begin{macrocode}
\def\HoLogo@XeTeX#1{%
  \hologo{Xe}%
  \kern-.15em\relax
  \hologo{TeX}%
}
%    \end{macrocode}
%    \end{macro}
%
%    \begin{macro}{\HoLogoHtml@XeTeX}
%    \begin{macrocode}
\def\HoLogoHtml@XeTeX#1{%
  \HoLogoCss@XeTeX
  \HOLOGO@Span{XeTeX}{%
    \hologo{Xe}%
    \hologo{TeX}%
  }%
}
%    \end{macrocode}
%    \end{macro}
%    \begin{macro}{\HoLogoCss@XeTeX}
%    \begin{macrocode}
\def\HoLogoCss@XeTeX{%
  \Css{%
    span.HoLogo-XeTeX span.HoLogo-TeX{%
      margin-left:-.15em;%
    }%
  }%
  \global\let\HoLogoCss@XeTeX\relax
}
%    \end{macrocode}
%    \end{macro}
%
%    \begin{macro}{\HoLogo@XeLaTeX}
%    \begin{macrocode}
\def\HoLogo@XeLaTeX#1{%
  \hologo{Xe}%
  \kern-.13em%
  \hologo{LaTeX}%
}
%    \end{macrocode}
%    \end{macro}
%    \begin{macro}{\HoLogoHtml@XeLaTeX}
%    \begin{macrocode}
\def\HoLogoHtml@XeLaTeX#1{%
  \HoLogoCss@XeLaTeX
  \HOLOGO@Span{XeLaTeX}{%
    \hologo{Xe}%
    \hologo{LaTeX}%
  }%
}
%    \end{macrocode}
%    \end{macro}
%    \begin{macro}{\HoLogoCss@XeLaTeX}
%    \begin{macrocode}
\def\HoLogoCss@XeLaTeX{%
  \Css{%
    span.HoLogo-XeLaTeX span.HoLogo-Xe{%
      margin-right:-.13em;%
    }%
  }%
  \global\let\HoLogoCss@XeLaTeX\relax
}
%    \end{macrocode}
%    \end{macro}
%
% \subsubsection{\hologo{pdfTeX}, \hologo{pdfLaTeX}}
%
%    \begin{macro}{\HoLogo@pdfTeX}
%    \begin{macrocode}
\def\HoLogo@pdfTeX#1{%
  \HOLOGO@mbox{%
    #1{p}{P}df\hologo{TeX}%
  }%
}
%    \end{macrocode}
%    \end{macro}
%    \begin{macro}{\HoLogoCs@pdfTeX}
%    \begin{macrocode}
\def\HoLogoCs@pdfTeX#1{#1{p}{P}dfTeX}
%    \end{macrocode}
%    \end{macro}
%    \begin{macro}{\HoLogoBkm@pdfTeX}
%    \begin{macrocode}
\def\HoLogoBkm@pdfTeX#1{%
  #1{p}{P}df\hologo{TeX}%
}
%    \end{macrocode}
%    \end{macro}
%    \begin{macro}{\HoLogoHtml@pdfTeX}
%    \begin{macrocode}
\let\HoLogoHtml@pdfTeX\HoLogo@pdfTeX
%    \end{macrocode}
%    \end{macro}
%
%    \begin{macro}{\HoLogo@pdfLaTeX}
%    \begin{macrocode}
\def\HoLogo@pdfLaTeX#1{%
  \HOLOGO@mbox{%
    #1{p}{P}df\hologo{LaTeX}%
  }%
}
%    \end{macrocode}
%    \end{macro}
%    \begin{macro}{\HoLogoCs@pdfLaTeX}
%    \begin{macrocode}
\def\HoLogoCs@pdfLaTeX#1{#1{p}{P}dfLaTeX}
%    \end{macrocode}
%    \end{macro}
%    \begin{macro}{\HoLogoBkm@pdfLaTeX}
%    \begin{macrocode}
\def\HoLogoBkm@pdfLaTeX#1{%
  #1{p}{P}df\hologo{LaTeX}%
}
%    \end{macrocode}
%    \end{macro}
%    \begin{macro}{\HoLogoHtml@pdfLaTeX}
%    \begin{macrocode}
\let\HoLogoHtml@pdfLaTeX\HoLogo@pdfLaTeX
%    \end{macrocode}
%    \end{macro}
%
% \subsubsection{\hologo{VTeX}}
%
%    \begin{macro}{\HoLogo@VTeX}
%    \begin{macrocode}
\def\HoLogo@VTeX#1{%
  \HOLOGO@mbox{%
    V\hologo{TeX}%
  }%
}
%    \end{macrocode}
%    \end{macro}
%    \begin{macro}{\HoLogoHtml@VTeX}
%    \begin{macrocode}
\let\HoLogoHtml@VTeX\HoLogo@VTeX
%    \end{macrocode}
%    \end{macro}
%
% \subsubsection{\hologo{AmS}, \dots}
%
%    Source: class \xclass{amsdtx}
%
%    \begin{macro}{\HoLogo@AmS}
%    \begin{macrocode}
\def\HoLogo@AmS#1{%
  \HoLogoFont@font{AmS}{sy}{%
    A%
    \kern-.1667em%
    \lower.5ex\hbox{M}%
    \kern-.125em%
    S%
  }%
}
%    \end{macrocode}
%    \end{macro}
%    \begin{macro}{\HoLogoBkm@AmS}
%    \begin{macrocode}
\def\HoLogoBkm@AmS#1{AmS}
%    \end{macrocode}
%    \end{macro}
%    \begin{macro}{\HoLogoHtml@AmS}
%    \begin{macrocode}
\def\HoLogoHtml@AmS#1{%
  \HoLogoCss@AmS
%  \HoLogoFont@font{AmS}{sy}{%
    \HOLOGO@Span{AmS}{%
      A%
      \HOLOGO@Span{M}{M}%
      S%
    }%
%   }%
}
%    \end{macrocode}
%    \end{macro}
%    \begin{macro}{\HoLogoCss@AmS}
%    \begin{macrocode}
\def\HoLogoCss@AmS{%
  \Css{%
    span.HoLogo-AmS span.HoLogo-M{%
      position:relative;%
      top:.5ex;%
      margin-left:-.1667em;%
      margin-right:-.125em;%
      text-decoration:none;%
    }%
  }%
  \global\let\HoLogoCss@AmS\relax
}
%    \end{macrocode}
%    \end{macro}
%
%    \begin{macro}{\HoLogo@AmSTeX}
%    \begin{macrocode}
\def\HoLogo@AmSTeX#1{%
  \hologo{AmS}%
  \HOLOGO@hyphen
  \hologo{TeX}%
}
%    \end{macrocode}
%    \end{macro}
%    \begin{macro}{\HoLogoBkm@AmSTeX}
%    \begin{macrocode}
\def\HoLogoBkm@AmSTeX#1{AmS-TeX}%
%    \end{macrocode}
%    \end{macro}
%    \begin{macro}{\HoLogoHtml@AmSTeX}
%    \begin{macrocode}
\let\HoLogoHtml@AmSTeX\HoLogo@AmSTeX
%    \end{macrocode}
%    \end{macro}
%
%    \begin{macro}{\HoLogo@AmSLaTeX}
%    \begin{macrocode}
\def\HoLogo@AmSLaTeX#1{%
  \hologo{AmS}%
  \HOLOGO@hyphen
  \hologo{LaTeX}%
}
%    \end{macrocode}
%    \end{macro}
%    \begin{macro}{\HoLogoBkm@AmSLaTeX}
%    \begin{macrocode}
\def\HoLogoBkm@AmSLaTeX#1{AmS-LaTeX}%
%    \end{macrocode}
%    \end{macro}
%    \begin{macro}{\HoLogoHtml@AmSLaTeX}
%    \begin{macrocode}
\let\HoLogoHtml@AmSLaTeX\HoLogo@AmSLaTeX
%    \end{macrocode}
%    \end{macro}
%
% \subsubsection{\hologo{BibTeX}}
%
%    \begin{macro}{\HoLogo@BibTeX@sc}
%    A definition of \hologo{BibTeX} is provided in
%    the documentation source for the manual of \hologo{BibTeX}
%    \cite{btxdoc}.
%\begin{quote}
%\begin{verbatim}
%\def\BibTeX{%
%  {%
%    \rm
%    B%
%    \kern-.05em%
%    {%
%      \sc
%      i%
%      \kern-.025em %
%      b%
%    }%
%    \kern-.08em
%    T%
%    \kern-.1667em%
%    \lower.7ex\hbox{E}%
%    \kern-.125em%
%    X%
%  }%
%}
%\end{verbatim}
%\end{quote}
%    \begin{macrocode}
\def\HoLogo@BibTeX@sc#1{%
  B%
  \kern-.05em%
  \HoLogoFont@font{BibTeX}{sc}{%
    i%
    \kern-.025em%
    b%
  }%
  \HOLOGO@discretionary
  \kern-.08em%
  \hologo{TeX}%
}
%    \end{macrocode}
%    \end{macro}
%    \begin{macro}{\HoLogoHtml@BibTeX@sc}
%    \begin{macrocode}
\def\HoLogoHtml@BibTeX@sc#1{%
  \HoLogoCss@BibTeX@sc
  \HOLOGO@Span{BibTeX-sc}{%
    B%
    \HOLOGO@Span{i}{i}%
    \HOLOGO@Span{b}{b}%
    \hologo{TeX}%
  }%
}
%    \end{macrocode}
%    \end{macro}
%    \begin{macro}{\HoLogoCss@BibTeX@sc}
%    \begin{macrocode}
\def\HoLogoCss@BibTeX@sc{%
  \Css{%
    span.HoLogo-BibTeX-sc span.HoLogo-i{%
      margin-left:-.05em;%
      margin-right:-.025em;%
      font-variant:small-caps;%
    }%
  }%
  \Css{%
    span.HoLogo-BibTeX-sc span.HoLogo-b{%
      margin-right:-.08em;%
      font-variant:small-caps;%
    }%
  }%
  \global\let\HoLogoCss@BibTeX@sc\relax
}
%    \end{macrocode}
%    \end{macro}
%
%    \begin{macro}{\HoLogo@BibTeX@sf}
%    Variant \xoption{sf} avoids trouble with unavailable
%    small caps fonts (e.g., bold versions of Computer Modern or
%    Latin Modern). The definition is taken from
%    package \xpackage{dtklogos} \cite{dtklogos}.
%\begin{quote}
%\begin{verbatim}
%\DeclareRobustCommand{\BibTeX}{%
%  B%
%  \kern-.05em%
%  \hbox{%
%    $\m@th$% %% force math size calculations
%    \csname S@\f@size\endcsname
%    \fontsize\sf@size\z@
%    \math@fontsfalse
%    \selectfont
%    I%
%    \kern-.025em%
%    B
%  }%
%  \kern-.08em%
%  \-%
%  \TeX
%}
%\end{verbatim}
%\end{quote}
%    \begin{macrocode}
\def\HoLogo@BibTeX@sf#1{%
  B%
  \kern-.05em%
  \HoLogoFont@font{BibTeX}{bibsf}{%
    I%
    \kern-.025em%
    B%
  }%
  \HOLOGO@discretionary
  \kern-.08em%
  \hologo{TeX}%
}
%    \end{macrocode}
%    \end{macro}
%    \begin{macro}{\HoLogoHtml@BibTeX@sf}
%    \begin{macrocode}
\def\HoLogoHtml@BibTeX@sf#1{%
  \HoLogoCss@BibTeX@sf
  \HOLOGO@Span{BibTeX-sf}{%
    B%
    \HoLogoFont@font{BibTeX}{bibsf}{%
      \HOLOGO@Span{i}{I}%
      B%
    }%
    \hologo{TeX}%
  }%
}
%    \end{macrocode}
%    \end{macro}
%    \begin{macro}{\HoLogoCss@BibTeX@sf}
%    \begin{macrocode}
\def\HoLogoCss@BibTeX@sf{%
  \Css{%
    span.HoLogo-BibTeX-sf span.HoLogo-i{%
      margin-left:-.05em;%
      margin-right:-.025em;%
    }%
  }%
  \Css{%
    span.HoLogo-BibTeX-sf span.HoLogo-TeX{%
      margin-left:-.08em;%
    }%
  }%
  \global\let\HoLogoCss@BibTeX@sf\relax
}
%    \end{macrocode}
%    \end{macro}
%
%    \begin{macro}{\HoLogo@BibTeX}
%    \begin{macrocode}
\def\HoLogo@BibTeX{\HoLogo@BibTeX@sf}
%    \end{macrocode}
%    \end{macro}
%    \begin{macro}{\HoLogoHtml@BibTeX}
%    \begin{macrocode}
\def\HoLogoHtml@BibTeX{\HoLogoHtml@BibTeX@sf}
%    \end{macrocode}
%    \end{macro}
%
% \subsubsection{\hologo{BibTeX8}}
%
%    \begin{macro}{\HoLogo@BibTeX8}
%    \begin{macrocode}
\expandafter\def\csname HoLogo@BibTeX8\endcsname#1{%
  \hologo{BibTeX}%
  8%
}
%    \end{macrocode}
%    \end{macro}
%
%    \begin{macro}{\HoLogoBkm@BibTeX8}
%    \begin{macrocode}
\expandafter\def\csname HoLogoBkm@BibTeX8\endcsname#1{%
  \hologo{BibTeX}%
  8%
}
%    \end{macrocode}
%    \end{macro}
%    \begin{macro}{\HoLogoHtml@BibTeX8}
%    \begin{macrocode}
\expandafter
\let\csname HoLogoHtml@BibTeX8\expandafter\endcsname
\csname HoLogo@BibTeX8\endcsname
%    \end{macrocode}
%    \end{macro}
%
% \subsubsection{\hologo{ConTeXt}}
%
%    \begin{macro}{\HoLogo@ConTeXt@simple}
%    \begin{macrocode}
\def\HoLogo@ConTeXt@simple#1{%
  \HOLOGO@mbox{Con}%
  \HOLOGO@discretionary
  \HOLOGO@mbox{\hologo{TeX}t}%
}
%    \end{macrocode}
%    \end{macro}
%    \begin{macro}{\HoLogoHtml@ConTeXt@simple}
%    \begin{macrocode}
\let\HoLogoHtml@ConTeXt@simple\HoLogo@ConTeXt@simple
%    \end{macrocode}
%    \end{macro}
%
%    \begin{macro}{\HoLogo@ConTeXt@narrow}
%    This definition of logo \hologo{ConTeXt} with variant \xoption{narrow}
%    comes from TUGboat's class \xclass{ltugboat} (version 2010/11/15 v2.8).
%    \begin{macrocode}
\def\HoLogo@ConTeXt@narrow#1{%
  \HOLOGO@mbox{C\kern-.0333emon}%
  \HOLOGO@discretionary
  \kern-.0667em%
  \HOLOGO@mbox{\hologo{TeX}\kern-.0333emt}%
}
%    \end{macrocode}
%    \end{macro}
%    \begin{macro}{\HoLogoHtml@ConTeXt@narrow}
%    \begin{macrocode}
\def\HoLogoHtml@ConTeXt@narrow#1{%
  \HoLogoCss@ConTeXt@narrow
  \HOLOGO@Span{ConTeXt-narrow}{%
    \HOLOGO@Span{C}{C}%
    on%
    \hologo{TeX}%
    t%
  }%
}
%    \end{macrocode}
%    \end{macro}
%    \begin{macro}{\HoLogoCss@ConTeXt@narrow}
%    \begin{macrocode}
\def\HoLogoCss@ConTeXt@narrow{%
  \Css{%
    span.HoLogo-ConTeXt-narrow span.HoLogo-C{%
      margin-left:-.0333em;%
    }%
  }%
  \Css{%
    span.HoLogo-ConTeXt-narrow span.HoLogo-TeX{%
      margin-left:-.0667em;%
      margin-right:-.0333em;%
    }%
  }%
  \global\let\HoLogoCss@ConTeXt@narrow\relax
}
%    \end{macrocode}
%    \end{macro}
%
%    \begin{macro}{\HoLogo@ConTeXt}
%    \begin{macrocode}
\def\HoLogo@ConTeXt{\HoLogo@ConTeXt@narrow}
%    \end{macrocode}
%    \end{macro}
%    \begin{macro}{\HoLogoHtml@ConTeXt}
%    \begin{macrocode}
\def\HoLogoHtml@ConTeXt{\HoLogoHtml@ConTeXt@narrow}
%    \end{macrocode}
%    \end{macro}
%
% \subsubsection{\hologo{emTeX}}
%
%    \begin{macro}{\HoLogo@emTeX}
%    \begin{macrocode}
\def\HoLogo@emTeX#1{%
  \HOLOGO@mbox{#1{e}{E}m}%
  \HOLOGO@discretionary
  \hologo{TeX}%
}
%    \end{macrocode}
%    \end{macro}
%    \begin{macro}{\HoLogoCs@emTeX}
%    \begin{macrocode}
\def\HoLogoCs@emTeX#1{#1{e}{E}mTeX}%
%    \end{macrocode}
%    \end{macro}
%    \begin{macro}{\HoLogoBkm@emTeX}
%    \begin{macrocode}
\def\HoLogoBkm@emTeX#1{%
  #1{e}{E}m\hologo{TeX}%
}
%    \end{macrocode}
%    \end{macro}
%    \begin{macro}{\HoLogoHtml@emTeX}
%    \begin{macrocode}
\let\HoLogoHtml@emTeX\HoLogo@emTeX
%    \end{macrocode}
%    \end{macro}
%
% \subsubsection{\hologo{ExTeX}}
%
%    \begin{macro}{\HoLogo@ExTeX}
%    The definition is taken from the FAQ of the
%    project \hologo{ExTeX}
%    \cite{ExTeX-FAQ}.
%\begin{quote}
%\begin{verbatim}
%\def\ExTeX{%
%  \textrm{% Logo always with serifs
%    \ensuremath{%
%      \textstyle
%      \varepsilon_{%
%        \kern-0.15em%
%        \mathcal{X}%
%      }%
%    }%
%    \kern-.15em%
%    \TeX
%  }%
%}
%\end{verbatim}
%\end{quote}
%    \begin{macrocode}
\def\HoLogo@ExTeX#1{%
  \HoLogoFont@font{ExTeX}{rm}{%
    \ltx@mbox{%
      \HOLOGO@MathSetup
      $%
        \textstyle
        \varepsilon_{%
          \kern-0.15em%
          \HoLogoFont@font{ExTeX}{sy}{X}%
        }%
      $%
    }%
    \HOLOGO@discretionary
    \kern-.15em%
    \hologo{TeX}%
  }%
}
%    \end{macrocode}
%    \end{macro}
%    \begin{macro}{\HoLogoHtml@ExTeX}
%    \begin{macrocode}
\def\HoLogoHtml@ExTeX#1{%
  \HoLogoCss@ExTeX
  \HoLogoFont@font{ExTeX}{rm}{%
    \HOLOGO@Span{ExTeX}{%
      \ltx@mbox{%
        \HOLOGO@MathSetup
        $\textstyle\varepsilon$%
        \HOLOGO@Span{X}{$\textstyle\chi$}%
        \hologo{TeX}%
      }%
    }%
  }%
}
%    \end{macrocode}
%    \end{macro}
%    \begin{macro}{\HoLogoBkm@ExTeX}
%    \begin{macrocode}
\def\HoLogoBkm@ExTeX#1{%
  \HOLOGO@PdfdocUnicode{#1{e}{E}x}{\textepsilon\textchi}%
  \hologo{TeX}%
}
%    \end{macrocode}
%    \end{macro}
%    \begin{macro}{\HoLogoCss@ExTeX}
%    \begin{macrocode}
\def\HoLogoCss@ExTeX{%
  \Css{%
    span.HoLogo-ExTeX{%
      font-family:serif;%
    }%
  }%
  \Css{%
    span.HoLogo-ExTeX span.HoLogo-TeX{%
      margin-left:-.15em;%
    }%
  }%
  \global\let\HoLogoCss@ExTeX\relax
}
%    \end{macrocode}
%    \end{macro}
%
% \subsubsection{\hologo{MiKTeX}}
%
%    \begin{macro}{\HoLogo@MiKTeX}
%    \begin{macrocode}
\def\HoLogo@MiKTeX#1{%
  \HOLOGO@mbox{MiK}%
  \HOLOGO@discretionary
  \hologo{TeX}%
}
%    \end{macrocode}
%    \end{macro}
%    \begin{macro}{\HoLogoHtml@MiKTeX}
%    \begin{macrocode}
\let\HoLogoHtml@MiKTeX\HoLogo@MiKTeX
%    \end{macrocode}
%    \end{macro}
%
% \subsubsection{\hologo{OzTeX} and friends}
%
%    Source: \hologo{OzTeX} FAQ \cite{OzTeX}:
%    \begin{quote}
%      |\def\OzTeX{O\kern-.03em z\kern-.15em\TeX}|\\
%      (There is no kerning in OzMF, OzMP and OzTtH.)
%    \end{quote}
%
%    \begin{macro}{\HoLogo@OzTeX}
%    \begin{macrocode}
\def\HoLogo@OzTeX#1{%
  O%
  \kern-.03em %
  z%
  \kern-.15em %
  \hologo{TeX}%
}
%    \end{macrocode}
%    \end{macro}
%    \begin{macro}{\HoLogoHtml@OzTeX}
%    \begin{macrocode}
\def\HoLogoHtml@OzTeX#1{%
  \HoLogoCss@OzTeX
  \HOLOGO@Span{OzTeX}{%
    O%
    \HOLOGO@Span{z}{z}%
    \hologo{TeX}%
  }%
}
%    \end{macrocode}
%    \end{macro}
%    \begin{macro}{\HoLogoCss@OzTeX}
%    \begin{macrocode}
\def\HoLogoCss@OzTeX{%
  \Css{%
    span.HoLogo-OzTeX span.HoLogo-z{%
      margin-left:-.03em;%
      margin-right:-.15em;%
    }%
  }%
  \global\let\HoLogoCss@OzTeX\relax
}
%    \end{macrocode}
%    \end{macro}
%
%    \begin{macro}{\HoLogo@OzMF}
%    \begin{macrocode}
\def\HoLogo@OzMF#1{%
  \HOLOGO@mbox{OzMF}%
}
%    \end{macrocode}
%    \end{macro}
%    \begin{macro}{\HoLogo@OzMP}
%    \begin{macrocode}
\def\HoLogo@OzMP#1{%
  \HOLOGO@mbox{OzMP}%
}
%    \end{macrocode}
%    \end{macro}
%    \begin{macro}{\HoLogo@OzTtH}
%    \begin{macrocode}
\def\HoLogo@OzTtH#1{%
  \HOLOGO@mbox{OzTtH}%
}
%    \end{macrocode}
%    \end{macro}
%
% \subsubsection{\hologo{PCTeX}}
%
%    \begin{macro}{\HoLogo@PCTeX}
%    \begin{macrocode}
\def\HoLogo@PCTeX#1{%
  \HOLOGO@mbox{PC}%
  \hologo{TeX}%
}
%    \end{macrocode}
%    \end{macro}
%    \begin{macro}{\HoLogoHtml@PCTeX}
%    \begin{macrocode}
\let\HoLogoHtml@PCTeX\HoLogo@PCTeX
%    \end{macrocode}
%    \end{macro}
%
% \subsubsection{\hologo{PiCTeX}}
%
%    The original definitions from \xfile{pictex.tex} \cite{PiCTeX}:
%\begin{quote}
%\begin{verbatim}
%\def\PiC{%
%  P%
%  \kern-.12em%
%  \lower.5ex\hbox{I}%
%  \kern-.075em%
%  C%
%}
%\def\PiCTeX{%
%  \PiC
%  \kern-.11em%
%  \TeX
%}
%\end{verbatim}
%\end{quote}
%
%    \begin{macro}{\HoLogo@PiC}
%    \begin{macrocode}
\def\HoLogo@PiC#1{%
  P%
  \kern-.12em%
  \lower.5ex\hbox{I}%
  \kern-.075em%
  C%
  \HOLOGO@SpaceFactor
}
%    \end{macrocode}
%    \end{macro}
%    \begin{macro}{\HoLogoHtml@PiC}
%    \begin{macrocode}
\def\HoLogoHtml@PiC#1{%
  \HoLogoCss@PiC
  \HOLOGO@Span{PiC}{%
    P%
    \HOLOGO@Span{i}{I}%
    C%
  }%
}
%    \end{macrocode}
%    \end{macro}
%    \begin{macro}{\HoLogoCss@PiC}
%    \begin{macrocode}
\def\HoLogoCss@PiC{%
  \Css{%
    span.HoLogo-PiC span.HoLogo-i{%
      position:relative;%
      top:.5ex;%
      margin-left:-.12em;%
      margin-right:-.075em;%
      text-decoration:none;%
    }%
  }%
  \global\let\HoLogoCss@PiC\relax
}
%    \end{macrocode}
%    \end{macro}
%
%    \begin{macro}{\HoLogo@PiCTeX}
%    \begin{macrocode}
\def\HoLogo@PiCTeX#1{%
  \hologo{PiC}%
  \HOLOGO@discretionary
  \kern-.11em%
  \hologo{TeX}%
}
%    \end{macrocode}
%    \end{macro}
%    \begin{macro}{\HoLogoHtml@PiCTeX}
%    \begin{macrocode}
\def\HoLogoHtml@PiCTeX#1{%
  \HoLogoCss@PiCTeX
  \HOLOGO@Span{PiCTeX}{%
    \hologo{PiC}%
    \hologo{TeX}%
  }%
}
%    \end{macrocode}
%    \end{macro}
%    \begin{macro}{\HoLogoCss@PiCTeX}
%    \begin{macrocode}
\def\HoLogoCss@PiCTeX{%
  \Css{%
    span.HoLogo-PiCTeX span.HoLogo-PiC{%
      margin-right:-.11em;%
    }%
  }%
  \global\let\HoLogoCss@PiCTeX\relax
}
%    \end{macrocode}
%    \end{macro}
%
% \subsubsection{\hologo{teTeX}}
%
%    \begin{macro}{\HoLogo@teTeX}
%    \begin{macrocode}
\def\HoLogo@teTeX#1{%
  \HOLOGO@mbox{#1{t}{T}e}%
  \HOLOGO@discretionary
  \hologo{TeX}%
}
%    \end{macrocode}
%    \end{macro}
%    \begin{macro}{\HoLogoCs@teTeX}
%    \begin{macrocode}
\def\HoLogoCs@teTeX#1{#1{t}{T}dfTeX}
%    \end{macrocode}
%    \end{macro}
%    \begin{macro}{\HoLogoBkm@teTeX}
%    \begin{macrocode}
\def\HoLogoBkm@teTeX#1{%
  #1{t}{T}e\hologo{TeX}%
}
%    \end{macrocode}
%    \end{macro}
%    \begin{macro}{\HoLogoHtml@teTeX}
%    \begin{macrocode}
\let\HoLogoHtml@teTeX\HoLogo@teTeX
%    \end{macrocode}
%    \end{macro}
%
% \subsubsection{\hologo{TeX4ht}}
%
%    \begin{macro}{\HoLogo@TeX4ht}
%    \begin{macrocode}
\expandafter\def\csname HoLogo@TeX4ht\endcsname#1{%
  \HOLOGO@mbox{\hologo{TeX}4ht}%
}
%    \end{macrocode}
%    \end{macro}
%    \begin{macro}{\HoLogoHtml@TeX4ht}
%    \begin{macrocode}
\expandafter
\let\csname HoLogoHtml@TeX4ht\expandafter\endcsname
\csname HoLogo@TeX4ht\endcsname
%    \end{macrocode}
%    \end{macro}
%
%
% \subsubsection{\hologo{SageTeX}}
%
%    \begin{macro}{\HoLogo@SageTeX}
%    \begin{macrocode}
\def\HoLogo@SageTeX#1{%
  \HOLOGO@mbox{Sage}%
  \HOLOGO@discretionary
  \HOLOGO@NegativeKerning{eT,oT,To}%
  \hologo{TeX}%
}
%    \end{macrocode}
%    \end{macro}
%    \begin{macro}{\HoLogoHtml@SageTeX}
%    \begin{macrocode}
\let\HoLogoHtml@SageTeX\HoLogo@SageTeX
%    \end{macrocode}
%    \end{macro}
%
% \subsection{\hologo{METAFONT} and friends}
%
%    \begin{macro}{\HoLogo@METAFONT}
%    \begin{macrocode}
\def\HoLogo@METAFONT#1{%
  \HoLogoFont@font{METAFONT}{logo}{%
    \HOLOGO@mbox{META}%
    \HOLOGO@discretionary
    \HOLOGO@mbox{FONT}%
  }%
}
%    \end{macrocode}
%    \end{macro}
%
%    \begin{macro}{\HoLogo@METAPOST}
%    \begin{macrocode}
\def\HoLogo@METAPOST#1{%
  \HoLogoFont@font{METAPOST}{logo}{%
    \HOLOGO@mbox{META}%
    \HOLOGO@discretionary
    \HOLOGO@mbox{POST}%
  }%
}
%    \end{macrocode}
%    \end{macro}
%
%    \begin{macro}{\HoLogo@MetaFun}
%    \begin{macrocode}
\def\HoLogo@MetaFun#1{%
  \HOLOGO@mbox{Meta}%
  \HOLOGO@discretionary
  \HOLOGO@mbox{Fun}%
}
%    \end{macrocode}
%    \end{macro}
%
%    \begin{macro}{\HoLogo@MetaPost}
%    \begin{macrocode}
\def\HoLogo@MetaPost#1{%
  \HOLOGO@mbox{Meta}%
  \HOLOGO@discretionary
  \HOLOGO@mbox{Post}%
}
%    \end{macrocode}
%    \end{macro}
%
% \subsection{Others}
%
% \subsubsection{\hologo{biber}}
%
%    \begin{macro}{\HoLogo@biber}
%    \begin{macrocode}
\def\HoLogo@biber#1{%
  \HOLOGO@mbox{#1{b}{B}i}%
  \HOLOGO@discretionary
  \HOLOGO@mbox{ber}%
}
%    \end{macrocode}
%    \end{macro}
%    \begin{macro}{\HoLogoCs@biber}
%    \begin{macrocode}
\def\HoLogoCs@biber#1{#1{b}{B}iber}
%    \end{macrocode}
%    \end{macro}
%    \begin{macro}{\HoLogoBkm@biber}
%    \begin{macrocode}
\def\HoLogoBkm@biber#1{%
  #1{b}{B}iber%
}
%    \end{macrocode}
%    \end{macro}
%    \begin{macro}{\HoLogoHtml@biber}
%    \begin{macrocode}
\let\HoLogoHtml@biber\HoLogo@biber
%    \end{macrocode}
%    \end{macro}
%
% \subsubsection{\hologo{KOMAScript}}
%
%    \begin{macro}{\HoLogo@KOMAScript}
%    The definition for \hologo{KOMAScript} is taken
%    from \hologo{KOMAScript} (\xfile{scrlogo.dtx}, reformatted) \cite{scrlogo}:
%\begin{quote}
%\begin{verbatim}
%\@ifundefined{KOMAScript}{%
%  \DeclareRobustCommand{\KOMAScript}{%
%    \textsf{%
%      K\kern.05em O\kern.05emM\kern.05em A%
%      \kern.1em-\kern.1em %
%      Script%
%    }%
%  }%
%}{}
%\end{verbatim}
%\end{quote}
%    \begin{macrocode}
\def\HoLogo@KOMAScript#1{%
  \HoLogoFont@font{KOMAScript}{sf}{%
    \HOLOGO@mbox{%
      K\kern.05em%
      O\kern.05em%
      M\kern.05em%
      A%
    }%
    \kern.1em%
    \HOLOGO@hyphen
    \kern.1em%
    \HOLOGO@mbox{Script}%
  }%
}
%    \end{macrocode}
%    \end{macro}
%    \begin{macro}{\HoLogoBkm@KOMAScript}
%    \begin{macrocode}
\def\HoLogoBkm@KOMAScript#1{%
  KOMA-Script%
}
%    \end{macrocode}
%    \end{macro}
%    \begin{macro}{\HoLogoHtml@KOMAScript}
%    \begin{macrocode}
\def\HoLogoHtml@KOMAScript#1{%
  \HoLogoCss@KOMAScript
  \HoLogoFont@font{KOMAScript}{sf}{%
    \HOLOGO@Span{KOMAScript}{%
      K%
      \HOLOGO@Span{O}{O}%
      M%
      \HOLOGO@Span{A}{A}%
      \HOLOGO@Span{hyphen}{-}%
      Script%
    }%
  }%
}
%    \end{macrocode}
%    \end{macro}
%    \begin{macro}{\HoLogoCss@KOMAScript}
%    \begin{macrocode}
\def\HoLogoCss@KOMAScript{%
  \Css{%
    span.HoLogo-KOMAScript{%
      font-family:sans-serif;%
    }%
  }%
  \Css{%
    span.HoLogo-KOMAScript span.HoLogo-O{%
      padding-left:.05em;%
      padding-right:.05em;%
    }%
  }%
  \Css{%
    span.HoLogo-KOMAScript span.HoLogo-A{%
      padding-left:.05em;%
    }%
  }%
  \Css{%
    span.HoLogo-KOMAScript span.HoLogo-hyphen{%
      padding-left:.1em;%
      padding-right:.1em;%
    }%
  }%
  \global\let\HoLogoCss@KOMAScript\relax
}
%    \end{macrocode}
%    \end{macro}
%
% \subsubsection{\hologo{LyX}}
%
%    \begin{macro}{\HoLogo@LyX}
%    The definition is taken from the documentation source files
%    of \hologo{LyX}, \xfile{Intro.lyx} \cite{LyX}:
%\begin{quote}
%\begin{verbatim}
%\def\LyX{%
%  \texorpdfstring{%
%    L\kern-.1667em\lower.25em\hbox{Y}\kern-.125emX\@%
%  }{%
%    LyX%
%  }%
%}
%\end{verbatim}
%\end{quote}
%    \begin{macrocode}
\def\HoLogo@LyX#1{%
  L%
  \kern-.1667em%
  \lower.25em\hbox{Y}%
  \kern-.125em%
  X%
  \HOLOGO@SpaceFactor
}
%    \end{macrocode}
%    \end{macro}
%    \begin{macro}{\HoLogoHtml@LyX}
%    \begin{macrocode}
\def\HoLogoHtml@LyX#1{%
  \HoLogoCss@LyX
  \HOLOGO@Span{LyX}{%
    L%
    \HOLOGO@Span{y}{Y}%
    X%
  }%
}
%    \end{macrocode}
%    \end{macro}
%    \begin{macro}{\HoLogoCss@LyX}
%    \begin{macrocode}
\def\HoLogoCss@LyX{%
  \Css{%
    span.HoLogo-LyX span.HoLogo-y{%
      position:relative;%
      top:.25em;%
      margin-left:-.1667em;%
      margin-right:-.125em;%
      text-decoration:none;%
    }%
  }%
  \global\let\HoLogoCss@LyX\relax
}
%    \end{macrocode}
%    \end{macro}
%
% \subsubsection{\hologo{NTS}}
%
%    \begin{macro}{\HoLogo@NTS}
%    Definition for \hologo{NTS} can be found in
%    package \xpackage{etex\textunderscore man} for the \hologo{eTeX} manual \cite{etexman}
%    and in package \xpackage{dtklogos} \cite{dtklogos}:
%\begin{quote}
%\begin{verbatim}
%\def\NTS{%
%  \leavevmode
%  \hbox{%
%    $%
%      \cal N%
%      \kern-0.35em%
%      \lower0.5ex\hbox{$\cal T$}%
%      \kern-0.2em%
%      S%
%    $%
%  }%
%}
%\end{verbatim}
%\end{quote}
%    \begin{macrocode}
\def\HoLogo@NTS#1{%
  \HoLogoFont@font{NTS}{sy}{%
    N\/%
    \kern-.35em%
    \lower.5ex\hbox{T\/}%
    \kern-.2em%
    S\/%
  }%
  \HOLOGO@SpaceFactor
}
%    \end{macrocode}
%    \end{macro}
%
% \subsubsection{\Hologo{TTH} (\hologo{TeX} to HTML translator)}
%
%    Source: \url{http://hutchinson.belmont.ma.us/tth/}
%    In the HTML source the second `T' is printed as subscript.
%\begin{quote}
%\begin{verbatim}
%T<sub>T</sub>H
%\end{verbatim}
%\end{quote}
%    \begin{macro}{\HoLogo@TTH}
%    \begin{macrocode}
\def\HoLogo@TTH#1{%
  \ltx@mbox{%
    T\HOLOGO@SubScript{T}H%
  }%
  \HOLOGO@SpaceFactor
}
%    \end{macrocode}
%    \end{macro}
%
%    \begin{macro}{\HoLogoHtml@TTH}
%    \begin{macrocode}
\def\HoLogoHtml@TTH#1{%
  T\HCode{<sub>}T\HCode{</sub>}H%
}
%    \end{macrocode}
%    \end{macro}
%
% \subsubsection{\Hologo{HanTheThanh}}
%
%    Partial source: Package \xpackage{dtklogos}.
%    The double accent is U+1EBF (latin small letter e with circumflex
%    and acute).
%    \begin{macro}{\HoLogo@HanTheThanh}
%    \begin{macrocode}
\def\HoLogo@HanTheThanh#1{%
  \ltx@mbox{H\`an}%
  \HOLOGO@space
  \ltx@mbox{%
    Th%
    \HOLOGO@IfCharExists{"1EBF}{%
      \char"1EBF\relax
    }{%
      \^e\hbox to 0pt{\hss\raise .5ex\hbox{\'{}}}%
    }%
  }%
  \HOLOGO@space
  \ltx@mbox{Th\`anh}%
}
%    \end{macrocode}
%    \end{macro}
%    \begin{macro}{\HoLogoBkm@HanTheThanh}
%    \begin{macrocode}
\def\HoLogoBkm@HanTheThanh#1{%
  H\`an %
  Th\HOLOGO@PdfdocUnicode{\^e}{\9036\277} %
  Th\`anh%
}
%    \end{macrocode}
%    \end{macro}
%    \begin{macro}{\HoLogoHtml@HanTheThanh}
%    \begin{macrocode}
\def\HoLogoHtml@HanTheThanh#1{%
  H\`an %
  Th\HCode{&\ltx@hashchar x1ebf;} %
  Th\`anh%
}
%    \end{macrocode}
%    \end{macro}
%
% \subsection{Driver detection}
%
%    \begin{macrocode}
\HOLOGO@IfExists\InputIfFileExists{%
  \InputIfFileExists{hologo.cfg}{}{}%
}{%
  \ltx@IfUndefined{pdf@filesize}{%
    \def\HOLOGO@InputIfExists{%
      \openin\HOLOGO@temp=hologo.cfg\relax
      \ifeof\HOLOGO@temp
        \closein\HOLOGO@temp
      \else
        \closein\HOLOGO@temp
        \begingroup
          \def\x{LaTeX2e}%
        \expandafter\endgroup
        \ifx\fmtname\x
          \input{hologo.cfg}%
        \else
          \input hologo.cfg\relax
        \fi
      \fi
    }%
    \ltx@IfUndefined{newread}{%
      \chardef\HOLOGO@temp=15 %
      \def\HOLOGO@CheckRead{%
        \ifeof\HOLOGO@temp
          \HOLOGO@InputIfExists
        \else
          \ifcase\HOLOGO@temp
            \@PackageWarningNoLine{hologo}{%
              Configuration file ignored, because\MessageBreak
              a free read register could not be found%
            }%
          \else
            \begingroup
              \count\ltx@cclv=\HOLOGO@temp
              \advance\ltx@cclv by \ltx@minusone
              \edef\x{\endgroup
                \chardef\noexpand\HOLOGO@temp=\the\count\ltx@cclv
                \relax
              }%
            \x
          \fi
        \fi
      }%
    }{%
      \csname newread\endcsname\HOLOGO@temp
      \HOLOGO@InputIfExists
    }%
  }{%
    \edef\HOLOGO@temp{\pdf@filesize{hologo.cfg}}%
    \ifx\HOLOGO@temp\ltx@empty
    \else
      \ifnum\HOLOGO@temp>0 %
        \begingroup
          \def\x{LaTeX2e}%
        \expandafter\endgroup
        \ifx\fmtname\x
          \input{hologo.cfg}%
        \else
          \input hologo.cfg\relax
        \fi
      \else
        \@PackageInfoNoLine{hologo}{%
          Empty configuration file `hologo.cfg' ignored%
        }%
      \fi
    \fi
  }%
}
%    \end{macrocode}
%
%    \begin{macrocode}
\def\HOLOGO@temp#1#2{%
  \kv@define@key{HoLogoDriver}{#1}[]{%
    \begingroup
      \def\HOLOGO@temp{##1}%
      \ltx@onelevel@sanitize\HOLOGO@temp
      \ifx\HOLOGO@temp\ltx@empty
      \else
        \@PackageError{hologo}{%
          Value (\HOLOGO@temp) not permitted for option `#1'%
        }%
        \@ehc
      \fi
    \endgroup
    \def\hologoDriver{#2}%
  }%
}%
\def\HOLOGO@@temp#1#2{%
  \ifx\kv@value\relax
    \HOLOGO@temp{#1}{#1}%
  \else
    \HOLOGO@temp{#1}{#2}%
  \fi
}%
\kv@parse@normalized{%
  pdftex,%
  luatex=pdftex,%
  dvipdfm,%
  dvipdfmx=dvipdfm,%
  dvips,%
  dvipsone=dvips,%
  xdvi=dvips,%
  xetex,%
  vtex,%
}\HOLOGO@@temp
%    \end{macrocode}
%
%    \begin{macrocode}
\kv@define@key{HoLogoDriver}{driverfallback}{%
  \def\HOLOGO@DriverFallback{#1}%
}
%    \end{macrocode}
%
%    \begin{macro}{\HOLOGO@DriverFallback}
%    \begin{macrocode}
\def\HOLOGO@DriverFallback{dvips}
%    \end{macrocode}
%    \end{macro}
%
%    \begin{macro}{\hologoDriverSetup}
%    \begin{macrocode}
\def\hologoDriverSetup{%
  \let\hologoDriver\ltx@undefined
  \HOLOGO@DriverSetup
}
%    \end{macrocode}
%    \end{macro}
%
%    \begin{macro}{\HOLOGO@DriverSetup}
%    \begin{macrocode}
\def\HOLOGO@DriverSetup#1{%
  \kvsetkeys{HoLogoDriver}{#1}%
  \HOLOGO@CheckDriver
  \ltx@ifundefined{hologoDriver}{%
    \begingroup
    \edef\x{\endgroup
      \noexpand\kvsetkeys{HoLogoDriver}{\HOLOGO@DriverFallback}%
    }\x
  }{}%
  \@PackageInfoNoLine{hologo}{Using driver `\hologoDriver'}%
}
%    \end{macrocode}
%    \end{macro}
%
%    \begin{macro}{\HOLOGO@CheckDriver}
%    \begin{macrocode}
\def\HOLOGO@CheckDriver{%
  \ifpdf
    \def\hologoDriver{pdftex}%
    \let\HOLOGO@pdfliteral\pdfliteral
    \ifluatex
      \ifx\pdfextension\@undefined\else
        \protected\def\pdfliteral{\pdfextension literal}%
        \let\HOLOGO@pdfliteral\pdfliteral
      \fi
      \ltx@IfUndefined{HOLOGO@pdfliteral}{%
        \ifnum\luatexversion<36 %
        \else
          \begingroup
            \let\HOLOGO@temp\endgroup
            \ifcase0%
                \directlua{%
                  if tex.enableprimitives then %
                    tex.enableprimitives('HOLOGO@', {'pdfliteral'})%
                  else %
                    tex.print('1')%
                  end%
                }%
                \ifx\HOLOGO@pdfliteral\@undefined 1\fi%
                \relax%
              \endgroup
              \let\HOLOGO@temp\relax
              \global\let\HOLOGO@pdfliteral\HOLOGO@pdfliteral
            \fi%
          \HOLOGO@temp
        \fi
      }{}%
    \fi
    \ltx@IfUndefined{HOLOGO@pdfliteral}{%
      \@PackageWarningNoLine{hologo}{%
        Cannot find \string\pdfliteral
      }%
    }{}%
  \else
    \ifxetex
      \def\hologoDriver{xetex}%
    \else
      \ifvtex
        \def\hologoDriver{vtex}%
      \fi
    \fi
  \fi
}
%    \end{macrocode}
%    \end{macro}
%
%    \begin{macro}{\HOLOGO@WarningUnsupportedDriver}
%    \begin{macrocode}
\def\HOLOGO@WarningUnsupportedDriver#1{%
  \@PackageWarningNoLine{hologo}{%
    Logo `#1' needs driver specific macros,\MessageBreak
    but driver `\hologoDriver' is not supported.\MessageBreak
    Use a different driver or\MessageBreak
    load package `graphics' or `pgf'%
  }%
}
%    \end{macrocode}
%    \end{macro}
%
% \subsubsection{Reflect box macros}
%
%    Skip driver part if not needed.
%    \begin{macrocode}
\ltx@IfUndefined{reflectbox}{}{%
  \ltx@IfUndefined{rotatebox}{}{%
    \HOLOGO@AtEnd
  }%
}
\ltx@IfUndefined{pgftext}{}{%
  \HOLOGO@AtEnd
}
\ltx@IfUndefined{psscalebox}{}{%
  \HOLOGO@AtEnd
}
%    \end{macrocode}
%
%    \begin{macrocode}
\def\HOLOGO@temp{LaTeX2e}
\ifx\fmtname\HOLOGO@temp
  \RequirePackage{kvoptions}[2011/06/30]%
  \ProcessKeyvalOptions{HoLogoDriver}%
\fi
\HOLOGO@DriverSetup{}
%    \end{macrocode}
%
%    \begin{macro}{\HOLOGO@ReflectBox}
%    \begin{macrocode}
\def\HOLOGO@ReflectBox#1{%
  \begingroup
    \setbox\ltx@zero\hbox{\begingroup#1\endgroup}%
    \setbox\ltx@two\hbox{%
      \kern\wd\ltx@zero
      \csname HOLOGO@ScaleBox@\hologoDriver\endcsname{-1}{1}{%
        \hbox to 0pt{\copy\ltx@zero\hss}%
      }%
    }%
    \wd\ltx@two=\wd\ltx@zero
    \box\ltx@two
  \endgroup
}
%    \end{macrocode}
%    \end{macro}
%
%    \begin{macro}{\HOLOGO@PointReflectBox}
%    \begin{macrocode}
\def\HOLOGO@PointReflectBox#1{%
  \begingroup
    \setbox\ltx@zero\hbox{\begingroup#1\endgroup}%
    \setbox\ltx@two\hbox{%
      \kern\wd\ltx@zero
      \raise\ht\ltx@zero\hbox{%
        \csname HOLOGO@ScaleBox@\hologoDriver\endcsname{-1}{-1}{%
          \hbox to 0pt{\copy\ltx@zero\hss}%
        }%
      }%
    }%
    \wd\ltx@two=\wd\ltx@zero
    \box\ltx@two
  \endgroup
}
%    \end{macrocode}
%    \end{macro}
%
%    We must define all variants because of dynamic driver setup.
%    \begin{macrocode}
\def\HOLOGO@temp#1#2{#2}
%    \end{macrocode}
%
%    \begin{macro}{\HOLOGO@ScaleBox@pdftex}
%    \begin{macrocode}
\HOLOGO@temp{pdftex}{%
  \def\HOLOGO@ScaleBox@pdftex#1#2#3{%
    \HOLOGO@pdfliteral{%
      q #1 0 0 #2 0 0 cm%
    }%
    #3%
    \HOLOGO@pdfliteral{%
      Q%
    }%
  }%
}
%    \end{macrocode}
%    \end{macro}
%    \begin{macro}{\HOLOGO@ScaleBox@dvips}
%    \begin{macrocode}
\HOLOGO@temp{dvips}{%
  \def\HOLOGO@ScaleBox@dvips#1#2#3{%
    \special{ps:%
      gsave %
      currentpoint %
      currentpoint translate %
      #1 #2 scale %
      neg exch neg exch translate%
    }%
    #3%
    \special{ps:%
      currentpoint %
      grestore %
      moveto%
    }%
  }%
}
%    \end{macrocode}
%    \end{macro}
%    \begin{macro}{\HOLOGO@ScaleBox@dvipdfm}
%    \begin{macrocode}
\HOLOGO@temp{dvipdfm}{%
  \let\HOLOGO@ScaleBox@dvipdfm\HOLOGO@ScaleBox@dvips
}
%    \end{macrocode}
%    \end{macro}
%    Since \hologo{XeTeX} v0.6.
%    \begin{macro}{\HOLOGO@ScaleBox@xetex}
%    \begin{macrocode}
\HOLOGO@temp{xetex}{%
  \def\HOLOGO@ScaleBox@xetex#1#2#3{%
    \special{x:gsave}%
    \special{x:scale #1 #2}%
    #3%
    \special{x:grestore}%
  }%
}
%    \end{macrocode}
%    \end{macro}
%    \begin{macro}{\HOLOGO@ScaleBox@vtex}
%    \begin{macrocode}
\HOLOGO@temp{vtex}{%
  \def\HOLOGO@ScaleBox@vtex#1#2#3{%
    \special{r(#1,0,0,#2,0,0}%
    #3%
    \special{r)}%
  }%
}
%    \end{macrocode}
%    \end{macro}
%
%    \begin{macrocode}
\HOLOGO@AtEnd%
%</package>
%    \end{macrocode}
%
% \section{Test}
%
% \subsection{Catcode checks for loading}
%
%    \begin{macrocode}
%<*test1>
%    \end{macrocode}
%    \begin{macrocode}
\catcode`\{=1 %
\catcode`\}=2 %
\catcode`\#=6 %
\catcode`\@=11 %
\expandafter\ifx\csname count@\endcsname\relax
  \countdef\count@=255 %
\fi
\expandafter\ifx\csname @gobble\endcsname\relax
  \long\def\@gobble#1{}%
\fi
\expandafter\ifx\csname @firstofone\endcsname\relax
  \long\def\@firstofone#1{#1}%
\fi
\expandafter\ifx\csname loop\endcsname\relax
  \expandafter\@firstofone
\else
  \expandafter\@gobble
\fi
{%
  \def\loop#1\repeat{%
    \def\body{#1}%
    \iterate
  }%
  \def\iterate{%
    \body
      \let\next\iterate
    \else
      \let\next\relax
    \fi
    \next
  }%
  \let\repeat=\fi
}%
\def\RestoreCatcodes{}
\count@=0 %
\loop
  \edef\RestoreCatcodes{%
    \RestoreCatcodes
    \catcode\the\count@=\the\catcode\count@\relax
  }%
\ifnum\count@<255 %
  \advance\count@ 1 %
\repeat

\def\RangeCatcodeInvalid#1#2{%
  \count@=#1\relax
  \loop
    \catcode\count@=15 %
  \ifnum\count@<#2\relax
    \advance\count@ 1 %
  \repeat
}
\def\RangeCatcodeCheck#1#2#3{%
  \count@=#1\relax
  \loop
    \ifnum#3=\catcode\count@
    \else
      \errmessage{%
        Character \the\count@\space
        with wrong catcode \the\catcode\count@\space
        instead of \number#3%
      }%
    \fi
  \ifnum\count@<#2\relax
    \advance\count@ 1 %
  \repeat
}
\def\space{ }
\expandafter\ifx\csname LoadCommand\endcsname\relax
  \def\LoadCommand{\input hologo.sty\relax}%
\fi
\def\Test{%
  \RangeCatcodeInvalid{0}{47}%
  \RangeCatcodeInvalid{58}{64}%
  \RangeCatcodeInvalid{91}{96}%
  \RangeCatcodeInvalid{123}{255}%
  \catcode`\@=12 %
  \catcode`\\=0 %
  \catcode`\%=14 %
  \LoadCommand
  \RangeCatcodeCheck{0}{36}{15}%
  \RangeCatcodeCheck{37}{37}{14}%
  \RangeCatcodeCheck{38}{47}{15}%
  \RangeCatcodeCheck{48}{57}{12}%
  \RangeCatcodeCheck{58}{63}{15}%
  \RangeCatcodeCheck{64}{64}{12}%
  \RangeCatcodeCheck{65}{90}{11}%
  \RangeCatcodeCheck{91}{91}{15}%
  \RangeCatcodeCheck{92}{92}{0}%
  \RangeCatcodeCheck{93}{96}{15}%
  \RangeCatcodeCheck{97}{122}{11}%
  \RangeCatcodeCheck{123}{255}{15}%
  \RestoreCatcodes
}
\Test
\csname @@end\endcsname
\end
%    \end{macrocode}
%    \begin{macrocode}
%</test1>
%    \end{macrocode}
%
% \subsection{Spacefactor}
%
%    The space factor must be 1000 after a logo. If it is greater 1000
%    then the following space is a space after a sentence closing point.
%    If the space factor is smaller 1000 then an immediate following
%    dot is interpreted as abbreviation, not sentence closing point.
%
%    \begin{macrocode}
%<*test-spacefactor>
\NeedsTeXFormat{LaTeX2e}
\documentclass{article}
\usepackage{hologo}[2016/05/12]
\usepackage{kvsetkeys}
\usepackage{qstest}
\IncludeTests{*}
\LogTests{log}{*}{*}
\begin{document}
\begin{qstest}{spacefactor}{spacefactor}
\newcommand*{\Test}[1]{%
  \sbox0{%
    \hologo{#1}%
    \Expect*{1000 (#1)}*{\the\spacefactor\space(#1)}%
  }%
}%
\makeatletter
\def\TestList{}
\def\hologoEntry#1#2#3{%
  \edef\TestList{%
    \ifx\TestList\@empty
    \else
      \TestList,%
    \fi
    #1%
    \ifx\\#2\\%
    \else
      ={variant=#2}%
    \fi
  }%
}
\hologoList
\expandafter\kv@parse@normalized\expandafter{%
  \TestList
}{%
  \begingroup
    \let\@logo=\kv@key
    \ifx\kv@value\relax
    \else
      \expandafter\hologoLogoSetup\expandafter\@logo\expandafter{%
        \kv@value
      }%
    \fi
    \Test\@logo
  \endgroup
  \@gobbletwo
}
\end{qstest}
\end{document}
%</test-spacefactor>
%    \end{macrocode}
%
% \subsection{Complete list}
%
%    \begin{macrocode}
%<*test-list>
\NeedsTeXFormat{LaTeX2e}
\documentclass[12pt,a4paper]{article}
\usepackage{hologo}[2016/05/12]
\usepackage[T1]{fontenc}
\usepackage{lmodern}
\usepackage{parskip}
\usepackage[unicode]{hyperref}[2011/09/28]
\usepackage{bookmark}[2011/09/19]
\bookmarksetup{%
  numbered,%
  open,%
  openlevel=2,%
}
\renewcommand*{\contentsname}{List of logos}
\begin{document}
\tableofcontents
\def\TestFont#1#2#3#4#5#6{%
  \begingroup
    \usefont{#3}{#4}{#5}{#6}%
    \HologoVariant{#1}{#2}/\hologoVariant{#1}{#2}%
    \quad
    \begingroup\scriptsize\hologoVariant{#1}{#2}\endgroup
    \quad
  \endgroup
  (#3/#4/#5/#6)%
  \par
}
\makeatletter
\def\hologoEntry#1#2#3{%
  \section{%
    \HologoVariant{#1}{#2}/\hologoVariant{#1}{#2} %
    {[#1\ifx\\#2\\\else\space(#2)\fi]}% hash-ok
  }% braces around [] because of bug in tex4ht
  \begingroup
    \hypersetup{unicode=false}%
    \bookmark[%
      dest=\@currentHref,%
      rellevel=1,%
      keeplevel,%
    ]{%
      \HologoVariant{#1}{#2}/\hologoVariant{#1}{#2} %
      (PDFDocEncoding)%
    }%
  \endgroup
  \TestFont{#1}{#2}{OT1}{cmr}{m}{n}%
  \TestFont{#1}{#2}{OT1}{cmss}{m}{n}%
  \TestFont{#1}{#2}{OT1}{cmr}{b}{n}%
  \TestFont{#1}{#2}{OT1}{cmr}{m}{it}%
  \TestFont{#1}{#2}{OT1}{cmtt}{m}{n}%
  \TestFont{#1}{#2}{T1}{lmr}{m}{n}%
  \TestFont{#1}{#2}{T1}{lmss}{m}{n}%
  \TestFont{#1}{#2}{T1}{lmr}{b}{n}%
  \TestFont{#1}{#2}{T1}{lmr}{m}{it}%
  \TestFont{#1}{#2}{T1}{lmtt}{m}{n}%
  \TestFont{#1}{#2}{T1}{lmvtt}{m}{n}%
  \TestFont{#1}{#2}{T1}{qtm}{m}{n}%
  \TestFont{#1}{#2}{T1}{qhv}{m}{n}%
  \TestFont{#1}{#2}{T1}{qtm}{b}{n}%
  \TestFont{#1}{#2}{T1}{qtm}{m}{it}%
  \TestFont{#1}{#2}{T1}{qcr}{m}{n}%
  \newpage
}
\makeatother
\hologoList
\end{document}
%</test-list>
%    \end{macrocode}
%
% \section{Installation}
%
% \subsection{Download}
%
% \paragraph{Package.} This package is available on
% CTAN\footnote{\url{ftp://ftp.ctan.org/tex-archive/}}:
% \begin{description}
% \item[\CTAN{macros/latex/contrib/oberdiek/hologo.dtx}] The source file.
% \item[\CTAN{macros/latex/contrib/oberdiek/hologo.pdf}] Documentation.
% \end{description}
%
%
% \paragraph{Bundle.} All the packages of the bundle `oberdiek'
% are also available in a TDS compliant ZIP archive. There
% the packages are already unpacked and the documentation files
% are generated. The files and directories obey the TDS standard.
% \begin{description}
% \item[\CTAN{install/macros/latex/contrib/oberdiek.tds.zip}]
% \end{description}
% \emph{TDS} refers to the standard ``A Directory Structure
% for \TeX\ Files'' (\CTAN{tds/tds.pdf}). Directories
% with \xfile{texmf} in their name are usually organized this way.
%
% \subsection{Bundle installation}
%
% \paragraph{Unpacking.} Unpack the \xfile{oberdiek.tds.zip} in the
% TDS tree (also known as \xfile{texmf} tree) of your choice.
% Example (linux):
% \begin{quote}
%   |unzip oberdiek.tds.zip -d ~/texmf|
% \end{quote}
%
% \paragraph{Script installation.}
% Check the directory \xfile{TDS:scripts/oberdiek/} for
% scripts that need further installation steps.
% Package \xpackage{attachfile2} comes with the Perl script
% \xfile{pdfatfi.pl} that should be installed in such a way
% that it can be called as \texttt{pdfatfi}.
% Example (linux):
% \begin{quote}
%   |chmod +x scripts/oberdiek/pdfatfi.pl|\\
%   |cp scripts/oberdiek/pdfatfi.pl /usr/local/bin/|
% \end{quote}
%
% \subsection{Package installation}
%
% \paragraph{Unpacking.} The \xfile{.dtx} file is a self-extracting
% \docstrip\ archive. The files are extracted by running the
% \xfile{.dtx} through \plainTeX:
% \begin{quote}
%   \verb|tex hologo.dtx|
% \end{quote}
%
% \paragraph{TDS.} Now the different files must be moved into
% the different directories in your installation TDS tree
% (also known as \xfile{texmf} tree):
% \begin{quote}
% \def\t{^^A
% \begin{tabular}{@{}>{\ttfamily}l@{ $\rightarrow$ }>{\ttfamily}l@{}}
%   hologo.sty & tex/generic/oberdiek/hologo.sty\\
%   hologo.pdf & doc/latex/oberdiek/hologo.pdf\\
%   example/hologo-example.tex & doc/latex/oberdiek/example/hologo-example.tex\\
%   test/hologo-test1.tex & doc/latex/oberdiek/test/hologo-test1.tex\\
%   test/hologo-test-spacefactor.tex & doc/latex/oberdiek/test/hologo-test-spacefactor.tex\\
%   test/hologo-test-list.tex & doc/latex/oberdiek/test/hologo-test-list.tex\\
%   hologo.dtx & source/latex/oberdiek/hologo.dtx\\
% \end{tabular}^^A
% }^^A
% \sbox0{\t}^^A
% \ifdim\wd0>\linewidth
%   \begingroup
%     \advance\linewidth by\leftmargin
%     \advance\linewidth by\rightmargin
%   \edef\x{\endgroup
%     \def\noexpand\lw{\the\linewidth}^^A
%   }\x
%   \def\lwbox{^^A
%     \leavevmode
%     \hbox to \linewidth{^^A
%       \kern-\leftmargin\relax
%       \hss
%       \usebox0
%       \hss
%       \kern-\rightmargin\relax
%     }^^A
%   }^^A
%   \ifdim\wd0>\lw
%     \sbox0{\small\t}^^A
%     \ifdim\wd0>\linewidth
%       \ifdim\wd0>\lw
%         \sbox0{\footnotesize\t}^^A
%         \ifdim\wd0>\linewidth
%           \ifdim\wd0>\lw
%             \sbox0{\scriptsize\t}^^A
%             \ifdim\wd0>\linewidth
%               \ifdim\wd0>\lw
%                 \sbox0{\tiny\t}^^A
%                 \ifdim\wd0>\linewidth
%                   \lwbox
%                 \else
%                   \usebox0
%                 \fi
%               \else
%                 \lwbox
%               \fi
%             \else
%               \usebox0
%             \fi
%           \else
%             \lwbox
%           \fi
%         \else
%           \usebox0
%         \fi
%       \else
%         \lwbox
%       \fi
%     \else
%       \usebox0
%     \fi
%   \else
%     \lwbox
%   \fi
% \else
%   \usebox0
% \fi
% \end{quote}
% If you have a \xfile{docstrip.cfg} that configures and enables \docstrip's
% TDS installing feature, then some files can already be in the right
% place, see the documentation of \docstrip.
%
% \subsection{Refresh file name databases}
%
% If your \TeX~distribution
% (\teTeX, \mikTeX, \dots) relies on file name databases, you must refresh
% these. For example, \teTeX\ users run \verb|texhash| or
% \verb|mktexlsr|.
%
% \subsection{Some details for the interested}
%
% \paragraph{Attached source.}
%
% The PDF documentation on CTAN also includes the
% \xfile{.dtx} source file. It can be extracted by
% AcrobatReader 6 or higher. Another option is \textsf{pdftk},
% e.g. unpack the file into the current directory:
% \begin{quote}
%   \verb|pdftk hologo.pdf unpack_files output .|
% \end{quote}
%
% \paragraph{Unpacking with \LaTeX.}
% The \xfile{.dtx} chooses its action depending on the format:
% \begin{description}
% \item[\plainTeX:] Run \docstrip\ and extract the files.
% \item[\LaTeX:] Generate the documentation.
% \end{description}
% If you insist on using \LaTeX\ for \docstrip\ (really,
% \docstrip\ does not need \LaTeX), then inform the autodetect routine
% about your intention:
% \begin{quote}
%   \verb|latex \let\install=y\input{hologo.dtx}|
% \end{quote}
% Do not forget to quote the argument according to the demands
% of your shell.
%
% \paragraph{Generating the documentation.}
% You can use both the \xfile{.dtx} or the \xfile{.drv} to generate
% the documentation. The process can be configured by the
% configuration file \xfile{ltxdoc.cfg}. For instance, put this
% line into this file, if you want to have A4 as paper format:
% \begin{quote}
%   \verb|\PassOptionsToClass{a4paper}{article}|
% \end{quote}
% An example follows how to generate the
% documentation with pdf\LaTeX:
% \begin{quote}
%\begin{verbatim}
%pdflatex hologo.dtx
%makeindex -s gind.ist hologo.idx
%pdflatex hologo.dtx
%makeindex -s gind.ist hologo.idx
%pdflatex hologo.dtx
%\end{verbatim}
% \end{quote}
%
% \section{Catalogue}
%
% The following XML file can be used as source for the
% \href{http://mirror.ctan.org/help/Catalogue/catalogue.html}{\TeX\ Catalogue}.
% The elements \texttt{caption} and \texttt{description} are imported
% from the original XML file from the Catalogue.
% The name of the XML file in the Catalogue is \xfile{hologo.xml}.
%    \begin{macrocode}
%<*catalogue>
<?xml version='1.0' encoding='us-ascii'?>
<!DOCTYPE entry SYSTEM 'catalogue.dtd'>
<entry datestamp='$Date$' modifier='$Author$' id='hologo'>
  <name>hologo</name>
  <caption>A collection of logos with bookmark support.</caption>
  <authorref id='auth:oberdiek'/>
  <copyright owner='Heiko Oberdiek' year='2010-2012'/>
  <license type='lppl1.3'/>
  <version number='1.10'/>
  <description>
    The package defines a single command <tt>\hologo</tt>, whose
    argument is the usual case-confused ASCII version of the logo.
    The command is bookmark-enabled, so that every logo becomes
    available in bookmarks without further work.
    <p/>
    The package is part of the <xref refid='oberdiek'>oberdiek</xref>
    bundle.
  </description>
  <documentation details='Package documentation'
      href='ctan:/macros/latex/contrib/oberdiek/hologo.pdf'/>
  <ctan file='true' path='/macros/latex/contrib/oberdiek/hologo.dtx'/>
  <miktex location='oberdiek'/>
  <texlive location='oberdiek'/>
  <install path='/macros/latex/contrib/oberdiek/oberdiek.tds.zip'/>
</entry>
%</catalogue>
%    \end{macrocode}
%
% \begin{thebibliography}{9}
% \raggedright
%
% \bibitem{btxdoc}
% Oren Patashnik,
% \textit{\hologo{BibTeX}ing},
% 1988-02-08.\\
% \CTAN{biblio/bibtex/base/}
%
% \bibitem{dtklogos}
% Gerd Neugebauer, DANTE,
% \textit{Package \xpackage{dtklogos}},
% 2011-04-25.\\
% \CTAN{usergrps/dante/dtk/dtklogos.sty}
%
% \bibitem{etexman}
% The \hologo{NTS} Team,
% \textit{The \hologo{eTeX} manual},
% 1998-02.\\
% \CTAN{systems/e-tex/v2/doc/}
%
% \bibitem{ExTeX-FAQ}
% The \hologo{ExTeX} group,
% \textit{\hologo{ExTeX}: FAQ -- How is \hologo{ExTeX} typeset?},
% 2007-04-14.\\
% \url{http://www.extex.org/documentation/faq.html}
%
% \bibitem{LyX}
% %@MISC{ LyX,
% %  title = {{LyX 2.0.0 -- The Document Processor [Computer software and manual]}},
% %  author = {{The LyX Team}},
% %  howpublished = {Internet: http://www.lyx.org},
% %  year = {2011-05-08},
% %  note = {Retrieved May 10, 2011, from http://www.lyx.org},
% %  url = {http://www.lyx.org/}
% %}
% The \hologo{LyX} Team,
% \textit{\hologo{LyX} -- The Document Processor},
% 2011-05-08.\\
% \url{http://www.lyx.org/}
%
% \bibitem{OzTeX}
% Andrew Trevorrow,
% \hologo{OzTeX} FAQ: What is the correct way to typeset ``\hologo{OzTeX}''?,
% 2011-09-15 (visited).
% \url{http://www.trevorrow.com/oztex/ozfaq.html#oztex-logo}
%
% \bibitem{PiCTeX}
% Michael Wichura,
% \textit{The \hologo{PiCTeX} macro package},
% 1987-09-21.
% \CTAN{graphics/pictex/}
%
% \bibitem{scrlogo}
% Markus Kohm,
% \textit{\hologo{KOMAScript} Datei \xfile{scrlogo.dtx}},
% 2009-01-30.\\
% \CTAN{install/macros/latex/contrib/komascript.tds.zip}
%
% \end{thebibliography}
%
% \begin{History}
%   \begin{Version}{2010/04/08 v1.0}
%   \item
%     The first version.
%   \end{Version}
%   \begin{Version}{2010/04/16 v1.1}
%   \item
%     \cs{Hologo} added for support of logos at start of a sentence.
%   \item
%     \cs{hologoSetup} and \cs{hologoLogoSetup} added.
%   \item
%     Options \xoption{break}, \xoption{hyphenbreak}, \xoption{spacebreak}
%     added.
%   \item
%     Variant support added by option \xoption{variant}.
%   \end{Version}
%   \begin{Version}{2010/04/24 v1.2}
%   \item
%     \hologo{LaTeX3} added.
%   \item
%     \hologo{VTeX} added.
%   \end{Version}
%   \begin{Version}{2010/11/21 v1.3}
%   \item
%     \hologo{iniTeX}, \hologo{virTeX} added.
%   \end{Version}
%   \begin{Version}{2011/03/25 v1.4}
%   \item
%     \hologo{ConTeXt} with variants added.
%   \item
%     Option \xoption{discretionarybreak} added as refinement for
%     option \xoption{break}.
%   \end{Version}
%   \begin{Version}{2011/04/21 v1.5}
%   \item
%     Wrong TDS directory for test files fixed.
%   \end{Version}
%   \begin{Version}{2011/10/01 v1.6}
%   \item
%     Support for package \xpackage{tex4ht} added.
%   \item
%     Support for \cs{csname} added if \cs{ifincsname} is available.
%   \item
%     New logos:
%     \hologo{(La)TeX},
%     \hologo{biber},
%     \hologo{BibTeX} (\xoption{sc}, \xoption{sf}),
%     \hologo{emTeX},
%     \hologo{ExTeX},
%     \hologo{KOMAScript},
%     \hologo{La},
%     \hologo{LyX},
%     \hologo{MiKTeX},
%     \hologo{NTS},
%     \hologo{OzMF},
%     \hologo{OzMP},
%     \hologo{OzTeX},
%     \hologo{OzTtH},
%     \hologo{PCTeX},
%     \hologo{PiC},
%     \hologo{PiCTeX},
%     \hologo{METAFONT},
%     \hologo{MetaFun},
%     \hologo{METAPOST},
%     \hologo{MetaPost},
%     \hologo{SLiTeX} (\xoption{lift}, \xoption{narrow}, \xoption{simple}),
%     \hologo{SliTeX} (\xoption{narrow}, \xoption{simple}, \xoption{lift}),
%     \hologo{teTeX}.
%   \item
%     Fixes:
%     \hologo{iniTeX},
%     \hologo{pdfLaTeX},
%     \hologo{pdfTeX},
%     \hologo{virTeX}.
%   \item
%     \cs{hologoFontSetup} and \cs{hologoLogoFontSetup} added.
%   \item
%     \cs{hologoVariant} and \cs{HologoVariant} added.
%   \end{Version}
%   \begin{Version}{2011/11/22 v1.7}
%   \item
%     New logos:
%     \hologo{BibTeX8},
%     \hologo{LaTeXML},
%     \hologo{SageTeX},
%     \hologo{TeX4ht},
%     \hologo{TTH}.
%   \item
%     \hologo{Xe} and friends: Driver stuff fixed.
%   \item
%     \hologo{Xe} and friends: Support for italic added.
%   \item
%     \hologo{Xe} and friends: Package support for \xpackage{pgf}
%     and \xpackage{pstricks} added.
%   \end{Version}
%   \begin{Version}{2011/11/29 v1.8}
%   \item
%     New logos:
%     \hologo{HanTheThanh}.
%   \end{Version}
%   \begin{Version}{2011/12/21 v1.9}
%   \item
%     Patch for package \xpackage{ifxetex} added for the case that
%     \cs{newif} is undefined in \hologo{iniTeX}.
%   \item
%     Some fixes for \hologo{iniTeX}.
%   \end{Version}
%   \begin{Version}{2012/04/26 v1.10}
%   \item
%     Fix in bookmark version of logo ``\hologo{HanTheThanh}''.
%   \end{Version}
%   \begin{Version}{2016/05/12 v1.11}
%   \item
%     Update HOLOGO@IfCharExists (previously in texlive)
%   \item define pdfliteral in current luatex.
%   \end{Version}
% \end{History}
%
% \PrintIndex
%
% \Finale
\endinput
|
% \end{quote}
% Do not forget to quote the argument according to the demands
% of your shell.
%
% \paragraph{Generating the documentation.}
% You can use both the \xfile{.dtx} or the \xfile{.drv} to generate
% the documentation. The process can be configured by the
% configuration file \xfile{ltxdoc.cfg}. For instance, put this
% line into this file, if you want to have A4 as paper format:
% \begin{quote}
%   \verb|\PassOptionsToClass{a4paper}{article}|
% \end{quote}
% An example follows how to generate the
% documentation with pdf\LaTeX:
% \begin{quote}
%\begin{verbatim}
%pdflatex hologo.dtx
%makeindex -s gind.ist hologo.idx
%pdflatex hologo.dtx
%makeindex -s gind.ist hologo.idx
%pdflatex hologo.dtx
%\end{verbatim}
% \end{quote}
%
% \section{Catalogue}
%
% The following XML file can be used as source for the
% \href{http://mirror.ctan.org/help/Catalogue/catalogue.html}{\TeX\ Catalogue}.
% The elements \texttt{caption} and \texttt{description} are imported
% from the original XML file from the Catalogue.
% The name of the XML file in the Catalogue is \xfile{hologo.xml}.
%    \begin{macrocode}
%<*catalogue>
<?xml version='1.0' encoding='us-ascii'?>
<!DOCTYPE entry SYSTEM 'catalogue.dtd'>
<entry datestamp='$Date$' modifier='$Author$' id='hologo'>
  <name>hologo</name>
  <caption>A collection of logos with bookmark support.</caption>
  <authorref id='auth:oberdiek'/>
  <copyright owner='Heiko Oberdiek' year='2010-2012'/>
  <license type='lppl1.3'/>
  <version number='1.10'/>
  <description>
    The package defines a single command <tt>\hologo</tt>, whose
    argument is the usual case-confused ASCII version of the logo.
    The command is bookmark-enabled, so that every logo becomes
    available in bookmarks without further work.
    <p/>
    The package is part of the <xref refid='oberdiek'>oberdiek</xref>
    bundle.
  </description>
  <documentation details='Package documentation'
      href='ctan:/macros/latex/contrib/oberdiek/hologo.pdf'/>
  <ctan file='true' path='/macros/latex/contrib/oberdiek/hologo.dtx'/>
  <miktex location='oberdiek'/>
  <texlive location='oberdiek'/>
  <install path='/macros/latex/contrib/oberdiek/oberdiek.tds.zip'/>
</entry>
%</catalogue>
%    \end{macrocode}
%
% \begin{thebibliography}{9}
% \raggedright
%
% \bibitem{btxdoc}
% Oren Patashnik,
% \textit{\hologo{BibTeX}ing},
% 1988-02-08.\\
% \CTAN{biblio/bibtex/base/}
%
% \bibitem{dtklogos}
% Gerd Neugebauer, DANTE,
% \textit{Package \xpackage{dtklogos}},
% 2011-04-25.\\
% \CTAN{usergrps/dante/dtk/dtklogos.sty}
%
% \bibitem{etexman}
% The \hologo{NTS} Team,
% \textit{The \hologo{eTeX} manual},
% 1998-02.\\
% \CTAN{systems/e-tex/v2/doc/}
%
% \bibitem{ExTeX-FAQ}
% The \hologo{ExTeX} group,
% \textit{\hologo{ExTeX}: FAQ -- How is \hologo{ExTeX} typeset?},
% 2007-04-14.\\
% \url{http://www.extex.org/documentation/faq.html}
%
% \bibitem{LyX}
% %@MISC{ LyX,
% %  title = {{LyX 2.0.0 -- The Document Processor [Computer software and manual]}},
% %  author = {{The LyX Team}},
% %  howpublished = {Internet: http://www.lyx.org},
% %  year = {2011-05-08},
% %  note = {Retrieved May 10, 2011, from http://www.lyx.org},
% %  url = {http://www.lyx.org/}
% %}
% The \hologo{LyX} Team,
% \textit{\hologo{LyX} -- The Document Processor},
% 2011-05-08.\\
% \url{http://www.lyx.org/}
%
% \bibitem{OzTeX}
% Andrew Trevorrow,
% \hologo{OzTeX} FAQ: What is the correct way to typeset ``\hologo{OzTeX}''?,
% 2011-09-15 (visited).
% \url{http://www.trevorrow.com/oztex/ozfaq.html#oztex-logo}
%
% \bibitem{PiCTeX}
% Michael Wichura,
% \textit{The \hologo{PiCTeX} macro package},
% 1987-09-21.
% \CTAN{graphics/pictex/}
%
% \bibitem{scrlogo}
% Markus Kohm,
% \textit{\hologo{KOMAScript} Datei \xfile{scrlogo.dtx}},
% 2009-01-30.\\
% \CTAN{install/macros/latex/contrib/komascript.tds.zip}
%
% \end{thebibliography}
%
% \begin{History}
%   \begin{Version}{2010/04/08 v1.0}
%   \item
%     The first version.
%   \end{Version}
%   \begin{Version}{2010/04/16 v1.1}
%   \item
%     \cs{Hologo} added for support of logos at start of a sentence.
%   \item
%     \cs{hologoSetup} and \cs{hologoLogoSetup} added.
%   \item
%     Options \xoption{break}, \xoption{hyphenbreak}, \xoption{spacebreak}
%     added.
%   \item
%     Variant support added by option \xoption{variant}.
%   \end{Version}
%   \begin{Version}{2010/04/24 v1.2}
%   \item
%     \hologo{LaTeX3} added.
%   \item
%     \hologo{VTeX} added.
%   \end{Version}
%   \begin{Version}{2010/11/21 v1.3}
%   \item
%     \hologo{iniTeX}, \hologo{virTeX} added.
%   \end{Version}
%   \begin{Version}{2011/03/25 v1.4}
%   \item
%     \hologo{ConTeXt} with variants added.
%   \item
%     Option \xoption{discretionarybreak} added as refinement for
%     option \xoption{break}.
%   \end{Version}
%   \begin{Version}{2011/04/21 v1.5}
%   \item
%     Wrong TDS directory for test files fixed.
%   \end{Version}
%   \begin{Version}{2011/10/01 v1.6}
%   \item
%     Support for package \xpackage{tex4ht} added.
%   \item
%     Support for \cs{csname} added if \cs{ifincsname} is available.
%   \item
%     New logos:
%     \hologo{(La)TeX},
%     \hologo{biber},
%     \hologo{BibTeX} (\xoption{sc}, \xoption{sf}),
%     \hologo{emTeX},
%     \hologo{ExTeX},
%     \hologo{KOMAScript},
%     \hologo{La},
%     \hologo{LyX},
%     \hologo{MiKTeX},
%     \hologo{NTS},
%     \hologo{OzMF},
%     \hologo{OzMP},
%     \hologo{OzTeX},
%     \hologo{OzTtH},
%     \hologo{PCTeX},
%     \hologo{PiC},
%     \hologo{PiCTeX},
%     \hologo{METAFONT},
%     \hologo{MetaFun},
%     \hologo{METAPOST},
%     \hologo{MetaPost},
%     \hologo{SLiTeX} (\xoption{lift}, \xoption{narrow}, \xoption{simple}),
%     \hologo{SliTeX} (\xoption{narrow}, \xoption{simple}, \xoption{lift}),
%     \hologo{teTeX}.
%   \item
%     Fixes:
%     \hologo{iniTeX},
%     \hologo{pdfLaTeX},
%     \hologo{pdfTeX},
%     \hologo{virTeX}.
%   \item
%     \cs{hologoFontSetup} and \cs{hologoLogoFontSetup} added.
%   \item
%     \cs{hologoVariant} and \cs{HologoVariant} added.
%   \end{Version}
%   \begin{Version}{2011/11/22 v1.7}
%   \item
%     New logos:
%     \hologo{BibTeX8},
%     \hologo{LaTeXML},
%     \hologo{SageTeX},
%     \hologo{TeX4ht},
%     \hologo{TTH}.
%   \item
%     \hologo{Xe} and friends: Driver stuff fixed.
%   \item
%     \hologo{Xe} and friends: Support for italic added.
%   \item
%     \hologo{Xe} and friends: Package support for \xpackage{pgf}
%     and \xpackage{pstricks} added.
%   \end{Version}
%   \begin{Version}{2011/11/29 v1.8}
%   \item
%     New logos:
%     \hologo{HanTheThanh}.
%   \end{Version}
%   \begin{Version}{2011/12/21 v1.9}
%   \item
%     Patch for package \xpackage{ifxetex} added for the case that
%     \cs{newif} is undefined in \hologo{iniTeX}.
%   \item
%     Some fixes for \hologo{iniTeX}.
%   \end{Version}
%   \begin{Version}{2012/04/26 v1.10}
%   \item
%     Fix in bookmark version of logo ``\hologo{HanTheThanh}''.
%   \end{Version}
%   \begin{Version}{2016/05/12 v1.11}
%   \item
%     Update HOLOGO@IfCharExists (previously in texlive)
%   \item define pdfliteral in current luatex.
%   \end{Version}
% \end{History}
%
% \PrintIndex
%
% \Finale
\endinput
%
        \else
          \input hologo.cfg\relax
        \fi
      \else
        \@PackageInfoNoLine{hologo}{%
          Empty configuration file `hologo.cfg' ignored%
        }%
      \fi
    \fi
  }%
}
%    \end{macrocode}
%
%    \begin{macrocode}
\def\HOLOGO@temp#1#2{%
  \kv@define@key{HoLogoDriver}{#1}[]{%
    \begingroup
      \def\HOLOGO@temp{##1}%
      \ltx@onelevel@sanitize\HOLOGO@temp
      \ifx\HOLOGO@temp\ltx@empty
      \else
        \@PackageError{hologo}{%
          Value (\HOLOGO@temp) not permitted for option `#1'%
        }%
        \@ehc
      \fi
    \endgroup
    \def\hologoDriver{#2}%
  }%
}%
\def\HOLOGO@@temp#1#2{%
  \ifx\kv@value\relax
    \HOLOGO@temp{#1}{#1}%
  \else
    \HOLOGO@temp{#1}{#2}%
  \fi
}%
\kv@parse@normalized{%
  pdftex,%
  luatex=pdftex,%
  dvipdfm,%
  dvipdfmx=dvipdfm,%
  dvips,%
  dvipsone=dvips,%
  xdvi=dvips,%
  xetex,%
  vtex,%
}\HOLOGO@@temp
%    \end{macrocode}
%
%    \begin{macrocode}
\kv@define@key{HoLogoDriver}{driverfallback}{%
  \def\HOLOGO@DriverFallback{#1}%
}
%    \end{macrocode}
%
%    \begin{macro}{\HOLOGO@DriverFallback}
%    \begin{macrocode}
\def\HOLOGO@DriverFallback{dvips}
%    \end{macrocode}
%    \end{macro}
%
%    \begin{macro}{\hologoDriverSetup}
%    \begin{macrocode}
\def\hologoDriverSetup{%
  \let\hologoDriver\ltx@undefined
  \HOLOGO@DriverSetup
}
%    \end{macrocode}
%    \end{macro}
%
%    \begin{macro}{\HOLOGO@DriverSetup}
%    \begin{macrocode}
\def\HOLOGO@DriverSetup#1{%
  \kvsetkeys{HoLogoDriver}{#1}%
  \HOLOGO@CheckDriver
  \ltx@ifundefined{hologoDriver}{%
    \begingroup
    \edef\x{\endgroup
      \noexpand\kvsetkeys{HoLogoDriver}{\HOLOGO@DriverFallback}%
    }\x
  }{}%
  \@PackageInfoNoLine{hologo}{Using driver `\hologoDriver'}%
}
%    \end{macrocode}
%    \end{macro}
%
%    \begin{macro}{\HOLOGO@CheckDriver}
%    \begin{macrocode}
\def\HOLOGO@CheckDriver{%
  \ifpdf
    \def\hologoDriver{pdftex}%
    \let\HOLOGO@pdfliteral\pdfliteral
    \ifluatex
      \ifx\pdfextension\@undefined\else
        \protected\def\pdfliteral{\pdfextension literal}%
        \let\HOLOGO@pdfliteral\pdfliteral
      \fi
      \ltx@IfUndefined{HOLOGO@pdfliteral}{%
        \ifnum\luatexversion<36 %
        \else
          \begingroup
            \let\HOLOGO@temp\endgroup
            \ifcase0%
                \directlua{%
                  if tex.enableprimitives then %
                    tex.enableprimitives('HOLOGO@', {'pdfliteral'})%
                  else %
                    tex.print('1')%
                  end%
                }%
                \ifx\HOLOGO@pdfliteral\@undefined 1\fi%
                \relax%
              \endgroup
              \let\HOLOGO@temp\relax
              \global\let\HOLOGO@pdfliteral\HOLOGO@pdfliteral
            \fi%
          \HOLOGO@temp
        \fi
      }{}%
    \fi
    \ltx@IfUndefined{HOLOGO@pdfliteral}{%
      \@PackageWarningNoLine{hologo}{%
        Cannot find \string\pdfliteral
      }%
    }{}%
  \else
    \ifxetex
      \def\hologoDriver{xetex}%
    \else
      \ifvtex
        \def\hologoDriver{vtex}%
      \fi
    \fi
  \fi
}
%    \end{macrocode}
%    \end{macro}
%
%    \begin{macro}{\HOLOGO@WarningUnsupportedDriver}
%    \begin{macrocode}
\def\HOLOGO@WarningUnsupportedDriver#1{%
  \@PackageWarningNoLine{hologo}{%
    Logo `#1' needs driver specific macros,\MessageBreak
    but driver `\hologoDriver' is not supported.\MessageBreak
    Use a different driver or\MessageBreak
    load package `graphics' or `pgf'%
  }%
}
%    \end{macrocode}
%    \end{macro}
%
% \subsubsection{Reflect box macros}
%
%    Skip driver part if not needed.
%    \begin{macrocode}
\ltx@IfUndefined{reflectbox}{}{%
  \ltx@IfUndefined{rotatebox}{}{%
    \HOLOGO@AtEnd
  }%
}
\ltx@IfUndefined{pgftext}{}{%
  \HOLOGO@AtEnd
}
\ltx@IfUndefined{psscalebox}{}{%
  \HOLOGO@AtEnd
}
%    \end{macrocode}
%
%    \begin{macrocode}
\def\HOLOGO@temp{LaTeX2e}
\ifx\fmtname\HOLOGO@temp
  \RequirePackage{kvoptions}[2011/06/30]%
  \ProcessKeyvalOptions{HoLogoDriver}%
\fi
\HOLOGO@DriverSetup{}
%    \end{macrocode}
%
%    \begin{macro}{\HOLOGO@ReflectBox}
%    \begin{macrocode}
\def\HOLOGO@ReflectBox#1{%
  \begingroup
    \setbox\ltx@zero\hbox{\begingroup#1\endgroup}%
    \setbox\ltx@two\hbox{%
      \kern\wd\ltx@zero
      \csname HOLOGO@ScaleBox@\hologoDriver\endcsname{-1}{1}{%
        \hbox to 0pt{\copy\ltx@zero\hss}%
      }%
    }%
    \wd\ltx@two=\wd\ltx@zero
    \box\ltx@two
  \endgroup
}
%    \end{macrocode}
%    \end{macro}
%
%    \begin{macro}{\HOLOGO@PointReflectBox}
%    \begin{macrocode}
\def\HOLOGO@PointReflectBox#1{%
  \begingroup
    \setbox\ltx@zero\hbox{\begingroup#1\endgroup}%
    \setbox\ltx@two\hbox{%
      \kern\wd\ltx@zero
      \raise\ht\ltx@zero\hbox{%
        \csname HOLOGO@ScaleBox@\hologoDriver\endcsname{-1}{-1}{%
          \hbox to 0pt{\copy\ltx@zero\hss}%
        }%
      }%
    }%
    \wd\ltx@two=\wd\ltx@zero
    \box\ltx@two
  \endgroup
}
%    \end{macrocode}
%    \end{macro}
%
%    We must define all variants because of dynamic driver setup.
%    \begin{macrocode}
\def\HOLOGO@temp#1#2{#2}
%    \end{macrocode}
%
%    \begin{macro}{\HOLOGO@ScaleBox@pdftex}
%    \begin{macrocode}
\HOLOGO@temp{pdftex}{%
  \def\HOLOGO@ScaleBox@pdftex#1#2#3{%
    \HOLOGO@pdfliteral{%
      q #1 0 0 #2 0 0 cm%
    }%
    #3%
    \HOLOGO@pdfliteral{%
      Q%
    }%
  }%
}
%    \end{macrocode}
%    \end{macro}
%    \begin{macro}{\HOLOGO@ScaleBox@dvips}
%    \begin{macrocode}
\HOLOGO@temp{dvips}{%
  \def\HOLOGO@ScaleBox@dvips#1#2#3{%
    \special{ps:%
      gsave %
      currentpoint %
      currentpoint translate %
      #1 #2 scale %
      neg exch neg exch translate%
    }%
    #3%
    \special{ps:%
      currentpoint %
      grestore %
      moveto%
    }%
  }%
}
%    \end{macrocode}
%    \end{macro}
%    \begin{macro}{\HOLOGO@ScaleBox@dvipdfm}
%    \begin{macrocode}
\HOLOGO@temp{dvipdfm}{%
  \let\HOLOGO@ScaleBox@dvipdfm\HOLOGO@ScaleBox@dvips
}
%    \end{macrocode}
%    \end{macro}
%    Since \hologo{XeTeX} v0.6.
%    \begin{macro}{\HOLOGO@ScaleBox@xetex}
%    \begin{macrocode}
\HOLOGO@temp{xetex}{%
  \def\HOLOGO@ScaleBox@xetex#1#2#3{%
    \special{x:gsave}%
    \special{x:scale #1 #2}%
    #3%
    \special{x:grestore}%
  }%
}
%    \end{macrocode}
%    \end{macro}
%    \begin{macro}{\HOLOGO@ScaleBox@vtex}
%    \begin{macrocode}
\HOLOGO@temp{vtex}{%
  \def\HOLOGO@ScaleBox@vtex#1#2#3{%
    \special{r(#1,0,0,#2,0,0}%
    #3%
    \special{r)}%
  }%
}
%    \end{macrocode}
%    \end{macro}
%
%    \begin{macrocode}
\HOLOGO@AtEnd%
%</package>
%    \end{macrocode}
%
% \section{Test}
%
% \subsection{Catcode checks for loading}
%
%    \begin{macrocode}
%<*test1>
%    \end{macrocode}
%    \begin{macrocode}
\catcode`\{=1 %
\catcode`\}=2 %
\catcode`\#=6 %
\catcode`\@=11 %
\expandafter\ifx\csname count@\endcsname\relax
  \countdef\count@=255 %
\fi
\expandafter\ifx\csname @gobble\endcsname\relax
  \long\def\@gobble#1{}%
\fi
\expandafter\ifx\csname @firstofone\endcsname\relax
  \long\def\@firstofone#1{#1}%
\fi
\expandafter\ifx\csname loop\endcsname\relax
  \expandafter\@firstofone
\else
  \expandafter\@gobble
\fi
{%
  \def\loop#1\repeat{%
    \def\body{#1}%
    \iterate
  }%
  \def\iterate{%
    \body
      \let\next\iterate
    \else
      \let\next\relax
    \fi
    \next
  }%
  \let\repeat=\fi
}%
\def\RestoreCatcodes{}
\count@=0 %
\loop
  \edef\RestoreCatcodes{%
    \RestoreCatcodes
    \catcode\the\count@=\the\catcode\count@\relax
  }%
\ifnum\count@<255 %
  \advance\count@ 1 %
\repeat

\def\RangeCatcodeInvalid#1#2{%
  \count@=#1\relax
  \loop
    \catcode\count@=15 %
  \ifnum\count@<#2\relax
    \advance\count@ 1 %
  \repeat
}
\def\RangeCatcodeCheck#1#2#3{%
  \count@=#1\relax
  \loop
    \ifnum#3=\catcode\count@
    \else
      \errmessage{%
        Character \the\count@\space
        with wrong catcode \the\catcode\count@\space
        instead of \number#3%
      }%
    \fi
  \ifnum\count@<#2\relax
    \advance\count@ 1 %
  \repeat
}
\def\space{ }
\expandafter\ifx\csname LoadCommand\endcsname\relax
  \def\LoadCommand{\input hologo.sty\relax}%
\fi
\def\Test{%
  \RangeCatcodeInvalid{0}{47}%
  \RangeCatcodeInvalid{58}{64}%
  \RangeCatcodeInvalid{91}{96}%
  \RangeCatcodeInvalid{123}{255}%
  \catcode`\@=12 %
  \catcode`\\=0 %
  \catcode`\%=14 %
  \LoadCommand
  \RangeCatcodeCheck{0}{36}{15}%
  \RangeCatcodeCheck{37}{37}{14}%
  \RangeCatcodeCheck{38}{47}{15}%
  \RangeCatcodeCheck{48}{57}{12}%
  \RangeCatcodeCheck{58}{63}{15}%
  \RangeCatcodeCheck{64}{64}{12}%
  \RangeCatcodeCheck{65}{90}{11}%
  \RangeCatcodeCheck{91}{91}{15}%
  \RangeCatcodeCheck{92}{92}{0}%
  \RangeCatcodeCheck{93}{96}{15}%
  \RangeCatcodeCheck{97}{122}{11}%
  \RangeCatcodeCheck{123}{255}{15}%
  \RestoreCatcodes
}
\Test
\csname @@end\endcsname
\end
%    \end{macrocode}
%    \begin{macrocode}
%</test1>
%    \end{macrocode}
%
% \subsection{Spacefactor}
%
%    The space factor must be 1000 after a logo. If it is greater 1000
%    then the following space is a space after a sentence closing point.
%    If the space factor is smaller 1000 then an immediate following
%    dot is interpreted as abbreviation, not sentence closing point.
%
%    \begin{macrocode}
%<*test-spacefactor>
\NeedsTeXFormat{LaTeX2e}
\documentclass{article}
\usepackage{hologo}[2016/05/12]
\usepackage{kvsetkeys}
\usepackage{qstest}
\IncludeTests{*}
\LogTests{log}{*}{*}
\begin{document}
\begin{qstest}{spacefactor}{spacefactor}
\newcommand*{\Test}[1]{%
  \sbox0{%
    \hologo{#1}%
    \Expect*{1000 (#1)}*{\the\spacefactor\space(#1)}%
  }%
}%
\makeatletter
\def\TestList{}
\def\hologoEntry#1#2#3{%
  \edef\TestList{%
    \ifx\TestList\@empty
    \else
      \TestList,%
    \fi
    #1%
    \ifx\\#2\\%
    \else
      ={variant=#2}%
    \fi
  }%
}
\hologoList
\expandafter\kv@parse@normalized\expandafter{%
  \TestList
}{%
  \begingroup
    \let\@logo=\kv@key
    \ifx\kv@value\relax
    \else
      \expandafter\hologoLogoSetup\expandafter\@logo\expandafter{%
        \kv@value
      }%
    \fi
    \Test\@logo
  \endgroup
  \@gobbletwo
}
\end{qstest}
\end{document}
%</test-spacefactor>
%    \end{macrocode}
%
% \subsection{Complete list}
%
%    \begin{macrocode}
%<*test-list>
\NeedsTeXFormat{LaTeX2e}
\documentclass[12pt,a4paper]{article}
\usepackage{hologo}[2016/05/12]
\usepackage[T1]{fontenc}
\usepackage{lmodern}
\usepackage{parskip}
\usepackage[unicode]{hyperref}[2011/09/28]
\usepackage{bookmark}[2011/09/19]
\bookmarksetup{%
  numbered,%
  open,%
  openlevel=2,%
}
\renewcommand*{\contentsname}{List of logos}
\begin{document}
\tableofcontents
\def\TestFont#1#2#3#4#5#6{%
  \begingroup
    \usefont{#3}{#4}{#5}{#6}%
    \HologoVariant{#1}{#2}/\hologoVariant{#1}{#2}%
    \quad
    \begingroup\scriptsize\hologoVariant{#1}{#2}\endgroup
    \quad
  \endgroup
  (#3/#4/#5/#6)%
  \par
}
\makeatletter
\def\hologoEntry#1#2#3{%
  \section{%
    \HologoVariant{#1}{#2}/\hologoVariant{#1}{#2} %
    {[#1\ifx\\#2\\\else\space(#2)\fi]}% hash-ok
  }% braces around [] because of bug in tex4ht
  \begingroup
    \hypersetup{unicode=false}%
    \bookmark[%
      dest=\@currentHref,%
      rellevel=1,%
      keeplevel,%
    ]{%
      \HologoVariant{#1}{#2}/\hologoVariant{#1}{#2} %
      (PDFDocEncoding)%
    }%
  \endgroup
  \TestFont{#1}{#2}{OT1}{cmr}{m}{n}%
  \TestFont{#1}{#2}{OT1}{cmss}{m}{n}%
  \TestFont{#1}{#2}{OT1}{cmr}{b}{n}%
  \TestFont{#1}{#2}{OT1}{cmr}{m}{it}%
  \TestFont{#1}{#2}{OT1}{cmtt}{m}{n}%
  \TestFont{#1}{#2}{T1}{lmr}{m}{n}%
  \TestFont{#1}{#2}{T1}{lmss}{m}{n}%
  \TestFont{#1}{#2}{T1}{lmr}{b}{n}%
  \TestFont{#1}{#2}{T1}{lmr}{m}{it}%
  \TestFont{#1}{#2}{T1}{lmtt}{m}{n}%
  \TestFont{#1}{#2}{T1}{lmvtt}{m}{n}%
  \TestFont{#1}{#2}{T1}{qtm}{m}{n}%
  \TestFont{#1}{#2}{T1}{qhv}{m}{n}%
  \TestFont{#1}{#2}{T1}{qtm}{b}{n}%
  \TestFont{#1}{#2}{T1}{qtm}{m}{it}%
  \TestFont{#1}{#2}{T1}{qcr}{m}{n}%
  \newpage
}
\makeatother
\hologoList
\end{document}
%</test-list>
%    \end{macrocode}
%
% \section{Installation}
%
% \subsection{Download}
%
% \paragraph{Package.} This package is available on
% CTAN\footnote{\url{ftp://ftp.ctan.org/tex-archive/}}:
% \begin{description}
% \item[\CTAN{macros/latex/contrib/oberdiek/hologo.dtx}] The source file.
% \item[\CTAN{macros/latex/contrib/oberdiek/hologo.pdf}] Documentation.
% \end{description}
%
%
% \paragraph{Bundle.} All the packages of the bundle `oberdiek'
% are also available in a TDS compliant ZIP archive. There
% the packages are already unpacked and the documentation files
% are generated. The files and directories obey the TDS standard.
% \begin{description}
% \item[\CTAN{install/macros/latex/contrib/oberdiek.tds.zip}]
% \end{description}
% \emph{TDS} refers to the standard ``A Directory Structure
% for \TeX\ Files'' (\CTAN{tds/tds.pdf}). Directories
% with \xfile{texmf} in their name are usually organized this way.
%
% \subsection{Bundle installation}
%
% \paragraph{Unpacking.} Unpack the \xfile{oberdiek.tds.zip} in the
% TDS tree (also known as \xfile{texmf} tree) of your choice.
% Example (linux):
% \begin{quote}
%   |unzip oberdiek.tds.zip -d ~/texmf|
% \end{quote}
%
% \paragraph{Script installation.}
% Check the directory \xfile{TDS:scripts/oberdiek/} for
% scripts that need further installation steps.
% Package \xpackage{attachfile2} comes with the Perl script
% \xfile{pdfatfi.pl} that should be installed in such a way
% that it can be called as \texttt{pdfatfi}.
% Example (linux):
% \begin{quote}
%   |chmod +x scripts/oberdiek/pdfatfi.pl|\\
%   |cp scripts/oberdiek/pdfatfi.pl /usr/local/bin/|
% \end{quote}
%
% \subsection{Package installation}
%
% \paragraph{Unpacking.} The \xfile{.dtx} file is a self-extracting
% \docstrip\ archive. The files are extracted by running the
% \xfile{.dtx} through \plainTeX:
% \begin{quote}
%   \verb|tex hologo.dtx|
% \end{quote}
%
% \paragraph{TDS.} Now the different files must be moved into
% the different directories in your installation TDS tree
% (also known as \xfile{texmf} tree):
% \begin{quote}
% \def\t{^^A
% \begin{tabular}{@{}>{\ttfamily}l@{ $\rightarrow$ }>{\ttfamily}l@{}}
%   hologo.sty & tex/generic/oberdiek/hologo.sty\\
%   hologo.pdf & doc/latex/oberdiek/hologo.pdf\\
%   example/hologo-example.tex & doc/latex/oberdiek/example/hologo-example.tex\\
%   test/hologo-test1.tex & doc/latex/oberdiek/test/hologo-test1.tex\\
%   test/hologo-test-spacefactor.tex & doc/latex/oberdiek/test/hologo-test-spacefactor.tex\\
%   test/hologo-test-list.tex & doc/latex/oberdiek/test/hologo-test-list.tex\\
%   hologo.dtx & source/latex/oberdiek/hologo.dtx\\
% \end{tabular}^^A
% }^^A
% \sbox0{\t}^^A
% \ifdim\wd0>\linewidth
%   \begingroup
%     \advance\linewidth by\leftmargin
%     \advance\linewidth by\rightmargin
%   \edef\x{\endgroup
%     \def\noexpand\lw{\the\linewidth}^^A
%   }\x
%   \def\lwbox{^^A
%     \leavevmode
%     \hbox to \linewidth{^^A
%       \kern-\leftmargin\relax
%       \hss
%       \usebox0
%       \hss
%       \kern-\rightmargin\relax
%     }^^A
%   }^^A
%   \ifdim\wd0>\lw
%     \sbox0{\small\t}^^A
%     \ifdim\wd0>\linewidth
%       \ifdim\wd0>\lw
%         \sbox0{\footnotesize\t}^^A
%         \ifdim\wd0>\linewidth
%           \ifdim\wd0>\lw
%             \sbox0{\scriptsize\t}^^A
%             \ifdim\wd0>\linewidth
%               \ifdim\wd0>\lw
%                 \sbox0{\tiny\t}^^A
%                 \ifdim\wd0>\linewidth
%                   \lwbox
%                 \else
%                   \usebox0
%                 \fi
%               \else
%                 \lwbox
%               \fi
%             \else
%               \usebox0
%             \fi
%           \else
%             \lwbox
%           \fi
%         \else
%           \usebox0
%         \fi
%       \else
%         \lwbox
%       \fi
%     \else
%       \usebox0
%     \fi
%   \else
%     \lwbox
%   \fi
% \else
%   \usebox0
% \fi
% \end{quote}
% If you have a \xfile{docstrip.cfg} that configures and enables \docstrip's
% TDS installing feature, then some files can already be in the right
% place, see the documentation of \docstrip.
%
% \subsection{Refresh file name databases}
%
% If your \TeX~distribution
% (\teTeX, \mikTeX, \dots) relies on file name databases, you must refresh
% these. For example, \teTeX\ users run \verb|texhash| or
% \verb|mktexlsr|.
%
% \subsection{Some details for the interested}
%
% \paragraph{Attached source.}
%
% The PDF documentation on CTAN also includes the
% \xfile{.dtx} source file. It can be extracted by
% AcrobatReader 6 or higher. Another option is \textsf{pdftk},
% e.g. unpack the file into the current directory:
% \begin{quote}
%   \verb|pdftk hologo.pdf unpack_files output .|
% \end{quote}
%
% \paragraph{Unpacking with \LaTeX.}
% The \xfile{.dtx} chooses its action depending on the format:
% \begin{description}
% \item[\plainTeX:] Run \docstrip\ and extract the files.
% \item[\LaTeX:] Generate the documentation.
% \end{description}
% If you insist on using \LaTeX\ for \docstrip\ (really,
% \docstrip\ does not need \LaTeX), then inform the autodetect routine
% about your intention:
% \begin{quote}
%   \verb|latex \let\install=y% \iffalse meta-comment
%
% File: hologo.dtx
% Version: 2016/05/12 v1.11
% Info: A logo collection with bookmark support
%
% Copyright (C) 2010-2012 by
%    Heiko Oberdiek <heiko.oberdiek at googlemail.com>
%
% This work may be distributed and/or modified under the
% conditions of the LaTeX Project Public License, either
% version 1.3c of this license or (at your option) any later
% version. This version of this license is in
%    http://www.latex-project.org/lppl/lppl-1-3c.txt
% and the latest version of this license is in
%    http://www.latex-project.org/lppl.txt
% and version 1.3 or later is part of all distributions of
% LaTeX version 2005/12/01 or later.
%
% This work has the LPPL maintenance status "maintained".
%
% This Current Maintainer of this work is Heiko Oberdiek.
%
% The Base Interpreter refers to any `TeX-Format',
% because some files are installed in TDS:tex/generic//.
%
% This work consists of the main source file hologo.dtx
% and the derived files
%    hologo.sty, hologo.pdf, hologo.ins, hologo.drv, hologo-example.tex,
%    hologo-test1.tex, hologo-test-spacefactor.tex,
%    hologo-test-list.tex.
%
% Distribution:
%    CTAN:macros/latex/contrib/oberdiek/hologo.dtx
%    CTAN:macros/latex/contrib/oberdiek/hologo.pdf
%
% Unpacking:
%    (a) If hologo.ins is present:
%           tex hologo.ins
%    (b) Without hologo.ins:
%           tex hologo.dtx
%    (c) If you insist on using LaTeX
%           latex \let\install=y% \iffalse meta-comment
%
% File: hologo.dtx
% Version: 2016/05/12 v1.11
% Info: A logo collection with bookmark support
%
% Copyright (C) 2010-2012 by
%    Heiko Oberdiek <heiko.oberdiek at googlemail.com>
%
% This work may be distributed and/or modified under the
% conditions of the LaTeX Project Public License, either
% version 1.3c of this license or (at your option) any later
% version. This version of this license is in
%    http://www.latex-project.org/lppl/lppl-1-3c.txt
% and the latest version of this license is in
%    http://www.latex-project.org/lppl.txt
% and version 1.3 or later is part of all distributions of
% LaTeX version 2005/12/01 or later.
%
% This work has the LPPL maintenance status "maintained".
%
% This Current Maintainer of this work is Heiko Oberdiek.
%
% The Base Interpreter refers to any `TeX-Format',
% because some files are installed in TDS:tex/generic//.
%
% This work consists of the main source file hologo.dtx
% and the derived files
%    hologo.sty, hologo.pdf, hologo.ins, hologo.drv, hologo-example.tex,
%    hologo-test1.tex, hologo-test-spacefactor.tex,
%    hologo-test-list.tex.
%
% Distribution:
%    CTAN:macros/latex/contrib/oberdiek/hologo.dtx
%    CTAN:macros/latex/contrib/oberdiek/hologo.pdf
%
% Unpacking:
%    (a) If hologo.ins is present:
%           tex hologo.ins
%    (b) Without hologo.ins:
%           tex hologo.dtx
%    (c) If you insist on using LaTeX
%           latex \let\install=y\input{hologo.dtx}
%        (quote the arguments according to the demands of your shell)
%
% Documentation:
%    (a) If hologo.drv is present:
%           latex hologo.drv
%    (b) Without hologo.drv:
%           latex hologo.dtx; ...
%    The class ltxdoc loads the configuration file ltxdoc.cfg
%    if available. Here you can specify further options, e.g.
%    use A4 as paper format:
%       \PassOptionsToClass{a4paper}{article}
%
%    Programm calls to get the documentation (example):
%       pdflatex hologo.dtx
%       makeindex -s gind.ist hologo.idx
%       pdflatex hologo.dtx
%       makeindex -s gind.ist hologo.idx
%       pdflatex hologo.dtx
%
% Installation:
%    TDS:tex/generic/oberdiek/hologo.sty
%    TDS:doc/latex/oberdiek/hologo.pdf
%    TDS:doc/latex/oberdiek/example/hologo-example.tex
%    TDS:doc/latex/oberdiek/test/hologo-test1.tex
%    TDS:doc/latex/oberdiek/test/hologo-test-spacefactor.tex
%    TDS:doc/latex/oberdiek/test/hologo-test-list.tex
%    TDS:source/latex/oberdiek/hologo.dtx
%
%<*ignore>
\begingroup
  \catcode123=1 %
  \catcode125=2 %
  \def\x{LaTeX2e}%
\expandafter\endgroup
\ifcase 0\ifx\install y1\fi\expandafter
         \ifx\csname processbatchFile\endcsname\relax\else1\fi
         \ifx\fmtname\x\else 1\fi\relax
\else\csname fi\endcsname
%</ignore>
%<*install>
\input docstrip.tex
\Msg{************************************************************************}
\Msg{* Installation}
\Msg{* Package: hologo 2016/05/12 v1.11 A logo collection with bookmark support (HO)}
\Msg{************************************************************************}

\keepsilent
\askforoverwritefalse

\let\MetaPrefix\relax
\preamble

This is a generated file.

Project: hologo
Version: 2016/05/12 v1.11

Copyright (C) 2010-2012 by
   Heiko Oberdiek <heiko.oberdiek at googlemail.com>

This work may be distributed and/or modified under the
conditions of the LaTeX Project Public License, either
version 1.3c of this license or (at your option) any later
version. This version of this license is in
   http://www.latex-project.org/lppl/lppl-1-3c.txt
and the latest version of this license is in
   http://www.latex-project.org/lppl.txt
and version 1.3 or later is part of all distributions of
LaTeX version 2005/12/01 or later.

This work has the LPPL maintenance status "maintained".

This Current Maintainer of this work is Heiko Oberdiek.

The Base Interpreter refers to any `TeX-Format',
because some files are installed in TDS:tex/generic//.

This work consists of the main source file hologo.dtx
and the derived files
   hologo.sty, hologo.pdf, hologo.ins, hologo.drv, hologo-example.tex,
   hologo-test1.tex, hologo-test-spacefactor.tex,
   hologo-test-list.tex.

\endpreamble
\let\MetaPrefix\DoubleperCent

\generate{%
  \file{hologo.ins}{\from{hologo.dtx}{install}}%
  \file{hologo.drv}{\from{hologo.dtx}{driver}}%
  \usedir{tex/generic/oberdiek}%
  \file{hologo.sty}{\from{hologo.dtx}{package}}%
  \usedir{doc/latex/oberdiek/example}%
  \file{hologo-example.tex}{\from{hologo.dtx}{example}}%
  \usedir{doc/latex/oberdiek/test}%
  \file{hologo-test1.tex}{\from{hologo.dtx}{test1}}%
  \file{hologo-test-spacefactor.tex}{\from{hologo.dtx}{test-spacefactor}}%
  \file{hologo-test-list.tex}{\from{hologo.dtx}{test-list}}%
  \nopreamble
  \nopostamble
  \usedir{source/latex/oberdiek/catalogue}%
  \file{hologo.xml}{\from{hologo.dtx}{catalogue}}%
}

\catcode32=13\relax% active space
\let =\space%
\Msg{************************************************************************}
\Msg{*}
\Msg{* To finish the installation you have to move the following}
\Msg{* file into a directory searched by TeX:}
\Msg{*}
\Msg{*     hologo.sty}
\Msg{*}
\Msg{* To produce the documentation run the file `hologo.drv'}
\Msg{* through LaTeX.}
\Msg{*}
\Msg{* Happy TeXing!}
\Msg{*}
\Msg{************************************************************************}

\endbatchfile
%</install>
%<*ignore>
\fi
%</ignore>
%<*driver>
\NeedsTeXFormat{LaTeX2e}
\ProvidesFile{hologo.drv}%
  [2016/05/12 v1.11 A logo collection with bookmark support (HO)]%
\documentclass{ltxdoc}
\usepackage{holtxdoc}[2011/11/22]
\usepackage{hologo}[2016/05/12]
\usepackage{longtable}
\usepackage{array}
\usepackage{paralist}
%\usepackage[T1]{fontenc}
%\usepackage{lmodern}
\begin{document}
  \DocInput{hologo.dtx}%
\end{document}
%</driver>
% \fi
%
%
% \CharacterTable
%  {Upper-case    \A\B\C\D\E\F\G\H\I\J\K\L\M\N\O\P\Q\R\S\T\U\V\W\X\Y\Z
%   Lower-case    \a\b\c\d\e\f\g\h\i\j\k\l\m\n\o\p\q\r\s\t\u\v\w\x\y\z
%   Digits        \0\1\2\3\4\5\6\7\8\9
%   Exclamation   \!     Double quote  \"     Hash (number) \#
%   Dollar        \$     Percent       \%     Ampersand     \&
%   Acute accent  \'     Left paren    \(     Right paren   \)
%   Asterisk      \*     Plus          \+     Comma         \,
%   Minus         \-     Point         \.     Solidus       \/
%   Colon         \:     Semicolon     \;     Less than     \<
%   Equals        \=     Greater than  \>     Question mark \?
%   Commercial at \@     Left bracket  \[     Backslash     \\
%   Right bracket \]     Circumflex    \^     Underscore    \_
%   Grave accent  \`     Left brace    \{     Vertical bar  \|
%   Right brace   \}     Tilde         \~}
%
% \GetFileInfo{hologo.drv}
%
% \title{The \xpackage{hologo} package}
% \date{2016/05/12 v1.11}
% \author{Heiko Oberdiek\\\xemail{heiko.oberdiek at googlemail.com}}
%
% \maketitle
%
% \begin{abstract}
% This package starts a collection of logos with support for bookmarks
% strings.
% \end{abstract}
%
% \tableofcontents
%
% \section{Documentation}
%
% \subsection{Logo macros}
%
% \begin{declcs}{hologo} \M{name}
% \end{declcs}
% Macro \cs{hologo} sets the logo with name \meta{name}.
% The following table shows the supported names.
%
% \begingroup
%   \def\hologoEntry#1#2#3{^^A
%     #1&#2&\hologoLogoSetup{#1}{variant=#2}\hologo{#1}&#3\tabularnewline
%   }
%   \begin{longtable}{>{\ttfamily}l>{\ttfamily}lll}
%     \rmfamily\bfseries{name} & \rmfamily\bfseries variant
%     & \bfseries logo & \bfseries since\\
%     \hline
%     \endhead
%     \hologoList
%   \end{longtable}
% \endgroup
%
% \begin{declcs}{Hologo} \M{name}
% \end{declcs}
% Macro \cs{Hologo} starts the logo \meta{name} with an uppercase
% letter. As an exception small greek letters are not converted
% to uppercase. Examples, see \hologo{eTeX} and \hologo{ExTeX}.
%
% \subsection{Setup macros}
%
% The package does not support package options, but the following
% setup macros can be used to set options.
%
% \begin{declcs}{hologoSetup} \M{key value list}
% \end{declcs}
% Macro \cs{hologoSetup} sets global options.
%
% \begin{declcs}{hologoLogoSetup} \M{logo} \M{key value list}
% \end{declcs}
% Some options can also be used to configure a logo.
% These settings take precedence over global option settings.
%
% \subsection{Options}\label{sec:options}
%
% There are boolean and string options:
% \begin{description}
% \item[Boolean option:]
% It takes |true| or |false|
% as value. If the value is omitted, then |true| is used.
% \item[String option:]
% A value must be given as string. (But the string might be empty.)
% \end{description}
% The following options can be used both in \cs{hologoSetup}
% and \cs{hologoLogoSetup}:
% \begin{description}
% \def\entry#1{\item[\xoption{#1}:]}
% \entry{break}
%   enables or disables line breaks inside the logo. This setting is
%   refined by options \xoption{hyphenbreak}, \xoption{spacebreak}
%   or \xoption{discretionarybreak}.
%   Default is |false|.
% \entry{hyphenbreak}
%   enables or disables the line break right after the hyphen character.
% \entry{spacebreak}
%   enables or disables line breaks at space characters.
% \entry{discretionarybreak}
%   enables or disables line breaks at hyphenation points
%   (inserted by \cs{-}).
% \end{description}
% Macro \cs{hologoLogoSetup} also knows:
% \begin{description}
% \item[\xoption{variant}:]
%   This is a string option. It specifies a variant of a logo that
%   must exist. An empty string selects the package default variant.
% \end{description}
% Example:
% \begin{quote}
%   |\hologoSetup{break=false}|\\
%   |\hologoLogoSetup{plainTeX}{variant=hyphen,hyphenbreak}|\\
%   Then ``plain-\TeX'' contains one break point after the hyphen.
% \end{quote}
%
% \subsection{Driver options}
%
% Sometimes graphical operations are needed to construct some
% glyphs (e.g.\ \hologo{XeTeX}). If package \xpackage{graphics}
% or package \xpackage{pgf} are found, then the macros are taken
% from there. Otherwise the packge defines its own operations
% and therefore needs the driver information. Many drivers are
% detected automatically (\hologo{pdfTeX}/\hologo{LuaTeX}
% in PDF mode, \hologo{XeTeX}, \hologo{VTeX}). These have precedence
% over a driver option. The driver can be given as package option
% or using \cs{hologoDriverSetup}.
% The following list contains the recognized driver options:
% \begin{itemize}
% \item \xoption{pdftex}, \xoption{luatex}
% \item \xoption{dvipdfm}, \xoption{dvipdfmx}
% \item \xoption{dvips}, \xoption{dvipsone}, \xoption{xdvi}
% \item \xoption{xetex}
% \item \xoption{vtex}
% \end{itemize}
% The left driver of a line is the driver name that is used internally.
% The following names are aliases for drivers that use the
% same method. Therefore the entry in the \xext{log} file for
% the used driver prints the internally used driver name.
% \begin{description}
% \item[\xoption{driverfallback}:]
%   This option expects a driver that is used,
%   if the driver could not be detected automatically.
% \end{description}
%
% \begin{declcs}{hologoDriverSetup} \M{driver option}
% \end{declcs}
% The driver can also be configured after package loading
% using \cs{hologoDriverSetup}, also the way for \hologo{plainTeX}
% to setup the driver.
%
% \subsection{Font setup}
%
% Some logos require a special font, but should also be usable by
% \hologo{plainTeX}. Therefore the package provides some ways
% to influence the font settings. The options below
% take font settings as values. Both font commands
% such as \cs{sffamily} and macros that take one argument
% like \cs{textsf} can be used.
%
% \begin{declcs}{hologoFontSetup} \M{key value list}
% \end{declcs}
% Macro \cs{hologoFontSetup} sets the fonts for all logos.
% Supported keys:
% \begin{description}
% \def\entry#1{\item[\xoption{#1}:]}
% \entry{general}
%   This font is used for all logos. The default is empty.
%   That means no special font is used.
% \entry{bibsf}
%   This font is used for
%   {\hologoLogoSetup{BibTeX}{variant=sf}\hologo{BibTeX}}
%   with variant \xoption{sf}.
% \entry{rm}
%   This font is a serif font. It is used for \hologo{ExTeX}.
% \entry{sc}
%   This font specifies a small caps font. It is used for
%   {\hologoLogoSetup{BibTeX}{variant=sc}\hologo{BibTeX}}
%   with variant \xoption{sc}.
% \entry{sf}
%   This font specifies a sans serif font. The default
%   is \cs{sffamily}, then \cs{sf} is tried. Otherwise
%   a warning is given. It is used by \hologo{KOMAScript}.
% \entry{sy}
%   This is the font for math symbols (e.g. cmsy).
%   It is used by \hologo{AmS}, \hologo{NTS}, \hologo{ExTeX}.
% \entry{logo}
%   \hologo{METAFONT} and \hologo{METAPOST} are using that font.
%   In \hologo{LaTeX} \cs{logofamily} is used and
%   the definitions of package \xpackage{mflogo} are used
%   if the package is not loaded.
%   Otherwise the \cs{tenlogo} is used and defined
%   if it does not already exists.
% \end{description}
%
% \begin{declcs}{hologoLogoFontSetup} \M{logo} \M{key value list}
% \end{declcs}
% Fonts can also be set for a logo or logo component separately,
% see the following list.
% The keys are the same as for \cs{hologoFontSetup}.
%
% \begin{longtable}{>{\ttfamily}l>{\sffamily}ll}
%   \meta{logo} & keys & result\\
%   \hline
%   \endhead
%   BibTeX & bibsf & {\hologoLogoSetup{BibTeX}{variant=sf}\hologo{BibTeX}}\\[.5ex]
%   BibTeX & sc & {\hologoLogoSetup{BibTeX}{variant=sc}\hologo{BibTeX}}\\[.5ex]
%   ExTeX & rm & \hologo{ExTeX}\\
%   SliTeX & rm & \hologo{SliTeX}\\[.5ex]
%   AmS & sy & \hologo{AmS}\\
%   ExTeX & sy & \hologo{ExTeX}\\
%   NTS & sy & \hologo{NTS}\\[.5ex]
%   KOMAScript & sf & \hologo{KOMAScript}\\[.5ex]
%   METAFONT & logo & \hologo{METAFONT}\\
%   METAPOST & logo & \hologo{METAPOST}\\[.5ex]
%   SliTeX & sc \hologo{SliTeX}
% \end{longtable}
%
% \subsubsection{Font order}
%
% For all logos the font \xoption{general} is applied first.
% Example:
%\begin{quote}
%|\hologoFontSetup{general=\color{red}}|
%\end{quote}
% will print red logos.
% Then if the font uses a special font \xoption{sf}, for example,
% the font is applied that is setup by \cs{hologoLogoFontSetup}.
% If this font is not setup, then the common font setup
% by \cs{hologoFontSetup} is used. Otherwise a warning is given,
% that there is no font configured.
%
% \subsection{Additional user macros}
%
% Usually a variant of a logo is configured by using
% \cs{hologoLogoSetup}, because it is bad style to mix
% different variants of the same logo in the same text.
% There the following macros are a convenience for testing.
%
% \begin{declcs}{hologoVariant} \M{name} \M{variant}\\
%   \cs{HologoVariant} \M{name} \M{variant}
% \end{declcs}
% Logo \meta{name} is set using \meta{variant} that specifies
% explicitely which variant of the macro is used. If the argument
% is empty, then the default form of the logo is used
% (configurable by \cs{hologoLogoSetup}).
%
% \cs{HologoVariant} is used if the logo is set in a context
% that needs an uppercase first letter (beginning of a sentence, \dots).
%
% \begin{declcs}{hologoList}\\
%   \cs{hologoEntry} \M{logo} \M{variant} \M{since}
% \end{declcs}
% Macro \cs{hologoList} contains all logos that are provided
% by the package including variants. The list consists of calls
% of \cs{hologoEntry} with three arguments starting with the
% logo name \meta{logo} and its variant \meta{variant}. An empty
% variant means the current default. Argument \meta{since} specifies
% with version of the package \xpackage{hologo} is needed to get
% the logo. If the logo is fixed, then the date gets updated.
% Therefore the date \meta{since} is not exactly the date of
% the first introduction, but rather the date of the latest fix.
%
% Before \cs{hologoList} can be used, macro \cs{hologoEntry} needs
% a definition. The example file in section \ref{sec:example}
% shows applications of \cs{hologoList}.
%
% \subsection{Supported contexts}
%
% Macros \cs{hologo} and friends support special contexts:
% \begin{itemize}
% \item \hologo{LaTeX}'s protection mechanism.
% \item Bookmarks of package \xpackage{hyperref}.
% \item Package \xpackage{tex4ht}.
% \item The macros can be used inside \cs{csname} constructs,
%   if \cs{ifincsname} is available (\hologo{pdfTeX}, \hologo{XeTeX},
%   \hologo{LuaTeX}).
% \end{itemize}
%
% \subsection{Example}
% \label{sec:example}
%
% The following example prints the logos in different fonts.
%    \begin{macrocode}
%<*example>
%<<verbatim
\NeedsTeXFormat{LaTeX2e}
\documentclass[a4paper]{article}
\usepackage[
  hmargin=20mm,
  vmargin=20mm,
]{geometry}
\pagestyle{empty}
\usepackage{hologo}[2016/05/12]
\usepackage{longtable}
\usepackage{array}
\setlength{\extrarowheight}{2pt}
\usepackage[T1]{fontenc}
\usepackage{lmodern}
\usepackage{pdflscape}
\usepackage[
  pdfencoding=auto,
]{hyperref}
\hypersetup{
  pdfauthor={Heiko Oberdiek},
  pdftitle={Example for package `hologo'},
  pdfsubject={Logos with fonts lmr, lmss, qtm, qpl, qhv},
}
\usepackage{bookmark}

% Print the logo list on the console

\begingroup
  \typeout{}%
  \typeout{*** Begin of logo list ***}%
  \newcommand*{\hologoEntry}[3]{%
    \typeout{#1 \ifx\\#2\\\else(#2) \fi[#3]}%
  }%
  \hologoList
  \typeout{*** End of logo list ***}%
  \typeout{}%
\endgroup

\begin{document}
\begin{landscape}

  \section{Example file for package `hologo'}

  % Table for font names

  \begin{longtable}{>{\bfseries}ll}
    \textbf{font} & \textbf{Font name}\\
    \hline
    lmr & Latin Modern Roman\\
    lmss & Latin Modern Sans\\
    qtm & \TeX\ Gyre Termes\\
    qhv & \TeX\ Gyre Heros\\
    qpl & \TeX\ Gyre Pagella\\
  \end{longtable}

  % Logo list with logos in different fonts

  \begingroup
    \newcommand*{\SetVariant}[2]{%
      \ifx\\#2\\%
      \else
        \hologoLogoSetup{#1}{variant=#2}%
      \fi
    }%
    \newcommand*{\hologoEntry}[3]{%
      \SetVariant{#1}{#2}%
      \raisebox{1em}[0pt][0pt]{\hypertarget{#1@#2}{}}%
      \bookmark[%
        dest={#1@#2},%
      ]{%
        #1\ifx\\#2\\\else\space(#2)\fi: \Hologo{#1}, \hologo{#1} %
        [Unicode]%
      }%
      \hypersetup{unicode=false}%
      \bookmark[%
        dest={#1@#2},%
      ]{%
        #1\ifx\\#2\\\else\space(#2)\fi: \Hologo{#1}, \hologo{#1} %
        [PDFDocEncoding]%
      }%
      \texttt{#1}%
      &%
      \texttt{#2}%
      &%
      \Hologo{#1}%
      &%
      \SetVariant{#1}{#2}%
      \hologo{#1}%
      &%
      \SetVariant{#1}{#2}%
      \fontfamily{qtm}\selectfont
      \hologo{#1}%
      &%
      \SetVariant{#1}{#2}%
      \fontfamily{qpl}\selectfont
      \hologo{#1}%
      &%
      \SetVariant{#1}{#2}%
      \textsf{\hologo{#1}}%
      &%
      \SetVariant{#1}{#2}%
      \fontfamily{qhv}\selectfont
      \hologo{#1}%
      \tabularnewline
    }%
    \begin{longtable}{llllllll}%
      \textbf{\textit{logo}} & \textbf{\textit{variant}} &
      \texttt{\string\Hologo} &
      \textbf{lmr} & \textbf{qtm} & \textbf{qpl} &
      \textbf{lmss} & \textbf{qhv}
      \tabularnewline
      \hline
      \endhead
      \hologoList
    \end{longtable}%
  \endgroup

\end{landscape}
\end{document}
%verbatim
%</example>
%    \end{macrocode}
%
% \StopEventually{
% }
%
% \section{Implementation}
%    \begin{macrocode}
%<*package>
%    \end{macrocode}
%    Reload check, especially if the package is not used with \LaTeX.
%    \begin{macrocode}
\begingroup\catcode61\catcode48\catcode32=10\relax%
  \catcode13=5 % ^^M
  \endlinechar=13 %
  \catcode35=6 % #
  \catcode39=12 % '
  \catcode44=12 % ,
  \catcode45=12 % -
  \catcode46=12 % .
  \catcode58=12 % :
  \catcode64=11 % @
  \catcode123=1 % {
  \catcode125=2 % }
  \expandafter\let\expandafter\x\csname ver@hologo.sty\endcsname
  \ifx\x\relax % plain-TeX, first loading
  \else
    \def\empty{}%
    \ifx\x\empty % LaTeX, first loading,
      % variable is initialized, but \ProvidesPackage not yet seen
    \else
      \expandafter\ifx\csname PackageInfo\endcsname\relax
        \def\x#1#2{%
          \immediate\write-1{Package #1 Info: #2.}%
        }%
      \else
        \def\x#1#2{\PackageInfo{#1}{#2, stopped}}%
      \fi
      \x{hologo}{The package is already loaded}%
      \aftergroup\endinput
    \fi
  \fi
\endgroup%
%    \end{macrocode}
%    Package identification:
%    \begin{macrocode}
\begingroup\catcode61\catcode48\catcode32=10\relax%
  \catcode13=5 % ^^M
  \endlinechar=13 %
  \catcode35=6 % #
  \catcode39=12 % '
  \catcode40=12 % (
  \catcode41=12 % )
  \catcode44=12 % ,
  \catcode45=12 % -
  \catcode46=12 % .
  \catcode47=12 % /
  \catcode58=12 % :
  \catcode64=11 % @
  \catcode91=12 % [
  \catcode93=12 % ]
  \catcode123=1 % {
  \catcode125=2 % }
  \expandafter\ifx\csname ProvidesPackage\endcsname\relax
    \def\x#1#2#3[#4]{\endgroup
      \immediate\write-1{Package: #3 #4}%
      \xdef#1{#4}%
    }%
  \else
    \def\x#1#2[#3]{\endgroup
      #2[{#3}]%
      \ifx#1\@undefined
        \xdef#1{#3}%
      \fi
      \ifx#1\relax
        \xdef#1{#3}%
      \fi
    }%
  \fi
\expandafter\x\csname ver@hologo.sty\endcsname
\ProvidesPackage{hologo}%
  [2016/05/12 v1.11 A logo collection with bookmark support (HO)]%
%    \end{macrocode}
%
%    \begin{macrocode}
\begingroup\catcode61\catcode48\catcode32=10\relax%
  \catcode13=5 % ^^M
  \endlinechar=13 %
  \catcode123=1 % {
  \catcode125=2 % }
  \catcode64=11 % @
  \def\x{\endgroup
    \expandafter\edef\csname HOLOGO@AtEnd\endcsname{%
      \endlinechar=\the\endlinechar\relax
      \catcode13=\the\catcode13\relax
      \catcode32=\the\catcode32\relax
      \catcode35=\the\catcode35\relax
      \catcode61=\the\catcode61\relax
      \catcode64=\the\catcode64\relax
      \catcode123=\the\catcode123\relax
      \catcode125=\the\catcode125\relax
    }%
  }%
\x\catcode61\catcode48\catcode32=10\relax%
\catcode13=5 % ^^M
\endlinechar=13 %
\catcode35=6 % #
\catcode64=11 % @
\catcode123=1 % {
\catcode125=2 % }
\def\TMP@EnsureCode#1#2{%
  \edef\HOLOGO@AtEnd{%
    \HOLOGO@AtEnd
    \catcode#1=\the\catcode#1\relax
  }%
  \catcode#1=#2\relax
}
\TMP@EnsureCode{10}{12}% ^^J
\TMP@EnsureCode{33}{12}% !
\TMP@EnsureCode{34}{12}% "
\TMP@EnsureCode{36}{3}% $
\TMP@EnsureCode{38}{4}% &
\TMP@EnsureCode{39}{12}% '
\TMP@EnsureCode{40}{12}% (
\TMP@EnsureCode{41}{12}% )
\TMP@EnsureCode{42}{12}% *
\TMP@EnsureCode{43}{12}% +
\TMP@EnsureCode{44}{12}% ,
\TMP@EnsureCode{45}{12}% -
\TMP@EnsureCode{46}{12}% .
\TMP@EnsureCode{47}{12}% /
\TMP@EnsureCode{58}{12}% :
\TMP@EnsureCode{59}{12}% ;
\TMP@EnsureCode{60}{12}% <
\TMP@EnsureCode{62}{12}% >
\TMP@EnsureCode{63}{12}% ?
\TMP@EnsureCode{91}{12}% [
\TMP@EnsureCode{93}{12}% ]
\TMP@EnsureCode{94}{7}% ^ (superscript)
\TMP@EnsureCode{95}{8}% _ (subscript)
\TMP@EnsureCode{96}{12}% `
\TMP@EnsureCode{124}{12}% |
\edef\HOLOGO@AtEnd{%
  \HOLOGO@AtEnd
  \escapechar\the\escapechar\relax
  \noexpand\endinput
}
\escapechar=92 %
%    \end{macrocode}
%
% \subsection{Logo list}
%
%    \begin{macro}{\hologoList}
%    \begin{macrocode}
\def\hologoList{%
  \hologoEntry{(La)TeX}{}{2011/10/01}%
  \hologoEntry{AmSLaTeX}{}{2010/04/16}%
  \hologoEntry{AmSTeX}{}{2010/04/16}%
  \hologoEntry{biber}{}{2011/10/01}%
  \hologoEntry{BibTeX}{}{2011/10/01}%
  \hologoEntry{BibTeX}{sf}{2011/10/01}%
  \hologoEntry{BibTeX}{sc}{2011/10/01}%
  \hologoEntry{BibTeX8}{}{2011/11/22}%
  \hologoEntry{ConTeXt}{}{2011/03/25}%
  \hologoEntry{ConTeXt}{narrow}{2011/03/25}%
  \hologoEntry{ConTeXt}{simple}{2011/03/25}%
  \hologoEntry{emTeX}{}{2010/04/26}%
  \hologoEntry{eTeX}{}{2010/04/08}%
  \hologoEntry{ExTeX}{}{2011/10/01}%
  \hologoEntry{HanTheThanh}{}{2011/11/29}%
  \hologoEntry{iniTeX}{}{2011/10/01}%
  \hologoEntry{KOMAScript}{}{2011/10/01}%
  \hologoEntry{La}{}{2010/05/08}%
  \hologoEntry{LaTeX}{}{2010/04/08}%
  \hologoEntry{LaTeX2e}{}{2010/04/08}%
  \hologoEntry{LaTeX3}{}{2010/04/24}%
  \hologoEntry{LaTeXe}{}{2010/04/08}%
  \hologoEntry{LaTeXML}{}{2011/11/22}%
  \hologoEntry{LaTeXTeX}{}{2011/10/01}%
  \hologoEntry{LuaLaTeX}{}{2010/04/08}%
  \hologoEntry{LuaTeX}{}{2010/04/08}%
  \hologoEntry{LyX}{}{2011/10/01}%
  \hologoEntry{METAFONT}{}{2011/10/01}%
  \hologoEntry{MetaFun}{}{2011/10/01}%
  \hologoEntry{METAPOST}{}{2011/10/01}%
  \hologoEntry{MetaPost}{}{2011/10/01}%
  \hologoEntry{MiKTeX}{}{2011/10/01}%
  \hologoEntry{NTS}{}{2011/10/01}%
  \hologoEntry{OzMF}{}{2011/10/01}%
  \hologoEntry{OzMP}{}{2011/10/01}%
  \hologoEntry{OzTeX}{}{2011/10/01}%
  \hologoEntry{OzTtH}{}{2011/10/01}%
  \hologoEntry{PCTeX}{}{2011/10/01}%
  \hologoEntry{pdfTeX}{}{2011/10/01}%
  \hologoEntry{pdfLaTeX}{}{2011/10/01}%
  \hologoEntry{PiC}{}{2011/10/01}%
  \hologoEntry{PiCTeX}{}{2011/10/01}%
  \hologoEntry{plainTeX}{}{2010/04/08}%
  \hologoEntry{plainTeX}{space}{2010/04/16}%
  \hologoEntry{plainTeX}{hyphen}{2010/04/16}%
  \hologoEntry{plainTeX}{runtogether}{2010/04/16}%
  \hologoEntry{SageTeX}{}{2011/11/22}%
  \hologoEntry{SLiTeX}{}{2011/10/01}%
  \hologoEntry{SLiTeX}{lift}{2011/10/01}%
  \hologoEntry{SLiTeX}{narrow}{2011/10/01}%
  \hologoEntry{SLiTeX}{simple}{2011/10/01}%
  \hologoEntry{SliTeX}{}{2011/10/01}%
  \hologoEntry{SliTeX}{narrow}{2011/10/01}%
  \hologoEntry{SliTeX}{simple}{2011/10/01}%
  \hologoEntry{SliTeX}{lift}{2011/10/01}%
  \hologoEntry{teTeX}{}{2011/10/01}%
  \hologoEntry{TeX}{}{2010/04/08}%
  \hologoEntry{TeX4ht}{}{2011/11/22}%
  \hologoEntry{TTH}{}{2011/11/22}%
  \hologoEntry{virTeX}{}{2011/10/01}%
  \hologoEntry{VTeX}{}{2010/04/24}%
  \hologoEntry{Xe}{}{2010/04/08}%
  \hologoEntry{XeLaTeX}{}{2010/04/08}%
  \hologoEntry{XeTeX}{}{2010/04/08}%
}
%    \end{macrocode}
%    \end{macro}
%
% \subsection{Load resources}
%
%    \begin{macrocode}
\begingroup\expandafter\expandafter\expandafter\endgroup
\expandafter\ifx\csname RequirePackage\endcsname\relax
  \def\TMP@RequirePackage#1[#2]{%
    \begingroup\expandafter\expandafter\expandafter\endgroup
    \expandafter\ifx\csname ver@#1.sty\endcsname\relax
      \input #1.sty\relax
    \fi
  }%
  \TMP@RequirePackage{ltxcmds}[2011/02/04]%
  \TMP@RequirePackage{infwarerr}[2010/04/08]%
  \TMP@RequirePackage{kvsetkeys}[2010/03/01]%
  \TMP@RequirePackage{kvdefinekeys}[2010/03/01]%
  \TMP@RequirePackage{pdftexcmds}[2010/04/01]%
  \TMP@RequirePackage{ifpdf}[2010/01/28]%
  \TMP@RequirePackage{ifluatex}[2010/03/01]%
  \ltx@IfUndefined{newif}{%
    \expandafter\let\csname newif\endcsname\ltx@newif
  }{}%
  \TMP@RequirePackage{ifxetex}[2009/01/23]%
  \TMP@RequirePackage{ifvtex}[2010/03/01]%
\else
  \RequirePackage{ltxcmds}[2011/02/04]%
  \RequirePackage{infwarerr}[2010/04/08]%
  \RequirePackage{kvsetkeys}[2010/03/01]%
  \RequirePackage{kvdefinekeys}[2010/03/01]%
  \RequirePackage{pdftexcmds}[2010/04/01]%
  \RequirePackage{ifpdf}[2010/01/28]%
  \RequirePackage{ifluatex}[2010/03/01]%
  \RequirePackage{ifxetex}[2009/01/23]%
  \RequirePackage{ifvtex}[2010/03/01]%
\fi
%    \end{macrocode}
%
%    \begin{macro}{\HOLOGO@IfDefined}
%    \begin{macrocode}
\def\HOLOGO@IfExists#1{%
  \ifx\@undefined#1%
    \expandafter\ltx@secondoftwo
  \else
    \ifx\relax#1%
      \expandafter\ltx@secondoftwo
    \else
      \expandafter\expandafter\expandafter\ltx@firstoftwo
    \fi
  \fi
}
%    \end{macrocode}
%    \end{macro}
%
% \subsection{Setup macros}
%
%    \begin{macro}{\hologoSetup}
%    \begin{macrocode}
\def\hologoSetup{%
  \let\HOLOGO@name\relax
  \HOLOGO@Setup
}
%    \end{macrocode}
%    \end{macro}
%
%    \begin{macro}{\hologoLogoSetup}
%    \begin{macrocode}
\def\hologoLogoSetup#1{%
  \edef\HOLOGO@name{#1}%
  \ltx@IfUndefined{HoLogo@\HOLOGO@name}{%
    \@PackageError{hologo}{%
      Unknown logo `\HOLOGO@name'%
    }\@ehc
    \ltx@gobble
  }{%
    \HOLOGO@Setup
  }%
}
%    \end{macrocode}
%    \end{macro}
%
%    \begin{macro}{\HOLOGO@Setup}
%    \begin{macrocode}
\def\HOLOGO@Setup{%
  \kvsetkeys{HoLogo}%
}
%    \end{macrocode}
%    \end{macro}
%
% \subsection{Options}
%
%    \begin{macro}{\HOLOGO@DeclareBoolOption}
%    \begin{macrocode}
\def\HOLOGO@DeclareBoolOption#1{%
  \expandafter\chardef\csname HOLOGOOPT@#1\endcsname\ltx@zero
  \kv@define@key{HoLogo}{#1}[true]{%
    \def\HOLOGO@temp{##1}%
    \ifx\HOLOGO@temp\HOLOGO@true
      \ifx\HOLOGO@name\relax
        \expandafter\chardef\csname HOLOGOOPT@#1\endcsname=\ltx@one
      \else
        \expandafter\chardef\csname
        HoLogoOpt@#1@\HOLOGO@name\endcsname\ltx@one
      \fi
      \HOLOGO@SetBreakAll{#1}%
    \else
      \ifx\HOLOGO@temp\HOLOGO@false
        \ifx\HOLOGO@name\relax
          \expandafter\chardef\csname HOLOGOOPT@#1\endcsname=\ltx@zero
        \else
          \expandafter\chardef\csname
          HoLogoOpt@#1@\HOLOGO@name\endcsname=\ltx@zero
        \fi
        \HOLOGO@SetBreakAll{#1}%
      \else
        \@PackageError{hologo}{%
          Unknown value `##1' for boolean option `#1'.\MessageBreak
          Known values are `true' and `false'%
        }\@ehc
      \fi
    \fi
  }%
}
%    \end{macrocode}
%    \end{macro}
%
%    \begin{macro}{\HOLOGO@SetBreakAll}
%    \begin{macrocode}
\def\HOLOGO@SetBreakAll#1{%
  \def\HOLOGO@temp{#1}%
  \ifx\HOLOGO@temp\HOLOGO@break
    \ifx\HOLOGO@name\relax
      \chardef\HOLOGOOPT@hyphenbreak=\HOLOGOOPT@break
      \chardef\HOLOGOOPT@spacebreak=\HOLOGOOPT@break
      \chardef\HOLOGOOPT@discretionarybreak=\HOLOGOOPT@break
    \else
      \expandafter\chardef
         \csname HoLogoOpt@hyphenbreak@\HOLOGO@name\endcsname=%
         \csname HoLogoOpt@break@\HOLOGO@name\endcsname
      \expandafter\chardef
         \csname HoLogoOpt@spacebreak@\HOLOGO@name\endcsname=%
         \csname HoLogoOpt@break@\HOLOGO@name\endcsname
      \expandafter\chardef
         \csname HoLogoOpt@discretionarybreak@\HOLOGO@name
             \endcsname=%
         \csname HoLogoOpt@break@\HOLOGO@name\endcsname
    \fi
  \fi
}
%    \end{macrocode}
%    \end{macro}
%
%    \begin{macro}{\HOLOGO@true}
%    \begin{macrocode}
\def\HOLOGO@true{true}
%    \end{macrocode}
%    \end{macro}
%    \begin{macro}{\HOLOGO@false}
%    \begin{macrocode}
\def\HOLOGO@false{false}
%    \end{macrocode}
%    \end{macro}
%    \begin{macro}{\HOLOGO@break}
%    \begin{macrocode}
\def\HOLOGO@break{break}
%    \end{macrocode}
%    \end{macro}
%
%    \begin{macrocode}
\HOLOGO@DeclareBoolOption{break}
\HOLOGO@DeclareBoolOption{hyphenbreak}
\HOLOGO@DeclareBoolOption{spacebreak}
\HOLOGO@DeclareBoolOption{discretionarybreak}
%    \end{macrocode}
%
%    \begin{macrocode}
\kv@define@key{HoLogo}{variant}{%
  \ifx\HOLOGO@name\relax
    \@PackageError{hologo}{%
      Option `variant' is not available in \string\hologoSetup,%
      \MessageBreak
      Use \string\hologoLogoSetup\space instead%
    }\@ehc
  \else
    \edef\HOLOGO@temp{#1}%
    \ifx\HOLOGO@temp\ltx@empty
      \expandafter
      \let\csname HoLogoOpt@variant@\HOLOGO@name\endcsname\@undefined
    \else
      \ltx@IfUndefined{HoLogo@\HOLOGO@name @\HOLOGO@temp}{%
        \@PackageError{hologo}{%
          Unknown variant `\HOLOGO@temp' of logo `\HOLOGO@name'%
        }\@ehc
      }{%
        \expandafter
        \let\csname HoLogoOpt@variant@\HOLOGO@name\endcsname
            \HOLOGO@temp
      }%
    \fi
  \fi
}
%    \end{macrocode}
%
%    \begin{macro}{\HOLOGO@Variant}
%    \begin{macrocode}
\def\HOLOGO@Variant#1{%
  #1%
  \ltx@ifundefined{HoLogoOpt@variant@#1}{%
  }{%
    @\csname HoLogoOpt@variant@#1\endcsname
  }%
}
%    \end{macrocode}
%    \end{macro}
%
% \subsection{Break/no-break support}
%
%    \begin{macro}{\HOLOGO@space}
%    \begin{macrocode}
\def\HOLOGO@space{%
  \ltx@ifundefined{HoLogoOpt@spacebreak@\HOLOGO@name}{%
    \ltx@ifundefined{HoLogoOpt@break@\HOLOGO@name}{%
      \chardef\HOLOGO@temp=\HOLOGOOPT@spacebreak
    }{%
      \chardef\HOLOGO@temp=%
        \csname HoLogoOpt@break@\HOLOGO@name\endcsname
    }%
  }{%
    \chardef\HOLOGO@temp=%
      \csname HoLogoOpt@spacebreak@\HOLOGO@name\endcsname
  }%
  \ifcase\HOLOGO@temp
    \penalty10000 %
  \fi
  \ltx@space
}
%    \end{macrocode}
%    \end{macro}
%
%    \begin{macro}{\HOLOGO@hyphen}
%    \begin{macrocode}
\def\HOLOGO@hyphen{%
  \ltx@ifundefined{HoLogoOpt@hyphenbreak@\HOLOGO@name}{%
    \ltx@ifundefined{HoLogoOpt@break@\HOLOGO@name}{%
      \chardef\HOLOGO@temp=\HOLOGOOPT@hyphenbreak
    }{%
      \chardef\HOLOGO@temp=%
        \csname HoLogoOpt@break@\HOLOGO@name\endcsname
    }%
  }{%
    \chardef\HOLOGO@temp=%
      \csname HoLogoOpt@hyphenbreak@\HOLOGO@name\endcsname
  }%
  \ifcase\HOLOGO@temp
    \ltx@mbox{-}%
  \else
    -%
  \fi
}
%    \end{macrocode}
%    \end{macro}
%
%    \begin{macro}{\HOLOGO@discretionary}
%    \begin{macrocode}
\def\HOLOGO@discretionary{%
  \ltx@ifundefined{HoLogoOpt@discretionarybreak@\HOLOGO@name}{%
    \ltx@ifundefined{HoLogoOpt@break@\HOLOGO@name}{%
      \chardef\HOLOGO@temp=\HOLOGOOPT@discretionarybreak
    }{%
      \chardef\HOLOGO@temp=%
        \csname HoLogoOpt@break@\HOLOGO@name\endcsname
    }%
  }{%
    \chardef\HOLOGO@temp=%
      \csname HoLogoOpt@discretionarybreak@\HOLOGO@name\endcsname
  }%
  \ifcase\HOLOGO@temp
  \else
    \-%
  \fi
}
%    \end{macrocode}
%    \end{macro}
%
%    \begin{macro}{\HOLOGO@mbox}
%    \begin{macrocode}
\def\HOLOGO@mbox#1{%
  \ltx@ifundefined{HoLogoOpt@break@\HOLOGO@name}{%
    \chardef\HOLOGO@temp=\HOLOGOOPT@hyphenbreak
  }{%
    \chardef\HOLOGO@temp=%
      \csname HoLogoOpt@break@\HOLOGO@name\endcsname
  }%
  \ifcase\HOLOGO@temp
    \ltx@mbox{#1}%
  \else
    #1%
  \fi
}
%    \end{macrocode}
%    \end{macro}
%
% \subsection{Font support}
%
%    \begin{macro}{\HoLogoFont@font}
%    \begin{tabular}{@{}ll@{}}
%    |#1|:& logo name\\
%    |#2|:& font short name\\
%    |#3|:& text
%    \end{tabular}
%    \begin{macrocode}
\def\HoLogoFont@font#1#2#3{%
  \begingroup
    \ltx@IfUndefined{HoLogoFont@logo@#1.#2}{%
      \ltx@IfUndefined{HoLogoFont@font@#2}{%
        \@PackageWarning{hologo}{%
          Missing font `#2' for logo `#1'%
        }%
        #3%
      }{%
        \csname HoLogoFont@font@#2\endcsname{#3}%
      }%
    }{%
      \csname HoLogoFont@logo@#1.#2\endcsname{#3}%
    }%
  \endgroup
}
%    \end{macrocode}
%    \end{macro}
%
%    \begin{macro}{\HoLogoFont@Def}
%    \begin{macrocode}
\def\HoLogoFont@Def#1{%
  \expandafter\def\csname HoLogoFont@font@#1\endcsname
}
%    \end{macrocode}
%    \end{macro}
%    \begin{macro}{\HoLogoFont@LogoDef}
%    \begin{macrocode}
\def\HoLogoFont@LogoDef#1#2{%
  \expandafter\def\csname HoLogoFont@logo@#1.#2\endcsname
}
%    \end{macrocode}
%    \end{macro}
%
% \subsubsection{Font defaults}
%
%    \begin{macro}{\HoLogoFont@font@general}
%    \begin{macrocode}
\HoLogoFont@Def{general}{}%
%    \end{macrocode}
%    \end{macro}
%
%    \begin{macro}{\HoLogoFont@font@rm}
%    \begin{macrocode}
\ltx@IfUndefined{rmfamily}{%
  \ltx@IfUndefined{rm}{%
  }{%
    \HoLogoFont@Def{rm}{\rm}%
  }%
}{%
  \HoLogoFont@Def{rm}{\rmfamily}%
}
%    \end{macrocode}
%    \end{macro}
%
%    \begin{macro}{\HoLogoFont@font@sf}
%    \begin{macrocode}
\ltx@IfUndefined{sffamily}{%
  \ltx@IfUndefined{sf}{%
  }{%
    \HoLogoFont@Def{sf}{\sf}%
  }%
}{%
  \HoLogoFont@Def{sf}{\sffamily}%
}
%    \end{macrocode}
%    \end{macro}
%
%    \begin{macro}{\HoLogoFont@font@bibsf}
%    In case of \hologo{plainTeX} the original small caps
%    variant is used as default. In \hologo{LaTeX}
%    the definition of package \xpackage{dtklogos} \cite{dtklogos}
%    is used.
%\begin{quote}
%\begin{verbatim}
%\DeclareRobustCommand{\BibTeX}{%
%  B%
%  \kern-.05em%
%  \hbox{%
%    $\m@th$% %% force math size calculations
%    \csname S@\f@size\endcsname
%    \fontsize\sf@size\z@
%    \math@fontsfalse
%    \selectfont
%    I%
%    \kern-.025em%
%    B
%  }%
%  \kern-.08em%
%  \-%
%  \TeX
%}
%\end{verbatim}
%\end{quote}
%    \begin{macrocode}
\ltx@IfUndefined{selectfont}{%
  \ltx@IfUndefined{tensc}{%
    \font\tensc=cmcsc10\relax
  }{}%
  \HoLogoFont@Def{bibsf}{\tensc}%
}{%
  \HoLogoFont@Def{bibsf}{%
    $\mathsurround=0pt$%
    \csname S@\f@size\endcsname
    \fontsize\sf@size{0pt}%
    \math@fontsfalse
    \selectfont
  }%
}
%    \end{macrocode}
%    \end{macro}
%
%    \begin{macro}{\HoLogoFont@font@sc}
%    \begin{macrocode}
\ltx@IfUndefined{scshape}{%
  \ltx@IfUndefined{tensc}{%
    \font\tensc=cmcsc10\relax
  }{}%
  \HoLogoFont@Def{sc}{\tensc}%
}{%
  \HoLogoFont@Def{sc}{\scshape}%
}
%    \end{macrocode}
%    \end{macro}
%
%    \begin{macro}{\HoLogoFont@font@sy}
%    \begin{macrocode}
\ltx@IfUndefined{usefont}{%
  \ltx@IfUndefined{tensy}{%
  }{%
    \HoLogoFont@Def{sy}{\tensy}%
  }%
}{%
  \HoLogoFont@Def{sy}{%
    \usefont{OMS}{cmsy}{m}{n}%
  }%
}
%    \end{macrocode}
%    \end{macro}
%
%    \begin{macro}{\HoLogoFont@font@logo}
%    \begin{macrocode}
\begingroup
  \def\x{LaTeX2e}%
\expandafter\endgroup
\ifx\fmtname\x
  \ltx@IfUndefined{logofamily}{%
    \DeclareRobustCommand\logofamily{%
      \not@math@alphabet\logofamily\relax
      \fontencoding{U}%
      \fontfamily{logo}%
      \selectfont
    }%
  }{}%
  \ltx@IfUndefined{logofamily}{%
  }{%
    \HoLogoFont@Def{logo}{\logofamily}%
  }%
\else
  \ltx@IfUndefined{tenlogo}{%
    \font\tenlogo=logo10\relax
  }{}%
  \HoLogoFont@Def{logo}{\tenlogo}%
\fi
%    \end{macrocode}
%    \end{macro}
%
% \subsubsection{Font setup}
%
%    \begin{macro}{\hologoFontSetup}
%    \begin{macrocode}
\def\hologoFontSetup{%
  \let\HOLOGO@name\relax
  \HOLOGO@FontSetup
}
%    \end{macrocode}
%    \end{macro}
%
%    \begin{macro}{\hologoLogoFontSetup}
%    \begin{macrocode}
\def\hologoLogoFontSetup#1{%
  \edef\HOLOGO@name{#1}%
  \ltx@IfUndefined{HoLogo@\HOLOGO@name}{%
    \@PackageError{hologo}{%
      Unknown logo `\HOLOGO@name'%
    }\@ehc
    \ltx@gobble
  }{%
    \HOLOGO@FontSetup
  }%
}
%    \end{macrocode}
%    \end{macro}
%
%    \begin{macro}{\HOLOGO@FontSetup}
%    \begin{macrocode}
\def\HOLOGO@FontSetup{%
  \kvsetkeys{HoLogoFont}%
}
%    \end{macrocode}
%    \end{macro}
%
%    \begin{macrocode}
\def\HOLOGO@temp#1{%
  \kv@define@key{HoLogoFont}{#1}{%
    \ifx\HOLOGO@name\relax
      \HoLogoFont@Def{#1}{##1}%
    \else
      \HoLogoFont@LogoDef\HOLOGO@name{#1}{##1}%
    \fi
  }%
}
\HOLOGO@temp{general}
\HOLOGO@temp{sf}
%    \end{macrocode}
%
% \subsection{Generic logo commands}
%
%    \begin{macrocode}
\HOLOGO@IfExists\hologo{%
  \@PackageError{hologo}{%
    \string\hologo\ltx@space is already defined.\MessageBreak
    Package loading is aborted%
  }\@ehc
  \HOLOGO@AtEnd
}%
\HOLOGO@IfExists\hologoRobust{%
  \@PackageError{hologo}{%
    \string\hologoRobust\ltx@space is already defined.\MessageBreak
    Package loading is aborted%
  }\@ehc
  \HOLOGO@AtEnd
}%
%    \end{macrocode}
%
% \subsubsection{\cs{hologo} and friends}
%
%    \begin{macrocode}
\ifluatex
  \expandafter\ltx@firstofone
\else
  \expandafter\ltx@gobble
\fi
{%
  \ltx@IfUndefined{ifincsname}{%
    \ifnum\luatexversion<36 %
      \expandafter\ltx@gobble
    \else
      \expandafter\ltx@firstofone
    \fi
    {%
      \begingroup
        \ifcase0%
            \directlua{%
              if tex.enableprimitives then %
                tex.enableprimitives('HOLOGO@', {'ifincsname'})%
              else %
                tex.print('1')%
              end%
            }%
            \ifx\HOLOGO@ifincsname\@undefined 1\fi%
            \relax
          \expandafter\ltx@firstofone
        \else
          \endgroup
          \expandafter\ltx@gobble
        \fi
        {%
          \global\let\ifincsname\HOLOGO@ifincsname
        }%
      \HOLOGO@temp
    }%
  }{}%
}
%    \end{macrocode}
%    \begin{macrocode}
\ltx@IfUndefined{ifincsname}{%
  \catcode`$=14 %
}{%
  \catcode`$=9 %
}
%    \end{macrocode}
%
%    \begin{macro}{\hologo}
%    \begin{macrocode}
\def\hologo#1{%
$ \ifincsname
$   \ltx@ifundefined{HoLogoCs@\HOLOGO@Variant{#1}}{%
$     #1%
$   }{%
$     \csname HoLogoCs@\HOLOGO@Variant{#1}\endcsname\ltx@firstoftwo
$   }%
$ \else
    \HOLOGO@IfExists\texorpdfstring\texorpdfstring\ltx@firstoftwo
    {%
      \hologoRobust{#1}%
    }{%
      \ltx@ifundefined{HoLogoBkm@\HOLOGO@Variant{#1}}{%
        \ltx@ifundefined{HoLogo@#1}{?#1?}{#1}%
      }{%
        \csname HoLogoBkm@\HOLOGO@Variant{#1}\endcsname
        \ltx@firstoftwo
      }%
    }%
$ \fi
}
%    \end{macrocode}
%    \end{macro}
%    \begin{macro}{\Hologo}
%    \begin{macrocode}
\def\Hologo#1{%
$ \ifincsname
$   \ltx@ifundefined{HoLogoCs@\HOLOGO@Variant{#1}}{%
$     #1%
$   }{%
$     \csname HoLogoCs@\HOLOGO@Variant{#1}\endcsname\ltx@secondoftwo
$   }%
$ \else
    \HOLOGO@IfExists\texorpdfstring\texorpdfstring\ltx@firstoftwo
    {%
      \HologoRobust{#1}%
    }{%
      \ltx@ifundefined{HoLogoBkm@\HOLOGO@Variant{#1}}{%
        \ltx@ifundefined{HoLogo@#1}{?#1?}{#1}%
      }{%
        \csname HoLogoBkm@\HOLOGO@Variant{#1}\endcsname
        \ltx@secondoftwo
      }%
    }%
$ \fi
}
%    \end{macrocode}
%    \end{macro}
%
%    \begin{macro}{\hologoVariant}
%    \begin{macrocode}
\def\hologoVariant#1#2{%
  \ifx\relax#2\relax
    \hologo{#1}%
  \else
$   \ifincsname
$     \ltx@ifundefined{HoLogoCs@#1@#2}{%
$       #1%
$     }{%
$       \csname HoLogoCs@#1@#2\endcsname\ltx@firstoftwo
$     }%
$   \else
      \HOLOGO@IfExists\texorpdfstring\texorpdfstring\ltx@firstoftwo
      {%
        \hologoVariantRobust{#1}{#2}%
      }{%
        \ltx@ifundefined{HoLogoBkm@#1@#2}{%
          \ltx@ifundefined{HoLogo@#1}{?#1?}{#1}%
        }{%
          \csname HoLogoBkm@#1@#2\endcsname
          \ltx@firstoftwo
        }%
      }%
$   \fi
  \fi
}
%    \end{macrocode}
%    \end{macro}
%    \begin{macro}{\HologoVariant}
%    \begin{macrocode}
\def\HologoVariant#1#2{%
  \ifx\relax#2\relax
    \Hologo{#1}%
  \else
$   \ifincsname
$     \ltx@ifundefined{HoLogoCs@#1@#2}{%
$       #1%
$     }{%
$       \csname HoLogoCs@#1@#2\endcsname\ltx@secondoftwo
$     }%
$   \else
      \HOLOGO@IfExists\texorpdfstring\texorpdfstring\ltx@firstoftwo
      {%
        \HologoVariantRobust{#1}{#2}%
      }{%
        \ltx@ifundefined{HoLogoBkm@#1@#2}{%
          \ltx@ifundefined{HoLogo@#1}{?#1?}{#1}%
        }{%
          \csname HoLogoBkm@#1@#2\endcsname
          \ltx@secondoftwo
        }%
      }%
$   \fi
  \fi
}
%    \end{macrocode}
%    \end{macro}
%
%    \begin{macrocode}
\catcode`\$=3 %
%    \end{macrocode}
%
% \subsubsection{\cs{hologoRobust} and friends}
%
%    \begin{macro}{\hologoRobust}
%    \begin{macrocode}
\ltx@IfUndefined{protected}{%
  \ltx@IfUndefined{DeclareRobustCommand}{%
    \def\hologoRobust#1%
  }{%
    \DeclareRobustCommand*\hologoRobust[1]%
  }%
}{%
  \protected\def\hologoRobust#1%
}%
{%
  \edef\HOLOGO@name{#1}%
  \ltx@IfUndefined{HoLogo@\HOLOGO@Variant\HOLOGO@name}{%
    \@PackageError{hologo}{%
      Unknown logo `\HOLOGO@name'%
    }\@ehc
    ?\HOLOGO@name?%
  }{%
    \ltx@IfUndefined{ver@tex4ht.sty}{%
      \HoLogoFont@font\HOLOGO@name{general}{%
        \csname HoLogo@\HOLOGO@Variant\HOLOGO@name\endcsname
        \ltx@firstoftwo
      }%
    }{%
      \ltx@IfUndefined{HoLogoHtml@\HOLOGO@Variant\HOLOGO@name}{%
        \HOLOGO@name
      }{%
        \csname HoLogoHtml@\HOLOGO@Variant\HOLOGO@name\endcsname
        \ltx@firstoftwo
      }%
    }%
  }%
}
%    \end{macrocode}
%    \end{macro}
%    \begin{macro}{\HologoRobust}
%    \begin{macrocode}
\ltx@IfUndefined{protected}{%
  \ltx@IfUndefined{DeclareRobustCommand}{%
    \def\HologoRobust#1%
  }{%
    \DeclareRobustCommand*\HologoRobust[1]%
  }%
}{%
  \protected\def\HologoRobust#1%
}%
{%
  \edef\HOLOGO@name{#1}%
  \ltx@IfUndefined{HoLogo@\HOLOGO@Variant\HOLOGO@name}{%
    \@PackageError{hologo}{%
      Unknown logo `\HOLOGO@name'%
    }\@ehc
    ?\HOLOGO@name?%
  }{%
    \ltx@IfUndefined{ver@tex4ht.sty}{%
      \HoLogoFont@font\HOLOGO@name{general}{%
        \csname HoLogo@\HOLOGO@Variant\HOLOGO@name\endcsname
        \ltx@secondoftwo
      }%
    }{%
      \ltx@IfUndefined{HoLogoHtml@\HOLOGO@Variant\HOLOGO@name}{%
        \expandafter\HOLOGO@Uppercase\HOLOGO@name
      }{%
        \csname HoLogoHtml@\HOLOGO@Variant\HOLOGO@name\endcsname
        \ltx@secondoftwo
      }%
    }%
  }%
}
%    \end{macrocode}
%    \end{macro}
%    \begin{macro}{\hologoVariantRobust}
%    \begin{macrocode}
\ltx@IfUndefined{protected}{%
  \ltx@IfUndefined{DeclareRobustCommand}{%
    \def\hologoVariantRobust#1#2%
  }{%
    \DeclareRobustCommand*\hologoVariantRobust[2]%
  }%
}{%
  \protected\def\hologoVariantRobust#1#2%
}%
{%
  \begingroup
    \hologoLogoSetup{#1}{variant={#2}}%
    \hologoRobust{#1}%
  \endgroup
}
%    \end{macrocode}
%    \end{macro}
%    \begin{macro}{\HologoVariantRobust}
%    \begin{macrocode}
\ltx@IfUndefined{protected}{%
  \ltx@IfUndefined{DeclareRobustCommand}{%
    \def\HologoVariantRobust#1#2%
  }{%
    \DeclareRobustCommand*\HologoVariantRobust[2]%
  }%
}{%
  \protected\def\HologoVariantRobust#1#2%
}%
{%
  \begingroup
    \hologoLogoSetup{#1}{variant={#2}}%
    \HologoRobust{#1}%
  \endgroup
}
%    \end{macrocode}
%    \end{macro}
%
%    \begin{macro}{\hologorobust}
%    Macro \cs{hologorobust} is only defined for compatibility.
%    Its use is deprecated.
%    \begin{macrocode}
\def\hologorobust{\hologoRobust}
%    \end{macrocode}
%    \end{macro}
%
% \subsection{Helpers}
%
%    \begin{macro}{\HOLOGO@Uppercase}
%    Macro \cs{HOLOGO@Uppercase} is restricted to \cs{uppercase},
%    because \hologo{plainTeX} or \hologo{iniTeX} do not provide
%    \cs{MakeUppercase}.
%    \begin{macrocode}
\def\HOLOGO@Uppercase#1{\uppercase{#1}}
%    \end{macrocode}
%    \end{macro}
%
%    \begin{macro}{\HOLOGO@PdfdocUnicode}
%    \begin{macrocode}
\def\HOLOGO@PdfdocUnicode{%
  \ifx\ifHy@unicode\iftrue
    \expandafter\ltx@secondoftwo
  \else
    \expandafter\ltx@firstoftwo
  \fi
}
%    \end{macrocode}
%    \end{macro}
%
%    \begin{macro}{\HOLOGO@Math}
%    \begin{macrocode}
\def\HOLOGO@MathSetup{%
  \mathsurround0pt\relax
  \HOLOGO@IfExists\f@series{%
    \if b\expandafter\ltx@car\f@series x\@nil
      \csname boldmath\endcsname
   \fi
  }{}%
}
%    \end{macrocode}
%    \end{macro}
%
%    \begin{macro}{\HOLOGO@TempDimen}
%    \begin{macrocode}
\dimendef\HOLOGO@TempDimen=\ltx@zero
%    \end{macrocode}
%    \end{macro}
%    \begin{macro}{\HOLOGO@NegativeKerning}
%    \begin{macrocode}
\def\HOLOGO@NegativeKerning#1{%
  \begingroup
    \HOLOGO@TempDimen=0pt\relax
    \comma@parse@normalized{#1}{%
      \ifdim\HOLOGO@TempDimen=0pt %
        \expandafter\HOLOGO@@NegativeKerning\comma@entry
      \fi
      \ltx@gobble
    }%
    \ifdim\HOLOGO@TempDimen<0pt %
      \kern\HOLOGO@TempDimen
    \fi
  \endgroup
}
%    \end{macrocode}
%    \end{macro}
%    \begin{macro}{\HOLOGO@@NegativeKerning}
%    \begin{macrocode}
\def\HOLOGO@@NegativeKerning#1#2{%
  \setbox\ltx@zero\hbox{#1#2}%
  \HOLOGO@TempDimen=\wd\ltx@zero
  \setbox\ltx@zero\hbox{#1\kern0pt#2}%
  \advance\HOLOGO@TempDimen by -\wd\ltx@zero
}
%    \end{macrocode}
%    \end{macro}
%
%    \begin{macro}{\HOLOGO@SpaceFactor}
%    \begin{macrocode}
\def\HOLOGO@SpaceFactor{%
  \spacefactor1000 %
}
%    \end{macrocode}
%    \end{macro}
%
%    \begin{macro}{\HOLOGO@Span}
%    \begin{macrocode}
\def\HOLOGO@Span#1#2{%
  \HCode{<span class="HoLogo-#1">}%
  #2%
  \HCode{</span>}%
}
%    \end{macrocode}
%    \end{macro}
%
% \subsubsection{Text subscript}
%
%    \begin{macro}{\HOLOGO@SubScript}%
%    \begin{macrocode}
\def\HOLOGO@SubScript#1{%
  \ltx@IfUndefined{textsubscript}{%
    \ltx@IfUndefined{text}{%
      \ltx@mbox{%
        \mathsurround=0pt\relax
        $%
          _{%
            \ltx@IfUndefined{sf@size}{%
              \mathrm{#1}%
            }{%
              \mbox{%
                \fontsize\sf@size{0pt}\selectfont
                #1%
              }%
            }%
          }%
        $%
      }%
    }{%
      \ltx@mbox{%
        \mathsurround=0pt\relax
        $_{\text{#1}}$%
      }%
    }%
  }{%
    \textsubscript{#1}%
  }%
}
%    \end{macrocode}
%    \end{macro}
%
% \subsection{\hologo{TeX} and friends}
%
% \subsubsection{\hologo{TeX}}
%
%    \begin{macro}{\HoLogo@TeX}
%    Source: \hologo{LaTeX} kernel.
%    \begin{macrocode}
\def\HoLogo@TeX#1{%
  T\kern-.1667em\lower.5ex\hbox{E}\kern-.125emX\HOLOGO@SpaceFactor
}
%    \end{macrocode}
%    \end{macro}
%    \begin{macro}{\HoLogoHtml@TeX}
%    \begin{macrocode}
\def\HoLogoHtml@TeX#1{%
  \HoLogoCss@TeX
  \HOLOGO@Span{TeX}{%
    T%
    \HOLOGO@Span{e}{%
      E%
    }%
    X%
  }%
}
%    \end{macrocode}
%    \end{macro}
%    \begin{macro}{\HoLogoCss@TeX}
%    \begin{macrocode}
\def\HoLogoCss@TeX{%
  \Css{%
    span.HoLogo-TeX span.HoLogo-e{%
      position:relative;%
      top:.5ex;%
      margin-left:-.1667em;%
      margin-right:-.125em;%
    }%
  }%
  \Css{%
    a span.HoLogo-TeX span.HoLogo-e{%
      text-decoration:none;%
    }%
  }%
  \global\let\HoLogoCss@TeX\relax
}
%    \end{macrocode}
%    \end{macro}
%
% \subsubsection{\hologo{plainTeX}}
%
%    \begin{macro}{\HoLogo@plainTeX@space}
%    Source: ``The \hologo{TeX}book''
%    \begin{macrocode}
\def\HoLogo@plainTeX@space#1{%
  \HOLOGO@mbox{#1{p}{P}lain}\HOLOGO@space\hologo{TeX}%
}
%    \end{macrocode}
%    \end{macro}
%    \begin{macro}{\HoLogoCs@plainTeX@space}
%    \begin{macrocode}
\def\HoLogoCs@plainTeX@space#1{#1{p}{P}lain TeX}%
%    \end{macrocode}
%    \end{macro}
%    \begin{macro}{\HoLogoBkm@plainTeX@space}
%    \begin{macrocode}
\def\HoLogoBkm@plainTeX@space#1{%
  #1{p}{P}lain \hologo{TeX}%
}
%    \end{macrocode}
%    \end{macro}
%    \begin{macro}{\HoLogoHtml@plainTeX@space}
%    \begin{macrocode}
\def\HoLogoHtml@plainTeX@space#1{%
  #1{p}{P}lain \hologo{TeX}%
}
%    \end{macrocode}
%    \end{macro}
%
%    \begin{macro}{\HoLogo@plainTeX@hyphen}
%    \begin{macrocode}
\def\HoLogo@plainTeX@hyphen#1{%
  \HOLOGO@mbox{#1{p}{P}lain}\HOLOGO@hyphen\hologo{TeX}%
}
%    \end{macrocode}
%    \end{macro}
%    \begin{macro}{\HoLogoCs@plainTeX@hyphen}
%    \begin{macrocode}
\def\HoLogoCs@plainTeX@hyphen#1{#1{p}{P}lain-TeX}
%    \end{macrocode}
%    \end{macro}
%    \begin{macro}{\HoLogoBkm@plainTeX@hyphen}
%    \begin{macrocode}
\def\HoLogoBkm@plainTeX@hyphen#1{%
  #1{p}{P}lain-\hologo{TeX}%
}
%    \end{macrocode}
%    \end{macro}
%    \begin{macro}{\HoLogoHtml@plainTeX@hyphen}
%    \begin{macrocode}
\def\HoLogoHtml@plainTeX@hyphen#1{%
  #1{p}{P}lain-\hologo{TeX}%
}
%    \end{macrocode}
%    \end{macro}
%
%    \begin{macro}{\HoLogo@plainTeX@runtogether}
%    \begin{macrocode}
\def\HoLogo@plainTeX@runtogether#1{%
  \HOLOGO@mbox{#1{p}{P}lain\hologo{TeX}}%
}
%    \end{macrocode}
%    \end{macro}
%    \begin{macro}{\HoLogoCs@plainTeX@runtogether}
%    \begin{macrocode}
\def\HoLogoCs@plainTeX@runtogether#1{#1{p}{P}lainTeX}
%    \end{macrocode}
%    \end{macro}
%    \begin{macro}{\HoLogoBkm@plainTeX@runtogether}
%    \begin{macrocode}
\def\HoLogoBkm@plainTeX@runtogether#1{%
  #1{p}{P}lain\hologo{TeX}%
}
%    \end{macrocode}
%    \end{macro}
%    \begin{macro}{\HoLogoHtml@plainTeX@runtogether}
%    \begin{macrocode}
\def\HoLogoHtml@plainTeX@runtogether#1{%
  #1{p}{P}lain\hologo{TeX}%
}
%    \end{macrocode}
%    \end{macro}
%
%    \begin{macro}{\HoLogo@plainTeX}
%    \begin{macrocode}
\def\HoLogo@plainTeX{\HoLogo@plainTeX@space}
%    \end{macrocode}
%    \end{macro}
%    \begin{macro}{\HoLogoCs@plainTeX}
%    \begin{macrocode}
\def\HoLogoCs@plainTeX{\HoLogoCs@plainTeX@space}
%    \end{macrocode}
%    \end{macro}
%    \begin{macro}{\HoLogoBkm@plainTeX}
%    \begin{macrocode}
\def\HoLogoBkm@plainTeX{\HoLogoBkm@plainTeX@space}
%    \end{macrocode}
%    \end{macro}
%    \begin{macro}{\HoLogoHtml@plainTeX}
%    \begin{macrocode}
\def\HoLogoHtml@plainTeX{\HoLogoHtml@plainTeX@space}
%    \end{macrocode}
%    \end{macro}
%
% \subsubsection{\hologo{LaTeX}}
%
%    Source: \hologo{LaTeX} kernel.
%\begin{quote}
%\begin{verbatim}
%\DeclareRobustCommand{\LaTeX}{%
%  L%
%  \kern-.36em%
%  {%
%    \sbox\z@ T%
%    \vbox to\ht\z@{%
%      \hbox{%
%        \check@mathfonts
%        \fontsize\sf@size\z@
%        \math@fontsfalse
%        \selectfont
%        A%
%      }%
%      \vss
%    }%
%  }%
%  \kern-.15em%
%  \TeX
%}
%\end{verbatim}
%\end{quote}
%
%    \begin{macro}{\HoLogo@La}
%    \begin{macrocode}
\def\HoLogo@La#1{%
  L%
  \kern-.36em%
  \begingroup
    \setbox\ltx@zero\hbox{T}%
    \vbox to\ht\ltx@zero{%
      \hbox{%
        \ltx@ifundefined{check@mathfonts}{%
          \csname sevenrm\endcsname
        }{%
          \check@mathfonts
          \fontsize\sf@size{0pt}%
          \math@fontsfalse\selectfont
        }%
        A%
      }%
      \vss
    }%
  \endgroup
}
%    \end{macrocode}
%    \end{macro}
%
%    \begin{macro}{\HoLogo@LaTeX}
%    Source: \hologo{LaTeX} kernel.
%    \begin{macrocode}
\def\HoLogo@LaTeX#1{%
  \hologo{La}%
  \kern-.15em%
  \hologo{TeX}%
}
%    \end{macrocode}
%    \end{macro}
%    \begin{macro}{\HoLogoHtml@LaTeX}
%    \begin{macrocode}
\def\HoLogoHtml@LaTeX#1{%
  \HoLogoCss@LaTeX
  \HOLOGO@Span{LaTeX}{%
    L%
    \HOLOGO@Span{a}{%
      A%
    }%
    \hologo{TeX}%
  }%
}
%    \end{macrocode}
%    \end{macro}
%    \begin{macro}{\HoLogoCss@LaTeX}
%    \begin{macrocode}
\def\HoLogoCss@LaTeX{%
  \Css{%
    span.HoLogo-LaTeX span.HoLogo-a{%
      position:relative;%
      top:-.5ex;%
      margin-left:-.36em;%
      margin-right:-.15em;%
      font-size:85\%;%
    }%
  }%
  \global\let\HoLogoCss@LaTeX\relax
}
%    \end{macrocode}
%    \end{macro}
%
% \subsubsection{\hologo{(La)TeX}}
%
%    \begin{macro}{\HoLogo@LaTeXTeX}
%    The kerning around the parentheses is taken
%    from package \xpackage{dtklogos} \cite{dtklogos}.
%\begin{quote}
%\begin{verbatim}
%\DeclareRobustCommand{\LaTeXTeX}{%
%  (%
%  \kern-.15em%
%  L%
%  \kern-.36em%
%  {%
%    \sbox\z@ T%
%    \vbox to\ht0{%
%      \hbox{%
%        $\m@th$%
%        \csname S@\f@size\endcsname
%        \fontsize\sf@size\z@
%        \math@fontsfalse
%        \selectfont
%        A%
%      }%
%      \vss
%    }%
%  }%
%  \kern-.2em%
%  )%
%  \kern-.15em%
%  \TeX
%}
%\end{verbatim}
%\end{quote}
%    \begin{macrocode}
\def\HoLogo@LaTeXTeX#1{%
  (%
  \kern-.15em%
  \hologo{La}%
  \kern-.2em%
  )%
  \kern-.15em%
  \hologo{TeX}%
}
%    \end{macrocode}
%    \end{macro}
%    \begin{macro}{\HoLogoBkm@LaTeXTeX}
%    \begin{macrocode}
\def\HoLogoBkm@LaTeXTeX#1{(La)TeX}
%    \end{macrocode}
%    \end{macro}
%
%    \begin{macro}{\HoLogo@(La)TeX}
%    \begin{macrocode}
\expandafter
\let\csname HoLogo@(La)TeX\endcsname\HoLogo@LaTeXTeX
%    \end{macrocode}
%    \end{macro}
%    \begin{macro}{\HoLogoBkm@(La)TeX}
%    \begin{macrocode}
\expandafter
\let\csname HoLogoBkm@(La)TeX\endcsname\HoLogoBkm@LaTeXTeX
%    \end{macrocode}
%    \end{macro}
%    \begin{macro}{\HoLogoHtml@LaTeXTeX}
%    \begin{macrocode}
\def\HoLogoHtml@LaTeXTeX#1{%
  \HoLogoCss@LaTeXTeX
  \HOLOGO@Span{LaTeXTeX}{%
    (%
    \HOLOGO@Span{L}{L}%
    \HOLOGO@Span{a}{A}%
    \HOLOGO@Span{ParenRight}{)}%
    \hologo{TeX}%
  }%
}
%    \end{macrocode}
%    \end{macro}
%    \begin{macro}{\HoLogoHtml@(La)TeX}
%    Kerning after opening parentheses and before closing parentheses
%    is $-0.1$\,em. The original values $-0.15$\,em
%    looked too ugly for a serif font.
%    \begin{macrocode}
\expandafter
\let\csname HoLogoHtml@(La)TeX\endcsname\HoLogoHtml@LaTeXTeX
%    \end{macrocode}
%    \end{macro}
%    \begin{macro}{\HoLogoCss@LaTeXTeX}
%    \begin{macrocode}
\def\HoLogoCss@LaTeXTeX{%
  \Css{%
    span.HoLogo-LaTeXTeX span.HoLogo-L{%
      margin-left:-.1em;%
    }%
  }%
  \Css{%
    span.HoLogo-LaTeXTeX span.HoLogo-a{%
      position:relative;%
      top:-.5ex;%
      margin-left:-.36em;%
      margin-right:-.1em;%
      font-size:85\%;%
    }%
  }%
  \Css{%
    span.HoLogo-LaTeXTeX span.HoLogo-ParenRight{%
      margin-right:-.15em;%
    }%
  }%
  \global\let\HoLogoCss@LaTeXTeX\relax
}
%    \end{macrocode}
%    \end{macro}
%
% \subsubsection{\hologo{LaTeXe}}
%
%    \begin{macro}{\HoLogo@LaTeXe}
%    Source: \hologo{LaTeX} kernel
%    \begin{macrocode}
\def\HoLogo@LaTeXe#1{%
  \hologo{LaTeX}%
  \kern.15em%
  \hbox{%
    \HOLOGO@MathSetup
    2%
    $_{\textstyle\varepsilon}$%
  }%
}
%    \end{macrocode}
%    \end{macro}
%
%    \begin{macro}{\HoLogoCs@LaTeXe}
%    \begin{macrocode}
\ifnum64=`\^^^^0040\relax % test for big chars of LuaTeX/XeTeX
  \catcode`\$=9 %
  \catcode`\&=14 %
\else
  \catcode`\$=14 %
  \catcode`\&=9 %
\fi
\def\HoLogoCs@LaTeXe#1{%
  LaTeX2%
$ \string ^^^^0395%
& e%
}%
\catcode`\$=3 %
\catcode`\&=4 %
%    \end{macrocode}
%    \end{macro}
%
%    \begin{macro}{\HoLogoBkm@LaTeXe}
%    \begin{macrocode}
\def\HoLogoBkm@LaTeXe#1{%
  \hologo{LaTeX}%
  2%
  \HOLOGO@PdfdocUnicode{e}{\textepsilon}%
}
%    \end{macrocode}
%    \end{macro}
%
%    \begin{macro}{\HoLogoHtml@LaTeXe}
%    \begin{macrocode}
\def\HoLogoHtml@LaTeXe#1{%
  \HoLogoCss@LaTeXe
  \HOLOGO@Span{LaTeX2e}{%
    \hologo{LaTeX}%
    \HOLOGO@Span{2}{2}%
    \HOLOGO@Span{e}{%
      \HOLOGO@MathSetup
      \ensuremath{\textstyle\varepsilon}%
    }%
  }%
}
%    \end{macrocode}
%    \end{macro}
%    \begin{macro}{\HoLogoCss@LaTeXe}
%    \begin{macrocode}
\def\HoLogoCss@LaTeXe{%
  \Css{%
    span.HoLogo-LaTeX2e span.HoLogo-2{%
      padding-left:.15em;%
    }%
  }%
  \Css{%
    span.HoLogo-LaTeX2e span.HoLogo-e{%
      position:relative;%
      top:.35ex;%
      text-decoration:none;%
    }%
  }%
  \global\let\HoLogoCss@LaTeXe\relax
}
%    \end{macrocode}
%    \end{macro}
%
%    \begin{macro}{\HoLogo@LaTeX2e}
%    \begin{macrocode}
\expandafter
\let\csname HoLogo@LaTeX2e\endcsname\HoLogo@LaTeXe
%    \end{macrocode}
%    \end{macro}
%    \begin{macro}{\HoLogoCs@LaTeX2e}
%    \begin{macrocode}
\expandafter
\let\csname HoLogoCs@LaTeX2e\endcsname\HoLogoCs@LaTeXe
%    \end{macrocode}
%    \end{macro}
%    \begin{macro}{\HoLogoBkm@LaTeX2e}
%    \begin{macrocode}
\expandafter
\let\csname HoLogoBkm@LaTeX2e\endcsname\HoLogoBkm@LaTeXe
%    \end{macrocode}
%    \end{macro}
%    \begin{macro}{\HoLogoHtml@LaTeX2e}
%    \begin{macrocode}
\expandafter
\let\csname HoLogoHtml@LaTeX2e\endcsname\HoLogoHtml@LaTeXe
%    \end{macrocode}
%    \end{macro}
%
% \subsubsection{\hologo{LaTeX3}}
%
%    \begin{macro}{\HoLogo@LaTeX3}
%    Source: \hologo{LaTeX} kernel
%    \begin{macrocode}
\expandafter\def\csname HoLogo@LaTeX3\endcsname#1{%
  \hologo{LaTeX}%
  3%
}
%    \end{macrocode}
%    \end{macro}
%
%    \begin{macro}{\HoLogoBkm@LaTeX3}
%    \begin{macrocode}
\expandafter\def\csname HoLogoBkm@LaTeX3\endcsname#1{%
  \hologo{LaTeX}%
  3%
}
%    \end{macrocode}
%    \end{macro}
%    \begin{macro}{\HoLogoHtml@LaTeX3}
%    \begin{macrocode}
\expandafter
\let\csname HoLogoHtml@LaTeX3\expandafter\endcsname
\csname HoLogo@LaTeX3\endcsname
%    \end{macrocode}
%    \end{macro}
%
% \subsubsection{\hologo{LaTeXML}}
%
%    \begin{macro}{\HoLogo@LaTeXML}
%    \begin{macrocode}
\def\HoLogo@LaTeXML#1{%
  \HOLOGO@mbox{%
    \hologo{La}%
    \kern-.15em%
    T%
    \kern-.1667em%
    \lower.5ex\hbox{E}%
    \kern-.125em%
    \HoLogoFont@font{LaTeXML}{sc}{xml}%
  }%
}
%    \end{macrocode}
%    \end{macro}
%    \begin{macro}{\HoLogoHtml@pdfLaTeX}
%    \begin{macrocode}
\def\HoLogoHtml@LaTeXML#1{%
  \HOLOGO@Span{LaTeXML}{%
    \HoLogoCss@LaTeX
    \HoLogoCss@TeX
    \HOLOGO@Span{LaTeX}{%
      L%
      \HOLOGO@Span{a}{%
        A%
      }%
    }%
    \HOLOGO@Span{TeX}{%
      T%
      \HOLOGO@Span{e}{%
        E%
      }%
    }%
    \HCode{<span style="font-variant: small-caps;">}%
    xml%
    \HCode{</span>}%
  }%
}
%    \end{macrocode}
%    \end{macro}
%
% \subsubsection{\hologo{eTeX}}
%
%    \begin{macro}{\HoLogo@eTeX}
%    Source: package \xpackage{etex}
%    \begin{macrocode}
\def\HoLogo@eTeX#1{%
  \ltx@mbox{%
    \HOLOGO@MathSetup
    $\varepsilon$%
    -%
    \HOLOGO@NegativeKerning{-T,T-,To}%
    \hologo{TeX}%
  }%
}
%    \end{macrocode}
%    \end{macro}
%    \begin{macro}{\HoLogoCs@eTeX}
%    \begin{macrocode}
\ifnum64=`\^^^^0040\relax % test for big chars of LuaTeX/XeTeX
  \catcode`\$=9 %
  \catcode`\&=14 %
\else
  \catcode`\$=14 %
  \catcode`\&=9 %
\fi
\def\HoLogoCs@eTeX#1{%
$ #1{\string ^^^^0395}{\string ^^^^03b5}%
& #1{e}{E}%
  TeX%
}%
\catcode`\$=3 %
\catcode`\&=4 %
%    \end{macrocode}
%    \end{macro}
%    \begin{macro}{\HoLogoBkm@eTeX}
%    \begin{macrocode}
\def\HoLogoBkm@eTeX#1{%
  \HOLOGO@PdfdocUnicode{#1{e}{E}}{\textepsilon}%
  -%
  \hologo{TeX}%
}
%    \end{macrocode}
%    \end{macro}
%    \begin{macro}{\HoLogoHtml@eTeX}
%    \begin{macrocode}
\def\HoLogoHtml@eTeX#1{%
  \ltx@mbox{%
    \HOLOGO@MathSetup
    $\varepsilon$%
    -%
    \hologo{TeX}%
  }%
}
%    \end{macrocode}
%    \end{macro}
%
% \subsubsection{\hologo{iniTeX}}
%
%    \begin{macro}{\HoLogo@iniTeX}
%    \begin{macrocode}
\def\HoLogo@iniTeX#1{%
  \HOLOGO@mbox{%
    #1{i}{I}ni\hologo{TeX}%
  }%
}
%    \end{macrocode}
%    \end{macro}
%    \begin{macro}{\HoLogoCs@iniTeX}
%    \begin{macrocode}
\def\HoLogoCs@iniTeX#1{#1{i}{I}niTeX}
%    \end{macrocode}
%    \end{macro}
%    \begin{macro}{\HoLogoBkm@iniTeX}
%    \begin{macrocode}
\def\HoLogoBkm@iniTeX#1{%
  #1{i}{I}ni\hologo{TeX}%
}
%    \end{macrocode}
%    \end{macro}
%    \begin{macro}{\HoLogoHtml@iniTeX}
%    \begin{macrocode}
\let\HoLogoHtml@iniTeX\HoLogo@iniTeX
%    \end{macrocode}
%    \end{macro}
%
% \subsubsection{\hologo{virTeX}}
%
%    \begin{macro}{\HoLogo@virTeX}
%    \begin{macrocode}
\def\HoLogo@virTeX#1{%
  \HOLOGO@mbox{%
    #1{v}{V}ir\hologo{TeX}%
  }%
}
%    \end{macrocode}
%    \end{macro}
%    \begin{macro}{\HoLogoCs@virTeX}
%    \begin{macrocode}
\def\HoLogoCs@virTeX#1{#1{v}{V}irTeX}
%    \end{macrocode}
%    \end{macro}
%    \begin{macro}{\HoLogoBkm@virTeX}
%    \begin{macrocode}
\def\HoLogoBkm@virTeX#1{%
  #1{v}{V}ir\hologo{TeX}%
}
%    \end{macrocode}
%    \end{macro}
%    \begin{macro}{\HoLogoHtml@virTeX}
%    \begin{macrocode}
\let\HoLogoHtml@virTeX\HoLogo@virTeX
%    \end{macrocode}
%    \end{macro}
%
% \subsubsection{\hologo{SliTeX}}
%
% \paragraph{Definitions of the three variants.}
%
%    \begin{macro}{\HoLogo@SLiTeX@lift}
%    \begin{macrocode}
\def\HoLogo@SLiTeX@lift#1{%
  \HoLogoFont@font{SliTeX}{rm}{%
    S%
    \kern-.06em%
    L%
    \kern-.18em%
    \raise.32ex\hbox{\HoLogoFont@font{SliTeX}{sc}{i}}%
    \HOLOGO@discretionary
    \kern-.06em%
    \hologo{TeX}%
  }%
}
%    \end{macrocode}
%    \end{macro}
%    \begin{macro}{\HoLogoBkm@SLiTeX@lift}
%    \begin{macrocode}
\def\HoLogoBkm@SLiTeX@lift#1{SLiTeX}
%    \end{macrocode}
%    \end{macro}
%    \begin{macro}{\HoLogoHtml@SLiTeX@lift}
%    \begin{macrocode}
\def\HoLogoHtml@SLiTeX@lift#1{%
  \HoLogoCss@SLiTeX@lift
  \HOLOGO@Span{SLiTeX-lift}{%
    \HoLogoFont@font{SliTeX}{rm}{%
      S%
      \HOLOGO@Span{L}{L}%
      \HOLOGO@Span{i}{i}%
      \hologo{TeX}%
    }%
  }%
}
%    \end{macrocode}
%    \end{macro}
%    \begin{macro}{\HoLogoCss@SLiTeX@lift}
%    \begin{macrocode}
\def\HoLogoCss@SLiTeX@lift{%
  \Css{%
    span.HoLogo-SLiTeX-lift span.HoLogo-L{%
      margin-left:-.06em;%
      margin-right:-.18em;%
    }%
  }%
  \Css{%
    span.HoLogo-SLiTeX-lift span.HoLogo-i{%
      position:relative;%
      top:-.32ex;%
      margin-right:-.06em;%
      font-variant:small-caps;%
    }%
  }%
  \global\let\HoLogoCss@SLiTeX@lift\relax
}
%    \end{macrocode}
%    \end{macro}
%
%    \begin{macro}{\HoLogo@SliTeX@simple}
%    \begin{macrocode}
\def\HoLogo@SliTeX@simple#1{%
  \HoLogoFont@font{SliTeX}{rm}{%
    \ltx@mbox{%
      \HoLogoFont@font{SliTeX}{sc}{Sli}%
    }%
    \HOLOGO@discretionary
    \hologo{TeX}%
  }%
}
%    \end{macrocode}
%    \end{macro}
%    \begin{macro}{\HoLogoBkm@SliTeX@simple}
%    \begin{macrocode}
\def\HoLogoBkm@SliTeX@simple#1{SliTeX}
%    \end{macrocode}
%    \end{macro}
%    \begin{macro}{\HoLogoHtml@SliTeX@simple}
%    \begin{macrocode}
\let\HoLogoHtml@SliTeX@simple\HoLogo@SliTeX@simple
%    \end{macrocode}
%    \end{macro}
%
%    \begin{macro}{\HoLogo@SliTeX@narrow}
%    \begin{macrocode}
\def\HoLogo@SliTeX@narrow#1{%
  \HoLogoFont@font{SliTeX}{rm}{%
    \ltx@mbox{%
      S%
      \kern-.06em%
      \HoLogoFont@font{SliTeX}{sc}{%
        l%
        \kern-.035em%
        i%
      }%
    }%
    \HOLOGO@discretionary
    \kern-.06em%
    \hologo{TeX}%
  }%
}
%    \end{macrocode}
%    \end{macro}
%    \begin{macro}{\HoLogoBkm@SliTeX@narrow}
%    \begin{macrocode}
\def\HoLogoBkm@SliTeX@narrow#1{SliTeX}
%    \end{macrocode}
%    \end{macro}
%    \begin{macro}{\HoLogoHtml@SliTeX@narrow}
%    \begin{macrocode}
\def\HoLogoHtml@SliTeX@narrow#1{%
  \HoLogoCss@SliTeX@narrow
  \HOLOGO@Span{SliTeX-narrow}{%
    \HoLogoFont@font{SliTeX}{rm}{%
      S%
        \HOLOGO@Span{l}{l}%
        \HOLOGO@Span{i}{i}%
      \hologo{TeX}%
    }%
  }%
}
%    \end{macrocode}
%    \end{macro}
%    \begin{macro}{\HoLogoCss@SliTeX@narrow}
%    \begin{macrocode}
\def\HoLogoCss@SliTeX@narrow{%
  \Css{%
    span.HoLogo-SliTeX-narrow span.HoLogo-l{%
      margin-left:-.06em;%
      margin-right:-.035em;%
      font-variant:small-caps;%
    }%
  }%
  \Css{%
    span.HoLogo-SliTeX-narrow span.HoLogo-i{%
      margin-right:-.06em;%
      font-variant:small-caps;%
    }%
  }%
  \global\let\HoLogoCss@SliTeX@narrow\relax
}
%    \end{macrocode}
%    \end{macro}
%
% \paragraph{Macro set completion.}
%
%    \begin{macro}{\HoLogo@SLiTeX@simple}
%    \begin{macrocode}
\def\HoLogo@SLiTeX@simple{\HoLogo@SliTeX@simple}
%    \end{macrocode}
%    \end{macro}
%    \begin{macro}{\HoLogoBkm@SLiTeX@simple}
%    \begin{macrocode}
\def\HoLogoBkm@SLiTeX@simple{\HoLogoBkm@SliTeX@simple}
%    \end{macrocode}
%    \end{macro}
%    \begin{macro}{\HoLogoHtml@SLiTeX@simple}
%    \begin{macrocode}
\def\HoLogoHtml@SLiTeX@simple{\HoLogoHtml@SliTeX@simple}
%    \end{macrocode}
%    \end{macro}
%
%    \begin{macro}{\HoLogo@SLiTeX@narrow}
%    \begin{macrocode}
\def\HoLogo@SLiTeX@narrow{\HoLogo@SliTeX@narrow}
%    \end{macrocode}
%    \end{macro}
%    \begin{macro}{\HoLogoBkm@SLiTeX@narrow}
%    \begin{macrocode}
\def\HoLogoBkm@SLiTeX@narrow{\HoLogoBkm@SliTeX@narrow}
%    \end{macrocode}
%    \end{macro}
%    \begin{macro}{\HoLogoHtml@SLiTeX@narrow}
%    \begin{macrocode}
\def\HoLogoHtml@SLiTeX@narrow{\HoLogoHtml@SliTeX@narrow}
%    \end{macrocode}
%    \end{macro}
%
%    \begin{macro}{\HoLogo@SliTeX@lift}
%    \begin{macrocode}
\def\HoLogo@SliTeX@lift{\HoLogo@SLiTeX@lift}
%    \end{macrocode}
%    \end{macro}
%    \begin{macro}{\HoLogoBkm@SliTeX@lift}
%    \begin{macrocode}
\def\HoLogoBkm@SliTeX@lift{\HoLogoBkm@SLiTeX@lift}
%    \end{macrocode}
%    \end{macro}
%    \begin{macro}{\HoLogoHtml@SliTeX@lift}
%    \begin{macrocode}
\def\HoLogoHtml@SliTeX@lift{\HoLogoHtml@SLiTeX@lift}
%    \end{macrocode}
%    \end{macro}
%
% \paragraph{Defaults.}
%
%    \begin{macro}{\HoLogo@SLiTeX}
%    \begin{macrocode}
\def\HoLogo@SLiTeX{\HoLogo@SLiTeX@lift}
%    \end{macrocode}
%    \end{macro}
%    \begin{macro}{\HoLogoBkm@SLiTeX}
%    \begin{macrocode}
\def\HoLogoBkm@SLiTeX{\HoLogoBkm@SLiTeX@lift}
%    \end{macrocode}
%    \end{macro}
%    \begin{macro}{\HoLogoHtml@SLiTeX}
%    \begin{macrocode}
\def\HoLogoHtml@SLiTeX{\HoLogoHtml@SLiTeX@lift}
%    \end{macrocode}
%    \end{macro}
%
%    \begin{macro}{\HoLogo@SliTeX}
%    \begin{macrocode}
\def\HoLogo@SliTeX{\HoLogo@SliTeX@narrow}
%    \end{macrocode}
%    \end{macro}
%    \begin{macro}{\HoLogoBkm@SliTeX}
%    \begin{macrocode}
\def\HoLogoBkm@SliTeX{\HoLogoBkm@SliTeX@narrow}
%    \end{macrocode}
%    \end{macro}
%    \begin{macro}{\HoLogoHtml@SliTeX}
%    \begin{macrocode}
\def\HoLogoHtml@SliTeX{\HoLogoHtml@SliTeX@narrow}
%    \end{macrocode}
%    \end{macro}
%
% \subsubsection{\hologo{LuaTeX}}
%
%    \begin{macro}{\HoLogo@LuaTeX}
%    The kerning is an idea of Hans Hagen, see mailing list
%    `luatex at tug dot org' in March 2010.
%    \begin{macrocode}
\def\HoLogo@LuaTeX#1{%
  \HOLOGO@mbox{%
    Lua%
    \HOLOGO@NegativeKerning{aT,oT,To}%
    \hologo{TeX}%
  }%
}
%    \end{macrocode}
%    \end{macro}
%    \begin{macro}{\HoLogoHtml@LuaTeX}
%    \begin{macrocode}
\let\HoLogoHtml@LuaTeX\HoLogo@LuaTeX
%    \end{macrocode}
%    \end{macro}
%
% \subsubsection{\hologo{LuaLaTeX}}
%
%    \begin{macro}{\HoLogo@LuaLaTeX}
%    \begin{macrocode}
\def\HoLogo@LuaLaTeX#1{%
  \HOLOGO@mbox{%
    Lua%
    \hologo{LaTeX}%
  }%
}
%    \end{macrocode}
%    \end{macro}
%    \begin{macro}{\HoLogoHtml@LuaLaTeX}
%    \begin{macrocode}
\let\HoLogoHtml@LuaLaTeX\HoLogo@LuaLaTeX
%    \end{macrocode}
%    \end{macro}
%
% \subsubsection{\hologo{XeTeX}, \hologo{XeLaTeX}}
%
%    \begin{macro}{\HOLOGO@IfCharExists}
%    \begin{macrocode}
\ifluatex
  \ifnum\luatexversion<36 %
  \else
    \def\HOLOGO@IfCharExists#1{%
      \ifnum
        \directlua{%
           if luaotfload and luaotfload.aux then
             if luaotfload.aux.font_has_glyph(%
                    font.current(), \number#1) then % 	 
	       tex.print("1") % 	 
	     end % 	 
	   elseif font and font.fonts and font.current then %
            local f = font.fonts[font.current()]%
            if f.characters and f.characters[\number#1] then %
              tex.print("1")%
            end %
          end%
        }0=\ltx@zero
        \expandafter\ltx@secondoftwo
      \else
        \expandafter\ltx@firstoftwo
      \fi
    }%
  \fi
\fi
\ltx@IfUndefined{HOLOGO@IfCharExists}{%
  \def\HOLOGO@@IfCharExists#1{%
    \begingroup
      \tracinglostchars=\ltx@zero
      \setbox\ltx@zero=\hbox{%
        \kern7sp\char#1\relax
        \ifnum\lastkern>\ltx@zero
          \expandafter\aftergroup\csname iffalse\endcsname
        \else
          \expandafter\aftergroup\csname iftrue\endcsname
        \fi
      }%
      % \if{true|false} from \aftergroup
      \endgroup
      \expandafter\ltx@firstoftwo
    \else
      \endgroup
      \expandafter\ltx@secondoftwo
    \fi
  }%
  \ifxetex
    \ltx@IfUndefined{XeTeXfonttype}{}{%
      \ltx@IfUndefined{XeTeXcharglyph}{}{%
        \def\HOLOGO@IfCharExists#1{%
          \ifnum\XeTeXfonttype\font>\ltx@zero
            \expandafter\ltx@firstofthree
          \else
            \expandafter\ltx@gobble
          \fi
          {%
            \ifnum\XeTeXcharglyph#1>\ltx@zero
              \expandafter\ltx@firstoftwo
            \else
              \expandafter\ltx@secondoftwo
            \fi
          }%
          \HOLOGO@@IfCharExists{#1}%
        }%
      }%
    }%
  \fi
}{}
\ltx@ifundefined{HOLOGO@IfCharExists}{%
  \ifnum64=`\^^^^0040\relax % test for big chars of LuaTeX/XeTeX
    \let\HOLOGO@IfCharExists\HOLOGO@@IfCharExists
  \else
    \def\HOLOGO@IfCharExists#1{%
      \ifnum#1>255 %
        \expandafter\ltx@fourthoffour
      \fi
      \HOLOGO@@IfCharExists{#1}%
    }%
  \fi
}{}
%    \end{macrocode}
%    \end{macro}
%
%    \begin{macro}{\HoLogo@Xe}
%    Source: package \xpackage{dtklogos}
%    \begin{macrocode}
\def\HoLogo@Xe#1{%
  X%
  \kern-.1em\relax
  \HOLOGO@IfCharExists{"018E}{%
    \lower.5ex\hbox{\char"018E}%
  }{%
    \chardef\HOLOGO@choice=\ltx@zero
    \ifdim\fontdimen\ltx@one\font>0pt %
      \ltx@IfUndefined{rotatebox}{%
        \ltx@IfUndefined{pgftext}{%
          \ltx@IfUndefined{psscalebox}{%
            \ltx@IfUndefined{HOLOGO@ScaleBox@\hologoDriver}{%
            }{%
              \chardef\HOLOGO@choice=4 %
            }%
          }{%
            \chardef\HOLOGO@choice=3 %
          }%
        }{%
          \chardef\HOLOGO@choice=2 %
        }%
      }{%
        \chardef\HOLOGO@choice=1 %
      }%
      \ifcase\HOLOGO@choice
        \HOLOGO@WarningUnsupportedDriver{Xe}%
        e%
      \or % 1: \rotatebox
        \begingroup
          \setbox\ltx@zero\hbox{\rotatebox{180}{E}}%
          \ltx@LocDimenA=\dp\ltx@zero
          \advance\ltx@LocDimenA by -.5ex\relax
          \raise\ltx@LocDimenA\box\ltx@zero
        \endgroup
      \or % 2: \pgftext
        \lower.5ex\hbox{%
          \pgfpicture
            \pgftext[rotate=180]{E}%
          \endpgfpicture
        }%
      \or % 3: \psscalebox
        \begingroup
          \setbox\ltx@zero\hbox{\psscalebox{-1 -1}{E}}%
          \ltx@LocDimenA=\dp\ltx@zero
          \advance\ltx@LocDimenA by -.5ex\relax
          \raise\ltx@LocDimenA\box\ltx@zero
        \endgroup
      \or % 4: \HOLOGO@PointReflectBox
        \lower.5ex\hbox{\HOLOGO@PointReflectBox{E}}%
      \else
        \@PackageError{hologo}{Internal error (choice/it}\@ehc
      \fi
    \else
      \ltx@IfUndefined{reflectbox}{%
        \ltx@IfUndefined{pgftext}{%
          \ltx@IfUndefined{psscalebox}{%
            \ltx@IfUndefined{HOLOGO@ScaleBox@\hologoDriver}{%
            }{%
              \chardef\HOLOGO@choice=4 %
            }%
          }{%
            \chardef\HOLOGO@choice=3 %
          }%
        }{%
          \chardef\HOLOGO@choice=2 %
        }%
      }{%
        \chardef\HOLOGO@choice=1 %
      }%
      \ifcase\HOLOGO@choice
        \HOLOGO@WarningUnsupportedDriver{Xe}%
        e%
      \or % 1: reflectbox
        \lower.5ex\hbox{%
          \reflectbox{E}%
        }%
      \or % 2: \pgftext
        \lower.5ex\hbox{%
          \pgfpicture
            \pgftransformxscale{-1}%
            \pgftext{E}%
          \endpgfpicture
        }%
      \or % 3: \psscalebox
        \lower.5ex\hbox{%
          \psscalebox{-1 1}{E}%
        }%
      \or % 4: \HOLOGO@Reflectbox
        \lower.5ex\hbox{%
          \HOLOGO@ReflectBox{E}%
        }%
      \else
        \@PackageError{hologo}{Internal error (choice/up)}\@ehc
      \fi
    \fi
  }%
}
%    \end{macrocode}
%    \end{macro}
%    \begin{macro}{\HoLogoHtml@Xe}
%    \begin{macrocode}
\def\HoLogoHtml@Xe#1{%
  \HoLogoCss@Xe
  \HOLOGO@Span{Xe}{%
    X%
    \HOLOGO@Span{e}{%
      \HCode{&\ltx@hashchar x018e;}%
    }%
  }%
}
%    \end{macrocode}
%    \end{macro}
%    \begin{macro}{\HoLogoCss@Xe}
%    \begin{macrocode}
\def\HoLogoCss@Xe{%
  \Css{%
    span.HoLogo-Xe span.HoLogo-e{%
      position:relative;%
      top:.5ex;%
      left-margin:-.1em;%
    }%
  }%
  \global\let\HoLogoCss@Xe\relax
}
%    \end{macrocode}
%    \end{macro}
%
%    \begin{macro}{\HoLogo@XeTeX}
%    \begin{macrocode}
\def\HoLogo@XeTeX#1{%
  \hologo{Xe}%
  \kern-.15em\relax
  \hologo{TeX}%
}
%    \end{macrocode}
%    \end{macro}
%
%    \begin{macro}{\HoLogoHtml@XeTeX}
%    \begin{macrocode}
\def\HoLogoHtml@XeTeX#1{%
  \HoLogoCss@XeTeX
  \HOLOGO@Span{XeTeX}{%
    \hologo{Xe}%
    \hologo{TeX}%
  }%
}
%    \end{macrocode}
%    \end{macro}
%    \begin{macro}{\HoLogoCss@XeTeX}
%    \begin{macrocode}
\def\HoLogoCss@XeTeX{%
  \Css{%
    span.HoLogo-XeTeX span.HoLogo-TeX{%
      margin-left:-.15em;%
    }%
  }%
  \global\let\HoLogoCss@XeTeX\relax
}
%    \end{macrocode}
%    \end{macro}
%
%    \begin{macro}{\HoLogo@XeLaTeX}
%    \begin{macrocode}
\def\HoLogo@XeLaTeX#1{%
  \hologo{Xe}%
  \kern-.13em%
  \hologo{LaTeX}%
}
%    \end{macrocode}
%    \end{macro}
%    \begin{macro}{\HoLogoHtml@XeLaTeX}
%    \begin{macrocode}
\def\HoLogoHtml@XeLaTeX#1{%
  \HoLogoCss@XeLaTeX
  \HOLOGO@Span{XeLaTeX}{%
    \hologo{Xe}%
    \hologo{LaTeX}%
  }%
}
%    \end{macrocode}
%    \end{macro}
%    \begin{macro}{\HoLogoCss@XeLaTeX}
%    \begin{macrocode}
\def\HoLogoCss@XeLaTeX{%
  \Css{%
    span.HoLogo-XeLaTeX span.HoLogo-Xe{%
      margin-right:-.13em;%
    }%
  }%
  \global\let\HoLogoCss@XeLaTeX\relax
}
%    \end{macrocode}
%    \end{macro}
%
% \subsubsection{\hologo{pdfTeX}, \hologo{pdfLaTeX}}
%
%    \begin{macro}{\HoLogo@pdfTeX}
%    \begin{macrocode}
\def\HoLogo@pdfTeX#1{%
  \HOLOGO@mbox{%
    #1{p}{P}df\hologo{TeX}%
  }%
}
%    \end{macrocode}
%    \end{macro}
%    \begin{macro}{\HoLogoCs@pdfTeX}
%    \begin{macrocode}
\def\HoLogoCs@pdfTeX#1{#1{p}{P}dfTeX}
%    \end{macrocode}
%    \end{macro}
%    \begin{macro}{\HoLogoBkm@pdfTeX}
%    \begin{macrocode}
\def\HoLogoBkm@pdfTeX#1{%
  #1{p}{P}df\hologo{TeX}%
}
%    \end{macrocode}
%    \end{macro}
%    \begin{macro}{\HoLogoHtml@pdfTeX}
%    \begin{macrocode}
\let\HoLogoHtml@pdfTeX\HoLogo@pdfTeX
%    \end{macrocode}
%    \end{macro}
%
%    \begin{macro}{\HoLogo@pdfLaTeX}
%    \begin{macrocode}
\def\HoLogo@pdfLaTeX#1{%
  \HOLOGO@mbox{%
    #1{p}{P}df\hologo{LaTeX}%
  }%
}
%    \end{macrocode}
%    \end{macro}
%    \begin{macro}{\HoLogoCs@pdfLaTeX}
%    \begin{macrocode}
\def\HoLogoCs@pdfLaTeX#1{#1{p}{P}dfLaTeX}
%    \end{macrocode}
%    \end{macro}
%    \begin{macro}{\HoLogoBkm@pdfLaTeX}
%    \begin{macrocode}
\def\HoLogoBkm@pdfLaTeX#1{%
  #1{p}{P}df\hologo{LaTeX}%
}
%    \end{macrocode}
%    \end{macro}
%    \begin{macro}{\HoLogoHtml@pdfLaTeX}
%    \begin{macrocode}
\let\HoLogoHtml@pdfLaTeX\HoLogo@pdfLaTeX
%    \end{macrocode}
%    \end{macro}
%
% \subsubsection{\hologo{VTeX}}
%
%    \begin{macro}{\HoLogo@VTeX}
%    \begin{macrocode}
\def\HoLogo@VTeX#1{%
  \HOLOGO@mbox{%
    V\hologo{TeX}%
  }%
}
%    \end{macrocode}
%    \end{macro}
%    \begin{macro}{\HoLogoHtml@VTeX}
%    \begin{macrocode}
\let\HoLogoHtml@VTeX\HoLogo@VTeX
%    \end{macrocode}
%    \end{macro}
%
% \subsubsection{\hologo{AmS}, \dots}
%
%    Source: class \xclass{amsdtx}
%
%    \begin{macro}{\HoLogo@AmS}
%    \begin{macrocode}
\def\HoLogo@AmS#1{%
  \HoLogoFont@font{AmS}{sy}{%
    A%
    \kern-.1667em%
    \lower.5ex\hbox{M}%
    \kern-.125em%
    S%
  }%
}
%    \end{macrocode}
%    \end{macro}
%    \begin{macro}{\HoLogoBkm@AmS}
%    \begin{macrocode}
\def\HoLogoBkm@AmS#1{AmS}
%    \end{macrocode}
%    \end{macro}
%    \begin{macro}{\HoLogoHtml@AmS}
%    \begin{macrocode}
\def\HoLogoHtml@AmS#1{%
  \HoLogoCss@AmS
%  \HoLogoFont@font{AmS}{sy}{%
    \HOLOGO@Span{AmS}{%
      A%
      \HOLOGO@Span{M}{M}%
      S%
    }%
%   }%
}
%    \end{macrocode}
%    \end{macro}
%    \begin{macro}{\HoLogoCss@AmS}
%    \begin{macrocode}
\def\HoLogoCss@AmS{%
  \Css{%
    span.HoLogo-AmS span.HoLogo-M{%
      position:relative;%
      top:.5ex;%
      margin-left:-.1667em;%
      margin-right:-.125em;%
      text-decoration:none;%
    }%
  }%
  \global\let\HoLogoCss@AmS\relax
}
%    \end{macrocode}
%    \end{macro}
%
%    \begin{macro}{\HoLogo@AmSTeX}
%    \begin{macrocode}
\def\HoLogo@AmSTeX#1{%
  \hologo{AmS}%
  \HOLOGO@hyphen
  \hologo{TeX}%
}
%    \end{macrocode}
%    \end{macro}
%    \begin{macro}{\HoLogoBkm@AmSTeX}
%    \begin{macrocode}
\def\HoLogoBkm@AmSTeX#1{AmS-TeX}%
%    \end{macrocode}
%    \end{macro}
%    \begin{macro}{\HoLogoHtml@AmSTeX}
%    \begin{macrocode}
\let\HoLogoHtml@AmSTeX\HoLogo@AmSTeX
%    \end{macrocode}
%    \end{macro}
%
%    \begin{macro}{\HoLogo@AmSLaTeX}
%    \begin{macrocode}
\def\HoLogo@AmSLaTeX#1{%
  \hologo{AmS}%
  \HOLOGO@hyphen
  \hologo{LaTeX}%
}
%    \end{macrocode}
%    \end{macro}
%    \begin{macro}{\HoLogoBkm@AmSLaTeX}
%    \begin{macrocode}
\def\HoLogoBkm@AmSLaTeX#1{AmS-LaTeX}%
%    \end{macrocode}
%    \end{macro}
%    \begin{macro}{\HoLogoHtml@AmSLaTeX}
%    \begin{macrocode}
\let\HoLogoHtml@AmSLaTeX\HoLogo@AmSLaTeX
%    \end{macrocode}
%    \end{macro}
%
% \subsubsection{\hologo{BibTeX}}
%
%    \begin{macro}{\HoLogo@BibTeX@sc}
%    A definition of \hologo{BibTeX} is provided in
%    the documentation source for the manual of \hologo{BibTeX}
%    \cite{btxdoc}.
%\begin{quote}
%\begin{verbatim}
%\def\BibTeX{%
%  {%
%    \rm
%    B%
%    \kern-.05em%
%    {%
%      \sc
%      i%
%      \kern-.025em %
%      b%
%    }%
%    \kern-.08em
%    T%
%    \kern-.1667em%
%    \lower.7ex\hbox{E}%
%    \kern-.125em%
%    X%
%  }%
%}
%\end{verbatim}
%\end{quote}
%    \begin{macrocode}
\def\HoLogo@BibTeX@sc#1{%
  B%
  \kern-.05em%
  \HoLogoFont@font{BibTeX}{sc}{%
    i%
    \kern-.025em%
    b%
  }%
  \HOLOGO@discretionary
  \kern-.08em%
  \hologo{TeX}%
}
%    \end{macrocode}
%    \end{macro}
%    \begin{macro}{\HoLogoHtml@BibTeX@sc}
%    \begin{macrocode}
\def\HoLogoHtml@BibTeX@sc#1{%
  \HoLogoCss@BibTeX@sc
  \HOLOGO@Span{BibTeX-sc}{%
    B%
    \HOLOGO@Span{i}{i}%
    \HOLOGO@Span{b}{b}%
    \hologo{TeX}%
  }%
}
%    \end{macrocode}
%    \end{macro}
%    \begin{macro}{\HoLogoCss@BibTeX@sc}
%    \begin{macrocode}
\def\HoLogoCss@BibTeX@sc{%
  \Css{%
    span.HoLogo-BibTeX-sc span.HoLogo-i{%
      margin-left:-.05em;%
      margin-right:-.025em;%
      font-variant:small-caps;%
    }%
  }%
  \Css{%
    span.HoLogo-BibTeX-sc span.HoLogo-b{%
      margin-right:-.08em;%
      font-variant:small-caps;%
    }%
  }%
  \global\let\HoLogoCss@BibTeX@sc\relax
}
%    \end{macrocode}
%    \end{macro}
%
%    \begin{macro}{\HoLogo@BibTeX@sf}
%    Variant \xoption{sf} avoids trouble with unavailable
%    small caps fonts (e.g., bold versions of Computer Modern or
%    Latin Modern). The definition is taken from
%    package \xpackage{dtklogos} \cite{dtklogos}.
%\begin{quote}
%\begin{verbatim}
%\DeclareRobustCommand{\BibTeX}{%
%  B%
%  \kern-.05em%
%  \hbox{%
%    $\m@th$% %% force math size calculations
%    \csname S@\f@size\endcsname
%    \fontsize\sf@size\z@
%    \math@fontsfalse
%    \selectfont
%    I%
%    \kern-.025em%
%    B
%  }%
%  \kern-.08em%
%  \-%
%  \TeX
%}
%\end{verbatim}
%\end{quote}
%    \begin{macrocode}
\def\HoLogo@BibTeX@sf#1{%
  B%
  \kern-.05em%
  \HoLogoFont@font{BibTeX}{bibsf}{%
    I%
    \kern-.025em%
    B%
  }%
  \HOLOGO@discretionary
  \kern-.08em%
  \hologo{TeX}%
}
%    \end{macrocode}
%    \end{macro}
%    \begin{macro}{\HoLogoHtml@BibTeX@sf}
%    \begin{macrocode}
\def\HoLogoHtml@BibTeX@sf#1{%
  \HoLogoCss@BibTeX@sf
  \HOLOGO@Span{BibTeX-sf}{%
    B%
    \HoLogoFont@font{BibTeX}{bibsf}{%
      \HOLOGO@Span{i}{I}%
      B%
    }%
    \hologo{TeX}%
  }%
}
%    \end{macrocode}
%    \end{macro}
%    \begin{macro}{\HoLogoCss@BibTeX@sf}
%    \begin{macrocode}
\def\HoLogoCss@BibTeX@sf{%
  \Css{%
    span.HoLogo-BibTeX-sf span.HoLogo-i{%
      margin-left:-.05em;%
      margin-right:-.025em;%
    }%
  }%
  \Css{%
    span.HoLogo-BibTeX-sf span.HoLogo-TeX{%
      margin-left:-.08em;%
    }%
  }%
  \global\let\HoLogoCss@BibTeX@sf\relax
}
%    \end{macrocode}
%    \end{macro}
%
%    \begin{macro}{\HoLogo@BibTeX}
%    \begin{macrocode}
\def\HoLogo@BibTeX{\HoLogo@BibTeX@sf}
%    \end{macrocode}
%    \end{macro}
%    \begin{macro}{\HoLogoHtml@BibTeX}
%    \begin{macrocode}
\def\HoLogoHtml@BibTeX{\HoLogoHtml@BibTeX@sf}
%    \end{macrocode}
%    \end{macro}
%
% \subsubsection{\hologo{BibTeX8}}
%
%    \begin{macro}{\HoLogo@BibTeX8}
%    \begin{macrocode}
\expandafter\def\csname HoLogo@BibTeX8\endcsname#1{%
  \hologo{BibTeX}%
  8%
}
%    \end{macrocode}
%    \end{macro}
%
%    \begin{macro}{\HoLogoBkm@BibTeX8}
%    \begin{macrocode}
\expandafter\def\csname HoLogoBkm@BibTeX8\endcsname#1{%
  \hologo{BibTeX}%
  8%
}
%    \end{macrocode}
%    \end{macro}
%    \begin{macro}{\HoLogoHtml@BibTeX8}
%    \begin{macrocode}
\expandafter
\let\csname HoLogoHtml@BibTeX8\expandafter\endcsname
\csname HoLogo@BibTeX8\endcsname
%    \end{macrocode}
%    \end{macro}
%
% \subsubsection{\hologo{ConTeXt}}
%
%    \begin{macro}{\HoLogo@ConTeXt@simple}
%    \begin{macrocode}
\def\HoLogo@ConTeXt@simple#1{%
  \HOLOGO@mbox{Con}%
  \HOLOGO@discretionary
  \HOLOGO@mbox{\hologo{TeX}t}%
}
%    \end{macrocode}
%    \end{macro}
%    \begin{macro}{\HoLogoHtml@ConTeXt@simple}
%    \begin{macrocode}
\let\HoLogoHtml@ConTeXt@simple\HoLogo@ConTeXt@simple
%    \end{macrocode}
%    \end{macro}
%
%    \begin{macro}{\HoLogo@ConTeXt@narrow}
%    This definition of logo \hologo{ConTeXt} with variant \xoption{narrow}
%    comes from TUGboat's class \xclass{ltugboat} (version 2010/11/15 v2.8).
%    \begin{macrocode}
\def\HoLogo@ConTeXt@narrow#1{%
  \HOLOGO@mbox{C\kern-.0333emon}%
  \HOLOGO@discretionary
  \kern-.0667em%
  \HOLOGO@mbox{\hologo{TeX}\kern-.0333emt}%
}
%    \end{macrocode}
%    \end{macro}
%    \begin{macro}{\HoLogoHtml@ConTeXt@narrow}
%    \begin{macrocode}
\def\HoLogoHtml@ConTeXt@narrow#1{%
  \HoLogoCss@ConTeXt@narrow
  \HOLOGO@Span{ConTeXt-narrow}{%
    \HOLOGO@Span{C}{C}%
    on%
    \hologo{TeX}%
    t%
  }%
}
%    \end{macrocode}
%    \end{macro}
%    \begin{macro}{\HoLogoCss@ConTeXt@narrow}
%    \begin{macrocode}
\def\HoLogoCss@ConTeXt@narrow{%
  \Css{%
    span.HoLogo-ConTeXt-narrow span.HoLogo-C{%
      margin-left:-.0333em;%
    }%
  }%
  \Css{%
    span.HoLogo-ConTeXt-narrow span.HoLogo-TeX{%
      margin-left:-.0667em;%
      margin-right:-.0333em;%
    }%
  }%
  \global\let\HoLogoCss@ConTeXt@narrow\relax
}
%    \end{macrocode}
%    \end{macro}
%
%    \begin{macro}{\HoLogo@ConTeXt}
%    \begin{macrocode}
\def\HoLogo@ConTeXt{\HoLogo@ConTeXt@narrow}
%    \end{macrocode}
%    \end{macro}
%    \begin{macro}{\HoLogoHtml@ConTeXt}
%    \begin{macrocode}
\def\HoLogoHtml@ConTeXt{\HoLogoHtml@ConTeXt@narrow}
%    \end{macrocode}
%    \end{macro}
%
% \subsubsection{\hologo{emTeX}}
%
%    \begin{macro}{\HoLogo@emTeX}
%    \begin{macrocode}
\def\HoLogo@emTeX#1{%
  \HOLOGO@mbox{#1{e}{E}m}%
  \HOLOGO@discretionary
  \hologo{TeX}%
}
%    \end{macrocode}
%    \end{macro}
%    \begin{macro}{\HoLogoCs@emTeX}
%    \begin{macrocode}
\def\HoLogoCs@emTeX#1{#1{e}{E}mTeX}%
%    \end{macrocode}
%    \end{macro}
%    \begin{macro}{\HoLogoBkm@emTeX}
%    \begin{macrocode}
\def\HoLogoBkm@emTeX#1{%
  #1{e}{E}m\hologo{TeX}%
}
%    \end{macrocode}
%    \end{macro}
%    \begin{macro}{\HoLogoHtml@emTeX}
%    \begin{macrocode}
\let\HoLogoHtml@emTeX\HoLogo@emTeX
%    \end{macrocode}
%    \end{macro}
%
% \subsubsection{\hologo{ExTeX}}
%
%    \begin{macro}{\HoLogo@ExTeX}
%    The definition is taken from the FAQ of the
%    project \hologo{ExTeX}
%    \cite{ExTeX-FAQ}.
%\begin{quote}
%\begin{verbatim}
%\def\ExTeX{%
%  \textrm{% Logo always with serifs
%    \ensuremath{%
%      \textstyle
%      \varepsilon_{%
%        \kern-0.15em%
%        \mathcal{X}%
%      }%
%    }%
%    \kern-.15em%
%    \TeX
%  }%
%}
%\end{verbatim}
%\end{quote}
%    \begin{macrocode}
\def\HoLogo@ExTeX#1{%
  \HoLogoFont@font{ExTeX}{rm}{%
    \ltx@mbox{%
      \HOLOGO@MathSetup
      $%
        \textstyle
        \varepsilon_{%
          \kern-0.15em%
          \HoLogoFont@font{ExTeX}{sy}{X}%
        }%
      $%
    }%
    \HOLOGO@discretionary
    \kern-.15em%
    \hologo{TeX}%
  }%
}
%    \end{macrocode}
%    \end{macro}
%    \begin{macro}{\HoLogoHtml@ExTeX}
%    \begin{macrocode}
\def\HoLogoHtml@ExTeX#1{%
  \HoLogoCss@ExTeX
  \HoLogoFont@font{ExTeX}{rm}{%
    \HOLOGO@Span{ExTeX}{%
      \ltx@mbox{%
        \HOLOGO@MathSetup
        $\textstyle\varepsilon$%
        \HOLOGO@Span{X}{$\textstyle\chi$}%
        \hologo{TeX}%
      }%
    }%
  }%
}
%    \end{macrocode}
%    \end{macro}
%    \begin{macro}{\HoLogoBkm@ExTeX}
%    \begin{macrocode}
\def\HoLogoBkm@ExTeX#1{%
  \HOLOGO@PdfdocUnicode{#1{e}{E}x}{\textepsilon\textchi}%
  \hologo{TeX}%
}
%    \end{macrocode}
%    \end{macro}
%    \begin{macro}{\HoLogoCss@ExTeX}
%    \begin{macrocode}
\def\HoLogoCss@ExTeX{%
  \Css{%
    span.HoLogo-ExTeX{%
      font-family:serif;%
    }%
  }%
  \Css{%
    span.HoLogo-ExTeX span.HoLogo-TeX{%
      margin-left:-.15em;%
    }%
  }%
  \global\let\HoLogoCss@ExTeX\relax
}
%    \end{macrocode}
%    \end{macro}
%
% \subsubsection{\hologo{MiKTeX}}
%
%    \begin{macro}{\HoLogo@MiKTeX}
%    \begin{macrocode}
\def\HoLogo@MiKTeX#1{%
  \HOLOGO@mbox{MiK}%
  \HOLOGO@discretionary
  \hologo{TeX}%
}
%    \end{macrocode}
%    \end{macro}
%    \begin{macro}{\HoLogoHtml@MiKTeX}
%    \begin{macrocode}
\let\HoLogoHtml@MiKTeX\HoLogo@MiKTeX
%    \end{macrocode}
%    \end{macro}
%
% \subsubsection{\hologo{OzTeX} and friends}
%
%    Source: \hologo{OzTeX} FAQ \cite{OzTeX}:
%    \begin{quote}
%      |\def\OzTeX{O\kern-.03em z\kern-.15em\TeX}|\\
%      (There is no kerning in OzMF, OzMP and OzTtH.)
%    \end{quote}
%
%    \begin{macro}{\HoLogo@OzTeX}
%    \begin{macrocode}
\def\HoLogo@OzTeX#1{%
  O%
  \kern-.03em %
  z%
  \kern-.15em %
  \hologo{TeX}%
}
%    \end{macrocode}
%    \end{macro}
%    \begin{macro}{\HoLogoHtml@OzTeX}
%    \begin{macrocode}
\def\HoLogoHtml@OzTeX#1{%
  \HoLogoCss@OzTeX
  \HOLOGO@Span{OzTeX}{%
    O%
    \HOLOGO@Span{z}{z}%
    \hologo{TeX}%
  }%
}
%    \end{macrocode}
%    \end{macro}
%    \begin{macro}{\HoLogoCss@OzTeX}
%    \begin{macrocode}
\def\HoLogoCss@OzTeX{%
  \Css{%
    span.HoLogo-OzTeX span.HoLogo-z{%
      margin-left:-.03em;%
      margin-right:-.15em;%
    }%
  }%
  \global\let\HoLogoCss@OzTeX\relax
}
%    \end{macrocode}
%    \end{macro}
%
%    \begin{macro}{\HoLogo@OzMF}
%    \begin{macrocode}
\def\HoLogo@OzMF#1{%
  \HOLOGO@mbox{OzMF}%
}
%    \end{macrocode}
%    \end{macro}
%    \begin{macro}{\HoLogo@OzMP}
%    \begin{macrocode}
\def\HoLogo@OzMP#1{%
  \HOLOGO@mbox{OzMP}%
}
%    \end{macrocode}
%    \end{macro}
%    \begin{macro}{\HoLogo@OzTtH}
%    \begin{macrocode}
\def\HoLogo@OzTtH#1{%
  \HOLOGO@mbox{OzTtH}%
}
%    \end{macrocode}
%    \end{macro}
%
% \subsubsection{\hologo{PCTeX}}
%
%    \begin{macro}{\HoLogo@PCTeX}
%    \begin{macrocode}
\def\HoLogo@PCTeX#1{%
  \HOLOGO@mbox{PC}%
  \hologo{TeX}%
}
%    \end{macrocode}
%    \end{macro}
%    \begin{macro}{\HoLogoHtml@PCTeX}
%    \begin{macrocode}
\let\HoLogoHtml@PCTeX\HoLogo@PCTeX
%    \end{macrocode}
%    \end{macro}
%
% \subsubsection{\hologo{PiCTeX}}
%
%    The original definitions from \xfile{pictex.tex} \cite{PiCTeX}:
%\begin{quote}
%\begin{verbatim}
%\def\PiC{%
%  P%
%  \kern-.12em%
%  \lower.5ex\hbox{I}%
%  \kern-.075em%
%  C%
%}
%\def\PiCTeX{%
%  \PiC
%  \kern-.11em%
%  \TeX
%}
%\end{verbatim}
%\end{quote}
%
%    \begin{macro}{\HoLogo@PiC}
%    \begin{macrocode}
\def\HoLogo@PiC#1{%
  P%
  \kern-.12em%
  \lower.5ex\hbox{I}%
  \kern-.075em%
  C%
  \HOLOGO@SpaceFactor
}
%    \end{macrocode}
%    \end{macro}
%    \begin{macro}{\HoLogoHtml@PiC}
%    \begin{macrocode}
\def\HoLogoHtml@PiC#1{%
  \HoLogoCss@PiC
  \HOLOGO@Span{PiC}{%
    P%
    \HOLOGO@Span{i}{I}%
    C%
  }%
}
%    \end{macrocode}
%    \end{macro}
%    \begin{macro}{\HoLogoCss@PiC}
%    \begin{macrocode}
\def\HoLogoCss@PiC{%
  \Css{%
    span.HoLogo-PiC span.HoLogo-i{%
      position:relative;%
      top:.5ex;%
      margin-left:-.12em;%
      margin-right:-.075em;%
      text-decoration:none;%
    }%
  }%
  \global\let\HoLogoCss@PiC\relax
}
%    \end{macrocode}
%    \end{macro}
%
%    \begin{macro}{\HoLogo@PiCTeX}
%    \begin{macrocode}
\def\HoLogo@PiCTeX#1{%
  \hologo{PiC}%
  \HOLOGO@discretionary
  \kern-.11em%
  \hologo{TeX}%
}
%    \end{macrocode}
%    \end{macro}
%    \begin{macro}{\HoLogoHtml@PiCTeX}
%    \begin{macrocode}
\def\HoLogoHtml@PiCTeX#1{%
  \HoLogoCss@PiCTeX
  \HOLOGO@Span{PiCTeX}{%
    \hologo{PiC}%
    \hologo{TeX}%
  }%
}
%    \end{macrocode}
%    \end{macro}
%    \begin{macro}{\HoLogoCss@PiCTeX}
%    \begin{macrocode}
\def\HoLogoCss@PiCTeX{%
  \Css{%
    span.HoLogo-PiCTeX span.HoLogo-PiC{%
      margin-right:-.11em;%
    }%
  }%
  \global\let\HoLogoCss@PiCTeX\relax
}
%    \end{macrocode}
%    \end{macro}
%
% \subsubsection{\hologo{teTeX}}
%
%    \begin{macro}{\HoLogo@teTeX}
%    \begin{macrocode}
\def\HoLogo@teTeX#1{%
  \HOLOGO@mbox{#1{t}{T}e}%
  \HOLOGO@discretionary
  \hologo{TeX}%
}
%    \end{macrocode}
%    \end{macro}
%    \begin{macro}{\HoLogoCs@teTeX}
%    \begin{macrocode}
\def\HoLogoCs@teTeX#1{#1{t}{T}dfTeX}
%    \end{macrocode}
%    \end{macro}
%    \begin{macro}{\HoLogoBkm@teTeX}
%    \begin{macrocode}
\def\HoLogoBkm@teTeX#1{%
  #1{t}{T}e\hologo{TeX}%
}
%    \end{macrocode}
%    \end{macro}
%    \begin{macro}{\HoLogoHtml@teTeX}
%    \begin{macrocode}
\let\HoLogoHtml@teTeX\HoLogo@teTeX
%    \end{macrocode}
%    \end{macro}
%
% \subsubsection{\hologo{TeX4ht}}
%
%    \begin{macro}{\HoLogo@TeX4ht}
%    \begin{macrocode}
\expandafter\def\csname HoLogo@TeX4ht\endcsname#1{%
  \HOLOGO@mbox{\hologo{TeX}4ht}%
}
%    \end{macrocode}
%    \end{macro}
%    \begin{macro}{\HoLogoHtml@TeX4ht}
%    \begin{macrocode}
\expandafter
\let\csname HoLogoHtml@TeX4ht\expandafter\endcsname
\csname HoLogo@TeX4ht\endcsname
%    \end{macrocode}
%    \end{macro}
%
%
% \subsubsection{\hologo{SageTeX}}
%
%    \begin{macro}{\HoLogo@SageTeX}
%    \begin{macrocode}
\def\HoLogo@SageTeX#1{%
  \HOLOGO@mbox{Sage}%
  \HOLOGO@discretionary
  \HOLOGO@NegativeKerning{eT,oT,To}%
  \hologo{TeX}%
}
%    \end{macrocode}
%    \end{macro}
%    \begin{macro}{\HoLogoHtml@SageTeX}
%    \begin{macrocode}
\let\HoLogoHtml@SageTeX\HoLogo@SageTeX
%    \end{macrocode}
%    \end{macro}
%
% \subsection{\hologo{METAFONT} and friends}
%
%    \begin{macro}{\HoLogo@METAFONT}
%    \begin{macrocode}
\def\HoLogo@METAFONT#1{%
  \HoLogoFont@font{METAFONT}{logo}{%
    \HOLOGO@mbox{META}%
    \HOLOGO@discretionary
    \HOLOGO@mbox{FONT}%
  }%
}
%    \end{macrocode}
%    \end{macro}
%
%    \begin{macro}{\HoLogo@METAPOST}
%    \begin{macrocode}
\def\HoLogo@METAPOST#1{%
  \HoLogoFont@font{METAPOST}{logo}{%
    \HOLOGO@mbox{META}%
    \HOLOGO@discretionary
    \HOLOGO@mbox{POST}%
  }%
}
%    \end{macrocode}
%    \end{macro}
%
%    \begin{macro}{\HoLogo@MetaFun}
%    \begin{macrocode}
\def\HoLogo@MetaFun#1{%
  \HOLOGO@mbox{Meta}%
  \HOLOGO@discretionary
  \HOLOGO@mbox{Fun}%
}
%    \end{macrocode}
%    \end{macro}
%
%    \begin{macro}{\HoLogo@MetaPost}
%    \begin{macrocode}
\def\HoLogo@MetaPost#1{%
  \HOLOGO@mbox{Meta}%
  \HOLOGO@discretionary
  \HOLOGO@mbox{Post}%
}
%    \end{macrocode}
%    \end{macro}
%
% \subsection{Others}
%
% \subsubsection{\hologo{biber}}
%
%    \begin{macro}{\HoLogo@biber}
%    \begin{macrocode}
\def\HoLogo@biber#1{%
  \HOLOGO@mbox{#1{b}{B}i}%
  \HOLOGO@discretionary
  \HOLOGO@mbox{ber}%
}
%    \end{macrocode}
%    \end{macro}
%    \begin{macro}{\HoLogoCs@biber}
%    \begin{macrocode}
\def\HoLogoCs@biber#1{#1{b}{B}iber}
%    \end{macrocode}
%    \end{macro}
%    \begin{macro}{\HoLogoBkm@biber}
%    \begin{macrocode}
\def\HoLogoBkm@biber#1{%
  #1{b}{B}iber%
}
%    \end{macrocode}
%    \end{macro}
%    \begin{macro}{\HoLogoHtml@biber}
%    \begin{macrocode}
\let\HoLogoHtml@biber\HoLogo@biber
%    \end{macrocode}
%    \end{macro}
%
% \subsubsection{\hologo{KOMAScript}}
%
%    \begin{macro}{\HoLogo@KOMAScript}
%    The definition for \hologo{KOMAScript} is taken
%    from \hologo{KOMAScript} (\xfile{scrlogo.dtx}, reformatted) \cite{scrlogo}:
%\begin{quote}
%\begin{verbatim}
%\@ifundefined{KOMAScript}{%
%  \DeclareRobustCommand{\KOMAScript}{%
%    \textsf{%
%      K\kern.05em O\kern.05emM\kern.05em A%
%      \kern.1em-\kern.1em %
%      Script%
%    }%
%  }%
%}{}
%\end{verbatim}
%\end{quote}
%    \begin{macrocode}
\def\HoLogo@KOMAScript#1{%
  \HoLogoFont@font{KOMAScript}{sf}{%
    \HOLOGO@mbox{%
      K\kern.05em%
      O\kern.05em%
      M\kern.05em%
      A%
    }%
    \kern.1em%
    \HOLOGO@hyphen
    \kern.1em%
    \HOLOGO@mbox{Script}%
  }%
}
%    \end{macrocode}
%    \end{macro}
%    \begin{macro}{\HoLogoBkm@KOMAScript}
%    \begin{macrocode}
\def\HoLogoBkm@KOMAScript#1{%
  KOMA-Script%
}
%    \end{macrocode}
%    \end{macro}
%    \begin{macro}{\HoLogoHtml@KOMAScript}
%    \begin{macrocode}
\def\HoLogoHtml@KOMAScript#1{%
  \HoLogoCss@KOMAScript
  \HoLogoFont@font{KOMAScript}{sf}{%
    \HOLOGO@Span{KOMAScript}{%
      K%
      \HOLOGO@Span{O}{O}%
      M%
      \HOLOGO@Span{A}{A}%
      \HOLOGO@Span{hyphen}{-}%
      Script%
    }%
  }%
}
%    \end{macrocode}
%    \end{macro}
%    \begin{macro}{\HoLogoCss@KOMAScript}
%    \begin{macrocode}
\def\HoLogoCss@KOMAScript{%
  \Css{%
    span.HoLogo-KOMAScript{%
      font-family:sans-serif;%
    }%
  }%
  \Css{%
    span.HoLogo-KOMAScript span.HoLogo-O{%
      padding-left:.05em;%
      padding-right:.05em;%
    }%
  }%
  \Css{%
    span.HoLogo-KOMAScript span.HoLogo-A{%
      padding-left:.05em;%
    }%
  }%
  \Css{%
    span.HoLogo-KOMAScript span.HoLogo-hyphen{%
      padding-left:.1em;%
      padding-right:.1em;%
    }%
  }%
  \global\let\HoLogoCss@KOMAScript\relax
}
%    \end{macrocode}
%    \end{macro}
%
% \subsubsection{\hologo{LyX}}
%
%    \begin{macro}{\HoLogo@LyX}
%    The definition is taken from the documentation source files
%    of \hologo{LyX}, \xfile{Intro.lyx} \cite{LyX}:
%\begin{quote}
%\begin{verbatim}
%\def\LyX{%
%  \texorpdfstring{%
%    L\kern-.1667em\lower.25em\hbox{Y}\kern-.125emX\@%
%  }{%
%    LyX%
%  }%
%}
%\end{verbatim}
%\end{quote}
%    \begin{macrocode}
\def\HoLogo@LyX#1{%
  L%
  \kern-.1667em%
  \lower.25em\hbox{Y}%
  \kern-.125em%
  X%
  \HOLOGO@SpaceFactor
}
%    \end{macrocode}
%    \end{macro}
%    \begin{macro}{\HoLogoHtml@LyX}
%    \begin{macrocode}
\def\HoLogoHtml@LyX#1{%
  \HoLogoCss@LyX
  \HOLOGO@Span{LyX}{%
    L%
    \HOLOGO@Span{y}{Y}%
    X%
  }%
}
%    \end{macrocode}
%    \end{macro}
%    \begin{macro}{\HoLogoCss@LyX}
%    \begin{macrocode}
\def\HoLogoCss@LyX{%
  \Css{%
    span.HoLogo-LyX span.HoLogo-y{%
      position:relative;%
      top:.25em;%
      margin-left:-.1667em;%
      margin-right:-.125em;%
      text-decoration:none;%
    }%
  }%
  \global\let\HoLogoCss@LyX\relax
}
%    \end{macrocode}
%    \end{macro}
%
% \subsubsection{\hologo{NTS}}
%
%    \begin{macro}{\HoLogo@NTS}
%    Definition for \hologo{NTS} can be found in
%    package \xpackage{etex\textunderscore man} for the \hologo{eTeX} manual \cite{etexman}
%    and in package \xpackage{dtklogos} \cite{dtklogos}:
%\begin{quote}
%\begin{verbatim}
%\def\NTS{%
%  \leavevmode
%  \hbox{%
%    $%
%      \cal N%
%      \kern-0.35em%
%      \lower0.5ex\hbox{$\cal T$}%
%      \kern-0.2em%
%      S%
%    $%
%  }%
%}
%\end{verbatim}
%\end{quote}
%    \begin{macrocode}
\def\HoLogo@NTS#1{%
  \HoLogoFont@font{NTS}{sy}{%
    N\/%
    \kern-.35em%
    \lower.5ex\hbox{T\/}%
    \kern-.2em%
    S\/%
  }%
  \HOLOGO@SpaceFactor
}
%    \end{macrocode}
%    \end{macro}
%
% \subsubsection{\Hologo{TTH} (\hologo{TeX} to HTML translator)}
%
%    Source: \url{http://hutchinson.belmont.ma.us/tth/}
%    In the HTML source the second `T' is printed as subscript.
%\begin{quote}
%\begin{verbatim}
%T<sub>T</sub>H
%\end{verbatim}
%\end{quote}
%    \begin{macro}{\HoLogo@TTH}
%    \begin{macrocode}
\def\HoLogo@TTH#1{%
  \ltx@mbox{%
    T\HOLOGO@SubScript{T}H%
  }%
  \HOLOGO@SpaceFactor
}
%    \end{macrocode}
%    \end{macro}
%
%    \begin{macro}{\HoLogoHtml@TTH}
%    \begin{macrocode}
\def\HoLogoHtml@TTH#1{%
  T\HCode{<sub>}T\HCode{</sub>}H%
}
%    \end{macrocode}
%    \end{macro}
%
% \subsubsection{\Hologo{HanTheThanh}}
%
%    Partial source: Package \xpackage{dtklogos}.
%    The double accent is U+1EBF (latin small letter e with circumflex
%    and acute).
%    \begin{macro}{\HoLogo@HanTheThanh}
%    \begin{macrocode}
\def\HoLogo@HanTheThanh#1{%
  \ltx@mbox{H\`an}%
  \HOLOGO@space
  \ltx@mbox{%
    Th%
    \HOLOGO@IfCharExists{"1EBF}{%
      \char"1EBF\relax
    }{%
      \^e\hbox to 0pt{\hss\raise .5ex\hbox{\'{}}}%
    }%
  }%
  \HOLOGO@space
  \ltx@mbox{Th\`anh}%
}
%    \end{macrocode}
%    \end{macro}
%    \begin{macro}{\HoLogoBkm@HanTheThanh}
%    \begin{macrocode}
\def\HoLogoBkm@HanTheThanh#1{%
  H\`an %
  Th\HOLOGO@PdfdocUnicode{\^e}{\9036\277} %
  Th\`anh%
}
%    \end{macrocode}
%    \end{macro}
%    \begin{macro}{\HoLogoHtml@HanTheThanh}
%    \begin{macrocode}
\def\HoLogoHtml@HanTheThanh#1{%
  H\`an %
  Th\HCode{&\ltx@hashchar x1ebf;} %
  Th\`anh%
}
%    \end{macrocode}
%    \end{macro}
%
% \subsection{Driver detection}
%
%    \begin{macrocode}
\HOLOGO@IfExists\InputIfFileExists{%
  \InputIfFileExists{hologo.cfg}{}{}%
}{%
  \ltx@IfUndefined{pdf@filesize}{%
    \def\HOLOGO@InputIfExists{%
      \openin\HOLOGO@temp=hologo.cfg\relax
      \ifeof\HOLOGO@temp
        \closein\HOLOGO@temp
      \else
        \closein\HOLOGO@temp
        \begingroup
          \def\x{LaTeX2e}%
        \expandafter\endgroup
        \ifx\fmtname\x
          \input{hologo.cfg}%
        \else
          \input hologo.cfg\relax
        \fi
      \fi
    }%
    \ltx@IfUndefined{newread}{%
      \chardef\HOLOGO@temp=15 %
      \def\HOLOGO@CheckRead{%
        \ifeof\HOLOGO@temp
          \HOLOGO@InputIfExists
        \else
          \ifcase\HOLOGO@temp
            \@PackageWarningNoLine{hologo}{%
              Configuration file ignored, because\MessageBreak
              a free read register could not be found%
            }%
          \else
            \begingroup
              \count\ltx@cclv=\HOLOGO@temp
              \advance\ltx@cclv by \ltx@minusone
              \edef\x{\endgroup
                \chardef\noexpand\HOLOGO@temp=\the\count\ltx@cclv
                \relax
              }%
            \x
          \fi
        \fi
      }%
    }{%
      \csname newread\endcsname\HOLOGO@temp
      \HOLOGO@InputIfExists
    }%
  }{%
    \edef\HOLOGO@temp{\pdf@filesize{hologo.cfg}}%
    \ifx\HOLOGO@temp\ltx@empty
    \else
      \ifnum\HOLOGO@temp>0 %
        \begingroup
          \def\x{LaTeX2e}%
        \expandafter\endgroup
        \ifx\fmtname\x
          \input{hologo.cfg}%
        \else
          \input hologo.cfg\relax
        \fi
      \else
        \@PackageInfoNoLine{hologo}{%
          Empty configuration file `hologo.cfg' ignored%
        }%
      \fi
    \fi
  }%
}
%    \end{macrocode}
%
%    \begin{macrocode}
\def\HOLOGO@temp#1#2{%
  \kv@define@key{HoLogoDriver}{#1}[]{%
    \begingroup
      \def\HOLOGO@temp{##1}%
      \ltx@onelevel@sanitize\HOLOGO@temp
      \ifx\HOLOGO@temp\ltx@empty
      \else
        \@PackageError{hologo}{%
          Value (\HOLOGO@temp) not permitted for option `#1'%
        }%
        \@ehc
      \fi
    \endgroup
    \def\hologoDriver{#2}%
  }%
}%
\def\HOLOGO@@temp#1#2{%
  \ifx\kv@value\relax
    \HOLOGO@temp{#1}{#1}%
  \else
    \HOLOGO@temp{#1}{#2}%
  \fi
}%
\kv@parse@normalized{%
  pdftex,%
  luatex=pdftex,%
  dvipdfm,%
  dvipdfmx=dvipdfm,%
  dvips,%
  dvipsone=dvips,%
  xdvi=dvips,%
  xetex,%
  vtex,%
}\HOLOGO@@temp
%    \end{macrocode}
%
%    \begin{macrocode}
\kv@define@key{HoLogoDriver}{driverfallback}{%
  \def\HOLOGO@DriverFallback{#1}%
}
%    \end{macrocode}
%
%    \begin{macro}{\HOLOGO@DriverFallback}
%    \begin{macrocode}
\def\HOLOGO@DriverFallback{dvips}
%    \end{macrocode}
%    \end{macro}
%
%    \begin{macro}{\hologoDriverSetup}
%    \begin{macrocode}
\def\hologoDriverSetup{%
  \let\hologoDriver\ltx@undefined
  \HOLOGO@DriverSetup
}
%    \end{macrocode}
%    \end{macro}
%
%    \begin{macro}{\HOLOGO@DriverSetup}
%    \begin{macrocode}
\def\HOLOGO@DriverSetup#1{%
  \kvsetkeys{HoLogoDriver}{#1}%
  \HOLOGO@CheckDriver
  \ltx@ifundefined{hologoDriver}{%
    \begingroup
    \edef\x{\endgroup
      \noexpand\kvsetkeys{HoLogoDriver}{\HOLOGO@DriverFallback}%
    }\x
  }{}%
  \@PackageInfoNoLine{hologo}{Using driver `\hologoDriver'}%
}
%    \end{macrocode}
%    \end{macro}
%
%    \begin{macro}{\HOLOGO@CheckDriver}
%    \begin{macrocode}
\def\HOLOGO@CheckDriver{%
  \ifpdf
    \def\hologoDriver{pdftex}%
    \let\HOLOGO@pdfliteral\pdfliteral
    \ifluatex
      \ifx\pdfextension\@undefined\else
        \protected\def\pdfliteral{\pdfextension literal}%
        \let\HOLOGO@pdfliteral\pdfliteral
      \fi
      \ltx@IfUndefined{HOLOGO@pdfliteral}{%
        \ifnum\luatexversion<36 %
        \else
          \begingroup
            \let\HOLOGO@temp\endgroup
            \ifcase0%
                \directlua{%
                  if tex.enableprimitives then %
                    tex.enableprimitives('HOLOGO@', {'pdfliteral'})%
                  else %
                    tex.print('1')%
                  end%
                }%
                \ifx\HOLOGO@pdfliteral\@undefined 1\fi%
                \relax%
              \endgroup
              \let\HOLOGO@temp\relax
              \global\let\HOLOGO@pdfliteral\HOLOGO@pdfliteral
            \fi%
          \HOLOGO@temp
        \fi
      }{}%
    \fi
    \ltx@IfUndefined{HOLOGO@pdfliteral}{%
      \@PackageWarningNoLine{hologo}{%
        Cannot find \string\pdfliteral
      }%
    }{}%
  \else
    \ifxetex
      \def\hologoDriver{xetex}%
    \else
      \ifvtex
        \def\hologoDriver{vtex}%
      \fi
    \fi
  \fi
}
%    \end{macrocode}
%    \end{macro}
%
%    \begin{macro}{\HOLOGO@WarningUnsupportedDriver}
%    \begin{macrocode}
\def\HOLOGO@WarningUnsupportedDriver#1{%
  \@PackageWarningNoLine{hologo}{%
    Logo `#1' needs driver specific macros,\MessageBreak
    but driver `\hologoDriver' is not supported.\MessageBreak
    Use a different driver or\MessageBreak
    load package `graphics' or `pgf'%
  }%
}
%    \end{macrocode}
%    \end{macro}
%
% \subsubsection{Reflect box macros}
%
%    Skip driver part if not needed.
%    \begin{macrocode}
\ltx@IfUndefined{reflectbox}{}{%
  \ltx@IfUndefined{rotatebox}{}{%
    \HOLOGO@AtEnd
  }%
}
\ltx@IfUndefined{pgftext}{}{%
  \HOLOGO@AtEnd
}
\ltx@IfUndefined{psscalebox}{}{%
  \HOLOGO@AtEnd
}
%    \end{macrocode}
%
%    \begin{macrocode}
\def\HOLOGO@temp{LaTeX2e}
\ifx\fmtname\HOLOGO@temp
  \RequirePackage{kvoptions}[2011/06/30]%
  \ProcessKeyvalOptions{HoLogoDriver}%
\fi
\HOLOGO@DriverSetup{}
%    \end{macrocode}
%
%    \begin{macro}{\HOLOGO@ReflectBox}
%    \begin{macrocode}
\def\HOLOGO@ReflectBox#1{%
  \begingroup
    \setbox\ltx@zero\hbox{\begingroup#1\endgroup}%
    \setbox\ltx@two\hbox{%
      \kern\wd\ltx@zero
      \csname HOLOGO@ScaleBox@\hologoDriver\endcsname{-1}{1}{%
        \hbox to 0pt{\copy\ltx@zero\hss}%
      }%
    }%
    \wd\ltx@two=\wd\ltx@zero
    \box\ltx@two
  \endgroup
}
%    \end{macrocode}
%    \end{macro}
%
%    \begin{macro}{\HOLOGO@PointReflectBox}
%    \begin{macrocode}
\def\HOLOGO@PointReflectBox#1{%
  \begingroup
    \setbox\ltx@zero\hbox{\begingroup#1\endgroup}%
    \setbox\ltx@two\hbox{%
      \kern\wd\ltx@zero
      \raise\ht\ltx@zero\hbox{%
        \csname HOLOGO@ScaleBox@\hologoDriver\endcsname{-1}{-1}{%
          \hbox to 0pt{\copy\ltx@zero\hss}%
        }%
      }%
    }%
    \wd\ltx@two=\wd\ltx@zero
    \box\ltx@two
  \endgroup
}
%    \end{macrocode}
%    \end{macro}
%
%    We must define all variants because of dynamic driver setup.
%    \begin{macrocode}
\def\HOLOGO@temp#1#2{#2}
%    \end{macrocode}
%
%    \begin{macro}{\HOLOGO@ScaleBox@pdftex}
%    \begin{macrocode}
\HOLOGO@temp{pdftex}{%
  \def\HOLOGO@ScaleBox@pdftex#1#2#3{%
    \HOLOGO@pdfliteral{%
      q #1 0 0 #2 0 0 cm%
    }%
    #3%
    \HOLOGO@pdfliteral{%
      Q%
    }%
  }%
}
%    \end{macrocode}
%    \end{macro}
%    \begin{macro}{\HOLOGO@ScaleBox@dvips}
%    \begin{macrocode}
\HOLOGO@temp{dvips}{%
  \def\HOLOGO@ScaleBox@dvips#1#2#3{%
    \special{ps:%
      gsave %
      currentpoint %
      currentpoint translate %
      #1 #2 scale %
      neg exch neg exch translate%
    }%
    #3%
    \special{ps:%
      currentpoint %
      grestore %
      moveto%
    }%
  }%
}
%    \end{macrocode}
%    \end{macro}
%    \begin{macro}{\HOLOGO@ScaleBox@dvipdfm}
%    \begin{macrocode}
\HOLOGO@temp{dvipdfm}{%
  \let\HOLOGO@ScaleBox@dvipdfm\HOLOGO@ScaleBox@dvips
}
%    \end{macrocode}
%    \end{macro}
%    Since \hologo{XeTeX} v0.6.
%    \begin{macro}{\HOLOGO@ScaleBox@xetex}
%    \begin{macrocode}
\HOLOGO@temp{xetex}{%
  \def\HOLOGO@ScaleBox@xetex#1#2#3{%
    \special{x:gsave}%
    \special{x:scale #1 #2}%
    #3%
    \special{x:grestore}%
  }%
}
%    \end{macrocode}
%    \end{macro}
%    \begin{macro}{\HOLOGO@ScaleBox@vtex}
%    \begin{macrocode}
\HOLOGO@temp{vtex}{%
  \def\HOLOGO@ScaleBox@vtex#1#2#3{%
    \special{r(#1,0,0,#2,0,0}%
    #3%
    \special{r)}%
  }%
}
%    \end{macrocode}
%    \end{macro}
%
%    \begin{macrocode}
\HOLOGO@AtEnd%
%</package>
%    \end{macrocode}
%
% \section{Test}
%
% \subsection{Catcode checks for loading}
%
%    \begin{macrocode}
%<*test1>
%    \end{macrocode}
%    \begin{macrocode}
\catcode`\{=1 %
\catcode`\}=2 %
\catcode`\#=6 %
\catcode`\@=11 %
\expandafter\ifx\csname count@\endcsname\relax
  \countdef\count@=255 %
\fi
\expandafter\ifx\csname @gobble\endcsname\relax
  \long\def\@gobble#1{}%
\fi
\expandafter\ifx\csname @firstofone\endcsname\relax
  \long\def\@firstofone#1{#1}%
\fi
\expandafter\ifx\csname loop\endcsname\relax
  \expandafter\@firstofone
\else
  \expandafter\@gobble
\fi
{%
  \def\loop#1\repeat{%
    \def\body{#1}%
    \iterate
  }%
  \def\iterate{%
    \body
      \let\next\iterate
    \else
      \let\next\relax
    \fi
    \next
  }%
  \let\repeat=\fi
}%
\def\RestoreCatcodes{}
\count@=0 %
\loop
  \edef\RestoreCatcodes{%
    \RestoreCatcodes
    \catcode\the\count@=\the\catcode\count@\relax
  }%
\ifnum\count@<255 %
  \advance\count@ 1 %
\repeat

\def\RangeCatcodeInvalid#1#2{%
  \count@=#1\relax
  \loop
    \catcode\count@=15 %
  \ifnum\count@<#2\relax
    \advance\count@ 1 %
  \repeat
}
\def\RangeCatcodeCheck#1#2#3{%
  \count@=#1\relax
  \loop
    \ifnum#3=\catcode\count@
    \else
      \errmessage{%
        Character \the\count@\space
        with wrong catcode \the\catcode\count@\space
        instead of \number#3%
      }%
    \fi
  \ifnum\count@<#2\relax
    \advance\count@ 1 %
  \repeat
}
\def\space{ }
\expandafter\ifx\csname LoadCommand\endcsname\relax
  \def\LoadCommand{\input hologo.sty\relax}%
\fi
\def\Test{%
  \RangeCatcodeInvalid{0}{47}%
  \RangeCatcodeInvalid{58}{64}%
  \RangeCatcodeInvalid{91}{96}%
  \RangeCatcodeInvalid{123}{255}%
  \catcode`\@=12 %
  \catcode`\\=0 %
  \catcode`\%=14 %
  \LoadCommand
  \RangeCatcodeCheck{0}{36}{15}%
  \RangeCatcodeCheck{37}{37}{14}%
  \RangeCatcodeCheck{38}{47}{15}%
  \RangeCatcodeCheck{48}{57}{12}%
  \RangeCatcodeCheck{58}{63}{15}%
  \RangeCatcodeCheck{64}{64}{12}%
  \RangeCatcodeCheck{65}{90}{11}%
  \RangeCatcodeCheck{91}{91}{15}%
  \RangeCatcodeCheck{92}{92}{0}%
  \RangeCatcodeCheck{93}{96}{15}%
  \RangeCatcodeCheck{97}{122}{11}%
  \RangeCatcodeCheck{123}{255}{15}%
  \RestoreCatcodes
}
\Test
\csname @@end\endcsname
\end
%    \end{macrocode}
%    \begin{macrocode}
%</test1>
%    \end{macrocode}
%
% \subsection{Spacefactor}
%
%    The space factor must be 1000 after a logo. If it is greater 1000
%    then the following space is a space after a sentence closing point.
%    If the space factor is smaller 1000 then an immediate following
%    dot is interpreted as abbreviation, not sentence closing point.
%
%    \begin{macrocode}
%<*test-spacefactor>
\NeedsTeXFormat{LaTeX2e}
\documentclass{article}
\usepackage{hologo}[2016/05/12]
\usepackage{kvsetkeys}
\usepackage{qstest}
\IncludeTests{*}
\LogTests{log}{*}{*}
\begin{document}
\begin{qstest}{spacefactor}{spacefactor}
\newcommand*{\Test}[1]{%
  \sbox0{%
    \hologo{#1}%
    \Expect*{1000 (#1)}*{\the\spacefactor\space(#1)}%
  }%
}%
\makeatletter
\def\TestList{}
\def\hologoEntry#1#2#3{%
  \edef\TestList{%
    \ifx\TestList\@empty
    \else
      \TestList,%
    \fi
    #1%
    \ifx\\#2\\%
    \else
      ={variant=#2}%
    \fi
  }%
}
\hologoList
\expandafter\kv@parse@normalized\expandafter{%
  \TestList
}{%
  \begingroup
    \let\@logo=\kv@key
    \ifx\kv@value\relax
    \else
      \expandafter\hologoLogoSetup\expandafter\@logo\expandafter{%
        \kv@value
      }%
    \fi
    \Test\@logo
  \endgroup
  \@gobbletwo
}
\end{qstest}
\end{document}
%</test-spacefactor>
%    \end{macrocode}
%
% \subsection{Complete list}
%
%    \begin{macrocode}
%<*test-list>
\NeedsTeXFormat{LaTeX2e}
\documentclass[12pt,a4paper]{article}
\usepackage{hologo}[2016/05/12]
\usepackage[T1]{fontenc}
\usepackage{lmodern}
\usepackage{parskip}
\usepackage[unicode]{hyperref}[2011/09/28]
\usepackage{bookmark}[2011/09/19]
\bookmarksetup{%
  numbered,%
  open,%
  openlevel=2,%
}
\renewcommand*{\contentsname}{List of logos}
\begin{document}
\tableofcontents
\def\TestFont#1#2#3#4#5#6{%
  \begingroup
    \usefont{#3}{#4}{#5}{#6}%
    \HologoVariant{#1}{#2}/\hologoVariant{#1}{#2}%
    \quad
    \begingroup\scriptsize\hologoVariant{#1}{#2}\endgroup
    \quad
  \endgroup
  (#3/#4/#5/#6)%
  \par
}
\makeatletter
\def\hologoEntry#1#2#3{%
  \section{%
    \HologoVariant{#1}{#2}/\hologoVariant{#1}{#2} %
    {[#1\ifx\\#2\\\else\space(#2)\fi]}% hash-ok
  }% braces around [] because of bug in tex4ht
  \begingroup
    \hypersetup{unicode=false}%
    \bookmark[%
      dest=\@currentHref,%
      rellevel=1,%
      keeplevel,%
    ]{%
      \HologoVariant{#1}{#2}/\hologoVariant{#1}{#2} %
      (PDFDocEncoding)%
    }%
  \endgroup
  \TestFont{#1}{#2}{OT1}{cmr}{m}{n}%
  \TestFont{#1}{#2}{OT1}{cmss}{m}{n}%
  \TestFont{#1}{#2}{OT1}{cmr}{b}{n}%
  \TestFont{#1}{#2}{OT1}{cmr}{m}{it}%
  \TestFont{#1}{#2}{OT1}{cmtt}{m}{n}%
  \TestFont{#1}{#2}{T1}{lmr}{m}{n}%
  \TestFont{#1}{#2}{T1}{lmss}{m}{n}%
  \TestFont{#1}{#2}{T1}{lmr}{b}{n}%
  \TestFont{#1}{#2}{T1}{lmr}{m}{it}%
  \TestFont{#1}{#2}{T1}{lmtt}{m}{n}%
  \TestFont{#1}{#2}{T1}{lmvtt}{m}{n}%
  \TestFont{#1}{#2}{T1}{qtm}{m}{n}%
  \TestFont{#1}{#2}{T1}{qhv}{m}{n}%
  \TestFont{#1}{#2}{T1}{qtm}{b}{n}%
  \TestFont{#1}{#2}{T1}{qtm}{m}{it}%
  \TestFont{#1}{#2}{T1}{qcr}{m}{n}%
  \newpage
}
\makeatother
\hologoList
\end{document}
%</test-list>
%    \end{macrocode}
%
% \section{Installation}
%
% \subsection{Download}
%
% \paragraph{Package.} This package is available on
% CTAN\footnote{\url{ftp://ftp.ctan.org/tex-archive/}}:
% \begin{description}
% \item[\CTAN{macros/latex/contrib/oberdiek/hologo.dtx}] The source file.
% \item[\CTAN{macros/latex/contrib/oberdiek/hologo.pdf}] Documentation.
% \end{description}
%
%
% \paragraph{Bundle.} All the packages of the bundle `oberdiek'
% are also available in a TDS compliant ZIP archive. There
% the packages are already unpacked and the documentation files
% are generated. The files and directories obey the TDS standard.
% \begin{description}
% \item[\CTAN{install/macros/latex/contrib/oberdiek.tds.zip}]
% \end{description}
% \emph{TDS} refers to the standard ``A Directory Structure
% for \TeX\ Files'' (\CTAN{tds/tds.pdf}). Directories
% with \xfile{texmf} in their name are usually organized this way.
%
% \subsection{Bundle installation}
%
% \paragraph{Unpacking.} Unpack the \xfile{oberdiek.tds.zip} in the
% TDS tree (also known as \xfile{texmf} tree) of your choice.
% Example (linux):
% \begin{quote}
%   |unzip oberdiek.tds.zip -d ~/texmf|
% \end{quote}
%
% \paragraph{Script installation.}
% Check the directory \xfile{TDS:scripts/oberdiek/} for
% scripts that need further installation steps.
% Package \xpackage{attachfile2} comes with the Perl script
% \xfile{pdfatfi.pl} that should be installed in such a way
% that it can be called as \texttt{pdfatfi}.
% Example (linux):
% \begin{quote}
%   |chmod +x scripts/oberdiek/pdfatfi.pl|\\
%   |cp scripts/oberdiek/pdfatfi.pl /usr/local/bin/|
% \end{quote}
%
% \subsection{Package installation}
%
% \paragraph{Unpacking.} The \xfile{.dtx} file is a self-extracting
% \docstrip\ archive. The files are extracted by running the
% \xfile{.dtx} through \plainTeX:
% \begin{quote}
%   \verb|tex hologo.dtx|
% \end{quote}
%
% \paragraph{TDS.} Now the different files must be moved into
% the different directories in your installation TDS tree
% (also known as \xfile{texmf} tree):
% \begin{quote}
% \def\t{^^A
% \begin{tabular}{@{}>{\ttfamily}l@{ $\rightarrow$ }>{\ttfamily}l@{}}
%   hologo.sty & tex/generic/oberdiek/hologo.sty\\
%   hologo.pdf & doc/latex/oberdiek/hologo.pdf\\
%   example/hologo-example.tex & doc/latex/oberdiek/example/hologo-example.tex\\
%   test/hologo-test1.tex & doc/latex/oberdiek/test/hologo-test1.tex\\
%   test/hologo-test-spacefactor.tex & doc/latex/oberdiek/test/hologo-test-spacefactor.tex\\
%   test/hologo-test-list.tex & doc/latex/oberdiek/test/hologo-test-list.tex\\
%   hologo.dtx & source/latex/oberdiek/hologo.dtx\\
% \end{tabular}^^A
% }^^A
% \sbox0{\t}^^A
% \ifdim\wd0>\linewidth
%   \begingroup
%     \advance\linewidth by\leftmargin
%     \advance\linewidth by\rightmargin
%   \edef\x{\endgroup
%     \def\noexpand\lw{\the\linewidth}^^A
%   }\x
%   \def\lwbox{^^A
%     \leavevmode
%     \hbox to \linewidth{^^A
%       \kern-\leftmargin\relax
%       \hss
%       \usebox0
%       \hss
%       \kern-\rightmargin\relax
%     }^^A
%   }^^A
%   \ifdim\wd0>\lw
%     \sbox0{\small\t}^^A
%     \ifdim\wd0>\linewidth
%       \ifdim\wd0>\lw
%         \sbox0{\footnotesize\t}^^A
%         \ifdim\wd0>\linewidth
%           \ifdim\wd0>\lw
%             \sbox0{\scriptsize\t}^^A
%             \ifdim\wd0>\linewidth
%               \ifdim\wd0>\lw
%                 \sbox0{\tiny\t}^^A
%                 \ifdim\wd0>\linewidth
%                   \lwbox
%                 \else
%                   \usebox0
%                 \fi
%               \else
%                 \lwbox
%               \fi
%             \else
%               \usebox0
%             \fi
%           \else
%             \lwbox
%           \fi
%         \else
%           \usebox0
%         \fi
%       \else
%         \lwbox
%       \fi
%     \else
%       \usebox0
%     \fi
%   \else
%     \lwbox
%   \fi
% \else
%   \usebox0
% \fi
% \end{quote}
% If you have a \xfile{docstrip.cfg} that configures and enables \docstrip's
% TDS installing feature, then some files can already be in the right
% place, see the documentation of \docstrip.
%
% \subsection{Refresh file name databases}
%
% If your \TeX~distribution
% (\teTeX, \mikTeX, \dots) relies on file name databases, you must refresh
% these. For example, \teTeX\ users run \verb|texhash| or
% \verb|mktexlsr|.
%
% \subsection{Some details for the interested}
%
% \paragraph{Attached source.}
%
% The PDF documentation on CTAN also includes the
% \xfile{.dtx} source file. It can be extracted by
% AcrobatReader 6 or higher. Another option is \textsf{pdftk},
% e.g. unpack the file into the current directory:
% \begin{quote}
%   \verb|pdftk hologo.pdf unpack_files output .|
% \end{quote}
%
% \paragraph{Unpacking with \LaTeX.}
% The \xfile{.dtx} chooses its action depending on the format:
% \begin{description}
% \item[\plainTeX:] Run \docstrip\ and extract the files.
% \item[\LaTeX:] Generate the documentation.
% \end{description}
% If you insist on using \LaTeX\ for \docstrip\ (really,
% \docstrip\ does not need \LaTeX), then inform the autodetect routine
% about your intention:
% \begin{quote}
%   \verb|latex \let\install=y\input{hologo.dtx}|
% \end{quote}
% Do not forget to quote the argument according to the demands
% of your shell.
%
% \paragraph{Generating the documentation.}
% You can use both the \xfile{.dtx} or the \xfile{.drv} to generate
% the documentation. The process can be configured by the
% configuration file \xfile{ltxdoc.cfg}. For instance, put this
% line into this file, if you want to have A4 as paper format:
% \begin{quote}
%   \verb|\PassOptionsToClass{a4paper}{article}|
% \end{quote}
% An example follows how to generate the
% documentation with pdf\LaTeX:
% \begin{quote}
%\begin{verbatim}
%pdflatex hologo.dtx
%makeindex -s gind.ist hologo.idx
%pdflatex hologo.dtx
%makeindex -s gind.ist hologo.idx
%pdflatex hologo.dtx
%\end{verbatim}
% \end{quote}
%
% \section{Catalogue}
%
% The following XML file can be used as source for the
% \href{http://mirror.ctan.org/help/Catalogue/catalogue.html}{\TeX\ Catalogue}.
% The elements \texttt{caption} and \texttt{description} are imported
% from the original XML file from the Catalogue.
% The name of the XML file in the Catalogue is \xfile{hologo.xml}.
%    \begin{macrocode}
%<*catalogue>
<?xml version='1.0' encoding='us-ascii'?>
<!DOCTYPE entry SYSTEM 'catalogue.dtd'>
<entry datestamp='$Date$' modifier='$Author$' id='hologo'>
  <name>hologo</name>
  <caption>A collection of logos with bookmark support.</caption>
  <authorref id='auth:oberdiek'/>
  <copyright owner='Heiko Oberdiek' year='2010-2012'/>
  <license type='lppl1.3'/>
  <version number='1.10'/>
  <description>
    The package defines a single command <tt>\hologo</tt>, whose
    argument is the usual case-confused ASCII version of the logo.
    The command is bookmark-enabled, so that every logo becomes
    available in bookmarks without further work.
    <p/>
    The package is part of the <xref refid='oberdiek'>oberdiek</xref>
    bundle.
  </description>
  <documentation details='Package documentation'
      href='ctan:/macros/latex/contrib/oberdiek/hologo.pdf'/>
  <ctan file='true' path='/macros/latex/contrib/oberdiek/hologo.dtx'/>
  <miktex location='oberdiek'/>
  <texlive location='oberdiek'/>
  <install path='/macros/latex/contrib/oberdiek/oberdiek.tds.zip'/>
</entry>
%</catalogue>
%    \end{macrocode}
%
% \begin{thebibliography}{9}
% \raggedright
%
% \bibitem{btxdoc}
% Oren Patashnik,
% \textit{\hologo{BibTeX}ing},
% 1988-02-08.\\
% \CTAN{biblio/bibtex/base/}
%
% \bibitem{dtklogos}
% Gerd Neugebauer, DANTE,
% \textit{Package \xpackage{dtklogos}},
% 2011-04-25.\\
% \CTAN{usergrps/dante/dtk/dtklogos.sty}
%
% \bibitem{etexman}
% The \hologo{NTS} Team,
% \textit{The \hologo{eTeX} manual},
% 1998-02.\\
% \CTAN{systems/e-tex/v2/doc/}
%
% \bibitem{ExTeX-FAQ}
% The \hologo{ExTeX} group,
% \textit{\hologo{ExTeX}: FAQ -- How is \hologo{ExTeX} typeset?},
% 2007-04-14.\\
% \url{http://www.extex.org/documentation/faq.html}
%
% \bibitem{LyX}
% %@MISC{ LyX,
% %  title = {{LyX 2.0.0 -- The Document Processor [Computer software and manual]}},
% %  author = {{The LyX Team}},
% %  howpublished = {Internet: http://www.lyx.org},
% %  year = {2011-05-08},
% %  note = {Retrieved May 10, 2011, from http://www.lyx.org},
% %  url = {http://www.lyx.org/}
% %}
% The \hologo{LyX} Team,
% \textit{\hologo{LyX} -- The Document Processor},
% 2011-05-08.\\
% \url{http://www.lyx.org/}
%
% \bibitem{OzTeX}
% Andrew Trevorrow,
% \hologo{OzTeX} FAQ: What is the correct way to typeset ``\hologo{OzTeX}''?,
% 2011-09-15 (visited).
% \url{http://www.trevorrow.com/oztex/ozfaq.html#oztex-logo}
%
% \bibitem{PiCTeX}
% Michael Wichura,
% \textit{The \hologo{PiCTeX} macro package},
% 1987-09-21.
% \CTAN{graphics/pictex/}
%
% \bibitem{scrlogo}
% Markus Kohm,
% \textit{\hologo{KOMAScript} Datei \xfile{scrlogo.dtx}},
% 2009-01-30.\\
% \CTAN{install/macros/latex/contrib/komascript.tds.zip}
%
% \end{thebibliography}
%
% \begin{History}
%   \begin{Version}{2010/04/08 v1.0}
%   \item
%     The first version.
%   \end{Version}
%   \begin{Version}{2010/04/16 v1.1}
%   \item
%     \cs{Hologo} added for support of logos at start of a sentence.
%   \item
%     \cs{hologoSetup} and \cs{hologoLogoSetup} added.
%   \item
%     Options \xoption{break}, \xoption{hyphenbreak}, \xoption{spacebreak}
%     added.
%   \item
%     Variant support added by option \xoption{variant}.
%   \end{Version}
%   \begin{Version}{2010/04/24 v1.2}
%   \item
%     \hologo{LaTeX3} added.
%   \item
%     \hologo{VTeX} added.
%   \end{Version}
%   \begin{Version}{2010/11/21 v1.3}
%   \item
%     \hologo{iniTeX}, \hologo{virTeX} added.
%   \end{Version}
%   \begin{Version}{2011/03/25 v1.4}
%   \item
%     \hologo{ConTeXt} with variants added.
%   \item
%     Option \xoption{discretionarybreak} added as refinement for
%     option \xoption{break}.
%   \end{Version}
%   \begin{Version}{2011/04/21 v1.5}
%   \item
%     Wrong TDS directory for test files fixed.
%   \end{Version}
%   \begin{Version}{2011/10/01 v1.6}
%   \item
%     Support for package \xpackage{tex4ht} added.
%   \item
%     Support for \cs{csname} added if \cs{ifincsname} is available.
%   \item
%     New logos:
%     \hologo{(La)TeX},
%     \hologo{biber},
%     \hologo{BibTeX} (\xoption{sc}, \xoption{sf}),
%     \hologo{emTeX},
%     \hologo{ExTeX},
%     \hologo{KOMAScript},
%     \hologo{La},
%     \hologo{LyX},
%     \hologo{MiKTeX},
%     \hologo{NTS},
%     \hologo{OzMF},
%     \hologo{OzMP},
%     \hologo{OzTeX},
%     \hologo{OzTtH},
%     \hologo{PCTeX},
%     \hologo{PiC},
%     \hologo{PiCTeX},
%     \hologo{METAFONT},
%     \hologo{MetaFun},
%     \hologo{METAPOST},
%     \hologo{MetaPost},
%     \hologo{SLiTeX} (\xoption{lift}, \xoption{narrow}, \xoption{simple}),
%     \hologo{SliTeX} (\xoption{narrow}, \xoption{simple}, \xoption{lift}),
%     \hologo{teTeX}.
%   \item
%     Fixes:
%     \hologo{iniTeX},
%     \hologo{pdfLaTeX},
%     \hologo{pdfTeX},
%     \hologo{virTeX}.
%   \item
%     \cs{hologoFontSetup} and \cs{hologoLogoFontSetup} added.
%   \item
%     \cs{hologoVariant} and \cs{HologoVariant} added.
%   \end{Version}
%   \begin{Version}{2011/11/22 v1.7}
%   \item
%     New logos:
%     \hologo{BibTeX8},
%     \hologo{LaTeXML},
%     \hologo{SageTeX},
%     \hologo{TeX4ht},
%     \hologo{TTH}.
%   \item
%     \hologo{Xe} and friends: Driver stuff fixed.
%   \item
%     \hologo{Xe} and friends: Support for italic added.
%   \item
%     \hologo{Xe} and friends: Package support for \xpackage{pgf}
%     and \xpackage{pstricks} added.
%   \end{Version}
%   \begin{Version}{2011/11/29 v1.8}
%   \item
%     New logos:
%     \hologo{HanTheThanh}.
%   \end{Version}
%   \begin{Version}{2011/12/21 v1.9}
%   \item
%     Patch for package \xpackage{ifxetex} added for the case that
%     \cs{newif} is undefined in \hologo{iniTeX}.
%   \item
%     Some fixes for \hologo{iniTeX}.
%   \end{Version}
%   \begin{Version}{2012/04/26 v1.10}
%   \item
%     Fix in bookmark version of logo ``\hologo{HanTheThanh}''.
%   \end{Version}
%   \begin{Version}{2016/05/12 v1.11}
%   \item
%     Update HOLOGO@IfCharExists (previously in texlive)
%   \item define pdfliteral in current luatex.
%   \end{Version}
% \end{History}
%
% \PrintIndex
%
% \Finale
\endinput

%        (quote the arguments according to the demands of your shell)
%
% Documentation:
%    (a) If hologo.drv is present:
%           latex hologo.drv
%    (b) Without hologo.drv:
%           latex hologo.dtx; ...
%    The class ltxdoc loads the configuration file ltxdoc.cfg
%    if available. Here you can specify further options, e.g.
%    use A4 as paper format:
%       \PassOptionsToClass{a4paper}{article}
%
%    Programm calls to get the documentation (example):
%       pdflatex hologo.dtx
%       makeindex -s gind.ist hologo.idx
%       pdflatex hologo.dtx
%       makeindex -s gind.ist hologo.idx
%       pdflatex hologo.dtx
%
% Installation:
%    TDS:tex/generic/oberdiek/hologo.sty
%    TDS:doc/latex/oberdiek/hologo.pdf
%    TDS:doc/latex/oberdiek/example/hologo-example.tex
%    TDS:doc/latex/oberdiek/test/hologo-test1.tex
%    TDS:doc/latex/oberdiek/test/hologo-test-spacefactor.tex
%    TDS:doc/latex/oberdiek/test/hologo-test-list.tex
%    TDS:source/latex/oberdiek/hologo.dtx
%
%<*ignore>
\begingroup
  \catcode123=1 %
  \catcode125=2 %
  \def\x{LaTeX2e}%
\expandafter\endgroup
\ifcase 0\ifx\install y1\fi\expandafter
         \ifx\csname processbatchFile\endcsname\relax\else1\fi
         \ifx\fmtname\x\else 1\fi\relax
\else\csname fi\endcsname
%</ignore>
%<*install>
\input docstrip.tex
\Msg{************************************************************************}
\Msg{* Installation}
\Msg{* Package: hologo 2016/05/12 v1.11 A logo collection with bookmark support (HO)}
\Msg{************************************************************************}

\keepsilent
\askforoverwritefalse

\let\MetaPrefix\relax
\preamble

This is a generated file.

Project: hologo
Version: 2016/05/12 v1.11

Copyright (C) 2010-2012 by
   Heiko Oberdiek <heiko.oberdiek at googlemail.com>

This work may be distributed and/or modified under the
conditions of the LaTeX Project Public License, either
version 1.3c of this license or (at your option) any later
version. This version of this license is in
   http://www.latex-project.org/lppl/lppl-1-3c.txt
and the latest version of this license is in
   http://www.latex-project.org/lppl.txt
and version 1.3 or later is part of all distributions of
LaTeX version 2005/12/01 or later.

This work has the LPPL maintenance status "maintained".

This Current Maintainer of this work is Heiko Oberdiek.

The Base Interpreter refers to any `TeX-Format',
because some files are installed in TDS:tex/generic//.

This work consists of the main source file hologo.dtx
and the derived files
   hologo.sty, hologo.pdf, hologo.ins, hologo.drv, hologo-example.tex,
   hologo-test1.tex, hologo-test-spacefactor.tex,
   hologo-test-list.tex.

\endpreamble
\let\MetaPrefix\DoubleperCent

\generate{%
  \file{hologo.ins}{\from{hologo.dtx}{install}}%
  \file{hologo.drv}{\from{hologo.dtx}{driver}}%
  \usedir{tex/generic/oberdiek}%
  \file{hologo.sty}{\from{hologo.dtx}{package}}%
  \usedir{doc/latex/oberdiek/example}%
  \file{hologo-example.tex}{\from{hologo.dtx}{example}}%
  \usedir{doc/latex/oberdiek/test}%
  \file{hologo-test1.tex}{\from{hologo.dtx}{test1}}%
  \file{hologo-test-spacefactor.tex}{\from{hologo.dtx}{test-spacefactor}}%
  \file{hologo-test-list.tex}{\from{hologo.dtx}{test-list}}%
  \nopreamble
  \nopostamble
  \usedir{source/latex/oberdiek/catalogue}%
  \file{hologo.xml}{\from{hologo.dtx}{catalogue}}%
}

\catcode32=13\relax% active space
\let =\space%
\Msg{************************************************************************}
\Msg{*}
\Msg{* To finish the installation you have to move the following}
\Msg{* file into a directory searched by TeX:}
\Msg{*}
\Msg{*     hologo.sty}
\Msg{*}
\Msg{* To produce the documentation run the file `hologo.drv'}
\Msg{* through LaTeX.}
\Msg{*}
\Msg{* Happy TeXing!}
\Msg{*}
\Msg{************************************************************************}

\endbatchfile
%</install>
%<*ignore>
\fi
%</ignore>
%<*driver>
\NeedsTeXFormat{LaTeX2e}
\ProvidesFile{hologo.drv}%
  [2016/05/12 v1.11 A logo collection with bookmark support (HO)]%
\documentclass{ltxdoc}
\usepackage{holtxdoc}[2011/11/22]
\usepackage{hologo}[2016/05/12]
\usepackage{longtable}
\usepackage{array}
\usepackage{paralist}
%\usepackage[T1]{fontenc}
%\usepackage{lmodern}
\begin{document}
  \DocInput{hologo.dtx}%
\end{document}
%</driver>
% \fi
%
%
% \CharacterTable
%  {Upper-case    \A\B\C\D\E\F\G\H\I\J\K\L\M\N\O\P\Q\R\S\T\U\V\W\X\Y\Z
%   Lower-case    \a\b\c\d\e\f\g\h\i\j\k\l\m\n\o\p\q\r\s\t\u\v\w\x\y\z
%   Digits        \0\1\2\3\4\5\6\7\8\9
%   Exclamation   \!     Double quote  \"     Hash (number) \#
%   Dollar        \$     Percent       \%     Ampersand     \&
%   Acute accent  \'     Left paren    \(     Right paren   \)
%   Asterisk      \*     Plus          \+     Comma         \,
%   Minus         \-     Point         \.     Solidus       \/
%   Colon         \:     Semicolon     \;     Less than     \<
%   Equals        \=     Greater than  \>     Question mark \?
%   Commercial at \@     Left bracket  \[     Backslash     \\
%   Right bracket \]     Circumflex    \^     Underscore    \_
%   Grave accent  \`     Left brace    \{     Vertical bar  \|
%   Right brace   \}     Tilde         \~}
%
% \GetFileInfo{hologo.drv}
%
% \title{The \xpackage{hologo} package}
% \date{2016/05/12 v1.11}
% \author{Heiko Oberdiek\\\xemail{heiko.oberdiek at googlemail.com}}
%
% \maketitle
%
% \begin{abstract}
% This package starts a collection of logos with support for bookmarks
% strings.
% \end{abstract}
%
% \tableofcontents
%
% \section{Documentation}
%
% \subsection{Logo macros}
%
% \begin{declcs}{hologo} \M{name}
% \end{declcs}
% Macro \cs{hologo} sets the logo with name \meta{name}.
% The following table shows the supported names.
%
% \begingroup
%   \def\hologoEntry#1#2#3{^^A
%     #1&#2&\hologoLogoSetup{#1}{variant=#2}\hologo{#1}&#3\tabularnewline
%   }
%   \begin{longtable}{>{\ttfamily}l>{\ttfamily}lll}
%     \rmfamily\bfseries{name} & \rmfamily\bfseries variant
%     & \bfseries logo & \bfseries since\\
%     \hline
%     \endhead
%     \hologoList
%   \end{longtable}
% \endgroup
%
% \begin{declcs}{Hologo} \M{name}
% \end{declcs}
% Macro \cs{Hologo} starts the logo \meta{name} with an uppercase
% letter. As an exception small greek letters are not converted
% to uppercase. Examples, see \hologo{eTeX} and \hologo{ExTeX}.
%
% \subsection{Setup macros}
%
% The package does not support package options, but the following
% setup macros can be used to set options.
%
% \begin{declcs}{hologoSetup} \M{key value list}
% \end{declcs}
% Macro \cs{hologoSetup} sets global options.
%
% \begin{declcs}{hologoLogoSetup} \M{logo} \M{key value list}
% \end{declcs}
% Some options can also be used to configure a logo.
% These settings take precedence over global option settings.
%
% \subsection{Options}\label{sec:options}
%
% There are boolean and string options:
% \begin{description}
% \item[Boolean option:]
% It takes |true| or |false|
% as value. If the value is omitted, then |true| is used.
% \item[String option:]
% A value must be given as string. (But the string might be empty.)
% \end{description}
% The following options can be used both in \cs{hologoSetup}
% and \cs{hologoLogoSetup}:
% \begin{description}
% \def\entry#1{\item[\xoption{#1}:]}
% \entry{break}
%   enables or disables line breaks inside the logo. This setting is
%   refined by options \xoption{hyphenbreak}, \xoption{spacebreak}
%   or \xoption{discretionarybreak}.
%   Default is |false|.
% \entry{hyphenbreak}
%   enables or disables the line break right after the hyphen character.
% \entry{spacebreak}
%   enables or disables line breaks at space characters.
% \entry{discretionarybreak}
%   enables or disables line breaks at hyphenation points
%   (inserted by \cs{-}).
% \end{description}
% Macro \cs{hologoLogoSetup} also knows:
% \begin{description}
% \item[\xoption{variant}:]
%   This is a string option. It specifies a variant of a logo that
%   must exist. An empty string selects the package default variant.
% \end{description}
% Example:
% \begin{quote}
%   |\hologoSetup{break=false}|\\
%   |\hologoLogoSetup{plainTeX}{variant=hyphen,hyphenbreak}|\\
%   Then ``plain-\TeX'' contains one break point after the hyphen.
% \end{quote}
%
% \subsection{Driver options}
%
% Sometimes graphical operations are needed to construct some
% glyphs (e.g.\ \hologo{XeTeX}). If package \xpackage{graphics}
% or package \xpackage{pgf} are found, then the macros are taken
% from there. Otherwise the packge defines its own operations
% and therefore needs the driver information. Many drivers are
% detected automatically (\hologo{pdfTeX}/\hologo{LuaTeX}
% in PDF mode, \hologo{XeTeX}, \hologo{VTeX}). These have precedence
% over a driver option. The driver can be given as package option
% or using \cs{hologoDriverSetup}.
% The following list contains the recognized driver options:
% \begin{itemize}
% \item \xoption{pdftex}, \xoption{luatex}
% \item \xoption{dvipdfm}, \xoption{dvipdfmx}
% \item \xoption{dvips}, \xoption{dvipsone}, \xoption{xdvi}
% \item \xoption{xetex}
% \item \xoption{vtex}
% \end{itemize}
% The left driver of a line is the driver name that is used internally.
% The following names are aliases for drivers that use the
% same method. Therefore the entry in the \xext{log} file for
% the used driver prints the internally used driver name.
% \begin{description}
% \item[\xoption{driverfallback}:]
%   This option expects a driver that is used,
%   if the driver could not be detected automatically.
% \end{description}
%
% \begin{declcs}{hologoDriverSetup} \M{driver option}
% \end{declcs}
% The driver can also be configured after package loading
% using \cs{hologoDriverSetup}, also the way for \hologo{plainTeX}
% to setup the driver.
%
% \subsection{Font setup}
%
% Some logos require a special font, but should also be usable by
% \hologo{plainTeX}. Therefore the package provides some ways
% to influence the font settings. The options below
% take font settings as values. Both font commands
% such as \cs{sffamily} and macros that take one argument
% like \cs{textsf} can be used.
%
% \begin{declcs}{hologoFontSetup} \M{key value list}
% \end{declcs}
% Macro \cs{hologoFontSetup} sets the fonts for all logos.
% Supported keys:
% \begin{description}
% \def\entry#1{\item[\xoption{#1}:]}
% \entry{general}
%   This font is used for all logos. The default is empty.
%   That means no special font is used.
% \entry{bibsf}
%   This font is used for
%   {\hologoLogoSetup{BibTeX}{variant=sf}\hologo{BibTeX}}
%   with variant \xoption{sf}.
% \entry{rm}
%   This font is a serif font. It is used for \hologo{ExTeX}.
% \entry{sc}
%   This font specifies a small caps font. It is used for
%   {\hologoLogoSetup{BibTeX}{variant=sc}\hologo{BibTeX}}
%   with variant \xoption{sc}.
% \entry{sf}
%   This font specifies a sans serif font. The default
%   is \cs{sffamily}, then \cs{sf} is tried. Otherwise
%   a warning is given. It is used by \hologo{KOMAScript}.
% \entry{sy}
%   This is the font for math symbols (e.g. cmsy).
%   It is used by \hologo{AmS}, \hologo{NTS}, \hologo{ExTeX}.
% \entry{logo}
%   \hologo{METAFONT} and \hologo{METAPOST} are using that font.
%   In \hologo{LaTeX} \cs{logofamily} is used and
%   the definitions of package \xpackage{mflogo} are used
%   if the package is not loaded.
%   Otherwise the \cs{tenlogo} is used and defined
%   if it does not already exists.
% \end{description}
%
% \begin{declcs}{hologoLogoFontSetup} \M{logo} \M{key value list}
% \end{declcs}
% Fonts can also be set for a logo or logo component separately,
% see the following list.
% The keys are the same as for \cs{hologoFontSetup}.
%
% \begin{longtable}{>{\ttfamily}l>{\sffamily}ll}
%   \meta{logo} & keys & result\\
%   \hline
%   \endhead
%   BibTeX & bibsf & {\hologoLogoSetup{BibTeX}{variant=sf}\hologo{BibTeX}}\\[.5ex]
%   BibTeX & sc & {\hologoLogoSetup{BibTeX}{variant=sc}\hologo{BibTeX}}\\[.5ex]
%   ExTeX & rm & \hologo{ExTeX}\\
%   SliTeX & rm & \hologo{SliTeX}\\[.5ex]
%   AmS & sy & \hologo{AmS}\\
%   ExTeX & sy & \hologo{ExTeX}\\
%   NTS & sy & \hologo{NTS}\\[.5ex]
%   KOMAScript & sf & \hologo{KOMAScript}\\[.5ex]
%   METAFONT & logo & \hologo{METAFONT}\\
%   METAPOST & logo & \hologo{METAPOST}\\[.5ex]
%   SliTeX & sc \hologo{SliTeX}
% \end{longtable}
%
% \subsubsection{Font order}
%
% For all logos the font \xoption{general} is applied first.
% Example:
%\begin{quote}
%|\hologoFontSetup{general=\color{red}}|
%\end{quote}
% will print red logos.
% Then if the font uses a special font \xoption{sf}, for example,
% the font is applied that is setup by \cs{hologoLogoFontSetup}.
% If this font is not setup, then the common font setup
% by \cs{hologoFontSetup} is used. Otherwise a warning is given,
% that there is no font configured.
%
% \subsection{Additional user macros}
%
% Usually a variant of a logo is configured by using
% \cs{hologoLogoSetup}, because it is bad style to mix
% different variants of the same logo in the same text.
% There the following macros are a convenience for testing.
%
% \begin{declcs}{hologoVariant} \M{name} \M{variant}\\
%   \cs{HologoVariant} \M{name} \M{variant}
% \end{declcs}
% Logo \meta{name} is set using \meta{variant} that specifies
% explicitely which variant of the macro is used. If the argument
% is empty, then the default form of the logo is used
% (configurable by \cs{hologoLogoSetup}).
%
% \cs{HologoVariant} is used if the logo is set in a context
% that needs an uppercase first letter (beginning of a sentence, \dots).
%
% \begin{declcs}{hologoList}\\
%   \cs{hologoEntry} \M{logo} \M{variant} \M{since}
% \end{declcs}
% Macro \cs{hologoList} contains all logos that are provided
% by the package including variants. The list consists of calls
% of \cs{hologoEntry} with three arguments starting with the
% logo name \meta{logo} and its variant \meta{variant}. An empty
% variant means the current default. Argument \meta{since} specifies
% with version of the package \xpackage{hologo} is needed to get
% the logo. If the logo is fixed, then the date gets updated.
% Therefore the date \meta{since} is not exactly the date of
% the first introduction, but rather the date of the latest fix.
%
% Before \cs{hologoList} can be used, macro \cs{hologoEntry} needs
% a definition. The example file in section \ref{sec:example}
% shows applications of \cs{hologoList}.
%
% \subsection{Supported contexts}
%
% Macros \cs{hologo} and friends support special contexts:
% \begin{itemize}
% \item \hologo{LaTeX}'s protection mechanism.
% \item Bookmarks of package \xpackage{hyperref}.
% \item Package \xpackage{tex4ht}.
% \item The macros can be used inside \cs{csname} constructs,
%   if \cs{ifincsname} is available (\hologo{pdfTeX}, \hologo{XeTeX},
%   \hologo{LuaTeX}).
% \end{itemize}
%
% \subsection{Example}
% \label{sec:example}
%
% The following example prints the logos in different fonts.
%    \begin{macrocode}
%<*example>
%<<verbatim
\NeedsTeXFormat{LaTeX2e}
\documentclass[a4paper]{article}
\usepackage[
  hmargin=20mm,
  vmargin=20mm,
]{geometry}
\pagestyle{empty}
\usepackage{hologo}[2016/05/12]
\usepackage{longtable}
\usepackage{array}
\setlength{\extrarowheight}{2pt}
\usepackage[T1]{fontenc}
\usepackage{lmodern}
\usepackage{pdflscape}
\usepackage[
  pdfencoding=auto,
]{hyperref}
\hypersetup{
  pdfauthor={Heiko Oberdiek},
  pdftitle={Example for package `hologo'},
  pdfsubject={Logos with fonts lmr, lmss, qtm, qpl, qhv},
}
\usepackage{bookmark}

% Print the logo list on the console

\begingroup
  \typeout{}%
  \typeout{*** Begin of logo list ***}%
  \newcommand*{\hologoEntry}[3]{%
    \typeout{#1 \ifx\\#2\\\else(#2) \fi[#3]}%
  }%
  \hologoList
  \typeout{*** End of logo list ***}%
  \typeout{}%
\endgroup

\begin{document}
\begin{landscape}

  \section{Example file for package `hologo'}

  % Table for font names

  \begin{longtable}{>{\bfseries}ll}
    \textbf{font} & \textbf{Font name}\\
    \hline
    lmr & Latin Modern Roman\\
    lmss & Latin Modern Sans\\
    qtm & \TeX\ Gyre Termes\\
    qhv & \TeX\ Gyre Heros\\
    qpl & \TeX\ Gyre Pagella\\
  \end{longtable}

  % Logo list with logos in different fonts

  \begingroup
    \newcommand*{\SetVariant}[2]{%
      \ifx\\#2\\%
      \else
        \hologoLogoSetup{#1}{variant=#2}%
      \fi
    }%
    \newcommand*{\hologoEntry}[3]{%
      \SetVariant{#1}{#2}%
      \raisebox{1em}[0pt][0pt]{\hypertarget{#1@#2}{}}%
      \bookmark[%
        dest={#1@#2},%
      ]{%
        #1\ifx\\#2\\\else\space(#2)\fi: \Hologo{#1}, \hologo{#1} %
        [Unicode]%
      }%
      \hypersetup{unicode=false}%
      \bookmark[%
        dest={#1@#2},%
      ]{%
        #1\ifx\\#2\\\else\space(#2)\fi: \Hologo{#1}, \hologo{#1} %
        [PDFDocEncoding]%
      }%
      \texttt{#1}%
      &%
      \texttt{#2}%
      &%
      \Hologo{#1}%
      &%
      \SetVariant{#1}{#2}%
      \hologo{#1}%
      &%
      \SetVariant{#1}{#2}%
      \fontfamily{qtm}\selectfont
      \hologo{#1}%
      &%
      \SetVariant{#1}{#2}%
      \fontfamily{qpl}\selectfont
      \hologo{#1}%
      &%
      \SetVariant{#1}{#2}%
      \textsf{\hologo{#1}}%
      &%
      \SetVariant{#1}{#2}%
      \fontfamily{qhv}\selectfont
      \hologo{#1}%
      \tabularnewline
    }%
    \begin{longtable}{llllllll}%
      \textbf{\textit{logo}} & \textbf{\textit{variant}} &
      \texttt{\string\Hologo} &
      \textbf{lmr} & \textbf{qtm} & \textbf{qpl} &
      \textbf{lmss} & \textbf{qhv}
      \tabularnewline
      \hline
      \endhead
      \hologoList
    \end{longtable}%
  \endgroup

\end{landscape}
\end{document}
%verbatim
%</example>
%    \end{macrocode}
%
% \StopEventually{
% }
%
% \section{Implementation}
%    \begin{macrocode}
%<*package>
%    \end{macrocode}
%    Reload check, especially if the package is not used with \LaTeX.
%    \begin{macrocode}
\begingroup\catcode61\catcode48\catcode32=10\relax%
  \catcode13=5 % ^^M
  \endlinechar=13 %
  \catcode35=6 % #
  \catcode39=12 % '
  \catcode44=12 % ,
  \catcode45=12 % -
  \catcode46=12 % .
  \catcode58=12 % :
  \catcode64=11 % @
  \catcode123=1 % {
  \catcode125=2 % }
  \expandafter\let\expandafter\x\csname ver@hologo.sty\endcsname
  \ifx\x\relax % plain-TeX, first loading
  \else
    \def\empty{}%
    \ifx\x\empty % LaTeX, first loading,
      % variable is initialized, but \ProvidesPackage not yet seen
    \else
      \expandafter\ifx\csname PackageInfo\endcsname\relax
        \def\x#1#2{%
          \immediate\write-1{Package #1 Info: #2.}%
        }%
      \else
        \def\x#1#2{\PackageInfo{#1}{#2, stopped}}%
      \fi
      \x{hologo}{The package is already loaded}%
      \aftergroup\endinput
    \fi
  \fi
\endgroup%
%    \end{macrocode}
%    Package identification:
%    \begin{macrocode}
\begingroup\catcode61\catcode48\catcode32=10\relax%
  \catcode13=5 % ^^M
  \endlinechar=13 %
  \catcode35=6 % #
  \catcode39=12 % '
  \catcode40=12 % (
  \catcode41=12 % )
  \catcode44=12 % ,
  \catcode45=12 % -
  \catcode46=12 % .
  \catcode47=12 % /
  \catcode58=12 % :
  \catcode64=11 % @
  \catcode91=12 % [
  \catcode93=12 % ]
  \catcode123=1 % {
  \catcode125=2 % }
  \expandafter\ifx\csname ProvidesPackage\endcsname\relax
    \def\x#1#2#3[#4]{\endgroup
      \immediate\write-1{Package: #3 #4}%
      \xdef#1{#4}%
    }%
  \else
    \def\x#1#2[#3]{\endgroup
      #2[{#3}]%
      \ifx#1\@undefined
        \xdef#1{#3}%
      \fi
      \ifx#1\relax
        \xdef#1{#3}%
      \fi
    }%
  \fi
\expandafter\x\csname ver@hologo.sty\endcsname
\ProvidesPackage{hologo}%
  [2016/05/12 v1.11 A logo collection with bookmark support (HO)]%
%    \end{macrocode}
%
%    \begin{macrocode}
\begingroup\catcode61\catcode48\catcode32=10\relax%
  \catcode13=5 % ^^M
  \endlinechar=13 %
  \catcode123=1 % {
  \catcode125=2 % }
  \catcode64=11 % @
  \def\x{\endgroup
    \expandafter\edef\csname HOLOGO@AtEnd\endcsname{%
      \endlinechar=\the\endlinechar\relax
      \catcode13=\the\catcode13\relax
      \catcode32=\the\catcode32\relax
      \catcode35=\the\catcode35\relax
      \catcode61=\the\catcode61\relax
      \catcode64=\the\catcode64\relax
      \catcode123=\the\catcode123\relax
      \catcode125=\the\catcode125\relax
    }%
  }%
\x\catcode61\catcode48\catcode32=10\relax%
\catcode13=5 % ^^M
\endlinechar=13 %
\catcode35=6 % #
\catcode64=11 % @
\catcode123=1 % {
\catcode125=2 % }
\def\TMP@EnsureCode#1#2{%
  \edef\HOLOGO@AtEnd{%
    \HOLOGO@AtEnd
    \catcode#1=\the\catcode#1\relax
  }%
  \catcode#1=#2\relax
}
\TMP@EnsureCode{10}{12}% ^^J
\TMP@EnsureCode{33}{12}% !
\TMP@EnsureCode{34}{12}% "
\TMP@EnsureCode{36}{3}% $
\TMP@EnsureCode{38}{4}% &
\TMP@EnsureCode{39}{12}% '
\TMP@EnsureCode{40}{12}% (
\TMP@EnsureCode{41}{12}% )
\TMP@EnsureCode{42}{12}% *
\TMP@EnsureCode{43}{12}% +
\TMP@EnsureCode{44}{12}% ,
\TMP@EnsureCode{45}{12}% -
\TMP@EnsureCode{46}{12}% .
\TMP@EnsureCode{47}{12}% /
\TMP@EnsureCode{58}{12}% :
\TMP@EnsureCode{59}{12}% ;
\TMP@EnsureCode{60}{12}% <
\TMP@EnsureCode{62}{12}% >
\TMP@EnsureCode{63}{12}% ?
\TMP@EnsureCode{91}{12}% [
\TMP@EnsureCode{93}{12}% ]
\TMP@EnsureCode{94}{7}% ^ (superscript)
\TMP@EnsureCode{95}{8}% _ (subscript)
\TMP@EnsureCode{96}{12}% `
\TMP@EnsureCode{124}{12}% |
\edef\HOLOGO@AtEnd{%
  \HOLOGO@AtEnd
  \escapechar\the\escapechar\relax
  \noexpand\endinput
}
\escapechar=92 %
%    \end{macrocode}
%
% \subsection{Logo list}
%
%    \begin{macro}{\hologoList}
%    \begin{macrocode}
\def\hologoList{%
  \hologoEntry{(La)TeX}{}{2011/10/01}%
  \hologoEntry{AmSLaTeX}{}{2010/04/16}%
  \hologoEntry{AmSTeX}{}{2010/04/16}%
  \hologoEntry{biber}{}{2011/10/01}%
  \hologoEntry{BibTeX}{}{2011/10/01}%
  \hologoEntry{BibTeX}{sf}{2011/10/01}%
  \hologoEntry{BibTeX}{sc}{2011/10/01}%
  \hologoEntry{BibTeX8}{}{2011/11/22}%
  \hologoEntry{ConTeXt}{}{2011/03/25}%
  \hologoEntry{ConTeXt}{narrow}{2011/03/25}%
  \hologoEntry{ConTeXt}{simple}{2011/03/25}%
  \hologoEntry{emTeX}{}{2010/04/26}%
  \hologoEntry{eTeX}{}{2010/04/08}%
  \hologoEntry{ExTeX}{}{2011/10/01}%
  \hologoEntry{HanTheThanh}{}{2011/11/29}%
  \hologoEntry{iniTeX}{}{2011/10/01}%
  \hologoEntry{KOMAScript}{}{2011/10/01}%
  \hologoEntry{La}{}{2010/05/08}%
  \hologoEntry{LaTeX}{}{2010/04/08}%
  \hologoEntry{LaTeX2e}{}{2010/04/08}%
  \hologoEntry{LaTeX3}{}{2010/04/24}%
  \hologoEntry{LaTeXe}{}{2010/04/08}%
  \hologoEntry{LaTeXML}{}{2011/11/22}%
  \hologoEntry{LaTeXTeX}{}{2011/10/01}%
  \hologoEntry{LuaLaTeX}{}{2010/04/08}%
  \hologoEntry{LuaTeX}{}{2010/04/08}%
  \hologoEntry{LyX}{}{2011/10/01}%
  \hologoEntry{METAFONT}{}{2011/10/01}%
  \hologoEntry{MetaFun}{}{2011/10/01}%
  \hologoEntry{METAPOST}{}{2011/10/01}%
  \hologoEntry{MetaPost}{}{2011/10/01}%
  \hologoEntry{MiKTeX}{}{2011/10/01}%
  \hologoEntry{NTS}{}{2011/10/01}%
  \hologoEntry{OzMF}{}{2011/10/01}%
  \hologoEntry{OzMP}{}{2011/10/01}%
  \hologoEntry{OzTeX}{}{2011/10/01}%
  \hologoEntry{OzTtH}{}{2011/10/01}%
  \hologoEntry{PCTeX}{}{2011/10/01}%
  \hologoEntry{pdfTeX}{}{2011/10/01}%
  \hologoEntry{pdfLaTeX}{}{2011/10/01}%
  \hologoEntry{PiC}{}{2011/10/01}%
  \hologoEntry{PiCTeX}{}{2011/10/01}%
  \hologoEntry{plainTeX}{}{2010/04/08}%
  \hologoEntry{plainTeX}{space}{2010/04/16}%
  \hologoEntry{plainTeX}{hyphen}{2010/04/16}%
  \hologoEntry{plainTeX}{runtogether}{2010/04/16}%
  \hologoEntry{SageTeX}{}{2011/11/22}%
  \hologoEntry{SLiTeX}{}{2011/10/01}%
  \hologoEntry{SLiTeX}{lift}{2011/10/01}%
  \hologoEntry{SLiTeX}{narrow}{2011/10/01}%
  \hologoEntry{SLiTeX}{simple}{2011/10/01}%
  \hologoEntry{SliTeX}{}{2011/10/01}%
  \hologoEntry{SliTeX}{narrow}{2011/10/01}%
  \hologoEntry{SliTeX}{simple}{2011/10/01}%
  \hologoEntry{SliTeX}{lift}{2011/10/01}%
  \hologoEntry{teTeX}{}{2011/10/01}%
  \hologoEntry{TeX}{}{2010/04/08}%
  \hologoEntry{TeX4ht}{}{2011/11/22}%
  \hologoEntry{TTH}{}{2011/11/22}%
  \hologoEntry{virTeX}{}{2011/10/01}%
  \hologoEntry{VTeX}{}{2010/04/24}%
  \hologoEntry{Xe}{}{2010/04/08}%
  \hologoEntry{XeLaTeX}{}{2010/04/08}%
  \hologoEntry{XeTeX}{}{2010/04/08}%
}
%    \end{macrocode}
%    \end{macro}
%
% \subsection{Load resources}
%
%    \begin{macrocode}
\begingroup\expandafter\expandafter\expandafter\endgroup
\expandafter\ifx\csname RequirePackage\endcsname\relax
  \def\TMP@RequirePackage#1[#2]{%
    \begingroup\expandafter\expandafter\expandafter\endgroup
    \expandafter\ifx\csname ver@#1.sty\endcsname\relax
      \input #1.sty\relax
    \fi
  }%
  \TMP@RequirePackage{ltxcmds}[2011/02/04]%
  \TMP@RequirePackage{infwarerr}[2010/04/08]%
  \TMP@RequirePackage{kvsetkeys}[2010/03/01]%
  \TMP@RequirePackage{kvdefinekeys}[2010/03/01]%
  \TMP@RequirePackage{pdftexcmds}[2010/04/01]%
  \TMP@RequirePackage{ifpdf}[2010/01/28]%
  \TMP@RequirePackage{ifluatex}[2010/03/01]%
  \ltx@IfUndefined{newif}{%
    \expandafter\let\csname newif\endcsname\ltx@newif
  }{}%
  \TMP@RequirePackage{ifxetex}[2009/01/23]%
  \TMP@RequirePackage{ifvtex}[2010/03/01]%
\else
  \RequirePackage{ltxcmds}[2011/02/04]%
  \RequirePackage{infwarerr}[2010/04/08]%
  \RequirePackage{kvsetkeys}[2010/03/01]%
  \RequirePackage{kvdefinekeys}[2010/03/01]%
  \RequirePackage{pdftexcmds}[2010/04/01]%
  \RequirePackage{ifpdf}[2010/01/28]%
  \RequirePackage{ifluatex}[2010/03/01]%
  \RequirePackage{ifxetex}[2009/01/23]%
  \RequirePackage{ifvtex}[2010/03/01]%
\fi
%    \end{macrocode}
%
%    \begin{macro}{\HOLOGO@IfDefined}
%    \begin{macrocode}
\def\HOLOGO@IfExists#1{%
  \ifx\@undefined#1%
    \expandafter\ltx@secondoftwo
  \else
    \ifx\relax#1%
      \expandafter\ltx@secondoftwo
    \else
      \expandafter\expandafter\expandafter\ltx@firstoftwo
    \fi
  \fi
}
%    \end{macrocode}
%    \end{macro}
%
% \subsection{Setup macros}
%
%    \begin{macro}{\hologoSetup}
%    \begin{macrocode}
\def\hologoSetup{%
  \let\HOLOGO@name\relax
  \HOLOGO@Setup
}
%    \end{macrocode}
%    \end{macro}
%
%    \begin{macro}{\hologoLogoSetup}
%    \begin{macrocode}
\def\hologoLogoSetup#1{%
  \edef\HOLOGO@name{#1}%
  \ltx@IfUndefined{HoLogo@\HOLOGO@name}{%
    \@PackageError{hologo}{%
      Unknown logo `\HOLOGO@name'%
    }\@ehc
    \ltx@gobble
  }{%
    \HOLOGO@Setup
  }%
}
%    \end{macrocode}
%    \end{macro}
%
%    \begin{macro}{\HOLOGO@Setup}
%    \begin{macrocode}
\def\HOLOGO@Setup{%
  \kvsetkeys{HoLogo}%
}
%    \end{macrocode}
%    \end{macro}
%
% \subsection{Options}
%
%    \begin{macro}{\HOLOGO@DeclareBoolOption}
%    \begin{macrocode}
\def\HOLOGO@DeclareBoolOption#1{%
  \expandafter\chardef\csname HOLOGOOPT@#1\endcsname\ltx@zero
  \kv@define@key{HoLogo}{#1}[true]{%
    \def\HOLOGO@temp{##1}%
    \ifx\HOLOGO@temp\HOLOGO@true
      \ifx\HOLOGO@name\relax
        \expandafter\chardef\csname HOLOGOOPT@#1\endcsname=\ltx@one
      \else
        \expandafter\chardef\csname
        HoLogoOpt@#1@\HOLOGO@name\endcsname\ltx@one
      \fi
      \HOLOGO@SetBreakAll{#1}%
    \else
      \ifx\HOLOGO@temp\HOLOGO@false
        \ifx\HOLOGO@name\relax
          \expandafter\chardef\csname HOLOGOOPT@#1\endcsname=\ltx@zero
        \else
          \expandafter\chardef\csname
          HoLogoOpt@#1@\HOLOGO@name\endcsname=\ltx@zero
        \fi
        \HOLOGO@SetBreakAll{#1}%
      \else
        \@PackageError{hologo}{%
          Unknown value `##1' for boolean option `#1'.\MessageBreak
          Known values are `true' and `false'%
        }\@ehc
      \fi
    \fi
  }%
}
%    \end{macrocode}
%    \end{macro}
%
%    \begin{macro}{\HOLOGO@SetBreakAll}
%    \begin{macrocode}
\def\HOLOGO@SetBreakAll#1{%
  \def\HOLOGO@temp{#1}%
  \ifx\HOLOGO@temp\HOLOGO@break
    \ifx\HOLOGO@name\relax
      \chardef\HOLOGOOPT@hyphenbreak=\HOLOGOOPT@break
      \chardef\HOLOGOOPT@spacebreak=\HOLOGOOPT@break
      \chardef\HOLOGOOPT@discretionarybreak=\HOLOGOOPT@break
    \else
      \expandafter\chardef
         \csname HoLogoOpt@hyphenbreak@\HOLOGO@name\endcsname=%
         \csname HoLogoOpt@break@\HOLOGO@name\endcsname
      \expandafter\chardef
         \csname HoLogoOpt@spacebreak@\HOLOGO@name\endcsname=%
         \csname HoLogoOpt@break@\HOLOGO@name\endcsname
      \expandafter\chardef
         \csname HoLogoOpt@discretionarybreak@\HOLOGO@name
             \endcsname=%
         \csname HoLogoOpt@break@\HOLOGO@name\endcsname
    \fi
  \fi
}
%    \end{macrocode}
%    \end{macro}
%
%    \begin{macro}{\HOLOGO@true}
%    \begin{macrocode}
\def\HOLOGO@true{true}
%    \end{macrocode}
%    \end{macro}
%    \begin{macro}{\HOLOGO@false}
%    \begin{macrocode}
\def\HOLOGO@false{false}
%    \end{macrocode}
%    \end{macro}
%    \begin{macro}{\HOLOGO@break}
%    \begin{macrocode}
\def\HOLOGO@break{break}
%    \end{macrocode}
%    \end{macro}
%
%    \begin{macrocode}
\HOLOGO@DeclareBoolOption{break}
\HOLOGO@DeclareBoolOption{hyphenbreak}
\HOLOGO@DeclareBoolOption{spacebreak}
\HOLOGO@DeclareBoolOption{discretionarybreak}
%    \end{macrocode}
%
%    \begin{macrocode}
\kv@define@key{HoLogo}{variant}{%
  \ifx\HOLOGO@name\relax
    \@PackageError{hologo}{%
      Option `variant' is not available in \string\hologoSetup,%
      \MessageBreak
      Use \string\hologoLogoSetup\space instead%
    }\@ehc
  \else
    \edef\HOLOGO@temp{#1}%
    \ifx\HOLOGO@temp\ltx@empty
      \expandafter
      \let\csname HoLogoOpt@variant@\HOLOGO@name\endcsname\@undefined
    \else
      \ltx@IfUndefined{HoLogo@\HOLOGO@name @\HOLOGO@temp}{%
        \@PackageError{hologo}{%
          Unknown variant `\HOLOGO@temp' of logo `\HOLOGO@name'%
        }\@ehc
      }{%
        \expandafter
        \let\csname HoLogoOpt@variant@\HOLOGO@name\endcsname
            \HOLOGO@temp
      }%
    \fi
  \fi
}
%    \end{macrocode}
%
%    \begin{macro}{\HOLOGO@Variant}
%    \begin{macrocode}
\def\HOLOGO@Variant#1{%
  #1%
  \ltx@ifundefined{HoLogoOpt@variant@#1}{%
  }{%
    @\csname HoLogoOpt@variant@#1\endcsname
  }%
}
%    \end{macrocode}
%    \end{macro}
%
% \subsection{Break/no-break support}
%
%    \begin{macro}{\HOLOGO@space}
%    \begin{macrocode}
\def\HOLOGO@space{%
  \ltx@ifundefined{HoLogoOpt@spacebreak@\HOLOGO@name}{%
    \ltx@ifundefined{HoLogoOpt@break@\HOLOGO@name}{%
      \chardef\HOLOGO@temp=\HOLOGOOPT@spacebreak
    }{%
      \chardef\HOLOGO@temp=%
        \csname HoLogoOpt@break@\HOLOGO@name\endcsname
    }%
  }{%
    \chardef\HOLOGO@temp=%
      \csname HoLogoOpt@spacebreak@\HOLOGO@name\endcsname
  }%
  \ifcase\HOLOGO@temp
    \penalty10000 %
  \fi
  \ltx@space
}
%    \end{macrocode}
%    \end{macro}
%
%    \begin{macro}{\HOLOGO@hyphen}
%    \begin{macrocode}
\def\HOLOGO@hyphen{%
  \ltx@ifundefined{HoLogoOpt@hyphenbreak@\HOLOGO@name}{%
    \ltx@ifundefined{HoLogoOpt@break@\HOLOGO@name}{%
      \chardef\HOLOGO@temp=\HOLOGOOPT@hyphenbreak
    }{%
      \chardef\HOLOGO@temp=%
        \csname HoLogoOpt@break@\HOLOGO@name\endcsname
    }%
  }{%
    \chardef\HOLOGO@temp=%
      \csname HoLogoOpt@hyphenbreak@\HOLOGO@name\endcsname
  }%
  \ifcase\HOLOGO@temp
    \ltx@mbox{-}%
  \else
    -%
  \fi
}
%    \end{macrocode}
%    \end{macro}
%
%    \begin{macro}{\HOLOGO@discretionary}
%    \begin{macrocode}
\def\HOLOGO@discretionary{%
  \ltx@ifundefined{HoLogoOpt@discretionarybreak@\HOLOGO@name}{%
    \ltx@ifundefined{HoLogoOpt@break@\HOLOGO@name}{%
      \chardef\HOLOGO@temp=\HOLOGOOPT@discretionarybreak
    }{%
      \chardef\HOLOGO@temp=%
        \csname HoLogoOpt@break@\HOLOGO@name\endcsname
    }%
  }{%
    \chardef\HOLOGO@temp=%
      \csname HoLogoOpt@discretionarybreak@\HOLOGO@name\endcsname
  }%
  \ifcase\HOLOGO@temp
  \else
    \-%
  \fi
}
%    \end{macrocode}
%    \end{macro}
%
%    \begin{macro}{\HOLOGO@mbox}
%    \begin{macrocode}
\def\HOLOGO@mbox#1{%
  \ltx@ifundefined{HoLogoOpt@break@\HOLOGO@name}{%
    \chardef\HOLOGO@temp=\HOLOGOOPT@hyphenbreak
  }{%
    \chardef\HOLOGO@temp=%
      \csname HoLogoOpt@break@\HOLOGO@name\endcsname
  }%
  \ifcase\HOLOGO@temp
    \ltx@mbox{#1}%
  \else
    #1%
  \fi
}
%    \end{macrocode}
%    \end{macro}
%
% \subsection{Font support}
%
%    \begin{macro}{\HoLogoFont@font}
%    \begin{tabular}{@{}ll@{}}
%    |#1|:& logo name\\
%    |#2|:& font short name\\
%    |#3|:& text
%    \end{tabular}
%    \begin{macrocode}
\def\HoLogoFont@font#1#2#3{%
  \begingroup
    \ltx@IfUndefined{HoLogoFont@logo@#1.#2}{%
      \ltx@IfUndefined{HoLogoFont@font@#2}{%
        \@PackageWarning{hologo}{%
          Missing font `#2' for logo `#1'%
        }%
        #3%
      }{%
        \csname HoLogoFont@font@#2\endcsname{#3}%
      }%
    }{%
      \csname HoLogoFont@logo@#1.#2\endcsname{#3}%
    }%
  \endgroup
}
%    \end{macrocode}
%    \end{macro}
%
%    \begin{macro}{\HoLogoFont@Def}
%    \begin{macrocode}
\def\HoLogoFont@Def#1{%
  \expandafter\def\csname HoLogoFont@font@#1\endcsname
}
%    \end{macrocode}
%    \end{macro}
%    \begin{macro}{\HoLogoFont@LogoDef}
%    \begin{macrocode}
\def\HoLogoFont@LogoDef#1#2{%
  \expandafter\def\csname HoLogoFont@logo@#1.#2\endcsname
}
%    \end{macrocode}
%    \end{macro}
%
% \subsubsection{Font defaults}
%
%    \begin{macro}{\HoLogoFont@font@general}
%    \begin{macrocode}
\HoLogoFont@Def{general}{}%
%    \end{macrocode}
%    \end{macro}
%
%    \begin{macro}{\HoLogoFont@font@rm}
%    \begin{macrocode}
\ltx@IfUndefined{rmfamily}{%
  \ltx@IfUndefined{rm}{%
  }{%
    \HoLogoFont@Def{rm}{\rm}%
  }%
}{%
  \HoLogoFont@Def{rm}{\rmfamily}%
}
%    \end{macrocode}
%    \end{macro}
%
%    \begin{macro}{\HoLogoFont@font@sf}
%    \begin{macrocode}
\ltx@IfUndefined{sffamily}{%
  \ltx@IfUndefined{sf}{%
  }{%
    \HoLogoFont@Def{sf}{\sf}%
  }%
}{%
  \HoLogoFont@Def{sf}{\sffamily}%
}
%    \end{macrocode}
%    \end{macro}
%
%    \begin{macro}{\HoLogoFont@font@bibsf}
%    In case of \hologo{plainTeX} the original small caps
%    variant is used as default. In \hologo{LaTeX}
%    the definition of package \xpackage{dtklogos} \cite{dtklogos}
%    is used.
%\begin{quote}
%\begin{verbatim}
%\DeclareRobustCommand{\BibTeX}{%
%  B%
%  \kern-.05em%
%  \hbox{%
%    $\m@th$% %% force math size calculations
%    \csname S@\f@size\endcsname
%    \fontsize\sf@size\z@
%    \math@fontsfalse
%    \selectfont
%    I%
%    \kern-.025em%
%    B
%  }%
%  \kern-.08em%
%  \-%
%  \TeX
%}
%\end{verbatim}
%\end{quote}
%    \begin{macrocode}
\ltx@IfUndefined{selectfont}{%
  \ltx@IfUndefined{tensc}{%
    \font\tensc=cmcsc10\relax
  }{}%
  \HoLogoFont@Def{bibsf}{\tensc}%
}{%
  \HoLogoFont@Def{bibsf}{%
    $\mathsurround=0pt$%
    \csname S@\f@size\endcsname
    \fontsize\sf@size{0pt}%
    \math@fontsfalse
    \selectfont
  }%
}
%    \end{macrocode}
%    \end{macro}
%
%    \begin{macro}{\HoLogoFont@font@sc}
%    \begin{macrocode}
\ltx@IfUndefined{scshape}{%
  \ltx@IfUndefined{tensc}{%
    \font\tensc=cmcsc10\relax
  }{}%
  \HoLogoFont@Def{sc}{\tensc}%
}{%
  \HoLogoFont@Def{sc}{\scshape}%
}
%    \end{macrocode}
%    \end{macro}
%
%    \begin{macro}{\HoLogoFont@font@sy}
%    \begin{macrocode}
\ltx@IfUndefined{usefont}{%
  \ltx@IfUndefined{tensy}{%
  }{%
    \HoLogoFont@Def{sy}{\tensy}%
  }%
}{%
  \HoLogoFont@Def{sy}{%
    \usefont{OMS}{cmsy}{m}{n}%
  }%
}
%    \end{macrocode}
%    \end{macro}
%
%    \begin{macro}{\HoLogoFont@font@logo}
%    \begin{macrocode}
\begingroup
  \def\x{LaTeX2e}%
\expandafter\endgroup
\ifx\fmtname\x
  \ltx@IfUndefined{logofamily}{%
    \DeclareRobustCommand\logofamily{%
      \not@math@alphabet\logofamily\relax
      \fontencoding{U}%
      \fontfamily{logo}%
      \selectfont
    }%
  }{}%
  \ltx@IfUndefined{logofamily}{%
  }{%
    \HoLogoFont@Def{logo}{\logofamily}%
  }%
\else
  \ltx@IfUndefined{tenlogo}{%
    \font\tenlogo=logo10\relax
  }{}%
  \HoLogoFont@Def{logo}{\tenlogo}%
\fi
%    \end{macrocode}
%    \end{macro}
%
% \subsubsection{Font setup}
%
%    \begin{macro}{\hologoFontSetup}
%    \begin{macrocode}
\def\hologoFontSetup{%
  \let\HOLOGO@name\relax
  \HOLOGO@FontSetup
}
%    \end{macrocode}
%    \end{macro}
%
%    \begin{macro}{\hologoLogoFontSetup}
%    \begin{macrocode}
\def\hologoLogoFontSetup#1{%
  \edef\HOLOGO@name{#1}%
  \ltx@IfUndefined{HoLogo@\HOLOGO@name}{%
    \@PackageError{hologo}{%
      Unknown logo `\HOLOGO@name'%
    }\@ehc
    \ltx@gobble
  }{%
    \HOLOGO@FontSetup
  }%
}
%    \end{macrocode}
%    \end{macro}
%
%    \begin{macro}{\HOLOGO@FontSetup}
%    \begin{macrocode}
\def\HOLOGO@FontSetup{%
  \kvsetkeys{HoLogoFont}%
}
%    \end{macrocode}
%    \end{macro}
%
%    \begin{macrocode}
\def\HOLOGO@temp#1{%
  \kv@define@key{HoLogoFont}{#1}{%
    \ifx\HOLOGO@name\relax
      \HoLogoFont@Def{#1}{##1}%
    \else
      \HoLogoFont@LogoDef\HOLOGO@name{#1}{##1}%
    \fi
  }%
}
\HOLOGO@temp{general}
\HOLOGO@temp{sf}
%    \end{macrocode}
%
% \subsection{Generic logo commands}
%
%    \begin{macrocode}
\HOLOGO@IfExists\hologo{%
  \@PackageError{hologo}{%
    \string\hologo\ltx@space is already defined.\MessageBreak
    Package loading is aborted%
  }\@ehc
  \HOLOGO@AtEnd
}%
\HOLOGO@IfExists\hologoRobust{%
  \@PackageError{hologo}{%
    \string\hologoRobust\ltx@space is already defined.\MessageBreak
    Package loading is aborted%
  }\@ehc
  \HOLOGO@AtEnd
}%
%    \end{macrocode}
%
% \subsubsection{\cs{hologo} and friends}
%
%    \begin{macrocode}
\ifluatex
  \expandafter\ltx@firstofone
\else
  \expandafter\ltx@gobble
\fi
{%
  \ltx@IfUndefined{ifincsname}{%
    \ifnum\luatexversion<36 %
      \expandafter\ltx@gobble
    \else
      \expandafter\ltx@firstofone
    \fi
    {%
      \begingroup
        \ifcase0%
            \directlua{%
              if tex.enableprimitives then %
                tex.enableprimitives('HOLOGO@', {'ifincsname'})%
              else %
                tex.print('1')%
              end%
            }%
            \ifx\HOLOGO@ifincsname\@undefined 1\fi%
            \relax
          \expandafter\ltx@firstofone
        \else
          \endgroup
          \expandafter\ltx@gobble
        \fi
        {%
          \global\let\ifincsname\HOLOGO@ifincsname
        }%
      \HOLOGO@temp
    }%
  }{}%
}
%    \end{macrocode}
%    \begin{macrocode}
\ltx@IfUndefined{ifincsname}{%
  \catcode`$=14 %
}{%
  \catcode`$=9 %
}
%    \end{macrocode}
%
%    \begin{macro}{\hologo}
%    \begin{macrocode}
\def\hologo#1{%
$ \ifincsname
$   \ltx@ifundefined{HoLogoCs@\HOLOGO@Variant{#1}}{%
$     #1%
$   }{%
$     \csname HoLogoCs@\HOLOGO@Variant{#1}\endcsname\ltx@firstoftwo
$   }%
$ \else
    \HOLOGO@IfExists\texorpdfstring\texorpdfstring\ltx@firstoftwo
    {%
      \hologoRobust{#1}%
    }{%
      \ltx@ifundefined{HoLogoBkm@\HOLOGO@Variant{#1}}{%
        \ltx@ifundefined{HoLogo@#1}{?#1?}{#1}%
      }{%
        \csname HoLogoBkm@\HOLOGO@Variant{#1}\endcsname
        \ltx@firstoftwo
      }%
    }%
$ \fi
}
%    \end{macrocode}
%    \end{macro}
%    \begin{macro}{\Hologo}
%    \begin{macrocode}
\def\Hologo#1{%
$ \ifincsname
$   \ltx@ifundefined{HoLogoCs@\HOLOGO@Variant{#1}}{%
$     #1%
$   }{%
$     \csname HoLogoCs@\HOLOGO@Variant{#1}\endcsname\ltx@secondoftwo
$   }%
$ \else
    \HOLOGO@IfExists\texorpdfstring\texorpdfstring\ltx@firstoftwo
    {%
      \HologoRobust{#1}%
    }{%
      \ltx@ifundefined{HoLogoBkm@\HOLOGO@Variant{#1}}{%
        \ltx@ifundefined{HoLogo@#1}{?#1?}{#1}%
      }{%
        \csname HoLogoBkm@\HOLOGO@Variant{#1}\endcsname
        \ltx@secondoftwo
      }%
    }%
$ \fi
}
%    \end{macrocode}
%    \end{macro}
%
%    \begin{macro}{\hologoVariant}
%    \begin{macrocode}
\def\hologoVariant#1#2{%
  \ifx\relax#2\relax
    \hologo{#1}%
  \else
$   \ifincsname
$     \ltx@ifundefined{HoLogoCs@#1@#2}{%
$       #1%
$     }{%
$       \csname HoLogoCs@#1@#2\endcsname\ltx@firstoftwo
$     }%
$   \else
      \HOLOGO@IfExists\texorpdfstring\texorpdfstring\ltx@firstoftwo
      {%
        \hologoVariantRobust{#1}{#2}%
      }{%
        \ltx@ifundefined{HoLogoBkm@#1@#2}{%
          \ltx@ifundefined{HoLogo@#1}{?#1?}{#1}%
        }{%
          \csname HoLogoBkm@#1@#2\endcsname
          \ltx@firstoftwo
        }%
      }%
$   \fi
  \fi
}
%    \end{macrocode}
%    \end{macro}
%    \begin{macro}{\HologoVariant}
%    \begin{macrocode}
\def\HologoVariant#1#2{%
  \ifx\relax#2\relax
    \Hologo{#1}%
  \else
$   \ifincsname
$     \ltx@ifundefined{HoLogoCs@#1@#2}{%
$       #1%
$     }{%
$       \csname HoLogoCs@#1@#2\endcsname\ltx@secondoftwo
$     }%
$   \else
      \HOLOGO@IfExists\texorpdfstring\texorpdfstring\ltx@firstoftwo
      {%
        \HologoVariantRobust{#1}{#2}%
      }{%
        \ltx@ifundefined{HoLogoBkm@#1@#2}{%
          \ltx@ifundefined{HoLogo@#1}{?#1?}{#1}%
        }{%
          \csname HoLogoBkm@#1@#2\endcsname
          \ltx@secondoftwo
        }%
      }%
$   \fi
  \fi
}
%    \end{macrocode}
%    \end{macro}
%
%    \begin{macrocode}
\catcode`\$=3 %
%    \end{macrocode}
%
% \subsubsection{\cs{hologoRobust} and friends}
%
%    \begin{macro}{\hologoRobust}
%    \begin{macrocode}
\ltx@IfUndefined{protected}{%
  \ltx@IfUndefined{DeclareRobustCommand}{%
    \def\hologoRobust#1%
  }{%
    \DeclareRobustCommand*\hologoRobust[1]%
  }%
}{%
  \protected\def\hologoRobust#1%
}%
{%
  \edef\HOLOGO@name{#1}%
  \ltx@IfUndefined{HoLogo@\HOLOGO@Variant\HOLOGO@name}{%
    \@PackageError{hologo}{%
      Unknown logo `\HOLOGO@name'%
    }\@ehc
    ?\HOLOGO@name?%
  }{%
    \ltx@IfUndefined{ver@tex4ht.sty}{%
      \HoLogoFont@font\HOLOGO@name{general}{%
        \csname HoLogo@\HOLOGO@Variant\HOLOGO@name\endcsname
        \ltx@firstoftwo
      }%
    }{%
      \ltx@IfUndefined{HoLogoHtml@\HOLOGO@Variant\HOLOGO@name}{%
        \HOLOGO@name
      }{%
        \csname HoLogoHtml@\HOLOGO@Variant\HOLOGO@name\endcsname
        \ltx@firstoftwo
      }%
    }%
  }%
}
%    \end{macrocode}
%    \end{macro}
%    \begin{macro}{\HologoRobust}
%    \begin{macrocode}
\ltx@IfUndefined{protected}{%
  \ltx@IfUndefined{DeclareRobustCommand}{%
    \def\HologoRobust#1%
  }{%
    \DeclareRobustCommand*\HologoRobust[1]%
  }%
}{%
  \protected\def\HologoRobust#1%
}%
{%
  \edef\HOLOGO@name{#1}%
  \ltx@IfUndefined{HoLogo@\HOLOGO@Variant\HOLOGO@name}{%
    \@PackageError{hologo}{%
      Unknown logo `\HOLOGO@name'%
    }\@ehc
    ?\HOLOGO@name?%
  }{%
    \ltx@IfUndefined{ver@tex4ht.sty}{%
      \HoLogoFont@font\HOLOGO@name{general}{%
        \csname HoLogo@\HOLOGO@Variant\HOLOGO@name\endcsname
        \ltx@secondoftwo
      }%
    }{%
      \ltx@IfUndefined{HoLogoHtml@\HOLOGO@Variant\HOLOGO@name}{%
        \expandafter\HOLOGO@Uppercase\HOLOGO@name
      }{%
        \csname HoLogoHtml@\HOLOGO@Variant\HOLOGO@name\endcsname
        \ltx@secondoftwo
      }%
    }%
  }%
}
%    \end{macrocode}
%    \end{macro}
%    \begin{macro}{\hologoVariantRobust}
%    \begin{macrocode}
\ltx@IfUndefined{protected}{%
  \ltx@IfUndefined{DeclareRobustCommand}{%
    \def\hologoVariantRobust#1#2%
  }{%
    \DeclareRobustCommand*\hologoVariantRobust[2]%
  }%
}{%
  \protected\def\hologoVariantRobust#1#2%
}%
{%
  \begingroup
    \hologoLogoSetup{#1}{variant={#2}}%
    \hologoRobust{#1}%
  \endgroup
}
%    \end{macrocode}
%    \end{macro}
%    \begin{macro}{\HologoVariantRobust}
%    \begin{macrocode}
\ltx@IfUndefined{protected}{%
  \ltx@IfUndefined{DeclareRobustCommand}{%
    \def\HologoVariantRobust#1#2%
  }{%
    \DeclareRobustCommand*\HologoVariantRobust[2]%
  }%
}{%
  \protected\def\HologoVariantRobust#1#2%
}%
{%
  \begingroup
    \hologoLogoSetup{#1}{variant={#2}}%
    \HologoRobust{#1}%
  \endgroup
}
%    \end{macrocode}
%    \end{macro}
%
%    \begin{macro}{\hologorobust}
%    Macro \cs{hologorobust} is only defined for compatibility.
%    Its use is deprecated.
%    \begin{macrocode}
\def\hologorobust{\hologoRobust}
%    \end{macrocode}
%    \end{macro}
%
% \subsection{Helpers}
%
%    \begin{macro}{\HOLOGO@Uppercase}
%    Macro \cs{HOLOGO@Uppercase} is restricted to \cs{uppercase},
%    because \hologo{plainTeX} or \hologo{iniTeX} do not provide
%    \cs{MakeUppercase}.
%    \begin{macrocode}
\def\HOLOGO@Uppercase#1{\uppercase{#1}}
%    \end{macrocode}
%    \end{macro}
%
%    \begin{macro}{\HOLOGO@PdfdocUnicode}
%    \begin{macrocode}
\def\HOLOGO@PdfdocUnicode{%
  \ifx\ifHy@unicode\iftrue
    \expandafter\ltx@secondoftwo
  \else
    \expandafter\ltx@firstoftwo
  \fi
}
%    \end{macrocode}
%    \end{macro}
%
%    \begin{macro}{\HOLOGO@Math}
%    \begin{macrocode}
\def\HOLOGO@MathSetup{%
  \mathsurround0pt\relax
  \HOLOGO@IfExists\f@series{%
    \if b\expandafter\ltx@car\f@series x\@nil
      \csname boldmath\endcsname
   \fi
  }{}%
}
%    \end{macrocode}
%    \end{macro}
%
%    \begin{macro}{\HOLOGO@TempDimen}
%    \begin{macrocode}
\dimendef\HOLOGO@TempDimen=\ltx@zero
%    \end{macrocode}
%    \end{macro}
%    \begin{macro}{\HOLOGO@NegativeKerning}
%    \begin{macrocode}
\def\HOLOGO@NegativeKerning#1{%
  \begingroup
    \HOLOGO@TempDimen=0pt\relax
    \comma@parse@normalized{#1}{%
      \ifdim\HOLOGO@TempDimen=0pt %
        \expandafter\HOLOGO@@NegativeKerning\comma@entry
      \fi
      \ltx@gobble
    }%
    \ifdim\HOLOGO@TempDimen<0pt %
      \kern\HOLOGO@TempDimen
    \fi
  \endgroup
}
%    \end{macrocode}
%    \end{macro}
%    \begin{macro}{\HOLOGO@@NegativeKerning}
%    \begin{macrocode}
\def\HOLOGO@@NegativeKerning#1#2{%
  \setbox\ltx@zero\hbox{#1#2}%
  \HOLOGO@TempDimen=\wd\ltx@zero
  \setbox\ltx@zero\hbox{#1\kern0pt#2}%
  \advance\HOLOGO@TempDimen by -\wd\ltx@zero
}
%    \end{macrocode}
%    \end{macro}
%
%    \begin{macro}{\HOLOGO@SpaceFactor}
%    \begin{macrocode}
\def\HOLOGO@SpaceFactor{%
  \spacefactor1000 %
}
%    \end{macrocode}
%    \end{macro}
%
%    \begin{macro}{\HOLOGO@Span}
%    \begin{macrocode}
\def\HOLOGO@Span#1#2{%
  \HCode{<span class="HoLogo-#1">}%
  #2%
  \HCode{</span>}%
}
%    \end{macrocode}
%    \end{macro}
%
% \subsubsection{Text subscript}
%
%    \begin{macro}{\HOLOGO@SubScript}%
%    \begin{macrocode}
\def\HOLOGO@SubScript#1{%
  \ltx@IfUndefined{textsubscript}{%
    \ltx@IfUndefined{text}{%
      \ltx@mbox{%
        \mathsurround=0pt\relax
        $%
          _{%
            \ltx@IfUndefined{sf@size}{%
              \mathrm{#1}%
            }{%
              \mbox{%
                \fontsize\sf@size{0pt}\selectfont
                #1%
              }%
            }%
          }%
        $%
      }%
    }{%
      \ltx@mbox{%
        \mathsurround=0pt\relax
        $_{\text{#1}}$%
      }%
    }%
  }{%
    \textsubscript{#1}%
  }%
}
%    \end{macrocode}
%    \end{macro}
%
% \subsection{\hologo{TeX} and friends}
%
% \subsubsection{\hologo{TeX}}
%
%    \begin{macro}{\HoLogo@TeX}
%    Source: \hologo{LaTeX} kernel.
%    \begin{macrocode}
\def\HoLogo@TeX#1{%
  T\kern-.1667em\lower.5ex\hbox{E}\kern-.125emX\HOLOGO@SpaceFactor
}
%    \end{macrocode}
%    \end{macro}
%    \begin{macro}{\HoLogoHtml@TeX}
%    \begin{macrocode}
\def\HoLogoHtml@TeX#1{%
  \HoLogoCss@TeX
  \HOLOGO@Span{TeX}{%
    T%
    \HOLOGO@Span{e}{%
      E%
    }%
    X%
  }%
}
%    \end{macrocode}
%    \end{macro}
%    \begin{macro}{\HoLogoCss@TeX}
%    \begin{macrocode}
\def\HoLogoCss@TeX{%
  \Css{%
    span.HoLogo-TeX span.HoLogo-e{%
      position:relative;%
      top:.5ex;%
      margin-left:-.1667em;%
      margin-right:-.125em;%
    }%
  }%
  \Css{%
    a span.HoLogo-TeX span.HoLogo-e{%
      text-decoration:none;%
    }%
  }%
  \global\let\HoLogoCss@TeX\relax
}
%    \end{macrocode}
%    \end{macro}
%
% \subsubsection{\hologo{plainTeX}}
%
%    \begin{macro}{\HoLogo@plainTeX@space}
%    Source: ``The \hologo{TeX}book''
%    \begin{macrocode}
\def\HoLogo@plainTeX@space#1{%
  \HOLOGO@mbox{#1{p}{P}lain}\HOLOGO@space\hologo{TeX}%
}
%    \end{macrocode}
%    \end{macro}
%    \begin{macro}{\HoLogoCs@plainTeX@space}
%    \begin{macrocode}
\def\HoLogoCs@plainTeX@space#1{#1{p}{P}lain TeX}%
%    \end{macrocode}
%    \end{macro}
%    \begin{macro}{\HoLogoBkm@plainTeX@space}
%    \begin{macrocode}
\def\HoLogoBkm@plainTeX@space#1{%
  #1{p}{P}lain \hologo{TeX}%
}
%    \end{macrocode}
%    \end{macro}
%    \begin{macro}{\HoLogoHtml@plainTeX@space}
%    \begin{macrocode}
\def\HoLogoHtml@plainTeX@space#1{%
  #1{p}{P}lain \hologo{TeX}%
}
%    \end{macrocode}
%    \end{macro}
%
%    \begin{macro}{\HoLogo@plainTeX@hyphen}
%    \begin{macrocode}
\def\HoLogo@plainTeX@hyphen#1{%
  \HOLOGO@mbox{#1{p}{P}lain}\HOLOGO@hyphen\hologo{TeX}%
}
%    \end{macrocode}
%    \end{macro}
%    \begin{macro}{\HoLogoCs@plainTeX@hyphen}
%    \begin{macrocode}
\def\HoLogoCs@plainTeX@hyphen#1{#1{p}{P}lain-TeX}
%    \end{macrocode}
%    \end{macro}
%    \begin{macro}{\HoLogoBkm@plainTeX@hyphen}
%    \begin{macrocode}
\def\HoLogoBkm@plainTeX@hyphen#1{%
  #1{p}{P}lain-\hologo{TeX}%
}
%    \end{macrocode}
%    \end{macro}
%    \begin{macro}{\HoLogoHtml@plainTeX@hyphen}
%    \begin{macrocode}
\def\HoLogoHtml@plainTeX@hyphen#1{%
  #1{p}{P}lain-\hologo{TeX}%
}
%    \end{macrocode}
%    \end{macro}
%
%    \begin{macro}{\HoLogo@plainTeX@runtogether}
%    \begin{macrocode}
\def\HoLogo@plainTeX@runtogether#1{%
  \HOLOGO@mbox{#1{p}{P}lain\hologo{TeX}}%
}
%    \end{macrocode}
%    \end{macro}
%    \begin{macro}{\HoLogoCs@plainTeX@runtogether}
%    \begin{macrocode}
\def\HoLogoCs@plainTeX@runtogether#1{#1{p}{P}lainTeX}
%    \end{macrocode}
%    \end{macro}
%    \begin{macro}{\HoLogoBkm@plainTeX@runtogether}
%    \begin{macrocode}
\def\HoLogoBkm@plainTeX@runtogether#1{%
  #1{p}{P}lain\hologo{TeX}%
}
%    \end{macrocode}
%    \end{macro}
%    \begin{macro}{\HoLogoHtml@plainTeX@runtogether}
%    \begin{macrocode}
\def\HoLogoHtml@plainTeX@runtogether#1{%
  #1{p}{P}lain\hologo{TeX}%
}
%    \end{macrocode}
%    \end{macro}
%
%    \begin{macro}{\HoLogo@plainTeX}
%    \begin{macrocode}
\def\HoLogo@plainTeX{\HoLogo@plainTeX@space}
%    \end{macrocode}
%    \end{macro}
%    \begin{macro}{\HoLogoCs@plainTeX}
%    \begin{macrocode}
\def\HoLogoCs@plainTeX{\HoLogoCs@plainTeX@space}
%    \end{macrocode}
%    \end{macro}
%    \begin{macro}{\HoLogoBkm@plainTeX}
%    \begin{macrocode}
\def\HoLogoBkm@plainTeX{\HoLogoBkm@plainTeX@space}
%    \end{macrocode}
%    \end{macro}
%    \begin{macro}{\HoLogoHtml@plainTeX}
%    \begin{macrocode}
\def\HoLogoHtml@plainTeX{\HoLogoHtml@plainTeX@space}
%    \end{macrocode}
%    \end{macro}
%
% \subsubsection{\hologo{LaTeX}}
%
%    Source: \hologo{LaTeX} kernel.
%\begin{quote}
%\begin{verbatim}
%\DeclareRobustCommand{\LaTeX}{%
%  L%
%  \kern-.36em%
%  {%
%    \sbox\z@ T%
%    \vbox to\ht\z@{%
%      \hbox{%
%        \check@mathfonts
%        \fontsize\sf@size\z@
%        \math@fontsfalse
%        \selectfont
%        A%
%      }%
%      \vss
%    }%
%  }%
%  \kern-.15em%
%  \TeX
%}
%\end{verbatim}
%\end{quote}
%
%    \begin{macro}{\HoLogo@La}
%    \begin{macrocode}
\def\HoLogo@La#1{%
  L%
  \kern-.36em%
  \begingroup
    \setbox\ltx@zero\hbox{T}%
    \vbox to\ht\ltx@zero{%
      \hbox{%
        \ltx@ifundefined{check@mathfonts}{%
          \csname sevenrm\endcsname
        }{%
          \check@mathfonts
          \fontsize\sf@size{0pt}%
          \math@fontsfalse\selectfont
        }%
        A%
      }%
      \vss
    }%
  \endgroup
}
%    \end{macrocode}
%    \end{macro}
%
%    \begin{macro}{\HoLogo@LaTeX}
%    Source: \hologo{LaTeX} kernel.
%    \begin{macrocode}
\def\HoLogo@LaTeX#1{%
  \hologo{La}%
  \kern-.15em%
  \hologo{TeX}%
}
%    \end{macrocode}
%    \end{macro}
%    \begin{macro}{\HoLogoHtml@LaTeX}
%    \begin{macrocode}
\def\HoLogoHtml@LaTeX#1{%
  \HoLogoCss@LaTeX
  \HOLOGO@Span{LaTeX}{%
    L%
    \HOLOGO@Span{a}{%
      A%
    }%
    \hologo{TeX}%
  }%
}
%    \end{macrocode}
%    \end{macro}
%    \begin{macro}{\HoLogoCss@LaTeX}
%    \begin{macrocode}
\def\HoLogoCss@LaTeX{%
  \Css{%
    span.HoLogo-LaTeX span.HoLogo-a{%
      position:relative;%
      top:-.5ex;%
      margin-left:-.36em;%
      margin-right:-.15em;%
      font-size:85\%;%
    }%
  }%
  \global\let\HoLogoCss@LaTeX\relax
}
%    \end{macrocode}
%    \end{macro}
%
% \subsubsection{\hologo{(La)TeX}}
%
%    \begin{macro}{\HoLogo@LaTeXTeX}
%    The kerning around the parentheses is taken
%    from package \xpackage{dtklogos} \cite{dtklogos}.
%\begin{quote}
%\begin{verbatim}
%\DeclareRobustCommand{\LaTeXTeX}{%
%  (%
%  \kern-.15em%
%  L%
%  \kern-.36em%
%  {%
%    \sbox\z@ T%
%    \vbox to\ht0{%
%      \hbox{%
%        $\m@th$%
%        \csname S@\f@size\endcsname
%        \fontsize\sf@size\z@
%        \math@fontsfalse
%        \selectfont
%        A%
%      }%
%      \vss
%    }%
%  }%
%  \kern-.2em%
%  )%
%  \kern-.15em%
%  \TeX
%}
%\end{verbatim}
%\end{quote}
%    \begin{macrocode}
\def\HoLogo@LaTeXTeX#1{%
  (%
  \kern-.15em%
  \hologo{La}%
  \kern-.2em%
  )%
  \kern-.15em%
  \hologo{TeX}%
}
%    \end{macrocode}
%    \end{macro}
%    \begin{macro}{\HoLogoBkm@LaTeXTeX}
%    \begin{macrocode}
\def\HoLogoBkm@LaTeXTeX#1{(La)TeX}
%    \end{macrocode}
%    \end{macro}
%
%    \begin{macro}{\HoLogo@(La)TeX}
%    \begin{macrocode}
\expandafter
\let\csname HoLogo@(La)TeX\endcsname\HoLogo@LaTeXTeX
%    \end{macrocode}
%    \end{macro}
%    \begin{macro}{\HoLogoBkm@(La)TeX}
%    \begin{macrocode}
\expandafter
\let\csname HoLogoBkm@(La)TeX\endcsname\HoLogoBkm@LaTeXTeX
%    \end{macrocode}
%    \end{macro}
%    \begin{macro}{\HoLogoHtml@LaTeXTeX}
%    \begin{macrocode}
\def\HoLogoHtml@LaTeXTeX#1{%
  \HoLogoCss@LaTeXTeX
  \HOLOGO@Span{LaTeXTeX}{%
    (%
    \HOLOGO@Span{L}{L}%
    \HOLOGO@Span{a}{A}%
    \HOLOGO@Span{ParenRight}{)}%
    \hologo{TeX}%
  }%
}
%    \end{macrocode}
%    \end{macro}
%    \begin{macro}{\HoLogoHtml@(La)TeX}
%    Kerning after opening parentheses and before closing parentheses
%    is $-0.1$\,em. The original values $-0.15$\,em
%    looked too ugly for a serif font.
%    \begin{macrocode}
\expandafter
\let\csname HoLogoHtml@(La)TeX\endcsname\HoLogoHtml@LaTeXTeX
%    \end{macrocode}
%    \end{macro}
%    \begin{macro}{\HoLogoCss@LaTeXTeX}
%    \begin{macrocode}
\def\HoLogoCss@LaTeXTeX{%
  \Css{%
    span.HoLogo-LaTeXTeX span.HoLogo-L{%
      margin-left:-.1em;%
    }%
  }%
  \Css{%
    span.HoLogo-LaTeXTeX span.HoLogo-a{%
      position:relative;%
      top:-.5ex;%
      margin-left:-.36em;%
      margin-right:-.1em;%
      font-size:85\%;%
    }%
  }%
  \Css{%
    span.HoLogo-LaTeXTeX span.HoLogo-ParenRight{%
      margin-right:-.15em;%
    }%
  }%
  \global\let\HoLogoCss@LaTeXTeX\relax
}
%    \end{macrocode}
%    \end{macro}
%
% \subsubsection{\hologo{LaTeXe}}
%
%    \begin{macro}{\HoLogo@LaTeXe}
%    Source: \hologo{LaTeX} kernel
%    \begin{macrocode}
\def\HoLogo@LaTeXe#1{%
  \hologo{LaTeX}%
  \kern.15em%
  \hbox{%
    \HOLOGO@MathSetup
    2%
    $_{\textstyle\varepsilon}$%
  }%
}
%    \end{macrocode}
%    \end{macro}
%
%    \begin{macro}{\HoLogoCs@LaTeXe}
%    \begin{macrocode}
\ifnum64=`\^^^^0040\relax % test for big chars of LuaTeX/XeTeX
  \catcode`\$=9 %
  \catcode`\&=14 %
\else
  \catcode`\$=14 %
  \catcode`\&=9 %
\fi
\def\HoLogoCs@LaTeXe#1{%
  LaTeX2%
$ \string ^^^^0395%
& e%
}%
\catcode`\$=3 %
\catcode`\&=4 %
%    \end{macrocode}
%    \end{macro}
%
%    \begin{macro}{\HoLogoBkm@LaTeXe}
%    \begin{macrocode}
\def\HoLogoBkm@LaTeXe#1{%
  \hologo{LaTeX}%
  2%
  \HOLOGO@PdfdocUnicode{e}{\textepsilon}%
}
%    \end{macrocode}
%    \end{macro}
%
%    \begin{macro}{\HoLogoHtml@LaTeXe}
%    \begin{macrocode}
\def\HoLogoHtml@LaTeXe#1{%
  \HoLogoCss@LaTeXe
  \HOLOGO@Span{LaTeX2e}{%
    \hologo{LaTeX}%
    \HOLOGO@Span{2}{2}%
    \HOLOGO@Span{e}{%
      \HOLOGO@MathSetup
      \ensuremath{\textstyle\varepsilon}%
    }%
  }%
}
%    \end{macrocode}
%    \end{macro}
%    \begin{macro}{\HoLogoCss@LaTeXe}
%    \begin{macrocode}
\def\HoLogoCss@LaTeXe{%
  \Css{%
    span.HoLogo-LaTeX2e span.HoLogo-2{%
      padding-left:.15em;%
    }%
  }%
  \Css{%
    span.HoLogo-LaTeX2e span.HoLogo-e{%
      position:relative;%
      top:.35ex;%
      text-decoration:none;%
    }%
  }%
  \global\let\HoLogoCss@LaTeXe\relax
}
%    \end{macrocode}
%    \end{macro}
%
%    \begin{macro}{\HoLogo@LaTeX2e}
%    \begin{macrocode}
\expandafter
\let\csname HoLogo@LaTeX2e\endcsname\HoLogo@LaTeXe
%    \end{macrocode}
%    \end{macro}
%    \begin{macro}{\HoLogoCs@LaTeX2e}
%    \begin{macrocode}
\expandafter
\let\csname HoLogoCs@LaTeX2e\endcsname\HoLogoCs@LaTeXe
%    \end{macrocode}
%    \end{macro}
%    \begin{macro}{\HoLogoBkm@LaTeX2e}
%    \begin{macrocode}
\expandafter
\let\csname HoLogoBkm@LaTeX2e\endcsname\HoLogoBkm@LaTeXe
%    \end{macrocode}
%    \end{macro}
%    \begin{macro}{\HoLogoHtml@LaTeX2e}
%    \begin{macrocode}
\expandafter
\let\csname HoLogoHtml@LaTeX2e\endcsname\HoLogoHtml@LaTeXe
%    \end{macrocode}
%    \end{macro}
%
% \subsubsection{\hologo{LaTeX3}}
%
%    \begin{macro}{\HoLogo@LaTeX3}
%    Source: \hologo{LaTeX} kernel
%    \begin{macrocode}
\expandafter\def\csname HoLogo@LaTeX3\endcsname#1{%
  \hologo{LaTeX}%
  3%
}
%    \end{macrocode}
%    \end{macro}
%
%    \begin{macro}{\HoLogoBkm@LaTeX3}
%    \begin{macrocode}
\expandafter\def\csname HoLogoBkm@LaTeX3\endcsname#1{%
  \hologo{LaTeX}%
  3%
}
%    \end{macrocode}
%    \end{macro}
%    \begin{macro}{\HoLogoHtml@LaTeX3}
%    \begin{macrocode}
\expandafter
\let\csname HoLogoHtml@LaTeX3\expandafter\endcsname
\csname HoLogo@LaTeX3\endcsname
%    \end{macrocode}
%    \end{macro}
%
% \subsubsection{\hologo{LaTeXML}}
%
%    \begin{macro}{\HoLogo@LaTeXML}
%    \begin{macrocode}
\def\HoLogo@LaTeXML#1{%
  \HOLOGO@mbox{%
    \hologo{La}%
    \kern-.15em%
    T%
    \kern-.1667em%
    \lower.5ex\hbox{E}%
    \kern-.125em%
    \HoLogoFont@font{LaTeXML}{sc}{xml}%
  }%
}
%    \end{macrocode}
%    \end{macro}
%    \begin{macro}{\HoLogoHtml@pdfLaTeX}
%    \begin{macrocode}
\def\HoLogoHtml@LaTeXML#1{%
  \HOLOGO@Span{LaTeXML}{%
    \HoLogoCss@LaTeX
    \HoLogoCss@TeX
    \HOLOGO@Span{LaTeX}{%
      L%
      \HOLOGO@Span{a}{%
        A%
      }%
    }%
    \HOLOGO@Span{TeX}{%
      T%
      \HOLOGO@Span{e}{%
        E%
      }%
    }%
    \HCode{<span style="font-variant: small-caps;">}%
    xml%
    \HCode{</span>}%
  }%
}
%    \end{macrocode}
%    \end{macro}
%
% \subsubsection{\hologo{eTeX}}
%
%    \begin{macro}{\HoLogo@eTeX}
%    Source: package \xpackage{etex}
%    \begin{macrocode}
\def\HoLogo@eTeX#1{%
  \ltx@mbox{%
    \HOLOGO@MathSetup
    $\varepsilon$%
    -%
    \HOLOGO@NegativeKerning{-T,T-,To}%
    \hologo{TeX}%
  }%
}
%    \end{macrocode}
%    \end{macro}
%    \begin{macro}{\HoLogoCs@eTeX}
%    \begin{macrocode}
\ifnum64=`\^^^^0040\relax % test for big chars of LuaTeX/XeTeX
  \catcode`\$=9 %
  \catcode`\&=14 %
\else
  \catcode`\$=14 %
  \catcode`\&=9 %
\fi
\def\HoLogoCs@eTeX#1{%
$ #1{\string ^^^^0395}{\string ^^^^03b5}%
& #1{e}{E}%
  TeX%
}%
\catcode`\$=3 %
\catcode`\&=4 %
%    \end{macrocode}
%    \end{macro}
%    \begin{macro}{\HoLogoBkm@eTeX}
%    \begin{macrocode}
\def\HoLogoBkm@eTeX#1{%
  \HOLOGO@PdfdocUnicode{#1{e}{E}}{\textepsilon}%
  -%
  \hologo{TeX}%
}
%    \end{macrocode}
%    \end{macro}
%    \begin{macro}{\HoLogoHtml@eTeX}
%    \begin{macrocode}
\def\HoLogoHtml@eTeX#1{%
  \ltx@mbox{%
    \HOLOGO@MathSetup
    $\varepsilon$%
    -%
    \hologo{TeX}%
  }%
}
%    \end{macrocode}
%    \end{macro}
%
% \subsubsection{\hologo{iniTeX}}
%
%    \begin{macro}{\HoLogo@iniTeX}
%    \begin{macrocode}
\def\HoLogo@iniTeX#1{%
  \HOLOGO@mbox{%
    #1{i}{I}ni\hologo{TeX}%
  }%
}
%    \end{macrocode}
%    \end{macro}
%    \begin{macro}{\HoLogoCs@iniTeX}
%    \begin{macrocode}
\def\HoLogoCs@iniTeX#1{#1{i}{I}niTeX}
%    \end{macrocode}
%    \end{macro}
%    \begin{macro}{\HoLogoBkm@iniTeX}
%    \begin{macrocode}
\def\HoLogoBkm@iniTeX#1{%
  #1{i}{I}ni\hologo{TeX}%
}
%    \end{macrocode}
%    \end{macro}
%    \begin{macro}{\HoLogoHtml@iniTeX}
%    \begin{macrocode}
\let\HoLogoHtml@iniTeX\HoLogo@iniTeX
%    \end{macrocode}
%    \end{macro}
%
% \subsubsection{\hologo{virTeX}}
%
%    \begin{macro}{\HoLogo@virTeX}
%    \begin{macrocode}
\def\HoLogo@virTeX#1{%
  \HOLOGO@mbox{%
    #1{v}{V}ir\hologo{TeX}%
  }%
}
%    \end{macrocode}
%    \end{macro}
%    \begin{macro}{\HoLogoCs@virTeX}
%    \begin{macrocode}
\def\HoLogoCs@virTeX#1{#1{v}{V}irTeX}
%    \end{macrocode}
%    \end{macro}
%    \begin{macro}{\HoLogoBkm@virTeX}
%    \begin{macrocode}
\def\HoLogoBkm@virTeX#1{%
  #1{v}{V}ir\hologo{TeX}%
}
%    \end{macrocode}
%    \end{macro}
%    \begin{macro}{\HoLogoHtml@virTeX}
%    \begin{macrocode}
\let\HoLogoHtml@virTeX\HoLogo@virTeX
%    \end{macrocode}
%    \end{macro}
%
% \subsubsection{\hologo{SliTeX}}
%
% \paragraph{Definitions of the three variants.}
%
%    \begin{macro}{\HoLogo@SLiTeX@lift}
%    \begin{macrocode}
\def\HoLogo@SLiTeX@lift#1{%
  \HoLogoFont@font{SliTeX}{rm}{%
    S%
    \kern-.06em%
    L%
    \kern-.18em%
    \raise.32ex\hbox{\HoLogoFont@font{SliTeX}{sc}{i}}%
    \HOLOGO@discretionary
    \kern-.06em%
    \hologo{TeX}%
  }%
}
%    \end{macrocode}
%    \end{macro}
%    \begin{macro}{\HoLogoBkm@SLiTeX@lift}
%    \begin{macrocode}
\def\HoLogoBkm@SLiTeX@lift#1{SLiTeX}
%    \end{macrocode}
%    \end{macro}
%    \begin{macro}{\HoLogoHtml@SLiTeX@lift}
%    \begin{macrocode}
\def\HoLogoHtml@SLiTeX@lift#1{%
  \HoLogoCss@SLiTeX@lift
  \HOLOGO@Span{SLiTeX-lift}{%
    \HoLogoFont@font{SliTeX}{rm}{%
      S%
      \HOLOGO@Span{L}{L}%
      \HOLOGO@Span{i}{i}%
      \hologo{TeX}%
    }%
  }%
}
%    \end{macrocode}
%    \end{macro}
%    \begin{macro}{\HoLogoCss@SLiTeX@lift}
%    \begin{macrocode}
\def\HoLogoCss@SLiTeX@lift{%
  \Css{%
    span.HoLogo-SLiTeX-lift span.HoLogo-L{%
      margin-left:-.06em;%
      margin-right:-.18em;%
    }%
  }%
  \Css{%
    span.HoLogo-SLiTeX-lift span.HoLogo-i{%
      position:relative;%
      top:-.32ex;%
      margin-right:-.06em;%
      font-variant:small-caps;%
    }%
  }%
  \global\let\HoLogoCss@SLiTeX@lift\relax
}
%    \end{macrocode}
%    \end{macro}
%
%    \begin{macro}{\HoLogo@SliTeX@simple}
%    \begin{macrocode}
\def\HoLogo@SliTeX@simple#1{%
  \HoLogoFont@font{SliTeX}{rm}{%
    \ltx@mbox{%
      \HoLogoFont@font{SliTeX}{sc}{Sli}%
    }%
    \HOLOGO@discretionary
    \hologo{TeX}%
  }%
}
%    \end{macrocode}
%    \end{macro}
%    \begin{macro}{\HoLogoBkm@SliTeX@simple}
%    \begin{macrocode}
\def\HoLogoBkm@SliTeX@simple#1{SliTeX}
%    \end{macrocode}
%    \end{macro}
%    \begin{macro}{\HoLogoHtml@SliTeX@simple}
%    \begin{macrocode}
\let\HoLogoHtml@SliTeX@simple\HoLogo@SliTeX@simple
%    \end{macrocode}
%    \end{macro}
%
%    \begin{macro}{\HoLogo@SliTeX@narrow}
%    \begin{macrocode}
\def\HoLogo@SliTeX@narrow#1{%
  \HoLogoFont@font{SliTeX}{rm}{%
    \ltx@mbox{%
      S%
      \kern-.06em%
      \HoLogoFont@font{SliTeX}{sc}{%
        l%
        \kern-.035em%
        i%
      }%
    }%
    \HOLOGO@discretionary
    \kern-.06em%
    \hologo{TeX}%
  }%
}
%    \end{macrocode}
%    \end{macro}
%    \begin{macro}{\HoLogoBkm@SliTeX@narrow}
%    \begin{macrocode}
\def\HoLogoBkm@SliTeX@narrow#1{SliTeX}
%    \end{macrocode}
%    \end{macro}
%    \begin{macro}{\HoLogoHtml@SliTeX@narrow}
%    \begin{macrocode}
\def\HoLogoHtml@SliTeX@narrow#1{%
  \HoLogoCss@SliTeX@narrow
  \HOLOGO@Span{SliTeX-narrow}{%
    \HoLogoFont@font{SliTeX}{rm}{%
      S%
        \HOLOGO@Span{l}{l}%
        \HOLOGO@Span{i}{i}%
      \hologo{TeX}%
    }%
  }%
}
%    \end{macrocode}
%    \end{macro}
%    \begin{macro}{\HoLogoCss@SliTeX@narrow}
%    \begin{macrocode}
\def\HoLogoCss@SliTeX@narrow{%
  \Css{%
    span.HoLogo-SliTeX-narrow span.HoLogo-l{%
      margin-left:-.06em;%
      margin-right:-.035em;%
      font-variant:small-caps;%
    }%
  }%
  \Css{%
    span.HoLogo-SliTeX-narrow span.HoLogo-i{%
      margin-right:-.06em;%
      font-variant:small-caps;%
    }%
  }%
  \global\let\HoLogoCss@SliTeX@narrow\relax
}
%    \end{macrocode}
%    \end{macro}
%
% \paragraph{Macro set completion.}
%
%    \begin{macro}{\HoLogo@SLiTeX@simple}
%    \begin{macrocode}
\def\HoLogo@SLiTeX@simple{\HoLogo@SliTeX@simple}
%    \end{macrocode}
%    \end{macro}
%    \begin{macro}{\HoLogoBkm@SLiTeX@simple}
%    \begin{macrocode}
\def\HoLogoBkm@SLiTeX@simple{\HoLogoBkm@SliTeX@simple}
%    \end{macrocode}
%    \end{macro}
%    \begin{macro}{\HoLogoHtml@SLiTeX@simple}
%    \begin{macrocode}
\def\HoLogoHtml@SLiTeX@simple{\HoLogoHtml@SliTeX@simple}
%    \end{macrocode}
%    \end{macro}
%
%    \begin{macro}{\HoLogo@SLiTeX@narrow}
%    \begin{macrocode}
\def\HoLogo@SLiTeX@narrow{\HoLogo@SliTeX@narrow}
%    \end{macrocode}
%    \end{macro}
%    \begin{macro}{\HoLogoBkm@SLiTeX@narrow}
%    \begin{macrocode}
\def\HoLogoBkm@SLiTeX@narrow{\HoLogoBkm@SliTeX@narrow}
%    \end{macrocode}
%    \end{macro}
%    \begin{macro}{\HoLogoHtml@SLiTeX@narrow}
%    \begin{macrocode}
\def\HoLogoHtml@SLiTeX@narrow{\HoLogoHtml@SliTeX@narrow}
%    \end{macrocode}
%    \end{macro}
%
%    \begin{macro}{\HoLogo@SliTeX@lift}
%    \begin{macrocode}
\def\HoLogo@SliTeX@lift{\HoLogo@SLiTeX@lift}
%    \end{macrocode}
%    \end{macro}
%    \begin{macro}{\HoLogoBkm@SliTeX@lift}
%    \begin{macrocode}
\def\HoLogoBkm@SliTeX@lift{\HoLogoBkm@SLiTeX@lift}
%    \end{macrocode}
%    \end{macro}
%    \begin{macro}{\HoLogoHtml@SliTeX@lift}
%    \begin{macrocode}
\def\HoLogoHtml@SliTeX@lift{\HoLogoHtml@SLiTeX@lift}
%    \end{macrocode}
%    \end{macro}
%
% \paragraph{Defaults.}
%
%    \begin{macro}{\HoLogo@SLiTeX}
%    \begin{macrocode}
\def\HoLogo@SLiTeX{\HoLogo@SLiTeX@lift}
%    \end{macrocode}
%    \end{macro}
%    \begin{macro}{\HoLogoBkm@SLiTeX}
%    \begin{macrocode}
\def\HoLogoBkm@SLiTeX{\HoLogoBkm@SLiTeX@lift}
%    \end{macrocode}
%    \end{macro}
%    \begin{macro}{\HoLogoHtml@SLiTeX}
%    \begin{macrocode}
\def\HoLogoHtml@SLiTeX{\HoLogoHtml@SLiTeX@lift}
%    \end{macrocode}
%    \end{macro}
%
%    \begin{macro}{\HoLogo@SliTeX}
%    \begin{macrocode}
\def\HoLogo@SliTeX{\HoLogo@SliTeX@narrow}
%    \end{macrocode}
%    \end{macro}
%    \begin{macro}{\HoLogoBkm@SliTeX}
%    \begin{macrocode}
\def\HoLogoBkm@SliTeX{\HoLogoBkm@SliTeX@narrow}
%    \end{macrocode}
%    \end{macro}
%    \begin{macro}{\HoLogoHtml@SliTeX}
%    \begin{macrocode}
\def\HoLogoHtml@SliTeX{\HoLogoHtml@SliTeX@narrow}
%    \end{macrocode}
%    \end{macro}
%
% \subsubsection{\hologo{LuaTeX}}
%
%    \begin{macro}{\HoLogo@LuaTeX}
%    The kerning is an idea of Hans Hagen, see mailing list
%    `luatex at tug dot org' in March 2010.
%    \begin{macrocode}
\def\HoLogo@LuaTeX#1{%
  \HOLOGO@mbox{%
    Lua%
    \HOLOGO@NegativeKerning{aT,oT,To}%
    \hologo{TeX}%
  }%
}
%    \end{macrocode}
%    \end{macro}
%    \begin{macro}{\HoLogoHtml@LuaTeX}
%    \begin{macrocode}
\let\HoLogoHtml@LuaTeX\HoLogo@LuaTeX
%    \end{macrocode}
%    \end{macro}
%
% \subsubsection{\hologo{LuaLaTeX}}
%
%    \begin{macro}{\HoLogo@LuaLaTeX}
%    \begin{macrocode}
\def\HoLogo@LuaLaTeX#1{%
  \HOLOGO@mbox{%
    Lua%
    \hologo{LaTeX}%
  }%
}
%    \end{macrocode}
%    \end{macro}
%    \begin{macro}{\HoLogoHtml@LuaLaTeX}
%    \begin{macrocode}
\let\HoLogoHtml@LuaLaTeX\HoLogo@LuaLaTeX
%    \end{macrocode}
%    \end{macro}
%
% \subsubsection{\hologo{XeTeX}, \hologo{XeLaTeX}}
%
%    \begin{macro}{\HOLOGO@IfCharExists}
%    \begin{macrocode}
\ifluatex
  \ifnum\luatexversion<36 %
  \else
    \def\HOLOGO@IfCharExists#1{%
      \ifnum
        \directlua{%
           if luaotfload and luaotfload.aux then
             if luaotfload.aux.font_has_glyph(%
                    font.current(), \number#1) then % 	 
	       tex.print("1") % 	 
	     end % 	 
	   elseif font and font.fonts and font.current then %
            local f = font.fonts[font.current()]%
            if f.characters and f.characters[\number#1] then %
              tex.print("1")%
            end %
          end%
        }0=\ltx@zero
        \expandafter\ltx@secondoftwo
      \else
        \expandafter\ltx@firstoftwo
      \fi
    }%
  \fi
\fi
\ltx@IfUndefined{HOLOGO@IfCharExists}{%
  \def\HOLOGO@@IfCharExists#1{%
    \begingroup
      \tracinglostchars=\ltx@zero
      \setbox\ltx@zero=\hbox{%
        \kern7sp\char#1\relax
        \ifnum\lastkern>\ltx@zero
          \expandafter\aftergroup\csname iffalse\endcsname
        \else
          \expandafter\aftergroup\csname iftrue\endcsname
        \fi
      }%
      % \if{true|false} from \aftergroup
      \endgroup
      \expandafter\ltx@firstoftwo
    \else
      \endgroup
      \expandafter\ltx@secondoftwo
    \fi
  }%
  \ifxetex
    \ltx@IfUndefined{XeTeXfonttype}{}{%
      \ltx@IfUndefined{XeTeXcharglyph}{}{%
        \def\HOLOGO@IfCharExists#1{%
          \ifnum\XeTeXfonttype\font>\ltx@zero
            \expandafter\ltx@firstofthree
          \else
            \expandafter\ltx@gobble
          \fi
          {%
            \ifnum\XeTeXcharglyph#1>\ltx@zero
              \expandafter\ltx@firstoftwo
            \else
              \expandafter\ltx@secondoftwo
            \fi
          }%
          \HOLOGO@@IfCharExists{#1}%
        }%
      }%
    }%
  \fi
}{}
\ltx@ifundefined{HOLOGO@IfCharExists}{%
  \ifnum64=`\^^^^0040\relax % test for big chars of LuaTeX/XeTeX
    \let\HOLOGO@IfCharExists\HOLOGO@@IfCharExists
  \else
    \def\HOLOGO@IfCharExists#1{%
      \ifnum#1>255 %
        \expandafter\ltx@fourthoffour
      \fi
      \HOLOGO@@IfCharExists{#1}%
    }%
  \fi
}{}
%    \end{macrocode}
%    \end{macro}
%
%    \begin{macro}{\HoLogo@Xe}
%    Source: package \xpackage{dtklogos}
%    \begin{macrocode}
\def\HoLogo@Xe#1{%
  X%
  \kern-.1em\relax
  \HOLOGO@IfCharExists{"018E}{%
    \lower.5ex\hbox{\char"018E}%
  }{%
    \chardef\HOLOGO@choice=\ltx@zero
    \ifdim\fontdimen\ltx@one\font>0pt %
      \ltx@IfUndefined{rotatebox}{%
        \ltx@IfUndefined{pgftext}{%
          \ltx@IfUndefined{psscalebox}{%
            \ltx@IfUndefined{HOLOGO@ScaleBox@\hologoDriver}{%
            }{%
              \chardef\HOLOGO@choice=4 %
            }%
          }{%
            \chardef\HOLOGO@choice=3 %
          }%
        }{%
          \chardef\HOLOGO@choice=2 %
        }%
      }{%
        \chardef\HOLOGO@choice=1 %
      }%
      \ifcase\HOLOGO@choice
        \HOLOGO@WarningUnsupportedDriver{Xe}%
        e%
      \or % 1: \rotatebox
        \begingroup
          \setbox\ltx@zero\hbox{\rotatebox{180}{E}}%
          \ltx@LocDimenA=\dp\ltx@zero
          \advance\ltx@LocDimenA by -.5ex\relax
          \raise\ltx@LocDimenA\box\ltx@zero
        \endgroup
      \or % 2: \pgftext
        \lower.5ex\hbox{%
          \pgfpicture
            \pgftext[rotate=180]{E}%
          \endpgfpicture
        }%
      \or % 3: \psscalebox
        \begingroup
          \setbox\ltx@zero\hbox{\psscalebox{-1 -1}{E}}%
          \ltx@LocDimenA=\dp\ltx@zero
          \advance\ltx@LocDimenA by -.5ex\relax
          \raise\ltx@LocDimenA\box\ltx@zero
        \endgroup
      \or % 4: \HOLOGO@PointReflectBox
        \lower.5ex\hbox{\HOLOGO@PointReflectBox{E}}%
      \else
        \@PackageError{hologo}{Internal error (choice/it}\@ehc
      \fi
    \else
      \ltx@IfUndefined{reflectbox}{%
        \ltx@IfUndefined{pgftext}{%
          \ltx@IfUndefined{psscalebox}{%
            \ltx@IfUndefined{HOLOGO@ScaleBox@\hologoDriver}{%
            }{%
              \chardef\HOLOGO@choice=4 %
            }%
          }{%
            \chardef\HOLOGO@choice=3 %
          }%
        }{%
          \chardef\HOLOGO@choice=2 %
        }%
      }{%
        \chardef\HOLOGO@choice=1 %
      }%
      \ifcase\HOLOGO@choice
        \HOLOGO@WarningUnsupportedDriver{Xe}%
        e%
      \or % 1: reflectbox
        \lower.5ex\hbox{%
          \reflectbox{E}%
        }%
      \or % 2: \pgftext
        \lower.5ex\hbox{%
          \pgfpicture
            \pgftransformxscale{-1}%
            \pgftext{E}%
          \endpgfpicture
        }%
      \or % 3: \psscalebox
        \lower.5ex\hbox{%
          \psscalebox{-1 1}{E}%
        }%
      \or % 4: \HOLOGO@Reflectbox
        \lower.5ex\hbox{%
          \HOLOGO@ReflectBox{E}%
        }%
      \else
        \@PackageError{hologo}{Internal error (choice/up)}\@ehc
      \fi
    \fi
  }%
}
%    \end{macrocode}
%    \end{macro}
%    \begin{macro}{\HoLogoHtml@Xe}
%    \begin{macrocode}
\def\HoLogoHtml@Xe#1{%
  \HoLogoCss@Xe
  \HOLOGO@Span{Xe}{%
    X%
    \HOLOGO@Span{e}{%
      \HCode{&\ltx@hashchar x018e;}%
    }%
  }%
}
%    \end{macrocode}
%    \end{macro}
%    \begin{macro}{\HoLogoCss@Xe}
%    \begin{macrocode}
\def\HoLogoCss@Xe{%
  \Css{%
    span.HoLogo-Xe span.HoLogo-e{%
      position:relative;%
      top:.5ex;%
      left-margin:-.1em;%
    }%
  }%
  \global\let\HoLogoCss@Xe\relax
}
%    \end{macrocode}
%    \end{macro}
%
%    \begin{macro}{\HoLogo@XeTeX}
%    \begin{macrocode}
\def\HoLogo@XeTeX#1{%
  \hologo{Xe}%
  \kern-.15em\relax
  \hologo{TeX}%
}
%    \end{macrocode}
%    \end{macro}
%
%    \begin{macro}{\HoLogoHtml@XeTeX}
%    \begin{macrocode}
\def\HoLogoHtml@XeTeX#1{%
  \HoLogoCss@XeTeX
  \HOLOGO@Span{XeTeX}{%
    \hologo{Xe}%
    \hologo{TeX}%
  }%
}
%    \end{macrocode}
%    \end{macro}
%    \begin{macro}{\HoLogoCss@XeTeX}
%    \begin{macrocode}
\def\HoLogoCss@XeTeX{%
  \Css{%
    span.HoLogo-XeTeX span.HoLogo-TeX{%
      margin-left:-.15em;%
    }%
  }%
  \global\let\HoLogoCss@XeTeX\relax
}
%    \end{macrocode}
%    \end{macro}
%
%    \begin{macro}{\HoLogo@XeLaTeX}
%    \begin{macrocode}
\def\HoLogo@XeLaTeX#1{%
  \hologo{Xe}%
  \kern-.13em%
  \hologo{LaTeX}%
}
%    \end{macrocode}
%    \end{macro}
%    \begin{macro}{\HoLogoHtml@XeLaTeX}
%    \begin{macrocode}
\def\HoLogoHtml@XeLaTeX#1{%
  \HoLogoCss@XeLaTeX
  \HOLOGO@Span{XeLaTeX}{%
    \hologo{Xe}%
    \hologo{LaTeX}%
  }%
}
%    \end{macrocode}
%    \end{macro}
%    \begin{macro}{\HoLogoCss@XeLaTeX}
%    \begin{macrocode}
\def\HoLogoCss@XeLaTeX{%
  \Css{%
    span.HoLogo-XeLaTeX span.HoLogo-Xe{%
      margin-right:-.13em;%
    }%
  }%
  \global\let\HoLogoCss@XeLaTeX\relax
}
%    \end{macrocode}
%    \end{macro}
%
% \subsubsection{\hologo{pdfTeX}, \hologo{pdfLaTeX}}
%
%    \begin{macro}{\HoLogo@pdfTeX}
%    \begin{macrocode}
\def\HoLogo@pdfTeX#1{%
  \HOLOGO@mbox{%
    #1{p}{P}df\hologo{TeX}%
  }%
}
%    \end{macrocode}
%    \end{macro}
%    \begin{macro}{\HoLogoCs@pdfTeX}
%    \begin{macrocode}
\def\HoLogoCs@pdfTeX#1{#1{p}{P}dfTeX}
%    \end{macrocode}
%    \end{macro}
%    \begin{macro}{\HoLogoBkm@pdfTeX}
%    \begin{macrocode}
\def\HoLogoBkm@pdfTeX#1{%
  #1{p}{P}df\hologo{TeX}%
}
%    \end{macrocode}
%    \end{macro}
%    \begin{macro}{\HoLogoHtml@pdfTeX}
%    \begin{macrocode}
\let\HoLogoHtml@pdfTeX\HoLogo@pdfTeX
%    \end{macrocode}
%    \end{macro}
%
%    \begin{macro}{\HoLogo@pdfLaTeX}
%    \begin{macrocode}
\def\HoLogo@pdfLaTeX#1{%
  \HOLOGO@mbox{%
    #1{p}{P}df\hologo{LaTeX}%
  }%
}
%    \end{macrocode}
%    \end{macro}
%    \begin{macro}{\HoLogoCs@pdfLaTeX}
%    \begin{macrocode}
\def\HoLogoCs@pdfLaTeX#1{#1{p}{P}dfLaTeX}
%    \end{macrocode}
%    \end{macro}
%    \begin{macro}{\HoLogoBkm@pdfLaTeX}
%    \begin{macrocode}
\def\HoLogoBkm@pdfLaTeX#1{%
  #1{p}{P}df\hologo{LaTeX}%
}
%    \end{macrocode}
%    \end{macro}
%    \begin{macro}{\HoLogoHtml@pdfLaTeX}
%    \begin{macrocode}
\let\HoLogoHtml@pdfLaTeX\HoLogo@pdfLaTeX
%    \end{macrocode}
%    \end{macro}
%
% \subsubsection{\hologo{VTeX}}
%
%    \begin{macro}{\HoLogo@VTeX}
%    \begin{macrocode}
\def\HoLogo@VTeX#1{%
  \HOLOGO@mbox{%
    V\hologo{TeX}%
  }%
}
%    \end{macrocode}
%    \end{macro}
%    \begin{macro}{\HoLogoHtml@VTeX}
%    \begin{macrocode}
\let\HoLogoHtml@VTeX\HoLogo@VTeX
%    \end{macrocode}
%    \end{macro}
%
% \subsubsection{\hologo{AmS}, \dots}
%
%    Source: class \xclass{amsdtx}
%
%    \begin{macro}{\HoLogo@AmS}
%    \begin{macrocode}
\def\HoLogo@AmS#1{%
  \HoLogoFont@font{AmS}{sy}{%
    A%
    \kern-.1667em%
    \lower.5ex\hbox{M}%
    \kern-.125em%
    S%
  }%
}
%    \end{macrocode}
%    \end{macro}
%    \begin{macro}{\HoLogoBkm@AmS}
%    \begin{macrocode}
\def\HoLogoBkm@AmS#1{AmS}
%    \end{macrocode}
%    \end{macro}
%    \begin{macro}{\HoLogoHtml@AmS}
%    \begin{macrocode}
\def\HoLogoHtml@AmS#1{%
  \HoLogoCss@AmS
%  \HoLogoFont@font{AmS}{sy}{%
    \HOLOGO@Span{AmS}{%
      A%
      \HOLOGO@Span{M}{M}%
      S%
    }%
%   }%
}
%    \end{macrocode}
%    \end{macro}
%    \begin{macro}{\HoLogoCss@AmS}
%    \begin{macrocode}
\def\HoLogoCss@AmS{%
  \Css{%
    span.HoLogo-AmS span.HoLogo-M{%
      position:relative;%
      top:.5ex;%
      margin-left:-.1667em;%
      margin-right:-.125em;%
      text-decoration:none;%
    }%
  }%
  \global\let\HoLogoCss@AmS\relax
}
%    \end{macrocode}
%    \end{macro}
%
%    \begin{macro}{\HoLogo@AmSTeX}
%    \begin{macrocode}
\def\HoLogo@AmSTeX#1{%
  \hologo{AmS}%
  \HOLOGO@hyphen
  \hologo{TeX}%
}
%    \end{macrocode}
%    \end{macro}
%    \begin{macro}{\HoLogoBkm@AmSTeX}
%    \begin{macrocode}
\def\HoLogoBkm@AmSTeX#1{AmS-TeX}%
%    \end{macrocode}
%    \end{macro}
%    \begin{macro}{\HoLogoHtml@AmSTeX}
%    \begin{macrocode}
\let\HoLogoHtml@AmSTeX\HoLogo@AmSTeX
%    \end{macrocode}
%    \end{macro}
%
%    \begin{macro}{\HoLogo@AmSLaTeX}
%    \begin{macrocode}
\def\HoLogo@AmSLaTeX#1{%
  \hologo{AmS}%
  \HOLOGO@hyphen
  \hologo{LaTeX}%
}
%    \end{macrocode}
%    \end{macro}
%    \begin{macro}{\HoLogoBkm@AmSLaTeX}
%    \begin{macrocode}
\def\HoLogoBkm@AmSLaTeX#1{AmS-LaTeX}%
%    \end{macrocode}
%    \end{macro}
%    \begin{macro}{\HoLogoHtml@AmSLaTeX}
%    \begin{macrocode}
\let\HoLogoHtml@AmSLaTeX\HoLogo@AmSLaTeX
%    \end{macrocode}
%    \end{macro}
%
% \subsubsection{\hologo{BibTeX}}
%
%    \begin{macro}{\HoLogo@BibTeX@sc}
%    A definition of \hologo{BibTeX} is provided in
%    the documentation source for the manual of \hologo{BibTeX}
%    \cite{btxdoc}.
%\begin{quote}
%\begin{verbatim}
%\def\BibTeX{%
%  {%
%    \rm
%    B%
%    \kern-.05em%
%    {%
%      \sc
%      i%
%      \kern-.025em %
%      b%
%    }%
%    \kern-.08em
%    T%
%    \kern-.1667em%
%    \lower.7ex\hbox{E}%
%    \kern-.125em%
%    X%
%  }%
%}
%\end{verbatim}
%\end{quote}
%    \begin{macrocode}
\def\HoLogo@BibTeX@sc#1{%
  B%
  \kern-.05em%
  \HoLogoFont@font{BibTeX}{sc}{%
    i%
    \kern-.025em%
    b%
  }%
  \HOLOGO@discretionary
  \kern-.08em%
  \hologo{TeX}%
}
%    \end{macrocode}
%    \end{macro}
%    \begin{macro}{\HoLogoHtml@BibTeX@sc}
%    \begin{macrocode}
\def\HoLogoHtml@BibTeX@sc#1{%
  \HoLogoCss@BibTeX@sc
  \HOLOGO@Span{BibTeX-sc}{%
    B%
    \HOLOGO@Span{i}{i}%
    \HOLOGO@Span{b}{b}%
    \hologo{TeX}%
  }%
}
%    \end{macrocode}
%    \end{macro}
%    \begin{macro}{\HoLogoCss@BibTeX@sc}
%    \begin{macrocode}
\def\HoLogoCss@BibTeX@sc{%
  \Css{%
    span.HoLogo-BibTeX-sc span.HoLogo-i{%
      margin-left:-.05em;%
      margin-right:-.025em;%
      font-variant:small-caps;%
    }%
  }%
  \Css{%
    span.HoLogo-BibTeX-sc span.HoLogo-b{%
      margin-right:-.08em;%
      font-variant:small-caps;%
    }%
  }%
  \global\let\HoLogoCss@BibTeX@sc\relax
}
%    \end{macrocode}
%    \end{macro}
%
%    \begin{macro}{\HoLogo@BibTeX@sf}
%    Variant \xoption{sf} avoids trouble with unavailable
%    small caps fonts (e.g., bold versions of Computer Modern or
%    Latin Modern). The definition is taken from
%    package \xpackage{dtklogos} \cite{dtklogos}.
%\begin{quote}
%\begin{verbatim}
%\DeclareRobustCommand{\BibTeX}{%
%  B%
%  \kern-.05em%
%  \hbox{%
%    $\m@th$% %% force math size calculations
%    \csname S@\f@size\endcsname
%    \fontsize\sf@size\z@
%    \math@fontsfalse
%    \selectfont
%    I%
%    \kern-.025em%
%    B
%  }%
%  \kern-.08em%
%  \-%
%  \TeX
%}
%\end{verbatim}
%\end{quote}
%    \begin{macrocode}
\def\HoLogo@BibTeX@sf#1{%
  B%
  \kern-.05em%
  \HoLogoFont@font{BibTeX}{bibsf}{%
    I%
    \kern-.025em%
    B%
  }%
  \HOLOGO@discretionary
  \kern-.08em%
  \hologo{TeX}%
}
%    \end{macrocode}
%    \end{macro}
%    \begin{macro}{\HoLogoHtml@BibTeX@sf}
%    \begin{macrocode}
\def\HoLogoHtml@BibTeX@sf#1{%
  \HoLogoCss@BibTeX@sf
  \HOLOGO@Span{BibTeX-sf}{%
    B%
    \HoLogoFont@font{BibTeX}{bibsf}{%
      \HOLOGO@Span{i}{I}%
      B%
    }%
    \hologo{TeX}%
  }%
}
%    \end{macrocode}
%    \end{macro}
%    \begin{macro}{\HoLogoCss@BibTeX@sf}
%    \begin{macrocode}
\def\HoLogoCss@BibTeX@sf{%
  \Css{%
    span.HoLogo-BibTeX-sf span.HoLogo-i{%
      margin-left:-.05em;%
      margin-right:-.025em;%
    }%
  }%
  \Css{%
    span.HoLogo-BibTeX-sf span.HoLogo-TeX{%
      margin-left:-.08em;%
    }%
  }%
  \global\let\HoLogoCss@BibTeX@sf\relax
}
%    \end{macrocode}
%    \end{macro}
%
%    \begin{macro}{\HoLogo@BibTeX}
%    \begin{macrocode}
\def\HoLogo@BibTeX{\HoLogo@BibTeX@sf}
%    \end{macrocode}
%    \end{macro}
%    \begin{macro}{\HoLogoHtml@BibTeX}
%    \begin{macrocode}
\def\HoLogoHtml@BibTeX{\HoLogoHtml@BibTeX@sf}
%    \end{macrocode}
%    \end{macro}
%
% \subsubsection{\hologo{BibTeX8}}
%
%    \begin{macro}{\HoLogo@BibTeX8}
%    \begin{macrocode}
\expandafter\def\csname HoLogo@BibTeX8\endcsname#1{%
  \hologo{BibTeX}%
  8%
}
%    \end{macrocode}
%    \end{macro}
%
%    \begin{macro}{\HoLogoBkm@BibTeX8}
%    \begin{macrocode}
\expandafter\def\csname HoLogoBkm@BibTeX8\endcsname#1{%
  \hologo{BibTeX}%
  8%
}
%    \end{macrocode}
%    \end{macro}
%    \begin{macro}{\HoLogoHtml@BibTeX8}
%    \begin{macrocode}
\expandafter
\let\csname HoLogoHtml@BibTeX8\expandafter\endcsname
\csname HoLogo@BibTeX8\endcsname
%    \end{macrocode}
%    \end{macro}
%
% \subsubsection{\hologo{ConTeXt}}
%
%    \begin{macro}{\HoLogo@ConTeXt@simple}
%    \begin{macrocode}
\def\HoLogo@ConTeXt@simple#1{%
  \HOLOGO@mbox{Con}%
  \HOLOGO@discretionary
  \HOLOGO@mbox{\hologo{TeX}t}%
}
%    \end{macrocode}
%    \end{macro}
%    \begin{macro}{\HoLogoHtml@ConTeXt@simple}
%    \begin{macrocode}
\let\HoLogoHtml@ConTeXt@simple\HoLogo@ConTeXt@simple
%    \end{macrocode}
%    \end{macro}
%
%    \begin{macro}{\HoLogo@ConTeXt@narrow}
%    This definition of logo \hologo{ConTeXt} with variant \xoption{narrow}
%    comes from TUGboat's class \xclass{ltugboat} (version 2010/11/15 v2.8).
%    \begin{macrocode}
\def\HoLogo@ConTeXt@narrow#1{%
  \HOLOGO@mbox{C\kern-.0333emon}%
  \HOLOGO@discretionary
  \kern-.0667em%
  \HOLOGO@mbox{\hologo{TeX}\kern-.0333emt}%
}
%    \end{macrocode}
%    \end{macro}
%    \begin{macro}{\HoLogoHtml@ConTeXt@narrow}
%    \begin{macrocode}
\def\HoLogoHtml@ConTeXt@narrow#1{%
  \HoLogoCss@ConTeXt@narrow
  \HOLOGO@Span{ConTeXt-narrow}{%
    \HOLOGO@Span{C}{C}%
    on%
    \hologo{TeX}%
    t%
  }%
}
%    \end{macrocode}
%    \end{macro}
%    \begin{macro}{\HoLogoCss@ConTeXt@narrow}
%    \begin{macrocode}
\def\HoLogoCss@ConTeXt@narrow{%
  \Css{%
    span.HoLogo-ConTeXt-narrow span.HoLogo-C{%
      margin-left:-.0333em;%
    }%
  }%
  \Css{%
    span.HoLogo-ConTeXt-narrow span.HoLogo-TeX{%
      margin-left:-.0667em;%
      margin-right:-.0333em;%
    }%
  }%
  \global\let\HoLogoCss@ConTeXt@narrow\relax
}
%    \end{macrocode}
%    \end{macro}
%
%    \begin{macro}{\HoLogo@ConTeXt}
%    \begin{macrocode}
\def\HoLogo@ConTeXt{\HoLogo@ConTeXt@narrow}
%    \end{macrocode}
%    \end{macro}
%    \begin{macro}{\HoLogoHtml@ConTeXt}
%    \begin{macrocode}
\def\HoLogoHtml@ConTeXt{\HoLogoHtml@ConTeXt@narrow}
%    \end{macrocode}
%    \end{macro}
%
% \subsubsection{\hologo{emTeX}}
%
%    \begin{macro}{\HoLogo@emTeX}
%    \begin{macrocode}
\def\HoLogo@emTeX#1{%
  \HOLOGO@mbox{#1{e}{E}m}%
  \HOLOGO@discretionary
  \hologo{TeX}%
}
%    \end{macrocode}
%    \end{macro}
%    \begin{macro}{\HoLogoCs@emTeX}
%    \begin{macrocode}
\def\HoLogoCs@emTeX#1{#1{e}{E}mTeX}%
%    \end{macrocode}
%    \end{macro}
%    \begin{macro}{\HoLogoBkm@emTeX}
%    \begin{macrocode}
\def\HoLogoBkm@emTeX#1{%
  #1{e}{E}m\hologo{TeX}%
}
%    \end{macrocode}
%    \end{macro}
%    \begin{macro}{\HoLogoHtml@emTeX}
%    \begin{macrocode}
\let\HoLogoHtml@emTeX\HoLogo@emTeX
%    \end{macrocode}
%    \end{macro}
%
% \subsubsection{\hologo{ExTeX}}
%
%    \begin{macro}{\HoLogo@ExTeX}
%    The definition is taken from the FAQ of the
%    project \hologo{ExTeX}
%    \cite{ExTeX-FAQ}.
%\begin{quote}
%\begin{verbatim}
%\def\ExTeX{%
%  \textrm{% Logo always with serifs
%    \ensuremath{%
%      \textstyle
%      \varepsilon_{%
%        \kern-0.15em%
%        \mathcal{X}%
%      }%
%    }%
%    \kern-.15em%
%    \TeX
%  }%
%}
%\end{verbatim}
%\end{quote}
%    \begin{macrocode}
\def\HoLogo@ExTeX#1{%
  \HoLogoFont@font{ExTeX}{rm}{%
    \ltx@mbox{%
      \HOLOGO@MathSetup
      $%
        \textstyle
        \varepsilon_{%
          \kern-0.15em%
          \HoLogoFont@font{ExTeX}{sy}{X}%
        }%
      $%
    }%
    \HOLOGO@discretionary
    \kern-.15em%
    \hologo{TeX}%
  }%
}
%    \end{macrocode}
%    \end{macro}
%    \begin{macro}{\HoLogoHtml@ExTeX}
%    \begin{macrocode}
\def\HoLogoHtml@ExTeX#1{%
  \HoLogoCss@ExTeX
  \HoLogoFont@font{ExTeX}{rm}{%
    \HOLOGO@Span{ExTeX}{%
      \ltx@mbox{%
        \HOLOGO@MathSetup
        $\textstyle\varepsilon$%
        \HOLOGO@Span{X}{$\textstyle\chi$}%
        \hologo{TeX}%
      }%
    }%
  }%
}
%    \end{macrocode}
%    \end{macro}
%    \begin{macro}{\HoLogoBkm@ExTeX}
%    \begin{macrocode}
\def\HoLogoBkm@ExTeX#1{%
  \HOLOGO@PdfdocUnicode{#1{e}{E}x}{\textepsilon\textchi}%
  \hologo{TeX}%
}
%    \end{macrocode}
%    \end{macro}
%    \begin{macro}{\HoLogoCss@ExTeX}
%    \begin{macrocode}
\def\HoLogoCss@ExTeX{%
  \Css{%
    span.HoLogo-ExTeX{%
      font-family:serif;%
    }%
  }%
  \Css{%
    span.HoLogo-ExTeX span.HoLogo-TeX{%
      margin-left:-.15em;%
    }%
  }%
  \global\let\HoLogoCss@ExTeX\relax
}
%    \end{macrocode}
%    \end{macro}
%
% \subsubsection{\hologo{MiKTeX}}
%
%    \begin{macro}{\HoLogo@MiKTeX}
%    \begin{macrocode}
\def\HoLogo@MiKTeX#1{%
  \HOLOGO@mbox{MiK}%
  \HOLOGO@discretionary
  \hologo{TeX}%
}
%    \end{macrocode}
%    \end{macro}
%    \begin{macro}{\HoLogoHtml@MiKTeX}
%    \begin{macrocode}
\let\HoLogoHtml@MiKTeX\HoLogo@MiKTeX
%    \end{macrocode}
%    \end{macro}
%
% \subsubsection{\hologo{OzTeX} and friends}
%
%    Source: \hologo{OzTeX} FAQ \cite{OzTeX}:
%    \begin{quote}
%      |\def\OzTeX{O\kern-.03em z\kern-.15em\TeX}|\\
%      (There is no kerning in OzMF, OzMP and OzTtH.)
%    \end{quote}
%
%    \begin{macro}{\HoLogo@OzTeX}
%    \begin{macrocode}
\def\HoLogo@OzTeX#1{%
  O%
  \kern-.03em %
  z%
  \kern-.15em %
  \hologo{TeX}%
}
%    \end{macrocode}
%    \end{macro}
%    \begin{macro}{\HoLogoHtml@OzTeX}
%    \begin{macrocode}
\def\HoLogoHtml@OzTeX#1{%
  \HoLogoCss@OzTeX
  \HOLOGO@Span{OzTeX}{%
    O%
    \HOLOGO@Span{z}{z}%
    \hologo{TeX}%
  }%
}
%    \end{macrocode}
%    \end{macro}
%    \begin{macro}{\HoLogoCss@OzTeX}
%    \begin{macrocode}
\def\HoLogoCss@OzTeX{%
  \Css{%
    span.HoLogo-OzTeX span.HoLogo-z{%
      margin-left:-.03em;%
      margin-right:-.15em;%
    }%
  }%
  \global\let\HoLogoCss@OzTeX\relax
}
%    \end{macrocode}
%    \end{macro}
%
%    \begin{macro}{\HoLogo@OzMF}
%    \begin{macrocode}
\def\HoLogo@OzMF#1{%
  \HOLOGO@mbox{OzMF}%
}
%    \end{macrocode}
%    \end{macro}
%    \begin{macro}{\HoLogo@OzMP}
%    \begin{macrocode}
\def\HoLogo@OzMP#1{%
  \HOLOGO@mbox{OzMP}%
}
%    \end{macrocode}
%    \end{macro}
%    \begin{macro}{\HoLogo@OzTtH}
%    \begin{macrocode}
\def\HoLogo@OzTtH#1{%
  \HOLOGO@mbox{OzTtH}%
}
%    \end{macrocode}
%    \end{macro}
%
% \subsubsection{\hologo{PCTeX}}
%
%    \begin{macro}{\HoLogo@PCTeX}
%    \begin{macrocode}
\def\HoLogo@PCTeX#1{%
  \HOLOGO@mbox{PC}%
  \hologo{TeX}%
}
%    \end{macrocode}
%    \end{macro}
%    \begin{macro}{\HoLogoHtml@PCTeX}
%    \begin{macrocode}
\let\HoLogoHtml@PCTeX\HoLogo@PCTeX
%    \end{macrocode}
%    \end{macro}
%
% \subsubsection{\hologo{PiCTeX}}
%
%    The original definitions from \xfile{pictex.tex} \cite{PiCTeX}:
%\begin{quote}
%\begin{verbatim}
%\def\PiC{%
%  P%
%  \kern-.12em%
%  \lower.5ex\hbox{I}%
%  \kern-.075em%
%  C%
%}
%\def\PiCTeX{%
%  \PiC
%  \kern-.11em%
%  \TeX
%}
%\end{verbatim}
%\end{quote}
%
%    \begin{macro}{\HoLogo@PiC}
%    \begin{macrocode}
\def\HoLogo@PiC#1{%
  P%
  \kern-.12em%
  \lower.5ex\hbox{I}%
  \kern-.075em%
  C%
  \HOLOGO@SpaceFactor
}
%    \end{macrocode}
%    \end{macro}
%    \begin{macro}{\HoLogoHtml@PiC}
%    \begin{macrocode}
\def\HoLogoHtml@PiC#1{%
  \HoLogoCss@PiC
  \HOLOGO@Span{PiC}{%
    P%
    \HOLOGO@Span{i}{I}%
    C%
  }%
}
%    \end{macrocode}
%    \end{macro}
%    \begin{macro}{\HoLogoCss@PiC}
%    \begin{macrocode}
\def\HoLogoCss@PiC{%
  \Css{%
    span.HoLogo-PiC span.HoLogo-i{%
      position:relative;%
      top:.5ex;%
      margin-left:-.12em;%
      margin-right:-.075em;%
      text-decoration:none;%
    }%
  }%
  \global\let\HoLogoCss@PiC\relax
}
%    \end{macrocode}
%    \end{macro}
%
%    \begin{macro}{\HoLogo@PiCTeX}
%    \begin{macrocode}
\def\HoLogo@PiCTeX#1{%
  \hologo{PiC}%
  \HOLOGO@discretionary
  \kern-.11em%
  \hologo{TeX}%
}
%    \end{macrocode}
%    \end{macro}
%    \begin{macro}{\HoLogoHtml@PiCTeX}
%    \begin{macrocode}
\def\HoLogoHtml@PiCTeX#1{%
  \HoLogoCss@PiCTeX
  \HOLOGO@Span{PiCTeX}{%
    \hologo{PiC}%
    \hologo{TeX}%
  }%
}
%    \end{macrocode}
%    \end{macro}
%    \begin{macro}{\HoLogoCss@PiCTeX}
%    \begin{macrocode}
\def\HoLogoCss@PiCTeX{%
  \Css{%
    span.HoLogo-PiCTeX span.HoLogo-PiC{%
      margin-right:-.11em;%
    }%
  }%
  \global\let\HoLogoCss@PiCTeX\relax
}
%    \end{macrocode}
%    \end{macro}
%
% \subsubsection{\hologo{teTeX}}
%
%    \begin{macro}{\HoLogo@teTeX}
%    \begin{macrocode}
\def\HoLogo@teTeX#1{%
  \HOLOGO@mbox{#1{t}{T}e}%
  \HOLOGO@discretionary
  \hologo{TeX}%
}
%    \end{macrocode}
%    \end{macro}
%    \begin{macro}{\HoLogoCs@teTeX}
%    \begin{macrocode}
\def\HoLogoCs@teTeX#1{#1{t}{T}dfTeX}
%    \end{macrocode}
%    \end{macro}
%    \begin{macro}{\HoLogoBkm@teTeX}
%    \begin{macrocode}
\def\HoLogoBkm@teTeX#1{%
  #1{t}{T}e\hologo{TeX}%
}
%    \end{macrocode}
%    \end{macro}
%    \begin{macro}{\HoLogoHtml@teTeX}
%    \begin{macrocode}
\let\HoLogoHtml@teTeX\HoLogo@teTeX
%    \end{macrocode}
%    \end{macro}
%
% \subsubsection{\hologo{TeX4ht}}
%
%    \begin{macro}{\HoLogo@TeX4ht}
%    \begin{macrocode}
\expandafter\def\csname HoLogo@TeX4ht\endcsname#1{%
  \HOLOGO@mbox{\hologo{TeX}4ht}%
}
%    \end{macrocode}
%    \end{macro}
%    \begin{macro}{\HoLogoHtml@TeX4ht}
%    \begin{macrocode}
\expandafter
\let\csname HoLogoHtml@TeX4ht\expandafter\endcsname
\csname HoLogo@TeX4ht\endcsname
%    \end{macrocode}
%    \end{macro}
%
%
% \subsubsection{\hologo{SageTeX}}
%
%    \begin{macro}{\HoLogo@SageTeX}
%    \begin{macrocode}
\def\HoLogo@SageTeX#1{%
  \HOLOGO@mbox{Sage}%
  \HOLOGO@discretionary
  \HOLOGO@NegativeKerning{eT,oT,To}%
  \hologo{TeX}%
}
%    \end{macrocode}
%    \end{macro}
%    \begin{macro}{\HoLogoHtml@SageTeX}
%    \begin{macrocode}
\let\HoLogoHtml@SageTeX\HoLogo@SageTeX
%    \end{macrocode}
%    \end{macro}
%
% \subsection{\hologo{METAFONT} and friends}
%
%    \begin{macro}{\HoLogo@METAFONT}
%    \begin{macrocode}
\def\HoLogo@METAFONT#1{%
  \HoLogoFont@font{METAFONT}{logo}{%
    \HOLOGO@mbox{META}%
    \HOLOGO@discretionary
    \HOLOGO@mbox{FONT}%
  }%
}
%    \end{macrocode}
%    \end{macro}
%
%    \begin{macro}{\HoLogo@METAPOST}
%    \begin{macrocode}
\def\HoLogo@METAPOST#1{%
  \HoLogoFont@font{METAPOST}{logo}{%
    \HOLOGO@mbox{META}%
    \HOLOGO@discretionary
    \HOLOGO@mbox{POST}%
  }%
}
%    \end{macrocode}
%    \end{macro}
%
%    \begin{macro}{\HoLogo@MetaFun}
%    \begin{macrocode}
\def\HoLogo@MetaFun#1{%
  \HOLOGO@mbox{Meta}%
  \HOLOGO@discretionary
  \HOLOGO@mbox{Fun}%
}
%    \end{macrocode}
%    \end{macro}
%
%    \begin{macro}{\HoLogo@MetaPost}
%    \begin{macrocode}
\def\HoLogo@MetaPost#1{%
  \HOLOGO@mbox{Meta}%
  \HOLOGO@discretionary
  \HOLOGO@mbox{Post}%
}
%    \end{macrocode}
%    \end{macro}
%
% \subsection{Others}
%
% \subsubsection{\hologo{biber}}
%
%    \begin{macro}{\HoLogo@biber}
%    \begin{macrocode}
\def\HoLogo@biber#1{%
  \HOLOGO@mbox{#1{b}{B}i}%
  \HOLOGO@discretionary
  \HOLOGO@mbox{ber}%
}
%    \end{macrocode}
%    \end{macro}
%    \begin{macro}{\HoLogoCs@biber}
%    \begin{macrocode}
\def\HoLogoCs@biber#1{#1{b}{B}iber}
%    \end{macrocode}
%    \end{macro}
%    \begin{macro}{\HoLogoBkm@biber}
%    \begin{macrocode}
\def\HoLogoBkm@biber#1{%
  #1{b}{B}iber%
}
%    \end{macrocode}
%    \end{macro}
%    \begin{macro}{\HoLogoHtml@biber}
%    \begin{macrocode}
\let\HoLogoHtml@biber\HoLogo@biber
%    \end{macrocode}
%    \end{macro}
%
% \subsubsection{\hologo{KOMAScript}}
%
%    \begin{macro}{\HoLogo@KOMAScript}
%    The definition for \hologo{KOMAScript} is taken
%    from \hologo{KOMAScript} (\xfile{scrlogo.dtx}, reformatted) \cite{scrlogo}:
%\begin{quote}
%\begin{verbatim}
%\@ifundefined{KOMAScript}{%
%  \DeclareRobustCommand{\KOMAScript}{%
%    \textsf{%
%      K\kern.05em O\kern.05emM\kern.05em A%
%      \kern.1em-\kern.1em %
%      Script%
%    }%
%  }%
%}{}
%\end{verbatim}
%\end{quote}
%    \begin{macrocode}
\def\HoLogo@KOMAScript#1{%
  \HoLogoFont@font{KOMAScript}{sf}{%
    \HOLOGO@mbox{%
      K\kern.05em%
      O\kern.05em%
      M\kern.05em%
      A%
    }%
    \kern.1em%
    \HOLOGO@hyphen
    \kern.1em%
    \HOLOGO@mbox{Script}%
  }%
}
%    \end{macrocode}
%    \end{macro}
%    \begin{macro}{\HoLogoBkm@KOMAScript}
%    \begin{macrocode}
\def\HoLogoBkm@KOMAScript#1{%
  KOMA-Script%
}
%    \end{macrocode}
%    \end{macro}
%    \begin{macro}{\HoLogoHtml@KOMAScript}
%    \begin{macrocode}
\def\HoLogoHtml@KOMAScript#1{%
  \HoLogoCss@KOMAScript
  \HoLogoFont@font{KOMAScript}{sf}{%
    \HOLOGO@Span{KOMAScript}{%
      K%
      \HOLOGO@Span{O}{O}%
      M%
      \HOLOGO@Span{A}{A}%
      \HOLOGO@Span{hyphen}{-}%
      Script%
    }%
  }%
}
%    \end{macrocode}
%    \end{macro}
%    \begin{macro}{\HoLogoCss@KOMAScript}
%    \begin{macrocode}
\def\HoLogoCss@KOMAScript{%
  \Css{%
    span.HoLogo-KOMAScript{%
      font-family:sans-serif;%
    }%
  }%
  \Css{%
    span.HoLogo-KOMAScript span.HoLogo-O{%
      padding-left:.05em;%
      padding-right:.05em;%
    }%
  }%
  \Css{%
    span.HoLogo-KOMAScript span.HoLogo-A{%
      padding-left:.05em;%
    }%
  }%
  \Css{%
    span.HoLogo-KOMAScript span.HoLogo-hyphen{%
      padding-left:.1em;%
      padding-right:.1em;%
    }%
  }%
  \global\let\HoLogoCss@KOMAScript\relax
}
%    \end{macrocode}
%    \end{macro}
%
% \subsubsection{\hologo{LyX}}
%
%    \begin{macro}{\HoLogo@LyX}
%    The definition is taken from the documentation source files
%    of \hologo{LyX}, \xfile{Intro.lyx} \cite{LyX}:
%\begin{quote}
%\begin{verbatim}
%\def\LyX{%
%  \texorpdfstring{%
%    L\kern-.1667em\lower.25em\hbox{Y}\kern-.125emX\@%
%  }{%
%    LyX%
%  }%
%}
%\end{verbatim}
%\end{quote}
%    \begin{macrocode}
\def\HoLogo@LyX#1{%
  L%
  \kern-.1667em%
  \lower.25em\hbox{Y}%
  \kern-.125em%
  X%
  \HOLOGO@SpaceFactor
}
%    \end{macrocode}
%    \end{macro}
%    \begin{macro}{\HoLogoHtml@LyX}
%    \begin{macrocode}
\def\HoLogoHtml@LyX#1{%
  \HoLogoCss@LyX
  \HOLOGO@Span{LyX}{%
    L%
    \HOLOGO@Span{y}{Y}%
    X%
  }%
}
%    \end{macrocode}
%    \end{macro}
%    \begin{macro}{\HoLogoCss@LyX}
%    \begin{macrocode}
\def\HoLogoCss@LyX{%
  \Css{%
    span.HoLogo-LyX span.HoLogo-y{%
      position:relative;%
      top:.25em;%
      margin-left:-.1667em;%
      margin-right:-.125em;%
      text-decoration:none;%
    }%
  }%
  \global\let\HoLogoCss@LyX\relax
}
%    \end{macrocode}
%    \end{macro}
%
% \subsubsection{\hologo{NTS}}
%
%    \begin{macro}{\HoLogo@NTS}
%    Definition for \hologo{NTS} can be found in
%    package \xpackage{etex\textunderscore man} for the \hologo{eTeX} manual \cite{etexman}
%    and in package \xpackage{dtklogos} \cite{dtklogos}:
%\begin{quote}
%\begin{verbatim}
%\def\NTS{%
%  \leavevmode
%  \hbox{%
%    $%
%      \cal N%
%      \kern-0.35em%
%      \lower0.5ex\hbox{$\cal T$}%
%      \kern-0.2em%
%      S%
%    $%
%  }%
%}
%\end{verbatim}
%\end{quote}
%    \begin{macrocode}
\def\HoLogo@NTS#1{%
  \HoLogoFont@font{NTS}{sy}{%
    N\/%
    \kern-.35em%
    \lower.5ex\hbox{T\/}%
    \kern-.2em%
    S\/%
  }%
  \HOLOGO@SpaceFactor
}
%    \end{macrocode}
%    \end{macro}
%
% \subsubsection{\Hologo{TTH} (\hologo{TeX} to HTML translator)}
%
%    Source: \url{http://hutchinson.belmont.ma.us/tth/}
%    In the HTML source the second `T' is printed as subscript.
%\begin{quote}
%\begin{verbatim}
%T<sub>T</sub>H
%\end{verbatim}
%\end{quote}
%    \begin{macro}{\HoLogo@TTH}
%    \begin{macrocode}
\def\HoLogo@TTH#1{%
  \ltx@mbox{%
    T\HOLOGO@SubScript{T}H%
  }%
  \HOLOGO@SpaceFactor
}
%    \end{macrocode}
%    \end{macro}
%
%    \begin{macro}{\HoLogoHtml@TTH}
%    \begin{macrocode}
\def\HoLogoHtml@TTH#1{%
  T\HCode{<sub>}T\HCode{</sub>}H%
}
%    \end{macrocode}
%    \end{macro}
%
% \subsubsection{\Hologo{HanTheThanh}}
%
%    Partial source: Package \xpackage{dtklogos}.
%    The double accent is U+1EBF (latin small letter e with circumflex
%    and acute).
%    \begin{macro}{\HoLogo@HanTheThanh}
%    \begin{macrocode}
\def\HoLogo@HanTheThanh#1{%
  \ltx@mbox{H\`an}%
  \HOLOGO@space
  \ltx@mbox{%
    Th%
    \HOLOGO@IfCharExists{"1EBF}{%
      \char"1EBF\relax
    }{%
      \^e\hbox to 0pt{\hss\raise .5ex\hbox{\'{}}}%
    }%
  }%
  \HOLOGO@space
  \ltx@mbox{Th\`anh}%
}
%    \end{macrocode}
%    \end{macro}
%    \begin{macro}{\HoLogoBkm@HanTheThanh}
%    \begin{macrocode}
\def\HoLogoBkm@HanTheThanh#1{%
  H\`an %
  Th\HOLOGO@PdfdocUnicode{\^e}{\9036\277} %
  Th\`anh%
}
%    \end{macrocode}
%    \end{macro}
%    \begin{macro}{\HoLogoHtml@HanTheThanh}
%    \begin{macrocode}
\def\HoLogoHtml@HanTheThanh#1{%
  H\`an %
  Th\HCode{&\ltx@hashchar x1ebf;} %
  Th\`anh%
}
%    \end{macrocode}
%    \end{macro}
%
% \subsection{Driver detection}
%
%    \begin{macrocode}
\HOLOGO@IfExists\InputIfFileExists{%
  \InputIfFileExists{hologo.cfg}{}{}%
}{%
  \ltx@IfUndefined{pdf@filesize}{%
    \def\HOLOGO@InputIfExists{%
      \openin\HOLOGO@temp=hologo.cfg\relax
      \ifeof\HOLOGO@temp
        \closein\HOLOGO@temp
      \else
        \closein\HOLOGO@temp
        \begingroup
          \def\x{LaTeX2e}%
        \expandafter\endgroup
        \ifx\fmtname\x
          % \iffalse meta-comment
%
% File: hologo.dtx
% Version: 2016/05/12 v1.11
% Info: A logo collection with bookmark support
%
% Copyright (C) 2010-2012 by
%    Heiko Oberdiek <heiko.oberdiek at googlemail.com>
%
% This work may be distributed and/or modified under the
% conditions of the LaTeX Project Public License, either
% version 1.3c of this license or (at your option) any later
% version. This version of this license is in
%    http://www.latex-project.org/lppl/lppl-1-3c.txt
% and the latest version of this license is in
%    http://www.latex-project.org/lppl.txt
% and version 1.3 or later is part of all distributions of
% LaTeX version 2005/12/01 or later.
%
% This work has the LPPL maintenance status "maintained".
%
% This Current Maintainer of this work is Heiko Oberdiek.
%
% The Base Interpreter refers to any `TeX-Format',
% because some files are installed in TDS:tex/generic//.
%
% This work consists of the main source file hologo.dtx
% and the derived files
%    hologo.sty, hologo.pdf, hologo.ins, hologo.drv, hologo-example.tex,
%    hologo-test1.tex, hologo-test-spacefactor.tex,
%    hologo-test-list.tex.
%
% Distribution:
%    CTAN:macros/latex/contrib/oberdiek/hologo.dtx
%    CTAN:macros/latex/contrib/oberdiek/hologo.pdf
%
% Unpacking:
%    (a) If hologo.ins is present:
%           tex hologo.ins
%    (b) Without hologo.ins:
%           tex hologo.dtx
%    (c) If you insist on using LaTeX
%           latex \let\install=y\input{hologo.dtx}
%        (quote the arguments according to the demands of your shell)
%
% Documentation:
%    (a) If hologo.drv is present:
%           latex hologo.drv
%    (b) Without hologo.drv:
%           latex hologo.dtx; ...
%    The class ltxdoc loads the configuration file ltxdoc.cfg
%    if available. Here you can specify further options, e.g.
%    use A4 as paper format:
%       \PassOptionsToClass{a4paper}{article}
%
%    Programm calls to get the documentation (example):
%       pdflatex hologo.dtx
%       makeindex -s gind.ist hologo.idx
%       pdflatex hologo.dtx
%       makeindex -s gind.ist hologo.idx
%       pdflatex hologo.dtx
%
% Installation:
%    TDS:tex/generic/oberdiek/hologo.sty
%    TDS:doc/latex/oberdiek/hologo.pdf
%    TDS:doc/latex/oberdiek/example/hologo-example.tex
%    TDS:doc/latex/oberdiek/test/hologo-test1.tex
%    TDS:doc/latex/oberdiek/test/hologo-test-spacefactor.tex
%    TDS:doc/latex/oberdiek/test/hologo-test-list.tex
%    TDS:source/latex/oberdiek/hologo.dtx
%
%<*ignore>
\begingroup
  \catcode123=1 %
  \catcode125=2 %
  \def\x{LaTeX2e}%
\expandafter\endgroup
\ifcase 0\ifx\install y1\fi\expandafter
         \ifx\csname processbatchFile\endcsname\relax\else1\fi
         \ifx\fmtname\x\else 1\fi\relax
\else\csname fi\endcsname
%</ignore>
%<*install>
\input docstrip.tex
\Msg{************************************************************************}
\Msg{* Installation}
\Msg{* Package: hologo 2016/05/12 v1.11 A logo collection with bookmark support (HO)}
\Msg{************************************************************************}

\keepsilent
\askforoverwritefalse

\let\MetaPrefix\relax
\preamble

This is a generated file.

Project: hologo
Version: 2016/05/12 v1.11

Copyright (C) 2010-2012 by
   Heiko Oberdiek <heiko.oberdiek at googlemail.com>

This work may be distributed and/or modified under the
conditions of the LaTeX Project Public License, either
version 1.3c of this license or (at your option) any later
version. This version of this license is in
   http://www.latex-project.org/lppl/lppl-1-3c.txt
and the latest version of this license is in
   http://www.latex-project.org/lppl.txt
and version 1.3 or later is part of all distributions of
LaTeX version 2005/12/01 or later.

This work has the LPPL maintenance status "maintained".

This Current Maintainer of this work is Heiko Oberdiek.

The Base Interpreter refers to any `TeX-Format',
because some files are installed in TDS:tex/generic//.

This work consists of the main source file hologo.dtx
and the derived files
   hologo.sty, hologo.pdf, hologo.ins, hologo.drv, hologo-example.tex,
   hologo-test1.tex, hologo-test-spacefactor.tex,
   hologo-test-list.tex.

\endpreamble
\let\MetaPrefix\DoubleperCent

\generate{%
  \file{hologo.ins}{\from{hologo.dtx}{install}}%
  \file{hologo.drv}{\from{hologo.dtx}{driver}}%
  \usedir{tex/generic/oberdiek}%
  \file{hologo.sty}{\from{hologo.dtx}{package}}%
  \usedir{doc/latex/oberdiek/example}%
  \file{hologo-example.tex}{\from{hologo.dtx}{example}}%
  \usedir{doc/latex/oberdiek/test}%
  \file{hologo-test1.tex}{\from{hologo.dtx}{test1}}%
  \file{hologo-test-spacefactor.tex}{\from{hologo.dtx}{test-spacefactor}}%
  \file{hologo-test-list.tex}{\from{hologo.dtx}{test-list}}%
  \nopreamble
  \nopostamble
  \usedir{source/latex/oberdiek/catalogue}%
  \file{hologo.xml}{\from{hologo.dtx}{catalogue}}%
}

\catcode32=13\relax% active space
\let =\space%
\Msg{************************************************************************}
\Msg{*}
\Msg{* To finish the installation you have to move the following}
\Msg{* file into a directory searched by TeX:}
\Msg{*}
\Msg{*     hologo.sty}
\Msg{*}
\Msg{* To produce the documentation run the file `hologo.drv'}
\Msg{* through LaTeX.}
\Msg{*}
\Msg{* Happy TeXing!}
\Msg{*}
\Msg{************************************************************************}

\endbatchfile
%</install>
%<*ignore>
\fi
%</ignore>
%<*driver>
\NeedsTeXFormat{LaTeX2e}
\ProvidesFile{hologo.drv}%
  [2016/05/12 v1.11 A logo collection with bookmark support (HO)]%
\documentclass{ltxdoc}
\usepackage{holtxdoc}[2011/11/22]
\usepackage{hologo}[2016/05/12]
\usepackage{longtable}
\usepackage{array}
\usepackage{paralist}
%\usepackage[T1]{fontenc}
%\usepackage{lmodern}
\begin{document}
  \DocInput{hologo.dtx}%
\end{document}
%</driver>
% \fi
%
%
% \CharacterTable
%  {Upper-case    \A\B\C\D\E\F\G\H\I\J\K\L\M\N\O\P\Q\R\S\T\U\V\W\X\Y\Z
%   Lower-case    \a\b\c\d\e\f\g\h\i\j\k\l\m\n\o\p\q\r\s\t\u\v\w\x\y\z
%   Digits        \0\1\2\3\4\5\6\7\8\9
%   Exclamation   \!     Double quote  \"     Hash (number) \#
%   Dollar        \$     Percent       \%     Ampersand     \&
%   Acute accent  \'     Left paren    \(     Right paren   \)
%   Asterisk      \*     Plus          \+     Comma         \,
%   Minus         \-     Point         \.     Solidus       \/
%   Colon         \:     Semicolon     \;     Less than     \<
%   Equals        \=     Greater than  \>     Question mark \?
%   Commercial at \@     Left bracket  \[     Backslash     \\
%   Right bracket \]     Circumflex    \^     Underscore    \_
%   Grave accent  \`     Left brace    \{     Vertical bar  \|
%   Right brace   \}     Tilde         \~}
%
% \GetFileInfo{hologo.drv}
%
% \title{The \xpackage{hologo} package}
% \date{2016/05/12 v1.11}
% \author{Heiko Oberdiek\\\xemail{heiko.oberdiek at googlemail.com}}
%
% \maketitle
%
% \begin{abstract}
% This package starts a collection of logos with support for bookmarks
% strings.
% \end{abstract}
%
% \tableofcontents
%
% \section{Documentation}
%
% \subsection{Logo macros}
%
% \begin{declcs}{hologo} \M{name}
% \end{declcs}
% Macro \cs{hologo} sets the logo with name \meta{name}.
% The following table shows the supported names.
%
% \begingroup
%   \def\hologoEntry#1#2#3{^^A
%     #1&#2&\hologoLogoSetup{#1}{variant=#2}\hologo{#1}&#3\tabularnewline
%   }
%   \begin{longtable}{>{\ttfamily}l>{\ttfamily}lll}
%     \rmfamily\bfseries{name} & \rmfamily\bfseries variant
%     & \bfseries logo & \bfseries since\\
%     \hline
%     \endhead
%     \hologoList
%   \end{longtable}
% \endgroup
%
% \begin{declcs}{Hologo} \M{name}
% \end{declcs}
% Macro \cs{Hologo} starts the logo \meta{name} with an uppercase
% letter. As an exception small greek letters are not converted
% to uppercase. Examples, see \hologo{eTeX} and \hologo{ExTeX}.
%
% \subsection{Setup macros}
%
% The package does not support package options, but the following
% setup macros can be used to set options.
%
% \begin{declcs}{hologoSetup} \M{key value list}
% \end{declcs}
% Macro \cs{hologoSetup} sets global options.
%
% \begin{declcs}{hologoLogoSetup} \M{logo} \M{key value list}
% \end{declcs}
% Some options can also be used to configure a logo.
% These settings take precedence over global option settings.
%
% \subsection{Options}\label{sec:options}
%
% There are boolean and string options:
% \begin{description}
% \item[Boolean option:]
% It takes |true| or |false|
% as value. If the value is omitted, then |true| is used.
% \item[String option:]
% A value must be given as string. (But the string might be empty.)
% \end{description}
% The following options can be used both in \cs{hologoSetup}
% and \cs{hologoLogoSetup}:
% \begin{description}
% \def\entry#1{\item[\xoption{#1}:]}
% \entry{break}
%   enables or disables line breaks inside the logo. This setting is
%   refined by options \xoption{hyphenbreak}, \xoption{spacebreak}
%   or \xoption{discretionarybreak}.
%   Default is |false|.
% \entry{hyphenbreak}
%   enables or disables the line break right after the hyphen character.
% \entry{spacebreak}
%   enables or disables line breaks at space characters.
% \entry{discretionarybreak}
%   enables or disables line breaks at hyphenation points
%   (inserted by \cs{-}).
% \end{description}
% Macro \cs{hologoLogoSetup} also knows:
% \begin{description}
% \item[\xoption{variant}:]
%   This is a string option. It specifies a variant of a logo that
%   must exist. An empty string selects the package default variant.
% \end{description}
% Example:
% \begin{quote}
%   |\hologoSetup{break=false}|\\
%   |\hologoLogoSetup{plainTeX}{variant=hyphen,hyphenbreak}|\\
%   Then ``plain-\TeX'' contains one break point after the hyphen.
% \end{quote}
%
% \subsection{Driver options}
%
% Sometimes graphical operations are needed to construct some
% glyphs (e.g.\ \hologo{XeTeX}). If package \xpackage{graphics}
% or package \xpackage{pgf} are found, then the macros are taken
% from there. Otherwise the packge defines its own operations
% and therefore needs the driver information. Many drivers are
% detected automatically (\hologo{pdfTeX}/\hologo{LuaTeX}
% in PDF mode, \hologo{XeTeX}, \hologo{VTeX}). These have precedence
% over a driver option. The driver can be given as package option
% or using \cs{hologoDriverSetup}.
% The following list contains the recognized driver options:
% \begin{itemize}
% \item \xoption{pdftex}, \xoption{luatex}
% \item \xoption{dvipdfm}, \xoption{dvipdfmx}
% \item \xoption{dvips}, \xoption{dvipsone}, \xoption{xdvi}
% \item \xoption{xetex}
% \item \xoption{vtex}
% \end{itemize}
% The left driver of a line is the driver name that is used internally.
% The following names are aliases for drivers that use the
% same method. Therefore the entry in the \xext{log} file for
% the used driver prints the internally used driver name.
% \begin{description}
% \item[\xoption{driverfallback}:]
%   This option expects a driver that is used,
%   if the driver could not be detected automatically.
% \end{description}
%
% \begin{declcs}{hologoDriverSetup} \M{driver option}
% \end{declcs}
% The driver can also be configured after package loading
% using \cs{hologoDriverSetup}, also the way for \hologo{plainTeX}
% to setup the driver.
%
% \subsection{Font setup}
%
% Some logos require a special font, but should also be usable by
% \hologo{plainTeX}. Therefore the package provides some ways
% to influence the font settings. The options below
% take font settings as values. Both font commands
% such as \cs{sffamily} and macros that take one argument
% like \cs{textsf} can be used.
%
% \begin{declcs}{hologoFontSetup} \M{key value list}
% \end{declcs}
% Macro \cs{hologoFontSetup} sets the fonts for all logos.
% Supported keys:
% \begin{description}
% \def\entry#1{\item[\xoption{#1}:]}
% \entry{general}
%   This font is used for all logos. The default is empty.
%   That means no special font is used.
% \entry{bibsf}
%   This font is used for
%   {\hologoLogoSetup{BibTeX}{variant=sf}\hologo{BibTeX}}
%   with variant \xoption{sf}.
% \entry{rm}
%   This font is a serif font. It is used for \hologo{ExTeX}.
% \entry{sc}
%   This font specifies a small caps font. It is used for
%   {\hologoLogoSetup{BibTeX}{variant=sc}\hologo{BibTeX}}
%   with variant \xoption{sc}.
% \entry{sf}
%   This font specifies a sans serif font. The default
%   is \cs{sffamily}, then \cs{sf} is tried. Otherwise
%   a warning is given. It is used by \hologo{KOMAScript}.
% \entry{sy}
%   This is the font for math symbols (e.g. cmsy).
%   It is used by \hologo{AmS}, \hologo{NTS}, \hologo{ExTeX}.
% \entry{logo}
%   \hologo{METAFONT} and \hologo{METAPOST} are using that font.
%   In \hologo{LaTeX} \cs{logofamily} is used and
%   the definitions of package \xpackage{mflogo} are used
%   if the package is not loaded.
%   Otherwise the \cs{tenlogo} is used and defined
%   if it does not already exists.
% \end{description}
%
% \begin{declcs}{hologoLogoFontSetup} \M{logo} \M{key value list}
% \end{declcs}
% Fonts can also be set for a logo or logo component separately,
% see the following list.
% The keys are the same as for \cs{hologoFontSetup}.
%
% \begin{longtable}{>{\ttfamily}l>{\sffamily}ll}
%   \meta{logo} & keys & result\\
%   \hline
%   \endhead
%   BibTeX & bibsf & {\hologoLogoSetup{BibTeX}{variant=sf}\hologo{BibTeX}}\\[.5ex]
%   BibTeX & sc & {\hologoLogoSetup{BibTeX}{variant=sc}\hologo{BibTeX}}\\[.5ex]
%   ExTeX & rm & \hologo{ExTeX}\\
%   SliTeX & rm & \hologo{SliTeX}\\[.5ex]
%   AmS & sy & \hologo{AmS}\\
%   ExTeX & sy & \hologo{ExTeX}\\
%   NTS & sy & \hologo{NTS}\\[.5ex]
%   KOMAScript & sf & \hologo{KOMAScript}\\[.5ex]
%   METAFONT & logo & \hologo{METAFONT}\\
%   METAPOST & logo & \hologo{METAPOST}\\[.5ex]
%   SliTeX & sc \hologo{SliTeX}
% \end{longtable}
%
% \subsubsection{Font order}
%
% For all logos the font \xoption{general} is applied first.
% Example:
%\begin{quote}
%|\hologoFontSetup{general=\color{red}}|
%\end{quote}
% will print red logos.
% Then if the font uses a special font \xoption{sf}, for example,
% the font is applied that is setup by \cs{hologoLogoFontSetup}.
% If this font is not setup, then the common font setup
% by \cs{hologoFontSetup} is used. Otherwise a warning is given,
% that there is no font configured.
%
% \subsection{Additional user macros}
%
% Usually a variant of a logo is configured by using
% \cs{hologoLogoSetup}, because it is bad style to mix
% different variants of the same logo in the same text.
% There the following macros are a convenience for testing.
%
% \begin{declcs}{hologoVariant} \M{name} \M{variant}\\
%   \cs{HologoVariant} \M{name} \M{variant}
% \end{declcs}
% Logo \meta{name} is set using \meta{variant} that specifies
% explicitely which variant of the macro is used. If the argument
% is empty, then the default form of the logo is used
% (configurable by \cs{hologoLogoSetup}).
%
% \cs{HologoVariant} is used if the logo is set in a context
% that needs an uppercase first letter (beginning of a sentence, \dots).
%
% \begin{declcs}{hologoList}\\
%   \cs{hologoEntry} \M{logo} \M{variant} \M{since}
% \end{declcs}
% Macro \cs{hologoList} contains all logos that are provided
% by the package including variants. The list consists of calls
% of \cs{hologoEntry} with three arguments starting with the
% logo name \meta{logo} and its variant \meta{variant}. An empty
% variant means the current default. Argument \meta{since} specifies
% with version of the package \xpackage{hologo} is needed to get
% the logo. If the logo is fixed, then the date gets updated.
% Therefore the date \meta{since} is not exactly the date of
% the first introduction, but rather the date of the latest fix.
%
% Before \cs{hologoList} can be used, macro \cs{hologoEntry} needs
% a definition. The example file in section \ref{sec:example}
% shows applications of \cs{hologoList}.
%
% \subsection{Supported contexts}
%
% Macros \cs{hologo} and friends support special contexts:
% \begin{itemize}
% \item \hologo{LaTeX}'s protection mechanism.
% \item Bookmarks of package \xpackage{hyperref}.
% \item Package \xpackage{tex4ht}.
% \item The macros can be used inside \cs{csname} constructs,
%   if \cs{ifincsname} is available (\hologo{pdfTeX}, \hologo{XeTeX},
%   \hologo{LuaTeX}).
% \end{itemize}
%
% \subsection{Example}
% \label{sec:example}
%
% The following example prints the logos in different fonts.
%    \begin{macrocode}
%<*example>
%<<verbatim
\NeedsTeXFormat{LaTeX2e}
\documentclass[a4paper]{article}
\usepackage[
  hmargin=20mm,
  vmargin=20mm,
]{geometry}
\pagestyle{empty}
\usepackage{hologo}[2016/05/12]
\usepackage{longtable}
\usepackage{array}
\setlength{\extrarowheight}{2pt}
\usepackage[T1]{fontenc}
\usepackage{lmodern}
\usepackage{pdflscape}
\usepackage[
  pdfencoding=auto,
]{hyperref}
\hypersetup{
  pdfauthor={Heiko Oberdiek},
  pdftitle={Example for package `hologo'},
  pdfsubject={Logos with fonts lmr, lmss, qtm, qpl, qhv},
}
\usepackage{bookmark}

% Print the logo list on the console

\begingroup
  \typeout{}%
  \typeout{*** Begin of logo list ***}%
  \newcommand*{\hologoEntry}[3]{%
    \typeout{#1 \ifx\\#2\\\else(#2) \fi[#3]}%
  }%
  \hologoList
  \typeout{*** End of logo list ***}%
  \typeout{}%
\endgroup

\begin{document}
\begin{landscape}

  \section{Example file for package `hologo'}

  % Table for font names

  \begin{longtable}{>{\bfseries}ll}
    \textbf{font} & \textbf{Font name}\\
    \hline
    lmr & Latin Modern Roman\\
    lmss & Latin Modern Sans\\
    qtm & \TeX\ Gyre Termes\\
    qhv & \TeX\ Gyre Heros\\
    qpl & \TeX\ Gyre Pagella\\
  \end{longtable}

  % Logo list with logos in different fonts

  \begingroup
    \newcommand*{\SetVariant}[2]{%
      \ifx\\#2\\%
      \else
        \hologoLogoSetup{#1}{variant=#2}%
      \fi
    }%
    \newcommand*{\hologoEntry}[3]{%
      \SetVariant{#1}{#2}%
      \raisebox{1em}[0pt][0pt]{\hypertarget{#1@#2}{}}%
      \bookmark[%
        dest={#1@#2},%
      ]{%
        #1\ifx\\#2\\\else\space(#2)\fi: \Hologo{#1}, \hologo{#1} %
        [Unicode]%
      }%
      \hypersetup{unicode=false}%
      \bookmark[%
        dest={#1@#2},%
      ]{%
        #1\ifx\\#2\\\else\space(#2)\fi: \Hologo{#1}, \hologo{#1} %
        [PDFDocEncoding]%
      }%
      \texttt{#1}%
      &%
      \texttt{#2}%
      &%
      \Hologo{#1}%
      &%
      \SetVariant{#1}{#2}%
      \hologo{#1}%
      &%
      \SetVariant{#1}{#2}%
      \fontfamily{qtm}\selectfont
      \hologo{#1}%
      &%
      \SetVariant{#1}{#2}%
      \fontfamily{qpl}\selectfont
      \hologo{#1}%
      &%
      \SetVariant{#1}{#2}%
      \textsf{\hologo{#1}}%
      &%
      \SetVariant{#1}{#2}%
      \fontfamily{qhv}\selectfont
      \hologo{#1}%
      \tabularnewline
    }%
    \begin{longtable}{llllllll}%
      \textbf{\textit{logo}} & \textbf{\textit{variant}} &
      \texttt{\string\Hologo} &
      \textbf{lmr} & \textbf{qtm} & \textbf{qpl} &
      \textbf{lmss} & \textbf{qhv}
      \tabularnewline
      \hline
      \endhead
      \hologoList
    \end{longtable}%
  \endgroup

\end{landscape}
\end{document}
%verbatim
%</example>
%    \end{macrocode}
%
% \StopEventually{
% }
%
% \section{Implementation}
%    \begin{macrocode}
%<*package>
%    \end{macrocode}
%    Reload check, especially if the package is not used with \LaTeX.
%    \begin{macrocode}
\begingroup\catcode61\catcode48\catcode32=10\relax%
  \catcode13=5 % ^^M
  \endlinechar=13 %
  \catcode35=6 % #
  \catcode39=12 % '
  \catcode44=12 % ,
  \catcode45=12 % -
  \catcode46=12 % .
  \catcode58=12 % :
  \catcode64=11 % @
  \catcode123=1 % {
  \catcode125=2 % }
  \expandafter\let\expandafter\x\csname ver@hologo.sty\endcsname
  \ifx\x\relax % plain-TeX, first loading
  \else
    \def\empty{}%
    \ifx\x\empty % LaTeX, first loading,
      % variable is initialized, but \ProvidesPackage not yet seen
    \else
      \expandafter\ifx\csname PackageInfo\endcsname\relax
        \def\x#1#2{%
          \immediate\write-1{Package #1 Info: #2.}%
        }%
      \else
        \def\x#1#2{\PackageInfo{#1}{#2, stopped}}%
      \fi
      \x{hologo}{The package is already loaded}%
      \aftergroup\endinput
    \fi
  \fi
\endgroup%
%    \end{macrocode}
%    Package identification:
%    \begin{macrocode}
\begingroup\catcode61\catcode48\catcode32=10\relax%
  \catcode13=5 % ^^M
  \endlinechar=13 %
  \catcode35=6 % #
  \catcode39=12 % '
  \catcode40=12 % (
  \catcode41=12 % )
  \catcode44=12 % ,
  \catcode45=12 % -
  \catcode46=12 % .
  \catcode47=12 % /
  \catcode58=12 % :
  \catcode64=11 % @
  \catcode91=12 % [
  \catcode93=12 % ]
  \catcode123=1 % {
  \catcode125=2 % }
  \expandafter\ifx\csname ProvidesPackage\endcsname\relax
    \def\x#1#2#3[#4]{\endgroup
      \immediate\write-1{Package: #3 #4}%
      \xdef#1{#4}%
    }%
  \else
    \def\x#1#2[#3]{\endgroup
      #2[{#3}]%
      \ifx#1\@undefined
        \xdef#1{#3}%
      \fi
      \ifx#1\relax
        \xdef#1{#3}%
      \fi
    }%
  \fi
\expandafter\x\csname ver@hologo.sty\endcsname
\ProvidesPackage{hologo}%
  [2016/05/12 v1.11 A logo collection with bookmark support (HO)]%
%    \end{macrocode}
%
%    \begin{macrocode}
\begingroup\catcode61\catcode48\catcode32=10\relax%
  \catcode13=5 % ^^M
  \endlinechar=13 %
  \catcode123=1 % {
  \catcode125=2 % }
  \catcode64=11 % @
  \def\x{\endgroup
    \expandafter\edef\csname HOLOGO@AtEnd\endcsname{%
      \endlinechar=\the\endlinechar\relax
      \catcode13=\the\catcode13\relax
      \catcode32=\the\catcode32\relax
      \catcode35=\the\catcode35\relax
      \catcode61=\the\catcode61\relax
      \catcode64=\the\catcode64\relax
      \catcode123=\the\catcode123\relax
      \catcode125=\the\catcode125\relax
    }%
  }%
\x\catcode61\catcode48\catcode32=10\relax%
\catcode13=5 % ^^M
\endlinechar=13 %
\catcode35=6 % #
\catcode64=11 % @
\catcode123=1 % {
\catcode125=2 % }
\def\TMP@EnsureCode#1#2{%
  \edef\HOLOGO@AtEnd{%
    \HOLOGO@AtEnd
    \catcode#1=\the\catcode#1\relax
  }%
  \catcode#1=#2\relax
}
\TMP@EnsureCode{10}{12}% ^^J
\TMP@EnsureCode{33}{12}% !
\TMP@EnsureCode{34}{12}% "
\TMP@EnsureCode{36}{3}% $
\TMP@EnsureCode{38}{4}% &
\TMP@EnsureCode{39}{12}% '
\TMP@EnsureCode{40}{12}% (
\TMP@EnsureCode{41}{12}% )
\TMP@EnsureCode{42}{12}% *
\TMP@EnsureCode{43}{12}% +
\TMP@EnsureCode{44}{12}% ,
\TMP@EnsureCode{45}{12}% -
\TMP@EnsureCode{46}{12}% .
\TMP@EnsureCode{47}{12}% /
\TMP@EnsureCode{58}{12}% :
\TMP@EnsureCode{59}{12}% ;
\TMP@EnsureCode{60}{12}% <
\TMP@EnsureCode{62}{12}% >
\TMP@EnsureCode{63}{12}% ?
\TMP@EnsureCode{91}{12}% [
\TMP@EnsureCode{93}{12}% ]
\TMP@EnsureCode{94}{7}% ^ (superscript)
\TMP@EnsureCode{95}{8}% _ (subscript)
\TMP@EnsureCode{96}{12}% `
\TMP@EnsureCode{124}{12}% |
\edef\HOLOGO@AtEnd{%
  \HOLOGO@AtEnd
  \escapechar\the\escapechar\relax
  \noexpand\endinput
}
\escapechar=92 %
%    \end{macrocode}
%
% \subsection{Logo list}
%
%    \begin{macro}{\hologoList}
%    \begin{macrocode}
\def\hologoList{%
  \hologoEntry{(La)TeX}{}{2011/10/01}%
  \hologoEntry{AmSLaTeX}{}{2010/04/16}%
  \hologoEntry{AmSTeX}{}{2010/04/16}%
  \hologoEntry{biber}{}{2011/10/01}%
  \hologoEntry{BibTeX}{}{2011/10/01}%
  \hologoEntry{BibTeX}{sf}{2011/10/01}%
  \hologoEntry{BibTeX}{sc}{2011/10/01}%
  \hologoEntry{BibTeX8}{}{2011/11/22}%
  \hologoEntry{ConTeXt}{}{2011/03/25}%
  \hologoEntry{ConTeXt}{narrow}{2011/03/25}%
  \hologoEntry{ConTeXt}{simple}{2011/03/25}%
  \hologoEntry{emTeX}{}{2010/04/26}%
  \hologoEntry{eTeX}{}{2010/04/08}%
  \hologoEntry{ExTeX}{}{2011/10/01}%
  \hologoEntry{HanTheThanh}{}{2011/11/29}%
  \hologoEntry{iniTeX}{}{2011/10/01}%
  \hologoEntry{KOMAScript}{}{2011/10/01}%
  \hologoEntry{La}{}{2010/05/08}%
  \hologoEntry{LaTeX}{}{2010/04/08}%
  \hologoEntry{LaTeX2e}{}{2010/04/08}%
  \hologoEntry{LaTeX3}{}{2010/04/24}%
  \hologoEntry{LaTeXe}{}{2010/04/08}%
  \hologoEntry{LaTeXML}{}{2011/11/22}%
  \hologoEntry{LaTeXTeX}{}{2011/10/01}%
  \hologoEntry{LuaLaTeX}{}{2010/04/08}%
  \hologoEntry{LuaTeX}{}{2010/04/08}%
  \hologoEntry{LyX}{}{2011/10/01}%
  \hologoEntry{METAFONT}{}{2011/10/01}%
  \hologoEntry{MetaFun}{}{2011/10/01}%
  \hologoEntry{METAPOST}{}{2011/10/01}%
  \hologoEntry{MetaPost}{}{2011/10/01}%
  \hologoEntry{MiKTeX}{}{2011/10/01}%
  \hologoEntry{NTS}{}{2011/10/01}%
  \hologoEntry{OzMF}{}{2011/10/01}%
  \hologoEntry{OzMP}{}{2011/10/01}%
  \hologoEntry{OzTeX}{}{2011/10/01}%
  \hologoEntry{OzTtH}{}{2011/10/01}%
  \hologoEntry{PCTeX}{}{2011/10/01}%
  \hologoEntry{pdfTeX}{}{2011/10/01}%
  \hologoEntry{pdfLaTeX}{}{2011/10/01}%
  \hologoEntry{PiC}{}{2011/10/01}%
  \hologoEntry{PiCTeX}{}{2011/10/01}%
  \hologoEntry{plainTeX}{}{2010/04/08}%
  \hologoEntry{plainTeX}{space}{2010/04/16}%
  \hologoEntry{plainTeX}{hyphen}{2010/04/16}%
  \hologoEntry{plainTeX}{runtogether}{2010/04/16}%
  \hologoEntry{SageTeX}{}{2011/11/22}%
  \hologoEntry{SLiTeX}{}{2011/10/01}%
  \hologoEntry{SLiTeX}{lift}{2011/10/01}%
  \hologoEntry{SLiTeX}{narrow}{2011/10/01}%
  \hologoEntry{SLiTeX}{simple}{2011/10/01}%
  \hologoEntry{SliTeX}{}{2011/10/01}%
  \hologoEntry{SliTeX}{narrow}{2011/10/01}%
  \hologoEntry{SliTeX}{simple}{2011/10/01}%
  \hologoEntry{SliTeX}{lift}{2011/10/01}%
  \hologoEntry{teTeX}{}{2011/10/01}%
  \hologoEntry{TeX}{}{2010/04/08}%
  \hologoEntry{TeX4ht}{}{2011/11/22}%
  \hologoEntry{TTH}{}{2011/11/22}%
  \hologoEntry{virTeX}{}{2011/10/01}%
  \hologoEntry{VTeX}{}{2010/04/24}%
  \hologoEntry{Xe}{}{2010/04/08}%
  \hologoEntry{XeLaTeX}{}{2010/04/08}%
  \hologoEntry{XeTeX}{}{2010/04/08}%
}
%    \end{macrocode}
%    \end{macro}
%
% \subsection{Load resources}
%
%    \begin{macrocode}
\begingroup\expandafter\expandafter\expandafter\endgroup
\expandafter\ifx\csname RequirePackage\endcsname\relax
  \def\TMP@RequirePackage#1[#2]{%
    \begingroup\expandafter\expandafter\expandafter\endgroup
    \expandafter\ifx\csname ver@#1.sty\endcsname\relax
      \input #1.sty\relax
    \fi
  }%
  \TMP@RequirePackage{ltxcmds}[2011/02/04]%
  \TMP@RequirePackage{infwarerr}[2010/04/08]%
  \TMP@RequirePackage{kvsetkeys}[2010/03/01]%
  \TMP@RequirePackage{kvdefinekeys}[2010/03/01]%
  \TMP@RequirePackage{pdftexcmds}[2010/04/01]%
  \TMP@RequirePackage{ifpdf}[2010/01/28]%
  \TMP@RequirePackage{ifluatex}[2010/03/01]%
  \ltx@IfUndefined{newif}{%
    \expandafter\let\csname newif\endcsname\ltx@newif
  }{}%
  \TMP@RequirePackage{ifxetex}[2009/01/23]%
  \TMP@RequirePackage{ifvtex}[2010/03/01]%
\else
  \RequirePackage{ltxcmds}[2011/02/04]%
  \RequirePackage{infwarerr}[2010/04/08]%
  \RequirePackage{kvsetkeys}[2010/03/01]%
  \RequirePackage{kvdefinekeys}[2010/03/01]%
  \RequirePackage{pdftexcmds}[2010/04/01]%
  \RequirePackage{ifpdf}[2010/01/28]%
  \RequirePackage{ifluatex}[2010/03/01]%
  \RequirePackage{ifxetex}[2009/01/23]%
  \RequirePackage{ifvtex}[2010/03/01]%
\fi
%    \end{macrocode}
%
%    \begin{macro}{\HOLOGO@IfDefined}
%    \begin{macrocode}
\def\HOLOGO@IfExists#1{%
  \ifx\@undefined#1%
    \expandafter\ltx@secondoftwo
  \else
    \ifx\relax#1%
      \expandafter\ltx@secondoftwo
    \else
      \expandafter\expandafter\expandafter\ltx@firstoftwo
    \fi
  \fi
}
%    \end{macrocode}
%    \end{macro}
%
% \subsection{Setup macros}
%
%    \begin{macro}{\hologoSetup}
%    \begin{macrocode}
\def\hologoSetup{%
  \let\HOLOGO@name\relax
  \HOLOGO@Setup
}
%    \end{macrocode}
%    \end{macro}
%
%    \begin{macro}{\hologoLogoSetup}
%    \begin{macrocode}
\def\hologoLogoSetup#1{%
  \edef\HOLOGO@name{#1}%
  \ltx@IfUndefined{HoLogo@\HOLOGO@name}{%
    \@PackageError{hologo}{%
      Unknown logo `\HOLOGO@name'%
    }\@ehc
    \ltx@gobble
  }{%
    \HOLOGO@Setup
  }%
}
%    \end{macrocode}
%    \end{macro}
%
%    \begin{macro}{\HOLOGO@Setup}
%    \begin{macrocode}
\def\HOLOGO@Setup{%
  \kvsetkeys{HoLogo}%
}
%    \end{macrocode}
%    \end{macro}
%
% \subsection{Options}
%
%    \begin{macro}{\HOLOGO@DeclareBoolOption}
%    \begin{macrocode}
\def\HOLOGO@DeclareBoolOption#1{%
  \expandafter\chardef\csname HOLOGOOPT@#1\endcsname\ltx@zero
  \kv@define@key{HoLogo}{#1}[true]{%
    \def\HOLOGO@temp{##1}%
    \ifx\HOLOGO@temp\HOLOGO@true
      \ifx\HOLOGO@name\relax
        \expandafter\chardef\csname HOLOGOOPT@#1\endcsname=\ltx@one
      \else
        \expandafter\chardef\csname
        HoLogoOpt@#1@\HOLOGO@name\endcsname\ltx@one
      \fi
      \HOLOGO@SetBreakAll{#1}%
    \else
      \ifx\HOLOGO@temp\HOLOGO@false
        \ifx\HOLOGO@name\relax
          \expandafter\chardef\csname HOLOGOOPT@#1\endcsname=\ltx@zero
        \else
          \expandafter\chardef\csname
          HoLogoOpt@#1@\HOLOGO@name\endcsname=\ltx@zero
        \fi
        \HOLOGO@SetBreakAll{#1}%
      \else
        \@PackageError{hologo}{%
          Unknown value `##1' for boolean option `#1'.\MessageBreak
          Known values are `true' and `false'%
        }\@ehc
      \fi
    \fi
  }%
}
%    \end{macrocode}
%    \end{macro}
%
%    \begin{macro}{\HOLOGO@SetBreakAll}
%    \begin{macrocode}
\def\HOLOGO@SetBreakAll#1{%
  \def\HOLOGO@temp{#1}%
  \ifx\HOLOGO@temp\HOLOGO@break
    \ifx\HOLOGO@name\relax
      \chardef\HOLOGOOPT@hyphenbreak=\HOLOGOOPT@break
      \chardef\HOLOGOOPT@spacebreak=\HOLOGOOPT@break
      \chardef\HOLOGOOPT@discretionarybreak=\HOLOGOOPT@break
    \else
      \expandafter\chardef
         \csname HoLogoOpt@hyphenbreak@\HOLOGO@name\endcsname=%
         \csname HoLogoOpt@break@\HOLOGO@name\endcsname
      \expandafter\chardef
         \csname HoLogoOpt@spacebreak@\HOLOGO@name\endcsname=%
         \csname HoLogoOpt@break@\HOLOGO@name\endcsname
      \expandafter\chardef
         \csname HoLogoOpt@discretionarybreak@\HOLOGO@name
             \endcsname=%
         \csname HoLogoOpt@break@\HOLOGO@name\endcsname
    \fi
  \fi
}
%    \end{macrocode}
%    \end{macro}
%
%    \begin{macro}{\HOLOGO@true}
%    \begin{macrocode}
\def\HOLOGO@true{true}
%    \end{macrocode}
%    \end{macro}
%    \begin{macro}{\HOLOGO@false}
%    \begin{macrocode}
\def\HOLOGO@false{false}
%    \end{macrocode}
%    \end{macro}
%    \begin{macro}{\HOLOGO@break}
%    \begin{macrocode}
\def\HOLOGO@break{break}
%    \end{macrocode}
%    \end{macro}
%
%    \begin{macrocode}
\HOLOGO@DeclareBoolOption{break}
\HOLOGO@DeclareBoolOption{hyphenbreak}
\HOLOGO@DeclareBoolOption{spacebreak}
\HOLOGO@DeclareBoolOption{discretionarybreak}
%    \end{macrocode}
%
%    \begin{macrocode}
\kv@define@key{HoLogo}{variant}{%
  \ifx\HOLOGO@name\relax
    \@PackageError{hologo}{%
      Option `variant' is not available in \string\hologoSetup,%
      \MessageBreak
      Use \string\hologoLogoSetup\space instead%
    }\@ehc
  \else
    \edef\HOLOGO@temp{#1}%
    \ifx\HOLOGO@temp\ltx@empty
      \expandafter
      \let\csname HoLogoOpt@variant@\HOLOGO@name\endcsname\@undefined
    \else
      \ltx@IfUndefined{HoLogo@\HOLOGO@name @\HOLOGO@temp}{%
        \@PackageError{hologo}{%
          Unknown variant `\HOLOGO@temp' of logo `\HOLOGO@name'%
        }\@ehc
      }{%
        \expandafter
        \let\csname HoLogoOpt@variant@\HOLOGO@name\endcsname
            \HOLOGO@temp
      }%
    \fi
  \fi
}
%    \end{macrocode}
%
%    \begin{macro}{\HOLOGO@Variant}
%    \begin{macrocode}
\def\HOLOGO@Variant#1{%
  #1%
  \ltx@ifundefined{HoLogoOpt@variant@#1}{%
  }{%
    @\csname HoLogoOpt@variant@#1\endcsname
  }%
}
%    \end{macrocode}
%    \end{macro}
%
% \subsection{Break/no-break support}
%
%    \begin{macro}{\HOLOGO@space}
%    \begin{macrocode}
\def\HOLOGO@space{%
  \ltx@ifundefined{HoLogoOpt@spacebreak@\HOLOGO@name}{%
    \ltx@ifundefined{HoLogoOpt@break@\HOLOGO@name}{%
      \chardef\HOLOGO@temp=\HOLOGOOPT@spacebreak
    }{%
      \chardef\HOLOGO@temp=%
        \csname HoLogoOpt@break@\HOLOGO@name\endcsname
    }%
  }{%
    \chardef\HOLOGO@temp=%
      \csname HoLogoOpt@spacebreak@\HOLOGO@name\endcsname
  }%
  \ifcase\HOLOGO@temp
    \penalty10000 %
  \fi
  \ltx@space
}
%    \end{macrocode}
%    \end{macro}
%
%    \begin{macro}{\HOLOGO@hyphen}
%    \begin{macrocode}
\def\HOLOGO@hyphen{%
  \ltx@ifundefined{HoLogoOpt@hyphenbreak@\HOLOGO@name}{%
    \ltx@ifundefined{HoLogoOpt@break@\HOLOGO@name}{%
      \chardef\HOLOGO@temp=\HOLOGOOPT@hyphenbreak
    }{%
      \chardef\HOLOGO@temp=%
        \csname HoLogoOpt@break@\HOLOGO@name\endcsname
    }%
  }{%
    \chardef\HOLOGO@temp=%
      \csname HoLogoOpt@hyphenbreak@\HOLOGO@name\endcsname
  }%
  \ifcase\HOLOGO@temp
    \ltx@mbox{-}%
  \else
    -%
  \fi
}
%    \end{macrocode}
%    \end{macro}
%
%    \begin{macro}{\HOLOGO@discretionary}
%    \begin{macrocode}
\def\HOLOGO@discretionary{%
  \ltx@ifundefined{HoLogoOpt@discretionarybreak@\HOLOGO@name}{%
    \ltx@ifundefined{HoLogoOpt@break@\HOLOGO@name}{%
      \chardef\HOLOGO@temp=\HOLOGOOPT@discretionarybreak
    }{%
      \chardef\HOLOGO@temp=%
        \csname HoLogoOpt@break@\HOLOGO@name\endcsname
    }%
  }{%
    \chardef\HOLOGO@temp=%
      \csname HoLogoOpt@discretionarybreak@\HOLOGO@name\endcsname
  }%
  \ifcase\HOLOGO@temp
  \else
    \-%
  \fi
}
%    \end{macrocode}
%    \end{macro}
%
%    \begin{macro}{\HOLOGO@mbox}
%    \begin{macrocode}
\def\HOLOGO@mbox#1{%
  \ltx@ifundefined{HoLogoOpt@break@\HOLOGO@name}{%
    \chardef\HOLOGO@temp=\HOLOGOOPT@hyphenbreak
  }{%
    \chardef\HOLOGO@temp=%
      \csname HoLogoOpt@break@\HOLOGO@name\endcsname
  }%
  \ifcase\HOLOGO@temp
    \ltx@mbox{#1}%
  \else
    #1%
  \fi
}
%    \end{macrocode}
%    \end{macro}
%
% \subsection{Font support}
%
%    \begin{macro}{\HoLogoFont@font}
%    \begin{tabular}{@{}ll@{}}
%    |#1|:& logo name\\
%    |#2|:& font short name\\
%    |#3|:& text
%    \end{tabular}
%    \begin{macrocode}
\def\HoLogoFont@font#1#2#3{%
  \begingroup
    \ltx@IfUndefined{HoLogoFont@logo@#1.#2}{%
      \ltx@IfUndefined{HoLogoFont@font@#2}{%
        \@PackageWarning{hologo}{%
          Missing font `#2' for logo `#1'%
        }%
        #3%
      }{%
        \csname HoLogoFont@font@#2\endcsname{#3}%
      }%
    }{%
      \csname HoLogoFont@logo@#1.#2\endcsname{#3}%
    }%
  \endgroup
}
%    \end{macrocode}
%    \end{macro}
%
%    \begin{macro}{\HoLogoFont@Def}
%    \begin{macrocode}
\def\HoLogoFont@Def#1{%
  \expandafter\def\csname HoLogoFont@font@#1\endcsname
}
%    \end{macrocode}
%    \end{macro}
%    \begin{macro}{\HoLogoFont@LogoDef}
%    \begin{macrocode}
\def\HoLogoFont@LogoDef#1#2{%
  \expandafter\def\csname HoLogoFont@logo@#1.#2\endcsname
}
%    \end{macrocode}
%    \end{macro}
%
% \subsubsection{Font defaults}
%
%    \begin{macro}{\HoLogoFont@font@general}
%    \begin{macrocode}
\HoLogoFont@Def{general}{}%
%    \end{macrocode}
%    \end{macro}
%
%    \begin{macro}{\HoLogoFont@font@rm}
%    \begin{macrocode}
\ltx@IfUndefined{rmfamily}{%
  \ltx@IfUndefined{rm}{%
  }{%
    \HoLogoFont@Def{rm}{\rm}%
  }%
}{%
  \HoLogoFont@Def{rm}{\rmfamily}%
}
%    \end{macrocode}
%    \end{macro}
%
%    \begin{macro}{\HoLogoFont@font@sf}
%    \begin{macrocode}
\ltx@IfUndefined{sffamily}{%
  \ltx@IfUndefined{sf}{%
  }{%
    \HoLogoFont@Def{sf}{\sf}%
  }%
}{%
  \HoLogoFont@Def{sf}{\sffamily}%
}
%    \end{macrocode}
%    \end{macro}
%
%    \begin{macro}{\HoLogoFont@font@bibsf}
%    In case of \hologo{plainTeX} the original small caps
%    variant is used as default. In \hologo{LaTeX}
%    the definition of package \xpackage{dtklogos} \cite{dtklogos}
%    is used.
%\begin{quote}
%\begin{verbatim}
%\DeclareRobustCommand{\BibTeX}{%
%  B%
%  \kern-.05em%
%  \hbox{%
%    $\m@th$% %% force math size calculations
%    \csname S@\f@size\endcsname
%    \fontsize\sf@size\z@
%    \math@fontsfalse
%    \selectfont
%    I%
%    \kern-.025em%
%    B
%  }%
%  \kern-.08em%
%  \-%
%  \TeX
%}
%\end{verbatim}
%\end{quote}
%    \begin{macrocode}
\ltx@IfUndefined{selectfont}{%
  \ltx@IfUndefined{tensc}{%
    \font\tensc=cmcsc10\relax
  }{}%
  \HoLogoFont@Def{bibsf}{\tensc}%
}{%
  \HoLogoFont@Def{bibsf}{%
    $\mathsurround=0pt$%
    \csname S@\f@size\endcsname
    \fontsize\sf@size{0pt}%
    \math@fontsfalse
    \selectfont
  }%
}
%    \end{macrocode}
%    \end{macro}
%
%    \begin{macro}{\HoLogoFont@font@sc}
%    \begin{macrocode}
\ltx@IfUndefined{scshape}{%
  \ltx@IfUndefined{tensc}{%
    \font\tensc=cmcsc10\relax
  }{}%
  \HoLogoFont@Def{sc}{\tensc}%
}{%
  \HoLogoFont@Def{sc}{\scshape}%
}
%    \end{macrocode}
%    \end{macro}
%
%    \begin{macro}{\HoLogoFont@font@sy}
%    \begin{macrocode}
\ltx@IfUndefined{usefont}{%
  \ltx@IfUndefined{tensy}{%
  }{%
    \HoLogoFont@Def{sy}{\tensy}%
  }%
}{%
  \HoLogoFont@Def{sy}{%
    \usefont{OMS}{cmsy}{m}{n}%
  }%
}
%    \end{macrocode}
%    \end{macro}
%
%    \begin{macro}{\HoLogoFont@font@logo}
%    \begin{macrocode}
\begingroup
  \def\x{LaTeX2e}%
\expandafter\endgroup
\ifx\fmtname\x
  \ltx@IfUndefined{logofamily}{%
    \DeclareRobustCommand\logofamily{%
      \not@math@alphabet\logofamily\relax
      \fontencoding{U}%
      \fontfamily{logo}%
      \selectfont
    }%
  }{}%
  \ltx@IfUndefined{logofamily}{%
  }{%
    \HoLogoFont@Def{logo}{\logofamily}%
  }%
\else
  \ltx@IfUndefined{tenlogo}{%
    \font\tenlogo=logo10\relax
  }{}%
  \HoLogoFont@Def{logo}{\tenlogo}%
\fi
%    \end{macrocode}
%    \end{macro}
%
% \subsubsection{Font setup}
%
%    \begin{macro}{\hologoFontSetup}
%    \begin{macrocode}
\def\hologoFontSetup{%
  \let\HOLOGO@name\relax
  \HOLOGO@FontSetup
}
%    \end{macrocode}
%    \end{macro}
%
%    \begin{macro}{\hologoLogoFontSetup}
%    \begin{macrocode}
\def\hologoLogoFontSetup#1{%
  \edef\HOLOGO@name{#1}%
  \ltx@IfUndefined{HoLogo@\HOLOGO@name}{%
    \@PackageError{hologo}{%
      Unknown logo `\HOLOGO@name'%
    }\@ehc
    \ltx@gobble
  }{%
    \HOLOGO@FontSetup
  }%
}
%    \end{macrocode}
%    \end{macro}
%
%    \begin{macro}{\HOLOGO@FontSetup}
%    \begin{macrocode}
\def\HOLOGO@FontSetup{%
  \kvsetkeys{HoLogoFont}%
}
%    \end{macrocode}
%    \end{macro}
%
%    \begin{macrocode}
\def\HOLOGO@temp#1{%
  \kv@define@key{HoLogoFont}{#1}{%
    \ifx\HOLOGO@name\relax
      \HoLogoFont@Def{#1}{##1}%
    \else
      \HoLogoFont@LogoDef\HOLOGO@name{#1}{##1}%
    \fi
  }%
}
\HOLOGO@temp{general}
\HOLOGO@temp{sf}
%    \end{macrocode}
%
% \subsection{Generic logo commands}
%
%    \begin{macrocode}
\HOLOGO@IfExists\hologo{%
  \@PackageError{hologo}{%
    \string\hologo\ltx@space is already defined.\MessageBreak
    Package loading is aborted%
  }\@ehc
  \HOLOGO@AtEnd
}%
\HOLOGO@IfExists\hologoRobust{%
  \@PackageError{hologo}{%
    \string\hologoRobust\ltx@space is already defined.\MessageBreak
    Package loading is aborted%
  }\@ehc
  \HOLOGO@AtEnd
}%
%    \end{macrocode}
%
% \subsubsection{\cs{hologo} and friends}
%
%    \begin{macrocode}
\ifluatex
  \expandafter\ltx@firstofone
\else
  \expandafter\ltx@gobble
\fi
{%
  \ltx@IfUndefined{ifincsname}{%
    \ifnum\luatexversion<36 %
      \expandafter\ltx@gobble
    \else
      \expandafter\ltx@firstofone
    \fi
    {%
      \begingroup
        \ifcase0%
            \directlua{%
              if tex.enableprimitives then %
                tex.enableprimitives('HOLOGO@', {'ifincsname'})%
              else %
                tex.print('1')%
              end%
            }%
            \ifx\HOLOGO@ifincsname\@undefined 1\fi%
            \relax
          \expandafter\ltx@firstofone
        \else
          \endgroup
          \expandafter\ltx@gobble
        \fi
        {%
          \global\let\ifincsname\HOLOGO@ifincsname
        }%
      \HOLOGO@temp
    }%
  }{}%
}
%    \end{macrocode}
%    \begin{macrocode}
\ltx@IfUndefined{ifincsname}{%
  \catcode`$=14 %
}{%
  \catcode`$=9 %
}
%    \end{macrocode}
%
%    \begin{macro}{\hologo}
%    \begin{macrocode}
\def\hologo#1{%
$ \ifincsname
$   \ltx@ifundefined{HoLogoCs@\HOLOGO@Variant{#1}}{%
$     #1%
$   }{%
$     \csname HoLogoCs@\HOLOGO@Variant{#1}\endcsname\ltx@firstoftwo
$   }%
$ \else
    \HOLOGO@IfExists\texorpdfstring\texorpdfstring\ltx@firstoftwo
    {%
      \hologoRobust{#1}%
    }{%
      \ltx@ifundefined{HoLogoBkm@\HOLOGO@Variant{#1}}{%
        \ltx@ifundefined{HoLogo@#1}{?#1?}{#1}%
      }{%
        \csname HoLogoBkm@\HOLOGO@Variant{#1}\endcsname
        \ltx@firstoftwo
      }%
    }%
$ \fi
}
%    \end{macrocode}
%    \end{macro}
%    \begin{macro}{\Hologo}
%    \begin{macrocode}
\def\Hologo#1{%
$ \ifincsname
$   \ltx@ifundefined{HoLogoCs@\HOLOGO@Variant{#1}}{%
$     #1%
$   }{%
$     \csname HoLogoCs@\HOLOGO@Variant{#1}\endcsname\ltx@secondoftwo
$   }%
$ \else
    \HOLOGO@IfExists\texorpdfstring\texorpdfstring\ltx@firstoftwo
    {%
      \HologoRobust{#1}%
    }{%
      \ltx@ifundefined{HoLogoBkm@\HOLOGO@Variant{#1}}{%
        \ltx@ifundefined{HoLogo@#1}{?#1?}{#1}%
      }{%
        \csname HoLogoBkm@\HOLOGO@Variant{#1}\endcsname
        \ltx@secondoftwo
      }%
    }%
$ \fi
}
%    \end{macrocode}
%    \end{macro}
%
%    \begin{macro}{\hologoVariant}
%    \begin{macrocode}
\def\hologoVariant#1#2{%
  \ifx\relax#2\relax
    \hologo{#1}%
  \else
$   \ifincsname
$     \ltx@ifundefined{HoLogoCs@#1@#2}{%
$       #1%
$     }{%
$       \csname HoLogoCs@#1@#2\endcsname\ltx@firstoftwo
$     }%
$   \else
      \HOLOGO@IfExists\texorpdfstring\texorpdfstring\ltx@firstoftwo
      {%
        \hologoVariantRobust{#1}{#2}%
      }{%
        \ltx@ifundefined{HoLogoBkm@#1@#2}{%
          \ltx@ifundefined{HoLogo@#1}{?#1?}{#1}%
        }{%
          \csname HoLogoBkm@#1@#2\endcsname
          \ltx@firstoftwo
        }%
      }%
$   \fi
  \fi
}
%    \end{macrocode}
%    \end{macro}
%    \begin{macro}{\HologoVariant}
%    \begin{macrocode}
\def\HologoVariant#1#2{%
  \ifx\relax#2\relax
    \Hologo{#1}%
  \else
$   \ifincsname
$     \ltx@ifundefined{HoLogoCs@#1@#2}{%
$       #1%
$     }{%
$       \csname HoLogoCs@#1@#2\endcsname\ltx@secondoftwo
$     }%
$   \else
      \HOLOGO@IfExists\texorpdfstring\texorpdfstring\ltx@firstoftwo
      {%
        \HologoVariantRobust{#1}{#2}%
      }{%
        \ltx@ifundefined{HoLogoBkm@#1@#2}{%
          \ltx@ifundefined{HoLogo@#1}{?#1?}{#1}%
        }{%
          \csname HoLogoBkm@#1@#2\endcsname
          \ltx@secondoftwo
        }%
      }%
$   \fi
  \fi
}
%    \end{macrocode}
%    \end{macro}
%
%    \begin{macrocode}
\catcode`\$=3 %
%    \end{macrocode}
%
% \subsubsection{\cs{hologoRobust} and friends}
%
%    \begin{macro}{\hologoRobust}
%    \begin{macrocode}
\ltx@IfUndefined{protected}{%
  \ltx@IfUndefined{DeclareRobustCommand}{%
    \def\hologoRobust#1%
  }{%
    \DeclareRobustCommand*\hologoRobust[1]%
  }%
}{%
  \protected\def\hologoRobust#1%
}%
{%
  \edef\HOLOGO@name{#1}%
  \ltx@IfUndefined{HoLogo@\HOLOGO@Variant\HOLOGO@name}{%
    \@PackageError{hologo}{%
      Unknown logo `\HOLOGO@name'%
    }\@ehc
    ?\HOLOGO@name?%
  }{%
    \ltx@IfUndefined{ver@tex4ht.sty}{%
      \HoLogoFont@font\HOLOGO@name{general}{%
        \csname HoLogo@\HOLOGO@Variant\HOLOGO@name\endcsname
        \ltx@firstoftwo
      }%
    }{%
      \ltx@IfUndefined{HoLogoHtml@\HOLOGO@Variant\HOLOGO@name}{%
        \HOLOGO@name
      }{%
        \csname HoLogoHtml@\HOLOGO@Variant\HOLOGO@name\endcsname
        \ltx@firstoftwo
      }%
    }%
  }%
}
%    \end{macrocode}
%    \end{macro}
%    \begin{macro}{\HologoRobust}
%    \begin{macrocode}
\ltx@IfUndefined{protected}{%
  \ltx@IfUndefined{DeclareRobustCommand}{%
    \def\HologoRobust#1%
  }{%
    \DeclareRobustCommand*\HologoRobust[1]%
  }%
}{%
  \protected\def\HologoRobust#1%
}%
{%
  \edef\HOLOGO@name{#1}%
  \ltx@IfUndefined{HoLogo@\HOLOGO@Variant\HOLOGO@name}{%
    \@PackageError{hologo}{%
      Unknown logo `\HOLOGO@name'%
    }\@ehc
    ?\HOLOGO@name?%
  }{%
    \ltx@IfUndefined{ver@tex4ht.sty}{%
      \HoLogoFont@font\HOLOGO@name{general}{%
        \csname HoLogo@\HOLOGO@Variant\HOLOGO@name\endcsname
        \ltx@secondoftwo
      }%
    }{%
      \ltx@IfUndefined{HoLogoHtml@\HOLOGO@Variant\HOLOGO@name}{%
        \expandafter\HOLOGO@Uppercase\HOLOGO@name
      }{%
        \csname HoLogoHtml@\HOLOGO@Variant\HOLOGO@name\endcsname
        \ltx@secondoftwo
      }%
    }%
  }%
}
%    \end{macrocode}
%    \end{macro}
%    \begin{macro}{\hologoVariantRobust}
%    \begin{macrocode}
\ltx@IfUndefined{protected}{%
  \ltx@IfUndefined{DeclareRobustCommand}{%
    \def\hologoVariantRobust#1#2%
  }{%
    \DeclareRobustCommand*\hologoVariantRobust[2]%
  }%
}{%
  \protected\def\hologoVariantRobust#1#2%
}%
{%
  \begingroup
    \hologoLogoSetup{#1}{variant={#2}}%
    \hologoRobust{#1}%
  \endgroup
}
%    \end{macrocode}
%    \end{macro}
%    \begin{macro}{\HologoVariantRobust}
%    \begin{macrocode}
\ltx@IfUndefined{protected}{%
  \ltx@IfUndefined{DeclareRobustCommand}{%
    \def\HologoVariantRobust#1#2%
  }{%
    \DeclareRobustCommand*\HologoVariantRobust[2]%
  }%
}{%
  \protected\def\HologoVariantRobust#1#2%
}%
{%
  \begingroup
    \hologoLogoSetup{#1}{variant={#2}}%
    \HologoRobust{#1}%
  \endgroup
}
%    \end{macrocode}
%    \end{macro}
%
%    \begin{macro}{\hologorobust}
%    Macro \cs{hologorobust} is only defined for compatibility.
%    Its use is deprecated.
%    \begin{macrocode}
\def\hologorobust{\hologoRobust}
%    \end{macrocode}
%    \end{macro}
%
% \subsection{Helpers}
%
%    \begin{macro}{\HOLOGO@Uppercase}
%    Macro \cs{HOLOGO@Uppercase} is restricted to \cs{uppercase},
%    because \hologo{plainTeX} or \hologo{iniTeX} do not provide
%    \cs{MakeUppercase}.
%    \begin{macrocode}
\def\HOLOGO@Uppercase#1{\uppercase{#1}}
%    \end{macrocode}
%    \end{macro}
%
%    \begin{macro}{\HOLOGO@PdfdocUnicode}
%    \begin{macrocode}
\def\HOLOGO@PdfdocUnicode{%
  \ifx\ifHy@unicode\iftrue
    \expandafter\ltx@secondoftwo
  \else
    \expandafter\ltx@firstoftwo
  \fi
}
%    \end{macrocode}
%    \end{macro}
%
%    \begin{macro}{\HOLOGO@Math}
%    \begin{macrocode}
\def\HOLOGO@MathSetup{%
  \mathsurround0pt\relax
  \HOLOGO@IfExists\f@series{%
    \if b\expandafter\ltx@car\f@series x\@nil
      \csname boldmath\endcsname
   \fi
  }{}%
}
%    \end{macrocode}
%    \end{macro}
%
%    \begin{macro}{\HOLOGO@TempDimen}
%    \begin{macrocode}
\dimendef\HOLOGO@TempDimen=\ltx@zero
%    \end{macrocode}
%    \end{macro}
%    \begin{macro}{\HOLOGO@NegativeKerning}
%    \begin{macrocode}
\def\HOLOGO@NegativeKerning#1{%
  \begingroup
    \HOLOGO@TempDimen=0pt\relax
    \comma@parse@normalized{#1}{%
      \ifdim\HOLOGO@TempDimen=0pt %
        \expandafter\HOLOGO@@NegativeKerning\comma@entry
      \fi
      \ltx@gobble
    }%
    \ifdim\HOLOGO@TempDimen<0pt %
      \kern\HOLOGO@TempDimen
    \fi
  \endgroup
}
%    \end{macrocode}
%    \end{macro}
%    \begin{macro}{\HOLOGO@@NegativeKerning}
%    \begin{macrocode}
\def\HOLOGO@@NegativeKerning#1#2{%
  \setbox\ltx@zero\hbox{#1#2}%
  \HOLOGO@TempDimen=\wd\ltx@zero
  \setbox\ltx@zero\hbox{#1\kern0pt#2}%
  \advance\HOLOGO@TempDimen by -\wd\ltx@zero
}
%    \end{macrocode}
%    \end{macro}
%
%    \begin{macro}{\HOLOGO@SpaceFactor}
%    \begin{macrocode}
\def\HOLOGO@SpaceFactor{%
  \spacefactor1000 %
}
%    \end{macrocode}
%    \end{macro}
%
%    \begin{macro}{\HOLOGO@Span}
%    \begin{macrocode}
\def\HOLOGO@Span#1#2{%
  \HCode{<span class="HoLogo-#1">}%
  #2%
  \HCode{</span>}%
}
%    \end{macrocode}
%    \end{macro}
%
% \subsubsection{Text subscript}
%
%    \begin{macro}{\HOLOGO@SubScript}%
%    \begin{macrocode}
\def\HOLOGO@SubScript#1{%
  \ltx@IfUndefined{textsubscript}{%
    \ltx@IfUndefined{text}{%
      \ltx@mbox{%
        \mathsurround=0pt\relax
        $%
          _{%
            \ltx@IfUndefined{sf@size}{%
              \mathrm{#1}%
            }{%
              \mbox{%
                \fontsize\sf@size{0pt}\selectfont
                #1%
              }%
            }%
          }%
        $%
      }%
    }{%
      \ltx@mbox{%
        \mathsurround=0pt\relax
        $_{\text{#1}}$%
      }%
    }%
  }{%
    \textsubscript{#1}%
  }%
}
%    \end{macrocode}
%    \end{macro}
%
% \subsection{\hologo{TeX} and friends}
%
% \subsubsection{\hologo{TeX}}
%
%    \begin{macro}{\HoLogo@TeX}
%    Source: \hologo{LaTeX} kernel.
%    \begin{macrocode}
\def\HoLogo@TeX#1{%
  T\kern-.1667em\lower.5ex\hbox{E}\kern-.125emX\HOLOGO@SpaceFactor
}
%    \end{macrocode}
%    \end{macro}
%    \begin{macro}{\HoLogoHtml@TeX}
%    \begin{macrocode}
\def\HoLogoHtml@TeX#1{%
  \HoLogoCss@TeX
  \HOLOGO@Span{TeX}{%
    T%
    \HOLOGO@Span{e}{%
      E%
    }%
    X%
  }%
}
%    \end{macrocode}
%    \end{macro}
%    \begin{macro}{\HoLogoCss@TeX}
%    \begin{macrocode}
\def\HoLogoCss@TeX{%
  \Css{%
    span.HoLogo-TeX span.HoLogo-e{%
      position:relative;%
      top:.5ex;%
      margin-left:-.1667em;%
      margin-right:-.125em;%
    }%
  }%
  \Css{%
    a span.HoLogo-TeX span.HoLogo-e{%
      text-decoration:none;%
    }%
  }%
  \global\let\HoLogoCss@TeX\relax
}
%    \end{macrocode}
%    \end{macro}
%
% \subsubsection{\hologo{plainTeX}}
%
%    \begin{macro}{\HoLogo@plainTeX@space}
%    Source: ``The \hologo{TeX}book''
%    \begin{macrocode}
\def\HoLogo@plainTeX@space#1{%
  \HOLOGO@mbox{#1{p}{P}lain}\HOLOGO@space\hologo{TeX}%
}
%    \end{macrocode}
%    \end{macro}
%    \begin{macro}{\HoLogoCs@plainTeX@space}
%    \begin{macrocode}
\def\HoLogoCs@plainTeX@space#1{#1{p}{P}lain TeX}%
%    \end{macrocode}
%    \end{macro}
%    \begin{macro}{\HoLogoBkm@plainTeX@space}
%    \begin{macrocode}
\def\HoLogoBkm@plainTeX@space#1{%
  #1{p}{P}lain \hologo{TeX}%
}
%    \end{macrocode}
%    \end{macro}
%    \begin{macro}{\HoLogoHtml@plainTeX@space}
%    \begin{macrocode}
\def\HoLogoHtml@plainTeX@space#1{%
  #1{p}{P}lain \hologo{TeX}%
}
%    \end{macrocode}
%    \end{macro}
%
%    \begin{macro}{\HoLogo@plainTeX@hyphen}
%    \begin{macrocode}
\def\HoLogo@plainTeX@hyphen#1{%
  \HOLOGO@mbox{#1{p}{P}lain}\HOLOGO@hyphen\hologo{TeX}%
}
%    \end{macrocode}
%    \end{macro}
%    \begin{macro}{\HoLogoCs@plainTeX@hyphen}
%    \begin{macrocode}
\def\HoLogoCs@plainTeX@hyphen#1{#1{p}{P}lain-TeX}
%    \end{macrocode}
%    \end{macro}
%    \begin{macro}{\HoLogoBkm@plainTeX@hyphen}
%    \begin{macrocode}
\def\HoLogoBkm@plainTeX@hyphen#1{%
  #1{p}{P}lain-\hologo{TeX}%
}
%    \end{macrocode}
%    \end{macro}
%    \begin{macro}{\HoLogoHtml@plainTeX@hyphen}
%    \begin{macrocode}
\def\HoLogoHtml@plainTeX@hyphen#1{%
  #1{p}{P}lain-\hologo{TeX}%
}
%    \end{macrocode}
%    \end{macro}
%
%    \begin{macro}{\HoLogo@plainTeX@runtogether}
%    \begin{macrocode}
\def\HoLogo@plainTeX@runtogether#1{%
  \HOLOGO@mbox{#1{p}{P}lain\hologo{TeX}}%
}
%    \end{macrocode}
%    \end{macro}
%    \begin{macro}{\HoLogoCs@plainTeX@runtogether}
%    \begin{macrocode}
\def\HoLogoCs@plainTeX@runtogether#1{#1{p}{P}lainTeX}
%    \end{macrocode}
%    \end{macro}
%    \begin{macro}{\HoLogoBkm@plainTeX@runtogether}
%    \begin{macrocode}
\def\HoLogoBkm@plainTeX@runtogether#1{%
  #1{p}{P}lain\hologo{TeX}%
}
%    \end{macrocode}
%    \end{macro}
%    \begin{macro}{\HoLogoHtml@plainTeX@runtogether}
%    \begin{macrocode}
\def\HoLogoHtml@plainTeX@runtogether#1{%
  #1{p}{P}lain\hologo{TeX}%
}
%    \end{macrocode}
%    \end{macro}
%
%    \begin{macro}{\HoLogo@plainTeX}
%    \begin{macrocode}
\def\HoLogo@plainTeX{\HoLogo@plainTeX@space}
%    \end{macrocode}
%    \end{macro}
%    \begin{macro}{\HoLogoCs@plainTeX}
%    \begin{macrocode}
\def\HoLogoCs@plainTeX{\HoLogoCs@plainTeX@space}
%    \end{macrocode}
%    \end{macro}
%    \begin{macro}{\HoLogoBkm@plainTeX}
%    \begin{macrocode}
\def\HoLogoBkm@plainTeX{\HoLogoBkm@plainTeX@space}
%    \end{macrocode}
%    \end{macro}
%    \begin{macro}{\HoLogoHtml@plainTeX}
%    \begin{macrocode}
\def\HoLogoHtml@plainTeX{\HoLogoHtml@plainTeX@space}
%    \end{macrocode}
%    \end{macro}
%
% \subsubsection{\hologo{LaTeX}}
%
%    Source: \hologo{LaTeX} kernel.
%\begin{quote}
%\begin{verbatim}
%\DeclareRobustCommand{\LaTeX}{%
%  L%
%  \kern-.36em%
%  {%
%    \sbox\z@ T%
%    \vbox to\ht\z@{%
%      \hbox{%
%        \check@mathfonts
%        \fontsize\sf@size\z@
%        \math@fontsfalse
%        \selectfont
%        A%
%      }%
%      \vss
%    }%
%  }%
%  \kern-.15em%
%  \TeX
%}
%\end{verbatim}
%\end{quote}
%
%    \begin{macro}{\HoLogo@La}
%    \begin{macrocode}
\def\HoLogo@La#1{%
  L%
  \kern-.36em%
  \begingroup
    \setbox\ltx@zero\hbox{T}%
    \vbox to\ht\ltx@zero{%
      \hbox{%
        \ltx@ifundefined{check@mathfonts}{%
          \csname sevenrm\endcsname
        }{%
          \check@mathfonts
          \fontsize\sf@size{0pt}%
          \math@fontsfalse\selectfont
        }%
        A%
      }%
      \vss
    }%
  \endgroup
}
%    \end{macrocode}
%    \end{macro}
%
%    \begin{macro}{\HoLogo@LaTeX}
%    Source: \hologo{LaTeX} kernel.
%    \begin{macrocode}
\def\HoLogo@LaTeX#1{%
  \hologo{La}%
  \kern-.15em%
  \hologo{TeX}%
}
%    \end{macrocode}
%    \end{macro}
%    \begin{macro}{\HoLogoHtml@LaTeX}
%    \begin{macrocode}
\def\HoLogoHtml@LaTeX#1{%
  \HoLogoCss@LaTeX
  \HOLOGO@Span{LaTeX}{%
    L%
    \HOLOGO@Span{a}{%
      A%
    }%
    \hologo{TeX}%
  }%
}
%    \end{macrocode}
%    \end{macro}
%    \begin{macro}{\HoLogoCss@LaTeX}
%    \begin{macrocode}
\def\HoLogoCss@LaTeX{%
  \Css{%
    span.HoLogo-LaTeX span.HoLogo-a{%
      position:relative;%
      top:-.5ex;%
      margin-left:-.36em;%
      margin-right:-.15em;%
      font-size:85\%;%
    }%
  }%
  \global\let\HoLogoCss@LaTeX\relax
}
%    \end{macrocode}
%    \end{macro}
%
% \subsubsection{\hologo{(La)TeX}}
%
%    \begin{macro}{\HoLogo@LaTeXTeX}
%    The kerning around the parentheses is taken
%    from package \xpackage{dtklogos} \cite{dtklogos}.
%\begin{quote}
%\begin{verbatim}
%\DeclareRobustCommand{\LaTeXTeX}{%
%  (%
%  \kern-.15em%
%  L%
%  \kern-.36em%
%  {%
%    \sbox\z@ T%
%    \vbox to\ht0{%
%      \hbox{%
%        $\m@th$%
%        \csname S@\f@size\endcsname
%        \fontsize\sf@size\z@
%        \math@fontsfalse
%        \selectfont
%        A%
%      }%
%      \vss
%    }%
%  }%
%  \kern-.2em%
%  )%
%  \kern-.15em%
%  \TeX
%}
%\end{verbatim}
%\end{quote}
%    \begin{macrocode}
\def\HoLogo@LaTeXTeX#1{%
  (%
  \kern-.15em%
  \hologo{La}%
  \kern-.2em%
  )%
  \kern-.15em%
  \hologo{TeX}%
}
%    \end{macrocode}
%    \end{macro}
%    \begin{macro}{\HoLogoBkm@LaTeXTeX}
%    \begin{macrocode}
\def\HoLogoBkm@LaTeXTeX#1{(La)TeX}
%    \end{macrocode}
%    \end{macro}
%
%    \begin{macro}{\HoLogo@(La)TeX}
%    \begin{macrocode}
\expandafter
\let\csname HoLogo@(La)TeX\endcsname\HoLogo@LaTeXTeX
%    \end{macrocode}
%    \end{macro}
%    \begin{macro}{\HoLogoBkm@(La)TeX}
%    \begin{macrocode}
\expandafter
\let\csname HoLogoBkm@(La)TeX\endcsname\HoLogoBkm@LaTeXTeX
%    \end{macrocode}
%    \end{macro}
%    \begin{macro}{\HoLogoHtml@LaTeXTeX}
%    \begin{macrocode}
\def\HoLogoHtml@LaTeXTeX#1{%
  \HoLogoCss@LaTeXTeX
  \HOLOGO@Span{LaTeXTeX}{%
    (%
    \HOLOGO@Span{L}{L}%
    \HOLOGO@Span{a}{A}%
    \HOLOGO@Span{ParenRight}{)}%
    \hologo{TeX}%
  }%
}
%    \end{macrocode}
%    \end{macro}
%    \begin{macro}{\HoLogoHtml@(La)TeX}
%    Kerning after opening parentheses and before closing parentheses
%    is $-0.1$\,em. The original values $-0.15$\,em
%    looked too ugly for a serif font.
%    \begin{macrocode}
\expandafter
\let\csname HoLogoHtml@(La)TeX\endcsname\HoLogoHtml@LaTeXTeX
%    \end{macrocode}
%    \end{macro}
%    \begin{macro}{\HoLogoCss@LaTeXTeX}
%    \begin{macrocode}
\def\HoLogoCss@LaTeXTeX{%
  \Css{%
    span.HoLogo-LaTeXTeX span.HoLogo-L{%
      margin-left:-.1em;%
    }%
  }%
  \Css{%
    span.HoLogo-LaTeXTeX span.HoLogo-a{%
      position:relative;%
      top:-.5ex;%
      margin-left:-.36em;%
      margin-right:-.1em;%
      font-size:85\%;%
    }%
  }%
  \Css{%
    span.HoLogo-LaTeXTeX span.HoLogo-ParenRight{%
      margin-right:-.15em;%
    }%
  }%
  \global\let\HoLogoCss@LaTeXTeX\relax
}
%    \end{macrocode}
%    \end{macro}
%
% \subsubsection{\hologo{LaTeXe}}
%
%    \begin{macro}{\HoLogo@LaTeXe}
%    Source: \hologo{LaTeX} kernel
%    \begin{macrocode}
\def\HoLogo@LaTeXe#1{%
  \hologo{LaTeX}%
  \kern.15em%
  \hbox{%
    \HOLOGO@MathSetup
    2%
    $_{\textstyle\varepsilon}$%
  }%
}
%    \end{macrocode}
%    \end{macro}
%
%    \begin{macro}{\HoLogoCs@LaTeXe}
%    \begin{macrocode}
\ifnum64=`\^^^^0040\relax % test for big chars of LuaTeX/XeTeX
  \catcode`\$=9 %
  \catcode`\&=14 %
\else
  \catcode`\$=14 %
  \catcode`\&=9 %
\fi
\def\HoLogoCs@LaTeXe#1{%
  LaTeX2%
$ \string ^^^^0395%
& e%
}%
\catcode`\$=3 %
\catcode`\&=4 %
%    \end{macrocode}
%    \end{macro}
%
%    \begin{macro}{\HoLogoBkm@LaTeXe}
%    \begin{macrocode}
\def\HoLogoBkm@LaTeXe#1{%
  \hologo{LaTeX}%
  2%
  \HOLOGO@PdfdocUnicode{e}{\textepsilon}%
}
%    \end{macrocode}
%    \end{macro}
%
%    \begin{macro}{\HoLogoHtml@LaTeXe}
%    \begin{macrocode}
\def\HoLogoHtml@LaTeXe#1{%
  \HoLogoCss@LaTeXe
  \HOLOGO@Span{LaTeX2e}{%
    \hologo{LaTeX}%
    \HOLOGO@Span{2}{2}%
    \HOLOGO@Span{e}{%
      \HOLOGO@MathSetup
      \ensuremath{\textstyle\varepsilon}%
    }%
  }%
}
%    \end{macrocode}
%    \end{macro}
%    \begin{macro}{\HoLogoCss@LaTeXe}
%    \begin{macrocode}
\def\HoLogoCss@LaTeXe{%
  \Css{%
    span.HoLogo-LaTeX2e span.HoLogo-2{%
      padding-left:.15em;%
    }%
  }%
  \Css{%
    span.HoLogo-LaTeX2e span.HoLogo-e{%
      position:relative;%
      top:.35ex;%
      text-decoration:none;%
    }%
  }%
  \global\let\HoLogoCss@LaTeXe\relax
}
%    \end{macrocode}
%    \end{macro}
%
%    \begin{macro}{\HoLogo@LaTeX2e}
%    \begin{macrocode}
\expandafter
\let\csname HoLogo@LaTeX2e\endcsname\HoLogo@LaTeXe
%    \end{macrocode}
%    \end{macro}
%    \begin{macro}{\HoLogoCs@LaTeX2e}
%    \begin{macrocode}
\expandafter
\let\csname HoLogoCs@LaTeX2e\endcsname\HoLogoCs@LaTeXe
%    \end{macrocode}
%    \end{macro}
%    \begin{macro}{\HoLogoBkm@LaTeX2e}
%    \begin{macrocode}
\expandafter
\let\csname HoLogoBkm@LaTeX2e\endcsname\HoLogoBkm@LaTeXe
%    \end{macrocode}
%    \end{macro}
%    \begin{macro}{\HoLogoHtml@LaTeX2e}
%    \begin{macrocode}
\expandafter
\let\csname HoLogoHtml@LaTeX2e\endcsname\HoLogoHtml@LaTeXe
%    \end{macrocode}
%    \end{macro}
%
% \subsubsection{\hologo{LaTeX3}}
%
%    \begin{macro}{\HoLogo@LaTeX3}
%    Source: \hologo{LaTeX} kernel
%    \begin{macrocode}
\expandafter\def\csname HoLogo@LaTeX3\endcsname#1{%
  \hologo{LaTeX}%
  3%
}
%    \end{macrocode}
%    \end{macro}
%
%    \begin{macro}{\HoLogoBkm@LaTeX3}
%    \begin{macrocode}
\expandafter\def\csname HoLogoBkm@LaTeX3\endcsname#1{%
  \hologo{LaTeX}%
  3%
}
%    \end{macrocode}
%    \end{macro}
%    \begin{macro}{\HoLogoHtml@LaTeX3}
%    \begin{macrocode}
\expandafter
\let\csname HoLogoHtml@LaTeX3\expandafter\endcsname
\csname HoLogo@LaTeX3\endcsname
%    \end{macrocode}
%    \end{macro}
%
% \subsubsection{\hologo{LaTeXML}}
%
%    \begin{macro}{\HoLogo@LaTeXML}
%    \begin{macrocode}
\def\HoLogo@LaTeXML#1{%
  \HOLOGO@mbox{%
    \hologo{La}%
    \kern-.15em%
    T%
    \kern-.1667em%
    \lower.5ex\hbox{E}%
    \kern-.125em%
    \HoLogoFont@font{LaTeXML}{sc}{xml}%
  }%
}
%    \end{macrocode}
%    \end{macro}
%    \begin{macro}{\HoLogoHtml@pdfLaTeX}
%    \begin{macrocode}
\def\HoLogoHtml@LaTeXML#1{%
  \HOLOGO@Span{LaTeXML}{%
    \HoLogoCss@LaTeX
    \HoLogoCss@TeX
    \HOLOGO@Span{LaTeX}{%
      L%
      \HOLOGO@Span{a}{%
        A%
      }%
    }%
    \HOLOGO@Span{TeX}{%
      T%
      \HOLOGO@Span{e}{%
        E%
      }%
    }%
    \HCode{<span style="font-variant: small-caps;">}%
    xml%
    \HCode{</span>}%
  }%
}
%    \end{macrocode}
%    \end{macro}
%
% \subsubsection{\hologo{eTeX}}
%
%    \begin{macro}{\HoLogo@eTeX}
%    Source: package \xpackage{etex}
%    \begin{macrocode}
\def\HoLogo@eTeX#1{%
  \ltx@mbox{%
    \HOLOGO@MathSetup
    $\varepsilon$%
    -%
    \HOLOGO@NegativeKerning{-T,T-,To}%
    \hologo{TeX}%
  }%
}
%    \end{macrocode}
%    \end{macro}
%    \begin{macro}{\HoLogoCs@eTeX}
%    \begin{macrocode}
\ifnum64=`\^^^^0040\relax % test for big chars of LuaTeX/XeTeX
  \catcode`\$=9 %
  \catcode`\&=14 %
\else
  \catcode`\$=14 %
  \catcode`\&=9 %
\fi
\def\HoLogoCs@eTeX#1{%
$ #1{\string ^^^^0395}{\string ^^^^03b5}%
& #1{e}{E}%
  TeX%
}%
\catcode`\$=3 %
\catcode`\&=4 %
%    \end{macrocode}
%    \end{macro}
%    \begin{macro}{\HoLogoBkm@eTeX}
%    \begin{macrocode}
\def\HoLogoBkm@eTeX#1{%
  \HOLOGO@PdfdocUnicode{#1{e}{E}}{\textepsilon}%
  -%
  \hologo{TeX}%
}
%    \end{macrocode}
%    \end{macro}
%    \begin{macro}{\HoLogoHtml@eTeX}
%    \begin{macrocode}
\def\HoLogoHtml@eTeX#1{%
  \ltx@mbox{%
    \HOLOGO@MathSetup
    $\varepsilon$%
    -%
    \hologo{TeX}%
  }%
}
%    \end{macrocode}
%    \end{macro}
%
% \subsubsection{\hologo{iniTeX}}
%
%    \begin{macro}{\HoLogo@iniTeX}
%    \begin{macrocode}
\def\HoLogo@iniTeX#1{%
  \HOLOGO@mbox{%
    #1{i}{I}ni\hologo{TeX}%
  }%
}
%    \end{macrocode}
%    \end{macro}
%    \begin{macro}{\HoLogoCs@iniTeX}
%    \begin{macrocode}
\def\HoLogoCs@iniTeX#1{#1{i}{I}niTeX}
%    \end{macrocode}
%    \end{macro}
%    \begin{macro}{\HoLogoBkm@iniTeX}
%    \begin{macrocode}
\def\HoLogoBkm@iniTeX#1{%
  #1{i}{I}ni\hologo{TeX}%
}
%    \end{macrocode}
%    \end{macro}
%    \begin{macro}{\HoLogoHtml@iniTeX}
%    \begin{macrocode}
\let\HoLogoHtml@iniTeX\HoLogo@iniTeX
%    \end{macrocode}
%    \end{macro}
%
% \subsubsection{\hologo{virTeX}}
%
%    \begin{macro}{\HoLogo@virTeX}
%    \begin{macrocode}
\def\HoLogo@virTeX#1{%
  \HOLOGO@mbox{%
    #1{v}{V}ir\hologo{TeX}%
  }%
}
%    \end{macrocode}
%    \end{macro}
%    \begin{macro}{\HoLogoCs@virTeX}
%    \begin{macrocode}
\def\HoLogoCs@virTeX#1{#1{v}{V}irTeX}
%    \end{macrocode}
%    \end{macro}
%    \begin{macro}{\HoLogoBkm@virTeX}
%    \begin{macrocode}
\def\HoLogoBkm@virTeX#1{%
  #1{v}{V}ir\hologo{TeX}%
}
%    \end{macrocode}
%    \end{macro}
%    \begin{macro}{\HoLogoHtml@virTeX}
%    \begin{macrocode}
\let\HoLogoHtml@virTeX\HoLogo@virTeX
%    \end{macrocode}
%    \end{macro}
%
% \subsubsection{\hologo{SliTeX}}
%
% \paragraph{Definitions of the three variants.}
%
%    \begin{macro}{\HoLogo@SLiTeX@lift}
%    \begin{macrocode}
\def\HoLogo@SLiTeX@lift#1{%
  \HoLogoFont@font{SliTeX}{rm}{%
    S%
    \kern-.06em%
    L%
    \kern-.18em%
    \raise.32ex\hbox{\HoLogoFont@font{SliTeX}{sc}{i}}%
    \HOLOGO@discretionary
    \kern-.06em%
    \hologo{TeX}%
  }%
}
%    \end{macrocode}
%    \end{macro}
%    \begin{macro}{\HoLogoBkm@SLiTeX@lift}
%    \begin{macrocode}
\def\HoLogoBkm@SLiTeX@lift#1{SLiTeX}
%    \end{macrocode}
%    \end{macro}
%    \begin{macro}{\HoLogoHtml@SLiTeX@lift}
%    \begin{macrocode}
\def\HoLogoHtml@SLiTeX@lift#1{%
  \HoLogoCss@SLiTeX@lift
  \HOLOGO@Span{SLiTeX-lift}{%
    \HoLogoFont@font{SliTeX}{rm}{%
      S%
      \HOLOGO@Span{L}{L}%
      \HOLOGO@Span{i}{i}%
      \hologo{TeX}%
    }%
  }%
}
%    \end{macrocode}
%    \end{macro}
%    \begin{macro}{\HoLogoCss@SLiTeX@lift}
%    \begin{macrocode}
\def\HoLogoCss@SLiTeX@lift{%
  \Css{%
    span.HoLogo-SLiTeX-lift span.HoLogo-L{%
      margin-left:-.06em;%
      margin-right:-.18em;%
    }%
  }%
  \Css{%
    span.HoLogo-SLiTeX-lift span.HoLogo-i{%
      position:relative;%
      top:-.32ex;%
      margin-right:-.06em;%
      font-variant:small-caps;%
    }%
  }%
  \global\let\HoLogoCss@SLiTeX@lift\relax
}
%    \end{macrocode}
%    \end{macro}
%
%    \begin{macro}{\HoLogo@SliTeX@simple}
%    \begin{macrocode}
\def\HoLogo@SliTeX@simple#1{%
  \HoLogoFont@font{SliTeX}{rm}{%
    \ltx@mbox{%
      \HoLogoFont@font{SliTeX}{sc}{Sli}%
    }%
    \HOLOGO@discretionary
    \hologo{TeX}%
  }%
}
%    \end{macrocode}
%    \end{macro}
%    \begin{macro}{\HoLogoBkm@SliTeX@simple}
%    \begin{macrocode}
\def\HoLogoBkm@SliTeX@simple#1{SliTeX}
%    \end{macrocode}
%    \end{macro}
%    \begin{macro}{\HoLogoHtml@SliTeX@simple}
%    \begin{macrocode}
\let\HoLogoHtml@SliTeX@simple\HoLogo@SliTeX@simple
%    \end{macrocode}
%    \end{macro}
%
%    \begin{macro}{\HoLogo@SliTeX@narrow}
%    \begin{macrocode}
\def\HoLogo@SliTeX@narrow#1{%
  \HoLogoFont@font{SliTeX}{rm}{%
    \ltx@mbox{%
      S%
      \kern-.06em%
      \HoLogoFont@font{SliTeX}{sc}{%
        l%
        \kern-.035em%
        i%
      }%
    }%
    \HOLOGO@discretionary
    \kern-.06em%
    \hologo{TeX}%
  }%
}
%    \end{macrocode}
%    \end{macro}
%    \begin{macro}{\HoLogoBkm@SliTeX@narrow}
%    \begin{macrocode}
\def\HoLogoBkm@SliTeX@narrow#1{SliTeX}
%    \end{macrocode}
%    \end{macro}
%    \begin{macro}{\HoLogoHtml@SliTeX@narrow}
%    \begin{macrocode}
\def\HoLogoHtml@SliTeX@narrow#1{%
  \HoLogoCss@SliTeX@narrow
  \HOLOGO@Span{SliTeX-narrow}{%
    \HoLogoFont@font{SliTeX}{rm}{%
      S%
        \HOLOGO@Span{l}{l}%
        \HOLOGO@Span{i}{i}%
      \hologo{TeX}%
    }%
  }%
}
%    \end{macrocode}
%    \end{macro}
%    \begin{macro}{\HoLogoCss@SliTeX@narrow}
%    \begin{macrocode}
\def\HoLogoCss@SliTeX@narrow{%
  \Css{%
    span.HoLogo-SliTeX-narrow span.HoLogo-l{%
      margin-left:-.06em;%
      margin-right:-.035em;%
      font-variant:small-caps;%
    }%
  }%
  \Css{%
    span.HoLogo-SliTeX-narrow span.HoLogo-i{%
      margin-right:-.06em;%
      font-variant:small-caps;%
    }%
  }%
  \global\let\HoLogoCss@SliTeX@narrow\relax
}
%    \end{macrocode}
%    \end{macro}
%
% \paragraph{Macro set completion.}
%
%    \begin{macro}{\HoLogo@SLiTeX@simple}
%    \begin{macrocode}
\def\HoLogo@SLiTeX@simple{\HoLogo@SliTeX@simple}
%    \end{macrocode}
%    \end{macro}
%    \begin{macro}{\HoLogoBkm@SLiTeX@simple}
%    \begin{macrocode}
\def\HoLogoBkm@SLiTeX@simple{\HoLogoBkm@SliTeX@simple}
%    \end{macrocode}
%    \end{macro}
%    \begin{macro}{\HoLogoHtml@SLiTeX@simple}
%    \begin{macrocode}
\def\HoLogoHtml@SLiTeX@simple{\HoLogoHtml@SliTeX@simple}
%    \end{macrocode}
%    \end{macro}
%
%    \begin{macro}{\HoLogo@SLiTeX@narrow}
%    \begin{macrocode}
\def\HoLogo@SLiTeX@narrow{\HoLogo@SliTeX@narrow}
%    \end{macrocode}
%    \end{macro}
%    \begin{macro}{\HoLogoBkm@SLiTeX@narrow}
%    \begin{macrocode}
\def\HoLogoBkm@SLiTeX@narrow{\HoLogoBkm@SliTeX@narrow}
%    \end{macrocode}
%    \end{macro}
%    \begin{macro}{\HoLogoHtml@SLiTeX@narrow}
%    \begin{macrocode}
\def\HoLogoHtml@SLiTeX@narrow{\HoLogoHtml@SliTeX@narrow}
%    \end{macrocode}
%    \end{macro}
%
%    \begin{macro}{\HoLogo@SliTeX@lift}
%    \begin{macrocode}
\def\HoLogo@SliTeX@lift{\HoLogo@SLiTeX@lift}
%    \end{macrocode}
%    \end{macro}
%    \begin{macro}{\HoLogoBkm@SliTeX@lift}
%    \begin{macrocode}
\def\HoLogoBkm@SliTeX@lift{\HoLogoBkm@SLiTeX@lift}
%    \end{macrocode}
%    \end{macro}
%    \begin{macro}{\HoLogoHtml@SliTeX@lift}
%    \begin{macrocode}
\def\HoLogoHtml@SliTeX@lift{\HoLogoHtml@SLiTeX@lift}
%    \end{macrocode}
%    \end{macro}
%
% \paragraph{Defaults.}
%
%    \begin{macro}{\HoLogo@SLiTeX}
%    \begin{macrocode}
\def\HoLogo@SLiTeX{\HoLogo@SLiTeX@lift}
%    \end{macrocode}
%    \end{macro}
%    \begin{macro}{\HoLogoBkm@SLiTeX}
%    \begin{macrocode}
\def\HoLogoBkm@SLiTeX{\HoLogoBkm@SLiTeX@lift}
%    \end{macrocode}
%    \end{macro}
%    \begin{macro}{\HoLogoHtml@SLiTeX}
%    \begin{macrocode}
\def\HoLogoHtml@SLiTeX{\HoLogoHtml@SLiTeX@lift}
%    \end{macrocode}
%    \end{macro}
%
%    \begin{macro}{\HoLogo@SliTeX}
%    \begin{macrocode}
\def\HoLogo@SliTeX{\HoLogo@SliTeX@narrow}
%    \end{macrocode}
%    \end{macro}
%    \begin{macro}{\HoLogoBkm@SliTeX}
%    \begin{macrocode}
\def\HoLogoBkm@SliTeX{\HoLogoBkm@SliTeX@narrow}
%    \end{macrocode}
%    \end{macro}
%    \begin{macro}{\HoLogoHtml@SliTeX}
%    \begin{macrocode}
\def\HoLogoHtml@SliTeX{\HoLogoHtml@SliTeX@narrow}
%    \end{macrocode}
%    \end{macro}
%
% \subsubsection{\hologo{LuaTeX}}
%
%    \begin{macro}{\HoLogo@LuaTeX}
%    The kerning is an idea of Hans Hagen, see mailing list
%    `luatex at tug dot org' in March 2010.
%    \begin{macrocode}
\def\HoLogo@LuaTeX#1{%
  \HOLOGO@mbox{%
    Lua%
    \HOLOGO@NegativeKerning{aT,oT,To}%
    \hologo{TeX}%
  }%
}
%    \end{macrocode}
%    \end{macro}
%    \begin{macro}{\HoLogoHtml@LuaTeX}
%    \begin{macrocode}
\let\HoLogoHtml@LuaTeX\HoLogo@LuaTeX
%    \end{macrocode}
%    \end{macro}
%
% \subsubsection{\hologo{LuaLaTeX}}
%
%    \begin{macro}{\HoLogo@LuaLaTeX}
%    \begin{macrocode}
\def\HoLogo@LuaLaTeX#1{%
  \HOLOGO@mbox{%
    Lua%
    \hologo{LaTeX}%
  }%
}
%    \end{macrocode}
%    \end{macro}
%    \begin{macro}{\HoLogoHtml@LuaLaTeX}
%    \begin{macrocode}
\let\HoLogoHtml@LuaLaTeX\HoLogo@LuaLaTeX
%    \end{macrocode}
%    \end{macro}
%
% \subsubsection{\hologo{XeTeX}, \hologo{XeLaTeX}}
%
%    \begin{macro}{\HOLOGO@IfCharExists}
%    \begin{macrocode}
\ifluatex
  \ifnum\luatexversion<36 %
  \else
    \def\HOLOGO@IfCharExists#1{%
      \ifnum
        \directlua{%
           if luaotfload and luaotfload.aux then
             if luaotfload.aux.font_has_glyph(%
                    font.current(), \number#1) then % 	 
	       tex.print("1") % 	 
	     end % 	 
	   elseif font and font.fonts and font.current then %
            local f = font.fonts[font.current()]%
            if f.characters and f.characters[\number#1] then %
              tex.print("1")%
            end %
          end%
        }0=\ltx@zero
        \expandafter\ltx@secondoftwo
      \else
        \expandafter\ltx@firstoftwo
      \fi
    }%
  \fi
\fi
\ltx@IfUndefined{HOLOGO@IfCharExists}{%
  \def\HOLOGO@@IfCharExists#1{%
    \begingroup
      \tracinglostchars=\ltx@zero
      \setbox\ltx@zero=\hbox{%
        \kern7sp\char#1\relax
        \ifnum\lastkern>\ltx@zero
          \expandafter\aftergroup\csname iffalse\endcsname
        \else
          \expandafter\aftergroup\csname iftrue\endcsname
        \fi
      }%
      % \if{true|false} from \aftergroup
      \endgroup
      \expandafter\ltx@firstoftwo
    \else
      \endgroup
      \expandafter\ltx@secondoftwo
    \fi
  }%
  \ifxetex
    \ltx@IfUndefined{XeTeXfonttype}{}{%
      \ltx@IfUndefined{XeTeXcharglyph}{}{%
        \def\HOLOGO@IfCharExists#1{%
          \ifnum\XeTeXfonttype\font>\ltx@zero
            \expandafter\ltx@firstofthree
          \else
            \expandafter\ltx@gobble
          \fi
          {%
            \ifnum\XeTeXcharglyph#1>\ltx@zero
              \expandafter\ltx@firstoftwo
            \else
              \expandafter\ltx@secondoftwo
            \fi
          }%
          \HOLOGO@@IfCharExists{#1}%
        }%
      }%
    }%
  \fi
}{}
\ltx@ifundefined{HOLOGO@IfCharExists}{%
  \ifnum64=`\^^^^0040\relax % test for big chars of LuaTeX/XeTeX
    \let\HOLOGO@IfCharExists\HOLOGO@@IfCharExists
  \else
    \def\HOLOGO@IfCharExists#1{%
      \ifnum#1>255 %
        \expandafter\ltx@fourthoffour
      \fi
      \HOLOGO@@IfCharExists{#1}%
    }%
  \fi
}{}
%    \end{macrocode}
%    \end{macro}
%
%    \begin{macro}{\HoLogo@Xe}
%    Source: package \xpackage{dtklogos}
%    \begin{macrocode}
\def\HoLogo@Xe#1{%
  X%
  \kern-.1em\relax
  \HOLOGO@IfCharExists{"018E}{%
    \lower.5ex\hbox{\char"018E}%
  }{%
    \chardef\HOLOGO@choice=\ltx@zero
    \ifdim\fontdimen\ltx@one\font>0pt %
      \ltx@IfUndefined{rotatebox}{%
        \ltx@IfUndefined{pgftext}{%
          \ltx@IfUndefined{psscalebox}{%
            \ltx@IfUndefined{HOLOGO@ScaleBox@\hologoDriver}{%
            }{%
              \chardef\HOLOGO@choice=4 %
            }%
          }{%
            \chardef\HOLOGO@choice=3 %
          }%
        }{%
          \chardef\HOLOGO@choice=2 %
        }%
      }{%
        \chardef\HOLOGO@choice=1 %
      }%
      \ifcase\HOLOGO@choice
        \HOLOGO@WarningUnsupportedDriver{Xe}%
        e%
      \or % 1: \rotatebox
        \begingroup
          \setbox\ltx@zero\hbox{\rotatebox{180}{E}}%
          \ltx@LocDimenA=\dp\ltx@zero
          \advance\ltx@LocDimenA by -.5ex\relax
          \raise\ltx@LocDimenA\box\ltx@zero
        \endgroup
      \or % 2: \pgftext
        \lower.5ex\hbox{%
          \pgfpicture
            \pgftext[rotate=180]{E}%
          \endpgfpicture
        }%
      \or % 3: \psscalebox
        \begingroup
          \setbox\ltx@zero\hbox{\psscalebox{-1 -1}{E}}%
          \ltx@LocDimenA=\dp\ltx@zero
          \advance\ltx@LocDimenA by -.5ex\relax
          \raise\ltx@LocDimenA\box\ltx@zero
        \endgroup
      \or % 4: \HOLOGO@PointReflectBox
        \lower.5ex\hbox{\HOLOGO@PointReflectBox{E}}%
      \else
        \@PackageError{hologo}{Internal error (choice/it}\@ehc
      \fi
    \else
      \ltx@IfUndefined{reflectbox}{%
        \ltx@IfUndefined{pgftext}{%
          \ltx@IfUndefined{psscalebox}{%
            \ltx@IfUndefined{HOLOGO@ScaleBox@\hologoDriver}{%
            }{%
              \chardef\HOLOGO@choice=4 %
            }%
          }{%
            \chardef\HOLOGO@choice=3 %
          }%
        }{%
          \chardef\HOLOGO@choice=2 %
        }%
      }{%
        \chardef\HOLOGO@choice=1 %
      }%
      \ifcase\HOLOGO@choice
        \HOLOGO@WarningUnsupportedDriver{Xe}%
        e%
      \or % 1: reflectbox
        \lower.5ex\hbox{%
          \reflectbox{E}%
        }%
      \or % 2: \pgftext
        \lower.5ex\hbox{%
          \pgfpicture
            \pgftransformxscale{-1}%
            \pgftext{E}%
          \endpgfpicture
        }%
      \or % 3: \psscalebox
        \lower.5ex\hbox{%
          \psscalebox{-1 1}{E}%
        }%
      \or % 4: \HOLOGO@Reflectbox
        \lower.5ex\hbox{%
          \HOLOGO@ReflectBox{E}%
        }%
      \else
        \@PackageError{hologo}{Internal error (choice/up)}\@ehc
      \fi
    \fi
  }%
}
%    \end{macrocode}
%    \end{macro}
%    \begin{macro}{\HoLogoHtml@Xe}
%    \begin{macrocode}
\def\HoLogoHtml@Xe#1{%
  \HoLogoCss@Xe
  \HOLOGO@Span{Xe}{%
    X%
    \HOLOGO@Span{e}{%
      \HCode{&\ltx@hashchar x018e;}%
    }%
  }%
}
%    \end{macrocode}
%    \end{macro}
%    \begin{macro}{\HoLogoCss@Xe}
%    \begin{macrocode}
\def\HoLogoCss@Xe{%
  \Css{%
    span.HoLogo-Xe span.HoLogo-e{%
      position:relative;%
      top:.5ex;%
      left-margin:-.1em;%
    }%
  }%
  \global\let\HoLogoCss@Xe\relax
}
%    \end{macrocode}
%    \end{macro}
%
%    \begin{macro}{\HoLogo@XeTeX}
%    \begin{macrocode}
\def\HoLogo@XeTeX#1{%
  \hologo{Xe}%
  \kern-.15em\relax
  \hologo{TeX}%
}
%    \end{macrocode}
%    \end{macro}
%
%    \begin{macro}{\HoLogoHtml@XeTeX}
%    \begin{macrocode}
\def\HoLogoHtml@XeTeX#1{%
  \HoLogoCss@XeTeX
  \HOLOGO@Span{XeTeX}{%
    \hologo{Xe}%
    \hologo{TeX}%
  }%
}
%    \end{macrocode}
%    \end{macro}
%    \begin{macro}{\HoLogoCss@XeTeX}
%    \begin{macrocode}
\def\HoLogoCss@XeTeX{%
  \Css{%
    span.HoLogo-XeTeX span.HoLogo-TeX{%
      margin-left:-.15em;%
    }%
  }%
  \global\let\HoLogoCss@XeTeX\relax
}
%    \end{macrocode}
%    \end{macro}
%
%    \begin{macro}{\HoLogo@XeLaTeX}
%    \begin{macrocode}
\def\HoLogo@XeLaTeX#1{%
  \hologo{Xe}%
  \kern-.13em%
  \hologo{LaTeX}%
}
%    \end{macrocode}
%    \end{macro}
%    \begin{macro}{\HoLogoHtml@XeLaTeX}
%    \begin{macrocode}
\def\HoLogoHtml@XeLaTeX#1{%
  \HoLogoCss@XeLaTeX
  \HOLOGO@Span{XeLaTeX}{%
    \hologo{Xe}%
    \hologo{LaTeX}%
  }%
}
%    \end{macrocode}
%    \end{macro}
%    \begin{macro}{\HoLogoCss@XeLaTeX}
%    \begin{macrocode}
\def\HoLogoCss@XeLaTeX{%
  \Css{%
    span.HoLogo-XeLaTeX span.HoLogo-Xe{%
      margin-right:-.13em;%
    }%
  }%
  \global\let\HoLogoCss@XeLaTeX\relax
}
%    \end{macrocode}
%    \end{macro}
%
% \subsubsection{\hologo{pdfTeX}, \hologo{pdfLaTeX}}
%
%    \begin{macro}{\HoLogo@pdfTeX}
%    \begin{macrocode}
\def\HoLogo@pdfTeX#1{%
  \HOLOGO@mbox{%
    #1{p}{P}df\hologo{TeX}%
  }%
}
%    \end{macrocode}
%    \end{macro}
%    \begin{macro}{\HoLogoCs@pdfTeX}
%    \begin{macrocode}
\def\HoLogoCs@pdfTeX#1{#1{p}{P}dfTeX}
%    \end{macrocode}
%    \end{macro}
%    \begin{macro}{\HoLogoBkm@pdfTeX}
%    \begin{macrocode}
\def\HoLogoBkm@pdfTeX#1{%
  #1{p}{P}df\hologo{TeX}%
}
%    \end{macrocode}
%    \end{macro}
%    \begin{macro}{\HoLogoHtml@pdfTeX}
%    \begin{macrocode}
\let\HoLogoHtml@pdfTeX\HoLogo@pdfTeX
%    \end{macrocode}
%    \end{macro}
%
%    \begin{macro}{\HoLogo@pdfLaTeX}
%    \begin{macrocode}
\def\HoLogo@pdfLaTeX#1{%
  \HOLOGO@mbox{%
    #1{p}{P}df\hologo{LaTeX}%
  }%
}
%    \end{macrocode}
%    \end{macro}
%    \begin{macro}{\HoLogoCs@pdfLaTeX}
%    \begin{macrocode}
\def\HoLogoCs@pdfLaTeX#1{#1{p}{P}dfLaTeX}
%    \end{macrocode}
%    \end{macro}
%    \begin{macro}{\HoLogoBkm@pdfLaTeX}
%    \begin{macrocode}
\def\HoLogoBkm@pdfLaTeX#1{%
  #1{p}{P}df\hologo{LaTeX}%
}
%    \end{macrocode}
%    \end{macro}
%    \begin{macro}{\HoLogoHtml@pdfLaTeX}
%    \begin{macrocode}
\let\HoLogoHtml@pdfLaTeX\HoLogo@pdfLaTeX
%    \end{macrocode}
%    \end{macro}
%
% \subsubsection{\hologo{VTeX}}
%
%    \begin{macro}{\HoLogo@VTeX}
%    \begin{macrocode}
\def\HoLogo@VTeX#1{%
  \HOLOGO@mbox{%
    V\hologo{TeX}%
  }%
}
%    \end{macrocode}
%    \end{macro}
%    \begin{macro}{\HoLogoHtml@VTeX}
%    \begin{macrocode}
\let\HoLogoHtml@VTeX\HoLogo@VTeX
%    \end{macrocode}
%    \end{macro}
%
% \subsubsection{\hologo{AmS}, \dots}
%
%    Source: class \xclass{amsdtx}
%
%    \begin{macro}{\HoLogo@AmS}
%    \begin{macrocode}
\def\HoLogo@AmS#1{%
  \HoLogoFont@font{AmS}{sy}{%
    A%
    \kern-.1667em%
    \lower.5ex\hbox{M}%
    \kern-.125em%
    S%
  }%
}
%    \end{macrocode}
%    \end{macro}
%    \begin{macro}{\HoLogoBkm@AmS}
%    \begin{macrocode}
\def\HoLogoBkm@AmS#1{AmS}
%    \end{macrocode}
%    \end{macro}
%    \begin{macro}{\HoLogoHtml@AmS}
%    \begin{macrocode}
\def\HoLogoHtml@AmS#1{%
  \HoLogoCss@AmS
%  \HoLogoFont@font{AmS}{sy}{%
    \HOLOGO@Span{AmS}{%
      A%
      \HOLOGO@Span{M}{M}%
      S%
    }%
%   }%
}
%    \end{macrocode}
%    \end{macro}
%    \begin{macro}{\HoLogoCss@AmS}
%    \begin{macrocode}
\def\HoLogoCss@AmS{%
  \Css{%
    span.HoLogo-AmS span.HoLogo-M{%
      position:relative;%
      top:.5ex;%
      margin-left:-.1667em;%
      margin-right:-.125em;%
      text-decoration:none;%
    }%
  }%
  \global\let\HoLogoCss@AmS\relax
}
%    \end{macrocode}
%    \end{macro}
%
%    \begin{macro}{\HoLogo@AmSTeX}
%    \begin{macrocode}
\def\HoLogo@AmSTeX#1{%
  \hologo{AmS}%
  \HOLOGO@hyphen
  \hologo{TeX}%
}
%    \end{macrocode}
%    \end{macro}
%    \begin{macro}{\HoLogoBkm@AmSTeX}
%    \begin{macrocode}
\def\HoLogoBkm@AmSTeX#1{AmS-TeX}%
%    \end{macrocode}
%    \end{macro}
%    \begin{macro}{\HoLogoHtml@AmSTeX}
%    \begin{macrocode}
\let\HoLogoHtml@AmSTeX\HoLogo@AmSTeX
%    \end{macrocode}
%    \end{macro}
%
%    \begin{macro}{\HoLogo@AmSLaTeX}
%    \begin{macrocode}
\def\HoLogo@AmSLaTeX#1{%
  \hologo{AmS}%
  \HOLOGO@hyphen
  \hologo{LaTeX}%
}
%    \end{macrocode}
%    \end{macro}
%    \begin{macro}{\HoLogoBkm@AmSLaTeX}
%    \begin{macrocode}
\def\HoLogoBkm@AmSLaTeX#1{AmS-LaTeX}%
%    \end{macrocode}
%    \end{macro}
%    \begin{macro}{\HoLogoHtml@AmSLaTeX}
%    \begin{macrocode}
\let\HoLogoHtml@AmSLaTeX\HoLogo@AmSLaTeX
%    \end{macrocode}
%    \end{macro}
%
% \subsubsection{\hologo{BibTeX}}
%
%    \begin{macro}{\HoLogo@BibTeX@sc}
%    A definition of \hologo{BibTeX} is provided in
%    the documentation source for the manual of \hologo{BibTeX}
%    \cite{btxdoc}.
%\begin{quote}
%\begin{verbatim}
%\def\BibTeX{%
%  {%
%    \rm
%    B%
%    \kern-.05em%
%    {%
%      \sc
%      i%
%      \kern-.025em %
%      b%
%    }%
%    \kern-.08em
%    T%
%    \kern-.1667em%
%    \lower.7ex\hbox{E}%
%    \kern-.125em%
%    X%
%  }%
%}
%\end{verbatim}
%\end{quote}
%    \begin{macrocode}
\def\HoLogo@BibTeX@sc#1{%
  B%
  \kern-.05em%
  \HoLogoFont@font{BibTeX}{sc}{%
    i%
    \kern-.025em%
    b%
  }%
  \HOLOGO@discretionary
  \kern-.08em%
  \hologo{TeX}%
}
%    \end{macrocode}
%    \end{macro}
%    \begin{macro}{\HoLogoHtml@BibTeX@sc}
%    \begin{macrocode}
\def\HoLogoHtml@BibTeX@sc#1{%
  \HoLogoCss@BibTeX@sc
  \HOLOGO@Span{BibTeX-sc}{%
    B%
    \HOLOGO@Span{i}{i}%
    \HOLOGO@Span{b}{b}%
    \hologo{TeX}%
  }%
}
%    \end{macrocode}
%    \end{macro}
%    \begin{macro}{\HoLogoCss@BibTeX@sc}
%    \begin{macrocode}
\def\HoLogoCss@BibTeX@sc{%
  \Css{%
    span.HoLogo-BibTeX-sc span.HoLogo-i{%
      margin-left:-.05em;%
      margin-right:-.025em;%
      font-variant:small-caps;%
    }%
  }%
  \Css{%
    span.HoLogo-BibTeX-sc span.HoLogo-b{%
      margin-right:-.08em;%
      font-variant:small-caps;%
    }%
  }%
  \global\let\HoLogoCss@BibTeX@sc\relax
}
%    \end{macrocode}
%    \end{macro}
%
%    \begin{macro}{\HoLogo@BibTeX@sf}
%    Variant \xoption{sf} avoids trouble with unavailable
%    small caps fonts (e.g., bold versions of Computer Modern or
%    Latin Modern). The definition is taken from
%    package \xpackage{dtklogos} \cite{dtklogos}.
%\begin{quote}
%\begin{verbatim}
%\DeclareRobustCommand{\BibTeX}{%
%  B%
%  \kern-.05em%
%  \hbox{%
%    $\m@th$% %% force math size calculations
%    \csname S@\f@size\endcsname
%    \fontsize\sf@size\z@
%    \math@fontsfalse
%    \selectfont
%    I%
%    \kern-.025em%
%    B
%  }%
%  \kern-.08em%
%  \-%
%  \TeX
%}
%\end{verbatim}
%\end{quote}
%    \begin{macrocode}
\def\HoLogo@BibTeX@sf#1{%
  B%
  \kern-.05em%
  \HoLogoFont@font{BibTeX}{bibsf}{%
    I%
    \kern-.025em%
    B%
  }%
  \HOLOGO@discretionary
  \kern-.08em%
  \hologo{TeX}%
}
%    \end{macrocode}
%    \end{macro}
%    \begin{macro}{\HoLogoHtml@BibTeX@sf}
%    \begin{macrocode}
\def\HoLogoHtml@BibTeX@sf#1{%
  \HoLogoCss@BibTeX@sf
  \HOLOGO@Span{BibTeX-sf}{%
    B%
    \HoLogoFont@font{BibTeX}{bibsf}{%
      \HOLOGO@Span{i}{I}%
      B%
    }%
    \hologo{TeX}%
  }%
}
%    \end{macrocode}
%    \end{macro}
%    \begin{macro}{\HoLogoCss@BibTeX@sf}
%    \begin{macrocode}
\def\HoLogoCss@BibTeX@sf{%
  \Css{%
    span.HoLogo-BibTeX-sf span.HoLogo-i{%
      margin-left:-.05em;%
      margin-right:-.025em;%
    }%
  }%
  \Css{%
    span.HoLogo-BibTeX-sf span.HoLogo-TeX{%
      margin-left:-.08em;%
    }%
  }%
  \global\let\HoLogoCss@BibTeX@sf\relax
}
%    \end{macrocode}
%    \end{macro}
%
%    \begin{macro}{\HoLogo@BibTeX}
%    \begin{macrocode}
\def\HoLogo@BibTeX{\HoLogo@BibTeX@sf}
%    \end{macrocode}
%    \end{macro}
%    \begin{macro}{\HoLogoHtml@BibTeX}
%    \begin{macrocode}
\def\HoLogoHtml@BibTeX{\HoLogoHtml@BibTeX@sf}
%    \end{macrocode}
%    \end{macro}
%
% \subsubsection{\hologo{BibTeX8}}
%
%    \begin{macro}{\HoLogo@BibTeX8}
%    \begin{macrocode}
\expandafter\def\csname HoLogo@BibTeX8\endcsname#1{%
  \hologo{BibTeX}%
  8%
}
%    \end{macrocode}
%    \end{macro}
%
%    \begin{macro}{\HoLogoBkm@BibTeX8}
%    \begin{macrocode}
\expandafter\def\csname HoLogoBkm@BibTeX8\endcsname#1{%
  \hologo{BibTeX}%
  8%
}
%    \end{macrocode}
%    \end{macro}
%    \begin{macro}{\HoLogoHtml@BibTeX8}
%    \begin{macrocode}
\expandafter
\let\csname HoLogoHtml@BibTeX8\expandafter\endcsname
\csname HoLogo@BibTeX8\endcsname
%    \end{macrocode}
%    \end{macro}
%
% \subsubsection{\hologo{ConTeXt}}
%
%    \begin{macro}{\HoLogo@ConTeXt@simple}
%    \begin{macrocode}
\def\HoLogo@ConTeXt@simple#1{%
  \HOLOGO@mbox{Con}%
  \HOLOGO@discretionary
  \HOLOGO@mbox{\hologo{TeX}t}%
}
%    \end{macrocode}
%    \end{macro}
%    \begin{macro}{\HoLogoHtml@ConTeXt@simple}
%    \begin{macrocode}
\let\HoLogoHtml@ConTeXt@simple\HoLogo@ConTeXt@simple
%    \end{macrocode}
%    \end{macro}
%
%    \begin{macro}{\HoLogo@ConTeXt@narrow}
%    This definition of logo \hologo{ConTeXt} with variant \xoption{narrow}
%    comes from TUGboat's class \xclass{ltugboat} (version 2010/11/15 v2.8).
%    \begin{macrocode}
\def\HoLogo@ConTeXt@narrow#1{%
  \HOLOGO@mbox{C\kern-.0333emon}%
  \HOLOGO@discretionary
  \kern-.0667em%
  \HOLOGO@mbox{\hologo{TeX}\kern-.0333emt}%
}
%    \end{macrocode}
%    \end{macro}
%    \begin{macro}{\HoLogoHtml@ConTeXt@narrow}
%    \begin{macrocode}
\def\HoLogoHtml@ConTeXt@narrow#1{%
  \HoLogoCss@ConTeXt@narrow
  \HOLOGO@Span{ConTeXt-narrow}{%
    \HOLOGO@Span{C}{C}%
    on%
    \hologo{TeX}%
    t%
  }%
}
%    \end{macrocode}
%    \end{macro}
%    \begin{macro}{\HoLogoCss@ConTeXt@narrow}
%    \begin{macrocode}
\def\HoLogoCss@ConTeXt@narrow{%
  \Css{%
    span.HoLogo-ConTeXt-narrow span.HoLogo-C{%
      margin-left:-.0333em;%
    }%
  }%
  \Css{%
    span.HoLogo-ConTeXt-narrow span.HoLogo-TeX{%
      margin-left:-.0667em;%
      margin-right:-.0333em;%
    }%
  }%
  \global\let\HoLogoCss@ConTeXt@narrow\relax
}
%    \end{macrocode}
%    \end{macro}
%
%    \begin{macro}{\HoLogo@ConTeXt}
%    \begin{macrocode}
\def\HoLogo@ConTeXt{\HoLogo@ConTeXt@narrow}
%    \end{macrocode}
%    \end{macro}
%    \begin{macro}{\HoLogoHtml@ConTeXt}
%    \begin{macrocode}
\def\HoLogoHtml@ConTeXt{\HoLogoHtml@ConTeXt@narrow}
%    \end{macrocode}
%    \end{macro}
%
% \subsubsection{\hologo{emTeX}}
%
%    \begin{macro}{\HoLogo@emTeX}
%    \begin{macrocode}
\def\HoLogo@emTeX#1{%
  \HOLOGO@mbox{#1{e}{E}m}%
  \HOLOGO@discretionary
  \hologo{TeX}%
}
%    \end{macrocode}
%    \end{macro}
%    \begin{macro}{\HoLogoCs@emTeX}
%    \begin{macrocode}
\def\HoLogoCs@emTeX#1{#1{e}{E}mTeX}%
%    \end{macrocode}
%    \end{macro}
%    \begin{macro}{\HoLogoBkm@emTeX}
%    \begin{macrocode}
\def\HoLogoBkm@emTeX#1{%
  #1{e}{E}m\hologo{TeX}%
}
%    \end{macrocode}
%    \end{macro}
%    \begin{macro}{\HoLogoHtml@emTeX}
%    \begin{macrocode}
\let\HoLogoHtml@emTeX\HoLogo@emTeX
%    \end{macrocode}
%    \end{macro}
%
% \subsubsection{\hologo{ExTeX}}
%
%    \begin{macro}{\HoLogo@ExTeX}
%    The definition is taken from the FAQ of the
%    project \hologo{ExTeX}
%    \cite{ExTeX-FAQ}.
%\begin{quote}
%\begin{verbatim}
%\def\ExTeX{%
%  \textrm{% Logo always with serifs
%    \ensuremath{%
%      \textstyle
%      \varepsilon_{%
%        \kern-0.15em%
%        \mathcal{X}%
%      }%
%    }%
%    \kern-.15em%
%    \TeX
%  }%
%}
%\end{verbatim}
%\end{quote}
%    \begin{macrocode}
\def\HoLogo@ExTeX#1{%
  \HoLogoFont@font{ExTeX}{rm}{%
    \ltx@mbox{%
      \HOLOGO@MathSetup
      $%
        \textstyle
        \varepsilon_{%
          \kern-0.15em%
          \HoLogoFont@font{ExTeX}{sy}{X}%
        }%
      $%
    }%
    \HOLOGO@discretionary
    \kern-.15em%
    \hologo{TeX}%
  }%
}
%    \end{macrocode}
%    \end{macro}
%    \begin{macro}{\HoLogoHtml@ExTeX}
%    \begin{macrocode}
\def\HoLogoHtml@ExTeX#1{%
  \HoLogoCss@ExTeX
  \HoLogoFont@font{ExTeX}{rm}{%
    \HOLOGO@Span{ExTeX}{%
      \ltx@mbox{%
        \HOLOGO@MathSetup
        $\textstyle\varepsilon$%
        \HOLOGO@Span{X}{$\textstyle\chi$}%
        \hologo{TeX}%
      }%
    }%
  }%
}
%    \end{macrocode}
%    \end{macro}
%    \begin{macro}{\HoLogoBkm@ExTeX}
%    \begin{macrocode}
\def\HoLogoBkm@ExTeX#1{%
  \HOLOGO@PdfdocUnicode{#1{e}{E}x}{\textepsilon\textchi}%
  \hologo{TeX}%
}
%    \end{macrocode}
%    \end{macro}
%    \begin{macro}{\HoLogoCss@ExTeX}
%    \begin{macrocode}
\def\HoLogoCss@ExTeX{%
  \Css{%
    span.HoLogo-ExTeX{%
      font-family:serif;%
    }%
  }%
  \Css{%
    span.HoLogo-ExTeX span.HoLogo-TeX{%
      margin-left:-.15em;%
    }%
  }%
  \global\let\HoLogoCss@ExTeX\relax
}
%    \end{macrocode}
%    \end{macro}
%
% \subsubsection{\hologo{MiKTeX}}
%
%    \begin{macro}{\HoLogo@MiKTeX}
%    \begin{macrocode}
\def\HoLogo@MiKTeX#1{%
  \HOLOGO@mbox{MiK}%
  \HOLOGO@discretionary
  \hologo{TeX}%
}
%    \end{macrocode}
%    \end{macro}
%    \begin{macro}{\HoLogoHtml@MiKTeX}
%    \begin{macrocode}
\let\HoLogoHtml@MiKTeX\HoLogo@MiKTeX
%    \end{macrocode}
%    \end{macro}
%
% \subsubsection{\hologo{OzTeX} and friends}
%
%    Source: \hologo{OzTeX} FAQ \cite{OzTeX}:
%    \begin{quote}
%      |\def\OzTeX{O\kern-.03em z\kern-.15em\TeX}|\\
%      (There is no kerning in OzMF, OzMP and OzTtH.)
%    \end{quote}
%
%    \begin{macro}{\HoLogo@OzTeX}
%    \begin{macrocode}
\def\HoLogo@OzTeX#1{%
  O%
  \kern-.03em %
  z%
  \kern-.15em %
  \hologo{TeX}%
}
%    \end{macrocode}
%    \end{macro}
%    \begin{macro}{\HoLogoHtml@OzTeX}
%    \begin{macrocode}
\def\HoLogoHtml@OzTeX#1{%
  \HoLogoCss@OzTeX
  \HOLOGO@Span{OzTeX}{%
    O%
    \HOLOGO@Span{z}{z}%
    \hologo{TeX}%
  }%
}
%    \end{macrocode}
%    \end{macro}
%    \begin{macro}{\HoLogoCss@OzTeX}
%    \begin{macrocode}
\def\HoLogoCss@OzTeX{%
  \Css{%
    span.HoLogo-OzTeX span.HoLogo-z{%
      margin-left:-.03em;%
      margin-right:-.15em;%
    }%
  }%
  \global\let\HoLogoCss@OzTeX\relax
}
%    \end{macrocode}
%    \end{macro}
%
%    \begin{macro}{\HoLogo@OzMF}
%    \begin{macrocode}
\def\HoLogo@OzMF#1{%
  \HOLOGO@mbox{OzMF}%
}
%    \end{macrocode}
%    \end{macro}
%    \begin{macro}{\HoLogo@OzMP}
%    \begin{macrocode}
\def\HoLogo@OzMP#1{%
  \HOLOGO@mbox{OzMP}%
}
%    \end{macrocode}
%    \end{macro}
%    \begin{macro}{\HoLogo@OzTtH}
%    \begin{macrocode}
\def\HoLogo@OzTtH#1{%
  \HOLOGO@mbox{OzTtH}%
}
%    \end{macrocode}
%    \end{macro}
%
% \subsubsection{\hologo{PCTeX}}
%
%    \begin{macro}{\HoLogo@PCTeX}
%    \begin{macrocode}
\def\HoLogo@PCTeX#1{%
  \HOLOGO@mbox{PC}%
  \hologo{TeX}%
}
%    \end{macrocode}
%    \end{macro}
%    \begin{macro}{\HoLogoHtml@PCTeX}
%    \begin{macrocode}
\let\HoLogoHtml@PCTeX\HoLogo@PCTeX
%    \end{macrocode}
%    \end{macro}
%
% \subsubsection{\hologo{PiCTeX}}
%
%    The original definitions from \xfile{pictex.tex} \cite{PiCTeX}:
%\begin{quote}
%\begin{verbatim}
%\def\PiC{%
%  P%
%  \kern-.12em%
%  \lower.5ex\hbox{I}%
%  \kern-.075em%
%  C%
%}
%\def\PiCTeX{%
%  \PiC
%  \kern-.11em%
%  \TeX
%}
%\end{verbatim}
%\end{quote}
%
%    \begin{macro}{\HoLogo@PiC}
%    \begin{macrocode}
\def\HoLogo@PiC#1{%
  P%
  \kern-.12em%
  \lower.5ex\hbox{I}%
  \kern-.075em%
  C%
  \HOLOGO@SpaceFactor
}
%    \end{macrocode}
%    \end{macro}
%    \begin{macro}{\HoLogoHtml@PiC}
%    \begin{macrocode}
\def\HoLogoHtml@PiC#1{%
  \HoLogoCss@PiC
  \HOLOGO@Span{PiC}{%
    P%
    \HOLOGO@Span{i}{I}%
    C%
  }%
}
%    \end{macrocode}
%    \end{macro}
%    \begin{macro}{\HoLogoCss@PiC}
%    \begin{macrocode}
\def\HoLogoCss@PiC{%
  \Css{%
    span.HoLogo-PiC span.HoLogo-i{%
      position:relative;%
      top:.5ex;%
      margin-left:-.12em;%
      margin-right:-.075em;%
      text-decoration:none;%
    }%
  }%
  \global\let\HoLogoCss@PiC\relax
}
%    \end{macrocode}
%    \end{macro}
%
%    \begin{macro}{\HoLogo@PiCTeX}
%    \begin{macrocode}
\def\HoLogo@PiCTeX#1{%
  \hologo{PiC}%
  \HOLOGO@discretionary
  \kern-.11em%
  \hologo{TeX}%
}
%    \end{macrocode}
%    \end{macro}
%    \begin{macro}{\HoLogoHtml@PiCTeX}
%    \begin{macrocode}
\def\HoLogoHtml@PiCTeX#1{%
  \HoLogoCss@PiCTeX
  \HOLOGO@Span{PiCTeX}{%
    \hologo{PiC}%
    \hologo{TeX}%
  }%
}
%    \end{macrocode}
%    \end{macro}
%    \begin{macro}{\HoLogoCss@PiCTeX}
%    \begin{macrocode}
\def\HoLogoCss@PiCTeX{%
  \Css{%
    span.HoLogo-PiCTeX span.HoLogo-PiC{%
      margin-right:-.11em;%
    }%
  }%
  \global\let\HoLogoCss@PiCTeX\relax
}
%    \end{macrocode}
%    \end{macro}
%
% \subsubsection{\hologo{teTeX}}
%
%    \begin{macro}{\HoLogo@teTeX}
%    \begin{macrocode}
\def\HoLogo@teTeX#1{%
  \HOLOGO@mbox{#1{t}{T}e}%
  \HOLOGO@discretionary
  \hologo{TeX}%
}
%    \end{macrocode}
%    \end{macro}
%    \begin{macro}{\HoLogoCs@teTeX}
%    \begin{macrocode}
\def\HoLogoCs@teTeX#1{#1{t}{T}dfTeX}
%    \end{macrocode}
%    \end{macro}
%    \begin{macro}{\HoLogoBkm@teTeX}
%    \begin{macrocode}
\def\HoLogoBkm@teTeX#1{%
  #1{t}{T}e\hologo{TeX}%
}
%    \end{macrocode}
%    \end{macro}
%    \begin{macro}{\HoLogoHtml@teTeX}
%    \begin{macrocode}
\let\HoLogoHtml@teTeX\HoLogo@teTeX
%    \end{macrocode}
%    \end{macro}
%
% \subsubsection{\hologo{TeX4ht}}
%
%    \begin{macro}{\HoLogo@TeX4ht}
%    \begin{macrocode}
\expandafter\def\csname HoLogo@TeX4ht\endcsname#1{%
  \HOLOGO@mbox{\hologo{TeX}4ht}%
}
%    \end{macrocode}
%    \end{macro}
%    \begin{macro}{\HoLogoHtml@TeX4ht}
%    \begin{macrocode}
\expandafter
\let\csname HoLogoHtml@TeX4ht\expandafter\endcsname
\csname HoLogo@TeX4ht\endcsname
%    \end{macrocode}
%    \end{macro}
%
%
% \subsubsection{\hologo{SageTeX}}
%
%    \begin{macro}{\HoLogo@SageTeX}
%    \begin{macrocode}
\def\HoLogo@SageTeX#1{%
  \HOLOGO@mbox{Sage}%
  \HOLOGO@discretionary
  \HOLOGO@NegativeKerning{eT,oT,To}%
  \hologo{TeX}%
}
%    \end{macrocode}
%    \end{macro}
%    \begin{macro}{\HoLogoHtml@SageTeX}
%    \begin{macrocode}
\let\HoLogoHtml@SageTeX\HoLogo@SageTeX
%    \end{macrocode}
%    \end{macro}
%
% \subsection{\hologo{METAFONT} and friends}
%
%    \begin{macro}{\HoLogo@METAFONT}
%    \begin{macrocode}
\def\HoLogo@METAFONT#1{%
  \HoLogoFont@font{METAFONT}{logo}{%
    \HOLOGO@mbox{META}%
    \HOLOGO@discretionary
    \HOLOGO@mbox{FONT}%
  }%
}
%    \end{macrocode}
%    \end{macro}
%
%    \begin{macro}{\HoLogo@METAPOST}
%    \begin{macrocode}
\def\HoLogo@METAPOST#1{%
  \HoLogoFont@font{METAPOST}{logo}{%
    \HOLOGO@mbox{META}%
    \HOLOGO@discretionary
    \HOLOGO@mbox{POST}%
  }%
}
%    \end{macrocode}
%    \end{macro}
%
%    \begin{macro}{\HoLogo@MetaFun}
%    \begin{macrocode}
\def\HoLogo@MetaFun#1{%
  \HOLOGO@mbox{Meta}%
  \HOLOGO@discretionary
  \HOLOGO@mbox{Fun}%
}
%    \end{macrocode}
%    \end{macro}
%
%    \begin{macro}{\HoLogo@MetaPost}
%    \begin{macrocode}
\def\HoLogo@MetaPost#1{%
  \HOLOGO@mbox{Meta}%
  \HOLOGO@discretionary
  \HOLOGO@mbox{Post}%
}
%    \end{macrocode}
%    \end{macro}
%
% \subsection{Others}
%
% \subsubsection{\hologo{biber}}
%
%    \begin{macro}{\HoLogo@biber}
%    \begin{macrocode}
\def\HoLogo@biber#1{%
  \HOLOGO@mbox{#1{b}{B}i}%
  \HOLOGO@discretionary
  \HOLOGO@mbox{ber}%
}
%    \end{macrocode}
%    \end{macro}
%    \begin{macro}{\HoLogoCs@biber}
%    \begin{macrocode}
\def\HoLogoCs@biber#1{#1{b}{B}iber}
%    \end{macrocode}
%    \end{macro}
%    \begin{macro}{\HoLogoBkm@biber}
%    \begin{macrocode}
\def\HoLogoBkm@biber#1{%
  #1{b}{B}iber%
}
%    \end{macrocode}
%    \end{macro}
%    \begin{macro}{\HoLogoHtml@biber}
%    \begin{macrocode}
\let\HoLogoHtml@biber\HoLogo@biber
%    \end{macrocode}
%    \end{macro}
%
% \subsubsection{\hologo{KOMAScript}}
%
%    \begin{macro}{\HoLogo@KOMAScript}
%    The definition for \hologo{KOMAScript} is taken
%    from \hologo{KOMAScript} (\xfile{scrlogo.dtx}, reformatted) \cite{scrlogo}:
%\begin{quote}
%\begin{verbatim}
%\@ifundefined{KOMAScript}{%
%  \DeclareRobustCommand{\KOMAScript}{%
%    \textsf{%
%      K\kern.05em O\kern.05emM\kern.05em A%
%      \kern.1em-\kern.1em %
%      Script%
%    }%
%  }%
%}{}
%\end{verbatim}
%\end{quote}
%    \begin{macrocode}
\def\HoLogo@KOMAScript#1{%
  \HoLogoFont@font{KOMAScript}{sf}{%
    \HOLOGO@mbox{%
      K\kern.05em%
      O\kern.05em%
      M\kern.05em%
      A%
    }%
    \kern.1em%
    \HOLOGO@hyphen
    \kern.1em%
    \HOLOGO@mbox{Script}%
  }%
}
%    \end{macrocode}
%    \end{macro}
%    \begin{macro}{\HoLogoBkm@KOMAScript}
%    \begin{macrocode}
\def\HoLogoBkm@KOMAScript#1{%
  KOMA-Script%
}
%    \end{macrocode}
%    \end{macro}
%    \begin{macro}{\HoLogoHtml@KOMAScript}
%    \begin{macrocode}
\def\HoLogoHtml@KOMAScript#1{%
  \HoLogoCss@KOMAScript
  \HoLogoFont@font{KOMAScript}{sf}{%
    \HOLOGO@Span{KOMAScript}{%
      K%
      \HOLOGO@Span{O}{O}%
      M%
      \HOLOGO@Span{A}{A}%
      \HOLOGO@Span{hyphen}{-}%
      Script%
    }%
  }%
}
%    \end{macrocode}
%    \end{macro}
%    \begin{macro}{\HoLogoCss@KOMAScript}
%    \begin{macrocode}
\def\HoLogoCss@KOMAScript{%
  \Css{%
    span.HoLogo-KOMAScript{%
      font-family:sans-serif;%
    }%
  }%
  \Css{%
    span.HoLogo-KOMAScript span.HoLogo-O{%
      padding-left:.05em;%
      padding-right:.05em;%
    }%
  }%
  \Css{%
    span.HoLogo-KOMAScript span.HoLogo-A{%
      padding-left:.05em;%
    }%
  }%
  \Css{%
    span.HoLogo-KOMAScript span.HoLogo-hyphen{%
      padding-left:.1em;%
      padding-right:.1em;%
    }%
  }%
  \global\let\HoLogoCss@KOMAScript\relax
}
%    \end{macrocode}
%    \end{macro}
%
% \subsubsection{\hologo{LyX}}
%
%    \begin{macro}{\HoLogo@LyX}
%    The definition is taken from the documentation source files
%    of \hologo{LyX}, \xfile{Intro.lyx} \cite{LyX}:
%\begin{quote}
%\begin{verbatim}
%\def\LyX{%
%  \texorpdfstring{%
%    L\kern-.1667em\lower.25em\hbox{Y}\kern-.125emX\@%
%  }{%
%    LyX%
%  }%
%}
%\end{verbatim}
%\end{quote}
%    \begin{macrocode}
\def\HoLogo@LyX#1{%
  L%
  \kern-.1667em%
  \lower.25em\hbox{Y}%
  \kern-.125em%
  X%
  \HOLOGO@SpaceFactor
}
%    \end{macrocode}
%    \end{macro}
%    \begin{macro}{\HoLogoHtml@LyX}
%    \begin{macrocode}
\def\HoLogoHtml@LyX#1{%
  \HoLogoCss@LyX
  \HOLOGO@Span{LyX}{%
    L%
    \HOLOGO@Span{y}{Y}%
    X%
  }%
}
%    \end{macrocode}
%    \end{macro}
%    \begin{macro}{\HoLogoCss@LyX}
%    \begin{macrocode}
\def\HoLogoCss@LyX{%
  \Css{%
    span.HoLogo-LyX span.HoLogo-y{%
      position:relative;%
      top:.25em;%
      margin-left:-.1667em;%
      margin-right:-.125em;%
      text-decoration:none;%
    }%
  }%
  \global\let\HoLogoCss@LyX\relax
}
%    \end{macrocode}
%    \end{macro}
%
% \subsubsection{\hologo{NTS}}
%
%    \begin{macro}{\HoLogo@NTS}
%    Definition for \hologo{NTS} can be found in
%    package \xpackage{etex\textunderscore man} for the \hologo{eTeX} manual \cite{etexman}
%    and in package \xpackage{dtklogos} \cite{dtklogos}:
%\begin{quote}
%\begin{verbatim}
%\def\NTS{%
%  \leavevmode
%  \hbox{%
%    $%
%      \cal N%
%      \kern-0.35em%
%      \lower0.5ex\hbox{$\cal T$}%
%      \kern-0.2em%
%      S%
%    $%
%  }%
%}
%\end{verbatim}
%\end{quote}
%    \begin{macrocode}
\def\HoLogo@NTS#1{%
  \HoLogoFont@font{NTS}{sy}{%
    N\/%
    \kern-.35em%
    \lower.5ex\hbox{T\/}%
    \kern-.2em%
    S\/%
  }%
  \HOLOGO@SpaceFactor
}
%    \end{macrocode}
%    \end{macro}
%
% \subsubsection{\Hologo{TTH} (\hologo{TeX} to HTML translator)}
%
%    Source: \url{http://hutchinson.belmont.ma.us/tth/}
%    In the HTML source the second `T' is printed as subscript.
%\begin{quote}
%\begin{verbatim}
%T<sub>T</sub>H
%\end{verbatim}
%\end{quote}
%    \begin{macro}{\HoLogo@TTH}
%    \begin{macrocode}
\def\HoLogo@TTH#1{%
  \ltx@mbox{%
    T\HOLOGO@SubScript{T}H%
  }%
  \HOLOGO@SpaceFactor
}
%    \end{macrocode}
%    \end{macro}
%
%    \begin{macro}{\HoLogoHtml@TTH}
%    \begin{macrocode}
\def\HoLogoHtml@TTH#1{%
  T\HCode{<sub>}T\HCode{</sub>}H%
}
%    \end{macrocode}
%    \end{macro}
%
% \subsubsection{\Hologo{HanTheThanh}}
%
%    Partial source: Package \xpackage{dtklogos}.
%    The double accent is U+1EBF (latin small letter e with circumflex
%    and acute).
%    \begin{macro}{\HoLogo@HanTheThanh}
%    \begin{macrocode}
\def\HoLogo@HanTheThanh#1{%
  \ltx@mbox{H\`an}%
  \HOLOGO@space
  \ltx@mbox{%
    Th%
    \HOLOGO@IfCharExists{"1EBF}{%
      \char"1EBF\relax
    }{%
      \^e\hbox to 0pt{\hss\raise .5ex\hbox{\'{}}}%
    }%
  }%
  \HOLOGO@space
  \ltx@mbox{Th\`anh}%
}
%    \end{macrocode}
%    \end{macro}
%    \begin{macro}{\HoLogoBkm@HanTheThanh}
%    \begin{macrocode}
\def\HoLogoBkm@HanTheThanh#1{%
  H\`an %
  Th\HOLOGO@PdfdocUnicode{\^e}{\9036\277} %
  Th\`anh%
}
%    \end{macrocode}
%    \end{macro}
%    \begin{macro}{\HoLogoHtml@HanTheThanh}
%    \begin{macrocode}
\def\HoLogoHtml@HanTheThanh#1{%
  H\`an %
  Th\HCode{&\ltx@hashchar x1ebf;} %
  Th\`anh%
}
%    \end{macrocode}
%    \end{macro}
%
% \subsection{Driver detection}
%
%    \begin{macrocode}
\HOLOGO@IfExists\InputIfFileExists{%
  \InputIfFileExists{hologo.cfg}{}{}%
}{%
  \ltx@IfUndefined{pdf@filesize}{%
    \def\HOLOGO@InputIfExists{%
      \openin\HOLOGO@temp=hologo.cfg\relax
      \ifeof\HOLOGO@temp
        \closein\HOLOGO@temp
      \else
        \closein\HOLOGO@temp
        \begingroup
          \def\x{LaTeX2e}%
        \expandafter\endgroup
        \ifx\fmtname\x
          \input{hologo.cfg}%
        \else
          \input hologo.cfg\relax
        \fi
      \fi
    }%
    \ltx@IfUndefined{newread}{%
      \chardef\HOLOGO@temp=15 %
      \def\HOLOGO@CheckRead{%
        \ifeof\HOLOGO@temp
          \HOLOGO@InputIfExists
        \else
          \ifcase\HOLOGO@temp
            \@PackageWarningNoLine{hologo}{%
              Configuration file ignored, because\MessageBreak
              a free read register could not be found%
            }%
          \else
            \begingroup
              \count\ltx@cclv=\HOLOGO@temp
              \advance\ltx@cclv by \ltx@minusone
              \edef\x{\endgroup
                \chardef\noexpand\HOLOGO@temp=\the\count\ltx@cclv
                \relax
              }%
            \x
          \fi
        \fi
      }%
    }{%
      \csname newread\endcsname\HOLOGO@temp
      \HOLOGO@InputIfExists
    }%
  }{%
    \edef\HOLOGO@temp{\pdf@filesize{hologo.cfg}}%
    \ifx\HOLOGO@temp\ltx@empty
    \else
      \ifnum\HOLOGO@temp>0 %
        \begingroup
          \def\x{LaTeX2e}%
        \expandafter\endgroup
        \ifx\fmtname\x
          \input{hologo.cfg}%
        \else
          \input hologo.cfg\relax
        \fi
      \else
        \@PackageInfoNoLine{hologo}{%
          Empty configuration file `hologo.cfg' ignored%
        }%
      \fi
    \fi
  }%
}
%    \end{macrocode}
%
%    \begin{macrocode}
\def\HOLOGO@temp#1#2{%
  \kv@define@key{HoLogoDriver}{#1}[]{%
    \begingroup
      \def\HOLOGO@temp{##1}%
      \ltx@onelevel@sanitize\HOLOGO@temp
      \ifx\HOLOGO@temp\ltx@empty
      \else
        \@PackageError{hologo}{%
          Value (\HOLOGO@temp) not permitted for option `#1'%
        }%
        \@ehc
      \fi
    \endgroup
    \def\hologoDriver{#2}%
  }%
}%
\def\HOLOGO@@temp#1#2{%
  \ifx\kv@value\relax
    \HOLOGO@temp{#1}{#1}%
  \else
    \HOLOGO@temp{#1}{#2}%
  \fi
}%
\kv@parse@normalized{%
  pdftex,%
  luatex=pdftex,%
  dvipdfm,%
  dvipdfmx=dvipdfm,%
  dvips,%
  dvipsone=dvips,%
  xdvi=dvips,%
  xetex,%
  vtex,%
}\HOLOGO@@temp
%    \end{macrocode}
%
%    \begin{macrocode}
\kv@define@key{HoLogoDriver}{driverfallback}{%
  \def\HOLOGO@DriverFallback{#1}%
}
%    \end{macrocode}
%
%    \begin{macro}{\HOLOGO@DriverFallback}
%    \begin{macrocode}
\def\HOLOGO@DriverFallback{dvips}
%    \end{macrocode}
%    \end{macro}
%
%    \begin{macro}{\hologoDriverSetup}
%    \begin{macrocode}
\def\hologoDriverSetup{%
  \let\hologoDriver\ltx@undefined
  \HOLOGO@DriverSetup
}
%    \end{macrocode}
%    \end{macro}
%
%    \begin{macro}{\HOLOGO@DriverSetup}
%    \begin{macrocode}
\def\HOLOGO@DriverSetup#1{%
  \kvsetkeys{HoLogoDriver}{#1}%
  \HOLOGO@CheckDriver
  \ltx@ifundefined{hologoDriver}{%
    \begingroup
    \edef\x{\endgroup
      \noexpand\kvsetkeys{HoLogoDriver}{\HOLOGO@DriverFallback}%
    }\x
  }{}%
  \@PackageInfoNoLine{hologo}{Using driver `\hologoDriver'}%
}
%    \end{macrocode}
%    \end{macro}
%
%    \begin{macro}{\HOLOGO@CheckDriver}
%    \begin{macrocode}
\def\HOLOGO@CheckDriver{%
  \ifpdf
    \def\hologoDriver{pdftex}%
    \let\HOLOGO@pdfliteral\pdfliteral
    \ifluatex
      \ifx\pdfextension\@undefined\else
        \protected\def\pdfliteral{\pdfextension literal}%
        \let\HOLOGO@pdfliteral\pdfliteral
      \fi
      \ltx@IfUndefined{HOLOGO@pdfliteral}{%
        \ifnum\luatexversion<36 %
        \else
          \begingroup
            \let\HOLOGO@temp\endgroup
            \ifcase0%
                \directlua{%
                  if tex.enableprimitives then %
                    tex.enableprimitives('HOLOGO@', {'pdfliteral'})%
                  else %
                    tex.print('1')%
                  end%
                }%
                \ifx\HOLOGO@pdfliteral\@undefined 1\fi%
                \relax%
              \endgroup
              \let\HOLOGO@temp\relax
              \global\let\HOLOGO@pdfliteral\HOLOGO@pdfliteral
            \fi%
          \HOLOGO@temp
        \fi
      }{}%
    \fi
    \ltx@IfUndefined{HOLOGO@pdfliteral}{%
      \@PackageWarningNoLine{hologo}{%
        Cannot find \string\pdfliteral
      }%
    }{}%
  \else
    \ifxetex
      \def\hologoDriver{xetex}%
    \else
      \ifvtex
        \def\hologoDriver{vtex}%
      \fi
    \fi
  \fi
}
%    \end{macrocode}
%    \end{macro}
%
%    \begin{macro}{\HOLOGO@WarningUnsupportedDriver}
%    \begin{macrocode}
\def\HOLOGO@WarningUnsupportedDriver#1{%
  \@PackageWarningNoLine{hologo}{%
    Logo `#1' needs driver specific macros,\MessageBreak
    but driver `\hologoDriver' is not supported.\MessageBreak
    Use a different driver or\MessageBreak
    load package `graphics' or `pgf'%
  }%
}
%    \end{macrocode}
%    \end{macro}
%
% \subsubsection{Reflect box macros}
%
%    Skip driver part if not needed.
%    \begin{macrocode}
\ltx@IfUndefined{reflectbox}{}{%
  \ltx@IfUndefined{rotatebox}{}{%
    \HOLOGO@AtEnd
  }%
}
\ltx@IfUndefined{pgftext}{}{%
  \HOLOGO@AtEnd
}
\ltx@IfUndefined{psscalebox}{}{%
  \HOLOGO@AtEnd
}
%    \end{macrocode}
%
%    \begin{macrocode}
\def\HOLOGO@temp{LaTeX2e}
\ifx\fmtname\HOLOGO@temp
  \RequirePackage{kvoptions}[2011/06/30]%
  \ProcessKeyvalOptions{HoLogoDriver}%
\fi
\HOLOGO@DriverSetup{}
%    \end{macrocode}
%
%    \begin{macro}{\HOLOGO@ReflectBox}
%    \begin{macrocode}
\def\HOLOGO@ReflectBox#1{%
  \begingroup
    \setbox\ltx@zero\hbox{\begingroup#1\endgroup}%
    \setbox\ltx@two\hbox{%
      \kern\wd\ltx@zero
      \csname HOLOGO@ScaleBox@\hologoDriver\endcsname{-1}{1}{%
        \hbox to 0pt{\copy\ltx@zero\hss}%
      }%
    }%
    \wd\ltx@two=\wd\ltx@zero
    \box\ltx@two
  \endgroup
}
%    \end{macrocode}
%    \end{macro}
%
%    \begin{macro}{\HOLOGO@PointReflectBox}
%    \begin{macrocode}
\def\HOLOGO@PointReflectBox#1{%
  \begingroup
    \setbox\ltx@zero\hbox{\begingroup#1\endgroup}%
    \setbox\ltx@two\hbox{%
      \kern\wd\ltx@zero
      \raise\ht\ltx@zero\hbox{%
        \csname HOLOGO@ScaleBox@\hologoDriver\endcsname{-1}{-1}{%
          \hbox to 0pt{\copy\ltx@zero\hss}%
        }%
      }%
    }%
    \wd\ltx@two=\wd\ltx@zero
    \box\ltx@two
  \endgroup
}
%    \end{macrocode}
%    \end{macro}
%
%    We must define all variants because of dynamic driver setup.
%    \begin{macrocode}
\def\HOLOGO@temp#1#2{#2}
%    \end{macrocode}
%
%    \begin{macro}{\HOLOGO@ScaleBox@pdftex}
%    \begin{macrocode}
\HOLOGO@temp{pdftex}{%
  \def\HOLOGO@ScaleBox@pdftex#1#2#3{%
    \HOLOGO@pdfliteral{%
      q #1 0 0 #2 0 0 cm%
    }%
    #3%
    \HOLOGO@pdfliteral{%
      Q%
    }%
  }%
}
%    \end{macrocode}
%    \end{macro}
%    \begin{macro}{\HOLOGO@ScaleBox@dvips}
%    \begin{macrocode}
\HOLOGO@temp{dvips}{%
  \def\HOLOGO@ScaleBox@dvips#1#2#3{%
    \special{ps:%
      gsave %
      currentpoint %
      currentpoint translate %
      #1 #2 scale %
      neg exch neg exch translate%
    }%
    #3%
    \special{ps:%
      currentpoint %
      grestore %
      moveto%
    }%
  }%
}
%    \end{macrocode}
%    \end{macro}
%    \begin{macro}{\HOLOGO@ScaleBox@dvipdfm}
%    \begin{macrocode}
\HOLOGO@temp{dvipdfm}{%
  \let\HOLOGO@ScaleBox@dvipdfm\HOLOGO@ScaleBox@dvips
}
%    \end{macrocode}
%    \end{macro}
%    Since \hologo{XeTeX} v0.6.
%    \begin{macro}{\HOLOGO@ScaleBox@xetex}
%    \begin{macrocode}
\HOLOGO@temp{xetex}{%
  \def\HOLOGO@ScaleBox@xetex#1#2#3{%
    \special{x:gsave}%
    \special{x:scale #1 #2}%
    #3%
    \special{x:grestore}%
  }%
}
%    \end{macrocode}
%    \end{macro}
%    \begin{macro}{\HOLOGO@ScaleBox@vtex}
%    \begin{macrocode}
\HOLOGO@temp{vtex}{%
  \def\HOLOGO@ScaleBox@vtex#1#2#3{%
    \special{r(#1,0,0,#2,0,0}%
    #3%
    \special{r)}%
  }%
}
%    \end{macrocode}
%    \end{macro}
%
%    \begin{macrocode}
\HOLOGO@AtEnd%
%</package>
%    \end{macrocode}
%
% \section{Test}
%
% \subsection{Catcode checks for loading}
%
%    \begin{macrocode}
%<*test1>
%    \end{macrocode}
%    \begin{macrocode}
\catcode`\{=1 %
\catcode`\}=2 %
\catcode`\#=6 %
\catcode`\@=11 %
\expandafter\ifx\csname count@\endcsname\relax
  \countdef\count@=255 %
\fi
\expandafter\ifx\csname @gobble\endcsname\relax
  \long\def\@gobble#1{}%
\fi
\expandafter\ifx\csname @firstofone\endcsname\relax
  \long\def\@firstofone#1{#1}%
\fi
\expandafter\ifx\csname loop\endcsname\relax
  \expandafter\@firstofone
\else
  \expandafter\@gobble
\fi
{%
  \def\loop#1\repeat{%
    \def\body{#1}%
    \iterate
  }%
  \def\iterate{%
    \body
      \let\next\iterate
    \else
      \let\next\relax
    \fi
    \next
  }%
  \let\repeat=\fi
}%
\def\RestoreCatcodes{}
\count@=0 %
\loop
  \edef\RestoreCatcodes{%
    \RestoreCatcodes
    \catcode\the\count@=\the\catcode\count@\relax
  }%
\ifnum\count@<255 %
  \advance\count@ 1 %
\repeat

\def\RangeCatcodeInvalid#1#2{%
  \count@=#1\relax
  \loop
    \catcode\count@=15 %
  \ifnum\count@<#2\relax
    \advance\count@ 1 %
  \repeat
}
\def\RangeCatcodeCheck#1#2#3{%
  \count@=#1\relax
  \loop
    \ifnum#3=\catcode\count@
    \else
      \errmessage{%
        Character \the\count@\space
        with wrong catcode \the\catcode\count@\space
        instead of \number#3%
      }%
    \fi
  \ifnum\count@<#2\relax
    \advance\count@ 1 %
  \repeat
}
\def\space{ }
\expandafter\ifx\csname LoadCommand\endcsname\relax
  \def\LoadCommand{\input hologo.sty\relax}%
\fi
\def\Test{%
  \RangeCatcodeInvalid{0}{47}%
  \RangeCatcodeInvalid{58}{64}%
  \RangeCatcodeInvalid{91}{96}%
  \RangeCatcodeInvalid{123}{255}%
  \catcode`\@=12 %
  \catcode`\\=0 %
  \catcode`\%=14 %
  \LoadCommand
  \RangeCatcodeCheck{0}{36}{15}%
  \RangeCatcodeCheck{37}{37}{14}%
  \RangeCatcodeCheck{38}{47}{15}%
  \RangeCatcodeCheck{48}{57}{12}%
  \RangeCatcodeCheck{58}{63}{15}%
  \RangeCatcodeCheck{64}{64}{12}%
  \RangeCatcodeCheck{65}{90}{11}%
  \RangeCatcodeCheck{91}{91}{15}%
  \RangeCatcodeCheck{92}{92}{0}%
  \RangeCatcodeCheck{93}{96}{15}%
  \RangeCatcodeCheck{97}{122}{11}%
  \RangeCatcodeCheck{123}{255}{15}%
  \RestoreCatcodes
}
\Test
\csname @@end\endcsname
\end
%    \end{macrocode}
%    \begin{macrocode}
%</test1>
%    \end{macrocode}
%
% \subsection{Spacefactor}
%
%    The space factor must be 1000 after a logo. If it is greater 1000
%    then the following space is a space after a sentence closing point.
%    If the space factor is smaller 1000 then an immediate following
%    dot is interpreted as abbreviation, not sentence closing point.
%
%    \begin{macrocode}
%<*test-spacefactor>
\NeedsTeXFormat{LaTeX2e}
\documentclass{article}
\usepackage{hologo}[2016/05/12]
\usepackage{kvsetkeys}
\usepackage{qstest}
\IncludeTests{*}
\LogTests{log}{*}{*}
\begin{document}
\begin{qstest}{spacefactor}{spacefactor}
\newcommand*{\Test}[1]{%
  \sbox0{%
    \hologo{#1}%
    \Expect*{1000 (#1)}*{\the\spacefactor\space(#1)}%
  }%
}%
\makeatletter
\def\TestList{}
\def\hologoEntry#1#2#3{%
  \edef\TestList{%
    \ifx\TestList\@empty
    \else
      \TestList,%
    \fi
    #1%
    \ifx\\#2\\%
    \else
      ={variant=#2}%
    \fi
  }%
}
\hologoList
\expandafter\kv@parse@normalized\expandafter{%
  \TestList
}{%
  \begingroup
    \let\@logo=\kv@key
    \ifx\kv@value\relax
    \else
      \expandafter\hologoLogoSetup\expandafter\@logo\expandafter{%
        \kv@value
      }%
    \fi
    \Test\@logo
  \endgroup
  \@gobbletwo
}
\end{qstest}
\end{document}
%</test-spacefactor>
%    \end{macrocode}
%
% \subsection{Complete list}
%
%    \begin{macrocode}
%<*test-list>
\NeedsTeXFormat{LaTeX2e}
\documentclass[12pt,a4paper]{article}
\usepackage{hologo}[2016/05/12]
\usepackage[T1]{fontenc}
\usepackage{lmodern}
\usepackage{parskip}
\usepackage[unicode]{hyperref}[2011/09/28]
\usepackage{bookmark}[2011/09/19]
\bookmarksetup{%
  numbered,%
  open,%
  openlevel=2,%
}
\renewcommand*{\contentsname}{List of logos}
\begin{document}
\tableofcontents
\def\TestFont#1#2#3#4#5#6{%
  \begingroup
    \usefont{#3}{#4}{#5}{#6}%
    \HologoVariant{#1}{#2}/\hologoVariant{#1}{#2}%
    \quad
    \begingroup\scriptsize\hologoVariant{#1}{#2}\endgroup
    \quad
  \endgroup
  (#3/#4/#5/#6)%
  \par
}
\makeatletter
\def\hologoEntry#1#2#3{%
  \section{%
    \HologoVariant{#1}{#2}/\hologoVariant{#1}{#2} %
    {[#1\ifx\\#2\\\else\space(#2)\fi]}% hash-ok
  }% braces around [] because of bug in tex4ht
  \begingroup
    \hypersetup{unicode=false}%
    \bookmark[%
      dest=\@currentHref,%
      rellevel=1,%
      keeplevel,%
    ]{%
      \HologoVariant{#1}{#2}/\hologoVariant{#1}{#2} %
      (PDFDocEncoding)%
    }%
  \endgroup
  \TestFont{#1}{#2}{OT1}{cmr}{m}{n}%
  \TestFont{#1}{#2}{OT1}{cmss}{m}{n}%
  \TestFont{#1}{#2}{OT1}{cmr}{b}{n}%
  \TestFont{#1}{#2}{OT1}{cmr}{m}{it}%
  \TestFont{#1}{#2}{OT1}{cmtt}{m}{n}%
  \TestFont{#1}{#2}{T1}{lmr}{m}{n}%
  \TestFont{#1}{#2}{T1}{lmss}{m}{n}%
  \TestFont{#1}{#2}{T1}{lmr}{b}{n}%
  \TestFont{#1}{#2}{T1}{lmr}{m}{it}%
  \TestFont{#1}{#2}{T1}{lmtt}{m}{n}%
  \TestFont{#1}{#2}{T1}{lmvtt}{m}{n}%
  \TestFont{#1}{#2}{T1}{qtm}{m}{n}%
  \TestFont{#1}{#2}{T1}{qhv}{m}{n}%
  \TestFont{#1}{#2}{T1}{qtm}{b}{n}%
  \TestFont{#1}{#2}{T1}{qtm}{m}{it}%
  \TestFont{#1}{#2}{T1}{qcr}{m}{n}%
  \newpage
}
\makeatother
\hologoList
\end{document}
%</test-list>
%    \end{macrocode}
%
% \section{Installation}
%
% \subsection{Download}
%
% \paragraph{Package.} This package is available on
% CTAN\footnote{\url{ftp://ftp.ctan.org/tex-archive/}}:
% \begin{description}
% \item[\CTAN{macros/latex/contrib/oberdiek/hologo.dtx}] The source file.
% \item[\CTAN{macros/latex/contrib/oberdiek/hologo.pdf}] Documentation.
% \end{description}
%
%
% \paragraph{Bundle.} All the packages of the bundle `oberdiek'
% are also available in a TDS compliant ZIP archive. There
% the packages are already unpacked and the documentation files
% are generated. The files and directories obey the TDS standard.
% \begin{description}
% \item[\CTAN{install/macros/latex/contrib/oberdiek.tds.zip}]
% \end{description}
% \emph{TDS} refers to the standard ``A Directory Structure
% for \TeX\ Files'' (\CTAN{tds/tds.pdf}). Directories
% with \xfile{texmf} in their name are usually organized this way.
%
% \subsection{Bundle installation}
%
% \paragraph{Unpacking.} Unpack the \xfile{oberdiek.tds.zip} in the
% TDS tree (also known as \xfile{texmf} tree) of your choice.
% Example (linux):
% \begin{quote}
%   |unzip oberdiek.tds.zip -d ~/texmf|
% \end{quote}
%
% \paragraph{Script installation.}
% Check the directory \xfile{TDS:scripts/oberdiek/} for
% scripts that need further installation steps.
% Package \xpackage{attachfile2} comes with the Perl script
% \xfile{pdfatfi.pl} that should be installed in such a way
% that it can be called as \texttt{pdfatfi}.
% Example (linux):
% \begin{quote}
%   |chmod +x scripts/oberdiek/pdfatfi.pl|\\
%   |cp scripts/oberdiek/pdfatfi.pl /usr/local/bin/|
% \end{quote}
%
% \subsection{Package installation}
%
% \paragraph{Unpacking.} The \xfile{.dtx} file is a self-extracting
% \docstrip\ archive. The files are extracted by running the
% \xfile{.dtx} through \plainTeX:
% \begin{quote}
%   \verb|tex hologo.dtx|
% \end{quote}
%
% \paragraph{TDS.} Now the different files must be moved into
% the different directories in your installation TDS tree
% (also known as \xfile{texmf} tree):
% \begin{quote}
% \def\t{^^A
% \begin{tabular}{@{}>{\ttfamily}l@{ $\rightarrow$ }>{\ttfamily}l@{}}
%   hologo.sty & tex/generic/oberdiek/hologo.sty\\
%   hologo.pdf & doc/latex/oberdiek/hologo.pdf\\
%   example/hologo-example.tex & doc/latex/oberdiek/example/hologo-example.tex\\
%   test/hologo-test1.tex & doc/latex/oberdiek/test/hologo-test1.tex\\
%   test/hologo-test-spacefactor.tex & doc/latex/oberdiek/test/hologo-test-spacefactor.tex\\
%   test/hologo-test-list.tex & doc/latex/oberdiek/test/hologo-test-list.tex\\
%   hologo.dtx & source/latex/oberdiek/hologo.dtx\\
% \end{tabular}^^A
% }^^A
% \sbox0{\t}^^A
% \ifdim\wd0>\linewidth
%   \begingroup
%     \advance\linewidth by\leftmargin
%     \advance\linewidth by\rightmargin
%   \edef\x{\endgroup
%     \def\noexpand\lw{\the\linewidth}^^A
%   }\x
%   \def\lwbox{^^A
%     \leavevmode
%     \hbox to \linewidth{^^A
%       \kern-\leftmargin\relax
%       \hss
%       \usebox0
%       \hss
%       \kern-\rightmargin\relax
%     }^^A
%   }^^A
%   \ifdim\wd0>\lw
%     \sbox0{\small\t}^^A
%     \ifdim\wd0>\linewidth
%       \ifdim\wd0>\lw
%         \sbox0{\footnotesize\t}^^A
%         \ifdim\wd0>\linewidth
%           \ifdim\wd0>\lw
%             \sbox0{\scriptsize\t}^^A
%             \ifdim\wd0>\linewidth
%               \ifdim\wd0>\lw
%                 \sbox0{\tiny\t}^^A
%                 \ifdim\wd0>\linewidth
%                   \lwbox
%                 \else
%                   \usebox0
%                 \fi
%               \else
%                 \lwbox
%               \fi
%             \else
%               \usebox0
%             \fi
%           \else
%             \lwbox
%           \fi
%         \else
%           \usebox0
%         \fi
%       \else
%         \lwbox
%       \fi
%     \else
%       \usebox0
%     \fi
%   \else
%     \lwbox
%   \fi
% \else
%   \usebox0
% \fi
% \end{quote}
% If you have a \xfile{docstrip.cfg} that configures and enables \docstrip's
% TDS installing feature, then some files can already be in the right
% place, see the documentation of \docstrip.
%
% \subsection{Refresh file name databases}
%
% If your \TeX~distribution
% (\teTeX, \mikTeX, \dots) relies on file name databases, you must refresh
% these. For example, \teTeX\ users run \verb|texhash| or
% \verb|mktexlsr|.
%
% \subsection{Some details for the interested}
%
% \paragraph{Attached source.}
%
% The PDF documentation on CTAN also includes the
% \xfile{.dtx} source file. It can be extracted by
% AcrobatReader 6 or higher. Another option is \textsf{pdftk},
% e.g. unpack the file into the current directory:
% \begin{quote}
%   \verb|pdftk hologo.pdf unpack_files output .|
% \end{quote}
%
% \paragraph{Unpacking with \LaTeX.}
% The \xfile{.dtx} chooses its action depending on the format:
% \begin{description}
% \item[\plainTeX:] Run \docstrip\ and extract the files.
% \item[\LaTeX:] Generate the documentation.
% \end{description}
% If you insist on using \LaTeX\ for \docstrip\ (really,
% \docstrip\ does not need \LaTeX), then inform the autodetect routine
% about your intention:
% \begin{quote}
%   \verb|latex \let\install=y\input{hologo.dtx}|
% \end{quote}
% Do not forget to quote the argument according to the demands
% of your shell.
%
% \paragraph{Generating the documentation.}
% You can use both the \xfile{.dtx} or the \xfile{.drv} to generate
% the documentation. The process can be configured by the
% configuration file \xfile{ltxdoc.cfg}. For instance, put this
% line into this file, if you want to have A4 as paper format:
% \begin{quote}
%   \verb|\PassOptionsToClass{a4paper}{article}|
% \end{quote}
% An example follows how to generate the
% documentation with pdf\LaTeX:
% \begin{quote}
%\begin{verbatim}
%pdflatex hologo.dtx
%makeindex -s gind.ist hologo.idx
%pdflatex hologo.dtx
%makeindex -s gind.ist hologo.idx
%pdflatex hologo.dtx
%\end{verbatim}
% \end{quote}
%
% \section{Catalogue}
%
% The following XML file can be used as source for the
% \href{http://mirror.ctan.org/help/Catalogue/catalogue.html}{\TeX\ Catalogue}.
% The elements \texttt{caption} and \texttt{description} are imported
% from the original XML file from the Catalogue.
% The name of the XML file in the Catalogue is \xfile{hologo.xml}.
%    \begin{macrocode}
%<*catalogue>
<?xml version='1.0' encoding='us-ascii'?>
<!DOCTYPE entry SYSTEM 'catalogue.dtd'>
<entry datestamp='$Date$' modifier='$Author$' id='hologo'>
  <name>hologo</name>
  <caption>A collection of logos with bookmark support.</caption>
  <authorref id='auth:oberdiek'/>
  <copyright owner='Heiko Oberdiek' year='2010-2012'/>
  <license type='lppl1.3'/>
  <version number='1.10'/>
  <description>
    The package defines a single command <tt>\hologo</tt>, whose
    argument is the usual case-confused ASCII version of the logo.
    The command is bookmark-enabled, so that every logo becomes
    available in bookmarks without further work.
    <p/>
    The package is part of the <xref refid='oberdiek'>oberdiek</xref>
    bundle.
  </description>
  <documentation details='Package documentation'
      href='ctan:/macros/latex/contrib/oberdiek/hologo.pdf'/>
  <ctan file='true' path='/macros/latex/contrib/oberdiek/hologo.dtx'/>
  <miktex location='oberdiek'/>
  <texlive location='oberdiek'/>
  <install path='/macros/latex/contrib/oberdiek/oberdiek.tds.zip'/>
</entry>
%</catalogue>
%    \end{macrocode}
%
% \begin{thebibliography}{9}
% \raggedright
%
% \bibitem{btxdoc}
% Oren Patashnik,
% \textit{\hologo{BibTeX}ing},
% 1988-02-08.\\
% \CTAN{biblio/bibtex/base/}
%
% \bibitem{dtklogos}
% Gerd Neugebauer, DANTE,
% \textit{Package \xpackage{dtklogos}},
% 2011-04-25.\\
% \CTAN{usergrps/dante/dtk/dtklogos.sty}
%
% \bibitem{etexman}
% The \hologo{NTS} Team,
% \textit{The \hologo{eTeX} manual},
% 1998-02.\\
% \CTAN{systems/e-tex/v2/doc/}
%
% \bibitem{ExTeX-FAQ}
% The \hologo{ExTeX} group,
% \textit{\hologo{ExTeX}: FAQ -- How is \hologo{ExTeX} typeset?},
% 2007-04-14.\\
% \url{http://www.extex.org/documentation/faq.html}
%
% \bibitem{LyX}
% %@MISC{ LyX,
% %  title = {{LyX 2.0.0 -- The Document Processor [Computer software and manual]}},
% %  author = {{The LyX Team}},
% %  howpublished = {Internet: http://www.lyx.org},
% %  year = {2011-05-08},
% %  note = {Retrieved May 10, 2011, from http://www.lyx.org},
% %  url = {http://www.lyx.org/}
% %}
% The \hologo{LyX} Team,
% \textit{\hologo{LyX} -- The Document Processor},
% 2011-05-08.\\
% \url{http://www.lyx.org/}
%
% \bibitem{OzTeX}
% Andrew Trevorrow,
% \hologo{OzTeX} FAQ: What is the correct way to typeset ``\hologo{OzTeX}''?,
% 2011-09-15 (visited).
% \url{http://www.trevorrow.com/oztex/ozfaq.html#oztex-logo}
%
% \bibitem{PiCTeX}
% Michael Wichura,
% \textit{The \hologo{PiCTeX} macro package},
% 1987-09-21.
% \CTAN{graphics/pictex/}
%
% \bibitem{scrlogo}
% Markus Kohm,
% \textit{\hologo{KOMAScript} Datei \xfile{scrlogo.dtx}},
% 2009-01-30.\\
% \CTAN{install/macros/latex/contrib/komascript.tds.zip}
%
% \end{thebibliography}
%
% \begin{History}
%   \begin{Version}{2010/04/08 v1.0}
%   \item
%     The first version.
%   \end{Version}
%   \begin{Version}{2010/04/16 v1.1}
%   \item
%     \cs{Hologo} added for support of logos at start of a sentence.
%   \item
%     \cs{hologoSetup} and \cs{hologoLogoSetup} added.
%   \item
%     Options \xoption{break}, \xoption{hyphenbreak}, \xoption{spacebreak}
%     added.
%   \item
%     Variant support added by option \xoption{variant}.
%   \end{Version}
%   \begin{Version}{2010/04/24 v1.2}
%   \item
%     \hologo{LaTeX3} added.
%   \item
%     \hologo{VTeX} added.
%   \end{Version}
%   \begin{Version}{2010/11/21 v1.3}
%   \item
%     \hologo{iniTeX}, \hologo{virTeX} added.
%   \end{Version}
%   \begin{Version}{2011/03/25 v1.4}
%   \item
%     \hologo{ConTeXt} with variants added.
%   \item
%     Option \xoption{discretionarybreak} added as refinement for
%     option \xoption{break}.
%   \end{Version}
%   \begin{Version}{2011/04/21 v1.5}
%   \item
%     Wrong TDS directory for test files fixed.
%   \end{Version}
%   \begin{Version}{2011/10/01 v1.6}
%   \item
%     Support for package \xpackage{tex4ht} added.
%   \item
%     Support for \cs{csname} added if \cs{ifincsname} is available.
%   \item
%     New logos:
%     \hologo{(La)TeX},
%     \hologo{biber},
%     \hologo{BibTeX} (\xoption{sc}, \xoption{sf}),
%     \hologo{emTeX},
%     \hologo{ExTeX},
%     \hologo{KOMAScript},
%     \hologo{La},
%     \hologo{LyX},
%     \hologo{MiKTeX},
%     \hologo{NTS},
%     \hologo{OzMF},
%     \hologo{OzMP},
%     \hologo{OzTeX},
%     \hologo{OzTtH},
%     \hologo{PCTeX},
%     \hologo{PiC},
%     \hologo{PiCTeX},
%     \hologo{METAFONT},
%     \hologo{MetaFun},
%     \hologo{METAPOST},
%     \hologo{MetaPost},
%     \hologo{SLiTeX} (\xoption{lift}, \xoption{narrow}, \xoption{simple}),
%     \hologo{SliTeX} (\xoption{narrow}, \xoption{simple}, \xoption{lift}),
%     \hologo{teTeX}.
%   \item
%     Fixes:
%     \hologo{iniTeX},
%     \hologo{pdfLaTeX},
%     \hologo{pdfTeX},
%     \hologo{virTeX}.
%   \item
%     \cs{hologoFontSetup} and \cs{hologoLogoFontSetup} added.
%   \item
%     \cs{hologoVariant} and \cs{HologoVariant} added.
%   \end{Version}
%   \begin{Version}{2011/11/22 v1.7}
%   \item
%     New logos:
%     \hologo{BibTeX8},
%     \hologo{LaTeXML},
%     \hologo{SageTeX},
%     \hologo{TeX4ht},
%     \hologo{TTH}.
%   \item
%     \hologo{Xe} and friends: Driver stuff fixed.
%   \item
%     \hologo{Xe} and friends: Support for italic added.
%   \item
%     \hologo{Xe} and friends: Package support for \xpackage{pgf}
%     and \xpackage{pstricks} added.
%   \end{Version}
%   \begin{Version}{2011/11/29 v1.8}
%   \item
%     New logos:
%     \hologo{HanTheThanh}.
%   \end{Version}
%   \begin{Version}{2011/12/21 v1.9}
%   \item
%     Patch for package \xpackage{ifxetex} added for the case that
%     \cs{newif} is undefined in \hologo{iniTeX}.
%   \item
%     Some fixes for \hologo{iniTeX}.
%   \end{Version}
%   \begin{Version}{2012/04/26 v1.10}
%   \item
%     Fix in bookmark version of logo ``\hologo{HanTheThanh}''.
%   \end{Version}
%   \begin{Version}{2016/05/12 v1.11}
%   \item
%     Update HOLOGO@IfCharExists (previously in texlive)
%   \item define pdfliteral in current luatex.
%   \end{Version}
% \end{History}
%
% \PrintIndex
%
% \Finale
\endinput
%
        \else
          \input hologo.cfg\relax
        \fi
      \fi
    }%
    \ltx@IfUndefined{newread}{%
      \chardef\HOLOGO@temp=15 %
      \def\HOLOGO@CheckRead{%
        \ifeof\HOLOGO@temp
          \HOLOGO@InputIfExists
        \else
          \ifcase\HOLOGO@temp
            \@PackageWarningNoLine{hologo}{%
              Configuration file ignored, because\MessageBreak
              a free read register could not be found%
            }%
          \else
            \begingroup
              \count\ltx@cclv=\HOLOGO@temp
              \advance\ltx@cclv by \ltx@minusone
              \edef\x{\endgroup
                \chardef\noexpand\HOLOGO@temp=\the\count\ltx@cclv
                \relax
              }%
            \x
          \fi
        \fi
      }%
    }{%
      \csname newread\endcsname\HOLOGO@temp
      \HOLOGO@InputIfExists
    }%
  }{%
    \edef\HOLOGO@temp{\pdf@filesize{hologo.cfg}}%
    \ifx\HOLOGO@temp\ltx@empty
    \else
      \ifnum\HOLOGO@temp>0 %
        \begingroup
          \def\x{LaTeX2e}%
        \expandafter\endgroup
        \ifx\fmtname\x
          % \iffalse meta-comment
%
% File: hologo.dtx
% Version: 2016/05/12 v1.11
% Info: A logo collection with bookmark support
%
% Copyright (C) 2010-2012 by
%    Heiko Oberdiek <heiko.oberdiek at googlemail.com>
%
% This work may be distributed and/or modified under the
% conditions of the LaTeX Project Public License, either
% version 1.3c of this license or (at your option) any later
% version. This version of this license is in
%    http://www.latex-project.org/lppl/lppl-1-3c.txt
% and the latest version of this license is in
%    http://www.latex-project.org/lppl.txt
% and version 1.3 or later is part of all distributions of
% LaTeX version 2005/12/01 or later.
%
% This work has the LPPL maintenance status "maintained".
%
% This Current Maintainer of this work is Heiko Oberdiek.
%
% The Base Interpreter refers to any `TeX-Format',
% because some files are installed in TDS:tex/generic//.
%
% This work consists of the main source file hologo.dtx
% and the derived files
%    hologo.sty, hologo.pdf, hologo.ins, hologo.drv, hologo-example.tex,
%    hologo-test1.tex, hologo-test-spacefactor.tex,
%    hologo-test-list.tex.
%
% Distribution:
%    CTAN:macros/latex/contrib/oberdiek/hologo.dtx
%    CTAN:macros/latex/contrib/oberdiek/hologo.pdf
%
% Unpacking:
%    (a) If hologo.ins is present:
%           tex hologo.ins
%    (b) Without hologo.ins:
%           tex hologo.dtx
%    (c) If you insist on using LaTeX
%           latex \let\install=y\input{hologo.dtx}
%        (quote the arguments according to the demands of your shell)
%
% Documentation:
%    (a) If hologo.drv is present:
%           latex hologo.drv
%    (b) Without hologo.drv:
%           latex hologo.dtx; ...
%    The class ltxdoc loads the configuration file ltxdoc.cfg
%    if available. Here you can specify further options, e.g.
%    use A4 as paper format:
%       \PassOptionsToClass{a4paper}{article}
%
%    Programm calls to get the documentation (example):
%       pdflatex hologo.dtx
%       makeindex -s gind.ist hologo.idx
%       pdflatex hologo.dtx
%       makeindex -s gind.ist hologo.idx
%       pdflatex hologo.dtx
%
% Installation:
%    TDS:tex/generic/oberdiek/hologo.sty
%    TDS:doc/latex/oberdiek/hologo.pdf
%    TDS:doc/latex/oberdiek/example/hologo-example.tex
%    TDS:doc/latex/oberdiek/test/hologo-test1.tex
%    TDS:doc/latex/oberdiek/test/hologo-test-spacefactor.tex
%    TDS:doc/latex/oberdiek/test/hologo-test-list.tex
%    TDS:source/latex/oberdiek/hologo.dtx
%
%<*ignore>
\begingroup
  \catcode123=1 %
  \catcode125=2 %
  \def\x{LaTeX2e}%
\expandafter\endgroup
\ifcase 0\ifx\install y1\fi\expandafter
         \ifx\csname processbatchFile\endcsname\relax\else1\fi
         \ifx\fmtname\x\else 1\fi\relax
\else\csname fi\endcsname
%</ignore>
%<*install>
\input docstrip.tex
\Msg{************************************************************************}
\Msg{* Installation}
\Msg{* Package: hologo 2016/05/12 v1.11 A logo collection with bookmark support (HO)}
\Msg{************************************************************************}

\keepsilent
\askforoverwritefalse

\let\MetaPrefix\relax
\preamble

This is a generated file.

Project: hologo
Version: 2016/05/12 v1.11

Copyright (C) 2010-2012 by
   Heiko Oberdiek <heiko.oberdiek at googlemail.com>

This work may be distributed and/or modified under the
conditions of the LaTeX Project Public License, either
version 1.3c of this license or (at your option) any later
version. This version of this license is in
   http://www.latex-project.org/lppl/lppl-1-3c.txt
and the latest version of this license is in
   http://www.latex-project.org/lppl.txt
and version 1.3 or later is part of all distributions of
LaTeX version 2005/12/01 or later.

This work has the LPPL maintenance status "maintained".

This Current Maintainer of this work is Heiko Oberdiek.

The Base Interpreter refers to any `TeX-Format',
because some files are installed in TDS:tex/generic//.

This work consists of the main source file hologo.dtx
and the derived files
   hologo.sty, hologo.pdf, hologo.ins, hologo.drv, hologo-example.tex,
   hologo-test1.tex, hologo-test-spacefactor.tex,
   hologo-test-list.tex.

\endpreamble
\let\MetaPrefix\DoubleperCent

\generate{%
  \file{hologo.ins}{\from{hologo.dtx}{install}}%
  \file{hologo.drv}{\from{hologo.dtx}{driver}}%
  \usedir{tex/generic/oberdiek}%
  \file{hologo.sty}{\from{hologo.dtx}{package}}%
  \usedir{doc/latex/oberdiek/example}%
  \file{hologo-example.tex}{\from{hologo.dtx}{example}}%
  \usedir{doc/latex/oberdiek/test}%
  \file{hologo-test1.tex}{\from{hologo.dtx}{test1}}%
  \file{hologo-test-spacefactor.tex}{\from{hologo.dtx}{test-spacefactor}}%
  \file{hologo-test-list.tex}{\from{hologo.dtx}{test-list}}%
  \nopreamble
  \nopostamble
  \usedir{source/latex/oberdiek/catalogue}%
  \file{hologo.xml}{\from{hologo.dtx}{catalogue}}%
}

\catcode32=13\relax% active space
\let =\space%
\Msg{************************************************************************}
\Msg{*}
\Msg{* To finish the installation you have to move the following}
\Msg{* file into a directory searched by TeX:}
\Msg{*}
\Msg{*     hologo.sty}
\Msg{*}
\Msg{* To produce the documentation run the file `hologo.drv'}
\Msg{* through LaTeX.}
\Msg{*}
\Msg{* Happy TeXing!}
\Msg{*}
\Msg{************************************************************************}

\endbatchfile
%</install>
%<*ignore>
\fi
%</ignore>
%<*driver>
\NeedsTeXFormat{LaTeX2e}
\ProvidesFile{hologo.drv}%
  [2016/05/12 v1.11 A logo collection with bookmark support (HO)]%
\documentclass{ltxdoc}
\usepackage{holtxdoc}[2011/11/22]
\usepackage{hologo}[2016/05/12]
\usepackage{longtable}
\usepackage{array}
\usepackage{paralist}
%\usepackage[T1]{fontenc}
%\usepackage{lmodern}
\begin{document}
  \DocInput{hologo.dtx}%
\end{document}
%</driver>
% \fi
%
%
% \CharacterTable
%  {Upper-case    \A\B\C\D\E\F\G\H\I\J\K\L\M\N\O\P\Q\R\S\T\U\V\W\X\Y\Z
%   Lower-case    \a\b\c\d\e\f\g\h\i\j\k\l\m\n\o\p\q\r\s\t\u\v\w\x\y\z
%   Digits        \0\1\2\3\4\5\6\7\8\9
%   Exclamation   \!     Double quote  \"     Hash (number) \#
%   Dollar        \$     Percent       \%     Ampersand     \&
%   Acute accent  \'     Left paren    \(     Right paren   \)
%   Asterisk      \*     Plus          \+     Comma         \,
%   Minus         \-     Point         \.     Solidus       \/
%   Colon         \:     Semicolon     \;     Less than     \<
%   Equals        \=     Greater than  \>     Question mark \?
%   Commercial at \@     Left bracket  \[     Backslash     \\
%   Right bracket \]     Circumflex    \^     Underscore    \_
%   Grave accent  \`     Left brace    \{     Vertical bar  \|
%   Right brace   \}     Tilde         \~}
%
% \GetFileInfo{hologo.drv}
%
% \title{The \xpackage{hologo} package}
% \date{2016/05/12 v1.11}
% \author{Heiko Oberdiek\\\xemail{heiko.oberdiek at googlemail.com}}
%
% \maketitle
%
% \begin{abstract}
% This package starts a collection of logos with support for bookmarks
% strings.
% \end{abstract}
%
% \tableofcontents
%
% \section{Documentation}
%
% \subsection{Logo macros}
%
% \begin{declcs}{hologo} \M{name}
% \end{declcs}
% Macro \cs{hologo} sets the logo with name \meta{name}.
% The following table shows the supported names.
%
% \begingroup
%   \def\hologoEntry#1#2#3{^^A
%     #1&#2&\hologoLogoSetup{#1}{variant=#2}\hologo{#1}&#3\tabularnewline
%   }
%   \begin{longtable}{>{\ttfamily}l>{\ttfamily}lll}
%     \rmfamily\bfseries{name} & \rmfamily\bfseries variant
%     & \bfseries logo & \bfseries since\\
%     \hline
%     \endhead
%     \hologoList
%   \end{longtable}
% \endgroup
%
% \begin{declcs}{Hologo} \M{name}
% \end{declcs}
% Macro \cs{Hologo} starts the logo \meta{name} with an uppercase
% letter. As an exception small greek letters are not converted
% to uppercase. Examples, see \hologo{eTeX} and \hologo{ExTeX}.
%
% \subsection{Setup macros}
%
% The package does not support package options, but the following
% setup macros can be used to set options.
%
% \begin{declcs}{hologoSetup} \M{key value list}
% \end{declcs}
% Macro \cs{hologoSetup} sets global options.
%
% \begin{declcs}{hologoLogoSetup} \M{logo} \M{key value list}
% \end{declcs}
% Some options can also be used to configure a logo.
% These settings take precedence over global option settings.
%
% \subsection{Options}\label{sec:options}
%
% There are boolean and string options:
% \begin{description}
% \item[Boolean option:]
% It takes |true| or |false|
% as value. If the value is omitted, then |true| is used.
% \item[String option:]
% A value must be given as string. (But the string might be empty.)
% \end{description}
% The following options can be used both in \cs{hologoSetup}
% and \cs{hologoLogoSetup}:
% \begin{description}
% \def\entry#1{\item[\xoption{#1}:]}
% \entry{break}
%   enables or disables line breaks inside the logo. This setting is
%   refined by options \xoption{hyphenbreak}, \xoption{spacebreak}
%   or \xoption{discretionarybreak}.
%   Default is |false|.
% \entry{hyphenbreak}
%   enables or disables the line break right after the hyphen character.
% \entry{spacebreak}
%   enables or disables line breaks at space characters.
% \entry{discretionarybreak}
%   enables or disables line breaks at hyphenation points
%   (inserted by \cs{-}).
% \end{description}
% Macro \cs{hologoLogoSetup} also knows:
% \begin{description}
% \item[\xoption{variant}:]
%   This is a string option. It specifies a variant of a logo that
%   must exist. An empty string selects the package default variant.
% \end{description}
% Example:
% \begin{quote}
%   |\hologoSetup{break=false}|\\
%   |\hologoLogoSetup{plainTeX}{variant=hyphen,hyphenbreak}|\\
%   Then ``plain-\TeX'' contains one break point after the hyphen.
% \end{quote}
%
% \subsection{Driver options}
%
% Sometimes graphical operations are needed to construct some
% glyphs (e.g.\ \hologo{XeTeX}). If package \xpackage{graphics}
% or package \xpackage{pgf} are found, then the macros are taken
% from there. Otherwise the packge defines its own operations
% and therefore needs the driver information. Many drivers are
% detected automatically (\hologo{pdfTeX}/\hologo{LuaTeX}
% in PDF mode, \hologo{XeTeX}, \hologo{VTeX}). These have precedence
% over a driver option. The driver can be given as package option
% or using \cs{hologoDriverSetup}.
% The following list contains the recognized driver options:
% \begin{itemize}
% \item \xoption{pdftex}, \xoption{luatex}
% \item \xoption{dvipdfm}, \xoption{dvipdfmx}
% \item \xoption{dvips}, \xoption{dvipsone}, \xoption{xdvi}
% \item \xoption{xetex}
% \item \xoption{vtex}
% \end{itemize}
% The left driver of a line is the driver name that is used internally.
% The following names are aliases for drivers that use the
% same method. Therefore the entry in the \xext{log} file for
% the used driver prints the internally used driver name.
% \begin{description}
% \item[\xoption{driverfallback}:]
%   This option expects a driver that is used,
%   if the driver could not be detected automatically.
% \end{description}
%
% \begin{declcs}{hologoDriverSetup} \M{driver option}
% \end{declcs}
% The driver can also be configured after package loading
% using \cs{hologoDriverSetup}, also the way for \hologo{plainTeX}
% to setup the driver.
%
% \subsection{Font setup}
%
% Some logos require a special font, but should also be usable by
% \hologo{plainTeX}. Therefore the package provides some ways
% to influence the font settings. The options below
% take font settings as values. Both font commands
% such as \cs{sffamily} and macros that take one argument
% like \cs{textsf} can be used.
%
% \begin{declcs}{hologoFontSetup} \M{key value list}
% \end{declcs}
% Macro \cs{hologoFontSetup} sets the fonts for all logos.
% Supported keys:
% \begin{description}
% \def\entry#1{\item[\xoption{#1}:]}
% \entry{general}
%   This font is used for all logos. The default is empty.
%   That means no special font is used.
% \entry{bibsf}
%   This font is used for
%   {\hologoLogoSetup{BibTeX}{variant=sf}\hologo{BibTeX}}
%   with variant \xoption{sf}.
% \entry{rm}
%   This font is a serif font. It is used for \hologo{ExTeX}.
% \entry{sc}
%   This font specifies a small caps font. It is used for
%   {\hologoLogoSetup{BibTeX}{variant=sc}\hologo{BibTeX}}
%   with variant \xoption{sc}.
% \entry{sf}
%   This font specifies a sans serif font. The default
%   is \cs{sffamily}, then \cs{sf} is tried. Otherwise
%   a warning is given. It is used by \hologo{KOMAScript}.
% \entry{sy}
%   This is the font for math symbols (e.g. cmsy).
%   It is used by \hologo{AmS}, \hologo{NTS}, \hologo{ExTeX}.
% \entry{logo}
%   \hologo{METAFONT} and \hologo{METAPOST} are using that font.
%   In \hologo{LaTeX} \cs{logofamily} is used and
%   the definitions of package \xpackage{mflogo} are used
%   if the package is not loaded.
%   Otherwise the \cs{tenlogo} is used and defined
%   if it does not already exists.
% \end{description}
%
% \begin{declcs}{hologoLogoFontSetup} \M{logo} \M{key value list}
% \end{declcs}
% Fonts can also be set for a logo or logo component separately,
% see the following list.
% The keys are the same as for \cs{hologoFontSetup}.
%
% \begin{longtable}{>{\ttfamily}l>{\sffamily}ll}
%   \meta{logo} & keys & result\\
%   \hline
%   \endhead
%   BibTeX & bibsf & {\hologoLogoSetup{BibTeX}{variant=sf}\hologo{BibTeX}}\\[.5ex]
%   BibTeX & sc & {\hologoLogoSetup{BibTeX}{variant=sc}\hologo{BibTeX}}\\[.5ex]
%   ExTeX & rm & \hologo{ExTeX}\\
%   SliTeX & rm & \hologo{SliTeX}\\[.5ex]
%   AmS & sy & \hologo{AmS}\\
%   ExTeX & sy & \hologo{ExTeX}\\
%   NTS & sy & \hologo{NTS}\\[.5ex]
%   KOMAScript & sf & \hologo{KOMAScript}\\[.5ex]
%   METAFONT & logo & \hologo{METAFONT}\\
%   METAPOST & logo & \hologo{METAPOST}\\[.5ex]
%   SliTeX & sc \hologo{SliTeX}
% \end{longtable}
%
% \subsubsection{Font order}
%
% For all logos the font \xoption{general} is applied first.
% Example:
%\begin{quote}
%|\hologoFontSetup{general=\color{red}}|
%\end{quote}
% will print red logos.
% Then if the font uses a special font \xoption{sf}, for example,
% the font is applied that is setup by \cs{hologoLogoFontSetup}.
% If this font is not setup, then the common font setup
% by \cs{hologoFontSetup} is used. Otherwise a warning is given,
% that there is no font configured.
%
% \subsection{Additional user macros}
%
% Usually a variant of a logo is configured by using
% \cs{hologoLogoSetup}, because it is bad style to mix
% different variants of the same logo in the same text.
% There the following macros are a convenience for testing.
%
% \begin{declcs}{hologoVariant} \M{name} \M{variant}\\
%   \cs{HologoVariant} \M{name} \M{variant}
% \end{declcs}
% Logo \meta{name} is set using \meta{variant} that specifies
% explicitely which variant of the macro is used. If the argument
% is empty, then the default form of the logo is used
% (configurable by \cs{hologoLogoSetup}).
%
% \cs{HologoVariant} is used if the logo is set in a context
% that needs an uppercase first letter (beginning of a sentence, \dots).
%
% \begin{declcs}{hologoList}\\
%   \cs{hologoEntry} \M{logo} \M{variant} \M{since}
% \end{declcs}
% Macro \cs{hologoList} contains all logos that are provided
% by the package including variants. The list consists of calls
% of \cs{hologoEntry} with three arguments starting with the
% logo name \meta{logo} and its variant \meta{variant}. An empty
% variant means the current default. Argument \meta{since} specifies
% with version of the package \xpackage{hologo} is needed to get
% the logo. If the logo is fixed, then the date gets updated.
% Therefore the date \meta{since} is not exactly the date of
% the first introduction, but rather the date of the latest fix.
%
% Before \cs{hologoList} can be used, macro \cs{hologoEntry} needs
% a definition. The example file in section \ref{sec:example}
% shows applications of \cs{hologoList}.
%
% \subsection{Supported contexts}
%
% Macros \cs{hologo} and friends support special contexts:
% \begin{itemize}
% \item \hologo{LaTeX}'s protection mechanism.
% \item Bookmarks of package \xpackage{hyperref}.
% \item Package \xpackage{tex4ht}.
% \item The macros can be used inside \cs{csname} constructs,
%   if \cs{ifincsname} is available (\hologo{pdfTeX}, \hologo{XeTeX},
%   \hologo{LuaTeX}).
% \end{itemize}
%
% \subsection{Example}
% \label{sec:example}
%
% The following example prints the logos in different fonts.
%    \begin{macrocode}
%<*example>
%<<verbatim
\NeedsTeXFormat{LaTeX2e}
\documentclass[a4paper]{article}
\usepackage[
  hmargin=20mm,
  vmargin=20mm,
]{geometry}
\pagestyle{empty}
\usepackage{hologo}[2016/05/12]
\usepackage{longtable}
\usepackage{array}
\setlength{\extrarowheight}{2pt}
\usepackage[T1]{fontenc}
\usepackage{lmodern}
\usepackage{pdflscape}
\usepackage[
  pdfencoding=auto,
]{hyperref}
\hypersetup{
  pdfauthor={Heiko Oberdiek},
  pdftitle={Example for package `hologo'},
  pdfsubject={Logos with fonts lmr, lmss, qtm, qpl, qhv},
}
\usepackage{bookmark}

% Print the logo list on the console

\begingroup
  \typeout{}%
  \typeout{*** Begin of logo list ***}%
  \newcommand*{\hologoEntry}[3]{%
    \typeout{#1 \ifx\\#2\\\else(#2) \fi[#3]}%
  }%
  \hologoList
  \typeout{*** End of logo list ***}%
  \typeout{}%
\endgroup

\begin{document}
\begin{landscape}

  \section{Example file for package `hologo'}

  % Table for font names

  \begin{longtable}{>{\bfseries}ll}
    \textbf{font} & \textbf{Font name}\\
    \hline
    lmr & Latin Modern Roman\\
    lmss & Latin Modern Sans\\
    qtm & \TeX\ Gyre Termes\\
    qhv & \TeX\ Gyre Heros\\
    qpl & \TeX\ Gyre Pagella\\
  \end{longtable}

  % Logo list with logos in different fonts

  \begingroup
    \newcommand*{\SetVariant}[2]{%
      \ifx\\#2\\%
      \else
        \hologoLogoSetup{#1}{variant=#2}%
      \fi
    }%
    \newcommand*{\hologoEntry}[3]{%
      \SetVariant{#1}{#2}%
      \raisebox{1em}[0pt][0pt]{\hypertarget{#1@#2}{}}%
      \bookmark[%
        dest={#1@#2},%
      ]{%
        #1\ifx\\#2\\\else\space(#2)\fi: \Hologo{#1}, \hologo{#1} %
        [Unicode]%
      }%
      \hypersetup{unicode=false}%
      \bookmark[%
        dest={#1@#2},%
      ]{%
        #1\ifx\\#2\\\else\space(#2)\fi: \Hologo{#1}, \hologo{#1} %
        [PDFDocEncoding]%
      }%
      \texttt{#1}%
      &%
      \texttt{#2}%
      &%
      \Hologo{#1}%
      &%
      \SetVariant{#1}{#2}%
      \hologo{#1}%
      &%
      \SetVariant{#1}{#2}%
      \fontfamily{qtm}\selectfont
      \hologo{#1}%
      &%
      \SetVariant{#1}{#2}%
      \fontfamily{qpl}\selectfont
      \hologo{#1}%
      &%
      \SetVariant{#1}{#2}%
      \textsf{\hologo{#1}}%
      &%
      \SetVariant{#1}{#2}%
      \fontfamily{qhv}\selectfont
      \hologo{#1}%
      \tabularnewline
    }%
    \begin{longtable}{llllllll}%
      \textbf{\textit{logo}} & \textbf{\textit{variant}} &
      \texttt{\string\Hologo} &
      \textbf{lmr} & \textbf{qtm} & \textbf{qpl} &
      \textbf{lmss} & \textbf{qhv}
      \tabularnewline
      \hline
      \endhead
      \hologoList
    \end{longtable}%
  \endgroup

\end{landscape}
\end{document}
%verbatim
%</example>
%    \end{macrocode}
%
% \StopEventually{
% }
%
% \section{Implementation}
%    \begin{macrocode}
%<*package>
%    \end{macrocode}
%    Reload check, especially if the package is not used with \LaTeX.
%    \begin{macrocode}
\begingroup\catcode61\catcode48\catcode32=10\relax%
  \catcode13=5 % ^^M
  \endlinechar=13 %
  \catcode35=6 % #
  \catcode39=12 % '
  \catcode44=12 % ,
  \catcode45=12 % -
  \catcode46=12 % .
  \catcode58=12 % :
  \catcode64=11 % @
  \catcode123=1 % {
  \catcode125=2 % }
  \expandafter\let\expandafter\x\csname ver@hologo.sty\endcsname
  \ifx\x\relax % plain-TeX, first loading
  \else
    \def\empty{}%
    \ifx\x\empty % LaTeX, first loading,
      % variable is initialized, but \ProvidesPackage not yet seen
    \else
      \expandafter\ifx\csname PackageInfo\endcsname\relax
        \def\x#1#2{%
          \immediate\write-1{Package #1 Info: #2.}%
        }%
      \else
        \def\x#1#2{\PackageInfo{#1}{#2, stopped}}%
      \fi
      \x{hologo}{The package is already loaded}%
      \aftergroup\endinput
    \fi
  \fi
\endgroup%
%    \end{macrocode}
%    Package identification:
%    \begin{macrocode}
\begingroup\catcode61\catcode48\catcode32=10\relax%
  \catcode13=5 % ^^M
  \endlinechar=13 %
  \catcode35=6 % #
  \catcode39=12 % '
  \catcode40=12 % (
  \catcode41=12 % )
  \catcode44=12 % ,
  \catcode45=12 % -
  \catcode46=12 % .
  \catcode47=12 % /
  \catcode58=12 % :
  \catcode64=11 % @
  \catcode91=12 % [
  \catcode93=12 % ]
  \catcode123=1 % {
  \catcode125=2 % }
  \expandafter\ifx\csname ProvidesPackage\endcsname\relax
    \def\x#1#2#3[#4]{\endgroup
      \immediate\write-1{Package: #3 #4}%
      \xdef#1{#4}%
    }%
  \else
    \def\x#1#2[#3]{\endgroup
      #2[{#3}]%
      \ifx#1\@undefined
        \xdef#1{#3}%
      \fi
      \ifx#1\relax
        \xdef#1{#3}%
      \fi
    }%
  \fi
\expandafter\x\csname ver@hologo.sty\endcsname
\ProvidesPackage{hologo}%
  [2016/05/12 v1.11 A logo collection with bookmark support (HO)]%
%    \end{macrocode}
%
%    \begin{macrocode}
\begingroup\catcode61\catcode48\catcode32=10\relax%
  \catcode13=5 % ^^M
  \endlinechar=13 %
  \catcode123=1 % {
  \catcode125=2 % }
  \catcode64=11 % @
  \def\x{\endgroup
    \expandafter\edef\csname HOLOGO@AtEnd\endcsname{%
      \endlinechar=\the\endlinechar\relax
      \catcode13=\the\catcode13\relax
      \catcode32=\the\catcode32\relax
      \catcode35=\the\catcode35\relax
      \catcode61=\the\catcode61\relax
      \catcode64=\the\catcode64\relax
      \catcode123=\the\catcode123\relax
      \catcode125=\the\catcode125\relax
    }%
  }%
\x\catcode61\catcode48\catcode32=10\relax%
\catcode13=5 % ^^M
\endlinechar=13 %
\catcode35=6 % #
\catcode64=11 % @
\catcode123=1 % {
\catcode125=2 % }
\def\TMP@EnsureCode#1#2{%
  \edef\HOLOGO@AtEnd{%
    \HOLOGO@AtEnd
    \catcode#1=\the\catcode#1\relax
  }%
  \catcode#1=#2\relax
}
\TMP@EnsureCode{10}{12}% ^^J
\TMP@EnsureCode{33}{12}% !
\TMP@EnsureCode{34}{12}% "
\TMP@EnsureCode{36}{3}% $
\TMP@EnsureCode{38}{4}% &
\TMP@EnsureCode{39}{12}% '
\TMP@EnsureCode{40}{12}% (
\TMP@EnsureCode{41}{12}% )
\TMP@EnsureCode{42}{12}% *
\TMP@EnsureCode{43}{12}% +
\TMP@EnsureCode{44}{12}% ,
\TMP@EnsureCode{45}{12}% -
\TMP@EnsureCode{46}{12}% .
\TMP@EnsureCode{47}{12}% /
\TMP@EnsureCode{58}{12}% :
\TMP@EnsureCode{59}{12}% ;
\TMP@EnsureCode{60}{12}% <
\TMP@EnsureCode{62}{12}% >
\TMP@EnsureCode{63}{12}% ?
\TMP@EnsureCode{91}{12}% [
\TMP@EnsureCode{93}{12}% ]
\TMP@EnsureCode{94}{7}% ^ (superscript)
\TMP@EnsureCode{95}{8}% _ (subscript)
\TMP@EnsureCode{96}{12}% `
\TMP@EnsureCode{124}{12}% |
\edef\HOLOGO@AtEnd{%
  \HOLOGO@AtEnd
  \escapechar\the\escapechar\relax
  \noexpand\endinput
}
\escapechar=92 %
%    \end{macrocode}
%
% \subsection{Logo list}
%
%    \begin{macro}{\hologoList}
%    \begin{macrocode}
\def\hologoList{%
  \hologoEntry{(La)TeX}{}{2011/10/01}%
  \hologoEntry{AmSLaTeX}{}{2010/04/16}%
  \hologoEntry{AmSTeX}{}{2010/04/16}%
  \hologoEntry{biber}{}{2011/10/01}%
  \hologoEntry{BibTeX}{}{2011/10/01}%
  \hologoEntry{BibTeX}{sf}{2011/10/01}%
  \hologoEntry{BibTeX}{sc}{2011/10/01}%
  \hologoEntry{BibTeX8}{}{2011/11/22}%
  \hologoEntry{ConTeXt}{}{2011/03/25}%
  \hologoEntry{ConTeXt}{narrow}{2011/03/25}%
  \hologoEntry{ConTeXt}{simple}{2011/03/25}%
  \hologoEntry{emTeX}{}{2010/04/26}%
  \hologoEntry{eTeX}{}{2010/04/08}%
  \hologoEntry{ExTeX}{}{2011/10/01}%
  \hologoEntry{HanTheThanh}{}{2011/11/29}%
  \hologoEntry{iniTeX}{}{2011/10/01}%
  \hologoEntry{KOMAScript}{}{2011/10/01}%
  \hologoEntry{La}{}{2010/05/08}%
  \hologoEntry{LaTeX}{}{2010/04/08}%
  \hologoEntry{LaTeX2e}{}{2010/04/08}%
  \hologoEntry{LaTeX3}{}{2010/04/24}%
  \hologoEntry{LaTeXe}{}{2010/04/08}%
  \hologoEntry{LaTeXML}{}{2011/11/22}%
  \hologoEntry{LaTeXTeX}{}{2011/10/01}%
  \hologoEntry{LuaLaTeX}{}{2010/04/08}%
  \hologoEntry{LuaTeX}{}{2010/04/08}%
  \hologoEntry{LyX}{}{2011/10/01}%
  \hologoEntry{METAFONT}{}{2011/10/01}%
  \hologoEntry{MetaFun}{}{2011/10/01}%
  \hologoEntry{METAPOST}{}{2011/10/01}%
  \hologoEntry{MetaPost}{}{2011/10/01}%
  \hologoEntry{MiKTeX}{}{2011/10/01}%
  \hologoEntry{NTS}{}{2011/10/01}%
  \hologoEntry{OzMF}{}{2011/10/01}%
  \hologoEntry{OzMP}{}{2011/10/01}%
  \hologoEntry{OzTeX}{}{2011/10/01}%
  \hologoEntry{OzTtH}{}{2011/10/01}%
  \hologoEntry{PCTeX}{}{2011/10/01}%
  \hologoEntry{pdfTeX}{}{2011/10/01}%
  \hologoEntry{pdfLaTeX}{}{2011/10/01}%
  \hologoEntry{PiC}{}{2011/10/01}%
  \hologoEntry{PiCTeX}{}{2011/10/01}%
  \hologoEntry{plainTeX}{}{2010/04/08}%
  \hologoEntry{plainTeX}{space}{2010/04/16}%
  \hologoEntry{plainTeX}{hyphen}{2010/04/16}%
  \hologoEntry{plainTeX}{runtogether}{2010/04/16}%
  \hologoEntry{SageTeX}{}{2011/11/22}%
  \hologoEntry{SLiTeX}{}{2011/10/01}%
  \hologoEntry{SLiTeX}{lift}{2011/10/01}%
  \hologoEntry{SLiTeX}{narrow}{2011/10/01}%
  \hologoEntry{SLiTeX}{simple}{2011/10/01}%
  \hologoEntry{SliTeX}{}{2011/10/01}%
  \hologoEntry{SliTeX}{narrow}{2011/10/01}%
  \hologoEntry{SliTeX}{simple}{2011/10/01}%
  \hologoEntry{SliTeX}{lift}{2011/10/01}%
  \hologoEntry{teTeX}{}{2011/10/01}%
  \hologoEntry{TeX}{}{2010/04/08}%
  \hologoEntry{TeX4ht}{}{2011/11/22}%
  \hologoEntry{TTH}{}{2011/11/22}%
  \hologoEntry{virTeX}{}{2011/10/01}%
  \hologoEntry{VTeX}{}{2010/04/24}%
  \hologoEntry{Xe}{}{2010/04/08}%
  \hologoEntry{XeLaTeX}{}{2010/04/08}%
  \hologoEntry{XeTeX}{}{2010/04/08}%
}
%    \end{macrocode}
%    \end{macro}
%
% \subsection{Load resources}
%
%    \begin{macrocode}
\begingroup\expandafter\expandafter\expandafter\endgroup
\expandafter\ifx\csname RequirePackage\endcsname\relax
  \def\TMP@RequirePackage#1[#2]{%
    \begingroup\expandafter\expandafter\expandafter\endgroup
    \expandafter\ifx\csname ver@#1.sty\endcsname\relax
      \input #1.sty\relax
    \fi
  }%
  \TMP@RequirePackage{ltxcmds}[2011/02/04]%
  \TMP@RequirePackage{infwarerr}[2010/04/08]%
  \TMP@RequirePackage{kvsetkeys}[2010/03/01]%
  \TMP@RequirePackage{kvdefinekeys}[2010/03/01]%
  \TMP@RequirePackage{pdftexcmds}[2010/04/01]%
  \TMP@RequirePackage{ifpdf}[2010/01/28]%
  \TMP@RequirePackage{ifluatex}[2010/03/01]%
  \ltx@IfUndefined{newif}{%
    \expandafter\let\csname newif\endcsname\ltx@newif
  }{}%
  \TMP@RequirePackage{ifxetex}[2009/01/23]%
  \TMP@RequirePackage{ifvtex}[2010/03/01]%
\else
  \RequirePackage{ltxcmds}[2011/02/04]%
  \RequirePackage{infwarerr}[2010/04/08]%
  \RequirePackage{kvsetkeys}[2010/03/01]%
  \RequirePackage{kvdefinekeys}[2010/03/01]%
  \RequirePackage{pdftexcmds}[2010/04/01]%
  \RequirePackage{ifpdf}[2010/01/28]%
  \RequirePackage{ifluatex}[2010/03/01]%
  \RequirePackage{ifxetex}[2009/01/23]%
  \RequirePackage{ifvtex}[2010/03/01]%
\fi
%    \end{macrocode}
%
%    \begin{macro}{\HOLOGO@IfDefined}
%    \begin{macrocode}
\def\HOLOGO@IfExists#1{%
  \ifx\@undefined#1%
    \expandafter\ltx@secondoftwo
  \else
    \ifx\relax#1%
      \expandafter\ltx@secondoftwo
    \else
      \expandafter\expandafter\expandafter\ltx@firstoftwo
    \fi
  \fi
}
%    \end{macrocode}
%    \end{macro}
%
% \subsection{Setup macros}
%
%    \begin{macro}{\hologoSetup}
%    \begin{macrocode}
\def\hologoSetup{%
  \let\HOLOGO@name\relax
  \HOLOGO@Setup
}
%    \end{macrocode}
%    \end{macro}
%
%    \begin{macro}{\hologoLogoSetup}
%    \begin{macrocode}
\def\hologoLogoSetup#1{%
  \edef\HOLOGO@name{#1}%
  \ltx@IfUndefined{HoLogo@\HOLOGO@name}{%
    \@PackageError{hologo}{%
      Unknown logo `\HOLOGO@name'%
    }\@ehc
    \ltx@gobble
  }{%
    \HOLOGO@Setup
  }%
}
%    \end{macrocode}
%    \end{macro}
%
%    \begin{macro}{\HOLOGO@Setup}
%    \begin{macrocode}
\def\HOLOGO@Setup{%
  \kvsetkeys{HoLogo}%
}
%    \end{macrocode}
%    \end{macro}
%
% \subsection{Options}
%
%    \begin{macro}{\HOLOGO@DeclareBoolOption}
%    \begin{macrocode}
\def\HOLOGO@DeclareBoolOption#1{%
  \expandafter\chardef\csname HOLOGOOPT@#1\endcsname\ltx@zero
  \kv@define@key{HoLogo}{#1}[true]{%
    \def\HOLOGO@temp{##1}%
    \ifx\HOLOGO@temp\HOLOGO@true
      \ifx\HOLOGO@name\relax
        \expandafter\chardef\csname HOLOGOOPT@#1\endcsname=\ltx@one
      \else
        \expandafter\chardef\csname
        HoLogoOpt@#1@\HOLOGO@name\endcsname\ltx@one
      \fi
      \HOLOGO@SetBreakAll{#1}%
    \else
      \ifx\HOLOGO@temp\HOLOGO@false
        \ifx\HOLOGO@name\relax
          \expandafter\chardef\csname HOLOGOOPT@#1\endcsname=\ltx@zero
        \else
          \expandafter\chardef\csname
          HoLogoOpt@#1@\HOLOGO@name\endcsname=\ltx@zero
        \fi
        \HOLOGO@SetBreakAll{#1}%
      \else
        \@PackageError{hologo}{%
          Unknown value `##1' for boolean option `#1'.\MessageBreak
          Known values are `true' and `false'%
        }\@ehc
      \fi
    \fi
  }%
}
%    \end{macrocode}
%    \end{macro}
%
%    \begin{macro}{\HOLOGO@SetBreakAll}
%    \begin{macrocode}
\def\HOLOGO@SetBreakAll#1{%
  \def\HOLOGO@temp{#1}%
  \ifx\HOLOGO@temp\HOLOGO@break
    \ifx\HOLOGO@name\relax
      \chardef\HOLOGOOPT@hyphenbreak=\HOLOGOOPT@break
      \chardef\HOLOGOOPT@spacebreak=\HOLOGOOPT@break
      \chardef\HOLOGOOPT@discretionarybreak=\HOLOGOOPT@break
    \else
      \expandafter\chardef
         \csname HoLogoOpt@hyphenbreak@\HOLOGO@name\endcsname=%
         \csname HoLogoOpt@break@\HOLOGO@name\endcsname
      \expandafter\chardef
         \csname HoLogoOpt@spacebreak@\HOLOGO@name\endcsname=%
         \csname HoLogoOpt@break@\HOLOGO@name\endcsname
      \expandafter\chardef
         \csname HoLogoOpt@discretionarybreak@\HOLOGO@name
             \endcsname=%
         \csname HoLogoOpt@break@\HOLOGO@name\endcsname
    \fi
  \fi
}
%    \end{macrocode}
%    \end{macro}
%
%    \begin{macro}{\HOLOGO@true}
%    \begin{macrocode}
\def\HOLOGO@true{true}
%    \end{macrocode}
%    \end{macro}
%    \begin{macro}{\HOLOGO@false}
%    \begin{macrocode}
\def\HOLOGO@false{false}
%    \end{macrocode}
%    \end{macro}
%    \begin{macro}{\HOLOGO@break}
%    \begin{macrocode}
\def\HOLOGO@break{break}
%    \end{macrocode}
%    \end{macro}
%
%    \begin{macrocode}
\HOLOGO@DeclareBoolOption{break}
\HOLOGO@DeclareBoolOption{hyphenbreak}
\HOLOGO@DeclareBoolOption{spacebreak}
\HOLOGO@DeclareBoolOption{discretionarybreak}
%    \end{macrocode}
%
%    \begin{macrocode}
\kv@define@key{HoLogo}{variant}{%
  \ifx\HOLOGO@name\relax
    \@PackageError{hologo}{%
      Option `variant' is not available in \string\hologoSetup,%
      \MessageBreak
      Use \string\hologoLogoSetup\space instead%
    }\@ehc
  \else
    \edef\HOLOGO@temp{#1}%
    \ifx\HOLOGO@temp\ltx@empty
      \expandafter
      \let\csname HoLogoOpt@variant@\HOLOGO@name\endcsname\@undefined
    \else
      \ltx@IfUndefined{HoLogo@\HOLOGO@name @\HOLOGO@temp}{%
        \@PackageError{hologo}{%
          Unknown variant `\HOLOGO@temp' of logo `\HOLOGO@name'%
        }\@ehc
      }{%
        \expandafter
        \let\csname HoLogoOpt@variant@\HOLOGO@name\endcsname
            \HOLOGO@temp
      }%
    \fi
  \fi
}
%    \end{macrocode}
%
%    \begin{macro}{\HOLOGO@Variant}
%    \begin{macrocode}
\def\HOLOGO@Variant#1{%
  #1%
  \ltx@ifundefined{HoLogoOpt@variant@#1}{%
  }{%
    @\csname HoLogoOpt@variant@#1\endcsname
  }%
}
%    \end{macrocode}
%    \end{macro}
%
% \subsection{Break/no-break support}
%
%    \begin{macro}{\HOLOGO@space}
%    \begin{macrocode}
\def\HOLOGO@space{%
  \ltx@ifundefined{HoLogoOpt@spacebreak@\HOLOGO@name}{%
    \ltx@ifundefined{HoLogoOpt@break@\HOLOGO@name}{%
      \chardef\HOLOGO@temp=\HOLOGOOPT@spacebreak
    }{%
      \chardef\HOLOGO@temp=%
        \csname HoLogoOpt@break@\HOLOGO@name\endcsname
    }%
  }{%
    \chardef\HOLOGO@temp=%
      \csname HoLogoOpt@spacebreak@\HOLOGO@name\endcsname
  }%
  \ifcase\HOLOGO@temp
    \penalty10000 %
  \fi
  \ltx@space
}
%    \end{macrocode}
%    \end{macro}
%
%    \begin{macro}{\HOLOGO@hyphen}
%    \begin{macrocode}
\def\HOLOGO@hyphen{%
  \ltx@ifundefined{HoLogoOpt@hyphenbreak@\HOLOGO@name}{%
    \ltx@ifundefined{HoLogoOpt@break@\HOLOGO@name}{%
      \chardef\HOLOGO@temp=\HOLOGOOPT@hyphenbreak
    }{%
      \chardef\HOLOGO@temp=%
        \csname HoLogoOpt@break@\HOLOGO@name\endcsname
    }%
  }{%
    \chardef\HOLOGO@temp=%
      \csname HoLogoOpt@hyphenbreak@\HOLOGO@name\endcsname
  }%
  \ifcase\HOLOGO@temp
    \ltx@mbox{-}%
  \else
    -%
  \fi
}
%    \end{macrocode}
%    \end{macro}
%
%    \begin{macro}{\HOLOGO@discretionary}
%    \begin{macrocode}
\def\HOLOGO@discretionary{%
  \ltx@ifundefined{HoLogoOpt@discretionarybreak@\HOLOGO@name}{%
    \ltx@ifundefined{HoLogoOpt@break@\HOLOGO@name}{%
      \chardef\HOLOGO@temp=\HOLOGOOPT@discretionarybreak
    }{%
      \chardef\HOLOGO@temp=%
        \csname HoLogoOpt@break@\HOLOGO@name\endcsname
    }%
  }{%
    \chardef\HOLOGO@temp=%
      \csname HoLogoOpt@discretionarybreak@\HOLOGO@name\endcsname
  }%
  \ifcase\HOLOGO@temp
  \else
    \-%
  \fi
}
%    \end{macrocode}
%    \end{macro}
%
%    \begin{macro}{\HOLOGO@mbox}
%    \begin{macrocode}
\def\HOLOGO@mbox#1{%
  \ltx@ifundefined{HoLogoOpt@break@\HOLOGO@name}{%
    \chardef\HOLOGO@temp=\HOLOGOOPT@hyphenbreak
  }{%
    \chardef\HOLOGO@temp=%
      \csname HoLogoOpt@break@\HOLOGO@name\endcsname
  }%
  \ifcase\HOLOGO@temp
    \ltx@mbox{#1}%
  \else
    #1%
  \fi
}
%    \end{macrocode}
%    \end{macro}
%
% \subsection{Font support}
%
%    \begin{macro}{\HoLogoFont@font}
%    \begin{tabular}{@{}ll@{}}
%    |#1|:& logo name\\
%    |#2|:& font short name\\
%    |#3|:& text
%    \end{tabular}
%    \begin{macrocode}
\def\HoLogoFont@font#1#2#3{%
  \begingroup
    \ltx@IfUndefined{HoLogoFont@logo@#1.#2}{%
      \ltx@IfUndefined{HoLogoFont@font@#2}{%
        \@PackageWarning{hologo}{%
          Missing font `#2' for logo `#1'%
        }%
        #3%
      }{%
        \csname HoLogoFont@font@#2\endcsname{#3}%
      }%
    }{%
      \csname HoLogoFont@logo@#1.#2\endcsname{#3}%
    }%
  \endgroup
}
%    \end{macrocode}
%    \end{macro}
%
%    \begin{macro}{\HoLogoFont@Def}
%    \begin{macrocode}
\def\HoLogoFont@Def#1{%
  \expandafter\def\csname HoLogoFont@font@#1\endcsname
}
%    \end{macrocode}
%    \end{macro}
%    \begin{macro}{\HoLogoFont@LogoDef}
%    \begin{macrocode}
\def\HoLogoFont@LogoDef#1#2{%
  \expandafter\def\csname HoLogoFont@logo@#1.#2\endcsname
}
%    \end{macrocode}
%    \end{macro}
%
% \subsubsection{Font defaults}
%
%    \begin{macro}{\HoLogoFont@font@general}
%    \begin{macrocode}
\HoLogoFont@Def{general}{}%
%    \end{macrocode}
%    \end{macro}
%
%    \begin{macro}{\HoLogoFont@font@rm}
%    \begin{macrocode}
\ltx@IfUndefined{rmfamily}{%
  \ltx@IfUndefined{rm}{%
  }{%
    \HoLogoFont@Def{rm}{\rm}%
  }%
}{%
  \HoLogoFont@Def{rm}{\rmfamily}%
}
%    \end{macrocode}
%    \end{macro}
%
%    \begin{macro}{\HoLogoFont@font@sf}
%    \begin{macrocode}
\ltx@IfUndefined{sffamily}{%
  \ltx@IfUndefined{sf}{%
  }{%
    \HoLogoFont@Def{sf}{\sf}%
  }%
}{%
  \HoLogoFont@Def{sf}{\sffamily}%
}
%    \end{macrocode}
%    \end{macro}
%
%    \begin{macro}{\HoLogoFont@font@bibsf}
%    In case of \hologo{plainTeX} the original small caps
%    variant is used as default. In \hologo{LaTeX}
%    the definition of package \xpackage{dtklogos} \cite{dtklogos}
%    is used.
%\begin{quote}
%\begin{verbatim}
%\DeclareRobustCommand{\BibTeX}{%
%  B%
%  \kern-.05em%
%  \hbox{%
%    $\m@th$% %% force math size calculations
%    \csname S@\f@size\endcsname
%    \fontsize\sf@size\z@
%    \math@fontsfalse
%    \selectfont
%    I%
%    \kern-.025em%
%    B
%  }%
%  \kern-.08em%
%  \-%
%  \TeX
%}
%\end{verbatim}
%\end{quote}
%    \begin{macrocode}
\ltx@IfUndefined{selectfont}{%
  \ltx@IfUndefined{tensc}{%
    \font\tensc=cmcsc10\relax
  }{}%
  \HoLogoFont@Def{bibsf}{\tensc}%
}{%
  \HoLogoFont@Def{bibsf}{%
    $\mathsurround=0pt$%
    \csname S@\f@size\endcsname
    \fontsize\sf@size{0pt}%
    \math@fontsfalse
    \selectfont
  }%
}
%    \end{macrocode}
%    \end{macro}
%
%    \begin{macro}{\HoLogoFont@font@sc}
%    \begin{macrocode}
\ltx@IfUndefined{scshape}{%
  \ltx@IfUndefined{tensc}{%
    \font\tensc=cmcsc10\relax
  }{}%
  \HoLogoFont@Def{sc}{\tensc}%
}{%
  \HoLogoFont@Def{sc}{\scshape}%
}
%    \end{macrocode}
%    \end{macro}
%
%    \begin{macro}{\HoLogoFont@font@sy}
%    \begin{macrocode}
\ltx@IfUndefined{usefont}{%
  \ltx@IfUndefined{tensy}{%
  }{%
    \HoLogoFont@Def{sy}{\tensy}%
  }%
}{%
  \HoLogoFont@Def{sy}{%
    \usefont{OMS}{cmsy}{m}{n}%
  }%
}
%    \end{macrocode}
%    \end{macro}
%
%    \begin{macro}{\HoLogoFont@font@logo}
%    \begin{macrocode}
\begingroup
  \def\x{LaTeX2e}%
\expandafter\endgroup
\ifx\fmtname\x
  \ltx@IfUndefined{logofamily}{%
    \DeclareRobustCommand\logofamily{%
      \not@math@alphabet\logofamily\relax
      \fontencoding{U}%
      \fontfamily{logo}%
      \selectfont
    }%
  }{}%
  \ltx@IfUndefined{logofamily}{%
  }{%
    \HoLogoFont@Def{logo}{\logofamily}%
  }%
\else
  \ltx@IfUndefined{tenlogo}{%
    \font\tenlogo=logo10\relax
  }{}%
  \HoLogoFont@Def{logo}{\tenlogo}%
\fi
%    \end{macrocode}
%    \end{macro}
%
% \subsubsection{Font setup}
%
%    \begin{macro}{\hologoFontSetup}
%    \begin{macrocode}
\def\hologoFontSetup{%
  \let\HOLOGO@name\relax
  \HOLOGO@FontSetup
}
%    \end{macrocode}
%    \end{macro}
%
%    \begin{macro}{\hologoLogoFontSetup}
%    \begin{macrocode}
\def\hologoLogoFontSetup#1{%
  \edef\HOLOGO@name{#1}%
  \ltx@IfUndefined{HoLogo@\HOLOGO@name}{%
    \@PackageError{hologo}{%
      Unknown logo `\HOLOGO@name'%
    }\@ehc
    \ltx@gobble
  }{%
    \HOLOGO@FontSetup
  }%
}
%    \end{macrocode}
%    \end{macro}
%
%    \begin{macro}{\HOLOGO@FontSetup}
%    \begin{macrocode}
\def\HOLOGO@FontSetup{%
  \kvsetkeys{HoLogoFont}%
}
%    \end{macrocode}
%    \end{macro}
%
%    \begin{macrocode}
\def\HOLOGO@temp#1{%
  \kv@define@key{HoLogoFont}{#1}{%
    \ifx\HOLOGO@name\relax
      \HoLogoFont@Def{#1}{##1}%
    \else
      \HoLogoFont@LogoDef\HOLOGO@name{#1}{##1}%
    \fi
  }%
}
\HOLOGO@temp{general}
\HOLOGO@temp{sf}
%    \end{macrocode}
%
% \subsection{Generic logo commands}
%
%    \begin{macrocode}
\HOLOGO@IfExists\hologo{%
  \@PackageError{hologo}{%
    \string\hologo\ltx@space is already defined.\MessageBreak
    Package loading is aborted%
  }\@ehc
  \HOLOGO@AtEnd
}%
\HOLOGO@IfExists\hologoRobust{%
  \@PackageError{hologo}{%
    \string\hologoRobust\ltx@space is already defined.\MessageBreak
    Package loading is aborted%
  }\@ehc
  \HOLOGO@AtEnd
}%
%    \end{macrocode}
%
% \subsubsection{\cs{hologo} and friends}
%
%    \begin{macrocode}
\ifluatex
  \expandafter\ltx@firstofone
\else
  \expandafter\ltx@gobble
\fi
{%
  \ltx@IfUndefined{ifincsname}{%
    \ifnum\luatexversion<36 %
      \expandafter\ltx@gobble
    \else
      \expandafter\ltx@firstofone
    \fi
    {%
      \begingroup
        \ifcase0%
            \directlua{%
              if tex.enableprimitives then %
                tex.enableprimitives('HOLOGO@', {'ifincsname'})%
              else %
                tex.print('1')%
              end%
            }%
            \ifx\HOLOGO@ifincsname\@undefined 1\fi%
            \relax
          \expandafter\ltx@firstofone
        \else
          \endgroup
          \expandafter\ltx@gobble
        \fi
        {%
          \global\let\ifincsname\HOLOGO@ifincsname
        }%
      \HOLOGO@temp
    }%
  }{}%
}
%    \end{macrocode}
%    \begin{macrocode}
\ltx@IfUndefined{ifincsname}{%
  \catcode`$=14 %
}{%
  \catcode`$=9 %
}
%    \end{macrocode}
%
%    \begin{macro}{\hologo}
%    \begin{macrocode}
\def\hologo#1{%
$ \ifincsname
$   \ltx@ifundefined{HoLogoCs@\HOLOGO@Variant{#1}}{%
$     #1%
$   }{%
$     \csname HoLogoCs@\HOLOGO@Variant{#1}\endcsname\ltx@firstoftwo
$   }%
$ \else
    \HOLOGO@IfExists\texorpdfstring\texorpdfstring\ltx@firstoftwo
    {%
      \hologoRobust{#1}%
    }{%
      \ltx@ifundefined{HoLogoBkm@\HOLOGO@Variant{#1}}{%
        \ltx@ifundefined{HoLogo@#1}{?#1?}{#1}%
      }{%
        \csname HoLogoBkm@\HOLOGO@Variant{#1}\endcsname
        \ltx@firstoftwo
      }%
    }%
$ \fi
}
%    \end{macrocode}
%    \end{macro}
%    \begin{macro}{\Hologo}
%    \begin{macrocode}
\def\Hologo#1{%
$ \ifincsname
$   \ltx@ifundefined{HoLogoCs@\HOLOGO@Variant{#1}}{%
$     #1%
$   }{%
$     \csname HoLogoCs@\HOLOGO@Variant{#1}\endcsname\ltx@secondoftwo
$   }%
$ \else
    \HOLOGO@IfExists\texorpdfstring\texorpdfstring\ltx@firstoftwo
    {%
      \HologoRobust{#1}%
    }{%
      \ltx@ifundefined{HoLogoBkm@\HOLOGO@Variant{#1}}{%
        \ltx@ifundefined{HoLogo@#1}{?#1?}{#1}%
      }{%
        \csname HoLogoBkm@\HOLOGO@Variant{#1}\endcsname
        \ltx@secondoftwo
      }%
    }%
$ \fi
}
%    \end{macrocode}
%    \end{macro}
%
%    \begin{macro}{\hologoVariant}
%    \begin{macrocode}
\def\hologoVariant#1#2{%
  \ifx\relax#2\relax
    \hologo{#1}%
  \else
$   \ifincsname
$     \ltx@ifundefined{HoLogoCs@#1@#2}{%
$       #1%
$     }{%
$       \csname HoLogoCs@#1@#2\endcsname\ltx@firstoftwo
$     }%
$   \else
      \HOLOGO@IfExists\texorpdfstring\texorpdfstring\ltx@firstoftwo
      {%
        \hologoVariantRobust{#1}{#2}%
      }{%
        \ltx@ifundefined{HoLogoBkm@#1@#2}{%
          \ltx@ifundefined{HoLogo@#1}{?#1?}{#1}%
        }{%
          \csname HoLogoBkm@#1@#2\endcsname
          \ltx@firstoftwo
        }%
      }%
$   \fi
  \fi
}
%    \end{macrocode}
%    \end{macro}
%    \begin{macro}{\HologoVariant}
%    \begin{macrocode}
\def\HologoVariant#1#2{%
  \ifx\relax#2\relax
    \Hologo{#1}%
  \else
$   \ifincsname
$     \ltx@ifundefined{HoLogoCs@#1@#2}{%
$       #1%
$     }{%
$       \csname HoLogoCs@#1@#2\endcsname\ltx@secondoftwo
$     }%
$   \else
      \HOLOGO@IfExists\texorpdfstring\texorpdfstring\ltx@firstoftwo
      {%
        \HologoVariantRobust{#1}{#2}%
      }{%
        \ltx@ifundefined{HoLogoBkm@#1@#2}{%
          \ltx@ifundefined{HoLogo@#1}{?#1?}{#1}%
        }{%
          \csname HoLogoBkm@#1@#2\endcsname
          \ltx@secondoftwo
        }%
      }%
$   \fi
  \fi
}
%    \end{macrocode}
%    \end{macro}
%
%    \begin{macrocode}
\catcode`\$=3 %
%    \end{macrocode}
%
% \subsubsection{\cs{hologoRobust} and friends}
%
%    \begin{macro}{\hologoRobust}
%    \begin{macrocode}
\ltx@IfUndefined{protected}{%
  \ltx@IfUndefined{DeclareRobustCommand}{%
    \def\hologoRobust#1%
  }{%
    \DeclareRobustCommand*\hologoRobust[1]%
  }%
}{%
  \protected\def\hologoRobust#1%
}%
{%
  \edef\HOLOGO@name{#1}%
  \ltx@IfUndefined{HoLogo@\HOLOGO@Variant\HOLOGO@name}{%
    \@PackageError{hologo}{%
      Unknown logo `\HOLOGO@name'%
    }\@ehc
    ?\HOLOGO@name?%
  }{%
    \ltx@IfUndefined{ver@tex4ht.sty}{%
      \HoLogoFont@font\HOLOGO@name{general}{%
        \csname HoLogo@\HOLOGO@Variant\HOLOGO@name\endcsname
        \ltx@firstoftwo
      }%
    }{%
      \ltx@IfUndefined{HoLogoHtml@\HOLOGO@Variant\HOLOGO@name}{%
        \HOLOGO@name
      }{%
        \csname HoLogoHtml@\HOLOGO@Variant\HOLOGO@name\endcsname
        \ltx@firstoftwo
      }%
    }%
  }%
}
%    \end{macrocode}
%    \end{macro}
%    \begin{macro}{\HologoRobust}
%    \begin{macrocode}
\ltx@IfUndefined{protected}{%
  \ltx@IfUndefined{DeclareRobustCommand}{%
    \def\HologoRobust#1%
  }{%
    \DeclareRobustCommand*\HologoRobust[1]%
  }%
}{%
  \protected\def\HologoRobust#1%
}%
{%
  \edef\HOLOGO@name{#1}%
  \ltx@IfUndefined{HoLogo@\HOLOGO@Variant\HOLOGO@name}{%
    \@PackageError{hologo}{%
      Unknown logo `\HOLOGO@name'%
    }\@ehc
    ?\HOLOGO@name?%
  }{%
    \ltx@IfUndefined{ver@tex4ht.sty}{%
      \HoLogoFont@font\HOLOGO@name{general}{%
        \csname HoLogo@\HOLOGO@Variant\HOLOGO@name\endcsname
        \ltx@secondoftwo
      }%
    }{%
      \ltx@IfUndefined{HoLogoHtml@\HOLOGO@Variant\HOLOGO@name}{%
        \expandafter\HOLOGO@Uppercase\HOLOGO@name
      }{%
        \csname HoLogoHtml@\HOLOGO@Variant\HOLOGO@name\endcsname
        \ltx@secondoftwo
      }%
    }%
  }%
}
%    \end{macrocode}
%    \end{macro}
%    \begin{macro}{\hologoVariantRobust}
%    \begin{macrocode}
\ltx@IfUndefined{protected}{%
  \ltx@IfUndefined{DeclareRobustCommand}{%
    \def\hologoVariantRobust#1#2%
  }{%
    \DeclareRobustCommand*\hologoVariantRobust[2]%
  }%
}{%
  \protected\def\hologoVariantRobust#1#2%
}%
{%
  \begingroup
    \hologoLogoSetup{#1}{variant={#2}}%
    \hologoRobust{#1}%
  \endgroup
}
%    \end{macrocode}
%    \end{macro}
%    \begin{macro}{\HologoVariantRobust}
%    \begin{macrocode}
\ltx@IfUndefined{protected}{%
  \ltx@IfUndefined{DeclareRobustCommand}{%
    \def\HologoVariantRobust#1#2%
  }{%
    \DeclareRobustCommand*\HologoVariantRobust[2]%
  }%
}{%
  \protected\def\HologoVariantRobust#1#2%
}%
{%
  \begingroup
    \hologoLogoSetup{#1}{variant={#2}}%
    \HologoRobust{#1}%
  \endgroup
}
%    \end{macrocode}
%    \end{macro}
%
%    \begin{macro}{\hologorobust}
%    Macro \cs{hologorobust} is only defined for compatibility.
%    Its use is deprecated.
%    \begin{macrocode}
\def\hologorobust{\hologoRobust}
%    \end{macrocode}
%    \end{macro}
%
% \subsection{Helpers}
%
%    \begin{macro}{\HOLOGO@Uppercase}
%    Macro \cs{HOLOGO@Uppercase} is restricted to \cs{uppercase},
%    because \hologo{plainTeX} or \hologo{iniTeX} do not provide
%    \cs{MakeUppercase}.
%    \begin{macrocode}
\def\HOLOGO@Uppercase#1{\uppercase{#1}}
%    \end{macrocode}
%    \end{macro}
%
%    \begin{macro}{\HOLOGO@PdfdocUnicode}
%    \begin{macrocode}
\def\HOLOGO@PdfdocUnicode{%
  \ifx\ifHy@unicode\iftrue
    \expandafter\ltx@secondoftwo
  \else
    \expandafter\ltx@firstoftwo
  \fi
}
%    \end{macrocode}
%    \end{macro}
%
%    \begin{macro}{\HOLOGO@Math}
%    \begin{macrocode}
\def\HOLOGO@MathSetup{%
  \mathsurround0pt\relax
  \HOLOGO@IfExists\f@series{%
    \if b\expandafter\ltx@car\f@series x\@nil
      \csname boldmath\endcsname
   \fi
  }{}%
}
%    \end{macrocode}
%    \end{macro}
%
%    \begin{macro}{\HOLOGO@TempDimen}
%    \begin{macrocode}
\dimendef\HOLOGO@TempDimen=\ltx@zero
%    \end{macrocode}
%    \end{macro}
%    \begin{macro}{\HOLOGO@NegativeKerning}
%    \begin{macrocode}
\def\HOLOGO@NegativeKerning#1{%
  \begingroup
    \HOLOGO@TempDimen=0pt\relax
    \comma@parse@normalized{#1}{%
      \ifdim\HOLOGO@TempDimen=0pt %
        \expandafter\HOLOGO@@NegativeKerning\comma@entry
      \fi
      \ltx@gobble
    }%
    \ifdim\HOLOGO@TempDimen<0pt %
      \kern\HOLOGO@TempDimen
    \fi
  \endgroup
}
%    \end{macrocode}
%    \end{macro}
%    \begin{macro}{\HOLOGO@@NegativeKerning}
%    \begin{macrocode}
\def\HOLOGO@@NegativeKerning#1#2{%
  \setbox\ltx@zero\hbox{#1#2}%
  \HOLOGO@TempDimen=\wd\ltx@zero
  \setbox\ltx@zero\hbox{#1\kern0pt#2}%
  \advance\HOLOGO@TempDimen by -\wd\ltx@zero
}
%    \end{macrocode}
%    \end{macro}
%
%    \begin{macro}{\HOLOGO@SpaceFactor}
%    \begin{macrocode}
\def\HOLOGO@SpaceFactor{%
  \spacefactor1000 %
}
%    \end{macrocode}
%    \end{macro}
%
%    \begin{macro}{\HOLOGO@Span}
%    \begin{macrocode}
\def\HOLOGO@Span#1#2{%
  \HCode{<span class="HoLogo-#1">}%
  #2%
  \HCode{</span>}%
}
%    \end{macrocode}
%    \end{macro}
%
% \subsubsection{Text subscript}
%
%    \begin{macro}{\HOLOGO@SubScript}%
%    \begin{macrocode}
\def\HOLOGO@SubScript#1{%
  \ltx@IfUndefined{textsubscript}{%
    \ltx@IfUndefined{text}{%
      \ltx@mbox{%
        \mathsurround=0pt\relax
        $%
          _{%
            \ltx@IfUndefined{sf@size}{%
              \mathrm{#1}%
            }{%
              \mbox{%
                \fontsize\sf@size{0pt}\selectfont
                #1%
              }%
            }%
          }%
        $%
      }%
    }{%
      \ltx@mbox{%
        \mathsurround=0pt\relax
        $_{\text{#1}}$%
      }%
    }%
  }{%
    \textsubscript{#1}%
  }%
}
%    \end{macrocode}
%    \end{macro}
%
% \subsection{\hologo{TeX} and friends}
%
% \subsubsection{\hologo{TeX}}
%
%    \begin{macro}{\HoLogo@TeX}
%    Source: \hologo{LaTeX} kernel.
%    \begin{macrocode}
\def\HoLogo@TeX#1{%
  T\kern-.1667em\lower.5ex\hbox{E}\kern-.125emX\HOLOGO@SpaceFactor
}
%    \end{macrocode}
%    \end{macro}
%    \begin{macro}{\HoLogoHtml@TeX}
%    \begin{macrocode}
\def\HoLogoHtml@TeX#1{%
  \HoLogoCss@TeX
  \HOLOGO@Span{TeX}{%
    T%
    \HOLOGO@Span{e}{%
      E%
    }%
    X%
  }%
}
%    \end{macrocode}
%    \end{macro}
%    \begin{macro}{\HoLogoCss@TeX}
%    \begin{macrocode}
\def\HoLogoCss@TeX{%
  \Css{%
    span.HoLogo-TeX span.HoLogo-e{%
      position:relative;%
      top:.5ex;%
      margin-left:-.1667em;%
      margin-right:-.125em;%
    }%
  }%
  \Css{%
    a span.HoLogo-TeX span.HoLogo-e{%
      text-decoration:none;%
    }%
  }%
  \global\let\HoLogoCss@TeX\relax
}
%    \end{macrocode}
%    \end{macro}
%
% \subsubsection{\hologo{plainTeX}}
%
%    \begin{macro}{\HoLogo@plainTeX@space}
%    Source: ``The \hologo{TeX}book''
%    \begin{macrocode}
\def\HoLogo@plainTeX@space#1{%
  \HOLOGO@mbox{#1{p}{P}lain}\HOLOGO@space\hologo{TeX}%
}
%    \end{macrocode}
%    \end{macro}
%    \begin{macro}{\HoLogoCs@plainTeX@space}
%    \begin{macrocode}
\def\HoLogoCs@plainTeX@space#1{#1{p}{P}lain TeX}%
%    \end{macrocode}
%    \end{macro}
%    \begin{macro}{\HoLogoBkm@plainTeX@space}
%    \begin{macrocode}
\def\HoLogoBkm@plainTeX@space#1{%
  #1{p}{P}lain \hologo{TeX}%
}
%    \end{macrocode}
%    \end{macro}
%    \begin{macro}{\HoLogoHtml@plainTeX@space}
%    \begin{macrocode}
\def\HoLogoHtml@plainTeX@space#1{%
  #1{p}{P}lain \hologo{TeX}%
}
%    \end{macrocode}
%    \end{macro}
%
%    \begin{macro}{\HoLogo@plainTeX@hyphen}
%    \begin{macrocode}
\def\HoLogo@plainTeX@hyphen#1{%
  \HOLOGO@mbox{#1{p}{P}lain}\HOLOGO@hyphen\hologo{TeX}%
}
%    \end{macrocode}
%    \end{macro}
%    \begin{macro}{\HoLogoCs@plainTeX@hyphen}
%    \begin{macrocode}
\def\HoLogoCs@plainTeX@hyphen#1{#1{p}{P}lain-TeX}
%    \end{macrocode}
%    \end{macro}
%    \begin{macro}{\HoLogoBkm@plainTeX@hyphen}
%    \begin{macrocode}
\def\HoLogoBkm@plainTeX@hyphen#1{%
  #1{p}{P}lain-\hologo{TeX}%
}
%    \end{macrocode}
%    \end{macro}
%    \begin{macro}{\HoLogoHtml@plainTeX@hyphen}
%    \begin{macrocode}
\def\HoLogoHtml@plainTeX@hyphen#1{%
  #1{p}{P}lain-\hologo{TeX}%
}
%    \end{macrocode}
%    \end{macro}
%
%    \begin{macro}{\HoLogo@plainTeX@runtogether}
%    \begin{macrocode}
\def\HoLogo@plainTeX@runtogether#1{%
  \HOLOGO@mbox{#1{p}{P}lain\hologo{TeX}}%
}
%    \end{macrocode}
%    \end{macro}
%    \begin{macro}{\HoLogoCs@plainTeX@runtogether}
%    \begin{macrocode}
\def\HoLogoCs@plainTeX@runtogether#1{#1{p}{P}lainTeX}
%    \end{macrocode}
%    \end{macro}
%    \begin{macro}{\HoLogoBkm@plainTeX@runtogether}
%    \begin{macrocode}
\def\HoLogoBkm@plainTeX@runtogether#1{%
  #1{p}{P}lain\hologo{TeX}%
}
%    \end{macrocode}
%    \end{macro}
%    \begin{macro}{\HoLogoHtml@plainTeX@runtogether}
%    \begin{macrocode}
\def\HoLogoHtml@plainTeX@runtogether#1{%
  #1{p}{P}lain\hologo{TeX}%
}
%    \end{macrocode}
%    \end{macro}
%
%    \begin{macro}{\HoLogo@plainTeX}
%    \begin{macrocode}
\def\HoLogo@plainTeX{\HoLogo@plainTeX@space}
%    \end{macrocode}
%    \end{macro}
%    \begin{macro}{\HoLogoCs@plainTeX}
%    \begin{macrocode}
\def\HoLogoCs@plainTeX{\HoLogoCs@plainTeX@space}
%    \end{macrocode}
%    \end{macro}
%    \begin{macro}{\HoLogoBkm@plainTeX}
%    \begin{macrocode}
\def\HoLogoBkm@plainTeX{\HoLogoBkm@plainTeX@space}
%    \end{macrocode}
%    \end{macro}
%    \begin{macro}{\HoLogoHtml@plainTeX}
%    \begin{macrocode}
\def\HoLogoHtml@plainTeX{\HoLogoHtml@plainTeX@space}
%    \end{macrocode}
%    \end{macro}
%
% \subsubsection{\hologo{LaTeX}}
%
%    Source: \hologo{LaTeX} kernel.
%\begin{quote}
%\begin{verbatim}
%\DeclareRobustCommand{\LaTeX}{%
%  L%
%  \kern-.36em%
%  {%
%    \sbox\z@ T%
%    \vbox to\ht\z@{%
%      \hbox{%
%        \check@mathfonts
%        \fontsize\sf@size\z@
%        \math@fontsfalse
%        \selectfont
%        A%
%      }%
%      \vss
%    }%
%  }%
%  \kern-.15em%
%  \TeX
%}
%\end{verbatim}
%\end{quote}
%
%    \begin{macro}{\HoLogo@La}
%    \begin{macrocode}
\def\HoLogo@La#1{%
  L%
  \kern-.36em%
  \begingroup
    \setbox\ltx@zero\hbox{T}%
    \vbox to\ht\ltx@zero{%
      \hbox{%
        \ltx@ifundefined{check@mathfonts}{%
          \csname sevenrm\endcsname
        }{%
          \check@mathfonts
          \fontsize\sf@size{0pt}%
          \math@fontsfalse\selectfont
        }%
        A%
      }%
      \vss
    }%
  \endgroup
}
%    \end{macrocode}
%    \end{macro}
%
%    \begin{macro}{\HoLogo@LaTeX}
%    Source: \hologo{LaTeX} kernel.
%    \begin{macrocode}
\def\HoLogo@LaTeX#1{%
  \hologo{La}%
  \kern-.15em%
  \hologo{TeX}%
}
%    \end{macrocode}
%    \end{macro}
%    \begin{macro}{\HoLogoHtml@LaTeX}
%    \begin{macrocode}
\def\HoLogoHtml@LaTeX#1{%
  \HoLogoCss@LaTeX
  \HOLOGO@Span{LaTeX}{%
    L%
    \HOLOGO@Span{a}{%
      A%
    }%
    \hologo{TeX}%
  }%
}
%    \end{macrocode}
%    \end{macro}
%    \begin{macro}{\HoLogoCss@LaTeX}
%    \begin{macrocode}
\def\HoLogoCss@LaTeX{%
  \Css{%
    span.HoLogo-LaTeX span.HoLogo-a{%
      position:relative;%
      top:-.5ex;%
      margin-left:-.36em;%
      margin-right:-.15em;%
      font-size:85\%;%
    }%
  }%
  \global\let\HoLogoCss@LaTeX\relax
}
%    \end{macrocode}
%    \end{macro}
%
% \subsubsection{\hologo{(La)TeX}}
%
%    \begin{macro}{\HoLogo@LaTeXTeX}
%    The kerning around the parentheses is taken
%    from package \xpackage{dtklogos} \cite{dtklogos}.
%\begin{quote}
%\begin{verbatim}
%\DeclareRobustCommand{\LaTeXTeX}{%
%  (%
%  \kern-.15em%
%  L%
%  \kern-.36em%
%  {%
%    \sbox\z@ T%
%    \vbox to\ht0{%
%      \hbox{%
%        $\m@th$%
%        \csname S@\f@size\endcsname
%        \fontsize\sf@size\z@
%        \math@fontsfalse
%        \selectfont
%        A%
%      }%
%      \vss
%    }%
%  }%
%  \kern-.2em%
%  )%
%  \kern-.15em%
%  \TeX
%}
%\end{verbatim}
%\end{quote}
%    \begin{macrocode}
\def\HoLogo@LaTeXTeX#1{%
  (%
  \kern-.15em%
  \hologo{La}%
  \kern-.2em%
  )%
  \kern-.15em%
  \hologo{TeX}%
}
%    \end{macrocode}
%    \end{macro}
%    \begin{macro}{\HoLogoBkm@LaTeXTeX}
%    \begin{macrocode}
\def\HoLogoBkm@LaTeXTeX#1{(La)TeX}
%    \end{macrocode}
%    \end{macro}
%
%    \begin{macro}{\HoLogo@(La)TeX}
%    \begin{macrocode}
\expandafter
\let\csname HoLogo@(La)TeX\endcsname\HoLogo@LaTeXTeX
%    \end{macrocode}
%    \end{macro}
%    \begin{macro}{\HoLogoBkm@(La)TeX}
%    \begin{macrocode}
\expandafter
\let\csname HoLogoBkm@(La)TeX\endcsname\HoLogoBkm@LaTeXTeX
%    \end{macrocode}
%    \end{macro}
%    \begin{macro}{\HoLogoHtml@LaTeXTeX}
%    \begin{macrocode}
\def\HoLogoHtml@LaTeXTeX#1{%
  \HoLogoCss@LaTeXTeX
  \HOLOGO@Span{LaTeXTeX}{%
    (%
    \HOLOGO@Span{L}{L}%
    \HOLOGO@Span{a}{A}%
    \HOLOGO@Span{ParenRight}{)}%
    \hologo{TeX}%
  }%
}
%    \end{macrocode}
%    \end{macro}
%    \begin{macro}{\HoLogoHtml@(La)TeX}
%    Kerning after opening parentheses and before closing parentheses
%    is $-0.1$\,em. The original values $-0.15$\,em
%    looked too ugly for a serif font.
%    \begin{macrocode}
\expandafter
\let\csname HoLogoHtml@(La)TeX\endcsname\HoLogoHtml@LaTeXTeX
%    \end{macrocode}
%    \end{macro}
%    \begin{macro}{\HoLogoCss@LaTeXTeX}
%    \begin{macrocode}
\def\HoLogoCss@LaTeXTeX{%
  \Css{%
    span.HoLogo-LaTeXTeX span.HoLogo-L{%
      margin-left:-.1em;%
    }%
  }%
  \Css{%
    span.HoLogo-LaTeXTeX span.HoLogo-a{%
      position:relative;%
      top:-.5ex;%
      margin-left:-.36em;%
      margin-right:-.1em;%
      font-size:85\%;%
    }%
  }%
  \Css{%
    span.HoLogo-LaTeXTeX span.HoLogo-ParenRight{%
      margin-right:-.15em;%
    }%
  }%
  \global\let\HoLogoCss@LaTeXTeX\relax
}
%    \end{macrocode}
%    \end{macro}
%
% \subsubsection{\hologo{LaTeXe}}
%
%    \begin{macro}{\HoLogo@LaTeXe}
%    Source: \hologo{LaTeX} kernel
%    \begin{macrocode}
\def\HoLogo@LaTeXe#1{%
  \hologo{LaTeX}%
  \kern.15em%
  \hbox{%
    \HOLOGO@MathSetup
    2%
    $_{\textstyle\varepsilon}$%
  }%
}
%    \end{macrocode}
%    \end{macro}
%
%    \begin{macro}{\HoLogoCs@LaTeXe}
%    \begin{macrocode}
\ifnum64=`\^^^^0040\relax % test for big chars of LuaTeX/XeTeX
  \catcode`\$=9 %
  \catcode`\&=14 %
\else
  \catcode`\$=14 %
  \catcode`\&=9 %
\fi
\def\HoLogoCs@LaTeXe#1{%
  LaTeX2%
$ \string ^^^^0395%
& e%
}%
\catcode`\$=3 %
\catcode`\&=4 %
%    \end{macrocode}
%    \end{macro}
%
%    \begin{macro}{\HoLogoBkm@LaTeXe}
%    \begin{macrocode}
\def\HoLogoBkm@LaTeXe#1{%
  \hologo{LaTeX}%
  2%
  \HOLOGO@PdfdocUnicode{e}{\textepsilon}%
}
%    \end{macrocode}
%    \end{macro}
%
%    \begin{macro}{\HoLogoHtml@LaTeXe}
%    \begin{macrocode}
\def\HoLogoHtml@LaTeXe#1{%
  \HoLogoCss@LaTeXe
  \HOLOGO@Span{LaTeX2e}{%
    \hologo{LaTeX}%
    \HOLOGO@Span{2}{2}%
    \HOLOGO@Span{e}{%
      \HOLOGO@MathSetup
      \ensuremath{\textstyle\varepsilon}%
    }%
  }%
}
%    \end{macrocode}
%    \end{macro}
%    \begin{macro}{\HoLogoCss@LaTeXe}
%    \begin{macrocode}
\def\HoLogoCss@LaTeXe{%
  \Css{%
    span.HoLogo-LaTeX2e span.HoLogo-2{%
      padding-left:.15em;%
    }%
  }%
  \Css{%
    span.HoLogo-LaTeX2e span.HoLogo-e{%
      position:relative;%
      top:.35ex;%
      text-decoration:none;%
    }%
  }%
  \global\let\HoLogoCss@LaTeXe\relax
}
%    \end{macrocode}
%    \end{macro}
%
%    \begin{macro}{\HoLogo@LaTeX2e}
%    \begin{macrocode}
\expandafter
\let\csname HoLogo@LaTeX2e\endcsname\HoLogo@LaTeXe
%    \end{macrocode}
%    \end{macro}
%    \begin{macro}{\HoLogoCs@LaTeX2e}
%    \begin{macrocode}
\expandafter
\let\csname HoLogoCs@LaTeX2e\endcsname\HoLogoCs@LaTeXe
%    \end{macrocode}
%    \end{macro}
%    \begin{macro}{\HoLogoBkm@LaTeX2e}
%    \begin{macrocode}
\expandafter
\let\csname HoLogoBkm@LaTeX2e\endcsname\HoLogoBkm@LaTeXe
%    \end{macrocode}
%    \end{macro}
%    \begin{macro}{\HoLogoHtml@LaTeX2e}
%    \begin{macrocode}
\expandafter
\let\csname HoLogoHtml@LaTeX2e\endcsname\HoLogoHtml@LaTeXe
%    \end{macrocode}
%    \end{macro}
%
% \subsubsection{\hologo{LaTeX3}}
%
%    \begin{macro}{\HoLogo@LaTeX3}
%    Source: \hologo{LaTeX} kernel
%    \begin{macrocode}
\expandafter\def\csname HoLogo@LaTeX3\endcsname#1{%
  \hologo{LaTeX}%
  3%
}
%    \end{macrocode}
%    \end{macro}
%
%    \begin{macro}{\HoLogoBkm@LaTeX3}
%    \begin{macrocode}
\expandafter\def\csname HoLogoBkm@LaTeX3\endcsname#1{%
  \hologo{LaTeX}%
  3%
}
%    \end{macrocode}
%    \end{macro}
%    \begin{macro}{\HoLogoHtml@LaTeX3}
%    \begin{macrocode}
\expandafter
\let\csname HoLogoHtml@LaTeX3\expandafter\endcsname
\csname HoLogo@LaTeX3\endcsname
%    \end{macrocode}
%    \end{macro}
%
% \subsubsection{\hologo{LaTeXML}}
%
%    \begin{macro}{\HoLogo@LaTeXML}
%    \begin{macrocode}
\def\HoLogo@LaTeXML#1{%
  \HOLOGO@mbox{%
    \hologo{La}%
    \kern-.15em%
    T%
    \kern-.1667em%
    \lower.5ex\hbox{E}%
    \kern-.125em%
    \HoLogoFont@font{LaTeXML}{sc}{xml}%
  }%
}
%    \end{macrocode}
%    \end{macro}
%    \begin{macro}{\HoLogoHtml@pdfLaTeX}
%    \begin{macrocode}
\def\HoLogoHtml@LaTeXML#1{%
  \HOLOGO@Span{LaTeXML}{%
    \HoLogoCss@LaTeX
    \HoLogoCss@TeX
    \HOLOGO@Span{LaTeX}{%
      L%
      \HOLOGO@Span{a}{%
        A%
      }%
    }%
    \HOLOGO@Span{TeX}{%
      T%
      \HOLOGO@Span{e}{%
        E%
      }%
    }%
    \HCode{<span style="font-variant: small-caps;">}%
    xml%
    \HCode{</span>}%
  }%
}
%    \end{macrocode}
%    \end{macro}
%
% \subsubsection{\hologo{eTeX}}
%
%    \begin{macro}{\HoLogo@eTeX}
%    Source: package \xpackage{etex}
%    \begin{macrocode}
\def\HoLogo@eTeX#1{%
  \ltx@mbox{%
    \HOLOGO@MathSetup
    $\varepsilon$%
    -%
    \HOLOGO@NegativeKerning{-T,T-,To}%
    \hologo{TeX}%
  }%
}
%    \end{macrocode}
%    \end{macro}
%    \begin{macro}{\HoLogoCs@eTeX}
%    \begin{macrocode}
\ifnum64=`\^^^^0040\relax % test for big chars of LuaTeX/XeTeX
  \catcode`\$=9 %
  \catcode`\&=14 %
\else
  \catcode`\$=14 %
  \catcode`\&=9 %
\fi
\def\HoLogoCs@eTeX#1{%
$ #1{\string ^^^^0395}{\string ^^^^03b5}%
& #1{e}{E}%
  TeX%
}%
\catcode`\$=3 %
\catcode`\&=4 %
%    \end{macrocode}
%    \end{macro}
%    \begin{macro}{\HoLogoBkm@eTeX}
%    \begin{macrocode}
\def\HoLogoBkm@eTeX#1{%
  \HOLOGO@PdfdocUnicode{#1{e}{E}}{\textepsilon}%
  -%
  \hologo{TeX}%
}
%    \end{macrocode}
%    \end{macro}
%    \begin{macro}{\HoLogoHtml@eTeX}
%    \begin{macrocode}
\def\HoLogoHtml@eTeX#1{%
  \ltx@mbox{%
    \HOLOGO@MathSetup
    $\varepsilon$%
    -%
    \hologo{TeX}%
  }%
}
%    \end{macrocode}
%    \end{macro}
%
% \subsubsection{\hologo{iniTeX}}
%
%    \begin{macro}{\HoLogo@iniTeX}
%    \begin{macrocode}
\def\HoLogo@iniTeX#1{%
  \HOLOGO@mbox{%
    #1{i}{I}ni\hologo{TeX}%
  }%
}
%    \end{macrocode}
%    \end{macro}
%    \begin{macro}{\HoLogoCs@iniTeX}
%    \begin{macrocode}
\def\HoLogoCs@iniTeX#1{#1{i}{I}niTeX}
%    \end{macrocode}
%    \end{macro}
%    \begin{macro}{\HoLogoBkm@iniTeX}
%    \begin{macrocode}
\def\HoLogoBkm@iniTeX#1{%
  #1{i}{I}ni\hologo{TeX}%
}
%    \end{macrocode}
%    \end{macro}
%    \begin{macro}{\HoLogoHtml@iniTeX}
%    \begin{macrocode}
\let\HoLogoHtml@iniTeX\HoLogo@iniTeX
%    \end{macrocode}
%    \end{macro}
%
% \subsubsection{\hologo{virTeX}}
%
%    \begin{macro}{\HoLogo@virTeX}
%    \begin{macrocode}
\def\HoLogo@virTeX#1{%
  \HOLOGO@mbox{%
    #1{v}{V}ir\hologo{TeX}%
  }%
}
%    \end{macrocode}
%    \end{macro}
%    \begin{macro}{\HoLogoCs@virTeX}
%    \begin{macrocode}
\def\HoLogoCs@virTeX#1{#1{v}{V}irTeX}
%    \end{macrocode}
%    \end{macro}
%    \begin{macro}{\HoLogoBkm@virTeX}
%    \begin{macrocode}
\def\HoLogoBkm@virTeX#1{%
  #1{v}{V}ir\hologo{TeX}%
}
%    \end{macrocode}
%    \end{macro}
%    \begin{macro}{\HoLogoHtml@virTeX}
%    \begin{macrocode}
\let\HoLogoHtml@virTeX\HoLogo@virTeX
%    \end{macrocode}
%    \end{macro}
%
% \subsubsection{\hologo{SliTeX}}
%
% \paragraph{Definitions of the three variants.}
%
%    \begin{macro}{\HoLogo@SLiTeX@lift}
%    \begin{macrocode}
\def\HoLogo@SLiTeX@lift#1{%
  \HoLogoFont@font{SliTeX}{rm}{%
    S%
    \kern-.06em%
    L%
    \kern-.18em%
    \raise.32ex\hbox{\HoLogoFont@font{SliTeX}{sc}{i}}%
    \HOLOGO@discretionary
    \kern-.06em%
    \hologo{TeX}%
  }%
}
%    \end{macrocode}
%    \end{macro}
%    \begin{macro}{\HoLogoBkm@SLiTeX@lift}
%    \begin{macrocode}
\def\HoLogoBkm@SLiTeX@lift#1{SLiTeX}
%    \end{macrocode}
%    \end{macro}
%    \begin{macro}{\HoLogoHtml@SLiTeX@lift}
%    \begin{macrocode}
\def\HoLogoHtml@SLiTeX@lift#1{%
  \HoLogoCss@SLiTeX@lift
  \HOLOGO@Span{SLiTeX-lift}{%
    \HoLogoFont@font{SliTeX}{rm}{%
      S%
      \HOLOGO@Span{L}{L}%
      \HOLOGO@Span{i}{i}%
      \hologo{TeX}%
    }%
  }%
}
%    \end{macrocode}
%    \end{macro}
%    \begin{macro}{\HoLogoCss@SLiTeX@lift}
%    \begin{macrocode}
\def\HoLogoCss@SLiTeX@lift{%
  \Css{%
    span.HoLogo-SLiTeX-lift span.HoLogo-L{%
      margin-left:-.06em;%
      margin-right:-.18em;%
    }%
  }%
  \Css{%
    span.HoLogo-SLiTeX-lift span.HoLogo-i{%
      position:relative;%
      top:-.32ex;%
      margin-right:-.06em;%
      font-variant:small-caps;%
    }%
  }%
  \global\let\HoLogoCss@SLiTeX@lift\relax
}
%    \end{macrocode}
%    \end{macro}
%
%    \begin{macro}{\HoLogo@SliTeX@simple}
%    \begin{macrocode}
\def\HoLogo@SliTeX@simple#1{%
  \HoLogoFont@font{SliTeX}{rm}{%
    \ltx@mbox{%
      \HoLogoFont@font{SliTeX}{sc}{Sli}%
    }%
    \HOLOGO@discretionary
    \hologo{TeX}%
  }%
}
%    \end{macrocode}
%    \end{macro}
%    \begin{macro}{\HoLogoBkm@SliTeX@simple}
%    \begin{macrocode}
\def\HoLogoBkm@SliTeX@simple#1{SliTeX}
%    \end{macrocode}
%    \end{macro}
%    \begin{macro}{\HoLogoHtml@SliTeX@simple}
%    \begin{macrocode}
\let\HoLogoHtml@SliTeX@simple\HoLogo@SliTeX@simple
%    \end{macrocode}
%    \end{macro}
%
%    \begin{macro}{\HoLogo@SliTeX@narrow}
%    \begin{macrocode}
\def\HoLogo@SliTeX@narrow#1{%
  \HoLogoFont@font{SliTeX}{rm}{%
    \ltx@mbox{%
      S%
      \kern-.06em%
      \HoLogoFont@font{SliTeX}{sc}{%
        l%
        \kern-.035em%
        i%
      }%
    }%
    \HOLOGO@discretionary
    \kern-.06em%
    \hologo{TeX}%
  }%
}
%    \end{macrocode}
%    \end{macro}
%    \begin{macro}{\HoLogoBkm@SliTeX@narrow}
%    \begin{macrocode}
\def\HoLogoBkm@SliTeX@narrow#1{SliTeX}
%    \end{macrocode}
%    \end{macro}
%    \begin{macro}{\HoLogoHtml@SliTeX@narrow}
%    \begin{macrocode}
\def\HoLogoHtml@SliTeX@narrow#1{%
  \HoLogoCss@SliTeX@narrow
  \HOLOGO@Span{SliTeX-narrow}{%
    \HoLogoFont@font{SliTeX}{rm}{%
      S%
        \HOLOGO@Span{l}{l}%
        \HOLOGO@Span{i}{i}%
      \hologo{TeX}%
    }%
  }%
}
%    \end{macrocode}
%    \end{macro}
%    \begin{macro}{\HoLogoCss@SliTeX@narrow}
%    \begin{macrocode}
\def\HoLogoCss@SliTeX@narrow{%
  \Css{%
    span.HoLogo-SliTeX-narrow span.HoLogo-l{%
      margin-left:-.06em;%
      margin-right:-.035em;%
      font-variant:small-caps;%
    }%
  }%
  \Css{%
    span.HoLogo-SliTeX-narrow span.HoLogo-i{%
      margin-right:-.06em;%
      font-variant:small-caps;%
    }%
  }%
  \global\let\HoLogoCss@SliTeX@narrow\relax
}
%    \end{macrocode}
%    \end{macro}
%
% \paragraph{Macro set completion.}
%
%    \begin{macro}{\HoLogo@SLiTeX@simple}
%    \begin{macrocode}
\def\HoLogo@SLiTeX@simple{\HoLogo@SliTeX@simple}
%    \end{macrocode}
%    \end{macro}
%    \begin{macro}{\HoLogoBkm@SLiTeX@simple}
%    \begin{macrocode}
\def\HoLogoBkm@SLiTeX@simple{\HoLogoBkm@SliTeX@simple}
%    \end{macrocode}
%    \end{macro}
%    \begin{macro}{\HoLogoHtml@SLiTeX@simple}
%    \begin{macrocode}
\def\HoLogoHtml@SLiTeX@simple{\HoLogoHtml@SliTeX@simple}
%    \end{macrocode}
%    \end{macro}
%
%    \begin{macro}{\HoLogo@SLiTeX@narrow}
%    \begin{macrocode}
\def\HoLogo@SLiTeX@narrow{\HoLogo@SliTeX@narrow}
%    \end{macrocode}
%    \end{macro}
%    \begin{macro}{\HoLogoBkm@SLiTeX@narrow}
%    \begin{macrocode}
\def\HoLogoBkm@SLiTeX@narrow{\HoLogoBkm@SliTeX@narrow}
%    \end{macrocode}
%    \end{macro}
%    \begin{macro}{\HoLogoHtml@SLiTeX@narrow}
%    \begin{macrocode}
\def\HoLogoHtml@SLiTeX@narrow{\HoLogoHtml@SliTeX@narrow}
%    \end{macrocode}
%    \end{macro}
%
%    \begin{macro}{\HoLogo@SliTeX@lift}
%    \begin{macrocode}
\def\HoLogo@SliTeX@lift{\HoLogo@SLiTeX@lift}
%    \end{macrocode}
%    \end{macro}
%    \begin{macro}{\HoLogoBkm@SliTeX@lift}
%    \begin{macrocode}
\def\HoLogoBkm@SliTeX@lift{\HoLogoBkm@SLiTeX@lift}
%    \end{macrocode}
%    \end{macro}
%    \begin{macro}{\HoLogoHtml@SliTeX@lift}
%    \begin{macrocode}
\def\HoLogoHtml@SliTeX@lift{\HoLogoHtml@SLiTeX@lift}
%    \end{macrocode}
%    \end{macro}
%
% \paragraph{Defaults.}
%
%    \begin{macro}{\HoLogo@SLiTeX}
%    \begin{macrocode}
\def\HoLogo@SLiTeX{\HoLogo@SLiTeX@lift}
%    \end{macrocode}
%    \end{macro}
%    \begin{macro}{\HoLogoBkm@SLiTeX}
%    \begin{macrocode}
\def\HoLogoBkm@SLiTeX{\HoLogoBkm@SLiTeX@lift}
%    \end{macrocode}
%    \end{macro}
%    \begin{macro}{\HoLogoHtml@SLiTeX}
%    \begin{macrocode}
\def\HoLogoHtml@SLiTeX{\HoLogoHtml@SLiTeX@lift}
%    \end{macrocode}
%    \end{macro}
%
%    \begin{macro}{\HoLogo@SliTeX}
%    \begin{macrocode}
\def\HoLogo@SliTeX{\HoLogo@SliTeX@narrow}
%    \end{macrocode}
%    \end{macro}
%    \begin{macro}{\HoLogoBkm@SliTeX}
%    \begin{macrocode}
\def\HoLogoBkm@SliTeX{\HoLogoBkm@SliTeX@narrow}
%    \end{macrocode}
%    \end{macro}
%    \begin{macro}{\HoLogoHtml@SliTeX}
%    \begin{macrocode}
\def\HoLogoHtml@SliTeX{\HoLogoHtml@SliTeX@narrow}
%    \end{macrocode}
%    \end{macro}
%
% \subsubsection{\hologo{LuaTeX}}
%
%    \begin{macro}{\HoLogo@LuaTeX}
%    The kerning is an idea of Hans Hagen, see mailing list
%    `luatex at tug dot org' in March 2010.
%    \begin{macrocode}
\def\HoLogo@LuaTeX#1{%
  \HOLOGO@mbox{%
    Lua%
    \HOLOGO@NegativeKerning{aT,oT,To}%
    \hologo{TeX}%
  }%
}
%    \end{macrocode}
%    \end{macro}
%    \begin{macro}{\HoLogoHtml@LuaTeX}
%    \begin{macrocode}
\let\HoLogoHtml@LuaTeX\HoLogo@LuaTeX
%    \end{macrocode}
%    \end{macro}
%
% \subsubsection{\hologo{LuaLaTeX}}
%
%    \begin{macro}{\HoLogo@LuaLaTeX}
%    \begin{macrocode}
\def\HoLogo@LuaLaTeX#1{%
  \HOLOGO@mbox{%
    Lua%
    \hologo{LaTeX}%
  }%
}
%    \end{macrocode}
%    \end{macro}
%    \begin{macro}{\HoLogoHtml@LuaLaTeX}
%    \begin{macrocode}
\let\HoLogoHtml@LuaLaTeX\HoLogo@LuaLaTeX
%    \end{macrocode}
%    \end{macro}
%
% \subsubsection{\hologo{XeTeX}, \hologo{XeLaTeX}}
%
%    \begin{macro}{\HOLOGO@IfCharExists}
%    \begin{macrocode}
\ifluatex
  \ifnum\luatexversion<36 %
  \else
    \def\HOLOGO@IfCharExists#1{%
      \ifnum
        \directlua{%
           if luaotfload and luaotfload.aux then
             if luaotfload.aux.font_has_glyph(%
                    font.current(), \number#1) then % 	 
	       tex.print("1") % 	 
	     end % 	 
	   elseif font and font.fonts and font.current then %
            local f = font.fonts[font.current()]%
            if f.characters and f.characters[\number#1] then %
              tex.print("1")%
            end %
          end%
        }0=\ltx@zero
        \expandafter\ltx@secondoftwo
      \else
        \expandafter\ltx@firstoftwo
      \fi
    }%
  \fi
\fi
\ltx@IfUndefined{HOLOGO@IfCharExists}{%
  \def\HOLOGO@@IfCharExists#1{%
    \begingroup
      \tracinglostchars=\ltx@zero
      \setbox\ltx@zero=\hbox{%
        \kern7sp\char#1\relax
        \ifnum\lastkern>\ltx@zero
          \expandafter\aftergroup\csname iffalse\endcsname
        \else
          \expandafter\aftergroup\csname iftrue\endcsname
        \fi
      }%
      % \if{true|false} from \aftergroup
      \endgroup
      \expandafter\ltx@firstoftwo
    \else
      \endgroup
      \expandafter\ltx@secondoftwo
    \fi
  }%
  \ifxetex
    \ltx@IfUndefined{XeTeXfonttype}{}{%
      \ltx@IfUndefined{XeTeXcharglyph}{}{%
        \def\HOLOGO@IfCharExists#1{%
          \ifnum\XeTeXfonttype\font>\ltx@zero
            \expandafter\ltx@firstofthree
          \else
            \expandafter\ltx@gobble
          \fi
          {%
            \ifnum\XeTeXcharglyph#1>\ltx@zero
              \expandafter\ltx@firstoftwo
            \else
              \expandafter\ltx@secondoftwo
            \fi
          }%
          \HOLOGO@@IfCharExists{#1}%
        }%
      }%
    }%
  \fi
}{}
\ltx@ifundefined{HOLOGO@IfCharExists}{%
  \ifnum64=`\^^^^0040\relax % test for big chars of LuaTeX/XeTeX
    \let\HOLOGO@IfCharExists\HOLOGO@@IfCharExists
  \else
    \def\HOLOGO@IfCharExists#1{%
      \ifnum#1>255 %
        \expandafter\ltx@fourthoffour
      \fi
      \HOLOGO@@IfCharExists{#1}%
    }%
  \fi
}{}
%    \end{macrocode}
%    \end{macro}
%
%    \begin{macro}{\HoLogo@Xe}
%    Source: package \xpackage{dtklogos}
%    \begin{macrocode}
\def\HoLogo@Xe#1{%
  X%
  \kern-.1em\relax
  \HOLOGO@IfCharExists{"018E}{%
    \lower.5ex\hbox{\char"018E}%
  }{%
    \chardef\HOLOGO@choice=\ltx@zero
    \ifdim\fontdimen\ltx@one\font>0pt %
      \ltx@IfUndefined{rotatebox}{%
        \ltx@IfUndefined{pgftext}{%
          \ltx@IfUndefined{psscalebox}{%
            \ltx@IfUndefined{HOLOGO@ScaleBox@\hologoDriver}{%
            }{%
              \chardef\HOLOGO@choice=4 %
            }%
          }{%
            \chardef\HOLOGO@choice=3 %
          }%
        }{%
          \chardef\HOLOGO@choice=2 %
        }%
      }{%
        \chardef\HOLOGO@choice=1 %
      }%
      \ifcase\HOLOGO@choice
        \HOLOGO@WarningUnsupportedDriver{Xe}%
        e%
      \or % 1: \rotatebox
        \begingroup
          \setbox\ltx@zero\hbox{\rotatebox{180}{E}}%
          \ltx@LocDimenA=\dp\ltx@zero
          \advance\ltx@LocDimenA by -.5ex\relax
          \raise\ltx@LocDimenA\box\ltx@zero
        \endgroup
      \or % 2: \pgftext
        \lower.5ex\hbox{%
          \pgfpicture
            \pgftext[rotate=180]{E}%
          \endpgfpicture
        }%
      \or % 3: \psscalebox
        \begingroup
          \setbox\ltx@zero\hbox{\psscalebox{-1 -1}{E}}%
          \ltx@LocDimenA=\dp\ltx@zero
          \advance\ltx@LocDimenA by -.5ex\relax
          \raise\ltx@LocDimenA\box\ltx@zero
        \endgroup
      \or % 4: \HOLOGO@PointReflectBox
        \lower.5ex\hbox{\HOLOGO@PointReflectBox{E}}%
      \else
        \@PackageError{hologo}{Internal error (choice/it}\@ehc
      \fi
    \else
      \ltx@IfUndefined{reflectbox}{%
        \ltx@IfUndefined{pgftext}{%
          \ltx@IfUndefined{psscalebox}{%
            \ltx@IfUndefined{HOLOGO@ScaleBox@\hologoDriver}{%
            }{%
              \chardef\HOLOGO@choice=4 %
            }%
          }{%
            \chardef\HOLOGO@choice=3 %
          }%
        }{%
          \chardef\HOLOGO@choice=2 %
        }%
      }{%
        \chardef\HOLOGO@choice=1 %
      }%
      \ifcase\HOLOGO@choice
        \HOLOGO@WarningUnsupportedDriver{Xe}%
        e%
      \or % 1: reflectbox
        \lower.5ex\hbox{%
          \reflectbox{E}%
        }%
      \or % 2: \pgftext
        \lower.5ex\hbox{%
          \pgfpicture
            \pgftransformxscale{-1}%
            \pgftext{E}%
          \endpgfpicture
        }%
      \or % 3: \psscalebox
        \lower.5ex\hbox{%
          \psscalebox{-1 1}{E}%
        }%
      \or % 4: \HOLOGO@Reflectbox
        \lower.5ex\hbox{%
          \HOLOGO@ReflectBox{E}%
        }%
      \else
        \@PackageError{hologo}{Internal error (choice/up)}\@ehc
      \fi
    \fi
  }%
}
%    \end{macrocode}
%    \end{macro}
%    \begin{macro}{\HoLogoHtml@Xe}
%    \begin{macrocode}
\def\HoLogoHtml@Xe#1{%
  \HoLogoCss@Xe
  \HOLOGO@Span{Xe}{%
    X%
    \HOLOGO@Span{e}{%
      \HCode{&\ltx@hashchar x018e;}%
    }%
  }%
}
%    \end{macrocode}
%    \end{macro}
%    \begin{macro}{\HoLogoCss@Xe}
%    \begin{macrocode}
\def\HoLogoCss@Xe{%
  \Css{%
    span.HoLogo-Xe span.HoLogo-e{%
      position:relative;%
      top:.5ex;%
      left-margin:-.1em;%
    }%
  }%
  \global\let\HoLogoCss@Xe\relax
}
%    \end{macrocode}
%    \end{macro}
%
%    \begin{macro}{\HoLogo@XeTeX}
%    \begin{macrocode}
\def\HoLogo@XeTeX#1{%
  \hologo{Xe}%
  \kern-.15em\relax
  \hologo{TeX}%
}
%    \end{macrocode}
%    \end{macro}
%
%    \begin{macro}{\HoLogoHtml@XeTeX}
%    \begin{macrocode}
\def\HoLogoHtml@XeTeX#1{%
  \HoLogoCss@XeTeX
  \HOLOGO@Span{XeTeX}{%
    \hologo{Xe}%
    \hologo{TeX}%
  }%
}
%    \end{macrocode}
%    \end{macro}
%    \begin{macro}{\HoLogoCss@XeTeX}
%    \begin{macrocode}
\def\HoLogoCss@XeTeX{%
  \Css{%
    span.HoLogo-XeTeX span.HoLogo-TeX{%
      margin-left:-.15em;%
    }%
  }%
  \global\let\HoLogoCss@XeTeX\relax
}
%    \end{macrocode}
%    \end{macro}
%
%    \begin{macro}{\HoLogo@XeLaTeX}
%    \begin{macrocode}
\def\HoLogo@XeLaTeX#1{%
  \hologo{Xe}%
  \kern-.13em%
  \hologo{LaTeX}%
}
%    \end{macrocode}
%    \end{macro}
%    \begin{macro}{\HoLogoHtml@XeLaTeX}
%    \begin{macrocode}
\def\HoLogoHtml@XeLaTeX#1{%
  \HoLogoCss@XeLaTeX
  \HOLOGO@Span{XeLaTeX}{%
    \hologo{Xe}%
    \hologo{LaTeX}%
  }%
}
%    \end{macrocode}
%    \end{macro}
%    \begin{macro}{\HoLogoCss@XeLaTeX}
%    \begin{macrocode}
\def\HoLogoCss@XeLaTeX{%
  \Css{%
    span.HoLogo-XeLaTeX span.HoLogo-Xe{%
      margin-right:-.13em;%
    }%
  }%
  \global\let\HoLogoCss@XeLaTeX\relax
}
%    \end{macrocode}
%    \end{macro}
%
% \subsubsection{\hologo{pdfTeX}, \hologo{pdfLaTeX}}
%
%    \begin{macro}{\HoLogo@pdfTeX}
%    \begin{macrocode}
\def\HoLogo@pdfTeX#1{%
  \HOLOGO@mbox{%
    #1{p}{P}df\hologo{TeX}%
  }%
}
%    \end{macrocode}
%    \end{macro}
%    \begin{macro}{\HoLogoCs@pdfTeX}
%    \begin{macrocode}
\def\HoLogoCs@pdfTeX#1{#1{p}{P}dfTeX}
%    \end{macrocode}
%    \end{macro}
%    \begin{macro}{\HoLogoBkm@pdfTeX}
%    \begin{macrocode}
\def\HoLogoBkm@pdfTeX#1{%
  #1{p}{P}df\hologo{TeX}%
}
%    \end{macrocode}
%    \end{macro}
%    \begin{macro}{\HoLogoHtml@pdfTeX}
%    \begin{macrocode}
\let\HoLogoHtml@pdfTeX\HoLogo@pdfTeX
%    \end{macrocode}
%    \end{macro}
%
%    \begin{macro}{\HoLogo@pdfLaTeX}
%    \begin{macrocode}
\def\HoLogo@pdfLaTeX#1{%
  \HOLOGO@mbox{%
    #1{p}{P}df\hologo{LaTeX}%
  }%
}
%    \end{macrocode}
%    \end{macro}
%    \begin{macro}{\HoLogoCs@pdfLaTeX}
%    \begin{macrocode}
\def\HoLogoCs@pdfLaTeX#1{#1{p}{P}dfLaTeX}
%    \end{macrocode}
%    \end{macro}
%    \begin{macro}{\HoLogoBkm@pdfLaTeX}
%    \begin{macrocode}
\def\HoLogoBkm@pdfLaTeX#1{%
  #1{p}{P}df\hologo{LaTeX}%
}
%    \end{macrocode}
%    \end{macro}
%    \begin{macro}{\HoLogoHtml@pdfLaTeX}
%    \begin{macrocode}
\let\HoLogoHtml@pdfLaTeX\HoLogo@pdfLaTeX
%    \end{macrocode}
%    \end{macro}
%
% \subsubsection{\hologo{VTeX}}
%
%    \begin{macro}{\HoLogo@VTeX}
%    \begin{macrocode}
\def\HoLogo@VTeX#1{%
  \HOLOGO@mbox{%
    V\hologo{TeX}%
  }%
}
%    \end{macrocode}
%    \end{macro}
%    \begin{macro}{\HoLogoHtml@VTeX}
%    \begin{macrocode}
\let\HoLogoHtml@VTeX\HoLogo@VTeX
%    \end{macrocode}
%    \end{macro}
%
% \subsubsection{\hologo{AmS}, \dots}
%
%    Source: class \xclass{amsdtx}
%
%    \begin{macro}{\HoLogo@AmS}
%    \begin{macrocode}
\def\HoLogo@AmS#1{%
  \HoLogoFont@font{AmS}{sy}{%
    A%
    \kern-.1667em%
    \lower.5ex\hbox{M}%
    \kern-.125em%
    S%
  }%
}
%    \end{macrocode}
%    \end{macro}
%    \begin{macro}{\HoLogoBkm@AmS}
%    \begin{macrocode}
\def\HoLogoBkm@AmS#1{AmS}
%    \end{macrocode}
%    \end{macro}
%    \begin{macro}{\HoLogoHtml@AmS}
%    \begin{macrocode}
\def\HoLogoHtml@AmS#1{%
  \HoLogoCss@AmS
%  \HoLogoFont@font{AmS}{sy}{%
    \HOLOGO@Span{AmS}{%
      A%
      \HOLOGO@Span{M}{M}%
      S%
    }%
%   }%
}
%    \end{macrocode}
%    \end{macro}
%    \begin{macro}{\HoLogoCss@AmS}
%    \begin{macrocode}
\def\HoLogoCss@AmS{%
  \Css{%
    span.HoLogo-AmS span.HoLogo-M{%
      position:relative;%
      top:.5ex;%
      margin-left:-.1667em;%
      margin-right:-.125em;%
      text-decoration:none;%
    }%
  }%
  \global\let\HoLogoCss@AmS\relax
}
%    \end{macrocode}
%    \end{macro}
%
%    \begin{macro}{\HoLogo@AmSTeX}
%    \begin{macrocode}
\def\HoLogo@AmSTeX#1{%
  \hologo{AmS}%
  \HOLOGO@hyphen
  \hologo{TeX}%
}
%    \end{macrocode}
%    \end{macro}
%    \begin{macro}{\HoLogoBkm@AmSTeX}
%    \begin{macrocode}
\def\HoLogoBkm@AmSTeX#1{AmS-TeX}%
%    \end{macrocode}
%    \end{macro}
%    \begin{macro}{\HoLogoHtml@AmSTeX}
%    \begin{macrocode}
\let\HoLogoHtml@AmSTeX\HoLogo@AmSTeX
%    \end{macrocode}
%    \end{macro}
%
%    \begin{macro}{\HoLogo@AmSLaTeX}
%    \begin{macrocode}
\def\HoLogo@AmSLaTeX#1{%
  \hologo{AmS}%
  \HOLOGO@hyphen
  \hologo{LaTeX}%
}
%    \end{macrocode}
%    \end{macro}
%    \begin{macro}{\HoLogoBkm@AmSLaTeX}
%    \begin{macrocode}
\def\HoLogoBkm@AmSLaTeX#1{AmS-LaTeX}%
%    \end{macrocode}
%    \end{macro}
%    \begin{macro}{\HoLogoHtml@AmSLaTeX}
%    \begin{macrocode}
\let\HoLogoHtml@AmSLaTeX\HoLogo@AmSLaTeX
%    \end{macrocode}
%    \end{macro}
%
% \subsubsection{\hologo{BibTeX}}
%
%    \begin{macro}{\HoLogo@BibTeX@sc}
%    A definition of \hologo{BibTeX} is provided in
%    the documentation source for the manual of \hologo{BibTeX}
%    \cite{btxdoc}.
%\begin{quote}
%\begin{verbatim}
%\def\BibTeX{%
%  {%
%    \rm
%    B%
%    \kern-.05em%
%    {%
%      \sc
%      i%
%      \kern-.025em %
%      b%
%    }%
%    \kern-.08em
%    T%
%    \kern-.1667em%
%    \lower.7ex\hbox{E}%
%    \kern-.125em%
%    X%
%  }%
%}
%\end{verbatim}
%\end{quote}
%    \begin{macrocode}
\def\HoLogo@BibTeX@sc#1{%
  B%
  \kern-.05em%
  \HoLogoFont@font{BibTeX}{sc}{%
    i%
    \kern-.025em%
    b%
  }%
  \HOLOGO@discretionary
  \kern-.08em%
  \hologo{TeX}%
}
%    \end{macrocode}
%    \end{macro}
%    \begin{macro}{\HoLogoHtml@BibTeX@sc}
%    \begin{macrocode}
\def\HoLogoHtml@BibTeX@sc#1{%
  \HoLogoCss@BibTeX@sc
  \HOLOGO@Span{BibTeX-sc}{%
    B%
    \HOLOGO@Span{i}{i}%
    \HOLOGO@Span{b}{b}%
    \hologo{TeX}%
  }%
}
%    \end{macrocode}
%    \end{macro}
%    \begin{macro}{\HoLogoCss@BibTeX@sc}
%    \begin{macrocode}
\def\HoLogoCss@BibTeX@sc{%
  \Css{%
    span.HoLogo-BibTeX-sc span.HoLogo-i{%
      margin-left:-.05em;%
      margin-right:-.025em;%
      font-variant:small-caps;%
    }%
  }%
  \Css{%
    span.HoLogo-BibTeX-sc span.HoLogo-b{%
      margin-right:-.08em;%
      font-variant:small-caps;%
    }%
  }%
  \global\let\HoLogoCss@BibTeX@sc\relax
}
%    \end{macrocode}
%    \end{macro}
%
%    \begin{macro}{\HoLogo@BibTeX@sf}
%    Variant \xoption{sf} avoids trouble with unavailable
%    small caps fonts (e.g., bold versions of Computer Modern or
%    Latin Modern). The definition is taken from
%    package \xpackage{dtklogos} \cite{dtklogos}.
%\begin{quote}
%\begin{verbatim}
%\DeclareRobustCommand{\BibTeX}{%
%  B%
%  \kern-.05em%
%  \hbox{%
%    $\m@th$% %% force math size calculations
%    \csname S@\f@size\endcsname
%    \fontsize\sf@size\z@
%    \math@fontsfalse
%    \selectfont
%    I%
%    \kern-.025em%
%    B
%  }%
%  \kern-.08em%
%  \-%
%  \TeX
%}
%\end{verbatim}
%\end{quote}
%    \begin{macrocode}
\def\HoLogo@BibTeX@sf#1{%
  B%
  \kern-.05em%
  \HoLogoFont@font{BibTeX}{bibsf}{%
    I%
    \kern-.025em%
    B%
  }%
  \HOLOGO@discretionary
  \kern-.08em%
  \hologo{TeX}%
}
%    \end{macrocode}
%    \end{macro}
%    \begin{macro}{\HoLogoHtml@BibTeX@sf}
%    \begin{macrocode}
\def\HoLogoHtml@BibTeX@sf#1{%
  \HoLogoCss@BibTeX@sf
  \HOLOGO@Span{BibTeX-sf}{%
    B%
    \HoLogoFont@font{BibTeX}{bibsf}{%
      \HOLOGO@Span{i}{I}%
      B%
    }%
    \hologo{TeX}%
  }%
}
%    \end{macrocode}
%    \end{macro}
%    \begin{macro}{\HoLogoCss@BibTeX@sf}
%    \begin{macrocode}
\def\HoLogoCss@BibTeX@sf{%
  \Css{%
    span.HoLogo-BibTeX-sf span.HoLogo-i{%
      margin-left:-.05em;%
      margin-right:-.025em;%
    }%
  }%
  \Css{%
    span.HoLogo-BibTeX-sf span.HoLogo-TeX{%
      margin-left:-.08em;%
    }%
  }%
  \global\let\HoLogoCss@BibTeX@sf\relax
}
%    \end{macrocode}
%    \end{macro}
%
%    \begin{macro}{\HoLogo@BibTeX}
%    \begin{macrocode}
\def\HoLogo@BibTeX{\HoLogo@BibTeX@sf}
%    \end{macrocode}
%    \end{macro}
%    \begin{macro}{\HoLogoHtml@BibTeX}
%    \begin{macrocode}
\def\HoLogoHtml@BibTeX{\HoLogoHtml@BibTeX@sf}
%    \end{macrocode}
%    \end{macro}
%
% \subsubsection{\hologo{BibTeX8}}
%
%    \begin{macro}{\HoLogo@BibTeX8}
%    \begin{macrocode}
\expandafter\def\csname HoLogo@BibTeX8\endcsname#1{%
  \hologo{BibTeX}%
  8%
}
%    \end{macrocode}
%    \end{macro}
%
%    \begin{macro}{\HoLogoBkm@BibTeX8}
%    \begin{macrocode}
\expandafter\def\csname HoLogoBkm@BibTeX8\endcsname#1{%
  \hologo{BibTeX}%
  8%
}
%    \end{macrocode}
%    \end{macro}
%    \begin{macro}{\HoLogoHtml@BibTeX8}
%    \begin{macrocode}
\expandafter
\let\csname HoLogoHtml@BibTeX8\expandafter\endcsname
\csname HoLogo@BibTeX8\endcsname
%    \end{macrocode}
%    \end{macro}
%
% \subsubsection{\hologo{ConTeXt}}
%
%    \begin{macro}{\HoLogo@ConTeXt@simple}
%    \begin{macrocode}
\def\HoLogo@ConTeXt@simple#1{%
  \HOLOGO@mbox{Con}%
  \HOLOGO@discretionary
  \HOLOGO@mbox{\hologo{TeX}t}%
}
%    \end{macrocode}
%    \end{macro}
%    \begin{macro}{\HoLogoHtml@ConTeXt@simple}
%    \begin{macrocode}
\let\HoLogoHtml@ConTeXt@simple\HoLogo@ConTeXt@simple
%    \end{macrocode}
%    \end{macro}
%
%    \begin{macro}{\HoLogo@ConTeXt@narrow}
%    This definition of logo \hologo{ConTeXt} with variant \xoption{narrow}
%    comes from TUGboat's class \xclass{ltugboat} (version 2010/11/15 v2.8).
%    \begin{macrocode}
\def\HoLogo@ConTeXt@narrow#1{%
  \HOLOGO@mbox{C\kern-.0333emon}%
  \HOLOGO@discretionary
  \kern-.0667em%
  \HOLOGO@mbox{\hologo{TeX}\kern-.0333emt}%
}
%    \end{macrocode}
%    \end{macro}
%    \begin{macro}{\HoLogoHtml@ConTeXt@narrow}
%    \begin{macrocode}
\def\HoLogoHtml@ConTeXt@narrow#1{%
  \HoLogoCss@ConTeXt@narrow
  \HOLOGO@Span{ConTeXt-narrow}{%
    \HOLOGO@Span{C}{C}%
    on%
    \hologo{TeX}%
    t%
  }%
}
%    \end{macrocode}
%    \end{macro}
%    \begin{macro}{\HoLogoCss@ConTeXt@narrow}
%    \begin{macrocode}
\def\HoLogoCss@ConTeXt@narrow{%
  \Css{%
    span.HoLogo-ConTeXt-narrow span.HoLogo-C{%
      margin-left:-.0333em;%
    }%
  }%
  \Css{%
    span.HoLogo-ConTeXt-narrow span.HoLogo-TeX{%
      margin-left:-.0667em;%
      margin-right:-.0333em;%
    }%
  }%
  \global\let\HoLogoCss@ConTeXt@narrow\relax
}
%    \end{macrocode}
%    \end{macro}
%
%    \begin{macro}{\HoLogo@ConTeXt}
%    \begin{macrocode}
\def\HoLogo@ConTeXt{\HoLogo@ConTeXt@narrow}
%    \end{macrocode}
%    \end{macro}
%    \begin{macro}{\HoLogoHtml@ConTeXt}
%    \begin{macrocode}
\def\HoLogoHtml@ConTeXt{\HoLogoHtml@ConTeXt@narrow}
%    \end{macrocode}
%    \end{macro}
%
% \subsubsection{\hologo{emTeX}}
%
%    \begin{macro}{\HoLogo@emTeX}
%    \begin{macrocode}
\def\HoLogo@emTeX#1{%
  \HOLOGO@mbox{#1{e}{E}m}%
  \HOLOGO@discretionary
  \hologo{TeX}%
}
%    \end{macrocode}
%    \end{macro}
%    \begin{macro}{\HoLogoCs@emTeX}
%    \begin{macrocode}
\def\HoLogoCs@emTeX#1{#1{e}{E}mTeX}%
%    \end{macrocode}
%    \end{macro}
%    \begin{macro}{\HoLogoBkm@emTeX}
%    \begin{macrocode}
\def\HoLogoBkm@emTeX#1{%
  #1{e}{E}m\hologo{TeX}%
}
%    \end{macrocode}
%    \end{macro}
%    \begin{macro}{\HoLogoHtml@emTeX}
%    \begin{macrocode}
\let\HoLogoHtml@emTeX\HoLogo@emTeX
%    \end{macrocode}
%    \end{macro}
%
% \subsubsection{\hologo{ExTeX}}
%
%    \begin{macro}{\HoLogo@ExTeX}
%    The definition is taken from the FAQ of the
%    project \hologo{ExTeX}
%    \cite{ExTeX-FAQ}.
%\begin{quote}
%\begin{verbatim}
%\def\ExTeX{%
%  \textrm{% Logo always with serifs
%    \ensuremath{%
%      \textstyle
%      \varepsilon_{%
%        \kern-0.15em%
%        \mathcal{X}%
%      }%
%    }%
%    \kern-.15em%
%    \TeX
%  }%
%}
%\end{verbatim}
%\end{quote}
%    \begin{macrocode}
\def\HoLogo@ExTeX#1{%
  \HoLogoFont@font{ExTeX}{rm}{%
    \ltx@mbox{%
      \HOLOGO@MathSetup
      $%
        \textstyle
        \varepsilon_{%
          \kern-0.15em%
          \HoLogoFont@font{ExTeX}{sy}{X}%
        }%
      $%
    }%
    \HOLOGO@discretionary
    \kern-.15em%
    \hologo{TeX}%
  }%
}
%    \end{macrocode}
%    \end{macro}
%    \begin{macro}{\HoLogoHtml@ExTeX}
%    \begin{macrocode}
\def\HoLogoHtml@ExTeX#1{%
  \HoLogoCss@ExTeX
  \HoLogoFont@font{ExTeX}{rm}{%
    \HOLOGO@Span{ExTeX}{%
      \ltx@mbox{%
        \HOLOGO@MathSetup
        $\textstyle\varepsilon$%
        \HOLOGO@Span{X}{$\textstyle\chi$}%
        \hologo{TeX}%
      }%
    }%
  }%
}
%    \end{macrocode}
%    \end{macro}
%    \begin{macro}{\HoLogoBkm@ExTeX}
%    \begin{macrocode}
\def\HoLogoBkm@ExTeX#1{%
  \HOLOGO@PdfdocUnicode{#1{e}{E}x}{\textepsilon\textchi}%
  \hologo{TeX}%
}
%    \end{macrocode}
%    \end{macro}
%    \begin{macro}{\HoLogoCss@ExTeX}
%    \begin{macrocode}
\def\HoLogoCss@ExTeX{%
  \Css{%
    span.HoLogo-ExTeX{%
      font-family:serif;%
    }%
  }%
  \Css{%
    span.HoLogo-ExTeX span.HoLogo-TeX{%
      margin-left:-.15em;%
    }%
  }%
  \global\let\HoLogoCss@ExTeX\relax
}
%    \end{macrocode}
%    \end{macro}
%
% \subsubsection{\hologo{MiKTeX}}
%
%    \begin{macro}{\HoLogo@MiKTeX}
%    \begin{macrocode}
\def\HoLogo@MiKTeX#1{%
  \HOLOGO@mbox{MiK}%
  \HOLOGO@discretionary
  \hologo{TeX}%
}
%    \end{macrocode}
%    \end{macro}
%    \begin{macro}{\HoLogoHtml@MiKTeX}
%    \begin{macrocode}
\let\HoLogoHtml@MiKTeX\HoLogo@MiKTeX
%    \end{macrocode}
%    \end{macro}
%
% \subsubsection{\hologo{OzTeX} and friends}
%
%    Source: \hologo{OzTeX} FAQ \cite{OzTeX}:
%    \begin{quote}
%      |\def\OzTeX{O\kern-.03em z\kern-.15em\TeX}|\\
%      (There is no kerning in OzMF, OzMP and OzTtH.)
%    \end{quote}
%
%    \begin{macro}{\HoLogo@OzTeX}
%    \begin{macrocode}
\def\HoLogo@OzTeX#1{%
  O%
  \kern-.03em %
  z%
  \kern-.15em %
  \hologo{TeX}%
}
%    \end{macrocode}
%    \end{macro}
%    \begin{macro}{\HoLogoHtml@OzTeX}
%    \begin{macrocode}
\def\HoLogoHtml@OzTeX#1{%
  \HoLogoCss@OzTeX
  \HOLOGO@Span{OzTeX}{%
    O%
    \HOLOGO@Span{z}{z}%
    \hologo{TeX}%
  }%
}
%    \end{macrocode}
%    \end{macro}
%    \begin{macro}{\HoLogoCss@OzTeX}
%    \begin{macrocode}
\def\HoLogoCss@OzTeX{%
  \Css{%
    span.HoLogo-OzTeX span.HoLogo-z{%
      margin-left:-.03em;%
      margin-right:-.15em;%
    }%
  }%
  \global\let\HoLogoCss@OzTeX\relax
}
%    \end{macrocode}
%    \end{macro}
%
%    \begin{macro}{\HoLogo@OzMF}
%    \begin{macrocode}
\def\HoLogo@OzMF#1{%
  \HOLOGO@mbox{OzMF}%
}
%    \end{macrocode}
%    \end{macro}
%    \begin{macro}{\HoLogo@OzMP}
%    \begin{macrocode}
\def\HoLogo@OzMP#1{%
  \HOLOGO@mbox{OzMP}%
}
%    \end{macrocode}
%    \end{macro}
%    \begin{macro}{\HoLogo@OzTtH}
%    \begin{macrocode}
\def\HoLogo@OzTtH#1{%
  \HOLOGO@mbox{OzTtH}%
}
%    \end{macrocode}
%    \end{macro}
%
% \subsubsection{\hologo{PCTeX}}
%
%    \begin{macro}{\HoLogo@PCTeX}
%    \begin{macrocode}
\def\HoLogo@PCTeX#1{%
  \HOLOGO@mbox{PC}%
  \hologo{TeX}%
}
%    \end{macrocode}
%    \end{macro}
%    \begin{macro}{\HoLogoHtml@PCTeX}
%    \begin{macrocode}
\let\HoLogoHtml@PCTeX\HoLogo@PCTeX
%    \end{macrocode}
%    \end{macro}
%
% \subsubsection{\hologo{PiCTeX}}
%
%    The original definitions from \xfile{pictex.tex} \cite{PiCTeX}:
%\begin{quote}
%\begin{verbatim}
%\def\PiC{%
%  P%
%  \kern-.12em%
%  \lower.5ex\hbox{I}%
%  \kern-.075em%
%  C%
%}
%\def\PiCTeX{%
%  \PiC
%  \kern-.11em%
%  \TeX
%}
%\end{verbatim}
%\end{quote}
%
%    \begin{macro}{\HoLogo@PiC}
%    \begin{macrocode}
\def\HoLogo@PiC#1{%
  P%
  \kern-.12em%
  \lower.5ex\hbox{I}%
  \kern-.075em%
  C%
  \HOLOGO@SpaceFactor
}
%    \end{macrocode}
%    \end{macro}
%    \begin{macro}{\HoLogoHtml@PiC}
%    \begin{macrocode}
\def\HoLogoHtml@PiC#1{%
  \HoLogoCss@PiC
  \HOLOGO@Span{PiC}{%
    P%
    \HOLOGO@Span{i}{I}%
    C%
  }%
}
%    \end{macrocode}
%    \end{macro}
%    \begin{macro}{\HoLogoCss@PiC}
%    \begin{macrocode}
\def\HoLogoCss@PiC{%
  \Css{%
    span.HoLogo-PiC span.HoLogo-i{%
      position:relative;%
      top:.5ex;%
      margin-left:-.12em;%
      margin-right:-.075em;%
      text-decoration:none;%
    }%
  }%
  \global\let\HoLogoCss@PiC\relax
}
%    \end{macrocode}
%    \end{macro}
%
%    \begin{macro}{\HoLogo@PiCTeX}
%    \begin{macrocode}
\def\HoLogo@PiCTeX#1{%
  \hologo{PiC}%
  \HOLOGO@discretionary
  \kern-.11em%
  \hologo{TeX}%
}
%    \end{macrocode}
%    \end{macro}
%    \begin{macro}{\HoLogoHtml@PiCTeX}
%    \begin{macrocode}
\def\HoLogoHtml@PiCTeX#1{%
  \HoLogoCss@PiCTeX
  \HOLOGO@Span{PiCTeX}{%
    \hologo{PiC}%
    \hologo{TeX}%
  }%
}
%    \end{macrocode}
%    \end{macro}
%    \begin{macro}{\HoLogoCss@PiCTeX}
%    \begin{macrocode}
\def\HoLogoCss@PiCTeX{%
  \Css{%
    span.HoLogo-PiCTeX span.HoLogo-PiC{%
      margin-right:-.11em;%
    }%
  }%
  \global\let\HoLogoCss@PiCTeX\relax
}
%    \end{macrocode}
%    \end{macro}
%
% \subsubsection{\hologo{teTeX}}
%
%    \begin{macro}{\HoLogo@teTeX}
%    \begin{macrocode}
\def\HoLogo@teTeX#1{%
  \HOLOGO@mbox{#1{t}{T}e}%
  \HOLOGO@discretionary
  \hologo{TeX}%
}
%    \end{macrocode}
%    \end{macro}
%    \begin{macro}{\HoLogoCs@teTeX}
%    \begin{macrocode}
\def\HoLogoCs@teTeX#1{#1{t}{T}dfTeX}
%    \end{macrocode}
%    \end{macro}
%    \begin{macro}{\HoLogoBkm@teTeX}
%    \begin{macrocode}
\def\HoLogoBkm@teTeX#1{%
  #1{t}{T}e\hologo{TeX}%
}
%    \end{macrocode}
%    \end{macro}
%    \begin{macro}{\HoLogoHtml@teTeX}
%    \begin{macrocode}
\let\HoLogoHtml@teTeX\HoLogo@teTeX
%    \end{macrocode}
%    \end{macro}
%
% \subsubsection{\hologo{TeX4ht}}
%
%    \begin{macro}{\HoLogo@TeX4ht}
%    \begin{macrocode}
\expandafter\def\csname HoLogo@TeX4ht\endcsname#1{%
  \HOLOGO@mbox{\hologo{TeX}4ht}%
}
%    \end{macrocode}
%    \end{macro}
%    \begin{macro}{\HoLogoHtml@TeX4ht}
%    \begin{macrocode}
\expandafter
\let\csname HoLogoHtml@TeX4ht\expandafter\endcsname
\csname HoLogo@TeX4ht\endcsname
%    \end{macrocode}
%    \end{macro}
%
%
% \subsubsection{\hologo{SageTeX}}
%
%    \begin{macro}{\HoLogo@SageTeX}
%    \begin{macrocode}
\def\HoLogo@SageTeX#1{%
  \HOLOGO@mbox{Sage}%
  \HOLOGO@discretionary
  \HOLOGO@NegativeKerning{eT,oT,To}%
  \hologo{TeX}%
}
%    \end{macrocode}
%    \end{macro}
%    \begin{macro}{\HoLogoHtml@SageTeX}
%    \begin{macrocode}
\let\HoLogoHtml@SageTeX\HoLogo@SageTeX
%    \end{macrocode}
%    \end{macro}
%
% \subsection{\hologo{METAFONT} and friends}
%
%    \begin{macro}{\HoLogo@METAFONT}
%    \begin{macrocode}
\def\HoLogo@METAFONT#1{%
  \HoLogoFont@font{METAFONT}{logo}{%
    \HOLOGO@mbox{META}%
    \HOLOGO@discretionary
    \HOLOGO@mbox{FONT}%
  }%
}
%    \end{macrocode}
%    \end{macro}
%
%    \begin{macro}{\HoLogo@METAPOST}
%    \begin{macrocode}
\def\HoLogo@METAPOST#1{%
  \HoLogoFont@font{METAPOST}{logo}{%
    \HOLOGO@mbox{META}%
    \HOLOGO@discretionary
    \HOLOGO@mbox{POST}%
  }%
}
%    \end{macrocode}
%    \end{macro}
%
%    \begin{macro}{\HoLogo@MetaFun}
%    \begin{macrocode}
\def\HoLogo@MetaFun#1{%
  \HOLOGO@mbox{Meta}%
  \HOLOGO@discretionary
  \HOLOGO@mbox{Fun}%
}
%    \end{macrocode}
%    \end{macro}
%
%    \begin{macro}{\HoLogo@MetaPost}
%    \begin{macrocode}
\def\HoLogo@MetaPost#1{%
  \HOLOGO@mbox{Meta}%
  \HOLOGO@discretionary
  \HOLOGO@mbox{Post}%
}
%    \end{macrocode}
%    \end{macro}
%
% \subsection{Others}
%
% \subsubsection{\hologo{biber}}
%
%    \begin{macro}{\HoLogo@biber}
%    \begin{macrocode}
\def\HoLogo@biber#1{%
  \HOLOGO@mbox{#1{b}{B}i}%
  \HOLOGO@discretionary
  \HOLOGO@mbox{ber}%
}
%    \end{macrocode}
%    \end{macro}
%    \begin{macro}{\HoLogoCs@biber}
%    \begin{macrocode}
\def\HoLogoCs@biber#1{#1{b}{B}iber}
%    \end{macrocode}
%    \end{macro}
%    \begin{macro}{\HoLogoBkm@biber}
%    \begin{macrocode}
\def\HoLogoBkm@biber#1{%
  #1{b}{B}iber%
}
%    \end{macrocode}
%    \end{macro}
%    \begin{macro}{\HoLogoHtml@biber}
%    \begin{macrocode}
\let\HoLogoHtml@biber\HoLogo@biber
%    \end{macrocode}
%    \end{macro}
%
% \subsubsection{\hologo{KOMAScript}}
%
%    \begin{macro}{\HoLogo@KOMAScript}
%    The definition for \hologo{KOMAScript} is taken
%    from \hologo{KOMAScript} (\xfile{scrlogo.dtx}, reformatted) \cite{scrlogo}:
%\begin{quote}
%\begin{verbatim}
%\@ifundefined{KOMAScript}{%
%  \DeclareRobustCommand{\KOMAScript}{%
%    \textsf{%
%      K\kern.05em O\kern.05emM\kern.05em A%
%      \kern.1em-\kern.1em %
%      Script%
%    }%
%  }%
%}{}
%\end{verbatim}
%\end{quote}
%    \begin{macrocode}
\def\HoLogo@KOMAScript#1{%
  \HoLogoFont@font{KOMAScript}{sf}{%
    \HOLOGO@mbox{%
      K\kern.05em%
      O\kern.05em%
      M\kern.05em%
      A%
    }%
    \kern.1em%
    \HOLOGO@hyphen
    \kern.1em%
    \HOLOGO@mbox{Script}%
  }%
}
%    \end{macrocode}
%    \end{macro}
%    \begin{macro}{\HoLogoBkm@KOMAScript}
%    \begin{macrocode}
\def\HoLogoBkm@KOMAScript#1{%
  KOMA-Script%
}
%    \end{macrocode}
%    \end{macro}
%    \begin{macro}{\HoLogoHtml@KOMAScript}
%    \begin{macrocode}
\def\HoLogoHtml@KOMAScript#1{%
  \HoLogoCss@KOMAScript
  \HoLogoFont@font{KOMAScript}{sf}{%
    \HOLOGO@Span{KOMAScript}{%
      K%
      \HOLOGO@Span{O}{O}%
      M%
      \HOLOGO@Span{A}{A}%
      \HOLOGO@Span{hyphen}{-}%
      Script%
    }%
  }%
}
%    \end{macrocode}
%    \end{macro}
%    \begin{macro}{\HoLogoCss@KOMAScript}
%    \begin{macrocode}
\def\HoLogoCss@KOMAScript{%
  \Css{%
    span.HoLogo-KOMAScript{%
      font-family:sans-serif;%
    }%
  }%
  \Css{%
    span.HoLogo-KOMAScript span.HoLogo-O{%
      padding-left:.05em;%
      padding-right:.05em;%
    }%
  }%
  \Css{%
    span.HoLogo-KOMAScript span.HoLogo-A{%
      padding-left:.05em;%
    }%
  }%
  \Css{%
    span.HoLogo-KOMAScript span.HoLogo-hyphen{%
      padding-left:.1em;%
      padding-right:.1em;%
    }%
  }%
  \global\let\HoLogoCss@KOMAScript\relax
}
%    \end{macrocode}
%    \end{macro}
%
% \subsubsection{\hologo{LyX}}
%
%    \begin{macro}{\HoLogo@LyX}
%    The definition is taken from the documentation source files
%    of \hologo{LyX}, \xfile{Intro.lyx} \cite{LyX}:
%\begin{quote}
%\begin{verbatim}
%\def\LyX{%
%  \texorpdfstring{%
%    L\kern-.1667em\lower.25em\hbox{Y}\kern-.125emX\@%
%  }{%
%    LyX%
%  }%
%}
%\end{verbatim}
%\end{quote}
%    \begin{macrocode}
\def\HoLogo@LyX#1{%
  L%
  \kern-.1667em%
  \lower.25em\hbox{Y}%
  \kern-.125em%
  X%
  \HOLOGO@SpaceFactor
}
%    \end{macrocode}
%    \end{macro}
%    \begin{macro}{\HoLogoHtml@LyX}
%    \begin{macrocode}
\def\HoLogoHtml@LyX#1{%
  \HoLogoCss@LyX
  \HOLOGO@Span{LyX}{%
    L%
    \HOLOGO@Span{y}{Y}%
    X%
  }%
}
%    \end{macrocode}
%    \end{macro}
%    \begin{macro}{\HoLogoCss@LyX}
%    \begin{macrocode}
\def\HoLogoCss@LyX{%
  \Css{%
    span.HoLogo-LyX span.HoLogo-y{%
      position:relative;%
      top:.25em;%
      margin-left:-.1667em;%
      margin-right:-.125em;%
      text-decoration:none;%
    }%
  }%
  \global\let\HoLogoCss@LyX\relax
}
%    \end{macrocode}
%    \end{macro}
%
% \subsubsection{\hologo{NTS}}
%
%    \begin{macro}{\HoLogo@NTS}
%    Definition for \hologo{NTS} can be found in
%    package \xpackage{etex\textunderscore man} for the \hologo{eTeX} manual \cite{etexman}
%    and in package \xpackage{dtklogos} \cite{dtklogos}:
%\begin{quote}
%\begin{verbatim}
%\def\NTS{%
%  \leavevmode
%  \hbox{%
%    $%
%      \cal N%
%      \kern-0.35em%
%      \lower0.5ex\hbox{$\cal T$}%
%      \kern-0.2em%
%      S%
%    $%
%  }%
%}
%\end{verbatim}
%\end{quote}
%    \begin{macrocode}
\def\HoLogo@NTS#1{%
  \HoLogoFont@font{NTS}{sy}{%
    N\/%
    \kern-.35em%
    \lower.5ex\hbox{T\/}%
    \kern-.2em%
    S\/%
  }%
  \HOLOGO@SpaceFactor
}
%    \end{macrocode}
%    \end{macro}
%
% \subsubsection{\Hologo{TTH} (\hologo{TeX} to HTML translator)}
%
%    Source: \url{http://hutchinson.belmont.ma.us/tth/}
%    In the HTML source the second `T' is printed as subscript.
%\begin{quote}
%\begin{verbatim}
%T<sub>T</sub>H
%\end{verbatim}
%\end{quote}
%    \begin{macro}{\HoLogo@TTH}
%    \begin{macrocode}
\def\HoLogo@TTH#1{%
  \ltx@mbox{%
    T\HOLOGO@SubScript{T}H%
  }%
  \HOLOGO@SpaceFactor
}
%    \end{macrocode}
%    \end{macro}
%
%    \begin{macro}{\HoLogoHtml@TTH}
%    \begin{macrocode}
\def\HoLogoHtml@TTH#1{%
  T\HCode{<sub>}T\HCode{</sub>}H%
}
%    \end{macrocode}
%    \end{macro}
%
% \subsubsection{\Hologo{HanTheThanh}}
%
%    Partial source: Package \xpackage{dtklogos}.
%    The double accent is U+1EBF (latin small letter e with circumflex
%    and acute).
%    \begin{macro}{\HoLogo@HanTheThanh}
%    \begin{macrocode}
\def\HoLogo@HanTheThanh#1{%
  \ltx@mbox{H\`an}%
  \HOLOGO@space
  \ltx@mbox{%
    Th%
    \HOLOGO@IfCharExists{"1EBF}{%
      \char"1EBF\relax
    }{%
      \^e\hbox to 0pt{\hss\raise .5ex\hbox{\'{}}}%
    }%
  }%
  \HOLOGO@space
  \ltx@mbox{Th\`anh}%
}
%    \end{macrocode}
%    \end{macro}
%    \begin{macro}{\HoLogoBkm@HanTheThanh}
%    \begin{macrocode}
\def\HoLogoBkm@HanTheThanh#1{%
  H\`an %
  Th\HOLOGO@PdfdocUnicode{\^e}{\9036\277} %
  Th\`anh%
}
%    \end{macrocode}
%    \end{macro}
%    \begin{macro}{\HoLogoHtml@HanTheThanh}
%    \begin{macrocode}
\def\HoLogoHtml@HanTheThanh#1{%
  H\`an %
  Th\HCode{&\ltx@hashchar x1ebf;} %
  Th\`anh%
}
%    \end{macrocode}
%    \end{macro}
%
% \subsection{Driver detection}
%
%    \begin{macrocode}
\HOLOGO@IfExists\InputIfFileExists{%
  \InputIfFileExists{hologo.cfg}{}{}%
}{%
  \ltx@IfUndefined{pdf@filesize}{%
    \def\HOLOGO@InputIfExists{%
      \openin\HOLOGO@temp=hologo.cfg\relax
      \ifeof\HOLOGO@temp
        \closein\HOLOGO@temp
      \else
        \closein\HOLOGO@temp
        \begingroup
          \def\x{LaTeX2e}%
        \expandafter\endgroup
        \ifx\fmtname\x
          \input{hologo.cfg}%
        \else
          \input hologo.cfg\relax
        \fi
      \fi
    }%
    \ltx@IfUndefined{newread}{%
      \chardef\HOLOGO@temp=15 %
      \def\HOLOGO@CheckRead{%
        \ifeof\HOLOGO@temp
          \HOLOGO@InputIfExists
        \else
          \ifcase\HOLOGO@temp
            \@PackageWarningNoLine{hologo}{%
              Configuration file ignored, because\MessageBreak
              a free read register could not be found%
            }%
          \else
            \begingroup
              \count\ltx@cclv=\HOLOGO@temp
              \advance\ltx@cclv by \ltx@minusone
              \edef\x{\endgroup
                \chardef\noexpand\HOLOGO@temp=\the\count\ltx@cclv
                \relax
              }%
            \x
          \fi
        \fi
      }%
    }{%
      \csname newread\endcsname\HOLOGO@temp
      \HOLOGO@InputIfExists
    }%
  }{%
    \edef\HOLOGO@temp{\pdf@filesize{hologo.cfg}}%
    \ifx\HOLOGO@temp\ltx@empty
    \else
      \ifnum\HOLOGO@temp>0 %
        \begingroup
          \def\x{LaTeX2e}%
        \expandafter\endgroup
        \ifx\fmtname\x
          \input{hologo.cfg}%
        \else
          \input hologo.cfg\relax
        \fi
      \else
        \@PackageInfoNoLine{hologo}{%
          Empty configuration file `hologo.cfg' ignored%
        }%
      \fi
    \fi
  }%
}
%    \end{macrocode}
%
%    \begin{macrocode}
\def\HOLOGO@temp#1#2{%
  \kv@define@key{HoLogoDriver}{#1}[]{%
    \begingroup
      \def\HOLOGO@temp{##1}%
      \ltx@onelevel@sanitize\HOLOGO@temp
      \ifx\HOLOGO@temp\ltx@empty
      \else
        \@PackageError{hologo}{%
          Value (\HOLOGO@temp) not permitted for option `#1'%
        }%
        \@ehc
      \fi
    \endgroup
    \def\hologoDriver{#2}%
  }%
}%
\def\HOLOGO@@temp#1#2{%
  \ifx\kv@value\relax
    \HOLOGO@temp{#1}{#1}%
  \else
    \HOLOGO@temp{#1}{#2}%
  \fi
}%
\kv@parse@normalized{%
  pdftex,%
  luatex=pdftex,%
  dvipdfm,%
  dvipdfmx=dvipdfm,%
  dvips,%
  dvipsone=dvips,%
  xdvi=dvips,%
  xetex,%
  vtex,%
}\HOLOGO@@temp
%    \end{macrocode}
%
%    \begin{macrocode}
\kv@define@key{HoLogoDriver}{driverfallback}{%
  \def\HOLOGO@DriverFallback{#1}%
}
%    \end{macrocode}
%
%    \begin{macro}{\HOLOGO@DriverFallback}
%    \begin{macrocode}
\def\HOLOGO@DriverFallback{dvips}
%    \end{macrocode}
%    \end{macro}
%
%    \begin{macro}{\hologoDriverSetup}
%    \begin{macrocode}
\def\hologoDriverSetup{%
  \let\hologoDriver\ltx@undefined
  \HOLOGO@DriverSetup
}
%    \end{macrocode}
%    \end{macro}
%
%    \begin{macro}{\HOLOGO@DriverSetup}
%    \begin{macrocode}
\def\HOLOGO@DriverSetup#1{%
  \kvsetkeys{HoLogoDriver}{#1}%
  \HOLOGO@CheckDriver
  \ltx@ifundefined{hologoDriver}{%
    \begingroup
    \edef\x{\endgroup
      \noexpand\kvsetkeys{HoLogoDriver}{\HOLOGO@DriverFallback}%
    }\x
  }{}%
  \@PackageInfoNoLine{hologo}{Using driver `\hologoDriver'}%
}
%    \end{macrocode}
%    \end{macro}
%
%    \begin{macro}{\HOLOGO@CheckDriver}
%    \begin{macrocode}
\def\HOLOGO@CheckDriver{%
  \ifpdf
    \def\hologoDriver{pdftex}%
    \let\HOLOGO@pdfliteral\pdfliteral
    \ifluatex
      \ifx\pdfextension\@undefined\else
        \protected\def\pdfliteral{\pdfextension literal}%
        \let\HOLOGO@pdfliteral\pdfliteral
      \fi
      \ltx@IfUndefined{HOLOGO@pdfliteral}{%
        \ifnum\luatexversion<36 %
        \else
          \begingroup
            \let\HOLOGO@temp\endgroup
            \ifcase0%
                \directlua{%
                  if tex.enableprimitives then %
                    tex.enableprimitives('HOLOGO@', {'pdfliteral'})%
                  else %
                    tex.print('1')%
                  end%
                }%
                \ifx\HOLOGO@pdfliteral\@undefined 1\fi%
                \relax%
              \endgroup
              \let\HOLOGO@temp\relax
              \global\let\HOLOGO@pdfliteral\HOLOGO@pdfliteral
            \fi%
          \HOLOGO@temp
        \fi
      }{}%
    \fi
    \ltx@IfUndefined{HOLOGO@pdfliteral}{%
      \@PackageWarningNoLine{hologo}{%
        Cannot find \string\pdfliteral
      }%
    }{}%
  \else
    \ifxetex
      \def\hologoDriver{xetex}%
    \else
      \ifvtex
        \def\hologoDriver{vtex}%
      \fi
    \fi
  \fi
}
%    \end{macrocode}
%    \end{macro}
%
%    \begin{macro}{\HOLOGO@WarningUnsupportedDriver}
%    \begin{macrocode}
\def\HOLOGO@WarningUnsupportedDriver#1{%
  \@PackageWarningNoLine{hologo}{%
    Logo `#1' needs driver specific macros,\MessageBreak
    but driver `\hologoDriver' is not supported.\MessageBreak
    Use a different driver or\MessageBreak
    load package `graphics' or `pgf'%
  }%
}
%    \end{macrocode}
%    \end{macro}
%
% \subsubsection{Reflect box macros}
%
%    Skip driver part if not needed.
%    \begin{macrocode}
\ltx@IfUndefined{reflectbox}{}{%
  \ltx@IfUndefined{rotatebox}{}{%
    \HOLOGO@AtEnd
  }%
}
\ltx@IfUndefined{pgftext}{}{%
  \HOLOGO@AtEnd
}
\ltx@IfUndefined{psscalebox}{}{%
  \HOLOGO@AtEnd
}
%    \end{macrocode}
%
%    \begin{macrocode}
\def\HOLOGO@temp{LaTeX2e}
\ifx\fmtname\HOLOGO@temp
  \RequirePackage{kvoptions}[2011/06/30]%
  \ProcessKeyvalOptions{HoLogoDriver}%
\fi
\HOLOGO@DriverSetup{}
%    \end{macrocode}
%
%    \begin{macro}{\HOLOGO@ReflectBox}
%    \begin{macrocode}
\def\HOLOGO@ReflectBox#1{%
  \begingroup
    \setbox\ltx@zero\hbox{\begingroup#1\endgroup}%
    \setbox\ltx@two\hbox{%
      \kern\wd\ltx@zero
      \csname HOLOGO@ScaleBox@\hologoDriver\endcsname{-1}{1}{%
        \hbox to 0pt{\copy\ltx@zero\hss}%
      }%
    }%
    \wd\ltx@two=\wd\ltx@zero
    \box\ltx@two
  \endgroup
}
%    \end{macrocode}
%    \end{macro}
%
%    \begin{macro}{\HOLOGO@PointReflectBox}
%    \begin{macrocode}
\def\HOLOGO@PointReflectBox#1{%
  \begingroup
    \setbox\ltx@zero\hbox{\begingroup#1\endgroup}%
    \setbox\ltx@two\hbox{%
      \kern\wd\ltx@zero
      \raise\ht\ltx@zero\hbox{%
        \csname HOLOGO@ScaleBox@\hologoDriver\endcsname{-1}{-1}{%
          \hbox to 0pt{\copy\ltx@zero\hss}%
        }%
      }%
    }%
    \wd\ltx@two=\wd\ltx@zero
    \box\ltx@two
  \endgroup
}
%    \end{macrocode}
%    \end{macro}
%
%    We must define all variants because of dynamic driver setup.
%    \begin{macrocode}
\def\HOLOGO@temp#1#2{#2}
%    \end{macrocode}
%
%    \begin{macro}{\HOLOGO@ScaleBox@pdftex}
%    \begin{macrocode}
\HOLOGO@temp{pdftex}{%
  \def\HOLOGO@ScaleBox@pdftex#1#2#3{%
    \HOLOGO@pdfliteral{%
      q #1 0 0 #2 0 0 cm%
    }%
    #3%
    \HOLOGO@pdfliteral{%
      Q%
    }%
  }%
}
%    \end{macrocode}
%    \end{macro}
%    \begin{macro}{\HOLOGO@ScaleBox@dvips}
%    \begin{macrocode}
\HOLOGO@temp{dvips}{%
  \def\HOLOGO@ScaleBox@dvips#1#2#3{%
    \special{ps:%
      gsave %
      currentpoint %
      currentpoint translate %
      #1 #2 scale %
      neg exch neg exch translate%
    }%
    #3%
    \special{ps:%
      currentpoint %
      grestore %
      moveto%
    }%
  }%
}
%    \end{macrocode}
%    \end{macro}
%    \begin{macro}{\HOLOGO@ScaleBox@dvipdfm}
%    \begin{macrocode}
\HOLOGO@temp{dvipdfm}{%
  \let\HOLOGO@ScaleBox@dvipdfm\HOLOGO@ScaleBox@dvips
}
%    \end{macrocode}
%    \end{macro}
%    Since \hologo{XeTeX} v0.6.
%    \begin{macro}{\HOLOGO@ScaleBox@xetex}
%    \begin{macrocode}
\HOLOGO@temp{xetex}{%
  \def\HOLOGO@ScaleBox@xetex#1#2#3{%
    \special{x:gsave}%
    \special{x:scale #1 #2}%
    #3%
    \special{x:grestore}%
  }%
}
%    \end{macrocode}
%    \end{macro}
%    \begin{macro}{\HOLOGO@ScaleBox@vtex}
%    \begin{macrocode}
\HOLOGO@temp{vtex}{%
  \def\HOLOGO@ScaleBox@vtex#1#2#3{%
    \special{r(#1,0,0,#2,0,0}%
    #3%
    \special{r)}%
  }%
}
%    \end{macrocode}
%    \end{macro}
%
%    \begin{macrocode}
\HOLOGO@AtEnd%
%</package>
%    \end{macrocode}
%
% \section{Test}
%
% \subsection{Catcode checks for loading}
%
%    \begin{macrocode}
%<*test1>
%    \end{macrocode}
%    \begin{macrocode}
\catcode`\{=1 %
\catcode`\}=2 %
\catcode`\#=6 %
\catcode`\@=11 %
\expandafter\ifx\csname count@\endcsname\relax
  \countdef\count@=255 %
\fi
\expandafter\ifx\csname @gobble\endcsname\relax
  \long\def\@gobble#1{}%
\fi
\expandafter\ifx\csname @firstofone\endcsname\relax
  \long\def\@firstofone#1{#1}%
\fi
\expandafter\ifx\csname loop\endcsname\relax
  \expandafter\@firstofone
\else
  \expandafter\@gobble
\fi
{%
  \def\loop#1\repeat{%
    \def\body{#1}%
    \iterate
  }%
  \def\iterate{%
    \body
      \let\next\iterate
    \else
      \let\next\relax
    \fi
    \next
  }%
  \let\repeat=\fi
}%
\def\RestoreCatcodes{}
\count@=0 %
\loop
  \edef\RestoreCatcodes{%
    \RestoreCatcodes
    \catcode\the\count@=\the\catcode\count@\relax
  }%
\ifnum\count@<255 %
  \advance\count@ 1 %
\repeat

\def\RangeCatcodeInvalid#1#2{%
  \count@=#1\relax
  \loop
    \catcode\count@=15 %
  \ifnum\count@<#2\relax
    \advance\count@ 1 %
  \repeat
}
\def\RangeCatcodeCheck#1#2#3{%
  \count@=#1\relax
  \loop
    \ifnum#3=\catcode\count@
    \else
      \errmessage{%
        Character \the\count@\space
        with wrong catcode \the\catcode\count@\space
        instead of \number#3%
      }%
    \fi
  \ifnum\count@<#2\relax
    \advance\count@ 1 %
  \repeat
}
\def\space{ }
\expandafter\ifx\csname LoadCommand\endcsname\relax
  \def\LoadCommand{\input hologo.sty\relax}%
\fi
\def\Test{%
  \RangeCatcodeInvalid{0}{47}%
  \RangeCatcodeInvalid{58}{64}%
  \RangeCatcodeInvalid{91}{96}%
  \RangeCatcodeInvalid{123}{255}%
  \catcode`\@=12 %
  \catcode`\\=0 %
  \catcode`\%=14 %
  \LoadCommand
  \RangeCatcodeCheck{0}{36}{15}%
  \RangeCatcodeCheck{37}{37}{14}%
  \RangeCatcodeCheck{38}{47}{15}%
  \RangeCatcodeCheck{48}{57}{12}%
  \RangeCatcodeCheck{58}{63}{15}%
  \RangeCatcodeCheck{64}{64}{12}%
  \RangeCatcodeCheck{65}{90}{11}%
  \RangeCatcodeCheck{91}{91}{15}%
  \RangeCatcodeCheck{92}{92}{0}%
  \RangeCatcodeCheck{93}{96}{15}%
  \RangeCatcodeCheck{97}{122}{11}%
  \RangeCatcodeCheck{123}{255}{15}%
  \RestoreCatcodes
}
\Test
\csname @@end\endcsname
\end
%    \end{macrocode}
%    \begin{macrocode}
%</test1>
%    \end{macrocode}
%
% \subsection{Spacefactor}
%
%    The space factor must be 1000 after a logo. If it is greater 1000
%    then the following space is a space after a sentence closing point.
%    If the space factor is smaller 1000 then an immediate following
%    dot is interpreted as abbreviation, not sentence closing point.
%
%    \begin{macrocode}
%<*test-spacefactor>
\NeedsTeXFormat{LaTeX2e}
\documentclass{article}
\usepackage{hologo}[2016/05/12]
\usepackage{kvsetkeys}
\usepackage{qstest}
\IncludeTests{*}
\LogTests{log}{*}{*}
\begin{document}
\begin{qstest}{spacefactor}{spacefactor}
\newcommand*{\Test}[1]{%
  \sbox0{%
    \hologo{#1}%
    \Expect*{1000 (#1)}*{\the\spacefactor\space(#1)}%
  }%
}%
\makeatletter
\def\TestList{}
\def\hologoEntry#1#2#3{%
  \edef\TestList{%
    \ifx\TestList\@empty
    \else
      \TestList,%
    \fi
    #1%
    \ifx\\#2\\%
    \else
      ={variant=#2}%
    \fi
  }%
}
\hologoList
\expandafter\kv@parse@normalized\expandafter{%
  \TestList
}{%
  \begingroup
    \let\@logo=\kv@key
    \ifx\kv@value\relax
    \else
      \expandafter\hologoLogoSetup\expandafter\@logo\expandafter{%
        \kv@value
      }%
    \fi
    \Test\@logo
  \endgroup
  \@gobbletwo
}
\end{qstest}
\end{document}
%</test-spacefactor>
%    \end{macrocode}
%
% \subsection{Complete list}
%
%    \begin{macrocode}
%<*test-list>
\NeedsTeXFormat{LaTeX2e}
\documentclass[12pt,a4paper]{article}
\usepackage{hologo}[2016/05/12]
\usepackage[T1]{fontenc}
\usepackage{lmodern}
\usepackage{parskip}
\usepackage[unicode]{hyperref}[2011/09/28]
\usepackage{bookmark}[2011/09/19]
\bookmarksetup{%
  numbered,%
  open,%
  openlevel=2,%
}
\renewcommand*{\contentsname}{List of logos}
\begin{document}
\tableofcontents
\def\TestFont#1#2#3#4#5#6{%
  \begingroup
    \usefont{#3}{#4}{#5}{#6}%
    \HologoVariant{#1}{#2}/\hologoVariant{#1}{#2}%
    \quad
    \begingroup\scriptsize\hologoVariant{#1}{#2}\endgroup
    \quad
  \endgroup
  (#3/#4/#5/#6)%
  \par
}
\makeatletter
\def\hologoEntry#1#2#3{%
  \section{%
    \HologoVariant{#1}{#2}/\hologoVariant{#1}{#2} %
    {[#1\ifx\\#2\\\else\space(#2)\fi]}% hash-ok
  }% braces around [] because of bug in tex4ht
  \begingroup
    \hypersetup{unicode=false}%
    \bookmark[%
      dest=\@currentHref,%
      rellevel=1,%
      keeplevel,%
    ]{%
      \HologoVariant{#1}{#2}/\hologoVariant{#1}{#2} %
      (PDFDocEncoding)%
    }%
  \endgroup
  \TestFont{#1}{#2}{OT1}{cmr}{m}{n}%
  \TestFont{#1}{#2}{OT1}{cmss}{m}{n}%
  \TestFont{#1}{#2}{OT1}{cmr}{b}{n}%
  \TestFont{#1}{#2}{OT1}{cmr}{m}{it}%
  \TestFont{#1}{#2}{OT1}{cmtt}{m}{n}%
  \TestFont{#1}{#2}{T1}{lmr}{m}{n}%
  \TestFont{#1}{#2}{T1}{lmss}{m}{n}%
  \TestFont{#1}{#2}{T1}{lmr}{b}{n}%
  \TestFont{#1}{#2}{T1}{lmr}{m}{it}%
  \TestFont{#1}{#2}{T1}{lmtt}{m}{n}%
  \TestFont{#1}{#2}{T1}{lmvtt}{m}{n}%
  \TestFont{#1}{#2}{T1}{qtm}{m}{n}%
  \TestFont{#1}{#2}{T1}{qhv}{m}{n}%
  \TestFont{#1}{#2}{T1}{qtm}{b}{n}%
  \TestFont{#1}{#2}{T1}{qtm}{m}{it}%
  \TestFont{#1}{#2}{T1}{qcr}{m}{n}%
  \newpage
}
\makeatother
\hologoList
\end{document}
%</test-list>
%    \end{macrocode}
%
% \section{Installation}
%
% \subsection{Download}
%
% \paragraph{Package.} This package is available on
% CTAN\footnote{\url{ftp://ftp.ctan.org/tex-archive/}}:
% \begin{description}
% \item[\CTAN{macros/latex/contrib/oberdiek/hologo.dtx}] The source file.
% \item[\CTAN{macros/latex/contrib/oberdiek/hologo.pdf}] Documentation.
% \end{description}
%
%
% \paragraph{Bundle.} All the packages of the bundle `oberdiek'
% are also available in a TDS compliant ZIP archive. There
% the packages are already unpacked and the documentation files
% are generated. The files and directories obey the TDS standard.
% \begin{description}
% \item[\CTAN{install/macros/latex/contrib/oberdiek.tds.zip}]
% \end{description}
% \emph{TDS} refers to the standard ``A Directory Structure
% for \TeX\ Files'' (\CTAN{tds/tds.pdf}). Directories
% with \xfile{texmf} in their name are usually organized this way.
%
% \subsection{Bundle installation}
%
% \paragraph{Unpacking.} Unpack the \xfile{oberdiek.tds.zip} in the
% TDS tree (also known as \xfile{texmf} tree) of your choice.
% Example (linux):
% \begin{quote}
%   |unzip oberdiek.tds.zip -d ~/texmf|
% \end{quote}
%
% \paragraph{Script installation.}
% Check the directory \xfile{TDS:scripts/oberdiek/} for
% scripts that need further installation steps.
% Package \xpackage{attachfile2} comes with the Perl script
% \xfile{pdfatfi.pl} that should be installed in such a way
% that it can be called as \texttt{pdfatfi}.
% Example (linux):
% \begin{quote}
%   |chmod +x scripts/oberdiek/pdfatfi.pl|\\
%   |cp scripts/oberdiek/pdfatfi.pl /usr/local/bin/|
% \end{quote}
%
% \subsection{Package installation}
%
% \paragraph{Unpacking.} The \xfile{.dtx} file is a self-extracting
% \docstrip\ archive. The files are extracted by running the
% \xfile{.dtx} through \plainTeX:
% \begin{quote}
%   \verb|tex hologo.dtx|
% \end{quote}
%
% \paragraph{TDS.} Now the different files must be moved into
% the different directories in your installation TDS tree
% (also known as \xfile{texmf} tree):
% \begin{quote}
% \def\t{^^A
% \begin{tabular}{@{}>{\ttfamily}l@{ $\rightarrow$ }>{\ttfamily}l@{}}
%   hologo.sty & tex/generic/oberdiek/hologo.sty\\
%   hologo.pdf & doc/latex/oberdiek/hologo.pdf\\
%   example/hologo-example.tex & doc/latex/oberdiek/example/hologo-example.tex\\
%   test/hologo-test1.tex & doc/latex/oberdiek/test/hologo-test1.tex\\
%   test/hologo-test-spacefactor.tex & doc/latex/oberdiek/test/hologo-test-spacefactor.tex\\
%   test/hologo-test-list.tex & doc/latex/oberdiek/test/hologo-test-list.tex\\
%   hologo.dtx & source/latex/oberdiek/hologo.dtx\\
% \end{tabular}^^A
% }^^A
% \sbox0{\t}^^A
% \ifdim\wd0>\linewidth
%   \begingroup
%     \advance\linewidth by\leftmargin
%     \advance\linewidth by\rightmargin
%   \edef\x{\endgroup
%     \def\noexpand\lw{\the\linewidth}^^A
%   }\x
%   \def\lwbox{^^A
%     \leavevmode
%     \hbox to \linewidth{^^A
%       \kern-\leftmargin\relax
%       \hss
%       \usebox0
%       \hss
%       \kern-\rightmargin\relax
%     }^^A
%   }^^A
%   \ifdim\wd0>\lw
%     \sbox0{\small\t}^^A
%     \ifdim\wd0>\linewidth
%       \ifdim\wd0>\lw
%         \sbox0{\footnotesize\t}^^A
%         \ifdim\wd0>\linewidth
%           \ifdim\wd0>\lw
%             \sbox0{\scriptsize\t}^^A
%             \ifdim\wd0>\linewidth
%               \ifdim\wd0>\lw
%                 \sbox0{\tiny\t}^^A
%                 \ifdim\wd0>\linewidth
%                   \lwbox
%                 \else
%                   \usebox0
%                 \fi
%               \else
%                 \lwbox
%               \fi
%             \else
%               \usebox0
%             \fi
%           \else
%             \lwbox
%           \fi
%         \else
%           \usebox0
%         \fi
%       \else
%         \lwbox
%       \fi
%     \else
%       \usebox0
%     \fi
%   \else
%     \lwbox
%   \fi
% \else
%   \usebox0
% \fi
% \end{quote}
% If you have a \xfile{docstrip.cfg} that configures and enables \docstrip's
% TDS installing feature, then some files can already be in the right
% place, see the documentation of \docstrip.
%
% \subsection{Refresh file name databases}
%
% If your \TeX~distribution
% (\teTeX, \mikTeX, \dots) relies on file name databases, you must refresh
% these. For example, \teTeX\ users run \verb|texhash| or
% \verb|mktexlsr|.
%
% \subsection{Some details for the interested}
%
% \paragraph{Attached source.}
%
% The PDF documentation on CTAN also includes the
% \xfile{.dtx} source file. It can be extracted by
% AcrobatReader 6 or higher. Another option is \textsf{pdftk},
% e.g. unpack the file into the current directory:
% \begin{quote}
%   \verb|pdftk hologo.pdf unpack_files output .|
% \end{quote}
%
% \paragraph{Unpacking with \LaTeX.}
% The \xfile{.dtx} chooses its action depending on the format:
% \begin{description}
% \item[\plainTeX:] Run \docstrip\ and extract the files.
% \item[\LaTeX:] Generate the documentation.
% \end{description}
% If you insist on using \LaTeX\ for \docstrip\ (really,
% \docstrip\ does not need \LaTeX), then inform the autodetect routine
% about your intention:
% \begin{quote}
%   \verb|latex \let\install=y\input{hologo.dtx}|
% \end{quote}
% Do not forget to quote the argument according to the demands
% of your shell.
%
% \paragraph{Generating the documentation.}
% You can use both the \xfile{.dtx} or the \xfile{.drv} to generate
% the documentation. The process can be configured by the
% configuration file \xfile{ltxdoc.cfg}. For instance, put this
% line into this file, if you want to have A4 as paper format:
% \begin{quote}
%   \verb|\PassOptionsToClass{a4paper}{article}|
% \end{quote}
% An example follows how to generate the
% documentation with pdf\LaTeX:
% \begin{quote}
%\begin{verbatim}
%pdflatex hologo.dtx
%makeindex -s gind.ist hologo.idx
%pdflatex hologo.dtx
%makeindex -s gind.ist hologo.idx
%pdflatex hologo.dtx
%\end{verbatim}
% \end{quote}
%
% \section{Catalogue}
%
% The following XML file can be used as source for the
% \href{http://mirror.ctan.org/help/Catalogue/catalogue.html}{\TeX\ Catalogue}.
% The elements \texttt{caption} and \texttt{description} are imported
% from the original XML file from the Catalogue.
% The name of the XML file in the Catalogue is \xfile{hologo.xml}.
%    \begin{macrocode}
%<*catalogue>
<?xml version='1.0' encoding='us-ascii'?>
<!DOCTYPE entry SYSTEM 'catalogue.dtd'>
<entry datestamp='$Date$' modifier='$Author$' id='hologo'>
  <name>hologo</name>
  <caption>A collection of logos with bookmark support.</caption>
  <authorref id='auth:oberdiek'/>
  <copyright owner='Heiko Oberdiek' year='2010-2012'/>
  <license type='lppl1.3'/>
  <version number='1.10'/>
  <description>
    The package defines a single command <tt>\hologo</tt>, whose
    argument is the usual case-confused ASCII version of the logo.
    The command is bookmark-enabled, so that every logo becomes
    available in bookmarks without further work.
    <p/>
    The package is part of the <xref refid='oberdiek'>oberdiek</xref>
    bundle.
  </description>
  <documentation details='Package documentation'
      href='ctan:/macros/latex/contrib/oberdiek/hologo.pdf'/>
  <ctan file='true' path='/macros/latex/contrib/oberdiek/hologo.dtx'/>
  <miktex location='oberdiek'/>
  <texlive location='oberdiek'/>
  <install path='/macros/latex/contrib/oberdiek/oberdiek.tds.zip'/>
</entry>
%</catalogue>
%    \end{macrocode}
%
% \begin{thebibliography}{9}
% \raggedright
%
% \bibitem{btxdoc}
% Oren Patashnik,
% \textit{\hologo{BibTeX}ing},
% 1988-02-08.\\
% \CTAN{biblio/bibtex/base/}
%
% \bibitem{dtklogos}
% Gerd Neugebauer, DANTE,
% \textit{Package \xpackage{dtklogos}},
% 2011-04-25.\\
% \CTAN{usergrps/dante/dtk/dtklogos.sty}
%
% \bibitem{etexman}
% The \hologo{NTS} Team,
% \textit{The \hologo{eTeX} manual},
% 1998-02.\\
% \CTAN{systems/e-tex/v2/doc/}
%
% \bibitem{ExTeX-FAQ}
% The \hologo{ExTeX} group,
% \textit{\hologo{ExTeX}: FAQ -- How is \hologo{ExTeX} typeset?},
% 2007-04-14.\\
% \url{http://www.extex.org/documentation/faq.html}
%
% \bibitem{LyX}
% %@MISC{ LyX,
% %  title = {{LyX 2.0.0 -- The Document Processor [Computer software and manual]}},
% %  author = {{The LyX Team}},
% %  howpublished = {Internet: http://www.lyx.org},
% %  year = {2011-05-08},
% %  note = {Retrieved May 10, 2011, from http://www.lyx.org},
% %  url = {http://www.lyx.org/}
% %}
% The \hologo{LyX} Team,
% \textit{\hologo{LyX} -- The Document Processor},
% 2011-05-08.\\
% \url{http://www.lyx.org/}
%
% \bibitem{OzTeX}
% Andrew Trevorrow,
% \hologo{OzTeX} FAQ: What is the correct way to typeset ``\hologo{OzTeX}''?,
% 2011-09-15 (visited).
% \url{http://www.trevorrow.com/oztex/ozfaq.html#oztex-logo}
%
% \bibitem{PiCTeX}
% Michael Wichura,
% \textit{The \hologo{PiCTeX} macro package},
% 1987-09-21.
% \CTAN{graphics/pictex/}
%
% \bibitem{scrlogo}
% Markus Kohm,
% \textit{\hologo{KOMAScript} Datei \xfile{scrlogo.dtx}},
% 2009-01-30.\\
% \CTAN{install/macros/latex/contrib/komascript.tds.zip}
%
% \end{thebibliography}
%
% \begin{History}
%   \begin{Version}{2010/04/08 v1.0}
%   \item
%     The first version.
%   \end{Version}
%   \begin{Version}{2010/04/16 v1.1}
%   \item
%     \cs{Hologo} added for support of logos at start of a sentence.
%   \item
%     \cs{hologoSetup} and \cs{hologoLogoSetup} added.
%   \item
%     Options \xoption{break}, \xoption{hyphenbreak}, \xoption{spacebreak}
%     added.
%   \item
%     Variant support added by option \xoption{variant}.
%   \end{Version}
%   \begin{Version}{2010/04/24 v1.2}
%   \item
%     \hologo{LaTeX3} added.
%   \item
%     \hologo{VTeX} added.
%   \end{Version}
%   \begin{Version}{2010/11/21 v1.3}
%   \item
%     \hologo{iniTeX}, \hologo{virTeX} added.
%   \end{Version}
%   \begin{Version}{2011/03/25 v1.4}
%   \item
%     \hologo{ConTeXt} with variants added.
%   \item
%     Option \xoption{discretionarybreak} added as refinement for
%     option \xoption{break}.
%   \end{Version}
%   \begin{Version}{2011/04/21 v1.5}
%   \item
%     Wrong TDS directory for test files fixed.
%   \end{Version}
%   \begin{Version}{2011/10/01 v1.6}
%   \item
%     Support for package \xpackage{tex4ht} added.
%   \item
%     Support for \cs{csname} added if \cs{ifincsname} is available.
%   \item
%     New logos:
%     \hologo{(La)TeX},
%     \hologo{biber},
%     \hologo{BibTeX} (\xoption{sc}, \xoption{sf}),
%     \hologo{emTeX},
%     \hologo{ExTeX},
%     \hologo{KOMAScript},
%     \hologo{La},
%     \hologo{LyX},
%     \hologo{MiKTeX},
%     \hologo{NTS},
%     \hologo{OzMF},
%     \hologo{OzMP},
%     \hologo{OzTeX},
%     \hologo{OzTtH},
%     \hologo{PCTeX},
%     \hologo{PiC},
%     \hologo{PiCTeX},
%     \hologo{METAFONT},
%     \hologo{MetaFun},
%     \hologo{METAPOST},
%     \hologo{MetaPost},
%     \hologo{SLiTeX} (\xoption{lift}, \xoption{narrow}, \xoption{simple}),
%     \hologo{SliTeX} (\xoption{narrow}, \xoption{simple}, \xoption{lift}),
%     \hologo{teTeX}.
%   \item
%     Fixes:
%     \hologo{iniTeX},
%     \hologo{pdfLaTeX},
%     \hologo{pdfTeX},
%     \hologo{virTeX}.
%   \item
%     \cs{hologoFontSetup} and \cs{hologoLogoFontSetup} added.
%   \item
%     \cs{hologoVariant} and \cs{HologoVariant} added.
%   \end{Version}
%   \begin{Version}{2011/11/22 v1.7}
%   \item
%     New logos:
%     \hologo{BibTeX8},
%     \hologo{LaTeXML},
%     \hologo{SageTeX},
%     \hologo{TeX4ht},
%     \hologo{TTH}.
%   \item
%     \hologo{Xe} and friends: Driver stuff fixed.
%   \item
%     \hologo{Xe} and friends: Support for italic added.
%   \item
%     \hologo{Xe} and friends: Package support for \xpackage{pgf}
%     and \xpackage{pstricks} added.
%   \end{Version}
%   \begin{Version}{2011/11/29 v1.8}
%   \item
%     New logos:
%     \hologo{HanTheThanh}.
%   \end{Version}
%   \begin{Version}{2011/12/21 v1.9}
%   \item
%     Patch for package \xpackage{ifxetex} added for the case that
%     \cs{newif} is undefined in \hologo{iniTeX}.
%   \item
%     Some fixes for \hologo{iniTeX}.
%   \end{Version}
%   \begin{Version}{2012/04/26 v1.10}
%   \item
%     Fix in bookmark version of logo ``\hologo{HanTheThanh}''.
%   \end{Version}
%   \begin{Version}{2016/05/12 v1.11}
%   \item
%     Update HOLOGO@IfCharExists (previously in texlive)
%   \item define pdfliteral in current luatex.
%   \end{Version}
% \end{History}
%
% \PrintIndex
%
% \Finale
\endinput
%
        \else
          \input hologo.cfg\relax
        \fi
      \else
        \@PackageInfoNoLine{hologo}{%
          Empty configuration file `hologo.cfg' ignored%
        }%
      \fi
    \fi
  }%
}
%    \end{macrocode}
%
%    \begin{macrocode}
\def\HOLOGO@temp#1#2{%
  \kv@define@key{HoLogoDriver}{#1}[]{%
    \begingroup
      \def\HOLOGO@temp{##1}%
      \ltx@onelevel@sanitize\HOLOGO@temp
      \ifx\HOLOGO@temp\ltx@empty
      \else
        \@PackageError{hologo}{%
          Value (\HOLOGO@temp) not permitted for option `#1'%
        }%
        \@ehc
      \fi
    \endgroup
    \def\hologoDriver{#2}%
  }%
}%
\def\HOLOGO@@temp#1#2{%
  \ifx\kv@value\relax
    \HOLOGO@temp{#1}{#1}%
  \else
    \HOLOGO@temp{#1}{#2}%
  \fi
}%
\kv@parse@normalized{%
  pdftex,%
  luatex=pdftex,%
  dvipdfm,%
  dvipdfmx=dvipdfm,%
  dvips,%
  dvipsone=dvips,%
  xdvi=dvips,%
  xetex,%
  vtex,%
}\HOLOGO@@temp
%    \end{macrocode}
%
%    \begin{macrocode}
\kv@define@key{HoLogoDriver}{driverfallback}{%
  \def\HOLOGO@DriverFallback{#1}%
}
%    \end{macrocode}
%
%    \begin{macro}{\HOLOGO@DriverFallback}
%    \begin{macrocode}
\def\HOLOGO@DriverFallback{dvips}
%    \end{macrocode}
%    \end{macro}
%
%    \begin{macro}{\hologoDriverSetup}
%    \begin{macrocode}
\def\hologoDriverSetup{%
  \let\hologoDriver\ltx@undefined
  \HOLOGO@DriverSetup
}
%    \end{macrocode}
%    \end{macro}
%
%    \begin{macro}{\HOLOGO@DriverSetup}
%    \begin{macrocode}
\def\HOLOGO@DriverSetup#1{%
  \kvsetkeys{HoLogoDriver}{#1}%
  \HOLOGO@CheckDriver
  \ltx@ifundefined{hologoDriver}{%
    \begingroup
    \edef\x{\endgroup
      \noexpand\kvsetkeys{HoLogoDriver}{\HOLOGO@DriverFallback}%
    }\x
  }{}%
  \@PackageInfoNoLine{hologo}{Using driver `\hologoDriver'}%
}
%    \end{macrocode}
%    \end{macro}
%
%    \begin{macro}{\HOLOGO@CheckDriver}
%    \begin{macrocode}
\def\HOLOGO@CheckDriver{%
  \ifpdf
    \def\hologoDriver{pdftex}%
    \let\HOLOGO@pdfliteral\pdfliteral
    \ifluatex
      \ifx\pdfextension\@undefined\else
        \protected\def\pdfliteral{\pdfextension literal}%
        \let\HOLOGO@pdfliteral\pdfliteral
      \fi
      \ltx@IfUndefined{HOLOGO@pdfliteral}{%
        \ifnum\luatexversion<36 %
        \else
          \begingroup
            \let\HOLOGO@temp\endgroup
            \ifcase0%
                \directlua{%
                  if tex.enableprimitives then %
                    tex.enableprimitives('HOLOGO@', {'pdfliteral'})%
                  else %
                    tex.print('1')%
                  end%
                }%
                \ifx\HOLOGO@pdfliteral\@undefined 1\fi%
                \relax%
              \endgroup
              \let\HOLOGO@temp\relax
              \global\let\HOLOGO@pdfliteral\HOLOGO@pdfliteral
            \fi%
          \HOLOGO@temp
        \fi
      }{}%
    \fi
    \ltx@IfUndefined{HOLOGO@pdfliteral}{%
      \@PackageWarningNoLine{hologo}{%
        Cannot find \string\pdfliteral
      }%
    }{}%
  \else
    \ifxetex
      \def\hologoDriver{xetex}%
    \else
      \ifvtex
        \def\hologoDriver{vtex}%
      \fi
    \fi
  \fi
}
%    \end{macrocode}
%    \end{macro}
%
%    \begin{macro}{\HOLOGO@WarningUnsupportedDriver}
%    \begin{macrocode}
\def\HOLOGO@WarningUnsupportedDriver#1{%
  \@PackageWarningNoLine{hologo}{%
    Logo `#1' needs driver specific macros,\MessageBreak
    but driver `\hologoDriver' is not supported.\MessageBreak
    Use a different driver or\MessageBreak
    load package `graphics' or `pgf'%
  }%
}
%    \end{macrocode}
%    \end{macro}
%
% \subsubsection{Reflect box macros}
%
%    Skip driver part if not needed.
%    \begin{macrocode}
\ltx@IfUndefined{reflectbox}{}{%
  \ltx@IfUndefined{rotatebox}{}{%
    \HOLOGO@AtEnd
  }%
}
\ltx@IfUndefined{pgftext}{}{%
  \HOLOGO@AtEnd
}
\ltx@IfUndefined{psscalebox}{}{%
  \HOLOGO@AtEnd
}
%    \end{macrocode}
%
%    \begin{macrocode}
\def\HOLOGO@temp{LaTeX2e}
\ifx\fmtname\HOLOGO@temp
  \RequirePackage{kvoptions}[2011/06/30]%
  \ProcessKeyvalOptions{HoLogoDriver}%
\fi
\HOLOGO@DriverSetup{}
%    \end{macrocode}
%
%    \begin{macro}{\HOLOGO@ReflectBox}
%    \begin{macrocode}
\def\HOLOGO@ReflectBox#1{%
  \begingroup
    \setbox\ltx@zero\hbox{\begingroup#1\endgroup}%
    \setbox\ltx@two\hbox{%
      \kern\wd\ltx@zero
      \csname HOLOGO@ScaleBox@\hologoDriver\endcsname{-1}{1}{%
        \hbox to 0pt{\copy\ltx@zero\hss}%
      }%
    }%
    \wd\ltx@two=\wd\ltx@zero
    \box\ltx@two
  \endgroup
}
%    \end{macrocode}
%    \end{macro}
%
%    \begin{macro}{\HOLOGO@PointReflectBox}
%    \begin{macrocode}
\def\HOLOGO@PointReflectBox#1{%
  \begingroup
    \setbox\ltx@zero\hbox{\begingroup#1\endgroup}%
    \setbox\ltx@two\hbox{%
      \kern\wd\ltx@zero
      \raise\ht\ltx@zero\hbox{%
        \csname HOLOGO@ScaleBox@\hologoDriver\endcsname{-1}{-1}{%
          \hbox to 0pt{\copy\ltx@zero\hss}%
        }%
      }%
    }%
    \wd\ltx@two=\wd\ltx@zero
    \box\ltx@two
  \endgroup
}
%    \end{macrocode}
%    \end{macro}
%
%    We must define all variants because of dynamic driver setup.
%    \begin{macrocode}
\def\HOLOGO@temp#1#2{#2}
%    \end{macrocode}
%
%    \begin{macro}{\HOLOGO@ScaleBox@pdftex}
%    \begin{macrocode}
\HOLOGO@temp{pdftex}{%
  \def\HOLOGO@ScaleBox@pdftex#1#2#3{%
    \HOLOGO@pdfliteral{%
      q #1 0 0 #2 0 0 cm%
    }%
    #3%
    \HOLOGO@pdfliteral{%
      Q%
    }%
  }%
}
%    \end{macrocode}
%    \end{macro}
%    \begin{macro}{\HOLOGO@ScaleBox@dvips}
%    \begin{macrocode}
\HOLOGO@temp{dvips}{%
  \def\HOLOGO@ScaleBox@dvips#1#2#3{%
    \special{ps:%
      gsave %
      currentpoint %
      currentpoint translate %
      #1 #2 scale %
      neg exch neg exch translate%
    }%
    #3%
    \special{ps:%
      currentpoint %
      grestore %
      moveto%
    }%
  }%
}
%    \end{macrocode}
%    \end{macro}
%    \begin{macro}{\HOLOGO@ScaleBox@dvipdfm}
%    \begin{macrocode}
\HOLOGO@temp{dvipdfm}{%
  \let\HOLOGO@ScaleBox@dvipdfm\HOLOGO@ScaleBox@dvips
}
%    \end{macrocode}
%    \end{macro}
%    Since \hologo{XeTeX} v0.6.
%    \begin{macro}{\HOLOGO@ScaleBox@xetex}
%    \begin{macrocode}
\HOLOGO@temp{xetex}{%
  \def\HOLOGO@ScaleBox@xetex#1#2#3{%
    \special{x:gsave}%
    \special{x:scale #1 #2}%
    #3%
    \special{x:grestore}%
  }%
}
%    \end{macrocode}
%    \end{macro}
%    \begin{macro}{\HOLOGO@ScaleBox@vtex}
%    \begin{macrocode}
\HOLOGO@temp{vtex}{%
  \def\HOLOGO@ScaleBox@vtex#1#2#3{%
    \special{r(#1,0,0,#2,0,0}%
    #3%
    \special{r)}%
  }%
}
%    \end{macrocode}
%    \end{macro}
%
%    \begin{macrocode}
\HOLOGO@AtEnd%
%</package>
%    \end{macrocode}
%
% \section{Test}
%
% \subsection{Catcode checks for loading}
%
%    \begin{macrocode}
%<*test1>
%    \end{macrocode}
%    \begin{macrocode}
\catcode`\{=1 %
\catcode`\}=2 %
\catcode`\#=6 %
\catcode`\@=11 %
\expandafter\ifx\csname count@\endcsname\relax
  \countdef\count@=255 %
\fi
\expandafter\ifx\csname @gobble\endcsname\relax
  \long\def\@gobble#1{}%
\fi
\expandafter\ifx\csname @firstofone\endcsname\relax
  \long\def\@firstofone#1{#1}%
\fi
\expandafter\ifx\csname loop\endcsname\relax
  \expandafter\@firstofone
\else
  \expandafter\@gobble
\fi
{%
  \def\loop#1\repeat{%
    \def\body{#1}%
    \iterate
  }%
  \def\iterate{%
    \body
      \let\next\iterate
    \else
      \let\next\relax
    \fi
    \next
  }%
  \let\repeat=\fi
}%
\def\RestoreCatcodes{}
\count@=0 %
\loop
  \edef\RestoreCatcodes{%
    \RestoreCatcodes
    \catcode\the\count@=\the\catcode\count@\relax
  }%
\ifnum\count@<255 %
  \advance\count@ 1 %
\repeat

\def\RangeCatcodeInvalid#1#2{%
  \count@=#1\relax
  \loop
    \catcode\count@=15 %
  \ifnum\count@<#2\relax
    \advance\count@ 1 %
  \repeat
}
\def\RangeCatcodeCheck#1#2#3{%
  \count@=#1\relax
  \loop
    \ifnum#3=\catcode\count@
    \else
      \errmessage{%
        Character \the\count@\space
        with wrong catcode \the\catcode\count@\space
        instead of \number#3%
      }%
    \fi
  \ifnum\count@<#2\relax
    \advance\count@ 1 %
  \repeat
}
\def\space{ }
\expandafter\ifx\csname LoadCommand\endcsname\relax
  \def\LoadCommand{\input hologo.sty\relax}%
\fi
\def\Test{%
  \RangeCatcodeInvalid{0}{47}%
  \RangeCatcodeInvalid{58}{64}%
  \RangeCatcodeInvalid{91}{96}%
  \RangeCatcodeInvalid{123}{255}%
  \catcode`\@=12 %
  \catcode`\\=0 %
  \catcode`\%=14 %
  \LoadCommand
  \RangeCatcodeCheck{0}{36}{15}%
  \RangeCatcodeCheck{37}{37}{14}%
  \RangeCatcodeCheck{38}{47}{15}%
  \RangeCatcodeCheck{48}{57}{12}%
  \RangeCatcodeCheck{58}{63}{15}%
  \RangeCatcodeCheck{64}{64}{12}%
  \RangeCatcodeCheck{65}{90}{11}%
  \RangeCatcodeCheck{91}{91}{15}%
  \RangeCatcodeCheck{92}{92}{0}%
  \RangeCatcodeCheck{93}{96}{15}%
  \RangeCatcodeCheck{97}{122}{11}%
  \RangeCatcodeCheck{123}{255}{15}%
  \RestoreCatcodes
}
\Test
\csname @@end\endcsname
\end
%    \end{macrocode}
%    \begin{macrocode}
%</test1>
%    \end{macrocode}
%
% \subsection{Spacefactor}
%
%    The space factor must be 1000 after a logo. If it is greater 1000
%    then the following space is a space after a sentence closing point.
%    If the space factor is smaller 1000 then an immediate following
%    dot is interpreted as abbreviation, not sentence closing point.
%
%    \begin{macrocode}
%<*test-spacefactor>
\NeedsTeXFormat{LaTeX2e}
\documentclass{article}
\usepackage{hologo}[2016/05/12]
\usepackage{kvsetkeys}
\usepackage{qstest}
\IncludeTests{*}
\LogTests{log}{*}{*}
\begin{document}
\begin{qstest}{spacefactor}{spacefactor}
\newcommand*{\Test}[1]{%
  \sbox0{%
    \hologo{#1}%
    \Expect*{1000 (#1)}*{\the\spacefactor\space(#1)}%
  }%
}%
\makeatletter
\def\TestList{}
\def\hologoEntry#1#2#3{%
  \edef\TestList{%
    \ifx\TestList\@empty
    \else
      \TestList,%
    \fi
    #1%
    \ifx\\#2\\%
    \else
      ={variant=#2}%
    \fi
  }%
}
\hologoList
\expandafter\kv@parse@normalized\expandafter{%
  \TestList
}{%
  \begingroup
    \let\@logo=\kv@key
    \ifx\kv@value\relax
    \else
      \expandafter\hologoLogoSetup\expandafter\@logo\expandafter{%
        \kv@value
      }%
    \fi
    \Test\@logo
  \endgroup
  \@gobbletwo
}
\end{qstest}
\end{document}
%</test-spacefactor>
%    \end{macrocode}
%
% \subsection{Complete list}
%
%    \begin{macrocode}
%<*test-list>
\NeedsTeXFormat{LaTeX2e}
\documentclass[12pt,a4paper]{article}
\usepackage{hologo}[2016/05/12]
\usepackage[T1]{fontenc}
\usepackage{lmodern}
\usepackage{parskip}
\usepackage[unicode]{hyperref}[2011/09/28]
\usepackage{bookmark}[2011/09/19]
\bookmarksetup{%
  numbered,%
  open,%
  openlevel=2,%
}
\renewcommand*{\contentsname}{List of logos}
\begin{document}
\tableofcontents
\def\TestFont#1#2#3#4#5#6{%
  \begingroup
    \usefont{#3}{#4}{#5}{#6}%
    \HologoVariant{#1}{#2}/\hologoVariant{#1}{#2}%
    \quad
    \begingroup\scriptsize\hologoVariant{#1}{#2}\endgroup
    \quad
  \endgroup
  (#3/#4/#5/#6)%
  \par
}
\makeatletter
\def\hologoEntry#1#2#3{%
  \section{%
    \HologoVariant{#1}{#2}/\hologoVariant{#1}{#2} %
    {[#1\ifx\\#2\\\else\space(#2)\fi]}% hash-ok
  }% braces around [] because of bug in tex4ht
  \begingroup
    \hypersetup{unicode=false}%
    \bookmark[%
      dest=\@currentHref,%
      rellevel=1,%
      keeplevel,%
    ]{%
      \HologoVariant{#1}{#2}/\hologoVariant{#1}{#2} %
      (PDFDocEncoding)%
    }%
  \endgroup
  \TestFont{#1}{#2}{OT1}{cmr}{m}{n}%
  \TestFont{#1}{#2}{OT1}{cmss}{m}{n}%
  \TestFont{#1}{#2}{OT1}{cmr}{b}{n}%
  \TestFont{#1}{#2}{OT1}{cmr}{m}{it}%
  \TestFont{#1}{#2}{OT1}{cmtt}{m}{n}%
  \TestFont{#1}{#2}{T1}{lmr}{m}{n}%
  \TestFont{#1}{#2}{T1}{lmss}{m}{n}%
  \TestFont{#1}{#2}{T1}{lmr}{b}{n}%
  \TestFont{#1}{#2}{T1}{lmr}{m}{it}%
  \TestFont{#1}{#2}{T1}{lmtt}{m}{n}%
  \TestFont{#1}{#2}{T1}{lmvtt}{m}{n}%
  \TestFont{#1}{#2}{T1}{qtm}{m}{n}%
  \TestFont{#1}{#2}{T1}{qhv}{m}{n}%
  \TestFont{#1}{#2}{T1}{qtm}{b}{n}%
  \TestFont{#1}{#2}{T1}{qtm}{m}{it}%
  \TestFont{#1}{#2}{T1}{qcr}{m}{n}%
  \newpage
}
\makeatother
\hologoList
\end{document}
%</test-list>
%    \end{macrocode}
%
% \section{Installation}
%
% \subsection{Download}
%
% \paragraph{Package.} This package is available on
% CTAN\footnote{\url{ftp://ftp.ctan.org/tex-archive/}}:
% \begin{description}
% \item[\CTAN{macros/latex/contrib/oberdiek/hologo.dtx}] The source file.
% \item[\CTAN{macros/latex/contrib/oberdiek/hologo.pdf}] Documentation.
% \end{description}
%
%
% \paragraph{Bundle.} All the packages of the bundle `oberdiek'
% are also available in a TDS compliant ZIP archive. There
% the packages are already unpacked and the documentation files
% are generated. The files and directories obey the TDS standard.
% \begin{description}
% \item[\CTAN{install/macros/latex/contrib/oberdiek.tds.zip}]
% \end{description}
% \emph{TDS} refers to the standard ``A Directory Structure
% for \TeX\ Files'' (\CTAN{tds/tds.pdf}). Directories
% with \xfile{texmf} in their name are usually organized this way.
%
% \subsection{Bundle installation}
%
% \paragraph{Unpacking.} Unpack the \xfile{oberdiek.tds.zip} in the
% TDS tree (also known as \xfile{texmf} tree) of your choice.
% Example (linux):
% \begin{quote}
%   |unzip oberdiek.tds.zip -d ~/texmf|
% \end{quote}
%
% \paragraph{Script installation.}
% Check the directory \xfile{TDS:scripts/oberdiek/} for
% scripts that need further installation steps.
% Package \xpackage{attachfile2} comes with the Perl script
% \xfile{pdfatfi.pl} that should be installed in such a way
% that it can be called as \texttt{pdfatfi}.
% Example (linux):
% \begin{quote}
%   |chmod +x scripts/oberdiek/pdfatfi.pl|\\
%   |cp scripts/oberdiek/pdfatfi.pl /usr/local/bin/|
% \end{quote}
%
% \subsection{Package installation}
%
% \paragraph{Unpacking.} The \xfile{.dtx} file is a self-extracting
% \docstrip\ archive. The files are extracted by running the
% \xfile{.dtx} through \plainTeX:
% \begin{quote}
%   \verb|tex hologo.dtx|
% \end{quote}
%
% \paragraph{TDS.} Now the different files must be moved into
% the different directories in your installation TDS tree
% (also known as \xfile{texmf} tree):
% \begin{quote}
% \def\t{^^A
% \begin{tabular}{@{}>{\ttfamily}l@{ $\rightarrow$ }>{\ttfamily}l@{}}
%   hologo.sty & tex/generic/oberdiek/hologo.sty\\
%   hologo.pdf & doc/latex/oberdiek/hologo.pdf\\
%   example/hologo-example.tex & doc/latex/oberdiek/example/hologo-example.tex\\
%   test/hologo-test1.tex & doc/latex/oberdiek/test/hologo-test1.tex\\
%   test/hologo-test-spacefactor.tex & doc/latex/oberdiek/test/hologo-test-spacefactor.tex\\
%   test/hologo-test-list.tex & doc/latex/oberdiek/test/hologo-test-list.tex\\
%   hologo.dtx & source/latex/oberdiek/hologo.dtx\\
% \end{tabular}^^A
% }^^A
% \sbox0{\t}^^A
% \ifdim\wd0>\linewidth
%   \begingroup
%     \advance\linewidth by\leftmargin
%     \advance\linewidth by\rightmargin
%   \edef\x{\endgroup
%     \def\noexpand\lw{\the\linewidth}^^A
%   }\x
%   \def\lwbox{^^A
%     \leavevmode
%     \hbox to \linewidth{^^A
%       \kern-\leftmargin\relax
%       \hss
%       \usebox0
%       \hss
%       \kern-\rightmargin\relax
%     }^^A
%   }^^A
%   \ifdim\wd0>\lw
%     \sbox0{\small\t}^^A
%     \ifdim\wd0>\linewidth
%       \ifdim\wd0>\lw
%         \sbox0{\footnotesize\t}^^A
%         \ifdim\wd0>\linewidth
%           \ifdim\wd0>\lw
%             \sbox0{\scriptsize\t}^^A
%             \ifdim\wd0>\linewidth
%               \ifdim\wd0>\lw
%                 \sbox0{\tiny\t}^^A
%                 \ifdim\wd0>\linewidth
%                   \lwbox
%                 \else
%                   \usebox0
%                 \fi
%               \else
%                 \lwbox
%               \fi
%             \else
%               \usebox0
%             \fi
%           \else
%             \lwbox
%           \fi
%         \else
%           \usebox0
%         \fi
%       \else
%         \lwbox
%       \fi
%     \else
%       \usebox0
%     \fi
%   \else
%     \lwbox
%   \fi
% \else
%   \usebox0
% \fi
% \end{quote}
% If you have a \xfile{docstrip.cfg} that configures and enables \docstrip's
% TDS installing feature, then some files can already be in the right
% place, see the documentation of \docstrip.
%
% \subsection{Refresh file name databases}
%
% If your \TeX~distribution
% (\teTeX, \mikTeX, \dots) relies on file name databases, you must refresh
% these. For example, \teTeX\ users run \verb|texhash| or
% \verb|mktexlsr|.
%
% \subsection{Some details for the interested}
%
% \paragraph{Attached source.}
%
% The PDF documentation on CTAN also includes the
% \xfile{.dtx} source file. It can be extracted by
% AcrobatReader 6 or higher. Another option is \textsf{pdftk},
% e.g. unpack the file into the current directory:
% \begin{quote}
%   \verb|pdftk hologo.pdf unpack_files output .|
% \end{quote}
%
% \paragraph{Unpacking with \LaTeX.}
% The \xfile{.dtx} chooses its action depending on the format:
% \begin{description}
% \item[\plainTeX:] Run \docstrip\ and extract the files.
% \item[\LaTeX:] Generate the documentation.
% \end{description}
% If you insist on using \LaTeX\ for \docstrip\ (really,
% \docstrip\ does not need \LaTeX), then inform the autodetect routine
% about your intention:
% \begin{quote}
%   \verb|latex \let\install=y% \iffalse meta-comment
%
% File: hologo.dtx
% Version: 2016/05/12 v1.11
% Info: A logo collection with bookmark support
%
% Copyright (C) 2010-2012 by
%    Heiko Oberdiek <heiko.oberdiek at googlemail.com>
%
% This work may be distributed and/or modified under the
% conditions of the LaTeX Project Public License, either
% version 1.3c of this license or (at your option) any later
% version. This version of this license is in
%    http://www.latex-project.org/lppl/lppl-1-3c.txt
% and the latest version of this license is in
%    http://www.latex-project.org/lppl.txt
% and version 1.3 or later is part of all distributions of
% LaTeX version 2005/12/01 or later.
%
% This work has the LPPL maintenance status "maintained".
%
% This Current Maintainer of this work is Heiko Oberdiek.
%
% The Base Interpreter refers to any `TeX-Format',
% because some files are installed in TDS:tex/generic//.
%
% This work consists of the main source file hologo.dtx
% and the derived files
%    hologo.sty, hologo.pdf, hologo.ins, hologo.drv, hologo-example.tex,
%    hologo-test1.tex, hologo-test-spacefactor.tex,
%    hologo-test-list.tex.
%
% Distribution:
%    CTAN:macros/latex/contrib/oberdiek/hologo.dtx
%    CTAN:macros/latex/contrib/oberdiek/hologo.pdf
%
% Unpacking:
%    (a) If hologo.ins is present:
%           tex hologo.ins
%    (b) Without hologo.ins:
%           tex hologo.dtx
%    (c) If you insist on using LaTeX
%           latex \let\install=y\input{hologo.dtx}
%        (quote the arguments according to the demands of your shell)
%
% Documentation:
%    (a) If hologo.drv is present:
%           latex hologo.drv
%    (b) Without hologo.drv:
%           latex hologo.dtx; ...
%    The class ltxdoc loads the configuration file ltxdoc.cfg
%    if available. Here you can specify further options, e.g.
%    use A4 as paper format:
%       \PassOptionsToClass{a4paper}{article}
%
%    Programm calls to get the documentation (example):
%       pdflatex hologo.dtx
%       makeindex -s gind.ist hologo.idx
%       pdflatex hologo.dtx
%       makeindex -s gind.ist hologo.idx
%       pdflatex hologo.dtx
%
% Installation:
%    TDS:tex/generic/oberdiek/hologo.sty
%    TDS:doc/latex/oberdiek/hologo.pdf
%    TDS:doc/latex/oberdiek/example/hologo-example.tex
%    TDS:doc/latex/oberdiek/test/hologo-test1.tex
%    TDS:doc/latex/oberdiek/test/hologo-test-spacefactor.tex
%    TDS:doc/latex/oberdiek/test/hologo-test-list.tex
%    TDS:source/latex/oberdiek/hologo.dtx
%
%<*ignore>
\begingroup
  \catcode123=1 %
  \catcode125=2 %
  \def\x{LaTeX2e}%
\expandafter\endgroup
\ifcase 0\ifx\install y1\fi\expandafter
         \ifx\csname processbatchFile\endcsname\relax\else1\fi
         \ifx\fmtname\x\else 1\fi\relax
\else\csname fi\endcsname
%</ignore>
%<*install>
\input docstrip.tex
\Msg{************************************************************************}
\Msg{* Installation}
\Msg{* Package: hologo 2016/05/12 v1.11 A logo collection with bookmark support (HO)}
\Msg{************************************************************************}

\keepsilent
\askforoverwritefalse

\let\MetaPrefix\relax
\preamble

This is a generated file.

Project: hologo
Version: 2016/05/12 v1.11

Copyright (C) 2010-2012 by
   Heiko Oberdiek <heiko.oberdiek at googlemail.com>

This work may be distributed and/or modified under the
conditions of the LaTeX Project Public License, either
version 1.3c of this license or (at your option) any later
version. This version of this license is in
   http://www.latex-project.org/lppl/lppl-1-3c.txt
and the latest version of this license is in
   http://www.latex-project.org/lppl.txt
and version 1.3 or later is part of all distributions of
LaTeX version 2005/12/01 or later.

This work has the LPPL maintenance status "maintained".

This Current Maintainer of this work is Heiko Oberdiek.

The Base Interpreter refers to any `TeX-Format',
because some files are installed in TDS:tex/generic//.

This work consists of the main source file hologo.dtx
and the derived files
   hologo.sty, hologo.pdf, hologo.ins, hologo.drv, hologo-example.tex,
   hologo-test1.tex, hologo-test-spacefactor.tex,
   hologo-test-list.tex.

\endpreamble
\let\MetaPrefix\DoubleperCent

\generate{%
  \file{hologo.ins}{\from{hologo.dtx}{install}}%
  \file{hologo.drv}{\from{hologo.dtx}{driver}}%
  \usedir{tex/generic/oberdiek}%
  \file{hologo.sty}{\from{hologo.dtx}{package}}%
  \usedir{doc/latex/oberdiek/example}%
  \file{hologo-example.tex}{\from{hologo.dtx}{example}}%
  \usedir{doc/latex/oberdiek/test}%
  \file{hologo-test1.tex}{\from{hologo.dtx}{test1}}%
  \file{hologo-test-spacefactor.tex}{\from{hologo.dtx}{test-spacefactor}}%
  \file{hologo-test-list.tex}{\from{hologo.dtx}{test-list}}%
  \nopreamble
  \nopostamble
  \usedir{source/latex/oberdiek/catalogue}%
  \file{hologo.xml}{\from{hologo.dtx}{catalogue}}%
}

\catcode32=13\relax% active space
\let =\space%
\Msg{************************************************************************}
\Msg{*}
\Msg{* To finish the installation you have to move the following}
\Msg{* file into a directory searched by TeX:}
\Msg{*}
\Msg{*     hologo.sty}
\Msg{*}
\Msg{* To produce the documentation run the file `hologo.drv'}
\Msg{* through LaTeX.}
\Msg{*}
\Msg{* Happy TeXing!}
\Msg{*}
\Msg{************************************************************************}

\endbatchfile
%</install>
%<*ignore>
\fi
%</ignore>
%<*driver>
\NeedsTeXFormat{LaTeX2e}
\ProvidesFile{hologo.drv}%
  [2016/05/12 v1.11 A logo collection with bookmark support (HO)]%
\documentclass{ltxdoc}
\usepackage{holtxdoc}[2011/11/22]
\usepackage{hologo}[2016/05/12]
\usepackage{longtable}
\usepackage{array}
\usepackage{paralist}
%\usepackage[T1]{fontenc}
%\usepackage{lmodern}
\begin{document}
  \DocInput{hologo.dtx}%
\end{document}
%</driver>
% \fi
%
%
% \CharacterTable
%  {Upper-case    \A\B\C\D\E\F\G\H\I\J\K\L\M\N\O\P\Q\R\S\T\U\V\W\X\Y\Z
%   Lower-case    \a\b\c\d\e\f\g\h\i\j\k\l\m\n\o\p\q\r\s\t\u\v\w\x\y\z
%   Digits        \0\1\2\3\4\5\6\7\8\9
%   Exclamation   \!     Double quote  \"     Hash (number) \#
%   Dollar        \$     Percent       \%     Ampersand     \&
%   Acute accent  \'     Left paren    \(     Right paren   \)
%   Asterisk      \*     Plus          \+     Comma         \,
%   Minus         \-     Point         \.     Solidus       \/
%   Colon         \:     Semicolon     \;     Less than     \<
%   Equals        \=     Greater than  \>     Question mark \?
%   Commercial at \@     Left bracket  \[     Backslash     \\
%   Right bracket \]     Circumflex    \^     Underscore    \_
%   Grave accent  \`     Left brace    \{     Vertical bar  \|
%   Right brace   \}     Tilde         \~}
%
% \GetFileInfo{hologo.drv}
%
% \title{The \xpackage{hologo} package}
% \date{2016/05/12 v1.11}
% \author{Heiko Oberdiek\\\xemail{heiko.oberdiek at googlemail.com}}
%
% \maketitle
%
% \begin{abstract}
% This package starts a collection of logos with support for bookmarks
% strings.
% \end{abstract}
%
% \tableofcontents
%
% \section{Documentation}
%
% \subsection{Logo macros}
%
% \begin{declcs}{hologo} \M{name}
% \end{declcs}
% Macro \cs{hologo} sets the logo with name \meta{name}.
% The following table shows the supported names.
%
% \begingroup
%   \def\hologoEntry#1#2#3{^^A
%     #1&#2&\hologoLogoSetup{#1}{variant=#2}\hologo{#1}&#3\tabularnewline
%   }
%   \begin{longtable}{>{\ttfamily}l>{\ttfamily}lll}
%     \rmfamily\bfseries{name} & \rmfamily\bfseries variant
%     & \bfseries logo & \bfseries since\\
%     \hline
%     \endhead
%     \hologoList
%   \end{longtable}
% \endgroup
%
% \begin{declcs}{Hologo} \M{name}
% \end{declcs}
% Macro \cs{Hologo} starts the logo \meta{name} with an uppercase
% letter. As an exception small greek letters are not converted
% to uppercase. Examples, see \hologo{eTeX} and \hologo{ExTeX}.
%
% \subsection{Setup macros}
%
% The package does not support package options, but the following
% setup macros can be used to set options.
%
% \begin{declcs}{hologoSetup} \M{key value list}
% \end{declcs}
% Macro \cs{hologoSetup} sets global options.
%
% \begin{declcs}{hologoLogoSetup} \M{logo} \M{key value list}
% \end{declcs}
% Some options can also be used to configure a logo.
% These settings take precedence over global option settings.
%
% \subsection{Options}\label{sec:options}
%
% There are boolean and string options:
% \begin{description}
% \item[Boolean option:]
% It takes |true| or |false|
% as value. If the value is omitted, then |true| is used.
% \item[String option:]
% A value must be given as string. (But the string might be empty.)
% \end{description}
% The following options can be used both in \cs{hologoSetup}
% and \cs{hologoLogoSetup}:
% \begin{description}
% \def\entry#1{\item[\xoption{#1}:]}
% \entry{break}
%   enables or disables line breaks inside the logo. This setting is
%   refined by options \xoption{hyphenbreak}, \xoption{spacebreak}
%   or \xoption{discretionarybreak}.
%   Default is |false|.
% \entry{hyphenbreak}
%   enables or disables the line break right after the hyphen character.
% \entry{spacebreak}
%   enables or disables line breaks at space characters.
% \entry{discretionarybreak}
%   enables or disables line breaks at hyphenation points
%   (inserted by \cs{-}).
% \end{description}
% Macro \cs{hologoLogoSetup} also knows:
% \begin{description}
% \item[\xoption{variant}:]
%   This is a string option. It specifies a variant of a logo that
%   must exist. An empty string selects the package default variant.
% \end{description}
% Example:
% \begin{quote}
%   |\hologoSetup{break=false}|\\
%   |\hologoLogoSetup{plainTeX}{variant=hyphen,hyphenbreak}|\\
%   Then ``plain-\TeX'' contains one break point after the hyphen.
% \end{quote}
%
% \subsection{Driver options}
%
% Sometimes graphical operations are needed to construct some
% glyphs (e.g.\ \hologo{XeTeX}). If package \xpackage{graphics}
% or package \xpackage{pgf} are found, then the macros are taken
% from there. Otherwise the packge defines its own operations
% and therefore needs the driver information. Many drivers are
% detected automatically (\hologo{pdfTeX}/\hologo{LuaTeX}
% in PDF mode, \hologo{XeTeX}, \hologo{VTeX}). These have precedence
% over a driver option. The driver can be given as package option
% or using \cs{hologoDriverSetup}.
% The following list contains the recognized driver options:
% \begin{itemize}
% \item \xoption{pdftex}, \xoption{luatex}
% \item \xoption{dvipdfm}, \xoption{dvipdfmx}
% \item \xoption{dvips}, \xoption{dvipsone}, \xoption{xdvi}
% \item \xoption{xetex}
% \item \xoption{vtex}
% \end{itemize}
% The left driver of a line is the driver name that is used internally.
% The following names are aliases for drivers that use the
% same method. Therefore the entry in the \xext{log} file for
% the used driver prints the internally used driver name.
% \begin{description}
% \item[\xoption{driverfallback}:]
%   This option expects a driver that is used,
%   if the driver could not be detected automatically.
% \end{description}
%
% \begin{declcs}{hologoDriverSetup} \M{driver option}
% \end{declcs}
% The driver can also be configured after package loading
% using \cs{hologoDriverSetup}, also the way for \hologo{plainTeX}
% to setup the driver.
%
% \subsection{Font setup}
%
% Some logos require a special font, but should also be usable by
% \hologo{plainTeX}. Therefore the package provides some ways
% to influence the font settings. The options below
% take font settings as values. Both font commands
% such as \cs{sffamily} and macros that take one argument
% like \cs{textsf} can be used.
%
% \begin{declcs}{hologoFontSetup} \M{key value list}
% \end{declcs}
% Macro \cs{hologoFontSetup} sets the fonts for all logos.
% Supported keys:
% \begin{description}
% \def\entry#1{\item[\xoption{#1}:]}
% \entry{general}
%   This font is used for all logos. The default is empty.
%   That means no special font is used.
% \entry{bibsf}
%   This font is used for
%   {\hologoLogoSetup{BibTeX}{variant=sf}\hologo{BibTeX}}
%   with variant \xoption{sf}.
% \entry{rm}
%   This font is a serif font. It is used for \hologo{ExTeX}.
% \entry{sc}
%   This font specifies a small caps font. It is used for
%   {\hologoLogoSetup{BibTeX}{variant=sc}\hologo{BibTeX}}
%   with variant \xoption{sc}.
% \entry{sf}
%   This font specifies a sans serif font. The default
%   is \cs{sffamily}, then \cs{sf} is tried. Otherwise
%   a warning is given. It is used by \hologo{KOMAScript}.
% \entry{sy}
%   This is the font for math symbols (e.g. cmsy).
%   It is used by \hologo{AmS}, \hologo{NTS}, \hologo{ExTeX}.
% \entry{logo}
%   \hologo{METAFONT} and \hologo{METAPOST} are using that font.
%   In \hologo{LaTeX} \cs{logofamily} is used and
%   the definitions of package \xpackage{mflogo} are used
%   if the package is not loaded.
%   Otherwise the \cs{tenlogo} is used and defined
%   if it does not already exists.
% \end{description}
%
% \begin{declcs}{hologoLogoFontSetup} \M{logo} \M{key value list}
% \end{declcs}
% Fonts can also be set for a logo or logo component separately,
% see the following list.
% The keys are the same as for \cs{hologoFontSetup}.
%
% \begin{longtable}{>{\ttfamily}l>{\sffamily}ll}
%   \meta{logo} & keys & result\\
%   \hline
%   \endhead
%   BibTeX & bibsf & {\hologoLogoSetup{BibTeX}{variant=sf}\hologo{BibTeX}}\\[.5ex]
%   BibTeX & sc & {\hologoLogoSetup{BibTeX}{variant=sc}\hologo{BibTeX}}\\[.5ex]
%   ExTeX & rm & \hologo{ExTeX}\\
%   SliTeX & rm & \hologo{SliTeX}\\[.5ex]
%   AmS & sy & \hologo{AmS}\\
%   ExTeX & sy & \hologo{ExTeX}\\
%   NTS & sy & \hologo{NTS}\\[.5ex]
%   KOMAScript & sf & \hologo{KOMAScript}\\[.5ex]
%   METAFONT & logo & \hologo{METAFONT}\\
%   METAPOST & logo & \hologo{METAPOST}\\[.5ex]
%   SliTeX & sc \hologo{SliTeX}
% \end{longtable}
%
% \subsubsection{Font order}
%
% For all logos the font \xoption{general} is applied first.
% Example:
%\begin{quote}
%|\hologoFontSetup{general=\color{red}}|
%\end{quote}
% will print red logos.
% Then if the font uses a special font \xoption{sf}, for example,
% the font is applied that is setup by \cs{hologoLogoFontSetup}.
% If this font is not setup, then the common font setup
% by \cs{hologoFontSetup} is used. Otherwise a warning is given,
% that there is no font configured.
%
% \subsection{Additional user macros}
%
% Usually a variant of a logo is configured by using
% \cs{hologoLogoSetup}, because it is bad style to mix
% different variants of the same logo in the same text.
% There the following macros are a convenience for testing.
%
% \begin{declcs}{hologoVariant} \M{name} \M{variant}\\
%   \cs{HologoVariant} \M{name} \M{variant}
% \end{declcs}
% Logo \meta{name} is set using \meta{variant} that specifies
% explicitely which variant of the macro is used. If the argument
% is empty, then the default form of the logo is used
% (configurable by \cs{hologoLogoSetup}).
%
% \cs{HologoVariant} is used if the logo is set in a context
% that needs an uppercase first letter (beginning of a sentence, \dots).
%
% \begin{declcs}{hologoList}\\
%   \cs{hologoEntry} \M{logo} \M{variant} \M{since}
% \end{declcs}
% Macro \cs{hologoList} contains all logos that are provided
% by the package including variants. The list consists of calls
% of \cs{hologoEntry} with three arguments starting with the
% logo name \meta{logo} and its variant \meta{variant}. An empty
% variant means the current default. Argument \meta{since} specifies
% with version of the package \xpackage{hologo} is needed to get
% the logo. If the logo is fixed, then the date gets updated.
% Therefore the date \meta{since} is not exactly the date of
% the first introduction, but rather the date of the latest fix.
%
% Before \cs{hologoList} can be used, macro \cs{hologoEntry} needs
% a definition. The example file in section \ref{sec:example}
% shows applications of \cs{hologoList}.
%
% \subsection{Supported contexts}
%
% Macros \cs{hologo} and friends support special contexts:
% \begin{itemize}
% \item \hologo{LaTeX}'s protection mechanism.
% \item Bookmarks of package \xpackage{hyperref}.
% \item Package \xpackage{tex4ht}.
% \item The macros can be used inside \cs{csname} constructs,
%   if \cs{ifincsname} is available (\hologo{pdfTeX}, \hologo{XeTeX},
%   \hologo{LuaTeX}).
% \end{itemize}
%
% \subsection{Example}
% \label{sec:example}
%
% The following example prints the logos in different fonts.
%    \begin{macrocode}
%<*example>
%<<verbatim
\NeedsTeXFormat{LaTeX2e}
\documentclass[a4paper]{article}
\usepackage[
  hmargin=20mm,
  vmargin=20mm,
]{geometry}
\pagestyle{empty}
\usepackage{hologo}[2016/05/12]
\usepackage{longtable}
\usepackage{array}
\setlength{\extrarowheight}{2pt}
\usepackage[T1]{fontenc}
\usepackage{lmodern}
\usepackage{pdflscape}
\usepackage[
  pdfencoding=auto,
]{hyperref}
\hypersetup{
  pdfauthor={Heiko Oberdiek},
  pdftitle={Example for package `hologo'},
  pdfsubject={Logos with fonts lmr, lmss, qtm, qpl, qhv},
}
\usepackage{bookmark}

% Print the logo list on the console

\begingroup
  \typeout{}%
  \typeout{*** Begin of logo list ***}%
  \newcommand*{\hologoEntry}[3]{%
    \typeout{#1 \ifx\\#2\\\else(#2) \fi[#3]}%
  }%
  \hologoList
  \typeout{*** End of logo list ***}%
  \typeout{}%
\endgroup

\begin{document}
\begin{landscape}

  \section{Example file for package `hologo'}

  % Table for font names

  \begin{longtable}{>{\bfseries}ll}
    \textbf{font} & \textbf{Font name}\\
    \hline
    lmr & Latin Modern Roman\\
    lmss & Latin Modern Sans\\
    qtm & \TeX\ Gyre Termes\\
    qhv & \TeX\ Gyre Heros\\
    qpl & \TeX\ Gyre Pagella\\
  \end{longtable}

  % Logo list with logos in different fonts

  \begingroup
    \newcommand*{\SetVariant}[2]{%
      \ifx\\#2\\%
      \else
        \hologoLogoSetup{#1}{variant=#2}%
      \fi
    }%
    \newcommand*{\hologoEntry}[3]{%
      \SetVariant{#1}{#2}%
      \raisebox{1em}[0pt][0pt]{\hypertarget{#1@#2}{}}%
      \bookmark[%
        dest={#1@#2},%
      ]{%
        #1\ifx\\#2\\\else\space(#2)\fi: \Hologo{#1}, \hologo{#1} %
        [Unicode]%
      }%
      \hypersetup{unicode=false}%
      \bookmark[%
        dest={#1@#2},%
      ]{%
        #1\ifx\\#2\\\else\space(#2)\fi: \Hologo{#1}, \hologo{#1} %
        [PDFDocEncoding]%
      }%
      \texttt{#1}%
      &%
      \texttt{#2}%
      &%
      \Hologo{#1}%
      &%
      \SetVariant{#1}{#2}%
      \hologo{#1}%
      &%
      \SetVariant{#1}{#2}%
      \fontfamily{qtm}\selectfont
      \hologo{#1}%
      &%
      \SetVariant{#1}{#2}%
      \fontfamily{qpl}\selectfont
      \hologo{#1}%
      &%
      \SetVariant{#1}{#2}%
      \textsf{\hologo{#1}}%
      &%
      \SetVariant{#1}{#2}%
      \fontfamily{qhv}\selectfont
      \hologo{#1}%
      \tabularnewline
    }%
    \begin{longtable}{llllllll}%
      \textbf{\textit{logo}} & \textbf{\textit{variant}} &
      \texttt{\string\Hologo} &
      \textbf{lmr} & \textbf{qtm} & \textbf{qpl} &
      \textbf{lmss} & \textbf{qhv}
      \tabularnewline
      \hline
      \endhead
      \hologoList
    \end{longtable}%
  \endgroup

\end{landscape}
\end{document}
%verbatim
%</example>
%    \end{macrocode}
%
% \StopEventually{
% }
%
% \section{Implementation}
%    \begin{macrocode}
%<*package>
%    \end{macrocode}
%    Reload check, especially if the package is not used with \LaTeX.
%    \begin{macrocode}
\begingroup\catcode61\catcode48\catcode32=10\relax%
  \catcode13=5 % ^^M
  \endlinechar=13 %
  \catcode35=6 % #
  \catcode39=12 % '
  \catcode44=12 % ,
  \catcode45=12 % -
  \catcode46=12 % .
  \catcode58=12 % :
  \catcode64=11 % @
  \catcode123=1 % {
  \catcode125=2 % }
  \expandafter\let\expandafter\x\csname ver@hologo.sty\endcsname
  \ifx\x\relax % plain-TeX, first loading
  \else
    \def\empty{}%
    \ifx\x\empty % LaTeX, first loading,
      % variable is initialized, but \ProvidesPackage not yet seen
    \else
      \expandafter\ifx\csname PackageInfo\endcsname\relax
        \def\x#1#2{%
          \immediate\write-1{Package #1 Info: #2.}%
        }%
      \else
        \def\x#1#2{\PackageInfo{#1}{#2, stopped}}%
      \fi
      \x{hologo}{The package is already loaded}%
      \aftergroup\endinput
    \fi
  \fi
\endgroup%
%    \end{macrocode}
%    Package identification:
%    \begin{macrocode}
\begingroup\catcode61\catcode48\catcode32=10\relax%
  \catcode13=5 % ^^M
  \endlinechar=13 %
  \catcode35=6 % #
  \catcode39=12 % '
  \catcode40=12 % (
  \catcode41=12 % )
  \catcode44=12 % ,
  \catcode45=12 % -
  \catcode46=12 % .
  \catcode47=12 % /
  \catcode58=12 % :
  \catcode64=11 % @
  \catcode91=12 % [
  \catcode93=12 % ]
  \catcode123=1 % {
  \catcode125=2 % }
  \expandafter\ifx\csname ProvidesPackage\endcsname\relax
    \def\x#1#2#3[#4]{\endgroup
      \immediate\write-1{Package: #3 #4}%
      \xdef#1{#4}%
    }%
  \else
    \def\x#1#2[#3]{\endgroup
      #2[{#3}]%
      \ifx#1\@undefined
        \xdef#1{#3}%
      \fi
      \ifx#1\relax
        \xdef#1{#3}%
      \fi
    }%
  \fi
\expandafter\x\csname ver@hologo.sty\endcsname
\ProvidesPackage{hologo}%
  [2016/05/12 v1.11 A logo collection with bookmark support (HO)]%
%    \end{macrocode}
%
%    \begin{macrocode}
\begingroup\catcode61\catcode48\catcode32=10\relax%
  \catcode13=5 % ^^M
  \endlinechar=13 %
  \catcode123=1 % {
  \catcode125=2 % }
  \catcode64=11 % @
  \def\x{\endgroup
    \expandafter\edef\csname HOLOGO@AtEnd\endcsname{%
      \endlinechar=\the\endlinechar\relax
      \catcode13=\the\catcode13\relax
      \catcode32=\the\catcode32\relax
      \catcode35=\the\catcode35\relax
      \catcode61=\the\catcode61\relax
      \catcode64=\the\catcode64\relax
      \catcode123=\the\catcode123\relax
      \catcode125=\the\catcode125\relax
    }%
  }%
\x\catcode61\catcode48\catcode32=10\relax%
\catcode13=5 % ^^M
\endlinechar=13 %
\catcode35=6 % #
\catcode64=11 % @
\catcode123=1 % {
\catcode125=2 % }
\def\TMP@EnsureCode#1#2{%
  \edef\HOLOGO@AtEnd{%
    \HOLOGO@AtEnd
    \catcode#1=\the\catcode#1\relax
  }%
  \catcode#1=#2\relax
}
\TMP@EnsureCode{10}{12}% ^^J
\TMP@EnsureCode{33}{12}% !
\TMP@EnsureCode{34}{12}% "
\TMP@EnsureCode{36}{3}% $
\TMP@EnsureCode{38}{4}% &
\TMP@EnsureCode{39}{12}% '
\TMP@EnsureCode{40}{12}% (
\TMP@EnsureCode{41}{12}% )
\TMP@EnsureCode{42}{12}% *
\TMP@EnsureCode{43}{12}% +
\TMP@EnsureCode{44}{12}% ,
\TMP@EnsureCode{45}{12}% -
\TMP@EnsureCode{46}{12}% .
\TMP@EnsureCode{47}{12}% /
\TMP@EnsureCode{58}{12}% :
\TMP@EnsureCode{59}{12}% ;
\TMP@EnsureCode{60}{12}% <
\TMP@EnsureCode{62}{12}% >
\TMP@EnsureCode{63}{12}% ?
\TMP@EnsureCode{91}{12}% [
\TMP@EnsureCode{93}{12}% ]
\TMP@EnsureCode{94}{7}% ^ (superscript)
\TMP@EnsureCode{95}{8}% _ (subscript)
\TMP@EnsureCode{96}{12}% `
\TMP@EnsureCode{124}{12}% |
\edef\HOLOGO@AtEnd{%
  \HOLOGO@AtEnd
  \escapechar\the\escapechar\relax
  \noexpand\endinput
}
\escapechar=92 %
%    \end{macrocode}
%
% \subsection{Logo list}
%
%    \begin{macro}{\hologoList}
%    \begin{macrocode}
\def\hologoList{%
  \hologoEntry{(La)TeX}{}{2011/10/01}%
  \hologoEntry{AmSLaTeX}{}{2010/04/16}%
  \hologoEntry{AmSTeX}{}{2010/04/16}%
  \hologoEntry{biber}{}{2011/10/01}%
  \hologoEntry{BibTeX}{}{2011/10/01}%
  \hologoEntry{BibTeX}{sf}{2011/10/01}%
  \hologoEntry{BibTeX}{sc}{2011/10/01}%
  \hologoEntry{BibTeX8}{}{2011/11/22}%
  \hologoEntry{ConTeXt}{}{2011/03/25}%
  \hologoEntry{ConTeXt}{narrow}{2011/03/25}%
  \hologoEntry{ConTeXt}{simple}{2011/03/25}%
  \hologoEntry{emTeX}{}{2010/04/26}%
  \hologoEntry{eTeX}{}{2010/04/08}%
  \hologoEntry{ExTeX}{}{2011/10/01}%
  \hologoEntry{HanTheThanh}{}{2011/11/29}%
  \hologoEntry{iniTeX}{}{2011/10/01}%
  \hologoEntry{KOMAScript}{}{2011/10/01}%
  \hologoEntry{La}{}{2010/05/08}%
  \hologoEntry{LaTeX}{}{2010/04/08}%
  \hologoEntry{LaTeX2e}{}{2010/04/08}%
  \hologoEntry{LaTeX3}{}{2010/04/24}%
  \hologoEntry{LaTeXe}{}{2010/04/08}%
  \hologoEntry{LaTeXML}{}{2011/11/22}%
  \hologoEntry{LaTeXTeX}{}{2011/10/01}%
  \hologoEntry{LuaLaTeX}{}{2010/04/08}%
  \hologoEntry{LuaTeX}{}{2010/04/08}%
  \hologoEntry{LyX}{}{2011/10/01}%
  \hologoEntry{METAFONT}{}{2011/10/01}%
  \hologoEntry{MetaFun}{}{2011/10/01}%
  \hologoEntry{METAPOST}{}{2011/10/01}%
  \hologoEntry{MetaPost}{}{2011/10/01}%
  \hologoEntry{MiKTeX}{}{2011/10/01}%
  \hologoEntry{NTS}{}{2011/10/01}%
  \hologoEntry{OzMF}{}{2011/10/01}%
  \hologoEntry{OzMP}{}{2011/10/01}%
  \hologoEntry{OzTeX}{}{2011/10/01}%
  \hologoEntry{OzTtH}{}{2011/10/01}%
  \hologoEntry{PCTeX}{}{2011/10/01}%
  \hologoEntry{pdfTeX}{}{2011/10/01}%
  \hologoEntry{pdfLaTeX}{}{2011/10/01}%
  \hologoEntry{PiC}{}{2011/10/01}%
  \hologoEntry{PiCTeX}{}{2011/10/01}%
  \hologoEntry{plainTeX}{}{2010/04/08}%
  \hologoEntry{plainTeX}{space}{2010/04/16}%
  \hologoEntry{plainTeX}{hyphen}{2010/04/16}%
  \hologoEntry{plainTeX}{runtogether}{2010/04/16}%
  \hologoEntry{SageTeX}{}{2011/11/22}%
  \hologoEntry{SLiTeX}{}{2011/10/01}%
  \hologoEntry{SLiTeX}{lift}{2011/10/01}%
  \hologoEntry{SLiTeX}{narrow}{2011/10/01}%
  \hologoEntry{SLiTeX}{simple}{2011/10/01}%
  \hologoEntry{SliTeX}{}{2011/10/01}%
  \hologoEntry{SliTeX}{narrow}{2011/10/01}%
  \hologoEntry{SliTeX}{simple}{2011/10/01}%
  \hologoEntry{SliTeX}{lift}{2011/10/01}%
  \hologoEntry{teTeX}{}{2011/10/01}%
  \hologoEntry{TeX}{}{2010/04/08}%
  \hologoEntry{TeX4ht}{}{2011/11/22}%
  \hologoEntry{TTH}{}{2011/11/22}%
  \hologoEntry{virTeX}{}{2011/10/01}%
  \hologoEntry{VTeX}{}{2010/04/24}%
  \hologoEntry{Xe}{}{2010/04/08}%
  \hologoEntry{XeLaTeX}{}{2010/04/08}%
  \hologoEntry{XeTeX}{}{2010/04/08}%
}
%    \end{macrocode}
%    \end{macro}
%
% \subsection{Load resources}
%
%    \begin{macrocode}
\begingroup\expandafter\expandafter\expandafter\endgroup
\expandafter\ifx\csname RequirePackage\endcsname\relax
  \def\TMP@RequirePackage#1[#2]{%
    \begingroup\expandafter\expandafter\expandafter\endgroup
    \expandafter\ifx\csname ver@#1.sty\endcsname\relax
      \input #1.sty\relax
    \fi
  }%
  \TMP@RequirePackage{ltxcmds}[2011/02/04]%
  \TMP@RequirePackage{infwarerr}[2010/04/08]%
  \TMP@RequirePackage{kvsetkeys}[2010/03/01]%
  \TMP@RequirePackage{kvdefinekeys}[2010/03/01]%
  \TMP@RequirePackage{pdftexcmds}[2010/04/01]%
  \TMP@RequirePackage{ifpdf}[2010/01/28]%
  \TMP@RequirePackage{ifluatex}[2010/03/01]%
  \ltx@IfUndefined{newif}{%
    \expandafter\let\csname newif\endcsname\ltx@newif
  }{}%
  \TMP@RequirePackage{ifxetex}[2009/01/23]%
  \TMP@RequirePackage{ifvtex}[2010/03/01]%
\else
  \RequirePackage{ltxcmds}[2011/02/04]%
  \RequirePackage{infwarerr}[2010/04/08]%
  \RequirePackage{kvsetkeys}[2010/03/01]%
  \RequirePackage{kvdefinekeys}[2010/03/01]%
  \RequirePackage{pdftexcmds}[2010/04/01]%
  \RequirePackage{ifpdf}[2010/01/28]%
  \RequirePackage{ifluatex}[2010/03/01]%
  \RequirePackage{ifxetex}[2009/01/23]%
  \RequirePackage{ifvtex}[2010/03/01]%
\fi
%    \end{macrocode}
%
%    \begin{macro}{\HOLOGO@IfDefined}
%    \begin{macrocode}
\def\HOLOGO@IfExists#1{%
  \ifx\@undefined#1%
    \expandafter\ltx@secondoftwo
  \else
    \ifx\relax#1%
      \expandafter\ltx@secondoftwo
    \else
      \expandafter\expandafter\expandafter\ltx@firstoftwo
    \fi
  \fi
}
%    \end{macrocode}
%    \end{macro}
%
% \subsection{Setup macros}
%
%    \begin{macro}{\hologoSetup}
%    \begin{macrocode}
\def\hologoSetup{%
  \let\HOLOGO@name\relax
  \HOLOGO@Setup
}
%    \end{macrocode}
%    \end{macro}
%
%    \begin{macro}{\hologoLogoSetup}
%    \begin{macrocode}
\def\hologoLogoSetup#1{%
  \edef\HOLOGO@name{#1}%
  \ltx@IfUndefined{HoLogo@\HOLOGO@name}{%
    \@PackageError{hologo}{%
      Unknown logo `\HOLOGO@name'%
    }\@ehc
    \ltx@gobble
  }{%
    \HOLOGO@Setup
  }%
}
%    \end{macrocode}
%    \end{macro}
%
%    \begin{macro}{\HOLOGO@Setup}
%    \begin{macrocode}
\def\HOLOGO@Setup{%
  \kvsetkeys{HoLogo}%
}
%    \end{macrocode}
%    \end{macro}
%
% \subsection{Options}
%
%    \begin{macro}{\HOLOGO@DeclareBoolOption}
%    \begin{macrocode}
\def\HOLOGO@DeclareBoolOption#1{%
  \expandafter\chardef\csname HOLOGOOPT@#1\endcsname\ltx@zero
  \kv@define@key{HoLogo}{#1}[true]{%
    \def\HOLOGO@temp{##1}%
    \ifx\HOLOGO@temp\HOLOGO@true
      \ifx\HOLOGO@name\relax
        \expandafter\chardef\csname HOLOGOOPT@#1\endcsname=\ltx@one
      \else
        \expandafter\chardef\csname
        HoLogoOpt@#1@\HOLOGO@name\endcsname\ltx@one
      \fi
      \HOLOGO@SetBreakAll{#1}%
    \else
      \ifx\HOLOGO@temp\HOLOGO@false
        \ifx\HOLOGO@name\relax
          \expandafter\chardef\csname HOLOGOOPT@#1\endcsname=\ltx@zero
        \else
          \expandafter\chardef\csname
          HoLogoOpt@#1@\HOLOGO@name\endcsname=\ltx@zero
        \fi
        \HOLOGO@SetBreakAll{#1}%
      \else
        \@PackageError{hologo}{%
          Unknown value `##1' for boolean option `#1'.\MessageBreak
          Known values are `true' and `false'%
        }\@ehc
      \fi
    \fi
  }%
}
%    \end{macrocode}
%    \end{macro}
%
%    \begin{macro}{\HOLOGO@SetBreakAll}
%    \begin{macrocode}
\def\HOLOGO@SetBreakAll#1{%
  \def\HOLOGO@temp{#1}%
  \ifx\HOLOGO@temp\HOLOGO@break
    \ifx\HOLOGO@name\relax
      \chardef\HOLOGOOPT@hyphenbreak=\HOLOGOOPT@break
      \chardef\HOLOGOOPT@spacebreak=\HOLOGOOPT@break
      \chardef\HOLOGOOPT@discretionarybreak=\HOLOGOOPT@break
    \else
      \expandafter\chardef
         \csname HoLogoOpt@hyphenbreak@\HOLOGO@name\endcsname=%
         \csname HoLogoOpt@break@\HOLOGO@name\endcsname
      \expandafter\chardef
         \csname HoLogoOpt@spacebreak@\HOLOGO@name\endcsname=%
         \csname HoLogoOpt@break@\HOLOGO@name\endcsname
      \expandafter\chardef
         \csname HoLogoOpt@discretionarybreak@\HOLOGO@name
             \endcsname=%
         \csname HoLogoOpt@break@\HOLOGO@name\endcsname
    \fi
  \fi
}
%    \end{macrocode}
%    \end{macro}
%
%    \begin{macro}{\HOLOGO@true}
%    \begin{macrocode}
\def\HOLOGO@true{true}
%    \end{macrocode}
%    \end{macro}
%    \begin{macro}{\HOLOGO@false}
%    \begin{macrocode}
\def\HOLOGO@false{false}
%    \end{macrocode}
%    \end{macro}
%    \begin{macro}{\HOLOGO@break}
%    \begin{macrocode}
\def\HOLOGO@break{break}
%    \end{macrocode}
%    \end{macro}
%
%    \begin{macrocode}
\HOLOGO@DeclareBoolOption{break}
\HOLOGO@DeclareBoolOption{hyphenbreak}
\HOLOGO@DeclareBoolOption{spacebreak}
\HOLOGO@DeclareBoolOption{discretionarybreak}
%    \end{macrocode}
%
%    \begin{macrocode}
\kv@define@key{HoLogo}{variant}{%
  \ifx\HOLOGO@name\relax
    \@PackageError{hologo}{%
      Option `variant' is not available in \string\hologoSetup,%
      \MessageBreak
      Use \string\hologoLogoSetup\space instead%
    }\@ehc
  \else
    \edef\HOLOGO@temp{#1}%
    \ifx\HOLOGO@temp\ltx@empty
      \expandafter
      \let\csname HoLogoOpt@variant@\HOLOGO@name\endcsname\@undefined
    \else
      \ltx@IfUndefined{HoLogo@\HOLOGO@name @\HOLOGO@temp}{%
        \@PackageError{hologo}{%
          Unknown variant `\HOLOGO@temp' of logo `\HOLOGO@name'%
        }\@ehc
      }{%
        \expandafter
        \let\csname HoLogoOpt@variant@\HOLOGO@name\endcsname
            \HOLOGO@temp
      }%
    \fi
  \fi
}
%    \end{macrocode}
%
%    \begin{macro}{\HOLOGO@Variant}
%    \begin{macrocode}
\def\HOLOGO@Variant#1{%
  #1%
  \ltx@ifundefined{HoLogoOpt@variant@#1}{%
  }{%
    @\csname HoLogoOpt@variant@#1\endcsname
  }%
}
%    \end{macrocode}
%    \end{macro}
%
% \subsection{Break/no-break support}
%
%    \begin{macro}{\HOLOGO@space}
%    \begin{macrocode}
\def\HOLOGO@space{%
  \ltx@ifundefined{HoLogoOpt@spacebreak@\HOLOGO@name}{%
    \ltx@ifundefined{HoLogoOpt@break@\HOLOGO@name}{%
      \chardef\HOLOGO@temp=\HOLOGOOPT@spacebreak
    }{%
      \chardef\HOLOGO@temp=%
        \csname HoLogoOpt@break@\HOLOGO@name\endcsname
    }%
  }{%
    \chardef\HOLOGO@temp=%
      \csname HoLogoOpt@spacebreak@\HOLOGO@name\endcsname
  }%
  \ifcase\HOLOGO@temp
    \penalty10000 %
  \fi
  \ltx@space
}
%    \end{macrocode}
%    \end{macro}
%
%    \begin{macro}{\HOLOGO@hyphen}
%    \begin{macrocode}
\def\HOLOGO@hyphen{%
  \ltx@ifundefined{HoLogoOpt@hyphenbreak@\HOLOGO@name}{%
    \ltx@ifundefined{HoLogoOpt@break@\HOLOGO@name}{%
      \chardef\HOLOGO@temp=\HOLOGOOPT@hyphenbreak
    }{%
      \chardef\HOLOGO@temp=%
        \csname HoLogoOpt@break@\HOLOGO@name\endcsname
    }%
  }{%
    \chardef\HOLOGO@temp=%
      \csname HoLogoOpt@hyphenbreak@\HOLOGO@name\endcsname
  }%
  \ifcase\HOLOGO@temp
    \ltx@mbox{-}%
  \else
    -%
  \fi
}
%    \end{macrocode}
%    \end{macro}
%
%    \begin{macro}{\HOLOGO@discretionary}
%    \begin{macrocode}
\def\HOLOGO@discretionary{%
  \ltx@ifundefined{HoLogoOpt@discretionarybreak@\HOLOGO@name}{%
    \ltx@ifundefined{HoLogoOpt@break@\HOLOGO@name}{%
      \chardef\HOLOGO@temp=\HOLOGOOPT@discretionarybreak
    }{%
      \chardef\HOLOGO@temp=%
        \csname HoLogoOpt@break@\HOLOGO@name\endcsname
    }%
  }{%
    \chardef\HOLOGO@temp=%
      \csname HoLogoOpt@discretionarybreak@\HOLOGO@name\endcsname
  }%
  \ifcase\HOLOGO@temp
  \else
    \-%
  \fi
}
%    \end{macrocode}
%    \end{macro}
%
%    \begin{macro}{\HOLOGO@mbox}
%    \begin{macrocode}
\def\HOLOGO@mbox#1{%
  \ltx@ifundefined{HoLogoOpt@break@\HOLOGO@name}{%
    \chardef\HOLOGO@temp=\HOLOGOOPT@hyphenbreak
  }{%
    \chardef\HOLOGO@temp=%
      \csname HoLogoOpt@break@\HOLOGO@name\endcsname
  }%
  \ifcase\HOLOGO@temp
    \ltx@mbox{#1}%
  \else
    #1%
  \fi
}
%    \end{macrocode}
%    \end{macro}
%
% \subsection{Font support}
%
%    \begin{macro}{\HoLogoFont@font}
%    \begin{tabular}{@{}ll@{}}
%    |#1|:& logo name\\
%    |#2|:& font short name\\
%    |#3|:& text
%    \end{tabular}
%    \begin{macrocode}
\def\HoLogoFont@font#1#2#3{%
  \begingroup
    \ltx@IfUndefined{HoLogoFont@logo@#1.#2}{%
      \ltx@IfUndefined{HoLogoFont@font@#2}{%
        \@PackageWarning{hologo}{%
          Missing font `#2' for logo `#1'%
        }%
        #3%
      }{%
        \csname HoLogoFont@font@#2\endcsname{#3}%
      }%
    }{%
      \csname HoLogoFont@logo@#1.#2\endcsname{#3}%
    }%
  \endgroup
}
%    \end{macrocode}
%    \end{macro}
%
%    \begin{macro}{\HoLogoFont@Def}
%    \begin{macrocode}
\def\HoLogoFont@Def#1{%
  \expandafter\def\csname HoLogoFont@font@#1\endcsname
}
%    \end{macrocode}
%    \end{macro}
%    \begin{macro}{\HoLogoFont@LogoDef}
%    \begin{macrocode}
\def\HoLogoFont@LogoDef#1#2{%
  \expandafter\def\csname HoLogoFont@logo@#1.#2\endcsname
}
%    \end{macrocode}
%    \end{macro}
%
% \subsubsection{Font defaults}
%
%    \begin{macro}{\HoLogoFont@font@general}
%    \begin{macrocode}
\HoLogoFont@Def{general}{}%
%    \end{macrocode}
%    \end{macro}
%
%    \begin{macro}{\HoLogoFont@font@rm}
%    \begin{macrocode}
\ltx@IfUndefined{rmfamily}{%
  \ltx@IfUndefined{rm}{%
  }{%
    \HoLogoFont@Def{rm}{\rm}%
  }%
}{%
  \HoLogoFont@Def{rm}{\rmfamily}%
}
%    \end{macrocode}
%    \end{macro}
%
%    \begin{macro}{\HoLogoFont@font@sf}
%    \begin{macrocode}
\ltx@IfUndefined{sffamily}{%
  \ltx@IfUndefined{sf}{%
  }{%
    \HoLogoFont@Def{sf}{\sf}%
  }%
}{%
  \HoLogoFont@Def{sf}{\sffamily}%
}
%    \end{macrocode}
%    \end{macro}
%
%    \begin{macro}{\HoLogoFont@font@bibsf}
%    In case of \hologo{plainTeX} the original small caps
%    variant is used as default. In \hologo{LaTeX}
%    the definition of package \xpackage{dtklogos} \cite{dtklogos}
%    is used.
%\begin{quote}
%\begin{verbatim}
%\DeclareRobustCommand{\BibTeX}{%
%  B%
%  \kern-.05em%
%  \hbox{%
%    $\m@th$% %% force math size calculations
%    \csname S@\f@size\endcsname
%    \fontsize\sf@size\z@
%    \math@fontsfalse
%    \selectfont
%    I%
%    \kern-.025em%
%    B
%  }%
%  \kern-.08em%
%  \-%
%  \TeX
%}
%\end{verbatim}
%\end{quote}
%    \begin{macrocode}
\ltx@IfUndefined{selectfont}{%
  \ltx@IfUndefined{tensc}{%
    \font\tensc=cmcsc10\relax
  }{}%
  \HoLogoFont@Def{bibsf}{\tensc}%
}{%
  \HoLogoFont@Def{bibsf}{%
    $\mathsurround=0pt$%
    \csname S@\f@size\endcsname
    \fontsize\sf@size{0pt}%
    \math@fontsfalse
    \selectfont
  }%
}
%    \end{macrocode}
%    \end{macro}
%
%    \begin{macro}{\HoLogoFont@font@sc}
%    \begin{macrocode}
\ltx@IfUndefined{scshape}{%
  \ltx@IfUndefined{tensc}{%
    \font\tensc=cmcsc10\relax
  }{}%
  \HoLogoFont@Def{sc}{\tensc}%
}{%
  \HoLogoFont@Def{sc}{\scshape}%
}
%    \end{macrocode}
%    \end{macro}
%
%    \begin{macro}{\HoLogoFont@font@sy}
%    \begin{macrocode}
\ltx@IfUndefined{usefont}{%
  \ltx@IfUndefined{tensy}{%
  }{%
    \HoLogoFont@Def{sy}{\tensy}%
  }%
}{%
  \HoLogoFont@Def{sy}{%
    \usefont{OMS}{cmsy}{m}{n}%
  }%
}
%    \end{macrocode}
%    \end{macro}
%
%    \begin{macro}{\HoLogoFont@font@logo}
%    \begin{macrocode}
\begingroup
  \def\x{LaTeX2e}%
\expandafter\endgroup
\ifx\fmtname\x
  \ltx@IfUndefined{logofamily}{%
    \DeclareRobustCommand\logofamily{%
      \not@math@alphabet\logofamily\relax
      \fontencoding{U}%
      \fontfamily{logo}%
      \selectfont
    }%
  }{}%
  \ltx@IfUndefined{logofamily}{%
  }{%
    \HoLogoFont@Def{logo}{\logofamily}%
  }%
\else
  \ltx@IfUndefined{tenlogo}{%
    \font\tenlogo=logo10\relax
  }{}%
  \HoLogoFont@Def{logo}{\tenlogo}%
\fi
%    \end{macrocode}
%    \end{macro}
%
% \subsubsection{Font setup}
%
%    \begin{macro}{\hologoFontSetup}
%    \begin{macrocode}
\def\hologoFontSetup{%
  \let\HOLOGO@name\relax
  \HOLOGO@FontSetup
}
%    \end{macrocode}
%    \end{macro}
%
%    \begin{macro}{\hologoLogoFontSetup}
%    \begin{macrocode}
\def\hologoLogoFontSetup#1{%
  \edef\HOLOGO@name{#1}%
  \ltx@IfUndefined{HoLogo@\HOLOGO@name}{%
    \@PackageError{hologo}{%
      Unknown logo `\HOLOGO@name'%
    }\@ehc
    \ltx@gobble
  }{%
    \HOLOGO@FontSetup
  }%
}
%    \end{macrocode}
%    \end{macro}
%
%    \begin{macro}{\HOLOGO@FontSetup}
%    \begin{macrocode}
\def\HOLOGO@FontSetup{%
  \kvsetkeys{HoLogoFont}%
}
%    \end{macrocode}
%    \end{macro}
%
%    \begin{macrocode}
\def\HOLOGO@temp#1{%
  \kv@define@key{HoLogoFont}{#1}{%
    \ifx\HOLOGO@name\relax
      \HoLogoFont@Def{#1}{##1}%
    \else
      \HoLogoFont@LogoDef\HOLOGO@name{#1}{##1}%
    \fi
  }%
}
\HOLOGO@temp{general}
\HOLOGO@temp{sf}
%    \end{macrocode}
%
% \subsection{Generic logo commands}
%
%    \begin{macrocode}
\HOLOGO@IfExists\hologo{%
  \@PackageError{hologo}{%
    \string\hologo\ltx@space is already defined.\MessageBreak
    Package loading is aborted%
  }\@ehc
  \HOLOGO@AtEnd
}%
\HOLOGO@IfExists\hologoRobust{%
  \@PackageError{hologo}{%
    \string\hologoRobust\ltx@space is already defined.\MessageBreak
    Package loading is aborted%
  }\@ehc
  \HOLOGO@AtEnd
}%
%    \end{macrocode}
%
% \subsubsection{\cs{hologo} and friends}
%
%    \begin{macrocode}
\ifluatex
  \expandafter\ltx@firstofone
\else
  \expandafter\ltx@gobble
\fi
{%
  \ltx@IfUndefined{ifincsname}{%
    \ifnum\luatexversion<36 %
      \expandafter\ltx@gobble
    \else
      \expandafter\ltx@firstofone
    \fi
    {%
      \begingroup
        \ifcase0%
            \directlua{%
              if tex.enableprimitives then %
                tex.enableprimitives('HOLOGO@', {'ifincsname'})%
              else %
                tex.print('1')%
              end%
            }%
            \ifx\HOLOGO@ifincsname\@undefined 1\fi%
            \relax
          \expandafter\ltx@firstofone
        \else
          \endgroup
          \expandafter\ltx@gobble
        \fi
        {%
          \global\let\ifincsname\HOLOGO@ifincsname
        }%
      \HOLOGO@temp
    }%
  }{}%
}
%    \end{macrocode}
%    \begin{macrocode}
\ltx@IfUndefined{ifincsname}{%
  \catcode`$=14 %
}{%
  \catcode`$=9 %
}
%    \end{macrocode}
%
%    \begin{macro}{\hologo}
%    \begin{macrocode}
\def\hologo#1{%
$ \ifincsname
$   \ltx@ifundefined{HoLogoCs@\HOLOGO@Variant{#1}}{%
$     #1%
$   }{%
$     \csname HoLogoCs@\HOLOGO@Variant{#1}\endcsname\ltx@firstoftwo
$   }%
$ \else
    \HOLOGO@IfExists\texorpdfstring\texorpdfstring\ltx@firstoftwo
    {%
      \hologoRobust{#1}%
    }{%
      \ltx@ifundefined{HoLogoBkm@\HOLOGO@Variant{#1}}{%
        \ltx@ifundefined{HoLogo@#1}{?#1?}{#1}%
      }{%
        \csname HoLogoBkm@\HOLOGO@Variant{#1}\endcsname
        \ltx@firstoftwo
      }%
    }%
$ \fi
}
%    \end{macrocode}
%    \end{macro}
%    \begin{macro}{\Hologo}
%    \begin{macrocode}
\def\Hologo#1{%
$ \ifincsname
$   \ltx@ifundefined{HoLogoCs@\HOLOGO@Variant{#1}}{%
$     #1%
$   }{%
$     \csname HoLogoCs@\HOLOGO@Variant{#1}\endcsname\ltx@secondoftwo
$   }%
$ \else
    \HOLOGO@IfExists\texorpdfstring\texorpdfstring\ltx@firstoftwo
    {%
      \HologoRobust{#1}%
    }{%
      \ltx@ifundefined{HoLogoBkm@\HOLOGO@Variant{#1}}{%
        \ltx@ifundefined{HoLogo@#1}{?#1?}{#1}%
      }{%
        \csname HoLogoBkm@\HOLOGO@Variant{#1}\endcsname
        \ltx@secondoftwo
      }%
    }%
$ \fi
}
%    \end{macrocode}
%    \end{macro}
%
%    \begin{macro}{\hologoVariant}
%    \begin{macrocode}
\def\hologoVariant#1#2{%
  \ifx\relax#2\relax
    \hologo{#1}%
  \else
$   \ifincsname
$     \ltx@ifundefined{HoLogoCs@#1@#2}{%
$       #1%
$     }{%
$       \csname HoLogoCs@#1@#2\endcsname\ltx@firstoftwo
$     }%
$   \else
      \HOLOGO@IfExists\texorpdfstring\texorpdfstring\ltx@firstoftwo
      {%
        \hologoVariantRobust{#1}{#2}%
      }{%
        \ltx@ifundefined{HoLogoBkm@#1@#2}{%
          \ltx@ifundefined{HoLogo@#1}{?#1?}{#1}%
        }{%
          \csname HoLogoBkm@#1@#2\endcsname
          \ltx@firstoftwo
        }%
      }%
$   \fi
  \fi
}
%    \end{macrocode}
%    \end{macro}
%    \begin{macro}{\HologoVariant}
%    \begin{macrocode}
\def\HologoVariant#1#2{%
  \ifx\relax#2\relax
    \Hologo{#1}%
  \else
$   \ifincsname
$     \ltx@ifundefined{HoLogoCs@#1@#2}{%
$       #1%
$     }{%
$       \csname HoLogoCs@#1@#2\endcsname\ltx@secondoftwo
$     }%
$   \else
      \HOLOGO@IfExists\texorpdfstring\texorpdfstring\ltx@firstoftwo
      {%
        \HologoVariantRobust{#1}{#2}%
      }{%
        \ltx@ifundefined{HoLogoBkm@#1@#2}{%
          \ltx@ifundefined{HoLogo@#1}{?#1?}{#1}%
        }{%
          \csname HoLogoBkm@#1@#2\endcsname
          \ltx@secondoftwo
        }%
      }%
$   \fi
  \fi
}
%    \end{macrocode}
%    \end{macro}
%
%    \begin{macrocode}
\catcode`\$=3 %
%    \end{macrocode}
%
% \subsubsection{\cs{hologoRobust} and friends}
%
%    \begin{macro}{\hologoRobust}
%    \begin{macrocode}
\ltx@IfUndefined{protected}{%
  \ltx@IfUndefined{DeclareRobustCommand}{%
    \def\hologoRobust#1%
  }{%
    \DeclareRobustCommand*\hologoRobust[1]%
  }%
}{%
  \protected\def\hologoRobust#1%
}%
{%
  \edef\HOLOGO@name{#1}%
  \ltx@IfUndefined{HoLogo@\HOLOGO@Variant\HOLOGO@name}{%
    \@PackageError{hologo}{%
      Unknown logo `\HOLOGO@name'%
    }\@ehc
    ?\HOLOGO@name?%
  }{%
    \ltx@IfUndefined{ver@tex4ht.sty}{%
      \HoLogoFont@font\HOLOGO@name{general}{%
        \csname HoLogo@\HOLOGO@Variant\HOLOGO@name\endcsname
        \ltx@firstoftwo
      }%
    }{%
      \ltx@IfUndefined{HoLogoHtml@\HOLOGO@Variant\HOLOGO@name}{%
        \HOLOGO@name
      }{%
        \csname HoLogoHtml@\HOLOGO@Variant\HOLOGO@name\endcsname
        \ltx@firstoftwo
      }%
    }%
  }%
}
%    \end{macrocode}
%    \end{macro}
%    \begin{macro}{\HologoRobust}
%    \begin{macrocode}
\ltx@IfUndefined{protected}{%
  \ltx@IfUndefined{DeclareRobustCommand}{%
    \def\HologoRobust#1%
  }{%
    \DeclareRobustCommand*\HologoRobust[1]%
  }%
}{%
  \protected\def\HologoRobust#1%
}%
{%
  \edef\HOLOGO@name{#1}%
  \ltx@IfUndefined{HoLogo@\HOLOGO@Variant\HOLOGO@name}{%
    \@PackageError{hologo}{%
      Unknown logo `\HOLOGO@name'%
    }\@ehc
    ?\HOLOGO@name?%
  }{%
    \ltx@IfUndefined{ver@tex4ht.sty}{%
      \HoLogoFont@font\HOLOGO@name{general}{%
        \csname HoLogo@\HOLOGO@Variant\HOLOGO@name\endcsname
        \ltx@secondoftwo
      }%
    }{%
      \ltx@IfUndefined{HoLogoHtml@\HOLOGO@Variant\HOLOGO@name}{%
        \expandafter\HOLOGO@Uppercase\HOLOGO@name
      }{%
        \csname HoLogoHtml@\HOLOGO@Variant\HOLOGO@name\endcsname
        \ltx@secondoftwo
      }%
    }%
  }%
}
%    \end{macrocode}
%    \end{macro}
%    \begin{macro}{\hologoVariantRobust}
%    \begin{macrocode}
\ltx@IfUndefined{protected}{%
  \ltx@IfUndefined{DeclareRobustCommand}{%
    \def\hologoVariantRobust#1#2%
  }{%
    \DeclareRobustCommand*\hologoVariantRobust[2]%
  }%
}{%
  \protected\def\hologoVariantRobust#1#2%
}%
{%
  \begingroup
    \hologoLogoSetup{#1}{variant={#2}}%
    \hologoRobust{#1}%
  \endgroup
}
%    \end{macrocode}
%    \end{macro}
%    \begin{macro}{\HologoVariantRobust}
%    \begin{macrocode}
\ltx@IfUndefined{protected}{%
  \ltx@IfUndefined{DeclareRobustCommand}{%
    \def\HologoVariantRobust#1#2%
  }{%
    \DeclareRobustCommand*\HologoVariantRobust[2]%
  }%
}{%
  \protected\def\HologoVariantRobust#1#2%
}%
{%
  \begingroup
    \hologoLogoSetup{#1}{variant={#2}}%
    \HologoRobust{#1}%
  \endgroup
}
%    \end{macrocode}
%    \end{macro}
%
%    \begin{macro}{\hologorobust}
%    Macro \cs{hologorobust} is only defined for compatibility.
%    Its use is deprecated.
%    \begin{macrocode}
\def\hologorobust{\hologoRobust}
%    \end{macrocode}
%    \end{macro}
%
% \subsection{Helpers}
%
%    \begin{macro}{\HOLOGO@Uppercase}
%    Macro \cs{HOLOGO@Uppercase} is restricted to \cs{uppercase},
%    because \hologo{plainTeX} or \hologo{iniTeX} do not provide
%    \cs{MakeUppercase}.
%    \begin{macrocode}
\def\HOLOGO@Uppercase#1{\uppercase{#1}}
%    \end{macrocode}
%    \end{macro}
%
%    \begin{macro}{\HOLOGO@PdfdocUnicode}
%    \begin{macrocode}
\def\HOLOGO@PdfdocUnicode{%
  \ifx\ifHy@unicode\iftrue
    \expandafter\ltx@secondoftwo
  \else
    \expandafter\ltx@firstoftwo
  \fi
}
%    \end{macrocode}
%    \end{macro}
%
%    \begin{macro}{\HOLOGO@Math}
%    \begin{macrocode}
\def\HOLOGO@MathSetup{%
  \mathsurround0pt\relax
  \HOLOGO@IfExists\f@series{%
    \if b\expandafter\ltx@car\f@series x\@nil
      \csname boldmath\endcsname
   \fi
  }{}%
}
%    \end{macrocode}
%    \end{macro}
%
%    \begin{macro}{\HOLOGO@TempDimen}
%    \begin{macrocode}
\dimendef\HOLOGO@TempDimen=\ltx@zero
%    \end{macrocode}
%    \end{macro}
%    \begin{macro}{\HOLOGO@NegativeKerning}
%    \begin{macrocode}
\def\HOLOGO@NegativeKerning#1{%
  \begingroup
    \HOLOGO@TempDimen=0pt\relax
    \comma@parse@normalized{#1}{%
      \ifdim\HOLOGO@TempDimen=0pt %
        \expandafter\HOLOGO@@NegativeKerning\comma@entry
      \fi
      \ltx@gobble
    }%
    \ifdim\HOLOGO@TempDimen<0pt %
      \kern\HOLOGO@TempDimen
    \fi
  \endgroup
}
%    \end{macrocode}
%    \end{macro}
%    \begin{macro}{\HOLOGO@@NegativeKerning}
%    \begin{macrocode}
\def\HOLOGO@@NegativeKerning#1#2{%
  \setbox\ltx@zero\hbox{#1#2}%
  \HOLOGO@TempDimen=\wd\ltx@zero
  \setbox\ltx@zero\hbox{#1\kern0pt#2}%
  \advance\HOLOGO@TempDimen by -\wd\ltx@zero
}
%    \end{macrocode}
%    \end{macro}
%
%    \begin{macro}{\HOLOGO@SpaceFactor}
%    \begin{macrocode}
\def\HOLOGO@SpaceFactor{%
  \spacefactor1000 %
}
%    \end{macrocode}
%    \end{macro}
%
%    \begin{macro}{\HOLOGO@Span}
%    \begin{macrocode}
\def\HOLOGO@Span#1#2{%
  \HCode{<span class="HoLogo-#1">}%
  #2%
  \HCode{</span>}%
}
%    \end{macrocode}
%    \end{macro}
%
% \subsubsection{Text subscript}
%
%    \begin{macro}{\HOLOGO@SubScript}%
%    \begin{macrocode}
\def\HOLOGO@SubScript#1{%
  \ltx@IfUndefined{textsubscript}{%
    \ltx@IfUndefined{text}{%
      \ltx@mbox{%
        \mathsurround=0pt\relax
        $%
          _{%
            \ltx@IfUndefined{sf@size}{%
              \mathrm{#1}%
            }{%
              \mbox{%
                \fontsize\sf@size{0pt}\selectfont
                #1%
              }%
            }%
          }%
        $%
      }%
    }{%
      \ltx@mbox{%
        \mathsurround=0pt\relax
        $_{\text{#1}}$%
      }%
    }%
  }{%
    \textsubscript{#1}%
  }%
}
%    \end{macrocode}
%    \end{macro}
%
% \subsection{\hologo{TeX} and friends}
%
% \subsubsection{\hologo{TeX}}
%
%    \begin{macro}{\HoLogo@TeX}
%    Source: \hologo{LaTeX} kernel.
%    \begin{macrocode}
\def\HoLogo@TeX#1{%
  T\kern-.1667em\lower.5ex\hbox{E}\kern-.125emX\HOLOGO@SpaceFactor
}
%    \end{macrocode}
%    \end{macro}
%    \begin{macro}{\HoLogoHtml@TeX}
%    \begin{macrocode}
\def\HoLogoHtml@TeX#1{%
  \HoLogoCss@TeX
  \HOLOGO@Span{TeX}{%
    T%
    \HOLOGO@Span{e}{%
      E%
    }%
    X%
  }%
}
%    \end{macrocode}
%    \end{macro}
%    \begin{macro}{\HoLogoCss@TeX}
%    \begin{macrocode}
\def\HoLogoCss@TeX{%
  \Css{%
    span.HoLogo-TeX span.HoLogo-e{%
      position:relative;%
      top:.5ex;%
      margin-left:-.1667em;%
      margin-right:-.125em;%
    }%
  }%
  \Css{%
    a span.HoLogo-TeX span.HoLogo-e{%
      text-decoration:none;%
    }%
  }%
  \global\let\HoLogoCss@TeX\relax
}
%    \end{macrocode}
%    \end{macro}
%
% \subsubsection{\hologo{plainTeX}}
%
%    \begin{macro}{\HoLogo@plainTeX@space}
%    Source: ``The \hologo{TeX}book''
%    \begin{macrocode}
\def\HoLogo@plainTeX@space#1{%
  \HOLOGO@mbox{#1{p}{P}lain}\HOLOGO@space\hologo{TeX}%
}
%    \end{macrocode}
%    \end{macro}
%    \begin{macro}{\HoLogoCs@plainTeX@space}
%    \begin{macrocode}
\def\HoLogoCs@plainTeX@space#1{#1{p}{P}lain TeX}%
%    \end{macrocode}
%    \end{macro}
%    \begin{macro}{\HoLogoBkm@plainTeX@space}
%    \begin{macrocode}
\def\HoLogoBkm@plainTeX@space#1{%
  #1{p}{P}lain \hologo{TeX}%
}
%    \end{macrocode}
%    \end{macro}
%    \begin{macro}{\HoLogoHtml@plainTeX@space}
%    \begin{macrocode}
\def\HoLogoHtml@plainTeX@space#1{%
  #1{p}{P}lain \hologo{TeX}%
}
%    \end{macrocode}
%    \end{macro}
%
%    \begin{macro}{\HoLogo@plainTeX@hyphen}
%    \begin{macrocode}
\def\HoLogo@plainTeX@hyphen#1{%
  \HOLOGO@mbox{#1{p}{P}lain}\HOLOGO@hyphen\hologo{TeX}%
}
%    \end{macrocode}
%    \end{macro}
%    \begin{macro}{\HoLogoCs@plainTeX@hyphen}
%    \begin{macrocode}
\def\HoLogoCs@plainTeX@hyphen#1{#1{p}{P}lain-TeX}
%    \end{macrocode}
%    \end{macro}
%    \begin{macro}{\HoLogoBkm@plainTeX@hyphen}
%    \begin{macrocode}
\def\HoLogoBkm@plainTeX@hyphen#1{%
  #1{p}{P}lain-\hologo{TeX}%
}
%    \end{macrocode}
%    \end{macro}
%    \begin{macro}{\HoLogoHtml@plainTeX@hyphen}
%    \begin{macrocode}
\def\HoLogoHtml@plainTeX@hyphen#1{%
  #1{p}{P}lain-\hologo{TeX}%
}
%    \end{macrocode}
%    \end{macro}
%
%    \begin{macro}{\HoLogo@plainTeX@runtogether}
%    \begin{macrocode}
\def\HoLogo@plainTeX@runtogether#1{%
  \HOLOGO@mbox{#1{p}{P}lain\hologo{TeX}}%
}
%    \end{macrocode}
%    \end{macro}
%    \begin{macro}{\HoLogoCs@plainTeX@runtogether}
%    \begin{macrocode}
\def\HoLogoCs@plainTeX@runtogether#1{#1{p}{P}lainTeX}
%    \end{macrocode}
%    \end{macro}
%    \begin{macro}{\HoLogoBkm@plainTeX@runtogether}
%    \begin{macrocode}
\def\HoLogoBkm@plainTeX@runtogether#1{%
  #1{p}{P}lain\hologo{TeX}%
}
%    \end{macrocode}
%    \end{macro}
%    \begin{macro}{\HoLogoHtml@plainTeX@runtogether}
%    \begin{macrocode}
\def\HoLogoHtml@plainTeX@runtogether#1{%
  #1{p}{P}lain\hologo{TeX}%
}
%    \end{macrocode}
%    \end{macro}
%
%    \begin{macro}{\HoLogo@plainTeX}
%    \begin{macrocode}
\def\HoLogo@plainTeX{\HoLogo@plainTeX@space}
%    \end{macrocode}
%    \end{macro}
%    \begin{macro}{\HoLogoCs@plainTeX}
%    \begin{macrocode}
\def\HoLogoCs@plainTeX{\HoLogoCs@plainTeX@space}
%    \end{macrocode}
%    \end{macro}
%    \begin{macro}{\HoLogoBkm@plainTeX}
%    \begin{macrocode}
\def\HoLogoBkm@plainTeX{\HoLogoBkm@plainTeX@space}
%    \end{macrocode}
%    \end{macro}
%    \begin{macro}{\HoLogoHtml@plainTeX}
%    \begin{macrocode}
\def\HoLogoHtml@plainTeX{\HoLogoHtml@plainTeX@space}
%    \end{macrocode}
%    \end{macro}
%
% \subsubsection{\hologo{LaTeX}}
%
%    Source: \hologo{LaTeX} kernel.
%\begin{quote}
%\begin{verbatim}
%\DeclareRobustCommand{\LaTeX}{%
%  L%
%  \kern-.36em%
%  {%
%    \sbox\z@ T%
%    \vbox to\ht\z@{%
%      \hbox{%
%        \check@mathfonts
%        \fontsize\sf@size\z@
%        \math@fontsfalse
%        \selectfont
%        A%
%      }%
%      \vss
%    }%
%  }%
%  \kern-.15em%
%  \TeX
%}
%\end{verbatim}
%\end{quote}
%
%    \begin{macro}{\HoLogo@La}
%    \begin{macrocode}
\def\HoLogo@La#1{%
  L%
  \kern-.36em%
  \begingroup
    \setbox\ltx@zero\hbox{T}%
    \vbox to\ht\ltx@zero{%
      \hbox{%
        \ltx@ifundefined{check@mathfonts}{%
          \csname sevenrm\endcsname
        }{%
          \check@mathfonts
          \fontsize\sf@size{0pt}%
          \math@fontsfalse\selectfont
        }%
        A%
      }%
      \vss
    }%
  \endgroup
}
%    \end{macrocode}
%    \end{macro}
%
%    \begin{macro}{\HoLogo@LaTeX}
%    Source: \hologo{LaTeX} kernel.
%    \begin{macrocode}
\def\HoLogo@LaTeX#1{%
  \hologo{La}%
  \kern-.15em%
  \hologo{TeX}%
}
%    \end{macrocode}
%    \end{macro}
%    \begin{macro}{\HoLogoHtml@LaTeX}
%    \begin{macrocode}
\def\HoLogoHtml@LaTeX#1{%
  \HoLogoCss@LaTeX
  \HOLOGO@Span{LaTeX}{%
    L%
    \HOLOGO@Span{a}{%
      A%
    }%
    \hologo{TeX}%
  }%
}
%    \end{macrocode}
%    \end{macro}
%    \begin{macro}{\HoLogoCss@LaTeX}
%    \begin{macrocode}
\def\HoLogoCss@LaTeX{%
  \Css{%
    span.HoLogo-LaTeX span.HoLogo-a{%
      position:relative;%
      top:-.5ex;%
      margin-left:-.36em;%
      margin-right:-.15em;%
      font-size:85\%;%
    }%
  }%
  \global\let\HoLogoCss@LaTeX\relax
}
%    \end{macrocode}
%    \end{macro}
%
% \subsubsection{\hologo{(La)TeX}}
%
%    \begin{macro}{\HoLogo@LaTeXTeX}
%    The kerning around the parentheses is taken
%    from package \xpackage{dtklogos} \cite{dtklogos}.
%\begin{quote}
%\begin{verbatim}
%\DeclareRobustCommand{\LaTeXTeX}{%
%  (%
%  \kern-.15em%
%  L%
%  \kern-.36em%
%  {%
%    \sbox\z@ T%
%    \vbox to\ht0{%
%      \hbox{%
%        $\m@th$%
%        \csname S@\f@size\endcsname
%        \fontsize\sf@size\z@
%        \math@fontsfalse
%        \selectfont
%        A%
%      }%
%      \vss
%    }%
%  }%
%  \kern-.2em%
%  )%
%  \kern-.15em%
%  \TeX
%}
%\end{verbatim}
%\end{quote}
%    \begin{macrocode}
\def\HoLogo@LaTeXTeX#1{%
  (%
  \kern-.15em%
  \hologo{La}%
  \kern-.2em%
  )%
  \kern-.15em%
  \hologo{TeX}%
}
%    \end{macrocode}
%    \end{macro}
%    \begin{macro}{\HoLogoBkm@LaTeXTeX}
%    \begin{macrocode}
\def\HoLogoBkm@LaTeXTeX#1{(La)TeX}
%    \end{macrocode}
%    \end{macro}
%
%    \begin{macro}{\HoLogo@(La)TeX}
%    \begin{macrocode}
\expandafter
\let\csname HoLogo@(La)TeX\endcsname\HoLogo@LaTeXTeX
%    \end{macrocode}
%    \end{macro}
%    \begin{macro}{\HoLogoBkm@(La)TeX}
%    \begin{macrocode}
\expandafter
\let\csname HoLogoBkm@(La)TeX\endcsname\HoLogoBkm@LaTeXTeX
%    \end{macrocode}
%    \end{macro}
%    \begin{macro}{\HoLogoHtml@LaTeXTeX}
%    \begin{macrocode}
\def\HoLogoHtml@LaTeXTeX#1{%
  \HoLogoCss@LaTeXTeX
  \HOLOGO@Span{LaTeXTeX}{%
    (%
    \HOLOGO@Span{L}{L}%
    \HOLOGO@Span{a}{A}%
    \HOLOGO@Span{ParenRight}{)}%
    \hologo{TeX}%
  }%
}
%    \end{macrocode}
%    \end{macro}
%    \begin{macro}{\HoLogoHtml@(La)TeX}
%    Kerning after opening parentheses and before closing parentheses
%    is $-0.1$\,em. The original values $-0.15$\,em
%    looked too ugly for a serif font.
%    \begin{macrocode}
\expandafter
\let\csname HoLogoHtml@(La)TeX\endcsname\HoLogoHtml@LaTeXTeX
%    \end{macrocode}
%    \end{macro}
%    \begin{macro}{\HoLogoCss@LaTeXTeX}
%    \begin{macrocode}
\def\HoLogoCss@LaTeXTeX{%
  \Css{%
    span.HoLogo-LaTeXTeX span.HoLogo-L{%
      margin-left:-.1em;%
    }%
  }%
  \Css{%
    span.HoLogo-LaTeXTeX span.HoLogo-a{%
      position:relative;%
      top:-.5ex;%
      margin-left:-.36em;%
      margin-right:-.1em;%
      font-size:85\%;%
    }%
  }%
  \Css{%
    span.HoLogo-LaTeXTeX span.HoLogo-ParenRight{%
      margin-right:-.15em;%
    }%
  }%
  \global\let\HoLogoCss@LaTeXTeX\relax
}
%    \end{macrocode}
%    \end{macro}
%
% \subsubsection{\hologo{LaTeXe}}
%
%    \begin{macro}{\HoLogo@LaTeXe}
%    Source: \hologo{LaTeX} kernel
%    \begin{macrocode}
\def\HoLogo@LaTeXe#1{%
  \hologo{LaTeX}%
  \kern.15em%
  \hbox{%
    \HOLOGO@MathSetup
    2%
    $_{\textstyle\varepsilon}$%
  }%
}
%    \end{macrocode}
%    \end{macro}
%
%    \begin{macro}{\HoLogoCs@LaTeXe}
%    \begin{macrocode}
\ifnum64=`\^^^^0040\relax % test for big chars of LuaTeX/XeTeX
  \catcode`\$=9 %
  \catcode`\&=14 %
\else
  \catcode`\$=14 %
  \catcode`\&=9 %
\fi
\def\HoLogoCs@LaTeXe#1{%
  LaTeX2%
$ \string ^^^^0395%
& e%
}%
\catcode`\$=3 %
\catcode`\&=4 %
%    \end{macrocode}
%    \end{macro}
%
%    \begin{macro}{\HoLogoBkm@LaTeXe}
%    \begin{macrocode}
\def\HoLogoBkm@LaTeXe#1{%
  \hologo{LaTeX}%
  2%
  \HOLOGO@PdfdocUnicode{e}{\textepsilon}%
}
%    \end{macrocode}
%    \end{macro}
%
%    \begin{macro}{\HoLogoHtml@LaTeXe}
%    \begin{macrocode}
\def\HoLogoHtml@LaTeXe#1{%
  \HoLogoCss@LaTeXe
  \HOLOGO@Span{LaTeX2e}{%
    \hologo{LaTeX}%
    \HOLOGO@Span{2}{2}%
    \HOLOGO@Span{e}{%
      \HOLOGO@MathSetup
      \ensuremath{\textstyle\varepsilon}%
    }%
  }%
}
%    \end{macrocode}
%    \end{macro}
%    \begin{macro}{\HoLogoCss@LaTeXe}
%    \begin{macrocode}
\def\HoLogoCss@LaTeXe{%
  \Css{%
    span.HoLogo-LaTeX2e span.HoLogo-2{%
      padding-left:.15em;%
    }%
  }%
  \Css{%
    span.HoLogo-LaTeX2e span.HoLogo-e{%
      position:relative;%
      top:.35ex;%
      text-decoration:none;%
    }%
  }%
  \global\let\HoLogoCss@LaTeXe\relax
}
%    \end{macrocode}
%    \end{macro}
%
%    \begin{macro}{\HoLogo@LaTeX2e}
%    \begin{macrocode}
\expandafter
\let\csname HoLogo@LaTeX2e\endcsname\HoLogo@LaTeXe
%    \end{macrocode}
%    \end{macro}
%    \begin{macro}{\HoLogoCs@LaTeX2e}
%    \begin{macrocode}
\expandafter
\let\csname HoLogoCs@LaTeX2e\endcsname\HoLogoCs@LaTeXe
%    \end{macrocode}
%    \end{macro}
%    \begin{macro}{\HoLogoBkm@LaTeX2e}
%    \begin{macrocode}
\expandafter
\let\csname HoLogoBkm@LaTeX2e\endcsname\HoLogoBkm@LaTeXe
%    \end{macrocode}
%    \end{macro}
%    \begin{macro}{\HoLogoHtml@LaTeX2e}
%    \begin{macrocode}
\expandafter
\let\csname HoLogoHtml@LaTeX2e\endcsname\HoLogoHtml@LaTeXe
%    \end{macrocode}
%    \end{macro}
%
% \subsubsection{\hologo{LaTeX3}}
%
%    \begin{macro}{\HoLogo@LaTeX3}
%    Source: \hologo{LaTeX} kernel
%    \begin{macrocode}
\expandafter\def\csname HoLogo@LaTeX3\endcsname#1{%
  \hologo{LaTeX}%
  3%
}
%    \end{macrocode}
%    \end{macro}
%
%    \begin{macro}{\HoLogoBkm@LaTeX3}
%    \begin{macrocode}
\expandafter\def\csname HoLogoBkm@LaTeX3\endcsname#1{%
  \hologo{LaTeX}%
  3%
}
%    \end{macrocode}
%    \end{macro}
%    \begin{macro}{\HoLogoHtml@LaTeX3}
%    \begin{macrocode}
\expandafter
\let\csname HoLogoHtml@LaTeX3\expandafter\endcsname
\csname HoLogo@LaTeX3\endcsname
%    \end{macrocode}
%    \end{macro}
%
% \subsubsection{\hologo{LaTeXML}}
%
%    \begin{macro}{\HoLogo@LaTeXML}
%    \begin{macrocode}
\def\HoLogo@LaTeXML#1{%
  \HOLOGO@mbox{%
    \hologo{La}%
    \kern-.15em%
    T%
    \kern-.1667em%
    \lower.5ex\hbox{E}%
    \kern-.125em%
    \HoLogoFont@font{LaTeXML}{sc}{xml}%
  }%
}
%    \end{macrocode}
%    \end{macro}
%    \begin{macro}{\HoLogoHtml@pdfLaTeX}
%    \begin{macrocode}
\def\HoLogoHtml@LaTeXML#1{%
  \HOLOGO@Span{LaTeXML}{%
    \HoLogoCss@LaTeX
    \HoLogoCss@TeX
    \HOLOGO@Span{LaTeX}{%
      L%
      \HOLOGO@Span{a}{%
        A%
      }%
    }%
    \HOLOGO@Span{TeX}{%
      T%
      \HOLOGO@Span{e}{%
        E%
      }%
    }%
    \HCode{<span style="font-variant: small-caps;">}%
    xml%
    \HCode{</span>}%
  }%
}
%    \end{macrocode}
%    \end{macro}
%
% \subsubsection{\hologo{eTeX}}
%
%    \begin{macro}{\HoLogo@eTeX}
%    Source: package \xpackage{etex}
%    \begin{macrocode}
\def\HoLogo@eTeX#1{%
  \ltx@mbox{%
    \HOLOGO@MathSetup
    $\varepsilon$%
    -%
    \HOLOGO@NegativeKerning{-T,T-,To}%
    \hologo{TeX}%
  }%
}
%    \end{macrocode}
%    \end{macro}
%    \begin{macro}{\HoLogoCs@eTeX}
%    \begin{macrocode}
\ifnum64=`\^^^^0040\relax % test for big chars of LuaTeX/XeTeX
  \catcode`\$=9 %
  \catcode`\&=14 %
\else
  \catcode`\$=14 %
  \catcode`\&=9 %
\fi
\def\HoLogoCs@eTeX#1{%
$ #1{\string ^^^^0395}{\string ^^^^03b5}%
& #1{e}{E}%
  TeX%
}%
\catcode`\$=3 %
\catcode`\&=4 %
%    \end{macrocode}
%    \end{macro}
%    \begin{macro}{\HoLogoBkm@eTeX}
%    \begin{macrocode}
\def\HoLogoBkm@eTeX#1{%
  \HOLOGO@PdfdocUnicode{#1{e}{E}}{\textepsilon}%
  -%
  \hologo{TeX}%
}
%    \end{macrocode}
%    \end{macro}
%    \begin{macro}{\HoLogoHtml@eTeX}
%    \begin{macrocode}
\def\HoLogoHtml@eTeX#1{%
  \ltx@mbox{%
    \HOLOGO@MathSetup
    $\varepsilon$%
    -%
    \hologo{TeX}%
  }%
}
%    \end{macrocode}
%    \end{macro}
%
% \subsubsection{\hologo{iniTeX}}
%
%    \begin{macro}{\HoLogo@iniTeX}
%    \begin{macrocode}
\def\HoLogo@iniTeX#1{%
  \HOLOGO@mbox{%
    #1{i}{I}ni\hologo{TeX}%
  }%
}
%    \end{macrocode}
%    \end{macro}
%    \begin{macro}{\HoLogoCs@iniTeX}
%    \begin{macrocode}
\def\HoLogoCs@iniTeX#1{#1{i}{I}niTeX}
%    \end{macrocode}
%    \end{macro}
%    \begin{macro}{\HoLogoBkm@iniTeX}
%    \begin{macrocode}
\def\HoLogoBkm@iniTeX#1{%
  #1{i}{I}ni\hologo{TeX}%
}
%    \end{macrocode}
%    \end{macro}
%    \begin{macro}{\HoLogoHtml@iniTeX}
%    \begin{macrocode}
\let\HoLogoHtml@iniTeX\HoLogo@iniTeX
%    \end{macrocode}
%    \end{macro}
%
% \subsubsection{\hologo{virTeX}}
%
%    \begin{macro}{\HoLogo@virTeX}
%    \begin{macrocode}
\def\HoLogo@virTeX#1{%
  \HOLOGO@mbox{%
    #1{v}{V}ir\hologo{TeX}%
  }%
}
%    \end{macrocode}
%    \end{macro}
%    \begin{macro}{\HoLogoCs@virTeX}
%    \begin{macrocode}
\def\HoLogoCs@virTeX#1{#1{v}{V}irTeX}
%    \end{macrocode}
%    \end{macro}
%    \begin{macro}{\HoLogoBkm@virTeX}
%    \begin{macrocode}
\def\HoLogoBkm@virTeX#1{%
  #1{v}{V}ir\hologo{TeX}%
}
%    \end{macrocode}
%    \end{macro}
%    \begin{macro}{\HoLogoHtml@virTeX}
%    \begin{macrocode}
\let\HoLogoHtml@virTeX\HoLogo@virTeX
%    \end{macrocode}
%    \end{macro}
%
% \subsubsection{\hologo{SliTeX}}
%
% \paragraph{Definitions of the three variants.}
%
%    \begin{macro}{\HoLogo@SLiTeX@lift}
%    \begin{macrocode}
\def\HoLogo@SLiTeX@lift#1{%
  \HoLogoFont@font{SliTeX}{rm}{%
    S%
    \kern-.06em%
    L%
    \kern-.18em%
    \raise.32ex\hbox{\HoLogoFont@font{SliTeX}{sc}{i}}%
    \HOLOGO@discretionary
    \kern-.06em%
    \hologo{TeX}%
  }%
}
%    \end{macrocode}
%    \end{macro}
%    \begin{macro}{\HoLogoBkm@SLiTeX@lift}
%    \begin{macrocode}
\def\HoLogoBkm@SLiTeX@lift#1{SLiTeX}
%    \end{macrocode}
%    \end{macro}
%    \begin{macro}{\HoLogoHtml@SLiTeX@lift}
%    \begin{macrocode}
\def\HoLogoHtml@SLiTeX@lift#1{%
  \HoLogoCss@SLiTeX@lift
  \HOLOGO@Span{SLiTeX-lift}{%
    \HoLogoFont@font{SliTeX}{rm}{%
      S%
      \HOLOGO@Span{L}{L}%
      \HOLOGO@Span{i}{i}%
      \hologo{TeX}%
    }%
  }%
}
%    \end{macrocode}
%    \end{macro}
%    \begin{macro}{\HoLogoCss@SLiTeX@lift}
%    \begin{macrocode}
\def\HoLogoCss@SLiTeX@lift{%
  \Css{%
    span.HoLogo-SLiTeX-lift span.HoLogo-L{%
      margin-left:-.06em;%
      margin-right:-.18em;%
    }%
  }%
  \Css{%
    span.HoLogo-SLiTeX-lift span.HoLogo-i{%
      position:relative;%
      top:-.32ex;%
      margin-right:-.06em;%
      font-variant:small-caps;%
    }%
  }%
  \global\let\HoLogoCss@SLiTeX@lift\relax
}
%    \end{macrocode}
%    \end{macro}
%
%    \begin{macro}{\HoLogo@SliTeX@simple}
%    \begin{macrocode}
\def\HoLogo@SliTeX@simple#1{%
  \HoLogoFont@font{SliTeX}{rm}{%
    \ltx@mbox{%
      \HoLogoFont@font{SliTeX}{sc}{Sli}%
    }%
    \HOLOGO@discretionary
    \hologo{TeX}%
  }%
}
%    \end{macrocode}
%    \end{macro}
%    \begin{macro}{\HoLogoBkm@SliTeX@simple}
%    \begin{macrocode}
\def\HoLogoBkm@SliTeX@simple#1{SliTeX}
%    \end{macrocode}
%    \end{macro}
%    \begin{macro}{\HoLogoHtml@SliTeX@simple}
%    \begin{macrocode}
\let\HoLogoHtml@SliTeX@simple\HoLogo@SliTeX@simple
%    \end{macrocode}
%    \end{macro}
%
%    \begin{macro}{\HoLogo@SliTeX@narrow}
%    \begin{macrocode}
\def\HoLogo@SliTeX@narrow#1{%
  \HoLogoFont@font{SliTeX}{rm}{%
    \ltx@mbox{%
      S%
      \kern-.06em%
      \HoLogoFont@font{SliTeX}{sc}{%
        l%
        \kern-.035em%
        i%
      }%
    }%
    \HOLOGO@discretionary
    \kern-.06em%
    \hologo{TeX}%
  }%
}
%    \end{macrocode}
%    \end{macro}
%    \begin{macro}{\HoLogoBkm@SliTeX@narrow}
%    \begin{macrocode}
\def\HoLogoBkm@SliTeX@narrow#1{SliTeX}
%    \end{macrocode}
%    \end{macro}
%    \begin{macro}{\HoLogoHtml@SliTeX@narrow}
%    \begin{macrocode}
\def\HoLogoHtml@SliTeX@narrow#1{%
  \HoLogoCss@SliTeX@narrow
  \HOLOGO@Span{SliTeX-narrow}{%
    \HoLogoFont@font{SliTeX}{rm}{%
      S%
        \HOLOGO@Span{l}{l}%
        \HOLOGO@Span{i}{i}%
      \hologo{TeX}%
    }%
  }%
}
%    \end{macrocode}
%    \end{macro}
%    \begin{macro}{\HoLogoCss@SliTeX@narrow}
%    \begin{macrocode}
\def\HoLogoCss@SliTeX@narrow{%
  \Css{%
    span.HoLogo-SliTeX-narrow span.HoLogo-l{%
      margin-left:-.06em;%
      margin-right:-.035em;%
      font-variant:small-caps;%
    }%
  }%
  \Css{%
    span.HoLogo-SliTeX-narrow span.HoLogo-i{%
      margin-right:-.06em;%
      font-variant:small-caps;%
    }%
  }%
  \global\let\HoLogoCss@SliTeX@narrow\relax
}
%    \end{macrocode}
%    \end{macro}
%
% \paragraph{Macro set completion.}
%
%    \begin{macro}{\HoLogo@SLiTeX@simple}
%    \begin{macrocode}
\def\HoLogo@SLiTeX@simple{\HoLogo@SliTeX@simple}
%    \end{macrocode}
%    \end{macro}
%    \begin{macro}{\HoLogoBkm@SLiTeX@simple}
%    \begin{macrocode}
\def\HoLogoBkm@SLiTeX@simple{\HoLogoBkm@SliTeX@simple}
%    \end{macrocode}
%    \end{macro}
%    \begin{macro}{\HoLogoHtml@SLiTeX@simple}
%    \begin{macrocode}
\def\HoLogoHtml@SLiTeX@simple{\HoLogoHtml@SliTeX@simple}
%    \end{macrocode}
%    \end{macro}
%
%    \begin{macro}{\HoLogo@SLiTeX@narrow}
%    \begin{macrocode}
\def\HoLogo@SLiTeX@narrow{\HoLogo@SliTeX@narrow}
%    \end{macrocode}
%    \end{macro}
%    \begin{macro}{\HoLogoBkm@SLiTeX@narrow}
%    \begin{macrocode}
\def\HoLogoBkm@SLiTeX@narrow{\HoLogoBkm@SliTeX@narrow}
%    \end{macrocode}
%    \end{macro}
%    \begin{macro}{\HoLogoHtml@SLiTeX@narrow}
%    \begin{macrocode}
\def\HoLogoHtml@SLiTeX@narrow{\HoLogoHtml@SliTeX@narrow}
%    \end{macrocode}
%    \end{macro}
%
%    \begin{macro}{\HoLogo@SliTeX@lift}
%    \begin{macrocode}
\def\HoLogo@SliTeX@lift{\HoLogo@SLiTeX@lift}
%    \end{macrocode}
%    \end{macro}
%    \begin{macro}{\HoLogoBkm@SliTeX@lift}
%    \begin{macrocode}
\def\HoLogoBkm@SliTeX@lift{\HoLogoBkm@SLiTeX@lift}
%    \end{macrocode}
%    \end{macro}
%    \begin{macro}{\HoLogoHtml@SliTeX@lift}
%    \begin{macrocode}
\def\HoLogoHtml@SliTeX@lift{\HoLogoHtml@SLiTeX@lift}
%    \end{macrocode}
%    \end{macro}
%
% \paragraph{Defaults.}
%
%    \begin{macro}{\HoLogo@SLiTeX}
%    \begin{macrocode}
\def\HoLogo@SLiTeX{\HoLogo@SLiTeX@lift}
%    \end{macrocode}
%    \end{macro}
%    \begin{macro}{\HoLogoBkm@SLiTeX}
%    \begin{macrocode}
\def\HoLogoBkm@SLiTeX{\HoLogoBkm@SLiTeX@lift}
%    \end{macrocode}
%    \end{macro}
%    \begin{macro}{\HoLogoHtml@SLiTeX}
%    \begin{macrocode}
\def\HoLogoHtml@SLiTeX{\HoLogoHtml@SLiTeX@lift}
%    \end{macrocode}
%    \end{macro}
%
%    \begin{macro}{\HoLogo@SliTeX}
%    \begin{macrocode}
\def\HoLogo@SliTeX{\HoLogo@SliTeX@narrow}
%    \end{macrocode}
%    \end{macro}
%    \begin{macro}{\HoLogoBkm@SliTeX}
%    \begin{macrocode}
\def\HoLogoBkm@SliTeX{\HoLogoBkm@SliTeX@narrow}
%    \end{macrocode}
%    \end{macro}
%    \begin{macro}{\HoLogoHtml@SliTeX}
%    \begin{macrocode}
\def\HoLogoHtml@SliTeX{\HoLogoHtml@SliTeX@narrow}
%    \end{macrocode}
%    \end{macro}
%
% \subsubsection{\hologo{LuaTeX}}
%
%    \begin{macro}{\HoLogo@LuaTeX}
%    The kerning is an idea of Hans Hagen, see mailing list
%    `luatex at tug dot org' in March 2010.
%    \begin{macrocode}
\def\HoLogo@LuaTeX#1{%
  \HOLOGO@mbox{%
    Lua%
    \HOLOGO@NegativeKerning{aT,oT,To}%
    \hologo{TeX}%
  }%
}
%    \end{macrocode}
%    \end{macro}
%    \begin{macro}{\HoLogoHtml@LuaTeX}
%    \begin{macrocode}
\let\HoLogoHtml@LuaTeX\HoLogo@LuaTeX
%    \end{macrocode}
%    \end{macro}
%
% \subsubsection{\hologo{LuaLaTeX}}
%
%    \begin{macro}{\HoLogo@LuaLaTeX}
%    \begin{macrocode}
\def\HoLogo@LuaLaTeX#1{%
  \HOLOGO@mbox{%
    Lua%
    \hologo{LaTeX}%
  }%
}
%    \end{macrocode}
%    \end{macro}
%    \begin{macro}{\HoLogoHtml@LuaLaTeX}
%    \begin{macrocode}
\let\HoLogoHtml@LuaLaTeX\HoLogo@LuaLaTeX
%    \end{macrocode}
%    \end{macro}
%
% \subsubsection{\hologo{XeTeX}, \hologo{XeLaTeX}}
%
%    \begin{macro}{\HOLOGO@IfCharExists}
%    \begin{macrocode}
\ifluatex
  \ifnum\luatexversion<36 %
  \else
    \def\HOLOGO@IfCharExists#1{%
      \ifnum
        \directlua{%
           if luaotfload and luaotfload.aux then
             if luaotfload.aux.font_has_glyph(%
                    font.current(), \number#1) then % 	 
	       tex.print("1") % 	 
	     end % 	 
	   elseif font and font.fonts and font.current then %
            local f = font.fonts[font.current()]%
            if f.characters and f.characters[\number#1] then %
              tex.print("1")%
            end %
          end%
        }0=\ltx@zero
        \expandafter\ltx@secondoftwo
      \else
        \expandafter\ltx@firstoftwo
      \fi
    }%
  \fi
\fi
\ltx@IfUndefined{HOLOGO@IfCharExists}{%
  \def\HOLOGO@@IfCharExists#1{%
    \begingroup
      \tracinglostchars=\ltx@zero
      \setbox\ltx@zero=\hbox{%
        \kern7sp\char#1\relax
        \ifnum\lastkern>\ltx@zero
          \expandafter\aftergroup\csname iffalse\endcsname
        \else
          \expandafter\aftergroup\csname iftrue\endcsname
        \fi
      }%
      % \if{true|false} from \aftergroup
      \endgroup
      \expandafter\ltx@firstoftwo
    \else
      \endgroup
      \expandafter\ltx@secondoftwo
    \fi
  }%
  \ifxetex
    \ltx@IfUndefined{XeTeXfonttype}{}{%
      \ltx@IfUndefined{XeTeXcharglyph}{}{%
        \def\HOLOGO@IfCharExists#1{%
          \ifnum\XeTeXfonttype\font>\ltx@zero
            \expandafter\ltx@firstofthree
          \else
            \expandafter\ltx@gobble
          \fi
          {%
            \ifnum\XeTeXcharglyph#1>\ltx@zero
              \expandafter\ltx@firstoftwo
            \else
              \expandafter\ltx@secondoftwo
            \fi
          }%
          \HOLOGO@@IfCharExists{#1}%
        }%
      }%
    }%
  \fi
}{}
\ltx@ifundefined{HOLOGO@IfCharExists}{%
  \ifnum64=`\^^^^0040\relax % test for big chars of LuaTeX/XeTeX
    \let\HOLOGO@IfCharExists\HOLOGO@@IfCharExists
  \else
    \def\HOLOGO@IfCharExists#1{%
      \ifnum#1>255 %
        \expandafter\ltx@fourthoffour
      \fi
      \HOLOGO@@IfCharExists{#1}%
    }%
  \fi
}{}
%    \end{macrocode}
%    \end{macro}
%
%    \begin{macro}{\HoLogo@Xe}
%    Source: package \xpackage{dtklogos}
%    \begin{macrocode}
\def\HoLogo@Xe#1{%
  X%
  \kern-.1em\relax
  \HOLOGO@IfCharExists{"018E}{%
    \lower.5ex\hbox{\char"018E}%
  }{%
    \chardef\HOLOGO@choice=\ltx@zero
    \ifdim\fontdimen\ltx@one\font>0pt %
      \ltx@IfUndefined{rotatebox}{%
        \ltx@IfUndefined{pgftext}{%
          \ltx@IfUndefined{psscalebox}{%
            \ltx@IfUndefined{HOLOGO@ScaleBox@\hologoDriver}{%
            }{%
              \chardef\HOLOGO@choice=4 %
            }%
          }{%
            \chardef\HOLOGO@choice=3 %
          }%
        }{%
          \chardef\HOLOGO@choice=2 %
        }%
      }{%
        \chardef\HOLOGO@choice=1 %
      }%
      \ifcase\HOLOGO@choice
        \HOLOGO@WarningUnsupportedDriver{Xe}%
        e%
      \or % 1: \rotatebox
        \begingroup
          \setbox\ltx@zero\hbox{\rotatebox{180}{E}}%
          \ltx@LocDimenA=\dp\ltx@zero
          \advance\ltx@LocDimenA by -.5ex\relax
          \raise\ltx@LocDimenA\box\ltx@zero
        \endgroup
      \or % 2: \pgftext
        \lower.5ex\hbox{%
          \pgfpicture
            \pgftext[rotate=180]{E}%
          \endpgfpicture
        }%
      \or % 3: \psscalebox
        \begingroup
          \setbox\ltx@zero\hbox{\psscalebox{-1 -1}{E}}%
          \ltx@LocDimenA=\dp\ltx@zero
          \advance\ltx@LocDimenA by -.5ex\relax
          \raise\ltx@LocDimenA\box\ltx@zero
        \endgroup
      \or % 4: \HOLOGO@PointReflectBox
        \lower.5ex\hbox{\HOLOGO@PointReflectBox{E}}%
      \else
        \@PackageError{hologo}{Internal error (choice/it}\@ehc
      \fi
    \else
      \ltx@IfUndefined{reflectbox}{%
        \ltx@IfUndefined{pgftext}{%
          \ltx@IfUndefined{psscalebox}{%
            \ltx@IfUndefined{HOLOGO@ScaleBox@\hologoDriver}{%
            }{%
              \chardef\HOLOGO@choice=4 %
            }%
          }{%
            \chardef\HOLOGO@choice=3 %
          }%
        }{%
          \chardef\HOLOGO@choice=2 %
        }%
      }{%
        \chardef\HOLOGO@choice=1 %
      }%
      \ifcase\HOLOGO@choice
        \HOLOGO@WarningUnsupportedDriver{Xe}%
        e%
      \or % 1: reflectbox
        \lower.5ex\hbox{%
          \reflectbox{E}%
        }%
      \or % 2: \pgftext
        \lower.5ex\hbox{%
          \pgfpicture
            \pgftransformxscale{-1}%
            \pgftext{E}%
          \endpgfpicture
        }%
      \or % 3: \psscalebox
        \lower.5ex\hbox{%
          \psscalebox{-1 1}{E}%
        }%
      \or % 4: \HOLOGO@Reflectbox
        \lower.5ex\hbox{%
          \HOLOGO@ReflectBox{E}%
        }%
      \else
        \@PackageError{hologo}{Internal error (choice/up)}\@ehc
      \fi
    \fi
  }%
}
%    \end{macrocode}
%    \end{macro}
%    \begin{macro}{\HoLogoHtml@Xe}
%    \begin{macrocode}
\def\HoLogoHtml@Xe#1{%
  \HoLogoCss@Xe
  \HOLOGO@Span{Xe}{%
    X%
    \HOLOGO@Span{e}{%
      \HCode{&\ltx@hashchar x018e;}%
    }%
  }%
}
%    \end{macrocode}
%    \end{macro}
%    \begin{macro}{\HoLogoCss@Xe}
%    \begin{macrocode}
\def\HoLogoCss@Xe{%
  \Css{%
    span.HoLogo-Xe span.HoLogo-e{%
      position:relative;%
      top:.5ex;%
      left-margin:-.1em;%
    }%
  }%
  \global\let\HoLogoCss@Xe\relax
}
%    \end{macrocode}
%    \end{macro}
%
%    \begin{macro}{\HoLogo@XeTeX}
%    \begin{macrocode}
\def\HoLogo@XeTeX#1{%
  \hologo{Xe}%
  \kern-.15em\relax
  \hologo{TeX}%
}
%    \end{macrocode}
%    \end{macro}
%
%    \begin{macro}{\HoLogoHtml@XeTeX}
%    \begin{macrocode}
\def\HoLogoHtml@XeTeX#1{%
  \HoLogoCss@XeTeX
  \HOLOGO@Span{XeTeX}{%
    \hologo{Xe}%
    \hologo{TeX}%
  }%
}
%    \end{macrocode}
%    \end{macro}
%    \begin{macro}{\HoLogoCss@XeTeX}
%    \begin{macrocode}
\def\HoLogoCss@XeTeX{%
  \Css{%
    span.HoLogo-XeTeX span.HoLogo-TeX{%
      margin-left:-.15em;%
    }%
  }%
  \global\let\HoLogoCss@XeTeX\relax
}
%    \end{macrocode}
%    \end{macro}
%
%    \begin{macro}{\HoLogo@XeLaTeX}
%    \begin{macrocode}
\def\HoLogo@XeLaTeX#1{%
  \hologo{Xe}%
  \kern-.13em%
  \hologo{LaTeX}%
}
%    \end{macrocode}
%    \end{macro}
%    \begin{macro}{\HoLogoHtml@XeLaTeX}
%    \begin{macrocode}
\def\HoLogoHtml@XeLaTeX#1{%
  \HoLogoCss@XeLaTeX
  \HOLOGO@Span{XeLaTeX}{%
    \hologo{Xe}%
    \hologo{LaTeX}%
  }%
}
%    \end{macrocode}
%    \end{macro}
%    \begin{macro}{\HoLogoCss@XeLaTeX}
%    \begin{macrocode}
\def\HoLogoCss@XeLaTeX{%
  \Css{%
    span.HoLogo-XeLaTeX span.HoLogo-Xe{%
      margin-right:-.13em;%
    }%
  }%
  \global\let\HoLogoCss@XeLaTeX\relax
}
%    \end{macrocode}
%    \end{macro}
%
% \subsubsection{\hologo{pdfTeX}, \hologo{pdfLaTeX}}
%
%    \begin{macro}{\HoLogo@pdfTeX}
%    \begin{macrocode}
\def\HoLogo@pdfTeX#1{%
  \HOLOGO@mbox{%
    #1{p}{P}df\hologo{TeX}%
  }%
}
%    \end{macrocode}
%    \end{macro}
%    \begin{macro}{\HoLogoCs@pdfTeX}
%    \begin{macrocode}
\def\HoLogoCs@pdfTeX#1{#1{p}{P}dfTeX}
%    \end{macrocode}
%    \end{macro}
%    \begin{macro}{\HoLogoBkm@pdfTeX}
%    \begin{macrocode}
\def\HoLogoBkm@pdfTeX#1{%
  #1{p}{P}df\hologo{TeX}%
}
%    \end{macrocode}
%    \end{macro}
%    \begin{macro}{\HoLogoHtml@pdfTeX}
%    \begin{macrocode}
\let\HoLogoHtml@pdfTeX\HoLogo@pdfTeX
%    \end{macrocode}
%    \end{macro}
%
%    \begin{macro}{\HoLogo@pdfLaTeX}
%    \begin{macrocode}
\def\HoLogo@pdfLaTeX#1{%
  \HOLOGO@mbox{%
    #1{p}{P}df\hologo{LaTeX}%
  }%
}
%    \end{macrocode}
%    \end{macro}
%    \begin{macro}{\HoLogoCs@pdfLaTeX}
%    \begin{macrocode}
\def\HoLogoCs@pdfLaTeX#1{#1{p}{P}dfLaTeX}
%    \end{macrocode}
%    \end{macro}
%    \begin{macro}{\HoLogoBkm@pdfLaTeX}
%    \begin{macrocode}
\def\HoLogoBkm@pdfLaTeX#1{%
  #1{p}{P}df\hologo{LaTeX}%
}
%    \end{macrocode}
%    \end{macro}
%    \begin{macro}{\HoLogoHtml@pdfLaTeX}
%    \begin{macrocode}
\let\HoLogoHtml@pdfLaTeX\HoLogo@pdfLaTeX
%    \end{macrocode}
%    \end{macro}
%
% \subsubsection{\hologo{VTeX}}
%
%    \begin{macro}{\HoLogo@VTeX}
%    \begin{macrocode}
\def\HoLogo@VTeX#1{%
  \HOLOGO@mbox{%
    V\hologo{TeX}%
  }%
}
%    \end{macrocode}
%    \end{macro}
%    \begin{macro}{\HoLogoHtml@VTeX}
%    \begin{macrocode}
\let\HoLogoHtml@VTeX\HoLogo@VTeX
%    \end{macrocode}
%    \end{macro}
%
% \subsubsection{\hologo{AmS}, \dots}
%
%    Source: class \xclass{amsdtx}
%
%    \begin{macro}{\HoLogo@AmS}
%    \begin{macrocode}
\def\HoLogo@AmS#1{%
  \HoLogoFont@font{AmS}{sy}{%
    A%
    \kern-.1667em%
    \lower.5ex\hbox{M}%
    \kern-.125em%
    S%
  }%
}
%    \end{macrocode}
%    \end{macro}
%    \begin{macro}{\HoLogoBkm@AmS}
%    \begin{macrocode}
\def\HoLogoBkm@AmS#1{AmS}
%    \end{macrocode}
%    \end{macro}
%    \begin{macro}{\HoLogoHtml@AmS}
%    \begin{macrocode}
\def\HoLogoHtml@AmS#1{%
  \HoLogoCss@AmS
%  \HoLogoFont@font{AmS}{sy}{%
    \HOLOGO@Span{AmS}{%
      A%
      \HOLOGO@Span{M}{M}%
      S%
    }%
%   }%
}
%    \end{macrocode}
%    \end{macro}
%    \begin{macro}{\HoLogoCss@AmS}
%    \begin{macrocode}
\def\HoLogoCss@AmS{%
  \Css{%
    span.HoLogo-AmS span.HoLogo-M{%
      position:relative;%
      top:.5ex;%
      margin-left:-.1667em;%
      margin-right:-.125em;%
      text-decoration:none;%
    }%
  }%
  \global\let\HoLogoCss@AmS\relax
}
%    \end{macrocode}
%    \end{macro}
%
%    \begin{macro}{\HoLogo@AmSTeX}
%    \begin{macrocode}
\def\HoLogo@AmSTeX#1{%
  \hologo{AmS}%
  \HOLOGO@hyphen
  \hologo{TeX}%
}
%    \end{macrocode}
%    \end{macro}
%    \begin{macro}{\HoLogoBkm@AmSTeX}
%    \begin{macrocode}
\def\HoLogoBkm@AmSTeX#1{AmS-TeX}%
%    \end{macrocode}
%    \end{macro}
%    \begin{macro}{\HoLogoHtml@AmSTeX}
%    \begin{macrocode}
\let\HoLogoHtml@AmSTeX\HoLogo@AmSTeX
%    \end{macrocode}
%    \end{macro}
%
%    \begin{macro}{\HoLogo@AmSLaTeX}
%    \begin{macrocode}
\def\HoLogo@AmSLaTeX#1{%
  \hologo{AmS}%
  \HOLOGO@hyphen
  \hologo{LaTeX}%
}
%    \end{macrocode}
%    \end{macro}
%    \begin{macro}{\HoLogoBkm@AmSLaTeX}
%    \begin{macrocode}
\def\HoLogoBkm@AmSLaTeX#1{AmS-LaTeX}%
%    \end{macrocode}
%    \end{macro}
%    \begin{macro}{\HoLogoHtml@AmSLaTeX}
%    \begin{macrocode}
\let\HoLogoHtml@AmSLaTeX\HoLogo@AmSLaTeX
%    \end{macrocode}
%    \end{macro}
%
% \subsubsection{\hologo{BibTeX}}
%
%    \begin{macro}{\HoLogo@BibTeX@sc}
%    A definition of \hologo{BibTeX} is provided in
%    the documentation source for the manual of \hologo{BibTeX}
%    \cite{btxdoc}.
%\begin{quote}
%\begin{verbatim}
%\def\BibTeX{%
%  {%
%    \rm
%    B%
%    \kern-.05em%
%    {%
%      \sc
%      i%
%      \kern-.025em %
%      b%
%    }%
%    \kern-.08em
%    T%
%    \kern-.1667em%
%    \lower.7ex\hbox{E}%
%    \kern-.125em%
%    X%
%  }%
%}
%\end{verbatim}
%\end{quote}
%    \begin{macrocode}
\def\HoLogo@BibTeX@sc#1{%
  B%
  \kern-.05em%
  \HoLogoFont@font{BibTeX}{sc}{%
    i%
    \kern-.025em%
    b%
  }%
  \HOLOGO@discretionary
  \kern-.08em%
  \hologo{TeX}%
}
%    \end{macrocode}
%    \end{macro}
%    \begin{macro}{\HoLogoHtml@BibTeX@sc}
%    \begin{macrocode}
\def\HoLogoHtml@BibTeX@sc#1{%
  \HoLogoCss@BibTeX@sc
  \HOLOGO@Span{BibTeX-sc}{%
    B%
    \HOLOGO@Span{i}{i}%
    \HOLOGO@Span{b}{b}%
    \hologo{TeX}%
  }%
}
%    \end{macrocode}
%    \end{macro}
%    \begin{macro}{\HoLogoCss@BibTeX@sc}
%    \begin{macrocode}
\def\HoLogoCss@BibTeX@sc{%
  \Css{%
    span.HoLogo-BibTeX-sc span.HoLogo-i{%
      margin-left:-.05em;%
      margin-right:-.025em;%
      font-variant:small-caps;%
    }%
  }%
  \Css{%
    span.HoLogo-BibTeX-sc span.HoLogo-b{%
      margin-right:-.08em;%
      font-variant:small-caps;%
    }%
  }%
  \global\let\HoLogoCss@BibTeX@sc\relax
}
%    \end{macrocode}
%    \end{macro}
%
%    \begin{macro}{\HoLogo@BibTeX@sf}
%    Variant \xoption{sf} avoids trouble with unavailable
%    small caps fonts (e.g., bold versions of Computer Modern or
%    Latin Modern). The definition is taken from
%    package \xpackage{dtklogos} \cite{dtklogos}.
%\begin{quote}
%\begin{verbatim}
%\DeclareRobustCommand{\BibTeX}{%
%  B%
%  \kern-.05em%
%  \hbox{%
%    $\m@th$% %% force math size calculations
%    \csname S@\f@size\endcsname
%    \fontsize\sf@size\z@
%    \math@fontsfalse
%    \selectfont
%    I%
%    \kern-.025em%
%    B
%  }%
%  \kern-.08em%
%  \-%
%  \TeX
%}
%\end{verbatim}
%\end{quote}
%    \begin{macrocode}
\def\HoLogo@BibTeX@sf#1{%
  B%
  \kern-.05em%
  \HoLogoFont@font{BibTeX}{bibsf}{%
    I%
    \kern-.025em%
    B%
  }%
  \HOLOGO@discretionary
  \kern-.08em%
  \hologo{TeX}%
}
%    \end{macrocode}
%    \end{macro}
%    \begin{macro}{\HoLogoHtml@BibTeX@sf}
%    \begin{macrocode}
\def\HoLogoHtml@BibTeX@sf#1{%
  \HoLogoCss@BibTeX@sf
  \HOLOGO@Span{BibTeX-sf}{%
    B%
    \HoLogoFont@font{BibTeX}{bibsf}{%
      \HOLOGO@Span{i}{I}%
      B%
    }%
    \hologo{TeX}%
  }%
}
%    \end{macrocode}
%    \end{macro}
%    \begin{macro}{\HoLogoCss@BibTeX@sf}
%    \begin{macrocode}
\def\HoLogoCss@BibTeX@sf{%
  \Css{%
    span.HoLogo-BibTeX-sf span.HoLogo-i{%
      margin-left:-.05em;%
      margin-right:-.025em;%
    }%
  }%
  \Css{%
    span.HoLogo-BibTeX-sf span.HoLogo-TeX{%
      margin-left:-.08em;%
    }%
  }%
  \global\let\HoLogoCss@BibTeX@sf\relax
}
%    \end{macrocode}
%    \end{macro}
%
%    \begin{macro}{\HoLogo@BibTeX}
%    \begin{macrocode}
\def\HoLogo@BibTeX{\HoLogo@BibTeX@sf}
%    \end{macrocode}
%    \end{macro}
%    \begin{macro}{\HoLogoHtml@BibTeX}
%    \begin{macrocode}
\def\HoLogoHtml@BibTeX{\HoLogoHtml@BibTeX@sf}
%    \end{macrocode}
%    \end{macro}
%
% \subsubsection{\hologo{BibTeX8}}
%
%    \begin{macro}{\HoLogo@BibTeX8}
%    \begin{macrocode}
\expandafter\def\csname HoLogo@BibTeX8\endcsname#1{%
  \hologo{BibTeX}%
  8%
}
%    \end{macrocode}
%    \end{macro}
%
%    \begin{macro}{\HoLogoBkm@BibTeX8}
%    \begin{macrocode}
\expandafter\def\csname HoLogoBkm@BibTeX8\endcsname#1{%
  \hologo{BibTeX}%
  8%
}
%    \end{macrocode}
%    \end{macro}
%    \begin{macro}{\HoLogoHtml@BibTeX8}
%    \begin{macrocode}
\expandafter
\let\csname HoLogoHtml@BibTeX8\expandafter\endcsname
\csname HoLogo@BibTeX8\endcsname
%    \end{macrocode}
%    \end{macro}
%
% \subsubsection{\hologo{ConTeXt}}
%
%    \begin{macro}{\HoLogo@ConTeXt@simple}
%    \begin{macrocode}
\def\HoLogo@ConTeXt@simple#1{%
  \HOLOGO@mbox{Con}%
  \HOLOGO@discretionary
  \HOLOGO@mbox{\hologo{TeX}t}%
}
%    \end{macrocode}
%    \end{macro}
%    \begin{macro}{\HoLogoHtml@ConTeXt@simple}
%    \begin{macrocode}
\let\HoLogoHtml@ConTeXt@simple\HoLogo@ConTeXt@simple
%    \end{macrocode}
%    \end{macro}
%
%    \begin{macro}{\HoLogo@ConTeXt@narrow}
%    This definition of logo \hologo{ConTeXt} with variant \xoption{narrow}
%    comes from TUGboat's class \xclass{ltugboat} (version 2010/11/15 v2.8).
%    \begin{macrocode}
\def\HoLogo@ConTeXt@narrow#1{%
  \HOLOGO@mbox{C\kern-.0333emon}%
  \HOLOGO@discretionary
  \kern-.0667em%
  \HOLOGO@mbox{\hologo{TeX}\kern-.0333emt}%
}
%    \end{macrocode}
%    \end{macro}
%    \begin{macro}{\HoLogoHtml@ConTeXt@narrow}
%    \begin{macrocode}
\def\HoLogoHtml@ConTeXt@narrow#1{%
  \HoLogoCss@ConTeXt@narrow
  \HOLOGO@Span{ConTeXt-narrow}{%
    \HOLOGO@Span{C}{C}%
    on%
    \hologo{TeX}%
    t%
  }%
}
%    \end{macrocode}
%    \end{macro}
%    \begin{macro}{\HoLogoCss@ConTeXt@narrow}
%    \begin{macrocode}
\def\HoLogoCss@ConTeXt@narrow{%
  \Css{%
    span.HoLogo-ConTeXt-narrow span.HoLogo-C{%
      margin-left:-.0333em;%
    }%
  }%
  \Css{%
    span.HoLogo-ConTeXt-narrow span.HoLogo-TeX{%
      margin-left:-.0667em;%
      margin-right:-.0333em;%
    }%
  }%
  \global\let\HoLogoCss@ConTeXt@narrow\relax
}
%    \end{macrocode}
%    \end{macro}
%
%    \begin{macro}{\HoLogo@ConTeXt}
%    \begin{macrocode}
\def\HoLogo@ConTeXt{\HoLogo@ConTeXt@narrow}
%    \end{macrocode}
%    \end{macro}
%    \begin{macro}{\HoLogoHtml@ConTeXt}
%    \begin{macrocode}
\def\HoLogoHtml@ConTeXt{\HoLogoHtml@ConTeXt@narrow}
%    \end{macrocode}
%    \end{macro}
%
% \subsubsection{\hologo{emTeX}}
%
%    \begin{macro}{\HoLogo@emTeX}
%    \begin{macrocode}
\def\HoLogo@emTeX#1{%
  \HOLOGO@mbox{#1{e}{E}m}%
  \HOLOGO@discretionary
  \hologo{TeX}%
}
%    \end{macrocode}
%    \end{macro}
%    \begin{macro}{\HoLogoCs@emTeX}
%    \begin{macrocode}
\def\HoLogoCs@emTeX#1{#1{e}{E}mTeX}%
%    \end{macrocode}
%    \end{macro}
%    \begin{macro}{\HoLogoBkm@emTeX}
%    \begin{macrocode}
\def\HoLogoBkm@emTeX#1{%
  #1{e}{E}m\hologo{TeX}%
}
%    \end{macrocode}
%    \end{macro}
%    \begin{macro}{\HoLogoHtml@emTeX}
%    \begin{macrocode}
\let\HoLogoHtml@emTeX\HoLogo@emTeX
%    \end{macrocode}
%    \end{macro}
%
% \subsubsection{\hologo{ExTeX}}
%
%    \begin{macro}{\HoLogo@ExTeX}
%    The definition is taken from the FAQ of the
%    project \hologo{ExTeX}
%    \cite{ExTeX-FAQ}.
%\begin{quote}
%\begin{verbatim}
%\def\ExTeX{%
%  \textrm{% Logo always with serifs
%    \ensuremath{%
%      \textstyle
%      \varepsilon_{%
%        \kern-0.15em%
%        \mathcal{X}%
%      }%
%    }%
%    \kern-.15em%
%    \TeX
%  }%
%}
%\end{verbatim}
%\end{quote}
%    \begin{macrocode}
\def\HoLogo@ExTeX#1{%
  \HoLogoFont@font{ExTeX}{rm}{%
    \ltx@mbox{%
      \HOLOGO@MathSetup
      $%
        \textstyle
        \varepsilon_{%
          \kern-0.15em%
          \HoLogoFont@font{ExTeX}{sy}{X}%
        }%
      $%
    }%
    \HOLOGO@discretionary
    \kern-.15em%
    \hologo{TeX}%
  }%
}
%    \end{macrocode}
%    \end{macro}
%    \begin{macro}{\HoLogoHtml@ExTeX}
%    \begin{macrocode}
\def\HoLogoHtml@ExTeX#1{%
  \HoLogoCss@ExTeX
  \HoLogoFont@font{ExTeX}{rm}{%
    \HOLOGO@Span{ExTeX}{%
      \ltx@mbox{%
        \HOLOGO@MathSetup
        $\textstyle\varepsilon$%
        \HOLOGO@Span{X}{$\textstyle\chi$}%
        \hologo{TeX}%
      }%
    }%
  }%
}
%    \end{macrocode}
%    \end{macro}
%    \begin{macro}{\HoLogoBkm@ExTeX}
%    \begin{macrocode}
\def\HoLogoBkm@ExTeX#1{%
  \HOLOGO@PdfdocUnicode{#1{e}{E}x}{\textepsilon\textchi}%
  \hologo{TeX}%
}
%    \end{macrocode}
%    \end{macro}
%    \begin{macro}{\HoLogoCss@ExTeX}
%    \begin{macrocode}
\def\HoLogoCss@ExTeX{%
  \Css{%
    span.HoLogo-ExTeX{%
      font-family:serif;%
    }%
  }%
  \Css{%
    span.HoLogo-ExTeX span.HoLogo-TeX{%
      margin-left:-.15em;%
    }%
  }%
  \global\let\HoLogoCss@ExTeX\relax
}
%    \end{macrocode}
%    \end{macro}
%
% \subsubsection{\hologo{MiKTeX}}
%
%    \begin{macro}{\HoLogo@MiKTeX}
%    \begin{macrocode}
\def\HoLogo@MiKTeX#1{%
  \HOLOGO@mbox{MiK}%
  \HOLOGO@discretionary
  \hologo{TeX}%
}
%    \end{macrocode}
%    \end{macro}
%    \begin{macro}{\HoLogoHtml@MiKTeX}
%    \begin{macrocode}
\let\HoLogoHtml@MiKTeX\HoLogo@MiKTeX
%    \end{macrocode}
%    \end{macro}
%
% \subsubsection{\hologo{OzTeX} and friends}
%
%    Source: \hologo{OzTeX} FAQ \cite{OzTeX}:
%    \begin{quote}
%      |\def\OzTeX{O\kern-.03em z\kern-.15em\TeX}|\\
%      (There is no kerning in OzMF, OzMP and OzTtH.)
%    \end{quote}
%
%    \begin{macro}{\HoLogo@OzTeX}
%    \begin{macrocode}
\def\HoLogo@OzTeX#1{%
  O%
  \kern-.03em %
  z%
  \kern-.15em %
  \hologo{TeX}%
}
%    \end{macrocode}
%    \end{macro}
%    \begin{macro}{\HoLogoHtml@OzTeX}
%    \begin{macrocode}
\def\HoLogoHtml@OzTeX#1{%
  \HoLogoCss@OzTeX
  \HOLOGO@Span{OzTeX}{%
    O%
    \HOLOGO@Span{z}{z}%
    \hologo{TeX}%
  }%
}
%    \end{macrocode}
%    \end{macro}
%    \begin{macro}{\HoLogoCss@OzTeX}
%    \begin{macrocode}
\def\HoLogoCss@OzTeX{%
  \Css{%
    span.HoLogo-OzTeX span.HoLogo-z{%
      margin-left:-.03em;%
      margin-right:-.15em;%
    }%
  }%
  \global\let\HoLogoCss@OzTeX\relax
}
%    \end{macrocode}
%    \end{macro}
%
%    \begin{macro}{\HoLogo@OzMF}
%    \begin{macrocode}
\def\HoLogo@OzMF#1{%
  \HOLOGO@mbox{OzMF}%
}
%    \end{macrocode}
%    \end{macro}
%    \begin{macro}{\HoLogo@OzMP}
%    \begin{macrocode}
\def\HoLogo@OzMP#1{%
  \HOLOGO@mbox{OzMP}%
}
%    \end{macrocode}
%    \end{macro}
%    \begin{macro}{\HoLogo@OzTtH}
%    \begin{macrocode}
\def\HoLogo@OzTtH#1{%
  \HOLOGO@mbox{OzTtH}%
}
%    \end{macrocode}
%    \end{macro}
%
% \subsubsection{\hologo{PCTeX}}
%
%    \begin{macro}{\HoLogo@PCTeX}
%    \begin{macrocode}
\def\HoLogo@PCTeX#1{%
  \HOLOGO@mbox{PC}%
  \hologo{TeX}%
}
%    \end{macrocode}
%    \end{macro}
%    \begin{macro}{\HoLogoHtml@PCTeX}
%    \begin{macrocode}
\let\HoLogoHtml@PCTeX\HoLogo@PCTeX
%    \end{macrocode}
%    \end{macro}
%
% \subsubsection{\hologo{PiCTeX}}
%
%    The original definitions from \xfile{pictex.tex} \cite{PiCTeX}:
%\begin{quote}
%\begin{verbatim}
%\def\PiC{%
%  P%
%  \kern-.12em%
%  \lower.5ex\hbox{I}%
%  \kern-.075em%
%  C%
%}
%\def\PiCTeX{%
%  \PiC
%  \kern-.11em%
%  \TeX
%}
%\end{verbatim}
%\end{quote}
%
%    \begin{macro}{\HoLogo@PiC}
%    \begin{macrocode}
\def\HoLogo@PiC#1{%
  P%
  \kern-.12em%
  \lower.5ex\hbox{I}%
  \kern-.075em%
  C%
  \HOLOGO@SpaceFactor
}
%    \end{macrocode}
%    \end{macro}
%    \begin{macro}{\HoLogoHtml@PiC}
%    \begin{macrocode}
\def\HoLogoHtml@PiC#1{%
  \HoLogoCss@PiC
  \HOLOGO@Span{PiC}{%
    P%
    \HOLOGO@Span{i}{I}%
    C%
  }%
}
%    \end{macrocode}
%    \end{macro}
%    \begin{macro}{\HoLogoCss@PiC}
%    \begin{macrocode}
\def\HoLogoCss@PiC{%
  \Css{%
    span.HoLogo-PiC span.HoLogo-i{%
      position:relative;%
      top:.5ex;%
      margin-left:-.12em;%
      margin-right:-.075em;%
      text-decoration:none;%
    }%
  }%
  \global\let\HoLogoCss@PiC\relax
}
%    \end{macrocode}
%    \end{macro}
%
%    \begin{macro}{\HoLogo@PiCTeX}
%    \begin{macrocode}
\def\HoLogo@PiCTeX#1{%
  \hologo{PiC}%
  \HOLOGO@discretionary
  \kern-.11em%
  \hologo{TeX}%
}
%    \end{macrocode}
%    \end{macro}
%    \begin{macro}{\HoLogoHtml@PiCTeX}
%    \begin{macrocode}
\def\HoLogoHtml@PiCTeX#1{%
  \HoLogoCss@PiCTeX
  \HOLOGO@Span{PiCTeX}{%
    \hologo{PiC}%
    \hologo{TeX}%
  }%
}
%    \end{macrocode}
%    \end{macro}
%    \begin{macro}{\HoLogoCss@PiCTeX}
%    \begin{macrocode}
\def\HoLogoCss@PiCTeX{%
  \Css{%
    span.HoLogo-PiCTeX span.HoLogo-PiC{%
      margin-right:-.11em;%
    }%
  }%
  \global\let\HoLogoCss@PiCTeX\relax
}
%    \end{macrocode}
%    \end{macro}
%
% \subsubsection{\hologo{teTeX}}
%
%    \begin{macro}{\HoLogo@teTeX}
%    \begin{macrocode}
\def\HoLogo@teTeX#1{%
  \HOLOGO@mbox{#1{t}{T}e}%
  \HOLOGO@discretionary
  \hologo{TeX}%
}
%    \end{macrocode}
%    \end{macro}
%    \begin{macro}{\HoLogoCs@teTeX}
%    \begin{macrocode}
\def\HoLogoCs@teTeX#1{#1{t}{T}dfTeX}
%    \end{macrocode}
%    \end{macro}
%    \begin{macro}{\HoLogoBkm@teTeX}
%    \begin{macrocode}
\def\HoLogoBkm@teTeX#1{%
  #1{t}{T}e\hologo{TeX}%
}
%    \end{macrocode}
%    \end{macro}
%    \begin{macro}{\HoLogoHtml@teTeX}
%    \begin{macrocode}
\let\HoLogoHtml@teTeX\HoLogo@teTeX
%    \end{macrocode}
%    \end{macro}
%
% \subsubsection{\hologo{TeX4ht}}
%
%    \begin{macro}{\HoLogo@TeX4ht}
%    \begin{macrocode}
\expandafter\def\csname HoLogo@TeX4ht\endcsname#1{%
  \HOLOGO@mbox{\hologo{TeX}4ht}%
}
%    \end{macrocode}
%    \end{macro}
%    \begin{macro}{\HoLogoHtml@TeX4ht}
%    \begin{macrocode}
\expandafter
\let\csname HoLogoHtml@TeX4ht\expandafter\endcsname
\csname HoLogo@TeX4ht\endcsname
%    \end{macrocode}
%    \end{macro}
%
%
% \subsubsection{\hologo{SageTeX}}
%
%    \begin{macro}{\HoLogo@SageTeX}
%    \begin{macrocode}
\def\HoLogo@SageTeX#1{%
  \HOLOGO@mbox{Sage}%
  \HOLOGO@discretionary
  \HOLOGO@NegativeKerning{eT,oT,To}%
  \hologo{TeX}%
}
%    \end{macrocode}
%    \end{macro}
%    \begin{macro}{\HoLogoHtml@SageTeX}
%    \begin{macrocode}
\let\HoLogoHtml@SageTeX\HoLogo@SageTeX
%    \end{macrocode}
%    \end{macro}
%
% \subsection{\hologo{METAFONT} and friends}
%
%    \begin{macro}{\HoLogo@METAFONT}
%    \begin{macrocode}
\def\HoLogo@METAFONT#1{%
  \HoLogoFont@font{METAFONT}{logo}{%
    \HOLOGO@mbox{META}%
    \HOLOGO@discretionary
    \HOLOGO@mbox{FONT}%
  }%
}
%    \end{macrocode}
%    \end{macro}
%
%    \begin{macro}{\HoLogo@METAPOST}
%    \begin{macrocode}
\def\HoLogo@METAPOST#1{%
  \HoLogoFont@font{METAPOST}{logo}{%
    \HOLOGO@mbox{META}%
    \HOLOGO@discretionary
    \HOLOGO@mbox{POST}%
  }%
}
%    \end{macrocode}
%    \end{macro}
%
%    \begin{macro}{\HoLogo@MetaFun}
%    \begin{macrocode}
\def\HoLogo@MetaFun#1{%
  \HOLOGO@mbox{Meta}%
  \HOLOGO@discretionary
  \HOLOGO@mbox{Fun}%
}
%    \end{macrocode}
%    \end{macro}
%
%    \begin{macro}{\HoLogo@MetaPost}
%    \begin{macrocode}
\def\HoLogo@MetaPost#1{%
  \HOLOGO@mbox{Meta}%
  \HOLOGO@discretionary
  \HOLOGO@mbox{Post}%
}
%    \end{macrocode}
%    \end{macro}
%
% \subsection{Others}
%
% \subsubsection{\hologo{biber}}
%
%    \begin{macro}{\HoLogo@biber}
%    \begin{macrocode}
\def\HoLogo@biber#1{%
  \HOLOGO@mbox{#1{b}{B}i}%
  \HOLOGO@discretionary
  \HOLOGO@mbox{ber}%
}
%    \end{macrocode}
%    \end{macro}
%    \begin{macro}{\HoLogoCs@biber}
%    \begin{macrocode}
\def\HoLogoCs@biber#1{#1{b}{B}iber}
%    \end{macrocode}
%    \end{macro}
%    \begin{macro}{\HoLogoBkm@biber}
%    \begin{macrocode}
\def\HoLogoBkm@biber#1{%
  #1{b}{B}iber%
}
%    \end{macrocode}
%    \end{macro}
%    \begin{macro}{\HoLogoHtml@biber}
%    \begin{macrocode}
\let\HoLogoHtml@biber\HoLogo@biber
%    \end{macrocode}
%    \end{macro}
%
% \subsubsection{\hologo{KOMAScript}}
%
%    \begin{macro}{\HoLogo@KOMAScript}
%    The definition for \hologo{KOMAScript} is taken
%    from \hologo{KOMAScript} (\xfile{scrlogo.dtx}, reformatted) \cite{scrlogo}:
%\begin{quote}
%\begin{verbatim}
%\@ifundefined{KOMAScript}{%
%  \DeclareRobustCommand{\KOMAScript}{%
%    \textsf{%
%      K\kern.05em O\kern.05emM\kern.05em A%
%      \kern.1em-\kern.1em %
%      Script%
%    }%
%  }%
%}{}
%\end{verbatim}
%\end{quote}
%    \begin{macrocode}
\def\HoLogo@KOMAScript#1{%
  \HoLogoFont@font{KOMAScript}{sf}{%
    \HOLOGO@mbox{%
      K\kern.05em%
      O\kern.05em%
      M\kern.05em%
      A%
    }%
    \kern.1em%
    \HOLOGO@hyphen
    \kern.1em%
    \HOLOGO@mbox{Script}%
  }%
}
%    \end{macrocode}
%    \end{macro}
%    \begin{macro}{\HoLogoBkm@KOMAScript}
%    \begin{macrocode}
\def\HoLogoBkm@KOMAScript#1{%
  KOMA-Script%
}
%    \end{macrocode}
%    \end{macro}
%    \begin{macro}{\HoLogoHtml@KOMAScript}
%    \begin{macrocode}
\def\HoLogoHtml@KOMAScript#1{%
  \HoLogoCss@KOMAScript
  \HoLogoFont@font{KOMAScript}{sf}{%
    \HOLOGO@Span{KOMAScript}{%
      K%
      \HOLOGO@Span{O}{O}%
      M%
      \HOLOGO@Span{A}{A}%
      \HOLOGO@Span{hyphen}{-}%
      Script%
    }%
  }%
}
%    \end{macrocode}
%    \end{macro}
%    \begin{macro}{\HoLogoCss@KOMAScript}
%    \begin{macrocode}
\def\HoLogoCss@KOMAScript{%
  \Css{%
    span.HoLogo-KOMAScript{%
      font-family:sans-serif;%
    }%
  }%
  \Css{%
    span.HoLogo-KOMAScript span.HoLogo-O{%
      padding-left:.05em;%
      padding-right:.05em;%
    }%
  }%
  \Css{%
    span.HoLogo-KOMAScript span.HoLogo-A{%
      padding-left:.05em;%
    }%
  }%
  \Css{%
    span.HoLogo-KOMAScript span.HoLogo-hyphen{%
      padding-left:.1em;%
      padding-right:.1em;%
    }%
  }%
  \global\let\HoLogoCss@KOMAScript\relax
}
%    \end{macrocode}
%    \end{macro}
%
% \subsubsection{\hologo{LyX}}
%
%    \begin{macro}{\HoLogo@LyX}
%    The definition is taken from the documentation source files
%    of \hologo{LyX}, \xfile{Intro.lyx} \cite{LyX}:
%\begin{quote}
%\begin{verbatim}
%\def\LyX{%
%  \texorpdfstring{%
%    L\kern-.1667em\lower.25em\hbox{Y}\kern-.125emX\@%
%  }{%
%    LyX%
%  }%
%}
%\end{verbatim}
%\end{quote}
%    \begin{macrocode}
\def\HoLogo@LyX#1{%
  L%
  \kern-.1667em%
  \lower.25em\hbox{Y}%
  \kern-.125em%
  X%
  \HOLOGO@SpaceFactor
}
%    \end{macrocode}
%    \end{macro}
%    \begin{macro}{\HoLogoHtml@LyX}
%    \begin{macrocode}
\def\HoLogoHtml@LyX#1{%
  \HoLogoCss@LyX
  \HOLOGO@Span{LyX}{%
    L%
    \HOLOGO@Span{y}{Y}%
    X%
  }%
}
%    \end{macrocode}
%    \end{macro}
%    \begin{macro}{\HoLogoCss@LyX}
%    \begin{macrocode}
\def\HoLogoCss@LyX{%
  \Css{%
    span.HoLogo-LyX span.HoLogo-y{%
      position:relative;%
      top:.25em;%
      margin-left:-.1667em;%
      margin-right:-.125em;%
      text-decoration:none;%
    }%
  }%
  \global\let\HoLogoCss@LyX\relax
}
%    \end{macrocode}
%    \end{macro}
%
% \subsubsection{\hologo{NTS}}
%
%    \begin{macro}{\HoLogo@NTS}
%    Definition for \hologo{NTS} can be found in
%    package \xpackage{etex\textunderscore man} for the \hologo{eTeX} manual \cite{etexman}
%    and in package \xpackage{dtklogos} \cite{dtklogos}:
%\begin{quote}
%\begin{verbatim}
%\def\NTS{%
%  \leavevmode
%  \hbox{%
%    $%
%      \cal N%
%      \kern-0.35em%
%      \lower0.5ex\hbox{$\cal T$}%
%      \kern-0.2em%
%      S%
%    $%
%  }%
%}
%\end{verbatim}
%\end{quote}
%    \begin{macrocode}
\def\HoLogo@NTS#1{%
  \HoLogoFont@font{NTS}{sy}{%
    N\/%
    \kern-.35em%
    \lower.5ex\hbox{T\/}%
    \kern-.2em%
    S\/%
  }%
  \HOLOGO@SpaceFactor
}
%    \end{macrocode}
%    \end{macro}
%
% \subsubsection{\Hologo{TTH} (\hologo{TeX} to HTML translator)}
%
%    Source: \url{http://hutchinson.belmont.ma.us/tth/}
%    In the HTML source the second `T' is printed as subscript.
%\begin{quote}
%\begin{verbatim}
%T<sub>T</sub>H
%\end{verbatim}
%\end{quote}
%    \begin{macro}{\HoLogo@TTH}
%    \begin{macrocode}
\def\HoLogo@TTH#1{%
  \ltx@mbox{%
    T\HOLOGO@SubScript{T}H%
  }%
  \HOLOGO@SpaceFactor
}
%    \end{macrocode}
%    \end{macro}
%
%    \begin{macro}{\HoLogoHtml@TTH}
%    \begin{macrocode}
\def\HoLogoHtml@TTH#1{%
  T\HCode{<sub>}T\HCode{</sub>}H%
}
%    \end{macrocode}
%    \end{macro}
%
% \subsubsection{\Hologo{HanTheThanh}}
%
%    Partial source: Package \xpackage{dtklogos}.
%    The double accent is U+1EBF (latin small letter e with circumflex
%    and acute).
%    \begin{macro}{\HoLogo@HanTheThanh}
%    \begin{macrocode}
\def\HoLogo@HanTheThanh#1{%
  \ltx@mbox{H\`an}%
  \HOLOGO@space
  \ltx@mbox{%
    Th%
    \HOLOGO@IfCharExists{"1EBF}{%
      \char"1EBF\relax
    }{%
      \^e\hbox to 0pt{\hss\raise .5ex\hbox{\'{}}}%
    }%
  }%
  \HOLOGO@space
  \ltx@mbox{Th\`anh}%
}
%    \end{macrocode}
%    \end{macro}
%    \begin{macro}{\HoLogoBkm@HanTheThanh}
%    \begin{macrocode}
\def\HoLogoBkm@HanTheThanh#1{%
  H\`an %
  Th\HOLOGO@PdfdocUnicode{\^e}{\9036\277} %
  Th\`anh%
}
%    \end{macrocode}
%    \end{macro}
%    \begin{macro}{\HoLogoHtml@HanTheThanh}
%    \begin{macrocode}
\def\HoLogoHtml@HanTheThanh#1{%
  H\`an %
  Th\HCode{&\ltx@hashchar x1ebf;} %
  Th\`anh%
}
%    \end{macrocode}
%    \end{macro}
%
% \subsection{Driver detection}
%
%    \begin{macrocode}
\HOLOGO@IfExists\InputIfFileExists{%
  \InputIfFileExists{hologo.cfg}{}{}%
}{%
  \ltx@IfUndefined{pdf@filesize}{%
    \def\HOLOGO@InputIfExists{%
      \openin\HOLOGO@temp=hologo.cfg\relax
      \ifeof\HOLOGO@temp
        \closein\HOLOGO@temp
      \else
        \closein\HOLOGO@temp
        \begingroup
          \def\x{LaTeX2e}%
        \expandafter\endgroup
        \ifx\fmtname\x
          \input{hologo.cfg}%
        \else
          \input hologo.cfg\relax
        \fi
      \fi
    }%
    \ltx@IfUndefined{newread}{%
      \chardef\HOLOGO@temp=15 %
      \def\HOLOGO@CheckRead{%
        \ifeof\HOLOGO@temp
          \HOLOGO@InputIfExists
        \else
          \ifcase\HOLOGO@temp
            \@PackageWarningNoLine{hologo}{%
              Configuration file ignored, because\MessageBreak
              a free read register could not be found%
            }%
          \else
            \begingroup
              \count\ltx@cclv=\HOLOGO@temp
              \advance\ltx@cclv by \ltx@minusone
              \edef\x{\endgroup
                \chardef\noexpand\HOLOGO@temp=\the\count\ltx@cclv
                \relax
              }%
            \x
          \fi
        \fi
      }%
    }{%
      \csname newread\endcsname\HOLOGO@temp
      \HOLOGO@InputIfExists
    }%
  }{%
    \edef\HOLOGO@temp{\pdf@filesize{hologo.cfg}}%
    \ifx\HOLOGO@temp\ltx@empty
    \else
      \ifnum\HOLOGO@temp>0 %
        \begingroup
          \def\x{LaTeX2e}%
        \expandafter\endgroup
        \ifx\fmtname\x
          \input{hologo.cfg}%
        \else
          \input hologo.cfg\relax
        \fi
      \else
        \@PackageInfoNoLine{hologo}{%
          Empty configuration file `hologo.cfg' ignored%
        }%
      \fi
    \fi
  }%
}
%    \end{macrocode}
%
%    \begin{macrocode}
\def\HOLOGO@temp#1#2{%
  \kv@define@key{HoLogoDriver}{#1}[]{%
    \begingroup
      \def\HOLOGO@temp{##1}%
      \ltx@onelevel@sanitize\HOLOGO@temp
      \ifx\HOLOGO@temp\ltx@empty
      \else
        \@PackageError{hologo}{%
          Value (\HOLOGO@temp) not permitted for option `#1'%
        }%
        \@ehc
      \fi
    \endgroup
    \def\hologoDriver{#2}%
  }%
}%
\def\HOLOGO@@temp#1#2{%
  \ifx\kv@value\relax
    \HOLOGO@temp{#1}{#1}%
  \else
    \HOLOGO@temp{#1}{#2}%
  \fi
}%
\kv@parse@normalized{%
  pdftex,%
  luatex=pdftex,%
  dvipdfm,%
  dvipdfmx=dvipdfm,%
  dvips,%
  dvipsone=dvips,%
  xdvi=dvips,%
  xetex,%
  vtex,%
}\HOLOGO@@temp
%    \end{macrocode}
%
%    \begin{macrocode}
\kv@define@key{HoLogoDriver}{driverfallback}{%
  \def\HOLOGO@DriverFallback{#1}%
}
%    \end{macrocode}
%
%    \begin{macro}{\HOLOGO@DriverFallback}
%    \begin{macrocode}
\def\HOLOGO@DriverFallback{dvips}
%    \end{macrocode}
%    \end{macro}
%
%    \begin{macro}{\hologoDriverSetup}
%    \begin{macrocode}
\def\hologoDriverSetup{%
  \let\hologoDriver\ltx@undefined
  \HOLOGO@DriverSetup
}
%    \end{macrocode}
%    \end{macro}
%
%    \begin{macro}{\HOLOGO@DriverSetup}
%    \begin{macrocode}
\def\HOLOGO@DriverSetup#1{%
  \kvsetkeys{HoLogoDriver}{#1}%
  \HOLOGO@CheckDriver
  \ltx@ifundefined{hologoDriver}{%
    \begingroup
    \edef\x{\endgroup
      \noexpand\kvsetkeys{HoLogoDriver}{\HOLOGO@DriverFallback}%
    }\x
  }{}%
  \@PackageInfoNoLine{hologo}{Using driver `\hologoDriver'}%
}
%    \end{macrocode}
%    \end{macro}
%
%    \begin{macro}{\HOLOGO@CheckDriver}
%    \begin{macrocode}
\def\HOLOGO@CheckDriver{%
  \ifpdf
    \def\hologoDriver{pdftex}%
    \let\HOLOGO@pdfliteral\pdfliteral
    \ifluatex
      \ifx\pdfextension\@undefined\else
        \protected\def\pdfliteral{\pdfextension literal}%
        \let\HOLOGO@pdfliteral\pdfliteral
      \fi
      \ltx@IfUndefined{HOLOGO@pdfliteral}{%
        \ifnum\luatexversion<36 %
        \else
          \begingroup
            \let\HOLOGO@temp\endgroup
            \ifcase0%
                \directlua{%
                  if tex.enableprimitives then %
                    tex.enableprimitives('HOLOGO@', {'pdfliteral'})%
                  else %
                    tex.print('1')%
                  end%
                }%
                \ifx\HOLOGO@pdfliteral\@undefined 1\fi%
                \relax%
              \endgroup
              \let\HOLOGO@temp\relax
              \global\let\HOLOGO@pdfliteral\HOLOGO@pdfliteral
            \fi%
          \HOLOGO@temp
        \fi
      }{}%
    \fi
    \ltx@IfUndefined{HOLOGO@pdfliteral}{%
      \@PackageWarningNoLine{hologo}{%
        Cannot find \string\pdfliteral
      }%
    }{}%
  \else
    \ifxetex
      \def\hologoDriver{xetex}%
    \else
      \ifvtex
        \def\hologoDriver{vtex}%
      \fi
    \fi
  \fi
}
%    \end{macrocode}
%    \end{macro}
%
%    \begin{macro}{\HOLOGO@WarningUnsupportedDriver}
%    \begin{macrocode}
\def\HOLOGO@WarningUnsupportedDriver#1{%
  \@PackageWarningNoLine{hologo}{%
    Logo `#1' needs driver specific macros,\MessageBreak
    but driver `\hologoDriver' is not supported.\MessageBreak
    Use a different driver or\MessageBreak
    load package `graphics' or `pgf'%
  }%
}
%    \end{macrocode}
%    \end{macro}
%
% \subsubsection{Reflect box macros}
%
%    Skip driver part if not needed.
%    \begin{macrocode}
\ltx@IfUndefined{reflectbox}{}{%
  \ltx@IfUndefined{rotatebox}{}{%
    \HOLOGO@AtEnd
  }%
}
\ltx@IfUndefined{pgftext}{}{%
  \HOLOGO@AtEnd
}
\ltx@IfUndefined{psscalebox}{}{%
  \HOLOGO@AtEnd
}
%    \end{macrocode}
%
%    \begin{macrocode}
\def\HOLOGO@temp{LaTeX2e}
\ifx\fmtname\HOLOGO@temp
  \RequirePackage{kvoptions}[2011/06/30]%
  \ProcessKeyvalOptions{HoLogoDriver}%
\fi
\HOLOGO@DriverSetup{}
%    \end{macrocode}
%
%    \begin{macro}{\HOLOGO@ReflectBox}
%    \begin{macrocode}
\def\HOLOGO@ReflectBox#1{%
  \begingroup
    \setbox\ltx@zero\hbox{\begingroup#1\endgroup}%
    \setbox\ltx@two\hbox{%
      \kern\wd\ltx@zero
      \csname HOLOGO@ScaleBox@\hologoDriver\endcsname{-1}{1}{%
        \hbox to 0pt{\copy\ltx@zero\hss}%
      }%
    }%
    \wd\ltx@two=\wd\ltx@zero
    \box\ltx@two
  \endgroup
}
%    \end{macrocode}
%    \end{macro}
%
%    \begin{macro}{\HOLOGO@PointReflectBox}
%    \begin{macrocode}
\def\HOLOGO@PointReflectBox#1{%
  \begingroup
    \setbox\ltx@zero\hbox{\begingroup#1\endgroup}%
    \setbox\ltx@two\hbox{%
      \kern\wd\ltx@zero
      \raise\ht\ltx@zero\hbox{%
        \csname HOLOGO@ScaleBox@\hologoDriver\endcsname{-1}{-1}{%
          \hbox to 0pt{\copy\ltx@zero\hss}%
        }%
      }%
    }%
    \wd\ltx@two=\wd\ltx@zero
    \box\ltx@two
  \endgroup
}
%    \end{macrocode}
%    \end{macro}
%
%    We must define all variants because of dynamic driver setup.
%    \begin{macrocode}
\def\HOLOGO@temp#1#2{#2}
%    \end{macrocode}
%
%    \begin{macro}{\HOLOGO@ScaleBox@pdftex}
%    \begin{macrocode}
\HOLOGO@temp{pdftex}{%
  \def\HOLOGO@ScaleBox@pdftex#1#2#3{%
    \HOLOGO@pdfliteral{%
      q #1 0 0 #2 0 0 cm%
    }%
    #3%
    \HOLOGO@pdfliteral{%
      Q%
    }%
  }%
}
%    \end{macrocode}
%    \end{macro}
%    \begin{macro}{\HOLOGO@ScaleBox@dvips}
%    \begin{macrocode}
\HOLOGO@temp{dvips}{%
  \def\HOLOGO@ScaleBox@dvips#1#2#3{%
    \special{ps:%
      gsave %
      currentpoint %
      currentpoint translate %
      #1 #2 scale %
      neg exch neg exch translate%
    }%
    #3%
    \special{ps:%
      currentpoint %
      grestore %
      moveto%
    }%
  }%
}
%    \end{macrocode}
%    \end{macro}
%    \begin{macro}{\HOLOGO@ScaleBox@dvipdfm}
%    \begin{macrocode}
\HOLOGO@temp{dvipdfm}{%
  \let\HOLOGO@ScaleBox@dvipdfm\HOLOGO@ScaleBox@dvips
}
%    \end{macrocode}
%    \end{macro}
%    Since \hologo{XeTeX} v0.6.
%    \begin{macro}{\HOLOGO@ScaleBox@xetex}
%    \begin{macrocode}
\HOLOGO@temp{xetex}{%
  \def\HOLOGO@ScaleBox@xetex#1#2#3{%
    \special{x:gsave}%
    \special{x:scale #1 #2}%
    #3%
    \special{x:grestore}%
  }%
}
%    \end{macrocode}
%    \end{macro}
%    \begin{macro}{\HOLOGO@ScaleBox@vtex}
%    \begin{macrocode}
\HOLOGO@temp{vtex}{%
  \def\HOLOGO@ScaleBox@vtex#1#2#3{%
    \special{r(#1,0,0,#2,0,0}%
    #3%
    \special{r)}%
  }%
}
%    \end{macrocode}
%    \end{macro}
%
%    \begin{macrocode}
\HOLOGO@AtEnd%
%</package>
%    \end{macrocode}
%
% \section{Test}
%
% \subsection{Catcode checks for loading}
%
%    \begin{macrocode}
%<*test1>
%    \end{macrocode}
%    \begin{macrocode}
\catcode`\{=1 %
\catcode`\}=2 %
\catcode`\#=6 %
\catcode`\@=11 %
\expandafter\ifx\csname count@\endcsname\relax
  \countdef\count@=255 %
\fi
\expandafter\ifx\csname @gobble\endcsname\relax
  \long\def\@gobble#1{}%
\fi
\expandafter\ifx\csname @firstofone\endcsname\relax
  \long\def\@firstofone#1{#1}%
\fi
\expandafter\ifx\csname loop\endcsname\relax
  \expandafter\@firstofone
\else
  \expandafter\@gobble
\fi
{%
  \def\loop#1\repeat{%
    \def\body{#1}%
    \iterate
  }%
  \def\iterate{%
    \body
      \let\next\iterate
    \else
      \let\next\relax
    \fi
    \next
  }%
  \let\repeat=\fi
}%
\def\RestoreCatcodes{}
\count@=0 %
\loop
  \edef\RestoreCatcodes{%
    \RestoreCatcodes
    \catcode\the\count@=\the\catcode\count@\relax
  }%
\ifnum\count@<255 %
  \advance\count@ 1 %
\repeat

\def\RangeCatcodeInvalid#1#2{%
  \count@=#1\relax
  \loop
    \catcode\count@=15 %
  \ifnum\count@<#2\relax
    \advance\count@ 1 %
  \repeat
}
\def\RangeCatcodeCheck#1#2#3{%
  \count@=#1\relax
  \loop
    \ifnum#3=\catcode\count@
    \else
      \errmessage{%
        Character \the\count@\space
        with wrong catcode \the\catcode\count@\space
        instead of \number#3%
      }%
    \fi
  \ifnum\count@<#2\relax
    \advance\count@ 1 %
  \repeat
}
\def\space{ }
\expandafter\ifx\csname LoadCommand\endcsname\relax
  \def\LoadCommand{\input hologo.sty\relax}%
\fi
\def\Test{%
  \RangeCatcodeInvalid{0}{47}%
  \RangeCatcodeInvalid{58}{64}%
  \RangeCatcodeInvalid{91}{96}%
  \RangeCatcodeInvalid{123}{255}%
  \catcode`\@=12 %
  \catcode`\\=0 %
  \catcode`\%=14 %
  \LoadCommand
  \RangeCatcodeCheck{0}{36}{15}%
  \RangeCatcodeCheck{37}{37}{14}%
  \RangeCatcodeCheck{38}{47}{15}%
  \RangeCatcodeCheck{48}{57}{12}%
  \RangeCatcodeCheck{58}{63}{15}%
  \RangeCatcodeCheck{64}{64}{12}%
  \RangeCatcodeCheck{65}{90}{11}%
  \RangeCatcodeCheck{91}{91}{15}%
  \RangeCatcodeCheck{92}{92}{0}%
  \RangeCatcodeCheck{93}{96}{15}%
  \RangeCatcodeCheck{97}{122}{11}%
  \RangeCatcodeCheck{123}{255}{15}%
  \RestoreCatcodes
}
\Test
\csname @@end\endcsname
\end
%    \end{macrocode}
%    \begin{macrocode}
%</test1>
%    \end{macrocode}
%
% \subsection{Spacefactor}
%
%    The space factor must be 1000 after a logo. If it is greater 1000
%    then the following space is a space after a sentence closing point.
%    If the space factor is smaller 1000 then an immediate following
%    dot is interpreted as abbreviation, not sentence closing point.
%
%    \begin{macrocode}
%<*test-spacefactor>
\NeedsTeXFormat{LaTeX2e}
\documentclass{article}
\usepackage{hologo}[2016/05/12]
\usepackage{kvsetkeys}
\usepackage{qstest}
\IncludeTests{*}
\LogTests{log}{*}{*}
\begin{document}
\begin{qstest}{spacefactor}{spacefactor}
\newcommand*{\Test}[1]{%
  \sbox0{%
    \hologo{#1}%
    \Expect*{1000 (#1)}*{\the\spacefactor\space(#1)}%
  }%
}%
\makeatletter
\def\TestList{}
\def\hologoEntry#1#2#3{%
  \edef\TestList{%
    \ifx\TestList\@empty
    \else
      \TestList,%
    \fi
    #1%
    \ifx\\#2\\%
    \else
      ={variant=#2}%
    \fi
  }%
}
\hologoList
\expandafter\kv@parse@normalized\expandafter{%
  \TestList
}{%
  \begingroup
    \let\@logo=\kv@key
    \ifx\kv@value\relax
    \else
      \expandafter\hologoLogoSetup\expandafter\@logo\expandafter{%
        \kv@value
      }%
    \fi
    \Test\@logo
  \endgroup
  \@gobbletwo
}
\end{qstest}
\end{document}
%</test-spacefactor>
%    \end{macrocode}
%
% \subsection{Complete list}
%
%    \begin{macrocode}
%<*test-list>
\NeedsTeXFormat{LaTeX2e}
\documentclass[12pt,a4paper]{article}
\usepackage{hologo}[2016/05/12]
\usepackage[T1]{fontenc}
\usepackage{lmodern}
\usepackage{parskip}
\usepackage[unicode]{hyperref}[2011/09/28]
\usepackage{bookmark}[2011/09/19]
\bookmarksetup{%
  numbered,%
  open,%
  openlevel=2,%
}
\renewcommand*{\contentsname}{List of logos}
\begin{document}
\tableofcontents
\def\TestFont#1#2#3#4#5#6{%
  \begingroup
    \usefont{#3}{#4}{#5}{#6}%
    \HologoVariant{#1}{#2}/\hologoVariant{#1}{#2}%
    \quad
    \begingroup\scriptsize\hologoVariant{#1}{#2}\endgroup
    \quad
  \endgroup
  (#3/#4/#5/#6)%
  \par
}
\makeatletter
\def\hologoEntry#1#2#3{%
  \section{%
    \HologoVariant{#1}{#2}/\hologoVariant{#1}{#2} %
    {[#1\ifx\\#2\\\else\space(#2)\fi]}% hash-ok
  }% braces around [] because of bug in tex4ht
  \begingroup
    \hypersetup{unicode=false}%
    \bookmark[%
      dest=\@currentHref,%
      rellevel=1,%
      keeplevel,%
    ]{%
      \HologoVariant{#1}{#2}/\hologoVariant{#1}{#2} %
      (PDFDocEncoding)%
    }%
  \endgroup
  \TestFont{#1}{#2}{OT1}{cmr}{m}{n}%
  \TestFont{#1}{#2}{OT1}{cmss}{m}{n}%
  \TestFont{#1}{#2}{OT1}{cmr}{b}{n}%
  \TestFont{#1}{#2}{OT1}{cmr}{m}{it}%
  \TestFont{#1}{#2}{OT1}{cmtt}{m}{n}%
  \TestFont{#1}{#2}{T1}{lmr}{m}{n}%
  \TestFont{#1}{#2}{T1}{lmss}{m}{n}%
  \TestFont{#1}{#2}{T1}{lmr}{b}{n}%
  \TestFont{#1}{#2}{T1}{lmr}{m}{it}%
  \TestFont{#1}{#2}{T1}{lmtt}{m}{n}%
  \TestFont{#1}{#2}{T1}{lmvtt}{m}{n}%
  \TestFont{#1}{#2}{T1}{qtm}{m}{n}%
  \TestFont{#1}{#2}{T1}{qhv}{m}{n}%
  \TestFont{#1}{#2}{T1}{qtm}{b}{n}%
  \TestFont{#1}{#2}{T1}{qtm}{m}{it}%
  \TestFont{#1}{#2}{T1}{qcr}{m}{n}%
  \newpage
}
\makeatother
\hologoList
\end{document}
%</test-list>
%    \end{macrocode}
%
% \section{Installation}
%
% \subsection{Download}
%
% \paragraph{Package.} This package is available on
% CTAN\footnote{\url{ftp://ftp.ctan.org/tex-archive/}}:
% \begin{description}
% \item[\CTAN{macros/latex/contrib/oberdiek/hologo.dtx}] The source file.
% \item[\CTAN{macros/latex/contrib/oberdiek/hologo.pdf}] Documentation.
% \end{description}
%
%
% \paragraph{Bundle.} All the packages of the bundle `oberdiek'
% are also available in a TDS compliant ZIP archive. There
% the packages are already unpacked and the documentation files
% are generated. The files and directories obey the TDS standard.
% \begin{description}
% \item[\CTAN{install/macros/latex/contrib/oberdiek.tds.zip}]
% \end{description}
% \emph{TDS} refers to the standard ``A Directory Structure
% for \TeX\ Files'' (\CTAN{tds/tds.pdf}). Directories
% with \xfile{texmf} in their name are usually organized this way.
%
% \subsection{Bundle installation}
%
% \paragraph{Unpacking.} Unpack the \xfile{oberdiek.tds.zip} in the
% TDS tree (also known as \xfile{texmf} tree) of your choice.
% Example (linux):
% \begin{quote}
%   |unzip oberdiek.tds.zip -d ~/texmf|
% \end{quote}
%
% \paragraph{Script installation.}
% Check the directory \xfile{TDS:scripts/oberdiek/} for
% scripts that need further installation steps.
% Package \xpackage{attachfile2} comes with the Perl script
% \xfile{pdfatfi.pl} that should be installed in such a way
% that it can be called as \texttt{pdfatfi}.
% Example (linux):
% \begin{quote}
%   |chmod +x scripts/oberdiek/pdfatfi.pl|\\
%   |cp scripts/oberdiek/pdfatfi.pl /usr/local/bin/|
% \end{quote}
%
% \subsection{Package installation}
%
% \paragraph{Unpacking.} The \xfile{.dtx} file is a self-extracting
% \docstrip\ archive. The files are extracted by running the
% \xfile{.dtx} through \plainTeX:
% \begin{quote}
%   \verb|tex hologo.dtx|
% \end{quote}
%
% \paragraph{TDS.} Now the different files must be moved into
% the different directories in your installation TDS tree
% (also known as \xfile{texmf} tree):
% \begin{quote}
% \def\t{^^A
% \begin{tabular}{@{}>{\ttfamily}l@{ $\rightarrow$ }>{\ttfamily}l@{}}
%   hologo.sty & tex/generic/oberdiek/hologo.sty\\
%   hologo.pdf & doc/latex/oberdiek/hologo.pdf\\
%   example/hologo-example.tex & doc/latex/oberdiek/example/hologo-example.tex\\
%   test/hologo-test1.tex & doc/latex/oberdiek/test/hologo-test1.tex\\
%   test/hologo-test-spacefactor.tex & doc/latex/oberdiek/test/hologo-test-spacefactor.tex\\
%   test/hologo-test-list.tex & doc/latex/oberdiek/test/hologo-test-list.tex\\
%   hologo.dtx & source/latex/oberdiek/hologo.dtx\\
% \end{tabular}^^A
% }^^A
% \sbox0{\t}^^A
% \ifdim\wd0>\linewidth
%   \begingroup
%     \advance\linewidth by\leftmargin
%     \advance\linewidth by\rightmargin
%   \edef\x{\endgroup
%     \def\noexpand\lw{\the\linewidth}^^A
%   }\x
%   \def\lwbox{^^A
%     \leavevmode
%     \hbox to \linewidth{^^A
%       \kern-\leftmargin\relax
%       \hss
%       \usebox0
%       \hss
%       \kern-\rightmargin\relax
%     }^^A
%   }^^A
%   \ifdim\wd0>\lw
%     \sbox0{\small\t}^^A
%     \ifdim\wd0>\linewidth
%       \ifdim\wd0>\lw
%         \sbox0{\footnotesize\t}^^A
%         \ifdim\wd0>\linewidth
%           \ifdim\wd0>\lw
%             \sbox0{\scriptsize\t}^^A
%             \ifdim\wd0>\linewidth
%               \ifdim\wd0>\lw
%                 \sbox0{\tiny\t}^^A
%                 \ifdim\wd0>\linewidth
%                   \lwbox
%                 \else
%                   \usebox0
%                 \fi
%               \else
%                 \lwbox
%               \fi
%             \else
%               \usebox0
%             \fi
%           \else
%             \lwbox
%           \fi
%         \else
%           \usebox0
%         \fi
%       \else
%         \lwbox
%       \fi
%     \else
%       \usebox0
%     \fi
%   \else
%     \lwbox
%   \fi
% \else
%   \usebox0
% \fi
% \end{quote}
% If you have a \xfile{docstrip.cfg} that configures and enables \docstrip's
% TDS installing feature, then some files can already be in the right
% place, see the documentation of \docstrip.
%
% \subsection{Refresh file name databases}
%
% If your \TeX~distribution
% (\teTeX, \mikTeX, \dots) relies on file name databases, you must refresh
% these. For example, \teTeX\ users run \verb|texhash| or
% \verb|mktexlsr|.
%
% \subsection{Some details for the interested}
%
% \paragraph{Attached source.}
%
% The PDF documentation on CTAN also includes the
% \xfile{.dtx} source file. It can be extracted by
% AcrobatReader 6 or higher. Another option is \textsf{pdftk},
% e.g. unpack the file into the current directory:
% \begin{quote}
%   \verb|pdftk hologo.pdf unpack_files output .|
% \end{quote}
%
% \paragraph{Unpacking with \LaTeX.}
% The \xfile{.dtx} chooses its action depending on the format:
% \begin{description}
% \item[\plainTeX:] Run \docstrip\ and extract the files.
% \item[\LaTeX:] Generate the documentation.
% \end{description}
% If you insist on using \LaTeX\ for \docstrip\ (really,
% \docstrip\ does not need \LaTeX), then inform the autodetect routine
% about your intention:
% \begin{quote}
%   \verb|latex \let\install=y\input{hologo.dtx}|
% \end{quote}
% Do not forget to quote the argument according to the demands
% of your shell.
%
% \paragraph{Generating the documentation.}
% You can use both the \xfile{.dtx} or the \xfile{.drv} to generate
% the documentation. The process can be configured by the
% configuration file \xfile{ltxdoc.cfg}. For instance, put this
% line into this file, if you want to have A4 as paper format:
% \begin{quote}
%   \verb|\PassOptionsToClass{a4paper}{article}|
% \end{quote}
% An example follows how to generate the
% documentation with pdf\LaTeX:
% \begin{quote}
%\begin{verbatim}
%pdflatex hologo.dtx
%makeindex -s gind.ist hologo.idx
%pdflatex hologo.dtx
%makeindex -s gind.ist hologo.idx
%pdflatex hologo.dtx
%\end{verbatim}
% \end{quote}
%
% \section{Catalogue}
%
% The following XML file can be used as source for the
% \href{http://mirror.ctan.org/help/Catalogue/catalogue.html}{\TeX\ Catalogue}.
% The elements \texttt{caption} and \texttt{description} are imported
% from the original XML file from the Catalogue.
% The name of the XML file in the Catalogue is \xfile{hologo.xml}.
%    \begin{macrocode}
%<*catalogue>
<?xml version='1.0' encoding='us-ascii'?>
<!DOCTYPE entry SYSTEM 'catalogue.dtd'>
<entry datestamp='$Date$' modifier='$Author$' id='hologo'>
  <name>hologo</name>
  <caption>A collection of logos with bookmark support.</caption>
  <authorref id='auth:oberdiek'/>
  <copyright owner='Heiko Oberdiek' year='2010-2012'/>
  <license type='lppl1.3'/>
  <version number='1.10'/>
  <description>
    The package defines a single command <tt>\hologo</tt>, whose
    argument is the usual case-confused ASCII version of the logo.
    The command is bookmark-enabled, so that every logo becomes
    available in bookmarks without further work.
    <p/>
    The package is part of the <xref refid='oberdiek'>oberdiek</xref>
    bundle.
  </description>
  <documentation details='Package documentation'
      href='ctan:/macros/latex/contrib/oberdiek/hologo.pdf'/>
  <ctan file='true' path='/macros/latex/contrib/oberdiek/hologo.dtx'/>
  <miktex location='oberdiek'/>
  <texlive location='oberdiek'/>
  <install path='/macros/latex/contrib/oberdiek/oberdiek.tds.zip'/>
</entry>
%</catalogue>
%    \end{macrocode}
%
% \begin{thebibliography}{9}
% \raggedright
%
% \bibitem{btxdoc}
% Oren Patashnik,
% \textit{\hologo{BibTeX}ing},
% 1988-02-08.\\
% \CTAN{biblio/bibtex/base/}
%
% \bibitem{dtklogos}
% Gerd Neugebauer, DANTE,
% \textit{Package \xpackage{dtklogos}},
% 2011-04-25.\\
% \CTAN{usergrps/dante/dtk/dtklogos.sty}
%
% \bibitem{etexman}
% The \hologo{NTS} Team,
% \textit{The \hologo{eTeX} manual},
% 1998-02.\\
% \CTAN{systems/e-tex/v2/doc/}
%
% \bibitem{ExTeX-FAQ}
% The \hologo{ExTeX} group,
% \textit{\hologo{ExTeX}: FAQ -- How is \hologo{ExTeX} typeset?},
% 2007-04-14.\\
% \url{http://www.extex.org/documentation/faq.html}
%
% \bibitem{LyX}
% %@MISC{ LyX,
% %  title = {{LyX 2.0.0 -- The Document Processor [Computer software and manual]}},
% %  author = {{The LyX Team}},
% %  howpublished = {Internet: http://www.lyx.org},
% %  year = {2011-05-08},
% %  note = {Retrieved May 10, 2011, from http://www.lyx.org},
% %  url = {http://www.lyx.org/}
% %}
% The \hologo{LyX} Team,
% \textit{\hologo{LyX} -- The Document Processor},
% 2011-05-08.\\
% \url{http://www.lyx.org/}
%
% \bibitem{OzTeX}
% Andrew Trevorrow,
% \hologo{OzTeX} FAQ: What is the correct way to typeset ``\hologo{OzTeX}''?,
% 2011-09-15 (visited).
% \url{http://www.trevorrow.com/oztex/ozfaq.html#oztex-logo}
%
% \bibitem{PiCTeX}
% Michael Wichura,
% \textit{The \hologo{PiCTeX} macro package},
% 1987-09-21.
% \CTAN{graphics/pictex/}
%
% \bibitem{scrlogo}
% Markus Kohm,
% \textit{\hologo{KOMAScript} Datei \xfile{scrlogo.dtx}},
% 2009-01-30.\\
% \CTAN{install/macros/latex/contrib/komascript.tds.zip}
%
% \end{thebibliography}
%
% \begin{History}
%   \begin{Version}{2010/04/08 v1.0}
%   \item
%     The first version.
%   \end{Version}
%   \begin{Version}{2010/04/16 v1.1}
%   \item
%     \cs{Hologo} added for support of logos at start of a sentence.
%   \item
%     \cs{hologoSetup} and \cs{hologoLogoSetup} added.
%   \item
%     Options \xoption{break}, \xoption{hyphenbreak}, \xoption{spacebreak}
%     added.
%   \item
%     Variant support added by option \xoption{variant}.
%   \end{Version}
%   \begin{Version}{2010/04/24 v1.2}
%   \item
%     \hologo{LaTeX3} added.
%   \item
%     \hologo{VTeX} added.
%   \end{Version}
%   \begin{Version}{2010/11/21 v1.3}
%   \item
%     \hologo{iniTeX}, \hologo{virTeX} added.
%   \end{Version}
%   \begin{Version}{2011/03/25 v1.4}
%   \item
%     \hologo{ConTeXt} with variants added.
%   \item
%     Option \xoption{discretionarybreak} added as refinement for
%     option \xoption{break}.
%   \end{Version}
%   \begin{Version}{2011/04/21 v1.5}
%   \item
%     Wrong TDS directory for test files fixed.
%   \end{Version}
%   \begin{Version}{2011/10/01 v1.6}
%   \item
%     Support for package \xpackage{tex4ht} added.
%   \item
%     Support for \cs{csname} added if \cs{ifincsname} is available.
%   \item
%     New logos:
%     \hologo{(La)TeX},
%     \hologo{biber},
%     \hologo{BibTeX} (\xoption{sc}, \xoption{sf}),
%     \hologo{emTeX},
%     \hologo{ExTeX},
%     \hologo{KOMAScript},
%     \hologo{La},
%     \hologo{LyX},
%     \hologo{MiKTeX},
%     \hologo{NTS},
%     \hologo{OzMF},
%     \hologo{OzMP},
%     \hologo{OzTeX},
%     \hologo{OzTtH},
%     \hologo{PCTeX},
%     \hologo{PiC},
%     \hologo{PiCTeX},
%     \hologo{METAFONT},
%     \hologo{MetaFun},
%     \hologo{METAPOST},
%     \hologo{MetaPost},
%     \hologo{SLiTeX} (\xoption{lift}, \xoption{narrow}, \xoption{simple}),
%     \hologo{SliTeX} (\xoption{narrow}, \xoption{simple}, \xoption{lift}),
%     \hologo{teTeX}.
%   \item
%     Fixes:
%     \hologo{iniTeX},
%     \hologo{pdfLaTeX},
%     \hologo{pdfTeX},
%     \hologo{virTeX}.
%   \item
%     \cs{hologoFontSetup} and \cs{hologoLogoFontSetup} added.
%   \item
%     \cs{hologoVariant} and \cs{HologoVariant} added.
%   \end{Version}
%   \begin{Version}{2011/11/22 v1.7}
%   \item
%     New logos:
%     \hologo{BibTeX8},
%     \hologo{LaTeXML},
%     \hologo{SageTeX},
%     \hologo{TeX4ht},
%     \hologo{TTH}.
%   \item
%     \hologo{Xe} and friends: Driver stuff fixed.
%   \item
%     \hologo{Xe} and friends: Support for italic added.
%   \item
%     \hologo{Xe} and friends: Package support for \xpackage{pgf}
%     and \xpackage{pstricks} added.
%   \end{Version}
%   \begin{Version}{2011/11/29 v1.8}
%   \item
%     New logos:
%     \hologo{HanTheThanh}.
%   \end{Version}
%   \begin{Version}{2011/12/21 v1.9}
%   \item
%     Patch for package \xpackage{ifxetex} added for the case that
%     \cs{newif} is undefined in \hologo{iniTeX}.
%   \item
%     Some fixes for \hologo{iniTeX}.
%   \end{Version}
%   \begin{Version}{2012/04/26 v1.10}
%   \item
%     Fix in bookmark version of logo ``\hologo{HanTheThanh}''.
%   \end{Version}
%   \begin{Version}{2016/05/12 v1.11}
%   \item
%     Update HOLOGO@IfCharExists (previously in texlive)
%   \item define pdfliteral in current luatex.
%   \end{Version}
% \end{History}
%
% \PrintIndex
%
% \Finale
\endinput
|
% \end{quote}
% Do not forget to quote the argument according to the demands
% of your shell.
%
% \paragraph{Generating the documentation.}
% You can use both the \xfile{.dtx} or the \xfile{.drv} to generate
% the documentation. The process can be configured by the
% configuration file \xfile{ltxdoc.cfg}. For instance, put this
% line into this file, if you want to have A4 as paper format:
% \begin{quote}
%   \verb|\PassOptionsToClass{a4paper}{article}|
% \end{quote}
% An example follows how to generate the
% documentation with pdf\LaTeX:
% \begin{quote}
%\begin{verbatim}
%pdflatex hologo.dtx
%makeindex -s gind.ist hologo.idx
%pdflatex hologo.dtx
%makeindex -s gind.ist hologo.idx
%pdflatex hologo.dtx
%\end{verbatim}
% \end{quote}
%
% \section{Catalogue}
%
% The following XML file can be used as source for the
% \href{http://mirror.ctan.org/help/Catalogue/catalogue.html}{\TeX\ Catalogue}.
% The elements \texttt{caption} and \texttt{description} are imported
% from the original XML file from the Catalogue.
% The name of the XML file in the Catalogue is \xfile{hologo.xml}.
%    \begin{macrocode}
%<*catalogue>
<?xml version='1.0' encoding='us-ascii'?>
<!DOCTYPE entry SYSTEM 'catalogue.dtd'>
<entry datestamp='$Date$' modifier='$Author$' id='hologo'>
  <name>hologo</name>
  <caption>A collection of logos with bookmark support.</caption>
  <authorref id='auth:oberdiek'/>
  <copyright owner='Heiko Oberdiek' year='2010-2012'/>
  <license type='lppl1.3'/>
  <version number='1.10'/>
  <description>
    The package defines a single command <tt>\hologo</tt>, whose
    argument is the usual case-confused ASCII version of the logo.
    The command is bookmark-enabled, so that every logo becomes
    available in bookmarks without further work.
    <p/>
    The package is part of the <xref refid='oberdiek'>oberdiek</xref>
    bundle.
  </description>
  <documentation details='Package documentation'
      href='ctan:/macros/latex/contrib/oberdiek/hologo.pdf'/>
  <ctan file='true' path='/macros/latex/contrib/oberdiek/hologo.dtx'/>
  <miktex location='oberdiek'/>
  <texlive location='oberdiek'/>
  <install path='/macros/latex/contrib/oberdiek/oberdiek.tds.zip'/>
</entry>
%</catalogue>
%    \end{macrocode}
%
% \begin{thebibliography}{9}
% \raggedright
%
% \bibitem{btxdoc}
% Oren Patashnik,
% \textit{\hologo{BibTeX}ing},
% 1988-02-08.\\
% \CTAN{biblio/bibtex/base/}
%
% \bibitem{dtklogos}
% Gerd Neugebauer, DANTE,
% \textit{Package \xpackage{dtklogos}},
% 2011-04-25.\\
% \CTAN{usergrps/dante/dtk/dtklogos.sty}
%
% \bibitem{etexman}
% The \hologo{NTS} Team,
% \textit{The \hologo{eTeX} manual},
% 1998-02.\\
% \CTAN{systems/e-tex/v2/doc/}
%
% \bibitem{ExTeX-FAQ}
% The \hologo{ExTeX} group,
% \textit{\hologo{ExTeX}: FAQ -- How is \hologo{ExTeX} typeset?},
% 2007-04-14.\\
% \url{http://www.extex.org/documentation/faq.html}
%
% \bibitem{LyX}
% %@MISC{ LyX,
% %  title = {{LyX 2.0.0 -- The Document Processor [Computer software and manual]}},
% %  author = {{The LyX Team}},
% %  howpublished = {Internet: http://www.lyx.org},
% %  year = {2011-05-08},
% %  note = {Retrieved May 10, 2011, from http://www.lyx.org},
% %  url = {http://www.lyx.org/}
% %}
% The \hologo{LyX} Team,
% \textit{\hologo{LyX} -- The Document Processor},
% 2011-05-08.\\
% \url{http://www.lyx.org/}
%
% \bibitem{OzTeX}
% Andrew Trevorrow,
% \hologo{OzTeX} FAQ: What is the correct way to typeset ``\hologo{OzTeX}''?,
% 2011-09-15 (visited).
% \url{http://www.trevorrow.com/oztex/ozfaq.html#oztex-logo}
%
% \bibitem{PiCTeX}
% Michael Wichura,
% \textit{The \hologo{PiCTeX} macro package},
% 1987-09-21.
% \CTAN{graphics/pictex/}
%
% \bibitem{scrlogo}
% Markus Kohm,
% \textit{\hologo{KOMAScript} Datei \xfile{scrlogo.dtx}},
% 2009-01-30.\\
% \CTAN{install/macros/latex/contrib/komascript.tds.zip}
%
% \end{thebibliography}
%
% \begin{History}
%   \begin{Version}{2010/04/08 v1.0}
%   \item
%     The first version.
%   \end{Version}
%   \begin{Version}{2010/04/16 v1.1}
%   \item
%     \cs{Hologo} added for support of logos at start of a sentence.
%   \item
%     \cs{hologoSetup} and \cs{hologoLogoSetup} added.
%   \item
%     Options \xoption{break}, \xoption{hyphenbreak}, \xoption{spacebreak}
%     added.
%   \item
%     Variant support added by option \xoption{variant}.
%   \end{Version}
%   \begin{Version}{2010/04/24 v1.2}
%   \item
%     \hologo{LaTeX3} added.
%   \item
%     \hologo{VTeX} added.
%   \end{Version}
%   \begin{Version}{2010/11/21 v1.3}
%   \item
%     \hologo{iniTeX}, \hologo{virTeX} added.
%   \end{Version}
%   \begin{Version}{2011/03/25 v1.4}
%   \item
%     \hologo{ConTeXt} with variants added.
%   \item
%     Option \xoption{discretionarybreak} added as refinement for
%     option \xoption{break}.
%   \end{Version}
%   \begin{Version}{2011/04/21 v1.5}
%   \item
%     Wrong TDS directory for test files fixed.
%   \end{Version}
%   \begin{Version}{2011/10/01 v1.6}
%   \item
%     Support for package \xpackage{tex4ht} added.
%   \item
%     Support for \cs{csname} added if \cs{ifincsname} is available.
%   \item
%     New logos:
%     \hologo{(La)TeX},
%     \hologo{biber},
%     \hologo{BibTeX} (\xoption{sc}, \xoption{sf}),
%     \hologo{emTeX},
%     \hologo{ExTeX},
%     \hologo{KOMAScript},
%     \hologo{La},
%     \hologo{LyX},
%     \hologo{MiKTeX},
%     \hologo{NTS},
%     \hologo{OzMF},
%     \hologo{OzMP},
%     \hologo{OzTeX},
%     \hologo{OzTtH},
%     \hologo{PCTeX},
%     \hologo{PiC},
%     \hologo{PiCTeX},
%     \hologo{METAFONT},
%     \hologo{MetaFun},
%     \hologo{METAPOST},
%     \hologo{MetaPost},
%     \hologo{SLiTeX} (\xoption{lift}, \xoption{narrow}, \xoption{simple}),
%     \hologo{SliTeX} (\xoption{narrow}, \xoption{simple}, \xoption{lift}),
%     \hologo{teTeX}.
%   \item
%     Fixes:
%     \hologo{iniTeX},
%     \hologo{pdfLaTeX},
%     \hologo{pdfTeX},
%     \hologo{virTeX}.
%   \item
%     \cs{hologoFontSetup} and \cs{hologoLogoFontSetup} added.
%   \item
%     \cs{hologoVariant} and \cs{HologoVariant} added.
%   \end{Version}
%   \begin{Version}{2011/11/22 v1.7}
%   \item
%     New logos:
%     \hologo{BibTeX8},
%     \hologo{LaTeXML},
%     \hologo{SageTeX},
%     \hologo{TeX4ht},
%     \hologo{TTH}.
%   \item
%     \hologo{Xe} and friends: Driver stuff fixed.
%   \item
%     \hologo{Xe} and friends: Support for italic added.
%   \item
%     \hologo{Xe} and friends: Package support for \xpackage{pgf}
%     and \xpackage{pstricks} added.
%   \end{Version}
%   \begin{Version}{2011/11/29 v1.8}
%   \item
%     New logos:
%     \hologo{HanTheThanh}.
%   \end{Version}
%   \begin{Version}{2011/12/21 v1.9}
%   \item
%     Patch for package \xpackage{ifxetex} added for the case that
%     \cs{newif} is undefined in \hologo{iniTeX}.
%   \item
%     Some fixes for \hologo{iniTeX}.
%   \end{Version}
%   \begin{Version}{2012/04/26 v1.10}
%   \item
%     Fix in bookmark version of logo ``\hologo{HanTheThanh}''.
%   \end{Version}
%   \begin{Version}{2016/05/12 v1.11}
%   \item
%     Update HOLOGO@IfCharExists (previously in texlive)
%   \item define pdfliteral in current luatex.
%   \end{Version}
% \end{History}
%
% \PrintIndex
%
% \Finale
\endinput
|
% \end{quote}
% Do not forget to quote the argument according to the demands
% of your shell.
%
% \paragraph{Generating the documentation.}
% You can use both the \xfile{.dtx} or the \xfile{.drv} to generate
% the documentation. The process can be configured by the
% configuration file \xfile{ltxdoc.cfg}. For instance, put this
% line into this file, if you want to have A4 as paper format:
% \begin{quote}
%   \verb|\PassOptionsToClass{a4paper}{article}|
% \end{quote}
% An example follows how to generate the
% documentation with pdf\LaTeX:
% \begin{quote}
%\begin{verbatim}
%pdflatex hologo.dtx
%makeindex -s gind.ist hologo.idx
%pdflatex hologo.dtx
%makeindex -s gind.ist hologo.idx
%pdflatex hologo.dtx
%\end{verbatim}
% \end{quote}
%
% \section{Catalogue}
%
% The following XML file can be used as source for the
% \href{http://mirror.ctan.org/help/Catalogue/catalogue.html}{\TeX\ Catalogue}.
% The elements \texttt{caption} and \texttt{description} are imported
% from the original XML file from the Catalogue.
% The name of the XML file in the Catalogue is \xfile{hologo.xml}.
%    \begin{macrocode}
%<*catalogue>
<?xml version='1.0' encoding='us-ascii'?>
<!DOCTYPE entry SYSTEM 'catalogue.dtd'>
<entry datestamp='$Date$' modifier='$Author$' id='hologo'>
  <name>hologo</name>
  <caption>A collection of logos with bookmark support.</caption>
  <authorref id='auth:oberdiek'/>
  <copyright owner='Heiko Oberdiek' year='2010-2012'/>
  <license type='lppl1.3'/>
  <version number='1.10'/>
  <description>
    The package defines a single command <tt>\hologo</tt>, whose
    argument is the usual case-confused ASCII version of the logo.
    The command is bookmark-enabled, so that every logo becomes
    available in bookmarks without further work.
    <p/>
    The package is part of the <xref refid='oberdiek'>oberdiek</xref>
    bundle.
  </description>
  <documentation details='Package documentation'
      href='ctan:/macros/latex/contrib/oberdiek/hologo.pdf'/>
  <ctan file='true' path='/macros/latex/contrib/oberdiek/hologo.dtx'/>
  <miktex location='oberdiek'/>
  <texlive location='oberdiek'/>
  <install path='/macros/latex/contrib/oberdiek/oberdiek.tds.zip'/>
</entry>
%</catalogue>
%    \end{macrocode}
%
% \begin{thebibliography}{9}
% \raggedright
%
% \bibitem{btxdoc}
% Oren Patashnik,
% \textit{\hologo{BibTeX}ing},
% 1988-02-08.\\
% \CTAN{biblio/bibtex/base/}
%
% \bibitem{dtklogos}
% Gerd Neugebauer, DANTE,
% \textit{Package \xpackage{dtklogos}},
% 2011-04-25.\\
% \CTAN{usergrps/dante/dtk/dtklogos.sty}
%
% \bibitem{etexman}
% The \hologo{NTS} Team,
% \textit{The \hologo{eTeX} manual},
% 1998-02.\\
% \CTAN{systems/e-tex/v2/doc/}
%
% \bibitem{ExTeX-FAQ}
% The \hologo{ExTeX} group,
% \textit{\hologo{ExTeX}: FAQ -- How is \hologo{ExTeX} typeset?},
% 2007-04-14.\\
% \url{http://www.extex.org/documentation/faq.html}
%
% \bibitem{LyX}
% %@MISC{ LyX,
% %  title = {{LyX 2.0.0 -- The Document Processor [Computer software and manual]}},
% %  author = {{The LyX Team}},
% %  howpublished = {Internet: http://www.lyx.org},
% %  year = {2011-05-08},
% %  note = {Retrieved May 10, 2011, from http://www.lyx.org},
% %  url = {http://www.lyx.org/}
% %}
% The \hologo{LyX} Team,
% \textit{\hologo{LyX} -- The Document Processor},
% 2011-05-08.\\
% \url{http://www.lyx.org/}
%
% \bibitem{OzTeX}
% Andrew Trevorrow,
% \hologo{OzTeX} FAQ: What is the correct way to typeset ``\hologo{OzTeX}''?,
% 2011-09-15 (visited).
% \url{http://www.trevorrow.com/oztex/ozfaq.html#oztex-logo}
%
% \bibitem{PiCTeX}
% Michael Wichura,
% \textit{The \hologo{PiCTeX} macro package},
% 1987-09-21.
% \CTAN{graphics/pictex/}
%
% \bibitem{scrlogo}
% Markus Kohm,
% \textit{\hologo{KOMAScript} Datei \xfile{scrlogo.dtx}},
% 2009-01-30.\\
% \CTAN{install/macros/latex/contrib/komascript.tds.zip}
%
% \end{thebibliography}
%
% \begin{History}
%   \begin{Version}{2010/04/08 v1.0}
%   \item
%     The first version.
%   \end{Version}
%   \begin{Version}{2010/04/16 v1.1}
%   \item
%     \cs{Hologo} added for support of logos at start of a sentence.
%   \item
%     \cs{hologoSetup} and \cs{hologoLogoSetup} added.
%   \item
%     Options \xoption{break}, \xoption{hyphenbreak}, \xoption{spacebreak}
%     added.
%   \item
%     Variant support added by option \xoption{variant}.
%   \end{Version}
%   \begin{Version}{2010/04/24 v1.2}
%   \item
%     \hologo{LaTeX3} added.
%   \item
%     \hologo{VTeX} added.
%   \end{Version}
%   \begin{Version}{2010/11/21 v1.3}
%   \item
%     \hologo{iniTeX}, \hologo{virTeX} added.
%   \end{Version}
%   \begin{Version}{2011/03/25 v1.4}
%   \item
%     \hologo{ConTeXt} with variants added.
%   \item
%     Option \xoption{discretionarybreak} added as refinement for
%     option \xoption{break}.
%   \end{Version}
%   \begin{Version}{2011/04/21 v1.5}
%   \item
%     Wrong TDS directory for test files fixed.
%   \end{Version}
%   \begin{Version}{2011/10/01 v1.6}
%   \item
%     Support for package \xpackage{tex4ht} added.
%   \item
%     Support for \cs{csname} added if \cs{ifincsname} is available.
%   \item
%     New logos:
%     \hologo{(La)TeX},
%     \hologo{biber},
%     \hologo{BibTeX} (\xoption{sc}, \xoption{sf}),
%     \hologo{emTeX},
%     \hologo{ExTeX},
%     \hologo{KOMAScript},
%     \hologo{La},
%     \hologo{LyX},
%     \hologo{MiKTeX},
%     \hologo{NTS},
%     \hologo{OzMF},
%     \hologo{OzMP},
%     \hologo{OzTeX},
%     \hologo{OzTtH},
%     \hologo{PCTeX},
%     \hologo{PiC},
%     \hologo{PiCTeX},
%     \hologo{METAFONT},
%     \hologo{MetaFun},
%     \hologo{METAPOST},
%     \hologo{MetaPost},
%     \hologo{SLiTeX} (\xoption{lift}, \xoption{narrow}, \xoption{simple}),
%     \hologo{SliTeX} (\xoption{narrow}, \xoption{simple}, \xoption{lift}),
%     \hologo{teTeX}.
%   \item
%     Fixes:
%     \hologo{iniTeX},
%     \hologo{pdfLaTeX},
%     \hologo{pdfTeX},
%     \hologo{virTeX}.
%   \item
%     \cs{hologoFontSetup} and \cs{hologoLogoFontSetup} added.
%   \item
%     \cs{hologoVariant} and \cs{HologoVariant} added.
%   \end{Version}
%   \begin{Version}{2011/11/22 v1.7}
%   \item
%     New logos:
%     \hologo{BibTeX8},
%     \hologo{LaTeXML},
%     \hologo{SageTeX},
%     \hologo{TeX4ht},
%     \hologo{TTH}.
%   \item
%     \hologo{Xe} and friends: Driver stuff fixed.
%   \item
%     \hologo{Xe} and friends: Support for italic added.
%   \item
%     \hologo{Xe} and friends: Package support for \xpackage{pgf}
%     and \xpackage{pstricks} added.
%   \end{Version}
%   \begin{Version}{2011/11/29 v1.8}
%   \item
%     New logos:
%     \hologo{HanTheThanh}.
%   \end{Version}
%   \begin{Version}{2011/12/21 v1.9}
%   \item
%     Patch for package \xpackage{ifxetex} added for the case that
%     \cs{newif} is undefined in \hologo{iniTeX}.
%   \item
%     Some fixes for \hologo{iniTeX}.
%   \end{Version}
%   \begin{Version}{2012/04/26 v1.10}
%   \item
%     Fix in bookmark version of logo ``\hologo{HanTheThanh}''.
%   \end{Version}
%   \begin{Version}{2016/05/12 v1.11}
%   \item
%     Update HOLOGO@IfCharExists (previously in texlive)
%   \item define pdfliteral in current luatex.
%   \end{Version}
% \end{History}
%
% \PrintIndex
%
% \Finale
\endinput
%
        \else
          \input hologo.cfg\relax
        \fi
      \else
        \@PackageInfoNoLine{hologo}{%
          Empty configuration file `hologo.cfg' ignored%
        }%
      \fi
    \fi
  }%
}
%    \end{macrocode}
%
%    \begin{macrocode}
\def\HOLOGO@temp#1#2{%
  \kv@define@key{HoLogoDriver}{#1}[]{%
    \begingroup
      \def\HOLOGO@temp{##1}%
      \ltx@onelevel@sanitize\HOLOGO@temp
      \ifx\HOLOGO@temp\ltx@empty
      \else
        \@PackageError{hologo}{%
          Value (\HOLOGO@temp) not permitted for option `#1'%
        }%
        \@ehc
      \fi
    \endgroup
    \def\hologoDriver{#2}%
  }%
}%
\def\HOLOGO@@temp#1#2{%
  \ifx\kv@value\relax
    \HOLOGO@temp{#1}{#1}%
  \else
    \HOLOGO@temp{#1}{#2}%
  \fi
}%
\kv@parse@normalized{%
  pdftex,%
  luatex=pdftex,%
  dvipdfm,%
  dvipdfmx=dvipdfm,%
  dvips,%
  dvipsone=dvips,%
  xdvi=dvips,%
  xetex,%
  vtex,%
}\HOLOGO@@temp
%    \end{macrocode}
%
%    \begin{macrocode}
\kv@define@key{HoLogoDriver}{driverfallback}{%
  \def\HOLOGO@DriverFallback{#1}%
}
%    \end{macrocode}
%
%    \begin{macro}{\HOLOGO@DriverFallback}
%    \begin{macrocode}
\def\HOLOGO@DriverFallback{dvips}
%    \end{macrocode}
%    \end{macro}
%
%    \begin{macro}{\hologoDriverSetup}
%    \begin{macrocode}
\def\hologoDriverSetup{%
  \let\hologoDriver\ltx@undefined
  \HOLOGO@DriverSetup
}
%    \end{macrocode}
%    \end{macro}
%
%    \begin{macro}{\HOLOGO@DriverSetup}
%    \begin{macrocode}
\def\HOLOGO@DriverSetup#1{%
  \kvsetkeys{HoLogoDriver}{#1}%
  \HOLOGO@CheckDriver
  \ltx@ifundefined{hologoDriver}{%
    \begingroup
    \edef\x{\endgroup
      \noexpand\kvsetkeys{HoLogoDriver}{\HOLOGO@DriverFallback}%
    }\x
  }{}%
  \@PackageInfoNoLine{hologo}{Using driver `\hologoDriver'}%
}
%    \end{macrocode}
%    \end{macro}
%
%    \begin{macro}{\HOLOGO@CheckDriver}
%    \begin{macrocode}
\def\HOLOGO@CheckDriver{%
  \ifpdf
    \def\hologoDriver{pdftex}%
    \let\HOLOGO@pdfliteral\pdfliteral
    \ifluatex
      \ifx\pdfextension\@undefined\else
        \protected\def\pdfliteral{\pdfextension literal}%
        \let\HOLOGO@pdfliteral\pdfliteral
      \fi
      \ltx@IfUndefined{HOLOGO@pdfliteral}{%
        \ifnum\luatexversion<36 %
        \else
          \begingroup
            \let\HOLOGO@temp\endgroup
            \ifcase0%
                \directlua{%
                  if tex.enableprimitives then %
                    tex.enableprimitives('HOLOGO@', {'pdfliteral'})%
                  else %
                    tex.print('1')%
                  end%
                }%
                \ifx\HOLOGO@pdfliteral\@undefined 1\fi%
                \relax%
              \endgroup
              \let\HOLOGO@temp\relax
              \global\let\HOLOGO@pdfliteral\HOLOGO@pdfliteral
            \fi%
          \HOLOGO@temp
        \fi
      }{}%
    \fi
    \ltx@IfUndefined{HOLOGO@pdfliteral}{%
      \@PackageWarningNoLine{hologo}{%
        Cannot find \string\pdfliteral
      }%
    }{}%
  \else
    \ifxetex
      \def\hologoDriver{xetex}%
    \else
      \ifvtex
        \def\hologoDriver{vtex}%
      \fi
    \fi
  \fi
}
%    \end{macrocode}
%    \end{macro}
%
%    \begin{macro}{\HOLOGO@WarningUnsupportedDriver}
%    \begin{macrocode}
\def\HOLOGO@WarningUnsupportedDriver#1{%
  \@PackageWarningNoLine{hologo}{%
    Logo `#1' needs driver specific macros,\MessageBreak
    but driver `\hologoDriver' is not supported.\MessageBreak
    Use a different driver or\MessageBreak
    load package `graphics' or `pgf'%
  }%
}
%    \end{macrocode}
%    \end{macro}
%
% \subsubsection{Reflect box macros}
%
%    Skip driver part if not needed.
%    \begin{macrocode}
\ltx@IfUndefined{reflectbox}{}{%
  \ltx@IfUndefined{rotatebox}{}{%
    \HOLOGO@AtEnd
  }%
}
\ltx@IfUndefined{pgftext}{}{%
  \HOLOGO@AtEnd
}
\ltx@IfUndefined{psscalebox}{}{%
  \HOLOGO@AtEnd
}
%    \end{macrocode}
%
%    \begin{macrocode}
\def\HOLOGO@temp{LaTeX2e}
\ifx\fmtname\HOLOGO@temp
  \RequirePackage{kvoptions}[2011/06/30]%
  \ProcessKeyvalOptions{HoLogoDriver}%
\fi
\HOLOGO@DriverSetup{}
%    \end{macrocode}
%
%    \begin{macro}{\HOLOGO@ReflectBox}
%    \begin{macrocode}
\def\HOLOGO@ReflectBox#1{%
  \begingroup
    \setbox\ltx@zero\hbox{\begingroup#1\endgroup}%
    \setbox\ltx@two\hbox{%
      \kern\wd\ltx@zero
      \csname HOLOGO@ScaleBox@\hologoDriver\endcsname{-1}{1}{%
        \hbox to 0pt{\copy\ltx@zero\hss}%
      }%
    }%
    \wd\ltx@two=\wd\ltx@zero
    \box\ltx@two
  \endgroup
}
%    \end{macrocode}
%    \end{macro}
%
%    \begin{macro}{\HOLOGO@PointReflectBox}
%    \begin{macrocode}
\def\HOLOGO@PointReflectBox#1{%
  \begingroup
    \setbox\ltx@zero\hbox{\begingroup#1\endgroup}%
    \setbox\ltx@two\hbox{%
      \kern\wd\ltx@zero
      \raise\ht\ltx@zero\hbox{%
        \csname HOLOGO@ScaleBox@\hologoDriver\endcsname{-1}{-1}{%
          \hbox to 0pt{\copy\ltx@zero\hss}%
        }%
      }%
    }%
    \wd\ltx@two=\wd\ltx@zero
    \box\ltx@two
  \endgroup
}
%    \end{macrocode}
%    \end{macro}
%
%    We must define all variants because of dynamic driver setup.
%    \begin{macrocode}
\def\HOLOGO@temp#1#2{#2}
%    \end{macrocode}
%
%    \begin{macro}{\HOLOGO@ScaleBox@pdftex}
%    \begin{macrocode}
\HOLOGO@temp{pdftex}{%
  \def\HOLOGO@ScaleBox@pdftex#1#2#3{%
    \HOLOGO@pdfliteral{%
      q #1 0 0 #2 0 0 cm%
    }%
    #3%
    \HOLOGO@pdfliteral{%
      Q%
    }%
  }%
}
%    \end{macrocode}
%    \end{macro}
%    \begin{macro}{\HOLOGO@ScaleBox@dvips}
%    \begin{macrocode}
\HOLOGO@temp{dvips}{%
  \def\HOLOGO@ScaleBox@dvips#1#2#3{%
    \special{ps:%
      gsave %
      currentpoint %
      currentpoint translate %
      #1 #2 scale %
      neg exch neg exch translate%
    }%
    #3%
    \special{ps:%
      currentpoint %
      grestore %
      moveto%
    }%
  }%
}
%    \end{macrocode}
%    \end{macro}
%    \begin{macro}{\HOLOGO@ScaleBox@dvipdfm}
%    \begin{macrocode}
\HOLOGO@temp{dvipdfm}{%
  \let\HOLOGO@ScaleBox@dvipdfm\HOLOGO@ScaleBox@dvips
}
%    \end{macrocode}
%    \end{macro}
%    Since \hologo{XeTeX} v0.6.
%    \begin{macro}{\HOLOGO@ScaleBox@xetex}
%    \begin{macrocode}
\HOLOGO@temp{xetex}{%
  \def\HOLOGO@ScaleBox@xetex#1#2#3{%
    \special{x:gsave}%
    \special{x:scale #1 #2}%
    #3%
    \special{x:grestore}%
  }%
}
%    \end{macrocode}
%    \end{macro}
%    \begin{macro}{\HOLOGO@ScaleBox@vtex}
%    \begin{macrocode}
\HOLOGO@temp{vtex}{%
  \def\HOLOGO@ScaleBox@vtex#1#2#3{%
    \special{r(#1,0,0,#2,0,0}%
    #3%
    \special{r)}%
  }%
}
%    \end{macrocode}
%    \end{macro}
%
%    \begin{macrocode}
\HOLOGO@AtEnd%
%</package>
%    \end{macrocode}
%
% \section{Test}
%
% \subsection{Catcode checks for loading}
%
%    \begin{macrocode}
%<*test1>
%    \end{macrocode}
%    \begin{macrocode}
\catcode`\{=1 %
\catcode`\}=2 %
\catcode`\#=6 %
\catcode`\@=11 %
\expandafter\ifx\csname count@\endcsname\relax
  \countdef\count@=255 %
\fi
\expandafter\ifx\csname @gobble\endcsname\relax
  \long\def\@gobble#1{}%
\fi
\expandafter\ifx\csname @firstofone\endcsname\relax
  \long\def\@firstofone#1{#1}%
\fi
\expandafter\ifx\csname loop\endcsname\relax
  \expandafter\@firstofone
\else
  \expandafter\@gobble
\fi
{%
  \def\loop#1\repeat{%
    \def\body{#1}%
    \iterate
  }%
  \def\iterate{%
    \body
      \let\next\iterate
    \else
      \let\next\relax
    \fi
    \next
  }%
  \let\repeat=\fi
}%
\def\RestoreCatcodes{}
\count@=0 %
\loop
  \edef\RestoreCatcodes{%
    \RestoreCatcodes
    \catcode\the\count@=\the\catcode\count@\relax
  }%
\ifnum\count@<255 %
  \advance\count@ 1 %
\repeat

\def\RangeCatcodeInvalid#1#2{%
  \count@=#1\relax
  \loop
    \catcode\count@=15 %
  \ifnum\count@<#2\relax
    \advance\count@ 1 %
  \repeat
}
\def\RangeCatcodeCheck#1#2#3{%
  \count@=#1\relax
  \loop
    \ifnum#3=\catcode\count@
    \else
      \errmessage{%
        Character \the\count@\space
        with wrong catcode \the\catcode\count@\space
        instead of \number#3%
      }%
    \fi
  \ifnum\count@<#2\relax
    \advance\count@ 1 %
  \repeat
}
\def\space{ }
\expandafter\ifx\csname LoadCommand\endcsname\relax
  \def\LoadCommand{\input hologo.sty\relax}%
\fi
\def\Test{%
  \RangeCatcodeInvalid{0}{47}%
  \RangeCatcodeInvalid{58}{64}%
  \RangeCatcodeInvalid{91}{96}%
  \RangeCatcodeInvalid{123}{255}%
  \catcode`\@=12 %
  \catcode`\\=0 %
  \catcode`\%=14 %
  \LoadCommand
  \RangeCatcodeCheck{0}{36}{15}%
  \RangeCatcodeCheck{37}{37}{14}%
  \RangeCatcodeCheck{38}{47}{15}%
  \RangeCatcodeCheck{48}{57}{12}%
  \RangeCatcodeCheck{58}{63}{15}%
  \RangeCatcodeCheck{64}{64}{12}%
  \RangeCatcodeCheck{65}{90}{11}%
  \RangeCatcodeCheck{91}{91}{15}%
  \RangeCatcodeCheck{92}{92}{0}%
  \RangeCatcodeCheck{93}{96}{15}%
  \RangeCatcodeCheck{97}{122}{11}%
  \RangeCatcodeCheck{123}{255}{15}%
  \RestoreCatcodes
}
\Test
\csname @@end\endcsname
\end
%    \end{macrocode}
%    \begin{macrocode}
%</test1>
%    \end{macrocode}
%
% \subsection{Spacefactor}
%
%    The space factor must be 1000 after a logo. If it is greater 1000
%    then the following space is a space after a sentence closing point.
%    If the space factor is smaller 1000 then an immediate following
%    dot is interpreted as abbreviation, not sentence closing point.
%
%    \begin{macrocode}
%<*test-spacefactor>
\NeedsTeXFormat{LaTeX2e}
\documentclass{article}
\usepackage{hologo}[2016/05/12]
\usepackage{kvsetkeys}
\usepackage{qstest}
\IncludeTests{*}
\LogTests{log}{*}{*}
\begin{document}
\begin{qstest}{spacefactor}{spacefactor}
\newcommand*{\Test}[1]{%
  \sbox0{%
    \hologo{#1}%
    \Expect*{1000 (#1)}*{\the\spacefactor\space(#1)}%
  }%
}%
\makeatletter
\def\TestList{}
\def\hologoEntry#1#2#3{%
  \edef\TestList{%
    \ifx\TestList\@empty
    \else
      \TestList,%
    \fi
    #1%
    \ifx\\#2\\%
    \else
      ={variant=#2}%
    \fi
  }%
}
\hologoList
\expandafter\kv@parse@normalized\expandafter{%
  \TestList
}{%
  \begingroup
    \let\@logo=\kv@key
    \ifx\kv@value\relax
    \else
      \expandafter\hologoLogoSetup\expandafter\@logo\expandafter{%
        \kv@value
      }%
    \fi
    \Test\@logo
  \endgroup
  \@gobbletwo
}
\end{qstest}
\end{document}
%</test-spacefactor>
%    \end{macrocode}
%
% \subsection{Complete list}
%
%    \begin{macrocode}
%<*test-list>
\NeedsTeXFormat{LaTeX2e}
\documentclass[12pt,a4paper]{article}
\usepackage{hologo}[2016/05/12]
\usepackage[T1]{fontenc}
\usepackage{lmodern}
\usepackage{parskip}
\usepackage[unicode]{hyperref}[2011/09/28]
\usepackage{bookmark}[2011/09/19]
\bookmarksetup{%
  numbered,%
  open,%
  openlevel=2,%
}
\renewcommand*{\contentsname}{List of logos}
\begin{document}
\tableofcontents
\def\TestFont#1#2#3#4#5#6{%
  \begingroup
    \usefont{#3}{#4}{#5}{#6}%
    \HologoVariant{#1}{#2}/\hologoVariant{#1}{#2}%
    \quad
    \begingroup\scriptsize\hologoVariant{#1}{#2}\endgroup
    \quad
  \endgroup
  (#3/#4/#5/#6)%
  \par
}
\makeatletter
\def\hologoEntry#1#2#3{%
  \section{%
    \HologoVariant{#1}{#2}/\hologoVariant{#1}{#2} %
    {[#1\ifx\\#2\\\else\space(#2)\fi]}% hash-ok
  }% braces around [] because of bug in tex4ht
  \begingroup
    \hypersetup{unicode=false}%
    \bookmark[%
      dest=\@currentHref,%
      rellevel=1,%
      keeplevel,%
    ]{%
      \HologoVariant{#1}{#2}/\hologoVariant{#1}{#2} %
      (PDFDocEncoding)%
    }%
  \endgroup
  \TestFont{#1}{#2}{OT1}{cmr}{m}{n}%
  \TestFont{#1}{#2}{OT1}{cmss}{m}{n}%
  \TestFont{#1}{#2}{OT1}{cmr}{b}{n}%
  \TestFont{#1}{#2}{OT1}{cmr}{m}{it}%
  \TestFont{#1}{#2}{OT1}{cmtt}{m}{n}%
  \TestFont{#1}{#2}{T1}{lmr}{m}{n}%
  \TestFont{#1}{#2}{T1}{lmss}{m}{n}%
  \TestFont{#1}{#2}{T1}{lmr}{b}{n}%
  \TestFont{#1}{#2}{T1}{lmr}{m}{it}%
  \TestFont{#1}{#2}{T1}{lmtt}{m}{n}%
  \TestFont{#1}{#2}{T1}{lmvtt}{m}{n}%
  \TestFont{#1}{#2}{T1}{qtm}{m}{n}%
  \TestFont{#1}{#2}{T1}{qhv}{m}{n}%
  \TestFont{#1}{#2}{T1}{qtm}{b}{n}%
  \TestFont{#1}{#2}{T1}{qtm}{m}{it}%
  \TestFont{#1}{#2}{T1}{qcr}{m}{n}%
  \newpage
}
\makeatother
\hologoList
\end{document}
%</test-list>
%    \end{macrocode}
%
% \section{Installation}
%
% \subsection{Download}
%
% \paragraph{Package.} This package is available on
% CTAN\footnote{\url{ftp://ftp.ctan.org/tex-archive/}}:
% \begin{description}
% \item[\CTAN{macros/latex/contrib/oberdiek/hologo.dtx}] The source file.
% \item[\CTAN{macros/latex/contrib/oberdiek/hologo.pdf}] Documentation.
% \end{description}
%
%
% \paragraph{Bundle.} All the packages of the bundle `oberdiek'
% are also available in a TDS compliant ZIP archive. There
% the packages are already unpacked and the documentation files
% are generated. The files and directories obey the TDS standard.
% \begin{description}
% \item[\CTAN{install/macros/latex/contrib/oberdiek.tds.zip}]
% \end{description}
% \emph{TDS} refers to the standard ``A Directory Structure
% for \TeX\ Files'' (\CTAN{tds/tds.pdf}). Directories
% with \xfile{texmf} in their name are usually organized this way.
%
% \subsection{Bundle installation}
%
% \paragraph{Unpacking.} Unpack the \xfile{oberdiek.tds.zip} in the
% TDS tree (also known as \xfile{texmf} tree) of your choice.
% Example (linux):
% \begin{quote}
%   |unzip oberdiek.tds.zip -d ~/texmf|
% \end{quote}
%
% \paragraph{Script installation.}
% Check the directory \xfile{TDS:scripts/oberdiek/} for
% scripts that need further installation steps.
% Package \xpackage{attachfile2} comes with the Perl script
% \xfile{pdfatfi.pl} that should be installed in such a way
% that it can be called as \texttt{pdfatfi}.
% Example (linux):
% \begin{quote}
%   |chmod +x scripts/oberdiek/pdfatfi.pl|\\
%   |cp scripts/oberdiek/pdfatfi.pl /usr/local/bin/|
% \end{quote}
%
% \subsection{Package installation}
%
% \paragraph{Unpacking.} The \xfile{.dtx} file is a self-extracting
% \docstrip\ archive. The files are extracted by running the
% \xfile{.dtx} through \plainTeX:
% \begin{quote}
%   \verb|tex hologo.dtx|
% \end{quote}
%
% \paragraph{TDS.} Now the different files must be moved into
% the different directories in your installation TDS tree
% (also known as \xfile{texmf} tree):
% \begin{quote}
% \def\t{^^A
% \begin{tabular}{@{}>{\ttfamily}l@{ $\rightarrow$ }>{\ttfamily}l@{}}
%   hologo.sty & tex/generic/oberdiek/hologo.sty\\
%   hologo.pdf & doc/latex/oberdiek/hologo.pdf\\
%   example/hologo-example.tex & doc/latex/oberdiek/example/hologo-example.tex\\
%   test/hologo-test1.tex & doc/latex/oberdiek/test/hologo-test1.tex\\
%   test/hologo-test-spacefactor.tex & doc/latex/oberdiek/test/hologo-test-spacefactor.tex\\
%   test/hologo-test-list.tex & doc/latex/oberdiek/test/hologo-test-list.tex\\
%   hologo.dtx & source/latex/oberdiek/hologo.dtx\\
% \end{tabular}^^A
% }^^A
% \sbox0{\t}^^A
% \ifdim\wd0>\linewidth
%   \begingroup
%     \advance\linewidth by\leftmargin
%     \advance\linewidth by\rightmargin
%   \edef\x{\endgroup
%     \def\noexpand\lw{\the\linewidth}^^A
%   }\x
%   \def\lwbox{^^A
%     \leavevmode
%     \hbox to \linewidth{^^A
%       \kern-\leftmargin\relax
%       \hss
%       \usebox0
%       \hss
%       \kern-\rightmargin\relax
%     }^^A
%   }^^A
%   \ifdim\wd0>\lw
%     \sbox0{\small\t}^^A
%     \ifdim\wd0>\linewidth
%       \ifdim\wd0>\lw
%         \sbox0{\footnotesize\t}^^A
%         \ifdim\wd0>\linewidth
%           \ifdim\wd0>\lw
%             \sbox0{\scriptsize\t}^^A
%             \ifdim\wd0>\linewidth
%               \ifdim\wd0>\lw
%                 \sbox0{\tiny\t}^^A
%                 \ifdim\wd0>\linewidth
%                   \lwbox
%                 \else
%                   \usebox0
%                 \fi
%               \else
%                 \lwbox
%               \fi
%             \else
%               \usebox0
%             \fi
%           \else
%             \lwbox
%           \fi
%         \else
%           \usebox0
%         \fi
%       \else
%         \lwbox
%       \fi
%     \else
%       \usebox0
%     \fi
%   \else
%     \lwbox
%   \fi
% \else
%   \usebox0
% \fi
% \end{quote}
% If you have a \xfile{docstrip.cfg} that configures and enables \docstrip's
% TDS installing feature, then some files can already be in the right
% place, see the documentation of \docstrip.
%
% \subsection{Refresh file name databases}
%
% If your \TeX~distribution
% (\teTeX, \mikTeX, \dots) relies on file name databases, you must refresh
% these. For example, \teTeX\ users run \verb|texhash| or
% \verb|mktexlsr|.
%
% \subsection{Some details for the interested}
%
% \paragraph{Attached source.}
%
% The PDF documentation on CTAN also includes the
% \xfile{.dtx} source file. It can be extracted by
% AcrobatReader 6 or higher. Another option is \textsf{pdftk},
% e.g. unpack the file into the current directory:
% \begin{quote}
%   \verb|pdftk hologo.pdf unpack_files output .|
% \end{quote}
%
% \paragraph{Unpacking with \LaTeX.}
% The \xfile{.dtx} chooses its action depending on the format:
% \begin{description}
% \item[\plainTeX:] Run \docstrip\ and extract the files.
% \item[\LaTeX:] Generate the documentation.
% \end{description}
% If you insist on using \LaTeX\ for \docstrip\ (really,
% \docstrip\ does not need \LaTeX), then inform the autodetect routine
% about your intention:
% \begin{quote}
%   \verb|latex \let\install=y% \iffalse meta-comment
%
% File: hologo.dtx
% Version: 2016/05/12 v1.11
% Info: A logo collection with bookmark support
%
% Copyright (C) 2010-2012 by
%    Heiko Oberdiek <heiko.oberdiek at googlemail.com>
%
% This work may be distributed and/or modified under the
% conditions of the LaTeX Project Public License, either
% version 1.3c of this license or (at your option) any later
% version. This version of this license is in
%    http://www.latex-project.org/lppl/lppl-1-3c.txt
% and the latest version of this license is in
%    http://www.latex-project.org/lppl.txt
% and version 1.3 or later is part of all distributions of
% LaTeX version 2005/12/01 or later.
%
% This work has the LPPL maintenance status "maintained".
%
% This Current Maintainer of this work is Heiko Oberdiek.
%
% The Base Interpreter refers to any `TeX-Format',
% because some files are installed in TDS:tex/generic//.
%
% This work consists of the main source file hologo.dtx
% and the derived files
%    hologo.sty, hologo.pdf, hologo.ins, hologo.drv, hologo-example.tex,
%    hologo-test1.tex, hologo-test-spacefactor.tex,
%    hologo-test-list.tex.
%
% Distribution:
%    CTAN:macros/latex/contrib/oberdiek/hologo.dtx
%    CTAN:macros/latex/contrib/oberdiek/hologo.pdf
%
% Unpacking:
%    (a) If hologo.ins is present:
%           tex hologo.ins
%    (b) Without hologo.ins:
%           tex hologo.dtx
%    (c) If you insist on using LaTeX
%           latex \let\install=y% \iffalse meta-comment
%
% File: hologo.dtx
% Version: 2016/05/12 v1.11
% Info: A logo collection with bookmark support
%
% Copyright (C) 2010-2012 by
%    Heiko Oberdiek <heiko.oberdiek at googlemail.com>
%
% This work may be distributed and/or modified under the
% conditions of the LaTeX Project Public License, either
% version 1.3c of this license or (at your option) any later
% version. This version of this license is in
%    http://www.latex-project.org/lppl/lppl-1-3c.txt
% and the latest version of this license is in
%    http://www.latex-project.org/lppl.txt
% and version 1.3 or later is part of all distributions of
% LaTeX version 2005/12/01 or later.
%
% This work has the LPPL maintenance status "maintained".
%
% This Current Maintainer of this work is Heiko Oberdiek.
%
% The Base Interpreter refers to any `TeX-Format',
% because some files are installed in TDS:tex/generic//.
%
% This work consists of the main source file hologo.dtx
% and the derived files
%    hologo.sty, hologo.pdf, hologo.ins, hologo.drv, hologo-example.tex,
%    hologo-test1.tex, hologo-test-spacefactor.tex,
%    hologo-test-list.tex.
%
% Distribution:
%    CTAN:macros/latex/contrib/oberdiek/hologo.dtx
%    CTAN:macros/latex/contrib/oberdiek/hologo.pdf
%
% Unpacking:
%    (a) If hologo.ins is present:
%           tex hologo.ins
%    (b) Without hologo.ins:
%           tex hologo.dtx
%    (c) If you insist on using LaTeX
%           latex \let\install=y% \iffalse meta-comment
%
% File: hologo.dtx
% Version: 2016/05/12 v1.11
% Info: A logo collection with bookmark support
%
% Copyright (C) 2010-2012 by
%    Heiko Oberdiek <heiko.oberdiek at googlemail.com>
%
% This work may be distributed and/or modified under the
% conditions of the LaTeX Project Public License, either
% version 1.3c of this license or (at your option) any later
% version. This version of this license is in
%    http://www.latex-project.org/lppl/lppl-1-3c.txt
% and the latest version of this license is in
%    http://www.latex-project.org/lppl.txt
% and version 1.3 or later is part of all distributions of
% LaTeX version 2005/12/01 or later.
%
% This work has the LPPL maintenance status "maintained".
%
% This Current Maintainer of this work is Heiko Oberdiek.
%
% The Base Interpreter refers to any `TeX-Format',
% because some files are installed in TDS:tex/generic//.
%
% This work consists of the main source file hologo.dtx
% and the derived files
%    hologo.sty, hologo.pdf, hologo.ins, hologo.drv, hologo-example.tex,
%    hologo-test1.tex, hologo-test-spacefactor.tex,
%    hologo-test-list.tex.
%
% Distribution:
%    CTAN:macros/latex/contrib/oberdiek/hologo.dtx
%    CTAN:macros/latex/contrib/oberdiek/hologo.pdf
%
% Unpacking:
%    (a) If hologo.ins is present:
%           tex hologo.ins
%    (b) Without hologo.ins:
%           tex hologo.dtx
%    (c) If you insist on using LaTeX
%           latex \let\install=y\input{hologo.dtx}
%        (quote the arguments according to the demands of your shell)
%
% Documentation:
%    (a) If hologo.drv is present:
%           latex hologo.drv
%    (b) Without hologo.drv:
%           latex hologo.dtx; ...
%    The class ltxdoc loads the configuration file ltxdoc.cfg
%    if available. Here you can specify further options, e.g.
%    use A4 as paper format:
%       \PassOptionsToClass{a4paper}{article}
%
%    Programm calls to get the documentation (example):
%       pdflatex hologo.dtx
%       makeindex -s gind.ist hologo.idx
%       pdflatex hologo.dtx
%       makeindex -s gind.ist hologo.idx
%       pdflatex hologo.dtx
%
% Installation:
%    TDS:tex/generic/oberdiek/hologo.sty
%    TDS:doc/latex/oberdiek/hologo.pdf
%    TDS:doc/latex/oberdiek/example/hologo-example.tex
%    TDS:doc/latex/oberdiek/test/hologo-test1.tex
%    TDS:doc/latex/oberdiek/test/hologo-test-spacefactor.tex
%    TDS:doc/latex/oberdiek/test/hologo-test-list.tex
%    TDS:source/latex/oberdiek/hologo.dtx
%
%<*ignore>
\begingroup
  \catcode123=1 %
  \catcode125=2 %
  \def\x{LaTeX2e}%
\expandafter\endgroup
\ifcase 0\ifx\install y1\fi\expandafter
         \ifx\csname processbatchFile\endcsname\relax\else1\fi
         \ifx\fmtname\x\else 1\fi\relax
\else\csname fi\endcsname
%</ignore>
%<*install>
\input docstrip.tex
\Msg{************************************************************************}
\Msg{* Installation}
\Msg{* Package: hologo 2016/05/12 v1.11 A logo collection with bookmark support (HO)}
\Msg{************************************************************************}

\keepsilent
\askforoverwritefalse

\let\MetaPrefix\relax
\preamble

This is a generated file.

Project: hologo
Version: 2016/05/12 v1.11

Copyright (C) 2010-2012 by
   Heiko Oberdiek <heiko.oberdiek at googlemail.com>

This work may be distributed and/or modified under the
conditions of the LaTeX Project Public License, either
version 1.3c of this license or (at your option) any later
version. This version of this license is in
   http://www.latex-project.org/lppl/lppl-1-3c.txt
and the latest version of this license is in
   http://www.latex-project.org/lppl.txt
and version 1.3 or later is part of all distributions of
LaTeX version 2005/12/01 or later.

This work has the LPPL maintenance status "maintained".

This Current Maintainer of this work is Heiko Oberdiek.

The Base Interpreter refers to any `TeX-Format',
because some files are installed in TDS:tex/generic//.

This work consists of the main source file hologo.dtx
and the derived files
   hologo.sty, hologo.pdf, hologo.ins, hologo.drv, hologo-example.tex,
   hologo-test1.tex, hologo-test-spacefactor.tex,
   hologo-test-list.tex.

\endpreamble
\let\MetaPrefix\DoubleperCent

\generate{%
  \file{hologo.ins}{\from{hologo.dtx}{install}}%
  \file{hologo.drv}{\from{hologo.dtx}{driver}}%
  \usedir{tex/generic/oberdiek}%
  \file{hologo.sty}{\from{hologo.dtx}{package}}%
  \usedir{doc/latex/oberdiek/example}%
  \file{hologo-example.tex}{\from{hologo.dtx}{example}}%
  \usedir{doc/latex/oberdiek/test}%
  \file{hologo-test1.tex}{\from{hologo.dtx}{test1}}%
  \file{hologo-test-spacefactor.tex}{\from{hologo.dtx}{test-spacefactor}}%
  \file{hologo-test-list.tex}{\from{hologo.dtx}{test-list}}%
  \nopreamble
  \nopostamble
  \usedir{source/latex/oberdiek/catalogue}%
  \file{hologo.xml}{\from{hologo.dtx}{catalogue}}%
}

\catcode32=13\relax% active space
\let =\space%
\Msg{************************************************************************}
\Msg{*}
\Msg{* To finish the installation you have to move the following}
\Msg{* file into a directory searched by TeX:}
\Msg{*}
\Msg{*     hologo.sty}
\Msg{*}
\Msg{* To produce the documentation run the file `hologo.drv'}
\Msg{* through LaTeX.}
\Msg{*}
\Msg{* Happy TeXing!}
\Msg{*}
\Msg{************************************************************************}

\endbatchfile
%</install>
%<*ignore>
\fi
%</ignore>
%<*driver>
\NeedsTeXFormat{LaTeX2e}
\ProvidesFile{hologo.drv}%
  [2016/05/12 v1.11 A logo collection with bookmark support (HO)]%
\documentclass{ltxdoc}
\usepackage{holtxdoc}[2011/11/22]
\usepackage{hologo}[2016/05/12]
\usepackage{longtable}
\usepackage{array}
\usepackage{paralist}
%\usepackage[T1]{fontenc}
%\usepackage{lmodern}
\begin{document}
  \DocInput{hologo.dtx}%
\end{document}
%</driver>
% \fi
%
%
% \CharacterTable
%  {Upper-case    \A\B\C\D\E\F\G\H\I\J\K\L\M\N\O\P\Q\R\S\T\U\V\W\X\Y\Z
%   Lower-case    \a\b\c\d\e\f\g\h\i\j\k\l\m\n\o\p\q\r\s\t\u\v\w\x\y\z
%   Digits        \0\1\2\3\4\5\6\7\8\9
%   Exclamation   \!     Double quote  \"     Hash (number) \#
%   Dollar        \$     Percent       \%     Ampersand     \&
%   Acute accent  \'     Left paren    \(     Right paren   \)
%   Asterisk      \*     Plus          \+     Comma         \,
%   Minus         \-     Point         \.     Solidus       \/
%   Colon         \:     Semicolon     \;     Less than     \<
%   Equals        \=     Greater than  \>     Question mark \?
%   Commercial at \@     Left bracket  \[     Backslash     \\
%   Right bracket \]     Circumflex    \^     Underscore    \_
%   Grave accent  \`     Left brace    \{     Vertical bar  \|
%   Right brace   \}     Tilde         \~}
%
% \GetFileInfo{hologo.drv}
%
% \title{The \xpackage{hologo} package}
% \date{2016/05/12 v1.11}
% \author{Heiko Oberdiek\\\xemail{heiko.oberdiek at googlemail.com}}
%
% \maketitle
%
% \begin{abstract}
% This package starts a collection of logos with support for bookmarks
% strings.
% \end{abstract}
%
% \tableofcontents
%
% \section{Documentation}
%
% \subsection{Logo macros}
%
% \begin{declcs}{hologo} \M{name}
% \end{declcs}
% Macro \cs{hologo} sets the logo with name \meta{name}.
% The following table shows the supported names.
%
% \begingroup
%   \def\hologoEntry#1#2#3{^^A
%     #1&#2&\hologoLogoSetup{#1}{variant=#2}\hologo{#1}&#3\tabularnewline
%   }
%   \begin{longtable}{>{\ttfamily}l>{\ttfamily}lll}
%     \rmfamily\bfseries{name} & \rmfamily\bfseries variant
%     & \bfseries logo & \bfseries since\\
%     \hline
%     \endhead
%     \hologoList
%   \end{longtable}
% \endgroup
%
% \begin{declcs}{Hologo} \M{name}
% \end{declcs}
% Macro \cs{Hologo} starts the logo \meta{name} with an uppercase
% letter. As an exception small greek letters are not converted
% to uppercase. Examples, see \hologo{eTeX} and \hologo{ExTeX}.
%
% \subsection{Setup macros}
%
% The package does not support package options, but the following
% setup macros can be used to set options.
%
% \begin{declcs}{hologoSetup} \M{key value list}
% \end{declcs}
% Macro \cs{hologoSetup} sets global options.
%
% \begin{declcs}{hologoLogoSetup} \M{logo} \M{key value list}
% \end{declcs}
% Some options can also be used to configure a logo.
% These settings take precedence over global option settings.
%
% \subsection{Options}\label{sec:options}
%
% There are boolean and string options:
% \begin{description}
% \item[Boolean option:]
% It takes |true| or |false|
% as value. If the value is omitted, then |true| is used.
% \item[String option:]
% A value must be given as string. (But the string might be empty.)
% \end{description}
% The following options can be used both in \cs{hologoSetup}
% and \cs{hologoLogoSetup}:
% \begin{description}
% \def\entry#1{\item[\xoption{#1}:]}
% \entry{break}
%   enables or disables line breaks inside the logo. This setting is
%   refined by options \xoption{hyphenbreak}, \xoption{spacebreak}
%   or \xoption{discretionarybreak}.
%   Default is |false|.
% \entry{hyphenbreak}
%   enables or disables the line break right after the hyphen character.
% \entry{spacebreak}
%   enables or disables line breaks at space characters.
% \entry{discretionarybreak}
%   enables or disables line breaks at hyphenation points
%   (inserted by \cs{-}).
% \end{description}
% Macro \cs{hologoLogoSetup} also knows:
% \begin{description}
% \item[\xoption{variant}:]
%   This is a string option. It specifies a variant of a logo that
%   must exist. An empty string selects the package default variant.
% \end{description}
% Example:
% \begin{quote}
%   |\hologoSetup{break=false}|\\
%   |\hologoLogoSetup{plainTeX}{variant=hyphen,hyphenbreak}|\\
%   Then ``plain-\TeX'' contains one break point after the hyphen.
% \end{quote}
%
% \subsection{Driver options}
%
% Sometimes graphical operations are needed to construct some
% glyphs (e.g.\ \hologo{XeTeX}). If package \xpackage{graphics}
% or package \xpackage{pgf} are found, then the macros are taken
% from there. Otherwise the packge defines its own operations
% and therefore needs the driver information. Many drivers are
% detected automatically (\hologo{pdfTeX}/\hologo{LuaTeX}
% in PDF mode, \hologo{XeTeX}, \hologo{VTeX}). These have precedence
% over a driver option. The driver can be given as package option
% or using \cs{hologoDriverSetup}.
% The following list contains the recognized driver options:
% \begin{itemize}
% \item \xoption{pdftex}, \xoption{luatex}
% \item \xoption{dvipdfm}, \xoption{dvipdfmx}
% \item \xoption{dvips}, \xoption{dvipsone}, \xoption{xdvi}
% \item \xoption{xetex}
% \item \xoption{vtex}
% \end{itemize}
% The left driver of a line is the driver name that is used internally.
% The following names are aliases for drivers that use the
% same method. Therefore the entry in the \xext{log} file for
% the used driver prints the internally used driver name.
% \begin{description}
% \item[\xoption{driverfallback}:]
%   This option expects a driver that is used,
%   if the driver could not be detected automatically.
% \end{description}
%
% \begin{declcs}{hologoDriverSetup} \M{driver option}
% \end{declcs}
% The driver can also be configured after package loading
% using \cs{hologoDriverSetup}, also the way for \hologo{plainTeX}
% to setup the driver.
%
% \subsection{Font setup}
%
% Some logos require a special font, but should also be usable by
% \hologo{plainTeX}. Therefore the package provides some ways
% to influence the font settings. The options below
% take font settings as values. Both font commands
% such as \cs{sffamily} and macros that take one argument
% like \cs{textsf} can be used.
%
% \begin{declcs}{hologoFontSetup} \M{key value list}
% \end{declcs}
% Macro \cs{hologoFontSetup} sets the fonts for all logos.
% Supported keys:
% \begin{description}
% \def\entry#1{\item[\xoption{#1}:]}
% \entry{general}
%   This font is used for all logos. The default is empty.
%   That means no special font is used.
% \entry{bibsf}
%   This font is used for
%   {\hologoLogoSetup{BibTeX}{variant=sf}\hologo{BibTeX}}
%   with variant \xoption{sf}.
% \entry{rm}
%   This font is a serif font. It is used for \hologo{ExTeX}.
% \entry{sc}
%   This font specifies a small caps font. It is used for
%   {\hologoLogoSetup{BibTeX}{variant=sc}\hologo{BibTeX}}
%   with variant \xoption{sc}.
% \entry{sf}
%   This font specifies a sans serif font. The default
%   is \cs{sffamily}, then \cs{sf} is tried. Otherwise
%   a warning is given. It is used by \hologo{KOMAScript}.
% \entry{sy}
%   This is the font for math symbols (e.g. cmsy).
%   It is used by \hologo{AmS}, \hologo{NTS}, \hologo{ExTeX}.
% \entry{logo}
%   \hologo{METAFONT} and \hologo{METAPOST} are using that font.
%   In \hologo{LaTeX} \cs{logofamily} is used and
%   the definitions of package \xpackage{mflogo} are used
%   if the package is not loaded.
%   Otherwise the \cs{tenlogo} is used and defined
%   if it does not already exists.
% \end{description}
%
% \begin{declcs}{hologoLogoFontSetup} \M{logo} \M{key value list}
% \end{declcs}
% Fonts can also be set for a logo or logo component separately,
% see the following list.
% The keys are the same as for \cs{hologoFontSetup}.
%
% \begin{longtable}{>{\ttfamily}l>{\sffamily}ll}
%   \meta{logo} & keys & result\\
%   \hline
%   \endhead
%   BibTeX & bibsf & {\hologoLogoSetup{BibTeX}{variant=sf}\hologo{BibTeX}}\\[.5ex]
%   BibTeX & sc & {\hologoLogoSetup{BibTeX}{variant=sc}\hologo{BibTeX}}\\[.5ex]
%   ExTeX & rm & \hologo{ExTeX}\\
%   SliTeX & rm & \hologo{SliTeX}\\[.5ex]
%   AmS & sy & \hologo{AmS}\\
%   ExTeX & sy & \hologo{ExTeX}\\
%   NTS & sy & \hologo{NTS}\\[.5ex]
%   KOMAScript & sf & \hologo{KOMAScript}\\[.5ex]
%   METAFONT & logo & \hologo{METAFONT}\\
%   METAPOST & logo & \hologo{METAPOST}\\[.5ex]
%   SliTeX & sc \hologo{SliTeX}
% \end{longtable}
%
% \subsubsection{Font order}
%
% For all logos the font \xoption{general} is applied first.
% Example:
%\begin{quote}
%|\hologoFontSetup{general=\color{red}}|
%\end{quote}
% will print red logos.
% Then if the font uses a special font \xoption{sf}, for example,
% the font is applied that is setup by \cs{hologoLogoFontSetup}.
% If this font is not setup, then the common font setup
% by \cs{hologoFontSetup} is used. Otherwise a warning is given,
% that there is no font configured.
%
% \subsection{Additional user macros}
%
% Usually a variant of a logo is configured by using
% \cs{hologoLogoSetup}, because it is bad style to mix
% different variants of the same logo in the same text.
% There the following macros are a convenience for testing.
%
% \begin{declcs}{hologoVariant} \M{name} \M{variant}\\
%   \cs{HologoVariant} \M{name} \M{variant}
% \end{declcs}
% Logo \meta{name} is set using \meta{variant} that specifies
% explicitely which variant of the macro is used. If the argument
% is empty, then the default form of the logo is used
% (configurable by \cs{hologoLogoSetup}).
%
% \cs{HologoVariant} is used if the logo is set in a context
% that needs an uppercase first letter (beginning of a sentence, \dots).
%
% \begin{declcs}{hologoList}\\
%   \cs{hologoEntry} \M{logo} \M{variant} \M{since}
% \end{declcs}
% Macro \cs{hologoList} contains all logos that are provided
% by the package including variants. The list consists of calls
% of \cs{hologoEntry} with three arguments starting with the
% logo name \meta{logo} and its variant \meta{variant}. An empty
% variant means the current default. Argument \meta{since} specifies
% with version of the package \xpackage{hologo} is needed to get
% the logo. If the logo is fixed, then the date gets updated.
% Therefore the date \meta{since} is not exactly the date of
% the first introduction, but rather the date of the latest fix.
%
% Before \cs{hologoList} can be used, macro \cs{hologoEntry} needs
% a definition. The example file in section \ref{sec:example}
% shows applications of \cs{hologoList}.
%
% \subsection{Supported contexts}
%
% Macros \cs{hologo} and friends support special contexts:
% \begin{itemize}
% \item \hologo{LaTeX}'s protection mechanism.
% \item Bookmarks of package \xpackage{hyperref}.
% \item Package \xpackage{tex4ht}.
% \item The macros can be used inside \cs{csname} constructs,
%   if \cs{ifincsname} is available (\hologo{pdfTeX}, \hologo{XeTeX},
%   \hologo{LuaTeX}).
% \end{itemize}
%
% \subsection{Example}
% \label{sec:example}
%
% The following example prints the logos in different fonts.
%    \begin{macrocode}
%<*example>
%<<verbatim
\NeedsTeXFormat{LaTeX2e}
\documentclass[a4paper]{article}
\usepackage[
  hmargin=20mm,
  vmargin=20mm,
]{geometry}
\pagestyle{empty}
\usepackage{hologo}[2016/05/12]
\usepackage{longtable}
\usepackage{array}
\setlength{\extrarowheight}{2pt}
\usepackage[T1]{fontenc}
\usepackage{lmodern}
\usepackage{pdflscape}
\usepackage[
  pdfencoding=auto,
]{hyperref}
\hypersetup{
  pdfauthor={Heiko Oberdiek},
  pdftitle={Example for package `hologo'},
  pdfsubject={Logos with fonts lmr, lmss, qtm, qpl, qhv},
}
\usepackage{bookmark}

% Print the logo list on the console

\begingroup
  \typeout{}%
  \typeout{*** Begin of logo list ***}%
  \newcommand*{\hologoEntry}[3]{%
    \typeout{#1 \ifx\\#2\\\else(#2) \fi[#3]}%
  }%
  \hologoList
  \typeout{*** End of logo list ***}%
  \typeout{}%
\endgroup

\begin{document}
\begin{landscape}

  \section{Example file for package `hologo'}

  % Table for font names

  \begin{longtable}{>{\bfseries}ll}
    \textbf{font} & \textbf{Font name}\\
    \hline
    lmr & Latin Modern Roman\\
    lmss & Latin Modern Sans\\
    qtm & \TeX\ Gyre Termes\\
    qhv & \TeX\ Gyre Heros\\
    qpl & \TeX\ Gyre Pagella\\
  \end{longtable}

  % Logo list with logos in different fonts

  \begingroup
    \newcommand*{\SetVariant}[2]{%
      \ifx\\#2\\%
      \else
        \hologoLogoSetup{#1}{variant=#2}%
      \fi
    }%
    \newcommand*{\hologoEntry}[3]{%
      \SetVariant{#1}{#2}%
      \raisebox{1em}[0pt][0pt]{\hypertarget{#1@#2}{}}%
      \bookmark[%
        dest={#1@#2},%
      ]{%
        #1\ifx\\#2\\\else\space(#2)\fi: \Hologo{#1}, \hologo{#1} %
        [Unicode]%
      }%
      \hypersetup{unicode=false}%
      \bookmark[%
        dest={#1@#2},%
      ]{%
        #1\ifx\\#2\\\else\space(#2)\fi: \Hologo{#1}, \hologo{#1} %
        [PDFDocEncoding]%
      }%
      \texttt{#1}%
      &%
      \texttt{#2}%
      &%
      \Hologo{#1}%
      &%
      \SetVariant{#1}{#2}%
      \hologo{#1}%
      &%
      \SetVariant{#1}{#2}%
      \fontfamily{qtm}\selectfont
      \hologo{#1}%
      &%
      \SetVariant{#1}{#2}%
      \fontfamily{qpl}\selectfont
      \hologo{#1}%
      &%
      \SetVariant{#1}{#2}%
      \textsf{\hologo{#1}}%
      &%
      \SetVariant{#1}{#2}%
      \fontfamily{qhv}\selectfont
      \hologo{#1}%
      \tabularnewline
    }%
    \begin{longtable}{llllllll}%
      \textbf{\textit{logo}} & \textbf{\textit{variant}} &
      \texttt{\string\Hologo} &
      \textbf{lmr} & \textbf{qtm} & \textbf{qpl} &
      \textbf{lmss} & \textbf{qhv}
      \tabularnewline
      \hline
      \endhead
      \hologoList
    \end{longtable}%
  \endgroup

\end{landscape}
\end{document}
%verbatim
%</example>
%    \end{macrocode}
%
% \StopEventually{
% }
%
% \section{Implementation}
%    \begin{macrocode}
%<*package>
%    \end{macrocode}
%    Reload check, especially if the package is not used with \LaTeX.
%    \begin{macrocode}
\begingroup\catcode61\catcode48\catcode32=10\relax%
  \catcode13=5 % ^^M
  \endlinechar=13 %
  \catcode35=6 % #
  \catcode39=12 % '
  \catcode44=12 % ,
  \catcode45=12 % -
  \catcode46=12 % .
  \catcode58=12 % :
  \catcode64=11 % @
  \catcode123=1 % {
  \catcode125=2 % }
  \expandafter\let\expandafter\x\csname ver@hologo.sty\endcsname
  \ifx\x\relax % plain-TeX, first loading
  \else
    \def\empty{}%
    \ifx\x\empty % LaTeX, first loading,
      % variable is initialized, but \ProvidesPackage not yet seen
    \else
      \expandafter\ifx\csname PackageInfo\endcsname\relax
        \def\x#1#2{%
          \immediate\write-1{Package #1 Info: #2.}%
        }%
      \else
        \def\x#1#2{\PackageInfo{#1}{#2, stopped}}%
      \fi
      \x{hologo}{The package is already loaded}%
      \aftergroup\endinput
    \fi
  \fi
\endgroup%
%    \end{macrocode}
%    Package identification:
%    \begin{macrocode}
\begingroup\catcode61\catcode48\catcode32=10\relax%
  \catcode13=5 % ^^M
  \endlinechar=13 %
  \catcode35=6 % #
  \catcode39=12 % '
  \catcode40=12 % (
  \catcode41=12 % )
  \catcode44=12 % ,
  \catcode45=12 % -
  \catcode46=12 % .
  \catcode47=12 % /
  \catcode58=12 % :
  \catcode64=11 % @
  \catcode91=12 % [
  \catcode93=12 % ]
  \catcode123=1 % {
  \catcode125=2 % }
  \expandafter\ifx\csname ProvidesPackage\endcsname\relax
    \def\x#1#2#3[#4]{\endgroup
      \immediate\write-1{Package: #3 #4}%
      \xdef#1{#4}%
    }%
  \else
    \def\x#1#2[#3]{\endgroup
      #2[{#3}]%
      \ifx#1\@undefined
        \xdef#1{#3}%
      \fi
      \ifx#1\relax
        \xdef#1{#3}%
      \fi
    }%
  \fi
\expandafter\x\csname ver@hologo.sty\endcsname
\ProvidesPackage{hologo}%
  [2016/05/12 v1.11 A logo collection with bookmark support (HO)]%
%    \end{macrocode}
%
%    \begin{macrocode}
\begingroup\catcode61\catcode48\catcode32=10\relax%
  \catcode13=5 % ^^M
  \endlinechar=13 %
  \catcode123=1 % {
  \catcode125=2 % }
  \catcode64=11 % @
  \def\x{\endgroup
    \expandafter\edef\csname HOLOGO@AtEnd\endcsname{%
      \endlinechar=\the\endlinechar\relax
      \catcode13=\the\catcode13\relax
      \catcode32=\the\catcode32\relax
      \catcode35=\the\catcode35\relax
      \catcode61=\the\catcode61\relax
      \catcode64=\the\catcode64\relax
      \catcode123=\the\catcode123\relax
      \catcode125=\the\catcode125\relax
    }%
  }%
\x\catcode61\catcode48\catcode32=10\relax%
\catcode13=5 % ^^M
\endlinechar=13 %
\catcode35=6 % #
\catcode64=11 % @
\catcode123=1 % {
\catcode125=2 % }
\def\TMP@EnsureCode#1#2{%
  \edef\HOLOGO@AtEnd{%
    \HOLOGO@AtEnd
    \catcode#1=\the\catcode#1\relax
  }%
  \catcode#1=#2\relax
}
\TMP@EnsureCode{10}{12}% ^^J
\TMP@EnsureCode{33}{12}% !
\TMP@EnsureCode{34}{12}% "
\TMP@EnsureCode{36}{3}% $
\TMP@EnsureCode{38}{4}% &
\TMP@EnsureCode{39}{12}% '
\TMP@EnsureCode{40}{12}% (
\TMP@EnsureCode{41}{12}% )
\TMP@EnsureCode{42}{12}% *
\TMP@EnsureCode{43}{12}% +
\TMP@EnsureCode{44}{12}% ,
\TMP@EnsureCode{45}{12}% -
\TMP@EnsureCode{46}{12}% .
\TMP@EnsureCode{47}{12}% /
\TMP@EnsureCode{58}{12}% :
\TMP@EnsureCode{59}{12}% ;
\TMP@EnsureCode{60}{12}% <
\TMP@EnsureCode{62}{12}% >
\TMP@EnsureCode{63}{12}% ?
\TMP@EnsureCode{91}{12}% [
\TMP@EnsureCode{93}{12}% ]
\TMP@EnsureCode{94}{7}% ^ (superscript)
\TMP@EnsureCode{95}{8}% _ (subscript)
\TMP@EnsureCode{96}{12}% `
\TMP@EnsureCode{124}{12}% |
\edef\HOLOGO@AtEnd{%
  \HOLOGO@AtEnd
  \escapechar\the\escapechar\relax
  \noexpand\endinput
}
\escapechar=92 %
%    \end{macrocode}
%
% \subsection{Logo list}
%
%    \begin{macro}{\hologoList}
%    \begin{macrocode}
\def\hologoList{%
  \hologoEntry{(La)TeX}{}{2011/10/01}%
  \hologoEntry{AmSLaTeX}{}{2010/04/16}%
  \hologoEntry{AmSTeX}{}{2010/04/16}%
  \hologoEntry{biber}{}{2011/10/01}%
  \hologoEntry{BibTeX}{}{2011/10/01}%
  \hologoEntry{BibTeX}{sf}{2011/10/01}%
  \hologoEntry{BibTeX}{sc}{2011/10/01}%
  \hologoEntry{BibTeX8}{}{2011/11/22}%
  \hologoEntry{ConTeXt}{}{2011/03/25}%
  \hologoEntry{ConTeXt}{narrow}{2011/03/25}%
  \hologoEntry{ConTeXt}{simple}{2011/03/25}%
  \hologoEntry{emTeX}{}{2010/04/26}%
  \hologoEntry{eTeX}{}{2010/04/08}%
  \hologoEntry{ExTeX}{}{2011/10/01}%
  \hologoEntry{HanTheThanh}{}{2011/11/29}%
  \hologoEntry{iniTeX}{}{2011/10/01}%
  \hologoEntry{KOMAScript}{}{2011/10/01}%
  \hologoEntry{La}{}{2010/05/08}%
  \hologoEntry{LaTeX}{}{2010/04/08}%
  \hologoEntry{LaTeX2e}{}{2010/04/08}%
  \hologoEntry{LaTeX3}{}{2010/04/24}%
  \hologoEntry{LaTeXe}{}{2010/04/08}%
  \hologoEntry{LaTeXML}{}{2011/11/22}%
  \hologoEntry{LaTeXTeX}{}{2011/10/01}%
  \hologoEntry{LuaLaTeX}{}{2010/04/08}%
  \hologoEntry{LuaTeX}{}{2010/04/08}%
  \hologoEntry{LyX}{}{2011/10/01}%
  \hologoEntry{METAFONT}{}{2011/10/01}%
  \hologoEntry{MetaFun}{}{2011/10/01}%
  \hologoEntry{METAPOST}{}{2011/10/01}%
  \hologoEntry{MetaPost}{}{2011/10/01}%
  \hologoEntry{MiKTeX}{}{2011/10/01}%
  \hologoEntry{NTS}{}{2011/10/01}%
  \hologoEntry{OzMF}{}{2011/10/01}%
  \hologoEntry{OzMP}{}{2011/10/01}%
  \hologoEntry{OzTeX}{}{2011/10/01}%
  \hologoEntry{OzTtH}{}{2011/10/01}%
  \hologoEntry{PCTeX}{}{2011/10/01}%
  \hologoEntry{pdfTeX}{}{2011/10/01}%
  \hologoEntry{pdfLaTeX}{}{2011/10/01}%
  \hologoEntry{PiC}{}{2011/10/01}%
  \hologoEntry{PiCTeX}{}{2011/10/01}%
  \hologoEntry{plainTeX}{}{2010/04/08}%
  \hologoEntry{plainTeX}{space}{2010/04/16}%
  \hologoEntry{plainTeX}{hyphen}{2010/04/16}%
  \hologoEntry{plainTeX}{runtogether}{2010/04/16}%
  \hologoEntry{SageTeX}{}{2011/11/22}%
  \hologoEntry{SLiTeX}{}{2011/10/01}%
  \hologoEntry{SLiTeX}{lift}{2011/10/01}%
  \hologoEntry{SLiTeX}{narrow}{2011/10/01}%
  \hologoEntry{SLiTeX}{simple}{2011/10/01}%
  \hologoEntry{SliTeX}{}{2011/10/01}%
  \hologoEntry{SliTeX}{narrow}{2011/10/01}%
  \hologoEntry{SliTeX}{simple}{2011/10/01}%
  \hologoEntry{SliTeX}{lift}{2011/10/01}%
  \hologoEntry{teTeX}{}{2011/10/01}%
  \hologoEntry{TeX}{}{2010/04/08}%
  \hologoEntry{TeX4ht}{}{2011/11/22}%
  \hologoEntry{TTH}{}{2011/11/22}%
  \hologoEntry{virTeX}{}{2011/10/01}%
  \hologoEntry{VTeX}{}{2010/04/24}%
  \hologoEntry{Xe}{}{2010/04/08}%
  \hologoEntry{XeLaTeX}{}{2010/04/08}%
  \hologoEntry{XeTeX}{}{2010/04/08}%
}
%    \end{macrocode}
%    \end{macro}
%
% \subsection{Load resources}
%
%    \begin{macrocode}
\begingroup\expandafter\expandafter\expandafter\endgroup
\expandafter\ifx\csname RequirePackage\endcsname\relax
  \def\TMP@RequirePackage#1[#2]{%
    \begingroup\expandafter\expandafter\expandafter\endgroup
    \expandafter\ifx\csname ver@#1.sty\endcsname\relax
      \input #1.sty\relax
    \fi
  }%
  \TMP@RequirePackage{ltxcmds}[2011/02/04]%
  \TMP@RequirePackage{infwarerr}[2010/04/08]%
  \TMP@RequirePackage{kvsetkeys}[2010/03/01]%
  \TMP@RequirePackage{kvdefinekeys}[2010/03/01]%
  \TMP@RequirePackage{pdftexcmds}[2010/04/01]%
  \TMP@RequirePackage{ifpdf}[2010/01/28]%
  \TMP@RequirePackage{ifluatex}[2010/03/01]%
  \ltx@IfUndefined{newif}{%
    \expandafter\let\csname newif\endcsname\ltx@newif
  }{}%
  \TMP@RequirePackage{ifxetex}[2009/01/23]%
  \TMP@RequirePackage{ifvtex}[2010/03/01]%
\else
  \RequirePackage{ltxcmds}[2011/02/04]%
  \RequirePackage{infwarerr}[2010/04/08]%
  \RequirePackage{kvsetkeys}[2010/03/01]%
  \RequirePackage{kvdefinekeys}[2010/03/01]%
  \RequirePackage{pdftexcmds}[2010/04/01]%
  \RequirePackage{ifpdf}[2010/01/28]%
  \RequirePackage{ifluatex}[2010/03/01]%
  \RequirePackage{ifxetex}[2009/01/23]%
  \RequirePackage{ifvtex}[2010/03/01]%
\fi
%    \end{macrocode}
%
%    \begin{macro}{\HOLOGO@IfDefined}
%    \begin{macrocode}
\def\HOLOGO@IfExists#1{%
  \ifx\@undefined#1%
    \expandafter\ltx@secondoftwo
  \else
    \ifx\relax#1%
      \expandafter\ltx@secondoftwo
    \else
      \expandafter\expandafter\expandafter\ltx@firstoftwo
    \fi
  \fi
}
%    \end{macrocode}
%    \end{macro}
%
% \subsection{Setup macros}
%
%    \begin{macro}{\hologoSetup}
%    \begin{macrocode}
\def\hologoSetup{%
  \let\HOLOGO@name\relax
  \HOLOGO@Setup
}
%    \end{macrocode}
%    \end{macro}
%
%    \begin{macro}{\hologoLogoSetup}
%    \begin{macrocode}
\def\hologoLogoSetup#1{%
  \edef\HOLOGO@name{#1}%
  \ltx@IfUndefined{HoLogo@\HOLOGO@name}{%
    \@PackageError{hologo}{%
      Unknown logo `\HOLOGO@name'%
    }\@ehc
    \ltx@gobble
  }{%
    \HOLOGO@Setup
  }%
}
%    \end{macrocode}
%    \end{macro}
%
%    \begin{macro}{\HOLOGO@Setup}
%    \begin{macrocode}
\def\HOLOGO@Setup{%
  \kvsetkeys{HoLogo}%
}
%    \end{macrocode}
%    \end{macro}
%
% \subsection{Options}
%
%    \begin{macro}{\HOLOGO@DeclareBoolOption}
%    \begin{macrocode}
\def\HOLOGO@DeclareBoolOption#1{%
  \expandafter\chardef\csname HOLOGOOPT@#1\endcsname\ltx@zero
  \kv@define@key{HoLogo}{#1}[true]{%
    \def\HOLOGO@temp{##1}%
    \ifx\HOLOGO@temp\HOLOGO@true
      \ifx\HOLOGO@name\relax
        \expandafter\chardef\csname HOLOGOOPT@#1\endcsname=\ltx@one
      \else
        \expandafter\chardef\csname
        HoLogoOpt@#1@\HOLOGO@name\endcsname\ltx@one
      \fi
      \HOLOGO@SetBreakAll{#1}%
    \else
      \ifx\HOLOGO@temp\HOLOGO@false
        \ifx\HOLOGO@name\relax
          \expandafter\chardef\csname HOLOGOOPT@#1\endcsname=\ltx@zero
        \else
          \expandafter\chardef\csname
          HoLogoOpt@#1@\HOLOGO@name\endcsname=\ltx@zero
        \fi
        \HOLOGO@SetBreakAll{#1}%
      \else
        \@PackageError{hologo}{%
          Unknown value `##1' for boolean option `#1'.\MessageBreak
          Known values are `true' and `false'%
        }\@ehc
      \fi
    \fi
  }%
}
%    \end{macrocode}
%    \end{macro}
%
%    \begin{macro}{\HOLOGO@SetBreakAll}
%    \begin{macrocode}
\def\HOLOGO@SetBreakAll#1{%
  \def\HOLOGO@temp{#1}%
  \ifx\HOLOGO@temp\HOLOGO@break
    \ifx\HOLOGO@name\relax
      \chardef\HOLOGOOPT@hyphenbreak=\HOLOGOOPT@break
      \chardef\HOLOGOOPT@spacebreak=\HOLOGOOPT@break
      \chardef\HOLOGOOPT@discretionarybreak=\HOLOGOOPT@break
    \else
      \expandafter\chardef
         \csname HoLogoOpt@hyphenbreak@\HOLOGO@name\endcsname=%
         \csname HoLogoOpt@break@\HOLOGO@name\endcsname
      \expandafter\chardef
         \csname HoLogoOpt@spacebreak@\HOLOGO@name\endcsname=%
         \csname HoLogoOpt@break@\HOLOGO@name\endcsname
      \expandafter\chardef
         \csname HoLogoOpt@discretionarybreak@\HOLOGO@name
             \endcsname=%
         \csname HoLogoOpt@break@\HOLOGO@name\endcsname
    \fi
  \fi
}
%    \end{macrocode}
%    \end{macro}
%
%    \begin{macro}{\HOLOGO@true}
%    \begin{macrocode}
\def\HOLOGO@true{true}
%    \end{macrocode}
%    \end{macro}
%    \begin{macro}{\HOLOGO@false}
%    \begin{macrocode}
\def\HOLOGO@false{false}
%    \end{macrocode}
%    \end{macro}
%    \begin{macro}{\HOLOGO@break}
%    \begin{macrocode}
\def\HOLOGO@break{break}
%    \end{macrocode}
%    \end{macro}
%
%    \begin{macrocode}
\HOLOGO@DeclareBoolOption{break}
\HOLOGO@DeclareBoolOption{hyphenbreak}
\HOLOGO@DeclareBoolOption{spacebreak}
\HOLOGO@DeclareBoolOption{discretionarybreak}
%    \end{macrocode}
%
%    \begin{macrocode}
\kv@define@key{HoLogo}{variant}{%
  \ifx\HOLOGO@name\relax
    \@PackageError{hologo}{%
      Option `variant' is not available in \string\hologoSetup,%
      \MessageBreak
      Use \string\hologoLogoSetup\space instead%
    }\@ehc
  \else
    \edef\HOLOGO@temp{#1}%
    \ifx\HOLOGO@temp\ltx@empty
      \expandafter
      \let\csname HoLogoOpt@variant@\HOLOGO@name\endcsname\@undefined
    \else
      \ltx@IfUndefined{HoLogo@\HOLOGO@name @\HOLOGO@temp}{%
        \@PackageError{hologo}{%
          Unknown variant `\HOLOGO@temp' of logo `\HOLOGO@name'%
        }\@ehc
      }{%
        \expandafter
        \let\csname HoLogoOpt@variant@\HOLOGO@name\endcsname
            \HOLOGO@temp
      }%
    \fi
  \fi
}
%    \end{macrocode}
%
%    \begin{macro}{\HOLOGO@Variant}
%    \begin{macrocode}
\def\HOLOGO@Variant#1{%
  #1%
  \ltx@ifundefined{HoLogoOpt@variant@#1}{%
  }{%
    @\csname HoLogoOpt@variant@#1\endcsname
  }%
}
%    \end{macrocode}
%    \end{macro}
%
% \subsection{Break/no-break support}
%
%    \begin{macro}{\HOLOGO@space}
%    \begin{macrocode}
\def\HOLOGO@space{%
  \ltx@ifundefined{HoLogoOpt@spacebreak@\HOLOGO@name}{%
    \ltx@ifundefined{HoLogoOpt@break@\HOLOGO@name}{%
      \chardef\HOLOGO@temp=\HOLOGOOPT@spacebreak
    }{%
      \chardef\HOLOGO@temp=%
        \csname HoLogoOpt@break@\HOLOGO@name\endcsname
    }%
  }{%
    \chardef\HOLOGO@temp=%
      \csname HoLogoOpt@spacebreak@\HOLOGO@name\endcsname
  }%
  \ifcase\HOLOGO@temp
    \penalty10000 %
  \fi
  \ltx@space
}
%    \end{macrocode}
%    \end{macro}
%
%    \begin{macro}{\HOLOGO@hyphen}
%    \begin{macrocode}
\def\HOLOGO@hyphen{%
  \ltx@ifundefined{HoLogoOpt@hyphenbreak@\HOLOGO@name}{%
    \ltx@ifundefined{HoLogoOpt@break@\HOLOGO@name}{%
      \chardef\HOLOGO@temp=\HOLOGOOPT@hyphenbreak
    }{%
      \chardef\HOLOGO@temp=%
        \csname HoLogoOpt@break@\HOLOGO@name\endcsname
    }%
  }{%
    \chardef\HOLOGO@temp=%
      \csname HoLogoOpt@hyphenbreak@\HOLOGO@name\endcsname
  }%
  \ifcase\HOLOGO@temp
    \ltx@mbox{-}%
  \else
    -%
  \fi
}
%    \end{macrocode}
%    \end{macro}
%
%    \begin{macro}{\HOLOGO@discretionary}
%    \begin{macrocode}
\def\HOLOGO@discretionary{%
  \ltx@ifundefined{HoLogoOpt@discretionarybreak@\HOLOGO@name}{%
    \ltx@ifundefined{HoLogoOpt@break@\HOLOGO@name}{%
      \chardef\HOLOGO@temp=\HOLOGOOPT@discretionarybreak
    }{%
      \chardef\HOLOGO@temp=%
        \csname HoLogoOpt@break@\HOLOGO@name\endcsname
    }%
  }{%
    \chardef\HOLOGO@temp=%
      \csname HoLogoOpt@discretionarybreak@\HOLOGO@name\endcsname
  }%
  \ifcase\HOLOGO@temp
  \else
    \-%
  \fi
}
%    \end{macrocode}
%    \end{macro}
%
%    \begin{macro}{\HOLOGO@mbox}
%    \begin{macrocode}
\def\HOLOGO@mbox#1{%
  \ltx@ifundefined{HoLogoOpt@break@\HOLOGO@name}{%
    \chardef\HOLOGO@temp=\HOLOGOOPT@hyphenbreak
  }{%
    \chardef\HOLOGO@temp=%
      \csname HoLogoOpt@break@\HOLOGO@name\endcsname
  }%
  \ifcase\HOLOGO@temp
    \ltx@mbox{#1}%
  \else
    #1%
  \fi
}
%    \end{macrocode}
%    \end{macro}
%
% \subsection{Font support}
%
%    \begin{macro}{\HoLogoFont@font}
%    \begin{tabular}{@{}ll@{}}
%    |#1|:& logo name\\
%    |#2|:& font short name\\
%    |#3|:& text
%    \end{tabular}
%    \begin{macrocode}
\def\HoLogoFont@font#1#2#3{%
  \begingroup
    \ltx@IfUndefined{HoLogoFont@logo@#1.#2}{%
      \ltx@IfUndefined{HoLogoFont@font@#2}{%
        \@PackageWarning{hologo}{%
          Missing font `#2' for logo `#1'%
        }%
        #3%
      }{%
        \csname HoLogoFont@font@#2\endcsname{#3}%
      }%
    }{%
      \csname HoLogoFont@logo@#1.#2\endcsname{#3}%
    }%
  \endgroup
}
%    \end{macrocode}
%    \end{macro}
%
%    \begin{macro}{\HoLogoFont@Def}
%    \begin{macrocode}
\def\HoLogoFont@Def#1{%
  \expandafter\def\csname HoLogoFont@font@#1\endcsname
}
%    \end{macrocode}
%    \end{macro}
%    \begin{macro}{\HoLogoFont@LogoDef}
%    \begin{macrocode}
\def\HoLogoFont@LogoDef#1#2{%
  \expandafter\def\csname HoLogoFont@logo@#1.#2\endcsname
}
%    \end{macrocode}
%    \end{macro}
%
% \subsubsection{Font defaults}
%
%    \begin{macro}{\HoLogoFont@font@general}
%    \begin{macrocode}
\HoLogoFont@Def{general}{}%
%    \end{macrocode}
%    \end{macro}
%
%    \begin{macro}{\HoLogoFont@font@rm}
%    \begin{macrocode}
\ltx@IfUndefined{rmfamily}{%
  \ltx@IfUndefined{rm}{%
  }{%
    \HoLogoFont@Def{rm}{\rm}%
  }%
}{%
  \HoLogoFont@Def{rm}{\rmfamily}%
}
%    \end{macrocode}
%    \end{macro}
%
%    \begin{macro}{\HoLogoFont@font@sf}
%    \begin{macrocode}
\ltx@IfUndefined{sffamily}{%
  \ltx@IfUndefined{sf}{%
  }{%
    \HoLogoFont@Def{sf}{\sf}%
  }%
}{%
  \HoLogoFont@Def{sf}{\sffamily}%
}
%    \end{macrocode}
%    \end{macro}
%
%    \begin{macro}{\HoLogoFont@font@bibsf}
%    In case of \hologo{plainTeX} the original small caps
%    variant is used as default. In \hologo{LaTeX}
%    the definition of package \xpackage{dtklogos} \cite{dtklogos}
%    is used.
%\begin{quote}
%\begin{verbatim}
%\DeclareRobustCommand{\BibTeX}{%
%  B%
%  \kern-.05em%
%  \hbox{%
%    $\m@th$% %% force math size calculations
%    \csname S@\f@size\endcsname
%    \fontsize\sf@size\z@
%    \math@fontsfalse
%    \selectfont
%    I%
%    \kern-.025em%
%    B
%  }%
%  \kern-.08em%
%  \-%
%  \TeX
%}
%\end{verbatim}
%\end{quote}
%    \begin{macrocode}
\ltx@IfUndefined{selectfont}{%
  \ltx@IfUndefined{tensc}{%
    \font\tensc=cmcsc10\relax
  }{}%
  \HoLogoFont@Def{bibsf}{\tensc}%
}{%
  \HoLogoFont@Def{bibsf}{%
    $\mathsurround=0pt$%
    \csname S@\f@size\endcsname
    \fontsize\sf@size{0pt}%
    \math@fontsfalse
    \selectfont
  }%
}
%    \end{macrocode}
%    \end{macro}
%
%    \begin{macro}{\HoLogoFont@font@sc}
%    \begin{macrocode}
\ltx@IfUndefined{scshape}{%
  \ltx@IfUndefined{tensc}{%
    \font\tensc=cmcsc10\relax
  }{}%
  \HoLogoFont@Def{sc}{\tensc}%
}{%
  \HoLogoFont@Def{sc}{\scshape}%
}
%    \end{macrocode}
%    \end{macro}
%
%    \begin{macro}{\HoLogoFont@font@sy}
%    \begin{macrocode}
\ltx@IfUndefined{usefont}{%
  \ltx@IfUndefined{tensy}{%
  }{%
    \HoLogoFont@Def{sy}{\tensy}%
  }%
}{%
  \HoLogoFont@Def{sy}{%
    \usefont{OMS}{cmsy}{m}{n}%
  }%
}
%    \end{macrocode}
%    \end{macro}
%
%    \begin{macro}{\HoLogoFont@font@logo}
%    \begin{macrocode}
\begingroup
  \def\x{LaTeX2e}%
\expandafter\endgroup
\ifx\fmtname\x
  \ltx@IfUndefined{logofamily}{%
    \DeclareRobustCommand\logofamily{%
      \not@math@alphabet\logofamily\relax
      \fontencoding{U}%
      \fontfamily{logo}%
      \selectfont
    }%
  }{}%
  \ltx@IfUndefined{logofamily}{%
  }{%
    \HoLogoFont@Def{logo}{\logofamily}%
  }%
\else
  \ltx@IfUndefined{tenlogo}{%
    \font\tenlogo=logo10\relax
  }{}%
  \HoLogoFont@Def{logo}{\tenlogo}%
\fi
%    \end{macrocode}
%    \end{macro}
%
% \subsubsection{Font setup}
%
%    \begin{macro}{\hologoFontSetup}
%    \begin{macrocode}
\def\hologoFontSetup{%
  \let\HOLOGO@name\relax
  \HOLOGO@FontSetup
}
%    \end{macrocode}
%    \end{macro}
%
%    \begin{macro}{\hologoLogoFontSetup}
%    \begin{macrocode}
\def\hologoLogoFontSetup#1{%
  \edef\HOLOGO@name{#1}%
  \ltx@IfUndefined{HoLogo@\HOLOGO@name}{%
    \@PackageError{hologo}{%
      Unknown logo `\HOLOGO@name'%
    }\@ehc
    \ltx@gobble
  }{%
    \HOLOGO@FontSetup
  }%
}
%    \end{macrocode}
%    \end{macro}
%
%    \begin{macro}{\HOLOGO@FontSetup}
%    \begin{macrocode}
\def\HOLOGO@FontSetup{%
  \kvsetkeys{HoLogoFont}%
}
%    \end{macrocode}
%    \end{macro}
%
%    \begin{macrocode}
\def\HOLOGO@temp#1{%
  \kv@define@key{HoLogoFont}{#1}{%
    \ifx\HOLOGO@name\relax
      \HoLogoFont@Def{#1}{##1}%
    \else
      \HoLogoFont@LogoDef\HOLOGO@name{#1}{##1}%
    \fi
  }%
}
\HOLOGO@temp{general}
\HOLOGO@temp{sf}
%    \end{macrocode}
%
% \subsection{Generic logo commands}
%
%    \begin{macrocode}
\HOLOGO@IfExists\hologo{%
  \@PackageError{hologo}{%
    \string\hologo\ltx@space is already defined.\MessageBreak
    Package loading is aborted%
  }\@ehc
  \HOLOGO@AtEnd
}%
\HOLOGO@IfExists\hologoRobust{%
  \@PackageError{hologo}{%
    \string\hologoRobust\ltx@space is already defined.\MessageBreak
    Package loading is aborted%
  }\@ehc
  \HOLOGO@AtEnd
}%
%    \end{macrocode}
%
% \subsubsection{\cs{hologo} and friends}
%
%    \begin{macrocode}
\ifluatex
  \expandafter\ltx@firstofone
\else
  \expandafter\ltx@gobble
\fi
{%
  \ltx@IfUndefined{ifincsname}{%
    \ifnum\luatexversion<36 %
      \expandafter\ltx@gobble
    \else
      \expandafter\ltx@firstofone
    \fi
    {%
      \begingroup
        \ifcase0%
            \directlua{%
              if tex.enableprimitives then %
                tex.enableprimitives('HOLOGO@', {'ifincsname'})%
              else %
                tex.print('1')%
              end%
            }%
            \ifx\HOLOGO@ifincsname\@undefined 1\fi%
            \relax
          \expandafter\ltx@firstofone
        \else
          \endgroup
          \expandafter\ltx@gobble
        \fi
        {%
          \global\let\ifincsname\HOLOGO@ifincsname
        }%
      \HOLOGO@temp
    }%
  }{}%
}
%    \end{macrocode}
%    \begin{macrocode}
\ltx@IfUndefined{ifincsname}{%
  \catcode`$=14 %
}{%
  \catcode`$=9 %
}
%    \end{macrocode}
%
%    \begin{macro}{\hologo}
%    \begin{macrocode}
\def\hologo#1{%
$ \ifincsname
$   \ltx@ifundefined{HoLogoCs@\HOLOGO@Variant{#1}}{%
$     #1%
$   }{%
$     \csname HoLogoCs@\HOLOGO@Variant{#1}\endcsname\ltx@firstoftwo
$   }%
$ \else
    \HOLOGO@IfExists\texorpdfstring\texorpdfstring\ltx@firstoftwo
    {%
      \hologoRobust{#1}%
    }{%
      \ltx@ifundefined{HoLogoBkm@\HOLOGO@Variant{#1}}{%
        \ltx@ifundefined{HoLogo@#1}{?#1?}{#1}%
      }{%
        \csname HoLogoBkm@\HOLOGO@Variant{#1}\endcsname
        \ltx@firstoftwo
      }%
    }%
$ \fi
}
%    \end{macrocode}
%    \end{macro}
%    \begin{macro}{\Hologo}
%    \begin{macrocode}
\def\Hologo#1{%
$ \ifincsname
$   \ltx@ifundefined{HoLogoCs@\HOLOGO@Variant{#1}}{%
$     #1%
$   }{%
$     \csname HoLogoCs@\HOLOGO@Variant{#1}\endcsname\ltx@secondoftwo
$   }%
$ \else
    \HOLOGO@IfExists\texorpdfstring\texorpdfstring\ltx@firstoftwo
    {%
      \HologoRobust{#1}%
    }{%
      \ltx@ifundefined{HoLogoBkm@\HOLOGO@Variant{#1}}{%
        \ltx@ifundefined{HoLogo@#1}{?#1?}{#1}%
      }{%
        \csname HoLogoBkm@\HOLOGO@Variant{#1}\endcsname
        \ltx@secondoftwo
      }%
    }%
$ \fi
}
%    \end{macrocode}
%    \end{macro}
%
%    \begin{macro}{\hologoVariant}
%    \begin{macrocode}
\def\hologoVariant#1#2{%
  \ifx\relax#2\relax
    \hologo{#1}%
  \else
$   \ifincsname
$     \ltx@ifundefined{HoLogoCs@#1@#2}{%
$       #1%
$     }{%
$       \csname HoLogoCs@#1@#2\endcsname\ltx@firstoftwo
$     }%
$   \else
      \HOLOGO@IfExists\texorpdfstring\texorpdfstring\ltx@firstoftwo
      {%
        \hologoVariantRobust{#1}{#2}%
      }{%
        \ltx@ifundefined{HoLogoBkm@#1@#2}{%
          \ltx@ifundefined{HoLogo@#1}{?#1?}{#1}%
        }{%
          \csname HoLogoBkm@#1@#2\endcsname
          \ltx@firstoftwo
        }%
      }%
$   \fi
  \fi
}
%    \end{macrocode}
%    \end{macro}
%    \begin{macro}{\HologoVariant}
%    \begin{macrocode}
\def\HologoVariant#1#2{%
  \ifx\relax#2\relax
    \Hologo{#1}%
  \else
$   \ifincsname
$     \ltx@ifundefined{HoLogoCs@#1@#2}{%
$       #1%
$     }{%
$       \csname HoLogoCs@#1@#2\endcsname\ltx@secondoftwo
$     }%
$   \else
      \HOLOGO@IfExists\texorpdfstring\texorpdfstring\ltx@firstoftwo
      {%
        \HologoVariantRobust{#1}{#2}%
      }{%
        \ltx@ifundefined{HoLogoBkm@#1@#2}{%
          \ltx@ifundefined{HoLogo@#1}{?#1?}{#1}%
        }{%
          \csname HoLogoBkm@#1@#2\endcsname
          \ltx@secondoftwo
        }%
      }%
$   \fi
  \fi
}
%    \end{macrocode}
%    \end{macro}
%
%    \begin{macrocode}
\catcode`\$=3 %
%    \end{macrocode}
%
% \subsubsection{\cs{hologoRobust} and friends}
%
%    \begin{macro}{\hologoRobust}
%    \begin{macrocode}
\ltx@IfUndefined{protected}{%
  \ltx@IfUndefined{DeclareRobustCommand}{%
    \def\hologoRobust#1%
  }{%
    \DeclareRobustCommand*\hologoRobust[1]%
  }%
}{%
  \protected\def\hologoRobust#1%
}%
{%
  \edef\HOLOGO@name{#1}%
  \ltx@IfUndefined{HoLogo@\HOLOGO@Variant\HOLOGO@name}{%
    \@PackageError{hologo}{%
      Unknown logo `\HOLOGO@name'%
    }\@ehc
    ?\HOLOGO@name?%
  }{%
    \ltx@IfUndefined{ver@tex4ht.sty}{%
      \HoLogoFont@font\HOLOGO@name{general}{%
        \csname HoLogo@\HOLOGO@Variant\HOLOGO@name\endcsname
        \ltx@firstoftwo
      }%
    }{%
      \ltx@IfUndefined{HoLogoHtml@\HOLOGO@Variant\HOLOGO@name}{%
        \HOLOGO@name
      }{%
        \csname HoLogoHtml@\HOLOGO@Variant\HOLOGO@name\endcsname
        \ltx@firstoftwo
      }%
    }%
  }%
}
%    \end{macrocode}
%    \end{macro}
%    \begin{macro}{\HologoRobust}
%    \begin{macrocode}
\ltx@IfUndefined{protected}{%
  \ltx@IfUndefined{DeclareRobustCommand}{%
    \def\HologoRobust#1%
  }{%
    \DeclareRobustCommand*\HologoRobust[1]%
  }%
}{%
  \protected\def\HologoRobust#1%
}%
{%
  \edef\HOLOGO@name{#1}%
  \ltx@IfUndefined{HoLogo@\HOLOGO@Variant\HOLOGO@name}{%
    \@PackageError{hologo}{%
      Unknown logo `\HOLOGO@name'%
    }\@ehc
    ?\HOLOGO@name?%
  }{%
    \ltx@IfUndefined{ver@tex4ht.sty}{%
      \HoLogoFont@font\HOLOGO@name{general}{%
        \csname HoLogo@\HOLOGO@Variant\HOLOGO@name\endcsname
        \ltx@secondoftwo
      }%
    }{%
      \ltx@IfUndefined{HoLogoHtml@\HOLOGO@Variant\HOLOGO@name}{%
        \expandafter\HOLOGO@Uppercase\HOLOGO@name
      }{%
        \csname HoLogoHtml@\HOLOGO@Variant\HOLOGO@name\endcsname
        \ltx@secondoftwo
      }%
    }%
  }%
}
%    \end{macrocode}
%    \end{macro}
%    \begin{macro}{\hologoVariantRobust}
%    \begin{macrocode}
\ltx@IfUndefined{protected}{%
  \ltx@IfUndefined{DeclareRobustCommand}{%
    \def\hologoVariantRobust#1#2%
  }{%
    \DeclareRobustCommand*\hologoVariantRobust[2]%
  }%
}{%
  \protected\def\hologoVariantRobust#1#2%
}%
{%
  \begingroup
    \hologoLogoSetup{#1}{variant={#2}}%
    \hologoRobust{#1}%
  \endgroup
}
%    \end{macrocode}
%    \end{macro}
%    \begin{macro}{\HologoVariantRobust}
%    \begin{macrocode}
\ltx@IfUndefined{protected}{%
  \ltx@IfUndefined{DeclareRobustCommand}{%
    \def\HologoVariantRobust#1#2%
  }{%
    \DeclareRobustCommand*\HologoVariantRobust[2]%
  }%
}{%
  \protected\def\HologoVariantRobust#1#2%
}%
{%
  \begingroup
    \hologoLogoSetup{#1}{variant={#2}}%
    \HologoRobust{#1}%
  \endgroup
}
%    \end{macrocode}
%    \end{macro}
%
%    \begin{macro}{\hologorobust}
%    Macro \cs{hologorobust} is only defined for compatibility.
%    Its use is deprecated.
%    \begin{macrocode}
\def\hologorobust{\hologoRobust}
%    \end{macrocode}
%    \end{macro}
%
% \subsection{Helpers}
%
%    \begin{macro}{\HOLOGO@Uppercase}
%    Macro \cs{HOLOGO@Uppercase} is restricted to \cs{uppercase},
%    because \hologo{plainTeX} or \hologo{iniTeX} do not provide
%    \cs{MakeUppercase}.
%    \begin{macrocode}
\def\HOLOGO@Uppercase#1{\uppercase{#1}}
%    \end{macrocode}
%    \end{macro}
%
%    \begin{macro}{\HOLOGO@PdfdocUnicode}
%    \begin{macrocode}
\def\HOLOGO@PdfdocUnicode{%
  \ifx\ifHy@unicode\iftrue
    \expandafter\ltx@secondoftwo
  \else
    \expandafter\ltx@firstoftwo
  \fi
}
%    \end{macrocode}
%    \end{macro}
%
%    \begin{macro}{\HOLOGO@Math}
%    \begin{macrocode}
\def\HOLOGO@MathSetup{%
  \mathsurround0pt\relax
  \HOLOGO@IfExists\f@series{%
    \if b\expandafter\ltx@car\f@series x\@nil
      \csname boldmath\endcsname
   \fi
  }{}%
}
%    \end{macrocode}
%    \end{macro}
%
%    \begin{macro}{\HOLOGO@TempDimen}
%    \begin{macrocode}
\dimendef\HOLOGO@TempDimen=\ltx@zero
%    \end{macrocode}
%    \end{macro}
%    \begin{macro}{\HOLOGO@NegativeKerning}
%    \begin{macrocode}
\def\HOLOGO@NegativeKerning#1{%
  \begingroup
    \HOLOGO@TempDimen=0pt\relax
    \comma@parse@normalized{#1}{%
      \ifdim\HOLOGO@TempDimen=0pt %
        \expandafter\HOLOGO@@NegativeKerning\comma@entry
      \fi
      \ltx@gobble
    }%
    \ifdim\HOLOGO@TempDimen<0pt %
      \kern\HOLOGO@TempDimen
    \fi
  \endgroup
}
%    \end{macrocode}
%    \end{macro}
%    \begin{macro}{\HOLOGO@@NegativeKerning}
%    \begin{macrocode}
\def\HOLOGO@@NegativeKerning#1#2{%
  \setbox\ltx@zero\hbox{#1#2}%
  \HOLOGO@TempDimen=\wd\ltx@zero
  \setbox\ltx@zero\hbox{#1\kern0pt#2}%
  \advance\HOLOGO@TempDimen by -\wd\ltx@zero
}
%    \end{macrocode}
%    \end{macro}
%
%    \begin{macro}{\HOLOGO@SpaceFactor}
%    \begin{macrocode}
\def\HOLOGO@SpaceFactor{%
  \spacefactor1000 %
}
%    \end{macrocode}
%    \end{macro}
%
%    \begin{macro}{\HOLOGO@Span}
%    \begin{macrocode}
\def\HOLOGO@Span#1#2{%
  \HCode{<span class="HoLogo-#1">}%
  #2%
  \HCode{</span>}%
}
%    \end{macrocode}
%    \end{macro}
%
% \subsubsection{Text subscript}
%
%    \begin{macro}{\HOLOGO@SubScript}%
%    \begin{macrocode}
\def\HOLOGO@SubScript#1{%
  \ltx@IfUndefined{textsubscript}{%
    \ltx@IfUndefined{text}{%
      \ltx@mbox{%
        \mathsurround=0pt\relax
        $%
          _{%
            \ltx@IfUndefined{sf@size}{%
              \mathrm{#1}%
            }{%
              \mbox{%
                \fontsize\sf@size{0pt}\selectfont
                #1%
              }%
            }%
          }%
        $%
      }%
    }{%
      \ltx@mbox{%
        \mathsurround=0pt\relax
        $_{\text{#1}}$%
      }%
    }%
  }{%
    \textsubscript{#1}%
  }%
}
%    \end{macrocode}
%    \end{macro}
%
% \subsection{\hologo{TeX} and friends}
%
% \subsubsection{\hologo{TeX}}
%
%    \begin{macro}{\HoLogo@TeX}
%    Source: \hologo{LaTeX} kernel.
%    \begin{macrocode}
\def\HoLogo@TeX#1{%
  T\kern-.1667em\lower.5ex\hbox{E}\kern-.125emX\HOLOGO@SpaceFactor
}
%    \end{macrocode}
%    \end{macro}
%    \begin{macro}{\HoLogoHtml@TeX}
%    \begin{macrocode}
\def\HoLogoHtml@TeX#1{%
  \HoLogoCss@TeX
  \HOLOGO@Span{TeX}{%
    T%
    \HOLOGO@Span{e}{%
      E%
    }%
    X%
  }%
}
%    \end{macrocode}
%    \end{macro}
%    \begin{macro}{\HoLogoCss@TeX}
%    \begin{macrocode}
\def\HoLogoCss@TeX{%
  \Css{%
    span.HoLogo-TeX span.HoLogo-e{%
      position:relative;%
      top:.5ex;%
      margin-left:-.1667em;%
      margin-right:-.125em;%
    }%
  }%
  \Css{%
    a span.HoLogo-TeX span.HoLogo-e{%
      text-decoration:none;%
    }%
  }%
  \global\let\HoLogoCss@TeX\relax
}
%    \end{macrocode}
%    \end{macro}
%
% \subsubsection{\hologo{plainTeX}}
%
%    \begin{macro}{\HoLogo@plainTeX@space}
%    Source: ``The \hologo{TeX}book''
%    \begin{macrocode}
\def\HoLogo@plainTeX@space#1{%
  \HOLOGO@mbox{#1{p}{P}lain}\HOLOGO@space\hologo{TeX}%
}
%    \end{macrocode}
%    \end{macro}
%    \begin{macro}{\HoLogoCs@plainTeX@space}
%    \begin{macrocode}
\def\HoLogoCs@plainTeX@space#1{#1{p}{P}lain TeX}%
%    \end{macrocode}
%    \end{macro}
%    \begin{macro}{\HoLogoBkm@plainTeX@space}
%    \begin{macrocode}
\def\HoLogoBkm@plainTeX@space#1{%
  #1{p}{P}lain \hologo{TeX}%
}
%    \end{macrocode}
%    \end{macro}
%    \begin{macro}{\HoLogoHtml@plainTeX@space}
%    \begin{macrocode}
\def\HoLogoHtml@plainTeX@space#1{%
  #1{p}{P}lain \hologo{TeX}%
}
%    \end{macrocode}
%    \end{macro}
%
%    \begin{macro}{\HoLogo@plainTeX@hyphen}
%    \begin{macrocode}
\def\HoLogo@plainTeX@hyphen#1{%
  \HOLOGO@mbox{#1{p}{P}lain}\HOLOGO@hyphen\hologo{TeX}%
}
%    \end{macrocode}
%    \end{macro}
%    \begin{macro}{\HoLogoCs@plainTeX@hyphen}
%    \begin{macrocode}
\def\HoLogoCs@plainTeX@hyphen#1{#1{p}{P}lain-TeX}
%    \end{macrocode}
%    \end{macro}
%    \begin{macro}{\HoLogoBkm@plainTeX@hyphen}
%    \begin{macrocode}
\def\HoLogoBkm@plainTeX@hyphen#1{%
  #1{p}{P}lain-\hologo{TeX}%
}
%    \end{macrocode}
%    \end{macro}
%    \begin{macro}{\HoLogoHtml@plainTeX@hyphen}
%    \begin{macrocode}
\def\HoLogoHtml@plainTeX@hyphen#1{%
  #1{p}{P}lain-\hologo{TeX}%
}
%    \end{macrocode}
%    \end{macro}
%
%    \begin{macro}{\HoLogo@plainTeX@runtogether}
%    \begin{macrocode}
\def\HoLogo@plainTeX@runtogether#1{%
  \HOLOGO@mbox{#1{p}{P}lain\hologo{TeX}}%
}
%    \end{macrocode}
%    \end{macro}
%    \begin{macro}{\HoLogoCs@plainTeX@runtogether}
%    \begin{macrocode}
\def\HoLogoCs@plainTeX@runtogether#1{#1{p}{P}lainTeX}
%    \end{macrocode}
%    \end{macro}
%    \begin{macro}{\HoLogoBkm@plainTeX@runtogether}
%    \begin{macrocode}
\def\HoLogoBkm@plainTeX@runtogether#1{%
  #1{p}{P}lain\hologo{TeX}%
}
%    \end{macrocode}
%    \end{macro}
%    \begin{macro}{\HoLogoHtml@plainTeX@runtogether}
%    \begin{macrocode}
\def\HoLogoHtml@plainTeX@runtogether#1{%
  #1{p}{P}lain\hologo{TeX}%
}
%    \end{macrocode}
%    \end{macro}
%
%    \begin{macro}{\HoLogo@plainTeX}
%    \begin{macrocode}
\def\HoLogo@plainTeX{\HoLogo@plainTeX@space}
%    \end{macrocode}
%    \end{macro}
%    \begin{macro}{\HoLogoCs@plainTeX}
%    \begin{macrocode}
\def\HoLogoCs@plainTeX{\HoLogoCs@plainTeX@space}
%    \end{macrocode}
%    \end{macro}
%    \begin{macro}{\HoLogoBkm@plainTeX}
%    \begin{macrocode}
\def\HoLogoBkm@plainTeX{\HoLogoBkm@plainTeX@space}
%    \end{macrocode}
%    \end{macro}
%    \begin{macro}{\HoLogoHtml@plainTeX}
%    \begin{macrocode}
\def\HoLogoHtml@plainTeX{\HoLogoHtml@plainTeX@space}
%    \end{macrocode}
%    \end{macro}
%
% \subsubsection{\hologo{LaTeX}}
%
%    Source: \hologo{LaTeX} kernel.
%\begin{quote}
%\begin{verbatim}
%\DeclareRobustCommand{\LaTeX}{%
%  L%
%  \kern-.36em%
%  {%
%    \sbox\z@ T%
%    \vbox to\ht\z@{%
%      \hbox{%
%        \check@mathfonts
%        \fontsize\sf@size\z@
%        \math@fontsfalse
%        \selectfont
%        A%
%      }%
%      \vss
%    }%
%  }%
%  \kern-.15em%
%  \TeX
%}
%\end{verbatim}
%\end{quote}
%
%    \begin{macro}{\HoLogo@La}
%    \begin{macrocode}
\def\HoLogo@La#1{%
  L%
  \kern-.36em%
  \begingroup
    \setbox\ltx@zero\hbox{T}%
    \vbox to\ht\ltx@zero{%
      \hbox{%
        \ltx@ifundefined{check@mathfonts}{%
          \csname sevenrm\endcsname
        }{%
          \check@mathfonts
          \fontsize\sf@size{0pt}%
          \math@fontsfalse\selectfont
        }%
        A%
      }%
      \vss
    }%
  \endgroup
}
%    \end{macrocode}
%    \end{macro}
%
%    \begin{macro}{\HoLogo@LaTeX}
%    Source: \hologo{LaTeX} kernel.
%    \begin{macrocode}
\def\HoLogo@LaTeX#1{%
  \hologo{La}%
  \kern-.15em%
  \hologo{TeX}%
}
%    \end{macrocode}
%    \end{macro}
%    \begin{macro}{\HoLogoHtml@LaTeX}
%    \begin{macrocode}
\def\HoLogoHtml@LaTeX#1{%
  \HoLogoCss@LaTeX
  \HOLOGO@Span{LaTeX}{%
    L%
    \HOLOGO@Span{a}{%
      A%
    }%
    \hologo{TeX}%
  }%
}
%    \end{macrocode}
%    \end{macro}
%    \begin{macro}{\HoLogoCss@LaTeX}
%    \begin{macrocode}
\def\HoLogoCss@LaTeX{%
  \Css{%
    span.HoLogo-LaTeX span.HoLogo-a{%
      position:relative;%
      top:-.5ex;%
      margin-left:-.36em;%
      margin-right:-.15em;%
      font-size:85\%;%
    }%
  }%
  \global\let\HoLogoCss@LaTeX\relax
}
%    \end{macrocode}
%    \end{macro}
%
% \subsubsection{\hologo{(La)TeX}}
%
%    \begin{macro}{\HoLogo@LaTeXTeX}
%    The kerning around the parentheses is taken
%    from package \xpackage{dtklogos} \cite{dtklogos}.
%\begin{quote}
%\begin{verbatim}
%\DeclareRobustCommand{\LaTeXTeX}{%
%  (%
%  \kern-.15em%
%  L%
%  \kern-.36em%
%  {%
%    \sbox\z@ T%
%    \vbox to\ht0{%
%      \hbox{%
%        $\m@th$%
%        \csname S@\f@size\endcsname
%        \fontsize\sf@size\z@
%        \math@fontsfalse
%        \selectfont
%        A%
%      }%
%      \vss
%    }%
%  }%
%  \kern-.2em%
%  )%
%  \kern-.15em%
%  \TeX
%}
%\end{verbatim}
%\end{quote}
%    \begin{macrocode}
\def\HoLogo@LaTeXTeX#1{%
  (%
  \kern-.15em%
  \hologo{La}%
  \kern-.2em%
  )%
  \kern-.15em%
  \hologo{TeX}%
}
%    \end{macrocode}
%    \end{macro}
%    \begin{macro}{\HoLogoBkm@LaTeXTeX}
%    \begin{macrocode}
\def\HoLogoBkm@LaTeXTeX#1{(La)TeX}
%    \end{macrocode}
%    \end{macro}
%
%    \begin{macro}{\HoLogo@(La)TeX}
%    \begin{macrocode}
\expandafter
\let\csname HoLogo@(La)TeX\endcsname\HoLogo@LaTeXTeX
%    \end{macrocode}
%    \end{macro}
%    \begin{macro}{\HoLogoBkm@(La)TeX}
%    \begin{macrocode}
\expandafter
\let\csname HoLogoBkm@(La)TeX\endcsname\HoLogoBkm@LaTeXTeX
%    \end{macrocode}
%    \end{macro}
%    \begin{macro}{\HoLogoHtml@LaTeXTeX}
%    \begin{macrocode}
\def\HoLogoHtml@LaTeXTeX#1{%
  \HoLogoCss@LaTeXTeX
  \HOLOGO@Span{LaTeXTeX}{%
    (%
    \HOLOGO@Span{L}{L}%
    \HOLOGO@Span{a}{A}%
    \HOLOGO@Span{ParenRight}{)}%
    \hologo{TeX}%
  }%
}
%    \end{macrocode}
%    \end{macro}
%    \begin{macro}{\HoLogoHtml@(La)TeX}
%    Kerning after opening parentheses and before closing parentheses
%    is $-0.1$\,em. The original values $-0.15$\,em
%    looked too ugly for a serif font.
%    \begin{macrocode}
\expandafter
\let\csname HoLogoHtml@(La)TeX\endcsname\HoLogoHtml@LaTeXTeX
%    \end{macrocode}
%    \end{macro}
%    \begin{macro}{\HoLogoCss@LaTeXTeX}
%    \begin{macrocode}
\def\HoLogoCss@LaTeXTeX{%
  \Css{%
    span.HoLogo-LaTeXTeX span.HoLogo-L{%
      margin-left:-.1em;%
    }%
  }%
  \Css{%
    span.HoLogo-LaTeXTeX span.HoLogo-a{%
      position:relative;%
      top:-.5ex;%
      margin-left:-.36em;%
      margin-right:-.1em;%
      font-size:85\%;%
    }%
  }%
  \Css{%
    span.HoLogo-LaTeXTeX span.HoLogo-ParenRight{%
      margin-right:-.15em;%
    }%
  }%
  \global\let\HoLogoCss@LaTeXTeX\relax
}
%    \end{macrocode}
%    \end{macro}
%
% \subsubsection{\hologo{LaTeXe}}
%
%    \begin{macro}{\HoLogo@LaTeXe}
%    Source: \hologo{LaTeX} kernel
%    \begin{macrocode}
\def\HoLogo@LaTeXe#1{%
  \hologo{LaTeX}%
  \kern.15em%
  \hbox{%
    \HOLOGO@MathSetup
    2%
    $_{\textstyle\varepsilon}$%
  }%
}
%    \end{macrocode}
%    \end{macro}
%
%    \begin{macro}{\HoLogoCs@LaTeXe}
%    \begin{macrocode}
\ifnum64=`\^^^^0040\relax % test for big chars of LuaTeX/XeTeX
  \catcode`\$=9 %
  \catcode`\&=14 %
\else
  \catcode`\$=14 %
  \catcode`\&=9 %
\fi
\def\HoLogoCs@LaTeXe#1{%
  LaTeX2%
$ \string ^^^^0395%
& e%
}%
\catcode`\$=3 %
\catcode`\&=4 %
%    \end{macrocode}
%    \end{macro}
%
%    \begin{macro}{\HoLogoBkm@LaTeXe}
%    \begin{macrocode}
\def\HoLogoBkm@LaTeXe#1{%
  \hologo{LaTeX}%
  2%
  \HOLOGO@PdfdocUnicode{e}{\textepsilon}%
}
%    \end{macrocode}
%    \end{macro}
%
%    \begin{macro}{\HoLogoHtml@LaTeXe}
%    \begin{macrocode}
\def\HoLogoHtml@LaTeXe#1{%
  \HoLogoCss@LaTeXe
  \HOLOGO@Span{LaTeX2e}{%
    \hologo{LaTeX}%
    \HOLOGO@Span{2}{2}%
    \HOLOGO@Span{e}{%
      \HOLOGO@MathSetup
      \ensuremath{\textstyle\varepsilon}%
    }%
  }%
}
%    \end{macrocode}
%    \end{macro}
%    \begin{macro}{\HoLogoCss@LaTeXe}
%    \begin{macrocode}
\def\HoLogoCss@LaTeXe{%
  \Css{%
    span.HoLogo-LaTeX2e span.HoLogo-2{%
      padding-left:.15em;%
    }%
  }%
  \Css{%
    span.HoLogo-LaTeX2e span.HoLogo-e{%
      position:relative;%
      top:.35ex;%
      text-decoration:none;%
    }%
  }%
  \global\let\HoLogoCss@LaTeXe\relax
}
%    \end{macrocode}
%    \end{macro}
%
%    \begin{macro}{\HoLogo@LaTeX2e}
%    \begin{macrocode}
\expandafter
\let\csname HoLogo@LaTeX2e\endcsname\HoLogo@LaTeXe
%    \end{macrocode}
%    \end{macro}
%    \begin{macro}{\HoLogoCs@LaTeX2e}
%    \begin{macrocode}
\expandafter
\let\csname HoLogoCs@LaTeX2e\endcsname\HoLogoCs@LaTeXe
%    \end{macrocode}
%    \end{macro}
%    \begin{macro}{\HoLogoBkm@LaTeX2e}
%    \begin{macrocode}
\expandafter
\let\csname HoLogoBkm@LaTeX2e\endcsname\HoLogoBkm@LaTeXe
%    \end{macrocode}
%    \end{macro}
%    \begin{macro}{\HoLogoHtml@LaTeX2e}
%    \begin{macrocode}
\expandafter
\let\csname HoLogoHtml@LaTeX2e\endcsname\HoLogoHtml@LaTeXe
%    \end{macrocode}
%    \end{macro}
%
% \subsubsection{\hologo{LaTeX3}}
%
%    \begin{macro}{\HoLogo@LaTeX3}
%    Source: \hologo{LaTeX} kernel
%    \begin{macrocode}
\expandafter\def\csname HoLogo@LaTeX3\endcsname#1{%
  \hologo{LaTeX}%
  3%
}
%    \end{macrocode}
%    \end{macro}
%
%    \begin{macro}{\HoLogoBkm@LaTeX3}
%    \begin{macrocode}
\expandafter\def\csname HoLogoBkm@LaTeX3\endcsname#1{%
  \hologo{LaTeX}%
  3%
}
%    \end{macrocode}
%    \end{macro}
%    \begin{macro}{\HoLogoHtml@LaTeX3}
%    \begin{macrocode}
\expandafter
\let\csname HoLogoHtml@LaTeX3\expandafter\endcsname
\csname HoLogo@LaTeX3\endcsname
%    \end{macrocode}
%    \end{macro}
%
% \subsubsection{\hologo{LaTeXML}}
%
%    \begin{macro}{\HoLogo@LaTeXML}
%    \begin{macrocode}
\def\HoLogo@LaTeXML#1{%
  \HOLOGO@mbox{%
    \hologo{La}%
    \kern-.15em%
    T%
    \kern-.1667em%
    \lower.5ex\hbox{E}%
    \kern-.125em%
    \HoLogoFont@font{LaTeXML}{sc}{xml}%
  }%
}
%    \end{macrocode}
%    \end{macro}
%    \begin{macro}{\HoLogoHtml@pdfLaTeX}
%    \begin{macrocode}
\def\HoLogoHtml@LaTeXML#1{%
  \HOLOGO@Span{LaTeXML}{%
    \HoLogoCss@LaTeX
    \HoLogoCss@TeX
    \HOLOGO@Span{LaTeX}{%
      L%
      \HOLOGO@Span{a}{%
        A%
      }%
    }%
    \HOLOGO@Span{TeX}{%
      T%
      \HOLOGO@Span{e}{%
        E%
      }%
    }%
    \HCode{<span style="font-variant: small-caps;">}%
    xml%
    \HCode{</span>}%
  }%
}
%    \end{macrocode}
%    \end{macro}
%
% \subsubsection{\hologo{eTeX}}
%
%    \begin{macro}{\HoLogo@eTeX}
%    Source: package \xpackage{etex}
%    \begin{macrocode}
\def\HoLogo@eTeX#1{%
  \ltx@mbox{%
    \HOLOGO@MathSetup
    $\varepsilon$%
    -%
    \HOLOGO@NegativeKerning{-T,T-,To}%
    \hologo{TeX}%
  }%
}
%    \end{macrocode}
%    \end{macro}
%    \begin{macro}{\HoLogoCs@eTeX}
%    \begin{macrocode}
\ifnum64=`\^^^^0040\relax % test for big chars of LuaTeX/XeTeX
  \catcode`\$=9 %
  \catcode`\&=14 %
\else
  \catcode`\$=14 %
  \catcode`\&=9 %
\fi
\def\HoLogoCs@eTeX#1{%
$ #1{\string ^^^^0395}{\string ^^^^03b5}%
& #1{e}{E}%
  TeX%
}%
\catcode`\$=3 %
\catcode`\&=4 %
%    \end{macrocode}
%    \end{macro}
%    \begin{macro}{\HoLogoBkm@eTeX}
%    \begin{macrocode}
\def\HoLogoBkm@eTeX#1{%
  \HOLOGO@PdfdocUnicode{#1{e}{E}}{\textepsilon}%
  -%
  \hologo{TeX}%
}
%    \end{macrocode}
%    \end{macro}
%    \begin{macro}{\HoLogoHtml@eTeX}
%    \begin{macrocode}
\def\HoLogoHtml@eTeX#1{%
  \ltx@mbox{%
    \HOLOGO@MathSetup
    $\varepsilon$%
    -%
    \hologo{TeX}%
  }%
}
%    \end{macrocode}
%    \end{macro}
%
% \subsubsection{\hologo{iniTeX}}
%
%    \begin{macro}{\HoLogo@iniTeX}
%    \begin{macrocode}
\def\HoLogo@iniTeX#1{%
  \HOLOGO@mbox{%
    #1{i}{I}ni\hologo{TeX}%
  }%
}
%    \end{macrocode}
%    \end{macro}
%    \begin{macro}{\HoLogoCs@iniTeX}
%    \begin{macrocode}
\def\HoLogoCs@iniTeX#1{#1{i}{I}niTeX}
%    \end{macrocode}
%    \end{macro}
%    \begin{macro}{\HoLogoBkm@iniTeX}
%    \begin{macrocode}
\def\HoLogoBkm@iniTeX#1{%
  #1{i}{I}ni\hologo{TeX}%
}
%    \end{macrocode}
%    \end{macro}
%    \begin{macro}{\HoLogoHtml@iniTeX}
%    \begin{macrocode}
\let\HoLogoHtml@iniTeX\HoLogo@iniTeX
%    \end{macrocode}
%    \end{macro}
%
% \subsubsection{\hologo{virTeX}}
%
%    \begin{macro}{\HoLogo@virTeX}
%    \begin{macrocode}
\def\HoLogo@virTeX#1{%
  \HOLOGO@mbox{%
    #1{v}{V}ir\hologo{TeX}%
  }%
}
%    \end{macrocode}
%    \end{macro}
%    \begin{macro}{\HoLogoCs@virTeX}
%    \begin{macrocode}
\def\HoLogoCs@virTeX#1{#1{v}{V}irTeX}
%    \end{macrocode}
%    \end{macro}
%    \begin{macro}{\HoLogoBkm@virTeX}
%    \begin{macrocode}
\def\HoLogoBkm@virTeX#1{%
  #1{v}{V}ir\hologo{TeX}%
}
%    \end{macrocode}
%    \end{macro}
%    \begin{macro}{\HoLogoHtml@virTeX}
%    \begin{macrocode}
\let\HoLogoHtml@virTeX\HoLogo@virTeX
%    \end{macrocode}
%    \end{macro}
%
% \subsubsection{\hologo{SliTeX}}
%
% \paragraph{Definitions of the three variants.}
%
%    \begin{macro}{\HoLogo@SLiTeX@lift}
%    \begin{macrocode}
\def\HoLogo@SLiTeX@lift#1{%
  \HoLogoFont@font{SliTeX}{rm}{%
    S%
    \kern-.06em%
    L%
    \kern-.18em%
    \raise.32ex\hbox{\HoLogoFont@font{SliTeX}{sc}{i}}%
    \HOLOGO@discretionary
    \kern-.06em%
    \hologo{TeX}%
  }%
}
%    \end{macrocode}
%    \end{macro}
%    \begin{macro}{\HoLogoBkm@SLiTeX@lift}
%    \begin{macrocode}
\def\HoLogoBkm@SLiTeX@lift#1{SLiTeX}
%    \end{macrocode}
%    \end{macro}
%    \begin{macro}{\HoLogoHtml@SLiTeX@lift}
%    \begin{macrocode}
\def\HoLogoHtml@SLiTeX@lift#1{%
  \HoLogoCss@SLiTeX@lift
  \HOLOGO@Span{SLiTeX-lift}{%
    \HoLogoFont@font{SliTeX}{rm}{%
      S%
      \HOLOGO@Span{L}{L}%
      \HOLOGO@Span{i}{i}%
      \hologo{TeX}%
    }%
  }%
}
%    \end{macrocode}
%    \end{macro}
%    \begin{macro}{\HoLogoCss@SLiTeX@lift}
%    \begin{macrocode}
\def\HoLogoCss@SLiTeX@lift{%
  \Css{%
    span.HoLogo-SLiTeX-lift span.HoLogo-L{%
      margin-left:-.06em;%
      margin-right:-.18em;%
    }%
  }%
  \Css{%
    span.HoLogo-SLiTeX-lift span.HoLogo-i{%
      position:relative;%
      top:-.32ex;%
      margin-right:-.06em;%
      font-variant:small-caps;%
    }%
  }%
  \global\let\HoLogoCss@SLiTeX@lift\relax
}
%    \end{macrocode}
%    \end{macro}
%
%    \begin{macro}{\HoLogo@SliTeX@simple}
%    \begin{macrocode}
\def\HoLogo@SliTeX@simple#1{%
  \HoLogoFont@font{SliTeX}{rm}{%
    \ltx@mbox{%
      \HoLogoFont@font{SliTeX}{sc}{Sli}%
    }%
    \HOLOGO@discretionary
    \hologo{TeX}%
  }%
}
%    \end{macrocode}
%    \end{macro}
%    \begin{macro}{\HoLogoBkm@SliTeX@simple}
%    \begin{macrocode}
\def\HoLogoBkm@SliTeX@simple#1{SliTeX}
%    \end{macrocode}
%    \end{macro}
%    \begin{macro}{\HoLogoHtml@SliTeX@simple}
%    \begin{macrocode}
\let\HoLogoHtml@SliTeX@simple\HoLogo@SliTeX@simple
%    \end{macrocode}
%    \end{macro}
%
%    \begin{macro}{\HoLogo@SliTeX@narrow}
%    \begin{macrocode}
\def\HoLogo@SliTeX@narrow#1{%
  \HoLogoFont@font{SliTeX}{rm}{%
    \ltx@mbox{%
      S%
      \kern-.06em%
      \HoLogoFont@font{SliTeX}{sc}{%
        l%
        \kern-.035em%
        i%
      }%
    }%
    \HOLOGO@discretionary
    \kern-.06em%
    \hologo{TeX}%
  }%
}
%    \end{macrocode}
%    \end{macro}
%    \begin{macro}{\HoLogoBkm@SliTeX@narrow}
%    \begin{macrocode}
\def\HoLogoBkm@SliTeX@narrow#1{SliTeX}
%    \end{macrocode}
%    \end{macro}
%    \begin{macro}{\HoLogoHtml@SliTeX@narrow}
%    \begin{macrocode}
\def\HoLogoHtml@SliTeX@narrow#1{%
  \HoLogoCss@SliTeX@narrow
  \HOLOGO@Span{SliTeX-narrow}{%
    \HoLogoFont@font{SliTeX}{rm}{%
      S%
        \HOLOGO@Span{l}{l}%
        \HOLOGO@Span{i}{i}%
      \hologo{TeX}%
    }%
  }%
}
%    \end{macrocode}
%    \end{macro}
%    \begin{macro}{\HoLogoCss@SliTeX@narrow}
%    \begin{macrocode}
\def\HoLogoCss@SliTeX@narrow{%
  \Css{%
    span.HoLogo-SliTeX-narrow span.HoLogo-l{%
      margin-left:-.06em;%
      margin-right:-.035em;%
      font-variant:small-caps;%
    }%
  }%
  \Css{%
    span.HoLogo-SliTeX-narrow span.HoLogo-i{%
      margin-right:-.06em;%
      font-variant:small-caps;%
    }%
  }%
  \global\let\HoLogoCss@SliTeX@narrow\relax
}
%    \end{macrocode}
%    \end{macro}
%
% \paragraph{Macro set completion.}
%
%    \begin{macro}{\HoLogo@SLiTeX@simple}
%    \begin{macrocode}
\def\HoLogo@SLiTeX@simple{\HoLogo@SliTeX@simple}
%    \end{macrocode}
%    \end{macro}
%    \begin{macro}{\HoLogoBkm@SLiTeX@simple}
%    \begin{macrocode}
\def\HoLogoBkm@SLiTeX@simple{\HoLogoBkm@SliTeX@simple}
%    \end{macrocode}
%    \end{macro}
%    \begin{macro}{\HoLogoHtml@SLiTeX@simple}
%    \begin{macrocode}
\def\HoLogoHtml@SLiTeX@simple{\HoLogoHtml@SliTeX@simple}
%    \end{macrocode}
%    \end{macro}
%
%    \begin{macro}{\HoLogo@SLiTeX@narrow}
%    \begin{macrocode}
\def\HoLogo@SLiTeX@narrow{\HoLogo@SliTeX@narrow}
%    \end{macrocode}
%    \end{macro}
%    \begin{macro}{\HoLogoBkm@SLiTeX@narrow}
%    \begin{macrocode}
\def\HoLogoBkm@SLiTeX@narrow{\HoLogoBkm@SliTeX@narrow}
%    \end{macrocode}
%    \end{macro}
%    \begin{macro}{\HoLogoHtml@SLiTeX@narrow}
%    \begin{macrocode}
\def\HoLogoHtml@SLiTeX@narrow{\HoLogoHtml@SliTeX@narrow}
%    \end{macrocode}
%    \end{macro}
%
%    \begin{macro}{\HoLogo@SliTeX@lift}
%    \begin{macrocode}
\def\HoLogo@SliTeX@lift{\HoLogo@SLiTeX@lift}
%    \end{macrocode}
%    \end{macro}
%    \begin{macro}{\HoLogoBkm@SliTeX@lift}
%    \begin{macrocode}
\def\HoLogoBkm@SliTeX@lift{\HoLogoBkm@SLiTeX@lift}
%    \end{macrocode}
%    \end{macro}
%    \begin{macro}{\HoLogoHtml@SliTeX@lift}
%    \begin{macrocode}
\def\HoLogoHtml@SliTeX@lift{\HoLogoHtml@SLiTeX@lift}
%    \end{macrocode}
%    \end{macro}
%
% \paragraph{Defaults.}
%
%    \begin{macro}{\HoLogo@SLiTeX}
%    \begin{macrocode}
\def\HoLogo@SLiTeX{\HoLogo@SLiTeX@lift}
%    \end{macrocode}
%    \end{macro}
%    \begin{macro}{\HoLogoBkm@SLiTeX}
%    \begin{macrocode}
\def\HoLogoBkm@SLiTeX{\HoLogoBkm@SLiTeX@lift}
%    \end{macrocode}
%    \end{macro}
%    \begin{macro}{\HoLogoHtml@SLiTeX}
%    \begin{macrocode}
\def\HoLogoHtml@SLiTeX{\HoLogoHtml@SLiTeX@lift}
%    \end{macrocode}
%    \end{macro}
%
%    \begin{macro}{\HoLogo@SliTeX}
%    \begin{macrocode}
\def\HoLogo@SliTeX{\HoLogo@SliTeX@narrow}
%    \end{macrocode}
%    \end{macro}
%    \begin{macro}{\HoLogoBkm@SliTeX}
%    \begin{macrocode}
\def\HoLogoBkm@SliTeX{\HoLogoBkm@SliTeX@narrow}
%    \end{macrocode}
%    \end{macro}
%    \begin{macro}{\HoLogoHtml@SliTeX}
%    \begin{macrocode}
\def\HoLogoHtml@SliTeX{\HoLogoHtml@SliTeX@narrow}
%    \end{macrocode}
%    \end{macro}
%
% \subsubsection{\hologo{LuaTeX}}
%
%    \begin{macro}{\HoLogo@LuaTeX}
%    The kerning is an idea of Hans Hagen, see mailing list
%    `luatex at tug dot org' in March 2010.
%    \begin{macrocode}
\def\HoLogo@LuaTeX#1{%
  \HOLOGO@mbox{%
    Lua%
    \HOLOGO@NegativeKerning{aT,oT,To}%
    \hologo{TeX}%
  }%
}
%    \end{macrocode}
%    \end{macro}
%    \begin{macro}{\HoLogoHtml@LuaTeX}
%    \begin{macrocode}
\let\HoLogoHtml@LuaTeX\HoLogo@LuaTeX
%    \end{macrocode}
%    \end{macro}
%
% \subsubsection{\hologo{LuaLaTeX}}
%
%    \begin{macro}{\HoLogo@LuaLaTeX}
%    \begin{macrocode}
\def\HoLogo@LuaLaTeX#1{%
  \HOLOGO@mbox{%
    Lua%
    \hologo{LaTeX}%
  }%
}
%    \end{macrocode}
%    \end{macro}
%    \begin{macro}{\HoLogoHtml@LuaLaTeX}
%    \begin{macrocode}
\let\HoLogoHtml@LuaLaTeX\HoLogo@LuaLaTeX
%    \end{macrocode}
%    \end{macro}
%
% \subsubsection{\hologo{XeTeX}, \hologo{XeLaTeX}}
%
%    \begin{macro}{\HOLOGO@IfCharExists}
%    \begin{macrocode}
\ifluatex
  \ifnum\luatexversion<36 %
  \else
    \def\HOLOGO@IfCharExists#1{%
      \ifnum
        \directlua{%
           if luaotfload and luaotfload.aux then
             if luaotfload.aux.font_has_glyph(%
                    font.current(), \number#1) then % 	 
	       tex.print("1") % 	 
	     end % 	 
	   elseif font and font.fonts and font.current then %
            local f = font.fonts[font.current()]%
            if f.characters and f.characters[\number#1] then %
              tex.print("1")%
            end %
          end%
        }0=\ltx@zero
        \expandafter\ltx@secondoftwo
      \else
        \expandafter\ltx@firstoftwo
      \fi
    }%
  \fi
\fi
\ltx@IfUndefined{HOLOGO@IfCharExists}{%
  \def\HOLOGO@@IfCharExists#1{%
    \begingroup
      \tracinglostchars=\ltx@zero
      \setbox\ltx@zero=\hbox{%
        \kern7sp\char#1\relax
        \ifnum\lastkern>\ltx@zero
          \expandafter\aftergroup\csname iffalse\endcsname
        \else
          \expandafter\aftergroup\csname iftrue\endcsname
        \fi
      }%
      % \if{true|false} from \aftergroup
      \endgroup
      \expandafter\ltx@firstoftwo
    \else
      \endgroup
      \expandafter\ltx@secondoftwo
    \fi
  }%
  \ifxetex
    \ltx@IfUndefined{XeTeXfonttype}{}{%
      \ltx@IfUndefined{XeTeXcharglyph}{}{%
        \def\HOLOGO@IfCharExists#1{%
          \ifnum\XeTeXfonttype\font>\ltx@zero
            \expandafter\ltx@firstofthree
          \else
            \expandafter\ltx@gobble
          \fi
          {%
            \ifnum\XeTeXcharglyph#1>\ltx@zero
              \expandafter\ltx@firstoftwo
            \else
              \expandafter\ltx@secondoftwo
            \fi
          }%
          \HOLOGO@@IfCharExists{#1}%
        }%
      }%
    }%
  \fi
}{}
\ltx@ifundefined{HOLOGO@IfCharExists}{%
  \ifnum64=`\^^^^0040\relax % test for big chars of LuaTeX/XeTeX
    \let\HOLOGO@IfCharExists\HOLOGO@@IfCharExists
  \else
    \def\HOLOGO@IfCharExists#1{%
      \ifnum#1>255 %
        \expandafter\ltx@fourthoffour
      \fi
      \HOLOGO@@IfCharExists{#1}%
    }%
  \fi
}{}
%    \end{macrocode}
%    \end{macro}
%
%    \begin{macro}{\HoLogo@Xe}
%    Source: package \xpackage{dtklogos}
%    \begin{macrocode}
\def\HoLogo@Xe#1{%
  X%
  \kern-.1em\relax
  \HOLOGO@IfCharExists{"018E}{%
    \lower.5ex\hbox{\char"018E}%
  }{%
    \chardef\HOLOGO@choice=\ltx@zero
    \ifdim\fontdimen\ltx@one\font>0pt %
      \ltx@IfUndefined{rotatebox}{%
        \ltx@IfUndefined{pgftext}{%
          \ltx@IfUndefined{psscalebox}{%
            \ltx@IfUndefined{HOLOGO@ScaleBox@\hologoDriver}{%
            }{%
              \chardef\HOLOGO@choice=4 %
            }%
          }{%
            \chardef\HOLOGO@choice=3 %
          }%
        }{%
          \chardef\HOLOGO@choice=2 %
        }%
      }{%
        \chardef\HOLOGO@choice=1 %
      }%
      \ifcase\HOLOGO@choice
        \HOLOGO@WarningUnsupportedDriver{Xe}%
        e%
      \or % 1: \rotatebox
        \begingroup
          \setbox\ltx@zero\hbox{\rotatebox{180}{E}}%
          \ltx@LocDimenA=\dp\ltx@zero
          \advance\ltx@LocDimenA by -.5ex\relax
          \raise\ltx@LocDimenA\box\ltx@zero
        \endgroup
      \or % 2: \pgftext
        \lower.5ex\hbox{%
          \pgfpicture
            \pgftext[rotate=180]{E}%
          \endpgfpicture
        }%
      \or % 3: \psscalebox
        \begingroup
          \setbox\ltx@zero\hbox{\psscalebox{-1 -1}{E}}%
          \ltx@LocDimenA=\dp\ltx@zero
          \advance\ltx@LocDimenA by -.5ex\relax
          \raise\ltx@LocDimenA\box\ltx@zero
        \endgroup
      \or % 4: \HOLOGO@PointReflectBox
        \lower.5ex\hbox{\HOLOGO@PointReflectBox{E}}%
      \else
        \@PackageError{hologo}{Internal error (choice/it}\@ehc
      \fi
    \else
      \ltx@IfUndefined{reflectbox}{%
        \ltx@IfUndefined{pgftext}{%
          \ltx@IfUndefined{psscalebox}{%
            \ltx@IfUndefined{HOLOGO@ScaleBox@\hologoDriver}{%
            }{%
              \chardef\HOLOGO@choice=4 %
            }%
          }{%
            \chardef\HOLOGO@choice=3 %
          }%
        }{%
          \chardef\HOLOGO@choice=2 %
        }%
      }{%
        \chardef\HOLOGO@choice=1 %
      }%
      \ifcase\HOLOGO@choice
        \HOLOGO@WarningUnsupportedDriver{Xe}%
        e%
      \or % 1: reflectbox
        \lower.5ex\hbox{%
          \reflectbox{E}%
        }%
      \or % 2: \pgftext
        \lower.5ex\hbox{%
          \pgfpicture
            \pgftransformxscale{-1}%
            \pgftext{E}%
          \endpgfpicture
        }%
      \or % 3: \psscalebox
        \lower.5ex\hbox{%
          \psscalebox{-1 1}{E}%
        }%
      \or % 4: \HOLOGO@Reflectbox
        \lower.5ex\hbox{%
          \HOLOGO@ReflectBox{E}%
        }%
      \else
        \@PackageError{hologo}{Internal error (choice/up)}\@ehc
      \fi
    \fi
  }%
}
%    \end{macrocode}
%    \end{macro}
%    \begin{macro}{\HoLogoHtml@Xe}
%    \begin{macrocode}
\def\HoLogoHtml@Xe#1{%
  \HoLogoCss@Xe
  \HOLOGO@Span{Xe}{%
    X%
    \HOLOGO@Span{e}{%
      \HCode{&\ltx@hashchar x018e;}%
    }%
  }%
}
%    \end{macrocode}
%    \end{macro}
%    \begin{macro}{\HoLogoCss@Xe}
%    \begin{macrocode}
\def\HoLogoCss@Xe{%
  \Css{%
    span.HoLogo-Xe span.HoLogo-e{%
      position:relative;%
      top:.5ex;%
      left-margin:-.1em;%
    }%
  }%
  \global\let\HoLogoCss@Xe\relax
}
%    \end{macrocode}
%    \end{macro}
%
%    \begin{macro}{\HoLogo@XeTeX}
%    \begin{macrocode}
\def\HoLogo@XeTeX#1{%
  \hologo{Xe}%
  \kern-.15em\relax
  \hologo{TeX}%
}
%    \end{macrocode}
%    \end{macro}
%
%    \begin{macro}{\HoLogoHtml@XeTeX}
%    \begin{macrocode}
\def\HoLogoHtml@XeTeX#1{%
  \HoLogoCss@XeTeX
  \HOLOGO@Span{XeTeX}{%
    \hologo{Xe}%
    \hologo{TeX}%
  }%
}
%    \end{macrocode}
%    \end{macro}
%    \begin{macro}{\HoLogoCss@XeTeX}
%    \begin{macrocode}
\def\HoLogoCss@XeTeX{%
  \Css{%
    span.HoLogo-XeTeX span.HoLogo-TeX{%
      margin-left:-.15em;%
    }%
  }%
  \global\let\HoLogoCss@XeTeX\relax
}
%    \end{macrocode}
%    \end{macro}
%
%    \begin{macro}{\HoLogo@XeLaTeX}
%    \begin{macrocode}
\def\HoLogo@XeLaTeX#1{%
  \hologo{Xe}%
  \kern-.13em%
  \hologo{LaTeX}%
}
%    \end{macrocode}
%    \end{macro}
%    \begin{macro}{\HoLogoHtml@XeLaTeX}
%    \begin{macrocode}
\def\HoLogoHtml@XeLaTeX#1{%
  \HoLogoCss@XeLaTeX
  \HOLOGO@Span{XeLaTeX}{%
    \hologo{Xe}%
    \hologo{LaTeX}%
  }%
}
%    \end{macrocode}
%    \end{macro}
%    \begin{macro}{\HoLogoCss@XeLaTeX}
%    \begin{macrocode}
\def\HoLogoCss@XeLaTeX{%
  \Css{%
    span.HoLogo-XeLaTeX span.HoLogo-Xe{%
      margin-right:-.13em;%
    }%
  }%
  \global\let\HoLogoCss@XeLaTeX\relax
}
%    \end{macrocode}
%    \end{macro}
%
% \subsubsection{\hologo{pdfTeX}, \hologo{pdfLaTeX}}
%
%    \begin{macro}{\HoLogo@pdfTeX}
%    \begin{macrocode}
\def\HoLogo@pdfTeX#1{%
  \HOLOGO@mbox{%
    #1{p}{P}df\hologo{TeX}%
  }%
}
%    \end{macrocode}
%    \end{macro}
%    \begin{macro}{\HoLogoCs@pdfTeX}
%    \begin{macrocode}
\def\HoLogoCs@pdfTeX#1{#1{p}{P}dfTeX}
%    \end{macrocode}
%    \end{macro}
%    \begin{macro}{\HoLogoBkm@pdfTeX}
%    \begin{macrocode}
\def\HoLogoBkm@pdfTeX#1{%
  #1{p}{P}df\hologo{TeX}%
}
%    \end{macrocode}
%    \end{macro}
%    \begin{macro}{\HoLogoHtml@pdfTeX}
%    \begin{macrocode}
\let\HoLogoHtml@pdfTeX\HoLogo@pdfTeX
%    \end{macrocode}
%    \end{macro}
%
%    \begin{macro}{\HoLogo@pdfLaTeX}
%    \begin{macrocode}
\def\HoLogo@pdfLaTeX#1{%
  \HOLOGO@mbox{%
    #1{p}{P}df\hologo{LaTeX}%
  }%
}
%    \end{macrocode}
%    \end{macro}
%    \begin{macro}{\HoLogoCs@pdfLaTeX}
%    \begin{macrocode}
\def\HoLogoCs@pdfLaTeX#1{#1{p}{P}dfLaTeX}
%    \end{macrocode}
%    \end{macro}
%    \begin{macro}{\HoLogoBkm@pdfLaTeX}
%    \begin{macrocode}
\def\HoLogoBkm@pdfLaTeX#1{%
  #1{p}{P}df\hologo{LaTeX}%
}
%    \end{macrocode}
%    \end{macro}
%    \begin{macro}{\HoLogoHtml@pdfLaTeX}
%    \begin{macrocode}
\let\HoLogoHtml@pdfLaTeX\HoLogo@pdfLaTeX
%    \end{macrocode}
%    \end{macro}
%
% \subsubsection{\hologo{VTeX}}
%
%    \begin{macro}{\HoLogo@VTeX}
%    \begin{macrocode}
\def\HoLogo@VTeX#1{%
  \HOLOGO@mbox{%
    V\hologo{TeX}%
  }%
}
%    \end{macrocode}
%    \end{macro}
%    \begin{macro}{\HoLogoHtml@VTeX}
%    \begin{macrocode}
\let\HoLogoHtml@VTeX\HoLogo@VTeX
%    \end{macrocode}
%    \end{macro}
%
% \subsubsection{\hologo{AmS}, \dots}
%
%    Source: class \xclass{amsdtx}
%
%    \begin{macro}{\HoLogo@AmS}
%    \begin{macrocode}
\def\HoLogo@AmS#1{%
  \HoLogoFont@font{AmS}{sy}{%
    A%
    \kern-.1667em%
    \lower.5ex\hbox{M}%
    \kern-.125em%
    S%
  }%
}
%    \end{macrocode}
%    \end{macro}
%    \begin{macro}{\HoLogoBkm@AmS}
%    \begin{macrocode}
\def\HoLogoBkm@AmS#1{AmS}
%    \end{macrocode}
%    \end{macro}
%    \begin{macro}{\HoLogoHtml@AmS}
%    \begin{macrocode}
\def\HoLogoHtml@AmS#1{%
  \HoLogoCss@AmS
%  \HoLogoFont@font{AmS}{sy}{%
    \HOLOGO@Span{AmS}{%
      A%
      \HOLOGO@Span{M}{M}%
      S%
    }%
%   }%
}
%    \end{macrocode}
%    \end{macro}
%    \begin{macro}{\HoLogoCss@AmS}
%    \begin{macrocode}
\def\HoLogoCss@AmS{%
  \Css{%
    span.HoLogo-AmS span.HoLogo-M{%
      position:relative;%
      top:.5ex;%
      margin-left:-.1667em;%
      margin-right:-.125em;%
      text-decoration:none;%
    }%
  }%
  \global\let\HoLogoCss@AmS\relax
}
%    \end{macrocode}
%    \end{macro}
%
%    \begin{macro}{\HoLogo@AmSTeX}
%    \begin{macrocode}
\def\HoLogo@AmSTeX#1{%
  \hologo{AmS}%
  \HOLOGO@hyphen
  \hologo{TeX}%
}
%    \end{macrocode}
%    \end{macro}
%    \begin{macro}{\HoLogoBkm@AmSTeX}
%    \begin{macrocode}
\def\HoLogoBkm@AmSTeX#1{AmS-TeX}%
%    \end{macrocode}
%    \end{macro}
%    \begin{macro}{\HoLogoHtml@AmSTeX}
%    \begin{macrocode}
\let\HoLogoHtml@AmSTeX\HoLogo@AmSTeX
%    \end{macrocode}
%    \end{macro}
%
%    \begin{macro}{\HoLogo@AmSLaTeX}
%    \begin{macrocode}
\def\HoLogo@AmSLaTeX#1{%
  \hologo{AmS}%
  \HOLOGO@hyphen
  \hologo{LaTeX}%
}
%    \end{macrocode}
%    \end{macro}
%    \begin{macro}{\HoLogoBkm@AmSLaTeX}
%    \begin{macrocode}
\def\HoLogoBkm@AmSLaTeX#1{AmS-LaTeX}%
%    \end{macrocode}
%    \end{macro}
%    \begin{macro}{\HoLogoHtml@AmSLaTeX}
%    \begin{macrocode}
\let\HoLogoHtml@AmSLaTeX\HoLogo@AmSLaTeX
%    \end{macrocode}
%    \end{macro}
%
% \subsubsection{\hologo{BibTeX}}
%
%    \begin{macro}{\HoLogo@BibTeX@sc}
%    A definition of \hologo{BibTeX} is provided in
%    the documentation source for the manual of \hologo{BibTeX}
%    \cite{btxdoc}.
%\begin{quote}
%\begin{verbatim}
%\def\BibTeX{%
%  {%
%    \rm
%    B%
%    \kern-.05em%
%    {%
%      \sc
%      i%
%      \kern-.025em %
%      b%
%    }%
%    \kern-.08em
%    T%
%    \kern-.1667em%
%    \lower.7ex\hbox{E}%
%    \kern-.125em%
%    X%
%  }%
%}
%\end{verbatim}
%\end{quote}
%    \begin{macrocode}
\def\HoLogo@BibTeX@sc#1{%
  B%
  \kern-.05em%
  \HoLogoFont@font{BibTeX}{sc}{%
    i%
    \kern-.025em%
    b%
  }%
  \HOLOGO@discretionary
  \kern-.08em%
  \hologo{TeX}%
}
%    \end{macrocode}
%    \end{macro}
%    \begin{macro}{\HoLogoHtml@BibTeX@sc}
%    \begin{macrocode}
\def\HoLogoHtml@BibTeX@sc#1{%
  \HoLogoCss@BibTeX@sc
  \HOLOGO@Span{BibTeX-sc}{%
    B%
    \HOLOGO@Span{i}{i}%
    \HOLOGO@Span{b}{b}%
    \hologo{TeX}%
  }%
}
%    \end{macrocode}
%    \end{macro}
%    \begin{macro}{\HoLogoCss@BibTeX@sc}
%    \begin{macrocode}
\def\HoLogoCss@BibTeX@sc{%
  \Css{%
    span.HoLogo-BibTeX-sc span.HoLogo-i{%
      margin-left:-.05em;%
      margin-right:-.025em;%
      font-variant:small-caps;%
    }%
  }%
  \Css{%
    span.HoLogo-BibTeX-sc span.HoLogo-b{%
      margin-right:-.08em;%
      font-variant:small-caps;%
    }%
  }%
  \global\let\HoLogoCss@BibTeX@sc\relax
}
%    \end{macrocode}
%    \end{macro}
%
%    \begin{macro}{\HoLogo@BibTeX@sf}
%    Variant \xoption{sf} avoids trouble with unavailable
%    small caps fonts (e.g., bold versions of Computer Modern or
%    Latin Modern). The definition is taken from
%    package \xpackage{dtklogos} \cite{dtklogos}.
%\begin{quote}
%\begin{verbatim}
%\DeclareRobustCommand{\BibTeX}{%
%  B%
%  \kern-.05em%
%  \hbox{%
%    $\m@th$% %% force math size calculations
%    \csname S@\f@size\endcsname
%    \fontsize\sf@size\z@
%    \math@fontsfalse
%    \selectfont
%    I%
%    \kern-.025em%
%    B
%  }%
%  \kern-.08em%
%  \-%
%  \TeX
%}
%\end{verbatim}
%\end{quote}
%    \begin{macrocode}
\def\HoLogo@BibTeX@sf#1{%
  B%
  \kern-.05em%
  \HoLogoFont@font{BibTeX}{bibsf}{%
    I%
    \kern-.025em%
    B%
  }%
  \HOLOGO@discretionary
  \kern-.08em%
  \hologo{TeX}%
}
%    \end{macrocode}
%    \end{macro}
%    \begin{macro}{\HoLogoHtml@BibTeX@sf}
%    \begin{macrocode}
\def\HoLogoHtml@BibTeX@sf#1{%
  \HoLogoCss@BibTeX@sf
  \HOLOGO@Span{BibTeX-sf}{%
    B%
    \HoLogoFont@font{BibTeX}{bibsf}{%
      \HOLOGO@Span{i}{I}%
      B%
    }%
    \hologo{TeX}%
  }%
}
%    \end{macrocode}
%    \end{macro}
%    \begin{macro}{\HoLogoCss@BibTeX@sf}
%    \begin{macrocode}
\def\HoLogoCss@BibTeX@sf{%
  \Css{%
    span.HoLogo-BibTeX-sf span.HoLogo-i{%
      margin-left:-.05em;%
      margin-right:-.025em;%
    }%
  }%
  \Css{%
    span.HoLogo-BibTeX-sf span.HoLogo-TeX{%
      margin-left:-.08em;%
    }%
  }%
  \global\let\HoLogoCss@BibTeX@sf\relax
}
%    \end{macrocode}
%    \end{macro}
%
%    \begin{macro}{\HoLogo@BibTeX}
%    \begin{macrocode}
\def\HoLogo@BibTeX{\HoLogo@BibTeX@sf}
%    \end{macrocode}
%    \end{macro}
%    \begin{macro}{\HoLogoHtml@BibTeX}
%    \begin{macrocode}
\def\HoLogoHtml@BibTeX{\HoLogoHtml@BibTeX@sf}
%    \end{macrocode}
%    \end{macro}
%
% \subsubsection{\hologo{BibTeX8}}
%
%    \begin{macro}{\HoLogo@BibTeX8}
%    \begin{macrocode}
\expandafter\def\csname HoLogo@BibTeX8\endcsname#1{%
  \hologo{BibTeX}%
  8%
}
%    \end{macrocode}
%    \end{macro}
%
%    \begin{macro}{\HoLogoBkm@BibTeX8}
%    \begin{macrocode}
\expandafter\def\csname HoLogoBkm@BibTeX8\endcsname#1{%
  \hologo{BibTeX}%
  8%
}
%    \end{macrocode}
%    \end{macro}
%    \begin{macro}{\HoLogoHtml@BibTeX8}
%    \begin{macrocode}
\expandafter
\let\csname HoLogoHtml@BibTeX8\expandafter\endcsname
\csname HoLogo@BibTeX8\endcsname
%    \end{macrocode}
%    \end{macro}
%
% \subsubsection{\hologo{ConTeXt}}
%
%    \begin{macro}{\HoLogo@ConTeXt@simple}
%    \begin{macrocode}
\def\HoLogo@ConTeXt@simple#1{%
  \HOLOGO@mbox{Con}%
  \HOLOGO@discretionary
  \HOLOGO@mbox{\hologo{TeX}t}%
}
%    \end{macrocode}
%    \end{macro}
%    \begin{macro}{\HoLogoHtml@ConTeXt@simple}
%    \begin{macrocode}
\let\HoLogoHtml@ConTeXt@simple\HoLogo@ConTeXt@simple
%    \end{macrocode}
%    \end{macro}
%
%    \begin{macro}{\HoLogo@ConTeXt@narrow}
%    This definition of logo \hologo{ConTeXt} with variant \xoption{narrow}
%    comes from TUGboat's class \xclass{ltugboat} (version 2010/11/15 v2.8).
%    \begin{macrocode}
\def\HoLogo@ConTeXt@narrow#1{%
  \HOLOGO@mbox{C\kern-.0333emon}%
  \HOLOGO@discretionary
  \kern-.0667em%
  \HOLOGO@mbox{\hologo{TeX}\kern-.0333emt}%
}
%    \end{macrocode}
%    \end{macro}
%    \begin{macro}{\HoLogoHtml@ConTeXt@narrow}
%    \begin{macrocode}
\def\HoLogoHtml@ConTeXt@narrow#1{%
  \HoLogoCss@ConTeXt@narrow
  \HOLOGO@Span{ConTeXt-narrow}{%
    \HOLOGO@Span{C}{C}%
    on%
    \hologo{TeX}%
    t%
  }%
}
%    \end{macrocode}
%    \end{macro}
%    \begin{macro}{\HoLogoCss@ConTeXt@narrow}
%    \begin{macrocode}
\def\HoLogoCss@ConTeXt@narrow{%
  \Css{%
    span.HoLogo-ConTeXt-narrow span.HoLogo-C{%
      margin-left:-.0333em;%
    }%
  }%
  \Css{%
    span.HoLogo-ConTeXt-narrow span.HoLogo-TeX{%
      margin-left:-.0667em;%
      margin-right:-.0333em;%
    }%
  }%
  \global\let\HoLogoCss@ConTeXt@narrow\relax
}
%    \end{macrocode}
%    \end{macro}
%
%    \begin{macro}{\HoLogo@ConTeXt}
%    \begin{macrocode}
\def\HoLogo@ConTeXt{\HoLogo@ConTeXt@narrow}
%    \end{macrocode}
%    \end{macro}
%    \begin{macro}{\HoLogoHtml@ConTeXt}
%    \begin{macrocode}
\def\HoLogoHtml@ConTeXt{\HoLogoHtml@ConTeXt@narrow}
%    \end{macrocode}
%    \end{macro}
%
% \subsubsection{\hologo{emTeX}}
%
%    \begin{macro}{\HoLogo@emTeX}
%    \begin{macrocode}
\def\HoLogo@emTeX#1{%
  \HOLOGO@mbox{#1{e}{E}m}%
  \HOLOGO@discretionary
  \hologo{TeX}%
}
%    \end{macrocode}
%    \end{macro}
%    \begin{macro}{\HoLogoCs@emTeX}
%    \begin{macrocode}
\def\HoLogoCs@emTeX#1{#1{e}{E}mTeX}%
%    \end{macrocode}
%    \end{macro}
%    \begin{macro}{\HoLogoBkm@emTeX}
%    \begin{macrocode}
\def\HoLogoBkm@emTeX#1{%
  #1{e}{E}m\hologo{TeX}%
}
%    \end{macrocode}
%    \end{macro}
%    \begin{macro}{\HoLogoHtml@emTeX}
%    \begin{macrocode}
\let\HoLogoHtml@emTeX\HoLogo@emTeX
%    \end{macrocode}
%    \end{macro}
%
% \subsubsection{\hologo{ExTeX}}
%
%    \begin{macro}{\HoLogo@ExTeX}
%    The definition is taken from the FAQ of the
%    project \hologo{ExTeX}
%    \cite{ExTeX-FAQ}.
%\begin{quote}
%\begin{verbatim}
%\def\ExTeX{%
%  \textrm{% Logo always with serifs
%    \ensuremath{%
%      \textstyle
%      \varepsilon_{%
%        \kern-0.15em%
%        \mathcal{X}%
%      }%
%    }%
%    \kern-.15em%
%    \TeX
%  }%
%}
%\end{verbatim}
%\end{quote}
%    \begin{macrocode}
\def\HoLogo@ExTeX#1{%
  \HoLogoFont@font{ExTeX}{rm}{%
    \ltx@mbox{%
      \HOLOGO@MathSetup
      $%
        \textstyle
        \varepsilon_{%
          \kern-0.15em%
          \HoLogoFont@font{ExTeX}{sy}{X}%
        }%
      $%
    }%
    \HOLOGO@discretionary
    \kern-.15em%
    \hologo{TeX}%
  }%
}
%    \end{macrocode}
%    \end{macro}
%    \begin{macro}{\HoLogoHtml@ExTeX}
%    \begin{macrocode}
\def\HoLogoHtml@ExTeX#1{%
  \HoLogoCss@ExTeX
  \HoLogoFont@font{ExTeX}{rm}{%
    \HOLOGO@Span{ExTeX}{%
      \ltx@mbox{%
        \HOLOGO@MathSetup
        $\textstyle\varepsilon$%
        \HOLOGO@Span{X}{$\textstyle\chi$}%
        \hologo{TeX}%
      }%
    }%
  }%
}
%    \end{macrocode}
%    \end{macro}
%    \begin{macro}{\HoLogoBkm@ExTeX}
%    \begin{macrocode}
\def\HoLogoBkm@ExTeX#1{%
  \HOLOGO@PdfdocUnicode{#1{e}{E}x}{\textepsilon\textchi}%
  \hologo{TeX}%
}
%    \end{macrocode}
%    \end{macro}
%    \begin{macro}{\HoLogoCss@ExTeX}
%    \begin{macrocode}
\def\HoLogoCss@ExTeX{%
  \Css{%
    span.HoLogo-ExTeX{%
      font-family:serif;%
    }%
  }%
  \Css{%
    span.HoLogo-ExTeX span.HoLogo-TeX{%
      margin-left:-.15em;%
    }%
  }%
  \global\let\HoLogoCss@ExTeX\relax
}
%    \end{macrocode}
%    \end{macro}
%
% \subsubsection{\hologo{MiKTeX}}
%
%    \begin{macro}{\HoLogo@MiKTeX}
%    \begin{macrocode}
\def\HoLogo@MiKTeX#1{%
  \HOLOGO@mbox{MiK}%
  \HOLOGO@discretionary
  \hologo{TeX}%
}
%    \end{macrocode}
%    \end{macro}
%    \begin{macro}{\HoLogoHtml@MiKTeX}
%    \begin{macrocode}
\let\HoLogoHtml@MiKTeX\HoLogo@MiKTeX
%    \end{macrocode}
%    \end{macro}
%
% \subsubsection{\hologo{OzTeX} and friends}
%
%    Source: \hologo{OzTeX} FAQ \cite{OzTeX}:
%    \begin{quote}
%      |\def\OzTeX{O\kern-.03em z\kern-.15em\TeX}|\\
%      (There is no kerning in OzMF, OzMP and OzTtH.)
%    \end{quote}
%
%    \begin{macro}{\HoLogo@OzTeX}
%    \begin{macrocode}
\def\HoLogo@OzTeX#1{%
  O%
  \kern-.03em %
  z%
  \kern-.15em %
  \hologo{TeX}%
}
%    \end{macrocode}
%    \end{macro}
%    \begin{macro}{\HoLogoHtml@OzTeX}
%    \begin{macrocode}
\def\HoLogoHtml@OzTeX#1{%
  \HoLogoCss@OzTeX
  \HOLOGO@Span{OzTeX}{%
    O%
    \HOLOGO@Span{z}{z}%
    \hologo{TeX}%
  }%
}
%    \end{macrocode}
%    \end{macro}
%    \begin{macro}{\HoLogoCss@OzTeX}
%    \begin{macrocode}
\def\HoLogoCss@OzTeX{%
  \Css{%
    span.HoLogo-OzTeX span.HoLogo-z{%
      margin-left:-.03em;%
      margin-right:-.15em;%
    }%
  }%
  \global\let\HoLogoCss@OzTeX\relax
}
%    \end{macrocode}
%    \end{macro}
%
%    \begin{macro}{\HoLogo@OzMF}
%    \begin{macrocode}
\def\HoLogo@OzMF#1{%
  \HOLOGO@mbox{OzMF}%
}
%    \end{macrocode}
%    \end{macro}
%    \begin{macro}{\HoLogo@OzMP}
%    \begin{macrocode}
\def\HoLogo@OzMP#1{%
  \HOLOGO@mbox{OzMP}%
}
%    \end{macrocode}
%    \end{macro}
%    \begin{macro}{\HoLogo@OzTtH}
%    \begin{macrocode}
\def\HoLogo@OzTtH#1{%
  \HOLOGO@mbox{OzTtH}%
}
%    \end{macrocode}
%    \end{macro}
%
% \subsubsection{\hologo{PCTeX}}
%
%    \begin{macro}{\HoLogo@PCTeX}
%    \begin{macrocode}
\def\HoLogo@PCTeX#1{%
  \HOLOGO@mbox{PC}%
  \hologo{TeX}%
}
%    \end{macrocode}
%    \end{macro}
%    \begin{macro}{\HoLogoHtml@PCTeX}
%    \begin{macrocode}
\let\HoLogoHtml@PCTeX\HoLogo@PCTeX
%    \end{macrocode}
%    \end{macro}
%
% \subsubsection{\hologo{PiCTeX}}
%
%    The original definitions from \xfile{pictex.tex} \cite{PiCTeX}:
%\begin{quote}
%\begin{verbatim}
%\def\PiC{%
%  P%
%  \kern-.12em%
%  \lower.5ex\hbox{I}%
%  \kern-.075em%
%  C%
%}
%\def\PiCTeX{%
%  \PiC
%  \kern-.11em%
%  \TeX
%}
%\end{verbatim}
%\end{quote}
%
%    \begin{macro}{\HoLogo@PiC}
%    \begin{macrocode}
\def\HoLogo@PiC#1{%
  P%
  \kern-.12em%
  \lower.5ex\hbox{I}%
  \kern-.075em%
  C%
  \HOLOGO@SpaceFactor
}
%    \end{macrocode}
%    \end{macro}
%    \begin{macro}{\HoLogoHtml@PiC}
%    \begin{macrocode}
\def\HoLogoHtml@PiC#1{%
  \HoLogoCss@PiC
  \HOLOGO@Span{PiC}{%
    P%
    \HOLOGO@Span{i}{I}%
    C%
  }%
}
%    \end{macrocode}
%    \end{macro}
%    \begin{macro}{\HoLogoCss@PiC}
%    \begin{macrocode}
\def\HoLogoCss@PiC{%
  \Css{%
    span.HoLogo-PiC span.HoLogo-i{%
      position:relative;%
      top:.5ex;%
      margin-left:-.12em;%
      margin-right:-.075em;%
      text-decoration:none;%
    }%
  }%
  \global\let\HoLogoCss@PiC\relax
}
%    \end{macrocode}
%    \end{macro}
%
%    \begin{macro}{\HoLogo@PiCTeX}
%    \begin{macrocode}
\def\HoLogo@PiCTeX#1{%
  \hologo{PiC}%
  \HOLOGO@discretionary
  \kern-.11em%
  \hologo{TeX}%
}
%    \end{macrocode}
%    \end{macro}
%    \begin{macro}{\HoLogoHtml@PiCTeX}
%    \begin{macrocode}
\def\HoLogoHtml@PiCTeX#1{%
  \HoLogoCss@PiCTeX
  \HOLOGO@Span{PiCTeX}{%
    \hologo{PiC}%
    \hologo{TeX}%
  }%
}
%    \end{macrocode}
%    \end{macro}
%    \begin{macro}{\HoLogoCss@PiCTeX}
%    \begin{macrocode}
\def\HoLogoCss@PiCTeX{%
  \Css{%
    span.HoLogo-PiCTeX span.HoLogo-PiC{%
      margin-right:-.11em;%
    }%
  }%
  \global\let\HoLogoCss@PiCTeX\relax
}
%    \end{macrocode}
%    \end{macro}
%
% \subsubsection{\hologo{teTeX}}
%
%    \begin{macro}{\HoLogo@teTeX}
%    \begin{macrocode}
\def\HoLogo@teTeX#1{%
  \HOLOGO@mbox{#1{t}{T}e}%
  \HOLOGO@discretionary
  \hologo{TeX}%
}
%    \end{macrocode}
%    \end{macro}
%    \begin{macro}{\HoLogoCs@teTeX}
%    \begin{macrocode}
\def\HoLogoCs@teTeX#1{#1{t}{T}dfTeX}
%    \end{macrocode}
%    \end{macro}
%    \begin{macro}{\HoLogoBkm@teTeX}
%    \begin{macrocode}
\def\HoLogoBkm@teTeX#1{%
  #1{t}{T}e\hologo{TeX}%
}
%    \end{macrocode}
%    \end{macro}
%    \begin{macro}{\HoLogoHtml@teTeX}
%    \begin{macrocode}
\let\HoLogoHtml@teTeX\HoLogo@teTeX
%    \end{macrocode}
%    \end{macro}
%
% \subsubsection{\hologo{TeX4ht}}
%
%    \begin{macro}{\HoLogo@TeX4ht}
%    \begin{macrocode}
\expandafter\def\csname HoLogo@TeX4ht\endcsname#1{%
  \HOLOGO@mbox{\hologo{TeX}4ht}%
}
%    \end{macrocode}
%    \end{macro}
%    \begin{macro}{\HoLogoHtml@TeX4ht}
%    \begin{macrocode}
\expandafter
\let\csname HoLogoHtml@TeX4ht\expandafter\endcsname
\csname HoLogo@TeX4ht\endcsname
%    \end{macrocode}
%    \end{macro}
%
%
% \subsubsection{\hologo{SageTeX}}
%
%    \begin{macro}{\HoLogo@SageTeX}
%    \begin{macrocode}
\def\HoLogo@SageTeX#1{%
  \HOLOGO@mbox{Sage}%
  \HOLOGO@discretionary
  \HOLOGO@NegativeKerning{eT,oT,To}%
  \hologo{TeX}%
}
%    \end{macrocode}
%    \end{macro}
%    \begin{macro}{\HoLogoHtml@SageTeX}
%    \begin{macrocode}
\let\HoLogoHtml@SageTeX\HoLogo@SageTeX
%    \end{macrocode}
%    \end{macro}
%
% \subsection{\hologo{METAFONT} and friends}
%
%    \begin{macro}{\HoLogo@METAFONT}
%    \begin{macrocode}
\def\HoLogo@METAFONT#1{%
  \HoLogoFont@font{METAFONT}{logo}{%
    \HOLOGO@mbox{META}%
    \HOLOGO@discretionary
    \HOLOGO@mbox{FONT}%
  }%
}
%    \end{macrocode}
%    \end{macro}
%
%    \begin{macro}{\HoLogo@METAPOST}
%    \begin{macrocode}
\def\HoLogo@METAPOST#1{%
  \HoLogoFont@font{METAPOST}{logo}{%
    \HOLOGO@mbox{META}%
    \HOLOGO@discretionary
    \HOLOGO@mbox{POST}%
  }%
}
%    \end{macrocode}
%    \end{macro}
%
%    \begin{macro}{\HoLogo@MetaFun}
%    \begin{macrocode}
\def\HoLogo@MetaFun#1{%
  \HOLOGO@mbox{Meta}%
  \HOLOGO@discretionary
  \HOLOGO@mbox{Fun}%
}
%    \end{macrocode}
%    \end{macro}
%
%    \begin{macro}{\HoLogo@MetaPost}
%    \begin{macrocode}
\def\HoLogo@MetaPost#1{%
  \HOLOGO@mbox{Meta}%
  \HOLOGO@discretionary
  \HOLOGO@mbox{Post}%
}
%    \end{macrocode}
%    \end{macro}
%
% \subsection{Others}
%
% \subsubsection{\hologo{biber}}
%
%    \begin{macro}{\HoLogo@biber}
%    \begin{macrocode}
\def\HoLogo@biber#1{%
  \HOLOGO@mbox{#1{b}{B}i}%
  \HOLOGO@discretionary
  \HOLOGO@mbox{ber}%
}
%    \end{macrocode}
%    \end{macro}
%    \begin{macro}{\HoLogoCs@biber}
%    \begin{macrocode}
\def\HoLogoCs@biber#1{#1{b}{B}iber}
%    \end{macrocode}
%    \end{macro}
%    \begin{macro}{\HoLogoBkm@biber}
%    \begin{macrocode}
\def\HoLogoBkm@biber#1{%
  #1{b}{B}iber%
}
%    \end{macrocode}
%    \end{macro}
%    \begin{macro}{\HoLogoHtml@biber}
%    \begin{macrocode}
\let\HoLogoHtml@biber\HoLogo@biber
%    \end{macrocode}
%    \end{macro}
%
% \subsubsection{\hologo{KOMAScript}}
%
%    \begin{macro}{\HoLogo@KOMAScript}
%    The definition for \hologo{KOMAScript} is taken
%    from \hologo{KOMAScript} (\xfile{scrlogo.dtx}, reformatted) \cite{scrlogo}:
%\begin{quote}
%\begin{verbatim}
%\@ifundefined{KOMAScript}{%
%  \DeclareRobustCommand{\KOMAScript}{%
%    \textsf{%
%      K\kern.05em O\kern.05emM\kern.05em A%
%      \kern.1em-\kern.1em %
%      Script%
%    }%
%  }%
%}{}
%\end{verbatim}
%\end{quote}
%    \begin{macrocode}
\def\HoLogo@KOMAScript#1{%
  \HoLogoFont@font{KOMAScript}{sf}{%
    \HOLOGO@mbox{%
      K\kern.05em%
      O\kern.05em%
      M\kern.05em%
      A%
    }%
    \kern.1em%
    \HOLOGO@hyphen
    \kern.1em%
    \HOLOGO@mbox{Script}%
  }%
}
%    \end{macrocode}
%    \end{macro}
%    \begin{macro}{\HoLogoBkm@KOMAScript}
%    \begin{macrocode}
\def\HoLogoBkm@KOMAScript#1{%
  KOMA-Script%
}
%    \end{macrocode}
%    \end{macro}
%    \begin{macro}{\HoLogoHtml@KOMAScript}
%    \begin{macrocode}
\def\HoLogoHtml@KOMAScript#1{%
  \HoLogoCss@KOMAScript
  \HoLogoFont@font{KOMAScript}{sf}{%
    \HOLOGO@Span{KOMAScript}{%
      K%
      \HOLOGO@Span{O}{O}%
      M%
      \HOLOGO@Span{A}{A}%
      \HOLOGO@Span{hyphen}{-}%
      Script%
    }%
  }%
}
%    \end{macrocode}
%    \end{macro}
%    \begin{macro}{\HoLogoCss@KOMAScript}
%    \begin{macrocode}
\def\HoLogoCss@KOMAScript{%
  \Css{%
    span.HoLogo-KOMAScript{%
      font-family:sans-serif;%
    }%
  }%
  \Css{%
    span.HoLogo-KOMAScript span.HoLogo-O{%
      padding-left:.05em;%
      padding-right:.05em;%
    }%
  }%
  \Css{%
    span.HoLogo-KOMAScript span.HoLogo-A{%
      padding-left:.05em;%
    }%
  }%
  \Css{%
    span.HoLogo-KOMAScript span.HoLogo-hyphen{%
      padding-left:.1em;%
      padding-right:.1em;%
    }%
  }%
  \global\let\HoLogoCss@KOMAScript\relax
}
%    \end{macrocode}
%    \end{macro}
%
% \subsubsection{\hologo{LyX}}
%
%    \begin{macro}{\HoLogo@LyX}
%    The definition is taken from the documentation source files
%    of \hologo{LyX}, \xfile{Intro.lyx} \cite{LyX}:
%\begin{quote}
%\begin{verbatim}
%\def\LyX{%
%  \texorpdfstring{%
%    L\kern-.1667em\lower.25em\hbox{Y}\kern-.125emX\@%
%  }{%
%    LyX%
%  }%
%}
%\end{verbatim}
%\end{quote}
%    \begin{macrocode}
\def\HoLogo@LyX#1{%
  L%
  \kern-.1667em%
  \lower.25em\hbox{Y}%
  \kern-.125em%
  X%
  \HOLOGO@SpaceFactor
}
%    \end{macrocode}
%    \end{macro}
%    \begin{macro}{\HoLogoHtml@LyX}
%    \begin{macrocode}
\def\HoLogoHtml@LyX#1{%
  \HoLogoCss@LyX
  \HOLOGO@Span{LyX}{%
    L%
    \HOLOGO@Span{y}{Y}%
    X%
  }%
}
%    \end{macrocode}
%    \end{macro}
%    \begin{macro}{\HoLogoCss@LyX}
%    \begin{macrocode}
\def\HoLogoCss@LyX{%
  \Css{%
    span.HoLogo-LyX span.HoLogo-y{%
      position:relative;%
      top:.25em;%
      margin-left:-.1667em;%
      margin-right:-.125em;%
      text-decoration:none;%
    }%
  }%
  \global\let\HoLogoCss@LyX\relax
}
%    \end{macrocode}
%    \end{macro}
%
% \subsubsection{\hologo{NTS}}
%
%    \begin{macro}{\HoLogo@NTS}
%    Definition for \hologo{NTS} can be found in
%    package \xpackage{etex\textunderscore man} for the \hologo{eTeX} manual \cite{etexman}
%    and in package \xpackage{dtklogos} \cite{dtklogos}:
%\begin{quote}
%\begin{verbatim}
%\def\NTS{%
%  \leavevmode
%  \hbox{%
%    $%
%      \cal N%
%      \kern-0.35em%
%      \lower0.5ex\hbox{$\cal T$}%
%      \kern-0.2em%
%      S%
%    $%
%  }%
%}
%\end{verbatim}
%\end{quote}
%    \begin{macrocode}
\def\HoLogo@NTS#1{%
  \HoLogoFont@font{NTS}{sy}{%
    N\/%
    \kern-.35em%
    \lower.5ex\hbox{T\/}%
    \kern-.2em%
    S\/%
  }%
  \HOLOGO@SpaceFactor
}
%    \end{macrocode}
%    \end{macro}
%
% \subsubsection{\Hologo{TTH} (\hologo{TeX} to HTML translator)}
%
%    Source: \url{http://hutchinson.belmont.ma.us/tth/}
%    In the HTML source the second `T' is printed as subscript.
%\begin{quote}
%\begin{verbatim}
%T<sub>T</sub>H
%\end{verbatim}
%\end{quote}
%    \begin{macro}{\HoLogo@TTH}
%    \begin{macrocode}
\def\HoLogo@TTH#1{%
  \ltx@mbox{%
    T\HOLOGO@SubScript{T}H%
  }%
  \HOLOGO@SpaceFactor
}
%    \end{macrocode}
%    \end{macro}
%
%    \begin{macro}{\HoLogoHtml@TTH}
%    \begin{macrocode}
\def\HoLogoHtml@TTH#1{%
  T\HCode{<sub>}T\HCode{</sub>}H%
}
%    \end{macrocode}
%    \end{macro}
%
% \subsubsection{\Hologo{HanTheThanh}}
%
%    Partial source: Package \xpackage{dtklogos}.
%    The double accent is U+1EBF (latin small letter e with circumflex
%    and acute).
%    \begin{macro}{\HoLogo@HanTheThanh}
%    \begin{macrocode}
\def\HoLogo@HanTheThanh#1{%
  \ltx@mbox{H\`an}%
  \HOLOGO@space
  \ltx@mbox{%
    Th%
    \HOLOGO@IfCharExists{"1EBF}{%
      \char"1EBF\relax
    }{%
      \^e\hbox to 0pt{\hss\raise .5ex\hbox{\'{}}}%
    }%
  }%
  \HOLOGO@space
  \ltx@mbox{Th\`anh}%
}
%    \end{macrocode}
%    \end{macro}
%    \begin{macro}{\HoLogoBkm@HanTheThanh}
%    \begin{macrocode}
\def\HoLogoBkm@HanTheThanh#1{%
  H\`an %
  Th\HOLOGO@PdfdocUnicode{\^e}{\9036\277} %
  Th\`anh%
}
%    \end{macrocode}
%    \end{macro}
%    \begin{macro}{\HoLogoHtml@HanTheThanh}
%    \begin{macrocode}
\def\HoLogoHtml@HanTheThanh#1{%
  H\`an %
  Th\HCode{&\ltx@hashchar x1ebf;} %
  Th\`anh%
}
%    \end{macrocode}
%    \end{macro}
%
% \subsection{Driver detection}
%
%    \begin{macrocode}
\HOLOGO@IfExists\InputIfFileExists{%
  \InputIfFileExists{hologo.cfg}{}{}%
}{%
  \ltx@IfUndefined{pdf@filesize}{%
    \def\HOLOGO@InputIfExists{%
      \openin\HOLOGO@temp=hologo.cfg\relax
      \ifeof\HOLOGO@temp
        \closein\HOLOGO@temp
      \else
        \closein\HOLOGO@temp
        \begingroup
          \def\x{LaTeX2e}%
        \expandafter\endgroup
        \ifx\fmtname\x
          \input{hologo.cfg}%
        \else
          \input hologo.cfg\relax
        \fi
      \fi
    }%
    \ltx@IfUndefined{newread}{%
      \chardef\HOLOGO@temp=15 %
      \def\HOLOGO@CheckRead{%
        \ifeof\HOLOGO@temp
          \HOLOGO@InputIfExists
        \else
          \ifcase\HOLOGO@temp
            \@PackageWarningNoLine{hologo}{%
              Configuration file ignored, because\MessageBreak
              a free read register could not be found%
            }%
          \else
            \begingroup
              \count\ltx@cclv=\HOLOGO@temp
              \advance\ltx@cclv by \ltx@minusone
              \edef\x{\endgroup
                \chardef\noexpand\HOLOGO@temp=\the\count\ltx@cclv
                \relax
              }%
            \x
          \fi
        \fi
      }%
    }{%
      \csname newread\endcsname\HOLOGO@temp
      \HOLOGO@InputIfExists
    }%
  }{%
    \edef\HOLOGO@temp{\pdf@filesize{hologo.cfg}}%
    \ifx\HOLOGO@temp\ltx@empty
    \else
      \ifnum\HOLOGO@temp>0 %
        \begingroup
          \def\x{LaTeX2e}%
        \expandafter\endgroup
        \ifx\fmtname\x
          \input{hologo.cfg}%
        \else
          \input hologo.cfg\relax
        \fi
      \else
        \@PackageInfoNoLine{hologo}{%
          Empty configuration file `hologo.cfg' ignored%
        }%
      \fi
    \fi
  }%
}
%    \end{macrocode}
%
%    \begin{macrocode}
\def\HOLOGO@temp#1#2{%
  \kv@define@key{HoLogoDriver}{#1}[]{%
    \begingroup
      \def\HOLOGO@temp{##1}%
      \ltx@onelevel@sanitize\HOLOGO@temp
      \ifx\HOLOGO@temp\ltx@empty
      \else
        \@PackageError{hologo}{%
          Value (\HOLOGO@temp) not permitted for option `#1'%
        }%
        \@ehc
      \fi
    \endgroup
    \def\hologoDriver{#2}%
  }%
}%
\def\HOLOGO@@temp#1#2{%
  \ifx\kv@value\relax
    \HOLOGO@temp{#1}{#1}%
  \else
    \HOLOGO@temp{#1}{#2}%
  \fi
}%
\kv@parse@normalized{%
  pdftex,%
  luatex=pdftex,%
  dvipdfm,%
  dvipdfmx=dvipdfm,%
  dvips,%
  dvipsone=dvips,%
  xdvi=dvips,%
  xetex,%
  vtex,%
}\HOLOGO@@temp
%    \end{macrocode}
%
%    \begin{macrocode}
\kv@define@key{HoLogoDriver}{driverfallback}{%
  \def\HOLOGO@DriverFallback{#1}%
}
%    \end{macrocode}
%
%    \begin{macro}{\HOLOGO@DriverFallback}
%    \begin{macrocode}
\def\HOLOGO@DriverFallback{dvips}
%    \end{macrocode}
%    \end{macro}
%
%    \begin{macro}{\hologoDriverSetup}
%    \begin{macrocode}
\def\hologoDriverSetup{%
  \let\hologoDriver\ltx@undefined
  \HOLOGO@DriverSetup
}
%    \end{macrocode}
%    \end{macro}
%
%    \begin{macro}{\HOLOGO@DriverSetup}
%    \begin{macrocode}
\def\HOLOGO@DriverSetup#1{%
  \kvsetkeys{HoLogoDriver}{#1}%
  \HOLOGO@CheckDriver
  \ltx@ifundefined{hologoDriver}{%
    \begingroup
    \edef\x{\endgroup
      \noexpand\kvsetkeys{HoLogoDriver}{\HOLOGO@DriverFallback}%
    }\x
  }{}%
  \@PackageInfoNoLine{hologo}{Using driver `\hologoDriver'}%
}
%    \end{macrocode}
%    \end{macro}
%
%    \begin{macro}{\HOLOGO@CheckDriver}
%    \begin{macrocode}
\def\HOLOGO@CheckDriver{%
  \ifpdf
    \def\hologoDriver{pdftex}%
    \let\HOLOGO@pdfliteral\pdfliteral
    \ifluatex
      \ifx\pdfextension\@undefined\else
        \protected\def\pdfliteral{\pdfextension literal}%
        \let\HOLOGO@pdfliteral\pdfliteral
      \fi
      \ltx@IfUndefined{HOLOGO@pdfliteral}{%
        \ifnum\luatexversion<36 %
        \else
          \begingroup
            \let\HOLOGO@temp\endgroup
            \ifcase0%
                \directlua{%
                  if tex.enableprimitives then %
                    tex.enableprimitives('HOLOGO@', {'pdfliteral'})%
                  else %
                    tex.print('1')%
                  end%
                }%
                \ifx\HOLOGO@pdfliteral\@undefined 1\fi%
                \relax%
              \endgroup
              \let\HOLOGO@temp\relax
              \global\let\HOLOGO@pdfliteral\HOLOGO@pdfliteral
            \fi%
          \HOLOGO@temp
        \fi
      }{}%
    \fi
    \ltx@IfUndefined{HOLOGO@pdfliteral}{%
      \@PackageWarningNoLine{hologo}{%
        Cannot find \string\pdfliteral
      }%
    }{}%
  \else
    \ifxetex
      \def\hologoDriver{xetex}%
    \else
      \ifvtex
        \def\hologoDriver{vtex}%
      \fi
    \fi
  \fi
}
%    \end{macrocode}
%    \end{macro}
%
%    \begin{macro}{\HOLOGO@WarningUnsupportedDriver}
%    \begin{macrocode}
\def\HOLOGO@WarningUnsupportedDriver#1{%
  \@PackageWarningNoLine{hologo}{%
    Logo `#1' needs driver specific macros,\MessageBreak
    but driver `\hologoDriver' is not supported.\MessageBreak
    Use a different driver or\MessageBreak
    load package `graphics' or `pgf'%
  }%
}
%    \end{macrocode}
%    \end{macro}
%
% \subsubsection{Reflect box macros}
%
%    Skip driver part if not needed.
%    \begin{macrocode}
\ltx@IfUndefined{reflectbox}{}{%
  \ltx@IfUndefined{rotatebox}{}{%
    \HOLOGO@AtEnd
  }%
}
\ltx@IfUndefined{pgftext}{}{%
  \HOLOGO@AtEnd
}
\ltx@IfUndefined{psscalebox}{}{%
  \HOLOGO@AtEnd
}
%    \end{macrocode}
%
%    \begin{macrocode}
\def\HOLOGO@temp{LaTeX2e}
\ifx\fmtname\HOLOGO@temp
  \RequirePackage{kvoptions}[2011/06/30]%
  \ProcessKeyvalOptions{HoLogoDriver}%
\fi
\HOLOGO@DriverSetup{}
%    \end{macrocode}
%
%    \begin{macro}{\HOLOGO@ReflectBox}
%    \begin{macrocode}
\def\HOLOGO@ReflectBox#1{%
  \begingroup
    \setbox\ltx@zero\hbox{\begingroup#1\endgroup}%
    \setbox\ltx@two\hbox{%
      \kern\wd\ltx@zero
      \csname HOLOGO@ScaleBox@\hologoDriver\endcsname{-1}{1}{%
        \hbox to 0pt{\copy\ltx@zero\hss}%
      }%
    }%
    \wd\ltx@two=\wd\ltx@zero
    \box\ltx@two
  \endgroup
}
%    \end{macrocode}
%    \end{macro}
%
%    \begin{macro}{\HOLOGO@PointReflectBox}
%    \begin{macrocode}
\def\HOLOGO@PointReflectBox#1{%
  \begingroup
    \setbox\ltx@zero\hbox{\begingroup#1\endgroup}%
    \setbox\ltx@two\hbox{%
      \kern\wd\ltx@zero
      \raise\ht\ltx@zero\hbox{%
        \csname HOLOGO@ScaleBox@\hologoDriver\endcsname{-1}{-1}{%
          \hbox to 0pt{\copy\ltx@zero\hss}%
        }%
      }%
    }%
    \wd\ltx@two=\wd\ltx@zero
    \box\ltx@two
  \endgroup
}
%    \end{macrocode}
%    \end{macro}
%
%    We must define all variants because of dynamic driver setup.
%    \begin{macrocode}
\def\HOLOGO@temp#1#2{#2}
%    \end{macrocode}
%
%    \begin{macro}{\HOLOGO@ScaleBox@pdftex}
%    \begin{macrocode}
\HOLOGO@temp{pdftex}{%
  \def\HOLOGO@ScaleBox@pdftex#1#2#3{%
    \HOLOGO@pdfliteral{%
      q #1 0 0 #2 0 0 cm%
    }%
    #3%
    \HOLOGO@pdfliteral{%
      Q%
    }%
  }%
}
%    \end{macrocode}
%    \end{macro}
%    \begin{macro}{\HOLOGO@ScaleBox@dvips}
%    \begin{macrocode}
\HOLOGO@temp{dvips}{%
  \def\HOLOGO@ScaleBox@dvips#1#2#3{%
    \special{ps:%
      gsave %
      currentpoint %
      currentpoint translate %
      #1 #2 scale %
      neg exch neg exch translate%
    }%
    #3%
    \special{ps:%
      currentpoint %
      grestore %
      moveto%
    }%
  }%
}
%    \end{macrocode}
%    \end{macro}
%    \begin{macro}{\HOLOGO@ScaleBox@dvipdfm}
%    \begin{macrocode}
\HOLOGO@temp{dvipdfm}{%
  \let\HOLOGO@ScaleBox@dvipdfm\HOLOGO@ScaleBox@dvips
}
%    \end{macrocode}
%    \end{macro}
%    Since \hologo{XeTeX} v0.6.
%    \begin{macro}{\HOLOGO@ScaleBox@xetex}
%    \begin{macrocode}
\HOLOGO@temp{xetex}{%
  \def\HOLOGO@ScaleBox@xetex#1#2#3{%
    \special{x:gsave}%
    \special{x:scale #1 #2}%
    #3%
    \special{x:grestore}%
  }%
}
%    \end{macrocode}
%    \end{macro}
%    \begin{macro}{\HOLOGO@ScaleBox@vtex}
%    \begin{macrocode}
\HOLOGO@temp{vtex}{%
  \def\HOLOGO@ScaleBox@vtex#1#2#3{%
    \special{r(#1,0,0,#2,0,0}%
    #3%
    \special{r)}%
  }%
}
%    \end{macrocode}
%    \end{macro}
%
%    \begin{macrocode}
\HOLOGO@AtEnd%
%</package>
%    \end{macrocode}
%
% \section{Test}
%
% \subsection{Catcode checks for loading}
%
%    \begin{macrocode}
%<*test1>
%    \end{macrocode}
%    \begin{macrocode}
\catcode`\{=1 %
\catcode`\}=2 %
\catcode`\#=6 %
\catcode`\@=11 %
\expandafter\ifx\csname count@\endcsname\relax
  \countdef\count@=255 %
\fi
\expandafter\ifx\csname @gobble\endcsname\relax
  \long\def\@gobble#1{}%
\fi
\expandafter\ifx\csname @firstofone\endcsname\relax
  \long\def\@firstofone#1{#1}%
\fi
\expandafter\ifx\csname loop\endcsname\relax
  \expandafter\@firstofone
\else
  \expandafter\@gobble
\fi
{%
  \def\loop#1\repeat{%
    \def\body{#1}%
    \iterate
  }%
  \def\iterate{%
    \body
      \let\next\iterate
    \else
      \let\next\relax
    \fi
    \next
  }%
  \let\repeat=\fi
}%
\def\RestoreCatcodes{}
\count@=0 %
\loop
  \edef\RestoreCatcodes{%
    \RestoreCatcodes
    \catcode\the\count@=\the\catcode\count@\relax
  }%
\ifnum\count@<255 %
  \advance\count@ 1 %
\repeat

\def\RangeCatcodeInvalid#1#2{%
  \count@=#1\relax
  \loop
    \catcode\count@=15 %
  \ifnum\count@<#2\relax
    \advance\count@ 1 %
  \repeat
}
\def\RangeCatcodeCheck#1#2#3{%
  \count@=#1\relax
  \loop
    \ifnum#3=\catcode\count@
    \else
      \errmessage{%
        Character \the\count@\space
        with wrong catcode \the\catcode\count@\space
        instead of \number#3%
      }%
    \fi
  \ifnum\count@<#2\relax
    \advance\count@ 1 %
  \repeat
}
\def\space{ }
\expandafter\ifx\csname LoadCommand\endcsname\relax
  \def\LoadCommand{\input hologo.sty\relax}%
\fi
\def\Test{%
  \RangeCatcodeInvalid{0}{47}%
  \RangeCatcodeInvalid{58}{64}%
  \RangeCatcodeInvalid{91}{96}%
  \RangeCatcodeInvalid{123}{255}%
  \catcode`\@=12 %
  \catcode`\\=0 %
  \catcode`\%=14 %
  \LoadCommand
  \RangeCatcodeCheck{0}{36}{15}%
  \RangeCatcodeCheck{37}{37}{14}%
  \RangeCatcodeCheck{38}{47}{15}%
  \RangeCatcodeCheck{48}{57}{12}%
  \RangeCatcodeCheck{58}{63}{15}%
  \RangeCatcodeCheck{64}{64}{12}%
  \RangeCatcodeCheck{65}{90}{11}%
  \RangeCatcodeCheck{91}{91}{15}%
  \RangeCatcodeCheck{92}{92}{0}%
  \RangeCatcodeCheck{93}{96}{15}%
  \RangeCatcodeCheck{97}{122}{11}%
  \RangeCatcodeCheck{123}{255}{15}%
  \RestoreCatcodes
}
\Test
\csname @@end\endcsname
\end
%    \end{macrocode}
%    \begin{macrocode}
%</test1>
%    \end{macrocode}
%
% \subsection{Spacefactor}
%
%    The space factor must be 1000 after a logo. If it is greater 1000
%    then the following space is a space after a sentence closing point.
%    If the space factor is smaller 1000 then an immediate following
%    dot is interpreted as abbreviation, not sentence closing point.
%
%    \begin{macrocode}
%<*test-spacefactor>
\NeedsTeXFormat{LaTeX2e}
\documentclass{article}
\usepackage{hologo}[2016/05/12]
\usepackage{kvsetkeys}
\usepackage{qstest}
\IncludeTests{*}
\LogTests{log}{*}{*}
\begin{document}
\begin{qstest}{spacefactor}{spacefactor}
\newcommand*{\Test}[1]{%
  \sbox0{%
    \hologo{#1}%
    \Expect*{1000 (#1)}*{\the\spacefactor\space(#1)}%
  }%
}%
\makeatletter
\def\TestList{}
\def\hologoEntry#1#2#3{%
  \edef\TestList{%
    \ifx\TestList\@empty
    \else
      \TestList,%
    \fi
    #1%
    \ifx\\#2\\%
    \else
      ={variant=#2}%
    \fi
  }%
}
\hologoList
\expandafter\kv@parse@normalized\expandafter{%
  \TestList
}{%
  \begingroup
    \let\@logo=\kv@key
    \ifx\kv@value\relax
    \else
      \expandafter\hologoLogoSetup\expandafter\@logo\expandafter{%
        \kv@value
      }%
    \fi
    \Test\@logo
  \endgroup
  \@gobbletwo
}
\end{qstest}
\end{document}
%</test-spacefactor>
%    \end{macrocode}
%
% \subsection{Complete list}
%
%    \begin{macrocode}
%<*test-list>
\NeedsTeXFormat{LaTeX2e}
\documentclass[12pt,a4paper]{article}
\usepackage{hologo}[2016/05/12]
\usepackage[T1]{fontenc}
\usepackage{lmodern}
\usepackage{parskip}
\usepackage[unicode]{hyperref}[2011/09/28]
\usepackage{bookmark}[2011/09/19]
\bookmarksetup{%
  numbered,%
  open,%
  openlevel=2,%
}
\renewcommand*{\contentsname}{List of logos}
\begin{document}
\tableofcontents
\def\TestFont#1#2#3#4#5#6{%
  \begingroup
    \usefont{#3}{#4}{#5}{#6}%
    \HologoVariant{#1}{#2}/\hologoVariant{#1}{#2}%
    \quad
    \begingroup\scriptsize\hologoVariant{#1}{#2}\endgroup
    \quad
  \endgroup
  (#3/#4/#5/#6)%
  \par
}
\makeatletter
\def\hologoEntry#1#2#3{%
  \section{%
    \HologoVariant{#1}{#2}/\hologoVariant{#1}{#2} %
    {[#1\ifx\\#2\\\else\space(#2)\fi]}% hash-ok
  }% braces around [] because of bug in tex4ht
  \begingroup
    \hypersetup{unicode=false}%
    \bookmark[%
      dest=\@currentHref,%
      rellevel=1,%
      keeplevel,%
    ]{%
      \HologoVariant{#1}{#2}/\hologoVariant{#1}{#2} %
      (PDFDocEncoding)%
    }%
  \endgroup
  \TestFont{#1}{#2}{OT1}{cmr}{m}{n}%
  \TestFont{#1}{#2}{OT1}{cmss}{m}{n}%
  \TestFont{#1}{#2}{OT1}{cmr}{b}{n}%
  \TestFont{#1}{#2}{OT1}{cmr}{m}{it}%
  \TestFont{#1}{#2}{OT1}{cmtt}{m}{n}%
  \TestFont{#1}{#2}{T1}{lmr}{m}{n}%
  \TestFont{#1}{#2}{T1}{lmss}{m}{n}%
  \TestFont{#1}{#2}{T1}{lmr}{b}{n}%
  \TestFont{#1}{#2}{T1}{lmr}{m}{it}%
  \TestFont{#1}{#2}{T1}{lmtt}{m}{n}%
  \TestFont{#1}{#2}{T1}{lmvtt}{m}{n}%
  \TestFont{#1}{#2}{T1}{qtm}{m}{n}%
  \TestFont{#1}{#2}{T1}{qhv}{m}{n}%
  \TestFont{#1}{#2}{T1}{qtm}{b}{n}%
  \TestFont{#1}{#2}{T1}{qtm}{m}{it}%
  \TestFont{#1}{#2}{T1}{qcr}{m}{n}%
  \newpage
}
\makeatother
\hologoList
\end{document}
%</test-list>
%    \end{macrocode}
%
% \section{Installation}
%
% \subsection{Download}
%
% \paragraph{Package.} This package is available on
% CTAN\footnote{\url{ftp://ftp.ctan.org/tex-archive/}}:
% \begin{description}
% \item[\CTAN{macros/latex/contrib/oberdiek/hologo.dtx}] The source file.
% \item[\CTAN{macros/latex/contrib/oberdiek/hologo.pdf}] Documentation.
% \end{description}
%
%
% \paragraph{Bundle.} All the packages of the bundle `oberdiek'
% are also available in a TDS compliant ZIP archive. There
% the packages are already unpacked and the documentation files
% are generated. The files and directories obey the TDS standard.
% \begin{description}
% \item[\CTAN{install/macros/latex/contrib/oberdiek.tds.zip}]
% \end{description}
% \emph{TDS} refers to the standard ``A Directory Structure
% for \TeX\ Files'' (\CTAN{tds/tds.pdf}). Directories
% with \xfile{texmf} in their name are usually organized this way.
%
% \subsection{Bundle installation}
%
% \paragraph{Unpacking.} Unpack the \xfile{oberdiek.tds.zip} in the
% TDS tree (also known as \xfile{texmf} tree) of your choice.
% Example (linux):
% \begin{quote}
%   |unzip oberdiek.tds.zip -d ~/texmf|
% \end{quote}
%
% \paragraph{Script installation.}
% Check the directory \xfile{TDS:scripts/oberdiek/} for
% scripts that need further installation steps.
% Package \xpackage{attachfile2} comes with the Perl script
% \xfile{pdfatfi.pl} that should be installed in such a way
% that it can be called as \texttt{pdfatfi}.
% Example (linux):
% \begin{quote}
%   |chmod +x scripts/oberdiek/pdfatfi.pl|\\
%   |cp scripts/oberdiek/pdfatfi.pl /usr/local/bin/|
% \end{quote}
%
% \subsection{Package installation}
%
% \paragraph{Unpacking.} The \xfile{.dtx} file is a self-extracting
% \docstrip\ archive. The files are extracted by running the
% \xfile{.dtx} through \plainTeX:
% \begin{quote}
%   \verb|tex hologo.dtx|
% \end{quote}
%
% \paragraph{TDS.} Now the different files must be moved into
% the different directories in your installation TDS tree
% (also known as \xfile{texmf} tree):
% \begin{quote}
% \def\t{^^A
% \begin{tabular}{@{}>{\ttfamily}l@{ $\rightarrow$ }>{\ttfamily}l@{}}
%   hologo.sty & tex/generic/oberdiek/hologo.sty\\
%   hologo.pdf & doc/latex/oberdiek/hologo.pdf\\
%   example/hologo-example.tex & doc/latex/oberdiek/example/hologo-example.tex\\
%   test/hologo-test1.tex & doc/latex/oberdiek/test/hologo-test1.tex\\
%   test/hologo-test-spacefactor.tex & doc/latex/oberdiek/test/hologo-test-spacefactor.tex\\
%   test/hologo-test-list.tex & doc/latex/oberdiek/test/hologo-test-list.tex\\
%   hologo.dtx & source/latex/oberdiek/hologo.dtx\\
% \end{tabular}^^A
% }^^A
% \sbox0{\t}^^A
% \ifdim\wd0>\linewidth
%   \begingroup
%     \advance\linewidth by\leftmargin
%     \advance\linewidth by\rightmargin
%   \edef\x{\endgroup
%     \def\noexpand\lw{\the\linewidth}^^A
%   }\x
%   \def\lwbox{^^A
%     \leavevmode
%     \hbox to \linewidth{^^A
%       \kern-\leftmargin\relax
%       \hss
%       \usebox0
%       \hss
%       \kern-\rightmargin\relax
%     }^^A
%   }^^A
%   \ifdim\wd0>\lw
%     \sbox0{\small\t}^^A
%     \ifdim\wd0>\linewidth
%       \ifdim\wd0>\lw
%         \sbox0{\footnotesize\t}^^A
%         \ifdim\wd0>\linewidth
%           \ifdim\wd0>\lw
%             \sbox0{\scriptsize\t}^^A
%             \ifdim\wd0>\linewidth
%               \ifdim\wd0>\lw
%                 \sbox0{\tiny\t}^^A
%                 \ifdim\wd0>\linewidth
%                   \lwbox
%                 \else
%                   \usebox0
%                 \fi
%               \else
%                 \lwbox
%               \fi
%             \else
%               \usebox0
%             \fi
%           \else
%             \lwbox
%           \fi
%         \else
%           \usebox0
%         \fi
%       \else
%         \lwbox
%       \fi
%     \else
%       \usebox0
%     \fi
%   \else
%     \lwbox
%   \fi
% \else
%   \usebox0
% \fi
% \end{quote}
% If you have a \xfile{docstrip.cfg} that configures and enables \docstrip's
% TDS installing feature, then some files can already be in the right
% place, see the documentation of \docstrip.
%
% \subsection{Refresh file name databases}
%
% If your \TeX~distribution
% (\teTeX, \mikTeX, \dots) relies on file name databases, you must refresh
% these. For example, \teTeX\ users run \verb|texhash| or
% \verb|mktexlsr|.
%
% \subsection{Some details for the interested}
%
% \paragraph{Attached source.}
%
% The PDF documentation on CTAN also includes the
% \xfile{.dtx} source file. It can be extracted by
% AcrobatReader 6 or higher. Another option is \textsf{pdftk},
% e.g. unpack the file into the current directory:
% \begin{quote}
%   \verb|pdftk hologo.pdf unpack_files output .|
% \end{quote}
%
% \paragraph{Unpacking with \LaTeX.}
% The \xfile{.dtx} chooses its action depending on the format:
% \begin{description}
% \item[\plainTeX:] Run \docstrip\ and extract the files.
% \item[\LaTeX:] Generate the documentation.
% \end{description}
% If you insist on using \LaTeX\ for \docstrip\ (really,
% \docstrip\ does not need \LaTeX), then inform the autodetect routine
% about your intention:
% \begin{quote}
%   \verb|latex \let\install=y\input{hologo.dtx}|
% \end{quote}
% Do not forget to quote the argument according to the demands
% of your shell.
%
% \paragraph{Generating the documentation.}
% You can use both the \xfile{.dtx} or the \xfile{.drv} to generate
% the documentation. The process can be configured by the
% configuration file \xfile{ltxdoc.cfg}. For instance, put this
% line into this file, if you want to have A4 as paper format:
% \begin{quote}
%   \verb|\PassOptionsToClass{a4paper}{article}|
% \end{quote}
% An example follows how to generate the
% documentation with pdf\LaTeX:
% \begin{quote}
%\begin{verbatim}
%pdflatex hologo.dtx
%makeindex -s gind.ist hologo.idx
%pdflatex hologo.dtx
%makeindex -s gind.ist hologo.idx
%pdflatex hologo.dtx
%\end{verbatim}
% \end{quote}
%
% \section{Catalogue}
%
% The following XML file can be used as source for the
% \href{http://mirror.ctan.org/help/Catalogue/catalogue.html}{\TeX\ Catalogue}.
% The elements \texttt{caption} and \texttt{description} are imported
% from the original XML file from the Catalogue.
% The name of the XML file in the Catalogue is \xfile{hologo.xml}.
%    \begin{macrocode}
%<*catalogue>
<?xml version='1.0' encoding='us-ascii'?>
<!DOCTYPE entry SYSTEM 'catalogue.dtd'>
<entry datestamp='$Date$' modifier='$Author$' id='hologo'>
  <name>hologo</name>
  <caption>A collection of logos with bookmark support.</caption>
  <authorref id='auth:oberdiek'/>
  <copyright owner='Heiko Oberdiek' year='2010-2012'/>
  <license type='lppl1.3'/>
  <version number='1.10'/>
  <description>
    The package defines a single command <tt>\hologo</tt>, whose
    argument is the usual case-confused ASCII version of the logo.
    The command is bookmark-enabled, so that every logo becomes
    available in bookmarks without further work.
    <p/>
    The package is part of the <xref refid='oberdiek'>oberdiek</xref>
    bundle.
  </description>
  <documentation details='Package documentation'
      href='ctan:/macros/latex/contrib/oberdiek/hologo.pdf'/>
  <ctan file='true' path='/macros/latex/contrib/oberdiek/hologo.dtx'/>
  <miktex location='oberdiek'/>
  <texlive location='oberdiek'/>
  <install path='/macros/latex/contrib/oberdiek/oberdiek.tds.zip'/>
</entry>
%</catalogue>
%    \end{macrocode}
%
% \begin{thebibliography}{9}
% \raggedright
%
% \bibitem{btxdoc}
% Oren Patashnik,
% \textit{\hologo{BibTeX}ing},
% 1988-02-08.\\
% \CTAN{biblio/bibtex/base/}
%
% \bibitem{dtklogos}
% Gerd Neugebauer, DANTE,
% \textit{Package \xpackage{dtklogos}},
% 2011-04-25.\\
% \CTAN{usergrps/dante/dtk/dtklogos.sty}
%
% \bibitem{etexman}
% The \hologo{NTS} Team,
% \textit{The \hologo{eTeX} manual},
% 1998-02.\\
% \CTAN{systems/e-tex/v2/doc/}
%
% \bibitem{ExTeX-FAQ}
% The \hologo{ExTeX} group,
% \textit{\hologo{ExTeX}: FAQ -- How is \hologo{ExTeX} typeset?},
% 2007-04-14.\\
% \url{http://www.extex.org/documentation/faq.html}
%
% \bibitem{LyX}
% %@MISC{ LyX,
% %  title = {{LyX 2.0.0 -- The Document Processor [Computer software and manual]}},
% %  author = {{The LyX Team}},
% %  howpublished = {Internet: http://www.lyx.org},
% %  year = {2011-05-08},
% %  note = {Retrieved May 10, 2011, from http://www.lyx.org},
% %  url = {http://www.lyx.org/}
% %}
% The \hologo{LyX} Team,
% \textit{\hologo{LyX} -- The Document Processor},
% 2011-05-08.\\
% \url{http://www.lyx.org/}
%
% \bibitem{OzTeX}
% Andrew Trevorrow,
% \hologo{OzTeX} FAQ: What is the correct way to typeset ``\hologo{OzTeX}''?,
% 2011-09-15 (visited).
% \url{http://www.trevorrow.com/oztex/ozfaq.html#oztex-logo}
%
% \bibitem{PiCTeX}
% Michael Wichura,
% \textit{The \hologo{PiCTeX} macro package},
% 1987-09-21.
% \CTAN{graphics/pictex/}
%
% \bibitem{scrlogo}
% Markus Kohm,
% \textit{\hologo{KOMAScript} Datei \xfile{scrlogo.dtx}},
% 2009-01-30.\\
% \CTAN{install/macros/latex/contrib/komascript.tds.zip}
%
% \end{thebibliography}
%
% \begin{History}
%   \begin{Version}{2010/04/08 v1.0}
%   \item
%     The first version.
%   \end{Version}
%   \begin{Version}{2010/04/16 v1.1}
%   \item
%     \cs{Hologo} added for support of logos at start of a sentence.
%   \item
%     \cs{hologoSetup} and \cs{hologoLogoSetup} added.
%   \item
%     Options \xoption{break}, \xoption{hyphenbreak}, \xoption{spacebreak}
%     added.
%   \item
%     Variant support added by option \xoption{variant}.
%   \end{Version}
%   \begin{Version}{2010/04/24 v1.2}
%   \item
%     \hologo{LaTeX3} added.
%   \item
%     \hologo{VTeX} added.
%   \end{Version}
%   \begin{Version}{2010/11/21 v1.3}
%   \item
%     \hologo{iniTeX}, \hologo{virTeX} added.
%   \end{Version}
%   \begin{Version}{2011/03/25 v1.4}
%   \item
%     \hologo{ConTeXt} with variants added.
%   \item
%     Option \xoption{discretionarybreak} added as refinement for
%     option \xoption{break}.
%   \end{Version}
%   \begin{Version}{2011/04/21 v1.5}
%   \item
%     Wrong TDS directory for test files fixed.
%   \end{Version}
%   \begin{Version}{2011/10/01 v1.6}
%   \item
%     Support for package \xpackage{tex4ht} added.
%   \item
%     Support for \cs{csname} added if \cs{ifincsname} is available.
%   \item
%     New logos:
%     \hologo{(La)TeX},
%     \hologo{biber},
%     \hologo{BibTeX} (\xoption{sc}, \xoption{sf}),
%     \hologo{emTeX},
%     \hologo{ExTeX},
%     \hologo{KOMAScript},
%     \hologo{La},
%     \hologo{LyX},
%     \hologo{MiKTeX},
%     \hologo{NTS},
%     \hologo{OzMF},
%     \hologo{OzMP},
%     \hologo{OzTeX},
%     \hologo{OzTtH},
%     \hologo{PCTeX},
%     \hologo{PiC},
%     \hologo{PiCTeX},
%     \hologo{METAFONT},
%     \hologo{MetaFun},
%     \hologo{METAPOST},
%     \hologo{MetaPost},
%     \hologo{SLiTeX} (\xoption{lift}, \xoption{narrow}, \xoption{simple}),
%     \hologo{SliTeX} (\xoption{narrow}, \xoption{simple}, \xoption{lift}),
%     \hologo{teTeX}.
%   \item
%     Fixes:
%     \hologo{iniTeX},
%     \hologo{pdfLaTeX},
%     \hologo{pdfTeX},
%     \hologo{virTeX}.
%   \item
%     \cs{hologoFontSetup} and \cs{hologoLogoFontSetup} added.
%   \item
%     \cs{hologoVariant} and \cs{HologoVariant} added.
%   \end{Version}
%   \begin{Version}{2011/11/22 v1.7}
%   \item
%     New logos:
%     \hologo{BibTeX8},
%     \hologo{LaTeXML},
%     \hologo{SageTeX},
%     \hologo{TeX4ht},
%     \hologo{TTH}.
%   \item
%     \hologo{Xe} and friends: Driver stuff fixed.
%   \item
%     \hologo{Xe} and friends: Support for italic added.
%   \item
%     \hologo{Xe} and friends: Package support for \xpackage{pgf}
%     and \xpackage{pstricks} added.
%   \end{Version}
%   \begin{Version}{2011/11/29 v1.8}
%   \item
%     New logos:
%     \hologo{HanTheThanh}.
%   \end{Version}
%   \begin{Version}{2011/12/21 v1.9}
%   \item
%     Patch for package \xpackage{ifxetex} added for the case that
%     \cs{newif} is undefined in \hologo{iniTeX}.
%   \item
%     Some fixes for \hologo{iniTeX}.
%   \end{Version}
%   \begin{Version}{2012/04/26 v1.10}
%   \item
%     Fix in bookmark version of logo ``\hologo{HanTheThanh}''.
%   \end{Version}
%   \begin{Version}{2016/05/12 v1.11}
%   \item
%     Update HOLOGO@IfCharExists (previously in texlive)
%   \item define pdfliteral in current luatex.
%   \end{Version}
% \end{History}
%
% \PrintIndex
%
% \Finale
\endinput

%        (quote the arguments according to the demands of your shell)
%
% Documentation:
%    (a) If hologo.drv is present:
%           latex hologo.drv
%    (b) Without hologo.drv:
%           latex hologo.dtx; ...
%    The class ltxdoc loads the configuration file ltxdoc.cfg
%    if available. Here you can specify further options, e.g.
%    use A4 as paper format:
%       \PassOptionsToClass{a4paper}{article}
%
%    Programm calls to get the documentation (example):
%       pdflatex hologo.dtx
%       makeindex -s gind.ist hologo.idx
%       pdflatex hologo.dtx
%       makeindex -s gind.ist hologo.idx
%       pdflatex hologo.dtx
%
% Installation:
%    TDS:tex/generic/oberdiek/hologo.sty
%    TDS:doc/latex/oberdiek/hologo.pdf
%    TDS:doc/latex/oberdiek/example/hologo-example.tex
%    TDS:doc/latex/oberdiek/test/hologo-test1.tex
%    TDS:doc/latex/oberdiek/test/hologo-test-spacefactor.tex
%    TDS:doc/latex/oberdiek/test/hologo-test-list.tex
%    TDS:source/latex/oberdiek/hologo.dtx
%
%<*ignore>
\begingroup
  \catcode123=1 %
  \catcode125=2 %
  \def\x{LaTeX2e}%
\expandafter\endgroup
\ifcase 0\ifx\install y1\fi\expandafter
         \ifx\csname processbatchFile\endcsname\relax\else1\fi
         \ifx\fmtname\x\else 1\fi\relax
\else\csname fi\endcsname
%</ignore>
%<*install>
\input docstrip.tex
\Msg{************************************************************************}
\Msg{* Installation}
\Msg{* Package: hologo 2016/05/12 v1.11 A logo collection with bookmark support (HO)}
\Msg{************************************************************************}

\keepsilent
\askforoverwritefalse

\let\MetaPrefix\relax
\preamble

This is a generated file.

Project: hologo
Version: 2016/05/12 v1.11

Copyright (C) 2010-2012 by
   Heiko Oberdiek <heiko.oberdiek at googlemail.com>

This work may be distributed and/or modified under the
conditions of the LaTeX Project Public License, either
version 1.3c of this license or (at your option) any later
version. This version of this license is in
   http://www.latex-project.org/lppl/lppl-1-3c.txt
and the latest version of this license is in
   http://www.latex-project.org/lppl.txt
and version 1.3 or later is part of all distributions of
LaTeX version 2005/12/01 or later.

This work has the LPPL maintenance status "maintained".

This Current Maintainer of this work is Heiko Oberdiek.

The Base Interpreter refers to any `TeX-Format',
because some files are installed in TDS:tex/generic//.

This work consists of the main source file hologo.dtx
and the derived files
   hologo.sty, hologo.pdf, hologo.ins, hologo.drv, hologo-example.tex,
   hologo-test1.tex, hologo-test-spacefactor.tex,
   hologo-test-list.tex.

\endpreamble
\let\MetaPrefix\DoubleperCent

\generate{%
  \file{hologo.ins}{\from{hologo.dtx}{install}}%
  \file{hologo.drv}{\from{hologo.dtx}{driver}}%
  \usedir{tex/generic/oberdiek}%
  \file{hologo.sty}{\from{hologo.dtx}{package}}%
  \usedir{doc/latex/oberdiek/example}%
  \file{hologo-example.tex}{\from{hologo.dtx}{example}}%
  \usedir{doc/latex/oberdiek/test}%
  \file{hologo-test1.tex}{\from{hologo.dtx}{test1}}%
  \file{hologo-test-spacefactor.tex}{\from{hologo.dtx}{test-spacefactor}}%
  \file{hologo-test-list.tex}{\from{hologo.dtx}{test-list}}%
  \nopreamble
  \nopostamble
  \usedir{source/latex/oberdiek/catalogue}%
  \file{hologo.xml}{\from{hologo.dtx}{catalogue}}%
}

\catcode32=13\relax% active space
\let =\space%
\Msg{************************************************************************}
\Msg{*}
\Msg{* To finish the installation you have to move the following}
\Msg{* file into a directory searched by TeX:}
\Msg{*}
\Msg{*     hologo.sty}
\Msg{*}
\Msg{* To produce the documentation run the file `hologo.drv'}
\Msg{* through LaTeX.}
\Msg{*}
\Msg{* Happy TeXing!}
\Msg{*}
\Msg{************************************************************************}

\endbatchfile
%</install>
%<*ignore>
\fi
%</ignore>
%<*driver>
\NeedsTeXFormat{LaTeX2e}
\ProvidesFile{hologo.drv}%
  [2016/05/12 v1.11 A logo collection with bookmark support (HO)]%
\documentclass{ltxdoc}
\usepackage{holtxdoc}[2011/11/22]
\usepackage{hologo}[2016/05/12]
\usepackage{longtable}
\usepackage{array}
\usepackage{paralist}
%\usepackage[T1]{fontenc}
%\usepackage{lmodern}
\begin{document}
  \DocInput{hologo.dtx}%
\end{document}
%</driver>
% \fi
%
%
% \CharacterTable
%  {Upper-case    \A\B\C\D\E\F\G\H\I\J\K\L\M\N\O\P\Q\R\S\T\U\V\W\X\Y\Z
%   Lower-case    \a\b\c\d\e\f\g\h\i\j\k\l\m\n\o\p\q\r\s\t\u\v\w\x\y\z
%   Digits        \0\1\2\3\4\5\6\7\8\9
%   Exclamation   \!     Double quote  \"     Hash (number) \#
%   Dollar        \$     Percent       \%     Ampersand     \&
%   Acute accent  \'     Left paren    \(     Right paren   \)
%   Asterisk      \*     Plus          \+     Comma         \,
%   Minus         \-     Point         \.     Solidus       \/
%   Colon         \:     Semicolon     \;     Less than     \<
%   Equals        \=     Greater than  \>     Question mark \?
%   Commercial at \@     Left bracket  \[     Backslash     \\
%   Right bracket \]     Circumflex    \^     Underscore    \_
%   Grave accent  \`     Left brace    \{     Vertical bar  \|
%   Right brace   \}     Tilde         \~}
%
% \GetFileInfo{hologo.drv}
%
% \title{The \xpackage{hologo} package}
% \date{2016/05/12 v1.11}
% \author{Heiko Oberdiek\\\xemail{heiko.oberdiek at googlemail.com}}
%
% \maketitle
%
% \begin{abstract}
% This package starts a collection of logos with support for bookmarks
% strings.
% \end{abstract}
%
% \tableofcontents
%
% \section{Documentation}
%
% \subsection{Logo macros}
%
% \begin{declcs}{hologo} \M{name}
% \end{declcs}
% Macro \cs{hologo} sets the logo with name \meta{name}.
% The following table shows the supported names.
%
% \begingroup
%   \def\hologoEntry#1#2#3{^^A
%     #1&#2&\hologoLogoSetup{#1}{variant=#2}\hologo{#1}&#3\tabularnewline
%   }
%   \begin{longtable}{>{\ttfamily}l>{\ttfamily}lll}
%     \rmfamily\bfseries{name} & \rmfamily\bfseries variant
%     & \bfseries logo & \bfseries since\\
%     \hline
%     \endhead
%     \hologoList
%   \end{longtable}
% \endgroup
%
% \begin{declcs}{Hologo} \M{name}
% \end{declcs}
% Macro \cs{Hologo} starts the logo \meta{name} with an uppercase
% letter. As an exception small greek letters are not converted
% to uppercase. Examples, see \hologo{eTeX} and \hologo{ExTeX}.
%
% \subsection{Setup macros}
%
% The package does not support package options, but the following
% setup macros can be used to set options.
%
% \begin{declcs}{hologoSetup} \M{key value list}
% \end{declcs}
% Macro \cs{hologoSetup} sets global options.
%
% \begin{declcs}{hologoLogoSetup} \M{logo} \M{key value list}
% \end{declcs}
% Some options can also be used to configure a logo.
% These settings take precedence over global option settings.
%
% \subsection{Options}\label{sec:options}
%
% There are boolean and string options:
% \begin{description}
% \item[Boolean option:]
% It takes |true| or |false|
% as value. If the value is omitted, then |true| is used.
% \item[String option:]
% A value must be given as string. (But the string might be empty.)
% \end{description}
% The following options can be used both in \cs{hologoSetup}
% and \cs{hologoLogoSetup}:
% \begin{description}
% \def\entry#1{\item[\xoption{#1}:]}
% \entry{break}
%   enables or disables line breaks inside the logo. This setting is
%   refined by options \xoption{hyphenbreak}, \xoption{spacebreak}
%   or \xoption{discretionarybreak}.
%   Default is |false|.
% \entry{hyphenbreak}
%   enables or disables the line break right after the hyphen character.
% \entry{spacebreak}
%   enables or disables line breaks at space characters.
% \entry{discretionarybreak}
%   enables or disables line breaks at hyphenation points
%   (inserted by \cs{-}).
% \end{description}
% Macro \cs{hologoLogoSetup} also knows:
% \begin{description}
% \item[\xoption{variant}:]
%   This is a string option. It specifies a variant of a logo that
%   must exist. An empty string selects the package default variant.
% \end{description}
% Example:
% \begin{quote}
%   |\hologoSetup{break=false}|\\
%   |\hologoLogoSetup{plainTeX}{variant=hyphen,hyphenbreak}|\\
%   Then ``plain-\TeX'' contains one break point after the hyphen.
% \end{quote}
%
% \subsection{Driver options}
%
% Sometimes graphical operations are needed to construct some
% glyphs (e.g.\ \hologo{XeTeX}). If package \xpackage{graphics}
% or package \xpackage{pgf} are found, then the macros are taken
% from there. Otherwise the packge defines its own operations
% and therefore needs the driver information. Many drivers are
% detected automatically (\hologo{pdfTeX}/\hologo{LuaTeX}
% in PDF mode, \hologo{XeTeX}, \hologo{VTeX}). These have precedence
% over a driver option. The driver can be given as package option
% or using \cs{hologoDriverSetup}.
% The following list contains the recognized driver options:
% \begin{itemize}
% \item \xoption{pdftex}, \xoption{luatex}
% \item \xoption{dvipdfm}, \xoption{dvipdfmx}
% \item \xoption{dvips}, \xoption{dvipsone}, \xoption{xdvi}
% \item \xoption{xetex}
% \item \xoption{vtex}
% \end{itemize}
% The left driver of a line is the driver name that is used internally.
% The following names are aliases for drivers that use the
% same method. Therefore the entry in the \xext{log} file for
% the used driver prints the internally used driver name.
% \begin{description}
% \item[\xoption{driverfallback}:]
%   This option expects a driver that is used,
%   if the driver could not be detected automatically.
% \end{description}
%
% \begin{declcs}{hologoDriverSetup} \M{driver option}
% \end{declcs}
% The driver can also be configured after package loading
% using \cs{hologoDriverSetup}, also the way for \hologo{plainTeX}
% to setup the driver.
%
% \subsection{Font setup}
%
% Some logos require a special font, but should also be usable by
% \hologo{plainTeX}. Therefore the package provides some ways
% to influence the font settings. The options below
% take font settings as values. Both font commands
% such as \cs{sffamily} and macros that take one argument
% like \cs{textsf} can be used.
%
% \begin{declcs}{hologoFontSetup} \M{key value list}
% \end{declcs}
% Macro \cs{hologoFontSetup} sets the fonts for all logos.
% Supported keys:
% \begin{description}
% \def\entry#1{\item[\xoption{#1}:]}
% \entry{general}
%   This font is used for all logos. The default is empty.
%   That means no special font is used.
% \entry{bibsf}
%   This font is used for
%   {\hologoLogoSetup{BibTeX}{variant=sf}\hologo{BibTeX}}
%   with variant \xoption{sf}.
% \entry{rm}
%   This font is a serif font. It is used for \hologo{ExTeX}.
% \entry{sc}
%   This font specifies a small caps font. It is used for
%   {\hologoLogoSetup{BibTeX}{variant=sc}\hologo{BibTeX}}
%   with variant \xoption{sc}.
% \entry{sf}
%   This font specifies a sans serif font. The default
%   is \cs{sffamily}, then \cs{sf} is tried. Otherwise
%   a warning is given. It is used by \hologo{KOMAScript}.
% \entry{sy}
%   This is the font for math symbols (e.g. cmsy).
%   It is used by \hologo{AmS}, \hologo{NTS}, \hologo{ExTeX}.
% \entry{logo}
%   \hologo{METAFONT} and \hologo{METAPOST} are using that font.
%   In \hologo{LaTeX} \cs{logofamily} is used and
%   the definitions of package \xpackage{mflogo} are used
%   if the package is not loaded.
%   Otherwise the \cs{tenlogo} is used and defined
%   if it does not already exists.
% \end{description}
%
% \begin{declcs}{hologoLogoFontSetup} \M{logo} \M{key value list}
% \end{declcs}
% Fonts can also be set for a logo or logo component separately,
% see the following list.
% The keys are the same as for \cs{hologoFontSetup}.
%
% \begin{longtable}{>{\ttfamily}l>{\sffamily}ll}
%   \meta{logo} & keys & result\\
%   \hline
%   \endhead
%   BibTeX & bibsf & {\hologoLogoSetup{BibTeX}{variant=sf}\hologo{BibTeX}}\\[.5ex]
%   BibTeX & sc & {\hologoLogoSetup{BibTeX}{variant=sc}\hologo{BibTeX}}\\[.5ex]
%   ExTeX & rm & \hologo{ExTeX}\\
%   SliTeX & rm & \hologo{SliTeX}\\[.5ex]
%   AmS & sy & \hologo{AmS}\\
%   ExTeX & sy & \hologo{ExTeX}\\
%   NTS & sy & \hologo{NTS}\\[.5ex]
%   KOMAScript & sf & \hologo{KOMAScript}\\[.5ex]
%   METAFONT & logo & \hologo{METAFONT}\\
%   METAPOST & logo & \hologo{METAPOST}\\[.5ex]
%   SliTeX & sc \hologo{SliTeX}
% \end{longtable}
%
% \subsubsection{Font order}
%
% For all logos the font \xoption{general} is applied first.
% Example:
%\begin{quote}
%|\hologoFontSetup{general=\color{red}}|
%\end{quote}
% will print red logos.
% Then if the font uses a special font \xoption{sf}, for example,
% the font is applied that is setup by \cs{hologoLogoFontSetup}.
% If this font is not setup, then the common font setup
% by \cs{hologoFontSetup} is used. Otherwise a warning is given,
% that there is no font configured.
%
% \subsection{Additional user macros}
%
% Usually a variant of a logo is configured by using
% \cs{hologoLogoSetup}, because it is bad style to mix
% different variants of the same logo in the same text.
% There the following macros are a convenience for testing.
%
% \begin{declcs}{hologoVariant} \M{name} \M{variant}\\
%   \cs{HologoVariant} \M{name} \M{variant}
% \end{declcs}
% Logo \meta{name} is set using \meta{variant} that specifies
% explicitely which variant of the macro is used. If the argument
% is empty, then the default form of the logo is used
% (configurable by \cs{hologoLogoSetup}).
%
% \cs{HologoVariant} is used if the logo is set in a context
% that needs an uppercase first letter (beginning of a sentence, \dots).
%
% \begin{declcs}{hologoList}\\
%   \cs{hologoEntry} \M{logo} \M{variant} \M{since}
% \end{declcs}
% Macro \cs{hologoList} contains all logos that are provided
% by the package including variants. The list consists of calls
% of \cs{hologoEntry} with three arguments starting with the
% logo name \meta{logo} and its variant \meta{variant}. An empty
% variant means the current default. Argument \meta{since} specifies
% with version of the package \xpackage{hologo} is needed to get
% the logo. If the logo is fixed, then the date gets updated.
% Therefore the date \meta{since} is not exactly the date of
% the first introduction, but rather the date of the latest fix.
%
% Before \cs{hologoList} can be used, macro \cs{hologoEntry} needs
% a definition. The example file in section \ref{sec:example}
% shows applications of \cs{hologoList}.
%
% \subsection{Supported contexts}
%
% Macros \cs{hologo} and friends support special contexts:
% \begin{itemize}
% \item \hologo{LaTeX}'s protection mechanism.
% \item Bookmarks of package \xpackage{hyperref}.
% \item Package \xpackage{tex4ht}.
% \item The macros can be used inside \cs{csname} constructs,
%   if \cs{ifincsname} is available (\hologo{pdfTeX}, \hologo{XeTeX},
%   \hologo{LuaTeX}).
% \end{itemize}
%
% \subsection{Example}
% \label{sec:example}
%
% The following example prints the logos in different fonts.
%    \begin{macrocode}
%<*example>
%<<verbatim
\NeedsTeXFormat{LaTeX2e}
\documentclass[a4paper]{article}
\usepackage[
  hmargin=20mm,
  vmargin=20mm,
]{geometry}
\pagestyle{empty}
\usepackage{hologo}[2016/05/12]
\usepackage{longtable}
\usepackage{array}
\setlength{\extrarowheight}{2pt}
\usepackage[T1]{fontenc}
\usepackage{lmodern}
\usepackage{pdflscape}
\usepackage[
  pdfencoding=auto,
]{hyperref}
\hypersetup{
  pdfauthor={Heiko Oberdiek},
  pdftitle={Example for package `hologo'},
  pdfsubject={Logos with fonts lmr, lmss, qtm, qpl, qhv},
}
\usepackage{bookmark}

% Print the logo list on the console

\begingroup
  \typeout{}%
  \typeout{*** Begin of logo list ***}%
  \newcommand*{\hologoEntry}[3]{%
    \typeout{#1 \ifx\\#2\\\else(#2) \fi[#3]}%
  }%
  \hologoList
  \typeout{*** End of logo list ***}%
  \typeout{}%
\endgroup

\begin{document}
\begin{landscape}

  \section{Example file for package `hologo'}

  % Table for font names

  \begin{longtable}{>{\bfseries}ll}
    \textbf{font} & \textbf{Font name}\\
    \hline
    lmr & Latin Modern Roman\\
    lmss & Latin Modern Sans\\
    qtm & \TeX\ Gyre Termes\\
    qhv & \TeX\ Gyre Heros\\
    qpl & \TeX\ Gyre Pagella\\
  \end{longtable}

  % Logo list with logos in different fonts

  \begingroup
    \newcommand*{\SetVariant}[2]{%
      \ifx\\#2\\%
      \else
        \hologoLogoSetup{#1}{variant=#2}%
      \fi
    }%
    \newcommand*{\hologoEntry}[3]{%
      \SetVariant{#1}{#2}%
      \raisebox{1em}[0pt][0pt]{\hypertarget{#1@#2}{}}%
      \bookmark[%
        dest={#1@#2},%
      ]{%
        #1\ifx\\#2\\\else\space(#2)\fi: \Hologo{#1}, \hologo{#1} %
        [Unicode]%
      }%
      \hypersetup{unicode=false}%
      \bookmark[%
        dest={#1@#2},%
      ]{%
        #1\ifx\\#2\\\else\space(#2)\fi: \Hologo{#1}, \hologo{#1} %
        [PDFDocEncoding]%
      }%
      \texttt{#1}%
      &%
      \texttt{#2}%
      &%
      \Hologo{#1}%
      &%
      \SetVariant{#1}{#2}%
      \hologo{#1}%
      &%
      \SetVariant{#1}{#2}%
      \fontfamily{qtm}\selectfont
      \hologo{#1}%
      &%
      \SetVariant{#1}{#2}%
      \fontfamily{qpl}\selectfont
      \hologo{#1}%
      &%
      \SetVariant{#1}{#2}%
      \textsf{\hologo{#1}}%
      &%
      \SetVariant{#1}{#2}%
      \fontfamily{qhv}\selectfont
      \hologo{#1}%
      \tabularnewline
    }%
    \begin{longtable}{llllllll}%
      \textbf{\textit{logo}} & \textbf{\textit{variant}} &
      \texttt{\string\Hologo} &
      \textbf{lmr} & \textbf{qtm} & \textbf{qpl} &
      \textbf{lmss} & \textbf{qhv}
      \tabularnewline
      \hline
      \endhead
      \hologoList
    \end{longtable}%
  \endgroup

\end{landscape}
\end{document}
%verbatim
%</example>
%    \end{macrocode}
%
% \StopEventually{
% }
%
% \section{Implementation}
%    \begin{macrocode}
%<*package>
%    \end{macrocode}
%    Reload check, especially if the package is not used with \LaTeX.
%    \begin{macrocode}
\begingroup\catcode61\catcode48\catcode32=10\relax%
  \catcode13=5 % ^^M
  \endlinechar=13 %
  \catcode35=6 % #
  \catcode39=12 % '
  \catcode44=12 % ,
  \catcode45=12 % -
  \catcode46=12 % .
  \catcode58=12 % :
  \catcode64=11 % @
  \catcode123=1 % {
  \catcode125=2 % }
  \expandafter\let\expandafter\x\csname ver@hologo.sty\endcsname
  \ifx\x\relax % plain-TeX, first loading
  \else
    \def\empty{}%
    \ifx\x\empty % LaTeX, first loading,
      % variable is initialized, but \ProvidesPackage not yet seen
    \else
      \expandafter\ifx\csname PackageInfo\endcsname\relax
        \def\x#1#2{%
          \immediate\write-1{Package #1 Info: #2.}%
        }%
      \else
        \def\x#1#2{\PackageInfo{#1}{#2, stopped}}%
      \fi
      \x{hologo}{The package is already loaded}%
      \aftergroup\endinput
    \fi
  \fi
\endgroup%
%    \end{macrocode}
%    Package identification:
%    \begin{macrocode}
\begingroup\catcode61\catcode48\catcode32=10\relax%
  \catcode13=5 % ^^M
  \endlinechar=13 %
  \catcode35=6 % #
  \catcode39=12 % '
  \catcode40=12 % (
  \catcode41=12 % )
  \catcode44=12 % ,
  \catcode45=12 % -
  \catcode46=12 % .
  \catcode47=12 % /
  \catcode58=12 % :
  \catcode64=11 % @
  \catcode91=12 % [
  \catcode93=12 % ]
  \catcode123=1 % {
  \catcode125=2 % }
  \expandafter\ifx\csname ProvidesPackage\endcsname\relax
    \def\x#1#2#3[#4]{\endgroup
      \immediate\write-1{Package: #3 #4}%
      \xdef#1{#4}%
    }%
  \else
    \def\x#1#2[#3]{\endgroup
      #2[{#3}]%
      \ifx#1\@undefined
        \xdef#1{#3}%
      \fi
      \ifx#1\relax
        \xdef#1{#3}%
      \fi
    }%
  \fi
\expandafter\x\csname ver@hologo.sty\endcsname
\ProvidesPackage{hologo}%
  [2016/05/12 v1.11 A logo collection with bookmark support (HO)]%
%    \end{macrocode}
%
%    \begin{macrocode}
\begingroup\catcode61\catcode48\catcode32=10\relax%
  \catcode13=5 % ^^M
  \endlinechar=13 %
  \catcode123=1 % {
  \catcode125=2 % }
  \catcode64=11 % @
  \def\x{\endgroup
    \expandafter\edef\csname HOLOGO@AtEnd\endcsname{%
      \endlinechar=\the\endlinechar\relax
      \catcode13=\the\catcode13\relax
      \catcode32=\the\catcode32\relax
      \catcode35=\the\catcode35\relax
      \catcode61=\the\catcode61\relax
      \catcode64=\the\catcode64\relax
      \catcode123=\the\catcode123\relax
      \catcode125=\the\catcode125\relax
    }%
  }%
\x\catcode61\catcode48\catcode32=10\relax%
\catcode13=5 % ^^M
\endlinechar=13 %
\catcode35=6 % #
\catcode64=11 % @
\catcode123=1 % {
\catcode125=2 % }
\def\TMP@EnsureCode#1#2{%
  \edef\HOLOGO@AtEnd{%
    \HOLOGO@AtEnd
    \catcode#1=\the\catcode#1\relax
  }%
  \catcode#1=#2\relax
}
\TMP@EnsureCode{10}{12}% ^^J
\TMP@EnsureCode{33}{12}% !
\TMP@EnsureCode{34}{12}% "
\TMP@EnsureCode{36}{3}% $
\TMP@EnsureCode{38}{4}% &
\TMP@EnsureCode{39}{12}% '
\TMP@EnsureCode{40}{12}% (
\TMP@EnsureCode{41}{12}% )
\TMP@EnsureCode{42}{12}% *
\TMP@EnsureCode{43}{12}% +
\TMP@EnsureCode{44}{12}% ,
\TMP@EnsureCode{45}{12}% -
\TMP@EnsureCode{46}{12}% .
\TMP@EnsureCode{47}{12}% /
\TMP@EnsureCode{58}{12}% :
\TMP@EnsureCode{59}{12}% ;
\TMP@EnsureCode{60}{12}% <
\TMP@EnsureCode{62}{12}% >
\TMP@EnsureCode{63}{12}% ?
\TMP@EnsureCode{91}{12}% [
\TMP@EnsureCode{93}{12}% ]
\TMP@EnsureCode{94}{7}% ^ (superscript)
\TMP@EnsureCode{95}{8}% _ (subscript)
\TMP@EnsureCode{96}{12}% `
\TMP@EnsureCode{124}{12}% |
\edef\HOLOGO@AtEnd{%
  \HOLOGO@AtEnd
  \escapechar\the\escapechar\relax
  \noexpand\endinput
}
\escapechar=92 %
%    \end{macrocode}
%
% \subsection{Logo list}
%
%    \begin{macro}{\hologoList}
%    \begin{macrocode}
\def\hologoList{%
  \hologoEntry{(La)TeX}{}{2011/10/01}%
  \hologoEntry{AmSLaTeX}{}{2010/04/16}%
  \hologoEntry{AmSTeX}{}{2010/04/16}%
  \hologoEntry{biber}{}{2011/10/01}%
  \hologoEntry{BibTeX}{}{2011/10/01}%
  \hologoEntry{BibTeX}{sf}{2011/10/01}%
  \hologoEntry{BibTeX}{sc}{2011/10/01}%
  \hologoEntry{BibTeX8}{}{2011/11/22}%
  \hologoEntry{ConTeXt}{}{2011/03/25}%
  \hologoEntry{ConTeXt}{narrow}{2011/03/25}%
  \hologoEntry{ConTeXt}{simple}{2011/03/25}%
  \hologoEntry{emTeX}{}{2010/04/26}%
  \hologoEntry{eTeX}{}{2010/04/08}%
  \hologoEntry{ExTeX}{}{2011/10/01}%
  \hologoEntry{HanTheThanh}{}{2011/11/29}%
  \hologoEntry{iniTeX}{}{2011/10/01}%
  \hologoEntry{KOMAScript}{}{2011/10/01}%
  \hologoEntry{La}{}{2010/05/08}%
  \hologoEntry{LaTeX}{}{2010/04/08}%
  \hologoEntry{LaTeX2e}{}{2010/04/08}%
  \hologoEntry{LaTeX3}{}{2010/04/24}%
  \hologoEntry{LaTeXe}{}{2010/04/08}%
  \hologoEntry{LaTeXML}{}{2011/11/22}%
  \hologoEntry{LaTeXTeX}{}{2011/10/01}%
  \hologoEntry{LuaLaTeX}{}{2010/04/08}%
  \hologoEntry{LuaTeX}{}{2010/04/08}%
  \hologoEntry{LyX}{}{2011/10/01}%
  \hologoEntry{METAFONT}{}{2011/10/01}%
  \hologoEntry{MetaFun}{}{2011/10/01}%
  \hologoEntry{METAPOST}{}{2011/10/01}%
  \hologoEntry{MetaPost}{}{2011/10/01}%
  \hologoEntry{MiKTeX}{}{2011/10/01}%
  \hologoEntry{NTS}{}{2011/10/01}%
  \hologoEntry{OzMF}{}{2011/10/01}%
  \hologoEntry{OzMP}{}{2011/10/01}%
  \hologoEntry{OzTeX}{}{2011/10/01}%
  \hologoEntry{OzTtH}{}{2011/10/01}%
  \hologoEntry{PCTeX}{}{2011/10/01}%
  \hologoEntry{pdfTeX}{}{2011/10/01}%
  \hologoEntry{pdfLaTeX}{}{2011/10/01}%
  \hologoEntry{PiC}{}{2011/10/01}%
  \hologoEntry{PiCTeX}{}{2011/10/01}%
  \hologoEntry{plainTeX}{}{2010/04/08}%
  \hologoEntry{plainTeX}{space}{2010/04/16}%
  \hologoEntry{plainTeX}{hyphen}{2010/04/16}%
  \hologoEntry{plainTeX}{runtogether}{2010/04/16}%
  \hologoEntry{SageTeX}{}{2011/11/22}%
  \hologoEntry{SLiTeX}{}{2011/10/01}%
  \hologoEntry{SLiTeX}{lift}{2011/10/01}%
  \hologoEntry{SLiTeX}{narrow}{2011/10/01}%
  \hologoEntry{SLiTeX}{simple}{2011/10/01}%
  \hologoEntry{SliTeX}{}{2011/10/01}%
  \hologoEntry{SliTeX}{narrow}{2011/10/01}%
  \hologoEntry{SliTeX}{simple}{2011/10/01}%
  \hologoEntry{SliTeX}{lift}{2011/10/01}%
  \hologoEntry{teTeX}{}{2011/10/01}%
  \hologoEntry{TeX}{}{2010/04/08}%
  \hologoEntry{TeX4ht}{}{2011/11/22}%
  \hologoEntry{TTH}{}{2011/11/22}%
  \hologoEntry{virTeX}{}{2011/10/01}%
  \hologoEntry{VTeX}{}{2010/04/24}%
  \hologoEntry{Xe}{}{2010/04/08}%
  \hologoEntry{XeLaTeX}{}{2010/04/08}%
  \hologoEntry{XeTeX}{}{2010/04/08}%
}
%    \end{macrocode}
%    \end{macro}
%
% \subsection{Load resources}
%
%    \begin{macrocode}
\begingroup\expandafter\expandafter\expandafter\endgroup
\expandafter\ifx\csname RequirePackage\endcsname\relax
  \def\TMP@RequirePackage#1[#2]{%
    \begingroup\expandafter\expandafter\expandafter\endgroup
    \expandafter\ifx\csname ver@#1.sty\endcsname\relax
      \input #1.sty\relax
    \fi
  }%
  \TMP@RequirePackage{ltxcmds}[2011/02/04]%
  \TMP@RequirePackage{infwarerr}[2010/04/08]%
  \TMP@RequirePackage{kvsetkeys}[2010/03/01]%
  \TMP@RequirePackage{kvdefinekeys}[2010/03/01]%
  \TMP@RequirePackage{pdftexcmds}[2010/04/01]%
  \TMP@RequirePackage{ifpdf}[2010/01/28]%
  \TMP@RequirePackage{ifluatex}[2010/03/01]%
  \ltx@IfUndefined{newif}{%
    \expandafter\let\csname newif\endcsname\ltx@newif
  }{}%
  \TMP@RequirePackage{ifxetex}[2009/01/23]%
  \TMP@RequirePackage{ifvtex}[2010/03/01]%
\else
  \RequirePackage{ltxcmds}[2011/02/04]%
  \RequirePackage{infwarerr}[2010/04/08]%
  \RequirePackage{kvsetkeys}[2010/03/01]%
  \RequirePackage{kvdefinekeys}[2010/03/01]%
  \RequirePackage{pdftexcmds}[2010/04/01]%
  \RequirePackage{ifpdf}[2010/01/28]%
  \RequirePackage{ifluatex}[2010/03/01]%
  \RequirePackage{ifxetex}[2009/01/23]%
  \RequirePackage{ifvtex}[2010/03/01]%
\fi
%    \end{macrocode}
%
%    \begin{macro}{\HOLOGO@IfDefined}
%    \begin{macrocode}
\def\HOLOGO@IfExists#1{%
  \ifx\@undefined#1%
    \expandafter\ltx@secondoftwo
  \else
    \ifx\relax#1%
      \expandafter\ltx@secondoftwo
    \else
      \expandafter\expandafter\expandafter\ltx@firstoftwo
    \fi
  \fi
}
%    \end{macrocode}
%    \end{macro}
%
% \subsection{Setup macros}
%
%    \begin{macro}{\hologoSetup}
%    \begin{macrocode}
\def\hologoSetup{%
  \let\HOLOGO@name\relax
  \HOLOGO@Setup
}
%    \end{macrocode}
%    \end{macro}
%
%    \begin{macro}{\hologoLogoSetup}
%    \begin{macrocode}
\def\hologoLogoSetup#1{%
  \edef\HOLOGO@name{#1}%
  \ltx@IfUndefined{HoLogo@\HOLOGO@name}{%
    \@PackageError{hologo}{%
      Unknown logo `\HOLOGO@name'%
    }\@ehc
    \ltx@gobble
  }{%
    \HOLOGO@Setup
  }%
}
%    \end{macrocode}
%    \end{macro}
%
%    \begin{macro}{\HOLOGO@Setup}
%    \begin{macrocode}
\def\HOLOGO@Setup{%
  \kvsetkeys{HoLogo}%
}
%    \end{macrocode}
%    \end{macro}
%
% \subsection{Options}
%
%    \begin{macro}{\HOLOGO@DeclareBoolOption}
%    \begin{macrocode}
\def\HOLOGO@DeclareBoolOption#1{%
  \expandafter\chardef\csname HOLOGOOPT@#1\endcsname\ltx@zero
  \kv@define@key{HoLogo}{#1}[true]{%
    \def\HOLOGO@temp{##1}%
    \ifx\HOLOGO@temp\HOLOGO@true
      \ifx\HOLOGO@name\relax
        \expandafter\chardef\csname HOLOGOOPT@#1\endcsname=\ltx@one
      \else
        \expandafter\chardef\csname
        HoLogoOpt@#1@\HOLOGO@name\endcsname\ltx@one
      \fi
      \HOLOGO@SetBreakAll{#1}%
    \else
      \ifx\HOLOGO@temp\HOLOGO@false
        \ifx\HOLOGO@name\relax
          \expandafter\chardef\csname HOLOGOOPT@#1\endcsname=\ltx@zero
        \else
          \expandafter\chardef\csname
          HoLogoOpt@#1@\HOLOGO@name\endcsname=\ltx@zero
        \fi
        \HOLOGO@SetBreakAll{#1}%
      \else
        \@PackageError{hologo}{%
          Unknown value `##1' for boolean option `#1'.\MessageBreak
          Known values are `true' and `false'%
        }\@ehc
      \fi
    \fi
  }%
}
%    \end{macrocode}
%    \end{macro}
%
%    \begin{macro}{\HOLOGO@SetBreakAll}
%    \begin{macrocode}
\def\HOLOGO@SetBreakAll#1{%
  \def\HOLOGO@temp{#1}%
  \ifx\HOLOGO@temp\HOLOGO@break
    \ifx\HOLOGO@name\relax
      \chardef\HOLOGOOPT@hyphenbreak=\HOLOGOOPT@break
      \chardef\HOLOGOOPT@spacebreak=\HOLOGOOPT@break
      \chardef\HOLOGOOPT@discretionarybreak=\HOLOGOOPT@break
    \else
      \expandafter\chardef
         \csname HoLogoOpt@hyphenbreak@\HOLOGO@name\endcsname=%
         \csname HoLogoOpt@break@\HOLOGO@name\endcsname
      \expandafter\chardef
         \csname HoLogoOpt@spacebreak@\HOLOGO@name\endcsname=%
         \csname HoLogoOpt@break@\HOLOGO@name\endcsname
      \expandafter\chardef
         \csname HoLogoOpt@discretionarybreak@\HOLOGO@name
             \endcsname=%
         \csname HoLogoOpt@break@\HOLOGO@name\endcsname
    \fi
  \fi
}
%    \end{macrocode}
%    \end{macro}
%
%    \begin{macro}{\HOLOGO@true}
%    \begin{macrocode}
\def\HOLOGO@true{true}
%    \end{macrocode}
%    \end{macro}
%    \begin{macro}{\HOLOGO@false}
%    \begin{macrocode}
\def\HOLOGO@false{false}
%    \end{macrocode}
%    \end{macro}
%    \begin{macro}{\HOLOGO@break}
%    \begin{macrocode}
\def\HOLOGO@break{break}
%    \end{macrocode}
%    \end{macro}
%
%    \begin{macrocode}
\HOLOGO@DeclareBoolOption{break}
\HOLOGO@DeclareBoolOption{hyphenbreak}
\HOLOGO@DeclareBoolOption{spacebreak}
\HOLOGO@DeclareBoolOption{discretionarybreak}
%    \end{macrocode}
%
%    \begin{macrocode}
\kv@define@key{HoLogo}{variant}{%
  \ifx\HOLOGO@name\relax
    \@PackageError{hologo}{%
      Option `variant' is not available in \string\hologoSetup,%
      \MessageBreak
      Use \string\hologoLogoSetup\space instead%
    }\@ehc
  \else
    \edef\HOLOGO@temp{#1}%
    \ifx\HOLOGO@temp\ltx@empty
      \expandafter
      \let\csname HoLogoOpt@variant@\HOLOGO@name\endcsname\@undefined
    \else
      \ltx@IfUndefined{HoLogo@\HOLOGO@name @\HOLOGO@temp}{%
        \@PackageError{hologo}{%
          Unknown variant `\HOLOGO@temp' of logo `\HOLOGO@name'%
        }\@ehc
      }{%
        \expandafter
        \let\csname HoLogoOpt@variant@\HOLOGO@name\endcsname
            \HOLOGO@temp
      }%
    \fi
  \fi
}
%    \end{macrocode}
%
%    \begin{macro}{\HOLOGO@Variant}
%    \begin{macrocode}
\def\HOLOGO@Variant#1{%
  #1%
  \ltx@ifundefined{HoLogoOpt@variant@#1}{%
  }{%
    @\csname HoLogoOpt@variant@#1\endcsname
  }%
}
%    \end{macrocode}
%    \end{macro}
%
% \subsection{Break/no-break support}
%
%    \begin{macro}{\HOLOGO@space}
%    \begin{macrocode}
\def\HOLOGO@space{%
  \ltx@ifundefined{HoLogoOpt@spacebreak@\HOLOGO@name}{%
    \ltx@ifundefined{HoLogoOpt@break@\HOLOGO@name}{%
      \chardef\HOLOGO@temp=\HOLOGOOPT@spacebreak
    }{%
      \chardef\HOLOGO@temp=%
        \csname HoLogoOpt@break@\HOLOGO@name\endcsname
    }%
  }{%
    \chardef\HOLOGO@temp=%
      \csname HoLogoOpt@spacebreak@\HOLOGO@name\endcsname
  }%
  \ifcase\HOLOGO@temp
    \penalty10000 %
  \fi
  \ltx@space
}
%    \end{macrocode}
%    \end{macro}
%
%    \begin{macro}{\HOLOGO@hyphen}
%    \begin{macrocode}
\def\HOLOGO@hyphen{%
  \ltx@ifundefined{HoLogoOpt@hyphenbreak@\HOLOGO@name}{%
    \ltx@ifundefined{HoLogoOpt@break@\HOLOGO@name}{%
      \chardef\HOLOGO@temp=\HOLOGOOPT@hyphenbreak
    }{%
      \chardef\HOLOGO@temp=%
        \csname HoLogoOpt@break@\HOLOGO@name\endcsname
    }%
  }{%
    \chardef\HOLOGO@temp=%
      \csname HoLogoOpt@hyphenbreak@\HOLOGO@name\endcsname
  }%
  \ifcase\HOLOGO@temp
    \ltx@mbox{-}%
  \else
    -%
  \fi
}
%    \end{macrocode}
%    \end{macro}
%
%    \begin{macro}{\HOLOGO@discretionary}
%    \begin{macrocode}
\def\HOLOGO@discretionary{%
  \ltx@ifundefined{HoLogoOpt@discretionarybreak@\HOLOGO@name}{%
    \ltx@ifundefined{HoLogoOpt@break@\HOLOGO@name}{%
      \chardef\HOLOGO@temp=\HOLOGOOPT@discretionarybreak
    }{%
      \chardef\HOLOGO@temp=%
        \csname HoLogoOpt@break@\HOLOGO@name\endcsname
    }%
  }{%
    \chardef\HOLOGO@temp=%
      \csname HoLogoOpt@discretionarybreak@\HOLOGO@name\endcsname
  }%
  \ifcase\HOLOGO@temp
  \else
    \-%
  \fi
}
%    \end{macrocode}
%    \end{macro}
%
%    \begin{macro}{\HOLOGO@mbox}
%    \begin{macrocode}
\def\HOLOGO@mbox#1{%
  \ltx@ifundefined{HoLogoOpt@break@\HOLOGO@name}{%
    \chardef\HOLOGO@temp=\HOLOGOOPT@hyphenbreak
  }{%
    \chardef\HOLOGO@temp=%
      \csname HoLogoOpt@break@\HOLOGO@name\endcsname
  }%
  \ifcase\HOLOGO@temp
    \ltx@mbox{#1}%
  \else
    #1%
  \fi
}
%    \end{macrocode}
%    \end{macro}
%
% \subsection{Font support}
%
%    \begin{macro}{\HoLogoFont@font}
%    \begin{tabular}{@{}ll@{}}
%    |#1|:& logo name\\
%    |#2|:& font short name\\
%    |#3|:& text
%    \end{tabular}
%    \begin{macrocode}
\def\HoLogoFont@font#1#2#3{%
  \begingroup
    \ltx@IfUndefined{HoLogoFont@logo@#1.#2}{%
      \ltx@IfUndefined{HoLogoFont@font@#2}{%
        \@PackageWarning{hologo}{%
          Missing font `#2' for logo `#1'%
        }%
        #3%
      }{%
        \csname HoLogoFont@font@#2\endcsname{#3}%
      }%
    }{%
      \csname HoLogoFont@logo@#1.#2\endcsname{#3}%
    }%
  \endgroup
}
%    \end{macrocode}
%    \end{macro}
%
%    \begin{macro}{\HoLogoFont@Def}
%    \begin{macrocode}
\def\HoLogoFont@Def#1{%
  \expandafter\def\csname HoLogoFont@font@#1\endcsname
}
%    \end{macrocode}
%    \end{macro}
%    \begin{macro}{\HoLogoFont@LogoDef}
%    \begin{macrocode}
\def\HoLogoFont@LogoDef#1#2{%
  \expandafter\def\csname HoLogoFont@logo@#1.#2\endcsname
}
%    \end{macrocode}
%    \end{macro}
%
% \subsubsection{Font defaults}
%
%    \begin{macro}{\HoLogoFont@font@general}
%    \begin{macrocode}
\HoLogoFont@Def{general}{}%
%    \end{macrocode}
%    \end{macro}
%
%    \begin{macro}{\HoLogoFont@font@rm}
%    \begin{macrocode}
\ltx@IfUndefined{rmfamily}{%
  \ltx@IfUndefined{rm}{%
  }{%
    \HoLogoFont@Def{rm}{\rm}%
  }%
}{%
  \HoLogoFont@Def{rm}{\rmfamily}%
}
%    \end{macrocode}
%    \end{macro}
%
%    \begin{macro}{\HoLogoFont@font@sf}
%    \begin{macrocode}
\ltx@IfUndefined{sffamily}{%
  \ltx@IfUndefined{sf}{%
  }{%
    \HoLogoFont@Def{sf}{\sf}%
  }%
}{%
  \HoLogoFont@Def{sf}{\sffamily}%
}
%    \end{macrocode}
%    \end{macro}
%
%    \begin{macro}{\HoLogoFont@font@bibsf}
%    In case of \hologo{plainTeX} the original small caps
%    variant is used as default. In \hologo{LaTeX}
%    the definition of package \xpackage{dtklogos} \cite{dtklogos}
%    is used.
%\begin{quote}
%\begin{verbatim}
%\DeclareRobustCommand{\BibTeX}{%
%  B%
%  \kern-.05em%
%  \hbox{%
%    $\m@th$% %% force math size calculations
%    \csname S@\f@size\endcsname
%    \fontsize\sf@size\z@
%    \math@fontsfalse
%    \selectfont
%    I%
%    \kern-.025em%
%    B
%  }%
%  \kern-.08em%
%  \-%
%  \TeX
%}
%\end{verbatim}
%\end{quote}
%    \begin{macrocode}
\ltx@IfUndefined{selectfont}{%
  \ltx@IfUndefined{tensc}{%
    \font\tensc=cmcsc10\relax
  }{}%
  \HoLogoFont@Def{bibsf}{\tensc}%
}{%
  \HoLogoFont@Def{bibsf}{%
    $\mathsurround=0pt$%
    \csname S@\f@size\endcsname
    \fontsize\sf@size{0pt}%
    \math@fontsfalse
    \selectfont
  }%
}
%    \end{macrocode}
%    \end{macro}
%
%    \begin{macro}{\HoLogoFont@font@sc}
%    \begin{macrocode}
\ltx@IfUndefined{scshape}{%
  \ltx@IfUndefined{tensc}{%
    \font\tensc=cmcsc10\relax
  }{}%
  \HoLogoFont@Def{sc}{\tensc}%
}{%
  \HoLogoFont@Def{sc}{\scshape}%
}
%    \end{macrocode}
%    \end{macro}
%
%    \begin{macro}{\HoLogoFont@font@sy}
%    \begin{macrocode}
\ltx@IfUndefined{usefont}{%
  \ltx@IfUndefined{tensy}{%
  }{%
    \HoLogoFont@Def{sy}{\tensy}%
  }%
}{%
  \HoLogoFont@Def{sy}{%
    \usefont{OMS}{cmsy}{m}{n}%
  }%
}
%    \end{macrocode}
%    \end{macro}
%
%    \begin{macro}{\HoLogoFont@font@logo}
%    \begin{macrocode}
\begingroup
  \def\x{LaTeX2e}%
\expandafter\endgroup
\ifx\fmtname\x
  \ltx@IfUndefined{logofamily}{%
    \DeclareRobustCommand\logofamily{%
      \not@math@alphabet\logofamily\relax
      \fontencoding{U}%
      \fontfamily{logo}%
      \selectfont
    }%
  }{}%
  \ltx@IfUndefined{logofamily}{%
  }{%
    \HoLogoFont@Def{logo}{\logofamily}%
  }%
\else
  \ltx@IfUndefined{tenlogo}{%
    \font\tenlogo=logo10\relax
  }{}%
  \HoLogoFont@Def{logo}{\tenlogo}%
\fi
%    \end{macrocode}
%    \end{macro}
%
% \subsubsection{Font setup}
%
%    \begin{macro}{\hologoFontSetup}
%    \begin{macrocode}
\def\hologoFontSetup{%
  \let\HOLOGO@name\relax
  \HOLOGO@FontSetup
}
%    \end{macrocode}
%    \end{macro}
%
%    \begin{macro}{\hologoLogoFontSetup}
%    \begin{macrocode}
\def\hologoLogoFontSetup#1{%
  \edef\HOLOGO@name{#1}%
  \ltx@IfUndefined{HoLogo@\HOLOGO@name}{%
    \@PackageError{hologo}{%
      Unknown logo `\HOLOGO@name'%
    }\@ehc
    \ltx@gobble
  }{%
    \HOLOGO@FontSetup
  }%
}
%    \end{macrocode}
%    \end{macro}
%
%    \begin{macro}{\HOLOGO@FontSetup}
%    \begin{macrocode}
\def\HOLOGO@FontSetup{%
  \kvsetkeys{HoLogoFont}%
}
%    \end{macrocode}
%    \end{macro}
%
%    \begin{macrocode}
\def\HOLOGO@temp#1{%
  \kv@define@key{HoLogoFont}{#1}{%
    \ifx\HOLOGO@name\relax
      \HoLogoFont@Def{#1}{##1}%
    \else
      \HoLogoFont@LogoDef\HOLOGO@name{#1}{##1}%
    \fi
  }%
}
\HOLOGO@temp{general}
\HOLOGO@temp{sf}
%    \end{macrocode}
%
% \subsection{Generic logo commands}
%
%    \begin{macrocode}
\HOLOGO@IfExists\hologo{%
  \@PackageError{hologo}{%
    \string\hologo\ltx@space is already defined.\MessageBreak
    Package loading is aborted%
  }\@ehc
  \HOLOGO@AtEnd
}%
\HOLOGO@IfExists\hologoRobust{%
  \@PackageError{hologo}{%
    \string\hologoRobust\ltx@space is already defined.\MessageBreak
    Package loading is aborted%
  }\@ehc
  \HOLOGO@AtEnd
}%
%    \end{macrocode}
%
% \subsubsection{\cs{hologo} and friends}
%
%    \begin{macrocode}
\ifluatex
  \expandafter\ltx@firstofone
\else
  \expandafter\ltx@gobble
\fi
{%
  \ltx@IfUndefined{ifincsname}{%
    \ifnum\luatexversion<36 %
      \expandafter\ltx@gobble
    \else
      \expandafter\ltx@firstofone
    \fi
    {%
      \begingroup
        \ifcase0%
            \directlua{%
              if tex.enableprimitives then %
                tex.enableprimitives('HOLOGO@', {'ifincsname'})%
              else %
                tex.print('1')%
              end%
            }%
            \ifx\HOLOGO@ifincsname\@undefined 1\fi%
            \relax
          \expandafter\ltx@firstofone
        \else
          \endgroup
          \expandafter\ltx@gobble
        \fi
        {%
          \global\let\ifincsname\HOLOGO@ifincsname
        }%
      \HOLOGO@temp
    }%
  }{}%
}
%    \end{macrocode}
%    \begin{macrocode}
\ltx@IfUndefined{ifincsname}{%
  \catcode`$=14 %
}{%
  \catcode`$=9 %
}
%    \end{macrocode}
%
%    \begin{macro}{\hologo}
%    \begin{macrocode}
\def\hologo#1{%
$ \ifincsname
$   \ltx@ifundefined{HoLogoCs@\HOLOGO@Variant{#1}}{%
$     #1%
$   }{%
$     \csname HoLogoCs@\HOLOGO@Variant{#1}\endcsname\ltx@firstoftwo
$   }%
$ \else
    \HOLOGO@IfExists\texorpdfstring\texorpdfstring\ltx@firstoftwo
    {%
      \hologoRobust{#1}%
    }{%
      \ltx@ifundefined{HoLogoBkm@\HOLOGO@Variant{#1}}{%
        \ltx@ifundefined{HoLogo@#1}{?#1?}{#1}%
      }{%
        \csname HoLogoBkm@\HOLOGO@Variant{#1}\endcsname
        \ltx@firstoftwo
      }%
    }%
$ \fi
}
%    \end{macrocode}
%    \end{macro}
%    \begin{macro}{\Hologo}
%    \begin{macrocode}
\def\Hologo#1{%
$ \ifincsname
$   \ltx@ifundefined{HoLogoCs@\HOLOGO@Variant{#1}}{%
$     #1%
$   }{%
$     \csname HoLogoCs@\HOLOGO@Variant{#1}\endcsname\ltx@secondoftwo
$   }%
$ \else
    \HOLOGO@IfExists\texorpdfstring\texorpdfstring\ltx@firstoftwo
    {%
      \HologoRobust{#1}%
    }{%
      \ltx@ifundefined{HoLogoBkm@\HOLOGO@Variant{#1}}{%
        \ltx@ifundefined{HoLogo@#1}{?#1?}{#1}%
      }{%
        \csname HoLogoBkm@\HOLOGO@Variant{#1}\endcsname
        \ltx@secondoftwo
      }%
    }%
$ \fi
}
%    \end{macrocode}
%    \end{macro}
%
%    \begin{macro}{\hologoVariant}
%    \begin{macrocode}
\def\hologoVariant#1#2{%
  \ifx\relax#2\relax
    \hologo{#1}%
  \else
$   \ifincsname
$     \ltx@ifundefined{HoLogoCs@#1@#2}{%
$       #1%
$     }{%
$       \csname HoLogoCs@#1@#2\endcsname\ltx@firstoftwo
$     }%
$   \else
      \HOLOGO@IfExists\texorpdfstring\texorpdfstring\ltx@firstoftwo
      {%
        \hologoVariantRobust{#1}{#2}%
      }{%
        \ltx@ifundefined{HoLogoBkm@#1@#2}{%
          \ltx@ifundefined{HoLogo@#1}{?#1?}{#1}%
        }{%
          \csname HoLogoBkm@#1@#2\endcsname
          \ltx@firstoftwo
        }%
      }%
$   \fi
  \fi
}
%    \end{macrocode}
%    \end{macro}
%    \begin{macro}{\HologoVariant}
%    \begin{macrocode}
\def\HologoVariant#1#2{%
  \ifx\relax#2\relax
    \Hologo{#1}%
  \else
$   \ifincsname
$     \ltx@ifundefined{HoLogoCs@#1@#2}{%
$       #1%
$     }{%
$       \csname HoLogoCs@#1@#2\endcsname\ltx@secondoftwo
$     }%
$   \else
      \HOLOGO@IfExists\texorpdfstring\texorpdfstring\ltx@firstoftwo
      {%
        \HologoVariantRobust{#1}{#2}%
      }{%
        \ltx@ifundefined{HoLogoBkm@#1@#2}{%
          \ltx@ifundefined{HoLogo@#1}{?#1?}{#1}%
        }{%
          \csname HoLogoBkm@#1@#2\endcsname
          \ltx@secondoftwo
        }%
      }%
$   \fi
  \fi
}
%    \end{macrocode}
%    \end{macro}
%
%    \begin{macrocode}
\catcode`\$=3 %
%    \end{macrocode}
%
% \subsubsection{\cs{hologoRobust} and friends}
%
%    \begin{macro}{\hologoRobust}
%    \begin{macrocode}
\ltx@IfUndefined{protected}{%
  \ltx@IfUndefined{DeclareRobustCommand}{%
    \def\hologoRobust#1%
  }{%
    \DeclareRobustCommand*\hologoRobust[1]%
  }%
}{%
  \protected\def\hologoRobust#1%
}%
{%
  \edef\HOLOGO@name{#1}%
  \ltx@IfUndefined{HoLogo@\HOLOGO@Variant\HOLOGO@name}{%
    \@PackageError{hologo}{%
      Unknown logo `\HOLOGO@name'%
    }\@ehc
    ?\HOLOGO@name?%
  }{%
    \ltx@IfUndefined{ver@tex4ht.sty}{%
      \HoLogoFont@font\HOLOGO@name{general}{%
        \csname HoLogo@\HOLOGO@Variant\HOLOGO@name\endcsname
        \ltx@firstoftwo
      }%
    }{%
      \ltx@IfUndefined{HoLogoHtml@\HOLOGO@Variant\HOLOGO@name}{%
        \HOLOGO@name
      }{%
        \csname HoLogoHtml@\HOLOGO@Variant\HOLOGO@name\endcsname
        \ltx@firstoftwo
      }%
    }%
  }%
}
%    \end{macrocode}
%    \end{macro}
%    \begin{macro}{\HologoRobust}
%    \begin{macrocode}
\ltx@IfUndefined{protected}{%
  \ltx@IfUndefined{DeclareRobustCommand}{%
    \def\HologoRobust#1%
  }{%
    \DeclareRobustCommand*\HologoRobust[1]%
  }%
}{%
  \protected\def\HologoRobust#1%
}%
{%
  \edef\HOLOGO@name{#1}%
  \ltx@IfUndefined{HoLogo@\HOLOGO@Variant\HOLOGO@name}{%
    \@PackageError{hologo}{%
      Unknown logo `\HOLOGO@name'%
    }\@ehc
    ?\HOLOGO@name?%
  }{%
    \ltx@IfUndefined{ver@tex4ht.sty}{%
      \HoLogoFont@font\HOLOGO@name{general}{%
        \csname HoLogo@\HOLOGO@Variant\HOLOGO@name\endcsname
        \ltx@secondoftwo
      }%
    }{%
      \ltx@IfUndefined{HoLogoHtml@\HOLOGO@Variant\HOLOGO@name}{%
        \expandafter\HOLOGO@Uppercase\HOLOGO@name
      }{%
        \csname HoLogoHtml@\HOLOGO@Variant\HOLOGO@name\endcsname
        \ltx@secondoftwo
      }%
    }%
  }%
}
%    \end{macrocode}
%    \end{macro}
%    \begin{macro}{\hologoVariantRobust}
%    \begin{macrocode}
\ltx@IfUndefined{protected}{%
  \ltx@IfUndefined{DeclareRobustCommand}{%
    \def\hologoVariantRobust#1#2%
  }{%
    \DeclareRobustCommand*\hologoVariantRobust[2]%
  }%
}{%
  \protected\def\hologoVariantRobust#1#2%
}%
{%
  \begingroup
    \hologoLogoSetup{#1}{variant={#2}}%
    \hologoRobust{#1}%
  \endgroup
}
%    \end{macrocode}
%    \end{macro}
%    \begin{macro}{\HologoVariantRobust}
%    \begin{macrocode}
\ltx@IfUndefined{protected}{%
  \ltx@IfUndefined{DeclareRobustCommand}{%
    \def\HologoVariantRobust#1#2%
  }{%
    \DeclareRobustCommand*\HologoVariantRobust[2]%
  }%
}{%
  \protected\def\HologoVariantRobust#1#2%
}%
{%
  \begingroup
    \hologoLogoSetup{#1}{variant={#2}}%
    \HologoRobust{#1}%
  \endgroup
}
%    \end{macrocode}
%    \end{macro}
%
%    \begin{macro}{\hologorobust}
%    Macro \cs{hologorobust} is only defined for compatibility.
%    Its use is deprecated.
%    \begin{macrocode}
\def\hologorobust{\hologoRobust}
%    \end{macrocode}
%    \end{macro}
%
% \subsection{Helpers}
%
%    \begin{macro}{\HOLOGO@Uppercase}
%    Macro \cs{HOLOGO@Uppercase} is restricted to \cs{uppercase},
%    because \hologo{plainTeX} or \hologo{iniTeX} do not provide
%    \cs{MakeUppercase}.
%    \begin{macrocode}
\def\HOLOGO@Uppercase#1{\uppercase{#1}}
%    \end{macrocode}
%    \end{macro}
%
%    \begin{macro}{\HOLOGO@PdfdocUnicode}
%    \begin{macrocode}
\def\HOLOGO@PdfdocUnicode{%
  \ifx\ifHy@unicode\iftrue
    \expandafter\ltx@secondoftwo
  \else
    \expandafter\ltx@firstoftwo
  \fi
}
%    \end{macrocode}
%    \end{macro}
%
%    \begin{macro}{\HOLOGO@Math}
%    \begin{macrocode}
\def\HOLOGO@MathSetup{%
  \mathsurround0pt\relax
  \HOLOGO@IfExists\f@series{%
    \if b\expandafter\ltx@car\f@series x\@nil
      \csname boldmath\endcsname
   \fi
  }{}%
}
%    \end{macrocode}
%    \end{macro}
%
%    \begin{macro}{\HOLOGO@TempDimen}
%    \begin{macrocode}
\dimendef\HOLOGO@TempDimen=\ltx@zero
%    \end{macrocode}
%    \end{macro}
%    \begin{macro}{\HOLOGO@NegativeKerning}
%    \begin{macrocode}
\def\HOLOGO@NegativeKerning#1{%
  \begingroup
    \HOLOGO@TempDimen=0pt\relax
    \comma@parse@normalized{#1}{%
      \ifdim\HOLOGO@TempDimen=0pt %
        \expandafter\HOLOGO@@NegativeKerning\comma@entry
      \fi
      \ltx@gobble
    }%
    \ifdim\HOLOGO@TempDimen<0pt %
      \kern\HOLOGO@TempDimen
    \fi
  \endgroup
}
%    \end{macrocode}
%    \end{macro}
%    \begin{macro}{\HOLOGO@@NegativeKerning}
%    \begin{macrocode}
\def\HOLOGO@@NegativeKerning#1#2{%
  \setbox\ltx@zero\hbox{#1#2}%
  \HOLOGO@TempDimen=\wd\ltx@zero
  \setbox\ltx@zero\hbox{#1\kern0pt#2}%
  \advance\HOLOGO@TempDimen by -\wd\ltx@zero
}
%    \end{macrocode}
%    \end{macro}
%
%    \begin{macro}{\HOLOGO@SpaceFactor}
%    \begin{macrocode}
\def\HOLOGO@SpaceFactor{%
  \spacefactor1000 %
}
%    \end{macrocode}
%    \end{macro}
%
%    \begin{macro}{\HOLOGO@Span}
%    \begin{macrocode}
\def\HOLOGO@Span#1#2{%
  \HCode{<span class="HoLogo-#1">}%
  #2%
  \HCode{</span>}%
}
%    \end{macrocode}
%    \end{macro}
%
% \subsubsection{Text subscript}
%
%    \begin{macro}{\HOLOGO@SubScript}%
%    \begin{macrocode}
\def\HOLOGO@SubScript#1{%
  \ltx@IfUndefined{textsubscript}{%
    \ltx@IfUndefined{text}{%
      \ltx@mbox{%
        \mathsurround=0pt\relax
        $%
          _{%
            \ltx@IfUndefined{sf@size}{%
              \mathrm{#1}%
            }{%
              \mbox{%
                \fontsize\sf@size{0pt}\selectfont
                #1%
              }%
            }%
          }%
        $%
      }%
    }{%
      \ltx@mbox{%
        \mathsurround=0pt\relax
        $_{\text{#1}}$%
      }%
    }%
  }{%
    \textsubscript{#1}%
  }%
}
%    \end{macrocode}
%    \end{macro}
%
% \subsection{\hologo{TeX} and friends}
%
% \subsubsection{\hologo{TeX}}
%
%    \begin{macro}{\HoLogo@TeX}
%    Source: \hologo{LaTeX} kernel.
%    \begin{macrocode}
\def\HoLogo@TeX#1{%
  T\kern-.1667em\lower.5ex\hbox{E}\kern-.125emX\HOLOGO@SpaceFactor
}
%    \end{macrocode}
%    \end{macro}
%    \begin{macro}{\HoLogoHtml@TeX}
%    \begin{macrocode}
\def\HoLogoHtml@TeX#1{%
  \HoLogoCss@TeX
  \HOLOGO@Span{TeX}{%
    T%
    \HOLOGO@Span{e}{%
      E%
    }%
    X%
  }%
}
%    \end{macrocode}
%    \end{macro}
%    \begin{macro}{\HoLogoCss@TeX}
%    \begin{macrocode}
\def\HoLogoCss@TeX{%
  \Css{%
    span.HoLogo-TeX span.HoLogo-e{%
      position:relative;%
      top:.5ex;%
      margin-left:-.1667em;%
      margin-right:-.125em;%
    }%
  }%
  \Css{%
    a span.HoLogo-TeX span.HoLogo-e{%
      text-decoration:none;%
    }%
  }%
  \global\let\HoLogoCss@TeX\relax
}
%    \end{macrocode}
%    \end{macro}
%
% \subsubsection{\hologo{plainTeX}}
%
%    \begin{macro}{\HoLogo@plainTeX@space}
%    Source: ``The \hologo{TeX}book''
%    \begin{macrocode}
\def\HoLogo@plainTeX@space#1{%
  \HOLOGO@mbox{#1{p}{P}lain}\HOLOGO@space\hologo{TeX}%
}
%    \end{macrocode}
%    \end{macro}
%    \begin{macro}{\HoLogoCs@plainTeX@space}
%    \begin{macrocode}
\def\HoLogoCs@plainTeX@space#1{#1{p}{P}lain TeX}%
%    \end{macrocode}
%    \end{macro}
%    \begin{macro}{\HoLogoBkm@plainTeX@space}
%    \begin{macrocode}
\def\HoLogoBkm@plainTeX@space#1{%
  #1{p}{P}lain \hologo{TeX}%
}
%    \end{macrocode}
%    \end{macro}
%    \begin{macro}{\HoLogoHtml@plainTeX@space}
%    \begin{macrocode}
\def\HoLogoHtml@plainTeX@space#1{%
  #1{p}{P}lain \hologo{TeX}%
}
%    \end{macrocode}
%    \end{macro}
%
%    \begin{macro}{\HoLogo@plainTeX@hyphen}
%    \begin{macrocode}
\def\HoLogo@plainTeX@hyphen#1{%
  \HOLOGO@mbox{#1{p}{P}lain}\HOLOGO@hyphen\hologo{TeX}%
}
%    \end{macrocode}
%    \end{macro}
%    \begin{macro}{\HoLogoCs@plainTeX@hyphen}
%    \begin{macrocode}
\def\HoLogoCs@plainTeX@hyphen#1{#1{p}{P}lain-TeX}
%    \end{macrocode}
%    \end{macro}
%    \begin{macro}{\HoLogoBkm@plainTeX@hyphen}
%    \begin{macrocode}
\def\HoLogoBkm@plainTeX@hyphen#1{%
  #1{p}{P}lain-\hologo{TeX}%
}
%    \end{macrocode}
%    \end{macro}
%    \begin{macro}{\HoLogoHtml@plainTeX@hyphen}
%    \begin{macrocode}
\def\HoLogoHtml@plainTeX@hyphen#1{%
  #1{p}{P}lain-\hologo{TeX}%
}
%    \end{macrocode}
%    \end{macro}
%
%    \begin{macro}{\HoLogo@plainTeX@runtogether}
%    \begin{macrocode}
\def\HoLogo@plainTeX@runtogether#1{%
  \HOLOGO@mbox{#1{p}{P}lain\hologo{TeX}}%
}
%    \end{macrocode}
%    \end{macro}
%    \begin{macro}{\HoLogoCs@plainTeX@runtogether}
%    \begin{macrocode}
\def\HoLogoCs@plainTeX@runtogether#1{#1{p}{P}lainTeX}
%    \end{macrocode}
%    \end{macro}
%    \begin{macro}{\HoLogoBkm@plainTeX@runtogether}
%    \begin{macrocode}
\def\HoLogoBkm@plainTeX@runtogether#1{%
  #1{p}{P}lain\hologo{TeX}%
}
%    \end{macrocode}
%    \end{macro}
%    \begin{macro}{\HoLogoHtml@plainTeX@runtogether}
%    \begin{macrocode}
\def\HoLogoHtml@plainTeX@runtogether#1{%
  #1{p}{P}lain\hologo{TeX}%
}
%    \end{macrocode}
%    \end{macro}
%
%    \begin{macro}{\HoLogo@plainTeX}
%    \begin{macrocode}
\def\HoLogo@plainTeX{\HoLogo@plainTeX@space}
%    \end{macrocode}
%    \end{macro}
%    \begin{macro}{\HoLogoCs@plainTeX}
%    \begin{macrocode}
\def\HoLogoCs@plainTeX{\HoLogoCs@plainTeX@space}
%    \end{macrocode}
%    \end{macro}
%    \begin{macro}{\HoLogoBkm@plainTeX}
%    \begin{macrocode}
\def\HoLogoBkm@plainTeX{\HoLogoBkm@plainTeX@space}
%    \end{macrocode}
%    \end{macro}
%    \begin{macro}{\HoLogoHtml@plainTeX}
%    \begin{macrocode}
\def\HoLogoHtml@plainTeX{\HoLogoHtml@plainTeX@space}
%    \end{macrocode}
%    \end{macro}
%
% \subsubsection{\hologo{LaTeX}}
%
%    Source: \hologo{LaTeX} kernel.
%\begin{quote}
%\begin{verbatim}
%\DeclareRobustCommand{\LaTeX}{%
%  L%
%  \kern-.36em%
%  {%
%    \sbox\z@ T%
%    \vbox to\ht\z@{%
%      \hbox{%
%        \check@mathfonts
%        \fontsize\sf@size\z@
%        \math@fontsfalse
%        \selectfont
%        A%
%      }%
%      \vss
%    }%
%  }%
%  \kern-.15em%
%  \TeX
%}
%\end{verbatim}
%\end{quote}
%
%    \begin{macro}{\HoLogo@La}
%    \begin{macrocode}
\def\HoLogo@La#1{%
  L%
  \kern-.36em%
  \begingroup
    \setbox\ltx@zero\hbox{T}%
    \vbox to\ht\ltx@zero{%
      \hbox{%
        \ltx@ifundefined{check@mathfonts}{%
          \csname sevenrm\endcsname
        }{%
          \check@mathfonts
          \fontsize\sf@size{0pt}%
          \math@fontsfalse\selectfont
        }%
        A%
      }%
      \vss
    }%
  \endgroup
}
%    \end{macrocode}
%    \end{macro}
%
%    \begin{macro}{\HoLogo@LaTeX}
%    Source: \hologo{LaTeX} kernel.
%    \begin{macrocode}
\def\HoLogo@LaTeX#1{%
  \hologo{La}%
  \kern-.15em%
  \hologo{TeX}%
}
%    \end{macrocode}
%    \end{macro}
%    \begin{macro}{\HoLogoHtml@LaTeX}
%    \begin{macrocode}
\def\HoLogoHtml@LaTeX#1{%
  \HoLogoCss@LaTeX
  \HOLOGO@Span{LaTeX}{%
    L%
    \HOLOGO@Span{a}{%
      A%
    }%
    \hologo{TeX}%
  }%
}
%    \end{macrocode}
%    \end{macro}
%    \begin{macro}{\HoLogoCss@LaTeX}
%    \begin{macrocode}
\def\HoLogoCss@LaTeX{%
  \Css{%
    span.HoLogo-LaTeX span.HoLogo-a{%
      position:relative;%
      top:-.5ex;%
      margin-left:-.36em;%
      margin-right:-.15em;%
      font-size:85\%;%
    }%
  }%
  \global\let\HoLogoCss@LaTeX\relax
}
%    \end{macrocode}
%    \end{macro}
%
% \subsubsection{\hologo{(La)TeX}}
%
%    \begin{macro}{\HoLogo@LaTeXTeX}
%    The kerning around the parentheses is taken
%    from package \xpackage{dtklogos} \cite{dtklogos}.
%\begin{quote}
%\begin{verbatim}
%\DeclareRobustCommand{\LaTeXTeX}{%
%  (%
%  \kern-.15em%
%  L%
%  \kern-.36em%
%  {%
%    \sbox\z@ T%
%    \vbox to\ht0{%
%      \hbox{%
%        $\m@th$%
%        \csname S@\f@size\endcsname
%        \fontsize\sf@size\z@
%        \math@fontsfalse
%        \selectfont
%        A%
%      }%
%      \vss
%    }%
%  }%
%  \kern-.2em%
%  )%
%  \kern-.15em%
%  \TeX
%}
%\end{verbatim}
%\end{quote}
%    \begin{macrocode}
\def\HoLogo@LaTeXTeX#1{%
  (%
  \kern-.15em%
  \hologo{La}%
  \kern-.2em%
  )%
  \kern-.15em%
  \hologo{TeX}%
}
%    \end{macrocode}
%    \end{macro}
%    \begin{macro}{\HoLogoBkm@LaTeXTeX}
%    \begin{macrocode}
\def\HoLogoBkm@LaTeXTeX#1{(La)TeX}
%    \end{macrocode}
%    \end{macro}
%
%    \begin{macro}{\HoLogo@(La)TeX}
%    \begin{macrocode}
\expandafter
\let\csname HoLogo@(La)TeX\endcsname\HoLogo@LaTeXTeX
%    \end{macrocode}
%    \end{macro}
%    \begin{macro}{\HoLogoBkm@(La)TeX}
%    \begin{macrocode}
\expandafter
\let\csname HoLogoBkm@(La)TeX\endcsname\HoLogoBkm@LaTeXTeX
%    \end{macrocode}
%    \end{macro}
%    \begin{macro}{\HoLogoHtml@LaTeXTeX}
%    \begin{macrocode}
\def\HoLogoHtml@LaTeXTeX#1{%
  \HoLogoCss@LaTeXTeX
  \HOLOGO@Span{LaTeXTeX}{%
    (%
    \HOLOGO@Span{L}{L}%
    \HOLOGO@Span{a}{A}%
    \HOLOGO@Span{ParenRight}{)}%
    \hologo{TeX}%
  }%
}
%    \end{macrocode}
%    \end{macro}
%    \begin{macro}{\HoLogoHtml@(La)TeX}
%    Kerning after opening parentheses and before closing parentheses
%    is $-0.1$\,em. The original values $-0.15$\,em
%    looked too ugly for a serif font.
%    \begin{macrocode}
\expandafter
\let\csname HoLogoHtml@(La)TeX\endcsname\HoLogoHtml@LaTeXTeX
%    \end{macrocode}
%    \end{macro}
%    \begin{macro}{\HoLogoCss@LaTeXTeX}
%    \begin{macrocode}
\def\HoLogoCss@LaTeXTeX{%
  \Css{%
    span.HoLogo-LaTeXTeX span.HoLogo-L{%
      margin-left:-.1em;%
    }%
  }%
  \Css{%
    span.HoLogo-LaTeXTeX span.HoLogo-a{%
      position:relative;%
      top:-.5ex;%
      margin-left:-.36em;%
      margin-right:-.1em;%
      font-size:85\%;%
    }%
  }%
  \Css{%
    span.HoLogo-LaTeXTeX span.HoLogo-ParenRight{%
      margin-right:-.15em;%
    }%
  }%
  \global\let\HoLogoCss@LaTeXTeX\relax
}
%    \end{macrocode}
%    \end{macro}
%
% \subsubsection{\hologo{LaTeXe}}
%
%    \begin{macro}{\HoLogo@LaTeXe}
%    Source: \hologo{LaTeX} kernel
%    \begin{macrocode}
\def\HoLogo@LaTeXe#1{%
  \hologo{LaTeX}%
  \kern.15em%
  \hbox{%
    \HOLOGO@MathSetup
    2%
    $_{\textstyle\varepsilon}$%
  }%
}
%    \end{macrocode}
%    \end{macro}
%
%    \begin{macro}{\HoLogoCs@LaTeXe}
%    \begin{macrocode}
\ifnum64=`\^^^^0040\relax % test for big chars of LuaTeX/XeTeX
  \catcode`\$=9 %
  \catcode`\&=14 %
\else
  \catcode`\$=14 %
  \catcode`\&=9 %
\fi
\def\HoLogoCs@LaTeXe#1{%
  LaTeX2%
$ \string ^^^^0395%
& e%
}%
\catcode`\$=3 %
\catcode`\&=4 %
%    \end{macrocode}
%    \end{macro}
%
%    \begin{macro}{\HoLogoBkm@LaTeXe}
%    \begin{macrocode}
\def\HoLogoBkm@LaTeXe#1{%
  \hologo{LaTeX}%
  2%
  \HOLOGO@PdfdocUnicode{e}{\textepsilon}%
}
%    \end{macrocode}
%    \end{macro}
%
%    \begin{macro}{\HoLogoHtml@LaTeXe}
%    \begin{macrocode}
\def\HoLogoHtml@LaTeXe#1{%
  \HoLogoCss@LaTeXe
  \HOLOGO@Span{LaTeX2e}{%
    \hologo{LaTeX}%
    \HOLOGO@Span{2}{2}%
    \HOLOGO@Span{e}{%
      \HOLOGO@MathSetup
      \ensuremath{\textstyle\varepsilon}%
    }%
  }%
}
%    \end{macrocode}
%    \end{macro}
%    \begin{macro}{\HoLogoCss@LaTeXe}
%    \begin{macrocode}
\def\HoLogoCss@LaTeXe{%
  \Css{%
    span.HoLogo-LaTeX2e span.HoLogo-2{%
      padding-left:.15em;%
    }%
  }%
  \Css{%
    span.HoLogo-LaTeX2e span.HoLogo-e{%
      position:relative;%
      top:.35ex;%
      text-decoration:none;%
    }%
  }%
  \global\let\HoLogoCss@LaTeXe\relax
}
%    \end{macrocode}
%    \end{macro}
%
%    \begin{macro}{\HoLogo@LaTeX2e}
%    \begin{macrocode}
\expandafter
\let\csname HoLogo@LaTeX2e\endcsname\HoLogo@LaTeXe
%    \end{macrocode}
%    \end{macro}
%    \begin{macro}{\HoLogoCs@LaTeX2e}
%    \begin{macrocode}
\expandafter
\let\csname HoLogoCs@LaTeX2e\endcsname\HoLogoCs@LaTeXe
%    \end{macrocode}
%    \end{macro}
%    \begin{macro}{\HoLogoBkm@LaTeX2e}
%    \begin{macrocode}
\expandafter
\let\csname HoLogoBkm@LaTeX2e\endcsname\HoLogoBkm@LaTeXe
%    \end{macrocode}
%    \end{macro}
%    \begin{macro}{\HoLogoHtml@LaTeX2e}
%    \begin{macrocode}
\expandafter
\let\csname HoLogoHtml@LaTeX2e\endcsname\HoLogoHtml@LaTeXe
%    \end{macrocode}
%    \end{macro}
%
% \subsubsection{\hologo{LaTeX3}}
%
%    \begin{macro}{\HoLogo@LaTeX3}
%    Source: \hologo{LaTeX} kernel
%    \begin{macrocode}
\expandafter\def\csname HoLogo@LaTeX3\endcsname#1{%
  \hologo{LaTeX}%
  3%
}
%    \end{macrocode}
%    \end{macro}
%
%    \begin{macro}{\HoLogoBkm@LaTeX3}
%    \begin{macrocode}
\expandafter\def\csname HoLogoBkm@LaTeX3\endcsname#1{%
  \hologo{LaTeX}%
  3%
}
%    \end{macrocode}
%    \end{macro}
%    \begin{macro}{\HoLogoHtml@LaTeX3}
%    \begin{macrocode}
\expandafter
\let\csname HoLogoHtml@LaTeX3\expandafter\endcsname
\csname HoLogo@LaTeX3\endcsname
%    \end{macrocode}
%    \end{macro}
%
% \subsubsection{\hologo{LaTeXML}}
%
%    \begin{macro}{\HoLogo@LaTeXML}
%    \begin{macrocode}
\def\HoLogo@LaTeXML#1{%
  \HOLOGO@mbox{%
    \hologo{La}%
    \kern-.15em%
    T%
    \kern-.1667em%
    \lower.5ex\hbox{E}%
    \kern-.125em%
    \HoLogoFont@font{LaTeXML}{sc}{xml}%
  }%
}
%    \end{macrocode}
%    \end{macro}
%    \begin{macro}{\HoLogoHtml@pdfLaTeX}
%    \begin{macrocode}
\def\HoLogoHtml@LaTeXML#1{%
  \HOLOGO@Span{LaTeXML}{%
    \HoLogoCss@LaTeX
    \HoLogoCss@TeX
    \HOLOGO@Span{LaTeX}{%
      L%
      \HOLOGO@Span{a}{%
        A%
      }%
    }%
    \HOLOGO@Span{TeX}{%
      T%
      \HOLOGO@Span{e}{%
        E%
      }%
    }%
    \HCode{<span style="font-variant: small-caps;">}%
    xml%
    \HCode{</span>}%
  }%
}
%    \end{macrocode}
%    \end{macro}
%
% \subsubsection{\hologo{eTeX}}
%
%    \begin{macro}{\HoLogo@eTeX}
%    Source: package \xpackage{etex}
%    \begin{macrocode}
\def\HoLogo@eTeX#1{%
  \ltx@mbox{%
    \HOLOGO@MathSetup
    $\varepsilon$%
    -%
    \HOLOGO@NegativeKerning{-T,T-,To}%
    \hologo{TeX}%
  }%
}
%    \end{macrocode}
%    \end{macro}
%    \begin{macro}{\HoLogoCs@eTeX}
%    \begin{macrocode}
\ifnum64=`\^^^^0040\relax % test for big chars of LuaTeX/XeTeX
  \catcode`\$=9 %
  \catcode`\&=14 %
\else
  \catcode`\$=14 %
  \catcode`\&=9 %
\fi
\def\HoLogoCs@eTeX#1{%
$ #1{\string ^^^^0395}{\string ^^^^03b5}%
& #1{e}{E}%
  TeX%
}%
\catcode`\$=3 %
\catcode`\&=4 %
%    \end{macrocode}
%    \end{macro}
%    \begin{macro}{\HoLogoBkm@eTeX}
%    \begin{macrocode}
\def\HoLogoBkm@eTeX#1{%
  \HOLOGO@PdfdocUnicode{#1{e}{E}}{\textepsilon}%
  -%
  \hologo{TeX}%
}
%    \end{macrocode}
%    \end{macro}
%    \begin{macro}{\HoLogoHtml@eTeX}
%    \begin{macrocode}
\def\HoLogoHtml@eTeX#1{%
  \ltx@mbox{%
    \HOLOGO@MathSetup
    $\varepsilon$%
    -%
    \hologo{TeX}%
  }%
}
%    \end{macrocode}
%    \end{macro}
%
% \subsubsection{\hologo{iniTeX}}
%
%    \begin{macro}{\HoLogo@iniTeX}
%    \begin{macrocode}
\def\HoLogo@iniTeX#1{%
  \HOLOGO@mbox{%
    #1{i}{I}ni\hologo{TeX}%
  }%
}
%    \end{macrocode}
%    \end{macro}
%    \begin{macro}{\HoLogoCs@iniTeX}
%    \begin{macrocode}
\def\HoLogoCs@iniTeX#1{#1{i}{I}niTeX}
%    \end{macrocode}
%    \end{macro}
%    \begin{macro}{\HoLogoBkm@iniTeX}
%    \begin{macrocode}
\def\HoLogoBkm@iniTeX#1{%
  #1{i}{I}ni\hologo{TeX}%
}
%    \end{macrocode}
%    \end{macro}
%    \begin{macro}{\HoLogoHtml@iniTeX}
%    \begin{macrocode}
\let\HoLogoHtml@iniTeX\HoLogo@iniTeX
%    \end{macrocode}
%    \end{macro}
%
% \subsubsection{\hologo{virTeX}}
%
%    \begin{macro}{\HoLogo@virTeX}
%    \begin{macrocode}
\def\HoLogo@virTeX#1{%
  \HOLOGO@mbox{%
    #1{v}{V}ir\hologo{TeX}%
  }%
}
%    \end{macrocode}
%    \end{macro}
%    \begin{macro}{\HoLogoCs@virTeX}
%    \begin{macrocode}
\def\HoLogoCs@virTeX#1{#1{v}{V}irTeX}
%    \end{macrocode}
%    \end{macro}
%    \begin{macro}{\HoLogoBkm@virTeX}
%    \begin{macrocode}
\def\HoLogoBkm@virTeX#1{%
  #1{v}{V}ir\hologo{TeX}%
}
%    \end{macrocode}
%    \end{macro}
%    \begin{macro}{\HoLogoHtml@virTeX}
%    \begin{macrocode}
\let\HoLogoHtml@virTeX\HoLogo@virTeX
%    \end{macrocode}
%    \end{macro}
%
% \subsubsection{\hologo{SliTeX}}
%
% \paragraph{Definitions of the three variants.}
%
%    \begin{macro}{\HoLogo@SLiTeX@lift}
%    \begin{macrocode}
\def\HoLogo@SLiTeX@lift#1{%
  \HoLogoFont@font{SliTeX}{rm}{%
    S%
    \kern-.06em%
    L%
    \kern-.18em%
    \raise.32ex\hbox{\HoLogoFont@font{SliTeX}{sc}{i}}%
    \HOLOGO@discretionary
    \kern-.06em%
    \hologo{TeX}%
  }%
}
%    \end{macrocode}
%    \end{macro}
%    \begin{macro}{\HoLogoBkm@SLiTeX@lift}
%    \begin{macrocode}
\def\HoLogoBkm@SLiTeX@lift#1{SLiTeX}
%    \end{macrocode}
%    \end{macro}
%    \begin{macro}{\HoLogoHtml@SLiTeX@lift}
%    \begin{macrocode}
\def\HoLogoHtml@SLiTeX@lift#1{%
  \HoLogoCss@SLiTeX@lift
  \HOLOGO@Span{SLiTeX-lift}{%
    \HoLogoFont@font{SliTeX}{rm}{%
      S%
      \HOLOGO@Span{L}{L}%
      \HOLOGO@Span{i}{i}%
      \hologo{TeX}%
    }%
  }%
}
%    \end{macrocode}
%    \end{macro}
%    \begin{macro}{\HoLogoCss@SLiTeX@lift}
%    \begin{macrocode}
\def\HoLogoCss@SLiTeX@lift{%
  \Css{%
    span.HoLogo-SLiTeX-lift span.HoLogo-L{%
      margin-left:-.06em;%
      margin-right:-.18em;%
    }%
  }%
  \Css{%
    span.HoLogo-SLiTeX-lift span.HoLogo-i{%
      position:relative;%
      top:-.32ex;%
      margin-right:-.06em;%
      font-variant:small-caps;%
    }%
  }%
  \global\let\HoLogoCss@SLiTeX@lift\relax
}
%    \end{macrocode}
%    \end{macro}
%
%    \begin{macro}{\HoLogo@SliTeX@simple}
%    \begin{macrocode}
\def\HoLogo@SliTeX@simple#1{%
  \HoLogoFont@font{SliTeX}{rm}{%
    \ltx@mbox{%
      \HoLogoFont@font{SliTeX}{sc}{Sli}%
    }%
    \HOLOGO@discretionary
    \hologo{TeX}%
  }%
}
%    \end{macrocode}
%    \end{macro}
%    \begin{macro}{\HoLogoBkm@SliTeX@simple}
%    \begin{macrocode}
\def\HoLogoBkm@SliTeX@simple#1{SliTeX}
%    \end{macrocode}
%    \end{macro}
%    \begin{macro}{\HoLogoHtml@SliTeX@simple}
%    \begin{macrocode}
\let\HoLogoHtml@SliTeX@simple\HoLogo@SliTeX@simple
%    \end{macrocode}
%    \end{macro}
%
%    \begin{macro}{\HoLogo@SliTeX@narrow}
%    \begin{macrocode}
\def\HoLogo@SliTeX@narrow#1{%
  \HoLogoFont@font{SliTeX}{rm}{%
    \ltx@mbox{%
      S%
      \kern-.06em%
      \HoLogoFont@font{SliTeX}{sc}{%
        l%
        \kern-.035em%
        i%
      }%
    }%
    \HOLOGO@discretionary
    \kern-.06em%
    \hologo{TeX}%
  }%
}
%    \end{macrocode}
%    \end{macro}
%    \begin{macro}{\HoLogoBkm@SliTeX@narrow}
%    \begin{macrocode}
\def\HoLogoBkm@SliTeX@narrow#1{SliTeX}
%    \end{macrocode}
%    \end{macro}
%    \begin{macro}{\HoLogoHtml@SliTeX@narrow}
%    \begin{macrocode}
\def\HoLogoHtml@SliTeX@narrow#1{%
  \HoLogoCss@SliTeX@narrow
  \HOLOGO@Span{SliTeX-narrow}{%
    \HoLogoFont@font{SliTeX}{rm}{%
      S%
        \HOLOGO@Span{l}{l}%
        \HOLOGO@Span{i}{i}%
      \hologo{TeX}%
    }%
  }%
}
%    \end{macrocode}
%    \end{macro}
%    \begin{macro}{\HoLogoCss@SliTeX@narrow}
%    \begin{macrocode}
\def\HoLogoCss@SliTeX@narrow{%
  \Css{%
    span.HoLogo-SliTeX-narrow span.HoLogo-l{%
      margin-left:-.06em;%
      margin-right:-.035em;%
      font-variant:small-caps;%
    }%
  }%
  \Css{%
    span.HoLogo-SliTeX-narrow span.HoLogo-i{%
      margin-right:-.06em;%
      font-variant:small-caps;%
    }%
  }%
  \global\let\HoLogoCss@SliTeX@narrow\relax
}
%    \end{macrocode}
%    \end{macro}
%
% \paragraph{Macro set completion.}
%
%    \begin{macro}{\HoLogo@SLiTeX@simple}
%    \begin{macrocode}
\def\HoLogo@SLiTeX@simple{\HoLogo@SliTeX@simple}
%    \end{macrocode}
%    \end{macro}
%    \begin{macro}{\HoLogoBkm@SLiTeX@simple}
%    \begin{macrocode}
\def\HoLogoBkm@SLiTeX@simple{\HoLogoBkm@SliTeX@simple}
%    \end{macrocode}
%    \end{macro}
%    \begin{macro}{\HoLogoHtml@SLiTeX@simple}
%    \begin{macrocode}
\def\HoLogoHtml@SLiTeX@simple{\HoLogoHtml@SliTeX@simple}
%    \end{macrocode}
%    \end{macro}
%
%    \begin{macro}{\HoLogo@SLiTeX@narrow}
%    \begin{macrocode}
\def\HoLogo@SLiTeX@narrow{\HoLogo@SliTeX@narrow}
%    \end{macrocode}
%    \end{macro}
%    \begin{macro}{\HoLogoBkm@SLiTeX@narrow}
%    \begin{macrocode}
\def\HoLogoBkm@SLiTeX@narrow{\HoLogoBkm@SliTeX@narrow}
%    \end{macrocode}
%    \end{macro}
%    \begin{macro}{\HoLogoHtml@SLiTeX@narrow}
%    \begin{macrocode}
\def\HoLogoHtml@SLiTeX@narrow{\HoLogoHtml@SliTeX@narrow}
%    \end{macrocode}
%    \end{macro}
%
%    \begin{macro}{\HoLogo@SliTeX@lift}
%    \begin{macrocode}
\def\HoLogo@SliTeX@lift{\HoLogo@SLiTeX@lift}
%    \end{macrocode}
%    \end{macro}
%    \begin{macro}{\HoLogoBkm@SliTeX@lift}
%    \begin{macrocode}
\def\HoLogoBkm@SliTeX@lift{\HoLogoBkm@SLiTeX@lift}
%    \end{macrocode}
%    \end{macro}
%    \begin{macro}{\HoLogoHtml@SliTeX@lift}
%    \begin{macrocode}
\def\HoLogoHtml@SliTeX@lift{\HoLogoHtml@SLiTeX@lift}
%    \end{macrocode}
%    \end{macro}
%
% \paragraph{Defaults.}
%
%    \begin{macro}{\HoLogo@SLiTeX}
%    \begin{macrocode}
\def\HoLogo@SLiTeX{\HoLogo@SLiTeX@lift}
%    \end{macrocode}
%    \end{macro}
%    \begin{macro}{\HoLogoBkm@SLiTeX}
%    \begin{macrocode}
\def\HoLogoBkm@SLiTeX{\HoLogoBkm@SLiTeX@lift}
%    \end{macrocode}
%    \end{macro}
%    \begin{macro}{\HoLogoHtml@SLiTeX}
%    \begin{macrocode}
\def\HoLogoHtml@SLiTeX{\HoLogoHtml@SLiTeX@lift}
%    \end{macrocode}
%    \end{macro}
%
%    \begin{macro}{\HoLogo@SliTeX}
%    \begin{macrocode}
\def\HoLogo@SliTeX{\HoLogo@SliTeX@narrow}
%    \end{macrocode}
%    \end{macro}
%    \begin{macro}{\HoLogoBkm@SliTeX}
%    \begin{macrocode}
\def\HoLogoBkm@SliTeX{\HoLogoBkm@SliTeX@narrow}
%    \end{macrocode}
%    \end{macro}
%    \begin{macro}{\HoLogoHtml@SliTeX}
%    \begin{macrocode}
\def\HoLogoHtml@SliTeX{\HoLogoHtml@SliTeX@narrow}
%    \end{macrocode}
%    \end{macro}
%
% \subsubsection{\hologo{LuaTeX}}
%
%    \begin{macro}{\HoLogo@LuaTeX}
%    The kerning is an idea of Hans Hagen, see mailing list
%    `luatex at tug dot org' in March 2010.
%    \begin{macrocode}
\def\HoLogo@LuaTeX#1{%
  \HOLOGO@mbox{%
    Lua%
    \HOLOGO@NegativeKerning{aT,oT,To}%
    \hologo{TeX}%
  }%
}
%    \end{macrocode}
%    \end{macro}
%    \begin{macro}{\HoLogoHtml@LuaTeX}
%    \begin{macrocode}
\let\HoLogoHtml@LuaTeX\HoLogo@LuaTeX
%    \end{macrocode}
%    \end{macro}
%
% \subsubsection{\hologo{LuaLaTeX}}
%
%    \begin{macro}{\HoLogo@LuaLaTeX}
%    \begin{macrocode}
\def\HoLogo@LuaLaTeX#1{%
  \HOLOGO@mbox{%
    Lua%
    \hologo{LaTeX}%
  }%
}
%    \end{macrocode}
%    \end{macro}
%    \begin{macro}{\HoLogoHtml@LuaLaTeX}
%    \begin{macrocode}
\let\HoLogoHtml@LuaLaTeX\HoLogo@LuaLaTeX
%    \end{macrocode}
%    \end{macro}
%
% \subsubsection{\hologo{XeTeX}, \hologo{XeLaTeX}}
%
%    \begin{macro}{\HOLOGO@IfCharExists}
%    \begin{macrocode}
\ifluatex
  \ifnum\luatexversion<36 %
  \else
    \def\HOLOGO@IfCharExists#1{%
      \ifnum
        \directlua{%
           if luaotfload and luaotfload.aux then
             if luaotfload.aux.font_has_glyph(%
                    font.current(), \number#1) then % 	 
	       tex.print("1") % 	 
	     end % 	 
	   elseif font and font.fonts and font.current then %
            local f = font.fonts[font.current()]%
            if f.characters and f.characters[\number#1] then %
              tex.print("1")%
            end %
          end%
        }0=\ltx@zero
        \expandafter\ltx@secondoftwo
      \else
        \expandafter\ltx@firstoftwo
      \fi
    }%
  \fi
\fi
\ltx@IfUndefined{HOLOGO@IfCharExists}{%
  \def\HOLOGO@@IfCharExists#1{%
    \begingroup
      \tracinglostchars=\ltx@zero
      \setbox\ltx@zero=\hbox{%
        \kern7sp\char#1\relax
        \ifnum\lastkern>\ltx@zero
          \expandafter\aftergroup\csname iffalse\endcsname
        \else
          \expandafter\aftergroup\csname iftrue\endcsname
        \fi
      }%
      % \if{true|false} from \aftergroup
      \endgroup
      \expandafter\ltx@firstoftwo
    \else
      \endgroup
      \expandafter\ltx@secondoftwo
    \fi
  }%
  \ifxetex
    \ltx@IfUndefined{XeTeXfonttype}{}{%
      \ltx@IfUndefined{XeTeXcharglyph}{}{%
        \def\HOLOGO@IfCharExists#1{%
          \ifnum\XeTeXfonttype\font>\ltx@zero
            \expandafter\ltx@firstofthree
          \else
            \expandafter\ltx@gobble
          \fi
          {%
            \ifnum\XeTeXcharglyph#1>\ltx@zero
              \expandafter\ltx@firstoftwo
            \else
              \expandafter\ltx@secondoftwo
            \fi
          }%
          \HOLOGO@@IfCharExists{#1}%
        }%
      }%
    }%
  \fi
}{}
\ltx@ifundefined{HOLOGO@IfCharExists}{%
  \ifnum64=`\^^^^0040\relax % test for big chars of LuaTeX/XeTeX
    \let\HOLOGO@IfCharExists\HOLOGO@@IfCharExists
  \else
    \def\HOLOGO@IfCharExists#1{%
      \ifnum#1>255 %
        \expandafter\ltx@fourthoffour
      \fi
      \HOLOGO@@IfCharExists{#1}%
    }%
  \fi
}{}
%    \end{macrocode}
%    \end{macro}
%
%    \begin{macro}{\HoLogo@Xe}
%    Source: package \xpackage{dtklogos}
%    \begin{macrocode}
\def\HoLogo@Xe#1{%
  X%
  \kern-.1em\relax
  \HOLOGO@IfCharExists{"018E}{%
    \lower.5ex\hbox{\char"018E}%
  }{%
    \chardef\HOLOGO@choice=\ltx@zero
    \ifdim\fontdimen\ltx@one\font>0pt %
      \ltx@IfUndefined{rotatebox}{%
        \ltx@IfUndefined{pgftext}{%
          \ltx@IfUndefined{psscalebox}{%
            \ltx@IfUndefined{HOLOGO@ScaleBox@\hologoDriver}{%
            }{%
              \chardef\HOLOGO@choice=4 %
            }%
          }{%
            \chardef\HOLOGO@choice=3 %
          }%
        }{%
          \chardef\HOLOGO@choice=2 %
        }%
      }{%
        \chardef\HOLOGO@choice=1 %
      }%
      \ifcase\HOLOGO@choice
        \HOLOGO@WarningUnsupportedDriver{Xe}%
        e%
      \or % 1: \rotatebox
        \begingroup
          \setbox\ltx@zero\hbox{\rotatebox{180}{E}}%
          \ltx@LocDimenA=\dp\ltx@zero
          \advance\ltx@LocDimenA by -.5ex\relax
          \raise\ltx@LocDimenA\box\ltx@zero
        \endgroup
      \or % 2: \pgftext
        \lower.5ex\hbox{%
          \pgfpicture
            \pgftext[rotate=180]{E}%
          \endpgfpicture
        }%
      \or % 3: \psscalebox
        \begingroup
          \setbox\ltx@zero\hbox{\psscalebox{-1 -1}{E}}%
          \ltx@LocDimenA=\dp\ltx@zero
          \advance\ltx@LocDimenA by -.5ex\relax
          \raise\ltx@LocDimenA\box\ltx@zero
        \endgroup
      \or % 4: \HOLOGO@PointReflectBox
        \lower.5ex\hbox{\HOLOGO@PointReflectBox{E}}%
      \else
        \@PackageError{hologo}{Internal error (choice/it}\@ehc
      \fi
    \else
      \ltx@IfUndefined{reflectbox}{%
        \ltx@IfUndefined{pgftext}{%
          \ltx@IfUndefined{psscalebox}{%
            \ltx@IfUndefined{HOLOGO@ScaleBox@\hologoDriver}{%
            }{%
              \chardef\HOLOGO@choice=4 %
            }%
          }{%
            \chardef\HOLOGO@choice=3 %
          }%
        }{%
          \chardef\HOLOGO@choice=2 %
        }%
      }{%
        \chardef\HOLOGO@choice=1 %
      }%
      \ifcase\HOLOGO@choice
        \HOLOGO@WarningUnsupportedDriver{Xe}%
        e%
      \or % 1: reflectbox
        \lower.5ex\hbox{%
          \reflectbox{E}%
        }%
      \or % 2: \pgftext
        \lower.5ex\hbox{%
          \pgfpicture
            \pgftransformxscale{-1}%
            \pgftext{E}%
          \endpgfpicture
        }%
      \or % 3: \psscalebox
        \lower.5ex\hbox{%
          \psscalebox{-1 1}{E}%
        }%
      \or % 4: \HOLOGO@Reflectbox
        \lower.5ex\hbox{%
          \HOLOGO@ReflectBox{E}%
        }%
      \else
        \@PackageError{hologo}{Internal error (choice/up)}\@ehc
      \fi
    \fi
  }%
}
%    \end{macrocode}
%    \end{macro}
%    \begin{macro}{\HoLogoHtml@Xe}
%    \begin{macrocode}
\def\HoLogoHtml@Xe#1{%
  \HoLogoCss@Xe
  \HOLOGO@Span{Xe}{%
    X%
    \HOLOGO@Span{e}{%
      \HCode{&\ltx@hashchar x018e;}%
    }%
  }%
}
%    \end{macrocode}
%    \end{macro}
%    \begin{macro}{\HoLogoCss@Xe}
%    \begin{macrocode}
\def\HoLogoCss@Xe{%
  \Css{%
    span.HoLogo-Xe span.HoLogo-e{%
      position:relative;%
      top:.5ex;%
      left-margin:-.1em;%
    }%
  }%
  \global\let\HoLogoCss@Xe\relax
}
%    \end{macrocode}
%    \end{macro}
%
%    \begin{macro}{\HoLogo@XeTeX}
%    \begin{macrocode}
\def\HoLogo@XeTeX#1{%
  \hologo{Xe}%
  \kern-.15em\relax
  \hologo{TeX}%
}
%    \end{macrocode}
%    \end{macro}
%
%    \begin{macro}{\HoLogoHtml@XeTeX}
%    \begin{macrocode}
\def\HoLogoHtml@XeTeX#1{%
  \HoLogoCss@XeTeX
  \HOLOGO@Span{XeTeX}{%
    \hologo{Xe}%
    \hologo{TeX}%
  }%
}
%    \end{macrocode}
%    \end{macro}
%    \begin{macro}{\HoLogoCss@XeTeX}
%    \begin{macrocode}
\def\HoLogoCss@XeTeX{%
  \Css{%
    span.HoLogo-XeTeX span.HoLogo-TeX{%
      margin-left:-.15em;%
    }%
  }%
  \global\let\HoLogoCss@XeTeX\relax
}
%    \end{macrocode}
%    \end{macro}
%
%    \begin{macro}{\HoLogo@XeLaTeX}
%    \begin{macrocode}
\def\HoLogo@XeLaTeX#1{%
  \hologo{Xe}%
  \kern-.13em%
  \hologo{LaTeX}%
}
%    \end{macrocode}
%    \end{macro}
%    \begin{macro}{\HoLogoHtml@XeLaTeX}
%    \begin{macrocode}
\def\HoLogoHtml@XeLaTeX#1{%
  \HoLogoCss@XeLaTeX
  \HOLOGO@Span{XeLaTeX}{%
    \hologo{Xe}%
    \hologo{LaTeX}%
  }%
}
%    \end{macrocode}
%    \end{macro}
%    \begin{macro}{\HoLogoCss@XeLaTeX}
%    \begin{macrocode}
\def\HoLogoCss@XeLaTeX{%
  \Css{%
    span.HoLogo-XeLaTeX span.HoLogo-Xe{%
      margin-right:-.13em;%
    }%
  }%
  \global\let\HoLogoCss@XeLaTeX\relax
}
%    \end{macrocode}
%    \end{macro}
%
% \subsubsection{\hologo{pdfTeX}, \hologo{pdfLaTeX}}
%
%    \begin{macro}{\HoLogo@pdfTeX}
%    \begin{macrocode}
\def\HoLogo@pdfTeX#1{%
  \HOLOGO@mbox{%
    #1{p}{P}df\hologo{TeX}%
  }%
}
%    \end{macrocode}
%    \end{macro}
%    \begin{macro}{\HoLogoCs@pdfTeX}
%    \begin{macrocode}
\def\HoLogoCs@pdfTeX#1{#1{p}{P}dfTeX}
%    \end{macrocode}
%    \end{macro}
%    \begin{macro}{\HoLogoBkm@pdfTeX}
%    \begin{macrocode}
\def\HoLogoBkm@pdfTeX#1{%
  #1{p}{P}df\hologo{TeX}%
}
%    \end{macrocode}
%    \end{macro}
%    \begin{macro}{\HoLogoHtml@pdfTeX}
%    \begin{macrocode}
\let\HoLogoHtml@pdfTeX\HoLogo@pdfTeX
%    \end{macrocode}
%    \end{macro}
%
%    \begin{macro}{\HoLogo@pdfLaTeX}
%    \begin{macrocode}
\def\HoLogo@pdfLaTeX#1{%
  \HOLOGO@mbox{%
    #1{p}{P}df\hologo{LaTeX}%
  }%
}
%    \end{macrocode}
%    \end{macro}
%    \begin{macro}{\HoLogoCs@pdfLaTeX}
%    \begin{macrocode}
\def\HoLogoCs@pdfLaTeX#1{#1{p}{P}dfLaTeX}
%    \end{macrocode}
%    \end{macro}
%    \begin{macro}{\HoLogoBkm@pdfLaTeX}
%    \begin{macrocode}
\def\HoLogoBkm@pdfLaTeX#1{%
  #1{p}{P}df\hologo{LaTeX}%
}
%    \end{macrocode}
%    \end{macro}
%    \begin{macro}{\HoLogoHtml@pdfLaTeX}
%    \begin{macrocode}
\let\HoLogoHtml@pdfLaTeX\HoLogo@pdfLaTeX
%    \end{macrocode}
%    \end{macro}
%
% \subsubsection{\hologo{VTeX}}
%
%    \begin{macro}{\HoLogo@VTeX}
%    \begin{macrocode}
\def\HoLogo@VTeX#1{%
  \HOLOGO@mbox{%
    V\hologo{TeX}%
  }%
}
%    \end{macrocode}
%    \end{macro}
%    \begin{macro}{\HoLogoHtml@VTeX}
%    \begin{macrocode}
\let\HoLogoHtml@VTeX\HoLogo@VTeX
%    \end{macrocode}
%    \end{macro}
%
% \subsubsection{\hologo{AmS}, \dots}
%
%    Source: class \xclass{amsdtx}
%
%    \begin{macro}{\HoLogo@AmS}
%    \begin{macrocode}
\def\HoLogo@AmS#1{%
  \HoLogoFont@font{AmS}{sy}{%
    A%
    \kern-.1667em%
    \lower.5ex\hbox{M}%
    \kern-.125em%
    S%
  }%
}
%    \end{macrocode}
%    \end{macro}
%    \begin{macro}{\HoLogoBkm@AmS}
%    \begin{macrocode}
\def\HoLogoBkm@AmS#1{AmS}
%    \end{macrocode}
%    \end{macro}
%    \begin{macro}{\HoLogoHtml@AmS}
%    \begin{macrocode}
\def\HoLogoHtml@AmS#1{%
  \HoLogoCss@AmS
%  \HoLogoFont@font{AmS}{sy}{%
    \HOLOGO@Span{AmS}{%
      A%
      \HOLOGO@Span{M}{M}%
      S%
    }%
%   }%
}
%    \end{macrocode}
%    \end{macro}
%    \begin{macro}{\HoLogoCss@AmS}
%    \begin{macrocode}
\def\HoLogoCss@AmS{%
  \Css{%
    span.HoLogo-AmS span.HoLogo-M{%
      position:relative;%
      top:.5ex;%
      margin-left:-.1667em;%
      margin-right:-.125em;%
      text-decoration:none;%
    }%
  }%
  \global\let\HoLogoCss@AmS\relax
}
%    \end{macrocode}
%    \end{macro}
%
%    \begin{macro}{\HoLogo@AmSTeX}
%    \begin{macrocode}
\def\HoLogo@AmSTeX#1{%
  \hologo{AmS}%
  \HOLOGO@hyphen
  \hologo{TeX}%
}
%    \end{macrocode}
%    \end{macro}
%    \begin{macro}{\HoLogoBkm@AmSTeX}
%    \begin{macrocode}
\def\HoLogoBkm@AmSTeX#1{AmS-TeX}%
%    \end{macrocode}
%    \end{macro}
%    \begin{macro}{\HoLogoHtml@AmSTeX}
%    \begin{macrocode}
\let\HoLogoHtml@AmSTeX\HoLogo@AmSTeX
%    \end{macrocode}
%    \end{macro}
%
%    \begin{macro}{\HoLogo@AmSLaTeX}
%    \begin{macrocode}
\def\HoLogo@AmSLaTeX#1{%
  \hologo{AmS}%
  \HOLOGO@hyphen
  \hologo{LaTeX}%
}
%    \end{macrocode}
%    \end{macro}
%    \begin{macro}{\HoLogoBkm@AmSLaTeX}
%    \begin{macrocode}
\def\HoLogoBkm@AmSLaTeX#1{AmS-LaTeX}%
%    \end{macrocode}
%    \end{macro}
%    \begin{macro}{\HoLogoHtml@AmSLaTeX}
%    \begin{macrocode}
\let\HoLogoHtml@AmSLaTeX\HoLogo@AmSLaTeX
%    \end{macrocode}
%    \end{macro}
%
% \subsubsection{\hologo{BibTeX}}
%
%    \begin{macro}{\HoLogo@BibTeX@sc}
%    A definition of \hologo{BibTeX} is provided in
%    the documentation source for the manual of \hologo{BibTeX}
%    \cite{btxdoc}.
%\begin{quote}
%\begin{verbatim}
%\def\BibTeX{%
%  {%
%    \rm
%    B%
%    \kern-.05em%
%    {%
%      \sc
%      i%
%      \kern-.025em %
%      b%
%    }%
%    \kern-.08em
%    T%
%    \kern-.1667em%
%    \lower.7ex\hbox{E}%
%    \kern-.125em%
%    X%
%  }%
%}
%\end{verbatim}
%\end{quote}
%    \begin{macrocode}
\def\HoLogo@BibTeX@sc#1{%
  B%
  \kern-.05em%
  \HoLogoFont@font{BibTeX}{sc}{%
    i%
    \kern-.025em%
    b%
  }%
  \HOLOGO@discretionary
  \kern-.08em%
  \hologo{TeX}%
}
%    \end{macrocode}
%    \end{macro}
%    \begin{macro}{\HoLogoHtml@BibTeX@sc}
%    \begin{macrocode}
\def\HoLogoHtml@BibTeX@sc#1{%
  \HoLogoCss@BibTeX@sc
  \HOLOGO@Span{BibTeX-sc}{%
    B%
    \HOLOGO@Span{i}{i}%
    \HOLOGO@Span{b}{b}%
    \hologo{TeX}%
  }%
}
%    \end{macrocode}
%    \end{macro}
%    \begin{macro}{\HoLogoCss@BibTeX@sc}
%    \begin{macrocode}
\def\HoLogoCss@BibTeX@sc{%
  \Css{%
    span.HoLogo-BibTeX-sc span.HoLogo-i{%
      margin-left:-.05em;%
      margin-right:-.025em;%
      font-variant:small-caps;%
    }%
  }%
  \Css{%
    span.HoLogo-BibTeX-sc span.HoLogo-b{%
      margin-right:-.08em;%
      font-variant:small-caps;%
    }%
  }%
  \global\let\HoLogoCss@BibTeX@sc\relax
}
%    \end{macrocode}
%    \end{macro}
%
%    \begin{macro}{\HoLogo@BibTeX@sf}
%    Variant \xoption{sf} avoids trouble with unavailable
%    small caps fonts (e.g., bold versions of Computer Modern or
%    Latin Modern). The definition is taken from
%    package \xpackage{dtklogos} \cite{dtklogos}.
%\begin{quote}
%\begin{verbatim}
%\DeclareRobustCommand{\BibTeX}{%
%  B%
%  \kern-.05em%
%  \hbox{%
%    $\m@th$% %% force math size calculations
%    \csname S@\f@size\endcsname
%    \fontsize\sf@size\z@
%    \math@fontsfalse
%    \selectfont
%    I%
%    \kern-.025em%
%    B
%  }%
%  \kern-.08em%
%  \-%
%  \TeX
%}
%\end{verbatim}
%\end{quote}
%    \begin{macrocode}
\def\HoLogo@BibTeX@sf#1{%
  B%
  \kern-.05em%
  \HoLogoFont@font{BibTeX}{bibsf}{%
    I%
    \kern-.025em%
    B%
  }%
  \HOLOGO@discretionary
  \kern-.08em%
  \hologo{TeX}%
}
%    \end{macrocode}
%    \end{macro}
%    \begin{macro}{\HoLogoHtml@BibTeX@sf}
%    \begin{macrocode}
\def\HoLogoHtml@BibTeX@sf#1{%
  \HoLogoCss@BibTeX@sf
  \HOLOGO@Span{BibTeX-sf}{%
    B%
    \HoLogoFont@font{BibTeX}{bibsf}{%
      \HOLOGO@Span{i}{I}%
      B%
    }%
    \hologo{TeX}%
  }%
}
%    \end{macrocode}
%    \end{macro}
%    \begin{macro}{\HoLogoCss@BibTeX@sf}
%    \begin{macrocode}
\def\HoLogoCss@BibTeX@sf{%
  \Css{%
    span.HoLogo-BibTeX-sf span.HoLogo-i{%
      margin-left:-.05em;%
      margin-right:-.025em;%
    }%
  }%
  \Css{%
    span.HoLogo-BibTeX-sf span.HoLogo-TeX{%
      margin-left:-.08em;%
    }%
  }%
  \global\let\HoLogoCss@BibTeX@sf\relax
}
%    \end{macrocode}
%    \end{macro}
%
%    \begin{macro}{\HoLogo@BibTeX}
%    \begin{macrocode}
\def\HoLogo@BibTeX{\HoLogo@BibTeX@sf}
%    \end{macrocode}
%    \end{macro}
%    \begin{macro}{\HoLogoHtml@BibTeX}
%    \begin{macrocode}
\def\HoLogoHtml@BibTeX{\HoLogoHtml@BibTeX@sf}
%    \end{macrocode}
%    \end{macro}
%
% \subsubsection{\hologo{BibTeX8}}
%
%    \begin{macro}{\HoLogo@BibTeX8}
%    \begin{macrocode}
\expandafter\def\csname HoLogo@BibTeX8\endcsname#1{%
  \hologo{BibTeX}%
  8%
}
%    \end{macrocode}
%    \end{macro}
%
%    \begin{macro}{\HoLogoBkm@BibTeX8}
%    \begin{macrocode}
\expandafter\def\csname HoLogoBkm@BibTeX8\endcsname#1{%
  \hologo{BibTeX}%
  8%
}
%    \end{macrocode}
%    \end{macro}
%    \begin{macro}{\HoLogoHtml@BibTeX8}
%    \begin{macrocode}
\expandafter
\let\csname HoLogoHtml@BibTeX8\expandafter\endcsname
\csname HoLogo@BibTeX8\endcsname
%    \end{macrocode}
%    \end{macro}
%
% \subsubsection{\hologo{ConTeXt}}
%
%    \begin{macro}{\HoLogo@ConTeXt@simple}
%    \begin{macrocode}
\def\HoLogo@ConTeXt@simple#1{%
  \HOLOGO@mbox{Con}%
  \HOLOGO@discretionary
  \HOLOGO@mbox{\hologo{TeX}t}%
}
%    \end{macrocode}
%    \end{macro}
%    \begin{macro}{\HoLogoHtml@ConTeXt@simple}
%    \begin{macrocode}
\let\HoLogoHtml@ConTeXt@simple\HoLogo@ConTeXt@simple
%    \end{macrocode}
%    \end{macro}
%
%    \begin{macro}{\HoLogo@ConTeXt@narrow}
%    This definition of logo \hologo{ConTeXt} with variant \xoption{narrow}
%    comes from TUGboat's class \xclass{ltugboat} (version 2010/11/15 v2.8).
%    \begin{macrocode}
\def\HoLogo@ConTeXt@narrow#1{%
  \HOLOGO@mbox{C\kern-.0333emon}%
  \HOLOGO@discretionary
  \kern-.0667em%
  \HOLOGO@mbox{\hologo{TeX}\kern-.0333emt}%
}
%    \end{macrocode}
%    \end{macro}
%    \begin{macro}{\HoLogoHtml@ConTeXt@narrow}
%    \begin{macrocode}
\def\HoLogoHtml@ConTeXt@narrow#1{%
  \HoLogoCss@ConTeXt@narrow
  \HOLOGO@Span{ConTeXt-narrow}{%
    \HOLOGO@Span{C}{C}%
    on%
    \hologo{TeX}%
    t%
  }%
}
%    \end{macrocode}
%    \end{macro}
%    \begin{macro}{\HoLogoCss@ConTeXt@narrow}
%    \begin{macrocode}
\def\HoLogoCss@ConTeXt@narrow{%
  \Css{%
    span.HoLogo-ConTeXt-narrow span.HoLogo-C{%
      margin-left:-.0333em;%
    }%
  }%
  \Css{%
    span.HoLogo-ConTeXt-narrow span.HoLogo-TeX{%
      margin-left:-.0667em;%
      margin-right:-.0333em;%
    }%
  }%
  \global\let\HoLogoCss@ConTeXt@narrow\relax
}
%    \end{macrocode}
%    \end{macro}
%
%    \begin{macro}{\HoLogo@ConTeXt}
%    \begin{macrocode}
\def\HoLogo@ConTeXt{\HoLogo@ConTeXt@narrow}
%    \end{macrocode}
%    \end{macro}
%    \begin{macro}{\HoLogoHtml@ConTeXt}
%    \begin{macrocode}
\def\HoLogoHtml@ConTeXt{\HoLogoHtml@ConTeXt@narrow}
%    \end{macrocode}
%    \end{macro}
%
% \subsubsection{\hologo{emTeX}}
%
%    \begin{macro}{\HoLogo@emTeX}
%    \begin{macrocode}
\def\HoLogo@emTeX#1{%
  \HOLOGO@mbox{#1{e}{E}m}%
  \HOLOGO@discretionary
  \hologo{TeX}%
}
%    \end{macrocode}
%    \end{macro}
%    \begin{macro}{\HoLogoCs@emTeX}
%    \begin{macrocode}
\def\HoLogoCs@emTeX#1{#1{e}{E}mTeX}%
%    \end{macrocode}
%    \end{macro}
%    \begin{macro}{\HoLogoBkm@emTeX}
%    \begin{macrocode}
\def\HoLogoBkm@emTeX#1{%
  #1{e}{E}m\hologo{TeX}%
}
%    \end{macrocode}
%    \end{macro}
%    \begin{macro}{\HoLogoHtml@emTeX}
%    \begin{macrocode}
\let\HoLogoHtml@emTeX\HoLogo@emTeX
%    \end{macrocode}
%    \end{macro}
%
% \subsubsection{\hologo{ExTeX}}
%
%    \begin{macro}{\HoLogo@ExTeX}
%    The definition is taken from the FAQ of the
%    project \hologo{ExTeX}
%    \cite{ExTeX-FAQ}.
%\begin{quote}
%\begin{verbatim}
%\def\ExTeX{%
%  \textrm{% Logo always with serifs
%    \ensuremath{%
%      \textstyle
%      \varepsilon_{%
%        \kern-0.15em%
%        \mathcal{X}%
%      }%
%    }%
%    \kern-.15em%
%    \TeX
%  }%
%}
%\end{verbatim}
%\end{quote}
%    \begin{macrocode}
\def\HoLogo@ExTeX#1{%
  \HoLogoFont@font{ExTeX}{rm}{%
    \ltx@mbox{%
      \HOLOGO@MathSetup
      $%
        \textstyle
        \varepsilon_{%
          \kern-0.15em%
          \HoLogoFont@font{ExTeX}{sy}{X}%
        }%
      $%
    }%
    \HOLOGO@discretionary
    \kern-.15em%
    \hologo{TeX}%
  }%
}
%    \end{macrocode}
%    \end{macro}
%    \begin{macro}{\HoLogoHtml@ExTeX}
%    \begin{macrocode}
\def\HoLogoHtml@ExTeX#1{%
  \HoLogoCss@ExTeX
  \HoLogoFont@font{ExTeX}{rm}{%
    \HOLOGO@Span{ExTeX}{%
      \ltx@mbox{%
        \HOLOGO@MathSetup
        $\textstyle\varepsilon$%
        \HOLOGO@Span{X}{$\textstyle\chi$}%
        \hologo{TeX}%
      }%
    }%
  }%
}
%    \end{macrocode}
%    \end{macro}
%    \begin{macro}{\HoLogoBkm@ExTeX}
%    \begin{macrocode}
\def\HoLogoBkm@ExTeX#1{%
  \HOLOGO@PdfdocUnicode{#1{e}{E}x}{\textepsilon\textchi}%
  \hologo{TeX}%
}
%    \end{macrocode}
%    \end{macro}
%    \begin{macro}{\HoLogoCss@ExTeX}
%    \begin{macrocode}
\def\HoLogoCss@ExTeX{%
  \Css{%
    span.HoLogo-ExTeX{%
      font-family:serif;%
    }%
  }%
  \Css{%
    span.HoLogo-ExTeX span.HoLogo-TeX{%
      margin-left:-.15em;%
    }%
  }%
  \global\let\HoLogoCss@ExTeX\relax
}
%    \end{macrocode}
%    \end{macro}
%
% \subsubsection{\hologo{MiKTeX}}
%
%    \begin{macro}{\HoLogo@MiKTeX}
%    \begin{macrocode}
\def\HoLogo@MiKTeX#1{%
  \HOLOGO@mbox{MiK}%
  \HOLOGO@discretionary
  \hologo{TeX}%
}
%    \end{macrocode}
%    \end{macro}
%    \begin{macro}{\HoLogoHtml@MiKTeX}
%    \begin{macrocode}
\let\HoLogoHtml@MiKTeX\HoLogo@MiKTeX
%    \end{macrocode}
%    \end{macro}
%
% \subsubsection{\hologo{OzTeX} and friends}
%
%    Source: \hologo{OzTeX} FAQ \cite{OzTeX}:
%    \begin{quote}
%      |\def\OzTeX{O\kern-.03em z\kern-.15em\TeX}|\\
%      (There is no kerning in OzMF, OzMP and OzTtH.)
%    \end{quote}
%
%    \begin{macro}{\HoLogo@OzTeX}
%    \begin{macrocode}
\def\HoLogo@OzTeX#1{%
  O%
  \kern-.03em %
  z%
  \kern-.15em %
  \hologo{TeX}%
}
%    \end{macrocode}
%    \end{macro}
%    \begin{macro}{\HoLogoHtml@OzTeX}
%    \begin{macrocode}
\def\HoLogoHtml@OzTeX#1{%
  \HoLogoCss@OzTeX
  \HOLOGO@Span{OzTeX}{%
    O%
    \HOLOGO@Span{z}{z}%
    \hologo{TeX}%
  }%
}
%    \end{macrocode}
%    \end{macro}
%    \begin{macro}{\HoLogoCss@OzTeX}
%    \begin{macrocode}
\def\HoLogoCss@OzTeX{%
  \Css{%
    span.HoLogo-OzTeX span.HoLogo-z{%
      margin-left:-.03em;%
      margin-right:-.15em;%
    }%
  }%
  \global\let\HoLogoCss@OzTeX\relax
}
%    \end{macrocode}
%    \end{macro}
%
%    \begin{macro}{\HoLogo@OzMF}
%    \begin{macrocode}
\def\HoLogo@OzMF#1{%
  \HOLOGO@mbox{OzMF}%
}
%    \end{macrocode}
%    \end{macro}
%    \begin{macro}{\HoLogo@OzMP}
%    \begin{macrocode}
\def\HoLogo@OzMP#1{%
  \HOLOGO@mbox{OzMP}%
}
%    \end{macrocode}
%    \end{macro}
%    \begin{macro}{\HoLogo@OzTtH}
%    \begin{macrocode}
\def\HoLogo@OzTtH#1{%
  \HOLOGO@mbox{OzTtH}%
}
%    \end{macrocode}
%    \end{macro}
%
% \subsubsection{\hologo{PCTeX}}
%
%    \begin{macro}{\HoLogo@PCTeX}
%    \begin{macrocode}
\def\HoLogo@PCTeX#1{%
  \HOLOGO@mbox{PC}%
  \hologo{TeX}%
}
%    \end{macrocode}
%    \end{macro}
%    \begin{macro}{\HoLogoHtml@PCTeX}
%    \begin{macrocode}
\let\HoLogoHtml@PCTeX\HoLogo@PCTeX
%    \end{macrocode}
%    \end{macro}
%
% \subsubsection{\hologo{PiCTeX}}
%
%    The original definitions from \xfile{pictex.tex} \cite{PiCTeX}:
%\begin{quote}
%\begin{verbatim}
%\def\PiC{%
%  P%
%  \kern-.12em%
%  \lower.5ex\hbox{I}%
%  \kern-.075em%
%  C%
%}
%\def\PiCTeX{%
%  \PiC
%  \kern-.11em%
%  \TeX
%}
%\end{verbatim}
%\end{quote}
%
%    \begin{macro}{\HoLogo@PiC}
%    \begin{macrocode}
\def\HoLogo@PiC#1{%
  P%
  \kern-.12em%
  \lower.5ex\hbox{I}%
  \kern-.075em%
  C%
  \HOLOGO@SpaceFactor
}
%    \end{macrocode}
%    \end{macro}
%    \begin{macro}{\HoLogoHtml@PiC}
%    \begin{macrocode}
\def\HoLogoHtml@PiC#1{%
  \HoLogoCss@PiC
  \HOLOGO@Span{PiC}{%
    P%
    \HOLOGO@Span{i}{I}%
    C%
  }%
}
%    \end{macrocode}
%    \end{macro}
%    \begin{macro}{\HoLogoCss@PiC}
%    \begin{macrocode}
\def\HoLogoCss@PiC{%
  \Css{%
    span.HoLogo-PiC span.HoLogo-i{%
      position:relative;%
      top:.5ex;%
      margin-left:-.12em;%
      margin-right:-.075em;%
      text-decoration:none;%
    }%
  }%
  \global\let\HoLogoCss@PiC\relax
}
%    \end{macrocode}
%    \end{macro}
%
%    \begin{macro}{\HoLogo@PiCTeX}
%    \begin{macrocode}
\def\HoLogo@PiCTeX#1{%
  \hologo{PiC}%
  \HOLOGO@discretionary
  \kern-.11em%
  \hologo{TeX}%
}
%    \end{macrocode}
%    \end{macro}
%    \begin{macro}{\HoLogoHtml@PiCTeX}
%    \begin{macrocode}
\def\HoLogoHtml@PiCTeX#1{%
  \HoLogoCss@PiCTeX
  \HOLOGO@Span{PiCTeX}{%
    \hologo{PiC}%
    \hologo{TeX}%
  }%
}
%    \end{macrocode}
%    \end{macro}
%    \begin{macro}{\HoLogoCss@PiCTeX}
%    \begin{macrocode}
\def\HoLogoCss@PiCTeX{%
  \Css{%
    span.HoLogo-PiCTeX span.HoLogo-PiC{%
      margin-right:-.11em;%
    }%
  }%
  \global\let\HoLogoCss@PiCTeX\relax
}
%    \end{macrocode}
%    \end{macro}
%
% \subsubsection{\hologo{teTeX}}
%
%    \begin{macro}{\HoLogo@teTeX}
%    \begin{macrocode}
\def\HoLogo@teTeX#1{%
  \HOLOGO@mbox{#1{t}{T}e}%
  \HOLOGO@discretionary
  \hologo{TeX}%
}
%    \end{macrocode}
%    \end{macro}
%    \begin{macro}{\HoLogoCs@teTeX}
%    \begin{macrocode}
\def\HoLogoCs@teTeX#1{#1{t}{T}dfTeX}
%    \end{macrocode}
%    \end{macro}
%    \begin{macro}{\HoLogoBkm@teTeX}
%    \begin{macrocode}
\def\HoLogoBkm@teTeX#1{%
  #1{t}{T}e\hologo{TeX}%
}
%    \end{macrocode}
%    \end{macro}
%    \begin{macro}{\HoLogoHtml@teTeX}
%    \begin{macrocode}
\let\HoLogoHtml@teTeX\HoLogo@teTeX
%    \end{macrocode}
%    \end{macro}
%
% \subsubsection{\hologo{TeX4ht}}
%
%    \begin{macro}{\HoLogo@TeX4ht}
%    \begin{macrocode}
\expandafter\def\csname HoLogo@TeX4ht\endcsname#1{%
  \HOLOGO@mbox{\hologo{TeX}4ht}%
}
%    \end{macrocode}
%    \end{macro}
%    \begin{macro}{\HoLogoHtml@TeX4ht}
%    \begin{macrocode}
\expandafter
\let\csname HoLogoHtml@TeX4ht\expandafter\endcsname
\csname HoLogo@TeX4ht\endcsname
%    \end{macrocode}
%    \end{macro}
%
%
% \subsubsection{\hologo{SageTeX}}
%
%    \begin{macro}{\HoLogo@SageTeX}
%    \begin{macrocode}
\def\HoLogo@SageTeX#1{%
  \HOLOGO@mbox{Sage}%
  \HOLOGO@discretionary
  \HOLOGO@NegativeKerning{eT,oT,To}%
  \hologo{TeX}%
}
%    \end{macrocode}
%    \end{macro}
%    \begin{macro}{\HoLogoHtml@SageTeX}
%    \begin{macrocode}
\let\HoLogoHtml@SageTeX\HoLogo@SageTeX
%    \end{macrocode}
%    \end{macro}
%
% \subsection{\hologo{METAFONT} and friends}
%
%    \begin{macro}{\HoLogo@METAFONT}
%    \begin{macrocode}
\def\HoLogo@METAFONT#1{%
  \HoLogoFont@font{METAFONT}{logo}{%
    \HOLOGO@mbox{META}%
    \HOLOGO@discretionary
    \HOLOGO@mbox{FONT}%
  }%
}
%    \end{macrocode}
%    \end{macro}
%
%    \begin{macro}{\HoLogo@METAPOST}
%    \begin{macrocode}
\def\HoLogo@METAPOST#1{%
  \HoLogoFont@font{METAPOST}{logo}{%
    \HOLOGO@mbox{META}%
    \HOLOGO@discretionary
    \HOLOGO@mbox{POST}%
  }%
}
%    \end{macrocode}
%    \end{macro}
%
%    \begin{macro}{\HoLogo@MetaFun}
%    \begin{macrocode}
\def\HoLogo@MetaFun#1{%
  \HOLOGO@mbox{Meta}%
  \HOLOGO@discretionary
  \HOLOGO@mbox{Fun}%
}
%    \end{macrocode}
%    \end{macro}
%
%    \begin{macro}{\HoLogo@MetaPost}
%    \begin{macrocode}
\def\HoLogo@MetaPost#1{%
  \HOLOGO@mbox{Meta}%
  \HOLOGO@discretionary
  \HOLOGO@mbox{Post}%
}
%    \end{macrocode}
%    \end{macro}
%
% \subsection{Others}
%
% \subsubsection{\hologo{biber}}
%
%    \begin{macro}{\HoLogo@biber}
%    \begin{macrocode}
\def\HoLogo@biber#1{%
  \HOLOGO@mbox{#1{b}{B}i}%
  \HOLOGO@discretionary
  \HOLOGO@mbox{ber}%
}
%    \end{macrocode}
%    \end{macro}
%    \begin{macro}{\HoLogoCs@biber}
%    \begin{macrocode}
\def\HoLogoCs@biber#1{#1{b}{B}iber}
%    \end{macrocode}
%    \end{macro}
%    \begin{macro}{\HoLogoBkm@biber}
%    \begin{macrocode}
\def\HoLogoBkm@biber#1{%
  #1{b}{B}iber%
}
%    \end{macrocode}
%    \end{macro}
%    \begin{macro}{\HoLogoHtml@biber}
%    \begin{macrocode}
\let\HoLogoHtml@biber\HoLogo@biber
%    \end{macrocode}
%    \end{macro}
%
% \subsubsection{\hologo{KOMAScript}}
%
%    \begin{macro}{\HoLogo@KOMAScript}
%    The definition for \hologo{KOMAScript} is taken
%    from \hologo{KOMAScript} (\xfile{scrlogo.dtx}, reformatted) \cite{scrlogo}:
%\begin{quote}
%\begin{verbatim}
%\@ifundefined{KOMAScript}{%
%  \DeclareRobustCommand{\KOMAScript}{%
%    \textsf{%
%      K\kern.05em O\kern.05emM\kern.05em A%
%      \kern.1em-\kern.1em %
%      Script%
%    }%
%  }%
%}{}
%\end{verbatim}
%\end{quote}
%    \begin{macrocode}
\def\HoLogo@KOMAScript#1{%
  \HoLogoFont@font{KOMAScript}{sf}{%
    \HOLOGO@mbox{%
      K\kern.05em%
      O\kern.05em%
      M\kern.05em%
      A%
    }%
    \kern.1em%
    \HOLOGO@hyphen
    \kern.1em%
    \HOLOGO@mbox{Script}%
  }%
}
%    \end{macrocode}
%    \end{macro}
%    \begin{macro}{\HoLogoBkm@KOMAScript}
%    \begin{macrocode}
\def\HoLogoBkm@KOMAScript#1{%
  KOMA-Script%
}
%    \end{macrocode}
%    \end{macro}
%    \begin{macro}{\HoLogoHtml@KOMAScript}
%    \begin{macrocode}
\def\HoLogoHtml@KOMAScript#1{%
  \HoLogoCss@KOMAScript
  \HoLogoFont@font{KOMAScript}{sf}{%
    \HOLOGO@Span{KOMAScript}{%
      K%
      \HOLOGO@Span{O}{O}%
      M%
      \HOLOGO@Span{A}{A}%
      \HOLOGO@Span{hyphen}{-}%
      Script%
    }%
  }%
}
%    \end{macrocode}
%    \end{macro}
%    \begin{macro}{\HoLogoCss@KOMAScript}
%    \begin{macrocode}
\def\HoLogoCss@KOMAScript{%
  \Css{%
    span.HoLogo-KOMAScript{%
      font-family:sans-serif;%
    }%
  }%
  \Css{%
    span.HoLogo-KOMAScript span.HoLogo-O{%
      padding-left:.05em;%
      padding-right:.05em;%
    }%
  }%
  \Css{%
    span.HoLogo-KOMAScript span.HoLogo-A{%
      padding-left:.05em;%
    }%
  }%
  \Css{%
    span.HoLogo-KOMAScript span.HoLogo-hyphen{%
      padding-left:.1em;%
      padding-right:.1em;%
    }%
  }%
  \global\let\HoLogoCss@KOMAScript\relax
}
%    \end{macrocode}
%    \end{macro}
%
% \subsubsection{\hologo{LyX}}
%
%    \begin{macro}{\HoLogo@LyX}
%    The definition is taken from the documentation source files
%    of \hologo{LyX}, \xfile{Intro.lyx} \cite{LyX}:
%\begin{quote}
%\begin{verbatim}
%\def\LyX{%
%  \texorpdfstring{%
%    L\kern-.1667em\lower.25em\hbox{Y}\kern-.125emX\@%
%  }{%
%    LyX%
%  }%
%}
%\end{verbatim}
%\end{quote}
%    \begin{macrocode}
\def\HoLogo@LyX#1{%
  L%
  \kern-.1667em%
  \lower.25em\hbox{Y}%
  \kern-.125em%
  X%
  \HOLOGO@SpaceFactor
}
%    \end{macrocode}
%    \end{macro}
%    \begin{macro}{\HoLogoHtml@LyX}
%    \begin{macrocode}
\def\HoLogoHtml@LyX#1{%
  \HoLogoCss@LyX
  \HOLOGO@Span{LyX}{%
    L%
    \HOLOGO@Span{y}{Y}%
    X%
  }%
}
%    \end{macrocode}
%    \end{macro}
%    \begin{macro}{\HoLogoCss@LyX}
%    \begin{macrocode}
\def\HoLogoCss@LyX{%
  \Css{%
    span.HoLogo-LyX span.HoLogo-y{%
      position:relative;%
      top:.25em;%
      margin-left:-.1667em;%
      margin-right:-.125em;%
      text-decoration:none;%
    }%
  }%
  \global\let\HoLogoCss@LyX\relax
}
%    \end{macrocode}
%    \end{macro}
%
% \subsubsection{\hologo{NTS}}
%
%    \begin{macro}{\HoLogo@NTS}
%    Definition for \hologo{NTS} can be found in
%    package \xpackage{etex\textunderscore man} for the \hologo{eTeX} manual \cite{etexman}
%    and in package \xpackage{dtklogos} \cite{dtklogos}:
%\begin{quote}
%\begin{verbatim}
%\def\NTS{%
%  \leavevmode
%  \hbox{%
%    $%
%      \cal N%
%      \kern-0.35em%
%      \lower0.5ex\hbox{$\cal T$}%
%      \kern-0.2em%
%      S%
%    $%
%  }%
%}
%\end{verbatim}
%\end{quote}
%    \begin{macrocode}
\def\HoLogo@NTS#1{%
  \HoLogoFont@font{NTS}{sy}{%
    N\/%
    \kern-.35em%
    \lower.5ex\hbox{T\/}%
    \kern-.2em%
    S\/%
  }%
  \HOLOGO@SpaceFactor
}
%    \end{macrocode}
%    \end{macro}
%
% \subsubsection{\Hologo{TTH} (\hologo{TeX} to HTML translator)}
%
%    Source: \url{http://hutchinson.belmont.ma.us/tth/}
%    In the HTML source the second `T' is printed as subscript.
%\begin{quote}
%\begin{verbatim}
%T<sub>T</sub>H
%\end{verbatim}
%\end{quote}
%    \begin{macro}{\HoLogo@TTH}
%    \begin{macrocode}
\def\HoLogo@TTH#1{%
  \ltx@mbox{%
    T\HOLOGO@SubScript{T}H%
  }%
  \HOLOGO@SpaceFactor
}
%    \end{macrocode}
%    \end{macro}
%
%    \begin{macro}{\HoLogoHtml@TTH}
%    \begin{macrocode}
\def\HoLogoHtml@TTH#1{%
  T\HCode{<sub>}T\HCode{</sub>}H%
}
%    \end{macrocode}
%    \end{macro}
%
% \subsubsection{\Hologo{HanTheThanh}}
%
%    Partial source: Package \xpackage{dtklogos}.
%    The double accent is U+1EBF (latin small letter e with circumflex
%    and acute).
%    \begin{macro}{\HoLogo@HanTheThanh}
%    \begin{macrocode}
\def\HoLogo@HanTheThanh#1{%
  \ltx@mbox{H\`an}%
  \HOLOGO@space
  \ltx@mbox{%
    Th%
    \HOLOGO@IfCharExists{"1EBF}{%
      \char"1EBF\relax
    }{%
      \^e\hbox to 0pt{\hss\raise .5ex\hbox{\'{}}}%
    }%
  }%
  \HOLOGO@space
  \ltx@mbox{Th\`anh}%
}
%    \end{macrocode}
%    \end{macro}
%    \begin{macro}{\HoLogoBkm@HanTheThanh}
%    \begin{macrocode}
\def\HoLogoBkm@HanTheThanh#1{%
  H\`an %
  Th\HOLOGO@PdfdocUnicode{\^e}{\9036\277} %
  Th\`anh%
}
%    \end{macrocode}
%    \end{macro}
%    \begin{macro}{\HoLogoHtml@HanTheThanh}
%    \begin{macrocode}
\def\HoLogoHtml@HanTheThanh#1{%
  H\`an %
  Th\HCode{&\ltx@hashchar x1ebf;} %
  Th\`anh%
}
%    \end{macrocode}
%    \end{macro}
%
% \subsection{Driver detection}
%
%    \begin{macrocode}
\HOLOGO@IfExists\InputIfFileExists{%
  \InputIfFileExists{hologo.cfg}{}{}%
}{%
  \ltx@IfUndefined{pdf@filesize}{%
    \def\HOLOGO@InputIfExists{%
      \openin\HOLOGO@temp=hologo.cfg\relax
      \ifeof\HOLOGO@temp
        \closein\HOLOGO@temp
      \else
        \closein\HOLOGO@temp
        \begingroup
          \def\x{LaTeX2e}%
        \expandafter\endgroup
        \ifx\fmtname\x
          % \iffalse meta-comment
%
% File: hologo.dtx
% Version: 2016/05/12 v1.11
% Info: A logo collection with bookmark support
%
% Copyright (C) 2010-2012 by
%    Heiko Oberdiek <heiko.oberdiek at googlemail.com>
%
% This work may be distributed and/or modified under the
% conditions of the LaTeX Project Public License, either
% version 1.3c of this license or (at your option) any later
% version. This version of this license is in
%    http://www.latex-project.org/lppl/lppl-1-3c.txt
% and the latest version of this license is in
%    http://www.latex-project.org/lppl.txt
% and version 1.3 or later is part of all distributions of
% LaTeX version 2005/12/01 or later.
%
% This work has the LPPL maintenance status "maintained".
%
% This Current Maintainer of this work is Heiko Oberdiek.
%
% The Base Interpreter refers to any `TeX-Format',
% because some files are installed in TDS:tex/generic//.
%
% This work consists of the main source file hologo.dtx
% and the derived files
%    hologo.sty, hologo.pdf, hologo.ins, hologo.drv, hologo-example.tex,
%    hologo-test1.tex, hologo-test-spacefactor.tex,
%    hologo-test-list.tex.
%
% Distribution:
%    CTAN:macros/latex/contrib/oberdiek/hologo.dtx
%    CTAN:macros/latex/contrib/oberdiek/hologo.pdf
%
% Unpacking:
%    (a) If hologo.ins is present:
%           tex hologo.ins
%    (b) Without hologo.ins:
%           tex hologo.dtx
%    (c) If you insist on using LaTeX
%           latex \let\install=y\input{hologo.dtx}
%        (quote the arguments according to the demands of your shell)
%
% Documentation:
%    (a) If hologo.drv is present:
%           latex hologo.drv
%    (b) Without hologo.drv:
%           latex hologo.dtx; ...
%    The class ltxdoc loads the configuration file ltxdoc.cfg
%    if available. Here you can specify further options, e.g.
%    use A4 as paper format:
%       \PassOptionsToClass{a4paper}{article}
%
%    Programm calls to get the documentation (example):
%       pdflatex hologo.dtx
%       makeindex -s gind.ist hologo.idx
%       pdflatex hologo.dtx
%       makeindex -s gind.ist hologo.idx
%       pdflatex hologo.dtx
%
% Installation:
%    TDS:tex/generic/oberdiek/hologo.sty
%    TDS:doc/latex/oberdiek/hologo.pdf
%    TDS:doc/latex/oberdiek/example/hologo-example.tex
%    TDS:doc/latex/oberdiek/test/hologo-test1.tex
%    TDS:doc/latex/oberdiek/test/hologo-test-spacefactor.tex
%    TDS:doc/latex/oberdiek/test/hologo-test-list.tex
%    TDS:source/latex/oberdiek/hologo.dtx
%
%<*ignore>
\begingroup
  \catcode123=1 %
  \catcode125=2 %
  \def\x{LaTeX2e}%
\expandafter\endgroup
\ifcase 0\ifx\install y1\fi\expandafter
         \ifx\csname processbatchFile\endcsname\relax\else1\fi
         \ifx\fmtname\x\else 1\fi\relax
\else\csname fi\endcsname
%</ignore>
%<*install>
\input docstrip.tex
\Msg{************************************************************************}
\Msg{* Installation}
\Msg{* Package: hologo 2016/05/12 v1.11 A logo collection with bookmark support (HO)}
\Msg{************************************************************************}

\keepsilent
\askforoverwritefalse

\let\MetaPrefix\relax
\preamble

This is a generated file.

Project: hologo
Version: 2016/05/12 v1.11

Copyright (C) 2010-2012 by
   Heiko Oberdiek <heiko.oberdiek at googlemail.com>

This work may be distributed and/or modified under the
conditions of the LaTeX Project Public License, either
version 1.3c of this license or (at your option) any later
version. This version of this license is in
   http://www.latex-project.org/lppl/lppl-1-3c.txt
and the latest version of this license is in
   http://www.latex-project.org/lppl.txt
and version 1.3 or later is part of all distributions of
LaTeX version 2005/12/01 or later.

This work has the LPPL maintenance status "maintained".

This Current Maintainer of this work is Heiko Oberdiek.

The Base Interpreter refers to any `TeX-Format',
because some files are installed in TDS:tex/generic//.

This work consists of the main source file hologo.dtx
and the derived files
   hologo.sty, hologo.pdf, hologo.ins, hologo.drv, hologo-example.tex,
   hologo-test1.tex, hologo-test-spacefactor.tex,
   hologo-test-list.tex.

\endpreamble
\let\MetaPrefix\DoubleperCent

\generate{%
  \file{hologo.ins}{\from{hologo.dtx}{install}}%
  \file{hologo.drv}{\from{hologo.dtx}{driver}}%
  \usedir{tex/generic/oberdiek}%
  \file{hologo.sty}{\from{hologo.dtx}{package}}%
  \usedir{doc/latex/oberdiek/example}%
  \file{hologo-example.tex}{\from{hologo.dtx}{example}}%
  \usedir{doc/latex/oberdiek/test}%
  \file{hologo-test1.tex}{\from{hologo.dtx}{test1}}%
  \file{hologo-test-spacefactor.tex}{\from{hologo.dtx}{test-spacefactor}}%
  \file{hologo-test-list.tex}{\from{hologo.dtx}{test-list}}%
  \nopreamble
  \nopostamble
  \usedir{source/latex/oberdiek/catalogue}%
  \file{hologo.xml}{\from{hologo.dtx}{catalogue}}%
}

\catcode32=13\relax% active space
\let =\space%
\Msg{************************************************************************}
\Msg{*}
\Msg{* To finish the installation you have to move the following}
\Msg{* file into a directory searched by TeX:}
\Msg{*}
\Msg{*     hologo.sty}
\Msg{*}
\Msg{* To produce the documentation run the file `hologo.drv'}
\Msg{* through LaTeX.}
\Msg{*}
\Msg{* Happy TeXing!}
\Msg{*}
\Msg{************************************************************************}

\endbatchfile
%</install>
%<*ignore>
\fi
%</ignore>
%<*driver>
\NeedsTeXFormat{LaTeX2e}
\ProvidesFile{hologo.drv}%
  [2016/05/12 v1.11 A logo collection with bookmark support (HO)]%
\documentclass{ltxdoc}
\usepackage{holtxdoc}[2011/11/22]
\usepackage{hologo}[2016/05/12]
\usepackage{longtable}
\usepackage{array}
\usepackage{paralist}
%\usepackage[T1]{fontenc}
%\usepackage{lmodern}
\begin{document}
  \DocInput{hologo.dtx}%
\end{document}
%</driver>
% \fi
%
%
% \CharacterTable
%  {Upper-case    \A\B\C\D\E\F\G\H\I\J\K\L\M\N\O\P\Q\R\S\T\U\V\W\X\Y\Z
%   Lower-case    \a\b\c\d\e\f\g\h\i\j\k\l\m\n\o\p\q\r\s\t\u\v\w\x\y\z
%   Digits        \0\1\2\3\4\5\6\7\8\9
%   Exclamation   \!     Double quote  \"     Hash (number) \#
%   Dollar        \$     Percent       \%     Ampersand     \&
%   Acute accent  \'     Left paren    \(     Right paren   \)
%   Asterisk      \*     Plus          \+     Comma         \,
%   Minus         \-     Point         \.     Solidus       \/
%   Colon         \:     Semicolon     \;     Less than     \<
%   Equals        \=     Greater than  \>     Question mark \?
%   Commercial at \@     Left bracket  \[     Backslash     \\
%   Right bracket \]     Circumflex    \^     Underscore    \_
%   Grave accent  \`     Left brace    \{     Vertical bar  \|
%   Right brace   \}     Tilde         \~}
%
% \GetFileInfo{hologo.drv}
%
% \title{The \xpackage{hologo} package}
% \date{2016/05/12 v1.11}
% \author{Heiko Oberdiek\\\xemail{heiko.oberdiek at googlemail.com}}
%
% \maketitle
%
% \begin{abstract}
% This package starts a collection of logos with support for bookmarks
% strings.
% \end{abstract}
%
% \tableofcontents
%
% \section{Documentation}
%
% \subsection{Logo macros}
%
% \begin{declcs}{hologo} \M{name}
% \end{declcs}
% Macro \cs{hologo} sets the logo with name \meta{name}.
% The following table shows the supported names.
%
% \begingroup
%   \def\hologoEntry#1#2#3{^^A
%     #1&#2&\hologoLogoSetup{#1}{variant=#2}\hologo{#1}&#3\tabularnewline
%   }
%   \begin{longtable}{>{\ttfamily}l>{\ttfamily}lll}
%     \rmfamily\bfseries{name} & \rmfamily\bfseries variant
%     & \bfseries logo & \bfseries since\\
%     \hline
%     \endhead
%     \hologoList
%   \end{longtable}
% \endgroup
%
% \begin{declcs}{Hologo} \M{name}
% \end{declcs}
% Macro \cs{Hologo} starts the logo \meta{name} with an uppercase
% letter. As an exception small greek letters are not converted
% to uppercase. Examples, see \hologo{eTeX} and \hologo{ExTeX}.
%
% \subsection{Setup macros}
%
% The package does not support package options, but the following
% setup macros can be used to set options.
%
% \begin{declcs}{hologoSetup} \M{key value list}
% \end{declcs}
% Macro \cs{hologoSetup} sets global options.
%
% \begin{declcs}{hologoLogoSetup} \M{logo} \M{key value list}
% \end{declcs}
% Some options can also be used to configure a logo.
% These settings take precedence over global option settings.
%
% \subsection{Options}\label{sec:options}
%
% There are boolean and string options:
% \begin{description}
% \item[Boolean option:]
% It takes |true| or |false|
% as value. If the value is omitted, then |true| is used.
% \item[String option:]
% A value must be given as string. (But the string might be empty.)
% \end{description}
% The following options can be used both in \cs{hologoSetup}
% and \cs{hologoLogoSetup}:
% \begin{description}
% \def\entry#1{\item[\xoption{#1}:]}
% \entry{break}
%   enables or disables line breaks inside the logo. This setting is
%   refined by options \xoption{hyphenbreak}, \xoption{spacebreak}
%   or \xoption{discretionarybreak}.
%   Default is |false|.
% \entry{hyphenbreak}
%   enables or disables the line break right after the hyphen character.
% \entry{spacebreak}
%   enables or disables line breaks at space characters.
% \entry{discretionarybreak}
%   enables or disables line breaks at hyphenation points
%   (inserted by \cs{-}).
% \end{description}
% Macro \cs{hologoLogoSetup} also knows:
% \begin{description}
% \item[\xoption{variant}:]
%   This is a string option. It specifies a variant of a logo that
%   must exist. An empty string selects the package default variant.
% \end{description}
% Example:
% \begin{quote}
%   |\hologoSetup{break=false}|\\
%   |\hologoLogoSetup{plainTeX}{variant=hyphen,hyphenbreak}|\\
%   Then ``plain-\TeX'' contains one break point after the hyphen.
% \end{quote}
%
% \subsection{Driver options}
%
% Sometimes graphical operations are needed to construct some
% glyphs (e.g.\ \hologo{XeTeX}). If package \xpackage{graphics}
% or package \xpackage{pgf} are found, then the macros are taken
% from there. Otherwise the packge defines its own operations
% and therefore needs the driver information. Many drivers are
% detected automatically (\hologo{pdfTeX}/\hologo{LuaTeX}
% in PDF mode, \hologo{XeTeX}, \hologo{VTeX}). These have precedence
% over a driver option. The driver can be given as package option
% or using \cs{hologoDriverSetup}.
% The following list contains the recognized driver options:
% \begin{itemize}
% \item \xoption{pdftex}, \xoption{luatex}
% \item \xoption{dvipdfm}, \xoption{dvipdfmx}
% \item \xoption{dvips}, \xoption{dvipsone}, \xoption{xdvi}
% \item \xoption{xetex}
% \item \xoption{vtex}
% \end{itemize}
% The left driver of a line is the driver name that is used internally.
% The following names are aliases for drivers that use the
% same method. Therefore the entry in the \xext{log} file for
% the used driver prints the internally used driver name.
% \begin{description}
% \item[\xoption{driverfallback}:]
%   This option expects a driver that is used,
%   if the driver could not be detected automatically.
% \end{description}
%
% \begin{declcs}{hologoDriverSetup} \M{driver option}
% \end{declcs}
% The driver can also be configured after package loading
% using \cs{hologoDriverSetup}, also the way for \hologo{plainTeX}
% to setup the driver.
%
% \subsection{Font setup}
%
% Some logos require a special font, but should also be usable by
% \hologo{plainTeX}. Therefore the package provides some ways
% to influence the font settings. The options below
% take font settings as values. Both font commands
% such as \cs{sffamily} and macros that take one argument
% like \cs{textsf} can be used.
%
% \begin{declcs}{hologoFontSetup} \M{key value list}
% \end{declcs}
% Macro \cs{hologoFontSetup} sets the fonts for all logos.
% Supported keys:
% \begin{description}
% \def\entry#1{\item[\xoption{#1}:]}
% \entry{general}
%   This font is used for all logos. The default is empty.
%   That means no special font is used.
% \entry{bibsf}
%   This font is used for
%   {\hologoLogoSetup{BibTeX}{variant=sf}\hologo{BibTeX}}
%   with variant \xoption{sf}.
% \entry{rm}
%   This font is a serif font. It is used for \hologo{ExTeX}.
% \entry{sc}
%   This font specifies a small caps font. It is used for
%   {\hologoLogoSetup{BibTeX}{variant=sc}\hologo{BibTeX}}
%   with variant \xoption{sc}.
% \entry{sf}
%   This font specifies a sans serif font. The default
%   is \cs{sffamily}, then \cs{sf} is tried. Otherwise
%   a warning is given. It is used by \hologo{KOMAScript}.
% \entry{sy}
%   This is the font for math symbols (e.g. cmsy).
%   It is used by \hologo{AmS}, \hologo{NTS}, \hologo{ExTeX}.
% \entry{logo}
%   \hologo{METAFONT} and \hologo{METAPOST} are using that font.
%   In \hologo{LaTeX} \cs{logofamily} is used and
%   the definitions of package \xpackage{mflogo} are used
%   if the package is not loaded.
%   Otherwise the \cs{tenlogo} is used and defined
%   if it does not already exists.
% \end{description}
%
% \begin{declcs}{hologoLogoFontSetup} \M{logo} \M{key value list}
% \end{declcs}
% Fonts can also be set for a logo or logo component separately,
% see the following list.
% The keys are the same as for \cs{hologoFontSetup}.
%
% \begin{longtable}{>{\ttfamily}l>{\sffamily}ll}
%   \meta{logo} & keys & result\\
%   \hline
%   \endhead
%   BibTeX & bibsf & {\hologoLogoSetup{BibTeX}{variant=sf}\hologo{BibTeX}}\\[.5ex]
%   BibTeX & sc & {\hologoLogoSetup{BibTeX}{variant=sc}\hologo{BibTeX}}\\[.5ex]
%   ExTeX & rm & \hologo{ExTeX}\\
%   SliTeX & rm & \hologo{SliTeX}\\[.5ex]
%   AmS & sy & \hologo{AmS}\\
%   ExTeX & sy & \hologo{ExTeX}\\
%   NTS & sy & \hologo{NTS}\\[.5ex]
%   KOMAScript & sf & \hologo{KOMAScript}\\[.5ex]
%   METAFONT & logo & \hologo{METAFONT}\\
%   METAPOST & logo & \hologo{METAPOST}\\[.5ex]
%   SliTeX & sc \hologo{SliTeX}
% \end{longtable}
%
% \subsubsection{Font order}
%
% For all logos the font \xoption{general} is applied first.
% Example:
%\begin{quote}
%|\hologoFontSetup{general=\color{red}}|
%\end{quote}
% will print red logos.
% Then if the font uses a special font \xoption{sf}, for example,
% the font is applied that is setup by \cs{hologoLogoFontSetup}.
% If this font is not setup, then the common font setup
% by \cs{hologoFontSetup} is used. Otherwise a warning is given,
% that there is no font configured.
%
% \subsection{Additional user macros}
%
% Usually a variant of a logo is configured by using
% \cs{hologoLogoSetup}, because it is bad style to mix
% different variants of the same logo in the same text.
% There the following macros are a convenience for testing.
%
% \begin{declcs}{hologoVariant} \M{name} \M{variant}\\
%   \cs{HologoVariant} \M{name} \M{variant}
% \end{declcs}
% Logo \meta{name} is set using \meta{variant} that specifies
% explicitely which variant of the macro is used. If the argument
% is empty, then the default form of the logo is used
% (configurable by \cs{hologoLogoSetup}).
%
% \cs{HologoVariant} is used if the logo is set in a context
% that needs an uppercase first letter (beginning of a sentence, \dots).
%
% \begin{declcs}{hologoList}\\
%   \cs{hologoEntry} \M{logo} \M{variant} \M{since}
% \end{declcs}
% Macro \cs{hologoList} contains all logos that are provided
% by the package including variants. The list consists of calls
% of \cs{hologoEntry} with three arguments starting with the
% logo name \meta{logo} and its variant \meta{variant}. An empty
% variant means the current default. Argument \meta{since} specifies
% with version of the package \xpackage{hologo} is needed to get
% the logo. If the logo is fixed, then the date gets updated.
% Therefore the date \meta{since} is not exactly the date of
% the first introduction, but rather the date of the latest fix.
%
% Before \cs{hologoList} can be used, macro \cs{hologoEntry} needs
% a definition. The example file in section \ref{sec:example}
% shows applications of \cs{hologoList}.
%
% \subsection{Supported contexts}
%
% Macros \cs{hologo} and friends support special contexts:
% \begin{itemize}
% \item \hologo{LaTeX}'s protection mechanism.
% \item Bookmarks of package \xpackage{hyperref}.
% \item Package \xpackage{tex4ht}.
% \item The macros can be used inside \cs{csname} constructs,
%   if \cs{ifincsname} is available (\hologo{pdfTeX}, \hologo{XeTeX},
%   \hologo{LuaTeX}).
% \end{itemize}
%
% \subsection{Example}
% \label{sec:example}
%
% The following example prints the logos in different fonts.
%    \begin{macrocode}
%<*example>
%<<verbatim
\NeedsTeXFormat{LaTeX2e}
\documentclass[a4paper]{article}
\usepackage[
  hmargin=20mm,
  vmargin=20mm,
]{geometry}
\pagestyle{empty}
\usepackage{hologo}[2016/05/12]
\usepackage{longtable}
\usepackage{array}
\setlength{\extrarowheight}{2pt}
\usepackage[T1]{fontenc}
\usepackage{lmodern}
\usepackage{pdflscape}
\usepackage[
  pdfencoding=auto,
]{hyperref}
\hypersetup{
  pdfauthor={Heiko Oberdiek},
  pdftitle={Example for package `hologo'},
  pdfsubject={Logos with fonts lmr, lmss, qtm, qpl, qhv},
}
\usepackage{bookmark}

% Print the logo list on the console

\begingroup
  \typeout{}%
  \typeout{*** Begin of logo list ***}%
  \newcommand*{\hologoEntry}[3]{%
    \typeout{#1 \ifx\\#2\\\else(#2) \fi[#3]}%
  }%
  \hologoList
  \typeout{*** End of logo list ***}%
  \typeout{}%
\endgroup

\begin{document}
\begin{landscape}

  \section{Example file for package `hologo'}

  % Table for font names

  \begin{longtable}{>{\bfseries}ll}
    \textbf{font} & \textbf{Font name}\\
    \hline
    lmr & Latin Modern Roman\\
    lmss & Latin Modern Sans\\
    qtm & \TeX\ Gyre Termes\\
    qhv & \TeX\ Gyre Heros\\
    qpl & \TeX\ Gyre Pagella\\
  \end{longtable}

  % Logo list with logos in different fonts

  \begingroup
    \newcommand*{\SetVariant}[2]{%
      \ifx\\#2\\%
      \else
        \hologoLogoSetup{#1}{variant=#2}%
      \fi
    }%
    \newcommand*{\hologoEntry}[3]{%
      \SetVariant{#1}{#2}%
      \raisebox{1em}[0pt][0pt]{\hypertarget{#1@#2}{}}%
      \bookmark[%
        dest={#1@#2},%
      ]{%
        #1\ifx\\#2\\\else\space(#2)\fi: \Hologo{#1}, \hologo{#1} %
        [Unicode]%
      }%
      \hypersetup{unicode=false}%
      \bookmark[%
        dest={#1@#2},%
      ]{%
        #1\ifx\\#2\\\else\space(#2)\fi: \Hologo{#1}, \hologo{#1} %
        [PDFDocEncoding]%
      }%
      \texttt{#1}%
      &%
      \texttt{#2}%
      &%
      \Hologo{#1}%
      &%
      \SetVariant{#1}{#2}%
      \hologo{#1}%
      &%
      \SetVariant{#1}{#2}%
      \fontfamily{qtm}\selectfont
      \hologo{#1}%
      &%
      \SetVariant{#1}{#2}%
      \fontfamily{qpl}\selectfont
      \hologo{#1}%
      &%
      \SetVariant{#1}{#2}%
      \textsf{\hologo{#1}}%
      &%
      \SetVariant{#1}{#2}%
      \fontfamily{qhv}\selectfont
      \hologo{#1}%
      \tabularnewline
    }%
    \begin{longtable}{llllllll}%
      \textbf{\textit{logo}} & \textbf{\textit{variant}} &
      \texttt{\string\Hologo} &
      \textbf{lmr} & \textbf{qtm} & \textbf{qpl} &
      \textbf{lmss} & \textbf{qhv}
      \tabularnewline
      \hline
      \endhead
      \hologoList
    \end{longtable}%
  \endgroup

\end{landscape}
\end{document}
%verbatim
%</example>
%    \end{macrocode}
%
% \StopEventually{
% }
%
% \section{Implementation}
%    \begin{macrocode}
%<*package>
%    \end{macrocode}
%    Reload check, especially if the package is not used with \LaTeX.
%    \begin{macrocode}
\begingroup\catcode61\catcode48\catcode32=10\relax%
  \catcode13=5 % ^^M
  \endlinechar=13 %
  \catcode35=6 % #
  \catcode39=12 % '
  \catcode44=12 % ,
  \catcode45=12 % -
  \catcode46=12 % .
  \catcode58=12 % :
  \catcode64=11 % @
  \catcode123=1 % {
  \catcode125=2 % }
  \expandafter\let\expandafter\x\csname ver@hologo.sty\endcsname
  \ifx\x\relax % plain-TeX, first loading
  \else
    \def\empty{}%
    \ifx\x\empty % LaTeX, first loading,
      % variable is initialized, but \ProvidesPackage not yet seen
    \else
      \expandafter\ifx\csname PackageInfo\endcsname\relax
        \def\x#1#2{%
          \immediate\write-1{Package #1 Info: #2.}%
        }%
      \else
        \def\x#1#2{\PackageInfo{#1}{#2, stopped}}%
      \fi
      \x{hologo}{The package is already loaded}%
      \aftergroup\endinput
    \fi
  \fi
\endgroup%
%    \end{macrocode}
%    Package identification:
%    \begin{macrocode}
\begingroup\catcode61\catcode48\catcode32=10\relax%
  \catcode13=5 % ^^M
  \endlinechar=13 %
  \catcode35=6 % #
  \catcode39=12 % '
  \catcode40=12 % (
  \catcode41=12 % )
  \catcode44=12 % ,
  \catcode45=12 % -
  \catcode46=12 % .
  \catcode47=12 % /
  \catcode58=12 % :
  \catcode64=11 % @
  \catcode91=12 % [
  \catcode93=12 % ]
  \catcode123=1 % {
  \catcode125=2 % }
  \expandafter\ifx\csname ProvidesPackage\endcsname\relax
    \def\x#1#2#3[#4]{\endgroup
      \immediate\write-1{Package: #3 #4}%
      \xdef#1{#4}%
    }%
  \else
    \def\x#1#2[#3]{\endgroup
      #2[{#3}]%
      \ifx#1\@undefined
        \xdef#1{#3}%
      \fi
      \ifx#1\relax
        \xdef#1{#3}%
      \fi
    }%
  \fi
\expandafter\x\csname ver@hologo.sty\endcsname
\ProvidesPackage{hologo}%
  [2016/05/12 v1.11 A logo collection with bookmark support (HO)]%
%    \end{macrocode}
%
%    \begin{macrocode}
\begingroup\catcode61\catcode48\catcode32=10\relax%
  \catcode13=5 % ^^M
  \endlinechar=13 %
  \catcode123=1 % {
  \catcode125=2 % }
  \catcode64=11 % @
  \def\x{\endgroup
    \expandafter\edef\csname HOLOGO@AtEnd\endcsname{%
      \endlinechar=\the\endlinechar\relax
      \catcode13=\the\catcode13\relax
      \catcode32=\the\catcode32\relax
      \catcode35=\the\catcode35\relax
      \catcode61=\the\catcode61\relax
      \catcode64=\the\catcode64\relax
      \catcode123=\the\catcode123\relax
      \catcode125=\the\catcode125\relax
    }%
  }%
\x\catcode61\catcode48\catcode32=10\relax%
\catcode13=5 % ^^M
\endlinechar=13 %
\catcode35=6 % #
\catcode64=11 % @
\catcode123=1 % {
\catcode125=2 % }
\def\TMP@EnsureCode#1#2{%
  \edef\HOLOGO@AtEnd{%
    \HOLOGO@AtEnd
    \catcode#1=\the\catcode#1\relax
  }%
  \catcode#1=#2\relax
}
\TMP@EnsureCode{10}{12}% ^^J
\TMP@EnsureCode{33}{12}% !
\TMP@EnsureCode{34}{12}% "
\TMP@EnsureCode{36}{3}% $
\TMP@EnsureCode{38}{4}% &
\TMP@EnsureCode{39}{12}% '
\TMP@EnsureCode{40}{12}% (
\TMP@EnsureCode{41}{12}% )
\TMP@EnsureCode{42}{12}% *
\TMP@EnsureCode{43}{12}% +
\TMP@EnsureCode{44}{12}% ,
\TMP@EnsureCode{45}{12}% -
\TMP@EnsureCode{46}{12}% .
\TMP@EnsureCode{47}{12}% /
\TMP@EnsureCode{58}{12}% :
\TMP@EnsureCode{59}{12}% ;
\TMP@EnsureCode{60}{12}% <
\TMP@EnsureCode{62}{12}% >
\TMP@EnsureCode{63}{12}% ?
\TMP@EnsureCode{91}{12}% [
\TMP@EnsureCode{93}{12}% ]
\TMP@EnsureCode{94}{7}% ^ (superscript)
\TMP@EnsureCode{95}{8}% _ (subscript)
\TMP@EnsureCode{96}{12}% `
\TMP@EnsureCode{124}{12}% |
\edef\HOLOGO@AtEnd{%
  \HOLOGO@AtEnd
  \escapechar\the\escapechar\relax
  \noexpand\endinput
}
\escapechar=92 %
%    \end{macrocode}
%
% \subsection{Logo list}
%
%    \begin{macro}{\hologoList}
%    \begin{macrocode}
\def\hologoList{%
  \hologoEntry{(La)TeX}{}{2011/10/01}%
  \hologoEntry{AmSLaTeX}{}{2010/04/16}%
  \hologoEntry{AmSTeX}{}{2010/04/16}%
  \hologoEntry{biber}{}{2011/10/01}%
  \hologoEntry{BibTeX}{}{2011/10/01}%
  \hologoEntry{BibTeX}{sf}{2011/10/01}%
  \hologoEntry{BibTeX}{sc}{2011/10/01}%
  \hologoEntry{BibTeX8}{}{2011/11/22}%
  \hologoEntry{ConTeXt}{}{2011/03/25}%
  \hologoEntry{ConTeXt}{narrow}{2011/03/25}%
  \hologoEntry{ConTeXt}{simple}{2011/03/25}%
  \hologoEntry{emTeX}{}{2010/04/26}%
  \hologoEntry{eTeX}{}{2010/04/08}%
  \hologoEntry{ExTeX}{}{2011/10/01}%
  \hologoEntry{HanTheThanh}{}{2011/11/29}%
  \hologoEntry{iniTeX}{}{2011/10/01}%
  \hologoEntry{KOMAScript}{}{2011/10/01}%
  \hologoEntry{La}{}{2010/05/08}%
  \hologoEntry{LaTeX}{}{2010/04/08}%
  \hologoEntry{LaTeX2e}{}{2010/04/08}%
  \hologoEntry{LaTeX3}{}{2010/04/24}%
  \hologoEntry{LaTeXe}{}{2010/04/08}%
  \hologoEntry{LaTeXML}{}{2011/11/22}%
  \hologoEntry{LaTeXTeX}{}{2011/10/01}%
  \hologoEntry{LuaLaTeX}{}{2010/04/08}%
  \hologoEntry{LuaTeX}{}{2010/04/08}%
  \hologoEntry{LyX}{}{2011/10/01}%
  \hologoEntry{METAFONT}{}{2011/10/01}%
  \hologoEntry{MetaFun}{}{2011/10/01}%
  \hologoEntry{METAPOST}{}{2011/10/01}%
  \hologoEntry{MetaPost}{}{2011/10/01}%
  \hologoEntry{MiKTeX}{}{2011/10/01}%
  \hologoEntry{NTS}{}{2011/10/01}%
  \hologoEntry{OzMF}{}{2011/10/01}%
  \hologoEntry{OzMP}{}{2011/10/01}%
  \hologoEntry{OzTeX}{}{2011/10/01}%
  \hologoEntry{OzTtH}{}{2011/10/01}%
  \hologoEntry{PCTeX}{}{2011/10/01}%
  \hologoEntry{pdfTeX}{}{2011/10/01}%
  \hologoEntry{pdfLaTeX}{}{2011/10/01}%
  \hologoEntry{PiC}{}{2011/10/01}%
  \hologoEntry{PiCTeX}{}{2011/10/01}%
  \hologoEntry{plainTeX}{}{2010/04/08}%
  \hologoEntry{plainTeX}{space}{2010/04/16}%
  \hologoEntry{plainTeX}{hyphen}{2010/04/16}%
  \hologoEntry{plainTeX}{runtogether}{2010/04/16}%
  \hologoEntry{SageTeX}{}{2011/11/22}%
  \hologoEntry{SLiTeX}{}{2011/10/01}%
  \hologoEntry{SLiTeX}{lift}{2011/10/01}%
  \hologoEntry{SLiTeX}{narrow}{2011/10/01}%
  \hologoEntry{SLiTeX}{simple}{2011/10/01}%
  \hologoEntry{SliTeX}{}{2011/10/01}%
  \hologoEntry{SliTeX}{narrow}{2011/10/01}%
  \hologoEntry{SliTeX}{simple}{2011/10/01}%
  \hologoEntry{SliTeX}{lift}{2011/10/01}%
  \hologoEntry{teTeX}{}{2011/10/01}%
  \hologoEntry{TeX}{}{2010/04/08}%
  \hologoEntry{TeX4ht}{}{2011/11/22}%
  \hologoEntry{TTH}{}{2011/11/22}%
  \hologoEntry{virTeX}{}{2011/10/01}%
  \hologoEntry{VTeX}{}{2010/04/24}%
  \hologoEntry{Xe}{}{2010/04/08}%
  \hologoEntry{XeLaTeX}{}{2010/04/08}%
  \hologoEntry{XeTeX}{}{2010/04/08}%
}
%    \end{macrocode}
%    \end{macro}
%
% \subsection{Load resources}
%
%    \begin{macrocode}
\begingroup\expandafter\expandafter\expandafter\endgroup
\expandafter\ifx\csname RequirePackage\endcsname\relax
  \def\TMP@RequirePackage#1[#2]{%
    \begingroup\expandafter\expandafter\expandafter\endgroup
    \expandafter\ifx\csname ver@#1.sty\endcsname\relax
      \input #1.sty\relax
    \fi
  }%
  \TMP@RequirePackage{ltxcmds}[2011/02/04]%
  \TMP@RequirePackage{infwarerr}[2010/04/08]%
  \TMP@RequirePackage{kvsetkeys}[2010/03/01]%
  \TMP@RequirePackage{kvdefinekeys}[2010/03/01]%
  \TMP@RequirePackage{pdftexcmds}[2010/04/01]%
  \TMP@RequirePackage{ifpdf}[2010/01/28]%
  \TMP@RequirePackage{ifluatex}[2010/03/01]%
  \ltx@IfUndefined{newif}{%
    \expandafter\let\csname newif\endcsname\ltx@newif
  }{}%
  \TMP@RequirePackage{ifxetex}[2009/01/23]%
  \TMP@RequirePackage{ifvtex}[2010/03/01]%
\else
  \RequirePackage{ltxcmds}[2011/02/04]%
  \RequirePackage{infwarerr}[2010/04/08]%
  \RequirePackage{kvsetkeys}[2010/03/01]%
  \RequirePackage{kvdefinekeys}[2010/03/01]%
  \RequirePackage{pdftexcmds}[2010/04/01]%
  \RequirePackage{ifpdf}[2010/01/28]%
  \RequirePackage{ifluatex}[2010/03/01]%
  \RequirePackage{ifxetex}[2009/01/23]%
  \RequirePackage{ifvtex}[2010/03/01]%
\fi
%    \end{macrocode}
%
%    \begin{macro}{\HOLOGO@IfDefined}
%    \begin{macrocode}
\def\HOLOGO@IfExists#1{%
  \ifx\@undefined#1%
    \expandafter\ltx@secondoftwo
  \else
    \ifx\relax#1%
      \expandafter\ltx@secondoftwo
    \else
      \expandafter\expandafter\expandafter\ltx@firstoftwo
    \fi
  \fi
}
%    \end{macrocode}
%    \end{macro}
%
% \subsection{Setup macros}
%
%    \begin{macro}{\hologoSetup}
%    \begin{macrocode}
\def\hologoSetup{%
  \let\HOLOGO@name\relax
  \HOLOGO@Setup
}
%    \end{macrocode}
%    \end{macro}
%
%    \begin{macro}{\hologoLogoSetup}
%    \begin{macrocode}
\def\hologoLogoSetup#1{%
  \edef\HOLOGO@name{#1}%
  \ltx@IfUndefined{HoLogo@\HOLOGO@name}{%
    \@PackageError{hologo}{%
      Unknown logo `\HOLOGO@name'%
    }\@ehc
    \ltx@gobble
  }{%
    \HOLOGO@Setup
  }%
}
%    \end{macrocode}
%    \end{macro}
%
%    \begin{macro}{\HOLOGO@Setup}
%    \begin{macrocode}
\def\HOLOGO@Setup{%
  \kvsetkeys{HoLogo}%
}
%    \end{macrocode}
%    \end{macro}
%
% \subsection{Options}
%
%    \begin{macro}{\HOLOGO@DeclareBoolOption}
%    \begin{macrocode}
\def\HOLOGO@DeclareBoolOption#1{%
  \expandafter\chardef\csname HOLOGOOPT@#1\endcsname\ltx@zero
  \kv@define@key{HoLogo}{#1}[true]{%
    \def\HOLOGO@temp{##1}%
    \ifx\HOLOGO@temp\HOLOGO@true
      \ifx\HOLOGO@name\relax
        \expandafter\chardef\csname HOLOGOOPT@#1\endcsname=\ltx@one
      \else
        \expandafter\chardef\csname
        HoLogoOpt@#1@\HOLOGO@name\endcsname\ltx@one
      \fi
      \HOLOGO@SetBreakAll{#1}%
    \else
      \ifx\HOLOGO@temp\HOLOGO@false
        \ifx\HOLOGO@name\relax
          \expandafter\chardef\csname HOLOGOOPT@#1\endcsname=\ltx@zero
        \else
          \expandafter\chardef\csname
          HoLogoOpt@#1@\HOLOGO@name\endcsname=\ltx@zero
        \fi
        \HOLOGO@SetBreakAll{#1}%
      \else
        \@PackageError{hologo}{%
          Unknown value `##1' for boolean option `#1'.\MessageBreak
          Known values are `true' and `false'%
        }\@ehc
      \fi
    \fi
  }%
}
%    \end{macrocode}
%    \end{macro}
%
%    \begin{macro}{\HOLOGO@SetBreakAll}
%    \begin{macrocode}
\def\HOLOGO@SetBreakAll#1{%
  \def\HOLOGO@temp{#1}%
  \ifx\HOLOGO@temp\HOLOGO@break
    \ifx\HOLOGO@name\relax
      \chardef\HOLOGOOPT@hyphenbreak=\HOLOGOOPT@break
      \chardef\HOLOGOOPT@spacebreak=\HOLOGOOPT@break
      \chardef\HOLOGOOPT@discretionarybreak=\HOLOGOOPT@break
    \else
      \expandafter\chardef
         \csname HoLogoOpt@hyphenbreak@\HOLOGO@name\endcsname=%
         \csname HoLogoOpt@break@\HOLOGO@name\endcsname
      \expandafter\chardef
         \csname HoLogoOpt@spacebreak@\HOLOGO@name\endcsname=%
         \csname HoLogoOpt@break@\HOLOGO@name\endcsname
      \expandafter\chardef
         \csname HoLogoOpt@discretionarybreak@\HOLOGO@name
             \endcsname=%
         \csname HoLogoOpt@break@\HOLOGO@name\endcsname
    \fi
  \fi
}
%    \end{macrocode}
%    \end{macro}
%
%    \begin{macro}{\HOLOGO@true}
%    \begin{macrocode}
\def\HOLOGO@true{true}
%    \end{macrocode}
%    \end{macro}
%    \begin{macro}{\HOLOGO@false}
%    \begin{macrocode}
\def\HOLOGO@false{false}
%    \end{macrocode}
%    \end{macro}
%    \begin{macro}{\HOLOGO@break}
%    \begin{macrocode}
\def\HOLOGO@break{break}
%    \end{macrocode}
%    \end{macro}
%
%    \begin{macrocode}
\HOLOGO@DeclareBoolOption{break}
\HOLOGO@DeclareBoolOption{hyphenbreak}
\HOLOGO@DeclareBoolOption{spacebreak}
\HOLOGO@DeclareBoolOption{discretionarybreak}
%    \end{macrocode}
%
%    \begin{macrocode}
\kv@define@key{HoLogo}{variant}{%
  \ifx\HOLOGO@name\relax
    \@PackageError{hologo}{%
      Option `variant' is not available in \string\hologoSetup,%
      \MessageBreak
      Use \string\hologoLogoSetup\space instead%
    }\@ehc
  \else
    \edef\HOLOGO@temp{#1}%
    \ifx\HOLOGO@temp\ltx@empty
      \expandafter
      \let\csname HoLogoOpt@variant@\HOLOGO@name\endcsname\@undefined
    \else
      \ltx@IfUndefined{HoLogo@\HOLOGO@name @\HOLOGO@temp}{%
        \@PackageError{hologo}{%
          Unknown variant `\HOLOGO@temp' of logo `\HOLOGO@name'%
        }\@ehc
      }{%
        \expandafter
        \let\csname HoLogoOpt@variant@\HOLOGO@name\endcsname
            \HOLOGO@temp
      }%
    \fi
  \fi
}
%    \end{macrocode}
%
%    \begin{macro}{\HOLOGO@Variant}
%    \begin{macrocode}
\def\HOLOGO@Variant#1{%
  #1%
  \ltx@ifundefined{HoLogoOpt@variant@#1}{%
  }{%
    @\csname HoLogoOpt@variant@#1\endcsname
  }%
}
%    \end{macrocode}
%    \end{macro}
%
% \subsection{Break/no-break support}
%
%    \begin{macro}{\HOLOGO@space}
%    \begin{macrocode}
\def\HOLOGO@space{%
  \ltx@ifundefined{HoLogoOpt@spacebreak@\HOLOGO@name}{%
    \ltx@ifundefined{HoLogoOpt@break@\HOLOGO@name}{%
      \chardef\HOLOGO@temp=\HOLOGOOPT@spacebreak
    }{%
      \chardef\HOLOGO@temp=%
        \csname HoLogoOpt@break@\HOLOGO@name\endcsname
    }%
  }{%
    \chardef\HOLOGO@temp=%
      \csname HoLogoOpt@spacebreak@\HOLOGO@name\endcsname
  }%
  \ifcase\HOLOGO@temp
    \penalty10000 %
  \fi
  \ltx@space
}
%    \end{macrocode}
%    \end{macro}
%
%    \begin{macro}{\HOLOGO@hyphen}
%    \begin{macrocode}
\def\HOLOGO@hyphen{%
  \ltx@ifundefined{HoLogoOpt@hyphenbreak@\HOLOGO@name}{%
    \ltx@ifundefined{HoLogoOpt@break@\HOLOGO@name}{%
      \chardef\HOLOGO@temp=\HOLOGOOPT@hyphenbreak
    }{%
      \chardef\HOLOGO@temp=%
        \csname HoLogoOpt@break@\HOLOGO@name\endcsname
    }%
  }{%
    \chardef\HOLOGO@temp=%
      \csname HoLogoOpt@hyphenbreak@\HOLOGO@name\endcsname
  }%
  \ifcase\HOLOGO@temp
    \ltx@mbox{-}%
  \else
    -%
  \fi
}
%    \end{macrocode}
%    \end{macro}
%
%    \begin{macro}{\HOLOGO@discretionary}
%    \begin{macrocode}
\def\HOLOGO@discretionary{%
  \ltx@ifundefined{HoLogoOpt@discretionarybreak@\HOLOGO@name}{%
    \ltx@ifundefined{HoLogoOpt@break@\HOLOGO@name}{%
      \chardef\HOLOGO@temp=\HOLOGOOPT@discretionarybreak
    }{%
      \chardef\HOLOGO@temp=%
        \csname HoLogoOpt@break@\HOLOGO@name\endcsname
    }%
  }{%
    \chardef\HOLOGO@temp=%
      \csname HoLogoOpt@discretionarybreak@\HOLOGO@name\endcsname
  }%
  \ifcase\HOLOGO@temp
  \else
    \-%
  \fi
}
%    \end{macrocode}
%    \end{macro}
%
%    \begin{macro}{\HOLOGO@mbox}
%    \begin{macrocode}
\def\HOLOGO@mbox#1{%
  \ltx@ifundefined{HoLogoOpt@break@\HOLOGO@name}{%
    \chardef\HOLOGO@temp=\HOLOGOOPT@hyphenbreak
  }{%
    \chardef\HOLOGO@temp=%
      \csname HoLogoOpt@break@\HOLOGO@name\endcsname
  }%
  \ifcase\HOLOGO@temp
    \ltx@mbox{#1}%
  \else
    #1%
  \fi
}
%    \end{macrocode}
%    \end{macro}
%
% \subsection{Font support}
%
%    \begin{macro}{\HoLogoFont@font}
%    \begin{tabular}{@{}ll@{}}
%    |#1|:& logo name\\
%    |#2|:& font short name\\
%    |#3|:& text
%    \end{tabular}
%    \begin{macrocode}
\def\HoLogoFont@font#1#2#3{%
  \begingroup
    \ltx@IfUndefined{HoLogoFont@logo@#1.#2}{%
      \ltx@IfUndefined{HoLogoFont@font@#2}{%
        \@PackageWarning{hologo}{%
          Missing font `#2' for logo `#1'%
        }%
        #3%
      }{%
        \csname HoLogoFont@font@#2\endcsname{#3}%
      }%
    }{%
      \csname HoLogoFont@logo@#1.#2\endcsname{#3}%
    }%
  \endgroup
}
%    \end{macrocode}
%    \end{macro}
%
%    \begin{macro}{\HoLogoFont@Def}
%    \begin{macrocode}
\def\HoLogoFont@Def#1{%
  \expandafter\def\csname HoLogoFont@font@#1\endcsname
}
%    \end{macrocode}
%    \end{macro}
%    \begin{macro}{\HoLogoFont@LogoDef}
%    \begin{macrocode}
\def\HoLogoFont@LogoDef#1#2{%
  \expandafter\def\csname HoLogoFont@logo@#1.#2\endcsname
}
%    \end{macrocode}
%    \end{macro}
%
% \subsubsection{Font defaults}
%
%    \begin{macro}{\HoLogoFont@font@general}
%    \begin{macrocode}
\HoLogoFont@Def{general}{}%
%    \end{macrocode}
%    \end{macro}
%
%    \begin{macro}{\HoLogoFont@font@rm}
%    \begin{macrocode}
\ltx@IfUndefined{rmfamily}{%
  \ltx@IfUndefined{rm}{%
  }{%
    \HoLogoFont@Def{rm}{\rm}%
  }%
}{%
  \HoLogoFont@Def{rm}{\rmfamily}%
}
%    \end{macrocode}
%    \end{macro}
%
%    \begin{macro}{\HoLogoFont@font@sf}
%    \begin{macrocode}
\ltx@IfUndefined{sffamily}{%
  \ltx@IfUndefined{sf}{%
  }{%
    \HoLogoFont@Def{sf}{\sf}%
  }%
}{%
  \HoLogoFont@Def{sf}{\sffamily}%
}
%    \end{macrocode}
%    \end{macro}
%
%    \begin{macro}{\HoLogoFont@font@bibsf}
%    In case of \hologo{plainTeX} the original small caps
%    variant is used as default. In \hologo{LaTeX}
%    the definition of package \xpackage{dtklogos} \cite{dtklogos}
%    is used.
%\begin{quote}
%\begin{verbatim}
%\DeclareRobustCommand{\BibTeX}{%
%  B%
%  \kern-.05em%
%  \hbox{%
%    $\m@th$% %% force math size calculations
%    \csname S@\f@size\endcsname
%    \fontsize\sf@size\z@
%    \math@fontsfalse
%    \selectfont
%    I%
%    \kern-.025em%
%    B
%  }%
%  \kern-.08em%
%  \-%
%  \TeX
%}
%\end{verbatim}
%\end{quote}
%    \begin{macrocode}
\ltx@IfUndefined{selectfont}{%
  \ltx@IfUndefined{tensc}{%
    \font\tensc=cmcsc10\relax
  }{}%
  \HoLogoFont@Def{bibsf}{\tensc}%
}{%
  \HoLogoFont@Def{bibsf}{%
    $\mathsurround=0pt$%
    \csname S@\f@size\endcsname
    \fontsize\sf@size{0pt}%
    \math@fontsfalse
    \selectfont
  }%
}
%    \end{macrocode}
%    \end{macro}
%
%    \begin{macro}{\HoLogoFont@font@sc}
%    \begin{macrocode}
\ltx@IfUndefined{scshape}{%
  \ltx@IfUndefined{tensc}{%
    \font\tensc=cmcsc10\relax
  }{}%
  \HoLogoFont@Def{sc}{\tensc}%
}{%
  \HoLogoFont@Def{sc}{\scshape}%
}
%    \end{macrocode}
%    \end{macro}
%
%    \begin{macro}{\HoLogoFont@font@sy}
%    \begin{macrocode}
\ltx@IfUndefined{usefont}{%
  \ltx@IfUndefined{tensy}{%
  }{%
    \HoLogoFont@Def{sy}{\tensy}%
  }%
}{%
  \HoLogoFont@Def{sy}{%
    \usefont{OMS}{cmsy}{m}{n}%
  }%
}
%    \end{macrocode}
%    \end{macro}
%
%    \begin{macro}{\HoLogoFont@font@logo}
%    \begin{macrocode}
\begingroup
  \def\x{LaTeX2e}%
\expandafter\endgroup
\ifx\fmtname\x
  \ltx@IfUndefined{logofamily}{%
    \DeclareRobustCommand\logofamily{%
      \not@math@alphabet\logofamily\relax
      \fontencoding{U}%
      \fontfamily{logo}%
      \selectfont
    }%
  }{}%
  \ltx@IfUndefined{logofamily}{%
  }{%
    \HoLogoFont@Def{logo}{\logofamily}%
  }%
\else
  \ltx@IfUndefined{tenlogo}{%
    \font\tenlogo=logo10\relax
  }{}%
  \HoLogoFont@Def{logo}{\tenlogo}%
\fi
%    \end{macrocode}
%    \end{macro}
%
% \subsubsection{Font setup}
%
%    \begin{macro}{\hologoFontSetup}
%    \begin{macrocode}
\def\hologoFontSetup{%
  \let\HOLOGO@name\relax
  \HOLOGO@FontSetup
}
%    \end{macrocode}
%    \end{macro}
%
%    \begin{macro}{\hologoLogoFontSetup}
%    \begin{macrocode}
\def\hologoLogoFontSetup#1{%
  \edef\HOLOGO@name{#1}%
  \ltx@IfUndefined{HoLogo@\HOLOGO@name}{%
    \@PackageError{hologo}{%
      Unknown logo `\HOLOGO@name'%
    }\@ehc
    \ltx@gobble
  }{%
    \HOLOGO@FontSetup
  }%
}
%    \end{macrocode}
%    \end{macro}
%
%    \begin{macro}{\HOLOGO@FontSetup}
%    \begin{macrocode}
\def\HOLOGO@FontSetup{%
  \kvsetkeys{HoLogoFont}%
}
%    \end{macrocode}
%    \end{macro}
%
%    \begin{macrocode}
\def\HOLOGO@temp#1{%
  \kv@define@key{HoLogoFont}{#1}{%
    \ifx\HOLOGO@name\relax
      \HoLogoFont@Def{#1}{##1}%
    \else
      \HoLogoFont@LogoDef\HOLOGO@name{#1}{##1}%
    \fi
  }%
}
\HOLOGO@temp{general}
\HOLOGO@temp{sf}
%    \end{macrocode}
%
% \subsection{Generic logo commands}
%
%    \begin{macrocode}
\HOLOGO@IfExists\hologo{%
  \@PackageError{hologo}{%
    \string\hologo\ltx@space is already defined.\MessageBreak
    Package loading is aborted%
  }\@ehc
  \HOLOGO@AtEnd
}%
\HOLOGO@IfExists\hologoRobust{%
  \@PackageError{hologo}{%
    \string\hologoRobust\ltx@space is already defined.\MessageBreak
    Package loading is aborted%
  }\@ehc
  \HOLOGO@AtEnd
}%
%    \end{macrocode}
%
% \subsubsection{\cs{hologo} and friends}
%
%    \begin{macrocode}
\ifluatex
  \expandafter\ltx@firstofone
\else
  \expandafter\ltx@gobble
\fi
{%
  \ltx@IfUndefined{ifincsname}{%
    \ifnum\luatexversion<36 %
      \expandafter\ltx@gobble
    \else
      \expandafter\ltx@firstofone
    \fi
    {%
      \begingroup
        \ifcase0%
            \directlua{%
              if tex.enableprimitives then %
                tex.enableprimitives('HOLOGO@', {'ifincsname'})%
              else %
                tex.print('1')%
              end%
            }%
            \ifx\HOLOGO@ifincsname\@undefined 1\fi%
            \relax
          \expandafter\ltx@firstofone
        \else
          \endgroup
          \expandafter\ltx@gobble
        \fi
        {%
          \global\let\ifincsname\HOLOGO@ifincsname
        }%
      \HOLOGO@temp
    }%
  }{}%
}
%    \end{macrocode}
%    \begin{macrocode}
\ltx@IfUndefined{ifincsname}{%
  \catcode`$=14 %
}{%
  \catcode`$=9 %
}
%    \end{macrocode}
%
%    \begin{macro}{\hologo}
%    \begin{macrocode}
\def\hologo#1{%
$ \ifincsname
$   \ltx@ifundefined{HoLogoCs@\HOLOGO@Variant{#1}}{%
$     #1%
$   }{%
$     \csname HoLogoCs@\HOLOGO@Variant{#1}\endcsname\ltx@firstoftwo
$   }%
$ \else
    \HOLOGO@IfExists\texorpdfstring\texorpdfstring\ltx@firstoftwo
    {%
      \hologoRobust{#1}%
    }{%
      \ltx@ifundefined{HoLogoBkm@\HOLOGO@Variant{#1}}{%
        \ltx@ifundefined{HoLogo@#1}{?#1?}{#1}%
      }{%
        \csname HoLogoBkm@\HOLOGO@Variant{#1}\endcsname
        \ltx@firstoftwo
      }%
    }%
$ \fi
}
%    \end{macrocode}
%    \end{macro}
%    \begin{macro}{\Hologo}
%    \begin{macrocode}
\def\Hologo#1{%
$ \ifincsname
$   \ltx@ifundefined{HoLogoCs@\HOLOGO@Variant{#1}}{%
$     #1%
$   }{%
$     \csname HoLogoCs@\HOLOGO@Variant{#1}\endcsname\ltx@secondoftwo
$   }%
$ \else
    \HOLOGO@IfExists\texorpdfstring\texorpdfstring\ltx@firstoftwo
    {%
      \HologoRobust{#1}%
    }{%
      \ltx@ifundefined{HoLogoBkm@\HOLOGO@Variant{#1}}{%
        \ltx@ifundefined{HoLogo@#1}{?#1?}{#1}%
      }{%
        \csname HoLogoBkm@\HOLOGO@Variant{#1}\endcsname
        \ltx@secondoftwo
      }%
    }%
$ \fi
}
%    \end{macrocode}
%    \end{macro}
%
%    \begin{macro}{\hologoVariant}
%    \begin{macrocode}
\def\hologoVariant#1#2{%
  \ifx\relax#2\relax
    \hologo{#1}%
  \else
$   \ifincsname
$     \ltx@ifundefined{HoLogoCs@#1@#2}{%
$       #1%
$     }{%
$       \csname HoLogoCs@#1@#2\endcsname\ltx@firstoftwo
$     }%
$   \else
      \HOLOGO@IfExists\texorpdfstring\texorpdfstring\ltx@firstoftwo
      {%
        \hologoVariantRobust{#1}{#2}%
      }{%
        \ltx@ifundefined{HoLogoBkm@#1@#2}{%
          \ltx@ifundefined{HoLogo@#1}{?#1?}{#1}%
        }{%
          \csname HoLogoBkm@#1@#2\endcsname
          \ltx@firstoftwo
        }%
      }%
$   \fi
  \fi
}
%    \end{macrocode}
%    \end{macro}
%    \begin{macro}{\HologoVariant}
%    \begin{macrocode}
\def\HologoVariant#1#2{%
  \ifx\relax#2\relax
    \Hologo{#1}%
  \else
$   \ifincsname
$     \ltx@ifundefined{HoLogoCs@#1@#2}{%
$       #1%
$     }{%
$       \csname HoLogoCs@#1@#2\endcsname\ltx@secondoftwo
$     }%
$   \else
      \HOLOGO@IfExists\texorpdfstring\texorpdfstring\ltx@firstoftwo
      {%
        \HologoVariantRobust{#1}{#2}%
      }{%
        \ltx@ifundefined{HoLogoBkm@#1@#2}{%
          \ltx@ifundefined{HoLogo@#1}{?#1?}{#1}%
        }{%
          \csname HoLogoBkm@#1@#2\endcsname
          \ltx@secondoftwo
        }%
      }%
$   \fi
  \fi
}
%    \end{macrocode}
%    \end{macro}
%
%    \begin{macrocode}
\catcode`\$=3 %
%    \end{macrocode}
%
% \subsubsection{\cs{hologoRobust} and friends}
%
%    \begin{macro}{\hologoRobust}
%    \begin{macrocode}
\ltx@IfUndefined{protected}{%
  \ltx@IfUndefined{DeclareRobustCommand}{%
    \def\hologoRobust#1%
  }{%
    \DeclareRobustCommand*\hologoRobust[1]%
  }%
}{%
  \protected\def\hologoRobust#1%
}%
{%
  \edef\HOLOGO@name{#1}%
  \ltx@IfUndefined{HoLogo@\HOLOGO@Variant\HOLOGO@name}{%
    \@PackageError{hologo}{%
      Unknown logo `\HOLOGO@name'%
    }\@ehc
    ?\HOLOGO@name?%
  }{%
    \ltx@IfUndefined{ver@tex4ht.sty}{%
      \HoLogoFont@font\HOLOGO@name{general}{%
        \csname HoLogo@\HOLOGO@Variant\HOLOGO@name\endcsname
        \ltx@firstoftwo
      }%
    }{%
      \ltx@IfUndefined{HoLogoHtml@\HOLOGO@Variant\HOLOGO@name}{%
        \HOLOGO@name
      }{%
        \csname HoLogoHtml@\HOLOGO@Variant\HOLOGO@name\endcsname
        \ltx@firstoftwo
      }%
    }%
  }%
}
%    \end{macrocode}
%    \end{macro}
%    \begin{macro}{\HologoRobust}
%    \begin{macrocode}
\ltx@IfUndefined{protected}{%
  \ltx@IfUndefined{DeclareRobustCommand}{%
    \def\HologoRobust#1%
  }{%
    \DeclareRobustCommand*\HologoRobust[1]%
  }%
}{%
  \protected\def\HologoRobust#1%
}%
{%
  \edef\HOLOGO@name{#1}%
  \ltx@IfUndefined{HoLogo@\HOLOGO@Variant\HOLOGO@name}{%
    \@PackageError{hologo}{%
      Unknown logo `\HOLOGO@name'%
    }\@ehc
    ?\HOLOGO@name?%
  }{%
    \ltx@IfUndefined{ver@tex4ht.sty}{%
      \HoLogoFont@font\HOLOGO@name{general}{%
        \csname HoLogo@\HOLOGO@Variant\HOLOGO@name\endcsname
        \ltx@secondoftwo
      }%
    }{%
      \ltx@IfUndefined{HoLogoHtml@\HOLOGO@Variant\HOLOGO@name}{%
        \expandafter\HOLOGO@Uppercase\HOLOGO@name
      }{%
        \csname HoLogoHtml@\HOLOGO@Variant\HOLOGO@name\endcsname
        \ltx@secondoftwo
      }%
    }%
  }%
}
%    \end{macrocode}
%    \end{macro}
%    \begin{macro}{\hologoVariantRobust}
%    \begin{macrocode}
\ltx@IfUndefined{protected}{%
  \ltx@IfUndefined{DeclareRobustCommand}{%
    \def\hologoVariantRobust#1#2%
  }{%
    \DeclareRobustCommand*\hologoVariantRobust[2]%
  }%
}{%
  \protected\def\hologoVariantRobust#1#2%
}%
{%
  \begingroup
    \hologoLogoSetup{#1}{variant={#2}}%
    \hologoRobust{#1}%
  \endgroup
}
%    \end{macrocode}
%    \end{macro}
%    \begin{macro}{\HologoVariantRobust}
%    \begin{macrocode}
\ltx@IfUndefined{protected}{%
  \ltx@IfUndefined{DeclareRobustCommand}{%
    \def\HologoVariantRobust#1#2%
  }{%
    \DeclareRobustCommand*\HologoVariantRobust[2]%
  }%
}{%
  \protected\def\HologoVariantRobust#1#2%
}%
{%
  \begingroup
    \hologoLogoSetup{#1}{variant={#2}}%
    \HologoRobust{#1}%
  \endgroup
}
%    \end{macrocode}
%    \end{macro}
%
%    \begin{macro}{\hologorobust}
%    Macro \cs{hologorobust} is only defined for compatibility.
%    Its use is deprecated.
%    \begin{macrocode}
\def\hologorobust{\hologoRobust}
%    \end{macrocode}
%    \end{macro}
%
% \subsection{Helpers}
%
%    \begin{macro}{\HOLOGO@Uppercase}
%    Macro \cs{HOLOGO@Uppercase} is restricted to \cs{uppercase},
%    because \hologo{plainTeX} or \hologo{iniTeX} do not provide
%    \cs{MakeUppercase}.
%    \begin{macrocode}
\def\HOLOGO@Uppercase#1{\uppercase{#1}}
%    \end{macrocode}
%    \end{macro}
%
%    \begin{macro}{\HOLOGO@PdfdocUnicode}
%    \begin{macrocode}
\def\HOLOGO@PdfdocUnicode{%
  \ifx\ifHy@unicode\iftrue
    \expandafter\ltx@secondoftwo
  \else
    \expandafter\ltx@firstoftwo
  \fi
}
%    \end{macrocode}
%    \end{macro}
%
%    \begin{macro}{\HOLOGO@Math}
%    \begin{macrocode}
\def\HOLOGO@MathSetup{%
  \mathsurround0pt\relax
  \HOLOGO@IfExists\f@series{%
    \if b\expandafter\ltx@car\f@series x\@nil
      \csname boldmath\endcsname
   \fi
  }{}%
}
%    \end{macrocode}
%    \end{macro}
%
%    \begin{macro}{\HOLOGO@TempDimen}
%    \begin{macrocode}
\dimendef\HOLOGO@TempDimen=\ltx@zero
%    \end{macrocode}
%    \end{macro}
%    \begin{macro}{\HOLOGO@NegativeKerning}
%    \begin{macrocode}
\def\HOLOGO@NegativeKerning#1{%
  \begingroup
    \HOLOGO@TempDimen=0pt\relax
    \comma@parse@normalized{#1}{%
      \ifdim\HOLOGO@TempDimen=0pt %
        \expandafter\HOLOGO@@NegativeKerning\comma@entry
      \fi
      \ltx@gobble
    }%
    \ifdim\HOLOGO@TempDimen<0pt %
      \kern\HOLOGO@TempDimen
    \fi
  \endgroup
}
%    \end{macrocode}
%    \end{macro}
%    \begin{macro}{\HOLOGO@@NegativeKerning}
%    \begin{macrocode}
\def\HOLOGO@@NegativeKerning#1#2{%
  \setbox\ltx@zero\hbox{#1#2}%
  \HOLOGO@TempDimen=\wd\ltx@zero
  \setbox\ltx@zero\hbox{#1\kern0pt#2}%
  \advance\HOLOGO@TempDimen by -\wd\ltx@zero
}
%    \end{macrocode}
%    \end{macro}
%
%    \begin{macro}{\HOLOGO@SpaceFactor}
%    \begin{macrocode}
\def\HOLOGO@SpaceFactor{%
  \spacefactor1000 %
}
%    \end{macrocode}
%    \end{macro}
%
%    \begin{macro}{\HOLOGO@Span}
%    \begin{macrocode}
\def\HOLOGO@Span#1#2{%
  \HCode{<span class="HoLogo-#1">}%
  #2%
  \HCode{</span>}%
}
%    \end{macrocode}
%    \end{macro}
%
% \subsubsection{Text subscript}
%
%    \begin{macro}{\HOLOGO@SubScript}%
%    \begin{macrocode}
\def\HOLOGO@SubScript#1{%
  \ltx@IfUndefined{textsubscript}{%
    \ltx@IfUndefined{text}{%
      \ltx@mbox{%
        \mathsurround=0pt\relax
        $%
          _{%
            \ltx@IfUndefined{sf@size}{%
              \mathrm{#1}%
            }{%
              \mbox{%
                \fontsize\sf@size{0pt}\selectfont
                #1%
              }%
            }%
          }%
        $%
      }%
    }{%
      \ltx@mbox{%
        \mathsurround=0pt\relax
        $_{\text{#1}}$%
      }%
    }%
  }{%
    \textsubscript{#1}%
  }%
}
%    \end{macrocode}
%    \end{macro}
%
% \subsection{\hologo{TeX} and friends}
%
% \subsubsection{\hologo{TeX}}
%
%    \begin{macro}{\HoLogo@TeX}
%    Source: \hologo{LaTeX} kernel.
%    \begin{macrocode}
\def\HoLogo@TeX#1{%
  T\kern-.1667em\lower.5ex\hbox{E}\kern-.125emX\HOLOGO@SpaceFactor
}
%    \end{macrocode}
%    \end{macro}
%    \begin{macro}{\HoLogoHtml@TeX}
%    \begin{macrocode}
\def\HoLogoHtml@TeX#1{%
  \HoLogoCss@TeX
  \HOLOGO@Span{TeX}{%
    T%
    \HOLOGO@Span{e}{%
      E%
    }%
    X%
  }%
}
%    \end{macrocode}
%    \end{macro}
%    \begin{macro}{\HoLogoCss@TeX}
%    \begin{macrocode}
\def\HoLogoCss@TeX{%
  \Css{%
    span.HoLogo-TeX span.HoLogo-e{%
      position:relative;%
      top:.5ex;%
      margin-left:-.1667em;%
      margin-right:-.125em;%
    }%
  }%
  \Css{%
    a span.HoLogo-TeX span.HoLogo-e{%
      text-decoration:none;%
    }%
  }%
  \global\let\HoLogoCss@TeX\relax
}
%    \end{macrocode}
%    \end{macro}
%
% \subsubsection{\hologo{plainTeX}}
%
%    \begin{macro}{\HoLogo@plainTeX@space}
%    Source: ``The \hologo{TeX}book''
%    \begin{macrocode}
\def\HoLogo@plainTeX@space#1{%
  \HOLOGO@mbox{#1{p}{P}lain}\HOLOGO@space\hologo{TeX}%
}
%    \end{macrocode}
%    \end{macro}
%    \begin{macro}{\HoLogoCs@plainTeX@space}
%    \begin{macrocode}
\def\HoLogoCs@plainTeX@space#1{#1{p}{P}lain TeX}%
%    \end{macrocode}
%    \end{macro}
%    \begin{macro}{\HoLogoBkm@plainTeX@space}
%    \begin{macrocode}
\def\HoLogoBkm@plainTeX@space#1{%
  #1{p}{P}lain \hologo{TeX}%
}
%    \end{macrocode}
%    \end{macro}
%    \begin{macro}{\HoLogoHtml@plainTeX@space}
%    \begin{macrocode}
\def\HoLogoHtml@plainTeX@space#1{%
  #1{p}{P}lain \hologo{TeX}%
}
%    \end{macrocode}
%    \end{macro}
%
%    \begin{macro}{\HoLogo@plainTeX@hyphen}
%    \begin{macrocode}
\def\HoLogo@plainTeX@hyphen#1{%
  \HOLOGO@mbox{#1{p}{P}lain}\HOLOGO@hyphen\hologo{TeX}%
}
%    \end{macrocode}
%    \end{macro}
%    \begin{macro}{\HoLogoCs@plainTeX@hyphen}
%    \begin{macrocode}
\def\HoLogoCs@plainTeX@hyphen#1{#1{p}{P}lain-TeX}
%    \end{macrocode}
%    \end{macro}
%    \begin{macro}{\HoLogoBkm@plainTeX@hyphen}
%    \begin{macrocode}
\def\HoLogoBkm@plainTeX@hyphen#1{%
  #1{p}{P}lain-\hologo{TeX}%
}
%    \end{macrocode}
%    \end{macro}
%    \begin{macro}{\HoLogoHtml@plainTeX@hyphen}
%    \begin{macrocode}
\def\HoLogoHtml@plainTeX@hyphen#1{%
  #1{p}{P}lain-\hologo{TeX}%
}
%    \end{macrocode}
%    \end{macro}
%
%    \begin{macro}{\HoLogo@plainTeX@runtogether}
%    \begin{macrocode}
\def\HoLogo@plainTeX@runtogether#1{%
  \HOLOGO@mbox{#1{p}{P}lain\hologo{TeX}}%
}
%    \end{macrocode}
%    \end{macro}
%    \begin{macro}{\HoLogoCs@plainTeX@runtogether}
%    \begin{macrocode}
\def\HoLogoCs@plainTeX@runtogether#1{#1{p}{P}lainTeX}
%    \end{macrocode}
%    \end{macro}
%    \begin{macro}{\HoLogoBkm@plainTeX@runtogether}
%    \begin{macrocode}
\def\HoLogoBkm@plainTeX@runtogether#1{%
  #1{p}{P}lain\hologo{TeX}%
}
%    \end{macrocode}
%    \end{macro}
%    \begin{macro}{\HoLogoHtml@plainTeX@runtogether}
%    \begin{macrocode}
\def\HoLogoHtml@plainTeX@runtogether#1{%
  #1{p}{P}lain\hologo{TeX}%
}
%    \end{macrocode}
%    \end{macro}
%
%    \begin{macro}{\HoLogo@plainTeX}
%    \begin{macrocode}
\def\HoLogo@plainTeX{\HoLogo@plainTeX@space}
%    \end{macrocode}
%    \end{macro}
%    \begin{macro}{\HoLogoCs@plainTeX}
%    \begin{macrocode}
\def\HoLogoCs@plainTeX{\HoLogoCs@plainTeX@space}
%    \end{macrocode}
%    \end{macro}
%    \begin{macro}{\HoLogoBkm@plainTeX}
%    \begin{macrocode}
\def\HoLogoBkm@plainTeX{\HoLogoBkm@plainTeX@space}
%    \end{macrocode}
%    \end{macro}
%    \begin{macro}{\HoLogoHtml@plainTeX}
%    \begin{macrocode}
\def\HoLogoHtml@plainTeX{\HoLogoHtml@plainTeX@space}
%    \end{macrocode}
%    \end{macro}
%
% \subsubsection{\hologo{LaTeX}}
%
%    Source: \hologo{LaTeX} kernel.
%\begin{quote}
%\begin{verbatim}
%\DeclareRobustCommand{\LaTeX}{%
%  L%
%  \kern-.36em%
%  {%
%    \sbox\z@ T%
%    \vbox to\ht\z@{%
%      \hbox{%
%        \check@mathfonts
%        \fontsize\sf@size\z@
%        \math@fontsfalse
%        \selectfont
%        A%
%      }%
%      \vss
%    }%
%  }%
%  \kern-.15em%
%  \TeX
%}
%\end{verbatim}
%\end{quote}
%
%    \begin{macro}{\HoLogo@La}
%    \begin{macrocode}
\def\HoLogo@La#1{%
  L%
  \kern-.36em%
  \begingroup
    \setbox\ltx@zero\hbox{T}%
    \vbox to\ht\ltx@zero{%
      \hbox{%
        \ltx@ifundefined{check@mathfonts}{%
          \csname sevenrm\endcsname
        }{%
          \check@mathfonts
          \fontsize\sf@size{0pt}%
          \math@fontsfalse\selectfont
        }%
        A%
      }%
      \vss
    }%
  \endgroup
}
%    \end{macrocode}
%    \end{macro}
%
%    \begin{macro}{\HoLogo@LaTeX}
%    Source: \hologo{LaTeX} kernel.
%    \begin{macrocode}
\def\HoLogo@LaTeX#1{%
  \hologo{La}%
  \kern-.15em%
  \hologo{TeX}%
}
%    \end{macrocode}
%    \end{macro}
%    \begin{macro}{\HoLogoHtml@LaTeX}
%    \begin{macrocode}
\def\HoLogoHtml@LaTeX#1{%
  \HoLogoCss@LaTeX
  \HOLOGO@Span{LaTeX}{%
    L%
    \HOLOGO@Span{a}{%
      A%
    }%
    \hologo{TeX}%
  }%
}
%    \end{macrocode}
%    \end{macro}
%    \begin{macro}{\HoLogoCss@LaTeX}
%    \begin{macrocode}
\def\HoLogoCss@LaTeX{%
  \Css{%
    span.HoLogo-LaTeX span.HoLogo-a{%
      position:relative;%
      top:-.5ex;%
      margin-left:-.36em;%
      margin-right:-.15em;%
      font-size:85\%;%
    }%
  }%
  \global\let\HoLogoCss@LaTeX\relax
}
%    \end{macrocode}
%    \end{macro}
%
% \subsubsection{\hologo{(La)TeX}}
%
%    \begin{macro}{\HoLogo@LaTeXTeX}
%    The kerning around the parentheses is taken
%    from package \xpackage{dtklogos} \cite{dtklogos}.
%\begin{quote}
%\begin{verbatim}
%\DeclareRobustCommand{\LaTeXTeX}{%
%  (%
%  \kern-.15em%
%  L%
%  \kern-.36em%
%  {%
%    \sbox\z@ T%
%    \vbox to\ht0{%
%      \hbox{%
%        $\m@th$%
%        \csname S@\f@size\endcsname
%        \fontsize\sf@size\z@
%        \math@fontsfalse
%        \selectfont
%        A%
%      }%
%      \vss
%    }%
%  }%
%  \kern-.2em%
%  )%
%  \kern-.15em%
%  \TeX
%}
%\end{verbatim}
%\end{quote}
%    \begin{macrocode}
\def\HoLogo@LaTeXTeX#1{%
  (%
  \kern-.15em%
  \hologo{La}%
  \kern-.2em%
  )%
  \kern-.15em%
  \hologo{TeX}%
}
%    \end{macrocode}
%    \end{macro}
%    \begin{macro}{\HoLogoBkm@LaTeXTeX}
%    \begin{macrocode}
\def\HoLogoBkm@LaTeXTeX#1{(La)TeX}
%    \end{macrocode}
%    \end{macro}
%
%    \begin{macro}{\HoLogo@(La)TeX}
%    \begin{macrocode}
\expandafter
\let\csname HoLogo@(La)TeX\endcsname\HoLogo@LaTeXTeX
%    \end{macrocode}
%    \end{macro}
%    \begin{macro}{\HoLogoBkm@(La)TeX}
%    \begin{macrocode}
\expandafter
\let\csname HoLogoBkm@(La)TeX\endcsname\HoLogoBkm@LaTeXTeX
%    \end{macrocode}
%    \end{macro}
%    \begin{macro}{\HoLogoHtml@LaTeXTeX}
%    \begin{macrocode}
\def\HoLogoHtml@LaTeXTeX#1{%
  \HoLogoCss@LaTeXTeX
  \HOLOGO@Span{LaTeXTeX}{%
    (%
    \HOLOGO@Span{L}{L}%
    \HOLOGO@Span{a}{A}%
    \HOLOGO@Span{ParenRight}{)}%
    \hologo{TeX}%
  }%
}
%    \end{macrocode}
%    \end{macro}
%    \begin{macro}{\HoLogoHtml@(La)TeX}
%    Kerning after opening parentheses and before closing parentheses
%    is $-0.1$\,em. The original values $-0.15$\,em
%    looked too ugly for a serif font.
%    \begin{macrocode}
\expandafter
\let\csname HoLogoHtml@(La)TeX\endcsname\HoLogoHtml@LaTeXTeX
%    \end{macrocode}
%    \end{macro}
%    \begin{macro}{\HoLogoCss@LaTeXTeX}
%    \begin{macrocode}
\def\HoLogoCss@LaTeXTeX{%
  \Css{%
    span.HoLogo-LaTeXTeX span.HoLogo-L{%
      margin-left:-.1em;%
    }%
  }%
  \Css{%
    span.HoLogo-LaTeXTeX span.HoLogo-a{%
      position:relative;%
      top:-.5ex;%
      margin-left:-.36em;%
      margin-right:-.1em;%
      font-size:85\%;%
    }%
  }%
  \Css{%
    span.HoLogo-LaTeXTeX span.HoLogo-ParenRight{%
      margin-right:-.15em;%
    }%
  }%
  \global\let\HoLogoCss@LaTeXTeX\relax
}
%    \end{macrocode}
%    \end{macro}
%
% \subsubsection{\hologo{LaTeXe}}
%
%    \begin{macro}{\HoLogo@LaTeXe}
%    Source: \hologo{LaTeX} kernel
%    \begin{macrocode}
\def\HoLogo@LaTeXe#1{%
  \hologo{LaTeX}%
  \kern.15em%
  \hbox{%
    \HOLOGO@MathSetup
    2%
    $_{\textstyle\varepsilon}$%
  }%
}
%    \end{macrocode}
%    \end{macro}
%
%    \begin{macro}{\HoLogoCs@LaTeXe}
%    \begin{macrocode}
\ifnum64=`\^^^^0040\relax % test for big chars of LuaTeX/XeTeX
  \catcode`\$=9 %
  \catcode`\&=14 %
\else
  \catcode`\$=14 %
  \catcode`\&=9 %
\fi
\def\HoLogoCs@LaTeXe#1{%
  LaTeX2%
$ \string ^^^^0395%
& e%
}%
\catcode`\$=3 %
\catcode`\&=4 %
%    \end{macrocode}
%    \end{macro}
%
%    \begin{macro}{\HoLogoBkm@LaTeXe}
%    \begin{macrocode}
\def\HoLogoBkm@LaTeXe#1{%
  \hologo{LaTeX}%
  2%
  \HOLOGO@PdfdocUnicode{e}{\textepsilon}%
}
%    \end{macrocode}
%    \end{macro}
%
%    \begin{macro}{\HoLogoHtml@LaTeXe}
%    \begin{macrocode}
\def\HoLogoHtml@LaTeXe#1{%
  \HoLogoCss@LaTeXe
  \HOLOGO@Span{LaTeX2e}{%
    \hologo{LaTeX}%
    \HOLOGO@Span{2}{2}%
    \HOLOGO@Span{e}{%
      \HOLOGO@MathSetup
      \ensuremath{\textstyle\varepsilon}%
    }%
  }%
}
%    \end{macrocode}
%    \end{macro}
%    \begin{macro}{\HoLogoCss@LaTeXe}
%    \begin{macrocode}
\def\HoLogoCss@LaTeXe{%
  \Css{%
    span.HoLogo-LaTeX2e span.HoLogo-2{%
      padding-left:.15em;%
    }%
  }%
  \Css{%
    span.HoLogo-LaTeX2e span.HoLogo-e{%
      position:relative;%
      top:.35ex;%
      text-decoration:none;%
    }%
  }%
  \global\let\HoLogoCss@LaTeXe\relax
}
%    \end{macrocode}
%    \end{macro}
%
%    \begin{macro}{\HoLogo@LaTeX2e}
%    \begin{macrocode}
\expandafter
\let\csname HoLogo@LaTeX2e\endcsname\HoLogo@LaTeXe
%    \end{macrocode}
%    \end{macro}
%    \begin{macro}{\HoLogoCs@LaTeX2e}
%    \begin{macrocode}
\expandafter
\let\csname HoLogoCs@LaTeX2e\endcsname\HoLogoCs@LaTeXe
%    \end{macrocode}
%    \end{macro}
%    \begin{macro}{\HoLogoBkm@LaTeX2e}
%    \begin{macrocode}
\expandafter
\let\csname HoLogoBkm@LaTeX2e\endcsname\HoLogoBkm@LaTeXe
%    \end{macrocode}
%    \end{macro}
%    \begin{macro}{\HoLogoHtml@LaTeX2e}
%    \begin{macrocode}
\expandafter
\let\csname HoLogoHtml@LaTeX2e\endcsname\HoLogoHtml@LaTeXe
%    \end{macrocode}
%    \end{macro}
%
% \subsubsection{\hologo{LaTeX3}}
%
%    \begin{macro}{\HoLogo@LaTeX3}
%    Source: \hologo{LaTeX} kernel
%    \begin{macrocode}
\expandafter\def\csname HoLogo@LaTeX3\endcsname#1{%
  \hologo{LaTeX}%
  3%
}
%    \end{macrocode}
%    \end{macro}
%
%    \begin{macro}{\HoLogoBkm@LaTeX3}
%    \begin{macrocode}
\expandafter\def\csname HoLogoBkm@LaTeX3\endcsname#1{%
  \hologo{LaTeX}%
  3%
}
%    \end{macrocode}
%    \end{macro}
%    \begin{macro}{\HoLogoHtml@LaTeX3}
%    \begin{macrocode}
\expandafter
\let\csname HoLogoHtml@LaTeX3\expandafter\endcsname
\csname HoLogo@LaTeX3\endcsname
%    \end{macrocode}
%    \end{macro}
%
% \subsubsection{\hologo{LaTeXML}}
%
%    \begin{macro}{\HoLogo@LaTeXML}
%    \begin{macrocode}
\def\HoLogo@LaTeXML#1{%
  \HOLOGO@mbox{%
    \hologo{La}%
    \kern-.15em%
    T%
    \kern-.1667em%
    \lower.5ex\hbox{E}%
    \kern-.125em%
    \HoLogoFont@font{LaTeXML}{sc}{xml}%
  }%
}
%    \end{macrocode}
%    \end{macro}
%    \begin{macro}{\HoLogoHtml@pdfLaTeX}
%    \begin{macrocode}
\def\HoLogoHtml@LaTeXML#1{%
  \HOLOGO@Span{LaTeXML}{%
    \HoLogoCss@LaTeX
    \HoLogoCss@TeX
    \HOLOGO@Span{LaTeX}{%
      L%
      \HOLOGO@Span{a}{%
        A%
      }%
    }%
    \HOLOGO@Span{TeX}{%
      T%
      \HOLOGO@Span{e}{%
        E%
      }%
    }%
    \HCode{<span style="font-variant: small-caps;">}%
    xml%
    \HCode{</span>}%
  }%
}
%    \end{macrocode}
%    \end{macro}
%
% \subsubsection{\hologo{eTeX}}
%
%    \begin{macro}{\HoLogo@eTeX}
%    Source: package \xpackage{etex}
%    \begin{macrocode}
\def\HoLogo@eTeX#1{%
  \ltx@mbox{%
    \HOLOGO@MathSetup
    $\varepsilon$%
    -%
    \HOLOGO@NegativeKerning{-T,T-,To}%
    \hologo{TeX}%
  }%
}
%    \end{macrocode}
%    \end{macro}
%    \begin{macro}{\HoLogoCs@eTeX}
%    \begin{macrocode}
\ifnum64=`\^^^^0040\relax % test for big chars of LuaTeX/XeTeX
  \catcode`\$=9 %
  \catcode`\&=14 %
\else
  \catcode`\$=14 %
  \catcode`\&=9 %
\fi
\def\HoLogoCs@eTeX#1{%
$ #1{\string ^^^^0395}{\string ^^^^03b5}%
& #1{e}{E}%
  TeX%
}%
\catcode`\$=3 %
\catcode`\&=4 %
%    \end{macrocode}
%    \end{macro}
%    \begin{macro}{\HoLogoBkm@eTeX}
%    \begin{macrocode}
\def\HoLogoBkm@eTeX#1{%
  \HOLOGO@PdfdocUnicode{#1{e}{E}}{\textepsilon}%
  -%
  \hologo{TeX}%
}
%    \end{macrocode}
%    \end{macro}
%    \begin{macro}{\HoLogoHtml@eTeX}
%    \begin{macrocode}
\def\HoLogoHtml@eTeX#1{%
  \ltx@mbox{%
    \HOLOGO@MathSetup
    $\varepsilon$%
    -%
    \hologo{TeX}%
  }%
}
%    \end{macrocode}
%    \end{macro}
%
% \subsubsection{\hologo{iniTeX}}
%
%    \begin{macro}{\HoLogo@iniTeX}
%    \begin{macrocode}
\def\HoLogo@iniTeX#1{%
  \HOLOGO@mbox{%
    #1{i}{I}ni\hologo{TeX}%
  }%
}
%    \end{macrocode}
%    \end{macro}
%    \begin{macro}{\HoLogoCs@iniTeX}
%    \begin{macrocode}
\def\HoLogoCs@iniTeX#1{#1{i}{I}niTeX}
%    \end{macrocode}
%    \end{macro}
%    \begin{macro}{\HoLogoBkm@iniTeX}
%    \begin{macrocode}
\def\HoLogoBkm@iniTeX#1{%
  #1{i}{I}ni\hologo{TeX}%
}
%    \end{macrocode}
%    \end{macro}
%    \begin{macro}{\HoLogoHtml@iniTeX}
%    \begin{macrocode}
\let\HoLogoHtml@iniTeX\HoLogo@iniTeX
%    \end{macrocode}
%    \end{macro}
%
% \subsubsection{\hologo{virTeX}}
%
%    \begin{macro}{\HoLogo@virTeX}
%    \begin{macrocode}
\def\HoLogo@virTeX#1{%
  \HOLOGO@mbox{%
    #1{v}{V}ir\hologo{TeX}%
  }%
}
%    \end{macrocode}
%    \end{macro}
%    \begin{macro}{\HoLogoCs@virTeX}
%    \begin{macrocode}
\def\HoLogoCs@virTeX#1{#1{v}{V}irTeX}
%    \end{macrocode}
%    \end{macro}
%    \begin{macro}{\HoLogoBkm@virTeX}
%    \begin{macrocode}
\def\HoLogoBkm@virTeX#1{%
  #1{v}{V}ir\hologo{TeX}%
}
%    \end{macrocode}
%    \end{macro}
%    \begin{macro}{\HoLogoHtml@virTeX}
%    \begin{macrocode}
\let\HoLogoHtml@virTeX\HoLogo@virTeX
%    \end{macrocode}
%    \end{macro}
%
% \subsubsection{\hologo{SliTeX}}
%
% \paragraph{Definitions of the three variants.}
%
%    \begin{macro}{\HoLogo@SLiTeX@lift}
%    \begin{macrocode}
\def\HoLogo@SLiTeX@lift#1{%
  \HoLogoFont@font{SliTeX}{rm}{%
    S%
    \kern-.06em%
    L%
    \kern-.18em%
    \raise.32ex\hbox{\HoLogoFont@font{SliTeX}{sc}{i}}%
    \HOLOGO@discretionary
    \kern-.06em%
    \hologo{TeX}%
  }%
}
%    \end{macrocode}
%    \end{macro}
%    \begin{macro}{\HoLogoBkm@SLiTeX@lift}
%    \begin{macrocode}
\def\HoLogoBkm@SLiTeX@lift#1{SLiTeX}
%    \end{macrocode}
%    \end{macro}
%    \begin{macro}{\HoLogoHtml@SLiTeX@lift}
%    \begin{macrocode}
\def\HoLogoHtml@SLiTeX@lift#1{%
  \HoLogoCss@SLiTeX@lift
  \HOLOGO@Span{SLiTeX-lift}{%
    \HoLogoFont@font{SliTeX}{rm}{%
      S%
      \HOLOGO@Span{L}{L}%
      \HOLOGO@Span{i}{i}%
      \hologo{TeX}%
    }%
  }%
}
%    \end{macrocode}
%    \end{macro}
%    \begin{macro}{\HoLogoCss@SLiTeX@lift}
%    \begin{macrocode}
\def\HoLogoCss@SLiTeX@lift{%
  \Css{%
    span.HoLogo-SLiTeX-lift span.HoLogo-L{%
      margin-left:-.06em;%
      margin-right:-.18em;%
    }%
  }%
  \Css{%
    span.HoLogo-SLiTeX-lift span.HoLogo-i{%
      position:relative;%
      top:-.32ex;%
      margin-right:-.06em;%
      font-variant:small-caps;%
    }%
  }%
  \global\let\HoLogoCss@SLiTeX@lift\relax
}
%    \end{macrocode}
%    \end{macro}
%
%    \begin{macro}{\HoLogo@SliTeX@simple}
%    \begin{macrocode}
\def\HoLogo@SliTeX@simple#1{%
  \HoLogoFont@font{SliTeX}{rm}{%
    \ltx@mbox{%
      \HoLogoFont@font{SliTeX}{sc}{Sli}%
    }%
    \HOLOGO@discretionary
    \hologo{TeX}%
  }%
}
%    \end{macrocode}
%    \end{macro}
%    \begin{macro}{\HoLogoBkm@SliTeX@simple}
%    \begin{macrocode}
\def\HoLogoBkm@SliTeX@simple#1{SliTeX}
%    \end{macrocode}
%    \end{macro}
%    \begin{macro}{\HoLogoHtml@SliTeX@simple}
%    \begin{macrocode}
\let\HoLogoHtml@SliTeX@simple\HoLogo@SliTeX@simple
%    \end{macrocode}
%    \end{macro}
%
%    \begin{macro}{\HoLogo@SliTeX@narrow}
%    \begin{macrocode}
\def\HoLogo@SliTeX@narrow#1{%
  \HoLogoFont@font{SliTeX}{rm}{%
    \ltx@mbox{%
      S%
      \kern-.06em%
      \HoLogoFont@font{SliTeX}{sc}{%
        l%
        \kern-.035em%
        i%
      }%
    }%
    \HOLOGO@discretionary
    \kern-.06em%
    \hologo{TeX}%
  }%
}
%    \end{macrocode}
%    \end{macro}
%    \begin{macro}{\HoLogoBkm@SliTeX@narrow}
%    \begin{macrocode}
\def\HoLogoBkm@SliTeX@narrow#1{SliTeX}
%    \end{macrocode}
%    \end{macro}
%    \begin{macro}{\HoLogoHtml@SliTeX@narrow}
%    \begin{macrocode}
\def\HoLogoHtml@SliTeX@narrow#1{%
  \HoLogoCss@SliTeX@narrow
  \HOLOGO@Span{SliTeX-narrow}{%
    \HoLogoFont@font{SliTeX}{rm}{%
      S%
        \HOLOGO@Span{l}{l}%
        \HOLOGO@Span{i}{i}%
      \hologo{TeX}%
    }%
  }%
}
%    \end{macrocode}
%    \end{macro}
%    \begin{macro}{\HoLogoCss@SliTeX@narrow}
%    \begin{macrocode}
\def\HoLogoCss@SliTeX@narrow{%
  \Css{%
    span.HoLogo-SliTeX-narrow span.HoLogo-l{%
      margin-left:-.06em;%
      margin-right:-.035em;%
      font-variant:small-caps;%
    }%
  }%
  \Css{%
    span.HoLogo-SliTeX-narrow span.HoLogo-i{%
      margin-right:-.06em;%
      font-variant:small-caps;%
    }%
  }%
  \global\let\HoLogoCss@SliTeX@narrow\relax
}
%    \end{macrocode}
%    \end{macro}
%
% \paragraph{Macro set completion.}
%
%    \begin{macro}{\HoLogo@SLiTeX@simple}
%    \begin{macrocode}
\def\HoLogo@SLiTeX@simple{\HoLogo@SliTeX@simple}
%    \end{macrocode}
%    \end{macro}
%    \begin{macro}{\HoLogoBkm@SLiTeX@simple}
%    \begin{macrocode}
\def\HoLogoBkm@SLiTeX@simple{\HoLogoBkm@SliTeX@simple}
%    \end{macrocode}
%    \end{macro}
%    \begin{macro}{\HoLogoHtml@SLiTeX@simple}
%    \begin{macrocode}
\def\HoLogoHtml@SLiTeX@simple{\HoLogoHtml@SliTeX@simple}
%    \end{macrocode}
%    \end{macro}
%
%    \begin{macro}{\HoLogo@SLiTeX@narrow}
%    \begin{macrocode}
\def\HoLogo@SLiTeX@narrow{\HoLogo@SliTeX@narrow}
%    \end{macrocode}
%    \end{macro}
%    \begin{macro}{\HoLogoBkm@SLiTeX@narrow}
%    \begin{macrocode}
\def\HoLogoBkm@SLiTeX@narrow{\HoLogoBkm@SliTeX@narrow}
%    \end{macrocode}
%    \end{macro}
%    \begin{macro}{\HoLogoHtml@SLiTeX@narrow}
%    \begin{macrocode}
\def\HoLogoHtml@SLiTeX@narrow{\HoLogoHtml@SliTeX@narrow}
%    \end{macrocode}
%    \end{macro}
%
%    \begin{macro}{\HoLogo@SliTeX@lift}
%    \begin{macrocode}
\def\HoLogo@SliTeX@lift{\HoLogo@SLiTeX@lift}
%    \end{macrocode}
%    \end{macro}
%    \begin{macro}{\HoLogoBkm@SliTeX@lift}
%    \begin{macrocode}
\def\HoLogoBkm@SliTeX@lift{\HoLogoBkm@SLiTeX@lift}
%    \end{macrocode}
%    \end{macro}
%    \begin{macro}{\HoLogoHtml@SliTeX@lift}
%    \begin{macrocode}
\def\HoLogoHtml@SliTeX@lift{\HoLogoHtml@SLiTeX@lift}
%    \end{macrocode}
%    \end{macro}
%
% \paragraph{Defaults.}
%
%    \begin{macro}{\HoLogo@SLiTeX}
%    \begin{macrocode}
\def\HoLogo@SLiTeX{\HoLogo@SLiTeX@lift}
%    \end{macrocode}
%    \end{macro}
%    \begin{macro}{\HoLogoBkm@SLiTeX}
%    \begin{macrocode}
\def\HoLogoBkm@SLiTeX{\HoLogoBkm@SLiTeX@lift}
%    \end{macrocode}
%    \end{macro}
%    \begin{macro}{\HoLogoHtml@SLiTeX}
%    \begin{macrocode}
\def\HoLogoHtml@SLiTeX{\HoLogoHtml@SLiTeX@lift}
%    \end{macrocode}
%    \end{macro}
%
%    \begin{macro}{\HoLogo@SliTeX}
%    \begin{macrocode}
\def\HoLogo@SliTeX{\HoLogo@SliTeX@narrow}
%    \end{macrocode}
%    \end{macro}
%    \begin{macro}{\HoLogoBkm@SliTeX}
%    \begin{macrocode}
\def\HoLogoBkm@SliTeX{\HoLogoBkm@SliTeX@narrow}
%    \end{macrocode}
%    \end{macro}
%    \begin{macro}{\HoLogoHtml@SliTeX}
%    \begin{macrocode}
\def\HoLogoHtml@SliTeX{\HoLogoHtml@SliTeX@narrow}
%    \end{macrocode}
%    \end{macro}
%
% \subsubsection{\hologo{LuaTeX}}
%
%    \begin{macro}{\HoLogo@LuaTeX}
%    The kerning is an idea of Hans Hagen, see mailing list
%    `luatex at tug dot org' in March 2010.
%    \begin{macrocode}
\def\HoLogo@LuaTeX#1{%
  \HOLOGO@mbox{%
    Lua%
    \HOLOGO@NegativeKerning{aT,oT,To}%
    \hologo{TeX}%
  }%
}
%    \end{macrocode}
%    \end{macro}
%    \begin{macro}{\HoLogoHtml@LuaTeX}
%    \begin{macrocode}
\let\HoLogoHtml@LuaTeX\HoLogo@LuaTeX
%    \end{macrocode}
%    \end{macro}
%
% \subsubsection{\hologo{LuaLaTeX}}
%
%    \begin{macro}{\HoLogo@LuaLaTeX}
%    \begin{macrocode}
\def\HoLogo@LuaLaTeX#1{%
  \HOLOGO@mbox{%
    Lua%
    \hologo{LaTeX}%
  }%
}
%    \end{macrocode}
%    \end{macro}
%    \begin{macro}{\HoLogoHtml@LuaLaTeX}
%    \begin{macrocode}
\let\HoLogoHtml@LuaLaTeX\HoLogo@LuaLaTeX
%    \end{macrocode}
%    \end{macro}
%
% \subsubsection{\hologo{XeTeX}, \hologo{XeLaTeX}}
%
%    \begin{macro}{\HOLOGO@IfCharExists}
%    \begin{macrocode}
\ifluatex
  \ifnum\luatexversion<36 %
  \else
    \def\HOLOGO@IfCharExists#1{%
      \ifnum
        \directlua{%
           if luaotfload and luaotfload.aux then
             if luaotfload.aux.font_has_glyph(%
                    font.current(), \number#1) then % 	 
	       tex.print("1") % 	 
	     end % 	 
	   elseif font and font.fonts and font.current then %
            local f = font.fonts[font.current()]%
            if f.characters and f.characters[\number#1] then %
              tex.print("1")%
            end %
          end%
        }0=\ltx@zero
        \expandafter\ltx@secondoftwo
      \else
        \expandafter\ltx@firstoftwo
      \fi
    }%
  \fi
\fi
\ltx@IfUndefined{HOLOGO@IfCharExists}{%
  \def\HOLOGO@@IfCharExists#1{%
    \begingroup
      \tracinglostchars=\ltx@zero
      \setbox\ltx@zero=\hbox{%
        \kern7sp\char#1\relax
        \ifnum\lastkern>\ltx@zero
          \expandafter\aftergroup\csname iffalse\endcsname
        \else
          \expandafter\aftergroup\csname iftrue\endcsname
        \fi
      }%
      % \if{true|false} from \aftergroup
      \endgroup
      \expandafter\ltx@firstoftwo
    \else
      \endgroup
      \expandafter\ltx@secondoftwo
    \fi
  }%
  \ifxetex
    \ltx@IfUndefined{XeTeXfonttype}{}{%
      \ltx@IfUndefined{XeTeXcharglyph}{}{%
        \def\HOLOGO@IfCharExists#1{%
          \ifnum\XeTeXfonttype\font>\ltx@zero
            \expandafter\ltx@firstofthree
          \else
            \expandafter\ltx@gobble
          \fi
          {%
            \ifnum\XeTeXcharglyph#1>\ltx@zero
              \expandafter\ltx@firstoftwo
            \else
              \expandafter\ltx@secondoftwo
            \fi
          }%
          \HOLOGO@@IfCharExists{#1}%
        }%
      }%
    }%
  \fi
}{}
\ltx@ifundefined{HOLOGO@IfCharExists}{%
  \ifnum64=`\^^^^0040\relax % test for big chars of LuaTeX/XeTeX
    \let\HOLOGO@IfCharExists\HOLOGO@@IfCharExists
  \else
    \def\HOLOGO@IfCharExists#1{%
      \ifnum#1>255 %
        \expandafter\ltx@fourthoffour
      \fi
      \HOLOGO@@IfCharExists{#1}%
    }%
  \fi
}{}
%    \end{macrocode}
%    \end{macro}
%
%    \begin{macro}{\HoLogo@Xe}
%    Source: package \xpackage{dtklogos}
%    \begin{macrocode}
\def\HoLogo@Xe#1{%
  X%
  \kern-.1em\relax
  \HOLOGO@IfCharExists{"018E}{%
    \lower.5ex\hbox{\char"018E}%
  }{%
    \chardef\HOLOGO@choice=\ltx@zero
    \ifdim\fontdimen\ltx@one\font>0pt %
      \ltx@IfUndefined{rotatebox}{%
        \ltx@IfUndefined{pgftext}{%
          \ltx@IfUndefined{psscalebox}{%
            \ltx@IfUndefined{HOLOGO@ScaleBox@\hologoDriver}{%
            }{%
              \chardef\HOLOGO@choice=4 %
            }%
          }{%
            \chardef\HOLOGO@choice=3 %
          }%
        }{%
          \chardef\HOLOGO@choice=2 %
        }%
      }{%
        \chardef\HOLOGO@choice=1 %
      }%
      \ifcase\HOLOGO@choice
        \HOLOGO@WarningUnsupportedDriver{Xe}%
        e%
      \or % 1: \rotatebox
        \begingroup
          \setbox\ltx@zero\hbox{\rotatebox{180}{E}}%
          \ltx@LocDimenA=\dp\ltx@zero
          \advance\ltx@LocDimenA by -.5ex\relax
          \raise\ltx@LocDimenA\box\ltx@zero
        \endgroup
      \or % 2: \pgftext
        \lower.5ex\hbox{%
          \pgfpicture
            \pgftext[rotate=180]{E}%
          \endpgfpicture
        }%
      \or % 3: \psscalebox
        \begingroup
          \setbox\ltx@zero\hbox{\psscalebox{-1 -1}{E}}%
          \ltx@LocDimenA=\dp\ltx@zero
          \advance\ltx@LocDimenA by -.5ex\relax
          \raise\ltx@LocDimenA\box\ltx@zero
        \endgroup
      \or % 4: \HOLOGO@PointReflectBox
        \lower.5ex\hbox{\HOLOGO@PointReflectBox{E}}%
      \else
        \@PackageError{hologo}{Internal error (choice/it}\@ehc
      \fi
    \else
      \ltx@IfUndefined{reflectbox}{%
        \ltx@IfUndefined{pgftext}{%
          \ltx@IfUndefined{psscalebox}{%
            \ltx@IfUndefined{HOLOGO@ScaleBox@\hologoDriver}{%
            }{%
              \chardef\HOLOGO@choice=4 %
            }%
          }{%
            \chardef\HOLOGO@choice=3 %
          }%
        }{%
          \chardef\HOLOGO@choice=2 %
        }%
      }{%
        \chardef\HOLOGO@choice=1 %
      }%
      \ifcase\HOLOGO@choice
        \HOLOGO@WarningUnsupportedDriver{Xe}%
        e%
      \or % 1: reflectbox
        \lower.5ex\hbox{%
          \reflectbox{E}%
        }%
      \or % 2: \pgftext
        \lower.5ex\hbox{%
          \pgfpicture
            \pgftransformxscale{-1}%
            \pgftext{E}%
          \endpgfpicture
        }%
      \or % 3: \psscalebox
        \lower.5ex\hbox{%
          \psscalebox{-1 1}{E}%
        }%
      \or % 4: \HOLOGO@Reflectbox
        \lower.5ex\hbox{%
          \HOLOGO@ReflectBox{E}%
        }%
      \else
        \@PackageError{hologo}{Internal error (choice/up)}\@ehc
      \fi
    \fi
  }%
}
%    \end{macrocode}
%    \end{macro}
%    \begin{macro}{\HoLogoHtml@Xe}
%    \begin{macrocode}
\def\HoLogoHtml@Xe#1{%
  \HoLogoCss@Xe
  \HOLOGO@Span{Xe}{%
    X%
    \HOLOGO@Span{e}{%
      \HCode{&\ltx@hashchar x018e;}%
    }%
  }%
}
%    \end{macrocode}
%    \end{macro}
%    \begin{macro}{\HoLogoCss@Xe}
%    \begin{macrocode}
\def\HoLogoCss@Xe{%
  \Css{%
    span.HoLogo-Xe span.HoLogo-e{%
      position:relative;%
      top:.5ex;%
      left-margin:-.1em;%
    }%
  }%
  \global\let\HoLogoCss@Xe\relax
}
%    \end{macrocode}
%    \end{macro}
%
%    \begin{macro}{\HoLogo@XeTeX}
%    \begin{macrocode}
\def\HoLogo@XeTeX#1{%
  \hologo{Xe}%
  \kern-.15em\relax
  \hologo{TeX}%
}
%    \end{macrocode}
%    \end{macro}
%
%    \begin{macro}{\HoLogoHtml@XeTeX}
%    \begin{macrocode}
\def\HoLogoHtml@XeTeX#1{%
  \HoLogoCss@XeTeX
  \HOLOGO@Span{XeTeX}{%
    \hologo{Xe}%
    \hologo{TeX}%
  }%
}
%    \end{macrocode}
%    \end{macro}
%    \begin{macro}{\HoLogoCss@XeTeX}
%    \begin{macrocode}
\def\HoLogoCss@XeTeX{%
  \Css{%
    span.HoLogo-XeTeX span.HoLogo-TeX{%
      margin-left:-.15em;%
    }%
  }%
  \global\let\HoLogoCss@XeTeX\relax
}
%    \end{macrocode}
%    \end{macro}
%
%    \begin{macro}{\HoLogo@XeLaTeX}
%    \begin{macrocode}
\def\HoLogo@XeLaTeX#1{%
  \hologo{Xe}%
  \kern-.13em%
  \hologo{LaTeX}%
}
%    \end{macrocode}
%    \end{macro}
%    \begin{macro}{\HoLogoHtml@XeLaTeX}
%    \begin{macrocode}
\def\HoLogoHtml@XeLaTeX#1{%
  \HoLogoCss@XeLaTeX
  \HOLOGO@Span{XeLaTeX}{%
    \hologo{Xe}%
    \hologo{LaTeX}%
  }%
}
%    \end{macrocode}
%    \end{macro}
%    \begin{macro}{\HoLogoCss@XeLaTeX}
%    \begin{macrocode}
\def\HoLogoCss@XeLaTeX{%
  \Css{%
    span.HoLogo-XeLaTeX span.HoLogo-Xe{%
      margin-right:-.13em;%
    }%
  }%
  \global\let\HoLogoCss@XeLaTeX\relax
}
%    \end{macrocode}
%    \end{macro}
%
% \subsubsection{\hologo{pdfTeX}, \hologo{pdfLaTeX}}
%
%    \begin{macro}{\HoLogo@pdfTeX}
%    \begin{macrocode}
\def\HoLogo@pdfTeX#1{%
  \HOLOGO@mbox{%
    #1{p}{P}df\hologo{TeX}%
  }%
}
%    \end{macrocode}
%    \end{macro}
%    \begin{macro}{\HoLogoCs@pdfTeX}
%    \begin{macrocode}
\def\HoLogoCs@pdfTeX#1{#1{p}{P}dfTeX}
%    \end{macrocode}
%    \end{macro}
%    \begin{macro}{\HoLogoBkm@pdfTeX}
%    \begin{macrocode}
\def\HoLogoBkm@pdfTeX#1{%
  #1{p}{P}df\hologo{TeX}%
}
%    \end{macrocode}
%    \end{macro}
%    \begin{macro}{\HoLogoHtml@pdfTeX}
%    \begin{macrocode}
\let\HoLogoHtml@pdfTeX\HoLogo@pdfTeX
%    \end{macrocode}
%    \end{macro}
%
%    \begin{macro}{\HoLogo@pdfLaTeX}
%    \begin{macrocode}
\def\HoLogo@pdfLaTeX#1{%
  \HOLOGO@mbox{%
    #1{p}{P}df\hologo{LaTeX}%
  }%
}
%    \end{macrocode}
%    \end{macro}
%    \begin{macro}{\HoLogoCs@pdfLaTeX}
%    \begin{macrocode}
\def\HoLogoCs@pdfLaTeX#1{#1{p}{P}dfLaTeX}
%    \end{macrocode}
%    \end{macro}
%    \begin{macro}{\HoLogoBkm@pdfLaTeX}
%    \begin{macrocode}
\def\HoLogoBkm@pdfLaTeX#1{%
  #1{p}{P}df\hologo{LaTeX}%
}
%    \end{macrocode}
%    \end{macro}
%    \begin{macro}{\HoLogoHtml@pdfLaTeX}
%    \begin{macrocode}
\let\HoLogoHtml@pdfLaTeX\HoLogo@pdfLaTeX
%    \end{macrocode}
%    \end{macro}
%
% \subsubsection{\hologo{VTeX}}
%
%    \begin{macro}{\HoLogo@VTeX}
%    \begin{macrocode}
\def\HoLogo@VTeX#1{%
  \HOLOGO@mbox{%
    V\hologo{TeX}%
  }%
}
%    \end{macrocode}
%    \end{macro}
%    \begin{macro}{\HoLogoHtml@VTeX}
%    \begin{macrocode}
\let\HoLogoHtml@VTeX\HoLogo@VTeX
%    \end{macrocode}
%    \end{macro}
%
% \subsubsection{\hologo{AmS}, \dots}
%
%    Source: class \xclass{amsdtx}
%
%    \begin{macro}{\HoLogo@AmS}
%    \begin{macrocode}
\def\HoLogo@AmS#1{%
  \HoLogoFont@font{AmS}{sy}{%
    A%
    \kern-.1667em%
    \lower.5ex\hbox{M}%
    \kern-.125em%
    S%
  }%
}
%    \end{macrocode}
%    \end{macro}
%    \begin{macro}{\HoLogoBkm@AmS}
%    \begin{macrocode}
\def\HoLogoBkm@AmS#1{AmS}
%    \end{macrocode}
%    \end{macro}
%    \begin{macro}{\HoLogoHtml@AmS}
%    \begin{macrocode}
\def\HoLogoHtml@AmS#1{%
  \HoLogoCss@AmS
%  \HoLogoFont@font{AmS}{sy}{%
    \HOLOGO@Span{AmS}{%
      A%
      \HOLOGO@Span{M}{M}%
      S%
    }%
%   }%
}
%    \end{macrocode}
%    \end{macro}
%    \begin{macro}{\HoLogoCss@AmS}
%    \begin{macrocode}
\def\HoLogoCss@AmS{%
  \Css{%
    span.HoLogo-AmS span.HoLogo-M{%
      position:relative;%
      top:.5ex;%
      margin-left:-.1667em;%
      margin-right:-.125em;%
      text-decoration:none;%
    }%
  }%
  \global\let\HoLogoCss@AmS\relax
}
%    \end{macrocode}
%    \end{macro}
%
%    \begin{macro}{\HoLogo@AmSTeX}
%    \begin{macrocode}
\def\HoLogo@AmSTeX#1{%
  \hologo{AmS}%
  \HOLOGO@hyphen
  \hologo{TeX}%
}
%    \end{macrocode}
%    \end{macro}
%    \begin{macro}{\HoLogoBkm@AmSTeX}
%    \begin{macrocode}
\def\HoLogoBkm@AmSTeX#1{AmS-TeX}%
%    \end{macrocode}
%    \end{macro}
%    \begin{macro}{\HoLogoHtml@AmSTeX}
%    \begin{macrocode}
\let\HoLogoHtml@AmSTeX\HoLogo@AmSTeX
%    \end{macrocode}
%    \end{macro}
%
%    \begin{macro}{\HoLogo@AmSLaTeX}
%    \begin{macrocode}
\def\HoLogo@AmSLaTeX#1{%
  \hologo{AmS}%
  \HOLOGO@hyphen
  \hologo{LaTeX}%
}
%    \end{macrocode}
%    \end{macro}
%    \begin{macro}{\HoLogoBkm@AmSLaTeX}
%    \begin{macrocode}
\def\HoLogoBkm@AmSLaTeX#1{AmS-LaTeX}%
%    \end{macrocode}
%    \end{macro}
%    \begin{macro}{\HoLogoHtml@AmSLaTeX}
%    \begin{macrocode}
\let\HoLogoHtml@AmSLaTeX\HoLogo@AmSLaTeX
%    \end{macrocode}
%    \end{macro}
%
% \subsubsection{\hologo{BibTeX}}
%
%    \begin{macro}{\HoLogo@BibTeX@sc}
%    A definition of \hologo{BibTeX} is provided in
%    the documentation source for the manual of \hologo{BibTeX}
%    \cite{btxdoc}.
%\begin{quote}
%\begin{verbatim}
%\def\BibTeX{%
%  {%
%    \rm
%    B%
%    \kern-.05em%
%    {%
%      \sc
%      i%
%      \kern-.025em %
%      b%
%    }%
%    \kern-.08em
%    T%
%    \kern-.1667em%
%    \lower.7ex\hbox{E}%
%    \kern-.125em%
%    X%
%  }%
%}
%\end{verbatim}
%\end{quote}
%    \begin{macrocode}
\def\HoLogo@BibTeX@sc#1{%
  B%
  \kern-.05em%
  \HoLogoFont@font{BibTeX}{sc}{%
    i%
    \kern-.025em%
    b%
  }%
  \HOLOGO@discretionary
  \kern-.08em%
  \hologo{TeX}%
}
%    \end{macrocode}
%    \end{macro}
%    \begin{macro}{\HoLogoHtml@BibTeX@sc}
%    \begin{macrocode}
\def\HoLogoHtml@BibTeX@sc#1{%
  \HoLogoCss@BibTeX@sc
  \HOLOGO@Span{BibTeX-sc}{%
    B%
    \HOLOGO@Span{i}{i}%
    \HOLOGO@Span{b}{b}%
    \hologo{TeX}%
  }%
}
%    \end{macrocode}
%    \end{macro}
%    \begin{macro}{\HoLogoCss@BibTeX@sc}
%    \begin{macrocode}
\def\HoLogoCss@BibTeX@sc{%
  \Css{%
    span.HoLogo-BibTeX-sc span.HoLogo-i{%
      margin-left:-.05em;%
      margin-right:-.025em;%
      font-variant:small-caps;%
    }%
  }%
  \Css{%
    span.HoLogo-BibTeX-sc span.HoLogo-b{%
      margin-right:-.08em;%
      font-variant:small-caps;%
    }%
  }%
  \global\let\HoLogoCss@BibTeX@sc\relax
}
%    \end{macrocode}
%    \end{macro}
%
%    \begin{macro}{\HoLogo@BibTeX@sf}
%    Variant \xoption{sf} avoids trouble with unavailable
%    small caps fonts (e.g., bold versions of Computer Modern or
%    Latin Modern). The definition is taken from
%    package \xpackage{dtklogos} \cite{dtklogos}.
%\begin{quote}
%\begin{verbatim}
%\DeclareRobustCommand{\BibTeX}{%
%  B%
%  \kern-.05em%
%  \hbox{%
%    $\m@th$% %% force math size calculations
%    \csname S@\f@size\endcsname
%    \fontsize\sf@size\z@
%    \math@fontsfalse
%    \selectfont
%    I%
%    \kern-.025em%
%    B
%  }%
%  \kern-.08em%
%  \-%
%  \TeX
%}
%\end{verbatim}
%\end{quote}
%    \begin{macrocode}
\def\HoLogo@BibTeX@sf#1{%
  B%
  \kern-.05em%
  \HoLogoFont@font{BibTeX}{bibsf}{%
    I%
    \kern-.025em%
    B%
  }%
  \HOLOGO@discretionary
  \kern-.08em%
  \hologo{TeX}%
}
%    \end{macrocode}
%    \end{macro}
%    \begin{macro}{\HoLogoHtml@BibTeX@sf}
%    \begin{macrocode}
\def\HoLogoHtml@BibTeX@sf#1{%
  \HoLogoCss@BibTeX@sf
  \HOLOGO@Span{BibTeX-sf}{%
    B%
    \HoLogoFont@font{BibTeX}{bibsf}{%
      \HOLOGO@Span{i}{I}%
      B%
    }%
    \hologo{TeX}%
  }%
}
%    \end{macrocode}
%    \end{macro}
%    \begin{macro}{\HoLogoCss@BibTeX@sf}
%    \begin{macrocode}
\def\HoLogoCss@BibTeX@sf{%
  \Css{%
    span.HoLogo-BibTeX-sf span.HoLogo-i{%
      margin-left:-.05em;%
      margin-right:-.025em;%
    }%
  }%
  \Css{%
    span.HoLogo-BibTeX-sf span.HoLogo-TeX{%
      margin-left:-.08em;%
    }%
  }%
  \global\let\HoLogoCss@BibTeX@sf\relax
}
%    \end{macrocode}
%    \end{macro}
%
%    \begin{macro}{\HoLogo@BibTeX}
%    \begin{macrocode}
\def\HoLogo@BibTeX{\HoLogo@BibTeX@sf}
%    \end{macrocode}
%    \end{macro}
%    \begin{macro}{\HoLogoHtml@BibTeX}
%    \begin{macrocode}
\def\HoLogoHtml@BibTeX{\HoLogoHtml@BibTeX@sf}
%    \end{macrocode}
%    \end{macro}
%
% \subsubsection{\hologo{BibTeX8}}
%
%    \begin{macro}{\HoLogo@BibTeX8}
%    \begin{macrocode}
\expandafter\def\csname HoLogo@BibTeX8\endcsname#1{%
  \hologo{BibTeX}%
  8%
}
%    \end{macrocode}
%    \end{macro}
%
%    \begin{macro}{\HoLogoBkm@BibTeX8}
%    \begin{macrocode}
\expandafter\def\csname HoLogoBkm@BibTeX8\endcsname#1{%
  \hologo{BibTeX}%
  8%
}
%    \end{macrocode}
%    \end{macro}
%    \begin{macro}{\HoLogoHtml@BibTeX8}
%    \begin{macrocode}
\expandafter
\let\csname HoLogoHtml@BibTeX8\expandafter\endcsname
\csname HoLogo@BibTeX8\endcsname
%    \end{macrocode}
%    \end{macro}
%
% \subsubsection{\hologo{ConTeXt}}
%
%    \begin{macro}{\HoLogo@ConTeXt@simple}
%    \begin{macrocode}
\def\HoLogo@ConTeXt@simple#1{%
  \HOLOGO@mbox{Con}%
  \HOLOGO@discretionary
  \HOLOGO@mbox{\hologo{TeX}t}%
}
%    \end{macrocode}
%    \end{macro}
%    \begin{macro}{\HoLogoHtml@ConTeXt@simple}
%    \begin{macrocode}
\let\HoLogoHtml@ConTeXt@simple\HoLogo@ConTeXt@simple
%    \end{macrocode}
%    \end{macro}
%
%    \begin{macro}{\HoLogo@ConTeXt@narrow}
%    This definition of logo \hologo{ConTeXt} with variant \xoption{narrow}
%    comes from TUGboat's class \xclass{ltugboat} (version 2010/11/15 v2.8).
%    \begin{macrocode}
\def\HoLogo@ConTeXt@narrow#1{%
  \HOLOGO@mbox{C\kern-.0333emon}%
  \HOLOGO@discretionary
  \kern-.0667em%
  \HOLOGO@mbox{\hologo{TeX}\kern-.0333emt}%
}
%    \end{macrocode}
%    \end{macro}
%    \begin{macro}{\HoLogoHtml@ConTeXt@narrow}
%    \begin{macrocode}
\def\HoLogoHtml@ConTeXt@narrow#1{%
  \HoLogoCss@ConTeXt@narrow
  \HOLOGO@Span{ConTeXt-narrow}{%
    \HOLOGO@Span{C}{C}%
    on%
    \hologo{TeX}%
    t%
  }%
}
%    \end{macrocode}
%    \end{macro}
%    \begin{macro}{\HoLogoCss@ConTeXt@narrow}
%    \begin{macrocode}
\def\HoLogoCss@ConTeXt@narrow{%
  \Css{%
    span.HoLogo-ConTeXt-narrow span.HoLogo-C{%
      margin-left:-.0333em;%
    }%
  }%
  \Css{%
    span.HoLogo-ConTeXt-narrow span.HoLogo-TeX{%
      margin-left:-.0667em;%
      margin-right:-.0333em;%
    }%
  }%
  \global\let\HoLogoCss@ConTeXt@narrow\relax
}
%    \end{macrocode}
%    \end{macro}
%
%    \begin{macro}{\HoLogo@ConTeXt}
%    \begin{macrocode}
\def\HoLogo@ConTeXt{\HoLogo@ConTeXt@narrow}
%    \end{macrocode}
%    \end{macro}
%    \begin{macro}{\HoLogoHtml@ConTeXt}
%    \begin{macrocode}
\def\HoLogoHtml@ConTeXt{\HoLogoHtml@ConTeXt@narrow}
%    \end{macrocode}
%    \end{macro}
%
% \subsubsection{\hologo{emTeX}}
%
%    \begin{macro}{\HoLogo@emTeX}
%    \begin{macrocode}
\def\HoLogo@emTeX#1{%
  \HOLOGO@mbox{#1{e}{E}m}%
  \HOLOGO@discretionary
  \hologo{TeX}%
}
%    \end{macrocode}
%    \end{macro}
%    \begin{macro}{\HoLogoCs@emTeX}
%    \begin{macrocode}
\def\HoLogoCs@emTeX#1{#1{e}{E}mTeX}%
%    \end{macrocode}
%    \end{macro}
%    \begin{macro}{\HoLogoBkm@emTeX}
%    \begin{macrocode}
\def\HoLogoBkm@emTeX#1{%
  #1{e}{E}m\hologo{TeX}%
}
%    \end{macrocode}
%    \end{macro}
%    \begin{macro}{\HoLogoHtml@emTeX}
%    \begin{macrocode}
\let\HoLogoHtml@emTeX\HoLogo@emTeX
%    \end{macrocode}
%    \end{macro}
%
% \subsubsection{\hologo{ExTeX}}
%
%    \begin{macro}{\HoLogo@ExTeX}
%    The definition is taken from the FAQ of the
%    project \hologo{ExTeX}
%    \cite{ExTeX-FAQ}.
%\begin{quote}
%\begin{verbatim}
%\def\ExTeX{%
%  \textrm{% Logo always with serifs
%    \ensuremath{%
%      \textstyle
%      \varepsilon_{%
%        \kern-0.15em%
%        \mathcal{X}%
%      }%
%    }%
%    \kern-.15em%
%    \TeX
%  }%
%}
%\end{verbatim}
%\end{quote}
%    \begin{macrocode}
\def\HoLogo@ExTeX#1{%
  \HoLogoFont@font{ExTeX}{rm}{%
    \ltx@mbox{%
      \HOLOGO@MathSetup
      $%
        \textstyle
        \varepsilon_{%
          \kern-0.15em%
          \HoLogoFont@font{ExTeX}{sy}{X}%
        }%
      $%
    }%
    \HOLOGO@discretionary
    \kern-.15em%
    \hologo{TeX}%
  }%
}
%    \end{macrocode}
%    \end{macro}
%    \begin{macro}{\HoLogoHtml@ExTeX}
%    \begin{macrocode}
\def\HoLogoHtml@ExTeX#1{%
  \HoLogoCss@ExTeX
  \HoLogoFont@font{ExTeX}{rm}{%
    \HOLOGO@Span{ExTeX}{%
      \ltx@mbox{%
        \HOLOGO@MathSetup
        $\textstyle\varepsilon$%
        \HOLOGO@Span{X}{$\textstyle\chi$}%
        \hologo{TeX}%
      }%
    }%
  }%
}
%    \end{macrocode}
%    \end{macro}
%    \begin{macro}{\HoLogoBkm@ExTeX}
%    \begin{macrocode}
\def\HoLogoBkm@ExTeX#1{%
  \HOLOGO@PdfdocUnicode{#1{e}{E}x}{\textepsilon\textchi}%
  \hologo{TeX}%
}
%    \end{macrocode}
%    \end{macro}
%    \begin{macro}{\HoLogoCss@ExTeX}
%    \begin{macrocode}
\def\HoLogoCss@ExTeX{%
  \Css{%
    span.HoLogo-ExTeX{%
      font-family:serif;%
    }%
  }%
  \Css{%
    span.HoLogo-ExTeX span.HoLogo-TeX{%
      margin-left:-.15em;%
    }%
  }%
  \global\let\HoLogoCss@ExTeX\relax
}
%    \end{macrocode}
%    \end{macro}
%
% \subsubsection{\hologo{MiKTeX}}
%
%    \begin{macro}{\HoLogo@MiKTeX}
%    \begin{macrocode}
\def\HoLogo@MiKTeX#1{%
  \HOLOGO@mbox{MiK}%
  \HOLOGO@discretionary
  \hologo{TeX}%
}
%    \end{macrocode}
%    \end{macro}
%    \begin{macro}{\HoLogoHtml@MiKTeX}
%    \begin{macrocode}
\let\HoLogoHtml@MiKTeX\HoLogo@MiKTeX
%    \end{macrocode}
%    \end{macro}
%
% \subsubsection{\hologo{OzTeX} and friends}
%
%    Source: \hologo{OzTeX} FAQ \cite{OzTeX}:
%    \begin{quote}
%      |\def\OzTeX{O\kern-.03em z\kern-.15em\TeX}|\\
%      (There is no kerning in OzMF, OzMP and OzTtH.)
%    \end{quote}
%
%    \begin{macro}{\HoLogo@OzTeX}
%    \begin{macrocode}
\def\HoLogo@OzTeX#1{%
  O%
  \kern-.03em %
  z%
  \kern-.15em %
  \hologo{TeX}%
}
%    \end{macrocode}
%    \end{macro}
%    \begin{macro}{\HoLogoHtml@OzTeX}
%    \begin{macrocode}
\def\HoLogoHtml@OzTeX#1{%
  \HoLogoCss@OzTeX
  \HOLOGO@Span{OzTeX}{%
    O%
    \HOLOGO@Span{z}{z}%
    \hologo{TeX}%
  }%
}
%    \end{macrocode}
%    \end{macro}
%    \begin{macro}{\HoLogoCss@OzTeX}
%    \begin{macrocode}
\def\HoLogoCss@OzTeX{%
  \Css{%
    span.HoLogo-OzTeX span.HoLogo-z{%
      margin-left:-.03em;%
      margin-right:-.15em;%
    }%
  }%
  \global\let\HoLogoCss@OzTeX\relax
}
%    \end{macrocode}
%    \end{macro}
%
%    \begin{macro}{\HoLogo@OzMF}
%    \begin{macrocode}
\def\HoLogo@OzMF#1{%
  \HOLOGO@mbox{OzMF}%
}
%    \end{macrocode}
%    \end{macro}
%    \begin{macro}{\HoLogo@OzMP}
%    \begin{macrocode}
\def\HoLogo@OzMP#1{%
  \HOLOGO@mbox{OzMP}%
}
%    \end{macrocode}
%    \end{macro}
%    \begin{macro}{\HoLogo@OzTtH}
%    \begin{macrocode}
\def\HoLogo@OzTtH#1{%
  \HOLOGO@mbox{OzTtH}%
}
%    \end{macrocode}
%    \end{macro}
%
% \subsubsection{\hologo{PCTeX}}
%
%    \begin{macro}{\HoLogo@PCTeX}
%    \begin{macrocode}
\def\HoLogo@PCTeX#1{%
  \HOLOGO@mbox{PC}%
  \hologo{TeX}%
}
%    \end{macrocode}
%    \end{macro}
%    \begin{macro}{\HoLogoHtml@PCTeX}
%    \begin{macrocode}
\let\HoLogoHtml@PCTeX\HoLogo@PCTeX
%    \end{macrocode}
%    \end{macro}
%
% \subsubsection{\hologo{PiCTeX}}
%
%    The original definitions from \xfile{pictex.tex} \cite{PiCTeX}:
%\begin{quote}
%\begin{verbatim}
%\def\PiC{%
%  P%
%  \kern-.12em%
%  \lower.5ex\hbox{I}%
%  \kern-.075em%
%  C%
%}
%\def\PiCTeX{%
%  \PiC
%  \kern-.11em%
%  \TeX
%}
%\end{verbatim}
%\end{quote}
%
%    \begin{macro}{\HoLogo@PiC}
%    \begin{macrocode}
\def\HoLogo@PiC#1{%
  P%
  \kern-.12em%
  \lower.5ex\hbox{I}%
  \kern-.075em%
  C%
  \HOLOGO@SpaceFactor
}
%    \end{macrocode}
%    \end{macro}
%    \begin{macro}{\HoLogoHtml@PiC}
%    \begin{macrocode}
\def\HoLogoHtml@PiC#1{%
  \HoLogoCss@PiC
  \HOLOGO@Span{PiC}{%
    P%
    \HOLOGO@Span{i}{I}%
    C%
  }%
}
%    \end{macrocode}
%    \end{macro}
%    \begin{macro}{\HoLogoCss@PiC}
%    \begin{macrocode}
\def\HoLogoCss@PiC{%
  \Css{%
    span.HoLogo-PiC span.HoLogo-i{%
      position:relative;%
      top:.5ex;%
      margin-left:-.12em;%
      margin-right:-.075em;%
      text-decoration:none;%
    }%
  }%
  \global\let\HoLogoCss@PiC\relax
}
%    \end{macrocode}
%    \end{macro}
%
%    \begin{macro}{\HoLogo@PiCTeX}
%    \begin{macrocode}
\def\HoLogo@PiCTeX#1{%
  \hologo{PiC}%
  \HOLOGO@discretionary
  \kern-.11em%
  \hologo{TeX}%
}
%    \end{macrocode}
%    \end{macro}
%    \begin{macro}{\HoLogoHtml@PiCTeX}
%    \begin{macrocode}
\def\HoLogoHtml@PiCTeX#1{%
  \HoLogoCss@PiCTeX
  \HOLOGO@Span{PiCTeX}{%
    \hologo{PiC}%
    \hologo{TeX}%
  }%
}
%    \end{macrocode}
%    \end{macro}
%    \begin{macro}{\HoLogoCss@PiCTeX}
%    \begin{macrocode}
\def\HoLogoCss@PiCTeX{%
  \Css{%
    span.HoLogo-PiCTeX span.HoLogo-PiC{%
      margin-right:-.11em;%
    }%
  }%
  \global\let\HoLogoCss@PiCTeX\relax
}
%    \end{macrocode}
%    \end{macro}
%
% \subsubsection{\hologo{teTeX}}
%
%    \begin{macro}{\HoLogo@teTeX}
%    \begin{macrocode}
\def\HoLogo@teTeX#1{%
  \HOLOGO@mbox{#1{t}{T}e}%
  \HOLOGO@discretionary
  \hologo{TeX}%
}
%    \end{macrocode}
%    \end{macro}
%    \begin{macro}{\HoLogoCs@teTeX}
%    \begin{macrocode}
\def\HoLogoCs@teTeX#1{#1{t}{T}dfTeX}
%    \end{macrocode}
%    \end{macro}
%    \begin{macro}{\HoLogoBkm@teTeX}
%    \begin{macrocode}
\def\HoLogoBkm@teTeX#1{%
  #1{t}{T}e\hologo{TeX}%
}
%    \end{macrocode}
%    \end{macro}
%    \begin{macro}{\HoLogoHtml@teTeX}
%    \begin{macrocode}
\let\HoLogoHtml@teTeX\HoLogo@teTeX
%    \end{macrocode}
%    \end{macro}
%
% \subsubsection{\hologo{TeX4ht}}
%
%    \begin{macro}{\HoLogo@TeX4ht}
%    \begin{macrocode}
\expandafter\def\csname HoLogo@TeX4ht\endcsname#1{%
  \HOLOGO@mbox{\hologo{TeX}4ht}%
}
%    \end{macrocode}
%    \end{macro}
%    \begin{macro}{\HoLogoHtml@TeX4ht}
%    \begin{macrocode}
\expandafter
\let\csname HoLogoHtml@TeX4ht\expandafter\endcsname
\csname HoLogo@TeX4ht\endcsname
%    \end{macrocode}
%    \end{macro}
%
%
% \subsubsection{\hologo{SageTeX}}
%
%    \begin{macro}{\HoLogo@SageTeX}
%    \begin{macrocode}
\def\HoLogo@SageTeX#1{%
  \HOLOGO@mbox{Sage}%
  \HOLOGO@discretionary
  \HOLOGO@NegativeKerning{eT,oT,To}%
  \hologo{TeX}%
}
%    \end{macrocode}
%    \end{macro}
%    \begin{macro}{\HoLogoHtml@SageTeX}
%    \begin{macrocode}
\let\HoLogoHtml@SageTeX\HoLogo@SageTeX
%    \end{macrocode}
%    \end{macro}
%
% \subsection{\hologo{METAFONT} and friends}
%
%    \begin{macro}{\HoLogo@METAFONT}
%    \begin{macrocode}
\def\HoLogo@METAFONT#1{%
  \HoLogoFont@font{METAFONT}{logo}{%
    \HOLOGO@mbox{META}%
    \HOLOGO@discretionary
    \HOLOGO@mbox{FONT}%
  }%
}
%    \end{macrocode}
%    \end{macro}
%
%    \begin{macro}{\HoLogo@METAPOST}
%    \begin{macrocode}
\def\HoLogo@METAPOST#1{%
  \HoLogoFont@font{METAPOST}{logo}{%
    \HOLOGO@mbox{META}%
    \HOLOGO@discretionary
    \HOLOGO@mbox{POST}%
  }%
}
%    \end{macrocode}
%    \end{macro}
%
%    \begin{macro}{\HoLogo@MetaFun}
%    \begin{macrocode}
\def\HoLogo@MetaFun#1{%
  \HOLOGO@mbox{Meta}%
  \HOLOGO@discretionary
  \HOLOGO@mbox{Fun}%
}
%    \end{macrocode}
%    \end{macro}
%
%    \begin{macro}{\HoLogo@MetaPost}
%    \begin{macrocode}
\def\HoLogo@MetaPost#1{%
  \HOLOGO@mbox{Meta}%
  \HOLOGO@discretionary
  \HOLOGO@mbox{Post}%
}
%    \end{macrocode}
%    \end{macro}
%
% \subsection{Others}
%
% \subsubsection{\hologo{biber}}
%
%    \begin{macro}{\HoLogo@biber}
%    \begin{macrocode}
\def\HoLogo@biber#1{%
  \HOLOGO@mbox{#1{b}{B}i}%
  \HOLOGO@discretionary
  \HOLOGO@mbox{ber}%
}
%    \end{macrocode}
%    \end{macro}
%    \begin{macro}{\HoLogoCs@biber}
%    \begin{macrocode}
\def\HoLogoCs@biber#1{#1{b}{B}iber}
%    \end{macrocode}
%    \end{macro}
%    \begin{macro}{\HoLogoBkm@biber}
%    \begin{macrocode}
\def\HoLogoBkm@biber#1{%
  #1{b}{B}iber%
}
%    \end{macrocode}
%    \end{macro}
%    \begin{macro}{\HoLogoHtml@biber}
%    \begin{macrocode}
\let\HoLogoHtml@biber\HoLogo@biber
%    \end{macrocode}
%    \end{macro}
%
% \subsubsection{\hologo{KOMAScript}}
%
%    \begin{macro}{\HoLogo@KOMAScript}
%    The definition for \hologo{KOMAScript} is taken
%    from \hologo{KOMAScript} (\xfile{scrlogo.dtx}, reformatted) \cite{scrlogo}:
%\begin{quote}
%\begin{verbatim}
%\@ifundefined{KOMAScript}{%
%  \DeclareRobustCommand{\KOMAScript}{%
%    \textsf{%
%      K\kern.05em O\kern.05emM\kern.05em A%
%      \kern.1em-\kern.1em %
%      Script%
%    }%
%  }%
%}{}
%\end{verbatim}
%\end{quote}
%    \begin{macrocode}
\def\HoLogo@KOMAScript#1{%
  \HoLogoFont@font{KOMAScript}{sf}{%
    \HOLOGO@mbox{%
      K\kern.05em%
      O\kern.05em%
      M\kern.05em%
      A%
    }%
    \kern.1em%
    \HOLOGO@hyphen
    \kern.1em%
    \HOLOGO@mbox{Script}%
  }%
}
%    \end{macrocode}
%    \end{macro}
%    \begin{macro}{\HoLogoBkm@KOMAScript}
%    \begin{macrocode}
\def\HoLogoBkm@KOMAScript#1{%
  KOMA-Script%
}
%    \end{macrocode}
%    \end{macro}
%    \begin{macro}{\HoLogoHtml@KOMAScript}
%    \begin{macrocode}
\def\HoLogoHtml@KOMAScript#1{%
  \HoLogoCss@KOMAScript
  \HoLogoFont@font{KOMAScript}{sf}{%
    \HOLOGO@Span{KOMAScript}{%
      K%
      \HOLOGO@Span{O}{O}%
      M%
      \HOLOGO@Span{A}{A}%
      \HOLOGO@Span{hyphen}{-}%
      Script%
    }%
  }%
}
%    \end{macrocode}
%    \end{macro}
%    \begin{macro}{\HoLogoCss@KOMAScript}
%    \begin{macrocode}
\def\HoLogoCss@KOMAScript{%
  \Css{%
    span.HoLogo-KOMAScript{%
      font-family:sans-serif;%
    }%
  }%
  \Css{%
    span.HoLogo-KOMAScript span.HoLogo-O{%
      padding-left:.05em;%
      padding-right:.05em;%
    }%
  }%
  \Css{%
    span.HoLogo-KOMAScript span.HoLogo-A{%
      padding-left:.05em;%
    }%
  }%
  \Css{%
    span.HoLogo-KOMAScript span.HoLogo-hyphen{%
      padding-left:.1em;%
      padding-right:.1em;%
    }%
  }%
  \global\let\HoLogoCss@KOMAScript\relax
}
%    \end{macrocode}
%    \end{macro}
%
% \subsubsection{\hologo{LyX}}
%
%    \begin{macro}{\HoLogo@LyX}
%    The definition is taken from the documentation source files
%    of \hologo{LyX}, \xfile{Intro.lyx} \cite{LyX}:
%\begin{quote}
%\begin{verbatim}
%\def\LyX{%
%  \texorpdfstring{%
%    L\kern-.1667em\lower.25em\hbox{Y}\kern-.125emX\@%
%  }{%
%    LyX%
%  }%
%}
%\end{verbatim}
%\end{quote}
%    \begin{macrocode}
\def\HoLogo@LyX#1{%
  L%
  \kern-.1667em%
  \lower.25em\hbox{Y}%
  \kern-.125em%
  X%
  \HOLOGO@SpaceFactor
}
%    \end{macrocode}
%    \end{macro}
%    \begin{macro}{\HoLogoHtml@LyX}
%    \begin{macrocode}
\def\HoLogoHtml@LyX#1{%
  \HoLogoCss@LyX
  \HOLOGO@Span{LyX}{%
    L%
    \HOLOGO@Span{y}{Y}%
    X%
  }%
}
%    \end{macrocode}
%    \end{macro}
%    \begin{macro}{\HoLogoCss@LyX}
%    \begin{macrocode}
\def\HoLogoCss@LyX{%
  \Css{%
    span.HoLogo-LyX span.HoLogo-y{%
      position:relative;%
      top:.25em;%
      margin-left:-.1667em;%
      margin-right:-.125em;%
      text-decoration:none;%
    }%
  }%
  \global\let\HoLogoCss@LyX\relax
}
%    \end{macrocode}
%    \end{macro}
%
% \subsubsection{\hologo{NTS}}
%
%    \begin{macro}{\HoLogo@NTS}
%    Definition for \hologo{NTS} can be found in
%    package \xpackage{etex\textunderscore man} for the \hologo{eTeX} manual \cite{etexman}
%    and in package \xpackage{dtklogos} \cite{dtklogos}:
%\begin{quote}
%\begin{verbatim}
%\def\NTS{%
%  \leavevmode
%  \hbox{%
%    $%
%      \cal N%
%      \kern-0.35em%
%      \lower0.5ex\hbox{$\cal T$}%
%      \kern-0.2em%
%      S%
%    $%
%  }%
%}
%\end{verbatim}
%\end{quote}
%    \begin{macrocode}
\def\HoLogo@NTS#1{%
  \HoLogoFont@font{NTS}{sy}{%
    N\/%
    \kern-.35em%
    \lower.5ex\hbox{T\/}%
    \kern-.2em%
    S\/%
  }%
  \HOLOGO@SpaceFactor
}
%    \end{macrocode}
%    \end{macro}
%
% \subsubsection{\Hologo{TTH} (\hologo{TeX} to HTML translator)}
%
%    Source: \url{http://hutchinson.belmont.ma.us/tth/}
%    In the HTML source the second `T' is printed as subscript.
%\begin{quote}
%\begin{verbatim}
%T<sub>T</sub>H
%\end{verbatim}
%\end{quote}
%    \begin{macro}{\HoLogo@TTH}
%    \begin{macrocode}
\def\HoLogo@TTH#1{%
  \ltx@mbox{%
    T\HOLOGO@SubScript{T}H%
  }%
  \HOLOGO@SpaceFactor
}
%    \end{macrocode}
%    \end{macro}
%
%    \begin{macro}{\HoLogoHtml@TTH}
%    \begin{macrocode}
\def\HoLogoHtml@TTH#1{%
  T\HCode{<sub>}T\HCode{</sub>}H%
}
%    \end{macrocode}
%    \end{macro}
%
% \subsubsection{\Hologo{HanTheThanh}}
%
%    Partial source: Package \xpackage{dtklogos}.
%    The double accent is U+1EBF (latin small letter e with circumflex
%    and acute).
%    \begin{macro}{\HoLogo@HanTheThanh}
%    \begin{macrocode}
\def\HoLogo@HanTheThanh#1{%
  \ltx@mbox{H\`an}%
  \HOLOGO@space
  \ltx@mbox{%
    Th%
    \HOLOGO@IfCharExists{"1EBF}{%
      \char"1EBF\relax
    }{%
      \^e\hbox to 0pt{\hss\raise .5ex\hbox{\'{}}}%
    }%
  }%
  \HOLOGO@space
  \ltx@mbox{Th\`anh}%
}
%    \end{macrocode}
%    \end{macro}
%    \begin{macro}{\HoLogoBkm@HanTheThanh}
%    \begin{macrocode}
\def\HoLogoBkm@HanTheThanh#1{%
  H\`an %
  Th\HOLOGO@PdfdocUnicode{\^e}{\9036\277} %
  Th\`anh%
}
%    \end{macrocode}
%    \end{macro}
%    \begin{macro}{\HoLogoHtml@HanTheThanh}
%    \begin{macrocode}
\def\HoLogoHtml@HanTheThanh#1{%
  H\`an %
  Th\HCode{&\ltx@hashchar x1ebf;} %
  Th\`anh%
}
%    \end{macrocode}
%    \end{macro}
%
% \subsection{Driver detection}
%
%    \begin{macrocode}
\HOLOGO@IfExists\InputIfFileExists{%
  \InputIfFileExists{hologo.cfg}{}{}%
}{%
  \ltx@IfUndefined{pdf@filesize}{%
    \def\HOLOGO@InputIfExists{%
      \openin\HOLOGO@temp=hologo.cfg\relax
      \ifeof\HOLOGO@temp
        \closein\HOLOGO@temp
      \else
        \closein\HOLOGO@temp
        \begingroup
          \def\x{LaTeX2e}%
        \expandafter\endgroup
        \ifx\fmtname\x
          \input{hologo.cfg}%
        \else
          \input hologo.cfg\relax
        \fi
      \fi
    }%
    \ltx@IfUndefined{newread}{%
      \chardef\HOLOGO@temp=15 %
      \def\HOLOGO@CheckRead{%
        \ifeof\HOLOGO@temp
          \HOLOGO@InputIfExists
        \else
          \ifcase\HOLOGO@temp
            \@PackageWarningNoLine{hologo}{%
              Configuration file ignored, because\MessageBreak
              a free read register could not be found%
            }%
          \else
            \begingroup
              \count\ltx@cclv=\HOLOGO@temp
              \advance\ltx@cclv by \ltx@minusone
              \edef\x{\endgroup
                \chardef\noexpand\HOLOGO@temp=\the\count\ltx@cclv
                \relax
              }%
            \x
          \fi
        \fi
      }%
    }{%
      \csname newread\endcsname\HOLOGO@temp
      \HOLOGO@InputIfExists
    }%
  }{%
    \edef\HOLOGO@temp{\pdf@filesize{hologo.cfg}}%
    \ifx\HOLOGO@temp\ltx@empty
    \else
      \ifnum\HOLOGO@temp>0 %
        \begingroup
          \def\x{LaTeX2e}%
        \expandafter\endgroup
        \ifx\fmtname\x
          \input{hologo.cfg}%
        \else
          \input hologo.cfg\relax
        \fi
      \else
        \@PackageInfoNoLine{hologo}{%
          Empty configuration file `hologo.cfg' ignored%
        }%
      \fi
    \fi
  }%
}
%    \end{macrocode}
%
%    \begin{macrocode}
\def\HOLOGO@temp#1#2{%
  \kv@define@key{HoLogoDriver}{#1}[]{%
    \begingroup
      \def\HOLOGO@temp{##1}%
      \ltx@onelevel@sanitize\HOLOGO@temp
      \ifx\HOLOGO@temp\ltx@empty
      \else
        \@PackageError{hologo}{%
          Value (\HOLOGO@temp) not permitted for option `#1'%
        }%
        \@ehc
      \fi
    \endgroup
    \def\hologoDriver{#2}%
  }%
}%
\def\HOLOGO@@temp#1#2{%
  \ifx\kv@value\relax
    \HOLOGO@temp{#1}{#1}%
  \else
    \HOLOGO@temp{#1}{#2}%
  \fi
}%
\kv@parse@normalized{%
  pdftex,%
  luatex=pdftex,%
  dvipdfm,%
  dvipdfmx=dvipdfm,%
  dvips,%
  dvipsone=dvips,%
  xdvi=dvips,%
  xetex,%
  vtex,%
}\HOLOGO@@temp
%    \end{macrocode}
%
%    \begin{macrocode}
\kv@define@key{HoLogoDriver}{driverfallback}{%
  \def\HOLOGO@DriverFallback{#1}%
}
%    \end{macrocode}
%
%    \begin{macro}{\HOLOGO@DriverFallback}
%    \begin{macrocode}
\def\HOLOGO@DriverFallback{dvips}
%    \end{macrocode}
%    \end{macro}
%
%    \begin{macro}{\hologoDriverSetup}
%    \begin{macrocode}
\def\hologoDriverSetup{%
  \let\hologoDriver\ltx@undefined
  \HOLOGO@DriverSetup
}
%    \end{macrocode}
%    \end{macro}
%
%    \begin{macro}{\HOLOGO@DriverSetup}
%    \begin{macrocode}
\def\HOLOGO@DriverSetup#1{%
  \kvsetkeys{HoLogoDriver}{#1}%
  \HOLOGO@CheckDriver
  \ltx@ifundefined{hologoDriver}{%
    \begingroup
    \edef\x{\endgroup
      \noexpand\kvsetkeys{HoLogoDriver}{\HOLOGO@DriverFallback}%
    }\x
  }{}%
  \@PackageInfoNoLine{hologo}{Using driver `\hologoDriver'}%
}
%    \end{macrocode}
%    \end{macro}
%
%    \begin{macro}{\HOLOGO@CheckDriver}
%    \begin{macrocode}
\def\HOLOGO@CheckDriver{%
  \ifpdf
    \def\hologoDriver{pdftex}%
    \let\HOLOGO@pdfliteral\pdfliteral
    \ifluatex
      \ifx\pdfextension\@undefined\else
        \protected\def\pdfliteral{\pdfextension literal}%
        \let\HOLOGO@pdfliteral\pdfliteral
      \fi
      \ltx@IfUndefined{HOLOGO@pdfliteral}{%
        \ifnum\luatexversion<36 %
        \else
          \begingroup
            \let\HOLOGO@temp\endgroup
            \ifcase0%
                \directlua{%
                  if tex.enableprimitives then %
                    tex.enableprimitives('HOLOGO@', {'pdfliteral'})%
                  else %
                    tex.print('1')%
                  end%
                }%
                \ifx\HOLOGO@pdfliteral\@undefined 1\fi%
                \relax%
              \endgroup
              \let\HOLOGO@temp\relax
              \global\let\HOLOGO@pdfliteral\HOLOGO@pdfliteral
            \fi%
          \HOLOGO@temp
        \fi
      }{}%
    \fi
    \ltx@IfUndefined{HOLOGO@pdfliteral}{%
      \@PackageWarningNoLine{hologo}{%
        Cannot find \string\pdfliteral
      }%
    }{}%
  \else
    \ifxetex
      \def\hologoDriver{xetex}%
    \else
      \ifvtex
        \def\hologoDriver{vtex}%
      \fi
    \fi
  \fi
}
%    \end{macrocode}
%    \end{macro}
%
%    \begin{macro}{\HOLOGO@WarningUnsupportedDriver}
%    \begin{macrocode}
\def\HOLOGO@WarningUnsupportedDriver#1{%
  \@PackageWarningNoLine{hologo}{%
    Logo `#1' needs driver specific macros,\MessageBreak
    but driver `\hologoDriver' is not supported.\MessageBreak
    Use a different driver or\MessageBreak
    load package `graphics' or `pgf'%
  }%
}
%    \end{macrocode}
%    \end{macro}
%
% \subsubsection{Reflect box macros}
%
%    Skip driver part if not needed.
%    \begin{macrocode}
\ltx@IfUndefined{reflectbox}{}{%
  \ltx@IfUndefined{rotatebox}{}{%
    \HOLOGO@AtEnd
  }%
}
\ltx@IfUndefined{pgftext}{}{%
  \HOLOGO@AtEnd
}
\ltx@IfUndefined{psscalebox}{}{%
  \HOLOGO@AtEnd
}
%    \end{macrocode}
%
%    \begin{macrocode}
\def\HOLOGO@temp{LaTeX2e}
\ifx\fmtname\HOLOGO@temp
  \RequirePackage{kvoptions}[2011/06/30]%
  \ProcessKeyvalOptions{HoLogoDriver}%
\fi
\HOLOGO@DriverSetup{}
%    \end{macrocode}
%
%    \begin{macro}{\HOLOGO@ReflectBox}
%    \begin{macrocode}
\def\HOLOGO@ReflectBox#1{%
  \begingroup
    \setbox\ltx@zero\hbox{\begingroup#1\endgroup}%
    \setbox\ltx@two\hbox{%
      \kern\wd\ltx@zero
      \csname HOLOGO@ScaleBox@\hologoDriver\endcsname{-1}{1}{%
        \hbox to 0pt{\copy\ltx@zero\hss}%
      }%
    }%
    \wd\ltx@two=\wd\ltx@zero
    \box\ltx@two
  \endgroup
}
%    \end{macrocode}
%    \end{macro}
%
%    \begin{macro}{\HOLOGO@PointReflectBox}
%    \begin{macrocode}
\def\HOLOGO@PointReflectBox#1{%
  \begingroup
    \setbox\ltx@zero\hbox{\begingroup#1\endgroup}%
    \setbox\ltx@two\hbox{%
      \kern\wd\ltx@zero
      \raise\ht\ltx@zero\hbox{%
        \csname HOLOGO@ScaleBox@\hologoDriver\endcsname{-1}{-1}{%
          \hbox to 0pt{\copy\ltx@zero\hss}%
        }%
      }%
    }%
    \wd\ltx@two=\wd\ltx@zero
    \box\ltx@two
  \endgroup
}
%    \end{macrocode}
%    \end{macro}
%
%    We must define all variants because of dynamic driver setup.
%    \begin{macrocode}
\def\HOLOGO@temp#1#2{#2}
%    \end{macrocode}
%
%    \begin{macro}{\HOLOGO@ScaleBox@pdftex}
%    \begin{macrocode}
\HOLOGO@temp{pdftex}{%
  \def\HOLOGO@ScaleBox@pdftex#1#2#3{%
    \HOLOGO@pdfliteral{%
      q #1 0 0 #2 0 0 cm%
    }%
    #3%
    \HOLOGO@pdfliteral{%
      Q%
    }%
  }%
}
%    \end{macrocode}
%    \end{macro}
%    \begin{macro}{\HOLOGO@ScaleBox@dvips}
%    \begin{macrocode}
\HOLOGO@temp{dvips}{%
  \def\HOLOGO@ScaleBox@dvips#1#2#3{%
    \special{ps:%
      gsave %
      currentpoint %
      currentpoint translate %
      #1 #2 scale %
      neg exch neg exch translate%
    }%
    #3%
    \special{ps:%
      currentpoint %
      grestore %
      moveto%
    }%
  }%
}
%    \end{macrocode}
%    \end{macro}
%    \begin{macro}{\HOLOGO@ScaleBox@dvipdfm}
%    \begin{macrocode}
\HOLOGO@temp{dvipdfm}{%
  \let\HOLOGO@ScaleBox@dvipdfm\HOLOGO@ScaleBox@dvips
}
%    \end{macrocode}
%    \end{macro}
%    Since \hologo{XeTeX} v0.6.
%    \begin{macro}{\HOLOGO@ScaleBox@xetex}
%    \begin{macrocode}
\HOLOGO@temp{xetex}{%
  \def\HOLOGO@ScaleBox@xetex#1#2#3{%
    \special{x:gsave}%
    \special{x:scale #1 #2}%
    #3%
    \special{x:grestore}%
  }%
}
%    \end{macrocode}
%    \end{macro}
%    \begin{macro}{\HOLOGO@ScaleBox@vtex}
%    \begin{macrocode}
\HOLOGO@temp{vtex}{%
  \def\HOLOGO@ScaleBox@vtex#1#2#3{%
    \special{r(#1,0,0,#2,0,0}%
    #3%
    \special{r)}%
  }%
}
%    \end{macrocode}
%    \end{macro}
%
%    \begin{macrocode}
\HOLOGO@AtEnd%
%</package>
%    \end{macrocode}
%
% \section{Test}
%
% \subsection{Catcode checks for loading}
%
%    \begin{macrocode}
%<*test1>
%    \end{macrocode}
%    \begin{macrocode}
\catcode`\{=1 %
\catcode`\}=2 %
\catcode`\#=6 %
\catcode`\@=11 %
\expandafter\ifx\csname count@\endcsname\relax
  \countdef\count@=255 %
\fi
\expandafter\ifx\csname @gobble\endcsname\relax
  \long\def\@gobble#1{}%
\fi
\expandafter\ifx\csname @firstofone\endcsname\relax
  \long\def\@firstofone#1{#1}%
\fi
\expandafter\ifx\csname loop\endcsname\relax
  \expandafter\@firstofone
\else
  \expandafter\@gobble
\fi
{%
  \def\loop#1\repeat{%
    \def\body{#1}%
    \iterate
  }%
  \def\iterate{%
    \body
      \let\next\iterate
    \else
      \let\next\relax
    \fi
    \next
  }%
  \let\repeat=\fi
}%
\def\RestoreCatcodes{}
\count@=0 %
\loop
  \edef\RestoreCatcodes{%
    \RestoreCatcodes
    \catcode\the\count@=\the\catcode\count@\relax
  }%
\ifnum\count@<255 %
  \advance\count@ 1 %
\repeat

\def\RangeCatcodeInvalid#1#2{%
  \count@=#1\relax
  \loop
    \catcode\count@=15 %
  \ifnum\count@<#2\relax
    \advance\count@ 1 %
  \repeat
}
\def\RangeCatcodeCheck#1#2#3{%
  \count@=#1\relax
  \loop
    \ifnum#3=\catcode\count@
    \else
      \errmessage{%
        Character \the\count@\space
        with wrong catcode \the\catcode\count@\space
        instead of \number#3%
      }%
    \fi
  \ifnum\count@<#2\relax
    \advance\count@ 1 %
  \repeat
}
\def\space{ }
\expandafter\ifx\csname LoadCommand\endcsname\relax
  \def\LoadCommand{\input hologo.sty\relax}%
\fi
\def\Test{%
  \RangeCatcodeInvalid{0}{47}%
  \RangeCatcodeInvalid{58}{64}%
  \RangeCatcodeInvalid{91}{96}%
  \RangeCatcodeInvalid{123}{255}%
  \catcode`\@=12 %
  \catcode`\\=0 %
  \catcode`\%=14 %
  \LoadCommand
  \RangeCatcodeCheck{0}{36}{15}%
  \RangeCatcodeCheck{37}{37}{14}%
  \RangeCatcodeCheck{38}{47}{15}%
  \RangeCatcodeCheck{48}{57}{12}%
  \RangeCatcodeCheck{58}{63}{15}%
  \RangeCatcodeCheck{64}{64}{12}%
  \RangeCatcodeCheck{65}{90}{11}%
  \RangeCatcodeCheck{91}{91}{15}%
  \RangeCatcodeCheck{92}{92}{0}%
  \RangeCatcodeCheck{93}{96}{15}%
  \RangeCatcodeCheck{97}{122}{11}%
  \RangeCatcodeCheck{123}{255}{15}%
  \RestoreCatcodes
}
\Test
\csname @@end\endcsname
\end
%    \end{macrocode}
%    \begin{macrocode}
%</test1>
%    \end{macrocode}
%
% \subsection{Spacefactor}
%
%    The space factor must be 1000 after a logo. If it is greater 1000
%    then the following space is a space after a sentence closing point.
%    If the space factor is smaller 1000 then an immediate following
%    dot is interpreted as abbreviation, not sentence closing point.
%
%    \begin{macrocode}
%<*test-spacefactor>
\NeedsTeXFormat{LaTeX2e}
\documentclass{article}
\usepackage{hologo}[2016/05/12]
\usepackage{kvsetkeys}
\usepackage{qstest}
\IncludeTests{*}
\LogTests{log}{*}{*}
\begin{document}
\begin{qstest}{spacefactor}{spacefactor}
\newcommand*{\Test}[1]{%
  \sbox0{%
    \hologo{#1}%
    \Expect*{1000 (#1)}*{\the\spacefactor\space(#1)}%
  }%
}%
\makeatletter
\def\TestList{}
\def\hologoEntry#1#2#3{%
  \edef\TestList{%
    \ifx\TestList\@empty
    \else
      \TestList,%
    \fi
    #1%
    \ifx\\#2\\%
    \else
      ={variant=#2}%
    \fi
  }%
}
\hologoList
\expandafter\kv@parse@normalized\expandafter{%
  \TestList
}{%
  \begingroup
    \let\@logo=\kv@key
    \ifx\kv@value\relax
    \else
      \expandafter\hologoLogoSetup\expandafter\@logo\expandafter{%
        \kv@value
      }%
    \fi
    \Test\@logo
  \endgroup
  \@gobbletwo
}
\end{qstest}
\end{document}
%</test-spacefactor>
%    \end{macrocode}
%
% \subsection{Complete list}
%
%    \begin{macrocode}
%<*test-list>
\NeedsTeXFormat{LaTeX2e}
\documentclass[12pt,a4paper]{article}
\usepackage{hologo}[2016/05/12]
\usepackage[T1]{fontenc}
\usepackage{lmodern}
\usepackage{parskip}
\usepackage[unicode]{hyperref}[2011/09/28]
\usepackage{bookmark}[2011/09/19]
\bookmarksetup{%
  numbered,%
  open,%
  openlevel=2,%
}
\renewcommand*{\contentsname}{List of logos}
\begin{document}
\tableofcontents
\def\TestFont#1#2#3#4#5#6{%
  \begingroup
    \usefont{#3}{#4}{#5}{#6}%
    \HologoVariant{#1}{#2}/\hologoVariant{#1}{#2}%
    \quad
    \begingroup\scriptsize\hologoVariant{#1}{#2}\endgroup
    \quad
  \endgroup
  (#3/#4/#5/#6)%
  \par
}
\makeatletter
\def\hologoEntry#1#2#3{%
  \section{%
    \HologoVariant{#1}{#2}/\hologoVariant{#1}{#2} %
    {[#1\ifx\\#2\\\else\space(#2)\fi]}% hash-ok
  }% braces around [] because of bug in tex4ht
  \begingroup
    \hypersetup{unicode=false}%
    \bookmark[%
      dest=\@currentHref,%
      rellevel=1,%
      keeplevel,%
    ]{%
      \HologoVariant{#1}{#2}/\hologoVariant{#1}{#2} %
      (PDFDocEncoding)%
    }%
  \endgroup
  \TestFont{#1}{#2}{OT1}{cmr}{m}{n}%
  \TestFont{#1}{#2}{OT1}{cmss}{m}{n}%
  \TestFont{#1}{#2}{OT1}{cmr}{b}{n}%
  \TestFont{#1}{#2}{OT1}{cmr}{m}{it}%
  \TestFont{#1}{#2}{OT1}{cmtt}{m}{n}%
  \TestFont{#1}{#2}{T1}{lmr}{m}{n}%
  \TestFont{#1}{#2}{T1}{lmss}{m}{n}%
  \TestFont{#1}{#2}{T1}{lmr}{b}{n}%
  \TestFont{#1}{#2}{T1}{lmr}{m}{it}%
  \TestFont{#1}{#2}{T1}{lmtt}{m}{n}%
  \TestFont{#1}{#2}{T1}{lmvtt}{m}{n}%
  \TestFont{#1}{#2}{T1}{qtm}{m}{n}%
  \TestFont{#1}{#2}{T1}{qhv}{m}{n}%
  \TestFont{#1}{#2}{T1}{qtm}{b}{n}%
  \TestFont{#1}{#2}{T1}{qtm}{m}{it}%
  \TestFont{#1}{#2}{T1}{qcr}{m}{n}%
  \newpage
}
\makeatother
\hologoList
\end{document}
%</test-list>
%    \end{macrocode}
%
% \section{Installation}
%
% \subsection{Download}
%
% \paragraph{Package.} This package is available on
% CTAN\footnote{\url{ftp://ftp.ctan.org/tex-archive/}}:
% \begin{description}
% \item[\CTAN{macros/latex/contrib/oberdiek/hologo.dtx}] The source file.
% \item[\CTAN{macros/latex/contrib/oberdiek/hologo.pdf}] Documentation.
% \end{description}
%
%
% \paragraph{Bundle.} All the packages of the bundle `oberdiek'
% are also available in a TDS compliant ZIP archive. There
% the packages are already unpacked and the documentation files
% are generated. The files and directories obey the TDS standard.
% \begin{description}
% \item[\CTAN{install/macros/latex/contrib/oberdiek.tds.zip}]
% \end{description}
% \emph{TDS} refers to the standard ``A Directory Structure
% for \TeX\ Files'' (\CTAN{tds/tds.pdf}). Directories
% with \xfile{texmf} in their name are usually organized this way.
%
% \subsection{Bundle installation}
%
% \paragraph{Unpacking.} Unpack the \xfile{oberdiek.tds.zip} in the
% TDS tree (also known as \xfile{texmf} tree) of your choice.
% Example (linux):
% \begin{quote}
%   |unzip oberdiek.tds.zip -d ~/texmf|
% \end{quote}
%
% \paragraph{Script installation.}
% Check the directory \xfile{TDS:scripts/oberdiek/} for
% scripts that need further installation steps.
% Package \xpackage{attachfile2} comes with the Perl script
% \xfile{pdfatfi.pl} that should be installed in such a way
% that it can be called as \texttt{pdfatfi}.
% Example (linux):
% \begin{quote}
%   |chmod +x scripts/oberdiek/pdfatfi.pl|\\
%   |cp scripts/oberdiek/pdfatfi.pl /usr/local/bin/|
% \end{quote}
%
% \subsection{Package installation}
%
% \paragraph{Unpacking.} The \xfile{.dtx} file is a self-extracting
% \docstrip\ archive. The files are extracted by running the
% \xfile{.dtx} through \plainTeX:
% \begin{quote}
%   \verb|tex hologo.dtx|
% \end{quote}
%
% \paragraph{TDS.} Now the different files must be moved into
% the different directories in your installation TDS tree
% (also known as \xfile{texmf} tree):
% \begin{quote}
% \def\t{^^A
% \begin{tabular}{@{}>{\ttfamily}l@{ $\rightarrow$ }>{\ttfamily}l@{}}
%   hologo.sty & tex/generic/oberdiek/hologo.sty\\
%   hologo.pdf & doc/latex/oberdiek/hologo.pdf\\
%   example/hologo-example.tex & doc/latex/oberdiek/example/hologo-example.tex\\
%   test/hologo-test1.tex & doc/latex/oberdiek/test/hologo-test1.tex\\
%   test/hologo-test-spacefactor.tex & doc/latex/oberdiek/test/hologo-test-spacefactor.tex\\
%   test/hologo-test-list.tex & doc/latex/oberdiek/test/hologo-test-list.tex\\
%   hologo.dtx & source/latex/oberdiek/hologo.dtx\\
% \end{tabular}^^A
% }^^A
% \sbox0{\t}^^A
% \ifdim\wd0>\linewidth
%   \begingroup
%     \advance\linewidth by\leftmargin
%     \advance\linewidth by\rightmargin
%   \edef\x{\endgroup
%     \def\noexpand\lw{\the\linewidth}^^A
%   }\x
%   \def\lwbox{^^A
%     \leavevmode
%     \hbox to \linewidth{^^A
%       \kern-\leftmargin\relax
%       \hss
%       \usebox0
%       \hss
%       \kern-\rightmargin\relax
%     }^^A
%   }^^A
%   \ifdim\wd0>\lw
%     \sbox0{\small\t}^^A
%     \ifdim\wd0>\linewidth
%       \ifdim\wd0>\lw
%         \sbox0{\footnotesize\t}^^A
%         \ifdim\wd0>\linewidth
%           \ifdim\wd0>\lw
%             \sbox0{\scriptsize\t}^^A
%             \ifdim\wd0>\linewidth
%               \ifdim\wd0>\lw
%                 \sbox0{\tiny\t}^^A
%                 \ifdim\wd0>\linewidth
%                   \lwbox
%                 \else
%                   \usebox0
%                 \fi
%               \else
%                 \lwbox
%               \fi
%             \else
%               \usebox0
%             \fi
%           \else
%             \lwbox
%           \fi
%         \else
%           \usebox0
%         \fi
%       \else
%         \lwbox
%       \fi
%     \else
%       \usebox0
%     \fi
%   \else
%     \lwbox
%   \fi
% \else
%   \usebox0
% \fi
% \end{quote}
% If you have a \xfile{docstrip.cfg} that configures and enables \docstrip's
% TDS installing feature, then some files can already be in the right
% place, see the documentation of \docstrip.
%
% \subsection{Refresh file name databases}
%
% If your \TeX~distribution
% (\teTeX, \mikTeX, \dots) relies on file name databases, you must refresh
% these. For example, \teTeX\ users run \verb|texhash| or
% \verb|mktexlsr|.
%
% \subsection{Some details for the interested}
%
% \paragraph{Attached source.}
%
% The PDF documentation on CTAN also includes the
% \xfile{.dtx} source file. It can be extracted by
% AcrobatReader 6 or higher. Another option is \textsf{pdftk},
% e.g. unpack the file into the current directory:
% \begin{quote}
%   \verb|pdftk hologo.pdf unpack_files output .|
% \end{quote}
%
% \paragraph{Unpacking with \LaTeX.}
% The \xfile{.dtx} chooses its action depending on the format:
% \begin{description}
% \item[\plainTeX:] Run \docstrip\ and extract the files.
% \item[\LaTeX:] Generate the documentation.
% \end{description}
% If you insist on using \LaTeX\ for \docstrip\ (really,
% \docstrip\ does not need \LaTeX), then inform the autodetect routine
% about your intention:
% \begin{quote}
%   \verb|latex \let\install=y\input{hologo.dtx}|
% \end{quote}
% Do not forget to quote the argument according to the demands
% of your shell.
%
% \paragraph{Generating the documentation.}
% You can use both the \xfile{.dtx} or the \xfile{.drv} to generate
% the documentation. The process can be configured by the
% configuration file \xfile{ltxdoc.cfg}. For instance, put this
% line into this file, if you want to have A4 as paper format:
% \begin{quote}
%   \verb|\PassOptionsToClass{a4paper}{article}|
% \end{quote}
% An example follows how to generate the
% documentation with pdf\LaTeX:
% \begin{quote}
%\begin{verbatim}
%pdflatex hologo.dtx
%makeindex -s gind.ist hologo.idx
%pdflatex hologo.dtx
%makeindex -s gind.ist hologo.idx
%pdflatex hologo.dtx
%\end{verbatim}
% \end{quote}
%
% \section{Catalogue}
%
% The following XML file can be used as source for the
% \href{http://mirror.ctan.org/help/Catalogue/catalogue.html}{\TeX\ Catalogue}.
% The elements \texttt{caption} and \texttt{description} are imported
% from the original XML file from the Catalogue.
% The name of the XML file in the Catalogue is \xfile{hologo.xml}.
%    \begin{macrocode}
%<*catalogue>
<?xml version='1.0' encoding='us-ascii'?>
<!DOCTYPE entry SYSTEM 'catalogue.dtd'>
<entry datestamp='$Date$' modifier='$Author$' id='hologo'>
  <name>hologo</name>
  <caption>A collection of logos with bookmark support.</caption>
  <authorref id='auth:oberdiek'/>
  <copyright owner='Heiko Oberdiek' year='2010-2012'/>
  <license type='lppl1.3'/>
  <version number='1.10'/>
  <description>
    The package defines a single command <tt>\hologo</tt>, whose
    argument is the usual case-confused ASCII version of the logo.
    The command is bookmark-enabled, so that every logo becomes
    available in bookmarks without further work.
    <p/>
    The package is part of the <xref refid='oberdiek'>oberdiek</xref>
    bundle.
  </description>
  <documentation details='Package documentation'
      href='ctan:/macros/latex/contrib/oberdiek/hologo.pdf'/>
  <ctan file='true' path='/macros/latex/contrib/oberdiek/hologo.dtx'/>
  <miktex location='oberdiek'/>
  <texlive location='oberdiek'/>
  <install path='/macros/latex/contrib/oberdiek/oberdiek.tds.zip'/>
</entry>
%</catalogue>
%    \end{macrocode}
%
% \begin{thebibliography}{9}
% \raggedright
%
% \bibitem{btxdoc}
% Oren Patashnik,
% \textit{\hologo{BibTeX}ing},
% 1988-02-08.\\
% \CTAN{biblio/bibtex/base/}
%
% \bibitem{dtklogos}
% Gerd Neugebauer, DANTE,
% \textit{Package \xpackage{dtklogos}},
% 2011-04-25.\\
% \CTAN{usergrps/dante/dtk/dtklogos.sty}
%
% \bibitem{etexman}
% The \hologo{NTS} Team,
% \textit{The \hologo{eTeX} manual},
% 1998-02.\\
% \CTAN{systems/e-tex/v2/doc/}
%
% \bibitem{ExTeX-FAQ}
% The \hologo{ExTeX} group,
% \textit{\hologo{ExTeX}: FAQ -- How is \hologo{ExTeX} typeset?},
% 2007-04-14.\\
% \url{http://www.extex.org/documentation/faq.html}
%
% \bibitem{LyX}
% %@MISC{ LyX,
% %  title = {{LyX 2.0.0 -- The Document Processor [Computer software and manual]}},
% %  author = {{The LyX Team}},
% %  howpublished = {Internet: http://www.lyx.org},
% %  year = {2011-05-08},
% %  note = {Retrieved May 10, 2011, from http://www.lyx.org},
% %  url = {http://www.lyx.org/}
% %}
% The \hologo{LyX} Team,
% \textit{\hologo{LyX} -- The Document Processor},
% 2011-05-08.\\
% \url{http://www.lyx.org/}
%
% \bibitem{OzTeX}
% Andrew Trevorrow,
% \hologo{OzTeX} FAQ: What is the correct way to typeset ``\hologo{OzTeX}''?,
% 2011-09-15 (visited).
% \url{http://www.trevorrow.com/oztex/ozfaq.html#oztex-logo}
%
% \bibitem{PiCTeX}
% Michael Wichura,
% \textit{The \hologo{PiCTeX} macro package},
% 1987-09-21.
% \CTAN{graphics/pictex/}
%
% \bibitem{scrlogo}
% Markus Kohm,
% \textit{\hologo{KOMAScript} Datei \xfile{scrlogo.dtx}},
% 2009-01-30.\\
% \CTAN{install/macros/latex/contrib/komascript.tds.zip}
%
% \end{thebibliography}
%
% \begin{History}
%   \begin{Version}{2010/04/08 v1.0}
%   \item
%     The first version.
%   \end{Version}
%   \begin{Version}{2010/04/16 v1.1}
%   \item
%     \cs{Hologo} added for support of logos at start of a sentence.
%   \item
%     \cs{hologoSetup} and \cs{hologoLogoSetup} added.
%   \item
%     Options \xoption{break}, \xoption{hyphenbreak}, \xoption{spacebreak}
%     added.
%   \item
%     Variant support added by option \xoption{variant}.
%   \end{Version}
%   \begin{Version}{2010/04/24 v1.2}
%   \item
%     \hologo{LaTeX3} added.
%   \item
%     \hologo{VTeX} added.
%   \end{Version}
%   \begin{Version}{2010/11/21 v1.3}
%   \item
%     \hologo{iniTeX}, \hologo{virTeX} added.
%   \end{Version}
%   \begin{Version}{2011/03/25 v1.4}
%   \item
%     \hologo{ConTeXt} with variants added.
%   \item
%     Option \xoption{discretionarybreak} added as refinement for
%     option \xoption{break}.
%   \end{Version}
%   \begin{Version}{2011/04/21 v1.5}
%   \item
%     Wrong TDS directory for test files fixed.
%   \end{Version}
%   \begin{Version}{2011/10/01 v1.6}
%   \item
%     Support for package \xpackage{tex4ht} added.
%   \item
%     Support for \cs{csname} added if \cs{ifincsname} is available.
%   \item
%     New logos:
%     \hologo{(La)TeX},
%     \hologo{biber},
%     \hologo{BibTeX} (\xoption{sc}, \xoption{sf}),
%     \hologo{emTeX},
%     \hologo{ExTeX},
%     \hologo{KOMAScript},
%     \hologo{La},
%     \hologo{LyX},
%     \hologo{MiKTeX},
%     \hologo{NTS},
%     \hologo{OzMF},
%     \hologo{OzMP},
%     \hologo{OzTeX},
%     \hologo{OzTtH},
%     \hologo{PCTeX},
%     \hologo{PiC},
%     \hologo{PiCTeX},
%     \hologo{METAFONT},
%     \hologo{MetaFun},
%     \hologo{METAPOST},
%     \hologo{MetaPost},
%     \hologo{SLiTeX} (\xoption{lift}, \xoption{narrow}, \xoption{simple}),
%     \hologo{SliTeX} (\xoption{narrow}, \xoption{simple}, \xoption{lift}),
%     \hologo{teTeX}.
%   \item
%     Fixes:
%     \hologo{iniTeX},
%     \hologo{pdfLaTeX},
%     \hologo{pdfTeX},
%     \hologo{virTeX}.
%   \item
%     \cs{hologoFontSetup} and \cs{hologoLogoFontSetup} added.
%   \item
%     \cs{hologoVariant} and \cs{HologoVariant} added.
%   \end{Version}
%   \begin{Version}{2011/11/22 v1.7}
%   \item
%     New logos:
%     \hologo{BibTeX8},
%     \hologo{LaTeXML},
%     \hologo{SageTeX},
%     \hologo{TeX4ht},
%     \hologo{TTH}.
%   \item
%     \hologo{Xe} and friends: Driver stuff fixed.
%   \item
%     \hologo{Xe} and friends: Support for italic added.
%   \item
%     \hologo{Xe} and friends: Package support for \xpackage{pgf}
%     and \xpackage{pstricks} added.
%   \end{Version}
%   \begin{Version}{2011/11/29 v1.8}
%   \item
%     New logos:
%     \hologo{HanTheThanh}.
%   \end{Version}
%   \begin{Version}{2011/12/21 v1.9}
%   \item
%     Patch for package \xpackage{ifxetex} added for the case that
%     \cs{newif} is undefined in \hologo{iniTeX}.
%   \item
%     Some fixes for \hologo{iniTeX}.
%   \end{Version}
%   \begin{Version}{2012/04/26 v1.10}
%   \item
%     Fix in bookmark version of logo ``\hologo{HanTheThanh}''.
%   \end{Version}
%   \begin{Version}{2016/05/12 v1.11}
%   \item
%     Update HOLOGO@IfCharExists (previously in texlive)
%   \item define pdfliteral in current luatex.
%   \end{Version}
% \end{History}
%
% \PrintIndex
%
% \Finale
\endinput
%
        \else
          \input hologo.cfg\relax
        \fi
      \fi
    }%
    \ltx@IfUndefined{newread}{%
      \chardef\HOLOGO@temp=15 %
      \def\HOLOGO@CheckRead{%
        \ifeof\HOLOGO@temp
          \HOLOGO@InputIfExists
        \else
          \ifcase\HOLOGO@temp
            \@PackageWarningNoLine{hologo}{%
              Configuration file ignored, because\MessageBreak
              a free read register could not be found%
            }%
          \else
            \begingroup
              \count\ltx@cclv=\HOLOGO@temp
              \advance\ltx@cclv by \ltx@minusone
              \edef\x{\endgroup
                \chardef\noexpand\HOLOGO@temp=\the\count\ltx@cclv
                \relax
              }%
            \x
          \fi
        \fi
      }%
    }{%
      \csname newread\endcsname\HOLOGO@temp
      \HOLOGO@InputIfExists
    }%
  }{%
    \edef\HOLOGO@temp{\pdf@filesize{hologo.cfg}}%
    \ifx\HOLOGO@temp\ltx@empty
    \else
      \ifnum\HOLOGO@temp>0 %
        \begingroup
          \def\x{LaTeX2e}%
        \expandafter\endgroup
        \ifx\fmtname\x
          % \iffalse meta-comment
%
% File: hologo.dtx
% Version: 2016/05/12 v1.11
% Info: A logo collection with bookmark support
%
% Copyright (C) 2010-2012 by
%    Heiko Oberdiek <heiko.oberdiek at googlemail.com>
%
% This work may be distributed and/or modified under the
% conditions of the LaTeX Project Public License, either
% version 1.3c of this license or (at your option) any later
% version. This version of this license is in
%    http://www.latex-project.org/lppl/lppl-1-3c.txt
% and the latest version of this license is in
%    http://www.latex-project.org/lppl.txt
% and version 1.3 or later is part of all distributions of
% LaTeX version 2005/12/01 or later.
%
% This work has the LPPL maintenance status "maintained".
%
% This Current Maintainer of this work is Heiko Oberdiek.
%
% The Base Interpreter refers to any `TeX-Format',
% because some files are installed in TDS:tex/generic//.
%
% This work consists of the main source file hologo.dtx
% and the derived files
%    hologo.sty, hologo.pdf, hologo.ins, hologo.drv, hologo-example.tex,
%    hologo-test1.tex, hologo-test-spacefactor.tex,
%    hologo-test-list.tex.
%
% Distribution:
%    CTAN:macros/latex/contrib/oberdiek/hologo.dtx
%    CTAN:macros/latex/contrib/oberdiek/hologo.pdf
%
% Unpacking:
%    (a) If hologo.ins is present:
%           tex hologo.ins
%    (b) Without hologo.ins:
%           tex hologo.dtx
%    (c) If you insist on using LaTeX
%           latex \let\install=y\input{hologo.dtx}
%        (quote the arguments according to the demands of your shell)
%
% Documentation:
%    (a) If hologo.drv is present:
%           latex hologo.drv
%    (b) Without hologo.drv:
%           latex hologo.dtx; ...
%    The class ltxdoc loads the configuration file ltxdoc.cfg
%    if available. Here you can specify further options, e.g.
%    use A4 as paper format:
%       \PassOptionsToClass{a4paper}{article}
%
%    Programm calls to get the documentation (example):
%       pdflatex hologo.dtx
%       makeindex -s gind.ist hologo.idx
%       pdflatex hologo.dtx
%       makeindex -s gind.ist hologo.idx
%       pdflatex hologo.dtx
%
% Installation:
%    TDS:tex/generic/oberdiek/hologo.sty
%    TDS:doc/latex/oberdiek/hologo.pdf
%    TDS:doc/latex/oberdiek/example/hologo-example.tex
%    TDS:doc/latex/oberdiek/test/hologo-test1.tex
%    TDS:doc/latex/oberdiek/test/hologo-test-spacefactor.tex
%    TDS:doc/latex/oberdiek/test/hologo-test-list.tex
%    TDS:source/latex/oberdiek/hologo.dtx
%
%<*ignore>
\begingroup
  \catcode123=1 %
  \catcode125=2 %
  \def\x{LaTeX2e}%
\expandafter\endgroup
\ifcase 0\ifx\install y1\fi\expandafter
         \ifx\csname processbatchFile\endcsname\relax\else1\fi
         \ifx\fmtname\x\else 1\fi\relax
\else\csname fi\endcsname
%</ignore>
%<*install>
\input docstrip.tex
\Msg{************************************************************************}
\Msg{* Installation}
\Msg{* Package: hologo 2016/05/12 v1.11 A logo collection with bookmark support (HO)}
\Msg{************************************************************************}

\keepsilent
\askforoverwritefalse

\let\MetaPrefix\relax
\preamble

This is a generated file.

Project: hologo
Version: 2016/05/12 v1.11

Copyright (C) 2010-2012 by
   Heiko Oberdiek <heiko.oberdiek at googlemail.com>

This work may be distributed and/or modified under the
conditions of the LaTeX Project Public License, either
version 1.3c of this license or (at your option) any later
version. This version of this license is in
   http://www.latex-project.org/lppl/lppl-1-3c.txt
and the latest version of this license is in
   http://www.latex-project.org/lppl.txt
and version 1.3 or later is part of all distributions of
LaTeX version 2005/12/01 or later.

This work has the LPPL maintenance status "maintained".

This Current Maintainer of this work is Heiko Oberdiek.

The Base Interpreter refers to any `TeX-Format',
because some files are installed in TDS:tex/generic//.

This work consists of the main source file hologo.dtx
and the derived files
   hologo.sty, hologo.pdf, hologo.ins, hologo.drv, hologo-example.tex,
   hologo-test1.tex, hologo-test-spacefactor.tex,
   hologo-test-list.tex.

\endpreamble
\let\MetaPrefix\DoubleperCent

\generate{%
  \file{hologo.ins}{\from{hologo.dtx}{install}}%
  \file{hologo.drv}{\from{hologo.dtx}{driver}}%
  \usedir{tex/generic/oberdiek}%
  \file{hologo.sty}{\from{hologo.dtx}{package}}%
  \usedir{doc/latex/oberdiek/example}%
  \file{hologo-example.tex}{\from{hologo.dtx}{example}}%
  \usedir{doc/latex/oberdiek/test}%
  \file{hologo-test1.tex}{\from{hologo.dtx}{test1}}%
  \file{hologo-test-spacefactor.tex}{\from{hologo.dtx}{test-spacefactor}}%
  \file{hologo-test-list.tex}{\from{hologo.dtx}{test-list}}%
  \nopreamble
  \nopostamble
  \usedir{source/latex/oberdiek/catalogue}%
  \file{hologo.xml}{\from{hologo.dtx}{catalogue}}%
}

\catcode32=13\relax% active space
\let =\space%
\Msg{************************************************************************}
\Msg{*}
\Msg{* To finish the installation you have to move the following}
\Msg{* file into a directory searched by TeX:}
\Msg{*}
\Msg{*     hologo.sty}
\Msg{*}
\Msg{* To produce the documentation run the file `hologo.drv'}
\Msg{* through LaTeX.}
\Msg{*}
\Msg{* Happy TeXing!}
\Msg{*}
\Msg{************************************************************************}

\endbatchfile
%</install>
%<*ignore>
\fi
%</ignore>
%<*driver>
\NeedsTeXFormat{LaTeX2e}
\ProvidesFile{hologo.drv}%
  [2016/05/12 v1.11 A logo collection with bookmark support (HO)]%
\documentclass{ltxdoc}
\usepackage{holtxdoc}[2011/11/22]
\usepackage{hologo}[2016/05/12]
\usepackage{longtable}
\usepackage{array}
\usepackage{paralist}
%\usepackage[T1]{fontenc}
%\usepackage{lmodern}
\begin{document}
  \DocInput{hologo.dtx}%
\end{document}
%</driver>
% \fi
%
%
% \CharacterTable
%  {Upper-case    \A\B\C\D\E\F\G\H\I\J\K\L\M\N\O\P\Q\R\S\T\U\V\W\X\Y\Z
%   Lower-case    \a\b\c\d\e\f\g\h\i\j\k\l\m\n\o\p\q\r\s\t\u\v\w\x\y\z
%   Digits        \0\1\2\3\4\5\6\7\8\9
%   Exclamation   \!     Double quote  \"     Hash (number) \#
%   Dollar        \$     Percent       \%     Ampersand     \&
%   Acute accent  \'     Left paren    \(     Right paren   \)
%   Asterisk      \*     Plus          \+     Comma         \,
%   Minus         \-     Point         \.     Solidus       \/
%   Colon         \:     Semicolon     \;     Less than     \<
%   Equals        \=     Greater than  \>     Question mark \?
%   Commercial at \@     Left bracket  \[     Backslash     \\
%   Right bracket \]     Circumflex    \^     Underscore    \_
%   Grave accent  \`     Left brace    \{     Vertical bar  \|
%   Right brace   \}     Tilde         \~}
%
% \GetFileInfo{hologo.drv}
%
% \title{The \xpackage{hologo} package}
% \date{2016/05/12 v1.11}
% \author{Heiko Oberdiek\\\xemail{heiko.oberdiek at googlemail.com}}
%
% \maketitle
%
% \begin{abstract}
% This package starts a collection of logos with support for bookmarks
% strings.
% \end{abstract}
%
% \tableofcontents
%
% \section{Documentation}
%
% \subsection{Logo macros}
%
% \begin{declcs}{hologo} \M{name}
% \end{declcs}
% Macro \cs{hologo} sets the logo with name \meta{name}.
% The following table shows the supported names.
%
% \begingroup
%   \def\hologoEntry#1#2#3{^^A
%     #1&#2&\hologoLogoSetup{#1}{variant=#2}\hologo{#1}&#3\tabularnewline
%   }
%   \begin{longtable}{>{\ttfamily}l>{\ttfamily}lll}
%     \rmfamily\bfseries{name} & \rmfamily\bfseries variant
%     & \bfseries logo & \bfseries since\\
%     \hline
%     \endhead
%     \hologoList
%   \end{longtable}
% \endgroup
%
% \begin{declcs}{Hologo} \M{name}
% \end{declcs}
% Macro \cs{Hologo} starts the logo \meta{name} with an uppercase
% letter. As an exception small greek letters are not converted
% to uppercase. Examples, see \hologo{eTeX} and \hologo{ExTeX}.
%
% \subsection{Setup macros}
%
% The package does not support package options, but the following
% setup macros can be used to set options.
%
% \begin{declcs}{hologoSetup} \M{key value list}
% \end{declcs}
% Macro \cs{hologoSetup} sets global options.
%
% \begin{declcs}{hologoLogoSetup} \M{logo} \M{key value list}
% \end{declcs}
% Some options can also be used to configure a logo.
% These settings take precedence over global option settings.
%
% \subsection{Options}\label{sec:options}
%
% There are boolean and string options:
% \begin{description}
% \item[Boolean option:]
% It takes |true| or |false|
% as value. If the value is omitted, then |true| is used.
% \item[String option:]
% A value must be given as string. (But the string might be empty.)
% \end{description}
% The following options can be used both in \cs{hologoSetup}
% and \cs{hologoLogoSetup}:
% \begin{description}
% \def\entry#1{\item[\xoption{#1}:]}
% \entry{break}
%   enables or disables line breaks inside the logo. This setting is
%   refined by options \xoption{hyphenbreak}, \xoption{spacebreak}
%   or \xoption{discretionarybreak}.
%   Default is |false|.
% \entry{hyphenbreak}
%   enables or disables the line break right after the hyphen character.
% \entry{spacebreak}
%   enables or disables line breaks at space characters.
% \entry{discretionarybreak}
%   enables or disables line breaks at hyphenation points
%   (inserted by \cs{-}).
% \end{description}
% Macro \cs{hologoLogoSetup} also knows:
% \begin{description}
% \item[\xoption{variant}:]
%   This is a string option. It specifies a variant of a logo that
%   must exist. An empty string selects the package default variant.
% \end{description}
% Example:
% \begin{quote}
%   |\hologoSetup{break=false}|\\
%   |\hologoLogoSetup{plainTeX}{variant=hyphen,hyphenbreak}|\\
%   Then ``plain-\TeX'' contains one break point after the hyphen.
% \end{quote}
%
% \subsection{Driver options}
%
% Sometimes graphical operations are needed to construct some
% glyphs (e.g.\ \hologo{XeTeX}). If package \xpackage{graphics}
% or package \xpackage{pgf} are found, then the macros are taken
% from there. Otherwise the packge defines its own operations
% and therefore needs the driver information. Many drivers are
% detected automatically (\hologo{pdfTeX}/\hologo{LuaTeX}
% in PDF mode, \hologo{XeTeX}, \hologo{VTeX}). These have precedence
% over a driver option. The driver can be given as package option
% or using \cs{hologoDriverSetup}.
% The following list contains the recognized driver options:
% \begin{itemize}
% \item \xoption{pdftex}, \xoption{luatex}
% \item \xoption{dvipdfm}, \xoption{dvipdfmx}
% \item \xoption{dvips}, \xoption{dvipsone}, \xoption{xdvi}
% \item \xoption{xetex}
% \item \xoption{vtex}
% \end{itemize}
% The left driver of a line is the driver name that is used internally.
% The following names are aliases for drivers that use the
% same method. Therefore the entry in the \xext{log} file for
% the used driver prints the internally used driver name.
% \begin{description}
% \item[\xoption{driverfallback}:]
%   This option expects a driver that is used,
%   if the driver could not be detected automatically.
% \end{description}
%
% \begin{declcs}{hologoDriverSetup} \M{driver option}
% \end{declcs}
% The driver can also be configured after package loading
% using \cs{hologoDriverSetup}, also the way for \hologo{plainTeX}
% to setup the driver.
%
% \subsection{Font setup}
%
% Some logos require a special font, but should also be usable by
% \hologo{plainTeX}. Therefore the package provides some ways
% to influence the font settings. The options below
% take font settings as values. Both font commands
% such as \cs{sffamily} and macros that take one argument
% like \cs{textsf} can be used.
%
% \begin{declcs}{hologoFontSetup} \M{key value list}
% \end{declcs}
% Macro \cs{hologoFontSetup} sets the fonts for all logos.
% Supported keys:
% \begin{description}
% \def\entry#1{\item[\xoption{#1}:]}
% \entry{general}
%   This font is used for all logos. The default is empty.
%   That means no special font is used.
% \entry{bibsf}
%   This font is used for
%   {\hologoLogoSetup{BibTeX}{variant=sf}\hologo{BibTeX}}
%   with variant \xoption{sf}.
% \entry{rm}
%   This font is a serif font. It is used for \hologo{ExTeX}.
% \entry{sc}
%   This font specifies a small caps font. It is used for
%   {\hologoLogoSetup{BibTeX}{variant=sc}\hologo{BibTeX}}
%   with variant \xoption{sc}.
% \entry{sf}
%   This font specifies a sans serif font. The default
%   is \cs{sffamily}, then \cs{sf} is tried. Otherwise
%   a warning is given. It is used by \hologo{KOMAScript}.
% \entry{sy}
%   This is the font for math symbols (e.g. cmsy).
%   It is used by \hologo{AmS}, \hologo{NTS}, \hologo{ExTeX}.
% \entry{logo}
%   \hologo{METAFONT} and \hologo{METAPOST} are using that font.
%   In \hologo{LaTeX} \cs{logofamily} is used and
%   the definitions of package \xpackage{mflogo} are used
%   if the package is not loaded.
%   Otherwise the \cs{tenlogo} is used and defined
%   if it does not already exists.
% \end{description}
%
% \begin{declcs}{hologoLogoFontSetup} \M{logo} \M{key value list}
% \end{declcs}
% Fonts can also be set for a logo or logo component separately,
% see the following list.
% The keys are the same as for \cs{hologoFontSetup}.
%
% \begin{longtable}{>{\ttfamily}l>{\sffamily}ll}
%   \meta{logo} & keys & result\\
%   \hline
%   \endhead
%   BibTeX & bibsf & {\hologoLogoSetup{BibTeX}{variant=sf}\hologo{BibTeX}}\\[.5ex]
%   BibTeX & sc & {\hologoLogoSetup{BibTeX}{variant=sc}\hologo{BibTeX}}\\[.5ex]
%   ExTeX & rm & \hologo{ExTeX}\\
%   SliTeX & rm & \hologo{SliTeX}\\[.5ex]
%   AmS & sy & \hologo{AmS}\\
%   ExTeX & sy & \hologo{ExTeX}\\
%   NTS & sy & \hologo{NTS}\\[.5ex]
%   KOMAScript & sf & \hologo{KOMAScript}\\[.5ex]
%   METAFONT & logo & \hologo{METAFONT}\\
%   METAPOST & logo & \hologo{METAPOST}\\[.5ex]
%   SliTeX & sc \hologo{SliTeX}
% \end{longtable}
%
% \subsubsection{Font order}
%
% For all logos the font \xoption{general} is applied first.
% Example:
%\begin{quote}
%|\hologoFontSetup{general=\color{red}}|
%\end{quote}
% will print red logos.
% Then if the font uses a special font \xoption{sf}, for example,
% the font is applied that is setup by \cs{hologoLogoFontSetup}.
% If this font is not setup, then the common font setup
% by \cs{hologoFontSetup} is used. Otherwise a warning is given,
% that there is no font configured.
%
% \subsection{Additional user macros}
%
% Usually a variant of a logo is configured by using
% \cs{hologoLogoSetup}, because it is bad style to mix
% different variants of the same logo in the same text.
% There the following macros are a convenience for testing.
%
% \begin{declcs}{hologoVariant} \M{name} \M{variant}\\
%   \cs{HologoVariant} \M{name} \M{variant}
% \end{declcs}
% Logo \meta{name} is set using \meta{variant} that specifies
% explicitely which variant of the macro is used. If the argument
% is empty, then the default form of the logo is used
% (configurable by \cs{hologoLogoSetup}).
%
% \cs{HologoVariant} is used if the logo is set in a context
% that needs an uppercase first letter (beginning of a sentence, \dots).
%
% \begin{declcs}{hologoList}\\
%   \cs{hologoEntry} \M{logo} \M{variant} \M{since}
% \end{declcs}
% Macro \cs{hologoList} contains all logos that are provided
% by the package including variants. The list consists of calls
% of \cs{hologoEntry} with three arguments starting with the
% logo name \meta{logo} and its variant \meta{variant}. An empty
% variant means the current default. Argument \meta{since} specifies
% with version of the package \xpackage{hologo} is needed to get
% the logo. If the logo is fixed, then the date gets updated.
% Therefore the date \meta{since} is not exactly the date of
% the first introduction, but rather the date of the latest fix.
%
% Before \cs{hologoList} can be used, macro \cs{hologoEntry} needs
% a definition. The example file in section \ref{sec:example}
% shows applications of \cs{hologoList}.
%
% \subsection{Supported contexts}
%
% Macros \cs{hologo} and friends support special contexts:
% \begin{itemize}
% \item \hologo{LaTeX}'s protection mechanism.
% \item Bookmarks of package \xpackage{hyperref}.
% \item Package \xpackage{tex4ht}.
% \item The macros can be used inside \cs{csname} constructs,
%   if \cs{ifincsname} is available (\hologo{pdfTeX}, \hologo{XeTeX},
%   \hologo{LuaTeX}).
% \end{itemize}
%
% \subsection{Example}
% \label{sec:example}
%
% The following example prints the logos in different fonts.
%    \begin{macrocode}
%<*example>
%<<verbatim
\NeedsTeXFormat{LaTeX2e}
\documentclass[a4paper]{article}
\usepackage[
  hmargin=20mm,
  vmargin=20mm,
]{geometry}
\pagestyle{empty}
\usepackage{hologo}[2016/05/12]
\usepackage{longtable}
\usepackage{array}
\setlength{\extrarowheight}{2pt}
\usepackage[T1]{fontenc}
\usepackage{lmodern}
\usepackage{pdflscape}
\usepackage[
  pdfencoding=auto,
]{hyperref}
\hypersetup{
  pdfauthor={Heiko Oberdiek},
  pdftitle={Example for package `hologo'},
  pdfsubject={Logos with fonts lmr, lmss, qtm, qpl, qhv},
}
\usepackage{bookmark}

% Print the logo list on the console

\begingroup
  \typeout{}%
  \typeout{*** Begin of logo list ***}%
  \newcommand*{\hologoEntry}[3]{%
    \typeout{#1 \ifx\\#2\\\else(#2) \fi[#3]}%
  }%
  \hologoList
  \typeout{*** End of logo list ***}%
  \typeout{}%
\endgroup

\begin{document}
\begin{landscape}

  \section{Example file for package `hologo'}

  % Table for font names

  \begin{longtable}{>{\bfseries}ll}
    \textbf{font} & \textbf{Font name}\\
    \hline
    lmr & Latin Modern Roman\\
    lmss & Latin Modern Sans\\
    qtm & \TeX\ Gyre Termes\\
    qhv & \TeX\ Gyre Heros\\
    qpl & \TeX\ Gyre Pagella\\
  \end{longtable}

  % Logo list with logos in different fonts

  \begingroup
    \newcommand*{\SetVariant}[2]{%
      \ifx\\#2\\%
      \else
        \hologoLogoSetup{#1}{variant=#2}%
      \fi
    }%
    \newcommand*{\hologoEntry}[3]{%
      \SetVariant{#1}{#2}%
      \raisebox{1em}[0pt][0pt]{\hypertarget{#1@#2}{}}%
      \bookmark[%
        dest={#1@#2},%
      ]{%
        #1\ifx\\#2\\\else\space(#2)\fi: \Hologo{#1}, \hologo{#1} %
        [Unicode]%
      }%
      \hypersetup{unicode=false}%
      \bookmark[%
        dest={#1@#2},%
      ]{%
        #1\ifx\\#2\\\else\space(#2)\fi: \Hologo{#1}, \hologo{#1} %
        [PDFDocEncoding]%
      }%
      \texttt{#1}%
      &%
      \texttt{#2}%
      &%
      \Hologo{#1}%
      &%
      \SetVariant{#1}{#2}%
      \hologo{#1}%
      &%
      \SetVariant{#1}{#2}%
      \fontfamily{qtm}\selectfont
      \hologo{#1}%
      &%
      \SetVariant{#1}{#2}%
      \fontfamily{qpl}\selectfont
      \hologo{#1}%
      &%
      \SetVariant{#1}{#2}%
      \textsf{\hologo{#1}}%
      &%
      \SetVariant{#1}{#2}%
      \fontfamily{qhv}\selectfont
      \hologo{#1}%
      \tabularnewline
    }%
    \begin{longtable}{llllllll}%
      \textbf{\textit{logo}} & \textbf{\textit{variant}} &
      \texttt{\string\Hologo} &
      \textbf{lmr} & \textbf{qtm} & \textbf{qpl} &
      \textbf{lmss} & \textbf{qhv}
      \tabularnewline
      \hline
      \endhead
      \hologoList
    \end{longtable}%
  \endgroup

\end{landscape}
\end{document}
%verbatim
%</example>
%    \end{macrocode}
%
% \StopEventually{
% }
%
% \section{Implementation}
%    \begin{macrocode}
%<*package>
%    \end{macrocode}
%    Reload check, especially if the package is not used with \LaTeX.
%    \begin{macrocode}
\begingroup\catcode61\catcode48\catcode32=10\relax%
  \catcode13=5 % ^^M
  \endlinechar=13 %
  \catcode35=6 % #
  \catcode39=12 % '
  \catcode44=12 % ,
  \catcode45=12 % -
  \catcode46=12 % .
  \catcode58=12 % :
  \catcode64=11 % @
  \catcode123=1 % {
  \catcode125=2 % }
  \expandafter\let\expandafter\x\csname ver@hologo.sty\endcsname
  \ifx\x\relax % plain-TeX, first loading
  \else
    \def\empty{}%
    \ifx\x\empty % LaTeX, first loading,
      % variable is initialized, but \ProvidesPackage not yet seen
    \else
      \expandafter\ifx\csname PackageInfo\endcsname\relax
        \def\x#1#2{%
          \immediate\write-1{Package #1 Info: #2.}%
        }%
      \else
        \def\x#1#2{\PackageInfo{#1}{#2, stopped}}%
      \fi
      \x{hologo}{The package is already loaded}%
      \aftergroup\endinput
    \fi
  \fi
\endgroup%
%    \end{macrocode}
%    Package identification:
%    \begin{macrocode}
\begingroup\catcode61\catcode48\catcode32=10\relax%
  \catcode13=5 % ^^M
  \endlinechar=13 %
  \catcode35=6 % #
  \catcode39=12 % '
  \catcode40=12 % (
  \catcode41=12 % )
  \catcode44=12 % ,
  \catcode45=12 % -
  \catcode46=12 % .
  \catcode47=12 % /
  \catcode58=12 % :
  \catcode64=11 % @
  \catcode91=12 % [
  \catcode93=12 % ]
  \catcode123=1 % {
  \catcode125=2 % }
  \expandafter\ifx\csname ProvidesPackage\endcsname\relax
    \def\x#1#2#3[#4]{\endgroup
      \immediate\write-1{Package: #3 #4}%
      \xdef#1{#4}%
    }%
  \else
    \def\x#1#2[#3]{\endgroup
      #2[{#3}]%
      \ifx#1\@undefined
        \xdef#1{#3}%
      \fi
      \ifx#1\relax
        \xdef#1{#3}%
      \fi
    }%
  \fi
\expandafter\x\csname ver@hologo.sty\endcsname
\ProvidesPackage{hologo}%
  [2016/05/12 v1.11 A logo collection with bookmark support (HO)]%
%    \end{macrocode}
%
%    \begin{macrocode}
\begingroup\catcode61\catcode48\catcode32=10\relax%
  \catcode13=5 % ^^M
  \endlinechar=13 %
  \catcode123=1 % {
  \catcode125=2 % }
  \catcode64=11 % @
  \def\x{\endgroup
    \expandafter\edef\csname HOLOGO@AtEnd\endcsname{%
      \endlinechar=\the\endlinechar\relax
      \catcode13=\the\catcode13\relax
      \catcode32=\the\catcode32\relax
      \catcode35=\the\catcode35\relax
      \catcode61=\the\catcode61\relax
      \catcode64=\the\catcode64\relax
      \catcode123=\the\catcode123\relax
      \catcode125=\the\catcode125\relax
    }%
  }%
\x\catcode61\catcode48\catcode32=10\relax%
\catcode13=5 % ^^M
\endlinechar=13 %
\catcode35=6 % #
\catcode64=11 % @
\catcode123=1 % {
\catcode125=2 % }
\def\TMP@EnsureCode#1#2{%
  \edef\HOLOGO@AtEnd{%
    \HOLOGO@AtEnd
    \catcode#1=\the\catcode#1\relax
  }%
  \catcode#1=#2\relax
}
\TMP@EnsureCode{10}{12}% ^^J
\TMP@EnsureCode{33}{12}% !
\TMP@EnsureCode{34}{12}% "
\TMP@EnsureCode{36}{3}% $
\TMP@EnsureCode{38}{4}% &
\TMP@EnsureCode{39}{12}% '
\TMP@EnsureCode{40}{12}% (
\TMP@EnsureCode{41}{12}% )
\TMP@EnsureCode{42}{12}% *
\TMP@EnsureCode{43}{12}% +
\TMP@EnsureCode{44}{12}% ,
\TMP@EnsureCode{45}{12}% -
\TMP@EnsureCode{46}{12}% .
\TMP@EnsureCode{47}{12}% /
\TMP@EnsureCode{58}{12}% :
\TMP@EnsureCode{59}{12}% ;
\TMP@EnsureCode{60}{12}% <
\TMP@EnsureCode{62}{12}% >
\TMP@EnsureCode{63}{12}% ?
\TMP@EnsureCode{91}{12}% [
\TMP@EnsureCode{93}{12}% ]
\TMP@EnsureCode{94}{7}% ^ (superscript)
\TMP@EnsureCode{95}{8}% _ (subscript)
\TMP@EnsureCode{96}{12}% `
\TMP@EnsureCode{124}{12}% |
\edef\HOLOGO@AtEnd{%
  \HOLOGO@AtEnd
  \escapechar\the\escapechar\relax
  \noexpand\endinput
}
\escapechar=92 %
%    \end{macrocode}
%
% \subsection{Logo list}
%
%    \begin{macro}{\hologoList}
%    \begin{macrocode}
\def\hologoList{%
  \hologoEntry{(La)TeX}{}{2011/10/01}%
  \hologoEntry{AmSLaTeX}{}{2010/04/16}%
  \hologoEntry{AmSTeX}{}{2010/04/16}%
  \hologoEntry{biber}{}{2011/10/01}%
  \hologoEntry{BibTeX}{}{2011/10/01}%
  \hologoEntry{BibTeX}{sf}{2011/10/01}%
  \hologoEntry{BibTeX}{sc}{2011/10/01}%
  \hologoEntry{BibTeX8}{}{2011/11/22}%
  \hologoEntry{ConTeXt}{}{2011/03/25}%
  \hologoEntry{ConTeXt}{narrow}{2011/03/25}%
  \hologoEntry{ConTeXt}{simple}{2011/03/25}%
  \hologoEntry{emTeX}{}{2010/04/26}%
  \hologoEntry{eTeX}{}{2010/04/08}%
  \hologoEntry{ExTeX}{}{2011/10/01}%
  \hologoEntry{HanTheThanh}{}{2011/11/29}%
  \hologoEntry{iniTeX}{}{2011/10/01}%
  \hologoEntry{KOMAScript}{}{2011/10/01}%
  \hologoEntry{La}{}{2010/05/08}%
  \hologoEntry{LaTeX}{}{2010/04/08}%
  \hologoEntry{LaTeX2e}{}{2010/04/08}%
  \hologoEntry{LaTeX3}{}{2010/04/24}%
  \hologoEntry{LaTeXe}{}{2010/04/08}%
  \hologoEntry{LaTeXML}{}{2011/11/22}%
  \hologoEntry{LaTeXTeX}{}{2011/10/01}%
  \hologoEntry{LuaLaTeX}{}{2010/04/08}%
  \hologoEntry{LuaTeX}{}{2010/04/08}%
  \hologoEntry{LyX}{}{2011/10/01}%
  \hologoEntry{METAFONT}{}{2011/10/01}%
  \hologoEntry{MetaFun}{}{2011/10/01}%
  \hologoEntry{METAPOST}{}{2011/10/01}%
  \hologoEntry{MetaPost}{}{2011/10/01}%
  \hologoEntry{MiKTeX}{}{2011/10/01}%
  \hologoEntry{NTS}{}{2011/10/01}%
  \hologoEntry{OzMF}{}{2011/10/01}%
  \hologoEntry{OzMP}{}{2011/10/01}%
  \hologoEntry{OzTeX}{}{2011/10/01}%
  \hologoEntry{OzTtH}{}{2011/10/01}%
  \hologoEntry{PCTeX}{}{2011/10/01}%
  \hologoEntry{pdfTeX}{}{2011/10/01}%
  \hologoEntry{pdfLaTeX}{}{2011/10/01}%
  \hologoEntry{PiC}{}{2011/10/01}%
  \hologoEntry{PiCTeX}{}{2011/10/01}%
  \hologoEntry{plainTeX}{}{2010/04/08}%
  \hologoEntry{plainTeX}{space}{2010/04/16}%
  \hologoEntry{plainTeX}{hyphen}{2010/04/16}%
  \hologoEntry{plainTeX}{runtogether}{2010/04/16}%
  \hologoEntry{SageTeX}{}{2011/11/22}%
  \hologoEntry{SLiTeX}{}{2011/10/01}%
  \hologoEntry{SLiTeX}{lift}{2011/10/01}%
  \hologoEntry{SLiTeX}{narrow}{2011/10/01}%
  \hologoEntry{SLiTeX}{simple}{2011/10/01}%
  \hologoEntry{SliTeX}{}{2011/10/01}%
  \hologoEntry{SliTeX}{narrow}{2011/10/01}%
  \hologoEntry{SliTeX}{simple}{2011/10/01}%
  \hologoEntry{SliTeX}{lift}{2011/10/01}%
  \hologoEntry{teTeX}{}{2011/10/01}%
  \hologoEntry{TeX}{}{2010/04/08}%
  \hologoEntry{TeX4ht}{}{2011/11/22}%
  \hologoEntry{TTH}{}{2011/11/22}%
  \hologoEntry{virTeX}{}{2011/10/01}%
  \hologoEntry{VTeX}{}{2010/04/24}%
  \hologoEntry{Xe}{}{2010/04/08}%
  \hologoEntry{XeLaTeX}{}{2010/04/08}%
  \hologoEntry{XeTeX}{}{2010/04/08}%
}
%    \end{macrocode}
%    \end{macro}
%
% \subsection{Load resources}
%
%    \begin{macrocode}
\begingroup\expandafter\expandafter\expandafter\endgroup
\expandafter\ifx\csname RequirePackage\endcsname\relax
  \def\TMP@RequirePackage#1[#2]{%
    \begingroup\expandafter\expandafter\expandafter\endgroup
    \expandafter\ifx\csname ver@#1.sty\endcsname\relax
      \input #1.sty\relax
    \fi
  }%
  \TMP@RequirePackage{ltxcmds}[2011/02/04]%
  \TMP@RequirePackage{infwarerr}[2010/04/08]%
  \TMP@RequirePackage{kvsetkeys}[2010/03/01]%
  \TMP@RequirePackage{kvdefinekeys}[2010/03/01]%
  \TMP@RequirePackage{pdftexcmds}[2010/04/01]%
  \TMP@RequirePackage{ifpdf}[2010/01/28]%
  \TMP@RequirePackage{ifluatex}[2010/03/01]%
  \ltx@IfUndefined{newif}{%
    \expandafter\let\csname newif\endcsname\ltx@newif
  }{}%
  \TMP@RequirePackage{ifxetex}[2009/01/23]%
  \TMP@RequirePackage{ifvtex}[2010/03/01]%
\else
  \RequirePackage{ltxcmds}[2011/02/04]%
  \RequirePackage{infwarerr}[2010/04/08]%
  \RequirePackage{kvsetkeys}[2010/03/01]%
  \RequirePackage{kvdefinekeys}[2010/03/01]%
  \RequirePackage{pdftexcmds}[2010/04/01]%
  \RequirePackage{ifpdf}[2010/01/28]%
  \RequirePackage{ifluatex}[2010/03/01]%
  \RequirePackage{ifxetex}[2009/01/23]%
  \RequirePackage{ifvtex}[2010/03/01]%
\fi
%    \end{macrocode}
%
%    \begin{macro}{\HOLOGO@IfDefined}
%    \begin{macrocode}
\def\HOLOGO@IfExists#1{%
  \ifx\@undefined#1%
    \expandafter\ltx@secondoftwo
  \else
    \ifx\relax#1%
      \expandafter\ltx@secondoftwo
    \else
      \expandafter\expandafter\expandafter\ltx@firstoftwo
    \fi
  \fi
}
%    \end{macrocode}
%    \end{macro}
%
% \subsection{Setup macros}
%
%    \begin{macro}{\hologoSetup}
%    \begin{macrocode}
\def\hologoSetup{%
  \let\HOLOGO@name\relax
  \HOLOGO@Setup
}
%    \end{macrocode}
%    \end{macro}
%
%    \begin{macro}{\hologoLogoSetup}
%    \begin{macrocode}
\def\hologoLogoSetup#1{%
  \edef\HOLOGO@name{#1}%
  \ltx@IfUndefined{HoLogo@\HOLOGO@name}{%
    \@PackageError{hologo}{%
      Unknown logo `\HOLOGO@name'%
    }\@ehc
    \ltx@gobble
  }{%
    \HOLOGO@Setup
  }%
}
%    \end{macrocode}
%    \end{macro}
%
%    \begin{macro}{\HOLOGO@Setup}
%    \begin{macrocode}
\def\HOLOGO@Setup{%
  \kvsetkeys{HoLogo}%
}
%    \end{macrocode}
%    \end{macro}
%
% \subsection{Options}
%
%    \begin{macro}{\HOLOGO@DeclareBoolOption}
%    \begin{macrocode}
\def\HOLOGO@DeclareBoolOption#1{%
  \expandafter\chardef\csname HOLOGOOPT@#1\endcsname\ltx@zero
  \kv@define@key{HoLogo}{#1}[true]{%
    \def\HOLOGO@temp{##1}%
    \ifx\HOLOGO@temp\HOLOGO@true
      \ifx\HOLOGO@name\relax
        \expandafter\chardef\csname HOLOGOOPT@#1\endcsname=\ltx@one
      \else
        \expandafter\chardef\csname
        HoLogoOpt@#1@\HOLOGO@name\endcsname\ltx@one
      \fi
      \HOLOGO@SetBreakAll{#1}%
    \else
      \ifx\HOLOGO@temp\HOLOGO@false
        \ifx\HOLOGO@name\relax
          \expandafter\chardef\csname HOLOGOOPT@#1\endcsname=\ltx@zero
        \else
          \expandafter\chardef\csname
          HoLogoOpt@#1@\HOLOGO@name\endcsname=\ltx@zero
        \fi
        \HOLOGO@SetBreakAll{#1}%
      \else
        \@PackageError{hologo}{%
          Unknown value `##1' for boolean option `#1'.\MessageBreak
          Known values are `true' and `false'%
        }\@ehc
      \fi
    \fi
  }%
}
%    \end{macrocode}
%    \end{macro}
%
%    \begin{macro}{\HOLOGO@SetBreakAll}
%    \begin{macrocode}
\def\HOLOGO@SetBreakAll#1{%
  \def\HOLOGO@temp{#1}%
  \ifx\HOLOGO@temp\HOLOGO@break
    \ifx\HOLOGO@name\relax
      \chardef\HOLOGOOPT@hyphenbreak=\HOLOGOOPT@break
      \chardef\HOLOGOOPT@spacebreak=\HOLOGOOPT@break
      \chardef\HOLOGOOPT@discretionarybreak=\HOLOGOOPT@break
    \else
      \expandafter\chardef
         \csname HoLogoOpt@hyphenbreak@\HOLOGO@name\endcsname=%
         \csname HoLogoOpt@break@\HOLOGO@name\endcsname
      \expandafter\chardef
         \csname HoLogoOpt@spacebreak@\HOLOGO@name\endcsname=%
         \csname HoLogoOpt@break@\HOLOGO@name\endcsname
      \expandafter\chardef
         \csname HoLogoOpt@discretionarybreak@\HOLOGO@name
             \endcsname=%
         \csname HoLogoOpt@break@\HOLOGO@name\endcsname
    \fi
  \fi
}
%    \end{macrocode}
%    \end{macro}
%
%    \begin{macro}{\HOLOGO@true}
%    \begin{macrocode}
\def\HOLOGO@true{true}
%    \end{macrocode}
%    \end{macro}
%    \begin{macro}{\HOLOGO@false}
%    \begin{macrocode}
\def\HOLOGO@false{false}
%    \end{macrocode}
%    \end{macro}
%    \begin{macro}{\HOLOGO@break}
%    \begin{macrocode}
\def\HOLOGO@break{break}
%    \end{macrocode}
%    \end{macro}
%
%    \begin{macrocode}
\HOLOGO@DeclareBoolOption{break}
\HOLOGO@DeclareBoolOption{hyphenbreak}
\HOLOGO@DeclareBoolOption{spacebreak}
\HOLOGO@DeclareBoolOption{discretionarybreak}
%    \end{macrocode}
%
%    \begin{macrocode}
\kv@define@key{HoLogo}{variant}{%
  \ifx\HOLOGO@name\relax
    \@PackageError{hologo}{%
      Option `variant' is not available in \string\hologoSetup,%
      \MessageBreak
      Use \string\hologoLogoSetup\space instead%
    }\@ehc
  \else
    \edef\HOLOGO@temp{#1}%
    \ifx\HOLOGO@temp\ltx@empty
      \expandafter
      \let\csname HoLogoOpt@variant@\HOLOGO@name\endcsname\@undefined
    \else
      \ltx@IfUndefined{HoLogo@\HOLOGO@name @\HOLOGO@temp}{%
        \@PackageError{hologo}{%
          Unknown variant `\HOLOGO@temp' of logo `\HOLOGO@name'%
        }\@ehc
      }{%
        \expandafter
        \let\csname HoLogoOpt@variant@\HOLOGO@name\endcsname
            \HOLOGO@temp
      }%
    \fi
  \fi
}
%    \end{macrocode}
%
%    \begin{macro}{\HOLOGO@Variant}
%    \begin{macrocode}
\def\HOLOGO@Variant#1{%
  #1%
  \ltx@ifundefined{HoLogoOpt@variant@#1}{%
  }{%
    @\csname HoLogoOpt@variant@#1\endcsname
  }%
}
%    \end{macrocode}
%    \end{macro}
%
% \subsection{Break/no-break support}
%
%    \begin{macro}{\HOLOGO@space}
%    \begin{macrocode}
\def\HOLOGO@space{%
  \ltx@ifundefined{HoLogoOpt@spacebreak@\HOLOGO@name}{%
    \ltx@ifundefined{HoLogoOpt@break@\HOLOGO@name}{%
      \chardef\HOLOGO@temp=\HOLOGOOPT@spacebreak
    }{%
      \chardef\HOLOGO@temp=%
        \csname HoLogoOpt@break@\HOLOGO@name\endcsname
    }%
  }{%
    \chardef\HOLOGO@temp=%
      \csname HoLogoOpt@spacebreak@\HOLOGO@name\endcsname
  }%
  \ifcase\HOLOGO@temp
    \penalty10000 %
  \fi
  \ltx@space
}
%    \end{macrocode}
%    \end{macro}
%
%    \begin{macro}{\HOLOGO@hyphen}
%    \begin{macrocode}
\def\HOLOGO@hyphen{%
  \ltx@ifundefined{HoLogoOpt@hyphenbreak@\HOLOGO@name}{%
    \ltx@ifundefined{HoLogoOpt@break@\HOLOGO@name}{%
      \chardef\HOLOGO@temp=\HOLOGOOPT@hyphenbreak
    }{%
      \chardef\HOLOGO@temp=%
        \csname HoLogoOpt@break@\HOLOGO@name\endcsname
    }%
  }{%
    \chardef\HOLOGO@temp=%
      \csname HoLogoOpt@hyphenbreak@\HOLOGO@name\endcsname
  }%
  \ifcase\HOLOGO@temp
    \ltx@mbox{-}%
  \else
    -%
  \fi
}
%    \end{macrocode}
%    \end{macro}
%
%    \begin{macro}{\HOLOGO@discretionary}
%    \begin{macrocode}
\def\HOLOGO@discretionary{%
  \ltx@ifundefined{HoLogoOpt@discretionarybreak@\HOLOGO@name}{%
    \ltx@ifundefined{HoLogoOpt@break@\HOLOGO@name}{%
      \chardef\HOLOGO@temp=\HOLOGOOPT@discretionarybreak
    }{%
      \chardef\HOLOGO@temp=%
        \csname HoLogoOpt@break@\HOLOGO@name\endcsname
    }%
  }{%
    \chardef\HOLOGO@temp=%
      \csname HoLogoOpt@discretionarybreak@\HOLOGO@name\endcsname
  }%
  \ifcase\HOLOGO@temp
  \else
    \-%
  \fi
}
%    \end{macrocode}
%    \end{macro}
%
%    \begin{macro}{\HOLOGO@mbox}
%    \begin{macrocode}
\def\HOLOGO@mbox#1{%
  \ltx@ifundefined{HoLogoOpt@break@\HOLOGO@name}{%
    \chardef\HOLOGO@temp=\HOLOGOOPT@hyphenbreak
  }{%
    \chardef\HOLOGO@temp=%
      \csname HoLogoOpt@break@\HOLOGO@name\endcsname
  }%
  \ifcase\HOLOGO@temp
    \ltx@mbox{#1}%
  \else
    #1%
  \fi
}
%    \end{macrocode}
%    \end{macro}
%
% \subsection{Font support}
%
%    \begin{macro}{\HoLogoFont@font}
%    \begin{tabular}{@{}ll@{}}
%    |#1|:& logo name\\
%    |#2|:& font short name\\
%    |#3|:& text
%    \end{tabular}
%    \begin{macrocode}
\def\HoLogoFont@font#1#2#3{%
  \begingroup
    \ltx@IfUndefined{HoLogoFont@logo@#1.#2}{%
      \ltx@IfUndefined{HoLogoFont@font@#2}{%
        \@PackageWarning{hologo}{%
          Missing font `#2' for logo `#1'%
        }%
        #3%
      }{%
        \csname HoLogoFont@font@#2\endcsname{#3}%
      }%
    }{%
      \csname HoLogoFont@logo@#1.#2\endcsname{#3}%
    }%
  \endgroup
}
%    \end{macrocode}
%    \end{macro}
%
%    \begin{macro}{\HoLogoFont@Def}
%    \begin{macrocode}
\def\HoLogoFont@Def#1{%
  \expandafter\def\csname HoLogoFont@font@#1\endcsname
}
%    \end{macrocode}
%    \end{macro}
%    \begin{macro}{\HoLogoFont@LogoDef}
%    \begin{macrocode}
\def\HoLogoFont@LogoDef#1#2{%
  \expandafter\def\csname HoLogoFont@logo@#1.#2\endcsname
}
%    \end{macrocode}
%    \end{macro}
%
% \subsubsection{Font defaults}
%
%    \begin{macro}{\HoLogoFont@font@general}
%    \begin{macrocode}
\HoLogoFont@Def{general}{}%
%    \end{macrocode}
%    \end{macro}
%
%    \begin{macro}{\HoLogoFont@font@rm}
%    \begin{macrocode}
\ltx@IfUndefined{rmfamily}{%
  \ltx@IfUndefined{rm}{%
  }{%
    \HoLogoFont@Def{rm}{\rm}%
  }%
}{%
  \HoLogoFont@Def{rm}{\rmfamily}%
}
%    \end{macrocode}
%    \end{macro}
%
%    \begin{macro}{\HoLogoFont@font@sf}
%    \begin{macrocode}
\ltx@IfUndefined{sffamily}{%
  \ltx@IfUndefined{sf}{%
  }{%
    \HoLogoFont@Def{sf}{\sf}%
  }%
}{%
  \HoLogoFont@Def{sf}{\sffamily}%
}
%    \end{macrocode}
%    \end{macro}
%
%    \begin{macro}{\HoLogoFont@font@bibsf}
%    In case of \hologo{plainTeX} the original small caps
%    variant is used as default. In \hologo{LaTeX}
%    the definition of package \xpackage{dtklogos} \cite{dtklogos}
%    is used.
%\begin{quote}
%\begin{verbatim}
%\DeclareRobustCommand{\BibTeX}{%
%  B%
%  \kern-.05em%
%  \hbox{%
%    $\m@th$% %% force math size calculations
%    \csname S@\f@size\endcsname
%    \fontsize\sf@size\z@
%    \math@fontsfalse
%    \selectfont
%    I%
%    \kern-.025em%
%    B
%  }%
%  \kern-.08em%
%  \-%
%  \TeX
%}
%\end{verbatim}
%\end{quote}
%    \begin{macrocode}
\ltx@IfUndefined{selectfont}{%
  \ltx@IfUndefined{tensc}{%
    \font\tensc=cmcsc10\relax
  }{}%
  \HoLogoFont@Def{bibsf}{\tensc}%
}{%
  \HoLogoFont@Def{bibsf}{%
    $\mathsurround=0pt$%
    \csname S@\f@size\endcsname
    \fontsize\sf@size{0pt}%
    \math@fontsfalse
    \selectfont
  }%
}
%    \end{macrocode}
%    \end{macro}
%
%    \begin{macro}{\HoLogoFont@font@sc}
%    \begin{macrocode}
\ltx@IfUndefined{scshape}{%
  \ltx@IfUndefined{tensc}{%
    \font\tensc=cmcsc10\relax
  }{}%
  \HoLogoFont@Def{sc}{\tensc}%
}{%
  \HoLogoFont@Def{sc}{\scshape}%
}
%    \end{macrocode}
%    \end{macro}
%
%    \begin{macro}{\HoLogoFont@font@sy}
%    \begin{macrocode}
\ltx@IfUndefined{usefont}{%
  \ltx@IfUndefined{tensy}{%
  }{%
    \HoLogoFont@Def{sy}{\tensy}%
  }%
}{%
  \HoLogoFont@Def{sy}{%
    \usefont{OMS}{cmsy}{m}{n}%
  }%
}
%    \end{macrocode}
%    \end{macro}
%
%    \begin{macro}{\HoLogoFont@font@logo}
%    \begin{macrocode}
\begingroup
  \def\x{LaTeX2e}%
\expandafter\endgroup
\ifx\fmtname\x
  \ltx@IfUndefined{logofamily}{%
    \DeclareRobustCommand\logofamily{%
      \not@math@alphabet\logofamily\relax
      \fontencoding{U}%
      \fontfamily{logo}%
      \selectfont
    }%
  }{}%
  \ltx@IfUndefined{logofamily}{%
  }{%
    \HoLogoFont@Def{logo}{\logofamily}%
  }%
\else
  \ltx@IfUndefined{tenlogo}{%
    \font\tenlogo=logo10\relax
  }{}%
  \HoLogoFont@Def{logo}{\tenlogo}%
\fi
%    \end{macrocode}
%    \end{macro}
%
% \subsubsection{Font setup}
%
%    \begin{macro}{\hologoFontSetup}
%    \begin{macrocode}
\def\hologoFontSetup{%
  \let\HOLOGO@name\relax
  \HOLOGO@FontSetup
}
%    \end{macrocode}
%    \end{macro}
%
%    \begin{macro}{\hologoLogoFontSetup}
%    \begin{macrocode}
\def\hologoLogoFontSetup#1{%
  \edef\HOLOGO@name{#1}%
  \ltx@IfUndefined{HoLogo@\HOLOGO@name}{%
    \@PackageError{hologo}{%
      Unknown logo `\HOLOGO@name'%
    }\@ehc
    \ltx@gobble
  }{%
    \HOLOGO@FontSetup
  }%
}
%    \end{macrocode}
%    \end{macro}
%
%    \begin{macro}{\HOLOGO@FontSetup}
%    \begin{macrocode}
\def\HOLOGO@FontSetup{%
  \kvsetkeys{HoLogoFont}%
}
%    \end{macrocode}
%    \end{macro}
%
%    \begin{macrocode}
\def\HOLOGO@temp#1{%
  \kv@define@key{HoLogoFont}{#1}{%
    \ifx\HOLOGO@name\relax
      \HoLogoFont@Def{#1}{##1}%
    \else
      \HoLogoFont@LogoDef\HOLOGO@name{#1}{##1}%
    \fi
  }%
}
\HOLOGO@temp{general}
\HOLOGO@temp{sf}
%    \end{macrocode}
%
% \subsection{Generic logo commands}
%
%    \begin{macrocode}
\HOLOGO@IfExists\hologo{%
  \@PackageError{hologo}{%
    \string\hologo\ltx@space is already defined.\MessageBreak
    Package loading is aborted%
  }\@ehc
  \HOLOGO@AtEnd
}%
\HOLOGO@IfExists\hologoRobust{%
  \@PackageError{hologo}{%
    \string\hologoRobust\ltx@space is already defined.\MessageBreak
    Package loading is aborted%
  }\@ehc
  \HOLOGO@AtEnd
}%
%    \end{macrocode}
%
% \subsubsection{\cs{hologo} and friends}
%
%    \begin{macrocode}
\ifluatex
  \expandafter\ltx@firstofone
\else
  \expandafter\ltx@gobble
\fi
{%
  \ltx@IfUndefined{ifincsname}{%
    \ifnum\luatexversion<36 %
      \expandafter\ltx@gobble
    \else
      \expandafter\ltx@firstofone
    \fi
    {%
      \begingroup
        \ifcase0%
            \directlua{%
              if tex.enableprimitives then %
                tex.enableprimitives('HOLOGO@', {'ifincsname'})%
              else %
                tex.print('1')%
              end%
            }%
            \ifx\HOLOGO@ifincsname\@undefined 1\fi%
            \relax
          \expandafter\ltx@firstofone
        \else
          \endgroup
          \expandafter\ltx@gobble
        \fi
        {%
          \global\let\ifincsname\HOLOGO@ifincsname
        }%
      \HOLOGO@temp
    }%
  }{}%
}
%    \end{macrocode}
%    \begin{macrocode}
\ltx@IfUndefined{ifincsname}{%
  \catcode`$=14 %
}{%
  \catcode`$=9 %
}
%    \end{macrocode}
%
%    \begin{macro}{\hologo}
%    \begin{macrocode}
\def\hologo#1{%
$ \ifincsname
$   \ltx@ifundefined{HoLogoCs@\HOLOGO@Variant{#1}}{%
$     #1%
$   }{%
$     \csname HoLogoCs@\HOLOGO@Variant{#1}\endcsname\ltx@firstoftwo
$   }%
$ \else
    \HOLOGO@IfExists\texorpdfstring\texorpdfstring\ltx@firstoftwo
    {%
      \hologoRobust{#1}%
    }{%
      \ltx@ifundefined{HoLogoBkm@\HOLOGO@Variant{#1}}{%
        \ltx@ifundefined{HoLogo@#1}{?#1?}{#1}%
      }{%
        \csname HoLogoBkm@\HOLOGO@Variant{#1}\endcsname
        \ltx@firstoftwo
      }%
    }%
$ \fi
}
%    \end{macrocode}
%    \end{macro}
%    \begin{macro}{\Hologo}
%    \begin{macrocode}
\def\Hologo#1{%
$ \ifincsname
$   \ltx@ifundefined{HoLogoCs@\HOLOGO@Variant{#1}}{%
$     #1%
$   }{%
$     \csname HoLogoCs@\HOLOGO@Variant{#1}\endcsname\ltx@secondoftwo
$   }%
$ \else
    \HOLOGO@IfExists\texorpdfstring\texorpdfstring\ltx@firstoftwo
    {%
      \HologoRobust{#1}%
    }{%
      \ltx@ifundefined{HoLogoBkm@\HOLOGO@Variant{#1}}{%
        \ltx@ifundefined{HoLogo@#1}{?#1?}{#1}%
      }{%
        \csname HoLogoBkm@\HOLOGO@Variant{#1}\endcsname
        \ltx@secondoftwo
      }%
    }%
$ \fi
}
%    \end{macrocode}
%    \end{macro}
%
%    \begin{macro}{\hologoVariant}
%    \begin{macrocode}
\def\hologoVariant#1#2{%
  \ifx\relax#2\relax
    \hologo{#1}%
  \else
$   \ifincsname
$     \ltx@ifundefined{HoLogoCs@#1@#2}{%
$       #1%
$     }{%
$       \csname HoLogoCs@#1@#2\endcsname\ltx@firstoftwo
$     }%
$   \else
      \HOLOGO@IfExists\texorpdfstring\texorpdfstring\ltx@firstoftwo
      {%
        \hologoVariantRobust{#1}{#2}%
      }{%
        \ltx@ifundefined{HoLogoBkm@#1@#2}{%
          \ltx@ifundefined{HoLogo@#1}{?#1?}{#1}%
        }{%
          \csname HoLogoBkm@#1@#2\endcsname
          \ltx@firstoftwo
        }%
      }%
$   \fi
  \fi
}
%    \end{macrocode}
%    \end{macro}
%    \begin{macro}{\HologoVariant}
%    \begin{macrocode}
\def\HologoVariant#1#2{%
  \ifx\relax#2\relax
    \Hologo{#1}%
  \else
$   \ifincsname
$     \ltx@ifundefined{HoLogoCs@#1@#2}{%
$       #1%
$     }{%
$       \csname HoLogoCs@#1@#2\endcsname\ltx@secondoftwo
$     }%
$   \else
      \HOLOGO@IfExists\texorpdfstring\texorpdfstring\ltx@firstoftwo
      {%
        \HologoVariantRobust{#1}{#2}%
      }{%
        \ltx@ifundefined{HoLogoBkm@#1@#2}{%
          \ltx@ifundefined{HoLogo@#1}{?#1?}{#1}%
        }{%
          \csname HoLogoBkm@#1@#2\endcsname
          \ltx@secondoftwo
        }%
      }%
$   \fi
  \fi
}
%    \end{macrocode}
%    \end{macro}
%
%    \begin{macrocode}
\catcode`\$=3 %
%    \end{macrocode}
%
% \subsubsection{\cs{hologoRobust} and friends}
%
%    \begin{macro}{\hologoRobust}
%    \begin{macrocode}
\ltx@IfUndefined{protected}{%
  \ltx@IfUndefined{DeclareRobustCommand}{%
    \def\hologoRobust#1%
  }{%
    \DeclareRobustCommand*\hologoRobust[1]%
  }%
}{%
  \protected\def\hologoRobust#1%
}%
{%
  \edef\HOLOGO@name{#1}%
  \ltx@IfUndefined{HoLogo@\HOLOGO@Variant\HOLOGO@name}{%
    \@PackageError{hologo}{%
      Unknown logo `\HOLOGO@name'%
    }\@ehc
    ?\HOLOGO@name?%
  }{%
    \ltx@IfUndefined{ver@tex4ht.sty}{%
      \HoLogoFont@font\HOLOGO@name{general}{%
        \csname HoLogo@\HOLOGO@Variant\HOLOGO@name\endcsname
        \ltx@firstoftwo
      }%
    }{%
      \ltx@IfUndefined{HoLogoHtml@\HOLOGO@Variant\HOLOGO@name}{%
        \HOLOGO@name
      }{%
        \csname HoLogoHtml@\HOLOGO@Variant\HOLOGO@name\endcsname
        \ltx@firstoftwo
      }%
    }%
  }%
}
%    \end{macrocode}
%    \end{macro}
%    \begin{macro}{\HologoRobust}
%    \begin{macrocode}
\ltx@IfUndefined{protected}{%
  \ltx@IfUndefined{DeclareRobustCommand}{%
    \def\HologoRobust#1%
  }{%
    \DeclareRobustCommand*\HologoRobust[1]%
  }%
}{%
  \protected\def\HologoRobust#1%
}%
{%
  \edef\HOLOGO@name{#1}%
  \ltx@IfUndefined{HoLogo@\HOLOGO@Variant\HOLOGO@name}{%
    \@PackageError{hologo}{%
      Unknown logo `\HOLOGO@name'%
    }\@ehc
    ?\HOLOGO@name?%
  }{%
    \ltx@IfUndefined{ver@tex4ht.sty}{%
      \HoLogoFont@font\HOLOGO@name{general}{%
        \csname HoLogo@\HOLOGO@Variant\HOLOGO@name\endcsname
        \ltx@secondoftwo
      }%
    }{%
      \ltx@IfUndefined{HoLogoHtml@\HOLOGO@Variant\HOLOGO@name}{%
        \expandafter\HOLOGO@Uppercase\HOLOGO@name
      }{%
        \csname HoLogoHtml@\HOLOGO@Variant\HOLOGO@name\endcsname
        \ltx@secondoftwo
      }%
    }%
  }%
}
%    \end{macrocode}
%    \end{macro}
%    \begin{macro}{\hologoVariantRobust}
%    \begin{macrocode}
\ltx@IfUndefined{protected}{%
  \ltx@IfUndefined{DeclareRobustCommand}{%
    \def\hologoVariantRobust#1#2%
  }{%
    \DeclareRobustCommand*\hologoVariantRobust[2]%
  }%
}{%
  \protected\def\hologoVariantRobust#1#2%
}%
{%
  \begingroup
    \hologoLogoSetup{#1}{variant={#2}}%
    \hologoRobust{#1}%
  \endgroup
}
%    \end{macrocode}
%    \end{macro}
%    \begin{macro}{\HologoVariantRobust}
%    \begin{macrocode}
\ltx@IfUndefined{protected}{%
  \ltx@IfUndefined{DeclareRobustCommand}{%
    \def\HologoVariantRobust#1#2%
  }{%
    \DeclareRobustCommand*\HologoVariantRobust[2]%
  }%
}{%
  \protected\def\HologoVariantRobust#1#2%
}%
{%
  \begingroup
    \hologoLogoSetup{#1}{variant={#2}}%
    \HologoRobust{#1}%
  \endgroup
}
%    \end{macrocode}
%    \end{macro}
%
%    \begin{macro}{\hologorobust}
%    Macro \cs{hologorobust} is only defined for compatibility.
%    Its use is deprecated.
%    \begin{macrocode}
\def\hologorobust{\hologoRobust}
%    \end{macrocode}
%    \end{macro}
%
% \subsection{Helpers}
%
%    \begin{macro}{\HOLOGO@Uppercase}
%    Macro \cs{HOLOGO@Uppercase} is restricted to \cs{uppercase},
%    because \hologo{plainTeX} or \hologo{iniTeX} do not provide
%    \cs{MakeUppercase}.
%    \begin{macrocode}
\def\HOLOGO@Uppercase#1{\uppercase{#1}}
%    \end{macrocode}
%    \end{macro}
%
%    \begin{macro}{\HOLOGO@PdfdocUnicode}
%    \begin{macrocode}
\def\HOLOGO@PdfdocUnicode{%
  \ifx\ifHy@unicode\iftrue
    \expandafter\ltx@secondoftwo
  \else
    \expandafter\ltx@firstoftwo
  \fi
}
%    \end{macrocode}
%    \end{macro}
%
%    \begin{macro}{\HOLOGO@Math}
%    \begin{macrocode}
\def\HOLOGO@MathSetup{%
  \mathsurround0pt\relax
  \HOLOGO@IfExists\f@series{%
    \if b\expandafter\ltx@car\f@series x\@nil
      \csname boldmath\endcsname
   \fi
  }{}%
}
%    \end{macrocode}
%    \end{macro}
%
%    \begin{macro}{\HOLOGO@TempDimen}
%    \begin{macrocode}
\dimendef\HOLOGO@TempDimen=\ltx@zero
%    \end{macrocode}
%    \end{macro}
%    \begin{macro}{\HOLOGO@NegativeKerning}
%    \begin{macrocode}
\def\HOLOGO@NegativeKerning#1{%
  \begingroup
    \HOLOGO@TempDimen=0pt\relax
    \comma@parse@normalized{#1}{%
      \ifdim\HOLOGO@TempDimen=0pt %
        \expandafter\HOLOGO@@NegativeKerning\comma@entry
      \fi
      \ltx@gobble
    }%
    \ifdim\HOLOGO@TempDimen<0pt %
      \kern\HOLOGO@TempDimen
    \fi
  \endgroup
}
%    \end{macrocode}
%    \end{macro}
%    \begin{macro}{\HOLOGO@@NegativeKerning}
%    \begin{macrocode}
\def\HOLOGO@@NegativeKerning#1#2{%
  \setbox\ltx@zero\hbox{#1#2}%
  \HOLOGO@TempDimen=\wd\ltx@zero
  \setbox\ltx@zero\hbox{#1\kern0pt#2}%
  \advance\HOLOGO@TempDimen by -\wd\ltx@zero
}
%    \end{macrocode}
%    \end{macro}
%
%    \begin{macro}{\HOLOGO@SpaceFactor}
%    \begin{macrocode}
\def\HOLOGO@SpaceFactor{%
  \spacefactor1000 %
}
%    \end{macrocode}
%    \end{macro}
%
%    \begin{macro}{\HOLOGO@Span}
%    \begin{macrocode}
\def\HOLOGO@Span#1#2{%
  \HCode{<span class="HoLogo-#1">}%
  #2%
  \HCode{</span>}%
}
%    \end{macrocode}
%    \end{macro}
%
% \subsubsection{Text subscript}
%
%    \begin{macro}{\HOLOGO@SubScript}%
%    \begin{macrocode}
\def\HOLOGO@SubScript#1{%
  \ltx@IfUndefined{textsubscript}{%
    \ltx@IfUndefined{text}{%
      \ltx@mbox{%
        \mathsurround=0pt\relax
        $%
          _{%
            \ltx@IfUndefined{sf@size}{%
              \mathrm{#1}%
            }{%
              \mbox{%
                \fontsize\sf@size{0pt}\selectfont
                #1%
              }%
            }%
          }%
        $%
      }%
    }{%
      \ltx@mbox{%
        \mathsurround=0pt\relax
        $_{\text{#1}}$%
      }%
    }%
  }{%
    \textsubscript{#1}%
  }%
}
%    \end{macrocode}
%    \end{macro}
%
% \subsection{\hologo{TeX} and friends}
%
% \subsubsection{\hologo{TeX}}
%
%    \begin{macro}{\HoLogo@TeX}
%    Source: \hologo{LaTeX} kernel.
%    \begin{macrocode}
\def\HoLogo@TeX#1{%
  T\kern-.1667em\lower.5ex\hbox{E}\kern-.125emX\HOLOGO@SpaceFactor
}
%    \end{macrocode}
%    \end{macro}
%    \begin{macro}{\HoLogoHtml@TeX}
%    \begin{macrocode}
\def\HoLogoHtml@TeX#1{%
  \HoLogoCss@TeX
  \HOLOGO@Span{TeX}{%
    T%
    \HOLOGO@Span{e}{%
      E%
    }%
    X%
  }%
}
%    \end{macrocode}
%    \end{macro}
%    \begin{macro}{\HoLogoCss@TeX}
%    \begin{macrocode}
\def\HoLogoCss@TeX{%
  \Css{%
    span.HoLogo-TeX span.HoLogo-e{%
      position:relative;%
      top:.5ex;%
      margin-left:-.1667em;%
      margin-right:-.125em;%
    }%
  }%
  \Css{%
    a span.HoLogo-TeX span.HoLogo-e{%
      text-decoration:none;%
    }%
  }%
  \global\let\HoLogoCss@TeX\relax
}
%    \end{macrocode}
%    \end{macro}
%
% \subsubsection{\hologo{plainTeX}}
%
%    \begin{macro}{\HoLogo@plainTeX@space}
%    Source: ``The \hologo{TeX}book''
%    \begin{macrocode}
\def\HoLogo@plainTeX@space#1{%
  \HOLOGO@mbox{#1{p}{P}lain}\HOLOGO@space\hologo{TeX}%
}
%    \end{macrocode}
%    \end{macro}
%    \begin{macro}{\HoLogoCs@plainTeX@space}
%    \begin{macrocode}
\def\HoLogoCs@plainTeX@space#1{#1{p}{P}lain TeX}%
%    \end{macrocode}
%    \end{macro}
%    \begin{macro}{\HoLogoBkm@plainTeX@space}
%    \begin{macrocode}
\def\HoLogoBkm@plainTeX@space#1{%
  #1{p}{P}lain \hologo{TeX}%
}
%    \end{macrocode}
%    \end{macro}
%    \begin{macro}{\HoLogoHtml@plainTeX@space}
%    \begin{macrocode}
\def\HoLogoHtml@plainTeX@space#1{%
  #1{p}{P}lain \hologo{TeX}%
}
%    \end{macrocode}
%    \end{macro}
%
%    \begin{macro}{\HoLogo@plainTeX@hyphen}
%    \begin{macrocode}
\def\HoLogo@plainTeX@hyphen#1{%
  \HOLOGO@mbox{#1{p}{P}lain}\HOLOGO@hyphen\hologo{TeX}%
}
%    \end{macrocode}
%    \end{macro}
%    \begin{macro}{\HoLogoCs@plainTeX@hyphen}
%    \begin{macrocode}
\def\HoLogoCs@plainTeX@hyphen#1{#1{p}{P}lain-TeX}
%    \end{macrocode}
%    \end{macro}
%    \begin{macro}{\HoLogoBkm@plainTeX@hyphen}
%    \begin{macrocode}
\def\HoLogoBkm@plainTeX@hyphen#1{%
  #1{p}{P}lain-\hologo{TeX}%
}
%    \end{macrocode}
%    \end{macro}
%    \begin{macro}{\HoLogoHtml@plainTeX@hyphen}
%    \begin{macrocode}
\def\HoLogoHtml@plainTeX@hyphen#1{%
  #1{p}{P}lain-\hologo{TeX}%
}
%    \end{macrocode}
%    \end{macro}
%
%    \begin{macro}{\HoLogo@plainTeX@runtogether}
%    \begin{macrocode}
\def\HoLogo@plainTeX@runtogether#1{%
  \HOLOGO@mbox{#1{p}{P}lain\hologo{TeX}}%
}
%    \end{macrocode}
%    \end{macro}
%    \begin{macro}{\HoLogoCs@plainTeX@runtogether}
%    \begin{macrocode}
\def\HoLogoCs@plainTeX@runtogether#1{#1{p}{P}lainTeX}
%    \end{macrocode}
%    \end{macro}
%    \begin{macro}{\HoLogoBkm@plainTeX@runtogether}
%    \begin{macrocode}
\def\HoLogoBkm@plainTeX@runtogether#1{%
  #1{p}{P}lain\hologo{TeX}%
}
%    \end{macrocode}
%    \end{macro}
%    \begin{macro}{\HoLogoHtml@plainTeX@runtogether}
%    \begin{macrocode}
\def\HoLogoHtml@plainTeX@runtogether#1{%
  #1{p}{P}lain\hologo{TeX}%
}
%    \end{macrocode}
%    \end{macro}
%
%    \begin{macro}{\HoLogo@plainTeX}
%    \begin{macrocode}
\def\HoLogo@plainTeX{\HoLogo@plainTeX@space}
%    \end{macrocode}
%    \end{macro}
%    \begin{macro}{\HoLogoCs@plainTeX}
%    \begin{macrocode}
\def\HoLogoCs@plainTeX{\HoLogoCs@plainTeX@space}
%    \end{macrocode}
%    \end{macro}
%    \begin{macro}{\HoLogoBkm@plainTeX}
%    \begin{macrocode}
\def\HoLogoBkm@plainTeX{\HoLogoBkm@plainTeX@space}
%    \end{macrocode}
%    \end{macro}
%    \begin{macro}{\HoLogoHtml@plainTeX}
%    \begin{macrocode}
\def\HoLogoHtml@plainTeX{\HoLogoHtml@plainTeX@space}
%    \end{macrocode}
%    \end{macro}
%
% \subsubsection{\hologo{LaTeX}}
%
%    Source: \hologo{LaTeX} kernel.
%\begin{quote}
%\begin{verbatim}
%\DeclareRobustCommand{\LaTeX}{%
%  L%
%  \kern-.36em%
%  {%
%    \sbox\z@ T%
%    \vbox to\ht\z@{%
%      \hbox{%
%        \check@mathfonts
%        \fontsize\sf@size\z@
%        \math@fontsfalse
%        \selectfont
%        A%
%      }%
%      \vss
%    }%
%  }%
%  \kern-.15em%
%  \TeX
%}
%\end{verbatim}
%\end{quote}
%
%    \begin{macro}{\HoLogo@La}
%    \begin{macrocode}
\def\HoLogo@La#1{%
  L%
  \kern-.36em%
  \begingroup
    \setbox\ltx@zero\hbox{T}%
    \vbox to\ht\ltx@zero{%
      \hbox{%
        \ltx@ifundefined{check@mathfonts}{%
          \csname sevenrm\endcsname
        }{%
          \check@mathfonts
          \fontsize\sf@size{0pt}%
          \math@fontsfalse\selectfont
        }%
        A%
      }%
      \vss
    }%
  \endgroup
}
%    \end{macrocode}
%    \end{macro}
%
%    \begin{macro}{\HoLogo@LaTeX}
%    Source: \hologo{LaTeX} kernel.
%    \begin{macrocode}
\def\HoLogo@LaTeX#1{%
  \hologo{La}%
  \kern-.15em%
  \hologo{TeX}%
}
%    \end{macrocode}
%    \end{macro}
%    \begin{macro}{\HoLogoHtml@LaTeX}
%    \begin{macrocode}
\def\HoLogoHtml@LaTeX#1{%
  \HoLogoCss@LaTeX
  \HOLOGO@Span{LaTeX}{%
    L%
    \HOLOGO@Span{a}{%
      A%
    }%
    \hologo{TeX}%
  }%
}
%    \end{macrocode}
%    \end{macro}
%    \begin{macro}{\HoLogoCss@LaTeX}
%    \begin{macrocode}
\def\HoLogoCss@LaTeX{%
  \Css{%
    span.HoLogo-LaTeX span.HoLogo-a{%
      position:relative;%
      top:-.5ex;%
      margin-left:-.36em;%
      margin-right:-.15em;%
      font-size:85\%;%
    }%
  }%
  \global\let\HoLogoCss@LaTeX\relax
}
%    \end{macrocode}
%    \end{macro}
%
% \subsubsection{\hologo{(La)TeX}}
%
%    \begin{macro}{\HoLogo@LaTeXTeX}
%    The kerning around the parentheses is taken
%    from package \xpackage{dtklogos} \cite{dtklogos}.
%\begin{quote}
%\begin{verbatim}
%\DeclareRobustCommand{\LaTeXTeX}{%
%  (%
%  \kern-.15em%
%  L%
%  \kern-.36em%
%  {%
%    \sbox\z@ T%
%    \vbox to\ht0{%
%      \hbox{%
%        $\m@th$%
%        \csname S@\f@size\endcsname
%        \fontsize\sf@size\z@
%        \math@fontsfalse
%        \selectfont
%        A%
%      }%
%      \vss
%    }%
%  }%
%  \kern-.2em%
%  )%
%  \kern-.15em%
%  \TeX
%}
%\end{verbatim}
%\end{quote}
%    \begin{macrocode}
\def\HoLogo@LaTeXTeX#1{%
  (%
  \kern-.15em%
  \hologo{La}%
  \kern-.2em%
  )%
  \kern-.15em%
  \hologo{TeX}%
}
%    \end{macrocode}
%    \end{macro}
%    \begin{macro}{\HoLogoBkm@LaTeXTeX}
%    \begin{macrocode}
\def\HoLogoBkm@LaTeXTeX#1{(La)TeX}
%    \end{macrocode}
%    \end{macro}
%
%    \begin{macro}{\HoLogo@(La)TeX}
%    \begin{macrocode}
\expandafter
\let\csname HoLogo@(La)TeX\endcsname\HoLogo@LaTeXTeX
%    \end{macrocode}
%    \end{macro}
%    \begin{macro}{\HoLogoBkm@(La)TeX}
%    \begin{macrocode}
\expandafter
\let\csname HoLogoBkm@(La)TeX\endcsname\HoLogoBkm@LaTeXTeX
%    \end{macrocode}
%    \end{macro}
%    \begin{macro}{\HoLogoHtml@LaTeXTeX}
%    \begin{macrocode}
\def\HoLogoHtml@LaTeXTeX#1{%
  \HoLogoCss@LaTeXTeX
  \HOLOGO@Span{LaTeXTeX}{%
    (%
    \HOLOGO@Span{L}{L}%
    \HOLOGO@Span{a}{A}%
    \HOLOGO@Span{ParenRight}{)}%
    \hologo{TeX}%
  }%
}
%    \end{macrocode}
%    \end{macro}
%    \begin{macro}{\HoLogoHtml@(La)TeX}
%    Kerning after opening parentheses and before closing parentheses
%    is $-0.1$\,em. The original values $-0.15$\,em
%    looked too ugly for a serif font.
%    \begin{macrocode}
\expandafter
\let\csname HoLogoHtml@(La)TeX\endcsname\HoLogoHtml@LaTeXTeX
%    \end{macrocode}
%    \end{macro}
%    \begin{macro}{\HoLogoCss@LaTeXTeX}
%    \begin{macrocode}
\def\HoLogoCss@LaTeXTeX{%
  \Css{%
    span.HoLogo-LaTeXTeX span.HoLogo-L{%
      margin-left:-.1em;%
    }%
  }%
  \Css{%
    span.HoLogo-LaTeXTeX span.HoLogo-a{%
      position:relative;%
      top:-.5ex;%
      margin-left:-.36em;%
      margin-right:-.1em;%
      font-size:85\%;%
    }%
  }%
  \Css{%
    span.HoLogo-LaTeXTeX span.HoLogo-ParenRight{%
      margin-right:-.15em;%
    }%
  }%
  \global\let\HoLogoCss@LaTeXTeX\relax
}
%    \end{macrocode}
%    \end{macro}
%
% \subsubsection{\hologo{LaTeXe}}
%
%    \begin{macro}{\HoLogo@LaTeXe}
%    Source: \hologo{LaTeX} kernel
%    \begin{macrocode}
\def\HoLogo@LaTeXe#1{%
  \hologo{LaTeX}%
  \kern.15em%
  \hbox{%
    \HOLOGO@MathSetup
    2%
    $_{\textstyle\varepsilon}$%
  }%
}
%    \end{macrocode}
%    \end{macro}
%
%    \begin{macro}{\HoLogoCs@LaTeXe}
%    \begin{macrocode}
\ifnum64=`\^^^^0040\relax % test for big chars of LuaTeX/XeTeX
  \catcode`\$=9 %
  \catcode`\&=14 %
\else
  \catcode`\$=14 %
  \catcode`\&=9 %
\fi
\def\HoLogoCs@LaTeXe#1{%
  LaTeX2%
$ \string ^^^^0395%
& e%
}%
\catcode`\$=3 %
\catcode`\&=4 %
%    \end{macrocode}
%    \end{macro}
%
%    \begin{macro}{\HoLogoBkm@LaTeXe}
%    \begin{macrocode}
\def\HoLogoBkm@LaTeXe#1{%
  \hologo{LaTeX}%
  2%
  \HOLOGO@PdfdocUnicode{e}{\textepsilon}%
}
%    \end{macrocode}
%    \end{macro}
%
%    \begin{macro}{\HoLogoHtml@LaTeXe}
%    \begin{macrocode}
\def\HoLogoHtml@LaTeXe#1{%
  \HoLogoCss@LaTeXe
  \HOLOGO@Span{LaTeX2e}{%
    \hologo{LaTeX}%
    \HOLOGO@Span{2}{2}%
    \HOLOGO@Span{e}{%
      \HOLOGO@MathSetup
      \ensuremath{\textstyle\varepsilon}%
    }%
  }%
}
%    \end{macrocode}
%    \end{macro}
%    \begin{macro}{\HoLogoCss@LaTeXe}
%    \begin{macrocode}
\def\HoLogoCss@LaTeXe{%
  \Css{%
    span.HoLogo-LaTeX2e span.HoLogo-2{%
      padding-left:.15em;%
    }%
  }%
  \Css{%
    span.HoLogo-LaTeX2e span.HoLogo-e{%
      position:relative;%
      top:.35ex;%
      text-decoration:none;%
    }%
  }%
  \global\let\HoLogoCss@LaTeXe\relax
}
%    \end{macrocode}
%    \end{macro}
%
%    \begin{macro}{\HoLogo@LaTeX2e}
%    \begin{macrocode}
\expandafter
\let\csname HoLogo@LaTeX2e\endcsname\HoLogo@LaTeXe
%    \end{macrocode}
%    \end{macro}
%    \begin{macro}{\HoLogoCs@LaTeX2e}
%    \begin{macrocode}
\expandafter
\let\csname HoLogoCs@LaTeX2e\endcsname\HoLogoCs@LaTeXe
%    \end{macrocode}
%    \end{macro}
%    \begin{macro}{\HoLogoBkm@LaTeX2e}
%    \begin{macrocode}
\expandafter
\let\csname HoLogoBkm@LaTeX2e\endcsname\HoLogoBkm@LaTeXe
%    \end{macrocode}
%    \end{macro}
%    \begin{macro}{\HoLogoHtml@LaTeX2e}
%    \begin{macrocode}
\expandafter
\let\csname HoLogoHtml@LaTeX2e\endcsname\HoLogoHtml@LaTeXe
%    \end{macrocode}
%    \end{macro}
%
% \subsubsection{\hologo{LaTeX3}}
%
%    \begin{macro}{\HoLogo@LaTeX3}
%    Source: \hologo{LaTeX} kernel
%    \begin{macrocode}
\expandafter\def\csname HoLogo@LaTeX3\endcsname#1{%
  \hologo{LaTeX}%
  3%
}
%    \end{macrocode}
%    \end{macro}
%
%    \begin{macro}{\HoLogoBkm@LaTeX3}
%    \begin{macrocode}
\expandafter\def\csname HoLogoBkm@LaTeX3\endcsname#1{%
  \hologo{LaTeX}%
  3%
}
%    \end{macrocode}
%    \end{macro}
%    \begin{macro}{\HoLogoHtml@LaTeX3}
%    \begin{macrocode}
\expandafter
\let\csname HoLogoHtml@LaTeX3\expandafter\endcsname
\csname HoLogo@LaTeX3\endcsname
%    \end{macrocode}
%    \end{macro}
%
% \subsubsection{\hologo{LaTeXML}}
%
%    \begin{macro}{\HoLogo@LaTeXML}
%    \begin{macrocode}
\def\HoLogo@LaTeXML#1{%
  \HOLOGO@mbox{%
    \hologo{La}%
    \kern-.15em%
    T%
    \kern-.1667em%
    \lower.5ex\hbox{E}%
    \kern-.125em%
    \HoLogoFont@font{LaTeXML}{sc}{xml}%
  }%
}
%    \end{macrocode}
%    \end{macro}
%    \begin{macro}{\HoLogoHtml@pdfLaTeX}
%    \begin{macrocode}
\def\HoLogoHtml@LaTeXML#1{%
  \HOLOGO@Span{LaTeXML}{%
    \HoLogoCss@LaTeX
    \HoLogoCss@TeX
    \HOLOGO@Span{LaTeX}{%
      L%
      \HOLOGO@Span{a}{%
        A%
      }%
    }%
    \HOLOGO@Span{TeX}{%
      T%
      \HOLOGO@Span{e}{%
        E%
      }%
    }%
    \HCode{<span style="font-variant: small-caps;">}%
    xml%
    \HCode{</span>}%
  }%
}
%    \end{macrocode}
%    \end{macro}
%
% \subsubsection{\hologo{eTeX}}
%
%    \begin{macro}{\HoLogo@eTeX}
%    Source: package \xpackage{etex}
%    \begin{macrocode}
\def\HoLogo@eTeX#1{%
  \ltx@mbox{%
    \HOLOGO@MathSetup
    $\varepsilon$%
    -%
    \HOLOGO@NegativeKerning{-T,T-,To}%
    \hologo{TeX}%
  }%
}
%    \end{macrocode}
%    \end{macro}
%    \begin{macro}{\HoLogoCs@eTeX}
%    \begin{macrocode}
\ifnum64=`\^^^^0040\relax % test for big chars of LuaTeX/XeTeX
  \catcode`\$=9 %
  \catcode`\&=14 %
\else
  \catcode`\$=14 %
  \catcode`\&=9 %
\fi
\def\HoLogoCs@eTeX#1{%
$ #1{\string ^^^^0395}{\string ^^^^03b5}%
& #1{e}{E}%
  TeX%
}%
\catcode`\$=3 %
\catcode`\&=4 %
%    \end{macrocode}
%    \end{macro}
%    \begin{macro}{\HoLogoBkm@eTeX}
%    \begin{macrocode}
\def\HoLogoBkm@eTeX#1{%
  \HOLOGO@PdfdocUnicode{#1{e}{E}}{\textepsilon}%
  -%
  \hologo{TeX}%
}
%    \end{macrocode}
%    \end{macro}
%    \begin{macro}{\HoLogoHtml@eTeX}
%    \begin{macrocode}
\def\HoLogoHtml@eTeX#1{%
  \ltx@mbox{%
    \HOLOGO@MathSetup
    $\varepsilon$%
    -%
    \hologo{TeX}%
  }%
}
%    \end{macrocode}
%    \end{macro}
%
% \subsubsection{\hologo{iniTeX}}
%
%    \begin{macro}{\HoLogo@iniTeX}
%    \begin{macrocode}
\def\HoLogo@iniTeX#1{%
  \HOLOGO@mbox{%
    #1{i}{I}ni\hologo{TeX}%
  }%
}
%    \end{macrocode}
%    \end{macro}
%    \begin{macro}{\HoLogoCs@iniTeX}
%    \begin{macrocode}
\def\HoLogoCs@iniTeX#1{#1{i}{I}niTeX}
%    \end{macrocode}
%    \end{macro}
%    \begin{macro}{\HoLogoBkm@iniTeX}
%    \begin{macrocode}
\def\HoLogoBkm@iniTeX#1{%
  #1{i}{I}ni\hologo{TeX}%
}
%    \end{macrocode}
%    \end{macro}
%    \begin{macro}{\HoLogoHtml@iniTeX}
%    \begin{macrocode}
\let\HoLogoHtml@iniTeX\HoLogo@iniTeX
%    \end{macrocode}
%    \end{macro}
%
% \subsubsection{\hologo{virTeX}}
%
%    \begin{macro}{\HoLogo@virTeX}
%    \begin{macrocode}
\def\HoLogo@virTeX#1{%
  \HOLOGO@mbox{%
    #1{v}{V}ir\hologo{TeX}%
  }%
}
%    \end{macrocode}
%    \end{macro}
%    \begin{macro}{\HoLogoCs@virTeX}
%    \begin{macrocode}
\def\HoLogoCs@virTeX#1{#1{v}{V}irTeX}
%    \end{macrocode}
%    \end{macro}
%    \begin{macro}{\HoLogoBkm@virTeX}
%    \begin{macrocode}
\def\HoLogoBkm@virTeX#1{%
  #1{v}{V}ir\hologo{TeX}%
}
%    \end{macrocode}
%    \end{macro}
%    \begin{macro}{\HoLogoHtml@virTeX}
%    \begin{macrocode}
\let\HoLogoHtml@virTeX\HoLogo@virTeX
%    \end{macrocode}
%    \end{macro}
%
% \subsubsection{\hologo{SliTeX}}
%
% \paragraph{Definitions of the three variants.}
%
%    \begin{macro}{\HoLogo@SLiTeX@lift}
%    \begin{macrocode}
\def\HoLogo@SLiTeX@lift#1{%
  \HoLogoFont@font{SliTeX}{rm}{%
    S%
    \kern-.06em%
    L%
    \kern-.18em%
    \raise.32ex\hbox{\HoLogoFont@font{SliTeX}{sc}{i}}%
    \HOLOGO@discretionary
    \kern-.06em%
    \hologo{TeX}%
  }%
}
%    \end{macrocode}
%    \end{macro}
%    \begin{macro}{\HoLogoBkm@SLiTeX@lift}
%    \begin{macrocode}
\def\HoLogoBkm@SLiTeX@lift#1{SLiTeX}
%    \end{macrocode}
%    \end{macro}
%    \begin{macro}{\HoLogoHtml@SLiTeX@lift}
%    \begin{macrocode}
\def\HoLogoHtml@SLiTeX@lift#1{%
  \HoLogoCss@SLiTeX@lift
  \HOLOGO@Span{SLiTeX-lift}{%
    \HoLogoFont@font{SliTeX}{rm}{%
      S%
      \HOLOGO@Span{L}{L}%
      \HOLOGO@Span{i}{i}%
      \hologo{TeX}%
    }%
  }%
}
%    \end{macrocode}
%    \end{macro}
%    \begin{macro}{\HoLogoCss@SLiTeX@lift}
%    \begin{macrocode}
\def\HoLogoCss@SLiTeX@lift{%
  \Css{%
    span.HoLogo-SLiTeX-lift span.HoLogo-L{%
      margin-left:-.06em;%
      margin-right:-.18em;%
    }%
  }%
  \Css{%
    span.HoLogo-SLiTeX-lift span.HoLogo-i{%
      position:relative;%
      top:-.32ex;%
      margin-right:-.06em;%
      font-variant:small-caps;%
    }%
  }%
  \global\let\HoLogoCss@SLiTeX@lift\relax
}
%    \end{macrocode}
%    \end{macro}
%
%    \begin{macro}{\HoLogo@SliTeX@simple}
%    \begin{macrocode}
\def\HoLogo@SliTeX@simple#1{%
  \HoLogoFont@font{SliTeX}{rm}{%
    \ltx@mbox{%
      \HoLogoFont@font{SliTeX}{sc}{Sli}%
    }%
    \HOLOGO@discretionary
    \hologo{TeX}%
  }%
}
%    \end{macrocode}
%    \end{macro}
%    \begin{macro}{\HoLogoBkm@SliTeX@simple}
%    \begin{macrocode}
\def\HoLogoBkm@SliTeX@simple#1{SliTeX}
%    \end{macrocode}
%    \end{macro}
%    \begin{macro}{\HoLogoHtml@SliTeX@simple}
%    \begin{macrocode}
\let\HoLogoHtml@SliTeX@simple\HoLogo@SliTeX@simple
%    \end{macrocode}
%    \end{macro}
%
%    \begin{macro}{\HoLogo@SliTeX@narrow}
%    \begin{macrocode}
\def\HoLogo@SliTeX@narrow#1{%
  \HoLogoFont@font{SliTeX}{rm}{%
    \ltx@mbox{%
      S%
      \kern-.06em%
      \HoLogoFont@font{SliTeX}{sc}{%
        l%
        \kern-.035em%
        i%
      }%
    }%
    \HOLOGO@discretionary
    \kern-.06em%
    \hologo{TeX}%
  }%
}
%    \end{macrocode}
%    \end{macro}
%    \begin{macro}{\HoLogoBkm@SliTeX@narrow}
%    \begin{macrocode}
\def\HoLogoBkm@SliTeX@narrow#1{SliTeX}
%    \end{macrocode}
%    \end{macro}
%    \begin{macro}{\HoLogoHtml@SliTeX@narrow}
%    \begin{macrocode}
\def\HoLogoHtml@SliTeX@narrow#1{%
  \HoLogoCss@SliTeX@narrow
  \HOLOGO@Span{SliTeX-narrow}{%
    \HoLogoFont@font{SliTeX}{rm}{%
      S%
        \HOLOGO@Span{l}{l}%
        \HOLOGO@Span{i}{i}%
      \hologo{TeX}%
    }%
  }%
}
%    \end{macrocode}
%    \end{macro}
%    \begin{macro}{\HoLogoCss@SliTeX@narrow}
%    \begin{macrocode}
\def\HoLogoCss@SliTeX@narrow{%
  \Css{%
    span.HoLogo-SliTeX-narrow span.HoLogo-l{%
      margin-left:-.06em;%
      margin-right:-.035em;%
      font-variant:small-caps;%
    }%
  }%
  \Css{%
    span.HoLogo-SliTeX-narrow span.HoLogo-i{%
      margin-right:-.06em;%
      font-variant:small-caps;%
    }%
  }%
  \global\let\HoLogoCss@SliTeX@narrow\relax
}
%    \end{macrocode}
%    \end{macro}
%
% \paragraph{Macro set completion.}
%
%    \begin{macro}{\HoLogo@SLiTeX@simple}
%    \begin{macrocode}
\def\HoLogo@SLiTeX@simple{\HoLogo@SliTeX@simple}
%    \end{macrocode}
%    \end{macro}
%    \begin{macro}{\HoLogoBkm@SLiTeX@simple}
%    \begin{macrocode}
\def\HoLogoBkm@SLiTeX@simple{\HoLogoBkm@SliTeX@simple}
%    \end{macrocode}
%    \end{macro}
%    \begin{macro}{\HoLogoHtml@SLiTeX@simple}
%    \begin{macrocode}
\def\HoLogoHtml@SLiTeX@simple{\HoLogoHtml@SliTeX@simple}
%    \end{macrocode}
%    \end{macro}
%
%    \begin{macro}{\HoLogo@SLiTeX@narrow}
%    \begin{macrocode}
\def\HoLogo@SLiTeX@narrow{\HoLogo@SliTeX@narrow}
%    \end{macrocode}
%    \end{macro}
%    \begin{macro}{\HoLogoBkm@SLiTeX@narrow}
%    \begin{macrocode}
\def\HoLogoBkm@SLiTeX@narrow{\HoLogoBkm@SliTeX@narrow}
%    \end{macrocode}
%    \end{macro}
%    \begin{macro}{\HoLogoHtml@SLiTeX@narrow}
%    \begin{macrocode}
\def\HoLogoHtml@SLiTeX@narrow{\HoLogoHtml@SliTeX@narrow}
%    \end{macrocode}
%    \end{macro}
%
%    \begin{macro}{\HoLogo@SliTeX@lift}
%    \begin{macrocode}
\def\HoLogo@SliTeX@lift{\HoLogo@SLiTeX@lift}
%    \end{macrocode}
%    \end{macro}
%    \begin{macro}{\HoLogoBkm@SliTeX@lift}
%    \begin{macrocode}
\def\HoLogoBkm@SliTeX@lift{\HoLogoBkm@SLiTeX@lift}
%    \end{macrocode}
%    \end{macro}
%    \begin{macro}{\HoLogoHtml@SliTeX@lift}
%    \begin{macrocode}
\def\HoLogoHtml@SliTeX@lift{\HoLogoHtml@SLiTeX@lift}
%    \end{macrocode}
%    \end{macro}
%
% \paragraph{Defaults.}
%
%    \begin{macro}{\HoLogo@SLiTeX}
%    \begin{macrocode}
\def\HoLogo@SLiTeX{\HoLogo@SLiTeX@lift}
%    \end{macrocode}
%    \end{macro}
%    \begin{macro}{\HoLogoBkm@SLiTeX}
%    \begin{macrocode}
\def\HoLogoBkm@SLiTeX{\HoLogoBkm@SLiTeX@lift}
%    \end{macrocode}
%    \end{macro}
%    \begin{macro}{\HoLogoHtml@SLiTeX}
%    \begin{macrocode}
\def\HoLogoHtml@SLiTeX{\HoLogoHtml@SLiTeX@lift}
%    \end{macrocode}
%    \end{macro}
%
%    \begin{macro}{\HoLogo@SliTeX}
%    \begin{macrocode}
\def\HoLogo@SliTeX{\HoLogo@SliTeX@narrow}
%    \end{macrocode}
%    \end{macro}
%    \begin{macro}{\HoLogoBkm@SliTeX}
%    \begin{macrocode}
\def\HoLogoBkm@SliTeX{\HoLogoBkm@SliTeX@narrow}
%    \end{macrocode}
%    \end{macro}
%    \begin{macro}{\HoLogoHtml@SliTeX}
%    \begin{macrocode}
\def\HoLogoHtml@SliTeX{\HoLogoHtml@SliTeX@narrow}
%    \end{macrocode}
%    \end{macro}
%
% \subsubsection{\hologo{LuaTeX}}
%
%    \begin{macro}{\HoLogo@LuaTeX}
%    The kerning is an idea of Hans Hagen, see mailing list
%    `luatex at tug dot org' in March 2010.
%    \begin{macrocode}
\def\HoLogo@LuaTeX#1{%
  \HOLOGO@mbox{%
    Lua%
    \HOLOGO@NegativeKerning{aT,oT,To}%
    \hologo{TeX}%
  }%
}
%    \end{macrocode}
%    \end{macro}
%    \begin{macro}{\HoLogoHtml@LuaTeX}
%    \begin{macrocode}
\let\HoLogoHtml@LuaTeX\HoLogo@LuaTeX
%    \end{macrocode}
%    \end{macro}
%
% \subsubsection{\hologo{LuaLaTeX}}
%
%    \begin{macro}{\HoLogo@LuaLaTeX}
%    \begin{macrocode}
\def\HoLogo@LuaLaTeX#1{%
  \HOLOGO@mbox{%
    Lua%
    \hologo{LaTeX}%
  }%
}
%    \end{macrocode}
%    \end{macro}
%    \begin{macro}{\HoLogoHtml@LuaLaTeX}
%    \begin{macrocode}
\let\HoLogoHtml@LuaLaTeX\HoLogo@LuaLaTeX
%    \end{macrocode}
%    \end{macro}
%
% \subsubsection{\hologo{XeTeX}, \hologo{XeLaTeX}}
%
%    \begin{macro}{\HOLOGO@IfCharExists}
%    \begin{macrocode}
\ifluatex
  \ifnum\luatexversion<36 %
  \else
    \def\HOLOGO@IfCharExists#1{%
      \ifnum
        \directlua{%
           if luaotfload and luaotfload.aux then
             if luaotfload.aux.font_has_glyph(%
                    font.current(), \number#1) then % 	 
	       tex.print("1") % 	 
	     end % 	 
	   elseif font and font.fonts and font.current then %
            local f = font.fonts[font.current()]%
            if f.characters and f.characters[\number#1] then %
              tex.print("1")%
            end %
          end%
        }0=\ltx@zero
        \expandafter\ltx@secondoftwo
      \else
        \expandafter\ltx@firstoftwo
      \fi
    }%
  \fi
\fi
\ltx@IfUndefined{HOLOGO@IfCharExists}{%
  \def\HOLOGO@@IfCharExists#1{%
    \begingroup
      \tracinglostchars=\ltx@zero
      \setbox\ltx@zero=\hbox{%
        \kern7sp\char#1\relax
        \ifnum\lastkern>\ltx@zero
          \expandafter\aftergroup\csname iffalse\endcsname
        \else
          \expandafter\aftergroup\csname iftrue\endcsname
        \fi
      }%
      % \if{true|false} from \aftergroup
      \endgroup
      \expandafter\ltx@firstoftwo
    \else
      \endgroup
      \expandafter\ltx@secondoftwo
    \fi
  }%
  \ifxetex
    \ltx@IfUndefined{XeTeXfonttype}{}{%
      \ltx@IfUndefined{XeTeXcharglyph}{}{%
        \def\HOLOGO@IfCharExists#1{%
          \ifnum\XeTeXfonttype\font>\ltx@zero
            \expandafter\ltx@firstofthree
          \else
            \expandafter\ltx@gobble
          \fi
          {%
            \ifnum\XeTeXcharglyph#1>\ltx@zero
              \expandafter\ltx@firstoftwo
            \else
              \expandafter\ltx@secondoftwo
            \fi
          }%
          \HOLOGO@@IfCharExists{#1}%
        }%
      }%
    }%
  \fi
}{}
\ltx@ifundefined{HOLOGO@IfCharExists}{%
  \ifnum64=`\^^^^0040\relax % test for big chars of LuaTeX/XeTeX
    \let\HOLOGO@IfCharExists\HOLOGO@@IfCharExists
  \else
    \def\HOLOGO@IfCharExists#1{%
      \ifnum#1>255 %
        \expandafter\ltx@fourthoffour
      \fi
      \HOLOGO@@IfCharExists{#1}%
    }%
  \fi
}{}
%    \end{macrocode}
%    \end{macro}
%
%    \begin{macro}{\HoLogo@Xe}
%    Source: package \xpackage{dtklogos}
%    \begin{macrocode}
\def\HoLogo@Xe#1{%
  X%
  \kern-.1em\relax
  \HOLOGO@IfCharExists{"018E}{%
    \lower.5ex\hbox{\char"018E}%
  }{%
    \chardef\HOLOGO@choice=\ltx@zero
    \ifdim\fontdimen\ltx@one\font>0pt %
      \ltx@IfUndefined{rotatebox}{%
        \ltx@IfUndefined{pgftext}{%
          \ltx@IfUndefined{psscalebox}{%
            \ltx@IfUndefined{HOLOGO@ScaleBox@\hologoDriver}{%
            }{%
              \chardef\HOLOGO@choice=4 %
            }%
          }{%
            \chardef\HOLOGO@choice=3 %
          }%
        }{%
          \chardef\HOLOGO@choice=2 %
        }%
      }{%
        \chardef\HOLOGO@choice=1 %
      }%
      \ifcase\HOLOGO@choice
        \HOLOGO@WarningUnsupportedDriver{Xe}%
        e%
      \or % 1: \rotatebox
        \begingroup
          \setbox\ltx@zero\hbox{\rotatebox{180}{E}}%
          \ltx@LocDimenA=\dp\ltx@zero
          \advance\ltx@LocDimenA by -.5ex\relax
          \raise\ltx@LocDimenA\box\ltx@zero
        \endgroup
      \or % 2: \pgftext
        \lower.5ex\hbox{%
          \pgfpicture
            \pgftext[rotate=180]{E}%
          \endpgfpicture
        }%
      \or % 3: \psscalebox
        \begingroup
          \setbox\ltx@zero\hbox{\psscalebox{-1 -1}{E}}%
          \ltx@LocDimenA=\dp\ltx@zero
          \advance\ltx@LocDimenA by -.5ex\relax
          \raise\ltx@LocDimenA\box\ltx@zero
        \endgroup
      \or % 4: \HOLOGO@PointReflectBox
        \lower.5ex\hbox{\HOLOGO@PointReflectBox{E}}%
      \else
        \@PackageError{hologo}{Internal error (choice/it}\@ehc
      \fi
    \else
      \ltx@IfUndefined{reflectbox}{%
        \ltx@IfUndefined{pgftext}{%
          \ltx@IfUndefined{psscalebox}{%
            \ltx@IfUndefined{HOLOGO@ScaleBox@\hologoDriver}{%
            }{%
              \chardef\HOLOGO@choice=4 %
            }%
          }{%
            \chardef\HOLOGO@choice=3 %
          }%
        }{%
          \chardef\HOLOGO@choice=2 %
        }%
      }{%
        \chardef\HOLOGO@choice=1 %
      }%
      \ifcase\HOLOGO@choice
        \HOLOGO@WarningUnsupportedDriver{Xe}%
        e%
      \or % 1: reflectbox
        \lower.5ex\hbox{%
          \reflectbox{E}%
        }%
      \or % 2: \pgftext
        \lower.5ex\hbox{%
          \pgfpicture
            \pgftransformxscale{-1}%
            \pgftext{E}%
          \endpgfpicture
        }%
      \or % 3: \psscalebox
        \lower.5ex\hbox{%
          \psscalebox{-1 1}{E}%
        }%
      \or % 4: \HOLOGO@Reflectbox
        \lower.5ex\hbox{%
          \HOLOGO@ReflectBox{E}%
        }%
      \else
        \@PackageError{hologo}{Internal error (choice/up)}\@ehc
      \fi
    \fi
  }%
}
%    \end{macrocode}
%    \end{macro}
%    \begin{macro}{\HoLogoHtml@Xe}
%    \begin{macrocode}
\def\HoLogoHtml@Xe#1{%
  \HoLogoCss@Xe
  \HOLOGO@Span{Xe}{%
    X%
    \HOLOGO@Span{e}{%
      \HCode{&\ltx@hashchar x018e;}%
    }%
  }%
}
%    \end{macrocode}
%    \end{macro}
%    \begin{macro}{\HoLogoCss@Xe}
%    \begin{macrocode}
\def\HoLogoCss@Xe{%
  \Css{%
    span.HoLogo-Xe span.HoLogo-e{%
      position:relative;%
      top:.5ex;%
      left-margin:-.1em;%
    }%
  }%
  \global\let\HoLogoCss@Xe\relax
}
%    \end{macrocode}
%    \end{macro}
%
%    \begin{macro}{\HoLogo@XeTeX}
%    \begin{macrocode}
\def\HoLogo@XeTeX#1{%
  \hologo{Xe}%
  \kern-.15em\relax
  \hologo{TeX}%
}
%    \end{macrocode}
%    \end{macro}
%
%    \begin{macro}{\HoLogoHtml@XeTeX}
%    \begin{macrocode}
\def\HoLogoHtml@XeTeX#1{%
  \HoLogoCss@XeTeX
  \HOLOGO@Span{XeTeX}{%
    \hologo{Xe}%
    \hologo{TeX}%
  }%
}
%    \end{macrocode}
%    \end{macro}
%    \begin{macro}{\HoLogoCss@XeTeX}
%    \begin{macrocode}
\def\HoLogoCss@XeTeX{%
  \Css{%
    span.HoLogo-XeTeX span.HoLogo-TeX{%
      margin-left:-.15em;%
    }%
  }%
  \global\let\HoLogoCss@XeTeX\relax
}
%    \end{macrocode}
%    \end{macro}
%
%    \begin{macro}{\HoLogo@XeLaTeX}
%    \begin{macrocode}
\def\HoLogo@XeLaTeX#1{%
  \hologo{Xe}%
  \kern-.13em%
  \hologo{LaTeX}%
}
%    \end{macrocode}
%    \end{macro}
%    \begin{macro}{\HoLogoHtml@XeLaTeX}
%    \begin{macrocode}
\def\HoLogoHtml@XeLaTeX#1{%
  \HoLogoCss@XeLaTeX
  \HOLOGO@Span{XeLaTeX}{%
    \hologo{Xe}%
    \hologo{LaTeX}%
  }%
}
%    \end{macrocode}
%    \end{macro}
%    \begin{macro}{\HoLogoCss@XeLaTeX}
%    \begin{macrocode}
\def\HoLogoCss@XeLaTeX{%
  \Css{%
    span.HoLogo-XeLaTeX span.HoLogo-Xe{%
      margin-right:-.13em;%
    }%
  }%
  \global\let\HoLogoCss@XeLaTeX\relax
}
%    \end{macrocode}
%    \end{macro}
%
% \subsubsection{\hologo{pdfTeX}, \hologo{pdfLaTeX}}
%
%    \begin{macro}{\HoLogo@pdfTeX}
%    \begin{macrocode}
\def\HoLogo@pdfTeX#1{%
  \HOLOGO@mbox{%
    #1{p}{P}df\hologo{TeX}%
  }%
}
%    \end{macrocode}
%    \end{macro}
%    \begin{macro}{\HoLogoCs@pdfTeX}
%    \begin{macrocode}
\def\HoLogoCs@pdfTeX#1{#1{p}{P}dfTeX}
%    \end{macrocode}
%    \end{macro}
%    \begin{macro}{\HoLogoBkm@pdfTeX}
%    \begin{macrocode}
\def\HoLogoBkm@pdfTeX#1{%
  #1{p}{P}df\hologo{TeX}%
}
%    \end{macrocode}
%    \end{macro}
%    \begin{macro}{\HoLogoHtml@pdfTeX}
%    \begin{macrocode}
\let\HoLogoHtml@pdfTeX\HoLogo@pdfTeX
%    \end{macrocode}
%    \end{macro}
%
%    \begin{macro}{\HoLogo@pdfLaTeX}
%    \begin{macrocode}
\def\HoLogo@pdfLaTeX#1{%
  \HOLOGO@mbox{%
    #1{p}{P}df\hologo{LaTeX}%
  }%
}
%    \end{macrocode}
%    \end{macro}
%    \begin{macro}{\HoLogoCs@pdfLaTeX}
%    \begin{macrocode}
\def\HoLogoCs@pdfLaTeX#1{#1{p}{P}dfLaTeX}
%    \end{macrocode}
%    \end{macro}
%    \begin{macro}{\HoLogoBkm@pdfLaTeX}
%    \begin{macrocode}
\def\HoLogoBkm@pdfLaTeX#1{%
  #1{p}{P}df\hologo{LaTeX}%
}
%    \end{macrocode}
%    \end{macro}
%    \begin{macro}{\HoLogoHtml@pdfLaTeX}
%    \begin{macrocode}
\let\HoLogoHtml@pdfLaTeX\HoLogo@pdfLaTeX
%    \end{macrocode}
%    \end{macro}
%
% \subsubsection{\hologo{VTeX}}
%
%    \begin{macro}{\HoLogo@VTeX}
%    \begin{macrocode}
\def\HoLogo@VTeX#1{%
  \HOLOGO@mbox{%
    V\hologo{TeX}%
  }%
}
%    \end{macrocode}
%    \end{macro}
%    \begin{macro}{\HoLogoHtml@VTeX}
%    \begin{macrocode}
\let\HoLogoHtml@VTeX\HoLogo@VTeX
%    \end{macrocode}
%    \end{macro}
%
% \subsubsection{\hologo{AmS}, \dots}
%
%    Source: class \xclass{amsdtx}
%
%    \begin{macro}{\HoLogo@AmS}
%    \begin{macrocode}
\def\HoLogo@AmS#1{%
  \HoLogoFont@font{AmS}{sy}{%
    A%
    \kern-.1667em%
    \lower.5ex\hbox{M}%
    \kern-.125em%
    S%
  }%
}
%    \end{macrocode}
%    \end{macro}
%    \begin{macro}{\HoLogoBkm@AmS}
%    \begin{macrocode}
\def\HoLogoBkm@AmS#1{AmS}
%    \end{macrocode}
%    \end{macro}
%    \begin{macro}{\HoLogoHtml@AmS}
%    \begin{macrocode}
\def\HoLogoHtml@AmS#1{%
  \HoLogoCss@AmS
%  \HoLogoFont@font{AmS}{sy}{%
    \HOLOGO@Span{AmS}{%
      A%
      \HOLOGO@Span{M}{M}%
      S%
    }%
%   }%
}
%    \end{macrocode}
%    \end{macro}
%    \begin{macro}{\HoLogoCss@AmS}
%    \begin{macrocode}
\def\HoLogoCss@AmS{%
  \Css{%
    span.HoLogo-AmS span.HoLogo-M{%
      position:relative;%
      top:.5ex;%
      margin-left:-.1667em;%
      margin-right:-.125em;%
      text-decoration:none;%
    }%
  }%
  \global\let\HoLogoCss@AmS\relax
}
%    \end{macrocode}
%    \end{macro}
%
%    \begin{macro}{\HoLogo@AmSTeX}
%    \begin{macrocode}
\def\HoLogo@AmSTeX#1{%
  \hologo{AmS}%
  \HOLOGO@hyphen
  \hologo{TeX}%
}
%    \end{macrocode}
%    \end{macro}
%    \begin{macro}{\HoLogoBkm@AmSTeX}
%    \begin{macrocode}
\def\HoLogoBkm@AmSTeX#1{AmS-TeX}%
%    \end{macrocode}
%    \end{macro}
%    \begin{macro}{\HoLogoHtml@AmSTeX}
%    \begin{macrocode}
\let\HoLogoHtml@AmSTeX\HoLogo@AmSTeX
%    \end{macrocode}
%    \end{macro}
%
%    \begin{macro}{\HoLogo@AmSLaTeX}
%    \begin{macrocode}
\def\HoLogo@AmSLaTeX#1{%
  \hologo{AmS}%
  \HOLOGO@hyphen
  \hologo{LaTeX}%
}
%    \end{macrocode}
%    \end{macro}
%    \begin{macro}{\HoLogoBkm@AmSLaTeX}
%    \begin{macrocode}
\def\HoLogoBkm@AmSLaTeX#1{AmS-LaTeX}%
%    \end{macrocode}
%    \end{macro}
%    \begin{macro}{\HoLogoHtml@AmSLaTeX}
%    \begin{macrocode}
\let\HoLogoHtml@AmSLaTeX\HoLogo@AmSLaTeX
%    \end{macrocode}
%    \end{macro}
%
% \subsubsection{\hologo{BibTeX}}
%
%    \begin{macro}{\HoLogo@BibTeX@sc}
%    A definition of \hologo{BibTeX} is provided in
%    the documentation source for the manual of \hologo{BibTeX}
%    \cite{btxdoc}.
%\begin{quote}
%\begin{verbatim}
%\def\BibTeX{%
%  {%
%    \rm
%    B%
%    \kern-.05em%
%    {%
%      \sc
%      i%
%      \kern-.025em %
%      b%
%    }%
%    \kern-.08em
%    T%
%    \kern-.1667em%
%    \lower.7ex\hbox{E}%
%    \kern-.125em%
%    X%
%  }%
%}
%\end{verbatim}
%\end{quote}
%    \begin{macrocode}
\def\HoLogo@BibTeX@sc#1{%
  B%
  \kern-.05em%
  \HoLogoFont@font{BibTeX}{sc}{%
    i%
    \kern-.025em%
    b%
  }%
  \HOLOGO@discretionary
  \kern-.08em%
  \hologo{TeX}%
}
%    \end{macrocode}
%    \end{macro}
%    \begin{macro}{\HoLogoHtml@BibTeX@sc}
%    \begin{macrocode}
\def\HoLogoHtml@BibTeX@sc#1{%
  \HoLogoCss@BibTeX@sc
  \HOLOGO@Span{BibTeX-sc}{%
    B%
    \HOLOGO@Span{i}{i}%
    \HOLOGO@Span{b}{b}%
    \hologo{TeX}%
  }%
}
%    \end{macrocode}
%    \end{macro}
%    \begin{macro}{\HoLogoCss@BibTeX@sc}
%    \begin{macrocode}
\def\HoLogoCss@BibTeX@sc{%
  \Css{%
    span.HoLogo-BibTeX-sc span.HoLogo-i{%
      margin-left:-.05em;%
      margin-right:-.025em;%
      font-variant:small-caps;%
    }%
  }%
  \Css{%
    span.HoLogo-BibTeX-sc span.HoLogo-b{%
      margin-right:-.08em;%
      font-variant:small-caps;%
    }%
  }%
  \global\let\HoLogoCss@BibTeX@sc\relax
}
%    \end{macrocode}
%    \end{macro}
%
%    \begin{macro}{\HoLogo@BibTeX@sf}
%    Variant \xoption{sf} avoids trouble with unavailable
%    small caps fonts (e.g., bold versions of Computer Modern or
%    Latin Modern). The definition is taken from
%    package \xpackage{dtklogos} \cite{dtklogos}.
%\begin{quote}
%\begin{verbatim}
%\DeclareRobustCommand{\BibTeX}{%
%  B%
%  \kern-.05em%
%  \hbox{%
%    $\m@th$% %% force math size calculations
%    \csname S@\f@size\endcsname
%    \fontsize\sf@size\z@
%    \math@fontsfalse
%    \selectfont
%    I%
%    \kern-.025em%
%    B
%  }%
%  \kern-.08em%
%  \-%
%  \TeX
%}
%\end{verbatim}
%\end{quote}
%    \begin{macrocode}
\def\HoLogo@BibTeX@sf#1{%
  B%
  \kern-.05em%
  \HoLogoFont@font{BibTeX}{bibsf}{%
    I%
    \kern-.025em%
    B%
  }%
  \HOLOGO@discretionary
  \kern-.08em%
  \hologo{TeX}%
}
%    \end{macrocode}
%    \end{macro}
%    \begin{macro}{\HoLogoHtml@BibTeX@sf}
%    \begin{macrocode}
\def\HoLogoHtml@BibTeX@sf#1{%
  \HoLogoCss@BibTeX@sf
  \HOLOGO@Span{BibTeX-sf}{%
    B%
    \HoLogoFont@font{BibTeX}{bibsf}{%
      \HOLOGO@Span{i}{I}%
      B%
    }%
    \hologo{TeX}%
  }%
}
%    \end{macrocode}
%    \end{macro}
%    \begin{macro}{\HoLogoCss@BibTeX@sf}
%    \begin{macrocode}
\def\HoLogoCss@BibTeX@sf{%
  \Css{%
    span.HoLogo-BibTeX-sf span.HoLogo-i{%
      margin-left:-.05em;%
      margin-right:-.025em;%
    }%
  }%
  \Css{%
    span.HoLogo-BibTeX-sf span.HoLogo-TeX{%
      margin-left:-.08em;%
    }%
  }%
  \global\let\HoLogoCss@BibTeX@sf\relax
}
%    \end{macrocode}
%    \end{macro}
%
%    \begin{macro}{\HoLogo@BibTeX}
%    \begin{macrocode}
\def\HoLogo@BibTeX{\HoLogo@BibTeX@sf}
%    \end{macrocode}
%    \end{macro}
%    \begin{macro}{\HoLogoHtml@BibTeX}
%    \begin{macrocode}
\def\HoLogoHtml@BibTeX{\HoLogoHtml@BibTeX@sf}
%    \end{macrocode}
%    \end{macro}
%
% \subsubsection{\hologo{BibTeX8}}
%
%    \begin{macro}{\HoLogo@BibTeX8}
%    \begin{macrocode}
\expandafter\def\csname HoLogo@BibTeX8\endcsname#1{%
  \hologo{BibTeX}%
  8%
}
%    \end{macrocode}
%    \end{macro}
%
%    \begin{macro}{\HoLogoBkm@BibTeX8}
%    \begin{macrocode}
\expandafter\def\csname HoLogoBkm@BibTeX8\endcsname#1{%
  \hologo{BibTeX}%
  8%
}
%    \end{macrocode}
%    \end{macro}
%    \begin{macro}{\HoLogoHtml@BibTeX8}
%    \begin{macrocode}
\expandafter
\let\csname HoLogoHtml@BibTeX8\expandafter\endcsname
\csname HoLogo@BibTeX8\endcsname
%    \end{macrocode}
%    \end{macro}
%
% \subsubsection{\hologo{ConTeXt}}
%
%    \begin{macro}{\HoLogo@ConTeXt@simple}
%    \begin{macrocode}
\def\HoLogo@ConTeXt@simple#1{%
  \HOLOGO@mbox{Con}%
  \HOLOGO@discretionary
  \HOLOGO@mbox{\hologo{TeX}t}%
}
%    \end{macrocode}
%    \end{macro}
%    \begin{macro}{\HoLogoHtml@ConTeXt@simple}
%    \begin{macrocode}
\let\HoLogoHtml@ConTeXt@simple\HoLogo@ConTeXt@simple
%    \end{macrocode}
%    \end{macro}
%
%    \begin{macro}{\HoLogo@ConTeXt@narrow}
%    This definition of logo \hologo{ConTeXt} with variant \xoption{narrow}
%    comes from TUGboat's class \xclass{ltugboat} (version 2010/11/15 v2.8).
%    \begin{macrocode}
\def\HoLogo@ConTeXt@narrow#1{%
  \HOLOGO@mbox{C\kern-.0333emon}%
  \HOLOGO@discretionary
  \kern-.0667em%
  \HOLOGO@mbox{\hologo{TeX}\kern-.0333emt}%
}
%    \end{macrocode}
%    \end{macro}
%    \begin{macro}{\HoLogoHtml@ConTeXt@narrow}
%    \begin{macrocode}
\def\HoLogoHtml@ConTeXt@narrow#1{%
  \HoLogoCss@ConTeXt@narrow
  \HOLOGO@Span{ConTeXt-narrow}{%
    \HOLOGO@Span{C}{C}%
    on%
    \hologo{TeX}%
    t%
  }%
}
%    \end{macrocode}
%    \end{macro}
%    \begin{macro}{\HoLogoCss@ConTeXt@narrow}
%    \begin{macrocode}
\def\HoLogoCss@ConTeXt@narrow{%
  \Css{%
    span.HoLogo-ConTeXt-narrow span.HoLogo-C{%
      margin-left:-.0333em;%
    }%
  }%
  \Css{%
    span.HoLogo-ConTeXt-narrow span.HoLogo-TeX{%
      margin-left:-.0667em;%
      margin-right:-.0333em;%
    }%
  }%
  \global\let\HoLogoCss@ConTeXt@narrow\relax
}
%    \end{macrocode}
%    \end{macro}
%
%    \begin{macro}{\HoLogo@ConTeXt}
%    \begin{macrocode}
\def\HoLogo@ConTeXt{\HoLogo@ConTeXt@narrow}
%    \end{macrocode}
%    \end{macro}
%    \begin{macro}{\HoLogoHtml@ConTeXt}
%    \begin{macrocode}
\def\HoLogoHtml@ConTeXt{\HoLogoHtml@ConTeXt@narrow}
%    \end{macrocode}
%    \end{macro}
%
% \subsubsection{\hologo{emTeX}}
%
%    \begin{macro}{\HoLogo@emTeX}
%    \begin{macrocode}
\def\HoLogo@emTeX#1{%
  \HOLOGO@mbox{#1{e}{E}m}%
  \HOLOGO@discretionary
  \hologo{TeX}%
}
%    \end{macrocode}
%    \end{macro}
%    \begin{macro}{\HoLogoCs@emTeX}
%    \begin{macrocode}
\def\HoLogoCs@emTeX#1{#1{e}{E}mTeX}%
%    \end{macrocode}
%    \end{macro}
%    \begin{macro}{\HoLogoBkm@emTeX}
%    \begin{macrocode}
\def\HoLogoBkm@emTeX#1{%
  #1{e}{E}m\hologo{TeX}%
}
%    \end{macrocode}
%    \end{macro}
%    \begin{macro}{\HoLogoHtml@emTeX}
%    \begin{macrocode}
\let\HoLogoHtml@emTeX\HoLogo@emTeX
%    \end{macrocode}
%    \end{macro}
%
% \subsubsection{\hologo{ExTeX}}
%
%    \begin{macro}{\HoLogo@ExTeX}
%    The definition is taken from the FAQ of the
%    project \hologo{ExTeX}
%    \cite{ExTeX-FAQ}.
%\begin{quote}
%\begin{verbatim}
%\def\ExTeX{%
%  \textrm{% Logo always with serifs
%    \ensuremath{%
%      \textstyle
%      \varepsilon_{%
%        \kern-0.15em%
%        \mathcal{X}%
%      }%
%    }%
%    \kern-.15em%
%    \TeX
%  }%
%}
%\end{verbatim}
%\end{quote}
%    \begin{macrocode}
\def\HoLogo@ExTeX#1{%
  \HoLogoFont@font{ExTeX}{rm}{%
    \ltx@mbox{%
      \HOLOGO@MathSetup
      $%
        \textstyle
        \varepsilon_{%
          \kern-0.15em%
          \HoLogoFont@font{ExTeX}{sy}{X}%
        }%
      $%
    }%
    \HOLOGO@discretionary
    \kern-.15em%
    \hologo{TeX}%
  }%
}
%    \end{macrocode}
%    \end{macro}
%    \begin{macro}{\HoLogoHtml@ExTeX}
%    \begin{macrocode}
\def\HoLogoHtml@ExTeX#1{%
  \HoLogoCss@ExTeX
  \HoLogoFont@font{ExTeX}{rm}{%
    \HOLOGO@Span{ExTeX}{%
      \ltx@mbox{%
        \HOLOGO@MathSetup
        $\textstyle\varepsilon$%
        \HOLOGO@Span{X}{$\textstyle\chi$}%
        \hologo{TeX}%
      }%
    }%
  }%
}
%    \end{macrocode}
%    \end{macro}
%    \begin{macro}{\HoLogoBkm@ExTeX}
%    \begin{macrocode}
\def\HoLogoBkm@ExTeX#1{%
  \HOLOGO@PdfdocUnicode{#1{e}{E}x}{\textepsilon\textchi}%
  \hologo{TeX}%
}
%    \end{macrocode}
%    \end{macro}
%    \begin{macro}{\HoLogoCss@ExTeX}
%    \begin{macrocode}
\def\HoLogoCss@ExTeX{%
  \Css{%
    span.HoLogo-ExTeX{%
      font-family:serif;%
    }%
  }%
  \Css{%
    span.HoLogo-ExTeX span.HoLogo-TeX{%
      margin-left:-.15em;%
    }%
  }%
  \global\let\HoLogoCss@ExTeX\relax
}
%    \end{macrocode}
%    \end{macro}
%
% \subsubsection{\hologo{MiKTeX}}
%
%    \begin{macro}{\HoLogo@MiKTeX}
%    \begin{macrocode}
\def\HoLogo@MiKTeX#1{%
  \HOLOGO@mbox{MiK}%
  \HOLOGO@discretionary
  \hologo{TeX}%
}
%    \end{macrocode}
%    \end{macro}
%    \begin{macro}{\HoLogoHtml@MiKTeX}
%    \begin{macrocode}
\let\HoLogoHtml@MiKTeX\HoLogo@MiKTeX
%    \end{macrocode}
%    \end{macro}
%
% \subsubsection{\hologo{OzTeX} and friends}
%
%    Source: \hologo{OzTeX} FAQ \cite{OzTeX}:
%    \begin{quote}
%      |\def\OzTeX{O\kern-.03em z\kern-.15em\TeX}|\\
%      (There is no kerning in OzMF, OzMP and OzTtH.)
%    \end{quote}
%
%    \begin{macro}{\HoLogo@OzTeX}
%    \begin{macrocode}
\def\HoLogo@OzTeX#1{%
  O%
  \kern-.03em %
  z%
  \kern-.15em %
  \hologo{TeX}%
}
%    \end{macrocode}
%    \end{macro}
%    \begin{macro}{\HoLogoHtml@OzTeX}
%    \begin{macrocode}
\def\HoLogoHtml@OzTeX#1{%
  \HoLogoCss@OzTeX
  \HOLOGO@Span{OzTeX}{%
    O%
    \HOLOGO@Span{z}{z}%
    \hologo{TeX}%
  }%
}
%    \end{macrocode}
%    \end{macro}
%    \begin{macro}{\HoLogoCss@OzTeX}
%    \begin{macrocode}
\def\HoLogoCss@OzTeX{%
  \Css{%
    span.HoLogo-OzTeX span.HoLogo-z{%
      margin-left:-.03em;%
      margin-right:-.15em;%
    }%
  }%
  \global\let\HoLogoCss@OzTeX\relax
}
%    \end{macrocode}
%    \end{macro}
%
%    \begin{macro}{\HoLogo@OzMF}
%    \begin{macrocode}
\def\HoLogo@OzMF#1{%
  \HOLOGO@mbox{OzMF}%
}
%    \end{macrocode}
%    \end{macro}
%    \begin{macro}{\HoLogo@OzMP}
%    \begin{macrocode}
\def\HoLogo@OzMP#1{%
  \HOLOGO@mbox{OzMP}%
}
%    \end{macrocode}
%    \end{macro}
%    \begin{macro}{\HoLogo@OzTtH}
%    \begin{macrocode}
\def\HoLogo@OzTtH#1{%
  \HOLOGO@mbox{OzTtH}%
}
%    \end{macrocode}
%    \end{macro}
%
% \subsubsection{\hologo{PCTeX}}
%
%    \begin{macro}{\HoLogo@PCTeX}
%    \begin{macrocode}
\def\HoLogo@PCTeX#1{%
  \HOLOGO@mbox{PC}%
  \hologo{TeX}%
}
%    \end{macrocode}
%    \end{macro}
%    \begin{macro}{\HoLogoHtml@PCTeX}
%    \begin{macrocode}
\let\HoLogoHtml@PCTeX\HoLogo@PCTeX
%    \end{macrocode}
%    \end{macro}
%
% \subsubsection{\hologo{PiCTeX}}
%
%    The original definitions from \xfile{pictex.tex} \cite{PiCTeX}:
%\begin{quote}
%\begin{verbatim}
%\def\PiC{%
%  P%
%  \kern-.12em%
%  \lower.5ex\hbox{I}%
%  \kern-.075em%
%  C%
%}
%\def\PiCTeX{%
%  \PiC
%  \kern-.11em%
%  \TeX
%}
%\end{verbatim}
%\end{quote}
%
%    \begin{macro}{\HoLogo@PiC}
%    \begin{macrocode}
\def\HoLogo@PiC#1{%
  P%
  \kern-.12em%
  \lower.5ex\hbox{I}%
  \kern-.075em%
  C%
  \HOLOGO@SpaceFactor
}
%    \end{macrocode}
%    \end{macro}
%    \begin{macro}{\HoLogoHtml@PiC}
%    \begin{macrocode}
\def\HoLogoHtml@PiC#1{%
  \HoLogoCss@PiC
  \HOLOGO@Span{PiC}{%
    P%
    \HOLOGO@Span{i}{I}%
    C%
  }%
}
%    \end{macrocode}
%    \end{macro}
%    \begin{macro}{\HoLogoCss@PiC}
%    \begin{macrocode}
\def\HoLogoCss@PiC{%
  \Css{%
    span.HoLogo-PiC span.HoLogo-i{%
      position:relative;%
      top:.5ex;%
      margin-left:-.12em;%
      margin-right:-.075em;%
      text-decoration:none;%
    }%
  }%
  \global\let\HoLogoCss@PiC\relax
}
%    \end{macrocode}
%    \end{macro}
%
%    \begin{macro}{\HoLogo@PiCTeX}
%    \begin{macrocode}
\def\HoLogo@PiCTeX#1{%
  \hologo{PiC}%
  \HOLOGO@discretionary
  \kern-.11em%
  \hologo{TeX}%
}
%    \end{macrocode}
%    \end{macro}
%    \begin{macro}{\HoLogoHtml@PiCTeX}
%    \begin{macrocode}
\def\HoLogoHtml@PiCTeX#1{%
  \HoLogoCss@PiCTeX
  \HOLOGO@Span{PiCTeX}{%
    \hologo{PiC}%
    \hologo{TeX}%
  }%
}
%    \end{macrocode}
%    \end{macro}
%    \begin{macro}{\HoLogoCss@PiCTeX}
%    \begin{macrocode}
\def\HoLogoCss@PiCTeX{%
  \Css{%
    span.HoLogo-PiCTeX span.HoLogo-PiC{%
      margin-right:-.11em;%
    }%
  }%
  \global\let\HoLogoCss@PiCTeX\relax
}
%    \end{macrocode}
%    \end{macro}
%
% \subsubsection{\hologo{teTeX}}
%
%    \begin{macro}{\HoLogo@teTeX}
%    \begin{macrocode}
\def\HoLogo@teTeX#1{%
  \HOLOGO@mbox{#1{t}{T}e}%
  \HOLOGO@discretionary
  \hologo{TeX}%
}
%    \end{macrocode}
%    \end{macro}
%    \begin{macro}{\HoLogoCs@teTeX}
%    \begin{macrocode}
\def\HoLogoCs@teTeX#1{#1{t}{T}dfTeX}
%    \end{macrocode}
%    \end{macro}
%    \begin{macro}{\HoLogoBkm@teTeX}
%    \begin{macrocode}
\def\HoLogoBkm@teTeX#1{%
  #1{t}{T}e\hologo{TeX}%
}
%    \end{macrocode}
%    \end{macro}
%    \begin{macro}{\HoLogoHtml@teTeX}
%    \begin{macrocode}
\let\HoLogoHtml@teTeX\HoLogo@teTeX
%    \end{macrocode}
%    \end{macro}
%
% \subsubsection{\hologo{TeX4ht}}
%
%    \begin{macro}{\HoLogo@TeX4ht}
%    \begin{macrocode}
\expandafter\def\csname HoLogo@TeX4ht\endcsname#1{%
  \HOLOGO@mbox{\hologo{TeX}4ht}%
}
%    \end{macrocode}
%    \end{macro}
%    \begin{macro}{\HoLogoHtml@TeX4ht}
%    \begin{macrocode}
\expandafter
\let\csname HoLogoHtml@TeX4ht\expandafter\endcsname
\csname HoLogo@TeX4ht\endcsname
%    \end{macrocode}
%    \end{macro}
%
%
% \subsubsection{\hologo{SageTeX}}
%
%    \begin{macro}{\HoLogo@SageTeX}
%    \begin{macrocode}
\def\HoLogo@SageTeX#1{%
  \HOLOGO@mbox{Sage}%
  \HOLOGO@discretionary
  \HOLOGO@NegativeKerning{eT,oT,To}%
  \hologo{TeX}%
}
%    \end{macrocode}
%    \end{macro}
%    \begin{macro}{\HoLogoHtml@SageTeX}
%    \begin{macrocode}
\let\HoLogoHtml@SageTeX\HoLogo@SageTeX
%    \end{macrocode}
%    \end{macro}
%
% \subsection{\hologo{METAFONT} and friends}
%
%    \begin{macro}{\HoLogo@METAFONT}
%    \begin{macrocode}
\def\HoLogo@METAFONT#1{%
  \HoLogoFont@font{METAFONT}{logo}{%
    \HOLOGO@mbox{META}%
    \HOLOGO@discretionary
    \HOLOGO@mbox{FONT}%
  }%
}
%    \end{macrocode}
%    \end{macro}
%
%    \begin{macro}{\HoLogo@METAPOST}
%    \begin{macrocode}
\def\HoLogo@METAPOST#1{%
  \HoLogoFont@font{METAPOST}{logo}{%
    \HOLOGO@mbox{META}%
    \HOLOGO@discretionary
    \HOLOGO@mbox{POST}%
  }%
}
%    \end{macrocode}
%    \end{macro}
%
%    \begin{macro}{\HoLogo@MetaFun}
%    \begin{macrocode}
\def\HoLogo@MetaFun#1{%
  \HOLOGO@mbox{Meta}%
  \HOLOGO@discretionary
  \HOLOGO@mbox{Fun}%
}
%    \end{macrocode}
%    \end{macro}
%
%    \begin{macro}{\HoLogo@MetaPost}
%    \begin{macrocode}
\def\HoLogo@MetaPost#1{%
  \HOLOGO@mbox{Meta}%
  \HOLOGO@discretionary
  \HOLOGO@mbox{Post}%
}
%    \end{macrocode}
%    \end{macro}
%
% \subsection{Others}
%
% \subsubsection{\hologo{biber}}
%
%    \begin{macro}{\HoLogo@biber}
%    \begin{macrocode}
\def\HoLogo@biber#1{%
  \HOLOGO@mbox{#1{b}{B}i}%
  \HOLOGO@discretionary
  \HOLOGO@mbox{ber}%
}
%    \end{macrocode}
%    \end{macro}
%    \begin{macro}{\HoLogoCs@biber}
%    \begin{macrocode}
\def\HoLogoCs@biber#1{#1{b}{B}iber}
%    \end{macrocode}
%    \end{macro}
%    \begin{macro}{\HoLogoBkm@biber}
%    \begin{macrocode}
\def\HoLogoBkm@biber#1{%
  #1{b}{B}iber%
}
%    \end{macrocode}
%    \end{macro}
%    \begin{macro}{\HoLogoHtml@biber}
%    \begin{macrocode}
\let\HoLogoHtml@biber\HoLogo@biber
%    \end{macrocode}
%    \end{macro}
%
% \subsubsection{\hologo{KOMAScript}}
%
%    \begin{macro}{\HoLogo@KOMAScript}
%    The definition for \hologo{KOMAScript} is taken
%    from \hologo{KOMAScript} (\xfile{scrlogo.dtx}, reformatted) \cite{scrlogo}:
%\begin{quote}
%\begin{verbatim}
%\@ifundefined{KOMAScript}{%
%  \DeclareRobustCommand{\KOMAScript}{%
%    \textsf{%
%      K\kern.05em O\kern.05emM\kern.05em A%
%      \kern.1em-\kern.1em %
%      Script%
%    }%
%  }%
%}{}
%\end{verbatim}
%\end{quote}
%    \begin{macrocode}
\def\HoLogo@KOMAScript#1{%
  \HoLogoFont@font{KOMAScript}{sf}{%
    \HOLOGO@mbox{%
      K\kern.05em%
      O\kern.05em%
      M\kern.05em%
      A%
    }%
    \kern.1em%
    \HOLOGO@hyphen
    \kern.1em%
    \HOLOGO@mbox{Script}%
  }%
}
%    \end{macrocode}
%    \end{macro}
%    \begin{macro}{\HoLogoBkm@KOMAScript}
%    \begin{macrocode}
\def\HoLogoBkm@KOMAScript#1{%
  KOMA-Script%
}
%    \end{macrocode}
%    \end{macro}
%    \begin{macro}{\HoLogoHtml@KOMAScript}
%    \begin{macrocode}
\def\HoLogoHtml@KOMAScript#1{%
  \HoLogoCss@KOMAScript
  \HoLogoFont@font{KOMAScript}{sf}{%
    \HOLOGO@Span{KOMAScript}{%
      K%
      \HOLOGO@Span{O}{O}%
      M%
      \HOLOGO@Span{A}{A}%
      \HOLOGO@Span{hyphen}{-}%
      Script%
    }%
  }%
}
%    \end{macrocode}
%    \end{macro}
%    \begin{macro}{\HoLogoCss@KOMAScript}
%    \begin{macrocode}
\def\HoLogoCss@KOMAScript{%
  \Css{%
    span.HoLogo-KOMAScript{%
      font-family:sans-serif;%
    }%
  }%
  \Css{%
    span.HoLogo-KOMAScript span.HoLogo-O{%
      padding-left:.05em;%
      padding-right:.05em;%
    }%
  }%
  \Css{%
    span.HoLogo-KOMAScript span.HoLogo-A{%
      padding-left:.05em;%
    }%
  }%
  \Css{%
    span.HoLogo-KOMAScript span.HoLogo-hyphen{%
      padding-left:.1em;%
      padding-right:.1em;%
    }%
  }%
  \global\let\HoLogoCss@KOMAScript\relax
}
%    \end{macrocode}
%    \end{macro}
%
% \subsubsection{\hologo{LyX}}
%
%    \begin{macro}{\HoLogo@LyX}
%    The definition is taken from the documentation source files
%    of \hologo{LyX}, \xfile{Intro.lyx} \cite{LyX}:
%\begin{quote}
%\begin{verbatim}
%\def\LyX{%
%  \texorpdfstring{%
%    L\kern-.1667em\lower.25em\hbox{Y}\kern-.125emX\@%
%  }{%
%    LyX%
%  }%
%}
%\end{verbatim}
%\end{quote}
%    \begin{macrocode}
\def\HoLogo@LyX#1{%
  L%
  \kern-.1667em%
  \lower.25em\hbox{Y}%
  \kern-.125em%
  X%
  \HOLOGO@SpaceFactor
}
%    \end{macrocode}
%    \end{macro}
%    \begin{macro}{\HoLogoHtml@LyX}
%    \begin{macrocode}
\def\HoLogoHtml@LyX#1{%
  \HoLogoCss@LyX
  \HOLOGO@Span{LyX}{%
    L%
    \HOLOGO@Span{y}{Y}%
    X%
  }%
}
%    \end{macrocode}
%    \end{macro}
%    \begin{macro}{\HoLogoCss@LyX}
%    \begin{macrocode}
\def\HoLogoCss@LyX{%
  \Css{%
    span.HoLogo-LyX span.HoLogo-y{%
      position:relative;%
      top:.25em;%
      margin-left:-.1667em;%
      margin-right:-.125em;%
      text-decoration:none;%
    }%
  }%
  \global\let\HoLogoCss@LyX\relax
}
%    \end{macrocode}
%    \end{macro}
%
% \subsubsection{\hologo{NTS}}
%
%    \begin{macro}{\HoLogo@NTS}
%    Definition for \hologo{NTS} can be found in
%    package \xpackage{etex\textunderscore man} for the \hologo{eTeX} manual \cite{etexman}
%    and in package \xpackage{dtklogos} \cite{dtklogos}:
%\begin{quote}
%\begin{verbatim}
%\def\NTS{%
%  \leavevmode
%  \hbox{%
%    $%
%      \cal N%
%      \kern-0.35em%
%      \lower0.5ex\hbox{$\cal T$}%
%      \kern-0.2em%
%      S%
%    $%
%  }%
%}
%\end{verbatim}
%\end{quote}
%    \begin{macrocode}
\def\HoLogo@NTS#1{%
  \HoLogoFont@font{NTS}{sy}{%
    N\/%
    \kern-.35em%
    \lower.5ex\hbox{T\/}%
    \kern-.2em%
    S\/%
  }%
  \HOLOGO@SpaceFactor
}
%    \end{macrocode}
%    \end{macro}
%
% \subsubsection{\Hologo{TTH} (\hologo{TeX} to HTML translator)}
%
%    Source: \url{http://hutchinson.belmont.ma.us/tth/}
%    In the HTML source the second `T' is printed as subscript.
%\begin{quote}
%\begin{verbatim}
%T<sub>T</sub>H
%\end{verbatim}
%\end{quote}
%    \begin{macro}{\HoLogo@TTH}
%    \begin{macrocode}
\def\HoLogo@TTH#1{%
  \ltx@mbox{%
    T\HOLOGO@SubScript{T}H%
  }%
  \HOLOGO@SpaceFactor
}
%    \end{macrocode}
%    \end{macro}
%
%    \begin{macro}{\HoLogoHtml@TTH}
%    \begin{macrocode}
\def\HoLogoHtml@TTH#1{%
  T\HCode{<sub>}T\HCode{</sub>}H%
}
%    \end{macrocode}
%    \end{macro}
%
% \subsubsection{\Hologo{HanTheThanh}}
%
%    Partial source: Package \xpackage{dtklogos}.
%    The double accent is U+1EBF (latin small letter e with circumflex
%    and acute).
%    \begin{macro}{\HoLogo@HanTheThanh}
%    \begin{macrocode}
\def\HoLogo@HanTheThanh#1{%
  \ltx@mbox{H\`an}%
  \HOLOGO@space
  \ltx@mbox{%
    Th%
    \HOLOGO@IfCharExists{"1EBF}{%
      \char"1EBF\relax
    }{%
      \^e\hbox to 0pt{\hss\raise .5ex\hbox{\'{}}}%
    }%
  }%
  \HOLOGO@space
  \ltx@mbox{Th\`anh}%
}
%    \end{macrocode}
%    \end{macro}
%    \begin{macro}{\HoLogoBkm@HanTheThanh}
%    \begin{macrocode}
\def\HoLogoBkm@HanTheThanh#1{%
  H\`an %
  Th\HOLOGO@PdfdocUnicode{\^e}{\9036\277} %
  Th\`anh%
}
%    \end{macrocode}
%    \end{macro}
%    \begin{macro}{\HoLogoHtml@HanTheThanh}
%    \begin{macrocode}
\def\HoLogoHtml@HanTheThanh#1{%
  H\`an %
  Th\HCode{&\ltx@hashchar x1ebf;} %
  Th\`anh%
}
%    \end{macrocode}
%    \end{macro}
%
% \subsection{Driver detection}
%
%    \begin{macrocode}
\HOLOGO@IfExists\InputIfFileExists{%
  \InputIfFileExists{hologo.cfg}{}{}%
}{%
  \ltx@IfUndefined{pdf@filesize}{%
    \def\HOLOGO@InputIfExists{%
      \openin\HOLOGO@temp=hologo.cfg\relax
      \ifeof\HOLOGO@temp
        \closein\HOLOGO@temp
      \else
        \closein\HOLOGO@temp
        \begingroup
          \def\x{LaTeX2e}%
        \expandafter\endgroup
        \ifx\fmtname\x
          \input{hologo.cfg}%
        \else
          \input hologo.cfg\relax
        \fi
      \fi
    }%
    \ltx@IfUndefined{newread}{%
      \chardef\HOLOGO@temp=15 %
      \def\HOLOGO@CheckRead{%
        \ifeof\HOLOGO@temp
          \HOLOGO@InputIfExists
        \else
          \ifcase\HOLOGO@temp
            \@PackageWarningNoLine{hologo}{%
              Configuration file ignored, because\MessageBreak
              a free read register could not be found%
            }%
          \else
            \begingroup
              \count\ltx@cclv=\HOLOGO@temp
              \advance\ltx@cclv by \ltx@minusone
              \edef\x{\endgroup
                \chardef\noexpand\HOLOGO@temp=\the\count\ltx@cclv
                \relax
              }%
            \x
          \fi
        \fi
      }%
    }{%
      \csname newread\endcsname\HOLOGO@temp
      \HOLOGO@InputIfExists
    }%
  }{%
    \edef\HOLOGO@temp{\pdf@filesize{hologo.cfg}}%
    \ifx\HOLOGO@temp\ltx@empty
    \else
      \ifnum\HOLOGO@temp>0 %
        \begingroup
          \def\x{LaTeX2e}%
        \expandafter\endgroup
        \ifx\fmtname\x
          \input{hologo.cfg}%
        \else
          \input hologo.cfg\relax
        \fi
      \else
        \@PackageInfoNoLine{hologo}{%
          Empty configuration file `hologo.cfg' ignored%
        }%
      \fi
    \fi
  }%
}
%    \end{macrocode}
%
%    \begin{macrocode}
\def\HOLOGO@temp#1#2{%
  \kv@define@key{HoLogoDriver}{#1}[]{%
    \begingroup
      \def\HOLOGO@temp{##1}%
      \ltx@onelevel@sanitize\HOLOGO@temp
      \ifx\HOLOGO@temp\ltx@empty
      \else
        \@PackageError{hologo}{%
          Value (\HOLOGO@temp) not permitted for option `#1'%
        }%
        \@ehc
      \fi
    \endgroup
    \def\hologoDriver{#2}%
  }%
}%
\def\HOLOGO@@temp#1#2{%
  \ifx\kv@value\relax
    \HOLOGO@temp{#1}{#1}%
  \else
    \HOLOGO@temp{#1}{#2}%
  \fi
}%
\kv@parse@normalized{%
  pdftex,%
  luatex=pdftex,%
  dvipdfm,%
  dvipdfmx=dvipdfm,%
  dvips,%
  dvipsone=dvips,%
  xdvi=dvips,%
  xetex,%
  vtex,%
}\HOLOGO@@temp
%    \end{macrocode}
%
%    \begin{macrocode}
\kv@define@key{HoLogoDriver}{driverfallback}{%
  \def\HOLOGO@DriverFallback{#1}%
}
%    \end{macrocode}
%
%    \begin{macro}{\HOLOGO@DriverFallback}
%    \begin{macrocode}
\def\HOLOGO@DriverFallback{dvips}
%    \end{macrocode}
%    \end{macro}
%
%    \begin{macro}{\hologoDriverSetup}
%    \begin{macrocode}
\def\hologoDriverSetup{%
  \let\hologoDriver\ltx@undefined
  \HOLOGO@DriverSetup
}
%    \end{macrocode}
%    \end{macro}
%
%    \begin{macro}{\HOLOGO@DriverSetup}
%    \begin{macrocode}
\def\HOLOGO@DriverSetup#1{%
  \kvsetkeys{HoLogoDriver}{#1}%
  \HOLOGO@CheckDriver
  \ltx@ifundefined{hologoDriver}{%
    \begingroup
    \edef\x{\endgroup
      \noexpand\kvsetkeys{HoLogoDriver}{\HOLOGO@DriverFallback}%
    }\x
  }{}%
  \@PackageInfoNoLine{hologo}{Using driver `\hologoDriver'}%
}
%    \end{macrocode}
%    \end{macro}
%
%    \begin{macro}{\HOLOGO@CheckDriver}
%    \begin{macrocode}
\def\HOLOGO@CheckDriver{%
  \ifpdf
    \def\hologoDriver{pdftex}%
    \let\HOLOGO@pdfliteral\pdfliteral
    \ifluatex
      \ifx\pdfextension\@undefined\else
        \protected\def\pdfliteral{\pdfextension literal}%
        \let\HOLOGO@pdfliteral\pdfliteral
      \fi
      \ltx@IfUndefined{HOLOGO@pdfliteral}{%
        \ifnum\luatexversion<36 %
        \else
          \begingroup
            \let\HOLOGO@temp\endgroup
            \ifcase0%
                \directlua{%
                  if tex.enableprimitives then %
                    tex.enableprimitives('HOLOGO@', {'pdfliteral'})%
                  else %
                    tex.print('1')%
                  end%
                }%
                \ifx\HOLOGO@pdfliteral\@undefined 1\fi%
                \relax%
              \endgroup
              \let\HOLOGO@temp\relax
              \global\let\HOLOGO@pdfliteral\HOLOGO@pdfliteral
            \fi%
          \HOLOGO@temp
        \fi
      }{}%
    \fi
    \ltx@IfUndefined{HOLOGO@pdfliteral}{%
      \@PackageWarningNoLine{hologo}{%
        Cannot find \string\pdfliteral
      }%
    }{}%
  \else
    \ifxetex
      \def\hologoDriver{xetex}%
    \else
      \ifvtex
        \def\hologoDriver{vtex}%
      \fi
    \fi
  \fi
}
%    \end{macrocode}
%    \end{macro}
%
%    \begin{macro}{\HOLOGO@WarningUnsupportedDriver}
%    \begin{macrocode}
\def\HOLOGO@WarningUnsupportedDriver#1{%
  \@PackageWarningNoLine{hologo}{%
    Logo `#1' needs driver specific macros,\MessageBreak
    but driver `\hologoDriver' is not supported.\MessageBreak
    Use a different driver or\MessageBreak
    load package `graphics' or `pgf'%
  }%
}
%    \end{macrocode}
%    \end{macro}
%
% \subsubsection{Reflect box macros}
%
%    Skip driver part if not needed.
%    \begin{macrocode}
\ltx@IfUndefined{reflectbox}{}{%
  \ltx@IfUndefined{rotatebox}{}{%
    \HOLOGO@AtEnd
  }%
}
\ltx@IfUndefined{pgftext}{}{%
  \HOLOGO@AtEnd
}
\ltx@IfUndefined{psscalebox}{}{%
  \HOLOGO@AtEnd
}
%    \end{macrocode}
%
%    \begin{macrocode}
\def\HOLOGO@temp{LaTeX2e}
\ifx\fmtname\HOLOGO@temp
  \RequirePackage{kvoptions}[2011/06/30]%
  \ProcessKeyvalOptions{HoLogoDriver}%
\fi
\HOLOGO@DriverSetup{}
%    \end{macrocode}
%
%    \begin{macro}{\HOLOGO@ReflectBox}
%    \begin{macrocode}
\def\HOLOGO@ReflectBox#1{%
  \begingroup
    \setbox\ltx@zero\hbox{\begingroup#1\endgroup}%
    \setbox\ltx@two\hbox{%
      \kern\wd\ltx@zero
      \csname HOLOGO@ScaleBox@\hologoDriver\endcsname{-1}{1}{%
        \hbox to 0pt{\copy\ltx@zero\hss}%
      }%
    }%
    \wd\ltx@two=\wd\ltx@zero
    \box\ltx@two
  \endgroup
}
%    \end{macrocode}
%    \end{macro}
%
%    \begin{macro}{\HOLOGO@PointReflectBox}
%    \begin{macrocode}
\def\HOLOGO@PointReflectBox#1{%
  \begingroup
    \setbox\ltx@zero\hbox{\begingroup#1\endgroup}%
    \setbox\ltx@two\hbox{%
      \kern\wd\ltx@zero
      \raise\ht\ltx@zero\hbox{%
        \csname HOLOGO@ScaleBox@\hologoDriver\endcsname{-1}{-1}{%
          \hbox to 0pt{\copy\ltx@zero\hss}%
        }%
      }%
    }%
    \wd\ltx@two=\wd\ltx@zero
    \box\ltx@two
  \endgroup
}
%    \end{macrocode}
%    \end{macro}
%
%    We must define all variants because of dynamic driver setup.
%    \begin{macrocode}
\def\HOLOGO@temp#1#2{#2}
%    \end{macrocode}
%
%    \begin{macro}{\HOLOGO@ScaleBox@pdftex}
%    \begin{macrocode}
\HOLOGO@temp{pdftex}{%
  \def\HOLOGO@ScaleBox@pdftex#1#2#3{%
    \HOLOGO@pdfliteral{%
      q #1 0 0 #2 0 0 cm%
    }%
    #3%
    \HOLOGO@pdfliteral{%
      Q%
    }%
  }%
}
%    \end{macrocode}
%    \end{macro}
%    \begin{macro}{\HOLOGO@ScaleBox@dvips}
%    \begin{macrocode}
\HOLOGO@temp{dvips}{%
  \def\HOLOGO@ScaleBox@dvips#1#2#3{%
    \special{ps:%
      gsave %
      currentpoint %
      currentpoint translate %
      #1 #2 scale %
      neg exch neg exch translate%
    }%
    #3%
    \special{ps:%
      currentpoint %
      grestore %
      moveto%
    }%
  }%
}
%    \end{macrocode}
%    \end{macro}
%    \begin{macro}{\HOLOGO@ScaleBox@dvipdfm}
%    \begin{macrocode}
\HOLOGO@temp{dvipdfm}{%
  \let\HOLOGO@ScaleBox@dvipdfm\HOLOGO@ScaleBox@dvips
}
%    \end{macrocode}
%    \end{macro}
%    Since \hologo{XeTeX} v0.6.
%    \begin{macro}{\HOLOGO@ScaleBox@xetex}
%    \begin{macrocode}
\HOLOGO@temp{xetex}{%
  \def\HOLOGO@ScaleBox@xetex#1#2#3{%
    \special{x:gsave}%
    \special{x:scale #1 #2}%
    #3%
    \special{x:grestore}%
  }%
}
%    \end{macrocode}
%    \end{macro}
%    \begin{macro}{\HOLOGO@ScaleBox@vtex}
%    \begin{macrocode}
\HOLOGO@temp{vtex}{%
  \def\HOLOGO@ScaleBox@vtex#1#2#3{%
    \special{r(#1,0,0,#2,0,0}%
    #3%
    \special{r)}%
  }%
}
%    \end{macrocode}
%    \end{macro}
%
%    \begin{macrocode}
\HOLOGO@AtEnd%
%</package>
%    \end{macrocode}
%
% \section{Test}
%
% \subsection{Catcode checks for loading}
%
%    \begin{macrocode}
%<*test1>
%    \end{macrocode}
%    \begin{macrocode}
\catcode`\{=1 %
\catcode`\}=2 %
\catcode`\#=6 %
\catcode`\@=11 %
\expandafter\ifx\csname count@\endcsname\relax
  \countdef\count@=255 %
\fi
\expandafter\ifx\csname @gobble\endcsname\relax
  \long\def\@gobble#1{}%
\fi
\expandafter\ifx\csname @firstofone\endcsname\relax
  \long\def\@firstofone#1{#1}%
\fi
\expandafter\ifx\csname loop\endcsname\relax
  \expandafter\@firstofone
\else
  \expandafter\@gobble
\fi
{%
  \def\loop#1\repeat{%
    \def\body{#1}%
    \iterate
  }%
  \def\iterate{%
    \body
      \let\next\iterate
    \else
      \let\next\relax
    \fi
    \next
  }%
  \let\repeat=\fi
}%
\def\RestoreCatcodes{}
\count@=0 %
\loop
  \edef\RestoreCatcodes{%
    \RestoreCatcodes
    \catcode\the\count@=\the\catcode\count@\relax
  }%
\ifnum\count@<255 %
  \advance\count@ 1 %
\repeat

\def\RangeCatcodeInvalid#1#2{%
  \count@=#1\relax
  \loop
    \catcode\count@=15 %
  \ifnum\count@<#2\relax
    \advance\count@ 1 %
  \repeat
}
\def\RangeCatcodeCheck#1#2#3{%
  \count@=#1\relax
  \loop
    \ifnum#3=\catcode\count@
    \else
      \errmessage{%
        Character \the\count@\space
        with wrong catcode \the\catcode\count@\space
        instead of \number#3%
      }%
    \fi
  \ifnum\count@<#2\relax
    \advance\count@ 1 %
  \repeat
}
\def\space{ }
\expandafter\ifx\csname LoadCommand\endcsname\relax
  \def\LoadCommand{\input hologo.sty\relax}%
\fi
\def\Test{%
  \RangeCatcodeInvalid{0}{47}%
  \RangeCatcodeInvalid{58}{64}%
  \RangeCatcodeInvalid{91}{96}%
  \RangeCatcodeInvalid{123}{255}%
  \catcode`\@=12 %
  \catcode`\\=0 %
  \catcode`\%=14 %
  \LoadCommand
  \RangeCatcodeCheck{0}{36}{15}%
  \RangeCatcodeCheck{37}{37}{14}%
  \RangeCatcodeCheck{38}{47}{15}%
  \RangeCatcodeCheck{48}{57}{12}%
  \RangeCatcodeCheck{58}{63}{15}%
  \RangeCatcodeCheck{64}{64}{12}%
  \RangeCatcodeCheck{65}{90}{11}%
  \RangeCatcodeCheck{91}{91}{15}%
  \RangeCatcodeCheck{92}{92}{0}%
  \RangeCatcodeCheck{93}{96}{15}%
  \RangeCatcodeCheck{97}{122}{11}%
  \RangeCatcodeCheck{123}{255}{15}%
  \RestoreCatcodes
}
\Test
\csname @@end\endcsname
\end
%    \end{macrocode}
%    \begin{macrocode}
%</test1>
%    \end{macrocode}
%
% \subsection{Spacefactor}
%
%    The space factor must be 1000 after a logo. If it is greater 1000
%    then the following space is a space after a sentence closing point.
%    If the space factor is smaller 1000 then an immediate following
%    dot is interpreted as abbreviation, not sentence closing point.
%
%    \begin{macrocode}
%<*test-spacefactor>
\NeedsTeXFormat{LaTeX2e}
\documentclass{article}
\usepackage{hologo}[2016/05/12]
\usepackage{kvsetkeys}
\usepackage{qstest}
\IncludeTests{*}
\LogTests{log}{*}{*}
\begin{document}
\begin{qstest}{spacefactor}{spacefactor}
\newcommand*{\Test}[1]{%
  \sbox0{%
    \hologo{#1}%
    \Expect*{1000 (#1)}*{\the\spacefactor\space(#1)}%
  }%
}%
\makeatletter
\def\TestList{}
\def\hologoEntry#1#2#3{%
  \edef\TestList{%
    \ifx\TestList\@empty
    \else
      \TestList,%
    \fi
    #1%
    \ifx\\#2\\%
    \else
      ={variant=#2}%
    \fi
  }%
}
\hologoList
\expandafter\kv@parse@normalized\expandafter{%
  \TestList
}{%
  \begingroup
    \let\@logo=\kv@key
    \ifx\kv@value\relax
    \else
      \expandafter\hologoLogoSetup\expandafter\@logo\expandafter{%
        \kv@value
      }%
    \fi
    \Test\@logo
  \endgroup
  \@gobbletwo
}
\end{qstest}
\end{document}
%</test-spacefactor>
%    \end{macrocode}
%
% \subsection{Complete list}
%
%    \begin{macrocode}
%<*test-list>
\NeedsTeXFormat{LaTeX2e}
\documentclass[12pt,a4paper]{article}
\usepackage{hologo}[2016/05/12]
\usepackage[T1]{fontenc}
\usepackage{lmodern}
\usepackage{parskip}
\usepackage[unicode]{hyperref}[2011/09/28]
\usepackage{bookmark}[2011/09/19]
\bookmarksetup{%
  numbered,%
  open,%
  openlevel=2,%
}
\renewcommand*{\contentsname}{List of logos}
\begin{document}
\tableofcontents
\def\TestFont#1#2#3#4#5#6{%
  \begingroup
    \usefont{#3}{#4}{#5}{#6}%
    \HologoVariant{#1}{#2}/\hologoVariant{#1}{#2}%
    \quad
    \begingroup\scriptsize\hologoVariant{#1}{#2}\endgroup
    \quad
  \endgroup
  (#3/#4/#5/#6)%
  \par
}
\makeatletter
\def\hologoEntry#1#2#3{%
  \section{%
    \HologoVariant{#1}{#2}/\hologoVariant{#1}{#2} %
    {[#1\ifx\\#2\\\else\space(#2)\fi]}% hash-ok
  }% braces around [] because of bug in tex4ht
  \begingroup
    \hypersetup{unicode=false}%
    \bookmark[%
      dest=\@currentHref,%
      rellevel=1,%
      keeplevel,%
    ]{%
      \HologoVariant{#1}{#2}/\hologoVariant{#1}{#2} %
      (PDFDocEncoding)%
    }%
  \endgroup
  \TestFont{#1}{#2}{OT1}{cmr}{m}{n}%
  \TestFont{#1}{#2}{OT1}{cmss}{m}{n}%
  \TestFont{#1}{#2}{OT1}{cmr}{b}{n}%
  \TestFont{#1}{#2}{OT1}{cmr}{m}{it}%
  \TestFont{#1}{#2}{OT1}{cmtt}{m}{n}%
  \TestFont{#1}{#2}{T1}{lmr}{m}{n}%
  \TestFont{#1}{#2}{T1}{lmss}{m}{n}%
  \TestFont{#1}{#2}{T1}{lmr}{b}{n}%
  \TestFont{#1}{#2}{T1}{lmr}{m}{it}%
  \TestFont{#1}{#2}{T1}{lmtt}{m}{n}%
  \TestFont{#1}{#2}{T1}{lmvtt}{m}{n}%
  \TestFont{#1}{#2}{T1}{qtm}{m}{n}%
  \TestFont{#1}{#2}{T1}{qhv}{m}{n}%
  \TestFont{#1}{#2}{T1}{qtm}{b}{n}%
  \TestFont{#1}{#2}{T1}{qtm}{m}{it}%
  \TestFont{#1}{#2}{T1}{qcr}{m}{n}%
  \newpage
}
\makeatother
\hologoList
\end{document}
%</test-list>
%    \end{macrocode}
%
% \section{Installation}
%
% \subsection{Download}
%
% \paragraph{Package.} This package is available on
% CTAN\footnote{\url{ftp://ftp.ctan.org/tex-archive/}}:
% \begin{description}
% \item[\CTAN{macros/latex/contrib/oberdiek/hologo.dtx}] The source file.
% \item[\CTAN{macros/latex/contrib/oberdiek/hologo.pdf}] Documentation.
% \end{description}
%
%
% \paragraph{Bundle.} All the packages of the bundle `oberdiek'
% are also available in a TDS compliant ZIP archive. There
% the packages are already unpacked and the documentation files
% are generated. The files and directories obey the TDS standard.
% \begin{description}
% \item[\CTAN{install/macros/latex/contrib/oberdiek.tds.zip}]
% \end{description}
% \emph{TDS} refers to the standard ``A Directory Structure
% for \TeX\ Files'' (\CTAN{tds/tds.pdf}). Directories
% with \xfile{texmf} in their name are usually organized this way.
%
% \subsection{Bundle installation}
%
% \paragraph{Unpacking.} Unpack the \xfile{oberdiek.tds.zip} in the
% TDS tree (also known as \xfile{texmf} tree) of your choice.
% Example (linux):
% \begin{quote}
%   |unzip oberdiek.tds.zip -d ~/texmf|
% \end{quote}
%
% \paragraph{Script installation.}
% Check the directory \xfile{TDS:scripts/oberdiek/} for
% scripts that need further installation steps.
% Package \xpackage{attachfile2} comes with the Perl script
% \xfile{pdfatfi.pl} that should be installed in such a way
% that it can be called as \texttt{pdfatfi}.
% Example (linux):
% \begin{quote}
%   |chmod +x scripts/oberdiek/pdfatfi.pl|\\
%   |cp scripts/oberdiek/pdfatfi.pl /usr/local/bin/|
% \end{quote}
%
% \subsection{Package installation}
%
% \paragraph{Unpacking.} The \xfile{.dtx} file is a self-extracting
% \docstrip\ archive. The files are extracted by running the
% \xfile{.dtx} through \plainTeX:
% \begin{quote}
%   \verb|tex hologo.dtx|
% \end{quote}
%
% \paragraph{TDS.} Now the different files must be moved into
% the different directories in your installation TDS tree
% (also known as \xfile{texmf} tree):
% \begin{quote}
% \def\t{^^A
% \begin{tabular}{@{}>{\ttfamily}l@{ $\rightarrow$ }>{\ttfamily}l@{}}
%   hologo.sty & tex/generic/oberdiek/hologo.sty\\
%   hologo.pdf & doc/latex/oberdiek/hologo.pdf\\
%   example/hologo-example.tex & doc/latex/oberdiek/example/hologo-example.tex\\
%   test/hologo-test1.tex & doc/latex/oberdiek/test/hologo-test1.tex\\
%   test/hologo-test-spacefactor.tex & doc/latex/oberdiek/test/hologo-test-spacefactor.tex\\
%   test/hologo-test-list.tex & doc/latex/oberdiek/test/hologo-test-list.tex\\
%   hologo.dtx & source/latex/oberdiek/hologo.dtx\\
% \end{tabular}^^A
% }^^A
% \sbox0{\t}^^A
% \ifdim\wd0>\linewidth
%   \begingroup
%     \advance\linewidth by\leftmargin
%     \advance\linewidth by\rightmargin
%   \edef\x{\endgroup
%     \def\noexpand\lw{\the\linewidth}^^A
%   }\x
%   \def\lwbox{^^A
%     \leavevmode
%     \hbox to \linewidth{^^A
%       \kern-\leftmargin\relax
%       \hss
%       \usebox0
%       \hss
%       \kern-\rightmargin\relax
%     }^^A
%   }^^A
%   \ifdim\wd0>\lw
%     \sbox0{\small\t}^^A
%     \ifdim\wd0>\linewidth
%       \ifdim\wd0>\lw
%         \sbox0{\footnotesize\t}^^A
%         \ifdim\wd0>\linewidth
%           \ifdim\wd0>\lw
%             \sbox0{\scriptsize\t}^^A
%             \ifdim\wd0>\linewidth
%               \ifdim\wd0>\lw
%                 \sbox0{\tiny\t}^^A
%                 \ifdim\wd0>\linewidth
%                   \lwbox
%                 \else
%                   \usebox0
%                 \fi
%               \else
%                 \lwbox
%               \fi
%             \else
%               \usebox0
%             \fi
%           \else
%             \lwbox
%           \fi
%         \else
%           \usebox0
%         \fi
%       \else
%         \lwbox
%       \fi
%     \else
%       \usebox0
%     \fi
%   \else
%     \lwbox
%   \fi
% \else
%   \usebox0
% \fi
% \end{quote}
% If you have a \xfile{docstrip.cfg} that configures and enables \docstrip's
% TDS installing feature, then some files can already be in the right
% place, see the documentation of \docstrip.
%
% \subsection{Refresh file name databases}
%
% If your \TeX~distribution
% (\teTeX, \mikTeX, \dots) relies on file name databases, you must refresh
% these. For example, \teTeX\ users run \verb|texhash| or
% \verb|mktexlsr|.
%
% \subsection{Some details for the interested}
%
% \paragraph{Attached source.}
%
% The PDF documentation on CTAN also includes the
% \xfile{.dtx} source file. It can be extracted by
% AcrobatReader 6 or higher. Another option is \textsf{pdftk},
% e.g. unpack the file into the current directory:
% \begin{quote}
%   \verb|pdftk hologo.pdf unpack_files output .|
% \end{quote}
%
% \paragraph{Unpacking with \LaTeX.}
% The \xfile{.dtx} chooses its action depending on the format:
% \begin{description}
% \item[\plainTeX:] Run \docstrip\ and extract the files.
% \item[\LaTeX:] Generate the documentation.
% \end{description}
% If you insist on using \LaTeX\ for \docstrip\ (really,
% \docstrip\ does not need \LaTeX), then inform the autodetect routine
% about your intention:
% \begin{quote}
%   \verb|latex \let\install=y\input{hologo.dtx}|
% \end{quote}
% Do not forget to quote the argument according to the demands
% of your shell.
%
% \paragraph{Generating the documentation.}
% You can use both the \xfile{.dtx} or the \xfile{.drv} to generate
% the documentation. The process can be configured by the
% configuration file \xfile{ltxdoc.cfg}. For instance, put this
% line into this file, if you want to have A4 as paper format:
% \begin{quote}
%   \verb|\PassOptionsToClass{a4paper}{article}|
% \end{quote}
% An example follows how to generate the
% documentation with pdf\LaTeX:
% \begin{quote}
%\begin{verbatim}
%pdflatex hologo.dtx
%makeindex -s gind.ist hologo.idx
%pdflatex hologo.dtx
%makeindex -s gind.ist hologo.idx
%pdflatex hologo.dtx
%\end{verbatim}
% \end{quote}
%
% \section{Catalogue}
%
% The following XML file can be used as source for the
% \href{http://mirror.ctan.org/help/Catalogue/catalogue.html}{\TeX\ Catalogue}.
% The elements \texttt{caption} and \texttt{description} are imported
% from the original XML file from the Catalogue.
% The name of the XML file in the Catalogue is \xfile{hologo.xml}.
%    \begin{macrocode}
%<*catalogue>
<?xml version='1.0' encoding='us-ascii'?>
<!DOCTYPE entry SYSTEM 'catalogue.dtd'>
<entry datestamp='$Date$' modifier='$Author$' id='hologo'>
  <name>hologo</name>
  <caption>A collection of logos with bookmark support.</caption>
  <authorref id='auth:oberdiek'/>
  <copyright owner='Heiko Oberdiek' year='2010-2012'/>
  <license type='lppl1.3'/>
  <version number='1.10'/>
  <description>
    The package defines a single command <tt>\hologo</tt>, whose
    argument is the usual case-confused ASCII version of the logo.
    The command is bookmark-enabled, so that every logo becomes
    available in bookmarks without further work.
    <p/>
    The package is part of the <xref refid='oberdiek'>oberdiek</xref>
    bundle.
  </description>
  <documentation details='Package documentation'
      href='ctan:/macros/latex/contrib/oberdiek/hologo.pdf'/>
  <ctan file='true' path='/macros/latex/contrib/oberdiek/hologo.dtx'/>
  <miktex location='oberdiek'/>
  <texlive location='oberdiek'/>
  <install path='/macros/latex/contrib/oberdiek/oberdiek.tds.zip'/>
</entry>
%</catalogue>
%    \end{macrocode}
%
% \begin{thebibliography}{9}
% \raggedright
%
% \bibitem{btxdoc}
% Oren Patashnik,
% \textit{\hologo{BibTeX}ing},
% 1988-02-08.\\
% \CTAN{biblio/bibtex/base/}
%
% \bibitem{dtklogos}
% Gerd Neugebauer, DANTE,
% \textit{Package \xpackage{dtklogos}},
% 2011-04-25.\\
% \CTAN{usergrps/dante/dtk/dtklogos.sty}
%
% \bibitem{etexman}
% The \hologo{NTS} Team,
% \textit{The \hologo{eTeX} manual},
% 1998-02.\\
% \CTAN{systems/e-tex/v2/doc/}
%
% \bibitem{ExTeX-FAQ}
% The \hologo{ExTeX} group,
% \textit{\hologo{ExTeX}: FAQ -- How is \hologo{ExTeX} typeset?},
% 2007-04-14.\\
% \url{http://www.extex.org/documentation/faq.html}
%
% \bibitem{LyX}
% %@MISC{ LyX,
% %  title = {{LyX 2.0.0 -- The Document Processor [Computer software and manual]}},
% %  author = {{The LyX Team}},
% %  howpublished = {Internet: http://www.lyx.org},
% %  year = {2011-05-08},
% %  note = {Retrieved May 10, 2011, from http://www.lyx.org},
% %  url = {http://www.lyx.org/}
% %}
% The \hologo{LyX} Team,
% \textit{\hologo{LyX} -- The Document Processor},
% 2011-05-08.\\
% \url{http://www.lyx.org/}
%
% \bibitem{OzTeX}
% Andrew Trevorrow,
% \hologo{OzTeX} FAQ: What is the correct way to typeset ``\hologo{OzTeX}''?,
% 2011-09-15 (visited).
% \url{http://www.trevorrow.com/oztex/ozfaq.html#oztex-logo}
%
% \bibitem{PiCTeX}
% Michael Wichura,
% \textit{The \hologo{PiCTeX} macro package},
% 1987-09-21.
% \CTAN{graphics/pictex/}
%
% \bibitem{scrlogo}
% Markus Kohm,
% \textit{\hologo{KOMAScript} Datei \xfile{scrlogo.dtx}},
% 2009-01-30.\\
% \CTAN{install/macros/latex/contrib/komascript.tds.zip}
%
% \end{thebibliography}
%
% \begin{History}
%   \begin{Version}{2010/04/08 v1.0}
%   \item
%     The first version.
%   \end{Version}
%   \begin{Version}{2010/04/16 v1.1}
%   \item
%     \cs{Hologo} added for support of logos at start of a sentence.
%   \item
%     \cs{hologoSetup} and \cs{hologoLogoSetup} added.
%   \item
%     Options \xoption{break}, \xoption{hyphenbreak}, \xoption{spacebreak}
%     added.
%   \item
%     Variant support added by option \xoption{variant}.
%   \end{Version}
%   \begin{Version}{2010/04/24 v1.2}
%   \item
%     \hologo{LaTeX3} added.
%   \item
%     \hologo{VTeX} added.
%   \end{Version}
%   \begin{Version}{2010/11/21 v1.3}
%   \item
%     \hologo{iniTeX}, \hologo{virTeX} added.
%   \end{Version}
%   \begin{Version}{2011/03/25 v1.4}
%   \item
%     \hologo{ConTeXt} with variants added.
%   \item
%     Option \xoption{discretionarybreak} added as refinement for
%     option \xoption{break}.
%   \end{Version}
%   \begin{Version}{2011/04/21 v1.5}
%   \item
%     Wrong TDS directory for test files fixed.
%   \end{Version}
%   \begin{Version}{2011/10/01 v1.6}
%   \item
%     Support for package \xpackage{tex4ht} added.
%   \item
%     Support for \cs{csname} added if \cs{ifincsname} is available.
%   \item
%     New logos:
%     \hologo{(La)TeX},
%     \hologo{biber},
%     \hologo{BibTeX} (\xoption{sc}, \xoption{sf}),
%     \hologo{emTeX},
%     \hologo{ExTeX},
%     \hologo{KOMAScript},
%     \hologo{La},
%     \hologo{LyX},
%     \hologo{MiKTeX},
%     \hologo{NTS},
%     \hologo{OzMF},
%     \hologo{OzMP},
%     \hologo{OzTeX},
%     \hologo{OzTtH},
%     \hologo{PCTeX},
%     \hologo{PiC},
%     \hologo{PiCTeX},
%     \hologo{METAFONT},
%     \hologo{MetaFun},
%     \hologo{METAPOST},
%     \hologo{MetaPost},
%     \hologo{SLiTeX} (\xoption{lift}, \xoption{narrow}, \xoption{simple}),
%     \hologo{SliTeX} (\xoption{narrow}, \xoption{simple}, \xoption{lift}),
%     \hologo{teTeX}.
%   \item
%     Fixes:
%     \hologo{iniTeX},
%     \hologo{pdfLaTeX},
%     \hologo{pdfTeX},
%     \hologo{virTeX}.
%   \item
%     \cs{hologoFontSetup} and \cs{hologoLogoFontSetup} added.
%   \item
%     \cs{hologoVariant} and \cs{HologoVariant} added.
%   \end{Version}
%   \begin{Version}{2011/11/22 v1.7}
%   \item
%     New logos:
%     \hologo{BibTeX8},
%     \hologo{LaTeXML},
%     \hologo{SageTeX},
%     \hologo{TeX4ht},
%     \hologo{TTH}.
%   \item
%     \hologo{Xe} and friends: Driver stuff fixed.
%   \item
%     \hologo{Xe} and friends: Support for italic added.
%   \item
%     \hologo{Xe} and friends: Package support for \xpackage{pgf}
%     and \xpackage{pstricks} added.
%   \end{Version}
%   \begin{Version}{2011/11/29 v1.8}
%   \item
%     New logos:
%     \hologo{HanTheThanh}.
%   \end{Version}
%   \begin{Version}{2011/12/21 v1.9}
%   \item
%     Patch for package \xpackage{ifxetex} added for the case that
%     \cs{newif} is undefined in \hologo{iniTeX}.
%   \item
%     Some fixes for \hologo{iniTeX}.
%   \end{Version}
%   \begin{Version}{2012/04/26 v1.10}
%   \item
%     Fix in bookmark version of logo ``\hologo{HanTheThanh}''.
%   \end{Version}
%   \begin{Version}{2016/05/12 v1.11}
%   \item
%     Update HOLOGO@IfCharExists (previously in texlive)
%   \item define pdfliteral in current luatex.
%   \end{Version}
% \end{History}
%
% \PrintIndex
%
% \Finale
\endinput
%
        \else
          \input hologo.cfg\relax
        \fi
      \else
        \@PackageInfoNoLine{hologo}{%
          Empty configuration file `hologo.cfg' ignored%
        }%
      \fi
    \fi
  }%
}
%    \end{macrocode}
%
%    \begin{macrocode}
\def\HOLOGO@temp#1#2{%
  \kv@define@key{HoLogoDriver}{#1}[]{%
    \begingroup
      \def\HOLOGO@temp{##1}%
      \ltx@onelevel@sanitize\HOLOGO@temp
      \ifx\HOLOGO@temp\ltx@empty
      \else
        \@PackageError{hologo}{%
          Value (\HOLOGO@temp) not permitted for option `#1'%
        }%
        \@ehc
      \fi
    \endgroup
    \def\hologoDriver{#2}%
  }%
}%
\def\HOLOGO@@temp#1#2{%
  \ifx\kv@value\relax
    \HOLOGO@temp{#1}{#1}%
  \else
    \HOLOGO@temp{#1}{#2}%
  \fi
}%
\kv@parse@normalized{%
  pdftex,%
  luatex=pdftex,%
  dvipdfm,%
  dvipdfmx=dvipdfm,%
  dvips,%
  dvipsone=dvips,%
  xdvi=dvips,%
  xetex,%
  vtex,%
}\HOLOGO@@temp
%    \end{macrocode}
%
%    \begin{macrocode}
\kv@define@key{HoLogoDriver}{driverfallback}{%
  \def\HOLOGO@DriverFallback{#1}%
}
%    \end{macrocode}
%
%    \begin{macro}{\HOLOGO@DriverFallback}
%    \begin{macrocode}
\def\HOLOGO@DriverFallback{dvips}
%    \end{macrocode}
%    \end{macro}
%
%    \begin{macro}{\hologoDriverSetup}
%    \begin{macrocode}
\def\hologoDriverSetup{%
  \let\hologoDriver\ltx@undefined
  \HOLOGO@DriverSetup
}
%    \end{macrocode}
%    \end{macro}
%
%    \begin{macro}{\HOLOGO@DriverSetup}
%    \begin{macrocode}
\def\HOLOGO@DriverSetup#1{%
  \kvsetkeys{HoLogoDriver}{#1}%
  \HOLOGO@CheckDriver
  \ltx@ifundefined{hologoDriver}{%
    \begingroup
    \edef\x{\endgroup
      \noexpand\kvsetkeys{HoLogoDriver}{\HOLOGO@DriverFallback}%
    }\x
  }{}%
  \@PackageInfoNoLine{hologo}{Using driver `\hologoDriver'}%
}
%    \end{macrocode}
%    \end{macro}
%
%    \begin{macro}{\HOLOGO@CheckDriver}
%    \begin{macrocode}
\def\HOLOGO@CheckDriver{%
  \ifpdf
    \def\hologoDriver{pdftex}%
    \let\HOLOGO@pdfliteral\pdfliteral
    \ifluatex
      \ifx\pdfextension\@undefined\else
        \protected\def\pdfliteral{\pdfextension literal}%
        \let\HOLOGO@pdfliteral\pdfliteral
      \fi
      \ltx@IfUndefined{HOLOGO@pdfliteral}{%
        \ifnum\luatexversion<36 %
        \else
          \begingroup
            \let\HOLOGO@temp\endgroup
            \ifcase0%
                \directlua{%
                  if tex.enableprimitives then %
                    tex.enableprimitives('HOLOGO@', {'pdfliteral'})%
                  else %
                    tex.print('1')%
                  end%
                }%
                \ifx\HOLOGO@pdfliteral\@undefined 1\fi%
                \relax%
              \endgroup
              \let\HOLOGO@temp\relax
              \global\let\HOLOGO@pdfliteral\HOLOGO@pdfliteral
            \fi%
          \HOLOGO@temp
        \fi
      }{}%
    \fi
    \ltx@IfUndefined{HOLOGO@pdfliteral}{%
      \@PackageWarningNoLine{hologo}{%
        Cannot find \string\pdfliteral
      }%
    }{}%
  \else
    \ifxetex
      \def\hologoDriver{xetex}%
    \else
      \ifvtex
        \def\hologoDriver{vtex}%
      \fi
    \fi
  \fi
}
%    \end{macrocode}
%    \end{macro}
%
%    \begin{macro}{\HOLOGO@WarningUnsupportedDriver}
%    \begin{macrocode}
\def\HOLOGO@WarningUnsupportedDriver#1{%
  \@PackageWarningNoLine{hologo}{%
    Logo `#1' needs driver specific macros,\MessageBreak
    but driver `\hologoDriver' is not supported.\MessageBreak
    Use a different driver or\MessageBreak
    load package `graphics' or `pgf'%
  }%
}
%    \end{macrocode}
%    \end{macro}
%
% \subsubsection{Reflect box macros}
%
%    Skip driver part if not needed.
%    \begin{macrocode}
\ltx@IfUndefined{reflectbox}{}{%
  \ltx@IfUndefined{rotatebox}{}{%
    \HOLOGO@AtEnd
  }%
}
\ltx@IfUndefined{pgftext}{}{%
  \HOLOGO@AtEnd
}
\ltx@IfUndefined{psscalebox}{}{%
  \HOLOGO@AtEnd
}
%    \end{macrocode}
%
%    \begin{macrocode}
\def\HOLOGO@temp{LaTeX2e}
\ifx\fmtname\HOLOGO@temp
  \RequirePackage{kvoptions}[2011/06/30]%
  \ProcessKeyvalOptions{HoLogoDriver}%
\fi
\HOLOGO@DriverSetup{}
%    \end{macrocode}
%
%    \begin{macro}{\HOLOGO@ReflectBox}
%    \begin{macrocode}
\def\HOLOGO@ReflectBox#1{%
  \begingroup
    \setbox\ltx@zero\hbox{\begingroup#1\endgroup}%
    \setbox\ltx@two\hbox{%
      \kern\wd\ltx@zero
      \csname HOLOGO@ScaleBox@\hologoDriver\endcsname{-1}{1}{%
        \hbox to 0pt{\copy\ltx@zero\hss}%
      }%
    }%
    \wd\ltx@two=\wd\ltx@zero
    \box\ltx@two
  \endgroup
}
%    \end{macrocode}
%    \end{macro}
%
%    \begin{macro}{\HOLOGO@PointReflectBox}
%    \begin{macrocode}
\def\HOLOGO@PointReflectBox#1{%
  \begingroup
    \setbox\ltx@zero\hbox{\begingroup#1\endgroup}%
    \setbox\ltx@two\hbox{%
      \kern\wd\ltx@zero
      \raise\ht\ltx@zero\hbox{%
        \csname HOLOGO@ScaleBox@\hologoDriver\endcsname{-1}{-1}{%
          \hbox to 0pt{\copy\ltx@zero\hss}%
        }%
      }%
    }%
    \wd\ltx@two=\wd\ltx@zero
    \box\ltx@two
  \endgroup
}
%    \end{macrocode}
%    \end{macro}
%
%    We must define all variants because of dynamic driver setup.
%    \begin{macrocode}
\def\HOLOGO@temp#1#2{#2}
%    \end{macrocode}
%
%    \begin{macro}{\HOLOGO@ScaleBox@pdftex}
%    \begin{macrocode}
\HOLOGO@temp{pdftex}{%
  \def\HOLOGO@ScaleBox@pdftex#1#2#3{%
    \HOLOGO@pdfliteral{%
      q #1 0 0 #2 0 0 cm%
    }%
    #3%
    \HOLOGO@pdfliteral{%
      Q%
    }%
  }%
}
%    \end{macrocode}
%    \end{macro}
%    \begin{macro}{\HOLOGO@ScaleBox@dvips}
%    \begin{macrocode}
\HOLOGO@temp{dvips}{%
  \def\HOLOGO@ScaleBox@dvips#1#2#3{%
    \special{ps:%
      gsave %
      currentpoint %
      currentpoint translate %
      #1 #2 scale %
      neg exch neg exch translate%
    }%
    #3%
    \special{ps:%
      currentpoint %
      grestore %
      moveto%
    }%
  }%
}
%    \end{macrocode}
%    \end{macro}
%    \begin{macro}{\HOLOGO@ScaleBox@dvipdfm}
%    \begin{macrocode}
\HOLOGO@temp{dvipdfm}{%
  \let\HOLOGO@ScaleBox@dvipdfm\HOLOGO@ScaleBox@dvips
}
%    \end{macrocode}
%    \end{macro}
%    Since \hologo{XeTeX} v0.6.
%    \begin{macro}{\HOLOGO@ScaleBox@xetex}
%    \begin{macrocode}
\HOLOGO@temp{xetex}{%
  \def\HOLOGO@ScaleBox@xetex#1#2#3{%
    \special{x:gsave}%
    \special{x:scale #1 #2}%
    #3%
    \special{x:grestore}%
  }%
}
%    \end{macrocode}
%    \end{macro}
%    \begin{macro}{\HOLOGO@ScaleBox@vtex}
%    \begin{macrocode}
\HOLOGO@temp{vtex}{%
  \def\HOLOGO@ScaleBox@vtex#1#2#3{%
    \special{r(#1,0,0,#2,0,0}%
    #3%
    \special{r)}%
  }%
}
%    \end{macrocode}
%    \end{macro}
%
%    \begin{macrocode}
\HOLOGO@AtEnd%
%</package>
%    \end{macrocode}
%
% \section{Test}
%
% \subsection{Catcode checks for loading}
%
%    \begin{macrocode}
%<*test1>
%    \end{macrocode}
%    \begin{macrocode}
\catcode`\{=1 %
\catcode`\}=2 %
\catcode`\#=6 %
\catcode`\@=11 %
\expandafter\ifx\csname count@\endcsname\relax
  \countdef\count@=255 %
\fi
\expandafter\ifx\csname @gobble\endcsname\relax
  \long\def\@gobble#1{}%
\fi
\expandafter\ifx\csname @firstofone\endcsname\relax
  \long\def\@firstofone#1{#1}%
\fi
\expandafter\ifx\csname loop\endcsname\relax
  \expandafter\@firstofone
\else
  \expandafter\@gobble
\fi
{%
  \def\loop#1\repeat{%
    \def\body{#1}%
    \iterate
  }%
  \def\iterate{%
    \body
      \let\next\iterate
    \else
      \let\next\relax
    \fi
    \next
  }%
  \let\repeat=\fi
}%
\def\RestoreCatcodes{}
\count@=0 %
\loop
  \edef\RestoreCatcodes{%
    \RestoreCatcodes
    \catcode\the\count@=\the\catcode\count@\relax
  }%
\ifnum\count@<255 %
  \advance\count@ 1 %
\repeat

\def\RangeCatcodeInvalid#1#2{%
  \count@=#1\relax
  \loop
    \catcode\count@=15 %
  \ifnum\count@<#2\relax
    \advance\count@ 1 %
  \repeat
}
\def\RangeCatcodeCheck#1#2#3{%
  \count@=#1\relax
  \loop
    \ifnum#3=\catcode\count@
    \else
      \errmessage{%
        Character \the\count@\space
        with wrong catcode \the\catcode\count@\space
        instead of \number#3%
      }%
    \fi
  \ifnum\count@<#2\relax
    \advance\count@ 1 %
  \repeat
}
\def\space{ }
\expandafter\ifx\csname LoadCommand\endcsname\relax
  \def\LoadCommand{\input hologo.sty\relax}%
\fi
\def\Test{%
  \RangeCatcodeInvalid{0}{47}%
  \RangeCatcodeInvalid{58}{64}%
  \RangeCatcodeInvalid{91}{96}%
  \RangeCatcodeInvalid{123}{255}%
  \catcode`\@=12 %
  \catcode`\\=0 %
  \catcode`\%=14 %
  \LoadCommand
  \RangeCatcodeCheck{0}{36}{15}%
  \RangeCatcodeCheck{37}{37}{14}%
  \RangeCatcodeCheck{38}{47}{15}%
  \RangeCatcodeCheck{48}{57}{12}%
  \RangeCatcodeCheck{58}{63}{15}%
  \RangeCatcodeCheck{64}{64}{12}%
  \RangeCatcodeCheck{65}{90}{11}%
  \RangeCatcodeCheck{91}{91}{15}%
  \RangeCatcodeCheck{92}{92}{0}%
  \RangeCatcodeCheck{93}{96}{15}%
  \RangeCatcodeCheck{97}{122}{11}%
  \RangeCatcodeCheck{123}{255}{15}%
  \RestoreCatcodes
}
\Test
\csname @@end\endcsname
\end
%    \end{macrocode}
%    \begin{macrocode}
%</test1>
%    \end{macrocode}
%
% \subsection{Spacefactor}
%
%    The space factor must be 1000 after a logo. If it is greater 1000
%    then the following space is a space after a sentence closing point.
%    If the space factor is smaller 1000 then an immediate following
%    dot is interpreted as abbreviation, not sentence closing point.
%
%    \begin{macrocode}
%<*test-spacefactor>
\NeedsTeXFormat{LaTeX2e}
\documentclass{article}
\usepackage{hologo}[2016/05/12]
\usepackage{kvsetkeys}
\usepackage{qstest}
\IncludeTests{*}
\LogTests{log}{*}{*}
\begin{document}
\begin{qstest}{spacefactor}{spacefactor}
\newcommand*{\Test}[1]{%
  \sbox0{%
    \hologo{#1}%
    \Expect*{1000 (#1)}*{\the\spacefactor\space(#1)}%
  }%
}%
\makeatletter
\def\TestList{}
\def\hologoEntry#1#2#3{%
  \edef\TestList{%
    \ifx\TestList\@empty
    \else
      \TestList,%
    \fi
    #1%
    \ifx\\#2\\%
    \else
      ={variant=#2}%
    \fi
  }%
}
\hologoList
\expandafter\kv@parse@normalized\expandafter{%
  \TestList
}{%
  \begingroup
    \let\@logo=\kv@key
    \ifx\kv@value\relax
    \else
      \expandafter\hologoLogoSetup\expandafter\@logo\expandafter{%
        \kv@value
      }%
    \fi
    \Test\@logo
  \endgroup
  \@gobbletwo
}
\end{qstest}
\end{document}
%</test-spacefactor>
%    \end{macrocode}
%
% \subsection{Complete list}
%
%    \begin{macrocode}
%<*test-list>
\NeedsTeXFormat{LaTeX2e}
\documentclass[12pt,a4paper]{article}
\usepackage{hologo}[2016/05/12]
\usepackage[T1]{fontenc}
\usepackage{lmodern}
\usepackage{parskip}
\usepackage[unicode]{hyperref}[2011/09/28]
\usepackage{bookmark}[2011/09/19]
\bookmarksetup{%
  numbered,%
  open,%
  openlevel=2,%
}
\renewcommand*{\contentsname}{List of logos}
\begin{document}
\tableofcontents
\def\TestFont#1#2#3#4#5#6{%
  \begingroup
    \usefont{#3}{#4}{#5}{#6}%
    \HologoVariant{#1}{#2}/\hologoVariant{#1}{#2}%
    \quad
    \begingroup\scriptsize\hologoVariant{#1}{#2}\endgroup
    \quad
  \endgroup
  (#3/#4/#5/#6)%
  \par
}
\makeatletter
\def\hologoEntry#1#2#3{%
  \section{%
    \HologoVariant{#1}{#2}/\hologoVariant{#1}{#2} %
    {[#1\ifx\\#2\\\else\space(#2)\fi]}% hash-ok
  }% braces around [] because of bug in tex4ht
  \begingroup
    \hypersetup{unicode=false}%
    \bookmark[%
      dest=\@currentHref,%
      rellevel=1,%
      keeplevel,%
    ]{%
      \HologoVariant{#1}{#2}/\hologoVariant{#1}{#2} %
      (PDFDocEncoding)%
    }%
  \endgroup
  \TestFont{#1}{#2}{OT1}{cmr}{m}{n}%
  \TestFont{#1}{#2}{OT1}{cmss}{m}{n}%
  \TestFont{#1}{#2}{OT1}{cmr}{b}{n}%
  \TestFont{#1}{#2}{OT1}{cmr}{m}{it}%
  \TestFont{#1}{#2}{OT1}{cmtt}{m}{n}%
  \TestFont{#1}{#2}{T1}{lmr}{m}{n}%
  \TestFont{#1}{#2}{T1}{lmss}{m}{n}%
  \TestFont{#1}{#2}{T1}{lmr}{b}{n}%
  \TestFont{#1}{#2}{T1}{lmr}{m}{it}%
  \TestFont{#1}{#2}{T1}{lmtt}{m}{n}%
  \TestFont{#1}{#2}{T1}{lmvtt}{m}{n}%
  \TestFont{#1}{#2}{T1}{qtm}{m}{n}%
  \TestFont{#1}{#2}{T1}{qhv}{m}{n}%
  \TestFont{#1}{#2}{T1}{qtm}{b}{n}%
  \TestFont{#1}{#2}{T1}{qtm}{m}{it}%
  \TestFont{#1}{#2}{T1}{qcr}{m}{n}%
  \newpage
}
\makeatother
\hologoList
\end{document}
%</test-list>
%    \end{macrocode}
%
% \section{Installation}
%
% \subsection{Download}
%
% \paragraph{Package.} This package is available on
% CTAN\footnote{\url{ftp://ftp.ctan.org/tex-archive/}}:
% \begin{description}
% \item[\CTAN{macros/latex/contrib/oberdiek/hologo.dtx}] The source file.
% \item[\CTAN{macros/latex/contrib/oberdiek/hologo.pdf}] Documentation.
% \end{description}
%
%
% \paragraph{Bundle.} All the packages of the bundle `oberdiek'
% are also available in a TDS compliant ZIP archive. There
% the packages are already unpacked and the documentation files
% are generated. The files and directories obey the TDS standard.
% \begin{description}
% \item[\CTAN{install/macros/latex/contrib/oberdiek.tds.zip}]
% \end{description}
% \emph{TDS} refers to the standard ``A Directory Structure
% for \TeX\ Files'' (\CTAN{tds/tds.pdf}). Directories
% with \xfile{texmf} in their name are usually organized this way.
%
% \subsection{Bundle installation}
%
% \paragraph{Unpacking.} Unpack the \xfile{oberdiek.tds.zip} in the
% TDS tree (also known as \xfile{texmf} tree) of your choice.
% Example (linux):
% \begin{quote}
%   |unzip oberdiek.tds.zip -d ~/texmf|
% \end{quote}
%
% \paragraph{Script installation.}
% Check the directory \xfile{TDS:scripts/oberdiek/} for
% scripts that need further installation steps.
% Package \xpackage{attachfile2} comes with the Perl script
% \xfile{pdfatfi.pl} that should be installed in such a way
% that it can be called as \texttt{pdfatfi}.
% Example (linux):
% \begin{quote}
%   |chmod +x scripts/oberdiek/pdfatfi.pl|\\
%   |cp scripts/oberdiek/pdfatfi.pl /usr/local/bin/|
% \end{quote}
%
% \subsection{Package installation}
%
% \paragraph{Unpacking.} The \xfile{.dtx} file is a self-extracting
% \docstrip\ archive. The files are extracted by running the
% \xfile{.dtx} through \plainTeX:
% \begin{quote}
%   \verb|tex hologo.dtx|
% \end{quote}
%
% \paragraph{TDS.} Now the different files must be moved into
% the different directories in your installation TDS tree
% (also known as \xfile{texmf} tree):
% \begin{quote}
% \def\t{^^A
% \begin{tabular}{@{}>{\ttfamily}l@{ $\rightarrow$ }>{\ttfamily}l@{}}
%   hologo.sty & tex/generic/oberdiek/hologo.sty\\
%   hologo.pdf & doc/latex/oberdiek/hologo.pdf\\
%   example/hologo-example.tex & doc/latex/oberdiek/example/hologo-example.tex\\
%   test/hologo-test1.tex & doc/latex/oberdiek/test/hologo-test1.tex\\
%   test/hologo-test-spacefactor.tex & doc/latex/oberdiek/test/hologo-test-spacefactor.tex\\
%   test/hologo-test-list.tex & doc/latex/oberdiek/test/hologo-test-list.tex\\
%   hologo.dtx & source/latex/oberdiek/hologo.dtx\\
% \end{tabular}^^A
% }^^A
% \sbox0{\t}^^A
% \ifdim\wd0>\linewidth
%   \begingroup
%     \advance\linewidth by\leftmargin
%     \advance\linewidth by\rightmargin
%   \edef\x{\endgroup
%     \def\noexpand\lw{\the\linewidth}^^A
%   }\x
%   \def\lwbox{^^A
%     \leavevmode
%     \hbox to \linewidth{^^A
%       \kern-\leftmargin\relax
%       \hss
%       \usebox0
%       \hss
%       \kern-\rightmargin\relax
%     }^^A
%   }^^A
%   \ifdim\wd0>\lw
%     \sbox0{\small\t}^^A
%     \ifdim\wd0>\linewidth
%       \ifdim\wd0>\lw
%         \sbox0{\footnotesize\t}^^A
%         \ifdim\wd0>\linewidth
%           \ifdim\wd0>\lw
%             \sbox0{\scriptsize\t}^^A
%             \ifdim\wd0>\linewidth
%               \ifdim\wd0>\lw
%                 \sbox0{\tiny\t}^^A
%                 \ifdim\wd0>\linewidth
%                   \lwbox
%                 \else
%                   \usebox0
%                 \fi
%               \else
%                 \lwbox
%               \fi
%             \else
%               \usebox0
%             \fi
%           \else
%             \lwbox
%           \fi
%         \else
%           \usebox0
%         \fi
%       \else
%         \lwbox
%       \fi
%     \else
%       \usebox0
%     \fi
%   \else
%     \lwbox
%   \fi
% \else
%   \usebox0
% \fi
% \end{quote}
% If you have a \xfile{docstrip.cfg} that configures and enables \docstrip's
% TDS installing feature, then some files can already be in the right
% place, see the documentation of \docstrip.
%
% \subsection{Refresh file name databases}
%
% If your \TeX~distribution
% (\teTeX, \mikTeX, \dots) relies on file name databases, you must refresh
% these. For example, \teTeX\ users run \verb|texhash| or
% \verb|mktexlsr|.
%
% \subsection{Some details for the interested}
%
% \paragraph{Attached source.}
%
% The PDF documentation on CTAN also includes the
% \xfile{.dtx} source file. It can be extracted by
% AcrobatReader 6 or higher. Another option is \textsf{pdftk},
% e.g. unpack the file into the current directory:
% \begin{quote}
%   \verb|pdftk hologo.pdf unpack_files output .|
% \end{quote}
%
% \paragraph{Unpacking with \LaTeX.}
% The \xfile{.dtx} chooses its action depending on the format:
% \begin{description}
% \item[\plainTeX:] Run \docstrip\ and extract the files.
% \item[\LaTeX:] Generate the documentation.
% \end{description}
% If you insist on using \LaTeX\ for \docstrip\ (really,
% \docstrip\ does not need \LaTeX), then inform the autodetect routine
% about your intention:
% \begin{quote}
%   \verb|latex \let\install=y% \iffalse meta-comment
%
% File: hologo.dtx
% Version: 2016/05/12 v1.11
% Info: A logo collection with bookmark support
%
% Copyright (C) 2010-2012 by
%    Heiko Oberdiek <heiko.oberdiek at googlemail.com>
%
% This work may be distributed and/or modified under the
% conditions of the LaTeX Project Public License, either
% version 1.3c of this license or (at your option) any later
% version. This version of this license is in
%    http://www.latex-project.org/lppl/lppl-1-3c.txt
% and the latest version of this license is in
%    http://www.latex-project.org/lppl.txt
% and version 1.3 or later is part of all distributions of
% LaTeX version 2005/12/01 or later.
%
% This work has the LPPL maintenance status "maintained".
%
% This Current Maintainer of this work is Heiko Oberdiek.
%
% The Base Interpreter refers to any `TeX-Format',
% because some files are installed in TDS:tex/generic//.
%
% This work consists of the main source file hologo.dtx
% and the derived files
%    hologo.sty, hologo.pdf, hologo.ins, hologo.drv, hologo-example.tex,
%    hologo-test1.tex, hologo-test-spacefactor.tex,
%    hologo-test-list.tex.
%
% Distribution:
%    CTAN:macros/latex/contrib/oberdiek/hologo.dtx
%    CTAN:macros/latex/contrib/oberdiek/hologo.pdf
%
% Unpacking:
%    (a) If hologo.ins is present:
%           tex hologo.ins
%    (b) Without hologo.ins:
%           tex hologo.dtx
%    (c) If you insist on using LaTeX
%           latex \let\install=y\input{hologo.dtx}
%        (quote the arguments according to the demands of your shell)
%
% Documentation:
%    (a) If hologo.drv is present:
%           latex hologo.drv
%    (b) Without hologo.drv:
%           latex hologo.dtx; ...
%    The class ltxdoc loads the configuration file ltxdoc.cfg
%    if available. Here you can specify further options, e.g.
%    use A4 as paper format:
%       \PassOptionsToClass{a4paper}{article}
%
%    Programm calls to get the documentation (example):
%       pdflatex hologo.dtx
%       makeindex -s gind.ist hologo.idx
%       pdflatex hologo.dtx
%       makeindex -s gind.ist hologo.idx
%       pdflatex hologo.dtx
%
% Installation:
%    TDS:tex/generic/oberdiek/hologo.sty
%    TDS:doc/latex/oberdiek/hologo.pdf
%    TDS:doc/latex/oberdiek/example/hologo-example.tex
%    TDS:doc/latex/oberdiek/test/hologo-test1.tex
%    TDS:doc/latex/oberdiek/test/hologo-test-spacefactor.tex
%    TDS:doc/latex/oberdiek/test/hologo-test-list.tex
%    TDS:source/latex/oberdiek/hologo.dtx
%
%<*ignore>
\begingroup
  \catcode123=1 %
  \catcode125=2 %
  \def\x{LaTeX2e}%
\expandafter\endgroup
\ifcase 0\ifx\install y1\fi\expandafter
         \ifx\csname processbatchFile\endcsname\relax\else1\fi
         \ifx\fmtname\x\else 1\fi\relax
\else\csname fi\endcsname
%</ignore>
%<*install>
\input docstrip.tex
\Msg{************************************************************************}
\Msg{* Installation}
\Msg{* Package: hologo 2016/05/12 v1.11 A logo collection with bookmark support (HO)}
\Msg{************************************************************************}

\keepsilent
\askforoverwritefalse

\let\MetaPrefix\relax
\preamble

This is a generated file.

Project: hologo
Version: 2016/05/12 v1.11

Copyright (C) 2010-2012 by
   Heiko Oberdiek <heiko.oberdiek at googlemail.com>

This work may be distributed and/or modified under the
conditions of the LaTeX Project Public License, either
version 1.3c of this license or (at your option) any later
version. This version of this license is in
   http://www.latex-project.org/lppl/lppl-1-3c.txt
and the latest version of this license is in
   http://www.latex-project.org/lppl.txt
and version 1.3 or later is part of all distributions of
LaTeX version 2005/12/01 or later.

This work has the LPPL maintenance status "maintained".

This Current Maintainer of this work is Heiko Oberdiek.

The Base Interpreter refers to any `TeX-Format',
because some files are installed in TDS:tex/generic//.

This work consists of the main source file hologo.dtx
and the derived files
   hologo.sty, hologo.pdf, hologo.ins, hologo.drv, hologo-example.tex,
   hologo-test1.tex, hologo-test-spacefactor.tex,
   hologo-test-list.tex.

\endpreamble
\let\MetaPrefix\DoubleperCent

\generate{%
  \file{hologo.ins}{\from{hologo.dtx}{install}}%
  \file{hologo.drv}{\from{hologo.dtx}{driver}}%
  \usedir{tex/generic/oberdiek}%
  \file{hologo.sty}{\from{hologo.dtx}{package}}%
  \usedir{doc/latex/oberdiek/example}%
  \file{hologo-example.tex}{\from{hologo.dtx}{example}}%
  \usedir{doc/latex/oberdiek/test}%
  \file{hologo-test1.tex}{\from{hologo.dtx}{test1}}%
  \file{hologo-test-spacefactor.tex}{\from{hologo.dtx}{test-spacefactor}}%
  \file{hologo-test-list.tex}{\from{hologo.dtx}{test-list}}%
  \nopreamble
  \nopostamble
  \usedir{source/latex/oberdiek/catalogue}%
  \file{hologo.xml}{\from{hologo.dtx}{catalogue}}%
}

\catcode32=13\relax% active space
\let =\space%
\Msg{************************************************************************}
\Msg{*}
\Msg{* To finish the installation you have to move the following}
\Msg{* file into a directory searched by TeX:}
\Msg{*}
\Msg{*     hologo.sty}
\Msg{*}
\Msg{* To produce the documentation run the file `hologo.drv'}
\Msg{* through LaTeX.}
\Msg{*}
\Msg{* Happy TeXing!}
\Msg{*}
\Msg{************************************************************************}

\endbatchfile
%</install>
%<*ignore>
\fi
%</ignore>
%<*driver>
\NeedsTeXFormat{LaTeX2e}
\ProvidesFile{hologo.drv}%
  [2016/05/12 v1.11 A logo collection with bookmark support (HO)]%
\documentclass{ltxdoc}
\usepackage{holtxdoc}[2011/11/22]
\usepackage{hologo}[2016/05/12]
\usepackage{longtable}
\usepackage{array}
\usepackage{paralist}
%\usepackage[T1]{fontenc}
%\usepackage{lmodern}
\begin{document}
  \DocInput{hologo.dtx}%
\end{document}
%</driver>
% \fi
%
%
% \CharacterTable
%  {Upper-case    \A\B\C\D\E\F\G\H\I\J\K\L\M\N\O\P\Q\R\S\T\U\V\W\X\Y\Z
%   Lower-case    \a\b\c\d\e\f\g\h\i\j\k\l\m\n\o\p\q\r\s\t\u\v\w\x\y\z
%   Digits        \0\1\2\3\4\5\6\7\8\9
%   Exclamation   \!     Double quote  \"     Hash (number) \#
%   Dollar        \$     Percent       \%     Ampersand     \&
%   Acute accent  \'     Left paren    \(     Right paren   \)
%   Asterisk      \*     Plus          \+     Comma         \,
%   Minus         \-     Point         \.     Solidus       \/
%   Colon         \:     Semicolon     \;     Less than     \<
%   Equals        \=     Greater than  \>     Question mark \?
%   Commercial at \@     Left bracket  \[     Backslash     \\
%   Right bracket \]     Circumflex    \^     Underscore    \_
%   Grave accent  \`     Left brace    \{     Vertical bar  \|
%   Right brace   \}     Tilde         \~}
%
% \GetFileInfo{hologo.drv}
%
% \title{The \xpackage{hologo} package}
% \date{2016/05/12 v1.11}
% \author{Heiko Oberdiek\\\xemail{heiko.oberdiek at googlemail.com}}
%
% \maketitle
%
% \begin{abstract}
% This package starts a collection of logos with support for bookmarks
% strings.
% \end{abstract}
%
% \tableofcontents
%
% \section{Documentation}
%
% \subsection{Logo macros}
%
% \begin{declcs}{hologo} \M{name}
% \end{declcs}
% Macro \cs{hologo} sets the logo with name \meta{name}.
% The following table shows the supported names.
%
% \begingroup
%   \def\hologoEntry#1#2#3{^^A
%     #1&#2&\hologoLogoSetup{#1}{variant=#2}\hologo{#1}&#3\tabularnewline
%   }
%   \begin{longtable}{>{\ttfamily}l>{\ttfamily}lll}
%     \rmfamily\bfseries{name} & \rmfamily\bfseries variant
%     & \bfseries logo & \bfseries since\\
%     \hline
%     \endhead
%     \hologoList
%   \end{longtable}
% \endgroup
%
% \begin{declcs}{Hologo} \M{name}
% \end{declcs}
% Macro \cs{Hologo} starts the logo \meta{name} with an uppercase
% letter. As an exception small greek letters are not converted
% to uppercase. Examples, see \hologo{eTeX} and \hologo{ExTeX}.
%
% \subsection{Setup macros}
%
% The package does not support package options, but the following
% setup macros can be used to set options.
%
% \begin{declcs}{hologoSetup} \M{key value list}
% \end{declcs}
% Macro \cs{hologoSetup} sets global options.
%
% \begin{declcs}{hologoLogoSetup} \M{logo} \M{key value list}
% \end{declcs}
% Some options can also be used to configure a logo.
% These settings take precedence over global option settings.
%
% \subsection{Options}\label{sec:options}
%
% There are boolean and string options:
% \begin{description}
% \item[Boolean option:]
% It takes |true| or |false|
% as value. If the value is omitted, then |true| is used.
% \item[String option:]
% A value must be given as string. (But the string might be empty.)
% \end{description}
% The following options can be used both in \cs{hologoSetup}
% and \cs{hologoLogoSetup}:
% \begin{description}
% \def\entry#1{\item[\xoption{#1}:]}
% \entry{break}
%   enables or disables line breaks inside the logo. This setting is
%   refined by options \xoption{hyphenbreak}, \xoption{spacebreak}
%   or \xoption{discretionarybreak}.
%   Default is |false|.
% \entry{hyphenbreak}
%   enables or disables the line break right after the hyphen character.
% \entry{spacebreak}
%   enables or disables line breaks at space characters.
% \entry{discretionarybreak}
%   enables or disables line breaks at hyphenation points
%   (inserted by \cs{-}).
% \end{description}
% Macro \cs{hologoLogoSetup} also knows:
% \begin{description}
% \item[\xoption{variant}:]
%   This is a string option. It specifies a variant of a logo that
%   must exist. An empty string selects the package default variant.
% \end{description}
% Example:
% \begin{quote}
%   |\hologoSetup{break=false}|\\
%   |\hologoLogoSetup{plainTeX}{variant=hyphen,hyphenbreak}|\\
%   Then ``plain-\TeX'' contains one break point after the hyphen.
% \end{quote}
%
% \subsection{Driver options}
%
% Sometimes graphical operations are needed to construct some
% glyphs (e.g.\ \hologo{XeTeX}). If package \xpackage{graphics}
% or package \xpackage{pgf} are found, then the macros are taken
% from there. Otherwise the packge defines its own operations
% and therefore needs the driver information. Many drivers are
% detected automatically (\hologo{pdfTeX}/\hologo{LuaTeX}
% in PDF mode, \hologo{XeTeX}, \hologo{VTeX}). These have precedence
% over a driver option. The driver can be given as package option
% or using \cs{hologoDriverSetup}.
% The following list contains the recognized driver options:
% \begin{itemize}
% \item \xoption{pdftex}, \xoption{luatex}
% \item \xoption{dvipdfm}, \xoption{dvipdfmx}
% \item \xoption{dvips}, \xoption{dvipsone}, \xoption{xdvi}
% \item \xoption{xetex}
% \item \xoption{vtex}
% \end{itemize}
% The left driver of a line is the driver name that is used internally.
% The following names are aliases for drivers that use the
% same method. Therefore the entry in the \xext{log} file for
% the used driver prints the internally used driver name.
% \begin{description}
% \item[\xoption{driverfallback}:]
%   This option expects a driver that is used,
%   if the driver could not be detected automatically.
% \end{description}
%
% \begin{declcs}{hologoDriverSetup} \M{driver option}
% \end{declcs}
% The driver can also be configured after package loading
% using \cs{hologoDriverSetup}, also the way for \hologo{plainTeX}
% to setup the driver.
%
% \subsection{Font setup}
%
% Some logos require a special font, but should also be usable by
% \hologo{plainTeX}. Therefore the package provides some ways
% to influence the font settings. The options below
% take font settings as values. Both font commands
% such as \cs{sffamily} and macros that take one argument
% like \cs{textsf} can be used.
%
% \begin{declcs}{hologoFontSetup} \M{key value list}
% \end{declcs}
% Macro \cs{hologoFontSetup} sets the fonts for all logos.
% Supported keys:
% \begin{description}
% \def\entry#1{\item[\xoption{#1}:]}
% \entry{general}
%   This font is used for all logos. The default is empty.
%   That means no special font is used.
% \entry{bibsf}
%   This font is used for
%   {\hologoLogoSetup{BibTeX}{variant=sf}\hologo{BibTeX}}
%   with variant \xoption{sf}.
% \entry{rm}
%   This font is a serif font. It is used for \hologo{ExTeX}.
% \entry{sc}
%   This font specifies a small caps font. It is used for
%   {\hologoLogoSetup{BibTeX}{variant=sc}\hologo{BibTeX}}
%   with variant \xoption{sc}.
% \entry{sf}
%   This font specifies a sans serif font. The default
%   is \cs{sffamily}, then \cs{sf} is tried. Otherwise
%   a warning is given. It is used by \hologo{KOMAScript}.
% \entry{sy}
%   This is the font for math symbols (e.g. cmsy).
%   It is used by \hologo{AmS}, \hologo{NTS}, \hologo{ExTeX}.
% \entry{logo}
%   \hologo{METAFONT} and \hologo{METAPOST} are using that font.
%   In \hologo{LaTeX} \cs{logofamily} is used and
%   the definitions of package \xpackage{mflogo} are used
%   if the package is not loaded.
%   Otherwise the \cs{tenlogo} is used and defined
%   if it does not already exists.
% \end{description}
%
% \begin{declcs}{hologoLogoFontSetup} \M{logo} \M{key value list}
% \end{declcs}
% Fonts can also be set for a logo or logo component separately,
% see the following list.
% The keys are the same as for \cs{hologoFontSetup}.
%
% \begin{longtable}{>{\ttfamily}l>{\sffamily}ll}
%   \meta{logo} & keys & result\\
%   \hline
%   \endhead
%   BibTeX & bibsf & {\hologoLogoSetup{BibTeX}{variant=sf}\hologo{BibTeX}}\\[.5ex]
%   BibTeX & sc & {\hologoLogoSetup{BibTeX}{variant=sc}\hologo{BibTeX}}\\[.5ex]
%   ExTeX & rm & \hologo{ExTeX}\\
%   SliTeX & rm & \hologo{SliTeX}\\[.5ex]
%   AmS & sy & \hologo{AmS}\\
%   ExTeX & sy & \hologo{ExTeX}\\
%   NTS & sy & \hologo{NTS}\\[.5ex]
%   KOMAScript & sf & \hologo{KOMAScript}\\[.5ex]
%   METAFONT & logo & \hologo{METAFONT}\\
%   METAPOST & logo & \hologo{METAPOST}\\[.5ex]
%   SliTeX & sc \hologo{SliTeX}
% \end{longtable}
%
% \subsubsection{Font order}
%
% For all logos the font \xoption{general} is applied first.
% Example:
%\begin{quote}
%|\hologoFontSetup{general=\color{red}}|
%\end{quote}
% will print red logos.
% Then if the font uses a special font \xoption{sf}, for example,
% the font is applied that is setup by \cs{hologoLogoFontSetup}.
% If this font is not setup, then the common font setup
% by \cs{hologoFontSetup} is used. Otherwise a warning is given,
% that there is no font configured.
%
% \subsection{Additional user macros}
%
% Usually a variant of a logo is configured by using
% \cs{hologoLogoSetup}, because it is bad style to mix
% different variants of the same logo in the same text.
% There the following macros are a convenience for testing.
%
% \begin{declcs}{hologoVariant} \M{name} \M{variant}\\
%   \cs{HologoVariant} \M{name} \M{variant}
% \end{declcs}
% Logo \meta{name} is set using \meta{variant} that specifies
% explicitely which variant of the macro is used. If the argument
% is empty, then the default form of the logo is used
% (configurable by \cs{hologoLogoSetup}).
%
% \cs{HologoVariant} is used if the logo is set in a context
% that needs an uppercase first letter (beginning of a sentence, \dots).
%
% \begin{declcs}{hologoList}\\
%   \cs{hologoEntry} \M{logo} \M{variant} \M{since}
% \end{declcs}
% Macro \cs{hologoList} contains all logos that are provided
% by the package including variants. The list consists of calls
% of \cs{hologoEntry} with three arguments starting with the
% logo name \meta{logo} and its variant \meta{variant}. An empty
% variant means the current default. Argument \meta{since} specifies
% with version of the package \xpackage{hologo} is needed to get
% the logo. If the logo is fixed, then the date gets updated.
% Therefore the date \meta{since} is not exactly the date of
% the first introduction, but rather the date of the latest fix.
%
% Before \cs{hologoList} can be used, macro \cs{hologoEntry} needs
% a definition. The example file in section \ref{sec:example}
% shows applications of \cs{hologoList}.
%
% \subsection{Supported contexts}
%
% Macros \cs{hologo} and friends support special contexts:
% \begin{itemize}
% \item \hologo{LaTeX}'s protection mechanism.
% \item Bookmarks of package \xpackage{hyperref}.
% \item Package \xpackage{tex4ht}.
% \item The macros can be used inside \cs{csname} constructs,
%   if \cs{ifincsname} is available (\hologo{pdfTeX}, \hologo{XeTeX},
%   \hologo{LuaTeX}).
% \end{itemize}
%
% \subsection{Example}
% \label{sec:example}
%
% The following example prints the logos in different fonts.
%    \begin{macrocode}
%<*example>
%<<verbatim
\NeedsTeXFormat{LaTeX2e}
\documentclass[a4paper]{article}
\usepackage[
  hmargin=20mm,
  vmargin=20mm,
]{geometry}
\pagestyle{empty}
\usepackage{hologo}[2016/05/12]
\usepackage{longtable}
\usepackage{array}
\setlength{\extrarowheight}{2pt}
\usepackage[T1]{fontenc}
\usepackage{lmodern}
\usepackage{pdflscape}
\usepackage[
  pdfencoding=auto,
]{hyperref}
\hypersetup{
  pdfauthor={Heiko Oberdiek},
  pdftitle={Example for package `hologo'},
  pdfsubject={Logos with fonts lmr, lmss, qtm, qpl, qhv},
}
\usepackage{bookmark}

% Print the logo list on the console

\begingroup
  \typeout{}%
  \typeout{*** Begin of logo list ***}%
  \newcommand*{\hologoEntry}[3]{%
    \typeout{#1 \ifx\\#2\\\else(#2) \fi[#3]}%
  }%
  \hologoList
  \typeout{*** End of logo list ***}%
  \typeout{}%
\endgroup

\begin{document}
\begin{landscape}

  \section{Example file for package `hologo'}

  % Table for font names

  \begin{longtable}{>{\bfseries}ll}
    \textbf{font} & \textbf{Font name}\\
    \hline
    lmr & Latin Modern Roman\\
    lmss & Latin Modern Sans\\
    qtm & \TeX\ Gyre Termes\\
    qhv & \TeX\ Gyre Heros\\
    qpl & \TeX\ Gyre Pagella\\
  \end{longtable}

  % Logo list with logos in different fonts

  \begingroup
    \newcommand*{\SetVariant}[2]{%
      \ifx\\#2\\%
      \else
        \hologoLogoSetup{#1}{variant=#2}%
      \fi
    }%
    \newcommand*{\hologoEntry}[3]{%
      \SetVariant{#1}{#2}%
      \raisebox{1em}[0pt][0pt]{\hypertarget{#1@#2}{}}%
      \bookmark[%
        dest={#1@#2},%
      ]{%
        #1\ifx\\#2\\\else\space(#2)\fi: \Hologo{#1}, \hologo{#1} %
        [Unicode]%
      }%
      \hypersetup{unicode=false}%
      \bookmark[%
        dest={#1@#2},%
      ]{%
        #1\ifx\\#2\\\else\space(#2)\fi: \Hologo{#1}, \hologo{#1} %
        [PDFDocEncoding]%
      }%
      \texttt{#1}%
      &%
      \texttt{#2}%
      &%
      \Hologo{#1}%
      &%
      \SetVariant{#1}{#2}%
      \hologo{#1}%
      &%
      \SetVariant{#1}{#2}%
      \fontfamily{qtm}\selectfont
      \hologo{#1}%
      &%
      \SetVariant{#1}{#2}%
      \fontfamily{qpl}\selectfont
      \hologo{#1}%
      &%
      \SetVariant{#1}{#2}%
      \textsf{\hologo{#1}}%
      &%
      \SetVariant{#1}{#2}%
      \fontfamily{qhv}\selectfont
      \hologo{#1}%
      \tabularnewline
    }%
    \begin{longtable}{llllllll}%
      \textbf{\textit{logo}} & \textbf{\textit{variant}} &
      \texttt{\string\Hologo} &
      \textbf{lmr} & \textbf{qtm} & \textbf{qpl} &
      \textbf{lmss} & \textbf{qhv}
      \tabularnewline
      \hline
      \endhead
      \hologoList
    \end{longtable}%
  \endgroup

\end{landscape}
\end{document}
%verbatim
%</example>
%    \end{macrocode}
%
% \StopEventually{
% }
%
% \section{Implementation}
%    \begin{macrocode}
%<*package>
%    \end{macrocode}
%    Reload check, especially if the package is not used with \LaTeX.
%    \begin{macrocode}
\begingroup\catcode61\catcode48\catcode32=10\relax%
  \catcode13=5 % ^^M
  \endlinechar=13 %
  \catcode35=6 % #
  \catcode39=12 % '
  \catcode44=12 % ,
  \catcode45=12 % -
  \catcode46=12 % .
  \catcode58=12 % :
  \catcode64=11 % @
  \catcode123=1 % {
  \catcode125=2 % }
  \expandafter\let\expandafter\x\csname ver@hologo.sty\endcsname
  \ifx\x\relax % plain-TeX, first loading
  \else
    \def\empty{}%
    \ifx\x\empty % LaTeX, first loading,
      % variable is initialized, but \ProvidesPackage not yet seen
    \else
      \expandafter\ifx\csname PackageInfo\endcsname\relax
        \def\x#1#2{%
          \immediate\write-1{Package #1 Info: #2.}%
        }%
      \else
        \def\x#1#2{\PackageInfo{#1}{#2, stopped}}%
      \fi
      \x{hologo}{The package is already loaded}%
      \aftergroup\endinput
    \fi
  \fi
\endgroup%
%    \end{macrocode}
%    Package identification:
%    \begin{macrocode}
\begingroup\catcode61\catcode48\catcode32=10\relax%
  \catcode13=5 % ^^M
  \endlinechar=13 %
  \catcode35=6 % #
  \catcode39=12 % '
  \catcode40=12 % (
  \catcode41=12 % )
  \catcode44=12 % ,
  \catcode45=12 % -
  \catcode46=12 % .
  \catcode47=12 % /
  \catcode58=12 % :
  \catcode64=11 % @
  \catcode91=12 % [
  \catcode93=12 % ]
  \catcode123=1 % {
  \catcode125=2 % }
  \expandafter\ifx\csname ProvidesPackage\endcsname\relax
    \def\x#1#2#3[#4]{\endgroup
      \immediate\write-1{Package: #3 #4}%
      \xdef#1{#4}%
    }%
  \else
    \def\x#1#2[#3]{\endgroup
      #2[{#3}]%
      \ifx#1\@undefined
        \xdef#1{#3}%
      \fi
      \ifx#1\relax
        \xdef#1{#3}%
      \fi
    }%
  \fi
\expandafter\x\csname ver@hologo.sty\endcsname
\ProvidesPackage{hologo}%
  [2016/05/12 v1.11 A logo collection with bookmark support (HO)]%
%    \end{macrocode}
%
%    \begin{macrocode}
\begingroup\catcode61\catcode48\catcode32=10\relax%
  \catcode13=5 % ^^M
  \endlinechar=13 %
  \catcode123=1 % {
  \catcode125=2 % }
  \catcode64=11 % @
  \def\x{\endgroup
    \expandafter\edef\csname HOLOGO@AtEnd\endcsname{%
      \endlinechar=\the\endlinechar\relax
      \catcode13=\the\catcode13\relax
      \catcode32=\the\catcode32\relax
      \catcode35=\the\catcode35\relax
      \catcode61=\the\catcode61\relax
      \catcode64=\the\catcode64\relax
      \catcode123=\the\catcode123\relax
      \catcode125=\the\catcode125\relax
    }%
  }%
\x\catcode61\catcode48\catcode32=10\relax%
\catcode13=5 % ^^M
\endlinechar=13 %
\catcode35=6 % #
\catcode64=11 % @
\catcode123=1 % {
\catcode125=2 % }
\def\TMP@EnsureCode#1#2{%
  \edef\HOLOGO@AtEnd{%
    \HOLOGO@AtEnd
    \catcode#1=\the\catcode#1\relax
  }%
  \catcode#1=#2\relax
}
\TMP@EnsureCode{10}{12}% ^^J
\TMP@EnsureCode{33}{12}% !
\TMP@EnsureCode{34}{12}% "
\TMP@EnsureCode{36}{3}% $
\TMP@EnsureCode{38}{4}% &
\TMP@EnsureCode{39}{12}% '
\TMP@EnsureCode{40}{12}% (
\TMP@EnsureCode{41}{12}% )
\TMP@EnsureCode{42}{12}% *
\TMP@EnsureCode{43}{12}% +
\TMP@EnsureCode{44}{12}% ,
\TMP@EnsureCode{45}{12}% -
\TMP@EnsureCode{46}{12}% .
\TMP@EnsureCode{47}{12}% /
\TMP@EnsureCode{58}{12}% :
\TMP@EnsureCode{59}{12}% ;
\TMP@EnsureCode{60}{12}% <
\TMP@EnsureCode{62}{12}% >
\TMP@EnsureCode{63}{12}% ?
\TMP@EnsureCode{91}{12}% [
\TMP@EnsureCode{93}{12}% ]
\TMP@EnsureCode{94}{7}% ^ (superscript)
\TMP@EnsureCode{95}{8}% _ (subscript)
\TMP@EnsureCode{96}{12}% `
\TMP@EnsureCode{124}{12}% |
\edef\HOLOGO@AtEnd{%
  \HOLOGO@AtEnd
  \escapechar\the\escapechar\relax
  \noexpand\endinput
}
\escapechar=92 %
%    \end{macrocode}
%
% \subsection{Logo list}
%
%    \begin{macro}{\hologoList}
%    \begin{macrocode}
\def\hologoList{%
  \hologoEntry{(La)TeX}{}{2011/10/01}%
  \hologoEntry{AmSLaTeX}{}{2010/04/16}%
  \hologoEntry{AmSTeX}{}{2010/04/16}%
  \hologoEntry{biber}{}{2011/10/01}%
  \hologoEntry{BibTeX}{}{2011/10/01}%
  \hologoEntry{BibTeX}{sf}{2011/10/01}%
  \hologoEntry{BibTeX}{sc}{2011/10/01}%
  \hologoEntry{BibTeX8}{}{2011/11/22}%
  \hologoEntry{ConTeXt}{}{2011/03/25}%
  \hologoEntry{ConTeXt}{narrow}{2011/03/25}%
  \hologoEntry{ConTeXt}{simple}{2011/03/25}%
  \hologoEntry{emTeX}{}{2010/04/26}%
  \hologoEntry{eTeX}{}{2010/04/08}%
  \hologoEntry{ExTeX}{}{2011/10/01}%
  \hologoEntry{HanTheThanh}{}{2011/11/29}%
  \hologoEntry{iniTeX}{}{2011/10/01}%
  \hologoEntry{KOMAScript}{}{2011/10/01}%
  \hologoEntry{La}{}{2010/05/08}%
  \hologoEntry{LaTeX}{}{2010/04/08}%
  \hologoEntry{LaTeX2e}{}{2010/04/08}%
  \hologoEntry{LaTeX3}{}{2010/04/24}%
  \hologoEntry{LaTeXe}{}{2010/04/08}%
  \hologoEntry{LaTeXML}{}{2011/11/22}%
  \hologoEntry{LaTeXTeX}{}{2011/10/01}%
  \hologoEntry{LuaLaTeX}{}{2010/04/08}%
  \hologoEntry{LuaTeX}{}{2010/04/08}%
  \hologoEntry{LyX}{}{2011/10/01}%
  \hologoEntry{METAFONT}{}{2011/10/01}%
  \hologoEntry{MetaFun}{}{2011/10/01}%
  \hologoEntry{METAPOST}{}{2011/10/01}%
  \hologoEntry{MetaPost}{}{2011/10/01}%
  \hologoEntry{MiKTeX}{}{2011/10/01}%
  \hologoEntry{NTS}{}{2011/10/01}%
  \hologoEntry{OzMF}{}{2011/10/01}%
  \hologoEntry{OzMP}{}{2011/10/01}%
  \hologoEntry{OzTeX}{}{2011/10/01}%
  \hologoEntry{OzTtH}{}{2011/10/01}%
  \hologoEntry{PCTeX}{}{2011/10/01}%
  \hologoEntry{pdfTeX}{}{2011/10/01}%
  \hologoEntry{pdfLaTeX}{}{2011/10/01}%
  \hologoEntry{PiC}{}{2011/10/01}%
  \hologoEntry{PiCTeX}{}{2011/10/01}%
  \hologoEntry{plainTeX}{}{2010/04/08}%
  \hologoEntry{plainTeX}{space}{2010/04/16}%
  \hologoEntry{plainTeX}{hyphen}{2010/04/16}%
  \hologoEntry{plainTeX}{runtogether}{2010/04/16}%
  \hologoEntry{SageTeX}{}{2011/11/22}%
  \hologoEntry{SLiTeX}{}{2011/10/01}%
  \hologoEntry{SLiTeX}{lift}{2011/10/01}%
  \hologoEntry{SLiTeX}{narrow}{2011/10/01}%
  \hologoEntry{SLiTeX}{simple}{2011/10/01}%
  \hologoEntry{SliTeX}{}{2011/10/01}%
  \hologoEntry{SliTeX}{narrow}{2011/10/01}%
  \hologoEntry{SliTeX}{simple}{2011/10/01}%
  \hologoEntry{SliTeX}{lift}{2011/10/01}%
  \hologoEntry{teTeX}{}{2011/10/01}%
  \hologoEntry{TeX}{}{2010/04/08}%
  \hologoEntry{TeX4ht}{}{2011/11/22}%
  \hologoEntry{TTH}{}{2011/11/22}%
  \hologoEntry{virTeX}{}{2011/10/01}%
  \hologoEntry{VTeX}{}{2010/04/24}%
  \hologoEntry{Xe}{}{2010/04/08}%
  \hologoEntry{XeLaTeX}{}{2010/04/08}%
  \hologoEntry{XeTeX}{}{2010/04/08}%
}
%    \end{macrocode}
%    \end{macro}
%
% \subsection{Load resources}
%
%    \begin{macrocode}
\begingroup\expandafter\expandafter\expandafter\endgroup
\expandafter\ifx\csname RequirePackage\endcsname\relax
  \def\TMP@RequirePackage#1[#2]{%
    \begingroup\expandafter\expandafter\expandafter\endgroup
    \expandafter\ifx\csname ver@#1.sty\endcsname\relax
      \input #1.sty\relax
    \fi
  }%
  \TMP@RequirePackage{ltxcmds}[2011/02/04]%
  \TMP@RequirePackage{infwarerr}[2010/04/08]%
  \TMP@RequirePackage{kvsetkeys}[2010/03/01]%
  \TMP@RequirePackage{kvdefinekeys}[2010/03/01]%
  \TMP@RequirePackage{pdftexcmds}[2010/04/01]%
  \TMP@RequirePackage{ifpdf}[2010/01/28]%
  \TMP@RequirePackage{ifluatex}[2010/03/01]%
  \ltx@IfUndefined{newif}{%
    \expandafter\let\csname newif\endcsname\ltx@newif
  }{}%
  \TMP@RequirePackage{ifxetex}[2009/01/23]%
  \TMP@RequirePackage{ifvtex}[2010/03/01]%
\else
  \RequirePackage{ltxcmds}[2011/02/04]%
  \RequirePackage{infwarerr}[2010/04/08]%
  \RequirePackage{kvsetkeys}[2010/03/01]%
  \RequirePackage{kvdefinekeys}[2010/03/01]%
  \RequirePackage{pdftexcmds}[2010/04/01]%
  \RequirePackage{ifpdf}[2010/01/28]%
  \RequirePackage{ifluatex}[2010/03/01]%
  \RequirePackage{ifxetex}[2009/01/23]%
  \RequirePackage{ifvtex}[2010/03/01]%
\fi
%    \end{macrocode}
%
%    \begin{macro}{\HOLOGO@IfDefined}
%    \begin{macrocode}
\def\HOLOGO@IfExists#1{%
  \ifx\@undefined#1%
    \expandafter\ltx@secondoftwo
  \else
    \ifx\relax#1%
      \expandafter\ltx@secondoftwo
    \else
      \expandafter\expandafter\expandafter\ltx@firstoftwo
    \fi
  \fi
}
%    \end{macrocode}
%    \end{macro}
%
% \subsection{Setup macros}
%
%    \begin{macro}{\hologoSetup}
%    \begin{macrocode}
\def\hologoSetup{%
  \let\HOLOGO@name\relax
  \HOLOGO@Setup
}
%    \end{macrocode}
%    \end{macro}
%
%    \begin{macro}{\hologoLogoSetup}
%    \begin{macrocode}
\def\hologoLogoSetup#1{%
  \edef\HOLOGO@name{#1}%
  \ltx@IfUndefined{HoLogo@\HOLOGO@name}{%
    \@PackageError{hologo}{%
      Unknown logo `\HOLOGO@name'%
    }\@ehc
    \ltx@gobble
  }{%
    \HOLOGO@Setup
  }%
}
%    \end{macrocode}
%    \end{macro}
%
%    \begin{macro}{\HOLOGO@Setup}
%    \begin{macrocode}
\def\HOLOGO@Setup{%
  \kvsetkeys{HoLogo}%
}
%    \end{macrocode}
%    \end{macro}
%
% \subsection{Options}
%
%    \begin{macro}{\HOLOGO@DeclareBoolOption}
%    \begin{macrocode}
\def\HOLOGO@DeclareBoolOption#1{%
  \expandafter\chardef\csname HOLOGOOPT@#1\endcsname\ltx@zero
  \kv@define@key{HoLogo}{#1}[true]{%
    \def\HOLOGO@temp{##1}%
    \ifx\HOLOGO@temp\HOLOGO@true
      \ifx\HOLOGO@name\relax
        \expandafter\chardef\csname HOLOGOOPT@#1\endcsname=\ltx@one
      \else
        \expandafter\chardef\csname
        HoLogoOpt@#1@\HOLOGO@name\endcsname\ltx@one
      \fi
      \HOLOGO@SetBreakAll{#1}%
    \else
      \ifx\HOLOGO@temp\HOLOGO@false
        \ifx\HOLOGO@name\relax
          \expandafter\chardef\csname HOLOGOOPT@#1\endcsname=\ltx@zero
        \else
          \expandafter\chardef\csname
          HoLogoOpt@#1@\HOLOGO@name\endcsname=\ltx@zero
        \fi
        \HOLOGO@SetBreakAll{#1}%
      \else
        \@PackageError{hologo}{%
          Unknown value `##1' for boolean option `#1'.\MessageBreak
          Known values are `true' and `false'%
        }\@ehc
      \fi
    \fi
  }%
}
%    \end{macrocode}
%    \end{macro}
%
%    \begin{macro}{\HOLOGO@SetBreakAll}
%    \begin{macrocode}
\def\HOLOGO@SetBreakAll#1{%
  \def\HOLOGO@temp{#1}%
  \ifx\HOLOGO@temp\HOLOGO@break
    \ifx\HOLOGO@name\relax
      \chardef\HOLOGOOPT@hyphenbreak=\HOLOGOOPT@break
      \chardef\HOLOGOOPT@spacebreak=\HOLOGOOPT@break
      \chardef\HOLOGOOPT@discretionarybreak=\HOLOGOOPT@break
    \else
      \expandafter\chardef
         \csname HoLogoOpt@hyphenbreak@\HOLOGO@name\endcsname=%
         \csname HoLogoOpt@break@\HOLOGO@name\endcsname
      \expandafter\chardef
         \csname HoLogoOpt@spacebreak@\HOLOGO@name\endcsname=%
         \csname HoLogoOpt@break@\HOLOGO@name\endcsname
      \expandafter\chardef
         \csname HoLogoOpt@discretionarybreak@\HOLOGO@name
             \endcsname=%
         \csname HoLogoOpt@break@\HOLOGO@name\endcsname
    \fi
  \fi
}
%    \end{macrocode}
%    \end{macro}
%
%    \begin{macro}{\HOLOGO@true}
%    \begin{macrocode}
\def\HOLOGO@true{true}
%    \end{macrocode}
%    \end{macro}
%    \begin{macro}{\HOLOGO@false}
%    \begin{macrocode}
\def\HOLOGO@false{false}
%    \end{macrocode}
%    \end{macro}
%    \begin{macro}{\HOLOGO@break}
%    \begin{macrocode}
\def\HOLOGO@break{break}
%    \end{macrocode}
%    \end{macro}
%
%    \begin{macrocode}
\HOLOGO@DeclareBoolOption{break}
\HOLOGO@DeclareBoolOption{hyphenbreak}
\HOLOGO@DeclareBoolOption{spacebreak}
\HOLOGO@DeclareBoolOption{discretionarybreak}
%    \end{macrocode}
%
%    \begin{macrocode}
\kv@define@key{HoLogo}{variant}{%
  \ifx\HOLOGO@name\relax
    \@PackageError{hologo}{%
      Option `variant' is not available in \string\hologoSetup,%
      \MessageBreak
      Use \string\hologoLogoSetup\space instead%
    }\@ehc
  \else
    \edef\HOLOGO@temp{#1}%
    \ifx\HOLOGO@temp\ltx@empty
      \expandafter
      \let\csname HoLogoOpt@variant@\HOLOGO@name\endcsname\@undefined
    \else
      \ltx@IfUndefined{HoLogo@\HOLOGO@name @\HOLOGO@temp}{%
        \@PackageError{hologo}{%
          Unknown variant `\HOLOGO@temp' of logo `\HOLOGO@name'%
        }\@ehc
      }{%
        \expandafter
        \let\csname HoLogoOpt@variant@\HOLOGO@name\endcsname
            \HOLOGO@temp
      }%
    \fi
  \fi
}
%    \end{macrocode}
%
%    \begin{macro}{\HOLOGO@Variant}
%    \begin{macrocode}
\def\HOLOGO@Variant#1{%
  #1%
  \ltx@ifundefined{HoLogoOpt@variant@#1}{%
  }{%
    @\csname HoLogoOpt@variant@#1\endcsname
  }%
}
%    \end{macrocode}
%    \end{macro}
%
% \subsection{Break/no-break support}
%
%    \begin{macro}{\HOLOGO@space}
%    \begin{macrocode}
\def\HOLOGO@space{%
  \ltx@ifundefined{HoLogoOpt@spacebreak@\HOLOGO@name}{%
    \ltx@ifundefined{HoLogoOpt@break@\HOLOGO@name}{%
      \chardef\HOLOGO@temp=\HOLOGOOPT@spacebreak
    }{%
      \chardef\HOLOGO@temp=%
        \csname HoLogoOpt@break@\HOLOGO@name\endcsname
    }%
  }{%
    \chardef\HOLOGO@temp=%
      \csname HoLogoOpt@spacebreak@\HOLOGO@name\endcsname
  }%
  \ifcase\HOLOGO@temp
    \penalty10000 %
  \fi
  \ltx@space
}
%    \end{macrocode}
%    \end{macro}
%
%    \begin{macro}{\HOLOGO@hyphen}
%    \begin{macrocode}
\def\HOLOGO@hyphen{%
  \ltx@ifundefined{HoLogoOpt@hyphenbreak@\HOLOGO@name}{%
    \ltx@ifundefined{HoLogoOpt@break@\HOLOGO@name}{%
      \chardef\HOLOGO@temp=\HOLOGOOPT@hyphenbreak
    }{%
      \chardef\HOLOGO@temp=%
        \csname HoLogoOpt@break@\HOLOGO@name\endcsname
    }%
  }{%
    \chardef\HOLOGO@temp=%
      \csname HoLogoOpt@hyphenbreak@\HOLOGO@name\endcsname
  }%
  \ifcase\HOLOGO@temp
    \ltx@mbox{-}%
  \else
    -%
  \fi
}
%    \end{macrocode}
%    \end{macro}
%
%    \begin{macro}{\HOLOGO@discretionary}
%    \begin{macrocode}
\def\HOLOGO@discretionary{%
  \ltx@ifundefined{HoLogoOpt@discretionarybreak@\HOLOGO@name}{%
    \ltx@ifundefined{HoLogoOpt@break@\HOLOGO@name}{%
      \chardef\HOLOGO@temp=\HOLOGOOPT@discretionarybreak
    }{%
      \chardef\HOLOGO@temp=%
        \csname HoLogoOpt@break@\HOLOGO@name\endcsname
    }%
  }{%
    \chardef\HOLOGO@temp=%
      \csname HoLogoOpt@discretionarybreak@\HOLOGO@name\endcsname
  }%
  \ifcase\HOLOGO@temp
  \else
    \-%
  \fi
}
%    \end{macrocode}
%    \end{macro}
%
%    \begin{macro}{\HOLOGO@mbox}
%    \begin{macrocode}
\def\HOLOGO@mbox#1{%
  \ltx@ifundefined{HoLogoOpt@break@\HOLOGO@name}{%
    \chardef\HOLOGO@temp=\HOLOGOOPT@hyphenbreak
  }{%
    \chardef\HOLOGO@temp=%
      \csname HoLogoOpt@break@\HOLOGO@name\endcsname
  }%
  \ifcase\HOLOGO@temp
    \ltx@mbox{#1}%
  \else
    #1%
  \fi
}
%    \end{macrocode}
%    \end{macro}
%
% \subsection{Font support}
%
%    \begin{macro}{\HoLogoFont@font}
%    \begin{tabular}{@{}ll@{}}
%    |#1|:& logo name\\
%    |#2|:& font short name\\
%    |#3|:& text
%    \end{tabular}
%    \begin{macrocode}
\def\HoLogoFont@font#1#2#3{%
  \begingroup
    \ltx@IfUndefined{HoLogoFont@logo@#1.#2}{%
      \ltx@IfUndefined{HoLogoFont@font@#2}{%
        \@PackageWarning{hologo}{%
          Missing font `#2' for logo `#1'%
        }%
        #3%
      }{%
        \csname HoLogoFont@font@#2\endcsname{#3}%
      }%
    }{%
      \csname HoLogoFont@logo@#1.#2\endcsname{#3}%
    }%
  \endgroup
}
%    \end{macrocode}
%    \end{macro}
%
%    \begin{macro}{\HoLogoFont@Def}
%    \begin{macrocode}
\def\HoLogoFont@Def#1{%
  \expandafter\def\csname HoLogoFont@font@#1\endcsname
}
%    \end{macrocode}
%    \end{macro}
%    \begin{macro}{\HoLogoFont@LogoDef}
%    \begin{macrocode}
\def\HoLogoFont@LogoDef#1#2{%
  \expandafter\def\csname HoLogoFont@logo@#1.#2\endcsname
}
%    \end{macrocode}
%    \end{macro}
%
% \subsubsection{Font defaults}
%
%    \begin{macro}{\HoLogoFont@font@general}
%    \begin{macrocode}
\HoLogoFont@Def{general}{}%
%    \end{macrocode}
%    \end{macro}
%
%    \begin{macro}{\HoLogoFont@font@rm}
%    \begin{macrocode}
\ltx@IfUndefined{rmfamily}{%
  \ltx@IfUndefined{rm}{%
  }{%
    \HoLogoFont@Def{rm}{\rm}%
  }%
}{%
  \HoLogoFont@Def{rm}{\rmfamily}%
}
%    \end{macrocode}
%    \end{macro}
%
%    \begin{macro}{\HoLogoFont@font@sf}
%    \begin{macrocode}
\ltx@IfUndefined{sffamily}{%
  \ltx@IfUndefined{sf}{%
  }{%
    \HoLogoFont@Def{sf}{\sf}%
  }%
}{%
  \HoLogoFont@Def{sf}{\sffamily}%
}
%    \end{macrocode}
%    \end{macro}
%
%    \begin{macro}{\HoLogoFont@font@bibsf}
%    In case of \hologo{plainTeX} the original small caps
%    variant is used as default. In \hologo{LaTeX}
%    the definition of package \xpackage{dtklogos} \cite{dtklogos}
%    is used.
%\begin{quote}
%\begin{verbatim}
%\DeclareRobustCommand{\BibTeX}{%
%  B%
%  \kern-.05em%
%  \hbox{%
%    $\m@th$% %% force math size calculations
%    \csname S@\f@size\endcsname
%    \fontsize\sf@size\z@
%    \math@fontsfalse
%    \selectfont
%    I%
%    \kern-.025em%
%    B
%  }%
%  \kern-.08em%
%  \-%
%  \TeX
%}
%\end{verbatim}
%\end{quote}
%    \begin{macrocode}
\ltx@IfUndefined{selectfont}{%
  \ltx@IfUndefined{tensc}{%
    \font\tensc=cmcsc10\relax
  }{}%
  \HoLogoFont@Def{bibsf}{\tensc}%
}{%
  \HoLogoFont@Def{bibsf}{%
    $\mathsurround=0pt$%
    \csname S@\f@size\endcsname
    \fontsize\sf@size{0pt}%
    \math@fontsfalse
    \selectfont
  }%
}
%    \end{macrocode}
%    \end{macro}
%
%    \begin{macro}{\HoLogoFont@font@sc}
%    \begin{macrocode}
\ltx@IfUndefined{scshape}{%
  \ltx@IfUndefined{tensc}{%
    \font\tensc=cmcsc10\relax
  }{}%
  \HoLogoFont@Def{sc}{\tensc}%
}{%
  \HoLogoFont@Def{sc}{\scshape}%
}
%    \end{macrocode}
%    \end{macro}
%
%    \begin{macro}{\HoLogoFont@font@sy}
%    \begin{macrocode}
\ltx@IfUndefined{usefont}{%
  \ltx@IfUndefined{tensy}{%
  }{%
    \HoLogoFont@Def{sy}{\tensy}%
  }%
}{%
  \HoLogoFont@Def{sy}{%
    \usefont{OMS}{cmsy}{m}{n}%
  }%
}
%    \end{macrocode}
%    \end{macro}
%
%    \begin{macro}{\HoLogoFont@font@logo}
%    \begin{macrocode}
\begingroup
  \def\x{LaTeX2e}%
\expandafter\endgroup
\ifx\fmtname\x
  \ltx@IfUndefined{logofamily}{%
    \DeclareRobustCommand\logofamily{%
      \not@math@alphabet\logofamily\relax
      \fontencoding{U}%
      \fontfamily{logo}%
      \selectfont
    }%
  }{}%
  \ltx@IfUndefined{logofamily}{%
  }{%
    \HoLogoFont@Def{logo}{\logofamily}%
  }%
\else
  \ltx@IfUndefined{tenlogo}{%
    \font\tenlogo=logo10\relax
  }{}%
  \HoLogoFont@Def{logo}{\tenlogo}%
\fi
%    \end{macrocode}
%    \end{macro}
%
% \subsubsection{Font setup}
%
%    \begin{macro}{\hologoFontSetup}
%    \begin{macrocode}
\def\hologoFontSetup{%
  \let\HOLOGO@name\relax
  \HOLOGO@FontSetup
}
%    \end{macrocode}
%    \end{macro}
%
%    \begin{macro}{\hologoLogoFontSetup}
%    \begin{macrocode}
\def\hologoLogoFontSetup#1{%
  \edef\HOLOGO@name{#1}%
  \ltx@IfUndefined{HoLogo@\HOLOGO@name}{%
    \@PackageError{hologo}{%
      Unknown logo `\HOLOGO@name'%
    }\@ehc
    \ltx@gobble
  }{%
    \HOLOGO@FontSetup
  }%
}
%    \end{macrocode}
%    \end{macro}
%
%    \begin{macro}{\HOLOGO@FontSetup}
%    \begin{macrocode}
\def\HOLOGO@FontSetup{%
  \kvsetkeys{HoLogoFont}%
}
%    \end{macrocode}
%    \end{macro}
%
%    \begin{macrocode}
\def\HOLOGO@temp#1{%
  \kv@define@key{HoLogoFont}{#1}{%
    \ifx\HOLOGO@name\relax
      \HoLogoFont@Def{#1}{##1}%
    \else
      \HoLogoFont@LogoDef\HOLOGO@name{#1}{##1}%
    \fi
  }%
}
\HOLOGO@temp{general}
\HOLOGO@temp{sf}
%    \end{macrocode}
%
% \subsection{Generic logo commands}
%
%    \begin{macrocode}
\HOLOGO@IfExists\hologo{%
  \@PackageError{hologo}{%
    \string\hologo\ltx@space is already defined.\MessageBreak
    Package loading is aborted%
  }\@ehc
  \HOLOGO@AtEnd
}%
\HOLOGO@IfExists\hologoRobust{%
  \@PackageError{hologo}{%
    \string\hologoRobust\ltx@space is already defined.\MessageBreak
    Package loading is aborted%
  }\@ehc
  \HOLOGO@AtEnd
}%
%    \end{macrocode}
%
% \subsubsection{\cs{hologo} and friends}
%
%    \begin{macrocode}
\ifluatex
  \expandafter\ltx@firstofone
\else
  \expandafter\ltx@gobble
\fi
{%
  \ltx@IfUndefined{ifincsname}{%
    \ifnum\luatexversion<36 %
      \expandafter\ltx@gobble
    \else
      \expandafter\ltx@firstofone
    \fi
    {%
      \begingroup
        \ifcase0%
            \directlua{%
              if tex.enableprimitives then %
                tex.enableprimitives('HOLOGO@', {'ifincsname'})%
              else %
                tex.print('1')%
              end%
            }%
            \ifx\HOLOGO@ifincsname\@undefined 1\fi%
            \relax
          \expandafter\ltx@firstofone
        \else
          \endgroup
          \expandafter\ltx@gobble
        \fi
        {%
          \global\let\ifincsname\HOLOGO@ifincsname
        }%
      \HOLOGO@temp
    }%
  }{}%
}
%    \end{macrocode}
%    \begin{macrocode}
\ltx@IfUndefined{ifincsname}{%
  \catcode`$=14 %
}{%
  \catcode`$=9 %
}
%    \end{macrocode}
%
%    \begin{macro}{\hologo}
%    \begin{macrocode}
\def\hologo#1{%
$ \ifincsname
$   \ltx@ifundefined{HoLogoCs@\HOLOGO@Variant{#1}}{%
$     #1%
$   }{%
$     \csname HoLogoCs@\HOLOGO@Variant{#1}\endcsname\ltx@firstoftwo
$   }%
$ \else
    \HOLOGO@IfExists\texorpdfstring\texorpdfstring\ltx@firstoftwo
    {%
      \hologoRobust{#1}%
    }{%
      \ltx@ifundefined{HoLogoBkm@\HOLOGO@Variant{#1}}{%
        \ltx@ifundefined{HoLogo@#1}{?#1?}{#1}%
      }{%
        \csname HoLogoBkm@\HOLOGO@Variant{#1}\endcsname
        \ltx@firstoftwo
      }%
    }%
$ \fi
}
%    \end{macrocode}
%    \end{macro}
%    \begin{macro}{\Hologo}
%    \begin{macrocode}
\def\Hologo#1{%
$ \ifincsname
$   \ltx@ifundefined{HoLogoCs@\HOLOGO@Variant{#1}}{%
$     #1%
$   }{%
$     \csname HoLogoCs@\HOLOGO@Variant{#1}\endcsname\ltx@secondoftwo
$   }%
$ \else
    \HOLOGO@IfExists\texorpdfstring\texorpdfstring\ltx@firstoftwo
    {%
      \HologoRobust{#1}%
    }{%
      \ltx@ifundefined{HoLogoBkm@\HOLOGO@Variant{#1}}{%
        \ltx@ifundefined{HoLogo@#1}{?#1?}{#1}%
      }{%
        \csname HoLogoBkm@\HOLOGO@Variant{#1}\endcsname
        \ltx@secondoftwo
      }%
    }%
$ \fi
}
%    \end{macrocode}
%    \end{macro}
%
%    \begin{macro}{\hologoVariant}
%    \begin{macrocode}
\def\hologoVariant#1#2{%
  \ifx\relax#2\relax
    \hologo{#1}%
  \else
$   \ifincsname
$     \ltx@ifundefined{HoLogoCs@#1@#2}{%
$       #1%
$     }{%
$       \csname HoLogoCs@#1@#2\endcsname\ltx@firstoftwo
$     }%
$   \else
      \HOLOGO@IfExists\texorpdfstring\texorpdfstring\ltx@firstoftwo
      {%
        \hologoVariantRobust{#1}{#2}%
      }{%
        \ltx@ifundefined{HoLogoBkm@#1@#2}{%
          \ltx@ifundefined{HoLogo@#1}{?#1?}{#1}%
        }{%
          \csname HoLogoBkm@#1@#2\endcsname
          \ltx@firstoftwo
        }%
      }%
$   \fi
  \fi
}
%    \end{macrocode}
%    \end{macro}
%    \begin{macro}{\HologoVariant}
%    \begin{macrocode}
\def\HologoVariant#1#2{%
  \ifx\relax#2\relax
    \Hologo{#1}%
  \else
$   \ifincsname
$     \ltx@ifundefined{HoLogoCs@#1@#2}{%
$       #1%
$     }{%
$       \csname HoLogoCs@#1@#2\endcsname\ltx@secondoftwo
$     }%
$   \else
      \HOLOGO@IfExists\texorpdfstring\texorpdfstring\ltx@firstoftwo
      {%
        \HologoVariantRobust{#1}{#2}%
      }{%
        \ltx@ifundefined{HoLogoBkm@#1@#2}{%
          \ltx@ifundefined{HoLogo@#1}{?#1?}{#1}%
        }{%
          \csname HoLogoBkm@#1@#2\endcsname
          \ltx@secondoftwo
        }%
      }%
$   \fi
  \fi
}
%    \end{macrocode}
%    \end{macro}
%
%    \begin{macrocode}
\catcode`\$=3 %
%    \end{macrocode}
%
% \subsubsection{\cs{hologoRobust} and friends}
%
%    \begin{macro}{\hologoRobust}
%    \begin{macrocode}
\ltx@IfUndefined{protected}{%
  \ltx@IfUndefined{DeclareRobustCommand}{%
    \def\hologoRobust#1%
  }{%
    \DeclareRobustCommand*\hologoRobust[1]%
  }%
}{%
  \protected\def\hologoRobust#1%
}%
{%
  \edef\HOLOGO@name{#1}%
  \ltx@IfUndefined{HoLogo@\HOLOGO@Variant\HOLOGO@name}{%
    \@PackageError{hologo}{%
      Unknown logo `\HOLOGO@name'%
    }\@ehc
    ?\HOLOGO@name?%
  }{%
    \ltx@IfUndefined{ver@tex4ht.sty}{%
      \HoLogoFont@font\HOLOGO@name{general}{%
        \csname HoLogo@\HOLOGO@Variant\HOLOGO@name\endcsname
        \ltx@firstoftwo
      }%
    }{%
      \ltx@IfUndefined{HoLogoHtml@\HOLOGO@Variant\HOLOGO@name}{%
        \HOLOGO@name
      }{%
        \csname HoLogoHtml@\HOLOGO@Variant\HOLOGO@name\endcsname
        \ltx@firstoftwo
      }%
    }%
  }%
}
%    \end{macrocode}
%    \end{macro}
%    \begin{macro}{\HologoRobust}
%    \begin{macrocode}
\ltx@IfUndefined{protected}{%
  \ltx@IfUndefined{DeclareRobustCommand}{%
    \def\HologoRobust#1%
  }{%
    \DeclareRobustCommand*\HologoRobust[1]%
  }%
}{%
  \protected\def\HologoRobust#1%
}%
{%
  \edef\HOLOGO@name{#1}%
  \ltx@IfUndefined{HoLogo@\HOLOGO@Variant\HOLOGO@name}{%
    \@PackageError{hologo}{%
      Unknown logo `\HOLOGO@name'%
    }\@ehc
    ?\HOLOGO@name?%
  }{%
    \ltx@IfUndefined{ver@tex4ht.sty}{%
      \HoLogoFont@font\HOLOGO@name{general}{%
        \csname HoLogo@\HOLOGO@Variant\HOLOGO@name\endcsname
        \ltx@secondoftwo
      }%
    }{%
      \ltx@IfUndefined{HoLogoHtml@\HOLOGO@Variant\HOLOGO@name}{%
        \expandafter\HOLOGO@Uppercase\HOLOGO@name
      }{%
        \csname HoLogoHtml@\HOLOGO@Variant\HOLOGO@name\endcsname
        \ltx@secondoftwo
      }%
    }%
  }%
}
%    \end{macrocode}
%    \end{macro}
%    \begin{macro}{\hologoVariantRobust}
%    \begin{macrocode}
\ltx@IfUndefined{protected}{%
  \ltx@IfUndefined{DeclareRobustCommand}{%
    \def\hologoVariantRobust#1#2%
  }{%
    \DeclareRobustCommand*\hologoVariantRobust[2]%
  }%
}{%
  \protected\def\hologoVariantRobust#1#2%
}%
{%
  \begingroup
    \hologoLogoSetup{#1}{variant={#2}}%
    \hologoRobust{#1}%
  \endgroup
}
%    \end{macrocode}
%    \end{macro}
%    \begin{macro}{\HologoVariantRobust}
%    \begin{macrocode}
\ltx@IfUndefined{protected}{%
  \ltx@IfUndefined{DeclareRobustCommand}{%
    \def\HologoVariantRobust#1#2%
  }{%
    \DeclareRobustCommand*\HologoVariantRobust[2]%
  }%
}{%
  \protected\def\HologoVariantRobust#1#2%
}%
{%
  \begingroup
    \hologoLogoSetup{#1}{variant={#2}}%
    \HologoRobust{#1}%
  \endgroup
}
%    \end{macrocode}
%    \end{macro}
%
%    \begin{macro}{\hologorobust}
%    Macro \cs{hologorobust} is only defined for compatibility.
%    Its use is deprecated.
%    \begin{macrocode}
\def\hologorobust{\hologoRobust}
%    \end{macrocode}
%    \end{macro}
%
% \subsection{Helpers}
%
%    \begin{macro}{\HOLOGO@Uppercase}
%    Macro \cs{HOLOGO@Uppercase} is restricted to \cs{uppercase},
%    because \hologo{plainTeX} or \hologo{iniTeX} do not provide
%    \cs{MakeUppercase}.
%    \begin{macrocode}
\def\HOLOGO@Uppercase#1{\uppercase{#1}}
%    \end{macrocode}
%    \end{macro}
%
%    \begin{macro}{\HOLOGO@PdfdocUnicode}
%    \begin{macrocode}
\def\HOLOGO@PdfdocUnicode{%
  \ifx\ifHy@unicode\iftrue
    \expandafter\ltx@secondoftwo
  \else
    \expandafter\ltx@firstoftwo
  \fi
}
%    \end{macrocode}
%    \end{macro}
%
%    \begin{macro}{\HOLOGO@Math}
%    \begin{macrocode}
\def\HOLOGO@MathSetup{%
  \mathsurround0pt\relax
  \HOLOGO@IfExists\f@series{%
    \if b\expandafter\ltx@car\f@series x\@nil
      \csname boldmath\endcsname
   \fi
  }{}%
}
%    \end{macrocode}
%    \end{macro}
%
%    \begin{macro}{\HOLOGO@TempDimen}
%    \begin{macrocode}
\dimendef\HOLOGO@TempDimen=\ltx@zero
%    \end{macrocode}
%    \end{macro}
%    \begin{macro}{\HOLOGO@NegativeKerning}
%    \begin{macrocode}
\def\HOLOGO@NegativeKerning#1{%
  \begingroup
    \HOLOGO@TempDimen=0pt\relax
    \comma@parse@normalized{#1}{%
      \ifdim\HOLOGO@TempDimen=0pt %
        \expandafter\HOLOGO@@NegativeKerning\comma@entry
      \fi
      \ltx@gobble
    }%
    \ifdim\HOLOGO@TempDimen<0pt %
      \kern\HOLOGO@TempDimen
    \fi
  \endgroup
}
%    \end{macrocode}
%    \end{macro}
%    \begin{macro}{\HOLOGO@@NegativeKerning}
%    \begin{macrocode}
\def\HOLOGO@@NegativeKerning#1#2{%
  \setbox\ltx@zero\hbox{#1#2}%
  \HOLOGO@TempDimen=\wd\ltx@zero
  \setbox\ltx@zero\hbox{#1\kern0pt#2}%
  \advance\HOLOGO@TempDimen by -\wd\ltx@zero
}
%    \end{macrocode}
%    \end{macro}
%
%    \begin{macro}{\HOLOGO@SpaceFactor}
%    \begin{macrocode}
\def\HOLOGO@SpaceFactor{%
  \spacefactor1000 %
}
%    \end{macrocode}
%    \end{macro}
%
%    \begin{macro}{\HOLOGO@Span}
%    \begin{macrocode}
\def\HOLOGO@Span#1#2{%
  \HCode{<span class="HoLogo-#1">}%
  #2%
  \HCode{</span>}%
}
%    \end{macrocode}
%    \end{macro}
%
% \subsubsection{Text subscript}
%
%    \begin{macro}{\HOLOGO@SubScript}%
%    \begin{macrocode}
\def\HOLOGO@SubScript#1{%
  \ltx@IfUndefined{textsubscript}{%
    \ltx@IfUndefined{text}{%
      \ltx@mbox{%
        \mathsurround=0pt\relax
        $%
          _{%
            \ltx@IfUndefined{sf@size}{%
              \mathrm{#1}%
            }{%
              \mbox{%
                \fontsize\sf@size{0pt}\selectfont
                #1%
              }%
            }%
          }%
        $%
      }%
    }{%
      \ltx@mbox{%
        \mathsurround=0pt\relax
        $_{\text{#1}}$%
      }%
    }%
  }{%
    \textsubscript{#1}%
  }%
}
%    \end{macrocode}
%    \end{macro}
%
% \subsection{\hologo{TeX} and friends}
%
% \subsubsection{\hologo{TeX}}
%
%    \begin{macro}{\HoLogo@TeX}
%    Source: \hologo{LaTeX} kernel.
%    \begin{macrocode}
\def\HoLogo@TeX#1{%
  T\kern-.1667em\lower.5ex\hbox{E}\kern-.125emX\HOLOGO@SpaceFactor
}
%    \end{macrocode}
%    \end{macro}
%    \begin{macro}{\HoLogoHtml@TeX}
%    \begin{macrocode}
\def\HoLogoHtml@TeX#1{%
  \HoLogoCss@TeX
  \HOLOGO@Span{TeX}{%
    T%
    \HOLOGO@Span{e}{%
      E%
    }%
    X%
  }%
}
%    \end{macrocode}
%    \end{macro}
%    \begin{macro}{\HoLogoCss@TeX}
%    \begin{macrocode}
\def\HoLogoCss@TeX{%
  \Css{%
    span.HoLogo-TeX span.HoLogo-e{%
      position:relative;%
      top:.5ex;%
      margin-left:-.1667em;%
      margin-right:-.125em;%
    }%
  }%
  \Css{%
    a span.HoLogo-TeX span.HoLogo-e{%
      text-decoration:none;%
    }%
  }%
  \global\let\HoLogoCss@TeX\relax
}
%    \end{macrocode}
%    \end{macro}
%
% \subsubsection{\hologo{plainTeX}}
%
%    \begin{macro}{\HoLogo@plainTeX@space}
%    Source: ``The \hologo{TeX}book''
%    \begin{macrocode}
\def\HoLogo@plainTeX@space#1{%
  \HOLOGO@mbox{#1{p}{P}lain}\HOLOGO@space\hologo{TeX}%
}
%    \end{macrocode}
%    \end{macro}
%    \begin{macro}{\HoLogoCs@plainTeX@space}
%    \begin{macrocode}
\def\HoLogoCs@plainTeX@space#1{#1{p}{P}lain TeX}%
%    \end{macrocode}
%    \end{macro}
%    \begin{macro}{\HoLogoBkm@plainTeX@space}
%    \begin{macrocode}
\def\HoLogoBkm@plainTeX@space#1{%
  #1{p}{P}lain \hologo{TeX}%
}
%    \end{macrocode}
%    \end{macro}
%    \begin{macro}{\HoLogoHtml@plainTeX@space}
%    \begin{macrocode}
\def\HoLogoHtml@plainTeX@space#1{%
  #1{p}{P}lain \hologo{TeX}%
}
%    \end{macrocode}
%    \end{macro}
%
%    \begin{macro}{\HoLogo@plainTeX@hyphen}
%    \begin{macrocode}
\def\HoLogo@plainTeX@hyphen#1{%
  \HOLOGO@mbox{#1{p}{P}lain}\HOLOGO@hyphen\hologo{TeX}%
}
%    \end{macrocode}
%    \end{macro}
%    \begin{macro}{\HoLogoCs@plainTeX@hyphen}
%    \begin{macrocode}
\def\HoLogoCs@plainTeX@hyphen#1{#1{p}{P}lain-TeX}
%    \end{macrocode}
%    \end{macro}
%    \begin{macro}{\HoLogoBkm@plainTeX@hyphen}
%    \begin{macrocode}
\def\HoLogoBkm@plainTeX@hyphen#1{%
  #1{p}{P}lain-\hologo{TeX}%
}
%    \end{macrocode}
%    \end{macro}
%    \begin{macro}{\HoLogoHtml@plainTeX@hyphen}
%    \begin{macrocode}
\def\HoLogoHtml@plainTeX@hyphen#1{%
  #1{p}{P}lain-\hologo{TeX}%
}
%    \end{macrocode}
%    \end{macro}
%
%    \begin{macro}{\HoLogo@plainTeX@runtogether}
%    \begin{macrocode}
\def\HoLogo@plainTeX@runtogether#1{%
  \HOLOGO@mbox{#1{p}{P}lain\hologo{TeX}}%
}
%    \end{macrocode}
%    \end{macro}
%    \begin{macro}{\HoLogoCs@plainTeX@runtogether}
%    \begin{macrocode}
\def\HoLogoCs@plainTeX@runtogether#1{#1{p}{P}lainTeX}
%    \end{macrocode}
%    \end{macro}
%    \begin{macro}{\HoLogoBkm@plainTeX@runtogether}
%    \begin{macrocode}
\def\HoLogoBkm@plainTeX@runtogether#1{%
  #1{p}{P}lain\hologo{TeX}%
}
%    \end{macrocode}
%    \end{macro}
%    \begin{macro}{\HoLogoHtml@plainTeX@runtogether}
%    \begin{macrocode}
\def\HoLogoHtml@plainTeX@runtogether#1{%
  #1{p}{P}lain\hologo{TeX}%
}
%    \end{macrocode}
%    \end{macro}
%
%    \begin{macro}{\HoLogo@plainTeX}
%    \begin{macrocode}
\def\HoLogo@plainTeX{\HoLogo@plainTeX@space}
%    \end{macrocode}
%    \end{macro}
%    \begin{macro}{\HoLogoCs@plainTeX}
%    \begin{macrocode}
\def\HoLogoCs@plainTeX{\HoLogoCs@plainTeX@space}
%    \end{macrocode}
%    \end{macro}
%    \begin{macro}{\HoLogoBkm@plainTeX}
%    \begin{macrocode}
\def\HoLogoBkm@plainTeX{\HoLogoBkm@plainTeX@space}
%    \end{macrocode}
%    \end{macro}
%    \begin{macro}{\HoLogoHtml@plainTeX}
%    \begin{macrocode}
\def\HoLogoHtml@plainTeX{\HoLogoHtml@plainTeX@space}
%    \end{macrocode}
%    \end{macro}
%
% \subsubsection{\hologo{LaTeX}}
%
%    Source: \hologo{LaTeX} kernel.
%\begin{quote}
%\begin{verbatim}
%\DeclareRobustCommand{\LaTeX}{%
%  L%
%  \kern-.36em%
%  {%
%    \sbox\z@ T%
%    \vbox to\ht\z@{%
%      \hbox{%
%        \check@mathfonts
%        \fontsize\sf@size\z@
%        \math@fontsfalse
%        \selectfont
%        A%
%      }%
%      \vss
%    }%
%  }%
%  \kern-.15em%
%  \TeX
%}
%\end{verbatim}
%\end{quote}
%
%    \begin{macro}{\HoLogo@La}
%    \begin{macrocode}
\def\HoLogo@La#1{%
  L%
  \kern-.36em%
  \begingroup
    \setbox\ltx@zero\hbox{T}%
    \vbox to\ht\ltx@zero{%
      \hbox{%
        \ltx@ifundefined{check@mathfonts}{%
          \csname sevenrm\endcsname
        }{%
          \check@mathfonts
          \fontsize\sf@size{0pt}%
          \math@fontsfalse\selectfont
        }%
        A%
      }%
      \vss
    }%
  \endgroup
}
%    \end{macrocode}
%    \end{macro}
%
%    \begin{macro}{\HoLogo@LaTeX}
%    Source: \hologo{LaTeX} kernel.
%    \begin{macrocode}
\def\HoLogo@LaTeX#1{%
  \hologo{La}%
  \kern-.15em%
  \hologo{TeX}%
}
%    \end{macrocode}
%    \end{macro}
%    \begin{macro}{\HoLogoHtml@LaTeX}
%    \begin{macrocode}
\def\HoLogoHtml@LaTeX#1{%
  \HoLogoCss@LaTeX
  \HOLOGO@Span{LaTeX}{%
    L%
    \HOLOGO@Span{a}{%
      A%
    }%
    \hologo{TeX}%
  }%
}
%    \end{macrocode}
%    \end{macro}
%    \begin{macro}{\HoLogoCss@LaTeX}
%    \begin{macrocode}
\def\HoLogoCss@LaTeX{%
  \Css{%
    span.HoLogo-LaTeX span.HoLogo-a{%
      position:relative;%
      top:-.5ex;%
      margin-left:-.36em;%
      margin-right:-.15em;%
      font-size:85\%;%
    }%
  }%
  \global\let\HoLogoCss@LaTeX\relax
}
%    \end{macrocode}
%    \end{macro}
%
% \subsubsection{\hologo{(La)TeX}}
%
%    \begin{macro}{\HoLogo@LaTeXTeX}
%    The kerning around the parentheses is taken
%    from package \xpackage{dtklogos} \cite{dtklogos}.
%\begin{quote}
%\begin{verbatim}
%\DeclareRobustCommand{\LaTeXTeX}{%
%  (%
%  \kern-.15em%
%  L%
%  \kern-.36em%
%  {%
%    \sbox\z@ T%
%    \vbox to\ht0{%
%      \hbox{%
%        $\m@th$%
%        \csname S@\f@size\endcsname
%        \fontsize\sf@size\z@
%        \math@fontsfalse
%        \selectfont
%        A%
%      }%
%      \vss
%    }%
%  }%
%  \kern-.2em%
%  )%
%  \kern-.15em%
%  \TeX
%}
%\end{verbatim}
%\end{quote}
%    \begin{macrocode}
\def\HoLogo@LaTeXTeX#1{%
  (%
  \kern-.15em%
  \hologo{La}%
  \kern-.2em%
  )%
  \kern-.15em%
  \hologo{TeX}%
}
%    \end{macrocode}
%    \end{macro}
%    \begin{macro}{\HoLogoBkm@LaTeXTeX}
%    \begin{macrocode}
\def\HoLogoBkm@LaTeXTeX#1{(La)TeX}
%    \end{macrocode}
%    \end{macro}
%
%    \begin{macro}{\HoLogo@(La)TeX}
%    \begin{macrocode}
\expandafter
\let\csname HoLogo@(La)TeX\endcsname\HoLogo@LaTeXTeX
%    \end{macrocode}
%    \end{macro}
%    \begin{macro}{\HoLogoBkm@(La)TeX}
%    \begin{macrocode}
\expandafter
\let\csname HoLogoBkm@(La)TeX\endcsname\HoLogoBkm@LaTeXTeX
%    \end{macrocode}
%    \end{macro}
%    \begin{macro}{\HoLogoHtml@LaTeXTeX}
%    \begin{macrocode}
\def\HoLogoHtml@LaTeXTeX#1{%
  \HoLogoCss@LaTeXTeX
  \HOLOGO@Span{LaTeXTeX}{%
    (%
    \HOLOGO@Span{L}{L}%
    \HOLOGO@Span{a}{A}%
    \HOLOGO@Span{ParenRight}{)}%
    \hologo{TeX}%
  }%
}
%    \end{macrocode}
%    \end{macro}
%    \begin{macro}{\HoLogoHtml@(La)TeX}
%    Kerning after opening parentheses and before closing parentheses
%    is $-0.1$\,em. The original values $-0.15$\,em
%    looked too ugly for a serif font.
%    \begin{macrocode}
\expandafter
\let\csname HoLogoHtml@(La)TeX\endcsname\HoLogoHtml@LaTeXTeX
%    \end{macrocode}
%    \end{macro}
%    \begin{macro}{\HoLogoCss@LaTeXTeX}
%    \begin{macrocode}
\def\HoLogoCss@LaTeXTeX{%
  \Css{%
    span.HoLogo-LaTeXTeX span.HoLogo-L{%
      margin-left:-.1em;%
    }%
  }%
  \Css{%
    span.HoLogo-LaTeXTeX span.HoLogo-a{%
      position:relative;%
      top:-.5ex;%
      margin-left:-.36em;%
      margin-right:-.1em;%
      font-size:85\%;%
    }%
  }%
  \Css{%
    span.HoLogo-LaTeXTeX span.HoLogo-ParenRight{%
      margin-right:-.15em;%
    }%
  }%
  \global\let\HoLogoCss@LaTeXTeX\relax
}
%    \end{macrocode}
%    \end{macro}
%
% \subsubsection{\hologo{LaTeXe}}
%
%    \begin{macro}{\HoLogo@LaTeXe}
%    Source: \hologo{LaTeX} kernel
%    \begin{macrocode}
\def\HoLogo@LaTeXe#1{%
  \hologo{LaTeX}%
  \kern.15em%
  \hbox{%
    \HOLOGO@MathSetup
    2%
    $_{\textstyle\varepsilon}$%
  }%
}
%    \end{macrocode}
%    \end{macro}
%
%    \begin{macro}{\HoLogoCs@LaTeXe}
%    \begin{macrocode}
\ifnum64=`\^^^^0040\relax % test for big chars of LuaTeX/XeTeX
  \catcode`\$=9 %
  \catcode`\&=14 %
\else
  \catcode`\$=14 %
  \catcode`\&=9 %
\fi
\def\HoLogoCs@LaTeXe#1{%
  LaTeX2%
$ \string ^^^^0395%
& e%
}%
\catcode`\$=3 %
\catcode`\&=4 %
%    \end{macrocode}
%    \end{macro}
%
%    \begin{macro}{\HoLogoBkm@LaTeXe}
%    \begin{macrocode}
\def\HoLogoBkm@LaTeXe#1{%
  \hologo{LaTeX}%
  2%
  \HOLOGO@PdfdocUnicode{e}{\textepsilon}%
}
%    \end{macrocode}
%    \end{macro}
%
%    \begin{macro}{\HoLogoHtml@LaTeXe}
%    \begin{macrocode}
\def\HoLogoHtml@LaTeXe#1{%
  \HoLogoCss@LaTeXe
  \HOLOGO@Span{LaTeX2e}{%
    \hologo{LaTeX}%
    \HOLOGO@Span{2}{2}%
    \HOLOGO@Span{e}{%
      \HOLOGO@MathSetup
      \ensuremath{\textstyle\varepsilon}%
    }%
  }%
}
%    \end{macrocode}
%    \end{macro}
%    \begin{macro}{\HoLogoCss@LaTeXe}
%    \begin{macrocode}
\def\HoLogoCss@LaTeXe{%
  \Css{%
    span.HoLogo-LaTeX2e span.HoLogo-2{%
      padding-left:.15em;%
    }%
  }%
  \Css{%
    span.HoLogo-LaTeX2e span.HoLogo-e{%
      position:relative;%
      top:.35ex;%
      text-decoration:none;%
    }%
  }%
  \global\let\HoLogoCss@LaTeXe\relax
}
%    \end{macrocode}
%    \end{macro}
%
%    \begin{macro}{\HoLogo@LaTeX2e}
%    \begin{macrocode}
\expandafter
\let\csname HoLogo@LaTeX2e\endcsname\HoLogo@LaTeXe
%    \end{macrocode}
%    \end{macro}
%    \begin{macro}{\HoLogoCs@LaTeX2e}
%    \begin{macrocode}
\expandafter
\let\csname HoLogoCs@LaTeX2e\endcsname\HoLogoCs@LaTeXe
%    \end{macrocode}
%    \end{macro}
%    \begin{macro}{\HoLogoBkm@LaTeX2e}
%    \begin{macrocode}
\expandafter
\let\csname HoLogoBkm@LaTeX2e\endcsname\HoLogoBkm@LaTeXe
%    \end{macrocode}
%    \end{macro}
%    \begin{macro}{\HoLogoHtml@LaTeX2e}
%    \begin{macrocode}
\expandafter
\let\csname HoLogoHtml@LaTeX2e\endcsname\HoLogoHtml@LaTeXe
%    \end{macrocode}
%    \end{macro}
%
% \subsubsection{\hologo{LaTeX3}}
%
%    \begin{macro}{\HoLogo@LaTeX3}
%    Source: \hologo{LaTeX} kernel
%    \begin{macrocode}
\expandafter\def\csname HoLogo@LaTeX3\endcsname#1{%
  \hologo{LaTeX}%
  3%
}
%    \end{macrocode}
%    \end{macro}
%
%    \begin{macro}{\HoLogoBkm@LaTeX3}
%    \begin{macrocode}
\expandafter\def\csname HoLogoBkm@LaTeX3\endcsname#1{%
  \hologo{LaTeX}%
  3%
}
%    \end{macrocode}
%    \end{macro}
%    \begin{macro}{\HoLogoHtml@LaTeX3}
%    \begin{macrocode}
\expandafter
\let\csname HoLogoHtml@LaTeX3\expandafter\endcsname
\csname HoLogo@LaTeX3\endcsname
%    \end{macrocode}
%    \end{macro}
%
% \subsubsection{\hologo{LaTeXML}}
%
%    \begin{macro}{\HoLogo@LaTeXML}
%    \begin{macrocode}
\def\HoLogo@LaTeXML#1{%
  \HOLOGO@mbox{%
    \hologo{La}%
    \kern-.15em%
    T%
    \kern-.1667em%
    \lower.5ex\hbox{E}%
    \kern-.125em%
    \HoLogoFont@font{LaTeXML}{sc}{xml}%
  }%
}
%    \end{macrocode}
%    \end{macro}
%    \begin{macro}{\HoLogoHtml@pdfLaTeX}
%    \begin{macrocode}
\def\HoLogoHtml@LaTeXML#1{%
  \HOLOGO@Span{LaTeXML}{%
    \HoLogoCss@LaTeX
    \HoLogoCss@TeX
    \HOLOGO@Span{LaTeX}{%
      L%
      \HOLOGO@Span{a}{%
        A%
      }%
    }%
    \HOLOGO@Span{TeX}{%
      T%
      \HOLOGO@Span{e}{%
        E%
      }%
    }%
    \HCode{<span style="font-variant: small-caps;">}%
    xml%
    \HCode{</span>}%
  }%
}
%    \end{macrocode}
%    \end{macro}
%
% \subsubsection{\hologo{eTeX}}
%
%    \begin{macro}{\HoLogo@eTeX}
%    Source: package \xpackage{etex}
%    \begin{macrocode}
\def\HoLogo@eTeX#1{%
  \ltx@mbox{%
    \HOLOGO@MathSetup
    $\varepsilon$%
    -%
    \HOLOGO@NegativeKerning{-T,T-,To}%
    \hologo{TeX}%
  }%
}
%    \end{macrocode}
%    \end{macro}
%    \begin{macro}{\HoLogoCs@eTeX}
%    \begin{macrocode}
\ifnum64=`\^^^^0040\relax % test for big chars of LuaTeX/XeTeX
  \catcode`\$=9 %
  \catcode`\&=14 %
\else
  \catcode`\$=14 %
  \catcode`\&=9 %
\fi
\def\HoLogoCs@eTeX#1{%
$ #1{\string ^^^^0395}{\string ^^^^03b5}%
& #1{e}{E}%
  TeX%
}%
\catcode`\$=3 %
\catcode`\&=4 %
%    \end{macrocode}
%    \end{macro}
%    \begin{macro}{\HoLogoBkm@eTeX}
%    \begin{macrocode}
\def\HoLogoBkm@eTeX#1{%
  \HOLOGO@PdfdocUnicode{#1{e}{E}}{\textepsilon}%
  -%
  \hologo{TeX}%
}
%    \end{macrocode}
%    \end{macro}
%    \begin{macro}{\HoLogoHtml@eTeX}
%    \begin{macrocode}
\def\HoLogoHtml@eTeX#1{%
  \ltx@mbox{%
    \HOLOGO@MathSetup
    $\varepsilon$%
    -%
    \hologo{TeX}%
  }%
}
%    \end{macrocode}
%    \end{macro}
%
% \subsubsection{\hologo{iniTeX}}
%
%    \begin{macro}{\HoLogo@iniTeX}
%    \begin{macrocode}
\def\HoLogo@iniTeX#1{%
  \HOLOGO@mbox{%
    #1{i}{I}ni\hologo{TeX}%
  }%
}
%    \end{macrocode}
%    \end{macro}
%    \begin{macro}{\HoLogoCs@iniTeX}
%    \begin{macrocode}
\def\HoLogoCs@iniTeX#1{#1{i}{I}niTeX}
%    \end{macrocode}
%    \end{macro}
%    \begin{macro}{\HoLogoBkm@iniTeX}
%    \begin{macrocode}
\def\HoLogoBkm@iniTeX#1{%
  #1{i}{I}ni\hologo{TeX}%
}
%    \end{macrocode}
%    \end{macro}
%    \begin{macro}{\HoLogoHtml@iniTeX}
%    \begin{macrocode}
\let\HoLogoHtml@iniTeX\HoLogo@iniTeX
%    \end{macrocode}
%    \end{macro}
%
% \subsubsection{\hologo{virTeX}}
%
%    \begin{macro}{\HoLogo@virTeX}
%    \begin{macrocode}
\def\HoLogo@virTeX#1{%
  \HOLOGO@mbox{%
    #1{v}{V}ir\hologo{TeX}%
  }%
}
%    \end{macrocode}
%    \end{macro}
%    \begin{macro}{\HoLogoCs@virTeX}
%    \begin{macrocode}
\def\HoLogoCs@virTeX#1{#1{v}{V}irTeX}
%    \end{macrocode}
%    \end{macro}
%    \begin{macro}{\HoLogoBkm@virTeX}
%    \begin{macrocode}
\def\HoLogoBkm@virTeX#1{%
  #1{v}{V}ir\hologo{TeX}%
}
%    \end{macrocode}
%    \end{macro}
%    \begin{macro}{\HoLogoHtml@virTeX}
%    \begin{macrocode}
\let\HoLogoHtml@virTeX\HoLogo@virTeX
%    \end{macrocode}
%    \end{macro}
%
% \subsubsection{\hologo{SliTeX}}
%
% \paragraph{Definitions of the three variants.}
%
%    \begin{macro}{\HoLogo@SLiTeX@lift}
%    \begin{macrocode}
\def\HoLogo@SLiTeX@lift#1{%
  \HoLogoFont@font{SliTeX}{rm}{%
    S%
    \kern-.06em%
    L%
    \kern-.18em%
    \raise.32ex\hbox{\HoLogoFont@font{SliTeX}{sc}{i}}%
    \HOLOGO@discretionary
    \kern-.06em%
    \hologo{TeX}%
  }%
}
%    \end{macrocode}
%    \end{macro}
%    \begin{macro}{\HoLogoBkm@SLiTeX@lift}
%    \begin{macrocode}
\def\HoLogoBkm@SLiTeX@lift#1{SLiTeX}
%    \end{macrocode}
%    \end{macro}
%    \begin{macro}{\HoLogoHtml@SLiTeX@lift}
%    \begin{macrocode}
\def\HoLogoHtml@SLiTeX@lift#1{%
  \HoLogoCss@SLiTeX@lift
  \HOLOGO@Span{SLiTeX-lift}{%
    \HoLogoFont@font{SliTeX}{rm}{%
      S%
      \HOLOGO@Span{L}{L}%
      \HOLOGO@Span{i}{i}%
      \hologo{TeX}%
    }%
  }%
}
%    \end{macrocode}
%    \end{macro}
%    \begin{macro}{\HoLogoCss@SLiTeX@lift}
%    \begin{macrocode}
\def\HoLogoCss@SLiTeX@lift{%
  \Css{%
    span.HoLogo-SLiTeX-lift span.HoLogo-L{%
      margin-left:-.06em;%
      margin-right:-.18em;%
    }%
  }%
  \Css{%
    span.HoLogo-SLiTeX-lift span.HoLogo-i{%
      position:relative;%
      top:-.32ex;%
      margin-right:-.06em;%
      font-variant:small-caps;%
    }%
  }%
  \global\let\HoLogoCss@SLiTeX@lift\relax
}
%    \end{macrocode}
%    \end{macro}
%
%    \begin{macro}{\HoLogo@SliTeX@simple}
%    \begin{macrocode}
\def\HoLogo@SliTeX@simple#1{%
  \HoLogoFont@font{SliTeX}{rm}{%
    \ltx@mbox{%
      \HoLogoFont@font{SliTeX}{sc}{Sli}%
    }%
    \HOLOGO@discretionary
    \hologo{TeX}%
  }%
}
%    \end{macrocode}
%    \end{macro}
%    \begin{macro}{\HoLogoBkm@SliTeX@simple}
%    \begin{macrocode}
\def\HoLogoBkm@SliTeX@simple#1{SliTeX}
%    \end{macrocode}
%    \end{macro}
%    \begin{macro}{\HoLogoHtml@SliTeX@simple}
%    \begin{macrocode}
\let\HoLogoHtml@SliTeX@simple\HoLogo@SliTeX@simple
%    \end{macrocode}
%    \end{macro}
%
%    \begin{macro}{\HoLogo@SliTeX@narrow}
%    \begin{macrocode}
\def\HoLogo@SliTeX@narrow#1{%
  \HoLogoFont@font{SliTeX}{rm}{%
    \ltx@mbox{%
      S%
      \kern-.06em%
      \HoLogoFont@font{SliTeX}{sc}{%
        l%
        \kern-.035em%
        i%
      }%
    }%
    \HOLOGO@discretionary
    \kern-.06em%
    \hologo{TeX}%
  }%
}
%    \end{macrocode}
%    \end{macro}
%    \begin{macro}{\HoLogoBkm@SliTeX@narrow}
%    \begin{macrocode}
\def\HoLogoBkm@SliTeX@narrow#1{SliTeX}
%    \end{macrocode}
%    \end{macro}
%    \begin{macro}{\HoLogoHtml@SliTeX@narrow}
%    \begin{macrocode}
\def\HoLogoHtml@SliTeX@narrow#1{%
  \HoLogoCss@SliTeX@narrow
  \HOLOGO@Span{SliTeX-narrow}{%
    \HoLogoFont@font{SliTeX}{rm}{%
      S%
        \HOLOGO@Span{l}{l}%
        \HOLOGO@Span{i}{i}%
      \hologo{TeX}%
    }%
  }%
}
%    \end{macrocode}
%    \end{macro}
%    \begin{macro}{\HoLogoCss@SliTeX@narrow}
%    \begin{macrocode}
\def\HoLogoCss@SliTeX@narrow{%
  \Css{%
    span.HoLogo-SliTeX-narrow span.HoLogo-l{%
      margin-left:-.06em;%
      margin-right:-.035em;%
      font-variant:small-caps;%
    }%
  }%
  \Css{%
    span.HoLogo-SliTeX-narrow span.HoLogo-i{%
      margin-right:-.06em;%
      font-variant:small-caps;%
    }%
  }%
  \global\let\HoLogoCss@SliTeX@narrow\relax
}
%    \end{macrocode}
%    \end{macro}
%
% \paragraph{Macro set completion.}
%
%    \begin{macro}{\HoLogo@SLiTeX@simple}
%    \begin{macrocode}
\def\HoLogo@SLiTeX@simple{\HoLogo@SliTeX@simple}
%    \end{macrocode}
%    \end{macro}
%    \begin{macro}{\HoLogoBkm@SLiTeX@simple}
%    \begin{macrocode}
\def\HoLogoBkm@SLiTeX@simple{\HoLogoBkm@SliTeX@simple}
%    \end{macrocode}
%    \end{macro}
%    \begin{macro}{\HoLogoHtml@SLiTeX@simple}
%    \begin{macrocode}
\def\HoLogoHtml@SLiTeX@simple{\HoLogoHtml@SliTeX@simple}
%    \end{macrocode}
%    \end{macro}
%
%    \begin{macro}{\HoLogo@SLiTeX@narrow}
%    \begin{macrocode}
\def\HoLogo@SLiTeX@narrow{\HoLogo@SliTeX@narrow}
%    \end{macrocode}
%    \end{macro}
%    \begin{macro}{\HoLogoBkm@SLiTeX@narrow}
%    \begin{macrocode}
\def\HoLogoBkm@SLiTeX@narrow{\HoLogoBkm@SliTeX@narrow}
%    \end{macrocode}
%    \end{macro}
%    \begin{macro}{\HoLogoHtml@SLiTeX@narrow}
%    \begin{macrocode}
\def\HoLogoHtml@SLiTeX@narrow{\HoLogoHtml@SliTeX@narrow}
%    \end{macrocode}
%    \end{macro}
%
%    \begin{macro}{\HoLogo@SliTeX@lift}
%    \begin{macrocode}
\def\HoLogo@SliTeX@lift{\HoLogo@SLiTeX@lift}
%    \end{macrocode}
%    \end{macro}
%    \begin{macro}{\HoLogoBkm@SliTeX@lift}
%    \begin{macrocode}
\def\HoLogoBkm@SliTeX@lift{\HoLogoBkm@SLiTeX@lift}
%    \end{macrocode}
%    \end{macro}
%    \begin{macro}{\HoLogoHtml@SliTeX@lift}
%    \begin{macrocode}
\def\HoLogoHtml@SliTeX@lift{\HoLogoHtml@SLiTeX@lift}
%    \end{macrocode}
%    \end{macro}
%
% \paragraph{Defaults.}
%
%    \begin{macro}{\HoLogo@SLiTeX}
%    \begin{macrocode}
\def\HoLogo@SLiTeX{\HoLogo@SLiTeX@lift}
%    \end{macrocode}
%    \end{macro}
%    \begin{macro}{\HoLogoBkm@SLiTeX}
%    \begin{macrocode}
\def\HoLogoBkm@SLiTeX{\HoLogoBkm@SLiTeX@lift}
%    \end{macrocode}
%    \end{macro}
%    \begin{macro}{\HoLogoHtml@SLiTeX}
%    \begin{macrocode}
\def\HoLogoHtml@SLiTeX{\HoLogoHtml@SLiTeX@lift}
%    \end{macrocode}
%    \end{macro}
%
%    \begin{macro}{\HoLogo@SliTeX}
%    \begin{macrocode}
\def\HoLogo@SliTeX{\HoLogo@SliTeX@narrow}
%    \end{macrocode}
%    \end{macro}
%    \begin{macro}{\HoLogoBkm@SliTeX}
%    \begin{macrocode}
\def\HoLogoBkm@SliTeX{\HoLogoBkm@SliTeX@narrow}
%    \end{macrocode}
%    \end{macro}
%    \begin{macro}{\HoLogoHtml@SliTeX}
%    \begin{macrocode}
\def\HoLogoHtml@SliTeX{\HoLogoHtml@SliTeX@narrow}
%    \end{macrocode}
%    \end{macro}
%
% \subsubsection{\hologo{LuaTeX}}
%
%    \begin{macro}{\HoLogo@LuaTeX}
%    The kerning is an idea of Hans Hagen, see mailing list
%    `luatex at tug dot org' in March 2010.
%    \begin{macrocode}
\def\HoLogo@LuaTeX#1{%
  \HOLOGO@mbox{%
    Lua%
    \HOLOGO@NegativeKerning{aT,oT,To}%
    \hologo{TeX}%
  }%
}
%    \end{macrocode}
%    \end{macro}
%    \begin{macro}{\HoLogoHtml@LuaTeX}
%    \begin{macrocode}
\let\HoLogoHtml@LuaTeX\HoLogo@LuaTeX
%    \end{macrocode}
%    \end{macro}
%
% \subsubsection{\hologo{LuaLaTeX}}
%
%    \begin{macro}{\HoLogo@LuaLaTeX}
%    \begin{macrocode}
\def\HoLogo@LuaLaTeX#1{%
  \HOLOGO@mbox{%
    Lua%
    \hologo{LaTeX}%
  }%
}
%    \end{macrocode}
%    \end{macro}
%    \begin{macro}{\HoLogoHtml@LuaLaTeX}
%    \begin{macrocode}
\let\HoLogoHtml@LuaLaTeX\HoLogo@LuaLaTeX
%    \end{macrocode}
%    \end{macro}
%
% \subsubsection{\hologo{XeTeX}, \hologo{XeLaTeX}}
%
%    \begin{macro}{\HOLOGO@IfCharExists}
%    \begin{macrocode}
\ifluatex
  \ifnum\luatexversion<36 %
  \else
    \def\HOLOGO@IfCharExists#1{%
      \ifnum
        \directlua{%
           if luaotfload and luaotfload.aux then
             if luaotfload.aux.font_has_glyph(%
                    font.current(), \number#1) then % 	 
	       tex.print("1") % 	 
	     end % 	 
	   elseif font and font.fonts and font.current then %
            local f = font.fonts[font.current()]%
            if f.characters and f.characters[\number#1] then %
              tex.print("1")%
            end %
          end%
        }0=\ltx@zero
        \expandafter\ltx@secondoftwo
      \else
        \expandafter\ltx@firstoftwo
      \fi
    }%
  \fi
\fi
\ltx@IfUndefined{HOLOGO@IfCharExists}{%
  \def\HOLOGO@@IfCharExists#1{%
    \begingroup
      \tracinglostchars=\ltx@zero
      \setbox\ltx@zero=\hbox{%
        \kern7sp\char#1\relax
        \ifnum\lastkern>\ltx@zero
          \expandafter\aftergroup\csname iffalse\endcsname
        \else
          \expandafter\aftergroup\csname iftrue\endcsname
        \fi
      }%
      % \if{true|false} from \aftergroup
      \endgroup
      \expandafter\ltx@firstoftwo
    \else
      \endgroup
      \expandafter\ltx@secondoftwo
    \fi
  }%
  \ifxetex
    \ltx@IfUndefined{XeTeXfonttype}{}{%
      \ltx@IfUndefined{XeTeXcharglyph}{}{%
        \def\HOLOGO@IfCharExists#1{%
          \ifnum\XeTeXfonttype\font>\ltx@zero
            \expandafter\ltx@firstofthree
          \else
            \expandafter\ltx@gobble
          \fi
          {%
            \ifnum\XeTeXcharglyph#1>\ltx@zero
              \expandafter\ltx@firstoftwo
            \else
              \expandafter\ltx@secondoftwo
            \fi
          }%
          \HOLOGO@@IfCharExists{#1}%
        }%
      }%
    }%
  \fi
}{}
\ltx@ifundefined{HOLOGO@IfCharExists}{%
  \ifnum64=`\^^^^0040\relax % test for big chars of LuaTeX/XeTeX
    \let\HOLOGO@IfCharExists\HOLOGO@@IfCharExists
  \else
    \def\HOLOGO@IfCharExists#1{%
      \ifnum#1>255 %
        \expandafter\ltx@fourthoffour
      \fi
      \HOLOGO@@IfCharExists{#1}%
    }%
  \fi
}{}
%    \end{macrocode}
%    \end{macro}
%
%    \begin{macro}{\HoLogo@Xe}
%    Source: package \xpackage{dtklogos}
%    \begin{macrocode}
\def\HoLogo@Xe#1{%
  X%
  \kern-.1em\relax
  \HOLOGO@IfCharExists{"018E}{%
    \lower.5ex\hbox{\char"018E}%
  }{%
    \chardef\HOLOGO@choice=\ltx@zero
    \ifdim\fontdimen\ltx@one\font>0pt %
      \ltx@IfUndefined{rotatebox}{%
        \ltx@IfUndefined{pgftext}{%
          \ltx@IfUndefined{psscalebox}{%
            \ltx@IfUndefined{HOLOGO@ScaleBox@\hologoDriver}{%
            }{%
              \chardef\HOLOGO@choice=4 %
            }%
          }{%
            \chardef\HOLOGO@choice=3 %
          }%
        }{%
          \chardef\HOLOGO@choice=2 %
        }%
      }{%
        \chardef\HOLOGO@choice=1 %
      }%
      \ifcase\HOLOGO@choice
        \HOLOGO@WarningUnsupportedDriver{Xe}%
        e%
      \or % 1: \rotatebox
        \begingroup
          \setbox\ltx@zero\hbox{\rotatebox{180}{E}}%
          \ltx@LocDimenA=\dp\ltx@zero
          \advance\ltx@LocDimenA by -.5ex\relax
          \raise\ltx@LocDimenA\box\ltx@zero
        \endgroup
      \or % 2: \pgftext
        \lower.5ex\hbox{%
          \pgfpicture
            \pgftext[rotate=180]{E}%
          \endpgfpicture
        }%
      \or % 3: \psscalebox
        \begingroup
          \setbox\ltx@zero\hbox{\psscalebox{-1 -1}{E}}%
          \ltx@LocDimenA=\dp\ltx@zero
          \advance\ltx@LocDimenA by -.5ex\relax
          \raise\ltx@LocDimenA\box\ltx@zero
        \endgroup
      \or % 4: \HOLOGO@PointReflectBox
        \lower.5ex\hbox{\HOLOGO@PointReflectBox{E}}%
      \else
        \@PackageError{hologo}{Internal error (choice/it}\@ehc
      \fi
    \else
      \ltx@IfUndefined{reflectbox}{%
        \ltx@IfUndefined{pgftext}{%
          \ltx@IfUndefined{psscalebox}{%
            \ltx@IfUndefined{HOLOGO@ScaleBox@\hologoDriver}{%
            }{%
              \chardef\HOLOGO@choice=4 %
            }%
          }{%
            \chardef\HOLOGO@choice=3 %
          }%
        }{%
          \chardef\HOLOGO@choice=2 %
        }%
      }{%
        \chardef\HOLOGO@choice=1 %
      }%
      \ifcase\HOLOGO@choice
        \HOLOGO@WarningUnsupportedDriver{Xe}%
        e%
      \or % 1: reflectbox
        \lower.5ex\hbox{%
          \reflectbox{E}%
        }%
      \or % 2: \pgftext
        \lower.5ex\hbox{%
          \pgfpicture
            \pgftransformxscale{-1}%
            \pgftext{E}%
          \endpgfpicture
        }%
      \or % 3: \psscalebox
        \lower.5ex\hbox{%
          \psscalebox{-1 1}{E}%
        }%
      \or % 4: \HOLOGO@Reflectbox
        \lower.5ex\hbox{%
          \HOLOGO@ReflectBox{E}%
        }%
      \else
        \@PackageError{hologo}{Internal error (choice/up)}\@ehc
      \fi
    \fi
  }%
}
%    \end{macrocode}
%    \end{macro}
%    \begin{macro}{\HoLogoHtml@Xe}
%    \begin{macrocode}
\def\HoLogoHtml@Xe#1{%
  \HoLogoCss@Xe
  \HOLOGO@Span{Xe}{%
    X%
    \HOLOGO@Span{e}{%
      \HCode{&\ltx@hashchar x018e;}%
    }%
  }%
}
%    \end{macrocode}
%    \end{macro}
%    \begin{macro}{\HoLogoCss@Xe}
%    \begin{macrocode}
\def\HoLogoCss@Xe{%
  \Css{%
    span.HoLogo-Xe span.HoLogo-e{%
      position:relative;%
      top:.5ex;%
      left-margin:-.1em;%
    }%
  }%
  \global\let\HoLogoCss@Xe\relax
}
%    \end{macrocode}
%    \end{macro}
%
%    \begin{macro}{\HoLogo@XeTeX}
%    \begin{macrocode}
\def\HoLogo@XeTeX#1{%
  \hologo{Xe}%
  \kern-.15em\relax
  \hologo{TeX}%
}
%    \end{macrocode}
%    \end{macro}
%
%    \begin{macro}{\HoLogoHtml@XeTeX}
%    \begin{macrocode}
\def\HoLogoHtml@XeTeX#1{%
  \HoLogoCss@XeTeX
  \HOLOGO@Span{XeTeX}{%
    \hologo{Xe}%
    \hologo{TeX}%
  }%
}
%    \end{macrocode}
%    \end{macro}
%    \begin{macro}{\HoLogoCss@XeTeX}
%    \begin{macrocode}
\def\HoLogoCss@XeTeX{%
  \Css{%
    span.HoLogo-XeTeX span.HoLogo-TeX{%
      margin-left:-.15em;%
    }%
  }%
  \global\let\HoLogoCss@XeTeX\relax
}
%    \end{macrocode}
%    \end{macro}
%
%    \begin{macro}{\HoLogo@XeLaTeX}
%    \begin{macrocode}
\def\HoLogo@XeLaTeX#1{%
  \hologo{Xe}%
  \kern-.13em%
  \hologo{LaTeX}%
}
%    \end{macrocode}
%    \end{macro}
%    \begin{macro}{\HoLogoHtml@XeLaTeX}
%    \begin{macrocode}
\def\HoLogoHtml@XeLaTeX#1{%
  \HoLogoCss@XeLaTeX
  \HOLOGO@Span{XeLaTeX}{%
    \hologo{Xe}%
    \hologo{LaTeX}%
  }%
}
%    \end{macrocode}
%    \end{macro}
%    \begin{macro}{\HoLogoCss@XeLaTeX}
%    \begin{macrocode}
\def\HoLogoCss@XeLaTeX{%
  \Css{%
    span.HoLogo-XeLaTeX span.HoLogo-Xe{%
      margin-right:-.13em;%
    }%
  }%
  \global\let\HoLogoCss@XeLaTeX\relax
}
%    \end{macrocode}
%    \end{macro}
%
% \subsubsection{\hologo{pdfTeX}, \hologo{pdfLaTeX}}
%
%    \begin{macro}{\HoLogo@pdfTeX}
%    \begin{macrocode}
\def\HoLogo@pdfTeX#1{%
  \HOLOGO@mbox{%
    #1{p}{P}df\hologo{TeX}%
  }%
}
%    \end{macrocode}
%    \end{macro}
%    \begin{macro}{\HoLogoCs@pdfTeX}
%    \begin{macrocode}
\def\HoLogoCs@pdfTeX#1{#1{p}{P}dfTeX}
%    \end{macrocode}
%    \end{macro}
%    \begin{macro}{\HoLogoBkm@pdfTeX}
%    \begin{macrocode}
\def\HoLogoBkm@pdfTeX#1{%
  #1{p}{P}df\hologo{TeX}%
}
%    \end{macrocode}
%    \end{macro}
%    \begin{macro}{\HoLogoHtml@pdfTeX}
%    \begin{macrocode}
\let\HoLogoHtml@pdfTeX\HoLogo@pdfTeX
%    \end{macrocode}
%    \end{macro}
%
%    \begin{macro}{\HoLogo@pdfLaTeX}
%    \begin{macrocode}
\def\HoLogo@pdfLaTeX#1{%
  \HOLOGO@mbox{%
    #1{p}{P}df\hologo{LaTeX}%
  }%
}
%    \end{macrocode}
%    \end{macro}
%    \begin{macro}{\HoLogoCs@pdfLaTeX}
%    \begin{macrocode}
\def\HoLogoCs@pdfLaTeX#1{#1{p}{P}dfLaTeX}
%    \end{macrocode}
%    \end{macro}
%    \begin{macro}{\HoLogoBkm@pdfLaTeX}
%    \begin{macrocode}
\def\HoLogoBkm@pdfLaTeX#1{%
  #1{p}{P}df\hologo{LaTeX}%
}
%    \end{macrocode}
%    \end{macro}
%    \begin{macro}{\HoLogoHtml@pdfLaTeX}
%    \begin{macrocode}
\let\HoLogoHtml@pdfLaTeX\HoLogo@pdfLaTeX
%    \end{macrocode}
%    \end{macro}
%
% \subsubsection{\hologo{VTeX}}
%
%    \begin{macro}{\HoLogo@VTeX}
%    \begin{macrocode}
\def\HoLogo@VTeX#1{%
  \HOLOGO@mbox{%
    V\hologo{TeX}%
  }%
}
%    \end{macrocode}
%    \end{macro}
%    \begin{macro}{\HoLogoHtml@VTeX}
%    \begin{macrocode}
\let\HoLogoHtml@VTeX\HoLogo@VTeX
%    \end{macrocode}
%    \end{macro}
%
% \subsubsection{\hologo{AmS}, \dots}
%
%    Source: class \xclass{amsdtx}
%
%    \begin{macro}{\HoLogo@AmS}
%    \begin{macrocode}
\def\HoLogo@AmS#1{%
  \HoLogoFont@font{AmS}{sy}{%
    A%
    \kern-.1667em%
    \lower.5ex\hbox{M}%
    \kern-.125em%
    S%
  }%
}
%    \end{macrocode}
%    \end{macro}
%    \begin{macro}{\HoLogoBkm@AmS}
%    \begin{macrocode}
\def\HoLogoBkm@AmS#1{AmS}
%    \end{macrocode}
%    \end{macro}
%    \begin{macro}{\HoLogoHtml@AmS}
%    \begin{macrocode}
\def\HoLogoHtml@AmS#1{%
  \HoLogoCss@AmS
%  \HoLogoFont@font{AmS}{sy}{%
    \HOLOGO@Span{AmS}{%
      A%
      \HOLOGO@Span{M}{M}%
      S%
    }%
%   }%
}
%    \end{macrocode}
%    \end{macro}
%    \begin{macro}{\HoLogoCss@AmS}
%    \begin{macrocode}
\def\HoLogoCss@AmS{%
  \Css{%
    span.HoLogo-AmS span.HoLogo-M{%
      position:relative;%
      top:.5ex;%
      margin-left:-.1667em;%
      margin-right:-.125em;%
      text-decoration:none;%
    }%
  }%
  \global\let\HoLogoCss@AmS\relax
}
%    \end{macrocode}
%    \end{macro}
%
%    \begin{macro}{\HoLogo@AmSTeX}
%    \begin{macrocode}
\def\HoLogo@AmSTeX#1{%
  \hologo{AmS}%
  \HOLOGO@hyphen
  \hologo{TeX}%
}
%    \end{macrocode}
%    \end{macro}
%    \begin{macro}{\HoLogoBkm@AmSTeX}
%    \begin{macrocode}
\def\HoLogoBkm@AmSTeX#1{AmS-TeX}%
%    \end{macrocode}
%    \end{macro}
%    \begin{macro}{\HoLogoHtml@AmSTeX}
%    \begin{macrocode}
\let\HoLogoHtml@AmSTeX\HoLogo@AmSTeX
%    \end{macrocode}
%    \end{macro}
%
%    \begin{macro}{\HoLogo@AmSLaTeX}
%    \begin{macrocode}
\def\HoLogo@AmSLaTeX#1{%
  \hologo{AmS}%
  \HOLOGO@hyphen
  \hologo{LaTeX}%
}
%    \end{macrocode}
%    \end{macro}
%    \begin{macro}{\HoLogoBkm@AmSLaTeX}
%    \begin{macrocode}
\def\HoLogoBkm@AmSLaTeX#1{AmS-LaTeX}%
%    \end{macrocode}
%    \end{macro}
%    \begin{macro}{\HoLogoHtml@AmSLaTeX}
%    \begin{macrocode}
\let\HoLogoHtml@AmSLaTeX\HoLogo@AmSLaTeX
%    \end{macrocode}
%    \end{macro}
%
% \subsubsection{\hologo{BibTeX}}
%
%    \begin{macro}{\HoLogo@BibTeX@sc}
%    A definition of \hologo{BibTeX} is provided in
%    the documentation source for the manual of \hologo{BibTeX}
%    \cite{btxdoc}.
%\begin{quote}
%\begin{verbatim}
%\def\BibTeX{%
%  {%
%    \rm
%    B%
%    \kern-.05em%
%    {%
%      \sc
%      i%
%      \kern-.025em %
%      b%
%    }%
%    \kern-.08em
%    T%
%    \kern-.1667em%
%    \lower.7ex\hbox{E}%
%    \kern-.125em%
%    X%
%  }%
%}
%\end{verbatim}
%\end{quote}
%    \begin{macrocode}
\def\HoLogo@BibTeX@sc#1{%
  B%
  \kern-.05em%
  \HoLogoFont@font{BibTeX}{sc}{%
    i%
    \kern-.025em%
    b%
  }%
  \HOLOGO@discretionary
  \kern-.08em%
  \hologo{TeX}%
}
%    \end{macrocode}
%    \end{macro}
%    \begin{macro}{\HoLogoHtml@BibTeX@sc}
%    \begin{macrocode}
\def\HoLogoHtml@BibTeX@sc#1{%
  \HoLogoCss@BibTeX@sc
  \HOLOGO@Span{BibTeX-sc}{%
    B%
    \HOLOGO@Span{i}{i}%
    \HOLOGO@Span{b}{b}%
    \hologo{TeX}%
  }%
}
%    \end{macrocode}
%    \end{macro}
%    \begin{macro}{\HoLogoCss@BibTeX@sc}
%    \begin{macrocode}
\def\HoLogoCss@BibTeX@sc{%
  \Css{%
    span.HoLogo-BibTeX-sc span.HoLogo-i{%
      margin-left:-.05em;%
      margin-right:-.025em;%
      font-variant:small-caps;%
    }%
  }%
  \Css{%
    span.HoLogo-BibTeX-sc span.HoLogo-b{%
      margin-right:-.08em;%
      font-variant:small-caps;%
    }%
  }%
  \global\let\HoLogoCss@BibTeX@sc\relax
}
%    \end{macrocode}
%    \end{macro}
%
%    \begin{macro}{\HoLogo@BibTeX@sf}
%    Variant \xoption{sf} avoids trouble with unavailable
%    small caps fonts (e.g., bold versions of Computer Modern or
%    Latin Modern). The definition is taken from
%    package \xpackage{dtklogos} \cite{dtklogos}.
%\begin{quote}
%\begin{verbatim}
%\DeclareRobustCommand{\BibTeX}{%
%  B%
%  \kern-.05em%
%  \hbox{%
%    $\m@th$% %% force math size calculations
%    \csname S@\f@size\endcsname
%    \fontsize\sf@size\z@
%    \math@fontsfalse
%    \selectfont
%    I%
%    \kern-.025em%
%    B
%  }%
%  \kern-.08em%
%  \-%
%  \TeX
%}
%\end{verbatim}
%\end{quote}
%    \begin{macrocode}
\def\HoLogo@BibTeX@sf#1{%
  B%
  \kern-.05em%
  \HoLogoFont@font{BibTeX}{bibsf}{%
    I%
    \kern-.025em%
    B%
  }%
  \HOLOGO@discretionary
  \kern-.08em%
  \hologo{TeX}%
}
%    \end{macrocode}
%    \end{macro}
%    \begin{macro}{\HoLogoHtml@BibTeX@sf}
%    \begin{macrocode}
\def\HoLogoHtml@BibTeX@sf#1{%
  \HoLogoCss@BibTeX@sf
  \HOLOGO@Span{BibTeX-sf}{%
    B%
    \HoLogoFont@font{BibTeX}{bibsf}{%
      \HOLOGO@Span{i}{I}%
      B%
    }%
    \hologo{TeX}%
  }%
}
%    \end{macrocode}
%    \end{macro}
%    \begin{macro}{\HoLogoCss@BibTeX@sf}
%    \begin{macrocode}
\def\HoLogoCss@BibTeX@sf{%
  \Css{%
    span.HoLogo-BibTeX-sf span.HoLogo-i{%
      margin-left:-.05em;%
      margin-right:-.025em;%
    }%
  }%
  \Css{%
    span.HoLogo-BibTeX-sf span.HoLogo-TeX{%
      margin-left:-.08em;%
    }%
  }%
  \global\let\HoLogoCss@BibTeX@sf\relax
}
%    \end{macrocode}
%    \end{macro}
%
%    \begin{macro}{\HoLogo@BibTeX}
%    \begin{macrocode}
\def\HoLogo@BibTeX{\HoLogo@BibTeX@sf}
%    \end{macrocode}
%    \end{macro}
%    \begin{macro}{\HoLogoHtml@BibTeX}
%    \begin{macrocode}
\def\HoLogoHtml@BibTeX{\HoLogoHtml@BibTeX@sf}
%    \end{macrocode}
%    \end{macro}
%
% \subsubsection{\hologo{BibTeX8}}
%
%    \begin{macro}{\HoLogo@BibTeX8}
%    \begin{macrocode}
\expandafter\def\csname HoLogo@BibTeX8\endcsname#1{%
  \hologo{BibTeX}%
  8%
}
%    \end{macrocode}
%    \end{macro}
%
%    \begin{macro}{\HoLogoBkm@BibTeX8}
%    \begin{macrocode}
\expandafter\def\csname HoLogoBkm@BibTeX8\endcsname#1{%
  \hologo{BibTeX}%
  8%
}
%    \end{macrocode}
%    \end{macro}
%    \begin{macro}{\HoLogoHtml@BibTeX8}
%    \begin{macrocode}
\expandafter
\let\csname HoLogoHtml@BibTeX8\expandafter\endcsname
\csname HoLogo@BibTeX8\endcsname
%    \end{macrocode}
%    \end{macro}
%
% \subsubsection{\hologo{ConTeXt}}
%
%    \begin{macro}{\HoLogo@ConTeXt@simple}
%    \begin{macrocode}
\def\HoLogo@ConTeXt@simple#1{%
  \HOLOGO@mbox{Con}%
  \HOLOGO@discretionary
  \HOLOGO@mbox{\hologo{TeX}t}%
}
%    \end{macrocode}
%    \end{macro}
%    \begin{macro}{\HoLogoHtml@ConTeXt@simple}
%    \begin{macrocode}
\let\HoLogoHtml@ConTeXt@simple\HoLogo@ConTeXt@simple
%    \end{macrocode}
%    \end{macro}
%
%    \begin{macro}{\HoLogo@ConTeXt@narrow}
%    This definition of logo \hologo{ConTeXt} with variant \xoption{narrow}
%    comes from TUGboat's class \xclass{ltugboat} (version 2010/11/15 v2.8).
%    \begin{macrocode}
\def\HoLogo@ConTeXt@narrow#1{%
  \HOLOGO@mbox{C\kern-.0333emon}%
  \HOLOGO@discretionary
  \kern-.0667em%
  \HOLOGO@mbox{\hologo{TeX}\kern-.0333emt}%
}
%    \end{macrocode}
%    \end{macro}
%    \begin{macro}{\HoLogoHtml@ConTeXt@narrow}
%    \begin{macrocode}
\def\HoLogoHtml@ConTeXt@narrow#1{%
  \HoLogoCss@ConTeXt@narrow
  \HOLOGO@Span{ConTeXt-narrow}{%
    \HOLOGO@Span{C}{C}%
    on%
    \hologo{TeX}%
    t%
  }%
}
%    \end{macrocode}
%    \end{macro}
%    \begin{macro}{\HoLogoCss@ConTeXt@narrow}
%    \begin{macrocode}
\def\HoLogoCss@ConTeXt@narrow{%
  \Css{%
    span.HoLogo-ConTeXt-narrow span.HoLogo-C{%
      margin-left:-.0333em;%
    }%
  }%
  \Css{%
    span.HoLogo-ConTeXt-narrow span.HoLogo-TeX{%
      margin-left:-.0667em;%
      margin-right:-.0333em;%
    }%
  }%
  \global\let\HoLogoCss@ConTeXt@narrow\relax
}
%    \end{macrocode}
%    \end{macro}
%
%    \begin{macro}{\HoLogo@ConTeXt}
%    \begin{macrocode}
\def\HoLogo@ConTeXt{\HoLogo@ConTeXt@narrow}
%    \end{macrocode}
%    \end{macro}
%    \begin{macro}{\HoLogoHtml@ConTeXt}
%    \begin{macrocode}
\def\HoLogoHtml@ConTeXt{\HoLogoHtml@ConTeXt@narrow}
%    \end{macrocode}
%    \end{macro}
%
% \subsubsection{\hologo{emTeX}}
%
%    \begin{macro}{\HoLogo@emTeX}
%    \begin{macrocode}
\def\HoLogo@emTeX#1{%
  \HOLOGO@mbox{#1{e}{E}m}%
  \HOLOGO@discretionary
  \hologo{TeX}%
}
%    \end{macrocode}
%    \end{macro}
%    \begin{macro}{\HoLogoCs@emTeX}
%    \begin{macrocode}
\def\HoLogoCs@emTeX#1{#1{e}{E}mTeX}%
%    \end{macrocode}
%    \end{macro}
%    \begin{macro}{\HoLogoBkm@emTeX}
%    \begin{macrocode}
\def\HoLogoBkm@emTeX#1{%
  #1{e}{E}m\hologo{TeX}%
}
%    \end{macrocode}
%    \end{macro}
%    \begin{macro}{\HoLogoHtml@emTeX}
%    \begin{macrocode}
\let\HoLogoHtml@emTeX\HoLogo@emTeX
%    \end{macrocode}
%    \end{macro}
%
% \subsubsection{\hologo{ExTeX}}
%
%    \begin{macro}{\HoLogo@ExTeX}
%    The definition is taken from the FAQ of the
%    project \hologo{ExTeX}
%    \cite{ExTeX-FAQ}.
%\begin{quote}
%\begin{verbatim}
%\def\ExTeX{%
%  \textrm{% Logo always with serifs
%    \ensuremath{%
%      \textstyle
%      \varepsilon_{%
%        \kern-0.15em%
%        \mathcal{X}%
%      }%
%    }%
%    \kern-.15em%
%    \TeX
%  }%
%}
%\end{verbatim}
%\end{quote}
%    \begin{macrocode}
\def\HoLogo@ExTeX#1{%
  \HoLogoFont@font{ExTeX}{rm}{%
    \ltx@mbox{%
      \HOLOGO@MathSetup
      $%
        \textstyle
        \varepsilon_{%
          \kern-0.15em%
          \HoLogoFont@font{ExTeX}{sy}{X}%
        }%
      $%
    }%
    \HOLOGO@discretionary
    \kern-.15em%
    \hologo{TeX}%
  }%
}
%    \end{macrocode}
%    \end{macro}
%    \begin{macro}{\HoLogoHtml@ExTeX}
%    \begin{macrocode}
\def\HoLogoHtml@ExTeX#1{%
  \HoLogoCss@ExTeX
  \HoLogoFont@font{ExTeX}{rm}{%
    \HOLOGO@Span{ExTeX}{%
      \ltx@mbox{%
        \HOLOGO@MathSetup
        $\textstyle\varepsilon$%
        \HOLOGO@Span{X}{$\textstyle\chi$}%
        \hologo{TeX}%
      }%
    }%
  }%
}
%    \end{macrocode}
%    \end{macro}
%    \begin{macro}{\HoLogoBkm@ExTeX}
%    \begin{macrocode}
\def\HoLogoBkm@ExTeX#1{%
  \HOLOGO@PdfdocUnicode{#1{e}{E}x}{\textepsilon\textchi}%
  \hologo{TeX}%
}
%    \end{macrocode}
%    \end{macro}
%    \begin{macro}{\HoLogoCss@ExTeX}
%    \begin{macrocode}
\def\HoLogoCss@ExTeX{%
  \Css{%
    span.HoLogo-ExTeX{%
      font-family:serif;%
    }%
  }%
  \Css{%
    span.HoLogo-ExTeX span.HoLogo-TeX{%
      margin-left:-.15em;%
    }%
  }%
  \global\let\HoLogoCss@ExTeX\relax
}
%    \end{macrocode}
%    \end{macro}
%
% \subsubsection{\hologo{MiKTeX}}
%
%    \begin{macro}{\HoLogo@MiKTeX}
%    \begin{macrocode}
\def\HoLogo@MiKTeX#1{%
  \HOLOGO@mbox{MiK}%
  \HOLOGO@discretionary
  \hologo{TeX}%
}
%    \end{macrocode}
%    \end{macro}
%    \begin{macro}{\HoLogoHtml@MiKTeX}
%    \begin{macrocode}
\let\HoLogoHtml@MiKTeX\HoLogo@MiKTeX
%    \end{macrocode}
%    \end{macro}
%
% \subsubsection{\hologo{OzTeX} and friends}
%
%    Source: \hologo{OzTeX} FAQ \cite{OzTeX}:
%    \begin{quote}
%      |\def\OzTeX{O\kern-.03em z\kern-.15em\TeX}|\\
%      (There is no kerning in OzMF, OzMP and OzTtH.)
%    \end{quote}
%
%    \begin{macro}{\HoLogo@OzTeX}
%    \begin{macrocode}
\def\HoLogo@OzTeX#1{%
  O%
  \kern-.03em %
  z%
  \kern-.15em %
  \hologo{TeX}%
}
%    \end{macrocode}
%    \end{macro}
%    \begin{macro}{\HoLogoHtml@OzTeX}
%    \begin{macrocode}
\def\HoLogoHtml@OzTeX#1{%
  \HoLogoCss@OzTeX
  \HOLOGO@Span{OzTeX}{%
    O%
    \HOLOGO@Span{z}{z}%
    \hologo{TeX}%
  }%
}
%    \end{macrocode}
%    \end{macro}
%    \begin{macro}{\HoLogoCss@OzTeX}
%    \begin{macrocode}
\def\HoLogoCss@OzTeX{%
  \Css{%
    span.HoLogo-OzTeX span.HoLogo-z{%
      margin-left:-.03em;%
      margin-right:-.15em;%
    }%
  }%
  \global\let\HoLogoCss@OzTeX\relax
}
%    \end{macrocode}
%    \end{macro}
%
%    \begin{macro}{\HoLogo@OzMF}
%    \begin{macrocode}
\def\HoLogo@OzMF#1{%
  \HOLOGO@mbox{OzMF}%
}
%    \end{macrocode}
%    \end{macro}
%    \begin{macro}{\HoLogo@OzMP}
%    \begin{macrocode}
\def\HoLogo@OzMP#1{%
  \HOLOGO@mbox{OzMP}%
}
%    \end{macrocode}
%    \end{macro}
%    \begin{macro}{\HoLogo@OzTtH}
%    \begin{macrocode}
\def\HoLogo@OzTtH#1{%
  \HOLOGO@mbox{OzTtH}%
}
%    \end{macrocode}
%    \end{macro}
%
% \subsubsection{\hologo{PCTeX}}
%
%    \begin{macro}{\HoLogo@PCTeX}
%    \begin{macrocode}
\def\HoLogo@PCTeX#1{%
  \HOLOGO@mbox{PC}%
  \hologo{TeX}%
}
%    \end{macrocode}
%    \end{macro}
%    \begin{macro}{\HoLogoHtml@PCTeX}
%    \begin{macrocode}
\let\HoLogoHtml@PCTeX\HoLogo@PCTeX
%    \end{macrocode}
%    \end{macro}
%
% \subsubsection{\hologo{PiCTeX}}
%
%    The original definitions from \xfile{pictex.tex} \cite{PiCTeX}:
%\begin{quote}
%\begin{verbatim}
%\def\PiC{%
%  P%
%  \kern-.12em%
%  \lower.5ex\hbox{I}%
%  \kern-.075em%
%  C%
%}
%\def\PiCTeX{%
%  \PiC
%  \kern-.11em%
%  \TeX
%}
%\end{verbatim}
%\end{quote}
%
%    \begin{macro}{\HoLogo@PiC}
%    \begin{macrocode}
\def\HoLogo@PiC#1{%
  P%
  \kern-.12em%
  \lower.5ex\hbox{I}%
  \kern-.075em%
  C%
  \HOLOGO@SpaceFactor
}
%    \end{macrocode}
%    \end{macro}
%    \begin{macro}{\HoLogoHtml@PiC}
%    \begin{macrocode}
\def\HoLogoHtml@PiC#1{%
  \HoLogoCss@PiC
  \HOLOGO@Span{PiC}{%
    P%
    \HOLOGO@Span{i}{I}%
    C%
  }%
}
%    \end{macrocode}
%    \end{macro}
%    \begin{macro}{\HoLogoCss@PiC}
%    \begin{macrocode}
\def\HoLogoCss@PiC{%
  \Css{%
    span.HoLogo-PiC span.HoLogo-i{%
      position:relative;%
      top:.5ex;%
      margin-left:-.12em;%
      margin-right:-.075em;%
      text-decoration:none;%
    }%
  }%
  \global\let\HoLogoCss@PiC\relax
}
%    \end{macrocode}
%    \end{macro}
%
%    \begin{macro}{\HoLogo@PiCTeX}
%    \begin{macrocode}
\def\HoLogo@PiCTeX#1{%
  \hologo{PiC}%
  \HOLOGO@discretionary
  \kern-.11em%
  \hologo{TeX}%
}
%    \end{macrocode}
%    \end{macro}
%    \begin{macro}{\HoLogoHtml@PiCTeX}
%    \begin{macrocode}
\def\HoLogoHtml@PiCTeX#1{%
  \HoLogoCss@PiCTeX
  \HOLOGO@Span{PiCTeX}{%
    \hologo{PiC}%
    \hologo{TeX}%
  }%
}
%    \end{macrocode}
%    \end{macro}
%    \begin{macro}{\HoLogoCss@PiCTeX}
%    \begin{macrocode}
\def\HoLogoCss@PiCTeX{%
  \Css{%
    span.HoLogo-PiCTeX span.HoLogo-PiC{%
      margin-right:-.11em;%
    }%
  }%
  \global\let\HoLogoCss@PiCTeX\relax
}
%    \end{macrocode}
%    \end{macro}
%
% \subsubsection{\hologo{teTeX}}
%
%    \begin{macro}{\HoLogo@teTeX}
%    \begin{macrocode}
\def\HoLogo@teTeX#1{%
  \HOLOGO@mbox{#1{t}{T}e}%
  \HOLOGO@discretionary
  \hologo{TeX}%
}
%    \end{macrocode}
%    \end{macro}
%    \begin{macro}{\HoLogoCs@teTeX}
%    \begin{macrocode}
\def\HoLogoCs@teTeX#1{#1{t}{T}dfTeX}
%    \end{macrocode}
%    \end{macro}
%    \begin{macro}{\HoLogoBkm@teTeX}
%    \begin{macrocode}
\def\HoLogoBkm@teTeX#1{%
  #1{t}{T}e\hologo{TeX}%
}
%    \end{macrocode}
%    \end{macro}
%    \begin{macro}{\HoLogoHtml@teTeX}
%    \begin{macrocode}
\let\HoLogoHtml@teTeX\HoLogo@teTeX
%    \end{macrocode}
%    \end{macro}
%
% \subsubsection{\hologo{TeX4ht}}
%
%    \begin{macro}{\HoLogo@TeX4ht}
%    \begin{macrocode}
\expandafter\def\csname HoLogo@TeX4ht\endcsname#1{%
  \HOLOGO@mbox{\hologo{TeX}4ht}%
}
%    \end{macrocode}
%    \end{macro}
%    \begin{macro}{\HoLogoHtml@TeX4ht}
%    \begin{macrocode}
\expandafter
\let\csname HoLogoHtml@TeX4ht\expandafter\endcsname
\csname HoLogo@TeX4ht\endcsname
%    \end{macrocode}
%    \end{macro}
%
%
% \subsubsection{\hologo{SageTeX}}
%
%    \begin{macro}{\HoLogo@SageTeX}
%    \begin{macrocode}
\def\HoLogo@SageTeX#1{%
  \HOLOGO@mbox{Sage}%
  \HOLOGO@discretionary
  \HOLOGO@NegativeKerning{eT,oT,To}%
  \hologo{TeX}%
}
%    \end{macrocode}
%    \end{macro}
%    \begin{macro}{\HoLogoHtml@SageTeX}
%    \begin{macrocode}
\let\HoLogoHtml@SageTeX\HoLogo@SageTeX
%    \end{macrocode}
%    \end{macro}
%
% \subsection{\hologo{METAFONT} and friends}
%
%    \begin{macro}{\HoLogo@METAFONT}
%    \begin{macrocode}
\def\HoLogo@METAFONT#1{%
  \HoLogoFont@font{METAFONT}{logo}{%
    \HOLOGO@mbox{META}%
    \HOLOGO@discretionary
    \HOLOGO@mbox{FONT}%
  }%
}
%    \end{macrocode}
%    \end{macro}
%
%    \begin{macro}{\HoLogo@METAPOST}
%    \begin{macrocode}
\def\HoLogo@METAPOST#1{%
  \HoLogoFont@font{METAPOST}{logo}{%
    \HOLOGO@mbox{META}%
    \HOLOGO@discretionary
    \HOLOGO@mbox{POST}%
  }%
}
%    \end{macrocode}
%    \end{macro}
%
%    \begin{macro}{\HoLogo@MetaFun}
%    \begin{macrocode}
\def\HoLogo@MetaFun#1{%
  \HOLOGO@mbox{Meta}%
  \HOLOGO@discretionary
  \HOLOGO@mbox{Fun}%
}
%    \end{macrocode}
%    \end{macro}
%
%    \begin{macro}{\HoLogo@MetaPost}
%    \begin{macrocode}
\def\HoLogo@MetaPost#1{%
  \HOLOGO@mbox{Meta}%
  \HOLOGO@discretionary
  \HOLOGO@mbox{Post}%
}
%    \end{macrocode}
%    \end{macro}
%
% \subsection{Others}
%
% \subsubsection{\hologo{biber}}
%
%    \begin{macro}{\HoLogo@biber}
%    \begin{macrocode}
\def\HoLogo@biber#1{%
  \HOLOGO@mbox{#1{b}{B}i}%
  \HOLOGO@discretionary
  \HOLOGO@mbox{ber}%
}
%    \end{macrocode}
%    \end{macro}
%    \begin{macro}{\HoLogoCs@biber}
%    \begin{macrocode}
\def\HoLogoCs@biber#1{#1{b}{B}iber}
%    \end{macrocode}
%    \end{macro}
%    \begin{macro}{\HoLogoBkm@biber}
%    \begin{macrocode}
\def\HoLogoBkm@biber#1{%
  #1{b}{B}iber%
}
%    \end{macrocode}
%    \end{macro}
%    \begin{macro}{\HoLogoHtml@biber}
%    \begin{macrocode}
\let\HoLogoHtml@biber\HoLogo@biber
%    \end{macrocode}
%    \end{macro}
%
% \subsubsection{\hologo{KOMAScript}}
%
%    \begin{macro}{\HoLogo@KOMAScript}
%    The definition for \hologo{KOMAScript} is taken
%    from \hologo{KOMAScript} (\xfile{scrlogo.dtx}, reformatted) \cite{scrlogo}:
%\begin{quote}
%\begin{verbatim}
%\@ifundefined{KOMAScript}{%
%  \DeclareRobustCommand{\KOMAScript}{%
%    \textsf{%
%      K\kern.05em O\kern.05emM\kern.05em A%
%      \kern.1em-\kern.1em %
%      Script%
%    }%
%  }%
%}{}
%\end{verbatim}
%\end{quote}
%    \begin{macrocode}
\def\HoLogo@KOMAScript#1{%
  \HoLogoFont@font{KOMAScript}{sf}{%
    \HOLOGO@mbox{%
      K\kern.05em%
      O\kern.05em%
      M\kern.05em%
      A%
    }%
    \kern.1em%
    \HOLOGO@hyphen
    \kern.1em%
    \HOLOGO@mbox{Script}%
  }%
}
%    \end{macrocode}
%    \end{macro}
%    \begin{macro}{\HoLogoBkm@KOMAScript}
%    \begin{macrocode}
\def\HoLogoBkm@KOMAScript#1{%
  KOMA-Script%
}
%    \end{macrocode}
%    \end{macro}
%    \begin{macro}{\HoLogoHtml@KOMAScript}
%    \begin{macrocode}
\def\HoLogoHtml@KOMAScript#1{%
  \HoLogoCss@KOMAScript
  \HoLogoFont@font{KOMAScript}{sf}{%
    \HOLOGO@Span{KOMAScript}{%
      K%
      \HOLOGO@Span{O}{O}%
      M%
      \HOLOGO@Span{A}{A}%
      \HOLOGO@Span{hyphen}{-}%
      Script%
    }%
  }%
}
%    \end{macrocode}
%    \end{macro}
%    \begin{macro}{\HoLogoCss@KOMAScript}
%    \begin{macrocode}
\def\HoLogoCss@KOMAScript{%
  \Css{%
    span.HoLogo-KOMAScript{%
      font-family:sans-serif;%
    }%
  }%
  \Css{%
    span.HoLogo-KOMAScript span.HoLogo-O{%
      padding-left:.05em;%
      padding-right:.05em;%
    }%
  }%
  \Css{%
    span.HoLogo-KOMAScript span.HoLogo-A{%
      padding-left:.05em;%
    }%
  }%
  \Css{%
    span.HoLogo-KOMAScript span.HoLogo-hyphen{%
      padding-left:.1em;%
      padding-right:.1em;%
    }%
  }%
  \global\let\HoLogoCss@KOMAScript\relax
}
%    \end{macrocode}
%    \end{macro}
%
% \subsubsection{\hologo{LyX}}
%
%    \begin{macro}{\HoLogo@LyX}
%    The definition is taken from the documentation source files
%    of \hologo{LyX}, \xfile{Intro.lyx} \cite{LyX}:
%\begin{quote}
%\begin{verbatim}
%\def\LyX{%
%  \texorpdfstring{%
%    L\kern-.1667em\lower.25em\hbox{Y}\kern-.125emX\@%
%  }{%
%    LyX%
%  }%
%}
%\end{verbatim}
%\end{quote}
%    \begin{macrocode}
\def\HoLogo@LyX#1{%
  L%
  \kern-.1667em%
  \lower.25em\hbox{Y}%
  \kern-.125em%
  X%
  \HOLOGO@SpaceFactor
}
%    \end{macrocode}
%    \end{macro}
%    \begin{macro}{\HoLogoHtml@LyX}
%    \begin{macrocode}
\def\HoLogoHtml@LyX#1{%
  \HoLogoCss@LyX
  \HOLOGO@Span{LyX}{%
    L%
    \HOLOGO@Span{y}{Y}%
    X%
  }%
}
%    \end{macrocode}
%    \end{macro}
%    \begin{macro}{\HoLogoCss@LyX}
%    \begin{macrocode}
\def\HoLogoCss@LyX{%
  \Css{%
    span.HoLogo-LyX span.HoLogo-y{%
      position:relative;%
      top:.25em;%
      margin-left:-.1667em;%
      margin-right:-.125em;%
      text-decoration:none;%
    }%
  }%
  \global\let\HoLogoCss@LyX\relax
}
%    \end{macrocode}
%    \end{macro}
%
% \subsubsection{\hologo{NTS}}
%
%    \begin{macro}{\HoLogo@NTS}
%    Definition for \hologo{NTS} can be found in
%    package \xpackage{etex\textunderscore man} for the \hologo{eTeX} manual \cite{etexman}
%    and in package \xpackage{dtklogos} \cite{dtklogos}:
%\begin{quote}
%\begin{verbatim}
%\def\NTS{%
%  \leavevmode
%  \hbox{%
%    $%
%      \cal N%
%      \kern-0.35em%
%      \lower0.5ex\hbox{$\cal T$}%
%      \kern-0.2em%
%      S%
%    $%
%  }%
%}
%\end{verbatim}
%\end{quote}
%    \begin{macrocode}
\def\HoLogo@NTS#1{%
  \HoLogoFont@font{NTS}{sy}{%
    N\/%
    \kern-.35em%
    \lower.5ex\hbox{T\/}%
    \kern-.2em%
    S\/%
  }%
  \HOLOGO@SpaceFactor
}
%    \end{macrocode}
%    \end{macro}
%
% \subsubsection{\Hologo{TTH} (\hologo{TeX} to HTML translator)}
%
%    Source: \url{http://hutchinson.belmont.ma.us/tth/}
%    In the HTML source the second `T' is printed as subscript.
%\begin{quote}
%\begin{verbatim}
%T<sub>T</sub>H
%\end{verbatim}
%\end{quote}
%    \begin{macro}{\HoLogo@TTH}
%    \begin{macrocode}
\def\HoLogo@TTH#1{%
  \ltx@mbox{%
    T\HOLOGO@SubScript{T}H%
  }%
  \HOLOGO@SpaceFactor
}
%    \end{macrocode}
%    \end{macro}
%
%    \begin{macro}{\HoLogoHtml@TTH}
%    \begin{macrocode}
\def\HoLogoHtml@TTH#1{%
  T\HCode{<sub>}T\HCode{</sub>}H%
}
%    \end{macrocode}
%    \end{macro}
%
% \subsubsection{\Hologo{HanTheThanh}}
%
%    Partial source: Package \xpackage{dtklogos}.
%    The double accent is U+1EBF (latin small letter e with circumflex
%    and acute).
%    \begin{macro}{\HoLogo@HanTheThanh}
%    \begin{macrocode}
\def\HoLogo@HanTheThanh#1{%
  \ltx@mbox{H\`an}%
  \HOLOGO@space
  \ltx@mbox{%
    Th%
    \HOLOGO@IfCharExists{"1EBF}{%
      \char"1EBF\relax
    }{%
      \^e\hbox to 0pt{\hss\raise .5ex\hbox{\'{}}}%
    }%
  }%
  \HOLOGO@space
  \ltx@mbox{Th\`anh}%
}
%    \end{macrocode}
%    \end{macro}
%    \begin{macro}{\HoLogoBkm@HanTheThanh}
%    \begin{macrocode}
\def\HoLogoBkm@HanTheThanh#1{%
  H\`an %
  Th\HOLOGO@PdfdocUnicode{\^e}{\9036\277} %
  Th\`anh%
}
%    \end{macrocode}
%    \end{macro}
%    \begin{macro}{\HoLogoHtml@HanTheThanh}
%    \begin{macrocode}
\def\HoLogoHtml@HanTheThanh#1{%
  H\`an %
  Th\HCode{&\ltx@hashchar x1ebf;} %
  Th\`anh%
}
%    \end{macrocode}
%    \end{macro}
%
% \subsection{Driver detection}
%
%    \begin{macrocode}
\HOLOGO@IfExists\InputIfFileExists{%
  \InputIfFileExists{hologo.cfg}{}{}%
}{%
  \ltx@IfUndefined{pdf@filesize}{%
    \def\HOLOGO@InputIfExists{%
      \openin\HOLOGO@temp=hologo.cfg\relax
      \ifeof\HOLOGO@temp
        \closein\HOLOGO@temp
      \else
        \closein\HOLOGO@temp
        \begingroup
          \def\x{LaTeX2e}%
        \expandafter\endgroup
        \ifx\fmtname\x
          \input{hologo.cfg}%
        \else
          \input hologo.cfg\relax
        \fi
      \fi
    }%
    \ltx@IfUndefined{newread}{%
      \chardef\HOLOGO@temp=15 %
      \def\HOLOGO@CheckRead{%
        \ifeof\HOLOGO@temp
          \HOLOGO@InputIfExists
        \else
          \ifcase\HOLOGO@temp
            \@PackageWarningNoLine{hologo}{%
              Configuration file ignored, because\MessageBreak
              a free read register could not be found%
            }%
          \else
            \begingroup
              \count\ltx@cclv=\HOLOGO@temp
              \advance\ltx@cclv by \ltx@minusone
              \edef\x{\endgroup
                \chardef\noexpand\HOLOGO@temp=\the\count\ltx@cclv
                \relax
              }%
            \x
          \fi
        \fi
      }%
    }{%
      \csname newread\endcsname\HOLOGO@temp
      \HOLOGO@InputIfExists
    }%
  }{%
    \edef\HOLOGO@temp{\pdf@filesize{hologo.cfg}}%
    \ifx\HOLOGO@temp\ltx@empty
    \else
      \ifnum\HOLOGO@temp>0 %
        \begingroup
          \def\x{LaTeX2e}%
        \expandafter\endgroup
        \ifx\fmtname\x
          \input{hologo.cfg}%
        \else
          \input hologo.cfg\relax
        \fi
      \else
        \@PackageInfoNoLine{hologo}{%
          Empty configuration file `hologo.cfg' ignored%
        }%
      \fi
    \fi
  }%
}
%    \end{macrocode}
%
%    \begin{macrocode}
\def\HOLOGO@temp#1#2{%
  \kv@define@key{HoLogoDriver}{#1}[]{%
    \begingroup
      \def\HOLOGO@temp{##1}%
      \ltx@onelevel@sanitize\HOLOGO@temp
      \ifx\HOLOGO@temp\ltx@empty
      \else
        \@PackageError{hologo}{%
          Value (\HOLOGO@temp) not permitted for option `#1'%
        }%
        \@ehc
      \fi
    \endgroup
    \def\hologoDriver{#2}%
  }%
}%
\def\HOLOGO@@temp#1#2{%
  \ifx\kv@value\relax
    \HOLOGO@temp{#1}{#1}%
  \else
    \HOLOGO@temp{#1}{#2}%
  \fi
}%
\kv@parse@normalized{%
  pdftex,%
  luatex=pdftex,%
  dvipdfm,%
  dvipdfmx=dvipdfm,%
  dvips,%
  dvipsone=dvips,%
  xdvi=dvips,%
  xetex,%
  vtex,%
}\HOLOGO@@temp
%    \end{macrocode}
%
%    \begin{macrocode}
\kv@define@key{HoLogoDriver}{driverfallback}{%
  \def\HOLOGO@DriverFallback{#1}%
}
%    \end{macrocode}
%
%    \begin{macro}{\HOLOGO@DriverFallback}
%    \begin{macrocode}
\def\HOLOGO@DriverFallback{dvips}
%    \end{macrocode}
%    \end{macro}
%
%    \begin{macro}{\hologoDriverSetup}
%    \begin{macrocode}
\def\hologoDriverSetup{%
  \let\hologoDriver\ltx@undefined
  \HOLOGO@DriverSetup
}
%    \end{macrocode}
%    \end{macro}
%
%    \begin{macro}{\HOLOGO@DriverSetup}
%    \begin{macrocode}
\def\HOLOGO@DriverSetup#1{%
  \kvsetkeys{HoLogoDriver}{#1}%
  \HOLOGO@CheckDriver
  \ltx@ifundefined{hologoDriver}{%
    \begingroup
    \edef\x{\endgroup
      \noexpand\kvsetkeys{HoLogoDriver}{\HOLOGO@DriverFallback}%
    }\x
  }{}%
  \@PackageInfoNoLine{hologo}{Using driver `\hologoDriver'}%
}
%    \end{macrocode}
%    \end{macro}
%
%    \begin{macro}{\HOLOGO@CheckDriver}
%    \begin{macrocode}
\def\HOLOGO@CheckDriver{%
  \ifpdf
    \def\hologoDriver{pdftex}%
    \let\HOLOGO@pdfliteral\pdfliteral
    \ifluatex
      \ifx\pdfextension\@undefined\else
        \protected\def\pdfliteral{\pdfextension literal}%
        \let\HOLOGO@pdfliteral\pdfliteral
      \fi
      \ltx@IfUndefined{HOLOGO@pdfliteral}{%
        \ifnum\luatexversion<36 %
        \else
          \begingroup
            \let\HOLOGO@temp\endgroup
            \ifcase0%
                \directlua{%
                  if tex.enableprimitives then %
                    tex.enableprimitives('HOLOGO@', {'pdfliteral'})%
                  else %
                    tex.print('1')%
                  end%
                }%
                \ifx\HOLOGO@pdfliteral\@undefined 1\fi%
                \relax%
              \endgroup
              \let\HOLOGO@temp\relax
              \global\let\HOLOGO@pdfliteral\HOLOGO@pdfliteral
            \fi%
          \HOLOGO@temp
        \fi
      }{}%
    \fi
    \ltx@IfUndefined{HOLOGO@pdfliteral}{%
      \@PackageWarningNoLine{hologo}{%
        Cannot find \string\pdfliteral
      }%
    }{}%
  \else
    \ifxetex
      \def\hologoDriver{xetex}%
    \else
      \ifvtex
        \def\hologoDriver{vtex}%
      \fi
    \fi
  \fi
}
%    \end{macrocode}
%    \end{macro}
%
%    \begin{macro}{\HOLOGO@WarningUnsupportedDriver}
%    \begin{macrocode}
\def\HOLOGO@WarningUnsupportedDriver#1{%
  \@PackageWarningNoLine{hologo}{%
    Logo `#1' needs driver specific macros,\MessageBreak
    but driver `\hologoDriver' is not supported.\MessageBreak
    Use a different driver or\MessageBreak
    load package `graphics' or `pgf'%
  }%
}
%    \end{macrocode}
%    \end{macro}
%
% \subsubsection{Reflect box macros}
%
%    Skip driver part if not needed.
%    \begin{macrocode}
\ltx@IfUndefined{reflectbox}{}{%
  \ltx@IfUndefined{rotatebox}{}{%
    \HOLOGO@AtEnd
  }%
}
\ltx@IfUndefined{pgftext}{}{%
  \HOLOGO@AtEnd
}
\ltx@IfUndefined{psscalebox}{}{%
  \HOLOGO@AtEnd
}
%    \end{macrocode}
%
%    \begin{macrocode}
\def\HOLOGO@temp{LaTeX2e}
\ifx\fmtname\HOLOGO@temp
  \RequirePackage{kvoptions}[2011/06/30]%
  \ProcessKeyvalOptions{HoLogoDriver}%
\fi
\HOLOGO@DriverSetup{}
%    \end{macrocode}
%
%    \begin{macro}{\HOLOGO@ReflectBox}
%    \begin{macrocode}
\def\HOLOGO@ReflectBox#1{%
  \begingroup
    \setbox\ltx@zero\hbox{\begingroup#1\endgroup}%
    \setbox\ltx@two\hbox{%
      \kern\wd\ltx@zero
      \csname HOLOGO@ScaleBox@\hologoDriver\endcsname{-1}{1}{%
        \hbox to 0pt{\copy\ltx@zero\hss}%
      }%
    }%
    \wd\ltx@two=\wd\ltx@zero
    \box\ltx@two
  \endgroup
}
%    \end{macrocode}
%    \end{macro}
%
%    \begin{macro}{\HOLOGO@PointReflectBox}
%    \begin{macrocode}
\def\HOLOGO@PointReflectBox#1{%
  \begingroup
    \setbox\ltx@zero\hbox{\begingroup#1\endgroup}%
    \setbox\ltx@two\hbox{%
      \kern\wd\ltx@zero
      \raise\ht\ltx@zero\hbox{%
        \csname HOLOGO@ScaleBox@\hologoDriver\endcsname{-1}{-1}{%
          \hbox to 0pt{\copy\ltx@zero\hss}%
        }%
      }%
    }%
    \wd\ltx@two=\wd\ltx@zero
    \box\ltx@two
  \endgroup
}
%    \end{macrocode}
%    \end{macro}
%
%    We must define all variants because of dynamic driver setup.
%    \begin{macrocode}
\def\HOLOGO@temp#1#2{#2}
%    \end{macrocode}
%
%    \begin{macro}{\HOLOGO@ScaleBox@pdftex}
%    \begin{macrocode}
\HOLOGO@temp{pdftex}{%
  \def\HOLOGO@ScaleBox@pdftex#1#2#3{%
    \HOLOGO@pdfliteral{%
      q #1 0 0 #2 0 0 cm%
    }%
    #3%
    \HOLOGO@pdfliteral{%
      Q%
    }%
  }%
}
%    \end{macrocode}
%    \end{macro}
%    \begin{macro}{\HOLOGO@ScaleBox@dvips}
%    \begin{macrocode}
\HOLOGO@temp{dvips}{%
  \def\HOLOGO@ScaleBox@dvips#1#2#3{%
    \special{ps:%
      gsave %
      currentpoint %
      currentpoint translate %
      #1 #2 scale %
      neg exch neg exch translate%
    }%
    #3%
    \special{ps:%
      currentpoint %
      grestore %
      moveto%
    }%
  }%
}
%    \end{macrocode}
%    \end{macro}
%    \begin{macro}{\HOLOGO@ScaleBox@dvipdfm}
%    \begin{macrocode}
\HOLOGO@temp{dvipdfm}{%
  \let\HOLOGO@ScaleBox@dvipdfm\HOLOGO@ScaleBox@dvips
}
%    \end{macrocode}
%    \end{macro}
%    Since \hologo{XeTeX} v0.6.
%    \begin{macro}{\HOLOGO@ScaleBox@xetex}
%    \begin{macrocode}
\HOLOGO@temp{xetex}{%
  \def\HOLOGO@ScaleBox@xetex#1#2#3{%
    \special{x:gsave}%
    \special{x:scale #1 #2}%
    #3%
    \special{x:grestore}%
  }%
}
%    \end{macrocode}
%    \end{macro}
%    \begin{macro}{\HOLOGO@ScaleBox@vtex}
%    \begin{macrocode}
\HOLOGO@temp{vtex}{%
  \def\HOLOGO@ScaleBox@vtex#1#2#3{%
    \special{r(#1,0,0,#2,0,0}%
    #3%
    \special{r)}%
  }%
}
%    \end{macrocode}
%    \end{macro}
%
%    \begin{macrocode}
\HOLOGO@AtEnd%
%</package>
%    \end{macrocode}
%
% \section{Test}
%
% \subsection{Catcode checks for loading}
%
%    \begin{macrocode}
%<*test1>
%    \end{macrocode}
%    \begin{macrocode}
\catcode`\{=1 %
\catcode`\}=2 %
\catcode`\#=6 %
\catcode`\@=11 %
\expandafter\ifx\csname count@\endcsname\relax
  \countdef\count@=255 %
\fi
\expandafter\ifx\csname @gobble\endcsname\relax
  \long\def\@gobble#1{}%
\fi
\expandafter\ifx\csname @firstofone\endcsname\relax
  \long\def\@firstofone#1{#1}%
\fi
\expandafter\ifx\csname loop\endcsname\relax
  \expandafter\@firstofone
\else
  \expandafter\@gobble
\fi
{%
  \def\loop#1\repeat{%
    \def\body{#1}%
    \iterate
  }%
  \def\iterate{%
    \body
      \let\next\iterate
    \else
      \let\next\relax
    \fi
    \next
  }%
  \let\repeat=\fi
}%
\def\RestoreCatcodes{}
\count@=0 %
\loop
  \edef\RestoreCatcodes{%
    \RestoreCatcodes
    \catcode\the\count@=\the\catcode\count@\relax
  }%
\ifnum\count@<255 %
  \advance\count@ 1 %
\repeat

\def\RangeCatcodeInvalid#1#2{%
  \count@=#1\relax
  \loop
    \catcode\count@=15 %
  \ifnum\count@<#2\relax
    \advance\count@ 1 %
  \repeat
}
\def\RangeCatcodeCheck#1#2#3{%
  \count@=#1\relax
  \loop
    \ifnum#3=\catcode\count@
    \else
      \errmessage{%
        Character \the\count@\space
        with wrong catcode \the\catcode\count@\space
        instead of \number#3%
      }%
    \fi
  \ifnum\count@<#2\relax
    \advance\count@ 1 %
  \repeat
}
\def\space{ }
\expandafter\ifx\csname LoadCommand\endcsname\relax
  \def\LoadCommand{\input hologo.sty\relax}%
\fi
\def\Test{%
  \RangeCatcodeInvalid{0}{47}%
  \RangeCatcodeInvalid{58}{64}%
  \RangeCatcodeInvalid{91}{96}%
  \RangeCatcodeInvalid{123}{255}%
  \catcode`\@=12 %
  \catcode`\\=0 %
  \catcode`\%=14 %
  \LoadCommand
  \RangeCatcodeCheck{0}{36}{15}%
  \RangeCatcodeCheck{37}{37}{14}%
  \RangeCatcodeCheck{38}{47}{15}%
  \RangeCatcodeCheck{48}{57}{12}%
  \RangeCatcodeCheck{58}{63}{15}%
  \RangeCatcodeCheck{64}{64}{12}%
  \RangeCatcodeCheck{65}{90}{11}%
  \RangeCatcodeCheck{91}{91}{15}%
  \RangeCatcodeCheck{92}{92}{0}%
  \RangeCatcodeCheck{93}{96}{15}%
  \RangeCatcodeCheck{97}{122}{11}%
  \RangeCatcodeCheck{123}{255}{15}%
  \RestoreCatcodes
}
\Test
\csname @@end\endcsname
\end
%    \end{macrocode}
%    \begin{macrocode}
%</test1>
%    \end{macrocode}
%
% \subsection{Spacefactor}
%
%    The space factor must be 1000 after a logo. If it is greater 1000
%    then the following space is a space after a sentence closing point.
%    If the space factor is smaller 1000 then an immediate following
%    dot is interpreted as abbreviation, not sentence closing point.
%
%    \begin{macrocode}
%<*test-spacefactor>
\NeedsTeXFormat{LaTeX2e}
\documentclass{article}
\usepackage{hologo}[2016/05/12]
\usepackage{kvsetkeys}
\usepackage{qstest}
\IncludeTests{*}
\LogTests{log}{*}{*}
\begin{document}
\begin{qstest}{spacefactor}{spacefactor}
\newcommand*{\Test}[1]{%
  \sbox0{%
    \hologo{#1}%
    \Expect*{1000 (#1)}*{\the\spacefactor\space(#1)}%
  }%
}%
\makeatletter
\def\TestList{}
\def\hologoEntry#1#2#3{%
  \edef\TestList{%
    \ifx\TestList\@empty
    \else
      \TestList,%
    \fi
    #1%
    \ifx\\#2\\%
    \else
      ={variant=#2}%
    \fi
  }%
}
\hologoList
\expandafter\kv@parse@normalized\expandafter{%
  \TestList
}{%
  \begingroup
    \let\@logo=\kv@key
    \ifx\kv@value\relax
    \else
      \expandafter\hologoLogoSetup\expandafter\@logo\expandafter{%
        \kv@value
      }%
    \fi
    \Test\@logo
  \endgroup
  \@gobbletwo
}
\end{qstest}
\end{document}
%</test-spacefactor>
%    \end{macrocode}
%
% \subsection{Complete list}
%
%    \begin{macrocode}
%<*test-list>
\NeedsTeXFormat{LaTeX2e}
\documentclass[12pt,a4paper]{article}
\usepackage{hologo}[2016/05/12]
\usepackage[T1]{fontenc}
\usepackage{lmodern}
\usepackage{parskip}
\usepackage[unicode]{hyperref}[2011/09/28]
\usepackage{bookmark}[2011/09/19]
\bookmarksetup{%
  numbered,%
  open,%
  openlevel=2,%
}
\renewcommand*{\contentsname}{List of logos}
\begin{document}
\tableofcontents
\def\TestFont#1#2#3#4#5#6{%
  \begingroup
    \usefont{#3}{#4}{#5}{#6}%
    \HologoVariant{#1}{#2}/\hologoVariant{#1}{#2}%
    \quad
    \begingroup\scriptsize\hologoVariant{#1}{#2}\endgroup
    \quad
  \endgroup
  (#3/#4/#5/#6)%
  \par
}
\makeatletter
\def\hologoEntry#1#2#3{%
  \section{%
    \HologoVariant{#1}{#2}/\hologoVariant{#1}{#2} %
    {[#1\ifx\\#2\\\else\space(#2)\fi]}% hash-ok
  }% braces around [] because of bug in tex4ht
  \begingroup
    \hypersetup{unicode=false}%
    \bookmark[%
      dest=\@currentHref,%
      rellevel=1,%
      keeplevel,%
    ]{%
      \HologoVariant{#1}{#2}/\hologoVariant{#1}{#2} %
      (PDFDocEncoding)%
    }%
  \endgroup
  \TestFont{#1}{#2}{OT1}{cmr}{m}{n}%
  \TestFont{#1}{#2}{OT1}{cmss}{m}{n}%
  \TestFont{#1}{#2}{OT1}{cmr}{b}{n}%
  \TestFont{#1}{#2}{OT1}{cmr}{m}{it}%
  \TestFont{#1}{#2}{OT1}{cmtt}{m}{n}%
  \TestFont{#1}{#2}{T1}{lmr}{m}{n}%
  \TestFont{#1}{#2}{T1}{lmss}{m}{n}%
  \TestFont{#1}{#2}{T1}{lmr}{b}{n}%
  \TestFont{#1}{#2}{T1}{lmr}{m}{it}%
  \TestFont{#1}{#2}{T1}{lmtt}{m}{n}%
  \TestFont{#1}{#2}{T1}{lmvtt}{m}{n}%
  \TestFont{#1}{#2}{T1}{qtm}{m}{n}%
  \TestFont{#1}{#2}{T1}{qhv}{m}{n}%
  \TestFont{#1}{#2}{T1}{qtm}{b}{n}%
  \TestFont{#1}{#2}{T1}{qtm}{m}{it}%
  \TestFont{#1}{#2}{T1}{qcr}{m}{n}%
  \newpage
}
\makeatother
\hologoList
\end{document}
%</test-list>
%    \end{macrocode}
%
% \section{Installation}
%
% \subsection{Download}
%
% \paragraph{Package.} This package is available on
% CTAN\footnote{\url{ftp://ftp.ctan.org/tex-archive/}}:
% \begin{description}
% \item[\CTAN{macros/latex/contrib/oberdiek/hologo.dtx}] The source file.
% \item[\CTAN{macros/latex/contrib/oberdiek/hologo.pdf}] Documentation.
% \end{description}
%
%
% \paragraph{Bundle.} All the packages of the bundle `oberdiek'
% are also available in a TDS compliant ZIP archive. There
% the packages are already unpacked and the documentation files
% are generated. The files and directories obey the TDS standard.
% \begin{description}
% \item[\CTAN{install/macros/latex/contrib/oberdiek.tds.zip}]
% \end{description}
% \emph{TDS} refers to the standard ``A Directory Structure
% for \TeX\ Files'' (\CTAN{tds/tds.pdf}). Directories
% with \xfile{texmf} in their name are usually organized this way.
%
% \subsection{Bundle installation}
%
% \paragraph{Unpacking.} Unpack the \xfile{oberdiek.tds.zip} in the
% TDS tree (also known as \xfile{texmf} tree) of your choice.
% Example (linux):
% \begin{quote}
%   |unzip oberdiek.tds.zip -d ~/texmf|
% \end{quote}
%
% \paragraph{Script installation.}
% Check the directory \xfile{TDS:scripts/oberdiek/} for
% scripts that need further installation steps.
% Package \xpackage{attachfile2} comes with the Perl script
% \xfile{pdfatfi.pl} that should be installed in such a way
% that it can be called as \texttt{pdfatfi}.
% Example (linux):
% \begin{quote}
%   |chmod +x scripts/oberdiek/pdfatfi.pl|\\
%   |cp scripts/oberdiek/pdfatfi.pl /usr/local/bin/|
% \end{quote}
%
% \subsection{Package installation}
%
% \paragraph{Unpacking.} The \xfile{.dtx} file is a self-extracting
% \docstrip\ archive. The files are extracted by running the
% \xfile{.dtx} through \plainTeX:
% \begin{quote}
%   \verb|tex hologo.dtx|
% \end{quote}
%
% \paragraph{TDS.} Now the different files must be moved into
% the different directories in your installation TDS tree
% (also known as \xfile{texmf} tree):
% \begin{quote}
% \def\t{^^A
% \begin{tabular}{@{}>{\ttfamily}l@{ $\rightarrow$ }>{\ttfamily}l@{}}
%   hologo.sty & tex/generic/oberdiek/hologo.sty\\
%   hologo.pdf & doc/latex/oberdiek/hologo.pdf\\
%   example/hologo-example.tex & doc/latex/oberdiek/example/hologo-example.tex\\
%   test/hologo-test1.tex & doc/latex/oberdiek/test/hologo-test1.tex\\
%   test/hologo-test-spacefactor.tex & doc/latex/oberdiek/test/hologo-test-spacefactor.tex\\
%   test/hologo-test-list.tex & doc/latex/oberdiek/test/hologo-test-list.tex\\
%   hologo.dtx & source/latex/oberdiek/hologo.dtx\\
% \end{tabular}^^A
% }^^A
% \sbox0{\t}^^A
% \ifdim\wd0>\linewidth
%   \begingroup
%     \advance\linewidth by\leftmargin
%     \advance\linewidth by\rightmargin
%   \edef\x{\endgroup
%     \def\noexpand\lw{\the\linewidth}^^A
%   }\x
%   \def\lwbox{^^A
%     \leavevmode
%     \hbox to \linewidth{^^A
%       \kern-\leftmargin\relax
%       \hss
%       \usebox0
%       \hss
%       \kern-\rightmargin\relax
%     }^^A
%   }^^A
%   \ifdim\wd0>\lw
%     \sbox0{\small\t}^^A
%     \ifdim\wd0>\linewidth
%       \ifdim\wd0>\lw
%         \sbox0{\footnotesize\t}^^A
%         \ifdim\wd0>\linewidth
%           \ifdim\wd0>\lw
%             \sbox0{\scriptsize\t}^^A
%             \ifdim\wd0>\linewidth
%               \ifdim\wd0>\lw
%                 \sbox0{\tiny\t}^^A
%                 \ifdim\wd0>\linewidth
%                   \lwbox
%                 \else
%                   \usebox0
%                 \fi
%               \else
%                 \lwbox
%               \fi
%             \else
%               \usebox0
%             \fi
%           \else
%             \lwbox
%           \fi
%         \else
%           \usebox0
%         \fi
%       \else
%         \lwbox
%       \fi
%     \else
%       \usebox0
%     \fi
%   \else
%     \lwbox
%   \fi
% \else
%   \usebox0
% \fi
% \end{quote}
% If you have a \xfile{docstrip.cfg} that configures and enables \docstrip's
% TDS installing feature, then some files can already be in the right
% place, see the documentation of \docstrip.
%
% \subsection{Refresh file name databases}
%
% If your \TeX~distribution
% (\teTeX, \mikTeX, \dots) relies on file name databases, you must refresh
% these. For example, \teTeX\ users run \verb|texhash| or
% \verb|mktexlsr|.
%
% \subsection{Some details for the interested}
%
% \paragraph{Attached source.}
%
% The PDF documentation on CTAN also includes the
% \xfile{.dtx} source file. It can be extracted by
% AcrobatReader 6 or higher. Another option is \textsf{pdftk},
% e.g. unpack the file into the current directory:
% \begin{quote}
%   \verb|pdftk hologo.pdf unpack_files output .|
% \end{quote}
%
% \paragraph{Unpacking with \LaTeX.}
% The \xfile{.dtx} chooses its action depending on the format:
% \begin{description}
% \item[\plainTeX:] Run \docstrip\ and extract the files.
% \item[\LaTeX:] Generate the documentation.
% \end{description}
% If you insist on using \LaTeX\ for \docstrip\ (really,
% \docstrip\ does not need \LaTeX), then inform the autodetect routine
% about your intention:
% \begin{quote}
%   \verb|latex \let\install=y\input{hologo.dtx}|
% \end{quote}
% Do not forget to quote the argument according to the demands
% of your shell.
%
% \paragraph{Generating the documentation.}
% You can use both the \xfile{.dtx} or the \xfile{.drv} to generate
% the documentation. The process can be configured by the
% configuration file \xfile{ltxdoc.cfg}. For instance, put this
% line into this file, if you want to have A4 as paper format:
% \begin{quote}
%   \verb|\PassOptionsToClass{a4paper}{article}|
% \end{quote}
% An example follows how to generate the
% documentation with pdf\LaTeX:
% \begin{quote}
%\begin{verbatim}
%pdflatex hologo.dtx
%makeindex -s gind.ist hologo.idx
%pdflatex hologo.dtx
%makeindex -s gind.ist hologo.idx
%pdflatex hologo.dtx
%\end{verbatim}
% \end{quote}
%
% \section{Catalogue}
%
% The following XML file can be used as source for the
% \href{http://mirror.ctan.org/help/Catalogue/catalogue.html}{\TeX\ Catalogue}.
% The elements \texttt{caption} and \texttt{description} are imported
% from the original XML file from the Catalogue.
% The name of the XML file in the Catalogue is \xfile{hologo.xml}.
%    \begin{macrocode}
%<*catalogue>
<?xml version='1.0' encoding='us-ascii'?>
<!DOCTYPE entry SYSTEM 'catalogue.dtd'>
<entry datestamp='$Date$' modifier='$Author$' id='hologo'>
  <name>hologo</name>
  <caption>A collection of logos with bookmark support.</caption>
  <authorref id='auth:oberdiek'/>
  <copyright owner='Heiko Oberdiek' year='2010-2012'/>
  <license type='lppl1.3'/>
  <version number='1.10'/>
  <description>
    The package defines a single command <tt>\hologo</tt>, whose
    argument is the usual case-confused ASCII version of the logo.
    The command is bookmark-enabled, so that every logo becomes
    available in bookmarks without further work.
    <p/>
    The package is part of the <xref refid='oberdiek'>oberdiek</xref>
    bundle.
  </description>
  <documentation details='Package documentation'
      href='ctan:/macros/latex/contrib/oberdiek/hologo.pdf'/>
  <ctan file='true' path='/macros/latex/contrib/oberdiek/hologo.dtx'/>
  <miktex location='oberdiek'/>
  <texlive location='oberdiek'/>
  <install path='/macros/latex/contrib/oberdiek/oberdiek.tds.zip'/>
</entry>
%</catalogue>
%    \end{macrocode}
%
% \begin{thebibliography}{9}
% \raggedright
%
% \bibitem{btxdoc}
% Oren Patashnik,
% \textit{\hologo{BibTeX}ing},
% 1988-02-08.\\
% \CTAN{biblio/bibtex/base/}
%
% \bibitem{dtklogos}
% Gerd Neugebauer, DANTE,
% \textit{Package \xpackage{dtklogos}},
% 2011-04-25.\\
% \CTAN{usergrps/dante/dtk/dtklogos.sty}
%
% \bibitem{etexman}
% The \hologo{NTS} Team,
% \textit{The \hologo{eTeX} manual},
% 1998-02.\\
% \CTAN{systems/e-tex/v2/doc/}
%
% \bibitem{ExTeX-FAQ}
% The \hologo{ExTeX} group,
% \textit{\hologo{ExTeX}: FAQ -- How is \hologo{ExTeX} typeset?},
% 2007-04-14.\\
% \url{http://www.extex.org/documentation/faq.html}
%
% \bibitem{LyX}
% %@MISC{ LyX,
% %  title = {{LyX 2.0.0 -- The Document Processor [Computer software and manual]}},
% %  author = {{The LyX Team}},
% %  howpublished = {Internet: http://www.lyx.org},
% %  year = {2011-05-08},
% %  note = {Retrieved May 10, 2011, from http://www.lyx.org},
% %  url = {http://www.lyx.org/}
% %}
% The \hologo{LyX} Team,
% \textit{\hologo{LyX} -- The Document Processor},
% 2011-05-08.\\
% \url{http://www.lyx.org/}
%
% \bibitem{OzTeX}
% Andrew Trevorrow,
% \hologo{OzTeX} FAQ: What is the correct way to typeset ``\hologo{OzTeX}''?,
% 2011-09-15 (visited).
% \url{http://www.trevorrow.com/oztex/ozfaq.html#oztex-logo}
%
% \bibitem{PiCTeX}
% Michael Wichura,
% \textit{The \hologo{PiCTeX} macro package},
% 1987-09-21.
% \CTAN{graphics/pictex/}
%
% \bibitem{scrlogo}
% Markus Kohm,
% \textit{\hologo{KOMAScript} Datei \xfile{scrlogo.dtx}},
% 2009-01-30.\\
% \CTAN{install/macros/latex/contrib/komascript.tds.zip}
%
% \end{thebibliography}
%
% \begin{History}
%   \begin{Version}{2010/04/08 v1.0}
%   \item
%     The first version.
%   \end{Version}
%   \begin{Version}{2010/04/16 v1.1}
%   \item
%     \cs{Hologo} added for support of logos at start of a sentence.
%   \item
%     \cs{hologoSetup} and \cs{hologoLogoSetup} added.
%   \item
%     Options \xoption{break}, \xoption{hyphenbreak}, \xoption{spacebreak}
%     added.
%   \item
%     Variant support added by option \xoption{variant}.
%   \end{Version}
%   \begin{Version}{2010/04/24 v1.2}
%   \item
%     \hologo{LaTeX3} added.
%   \item
%     \hologo{VTeX} added.
%   \end{Version}
%   \begin{Version}{2010/11/21 v1.3}
%   \item
%     \hologo{iniTeX}, \hologo{virTeX} added.
%   \end{Version}
%   \begin{Version}{2011/03/25 v1.4}
%   \item
%     \hologo{ConTeXt} with variants added.
%   \item
%     Option \xoption{discretionarybreak} added as refinement for
%     option \xoption{break}.
%   \end{Version}
%   \begin{Version}{2011/04/21 v1.5}
%   \item
%     Wrong TDS directory for test files fixed.
%   \end{Version}
%   \begin{Version}{2011/10/01 v1.6}
%   \item
%     Support for package \xpackage{tex4ht} added.
%   \item
%     Support for \cs{csname} added if \cs{ifincsname} is available.
%   \item
%     New logos:
%     \hologo{(La)TeX},
%     \hologo{biber},
%     \hologo{BibTeX} (\xoption{sc}, \xoption{sf}),
%     \hologo{emTeX},
%     \hologo{ExTeX},
%     \hologo{KOMAScript},
%     \hologo{La},
%     \hologo{LyX},
%     \hologo{MiKTeX},
%     \hologo{NTS},
%     \hologo{OzMF},
%     \hologo{OzMP},
%     \hologo{OzTeX},
%     \hologo{OzTtH},
%     \hologo{PCTeX},
%     \hologo{PiC},
%     \hologo{PiCTeX},
%     \hologo{METAFONT},
%     \hologo{MetaFun},
%     \hologo{METAPOST},
%     \hologo{MetaPost},
%     \hologo{SLiTeX} (\xoption{lift}, \xoption{narrow}, \xoption{simple}),
%     \hologo{SliTeX} (\xoption{narrow}, \xoption{simple}, \xoption{lift}),
%     \hologo{teTeX}.
%   \item
%     Fixes:
%     \hologo{iniTeX},
%     \hologo{pdfLaTeX},
%     \hologo{pdfTeX},
%     \hologo{virTeX}.
%   \item
%     \cs{hologoFontSetup} and \cs{hologoLogoFontSetup} added.
%   \item
%     \cs{hologoVariant} and \cs{HologoVariant} added.
%   \end{Version}
%   \begin{Version}{2011/11/22 v1.7}
%   \item
%     New logos:
%     \hologo{BibTeX8},
%     \hologo{LaTeXML},
%     \hologo{SageTeX},
%     \hologo{TeX4ht},
%     \hologo{TTH}.
%   \item
%     \hologo{Xe} and friends: Driver stuff fixed.
%   \item
%     \hologo{Xe} and friends: Support for italic added.
%   \item
%     \hologo{Xe} and friends: Package support for \xpackage{pgf}
%     and \xpackage{pstricks} added.
%   \end{Version}
%   \begin{Version}{2011/11/29 v1.8}
%   \item
%     New logos:
%     \hologo{HanTheThanh}.
%   \end{Version}
%   \begin{Version}{2011/12/21 v1.9}
%   \item
%     Patch for package \xpackage{ifxetex} added for the case that
%     \cs{newif} is undefined in \hologo{iniTeX}.
%   \item
%     Some fixes for \hologo{iniTeX}.
%   \end{Version}
%   \begin{Version}{2012/04/26 v1.10}
%   \item
%     Fix in bookmark version of logo ``\hologo{HanTheThanh}''.
%   \end{Version}
%   \begin{Version}{2016/05/12 v1.11}
%   \item
%     Update HOLOGO@IfCharExists (previously in texlive)
%   \item define pdfliteral in current luatex.
%   \end{Version}
% \end{History}
%
% \PrintIndex
%
% \Finale
\endinput
|
% \end{quote}
% Do not forget to quote the argument according to the demands
% of your shell.
%
% \paragraph{Generating the documentation.}
% You can use both the \xfile{.dtx} or the \xfile{.drv} to generate
% the documentation. The process can be configured by the
% configuration file \xfile{ltxdoc.cfg}. For instance, put this
% line into this file, if you want to have A4 as paper format:
% \begin{quote}
%   \verb|\PassOptionsToClass{a4paper}{article}|
% \end{quote}
% An example follows how to generate the
% documentation with pdf\LaTeX:
% \begin{quote}
%\begin{verbatim}
%pdflatex hologo.dtx
%makeindex -s gind.ist hologo.idx
%pdflatex hologo.dtx
%makeindex -s gind.ist hologo.idx
%pdflatex hologo.dtx
%\end{verbatim}
% \end{quote}
%
% \section{Catalogue}
%
% The following XML file can be used as source for the
% \href{http://mirror.ctan.org/help/Catalogue/catalogue.html}{\TeX\ Catalogue}.
% The elements \texttt{caption} and \texttt{description} are imported
% from the original XML file from the Catalogue.
% The name of the XML file in the Catalogue is \xfile{hologo.xml}.
%    \begin{macrocode}
%<*catalogue>
<?xml version='1.0' encoding='us-ascii'?>
<!DOCTYPE entry SYSTEM 'catalogue.dtd'>
<entry datestamp='$Date$' modifier='$Author$' id='hologo'>
  <name>hologo</name>
  <caption>A collection of logos with bookmark support.</caption>
  <authorref id='auth:oberdiek'/>
  <copyright owner='Heiko Oberdiek' year='2010-2012'/>
  <license type='lppl1.3'/>
  <version number='1.10'/>
  <description>
    The package defines a single command <tt>\hologo</tt>, whose
    argument is the usual case-confused ASCII version of the logo.
    The command is bookmark-enabled, so that every logo becomes
    available in bookmarks without further work.
    <p/>
    The package is part of the <xref refid='oberdiek'>oberdiek</xref>
    bundle.
  </description>
  <documentation details='Package documentation'
      href='ctan:/macros/latex/contrib/oberdiek/hologo.pdf'/>
  <ctan file='true' path='/macros/latex/contrib/oberdiek/hologo.dtx'/>
  <miktex location='oberdiek'/>
  <texlive location='oberdiek'/>
  <install path='/macros/latex/contrib/oberdiek/oberdiek.tds.zip'/>
</entry>
%</catalogue>
%    \end{macrocode}
%
% \begin{thebibliography}{9}
% \raggedright
%
% \bibitem{btxdoc}
% Oren Patashnik,
% \textit{\hologo{BibTeX}ing},
% 1988-02-08.\\
% \CTAN{biblio/bibtex/base/}
%
% \bibitem{dtklogos}
% Gerd Neugebauer, DANTE,
% \textit{Package \xpackage{dtklogos}},
% 2011-04-25.\\
% \CTAN{usergrps/dante/dtk/dtklogos.sty}
%
% \bibitem{etexman}
% The \hologo{NTS} Team,
% \textit{The \hologo{eTeX} manual},
% 1998-02.\\
% \CTAN{systems/e-tex/v2/doc/}
%
% \bibitem{ExTeX-FAQ}
% The \hologo{ExTeX} group,
% \textit{\hologo{ExTeX}: FAQ -- How is \hologo{ExTeX} typeset?},
% 2007-04-14.\\
% \url{http://www.extex.org/documentation/faq.html}
%
% \bibitem{LyX}
% %@MISC{ LyX,
% %  title = {{LyX 2.0.0 -- The Document Processor [Computer software and manual]}},
% %  author = {{The LyX Team}},
% %  howpublished = {Internet: http://www.lyx.org},
% %  year = {2011-05-08},
% %  note = {Retrieved May 10, 2011, from http://www.lyx.org},
% %  url = {http://www.lyx.org/}
% %}
% The \hologo{LyX} Team,
% \textit{\hologo{LyX} -- The Document Processor},
% 2011-05-08.\\
% \url{http://www.lyx.org/}
%
% \bibitem{OzTeX}
% Andrew Trevorrow,
% \hologo{OzTeX} FAQ: What is the correct way to typeset ``\hologo{OzTeX}''?,
% 2011-09-15 (visited).
% \url{http://www.trevorrow.com/oztex/ozfaq.html#oztex-logo}
%
% \bibitem{PiCTeX}
% Michael Wichura,
% \textit{The \hologo{PiCTeX} macro package},
% 1987-09-21.
% \CTAN{graphics/pictex/}
%
% \bibitem{scrlogo}
% Markus Kohm,
% \textit{\hologo{KOMAScript} Datei \xfile{scrlogo.dtx}},
% 2009-01-30.\\
% \CTAN{install/macros/latex/contrib/komascript.tds.zip}
%
% \end{thebibliography}
%
% \begin{History}
%   \begin{Version}{2010/04/08 v1.0}
%   \item
%     The first version.
%   \end{Version}
%   \begin{Version}{2010/04/16 v1.1}
%   \item
%     \cs{Hologo} added for support of logos at start of a sentence.
%   \item
%     \cs{hologoSetup} and \cs{hologoLogoSetup} added.
%   \item
%     Options \xoption{break}, \xoption{hyphenbreak}, \xoption{spacebreak}
%     added.
%   \item
%     Variant support added by option \xoption{variant}.
%   \end{Version}
%   \begin{Version}{2010/04/24 v1.2}
%   \item
%     \hologo{LaTeX3} added.
%   \item
%     \hologo{VTeX} added.
%   \end{Version}
%   \begin{Version}{2010/11/21 v1.3}
%   \item
%     \hologo{iniTeX}, \hologo{virTeX} added.
%   \end{Version}
%   \begin{Version}{2011/03/25 v1.4}
%   \item
%     \hologo{ConTeXt} with variants added.
%   \item
%     Option \xoption{discretionarybreak} added as refinement for
%     option \xoption{break}.
%   \end{Version}
%   \begin{Version}{2011/04/21 v1.5}
%   \item
%     Wrong TDS directory for test files fixed.
%   \end{Version}
%   \begin{Version}{2011/10/01 v1.6}
%   \item
%     Support for package \xpackage{tex4ht} added.
%   \item
%     Support for \cs{csname} added if \cs{ifincsname} is available.
%   \item
%     New logos:
%     \hologo{(La)TeX},
%     \hologo{biber},
%     \hologo{BibTeX} (\xoption{sc}, \xoption{sf}),
%     \hologo{emTeX},
%     \hologo{ExTeX},
%     \hologo{KOMAScript},
%     \hologo{La},
%     \hologo{LyX},
%     \hologo{MiKTeX},
%     \hologo{NTS},
%     \hologo{OzMF},
%     \hologo{OzMP},
%     \hologo{OzTeX},
%     \hologo{OzTtH},
%     \hologo{PCTeX},
%     \hologo{PiC},
%     \hologo{PiCTeX},
%     \hologo{METAFONT},
%     \hologo{MetaFun},
%     \hologo{METAPOST},
%     \hologo{MetaPost},
%     \hologo{SLiTeX} (\xoption{lift}, \xoption{narrow}, \xoption{simple}),
%     \hologo{SliTeX} (\xoption{narrow}, \xoption{simple}, \xoption{lift}),
%     \hologo{teTeX}.
%   \item
%     Fixes:
%     \hologo{iniTeX},
%     \hologo{pdfLaTeX},
%     \hologo{pdfTeX},
%     \hologo{virTeX}.
%   \item
%     \cs{hologoFontSetup} and \cs{hologoLogoFontSetup} added.
%   \item
%     \cs{hologoVariant} and \cs{HologoVariant} added.
%   \end{Version}
%   \begin{Version}{2011/11/22 v1.7}
%   \item
%     New logos:
%     \hologo{BibTeX8},
%     \hologo{LaTeXML},
%     \hologo{SageTeX},
%     \hologo{TeX4ht},
%     \hologo{TTH}.
%   \item
%     \hologo{Xe} and friends: Driver stuff fixed.
%   \item
%     \hologo{Xe} and friends: Support for italic added.
%   \item
%     \hologo{Xe} and friends: Package support for \xpackage{pgf}
%     and \xpackage{pstricks} added.
%   \end{Version}
%   \begin{Version}{2011/11/29 v1.8}
%   \item
%     New logos:
%     \hologo{HanTheThanh}.
%   \end{Version}
%   \begin{Version}{2011/12/21 v1.9}
%   \item
%     Patch for package \xpackage{ifxetex} added for the case that
%     \cs{newif} is undefined in \hologo{iniTeX}.
%   \item
%     Some fixes for \hologo{iniTeX}.
%   \end{Version}
%   \begin{Version}{2012/04/26 v1.10}
%   \item
%     Fix in bookmark version of logo ``\hologo{HanTheThanh}''.
%   \end{Version}
%   \begin{Version}{2016/05/12 v1.11}
%   \item
%     Update HOLOGO@IfCharExists (previously in texlive)
%   \item define pdfliteral in current luatex.
%   \end{Version}
% \end{History}
%
% \PrintIndex
%
% \Finale
\endinput

%        (quote the arguments according to the demands of your shell)
%
% Documentation:
%    (a) If hologo.drv is present:
%           latex hologo.drv
%    (b) Without hologo.drv:
%           latex hologo.dtx; ...
%    The class ltxdoc loads the configuration file ltxdoc.cfg
%    if available. Here you can specify further options, e.g.
%    use A4 as paper format:
%       \PassOptionsToClass{a4paper}{article}
%
%    Programm calls to get the documentation (example):
%       pdflatex hologo.dtx
%       makeindex -s gind.ist hologo.idx
%       pdflatex hologo.dtx
%       makeindex -s gind.ist hologo.idx
%       pdflatex hologo.dtx
%
% Installation:
%    TDS:tex/generic/oberdiek/hologo.sty
%    TDS:doc/latex/oberdiek/hologo.pdf
%    TDS:doc/latex/oberdiek/example/hologo-example.tex
%    TDS:doc/latex/oberdiek/test/hologo-test1.tex
%    TDS:doc/latex/oberdiek/test/hologo-test-spacefactor.tex
%    TDS:doc/latex/oberdiek/test/hologo-test-list.tex
%    TDS:source/latex/oberdiek/hologo.dtx
%
%<*ignore>
\begingroup
  \catcode123=1 %
  \catcode125=2 %
  \def\x{LaTeX2e}%
\expandafter\endgroup
\ifcase 0\ifx\install y1\fi\expandafter
         \ifx\csname processbatchFile\endcsname\relax\else1\fi
         \ifx\fmtname\x\else 1\fi\relax
\else\csname fi\endcsname
%</ignore>
%<*install>
\input docstrip.tex
\Msg{************************************************************************}
\Msg{* Installation}
\Msg{* Package: hologo 2016/05/12 v1.11 A logo collection with bookmark support (HO)}
\Msg{************************************************************************}

\keepsilent
\askforoverwritefalse

\let\MetaPrefix\relax
\preamble

This is a generated file.

Project: hologo
Version: 2016/05/12 v1.11

Copyright (C) 2010-2012 by
   Heiko Oberdiek <heiko.oberdiek at googlemail.com>

This work may be distributed and/or modified under the
conditions of the LaTeX Project Public License, either
version 1.3c of this license or (at your option) any later
version. This version of this license is in
   http://www.latex-project.org/lppl/lppl-1-3c.txt
and the latest version of this license is in
   http://www.latex-project.org/lppl.txt
and version 1.3 or later is part of all distributions of
LaTeX version 2005/12/01 or later.

This work has the LPPL maintenance status "maintained".

This Current Maintainer of this work is Heiko Oberdiek.

The Base Interpreter refers to any `TeX-Format',
because some files are installed in TDS:tex/generic//.

This work consists of the main source file hologo.dtx
and the derived files
   hologo.sty, hologo.pdf, hologo.ins, hologo.drv, hologo-example.tex,
   hologo-test1.tex, hologo-test-spacefactor.tex,
   hologo-test-list.tex.

\endpreamble
\let\MetaPrefix\DoubleperCent

\generate{%
  \file{hologo.ins}{\from{hologo.dtx}{install}}%
  \file{hologo.drv}{\from{hologo.dtx}{driver}}%
  \usedir{tex/generic/oberdiek}%
  \file{hologo.sty}{\from{hologo.dtx}{package}}%
  \usedir{doc/latex/oberdiek/example}%
  \file{hologo-example.tex}{\from{hologo.dtx}{example}}%
  \usedir{doc/latex/oberdiek/test}%
  \file{hologo-test1.tex}{\from{hologo.dtx}{test1}}%
  \file{hologo-test-spacefactor.tex}{\from{hologo.dtx}{test-spacefactor}}%
  \file{hologo-test-list.tex}{\from{hologo.dtx}{test-list}}%
  \nopreamble
  \nopostamble
  \usedir{source/latex/oberdiek/catalogue}%
  \file{hologo.xml}{\from{hologo.dtx}{catalogue}}%
}

\catcode32=13\relax% active space
\let =\space%
\Msg{************************************************************************}
\Msg{*}
\Msg{* To finish the installation you have to move the following}
\Msg{* file into a directory searched by TeX:}
\Msg{*}
\Msg{*     hologo.sty}
\Msg{*}
\Msg{* To produce the documentation run the file `hologo.drv'}
\Msg{* through LaTeX.}
\Msg{*}
\Msg{* Happy TeXing!}
\Msg{*}
\Msg{************************************************************************}

\endbatchfile
%</install>
%<*ignore>
\fi
%</ignore>
%<*driver>
\NeedsTeXFormat{LaTeX2e}
\ProvidesFile{hologo.drv}%
  [2016/05/12 v1.11 A logo collection with bookmark support (HO)]%
\documentclass{ltxdoc}
\usepackage{holtxdoc}[2011/11/22]
\usepackage{hologo}[2016/05/12]
\usepackage{longtable}
\usepackage{array}
\usepackage{paralist}
%\usepackage[T1]{fontenc}
%\usepackage{lmodern}
\begin{document}
  \DocInput{hologo.dtx}%
\end{document}
%</driver>
% \fi
%
%
% \CharacterTable
%  {Upper-case    \A\B\C\D\E\F\G\H\I\J\K\L\M\N\O\P\Q\R\S\T\U\V\W\X\Y\Z
%   Lower-case    \a\b\c\d\e\f\g\h\i\j\k\l\m\n\o\p\q\r\s\t\u\v\w\x\y\z
%   Digits        \0\1\2\3\4\5\6\7\8\9
%   Exclamation   \!     Double quote  \"     Hash (number) \#
%   Dollar        \$     Percent       \%     Ampersand     \&
%   Acute accent  \'     Left paren    \(     Right paren   \)
%   Asterisk      \*     Plus          \+     Comma         \,
%   Minus         \-     Point         \.     Solidus       \/
%   Colon         \:     Semicolon     \;     Less than     \<
%   Equals        \=     Greater than  \>     Question mark \?
%   Commercial at \@     Left bracket  \[     Backslash     \\
%   Right bracket \]     Circumflex    \^     Underscore    \_
%   Grave accent  \`     Left brace    \{     Vertical bar  \|
%   Right brace   \}     Tilde         \~}
%
% \GetFileInfo{hologo.drv}
%
% \title{The \xpackage{hologo} package}
% \date{2016/05/12 v1.11}
% \author{Heiko Oberdiek\\\xemail{heiko.oberdiek at googlemail.com}}
%
% \maketitle
%
% \begin{abstract}
% This package starts a collection of logos with support for bookmarks
% strings.
% \end{abstract}
%
% \tableofcontents
%
% \section{Documentation}
%
% \subsection{Logo macros}
%
% \begin{declcs}{hologo} \M{name}
% \end{declcs}
% Macro \cs{hologo} sets the logo with name \meta{name}.
% The following table shows the supported names.
%
% \begingroup
%   \def\hologoEntry#1#2#3{^^A
%     #1&#2&\hologoLogoSetup{#1}{variant=#2}\hologo{#1}&#3\tabularnewline
%   }
%   \begin{longtable}{>{\ttfamily}l>{\ttfamily}lll}
%     \rmfamily\bfseries{name} & \rmfamily\bfseries variant
%     & \bfseries logo & \bfseries since\\
%     \hline
%     \endhead
%     \hologoList
%   \end{longtable}
% \endgroup
%
% \begin{declcs}{Hologo} \M{name}
% \end{declcs}
% Macro \cs{Hologo} starts the logo \meta{name} with an uppercase
% letter. As an exception small greek letters are not converted
% to uppercase. Examples, see \hologo{eTeX} and \hologo{ExTeX}.
%
% \subsection{Setup macros}
%
% The package does not support package options, but the following
% setup macros can be used to set options.
%
% \begin{declcs}{hologoSetup} \M{key value list}
% \end{declcs}
% Macro \cs{hologoSetup} sets global options.
%
% \begin{declcs}{hologoLogoSetup} \M{logo} \M{key value list}
% \end{declcs}
% Some options can also be used to configure a logo.
% These settings take precedence over global option settings.
%
% \subsection{Options}\label{sec:options}
%
% There are boolean and string options:
% \begin{description}
% \item[Boolean option:]
% It takes |true| or |false|
% as value. If the value is omitted, then |true| is used.
% \item[String option:]
% A value must be given as string. (But the string might be empty.)
% \end{description}
% The following options can be used both in \cs{hologoSetup}
% and \cs{hologoLogoSetup}:
% \begin{description}
% \def\entry#1{\item[\xoption{#1}:]}
% \entry{break}
%   enables or disables line breaks inside the logo. This setting is
%   refined by options \xoption{hyphenbreak}, \xoption{spacebreak}
%   or \xoption{discretionarybreak}.
%   Default is |false|.
% \entry{hyphenbreak}
%   enables or disables the line break right after the hyphen character.
% \entry{spacebreak}
%   enables or disables line breaks at space characters.
% \entry{discretionarybreak}
%   enables or disables line breaks at hyphenation points
%   (inserted by \cs{-}).
% \end{description}
% Macro \cs{hologoLogoSetup} also knows:
% \begin{description}
% \item[\xoption{variant}:]
%   This is a string option. It specifies a variant of a logo that
%   must exist. An empty string selects the package default variant.
% \end{description}
% Example:
% \begin{quote}
%   |\hologoSetup{break=false}|\\
%   |\hologoLogoSetup{plainTeX}{variant=hyphen,hyphenbreak}|\\
%   Then ``plain-\TeX'' contains one break point after the hyphen.
% \end{quote}
%
% \subsection{Driver options}
%
% Sometimes graphical operations are needed to construct some
% glyphs (e.g.\ \hologo{XeTeX}). If package \xpackage{graphics}
% or package \xpackage{pgf} are found, then the macros are taken
% from there. Otherwise the packge defines its own operations
% and therefore needs the driver information. Many drivers are
% detected automatically (\hologo{pdfTeX}/\hologo{LuaTeX}
% in PDF mode, \hologo{XeTeX}, \hologo{VTeX}). These have precedence
% over a driver option. The driver can be given as package option
% or using \cs{hologoDriverSetup}.
% The following list contains the recognized driver options:
% \begin{itemize}
% \item \xoption{pdftex}, \xoption{luatex}
% \item \xoption{dvipdfm}, \xoption{dvipdfmx}
% \item \xoption{dvips}, \xoption{dvipsone}, \xoption{xdvi}
% \item \xoption{xetex}
% \item \xoption{vtex}
% \end{itemize}
% The left driver of a line is the driver name that is used internally.
% The following names are aliases for drivers that use the
% same method. Therefore the entry in the \xext{log} file for
% the used driver prints the internally used driver name.
% \begin{description}
% \item[\xoption{driverfallback}:]
%   This option expects a driver that is used,
%   if the driver could not be detected automatically.
% \end{description}
%
% \begin{declcs}{hologoDriverSetup} \M{driver option}
% \end{declcs}
% The driver can also be configured after package loading
% using \cs{hologoDriverSetup}, also the way for \hologo{plainTeX}
% to setup the driver.
%
% \subsection{Font setup}
%
% Some logos require a special font, but should also be usable by
% \hologo{plainTeX}. Therefore the package provides some ways
% to influence the font settings. The options below
% take font settings as values. Both font commands
% such as \cs{sffamily} and macros that take one argument
% like \cs{textsf} can be used.
%
% \begin{declcs}{hologoFontSetup} \M{key value list}
% \end{declcs}
% Macro \cs{hologoFontSetup} sets the fonts for all logos.
% Supported keys:
% \begin{description}
% \def\entry#1{\item[\xoption{#1}:]}
% \entry{general}
%   This font is used for all logos. The default is empty.
%   That means no special font is used.
% \entry{bibsf}
%   This font is used for
%   {\hologoLogoSetup{BibTeX}{variant=sf}\hologo{BibTeX}}
%   with variant \xoption{sf}.
% \entry{rm}
%   This font is a serif font. It is used for \hologo{ExTeX}.
% \entry{sc}
%   This font specifies a small caps font. It is used for
%   {\hologoLogoSetup{BibTeX}{variant=sc}\hologo{BibTeX}}
%   with variant \xoption{sc}.
% \entry{sf}
%   This font specifies a sans serif font. The default
%   is \cs{sffamily}, then \cs{sf} is tried. Otherwise
%   a warning is given. It is used by \hologo{KOMAScript}.
% \entry{sy}
%   This is the font for math symbols (e.g. cmsy).
%   It is used by \hologo{AmS}, \hologo{NTS}, \hologo{ExTeX}.
% \entry{logo}
%   \hologo{METAFONT} and \hologo{METAPOST} are using that font.
%   In \hologo{LaTeX} \cs{logofamily} is used and
%   the definitions of package \xpackage{mflogo} are used
%   if the package is not loaded.
%   Otherwise the \cs{tenlogo} is used and defined
%   if it does not already exists.
% \end{description}
%
% \begin{declcs}{hologoLogoFontSetup} \M{logo} \M{key value list}
% \end{declcs}
% Fonts can also be set for a logo or logo component separately,
% see the following list.
% The keys are the same as for \cs{hologoFontSetup}.
%
% \begin{longtable}{>{\ttfamily}l>{\sffamily}ll}
%   \meta{logo} & keys & result\\
%   \hline
%   \endhead
%   BibTeX & bibsf & {\hologoLogoSetup{BibTeX}{variant=sf}\hologo{BibTeX}}\\[.5ex]
%   BibTeX & sc & {\hologoLogoSetup{BibTeX}{variant=sc}\hologo{BibTeX}}\\[.5ex]
%   ExTeX & rm & \hologo{ExTeX}\\
%   SliTeX & rm & \hologo{SliTeX}\\[.5ex]
%   AmS & sy & \hologo{AmS}\\
%   ExTeX & sy & \hologo{ExTeX}\\
%   NTS & sy & \hologo{NTS}\\[.5ex]
%   KOMAScript & sf & \hologo{KOMAScript}\\[.5ex]
%   METAFONT & logo & \hologo{METAFONT}\\
%   METAPOST & logo & \hologo{METAPOST}\\[.5ex]
%   SliTeX & sc \hologo{SliTeX}
% \end{longtable}
%
% \subsubsection{Font order}
%
% For all logos the font \xoption{general} is applied first.
% Example:
%\begin{quote}
%|\hologoFontSetup{general=\color{red}}|
%\end{quote}
% will print red logos.
% Then if the font uses a special font \xoption{sf}, for example,
% the font is applied that is setup by \cs{hologoLogoFontSetup}.
% If this font is not setup, then the common font setup
% by \cs{hologoFontSetup} is used. Otherwise a warning is given,
% that there is no font configured.
%
% \subsection{Additional user macros}
%
% Usually a variant of a logo is configured by using
% \cs{hologoLogoSetup}, because it is bad style to mix
% different variants of the same logo in the same text.
% There the following macros are a convenience for testing.
%
% \begin{declcs}{hologoVariant} \M{name} \M{variant}\\
%   \cs{HologoVariant} \M{name} \M{variant}
% \end{declcs}
% Logo \meta{name} is set using \meta{variant} that specifies
% explicitely which variant of the macro is used. If the argument
% is empty, then the default form of the logo is used
% (configurable by \cs{hologoLogoSetup}).
%
% \cs{HologoVariant} is used if the logo is set in a context
% that needs an uppercase first letter (beginning of a sentence, \dots).
%
% \begin{declcs}{hologoList}\\
%   \cs{hologoEntry} \M{logo} \M{variant} \M{since}
% \end{declcs}
% Macro \cs{hologoList} contains all logos that are provided
% by the package including variants. The list consists of calls
% of \cs{hologoEntry} with three arguments starting with the
% logo name \meta{logo} and its variant \meta{variant}. An empty
% variant means the current default. Argument \meta{since} specifies
% with version of the package \xpackage{hologo} is needed to get
% the logo. If the logo is fixed, then the date gets updated.
% Therefore the date \meta{since} is not exactly the date of
% the first introduction, but rather the date of the latest fix.
%
% Before \cs{hologoList} can be used, macro \cs{hologoEntry} needs
% a definition. The example file in section \ref{sec:example}
% shows applications of \cs{hologoList}.
%
% \subsection{Supported contexts}
%
% Macros \cs{hologo} and friends support special contexts:
% \begin{itemize}
% \item \hologo{LaTeX}'s protection mechanism.
% \item Bookmarks of package \xpackage{hyperref}.
% \item Package \xpackage{tex4ht}.
% \item The macros can be used inside \cs{csname} constructs,
%   if \cs{ifincsname} is available (\hologo{pdfTeX}, \hologo{XeTeX},
%   \hologo{LuaTeX}).
% \end{itemize}
%
% \subsection{Example}
% \label{sec:example}
%
% The following example prints the logos in different fonts.
%    \begin{macrocode}
%<*example>
%<<verbatim
\NeedsTeXFormat{LaTeX2e}
\documentclass[a4paper]{article}
\usepackage[
  hmargin=20mm,
  vmargin=20mm,
]{geometry}
\pagestyle{empty}
\usepackage{hologo}[2016/05/12]
\usepackage{longtable}
\usepackage{array}
\setlength{\extrarowheight}{2pt}
\usepackage[T1]{fontenc}
\usepackage{lmodern}
\usepackage{pdflscape}
\usepackage[
  pdfencoding=auto,
]{hyperref}
\hypersetup{
  pdfauthor={Heiko Oberdiek},
  pdftitle={Example for package `hologo'},
  pdfsubject={Logos with fonts lmr, lmss, qtm, qpl, qhv},
}
\usepackage{bookmark}

% Print the logo list on the console

\begingroup
  \typeout{}%
  \typeout{*** Begin of logo list ***}%
  \newcommand*{\hologoEntry}[3]{%
    \typeout{#1 \ifx\\#2\\\else(#2) \fi[#3]}%
  }%
  \hologoList
  \typeout{*** End of logo list ***}%
  \typeout{}%
\endgroup

\begin{document}
\begin{landscape}

  \section{Example file for package `hologo'}

  % Table for font names

  \begin{longtable}{>{\bfseries}ll}
    \textbf{font} & \textbf{Font name}\\
    \hline
    lmr & Latin Modern Roman\\
    lmss & Latin Modern Sans\\
    qtm & \TeX\ Gyre Termes\\
    qhv & \TeX\ Gyre Heros\\
    qpl & \TeX\ Gyre Pagella\\
  \end{longtable}

  % Logo list with logos in different fonts

  \begingroup
    \newcommand*{\SetVariant}[2]{%
      \ifx\\#2\\%
      \else
        \hologoLogoSetup{#1}{variant=#2}%
      \fi
    }%
    \newcommand*{\hologoEntry}[3]{%
      \SetVariant{#1}{#2}%
      \raisebox{1em}[0pt][0pt]{\hypertarget{#1@#2}{}}%
      \bookmark[%
        dest={#1@#2},%
      ]{%
        #1\ifx\\#2\\\else\space(#2)\fi: \Hologo{#1}, \hologo{#1} %
        [Unicode]%
      }%
      \hypersetup{unicode=false}%
      \bookmark[%
        dest={#1@#2},%
      ]{%
        #1\ifx\\#2\\\else\space(#2)\fi: \Hologo{#1}, \hologo{#1} %
        [PDFDocEncoding]%
      }%
      \texttt{#1}%
      &%
      \texttt{#2}%
      &%
      \Hologo{#1}%
      &%
      \SetVariant{#1}{#2}%
      \hologo{#1}%
      &%
      \SetVariant{#1}{#2}%
      \fontfamily{qtm}\selectfont
      \hologo{#1}%
      &%
      \SetVariant{#1}{#2}%
      \fontfamily{qpl}\selectfont
      \hologo{#1}%
      &%
      \SetVariant{#1}{#2}%
      \textsf{\hologo{#1}}%
      &%
      \SetVariant{#1}{#2}%
      \fontfamily{qhv}\selectfont
      \hologo{#1}%
      \tabularnewline
    }%
    \begin{longtable}{llllllll}%
      \textbf{\textit{logo}} & \textbf{\textit{variant}} &
      \texttt{\string\Hologo} &
      \textbf{lmr} & \textbf{qtm} & \textbf{qpl} &
      \textbf{lmss} & \textbf{qhv}
      \tabularnewline
      \hline
      \endhead
      \hologoList
    \end{longtable}%
  \endgroup

\end{landscape}
\end{document}
%verbatim
%</example>
%    \end{macrocode}
%
% \StopEventually{
% }
%
% \section{Implementation}
%    \begin{macrocode}
%<*package>
%    \end{macrocode}
%    Reload check, especially if the package is not used with \LaTeX.
%    \begin{macrocode}
\begingroup\catcode61\catcode48\catcode32=10\relax%
  \catcode13=5 % ^^M
  \endlinechar=13 %
  \catcode35=6 % #
  \catcode39=12 % '
  \catcode44=12 % ,
  \catcode45=12 % -
  \catcode46=12 % .
  \catcode58=12 % :
  \catcode64=11 % @
  \catcode123=1 % {
  \catcode125=2 % }
  \expandafter\let\expandafter\x\csname ver@hologo.sty\endcsname
  \ifx\x\relax % plain-TeX, first loading
  \else
    \def\empty{}%
    \ifx\x\empty % LaTeX, first loading,
      % variable is initialized, but \ProvidesPackage not yet seen
    \else
      \expandafter\ifx\csname PackageInfo\endcsname\relax
        \def\x#1#2{%
          \immediate\write-1{Package #1 Info: #2.}%
        }%
      \else
        \def\x#1#2{\PackageInfo{#1}{#2, stopped}}%
      \fi
      \x{hologo}{The package is already loaded}%
      \aftergroup\endinput
    \fi
  \fi
\endgroup%
%    \end{macrocode}
%    Package identification:
%    \begin{macrocode}
\begingroup\catcode61\catcode48\catcode32=10\relax%
  \catcode13=5 % ^^M
  \endlinechar=13 %
  \catcode35=6 % #
  \catcode39=12 % '
  \catcode40=12 % (
  \catcode41=12 % )
  \catcode44=12 % ,
  \catcode45=12 % -
  \catcode46=12 % .
  \catcode47=12 % /
  \catcode58=12 % :
  \catcode64=11 % @
  \catcode91=12 % [
  \catcode93=12 % ]
  \catcode123=1 % {
  \catcode125=2 % }
  \expandafter\ifx\csname ProvidesPackage\endcsname\relax
    \def\x#1#2#3[#4]{\endgroup
      \immediate\write-1{Package: #3 #4}%
      \xdef#1{#4}%
    }%
  \else
    \def\x#1#2[#3]{\endgroup
      #2[{#3}]%
      \ifx#1\@undefined
        \xdef#1{#3}%
      \fi
      \ifx#1\relax
        \xdef#1{#3}%
      \fi
    }%
  \fi
\expandafter\x\csname ver@hologo.sty\endcsname
\ProvidesPackage{hologo}%
  [2016/05/12 v1.11 A logo collection with bookmark support (HO)]%
%    \end{macrocode}
%
%    \begin{macrocode}
\begingroup\catcode61\catcode48\catcode32=10\relax%
  \catcode13=5 % ^^M
  \endlinechar=13 %
  \catcode123=1 % {
  \catcode125=2 % }
  \catcode64=11 % @
  \def\x{\endgroup
    \expandafter\edef\csname HOLOGO@AtEnd\endcsname{%
      \endlinechar=\the\endlinechar\relax
      \catcode13=\the\catcode13\relax
      \catcode32=\the\catcode32\relax
      \catcode35=\the\catcode35\relax
      \catcode61=\the\catcode61\relax
      \catcode64=\the\catcode64\relax
      \catcode123=\the\catcode123\relax
      \catcode125=\the\catcode125\relax
    }%
  }%
\x\catcode61\catcode48\catcode32=10\relax%
\catcode13=5 % ^^M
\endlinechar=13 %
\catcode35=6 % #
\catcode64=11 % @
\catcode123=1 % {
\catcode125=2 % }
\def\TMP@EnsureCode#1#2{%
  \edef\HOLOGO@AtEnd{%
    \HOLOGO@AtEnd
    \catcode#1=\the\catcode#1\relax
  }%
  \catcode#1=#2\relax
}
\TMP@EnsureCode{10}{12}% ^^J
\TMP@EnsureCode{33}{12}% !
\TMP@EnsureCode{34}{12}% "
\TMP@EnsureCode{36}{3}% $
\TMP@EnsureCode{38}{4}% &
\TMP@EnsureCode{39}{12}% '
\TMP@EnsureCode{40}{12}% (
\TMP@EnsureCode{41}{12}% )
\TMP@EnsureCode{42}{12}% *
\TMP@EnsureCode{43}{12}% +
\TMP@EnsureCode{44}{12}% ,
\TMP@EnsureCode{45}{12}% -
\TMP@EnsureCode{46}{12}% .
\TMP@EnsureCode{47}{12}% /
\TMP@EnsureCode{58}{12}% :
\TMP@EnsureCode{59}{12}% ;
\TMP@EnsureCode{60}{12}% <
\TMP@EnsureCode{62}{12}% >
\TMP@EnsureCode{63}{12}% ?
\TMP@EnsureCode{91}{12}% [
\TMP@EnsureCode{93}{12}% ]
\TMP@EnsureCode{94}{7}% ^ (superscript)
\TMP@EnsureCode{95}{8}% _ (subscript)
\TMP@EnsureCode{96}{12}% `
\TMP@EnsureCode{124}{12}% |
\edef\HOLOGO@AtEnd{%
  \HOLOGO@AtEnd
  \escapechar\the\escapechar\relax
  \noexpand\endinput
}
\escapechar=92 %
%    \end{macrocode}
%
% \subsection{Logo list}
%
%    \begin{macro}{\hologoList}
%    \begin{macrocode}
\def\hologoList{%
  \hologoEntry{(La)TeX}{}{2011/10/01}%
  \hologoEntry{AmSLaTeX}{}{2010/04/16}%
  \hologoEntry{AmSTeX}{}{2010/04/16}%
  \hologoEntry{biber}{}{2011/10/01}%
  \hologoEntry{BibTeX}{}{2011/10/01}%
  \hologoEntry{BibTeX}{sf}{2011/10/01}%
  \hologoEntry{BibTeX}{sc}{2011/10/01}%
  \hologoEntry{BibTeX8}{}{2011/11/22}%
  \hologoEntry{ConTeXt}{}{2011/03/25}%
  \hologoEntry{ConTeXt}{narrow}{2011/03/25}%
  \hologoEntry{ConTeXt}{simple}{2011/03/25}%
  \hologoEntry{emTeX}{}{2010/04/26}%
  \hologoEntry{eTeX}{}{2010/04/08}%
  \hologoEntry{ExTeX}{}{2011/10/01}%
  \hologoEntry{HanTheThanh}{}{2011/11/29}%
  \hologoEntry{iniTeX}{}{2011/10/01}%
  \hologoEntry{KOMAScript}{}{2011/10/01}%
  \hologoEntry{La}{}{2010/05/08}%
  \hologoEntry{LaTeX}{}{2010/04/08}%
  \hologoEntry{LaTeX2e}{}{2010/04/08}%
  \hologoEntry{LaTeX3}{}{2010/04/24}%
  \hologoEntry{LaTeXe}{}{2010/04/08}%
  \hologoEntry{LaTeXML}{}{2011/11/22}%
  \hologoEntry{LaTeXTeX}{}{2011/10/01}%
  \hologoEntry{LuaLaTeX}{}{2010/04/08}%
  \hologoEntry{LuaTeX}{}{2010/04/08}%
  \hologoEntry{LyX}{}{2011/10/01}%
  \hologoEntry{METAFONT}{}{2011/10/01}%
  \hologoEntry{MetaFun}{}{2011/10/01}%
  \hologoEntry{METAPOST}{}{2011/10/01}%
  \hologoEntry{MetaPost}{}{2011/10/01}%
  \hologoEntry{MiKTeX}{}{2011/10/01}%
  \hologoEntry{NTS}{}{2011/10/01}%
  \hologoEntry{OzMF}{}{2011/10/01}%
  \hologoEntry{OzMP}{}{2011/10/01}%
  \hologoEntry{OzTeX}{}{2011/10/01}%
  \hologoEntry{OzTtH}{}{2011/10/01}%
  \hologoEntry{PCTeX}{}{2011/10/01}%
  \hologoEntry{pdfTeX}{}{2011/10/01}%
  \hologoEntry{pdfLaTeX}{}{2011/10/01}%
  \hologoEntry{PiC}{}{2011/10/01}%
  \hologoEntry{PiCTeX}{}{2011/10/01}%
  \hologoEntry{plainTeX}{}{2010/04/08}%
  \hologoEntry{plainTeX}{space}{2010/04/16}%
  \hologoEntry{plainTeX}{hyphen}{2010/04/16}%
  \hologoEntry{plainTeX}{runtogether}{2010/04/16}%
  \hologoEntry{SageTeX}{}{2011/11/22}%
  \hologoEntry{SLiTeX}{}{2011/10/01}%
  \hologoEntry{SLiTeX}{lift}{2011/10/01}%
  \hologoEntry{SLiTeX}{narrow}{2011/10/01}%
  \hologoEntry{SLiTeX}{simple}{2011/10/01}%
  \hologoEntry{SliTeX}{}{2011/10/01}%
  \hologoEntry{SliTeX}{narrow}{2011/10/01}%
  \hologoEntry{SliTeX}{simple}{2011/10/01}%
  \hologoEntry{SliTeX}{lift}{2011/10/01}%
  \hologoEntry{teTeX}{}{2011/10/01}%
  \hologoEntry{TeX}{}{2010/04/08}%
  \hologoEntry{TeX4ht}{}{2011/11/22}%
  \hologoEntry{TTH}{}{2011/11/22}%
  \hologoEntry{virTeX}{}{2011/10/01}%
  \hologoEntry{VTeX}{}{2010/04/24}%
  \hologoEntry{Xe}{}{2010/04/08}%
  \hologoEntry{XeLaTeX}{}{2010/04/08}%
  \hologoEntry{XeTeX}{}{2010/04/08}%
}
%    \end{macrocode}
%    \end{macro}
%
% \subsection{Load resources}
%
%    \begin{macrocode}
\begingroup\expandafter\expandafter\expandafter\endgroup
\expandafter\ifx\csname RequirePackage\endcsname\relax
  \def\TMP@RequirePackage#1[#2]{%
    \begingroup\expandafter\expandafter\expandafter\endgroup
    \expandafter\ifx\csname ver@#1.sty\endcsname\relax
      \input #1.sty\relax
    \fi
  }%
  \TMP@RequirePackage{ltxcmds}[2011/02/04]%
  \TMP@RequirePackage{infwarerr}[2010/04/08]%
  \TMP@RequirePackage{kvsetkeys}[2010/03/01]%
  \TMP@RequirePackage{kvdefinekeys}[2010/03/01]%
  \TMP@RequirePackage{pdftexcmds}[2010/04/01]%
  \TMP@RequirePackage{ifpdf}[2010/01/28]%
  \TMP@RequirePackage{ifluatex}[2010/03/01]%
  \ltx@IfUndefined{newif}{%
    \expandafter\let\csname newif\endcsname\ltx@newif
  }{}%
  \TMP@RequirePackage{ifxetex}[2009/01/23]%
  \TMP@RequirePackage{ifvtex}[2010/03/01]%
\else
  \RequirePackage{ltxcmds}[2011/02/04]%
  \RequirePackage{infwarerr}[2010/04/08]%
  \RequirePackage{kvsetkeys}[2010/03/01]%
  \RequirePackage{kvdefinekeys}[2010/03/01]%
  \RequirePackage{pdftexcmds}[2010/04/01]%
  \RequirePackage{ifpdf}[2010/01/28]%
  \RequirePackage{ifluatex}[2010/03/01]%
  \RequirePackage{ifxetex}[2009/01/23]%
  \RequirePackage{ifvtex}[2010/03/01]%
\fi
%    \end{macrocode}
%
%    \begin{macro}{\HOLOGO@IfDefined}
%    \begin{macrocode}
\def\HOLOGO@IfExists#1{%
  \ifx\@undefined#1%
    \expandafter\ltx@secondoftwo
  \else
    \ifx\relax#1%
      \expandafter\ltx@secondoftwo
    \else
      \expandafter\expandafter\expandafter\ltx@firstoftwo
    \fi
  \fi
}
%    \end{macrocode}
%    \end{macro}
%
% \subsection{Setup macros}
%
%    \begin{macro}{\hologoSetup}
%    \begin{macrocode}
\def\hologoSetup{%
  \let\HOLOGO@name\relax
  \HOLOGO@Setup
}
%    \end{macrocode}
%    \end{macro}
%
%    \begin{macro}{\hologoLogoSetup}
%    \begin{macrocode}
\def\hologoLogoSetup#1{%
  \edef\HOLOGO@name{#1}%
  \ltx@IfUndefined{HoLogo@\HOLOGO@name}{%
    \@PackageError{hologo}{%
      Unknown logo `\HOLOGO@name'%
    }\@ehc
    \ltx@gobble
  }{%
    \HOLOGO@Setup
  }%
}
%    \end{macrocode}
%    \end{macro}
%
%    \begin{macro}{\HOLOGO@Setup}
%    \begin{macrocode}
\def\HOLOGO@Setup{%
  \kvsetkeys{HoLogo}%
}
%    \end{macrocode}
%    \end{macro}
%
% \subsection{Options}
%
%    \begin{macro}{\HOLOGO@DeclareBoolOption}
%    \begin{macrocode}
\def\HOLOGO@DeclareBoolOption#1{%
  \expandafter\chardef\csname HOLOGOOPT@#1\endcsname\ltx@zero
  \kv@define@key{HoLogo}{#1}[true]{%
    \def\HOLOGO@temp{##1}%
    \ifx\HOLOGO@temp\HOLOGO@true
      \ifx\HOLOGO@name\relax
        \expandafter\chardef\csname HOLOGOOPT@#1\endcsname=\ltx@one
      \else
        \expandafter\chardef\csname
        HoLogoOpt@#1@\HOLOGO@name\endcsname\ltx@one
      \fi
      \HOLOGO@SetBreakAll{#1}%
    \else
      \ifx\HOLOGO@temp\HOLOGO@false
        \ifx\HOLOGO@name\relax
          \expandafter\chardef\csname HOLOGOOPT@#1\endcsname=\ltx@zero
        \else
          \expandafter\chardef\csname
          HoLogoOpt@#1@\HOLOGO@name\endcsname=\ltx@zero
        \fi
        \HOLOGO@SetBreakAll{#1}%
      \else
        \@PackageError{hologo}{%
          Unknown value `##1' for boolean option `#1'.\MessageBreak
          Known values are `true' and `false'%
        }\@ehc
      \fi
    \fi
  }%
}
%    \end{macrocode}
%    \end{macro}
%
%    \begin{macro}{\HOLOGO@SetBreakAll}
%    \begin{macrocode}
\def\HOLOGO@SetBreakAll#1{%
  \def\HOLOGO@temp{#1}%
  \ifx\HOLOGO@temp\HOLOGO@break
    \ifx\HOLOGO@name\relax
      \chardef\HOLOGOOPT@hyphenbreak=\HOLOGOOPT@break
      \chardef\HOLOGOOPT@spacebreak=\HOLOGOOPT@break
      \chardef\HOLOGOOPT@discretionarybreak=\HOLOGOOPT@break
    \else
      \expandafter\chardef
         \csname HoLogoOpt@hyphenbreak@\HOLOGO@name\endcsname=%
         \csname HoLogoOpt@break@\HOLOGO@name\endcsname
      \expandafter\chardef
         \csname HoLogoOpt@spacebreak@\HOLOGO@name\endcsname=%
         \csname HoLogoOpt@break@\HOLOGO@name\endcsname
      \expandafter\chardef
         \csname HoLogoOpt@discretionarybreak@\HOLOGO@name
             \endcsname=%
         \csname HoLogoOpt@break@\HOLOGO@name\endcsname
    \fi
  \fi
}
%    \end{macrocode}
%    \end{macro}
%
%    \begin{macro}{\HOLOGO@true}
%    \begin{macrocode}
\def\HOLOGO@true{true}
%    \end{macrocode}
%    \end{macro}
%    \begin{macro}{\HOLOGO@false}
%    \begin{macrocode}
\def\HOLOGO@false{false}
%    \end{macrocode}
%    \end{macro}
%    \begin{macro}{\HOLOGO@break}
%    \begin{macrocode}
\def\HOLOGO@break{break}
%    \end{macrocode}
%    \end{macro}
%
%    \begin{macrocode}
\HOLOGO@DeclareBoolOption{break}
\HOLOGO@DeclareBoolOption{hyphenbreak}
\HOLOGO@DeclareBoolOption{spacebreak}
\HOLOGO@DeclareBoolOption{discretionarybreak}
%    \end{macrocode}
%
%    \begin{macrocode}
\kv@define@key{HoLogo}{variant}{%
  \ifx\HOLOGO@name\relax
    \@PackageError{hologo}{%
      Option `variant' is not available in \string\hologoSetup,%
      \MessageBreak
      Use \string\hologoLogoSetup\space instead%
    }\@ehc
  \else
    \edef\HOLOGO@temp{#1}%
    \ifx\HOLOGO@temp\ltx@empty
      \expandafter
      \let\csname HoLogoOpt@variant@\HOLOGO@name\endcsname\@undefined
    \else
      \ltx@IfUndefined{HoLogo@\HOLOGO@name @\HOLOGO@temp}{%
        \@PackageError{hologo}{%
          Unknown variant `\HOLOGO@temp' of logo `\HOLOGO@name'%
        }\@ehc
      }{%
        \expandafter
        \let\csname HoLogoOpt@variant@\HOLOGO@name\endcsname
            \HOLOGO@temp
      }%
    \fi
  \fi
}
%    \end{macrocode}
%
%    \begin{macro}{\HOLOGO@Variant}
%    \begin{macrocode}
\def\HOLOGO@Variant#1{%
  #1%
  \ltx@ifundefined{HoLogoOpt@variant@#1}{%
  }{%
    @\csname HoLogoOpt@variant@#1\endcsname
  }%
}
%    \end{macrocode}
%    \end{macro}
%
% \subsection{Break/no-break support}
%
%    \begin{macro}{\HOLOGO@space}
%    \begin{macrocode}
\def\HOLOGO@space{%
  \ltx@ifundefined{HoLogoOpt@spacebreak@\HOLOGO@name}{%
    \ltx@ifundefined{HoLogoOpt@break@\HOLOGO@name}{%
      \chardef\HOLOGO@temp=\HOLOGOOPT@spacebreak
    }{%
      \chardef\HOLOGO@temp=%
        \csname HoLogoOpt@break@\HOLOGO@name\endcsname
    }%
  }{%
    \chardef\HOLOGO@temp=%
      \csname HoLogoOpt@spacebreak@\HOLOGO@name\endcsname
  }%
  \ifcase\HOLOGO@temp
    \penalty10000 %
  \fi
  \ltx@space
}
%    \end{macrocode}
%    \end{macro}
%
%    \begin{macro}{\HOLOGO@hyphen}
%    \begin{macrocode}
\def\HOLOGO@hyphen{%
  \ltx@ifundefined{HoLogoOpt@hyphenbreak@\HOLOGO@name}{%
    \ltx@ifundefined{HoLogoOpt@break@\HOLOGO@name}{%
      \chardef\HOLOGO@temp=\HOLOGOOPT@hyphenbreak
    }{%
      \chardef\HOLOGO@temp=%
        \csname HoLogoOpt@break@\HOLOGO@name\endcsname
    }%
  }{%
    \chardef\HOLOGO@temp=%
      \csname HoLogoOpt@hyphenbreak@\HOLOGO@name\endcsname
  }%
  \ifcase\HOLOGO@temp
    \ltx@mbox{-}%
  \else
    -%
  \fi
}
%    \end{macrocode}
%    \end{macro}
%
%    \begin{macro}{\HOLOGO@discretionary}
%    \begin{macrocode}
\def\HOLOGO@discretionary{%
  \ltx@ifundefined{HoLogoOpt@discretionarybreak@\HOLOGO@name}{%
    \ltx@ifundefined{HoLogoOpt@break@\HOLOGO@name}{%
      \chardef\HOLOGO@temp=\HOLOGOOPT@discretionarybreak
    }{%
      \chardef\HOLOGO@temp=%
        \csname HoLogoOpt@break@\HOLOGO@name\endcsname
    }%
  }{%
    \chardef\HOLOGO@temp=%
      \csname HoLogoOpt@discretionarybreak@\HOLOGO@name\endcsname
  }%
  \ifcase\HOLOGO@temp
  \else
    \-%
  \fi
}
%    \end{macrocode}
%    \end{macro}
%
%    \begin{macro}{\HOLOGO@mbox}
%    \begin{macrocode}
\def\HOLOGO@mbox#1{%
  \ltx@ifundefined{HoLogoOpt@break@\HOLOGO@name}{%
    \chardef\HOLOGO@temp=\HOLOGOOPT@hyphenbreak
  }{%
    \chardef\HOLOGO@temp=%
      \csname HoLogoOpt@break@\HOLOGO@name\endcsname
  }%
  \ifcase\HOLOGO@temp
    \ltx@mbox{#1}%
  \else
    #1%
  \fi
}
%    \end{macrocode}
%    \end{macro}
%
% \subsection{Font support}
%
%    \begin{macro}{\HoLogoFont@font}
%    \begin{tabular}{@{}ll@{}}
%    |#1|:& logo name\\
%    |#2|:& font short name\\
%    |#3|:& text
%    \end{tabular}
%    \begin{macrocode}
\def\HoLogoFont@font#1#2#3{%
  \begingroup
    \ltx@IfUndefined{HoLogoFont@logo@#1.#2}{%
      \ltx@IfUndefined{HoLogoFont@font@#2}{%
        \@PackageWarning{hologo}{%
          Missing font `#2' for logo `#1'%
        }%
        #3%
      }{%
        \csname HoLogoFont@font@#2\endcsname{#3}%
      }%
    }{%
      \csname HoLogoFont@logo@#1.#2\endcsname{#3}%
    }%
  \endgroup
}
%    \end{macrocode}
%    \end{macro}
%
%    \begin{macro}{\HoLogoFont@Def}
%    \begin{macrocode}
\def\HoLogoFont@Def#1{%
  \expandafter\def\csname HoLogoFont@font@#1\endcsname
}
%    \end{macrocode}
%    \end{macro}
%    \begin{macro}{\HoLogoFont@LogoDef}
%    \begin{macrocode}
\def\HoLogoFont@LogoDef#1#2{%
  \expandafter\def\csname HoLogoFont@logo@#1.#2\endcsname
}
%    \end{macrocode}
%    \end{macro}
%
% \subsubsection{Font defaults}
%
%    \begin{macro}{\HoLogoFont@font@general}
%    \begin{macrocode}
\HoLogoFont@Def{general}{}%
%    \end{macrocode}
%    \end{macro}
%
%    \begin{macro}{\HoLogoFont@font@rm}
%    \begin{macrocode}
\ltx@IfUndefined{rmfamily}{%
  \ltx@IfUndefined{rm}{%
  }{%
    \HoLogoFont@Def{rm}{\rm}%
  }%
}{%
  \HoLogoFont@Def{rm}{\rmfamily}%
}
%    \end{macrocode}
%    \end{macro}
%
%    \begin{macro}{\HoLogoFont@font@sf}
%    \begin{macrocode}
\ltx@IfUndefined{sffamily}{%
  \ltx@IfUndefined{sf}{%
  }{%
    \HoLogoFont@Def{sf}{\sf}%
  }%
}{%
  \HoLogoFont@Def{sf}{\sffamily}%
}
%    \end{macrocode}
%    \end{macro}
%
%    \begin{macro}{\HoLogoFont@font@bibsf}
%    In case of \hologo{plainTeX} the original small caps
%    variant is used as default. In \hologo{LaTeX}
%    the definition of package \xpackage{dtklogos} \cite{dtklogos}
%    is used.
%\begin{quote}
%\begin{verbatim}
%\DeclareRobustCommand{\BibTeX}{%
%  B%
%  \kern-.05em%
%  \hbox{%
%    $\m@th$% %% force math size calculations
%    \csname S@\f@size\endcsname
%    \fontsize\sf@size\z@
%    \math@fontsfalse
%    \selectfont
%    I%
%    \kern-.025em%
%    B
%  }%
%  \kern-.08em%
%  \-%
%  \TeX
%}
%\end{verbatim}
%\end{quote}
%    \begin{macrocode}
\ltx@IfUndefined{selectfont}{%
  \ltx@IfUndefined{tensc}{%
    \font\tensc=cmcsc10\relax
  }{}%
  \HoLogoFont@Def{bibsf}{\tensc}%
}{%
  \HoLogoFont@Def{bibsf}{%
    $\mathsurround=0pt$%
    \csname S@\f@size\endcsname
    \fontsize\sf@size{0pt}%
    \math@fontsfalse
    \selectfont
  }%
}
%    \end{macrocode}
%    \end{macro}
%
%    \begin{macro}{\HoLogoFont@font@sc}
%    \begin{macrocode}
\ltx@IfUndefined{scshape}{%
  \ltx@IfUndefined{tensc}{%
    \font\tensc=cmcsc10\relax
  }{}%
  \HoLogoFont@Def{sc}{\tensc}%
}{%
  \HoLogoFont@Def{sc}{\scshape}%
}
%    \end{macrocode}
%    \end{macro}
%
%    \begin{macro}{\HoLogoFont@font@sy}
%    \begin{macrocode}
\ltx@IfUndefined{usefont}{%
  \ltx@IfUndefined{tensy}{%
  }{%
    \HoLogoFont@Def{sy}{\tensy}%
  }%
}{%
  \HoLogoFont@Def{sy}{%
    \usefont{OMS}{cmsy}{m}{n}%
  }%
}
%    \end{macrocode}
%    \end{macro}
%
%    \begin{macro}{\HoLogoFont@font@logo}
%    \begin{macrocode}
\begingroup
  \def\x{LaTeX2e}%
\expandafter\endgroup
\ifx\fmtname\x
  \ltx@IfUndefined{logofamily}{%
    \DeclareRobustCommand\logofamily{%
      \not@math@alphabet\logofamily\relax
      \fontencoding{U}%
      \fontfamily{logo}%
      \selectfont
    }%
  }{}%
  \ltx@IfUndefined{logofamily}{%
  }{%
    \HoLogoFont@Def{logo}{\logofamily}%
  }%
\else
  \ltx@IfUndefined{tenlogo}{%
    \font\tenlogo=logo10\relax
  }{}%
  \HoLogoFont@Def{logo}{\tenlogo}%
\fi
%    \end{macrocode}
%    \end{macro}
%
% \subsubsection{Font setup}
%
%    \begin{macro}{\hologoFontSetup}
%    \begin{macrocode}
\def\hologoFontSetup{%
  \let\HOLOGO@name\relax
  \HOLOGO@FontSetup
}
%    \end{macrocode}
%    \end{macro}
%
%    \begin{macro}{\hologoLogoFontSetup}
%    \begin{macrocode}
\def\hologoLogoFontSetup#1{%
  \edef\HOLOGO@name{#1}%
  \ltx@IfUndefined{HoLogo@\HOLOGO@name}{%
    \@PackageError{hologo}{%
      Unknown logo `\HOLOGO@name'%
    }\@ehc
    \ltx@gobble
  }{%
    \HOLOGO@FontSetup
  }%
}
%    \end{macrocode}
%    \end{macro}
%
%    \begin{macro}{\HOLOGO@FontSetup}
%    \begin{macrocode}
\def\HOLOGO@FontSetup{%
  \kvsetkeys{HoLogoFont}%
}
%    \end{macrocode}
%    \end{macro}
%
%    \begin{macrocode}
\def\HOLOGO@temp#1{%
  \kv@define@key{HoLogoFont}{#1}{%
    \ifx\HOLOGO@name\relax
      \HoLogoFont@Def{#1}{##1}%
    \else
      \HoLogoFont@LogoDef\HOLOGO@name{#1}{##1}%
    \fi
  }%
}
\HOLOGO@temp{general}
\HOLOGO@temp{sf}
%    \end{macrocode}
%
% \subsection{Generic logo commands}
%
%    \begin{macrocode}
\HOLOGO@IfExists\hologo{%
  \@PackageError{hologo}{%
    \string\hologo\ltx@space is already defined.\MessageBreak
    Package loading is aborted%
  }\@ehc
  \HOLOGO@AtEnd
}%
\HOLOGO@IfExists\hologoRobust{%
  \@PackageError{hologo}{%
    \string\hologoRobust\ltx@space is already defined.\MessageBreak
    Package loading is aborted%
  }\@ehc
  \HOLOGO@AtEnd
}%
%    \end{macrocode}
%
% \subsubsection{\cs{hologo} and friends}
%
%    \begin{macrocode}
\ifluatex
  \expandafter\ltx@firstofone
\else
  \expandafter\ltx@gobble
\fi
{%
  \ltx@IfUndefined{ifincsname}{%
    \ifnum\luatexversion<36 %
      \expandafter\ltx@gobble
    \else
      \expandafter\ltx@firstofone
    \fi
    {%
      \begingroup
        \ifcase0%
            \directlua{%
              if tex.enableprimitives then %
                tex.enableprimitives('HOLOGO@', {'ifincsname'})%
              else %
                tex.print('1')%
              end%
            }%
            \ifx\HOLOGO@ifincsname\@undefined 1\fi%
            \relax
          \expandafter\ltx@firstofone
        \else
          \endgroup
          \expandafter\ltx@gobble
        \fi
        {%
          \global\let\ifincsname\HOLOGO@ifincsname
        }%
      \HOLOGO@temp
    }%
  }{}%
}
%    \end{macrocode}
%    \begin{macrocode}
\ltx@IfUndefined{ifincsname}{%
  \catcode`$=14 %
}{%
  \catcode`$=9 %
}
%    \end{macrocode}
%
%    \begin{macro}{\hologo}
%    \begin{macrocode}
\def\hologo#1{%
$ \ifincsname
$   \ltx@ifundefined{HoLogoCs@\HOLOGO@Variant{#1}}{%
$     #1%
$   }{%
$     \csname HoLogoCs@\HOLOGO@Variant{#1}\endcsname\ltx@firstoftwo
$   }%
$ \else
    \HOLOGO@IfExists\texorpdfstring\texorpdfstring\ltx@firstoftwo
    {%
      \hologoRobust{#1}%
    }{%
      \ltx@ifundefined{HoLogoBkm@\HOLOGO@Variant{#1}}{%
        \ltx@ifundefined{HoLogo@#1}{?#1?}{#1}%
      }{%
        \csname HoLogoBkm@\HOLOGO@Variant{#1}\endcsname
        \ltx@firstoftwo
      }%
    }%
$ \fi
}
%    \end{macrocode}
%    \end{macro}
%    \begin{macro}{\Hologo}
%    \begin{macrocode}
\def\Hologo#1{%
$ \ifincsname
$   \ltx@ifundefined{HoLogoCs@\HOLOGO@Variant{#1}}{%
$     #1%
$   }{%
$     \csname HoLogoCs@\HOLOGO@Variant{#1}\endcsname\ltx@secondoftwo
$   }%
$ \else
    \HOLOGO@IfExists\texorpdfstring\texorpdfstring\ltx@firstoftwo
    {%
      \HologoRobust{#1}%
    }{%
      \ltx@ifundefined{HoLogoBkm@\HOLOGO@Variant{#1}}{%
        \ltx@ifundefined{HoLogo@#1}{?#1?}{#1}%
      }{%
        \csname HoLogoBkm@\HOLOGO@Variant{#1}\endcsname
        \ltx@secondoftwo
      }%
    }%
$ \fi
}
%    \end{macrocode}
%    \end{macro}
%
%    \begin{macro}{\hologoVariant}
%    \begin{macrocode}
\def\hologoVariant#1#2{%
  \ifx\relax#2\relax
    \hologo{#1}%
  \else
$   \ifincsname
$     \ltx@ifundefined{HoLogoCs@#1@#2}{%
$       #1%
$     }{%
$       \csname HoLogoCs@#1@#2\endcsname\ltx@firstoftwo
$     }%
$   \else
      \HOLOGO@IfExists\texorpdfstring\texorpdfstring\ltx@firstoftwo
      {%
        \hologoVariantRobust{#1}{#2}%
      }{%
        \ltx@ifundefined{HoLogoBkm@#1@#2}{%
          \ltx@ifundefined{HoLogo@#1}{?#1?}{#1}%
        }{%
          \csname HoLogoBkm@#1@#2\endcsname
          \ltx@firstoftwo
        }%
      }%
$   \fi
  \fi
}
%    \end{macrocode}
%    \end{macro}
%    \begin{macro}{\HologoVariant}
%    \begin{macrocode}
\def\HologoVariant#1#2{%
  \ifx\relax#2\relax
    \Hologo{#1}%
  \else
$   \ifincsname
$     \ltx@ifundefined{HoLogoCs@#1@#2}{%
$       #1%
$     }{%
$       \csname HoLogoCs@#1@#2\endcsname\ltx@secondoftwo
$     }%
$   \else
      \HOLOGO@IfExists\texorpdfstring\texorpdfstring\ltx@firstoftwo
      {%
        \HologoVariantRobust{#1}{#2}%
      }{%
        \ltx@ifundefined{HoLogoBkm@#1@#2}{%
          \ltx@ifundefined{HoLogo@#1}{?#1?}{#1}%
        }{%
          \csname HoLogoBkm@#1@#2\endcsname
          \ltx@secondoftwo
        }%
      }%
$   \fi
  \fi
}
%    \end{macrocode}
%    \end{macro}
%
%    \begin{macrocode}
\catcode`\$=3 %
%    \end{macrocode}
%
% \subsubsection{\cs{hologoRobust} and friends}
%
%    \begin{macro}{\hologoRobust}
%    \begin{macrocode}
\ltx@IfUndefined{protected}{%
  \ltx@IfUndefined{DeclareRobustCommand}{%
    \def\hologoRobust#1%
  }{%
    \DeclareRobustCommand*\hologoRobust[1]%
  }%
}{%
  \protected\def\hologoRobust#1%
}%
{%
  \edef\HOLOGO@name{#1}%
  \ltx@IfUndefined{HoLogo@\HOLOGO@Variant\HOLOGO@name}{%
    \@PackageError{hologo}{%
      Unknown logo `\HOLOGO@name'%
    }\@ehc
    ?\HOLOGO@name?%
  }{%
    \ltx@IfUndefined{ver@tex4ht.sty}{%
      \HoLogoFont@font\HOLOGO@name{general}{%
        \csname HoLogo@\HOLOGO@Variant\HOLOGO@name\endcsname
        \ltx@firstoftwo
      }%
    }{%
      \ltx@IfUndefined{HoLogoHtml@\HOLOGO@Variant\HOLOGO@name}{%
        \HOLOGO@name
      }{%
        \csname HoLogoHtml@\HOLOGO@Variant\HOLOGO@name\endcsname
        \ltx@firstoftwo
      }%
    }%
  }%
}
%    \end{macrocode}
%    \end{macro}
%    \begin{macro}{\HologoRobust}
%    \begin{macrocode}
\ltx@IfUndefined{protected}{%
  \ltx@IfUndefined{DeclareRobustCommand}{%
    \def\HologoRobust#1%
  }{%
    \DeclareRobustCommand*\HologoRobust[1]%
  }%
}{%
  \protected\def\HologoRobust#1%
}%
{%
  \edef\HOLOGO@name{#1}%
  \ltx@IfUndefined{HoLogo@\HOLOGO@Variant\HOLOGO@name}{%
    \@PackageError{hologo}{%
      Unknown logo `\HOLOGO@name'%
    }\@ehc
    ?\HOLOGO@name?%
  }{%
    \ltx@IfUndefined{ver@tex4ht.sty}{%
      \HoLogoFont@font\HOLOGO@name{general}{%
        \csname HoLogo@\HOLOGO@Variant\HOLOGO@name\endcsname
        \ltx@secondoftwo
      }%
    }{%
      \ltx@IfUndefined{HoLogoHtml@\HOLOGO@Variant\HOLOGO@name}{%
        \expandafter\HOLOGO@Uppercase\HOLOGO@name
      }{%
        \csname HoLogoHtml@\HOLOGO@Variant\HOLOGO@name\endcsname
        \ltx@secondoftwo
      }%
    }%
  }%
}
%    \end{macrocode}
%    \end{macro}
%    \begin{macro}{\hologoVariantRobust}
%    \begin{macrocode}
\ltx@IfUndefined{protected}{%
  \ltx@IfUndefined{DeclareRobustCommand}{%
    \def\hologoVariantRobust#1#2%
  }{%
    \DeclareRobustCommand*\hologoVariantRobust[2]%
  }%
}{%
  \protected\def\hologoVariantRobust#1#2%
}%
{%
  \begingroup
    \hologoLogoSetup{#1}{variant={#2}}%
    \hologoRobust{#1}%
  \endgroup
}
%    \end{macrocode}
%    \end{macro}
%    \begin{macro}{\HologoVariantRobust}
%    \begin{macrocode}
\ltx@IfUndefined{protected}{%
  \ltx@IfUndefined{DeclareRobustCommand}{%
    \def\HologoVariantRobust#1#2%
  }{%
    \DeclareRobustCommand*\HologoVariantRobust[2]%
  }%
}{%
  \protected\def\HologoVariantRobust#1#2%
}%
{%
  \begingroup
    \hologoLogoSetup{#1}{variant={#2}}%
    \HologoRobust{#1}%
  \endgroup
}
%    \end{macrocode}
%    \end{macro}
%
%    \begin{macro}{\hologorobust}
%    Macro \cs{hologorobust} is only defined for compatibility.
%    Its use is deprecated.
%    \begin{macrocode}
\def\hologorobust{\hologoRobust}
%    \end{macrocode}
%    \end{macro}
%
% \subsection{Helpers}
%
%    \begin{macro}{\HOLOGO@Uppercase}
%    Macro \cs{HOLOGO@Uppercase} is restricted to \cs{uppercase},
%    because \hologo{plainTeX} or \hologo{iniTeX} do not provide
%    \cs{MakeUppercase}.
%    \begin{macrocode}
\def\HOLOGO@Uppercase#1{\uppercase{#1}}
%    \end{macrocode}
%    \end{macro}
%
%    \begin{macro}{\HOLOGO@PdfdocUnicode}
%    \begin{macrocode}
\def\HOLOGO@PdfdocUnicode{%
  \ifx\ifHy@unicode\iftrue
    \expandafter\ltx@secondoftwo
  \else
    \expandafter\ltx@firstoftwo
  \fi
}
%    \end{macrocode}
%    \end{macro}
%
%    \begin{macro}{\HOLOGO@Math}
%    \begin{macrocode}
\def\HOLOGO@MathSetup{%
  \mathsurround0pt\relax
  \HOLOGO@IfExists\f@series{%
    \if b\expandafter\ltx@car\f@series x\@nil
      \csname boldmath\endcsname
   \fi
  }{}%
}
%    \end{macrocode}
%    \end{macro}
%
%    \begin{macro}{\HOLOGO@TempDimen}
%    \begin{macrocode}
\dimendef\HOLOGO@TempDimen=\ltx@zero
%    \end{macrocode}
%    \end{macro}
%    \begin{macro}{\HOLOGO@NegativeKerning}
%    \begin{macrocode}
\def\HOLOGO@NegativeKerning#1{%
  \begingroup
    \HOLOGO@TempDimen=0pt\relax
    \comma@parse@normalized{#1}{%
      \ifdim\HOLOGO@TempDimen=0pt %
        \expandafter\HOLOGO@@NegativeKerning\comma@entry
      \fi
      \ltx@gobble
    }%
    \ifdim\HOLOGO@TempDimen<0pt %
      \kern\HOLOGO@TempDimen
    \fi
  \endgroup
}
%    \end{macrocode}
%    \end{macro}
%    \begin{macro}{\HOLOGO@@NegativeKerning}
%    \begin{macrocode}
\def\HOLOGO@@NegativeKerning#1#2{%
  \setbox\ltx@zero\hbox{#1#2}%
  \HOLOGO@TempDimen=\wd\ltx@zero
  \setbox\ltx@zero\hbox{#1\kern0pt#2}%
  \advance\HOLOGO@TempDimen by -\wd\ltx@zero
}
%    \end{macrocode}
%    \end{macro}
%
%    \begin{macro}{\HOLOGO@SpaceFactor}
%    \begin{macrocode}
\def\HOLOGO@SpaceFactor{%
  \spacefactor1000 %
}
%    \end{macrocode}
%    \end{macro}
%
%    \begin{macro}{\HOLOGO@Span}
%    \begin{macrocode}
\def\HOLOGO@Span#1#2{%
  \HCode{<span class="HoLogo-#1">}%
  #2%
  \HCode{</span>}%
}
%    \end{macrocode}
%    \end{macro}
%
% \subsubsection{Text subscript}
%
%    \begin{macro}{\HOLOGO@SubScript}%
%    \begin{macrocode}
\def\HOLOGO@SubScript#1{%
  \ltx@IfUndefined{textsubscript}{%
    \ltx@IfUndefined{text}{%
      \ltx@mbox{%
        \mathsurround=0pt\relax
        $%
          _{%
            \ltx@IfUndefined{sf@size}{%
              \mathrm{#1}%
            }{%
              \mbox{%
                \fontsize\sf@size{0pt}\selectfont
                #1%
              }%
            }%
          }%
        $%
      }%
    }{%
      \ltx@mbox{%
        \mathsurround=0pt\relax
        $_{\text{#1}}$%
      }%
    }%
  }{%
    \textsubscript{#1}%
  }%
}
%    \end{macrocode}
%    \end{macro}
%
% \subsection{\hologo{TeX} and friends}
%
% \subsubsection{\hologo{TeX}}
%
%    \begin{macro}{\HoLogo@TeX}
%    Source: \hologo{LaTeX} kernel.
%    \begin{macrocode}
\def\HoLogo@TeX#1{%
  T\kern-.1667em\lower.5ex\hbox{E}\kern-.125emX\HOLOGO@SpaceFactor
}
%    \end{macrocode}
%    \end{macro}
%    \begin{macro}{\HoLogoHtml@TeX}
%    \begin{macrocode}
\def\HoLogoHtml@TeX#1{%
  \HoLogoCss@TeX
  \HOLOGO@Span{TeX}{%
    T%
    \HOLOGO@Span{e}{%
      E%
    }%
    X%
  }%
}
%    \end{macrocode}
%    \end{macro}
%    \begin{macro}{\HoLogoCss@TeX}
%    \begin{macrocode}
\def\HoLogoCss@TeX{%
  \Css{%
    span.HoLogo-TeX span.HoLogo-e{%
      position:relative;%
      top:.5ex;%
      margin-left:-.1667em;%
      margin-right:-.125em;%
    }%
  }%
  \Css{%
    a span.HoLogo-TeX span.HoLogo-e{%
      text-decoration:none;%
    }%
  }%
  \global\let\HoLogoCss@TeX\relax
}
%    \end{macrocode}
%    \end{macro}
%
% \subsubsection{\hologo{plainTeX}}
%
%    \begin{macro}{\HoLogo@plainTeX@space}
%    Source: ``The \hologo{TeX}book''
%    \begin{macrocode}
\def\HoLogo@plainTeX@space#1{%
  \HOLOGO@mbox{#1{p}{P}lain}\HOLOGO@space\hologo{TeX}%
}
%    \end{macrocode}
%    \end{macro}
%    \begin{macro}{\HoLogoCs@plainTeX@space}
%    \begin{macrocode}
\def\HoLogoCs@plainTeX@space#1{#1{p}{P}lain TeX}%
%    \end{macrocode}
%    \end{macro}
%    \begin{macro}{\HoLogoBkm@plainTeX@space}
%    \begin{macrocode}
\def\HoLogoBkm@plainTeX@space#1{%
  #1{p}{P}lain \hologo{TeX}%
}
%    \end{macrocode}
%    \end{macro}
%    \begin{macro}{\HoLogoHtml@plainTeX@space}
%    \begin{macrocode}
\def\HoLogoHtml@plainTeX@space#1{%
  #1{p}{P}lain \hologo{TeX}%
}
%    \end{macrocode}
%    \end{macro}
%
%    \begin{macro}{\HoLogo@plainTeX@hyphen}
%    \begin{macrocode}
\def\HoLogo@plainTeX@hyphen#1{%
  \HOLOGO@mbox{#1{p}{P}lain}\HOLOGO@hyphen\hologo{TeX}%
}
%    \end{macrocode}
%    \end{macro}
%    \begin{macro}{\HoLogoCs@plainTeX@hyphen}
%    \begin{macrocode}
\def\HoLogoCs@plainTeX@hyphen#1{#1{p}{P}lain-TeX}
%    \end{macrocode}
%    \end{macro}
%    \begin{macro}{\HoLogoBkm@plainTeX@hyphen}
%    \begin{macrocode}
\def\HoLogoBkm@plainTeX@hyphen#1{%
  #1{p}{P}lain-\hologo{TeX}%
}
%    \end{macrocode}
%    \end{macro}
%    \begin{macro}{\HoLogoHtml@plainTeX@hyphen}
%    \begin{macrocode}
\def\HoLogoHtml@plainTeX@hyphen#1{%
  #1{p}{P}lain-\hologo{TeX}%
}
%    \end{macrocode}
%    \end{macro}
%
%    \begin{macro}{\HoLogo@plainTeX@runtogether}
%    \begin{macrocode}
\def\HoLogo@plainTeX@runtogether#1{%
  \HOLOGO@mbox{#1{p}{P}lain\hologo{TeX}}%
}
%    \end{macrocode}
%    \end{macro}
%    \begin{macro}{\HoLogoCs@plainTeX@runtogether}
%    \begin{macrocode}
\def\HoLogoCs@plainTeX@runtogether#1{#1{p}{P}lainTeX}
%    \end{macrocode}
%    \end{macro}
%    \begin{macro}{\HoLogoBkm@plainTeX@runtogether}
%    \begin{macrocode}
\def\HoLogoBkm@plainTeX@runtogether#1{%
  #1{p}{P}lain\hologo{TeX}%
}
%    \end{macrocode}
%    \end{macro}
%    \begin{macro}{\HoLogoHtml@plainTeX@runtogether}
%    \begin{macrocode}
\def\HoLogoHtml@plainTeX@runtogether#1{%
  #1{p}{P}lain\hologo{TeX}%
}
%    \end{macrocode}
%    \end{macro}
%
%    \begin{macro}{\HoLogo@plainTeX}
%    \begin{macrocode}
\def\HoLogo@plainTeX{\HoLogo@plainTeX@space}
%    \end{macrocode}
%    \end{macro}
%    \begin{macro}{\HoLogoCs@plainTeX}
%    \begin{macrocode}
\def\HoLogoCs@plainTeX{\HoLogoCs@plainTeX@space}
%    \end{macrocode}
%    \end{macro}
%    \begin{macro}{\HoLogoBkm@plainTeX}
%    \begin{macrocode}
\def\HoLogoBkm@plainTeX{\HoLogoBkm@plainTeX@space}
%    \end{macrocode}
%    \end{macro}
%    \begin{macro}{\HoLogoHtml@plainTeX}
%    \begin{macrocode}
\def\HoLogoHtml@plainTeX{\HoLogoHtml@plainTeX@space}
%    \end{macrocode}
%    \end{macro}
%
% \subsubsection{\hologo{LaTeX}}
%
%    Source: \hologo{LaTeX} kernel.
%\begin{quote}
%\begin{verbatim}
%\DeclareRobustCommand{\LaTeX}{%
%  L%
%  \kern-.36em%
%  {%
%    \sbox\z@ T%
%    \vbox to\ht\z@{%
%      \hbox{%
%        \check@mathfonts
%        \fontsize\sf@size\z@
%        \math@fontsfalse
%        \selectfont
%        A%
%      }%
%      \vss
%    }%
%  }%
%  \kern-.15em%
%  \TeX
%}
%\end{verbatim}
%\end{quote}
%
%    \begin{macro}{\HoLogo@La}
%    \begin{macrocode}
\def\HoLogo@La#1{%
  L%
  \kern-.36em%
  \begingroup
    \setbox\ltx@zero\hbox{T}%
    \vbox to\ht\ltx@zero{%
      \hbox{%
        \ltx@ifundefined{check@mathfonts}{%
          \csname sevenrm\endcsname
        }{%
          \check@mathfonts
          \fontsize\sf@size{0pt}%
          \math@fontsfalse\selectfont
        }%
        A%
      }%
      \vss
    }%
  \endgroup
}
%    \end{macrocode}
%    \end{macro}
%
%    \begin{macro}{\HoLogo@LaTeX}
%    Source: \hologo{LaTeX} kernel.
%    \begin{macrocode}
\def\HoLogo@LaTeX#1{%
  \hologo{La}%
  \kern-.15em%
  \hologo{TeX}%
}
%    \end{macrocode}
%    \end{macro}
%    \begin{macro}{\HoLogoHtml@LaTeX}
%    \begin{macrocode}
\def\HoLogoHtml@LaTeX#1{%
  \HoLogoCss@LaTeX
  \HOLOGO@Span{LaTeX}{%
    L%
    \HOLOGO@Span{a}{%
      A%
    }%
    \hologo{TeX}%
  }%
}
%    \end{macrocode}
%    \end{macro}
%    \begin{macro}{\HoLogoCss@LaTeX}
%    \begin{macrocode}
\def\HoLogoCss@LaTeX{%
  \Css{%
    span.HoLogo-LaTeX span.HoLogo-a{%
      position:relative;%
      top:-.5ex;%
      margin-left:-.36em;%
      margin-right:-.15em;%
      font-size:85\%;%
    }%
  }%
  \global\let\HoLogoCss@LaTeX\relax
}
%    \end{macrocode}
%    \end{macro}
%
% \subsubsection{\hologo{(La)TeX}}
%
%    \begin{macro}{\HoLogo@LaTeXTeX}
%    The kerning around the parentheses is taken
%    from package \xpackage{dtklogos} \cite{dtklogos}.
%\begin{quote}
%\begin{verbatim}
%\DeclareRobustCommand{\LaTeXTeX}{%
%  (%
%  \kern-.15em%
%  L%
%  \kern-.36em%
%  {%
%    \sbox\z@ T%
%    \vbox to\ht0{%
%      \hbox{%
%        $\m@th$%
%        \csname S@\f@size\endcsname
%        \fontsize\sf@size\z@
%        \math@fontsfalse
%        \selectfont
%        A%
%      }%
%      \vss
%    }%
%  }%
%  \kern-.2em%
%  )%
%  \kern-.15em%
%  \TeX
%}
%\end{verbatim}
%\end{quote}
%    \begin{macrocode}
\def\HoLogo@LaTeXTeX#1{%
  (%
  \kern-.15em%
  \hologo{La}%
  \kern-.2em%
  )%
  \kern-.15em%
  \hologo{TeX}%
}
%    \end{macrocode}
%    \end{macro}
%    \begin{macro}{\HoLogoBkm@LaTeXTeX}
%    \begin{macrocode}
\def\HoLogoBkm@LaTeXTeX#1{(La)TeX}
%    \end{macrocode}
%    \end{macro}
%
%    \begin{macro}{\HoLogo@(La)TeX}
%    \begin{macrocode}
\expandafter
\let\csname HoLogo@(La)TeX\endcsname\HoLogo@LaTeXTeX
%    \end{macrocode}
%    \end{macro}
%    \begin{macro}{\HoLogoBkm@(La)TeX}
%    \begin{macrocode}
\expandafter
\let\csname HoLogoBkm@(La)TeX\endcsname\HoLogoBkm@LaTeXTeX
%    \end{macrocode}
%    \end{macro}
%    \begin{macro}{\HoLogoHtml@LaTeXTeX}
%    \begin{macrocode}
\def\HoLogoHtml@LaTeXTeX#1{%
  \HoLogoCss@LaTeXTeX
  \HOLOGO@Span{LaTeXTeX}{%
    (%
    \HOLOGO@Span{L}{L}%
    \HOLOGO@Span{a}{A}%
    \HOLOGO@Span{ParenRight}{)}%
    \hologo{TeX}%
  }%
}
%    \end{macrocode}
%    \end{macro}
%    \begin{macro}{\HoLogoHtml@(La)TeX}
%    Kerning after opening parentheses and before closing parentheses
%    is $-0.1$\,em. The original values $-0.15$\,em
%    looked too ugly for a serif font.
%    \begin{macrocode}
\expandafter
\let\csname HoLogoHtml@(La)TeX\endcsname\HoLogoHtml@LaTeXTeX
%    \end{macrocode}
%    \end{macro}
%    \begin{macro}{\HoLogoCss@LaTeXTeX}
%    \begin{macrocode}
\def\HoLogoCss@LaTeXTeX{%
  \Css{%
    span.HoLogo-LaTeXTeX span.HoLogo-L{%
      margin-left:-.1em;%
    }%
  }%
  \Css{%
    span.HoLogo-LaTeXTeX span.HoLogo-a{%
      position:relative;%
      top:-.5ex;%
      margin-left:-.36em;%
      margin-right:-.1em;%
      font-size:85\%;%
    }%
  }%
  \Css{%
    span.HoLogo-LaTeXTeX span.HoLogo-ParenRight{%
      margin-right:-.15em;%
    }%
  }%
  \global\let\HoLogoCss@LaTeXTeX\relax
}
%    \end{macrocode}
%    \end{macro}
%
% \subsubsection{\hologo{LaTeXe}}
%
%    \begin{macro}{\HoLogo@LaTeXe}
%    Source: \hologo{LaTeX} kernel
%    \begin{macrocode}
\def\HoLogo@LaTeXe#1{%
  \hologo{LaTeX}%
  \kern.15em%
  \hbox{%
    \HOLOGO@MathSetup
    2%
    $_{\textstyle\varepsilon}$%
  }%
}
%    \end{macrocode}
%    \end{macro}
%
%    \begin{macro}{\HoLogoCs@LaTeXe}
%    \begin{macrocode}
\ifnum64=`\^^^^0040\relax % test for big chars of LuaTeX/XeTeX
  \catcode`\$=9 %
  \catcode`\&=14 %
\else
  \catcode`\$=14 %
  \catcode`\&=9 %
\fi
\def\HoLogoCs@LaTeXe#1{%
  LaTeX2%
$ \string ^^^^0395%
& e%
}%
\catcode`\$=3 %
\catcode`\&=4 %
%    \end{macrocode}
%    \end{macro}
%
%    \begin{macro}{\HoLogoBkm@LaTeXe}
%    \begin{macrocode}
\def\HoLogoBkm@LaTeXe#1{%
  \hologo{LaTeX}%
  2%
  \HOLOGO@PdfdocUnicode{e}{\textepsilon}%
}
%    \end{macrocode}
%    \end{macro}
%
%    \begin{macro}{\HoLogoHtml@LaTeXe}
%    \begin{macrocode}
\def\HoLogoHtml@LaTeXe#1{%
  \HoLogoCss@LaTeXe
  \HOLOGO@Span{LaTeX2e}{%
    \hologo{LaTeX}%
    \HOLOGO@Span{2}{2}%
    \HOLOGO@Span{e}{%
      \HOLOGO@MathSetup
      \ensuremath{\textstyle\varepsilon}%
    }%
  }%
}
%    \end{macrocode}
%    \end{macro}
%    \begin{macro}{\HoLogoCss@LaTeXe}
%    \begin{macrocode}
\def\HoLogoCss@LaTeXe{%
  \Css{%
    span.HoLogo-LaTeX2e span.HoLogo-2{%
      padding-left:.15em;%
    }%
  }%
  \Css{%
    span.HoLogo-LaTeX2e span.HoLogo-e{%
      position:relative;%
      top:.35ex;%
      text-decoration:none;%
    }%
  }%
  \global\let\HoLogoCss@LaTeXe\relax
}
%    \end{macrocode}
%    \end{macro}
%
%    \begin{macro}{\HoLogo@LaTeX2e}
%    \begin{macrocode}
\expandafter
\let\csname HoLogo@LaTeX2e\endcsname\HoLogo@LaTeXe
%    \end{macrocode}
%    \end{macro}
%    \begin{macro}{\HoLogoCs@LaTeX2e}
%    \begin{macrocode}
\expandafter
\let\csname HoLogoCs@LaTeX2e\endcsname\HoLogoCs@LaTeXe
%    \end{macrocode}
%    \end{macro}
%    \begin{macro}{\HoLogoBkm@LaTeX2e}
%    \begin{macrocode}
\expandafter
\let\csname HoLogoBkm@LaTeX2e\endcsname\HoLogoBkm@LaTeXe
%    \end{macrocode}
%    \end{macro}
%    \begin{macro}{\HoLogoHtml@LaTeX2e}
%    \begin{macrocode}
\expandafter
\let\csname HoLogoHtml@LaTeX2e\endcsname\HoLogoHtml@LaTeXe
%    \end{macrocode}
%    \end{macro}
%
% \subsubsection{\hologo{LaTeX3}}
%
%    \begin{macro}{\HoLogo@LaTeX3}
%    Source: \hologo{LaTeX} kernel
%    \begin{macrocode}
\expandafter\def\csname HoLogo@LaTeX3\endcsname#1{%
  \hologo{LaTeX}%
  3%
}
%    \end{macrocode}
%    \end{macro}
%
%    \begin{macro}{\HoLogoBkm@LaTeX3}
%    \begin{macrocode}
\expandafter\def\csname HoLogoBkm@LaTeX3\endcsname#1{%
  \hologo{LaTeX}%
  3%
}
%    \end{macrocode}
%    \end{macro}
%    \begin{macro}{\HoLogoHtml@LaTeX3}
%    \begin{macrocode}
\expandafter
\let\csname HoLogoHtml@LaTeX3\expandafter\endcsname
\csname HoLogo@LaTeX3\endcsname
%    \end{macrocode}
%    \end{macro}
%
% \subsubsection{\hologo{LaTeXML}}
%
%    \begin{macro}{\HoLogo@LaTeXML}
%    \begin{macrocode}
\def\HoLogo@LaTeXML#1{%
  \HOLOGO@mbox{%
    \hologo{La}%
    \kern-.15em%
    T%
    \kern-.1667em%
    \lower.5ex\hbox{E}%
    \kern-.125em%
    \HoLogoFont@font{LaTeXML}{sc}{xml}%
  }%
}
%    \end{macrocode}
%    \end{macro}
%    \begin{macro}{\HoLogoHtml@pdfLaTeX}
%    \begin{macrocode}
\def\HoLogoHtml@LaTeXML#1{%
  \HOLOGO@Span{LaTeXML}{%
    \HoLogoCss@LaTeX
    \HoLogoCss@TeX
    \HOLOGO@Span{LaTeX}{%
      L%
      \HOLOGO@Span{a}{%
        A%
      }%
    }%
    \HOLOGO@Span{TeX}{%
      T%
      \HOLOGO@Span{e}{%
        E%
      }%
    }%
    \HCode{<span style="font-variant: small-caps;">}%
    xml%
    \HCode{</span>}%
  }%
}
%    \end{macrocode}
%    \end{macro}
%
% \subsubsection{\hologo{eTeX}}
%
%    \begin{macro}{\HoLogo@eTeX}
%    Source: package \xpackage{etex}
%    \begin{macrocode}
\def\HoLogo@eTeX#1{%
  \ltx@mbox{%
    \HOLOGO@MathSetup
    $\varepsilon$%
    -%
    \HOLOGO@NegativeKerning{-T,T-,To}%
    \hologo{TeX}%
  }%
}
%    \end{macrocode}
%    \end{macro}
%    \begin{macro}{\HoLogoCs@eTeX}
%    \begin{macrocode}
\ifnum64=`\^^^^0040\relax % test for big chars of LuaTeX/XeTeX
  \catcode`\$=9 %
  \catcode`\&=14 %
\else
  \catcode`\$=14 %
  \catcode`\&=9 %
\fi
\def\HoLogoCs@eTeX#1{%
$ #1{\string ^^^^0395}{\string ^^^^03b5}%
& #1{e}{E}%
  TeX%
}%
\catcode`\$=3 %
\catcode`\&=4 %
%    \end{macrocode}
%    \end{macro}
%    \begin{macro}{\HoLogoBkm@eTeX}
%    \begin{macrocode}
\def\HoLogoBkm@eTeX#1{%
  \HOLOGO@PdfdocUnicode{#1{e}{E}}{\textepsilon}%
  -%
  \hologo{TeX}%
}
%    \end{macrocode}
%    \end{macro}
%    \begin{macro}{\HoLogoHtml@eTeX}
%    \begin{macrocode}
\def\HoLogoHtml@eTeX#1{%
  \ltx@mbox{%
    \HOLOGO@MathSetup
    $\varepsilon$%
    -%
    \hologo{TeX}%
  }%
}
%    \end{macrocode}
%    \end{macro}
%
% \subsubsection{\hologo{iniTeX}}
%
%    \begin{macro}{\HoLogo@iniTeX}
%    \begin{macrocode}
\def\HoLogo@iniTeX#1{%
  \HOLOGO@mbox{%
    #1{i}{I}ni\hologo{TeX}%
  }%
}
%    \end{macrocode}
%    \end{macro}
%    \begin{macro}{\HoLogoCs@iniTeX}
%    \begin{macrocode}
\def\HoLogoCs@iniTeX#1{#1{i}{I}niTeX}
%    \end{macrocode}
%    \end{macro}
%    \begin{macro}{\HoLogoBkm@iniTeX}
%    \begin{macrocode}
\def\HoLogoBkm@iniTeX#1{%
  #1{i}{I}ni\hologo{TeX}%
}
%    \end{macrocode}
%    \end{macro}
%    \begin{macro}{\HoLogoHtml@iniTeX}
%    \begin{macrocode}
\let\HoLogoHtml@iniTeX\HoLogo@iniTeX
%    \end{macrocode}
%    \end{macro}
%
% \subsubsection{\hologo{virTeX}}
%
%    \begin{macro}{\HoLogo@virTeX}
%    \begin{macrocode}
\def\HoLogo@virTeX#1{%
  \HOLOGO@mbox{%
    #1{v}{V}ir\hologo{TeX}%
  }%
}
%    \end{macrocode}
%    \end{macro}
%    \begin{macro}{\HoLogoCs@virTeX}
%    \begin{macrocode}
\def\HoLogoCs@virTeX#1{#1{v}{V}irTeX}
%    \end{macrocode}
%    \end{macro}
%    \begin{macro}{\HoLogoBkm@virTeX}
%    \begin{macrocode}
\def\HoLogoBkm@virTeX#1{%
  #1{v}{V}ir\hologo{TeX}%
}
%    \end{macrocode}
%    \end{macro}
%    \begin{macro}{\HoLogoHtml@virTeX}
%    \begin{macrocode}
\let\HoLogoHtml@virTeX\HoLogo@virTeX
%    \end{macrocode}
%    \end{macro}
%
% \subsubsection{\hologo{SliTeX}}
%
% \paragraph{Definitions of the three variants.}
%
%    \begin{macro}{\HoLogo@SLiTeX@lift}
%    \begin{macrocode}
\def\HoLogo@SLiTeX@lift#1{%
  \HoLogoFont@font{SliTeX}{rm}{%
    S%
    \kern-.06em%
    L%
    \kern-.18em%
    \raise.32ex\hbox{\HoLogoFont@font{SliTeX}{sc}{i}}%
    \HOLOGO@discretionary
    \kern-.06em%
    \hologo{TeX}%
  }%
}
%    \end{macrocode}
%    \end{macro}
%    \begin{macro}{\HoLogoBkm@SLiTeX@lift}
%    \begin{macrocode}
\def\HoLogoBkm@SLiTeX@lift#1{SLiTeX}
%    \end{macrocode}
%    \end{macro}
%    \begin{macro}{\HoLogoHtml@SLiTeX@lift}
%    \begin{macrocode}
\def\HoLogoHtml@SLiTeX@lift#1{%
  \HoLogoCss@SLiTeX@lift
  \HOLOGO@Span{SLiTeX-lift}{%
    \HoLogoFont@font{SliTeX}{rm}{%
      S%
      \HOLOGO@Span{L}{L}%
      \HOLOGO@Span{i}{i}%
      \hologo{TeX}%
    }%
  }%
}
%    \end{macrocode}
%    \end{macro}
%    \begin{macro}{\HoLogoCss@SLiTeX@lift}
%    \begin{macrocode}
\def\HoLogoCss@SLiTeX@lift{%
  \Css{%
    span.HoLogo-SLiTeX-lift span.HoLogo-L{%
      margin-left:-.06em;%
      margin-right:-.18em;%
    }%
  }%
  \Css{%
    span.HoLogo-SLiTeX-lift span.HoLogo-i{%
      position:relative;%
      top:-.32ex;%
      margin-right:-.06em;%
      font-variant:small-caps;%
    }%
  }%
  \global\let\HoLogoCss@SLiTeX@lift\relax
}
%    \end{macrocode}
%    \end{macro}
%
%    \begin{macro}{\HoLogo@SliTeX@simple}
%    \begin{macrocode}
\def\HoLogo@SliTeX@simple#1{%
  \HoLogoFont@font{SliTeX}{rm}{%
    \ltx@mbox{%
      \HoLogoFont@font{SliTeX}{sc}{Sli}%
    }%
    \HOLOGO@discretionary
    \hologo{TeX}%
  }%
}
%    \end{macrocode}
%    \end{macro}
%    \begin{macro}{\HoLogoBkm@SliTeX@simple}
%    \begin{macrocode}
\def\HoLogoBkm@SliTeX@simple#1{SliTeX}
%    \end{macrocode}
%    \end{macro}
%    \begin{macro}{\HoLogoHtml@SliTeX@simple}
%    \begin{macrocode}
\let\HoLogoHtml@SliTeX@simple\HoLogo@SliTeX@simple
%    \end{macrocode}
%    \end{macro}
%
%    \begin{macro}{\HoLogo@SliTeX@narrow}
%    \begin{macrocode}
\def\HoLogo@SliTeX@narrow#1{%
  \HoLogoFont@font{SliTeX}{rm}{%
    \ltx@mbox{%
      S%
      \kern-.06em%
      \HoLogoFont@font{SliTeX}{sc}{%
        l%
        \kern-.035em%
        i%
      }%
    }%
    \HOLOGO@discretionary
    \kern-.06em%
    \hologo{TeX}%
  }%
}
%    \end{macrocode}
%    \end{macro}
%    \begin{macro}{\HoLogoBkm@SliTeX@narrow}
%    \begin{macrocode}
\def\HoLogoBkm@SliTeX@narrow#1{SliTeX}
%    \end{macrocode}
%    \end{macro}
%    \begin{macro}{\HoLogoHtml@SliTeX@narrow}
%    \begin{macrocode}
\def\HoLogoHtml@SliTeX@narrow#1{%
  \HoLogoCss@SliTeX@narrow
  \HOLOGO@Span{SliTeX-narrow}{%
    \HoLogoFont@font{SliTeX}{rm}{%
      S%
        \HOLOGO@Span{l}{l}%
        \HOLOGO@Span{i}{i}%
      \hologo{TeX}%
    }%
  }%
}
%    \end{macrocode}
%    \end{macro}
%    \begin{macro}{\HoLogoCss@SliTeX@narrow}
%    \begin{macrocode}
\def\HoLogoCss@SliTeX@narrow{%
  \Css{%
    span.HoLogo-SliTeX-narrow span.HoLogo-l{%
      margin-left:-.06em;%
      margin-right:-.035em;%
      font-variant:small-caps;%
    }%
  }%
  \Css{%
    span.HoLogo-SliTeX-narrow span.HoLogo-i{%
      margin-right:-.06em;%
      font-variant:small-caps;%
    }%
  }%
  \global\let\HoLogoCss@SliTeX@narrow\relax
}
%    \end{macrocode}
%    \end{macro}
%
% \paragraph{Macro set completion.}
%
%    \begin{macro}{\HoLogo@SLiTeX@simple}
%    \begin{macrocode}
\def\HoLogo@SLiTeX@simple{\HoLogo@SliTeX@simple}
%    \end{macrocode}
%    \end{macro}
%    \begin{macro}{\HoLogoBkm@SLiTeX@simple}
%    \begin{macrocode}
\def\HoLogoBkm@SLiTeX@simple{\HoLogoBkm@SliTeX@simple}
%    \end{macrocode}
%    \end{macro}
%    \begin{macro}{\HoLogoHtml@SLiTeX@simple}
%    \begin{macrocode}
\def\HoLogoHtml@SLiTeX@simple{\HoLogoHtml@SliTeX@simple}
%    \end{macrocode}
%    \end{macro}
%
%    \begin{macro}{\HoLogo@SLiTeX@narrow}
%    \begin{macrocode}
\def\HoLogo@SLiTeX@narrow{\HoLogo@SliTeX@narrow}
%    \end{macrocode}
%    \end{macro}
%    \begin{macro}{\HoLogoBkm@SLiTeX@narrow}
%    \begin{macrocode}
\def\HoLogoBkm@SLiTeX@narrow{\HoLogoBkm@SliTeX@narrow}
%    \end{macrocode}
%    \end{macro}
%    \begin{macro}{\HoLogoHtml@SLiTeX@narrow}
%    \begin{macrocode}
\def\HoLogoHtml@SLiTeX@narrow{\HoLogoHtml@SliTeX@narrow}
%    \end{macrocode}
%    \end{macro}
%
%    \begin{macro}{\HoLogo@SliTeX@lift}
%    \begin{macrocode}
\def\HoLogo@SliTeX@lift{\HoLogo@SLiTeX@lift}
%    \end{macrocode}
%    \end{macro}
%    \begin{macro}{\HoLogoBkm@SliTeX@lift}
%    \begin{macrocode}
\def\HoLogoBkm@SliTeX@lift{\HoLogoBkm@SLiTeX@lift}
%    \end{macrocode}
%    \end{macro}
%    \begin{macro}{\HoLogoHtml@SliTeX@lift}
%    \begin{macrocode}
\def\HoLogoHtml@SliTeX@lift{\HoLogoHtml@SLiTeX@lift}
%    \end{macrocode}
%    \end{macro}
%
% \paragraph{Defaults.}
%
%    \begin{macro}{\HoLogo@SLiTeX}
%    \begin{macrocode}
\def\HoLogo@SLiTeX{\HoLogo@SLiTeX@lift}
%    \end{macrocode}
%    \end{macro}
%    \begin{macro}{\HoLogoBkm@SLiTeX}
%    \begin{macrocode}
\def\HoLogoBkm@SLiTeX{\HoLogoBkm@SLiTeX@lift}
%    \end{macrocode}
%    \end{macro}
%    \begin{macro}{\HoLogoHtml@SLiTeX}
%    \begin{macrocode}
\def\HoLogoHtml@SLiTeX{\HoLogoHtml@SLiTeX@lift}
%    \end{macrocode}
%    \end{macro}
%
%    \begin{macro}{\HoLogo@SliTeX}
%    \begin{macrocode}
\def\HoLogo@SliTeX{\HoLogo@SliTeX@narrow}
%    \end{macrocode}
%    \end{macro}
%    \begin{macro}{\HoLogoBkm@SliTeX}
%    \begin{macrocode}
\def\HoLogoBkm@SliTeX{\HoLogoBkm@SliTeX@narrow}
%    \end{macrocode}
%    \end{macro}
%    \begin{macro}{\HoLogoHtml@SliTeX}
%    \begin{macrocode}
\def\HoLogoHtml@SliTeX{\HoLogoHtml@SliTeX@narrow}
%    \end{macrocode}
%    \end{macro}
%
% \subsubsection{\hologo{LuaTeX}}
%
%    \begin{macro}{\HoLogo@LuaTeX}
%    The kerning is an idea of Hans Hagen, see mailing list
%    `luatex at tug dot org' in March 2010.
%    \begin{macrocode}
\def\HoLogo@LuaTeX#1{%
  \HOLOGO@mbox{%
    Lua%
    \HOLOGO@NegativeKerning{aT,oT,To}%
    \hologo{TeX}%
  }%
}
%    \end{macrocode}
%    \end{macro}
%    \begin{macro}{\HoLogoHtml@LuaTeX}
%    \begin{macrocode}
\let\HoLogoHtml@LuaTeX\HoLogo@LuaTeX
%    \end{macrocode}
%    \end{macro}
%
% \subsubsection{\hologo{LuaLaTeX}}
%
%    \begin{macro}{\HoLogo@LuaLaTeX}
%    \begin{macrocode}
\def\HoLogo@LuaLaTeX#1{%
  \HOLOGO@mbox{%
    Lua%
    \hologo{LaTeX}%
  }%
}
%    \end{macrocode}
%    \end{macro}
%    \begin{macro}{\HoLogoHtml@LuaLaTeX}
%    \begin{macrocode}
\let\HoLogoHtml@LuaLaTeX\HoLogo@LuaLaTeX
%    \end{macrocode}
%    \end{macro}
%
% \subsubsection{\hologo{XeTeX}, \hologo{XeLaTeX}}
%
%    \begin{macro}{\HOLOGO@IfCharExists}
%    \begin{macrocode}
\ifluatex
  \ifnum\luatexversion<36 %
  \else
    \def\HOLOGO@IfCharExists#1{%
      \ifnum
        \directlua{%
           if luaotfload and luaotfload.aux then
             if luaotfload.aux.font_has_glyph(%
                    font.current(), \number#1) then % 	 
	       tex.print("1") % 	 
	     end % 	 
	   elseif font and font.fonts and font.current then %
            local f = font.fonts[font.current()]%
            if f.characters and f.characters[\number#1] then %
              tex.print("1")%
            end %
          end%
        }0=\ltx@zero
        \expandafter\ltx@secondoftwo
      \else
        \expandafter\ltx@firstoftwo
      \fi
    }%
  \fi
\fi
\ltx@IfUndefined{HOLOGO@IfCharExists}{%
  \def\HOLOGO@@IfCharExists#1{%
    \begingroup
      \tracinglostchars=\ltx@zero
      \setbox\ltx@zero=\hbox{%
        \kern7sp\char#1\relax
        \ifnum\lastkern>\ltx@zero
          \expandafter\aftergroup\csname iffalse\endcsname
        \else
          \expandafter\aftergroup\csname iftrue\endcsname
        \fi
      }%
      % \if{true|false} from \aftergroup
      \endgroup
      \expandafter\ltx@firstoftwo
    \else
      \endgroup
      \expandafter\ltx@secondoftwo
    \fi
  }%
  \ifxetex
    \ltx@IfUndefined{XeTeXfonttype}{}{%
      \ltx@IfUndefined{XeTeXcharglyph}{}{%
        \def\HOLOGO@IfCharExists#1{%
          \ifnum\XeTeXfonttype\font>\ltx@zero
            \expandafter\ltx@firstofthree
          \else
            \expandafter\ltx@gobble
          \fi
          {%
            \ifnum\XeTeXcharglyph#1>\ltx@zero
              \expandafter\ltx@firstoftwo
            \else
              \expandafter\ltx@secondoftwo
            \fi
          }%
          \HOLOGO@@IfCharExists{#1}%
        }%
      }%
    }%
  \fi
}{}
\ltx@ifundefined{HOLOGO@IfCharExists}{%
  \ifnum64=`\^^^^0040\relax % test for big chars of LuaTeX/XeTeX
    \let\HOLOGO@IfCharExists\HOLOGO@@IfCharExists
  \else
    \def\HOLOGO@IfCharExists#1{%
      \ifnum#1>255 %
        \expandafter\ltx@fourthoffour
      \fi
      \HOLOGO@@IfCharExists{#1}%
    }%
  \fi
}{}
%    \end{macrocode}
%    \end{macro}
%
%    \begin{macro}{\HoLogo@Xe}
%    Source: package \xpackage{dtklogos}
%    \begin{macrocode}
\def\HoLogo@Xe#1{%
  X%
  \kern-.1em\relax
  \HOLOGO@IfCharExists{"018E}{%
    \lower.5ex\hbox{\char"018E}%
  }{%
    \chardef\HOLOGO@choice=\ltx@zero
    \ifdim\fontdimen\ltx@one\font>0pt %
      \ltx@IfUndefined{rotatebox}{%
        \ltx@IfUndefined{pgftext}{%
          \ltx@IfUndefined{psscalebox}{%
            \ltx@IfUndefined{HOLOGO@ScaleBox@\hologoDriver}{%
            }{%
              \chardef\HOLOGO@choice=4 %
            }%
          }{%
            \chardef\HOLOGO@choice=3 %
          }%
        }{%
          \chardef\HOLOGO@choice=2 %
        }%
      }{%
        \chardef\HOLOGO@choice=1 %
      }%
      \ifcase\HOLOGO@choice
        \HOLOGO@WarningUnsupportedDriver{Xe}%
        e%
      \or % 1: \rotatebox
        \begingroup
          \setbox\ltx@zero\hbox{\rotatebox{180}{E}}%
          \ltx@LocDimenA=\dp\ltx@zero
          \advance\ltx@LocDimenA by -.5ex\relax
          \raise\ltx@LocDimenA\box\ltx@zero
        \endgroup
      \or % 2: \pgftext
        \lower.5ex\hbox{%
          \pgfpicture
            \pgftext[rotate=180]{E}%
          \endpgfpicture
        }%
      \or % 3: \psscalebox
        \begingroup
          \setbox\ltx@zero\hbox{\psscalebox{-1 -1}{E}}%
          \ltx@LocDimenA=\dp\ltx@zero
          \advance\ltx@LocDimenA by -.5ex\relax
          \raise\ltx@LocDimenA\box\ltx@zero
        \endgroup
      \or % 4: \HOLOGO@PointReflectBox
        \lower.5ex\hbox{\HOLOGO@PointReflectBox{E}}%
      \else
        \@PackageError{hologo}{Internal error (choice/it}\@ehc
      \fi
    \else
      \ltx@IfUndefined{reflectbox}{%
        \ltx@IfUndefined{pgftext}{%
          \ltx@IfUndefined{psscalebox}{%
            \ltx@IfUndefined{HOLOGO@ScaleBox@\hologoDriver}{%
            }{%
              \chardef\HOLOGO@choice=4 %
            }%
          }{%
            \chardef\HOLOGO@choice=3 %
          }%
        }{%
          \chardef\HOLOGO@choice=2 %
        }%
      }{%
        \chardef\HOLOGO@choice=1 %
      }%
      \ifcase\HOLOGO@choice
        \HOLOGO@WarningUnsupportedDriver{Xe}%
        e%
      \or % 1: reflectbox
        \lower.5ex\hbox{%
          \reflectbox{E}%
        }%
      \or % 2: \pgftext
        \lower.5ex\hbox{%
          \pgfpicture
            \pgftransformxscale{-1}%
            \pgftext{E}%
          \endpgfpicture
        }%
      \or % 3: \psscalebox
        \lower.5ex\hbox{%
          \psscalebox{-1 1}{E}%
        }%
      \or % 4: \HOLOGO@Reflectbox
        \lower.5ex\hbox{%
          \HOLOGO@ReflectBox{E}%
        }%
      \else
        \@PackageError{hologo}{Internal error (choice/up)}\@ehc
      \fi
    \fi
  }%
}
%    \end{macrocode}
%    \end{macro}
%    \begin{macro}{\HoLogoHtml@Xe}
%    \begin{macrocode}
\def\HoLogoHtml@Xe#1{%
  \HoLogoCss@Xe
  \HOLOGO@Span{Xe}{%
    X%
    \HOLOGO@Span{e}{%
      \HCode{&\ltx@hashchar x018e;}%
    }%
  }%
}
%    \end{macrocode}
%    \end{macro}
%    \begin{macro}{\HoLogoCss@Xe}
%    \begin{macrocode}
\def\HoLogoCss@Xe{%
  \Css{%
    span.HoLogo-Xe span.HoLogo-e{%
      position:relative;%
      top:.5ex;%
      left-margin:-.1em;%
    }%
  }%
  \global\let\HoLogoCss@Xe\relax
}
%    \end{macrocode}
%    \end{macro}
%
%    \begin{macro}{\HoLogo@XeTeX}
%    \begin{macrocode}
\def\HoLogo@XeTeX#1{%
  \hologo{Xe}%
  \kern-.15em\relax
  \hologo{TeX}%
}
%    \end{macrocode}
%    \end{macro}
%
%    \begin{macro}{\HoLogoHtml@XeTeX}
%    \begin{macrocode}
\def\HoLogoHtml@XeTeX#1{%
  \HoLogoCss@XeTeX
  \HOLOGO@Span{XeTeX}{%
    \hologo{Xe}%
    \hologo{TeX}%
  }%
}
%    \end{macrocode}
%    \end{macro}
%    \begin{macro}{\HoLogoCss@XeTeX}
%    \begin{macrocode}
\def\HoLogoCss@XeTeX{%
  \Css{%
    span.HoLogo-XeTeX span.HoLogo-TeX{%
      margin-left:-.15em;%
    }%
  }%
  \global\let\HoLogoCss@XeTeX\relax
}
%    \end{macrocode}
%    \end{macro}
%
%    \begin{macro}{\HoLogo@XeLaTeX}
%    \begin{macrocode}
\def\HoLogo@XeLaTeX#1{%
  \hologo{Xe}%
  \kern-.13em%
  \hologo{LaTeX}%
}
%    \end{macrocode}
%    \end{macro}
%    \begin{macro}{\HoLogoHtml@XeLaTeX}
%    \begin{macrocode}
\def\HoLogoHtml@XeLaTeX#1{%
  \HoLogoCss@XeLaTeX
  \HOLOGO@Span{XeLaTeX}{%
    \hologo{Xe}%
    \hologo{LaTeX}%
  }%
}
%    \end{macrocode}
%    \end{macro}
%    \begin{macro}{\HoLogoCss@XeLaTeX}
%    \begin{macrocode}
\def\HoLogoCss@XeLaTeX{%
  \Css{%
    span.HoLogo-XeLaTeX span.HoLogo-Xe{%
      margin-right:-.13em;%
    }%
  }%
  \global\let\HoLogoCss@XeLaTeX\relax
}
%    \end{macrocode}
%    \end{macro}
%
% \subsubsection{\hologo{pdfTeX}, \hologo{pdfLaTeX}}
%
%    \begin{macro}{\HoLogo@pdfTeX}
%    \begin{macrocode}
\def\HoLogo@pdfTeX#1{%
  \HOLOGO@mbox{%
    #1{p}{P}df\hologo{TeX}%
  }%
}
%    \end{macrocode}
%    \end{macro}
%    \begin{macro}{\HoLogoCs@pdfTeX}
%    \begin{macrocode}
\def\HoLogoCs@pdfTeX#1{#1{p}{P}dfTeX}
%    \end{macrocode}
%    \end{macro}
%    \begin{macro}{\HoLogoBkm@pdfTeX}
%    \begin{macrocode}
\def\HoLogoBkm@pdfTeX#1{%
  #1{p}{P}df\hologo{TeX}%
}
%    \end{macrocode}
%    \end{macro}
%    \begin{macro}{\HoLogoHtml@pdfTeX}
%    \begin{macrocode}
\let\HoLogoHtml@pdfTeX\HoLogo@pdfTeX
%    \end{macrocode}
%    \end{macro}
%
%    \begin{macro}{\HoLogo@pdfLaTeX}
%    \begin{macrocode}
\def\HoLogo@pdfLaTeX#1{%
  \HOLOGO@mbox{%
    #1{p}{P}df\hologo{LaTeX}%
  }%
}
%    \end{macrocode}
%    \end{macro}
%    \begin{macro}{\HoLogoCs@pdfLaTeX}
%    \begin{macrocode}
\def\HoLogoCs@pdfLaTeX#1{#1{p}{P}dfLaTeX}
%    \end{macrocode}
%    \end{macro}
%    \begin{macro}{\HoLogoBkm@pdfLaTeX}
%    \begin{macrocode}
\def\HoLogoBkm@pdfLaTeX#1{%
  #1{p}{P}df\hologo{LaTeX}%
}
%    \end{macrocode}
%    \end{macro}
%    \begin{macro}{\HoLogoHtml@pdfLaTeX}
%    \begin{macrocode}
\let\HoLogoHtml@pdfLaTeX\HoLogo@pdfLaTeX
%    \end{macrocode}
%    \end{macro}
%
% \subsubsection{\hologo{VTeX}}
%
%    \begin{macro}{\HoLogo@VTeX}
%    \begin{macrocode}
\def\HoLogo@VTeX#1{%
  \HOLOGO@mbox{%
    V\hologo{TeX}%
  }%
}
%    \end{macrocode}
%    \end{macro}
%    \begin{macro}{\HoLogoHtml@VTeX}
%    \begin{macrocode}
\let\HoLogoHtml@VTeX\HoLogo@VTeX
%    \end{macrocode}
%    \end{macro}
%
% \subsubsection{\hologo{AmS}, \dots}
%
%    Source: class \xclass{amsdtx}
%
%    \begin{macro}{\HoLogo@AmS}
%    \begin{macrocode}
\def\HoLogo@AmS#1{%
  \HoLogoFont@font{AmS}{sy}{%
    A%
    \kern-.1667em%
    \lower.5ex\hbox{M}%
    \kern-.125em%
    S%
  }%
}
%    \end{macrocode}
%    \end{macro}
%    \begin{macro}{\HoLogoBkm@AmS}
%    \begin{macrocode}
\def\HoLogoBkm@AmS#1{AmS}
%    \end{macrocode}
%    \end{macro}
%    \begin{macro}{\HoLogoHtml@AmS}
%    \begin{macrocode}
\def\HoLogoHtml@AmS#1{%
  \HoLogoCss@AmS
%  \HoLogoFont@font{AmS}{sy}{%
    \HOLOGO@Span{AmS}{%
      A%
      \HOLOGO@Span{M}{M}%
      S%
    }%
%   }%
}
%    \end{macrocode}
%    \end{macro}
%    \begin{macro}{\HoLogoCss@AmS}
%    \begin{macrocode}
\def\HoLogoCss@AmS{%
  \Css{%
    span.HoLogo-AmS span.HoLogo-M{%
      position:relative;%
      top:.5ex;%
      margin-left:-.1667em;%
      margin-right:-.125em;%
      text-decoration:none;%
    }%
  }%
  \global\let\HoLogoCss@AmS\relax
}
%    \end{macrocode}
%    \end{macro}
%
%    \begin{macro}{\HoLogo@AmSTeX}
%    \begin{macrocode}
\def\HoLogo@AmSTeX#1{%
  \hologo{AmS}%
  \HOLOGO@hyphen
  \hologo{TeX}%
}
%    \end{macrocode}
%    \end{macro}
%    \begin{macro}{\HoLogoBkm@AmSTeX}
%    \begin{macrocode}
\def\HoLogoBkm@AmSTeX#1{AmS-TeX}%
%    \end{macrocode}
%    \end{macro}
%    \begin{macro}{\HoLogoHtml@AmSTeX}
%    \begin{macrocode}
\let\HoLogoHtml@AmSTeX\HoLogo@AmSTeX
%    \end{macrocode}
%    \end{macro}
%
%    \begin{macro}{\HoLogo@AmSLaTeX}
%    \begin{macrocode}
\def\HoLogo@AmSLaTeX#1{%
  \hologo{AmS}%
  \HOLOGO@hyphen
  \hologo{LaTeX}%
}
%    \end{macrocode}
%    \end{macro}
%    \begin{macro}{\HoLogoBkm@AmSLaTeX}
%    \begin{macrocode}
\def\HoLogoBkm@AmSLaTeX#1{AmS-LaTeX}%
%    \end{macrocode}
%    \end{macro}
%    \begin{macro}{\HoLogoHtml@AmSLaTeX}
%    \begin{macrocode}
\let\HoLogoHtml@AmSLaTeX\HoLogo@AmSLaTeX
%    \end{macrocode}
%    \end{macro}
%
% \subsubsection{\hologo{BibTeX}}
%
%    \begin{macro}{\HoLogo@BibTeX@sc}
%    A definition of \hologo{BibTeX} is provided in
%    the documentation source for the manual of \hologo{BibTeX}
%    \cite{btxdoc}.
%\begin{quote}
%\begin{verbatim}
%\def\BibTeX{%
%  {%
%    \rm
%    B%
%    \kern-.05em%
%    {%
%      \sc
%      i%
%      \kern-.025em %
%      b%
%    }%
%    \kern-.08em
%    T%
%    \kern-.1667em%
%    \lower.7ex\hbox{E}%
%    \kern-.125em%
%    X%
%  }%
%}
%\end{verbatim}
%\end{quote}
%    \begin{macrocode}
\def\HoLogo@BibTeX@sc#1{%
  B%
  \kern-.05em%
  \HoLogoFont@font{BibTeX}{sc}{%
    i%
    \kern-.025em%
    b%
  }%
  \HOLOGO@discretionary
  \kern-.08em%
  \hologo{TeX}%
}
%    \end{macrocode}
%    \end{macro}
%    \begin{macro}{\HoLogoHtml@BibTeX@sc}
%    \begin{macrocode}
\def\HoLogoHtml@BibTeX@sc#1{%
  \HoLogoCss@BibTeX@sc
  \HOLOGO@Span{BibTeX-sc}{%
    B%
    \HOLOGO@Span{i}{i}%
    \HOLOGO@Span{b}{b}%
    \hologo{TeX}%
  }%
}
%    \end{macrocode}
%    \end{macro}
%    \begin{macro}{\HoLogoCss@BibTeX@sc}
%    \begin{macrocode}
\def\HoLogoCss@BibTeX@sc{%
  \Css{%
    span.HoLogo-BibTeX-sc span.HoLogo-i{%
      margin-left:-.05em;%
      margin-right:-.025em;%
      font-variant:small-caps;%
    }%
  }%
  \Css{%
    span.HoLogo-BibTeX-sc span.HoLogo-b{%
      margin-right:-.08em;%
      font-variant:small-caps;%
    }%
  }%
  \global\let\HoLogoCss@BibTeX@sc\relax
}
%    \end{macrocode}
%    \end{macro}
%
%    \begin{macro}{\HoLogo@BibTeX@sf}
%    Variant \xoption{sf} avoids trouble with unavailable
%    small caps fonts (e.g., bold versions of Computer Modern or
%    Latin Modern). The definition is taken from
%    package \xpackage{dtklogos} \cite{dtklogos}.
%\begin{quote}
%\begin{verbatim}
%\DeclareRobustCommand{\BibTeX}{%
%  B%
%  \kern-.05em%
%  \hbox{%
%    $\m@th$% %% force math size calculations
%    \csname S@\f@size\endcsname
%    \fontsize\sf@size\z@
%    \math@fontsfalse
%    \selectfont
%    I%
%    \kern-.025em%
%    B
%  }%
%  \kern-.08em%
%  \-%
%  \TeX
%}
%\end{verbatim}
%\end{quote}
%    \begin{macrocode}
\def\HoLogo@BibTeX@sf#1{%
  B%
  \kern-.05em%
  \HoLogoFont@font{BibTeX}{bibsf}{%
    I%
    \kern-.025em%
    B%
  }%
  \HOLOGO@discretionary
  \kern-.08em%
  \hologo{TeX}%
}
%    \end{macrocode}
%    \end{macro}
%    \begin{macro}{\HoLogoHtml@BibTeX@sf}
%    \begin{macrocode}
\def\HoLogoHtml@BibTeX@sf#1{%
  \HoLogoCss@BibTeX@sf
  \HOLOGO@Span{BibTeX-sf}{%
    B%
    \HoLogoFont@font{BibTeX}{bibsf}{%
      \HOLOGO@Span{i}{I}%
      B%
    }%
    \hologo{TeX}%
  }%
}
%    \end{macrocode}
%    \end{macro}
%    \begin{macro}{\HoLogoCss@BibTeX@sf}
%    \begin{macrocode}
\def\HoLogoCss@BibTeX@sf{%
  \Css{%
    span.HoLogo-BibTeX-sf span.HoLogo-i{%
      margin-left:-.05em;%
      margin-right:-.025em;%
    }%
  }%
  \Css{%
    span.HoLogo-BibTeX-sf span.HoLogo-TeX{%
      margin-left:-.08em;%
    }%
  }%
  \global\let\HoLogoCss@BibTeX@sf\relax
}
%    \end{macrocode}
%    \end{macro}
%
%    \begin{macro}{\HoLogo@BibTeX}
%    \begin{macrocode}
\def\HoLogo@BibTeX{\HoLogo@BibTeX@sf}
%    \end{macrocode}
%    \end{macro}
%    \begin{macro}{\HoLogoHtml@BibTeX}
%    \begin{macrocode}
\def\HoLogoHtml@BibTeX{\HoLogoHtml@BibTeX@sf}
%    \end{macrocode}
%    \end{macro}
%
% \subsubsection{\hologo{BibTeX8}}
%
%    \begin{macro}{\HoLogo@BibTeX8}
%    \begin{macrocode}
\expandafter\def\csname HoLogo@BibTeX8\endcsname#1{%
  \hologo{BibTeX}%
  8%
}
%    \end{macrocode}
%    \end{macro}
%
%    \begin{macro}{\HoLogoBkm@BibTeX8}
%    \begin{macrocode}
\expandafter\def\csname HoLogoBkm@BibTeX8\endcsname#1{%
  \hologo{BibTeX}%
  8%
}
%    \end{macrocode}
%    \end{macro}
%    \begin{macro}{\HoLogoHtml@BibTeX8}
%    \begin{macrocode}
\expandafter
\let\csname HoLogoHtml@BibTeX8\expandafter\endcsname
\csname HoLogo@BibTeX8\endcsname
%    \end{macrocode}
%    \end{macro}
%
% \subsubsection{\hologo{ConTeXt}}
%
%    \begin{macro}{\HoLogo@ConTeXt@simple}
%    \begin{macrocode}
\def\HoLogo@ConTeXt@simple#1{%
  \HOLOGO@mbox{Con}%
  \HOLOGO@discretionary
  \HOLOGO@mbox{\hologo{TeX}t}%
}
%    \end{macrocode}
%    \end{macro}
%    \begin{macro}{\HoLogoHtml@ConTeXt@simple}
%    \begin{macrocode}
\let\HoLogoHtml@ConTeXt@simple\HoLogo@ConTeXt@simple
%    \end{macrocode}
%    \end{macro}
%
%    \begin{macro}{\HoLogo@ConTeXt@narrow}
%    This definition of logo \hologo{ConTeXt} with variant \xoption{narrow}
%    comes from TUGboat's class \xclass{ltugboat} (version 2010/11/15 v2.8).
%    \begin{macrocode}
\def\HoLogo@ConTeXt@narrow#1{%
  \HOLOGO@mbox{C\kern-.0333emon}%
  \HOLOGO@discretionary
  \kern-.0667em%
  \HOLOGO@mbox{\hologo{TeX}\kern-.0333emt}%
}
%    \end{macrocode}
%    \end{macro}
%    \begin{macro}{\HoLogoHtml@ConTeXt@narrow}
%    \begin{macrocode}
\def\HoLogoHtml@ConTeXt@narrow#1{%
  \HoLogoCss@ConTeXt@narrow
  \HOLOGO@Span{ConTeXt-narrow}{%
    \HOLOGO@Span{C}{C}%
    on%
    \hologo{TeX}%
    t%
  }%
}
%    \end{macrocode}
%    \end{macro}
%    \begin{macro}{\HoLogoCss@ConTeXt@narrow}
%    \begin{macrocode}
\def\HoLogoCss@ConTeXt@narrow{%
  \Css{%
    span.HoLogo-ConTeXt-narrow span.HoLogo-C{%
      margin-left:-.0333em;%
    }%
  }%
  \Css{%
    span.HoLogo-ConTeXt-narrow span.HoLogo-TeX{%
      margin-left:-.0667em;%
      margin-right:-.0333em;%
    }%
  }%
  \global\let\HoLogoCss@ConTeXt@narrow\relax
}
%    \end{macrocode}
%    \end{macro}
%
%    \begin{macro}{\HoLogo@ConTeXt}
%    \begin{macrocode}
\def\HoLogo@ConTeXt{\HoLogo@ConTeXt@narrow}
%    \end{macrocode}
%    \end{macro}
%    \begin{macro}{\HoLogoHtml@ConTeXt}
%    \begin{macrocode}
\def\HoLogoHtml@ConTeXt{\HoLogoHtml@ConTeXt@narrow}
%    \end{macrocode}
%    \end{macro}
%
% \subsubsection{\hologo{emTeX}}
%
%    \begin{macro}{\HoLogo@emTeX}
%    \begin{macrocode}
\def\HoLogo@emTeX#1{%
  \HOLOGO@mbox{#1{e}{E}m}%
  \HOLOGO@discretionary
  \hologo{TeX}%
}
%    \end{macrocode}
%    \end{macro}
%    \begin{macro}{\HoLogoCs@emTeX}
%    \begin{macrocode}
\def\HoLogoCs@emTeX#1{#1{e}{E}mTeX}%
%    \end{macrocode}
%    \end{macro}
%    \begin{macro}{\HoLogoBkm@emTeX}
%    \begin{macrocode}
\def\HoLogoBkm@emTeX#1{%
  #1{e}{E}m\hologo{TeX}%
}
%    \end{macrocode}
%    \end{macro}
%    \begin{macro}{\HoLogoHtml@emTeX}
%    \begin{macrocode}
\let\HoLogoHtml@emTeX\HoLogo@emTeX
%    \end{macrocode}
%    \end{macro}
%
% \subsubsection{\hologo{ExTeX}}
%
%    \begin{macro}{\HoLogo@ExTeX}
%    The definition is taken from the FAQ of the
%    project \hologo{ExTeX}
%    \cite{ExTeX-FAQ}.
%\begin{quote}
%\begin{verbatim}
%\def\ExTeX{%
%  \textrm{% Logo always with serifs
%    \ensuremath{%
%      \textstyle
%      \varepsilon_{%
%        \kern-0.15em%
%        \mathcal{X}%
%      }%
%    }%
%    \kern-.15em%
%    \TeX
%  }%
%}
%\end{verbatim}
%\end{quote}
%    \begin{macrocode}
\def\HoLogo@ExTeX#1{%
  \HoLogoFont@font{ExTeX}{rm}{%
    \ltx@mbox{%
      \HOLOGO@MathSetup
      $%
        \textstyle
        \varepsilon_{%
          \kern-0.15em%
          \HoLogoFont@font{ExTeX}{sy}{X}%
        }%
      $%
    }%
    \HOLOGO@discretionary
    \kern-.15em%
    \hologo{TeX}%
  }%
}
%    \end{macrocode}
%    \end{macro}
%    \begin{macro}{\HoLogoHtml@ExTeX}
%    \begin{macrocode}
\def\HoLogoHtml@ExTeX#1{%
  \HoLogoCss@ExTeX
  \HoLogoFont@font{ExTeX}{rm}{%
    \HOLOGO@Span{ExTeX}{%
      \ltx@mbox{%
        \HOLOGO@MathSetup
        $\textstyle\varepsilon$%
        \HOLOGO@Span{X}{$\textstyle\chi$}%
        \hologo{TeX}%
      }%
    }%
  }%
}
%    \end{macrocode}
%    \end{macro}
%    \begin{macro}{\HoLogoBkm@ExTeX}
%    \begin{macrocode}
\def\HoLogoBkm@ExTeX#1{%
  \HOLOGO@PdfdocUnicode{#1{e}{E}x}{\textepsilon\textchi}%
  \hologo{TeX}%
}
%    \end{macrocode}
%    \end{macro}
%    \begin{macro}{\HoLogoCss@ExTeX}
%    \begin{macrocode}
\def\HoLogoCss@ExTeX{%
  \Css{%
    span.HoLogo-ExTeX{%
      font-family:serif;%
    }%
  }%
  \Css{%
    span.HoLogo-ExTeX span.HoLogo-TeX{%
      margin-left:-.15em;%
    }%
  }%
  \global\let\HoLogoCss@ExTeX\relax
}
%    \end{macrocode}
%    \end{macro}
%
% \subsubsection{\hologo{MiKTeX}}
%
%    \begin{macro}{\HoLogo@MiKTeX}
%    \begin{macrocode}
\def\HoLogo@MiKTeX#1{%
  \HOLOGO@mbox{MiK}%
  \HOLOGO@discretionary
  \hologo{TeX}%
}
%    \end{macrocode}
%    \end{macro}
%    \begin{macro}{\HoLogoHtml@MiKTeX}
%    \begin{macrocode}
\let\HoLogoHtml@MiKTeX\HoLogo@MiKTeX
%    \end{macrocode}
%    \end{macro}
%
% \subsubsection{\hologo{OzTeX} and friends}
%
%    Source: \hologo{OzTeX} FAQ \cite{OzTeX}:
%    \begin{quote}
%      |\def\OzTeX{O\kern-.03em z\kern-.15em\TeX}|\\
%      (There is no kerning in OzMF, OzMP and OzTtH.)
%    \end{quote}
%
%    \begin{macro}{\HoLogo@OzTeX}
%    \begin{macrocode}
\def\HoLogo@OzTeX#1{%
  O%
  \kern-.03em %
  z%
  \kern-.15em %
  \hologo{TeX}%
}
%    \end{macrocode}
%    \end{macro}
%    \begin{macro}{\HoLogoHtml@OzTeX}
%    \begin{macrocode}
\def\HoLogoHtml@OzTeX#1{%
  \HoLogoCss@OzTeX
  \HOLOGO@Span{OzTeX}{%
    O%
    \HOLOGO@Span{z}{z}%
    \hologo{TeX}%
  }%
}
%    \end{macrocode}
%    \end{macro}
%    \begin{macro}{\HoLogoCss@OzTeX}
%    \begin{macrocode}
\def\HoLogoCss@OzTeX{%
  \Css{%
    span.HoLogo-OzTeX span.HoLogo-z{%
      margin-left:-.03em;%
      margin-right:-.15em;%
    }%
  }%
  \global\let\HoLogoCss@OzTeX\relax
}
%    \end{macrocode}
%    \end{macro}
%
%    \begin{macro}{\HoLogo@OzMF}
%    \begin{macrocode}
\def\HoLogo@OzMF#1{%
  \HOLOGO@mbox{OzMF}%
}
%    \end{macrocode}
%    \end{macro}
%    \begin{macro}{\HoLogo@OzMP}
%    \begin{macrocode}
\def\HoLogo@OzMP#1{%
  \HOLOGO@mbox{OzMP}%
}
%    \end{macrocode}
%    \end{macro}
%    \begin{macro}{\HoLogo@OzTtH}
%    \begin{macrocode}
\def\HoLogo@OzTtH#1{%
  \HOLOGO@mbox{OzTtH}%
}
%    \end{macrocode}
%    \end{macro}
%
% \subsubsection{\hologo{PCTeX}}
%
%    \begin{macro}{\HoLogo@PCTeX}
%    \begin{macrocode}
\def\HoLogo@PCTeX#1{%
  \HOLOGO@mbox{PC}%
  \hologo{TeX}%
}
%    \end{macrocode}
%    \end{macro}
%    \begin{macro}{\HoLogoHtml@PCTeX}
%    \begin{macrocode}
\let\HoLogoHtml@PCTeX\HoLogo@PCTeX
%    \end{macrocode}
%    \end{macro}
%
% \subsubsection{\hologo{PiCTeX}}
%
%    The original definitions from \xfile{pictex.tex} \cite{PiCTeX}:
%\begin{quote}
%\begin{verbatim}
%\def\PiC{%
%  P%
%  \kern-.12em%
%  \lower.5ex\hbox{I}%
%  \kern-.075em%
%  C%
%}
%\def\PiCTeX{%
%  \PiC
%  \kern-.11em%
%  \TeX
%}
%\end{verbatim}
%\end{quote}
%
%    \begin{macro}{\HoLogo@PiC}
%    \begin{macrocode}
\def\HoLogo@PiC#1{%
  P%
  \kern-.12em%
  \lower.5ex\hbox{I}%
  \kern-.075em%
  C%
  \HOLOGO@SpaceFactor
}
%    \end{macrocode}
%    \end{macro}
%    \begin{macro}{\HoLogoHtml@PiC}
%    \begin{macrocode}
\def\HoLogoHtml@PiC#1{%
  \HoLogoCss@PiC
  \HOLOGO@Span{PiC}{%
    P%
    \HOLOGO@Span{i}{I}%
    C%
  }%
}
%    \end{macrocode}
%    \end{macro}
%    \begin{macro}{\HoLogoCss@PiC}
%    \begin{macrocode}
\def\HoLogoCss@PiC{%
  \Css{%
    span.HoLogo-PiC span.HoLogo-i{%
      position:relative;%
      top:.5ex;%
      margin-left:-.12em;%
      margin-right:-.075em;%
      text-decoration:none;%
    }%
  }%
  \global\let\HoLogoCss@PiC\relax
}
%    \end{macrocode}
%    \end{macro}
%
%    \begin{macro}{\HoLogo@PiCTeX}
%    \begin{macrocode}
\def\HoLogo@PiCTeX#1{%
  \hologo{PiC}%
  \HOLOGO@discretionary
  \kern-.11em%
  \hologo{TeX}%
}
%    \end{macrocode}
%    \end{macro}
%    \begin{macro}{\HoLogoHtml@PiCTeX}
%    \begin{macrocode}
\def\HoLogoHtml@PiCTeX#1{%
  \HoLogoCss@PiCTeX
  \HOLOGO@Span{PiCTeX}{%
    \hologo{PiC}%
    \hologo{TeX}%
  }%
}
%    \end{macrocode}
%    \end{macro}
%    \begin{macro}{\HoLogoCss@PiCTeX}
%    \begin{macrocode}
\def\HoLogoCss@PiCTeX{%
  \Css{%
    span.HoLogo-PiCTeX span.HoLogo-PiC{%
      margin-right:-.11em;%
    }%
  }%
  \global\let\HoLogoCss@PiCTeX\relax
}
%    \end{macrocode}
%    \end{macro}
%
% \subsubsection{\hologo{teTeX}}
%
%    \begin{macro}{\HoLogo@teTeX}
%    \begin{macrocode}
\def\HoLogo@teTeX#1{%
  \HOLOGO@mbox{#1{t}{T}e}%
  \HOLOGO@discretionary
  \hologo{TeX}%
}
%    \end{macrocode}
%    \end{macro}
%    \begin{macro}{\HoLogoCs@teTeX}
%    \begin{macrocode}
\def\HoLogoCs@teTeX#1{#1{t}{T}dfTeX}
%    \end{macrocode}
%    \end{macro}
%    \begin{macro}{\HoLogoBkm@teTeX}
%    \begin{macrocode}
\def\HoLogoBkm@teTeX#1{%
  #1{t}{T}e\hologo{TeX}%
}
%    \end{macrocode}
%    \end{macro}
%    \begin{macro}{\HoLogoHtml@teTeX}
%    \begin{macrocode}
\let\HoLogoHtml@teTeX\HoLogo@teTeX
%    \end{macrocode}
%    \end{macro}
%
% \subsubsection{\hologo{TeX4ht}}
%
%    \begin{macro}{\HoLogo@TeX4ht}
%    \begin{macrocode}
\expandafter\def\csname HoLogo@TeX4ht\endcsname#1{%
  \HOLOGO@mbox{\hologo{TeX}4ht}%
}
%    \end{macrocode}
%    \end{macro}
%    \begin{macro}{\HoLogoHtml@TeX4ht}
%    \begin{macrocode}
\expandafter
\let\csname HoLogoHtml@TeX4ht\expandafter\endcsname
\csname HoLogo@TeX4ht\endcsname
%    \end{macrocode}
%    \end{macro}
%
%
% \subsubsection{\hologo{SageTeX}}
%
%    \begin{macro}{\HoLogo@SageTeX}
%    \begin{macrocode}
\def\HoLogo@SageTeX#1{%
  \HOLOGO@mbox{Sage}%
  \HOLOGO@discretionary
  \HOLOGO@NegativeKerning{eT,oT,To}%
  \hologo{TeX}%
}
%    \end{macrocode}
%    \end{macro}
%    \begin{macro}{\HoLogoHtml@SageTeX}
%    \begin{macrocode}
\let\HoLogoHtml@SageTeX\HoLogo@SageTeX
%    \end{macrocode}
%    \end{macro}
%
% \subsection{\hologo{METAFONT} and friends}
%
%    \begin{macro}{\HoLogo@METAFONT}
%    \begin{macrocode}
\def\HoLogo@METAFONT#1{%
  \HoLogoFont@font{METAFONT}{logo}{%
    \HOLOGO@mbox{META}%
    \HOLOGO@discretionary
    \HOLOGO@mbox{FONT}%
  }%
}
%    \end{macrocode}
%    \end{macro}
%
%    \begin{macro}{\HoLogo@METAPOST}
%    \begin{macrocode}
\def\HoLogo@METAPOST#1{%
  \HoLogoFont@font{METAPOST}{logo}{%
    \HOLOGO@mbox{META}%
    \HOLOGO@discretionary
    \HOLOGO@mbox{POST}%
  }%
}
%    \end{macrocode}
%    \end{macro}
%
%    \begin{macro}{\HoLogo@MetaFun}
%    \begin{macrocode}
\def\HoLogo@MetaFun#1{%
  \HOLOGO@mbox{Meta}%
  \HOLOGO@discretionary
  \HOLOGO@mbox{Fun}%
}
%    \end{macrocode}
%    \end{macro}
%
%    \begin{macro}{\HoLogo@MetaPost}
%    \begin{macrocode}
\def\HoLogo@MetaPost#1{%
  \HOLOGO@mbox{Meta}%
  \HOLOGO@discretionary
  \HOLOGO@mbox{Post}%
}
%    \end{macrocode}
%    \end{macro}
%
% \subsection{Others}
%
% \subsubsection{\hologo{biber}}
%
%    \begin{macro}{\HoLogo@biber}
%    \begin{macrocode}
\def\HoLogo@biber#1{%
  \HOLOGO@mbox{#1{b}{B}i}%
  \HOLOGO@discretionary
  \HOLOGO@mbox{ber}%
}
%    \end{macrocode}
%    \end{macro}
%    \begin{macro}{\HoLogoCs@biber}
%    \begin{macrocode}
\def\HoLogoCs@biber#1{#1{b}{B}iber}
%    \end{macrocode}
%    \end{macro}
%    \begin{macro}{\HoLogoBkm@biber}
%    \begin{macrocode}
\def\HoLogoBkm@biber#1{%
  #1{b}{B}iber%
}
%    \end{macrocode}
%    \end{macro}
%    \begin{macro}{\HoLogoHtml@biber}
%    \begin{macrocode}
\let\HoLogoHtml@biber\HoLogo@biber
%    \end{macrocode}
%    \end{macro}
%
% \subsubsection{\hologo{KOMAScript}}
%
%    \begin{macro}{\HoLogo@KOMAScript}
%    The definition for \hologo{KOMAScript} is taken
%    from \hologo{KOMAScript} (\xfile{scrlogo.dtx}, reformatted) \cite{scrlogo}:
%\begin{quote}
%\begin{verbatim}
%\@ifundefined{KOMAScript}{%
%  \DeclareRobustCommand{\KOMAScript}{%
%    \textsf{%
%      K\kern.05em O\kern.05emM\kern.05em A%
%      \kern.1em-\kern.1em %
%      Script%
%    }%
%  }%
%}{}
%\end{verbatim}
%\end{quote}
%    \begin{macrocode}
\def\HoLogo@KOMAScript#1{%
  \HoLogoFont@font{KOMAScript}{sf}{%
    \HOLOGO@mbox{%
      K\kern.05em%
      O\kern.05em%
      M\kern.05em%
      A%
    }%
    \kern.1em%
    \HOLOGO@hyphen
    \kern.1em%
    \HOLOGO@mbox{Script}%
  }%
}
%    \end{macrocode}
%    \end{macro}
%    \begin{macro}{\HoLogoBkm@KOMAScript}
%    \begin{macrocode}
\def\HoLogoBkm@KOMAScript#1{%
  KOMA-Script%
}
%    \end{macrocode}
%    \end{macro}
%    \begin{macro}{\HoLogoHtml@KOMAScript}
%    \begin{macrocode}
\def\HoLogoHtml@KOMAScript#1{%
  \HoLogoCss@KOMAScript
  \HoLogoFont@font{KOMAScript}{sf}{%
    \HOLOGO@Span{KOMAScript}{%
      K%
      \HOLOGO@Span{O}{O}%
      M%
      \HOLOGO@Span{A}{A}%
      \HOLOGO@Span{hyphen}{-}%
      Script%
    }%
  }%
}
%    \end{macrocode}
%    \end{macro}
%    \begin{macro}{\HoLogoCss@KOMAScript}
%    \begin{macrocode}
\def\HoLogoCss@KOMAScript{%
  \Css{%
    span.HoLogo-KOMAScript{%
      font-family:sans-serif;%
    }%
  }%
  \Css{%
    span.HoLogo-KOMAScript span.HoLogo-O{%
      padding-left:.05em;%
      padding-right:.05em;%
    }%
  }%
  \Css{%
    span.HoLogo-KOMAScript span.HoLogo-A{%
      padding-left:.05em;%
    }%
  }%
  \Css{%
    span.HoLogo-KOMAScript span.HoLogo-hyphen{%
      padding-left:.1em;%
      padding-right:.1em;%
    }%
  }%
  \global\let\HoLogoCss@KOMAScript\relax
}
%    \end{macrocode}
%    \end{macro}
%
% \subsubsection{\hologo{LyX}}
%
%    \begin{macro}{\HoLogo@LyX}
%    The definition is taken from the documentation source files
%    of \hologo{LyX}, \xfile{Intro.lyx} \cite{LyX}:
%\begin{quote}
%\begin{verbatim}
%\def\LyX{%
%  \texorpdfstring{%
%    L\kern-.1667em\lower.25em\hbox{Y}\kern-.125emX\@%
%  }{%
%    LyX%
%  }%
%}
%\end{verbatim}
%\end{quote}
%    \begin{macrocode}
\def\HoLogo@LyX#1{%
  L%
  \kern-.1667em%
  \lower.25em\hbox{Y}%
  \kern-.125em%
  X%
  \HOLOGO@SpaceFactor
}
%    \end{macrocode}
%    \end{macro}
%    \begin{macro}{\HoLogoHtml@LyX}
%    \begin{macrocode}
\def\HoLogoHtml@LyX#1{%
  \HoLogoCss@LyX
  \HOLOGO@Span{LyX}{%
    L%
    \HOLOGO@Span{y}{Y}%
    X%
  }%
}
%    \end{macrocode}
%    \end{macro}
%    \begin{macro}{\HoLogoCss@LyX}
%    \begin{macrocode}
\def\HoLogoCss@LyX{%
  \Css{%
    span.HoLogo-LyX span.HoLogo-y{%
      position:relative;%
      top:.25em;%
      margin-left:-.1667em;%
      margin-right:-.125em;%
      text-decoration:none;%
    }%
  }%
  \global\let\HoLogoCss@LyX\relax
}
%    \end{macrocode}
%    \end{macro}
%
% \subsubsection{\hologo{NTS}}
%
%    \begin{macro}{\HoLogo@NTS}
%    Definition for \hologo{NTS} can be found in
%    package \xpackage{etex\textunderscore man} for the \hologo{eTeX} manual \cite{etexman}
%    and in package \xpackage{dtklogos} \cite{dtklogos}:
%\begin{quote}
%\begin{verbatim}
%\def\NTS{%
%  \leavevmode
%  \hbox{%
%    $%
%      \cal N%
%      \kern-0.35em%
%      \lower0.5ex\hbox{$\cal T$}%
%      \kern-0.2em%
%      S%
%    $%
%  }%
%}
%\end{verbatim}
%\end{quote}
%    \begin{macrocode}
\def\HoLogo@NTS#1{%
  \HoLogoFont@font{NTS}{sy}{%
    N\/%
    \kern-.35em%
    \lower.5ex\hbox{T\/}%
    \kern-.2em%
    S\/%
  }%
  \HOLOGO@SpaceFactor
}
%    \end{macrocode}
%    \end{macro}
%
% \subsubsection{\Hologo{TTH} (\hologo{TeX} to HTML translator)}
%
%    Source: \url{http://hutchinson.belmont.ma.us/tth/}
%    In the HTML source the second `T' is printed as subscript.
%\begin{quote}
%\begin{verbatim}
%T<sub>T</sub>H
%\end{verbatim}
%\end{quote}
%    \begin{macro}{\HoLogo@TTH}
%    \begin{macrocode}
\def\HoLogo@TTH#1{%
  \ltx@mbox{%
    T\HOLOGO@SubScript{T}H%
  }%
  \HOLOGO@SpaceFactor
}
%    \end{macrocode}
%    \end{macro}
%
%    \begin{macro}{\HoLogoHtml@TTH}
%    \begin{macrocode}
\def\HoLogoHtml@TTH#1{%
  T\HCode{<sub>}T\HCode{</sub>}H%
}
%    \end{macrocode}
%    \end{macro}
%
% \subsubsection{\Hologo{HanTheThanh}}
%
%    Partial source: Package \xpackage{dtklogos}.
%    The double accent is U+1EBF (latin small letter e with circumflex
%    and acute).
%    \begin{macro}{\HoLogo@HanTheThanh}
%    \begin{macrocode}
\def\HoLogo@HanTheThanh#1{%
  \ltx@mbox{H\`an}%
  \HOLOGO@space
  \ltx@mbox{%
    Th%
    \HOLOGO@IfCharExists{"1EBF}{%
      \char"1EBF\relax
    }{%
      \^e\hbox to 0pt{\hss\raise .5ex\hbox{\'{}}}%
    }%
  }%
  \HOLOGO@space
  \ltx@mbox{Th\`anh}%
}
%    \end{macrocode}
%    \end{macro}
%    \begin{macro}{\HoLogoBkm@HanTheThanh}
%    \begin{macrocode}
\def\HoLogoBkm@HanTheThanh#1{%
  H\`an %
  Th\HOLOGO@PdfdocUnicode{\^e}{\9036\277} %
  Th\`anh%
}
%    \end{macrocode}
%    \end{macro}
%    \begin{macro}{\HoLogoHtml@HanTheThanh}
%    \begin{macrocode}
\def\HoLogoHtml@HanTheThanh#1{%
  H\`an %
  Th\HCode{&\ltx@hashchar x1ebf;} %
  Th\`anh%
}
%    \end{macrocode}
%    \end{macro}
%
% \subsection{Driver detection}
%
%    \begin{macrocode}
\HOLOGO@IfExists\InputIfFileExists{%
  \InputIfFileExists{hologo.cfg}{}{}%
}{%
  \ltx@IfUndefined{pdf@filesize}{%
    \def\HOLOGO@InputIfExists{%
      \openin\HOLOGO@temp=hologo.cfg\relax
      \ifeof\HOLOGO@temp
        \closein\HOLOGO@temp
      \else
        \closein\HOLOGO@temp
        \begingroup
          \def\x{LaTeX2e}%
        \expandafter\endgroup
        \ifx\fmtname\x
          % \iffalse meta-comment
%
% File: hologo.dtx
% Version: 2016/05/12 v1.11
% Info: A logo collection with bookmark support
%
% Copyright (C) 2010-2012 by
%    Heiko Oberdiek <heiko.oberdiek at googlemail.com>
%
% This work may be distributed and/or modified under the
% conditions of the LaTeX Project Public License, either
% version 1.3c of this license or (at your option) any later
% version. This version of this license is in
%    http://www.latex-project.org/lppl/lppl-1-3c.txt
% and the latest version of this license is in
%    http://www.latex-project.org/lppl.txt
% and version 1.3 or later is part of all distributions of
% LaTeX version 2005/12/01 or later.
%
% This work has the LPPL maintenance status "maintained".
%
% This Current Maintainer of this work is Heiko Oberdiek.
%
% The Base Interpreter refers to any `TeX-Format',
% because some files are installed in TDS:tex/generic//.
%
% This work consists of the main source file hologo.dtx
% and the derived files
%    hologo.sty, hologo.pdf, hologo.ins, hologo.drv, hologo-example.tex,
%    hologo-test1.tex, hologo-test-spacefactor.tex,
%    hologo-test-list.tex.
%
% Distribution:
%    CTAN:macros/latex/contrib/oberdiek/hologo.dtx
%    CTAN:macros/latex/contrib/oberdiek/hologo.pdf
%
% Unpacking:
%    (a) If hologo.ins is present:
%           tex hologo.ins
%    (b) Without hologo.ins:
%           tex hologo.dtx
%    (c) If you insist on using LaTeX
%           latex \let\install=y% \iffalse meta-comment
%
% File: hologo.dtx
% Version: 2016/05/12 v1.11
% Info: A logo collection with bookmark support
%
% Copyright (C) 2010-2012 by
%    Heiko Oberdiek <heiko.oberdiek at googlemail.com>
%
% This work may be distributed and/or modified under the
% conditions of the LaTeX Project Public License, either
% version 1.3c of this license or (at your option) any later
% version. This version of this license is in
%    http://www.latex-project.org/lppl/lppl-1-3c.txt
% and the latest version of this license is in
%    http://www.latex-project.org/lppl.txt
% and version 1.3 or later is part of all distributions of
% LaTeX version 2005/12/01 or later.
%
% This work has the LPPL maintenance status "maintained".
%
% This Current Maintainer of this work is Heiko Oberdiek.
%
% The Base Interpreter refers to any `TeX-Format',
% because some files are installed in TDS:tex/generic//.
%
% This work consists of the main source file hologo.dtx
% and the derived files
%    hologo.sty, hologo.pdf, hologo.ins, hologo.drv, hologo-example.tex,
%    hologo-test1.tex, hologo-test-spacefactor.tex,
%    hologo-test-list.tex.
%
% Distribution:
%    CTAN:macros/latex/contrib/oberdiek/hologo.dtx
%    CTAN:macros/latex/contrib/oberdiek/hologo.pdf
%
% Unpacking:
%    (a) If hologo.ins is present:
%           tex hologo.ins
%    (b) Without hologo.ins:
%           tex hologo.dtx
%    (c) If you insist on using LaTeX
%           latex \let\install=y\input{hologo.dtx}
%        (quote the arguments according to the demands of your shell)
%
% Documentation:
%    (a) If hologo.drv is present:
%           latex hologo.drv
%    (b) Without hologo.drv:
%           latex hologo.dtx; ...
%    The class ltxdoc loads the configuration file ltxdoc.cfg
%    if available. Here you can specify further options, e.g.
%    use A4 as paper format:
%       \PassOptionsToClass{a4paper}{article}
%
%    Programm calls to get the documentation (example):
%       pdflatex hologo.dtx
%       makeindex -s gind.ist hologo.idx
%       pdflatex hologo.dtx
%       makeindex -s gind.ist hologo.idx
%       pdflatex hologo.dtx
%
% Installation:
%    TDS:tex/generic/oberdiek/hologo.sty
%    TDS:doc/latex/oberdiek/hologo.pdf
%    TDS:doc/latex/oberdiek/example/hologo-example.tex
%    TDS:doc/latex/oberdiek/test/hologo-test1.tex
%    TDS:doc/latex/oberdiek/test/hologo-test-spacefactor.tex
%    TDS:doc/latex/oberdiek/test/hologo-test-list.tex
%    TDS:source/latex/oberdiek/hologo.dtx
%
%<*ignore>
\begingroup
  \catcode123=1 %
  \catcode125=2 %
  \def\x{LaTeX2e}%
\expandafter\endgroup
\ifcase 0\ifx\install y1\fi\expandafter
         \ifx\csname processbatchFile\endcsname\relax\else1\fi
         \ifx\fmtname\x\else 1\fi\relax
\else\csname fi\endcsname
%</ignore>
%<*install>
\input docstrip.tex
\Msg{************************************************************************}
\Msg{* Installation}
\Msg{* Package: hologo 2016/05/12 v1.11 A logo collection with bookmark support (HO)}
\Msg{************************************************************************}

\keepsilent
\askforoverwritefalse

\let\MetaPrefix\relax
\preamble

This is a generated file.

Project: hologo
Version: 2016/05/12 v1.11

Copyright (C) 2010-2012 by
   Heiko Oberdiek <heiko.oberdiek at googlemail.com>

This work may be distributed and/or modified under the
conditions of the LaTeX Project Public License, either
version 1.3c of this license or (at your option) any later
version. This version of this license is in
   http://www.latex-project.org/lppl/lppl-1-3c.txt
and the latest version of this license is in
   http://www.latex-project.org/lppl.txt
and version 1.3 or later is part of all distributions of
LaTeX version 2005/12/01 or later.

This work has the LPPL maintenance status "maintained".

This Current Maintainer of this work is Heiko Oberdiek.

The Base Interpreter refers to any `TeX-Format',
because some files are installed in TDS:tex/generic//.

This work consists of the main source file hologo.dtx
and the derived files
   hologo.sty, hologo.pdf, hologo.ins, hologo.drv, hologo-example.tex,
   hologo-test1.tex, hologo-test-spacefactor.tex,
   hologo-test-list.tex.

\endpreamble
\let\MetaPrefix\DoubleperCent

\generate{%
  \file{hologo.ins}{\from{hologo.dtx}{install}}%
  \file{hologo.drv}{\from{hologo.dtx}{driver}}%
  \usedir{tex/generic/oberdiek}%
  \file{hologo.sty}{\from{hologo.dtx}{package}}%
  \usedir{doc/latex/oberdiek/example}%
  \file{hologo-example.tex}{\from{hologo.dtx}{example}}%
  \usedir{doc/latex/oberdiek/test}%
  \file{hologo-test1.tex}{\from{hologo.dtx}{test1}}%
  \file{hologo-test-spacefactor.tex}{\from{hologo.dtx}{test-spacefactor}}%
  \file{hologo-test-list.tex}{\from{hologo.dtx}{test-list}}%
  \nopreamble
  \nopostamble
  \usedir{source/latex/oberdiek/catalogue}%
  \file{hologo.xml}{\from{hologo.dtx}{catalogue}}%
}

\catcode32=13\relax% active space
\let =\space%
\Msg{************************************************************************}
\Msg{*}
\Msg{* To finish the installation you have to move the following}
\Msg{* file into a directory searched by TeX:}
\Msg{*}
\Msg{*     hologo.sty}
\Msg{*}
\Msg{* To produce the documentation run the file `hologo.drv'}
\Msg{* through LaTeX.}
\Msg{*}
\Msg{* Happy TeXing!}
\Msg{*}
\Msg{************************************************************************}

\endbatchfile
%</install>
%<*ignore>
\fi
%</ignore>
%<*driver>
\NeedsTeXFormat{LaTeX2e}
\ProvidesFile{hologo.drv}%
  [2016/05/12 v1.11 A logo collection with bookmark support (HO)]%
\documentclass{ltxdoc}
\usepackage{holtxdoc}[2011/11/22]
\usepackage{hologo}[2016/05/12]
\usepackage{longtable}
\usepackage{array}
\usepackage{paralist}
%\usepackage[T1]{fontenc}
%\usepackage{lmodern}
\begin{document}
  \DocInput{hologo.dtx}%
\end{document}
%</driver>
% \fi
%
%
% \CharacterTable
%  {Upper-case    \A\B\C\D\E\F\G\H\I\J\K\L\M\N\O\P\Q\R\S\T\U\V\W\X\Y\Z
%   Lower-case    \a\b\c\d\e\f\g\h\i\j\k\l\m\n\o\p\q\r\s\t\u\v\w\x\y\z
%   Digits        \0\1\2\3\4\5\6\7\8\9
%   Exclamation   \!     Double quote  \"     Hash (number) \#
%   Dollar        \$     Percent       \%     Ampersand     \&
%   Acute accent  \'     Left paren    \(     Right paren   \)
%   Asterisk      \*     Plus          \+     Comma         \,
%   Minus         \-     Point         \.     Solidus       \/
%   Colon         \:     Semicolon     \;     Less than     \<
%   Equals        \=     Greater than  \>     Question mark \?
%   Commercial at \@     Left bracket  \[     Backslash     \\
%   Right bracket \]     Circumflex    \^     Underscore    \_
%   Grave accent  \`     Left brace    \{     Vertical bar  \|
%   Right brace   \}     Tilde         \~}
%
% \GetFileInfo{hologo.drv}
%
% \title{The \xpackage{hologo} package}
% \date{2016/05/12 v1.11}
% \author{Heiko Oberdiek\\\xemail{heiko.oberdiek at googlemail.com}}
%
% \maketitle
%
% \begin{abstract}
% This package starts a collection of logos with support for bookmarks
% strings.
% \end{abstract}
%
% \tableofcontents
%
% \section{Documentation}
%
% \subsection{Logo macros}
%
% \begin{declcs}{hologo} \M{name}
% \end{declcs}
% Macro \cs{hologo} sets the logo with name \meta{name}.
% The following table shows the supported names.
%
% \begingroup
%   \def\hologoEntry#1#2#3{^^A
%     #1&#2&\hologoLogoSetup{#1}{variant=#2}\hologo{#1}&#3\tabularnewline
%   }
%   \begin{longtable}{>{\ttfamily}l>{\ttfamily}lll}
%     \rmfamily\bfseries{name} & \rmfamily\bfseries variant
%     & \bfseries logo & \bfseries since\\
%     \hline
%     \endhead
%     \hologoList
%   \end{longtable}
% \endgroup
%
% \begin{declcs}{Hologo} \M{name}
% \end{declcs}
% Macro \cs{Hologo} starts the logo \meta{name} with an uppercase
% letter. As an exception small greek letters are not converted
% to uppercase. Examples, see \hologo{eTeX} and \hologo{ExTeX}.
%
% \subsection{Setup macros}
%
% The package does not support package options, but the following
% setup macros can be used to set options.
%
% \begin{declcs}{hologoSetup} \M{key value list}
% \end{declcs}
% Macro \cs{hologoSetup} sets global options.
%
% \begin{declcs}{hologoLogoSetup} \M{logo} \M{key value list}
% \end{declcs}
% Some options can also be used to configure a logo.
% These settings take precedence over global option settings.
%
% \subsection{Options}\label{sec:options}
%
% There are boolean and string options:
% \begin{description}
% \item[Boolean option:]
% It takes |true| or |false|
% as value. If the value is omitted, then |true| is used.
% \item[String option:]
% A value must be given as string. (But the string might be empty.)
% \end{description}
% The following options can be used both in \cs{hologoSetup}
% and \cs{hologoLogoSetup}:
% \begin{description}
% \def\entry#1{\item[\xoption{#1}:]}
% \entry{break}
%   enables or disables line breaks inside the logo. This setting is
%   refined by options \xoption{hyphenbreak}, \xoption{spacebreak}
%   or \xoption{discretionarybreak}.
%   Default is |false|.
% \entry{hyphenbreak}
%   enables or disables the line break right after the hyphen character.
% \entry{spacebreak}
%   enables or disables line breaks at space characters.
% \entry{discretionarybreak}
%   enables or disables line breaks at hyphenation points
%   (inserted by \cs{-}).
% \end{description}
% Macro \cs{hologoLogoSetup} also knows:
% \begin{description}
% \item[\xoption{variant}:]
%   This is a string option. It specifies a variant of a logo that
%   must exist. An empty string selects the package default variant.
% \end{description}
% Example:
% \begin{quote}
%   |\hologoSetup{break=false}|\\
%   |\hologoLogoSetup{plainTeX}{variant=hyphen,hyphenbreak}|\\
%   Then ``plain-\TeX'' contains one break point after the hyphen.
% \end{quote}
%
% \subsection{Driver options}
%
% Sometimes graphical operations are needed to construct some
% glyphs (e.g.\ \hologo{XeTeX}). If package \xpackage{graphics}
% or package \xpackage{pgf} are found, then the macros are taken
% from there. Otherwise the packge defines its own operations
% and therefore needs the driver information. Many drivers are
% detected automatically (\hologo{pdfTeX}/\hologo{LuaTeX}
% in PDF mode, \hologo{XeTeX}, \hologo{VTeX}). These have precedence
% over a driver option. The driver can be given as package option
% or using \cs{hologoDriverSetup}.
% The following list contains the recognized driver options:
% \begin{itemize}
% \item \xoption{pdftex}, \xoption{luatex}
% \item \xoption{dvipdfm}, \xoption{dvipdfmx}
% \item \xoption{dvips}, \xoption{dvipsone}, \xoption{xdvi}
% \item \xoption{xetex}
% \item \xoption{vtex}
% \end{itemize}
% The left driver of a line is the driver name that is used internally.
% The following names are aliases for drivers that use the
% same method. Therefore the entry in the \xext{log} file for
% the used driver prints the internally used driver name.
% \begin{description}
% \item[\xoption{driverfallback}:]
%   This option expects a driver that is used,
%   if the driver could not be detected automatically.
% \end{description}
%
% \begin{declcs}{hologoDriverSetup} \M{driver option}
% \end{declcs}
% The driver can also be configured after package loading
% using \cs{hologoDriverSetup}, also the way for \hologo{plainTeX}
% to setup the driver.
%
% \subsection{Font setup}
%
% Some logos require a special font, but should also be usable by
% \hologo{plainTeX}. Therefore the package provides some ways
% to influence the font settings. The options below
% take font settings as values. Both font commands
% such as \cs{sffamily} and macros that take one argument
% like \cs{textsf} can be used.
%
% \begin{declcs}{hologoFontSetup} \M{key value list}
% \end{declcs}
% Macro \cs{hologoFontSetup} sets the fonts for all logos.
% Supported keys:
% \begin{description}
% \def\entry#1{\item[\xoption{#1}:]}
% \entry{general}
%   This font is used for all logos. The default is empty.
%   That means no special font is used.
% \entry{bibsf}
%   This font is used for
%   {\hologoLogoSetup{BibTeX}{variant=sf}\hologo{BibTeX}}
%   with variant \xoption{sf}.
% \entry{rm}
%   This font is a serif font. It is used for \hologo{ExTeX}.
% \entry{sc}
%   This font specifies a small caps font. It is used for
%   {\hologoLogoSetup{BibTeX}{variant=sc}\hologo{BibTeX}}
%   with variant \xoption{sc}.
% \entry{sf}
%   This font specifies a sans serif font. The default
%   is \cs{sffamily}, then \cs{sf} is tried. Otherwise
%   a warning is given. It is used by \hologo{KOMAScript}.
% \entry{sy}
%   This is the font for math symbols (e.g. cmsy).
%   It is used by \hologo{AmS}, \hologo{NTS}, \hologo{ExTeX}.
% \entry{logo}
%   \hologo{METAFONT} and \hologo{METAPOST} are using that font.
%   In \hologo{LaTeX} \cs{logofamily} is used and
%   the definitions of package \xpackage{mflogo} are used
%   if the package is not loaded.
%   Otherwise the \cs{tenlogo} is used and defined
%   if it does not already exists.
% \end{description}
%
% \begin{declcs}{hologoLogoFontSetup} \M{logo} \M{key value list}
% \end{declcs}
% Fonts can also be set for a logo or logo component separately,
% see the following list.
% The keys are the same as for \cs{hologoFontSetup}.
%
% \begin{longtable}{>{\ttfamily}l>{\sffamily}ll}
%   \meta{logo} & keys & result\\
%   \hline
%   \endhead
%   BibTeX & bibsf & {\hologoLogoSetup{BibTeX}{variant=sf}\hologo{BibTeX}}\\[.5ex]
%   BibTeX & sc & {\hologoLogoSetup{BibTeX}{variant=sc}\hologo{BibTeX}}\\[.5ex]
%   ExTeX & rm & \hologo{ExTeX}\\
%   SliTeX & rm & \hologo{SliTeX}\\[.5ex]
%   AmS & sy & \hologo{AmS}\\
%   ExTeX & sy & \hologo{ExTeX}\\
%   NTS & sy & \hologo{NTS}\\[.5ex]
%   KOMAScript & sf & \hologo{KOMAScript}\\[.5ex]
%   METAFONT & logo & \hologo{METAFONT}\\
%   METAPOST & logo & \hologo{METAPOST}\\[.5ex]
%   SliTeX & sc \hologo{SliTeX}
% \end{longtable}
%
% \subsubsection{Font order}
%
% For all logos the font \xoption{general} is applied first.
% Example:
%\begin{quote}
%|\hologoFontSetup{general=\color{red}}|
%\end{quote}
% will print red logos.
% Then if the font uses a special font \xoption{sf}, for example,
% the font is applied that is setup by \cs{hologoLogoFontSetup}.
% If this font is not setup, then the common font setup
% by \cs{hologoFontSetup} is used. Otherwise a warning is given,
% that there is no font configured.
%
% \subsection{Additional user macros}
%
% Usually a variant of a logo is configured by using
% \cs{hologoLogoSetup}, because it is bad style to mix
% different variants of the same logo in the same text.
% There the following macros are a convenience for testing.
%
% \begin{declcs}{hologoVariant} \M{name} \M{variant}\\
%   \cs{HologoVariant} \M{name} \M{variant}
% \end{declcs}
% Logo \meta{name} is set using \meta{variant} that specifies
% explicitely which variant of the macro is used. If the argument
% is empty, then the default form of the logo is used
% (configurable by \cs{hologoLogoSetup}).
%
% \cs{HologoVariant} is used if the logo is set in a context
% that needs an uppercase first letter (beginning of a sentence, \dots).
%
% \begin{declcs}{hologoList}\\
%   \cs{hologoEntry} \M{logo} \M{variant} \M{since}
% \end{declcs}
% Macro \cs{hologoList} contains all logos that are provided
% by the package including variants. The list consists of calls
% of \cs{hologoEntry} with three arguments starting with the
% logo name \meta{logo} and its variant \meta{variant}. An empty
% variant means the current default. Argument \meta{since} specifies
% with version of the package \xpackage{hologo} is needed to get
% the logo. If the logo is fixed, then the date gets updated.
% Therefore the date \meta{since} is not exactly the date of
% the first introduction, but rather the date of the latest fix.
%
% Before \cs{hologoList} can be used, macro \cs{hologoEntry} needs
% a definition. The example file in section \ref{sec:example}
% shows applications of \cs{hologoList}.
%
% \subsection{Supported contexts}
%
% Macros \cs{hologo} and friends support special contexts:
% \begin{itemize}
% \item \hologo{LaTeX}'s protection mechanism.
% \item Bookmarks of package \xpackage{hyperref}.
% \item Package \xpackage{tex4ht}.
% \item The macros can be used inside \cs{csname} constructs,
%   if \cs{ifincsname} is available (\hologo{pdfTeX}, \hologo{XeTeX},
%   \hologo{LuaTeX}).
% \end{itemize}
%
% \subsection{Example}
% \label{sec:example}
%
% The following example prints the logos in different fonts.
%    \begin{macrocode}
%<*example>
%<<verbatim
\NeedsTeXFormat{LaTeX2e}
\documentclass[a4paper]{article}
\usepackage[
  hmargin=20mm,
  vmargin=20mm,
]{geometry}
\pagestyle{empty}
\usepackage{hologo}[2016/05/12]
\usepackage{longtable}
\usepackage{array}
\setlength{\extrarowheight}{2pt}
\usepackage[T1]{fontenc}
\usepackage{lmodern}
\usepackage{pdflscape}
\usepackage[
  pdfencoding=auto,
]{hyperref}
\hypersetup{
  pdfauthor={Heiko Oberdiek},
  pdftitle={Example for package `hologo'},
  pdfsubject={Logos with fonts lmr, lmss, qtm, qpl, qhv},
}
\usepackage{bookmark}

% Print the logo list on the console

\begingroup
  \typeout{}%
  \typeout{*** Begin of logo list ***}%
  \newcommand*{\hologoEntry}[3]{%
    \typeout{#1 \ifx\\#2\\\else(#2) \fi[#3]}%
  }%
  \hologoList
  \typeout{*** End of logo list ***}%
  \typeout{}%
\endgroup

\begin{document}
\begin{landscape}

  \section{Example file for package `hologo'}

  % Table for font names

  \begin{longtable}{>{\bfseries}ll}
    \textbf{font} & \textbf{Font name}\\
    \hline
    lmr & Latin Modern Roman\\
    lmss & Latin Modern Sans\\
    qtm & \TeX\ Gyre Termes\\
    qhv & \TeX\ Gyre Heros\\
    qpl & \TeX\ Gyre Pagella\\
  \end{longtable}

  % Logo list with logos in different fonts

  \begingroup
    \newcommand*{\SetVariant}[2]{%
      \ifx\\#2\\%
      \else
        \hologoLogoSetup{#1}{variant=#2}%
      \fi
    }%
    \newcommand*{\hologoEntry}[3]{%
      \SetVariant{#1}{#2}%
      \raisebox{1em}[0pt][0pt]{\hypertarget{#1@#2}{}}%
      \bookmark[%
        dest={#1@#2},%
      ]{%
        #1\ifx\\#2\\\else\space(#2)\fi: \Hologo{#1}, \hologo{#1} %
        [Unicode]%
      }%
      \hypersetup{unicode=false}%
      \bookmark[%
        dest={#1@#2},%
      ]{%
        #1\ifx\\#2\\\else\space(#2)\fi: \Hologo{#1}, \hologo{#1} %
        [PDFDocEncoding]%
      }%
      \texttt{#1}%
      &%
      \texttt{#2}%
      &%
      \Hologo{#1}%
      &%
      \SetVariant{#1}{#2}%
      \hologo{#1}%
      &%
      \SetVariant{#1}{#2}%
      \fontfamily{qtm}\selectfont
      \hologo{#1}%
      &%
      \SetVariant{#1}{#2}%
      \fontfamily{qpl}\selectfont
      \hologo{#1}%
      &%
      \SetVariant{#1}{#2}%
      \textsf{\hologo{#1}}%
      &%
      \SetVariant{#1}{#2}%
      \fontfamily{qhv}\selectfont
      \hologo{#1}%
      \tabularnewline
    }%
    \begin{longtable}{llllllll}%
      \textbf{\textit{logo}} & \textbf{\textit{variant}} &
      \texttt{\string\Hologo} &
      \textbf{lmr} & \textbf{qtm} & \textbf{qpl} &
      \textbf{lmss} & \textbf{qhv}
      \tabularnewline
      \hline
      \endhead
      \hologoList
    \end{longtable}%
  \endgroup

\end{landscape}
\end{document}
%verbatim
%</example>
%    \end{macrocode}
%
% \StopEventually{
% }
%
% \section{Implementation}
%    \begin{macrocode}
%<*package>
%    \end{macrocode}
%    Reload check, especially if the package is not used with \LaTeX.
%    \begin{macrocode}
\begingroup\catcode61\catcode48\catcode32=10\relax%
  \catcode13=5 % ^^M
  \endlinechar=13 %
  \catcode35=6 % #
  \catcode39=12 % '
  \catcode44=12 % ,
  \catcode45=12 % -
  \catcode46=12 % .
  \catcode58=12 % :
  \catcode64=11 % @
  \catcode123=1 % {
  \catcode125=2 % }
  \expandafter\let\expandafter\x\csname ver@hologo.sty\endcsname
  \ifx\x\relax % plain-TeX, first loading
  \else
    \def\empty{}%
    \ifx\x\empty % LaTeX, first loading,
      % variable is initialized, but \ProvidesPackage not yet seen
    \else
      \expandafter\ifx\csname PackageInfo\endcsname\relax
        \def\x#1#2{%
          \immediate\write-1{Package #1 Info: #2.}%
        }%
      \else
        \def\x#1#2{\PackageInfo{#1}{#2, stopped}}%
      \fi
      \x{hologo}{The package is already loaded}%
      \aftergroup\endinput
    \fi
  \fi
\endgroup%
%    \end{macrocode}
%    Package identification:
%    \begin{macrocode}
\begingroup\catcode61\catcode48\catcode32=10\relax%
  \catcode13=5 % ^^M
  \endlinechar=13 %
  \catcode35=6 % #
  \catcode39=12 % '
  \catcode40=12 % (
  \catcode41=12 % )
  \catcode44=12 % ,
  \catcode45=12 % -
  \catcode46=12 % .
  \catcode47=12 % /
  \catcode58=12 % :
  \catcode64=11 % @
  \catcode91=12 % [
  \catcode93=12 % ]
  \catcode123=1 % {
  \catcode125=2 % }
  \expandafter\ifx\csname ProvidesPackage\endcsname\relax
    \def\x#1#2#3[#4]{\endgroup
      \immediate\write-1{Package: #3 #4}%
      \xdef#1{#4}%
    }%
  \else
    \def\x#1#2[#3]{\endgroup
      #2[{#3}]%
      \ifx#1\@undefined
        \xdef#1{#3}%
      \fi
      \ifx#1\relax
        \xdef#1{#3}%
      \fi
    }%
  \fi
\expandafter\x\csname ver@hologo.sty\endcsname
\ProvidesPackage{hologo}%
  [2016/05/12 v1.11 A logo collection with bookmark support (HO)]%
%    \end{macrocode}
%
%    \begin{macrocode}
\begingroup\catcode61\catcode48\catcode32=10\relax%
  \catcode13=5 % ^^M
  \endlinechar=13 %
  \catcode123=1 % {
  \catcode125=2 % }
  \catcode64=11 % @
  \def\x{\endgroup
    \expandafter\edef\csname HOLOGO@AtEnd\endcsname{%
      \endlinechar=\the\endlinechar\relax
      \catcode13=\the\catcode13\relax
      \catcode32=\the\catcode32\relax
      \catcode35=\the\catcode35\relax
      \catcode61=\the\catcode61\relax
      \catcode64=\the\catcode64\relax
      \catcode123=\the\catcode123\relax
      \catcode125=\the\catcode125\relax
    }%
  }%
\x\catcode61\catcode48\catcode32=10\relax%
\catcode13=5 % ^^M
\endlinechar=13 %
\catcode35=6 % #
\catcode64=11 % @
\catcode123=1 % {
\catcode125=2 % }
\def\TMP@EnsureCode#1#2{%
  \edef\HOLOGO@AtEnd{%
    \HOLOGO@AtEnd
    \catcode#1=\the\catcode#1\relax
  }%
  \catcode#1=#2\relax
}
\TMP@EnsureCode{10}{12}% ^^J
\TMP@EnsureCode{33}{12}% !
\TMP@EnsureCode{34}{12}% "
\TMP@EnsureCode{36}{3}% $
\TMP@EnsureCode{38}{4}% &
\TMP@EnsureCode{39}{12}% '
\TMP@EnsureCode{40}{12}% (
\TMP@EnsureCode{41}{12}% )
\TMP@EnsureCode{42}{12}% *
\TMP@EnsureCode{43}{12}% +
\TMP@EnsureCode{44}{12}% ,
\TMP@EnsureCode{45}{12}% -
\TMP@EnsureCode{46}{12}% .
\TMP@EnsureCode{47}{12}% /
\TMP@EnsureCode{58}{12}% :
\TMP@EnsureCode{59}{12}% ;
\TMP@EnsureCode{60}{12}% <
\TMP@EnsureCode{62}{12}% >
\TMP@EnsureCode{63}{12}% ?
\TMP@EnsureCode{91}{12}% [
\TMP@EnsureCode{93}{12}% ]
\TMP@EnsureCode{94}{7}% ^ (superscript)
\TMP@EnsureCode{95}{8}% _ (subscript)
\TMP@EnsureCode{96}{12}% `
\TMP@EnsureCode{124}{12}% |
\edef\HOLOGO@AtEnd{%
  \HOLOGO@AtEnd
  \escapechar\the\escapechar\relax
  \noexpand\endinput
}
\escapechar=92 %
%    \end{macrocode}
%
% \subsection{Logo list}
%
%    \begin{macro}{\hologoList}
%    \begin{macrocode}
\def\hologoList{%
  \hologoEntry{(La)TeX}{}{2011/10/01}%
  \hologoEntry{AmSLaTeX}{}{2010/04/16}%
  \hologoEntry{AmSTeX}{}{2010/04/16}%
  \hologoEntry{biber}{}{2011/10/01}%
  \hologoEntry{BibTeX}{}{2011/10/01}%
  \hologoEntry{BibTeX}{sf}{2011/10/01}%
  \hologoEntry{BibTeX}{sc}{2011/10/01}%
  \hologoEntry{BibTeX8}{}{2011/11/22}%
  \hologoEntry{ConTeXt}{}{2011/03/25}%
  \hologoEntry{ConTeXt}{narrow}{2011/03/25}%
  \hologoEntry{ConTeXt}{simple}{2011/03/25}%
  \hologoEntry{emTeX}{}{2010/04/26}%
  \hologoEntry{eTeX}{}{2010/04/08}%
  \hologoEntry{ExTeX}{}{2011/10/01}%
  \hologoEntry{HanTheThanh}{}{2011/11/29}%
  \hologoEntry{iniTeX}{}{2011/10/01}%
  \hologoEntry{KOMAScript}{}{2011/10/01}%
  \hologoEntry{La}{}{2010/05/08}%
  \hologoEntry{LaTeX}{}{2010/04/08}%
  \hologoEntry{LaTeX2e}{}{2010/04/08}%
  \hologoEntry{LaTeX3}{}{2010/04/24}%
  \hologoEntry{LaTeXe}{}{2010/04/08}%
  \hologoEntry{LaTeXML}{}{2011/11/22}%
  \hologoEntry{LaTeXTeX}{}{2011/10/01}%
  \hologoEntry{LuaLaTeX}{}{2010/04/08}%
  \hologoEntry{LuaTeX}{}{2010/04/08}%
  \hologoEntry{LyX}{}{2011/10/01}%
  \hologoEntry{METAFONT}{}{2011/10/01}%
  \hologoEntry{MetaFun}{}{2011/10/01}%
  \hologoEntry{METAPOST}{}{2011/10/01}%
  \hologoEntry{MetaPost}{}{2011/10/01}%
  \hologoEntry{MiKTeX}{}{2011/10/01}%
  \hologoEntry{NTS}{}{2011/10/01}%
  \hologoEntry{OzMF}{}{2011/10/01}%
  \hologoEntry{OzMP}{}{2011/10/01}%
  \hologoEntry{OzTeX}{}{2011/10/01}%
  \hologoEntry{OzTtH}{}{2011/10/01}%
  \hologoEntry{PCTeX}{}{2011/10/01}%
  \hologoEntry{pdfTeX}{}{2011/10/01}%
  \hologoEntry{pdfLaTeX}{}{2011/10/01}%
  \hologoEntry{PiC}{}{2011/10/01}%
  \hologoEntry{PiCTeX}{}{2011/10/01}%
  \hologoEntry{plainTeX}{}{2010/04/08}%
  \hologoEntry{plainTeX}{space}{2010/04/16}%
  \hologoEntry{plainTeX}{hyphen}{2010/04/16}%
  \hologoEntry{plainTeX}{runtogether}{2010/04/16}%
  \hologoEntry{SageTeX}{}{2011/11/22}%
  \hologoEntry{SLiTeX}{}{2011/10/01}%
  \hologoEntry{SLiTeX}{lift}{2011/10/01}%
  \hologoEntry{SLiTeX}{narrow}{2011/10/01}%
  \hologoEntry{SLiTeX}{simple}{2011/10/01}%
  \hologoEntry{SliTeX}{}{2011/10/01}%
  \hologoEntry{SliTeX}{narrow}{2011/10/01}%
  \hologoEntry{SliTeX}{simple}{2011/10/01}%
  \hologoEntry{SliTeX}{lift}{2011/10/01}%
  \hologoEntry{teTeX}{}{2011/10/01}%
  \hologoEntry{TeX}{}{2010/04/08}%
  \hologoEntry{TeX4ht}{}{2011/11/22}%
  \hologoEntry{TTH}{}{2011/11/22}%
  \hologoEntry{virTeX}{}{2011/10/01}%
  \hologoEntry{VTeX}{}{2010/04/24}%
  \hologoEntry{Xe}{}{2010/04/08}%
  \hologoEntry{XeLaTeX}{}{2010/04/08}%
  \hologoEntry{XeTeX}{}{2010/04/08}%
}
%    \end{macrocode}
%    \end{macro}
%
% \subsection{Load resources}
%
%    \begin{macrocode}
\begingroup\expandafter\expandafter\expandafter\endgroup
\expandafter\ifx\csname RequirePackage\endcsname\relax
  \def\TMP@RequirePackage#1[#2]{%
    \begingroup\expandafter\expandafter\expandafter\endgroup
    \expandafter\ifx\csname ver@#1.sty\endcsname\relax
      \input #1.sty\relax
    \fi
  }%
  \TMP@RequirePackage{ltxcmds}[2011/02/04]%
  \TMP@RequirePackage{infwarerr}[2010/04/08]%
  \TMP@RequirePackage{kvsetkeys}[2010/03/01]%
  \TMP@RequirePackage{kvdefinekeys}[2010/03/01]%
  \TMP@RequirePackage{pdftexcmds}[2010/04/01]%
  \TMP@RequirePackage{ifpdf}[2010/01/28]%
  \TMP@RequirePackage{ifluatex}[2010/03/01]%
  \ltx@IfUndefined{newif}{%
    \expandafter\let\csname newif\endcsname\ltx@newif
  }{}%
  \TMP@RequirePackage{ifxetex}[2009/01/23]%
  \TMP@RequirePackage{ifvtex}[2010/03/01]%
\else
  \RequirePackage{ltxcmds}[2011/02/04]%
  \RequirePackage{infwarerr}[2010/04/08]%
  \RequirePackage{kvsetkeys}[2010/03/01]%
  \RequirePackage{kvdefinekeys}[2010/03/01]%
  \RequirePackage{pdftexcmds}[2010/04/01]%
  \RequirePackage{ifpdf}[2010/01/28]%
  \RequirePackage{ifluatex}[2010/03/01]%
  \RequirePackage{ifxetex}[2009/01/23]%
  \RequirePackage{ifvtex}[2010/03/01]%
\fi
%    \end{macrocode}
%
%    \begin{macro}{\HOLOGO@IfDefined}
%    \begin{macrocode}
\def\HOLOGO@IfExists#1{%
  \ifx\@undefined#1%
    \expandafter\ltx@secondoftwo
  \else
    \ifx\relax#1%
      \expandafter\ltx@secondoftwo
    \else
      \expandafter\expandafter\expandafter\ltx@firstoftwo
    \fi
  \fi
}
%    \end{macrocode}
%    \end{macro}
%
% \subsection{Setup macros}
%
%    \begin{macro}{\hologoSetup}
%    \begin{macrocode}
\def\hologoSetup{%
  \let\HOLOGO@name\relax
  \HOLOGO@Setup
}
%    \end{macrocode}
%    \end{macro}
%
%    \begin{macro}{\hologoLogoSetup}
%    \begin{macrocode}
\def\hologoLogoSetup#1{%
  \edef\HOLOGO@name{#1}%
  \ltx@IfUndefined{HoLogo@\HOLOGO@name}{%
    \@PackageError{hologo}{%
      Unknown logo `\HOLOGO@name'%
    }\@ehc
    \ltx@gobble
  }{%
    \HOLOGO@Setup
  }%
}
%    \end{macrocode}
%    \end{macro}
%
%    \begin{macro}{\HOLOGO@Setup}
%    \begin{macrocode}
\def\HOLOGO@Setup{%
  \kvsetkeys{HoLogo}%
}
%    \end{macrocode}
%    \end{macro}
%
% \subsection{Options}
%
%    \begin{macro}{\HOLOGO@DeclareBoolOption}
%    \begin{macrocode}
\def\HOLOGO@DeclareBoolOption#1{%
  \expandafter\chardef\csname HOLOGOOPT@#1\endcsname\ltx@zero
  \kv@define@key{HoLogo}{#1}[true]{%
    \def\HOLOGO@temp{##1}%
    \ifx\HOLOGO@temp\HOLOGO@true
      \ifx\HOLOGO@name\relax
        \expandafter\chardef\csname HOLOGOOPT@#1\endcsname=\ltx@one
      \else
        \expandafter\chardef\csname
        HoLogoOpt@#1@\HOLOGO@name\endcsname\ltx@one
      \fi
      \HOLOGO@SetBreakAll{#1}%
    \else
      \ifx\HOLOGO@temp\HOLOGO@false
        \ifx\HOLOGO@name\relax
          \expandafter\chardef\csname HOLOGOOPT@#1\endcsname=\ltx@zero
        \else
          \expandafter\chardef\csname
          HoLogoOpt@#1@\HOLOGO@name\endcsname=\ltx@zero
        \fi
        \HOLOGO@SetBreakAll{#1}%
      \else
        \@PackageError{hologo}{%
          Unknown value `##1' for boolean option `#1'.\MessageBreak
          Known values are `true' and `false'%
        }\@ehc
      \fi
    \fi
  }%
}
%    \end{macrocode}
%    \end{macro}
%
%    \begin{macro}{\HOLOGO@SetBreakAll}
%    \begin{macrocode}
\def\HOLOGO@SetBreakAll#1{%
  \def\HOLOGO@temp{#1}%
  \ifx\HOLOGO@temp\HOLOGO@break
    \ifx\HOLOGO@name\relax
      \chardef\HOLOGOOPT@hyphenbreak=\HOLOGOOPT@break
      \chardef\HOLOGOOPT@spacebreak=\HOLOGOOPT@break
      \chardef\HOLOGOOPT@discretionarybreak=\HOLOGOOPT@break
    \else
      \expandafter\chardef
         \csname HoLogoOpt@hyphenbreak@\HOLOGO@name\endcsname=%
         \csname HoLogoOpt@break@\HOLOGO@name\endcsname
      \expandafter\chardef
         \csname HoLogoOpt@spacebreak@\HOLOGO@name\endcsname=%
         \csname HoLogoOpt@break@\HOLOGO@name\endcsname
      \expandafter\chardef
         \csname HoLogoOpt@discretionarybreak@\HOLOGO@name
             \endcsname=%
         \csname HoLogoOpt@break@\HOLOGO@name\endcsname
    \fi
  \fi
}
%    \end{macrocode}
%    \end{macro}
%
%    \begin{macro}{\HOLOGO@true}
%    \begin{macrocode}
\def\HOLOGO@true{true}
%    \end{macrocode}
%    \end{macro}
%    \begin{macro}{\HOLOGO@false}
%    \begin{macrocode}
\def\HOLOGO@false{false}
%    \end{macrocode}
%    \end{macro}
%    \begin{macro}{\HOLOGO@break}
%    \begin{macrocode}
\def\HOLOGO@break{break}
%    \end{macrocode}
%    \end{macro}
%
%    \begin{macrocode}
\HOLOGO@DeclareBoolOption{break}
\HOLOGO@DeclareBoolOption{hyphenbreak}
\HOLOGO@DeclareBoolOption{spacebreak}
\HOLOGO@DeclareBoolOption{discretionarybreak}
%    \end{macrocode}
%
%    \begin{macrocode}
\kv@define@key{HoLogo}{variant}{%
  \ifx\HOLOGO@name\relax
    \@PackageError{hologo}{%
      Option `variant' is not available in \string\hologoSetup,%
      \MessageBreak
      Use \string\hologoLogoSetup\space instead%
    }\@ehc
  \else
    \edef\HOLOGO@temp{#1}%
    \ifx\HOLOGO@temp\ltx@empty
      \expandafter
      \let\csname HoLogoOpt@variant@\HOLOGO@name\endcsname\@undefined
    \else
      \ltx@IfUndefined{HoLogo@\HOLOGO@name @\HOLOGO@temp}{%
        \@PackageError{hologo}{%
          Unknown variant `\HOLOGO@temp' of logo `\HOLOGO@name'%
        }\@ehc
      }{%
        \expandafter
        \let\csname HoLogoOpt@variant@\HOLOGO@name\endcsname
            \HOLOGO@temp
      }%
    \fi
  \fi
}
%    \end{macrocode}
%
%    \begin{macro}{\HOLOGO@Variant}
%    \begin{macrocode}
\def\HOLOGO@Variant#1{%
  #1%
  \ltx@ifundefined{HoLogoOpt@variant@#1}{%
  }{%
    @\csname HoLogoOpt@variant@#1\endcsname
  }%
}
%    \end{macrocode}
%    \end{macro}
%
% \subsection{Break/no-break support}
%
%    \begin{macro}{\HOLOGO@space}
%    \begin{macrocode}
\def\HOLOGO@space{%
  \ltx@ifundefined{HoLogoOpt@spacebreak@\HOLOGO@name}{%
    \ltx@ifundefined{HoLogoOpt@break@\HOLOGO@name}{%
      \chardef\HOLOGO@temp=\HOLOGOOPT@spacebreak
    }{%
      \chardef\HOLOGO@temp=%
        \csname HoLogoOpt@break@\HOLOGO@name\endcsname
    }%
  }{%
    \chardef\HOLOGO@temp=%
      \csname HoLogoOpt@spacebreak@\HOLOGO@name\endcsname
  }%
  \ifcase\HOLOGO@temp
    \penalty10000 %
  \fi
  \ltx@space
}
%    \end{macrocode}
%    \end{macro}
%
%    \begin{macro}{\HOLOGO@hyphen}
%    \begin{macrocode}
\def\HOLOGO@hyphen{%
  \ltx@ifundefined{HoLogoOpt@hyphenbreak@\HOLOGO@name}{%
    \ltx@ifundefined{HoLogoOpt@break@\HOLOGO@name}{%
      \chardef\HOLOGO@temp=\HOLOGOOPT@hyphenbreak
    }{%
      \chardef\HOLOGO@temp=%
        \csname HoLogoOpt@break@\HOLOGO@name\endcsname
    }%
  }{%
    \chardef\HOLOGO@temp=%
      \csname HoLogoOpt@hyphenbreak@\HOLOGO@name\endcsname
  }%
  \ifcase\HOLOGO@temp
    \ltx@mbox{-}%
  \else
    -%
  \fi
}
%    \end{macrocode}
%    \end{macro}
%
%    \begin{macro}{\HOLOGO@discretionary}
%    \begin{macrocode}
\def\HOLOGO@discretionary{%
  \ltx@ifundefined{HoLogoOpt@discretionarybreak@\HOLOGO@name}{%
    \ltx@ifundefined{HoLogoOpt@break@\HOLOGO@name}{%
      \chardef\HOLOGO@temp=\HOLOGOOPT@discretionarybreak
    }{%
      \chardef\HOLOGO@temp=%
        \csname HoLogoOpt@break@\HOLOGO@name\endcsname
    }%
  }{%
    \chardef\HOLOGO@temp=%
      \csname HoLogoOpt@discretionarybreak@\HOLOGO@name\endcsname
  }%
  \ifcase\HOLOGO@temp
  \else
    \-%
  \fi
}
%    \end{macrocode}
%    \end{macro}
%
%    \begin{macro}{\HOLOGO@mbox}
%    \begin{macrocode}
\def\HOLOGO@mbox#1{%
  \ltx@ifundefined{HoLogoOpt@break@\HOLOGO@name}{%
    \chardef\HOLOGO@temp=\HOLOGOOPT@hyphenbreak
  }{%
    \chardef\HOLOGO@temp=%
      \csname HoLogoOpt@break@\HOLOGO@name\endcsname
  }%
  \ifcase\HOLOGO@temp
    \ltx@mbox{#1}%
  \else
    #1%
  \fi
}
%    \end{macrocode}
%    \end{macro}
%
% \subsection{Font support}
%
%    \begin{macro}{\HoLogoFont@font}
%    \begin{tabular}{@{}ll@{}}
%    |#1|:& logo name\\
%    |#2|:& font short name\\
%    |#3|:& text
%    \end{tabular}
%    \begin{macrocode}
\def\HoLogoFont@font#1#2#3{%
  \begingroup
    \ltx@IfUndefined{HoLogoFont@logo@#1.#2}{%
      \ltx@IfUndefined{HoLogoFont@font@#2}{%
        \@PackageWarning{hologo}{%
          Missing font `#2' for logo `#1'%
        }%
        #3%
      }{%
        \csname HoLogoFont@font@#2\endcsname{#3}%
      }%
    }{%
      \csname HoLogoFont@logo@#1.#2\endcsname{#3}%
    }%
  \endgroup
}
%    \end{macrocode}
%    \end{macro}
%
%    \begin{macro}{\HoLogoFont@Def}
%    \begin{macrocode}
\def\HoLogoFont@Def#1{%
  \expandafter\def\csname HoLogoFont@font@#1\endcsname
}
%    \end{macrocode}
%    \end{macro}
%    \begin{macro}{\HoLogoFont@LogoDef}
%    \begin{macrocode}
\def\HoLogoFont@LogoDef#1#2{%
  \expandafter\def\csname HoLogoFont@logo@#1.#2\endcsname
}
%    \end{macrocode}
%    \end{macro}
%
% \subsubsection{Font defaults}
%
%    \begin{macro}{\HoLogoFont@font@general}
%    \begin{macrocode}
\HoLogoFont@Def{general}{}%
%    \end{macrocode}
%    \end{macro}
%
%    \begin{macro}{\HoLogoFont@font@rm}
%    \begin{macrocode}
\ltx@IfUndefined{rmfamily}{%
  \ltx@IfUndefined{rm}{%
  }{%
    \HoLogoFont@Def{rm}{\rm}%
  }%
}{%
  \HoLogoFont@Def{rm}{\rmfamily}%
}
%    \end{macrocode}
%    \end{macro}
%
%    \begin{macro}{\HoLogoFont@font@sf}
%    \begin{macrocode}
\ltx@IfUndefined{sffamily}{%
  \ltx@IfUndefined{sf}{%
  }{%
    \HoLogoFont@Def{sf}{\sf}%
  }%
}{%
  \HoLogoFont@Def{sf}{\sffamily}%
}
%    \end{macrocode}
%    \end{macro}
%
%    \begin{macro}{\HoLogoFont@font@bibsf}
%    In case of \hologo{plainTeX} the original small caps
%    variant is used as default. In \hologo{LaTeX}
%    the definition of package \xpackage{dtklogos} \cite{dtklogos}
%    is used.
%\begin{quote}
%\begin{verbatim}
%\DeclareRobustCommand{\BibTeX}{%
%  B%
%  \kern-.05em%
%  \hbox{%
%    $\m@th$% %% force math size calculations
%    \csname S@\f@size\endcsname
%    \fontsize\sf@size\z@
%    \math@fontsfalse
%    \selectfont
%    I%
%    \kern-.025em%
%    B
%  }%
%  \kern-.08em%
%  \-%
%  \TeX
%}
%\end{verbatim}
%\end{quote}
%    \begin{macrocode}
\ltx@IfUndefined{selectfont}{%
  \ltx@IfUndefined{tensc}{%
    \font\tensc=cmcsc10\relax
  }{}%
  \HoLogoFont@Def{bibsf}{\tensc}%
}{%
  \HoLogoFont@Def{bibsf}{%
    $\mathsurround=0pt$%
    \csname S@\f@size\endcsname
    \fontsize\sf@size{0pt}%
    \math@fontsfalse
    \selectfont
  }%
}
%    \end{macrocode}
%    \end{macro}
%
%    \begin{macro}{\HoLogoFont@font@sc}
%    \begin{macrocode}
\ltx@IfUndefined{scshape}{%
  \ltx@IfUndefined{tensc}{%
    \font\tensc=cmcsc10\relax
  }{}%
  \HoLogoFont@Def{sc}{\tensc}%
}{%
  \HoLogoFont@Def{sc}{\scshape}%
}
%    \end{macrocode}
%    \end{macro}
%
%    \begin{macro}{\HoLogoFont@font@sy}
%    \begin{macrocode}
\ltx@IfUndefined{usefont}{%
  \ltx@IfUndefined{tensy}{%
  }{%
    \HoLogoFont@Def{sy}{\tensy}%
  }%
}{%
  \HoLogoFont@Def{sy}{%
    \usefont{OMS}{cmsy}{m}{n}%
  }%
}
%    \end{macrocode}
%    \end{macro}
%
%    \begin{macro}{\HoLogoFont@font@logo}
%    \begin{macrocode}
\begingroup
  \def\x{LaTeX2e}%
\expandafter\endgroup
\ifx\fmtname\x
  \ltx@IfUndefined{logofamily}{%
    \DeclareRobustCommand\logofamily{%
      \not@math@alphabet\logofamily\relax
      \fontencoding{U}%
      \fontfamily{logo}%
      \selectfont
    }%
  }{}%
  \ltx@IfUndefined{logofamily}{%
  }{%
    \HoLogoFont@Def{logo}{\logofamily}%
  }%
\else
  \ltx@IfUndefined{tenlogo}{%
    \font\tenlogo=logo10\relax
  }{}%
  \HoLogoFont@Def{logo}{\tenlogo}%
\fi
%    \end{macrocode}
%    \end{macro}
%
% \subsubsection{Font setup}
%
%    \begin{macro}{\hologoFontSetup}
%    \begin{macrocode}
\def\hologoFontSetup{%
  \let\HOLOGO@name\relax
  \HOLOGO@FontSetup
}
%    \end{macrocode}
%    \end{macro}
%
%    \begin{macro}{\hologoLogoFontSetup}
%    \begin{macrocode}
\def\hologoLogoFontSetup#1{%
  \edef\HOLOGO@name{#1}%
  \ltx@IfUndefined{HoLogo@\HOLOGO@name}{%
    \@PackageError{hologo}{%
      Unknown logo `\HOLOGO@name'%
    }\@ehc
    \ltx@gobble
  }{%
    \HOLOGO@FontSetup
  }%
}
%    \end{macrocode}
%    \end{macro}
%
%    \begin{macro}{\HOLOGO@FontSetup}
%    \begin{macrocode}
\def\HOLOGO@FontSetup{%
  \kvsetkeys{HoLogoFont}%
}
%    \end{macrocode}
%    \end{macro}
%
%    \begin{macrocode}
\def\HOLOGO@temp#1{%
  \kv@define@key{HoLogoFont}{#1}{%
    \ifx\HOLOGO@name\relax
      \HoLogoFont@Def{#1}{##1}%
    \else
      \HoLogoFont@LogoDef\HOLOGO@name{#1}{##1}%
    \fi
  }%
}
\HOLOGO@temp{general}
\HOLOGO@temp{sf}
%    \end{macrocode}
%
% \subsection{Generic logo commands}
%
%    \begin{macrocode}
\HOLOGO@IfExists\hologo{%
  \@PackageError{hologo}{%
    \string\hologo\ltx@space is already defined.\MessageBreak
    Package loading is aborted%
  }\@ehc
  \HOLOGO@AtEnd
}%
\HOLOGO@IfExists\hologoRobust{%
  \@PackageError{hologo}{%
    \string\hologoRobust\ltx@space is already defined.\MessageBreak
    Package loading is aborted%
  }\@ehc
  \HOLOGO@AtEnd
}%
%    \end{macrocode}
%
% \subsubsection{\cs{hologo} and friends}
%
%    \begin{macrocode}
\ifluatex
  \expandafter\ltx@firstofone
\else
  \expandafter\ltx@gobble
\fi
{%
  \ltx@IfUndefined{ifincsname}{%
    \ifnum\luatexversion<36 %
      \expandafter\ltx@gobble
    \else
      \expandafter\ltx@firstofone
    \fi
    {%
      \begingroup
        \ifcase0%
            \directlua{%
              if tex.enableprimitives then %
                tex.enableprimitives('HOLOGO@', {'ifincsname'})%
              else %
                tex.print('1')%
              end%
            }%
            \ifx\HOLOGO@ifincsname\@undefined 1\fi%
            \relax
          \expandafter\ltx@firstofone
        \else
          \endgroup
          \expandafter\ltx@gobble
        \fi
        {%
          \global\let\ifincsname\HOLOGO@ifincsname
        }%
      \HOLOGO@temp
    }%
  }{}%
}
%    \end{macrocode}
%    \begin{macrocode}
\ltx@IfUndefined{ifincsname}{%
  \catcode`$=14 %
}{%
  \catcode`$=9 %
}
%    \end{macrocode}
%
%    \begin{macro}{\hologo}
%    \begin{macrocode}
\def\hologo#1{%
$ \ifincsname
$   \ltx@ifundefined{HoLogoCs@\HOLOGO@Variant{#1}}{%
$     #1%
$   }{%
$     \csname HoLogoCs@\HOLOGO@Variant{#1}\endcsname\ltx@firstoftwo
$   }%
$ \else
    \HOLOGO@IfExists\texorpdfstring\texorpdfstring\ltx@firstoftwo
    {%
      \hologoRobust{#1}%
    }{%
      \ltx@ifundefined{HoLogoBkm@\HOLOGO@Variant{#1}}{%
        \ltx@ifundefined{HoLogo@#1}{?#1?}{#1}%
      }{%
        \csname HoLogoBkm@\HOLOGO@Variant{#1}\endcsname
        \ltx@firstoftwo
      }%
    }%
$ \fi
}
%    \end{macrocode}
%    \end{macro}
%    \begin{macro}{\Hologo}
%    \begin{macrocode}
\def\Hologo#1{%
$ \ifincsname
$   \ltx@ifundefined{HoLogoCs@\HOLOGO@Variant{#1}}{%
$     #1%
$   }{%
$     \csname HoLogoCs@\HOLOGO@Variant{#1}\endcsname\ltx@secondoftwo
$   }%
$ \else
    \HOLOGO@IfExists\texorpdfstring\texorpdfstring\ltx@firstoftwo
    {%
      \HologoRobust{#1}%
    }{%
      \ltx@ifundefined{HoLogoBkm@\HOLOGO@Variant{#1}}{%
        \ltx@ifundefined{HoLogo@#1}{?#1?}{#1}%
      }{%
        \csname HoLogoBkm@\HOLOGO@Variant{#1}\endcsname
        \ltx@secondoftwo
      }%
    }%
$ \fi
}
%    \end{macrocode}
%    \end{macro}
%
%    \begin{macro}{\hologoVariant}
%    \begin{macrocode}
\def\hologoVariant#1#2{%
  \ifx\relax#2\relax
    \hologo{#1}%
  \else
$   \ifincsname
$     \ltx@ifundefined{HoLogoCs@#1@#2}{%
$       #1%
$     }{%
$       \csname HoLogoCs@#1@#2\endcsname\ltx@firstoftwo
$     }%
$   \else
      \HOLOGO@IfExists\texorpdfstring\texorpdfstring\ltx@firstoftwo
      {%
        \hologoVariantRobust{#1}{#2}%
      }{%
        \ltx@ifundefined{HoLogoBkm@#1@#2}{%
          \ltx@ifundefined{HoLogo@#1}{?#1?}{#1}%
        }{%
          \csname HoLogoBkm@#1@#2\endcsname
          \ltx@firstoftwo
        }%
      }%
$   \fi
  \fi
}
%    \end{macrocode}
%    \end{macro}
%    \begin{macro}{\HologoVariant}
%    \begin{macrocode}
\def\HologoVariant#1#2{%
  \ifx\relax#2\relax
    \Hologo{#1}%
  \else
$   \ifincsname
$     \ltx@ifundefined{HoLogoCs@#1@#2}{%
$       #1%
$     }{%
$       \csname HoLogoCs@#1@#2\endcsname\ltx@secondoftwo
$     }%
$   \else
      \HOLOGO@IfExists\texorpdfstring\texorpdfstring\ltx@firstoftwo
      {%
        \HologoVariantRobust{#1}{#2}%
      }{%
        \ltx@ifundefined{HoLogoBkm@#1@#2}{%
          \ltx@ifundefined{HoLogo@#1}{?#1?}{#1}%
        }{%
          \csname HoLogoBkm@#1@#2\endcsname
          \ltx@secondoftwo
        }%
      }%
$   \fi
  \fi
}
%    \end{macrocode}
%    \end{macro}
%
%    \begin{macrocode}
\catcode`\$=3 %
%    \end{macrocode}
%
% \subsubsection{\cs{hologoRobust} and friends}
%
%    \begin{macro}{\hologoRobust}
%    \begin{macrocode}
\ltx@IfUndefined{protected}{%
  \ltx@IfUndefined{DeclareRobustCommand}{%
    \def\hologoRobust#1%
  }{%
    \DeclareRobustCommand*\hologoRobust[1]%
  }%
}{%
  \protected\def\hologoRobust#1%
}%
{%
  \edef\HOLOGO@name{#1}%
  \ltx@IfUndefined{HoLogo@\HOLOGO@Variant\HOLOGO@name}{%
    \@PackageError{hologo}{%
      Unknown logo `\HOLOGO@name'%
    }\@ehc
    ?\HOLOGO@name?%
  }{%
    \ltx@IfUndefined{ver@tex4ht.sty}{%
      \HoLogoFont@font\HOLOGO@name{general}{%
        \csname HoLogo@\HOLOGO@Variant\HOLOGO@name\endcsname
        \ltx@firstoftwo
      }%
    }{%
      \ltx@IfUndefined{HoLogoHtml@\HOLOGO@Variant\HOLOGO@name}{%
        \HOLOGO@name
      }{%
        \csname HoLogoHtml@\HOLOGO@Variant\HOLOGO@name\endcsname
        \ltx@firstoftwo
      }%
    }%
  }%
}
%    \end{macrocode}
%    \end{macro}
%    \begin{macro}{\HologoRobust}
%    \begin{macrocode}
\ltx@IfUndefined{protected}{%
  \ltx@IfUndefined{DeclareRobustCommand}{%
    \def\HologoRobust#1%
  }{%
    \DeclareRobustCommand*\HologoRobust[1]%
  }%
}{%
  \protected\def\HologoRobust#1%
}%
{%
  \edef\HOLOGO@name{#1}%
  \ltx@IfUndefined{HoLogo@\HOLOGO@Variant\HOLOGO@name}{%
    \@PackageError{hologo}{%
      Unknown logo `\HOLOGO@name'%
    }\@ehc
    ?\HOLOGO@name?%
  }{%
    \ltx@IfUndefined{ver@tex4ht.sty}{%
      \HoLogoFont@font\HOLOGO@name{general}{%
        \csname HoLogo@\HOLOGO@Variant\HOLOGO@name\endcsname
        \ltx@secondoftwo
      }%
    }{%
      \ltx@IfUndefined{HoLogoHtml@\HOLOGO@Variant\HOLOGO@name}{%
        \expandafter\HOLOGO@Uppercase\HOLOGO@name
      }{%
        \csname HoLogoHtml@\HOLOGO@Variant\HOLOGO@name\endcsname
        \ltx@secondoftwo
      }%
    }%
  }%
}
%    \end{macrocode}
%    \end{macro}
%    \begin{macro}{\hologoVariantRobust}
%    \begin{macrocode}
\ltx@IfUndefined{protected}{%
  \ltx@IfUndefined{DeclareRobustCommand}{%
    \def\hologoVariantRobust#1#2%
  }{%
    \DeclareRobustCommand*\hologoVariantRobust[2]%
  }%
}{%
  \protected\def\hologoVariantRobust#1#2%
}%
{%
  \begingroup
    \hologoLogoSetup{#1}{variant={#2}}%
    \hologoRobust{#1}%
  \endgroup
}
%    \end{macrocode}
%    \end{macro}
%    \begin{macro}{\HologoVariantRobust}
%    \begin{macrocode}
\ltx@IfUndefined{protected}{%
  \ltx@IfUndefined{DeclareRobustCommand}{%
    \def\HologoVariantRobust#1#2%
  }{%
    \DeclareRobustCommand*\HologoVariantRobust[2]%
  }%
}{%
  \protected\def\HologoVariantRobust#1#2%
}%
{%
  \begingroup
    \hologoLogoSetup{#1}{variant={#2}}%
    \HologoRobust{#1}%
  \endgroup
}
%    \end{macrocode}
%    \end{macro}
%
%    \begin{macro}{\hologorobust}
%    Macro \cs{hologorobust} is only defined for compatibility.
%    Its use is deprecated.
%    \begin{macrocode}
\def\hologorobust{\hologoRobust}
%    \end{macrocode}
%    \end{macro}
%
% \subsection{Helpers}
%
%    \begin{macro}{\HOLOGO@Uppercase}
%    Macro \cs{HOLOGO@Uppercase} is restricted to \cs{uppercase},
%    because \hologo{plainTeX} or \hologo{iniTeX} do not provide
%    \cs{MakeUppercase}.
%    \begin{macrocode}
\def\HOLOGO@Uppercase#1{\uppercase{#1}}
%    \end{macrocode}
%    \end{macro}
%
%    \begin{macro}{\HOLOGO@PdfdocUnicode}
%    \begin{macrocode}
\def\HOLOGO@PdfdocUnicode{%
  \ifx\ifHy@unicode\iftrue
    \expandafter\ltx@secondoftwo
  \else
    \expandafter\ltx@firstoftwo
  \fi
}
%    \end{macrocode}
%    \end{macro}
%
%    \begin{macro}{\HOLOGO@Math}
%    \begin{macrocode}
\def\HOLOGO@MathSetup{%
  \mathsurround0pt\relax
  \HOLOGO@IfExists\f@series{%
    \if b\expandafter\ltx@car\f@series x\@nil
      \csname boldmath\endcsname
   \fi
  }{}%
}
%    \end{macrocode}
%    \end{macro}
%
%    \begin{macro}{\HOLOGO@TempDimen}
%    \begin{macrocode}
\dimendef\HOLOGO@TempDimen=\ltx@zero
%    \end{macrocode}
%    \end{macro}
%    \begin{macro}{\HOLOGO@NegativeKerning}
%    \begin{macrocode}
\def\HOLOGO@NegativeKerning#1{%
  \begingroup
    \HOLOGO@TempDimen=0pt\relax
    \comma@parse@normalized{#1}{%
      \ifdim\HOLOGO@TempDimen=0pt %
        \expandafter\HOLOGO@@NegativeKerning\comma@entry
      \fi
      \ltx@gobble
    }%
    \ifdim\HOLOGO@TempDimen<0pt %
      \kern\HOLOGO@TempDimen
    \fi
  \endgroup
}
%    \end{macrocode}
%    \end{macro}
%    \begin{macro}{\HOLOGO@@NegativeKerning}
%    \begin{macrocode}
\def\HOLOGO@@NegativeKerning#1#2{%
  \setbox\ltx@zero\hbox{#1#2}%
  \HOLOGO@TempDimen=\wd\ltx@zero
  \setbox\ltx@zero\hbox{#1\kern0pt#2}%
  \advance\HOLOGO@TempDimen by -\wd\ltx@zero
}
%    \end{macrocode}
%    \end{macro}
%
%    \begin{macro}{\HOLOGO@SpaceFactor}
%    \begin{macrocode}
\def\HOLOGO@SpaceFactor{%
  \spacefactor1000 %
}
%    \end{macrocode}
%    \end{macro}
%
%    \begin{macro}{\HOLOGO@Span}
%    \begin{macrocode}
\def\HOLOGO@Span#1#2{%
  \HCode{<span class="HoLogo-#1">}%
  #2%
  \HCode{</span>}%
}
%    \end{macrocode}
%    \end{macro}
%
% \subsubsection{Text subscript}
%
%    \begin{macro}{\HOLOGO@SubScript}%
%    \begin{macrocode}
\def\HOLOGO@SubScript#1{%
  \ltx@IfUndefined{textsubscript}{%
    \ltx@IfUndefined{text}{%
      \ltx@mbox{%
        \mathsurround=0pt\relax
        $%
          _{%
            \ltx@IfUndefined{sf@size}{%
              \mathrm{#1}%
            }{%
              \mbox{%
                \fontsize\sf@size{0pt}\selectfont
                #1%
              }%
            }%
          }%
        $%
      }%
    }{%
      \ltx@mbox{%
        \mathsurround=0pt\relax
        $_{\text{#1}}$%
      }%
    }%
  }{%
    \textsubscript{#1}%
  }%
}
%    \end{macrocode}
%    \end{macro}
%
% \subsection{\hologo{TeX} and friends}
%
% \subsubsection{\hologo{TeX}}
%
%    \begin{macro}{\HoLogo@TeX}
%    Source: \hologo{LaTeX} kernel.
%    \begin{macrocode}
\def\HoLogo@TeX#1{%
  T\kern-.1667em\lower.5ex\hbox{E}\kern-.125emX\HOLOGO@SpaceFactor
}
%    \end{macrocode}
%    \end{macro}
%    \begin{macro}{\HoLogoHtml@TeX}
%    \begin{macrocode}
\def\HoLogoHtml@TeX#1{%
  \HoLogoCss@TeX
  \HOLOGO@Span{TeX}{%
    T%
    \HOLOGO@Span{e}{%
      E%
    }%
    X%
  }%
}
%    \end{macrocode}
%    \end{macro}
%    \begin{macro}{\HoLogoCss@TeX}
%    \begin{macrocode}
\def\HoLogoCss@TeX{%
  \Css{%
    span.HoLogo-TeX span.HoLogo-e{%
      position:relative;%
      top:.5ex;%
      margin-left:-.1667em;%
      margin-right:-.125em;%
    }%
  }%
  \Css{%
    a span.HoLogo-TeX span.HoLogo-e{%
      text-decoration:none;%
    }%
  }%
  \global\let\HoLogoCss@TeX\relax
}
%    \end{macrocode}
%    \end{macro}
%
% \subsubsection{\hologo{plainTeX}}
%
%    \begin{macro}{\HoLogo@plainTeX@space}
%    Source: ``The \hologo{TeX}book''
%    \begin{macrocode}
\def\HoLogo@plainTeX@space#1{%
  \HOLOGO@mbox{#1{p}{P}lain}\HOLOGO@space\hologo{TeX}%
}
%    \end{macrocode}
%    \end{macro}
%    \begin{macro}{\HoLogoCs@plainTeX@space}
%    \begin{macrocode}
\def\HoLogoCs@plainTeX@space#1{#1{p}{P}lain TeX}%
%    \end{macrocode}
%    \end{macro}
%    \begin{macro}{\HoLogoBkm@plainTeX@space}
%    \begin{macrocode}
\def\HoLogoBkm@plainTeX@space#1{%
  #1{p}{P}lain \hologo{TeX}%
}
%    \end{macrocode}
%    \end{macro}
%    \begin{macro}{\HoLogoHtml@plainTeX@space}
%    \begin{macrocode}
\def\HoLogoHtml@plainTeX@space#1{%
  #1{p}{P}lain \hologo{TeX}%
}
%    \end{macrocode}
%    \end{macro}
%
%    \begin{macro}{\HoLogo@plainTeX@hyphen}
%    \begin{macrocode}
\def\HoLogo@plainTeX@hyphen#1{%
  \HOLOGO@mbox{#1{p}{P}lain}\HOLOGO@hyphen\hologo{TeX}%
}
%    \end{macrocode}
%    \end{macro}
%    \begin{macro}{\HoLogoCs@plainTeX@hyphen}
%    \begin{macrocode}
\def\HoLogoCs@plainTeX@hyphen#1{#1{p}{P}lain-TeX}
%    \end{macrocode}
%    \end{macro}
%    \begin{macro}{\HoLogoBkm@plainTeX@hyphen}
%    \begin{macrocode}
\def\HoLogoBkm@plainTeX@hyphen#1{%
  #1{p}{P}lain-\hologo{TeX}%
}
%    \end{macrocode}
%    \end{macro}
%    \begin{macro}{\HoLogoHtml@plainTeX@hyphen}
%    \begin{macrocode}
\def\HoLogoHtml@plainTeX@hyphen#1{%
  #1{p}{P}lain-\hologo{TeX}%
}
%    \end{macrocode}
%    \end{macro}
%
%    \begin{macro}{\HoLogo@plainTeX@runtogether}
%    \begin{macrocode}
\def\HoLogo@plainTeX@runtogether#1{%
  \HOLOGO@mbox{#1{p}{P}lain\hologo{TeX}}%
}
%    \end{macrocode}
%    \end{macro}
%    \begin{macro}{\HoLogoCs@plainTeX@runtogether}
%    \begin{macrocode}
\def\HoLogoCs@plainTeX@runtogether#1{#1{p}{P}lainTeX}
%    \end{macrocode}
%    \end{macro}
%    \begin{macro}{\HoLogoBkm@plainTeX@runtogether}
%    \begin{macrocode}
\def\HoLogoBkm@plainTeX@runtogether#1{%
  #1{p}{P}lain\hologo{TeX}%
}
%    \end{macrocode}
%    \end{macro}
%    \begin{macro}{\HoLogoHtml@plainTeX@runtogether}
%    \begin{macrocode}
\def\HoLogoHtml@plainTeX@runtogether#1{%
  #1{p}{P}lain\hologo{TeX}%
}
%    \end{macrocode}
%    \end{macro}
%
%    \begin{macro}{\HoLogo@plainTeX}
%    \begin{macrocode}
\def\HoLogo@plainTeX{\HoLogo@plainTeX@space}
%    \end{macrocode}
%    \end{macro}
%    \begin{macro}{\HoLogoCs@plainTeX}
%    \begin{macrocode}
\def\HoLogoCs@plainTeX{\HoLogoCs@plainTeX@space}
%    \end{macrocode}
%    \end{macro}
%    \begin{macro}{\HoLogoBkm@plainTeX}
%    \begin{macrocode}
\def\HoLogoBkm@plainTeX{\HoLogoBkm@plainTeX@space}
%    \end{macrocode}
%    \end{macro}
%    \begin{macro}{\HoLogoHtml@plainTeX}
%    \begin{macrocode}
\def\HoLogoHtml@plainTeX{\HoLogoHtml@plainTeX@space}
%    \end{macrocode}
%    \end{macro}
%
% \subsubsection{\hologo{LaTeX}}
%
%    Source: \hologo{LaTeX} kernel.
%\begin{quote}
%\begin{verbatim}
%\DeclareRobustCommand{\LaTeX}{%
%  L%
%  \kern-.36em%
%  {%
%    \sbox\z@ T%
%    \vbox to\ht\z@{%
%      \hbox{%
%        \check@mathfonts
%        \fontsize\sf@size\z@
%        \math@fontsfalse
%        \selectfont
%        A%
%      }%
%      \vss
%    }%
%  }%
%  \kern-.15em%
%  \TeX
%}
%\end{verbatim}
%\end{quote}
%
%    \begin{macro}{\HoLogo@La}
%    \begin{macrocode}
\def\HoLogo@La#1{%
  L%
  \kern-.36em%
  \begingroup
    \setbox\ltx@zero\hbox{T}%
    \vbox to\ht\ltx@zero{%
      \hbox{%
        \ltx@ifundefined{check@mathfonts}{%
          \csname sevenrm\endcsname
        }{%
          \check@mathfonts
          \fontsize\sf@size{0pt}%
          \math@fontsfalse\selectfont
        }%
        A%
      }%
      \vss
    }%
  \endgroup
}
%    \end{macrocode}
%    \end{macro}
%
%    \begin{macro}{\HoLogo@LaTeX}
%    Source: \hologo{LaTeX} kernel.
%    \begin{macrocode}
\def\HoLogo@LaTeX#1{%
  \hologo{La}%
  \kern-.15em%
  \hologo{TeX}%
}
%    \end{macrocode}
%    \end{macro}
%    \begin{macro}{\HoLogoHtml@LaTeX}
%    \begin{macrocode}
\def\HoLogoHtml@LaTeX#1{%
  \HoLogoCss@LaTeX
  \HOLOGO@Span{LaTeX}{%
    L%
    \HOLOGO@Span{a}{%
      A%
    }%
    \hologo{TeX}%
  }%
}
%    \end{macrocode}
%    \end{macro}
%    \begin{macro}{\HoLogoCss@LaTeX}
%    \begin{macrocode}
\def\HoLogoCss@LaTeX{%
  \Css{%
    span.HoLogo-LaTeX span.HoLogo-a{%
      position:relative;%
      top:-.5ex;%
      margin-left:-.36em;%
      margin-right:-.15em;%
      font-size:85\%;%
    }%
  }%
  \global\let\HoLogoCss@LaTeX\relax
}
%    \end{macrocode}
%    \end{macro}
%
% \subsubsection{\hologo{(La)TeX}}
%
%    \begin{macro}{\HoLogo@LaTeXTeX}
%    The kerning around the parentheses is taken
%    from package \xpackage{dtklogos} \cite{dtklogos}.
%\begin{quote}
%\begin{verbatim}
%\DeclareRobustCommand{\LaTeXTeX}{%
%  (%
%  \kern-.15em%
%  L%
%  \kern-.36em%
%  {%
%    \sbox\z@ T%
%    \vbox to\ht0{%
%      \hbox{%
%        $\m@th$%
%        \csname S@\f@size\endcsname
%        \fontsize\sf@size\z@
%        \math@fontsfalse
%        \selectfont
%        A%
%      }%
%      \vss
%    }%
%  }%
%  \kern-.2em%
%  )%
%  \kern-.15em%
%  \TeX
%}
%\end{verbatim}
%\end{quote}
%    \begin{macrocode}
\def\HoLogo@LaTeXTeX#1{%
  (%
  \kern-.15em%
  \hologo{La}%
  \kern-.2em%
  )%
  \kern-.15em%
  \hologo{TeX}%
}
%    \end{macrocode}
%    \end{macro}
%    \begin{macro}{\HoLogoBkm@LaTeXTeX}
%    \begin{macrocode}
\def\HoLogoBkm@LaTeXTeX#1{(La)TeX}
%    \end{macrocode}
%    \end{macro}
%
%    \begin{macro}{\HoLogo@(La)TeX}
%    \begin{macrocode}
\expandafter
\let\csname HoLogo@(La)TeX\endcsname\HoLogo@LaTeXTeX
%    \end{macrocode}
%    \end{macro}
%    \begin{macro}{\HoLogoBkm@(La)TeX}
%    \begin{macrocode}
\expandafter
\let\csname HoLogoBkm@(La)TeX\endcsname\HoLogoBkm@LaTeXTeX
%    \end{macrocode}
%    \end{macro}
%    \begin{macro}{\HoLogoHtml@LaTeXTeX}
%    \begin{macrocode}
\def\HoLogoHtml@LaTeXTeX#1{%
  \HoLogoCss@LaTeXTeX
  \HOLOGO@Span{LaTeXTeX}{%
    (%
    \HOLOGO@Span{L}{L}%
    \HOLOGO@Span{a}{A}%
    \HOLOGO@Span{ParenRight}{)}%
    \hologo{TeX}%
  }%
}
%    \end{macrocode}
%    \end{macro}
%    \begin{macro}{\HoLogoHtml@(La)TeX}
%    Kerning after opening parentheses and before closing parentheses
%    is $-0.1$\,em. The original values $-0.15$\,em
%    looked too ugly for a serif font.
%    \begin{macrocode}
\expandafter
\let\csname HoLogoHtml@(La)TeX\endcsname\HoLogoHtml@LaTeXTeX
%    \end{macrocode}
%    \end{macro}
%    \begin{macro}{\HoLogoCss@LaTeXTeX}
%    \begin{macrocode}
\def\HoLogoCss@LaTeXTeX{%
  \Css{%
    span.HoLogo-LaTeXTeX span.HoLogo-L{%
      margin-left:-.1em;%
    }%
  }%
  \Css{%
    span.HoLogo-LaTeXTeX span.HoLogo-a{%
      position:relative;%
      top:-.5ex;%
      margin-left:-.36em;%
      margin-right:-.1em;%
      font-size:85\%;%
    }%
  }%
  \Css{%
    span.HoLogo-LaTeXTeX span.HoLogo-ParenRight{%
      margin-right:-.15em;%
    }%
  }%
  \global\let\HoLogoCss@LaTeXTeX\relax
}
%    \end{macrocode}
%    \end{macro}
%
% \subsubsection{\hologo{LaTeXe}}
%
%    \begin{macro}{\HoLogo@LaTeXe}
%    Source: \hologo{LaTeX} kernel
%    \begin{macrocode}
\def\HoLogo@LaTeXe#1{%
  \hologo{LaTeX}%
  \kern.15em%
  \hbox{%
    \HOLOGO@MathSetup
    2%
    $_{\textstyle\varepsilon}$%
  }%
}
%    \end{macrocode}
%    \end{macro}
%
%    \begin{macro}{\HoLogoCs@LaTeXe}
%    \begin{macrocode}
\ifnum64=`\^^^^0040\relax % test for big chars of LuaTeX/XeTeX
  \catcode`\$=9 %
  \catcode`\&=14 %
\else
  \catcode`\$=14 %
  \catcode`\&=9 %
\fi
\def\HoLogoCs@LaTeXe#1{%
  LaTeX2%
$ \string ^^^^0395%
& e%
}%
\catcode`\$=3 %
\catcode`\&=4 %
%    \end{macrocode}
%    \end{macro}
%
%    \begin{macro}{\HoLogoBkm@LaTeXe}
%    \begin{macrocode}
\def\HoLogoBkm@LaTeXe#1{%
  \hologo{LaTeX}%
  2%
  \HOLOGO@PdfdocUnicode{e}{\textepsilon}%
}
%    \end{macrocode}
%    \end{macro}
%
%    \begin{macro}{\HoLogoHtml@LaTeXe}
%    \begin{macrocode}
\def\HoLogoHtml@LaTeXe#1{%
  \HoLogoCss@LaTeXe
  \HOLOGO@Span{LaTeX2e}{%
    \hologo{LaTeX}%
    \HOLOGO@Span{2}{2}%
    \HOLOGO@Span{e}{%
      \HOLOGO@MathSetup
      \ensuremath{\textstyle\varepsilon}%
    }%
  }%
}
%    \end{macrocode}
%    \end{macro}
%    \begin{macro}{\HoLogoCss@LaTeXe}
%    \begin{macrocode}
\def\HoLogoCss@LaTeXe{%
  \Css{%
    span.HoLogo-LaTeX2e span.HoLogo-2{%
      padding-left:.15em;%
    }%
  }%
  \Css{%
    span.HoLogo-LaTeX2e span.HoLogo-e{%
      position:relative;%
      top:.35ex;%
      text-decoration:none;%
    }%
  }%
  \global\let\HoLogoCss@LaTeXe\relax
}
%    \end{macrocode}
%    \end{macro}
%
%    \begin{macro}{\HoLogo@LaTeX2e}
%    \begin{macrocode}
\expandafter
\let\csname HoLogo@LaTeX2e\endcsname\HoLogo@LaTeXe
%    \end{macrocode}
%    \end{macro}
%    \begin{macro}{\HoLogoCs@LaTeX2e}
%    \begin{macrocode}
\expandafter
\let\csname HoLogoCs@LaTeX2e\endcsname\HoLogoCs@LaTeXe
%    \end{macrocode}
%    \end{macro}
%    \begin{macro}{\HoLogoBkm@LaTeX2e}
%    \begin{macrocode}
\expandafter
\let\csname HoLogoBkm@LaTeX2e\endcsname\HoLogoBkm@LaTeXe
%    \end{macrocode}
%    \end{macro}
%    \begin{macro}{\HoLogoHtml@LaTeX2e}
%    \begin{macrocode}
\expandafter
\let\csname HoLogoHtml@LaTeX2e\endcsname\HoLogoHtml@LaTeXe
%    \end{macrocode}
%    \end{macro}
%
% \subsubsection{\hologo{LaTeX3}}
%
%    \begin{macro}{\HoLogo@LaTeX3}
%    Source: \hologo{LaTeX} kernel
%    \begin{macrocode}
\expandafter\def\csname HoLogo@LaTeX3\endcsname#1{%
  \hologo{LaTeX}%
  3%
}
%    \end{macrocode}
%    \end{macro}
%
%    \begin{macro}{\HoLogoBkm@LaTeX3}
%    \begin{macrocode}
\expandafter\def\csname HoLogoBkm@LaTeX3\endcsname#1{%
  \hologo{LaTeX}%
  3%
}
%    \end{macrocode}
%    \end{macro}
%    \begin{macro}{\HoLogoHtml@LaTeX3}
%    \begin{macrocode}
\expandafter
\let\csname HoLogoHtml@LaTeX3\expandafter\endcsname
\csname HoLogo@LaTeX3\endcsname
%    \end{macrocode}
%    \end{macro}
%
% \subsubsection{\hologo{LaTeXML}}
%
%    \begin{macro}{\HoLogo@LaTeXML}
%    \begin{macrocode}
\def\HoLogo@LaTeXML#1{%
  \HOLOGO@mbox{%
    \hologo{La}%
    \kern-.15em%
    T%
    \kern-.1667em%
    \lower.5ex\hbox{E}%
    \kern-.125em%
    \HoLogoFont@font{LaTeXML}{sc}{xml}%
  }%
}
%    \end{macrocode}
%    \end{macro}
%    \begin{macro}{\HoLogoHtml@pdfLaTeX}
%    \begin{macrocode}
\def\HoLogoHtml@LaTeXML#1{%
  \HOLOGO@Span{LaTeXML}{%
    \HoLogoCss@LaTeX
    \HoLogoCss@TeX
    \HOLOGO@Span{LaTeX}{%
      L%
      \HOLOGO@Span{a}{%
        A%
      }%
    }%
    \HOLOGO@Span{TeX}{%
      T%
      \HOLOGO@Span{e}{%
        E%
      }%
    }%
    \HCode{<span style="font-variant: small-caps;">}%
    xml%
    \HCode{</span>}%
  }%
}
%    \end{macrocode}
%    \end{macro}
%
% \subsubsection{\hologo{eTeX}}
%
%    \begin{macro}{\HoLogo@eTeX}
%    Source: package \xpackage{etex}
%    \begin{macrocode}
\def\HoLogo@eTeX#1{%
  \ltx@mbox{%
    \HOLOGO@MathSetup
    $\varepsilon$%
    -%
    \HOLOGO@NegativeKerning{-T,T-,To}%
    \hologo{TeX}%
  }%
}
%    \end{macrocode}
%    \end{macro}
%    \begin{macro}{\HoLogoCs@eTeX}
%    \begin{macrocode}
\ifnum64=`\^^^^0040\relax % test for big chars of LuaTeX/XeTeX
  \catcode`\$=9 %
  \catcode`\&=14 %
\else
  \catcode`\$=14 %
  \catcode`\&=9 %
\fi
\def\HoLogoCs@eTeX#1{%
$ #1{\string ^^^^0395}{\string ^^^^03b5}%
& #1{e}{E}%
  TeX%
}%
\catcode`\$=3 %
\catcode`\&=4 %
%    \end{macrocode}
%    \end{macro}
%    \begin{macro}{\HoLogoBkm@eTeX}
%    \begin{macrocode}
\def\HoLogoBkm@eTeX#1{%
  \HOLOGO@PdfdocUnicode{#1{e}{E}}{\textepsilon}%
  -%
  \hologo{TeX}%
}
%    \end{macrocode}
%    \end{macro}
%    \begin{macro}{\HoLogoHtml@eTeX}
%    \begin{macrocode}
\def\HoLogoHtml@eTeX#1{%
  \ltx@mbox{%
    \HOLOGO@MathSetup
    $\varepsilon$%
    -%
    \hologo{TeX}%
  }%
}
%    \end{macrocode}
%    \end{macro}
%
% \subsubsection{\hologo{iniTeX}}
%
%    \begin{macro}{\HoLogo@iniTeX}
%    \begin{macrocode}
\def\HoLogo@iniTeX#1{%
  \HOLOGO@mbox{%
    #1{i}{I}ni\hologo{TeX}%
  }%
}
%    \end{macrocode}
%    \end{macro}
%    \begin{macro}{\HoLogoCs@iniTeX}
%    \begin{macrocode}
\def\HoLogoCs@iniTeX#1{#1{i}{I}niTeX}
%    \end{macrocode}
%    \end{macro}
%    \begin{macro}{\HoLogoBkm@iniTeX}
%    \begin{macrocode}
\def\HoLogoBkm@iniTeX#1{%
  #1{i}{I}ni\hologo{TeX}%
}
%    \end{macrocode}
%    \end{macro}
%    \begin{macro}{\HoLogoHtml@iniTeX}
%    \begin{macrocode}
\let\HoLogoHtml@iniTeX\HoLogo@iniTeX
%    \end{macrocode}
%    \end{macro}
%
% \subsubsection{\hologo{virTeX}}
%
%    \begin{macro}{\HoLogo@virTeX}
%    \begin{macrocode}
\def\HoLogo@virTeX#1{%
  \HOLOGO@mbox{%
    #1{v}{V}ir\hologo{TeX}%
  }%
}
%    \end{macrocode}
%    \end{macro}
%    \begin{macro}{\HoLogoCs@virTeX}
%    \begin{macrocode}
\def\HoLogoCs@virTeX#1{#1{v}{V}irTeX}
%    \end{macrocode}
%    \end{macro}
%    \begin{macro}{\HoLogoBkm@virTeX}
%    \begin{macrocode}
\def\HoLogoBkm@virTeX#1{%
  #1{v}{V}ir\hologo{TeX}%
}
%    \end{macrocode}
%    \end{macro}
%    \begin{macro}{\HoLogoHtml@virTeX}
%    \begin{macrocode}
\let\HoLogoHtml@virTeX\HoLogo@virTeX
%    \end{macrocode}
%    \end{macro}
%
% \subsubsection{\hologo{SliTeX}}
%
% \paragraph{Definitions of the three variants.}
%
%    \begin{macro}{\HoLogo@SLiTeX@lift}
%    \begin{macrocode}
\def\HoLogo@SLiTeX@lift#1{%
  \HoLogoFont@font{SliTeX}{rm}{%
    S%
    \kern-.06em%
    L%
    \kern-.18em%
    \raise.32ex\hbox{\HoLogoFont@font{SliTeX}{sc}{i}}%
    \HOLOGO@discretionary
    \kern-.06em%
    \hologo{TeX}%
  }%
}
%    \end{macrocode}
%    \end{macro}
%    \begin{macro}{\HoLogoBkm@SLiTeX@lift}
%    \begin{macrocode}
\def\HoLogoBkm@SLiTeX@lift#1{SLiTeX}
%    \end{macrocode}
%    \end{macro}
%    \begin{macro}{\HoLogoHtml@SLiTeX@lift}
%    \begin{macrocode}
\def\HoLogoHtml@SLiTeX@lift#1{%
  \HoLogoCss@SLiTeX@lift
  \HOLOGO@Span{SLiTeX-lift}{%
    \HoLogoFont@font{SliTeX}{rm}{%
      S%
      \HOLOGO@Span{L}{L}%
      \HOLOGO@Span{i}{i}%
      \hologo{TeX}%
    }%
  }%
}
%    \end{macrocode}
%    \end{macro}
%    \begin{macro}{\HoLogoCss@SLiTeX@lift}
%    \begin{macrocode}
\def\HoLogoCss@SLiTeX@lift{%
  \Css{%
    span.HoLogo-SLiTeX-lift span.HoLogo-L{%
      margin-left:-.06em;%
      margin-right:-.18em;%
    }%
  }%
  \Css{%
    span.HoLogo-SLiTeX-lift span.HoLogo-i{%
      position:relative;%
      top:-.32ex;%
      margin-right:-.06em;%
      font-variant:small-caps;%
    }%
  }%
  \global\let\HoLogoCss@SLiTeX@lift\relax
}
%    \end{macrocode}
%    \end{macro}
%
%    \begin{macro}{\HoLogo@SliTeX@simple}
%    \begin{macrocode}
\def\HoLogo@SliTeX@simple#1{%
  \HoLogoFont@font{SliTeX}{rm}{%
    \ltx@mbox{%
      \HoLogoFont@font{SliTeX}{sc}{Sli}%
    }%
    \HOLOGO@discretionary
    \hologo{TeX}%
  }%
}
%    \end{macrocode}
%    \end{macro}
%    \begin{macro}{\HoLogoBkm@SliTeX@simple}
%    \begin{macrocode}
\def\HoLogoBkm@SliTeX@simple#1{SliTeX}
%    \end{macrocode}
%    \end{macro}
%    \begin{macro}{\HoLogoHtml@SliTeX@simple}
%    \begin{macrocode}
\let\HoLogoHtml@SliTeX@simple\HoLogo@SliTeX@simple
%    \end{macrocode}
%    \end{macro}
%
%    \begin{macro}{\HoLogo@SliTeX@narrow}
%    \begin{macrocode}
\def\HoLogo@SliTeX@narrow#1{%
  \HoLogoFont@font{SliTeX}{rm}{%
    \ltx@mbox{%
      S%
      \kern-.06em%
      \HoLogoFont@font{SliTeX}{sc}{%
        l%
        \kern-.035em%
        i%
      }%
    }%
    \HOLOGO@discretionary
    \kern-.06em%
    \hologo{TeX}%
  }%
}
%    \end{macrocode}
%    \end{macro}
%    \begin{macro}{\HoLogoBkm@SliTeX@narrow}
%    \begin{macrocode}
\def\HoLogoBkm@SliTeX@narrow#1{SliTeX}
%    \end{macrocode}
%    \end{macro}
%    \begin{macro}{\HoLogoHtml@SliTeX@narrow}
%    \begin{macrocode}
\def\HoLogoHtml@SliTeX@narrow#1{%
  \HoLogoCss@SliTeX@narrow
  \HOLOGO@Span{SliTeX-narrow}{%
    \HoLogoFont@font{SliTeX}{rm}{%
      S%
        \HOLOGO@Span{l}{l}%
        \HOLOGO@Span{i}{i}%
      \hologo{TeX}%
    }%
  }%
}
%    \end{macrocode}
%    \end{macro}
%    \begin{macro}{\HoLogoCss@SliTeX@narrow}
%    \begin{macrocode}
\def\HoLogoCss@SliTeX@narrow{%
  \Css{%
    span.HoLogo-SliTeX-narrow span.HoLogo-l{%
      margin-left:-.06em;%
      margin-right:-.035em;%
      font-variant:small-caps;%
    }%
  }%
  \Css{%
    span.HoLogo-SliTeX-narrow span.HoLogo-i{%
      margin-right:-.06em;%
      font-variant:small-caps;%
    }%
  }%
  \global\let\HoLogoCss@SliTeX@narrow\relax
}
%    \end{macrocode}
%    \end{macro}
%
% \paragraph{Macro set completion.}
%
%    \begin{macro}{\HoLogo@SLiTeX@simple}
%    \begin{macrocode}
\def\HoLogo@SLiTeX@simple{\HoLogo@SliTeX@simple}
%    \end{macrocode}
%    \end{macro}
%    \begin{macro}{\HoLogoBkm@SLiTeX@simple}
%    \begin{macrocode}
\def\HoLogoBkm@SLiTeX@simple{\HoLogoBkm@SliTeX@simple}
%    \end{macrocode}
%    \end{macro}
%    \begin{macro}{\HoLogoHtml@SLiTeX@simple}
%    \begin{macrocode}
\def\HoLogoHtml@SLiTeX@simple{\HoLogoHtml@SliTeX@simple}
%    \end{macrocode}
%    \end{macro}
%
%    \begin{macro}{\HoLogo@SLiTeX@narrow}
%    \begin{macrocode}
\def\HoLogo@SLiTeX@narrow{\HoLogo@SliTeX@narrow}
%    \end{macrocode}
%    \end{macro}
%    \begin{macro}{\HoLogoBkm@SLiTeX@narrow}
%    \begin{macrocode}
\def\HoLogoBkm@SLiTeX@narrow{\HoLogoBkm@SliTeX@narrow}
%    \end{macrocode}
%    \end{macro}
%    \begin{macro}{\HoLogoHtml@SLiTeX@narrow}
%    \begin{macrocode}
\def\HoLogoHtml@SLiTeX@narrow{\HoLogoHtml@SliTeX@narrow}
%    \end{macrocode}
%    \end{macro}
%
%    \begin{macro}{\HoLogo@SliTeX@lift}
%    \begin{macrocode}
\def\HoLogo@SliTeX@lift{\HoLogo@SLiTeX@lift}
%    \end{macrocode}
%    \end{macro}
%    \begin{macro}{\HoLogoBkm@SliTeX@lift}
%    \begin{macrocode}
\def\HoLogoBkm@SliTeX@lift{\HoLogoBkm@SLiTeX@lift}
%    \end{macrocode}
%    \end{macro}
%    \begin{macro}{\HoLogoHtml@SliTeX@lift}
%    \begin{macrocode}
\def\HoLogoHtml@SliTeX@lift{\HoLogoHtml@SLiTeX@lift}
%    \end{macrocode}
%    \end{macro}
%
% \paragraph{Defaults.}
%
%    \begin{macro}{\HoLogo@SLiTeX}
%    \begin{macrocode}
\def\HoLogo@SLiTeX{\HoLogo@SLiTeX@lift}
%    \end{macrocode}
%    \end{macro}
%    \begin{macro}{\HoLogoBkm@SLiTeX}
%    \begin{macrocode}
\def\HoLogoBkm@SLiTeX{\HoLogoBkm@SLiTeX@lift}
%    \end{macrocode}
%    \end{macro}
%    \begin{macro}{\HoLogoHtml@SLiTeX}
%    \begin{macrocode}
\def\HoLogoHtml@SLiTeX{\HoLogoHtml@SLiTeX@lift}
%    \end{macrocode}
%    \end{macro}
%
%    \begin{macro}{\HoLogo@SliTeX}
%    \begin{macrocode}
\def\HoLogo@SliTeX{\HoLogo@SliTeX@narrow}
%    \end{macrocode}
%    \end{macro}
%    \begin{macro}{\HoLogoBkm@SliTeX}
%    \begin{macrocode}
\def\HoLogoBkm@SliTeX{\HoLogoBkm@SliTeX@narrow}
%    \end{macrocode}
%    \end{macro}
%    \begin{macro}{\HoLogoHtml@SliTeX}
%    \begin{macrocode}
\def\HoLogoHtml@SliTeX{\HoLogoHtml@SliTeX@narrow}
%    \end{macrocode}
%    \end{macro}
%
% \subsubsection{\hologo{LuaTeX}}
%
%    \begin{macro}{\HoLogo@LuaTeX}
%    The kerning is an idea of Hans Hagen, see mailing list
%    `luatex at tug dot org' in March 2010.
%    \begin{macrocode}
\def\HoLogo@LuaTeX#1{%
  \HOLOGO@mbox{%
    Lua%
    \HOLOGO@NegativeKerning{aT,oT,To}%
    \hologo{TeX}%
  }%
}
%    \end{macrocode}
%    \end{macro}
%    \begin{macro}{\HoLogoHtml@LuaTeX}
%    \begin{macrocode}
\let\HoLogoHtml@LuaTeX\HoLogo@LuaTeX
%    \end{macrocode}
%    \end{macro}
%
% \subsubsection{\hologo{LuaLaTeX}}
%
%    \begin{macro}{\HoLogo@LuaLaTeX}
%    \begin{macrocode}
\def\HoLogo@LuaLaTeX#1{%
  \HOLOGO@mbox{%
    Lua%
    \hologo{LaTeX}%
  }%
}
%    \end{macrocode}
%    \end{macro}
%    \begin{macro}{\HoLogoHtml@LuaLaTeX}
%    \begin{macrocode}
\let\HoLogoHtml@LuaLaTeX\HoLogo@LuaLaTeX
%    \end{macrocode}
%    \end{macro}
%
% \subsubsection{\hologo{XeTeX}, \hologo{XeLaTeX}}
%
%    \begin{macro}{\HOLOGO@IfCharExists}
%    \begin{macrocode}
\ifluatex
  \ifnum\luatexversion<36 %
  \else
    \def\HOLOGO@IfCharExists#1{%
      \ifnum
        \directlua{%
           if luaotfload and luaotfload.aux then
             if luaotfload.aux.font_has_glyph(%
                    font.current(), \number#1) then % 	 
	       tex.print("1") % 	 
	     end % 	 
	   elseif font and font.fonts and font.current then %
            local f = font.fonts[font.current()]%
            if f.characters and f.characters[\number#1] then %
              tex.print("1")%
            end %
          end%
        }0=\ltx@zero
        \expandafter\ltx@secondoftwo
      \else
        \expandafter\ltx@firstoftwo
      \fi
    }%
  \fi
\fi
\ltx@IfUndefined{HOLOGO@IfCharExists}{%
  \def\HOLOGO@@IfCharExists#1{%
    \begingroup
      \tracinglostchars=\ltx@zero
      \setbox\ltx@zero=\hbox{%
        \kern7sp\char#1\relax
        \ifnum\lastkern>\ltx@zero
          \expandafter\aftergroup\csname iffalse\endcsname
        \else
          \expandafter\aftergroup\csname iftrue\endcsname
        \fi
      }%
      % \if{true|false} from \aftergroup
      \endgroup
      \expandafter\ltx@firstoftwo
    \else
      \endgroup
      \expandafter\ltx@secondoftwo
    \fi
  }%
  \ifxetex
    \ltx@IfUndefined{XeTeXfonttype}{}{%
      \ltx@IfUndefined{XeTeXcharglyph}{}{%
        \def\HOLOGO@IfCharExists#1{%
          \ifnum\XeTeXfonttype\font>\ltx@zero
            \expandafter\ltx@firstofthree
          \else
            \expandafter\ltx@gobble
          \fi
          {%
            \ifnum\XeTeXcharglyph#1>\ltx@zero
              \expandafter\ltx@firstoftwo
            \else
              \expandafter\ltx@secondoftwo
            \fi
          }%
          \HOLOGO@@IfCharExists{#1}%
        }%
      }%
    }%
  \fi
}{}
\ltx@ifundefined{HOLOGO@IfCharExists}{%
  \ifnum64=`\^^^^0040\relax % test for big chars of LuaTeX/XeTeX
    \let\HOLOGO@IfCharExists\HOLOGO@@IfCharExists
  \else
    \def\HOLOGO@IfCharExists#1{%
      \ifnum#1>255 %
        \expandafter\ltx@fourthoffour
      \fi
      \HOLOGO@@IfCharExists{#1}%
    }%
  \fi
}{}
%    \end{macrocode}
%    \end{macro}
%
%    \begin{macro}{\HoLogo@Xe}
%    Source: package \xpackage{dtklogos}
%    \begin{macrocode}
\def\HoLogo@Xe#1{%
  X%
  \kern-.1em\relax
  \HOLOGO@IfCharExists{"018E}{%
    \lower.5ex\hbox{\char"018E}%
  }{%
    \chardef\HOLOGO@choice=\ltx@zero
    \ifdim\fontdimen\ltx@one\font>0pt %
      \ltx@IfUndefined{rotatebox}{%
        \ltx@IfUndefined{pgftext}{%
          \ltx@IfUndefined{psscalebox}{%
            \ltx@IfUndefined{HOLOGO@ScaleBox@\hologoDriver}{%
            }{%
              \chardef\HOLOGO@choice=4 %
            }%
          }{%
            \chardef\HOLOGO@choice=3 %
          }%
        }{%
          \chardef\HOLOGO@choice=2 %
        }%
      }{%
        \chardef\HOLOGO@choice=1 %
      }%
      \ifcase\HOLOGO@choice
        \HOLOGO@WarningUnsupportedDriver{Xe}%
        e%
      \or % 1: \rotatebox
        \begingroup
          \setbox\ltx@zero\hbox{\rotatebox{180}{E}}%
          \ltx@LocDimenA=\dp\ltx@zero
          \advance\ltx@LocDimenA by -.5ex\relax
          \raise\ltx@LocDimenA\box\ltx@zero
        \endgroup
      \or % 2: \pgftext
        \lower.5ex\hbox{%
          \pgfpicture
            \pgftext[rotate=180]{E}%
          \endpgfpicture
        }%
      \or % 3: \psscalebox
        \begingroup
          \setbox\ltx@zero\hbox{\psscalebox{-1 -1}{E}}%
          \ltx@LocDimenA=\dp\ltx@zero
          \advance\ltx@LocDimenA by -.5ex\relax
          \raise\ltx@LocDimenA\box\ltx@zero
        \endgroup
      \or % 4: \HOLOGO@PointReflectBox
        \lower.5ex\hbox{\HOLOGO@PointReflectBox{E}}%
      \else
        \@PackageError{hologo}{Internal error (choice/it}\@ehc
      \fi
    \else
      \ltx@IfUndefined{reflectbox}{%
        \ltx@IfUndefined{pgftext}{%
          \ltx@IfUndefined{psscalebox}{%
            \ltx@IfUndefined{HOLOGO@ScaleBox@\hologoDriver}{%
            }{%
              \chardef\HOLOGO@choice=4 %
            }%
          }{%
            \chardef\HOLOGO@choice=3 %
          }%
        }{%
          \chardef\HOLOGO@choice=2 %
        }%
      }{%
        \chardef\HOLOGO@choice=1 %
      }%
      \ifcase\HOLOGO@choice
        \HOLOGO@WarningUnsupportedDriver{Xe}%
        e%
      \or % 1: reflectbox
        \lower.5ex\hbox{%
          \reflectbox{E}%
        }%
      \or % 2: \pgftext
        \lower.5ex\hbox{%
          \pgfpicture
            \pgftransformxscale{-1}%
            \pgftext{E}%
          \endpgfpicture
        }%
      \or % 3: \psscalebox
        \lower.5ex\hbox{%
          \psscalebox{-1 1}{E}%
        }%
      \or % 4: \HOLOGO@Reflectbox
        \lower.5ex\hbox{%
          \HOLOGO@ReflectBox{E}%
        }%
      \else
        \@PackageError{hologo}{Internal error (choice/up)}\@ehc
      \fi
    \fi
  }%
}
%    \end{macrocode}
%    \end{macro}
%    \begin{macro}{\HoLogoHtml@Xe}
%    \begin{macrocode}
\def\HoLogoHtml@Xe#1{%
  \HoLogoCss@Xe
  \HOLOGO@Span{Xe}{%
    X%
    \HOLOGO@Span{e}{%
      \HCode{&\ltx@hashchar x018e;}%
    }%
  }%
}
%    \end{macrocode}
%    \end{macro}
%    \begin{macro}{\HoLogoCss@Xe}
%    \begin{macrocode}
\def\HoLogoCss@Xe{%
  \Css{%
    span.HoLogo-Xe span.HoLogo-e{%
      position:relative;%
      top:.5ex;%
      left-margin:-.1em;%
    }%
  }%
  \global\let\HoLogoCss@Xe\relax
}
%    \end{macrocode}
%    \end{macro}
%
%    \begin{macro}{\HoLogo@XeTeX}
%    \begin{macrocode}
\def\HoLogo@XeTeX#1{%
  \hologo{Xe}%
  \kern-.15em\relax
  \hologo{TeX}%
}
%    \end{macrocode}
%    \end{macro}
%
%    \begin{macro}{\HoLogoHtml@XeTeX}
%    \begin{macrocode}
\def\HoLogoHtml@XeTeX#1{%
  \HoLogoCss@XeTeX
  \HOLOGO@Span{XeTeX}{%
    \hologo{Xe}%
    \hologo{TeX}%
  }%
}
%    \end{macrocode}
%    \end{macro}
%    \begin{macro}{\HoLogoCss@XeTeX}
%    \begin{macrocode}
\def\HoLogoCss@XeTeX{%
  \Css{%
    span.HoLogo-XeTeX span.HoLogo-TeX{%
      margin-left:-.15em;%
    }%
  }%
  \global\let\HoLogoCss@XeTeX\relax
}
%    \end{macrocode}
%    \end{macro}
%
%    \begin{macro}{\HoLogo@XeLaTeX}
%    \begin{macrocode}
\def\HoLogo@XeLaTeX#1{%
  \hologo{Xe}%
  \kern-.13em%
  \hologo{LaTeX}%
}
%    \end{macrocode}
%    \end{macro}
%    \begin{macro}{\HoLogoHtml@XeLaTeX}
%    \begin{macrocode}
\def\HoLogoHtml@XeLaTeX#1{%
  \HoLogoCss@XeLaTeX
  \HOLOGO@Span{XeLaTeX}{%
    \hologo{Xe}%
    \hologo{LaTeX}%
  }%
}
%    \end{macrocode}
%    \end{macro}
%    \begin{macro}{\HoLogoCss@XeLaTeX}
%    \begin{macrocode}
\def\HoLogoCss@XeLaTeX{%
  \Css{%
    span.HoLogo-XeLaTeX span.HoLogo-Xe{%
      margin-right:-.13em;%
    }%
  }%
  \global\let\HoLogoCss@XeLaTeX\relax
}
%    \end{macrocode}
%    \end{macro}
%
% \subsubsection{\hologo{pdfTeX}, \hologo{pdfLaTeX}}
%
%    \begin{macro}{\HoLogo@pdfTeX}
%    \begin{macrocode}
\def\HoLogo@pdfTeX#1{%
  \HOLOGO@mbox{%
    #1{p}{P}df\hologo{TeX}%
  }%
}
%    \end{macrocode}
%    \end{macro}
%    \begin{macro}{\HoLogoCs@pdfTeX}
%    \begin{macrocode}
\def\HoLogoCs@pdfTeX#1{#1{p}{P}dfTeX}
%    \end{macrocode}
%    \end{macro}
%    \begin{macro}{\HoLogoBkm@pdfTeX}
%    \begin{macrocode}
\def\HoLogoBkm@pdfTeX#1{%
  #1{p}{P}df\hologo{TeX}%
}
%    \end{macrocode}
%    \end{macro}
%    \begin{macro}{\HoLogoHtml@pdfTeX}
%    \begin{macrocode}
\let\HoLogoHtml@pdfTeX\HoLogo@pdfTeX
%    \end{macrocode}
%    \end{macro}
%
%    \begin{macro}{\HoLogo@pdfLaTeX}
%    \begin{macrocode}
\def\HoLogo@pdfLaTeX#1{%
  \HOLOGO@mbox{%
    #1{p}{P}df\hologo{LaTeX}%
  }%
}
%    \end{macrocode}
%    \end{macro}
%    \begin{macro}{\HoLogoCs@pdfLaTeX}
%    \begin{macrocode}
\def\HoLogoCs@pdfLaTeX#1{#1{p}{P}dfLaTeX}
%    \end{macrocode}
%    \end{macro}
%    \begin{macro}{\HoLogoBkm@pdfLaTeX}
%    \begin{macrocode}
\def\HoLogoBkm@pdfLaTeX#1{%
  #1{p}{P}df\hologo{LaTeX}%
}
%    \end{macrocode}
%    \end{macro}
%    \begin{macro}{\HoLogoHtml@pdfLaTeX}
%    \begin{macrocode}
\let\HoLogoHtml@pdfLaTeX\HoLogo@pdfLaTeX
%    \end{macrocode}
%    \end{macro}
%
% \subsubsection{\hologo{VTeX}}
%
%    \begin{macro}{\HoLogo@VTeX}
%    \begin{macrocode}
\def\HoLogo@VTeX#1{%
  \HOLOGO@mbox{%
    V\hologo{TeX}%
  }%
}
%    \end{macrocode}
%    \end{macro}
%    \begin{macro}{\HoLogoHtml@VTeX}
%    \begin{macrocode}
\let\HoLogoHtml@VTeX\HoLogo@VTeX
%    \end{macrocode}
%    \end{macro}
%
% \subsubsection{\hologo{AmS}, \dots}
%
%    Source: class \xclass{amsdtx}
%
%    \begin{macro}{\HoLogo@AmS}
%    \begin{macrocode}
\def\HoLogo@AmS#1{%
  \HoLogoFont@font{AmS}{sy}{%
    A%
    \kern-.1667em%
    \lower.5ex\hbox{M}%
    \kern-.125em%
    S%
  }%
}
%    \end{macrocode}
%    \end{macro}
%    \begin{macro}{\HoLogoBkm@AmS}
%    \begin{macrocode}
\def\HoLogoBkm@AmS#1{AmS}
%    \end{macrocode}
%    \end{macro}
%    \begin{macro}{\HoLogoHtml@AmS}
%    \begin{macrocode}
\def\HoLogoHtml@AmS#1{%
  \HoLogoCss@AmS
%  \HoLogoFont@font{AmS}{sy}{%
    \HOLOGO@Span{AmS}{%
      A%
      \HOLOGO@Span{M}{M}%
      S%
    }%
%   }%
}
%    \end{macrocode}
%    \end{macro}
%    \begin{macro}{\HoLogoCss@AmS}
%    \begin{macrocode}
\def\HoLogoCss@AmS{%
  \Css{%
    span.HoLogo-AmS span.HoLogo-M{%
      position:relative;%
      top:.5ex;%
      margin-left:-.1667em;%
      margin-right:-.125em;%
      text-decoration:none;%
    }%
  }%
  \global\let\HoLogoCss@AmS\relax
}
%    \end{macrocode}
%    \end{macro}
%
%    \begin{macro}{\HoLogo@AmSTeX}
%    \begin{macrocode}
\def\HoLogo@AmSTeX#1{%
  \hologo{AmS}%
  \HOLOGO@hyphen
  \hologo{TeX}%
}
%    \end{macrocode}
%    \end{macro}
%    \begin{macro}{\HoLogoBkm@AmSTeX}
%    \begin{macrocode}
\def\HoLogoBkm@AmSTeX#1{AmS-TeX}%
%    \end{macrocode}
%    \end{macro}
%    \begin{macro}{\HoLogoHtml@AmSTeX}
%    \begin{macrocode}
\let\HoLogoHtml@AmSTeX\HoLogo@AmSTeX
%    \end{macrocode}
%    \end{macro}
%
%    \begin{macro}{\HoLogo@AmSLaTeX}
%    \begin{macrocode}
\def\HoLogo@AmSLaTeX#1{%
  \hologo{AmS}%
  \HOLOGO@hyphen
  \hologo{LaTeX}%
}
%    \end{macrocode}
%    \end{macro}
%    \begin{macro}{\HoLogoBkm@AmSLaTeX}
%    \begin{macrocode}
\def\HoLogoBkm@AmSLaTeX#1{AmS-LaTeX}%
%    \end{macrocode}
%    \end{macro}
%    \begin{macro}{\HoLogoHtml@AmSLaTeX}
%    \begin{macrocode}
\let\HoLogoHtml@AmSLaTeX\HoLogo@AmSLaTeX
%    \end{macrocode}
%    \end{macro}
%
% \subsubsection{\hologo{BibTeX}}
%
%    \begin{macro}{\HoLogo@BibTeX@sc}
%    A definition of \hologo{BibTeX} is provided in
%    the documentation source for the manual of \hologo{BibTeX}
%    \cite{btxdoc}.
%\begin{quote}
%\begin{verbatim}
%\def\BibTeX{%
%  {%
%    \rm
%    B%
%    \kern-.05em%
%    {%
%      \sc
%      i%
%      \kern-.025em %
%      b%
%    }%
%    \kern-.08em
%    T%
%    \kern-.1667em%
%    \lower.7ex\hbox{E}%
%    \kern-.125em%
%    X%
%  }%
%}
%\end{verbatim}
%\end{quote}
%    \begin{macrocode}
\def\HoLogo@BibTeX@sc#1{%
  B%
  \kern-.05em%
  \HoLogoFont@font{BibTeX}{sc}{%
    i%
    \kern-.025em%
    b%
  }%
  \HOLOGO@discretionary
  \kern-.08em%
  \hologo{TeX}%
}
%    \end{macrocode}
%    \end{macro}
%    \begin{macro}{\HoLogoHtml@BibTeX@sc}
%    \begin{macrocode}
\def\HoLogoHtml@BibTeX@sc#1{%
  \HoLogoCss@BibTeX@sc
  \HOLOGO@Span{BibTeX-sc}{%
    B%
    \HOLOGO@Span{i}{i}%
    \HOLOGO@Span{b}{b}%
    \hologo{TeX}%
  }%
}
%    \end{macrocode}
%    \end{macro}
%    \begin{macro}{\HoLogoCss@BibTeX@sc}
%    \begin{macrocode}
\def\HoLogoCss@BibTeX@sc{%
  \Css{%
    span.HoLogo-BibTeX-sc span.HoLogo-i{%
      margin-left:-.05em;%
      margin-right:-.025em;%
      font-variant:small-caps;%
    }%
  }%
  \Css{%
    span.HoLogo-BibTeX-sc span.HoLogo-b{%
      margin-right:-.08em;%
      font-variant:small-caps;%
    }%
  }%
  \global\let\HoLogoCss@BibTeX@sc\relax
}
%    \end{macrocode}
%    \end{macro}
%
%    \begin{macro}{\HoLogo@BibTeX@sf}
%    Variant \xoption{sf} avoids trouble with unavailable
%    small caps fonts (e.g., bold versions of Computer Modern or
%    Latin Modern). The definition is taken from
%    package \xpackage{dtklogos} \cite{dtklogos}.
%\begin{quote}
%\begin{verbatim}
%\DeclareRobustCommand{\BibTeX}{%
%  B%
%  \kern-.05em%
%  \hbox{%
%    $\m@th$% %% force math size calculations
%    \csname S@\f@size\endcsname
%    \fontsize\sf@size\z@
%    \math@fontsfalse
%    \selectfont
%    I%
%    \kern-.025em%
%    B
%  }%
%  \kern-.08em%
%  \-%
%  \TeX
%}
%\end{verbatim}
%\end{quote}
%    \begin{macrocode}
\def\HoLogo@BibTeX@sf#1{%
  B%
  \kern-.05em%
  \HoLogoFont@font{BibTeX}{bibsf}{%
    I%
    \kern-.025em%
    B%
  }%
  \HOLOGO@discretionary
  \kern-.08em%
  \hologo{TeX}%
}
%    \end{macrocode}
%    \end{macro}
%    \begin{macro}{\HoLogoHtml@BibTeX@sf}
%    \begin{macrocode}
\def\HoLogoHtml@BibTeX@sf#1{%
  \HoLogoCss@BibTeX@sf
  \HOLOGO@Span{BibTeX-sf}{%
    B%
    \HoLogoFont@font{BibTeX}{bibsf}{%
      \HOLOGO@Span{i}{I}%
      B%
    }%
    \hologo{TeX}%
  }%
}
%    \end{macrocode}
%    \end{macro}
%    \begin{macro}{\HoLogoCss@BibTeX@sf}
%    \begin{macrocode}
\def\HoLogoCss@BibTeX@sf{%
  \Css{%
    span.HoLogo-BibTeX-sf span.HoLogo-i{%
      margin-left:-.05em;%
      margin-right:-.025em;%
    }%
  }%
  \Css{%
    span.HoLogo-BibTeX-sf span.HoLogo-TeX{%
      margin-left:-.08em;%
    }%
  }%
  \global\let\HoLogoCss@BibTeX@sf\relax
}
%    \end{macrocode}
%    \end{macro}
%
%    \begin{macro}{\HoLogo@BibTeX}
%    \begin{macrocode}
\def\HoLogo@BibTeX{\HoLogo@BibTeX@sf}
%    \end{macrocode}
%    \end{macro}
%    \begin{macro}{\HoLogoHtml@BibTeX}
%    \begin{macrocode}
\def\HoLogoHtml@BibTeX{\HoLogoHtml@BibTeX@sf}
%    \end{macrocode}
%    \end{macro}
%
% \subsubsection{\hologo{BibTeX8}}
%
%    \begin{macro}{\HoLogo@BibTeX8}
%    \begin{macrocode}
\expandafter\def\csname HoLogo@BibTeX8\endcsname#1{%
  \hologo{BibTeX}%
  8%
}
%    \end{macrocode}
%    \end{macro}
%
%    \begin{macro}{\HoLogoBkm@BibTeX8}
%    \begin{macrocode}
\expandafter\def\csname HoLogoBkm@BibTeX8\endcsname#1{%
  \hologo{BibTeX}%
  8%
}
%    \end{macrocode}
%    \end{macro}
%    \begin{macro}{\HoLogoHtml@BibTeX8}
%    \begin{macrocode}
\expandafter
\let\csname HoLogoHtml@BibTeX8\expandafter\endcsname
\csname HoLogo@BibTeX8\endcsname
%    \end{macrocode}
%    \end{macro}
%
% \subsubsection{\hologo{ConTeXt}}
%
%    \begin{macro}{\HoLogo@ConTeXt@simple}
%    \begin{macrocode}
\def\HoLogo@ConTeXt@simple#1{%
  \HOLOGO@mbox{Con}%
  \HOLOGO@discretionary
  \HOLOGO@mbox{\hologo{TeX}t}%
}
%    \end{macrocode}
%    \end{macro}
%    \begin{macro}{\HoLogoHtml@ConTeXt@simple}
%    \begin{macrocode}
\let\HoLogoHtml@ConTeXt@simple\HoLogo@ConTeXt@simple
%    \end{macrocode}
%    \end{macro}
%
%    \begin{macro}{\HoLogo@ConTeXt@narrow}
%    This definition of logo \hologo{ConTeXt} with variant \xoption{narrow}
%    comes from TUGboat's class \xclass{ltugboat} (version 2010/11/15 v2.8).
%    \begin{macrocode}
\def\HoLogo@ConTeXt@narrow#1{%
  \HOLOGO@mbox{C\kern-.0333emon}%
  \HOLOGO@discretionary
  \kern-.0667em%
  \HOLOGO@mbox{\hologo{TeX}\kern-.0333emt}%
}
%    \end{macrocode}
%    \end{macro}
%    \begin{macro}{\HoLogoHtml@ConTeXt@narrow}
%    \begin{macrocode}
\def\HoLogoHtml@ConTeXt@narrow#1{%
  \HoLogoCss@ConTeXt@narrow
  \HOLOGO@Span{ConTeXt-narrow}{%
    \HOLOGO@Span{C}{C}%
    on%
    \hologo{TeX}%
    t%
  }%
}
%    \end{macrocode}
%    \end{macro}
%    \begin{macro}{\HoLogoCss@ConTeXt@narrow}
%    \begin{macrocode}
\def\HoLogoCss@ConTeXt@narrow{%
  \Css{%
    span.HoLogo-ConTeXt-narrow span.HoLogo-C{%
      margin-left:-.0333em;%
    }%
  }%
  \Css{%
    span.HoLogo-ConTeXt-narrow span.HoLogo-TeX{%
      margin-left:-.0667em;%
      margin-right:-.0333em;%
    }%
  }%
  \global\let\HoLogoCss@ConTeXt@narrow\relax
}
%    \end{macrocode}
%    \end{macro}
%
%    \begin{macro}{\HoLogo@ConTeXt}
%    \begin{macrocode}
\def\HoLogo@ConTeXt{\HoLogo@ConTeXt@narrow}
%    \end{macrocode}
%    \end{macro}
%    \begin{macro}{\HoLogoHtml@ConTeXt}
%    \begin{macrocode}
\def\HoLogoHtml@ConTeXt{\HoLogoHtml@ConTeXt@narrow}
%    \end{macrocode}
%    \end{macro}
%
% \subsubsection{\hologo{emTeX}}
%
%    \begin{macro}{\HoLogo@emTeX}
%    \begin{macrocode}
\def\HoLogo@emTeX#1{%
  \HOLOGO@mbox{#1{e}{E}m}%
  \HOLOGO@discretionary
  \hologo{TeX}%
}
%    \end{macrocode}
%    \end{macro}
%    \begin{macro}{\HoLogoCs@emTeX}
%    \begin{macrocode}
\def\HoLogoCs@emTeX#1{#1{e}{E}mTeX}%
%    \end{macrocode}
%    \end{macro}
%    \begin{macro}{\HoLogoBkm@emTeX}
%    \begin{macrocode}
\def\HoLogoBkm@emTeX#1{%
  #1{e}{E}m\hologo{TeX}%
}
%    \end{macrocode}
%    \end{macro}
%    \begin{macro}{\HoLogoHtml@emTeX}
%    \begin{macrocode}
\let\HoLogoHtml@emTeX\HoLogo@emTeX
%    \end{macrocode}
%    \end{macro}
%
% \subsubsection{\hologo{ExTeX}}
%
%    \begin{macro}{\HoLogo@ExTeX}
%    The definition is taken from the FAQ of the
%    project \hologo{ExTeX}
%    \cite{ExTeX-FAQ}.
%\begin{quote}
%\begin{verbatim}
%\def\ExTeX{%
%  \textrm{% Logo always with serifs
%    \ensuremath{%
%      \textstyle
%      \varepsilon_{%
%        \kern-0.15em%
%        \mathcal{X}%
%      }%
%    }%
%    \kern-.15em%
%    \TeX
%  }%
%}
%\end{verbatim}
%\end{quote}
%    \begin{macrocode}
\def\HoLogo@ExTeX#1{%
  \HoLogoFont@font{ExTeX}{rm}{%
    \ltx@mbox{%
      \HOLOGO@MathSetup
      $%
        \textstyle
        \varepsilon_{%
          \kern-0.15em%
          \HoLogoFont@font{ExTeX}{sy}{X}%
        }%
      $%
    }%
    \HOLOGO@discretionary
    \kern-.15em%
    \hologo{TeX}%
  }%
}
%    \end{macrocode}
%    \end{macro}
%    \begin{macro}{\HoLogoHtml@ExTeX}
%    \begin{macrocode}
\def\HoLogoHtml@ExTeX#1{%
  \HoLogoCss@ExTeX
  \HoLogoFont@font{ExTeX}{rm}{%
    \HOLOGO@Span{ExTeX}{%
      \ltx@mbox{%
        \HOLOGO@MathSetup
        $\textstyle\varepsilon$%
        \HOLOGO@Span{X}{$\textstyle\chi$}%
        \hologo{TeX}%
      }%
    }%
  }%
}
%    \end{macrocode}
%    \end{macro}
%    \begin{macro}{\HoLogoBkm@ExTeX}
%    \begin{macrocode}
\def\HoLogoBkm@ExTeX#1{%
  \HOLOGO@PdfdocUnicode{#1{e}{E}x}{\textepsilon\textchi}%
  \hologo{TeX}%
}
%    \end{macrocode}
%    \end{macro}
%    \begin{macro}{\HoLogoCss@ExTeX}
%    \begin{macrocode}
\def\HoLogoCss@ExTeX{%
  \Css{%
    span.HoLogo-ExTeX{%
      font-family:serif;%
    }%
  }%
  \Css{%
    span.HoLogo-ExTeX span.HoLogo-TeX{%
      margin-left:-.15em;%
    }%
  }%
  \global\let\HoLogoCss@ExTeX\relax
}
%    \end{macrocode}
%    \end{macro}
%
% \subsubsection{\hologo{MiKTeX}}
%
%    \begin{macro}{\HoLogo@MiKTeX}
%    \begin{macrocode}
\def\HoLogo@MiKTeX#1{%
  \HOLOGO@mbox{MiK}%
  \HOLOGO@discretionary
  \hologo{TeX}%
}
%    \end{macrocode}
%    \end{macro}
%    \begin{macro}{\HoLogoHtml@MiKTeX}
%    \begin{macrocode}
\let\HoLogoHtml@MiKTeX\HoLogo@MiKTeX
%    \end{macrocode}
%    \end{macro}
%
% \subsubsection{\hologo{OzTeX} and friends}
%
%    Source: \hologo{OzTeX} FAQ \cite{OzTeX}:
%    \begin{quote}
%      |\def\OzTeX{O\kern-.03em z\kern-.15em\TeX}|\\
%      (There is no kerning in OzMF, OzMP and OzTtH.)
%    \end{quote}
%
%    \begin{macro}{\HoLogo@OzTeX}
%    \begin{macrocode}
\def\HoLogo@OzTeX#1{%
  O%
  \kern-.03em %
  z%
  \kern-.15em %
  \hologo{TeX}%
}
%    \end{macrocode}
%    \end{macro}
%    \begin{macro}{\HoLogoHtml@OzTeX}
%    \begin{macrocode}
\def\HoLogoHtml@OzTeX#1{%
  \HoLogoCss@OzTeX
  \HOLOGO@Span{OzTeX}{%
    O%
    \HOLOGO@Span{z}{z}%
    \hologo{TeX}%
  }%
}
%    \end{macrocode}
%    \end{macro}
%    \begin{macro}{\HoLogoCss@OzTeX}
%    \begin{macrocode}
\def\HoLogoCss@OzTeX{%
  \Css{%
    span.HoLogo-OzTeX span.HoLogo-z{%
      margin-left:-.03em;%
      margin-right:-.15em;%
    }%
  }%
  \global\let\HoLogoCss@OzTeX\relax
}
%    \end{macrocode}
%    \end{macro}
%
%    \begin{macro}{\HoLogo@OzMF}
%    \begin{macrocode}
\def\HoLogo@OzMF#1{%
  \HOLOGO@mbox{OzMF}%
}
%    \end{macrocode}
%    \end{macro}
%    \begin{macro}{\HoLogo@OzMP}
%    \begin{macrocode}
\def\HoLogo@OzMP#1{%
  \HOLOGO@mbox{OzMP}%
}
%    \end{macrocode}
%    \end{macro}
%    \begin{macro}{\HoLogo@OzTtH}
%    \begin{macrocode}
\def\HoLogo@OzTtH#1{%
  \HOLOGO@mbox{OzTtH}%
}
%    \end{macrocode}
%    \end{macro}
%
% \subsubsection{\hologo{PCTeX}}
%
%    \begin{macro}{\HoLogo@PCTeX}
%    \begin{macrocode}
\def\HoLogo@PCTeX#1{%
  \HOLOGO@mbox{PC}%
  \hologo{TeX}%
}
%    \end{macrocode}
%    \end{macro}
%    \begin{macro}{\HoLogoHtml@PCTeX}
%    \begin{macrocode}
\let\HoLogoHtml@PCTeX\HoLogo@PCTeX
%    \end{macrocode}
%    \end{macro}
%
% \subsubsection{\hologo{PiCTeX}}
%
%    The original definitions from \xfile{pictex.tex} \cite{PiCTeX}:
%\begin{quote}
%\begin{verbatim}
%\def\PiC{%
%  P%
%  \kern-.12em%
%  \lower.5ex\hbox{I}%
%  \kern-.075em%
%  C%
%}
%\def\PiCTeX{%
%  \PiC
%  \kern-.11em%
%  \TeX
%}
%\end{verbatim}
%\end{quote}
%
%    \begin{macro}{\HoLogo@PiC}
%    \begin{macrocode}
\def\HoLogo@PiC#1{%
  P%
  \kern-.12em%
  \lower.5ex\hbox{I}%
  \kern-.075em%
  C%
  \HOLOGO@SpaceFactor
}
%    \end{macrocode}
%    \end{macro}
%    \begin{macro}{\HoLogoHtml@PiC}
%    \begin{macrocode}
\def\HoLogoHtml@PiC#1{%
  \HoLogoCss@PiC
  \HOLOGO@Span{PiC}{%
    P%
    \HOLOGO@Span{i}{I}%
    C%
  }%
}
%    \end{macrocode}
%    \end{macro}
%    \begin{macro}{\HoLogoCss@PiC}
%    \begin{macrocode}
\def\HoLogoCss@PiC{%
  \Css{%
    span.HoLogo-PiC span.HoLogo-i{%
      position:relative;%
      top:.5ex;%
      margin-left:-.12em;%
      margin-right:-.075em;%
      text-decoration:none;%
    }%
  }%
  \global\let\HoLogoCss@PiC\relax
}
%    \end{macrocode}
%    \end{macro}
%
%    \begin{macro}{\HoLogo@PiCTeX}
%    \begin{macrocode}
\def\HoLogo@PiCTeX#1{%
  \hologo{PiC}%
  \HOLOGO@discretionary
  \kern-.11em%
  \hologo{TeX}%
}
%    \end{macrocode}
%    \end{macro}
%    \begin{macro}{\HoLogoHtml@PiCTeX}
%    \begin{macrocode}
\def\HoLogoHtml@PiCTeX#1{%
  \HoLogoCss@PiCTeX
  \HOLOGO@Span{PiCTeX}{%
    \hologo{PiC}%
    \hologo{TeX}%
  }%
}
%    \end{macrocode}
%    \end{macro}
%    \begin{macro}{\HoLogoCss@PiCTeX}
%    \begin{macrocode}
\def\HoLogoCss@PiCTeX{%
  \Css{%
    span.HoLogo-PiCTeX span.HoLogo-PiC{%
      margin-right:-.11em;%
    }%
  }%
  \global\let\HoLogoCss@PiCTeX\relax
}
%    \end{macrocode}
%    \end{macro}
%
% \subsubsection{\hologo{teTeX}}
%
%    \begin{macro}{\HoLogo@teTeX}
%    \begin{macrocode}
\def\HoLogo@teTeX#1{%
  \HOLOGO@mbox{#1{t}{T}e}%
  \HOLOGO@discretionary
  \hologo{TeX}%
}
%    \end{macrocode}
%    \end{macro}
%    \begin{macro}{\HoLogoCs@teTeX}
%    \begin{macrocode}
\def\HoLogoCs@teTeX#1{#1{t}{T}dfTeX}
%    \end{macrocode}
%    \end{macro}
%    \begin{macro}{\HoLogoBkm@teTeX}
%    \begin{macrocode}
\def\HoLogoBkm@teTeX#1{%
  #1{t}{T}e\hologo{TeX}%
}
%    \end{macrocode}
%    \end{macro}
%    \begin{macro}{\HoLogoHtml@teTeX}
%    \begin{macrocode}
\let\HoLogoHtml@teTeX\HoLogo@teTeX
%    \end{macrocode}
%    \end{macro}
%
% \subsubsection{\hologo{TeX4ht}}
%
%    \begin{macro}{\HoLogo@TeX4ht}
%    \begin{macrocode}
\expandafter\def\csname HoLogo@TeX4ht\endcsname#1{%
  \HOLOGO@mbox{\hologo{TeX}4ht}%
}
%    \end{macrocode}
%    \end{macro}
%    \begin{macro}{\HoLogoHtml@TeX4ht}
%    \begin{macrocode}
\expandafter
\let\csname HoLogoHtml@TeX4ht\expandafter\endcsname
\csname HoLogo@TeX4ht\endcsname
%    \end{macrocode}
%    \end{macro}
%
%
% \subsubsection{\hologo{SageTeX}}
%
%    \begin{macro}{\HoLogo@SageTeX}
%    \begin{macrocode}
\def\HoLogo@SageTeX#1{%
  \HOLOGO@mbox{Sage}%
  \HOLOGO@discretionary
  \HOLOGO@NegativeKerning{eT,oT,To}%
  \hologo{TeX}%
}
%    \end{macrocode}
%    \end{macro}
%    \begin{macro}{\HoLogoHtml@SageTeX}
%    \begin{macrocode}
\let\HoLogoHtml@SageTeX\HoLogo@SageTeX
%    \end{macrocode}
%    \end{macro}
%
% \subsection{\hologo{METAFONT} and friends}
%
%    \begin{macro}{\HoLogo@METAFONT}
%    \begin{macrocode}
\def\HoLogo@METAFONT#1{%
  \HoLogoFont@font{METAFONT}{logo}{%
    \HOLOGO@mbox{META}%
    \HOLOGO@discretionary
    \HOLOGO@mbox{FONT}%
  }%
}
%    \end{macrocode}
%    \end{macro}
%
%    \begin{macro}{\HoLogo@METAPOST}
%    \begin{macrocode}
\def\HoLogo@METAPOST#1{%
  \HoLogoFont@font{METAPOST}{logo}{%
    \HOLOGO@mbox{META}%
    \HOLOGO@discretionary
    \HOLOGO@mbox{POST}%
  }%
}
%    \end{macrocode}
%    \end{macro}
%
%    \begin{macro}{\HoLogo@MetaFun}
%    \begin{macrocode}
\def\HoLogo@MetaFun#1{%
  \HOLOGO@mbox{Meta}%
  \HOLOGO@discretionary
  \HOLOGO@mbox{Fun}%
}
%    \end{macrocode}
%    \end{macro}
%
%    \begin{macro}{\HoLogo@MetaPost}
%    \begin{macrocode}
\def\HoLogo@MetaPost#1{%
  \HOLOGO@mbox{Meta}%
  \HOLOGO@discretionary
  \HOLOGO@mbox{Post}%
}
%    \end{macrocode}
%    \end{macro}
%
% \subsection{Others}
%
% \subsubsection{\hologo{biber}}
%
%    \begin{macro}{\HoLogo@biber}
%    \begin{macrocode}
\def\HoLogo@biber#1{%
  \HOLOGO@mbox{#1{b}{B}i}%
  \HOLOGO@discretionary
  \HOLOGO@mbox{ber}%
}
%    \end{macrocode}
%    \end{macro}
%    \begin{macro}{\HoLogoCs@biber}
%    \begin{macrocode}
\def\HoLogoCs@biber#1{#1{b}{B}iber}
%    \end{macrocode}
%    \end{macro}
%    \begin{macro}{\HoLogoBkm@biber}
%    \begin{macrocode}
\def\HoLogoBkm@biber#1{%
  #1{b}{B}iber%
}
%    \end{macrocode}
%    \end{macro}
%    \begin{macro}{\HoLogoHtml@biber}
%    \begin{macrocode}
\let\HoLogoHtml@biber\HoLogo@biber
%    \end{macrocode}
%    \end{macro}
%
% \subsubsection{\hologo{KOMAScript}}
%
%    \begin{macro}{\HoLogo@KOMAScript}
%    The definition for \hologo{KOMAScript} is taken
%    from \hologo{KOMAScript} (\xfile{scrlogo.dtx}, reformatted) \cite{scrlogo}:
%\begin{quote}
%\begin{verbatim}
%\@ifundefined{KOMAScript}{%
%  \DeclareRobustCommand{\KOMAScript}{%
%    \textsf{%
%      K\kern.05em O\kern.05emM\kern.05em A%
%      \kern.1em-\kern.1em %
%      Script%
%    }%
%  }%
%}{}
%\end{verbatim}
%\end{quote}
%    \begin{macrocode}
\def\HoLogo@KOMAScript#1{%
  \HoLogoFont@font{KOMAScript}{sf}{%
    \HOLOGO@mbox{%
      K\kern.05em%
      O\kern.05em%
      M\kern.05em%
      A%
    }%
    \kern.1em%
    \HOLOGO@hyphen
    \kern.1em%
    \HOLOGO@mbox{Script}%
  }%
}
%    \end{macrocode}
%    \end{macro}
%    \begin{macro}{\HoLogoBkm@KOMAScript}
%    \begin{macrocode}
\def\HoLogoBkm@KOMAScript#1{%
  KOMA-Script%
}
%    \end{macrocode}
%    \end{macro}
%    \begin{macro}{\HoLogoHtml@KOMAScript}
%    \begin{macrocode}
\def\HoLogoHtml@KOMAScript#1{%
  \HoLogoCss@KOMAScript
  \HoLogoFont@font{KOMAScript}{sf}{%
    \HOLOGO@Span{KOMAScript}{%
      K%
      \HOLOGO@Span{O}{O}%
      M%
      \HOLOGO@Span{A}{A}%
      \HOLOGO@Span{hyphen}{-}%
      Script%
    }%
  }%
}
%    \end{macrocode}
%    \end{macro}
%    \begin{macro}{\HoLogoCss@KOMAScript}
%    \begin{macrocode}
\def\HoLogoCss@KOMAScript{%
  \Css{%
    span.HoLogo-KOMAScript{%
      font-family:sans-serif;%
    }%
  }%
  \Css{%
    span.HoLogo-KOMAScript span.HoLogo-O{%
      padding-left:.05em;%
      padding-right:.05em;%
    }%
  }%
  \Css{%
    span.HoLogo-KOMAScript span.HoLogo-A{%
      padding-left:.05em;%
    }%
  }%
  \Css{%
    span.HoLogo-KOMAScript span.HoLogo-hyphen{%
      padding-left:.1em;%
      padding-right:.1em;%
    }%
  }%
  \global\let\HoLogoCss@KOMAScript\relax
}
%    \end{macrocode}
%    \end{macro}
%
% \subsubsection{\hologo{LyX}}
%
%    \begin{macro}{\HoLogo@LyX}
%    The definition is taken from the documentation source files
%    of \hologo{LyX}, \xfile{Intro.lyx} \cite{LyX}:
%\begin{quote}
%\begin{verbatim}
%\def\LyX{%
%  \texorpdfstring{%
%    L\kern-.1667em\lower.25em\hbox{Y}\kern-.125emX\@%
%  }{%
%    LyX%
%  }%
%}
%\end{verbatim}
%\end{quote}
%    \begin{macrocode}
\def\HoLogo@LyX#1{%
  L%
  \kern-.1667em%
  \lower.25em\hbox{Y}%
  \kern-.125em%
  X%
  \HOLOGO@SpaceFactor
}
%    \end{macrocode}
%    \end{macro}
%    \begin{macro}{\HoLogoHtml@LyX}
%    \begin{macrocode}
\def\HoLogoHtml@LyX#1{%
  \HoLogoCss@LyX
  \HOLOGO@Span{LyX}{%
    L%
    \HOLOGO@Span{y}{Y}%
    X%
  }%
}
%    \end{macrocode}
%    \end{macro}
%    \begin{macro}{\HoLogoCss@LyX}
%    \begin{macrocode}
\def\HoLogoCss@LyX{%
  \Css{%
    span.HoLogo-LyX span.HoLogo-y{%
      position:relative;%
      top:.25em;%
      margin-left:-.1667em;%
      margin-right:-.125em;%
      text-decoration:none;%
    }%
  }%
  \global\let\HoLogoCss@LyX\relax
}
%    \end{macrocode}
%    \end{macro}
%
% \subsubsection{\hologo{NTS}}
%
%    \begin{macro}{\HoLogo@NTS}
%    Definition for \hologo{NTS} can be found in
%    package \xpackage{etex\textunderscore man} for the \hologo{eTeX} manual \cite{etexman}
%    and in package \xpackage{dtklogos} \cite{dtklogos}:
%\begin{quote}
%\begin{verbatim}
%\def\NTS{%
%  \leavevmode
%  \hbox{%
%    $%
%      \cal N%
%      \kern-0.35em%
%      \lower0.5ex\hbox{$\cal T$}%
%      \kern-0.2em%
%      S%
%    $%
%  }%
%}
%\end{verbatim}
%\end{quote}
%    \begin{macrocode}
\def\HoLogo@NTS#1{%
  \HoLogoFont@font{NTS}{sy}{%
    N\/%
    \kern-.35em%
    \lower.5ex\hbox{T\/}%
    \kern-.2em%
    S\/%
  }%
  \HOLOGO@SpaceFactor
}
%    \end{macrocode}
%    \end{macro}
%
% \subsubsection{\Hologo{TTH} (\hologo{TeX} to HTML translator)}
%
%    Source: \url{http://hutchinson.belmont.ma.us/tth/}
%    In the HTML source the second `T' is printed as subscript.
%\begin{quote}
%\begin{verbatim}
%T<sub>T</sub>H
%\end{verbatim}
%\end{quote}
%    \begin{macro}{\HoLogo@TTH}
%    \begin{macrocode}
\def\HoLogo@TTH#1{%
  \ltx@mbox{%
    T\HOLOGO@SubScript{T}H%
  }%
  \HOLOGO@SpaceFactor
}
%    \end{macrocode}
%    \end{macro}
%
%    \begin{macro}{\HoLogoHtml@TTH}
%    \begin{macrocode}
\def\HoLogoHtml@TTH#1{%
  T\HCode{<sub>}T\HCode{</sub>}H%
}
%    \end{macrocode}
%    \end{macro}
%
% \subsubsection{\Hologo{HanTheThanh}}
%
%    Partial source: Package \xpackage{dtklogos}.
%    The double accent is U+1EBF (latin small letter e with circumflex
%    and acute).
%    \begin{macro}{\HoLogo@HanTheThanh}
%    \begin{macrocode}
\def\HoLogo@HanTheThanh#1{%
  \ltx@mbox{H\`an}%
  \HOLOGO@space
  \ltx@mbox{%
    Th%
    \HOLOGO@IfCharExists{"1EBF}{%
      \char"1EBF\relax
    }{%
      \^e\hbox to 0pt{\hss\raise .5ex\hbox{\'{}}}%
    }%
  }%
  \HOLOGO@space
  \ltx@mbox{Th\`anh}%
}
%    \end{macrocode}
%    \end{macro}
%    \begin{macro}{\HoLogoBkm@HanTheThanh}
%    \begin{macrocode}
\def\HoLogoBkm@HanTheThanh#1{%
  H\`an %
  Th\HOLOGO@PdfdocUnicode{\^e}{\9036\277} %
  Th\`anh%
}
%    \end{macrocode}
%    \end{macro}
%    \begin{macro}{\HoLogoHtml@HanTheThanh}
%    \begin{macrocode}
\def\HoLogoHtml@HanTheThanh#1{%
  H\`an %
  Th\HCode{&\ltx@hashchar x1ebf;} %
  Th\`anh%
}
%    \end{macrocode}
%    \end{macro}
%
% \subsection{Driver detection}
%
%    \begin{macrocode}
\HOLOGO@IfExists\InputIfFileExists{%
  \InputIfFileExists{hologo.cfg}{}{}%
}{%
  \ltx@IfUndefined{pdf@filesize}{%
    \def\HOLOGO@InputIfExists{%
      \openin\HOLOGO@temp=hologo.cfg\relax
      \ifeof\HOLOGO@temp
        \closein\HOLOGO@temp
      \else
        \closein\HOLOGO@temp
        \begingroup
          \def\x{LaTeX2e}%
        \expandafter\endgroup
        \ifx\fmtname\x
          \input{hologo.cfg}%
        \else
          \input hologo.cfg\relax
        \fi
      \fi
    }%
    \ltx@IfUndefined{newread}{%
      \chardef\HOLOGO@temp=15 %
      \def\HOLOGO@CheckRead{%
        \ifeof\HOLOGO@temp
          \HOLOGO@InputIfExists
        \else
          \ifcase\HOLOGO@temp
            \@PackageWarningNoLine{hologo}{%
              Configuration file ignored, because\MessageBreak
              a free read register could not be found%
            }%
          \else
            \begingroup
              \count\ltx@cclv=\HOLOGO@temp
              \advance\ltx@cclv by \ltx@minusone
              \edef\x{\endgroup
                \chardef\noexpand\HOLOGO@temp=\the\count\ltx@cclv
                \relax
              }%
            \x
          \fi
        \fi
      }%
    }{%
      \csname newread\endcsname\HOLOGO@temp
      \HOLOGO@InputIfExists
    }%
  }{%
    \edef\HOLOGO@temp{\pdf@filesize{hologo.cfg}}%
    \ifx\HOLOGO@temp\ltx@empty
    \else
      \ifnum\HOLOGO@temp>0 %
        \begingroup
          \def\x{LaTeX2e}%
        \expandafter\endgroup
        \ifx\fmtname\x
          \input{hologo.cfg}%
        \else
          \input hologo.cfg\relax
        \fi
      \else
        \@PackageInfoNoLine{hologo}{%
          Empty configuration file `hologo.cfg' ignored%
        }%
      \fi
    \fi
  }%
}
%    \end{macrocode}
%
%    \begin{macrocode}
\def\HOLOGO@temp#1#2{%
  \kv@define@key{HoLogoDriver}{#1}[]{%
    \begingroup
      \def\HOLOGO@temp{##1}%
      \ltx@onelevel@sanitize\HOLOGO@temp
      \ifx\HOLOGO@temp\ltx@empty
      \else
        \@PackageError{hologo}{%
          Value (\HOLOGO@temp) not permitted for option `#1'%
        }%
        \@ehc
      \fi
    \endgroup
    \def\hologoDriver{#2}%
  }%
}%
\def\HOLOGO@@temp#1#2{%
  \ifx\kv@value\relax
    \HOLOGO@temp{#1}{#1}%
  \else
    \HOLOGO@temp{#1}{#2}%
  \fi
}%
\kv@parse@normalized{%
  pdftex,%
  luatex=pdftex,%
  dvipdfm,%
  dvipdfmx=dvipdfm,%
  dvips,%
  dvipsone=dvips,%
  xdvi=dvips,%
  xetex,%
  vtex,%
}\HOLOGO@@temp
%    \end{macrocode}
%
%    \begin{macrocode}
\kv@define@key{HoLogoDriver}{driverfallback}{%
  \def\HOLOGO@DriverFallback{#1}%
}
%    \end{macrocode}
%
%    \begin{macro}{\HOLOGO@DriverFallback}
%    \begin{macrocode}
\def\HOLOGO@DriverFallback{dvips}
%    \end{macrocode}
%    \end{macro}
%
%    \begin{macro}{\hologoDriverSetup}
%    \begin{macrocode}
\def\hologoDriverSetup{%
  \let\hologoDriver\ltx@undefined
  \HOLOGO@DriverSetup
}
%    \end{macrocode}
%    \end{macro}
%
%    \begin{macro}{\HOLOGO@DriverSetup}
%    \begin{macrocode}
\def\HOLOGO@DriverSetup#1{%
  \kvsetkeys{HoLogoDriver}{#1}%
  \HOLOGO@CheckDriver
  \ltx@ifundefined{hologoDriver}{%
    \begingroup
    \edef\x{\endgroup
      \noexpand\kvsetkeys{HoLogoDriver}{\HOLOGO@DriverFallback}%
    }\x
  }{}%
  \@PackageInfoNoLine{hologo}{Using driver `\hologoDriver'}%
}
%    \end{macrocode}
%    \end{macro}
%
%    \begin{macro}{\HOLOGO@CheckDriver}
%    \begin{macrocode}
\def\HOLOGO@CheckDriver{%
  \ifpdf
    \def\hologoDriver{pdftex}%
    \let\HOLOGO@pdfliteral\pdfliteral
    \ifluatex
      \ifx\pdfextension\@undefined\else
        \protected\def\pdfliteral{\pdfextension literal}%
        \let\HOLOGO@pdfliteral\pdfliteral
      \fi
      \ltx@IfUndefined{HOLOGO@pdfliteral}{%
        \ifnum\luatexversion<36 %
        \else
          \begingroup
            \let\HOLOGO@temp\endgroup
            \ifcase0%
                \directlua{%
                  if tex.enableprimitives then %
                    tex.enableprimitives('HOLOGO@', {'pdfliteral'})%
                  else %
                    tex.print('1')%
                  end%
                }%
                \ifx\HOLOGO@pdfliteral\@undefined 1\fi%
                \relax%
              \endgroup
              \let\HOLOGO@temp\relax
              \global\let\HOLOGO@pdfliteral\HOLOGO@pdfliteral
            \fi%
          \HOLOGO@temp
        \fi
      }{}%
    \fi
    \ltx@IfUndefined{HOLOGO@pdfliteral}{%
      \@PackageWarningNoLine{hologo}{%
        Cannot find \string\pdfliteral
      }%
    }{}%
  \else
    \ifxetex
      \def\hologoDriver{xetex}%
    \else
      \ifvtex
        \def\hologoDriver{vtex}%
      \fi
    \fi
  \fi
}
%    \end{macrocode}
%    \end{macro}
%
%    \begin{macro}{\HOLOGO@WarningUnsupportedDriver}
%    \begin{macrocode}
\def\HOLOGO@WarningUnsupportedDriver#1{%
  \@PackageWarningNoLine{hologo}{%
    Logo `#1' needs driver specific macros,\MessageBreak
    but driver `\hologoDriver' is not supported.\MessageBreak
    Use a different driver or\MessageBreak
    load package `graphics' or `pgf'%
  }%
}
%    \end{macrocode}
%    \end{macro}
%
% \subsubsection{Reflect box macros}
%
%    Skip driver part if not needed.
%    \begin{macrocode}
\ltx@IfUndefined{reflectbox}{}{%
  \ltx@IfUndefined{rotatebox}{}{%
    \HOLOGO@AtEnd
  }%
}
\ltx@IfUndefined{pgftext}{}{%
  \HOLOGO@AtEnd
}
\ltx@IfUndefined{psscalebox}{}{%
  \HOLOGO@AtEnd
}
%    \end{macrocode}
%
%    \begin{macrocode}
\def\HOLOGO@temp{LaTeX2e}
\ifx\fmtname\HOLOGO@temp
  \RequirePackage{kvoptions}[2011/06/30]%
  \ProcessKeyvalOptions{HoLogoDriver}%
\fi
\HOLOGO@DriverSetup{}
%    \end{macrocode}
%
%    \begin{macro}{\HOLOGO@ReflectBox}
%    \begin{macrocode}
\def\HOLOGO@ReflectBox#1{%
  \begingroup
    \setbox\ltx@zero\hbox{\begingroup#1\endgroup}%
    \setbox\ltx@two\hbox{%
      \kern\wd\ltx@zero
      \csname HOLOGO@ScaleBox@\hologoDriver\endcsname{-1}{1}{%
        \hbox to 0pt{\copy\ltx@zero\hss}%
      }%
    }%
    \wd\ltx@two=\wd\ltx@zero
    \box\ltx@two
  \endgroup
}
%    \end{macrocode}
%    \end{macro}
%
%    \begin{macro}{\HOLOGO@PointReflectBox}
%    \begin{macrocode}
\def\HOLOGO@PointReflectBox#1{%
  \begingroup
    \setbox\ltx@zero\hbox{\begingroup#1\endgroup}%
    \setbox\ltx@two\hbox{%
      \kern\wd\ltx@zero
      \raise\ht\ltx@zero\hbox{%
        \csname HOLOGO@ScaleBox@\hologoDriver\endcsname{-1}{-1}{%
          \hbox to 0pt{\copy\ltx@zero\hss}%
        }%
      }%
    }%
    \wd\ltx@two=\wd\ltx@zero
    \box\ltx@two
  \endgroup
}
%    \end{macrocode}
%    \end{macro}
%
%    We must define all variants because of dynamic driver setup.
%    \begin{macrocode}
\def\HOLOGO@temp#1#2{#2}
%    \end{macrocode}
%
%    \begin{macro}{\HOLOGO@ScaleBox@pdftex}
%    \begin{macrocode}
\HOLOGO@temp{pdftex}{%
  \def\HOLOGO@ScaleBox@pdftex#1#2#3{%
    \HOLOGO@pdfliteral{%
      q #1 0 0 #2 0 0 cm%
    }%
    #3%
    \HOLOGO@pdfliteral{%
      Q%
    }%
  }%
}
%    \end{macrocode}
%    \end{macro}
%    \begin{macro}{\HOLOGO@ScaleBox@dvips}
%    \begin{macrocode}
\HOLOGO@temp{dvips}{%
  \def\HOLOGO@ScaleBox@dvips#1#2#3{%
    \special{ps:%
      gsave %
      currentpoint %
      currentpoint translate %
      #1 #2 scale %
      neg exch neg exch translate%
    }%
    #3%
    \special{ps:%
      currentpoint %
      grestore %
      moveto%
    }%
  }%
}
%    \end{macrocode}
%    \end{macro}
%    \begin{macro}{\HOLOGO@ScaleBox@dvipdfm}
%    \begin{macrocode}
\HOLOGO@temp{dvipdfm}{%
  \let\HOLOGO@ScaleBox@dvipdfm\HOLOGO@ScaleBox@dvips
}
%    \end{macrocode}
%    \end{macro}
%    Since \hologo{XeTeX} v0.6.
%    \begin{macro}{\HOLOGO@ScaleBox@xetex}
%    \begin{macrocode}
\HOLOGO@temp{xetex}{%
  \def\HOLOGO@ScaleBox@xetex#1#2#3{%
    \special{x:gsave}%
    \special{x:scale #1 #2}%
    #3%
    \special{x:grestore}%
  }%
}
%    \end{macrocode}
%    \end{macro}
%    \begin{macro}{\HOLOGO@ScaleBox@vtex}
%    \begin{macrocode}
\HOLOGO@temp{vtex}{%
  \def\HOLOGO@ScaleBox@vtex#1#2#3{%
    \special{r(#1,0,0,#2,0,0}%
    #3%
    \special{r)}%
  }%
}
%    \end{macrocode}
%    \end{macro}
%
%    \begin{macrocode}
\HOLOGO@AtEnd%
%</package>
%    \end{macrocode}
%
% \section{Test}
%
% \subsection{Catcode checks for loading}
%
%    \begin{macrocode}
%<*test1>
%    \end{macrocode}
%    \begin{macrocode}
\catcode`\{=1 %
\catcode`\}=2 %
\catcode`\#=6 %
\catcode`\@=11 %
\expandafter\ifx\csname count@\endcsname\relax
  \countdef\count@=255 %
\fi
\expandafter\ifx\csname @gobble\endcsname\relax
  \long\def\@gobble#1{}%
\fi
\expandafter\ifx\csname @firstofone\endcsname\relax
  \long\def\@firstofone#1{#1}%
\fi
\expandafter\ifx\csname loop\endcsname\relax
  \expandafter\@firstofone
\else
  \expandafter\@gobble
\fi
{%
  \def\loop#1\repeat{%
    \def\body{#1}%
    \iterate
  }%
  \def\iterate{%
    \body
      \let\next\iterate
    \else
      \let\next\relax
    \fi
    \next
  }%
  \let\repeat=\fi
}%
\def\RestoreCatcodes{}
\count@=0 %
\loop
  \edef\RestoreCatcodes{%
    \RestoreCatcodes
    \catcode\the\count@=\the\catcode\count@\relax
  }%
\ifnum\count@<255 %
  \advance\count@ 1 %
\repeat

\def\RangeCatcodeInvalid#1#2{%
  \count@=#1\relax
  \loop
    \catcode\count@=15 %
  \ifnum\count@<#2\relax
    \advance\count@ 1 %
  \repeat
}
\def\RangeCatcodeCheck#1#2#3{%
  \count@=#1\relax
  \loop
    \ifnum#3=\catcode\count@
    \else
      \errmessage{%
        Character \the\count@\space
        with wrong catcode \the\catcode\count@\space
        instead of \number#3%
      }%
    \fi
  \ifnum\count@<#2\relax
    \advance\count@ 1 %
  \repeat
}
\def\space{ }
\expandafter\ifx\csname LoadCommand\endcsname\relax
  \def\LoadCommand{\input hologo.sty\relax}%
\fi
\def\Test{%
  \RangeCatcodeInvalid{0}{47}%
  \RangeCatcodeInvalid{58}{64}%
  \RangeCatcodeInvalid{91}{96}%
  \RangeCatcodeInvalid{123}{255}%
  \catcode`\@=12 %
  \catcode`\\=0 %
  \catcode`\%=14 %
  \LoadCommand
  \RangeCatcodeCheck{0}{36}{15}%
  \RangeCatcodeCheck{37}{37}{14}%
  \RangeCatcodeCheck{38}{47}{15}%
  \RangeCatcodeCheck{48}{57}{12}%
  \RangeCatcodeCheck{58}{63}{15}%
  \RangeCatcodeCheck{64}{64}{12}%
  \RangeCatcodeCheck{65}{90}{11}%
  \RangeCatcodeCheck{91}{91}{15}%
  \RangeCatcodeCheck{92}{92}{0}%
  \RangeCatcodeCheck{93}{96}{15}%
  \RangeCatcodeCheck{97}{122}{11}%
  \RangeCatcodeCheck{123}{255}{15}%
  \RestoreCatcodes
}
\Test
\csname @@end\endcsname
\end
%    \end{macrocode}
%    \begin{macrocode}
%</test1>
%    \end{macrocode}
%
% \subsection{Spacefactor}
%
%    The space factor must be 1000 after a logo. If it is greater 1000
%    then the following space is a space after a sentence closing point.
%    If the space factor is smaller 1000 then an immediate following
%    dot is interpreted as abbreviation, not sentence closing point.
%
%    \begin{macrocode}
%<*test-spacefactor>
\NeedsTeXFormat{LaTeX2e}
\documentclass{article}
\usepackage{hologo}[2016/05/12]
\usepackage{kvsetkeys}
\usepackage{qstest}
\IncludeTests{*}
\LogTests{log}{*}{*}
\begin{document}
\begin{qstest}{spacefactor}{spacefactor}
\newcommand*{\Test}[1]{%
  \sbox0{%
    \hologo{#1}%
    \Expect*{1000 (#1)}*{\the\spacefactor\space(#1)}%
  }%
}%
\makeatletter
\def\TestList{}
\def\hologoEntry#1#2#3{%
  \edef\TestList{%
    \ifx\TestList\@empty
    \else
      \TestList,%
    \fi
    #1%
    \ifx\\#2\\%
    \else
      ={variant=#2}%
    \fi
  }%
}
\hologoList
\expandafter\kv@parse@normalized\expandafter{%
  \TestList
}{%
  \begingroup
    \let\@logo=\kv@key
    \ifx\kv@value\relax
    \else
      \expandafter\hologoLogoSetup\expandafter\@logo\expandafter{%
        \kv@value
      }%
    \fi
    \Test\@logo
  \endgroup
  \@gobbletwo
}
\end{qstest}
\end{document}
%</test-spacefactor>
%    \end{macrocode}
%
% \subsection{Complete list}
%
%    \begin{macrocode}
%<*test-list>
\NeedsTeXFormat{LaTeX2e}
\documentclass[12pt,a4paper]{article}
\usepackage{hologo}[2016/05/12]
\usepackage[T1]{fontenc}
\usepackage{lmodern}
\usepackage{parskip}
\usepackage[unicode]{hyperref}[2011/09/28]
\usepackage{bookmark}[2011/09/19]
\bookmarksetup{%
  numbered,%
  open,%
  openlevel=2,%
}
\renewcommand*{\contentsname}{List of logos}
\begin{document}
\tableofcontents
\def\TestFont#1#2#3#4#5#6{%
  \begingroup
    \usefont{#3}{#4}{#5}{#6}%
    \HologoVariant{#1}{#2}/\hologoVariant{#1}{#2}%
    \quad
    \begingroup\scriptsize\hologoVariant{#1}{#2}\endgroup
    \quad
  \endgroup
  (#3/#4/#5/#6)%
  \par
}
\makeatletter
\def\hologoEntry#1#2#3{%
  \section{%
    \HologoVariant{#1}{#2}/\hologoVariant{#1}{#2} %
    {[#1\ifx\\#2\\\else\space(#2)\fi]}% hash-ok
  }% braces around [] because of bug in tex4ht
  \begingroup
    \hypersetup{unicode=false}%
    \bookmark[%
      dest=\@currentHref,%
      rellevel=1,%
      keeplevel,%
    ]{%
      \HologoVariant{#1}{#2}/\hologoVariant{#1}{#2} %
      (PDFDocEncoding)%
    }%
  \endgroup
  \TestFont{#1}{#2}{OT1}{cmr}{m}{n}%
  \TestFont{#1}{#2}{OT1}{cmss}{m}{n}%
  \TestFont{#1}{#2}{OT1}{cmr}{b}{n}%
  \TestFont{#1}{#2}{OT1}{cmr}{m}{it}%
  \TestFont{#1}{#2}{OT1}{cmtt}{m}{n}%
  \TestFont{#1}{#2}{T1}{lmr}{m}{n}%
  \TestFont{#1}{#2}{T1}{lmss}{m}{n}%
  \TestFont{#1}{#2}{T1}{lmr}{b}{n}%
  \TestFont{#1}{#2}{T1}{lmr}{m}{it}%
  \TestFont{#1}{#2}{T1}{lmtt}{m}{n}%
  \TestFont{#1}{#2}{T1}{lmvtt}{m}{n}%
  \TestFont{#1}{#2}{T1}{qtm}{m}{n}%
  \TestFont{#1}{#2}{T1}{qhv}{m}{n}%
  \TestFont{#1}{#2}{T1}{qtm}{b}{n}%
  \TestFont{#1}{#2}{T1}{qtm}{m}{it}%
  \TestFont{#1}{#2}{T1}{qcr}{m}{n}%
  \newpage
}
\makeatother
\hologoList
\end{document}
%</test-list>
%    \end{macrocode}
%
% \section{Installation}
%
% \subsection{Download}
%
% \paragraph{Package.} This package is available on
% CTAN\footnote{\url{ftp://ftp.ctan.org/tex-archive/}}:
% \begin{description}
% \item[\CTAN{macros/latex/contrib/oberdiek/hologo.dtx}] The source file.
% \item[\CTAN{macros/latex/contrib/oberdiek/hologo.pdf}] Documentation.
% \end{description}
%
%
% \paragraph{Bundle.} All the packages of the bundle `oberdiek'
% are also available in a TDS compliant ZIP archive. There
% the packages are already unpacked and the documentation files
% are generated. The files and directories obey the TDS standard.
% \begin{description}
% \item[\CTAN{install/macros/latex/contrib/oberdiek.tds.zip}]
% \end{description}
% \emph{TDS} refers to the standard ``A Directory Structure
% for \TeX\ Files'' (\CTAN{tds/tds.pdf}). Directories
% with \xfile{texmf} in their name are usually organized this way.
%
% \subsection{Bundle installation}
%
% \paragraph{Unpacking.} Unpack the \xfile{oberdiek.tds.zip} in the
% TDS tree (also known as \xfile{texmf} tree) of your choice.
% Example (linux):
% \begin{quote}
%   |unzip oberdiek.tds.zip -d ~/texmf|
% \end{quote}
%
% \paragraph{Script installation.}
% Check the directory \xfile{TDS:scripts/oberdiek/} for
% scripts that need further installation steps.
% Package \xpackage{attachfile2} comes with the Perl script
% \xfile{pdfatfi.pl} that should be installed in such a way
% that it can be called as \texttt{pdfatfi}.
% Example (linux):
% \begin{quote}
%   |chmod +x scripts/oberdiek/pdfatfi.pl|\\
%   |cp scripts/oberdiek/pdfatfi.pl /usr/local/bin/|
% \end{quote}
%
% \subsection{Package installation}
%
% \paragraph{Unpacking.} The \xfile{.dtx} file is a self-extracting
% \docstrip\ archive. The files are extracted by running the
% \xfile{.dtx} through \plainTeX:
% \begin{quote}
%   \verb|tex hologo.dtx|
% \end{quote}
%
% \paragraph{TDS.} Now the different files must be moved into
% the different directories in your installation TDS tree
% (also known as \xfile{texmf} tree):
% \begin{quote}
% \def\t{^^A
% \begin{tabular}{@{}>{\ttfamily}l@{ $\rightarrow$ }>{\ttfamily}l@{}}
%   hologo.sty & tex/generic/oberdiek/hologo.sty\\
%   hologo.pdf & doc/latex/oberdiek/hologo.pdf\\
%   example/hologo-example.tex & doc/latex/oberdiek/example/hologo-example.tex\\
%   test/hologo-test1.tex & doc/latex/oberdiek/test/hologo-test1.tex\\
%   test/hologo-test-spacefactor.tex & doc/latex/oberdiek/test/hologo-test-spacefactor.tex\\
%   test/hologo-test-list.tex & doc/latex/oberdiek/test/hologo-test-list.tex\\
%   hologo.dtx & source/latex/oberdiek/hologo.dtx\\
% \end{tabular}^^A
% }^^A
% \sbox0{\t}^^A
% \ifdim\wd0>\linewidth
%   \begingroup
%     \advance\linewidth by\leftmargin
%     \advance\linewidth by\rightmargin
%   \edef\x{\endgroup
%     \def\noexpand\lw{\the\linewidth}^^A
%   }\x
%   \def\lwbox{^^A
%     \leavevmode
%     \hbox to \linewidth{^^A
%       \kern-\leftmargin\relax
%       \hss
%       \usebox0
%       \hss
%       \kern-\rightmargin\relax
%     }^^A
%   }^^A
%   \ifdim\wd0>\lw
%     \sbox0{\small\t}^^A
%     \ifdim\wd0>\linewidth
%       \ifdim\wd0>\lw
%         \sbox0{\footnotesize\t}^^A
%         \ifdim\wd0>\linewidth
%           \ifdim\wd0>\lw
%             \sbox0{\scriptsize\t}^^A
%             \ifdim\wd0>\linewidth
%               \ifdim\wd0>\lw
%                 \sbox0{\tiny\t}^^A
%                 \ifdim\wd0>\linewidth
%                   \lwbox
%                 \else
%                   \usebox0
%                 \fi
%               \else
%                 \lwbox
%               \fi
%             \else
%               \usebox0
%             \fi
%           \else
%             \lwbox
%           \fi
%         \else
%           \usebox0
%         \fi
%       \else
%         \lwbox
%       \fi
%     \else
%       \usebox0
%     \fi
%   \else
%     \lwbox
%   \fi
% \else
%   \usebox0
% \fi
% \end{quote}
% If you have a \xfile{docstrip.cfg} that configures and enables \docstrip's
% TDS installing feature, then some files can already be in the right
% place, see the documentation of \docstrip.
%
% \subsection{Refresh file name databases}
%
% If your \TeX~distribution
% (\teTeX, \mikTeX, \dots) relies on file name databases, you must refresh
% these. For example, \teTeX\ users run \verb|texhash| or
% \verb|mktexlsr|.
%
% \subsection{Some details for the interested}
%
% \paragraph{Attached source.}
%
% The PDF documentation on CTAN also includes the
% \xfile{.dtx} source file. It can be extracted by
% AcrobatReader 6 or higher. Another option is \textsf{pdftk},
% e.g. unpack the file into the current directory:
% \begin{quote}
%   \verb|pdftk hologo.pdf unpack_files output .|
% \end{quote}
%
% \paragraph{Unpacking with \LaTeX.}
% The \xfile{.dtx} chooses its action depending on the format:
% \begin{description}
% \item[\plainTeX:] Run \docstrip\ and extract the files.
% \item[\LaTeX:] Generate the documentation.
% \end{description}
% If you insist on using \LaTeX\ for \docstrip\ (really,
% \docstrip\ does not need \LaTeX), then inform the autodetect routine
% about your intention:
% \begin{quote}
%   \verb|latex \let\install=y\input{hologo.dtx}|
% \end{quote}
% Do not forget to quote the argument according to the demands
% of your shell.
%
% \paragraph{Generating the documentation.}
% You can use both the \xfile{.dtx} or the \xfile{.drv} to generate
% the documentation. The process can be configured by the
% configuration file \xfile{ltxdoc.cfg}. For instance, put this
% line into this file, if you want to have A4 as paper format:
% \begin{quote}
%   \verb|\PassOptionsToClass{a4paper}{article}|
% \end{quote}
% An example follows how to generate the
% documentation with pdf\LaTeX:
% \begin{quote}
%\begin{verbatim}
%pdflatex hologo.dtx
%makeindex -s gind.ist hologo.idx
%pdflatex hologo.dtx
%makeindex -s gind.ist hologo.idx
%pdflatex hologo.dtx
%\end{verbatim}
% \end{quote}
%
% \section{Catalogue}
%
% The following XML file can be used as source for the
% \href{http://mirror.ctan.org/help/Catalogue/catalogue.html}{\TeX\ Catalogue}.
% The elements \texttt{caption} and \texttt{description} are imported
% from the original XML file from the Catalogue.
% The name of the XML file in the Catalogue is \xfile{hologo.xml}.
%    \begin{macrocode}
%<*catalogue>
<?xml version='1.0' encoding='us-ascii'?>
<!DOCTYPE entry SYSTEM 'catalogue.dtd'>
<entry datestamp='$Date$' modifier='$Author$' id='hologo'>
  <name>hologo</name>
  <caption>A collection of logos with bookmark support.</caption>
  <authorref id='auth:oberdiek'/>
  <copyright owner='Heiko Oberdiek' year='2010-2012'/>
  <license type='lppl1.3'/>
  <version number='1.10'/>
  <description>
    The package defines a single command <tt>\hologo</tt>, whose
    argument is the usual case-confused ASCII version of the logo.
    The command is bookmark-enabled, so that every logo becomes
    available in bookmarks without further work.
    <p/>
    The package is part of the <xref refid='oberdiek'>oberdiek</xref>
    bundle.
  </description>
  <documentation details='Package documentation'
      href='ctan:/macros/latex/contrib/oberdiek/hologo.pdf'/>
  <ctan file='true' path='/macros/latex/contrib/oberdiek/hologo.dtx'/>
  <miktex location='oberdiek'/>
  <texlive location='oberdiek'/>
  <install path='/macros/latex/contrib/oberdiek/oberdiek.tds.zip'/>
</entry>
%</catalogue>
%    \end{macrocode}
%
% \begin{thebibliography}{9}
% \raggedright
%
% \bibitem{btxdoc}
% Oren Patashnik,
% \textit{\hologo{BibTeX}ing},
% 1988-02-08.\\
% \CTAN{biblio/bibtex/base/}
%
% \bibitem{dtklogos}
% Gerd Neugebauer, DANTE,
% \textit{Package \xpackage{dtklogos}},
% 2011-04-25.\\
% \CTAN{usergrps/dante/dtk/dtklogos.sty}
%
% \bibitem{etexman}
% The \hologo{NTS} Team,
% \textit{The \hologo{eTeX} manual},
% 1998-02.\\
% \CTAN{systems/e-tex/v2/doc/}
%
% \bibitem{ExTeX-FAQ}
% The \hologo{ExTeX} group,
% \textit{\hologo{ExTeX}: FAQ -- How is \hologo{ExTeX} typeset?},
% 2007-04-14.\\
% \url{http://www.extex.org/documentation/faq.html}
%
% \bibitem{LyX}
% %@MISC{ LyX,
% %  title = {{LyX 2.0.0 -- The Document Processor [Computer software and manual]}},
% %  author = {{The LyX Team}},
% %  howpublished = {Internet: http://www.lyx.org},
% %  year = {2011-05-08},
% %  note = {Retrieved May 10, 2011, from http://www.lyx.org},
% %  url = {http://www.lyx.org/}
% %}
% The \hologo{LyX} Team,
% \textit{\hologo{LyX} -- The Document Processor},
% 2011-05-08.\\
% \url{http://www.lyx.org/}
%
% \bibitem{OzTeX}
% Andrew Trevorrow,
% \hologo{OzTeX} FAQ: What is the correct way to typeset ``\hologo{OzTeX}''?,
% 2011-09-15 (visited).
% \url{http://www.trevorrow.com/oztex/ozfaq.html#oztex-logo}
%
% \bibitem{PiCTeX}
% Michael Wichura,
% \textit{The \hologo{PiCTeX} macro package},
% 1987-09-21.
% \CTAN{graphics/pictex/}
%
% \bibitem{scrlogo}
% Markus Kohm,
% \textit{\hologo{KOMAScript} Datei \xfile{scrlogo.dtx}},
% 2009-01-30.\\
% \CTAN{install/macros/latex/contrib/komascript.tds.zip}
%
% \end{thebibliography}
%
% \begin{History}
%   \begin{Version}{2010/04/08 v1.0}
%   \item
%     The first version.
%   \end{Version}
%   \begin{Version}{2010/04/16 v1.1}
%   \item
%     \cs{Hologo} added for support of logos at start of a sentence.
%   \item
%     \cs{hologoSetup} and \cs{hologoLogoSetup} added.
%   \item
%     Options \xoption{break}, \xoption{hyphenbreak}, \xoption{spacebreak}
%     added.
%   \item
%     Variant support added by option \xoption{variant}.
%   \end{Version}
%   \begin{Version}{2010/04/24 v1.2}
%   \item
%     \hologo{LaTeX3} added.
%   \item
%     \hologo{VTeX} added.
%   \end{Version}
%   \begin{Version}{2010/11/21 v1.3}
%   \item
%     \hologo{iniTeX}, \hologo{virTeX} added.
%   \end{Version}
%   \begin{Version}{2011/03/25 v1.4}
%   \item
%     \hologo{ConTeXt} with variants added.
%   \item
%     Option \xoption{discretionarybreak} added as refinement for
%     option \xoption{break}.
%   \end{Version}
%   \begin{Version}{2011/04/21 v1.5}
%   \item
%     Wrong TDS directory for test files fixed.
%   \end{Version}
%   \begin{Version}{2011/10/01 v1.6}
%   \item
%     Support for package \xpackage{tex4ht} added.
%   \item
%     Support for \cs{csname} added if \cs{ifincsname} is available.
%   \item
%     New logos:
%     \hologo{(La)TeX},
%     \hologo{biber},
%     \hologo{BibTeX} (\xoption{sc}, \xoption{sf}),
%     \hologo{emTeX},
%     \hologo{ExTeX},
%     \hologo{KOMAScript},
%     \hologo{La},
%     \hologo{LyX},
%     \hologo{MiKTeX},
%     \hologo{NTS},
%     \hologo{OzMF},
%     \hologo{OzMP},
%     \hologo{OzTeX},
%     \hologo{OzTtH},
%     \hologo{PCTeX},
%     \hologo{PiC},
%     \hologo{PiCTeX},
%     \hologo{METAFONT},
%     \hologo{MetaFun},
%     \hologo{METAPOST},
%     \hologo{MetaPost},
%     \hologo{SLiTeX} (\xoption{lift}, \xoption{narrow}, \xoption{simple}),
%     \hologo{SliTeX} (\xoption{narrow}, \xoption{simple}, \xoption{lift}),
%     \hologo{teTeX}.
%   \item
%     Fixes:
%     \hologo{iniTeX},
%     \hologo{pdfLaTeX},
%     \hologo{pdfTeX},
%     \hologo{virTeX}.
%   \item
%     \cs{hologoFontSetup} and \cs{hologoLogoFontSetup} added.
%   \item
%     \cs{hologoVariant} and \cs{HologoVariant} added.
%   \end{Version}
%   \begin{Version}{2011/11/22 v1.7}
%   \item
%     New logos:
%     \hologo{BibTeX8},
%     \hologo{LaTeXML},
%     \hologo{SageTeX},
%     \hologo{TeX4ht},
%     \hologo{TTH}.
%   \item
%     \hologo{Xe} and friends: Driver stuff fixed.
%   \item
%     \hologo{Xe} and friends: Support for italic added.
%   \item
%     \hologo{Xe} and friends: Package support for \xpackage{pgf}
%     and \xpackage{pstricks} added.
%   \end{Version}
%   \begin{Version}{2011/11/29 v1.8}
%   \item
%     New logos:
%     \hologo{HanTheThanh}.
%   \end{Version}
%   \begin{Version}{2011/12/21 v1.9}
%   \item
%     Patch for package \xpackage{ifxetex} added for the case that
%     \cs{newif} is undefined in \hologo{iniTeX}.
%   \item
%     Some fixes for \hologo{iniTeX}.
%   \end{Version}
%   \begin{Version}{2012/04/26 v1.10}
%   \item
%     Fix in bookmark version of logo ``\hologo{HanTheThanh}''.
%   \end{Version}
%   \begin{Version}{2016/05/12 v1.11}
%   \item
%     Update HOLOGO@IfCharExists (previously in texlive)
%   \item define pdfliteral in current luatex.
%   \end{Version}
% \end{History}
%
% \PrintIndex
%
% \Finale
\endinput

%        (quote the arguments according to the demands of your shell)
%
% Documentation:
%    (a) If hologo.drv is present:
%           latex hologo.drv
%    (b) Without hologo.drv:
%           latex hologo.dtx; ...
%    The class ltxdoc loads the configuration file ltxdoc.cfg
%    if available. Here you can specify further options, e.g.
%    use A4 as paper format:
%       \PassOptionsToClass{a4paper}{article}
%
%    Programm calls to get the documentation (example):
%       pdflatex hologo.dtx
%       makeindex -s gind.ist hologo.idx
%       pdflatex hologo.dtx
%       makeindex -s gind.ist hologo.idx
%       pdflatex hologo.dtx
%
% Installation:
%    TDS:tex/generic/oberdiek/hologo.sty
%    TDS:doc/latex/oberdiek/hologo.pdf
%    TDS:doc/latex/oberdiek/example/hologo-example.tex
%    TDS:doc/latex/oberdiek/test/hologo-test1.tex
%    TDS:doc/latex/oberdiek/test/hologo-test-spacefactor.tex
%    TDS:doc/latex/oberdiek/test/hologo-test-list.tex
%    TDS:source/latex/oberdiek/hologo.dtx
%
%<*ignore>
\begingroup
  \catcode123=1 %
  \catcode125=2 %
  \def\x{LaTeX2e}%
\expandafter\endgroup
\ifcase 0\ifx\install y1\fi\expandafter
         \ifx\csname processbatchFile\endcsname\relax\else1\fi
         \ifx\fmtname\x\else 1\fi\relax
\else\csname fi\endcsname
%</ignore>
%<*install>
\input docstrip.tex
\Msg{************************************************************************}
\Msg{* Installation}
\Msg{* Package: hologo 2016/05/12 v1.11 A logo collection with bookmark support (HO)}
\Msg{************************************************************************}

\keepsilent
\askforoverwritefalse

\let\MetaPrefix\relax
\preamble

This is a generated file.

Project: hologo
Version: 2016/05/12 v1.11

Copyright (C) 2010-2012 by
   Heiko Oberdiek <heiko.oberdiek at googlemail.com>

This work may be distributed and/or modified under the
conditions of the LaTeX Project Public License, either
version 1.3c of this license or (at your option) any later
version. This version of this license is in
   http://www.latex-project.org/lppl/lppl-1-3c.txt
and the latest version of this license is in
   http://www.latex-project.org/lppl.txt
and version 1.3 or later is part of all distributions of
LaTeX version 2005/12/01 or later.

This work has the LPPL maintenance status "maintained".

This Current Maintainer of this work is Heiko Oberdiek.

The Base Interpreter refers to any `TeX-Format',
because some files are installed in TDS:tex/generic//.

This work consists of the main source file hologo.dtx
and the derived files
   hologo.sty, hologo.pdf, hologo.ins, hologo.drv, hologo-example.tex,
   hologo-test1.tex, hologo-test-spacefactor.tex,
   hologo-test-list.tex.

\endpreamble
\let\MetaPrefix\DoubleperCent

\generate{%
  \file{hologo.ins}{\from{hologo.dtx}{install}}%
  \file{hologo.drv}{\from{hologo.dtx}{driver}}%
  \usedir{tex/generic/oberdiek}%
  \file{hologo.sty}{\from{hologo.dtx}{package}}%
  \usedir{doc/latex/oberdiek/example}%
  \file{hologo-example.tex}{\from{hologo.dtx}{example}}%
  \usedir{doc/latex/oberdiek/test}%
  \file{hologo-test1.tex}{\from{hologo.dtx}{test1}}%
  \file{hologo-test-spacefactor.tex}{\from{hologo.dtx}{test-spacefactor}}%
  \file{hologo-test-list.tex}{\from{hologo.dtx}{test-list}}%
  \nopreamble
  \nopostamble
  \usedir{source/latex/oberdiek/catalogue}%
  \file{hologo.xml}{\from{hologo.dtx}{catalogue}}%
}

\catcode32=13\relax% active space
\let =\space%
\Msg{************************************************************************}
\Msg{*}
\Msg{* To finish the installation you have to move the following}
\Msg{* file into a directory searched by TeX:}
\Msg{*}
\Msg{*     hologo.sty}
\Msg{*}
\Msg{* To produce the documentation run the file `hologo.drv'}
\Msg{* through LaTeX.}
\Msg{*}
\Msg{* Happy TeXing!}
\Msg{*}
\Msg{************************************************************************}

\endbatchfile
%</install>
%<*ignore>
\fi
%</ignore>
%<*driver>
\NeedsTeXFormat{LaTeX2e}
\ProvidesFile{hologo.drv}%
  [2016/05/12 v1.11 A logo collection with bookmark support (HO)]%
\documentclass{ltxdoc}
\usepackage{holtxdoc}[2011/11/22]
\usepackage{hologo}[2016/05/12]
\usepackage{longtable}
\usepackage{array}
\usepackage{paralist}
%\usepackage[T1]{fontenc}
%\usepackage{lmodern}
\begin{document}
  \DocInput{hologo.dtx}%
\end{document}
%</driver>
% \fi
%
%
% \CharacterTable
%  {Upper-case    \A\B\C\D\E\F\G\H\I\J\K\L\M\N\O\P\Q\R\S\T\U\V\W\X\Y\Z
%   Lower-case    \a\b\c\d\e\f\g\h\i\j\k\l\m\n\o\p\q\r\s\t\u\v\w\x\y\z
%   Digits        \0\1\2\3\4\5\6\7\8\9
%   Exclamation   \!     Double quote  \"     Hash (number) \#
%   Dollar        \$     Percent       \%     Ampersand     \&
%   Acute accent  \'     Left paren    \(     Right paren   \)
%   Asterisk      \*     Plus          \+     Comma         \,
%   Minus         \-     Point         \.     Solidus       \/
%   Colon         \:     Semicolon     \;     Less than     \<
%   Equals        \=     Greater than  \>     Question mark \?
%   Commercial at \@     Left bracket  \[     Backslash     \\
%   Right bracket \]     Circumflex    \^     Underscore    \_
%   Grave accent  \`     Left brace    \{     Vertical bar  \|
%   Right brace   \}     Tilde         \~}
%
% \GetFileInfo{hologo.drv}
%
% \title{The \xpackage{hologo} package}
% \date{2016/05/12 v1.11}
% \author{Heiko Oberdiek\\\xemail{heiko.oberdiek at googlemail.com}}
%
% \maketitle
%
% \begin{abstract}
% This package starts a collection of logos with support for bookmarks
% strings.
% \end{abstract}
%
% \tableofcontents
%
% \section{Documentation}
%
% \subsection{Logo macros}
%
% \begin{declcs}{hologo} \M{name}
% \end{declcs}
% Macro \cs{hologo} sets the logo with name \meta{name}.
% The following table shows the supported names.
%
% \begingroup
%   \def\hologoEntry#1#2#3{^^A
%     #1&#2&\hologoLogoSetup{#1}{variant=#2}\hologo{#1}&#3\tabularnewline
%   }
%   \begin{longtable}{>{\ttfamily}l>{\ttfamily}lll}
%     \rmfamily\bfseries{name} & \rmfamily\bfseries variant
%     & \bfseries logo & \bfseries since\\
%     \hline
%     \endhead
%     \hologoList
%   \end{longtable}
% \endgroup
%
% \begin{declcs}{Hologo} \M{name}
% \end{declcs}
% Macro \cs{Hologo} starts the logo \meta{name} with an uppercase
% letter. As an exception small greek letters are not converted
% to uppercase. Examples, see \hologo{eTeX} and \hologo{ExTeX}.
%
% \subsection{Setup macros}
%
% The package does not support package options, but the following
% setup macros can be used to set options.
%
% \begin{declcs}{hologoSetup} \M{key value list}
% \end{declcs}
% Macro \cs{hologoSetup} sets global options.
%
% \begin{declcs}{hologoLogoSetup} \M{logo} \M{key value list}
% \end{declcs}
% Some options can also be used to configure a logo.
% These settings take precedence over global option settings.
%
% \subsection{Options}\label{sec:options}
%
% There are boolean and string options:
% \begin{description}
% \item[Boolean option:]
% It takes |true| or |false|
% as value. If the value is omitted, then |true| is used.
% \item[String option:]
% A value must be given as string. (But the string might be empty.)
% \end{description}
% The following options can be used both in \cs{hologoSetup}
% and \cs{hologoLogoSetup}:
% \begin{description}
% \def\entry#1{\item[\xoption{#1}:]}
% \entry{break}
%   enables or disables line breaks inside the logo. This setting is
%   refined by options \xoption{hyphenbreak}, \xoption{spacebreak}
%   or \xoption{discretionarybreak}.
%   Default is |false|.
% \entry{hyphenbreak}
%   enables or disables the line break right after the hyphen character.
% \entry{spacebreak}
%   enables or disables line breaks at space characters.
% \entry{discretionarybreak}
%   enables or disables line breaks at hyphenation points
%   (inserted by \cs{-}).
% \end{description}
% Macro \cs{hologoLogoSetup} also knows:
% \begin{description}
% \item[\xoption{variant}:]
%   This is a string option. It specifies a variant of a logo that
%   must exist. An empty string selects the package default variant.
% \end{description}
% Example:
% \begin{quote}
%   |\hologoSetup{break=false}|\\
%   |\hologoLogoSetup{plainTeX}{variant=hyphen,hyphenbreak}|\\
%   Then ``plain-\TeX'' contains one break point after the hyphen.
% \end{quote}
%
% \subsection{Driver options}
%
% Sometimes graphical operations are needed to construct some
% glyphs (e.g.\ \hologo{XeTeX}). If package \xpackage{graphics}
% or package \xpackage{pgf} are found, then the macros are taken
% from there. Otherwise the packge defines its own operations
% and therefore needs the driver information. Many drivers are
% detected automatically (\hologo{pdfTeX}/\hologo{LuaTeX}
% in PDF mode, \hologo{XeTeX}, \hologo{VTeX}). These have precedence
% over a driver option. The driver can be given as package option
% or using \cs{hologoDriverSetup}.
% The following list contains the recognized driver options:
% \begin{itemize}
% \item \xoption{pdftex}, \xoption{luatex}
% \item \xoption{dvipdfm}, \xoption{dvipdfmx}
% \item \xoption{dvips}, \xoption{dvipsone}, \xoption{xdvi}
% \item \xoption{xetex}
% \item \xoption{vtex}
% \end{itemize}
% The left driver of a line is the driver name that is used internally.
% The following names are aliases for drivers that use the
% same method. Therefore the entry in the \xext{log} file for
% the used driver prints the internally used driver name.
% \begin{description}
% \item[\xoption{driverfallback}:]
%   This option expects a driver that is used,
%   if the driver could not be detected automatically.
% \end{description}
%
% \begin{declcs}{hologoDriverSetup} \M{driver option}
% \end{declcs}
% The driver can also be configured after package loading
% using \cs{hologoDriverSetup}, also the way for \hologo{plainTeX}
% to setup the driver.
%
% \subsection{Font setup}
%
% Some logos require a special font, but should also be usable by
% \hologo{plainTeX}. Therefore the package provides some ways
% to influence the font settings. The options below
% take font settings as values. Both font commands
% such as \cs{sffamily} and macros that take one argument
% like \cs{textsf} can be used.
%
% \begin{declcs}{hologoFontSetup} \M{key value list}
% \end{declcs}
% Macro \cs{hologoFontSetup} sets the fonts for all logos.
% Supported keys:
% \begin{description}
% \def\entry#1{\item[\xoption{#1}:]}
% \entry{general}
%   This font is used for all logos. The default is empty.
%   That means no special font is used.
% \entry{bibsf}
%   This font is used for
%   {\hologoLogoSetup{BibTeX}{variant=sf}\hologo{BibTeX}}
%   with variant \xoption{sf}.
% \entry{rm}
%   This font is a serif font. It is used for \hologo{ExTeX}.
% \entry{sc}
%   This font specifies a small caps font. It is used for
%   {\hologoLogoSetup{BibTeX}{variant=sc}\hologo{BibTeX}}
%   with variant \xoption{sc}.
% \entry{sf}
%   This font specifies a sans serif font. The default
%   is \cs{sffamily}, then \cs{sf} is tried. Otherwise
%   a warning is given. It is used by \hologo{KOMAScript}.
% \entry{sy}
%   This is the font for math symbols (e.g. cmsy).
%   It is used by \hologo{AmS}, \hologo{NTS}, \hologo{ExTeX}.
% \entry{logo}
%   \hologo{METAFONT} and \hologo{METAPOST} are using that font.
%   In \hologo{LaTeX} \cs{logofamily} is used and
%   the definitions of package \xpackage{mflogo} are used
%   if the package is not loaded.
%   Otherwise the \cs{tenlogo} is used and defined
%   if it does not already exists.
% \end{description}
%
% \begin{declcs}{hologoLogoFontSetup} \M{logo} \M{key value list}
% \end{declcs}
% Fonts can also be set for a logo or logo component separately,
% see the following list.
% The keys are the same as for \cs{hologoFontSetup}.
%
% \begin{longtable}{>{\ttfamily}l>{\sffamily}ll}
%   \meta{logo} & keys & result\\
%   \hline
%   \endhead
%   BibTeX & bibsf & {\hologoLogoSetup{BibTeX}{variant=sf}\hologo{BibTeX}}\\[.5ex]
%   BibTeX & sc & {\hologoLogoSetup{BibTeX}{variant=sc}\hologo{BibTeX}}\\[.5ex]
%   ExTeX & rm & \hologo{ExTeX}\\
%   SliTeX & rm & \hologo{SliTeX}\\[.5ex]
%   AmS & sy & \hologo{AmS}\\
%   ExTeX & sy & \hologo{ExTeX}\\
%   NTS & sy & \hologo{NTS}\\[.5ex]
%   KOMAScript & sf & \hologo{KOMAScript}\\[.5ex]
%   METAFONT & logo & \hologo{METAFONT}\\
%   METAPOST & logo & \hologo{METAPOST}\\[.5ex]
%   SliTeX & sc \hologo{SliTeX}
% \end{longtable}
%
% \subsubsection{Font order}
%
% For all logos the font \xoption{general} is applied first.
% Example:
%\begin{quote}
%|\hologoFontSetup{general=\color{red}}|
%\end{quote}
% will print red logos.
% Then if the font uses a special font \xoption{sf}, for example,
% the font is applied that is setup by \cs{hologoLogoFontSetup}.
% If this font is not setup, then the common font setup
% by \cs{hologoFontSetup} is used. Otherwise a warning is given,
% that there is no font configured.
%
% \subsection{Additional user macros}
%
% Usually a variant of a logo is configured by using
% \cs{hologoLogoSetup}, because it is bad style to mix
% different variants of the same logo in the same text.
% There the following macros are a convenience for testing.
%
% \begin{declcs}{hologoVariant} \M{name} \M{variant}\\
%   \cs{HologoVariant} \M{name} \M{variant}
% \end{declcs}
% Logo \meta{name} is set using \meta{variant} that specifies
% explicitely which variant of the macro is used. If the argument
% is empty, then the default form of the logo is used
% (configurable by \cs{hologoLogoSetup}).
%
% \cs{HologoVariant} is used if the logo is set in a context
% that needs an uppercase first letter (beginning of a sentence, \dots).
%
% \begin{declcs}{hologoList}\\
%   \cs{hologoEntry} \M{logo} \M{variant} \M{since}
% \end{declcs}
% Macro \cs{hologoList} contains all logos that are provided
% by the package including variants. The list consists of calls
% of \cs{hologoEntry} with three arguments starting with the
% logo name \meta{logo} and its variant \meta{variant}. An empty
% variant means the current default. Argument \meta{since} specifies
% with version of the package \xpackage{hologo} is needed to get
% the logo. If the logo is fixed, then the date gets updated.
% Therefore the date \meta{since} is not exactly the date of
% the first introduction, but rather the date of the latest fix.
%
% Before \cs{hologoList} can be used, macro \cs{hologoEntry} needs
% a definition. The example file in section \ref{sec:example}
% shows applications of \cs{hologoList}.
%
% \subsection{Supported contexts}
%
% Macros \cs{hologo} and friends support special contexts:
% \begin{itemize}
% \item \hologo{LaTeX}'s protection mechanism.
% \item Bookmarks of package \xpackage{hyperref}.
% \item Package \xpackage{tex4ht}.
% \item The macros can be used inside \cs{csname} constructs,
%   if \cs{ifincsname} is available (\hologo{pdfTeX}, \hologo{XeTeX},
%   \hologo{LuaTeX}).
% \end{itemize}
%
% \subsection{Example}
% \label{sec:example}
%
% The following example prints the logos in different fonts.
%    \begin{macrocode}
%<*example>
%<<verbatim
\NeedsTeXFormat{LaTeX2e}
\documentclass[a4paper]{article}
\usepackage[
  hmargin=20mm,
  vmargin=20mm,
]{geometry}
\pagestyle{empty}
\usepackage{hologo}[2016/05/12]
\usepackage{longtable}
\usepackage{array}
\setlength{\extrarowheight}{2pt}
\usepackage[T1]{fontenc}
\usepackage{lmodern}
\usepackage{pdflscape}
\usepackage[
  pdfencoding=auto,
]{hyperref}
\hypersetup{
  pdfauthor={Heiko Oberdiek},
  pdftitle={Example for package `hologo'},
  pdfsubject={Logos with fonts lmr, lmss, qtm, qpl, qhv},
}
\usepackage{bookmark}

% Print the logo list on the console

\begingroup
  \typeout{}%
  \typeout{*** Begin of logo list ***}%
  \newcommand*{\hologoEntry}[3]{%
    \typeout{#1 \ifx\\#2\\\else(#2) \fi[#3]}%
  }%
  \hologoList
  \typeout{*** End of logo list ***}%
  \typeout{}%
\endgroup

\begin{document}
\begin{landscape}

  \section{Example file for package `hologo'}

  % Table for font names

  \begin{longtable}{>{\bfseries}ll}
    \textbf{font} & \textbf{Font name}\\
    \hline
    lmr & Latin Modern Roman\\
    lmss & Latin Modern Sans\\
    qtm & \TeX\ Gyre Termes\\
    qhv & \TeX\ Gyre Heros\\
    qpl & \TeX\ Gyre Pagella\\
  \end{longtable}

  % Logo list with logos in different fonts

  \begingroup
    \newcommand*{\SetVariant}[2]{%
      \ifx\\#2\\%
      \else
        \hologoLogoSetup{#1}{variant=#2}%
      \fi
    }%
    \newcommand*{\hologoEntry}[3]{%
      \SetVariant{#1}{#2}%
      \raisebox{1em}[0pt][0pt]{\hypertarget{#1@#2}{}}%
      \bookmark[%
        dest={#1@#2},%
      ]{%
        #1\ifx\\#2\\\else\space(#2)\fi: \Hologo{#1}, \hologo{#1} %
        [Unicode]%
      }%
      \hypersetup{unicode=false}%
      \bookmark[%
        dest={#1@#2},%
      ]{%
        #1\ifx\\#2\\\else\space(#2)\fi: \Hologo{#1}, \hologo{#1} %
        [PDFDocEncoding]%
      }%
      \texttt{#1}%
      &%
      \texttt{#2}%
      &%
      \Hologo{#1}%
      &%
      \SetVariant{#1}{#2}%
      \hologo{#1}%
      &%
      \SetVariant{#1}{#2}%
      \fontfamily{qtm}\selectfont
      \hologo{#1}%
      &%
      \SetVariant{#1}{#2}%
      \fontfamily{qpl}\selectfont
      \hologo{#1}%
      &%
      \SetVariant{#1}{#2}%
      \textsf{\hologo{#1}}%
      &%
      \SetVariant{#1}{#2}%
      \fontfamily{qhv}\selectfont
      \hologo{#1}%
      \tabularnewline
    }%
    \begin{longtable}{llllllll}%
      \textbf{\textit{logo}} & \textbf{\textit{variant}} &
      \texttt{\string\Hologo} &
      \textbf{lmr} & \textbf{qtm} & \textbf{qpl} &
      \textbf{lmss} & \textbf{qhv}
      \tabularnewline
      \hline
      \endhead
      \hologoList
    \end{longtable}%
  \endgroup

\end{landscape}
\end{document}
%verbatim
%</example>
%    \end{macrocode}
%
% \StopEventually{
% }
%
% \section{Implementation}
%    \begin{macrocode}
%<*package>
%    \end{macrocode}
%    Reload check, especially if the package is not used with \LaTeX.
%    \begin{macrocode}
\begingroup\catcode61\catcode48\catcode32=10\relax%
  \catcode13=5 % ^^M
  \endlinechar=13 %
  \catcode35=6 % #
  \catcode39=12 % '
  \catcode44=12 % ,
  \catcode45=12 % -
  \catcode46=12 % .
  \catcode58=12 % :
  \catcode64=11 % @
  \catcode123=1 % {
  \catcode125=2 % }
  \expandafter\let\expandafter\x\csname ver@hologo.sty\endcsname
  \ifx\x\relax % plain-TeX, first loading
  \else
    \def\empty{}%
    \ifx\x\empty % LaTeX, first loading,
      % variable is initialized, but \ProvidesPackage not yet seen
    \else
      \expandafter\ifx\csname PackageInfo\endcsname\relax
        \def\x#1#2{%
          \immediate\write-1{Package #1 Info: #2.}%
        }%
      \else
        \def\x#1#2{\PackageInfo{#1}{#2, stopped}}%
      \fi
      \x{hologo}{The package is already loaded}%
      \aftergroup\endinput
    \fi
  \fi
\endgroup%
%    \end{macrocode}
%    Package identification:
%    \begin{macrocode}
\begingroup\catcode61\catcode48\catcode32=10\relax%
  \catcode13=5 % ^^M
  \endlinechar=13 %
  \catcode35=6 % #
  \catcode39=12 % '
  \catcode40=12 % (
  \catcode41=12 % )
  \catcode44=12 % ,
  \catcode45=12 % -
  \catcode46=12 % .
  \catcode47=12 % /
  \catcode58=12 % :
  \catcode64=11 % @
  \catcode91=12 % [
  \catcode93=12 % ]
  \catcode123=1 % {
  \catcode125=2 % }
  \expandafter\ifx\csname ProvidesPackage\endcsname\relax
    \def\x#1#2#3[#4]{\endgroup
      \immediate\write-1{Package: #3 #4}%
      \xdef#1{#4}%
    }%
  \else
    \def\x#1#2[#3]{\endgroup
      #2[{#3}]%
      \ifx#1\@undefined
        \xdef#1{#3}%
      \fi
      \ifx#1\relax
        \xdef#1{#3}%
      \fi
    }%
  \fi
\expandafter\x\csname ver@hologo.sty\endcsname
\ProvidesPackage{hologo}%
  [2016/05/12 v1.11 A logo collection with bookmark support (HO)]%
%    \end{macrocode}
%
%    \begin{macrocode}
\begingroup\catcode61\catcode48\catcode32=10\relax%
  \catcode13=5 % ^^M
  \endlinechar=13 %
  \catcode123=1 % {
  \catcode125=2 % }
  \catcode64=11 % @
  \def\x{\endgroup
    \expandafter\edef\csname HOLOGO@AtEnd\endcsname{%
      \endlinechar=\the\endlinechar\relax
      \catcode13=\the\catcode13\relax
      \catcode32=\the\catcode32\relax
      \catcode35=\the\catcode35\relax
      \catcode61=\the\catcode61\relax
      \catcode64=\the\catcode64\relax
      \catcode123=\the\catcode123\relax
      \catcode125=\the\catcode125\relax
    }%
  }%
\x\catcode61\catcode48\catcode32=10\relax%
\catcode13=5 % ^^M
\endlinechar=13 %
\catcode35=6 % #
\catcode64=11 % @
\catcode123=1 % {
\catcode125=2 % }
\def\TMP@EnsureCode#1#2{%
  \edef\HOLOGO@AtEnd{%
    \HOLOGO@AtEnd
    \catcode#1=\the\catcode#1\relax
  }%
  \catcode#1=#2\relax
}
\TMP@EnsureCode{10}{12}% ^^J
\TMP@EnsureCode{33}{12}% !
\TMP@EnsureCode{34}{12}% "
\TMP@EnsureCode{36}{3}% $
\TMP@EnsureCode{38}{4}% &
\TMP@EnsureCode{39}{12}% '
\TMP@EnsureCode{40}{12}% (
\TMP@EnsureCode{41}{12}% )
\TMP@EnsureCode{42}{12}% *
\TMP@EnsureCode{43}{12}% +
\TMP@EnsureCode{44}{12}% ,
\TMP@EnsureCode{45}{12}% -
\TMP@EnsureCode{46}{12}% .
\TMP@EnsureCode{47}{12}% /
\TMP@EnsureCode{58}{12}% :
\TMP@EnsureCode{59}{12}% ;
\TMP@EnsureCode{60}{12}% <
\TMP@EnsureCode{62}{12}% >
\TMP@EnsureCode{63}{12}% ?
\TMP@EnsureCode{91}{12}% [
\TMP@EnsureCode{93}{12}% ]
\TMP@EnsureCode{94}{7}% ^ (superscript)
\TMP@EnsureCode{95}{8}% _ (subscript)
\TMP@EnsureCode{96}{12}% `
\TMP@EnsureCode{124}{12}% |
\edef\HOLOGO@AtEnd{%
  \HOLOGO@AtEnd
  \escapechar\the\escapechar\relax
  \noexpand\endinput
}
\escapechar=92 %
%    \end{macrocode}
%
% \subsection{Logo list}
%
%    \begin{macro}{\hologoList}
%    \begin{macrocode}
\def\hologoList{%
  \hologoEntry{(La)TeX}{}{2011/10/01}%
  \hologoEntry{AmSLaTeX}{}{2010/04/16}%
  \hologoEntry{AmSTeX}{}{2010/04/16}%
  \hologoEntry{biber}{}{2011/10/01}%
  \hologoEntry{BibTeX}{}{2011/10/01}%
  \hologoEntry{BibTeX}{sf}{2011/10/01}%
  \hologoEntry{BibTeX}{sc}{2011/10/01}%
  \hologoEntry{BibTeX8}{}{2011/11/22}%
  \hologoEntry{ConTeXt}{}{2011/03/25}%
  \hologoEntry{ConTeXt}{narrow}{2011/03/25}%
  \hologoEntry{ConTeXt}{simple}{2011/03/25}%
  \hologoEntry{emTeX}{}{2010/04/26}%
  \hologoEntry{eTeX}{}{2010/04/08}%
  \hologoEntry{ExTeX}{}{2011/10/01}%
  \hologoEntry{HanTheThanh}{}{2011/11/29}%
  \hologoEntry{iniTeX}{}{2011/10/01}%
  \hologoEntry{KOMAScript}{}{2011/10/01}%
  \hologoEntry{La}{}{2010/05/08}%
  \hologoEntry{LaTeX}{}{2010/04/08}%
  \hologoEntry{LaTeX2e}{}{2010/04/08}%
  \hologoEntry{LaTeX3}{}{2010/04/24}%
  \hologoEntry{LaTeXe}{}{2010/04/08}%
  \hologoEntry{LaTeXML}{}{2011/11/22}%
  \hologoEntry{LaTeXTeX}{}{2011/10/01}%
  \hologoEntry{LuaLaTeX}{}{2010/04/08}%
  \hologoEntry{LuaTeX}{}{2010/04/08}%
  \hologoEntry{LyX}{}{2011/10/01}%
  \hologoEntry{METAFONT}{}{2011/10/01}%
  \hologoEntry{MetaFun}{}{2011/10/01}%
  \hologoEntry{METAPOST}{}{2011/10/01}%
  \hologoEntry{MetaPost}{}{2011/10/01}%
  \hologoEntry{MiKTeX}{}{2011/10/01}%
  \hologoEntry{NTS}{}{2011/10/01}%
  \hologoEntry{OzMF}{}{2011/10/01}%
  \hologoEntry{OzMP}{}{2011/10/01}%
  \hologoEntry{OzTeX}{}{2011/10/01}%
  \hologoEntry{OzTtH}{}{2011/10/01}%
  \hologoEntry{PCTeX}{}{2011/10/01}%
  \hologoEntry{pdfTeX}{}{2011/10/01}%
  \hologoEntry{pdfLaTeX}{}{2011/10/01}%
  \hologoEntry{PiC}{}{2011/10/01}%
  \hologoEntry{PiCTeX}{}{2011/10/01}%
  \hologoEntry{plainTeX}{}{2010/04/08}%
  \hologoEntry{plainTeX}{space}{2010/04/16}%
  \hologoEntry{plainTeX}{hyphen}{2010/04/16}%
  \hologoEntry{plainTeX}{runtogether}{2010/04/16}%
  \hologoEntry{SageTeX}{}{2011/11/22}%
  \hologoEntry{SLiTeX}{}{2011/10/01}%
  \hologoEntry{SLiTeX}{lift}{2011/10/01}%
  \hologoEntry{SLiTeX}{narrow}{2011/10/01}%
  \hologoEntry{SLiTeX}{simple}{2011/10/01}%
  \hologoEntry{SliTeX}{}{2011/10/01}%
  \hologoEntry{SliTeX}{narrow}{2011/10/01}%
  \hologoEntry{SliTeX}{simple}{2011/10/01}%
  \hologoEntry{SliTeX}{lift}{2011/10/01}%
  \hologoEntry{teTeX}{}{2011/10/01}%
  \hologoEntry{TeX}{}{2010/04/08}%
  \hologoEntry{TeX4ht}{}{2011/11/22}%
  \hologoEntry{TTH}{}{2011/11/22}%
  \hologoEntry{virTeX}{}{2011/10/01}%
  \hologoEntry{VTeX}{}{2010/04/24}%
  \hologoEntry{Xe}{}{2010/04/08}%
  \hologoEntry{XeLaTeX}{}{2010/04/08}%
  \hologoEntry{XeTeX}{}{2010/04/08}%
}
%    \end{macrocode}
%    \end{macro}
%
% \subsection{Load resources}
%
%    \begin{macrocode}
\begingroup\expandafter\expandafter\expandafter\endgroup
\expandafter\ifx\csname RequirePackage\endcsname\relax
  \def\TMP@RequirePackage#1[#2]{%
    \begingroup\expandafter\expandafter\expandafter\endgroup
    \expandafter\ifx\csname ver@#1.sty\endcsname\relax
      \input #1.sty\relax
    \fi
  }%
  \TMP@RequirePackage{ltxcmds}[2011/02/04]%
  \TMP@RequirePackage{infwarerr}[2010/04/08]%
  \TMP@RequirePackage{kvsetkeys}[2010/03/01]%
  \TMP@RequirePackage{kvdefinekeys}[2010/03/01]%
  \TMP@RequirePackage{pdftexcmds}[2010/04/01]%
  \TMP@RequirePackage{ifpdf}[2010/01/28]%
  \TMP@RequirePackage{ifluatex}[2010/03/01]%
  \ltx@IfUndefined{newif}{%
    \expandafter\let\csname newif\endcsname\ltx@newif
  }{}%
  \TMP@RequirePackage{ifxetex}[2009/01/23]%
  \TMP@RequirePackage{ifvtex}[2010/03/01]%
\else
  \RequirePackage{ltxcmds}[2011/02/04]%
  \RequirePackage{infwarerr}[2010/04/08]%
  \RequirePackage{kvsetkeys}[2010/03/01]%
  \RequirePackage{kvdefinekeys}[2010/03/01]%
  \RequirePackage{pdftexcmds}[2010/04/01]%
  \RequirePackage{ifpdf}[2010/01/28]%
  \RequirePackage{ifluatex}[2010/03/01]%
  \RequirePackage{ifxetex}[2009/01/23]%
  \RequirePackage{ifvtex}[2010/03/01]%
\fi
%    \end{macrocode}
%
%    \begin{macro}{\HOLOGO@IfDefined}
%    \begin{macrocode}
\def\HOLOGO@IfExists#1{%
  \ifx\@undefined#1%
    \expandafter\ltx@secondoftwo
  \else
    \ifx\relax#1%
      \expandafter\ltx@secondoftwo
    \else
      \expandafter\expandafter\expandafter\ltx@firstoftwo
    \fi
  \fi
}
%    \end{macrocode}
%    \end{macro}
%
% \subsection{Setup macros}
%
%    \begin{macro}{\hologoSetup}
%    \begin{macrocode}
\def\hologoSetup{%
  \let\HOLOGO@name\relax
  \HOLOGO@Setup
}
%    \end{macrocode}
%    \end{macro}
%
%    \begin{macro}{\hologoLogoSetup}
%    \begin{macrocode}
\def\hologoLogoSetup#1{%
  \edef\HOLOGO@name{#1}%
  \ltx@IfUndefined{HoLogo@\HOLOGO@name}{%
    \@PackageError{hologo}{%
      Unknown logo `\HOLOGO@name'%
    }\@ehc
    \ltx@gobble
  }{%
    \HOLOGO@Setup
  }%
}
%    \end{macrocode}
%    \end{macro}
%
%    \begin{macro}{\HOLOGO@Setup}
%    \begin{macrocode}
\def\HOLOGO@Setup{%
  \kvsetkeys{HoLogo}%
}
%    \end{macrocode}
%    \end{macro}
%
% \subsection{Options}
%
%    \begin{macro}{\HOLOGO@DeclareBoolOption}
%    \begin{macrocode}
\def\HOLOGO@DeclareBoolOption#1{%
  \expandafter\chardef\csname HOLOGOOPT@#1\endcsname\ltx@zero
  \kv@define@key{HoLogo}{#1}[true]{%
    \def\HOLOGO@temp{##1}%
    \ifx\HOLOGO@temp\HOLOGO@true
      \ifx\HOLOGO@name\relax
        \expandafter\chardef\csname HOLOGOOPT@#1\endcsname=\ltx@one
      \else
        \expandafter\chardef\csname
        HoLogoOpt@#1@\HOLOGO@name\endcsname\ltx@one
      \fi
      \HOLOGO@SetBreakAll{#1}%
    \else
      \ifx\HOLOGO@temp\HOLOGO@false
        \ifx\HOLOGO@name\relax
          \expandafter\chardef\csname HOLOGOOPT@#1\endcsname=\ltx@zero
        \else
          \expandafter\chardef\csname
          HoLogoOpt@#1@\HOLOGO@name\endcsname=\ltx@zero
        \fi
        \HOLOGO@SetBreakAll{#1}%
      \else
        \@PackageError{hologo}{%
          Unknown value `##1' for boolean option `#1'.\MessageBreak
          Known values are `true' and `false'%
        }\@ehc
      \fi
    \fi
  }%
}
%    \end{macrocode}
%    \end{macro}
%
%    \begin{macro}{\HOLOGO@SetBreakAll}
%    \begin{macrocode}
\def\HOLOGO@SetBreakAll#1{%
  \def\HOLOGO@temp{#1}%
  \ifx\HOLOGO@temp\HOLOGO@break
    \ifx\HOLOGO@name\relax
      \chardef\HOLOGOOPT@hyphenbreak=\HOLOGOOPT@break
      \chardef\HOLOGOOPT@spacebreak=\HOLOGOOPT@break
      \chardef\HOLOGOOPT@discretionarybreak=\HOLOGOOPT@break
    \else
      \expandafter\chardef
         \csname HoLogoOpt@hyphenbreak@\HOLOGO@name\endcsname=%
         \csname HoLogoOpt@break@\HOLOGO@name\endcsname
      \expandafter\chardef
         \csname HoLogoOpt@spacebreak@\HOLOGO@name\endcsname=%
         \csname HoLogoOpt@break@\HOLOGO@name\endcsname
      \expandafter\chardef
         \csname HoLogoOpt@discretionarybreak@\HOLOGO@name
             \endcsname=%
         \csname HoLogoOpt@break@\HOLOGO@name\endcsname
    \fi
  \fi
}
%    \end{macrocode}
%    \end{macro}
%
%    \begin{macro}{\HOLOGO@true}
%    \begin{macrocode}
\def\HOLOGO@true{true}
%    \end{macrocode}
%    \end{macro}
%    \begin{macro}{\HOLOGO@false}
%    \begin{macrocode}
\def\HOLOGO@false{false}
%    \end{macrocode}
%    \end{macro}
%    \begin{macro}{\HOLOGO@break}
%    \begin{macrocode}
\def\HOLOGO@break{break}
%    \end{macrocode}
%    \end{macro}
%
%    \begin{macrocode}
\HOLOGO@DeclareBoolOption{break}
\HOLOGO@DeclareBoolOption{hyphenbreak}
\HOLOGO@DeclareBoolOption{spacebreak}
\HOLOGO@DeclareBoolOption{discretionarybreak}
%    \end{macrocode}
%
%    \begin{macrocode}
\kv@define@key{HoLogo}{variant}{%
  \ifx\HOLOGO@name\relax
    \@PackageError{hologo}{%
      Option `variant' is not available in \string\hologoSetup,%
      \MessageBreak
      Use \string\hologoLogoSetup\space instead%
    }\@ehc
  \else
    \edef\HOLOGO@temp{#1}%
    \ifx\HOLOGO@temp\ltx@empty
      \expandafter
      \let\csname HoLogoOpt@variant@\HOLOGO@name\endcsname\@undefined
    \else
      \ltx@IfUndefined{HoLogo@\HOLOGO@name @\HOLOGO@temp}{%
        \@PackageError{hologo}{%
          Unknown variant `\HOLOGO@temp' of logo `\HOLOGO@name'%
        }\@ehc
      }{%
        \expandafter
        \let\csname HoLogoOpt@variant@\HOLOGO@name\endcsname
            \HOLOGO@temp
      }%
    \fi
  \fi
}
%    \end{macrocode}
%
%    \begin{macro}{\HOLOGO@Variant}
%    \begin{macrocode}
\def\HOLOGO@Variant#1{%
  #1%
  \ltx@ifundefined{HoLogoOpt@variant@#1}{%
  }{%
    @\csname HoLogoOpt@variant@#1\endcsname
  }%
}
%    \end{macrocode}
%    \end{macro}
%
% \subsection{Break/no-break support}
%
%    \begin{macro}{\HOLOGO@space}
%    \begin{macrocode}
\def\HOLOGO@space{%
  \ltx@ifundefined{HoLogoOpt@spacebreak@\HOLOGO@name}{%
    \ltx@ifundefined{HoLogoOpt@break@\HOLOGO@name}{%
      \chardef\HOLOGO@temp=\HOLOGOOPT@spacebreak
    }{%
      \chardef\HOLOGO@temp=%
        \csname HoLogoOpt@break@\HOLOGO@name\endcsname
    }%
  }{%
    \chardef\HOLOGO@temp=%
      \csname HoLogoOpt@spacebreak@\HOLOGO@name\endcsname
  }%
  \ifcase\HOLOGO@temp
    \penalty10000 %
  \fi
  \ltx@space
}
%    \end{macrocode}
%    \end{macro}
%
%    \begin{macro}{\HOLOGO@hyphen}
%    \begin{macrocode}
\def\HOLOGO@hyphen{%
  \ltx@ifundefined{HoLogoOpt@hyphenbreak@\HOLOGO@name}{%
    \ltx@ifundefined{HoLogoOpt@break@\HOLOGO@name}{%
      \chardef\HOLOGO@temp=\HOLOGOOPT@hyphenbreak
    }{%
      \chardef\HOLOGO@temp=%
        \csname HoLogoOpt@break@\HOLOGO@name\endcsname
    }%
  }{%
    \chardef\HOLOGO@temp=%
      \csname HoLogoOpt@hyphenbreak@\HOLOGO@name\endcsname
  }%
  \ifcase\HOLOGO@temp
    \ltx@mbox{-}%
  \else
    -%
  \fi
}
%    \end{macrocode}
%    \end{macro}
%
%    \begin{macro}{\HOLOGO@discretionary}
%    \begin{macrocode}
\def\HOLOGO@discretionary{%
  \ltx@ifundefined{HoLogoOpt@discretionarybreak@\HOLOGO@name}{%
    \ltx@ifundefined{HoLogoOpt@break@\HOLOGO@name}{%
      \chardef\HOLOGO@temp=\HOLOGOOPT@discretionarybreak
    }{%
      \chardef\HOLOGO@temp=%
        \csname HoLogoOpt@break@\HOLOGO@name\endcsname
    }%
  }{%
    \chardef\HOLOGO@temp=%
      \csname HoLogoOpt@discretionarybreak@\HOLOGO@name\endcsname
  }%
  \ifcase\HOLOGO@temp
  \else
    \-%
  \fi
}
%    \end{macrocode}
%    \end{macro}
%
%    \begin{macro}{\HOLOGO@mbox}
%    \begin{macrocode}
\def\HOLOGO@mbox#1{%
  \ltx@ifundefined{HoLogoOpt@break@\HOLOGO@name}{%
    \chardef\HOLOGO@temp=\HOLOGOOPT@hyphenbreak
  }{%
    \chardef\HOLOGO@temp=%
      \csname HoLogoOpt@break@\HOLOGO@name\endcsname
  }%
  \ifcase\HOLOGO@temp
    \ltx@mbox{#1}%
  \else
    #1%
  \fi
}
%    \end{macrocode}
%    \end{macro}
%
% \subsection{Font support}
%
%    \begin{macro}{\HoLogoFont@font}
%    \begin{tabular}{@{}ll@{}}
%    |#1|:& logo name\\
%    |#2|:& font short name\\
%    |#3|:& text
%    \end{tabular}
%    \begin{macrocode}
\def\HoLogoFont@font#1#2#3{%
  \begingroup
    \ltx@IfUndefined{HoLogoFont@logo@#1.#2}{%
      \ltx@IfUndefined{HoLogoFont@font@#2}{%
        \@PackageWarning{hologo}{%
          Missing font `#2' for logo `#1'%
        }%
        #3%
      }{%
        \csname HoLogoFont@font@#2\endcsname{#3}%
      }%
    }{%
      \csname HoLogoFont@logo@#1.#2\endcsname{#3}%
    }%
  \endgroup
}
%    \end{macrocode}
%    \end{macro}
%
%    \begin{macro}{\HoLogoFont@Def}
%    \begin{macrocode}
\def\HoLogoFont@Def#1{%
  \expandafter\def\csname HoLogoFont@font@#1\endcsname
}
%    \end{macrocode}
%    \end{macro}
%    \begin{macro}{\HoLogoFont@LogoDef}
%    \begin{macrocode}
\def\HoLogoFont@LogoDef#1#2{%
  \expandafter\def\csname HoLogoFont@logo@#1.#2\endcsname
}
%    \end{macrocode}
%    \end{macro}
%
% \subsubsection{Font defaults}
%
%    \begin{macro}{\HoLogoFont@font@general}
%    \begin{macrocode}
\HoLogoFont@Def{general}{}%
%    \end{macrocode}
%    \end{macro}
%
%    \begin{macro}{\HoLogoFont@font@rm}
%    \begin{macrocode}
\ltx@IfUndefined{rmfamily}{%
  \ltx@IfUndefined{rm}{%
  }{%
    \HoLogoFont@Def{rm}{\rm}%
  }%
}{%
  \HoLogoFont@Def{rm}{\rmfamily}%
}
%    \end{macrocode}
%    \end{macro}
%
%    \begin{macro}{\HoLogoFont@font@sf}
%    \begin{macrocode}
\ltx@IfUndefined{sffamily}{%
  \ltx@IfUndefined{sf}{%
  }{%
    \HoLogoFont@Def{sf}{\sf}%
  }%
}{%
  \HoLogoFont@Def{sf}{\sffamily}%
}
%    \end{macrocode}
%    \end{macro}
%
%    \begin{macro}{\HoLogoFont@font@bibsf}
%    In case of \hologo{plainTeX} the original small caps
%    variant is used as default. In \hologo{LaTeX}
%    the definition of package \xpackage{dtklogos} \cite{dtklogos}
%    is used.
%\begin{quote}
%\begin{verbatim}
%\DeclareRobustCommand{\BibTeX}{%
%  B%
%  \kern-.05em%
%  \hbox{%
%    $\m@th$% %% force math size calculations
%    \csname S@\f@size\endcsname
%    \fontsize\sf@size\z@
%    \math@fontsfalse
%    \selectfont
%    I%
%    \kern-.025em%
%    B
%  }%
%  \kern-.08em%
%  \-%
%  \TeX
%}
%\end{verbatim}
%\end{quote}
%    \begin{macrocode}
\ltx@IfUndefined{selectfont}{%
  \ltx@IfUndefined{tensc}{%
    \font\tensc=cmcsc10\relax
  }{}%
  \HoLogoFont@Def{bibsf}{\tensc}%
}{%
  \HoLogoFont@Def{bibsf}{%
    $\mathsurround=0pt$%
    \csname S@\f@size\endcsname
    \fontsize\sf@size{0pt}%
    \math@fontsfalse
    \selectfont
  }%
}
%    \end{macrocode}
%    \end{macro}
%
%    \begin{macro}{\HoLogoFont@font@sc}
%    \begin{macrocode}
\ltx@IfUndefined{scshape}{%
  \ltx@IfUndefined{tensc}{%
    \font\tensc=cmcsc10\relax
  }{}%
  \HoLogoFont@Def{sc}{\tensc}%
}{%
  \HoLogoFont@Def{sc}{\scshape}%
}
%    \end{macrocode}
%    \end{macro}
%
%    \begin{macro}{\HoLogoFont@font@sy}
%    \begin{macrocode}
\ltx@IfUndefined{usefont}{%
  \ltx@IfUndefined{tensy}{%
  }{%
    \HoLogoFont@Def{sy}{\tensy}%
  }%
}{%
  \HoLogoFont@Def{sy}{%
    \usefont{OMS}{cmsy}{m}{n}%
  }%
}
%    \end{macrocode}
%    \end{macro}
%
%    \begin{macro}{\HoLogoFont@font@logo}
%    \begin{macrocode}
\begingroup
  \def\x{LaTeX2e}%
\expandafter\endgroup
\ifx\fmtname\x
  \ltx@IfUndefined{logofamily}{%
    \DeclareRobustCommand\logofamily{%
      \not@math@alphabet\logofamily\relax
      \fontencoding{U}%
      \fontfamily{logo}%
      \selectfont
    }%
  }{}%
  \ltx@IfUndefined{logofamily}{%
  }{%
    \HoLogoFont@Def{logo}{\logofamily}%
  }%
\else
  \ltx@IfUndefined{tenlogo}{%
    \font\tenlogo=logo10\relax
  }{}%
  \HoLogoFont@Def{logo}{\tenlogo}%
\fi
%    \end{macrocode}
%    \end{macro}
%
% \subsubsection{Font setup}
%
%    \begin{macro}{\hologoFontSetup}
%    \begin{macrocode}
\def\hologoFontSetup{%
  \let\HOLOGO@name\relax
  \HOLOGO@FontSetup
}
%    \end{macrocode}
%    \end{macro}
%
%    \begin{macro}{\hologoLogoFontSetup}
%    \begin{macrocode}
\def\hologoLogoFontSetup#1{%
  \edef\HOLOGO@name{#1}%
  \ltx@IfUndefined{HoLogo@\HOLOGO@name}{%
    \@PackageError{hologo}{%
      Unknown logo `\HOLOGO@name'%
    }\@ehc
    \ltx@gobble
  }{%
    \HOLOGO@FontSetup
  }%
}
%    \end{macrocode}
%    \end{macro}
%
%    \begin{macro}{\HOLOGO@FontSetup}
%    \begin{macrocode}
\def\HOLOGO@FontSetup{%
  \kvsetkeys{HoLogoFont}%
}
%    \end{macrocode}
%    \end{macro}
%
%    \begin{macrocode}
\def\HOLOGO@temp#1{%
  \kv@define@key{HoLogoFont}{#1}{%
    \ifx\HOLOGO@name\relax
      \HoLogoFont@Def{#1}{##1}%
    \else
      \HoLogoFont@LogoDef\HOLOGO@name{#1}{##1}%
    \fi
  }%
}
\HOLOGO@temp{general}
\HOLOGO@temp{sf}
%    \end{macrocode}
%
% \subsection{Generic logo commands}
%
%    \begin{macrocode}
\HOLOGO@IfExists\hologo{%
  \@PackageError{hologo}{%
    \string\hologo\ltx@space is already defined.\MessageBreak
    Package loading is aborted%
  }\@ehc
  \HOLOGO@AtEnd
}%
\HOLOGO@IfExists\hologoRobust{%
  \@PackageError{hologo}{%
    \string\hologoRobust\ltx@space is already defined.\MessageBreak
    Package loading is aborted%
  }\@ehc
  \HOLOGO@AtEnd
}%
%    \end{macrocode}
%
% \subsubsection{\cs{hologo} and friends}
%
%    \begin{macrocode}
\ifluatex
  \expandafter\ltx@firstofone
\else
  \expandafter\ltx@gobble
\fi
{%
  \ltx@IfUndefined{ifincsname}{%
    \ifnum\luatexversion<36 %
      \expandafter\ltx@gobble
    \else
      \expandafter\ltx@firstofone
    \fi
    {%
      \begingroup
        \ifcase0%
            \directlua{%
              if tex.enableprimitives then %
                tex.enableprimitives('HOLOGO@', {'ifincsname'})%
              else %
                tex.print('1')%
              end%
            }%
            \ifx\HOLOGO@ifincsname\@undefined 1\fi%
            \relax
          \expandafter\ltx@firstofone
        \else
          \endgroup
          \expandafter\ltx@gobble
        \fi
        {%
          \global\let\ifincsname\HOLOGO@ifincsname
        }%
      \HOLOGO@temp
    }%
  }{}%
}
%    \end{macrocode}
%    \begin{macrocode}
\ltx@IfUndefined{ifincsname}{%
  \catcode`$=14 %
}{%
  \catcode`$=9 %
}
%    \end{macrocode}
%
%    \begin{macro}{\hologo}
%    \begin{macrocode}
\def\hologo#1{%
$ \ifincsname
$   \ltx@ifundefined{HoLogoCs@\HOLOGO@Variant{#1}}{%
$     #1%
$   }{%
$     \csname HoLogoCs@\HOLOGO@Variant{#1}\endcsname\ltx@firstoftwo
$   }%
$ \else
    \HOLOGO@IfExists\texorpdfstring\texorpdfstring\ltx@firstoftwo
    {%
      \hologoRobust{#1}%
    }{%
      \ltx@ifundefined{HoLogoBkm@\HOLOGO@Variant{#1}}{%
        \ltx@ifundefined{HoLogo@#1}{?#1?}{#1}%
      }{%
        \csname HoLogoBkm@\HOLOGO@Variant{#1}\endcsname
        \ltx@firstoftwo
      }%
    }%
$ \fi
}
%    \end{macrocode}
%    \end{macro}
%    \begin{macro}{\Hologo}
%    \begin{macrocode}
\def\Hologo#1{%
$ \ifincsname
$   \ltx@ifundefined{HoLogoCs@\HOLOGO@Variant{#1}}{%
$     #1%
$   }{%
$     \csname HoLogoCs@\HOLOGO@Variant{#1}\endcsname\ltx@secondoftwo
$   }%
$ \else
    \HOLOGO@IfExists\texorpdfstring\texorpdfstring\ltx@firstoftwo
    {%
      \HologoRobust{#1}%
    }{%
      \ltx@ifundefined{HoLogoBkm@\HOLOGO@Variant{#1}}{%
        \ltx@ifundefined{HoLogo@#1}{?#1?}{#1}%
      }{%
        \csname HoLogoBkm@\HOLOGO@Variant{#1}\endcsname
        \ltx@secondoftwo
      }%
    }%
$ \fi
}
%    \end{macrocode}
%    \end{macro}
%
%    \begin{macro}{\hologoVariant}
%    \begin{macrocode}
\def\hologoVariant#1#2{%
  \ifx\relax#2\relax
    \hologo{#1}%
  \else
$   \ifincsname
$     \ltx@ifundefined{HoLogoCs@#1@#2}{%
$       #1%
$     }{%
$       \csname HoLogoCs@#1@#2\endcsname\ltx@firstoftwo
$     }%
$   \else
      \HOLOGO@IfExists\texorpdfstring\texorpdfstring\ltx@firstoftwo
      {%
        \hologoVariantRobust{#1}{#2}%
      }{%
        \ltx@ifundefined{HoLogoBkm@#1@#2}{%
          \ltx@ifundefined{HoLogo@#1}{?#1?}{#1}%
        }{%
          \csname HoLogoBkm@#1@#2\endcsname
          \ltx@firstoftwo
        }%
      }%
$   \fi
  \fi
}
%    \end{macrocode}
%    \end{macro}
%    \begin{macro}{\HologoVariant}
%    \begin{macrocode}
\def\HologoVariant#1#2{%
  \ifx\relax#2\relax
    \Hologo{#1}%
  \else
$   \ifincsname
$     \ltx@ifundefined{HoLogoCs@#1@#2}{%
$       #1%
$     }{%
$       \csname HoLogoCs@#1@#2\endcsname\ltx@secondoftwo
$     }%
$   \else
      \HOLOGO@IfExists\texorpdfstring\texorpdfstring\ltx@firstoftwo
      {%
        \HologoVariantRobust{#1}{#2}%
      }{%
        \ltx@ifundefined{HoLogoBkm@#1@#2}{%
          \ltx@ifundefined{HoLogo@#1}{?#1?}{#1}%
        }{%
          \csname HoLogoBkm@#1@#2\endcsname
          \ltx@secondoftwo
        }%
      }%
$   \fi
  \fi
}
%    \end{macrocode}
%    \end{macro}
%
%    \begin{macrocode}
\catcode`\$=3 %
%    \end{macrocode}
%
% \subsubsection{\cs{hologoRobust} and friends}
%
%    \begin{macro}{\hologoRobust}
%    \begin{macrocode}
\ltx@IfUndefined{protected}{%
  \ltx@IfUndefined{DeclareRobustCommand}{%
    \def\hologoRobust#1%
  }{%
    \DeclareRobustCommand*\hologoRobust[1]%
  }%
}{%
  \protected\def\hologoRobust#1%
}%
{%
  \edef\HOLOGO@name{#1}%
  \ltx@IfUndefined{HoLogo@\HOLOGO@Variant\HOLOGO@name}{%
    \@PackageError{hologo}{%
      Unknown logo `\HOLOGO@name'%
    }\@ehc
    ?\HOLOGO@name?%
  }{%
    \ltx@IfUndefined{ver@tex4ht.sty}{%
      \HoLogoFont@font\HOLOGO@name{general}{%
        \csname HoLogo@\HOLOGO@Variant\HOLOGO@name\endcsname
        \ltx@firstoftwo
      }%
    }{%
      \ltx@IfUndefined{HoLogoHtml@\HOLOGO@Variant\HOLOGO@name}{%
        \HOLOGO@name
      }{%
        \csname HoLogoHtml@\HOLOGO@Variant\HOLOGO@name\endcsname
        \ltx@firstoftwo
      }%
    }%
  }%
}
%    \end{macrocode}
%    \end{macro}
%    \begin{macro}{\HologoRobust}
%    \begin{macrocode}
\ltx@IfUndefined{protected}{%
  \ltx@IfUndefined{DeclareRobustCommand}{%
    \def\HologoRobust#1%
  }{%
    \DeclareRobustCommand*\HologoRobust[1]%
  }%
}{%
  \protected\def\HologoRobust#1%
}%
{%
  \edef\HOLOGO@name{#1}%
  \ltx@IfUndefined{HoLogo@\HOLOGO@Variant\HOLOGO@name}{%
    \@PackageError{hologo}{%
      Unknown logo `\HOLOGO@name'%
    }\@ehc
    ?\HOLOGO@name?%
  }{%
    \ltx@IfUndefined{ver@tex4ht.sty}{%
      \HoLogoFont@font\HOLOGO@name{general}{%
        \csname HoLogo@\HOLOGO@Variant\HOLOGO@name\endcsname
        \ltx@secondoftwo
      }%
    }{%
      \ltx@IfUndefined{HoLogoHtml@\HOLOGO@Variant\HOLOGO@name}{%
        \expandafter\HOLOGO@Uppercase\HOLOGO@name
      }{%
        \csname HoLogoHtml@\HOLOGO@Variant\HOLOGO@name\endcsname
        \ltx@secondoftwo
      }%
    }%
  }%
}
%    \end{macrocode}
%    \end{macro}
%    \begin{macro}{\hologoVariantRobust}
%    \begin{macrocode}
\ltx@IfUndefined{protected}{%
  \ltx@IfUndefined{DeclareRobustCommand}{%
    \def\hologoVariantRobust#1#2%
  }{%
    \DeclareRobustCommand*\hologoVariantRobust[2]%
  }%
}{%
  \protected\def\hologoVariantRobust#1#2%
}%
{%
  \begingroup
    \hologoLogoSetup{#1}{variant={#2}}%
    \hologoRobust{#1}%
  \endgroup
}
%    \end{macrocode}
%    \end{macro}
%    \begin{macro}{\HologoVariantRobust}
%    \begin{macrocode}
\ltx@IfUndefined{protected}{%
  \ltx@IfUndefined{DeclareRobustCommand}{%
    \def\HologoVariantRobust#1#2%
  }{%
    \DeclareRobustCommand*\HologoVariantRobust[2]%
  }%
}{%
  \protected\def\HologoVariantRobust#1#2%
}%
{%
  \begingroup
    \hologoLogoSetup{#1}{variant={#2}}%
    \HologoRobust{#1}%
  \endgroup
}
%    \end{macrocode}
%    \end{macro}
%
%    \begin{macro}{\hologorobust}
%    Macro \cs{hologorobust} is only defined for compatibility.
%    Its use is deprecated.
%    \begin{macrocode}
\def\hologorobust{\hologoRobust}
%    \end{macrocode}
%    \end{macro}
%
% \subsection{Helpers}
%
%    \begin{macro}{\HOLOGO@Uppercase}
%    Macro \cs{HOLOGO@Uppercase} is restricted to \cs{uppercase},
%    because \hologo{plainTeX} or \hologo{iniTeX} do not provide
%    \cs{MakeUppercase}.
%    \begin{macrocode}
\def\HOLOGO@Uppercase#1{\uppercase{#1}}
%    \end{macrocode}
%    \end{macro}
%
%    \begin{macro}{\HOLOGO@PdfdocUnicode}
%    \begin{macrocode}
\def\HOLOGO@PdfdocUnicode{%
  \ifx\ifHy@unicode\iftrue
    \expandafter\ltx@secondoftwo
  \else
    \expandafter\ltx@firstoftwo
  \fi
}
%    \end{macrocode}
%    \end{macro}
%
%    \begin{macro}{\HOLOGO@Math}
%    \begin{macrocode}
\def\HOLOGO@MathSetup{%
  \mathsurround0pt\relax
  \HOLOGO@IfExists\f@series{%
    \if b\expandafter\ltx@car\f@series x\@nil
      \csname boldmath\endcsname
   \fi
  }{}%
}
%    \end{macrocode}
%    \end{macro}
%
%    \begin{macro}{\HOLOGO@TempDimen}
%    \begin{macrocode}
\dimendef\HOLOGO@TempDimen=\ltx@zero
%    \end{macrocode}
%    \end{macro}
%    \begin{macro}{\HOLOGO@NegativeKerning}
%    \begin{macrocode}
\def\HOLOGO@NegativeKerning#1{%
  \begingroup
    \HOLOGO@TempDimen=0pt\relax
    \comma@parse@normalized{#1}{%
      \ifdim\HOLOGO@TempDimen=0pt %
        \expandafter\HOLOGO@@NegativeKerning\comma@entry
      \fi
      \ltx@gobble
    }%
    \ifdim\HOLOGO@TempDimen<0pt %
      \kern\HOLOGO@TempDimen
    \fi
  \endgroup
}
%    \end{macrocode}
%    \end{macro}
%    \begin{macro}{\HOLOGO@@NegativeKerning}
%    \begin{macrocode}
\def\HOLOGO@@NegativeKerning#1#2{%
  \setbox\ltx@zero\hbox{#1#2}%
  \HOLOGO@TempDimen=\wd\ltx@zero
  \setbox\ltx@zero\hbox{#1\kern0pt#2}%
  \advance\HOLOGO@TempDimen by -\wd\ltx@zero
}
%    \end{macrocode}
%    \end{macro}
%
%    \begin{macro}{\HOLOGO@SpaceFactor}
%    \begin{macrocode}
\def\HOLOGO@SpaceFactor{%
  \spacefactor1000 %
}
%    \end{macrocode}
%    \end{macro}
%
%    \begin{macro}{\HOLOGO@Span}
%    \begin{macrocode}
\def\HOLOGO@Span#1#2{%
  \HCode{<span class="HoLogo-#1">}%
  #2%
  \HCode{</span>}%
}
%    \end{macrocode}
%    \end{macro}
%
% \subsubsection{Text subscript}
%
%    \begin{macro}{\HOLOGO@SubScript}%
%    \begin{macrocode}
\def\HOLOGO@SubScript#1{%
  \ltx@IfUndefined{textsubscript}{%
    \ltx@IfUndefined{text}{%
      \ltx@mbox{%
        \mathsurround=0pt\relax
        $%
          _{%
            \ltx@IfUndefined{sf@size}{%
              \mathrm{#1}%
            }{%
              \mbox{%
                \fontsize\sf@size{0pt}\selectfont
                #1%
              }%
            }%
          }%
        $%
      }%
    }{%
      \ltx@mbox{%
        \mathsurround=0pt\relax
        $_{\text{#1}}$%
      }%
    }%
  }{%
    \textsubscript{#1}%
  }%
}
%    \end{macrocode}
%    \end{macro}
%
% \subsection{\hologo{TeX} and friends}
%
% \subsubsection{\hologo{TeX}}
%
%    \begin{macro}{\HoLogo@TeX}
%    Source: \hologo{LaTeX} kernel.
%    \begin{macrocode}
\def\HoLogo@TeX#1{%
  T\kern-.1667em\lower.5ex\hbox{E}\kern-.125emX\HOLOGO@SpaceFactor
}
%    \end{macrocode}
%    \end{macro}
%    \begin{macro}{\HoLogoHtml@TeX}
%    \begin{macrocode}
\def\HoLogoHtml@TeX#1{%
  \HoLogoCss@TeX
  \HOLOGO@Span{TeX}{%
    T%
    \HOLOGO@Span{e}{%
      E%
    }%
    X%
  }%
}
%    \end{macrocode}
%    \end{macro}
%    \begin{macro}{\HoLogoCss@TeX}
%    \begin{macrocode}
\def\HoLogoCss@TeX{%
  \Css{%
    span.HoLogo-TeX span.HoLogo-e{%
      position:relative;%
      top:.5ex;%
      margin-left:-.1667em;%
      margin-right:-.125em;%
    }%
  }%
  \Css{%
    a span.HoLogo-TeX span.HoLogo-e{%
      text-decoration:none;%
    }%
  }%
  \global\let\HoLogoCss@TeX\relax
}
%    \end{macrocode}
%    \end{macro}
%
% \subsubsection{\hologo{plainTeX}}
%
%    \begin{macro}{\HoLogo@plainTeX@space}
%    Source: ``The \hologo{TeX}book''
%    \begin{macrocode}
\def\HoLogo@plainTeX@space#1{%
  \HOLOGO@mbox{#1{p}{P}lain}\HOLOGO@space\hologo{TeX}%
}
%    \end{macrocode}
%    \end{macro}
%    \begin{macro}{\HoLogoCs@plainTeX@space}
%    \begin{macrocode}
\def\HoLogoCs@plainTeX@space#1{#1{p}{P}lain TeX}%
%    \end{macrocode}
%    \end{macro}
%    \begin{macro}{\HoLogoBkm@plainTeX@space}
%    \begin{macrocode}
\def\HoLogoBkm@plainTeX@space#1{%
  #1{p}{P}lain \hologo{TeX}%
}
%    \end{macrocode}
%    \end{macro}
%    \begin{macro}{\HoLogoHtml@plainTeX@space}
%    \begin{macrocode}
\def\HoLogoHtml@plainTeX@space#1{%
  #1{p}{P}lain \hologo{TeX}%
}
%    \end{macrocode}
%    \end{macro}
%
%    \begin{macro}{\HoLogo@plainTeX@hyphen}
%    \begin{macrocode}
\def\HoLogo@plainTeX@hyphen#1{%
  \HOLOGO@mbox{#1{p}{P}lain}\HOLOGO@hyphen\hologo{TeX}%
}
%    \end{macrocode}
%    \end{macro}
%    \begin{macro}{\HoLogoCs@plainTeX@hyphen}
%    \begin{macrocode}
\def\HoLogoCs@plainTeX@hyphen#1{#1{p}{P}lain-TeX}
%    \end{macrocode}
%    \end{macro}
%    \begin{macro}{\HoLogoBkm@plainTeX@hyphen}
%    \begin{macrocode}
\def\HoLogoBkm@plainTeX@hyphen#1{%
  #1{p}{P}lain-\hologo{TeX}%
}
%    \end{macrocode}
%    \end{macro}
%    \begin{macro}{\HoLogoHtml@plainTeX@hyphen}
%    \begin{macrocode}
\def\HoLogoHtml@plainTeX@hyphen#1{%
  #1{p}{P}lain-\hologo{TeX}%
}
%    \end{macrocode}
%    \end{macro}
%
%    \begin{macro}{\HoLogo@plainTeX@runtogether}
%    \begin{macrocode}
\def\HoLogo@plainTeX@runtogether#1{%
  \HOLOGO@mbox{#1{p}{P}lain\hologo{TeX}}%
}
%    \end{macrocode}
%    \end{macro}
%    \begin{macro}{\HoLogoCs@plainTeX@runtogether}
%    \begin{macrocode}
\def\HoLogoCs@plainTeX@runtogether#1{#1{p}{P}lainTeX}
%    \end{macrocode}
%    \end{macro}
%    \begin{macro}{\HoLogoBkm@plainTeX@runtogether}
%    \begin{macrocode}
\def\HoLogoBkm@plainTeX@runtogether#1{%
  #1{p}{P}lain\hologo{TeX}%
}
%    \end{macrocode}
%    \end{macro}
%    \begin{macro}{\HoLogoHtml@plainTeX@runtogether}
%    \begin{macrocode}
\def\HoLogoHtml@plainTeX@runtogether#1{%
  #1{p}{P}lain\hologo{TeX}%
}
%    \end{macrocode}
%    \end{macro}
%
%    \begin{macro}{\HoLogo@plainTeX}
%    \begin{macrocode}
\def\HoLogo@plainTeX{\HoLogo@plainTeX@space}
%    \end{macrocode}
%    \end{macro}
%    \begin{macro}{\HoLogoCs@plainTeX}
%    \begin{macrocode}
\def\HoLogoCs@plainTeX{\HoLogoCs@plainTeX@space}
%    \end{macrocode}
%    \end{macro}
%    \begin{macro}{\HoLogoBkm@plainTeX}
%    \begin{macrocode}
\def\HoLogoBkm@plainTeX{\HoLogoBkm@plainTeX@space}
%    \end{macrocode}
%    \end{macro}
%    \begin{macro}{\HoLogoHtml@plainTeX}
%    \begin{macrocode}
\def\HoLogoHtml@plainTeX{\HoLogoHtml@plainTeX@space}
%    \end{macrocode}
%    \end{macro}
%
% \subsubsection{\hologo{LaTeX}}
%
%    Source: \hologo{LaTeX} kernel.
%\begin{quote}
%\begin{verbatim}
%\DeclareRobustCommand{\LaTeX}{%
%  L%
%  \kern-.36em%
%  {%
%    \sbox\z@ T%
%    \vbox to\ht\z@{%
%      \hbox{%
%        \check@mathfonts
%        \fontsize\sf@size\z@
%        \math@fontsfalse
%        \selectfont
%        A%
%      }%
%      \vss
%    }%
%  }%
%  \kern-.15em%
%  \TeX
%}
%\end{verbatim}
%\end{quote}
%
%    \begin{macro}{\HoLogo@La}
%    \begin{macrocode}
\def\HoLogo@La#1{%
  L%
  \kern-.36em%
  \begingroup
    \setbox\ltx@zero\hbox{T}%
    \vbox to\ht\ltx@zero{%
      \hbox{%
        \ltx@ifundefined{check@mathfonts}{%
          \csname sevenrm\endcsname
        }{%
          \check@mathfonts
          \fontsize\sf@size{0pt}%
          \math@fontsfalse\selectfont
        }%
        A%
      }%
      \vss
    }%
  \endgroup
}
%    \end{macrocode}
%    \end{macro}
%
%    \begin{macro}{\HoLogo@LaTeX}
%    Source: \hologo{LaTeX} kernel.
%    \begin{macrocode}
\def\HoLogo@LaTeX#1{%
  \hologo{La}%
  \kern-.15em%
  \hologo{TeX}%
}
%    \end{macrocode}
%    \end{macro}
%    \begin{macro}{\HoLogoHtml@LaTeX}
%    \begin{macrocode}
\def\HoLogoHtml@LaTeX#1{%
  \HoLogoCss@LaTeX
  \HOLOGO@Span{LaTeX}{%
    L%
    \HOLOGO@Span{a}{%
      A%
    }%
    \hologo{TeX}%
  }%
}
%    \end{macrocode}
%    \end{macro}
%    \begin{macro}{\HoLogoCss@LaTeX}
%    \begin{macrocode}
\def\HoLogoCss@LaTeX{%
  \Css{%
    span.HoLogo-LaTeX span.HoLogo-a{%
      position:relative;%
      top:-.5ex;%
      margin-left:-.36em;%
      margin-right:-.15em;%
      font-size:85\%;%
    }%
  }%
  \global\let\HoLogoCss@LaTeX\relax
}
%    \end{macrocode}
%    \end{macro}
%
% \subsubsection{\hologo{(La)TeX}}
%
%    \begin{macro}{\HoLogo@LaTeXTeX}
%    The kerning around the parentheses is taken
%    from package \xpackage{dtklogos} \cite{dtklogos}.
%\begin{quote}
%\begin{verbatim}
%\DeclareRobustCommand{\LaTeXTeX}{%
%  (%
%  \kern-.15em%
%  L%
%  \kern-.36em%
%  {%
%    \sbox\z@ T%
%    \vbox to\ht0{%
%      \hbox{%
%        $\m@th$%
%        \csname S@\f@size\endcsname
%        \fontsize\sf@size\z@
%        \math@fontsfalse
%        \selectfont
%        A%
%      }%
%      \vss
%    }%
%  }%
%  \kern-.2em%
%  )%
%  \kern-.15em%
%  \TeX
%}
%\end{verbatim}
%\end{quote}
%    \begin{macrocode}
\def\HoLogo@LaTeXTeX#1{%
  (%
  \kern-.15em%
  \hologo{La}%
  \kern-.2em%
  )%
  \kern-.15em%
  \hologo{TeX}%
}
%    \end{macrocode}
%    \end{macro}
%    \begin{macro}{\HoLogoBkm@LaTeXTeX}
%    \begin{macrocode}
\def\HoLogoBkm@LaTeXTeX#1{(La)TeX}
%    \end{macrocode}
%    \end{macro}
%
%    \begin{macro}{\HoLogo@(La)TeX}
%    \begin{macrocode}
\expandafter
\let\csname HoLogo@(La)TeX\endcsname\HoLogo@LaTeXTeX
%    \end{macrocode}
%    \end{macro}
%    \begin{macro}{\HoLogoBkm@(La)TeX}
%    \begin{macrocode}
\expandafter
\let\csname HoLogoBkm@(La)TeX\endcsname\HoLogoBkm@LaTeXTeX
%    \end{macrocode}
%    \end{macro}
%    \begin{macro}{\HoLogoHtml@LaTeXTeX}
%    \begin{macrocode}
\def\HoLogoHtml@LaTeXTeX#1{%
  \HoLogoCss@LaTeXTeX
  \HOLOGO@Span{LaTeXTeX}{%
    (%
    \HOLOGO@Span{L}{L}%
    \HOLOGO@Span{a}{A}%
    \HOLOGO@Span{ParenRight}{)}%
    \hologo{TeX}%
  }%
}
%    \end{macrocode}
%    \end{macro}
%    \begin{macro}{\HoLogoHtml@(La)TeX}
%    Kerning after opening parentheses and before closing parentheses
%    is $-0.1$\,em. The original values $-0.15$\,em
%    looked too ugly for a serif font.
%    \begin{macrocode}
\expandafter
\let\csname HoLogoHtml@(La)TeX\endcsname\HoLogoHtml@LaTeXTeX
%    \end{macrocode}
%    \end{macro}
%    \begin{macro}{\HoLogoCss@LaTeXTeX}
%    \begin{macrocode}
\def\HoLogoCss@LaTeXTeX{%
  \Css{%
    span.HoLogo-LaTeXTeX span.HoLogo-L{%
      margin-left:-.1em;%
    }%
  }%
  \Css{%
    span.HoLogo-LaTeXTeX span.HoLogo-a{%
      position:relative;%
      top:-.5ex;%
      margin-left:-.36em;%
      margin-right:-.1em;%
      font-size:85\%;%
    }%
  }%
  \Css{%
    span.HoLogo-LaTeXTeX span.HoLogo-ParenRight{%
      margin-right:-.15em;%
    }%
  }%
  \global\let\HoLogoCss@LaTeXTeX\relax
}
%    \end{macrocode}
%    \end{macro}
%
% \subsubsection{\hologo{LaTeXe}}
%
%    \begin{macro}{\HoLogo@LaTeXe}
%    Source: \hologo{LaTeX} kernel
%    \begin{macrocode}
\def\HoLogo@LaTeXe#1{%
  \hologo{LaTeX}%
  \kern.15em%
  \hbox{%
    \HOLOGO@MathSetup
    2%
    $_{\textstyle\varepsilon}$%
  }%
}
%    \end{macrocode}
%    \end{macro}
%
%    \begin{macro}{\HoLogoCs@LaTeXe}
%    \begin{macrocode}
\ifnum64=`\^^^^0040\relax % test for big chars of LuaTeX/XeTeX
  \catcode`\$=9 %
  \catcode`\&=14 %
\else
  \catcode`\$=14 %
  \catcode`\&=9 %
\fi
\def\HoLogoCs@LaTeXe#1{%
  LaTeX2%
$ \string ^^^^0395%
& e%
}%
\catcode`\$=3 %
\catcode`\&=4 %
%    \end{macrocode}
%    \end{macro}
%
%    \begin{macro}{\HoLogoBkm@LaTeXe}
%    \begin{macrocode}
\def\HoLogoBkm@LaTeXe#1{%
  \hologo{LaTeX}%
  2%
  \HOLOGO@PdfdocUnicode{e}{\textepsilon}%
}
%    \end{macrocode}
%    \end{macro}
%
%    \begin{macro}{\HoLogoHtml@LaTeXe}
%    \begin{macrocode}
\def\HoLogoHtml@LaTeXe#1{%
  \HoLogoCss@LaTeXe
  \HOLOGO@Span{LaTeX2e}{%
    \hologo{LaTeX}%
    \HOLOGO@Span{2}{2}%
    \HOLOGO@Span{e}{%
      \HOLOGO@MathSetup
      \ensuremath{\textstyle\varepsilon}%
    }%
  }%
}
%    \end{macrocode}
%    \end{macro}
%    \begin{macro}{\HoLogoCss@LaTeXe}
%    \begin{macrocode}
\def\HoLogoCss@LaTeXe{%
  \Css{%
    span.HoLogo-LaTeX2e span.HoLogo-2{%
      padding-left:.15em;%
    }%
  }%
  \Css{%
    span.HoLogo-LaTeX2e span.HoLogo-e{%
      position:relative;%
      top:.35ex;%
      text-decoration:none;%
    }%
  }%
  \global\let\HoLogoCss@LaTeXe\relax
}
%    \end{macrocode}
%    \end{macro}
%
%    \begin{macro}{\HoLogo@LaTeX2e}
%    \begin{macrocode}
\expandafter
\let\csname HoLogo@LaTeX2e\endcsname\HoLogo@LaTeXe
%    \end{macrocode}
%    \end{macro}
%    \begin{macro}{\HoLogoCs@LaTeX2e}
%    \begin{macrocode}
\expandafter
\let\csname HoLogoCs@LaTeX2e\endcsname\HoLogoCs@LaTeXe
%    \end{macrocode}
%    \end{macro}
%    \begin{macro}{\HoLogoBkm@LaTeX2e}
%    \begin{macrocode}
\expandafter
\let\csname HoLogoBkm@LaTeX2e\endcsname\HoLogoBkm@LaTeXe
%    \end{macrocode}
%    \end{macro}
%    \begin{macro}{\HoLogoHtml@LaTeX2e}
%    \begin{macrocode}
\expandafter
\let\csname HoLogoHtml@LaTeX2e\endcsname\HoLogoHtml@LaTeXe
%    \end{macrocode}
%    \end{macro}
%
% \subsubsection{\hologo{LaTeX3}}
%
%    \begin{macro}{\HoLogo@LaTeX3}
%    Source: \hologo{LaTeX} kernel
%    \begin{macrocode}
\expandafter\def\csname HoLogo@LaTeX3\endcsname#1{%
  \hologo{LaTeX}%
  3%
}
%    \end{macrocode}
%    \end{macro}
%
%    \begin{macro}{\HoLogoBkm@LaTeX3}
%    \begin{macrocode}
\expandafter\def\csname HoLogoBkm@LaTeX3\endcsname#1{%
  \hologo{LaTeX}%
  3%
}
%    \end{macrocode}
%    \end{macro}
%    \begin{macro}{\HoLogoHtml@LaTeX3}
%    \begin{macrocode}
\expandafter
\let\csname HoLogoHtml@LaTeX3\expandafter\endcsname
\csname HoLogo@LaTeX3\endcsname
%    \end{macrocode}
%    \end{macro}
%
% \subsubsection{\hologo{LaTeXML}}
%
%    \begin{macro}{\HoLogo@LaTeXML}
%    \begin{macrocode}
\def\HoLogo@LaTeXML#1{%
  \HOLOGO@mbox{%
    \hologo{La}%
    \kern-.15em%
    T%
    \kern-.1667em%
    \lower.5ex\hbox{E}%
    \kern-.125em%
    \HoLogoFont@font{LaTeXML}{sc}{xml}%
  }%
}
%    \end{macrocode}
%    \end{macro}
%    \begin{macro}{\HoLogoHtml@pdfLaTeX}
%    \begin{macrocode}
\def\HoLogoHtml@LaTeXML#1{%
  \HOLOGO@Span{LaTeXML}{%
    \HoLogoCss@LaTeX
    \HoLogoCss@TeX
    \HOLOGO@Span{LaTeX}{%
      L%
      \HOLOGO@Span{a}{%
        A%
      }%
    }%
    \HOLOGO@Span{TeX}{%
      T%
      \HOLOGO@Span{e}{%
        E%
      }%
    }%
    \HCode{<span style="font-variant: small-caps;">}%
    xml%
    \HCode{</span>}%
  }%
}
%    \end{macrocode}
%    \end{macro}
%
% \subsubsection{\hologo{eTeX}}
%
%    \begin{macro}{\HoLogo@eTeX}
%    Source: package \xpackage{etex}
%    \begin{macrocode}
\def\HoLogo@eTeX#1{%
  \ltx@mbox{%
    \HOLOGO@MathSetup
    $\varepsilon$%
    -%
    \HOLOGO@NegativeKerning{-T,T-,To}%
    \hologo{TeX}%
  }%
}
%    \end{macrocode}
%    \end{macro}
%    \begin{macro}{\HoLogoCs@eTeX}
%    \begin{macrocode}
\ifnum64=`\^^^^0040\relax % test for big chars of LuaTeX/XeTeX
  \catcode`\$=9 %
  \catcode`\&=14 %
\else
  \catcode`\$=14 %
  \catcode`\&=9 %
\fi
\def\HoLogoCs@eTeX#1{%
$ #1{\string ^^^^0395}{\string ^^^^03b5}%
& #1{e}{E}%
  TeX%
}%
\catcode`\$=3 %
\catcode`\&=4 %
%    \end{macrocode}
%    \end{macro}
%    \begin{macro}{\HoLogoBkm@eTeX}
%    \begin{macrocode}
\def\HoLogoBkm@eTeX#1{%
  \HOLOGO@PdfdocUnicode{#1{e}{E}}{\textepsilon}%
  -%
  \hologo{TeX}%
}
%    \end{macrocode}
%    \end{macro}
%    \begin{macro}{\HoLogoHtml@eTeX}
%    \begin{macrocode}
\def\HoLogoHtml@eTeX#1{%
  \ltx@mbox{%
    \HOLOGO@MathSetup
    $\varepsilon$%
    -%
    \hologo{TeX}%
  }%
}
%    \end{macrocode}
%    \end{macro}
%
% \subsubsection{\hologo{iniTeX}}
%
%    \begin{macro}{\HoLogo@iniTeX}
%    \begin{macrocode}
\def\HoLogo@iniTeX#1{%
  \HOLOGO@mbox{%
    #1{i}{I}ni\hologo{TeX}%
  }%
}
%    \end{macrocode}
%    \end{macro}
%    \begin{macro}{\HoLogoCs@iniTeX}
%    \begin{macrocode}
\def\HoLogoCs@iniTeX#1{#1{i}{I}niTeX}
%    \end{macrocode}
%    \end{macro}
%    \begin{macro}{\HoLogoBkm@iniTeX}
%    \begin{macrocode}
\def\HoLogoBkm@iniTeX#1{%
  #1{i}{I}ni\hologo{TeX}%
}
%    \end{macrocode}
%    \end{macro}
%    \begin{macro}{\HoLogoHtml@iniTeX}
%    \begin{macrocode}
\let\HoLogoHtml@iniTeX\HoLogo@iniTeX
%    \end{macrocode}
%    \end{macro}
%
% \subsubsection{\hologo{virTeX}}
%
%    \begin{macro}{\HoLogo@virTeX}
%    \begin{macrocode}
\def\HoLogo@virTeX#1{%
  \HOLOGO@mbox{%
    #1{v}{V}ir\hologo{TeX}%
  }%
}
%    \end{macrocode}
%    \end{macro}
%    \begin{macro}{\HoLogoCs@virTeX}
%    \begin{macrocode}
\def\HoLogoCs@virTeX#1{#1{v}{V}irTeX}
%    \end{macrocode}
%    \end{macro}
%    \begin{macro}{\HoLogoBkm@virTeX}
%    \begin{macrocode}
\def\HoLogoBkm@virTeX#1{%
  #1{v}{V}ir\hologo{TeX}%
}
%    \end{macrocode}
%    \end{macro}
%    \begin{macro}{\HoLogoHtml@virTeX}
%    \begin{macrocode}
\let\HoLogoHtml@virTeX\HoLogo@virTeX
%    \end{macrocode}
%    \end{macro}
%
% \subsubsection{\hologo{SliTeX}}
%
% \paragraph{Definitions of the three variants.}
%
%    \begin{macro}{\HoLogo@SLiTeX@lift}
%    \begin{macrocode}
\def\HoLogo@SLiTeX@lift#1{%
  \HoLogoFont@font{SliTeX}{rm}{%
    S%
    \kern-.06em%
    L%
    \kern-.18em%
    \raise.32ex\hbox{\HoLogoFont@font{SliTeX}{sc}{i}}%
    \HOLOGO@discretionary
    \kern-.06em%
    \hologo{TeX}%
  }%
}
%    \end{macrocode}
%    \end{macro}
%    \begin{macro}{\HoLogoBkm@SLiTeX@lift}
%    \begin{macrocode}
\def\HoLogoBkm@SLiTeX@lift#1{SLiTeX}
%    \end{macrocode}
%    \end{macro}
%    \begin{macro}{\HoLogoHtml@SLiTeX@lift}
%    \begin{macrocode}
\def\HoLogoHtml@SLiTeX@lift#1{%
  \HoLogoCss@SLiTeX@lift
  \HOLOGO@Span{SLiTeX-lift}{%
    \HoLogoFont@font{SliTeX}{rm}{%
      S%
      \HOLOGO@Span{L}{L}%
      \HOLOGO@Span{i}{i}%
      \hologo{TeX}%
    }%
  }%
}
%    \end{macrocode}
%    \end{macro}
%    \begin{macro}{\HoLogoCss@SLiTeX@lift}
%    \begin{macrocode}
\def\HoLogoCss@SLiTeX@lift{%
  \Css{%
    span.HoLogo-SLiTeX-lift span.HoLogo-L{%
      margin-left:-.06em;%
      margin-right:-.18em;%
    }%
  }%
  \Css{%
    span.HoLogo-SLiTeX-lift span.HoLogo-i{%
      position:relative;%
      top:-.32ex;%
      margin-right:-.06em;%
      font-variant:small-caps;%
    }%
  }%
  \global\let\HoLogoCss@SLiTeX@lift\relax
}
%    \end{macrocode}
%    \end{macro}
%
%    \begin{macro}{\HoLogo@SliTeX@simple}
%    \begin{macrocode}
\def\HoLogo@SliTeX@simple#1{%
  \HoLogoFont@font{SliTeX}{rm}{%
    \ltx@mbox{%
      \HoLogoFont@font{SliTeX}{sc}{Sli}%
    }%
    \HOLOGO@discretionary
    \hologo{TeX}%
  }%
}
%    \end{macrocode}
%    \end{macro}
%    \begin{macro}{\HoLogoBkm@SliTeX@simple}
%    \begin{macrocode}
\def\HoLogoBkm@SliTeX@simple#1{SliTeX}
%    \end{macrocode}
%    \end{macro}
%    \begin{macro}{\HoLogoHtml@SliTeX@simple}
%    \begin{macrocode}
\let\HoLogoHtml@SliTeX@simple\HoLogo@SliTeX@simple
%    \end{macrocode}
%    \end{macro}
%
%    \begin{macro}{\HoLogo@SliTeX@narrow}
%    \begin{macrocode}
\def\HoLogo@SliTeX@narrow#1{%
  \HoLogoFont@font{SliTeX}{rm}{%
    \ltx@mbox{%
      S%
      \kern-.06em%
      \HoLogoFont@font{SliTeX}{sc}{%
        l%
        \kern-.035em%
        i%
      }%
    }%
    \HOLOGO@discretionary
    \kern-.06em%
    \hologo{TeX}%
  }%
}
%    \end{macrocode}
%    \end{macro}
%    \begin{macro}{\HoLogoBkm@SliTeX@narrow}
%    \begin{macrocode}
\def\HoLogoBkm@SliTeX@narrow#1{SliTeX}
%    \end{macrocode}
%    \end{macro}
%    \begin{macro}{\HoLogoHtml@SliTeX@narrow}
%    \begin{macrocode}
\def\HoLogoHtml@SliTeX@narrow#1{%
  \HoLogoCss@SliTeX@narrow
  \HOLOGO@Span{SliTeX-narrow}{%
    \HoLogoFont@font{SliTeX}{rm}{%
      S%
        \HOLOGO@Span{l}{l}%
        \HOLOGO@Span{i}{i}%
      \hologo{TeX}%
    }%
  }%
}
%    \end{macrocode}
%    \end{macro}
%    \begin{macro}{\HoLogoCss@SliTeX@narrow}
%    \begin{macrocode}
\def\HoLogoCss@SliTeX@narrow{%
  \Css{%
    span.HoLogo-SliTeX-narrow span.HoLogo-l{%
      margin-left:-.06em;%
      margin-right:-.035em;%
      font-variant:small-caps;%
    }%
  }%
  \Css{%
    span.HoLogo-SliTeX-narrow span.HoLogo-i{%
      margin-right:-.06em;%
      font-variant:small-caps;%
    }%
  }%
  \global\let\HoLogoCss@SliTeX@narrow\relax
}
%    \end{macrocode}
%    \end{macro}
%
% \paragraph{Macro set completion.}
%
%    \begin{macro}{\HoLogo@SLiTeX@simple}
%    \begin{macrocode}
\def\HoLogo@SLiTeX@simple{\HoLogo@SliTeX@simple}
%    \end{macrocode}
%    \end{macro}
%    \begin{macro}{\HoLogoBkm@SLiTeX@simple}
%    \begin{macrocode}
\def\HoLogoBkm@SLiTeX@simple{\HoLogoBkm@SliTeX@simple}
%    \end{macrocode}
%    \end{macro}
%    \begin{macro}{\HoLogoHtml@SLiTeX@simple}
%    \begin{macrocode}
\def\HoLogoHtml@SLiTeX@simple{\HoLogoHtml@SliTeX@simple}
%    \end{macrocode}
%    \end{macro}
%
%    \begin{macro}{\HoLogo@SLiTeX@narrow}
%    \begin{macrocode}
\def\HoLogo@SLiTeX@narrow{\HoLogo@SliTeX@narrow}
%    \end{macrocode}
%    \end{macro}
%    \begin{macro}{\HoLogoBkm@SLiTeX@narrow}
%    \begin{macrocode}
\def\HoLogoBkm@SLiTeX@narrow{\HoLogoBkm@SliTeX@narrow}
%    \end{macrocode}
%    \end{macro}
%    \begin{macro}{\HoLogoHtml@SLiTeX@narrow}
%    \begin{macrocode}
\def\HoLogoHtml@SLiTeX@narrow{\HoLogoHtml@SliTeX@narrow}
%    \end{macrocode}
%    \end{macro}
%
%    \begin{macro}{\HoLogo@SliTeX@lift}
%    \begin{macrocode}
\def\HoLogo@SliTeX@lift{\HoLogo@SLiTeX@lift}
%    \end{macrocode}
%    \end{macro}
%    \begin{macro}{\HoLogoBkm@SliTeX@lift}
%    \begin{macrocode}
\def\HoLogoBkm@SliTeX@lift{\HoLogoBkm@SLiTeX@lift}
%    \end{macrocode}
%    \end{macro}
%    \begin{macro}{\HoLogoHtml@SliTeX@lift}
%    \begin{macrocode}
\def\HoLogoHtml@SliTeX@lift{\HoLogoHtml@SLiTeX@lift}
%    \end{macrocode}
%    \end{macro}
%
% \paragraph{Defaults.}
%
%    \begin{macro}{\HoLogo@SLiTeX}
%    \begin{macrocode}
\def\HoLogo@SLiTeX{\HoLogo@SLiTeX@lift}
%    \end{macrocode}
%    \end{macro}
%    \begin{macro}{\HoLogoBkm@SLiTeX}
%    \begin{macrocode}
\def\HoLogoBkm@SLiTeX{\HoLogoBkm@SLiTeX@lift}
%    \end{macrocode}
%    \end{macro}
%    \begin{macro}{\HoLogoHtml@SLiTeX}
%    \begin{macrocode}
\def\HoLogoHtml@SLiTeX{\HoLogoHtml@SLiTeX@lift}
%    \end{macrocode}
%    \end{macro}
%
%    \begin{macro}{\HoLogo@SliTeX}
%    \begin{macrocode}
\def\HoLogo@SliTeX{\HoLogo@SliTeX@narrow}
%    \end{macrocode}
%    \end{macro}
%    \begin{macro}{\HoLogoBkm@SliTeX}
%    \begin{macrocode}
\def\HoLogoBkm@SliTeX{\HoLogoBkm@SliTeX@narrow}
%    \end{macrocode}
%    \end{macro}
%    \begin{macro}{\HoLogoHtml@SliTeX}
%    \begin{macrocode}
\def\HoLogoHtml@SliTeX{\HoLogoHtml@SliTeX@narrow}
%    \end{macrocode}
%    \end{macro}
%
% \subsubsection{\hologo{LuaTeX}}
%
%    \begin{macro}{\HoLogo@LuaTeX}
%    The kerning is an idea of Hans Hagen, see mailing list
%    `luatex at tug dot org' in March 2010.
%    \begin{macrocode}
\def\HoLogo@LuaTeX#1{%
  \HOLOGO@mbox{%
    Lua%
    \HOLOGO@NegativeKerning{aT,oT,To}%
    \hologo{TeX}%
  }%
}
%    \end{macrocode}
%    \end{macro}
%    \begin{macro}{\HoLogoHtml@LuaTeX}
%    \begin{macrocode}
\let\HoLogoHtml@LuaTeX\HoLogo@LuaTeX
%    \end{macrocode}
%    \end{macro}
%
% \subsubsection{\hologo{LuaLaTeX}}
%
%    \begin{macro}{\HoLogo@LuaLaTeX}
%    \begin{macrocode}
\def\HoLogo@LuaLaTeX#1{%
  \HOLOGO@mbox{%
    Lua%
    \hologo{LaTeX}%
  }%
}
%    \end{macrocode}
%    \end{macro}
%    \begin{macro}{\HoLogoHtml@LuaLaTeX}
%    \begin{macrocode}
\let\HoLogoHtml@LuaLaTeX\HoLogo@LuaLaTeX
%    \end{macrocode}
%    \end{macro}
%
% \subsubsection{\hologo{XeTeX}, \hologo{XeLaTeX}}
%
%    \begin{macro}{\HOLOGO@IfCharExists}
%    \begin{macrocode}
\ifluatex
  \ifnum\luatexversion<36 %
  \else
    \def\HOLOGO@IfCharExists#1{%
      \ifnum
        \directlua{%
           if luaotfload and luaotfload.aux then
             if luaotfload.aux.font_has_glyph(%
                    font.current(), \number#1) then % 	 
	       tex.print("1") % 	 
	     end % 	 
	   elseif font and font.fonts and font.current then %
            local f = font.fonts[font.current()]%
            if f.characters and f.characters[\number#1] then %
              tex.print("1")%
            end %
          end%
        }0=\ltx@zero
        \expandafter\ltx@secondoftwo
      \else
        \expandafter\ltx@firstoftwo
      \fi
    }%
  \fi
\fi
\ltx@IfUndefined{HOLOGO@IfCharExists}{%
  \def\HOLOGO@@IfCharExists#1{%
    \begingroup
      \tracinglostchars=\ltx@zero
      \setbox\ltx@zero=\hbox{%
        \kern7sp\char#1\relax
        \ifnum\lastkern>\ltx@zero
          \expandafter\aftergroup\csname iffalse\endcsname
        \else
          \expandafter\aftergroup\csname iftrue\endcsname
        \fi
      }%
      % \if{true|false} from \aftergroup
      \endgroup
      \expandafter\ltx@firstoftwo
    \else
      \endgroup
      \expandafter\ltx@secondoftwo
    \fi
  }%
  \ifxetex
    \ltx@IfUndefined{XeTeXfonttype}{}{%
      \ltx@IfUndefined{XeTeXcharglyph}{}{%
        \def\HOLOGO@IfCharExists#1{%
          \ifnum\XeTeXfonttype\font>\ltx@zero
            \expandafter\ltx@firstofthree
          \else
            \expandafter\ltx@gobble
          \fi
          {%
            \ifnum\XeTeXcharglyph#1>\ltx@zero
              \expandafter\ltx@firstoftwo
            \else
              \expandafter\ltx@secondoftwo
            \fi
          }%
          \HOLOGO@@IfCharExists{#1}%
        }%
      }%
    }%
  \fi
}{}
\ltx@ifundefined{HOLOGO@IfCharExists}{%
  \ifnum64=`\^^^^0040\relax % test for big chars of LuaTeX/XeTeX
    \let\HOLOGO@IfCharExists\HOLOGO@@IfCharExists
  \else
    \def\HOLOGO@IfCharExists#1{%
      \ifnum#1>255 %
        \expandafter\ltx@fourthoffour
      \fi
      \HOLOGO@@IfCharExists{#1}%
    }%
  \fi
}{}
%    \end{macrocode}
%    \end{macro}
%
%    \begin{macro}{\HoLogo@Xe}
%    Source: package \xpackage{dtklogos}
%    \begin{macrocode}
\def\HoLogo@Xe#1{%
  X%
  \kern-.1em\relax
  \HOLOGO@IfCharExists{"018E}{%
    \lower.5ex\hbox{\char"018E}%
  }{%
    \chardef\HOLOGO@choice=\ltx@zero
    \ifdim\fontdimen\ltx@one\font>0pt %
      \ltx@IfUndefined{rotatebox}{%
        \ltx@IfUndefined{pgftext}{%
          \ltx@IfUndefined{psscalebox}{%
            \ltx@IfUndefined{HOLOGO@ScaleBox@\hologoDriver}{%
            }{%
              \chardef\HOLOGO@choice=4 %
            }%
          }{%
            \chardef\HOLOGO@choice=3 %
          }%
        }{%
          \chardef\HOLOGO@choice=2 %
        }%
      }{%
        \chardef\HOLOGO@choice=1 %
      }%
      \ifcase\HOLOGO@choice
        \HOLOGO@WarningUnsupportedDriver{Xe}%
        e%
      \or % 1: \rotatebox
        \begingroup
          \setbox\ltx@zero\hbox{\rotatebox{180}{E}}%
          \ltx@LocDimenA=\dp\ltx@zero
          \advance\ltx@LocDimenA by -.5ex\relax
          \raise\ltx@LocDimenA\box\ltx@zero
        \endgroup
      \or % 2: \pgftext
        \lower.5ex\hbox{%
          \pgfpicture
            \pgftext[rotate=180]{E}%
          \endpgfpicture
        }%
      \or % 3: \psscalebox
        \begingroup
          \setbox\ltx@zero\hbox{\psscalebox{-1 -1}{E}}%
          \ltx@LocDimenA=\dp\ltx@zero
          \advance\ltx@LocDimenA by -.5ex\relax
          \raise\ltx@LocDimenA\box\ltx@zero
        \endgroup
      \or % 4: \HOLOGO@PointReflectBox
        \lower.5ex\hbox{\HOLOGO@PointReflectBox{E}}%
      \else
        \@PackageError{hologo}{Internal error (choice/it}\@ehc
      \fi
    \else
      \ltx@IfUndefined{reflectbox}{%
        \ltx@IfUndefined{pgftext}{%
          \ltx@IfUndefined{psscalebox}{%
            \ltx@IfUndefined{HOLOGO@ScaleBox@\hologoDriver}{%
            }{%
              \chardef\HOLOGO@choice=4 %
            }%
          }{%
            \chardef\HOLOGO@choice=3 %
          }%
        }{%
          \chardef\HOLOGO@choice=2 %
        }%
      }{%
        \chardef\HOLOGO@choice=1 %
      }%
      \ifcase\HOLOGO@choice
        \HOLOGO@WarningUnsupportedDriver{Xe}%
        e%
      \or % 1: reflectbox
        \lower.5ex\hbox{%
          \reflectbox{E}%
        }%
      \or % 2: \pgftext
        \lower.5ex\hbox{%
          \pgfpicture
            \pgftransformxscale{-1}%
            \pgftext{E}%
          \endpgfpicture
        }%
      \or % 3: \psscalebox
        \lower.5ex\hbox{%
          \psscalebox{-1 1}{E}%
        }%
      \or % 4: \HOLOGO@Reflectbox
        \lower.5ex\hbox{%
          \HOLOGO@ReflectBox{E}%
        }%
      \else
        \@PackageError{hologo}{Internal error (choice/up)}\@ehc
      \fi
    \fi
  }%
}
%    \end{macrocode}
%    \end{macro}
%    \begin{macro}{\HoLogoHtml@Xe}
%    \begin{macrocode}
\def\HoLogoHtml@Xe#1{%
  \HoLogoCss@Xe
  \HOLOGO@Span{Xe}{%
    X%
    \HOLOGO@Span{e}{%
      \HCode{&\ltx@hashchar x018e;}%
    }%
  }%
}
%    \end{macrocode}
%    \end{macro}
%    \begin{macro}{\HoLogoCss@Xe}
%    \begin{macrocode}
\def\HoLogoCss@Xe{%
  \Css{%
    span.HoLogo-Xe span.HoLogo-e{%
      position:relative;%
      top:.5ex;%
      left-margin:-.1em;%
    }%
  }%
  \global\let\HoLogoCss@Xe\relax
}
%    \end{macrocode}
%    \end{macro}
%
%    \begin{macro}{\HoLogo@XeTeX}
%    \begin{macrocode}
\def\HoLogo@XeTeX#1{%
  \hologo{Xe}%
  \kern-.15em\relax
  \hologo{TeX}%
}
%    \end{macrocode}
%    \end{macro}
%
%    \begin{macro}{\HoLogoHtml@XeTeX}
%    \begin{macrocode}
\def\HoLogoHtml@XeTeX#1{%
  \HoLogoCss@XeTeX
  \HOLOGO@Span{XeTeX}{%
    \hologo{Xe}%
    \hologo{TeX}%
  }%
}
%    \end{macrocode}
%    \end{macro}
%    \begin{macro}{\HoLogoCss@XeTeX}
%    \begin{macrocode}
\def\HoLogoCss@XeTeX{%
  \Css{%
    span.HoLogo-XeTeX span.HoLogo-TeX{%
      margin-left:-.15em;%
    }%
  }%
  \global\let\HoLogoCss@XeTeX\relax
}
%    \end{macrocode}
%    \end{macro}
%
%    \begin{macro}{\HoLogo@XeLaTeX}
%    \begin{macrocode}
\def\HoLogo@XeLaTeX#1{%
  \hologo{Xe}%
  \kern-.13em%
  \hologo{LaTeX}%
}
%    \end{macrocode}
%    \end{macro}
%    \begin{macro}{\HoLogoHtml@XeLaTeX}
%    \begin{macrocode}
\def\HoLogoHtml@XeLaTeX#1{%
  \HoLogoCss@XeLaTeX
  \HOLOGO@Span{XeLaTeX}{%
    \hologo{Xe}%
    \hologo{LaTeX}%
  }%
}
%    \end{macrocode}
%    \end{macro}
%    \begin{macro}{\HoLogoCss@XeLaTeX}
%    \begin{macrocode}
\def\HoLogoCss@XeLaTeX{%
  \Css{%
    span.HoLogo-XeLaTeX span.HoLogo-Xe{%
      margin-right:-.13em;%
    }%
  }%
  \global\let\HoLogoCss@XeLaTeX\relax
}
%    \end{macrocode}
%    \end{macro}
%
% \subsubsection{\hologo{pdfTeX}, \hologo{pdfLaTeX}}
%
%    \begin{macro}{\HoLogo@pdfTeX}
%    \begin{macrocode}
\def\HoLogo@pdfTeX#1{%
  \HOLOGO@mbox{%
    #1{p}{P}df\hologo{TeX}%
  }%
}
%    \end{macrocode}
%    \end{macro}
%    \begin{macro}{\HoLogoCs@pdfTeX}
%    \begin{macrocode}
\def\HoLogoCs@pdfTeX#1{#1{p}{P}dfTeX}
%    \end{macrocode}
%    \end{macro}
%    \begin{macro}{\HoLogoBkm@pdfTeX}
%    \begin{macrocode}
\def\HoLogoBkm@pdfTeX#1{%
  #1{p}{P}df\hologo{TeX}%
}
%    \end{macrocode}
%    \end{macro}
%    \begin{macro}{\HoLogoHtml@pdfTeX}
%    \begin{macrocode}
\let\HoLogoHtml@pdfTeX\HoLogo@pdfTeX
%    \end{macrocode}
%    \end{macro}
%
%    \begin{macro}{\HoLogo@pdfLaTeX}
%    \begin{macrocode}
\def\HoLogo@pdfLaTeX#1{%
  \HOLOGO@mbox{%
    #1{p}{P}df\hologo{LaTeX}%
  }%
}
%    \end{macrocode}
%    \end{macro}
%    \begin{macro}{\HoLogoCs@pdfLaTeX}
%    \begin{macrocode}
\def\HoLogoCs@pdfLaTeX#1{#1{p}{P}dfLaTeX}
%    \end{macrocode}
%    \end{macro}
%    \begin{macro}{\HoLogoBkm@pdfLaTeX}
%    \begin{macrocode}
\def\HoLogoBkm@pdfLaTeX#1{%
  #1{p}{P}df\hologo{LaTeX}%
}
%    \end{macrocode}
%    \end{macro}
%    \begin{macro}{\HoLogoHtml@pdfLaTeX}
%    \begin{macrocode}
\let\HoLogoHtml@pdfLaTeX\HoLogo@pdfLaTeX
%    \end{macrocode}
%    \end{macro}
%
% \subsubsection{\hologo{VTeX}}
%
%    \begin{macro}{\HoLogo@VTeX}
%    \begin{macrocode}
\def\HoLogo@VTeX#1{%
  \HOLOGO@mbox{%
    V\hologo{TeX}%
  }%
}
%    \end{macrocode}
%    \end{macro}
%    \begin{macro}{\HoLogoHtml@VTeX}
%    \begin{macrocode}
\let\HoLogoHtml@VTeX\HoLogo@VTeX
%    \end{macrocode}
%    \end{macro}
%
% \subsubsection{\hologo{AmS}, \dots}
%
%    Source: class \xclass{amsdtx}
%
%    \begin{macro}{\HoLogo@AmS}
%    \begin{macrocode}
\def\HoLogo@AmS#1{%
  \HoLogoFont@font{AmS}{sy}{%
    A%
    \kern-.1667em%
    \lower.5ex\hbox{M}%
    \kern-.125em%
    S%
  }%
}
%    \end{macrocode}
%    \end{macro}
%    \begin{macro}{\HoLogoBkm@AmS}
%    \begin{macrocode}
\def\HoLogoBkm@AmS#1{AmS}
%    \end{macrocode}
%    \end{macro}
%    \begin{macro}{\HoLogoHtml@AmS}
%    \begin{macrocode}
\def\HoLogoHtml@AmS#1{%
  \HoLogoCss@AmS
%  \HoLogoFont@font{AmS}{sy}{%
    \HOLOGO@Span{AmS}{%
      A%
      \HOLOGO@Span{M}{M}%
      S%
    }%
%   }%
}
%    \end{macrocode}
%    \end{macro}
%    \begin{macro}{\HoLogoCss@AmS}
%    \begin{macrocode}
\def\HoLogoCss@AmS{%
  \Css{%
    span.HoLogo-AmS span.HoLogo-M{%
      position:relative;%
      top:.5ex;%
      margin-left:-.1667em;%
      margin-right:-.125em;%
      text-decoration:none;%
    }%
  }%
  \global\let\HoLogoCss@AmS\relax
}
%    \end{macrocode}
%    \end{macro}
%
%    \begin{macro}{\HoLogo@AmSTeX}
%    \begin{macrocode}
\def\HoLogo@AmSTeX#1{%
  \hologo{AmS}%
  \HOLOGO@hyphen
  \hologo{TeX}%
}
%    \end{macrocode}
%    \end{macro}
%    \begin{macro}{\HoLogoBkm@AmSTeX}
%    \begin{macrocode}
\def\HoLogoBkm@AmSTeX#1{AmS-TeX}%
%    \end{macrocode}
%    \end{macro}
%    \begin{macro}{\HoLogoHtml@AmSTeX}
%    \begin{macrocode}
\let\HoLogoHtml@AmSTeX\HoLogo@AmSTeX
%    \end{macrocode}
%    \end{macro}
%
%    \begin{macro}{\HoLogo@AmSLaTeX}
%    \begin{macrocode}
\def\HoLogo@AmSLaTeX#1{%
  \hologo{AmS}%
  \HOLOGO@hyphen
  \hologo{LaTeX}%
}
%    \end{macrocode}
%    \end{macro}
%    \begin{macro}{\HoLogoBkm@AmSLaTeX}
%    \begin{macrocode}
\def\HoLogoBkm@AmSLaTeX#1{AmS-LaTeX}%
%    \end{macrocode}
%    \end{macro}
%    \begin{macro}{\HoLogoHtml@AmSLaTeX}
%    \begin{macrocode}
\let\HoLogoHtml@AmSLaTeX\HoLogo@AmSLaTeX
%    \end{macrocode}
%    \end{macro}
%
% \subsubsection{\hologo{BibTeX}}
%
%    \begin{macro}{\HoLogo@BibTeX@sc}
%    A definition of \hologo{BibTeX} is provided in
%    the documentation source for the manual of \hologo{BibTeX}
%    \cite{btxdoc}.
%\begin{quote}
%\begin{verbatim}
%\def\BibTeX{%
%  {%
%    \rm
%    B%
%    \kern-.05em%
%    {%
%      \sc
%      i%
%      \kern-.025em %
%      b%
%    }%
%    \kern-.08em
%    T%
%    \kern-.1667em%
%    \lower.7ex\hbox{E}%
%    \kern-.125em%
%    X%
%  }%
%}
%\end{verbatim}
%\end{quote}
%    \begin{macrocode}
\def\HoLogo@BibTeX@sc#1{%
  B%
  \kern-.05em%
  \HoLogoFont@font{BibTeX}{sc}{%
    i%
    \kern-.025em%
    b%
  }%
  \HOLOGO@discretionary
  \kern-.08em%
  \hologo{TeX}%
}
%    \end{macrocode}
%    \end{macro}
%    \begin{macro}{\HoLogoHtml@BibTeX@sc}
%    \begin{macrocode}
\def\HoLogoHtml@BibTeX@sc#1{%
  \HoLogoCss@BibTeX@sc
  \HOLOGO@Span{BibTeX-sc}{%
    B%
    \HOLOGO@Span{i}{i}%
    \HOLOGO@Span{b}{b}%
    \hologo{TeX}%
  }%
}
%    \end{macrocode}
%    \end{macro}
%    \begin{macro}{\HoLogoCss@BibTeX@sc}
%    \begin{macrocode}
\def\HoLogoCss@BibTeX@sc{%
  \Css{%
    span.HoLogo-BibTeX-sc span.HoLogo-i{%
      margin-left:-.05em;%
      margin-right:-.025em;%
      font-variant:small-caps;%
    }%
  }%
  \Css{%
    span.HoLogo-BibTeX-sc span.HoLogo-b{%
      margin-right:-.08em;%
      font-variant:small-caps;%
    }%
  }%
  \global\let\HoLogoCss@BibTeX@sc\relax
}
%    \end{macrocode}
%    \end{macro}
%
%    \begin{macro}{\HoLogo@BibTeX@sf}
%    Variant \xoption{sf} avoids trouble with unavailable
%    small caps fonts (e.g., bold versions of Computer Modern or
%    Latin Modern). The definition is taken from
%    package \xpackage{dtklogos} \cite{dtklogos}.
%\begin{quote}
%\begin{verbatim}
%\DeclareRobustCommand{\BibTeX}{%
%  B%
%  \kern-.05em%
%  \hbox{%
%    $\m@th$% %% force math size calculations
%    \csname S@\f@size\endcsname
%    \fontsize\sf@size\z@
%    \math@fontsfalse
%    \selectfont
%    I%
%    \kern-.025em%
%    B
%  }%
%  \kern-.08em%
%  \-%
%  \TeX
%}
%\end{verbatim}
%\end{quote}
%    \begin{macrocode}
\def\HoLogo@BibTeX@sf#1{%
  B%
  \kern-.05em%
  \HoLogoFont@font{BibTeX}{bibsf}{%
    I%
    \kern-.025em%
    B%
  }%
  \HOLOGO@discretionary
  \kern-.08em%
  \hologo{TeX}%
}
%    \end{macrocode}
%    \end{macro}
%    \begin{macro}{\HoLogoHtml@BibTeX@sf}
%    \begin{macrocode}
\def\HoLogoHtml@BibTeX@sf#1{%
  \HoLogoCss@BibTeX@sf
  \HOLOGO@Span{BibTeX-sf}{%
    B%
    \HoLogoFont@font{BibTeX}{bibsf}{%
      \HOLOGO@Span{i}{I}%
      B%
    }%
    \hologo{TeX}%
  }%
}
%    \end{macrocode}
%    \end{macro}
%    \begin{macro}{\HoLogoCss@BibTeX@sf}
%    \begin{macrocode}
\def\HoLogoCss@BibTeX@sf{%
  \Css{%
    span.HoLogo-BibTeX-sf span.HoLogo-i{%
      margin-left:-.05em;%
      margin-right:-.025em;%
    }%
  }%
  \Css{%
    span.HoLogo-BibTeX-sf span.HoLogo-TeX{%
      margin-left:-.08em;%
    }%
  }%
  \global\let\HoLogoCss@BibTeX@sf\relax
}
%    \end{macrocode}
%    \end{macro}
%
%    \begin{macro}{\HoLogo@BibTeX}
%    \begin{macrocode}
\def\HoLogo@BibTeX{\HoLogo@BibTeX@sf}
%    \end{macrocode}
%    \end{macro}
%    \begin{macro}{\HoLogoHtml@BibTeX}
%    \begin{macrocode}
\def\HoLogoHtml@BibTeX{\HoLogoHtml@BibTeX@sf}
%    \end{macrocode}
%    \end{macro}
%
% \subsubsection{\hologo{BibTeX8}}
%
%    \begin{macro}{\HoLogo@BibTeX8}
%    \begin{macrocode}
\expandafter\def\csname HoLogo@BibTeX8\endcsname#1{%
  \hologo{BibTeX}%
  8%
}
%    \end{macrocode}
%    \end{macro}
%
%    \begin{macro}{\HoLogoBkm@BibTeX8}
%    \begin{macrocode}
\expandafter\def\csname HoLogoBkm@BibTeX8\endcsname#1{%
  \hologo{BibTeX}%
  8%
}
%    \end{macrocode}
%    \end{macro}
%    \begin{macro}{\HoLogoHtml@BibTeX8}
%    \begin{macrocode}
\expandafter
\let\csname HoLogoHtml@BibTeX8\expandafter\endcsname
\csname HoLogo@BibTeX8\endcsname
%    \end{macrocode}
%    \end{macro}
%
% \subsubsection{\hologo{ConTeXt}}
%
%    \begin{macro}{\HoLogo@ConTeXt@simple}
%    \begin{macrocode}
\def\HoLogo@ConTeXt@simple#1{%
  \HOLOGO@mbox{Con}%
  \HOLOGO@discretionary
  \HOLOGO@mbox{\hologo{TeX}t}%
}
%    \end{macrocode}
%    \end{macro}
%    \begin{macro}{\HoLogoHtml@ConTeXt@simple}
%    \begin{macrocode}
\let\HoLogoHtml@ConTeXt@simple\HoLogo@ConTeXt@simple
%    \end{macrocode}
%    \end{macro}
%
%    \begin{macro}{\HoLogo@ConTeXt@narrow}
%    This definition of logo \hologo{ConTeXt} with variant \xoption{narrow}
%    comes from TUGboat's class \xclass{ltugboat} (version 2010/11/15 v2.8).
%    \begin{macrocode}
\def\HoLogo@ConTeXt@narrow#1{%
  \HOLOGO@mbox{C\kern-.0333emon}%
  \HOLOGO@discretionary
  \kern-.0667em%
  \HOLOGO@mbox{\hologo{TeX}\kern-.0333emt}%
}
%    \end{macrocode}
%    \end{macro}
%    \begin{macro}{\HoLogoHtml@ConTeXt@narrow}
%    \begin{macrocode}
\def\HoLogoHtml@ConTeXt@narrow#1{%
  \HoLogoCss@ConTeXt@narrow
  \HOLOGO@Span{ConTeXt-narrow}{%
    \HOLOGO@Span{C}{C}%
    on%
    \hologo{TeX}%
    t%
  }%
}
%    \end{macrocode}
%    \end{macro}
%    \begin{macro}{\HoLogoCss@ConTeXt@narrow}
%    \begin{macrocode}
\def\HoLogoCss@ConTeXt@narrow{%
  \Css{%
    span.HoLogo-ConTeXt-narrow span.HoLogo-C{%
      margin-left:-.0333em;%
    }%
  }%
  \Css{%
    span.HoLogo-ConTeXt-narrow span.HoLogo-TeX{%
      margin-left:-.0667em;%
      margin-right:-.0333em;%
    }%
  }%
  \global\let\HoLogoCss@ConTeXt@narrow\relax
}
%    \end{macrocode}
%    \end{macro}
%
%    \begin{macro}{\HoLogo@ConTeXt}
%    \begin{macrocode}
\def\HoLogo@ConTeXt{\HoLogo@ConTeXt@narrow}
%    \end{macrocode}
%    \end{macro}
%    \begin{macro}{\HoLogoHtml@ConTeXt}
%    \begin{macrocode}
\def\HoLogoHtml@ConTeXt{\HoLogoHtml@ConTeXt@narrow}
%    \end{macrocode}
%    \end{macro}
%
% \subsubsection{\hologo{emTeX}}
%
%    \begin{macro}{\HoLogo@emTeX}
%    \begin{macrocode}
\def\HoLogo@emTeX#1{%
  \HOLOGO@mbox{#1{e}{E}m}%
  \HOLOGO@discretionary
  \hologo{TeX}%
}
%    \end{macrocode}
%    \end{macro}
%    \begin{macro}{\HoLogoCs@emTeX}
%    \begin{macrocode}
\def\HoLogoCs@emTeX#1{#1{e}{E}mTeX}%
%    \end{macrocode}
%    \end{macro}
%    \begin{macro}{\HoLogoBkm@emTeX}
%    \begin{macrocode}
\def\HoLogoBkm@emTeX#1{%
  #1{e}{E}m\hologo{TeX}%
}
%    \end{macrocode}
%    \end{macro}
%    \begin{macro}{\HoLogoHtml@emTeX}
%    \begin{macrocode}
\let\HoLogoHtml@emTeX\HoLogo@emTeX
%    \end{macrocode}
%    \end{macro}
%
% \subsubsection{\hologo{ExTeX}}
%
%    \begin{macro}{\HoLogo@ExTeX}
%    The definition is taken from the FAQ of the
%    project \hologo{ExTeX}
%    \cite{ExTeX-FAQ}.
%\begin{quote}
%\begin{verbatim}
%\def\ExTeX{%
%  \textrm{% Logo always with serifs
%    \ensuremath{%
%      \textstyle
%      \varepsilon_{%
%        \kern-0.15em%
%        \mathcal{X}%
%      }%
%    }%
%    \kern-.15em%
%    \TeX
%  }%
%}
%\end{verbatim}
%\end{quote}
%    \begin{macrocode}
\def\HoLogo@ExTeX#1{%
  \HoLogoFont@font{ExTeX}{rm}{%
    \ltx@mbox{%
      \HOLOGO@MathSetup
      $%
        \textstyle
        \varepsilon_{%
          \kern-0.15em%
          \HoLogoFont@font{ExTeX}{sy}{X}%
        }%
      $%
    }%
    \HOLOGO@discretionary
    \kern-.15em%
    \hologo{TeX}%
  }%
}
%    \end{macrocode}
%    \end{macro}
%    \begin{macro}{\HoLogoHtml@ExTeX}
%    \begin{macrocode}
\def\HoLogoHtml@ExTeX#1{%
  \HoLogoCss@ExTeX
  \HoLogoFont@font{ExTeX}{rm}{%
    \HOLOGO@Span{ExTeX}{%
      \ltx@mbox{%
        \HOLOGO@MathSetup
        $\textstyle\varepsilon$%
        \HOLOGO@Span{X}{$\textstyle\chi$}%
        \hologo{TeX}%
      }%
    }%
  }%
}
%    \end{macrocode}
%    \end{macro}
%    \begin{macro}{\HoLogoBkm@ExTeX}
%    \begin{macrocode}
\def\HoLogoBkm@ExTeX#1{%
  \HOLOGO@PdfdocUnicode{#1{e}{E}x}{\textepsilon\textchi}%
  \hologo{TeX}%
}
%    \end{macrocode}
%    \end{macro}
%    \begin{macro}{\HoLogoCss@ExTeX}
%    \begin{macrocode}
\def\HoLogoCss@ExTeX{%
  \Css{%
    span.HoLogo-ExTeX{%
      font-family:serif;%
    }%
  }%
  \Css{%
    span.HoLogo-ExTeX span.HoLogo-TeX{%
      margin-left:-.15em;%
    }%
  }%
  \global\let\HoLogoCss@ExTeX\relax
}
%    \end{macrocode}
%    \end{macro}
%
% \subsubsection{\hologo{MiKTeX}}
%
%    \begin{macro}{\HoLogo@MiKTeX}
%    \begin{macrocode}
\def\HoLogo@MiKTeX#1{%
  \HOLOGO@mbox{MiK}%
  \HOLOGO@discretionary
  \hologo{TeX}%
}
%    \end{macrocode}
%    \end{macro}
%    \begin{macro}{\HoLogoHtml@MiKTeX}
%    \begin{macrocode}
\let\HoLogoHtml@MiKTeX\HoLogo@MiKTeX
%    \end{macrocode}
%    \end{macro}
%
% \subsubsection{\hologo{OzTeX} and friends}
%
%    Source: \hologo{OzTeX} FAQ \cite{OzTeX}:
%    \begin{quote}
%      |\def\OzTeX{O\kern-.03em z\kern-.15em\TeX}|\\
%      (There is no kerning in OzMF, OzMP and OzTtH.)
%    \end{quote}
%
%    \begin{macro}{\HoLogo@OzTeX}
%    \begin{macrocode}
\def\HoLogo@OzTeX#1{%
  O%
  \kern-.03em %
  z%
  \kern-.15em %
  \hologo{TeX}%
}
%    \end{macrocode}
%    \end{macro}
%    \begin{macro}{\HoLogoHtml@OzTeX}
%    \begin{macrocode}
\def\HoLogoHtml@OzTeX#1{%
  \HoLogoCss@OzTeX
  \HOLOGO@Span{OzTeX}{%
    O%
    \HOLOGO@Span{z}{z}%
    \hologo{TeX}%
  }%
}
%    \end{macrocode}
%    \end{macro}
%    \begin{macro}{\HoLogoCss@OzTeX}
%    \begin{macrocode}
\def\HoLogoCss@OzTeX{%
  \Css{%
    span.HoLogo-OzTeX span.HoLogo-z{%
      margin-left:-.03em;%
      margin-right:-.15em;%
    }%
  }%
  \global\let\HoLogoCss@OzTeX\relax
}
%    \end{macrocode}
%    \end{macro}
%
%    \begin{macro}{\HoLogo@OzMF}
%    \begin{macrocode}
\def\HoLogo@OzMF#1{%
  \HOLOGO@mbox{OzMF}%
}
%    \end{macrocode}
%    \end{macro}
%    \begin{macro}{\HoLogo@OzMP}
%    \begin{macrocode}
\def\HoLogo@OzMP#1{%
  \HOLOGO@mbox{OzMP}%
}
%    \end{macrocode}
%    \end{macro}
%    \begin{macro}{\HoLogo@OzTtH}
%    \begin{macrocode}
\def\HoLogo@OzTtH#1{%
  \HOLOGO@mbox{OzTtH}%
}
%    \end{macrocode}
%    \end{macro}
%
% \subsubsection{\hologo{PCTeX}}
%
%    \begin{macro}{\HoLogo@PCTeX}
%    \begin{macrocode}
\def\HoLogo@PCTeX#1{%
  \HOLOGO@mbox{PC}%
  \hologo{TeX}%
}
%    \end{macrocode}
%    \end{macro}
%    \begin{macro}{\HoLogoHtml@PCTeX}
%    \begin{macrocode}
\let\HoLogoHtml@PCTeX\HoLogo@PCTeX
%    \end{macrocode}
%    \end{macro}
%
% \subsubsection{\hologo{PiCTeX}}
%
%    The original definitions from \xfile{pictex.tex} \cite{PiCTeX}:
%\begin{quote}
%\begin{verbatim}
%\def\PiC{%
%  P%
%  \kern-.12em%
%  \lower.5ex\hbox{I}%
%  \kern-.075em%
%  C%
%}
%\def\PiCTeX{%
%  \PiC
%  \kern-.11em%
%  \TeX
%}
%\end{verbatim}
%\end{quote}
%
%    \begin{macro}{\HoLogo@PiC}
%    \begin{macrocode}
\def\HoLogo@PiC#1{%
  P%
  \kern-.12em%
  \lower.5ex\hbox{I}%
  \kern-.075em%
  C%
  \HOLOGO@SpaceFactor
}
%    \end{macrocode}
%    \end{macro}
%    \begin{macro}{\HoLogoHtml@PiC}
%    \begin{macrocode}
\def\HoLogoHtml@PiC#1{%
  \HoLogoCss@PiC
  \HOLOGO@Span{PiC}{%
    P%
    \HOLOGO@Span{i}{I}%
    C%
  }%
}
%    \end{macrocode}
%    \end{macro}
%    \begin{macro}{\HoLogoCss@PiC}
%    \begin{macrocode}
\def\HoLogoCss@PiC{%
  \Css{%
    span.HoLogo-PiC span.HoLogo-i{%
      position:relative;%
      top:.5ex;%
      margin-left:-.12em;%
      margin-right:-.075em;%
      text-decoration:none;%
    }%
  }%
  \global\let\HoLogoCss@PiC\relax
}
%    \end{macrocode}
%    \end{macro}
%
%    \begin{macro}{\HoLogo@PiCTeX}
%    \begin{macrocode}
\def\HoLogo@PiCTeX#1{%
  \hologo{PiC}%
  \HOLOGO@discretionary
  \kern-.11em%
  \hologo{TeX}%
}
%    \end{macrocode}
%    \end{macro}
%    \begin{macro}{\HoLogoHtml@PiCTeX}
%    \begin{macrocode}
\def\HoLogoHtml@PiCTeX#1{%
  \HoLogoCss@PiCTeX
  \HOLOGO@Span{PiCTeX}{%
    \hologo{PiC}%
    \hologo{TeX}%
  }%
}
%    \end{macrocode}
%    \end{macro}
%    \begin{macro}{\HoLogoCss@PiCTeX}
%    \begin{macrocode}
\def\HoLogoCss@PiCTeX{%
  \Css{%
    span.HoLogo-PiCTeX span.HoLogo-PiC{%
      margin-right:-.11em;%
    }%
  }%
  \global\let\HoLogoCss@PiCTeX\relax
}
%    \end{macrocode}
%    \end{macro}
%
% \subsubsection{\hologo{teTeX}}
%
%    \begin{macro}{\HoLogo@teTeX}
%    \begin{macrocode}
\def\HoLogo@teTeX#1{%
  \HOLOGO@mbox{#1{t}{T}e}%
  \HOLOGO@discretionary
  \hologo{TeX}%
}
%    \end{macrocode}
%    \end{macro}
%    \begin{macro}{\HoLogoCs@teTeX}
%    \begin{macrocode}
\def\HoLogoCs@teTeX#1{#1{t}{T}dfTeX}
%    \end{macrocode}
%    \end{macro}
%    \begin{macro}{\HoLogoBkm@teTeX}
%    \begin{macrocode}
\def\HoLogoBkm@teTeX#1{%
  #1{t}{T}e\hologo{TeX}%
}
%    \end{macrocode}
%    \end{macro}
%    \begin{macro}{\HoLogoHtml@teTeX}
%    \begin{macrocode}
\let\HoLogoHtml@teTeX\HoLogo@teTeX
%    \end{macrocode}
%    \end{macro}
%
% \subsubsection{\hologo{TeX4ht}}
%
%    \begin{macro}{\HoLogo@TeX4ht}
%    \begin{macrocode}
\expandafter\def\csname HoLogo@TeX4ht\endcsname#1{%
  \HOLOGO@mbox{\hologo{TeX}4ht}%
}
%    \end{macrocode}
%    \end{macro}
%    \begin{macro}{\HoLogoHtml@TeX4ht}
%    \begin{macrocode}
\expandafter
\let\csname HoLogoHtml@TeX4ht\expandafter\endcsname
\csname HoLogo@TeX4ht\endcsname
%    \end{macrocode}
%    \end{macro}
%
%
% \subsubsection{\hologo{SageTeX}}
%
%    \begin{macro}{\HoLogo@SageTeX}
%    \begin{macrocode}
\def\HoLogo@SageTeX#1{%
  \HOLOGO@mbox{Sage}%
  \HOLOGO@discretionary
  \HOLOGO@NegativeKerning{eT,oT,To}%
  \hologo{TeX}%
}
%    \end{macrocode}
%    \end{macro}
%    \begin{macro}{\HoLogoHtml@SageTeX}
%    \begin{macrocode}
\let\HoLogoHtml@SageTeX\HoLogo@SageTeX
%    \end{macrocode}
%    \end{macro}
%
% \subsection{\hologo{METAFONT} and friends}
%
%    \begin{macro}{\HoLogo@METAFONT}
%    \begin{macrocode}
\def\HoLogo@METAFONT#1{%
  \HoLogoFont@font{METAFONT}{logo}{%
    \HOLOGO@mbox{META}%
    \HOLOGO@discretionary
    \HOLOGO@mbox{FONT}%
  }%
}
%    \end{macrocode}
%    \end{macro}
%
%    \begin{macro}{\HoLogo@METAPOST}
%    \begin{macrocode}
\def\HoLogo@METAPOST#1{%
  \HoLogoFont@font{METAPOST}{logo}{%
    \HOLOGO@mbox{META}%
    \HOLOGO@discretionary
    \HOLOGO@mbox{POST}%
  }%
}
%    \end{macrocode}
%    \end{macro}
%
%    \begin{macro}{\HoLogo@MetaFun}
%    \begin{macrocode}
\def\HoLogo@MetaFun#1{%
  \HOLOGO@mbox{Meta}%
  \HOLOGO@discretionary
  \HOLOGO@mbox{Fun}%
}
%    \end{macrocode}
%    \end{macro}
%
%    \begin{macro}{\HoLogo@MetaPost}
%    \begin{macrocode}
\def\HoLogo@MetaPost#1{%
  \HOLOGO@mbox{Meta}%
  \HOLOGO@discretionary
  \HOLOGO@mbox{Post}%
}
%    \end{macrocode}
%    \end{macro}
%
% \subsection{Others}
%
% \subsubsection{\hologo{biber}}
%
%    \begin{macro}{\HoLogo@biber}
%    \begin{macrocode}
\def\HoLogo@biber#1{%
  \HOLOGO@mbox{#1{b}{B}i}%
  \HOLOGO@discretionary
  \HOLOGO@mbox{ber}%
}
%    \end{macrocode}
%    \end{macro}
%    \begin{macro}{\HoLogoCs@biber}
%    \begin{macrocode}
\def\HoLogoCs@biber#1{#1{b}{B}iber}
%    \end{macrocode}
%    \end{macro}
%    \begin{macro}{\HoLogoBkm@biber}
%    \begin{macrocode}
\def\HoLogoBkm@biber#1{%
  #1{b}{B}iber%
}
%    \end{macrocode}
%    \end{macro}
%    \begin{macro}{\HoLogoHtml@biber}
%    \begin{macrocode}
\let\HoLogoHtml@biber\HoLogo@biber
%    \end{macrocode}
%    \end{macro}
%
% \subsubsection{\hologo{KOMAScript}}
%
%    \begin{macro}{\HoLogo@KOMAScript}
%    The definition for \hologo{KOMAScript} is taken
%    from \hologo{KOMAScript} (\xfile{scrlogo.dtx}, reformatted) \cite{scrlogo}:
%\begin{quote}
%\begin{verbatim}
%\@ifundefined{KOMAScript}{%
%  \DeclareRobustCommand{\KOMAScript}{%
%    \textsf{%
%      K\kern.05em O\kern.05emM\kern.05em A%
%      \kern.1em-\kern.1em %
%      Script%
%    }%
%  }%
%}{}
%\end{verbatim}
%\end{quote}
%    \begin{macrocode}
\def\HoLogo@KOMAScript#1{%
  \HoLogoFont@font{KOMAScript}{sf}{%
    \HOLOGO@mbox{%
      K\kern.05em%
      O\kern.05em%
      M\kern.05em%
      A%
    }%
    \kern.1em%
    \HOLOGO@hyphen
    \kern.1em%
    \HOLOGO@mbox{Script}%
  }%
}
%    \end{macrocode}
%    \end{macro}
%    \begin{macro}{\HoLogoBkm@KOMAScript}
%    \begin{macrocode}
\def\HoLogoBkm@KOMAScript#1{%
  KOMA-Script%
}
%    \end{macrocode}
%    \end{macro}
%    \begin{macro}{\HoLogoHtml@KOMAScript}
%    \begin{macrocode}
\def\HoLogoHtml@KOMAScript#1{%
  \HoLogoCss@KOMAScript
  \HoLogoFont@font{KOMAScript}{sf}{%
    \HOLOGO@Span{KOMAScript}{%
      K%
      \HOLOGO@Span{O}{O}%
      M%
      \HOLOGO@Span{A}{A}%
      \HOLOGO@Span{hyphen}{-}%
      Script%
    }%
  }%
}
%    \end{macrocode}
%    \end{macro}
%    \begin{macro}{\HoLogoCss@KOMAScript}
%    \begin{macrocode}
\def\HoLogoCss@KOMAScript{%
  \Css{%
    span.HoLogo-KOMAScript{%
      font-family:sans-serif;%
    }%
  }%
  \Css{%
    span.HoLogo-KOMAScript span.HoLogo-O{%
      padding-left:.05em;%
      padding-right:.05em;%
    }%
  }%
  \Css{%
    span.HoLogo-KOMAScript span.HoLogo-A{%
      padding-left:.05em;%
    }%
  }%
  \Css{%
    span.HoLogo-KOMAScript span.HoLogo-hyphen{%
      padding-left:.1em;%
      padding-right:.1em;%
    }%
  }%
  \global\let\HoLogoCss@KOMAScript\relax
}
%    \end{macrocode}
%    \end{macro}
%
% \subsubsection{\hologo{LyX}}
%
%    \begin{macro}{\HoLogo@LyX}
%    The definition is taken from the documentation source files
%    of \hologo{LyX}, \xfile{Intro.lyx} \cite{LyX}:
%\begin{quote}
%\begin{verbatim}
%\def\LyX{%
%  \texorpdfstring{%
%    L\kern-.1667em\lower.25em\hbox{Y}\kern-.125emX\@%
%  }{%
%    LyX%
%  }%
%}
%\end{verbatim}
%\end{quote}
%    \begin{macrocode}
\def\HoLogo@LyX#1{%
  L%
  \kern-.1667em%
  \lower.25em\hbox{Y}%
  \kern-.125em%
  X%
  \HOLOGO@SpaceFactor
}
%    \end{macrocode}
%    \end{macro}
%    \begin{macro}{\HoLogoHtml@LyX}
%    \begin{macrocode}
\def\HoLogoHtml@LyX#1{%
  \HoLogoCss@LyX
  \HOLOGO@Span{LyX}{%
    L%
    \HOLOGO@Span{y}{Y}%
    X%
  }%
}
%    \end{macrocode}
%    \end{macro}
%    \begin{macro}{\HoLogoCss@LyX}
%    \begin{macrocode}
\def\HoLogoCss@LyX{%
  \Css{%
    span.HoLogo-LyX span.HoLogo-y{%
      position:relative;%
      top:.25em;%
      margin-left:-.1667em;%
      margin-right:-.125em;%
      text-decoration:none;%
    }%
  }%
  \global\let\HoLogoCss@LyX\relax
}
%    \end{macrocode}
%    \end{macro}
%
% \subsubsection{\hologo{NTS}}
%
%    \begin{macro}{\HoLogo@NTS}
%    Definition for \hologo{NTS} can be found in
%    package \xpackage{etex\textunderscore man} for the \hologo{eTeX} manual \cite{etexman}
%    and in package \xpackage{dtklogos} \cite{dtklogos}:
%\begin{quote}
%\begin{verbatim}
%\def\NTS{%
%  \leavevmode
%  \hbox{%
%    $%
%      \cal N%
%      \kern-0.35em%
%      \lower0.5ex\hbox{$\cal T$}%
%      \kern-0.2em%
%      S%
%    $%
%  }%
%}
%\end{verbatim}
%\end{quote}
%    \begin{macrocode}
\def\HoLogo@NTS#1{%
  \HoLogoFont@font{NTS}{sy}{%
    N\/%
    \kern-.35em%
    \lower.5ex\hbox{T\/}%
    \kern-.2em%
    S\/%
  }%
  \HOLOGO@SpaceFactor
}
%    \end{macrocode}
%    \end{macro}
%
% \subsubsection{\Hologo{TTH} (\hologo{TeX} to HTML translator)}
%
%    Source: \url{http://hutchinson.belmont.ma.us/tth/}
%    In the HTML source the second `T' is printed as subscript.
%\begin{quote}
%\begin{verbatim}
%T<sub>T</sub>H
%\end{verbatim}
%\end{quote}
%    \begin{macro}{\HoLogo@TTH}
%    \begin{macrocode}
\def\HoLogo@TTH#1{%
  \ltx@mbox{%
    T\HOLOGO@SubScript{T}H%
  }%
  \HOLOGO@SpaceFactor
}
%    \end{macrocode}
%    \end{macro}
%
%    \begin{macro}{\HoLogoHtml@TTH}
%    \begin{macrocode}
\def\HoLogoHtml@TTH#1{%
  T\HCode{<sub>}T\HCode{</sub>}H%
}
%    \end{macrocode}
%    \end{macro}
%
% \subsubsection{\Hologo{HanTheThanh}}
%
%    Partial source: Package \xpackage{dtklogos}.
%    The double accent is U+1EBF (latin small letter e with circumflex
%    and acute).
%    \begin{macro}{\HoLogo@HanTheThanh}
%    \begin{macrocode}
\def\HoLogo@HanTheThanh#1{%
  \ltx@mbox{H\`an}%
  \HOLOGO@space
  \ltx@mbox{%
    Th%
    \HOLOGO@IfCharExists{"1EBF}{%
      \char"1EBF\relax
    }{%
      \^e\hbox to 0pt{\hss\raise .5ex\hbox{\'{}}}%
    }%
  }%
  \HOLOGO@space
  \ltx@mbox{Th\`anh}%
}
%    \end{macrocode}
%    \end{macro}
%    \begin{macro}{\HoLogoBkm@HanTheThanh}
%    \begin{macrocode}
\def\HoLogoBkm@HanTheThanh#1{%
  H\`an %
  Th\HOLOGO@PdfdocUnicode{\^e}{\9036\277} %
  Th\`anh%
}
%    \end{macrocode}
%    \end{macro}
%    \begin{macro}{\HoLogoHtml@HanTheThanh}
%    \begin{macrocode}
\def\HoLogoHtml@HanTheThanh#1{%
  H\`an %
  Th\HCode{&\ltx@hashchar x1ebf;} %
  Th\`anh%
}
%    \end{macrocode}
%    \end{macro}
%
% \subsection{Driver detection}
%
%    \begin{macrocode}
\HOLOGO@IfExists\InputIfFileExists{%
  \InputIfFileExists{hologo.cfg}{}{}%
}{%
  \ltx@IfUndefined{pdf@filesize}{%
    \def\HOLOGO@InputIfExists{%
      \openin\HOLOGO@temp=hologo.cfg\relax
      \ifeof\HOLOGO@temp
        \closein\HOLOGO@temp
      \else
        \closein\HOLOGO@temp
        \begingroup
          \def\x{LaTeX2e}%
        \expandafter\endgroup
        \ifx\fmtname\x
          % \iffalse meta-comment
%
% File: hologo.dtx
% Version: 2016/05/12 v1.11
% Info: A logo collection with bookmark support
%
% Copyright (C) 2010-2012 by
%    Heiko Oberdiek <heiko.oberdiek at googlemail.com>
%
% This work may be distributed and/or modified under the
% conditions of the LaTeX Project Public License, either
% version 1.3c of this license or (at your option) any later
% version. This version of this license is in
%    http://www.latex-project.org/lppl/lppl-1-3c.txt
% and the latest version of this license is in
%    http://www.latex-project.org/lppl.txt
% and version 1.3 or later is part of all distributions of
% LaTeX version 2005/12/01 or later.
%
% This work has the LPPL maintenance status "maintained".
%
% This Current Maintainer of this work is Heiko Oberdiek.
%
% The Base Interpreter refers to any `TeX-Format',
% because some files are installed in TDS:tex/generic//.
%
% This work consists of the main source file hologo.dtx
% and the derived files
%    hologo.sty, hologo.pdf, hologo.ins, hologo.drv, hologo-example.tex,
%    hologo-test1.tex, hologo-test-spacefactor.tex,
%    hologo-test-list.tex.
%
% Distribution:
%    CTAN:macros/latex/contrib/oberdiek/hologo.dtx
%    CTAN:macros/latex/contrib/oberdiek/hologo.pdf
%
% Unpacking:
%    (a) If hologo.ins is present:
%           tex hologo.ins
%    (b) Without hologo.ins:
%           tex hologo.dtx
%    (c) If you insist on using LaTeX
%           latex \let\install=y\input{hologo.dtx}
%        (quote the arguments according to the demands of your shell)
%
% Documentation:
%    (a) If hologo.drv is present:
%           latex hologo.drv
%    (b) Without hologo.drv:
%           latex hologo.dtx; ...
%    The class ltxdoc loads the configuration file ltxdoc.cfg
%    if available. Here you can specify further options, e.g.
%    use A4 as paper format:
%       \PassOptionsToClass{a4paper}{article}
%
%    Programm calls to get the documentation (example):
%       pdflatex hologo.dtx
%       makeindex -s gind.ist hologo.idx
%       pdflatex hologo.dtx
%       makeindex -s gind.ist hologo.idx
%       pdflatex hologo.dtx
%
% Installation:
%    TDS:tex/generic/oberdiek/hologo.sty
%    TDS:doc/latex/oberdiek/hologo.pdf
%    TDS:doc/latex/oberdiek/example/hologo-example.tex
%    TDS:doc/latex/oberdiek/test/hologo-test1.tex
%    TDS:doc/latex/oberdiek/test/hologo-test-spacefactor.tex
%    TDS:doc/latex/oberdiek/test/hologo-test-list.tex
%    TDS:source/latex/oberdiek/hologo.dtx
%
%<*ignore>
\begingroup
  \catcode123=1 %
  \catcode125=2 %
  \def\x{LaTeX2e}%
\expandafter\endgroup
\ifcase 0\ifx\install y1\fi\expandafter
         \ifx\csname processbatchFile\endcsname\relax\else1\fi
         \ifx\fmtname\x\else 1\fi\relax
\else\csname fi\endcsname
%</ignore>
%<*install>
\input docstrip.tex
\Msg{************************************************************************}
\Msg{* Installation}
\Msg{* Package: hologo 2016/05/12 v1.11 A logo collection with bookmark support (HO)}
\Msg{************************************************************************}

\keepsilent
\askforoverwritefalse

\let\MetaPrefix\relax
\preamble

This is a generated file.

Project: hologo
Version: 2016/05/12 v1.11

Copyright (C) 2010-2012 by
   Heiko Oberdiek <heiko.oberdiek at googlemail.com>

This work may be distributed and/or modified under the
conditions of the LaTeX Project Public License, either
version 1.3c of this license or (at your option) any later
version. This version of this license is in
   http://www.latex-project.org/lppl/lppl-1-3c.txt
and the latest version of this license is in
   http://www.latex-project.org/lppl.txt
and version 1.3 or later is part of all distributions of
LaTeX version 2005/12/01 or later.

This work has the LPPL maintenance status "maintained".

This Current Maintainer of this work is Heiko Oberdiek.

The Base Interpreter refers to any `TeX-Format',
because some files are installed in TDS:tex/generic//.

This work consists of the main source file hologo.dtx
and the derived files
   hologo.sty, hologo.pdf, hologo.ins, hologo.drv, hologo-example.tex,
   hologo-test1.tex, hologo-test-spacefactor.tex,
   hologo-test-list.tex.

\endpreamble
\let\MetaPrefix\DoubleperCent

\generate{%
  \file{hologo.ins}{\from{hologo.dtx}{install}}%
  \file{hologo.drv}{\from{hologo.dtx}{driver}}%
  \usedir{tex/generic/oberdiek}%
  \file{hologo.sty}{\from{hologo.dtx}{package}}%
  \usedir{doc/latex/oberdiek/example}%
  \file{hologo-example.tex}{\from{hologo.dtx}{example}}%
  \usedir{doc/latex/oberdiek/test}%
  \file{hologo-test1.tex}{\from{hologo.dtx}{test1}}%
  \file{hologo-test-spacefactor.tex}{\from{hologo.dtx}{test-spacefactor}}%
  \file{hologo-test-list.tex}{\from{hologo.dtx}{test-list}}%
  \nopreamble
  \nopostamble
  \usedir{source/latex/oberdiek/catalogue}%
  \file{hologo.xml}{\from{hologo.dtx}{catalogue}}%
}

\catcode32=13\relax% active space
\let =\space%
\Msg{************************************************************************}
\Msg{*}
\Msg{* To finish the installation you have to move the following}
\Msg{* file into a directory searched by TeX:}
\Msg{*}
\Msg{*     hologo.sty}
\Msg{*}
\Msg{* To produce the documentation run the file `hologo.drv'}
\Msg{* through LaTeX.}
\Msg{*}
\Msg{* Happy TeXing!}
\Msg{*}
\Msg{************************************************************************}

\endbatchfile
%</install>
%<*ignore>
\fi
%</ignore>
%<*driver>
\NeedsTeXFormat{LaTeX2e}
\ProvidesFile{hologo.drv}%
  [2016/05/12 v1.11 A logo collection with bookmark support (HO)]%
\documentclass{ltxdoc}
\usepackage{holtxdoc}[2011/11/22]
\usepackage{hologo}[2016/05/12]
\usepackage{longtable}
\usepackage{array}
\usepackage{paralist}
%\usepackage[T1]{fontenc}
%\usepackage{lmodern}
\begin{document}
  \DocInput{hologo.dtx}%
\end{document}
%</driver>
% \fi
%
%
% \CharacterTable
%  {Upper-case    \A\B\C\D\E\F\G\H\I\J\K\L\M\N\O\P\Q\R\S\T\U\V\W\X\Y\Z
%   Lower-case    \a\b\c\d\e\f\g\h\i\j\k\l\m\n\o\p\q\r\s\t\u\v\w\x\y\z
%   Digits        \0\1\2\3\4\5\6\7\8\9
%   Exclamation   \!     Double quote  \"     Hash (number) \#
%   Dollar        \$     Percent       \%     Ampersand     \&
%   Acute accent  \'     Left paren    \(     Right paren   \)
%   Asterisk      \*     Plus          \+     Comma         \,
%   Minus         \-     Point         \.     Solidus       \/
%   Colon         \:     Semicolon     \;     Less than     \<
%   Equals        \=     Greater than  \>     Question mark \?
%   Commercial at \@     Left bracket  \[     Backslash     \\
%   Right bracket \]     Circumflex    \^     Underscore    \_
%   Grave accent  \`     Left brace    \{     Vertical bar  \|
%   Right brace   \}     Tilde         \~}
%
% \GetFileInfo{hologo.drv}
%
% \title{The \xpackage{hologo} package}
% \date{2016/05/12 v1.11}
% \author{Heiko Oberdiek\\\xemail{heiko.oberdiek at googlemail.com}}
%
% \maketitle
%
% \begin{abstract}
% This package starts a collection of logos with support for bookmarks
% strings.
% \end{abstract}
%
% \tableofcontents
%
% \section{Documentation}
%
% \subsection{Logo macros}
%
% \begin{declcs}{hologo} \M{name}
% \end{declcs}
% Macro \cs{hologo} sets the logo with name \meta{name}.
% The following table shows the supported names.
%
% \begingroup
%   \def\hologoEntry#1#2#3{^^A
%     #1&#2&\hologoLogoSetup{#1}{variant=#2}\hologo{#1}&#3\tabularnewline
%   }
%   \begin{longtable}{>{\ttfamily}l>{\ttfamily}lll}
%     \rmfamily\bfseries{name} & \rmfamily\bfseries variant
%     & \bfseries logo & \bfseries since\\
%     \hline
%     \endhead
%     \hologoList
%   \end{longtable}
% \endgroup
%
% \begin{declcs}{Hologo} \M{name}
% \end{declcs}
% Macro \cs{Hologo} starts the logo \meta{name} with an uppercase
% letter. As an exception small greek letters are not converted
% to uppercase. Examples, see \hologo{eTeX} and \hologo{ExTeX}.
%
% \subsection{Setup macros}
%
% The package does not support package options, but the following
% setup macros can be used to set options.
%
% \begin{declcs}{hologoSetup} \M{key value list}
% \end{declcs}
% Macro \cs{hologoSetup} sets global options.
%
% \begin{declcs}{hologoLogoSetup} \M{logo} \M{key value list}
% \end{declcs}
% Some options can also be used to configure a logo.
% These settings take precedence over global option settings.
%
% \subsection{Options}\label{sec:options}
%
% There are boolean and string options:
% \begin{description}
% \item[Boolean option:]
% It takes |true| or |false|
% as value. If the value is omitted, then |true| is used.
% \item[String option:]
% A value must be given as string. (But the string might be empty.)
% \end{description}
% The following options can be used both in \cs{hologoSetup}
% and \cs{hologoLogoSetup}:
% \begin{description}
% \def\entry#1{\item[\xoption{#1}:]}
% \entry{break}
%   enables or disables line breaks inside the logo. This setting is
%   refined by options \xoption{hyphenbreak}, \xoption{spacebreak}
%   or \xoption{discretionarybreak}.
%   Default is |false|.
% \entry{hyphenbreak}
%   enables or disables the line break right after the hyphen character.
% \entry{spacebreak}
%   enables or disables line breaks at space characters.
% \entry{discretionarybreak}
%   enables or disables line breaks at hyphenation points
%   (inserted by \cs{-}).
% \end{description}
% Macro \cs{hologoLogoSetup} also knows:
% \begin{description}
% \item[\xoption{variant}:]
%   This is a string option. It specifies a variant of a logo that
%   must exist. An empty string selects the package default variant.
% \end{description}
% Example:
% \begin{quote}
%   |\hologoSetup{break=false}|\\
%   |\hologoLogoSetup{plainTeX}{variant=hyphen,hyphenbreak}|\\
%   Then ``plain-\TeX'' contains one break point after the hyphen.
% \end{quote}
%
% \subsection{Driver options}
%
% Sometimes graphical operations are needed to construct some
% glyphs (e.g.\ \hologo{XeTeX}). If package \xpackage{graphics}
% or package \xpackage{pgf} are found, then the macros are taken
% from there. Otherwise the packge defines its own operations
% and therefore needs the driver information. Many drivers are
% detected automatically (\hologo{pdfTeX}/\hologo{LuaTeX}
% in PDF mode, \hologo{XeTeX}, \hologo{VTeX}). These have precedence
% over a driver option. The driver can be given as package option
% or using \cs{hologoDriverSetup}.
% The following list contains the recognized driver options:
% \begin{itemize}
% \item \xoption{pdftex}, \xoption{luatex}
% \item \xoption{dvipdfm}, \xoption{dvipdfmx}
% \item \xoption{dvips}, \xoption{dvipsone}, \xoption{xdvi}
% \item \xoption{xetex}
% \item \xoption{vtex}
% \end{itemize}
% The left driver of a line is the driver name that is used internally.
% The following names are aliases for drivers that use the
% same method. Therefore the entry in the \xext{log} file for
% the used driver prints the internally used driver name.
% \begin{description}
% \item[\xoption{driverfallback}:]
%   This option expects a driver that is used,
%   if the driver could not be detected automatically.
% \end{description}
%
% \begin{declcs}{hologoDriverSetup} \M{driver option}
% \end{declcs}
% The driver can also be configured after package loading
% using \cs{hologoDriverSetup}, also the way for \hologo{plainTeX}
% to setup the driver.
%
% \subsection{Font setup}
%
% Some logos require a special font, but should also be usable by
% \hologo{plainTeX}. Therefore the package provides some ways
% to influence the font settings. The options below
% take font settings as values. Both font commands
% such as \cs{sffamily} and macros that take one argument
% like \cs{textsf} can be used.
%
% \begin{declcs}{hologoFontSetup} \M{key value list}
% \end{declcs}
% Macro \cs{hologoFontSetup} sets the fonts for all logos.
% Supported keys:
% \begin{description}
% \def\entry#1{\item[\xoption{#1}:]}
% \entry{general}
%   This font is used for all logos. The default is empty.
%   That means no special font is used.
% \entry{bibsf}
%   This font is used for
%   {\hologoLogoSetup{BibTeX}{variant=sf}\hologo{BibTeX}}
%   with variant \xoption{sf}.
% \entry{rm}
%   This font is a serif font. It is used for \hologo{ExTeX}.
% \entry{sc}
%   This font specifies a small caps font. It is used for
%   {\hologoLogoSetup{BibTeX}{variant=sc}\hologo{BibTeX}}
%   with variant \xoption{sc}.
% \entry{sf}
%   This font specifies a sans serif font. The default
%   is \cs{sffamily}, then \cs{sf} is tried. Otherwise
%   a warning is given. It is used by \hologo{KOMAScript}.
% \entry{sy}
%   This is the font for math symbols (e.g. cmsy).
%   It is used by \hologo{AmS}, \hologo{NTS}, \hologo{ExTeX}.
% \entry{logo}
%   \hologo{METAFONT} and \hologo{METAPOST} are using that font.
%   In \hologo{LaTeX} \cs{logofamily} is used and
%   the definitions of package \xpackage{mflogo} are used
%   if the package is not loaded.
%   Otherwise the \cs{tenlogo} is used and defined
%   if it does not already exists.
% \end{description}
%
% \begin{declcs}{hologoLogoFontSetup} \M{logo} \M{key value list}
% \end{declcs}
% Fonts can also be set for a logo or logo component separately,
% see the following list.
% The keys are the same as for \cs{hologoFontSetup}.
%
% \begin{longtable}{>{\ttfamily}l>{\sffamily}ll}
%   \meta{logo} & keys & result\\
%   \hline
%   \endhead
%   BibTeX & bibsf & {\hologoLogoSetup{BibTeX}{variant=sf}\hologo{BibTeX}}\\[.5ex]
%   BibTeX & sc & {\hologoLogoSetup{BibTeX}{variant=sc}\hologo{BibTeX}}\\[.5ex]
%   ExTeX & rm & \hologo{ExTeX}\\
%   SliTeX & rm & \hologo{SliTeX}\\[.5ex]
%   AmS & sy & \hologo{AmS}\\
%   ExTeX & sy & \hologo{ExTeX}\\
%   NTS & sy & \hologo{NTS}\\[.5ex]
%   KOMAScript & sf & \hologo{KOMAScript}\\[.5ex]
%   METAFONT & logo & \hologo{METAFONT}\\
%   METAPOST & logo & \hologo{METAPOST}\\[.5ex]
%   SliTeX & sc \hologo{SliTeX}
% \end{longtable}
%
% \subsubsection{Font order}
%
% For all logos the font \xoption{general} is applied first.
% Example:
%\begin{quote}
%|\hologoFontSetup{general=\color{red}}|
%\end{quote}
% will print red logos.
% Then if the font uses a special font \xoption{sf}, for example,
% the font is applied that is setup by \cs{hologoLogoFontSetup}.
% If this font is not setup, then the common font setup
% by \cs{hologoFontSetup} is used. Otherwise a warning is given,
% that there is no font configured.
%
% \subsection{Additional user macros}
%
% Usually a variant of a logo is configured by using
% \cs{hologoLogoSetup}, because it is bad style to mix
% different variants of the same logo in the same text.
% There the following macros are a convenience for testing.
%
% \begin{declcs}{hologoVariant} \M{name} \M{variant}\\
%   \cs{HologoVariant} \M{name} \M{variant}
% \end{declcs}
% Logo \meta{name} is set using \meta{variant} that specifies
% explicitely which variant of the macro is used. If the argument
% is empty, then the default form of the logo is used
% (configurable by \cs{hologoLogoSetup}).
%
% \cs{HologoVariant} is used if the logo is set in a context
% that needs an uppercase first letter (beginning of a sentence, \dots).
%
% \begin{declcs}{hologoList}\\
%   \cs{hologoEntry} \M{logo} \M{variant} \M{since}
% \end{declcs}
% Macro \cs{hologoList} contains all logos that are provided
% by the package including variants. The list consists of calls
% of \cs{hologoEntry} with three arguments starting with the
% logo name \meta{logo} and its variant \meta{variant}. An empty
% variant means the current default. Argument \meta{since} specifies
% with version of the package \xpackage{hologo} is needed to get
% the logo. If the logo is fixed, then the date gets updated.
% Therefore the date \meta{since} is not exactly the date of
% the first introduction, but rather the date of the latest fix.
%
% Before \cs{hologoList} can be used, macro \cs{hologoEntry} needs
% a definition. The example file in section \ref{sec:example}
% shows applications of \cs{hologoList}.
%
% \subsection{Supported contexts}
%
% Macros \cs{hologo} and friends support special contexts:
% \begin{itemize}
% \item \hologo{LaTeX}'s protection mechanism.
% \item Bookmarks of package \xpackage{hyperref}.
% \item Package \xpackage{tex4ht}.
% \item The macros can be used inside \cs{csname} constructs,
%   if \cs{ifincsname} is available (\hologo{pdfTeX}, \hologo{XeTeX},
%   \hologo{LuaTeX}).
% \end{itemize}
%
% \subsection{Example}
% \label{sec:example}
%
% The following example prints the logos in different fonts.
%    \begin{macrocode}
%<*example>
%<<verbatim
\NeedsTeXFormat{LaTeX2e}
\documentclass[a4paper]{article}
\usepackage[
  hmargin=20mm,
  vmargin=20mm,
]{geometry}
\pagestyle{empty}
\usepackage{hologo}[2016/05/12]
\usepackage{longtable}
\usepackage{array}
\setlength{\extrarowheight}{2pt}
\usepackage[T1]{fontenc}
\usepackage{lmodern}
\usepackage{pdflscape}
\usepackage[
  pdfencoding=auto,
]{hyperref}
\hypersetup{
  pdfauthor={Heiko Oberdiek},
  pdftitle={Example for package `hologo'},
  pdfsubject={Logos with fonts lmr, lmss, qtm, qpl, qhv},
}
\usepackage{bookmark}

% Print the logo list on the console

\begingroup
  \typeout{}%
  \typeout{*** Begin of logo list ***}%
  \newcommand*{\hologoEntry}[3]{%
    \typeout{#1 \ifx\\#2\\\else(#2) \fi[#3]}%
  }%
  \hologoList
  \typeout{*** End of logo list ***}%
  \typeout{}%
\endgroup

\begin{document}
\begin{landscape}

  \section{Example file for package `hologo'}

  % Table for font names

  \begin{longtable}{>{\bfseries}ll}
    \textbf{font} & \textbf{Font name}\\
    \hline
    lmr & Latin Modern Roman\\
    lmss & Latin Modern Sans\\
    qtm & \TeX\ Gyre Termes\\
    qhv & \TeX\ Gyre Heros\\
    qpl & \TeX\ Gyre Pagella\\
  \end{longtable}

  % Logo list with logos in different fonts

  \begingroup
    \newcommand*{\SetVariant}[2]{%
      \ifx\\#2\\%
      \else
        \hologoLogoSetup{#1}{variant=#2}%
      \fi
    }%
    \newcommand*{\hologoEntry}[3]{%
      \SetVariant{#1}{#2}%
      \raisebox{1em}[0pt][0pt]{\hypertarget{#1@#2}{}}%
      \bookmark[%
        dest={#1@#2},%
      ]{%
        #1\ifx\\#2\\\else\space(#2)\fi: \Hologo{#1}, \hologo{#1} %
        [Unicode]%
      }%
      \hypersetup{unicode=false}%
      \bookmark[%
        dest={#1@#2},%
      ]{%
        #1\ifx\\#2\\\else\space(#2)\fi: \Hologo{#1}, \hologo{#1} %
        [PDFDocEncoding]%
      }%
      \texttt{#1}%
      &%
      \texttt{#2}%
      &%
      \Hologo{#1}%
      &%
      \SetVariant{#1}{#2}%
      \hologo{#1}%
      &%
      \SetVariant{#1}{#2}%
      \fontfamily{qtm}\selectfont
      \hologo{#1}%
      &%
      \SetVariant{#1}{#2}%
      \fontfamily{qpl}\selectfont
      \hologo{#1}%
      &%
      \SetVariant{#1}{#2}%
      \textsf{\hologo{#1}}%
      &%
      \SetVariant{#1}{#2}%
      \fontfamily{qhv}\selectfont
      \hologo{#1}%
      \tabularnewline
    }%
    \begin{longtable}{llllllll}%
      \textbf{\textit{logo}} & \textbf{\textit{variant}} &
      \texttt{\string\Hologo} &
      \textbf{lmr} & \textbf{qtm} & \textbf{qpl} &
      \textbf{lmss} & \textbf{qhv}
      \tabularnewline
      \hline
      \endhead
      \hologoList
    \end{longtable}%
  \endgroup

\end{landscape}
\end{document}
%verbatim
%</example>
%    \end{macrocode}
%
% \StopEventually{
% }
%
% \section{Implementation}
%    \begin{macrocode}
%<*package>
%    \end{macrocode}
%    Reload check, especially if the package is not used with \LaTeX.
%    \begin{macrocode}
\begingroup\catcode61\catcode48\catcode32=10\relax%
  \catcode13=5 % ^^M
  \endlinechar=13 %
  \catcode35=6 % #
  \catcode39=12 % '
  \catcode44=12 % ,
  \catcode45=12 % -
  \catcode46=12 % .
  \catcode58=12 % :
  \catcode64=11 % @
  \catcode123=1 % {
  \catcode125=2 % }
  \expandafter\let\expandafter\x\csname ver@hologo.sty\endcsname
  \ifx\x\relax % plain-TeX, first loading
  \else
    \def\empty{}%
    \ifx\x\empty % LaTeX, first loading,
      % variable is initialized, but \ProvidesPackage not yet seen
    \else
      \expandafter\ifx\csname PackageInfo\endcsname\relax
        \def\x#1#2{%
          \immediate\write-1{Package #1 Info: #2.}%
        }%
      \else
        \def\x#1#2{\PackageInfo{#1}{#2, stopped}}%
      \fi
      \x{hologo}{The package is already loaded}%
      \aftergroup\endinput
    \fi
  \fi
\endgroup%
%    \end{macrocode}
%    Package identification:
%    \begin{macrocode}
\begingroup\catcode61\catcode48\catcode32=10\relax%
  \catcode13=5 % ^^M
  \endlinechar=13 %
  \catcode35=6 % #
  \catcode39=12 % '
  \catcode40=12 % (
  \catcode41=12 % )
  \catcode44=12 % ,
  \catcode45=12 % -
  \catcode46=12 % .
  \catcode47=12 % /
  \catcode58=12 % :
  \catcode64=11 % @
  \catcode91=12 % [
  \catcode93=12 % ]
  \catcode123=1 % {
  \catcode125=2 % }
  \expandafter\ifx\csname ProvidesPackage\endcsname\relax
    \def\x#1#2#3[#4]{\endgroup
      \immediate\write-1{Package: #3 #4}%
      \xdef#1{#4}%
    }%
  \else
    \def\x#1#2[#3]{\endgroup
      #2[{#3}]%
      \ifx#1\@undefined
        \xdef#1{#3}%
      \fi
      \ifx#1\relax
        \xdef#1{#3}%
      \fi
    }%
  \fi
\expandafter\x\csname ver@hologo.sty\endcsname
\ProvidesPackage{hologo}%
  [2016/05/12 v1.11 A logo collection with bookmark support (HO)]%
%    \end{macrocode}
%
%    \begin{macrocode}
\begingroup\catcode61\catcode48\catcode32=10\relax%
  \catcode13=5 % ^^M
  \endlinechar=13 %
  \catcode123=1 % {
  \catcode125=2 % }
  \catcode64=11 % @
  \def\x{\endgroup
    \expandafter\edef\csname HOLOGO@AtEnd\endcsname{%
      \endlinechar=\the\endlinechar\relax
      \catcode13=\the\catcode13\relax
      \catcode32=\the\catcode32\relax
      \catcode35=\the\catcode35\relax
      \catcode61=\the\catcode61\relax
      \catcode64=\the\catcode64\relax
      \catcode123=\the\catcode123\relax
      \catcode125=\the\catcode125\relax
    }%
  }%
\x\catcode61\catcode48\catcode32=10\relax%
\catcode13=5 % ^^M
\endlinechar=13 %
\catcode35=6 % #
\catcode64=11 % @
\catcode123=1 % {
\catcode125=2 % }
\def\TMP@EnsureCode#1#2{%
  \edef\HOLOGO@AtEnd{%
    \HOLOGO@AtEnd
    \catcode#1=\the\catcode#1\relax
  }%
  \catcode#1=#2\relax
}
\TMP@EnsureCode{10}{12}% ^^J
\TMP@EnsureCode{33}{12}% !
\TMP@EnsureCode{34}{12}% "
\TMP@EnsureCode{36}{3}% $
\TMP@EnsureCode{38}{4}% &
\TMP@EnsureCode{39}{12}% '
\TMP@EnsureCode{40}{12}% (
\TMP@EnsureCode{41}{12}% )
\TMP@EnsureCode{42}{12}% *
\TMP@EnsureCode{43}{12}% +
\TMP@EnsureCode{44}{12}% ,
\TMP@EnsureCode{45}{12}% -
\TMP@EnsureCode{46}{12}% .
\TMP@EnsureCode{47}{12}% /
\TMP@EnsureCode{58}{12}% :
\TMP@EnsureCode{59}{12}% ;
\TMP@EnsureCode{60}{12}% <
\TMP@EnsureCode{62}{12}% >
\TMP@EnsureCode{63}{12}% ?
\TMP@EnsureCode{91}{12}% [
\TMP@EnsureCode{93}{12}% ]
\TMP@EnsureCode{94}{7}% ^ (superscript)
\TMP@EnsureCode{95}{8}% _ (subscript)
\TMP@EnsureCode{96}{12}% `
\TMP@EnsureCode{124}{12}% |
\edef\HOLOGO@AtEnd{%
  \HOLOGO@AtEnd
  \escapechar\the\escapechar\relax
  \noexpand\endinput
}
\escapechar=92 %
%    \end{macrocode}
%
% \subsection{Logo list}
%
%    \begin{macro}{\hologoList}
%    \begin{macrocode}
\def\hologoList{%
  \hologoEntry{(La)TeX}{}{2011/10/01}%
  \hologoEntry{AmSLaTeX}{}{2010/04/16}%
  \hologoEntry{AmSTeX}{}{2010/04/16}%
  \hologoEntry{biber}{}{2011/10/01}%
  \hologoEntry{BibTeX}{}{2011/10/01}%
  \hologoEntry{BibTeX}{sf}{2011/10/01}%
  \hologoEntry{BibTeX}{sc}{2011/10/01}%
  \hologoEntry{BibTeX8}{}{2011/11/22}%
  \hologoEntry{ConTeXt}{}{2011/03/25}%
  \hologoEntry{ConTeXt}{narrow}{2011/03/25}%
  \hologoEntry{ConTeXt}{simple}{2011/03/25}%
  \hologoEntry{emTeX}{}{2010/04/26}%
  \hologoEntry{eTeX}{}{2010/04/08}%
  \hologoEntry{ExTeX}{}{2011/10/01}%
  \hologoEntry{HanTheThanh}{}{2011/11/29}%
  \hologoEntry{iniTeX}{}{2011/10/01}%
  \hologoEntry{KOMAScript}{}{2011/10/01}%
  \hologoEntry{La}{}{2010/05/08}%
  \hologoEntry{LaTeX}{}{2010/04/08}%
  \hologoEntry{LaTeX2e}{}{2010/04/08}%
  \hologoEntry{LaTeX3}{}{2010/04/24}%
  \hologoEntry{LaTeXe}{}{2010/04/08}%
  \hologoEntry{LaTeXML}{}{2011/11/22}%
  \hologoEntry{LaTeXTeX}{}{2011/10/01}%
  \hologoEntry{LuaLaTeX}{}{2010/04/08}%
  \hologoEntry{LuaTeX}{}{2010/04/08}%
  \hologoEntry{LyX}{}{2011/10/01}%
  \hologoEntry{METAFONT}{}{2011/10/01}%
  \hologoEntry{MetaFun}{}{2011/10/01}%
  \hologoEntry{METAPOST}{}{2011/10/01}%
  \hologoEntry{MetaPost}{}{2011/10/01}%
  \hologoEntry{MiKTeX}{}{2011/10/01}%
  \hologoEntry{NTS}{}{2011/10/01}%
  \hologoEntry{OzMF}{}{2011/10/01}%
  \hologoEntry{OzMP}{}{2011/10/01}%
  \hologoEntry{OzTeX}{}{2011/10/01}%
  \hologoEntry{OzTtH}{}{2011/10/01}%
  \hologoEntry{PCTeX}{}{2011/10/01}%
  \hologoEntry{pdfTeX}{}{2011/10/01}%
  \hologoEntry{pdfLaTeX}{}{2011/10/01}%
  \hologoEntry{PiC}{}{2011/10/01}%
  \hologoEntry{PiCTeX}{}{2011/10/01}%
  \hologoEntry{plainTeX}{}{2010/04/08}%
  \hologoEntry{plainTeX}{space}{2010/04/16}%
  \hologoEntry{plainTeX}{hyphen}{2010/04/16}%
  \hologoEntry{plainTeX}{runtogether}{2010/04/16}%
  \hologoEntry{SageTeX}{}{2011/11/22}%
  \hologoEntry{SLiTeX}{}{2011/10/01}%
  \hologoEntry{SLiTeX}{lift}{2011/10/01}%
  \hologoEntry{SLiTeX}{narrow}{2011/10/01}%
  \hologoEntry{SLiTeX}{simple}{2011/10/01}%
  \hologoEntry{SliTeX}{}{2011/10/01}%
  \hologoEntry{SliTeX}{narrow}{2011/10/01}%
  \hologoEntry{SliTeX}{simple}{2011/10/01}%
  \hologoEntry{SliTeX}{lift}{2011/10/01}%
  \hologoEntry{teTeX}{}{2011/10/01}%
  \hologoEntry{TeX}{}{2010/04/08}%
  \hologoEntry{TeX4ht}{}{2011/11/22}%
  \hologoEntry{TTH}{}{2011/11/22}%
  \hologoEntry{virTeX}{}{2011/10/01}%
  \hologoEntry{VTeX}{}{2010/04/24}%
  \hologoEntry{Xe}{}{2010/04/08}%
  \hologoEntry{XeLaTeX}{}{2010/04/08}%
  \hologoEntry{XeTeX}{}{2010/04/08}%
}
%    \end{macrocode}
%    \end{macro}
%
% \subsection{Load resources}
%
%    \begin{macrocode}
\begingroup\expandafter\expandafter\expandafter\endgroup
\expandafter\ifx\csname RequirePackage\endcsname\relax
  \def\TMP@RequirePackage#1[#2]{%
    \begingroup\expandafter\expandafter\expandafter\endgroup
    \expandafter\ifx\csname ver@#1.sty\endcsname\relax
      \input #1.sty\relax
    \fi
  }%
  \TMP@RequirePackage{ltxcmds}[2011/02/04]%
  \TMP@RequirePackage{infwarerr}[2010/04/08]%
  \TMP@RequirePackage{kvsetkeys}[2010/03/01]%
  \TMP@RequirePackage{kvdefinekeys}[2010/03/01]%
  \TMP@RequirePackage{pdftexcmds}[2010/04/01]%
  \TMP@RequirePackage{ifpdf}[2010/01/28]%
  \TMP@RequirePackage{ifluatex}[2010/03/01]%
  \ltx@IfUndefined{newif}{%
    \expandafter\let\csname newif\endcsname\ltx@newif
  }{}%
  \TMP@RequirePackage{ifxetex}[2009/01/23]%
  \TMP@RequirePackage{ifvtex}[2010/03/01]%
\else
  \RequirePackage{ltxcmds}[2011/02/04]%
  \RequirePackage{infwarerr}[2010/04/08]%
  \RequirePackage{kvsetkeys}[2010/03/01]%
  \RequirePackage{kvdefinekeys}[2010/03/01]%
  \RequirePackage{pdftexcmds}[2010/04/01]%
  \RequirePackage{ifpdf}[2010/01/28]%
  \RequirePackage{ifluatex}[2010/03/01]%
  \RequirePackage{ifxetex}[2009/01/23]%
  \RequirePackage{ifvtex}[2010/03/01]%
\fi
%    \end{macrocode}
%
%    \begin{macro}{\HOLOGO@IfDefined}
%    \begin{macrocode}
\def\HOLOGO@IfExists#1{%
  \ifx\@undefined#1%
    \expandafter\ltx@secondoftwo
  \else
    \ifx\relax#1%
      \expandafter\ltx@secondoftwo
    \else
      \expandafter\expandafter\expandafter\ltx@firstoftwo
    \fi
  \fi
}
%    \end{macrocode}
%    \end{macro}
%
% \subsection{Setup macros}
%
%    \begin{macro}{\hologoSetup}
%    \begin{macrocode}
\def\hologoSetup{%
  \let\HOLOGO@name\relax
  \HOLOGO@Setup
}
%    \end{macrocode}
%    \end{macro}
%
%    \begin{macro}{\hologoLogoSetup}
%    \begin{macrocode}
\def\hologoLogoSetup#1{%
  \edef\HOLOGO@name{#1}%
  \ltx@IfUndefined{HoLogo@\HOLOGO@name}{%
    \@PackageError{hologo}{%
      Unknown logo `\HOLOGO@name'%
    }\@ehc
    \ltx@gobble
  }{%
    \HOLOGO@Setup
  }%
}
%    \end{macrocode}
%    \end{macro}
%
%    \begin{macro}{\HOLOGO@Setup}
%    \begin{macrocode}
\def\HOLOGO@Setup{%
  \kvsetkeys{HoLogo}%
}
%    \end{macrocode}
%    \end{macro}
%
% \subsection{Options}
%
%    \begin{macro}{\HOLOGO@DeclareBoolOption}
%    \begin{macrocode}
\def\HOLOGO@DeclareBoolOption#1{%
  \expandafter\chardef\csname HOLOGOOPT@#1\endcsname\ltx@zero
  \kv@define@key{HoLogo}{#1}[true]{%
    \def\HOLOGO@temp{##1}%
    \ifx\HOLOGO@temp\HOLOGO@true
      \ifx\HOLOGO@name\relax
        \expandafter\chardef\csname HOLOGOOPT@#1\endcsname=\ltx@one
      \else
        \expandafter\chardef\csname
        HoLogoOpt@#1@\HOLOGO@name\endcsname\ltx@one
      \fi
      \HOLOGO@SetBreakAll{#1}%
    \else
      \ifx\HOLOGO@temp\HOLOGO@false
        \ifx\HOLOGO@name\relax
          \expandafter\chardef\csname HOLOGOOPT@#1\endcsname=\ltx@zero
        \else
          \expandafter\chardef\csname
          HoLogoOpt@#1@\HOLOGO@name\endcsname=\ltx@zero
        \fi
        \HOLOGO@SetBreakAll{#1}%
      \else
        \@PackageError{hologo}{%
          Unknown value `##1' for boolean option `#1'.\MessageBreak
          Known values are `true' and `false'%
        }\@ehc
      \fi
    \fi
  }%
}
%    \end{macrocode}
%    \end{macro}
%
%    \begin{macro}{\HOLOGO@SetBreakAll}
%    \begin{macrocode}
\def\HOLOGO@SetBreakAll#1{%
  \def\HOLOGO@temp{#1}%
  \ifx\HOLOGO@temp\HOLOGO@break
    \ifx\HOLOGO@name\relax
      \chardef\HOLOGOOPT@hyphenbreak=\HOLOGOOPT@break
      \chardef\HOLOGOOPT@spacebreak=\HOLOGOOPT@break
      \chardef\HOLOGOOPT@discretionarybreak=\HOLOGOOPT@break
    \else
      \expandafter\chardef
         \csname HoLogoOpt@hyphenbreak@\HOLOGO@name\endcsname=%
         \csname HoLogoOpt@break@\HOLOGO@name\endcsname
      \expandafter\chardef
         \csname HoLogoOpt@spacebreak@\HOLOGO@name\endcsname=%
         \csname HoLogoOpt@break@\HOLOGO@name\endcsname
      \expandafter\chardef
         \csname HoLogoOpt@discretionarybreak@\HOLOGO@name
             \endcsname=%
         \csname HoLogoOpt@break@\HOLOGO@name\endcsname
    \fi
  \fi
}
%    \end{macrocode}
%    \end{macro}
%
%    \begin{macro}{\HOLOGO@true}
%    \begin{macrocode}
\def\HOLOGO@true{true}
%    \end{macrocode}
%    \end{macro}
%    \begin{macro}{\HOLOGO@false}
%    \begin{macrocode}
\def\HOLOGO@false{false}
%    \end{macrocode}
%    \end{macro}
%    \begin{macro}{\HOLOGO@break}
%    \begin{macrocode}
\def\HOLOGO@break{break}
%    \end{macrocode}
%    \end{macro}
%
%    \begin{macrocode}
\HOLOGO@DeclareBoolOption{break}
\HOLOGO@DeclareBoolOption{hyphenbreak}
\HOLOGO@DeclareBoolOption{spacebreak}
\HOLOGO@DeclareBoolOption{discretionarybreak}
%    \end{macrocode}
%
%    \begin{macrocode}
\kv@define@key{HoLogo}{variant}{%
  \ifx\HOLOGO@name\relax
    \@PackageError{hologo}{%
      Option `variant' is not available in \string\hologoSetup,%
      \MessageBreak
      Use \string\hologoLogoSetup\space instead%
    }\@ehc
  \else
    \edef\HOLOGO@temp{#1}%
    \ifx\HOLOGO@temp\ltx@empty
      \expandafter
      \let\csname HoLogoOpt@variant@\HOLOGO@name\endcsname\@undefined
    \else
      \ltx@IfUndefined{HoLogo@\HOLOGO@name @\HOLOGO@temp}{%
        \@PackageError{hologo}{%
          Unknown variant `\HOLOGO@temp' of logo `\HOLOGO@name'%
        }\@ehc
      }{%
        \expandafter
        \let\csname HoLogoOpt@variant@\HOLOGO@name\endcsname
            \HOLOGO@temp
      }%
    \fi
  \fi
}
%    \end{macrocode}
%
%    \begin{macro}{\HOLOGO@Variant}
%    \begin{macrocode}
\def\HOLOGO@Variant#1{%
  #1%
  \ltx@ifundefined{HoLogoOpt@variant@#1}{%
  }{%
    @\csname HoLogoOpt@variant@#1\endcsname
  }%
}
%    \end{macrocode}
%    \end{macro}
%
% \subsection{Break/no-break support}
%
%    \begin{macro}{\HOLOGO@space}
%    \begin{macrocode}
\def\HOLOGO@space{%
  \ltx@ifundefined{HoLogoOpt@spacebreak@\HOLOGO@name}{%
    \ltx@ifundefined{HoLogoOpt@break@\HOLOGO@name}{%
      \chardef\HOLOGO@temp=\HOLOGOOPT@spacebreak
    }{%
      \chardef\HOLOGO@temp=%
        \csname HoLogoOpt@break@\HOLOGO@name\endcsname
    }%
  }{%
    \chardef\HOLOGO@temp=%
      \csname HoLogoOpt@spacebreak@\HOLOGO@name\endcsname
  }%
  \ifcase\HOLOGO@temp
    \penalty10000 %
  \fi
  \ltx@space
}
%    \end{macrocode}
%    \end{macro}
%
%    \begin{macro}{\HOLOGO@hyphen}
%    \begin{macrocode}
\def\HOLOGO@hyphen{%
  \ltx@ifundefined{HoLogoOpt@hyphenbreak@\HOLOGO@name}{%
    \ltx@ifundefined{HoLogoOpt@break@\HOLOGO@name}{%
      \chardef\HOLOGO@temp=\HOLOGOOPT@hyphenbreak
    }{%
      \chardef\HOLOGO@temp=%
        \csname HoLogoOpt@break@\HOLOGO@name\endcsname
    }%
  }{%
    \chardef\HOLOGO@temp=%
      \csname HoLogoOpt@hyphenbreak@\HOLOGO@name\endcsname
  }%
  \ifcase\HOLOGO@temp
    \ltx@mbox{-}%
  \else
    -%
  \fi
}
%    \end{macrocode}
%    \end{macro}
%
%    \begin{macro}{\HOLOGO@discretionary}
%    \begin{macrocode}
\def\HOLOGO@discretionary{%
  \ltx@ifundefined{HoLogoOpt@discretionarybreak@\HOLOGO@name}{%
    \ltx@ifundefined{HoLogoOpt@break@\HOLOGO@name}{%
      \chardef\HOLOGO@temp=\HOLOGOOPT@discretionarybreak
    }{%
      \chardef\HOLOGO@temp=%
        \csname HoLogoOpt@break@\HOLOGO@name\endcsname
    }%
  }{%
    \chardef\HOLOGO@temp=%
      \csname HoLogoOpt@discretionarybreak@\HOLOGO@name\endcsname
  }%
  \ifcase\HOLOGO@temp
  \else
    \-%
  \fi
}
%    \end{macrocode}
%    \end{macro}
%
%    \begin{macro}{\HOLOGO@mbox}
%    \begin{macrocode}
\def\HOLOGO@mbox#1{%
  \ltx@ifundefined{HoLogoOpt@break@\HOLOGO@name}{%
    \chardef\HOLOGO@temp=\HOLOGOOPT@hyphenbreak
  }{%
    \chardef\HOLOGO@temp=%
      \csname HoLogoOpt@break@\HOLOGO@name\endcsname
  }%
  \ifcase\HOLOGO@temp
    \ltx@mbox{#1}%
  \else
    #1%
  \fi
}
%    \end{macrocode}
%    \end{macro}
%
% \subsection{Font support}
%
%    \begin{macro}{\HoLogoFont@font}
%    \begin{tabular}{@{}ll@{}}
%    |#1|:& logo name\\
%    |#2|:& font short name\\
%    |#3|:& text
%    \end{tabular}
%    \begin{macrocode}
\def\HoLogoFont@font#1#2#3{%
  \begingroup
    \ltx@IfUndefined{HoLogoFont@logo@#1.#2}{%
      \ltx@IfUndefined{HoLogoFont@font@#2}{%
        \@PackageWarning{hologo}{%
          Missing font `#2' for logo `#1'%
        }%
        #3%
      }{%
        \csname HoLogoFont@font@#2\endcsname{#3}%
      }%
    }{%
      \csname HoLogoFont@logo@#1.#2\endcsname{#3}%
    }%
  \endgroup
}
%    \end{macrocode}
%    \end{macro}
%
%    \begin{macro}{\HoLogoFont@Def}
%    \begin{macrocode}
\def\HoLogoFont@Def#1{%
  \expandafter\def\csname HoLogoFont@font@#1\endcsname
}
%    \end{macrocode}
%    \end{macro}
%    \begin{macro}{\HoLogoFont@LogoDef}
%    \begin{macrocode}
\def\HoLogoFont@LogoDef#1#2{%
  \expandafter\def\csname HoLogoFont@logo@#1.#2\endcsname
}
%    \end{macrocode}
%    \end{macro}
%
% \subsubsection{Font defaults}
%
%    \begin{macro}{\HoLogoFont@font@general}
%    \begin{macrocode}
\HoLogoFont@Def{general}{}%
%    \end{macrocode}
%    \end{macro}
%
%    \begin{macro}{\HoLogoFont@font@rm}
%    \begin{macrocode}
\ltx@IfUndefined{rmfamily}{%
  \ltx@IfUndefined{rm}{%
  }{%
    \HoLogoFont@Def{rm}{\rm}%
  }%
}{%
  \HoLogoFont@Def{rm}{\rmfamily}%
}
%    \end{macrocode}
%    \end{macro}
%
%    \begin{macro}{\HoLogoFont@font@sf}
%    \begin{macrocode}
\ltx@IfUndefined{sffamily}{%
  \ltx@IfUndefined{sf}{%
  }{%
    \HoLogoFont@Def{sf}{\sf}%
  }%
}{%
  \HoLogoFont@Def{sf}{\sffamily}%
}
%    \end{macrocode}
%    \end{macro}
%
%    \begin{macro}{\HoLogoFont@font@bibsf}
%    In case of \hologo{plainTeX} the original small caps
%    variant is used as default. In \hologo{LaTeX}
%    the definition of package \xpackage{dtklogos} \cite{dtklogos}
%    is used.
%\begin{quote}
%\begin{verbatim}
%\DeclareRobustCommand{\BibTeX}{%
%  B%
%  \kern-.05em%
%  \hbox{%
%    $\m@th$% %% force math size calculations
%    \csname S@\f@size\endcsname
%    \fontsize\sf@size\z@
%    \math@fontsfalse
%    \selectfont
%    I%
%    \kern-.025em%
%    B
%  }%
%  \kern-.08em%
%  \-%
%  \TeX
%}
%\end{verbatim}
%\end{quote}
%    \begin{macrocode}
\ltx@IfUndefined{selectfont}{%
  \ltx@IfUndefined{tensc}{%
    \font\tensc=cmcsc10\relax
  }{}%
  \HoLogoFont@Def{bibsf}{\tensc}%
}{%
  \HoLogoFont@Def{bibsf}{%
    $\mathsurround=0pt$%
    \csname S@\f@size\endcsname
    \fontsize\sf@size{0pt}%
    \math@fontsfalse
    \selectfont
  }%
}
%    \end{macrocode}
%    \end{macro}
%
%    \begin{macro}{\HoLogoFont@font@sc}
%    \begin{macrocode}
\ltx@IfUndefined{scshape}{%
  \ltx@IfUndefined{tensc}{%
    \font\tensc=cmcsc10\relax
  }{}%
  \HoLogoFont@Def{sc}{\tensc}%
}{%
  \HoLogoFont@Def{sc}{\scshape}%
}
%    \end{macrocode}
%    \end{macro}
%
%    \begin{macro}{\HoLogoFont@font@sy}
%    \begin{macrocode}
\ltx@IfUndefined{usefont}{%
  \ltx@IfUndefined{tensy}{%
  }{%
    \HoLogoFont@Def{sy}{\tensy}%
  }%
}{%
  \HoLogoFont@Def{sy}{%
    \usefont{OMS}{cmsy}{m}{n}%
  }%
}
%    \end{macrocode}
%    \end{macro}
%
%    \begin{macro}{\HoLogoFont@font@logo}
%    \begin{macrocode}
\begingroup
  \def\x{LaTeX2e}%
\expandafter\endgroup
\ifx\fmtname\x
  \ltx@IfUndefined{logofamily}{%
    \DeclareRobustCommand\logofamily{%
      \not@math@alphabet\logofamily\relax
      \fontencoding{U}%
      \fontfamily{logo}%
      \selectfont
    }%
  }{}%
  \ltx@IfUndefined{logofamily}{%
  }{%
    \HoLogoFont@Def{logo}{\logofamily}%
  }%
\else
  \ltx@IfUndefined{tenlogo}{%
    \font\tenlogo=logo10\relax
  }{}%
  \HoLogoFont@Def{logo}{\tenlogo}%
\fi
%    \end{macrocode}
%    \end{macro}
%
% \subsubsection{Font setup}
%
%    \begin{macro}{\hologoFontSetup}
%    \begin{macrocode}
\def\hologoFontSetup{%
  \let\HOLOGO@name\relax
  \HOLOGO@FontSetup
}
%    \end{macrocode}
%    \end{macro}
%
%    \begin{macro}{\hologoLogoFontSetup}
%    \begin{macrocode}
\def\hologoLogoFontSetup#1{%
  \edef\HOLOGO@name{#1}%
  \ltx@IfUndefined{HoLogo@\HOLOGO@name}{%
    \@PackageError{hologo}{%
      Unknown logo `\HOLOGO@name'%
    }\@ehc
    \ltx@gobble
  }{%
    \HOLOGO@FontSetup
  }%
}
%    \end{macrocode}
%    \end{macro}
%
%    \begin{macro}{\HOLOGO@FontSetup}
%    \begin{macrocode}
\def\HOLOGO@FontSetup{%
  \kvsetkeys{HoLogoFont}%
}
%    \end{macrocode}
%    \end{macro}
%
%    \begin{macrocode}
\def\HOLOGO@temp#1{%
  \kv@define@key{HoLogoFont}{#1}{%
    \ifx\HOLOGO@name\relax
      \HoLogoFont@Def{#1}{##1}%
    \else
      \HoLogoFont@LogoDef\HOLOGO@name{#1}{##1}%
    \fi
  }%
}
\HOLOGO@temp{general}
\HOLOGO@temp{sf}
%    \end{macrocode}
%
% \subsection{Generic logo commands}
%
%    \begin{macrocode}
\HOLOGO@IfExists\hologo{%
  \@PackageError{hologo}{%
    \string\hologo\ltx@space is already defined.\MessageBreak
    Package loading is aborted%
  }\@ehc
  \HOLOGO@AtEnd
}%
\HOLOGO@IfExists\hologoRobust{%
  \@PackageError{hologo}{%
    \string\hologoRobust\ltx@space is already defined.\MessageBreak
    Package loading is aborted%
  }\@ehc
  \HOLOGO@AtEnd
}%
%    \end{macrocode}
%
% \subsubsection{\cs{hologo} and friends}
%
%    \begin{macrocode}
\ifluatex
  \expandafter\ltx@firstofone
\else
  \expandafter\ltx@gobble
\fi
{%
  \ltx@IfUndefined{ifincsname}{%
    \ifnum\luatexversion<36 %
      \expandafter\ltx@gobble
    \else
      \expandafter\ltx@firstofone
    \fi
    {%
      \begingroup
        \ifcase0%
            \directlua{%
              if tex.enableprimitives then %
                tex.enableprimitives('HOLOGO@', {'ifincsname'})%
              else %
                tex.print('1')%
              end%
            }%
            \ifx\HOLOGO@ifincsname\@undefined 1\fi%
            \relax
          \expandafter\ltx@firstofone
        \else
          \endgroup
          \expandafter\ltx@gobble
        \fi
        {%
          \global\let\ifincsname\HOLOGO@ifincsname
        }%
      \HOLOGO@temp
    }%
  }{}%
}
%    \end{macrocode}
%    \begin{macrocode}
\ltx@IfUndefined{ifincsname}{%
  \catcode`$=14 %
}{%
  \catcode`$=9 %
}
%    \end{macrocode}
%
%    \begin{macro}{\hologo}
%    \begin{macrocode}
\def\hologo#1{%
$ \ifincsname
$   \ltx@ifundefined{HoLogoCs@\HOLOGO@Variant{#1}}{%
$     #1%
$   }{%
$     \csname HoLogoCs@\HOLOGO@Variant{#1}\endcsname\ltx@firstoftwo
$   }%
$ \else
    \HOLOGO@IfExists\texorpdfstring\texorpdfstring\ltx@firstoftwo
    {%
      \hologoRobust{#1}%
    }{%
      \ltx@ifundefined{HoLogoBkm@\HOLOGO@Variant{#1}}{%
        \ltx@ifundefined{HoLogo@#1}{?#1?}{#1}%
      }{%
        \csname HoLogoBkm@\HOLOGO@Variant{#1}\endcsname
        \ltx@firstoftwo
      }%
    }%
$ \fi
}
%    \end{macrocode}
%    \end{macro}
%    \begin{macro}{\Hologo}
%    \begin{macrocode}
\def\Hologo#1{%
$ \ifincsname
$   \ltx@ifundefined{HoLogoCs@\HOLOGO@Variant{#1}}{%
$     #1%
$   }{%
$     \csname HoLogoCs@\HOLOGO@Variant{#1}\endcsname\ltx@secondoftwo
$   }%
$ \else
    \HOLOGO@IfExists\texorpdfstring\texorpdfstring\ltx@firstoftwo
    {%
      \HologoRobust{#1}%
    }{%
      \ltx@ifundefined{HoLogoBkm@\HOLOGO@Variant{#1}}{%
        \ltx@ifundefined{HoLogo@#1}{?#1?}{#1}%
      }{%
        \csname HoLogoBkm@\HOLOGO@Variant{#1}\endcsname
        \ltx@secondoftwo
      }%
    }%
$ \fi
}
%    \end{macrocode}
%    \end{macro}
%
%    \begin{macro}{\hologoVariant}
%    \begin{macrocode}
\def\hologoVariant#1#2{%
  \ifx\relax#2\relax
    \hologo{#1}%
  \else
$   \ifincsname
$     \ltx@ifundefined{HoLogoCs@#1@#2}{%
$       #1%
$     }{%
$       \csname HoLogoCs@#1@#2\endcsname\ltx@firstoftwo
$     }%
$   \else
      \HOLOGO@IfExists\texorpdfstring\texorpdfstring\ltx@firstoftwo
      {%
        \hologoVariantRobust{#1}{#2}%
      }{%
        \ltx@ifundefined{HoLogoBkm@#1@#2}{%
          \ltx@ifundefined{HoLogo@#1}{?#1?}{#1}%
        }{%
          \csname HoLogoBkm@#1@#2\endcsname
          \ltx@firstoftwo
        }%
      }%
$   \fi
  \fi
}
%    \end{macrocode}
%    \end{macro}
%    \begin{macro}{\HologoVariant}
%    \begin{macrocode}
\def\HologoVariant#1#2{%
  \ifx\relax#2\relax
    \Hologo{#1}%
  \else
$   \ifincsname
$     \ltx@ifundefined{HoLogoCs@#1@#2}{%
$       #1%
$     }{%
$       \csname HoLogoCs@#1@#2\endcsname\ltx@secondoftwo
$     }%
$   \else
      \HOLOGO@IfExists\texorpdfstring\texorpdfstring\ltx@firstoftwo
      {%
        \HologoVariantRobust{#1}{#2}%
      }{%
        \ltx@ifundefined{HoLogoBkm@#1@#2}{%
          \ltx@ifundefined{HoLogo@#1}{?#1?}{#1}%
        }{%
          \csname HoLogoBkm@#1@#2\endcsname
          \ltx@secondoftwo
        }%
      }%
$   \fi
  \fi
}
%    \end{macrocode}
%    \end{macro}
%
%    \begin{macrocode}
\catcode`\$=3 %
%    \end{macrocode}
%
% \subsubsection{\cs{hologoRobust} and friends}
%
%    \begin{macro}{\hologoRobust}
%    \begin{macrocode}
\ltx@IfUndefined{protected}{%
  \ltx@IfUndefined{DeclareRobustCommand}{%
    \def\hologoRobust#1%
  }{%
    \DeclareRobustCommand*\hologoRobust[1]%
  }%
}{%
  \protected\def\hologoRobust#1%
}%
{%
  \edef\HOLOGO@name{#1}%
  \ltx@IfUndefined{HoLogo@\HOLOGO@Variant\HOLOGO@name}{%
    \@PackageError{hologo}{%
      Unknown logo `\HOLOGO@name'%
    }\@ehc
    ?\HOLOGO@name?%
  }{%
    \ltx@IfUndefined{ver@tex4ht.sty}{%
      \HoLogoFont@font\HOLOGO@name{general}{%
        \csname HoLogo@\HOLOGO@Variant\HOLOGO@name\endcsname
        \ltx@firstoftwo
      }%
    }{%
      \ltx@IfUndefined{HoLogoHtml@\HOLOGO@Variant\HOLOGO@name}{%
        \HOLOGO@name
      }{%
        \csname HoLogoHtml@\HOLOGO@Variant\HOLOGO@name\endcsname
        \ltx@firstoftwo
      }%
    }%
  }%
}
%    \end{macrocode}
%    \end{macro}
%    \begin{macro}{\HologoRobust}
%    \begin{macrocode}
\ltx@IfUndefined{protected}{%
  \ltx@IfUndefined{DeclareRobustCommand}{%
    \def\HologoRobust#1%
  }{%
    \DeclareRobustCommand*\HologoRobust[1]%
  }%
}{%
  \protected\def\HologoRobust#1%
}%
{%
  \edef\HOLOGO@name{#1}%
  \ltx@IfUndefined{HoLogo@\HOLOGO@Variant\HOLOGO@name}{%
    \@PackageError{hologo}{%
      Unknown logo `\HOLOGO@name'%
    }\@ehc
    ?\HOLOGO@name?%
  }{%
    \ltx@IfUndefined{ver@tex4ht.sty}{%
      \HoLogoFont@font\HOLOGO@name{general}{%
        \csname HoLogo@\HOLOGO@Variant\HOLOGO@name\endcsname
        \ltx@secondoftwo
      }%
    }{%
      \ltx@IfUndefined{HoLogoHtml@\HOLOGO@Variant\HOLOGO@name}{%
        \expandafter\HOLOGO@Uppercase\HOLOGO@name
      }{%
        \csname HoLogoHtml@\HOLOGO@Variant\HOLOGO@name\endcsname
        \ltx@secondoftwo
      }%
    }%
  }%
}
%    \end{macrocode}
%    \end{macro}
%    \begin{macro}{\hologoVariantRobust}
%    \begin{macrocode}
\ltx@IfUndefined{protected}{%
  \ltx@IfUndefined{DeclareRobustCommand}{%
    \def\hologoVariantRobust#1#2%
  }{%
    \DeclareRobustCommand*\hologoVariantRobust[2]%
  }%
}{%
  \protected\def\hologoVariantRobust#1#2%
}%
{%
  \begingroup
    \hologoLogoSetup{#1}{variant={#2}}%
    \hologoRobust{#1}%
  \endgroup
}
%    \end{macrocode}
%    \end{macro}
%    \begin{macro}{\HologoVariantRobust}
%    \begin{macrocode}
\ltx@IfUndefined{protected}{%
  \ltx@IfUndefined{DeclareRobustCommand}{%
    \def\HologoVariantRobust#1#2%
  }{%
    \DeclareRobustCommand*\HologoVariantRobust[2]%
  }%
}{%
  \protected\def\HologoVariantRobust#1#2%
}%
{%
  \begingroup
    \hologoLogoSetup{#1}{variant={#2}}%
    \HologoRobust{#1}%
  \endgroup
}
%    \end{macrocode}
%    \end{macro}
%
%    \begin{macro}{\hologorobust}
%    Macro \cs{hologorobust} is only defined for compatibility.
%    Its use is deprecated.
%    \begin{macrocode}
\def\hologorobust{\hologoRobust}
%    \end{macrocode}
%    \end{macro}
%
% \subsection{Helpers}
%
%    \begin{macro}{\HOLOGO@Uppercase}
%    Macro \cs{HOLOGO@Uppercase} is restricted to \cs{uppercase},
%    because \hologo{plainTeX} or \hologo{iniTeX} do not provide
%    \cs{MakeUppercase}.
%    \begin{macrocode}
\def\HOLOGO@Uppercase#1{\uppercase{#1}}
%    \end{macrocode}
%    \end{macro}
%
%    \begin{macro}{\HOLOGO@PdfdocUnicode}
%    \begin{macrocode}
\def\HOLOGO@PdfdocUnicode{%
  \ifx\ifHy@unicode\iftrue
    \expandafter\ltx@secondoftwo
  \else
    \expandafter\ltx@firstoftwo
  \fi
}
%    \end{macrocode}
%    \end{macro}
%
%    \begin{macro}{\HOLOGO@Math}
%    \begin{macrocode}
\def\HOLOGO@MathSetup{%
  \mathsurround0pt\relax
  \HOLOGO@IfExists\f@series{%
    \if b\expandafter\ltx@car\f@series x\@nil
      \csname boldmath\endcsname
   \fi
  }{}%
}
%    \end{macrocode}
%    \end{macro}
%
%    \begin{macro}{\HOLOGO@TempDimen}
%    \begin{macrocode}
\dimendef\HOLOGO@TempDimen=\ltx@zero
%    \end{macrocode}
%    \end{macro}
%    \begin{macro}{\HOLOGO@NegativeKerning}
%    \begin{macrocode}
\def\HOLOGO@NegativeKerning#1{%
  \begingroup
    \HOLOGO@TempDimen=0pt\relax
    \comma@parse@normalized{#1}{%
      \ifdim\HOLOGO@TempDimen=0pt %
        \expandafter\HOLOGO@@NegativeKerning\comma@entry
      \fi
      \ltx@gobble
    }%
    \ifdim\HOLOGO@TempDimen<0pt %
      \kern\HOLOGO@TempDimen
    \fi
  \endgroup
}
%    \end{macrocode}
%    \end{macro}
%    \begin{macro}{\HOLOGO@@NegativeKerning}
%    \begin{macrocode}
\def\HOLOGO@@NegativeKerning#1#2{%
  \setbox\ltx@zero\hbox{#1#2}%
  \HOLOGO@TempDimen=\wd\ltx@zero
  \setbox\ltx@zero\hbox{#1\kern0pt#2}%
  \advance\HOLOGO@TempDimen by -\wd\ltx@zero
}
%    \end{macrocode}
%    \end{macro}
%
%    \begin{macro}{\HOLOGO@SpaceFactor}
%    \begin{macrocode}
\def\HOLOGO@SpaceFactor{%
  \spacefactor1000 %
}
%    \end{macrocode}
%    \end{macro}
%
%    \begin{macro}{\HOLOGO@Span}
%    \begin{macrocode}
\def\HOLOGO@Span#1#2{%
  \HCode{<span class="HoLogo-#1">}%
  #2%
  \HCode{</span>}%
}
%    \end{macrocode}
%    \end{macro}
%
% \subsubsection{Text subscript}
%
%    \begin{macro}{\HOLOGO@SubScript}%
%    \begin{macrocode}
\def\HOLOGO@SubScript#1{%
  \ltx@IfUndefined{textsubscript}{%
    \ltx@IfUndefined{text}{%
      \ltx@mbox{%
        \mathsurround=0pt\relax
        $%
          _{%
            \ltx@IfUndefined{sf@size}{%
              \mathrm{#1}%
            }{%
              \mbox{%
                \fontsize\sf@size{0pt}\selectfont
                #1%
              }%
            }%
          }%
        $%
      }%
    }{%
      \ltx@mbox{%
        \mathsurround=0pt\relax
        $_{\text{#1}}$%
      }%
    }%
  }{%
    \textsubscript{#1}%
  }%
}
%    \end{macrocode}
%    \end{macro}
%
% \subsection{\hologo{TeX} and friends}
%
% \subsubsection{\hologo{TeX}}
%
%    \begin{macro}{\HoLogo@TeX}
%    Source: \hologo{LaTeX} kernel.
%    \begin{macrocode}
\def\HoLogo@TeX#1{%
  T\kern-.1667em\lower.5ex\hbox{E}\kern-.125emX\HOLOGO@SpaceFactor
}
%    \end{macrocode}
%    \end{macro}
%    \begin{macro}{\HoLogoHtml@TeX}
%    \begin{macrocode}
\def\HoLogoHtml@TeX#1{%
  \HoLogoCss@TeX
  \HOLOGO@Span{TeX}{%
    T%
    \HOLOGO@Span{e}{%
      E%
    }%
    X%
  }%
}
%    \end{macrocode}
%    \end{macro}
%    \begin{macro}{\HoLogoCss@TeX}
%    \begin{macrocode}
\def\HoLogoCss@TeX{%
  \Css{%
    span.HoLogo-TeX span.HoLogo-e{%
      position:relative;%
      top:.5ex;%
      margin-left:-.1667em;%
      margin-right:-.125em;%
    }%
  }%
  \Css{%
    a span.HoLogo-TeX span.HoLogo-e{%
      text-decoration:none;%
    }%
  }%
  \global\let\HoLogoCss@TeX\relax
}
%    \end{macrocode}
%    \end{macro}
%
% \subsubsection{\hologo{plainTeX}}
%
%    \begin{macro}{\HoLogo@plainTeX@space}
%    Source: ``The \hologo{TeX}book''
%    \begin{macrocode}
\def\HoLogo@plainTeX@space#1{%
  \HOLOGO@mbox{#1{p}{P}lain}\HOLOGO@space\hologo{TeX}%
}
%    \end{macrocode}
%    \end{macro}
%    \begin{macro}{\HoLogoCs@plainTeX@space}
%    \begin{macrocode}
\def\HoLogoCs@plainTeX@space#1{#1{p}{P}lain TeX}%
%    \end{macrocode}
%    \end{macro}
%    \begin{macro}{\HoLogoBkm@plainTeX@space}
%    \begin{macrocode}
\def\HoLogoBkm@plainTeX@space#1{%
  #1{p}{P}lain \hologo{TeX}%
}
%    \end{macrocode}
%    \end{macro}
%    \begin{macro}{\HoLogoHtml@plainTeX@space}
%    \begin{macrocode}
\def\HoLogoHtml@plainTeX@space#1{%
  #1{p}{P}lain \hologo{TeX}%
}
%    \end{macrocode}
%    \end{macro}
%
%    \begin{macro}{\HoLogo@plainTeX@hyphen}
%    \begin{macrocode}
\def\HoLogo@plainTeX@hyphen#1{%
  \HOLOGO@mbox{#1{p}{P}lain}\HOLOGO@hyphen\hologo{TeX}%
}
%    \end{macrocode}
%    \end{macro}
%    \begin{macro}{\HoLogoCs@plainTeX@hyphen}
%    \begin{macrocode}
\def\HoLogoCs@plainTeX@hyphen#1{#1{p}{P}lain-TeX}
%    \end{macrocode}
%    \end{macro}
%    \begin{macro}{\HoLogoBkm@plainTeX@hyphen}
%    \begin{macrocode}
\def\HoLogoBkm@plainTeX@hyphen#1{%
  #1{p}{P}lain-\hologo{TeX}%
}
%    \end{macrocode}
%    \end{macro}
%    \begin{macro}{\HoLogoHtml@plainTeX@hyphen}
%    \begin{macrocode}
\def\HoLogoHtml@plainTeX@hyphen#1{%
  #1{p}{P}lain-\hologo{TeX}%
}
%    \end{macrocode}
%    \end{macro}
%
%    \begin{macro}{\HoLogo@plainTeX@runtogether}
%    \begin{macrocode}
\def\HoLogo@plainTeX@runtogether#1{%
  \HOLOGO@mbox{#1{p}{P}lain\hologo{TeX}}%
}
%    \end{macrocode}
%    \end{macro}
%    \begin{macro}{\HoLogoCs@plainTeX@runtogether}
%    \begin{macrocode}
\def\HoLogoCs@plainTeX@runtogether#1{#1{p}{P}lainTeX}
%    \end{macrocode}
%    \end{macro}
%    \begin{macro}{\HoLogoBkm@plainTeX@runtogether}
%    \begin{macrocode}
\def\HoLogoBkm@plainTeX@runtogether#1{%
  #1{p}{P}lain\hologo{TeX}%
}
%    \end{macrocode}
%    \end{macro}
%    \begin{macro}{\HoLogoHtml@plainTeX@runtogether}
%    \begin{macrocode}
\def\HoLogoHtml@plainTeX@runtogether#1{%
  #1{p}{P}lain\hologo{TeX}%
}
%    \end{macrocode}
%    \end{macro}
%
%    \begin{macro}{\HoLogo@plainTeX}
%    \begin{macrocode}
\def\HoLogo@plainTeX{\HoLogo@plainTeX@space}
%    \end{macrocode}
%    \end{macro}
%    \begin{macro}{\HoLogoCs@plainTeX}
%    \begin{macrocode}
\def\HoLogoCs@plainTeX{\HoLogoCs@plainTeX@space}
%    \end{macrocode}
%    \end{macro}
%    \begin{macro}{\HoLogoBkm@plainTeX}
%    \begin{macrocode}
\def\HoLogoBkm@plainTeX{\HoLogoBkm@plainTeX@space}
%    \end{macrocode}
%    \end{macro}
%    \begin{macro}{\HoLogoHtml@plainTeX}
%    \begin{macrocode}
\def\HoLogoHtml@plainTeX{\HoLogoHtml@plainTeX@space}
%    \end{macrocode}
%    \end{macro}
%
% \subsubsection{\hologo{LaTeX}}
%
%    Source: \hologo{LaTeX} kernel.
%\begin{quote}
%\begin{verbatim}
%\DeclareRobustCommand{\LaTeX}{%
%  L%
%  \kern-.36em%
%  {%
%    \sbox\z@ T%
%    \vbox to\ht\z@{%
%      \hbox{%
%        \check@mathfonts
%        \fontsize\sf@size\z@
%        \math@fontsfalse
%        \selectfont
%        A%
%      }%
%      \vss
%    }%
%  }%
%  \kern-.15em%
%  \TeX
%}
%\end{verbatim}
%\end{quote}
%
%    \begin{macro}{\HoLogo@La}
%    \begin{macrocode}
\def\HoLogo@La#1{%
  L%
  \kern-.36em%
  \begingroup
    \setbox\ltx@zero\hbox{T}%
    \vbox to\ht\ltx@zero{%
      \hbox{%
        \ltx@ifundefined{check@mathfonts}{%
          \csname sevenrm\endcsname
        }{%
          \check@mathfonts
          \fontsize\sf@size{0pt}%
          \math@fontsfalse\selectfont
        }%
        A%
      }%
      \vss
    }%
  \endgroup
}
%    \end{macrocode}
%    \end{macro}
%
%    \begin{macro}{\HoLogo@LaTeX}
%    Source: \hologo{LaTeX} kernel.
%    \begin{macrocode}
\def\HoLogo@LaTeX#1{%
  \hologo{La}%
  \kern-.15em%
  \hologo{TeX}%
}
%    \end{macrocode}
%    \end{macro}
%    \begin{macro}{\HoLogoHtml@LaTeX}
%    \begin{macrocode}
\def\HoLogoHtml@LaTeX#1{%
  \HoLogoCss@LaTeX
  \HOLOGO@Span{LaTeX}{%
    L%
    \HOLOGO@Span{a}{%
      A%
    }%
    \hologo{TeX}%
  }%
}
%    \end{macrocode}
%    \end{macro}
%    \begin{macro}{\HoLogoCss@LaTeX}
%    \begin{macrocode}
\def\HoLogoCss@LaTeX{%
  \Css{%
    span.HoLogo-LaTeX span.HoLogo-a{%
      position:relative;%
      top:-.5ex;%
      margin-left:-.36em;%
      margin-right:-.15em;%
      font-size:85\%;%
    }%
  }%
  \global\let\HoLogoCss@LaTeX\relax
}
%    \end{macrocode}
%    \end{macro}
%
% \subsubsection{\hologo{(La)TeX}}
%
%    \begin{macro}{\HoLogo@LaTeXTeX}
%    The kerning around the parentheses is taken
%    from package \xpackage{dtklogos} \cite{dtklogos}.
%\begin{quote}
%\begin{verbatim}
%\DeclareRobustCommand{\LaTeXTeX}{%
%  (%
%  \kern-.15em%
%  L%
%  \kern-.36em%
%  {%
%    \sbox\z@ T%
%    \vbox to\ht0{%
%      \hbox{%
%        $\m@th$%
%        \csname S@\f@size\endcsname
%        \fontsize\sf@size\z@
%        \math@fontsfalse
%        \selectfont
%        A%
%      }%
%      \vss
%    }%
%  }%
%  \kern-.2em%
%  )%
%  \kern-.15em%
%  \TeX
%}
%\end{verbatim}
%\end{quote}
%    \begin{macrocode}
\def\HoLogo@LaTeXTeX#1{%
  (%
  \kern-.15em%
  \hologo{La}%
  \kern-.2em%
  )%
  \kern-.15em%
  \hologo{TeX}%
}
%    \end{macrocode}
%    \end{macro}
%    \begin{macro}{\HoLogoBkm@LaTeXTeX}
%    \begin{macrocode}
\def\HoLogoBkm@LaTeXTeX#1{(La)TeX}
%    \end{macrocode}
%    \end{macro}
%
%    \begin{macro}{\HoLogo@(La)TeX}
%    \begin{macrocode}
\expandafter
\let\csname HoLogo@(La)TeX\endcsname\HoLogo@LaTeXTeX
%    \end{macrocode}
%    \end{macro}
%    \begin{macro}{\HoLogoBkm@(La)TeX}
%    \begin{macrocode}
\expandafter
\let\csname HoLogoBkm@(La)TeX\endcsname\HoLogoBkm@LaTeXTeX
%    \end{macrocode}
%    \end{macro}
%    \begin{macro}{\HoLogoHtml@LaTeXTeX}
%    \begin{macrocode}
\def\HoLogoHtml@LaTeXTeX#1{%
  \HoLogoCss@LaTeXTeX
  \HOLOGO@Span{LaTeXTeX}{%
    (%
    \HOLOGO@Span{L}{L}%
    \HOLOGO@Span{a}{A}%
    \HOLOGO@Span{ParenRight}{)}%
    \hologo{TeX}%
  }%
}
%    \end{macrocode}
%    \end{macro}
%    \begin{macro}{\HoLogoHtml@(La)TeX}
%    Kerning after opening parentheses and before closing parentheses
%    is $-0.1$\,em. The original values $-0.15$\,em
%    looked too ugly for a serif font.
%    \begin{macrocode}
\expandafter
\let\csname HoLogoHtml@(La)TeX\endcsname\HoLogoHtml@LaTeXTeX
%    \end{macrocode}
%    \end{macro}
%    \begin{macro}{\HoLogoCss@LaTeXTeX}
%    \begin{macrocode}
\def\HoLogoCss@LaTeXTeX{%
  \Css{%
    span.HoLogo-LaTeXTeX span.HoLogo-L{%
      margin-left:-.1em;%
    }%
  }%
  \Css{%
    span.HoLogo-LaTeXTeX span.HoLogo-a{%
      position:relative;%
      top:-.5ex;%
      margin-left:-.36em;%
      margin-right:-.1em;%
      font-size:85\%;%
    }%
  }%
  \Css{%
    span.HoLogo-LaTeXTeX span.HoLogo-ParenRight{%
      margin-right:-.15em;%
    }%
  }%
  \global\let\HoLogoCss@LaTeXTeX\relax
}
%    \end{macrocode}
%    \end{macro}
%
% \subsubsection{\hologo{LaTeXe}}
%
%    \begin{macro}{\HoLogo@LaTeXe}
%    Source: \hologo{LaTeX} kernel
%    \begin{macrocode}
\def\HoLogo@LaTeXe#1{%
  \hologo{LaTeX}%
  \kern.15em%
  \hbox{%
    \HOLOGO@MathSetup
    2%
    $_{\textstyle\varepsilon}$%
  }%
}
%    \end{macrocode}
%    \end{macro}
%
%    \begin{macro}{\HoLogoCs@LaTeXe}
%    \begin{macrocode}
\ifnum64=`\^^^^0040\relax % test for big chars of LuaTeX/XeTeX
  \catcode`\$=9 %
  \catcode`\&=14 %
\else
  \catcode`\$=14 %
  \catcode`\&=9 %
\fi
\def\HoLogoCs@LaTeXe#1{%
  LaTeX2%
$ \string ^^^^0395%
& e%
}%
\catcode`\$=3 %
\catcode`\&=4 %
%    \end{macrocode}
%    \end{macro}
%
%    \begin{macro}{\HoLogoBkm@LaTeXe}
%    \begin{macrocode}
\def\HoLogoBkm@LaTeXe#1{%
  \hologo{LaTeX}%
  2%
  \HOLOGO@PdfdocUnicode{e}{\textepsilon}%
}
%    \end{macrocode}
%    \end{macro}
%
%    \begin{macro}{\HoLogoHtml@LaTeXe}
%    \begin{macrocode}
\def\HoLogoHtml@LaTeXe#1{%
  \HoLogoCss@LaTeXe
  \HOLOGO@Span{LaTeX2e}{%
    \hologo{LaTeX}%
    \HOLOGO@Span{2}{2}%
    \HOLOGO@Span{e}{%
      \HOLOGO@MathSetup
      \ensuremath{\textstyle\varepsilon}%
    }%
  }%
}
%    \end{macrocode}
%    \end{macro}
%    \begin{macro}{\HoLogoCss@LaTeXe}
%    \begin{macrocode}
\def\HoLogoCss@LaTeXe{%
  \Css{%
    span.HoLogo-LaTeX2e span.HoLogo-2{%
      padding-left:.15em;%
    }%
  }%
  \Css{%
    span.HoLogo-LaTeX2e span.HoLogo-e{%
      position:relative;%
      top:.35ex;%
      text-decoration:none;%
    }%
  }%
  \global\let\HoLogoCss@LaTeXe\relax
}
%    \end{macrocode}
%    \end{macro}
%
%    \begin{macro}{\HoLogo@LaTeX2e}
%    \begin{macrocode}
\expandafter
\let\csname HoLogo@LaTeX2e\endcsname\HoLogo@LaTeXe
%    \end{macrocode}
%    \end{macro}
%    \begin{macro}{\HoLogoCs@LaTeX2e}
%    \begin{macrocode}
\expandafter
\let\csname HoLogoCs@LaTeX2e\endcsname\HoLogoCs@LaTeXe
%    \end{macrocode}
%    \end{macro}
%    \begin{macro}{\HoLogoBkm@LaTeX2e}
%    \begin{macrocode}
\expandafter
\let\csname HoLogoBkm@LaTeX2e\endcsname\HoLogoBkm@LaTeXe
%    \end{macrocode}
%    \end{macro}
%    \begin{macro}{\HoLogoHtml@LaTeX2e}
%    \begin{macrocode}
\expandafter
\let\csname HoLogoHtml@LaTeX2e\endcsname\HoLogoHtml@LaTeXe
%    \end{macrocode}
%    \end{macro}
%
% \subsubsection{\hologo{LaTeX3}}
%
%    \begin{macro}{\HoLogo@LaTeX3}
%    Source: \hologo{LaTeX} kernel
%    \begin{macrocode}
\expandafter\def\csname HoLogo@LaTeX3\endcsname#1{%
  \hologo{LaTeX}%
  3%
}
%    \end{macrocode}
%    \end{macro}
%
%    \begin{macro}{\HoLogoBkm@LaTeX3}
%    \begin{macrocode}
\expandafter\def\csname HoLogoBkm@LaTeX3\endcsname#1{%
  \hologo{LaTeX}%
  3%
}
%    \end{macrocode}
%    \end{macro}
%    \begin{macro}{\HoLogoHtml@LaTeX3}
%    \begin{macrocode}
\expandafter
\let\csname HoLogoHtml@LaTeX3\expandafter\endcsname
\csname HoLogo@LaTeX3\endcsname
%    \end{macrocode}
%    \end{macro}
%
% \subsubsection{\hologo{LaTeXML}}
%
%    \begin{macro}{\HoLogo@LaTeXML}
%    \begin{macrocode}
\def\HoLogo@LaTeXML#1{%
  \HOLOGO@mbox{%
    \hologo{La}%
    \kern-.15em%
    T%
    \kern-.1667em%
    \lower.5ex\hbox{E}%
    \kern-.125em%
    \HoLogoFont@font{LaTeXML}{sc}{xml}%
  }%
}
%    \end{macrocode}
%    \end{macro}
%    \begin{macro}{\HoLogoHtml@pdfLaTeX}
%    \begin{macrocode}
\def\HoLogoHtml@LaTeXML#1{%
  \HOLOGO@Span{LaTeXML}{%
    \HoLogoCss@LaTeX
    \HoLogoCss@TeX
    \HOLOGO@Span{LaTeX}{%
      L%
      \HOLOGO@Span{a}{%
        A%
      }%
    }%
    \HOLOGO@Span{TeX}{%
      T%
      \HOLOGO@Span{e}{%
        E%
      }%
    }%
    \HCode{<span style="font-variant: small-caps;">}%
    xml%
    \HCode{</span>}%
  }%
}
%    \end{macrocode}
%    \end{macro}
%
% \subsubsection{\hologo{eTeX}}
%
%    \begin{macro}{\HoLogo@eTeX}
%    Source: package \xpackage{etex}
%    \begin{macrocode}
\def\HoLogo@eTeX#1{%
  \ltx@mbox{%
    \HOLOGO@MathSetup
    $\varepsilon$%
    -%
    \HOLOGO@NegativeKerning{-T,T-,To}%
    \hologo{TeX}%
  }%
}
%    \end{macrocode}
%    \end{macro}
%    \begin{macro}{\HoLogoCs@eTeX}
%    \begin{macrocode}
\ifnum64=`\^^^^0040\relax % test for big chars of LuaTeX/XeTeX
  \catcode`\$=9 %
  \catcode`\&=14 %
\else
  \catcode`\$=14 %
  \catcode`\&=9 %
\fi
\def\HoLogoCs@eTeX#1{%
$ #1{\string ^^^^0395}{\string ^^^^03b5}%
& #1{e}{E}%
  TeX%
}%
\catcode`\$=3 %
\catcode`\&=4 %
%    \end{macrocode}
%    \end{macro}
%    \begin{macro}{\HoLogoBkm@eTeX}
%    \begin{macrocode}
\def\HoLogoBkm@eTeX#1{%
  \HOLOGO@PdfdocUnicode{#1{e}{E}}{\textepsilon}%
  -%
  \hologo{TeX}%
}
%    \end{macrocode}
%    \end{macro}
%    \begin{macro}{\HoLogoHtml@eTeX}
%    \begin{macrocode}
\def\HoLogoHtml@eTeX#1{%
  \ltx@mbox{%
    \HOLOGO@MathSetup
    $\varepsilon$%
    -%
    \hologo{TeX}%
  }%
}
%    \end{macrocode}
%    \end{macro}
%
% \subsubsection{\hologo{iniTeX}}
%
%    \begin{macro}{\HoLogo@iniTeX}
%    \begin{macrocode}
\def\HoLogo@iniTeX#1{%
  \HOLOGO@mbox{%
    #1{i}{I}ni\hologo{TeX}%
  }%
}
%    \end{macrocode}
%    \end{macro}
%    \begin{macro}{\HoLogoCs@iniTeX}
%    \begin{macrocode}
\def\HoLogoCs@iniTeX#1{#1{i}{I}niTeX}
%    \end{macrocode}
%    \end{macro}
%    \begin{macro}{\HoLogoBkm@iniTeX}
%    \begin{macrocode}
\def\HoLogoBkm@iniTeX#1{%
  #1{i}{I}ni\hologo{TeX}%
}
%    \end{macrocode}
%    \end{macro}
%    \begin{macro}{\HoLogoHtml@iniTeX}
%    \begin{macrocode}
\let\HoLogoHtml@iniTeX\HoLogo@iniTeX
%    \end{macrocode}
%    \end{macro}
%
% \subsubsection{\hologo{virTeX}}
%
%    \begin{macro}{\HoLogo@virTeX}
%    \begin{macrocode}
\def\HoLogo@virTeX#1{%
  \HOLOGO@mbox{%
    #1{v}{V}ir\hologo{TeX}%
  }%
}
%    \end{macrocode}
%    \end{macro}
%    \begin{macro}{\HoLogoCs@virTeX}
%    \begin{macrocode}
\def\HoLogoCs@virTeX#1{#1{v}{V}irTeX}
%    \end{macrocode}
%    \end{macro}
%    \begin{macro}{\HoLogoBkm@virTeX}
%    \begin{macrocode}
\def\HoLogoBkm@virTeX#1{%
  #1{v}{V}ir\hologo{TeX}%
}
%    \end{macrocode}
%    \end{macro}
%    \begin{macro}{\HoLogoHtml@virTeX}
%    \begin{macrocode}
\let\HoLogoHtml@virTeX\HoLogo@virTeX
%    \end{macrocode}
%    \end{macro}
%
% \subsubsection{\hologo{SliTeX}}
%
% \paragraph{Definitions of the three variants.}
%
%    \begin{macro}{\HoLogo@SLiTeX@lift}
%    \begin{macrocode}
\def\HoLogo@SLiTeX@lift#1{%
  \HoLogoFont@font{SliTeX}{rm}{%
    S%
    \kern-.06em%
    L%
    \kern-.18em%
    \raise.32ex\hbox{\HoLogoFont@font{SliTeX}{sc}{i}}%
    \HOLOGO@discretionary
    \kern-.06em%
    \hologo{TeX}%
  }%
}
%    \end{macrocode}
%    \end{macro}
%    \begin{macro}{\HoLogoBkm@SLiTeX@lift}
%    \begin{macrocode}
\def\HoLogoBkm@SLiTeX@lift#1{SLiTeX}
%    \end{macrocode}
%    \end{macro}
%    \begin{macro}{\HoLogoHtml@SLiTeX@lift}
%    \begin{macrocode}
\def\HoLogoHtml@SLiTeX@lift#1{%
  \HoLogoCss@SLiTeX@lift
  \HOLOGO@Span{SLiTeX-lift}{%
    \HoLogoFont@font{SliTeX}{rm}{%
      S%
      \HOLOGO@Span{L}{L}%
      \HOLOGO@Span{i}{i}%
      \hologo{TeX}%
    }%
  }%
}
%    \end{macrocode}
%    \end{macro}
%    \begin{macro}{\HoLogoCss@SLiTeX@lift}
%    \begin{macrocode}
\def\HoLogoCss@SLiTeX@lift{%
  \Css{%
    span.HoLogo-SLiTeX-lift span.HoLogo-L{%
      margin-left:-.06em;%
      margin-right:-.18em;%
    }%
  }%
  \Css{%
    span.HoLogo-SLiTeX-lift span.HoLogo-i{%
      position:relative;%
      top:-.32ex;%
      margin-right:-.06em;%
      font-variant:small-caps;%
    }%
  }%
  \global\let\HoLogoCss@SLiTeX@lift\relax
}
%    \end{macrocode}
%    \end{macro}
%
%    \begin{macro}{\HoLogo@SliTeX@simple}
%    \begin{macrocode}
\def\HoLogo@SliTeX@simple#1{%
  \HoLogoFont@font{SliTeX}{rm}{%
    \ltx@mbox{%
      \HoLogoFont@font{SliTeX}{sc}{Sli}%
    }%
    \HOLOGO@discretionary
    \hologo{TeX}%
  }%
}
%    \end{macrocode}
%    \end{macro}
%    \begin{macro}{\HoLogoBkm@SliTeX@simple}
%    \begin{macrocode}
\def\HoLogoBkm@SliTeX@simple#1{SliTeX}
%    \end{macrocode}
%    \end{macro}
%    \begin{macro}{\HoLogoHtml@SliTeX@simple}
%    \begin{macrocode}
\let\HoLogoHtml@SliTeX@simple\HoLogo@SliTeX@simple
%    \end{macrocode}
%    \end{macro}
%
%    \begin{macro}{\HoLogo@SliTeX@narrow}
%    \begin{macrocode}
\def\HoLogo@SliTeX@narrow#1{%
  \HoLogoFont@font{SliTeX}{rm}{%
    \ltx@mbox{%
      S%
      \kern-.06em%
      \HoLogoFont@font{SliTeX}{sc}{%
        l%
        \kern-.035em%
        i%
      }%
    }%
    \HOLOGO@discretionary
    \kern-.06em%
    \hologo{TeX}%
  }%
}
%    \end{macrocode}
%    \end{macro}
%    \begin{macro}{\HoLogoBkm@SliTeX@narrow}
%    \begin{macrocode}
\def\HoLogoBkm@SliTeX@narrow#1{SliTeX}
%    \end{macrocode}
%    \end{macro}
%    \begin{macro}{\HoLogoHtml@SliTeX@narrow}
%    \begin{macrocode}
\def\HoLogoHtml@SliTeX@narrow#1{%
  \HoLogoCss@SliTeX@narrow
  \HOLOGO@Span{SliTeX-narrow}{%
    \HoLogoFont@font{SliTeX}{rm}{%
      S%
        \HOLOGO@Span{l}{l}%
        \HOLOGO@Span{i}{i}%
      \hologo{TeX}%
    }%
  }%
}
%    \end{macrocode}
%    \end{macro}
%    \begin{macro}{\HoLogoCss@SliTeX@narrow}
%    \begin{macrocode}
\def\HoLogoCss@SliTeX@narrow{%
  \Css{%
    span.HoLogo-SliTeX-narrow span.HoLogo-l{%
      margin-left:-.06em;%
      margin-right:-.035em;%
      font-variant:small-caps;%
    }%
  }%
  \Css{%
    span.HoLogo-SliTeX-narrow span.HoLogo-i{%
      margin-right:-.06em;%
      font-variant:small-caps;%
    }%
  }%
  \global\let\HoLogoCss@SliTeX@narrow\relax
}
%    \end{macrocode}
%    \end{macro}
%
% \paragraph{Macro set completion.}
%
%    \begin{macro}{\HoLogo@SLiTeX@simple}
%    \begin{macrocode}
\def\HoLogo@SLiTeX@simple{\HoLogo@SliTeX@simple}
%    \end{macrocode}
%    \end{macro}
%    \begin{macro}{\HoLogoBkm@SLiTeX@simple}
%    \begin{macrocode}
\def\HoLogoBkm@SLiTeX@simple{\HoLogoBkm@SliTeX@simple}
%    \end{macrocode}
%    \end{macro}
%    \begin{macro}{\HoLogoHtml@SLiTeX@simple}
%    \begin{macrocode}
\def\HoLogoHtml@SLiTeX@simple{\HoLogoHtml@SliTeX@simple}
%    \end{macrocode}
%    \end{macro}
%
%    \begin{macro}{\HoLogo@SLiTeX@narrow}
%    \begin{macrocode}
\def\HoLogo@SLiTeX@narrow{\HoLogo@SliTeX@narrow}
%    \end{macrocode}
%    \end{macro}
%    \begin{macro}{\HoLogoBkm@SLiTeX@narrow}
%    \begin{macrocode}
\def\HoLogoBkm@SLiTeX@narrow{\HoLogoBkm@SliTeX@narrow}
%    \end{macrocode}
%    \end{macro}
%    \begin{macro}{\HoLogoHtml@SLiTeX@narrow}
%    \begin{macrocode}
\def\HoLogoHtml@SLiTeX@narrow{\HoLogoHtml@SliTeX@narrow}
%    \end{macrocode}
%    \end{macro}
%
%    \begin{macro}{\HoLogo@SliTeX@lift}
%    \begin{macrocode}
\def\HoLogo@SliTeX@lift{\HoLogo@SLiTeX@lift}
%    \end{macrocode}
%    \end{macro}
%    \begin{macro}{\HoLogoBkm@SliTeX@lift}
%    \begin{macrocode}
\def\HoLogoBkm@SliTeX@lift{\HoLogoBkm@SLiTeX@lift}
%    \end{macrocode}
%    \end{macro}
%    \begin{macro}{\HoLogoHtml@SliTeX@lift}
%    \begin{macrocode}
\def\HoLogoHtml@SliTeX@lift{\HoLogoHtml@SLiTeX@lift}
%    \end{macrocode}
%    \end{macro}
%
% \paragraph{Defaults.}
%
%    \begin{macro}{\HoLogo@SLiTeX}
%    \begin{macrocode}
\def\HoLogo@SLiTeX{\HoLogo@SLiTeX@lift}
%    \end{macrocode}
%    \end{macro}
%    \begin{macro}{\HoLogoBkm@SLiTeX}
%    \begin{macrocode}
\def\HoLogoBkm@SLiTeX{\HoLogoBkm@SLiTeX@lift}
%    \end{macrocode}
%    \end{macro}
%    \begin{macro}{\HoLogoHtml@SLiTeX}
%    \begin{macrocode}
\def\HoLogoHtml@SLiTeX{\HoLogoHtml@SLiTeX@lift}
%    \end{macrocode}
%    \end{macro}
%
%    \begin{macro}{\HoLogo@SliTeX}
%    \begin{macrocode}
\def\HoLogo@SliTeX{\HoLogo@SliTeX@narrow}
%    \end{macrocode}
%    \end{macro}
%    \begin{macro}{\HoLogoBkm@SliTeX}
%    \begin{macrocode}
\def\HoLogoBkm@SliTeX{\HoLogoBkm@SliTeX@narrow}
%    \end{macrocode}
%    \end{macro}
%    \begin{macro}{\HoLogoHtml@SliTeX}
%    \begin{macrocode}
\def\HoLogoHtml@SliTeX{\HoLogoHtml@SliTeX@narrow}
%    \end{macrocode}
%    \end{macro}
%
% \subsubsection{\hologo{LuaTeX}}
%
%    \begin{macro}{\HoLogo@LuaTeX}
%    The kerning is an idea of Hans Hagen, see mailing list
%    `luatex at tug dot org' in March 2010.
%    \begin{macrocode}
\def\HoLogo@LuaTeX#1{%
  \HOLOGO@mbox{%
    Lua%
    \HOLOGO@NegativeKerning{aT,oT,To}%
    \hologo{TeX}%
  }%
}
%    \end{macrocode}
%    \end{macro}
%    \begin{macro}{\HoLogoHtml@LuaTeX}
%    \begin{macrocode}
\let\HoLogoHtml@LuaTeX\HoLogo@LuaTeX
%    \end{macrocode}
%    \end{macro}
%
% \subsubsection{\hologo{LuaLaTeX}}
%
%    \begin{macro}{\HoLogo@LuaLaTeX}
%    \begin{macrocode}
\def\HoLogo@LuaLaTeX#1{%
  \HOLOGO@mbox{%
    Lua%
    \hologo{LaTeX}%
  }%
}
%    \end{macrocode}
%    \end{macro}
%    \begin{macro}{\HoLogoHtml@LuaLaTeX}
%    \begin{macrocode}
\let\HoLogoHtml@LuaLaTeX\HoLogo@LuaLaTeX
%    \end{macrocode}
%    \end{macro}
%
% \subsubsection{\hologo{XeTeX}, \hologo{XeLaTeX}}
%
%    \begin{macro}{\HOLOGO@IfCharExists}
%    \begin{macrocode}
\ifluatex
  \ifnum\luatexversion<36 %
  \else
    \def\HOLOGO@IfCharExists#1{%
      \ifnum
        \directlua{%
           if luaotfload and luaotfload.aux then
             if luaotfload.aux.font_has_glyph(%
                    font.current(), \number#1) then % 	 
	       tex.print("1") % 	 
	     end % 	 
	   elseif font and font.fonts and font.current then %
            local f = font.fonts[font.current()]%
            if f.characters and f.characters[\number#1] then %
              tex.print("1")%
            end %
          end%
        }0=\ltx@zero
        \expandafter\ltx@secondoftwo
      \else
        \expandafter\ltx@firstoftwo
      \fi
    }%
  \fi
\fi
\ltx@IfUndefined{HOLOGO@IfCharExists}{%
  \def\HOLOGO@@IfCharExists#1{%
    \begingroup
      \tracinglostchars=\ltx@zero
      \setbox\ltx@zero=\hbox{%
        \kern7sp\char#1\relax
        \ifnum\lastkern>\ltx@zero
          \expandafter\aftergroup\csname iffalse\endcsname
        \else
          \expandafter\aftergroup\csname iftrue\endcsname
        \fi
      }%
      % \if{true|false} from \aftergroup
      \endgroup
      \expandafter\ltx@firstoftwo
    \else
      \endgroup
      \expandafter\ltx@secondoftwo
    \fi
  }%
  \ifxetex
    \ltx@IfUndefined{XeTeXfonttype}{}{%
      \ltx@IfUndefined{XeTeXcharglyph}{}{%
        \def\HOLOGO@IfCharExists#1{%
          \ifnum\XeTeXfonttype\font>\ltx@zero
            \expandafter\ltx@firstofthree
          \else
            \expandafter\ltx@gobble
          \fi
          {%
            \ifnum\XeTeXcharglyph#1>\ltx@zero
              \expandafter\ltx@firstoftwo
            \else
              \expandafter\ltx@secondoftwo
            \fi
          }%
          \HOLOGO@@IfCharExists{#1}%
        }%
      }%
    }%
  \fi
}{}
\ltx@ifundefined{HOLOGO@IfCharExists}{%
  \ifnum64=`\^^^^0040\relax % test for big chars of LuaTeX/XeTeX
    \let\HOLOGO@IfCharExists\HOLOGO@@IfCharExists
  \else
    \def\HOLOGO@IfCharExists#1{%
      \ifnum#1>255 %
        \expandafter\ltx@fourthoffour
      \fi
      \HOLOGO@@IfCharExists{#1}%
    }%
  \fi
}{}
%    \end{macrocode}
%    \end{macro}
%
%    \begin{macro}{\HoLogo@Xe}
%    Source: package \xpackage{dtklogos}
%    \begin{macrocode}
\def\HoLogo@Xe#1{%
  X%
  \kern-.1em\relax
  \HOLOGO@IfCharExists{"018E}{%
    \lower.5ex\hbox{\char"018E}%
  }{%
    \chardef\HOLOGO@choice=\ltx@zero
    \ifdim\fontdimen\ltx@one\font>0pt %
      \ltx@IfUndefined{rotatebox}{%
        \ltx@IfUndefined{pgftext}{%
          \ltx@IfUndefined{psscalebox}{%
            \ltx@IfUndefined{HOLOGO@ScaleBox@\hologoDriver}{%
            }{%
              \chardef\HOLOGO@choice=4 %
            }%
          }{%
            \chardef\HOLOGO@choice=3 %
          }%
        }{%
          \chardef\HOLOGO@choice=2 %
        }%
      }{%
        \chardef\HOLOGO@choice=1 %
      }%
      \ifcase\HOLOGO@choice
        \HOLOGO@WarningUnsupportedDriver{Xe}%
        e%
      \or % 1: \rotatebox
        \begingroup
          \setbox\ltx@zero\hbox{\rotatebox{180}{E}}%
          \ltx@LocDimenA=\dp\ltx@zero
          \advance\ltx@LocDimenA by -.5ex\relax
          \raise\ltx@LocDimenA\box\ltx@zero
        \endgroup
      \or % 2: \pgftext
        \lower.5ex\hbox{%
          \pgfpicture
            \pgftext[rotate=180]{E}%
          \endpgfpicture
        }%
      \or % 3: \psscalebox
        \begingroup
          \setbox\ltx@zero\hbox{\psscalebox{-1 -1}{E}}%
          \ltx@LocDimenA=\dp\ltx@zero
          \advance\ltx@LocDimenA by -.5ex\relax
          \raise\ltx@LocDimenA\box\ltx@zero
        \endgroup
      \or % 4: \HOLOGO@PointReflectBox
        \lower.5ex\hbox{\HOLOGO@PointReflectBox{E}}%
      \else
        \@PackageError{hologo}{Internal error (choice/it}\@ehc
      \fi
    \else
      \ltx@IfUndefined{reflectbox}{%
        \ltx@IfUndefined{pgftext}{%
          \ltx@IfUndefined{psscalebox}{%
            \ltx@IfUndefined{HOLOGO@ScaleBox@\hologoDriver}{%
            }{%
              \chardef\HOLOGO@choice=4 %
            }%
          }{%
            \chardef\HOLOGO@choice=3 %
          }%
        }{%
          \chardef\HOLOGO@choice=2 %
        }%
      }{%
        \chardef\HOLOGO@choice=1 %
      }%
      \ifcase\HOLOGO@choice
        \HOLOGO@WarningUnsupportedDriver{Xe}%
        e%
      \or % 1: reflectbox
        \lower.5ex\hbox{%
          \reflectbox{E}%
        }%
      \or % 2: \pgftext
        \lower.5ex\hbox{%
          \pgfpicture
            \pgftransformxscale{-1}%
            \pgftext{E}%
          \endpgfpicture
        }%
      \or % 3: \psscalebox
        \lower.5ex\hbox{%
          \psscalebox{-1 1}{E}%
        }%
      \or % 4: \HOLOGO@Reflectbox
        \lower.5ex\hbox{%
          \HOLOGO@ReflectBox{E}%
        }%
      \else
        \@PackageError{hologo}{Internal error (choice/up)}\@ehc
      \fi
    \fi
  }%
}
%    \end{macrocode}
%    \end{macro}
%    \begin{macro}{\HoLogoHtml@Xe}
%    \begin{macrocode}
\def\HoLogoHtml@Xe#1{%
  \HoLogoCss@Xe
  \HOLOGO@Span{Xe}{%
    X%
    \HOLOGO@Span{e}{%
      \HCode{&\ltx@hashchar x018e;}%
    }%
  }%
}
%    \end{macrocode}
%    \end{macro}
%    \begin{macro}{\HoLogoCss@Xe}
%    \begin{macrocode}
\def\HoLogoCss@Xe{%
  \Css{%
    span.HoLogo-Xe span.HoLogo-e{%
      position:relative;%
      top:.5ex;%
      left-margin:-.1em;%
    }%
  }%
  \global\let\HoLogoCss@Xe\relax
}
%    \end{macrocode}
%    \end{macro}
%
%    \begin{macro}{\HoLogo@XeTeX}
%    \begin{macrocode}
\def\HoLogo@XeTeX#1{%
  \hologo{Xe}%
  \kern-.15em\relax
  \hologo{TeX}%
}
%    \end{macrocode}
%    \end{macro}
%
%    \begin{macro}{\HoLogoHtml@XeTeX}
%    \begin{macrocode}
\def\HoLogoHtml@XeTeX#1{%
  \HoLogoCss@XeTeX
  \HOLOGO@Span{XeTeX}{%
    \hologo{Xe}%
    \hologo{TeX}%
  }%
}
%    \end{macrocode}
%    \end{macro}
%    \begin{macro}{\HoLogoCss@XeTeX}
%    \begin{macrocode}
\def\HoLogoCss@XeTeX{%
  \Css{%
    span.HoLogo-XeTeX span.HoLogo-TeX{%
      margin-left:-.15em;%
    }%
  }%
  \global\let\HoLogoCss@XeTeX\relax
}
%    \end{macrocode}
%    \end{macro}
%
%    \begin{macro}{\HoLogo@XeLaTeX}
%    \begin{macrocode}
\def\HoLogo@XeLaTeX#1{%
  \hologo{Xe}%
  \kern-.13em%
  \hologo{LaTeX}%
}
%    \end{macrocode}
%    \end{macro}
%    \begin{macro}{\HoLogoHtml@XeLaTeX}
%    \begin{macrocode}
\def\HoLogoHtml@XeLaTeX#1{%
  \HoLogoCss@XeLaTeX
  \HOLOGO@Span{XeLaTeX}{%
    \hologo{Xe}%
    \hologo{LaTeX}%
  }%
}
%    \end{macrocode}
%    \end{macro}
%    \begin{macro}{\HoLogoCss@XeLaTeX}
%    \begin{macrocode}
\def\HoLogoCss@XeLaTeX{%
  \Css{%
    span.HoLogo-XeLaTeX span.HoLogo-Xe{%
      margin-right:-.13em;%
    }%
  }%
  \global\let\HoLogoCss@XeLaTeX\relax
}
%    \end{macrocode}
%    \end{macro}
%
% \subsubsection{\hologo{pdfTeX}, \hologo{pdfLaTeX}}
%
%    \begin{macro}{\HoLogo@pdfTeX}
%    \begin{macrocode}
\def\HoLogo@pdfTeX#1{%
  \HOLOGO@mbox{%
    #1{p}{P}df\hologo{TeX}%
  }%
}
%    \end{macrocode}
%    \end{macro}
%    \begin{macro}{\HoLogoCs@pdfTeX}
%    \begin{macrocode}
\def\HoLogoCs@pdfTeX#1{#1{p}{P}dfTeX}
%    \end{macrocode}
%    \end{macro}
%    \begin{macro}{\HoLogoBkm@pdfTeX}
%    \begin{macrocode}
\def\HoLogoBkm@pdfTeX#1{%
  #1{p}{P}df\hologo{TeX}%
}
%    \end{macrocode}
%    \end{macro}
%    \begin{macro}{\HoLogoHtml@pdfTeX}
%    \begin{macrocode}
\let\HoLogoHtml@pdfTeX\HoLogo@pdfTeX
%    \end{macrocode}
%    \end{macro}
%
%    \begin{macro}{\HoLogo@pdfLaTeX}
%    \begin{macrocode}
\def\HoLogo@pdfLaTeX#1{%
  \HOLOGO@mbox{%
    #1{p}{P}df\hologo{LaTeX}%
  }%
}
%    \end{macrocode}
%    \end{macro}
%    \begin{macro}{\HoLogoCs@pdfLaTeX}
%    \begin{macrocode}
\def\HoLogoCs@pdfLaTeX#1{#1{p}{P}dfLaTeX}
%    \end{macrocode}
%    \end{macro}
%    \begin{macro}{\HoLogoBkm@pdfLaTeX}
%    \begin{macrocode}
\def\HoLogoBkm@pdfLaTeX#1{%
  #1{p}{P}df\hologo{LaTeX}%
}
%    \end{macrocode}
%    \end{macro}
%    \begin{macro}{\HoLogoHtml@pdfLaTeX}
%    \begin{macrocode}
\let\HoLogoHtml@pdfLaTeX\HoLogo@pdfLaTeX
%    \end{macrocode}
%    \end{macro}
%
% \subsubsection{\hologo{VTeX}}
%
%    \begin{macro}{\HoLogo@VTeX}
%    \begin{macrocode}
\def\HoLogo@VTeX#1{%
  \HOLOGO@mbox{%
    V\hologo{TeX}%
  }%
}
%    \end{macrocode}
%    \end{macro}
%    \begin{macro}{\HoLogoHtml@VTeX}
%    \begin{macrocode}
\let\HoLogoHtml@VTeX\HoLogo@VTeX
%    \end{macrocode}
%    \end{macro}
%
% \subsubsection{\hologo{AmS}, \dots}
%
%    Source: class \xclass{amsdtx}
%
%    \begin{macro}{\HoLogo@AmS}
%    \begin{macrocode}
\def\HoLogo@AmS#1{%
  \HoLogoFont@font{AmS}{sy}{%
    A%
    \kern-.1667em%
    \lower.5ex\hbox{M}%
    \kern-.125em%
    S%
  }%
}
%    \end{macrocode}
%    \end{macro}
%    \begin{macro}{\HoLogoBkm@AmS}
%    \begin{macrocode}
\def\HoLogoBkm@AmS#1{AmS}
%    \end{macrocode}
%    \end{macro}
%    \begin{macro}{\HoLogoHtml@AmS}
%    \begin{macrocode}
\def\HoLogoHtml@AmS#1{%
  \HoLogoCss@AmS
%  \HoLogoFont@font{AmS}{sy}{%
    \HOLOGO@Span{AmS}{%
      A%
      \HOLOGO@Span{M}{M}%
      S%
    }%
%   }%
}
%    \end{macrocode}
%    \end{macro}
%    \begin{macro}{\HoLogoCss@AmS}
%    \begin{macrocode}
\def\HoLogoCss@AmS{%
  \Css{%
    span.HoLogo-AmS span.HoLogo-M{%
      position:relative;%
      top:.5ex;%
      margin-left:-.1667em;%
      margin-right:-.125em;%
      text-decoration:none;%
    }%
  }%
  \global\let\HoLogoCss@AmS\relax
}
%    \end{macrocode}
%    \end{macro}
%
%    \begin{macro}{\HoLogo@AmSTeX}
%    \begin{macrocode}
\def\HoLogo@AmSTeX#1{%
  \hologo{AmS}%
  \HOLOGO@hyphen
  \hologo{TeX}%
}
%    \end{macrocode}
%    \end{macro}
%    \begin{macro}{\HoLogoBkm@AmSTeX}
%    \begin{macrocode}
\def\HoLogoBkm@AmSTeX#1{AmS-TeX}%
%    \end{macrocode}
%    \end{macro}
%    \begin{macro}{\HoLogoHtml@AmSTeX}
%    \begin{macrocode}
\let\HoLogoHtml@AmSTeX\HoLogo@AmSTeX
%    \end{macrocode}
%    \end{macro}
%
%    \begin{macro}{\HoLogo@AmSLaTeX}
%    \begin{macrocode}
\def\HoLogo@AmSLaTeX#1{%
  \hologo{AmS}%
  \HOLOGO@hyphen
  \hologo{LaTeX}%
}
%    \end{macrocode}
%    \end{macro}
%    \begin{macro}{\HoLogoBkm@AmSLaTeX}
%    \begin{macrocode}
\def\HoLogoBkm@AmSLaTeX#1{AmS-LaTeX}%
%    \end{macrocode}
%    \end{macro}
%    \begin{macro}{\HoLogoHtml@AmSLaTeX}
%    \begin{macrocode}
\let\HoLogoHtml@AmSLaTeX\HoLogo@AmSLaTeX
%    \end{macrocode}
%    \end{macro}
%
% \subsubsection{\hologo{BibTeX}}
%
%    \begin{macro}{\HoLogo@BibTeX@sc}
%    A definition of \hologo{BibTeX} is provided in
%    the documentation source for the manual of \hologo{BibTeX}
%    \cite{btxdoc}.
%\begin{quote}
%\begin{verbatim}
%\def\BibTeX{%
%  {%
%    \rm
%    B%
%    \kern-.05em%
%    {%
%      \sc
%      i%
%      \kern-.025em %
%      b%
%    }%
%    \kern-.08em
%    T%
%    \kern-.1667em%
%    \lower.7ex\hbox{E}%
%    \kern-.125em%
%    X%
%  }%
%}
%\end{verbatim}
%\end{quote}
%    \begin{macrocode}
\def\HoLogo@BibTeX@sc#1{%
  B%
  \kern-.05em%
  \HoLogoFont@font{BibTeX}{sc}{%
    i%
    \kern-.025em%
    b%
  }%
  \HOLOGO@discretionary
  \kern-.08em%
  \hologo{TeX}%
}
%    \end{macrocode}
%    \end{macro}
%    \begin{macro}{\HoLogoHtml@BibTeX@sc}
%    \begin{macrocode}
\def\HoLogoHtml@BibTeX@sc#1{%
  \HoLogoCss@BibTeX@sc
  \HOLOGO@Span{BibTeX-sc}{%
    B%
    \HOLOGO@Span{i}{i}%
    \HOLOGO@Span{b}{b}%
    \hologo{TeX}%
  }%
}
%    \end{macrocode}
%    \end{macro}
%    \begin{macro}{\HoLogoCss@BibTeX@sc}
%    \begin{macrocode}
\def\HoLogoCss@BibTeX@sc{%
  \Css{%
    span.HoLogo-BibTeX-sc span.HoLogo-i{%
      margin-left:-.05em;%
      margin-right:-.025em;%
      font-variant:small-caps;%
    }%
  }%
  \Css{%
    span.HoLogo-BibTeX-sc span.HoLogo-b{%
      margin-right:-.08em;%
      font-variant:small-caps;%
    }%
  }%
  \global\let\HoLogoCss@BibTeX@sc\relax
}
%    \end{macrocode}
%    \end{macro}
%
%    \begin{macro}{\HoLogo@BibTeX@sf}
%    Variant \xoption{sf} avoids trouble with unavailable
%    small caps fonts (e.g., bold versions of Computer Modern or
%    Latin Modern). The definition is taken from
%    package \xpackage{dtklogos} \cite{dtklogos}.
%\begin{quote}
%\begin{verbatim}
%\DeclareRobustCommand{\BibTeX}{%
%  B%
%  \kern-.05em%
%  \hbox{%
%    $\m@th$% %% force math size calculations
%    \csname S@\f@size\endcsname
%    \fontsize\sf@size\z@
%    \math@fontsfalse
%    \selectfont
%    I%
%    \kern-.025em%
%    B
%  }%
%  \kern-.08em%
%  \-%
%  \TeX
%}
%\end{verbatim}
%\end{quote}
%    \begin{macrocode}
\def\HoLogo@BibTeX@sf#1{%
  B%
  \kern-.05em%
  \HoLogoFont@font{BibTeX}{bibsf}{%
    I%
    \kern-.025em%
    B%
  }%
  \HOLOGO@discretionary
  \kern-.08em%
  \hologo{TeX}%
}
%    \end{macrocode}
%    \end{macro}
%    \begin{macro}{\HoLogoHtml@BibTeX@sf}
%    \begin{macrocode}
\def\HoLogoHtml@BibTeX@sf#1{%
  \HoLogoCss@BibTeX@sf
  \HOLOGO@Span{BibTeX-sf}{%
    B%
    \HoLogoFont@font{BibTeX}{bibsf}{%
      \HOLOGO@Span{i}{I}%
      B%
    }%
    \hologo{TeX}%
  }%
}
%    \end{macrocode}
%    \end{macro}
%    \begin{macro}{\HoLogoCss@BibTeX@sf}
%    \begin{macrocode}
\def\HoLogoCss@BibTeX@sf{%
  \Css{%
    span.HoLogo-BibTeX-sf span.HoLogo-i{%
      margin-left:-.05em;%
      margin-right:-.025em;%
    }%
  }%
  \Css{%
    span.HoLogo-BibTeX-sf span.HoLogo-TeX{%
      margin-left:-.08em;%
    }%
  }%
  \global\let\HoLogoCss@BibTeX@sf\relax
}
%    \end{macrocode}
%    \end{macro}
%
%    \begin{macro}{\HoLogo@BibTeX}
%    \begin{macrocode}
\def\HoLogo@BibTeX{\HoLogo@BibTeX@sf}
%    \end{macrocode}
%    \end{macro}
%    \begin{macro}{\HoLogoHtml@BibTeX}
%    \begin{macrocode}
\def\HoLogoHtml@BibTeX{\HoLogoHtml@BibTeX@sf}
%    \end{macrocode}
%    \end{macro}
%
% \subsubsection{\hologo{BibTeX8}}
%
%    \begin{macro}{\HoLogo@BibTeX8}
%    \begin{macrocode}
\expandafter\def\csname HoLogo@BibTeX8\endcsname#1{%
  \hologo{BibTeX}%
  8%
}
%    \end{macrocode}
%    \end{macro}
%
%    \begin{macro}{\HoLogoBkm@BibTeX8}
%    \begin{macrocode}
\expandafter\def\csname HoLogoBkm@BibTeX8\endcsname#1{%
  \hologo{BibTeX}%
  8%
}
%    \end{macrocode}
%    \end{macro}
%    \begin{macro}{\HoLogoHtml@BibTeX8}
%    \begin{macrocode}
\expandafter
\let\csname HoLogoHtml@BibTeX8\expandafter\endcsname
\csname HoLogo@BibTeX8\endcsname
%    \end{macrocode}
%    \end{macro}
%
% \subsubsection{\hologo{ConTeXt}}
%
%    \begin{macro}{\HoLogo@ConTeXt@simple}
%    \begin{macrocode}
\def\HoLogo@ConTeXt@simple#1{%
  \HOLOGO@mbox{Con}%
  \HOLOGO@discretionary
  \HOLOGO@mbox{\hologo{TeX}t}%
}
%    \end{macrocode}
%    \end{macro}
%    \begin{macro}{\HoLogoHtml@ConTeXt@simple}
%    \begin{macrocode}
\let\HoLogoHtml@ConTeXt@simple\HoLogo@ConTeXt@simple
%    \end{macrocode}
%    \end{macro}
%
%    \begin{macro}{\HoLogo@ConTeXt@narrow}
%    This definition of logo \hologo{ConTeXt} with variant \xoption{narrow}
%    comes from TUGboat's class \xclass{ltugboat} (version 2010/11/15 v2.8).
%    \begin{macrocode}
\def\HoLogo@ConTeXt@narrow#1{%
  \HOLOGO@mbox{C\kern-.0333emon}%
  \HOLOGO@discretionary
  \kern-.0667em%
  \HOLOGO@mbox{\hologo{TeX}\kern-.0333emt}%
}
%    \end{macrocode}
%    \end{macro}
%    \begin{macro}{\HoLogoHtml@ConTeXt@narrow}
%    \begin{macrocode}
\def\HoLogoHtml@ConTeXt@narrow#1{%
  \HoLogoCss@ConTeXt@narrow
  \HOLOGO@Span{ConTeXt-narrow}{%
    \HOLOGO@Span{C}{C}%
    on%
    \hologo{TeX}%
    t%
  }%
}
%    \end{macrocode}
%    \end{macro}
%    \begin{macro}{\HoLogoCss@ConTeXt@narrow}
%    \begin{macrocode}
\def\HoLogoCss@ConTeXt@narrow{%
  \Css{%
    span.HoLogo-ConTeXt-narrow span.HoLogo-C{%
      margin-left:-.0333em;%
    }%
  }%
  \Css{%
    span.HoLogo-ConTeXt-narrow span.HoLogo-TeX{%
      margin-left:-.0667em;%
      margin-right:-.0333em;%
    }%
  }%
  \global\let\HoLogoCss@ConTeXt@narrow\relax
}
%    \end{macrocode}
%    \end{macro}
%
%    \begin{macro}{\HoLogo@ConTeXt}
%    \begin{macrocode}
\def\HoLogo@ConTeXt{\HoLogo@ConTeXt@narrow}
%    \end{macrocode}
%    \end{macro}
%    \begin{macro}{\HoLogoHtml@ConTeXt}
%    \begin{macrocode}
\def\HoLogoHtml@ConTeXt{\HoLogoHtml@ConTeXt@narrow}
%    \end{macrocode}
%    \end{macro}
%
% \subsubsection{\hologo{emTeX}}
%
%    \begin{macro}{\HoLogo@emTeX}
%    \begin{macrocode}
\def\HoLogo@emTeX#1{%
  \HOLOGO@mbox{#1{e}{E}m}%
  \HOLOGO@discretionary
  \hologo{TeX}%
}
%    \end{macrocode}
%    \end{macro}
%    \begin{macro}{\HoLogoCs@emTeX}
%    \begin{macrocode}
\def\HoLogoCs@emTeX#1{#1{e}{E}mTeX}%
%    \end{macrocode}
%    \end{macro}
%    \begin{macro}{\HoLogoBkm@emTeX}
%    \begin{macrocode}
\def\HoLogoBkm@emTeX#1{%
  #1{e}{E}m\hologo{TeX}%
}
%    \end{macrocode}
%    \end{macro}
%    \begin{macro}{\HoLogoHtml@emTeX}
%    \begin{macrocode}
\let\HoLogoHtml@emTeX\HoLogo@emTeX
%    \end{macrocode}
%    \end{macro}
%
% \subsubsection{\hologo{ExTeX}}
%
%    \begin{macro}{\HoLogo@ExTeX}
%    The definition is taken from the FAQ of the
%    project \hologo{ExTeX}
%    \cite{ExTeX-FAQ}.
%\begin{quote}
%\begin{verbatim}
%\def\ExTeX{%
%  \textrm{% Logo always with serifs
%    \ensuremath{%
%      \textstyle
%      \varepsilon_{%
%        \kern-0.15em%
%        \mathcal{X}%
%      }%
%    }%
%    \kern-.15em%
%    \TeX
%  }%
%}
%\end{verbatim}
%\end{quote}
%    \begin{macrocode}
\def\HoLogo@ExTeX#1{%
  \HoLogoFont@font{ExTeX}{rm}{%
    \ltx@mbox{%
      \HOLOGO@MathSetup
      $%
        \textstyle
        \varepsilon_{%
          \kern-0.15em%
          \HoLogoFont@font{ExTeX}{sy}{X}%
        }%
      $%
    }%
    \HOLOGO@discretionary
    \kern-.15em%
    \hologo{TeX}%
  }%
}
%    \end{macrocode}
%    \end{macro}
%    \begin{macro}{\HoLogoHtml@ExTeX}
%    \begin{macrocode}
\def\HoLogoHtml@ExTeX#1{%
  \HoLogoCss@ExTeX
  \HoLogoFont@font{ExTeX}{rm}{%
    \HOLOGO@Span{ExTeX}{%
      \ltx@mbox{%
        \HOLOGO@MathSetup
        $\textstyle\varepsilon$%
        \HOLOGO@Span{X}{$\textstyle\chi$}%
        \hologo{TeX}%
      }%
    }%
  }%
}
%    \end{macrocode}
%    \end{macro}
%    \begin{macro}{\HoLogoBkm@ExTeX}
%    \begin{macrocode}
\def\HoLogoBkm@ExTeX#1{%
  \HOLOGO@PdfdocUnicode{#1{e}{E}x}{\textepsilon\textchi}%
  \hologo{TeX}%
}
%    \end{macrocode}
%    \end{macro}
%    \begin{macro}{\HoLogoCss@ExTeX}
%    \begin{macrocode}
\def\HoLogoCss@ExTeX{%
  \Css{%
    span.HoLogo-ExTeX{%
      font-family:serif;%
    }%
  }%
  \Css{%
    span.HoLogo-ExTeX span.HoLogo-TeX{%
      margin-left:-.15em;%
    }%
  }%
  \global\let\HoLogoCss@ExTeX\relax
}
%    \end{macrocode}
%    \end{macro}
%
% \subsubsection{\hologo{MiKTeX}}
%
%    \begin{macro}{\HoLogo@MiKTeX}
%    \begin{macrocode}
\def\HoLogo@MiKTeX#1{%
  \HOLOGO@mbox{MiK}%
  \HOLOGO@discretionary
  \hologo{TeX}%
}
%    \end{macrocode}
%    \end{macro}
%    \begin{macro}{\HoLogoHtml@MiKTeX}
%    \begin{macrocode}
\let\HoLogoHtml@MiKTeX\HoLogo@MiKTeX
%    \end{macrocode}
%    \end{macro}
%
% \subsubsection{\hologo{OzTeX} and friends}
%
%    Source: \hologo{OzTeX} FAQ \cite{OzTeX}:
%    \begin{quote}
%      |\def\OzTeX{O\kern-.03em z\kern-.15em\TeX}|\\
%      (There is no kerning in OzMF, OzMP and OzTtH.)
%    \end{quote}
%
%    \begin{macro}{\HoLogo@OzTeX}
%    \begin{macrocode}
\def\HoLogo@OzTeX#1{%
  O%
  \kern-.03em %
  z%
  \kern-.15em %
  \hologo{TeX}%
}
%    \end{macrocode}
%    \end{macro}
%    \begin{macro}{\HoLogoHtml@OzTeX}
%    \begin{macrocode}
\def\HoLogoHtml@OzTeX#1{%
  \HoLogoCss@OzTeX
  \HOLOGO@Span{OzTeX}{%
    O%
    \HOLOGO@Span{z}{z}%
    \hologo{TeX}%
  }%
}
%    \end{macrocode}
%    \end{macro}
%    \begin{macro}{\HoLogoCss@OzTeX}
%    \begin{macrocode}
\def\HoLogoCss@OzTeX{%
  \Css{%
    span.HoLogo-OzTeX span.HoLogo-z{%
      margin-left:-.03em;%
      margin-right:-.15em;%
    }%
  }%
  \global\let\HoLogoCss@OzTeX\relax
}
%    \end{macrocode}
%    \end{macro}
%
%    \begin{macro}{\HoLogo@OzMF}
%    \begin{macrocode}
\def\HoLogo@OzMF#1{%
  \HOLOGO@mbox{OzMF}%
}
%    \end{macrocode}
%    \end{macro}
%    \begin{macro}{\HoLogo@OzMP}
%    \begin{macrocode}
\def\HoLogo@OzMP#1{%
  \HOLOGO@mbox{OzMP}%
}
%    \end{macrocode}
%    \end{macro}
%    \begin{macro}{\HoLogo@OzTtH}
%    \begin{macrocode}
\def\HoLogo@OzTtH#1{%
  \HOLOGO@mbox{OzTtH}%
}
%    \end{macrocode}
%    \end{macro}
%
% \subsubsection{\hologo{PCTeX}}
%
%    \begin{macro}{\HoLogo@PCTeX}
%    \begin{macrocode}
\def\HoLogo@PCTeX#1{%
  \HOLOGO@mbox{PC}%
  \hologo{TeX}%
}
%    \end{macrocode}
%    \end{macro}
%    \begin{macro}{\HoLogoHtml@PCTeX}
%    \begin{macrocode}
\let\HoLogoHtml@PCTeX\HoLogo@PCTeX
%    \end{macrocode}
%    \end{macro}
%
% \subsubsection{\hologo{PiCTeX}}
%
%    The original definitions from \xfile{pictex.tex} \cite{PiCTeX}:
%\begin{quote}
%\begin{verbatim}
%\def\PiC{%
%  P%
%  \kern-.12em%
%  \lower.5ex\hbox{I}%
%  \kern-.075em%
%  C%
%}
%\def\PiCTeX{%
%  \PiC
%  \kern-.11em%
%  \TeX
%}
%\end{verbatim}
%\end{quote}
%
%    \begin{macro}{\HoLogo@PiC}
%    \begin{macrocode}
\def\HoLogo@PiC#1{%
  P%
  \kern-.12em%
  \lower.5ex\hbox{I}%
  \kern-.075em%
  C%
  \HOLOGO@SpaceFactor
}
%    \end{macrocode}
%    \end{macro}
%    \begin{macro}{\HoLogoHtml@PiC}
%    \begin{macrocode}
\def\HoLogoHtml@PiC#1{%
  \HoLogoCss@PiC
  \HOLOGO@Span{PiC}{%
    P%
    \HOLOGO@Span{i}{I}%
    C%
  }%
}
%    \end{macrocode}
%    \end{macro}
%    \begin{macro}{\HoLogoCss@PiC}
%    \begin{macrocode}
\def\HoLogoCss@PiC{%
  \Css{%
    span.HoLogo-PiC span.HoLogo-i{%
      position:relative;%
      top:.5ex;%
      margin-left:-.12em;%
      margin-right:-.075em;%
      text-decoration:none;%
    }%
  }%
  \global\let\HoLogoCss@PiC\relax
}
%    \end{macrocode}
%    \end{macro}
%
%    \begin{macro}{\HoLogo@PiCTeX}
%    \begin{macrocode}
\def\HoLogo@PiCTeX#1{%
  \hologo{PiC}%
  \HOLOGO@discretionary
  \kern-.11em%
  \hologo{TeX}%
}
%    \end{macrocode}
%    \end{macro}
%    \begin{macro}{\HoLogoHtml@PiCTeX}
%    \begin{macrocode}
\def\HoLogoHtml@PiCTeX#1{%
  \HoLogoCss@PiCTeX
  \HOLOGO@Span{PiCTeX}{%
    \hologo{PiC}%
    \hologo{TeX}%
  }%
}
%    \end{macrocode}
%    \end{macro}
%    \begin{macro}{\HoLogoCss@PiCTeX}
%    \begin{macrocode}
\def\HoLogoCss@PiCTeX{%
  \Css{%
    span.HoLogo-PiCTeX span.HoLogo-PiC{%
      margin-right:-.11em;%
    }%
  }%
  \global\let\HoLogoCss@PiCTeX\relax
}
%    \end{macrocode}
%    \end{macro}
%
% \subsubsection{\hologo{teTeX}}
%
%    \begin{macro}{\HoLogo@teTeX}
%    \begin{macrocode}
\def\HoLogo@teTeX#1{%
  \HOLOGO@mbox{#1{t}{T}e}%
  \HOLOGO@discretionary
  \hologo{TeX}%
}
%    \end{macrocode}
%    \end{macro}
%    \begin{macro}{\HoLogoCs@teTeX}
%    \begin{macrocode}
\def\HoLogoCs@teTeX#1{#1{t}{T}dfTeX}
%    \end{macrocode}
%    \end{macro}
%    \begin{macro}{\HoLogoBkm@teTeX}
%    \begin{macrocode}
\def\HoLogoBkm@teTeX#1{%
  #1{t}{T}e\hologo{TeX}%
}
%    \end{macrocode}
%    \end{macro}
%    \begin{macro}{\HoLogoHtml@teTeX}
%    \begin{macrocode}
\let\HoLogoHtml@teTeX\HoLogo@teTeX
%    \end{macrocode}
%    \end{macro}
%
% \subsubsection{\hologo{TeX4ht}}
%
%    \begin{macro}{\HoLogo@TeX4ht}
%    \begin{macrocode}
\expandafter\def\csname HoLogo@TeX4ht\endcsname#1{%
  \HOLOGO@mbox{\hologo{TeX}4ht}%
}
%    \end{macrocode}
%    \end{macro}
%    \begin{macro}{\HoLogoHtml@TeX4ht}
%    \begin{macrocode}
\expandafter
\let\csname HoLogoHtml@TeX4ht\expandafter\endcsname
\csname HoLogo@TeX4ht\endcsname
%    \end{macrocode}
%    \end{macro}
%
%
% \subsubsection{\hologo{SageTeX}}
%
%    \begin{macro}{\HoLogo@SageTeX}
%    \begin{macrocode}
\def\HoLogo@SageTeX#1{%
  \HOLOGO@mbox{Sage}%
  \HOLOGO@discretionary
  \HOLOGO@NegativeKerning{eT,oT,To}%
  \hologo{TeX}%
}
%    \end{macrocode}
%    \end{macro}
%    \begin{macro}{\HoLogoHtml@SageTeX}
%    \begin{macrocode}
\let\HoLogoHtml@SageTeX\HoLogo@SageTeX
%    \end{macrocode}
%    \end{macro}
%
% \subsection{\hologo{METAFONT} and friends}
%
%    \begin{macro}{\HoLogo@METAFONT}
%    \begin{macrocode}
\def\HoLogo@METAFONT#1{%
  \HoLogoFont@font{METAFONT}{logo}{%
    \HOLOGO@mbox{META}%
    \HOLOGO@discretionary
    \HOLOGO@mbox{FONT}%
  }%
}
%    \end{macrocode}
%    \end{macro}
%
%    \begin{macro}{\HoLogo@METAPOST}
%    \begin{macrocode}
\def\HoLogo@METAPOST#1{%
  \HoLogoFont@font{METAPOST}{logo}{%
    \HOLOGO@mbox{META}%
    \HOLOGO@discretionary
    \HOLOGO@mbox{POST}%
  }%
}
%    \end{macrocode}
%    \end{macro}
%
%    \begin{macro}{\HoLogo@MetaFun}
%    \begin{macrocode}
\def\HoLogo@MetaFun#1{%
  \HOLOGO@mbox{Meta}%
  \HOLOGO@discretionary
  \HOLOGO@mbox{Fun}%
}
%    \end{macrocode}
%    \end{macro}
%
%    \begin{macro}{\HoLogo@MetaPost}
%    \begin{macrocode}
\def\HoLogo@MetaPost#1{%
  \HOLOGO@mbox{Meta}%
  \HOLOGO@discretionary
  \HOLOGO@mbox{Post}%
}
%    \end{macrocode}
%    \end{macro}
%
% \subsection{Others}
%
% \subsubsection{\hologo{biber}}
%
%    \begin{macro}{\HoLogo@biber}
%    \begin{macrocode}
\def\HoLogo@biber#1{%
  \HOLOGO@mbox{#1{b}{B}i}%
  \HOLOGO@discretionary
  \HOLOGO@mbox{ber}%
}
%    \end{macrocode}
%    \end{macro}
%    \begin{macro}{\HoLogoCs@biber}
%    \begin{macrocode}
\def\HoLogoCs@biber#1{#1{b}{B}iber}
%    \end{macrocode}
%    \end{macro}
%    \begin{macro}{\HoLogoBkm@biber}
%    \begin{macrocode}
\def\HoLogoBkm@biber#1{%
  #1{b}{B}iber%
}
%    \end{macrocode}
%    \end{macro}
%    \begin{macro}{\HoLogoHtml@biber}
%    \begin{macrocode}
\let\HoLogoHtml@biber\HoLogo@biber
%    \end{macrocode}
%    \end{macro}
%
% \subsubsection{\hologo{KOMAScript}}
%
%    \begin{macro}{\HoLogo@KOMAScript}
%    The definition for \hologo{KOMAScript} is taken
%    from \hologo{KOMAScript} (\xfile{scrlogo.dtx}, reformatted) \cite{scrlogo}:
%\begin{quote}
%\begin{verbatim}
%\@ifundefined{KOMAScript}{%
%  \DeclareRobustCommand{\KOMAScript}{%
%    \textsf{%
%      K\kern.05em O\kern.05emM\kern.05em A%
%      \kern.1em-\kern.1em %
%      Script%
%    }%
%  }%
%}{}
%\end{verbatim}
%\end{quote}
%    \begin{macrocode}
\def\HoLogo@KOMAScript#1{%
  \HoLogoFont@font{KOMAScript}{sf}{%
    \HOLOGO@mbox{%
      K\kern.05em%
      O\kern.05em%
      M\kern.05em%
      A%
    }%
    \kern.1em%
    \HOLOGO@hyphen
    \kern.1em%
    \HOLOGO@mbox{Script}%
  }%
}
%    \end{macrocode}
%    \end{macro}
%    \begin{macro}{\HoLogoBkm@KOMAScript}
%    \begin{macrocode}
\def\HoLogoBkm@KOMAScript#1{%
  KOMA-Script%
}
%    \end{macrocode}
%    \end{macro}
%    \begin{macro}{\HoLogoHtml@KOMAScript}
%    \begin{macrocode}
\def\HoLogoHtml@KOMAScript#1{%
  \HoLogoCss@KOMAScript
  \HoLogoFont@font{KOMAScript}{sf}{%
    \HOLOGO@Span{KOMAScript}{%
      K%
      \HOLOGO@Span{O}{O}%
      M%
      \HOLOGO@Span{A}{A}%
      \HOLOGO@Span{hyphen}{-}%
      Script%
    }%
  }%
}
%    \end{macrocode}
%    \end{macro}
%    \begin{macro}{\HoLogoCss@KOMAScript}
%    \begin{macrocode}
\def\HoLogoCss@KOMAScript{%
  \Css{%
    span.HoLogo-KOMAScript{%
      font-family:sans-serif;%
    }%
  }%
  \Css{%
    span.HoLogo-KOMAScript span.HoLogo-O{%
      padding-left:.05em;%
      padding-right:.05em;%
    }%
  }%
  \Css{%
    span.HoLogo-KOMAScript span.HoLogo-A{%
      padding-left:.05em;%
    }%
  }%
  \Css{%
    span.HoLogo-KOMAScript span.HoLogo-hyphen{%
      padding-left:.1em;%
      padding-right:.1em;%
    }%
  }%
  \global\let\HoLogoCss@KOMAScript\relax
}
%    \end{macrocode}
%    \end{macro}
%
% \subsubsection{\hologo{LyX}}
%
%    \begin{macro}{\HoLogo@LyX}
%    The definition is taken from the documentation source files
%    of \hologo{LyX}, \xfile{Intro.lyx} \cite{LyX}:
%\begin{quote}
%\begin{verbatim}
%\def\LyX{%
%  \texorpdfstring{%
%    L\kern-.1667em\lower.25em\hbox{Y}\kern-.125emX\@%
%  }{%
%    LyX%
%  }%
%}
%\end{verbatim}
%\end{quote}
%    \begin{macrocode}
\def\HoLogo@LyX#1{%
  L%
  \kern-.1667em%
  \lower.25em\hbox{Y}%
  \kern-.125em%
  X%
  \HOLOGO@SpaceFactor
}
%    \end{macrocode}
%    \end{macro}
%    \begin{macro}{\HoLogoHtml@LyX}
%    \begin{macrocode}
\def\HoLogoHtml@LyX#1{%
  \HoLogoCss@LyX
  \HOLOGO@Span{LyX}{%
    L%
    \HOLOGO@Span{y}{Y}%
    X%
  }%
}
%    \end{macrocode}
%    \end{macro}
%    \begin{macro}{\HoLogoCss@LyX}
%    \begin{macrocode}
\def\HoLogoCss@LyX{%
  \Css{%
    span.HoLogo-LyX span.HoLogo-y{%
      position:relative;%
      top:.25em;%
      margin-left:-.1667em;%
      margin-right:-.125em;%
      text-decoration:none;%
    }%
  }%
  \global\let\HoLogoCss@LyX\relax
}
%    \end{macrocode}
%    \end{macro}
%
% \subsubsection{\hologo{NTS}}
%
%    \begin{macro}{\HoLogo@NTS}
%    Definition for \hologo{NTS} can be found in
%    package \xpackage{etex\textunderscore man} for the \hologo{eTeX} manual \cite{etexman}
%    and in package \xpackage{dtklogos} \cite{dtklogos}:
%\begin{quote}
%\begin{verbatim}
%\def\NTS{%
%  \leavevmode
%  \hbox{%
%    $%
%      \cal N%
%      \kern-0.35em%
%      \lower0.5ex\hbox{$\cal T$}%
%      \kern-0.2em%
%      S%
%    $%
%  }%
%}
%\end{verbatim}
%\end{quote}
%    \begin{macrocode}
\def\HoLogo@NTS#1{%
  \HoLogoFont@font{NTS}{sy}{%
    N\/%
    \kern-.35em%
    \lower.5ex\hbox{T\/}%
    \kern-.2em%
    S\/%
  }%
  \HOLOGO@SpaceFactor
}
%    \end{macrocode}
%    \end{macro}
%
% \subsubsection{\Hologo{TTH} (\hologo{TeX} to HTML translator)}
%
%    Source: \url{http://hutchinson.belmont.ma.us/tth/}
%    In the HTML source the second `T' is printed as subscript.
%\begin{quote}
%\begin{verbatim}
%T<sub>T</sub>H
%\end{verbatim}
%\end{quote}
%    \begin{macro}{\HoLogo@TTH}
%    \begin{macrocode}
\def\HoLogo@TTH#1{%
  \ltx@mbox{%
    T\HOLOGO@SubScript{T}H%
  }%
  \HOLOGO@SpaceFactor
}
%    \end{macrocode}
%    \end{macro}
%
%    \begin{macro}{\HoLogoHtml@TTH}
%    \begin{macrocode}
\def\HoLogoHtml@TTH#1{%
  T\HCode{<sub>}T\HCode{</sub>}H%
}
%    \end{macrocode}
%    \end{macro}
%
% \subsubsection{\Hologo{HanTheThanh}}
%
%    Partial source: Package \xpackage{dtklogos}.
%    The double accent is U+1EBF (latin small letter e with circumflex
%    and acute).
%    \begin{macro}{\HoLogo@HanTheThanh}
%    \begin{macrocode}
\def\HoLogo@HanTheThanh#1{%
  \ltx@mbox{H\`an}%
  \HOLOGO@space
  \ltx@mbox{%
    Th%
    \HOLOGO@IfCharExists{"1EBF}{%
      \char"1EBF\relax
    }{%
      \^e\hbox to 0pt{\hss\raise .5ex\hbox{\'{}}}%
    }%
  }%
  \HOLOGO@space
  \ltx@mbox{Th\`anh}%
}
%    \end{macrocode}
%    \end{macro}
%    \begin{macro}{\HoLogoBkm@HanTheThanh}
%    \begin{macrocode}
\def\HoLogoBkm@HanTheThanh#1{%
  H\`an %
  Th\HOLOGO@PdfdocUnicode{\^e}{\9036\277} %
  Th\`anh%
}
%    \end{macrocode}
%    \end{macro}
%    \begin{macro}{\HoLogoHtml@HanTheThanh}
%    \begin{macrocode}
\def\HoLogoHtml@HanTheThanh#1{%
  H\`an %
  Th\HCode{&\ltx@hashchar x1ebf;} %
  Th\`anh%
}
%    \end{macrocode}
%    \end{macro}
%
% \subsection{Driver detection}
%
%    \begin{macrocode}
\HOLOGO@IfExists\InputIfFileExists{%
  \InputIfFileExists{hologo.cfg}{}{}%
}{%
  \ltx@IfUndefined{pdf@filesize}{%
    \def\HOLOGO@InputIfExists{%
      \openin\HOLOGO@temp=hologo.cfg\relax
      \ifeof\HOLOGO@temp
        \closein\HOLOGO@temp
      \else
        \closein\HOLOGO@temp
        \begingroup
          \def\x{LaTeX2e}%
        \expandafter\endgroup
        \ifx\fmtname\x
          \input{hologo.cfg}%
        \else
          \input hologo.cfg\relax
        \fi
      \fi
    }%
    \ltx@IfUndefined{newread}{%
      \chardef\HOLOGO@temp=15 %
      \def\HOLOGO@CheckRead{%
        \ifeof\HOLOGO@temp
          \HOLOGO@InputIfExists
        \else
          \ifcase\HOLOGO@temp
            \@PackageWarningNoLine{hologo}{%
              Configuration file ignored, because\MessageBreak
              a free read register could not be found%
            }%
          \else
            \begingroup
              \count\ltx@cclv=\HOLOGO@temp
              \advance\ltx@cclv by \ltx@minusone
              \edef\x{\endgroup
                \chardef\noexpand\HOLOGO@temp=\the\count\ltx@cclv
                \relax
              }%
            \x
          \fi
        \fi
      }%
    }{%
      \csname newread\endcsname\HOLOGO@temp
      \HOLOGO@InputIfExists
    }%
  }{%
    \edef\HOLOGO@temp{\pdf@filesize{hologo.cfg}}%
    \ifx\HOLOGO@temp\ltx@empty
    \else
      \ifnum\HOLOGO@temp>0 %
        \begingroup
          \def\x{LaTeX2e}%
        \expandafter\endgroup
        \ifx\fmtname\x
          \input{hologo.cfg}%
        \else
          \input hologo.cfg\relax
        \fi
      \else
        \@PackageInfoNoLine{hologo}{%
          Empty configuration file `hologo.cfg' ignored%
        }%
      \fi
    \fi
  }%
}
%    \end{macrocode}
%
%    \begin{macrocode}
\def\HOLOGO@temp#1#2{%
  \kv@define@key{HoLogoDriver}{#1}[]{%
    \begingroup
      \def\HOLOGO@temp{##1}%
      \ltx@onelevel@sanitize\HOLOGO@temp
      \ifx\HOLOGO@temp\ltx@empty
      \else
        \@PackageError{hologo}{%
          Value (\HOLOGO@temp) not permitted for option `#1'%
        }%
        \@ehc
      \fi
    \endgroup
    \def\hologoDriver{#2}%
  }%
}%
\def\HOLOGO@@temp#1#2{%
  \ifx\kv@value\relax
    \HOLOGO@temp{#1}{#1}%
  \else
    \HOLOGO@temp{#1}{#2}%
  \fi
}%
\kv@parse@normalized{%
  pdftex,%
  luatex=pdftex,%
  dvipdfm,%
  dvipdfmx=dvipdfm,%
  dvips,%
  dvipsone=dvips,%
  xdvi=dvips,%
  xetex,%
  vtex,%
}\HOLOGO@@temp
%    \end{macrocode}
%
%    \begin{macrocode}
\kv@define@key{HoLogoDriver}{driverfallback}{%
  \def\HOLOGO@DriverFallback{#1}%
}
%    \end{macrocode}
%
%    \begin{macro}{\HOLOGO@DriverFallback}
%    \begin{macrocode}
\def\HOLOGO@DriverFallback{dvips}
%    \end{macrocode}
%    \end{macro}
%
%    \begin{macro}{\hologoDriverSetup}
%    \begin{macrocode}
\def\hologoDriverSetup{%
  \let\hologoDriver\ltx@undefined
  \HOLOGO@DriverSetup
}
%    \end{macrocode}
%    \end{macro}
%
%    \begin{macro}{\HOLOGO@DriverSetup}
%    \begin{macrocode}
\def\HOLOGO@DriverSetup#1{%
  \kvsetkeys{HoLogoDriver}{#1}%
  \HOLOGO@CheckDriver
  \ltx@ifundefined{hologoDriver}{%
    \begingroup
    \edef\x{\endgroup
      \noexpand\kvsetkeys{HoLogoDriver}{\HOLOGO@DriverFallback}%
    }\x
  }{}%
  \@PackageInfoNoLine{hologo}{Using driver `\hologoDriver'}%
}
%    \end{macrocode}
%    \end{macro}
%
%    \begin{macro}{\HOLOGO@CheckDriver}
%    \begin{macrocode}
\def\HOLOGO@CheckDriver{%
  \ifpdf
    \def\hologoDriver{pdftex}%
    \let\HOLOGO@pdfliteral\pdfliteral
    \ifluatex
      \ifx\pdfextension\@undefined\else
        \protected\def\pdfliteral{\pdfextension literal}%
        \let\HOLOGO@pdfliteral\pdfliteral
      \fi
      \ltx@IfUndefined{HOLOGO@pdfliteral}{%
        \ifnum\luatexversion<36 %
        \else
          \begingroup
            \let\HOLOGO@temp\endgroup
            \ifcase0%
                \directlua{%
                  if tex.enableprimitives then %
                    tex.enableprimitives('HOLOGO@', {'pdfliteral'})%
                  else %
                    tex.print('1')%
                  end%
                }%
                \ifx\HOLOGO@pdfliteral\@undefined 1\fi%
                \relax%
              \endgroup
              \let\HOLOGO@temp\relax
              \global\let\HOLOGO@pdfliteral\HOLOGO@pdfliteral
            \fi%
          \HOLOGO@temp
        \fi
      }{}%
    \fi
    \ltx@IfUndefined{HOLOGO@pdfliteral}{%
      \@PackageWarningNoLine{hologo}{%
        Cannot find \string\pdfliteral
      }%
    }{}%
  \else
    \ifxetex
      \def\hologoDriver{xetex}%
    \else
      \ifvtex
        \def\hologoDriver{vtex}%
      \fi
    \fi
  \fi
}
%    \end{macrocode}
%    \end{macro}
%
%    \begin{macro}{\HOLOGO@WarningUnsupportedDriver}
%    \begin{macrocode}
\def\HOLOGO@WarningUnsupportedDriver#1{%
  \@PackageWarningNoLine{hologo}{%
    Logo `#1' needs driver specific macros,\MessageBreak
    but driver `\hologoDriver' is not supported.\MessageBreak
    Use a different driver or\MessageBreak
    load package `graphics' or `pgf'%
  }%
}
%    \end{macrocode}
%    \end{macro}
%
% \subsubsection{Reflect box macros}
%
%    Skip driver part if not needed.
%    \begin{macrocode}
\ltx@IfUndefined{reflectbox}{}{%
  \ltx@IfUndefined{rotatebox}{}{%
    \HOLOGO@AtEnd
  }%
}
\ltx@IfUndefined{pgftext}{}{%
  \HOLOGO@AtEnd
}
\ltx@IfUndefined{psscalebox}{}{%
  \HOLOGO@AtEnd
}
%    \end{macrocode}
%
%    \begin{macrocode}
\def\HOLOGO@temp{LaTeX2e}
\ifx\fmtname\HOLOGO@temp
  \RequirePackage{kvoptions}[2011/06/30]%
  \ProcessKeyvalOptions{HoLogoDriver}%
\fi
\HOLOGO@DriverSetup{}
%    \end{macrocode}
%
%    \begin{macro}{\HOLOGO@ReflectBox}
%    \begin{macrocode}
\def\HOLOGO@ReflectBox#1{%
  \begingroup
    \setbox\ltx@zero\hbox{\begingroup#1\endgroup}%
    \setbox\ltx@two\hbox{%
      \kern\wd\ltx@zero
      \csname HOLOGO@ScaleBox@\hologoDriver\endcsname{-1}{1}{%
        \hbox to 0pt{\copy\ltx@zero\hss}%
      }%
    }%
    \wd\ltx@two=\wd\ltx@zero
    \box\ltx@two
  \endgroup
}
%    \end{macrocode}
%    \end{macro}
%
%    \begin{macro}{\HOLOGO@PointReflectBox}
%    \begin{macrocode}
\def\HOLOGO@PointReflectBox#1{%
  \begingroup
    \setbox\ltx@zero\hbox{\begingroup#1\endgroup}%
    \setbox\ltx@two\hbox{%
      \kern\wd\ltx@zero
      \raise\ht\ltx@zero\hbox{%
        \csname HOLOGO@ScaleBox@\hologoDriver\endcsname{-1}{-1}{%
          \hbox to 0pt{\copy\ltx@zero\hss}%
        }%
      }%
    }%
    \wd\ltx@two=\wd\ltx@zero
    \box\ltx@two
  \endgroup
}
%    \end{macrocode}
%    \end{macro}
%
%    We must define all variants because of dynamic driver setup.
%    \begin{macrocode}
\def\HOLOGO@temp#1#2{#2}
%    \end{macrocode}
%
%    \begin{macro}{\HOLOGO@ScaleBox@pdftex}
%    \begin{macrocode}
\HOLOGO@temp{pdftex}{%
  \def\HOLOGO@ScaleBox@pdftex#1#2#3{%
    \HOLOGO@pdfliteral{%
      q #1 0 0 #2 0 0 cm%
    }%
    #3%
    \HOLOGO@pdfliteral{%
      Q%
    }%
  }%
}
%    \end{macrocode}
%    \end{macro}
%    \begin{macro}{\HOLOGO@ScaleBox@dvips}
%    \begin{macrocode}
\HOLOGO@temp{dvips}{%
  \def\HOLOGO@ScaleBox@dvips#1#2#3{%
    \special{ps:%
      gsave %
      currentpoint %
      currentpoint translate %
      #1 #2 scale %
      neg exch neg exch translate%
    }%
    #3%
    \special{ps:%
      currentpoint %
      grestore %
      moveto%
    }%
  }%
}
%    \end{macrocode}
%    \end{macro}
%    \begin{macro}{\HOLOGO@ScaleBox@dvipdfm}
%    \begin{macrocode}
\HOLOGO@temp{dvipdfm}{%
  \let\HOLOGO@ScaleBox@dvipdfm\HOLOGO@ScaleBox@dvips
}
%    \end{macrocode}
%    \end{macro}
%    Since \hologo{XeTeX} v0.6.
%    \begin{macro}{\HOLOGO@ScaleBox@xetex}
%    \begin{macrocode}
\HOLOGO@temp{xetex}{%
  \def\HOLOGO@ScaleBox@xetex#1#2#3{%
    \special{x:gsave}%
    \special{x:scale #1 #2}%
    #3%
    \special{x:grestore}%
  }%
}
%    \end{macrocode}
%    \end{macro}
%    \begin{macro}{\HOLOGO@ScaleBox@vtex}
%    \begin{macrocode}
\HOLOGO@temp{vtex}{%
  \def\HOLOGO@ScaleBox@vtex#1#2#3{%
    \special{r(#1,0,0,#2,0,0}%
    #3%
    \special{r)}%
  }%
}
%    \end{macrocode}
%    \end{macro}
%
%    \begin{macrocode}
\HOLOGO@AtEnd%
%</package>
%    \end{macrocode}
%
% \section{Test}
%
% \subsection{Catcode checks for loading}
%
%    \begin{macrocode}
%<*test1>
%    \end{macrocode}
%    \begin{macrocode}
\catcode`\{=1 %
\catcode`\}=2 %
\catcode`\#=6 %
\catcode`\@=11 %
\expandafter\ifx\csname count@\endcsname\relax
  \countdef\count@=255 %
\fi
\expandafter\ifx\csname @gobble\endcsname\relax
  \long\def\@gobble#1{}%
\fi
\expandafter\ifx\csname @firstofone\endcsname\relax
  \long\def\@firstofone#1{#1}%
\fi
\expandafter\ifx\csname loop\endcsname\relax
  \expandafter\@firstofone
\else
  \expandafter\@gobble
\fi
{%
  \def\loop#1\repeat{%
    \def\body{#1}%
    \iterate
  }%
  \def\iterate{%
    \body
      \let\next\iterate
    \else
      \let\next\relax
    \fi
    \next
  }%
  \let\repeat=\fi
}%
\def\RestoreCatcodes{}
\count@=0 %
\loop
  \edef\RestoreCatcodes{%
    \RestoreCatcodes
    \catcode\the\count@=\the\catcode\count@\relax
  }%
\ifnum\count@<255 %
  \advance\count@ 1 %
\repeat

\def\RangeCatcodeInvalid#1#2{%
  \count@=#1\relax
  \loop
    \catcode\count@=15 %
  \ifnum\count@<#2\relax
    \advance\count@ 1 %
  \repeat
}
\def\RangeCatcodeCheck#1#2#3{%
  \count@=#1\relax
  \loop
    \ifnum#3=\catcode\count@
    \else
      \errmessage{%
        Character \the\count@\space
        with wrong catcode \the\catcode\count@\space
        instead of \number#3%
      }%
    \fi
  \ifnum\count@<#2\relax
    \advance\count@ 1 %
  \repeat
}
\def\space{ }
\expandafter\ifx\csname LoadCommand\endcsname\relax
  \def\LoadCommand{\input hologo.sty\relax}%
\fi
\def\Test{%
  \RangeCatcodeInvalid{0}{47}%
  \RangeCatcodeInvalid{58}{64}%
  \RangeCatcodeInvalid{91}{96}%
  \RangeCatcodeInvalid{123}{255}%
  \catcode`\@=12 %
  \catcode`\\=0 %
  \catcode`\%=14 %
  \LoadCommand
  \RangeCatcodeCheck{0}{36}{15}%
  \RangeCatcodeCheck{37}{37}{14}%
  \RangeCatcodeCheck{38}{47}{15}%
  \RangeCatcodeCheck{48}{57}{12}%
  \RangeCatcodeCheck{58}{63}{15}%
  \RangeCatcodeCheck{64}{64}{12}%
  \RangeCatcodeCheck{65}{90}{11}%
  \RangeCatcodeCheck{91}{91}{15}%
  \RangeCatcodeCheck{92}{92}{0}%
  \RangeCatcodeCheck{93}{96}{15}%
  \RangeCatcodeCheck{97}{122}{11}%
  \RangeCatcodeCheck{123}{255}{15}%
  \RestoreCatcodes
}
\Test
\csname @@end\endcsname
\end
%    \end{macrocode}
%    \begin{macrocode}
%</test1>
%    \end{macrocode}
%
% \subsection{Spacefactor}
%
%    The space factor must be 1000 after a logo. If it is greater 1000
%    then the following space is a space after a sentence closing point.
%    If the space factor is smaller 1000 then an immediate following
%    dot is interpreted as abbreviation, not sentence closing point.
%
%    \begin{macrocode}
%<*test-spacefactor>
\NeedsTeXFormat{LaTeX2e}
\documentclass{article}
\usepackage{hologo}[2016/05/12]
\usepackage{kvsetkeys}
\usepackage{qstest}
\IncludeTests{*}
\LogTests{log}{*}{*}
\begin{document}
\begin{qstest}{spacefactor}{spacefactor}
\newcommand*{\Test}[1]{%
  \sbox0{%
    \hologo{#1}%
    \Expect*{1000 (#1)}*{\the\spacefactor\space(#1)}%
  }%
}%
\makeatletter
\def\TestList{}
\def\hologoEntry#1#2#3{%
  \edef\TestList{%
    \ifx\TestList\@empty
    \else
      \TestList,%
    \fi
    #1%
    \ifx\\#2\\%
    \else
      ={variant=#2}%
    \fi
  }%
}
\hologoList
\expandafter\kv@parse@normalized\expandafter{%
  \TestList
}{%
  \begingroup
    \let\@logo=\kv@key
    \ifx\kv@value\relax
    \else
      \expandafter\hologoLogoSetup\expandafter\@logo\expandafter{%
        \kv@value
      }%
    \fi
    \Test\@logo
  \endgroup
  \@gobbletwo
}
\end{qstest}
\end{document}
%</test-spacefactor>
%    \end{macrocode}
%
% \subsection{Complete list}
%
%    \begin{macrocode}
%<*test-list>
\NeedsTeXFormat{LaTeX2e}
\documentclass[12pt,a4paper]{article}
\usepackage{hologo}[2016/05/12]
\usepackage[T1]{fontenc}
\usepackage{lmodern}
\usepackage{parskip}
\usepackage[unicode]{hyperref}[2011/09/28]
\usepackage{bookmark}[2011/09/19]
\bookmarksetup{%
  numbered,%
  open,%
  openlevel=2,%
}
\renewcommand*{\contentsname}{List of logos}
\begin{document}
\tableofcontents
\def\TestFont#1#2#3#4#5#6{%
  \begingroup
    \usefont{#3}{#4}{#5}{#6}%
    \HologoVariant{#1}{#2}/\hologoVariant{#1}{#2}%
    \quad
    \begingroup\scriptsize\hologoVariant{#1}{#2}\endgroup
    \quad
  \endgroup
  (#3/#4/#5/#6)%
  \par
}
\makeatletter
\def\hologoEntry#1#2#3{%
  \section{%
    \HologoVariant{#1}{#2}/\hologoVariant{#1}{#2} %
    {[#1\ifx\\#2\\\else\space(#2)\fi]}% hash-ok
  }% braces around [] because of bug in tex4ht
  \begingroup
    \hypersetup{unicode=false}%
    \bookmark[%
      dest=\@currentHref,%
      rellevel=1,%
      keeplevel,%
    ]{%
      \HologoVariant{#1}{#2}/\hologoVariant{#1}{#2} %
      (PDFDocEncoding)%
    }%
  \endgroup
  \TestFont{#1}{#2}{OT1}{cmr}{m}{n}%
  \TestFont{#1}{#2}{OT1}{cmss}{m}{n}%
  \TestFont{#1}{#2}{OT1}{cmr}{b}{n}%
  \TestFont{#1}{#2}{OT1}{cmr}{m}{it}%
  \TestFont{#1}{#2}{OT1}{cmtt}{m}{n}%
  \TestFont{#1}{#2}{T1}{lmr}{m}{n}%
  \TestFont{#1}{#2}{T1}{lmss}{m}{n}%
  \TestFont{#1}{#2}{T1}{lmr}{b}{n}%
  \TestFont{#1}{#2}{T1}{lmr}{m}{it}%
  \TestFont{#1}{#2}{T1}{lmtt}{m}{n}%
  \TestFont{#1}{#2}{T1}{lmvtt}{m}{n}%
  \TestFont{#1}{#2}{T1}{qtm}{m}{n}%
  \TestFont{#1}{#2}{T1}{qhv}{m}{n}%
  \TestFont{#1}{#2}{T1}{qtm}{b}{n}%
  \TestFont{#1}{#2}{T1}{qtm}{m}{it}%
  \TestFont{#1}{#2}{T1}{qcr}{m}{n}%
  \newpage
}
\makeatother
\hologoList
\end{document}
%</test-list>
%    \end{macrocode}
%
% \section{Installation}
%
% \subsection{Download}
%
% \paragraph{Package.} This package is available on
% CTAN\footnote{\url{ftp://ftp.ctan.org/tex-archive/}}:
% \begin{description}
% \item[\CTAN{macros/latex/contrib/oberdiek/hologo.dtx}] The source file.
% \item[\CTAN{macros/latex/contrib/oberdiek/hologo.pdf}] Documentation.
% \end{description}
%
%
% \paragraph{Bundle.} All the packages of the bundle `oberdiek'
% are also available in a TDS compliant ZIP archive. There
% the packages are already unpacked and the documentation files
% are generated. The files and directories obey the TDS standard.
% \begin{description}
% \item[\CTAN{install/macros/latex/contrib/oberdiek.tds.zip}]
% \end{description}
% \emph{TDS} refers to the standard ``A Directory Structure
% for \TeX\ Files'' (\CTAN{tds/tds.pdf}). Directories
% with \xfile{texmf} in their name are usually organized this way.
%
% \subsection{Bundle installation}
%
% \paragraph{Unpacking.} Unpack the \xfile{oberdiek.tds.zip} in the
% TDS tree (also known as \xfile{texmf} tree) of your choice.
% Example (linux):
% \begin{quote}
%   |unzip oberdiek.tds.zip -d ~/texmf|
% \end{quote}
%
% \paragraph{Script installation.}
% Check the directory \xfile{TDS:scripts/oberdiek/} for
% scripts that need further installation steps.
% Package \xpackage{attachfile2} comes with the Perl script
% \xfile{pdfatfi.pl} that should be installed in such a way
% that it can be called as \texttt{pdfatfi}.
% Example (linux):
% \begin{quote}
%   |chmod +x scripts/oberdiek/pdfatfi.pl|\\
%   |cp scripts/oberdiek/pdfatfi.pl /usr/local/bin/|
% \end{quote}
%
% \subsection{Package installation}
%
% \paragraph{Unpacking.} The \xfile{.dtx} file is a self-extracting
% \docstrip\ archive. The files are extracted by running the
% \xfile{.dtx} through \plainTeX:
% \begin{quote}
%   \verb|tex hologo.dtx|
% \end{quote}
%
% \paragraph{TDS.} Now the different files must be moved into
% the different directories in your installation TDS tree
% (also known as \xfile{texmf} tree):
% \begin{quote}
% \def\t{^^A
% \begin{tabular}{@{}>{\ttfamily}l@{ $\rightarrow$ }>{\ttfamily}l@{}}
%   hologo.sty & tex/generic/oberdiek/hologo.sty\\
%   hologo.pdf & doc/latex/oberdiek/hologo.pdf\\
%   example/hologo-example.tex & doc/latex/oberdiek/example/hologo-example.tex\\
%   test/hologo-test1.tex & doc/latex/oberdiek/test/hologo-test1.tex\\
%   test/hologo-test-spacefactor.tex & doc/latex/oberdiek/test/hologo-test-spacefactor.tex\\
%   test/hologo-test-list.tex & doc/latex/oberdiek/test/hologo-test-list.tex\\
%   hologo.dtx & source/latex/oberdiek/hologo.dtx\\
% \end{tabular}^^A
% }^^A
% \sbox0{\t}^^A
% \ifdim\wd0>\linewidth
%   \begingroup
%     \advance\linewidth by\leftmargin
%     \advance\linewidth by\rightmargin
%   \edef\x{\endgroup
%     \def\noexpand\lw{\the\linewidth}^^A
%   }\x
%   \def\lwbox{^^A
%     \leavevmode
%     \hbox to \linewidth{^^A
%       \kern-\leftmargin\relax
%       \hss
%       \usebox0
%       \hss
%       \kern-\rightmargin\relax
%     }^^A
%   }^^A
%   \ifdim\wd0>\lw
%     \sbox0{\small\t}^^A
%     \ifdim\wd0>\linewidth
%       \ifdim\wd0>\lw
%         \sbox0{\footnotesize\t}^^A
%         \ifdim\wd0>\linewidth
%           \ifdim\wd0>\lw
%             \sbox0{\scriptsize\t}^^A
%             \ifdim\wd0>\linewidth
%               \ifdim\wd0>\lw
%                 \sbox0{\tiny\t}^^A
%                 \ifdim\wd0>\linewidth
%                   \lwbox
%                 \else
%                   \usebox0
%                 \fi
%               \else
%                 \lwbox
%               \fi
%             \else
%               \usebox0
%             \fi
%           \else
%             \lwbox
%           \fi
%         \else
%           \usebox0
%         \fi
%       \else
%         \lwbox
%       \fi
%     \else
%       \usebox0
%     \fi
%   \else
%     \lwbox
%   \fi
% \else
%   \usebox0
% \fi
% \end{quote}
% If you have a \xfile{docstrip.cfg} that configures and enables \docstrip's
% TDS installing feature, then some files can already be in the right
% place, see the documentation of \docstrip.
%
% \subsection{Refresh file name databases}
%
% If your \TeX~distribution
% (\teTeX, \mikTeX, \dots) relies on file name databases, you must refresh
% these. For example, \teTeX\ users run \verb|texhash| or
% \verb|mktexlsr|.
%
% \subsection{Some details for the interested}
%
% \paragraph{Attached source.}
%
% The PDF documentation on CTAN also includes the
% \xfile{.dtx} source file. It can be extracted by
% AcrobatReader 6 or higher. Another option is \textsf{pdftk},
% e.g. unpack the file into the current directory:
% \begin{quote}
%   \verb|pdftk hologo.pdf unpack_files output .|
% \end{quote}
%
% \paragraph{Unpacking with \LaTeX.}
% The \xfile{.dtx} chooses its action depending on the format:
% \begin{description}
% \item[\plainTeX:] Run \docstrip\ and extract the files.
% \item[\LaTeX:] Generate the documentation.
% \end{description}
% If you insist on using \LaTeX\ for \docstrip\ (really,
% \docstrip\ does not need \LaTeX), then inform the autodetect routine
% about your intention:
% \begin{quote}
%   \verb|latex \let\install=y\input{hologo.dtx}|
% \end{quote}
% Do not forget to quote the argument according to the demands
% of your shell.
%
% \paragraph{Generating the documentation.}
% You can use both the \xfile{.dtx} or the \xfile{.drv} to generate
% the documentation. The process can be configured by the
% configuration file \xfile{ltxdoc.cfg}. For instance, put this
% line into this file, if you want to have A4 as paper format:
% \begin{quote}
%   \verb|\PassOptionsToClass{a4paper}{article}|
% \end{quote}
% An example follows how to generate the
% documentation with pdf\LaTeX:
% \begin{quote}
%\begin{verbatim}
%pdflatex hologo.dtx
%makeindex -s gind.ist hologo.idx
%pdflatex hologo.dtx
%makeindex -s gind.ist hologo.idx
%pdflatex hologo.dtx
%\end{verbatim}
% \end{quote}
%
% \section{Catalogue}
%
% The following XML file can be used as source for the
% \href{http://mirror.ctan.org/help/Catalogue/catalogue.html}{\TeX\ Catalogue}.
% The elements \texttt{caption} and \texttt{description} are imported
% from the original XML file from the Catalogue.
% The name of the XML file in the Catalogue is \xfile{hologo.xml}.
%    \begin{macrocode}
%<*catalogue>
<?xml version='1.0' encoding='us-ascii'?>
<!DOCTYPE entry SYSTEM 'catalogue.dtd'>
<entry datestamp='$Date$' modifier='$Author$' id='hologo'>
  <name>hologo</name>
  <caption>A collection of logos with bookmark support.</caption>
  <authorref id='auth:oberdiek'/>
  <copyright owner='Heiko Oberdiek' year='2010-2012'/>
  <license type='lppl1.3'/>
  <version number='1.10'/>
  <description>
    The package defines a single command <tt>\hologo</tt>, whose
    argument is the usual case-confused ASCII version of the logo.
    The command is bookmark-enabled, so that every logo becomes
    available in bookmarks without further work.
    <p/>
    The package is part of the <xref refid='oberdiek'>oberdiek</xref>
    bundle.
  </description>
  <documentation details='Package documentation'
      href='ctan:/macros/latex/contrib/oberdiek/hologo.pdf'/>
  <ctan file='true' path='/macros/latex/contrib/oberdiek/hologo.dtx'/>
  <miktex location='oberdiek'/>
  <texlive location='oberdiek'/>
  <install path='/macros/latex/contrib/oberdiek/oberdiek.tds.zip'/>
</entry>
%</catalogue>
%    \end{macrocode}
%
% \begin{thebibliography}{9}
% \raggedright
%
% \bibitem{btxdoc}
% Oren Patashnik,
% \textit{\hologo{BibTeX}ing},
% 1988-02-08.\\
% \CTAN{biblio/bibtex/base/}
%
% \bibitem{dtklogos}
% Gerd Neugebauer, DANTE,
% \textit{Package \xpackage{dtklogos}},
% 2011-04-25.\\
% \CTAN{usergrps/dante/dtk/dtklogos.sty}
%
% \bibitem{etexman}
% The \hologo{NTS} Team,
% \textit{The \hologo{eTeX} manual},
% 1998-02.\\
% \CTAN{systems/e-tex/v2/doc/}
%
% \bibitem{ExTeX-FAQ}
% The \hologo{ExTeX} group,
% \textit{\hologo{ExTeX}: FAQ -- How is \hologo{ExTeX} typeset?},
% 2007-04-14.\\
% \url{http://www.extex.org/documentation/faq.html}
%
% \bibitem{LyX}
% %@MISC{ LyX,
% %  title = {{LyX 2.0.0 -- The Document Processor [Computer software and manual]}},
% %  author = {{The LyX Team}},
% %  howpublished = {Internet: http://www.lyx.org},
% %  year = {2011-05-08},
% %  note = {Retrieved May 10, 2011, from http://www.lyx.org},
% %  url = {http://www.lyx.org/}
% %}
% The \hologo{LyX} Team,
% \textit{\hologo{LyX} -- The Document Processor},
% 2011-05-08.\\
% \url{http://www.lyx.org/}
%
% \bibitem{OzTeX}
% Andrew Trevorrow,
% \hologo{OzTeX} FAQ: What is the correct way to typeset ``\hologo{OzTeX}''?,
% 2011-09-15 (visited).
% \url{http://www.trevorrow.com/oztex/ozfaq.html#oztex-logo}
%
% \bibitem{PiCTeX}
% Michael Wichura,
% \textit{The \hologo{PiCTeX} macro package},
% 1987-09-21.
% \CTAN{graphics/pictex/}
%
% \bibitem{scrlogo}
% Markus Kohm,
% \textit{\hologo{KOMAScript} Datei \xfile{scrlogo.dtx}},
% 2009-01-30.\\
% \CTAN{install/macros/latex/contrib/komascript.tds.zip}
%
% \end{thebibliography}
%
% \begin{History}
%   \begin{Version}{2010/04/08 v1.0}
%   \item
%     The first version.
%   \end{Version}
%   \begin{Version}{2010/04/16 v1.1}
%   \item
%     \cs{Hologo} added for support of logos at start of a sentence.
%   \item
%     \cs{hologoSetup} and \cs{hologoLogoSetup} added.
%   \item
%     Options \xoption{break}, \xoption{hyphenbreak}, \xoption{spacebreak}
%     added.
%   \item
%     Variant support added by option \xoption{variant}.
%   \end{Version}
%   \begin{Version}{2010/04/24 v1.2}
%   \item
%     \hologo{LaTeX3} added.
%   \item
%     \hologo{VTeX} added.
%   \end{Version}
%   \begin{Version}{2010/11/21 v1.3}
%   \item
%     \hologo{iniTeX}, \hologo{virTeX} added.
%   \end{Version}
%   \begin{Version}{2011/03/25 v1.4}
%   \item
%     \hologo{ConTeXt} with variants added.
%   \item
%     Option \xoption{discretionarybreak} added as refinement for
%     option \xoption{break}.
%   \end{Version}
%   \begin{Version}{2011/04/21 v1.5}
%   \item
%     Wrong TDS directory for test files fixed.
%   \end{Version}
%   \begin{Version}{2011/10/01 v1.6}
%   \item
%     Support for package \xpackage{tex4ht} added.
%   \item
%     Support for \cs{csname} added if \cs{ifincsname} is available.
%   \item
%     New logos:
%     \hologo{(La)TeX},
%     \hologo{biber},
%     \hologo{BibTeX} (\xoption{sc}, \xoption{sf}),
%     \hologo{emTeX},
%     \hologo{ExTeX},
%     \hologo{KOMAScript},
%     \hologo{La},
%     \hologo{LyX},
%     \hologo{MiKTeX},
%     \hologo{NTS},
%     \hologo{OzMF},
%     \hologo{OzMP},
%     \hologo{OzTeX},
%     \hologo{OzTtH},
%     \hologo{PCTeX},
%     \hologo{PiC},
%     \hologo{PiCTeX},
%     \hologo{METAFONT},
%     \hologo{MetaFun},
%     \hologo{METAPOST},
%     \hologo{MetaPost},
%     \hologo{SLiTeX} (\xoption{lift}, \xoption{narrow}, \xoption{simple}),
%     \hologo{SliTeX} (\xoption{narrow}, \xoption{simple}, \xoption{lift}),
%     \hologo{teTeX}.
%   \item
%     Fixes:
%     \hologo{iniTeX},
%     \hologo{pdfLaTeX},
%     \hologo{pdfTeX},
%     \hologo{virTeX}.
%   \item
%     \cs{hologoFontSetup} and \cs{hologoLogoFontSetup} added.
%   \item
%     \cs{hologoVariant} and \cs{HologoVariant} added.
%   \end{Version}
%   \begin{Version}{2011/11/22 v1.7}
%   \item
%     New logos:
%     \hologo{BibTeX8},
%     \hologo{LaTeXML},
%     \hologo{SageTeX},
%     \hologo{TeX4ht},
%     \hologo{TTH}.
%   \item
%     \hologo{Xe} and friends: Driver stuff fixed.
%   \item
%     \hologo{Xe} and friends: Support for italic added.
%   \item
%     \hologo{Xe} and friends: Package support for \xpackage{pgf}
%     and \xpackage{pstricks} added.
%   \end{Version}
%   \begin{Version}{2011/11/29 v1.8}
%   \item
%     New logos:
%     \hologo{HanTheThanh}.
%   \end{Version}
%   \begin{Version}{2011/12/21 v1.9}
%   \item
%     Patch for package \xpackage{ifxetex} added for the case that
%     \cs{newif} is undefined in \hologo{iniTeX}.
%   \item
%     Some fixes for \hologo{iniTeX}.
%   \end{Version}
%   \begin{Version}{2012/04/26 v1.10}
%   \item
%     Fix in bookmark version of logo ``\hologo{HanTheThanh}''.
%   \end{Version}
%   \begin{Version}{2016/05/12 v1.11}
%   \item
%     Update HOLOGO@IfCharExists (previously in texlive)
%   \item define pdfliteral in current luatex.
%   \end{Version}
% \end{History}
%
% \PrintIndex
%
% \Finale
\endinput
%
        \else
          \input hologo.cfg\relax
        \fi
      \fi
    }%
    \ltx@IfUndefined{newread}{%
      \chardef\HOLOGO@temp=15 %
      \def\HOLOGO@CheckRead{%
        \ifeof\HOLOGO@temp
          \HOLOGO@InputIfExists
        \else
          \ifcase\HOLOGO@temp
            \@PackageWarningNoLine{hologo}{%
              Configuration file ignored, because\MessageBreak
              a free read register could not be found%
            }%
          \else
            \begingroup
              \count\ltx@cclv=\HOLOGO@temp
              \advance\ltx@cclv by \ltx@minusone
              \edef\x{\endgroup
                \chardef\noexpand\HOLOGO@temp=\the\count\ltx@cclv
                \relax
              }%
            \x
          \fi
        \fi
      }%
    }{%
      \csname newread\endcsname\HOLOGO@temp
      \HOLOGO@InputIfExists
    }%
  }{%
    \edef\HOLOGO@temp{\pdf@filesize{hologo.cfg}}%
    \ifx\HOLOGO@temp\ltx@empty
    \else
      \ifnum\HOLOGO@temp>0 %
        \begingroup
          \def\x{LaTeX2e}%
        \expandafter\endgroup
        \ifx\fmtname\x
          % \iffalse meta-comment
%
% File: hologo.dtx
% Version: 2016/05/12 v1.11
% Info: A logo collection with bookmark support
%
% Copyright (C) 2010-2012 by
%    Heiko Oberdiek <heiko.oberdiek at googlemail.com>
%
% This work may be distributed and/or modified under the
% conditions of the LaTeX Project Public License, either
% version 1.3c of this license or (at your option) any later
% version. This version of this license is in
%    http://www.latex-project.org/lppl/lppl-1-3c.txt
% and the latest version of this license is in
%    http://www.latex-project.org/lppl.txt
% and version 1.3 or later is part of all distributions of
% LaTeX version 2005/12/01 or later.
%
% This work has the LPPL maintenance status "maintained".
%
% This Current Maintainer of this work is Heiko Oberdiek.
%
% The Base Interpreter refers to any `TeX-Format',
% because some files are installed in TDS:tex/generic//.
%
% This work consists of the main source file hologo.dtx
% and the derived files
%    hologo.sty, hologo.pdf, hologo.ins, hologo.drv, hologo-example.tex,
%    hologo-test1.tex, hologo-test-spacefactor.tex,
%    hologo-test-list.tex.
%
% Distribution:
%    CTAN:macros/latex/contrib/oberdiek/hologo.dtx
%    CTAN:macros/latex/contrib/oberdiek/hologo.pdf
%
% Unpacking:
%    (a) If hologo.ins is present:
%           tex hologo.ins
%    (b) Without hologo.ins:
%           tex hologo.dtx
%    (c) If you insist on using LaTeX
%           latex \let\install=y\input{hologo.dtx}
%        (quote the arguments according to the demands of your shell)
%
% Documentation:
%    (a) If hologo.drv is present:
%           latex hologo.drv
%    (b) Without hologo.drv:
%           latex hologo.dtx; ...
%    The class ltxdoc loads the configuration file ltxdoc.cfg
%    if available. Here you can specify further options, e.g.
%    use A4 as paper format:
%       \PassOptionsToClass{a4paper}{article}
%
%    Programm calls to get the documentation (example):
%       pdflatex hologo.dtx
%       makeindex -s gind.ist hologo.idx
%       pdflatex hologo.dtx
%       makeindex -s gind.ist hologo.idx
%       pdflatex hologo.dtx
%
% Installation:
%    TDS:tex/generic/oberdiek/hologo.sty
%    TDS:doc/latex/oberdiek/hologo.pdf
%    TDS:doc/latex/oberdiek/example/hologo-example.tex
%    TDS:doc/latex/oberdiek/test/hologo-test1.tex
%    TDS:doc/latex/oberdiek/test/hologo-test-spacefactor.tex
%    TDS:doc/latex/oberdiek/test/hologo-test-list.tex
%    TDS:source/latex/oberdiek/hologo.dtx
%
%<*ignore>
\begingroup
  \catcode123=1 %
  \catcode125=2 %
  \def\x{LaTeX2e}%
\expandafter\endgroup
\ifcase 0\ifx\install y1\fi\expandafter
         \ifx\csname processbatchFile\endcsname\relax\else1\fi
         \ifx\fmtname\x\else 1\fi\relax
\else\csname fi\endcsname
%</ignore>
%<*install>
\input docstrip.tex
\Msg{************************************************************************}
\Msg{* Installation}
\Msg{* Package: hologo 2016/05/12 v1.11 A logo collection with bookmark support (HO)}
\Msg{************************************************************************}

\keepsilent
\askforoverwritefalse

\let\MetaPrefix\relax
\preamble

This is a generated file.

Project: hologo
Version: 2016/05/12 v1.11

Copyright (C) 2010-2012 by
   Heiko Oberdiek <heiko.oberdiek at googlemail.com>

This work may be distributed and/or modified under the
conditions of the LaTeX Project Public License, either
version 1.3c of this license or (at your option) any later
version. This version of this license is in
   http://www.latex-project.org/lppl/lppl-1-3c.txt
and the latest version of this license is in
   http://www.latex-project.org/lppl.txt
and version 1.3 or later is part of all distributions of
LaTeX version 2005/12/01 or later.

This work has the LPPL maintenance status "maintained".

This Current Maintainer of this work is Heiko Oberdiek.

The Base Interpreter refers to any `TeX-Format',
because some files are installed in TDS:tex/generic//.

This work consists of the main source file hologo.dtx
and the derived files
   hologo.sty, hologo.pdf, hologo.ins, hologo.drv, hologo-example.tex,
   hologo-test1.tex, hologo-test-spacefactor.tex,
   hologo-test-list.tex.

\endpreamble
\let\MetaPrefix\DoubleperCent

\generate{%
  \file{hologo.ins}{\from{hologo.dtx}{install}}%
  \file{hologo.drv}{\from{hologo.dtx}{driver}}%
  \usedir{tex/generic/oberdiek}%
  \file{hologo.sty}{\from{hologo.dtx}{package}}%
  \usedir{doc/latex/oberdiek/example}%
  \file{hologo-example.tex}{\from{hologo.dtx}{example}}%
  \usedir{doc/latex/oberdiek/test}%
  \file{hologo-test1.tex}{\from{hologo.dtx}{test1}}%
  \file{hologo-test-spacefactor.tex}{\from{hologo.dtx}{test-spacefactor}}%
  \file{hologo-test-list.tex}{\from{hologo.dtx}{test-list}}%
  \nopreamble
  \nopostamble
  \usedir{source/latex/oberdiek/catalogue}%
  \file{hologo.xml}{\from{hologo.dtx}{catalogue}}%
}

\catcode32=13\relax% active space
\let =\space%
\Msg{************************************************************************}
\Msg{*}
\Msg{* To finish the installation you have to move the following}
\Msg{* file into a directory searched by TeX:}
\Msg{*}
\Msg{*     hologo.sty}
\Msg{*}
\Msg{* To produce the documentation run the file `hologo.drv'}
\Msg{* through LaTeX.}
\Msg{*}
\Msg{* Happy TeXing!}
\Msg{*}
\Msg{************************************************************************}

\endbatchfile
%</install>
%<*ignore>
\fi
%</ignore>
%<*driver>
\NeedsTeXFormat{LaTeX2e}
\ProvidesFile{hologo.drv}%
  [2016/05/12 v1.11 A logo collection with bookmark support (HO)]%
\documentclass{ltxdoc}
\usepackage{holtxdoc}[2011/11/22]
\usepackage{hologo}[2016/05/12]
\usepackage{longtable}
\usepackage{array}
\usepackage{paralist}
%\usepackage[T1]{fontenc}
%\usepackage{lmodern}
\begin{document}
  \DocInput{hologo.dtx}%
\end{document}
%</driver>
% \fi
%
%
% \CharacterTable
%  {Upper-case    \A\B\C\D\E\F\G\H\I\J\K\L\M\N\O\P\Q\R\S\T\U\V\W\X\Y\Z
%   Lower-case    \a\b\c\d\e\f\g\h\i\j\k\l\m\n\o\p\q\r\s\t\u\v\w\x\y\z
%   Digits        \0\1\2\3\4\5\6\7\8\9
%   Exclamation   \!     Double quote  \"     Hash (number) \#
%   Dollar        \$     Percent       \%     Ampersand     \&
%   Acute accent  \'     Left paren    \(     Right paren   \)
%   Asterisk      \*     Plus          \+     Comma         \,
%   Minus         \-     Point         \.     Solidus       \/
%   Colon         \:     Semicolon     \;     Less than     \<
%   Equals        \=     Greater than  \>     Question mark \?
%   Commercial at \@     Left bracket  \[     Backslash     \\
%   Right bracket \]     Circumflex    \^     Underscore    \_
%   Grave accent  \`     Left brace    \{     Vertical bar  \|
%   Right brace   \}     Tilde         \~}
%
% \GetFileInfo{hologo.drv}
%
% \title{The \xpackage{hologo} package}
% \date{2016/05/12 v1.11}
% \author{Heiko Oberdiek\\\xemail{heiko.oberdiek at googlemail.com}}
%
% \maketitle
%
% \begin{abstract}
% This package starts a collection of logos with support for bookmarks
% strings.
% \end{abstract}
%
% \tableofcontents
%
% \section{Documentation}
%
% \subsection{Logo macros}
%
% \begin{declcs}{hologo} \M{name}
% \end{declcs}
% Macro \cs{hologo} sets the logo with name \meta{name}.
% The following table shows the supported names.
%
% \begingroup
%   \def\hologoEntry#1#2#3{^^A
%     #1&#2&\hologoLogoSetup{#1}{variant=#2}\hologo{#1}&#3\tabularnewline
%   }
%   \begin{longtable}{>{\ttfamily}l>{\ttfamily}lll}
%     \rmfamily\bfseries{name} & \rmfamily\bfseries variant
%     & \bfseries logo & \bfseries since\\
%     \hline
%     \endhead
%     \hologoList
%   \end{longtable}
% \endgroup
%
% \begin{declcs}{Hologo} \M{name}
% \end{declcs}
% Macro \cs{Hologo} starts the logo \meta{name} with an uppercase
% letter. As an exception small greek letters are not converted
% to uppercase. Examples, see \hologo{eTeX} and \hologo{ExTeX}.
%
% \subsection{Setup macros}
%
% The package does not support package options, but the following
% setup macros can be used to set options.
%
% \begin{declcs}{hologoSetup} \M{key value list}
% \end{declcs}
% Macro \cs{hologoSetup} sets global options.
%
% \begin{declcs}{hologoLogoSetup} \M{logo} \M{key value list}
% \end{declcs}
% Some options can also be used to configure a logo.
% These settings take precedence over global option settings.
%
% \subsection{Options}\label{sec:options}
%
% There are boolean and string options:
% \begin{description}
% \item[Boolean option:]
% It takes |true| or |false|
% as value. If the value is omitted, then |true| is used.
% \item[String option:]
% A value must be given as string. (But the string might be empty.)
% \end{description}
% The following options can be used both in \cs{hologoSetup}
% and \cs{hologoLogoSetup}:
% \begin{description}
% \def\entry#1{\item[\xoption{#1}:]}
% \entry{break}
%   enables or disables line breaks inside the logo. This setting is
%   refined by options \xoption{hyphenbreak}, \xoption{spacebreak}
%   or \xoption{discretionarybreak}.
%   Default is |false|.
% \entry{hyphenbreak}
%   enables or disables the line break right after the hyphen character.
% \entry{spacebreak}
%   enables or disables line breaks at space characters.
% \entry{discretionarybreak}
%   enables or disables line breaks at hyphenation points
%   (inserted by \cs{-}).
% \end{description}
% Macro \cs{hologoLogoSetup} also knows:
% \begin{description}
% \item[\xoption{variant}:]
%   This is a string option. It specifies a variant of a logo that
%   must exist. An empty string selects the package default variant.
% \end{description}
% Example:
% \begin{quote}
%   |\hologoSetup{break=false}|\\
%   |\hologoLogoSetup{plainTeX}{variant=hyphen,hyphenbreak}|\\
%   Then ``plain-\TeX'' contains one break point after the hyphen.
% \end{quote}
%
% \subsection{Driver options}
%
% Sometimes graphical operations are needed to construct some
% glyphs (e.g.\ \hologo{XeTeX}). If package \xpackage{graphics}
% or package \xpackage{pgf} are found, then the macros are taken
% from there. Otherwise the packge defines its own operations
% and therefore needs the driver information. Many drivers are
% detected automatically (\hologo{pdfTeX}/\hologo{LuaTeX}
% in PDF mode, \hologo{XeTeX}, \hologo{VTeX}). These have precedence
% over a driver option. The driver can be given as package option
% or using \cs{hologoDriverSetup}.
% The following list contains the recognized driver options:
% \begin{itemize}
% \item \xoption{pdftex}, \xoption{luatex}
% \item \xoption{dvipdfm}, \xoption{dvipdfmx}
% \item \xoption{dvips}, \xoption{dvipsone}, \xoption{xdvi}
% \item \xoption{xetex}
% \item \xoption{vtex}
% \end{itemize}
% The left driver of a line is the driver name that is used internally.
% The following names are aliases for drivers that use the
% same method. Therefore the entry in the \xext{log} file for
% the used driver prints the internally used driver name.
% \begin{description}
% \item[\xoption{driverfallback}:]
%   This option expects a driver that is used,
%   if the driver could not be detected automatically.
% \end{description}
%
% \begin{declcs}{hologoDriverSetup} \M{driver option}
% \end{declcs}
% The driver can also be configured after package loading
% using \cs{hologoDriverSetup}, also the way for \hologo{plainTeX}
% to setup the driver.
%
% \subsection{Font setup}
%
% Some logos require a special font, but should also be usable by
% \hologo{plainTeX}. Therefore the package provides some ways
% to influence the font settings. The options below
% take font settings as values. Both font commands
% such as \cs{sffamily} and macros that take one argument
% like \cs{textsf} can be used.
%
% \begin{declcs}{hologoFontSetup} \M{key value list}
% \end{declcs}
% Macro \cs{hologoFontSetup} sets the fonts for all logos.
% Supported keys:
% \begin{description}
% \def\entry#1{\item[\xoption{#1}:]}
% \entry{general}
%   This font is used for all logos. The default is empty.
%   That means no special font is used.
% \entry{bibsf}
%   This font is used for
%   {\hologoLogoSetup{BibTeX}{variant=sf}\hologo{BibTeX}}
%   with variant \xoption{sf}.
% \entry{rm}
%   This font is a serif font. It is used for \hologo{ExTeX}.
% \entry{sc}
%   This font specifies a small caps font. It is used for
%   {\hologoLogoSetup{BibTeX}{variant=sc}\hologo{BibTeX}}
%   with variant \xoption{sc}.
% \entry{sf}
%   This font specifies a sans serif font. The default
%   is \cs{sffamily}, then \cs{sf} is tried. Otherwise
%   a warning is given. It is used by \hologo{KOMAScript}.
% \entry{sy}
%   This is the font for math symbols (e.g. cmsy).
%   It is used by \hologo{AmS}, \hologo{NTS}, \hologo{ExTeX}.
% \entry{logo}
%   \hologo{METAFONT} and \hologo{METAPOST} are using that font.
%   In \hologo{LaTeX} \cs{logofamily} is used and
%   the definitions of package \xpackage{mflogo} are used
%   if the package is not loaded.
%   Otherwise the \cs{tenlogo} is used and defined
%   if it does not already exists.
% \end{description}
%
% \begin{declcs}{hologoLogoFontSetup} \M{logo} \M{key value list}
% \end{declcs}
% Fonts can also be set for a logo or logo component separately,
% see the following list.
% The keys are the same as for \cs{hologoFontSetup}.
%
% \begin{longtable}{>{\ttfamily}l>{\sffamily}ll}
%   \meta{logo} & keys & result\\
%   \hline
%   \endhead
%   BibTeX & bibsf & {\hologoLogoSetup{BibTeX}{variant=sf}\hologo{BibTeX}}\\[.5ex]
%   BibTeX & sc & {\hologoLogoSetup{BibTeX}{variant=sc}\hologo{BibTeX}}\\[.5ex]
%   ExTeX & rm & \hologo{ExTeX}\\
%   SliTeX & rm & \hologo{SliTeX}\\[.5ex]
%   AmS & sy & \hologo{AmS}\\
%   ExTeX & sy & \hologo{ExTeX}\\
%   NTS & sy & \hologo{NTS}\\[.5ex]
%   KOMAScript & sf & \hologo{KOMAScript}\\[.5ex]
%   METAFONT & logo & \hologo{METAFONT}\\
%   METAPOST & logo & \hologo{METAPOST}\\[.5ex]
%   SliTeX & sc \hologo{SliTeX}
% \end{longtable}
%
% \subsubsection{Font order}
%
% For all logos the font \xoption{general} is applied first.
% Example:
%\begin{quote}
%|\hologoFontSetup{general=\color{red}}|
%\end{quote}
% will print red logos.
% Then if the font uses a special font \xoption{sf}, for example,
% the font is applied that is setup by \cs{hologoLogoFontSetup}.
% If this font is not setup, then the common font setup
% by \cs{hologoFontSetup} is used. Otherwise a warning is given,
% that there is no font configured.
%
% \subsection{Additional user macros}
%
% Usually a variant of a logo is configured by using
% \cs{hologoLogoSetup}, because it is bad style to mix
% different variants of the same logo in the same text.
% There the following macros are a convenience for testing.
%
% \begin{declcs}{hologoVariant} \M{name} \M{variant}\\
%   \cs{HologoVariant} \M{name} \M{variant}
% \end{declcs}
% Logo \meta{name} is set using \meta{variant} that specifies
% explicitely which variant of the macro is used. If the argument
% is empty, then the default form of the logo is used
% (configurable by \cs{hologoLogoSetup}).
%
% \cs{HologoVariant} is used if the logo is set in a context
% that needs an uppercase first letter (beginning of a sentence, \dots).
%
% \begin{declcs}{hologoList}\\
%   \cs{hologoEntry} \M{logo} \M{variant} \M{since}
% \end{declcs}
% Macro \cs{hologoList} contains all logos that are provided
% by the package including variants. The list consists of calls
% of \cs{hologoEntry} with three arguments starting with the
% logo name \meta{logo} and its variant \meta{variant}. An empty
% variant means the current default. Argument \meta{since} specifies
% with version of the package \xpackage{hologo} is needed to get
% the logo. If the logo is fixed, then the date gets updated.
% Therefore the date \meta{since} is not exactly the date of
% the first introduction, but rather the date of the latest fix.
%
% Before \cs{hologoList} can be used, macro \cs{hologoEntry} needs
% a definition. The example file in section \ref{sec:example}
% shows applications of \cs{hologoList}.
%
% \subsection{Supported contexts}
%
% Macros \cs{hologo} and friends support special contexts:
% \begin{itemize}
% \item \hologo{LaTeX}'s protection mechanism.
% \item Bookmarks of package \xpackage{hyperref}.
% \item Package \xpackage{tex4ht}.
% \item The macros can be used inside \cs{csname} constructs,
%   if \cs{ifincsname} is available (\hologo{pdfTeX}, \hologo{XeTeX},
%   \hologo{LuaTeX}).
% \end{itemize}
%
% \subsection{Example}
% \label{sec:example}
%
% The following example prints the logos in different fonts.
%    \begin{macrocode}
%<*example>
%<<verbatim
\NeedsTeXFormat{LaTeX2e}
\documentclass[a4paper]{article}
\usepackage[
  hmargin=20mm,
  vmargin=20mm,
]{geometry}
\pagestyle{empty}
\usepackage{hologo}[2016/05/12]
\usepackage{longtable}
\usepackage{array}
\setlength{\extrarowheight}{2pt}
\usepackage[T1]{fontenc}
\usepackage{lmodern}
\usepackage{pdflscape}
\usepackage[
  pdfencoding=auto,
]{hyperref}
\hypersetup{
  pdfauthor={Heiko Oberdiek},
  pdftitle={Example for package `hologo'},
  pdfsubject={Logos with fonts lmr, lmss, qtm, qpl, qhv},
}
\usepackage{bookmark}

% Print the logo list on the console

\begingroup
  \typeout{}%
  \typeout{*** Begin of logo list ***}%
  \newcommand*{\hologoEntry}[3]{%
    \typeout{#1 \ifx\\#2\\\else(#2) \fi[#3]}%
  }%
  \hologoList
  \typeout{*** End of logo list ***}%
  \typeout{}%
\endgroup

\begin{document}
\begin{landscape}

  \section{Example file for package `hologo'}

  % Table for font names

  \begin{longtable}{>{\bfseries}ll}
    \textbf{font} & \textbf{Font name}\\
    \hline
    lmr & Latin Modern Roman\\
    lmss & Latin Modern Sans\\
    qtm & \TeX\ Gyre Termes\\
    qhv & \TeX\ Gyre Heros\\
    qpl & \TeX\ Gyre Pagella\\
  \end{longtable}

  % Logo list with logos in different fonts

  \begingroup
    \newcommand*{\SetVariant}[2]{%
      \ifx\\#2\\%
      \else
        \hologoLogoSetup{#1}{variant=#2}%
      \fi
    }%
    \newcommand*{\hologoEntry}[3]{%
      \SetVariant{#1}{#2}%
      \raisebox{1em}[0pt][0pt]{\hypertarget{#1@#2}{}}%
      \bookmark[%
        dest={#1@#2},%
      ]{%
        #1\ifx\\#2\\\else\space(#2)\fi: \Hologo{#1}, \hologo{#1} %
        [Unicode]%
      }%
      \hypersetup{unicode=false}%
      \bookmark[%
        dest={#1@#2},%
      ]{%
        #1\ifx\\#2\\\else\space(#2)\fi: \Hologo{#1}, \hologo{#1} %
        [PDFDocEncoding]%
      }%
      \texttt{#1}%
      &%
      \texttt{#2}%
      &%
      \Hologo{#1}%
      &%
      \SetVariant{#1}{#2}%
      \hologo{#1}%
      &%
      \SetVariant{#1}{#2}%
      \fontfamily{qtm}\selectfont
      \hologo{#1}%
      &%
      \SetVariant{#1}{#2}%
      \fontfamily{qpl}\selectfont
      \hologo{#1}%
      &%
      \SetVariant{#1}{#2}%
      \textsf{\hologo{#1}}%
      &%
      \SetVariant{#1}{#2}%
      \fontfamily{qhv}\selectfont
      \hologo{#1}%
      \tabularnewline
    }%
    \begin{longtable}{llllllll}%
      \textbf{\textit{logo}} & \textbf{\textit{variant}} &
      \texttt{\string\Hologo} &
      \textbf{lmr} & \textbf{qtm} & \textbf{qpl} &
      \textbf{lmss} & \textbf{qhv}
      \tabularnewline
      \hline
      \endhead
      \hologoList
    \end{longtable}%
  \endgroup

\end{landscape}
\end{document}
%verbatim
%</example>
%    \end{macrocode}
%
% \StopEventually{
% }
%
% \section{Implementation}
%    \begin{macrocode}
%<*package>
%    \end{macrocode}
%    Reload check, especially if the package is not used with \LaTeX.
%    \begin{macrocode}
\begingroup\catcode61\catcode48\catcode32=10\relax%
  \catcode13=5 % ^^M
  \endlinechar=13 %
  \catcode35=6 % #
  \catcode39=12 % '
  \catcode44=12 % ,
  \catcode45=12 % -
  \catcode46=12 % .
  \catcode58=12 % :
  \catcode64=11 % @
  \catcode123=1 % {
  \catcode125=2 % }
  \expandafter\let\expandafter\x\csname ver@hologo.sty\endcsname
  \ifx\x\relax % plain-TeX, first loading
  \else
    \def\empty{}%
    \ifx\x\empty % LaTeX, first loading,
      % variable is initialized, but \ProvidesPackage not yet seen
    \else
      \expandafter\ifx\csname PackageInfo\endcsname\relax
        \def\x#1#2{%
          \immediate\write-1{Package #1 Info: #2.}%
        }%
      \else
        \def\x#1#2{\PackageInfo{#1}{#2, stopped}}%
      \fi
      \x{hologo}{The package is already loaded}%
      \aftergroup\endinput
    \fi
  \fi
\endgroup%
%    \end{macrocode}
%    Package identification:
%    \begin{macrocode}
\begingroup\catcode61\catcode48\catcode32=10\relax%
  \catcode13=5 % ^^M
  \endlinechar=13 %
  \catcode35=6 % #
  \catcode39=12 % '
  \catcode40=12 % (
  \catcode41=12 % )
  \catcode44=12 % ,
  \catcode45=12 % -
  \catcode46=12 % .
  \catcode47=12 % /
  \catcode58=12 % :
  \catcode64=11 % @
  \catcode91=12 % [
  \catcode93=12 % ]
  \catcode123=1 % {
  \catcode125=2 % }
  \expandafter\ifx\csname ProvidesPackage\endcsname\relax
    \def\x#1#2#3[#4]{\endgroup
      \immediate\write-1{Package: #3 #4}%
      \xdef#1{#4}%
    }%
  \else
    \def\x#1#2[#3]{\endgroup
      #2[{#3}]%
      \ifx#1\@undefined
        \xdef#1{#3}%
      \fi
      \ifx#1\relax
        \xdef#1{#3}%
      \fi
    }%
  \fi
\expandafter\x\csname ver@hologo.sty\endcsname
\ProvidesPackage{hologo}%
  [2016/05/12 v1.11 A logo collection with bookmark support (HO)]%
%    \end{macrocode}
%
%    \begin{macrocode}
\begingroup\catcode61\catcode48\catcode32=10\relax%
  \catcode13=5 % ^^M
  \endlinechar=13 %
  \catcode123=1 % {
  \catcode125=2 % }
  \catcode64=11 % @
  \def\x{\endgroup
    \expandafter\edef\csname HOLOGO@AtEnd\endcsname{%
      \endlinechar=\the\endlinechar\relax
      \catcode13=\the\catcode13\relax
      \catcode32=\the\catcode32\relax
      \catcode35=\the\catcode35\relax
      \catcode61=\the\catcode61\relax
      \catcode64=\the\catcode64\relax
      \catcode123=\the\catcode123\relax
      \catcode125=\the\catcode125\relax
    }%
  }%
\x\catcode61\catcode48\catcode32=10\relax%
\catcode13=5 % ^^M
\endlinechar=13 %
\catcode35=6 % #
\catcode64=11 % @
\catcode123=1 % {
\catcode125=2 % }
\def\TMP@EnsureCode#1#2{%
  \edef\HOLOGO@AtEnd{%
    \HOLOGO@AtEnd
    \catcode#1=\the\catcode#1\relax
  }%
  \catcode#1=#2\relax
}
\TMP@EnsureCode{10}{12}% ^^J
\TMP@EnsureCode{33}{12}% !
\TMP@EnsureCode{34}{12}% "
\TMP@EnsureCode{36}{3}% $
\TMP@EnsureCode{38}{4}% &
\TMP@EnsureCode{39}{12}% '
\TMP@EnsureCode{40}{12}% (
\TMP@EnsureCode{41}{12}% )
\TMP@EnsureCode{42}{12}% *
\TMP@EnsureCode{43}{12}% +
\TMP@EnsureCode{44}{12}% ,
\TMP@EnsureCode{45}{12}% -
\TMP@EnsureCode{46}{12}% .
\TMP@EnsureCode{47}{12}% /
\TMP@EnsureCode{58}{12}% :
\TMP@EnsureCode{59}{12}% ;
\TMP@EnsureCode{60}{12}% <
\TMP@EnsureCode{62}{12}% >
\TMP@EnsureCode{63}{12}% ?
\TMP@EnsureCode{91}{12}% [
\TMP@EnsureCode{93}{12}% ]
\TMP@EnsureCode{94}{7}% ^ (superscript)
\TMP@EnsureCode{95}{8}% _ (subscript)
\TMP@EnsureCode{96}{12}% `
\TMP@EnsureCode{124}{12}% |
\edef\HOLOGO@AtEnd{%
  \HOLOGO@AtEnd
  \escapechar\the\escapechar\relax
  \noexpand\endinput
}
\escapechar=92 %
%    \end{macrocode}
%
% \subsection{Logo list}
%
%    \begin{macro}{\hologoList}
%    \begin{macrocode}
\def\hologoList{%
  \hologoEntry{(La)TeX}{}{2011/10/01}%
  \hologoEntry{AmSLaTeX}{}{2010/04/16}%
  \hologoEntry{AmSTeX}{}{2010/04/16}%
  \hologoEntry{biber}{}{2011/10/01}%
  \hologoEntry{BibTeX}{}{2011/10/01}%
  \hologoEntry{BibTeX}{sf}{2011/10/01}%
  \hologoEntry{BibTeX}{sc}{2011/10/01}%
  \hologoEntry{BibTeX8}{}{2011/11/22}%
  \hologoEntry{ConTeXt}{}{2011/03/25}%
  \hologoEntry{ConTeXt}{narrow}{2011/03/25}%
  \hologoEntry{ConTeXt}{simple}{2011/03/25}%
  \hologoEntry{emTeX}{}{2010/04/26}%
  \hologoEntry{eTeX}{}{2010/04/08}%
  \hologoEntry{ExTeX}{}{2011/10/01}%
  \hologoEntry{HanTheThanh}{}{2011/11/29}%
  \hologoEntry{iniTeX}{}{2011/10/01}%
  \hologoEntry{KOMAScript}{}{2011/10/01}%
  \hologoEntry{La}{}{2010/05/08}%
  \hologoEntry{LaTeX}{}{2010/04/08}%
  \hologoEntry{LaTeX2e}{}{2010/04/08}%
  \hologoEntry{LaTeX3}{}{2010/04/24}%
  \hologoEntry{LaTeXe}{}{2010/04/08}%
  \hologoEntry{LaTeXML}{}{2011/11/22}%
  \hologoEntry{LaTeXTeX}{}{2011/10/01}%
  \hologoEntry{LuaLaTeX}{}{2010/04/08}%
  \hologoEntry{LuaTeX}{}{2010/04/08}%
  \hologoEntry{LyX}{}{2011/10/01}%
  \hologoEntry{METAFONT}{}{2011/10/01}%
  \hologoEntry{MetaFun}{}{2011/10/01}%
  \hologoEntry{METAPOST}{}{2011/10/01}%
  \hologoEntry{MetaPost}{}{2011/10/01}%
  \hologoEntry{MiKTeX}{}{2011/10/01}%
  \hologoEntry{NTS}{}{2011/10/01}%
  \hologoEntry{OzMF}{}{2011/10/01}%
  \hologoEntry{OzMP}{}{2011/10/01}%
  \hologoEntry{OzTeX}{}{2011/10/01}%
  \hologoEntry{OzTtH}{}{2011/10/01}%
  \hologoEntry{PCTeX}{}{2011/10/01}%
  \hologoEntry{pdfTeX}{}{2011/10/01}%
  \hologoEntry{pdfLaTeX}{}{2011/10/01}%
  \hologoEntry{PiC}{}{2011/10/01}%
  \hologoEntry{PiCTeX}{}{2011/10/01}%
  \hologoEntry{plainTeX}{}{2010/04/08}%
  \hologoEntry{plainTeX}{space}{2010/04/16}%
  \hologoEntry{plainTeX}{hyphen}{2010/04/16}%
  \hologoEntry{plainTeX}{runtogether}{2010/04/16}%
  \hologoEntry{SageTeX}{}{2011/11/22}%
  \hologoEntry{SLiTeX}{}{2011/10/01}%
  \hologoEntry{SLiTeX}{lift}{2011/10/01}%
  \hologoEntry{SLiTeX}{narrow}{2011/10/01}%
  \hologoEntry{SLiTeX}{simple}{2011/10/01}%
  \hologoEntry{SliTeX}{}{2011/10/01}%
  \hologoEntry{SliTeX}{narrow}{2011/10/01}%
  \hologoEntry{SliTeX}{simple}{2011/10/01}%
  \hologoEntry{SliTeX}{lift}{2011/10/01}%
  \hologoEntry{teTeX}{}{2011/10/01}%
  \hologoEntry{TeX}{}{2010/04/08}%
  \hologoEntry{TeX4ht}{}{2011/11/22}%
  \hologoEntry{TTH}{}{2011/11/22}%
  \hologoEntry{virTeX}{}{2011/10/01}%
  \hologoEntry{VTeX}{}{2010/04/24}%
  \hologoEntry{Xe}{}{2010/04/08}%
  \hologoEntry{XeLaTeX}{}{2010/04/08}%
  \hologoEntry{XeTeX}{}{2010/04/08}%
}
%    \end{macrocode}
%    \end{macro}
%
% \subsection{Load resources}
%
%    \begin{macrocode}
\begingroup\expandafter\expandafter\expandafter\endgroup
\expandafter\ifx\csname RequirePackage\endcsname\relax
  \def\TMP@RequirePackage#1[#2]{%
    \begingroup\expandafter\expandafter\expandafter\endgroup
    \expandafter\ifx\csname ver@#1.sty\endcsname\relax
      \input #1.sty\relax
    \fi
  }%
  \TMP@RequirePackage{ltxcmds}[2011/02/04]%
  \TMP@RequirePackage{infwarerr}[2010/04/08]%
  \TMP@RequirePackage{kvsetkeys}[2010/03/01]%
  \TMP@RequirePackage{kvdefinekeys}[2010/03/01]%
  \TMP@RequirePackage{pdftexcmds}[2010/04/01]%
  \TMP@RequirePackage{ifpdf}[2010/01/28]%
  \TMP@RequirePackage{ifluatex}[2010/03/01]%
  \ltx@IfUndefined{newif}{%
    \expandafter\let\csname newif\endcsname\ltx@newif
  }{}%
  \TMP@RequirePackage{ifxetex}[2009/01/23]%
  \TMP@RequirePackage{ifvtex}[2010/03/01]%
\else
  \RequirePackage{ltxcmds}[2011/02/04]%
  \RequirePackage{infwarerr}[2010/04/08]%
  \RequirePackage{kvsetkeys}[2010/03/01]%
  \RequirePackage{kvdefinekeys}[2010/03/01]%
  \RequirePackage{pdftexcmds}[2010/04/01]%
  \RequirePackage{ifpdf}[2010/01/28]%
  \RequirePackage{ifluatex}[2010/03/01]%
  \RequirePackage{ifxetex}[2009/01/23]%
  \RequirePackage{ifvtex}[2010/03/01]%
\fi
%    \end{macrocode}
%
%    \begin{macro}{\HOLOGO@IfDefined}
%    \begin{macrocode}
\def\HOLOGO@IfExists#1{%
  \ifx\@undefined#1%
    \expandafter\ltx@secondoftwo
  \else
    \ifx\relax#1%
      \expandafter\ltx@secondoftwo
    \else
      \expandafter\expandafter\expandafter\ltx@firstoftwo
    \fi
  \fi
}
%    \end{macrocode}
%    \end{macro}
%
% \subsection{Setup macros}
%
%    \begin{macro}{\hologoSetup}
%    \begin{macrocode}
\def\hologoSetup{%
  \let\HOLOGO@name\relax
  \HOLOGO@Setup
}
%    \end{macrocode}
%    \end{macro}
%
%    \begin{macro}{\hologoLogoSetup}
%    \begin{macrocode}
\def\hologoLogoSetup#1{%
  \edef\HOLOGO@name{#1}%
  \ltx@IfUndefined{HoLogo@\HOLOGO@name}{%
    \@PackageError{hologo}{%
      Unknown logo `\HOLOGO@name'%
    }\@ehc
    \ltx@gobble
  }{%
    \HOLOGO@Setup
  }%
}
%    \end{macrocode}
%    \end{macro}
%
%    \begin{macro}{\HOLOGO@Setup}
%    \begin{macrocode}
\def\HOLOGO@Setup{%
  \kvsetkeys{HoLogo}%
}
%    \end{macrocode}
%    \end{macro}
%
% \subsection{Options}
%
%    \begin{macro}{\HOLOGO@DeclareBoolOption}
%    \begin{macrocode}
\def\HOLOGO@DeclareBoolOption#1{%
  \expandafter\chardef\csname HOLOGOOPT@#1\endcsname\ltx@zero
  \kv@define@key{HoLogo}{#1}[true]{%
    \def\HOLOGO@temp{##1}%
    \ifx\HOLOGO@temp\HOLOGO@true
      \ifx\HOLOGO@name\relax
        \expandafter\chardef\csname HOLOGOOPT@#1\endcsname=\ltx@one
      \else
        \expandafter\chardef\csname
        HoLogoOpt@#1@\HOLOGO@name\endcsname\ltx@one
      \fi
      \HOLOGO@SetBreakAll{#1}%
    \else
      \ifx\HOLOGO@temp\HOLOGO@false
        \ifx\HOLOGO@name\relax
          \expandafter\chardef\csname HOLOGOOPT@#1\endcsname=\ltx@zero
        \else
          \expandafter\chardef\csname
          HoLogoOpt@#1@\HOLOGO@name\endcsname=\ltx@zero
        \fi
        \HOLOGO@SetBreakAll{#1}%
      \else
        \@PackageError{hologo}{%
          Unknown value `##1' for boolean option `#1'.\MessageBreak
          Known values are `true' and `false'%
        }\@ehc
      \fi
    \fi
  }%
}
%    \end{macrocode}
%    \end{macro}
%
%    \begin{macro}{\HOLOGO@SetBreakAll}
%    \begin{macrocode}
\def\HOLOGO@SetBreakAll#1{%
  \def\HOLOGO@temp{#1}%
  \ifx\HOLOGO@temp\HOLOGO@break
    \ifx\HOLOGO@name\relax
      \chardef\HOLOGOOPT@hyphenbreak=\HOLOGOOPT@break
      \chardef\HOLOGOOPT@spacebreak=\HOLOGOOPT@break
      \chardef\HOLOGOOPT@discretionarybreak=\HOLOGOOPT@break
    \else
      \expandafter\chardef
         \csname HoLogoOpt@hyphenbreak@\HOLOGO@name\endcsname=%
         \csname HoLogoOpt@break@\HOLOGO@name\endcsname
      \expandafter\chardef
         \csname HoLogoOpt@spacebreak@\HOLOGO@name\endcsname=%
         \csname HoLogoOpt@break@\HOLOGO@name\endcsname
      \expandafter\chardef
         \csname HoLogoOpt@discretionarybreak@\HOLOGO@name
             \endcsname=%
         \csname HoLogoOpt@break@\HOLOGO@name\endcsname
    \fi
  \fi
}
%    \end{macrocode}
%    \end{macro}
%
%    \begin{macro}{\HOLOGO@true}
%    \begin{macrocode}
\def\HOLOGO@true{true}
%    \end{macrocode}
%    \end{macro}
%    \begin{macro}{\HOLOGO@false}
%    \begin{macrocode}
\def\HOLOGO@false{false}
%    \end{macrocode}
%    \end{macro}
%    \begin{macro}{\HOLOGO@break}
%    \begin{macrocode}
\def\HOLOGO@break{break}
%    \end{macrocode}
%    \end{macro}
%
%    \begin{macrocode}
\HOLOGO@DeclareBoolOption{break}
\HOLOGO@DeclareBoolOption{hyphenbreak}
\HOLOGO@DeclareBoolOption{spacebreak}
\HOLOGO@DeclareBoolOption{discretionarybreak}
%    \end{macrocode}
%
%    \begin{macrocode}
\kv@define@key{HoLogo}{variant}{%
  \ifx\HOLOGO@name\relax
    \@PackageError{hologo}{%
      Option `variant' is not available in \string\hologoSetup,%
      \MessageBreak
      Use \string\hologoLogoSetup\space instead%
    }\@ehc
  \else
    \edef\HOLOGO@temp{#1}%
    \ifx\HOLOGO@temp\ltx@empty
      \expandafter
      \let\csname HoLogoOpt@variant@\HOLOGO@name\endcsname\@undefined
    \else
      \ltx@IfUndefined{HoLogo@\HOLOGO@name @\HOLOGO@temp}{%
        \@PackageError{hologo}{%
          Unknown variant `\HOLOGO@temp' of logo `\HOLOGO@name'%
        }\@ehc
      }{%
        \expandafter
        \let\csname HoLogoOpt@variant@\HOLOGO@name\endcsname
            \HOLOGO@temp
      }%
    \fi
  \fi
}
%    \end{macrocode}
%
%    \begin{macro}{\HOLOGO@Variant}
%    \begin{macrocode}
\def\HOLOGO@Variant#1{%
  #1%
  \ltx@ifundefined{HoLogoOpt@variant@#1}{%
  }{%
    @\csname HoLogoOpt@variant@#1\endcsname
  }%
}
%    \end{macrocode}
%    \end{macro}
%
% \subsection{Break/no-break support}
%
%    \begin{macro}{\HOLOGO@space}
%    \begin{macrocode}
\def\HOLOGO@space{%
  \ltx@ifundefined{HoLogoOpt@spacebreak@\HOLOGO@name}{%
    \ltx@ifundefined{HoLogoOpt@break@\HOLOGO@name}{%
      \chardef\HOLOGO@temp=\HOLOGOOPT@spacebreak
    }{%
      \chardef\HOLOGO@temp=%
        \csname HoLogoOpt@break@\HOLOGO@name\endcsname
    }%
  }{%
    \chardef\HOLOGO@temp=%
      \csname HoLogoOpt@spacebreak@\HOLOGO@name\endcsname
  }%
  \ifcase\HOLOGO@temp
    \penalty10000 %
  \fi
  \ltx@space
}
%    \end{macrocode}
%    \end{macro}
%
%    \begin{macro}{\HOLOGO@hyphen}
%    \begin{macrocode}
\def\HOLOGO@hyphen{%
  \ltx@ifundefined{HoLogoOpt@hyphenbreak@\HOLOGO@name}{%
    \ltx@ifundefined{HoLogoOpt@break@\HOLOGO@name}{%
      \chardef\HOLOGO@temp=\HOLOGOOPT@hyphenbreak
    }{%
      \chardef\HOLOGO@temp=%
        \csname HoLogoOpt@break@\HOLOGO@name\endcsname
    }%
  }{%
    \chardef\HOLOGO@temp=%
      \csname HoLogoOpt@hyphenbreak@\HOLOGO@name\endcsname
  }%
  \ifcase\HOLOGO@temp
    \ltx@mbox{-}%
  \else
    -%
  \fi
}
%    \end{macrocode}
%    \end{macro}
%
%    \begin{macro}{\HOLOGO@discretionary}
%    \begin{macrocode}
\def\HOLOGO@discretionary{%
  \ltx@ifundefined{HoLogoOpt@discretionarybreak@\HOLOGO@name}{%
    \ltx@ifundefined{HoLogoOpt@break@\HOLOGO@name}{%
      \chardef\HOLOGO@temp=\HOLOGOOPT@discretionarybreak
    }{%
      \chardef\HOLOGO@temp=%
        \csname HoLogoOpt@break@\HOLOGO@name\endcsname
    }%
  }{%
    \chardef\HOLOGO@temp=%
      \csname HoLogoOpt@discretionarybreak@\HOLOGO@name\endcsname
  }%
  \ifcase\HOLOGO@temp
  \else
    \-%
  \fi
}
%    \end{macrocode}
%    \end{macro}
%
%    \begin{macro}{\HOLOGO@mbox}
%    \begin{macrocode}
\def\HOLOGO@mbox#1{%
  \ltx@ifundefined{HoLogoOpt@break@\HOLOGO@name}{%
    \chardef\HOLOGO@temp=\HOLOGOOPT@hyphenbreak
  }{%
    \chardef\HOLOGO@temp=%
      \csname HoLogoOpt@break@\HOLOGO@name\endcsname
  }%
  \ifcase\HOLOGO@temp
    \ltx@mbox{#1}%
  \else
    #1%
  \fi
}
%    \end{macrocode}
%    \end{macro}
%
% \subsection{Font support}
%
%    \begin{macro}{\HoLogoFont@font}
%    \begin{tabular}{@{}ll@{}}
%    |#1|:& logo name\\
%    |#2|:& font short name\\
%    |#3|:& text
%    \end{tabular}
%    \begin{macrocode}
\def\HoLogoFont@font#1#2#3{%
  \begingroup
    \ltx@IfUndefined{HoLogoFont@logo@#1.#2}{%
      \ltx@IfUndefined{HoLogoFont@font@#2}{%
        \@PackageWarning{hologo}{%
          Missing font `#2' for logo `#1'%
        }%
        #3%
      }{%
        \csname HoLogoFont@font@#2\endcsname{#3}%
      }%
    }{%
      \csname HoLogoFont@logo@#1.#2\endcsname{#3}%
    }%
  \endgroup
}
%    \end{macrocode}
%    \end{macro}
%
%    \begin{macro}{\HoLogoFont@Def}
%    \begin{macrocode}
\def\HoLogoFont@Def#1{%
  \expandafter\def\csname HoLogoFont@font@#1\endcsname
}
%    \end{macrocode}
%    \end{macro}
%    \begin{macro}{\HoLogoFont@LogoDef}
%    \begin{macrocode}
\def\HoLogoFont@LogoDef#1#2{%
  \expandafter\def\csname HoLogoFont@logo@#1.#2\endcsname
}
%    \end{macrocode}
%    \end{macro}
%
% \subsubsection{Font defaults}
%
%    \begin{macro}{\HoLogoFont@font@general}
%    \begin{macrocode}
\HoLogoFont@Def{general}{}%
%    \end{macrocode}
%    \end{macro}
%
%    \begin{macro}{\HoLogoFont@font@rm}
%    \begin{macrocode}
\ltx@IfUndefined{rmfamily}{%
  \ltx@IfUndefined{rm}{%
  }{%
    \HoLogoFont@Def{rm}{\rm}%
  }%
}{%
  \HoLogoFont@Def{rm}{\rmfamily}%
}
%    \end{macrocode}
%    \end{macro}
%
%    \begin{macro}{\HoLogoFont@font@sf}
%    \begin{macrocode}
\ltx@IfUndefined{sffamily}{%
  \ltx@IfUndefined{sf}{%
  }{%
    \HoLogoFont@Def{sf}{\sf}%
  }%
}{%
  \HoLogoFont@Def{sf}{\sffamily}%
}
%    \end{macrocode}
%    \end{macro}
%
%    \begin{macro}{\HoLogoFont@font@bibsf}
%    In case of \hologo{plainTeX} the original small caps
%    variant is used as default. In \hologo{LaTeX}
%    the definition of package \xpackage{dtklogos} \cite{dtklogos}
%    is used.
%\begin{quote}
%\begin{verbatim}
%\DeclareRobustCommand{\BibTeX}{%
%  B%
%  \kern-.05em%
%  \hbox{%
%    $\m@th$% %% force math size calculations
%    \csname S@\f@size\endcsname
%    \fontsize\sf@size\z@
%    \math@fontsfalse
%    \selectfont
%    I%
%    \kern-.025em%
%    B
%  }%
%  \kern-.08em%
%  \-%
%  \TeX
%}
%\end{verbatim}
%\end{quote}
%    \begin{macrocode}
\ltx@IfUndefined{selectfont}{%
  \ltx@IfUndefined{tensc}{%
    \font\tensc=cmcsc10\relax
  }{}%
  \HoLogoFont@Def{bibsf}{\tensc}%
}{%
  \HoLogoFont@Def{bibsf}{%
    $\mathsurround=0pt$%
    \csname S@\f@size\endcsname
    \fontsize\sf@size{0pt}%
    \math@fontsfalse
    \selectfont
  }%
}
%    \end{macrocode}
%    \end{macro}
%
%    \begin{macro}{\HoLogoFont@font@sc}
%    \begin{macrocode}
\ltx@IfUndefined{scshape}{%
  \ltx@IfUndefined{tensc}{%
    \font\tensc=cmcsc10\relax
  }{}%
  \HoLogoFont@Def{sc}{\tensc}%
}{%
  \HoLogoFont@Def{sc}{\scshape}%
}
%    \end{macrocode}
%    \end{macro}
%
%    \begin{macro}{\HoLogoFont@font@sy}
%    \begin{macrocode}
\ltx@IfUndefined{usefont}{%
  \ltx@IfUndefined{tensy}{%
  }{%
    \HoLogoFont@Def{sy}{\tensy}%
  }%
}{%
  \HoLogoFont@Def{sy}{%
    \usefont{OMS}{cmsy}{m}{n}%
  }%
}
%    \end{macrocode}
%    \end{macro}
%
%    \begin{macro}{\HoLogoFont@font@logo}
%    \begin{macrocode}
\begingroup
  \def\x{LaTeX2e}%
\expandafter\endgroup
\ifx\fmtname\x
  \ltx@IfUndefined{logofamily}{%
    \DeclareRobustCommand\logofamily{%
      \not@math@alphabet\logofamily\relax
      \fontencoding{U}%
      \fontfamily{logo}%
      \selectfont
    }%
  }{}%
  \ltx@IfUndefined{logofamily}{%
  }{%
    \HoLogoFont@Def{logo}{\logofamily}%
  }%
\else
  \ltx@IfUndefined{tenlogo}{%
    \font\tenlogo=logo10\relax
  }{}%
  \HoLogoFont@Def{logo}{\tenlogo}%
\fi
%    \end{macrocode}
%    \end{macro}
%
% \subsubsection{Font setup}
%
%    \begin{macro}{\hologoFontSetup}
%    \begin{macrocode}
\def\hologoFontSetup{%
  \let\HOLOGO@name\relax
  \HOLOGO@FontSetup
}
%    \end{macrocode}
%    \end{macro}
%
%    \begin{macro}{\hologoLogoFontSetup}
%    \begin{macrocode}
\def\hologoLogoFontSetup#1{%
  \edef\HOLOGO@name{#1}%
  \ltx@IfUndefined{HoLogo@\HOLOGO@name}{%
    \@PackageError{hologo}{%
      Unknown logo `\HOLOGO@name'%
    }\@ehc
    \ltx@gobble
  }{%
    \HOLOGO@FontSetup
  }%
}
%    \end{macrocode}
%    \end{macro}
%
%    \begin{macro}{\HOLOGO@FontSetup}
%    \begin{macrocode}
\def\HOLOGO@FontSetup{%
  \kvsetkeys{HoLogoFont}%
}
%    \end{macrocode}
%    \end{macro}
%
%    \begin{macrocode}
\def\HOLOGO@temp#1{%
  \kv@define@key{HoLogoFont}{#1}{%
    \ifx\HOLOGO@name\relax
      \HoLogoFont@Def{#1}{##1}%
    \else
      \HoLogoFont@LogoDef\HOLOGO@name{#1}{##1}%
    \fi
  }%
}
\HOLOGO@temp{general}
\HOLOGO@temp{sf}
%    \end{macrocode}
%
% \subsection{Generic logo commands}
%
%    \begin{macrocode}
\HOLOGO@IfExists\hologo{%
  \@PackageError{hologo}{%
    \string\hologo\ltx@space is already defined.\MessageBreak
    Package loading is aborted%
  }\@ehc
  \HOLOGO@AtEnd
}%
\HOLOGO@IfExists\hologoRobust{%
  \@PackageError{hologo}{%
    \string\hologoRobust\ltx@space is already defined.\MessageBreak
    Package loading is aborted%
  }\@ehc
  \HOLOGO@AtEnd
}%
%    \end{macrocode}
%
% \subsubsection{\cs{hologo} and friends}
%
%    \begin{macrocode}
\ifluatex
  \expandafter\ltx@firstofone
\else
  \expandafter\ltx@gobble
\fi
{%
  \ltx@IfUndefined{ifincsname}{%
    \ifnum\luatexversion<36 %
      \expandafter\ltx@gobble
    \else
      \expandafter\ltx@firstofone
    \fi
    {%
      \begingroup
        \ifcase0%
            \directlua{%
              if tex.enableprimitives then %
                tex.enableprimitives('HOLOGO@', {'ifincsname'})%
              else %
                tex.print('1')%
              end%
            }%
            \ifx\HOLOGO@ifincsname\@undefined 1\fi%
            \relax
          \expandafter\ltx@firstofone
        \else
          \endgroup
          \expandafter\ltx@gobble
        \fi
        {%
          \global\let\ifincsname\HOLOGO@ifincsname
        }%
      \HOLOGO@temp
    }%
  }{}%
}
%    \end{macrocode}
%    \begin{macrocode}
\ltx@IfUndefined{ifincsname}{%
  \catcode`$=14 %
}{%
  \catcode`$=9 %
}
%    \end{macrocode}
%
%    \begin{macro}{\hologo}
%    \begin{macrocode}
\def\hologo#1{%
$ \ifincsname
$   \ltx@ifundefined{HoLogoCs@\HOLOGO@Variant{#1}}{%
$     #1%
$   }{%
$     \csname HoLogoCs@\HOLOGO@Variant{#1}\endcsname\ltx@firstoftwo
$   }%
$ \else
    \HOLOGO@IfExists\texorpdfstring\texorpdfstring\ltx@firstoftwo
    {%
      \hologoRobust{#1}%
    }{%
      \ltx@ifundefined{HoLogoBkm@\HOLOGO@Variant{#1}}{%
        \ltx@ifundefined{HoLogo@#1}{?#1?}{#1}%
      }{%
        \csname HoLogoBkm@\HOLOGO@Variant{#1}\endcsname
        \ltx@firstoftwo
      }%
    }%
$ \fi
}
%    \end{macrocode}
%    \end{macro}
%    \begin{macro}{\Hologo}
%    \begin{macrocode}
\def\Hologo#1{%
$ \ifincsname
$   \ltx@ifundefined{HoLogoCs@\HOLOGO@Variant{#1}}{%
$     #1%
$   }{%
$     \csname HoLogoCs@\HOLOGO@Variant{#1}\endcsname\ltx@secondoftwo
$   }%
$ \else
    \HOLOGO@IfExists\texorpdfstring\texorpdfstring\ltx@firstoftwo
    {%
      \HologoRobust{#1}%
    }{%
      \ltx@ifundefined{HoLogoBkm@\HOLOGO@Variant{#1}}{%
        \ltx@ifundefined{HoLogo@#1}{?#1?}{#1}%
      }{%
        \csname HoLogoBkm@\HOLOGO@Variant{#1}\endcsname
        \ltx@secondoftwo
      }%
    }%
$ \fi
}
%    \end{macrocode}
%    \end{macro}
%
%    \begin{macro}{\hologoVariant}
%    \begin{macrocode}
\def\hologoVariant#1#2{%
  \ifx\relax#2\relax
    \hologo{#1}%
  \else
$   \ifincsname
$     \ltx@ifundefined{HoLogoCs@#1@#2}{%
$       #1%
$     }{%
$       \csname HoLogoCs@#1@#2\endcsname\ltx@firstoftwo
$     }%
$   \else
      \HOLOGO@IfExists\texorpdfstring\texorpdfstring\ltx@firstoftwo
      {%
        \hologoVariantRobust{#1}{#2}%
      }{%
        \ltx@ifundefined{HoLogoBkm@#1@#2}{%
          \ltx@ifundefined{HoLogo@#1}{?#1?}{#1}%
        }{%
          \csname HoLogoBkm@#1@#2\endcsname
          \ltx@firstoftwo
        }%
      }%
$   \fi
  \fi
}
%    \end{macrocode}
%    \end{macro}
%    \begin{macro}{\HologoVariant}
%    \begin{macrocode}
\def\HologoVariant#1#2{%
  \ifx\relax#2\relax
    \Hologo{#1}%
  \else
$   \ifincsname
$     \ltx@ifundefined{HoLogoCs@#1@#2}{%
$       #1%
$     }{%
$       \csname HoLogoCs@#1@#2\endcsname\ltx@secondoftwo
$     }%
$   \else
      \HOLOGO@IfExists\texorpdfstring\texorpdfstring\ltx@firstoftwo
      {%
        \HologoVariantRobust{#1}{#2}%
      }{%
        \ltx@ifundefined{HoLogoBkm@#1@#2}{%
          \ltx@ifundefined{HoLogo@#1}{?#1?}{#1}%
        }{%
          \csname HoLogoBkm@#1@#2\endcsname
          \ltx@secondoftwo
        }%
      }%
$   \fi
  \fi
}
%    \end{macrocode}
%    \end{macro}
%
%    \begin{macrocode}
\catcode`\$=3 %
%    \end{macrocode}
%
% \subsubsection{\cs{hologoRobust} and friends}
%
%    \begin{macro}{\hologoRobust}
%    \begin{macrocode}
\ltx@IfUndefined{protected}{%
  \ltx@IfUndefined{DeclareRobustCommand}{%
    \def\hologoRobust#1%
  }{%
    \DeclareRobustCommand*\hologoRobust[1]%
  }%
}{%
  \protected\def\hologoRobust#1%
}%
{%
  \edef\HOLOGO@name{#1}%
  \ltx@IfUndefined{HoLogo@\HOLOGO@Variant\HOLOGO@name}{%
    \@PackageError{hologo}{%
      Unknown logo `\HOLOGO@name'%
    }\@ehc
    ?\HOLOGO@name?%
  }{%
    \ltx@IfUndefined{ver@tex4ht.sty}{%
      \HoLogoFont@font\HOLOGO@name{general}{%
        \csname HoLogo@\HOLOGO@Variant\HOLOGO@name\endcsname
        \ltx@firstoftwo
      }%
    }{%
      \ltx@IfUndefined{HoLogoHtml@\HOLOGO@Variant\HOLOGO@name}{%
        \HOLOGO@name
      }{%
        \csname HoLogoHtml@\HOLOGO@Variant\HOLOGO@name\endcsname
        \ltx@firstoftwo
      }%
    }%
  }%
}
%    \end{macrocode}
%    \end{macro}
%    \begin{macro}{\HologoRobust}
%    \begin{macrocode}
\ltx@IfUndefined{protected}{%
  \ltx@IfUndefined{DeclareRobustCommand}{%
    \def\HologoRobust#1%
  }{%
    \DeclareRobustCommand*\HologoRobust[1]%
  }%
}{%
  \protected\def\HologoRobust#1%
}%
{%
  \edef\HOLOGO@name{#1}%
  \ltx@IfUndefined{HoLogo@\HOLOGO@Variant\HOLOGO@name}{%
    \@PackageError{hologo}{%
      Unknown logo `\HOLOGO@name'%
    }\@ehc
    ?\HOLOGO@name?%
  }{%
    \ltx@IfUndefined{ver@tex4ht.sty}{%
      \HoLogoFont@font\HOLOGO@name{general}{%
        \csname HoLogo@\HOLOGO@Variant\HOLOGO@name\endcsname
        \ltx@secondoftwo
      }%
    }{%
      \ltx@IfUndefined{HoLogoHtml@\HOLOGO@Variant\HOLOGO@name}{%
        \expandafter\HOLOGO@Uppercase\HOLOGO@name
      }{%
        \csname HoLogoHtml@\HOLOGO@Variant\HOLOGO@name\endcsname
        \ltx@secondoftwo
      }%
    }%
  }%
}
%    \end{macrocode}
%    \end{macro}
%    \begin{macro}{\hologoVariantRobust}
%    \begin{macrocode}
\ltx@IfUndefined{protected}{%
  \ltx@IfUndefined{DeclareRobustCommand}{%
    \def\hologoVariantRobust#1#2%
  }{%
    \DeclareRobustCommand*\hologoVariantRobust[2]%
  }%
}{%
  \protected\def\hologoVariantRobust#1#2%
}%
{%
  \begingroup
    \hologoLogoSetup{#1}{variant={#2}}%
    \hologoRobust{#1}%
  \endgroup
}
%    \end{macrocode}
%    \end{macro}
%    \begin{macro}{\HologoVariantRobust}
%    \begin{macrocode}
\ltx@IfUndefined{protected}{%
  \ltx@IfUndefined{DeclareRobustCommand}{%
    \def\HologoVariantRobust#1#2%
  }{%
    \DeclareRobustCommand*\HologoVariantRobust[2]%
  }%
}{%
  \protected\def\HologoVariantRobust#1#2%
}%
{%
  \begingroup
    \hologoLogoSetup{#1}{variant={#2}}%
    \HologoRobust{#1}%
  \endgroup
}
%    \end{macrocode}
%    \end{macro}
%
%    \begin{macro}{\hologorobust}
%    Macro \cs{hologorobust} is only defined for compatibility.
%    Its use is deprecated.
%    \begin{macrocode}
\def\hologorobust{\hologoRobust}
%    \end{macrocode}
%    \end{macro}
%
% \subsection{Helpers}
%
%    \begin{macro}{\HOLOGO@Uppercase}
%    Macro \cs{HOLOGO@Uppercase} is restricted to \cs{uppercase},
%    because \hologo{plainTeX} or \hologo{iniTeX} do not provide
%    \cs{MakeUppercase}.
%    \begin{macrocode}
\def\HOLOGO@Uppercase#1{\uppercase{#1}}
%    \end{macrocode}
%    \end{macro}
%
%    \begin{macro}{\HOLOGO@PdfdocUnicode}
%    \begin{macrocode}
\def\HOLOGO@PdfdocUnicode{%
  \ifx\ifHy@unicode\iftrue
    \expandafter\ltx@secondoftwo
  \else
    \expandafter\ltx@firstoftwo
  \fi
}
%    \end{macrocode}
%    \end{macro}
%
%    \begin{macro}{\HOLOGO@Math}
%    \begin{macrocode}
\def\HOLOGO@MathSetup{%
  \mathsurround0pt\relax
  \HOLOGO@IfExists\f@series{%
    \if b\expandafter\ltx@car\f@series x\@nil
      \csname boldmath\endcsname
   \fi
  }{}%
}
%    \end{macrocode}
%    \end{macro}
%
%    \begin{macro}{\HOLOGO@TempDimen}
%    \begin{macrocode}
\dimendef\HOLOGO@TempDimen=\ltx@zero
%    \end{macrocode}
%    \end{macro}
%    \begin{macro}{\HOLOGO@NegativeKerning}
%    \begin{macrocode}
\def\HOLOGO@NegativeKerning#1{%
  \begingroup
    \HOLOGO@TempDimen=0pt\relax
    \comma@parse@normalized{#1}{%
      \ifdim\HOLOGO@TempDimen=0pt %
        \expandafter\HOLOGO@@NegativeKerning\comma@entry
      \fi
      \ltx@gobble
    }%
    \ifdim\HOLOGO@TempDimen<0pt %
      \kern\HOLOGO@TempDimen
    \fi
  \endgroup
}
%    \end{macrocode}
%    \end{macro}
%    \begin{macro}{\HOLOGO@@NegativeKerning}
%    \begin{macrocode}
\def\HOLOGO@@NegativeKerning#1#2{%
  \setbox\ltx@zero\hbox{#1#2}%
  \HOLOGO@TempDimen=\wd\ltx@zero
  \setbox\ltx@zero\hbox{#1\kern0pt#2}%
  \advance\HOLOGO@TempDimen by -\wd\ltx@zero
}
%    \end{macrocode}
%    \end{macro}
%
%    \begin{macro}{\HOLOGO@SpaceFactor}
%    \begin{macrocode}
\def\HOLOGO@SpaceFactor{%
  \spacefactor1000 %
}
%    \end{macrocode}
%    \end{macro}
%
%    \begin{macro}{\HOLOGO@Span}
%    \begin{macrocode}
\def\HOLOGO@Span#1#2{%
  \HCode{<span class="HoLogo-#1">}%
  #2%
  \HCode{</span>}%
}
%    \end{macrocode}
%    \end{macro}
%
% \subsubsection{Text subscript}
%
%    \begin{macro}{\HOLOGO@SubScript}%
%    \begin{macrocode}
\def\HOLOGO@SubScript#1{%
  \ltx@IfUndefined{textsubscript}{%
    \ltx@IfUndefined{text}{%
      \ltx@mbox{%
        \mathsurround=0pt\relax
        $%
          _{%
            \ltx@IfUndefined{sf@size}{%
              \mathrm{#1}%
            }{%
              \mbox{%
                \fontsize\sf@size{0pt}\selectfont
                #1%
              }%
            }%
          }%
        $%
      }%
    }{%
      \ltx@mbox{%
        \mathsurround=0pt\relax
        $_{\text{#1}}$%
      }%
    }%
  }{%
    \textsubscript{#1}%
  }%
}
%    \end{macrocode}
%    \end{macro}
%
% \subsection{\hologo{TeX} and friends}
%
% \subsubsection{\hologo{TeX}}
%
%    \begin{macro}{\HoLogo@TeX}
%    Source: \hologo{LaTeX} kernel.
%    \begin{macrocode}
\def\HoLogo@TeX#1{%
  T\kern-.1667em\lower.5ex\hbox{E}\kern-.125emX\HOLOGO@SpaceFactor
}
%    \end{macrocode}
%    \end{macro}
%    \begin{macro}{\HoLogoHtml@TeX}
%    \begin{macrocode}
\def\HoLogoHtml@TeX#1{%
  \HoLogoCss@TeX
  \HOLOGO@Span{TeX}{%
    T%
    \HOLOGO@Span{e}{%
      E%
    }%
    X%
  }%
}
%    \end{macrocode}
%    \end{macro}
%    \begin{macro}{\HoLogoCss@TeX}
%    \begin{macrocode}
\def\HoLogoCss@TeX{%
  \Css{%
    span.HoLogo-TeX span.HoLogo-e{%
      position:relative;%
      top:.5ex;%
      margin-left:-.1667em;%
      margin-right:-.125em;%
    }%
  }%
  \Css{%
    a span.HoLogo-TeX span.HoLogo-e{%
      text-decoration:none;%
    }%
  }%
  \global\let\HoLogoCss@TeX\relax
}
%    \end{macrocode}
%    \end{macro}
%
% \subsubsection{\hologo{plainTeX}}
%
%    \begin{macro}{\HoLogo@plainTeX@space}
%    Source: ``The \hologo{TeX}book''
%    \begin{macrocode}
\def\HoLogo@plainTeX@space#1{%
  \HOLOGO@mbox{#1{p}{P}lain}\HOLOGO@space\hologo{TeX}%
}
%    \end{macrocode}
%    \end{macro}
%    \begin{macro}{\HoLogoCs@plainTeX@space}
%    \begin{macrocode}
\def\HoLogoCs@plainTeX@space#1{#1{p}{P}lain TeX}%
%    \end{macrocode}
%    \end{macro}
%    \begin{macro}{\HoLogoBkm@plainTeX@space}
%    \begin{macrocode}
\def\HoLogoBkm@plainTeX@space#1{%
  #1{p}{P}lain \hologo{TeX}%
}
%    \end{macrocode}
%    \end{macro}
%    \begin{macro}{\HoLogoHtml@plainTeX@space}
%    \begin{macrocode}
\def\HoLogoHtml@plainTeX@space#1{%
  #1{p}{P}lain \hologo{TeX}%
}
%    \end{macrocode}
%    \end{macro}
%
%    \begin{macro}{\HoLogo@plainTeX@hyphen}
%    \begin{macrocode}
\def\HoLogo@plainTeX@hyphen#1{%
  \HOLOGO@mbox{#1{p}{P}lain}\HOLOGO@hyphen\hologo{TeX}%
}
%    \end{macrocode}
%    \end{macro}
%    \begin{macro}{\HoLogoCs@plainTeX@hyphen}
%    \begin{macrocode}
\def\HoLogoCs@plainTeX@hyphen#1{#1{p}{P}lain-TeX}
%    \end{macrocode}
%    \end{macro}
%    \begin{macro}{\HoLogoBkm@plainTeX@hyphen}
%    \begin{macrocode}
\def\HoLogoBkm@plainTeX@hyphen#1{%
  #1{p}{P}lain-\hologo{TeX}%
}
%    \end{macrocode}
%    \end{macro}
%    \begin{macro}{\HoLogoHtml@plainTeX@hyphen}
%    \begin{macrocode}
\def\HoLogoHtml@plainTeX@hyphen#1{%
  #1{p}{P}lain-\hologo{TeX}%
}
%    \end{macrocode}
%    \end{macro}
%
%    \begin{macro}{\HoLogo@plainTeX@runtogether}
%    \begin{macrocode}
\def\HoLogo@plainTeX@runtogether#1{%
  \HOLOGO@mbox{#1{p}{P}lain\hologo{TeX}}%
}
%    \end{macrocode}
%    \end{macro}
%    \begin{macro}{\HoLogoCs@plainTeX@runtogether}
%    \begin{macrocode}
\def\HoLogoCs@plainTeX@runtogether#1{#1{p}{P}lainTeX}
%    \end{macrocode}
%    \end{macro}
%    \begin{macro}{\HoLogoBkm@plainTeX@runtogether}
%    \begin{macrocode}
\def\HoLogoBkm@plainTeX@runtogether#1{%
  #1{p}{P}lain\hologo{TeX}%
}
%    \end{macrocode}
%    \end{macro}
%    \begin{macro}{\HoLogoHtml@plainTeX@runtogether}
%    \begin{macrocode}
\def\HoLogoHtml@plainTeX@runtogether#1{%
  #1{p}{P}lain\hologo{TeX}%
}
%    \end{macrocode}
%    \end{macro}
%
%    \begin{macro}{\HoLogo@plainTeX}
%    \begin{macrocode}
\def\HoLogo@plainTeX{\HoLogo@plainTeX@space}
%    \end{macrocode}
%    \end{macro}
%    \begin{macro}{\HoLogoCs@plainTeX}
%    \begin{macrocode}
\def\HoLogoCs@plainTeX{\HoLogoCs@plainTeX@space}
%    \end{macrocode}
%    \end{macro}
%    \begin{macro}{\HoLogoBkm@plainTeX}
%    \begin{macrocode}
\def\HoLogoBkm@plainTeX{\HoLogoBkm@plainTeX@space}
%    \end{macrocode}
%    \end{macro}
%    \begin{macro}{\HoLogoHtml@plainTeX}
%    \begin{macrocode}
\def\HoLogoHtml@plainTeX{\HoLogoHtml@plainTeX@space}
%    \end{macrocode}
%    \end{macro}
%
% \subsubsection{\hologo{LaTeX}}
%
%    Source: \hologo{LaTeX} kernel.
%\begin{quote}
%\begin{verbatim}
%\DeclareRobustCommand{\LaTeX}{%
%  L%
%  \kern-.36em%
%  {%
%    \sbox\z@ T%
%    \vbox to\ht\z@{%
%      \hbox{%
%        \check@mathfonts
%        \fontsize\sf@size\z@
%        \math@fontsfalse
%        \selectfont
%        A%
%      }%
%      \vss
%    }%
%  }%
%  \kern-.15em%
%  \TeX
%}
%\end{verbatim}
%\end{quote}
%
%    \begin{macro}{\HoLogo@La}
%    \begin{macrocode}
\def\HoLogo@La#1{%
  L%
  \kern-.36em%
  \begingroup
    \setbox\ltx@zero\hbox{T}%
    \vbox to\ht\ltx@zero{%
      \hbox{%
        \ltx@ifundefined{check@mathfonts}{%
          \csname sevenrm\endcsname
        }{%
          \check@mathfonts
          \fontsize\sf@size{0pt}%
          \math@fontsfalse\selectfont
        }%
        A%
      }%
      \vss
    }%
  \endgroup
}
%    \end{macrocode}
%    \end{macro}
%
%    \begin{macro}{\HoLogo@LaTeX}
%    Source: \hologo{LaTeX} kernel.
%    \begin{macrocode}
\def\HoLogo@LaTeX#1{%
  \hologo{La}%
  \kern-.15em%
  \hologo{TeX}%
}
%    \end{macrocode}
%    \end{macro}
%    \begin{macro}{\HoLogoHtml@LaTeX}
%    \begin{macrocode}
\def\HoLogoHtml@LaTeX#1{%
  \HoLogoCss@LaTeX
  \HOLOGO@Span{LaTeX}{%
    L%
    \HOLOGO@Span{a}{%
      A%
    }%
    \hologo{TeX}%
  }%
}
%    \end{macrocode}
%    \end{macro}
%    \begin{macro}{\HoLogoCss@LaTeX}
%    \begin{macrocode}
\def\HoLogoCss@LaTeX{%
  \Css{%
    span.HoLogo-LaTeX span.HoLogo-a{%
      position:relative;%
      top:-.5ex;%
      margin-left:-.36em;%
      margin-right:-.15em;%
      font-size:85\%;%
    }%
  }%
  \global\let\HoLogoCss@LaTeX\relax
}
%    \end{macrocode}
%    \end{macro}
%
% \subsubsection{\hologo{(La)TeX}}
%
%    \begin{macro}{\HoLogo@LaTeXTeX}
%    The kerning around the parentheses is taken
%    from package \xpackage{dtklogos} \cite{dtklogos}.
%\begin{quote}
%\begin{verbatim}
%\DeclareRobustCommand{\LaTeXTeX}{%
%  (%
%  \kern-.15em%
%  L%
%  \kern-.36em%
%  {%
%    \sbox\z@ T%
%    \vbox to\ht0{%
%      \hbox{%
%        $\m@th$%
%        \csname S@\f@size\endcsname
%        \fontsize\sf@size\z@
%        \math@fontsfalse
%        \selectfont
%        A%
%      }%
%      \vss
%    }%
%  }%
%  \kern-.2em%
%  )%
%  \kern-.15em%
%  \TeX
%}
%\end{verbatim}
%\end{quote}
%    \begin{macrocode}
\def\HoLogo@LaTeXTeX#1{%
  (%
  \kern-.15em%
  \hologo{La}%
  \kern-.2em%
  )%
  \kern-.15em%
  \hologo{TeX}%
}
%    \end{macrocode}
%    \end{macro}
%    \begin{macro}{\HoLogoBkm@LaTeXTeX}
%    \begin{macrocode}
\def\HoLogoBkm@LaTeXTeX#1{(La)TeX}
%    \end{macrocode}
%    \end{macro}
%
%    \begin{macro}{\HoLogo@(La)TeX}
%    \begin{macrocode}
\expandafter
\let\csname HoLogo@(La)TeX\endcsname\HoLogo@LaTeXTeX
%    \end{macrocode}
%    \end{macro}
%    \begin{macro}{\HoLogoBkm@(La)TeX}
%    \begin{macrocode}
\expandafter
\let\csname HoLogoBkm@(La)TeX\endcsname\HoLogoBkm@LaTeXTeX
%    \end{macrocode}
%    \end{macro}
%    \begin{macro}{\HoLogoHtml@LaTeXTeX}
%    \begin{macrocode}
\def\HoLogoHtml@LaTeXTeX#1{%
  \HoLogoCss@LaTeXTeX
  \HOLOGO@Span{LaTeXTeX}{%
    (%
    \HOLOGO@Span{L}{L}%
    \HOLOGO@Span{a}{A}%
    \HOLOGO@Span{ParenRight}{)}%
    \hologo{TeX}%
  }%
}
%    \end{macrocode}
%    \end{macro}
%    \begin{macro}{\HoLogoHtml@(La)TeX}
%    Kerning after opening parentheses and before closing parentheses
%    is $-0.1$\,em. The original values $-0.15$\,em
%    looked too ugly for a serif font.
%    \begin{macrocode}
\expandafter
\let\csname HoLogoHtml@(La)TeX\endcsname\HoLogoHtml@LaTeXTeX
%    \end{macrocode}
%    \end{macro}
%    \begin{macro}{\HoLogoCss@LaTeXTeX}
%    \begin{macrocode}
\def\HoLogoCss@LaTeXTeX{%
  \Css{%
    span.HoLogo-LaTeXTeX span.HoLogo-L{%
      margin-left:-.1em;%
    }%
  }%
  \Css{%
    span.HoLogo-LaTeXTeX span.HoLogo-a{%
      position:relative;%
      top:-.5ex;%
      margin-left:-.36em;%
      margin-right:-.1em;%
      font-size:85\%;%
    }%
  }%
  \Css{%
    span.HoLogo-LaTeXTeX span.HoLogo-ParenRight{%
      margin-right:-.15em;%
    }%
  }%
  \global\let\HoLogoCss@LaTeXTeX\relax
}
%    \end{macrocode}
%    \end{macro}
%
% \subsubsection{\hologo{LaTeXe}}
%
%    \begin{macro}{\HoLogo@LaTeXe}
%    Source: \hologo{LaTeX} kernel
%    \begin{macrocode}
\def\HoLogo@LaTeXe#1{%
  \hologo{LaTeX}%
  \kern.15em%
  \hbox{%
    \HOLOGO@MathSetup
    2%
    $_{\textstyle\varepsilon}$%
  }%
}
%    \end{macrocode}
%    \end{macro}
%
%    \begin{macro}{\HoLogoCs@LaTeXe}
%    \begin{macrocode}
\ifnum64=`\^^^^0040\relax % test for big chars of LuaTeX/XeTeX
  \catcode`\$=9 %
  \catcode`\&=14 %
\else
  \catcode`\$=14 %
  \catcode`\&=9 %
\fi
\def\HoLogoCs@LaTeXe#1{%
  LaTeX2%
$ \string ^^^^0395%
& e%
}%
\catcode`\$=3 %
\catcode`\&=4 %
%    \end{macrocode}
%    \end{macro}
%
%    \begin{macro}{\HoLogoBkm@LaTeXe}
%    \begin{macrocode}
\def\HoLogoBkm@LaTeXe#1{%
  \hologo{LaTeX}%
  2%
  \HOLOGO@PdfdocUnicode{e}{\textepsilon}%
}
%    \end{macrocode}
%    \end{macro}
%
%    \begin{macro}{\HoLogoHtml@LaTeXe}
%    \begin{macrocode}
\def\HoLogoHtml@LaTeXe#1{%
  \HoLogoCss@LaTeXe
  \HOLOGO@Span{LaTeX2e}{%
    \hologo{LaTeX}%
    \HOLOGO@Span{2}{2}%
    \HOLOGO@Span{e}{%
      \HOLOGO@MathSetup
      \ensuremath{\textstyle\varepsilon}%
    }%
  }%
}
%    \end{macrocode}
%    \end{macro}
%    \begin{macro}{\HoLogoCss@LaTeXe}
%    \begin{macrocode}
\def\HoLogoCss@LaTeXe{%
  \Css{%
    span.HoLogo-LaTeX2e span.HoLogo-2{%
      padding-left:.15em;%
    }%
  }%
  \Css{%
    span.HoLogo-LaTeX2e span.HoLogo-e{%
      position:relative;%
      top:.35ex;%
      text-decoration:none;%
    }%
  }%
  \global\let\HoLogoCss@LaTeXe\relax
}
%    \end{macrocode}
%    \end{macro}
%
%    \begin{macro}{\HoLogo@LaTeX2e}
%    \begin{macrocode}
\expandafter
\let\csname HoLogo@LaTeX2e\endcsname\HoLogo@LaTeXe
%    \end{macrocode}
%    \end{macro}
%    \begin{macro}{\HoLogoCs@LaTeX2e}
%    \begin{macrocode}
\expandafter
\let\csname HoLogoCs@LaTeX2e\endcsname\HoLogoCs@LaTeXe
%    \end{macrocode}
%    \end{macro}
%    \begin{macro}{\HoLogoBkm@LaTeX2e}
%    \begin{macrocode}
\expandafter
\let\csname HoLogoBkm@LaTeX2e\endcsname\HoLogoBkm@LaTeXe
%    \end{macrocode}
%    \end{macro}
%    \begin{macro}{\HoLogoHtml@LaTeX2e}
%    \begin{macrocode}
\expandafter
\let\csname HoLogoHtml@LaTeX2e\endcsname\HoLogoHtml@LaTeXe
%    \end{macrocode}
%    \end{macro}
%
% \subsubsection{\hologo{LaTeX3}}
%
%    \begin{macro}{\HoLogo@LaTeX3}
%    Source: \hologo{LaTeX} kernel
%    \begin{macrocode}
\expandafter\def\csname HoLogo@LaTeX3\endcsname#1{%
  \hologo{LaTeX}%
  3%
}
%    \end{macrocode}
%    \end{macro}
%
%    \begin{macro}{\HoLogoBkm@LaTeX3}
%    \begin{macrocode}
\expandafter\def\csname HoLogoBkm@LaTeX3\endcsname#1{%
  \hologo{LaTeX}%
  3%
}
%    \end{macrocode}
%    \end{macro}
%    \begin{macro}{\HoLogoHtml@LaTeX3}
%    \begin{macrocode}
\expandafter
\let\csname HoLogoHtml@LaTeX3\expandafter\endcsname
\csname HoLogo@LaTeX3\endcsname
%    \end{macrocode}
%    \end{macro}
%
% \subsubsection{\hologo{LaTeXML}}
%
%    \begin{macro}{\HoLogo@LaTeXML}
%    \begin{macrocode}
\def\HoLogo@LaTeXML#1{%
  \HOLOGO@mbox{%
    \hologo{La}%
    \kern-.15em%
    T%
    \kern-.1667em%
    \lower.5ex\hbox{E}%
    \kern-.125em%
    \HoLogoFont@font{LaTeXML}{sc}{xml}%
  }%
}
%    \end{macrocode}
%    \end{macro}
%    \begin{macro}{\HoLogoHtml@pdfLaTeX}
%    \begin{macrocode}
\def\HoLogoHtml@LaTeXML#1{%
  \HOLOGO@Span{LaTeXML}{%
    \HoLogoCss@LaTeX
    \HoLogoCss@TeX
    \HOLOGO@Span{LaTeX}{%
      L%
      \HOLOGO@Span{a}{%
        A%
      }%
    }%
    \HOLOGO@Span{TeX}{%
      T%
      \HOLOGO@Span{e}{%
        E%
      }%
    }%
    \HCode{<span style="font-variant: small-caps;">}%
    xml%
    \HCode{</span>}%
  }%
}
%    \end{macrocode}
%    \end{macro}
%
% \subsubsection{\hologo{eTeX}}
%
%    \begin{macro}{\HoLogo@eTeX}
%    Source: package \xpackage{etex}
%    \begin{macrocode}
\def\HoLogo@eTeX#1{%
  \ltx@mbox{%
    \HOLOGO@MathSetup
    $\varepsilon$%
    -%
    \HOLOGO@NegativeKerning{-T,T-,To}%
    \hologo{TeX}%
  }%
}
%    \end{macrocode}
%    \end{macro}
%    \begin{macro}{\HoLogoCs@eTeX}
%    \begin{macrocode}
\ifnum64=`\^^^^0040\relax % test for big chars of LuaTeX/XeTeX
  \catcode`\$=9 %
  \catcode`\&=14 %
\else
  \catcode`\$=14 %
  \catcode`\&=9 %
\fi
\def\HoLogoCs@eTeX#1{%
$ #1{\string ^^^^0395}{\string ^^^^03b5}%
& #1{e}{E}%
  TeX%
}%
\catcode`\$=3 %
\catcode`\&=4 %
%    \end{macrocode}
%    \end{macro}
%    \begin{macro}{\HoLogoBkm@eTeX}
%    \begin{macrocode}
\def\HoLogoBkm@eTeX#1{%
  \HOLOGO@PdfdocUnicode{#1{e}{E}}{\textepsilon}%
  -%
  \hologo{TeX}%
}
%    \end{macrocode}
%    \end{macro}
%    \begin{macro}{\HoLogoHtml@eTeX}
%    \begin{macrocode}
\def\HoLogoHtml@eTeX#1{%
  \ltx@mbox{%
    \HOLOGO@MathSetup
    $\varepsilon$%
    -%
    \hologo{TeX}%
  }%
}
%    \end{macrocode}
%    \end{macro}
%
% \subsubsection{\hologo{iniTeX}}
%
%    \begin{macro}{\HoLogo@iniTeX}
%    \begin{macrocode}
\def\HoLogo@iniTeX#1{%
  \HOLOGO@mbox{%
    #1{i}{I}ni\hologo{TeX}%
  }%
}
%    \end{macrocode}
%    \end{macro}
%    \begin{macro}{\HoLogoCs@iniTeX}
%    \begin{macrocode}
\def\HoLogoCs@iniTeX#1{#1{i}{I}niTeX}
%    \end{macrocode}
%    \end{macro}
%    \begin{macro}{\HoLogoBkm@iniTeX}
%    \begin{macrocode}
\def\HoLogoBkm@iniTeX#1{%
  #1{i}{I}ni\hologo{TeX}%
}
%    \end{macrocode}
%    \end{macro}
%    \begin{macro}{\HoLogoHtml@iniTeX}
%    \begin{macrocode}
\let\HoLogoHtml@iniTeX\HoLogo@iniTeX
%    \end{macrocode}
%    \end{macro}
%
% \subsubsection{\hologo{virTeX}}
%
%    \begin{macro}{\HoLogo@virTeX}
%    \begin{macrocode}
\def\HoLogo@virTeX#1{%
  \HOLOGO@mbox{%
    #1{v}{V}ir\hologo{TeX}%
  }%
}
%    \end{macrocode}
%    \end{macro}
%    \begin{macro}{\HoLogoCs@virTeX}
%    \begin{macrocode}
\def\HoLogoCs@virTeX#1{#1{v}{V}irTeX}
%    \end{macrocode}
%    \end{macro}
%    \begin{macro}{\HoLogoBkm@virTeX}
%    \begin{macrocode}
\def\HoLogoBkm@virTeX#1{%
  #1{v}{V}ir\hologo{TeX}%
}
%    \end{macrocode}
%    \end{macro}
%    \begin{macro}{\HoLogoHtml@virTeX}
%    \begin{macrocode}
\let\HoLogoHtml@virTeX\HoLogo@virTeX
%    \end{macrocode}
%    \end{macro}
%
% \subsubsection{\hologo{SliTeX}}
%
% \paragraph{Definitions of the three variants.}
%
%    \begin{macro}{\HoLogo@SLiTeX@lift}
%    \begin{macrocode}
\def\HoLogo@SLiTeX@lift#1{%
  \HoLogoFont@font{SliTeX}{rm}{%
    S%
    \kern-.06em%
    L%
    \kern-.18em%
    \raise.32ex\hbox{\HoLogoFont@font{SliTeX}{sc}{i}}%
    \HOLOGO@discretionary
    \kern-.06em%
    \hologo{TeX}%
  }%
}
%    \end{macrocode}
%    \end{macro}
%    \begin{macro}{\HoLogoBkm@SLiTeX@lift}
%    \begin{macrocode}
\def\HoLogoBkm@SLiTeX@lift#1{SLiTeX}
%    \end{macrocode}
%    \end{macro}
%    \begin{macro}{\HoLogoHtml@SLiTeX@lift}
%    \begin{macrocode}
\def\HoLogoHtml@SLiTeX@lift#1{%
  \HoLogoCss@SLiTeX@lift
  \HOLOGO@Span{SLiTeX-lift}{%
    \HoLogoFont@font{SliTeX}{rm}{%
      S%
      \HOLOGO@Span{L}{L}%
      \HOLOGO@Span{i}{i}%
      \hologo{TeX}%
    }%
  }%
}
%    \end{macrocode}
%    \end{macro}
%    \begin{macro}{\HoLogoCss@SLiTeX@lift}
%    \begin{macrocode}
\def\HoLogoCss@SLiTeX@lift{%
  \Css{%
    span.HoLogo-SLiTeX-lift span.HoLogo-L{%
      margin-left:-.06em;%
      margin-right:-.18em;%
    }%
  }%
  \Css{%
    span.HoLogo-SLiTeX-lift span.HoLogo-i{%
      position:relative;%
      top:-.32ex;%
      margin-right:-.06em;%
      font-variant:small-caps;%
    }%
  }%
  \global\let\HoLogoCss@SLiTeX@lift\relax
}
%    \end{macrocode}
%    \end{macro}
%
%    \begin{macro}{\HoLogo@SliTeX@simple}
%    \begin{macrocode}
\def\HoLogo@SliTeX@simple#1{%
  \HoLogoFont@font{SliTeX}{rm}{%
    \ltx@mbox{%
      \HoLogoFont@font{SliTeX}{sc}{Sli}%
    }%
    \HOLOGO@discretionary
    \hologo{TeX}%
  }%
}
%    \end{macrocode}
%    \end{macro}
%    \begin{macro}{\HoLogoBkm@SliTeX@simple}
%    \begin{macrocode}
\def\HoLogoBkm@SliTeX@simple#1{SliTeX}
%    \end{macrocode}
%    \end{macro}
%    \begin{macro}{\HoLogoHtml@SliTeX@simple}
%    \begin{macrocode}
\let\HoLogoHtml@SliTeX@simple\HoLogo@SliTeX@simple
%    \end{macrocode}
%    \end{macro}
%
%    \begin{macro}{\HoLogo@SliTeX@narrow}
%    \begin{macrocode}
\def\HoLogo@SliTeX@narrow#1{%
  \HoLogoFont@font{SliTeX}{rm}{%
    \ltx@mbox{%
      S%
      \kern-.06em%
      \HoLogoFont@font{SliTeX}{sc}{%
        l%
        \kern-.035em%
        i%
      }%
    }%
    \HOLOGO@discretionary
    \kern-.06em%
    \hologo{TeX}%
  }%
}
%    \end{macrocode}
%    \end{macro}
%    \begin{macro}{\HoLogoBkm@SliTeX@narrow}
%    \begin{macrocode}
\def\HoLogoBkm@SliTeX@narrow#1{SliTeX}
%    \end{macrocode}
%    \end{macro}
%    \begin{macro}{\HoLogoHtml@SliTeX@narrow}
%    \begin{macrocode}
\def\HoLogoHtml@SliTeX@narrow#1{%
  \HoLogoCss@SliTeX@narrow
  \HOLOGO@Span{SliTeX-narrow}{%
    \HoLogoFont@font{SliTeX}{rm}{%
      S%
        \HOLOGO@Span{l}{l}%
        \HOLOGO@Span{i}{i}%
      \hologo{TeX}%
    }%
  }%
}
%    \end{macrocode}
%    \end{macro}
%    \begin{macro}{\HoLogoCss@SliTeX@narrow}
%    \begin{macrocode}
\def\HoLogoCss@SliTeX@narrow{%
  \Css{%
    span.HoLogo-SliTeX-narrow span.HoLogo-l{%
      margin-left:-.06em;%
      margin-right:-.035em;%
      font-variant:small-caps;%
    }%
  }%
  \Css{%
    span.HoLogo-SliTeX-narrow span.HoLogo-i{%
      margin-right:-.06em;%
      font-variant:small-caps;%
    }%
  }%
  \global\let\HoLogoCss@SliTeX@narrow\relax
}
%    \end{macrocode}
%    \end{macro}
%
% \paragraph{Macro set completion.}
%
%    \begin{macro}{\HoLogo@SLiTeX@simple}
%    \begin{macrocode}
\def\HoLogo@SLiTeX@simple{\HoLogo@SliTeX@simple}
%    \end{macrocode}
%    \end{macro}
%    \begin{macro}{\HoLogoBkm@SLiTeX@simple}
%    \begin{macrocode}
\def\HoLogoBkm@SLiTeX@simple{\HoLogoBkm@SliTeX@simple}
%    \end{macrocode}
%    \end{macro}
%    \begin{macro}{\HoLogoHtml@SLiTeX@simple}
%    \begin{macrocode}
\def\HoLogoHtml@SLiTeX@simple{\HoLogoHtml@SliTeX@simple}
%    \end{macrocode}
%    \end{macro}
%
%    \begin{macro}{\HoLogo@SLiTeX@narrow}
%    \begin{macrocode}
\def\HoLogo@SLiTeX@narrow{\HoLogo@SliTeX@narrow}
%    \end{macrocode}
%    \end{macro}
%    \begin{macro}{\HoLogoBkm@SLiTeX@narrow}
%    \begin{macrocode}
\def\HoLogoBkm@SLiTeX@narrow{\HoLogoBkm@SliTeX@narrow}
%    \end{macrocode}
%    \end{macro}
%    \begin{macro}{\HoLogoHtml@SLiTeX@narrow}
%    \begin{macrocode}
\def\HoLogoHtml@SLiTeX@narrow{\HoLogoHtml@SliTeX@narrow}
%    \end{macrocode}
%    \end{macro}
%
%    \begin{macro}{\HoLogo@SliTeX@lift}
%    \begin{macrocode}
\def\HoLogo@SliTeX@lift{\HoLogo@SLiTeX@lift}
%    \end{macrocode}
%    \end{macro}
%    \begin{macro}{\HoLogoBkm@SliTeX@lift}
%    \begin{macrocode}
\def\HoLogoBkm@SliTeX@lift{\HoLogoBkm@SLiTeX@lift}
%    \end{macrocode}
%    \end{macro}
%    \begin{macro}{\HoLogoHtml@SliTeX@lift}
%    \begin{macrocode}
\def\HoLogoHtml@SliTeX@lift{\HoLogoHtml@SLiTeX@lift}
%    \end{macrocode}
%    \end{macro}
%
% \paragraph{Defaults.}
%
%    \begin{macro}{\HoLogo@SLiTeX}
%    \begin{macrocode}
\def\HoLogo@SLiTeX{\HoLogo@SLiTeX@lift}
%    \end{macrocode}
%    \end{macro}
%    \begin{macro}{\HoLogoBkm@SLiTeX}
%    \begin{macrocode}
\def\HoLogoBkm@SLiTeX{\HoLogoBkm@SLiTeX@lift}
%    \end{macrocode}
%    \end{macro}
%    \begin{macro}{\HoLogoHtml@SLiTeX}
%    \begin{macrocode}
\def\HoLogoHtml@SLiTeX{\HoLogoHtml@SLiTeX@lift}
%    \end{macrocode}
%    \end{macro}
%
%    \begin{macro}{\HoLogo@SliTeX}
%    \begin{macrocode}
\def\HoLogo@SliTeX{\HoLogo@SliTeX@narrow}
%    \end{macrocode}
%    \end{macro}
%    \begin{macro}{\HoLogoBkm@SliTeX}
%    \begin{macrocode}
\def\HoLogoBkm@SliTeX{\HoLogoBkm@SliTeX@narrow}
%    \end{macrocode}
%    \end{macro}
%    \begin{macro}{\HoLogoHtml@SliTeX}
%    \begin{macrocode}
\def\HoLogoHtml@SliTeX{\HoLogoHtml@SliTeX@narrow}
%    \end{macrocode}
%    \end{macro}
%
% \subsubsection{\hologo{LuaTeX}}
%
%    \begin{macro}{\HoLogo@LuaTeX}
%    The kerning is an idea of Hans Hagen, see mailing list
%    `luatex at tug dot org' in March 2010.
%    \begin{macrocode}
\def\HoLogo@LuaTeX#1{%
  \HOLOGO@mbox{%
    Lua%
    \HOLOGO@NegativeKerning{aT,oT,To}%
    \hologo{TeX}%
  }%
}
%    \end{macrocode}
%    \end{macro}
%    \begin{macro}{\HoLogoHtml@LuaTeX}
%    \begin{macrocode}
\let\HoLogoHtml@LuaTeX\HoLogo@LuaTeX
%    \end{macrocode}
%    \end{macro}
%
% \subsubsection{\hologo{LuaLaTeX}}
%
%    \begin{macro}{\HoLogo@LuaLaTeX}
%    \begin{macrocode}
\def\HoLogo@LuaLaTeX#1{%
  \HOLOGO@mbox{%
    Lua%
    \hologo{LaTeX}%
  }%
}
%    \end{macrocode}
%    \end{macro}
%    \begin{macro}{\HoLogoHtml@LuaLaTeX}
%    \begin{macrocode}
\let\HoLogoHtml@LuaLaTeX\HoLogo@LuaLaTeX
%    \end{macrocode}
%    \end{macro}
%
% \subsubsection{\hologo{XeTeX}, \hologo{XeLaTeX}}
%
%    \begin{macro}{\HOLOGO@IfCharExists}
%    \begin{macrocode}
\ifluatex
  \ifnum\luatexversion<36 %
  \else
    \def\HOLOGO@IfCharExists#1{%
      \ifnum
        \directlua{%
           if luaotfload and luaotfload.aux then
             if luaotfload.aux.font_has_glyph(%
                    font.current(), \number#1) then % 	 
	       tex.print("1") % 	 
	     end % 	 
	   elseif font and font.fonts and font.current then %
            local f = font.fonts[font.current()]%
            if f.characters and f.characters[\number#1] then %
              tex.print("1")%
            end %
          end%
        }0=\ltx@zero
        \expandafter\ltx@secondoftwo
      \else
        \expandafter\ltx@firstoftwo
      \fi
    }%
  \fi
\fi
\ltx@IfUndefined{HOLOGO@IfCharExists}{%
  \def\HOLOGO@@IfCharExists#1{%
    \begingroup
      \tracinglostchars=\ltx@zero
      \setbox\ltx@zero=\hbox{%
        \kern7sp\char#1\relax
        \ifnum\lastkern>\ltx@zero
          \expandafter\aftergroup\csname iffalse\endcsname
        \else
          \expandafter\aftergroup\csname iftrue\endcsname
        \fi
      }%
      % \if{true|false} from \aftergroup
      \endgroup
      \expandafter\ltx@firstoftwo
    \else
      \endgroup
      \expandafter\ltx@secondoftwo
    \fi
  }%
  \ifxetex
    \ltx@IfUndefined{XeTeXfonttype}{}{%
      \ltx@IfUndefined{XeTeXcharglyph}{}{%
        \def\HOLOGO@IfCharExists#1{%
          \ifnum\XeTeXfonttype\font>\ltx@zero
            \expandafter\ltx@firstofthree
          \else
            \expandafter\ltx@gobble
          \fi
          {%
            \ifnum\XeTeXcharglyph#1>\ltx@zero
              \expandafter\ltx@firstoftwo
            \else
              \expandafter\ltx@secondoftwo
            \fi
          }%
          \HOLOGO@@IfCharExists{#1}%
        }%
      }%
    }%
  \fi
}{}
\ltx@ifundefined{HOLOGO@IfCharExists}{%
  \ifnum64=`\^^^^0040\relax % test for big chars of LuaTeX/XeTeX
    \let\HOLOGO@IfCharExists\HOLOGO@@IfCharExists
  \else
    \def\HOLOGO@IfCharExists#1{%
      \ifnum#1>255 %
        \expandafter\ltx@fourthoffour
      \fi
      \HOLOGO@@IfCharExists{#1}%
    }%
  \fi
}{}
%    \end{macrocode}
%    \end{macro}
%
%    \begin{macro}{\HoLogo@Xe}
%    Source: package \xpackage{dtklogos}
%    \begin{macrocode}
\def\HoLogo@Xe#1{%
  X%
  \kern-.1em\relax
  \HOLOGO@IfCharExists{"018E}{%
    \lower.5ex\hbox{\char"018E}%
  }{%
    \chardef\HOLOGO@choice=\ltx@zero
    \ifdim\fontdimen\ltx@one\font>0pt %
      \ltx@IfUndefined{rotatebox}{%
        \ltx@IfUndefined{pgftext}{%
          \ltx@IfUndefined{psscalebox}{%
            \ltx@IfUndefined{HOLOGO@ScaleBox@\hologoDriver}{%
            }{%
              \chardef\HOLOGO@choice=4 %
            }%
          }{%
            \chardef\HOLOGO@choice=3 %
          }%
        }{%
          \chardef\HOLOGO@choice=2 %
        }%
      }{%
        \chardef\HOLOGO@choice=1 %
      }%
      \ifcase\HOLOGO@choice
        \HOLOGO@WarningUnsupportedDriver{Xe}%
        e%
      \or % 1: \rotatebox
        \begingroup
          \setbox\ltx@zero\hbox{\rotatebox{180}{E}}%
          \ltx@LocDimenA=\dp\ltx@zero
          \advance\ltx@LocDimenA by -.5ex\relax
          \raise\ltx@LocDimenA\box\ltx@zero
        \endgroup
      \or % 2: \pgftext
        \lower.5ex\hbox{%
          \pgfpicture
            \pgftext[rotate=180]{E}%
          \endpgfpicture
        }%
      \or % 3: \psscalebox
        \begingroup
          \setbox\ltx@zero\hbox{\psscalebox{-1 -1}{E}}%
          \ltx@LocDimenA=\dp\ltx@zero
          \advance\ltx@LocDimenA by -.5ex\relax
          \raise\ltx@LocDimenA\box\ltx@zero
        \endgroup
      \or % 4: \HOLOGO@PointReflectBox
        \lower.5ex\hbox{\HOLOGO@PointReflectBox{E}}%
      \else
        \@PackageError{hologo}{Internal error (choice/it}\@ehc
      \fi
    \else
      \ltx@IfUndefined{reflectbox}{%
        \ltx@IfUndefined{pgftext}{%
          \ltx@IfUndefined{psscalebox}{%
            \ltx@IfUndefined{HOLOGO@ScaleBox@\hologoDriver}{%
            }{%
              \chardef\HOLOGO@choice=4 %
            }%
          }{%
            \chardef\HOLOGO@choice=3 %
          }%
        }{%
          \chardef\HOLOGO@choice=2 %
        }%
      }{%
        \chardef\HOLOGO@choice=1 %
      }%
      \ifcase\HOLOGO@choice
        \HOLOGO@WarningUnsupportedDriver{Xe}%
        e%
      \or % 1: reflectbox
        \lower.5ex\hbox{%
          \reflectbox{E}%
        }%
      \or % 2: \pgftext
        \lower.5ex\hbox{%
          \pgfpicture
            \pgftransformxscale{-1}%
            \pgftext{E}%
          \endpgfpicture
        }%
      \or % 3: \psscalebox
        \lower.5ex\hbox{%
          \psscalebox{-1 1}{E}%
        }%
      \or % 4: \HOLOGO@Reflectbox
        \lower.5ex\hbox{%
          \HOLOGO@ReflectBox{E}%
        }%
      \else
        \@PackageError{hologo}{Internal error (choice/up)}\@ehc
      \fi
    \fi
  }%
}
%    \end{macrocode}
%    \end{macro}
%    \begin{macro}{\HoLogoHtml@Xe}
%    \begin{macrocode}
\def\HoLogoHtml@Xe#1{%
  \HoLogoCss@Xe
  \HOLOGO@Span{Xe}{%
    X%
    \HOLOGO@Span{e}{%
      \HCode{&\ltx@hashchar x018e;}%
    }%
  }%
}
%    \end{macrocode}
%    \end{macro}
%    \begin{macro}{\HoLogoCss@Xe}
%    \begin{macrocode}
\def\HoLogoCss@Xe{%
  \Css{%
    span.HoLogo-Xe span.HoLogo-e{%
      position:relative;%
      top:.5ex;%
      left-margin:-.1em;%
    }%
  }%
  \global\let\HoLogoCss@Xe\relax
}
%    \end{macrocode}
%    \end{macro}
%
%    \begin{macro}{\HoLogo@XeTeX}
%    \begin{macrocode}
\def\HoLogo@XeTeX#1{%
  \hologo{Xe}%
  \kern-.15em\relax
  \hologo{TeX}%
}
%    \end{macrocode}
%    \end{macro}
%
%    \begin{macro}{\HoLogoHtml@XeTeX}
%    \begin{macrocode}
\def\HoLogoHtml@XeTeX#1{%
  \HoLogoCss@XeTeX
  \HOLOGO@Span{XeTeX}{%
    \hologo{Xe}%
    \hologo{TeX}%
  }%
}
%    \end{macrocode}
%    \end{macro}
%    \begin{macro}{\HoLogoCss@XeTeX}
%    \begin{macrocode}
\def\HoLogoCss@XeTeX{%
  \Css{%
    span.HoLogo-XeTeX span.HoLogo-TeX{%
      margin-left:-.15em;%
    }%
  }%
  \global\let\HoLogoCss@XeTeX\relax
}
%    \end{macrocode}
%    \end{macro}
%
%    \begin{macro}{\HoLogo@XeLaTeX}
%    \begin{macrocode}
\def\HoLogo@XeLaTeX#1{%
  \hologo{Xe}%
  \kern-.13em%
  \hologo{LaTeX}%
}
%    \end{macrocode}
%    \end{macro}
%    \begin{macro}{\HoLogoHtml@XeLaTeX}
%    \begin{macrocode}
\def\HoLogoHtml@XeLaTeX#1{%
  \HoLogoCss@XeLaTeX
  \HOLOGO@Span{XeLaTeX}{%
    \hologo{Xe}%
    \hologo{LaTeX}%
  }%
}
%    \end{macrocode}
%    \end{macro}
%    \begin{macro}{\HoLogoCss@XeLaTeX}
%    \begin{macrocode}
\def\HoLogoCss@XeLaTeX{%
  \Css{%
    span.HoLogo-XeLaTeX span.HoLogo-Xe{%
      margin-right:-.13em;%
    }%
  }%
  \global\let\HoLogoCss@XeLaTeX\relax
}
%    \end{macrocode}
%    \end{macro}
%
% \subsubsection{\hologo{pdfTeX}, \hologo{pdfLaTeX}}
%
%    \begin{macro}{\HoLogo@pdfTeX}
%    \begin{macrocode}
\def\HoLogo@pdfTeX#1{%
  \HOLOGO@mbox{%
    #1{p}{P}df\hologo{TeX}%
  }%
}
%    \end{macrocode}
%    \end{macro}
%    \begin{macro}{\HoLogoCs@pdfTeX}
%    \begin{macrocode}
\def\HoLogoCs@pdfTeX#1{#1{p}{P}dfTeX}
%    \end{macrocode}
%    \end{macro}
%    \begin{macro}{\HoLogoBkm@pdfTeX}
%    \begin{macrocode}
\def\HoLogoBkm@pdfTeX#1{%
  #1{p}{P}df\hologo{TeX}%
}
%    \end{macrocode}
%    \end{macro}
%    \begin{macro}{\HoLogoHtml@pdfTeX}
%    \begin{macrocode}
\let\HoLogoHtml@pdfTeX\HoLogo@pdfTeX
%    \end{macrocode}
%    \end{macro}
%
%    \begin{macro}{\HoLogo@pdfLaTeX}
%    \begin{macrocode}
\def\HoLogo@pdfLaTeX#1{%
  \HOLOGO@mbox{%
    #1{p}{P}df\hologo{LaTeX}%
  }%
}
%    \end{macrocode}
%    \end{macro}
%    \begin{macro}{\HoLogoCs@pdfLaTeX}
%    \begin{macrocode}
\def\HoLogoCs@pdfLaTeX#1{#1{p}{P}dfLaTeX}
%    \end{macrocode}
%    \end{macro}
%    \begin{macro}{\HoLogoBkm@pdfLaTeX}
%    \begin{macrocode}
\def\HoLogoBkm@pdfLaTeX#1{%
  #1{p}{P}df\hologo{LaTeX}%
}
%    \end{macrocode}
%    \end{macro}
%    \begin{macro}{\HoLogoHtml@pdfLaTeX}
%    \begin{macrocode}
\let\HoLogoHtml@pdfLaTeX\HoLogo@pdfLaTeX
%    \end{macrocode}
%    \end{macro}
%
% \subsubsection{\hologo{VTeX}}
%
%    \begin{macro}{\HoLogo@VTeX}
%    \begin{macrocode}
\def\HoLogo@VTeX#1{%
  \HOLOGO@mbox{%
    V\hologo{TeX}%
  }%
}
%    \end{macrocode}
%    \end{macro}
%    \begin{macro}{\HoLogoHtml@VTeX}
%    \begin{macrocode}
\let\HoLogoHtml@VTeX\HoLogo@VTeX
%    \end{macrocode}
%    \end{macro}
%
% \subsubsection{\hologo{AmS}, \dots}
%
%    Source: class \xclass{amsdtx}
%
%    \begin{macro}{\HoLogo@AmS}
%    \begin{macrocode}
\def\HoLogo@AmS#1{%
  \HoLogoFont@font{AmS}{sy}{%
    A%
    \kern-.1667em%
    \lower.5ex\hbox{M}%
    \kern-.125em%
    S%
  }%
}
%    \end{macrocode}
%    \end{macro}
%    \begin{macro}{\HoLogoBkm@AmS}
%    \begin{macrocode}
\def\HoLogoBkm@AmS#1{AmS}
%    \end{macrocode}
%    \end{macro}
%    \begin{macro}{\HoLogoHtml@AmS}
%    \begin{macrocode}
\def\HoLogoHtml@AmS#1{%
  \HoLogoCss@AmS
%  \HoLogoFont@font{AmS}{sy}{%
    \HOLOGO@Span{AmS}{%
      A%
      \HOLOGO@Span{M}{M}%
      S%
    }%
%   }%
}
%    \end{macrocode}
%    \end{macro}
%    \begin{macro}{\HoLogoCss@AmS}
%    \begin{macrocode}
\def\HoLogoCss@AmS{%
  \Css{%
    span.HoLogo-AmS span.HoLogo-M{%
      position:relative;%
      top:.5ex;%
      margin-left:-.1667em;%
      margin-right:-.125em;%
      text-decoration:none;%
    }%
  }%
  \global\let\HoLogoCss@AmS\relax
}
%    \end{macrocode}
%    \end{macro}
%
%    \begin{macro}{\HoLogo@AmSTeX}
%    \begin{macrocode}
\def\HoLogo@AmSTeX#1{%
  \hologo{AmS}%
  \HOLOGO@hyphen
  \hologo{TeX}%
}
%    \end{macrocode}
%    \end{macro}
%    \begin{macro}{\HoLogoBkm@AmSTeX}
%    \begin{macrocode}
\def\HoLogoBkm@AmSTeX#1{AmS-TeX}%
%    \end{macrocode}
%    \end{macro}
%    \begin{macro}{\HoLogoHtml@AmSTeX}
%    \begin{macrocode}
\let\HoLogoHtml@AmSTeX\HoLogo@AmSTeX
%    \end{macrocode}
%    \end{macro}
%
%    \begin{macro}{\HoLogo@AmSLaTeX}
%    \begin{macrocode}
\def\HoLogo@AmSLaTeX#1{%
  \hologo{AmS}%
  \HOLOGO@hyphen
  \hologo{LaTeX}%
}
%    \end{macrocode}
%    \end{macro}
%    \begin{macro}{\HoLogoBkm@AmSLaTeX}
%    \begin{macrocode}
\def\HoLogoBkm@AmSLaTeX#1{AmS-LaTeX}%
%    \end{macrocode}
%    \end{macro}
%    \begin{macro}{\HoLogoHtml@AmSLaTeX}
%    \begin{macrocode}
\let\HoLogoHtml@AmSLaTeX\HoLogo@AmSLaTeX
%    \end{macrocode}
%    \end{macro}
%
% \subsubsection{\hologo{BibTeX}}
%
%    \begin{macro}{\HoLogo@BibTeX@sc}
%    A definition of \hologo{BibTeX} is provided in
%    the documentation source for the manual of \hologo{BibTeX}
%    \cite{btxdoc}.
%\begin{quote}
%\begin{verbatim}
%\def\BibTeX{%
%  {%
%    \rm
%    B%
%    \kern-.05em%
%    {%
%      \sc
%      i%
%      \kern-.025em %
%      b%
%    }%
%    \kern-.08em
%    T%
%    \kern-.1667em%
%    \lower.7ex\hbox{E}%
%    \kern-.125em%
%    X%
%  }%
%}
%\end{verbatim}
%\end{quote}
%    \begin{macrocode}
\def\HoLogo@BibTeX@sc#1{%
  B%
  \kern-.05em%
  \HoLogoFont@font{BibTeX}{sc}{%
    i%
    \kern-.025em%
    b%
  }%
  \HOLOGO@discretionary
  \kern-.08em%
  \hologo{TeX}%
}
%    \end{macrocode}
%    \end{macro}
%    \begin{macro}{\HoLogoHtml@BibTeX@sc}
%    \begin{macrocode}
\def\HoLogoHtml@BibTeX@sc#1{%
  \HoLogoCss@BibTeX@sc
  \HOLOGO@Span{BibTeX-sc}{%
    B%
    \HOLOGO@Span{i}{i}%
    \HOLOGO@Span{b}{b}%
    \hologo{TeX}%
  }%
}
%    \end{macrocode}
%    \end{macro}
%    \begin{macro}{\HoLogoCss@BibTeX@sc}
%    \begin{macrocode}
\def\HoLogoCss@BibTeX@sc{%
  \Css{%
    span.HoLogo-BibTeX-sc span.HoLogo-i{%
      margin-left:-.05em;%
      margin-right:-.025em;%
      font-variant:small-caps;%
    }%
  }%
  \Css{%
    span.HoLogo-BibTeX-sc span.HoLogo-b{%
      margin-right:-.08em;%
      font-variant:small-caps;%
    }%
  }%
  \global\let\HoLogoCss@BibTeX@sc\relax
}
%    \end{macrocode}
%    \end{macro}
%
%    \begin{macro}{\HoLogo@BibTeX@sf}
%    Variant \xoption{sf} avoids trouble with unavailable
%    small caps fonts (e.g., bold versions of Computer Modern or
%    Latin Modern). The definition is taken from
%    package \xpackage{dtklogos} \cite{dtklogos}.
%\begin{quote}
%\begin{verbatim}
%\DeclareRobustCommand{\BibTeX}{%
%  B%
%  \kern-.05em%
%  \hbox{%
%    $\m@th$% %% force math size calculations
%    \csname S@\f@size\endcsname
%    \fontsize\sf@size\z@
%    \math@fontsfalse
%    \selectfont
%    I%
%    \kern-.025em%
%    B
%  }%
%  \kern-.08em%
%  \-%
%  \TeX
%}
%\end{verbatim}
%\end{quote}
%    \begin{macrocode}
\def\HoLogo@BibTeX@sf#1{%
  B%
  \kern-.05em%
  \HoLogoFont@font{BibTeX}{bibsf}{%
    I%
    \kern-.025em%
    B%
  }%
  \HOLOGO@discretionary
  \kern-.08em%
  \hologo{TeX}%
}
%    \end{macrocode}
%    \end{macro}
%    \begin{macro}{\HoLogoHtml@BibTeX@sf}
%    \begin{macrocode}
\def\HoLogoHtml@BibTeX@sf#1{%
  \HoLogoCss@BibTeX@sf
  \HOLOGO@Span{BibTeX-sf}{%
    B%
    \HoLogoFont@font{BibTeX}{bibsf}{%
      \HOLOGO@Span{i}{I}%
      B%
    }%
    \hologo{TeX}%
  }%
}
%    \end{macrocode}
%    \end{macro}
%    \begin{macro}{\HoLogoCss@BibTeX@sf}
%    \begin{macrocode}
\def\HoLogoCss@BibTeX@sf{%
  \Css{%
    span.HoLogo-BibTeX-sf span.HoLogo-i{%
      margin-left:-.05em;%
      margin-right:-.025em;%
    }%
  }%
  \Css{%
    span.HoLogo-BibTeX-sf span.HoLogo-TeX{%
      margin-left:-.08em;%
    }%
  }%
  \global\let\HoLogoCss@BibTeX@sf\relax
}
%    \end{macrocode}
%    \end{macro}
%
%    \begin{macro}{\HoLogo@BibTeX}
%    \begin{macrocode}
\def\HoLogo@BibTeX{\HoLogo@BibTeX@sf}
%    \end{macrocode}
%    \end{macro}
%    \begin{macro}{\HoLogoHtml@BibTeX}
%    \begin{macrocode}
\def\HoLogoHtml@BibTeX{\HoLogoHtml@BibTeX@sf}
%    \end{macrocode}
%    \end{macro}
%
% \subsubsection{\hologo{BibTeX8}}
%
%    \begin{macro}{\HoLogo@BibTeX8}
%    \begin{macrocode}
\expandafter\def\csname HoLogo@BibTeX8\endcsname#1{%
  \hologo{BibTeX}%
  8%
}
%    \end{macrocode}
%    \end{macro}
%
%    \begin{macro}{\HoLogoBkm@BibTeX8}
%    \begin{macrocode}
\expandafter\def\csname HoLogoBkm@BibTeX8\endcsname#1{%
  \hologo{BibTeX}%
  8%
}
%    \end{macrocode}
%    \end{macro}
%    \begin{macro}{\HoLogoHtml@BibTeX8}
%    \begin{macrocode}
\expandafter
\let\csname HoLogoHtml@BibTeX8\expandafter\endcsname
\csname HoLogo@BibTeX8\endcsname
%    \end{macrocode}
%    \end{macro}
%
% \subsubsection{\hologo{ConTeXt}}
%
%    \begin{macro}{\HoLogo@ConTeXt@simple}
%    \begin{macrocode}
\def\HoLogo@ConTeXt@simple#1{%
  \HOLOGO@mbox{Con}%
  \HOLOGO@discretionary
  \HOLOGO@mbox{\hologo{TeX}t}%
}
%    \end{macrocode}
%    \end{macro}
%    \begin{macro}{\HoLogoHtml@ConTeXt@simple}
%    \begin{macrocode}
\let\HoLogoHtml@ConTeXt@simple\HoLogo@ConTeXt@simple
%    \end{macrocode}
%    \end{macro}
%
%    \begin{macro}{\HoLogo@ConTeXt@narrow}
%    This definition of logo \hologo{ConTeXt} with variant \xoption{narrow}
%    comes from TUGboat's class \xclass{ltugboat} (version 2010/11/15 v2.8).
%    \begin{macrocode}
\def\HoLogo@ConTeXt@narrow#1{%
  \HOLOGO@mbox{C\kern-.0333emon}%
  \HOLOGO@discretionary
  \kern-.0667em%
  \HOLOGO@mbox{\hologo{TeX}\kern-.0333emt}%
}
%    \end{macrocode}
%    \end{macro}
%    \begin{macro}{\HoLogoHtml@ConTeXt@narrow}
%    \begin{macrocode}
\def\HoLogoHtml@ConTeXt@narrow#1{%
  \HoLogoCss@ConTeXt@narrow
  \HOLOGO@Span{ConTeXt-narrow}{%
    \HOLOGO@Span{C}{C}%
    on%
    \hologo{TeX}%
    t%
  }%
}
%    \end{macrocode}
%    \end{macro}
%    \begin{macro}{\HoLogoCss@ConTeXt@narrow}
%    \begin{macrocode}
\def\HoLogoCss@ConTeXt@narrow{%
  \Css{%
    span.HoLogo-ConTeXt-narrow span.HoLogo-C{%
      margin-left:-.0333em;%
    }%
  }%
  \Css{%
    span.HoLogo-ConTeXt-narrow span.HoLogo-TeX{%
      margin-left:-.0667em;%
      margin-right:-.0333em;%
    }%
  }%
  \global\let\HoLogoCss@ConTeXt@narrow\relax
}
%    \end{macrocode}
%    \end{macro}
%
%    \begin{macro}{\HoLogo@ConTeXt}
%    \begin{macrocode}
\def\HoLogo@ConTeXt{\HoLogo@ConTeXt@narrow}
%    \end{macrocode}
%    \end{macro}
%    \begin{macro}{\HoLogoHtml@ConTeXt}
%    \begin{macrocode}
\def\HoLogoHtml@ConTeXt{\HoLogoHtml@ConTeXt@narrow}
%    \end{macrocode}
%    \end{macro}
%
% \subsubsection{\hologo{emTeX}}
%
%    \begin{macro}{\HoLogo@emTeX}
%    \begin{macrocode}
\def\HoLogo@emTeX#1{%
  \HOLOGO@mbox{#1{e}{E}m}%
  \HOLOGO@discretionary
  \hologo{TeX}%
}
%    \end{macrocode}
%    \end{macro}
%    \begin{macro}{\HoLogoCs@emTeX}
%    \begin{macrocode}
\def\HoLogoCs@emTeX#1{#1{e}{E}mTeX}%
%    \end{macrocode}
%    \end{macro}
%    \begin{macro}{\HoLogoBkm@emTeX}
%    \begin{macrocode}
\def\HoLogoBkm@emTeX#1{%
  #1{e}{E}m\hologo{TeX}%
}
%    \end{macrocode}
%    \end{macro}
%    \begin{macro}{\HoLogoHtml@emTeX}
%    \begin{macrocode}
\let\HoLogoHtml@emTeX\HoLogo@emTeX
%    \end{macrocode}
%    \end{macro}
%
% \subsubsection{\hologo{ExTeX}}
%
%    \begin{macro}{\HoLogo@ExTeX}
%    The definition is taken from the FAQ of the
%    project \hologo{ExTeX}
%    \cite{ExTeX-FAQ}.
%\begin{quote}
%\begin{verbatim}
%\def\ExTeX{%
%  \textrm{% Logo always with serifs
%    \ensuremath{%
%      \textstyle
%      \varepsilon_{%
%        \kern-0.15em%
%        \mathcal{X}%
%      }%
%    }%
%    \kern-.15em%
%    \TeX
%  }%
%}
%\end{verbatim}
%\end{quote}
%    \begin{macrocode}
\def\HoLogo@ExTeX#1{%
  \HoLogoFont@font{ExTeX}{rm}{%
    \ltx@mbox{%
      \HOLOGO@MathSetup
      $%
        \textstyle
        \varepsilon_{%
          \kern-0.15em%
          \HoLogoFont@font{ExTeX}{sy}{X}%
        }%
      $%
    }%
    \HOLOGO@discretionary
    \kern-.15em%
    \hologo{TeX}%
  }%
}
%    \end{macrocode}
%    \end{macro}
%    \begin{macro}{\HoLogoHtml@ExTeX}
%    \begin{macrocode}
\def\HoLogoHtml@ExTeX#1{%
  \HoLogoCss@ExTeX
  \HoLogoFont@font{ExTeX}{rm}{%
    \HOLOGO@Span{ExTeX}{%
      \ltx@mbox{%
        \HOLOGO@MathSetup
        $\textstyle\varepsilon$%
        \HOLOGO@Span{X}{$\textstyle\chi$}%
        \hologo{TeX}%
      }%
    }%
  }%
}
%    \end{macrocode}
%    \end{macro}
%    \begin{macro}{\HoLogoBkm@ExTeX}
%    \begin{macrocode}
\def\HoLogoBkm@ExTeX#1{%
  \HOLOGO@PdfdocUnicode{#1{e}{E}x}{\textepsilon\textchi}%
  \hologo{TeX}%
}
%    \end{macrocode}
%    \end{macro}
%    \begin{macro}{\HoLogoCss@ExTeX}
%    \begin{macrocode}
\def\HoLogoCss@ExTeX{%
  \Css{%
    span.HoLogo-ExTeX{%
      font-family:serif;%
    }%
  }%
  \Css{%
    span.HoLogo-ExTeX span.HoLogo-TeX{%
      margin-left:-.15em;%
    }%
  }%
  \global\let\HoLogoCss@ExTeX\relax
}
%    \end{macrocode}
%    \end{macro}
%
% \subsubsection{\hologo{MiKTeX}}
%
%    \begin{macro}{\HoLogo@MiKTeX}
%    \begin{macrocode}
\def\HoLogo@MiKTeX#1{%
  \HOLOGO@mbox{MiK}%
  \HOLOGO@discretionary
  \hologo{TeX}%
}
%    \end{macrocode}
%    \end{macro}
%    \begin{macro}{\HoLogoHtml@MiKTeX}
%    \begin{macrocode}
\let\HoLogoHtml@MiKTeX\HoLogo@MiKTeX
%    \end{macrocode}
%    \end{macro}
%
% \subsubsection{\hologo{OzTeX} and friends}
%
%    Source: \hologo{OzTeX} FAQ \cite{OzTeX}:
%    \begin{quote}
%      |\def\OzTeX{O\kern-.03em z\kern-.15em\TeX}|\\
%      (There is no kerning in OzMF, OzMP and OzTtH.)
%    \end{quote}
%
%    \begin{macro}{\HoLogo@OzTeX}
%    \begin{macrocode}
\def\HoLogo@OzTeX#1{%
  O%
  \kern-.03em %
  z%
  \kern-.15em %
  \hologo{TeX}%
}
%    \end{macrocode}
%    \end{macro}
%    \begin{macro}{\HoLogoHtml@OzTeX}
%    \begin{macrocode}
\def\HoLogoHtml@OzTeX#1{%
  \HoLogoCss@OzTeX
  \HOLOGO@Span{OzTeX}{%
    O%
    \HOLOGO@Span{z}{z}%
    \hologo{TeX}%
  }%
}
%    \end{macrocode}
%    \end{macro}
%    \begin{macro}{\HoLogoCss@OzTeX}
%    \begin{macrocode}
\def\HoLogoCss@OzTeX{%
  \Css{%
    span.HoLogo-OzTeX span.HoLogo-z{%
      margin-left:-.03em;%
      margin-right:-.15em;%
    }%
  }%
  \global\let\HoLogoCss@OzTeX\relax
}
%    \end{macrocode}
%    \end{macro}
%
%    \begin{macro}{\HoLogo@OzMF}
%    \begin{macrocode}
\def\HoLogo@OzMF#1{%
  \HOLOGO@mbox{OzMF}%
}
%    \end{macrocode}
%    \end{macro}
%    \begin{macro}{\HoLogo@OzMP}
%    \begin{macrocode}
\def\HoLogo@OzMP#1{%
  \HOLOGO@mbox{OzMP}%
}
%    \end{macrocode}
%    \end{macro}
%    \begin{macro}{\HoLogo@OzTtH}
%    \begin{macrocode}
\def\HoLogo@OzTtH#1{%
  \HOLOGO@mbox{OzTtH}%
}
%    \end{macrocode}
%    \end{macro}
%
% \subsubsection{\hologo{PCTeX}}
%
%    \begin{macro}{\HoLogo@PCTeX}
%    \begin{macrocode}
\def\HoLogo@PCTeX#1{%
  \HOLOGO@mbox{PC}%
  \hologo{TeX}%
}
%    \end{macrocode}
%    \end{macro}
%    \begin{macro}{\HoLogoHtml@PCTeX}
%    \begin{macrocode}
\let\HoLogoHtml@PCTeX\HoLogo@PCTeX
%    \end{macrocode}
%    \end{macro}
%
% \subsubsection{\hologo{PiCTeX}}
%
%    The original definitions from \xfile{pictex.tex} \cite{PiCTeX}:
%\begin{quote}
%\begin{verbatim}
%\def\PiC{%
%  P%
%  \kern-.12em%
%  \lower.5ex\hbox{I}%
%  \kern-.075em%
%  C%
%}
%\def\PiCTeX{%
%  \PiC
%  \kern-.11em%
%  \TeX
%}
%\end{verbatim}
%\end{quote}
%
%    \begin{macro}{\HoLogo@PiC}
%    \begin{macrocode}
\def\HoLogo@PiC#1{%
  P%
  \kern-.12em%
  \lower.5ex\hbox{I}%
  \kern-.075em%
  C%
  \HOLOGO@SpaceFactor
}
%    \end{macrocode}
%    \end{macro}
%    \begin{macro}{\HoLogoHtml@PiC}
%    \begin{macrocode}
\def\HoLogoHtml@PiC#1{%
  \HoLogoCss@PiC
  \HOLOGO@Span{PiC}{%
    P%
    \HOLOGO@Span{i}{I}%
    C%
  }%
}
%    \end{macrocode}
%    \end{macro}
%    \begin{macro}{\HoLogoCss@PiC}
%    \begin{macrocode}
\def\HoLogoCss@PiC{%
  \Css{%
    span.HoLogo-PiC span.HoLogo-i{%
      position:relative;%
      top:.5ex;%
      margin-left:-.12em;%
      margin-right:-.075em;%
      text-decoration:none;%
    }%
  }%
  \global\let\HoLogoCss@PiC\relax
}
%    \end{macrocode}
%    \end{macro}
%
%    \begin{macro}{\HoLogo@PiCTeX}
%    \begin{macrocode}
\def\HoLogo@PiCTeX#1{%
  \hologo{PiC}%
  \HOLOGO@discretionary
  \kern-.11em%
  \hologo{TeX}%
}
%    \end{macrocode}
%    \end{macro}
%    \begin{macro}{\HoLogoHtml@PiCTeX}
%    \begin{macrocode}
\def\HoLogoHtml@PiCTeX#1{%
  \HoLogoCss@PiCTeX
  \HOLOGO@Span{PiCTeX}{%
    \hologo{PiC}%
    \hologo{TeX}%
  }%
}
%    \end{macrocode}
%    \end{macro}
%    \begin{macro}{\HoLogoCss@PiCTeX}
%    \begin{macrocode}
\def\HoLogoCss@PiCTeX{%
  \Css{%
    span.HoLogo-PiCTeX span.HoLogo-PiC{%
      margin-right:-.11em;%
    }%
  }%
  \global\let\HoLogoCss@PiCTeX\relax
}
%    \end{macrocode}
%    \end{macro}
%
% \subsubsection{\hologo{teTeX}}
%
%    \begin{macro}{\HoLogo@teTeX}
%    \begin{macrocode}
\def\HoLogo@teTeX#1{%
  \HOLOGO@mbox{#1{t}{T}e}%
  \HOLOGO@discretionary
  \hologo{TeX}%
}
%    \end{macrocode}
%    \end{macro}
%    \begin{macro}{\HoLogoCs@teTeX}
%    \begin{macrocode}
\def\HoLogoCs@teTeX#1{#1{t}{T}dfTeX}
%    \end{macrocode}
%    \end{macro}
%    \begin{macro}{\HoLogoBkm@teTeX}
%    \begin{macrocode}
\def\HoLogoBkm@teTeX#1{%
  #1{t}{T}e\hologo{TeX}%
}
%    \end{macrocode}
%    \end{macro}
%    \begin{macro}{\HoLogoHtml@teTeX}
%    \begin{macrocode}
\let\HoLogoHtml@teTeX\HoLogo@teTeX
%    \end{macrocode}
%    \end{macro}
%
% \subsubsection{\hologo{TeX4ht}}
%
%    \begin{macro}{\HoLogo@TeX4ht}
%    \begin{macrocode}
\expandafter\def\csname HoLogo@TeX4ht\endcsname#1{%
  \HOLOGO@mbox{\hologo{TeX}4ht}%
}
%    \end{macrocode}
%    \end{macro}
%    \begin{macro}{\HoLogoHtml@TeX4ht}
%    \begin{macrocode}
\expandafter
\let\csname HoLogoHtml@TeX4ht\expandafter\endcsname
\csname HoLogo@TeX4ht\endcsname
%    \end{macrocode}
%    \end{macro}
%
%
% \subsubsection{\hologo{SageTeX}}
%
%    \begin{macro}{\HoLogo@SageTeX}
%    \begin{macrocode}
\def\HoLogo@SageTeX#1{%
  \HOLOGO@mbox{Sage}%
  \HOLOGO@discretionary
  \HOLOGO@NegativeKerning{eT,oT,To}%
  \hologo{TeX}%
}
%    \end{macrocode}
%    \end{macro}
%    \begin{macro}{\HoLogoHtml@SageTeX}
%    \begin{macrocode}
\let\HoLogoHtml@SageTeX\HoLogo@SageTeX
%    \end{macrocode}
%    \end{macro}
%
% \subsection{\hologo{METAFONT} and friends}
%
%    \begin{macro}{\HoLogo@METAFONT}
%    \begin{macrocode}
\def\HoLogo@METAFONT#1{%
  \HoLogoFont@font{METAFONT}{logo}{%
    \HOLOGO@mbox{META}%
    \HOLOGO@discretionary
    \HOLOGO@mbox{FONT}%
  }%
}
%    \end{macrocode}
%    \end{macro}
%
%    \begin{macro}{\HoLogo@METAPOST}
%    \begin{macrocode}
\def\HoLogo@METAPOST#1{%
  \HoLogoFont@font{METAPOST}{logo}{%
    \HOLOGO@mbox{META}%
    \HOLOGO@discretionary
    \HOLOGO@mbox{POST}%
  }%
}
%    \end{macrocode}
%    \end{macro}
%
%    \begin{macro}{\HoLogo@MetaFun}
%    \begin{macrocode}
\def\HoLogo@MetaFun#1{%
  \HOLOGO@mbox{Meta}%
  \HOLOGO@discretionary
  \HOLOGO@mbox{Fun}%
}
%    \end{macrocode}
%    \end{macro}
%
%    \begin{macro}{\HoLogo@MetaPost}
%    \begin{macrocode}
\def\HoLogo@MetaPost#1{%
  \HOLOGO@mbox{Meta}%
  \HOLOGO@discretionary
  \HOLOGO@mbox{Post}%
}
%    \end{macrocode}
%    \end{macro}
%
% \subsection{Others}
%
% \subsubsection{\hologo{biber}}
%
%    \begin{macro}{\HoLogo@biber}
%    \begin{macrocode}
\def\HoLogo@biber#1{%
  \HOLOGO@mbox{#1{b}{B}i}%
  \HOLOGO@discretionary
  \HOLOGO@mbox{ber}%
}
%    \end{macrocode}
%    \end{macro}
%    \begin{macro}{\HoLogoCs@biber}
%    \begin{macrocode}
\def\HoLogoCs@biber#1{#1{b}{B}iber}
%    \end{macrocode}
%    \end{macro}
%    \begin{macro}{\HoLogoBkm@biber}
%    \begin{macrocode}
\def\HoLogoBkm@biber#1{%
  #1{b}{B}iber%
}
%    \end{macrocode}
%    \end{macro}
%    \begin{macro}{\HoLogoHtml@biber}
%    \begin{macrocode}
\let\HoLogoHtml@biber\HoLogo@biber
%    \end{macrocode}
%    \end{macro}
%
% \subsubsection{\hologo{KOMAScript}}
%
%    \begin{macro}{\HoLogo@KOMAScript}
%    The definition for \hologo{KOMAScript} is taken
%    from \hologo{KOMAScript} (\xfile{scrlogo.dtx}, reformatted) \cite{scrlogo}:
%\begin{quote}
%\begin{verbatim}
%\@ifundefined{KOMAScript}{%
%  \DeclareRobustCommand{\KOMAScript}{%
%    \textsf{%
%      K\kern.05em O\kern.05emM\kern.05em A%
%      \kern.1em-\kern.1em %
%      Script%
%    }%
%  }%
%}{}
%\end{verbatim}
%\end{quote}
%    \begin{macrocode}
\def\HoLogo@KOMAScript#1{%
  \HoLogoFont@font{KOMAScript}{sf}{%
    \HOLOGO@mbox{%
      K\kern.05em%
      O\kern.05em%
      M\kern.05em%
      A%
    }%
    \kern.1em%
    \HOLOGO@hyphen
    \kern.1em%
    \HOLOGO@mbox{Script}%
  }%
}
%    \end{macrocode}
%    \end{macro}
%    \begin{macro}{\HoLogoBkm@KOMAScript}
%    \begin{macrocode}
\def\HoLogoBkm@KOMAScript#1{%
  KOMA-Script%
}
%    \end{macrocode}
%    \end{macro}
%    \begin{macro}{\HoLogoHtml@KOMAScript}
%    \begin{macrocode}
\def\HoLogoHtml@KOMAScript#1{%
  \HoLogoCss@KOMAScript
  \HoLogoFont@font{KOMAScript}{sf}{%
    \HOLOGO@Span{KOMAScript}{%
      K%
      \HOLOGO@Span{O}{O}%
      M%
      \HOLOGO@Span{A}{A}%
      \HOLOGO@Span{hyphen}{-}%
      Script%
    }%
  }%
}
%    \end{macrocode}
%    \end{macro}
%    \begin{macro}{\HoLogoCss@KOMAScript}
%    \begin{macrocode}
\def\HoLogoCss@KOMAScript{%
  \Css{%
    span.HoLogo-KOMAScript{%
      font-family:sans-serif;%
    }%
  }%
  \Css{%
    span.HoLogo-KOMAScript span.HoLogo-O{%
      padding-left:.05em;%
      padding-right:.05em;%
    }%
  }%
  \Css{%
    span.HoLogo-KOMAScript span.HoLogo-A{%
      padding-left:.05em;%
    }%
  }%
  \Css{%
    span.HoLogo-KOMAScript span.HoLogo-hyphen{%
      padding-left:.1em;%
      padding-right:.1em;%
    }%
  }%
  \global\let\HoLogoCss@KOMAScript\relax
}
%    \end{macrocode}
%    \end{macro}
%
% \subsubsection{\hologo{LyX}}
%
%    \begin{macro}{\HoLogo@LyX}
%    The definition is taken from the documentation source files
%    of \hologo{LyX}, \xfile{Intro.lyx} \cite{LyX}:
%\begin{quote}
%\begin{verbatim}
%\def\LyX{%
%  \texorpdfstring{%
%    L\kern-.1667em\lower.25em\hbox{Y}\kern-.125emX\@%
%  }{%
%    LyX%
%  }%
%}
%\end{verbatim}
%\end{quote}
%    \begin{macrocode}
\def\HoLogo@LyX#1{%
  L%
  \kern-.1667em%
  \lower.25em\hbox{Y}%
  \kern-.125em%
  X%
  \HOLOGO@SpaceFactor
}
%    \end{macrocode}
%    \end{macro}
%    \begin{macro}{\HoLogoHtml@LyX}
%    \begin{macrocode}
\def\HoLogoHtml@LyX#1{%
  \HoLogoCss@LyX
  \HOLOGO@Span{LyX}{%
    L%
    \HOLOGO@Span{y}{Y}%
    X%
  }%
}
%    \end{macrocode}
%    \end{macro}
%    \begin{macro}{\HoLogoCss@LyX}
%    \begin{macrocode}
\def\HoLogoCss@LyX{%
  \Css{%
    span.HoLogo-LyX span.HoLogo-y{%
      position:relative;%
      top:.25em;%
      margin-left:-.1667em;%
      margin-right:-.125em;%
      text-decoration:none;%
    }%
  }%
  \global\let\HoLogoCss@LyX\relax
}
%    \end{macrocode}
%    \end{macro}
%
% \subsubsection{\hologo{NTS}}
%
%    \begin{macro}{\HoLogo@NTS}
%    Definition for \hologo{NTS} can be found in
%    package \xpackage{etex\textunderscore man} for the \hologo{eTeX} manual \cite{etexman}
%    and in package \xpackage{dtklogos} \cite{dtklogos}:
%\begin{quote}
%\begin{verbatim}
%\def\NTS{%
%  \leavevmode
%  \hbox{%
%    $%
%      \cal N%
%      \kern-0.35em%
%      \lower0.5ex\hbox{$\cal T$}%
%      \kern-0.2em%
%      S%
%    $%
%  }%
%}
%\end{verbatim}
%\end{quote}
%    \begin{macrocode}
\def\HoLogo@NTS#1{%
  \HoLogoFont@font{NTS}{sy}{%
    N\/%
    \kern-.35em%
    \lower.5ex\hbox{T\/}%
    \kern-.2em%
    S\/%
  }%
  \HOLOGO@SpaceFactor
}
%    \end{macrocode}
%    \end{macro}
%
% \subsubsection{\Hologo{TTH} (\hologo{TeX} to HTML translator)}
%
%    Source: \url{http://hutchinson.belmont.ma.us/tth/}
%    In the HTML source the second `T' is printed as subscript.
%\begin{quote}
%\begin{verbatim}
%T<sub>T</sub>H
%\end{verbatim}
%\end{quote}
%    \begin{macro}{\HoLogo@TTH}
%    \begin{macrocode}
\def\HoLogo@TTH#1{%
  \ltx@mbox{%
    T\HOLOGO@SubScript{T}H%
  }%
  \HOLOGO@SpaceFactor
}
%    \end{macrocode}
%    \end{macro}
%
%    \begin{macro}{\HoLogoHtml@TTH}
%    \begin{macrocode}
\def\HoLogoHtml@TTH#1{%
  T\HCode{<sub>}T\HCode{</sub>}H%
}
%    \end{macrocode}
%    \end{macro}
%
% \subsubsection{\Hologo{HanTheThanh}}
%
%    Partial source: Package \xpackage{dtklogos}.
%    The double accent is U+1EBF (latin small letter e with circumflex
%    and acute).
%    \begin{macro}{\HoLogo@HanTheThanh}
%    \begin{macrocode}
\def\HoLogo@HanTheThanh#1{%
  \ltx@mbox{H\`an}%
  \HOLOGO@space
  \ltx@mbox{%
    Th%
    \HOLOGO@IfCharExists{"1EBF}{%
      \char"1EBF\relax
    }{%
      \^e\hbox to 0pt{\hss\raise .5ex\hbox{\'{}}}%
    }%
  }%
  \HOLOGO@space
  \ltx@mbox{Th\`anh}%
}
%    \end{macrocode}
%    \end{macro}
%    \begin{macro}{\HoLogoBkm@HanTheThanh}
%    \begin{macrocode}
\def\HoLogoBkm@HanTheThanh#1{%
  H\`an %
  Th\HOLOGO@PdfdocUnicode{\^e}{\9036\277} %
  Th\`anh%
}
%    \end{macrocode}
%    \end{macro}
%    \begin{macro}{\HoLogoHtml@HanTheThanh}
%    \begin{macrocode}
\def\HoLogoHtml@HanTheThanh#1{%
  H\`an %
  Th\HCode{&\ltx@hashchar x1ebf;} %
  Th\`anh%
}
%    \end{macrocode}
%    \end{macro}
%
% \subsection{Driver detection}
%
%    \begin{macrocode}
\HOLOGO@IfExists\InputIfFileExists{%
  \InputIfFileExists{hologo.cfg}{}{}%
}{%
  \ltx@IfUndefined{pdf@filesize}{%
    \def\HOLOGO@InputIfExists{%
      \openin\HOLOGO@temp=hologo.cfg\relax
      \ifeof\HOLOGO@temp
        \closein\HOLOGO@temp
      \else
        \closein\HOLOGO@temp
        \begingroup
          \def\x{LaTeX2e}%
        \expandafter\endgroup
        \ifx\fmtname\x
          \input{hologo.cfg}%
        \else
          \input hologo.cfg\relax
        \fi
      \fi
    }%
    \ltx@IfUndefined{newread}{%
      \chardef\HOLOGO@temp=15 %
      \def\HOLOGO@CheckRead{%
        \ifeof\HOLOGO@temp
          \HOLOGO@InputIfExists
        \else
          \ifcase\HOLOGO@temp
            \@PackageWarningNoLine{hologo}{%
              Configuration file ignored, because\MessageBreak
              a free read register could not be found%
            }%
          \else
            \begingroup
              \count\ltx@cclv=\HOLOGO@temp
              \advance\ltx@cclv by \ltx@minusone
              \edef\x{\endgroup
                \chardef\noexpand\HOLOGO@temp=\the\count\ltx@cclv
                \relax
              }%
            \x
          \fi
        \fi
      }%
    }{%
      \csname newread\endcsname\HOLOGO@temp
      \HOLOGO@InputIfExists
    }%
  }{%
    \edef\HOLOGO@temp{\pdf@filesize{hologo.cfg}}%
    \ifx\HOLOGO@temp\ltx@empty
    \else
      \ifnum\HOLOGO@temp>0 %
        \begingroup
          \def\x{LaTeX2e}%
        \expandafter\endgroup
        \ifx\fmtname\x
          \input{hologo.cfg}%
        \else
          \input hologo.cfg\relax
        \fi
      \else
        \@PackageInfoNoLine{hologo}{%
          Empty configuration file `hologo.cfg' ignored%
        }%
      \fi
    \fi
  }%
}
%    \end{macrocode}
%
%    \begin{macrocode}
\def\HOLOGO@temp#1#2{%
  \kv@define@key{HoLogoDriver}{#1}[]{%
    \begingroup
      \def\HOLOGO@temp{##1}%
      \ltx@onelevel@sanitize\HOLOGO@temp
      \ifx\HOLOGO@temp\ltx@empty
      \else
        \@PackageError{hologo}{%
          Value (\HOLOGO@temp) not permitted for option `#1'%
        }%
        \@ehc
      \fi
    \endgroup
    \def\hologoDriver{#2}%
  }%
}%
\def\HOLOGO@@temp#1#2{%
  \ifx\kv@value\relax
    \HOLOGO@temp{#1}{#1}%
  \else
    \HOLOGO@temp{#1}{#2}%
  \fi
}%
\kv@parse@normalized{%
  pdftex,%
  luatex=pdftex,%
  dvipdfm,%
  dvipdfmx=dvipdfm,%
  dvips,%
  dvipsone=dvips,%
  xdvi=dvips,%
  xetex,%
  vtex,%
}\HOLOGO@@temp
%    \end{macrocode}
%
%    \begin{macrocode}
\kv@define@key{HoLogoDriver}{driverfallback}{%
  \def\HOLOGO@DriverFallback{#1}%
}
%    \end{macrocode}
%
%    \begin{macro}{\HOLOGO@DriverFallback}
%    \begin{macrocode}
\def\HOLOGO@DriverFallback{dvips}
%    \end{macrocode}
%    \end{macro}
%
%    \begin{macro}{\hologoDriverSetup}
%    \begin{macrocode}
\def\hologoDriverSetup{%
  \let\hologoDriver\ltx@undefined
  \HOLOGO@DriverSetup
}
%    \end{macrocode}
%    \end{macro}
%
%    \begin{macro}{\HOLOGO@DriverSetup}
%    \begin{macrocode}
\def\HOLOGO@DriverSetup#1{%
  \kvsetkeys{HoLogoDriver}{#1}%
  \HOLOGO@CheckDriver
  \ltx@ifundefined{hologoDriver}{%
    \begingroup
    \edef\x{\endgroup
      \noexpand\kvsetkeys{HoLogoDriver}{\HOLOGO@DriverFallback}%
    }\x
  }{}%
  \@PackageInfoNoLine{hologo}{Using driver `\hologoDriver'}%
}
%    \end{macrocode}
%    \end{macro}
%
%    \begin{macro}{\HOLOGO@CheckDriver}
%    \begin{macrocode}
\def\HOLOGO@CheckDriver{%
  \ifpdf
    \def\hologoDriver{pdftex}%
    \let\HOLOGO@pdfliteral\pdfliteral
    \ifluatex
      \ifx\pdfextension\@undefined\else
        \protected\def\pdfliteral{\pdfextension literal}%
        \let\HOLOGO@pdfliteral\pdfliteral
      \fi
      \ltx@IfUndefined{HOLOGO@pdfliteral}{%
        \ifnum\luatexversion<36 %
        \else
          \begingroup
            \let\HOLOGO@temp\endgroup
            \ifcase0%
                \directlua{%
                  if tex.enableprimitives then %
                    tex.enableprimitives('HOLOGO@', {'pdfliteral'})%
                  else %
                    tex.print('1')%
                  end%
                }%
                \ifx\HOLOGO@pdfliteral\@undefined 1\fi%
                \relax%
              \endgroup
              \let\HOLOGO@temp\relax
              \global\let\HOLOGO@pdfliteral\HOLOGO@pdfliteral
            \fi%
          \HOLOGO@temp
        \fi
      }{}%
    \fi
    \ltx@IfUndefined{HOLOGO@pdfliteral}{%
      \@PackageWarningNoLine{hologo}{%
        Cannot find \string\pdfliteral
      }%
    }{}%
  \else
    \ifxetex
      \def\hologoDriver{xetex}%
    \else
      \ifvtex
        \def\hologoDriver{vtex}%
      \fi
    \fi
  \fi
}
%    \end{macrocode}
%    \end{macro}
%
%    \begin{macro}{\HOLOGO@WarningUnsupportedDriver}
%    \begin{macrocode}
\def\HOLOGO@WarningUnsupportedDriver#1{%
  \@PackageWarningNoLine{hologo}{%
    Logo `#1' needs driver specific macros,\MessageBreak
    but driver `\hologoDriver' is not supported.\MessageBreak
    Use a different driver or\MessageBreak
    load package `graphics' or `pgf'%
  }%
}
%    \end{macrocode}
%    \end{macro}
%
% \subsubsection{Reflect box macros}
%
%    Skip driver part if not needed.
%    \begin{macrocode}
\ltx@IfUndefined{reflectbox}{}{%
  \ltx@IfUndefined{rotatebox}{}{%
    \HOLOGO@AtEnd
  }%
}
\ltx@IfUndefined{pgftext}{}{%
  \HOLOGO@AtEnd
}
\ltx@IfUndefined{psscalebox}{}{%
  \HOLOGO@AtEnd
}
%    \end{macrocode}
%
%    \begin{macrocode}
\def\HOLOGO@temp{LaTeX2e}
\ifx\fmtname\HOLOGO@temp
  \RequirePackage{kvoptions}[2011/06/30]%
  \ProcessKeyvalOptions{HoLogoDriver}%
\fi
\HOLOGO@DriverSetup{}
%    \end{macrocode}
%
%    \begin{macro}{\HOLOGO@ReflectBox}
%    \begin{macrocode}
\def\HOLOGO@ReflectBox#1{%
  \begingroup
    \setbox\ltx@zero\hbox{\begingroup#1\endgroup}%
    \setbox\ltx@two\hbox{%
      \kern\wd\ltx@zero
      \csname HOLOGO@ScaleBox@\hologoDriver\endcsname{-1}{1}{%
        \hbox to 0pt{\copy\ltx@zero\hss}%
      }%
    }%
    \wd\ltx@two=\wd\ltx@zero
    \box\ltx@two
  \endgroup
}
%    \end{macrocode}
%    \end{macro}
%
%    \begin{macro}{\HOLOGO@PointReflectBox}
%    \begin{macrocode}
\def\HOLOGO@PointReflectBox#1{%
  \begingroup
    \setbox\ltx@zero\hbox{\begingroup#1\endgroup}%
    \setbox\ltx@two\hbox{%
      \kern\wd\ltx@zero
      \raise\ht\ltx@zero\hbox{%
        \csname HOLOGO@ScaleBox@\hologoDriver\endcsname{-1}{-1}{%
          \hbox to 0pt{\copy\ltx@zero\hss}%
        }%
      }%
    }%
    \wd\ltx@two=\wd\ltx@zero
    \box\ltx@two
  \endgroup
}
%    \end{macrocode}
%    \end{macro}
%
%    We must define all variants because of dynamic driver setup.
%    \begin{macrocode}
\def\HOLOGO@temp#1#2{#2}
%    \end{macrocode}
%
%    \begin{macro}{\HOLOGO@ScaleBox@pdftex}
%    \begin{macrocode}
\HOLOGO@temp{pdftex}{%
  \def\HOLOGO@ScaleBox@pdftex#1#2#3{%
    \HOLOGO@pdfliteral{%
      q #1 0 0 #2 0 0 cm%
    }%
    #3%
    \HOLOGO@pdfliteral{%
      Q%
    }%
  }%
}
%    \end{macrocode}
%    \end{macro}
%    \begin{macro}{\HOLOGO@ScaleBox@dvips}
%    \begin{macrocode}
\HOLOGO@temp{dvips}{%
  \def\HOLOGO@ScaleBox@dvips#1#2#3{%
    \special{ps:%
      gsave %
      currentpoint %
      currentpoint translate %
      #1 #2 scale %
      neg exch neg exch translate%
    }%
    #3%
    \special{ps:%
      currentpoint %
      grestore %
      moveto%
    }%
  }%
}
%    \end{macrocode}
%    \end{macro}
%    \begin{macro}{\HOLOGO@ScaleBox@dvipdfm}
%    \begin{macrocode}
\HOLOGO@temp{dvipdfm}{%
  \let\HOLOGO@ScaleBox@dvipdfm\HOLOGO@ScaleBox@dvips
}
%    \end{macrocode}
%    \end{macro}
%    Since \hologo{XeTeX} v0.6.
%    \begin{macro}{\HOLOGO@ScaleBox@xetex}
%    \begin{macrocode}
\HOLOGO@temp{xetex}{%
  \def\HOLOGO@ScaleBox@xetex#1#2#3{%
    \special{x:gsave}%
    \special{x:scale #1 #2}%
    #3%
    \special{x:grestore}%
  }%
}
%    \end{macrocode}
%    \end{macro}
%    \begin{macro}{\HOLOGO@ScaleBox@vtex}
%    \begin{macrocode}
\HOLOGO@temp{vtex}{%
  \def\HOLOGO@ScaleBox@vtex#1#2#3{%
    \special{r(#1,0,0,#2,0,0}%
    #3%
    \special{r)}%
  }%
}
%    \end{macrocode}
%    \end{macro}
%
%    \begin{macrocode}
\HOLOGO@AtEnd%
%</package>
%    \end{macrocode}
%
% \section{Test}
%
% \subsection{Catcode checks for loading}
%
%    \begin{macrocode}
%<*test1>
%    \end{macrocode}
%    \begin{macrocode}
\catcode`\{=1 %
\catcode`\}=2 %
\catcode`\#=6 %
\catcode`\@=11 %
\expandafter\ifx\csname count@\endcsname\relax
  \countdef\count@=255 %
\fi
\expandafter\ifx\csname @gobble\endcsname\relax
  \long\def\@gobble#1{}%
\fi
\expandafter\ifx\csname @firstofone\endcsname\relax
  \long\def\@firstofone#1{#1}%
\fi
\expandafter\ifx\csname loop\endcsname\relax
  \expandafter\@firstofone
\else
  \expandafter\@gobble
\fi
{%
  \def\loop#1\repeat{%
    \def\body{#1}%
    \iterate
  }%
  \def\iterate{%
    \body
      \let\next\iterate
    \else
      \let\next\relax
    \fi
    \next
  }%
  \let\repeat=\fi
}%
\def\RestoreCatcodes{}
\count@=0 %
\loop
  \edef\RestoreCatcodes{%
    \RestoreCatcodes
    \catcode\the\count@=\the\catcode\count@\relax
  }%
\ifnum\count@<255 %
  \advance\count@ 1 %
\repeat

\def\RangeCatcodeInvalid#1#2{%
  \count@=#1\relax
  \loop
    \catcode\count@=15 %
  \ifnum\count@<#2\relax
    \advance\count@ 1 %
  \repeat
}
\def\RangeCatcodeCheck#1#2#3{%
  \count@=#1\relax
  \loop
    \ifnum#3=\catcode\count@
    \else
      \errmessage{%
        Character \the\count@\space
        with wrong catcode \the\catcode\count@\space
        instead of \number#3%
      }%
    \fi
  \ifnum\count@<#2\relax
    \advance\count@ 1 %
  \repeat
}
\def\space{ }
\expandafter\ifx\csname LoadCommand\endcsname\relax
  \def\LoadCommand{\input hologo.sty\relax}%
\fi
\def\Test{%
  \RangeCatcodeInvalid{0}{47}%
  \RangeCatcodeInvalid{58}{64}%
  \RangeCatcodeInvalid{91}{96}%
  \RangeCatcodeInvalid{123}{255}%
  \catcode`\@=12 %
  \catcode`\\=0 %
  \catcode`\%=14 %
  \LoadCommand
  \RangeCatcodeCheck{0}{36}{15}%
  \RangeCatcodeCheck{37}{37}{14}%
  \RangeCatcodeCheck{38}{47}{15}%
  \RangeCatcodeCheck{48}{57}{12}%
  \RangeCatcodeCheck{58}{63}{15}%
  \RangeCatcodeCheck{64}{64}{12}%
  \RangeCatcodeCheck{65}{90}{11}%
  \RangeCatcodeCheck{91}{91}{15}%
  \RangeCatcodeCheck{92}{92}{0}%
  \RangeCatcodeCheck{93}{96}{15}%
  \RangeCatcodeCheck{97}{122}{11}%
  \RangeCatcodeCheck{123}{255}{15}%
  \RestoreCatcodes
}
\Test
\csname @@end\endcsname
\end
%    \end{macrocode}
%    \begin{macrocode}
%</test1>
%    \end{macrocode}
%
% \subsection{Spacefactor}
%
%    The space factor must be 1000 after a logo. If it is greater 1000
%    then the following space is a space after a sentence closing point.
%    If the space factor is smaller 1000 then an immediate following
%    dot is interpreted as abbreviation, not sentence closing point.
%
%    \begin{macrocode}
%<*test-spacefactor>
\NeedsTeXFormat{LaTeX2e}
\documentclass{article}
\usepackage{hologo}[2016/05/12]
\usepackage{kvsetkeys}
\usepackage{qstest}
\IncludeTests{*}
\LogTests{log}{*}{*}
\begin{document}
\begin{qstest}{spacefactor}{spacefactor}
\newcommand*{\Test}[1]{%
  \sbox0{%
    \hologo{#1}%
    \Expect*{1000 (#1)}*{\the\spacefactor\space(#1)}%
  }%
}%
\makeatletter
\def\TestList{}
\def\hologoEntry#1#2#3{%
  \edef\TestList{%
    \ifx\TestList\@empty
    \else
      \TestList,%
    \fi
    #1%
    \ifx\\#2\\%
    \else
      ={variant=#2}%
    \fi
  }%
}
\hologoList
\expandafter\kv@parse@normalized\expandafter{%
  \TestList
}{%
  \begingroup
    \let\@logo=\kv@key
    \ifx\kv@value\relax
    \else
      \expandafter\hologoLogoSetup\expandafter\@logo\expandafter{%
        \kv@value
      }%
    \fi
    \Test\@logo
  \endgroup
  \@gobbletwo
}
\end{qstest}
\end{document}
%</test-spacefactor>
%    \end{macrocode}
%
% \subsection{Complete list}
%
%    \begin{macrocode}
%<*test-list>
\NeedsTeXFormat{LaTeX2e}
\documentclass[12pt,a4paper]{article}
\usepackage{hologo}[2016/05/12]
\usepackage[T1]{fontenc}
\usepackage{lmodern}
\usepackage{parskip}
\usepackage[unicode]{hyperref}[2011/09/28]
\usepackage{bookmark}[2011/09/19]
\bookmarksetup{%
  numbered,%
  open,%
  openlevel=2,%
}
\renewcommand*{\contentsname}{List of logos}
\begin{document}
\tableofcontents
\def\TestFont#1#2#3#4#5#6{%
  \begingroup
    \usefont{#3}{#4}{#5}{#6}%
    \HologoVariant{#1}{#2}/\hologoVariant{#1}{#2}%
    \quad
    \begingroup\scriptsize\hologoVariant{#1}{#2}\endgroup
    \quad
  \endgroup
  (#3/#4/#5/#6)%
  \par
}
\makeatletter
\def\hologoEntry#1#2#3{%
  \section{%
    \HologoVariant{#1}{#2}/\hologoVariant{#1}{#2} %
    {[#1\ifx\\#2\\\else\space(#2)\fi]}% hash-ok
  }% braces around [] because of bug in tex4ht
  \begingroup
    \hypersetup{unicode=false}%
    \bookmark[%
      dest=\@currentHref,%
      rellevel=1,%
      keeplevel,%
    ]{%
      \HologoVariant{#1}{#2}/\hologoVariant{#1}{#2} %
      (PDFDocEncoding)%
    }%
  \endgroup
  \TestFont{#1}{#2}{OT1}{cmr}{m}{n}%
  \TestFont{#1}{#2}{OT1}{cmss}{m}{n}%
  \TestFont{#1}{#2}{OT1}{cmr}{b}{n}%
  \TestFont{#1}{#2}{OT1}{cmr}{m}{it}%
  \TestFont{#1}{#2}{OT1}{cmtt}{m}{n}%
  \TestFont{#1}{#2}{T1}{lmr}{m}{n}%
  \TestFont{#1}{#2}{T1}{lmss}{m}{n}%
  \TestFont{#1}{#2}{T1}{lmr}{b}{n}%
  \TestFont{#1}{#2}{T1}{lmr}{m}{it}%
  \TestFont{#1}{#2}{T1}{lmtt}{m}{n}%
  \TestFont{#1}{#2}{T1}{lmvtt}{m}{n}%
  \TestFont{#1}{#2}{T1}{qtm}{m}{n}%
  \TestFont{#1}{#2}{T1}{qhv}{m}{n}%
  \TestFont{#1}{#2}{T1}{qtm}{b}{n}%
  \TestFont{#1}{#2}{T1}{qtm}{m}{it}%
  \TestFont{#1}{#2}{T1}{qcr}{m}{n}%
  \newpage
}
\makeatother
\hologoList
\end{document}
%</test-list>
%    \end{macrocode}
%
% \section{Installation}
%
% \subsection{Download}
%
% \paragraph{Package.} This package is available on
% CTAN\footnote{\url{ftp://ftp.ctan.org/tex-archive/}}:
% \begin{description}
% \item[\CTAN{macros/latex/contrib/oberdiek/hologo.dtx}] The source file.
% \item[\CTAN{macros/latex/contrib/oberdiek/hologo.pdf}] Documentation.
% \end{description}
%
%
% \paragraph{Bundle.} All the packages of the bundle `oberdiek'
% are also available in a TDS compliant ZIP archive. There
% the packages are already unpacked and the documentation files
% are generated. The files and directories obey the TDS standard.
% \begin{description}
% \item[\CTAN{install/macros/latex/contrib/oberdiek.tds.zip}]
% \end{description}
% \emph{TDS} refers to the standard ``A Directory Structure
% for \TeX\ Files'' (\CTAN{tds/tds.pdf}). Directories
% with \xfile{texmf} in their name are usually organized this way.
%
% \subsection{Bundle installation}
%
% \paragraph{Unpacking.} Unpack the \xfile{oberdiek.tds.zip} in the
% TDS tree (also known as \xfile{texmf} tree) of your choice.
% Example (linux):
% \begin{quote}
%   |unzip oberdiek.tds.zip -d ~/texmf|
% \end{quote}
%
% \paragraph{Script installation.}
% Check the directory \xfile{TDS:scripts/oberdiek/} for
% scripts that need further installation steps.
% Package \xpackage{attachfile2} comes with the Perl script
% \xfile{pdfatfi.pl} that should be installed in such a way
% that it can be called as \texttt{pdfatfi}.
% Example (linux):
% \begin{quote}
%   |chmod +x scripts/oberdiek/pdfatfi.pl|\\
%   |cp scripts/oberdiek/pdfatfi.pl /usr/local/bin/|
% \end{quote}
%
% \subsection{Package installation}
%
% \paragraph{Unpacking.} The \xfile{.dtx} file is a self-extracting
% \docstrip\ archive. The files are extracted by running the
% \xfile{.dtx} through \plainTeX:
% \begin{quote}
%   \verb|tex hologo.dtx|
% \end{quote}
%
% \paragraph{TDS.} Now the different files must be moved into
% the different directories in your installation TDS tree
% (also known as \xfile{texmf} tree):
% \begin{quote}
% \def\t{^^A
% \begin{tabular}{@{}>{\ttfamily}l@{ $\rightarrow$ }>{\ttfamily}l@{}}
%   hologo.sty & tex/generic/oberdiek/hologo.sty\\
%   hologo.pdf & doc/latex/oberdiek/hologo.pdf\\
%   example/hologo-example.tex & doc/latex/oberdiek/example/hologo-example.tex\\
%   test/hologo-test1.tex & doc/latex/oberdiek/test/hologo-test1.tex\\
%   test/hologo-test-spacefactor.tex & doc/latex/oberdiek/test/hologo-test-spacefactor.tex\\
%   test/hologo-test-list.tex & doc/latex/oberdiek/test/hologo-test-list.tex\\
%   hologo.dtx & source/latex/oberdiek/hologo.dtx\\
% \end{tabular}^^A
% }^^A
% \sbox0{\t}^^A
% \ifdim\wd0>\linewidth
%   \begingroup
%     \advance\linewidth by\leftmargin
%     \advance\linewidth by\rightmargin
%   \edef\x{\endgroup
%     \def\noexpand\lw{\the\linewidth}^^A
%   }\x
%   \def\lwbox{^^A
%     \leavevmode
%     \hbox to \linewidth{^^A
%       \kern-\leftmargin\relax
%       \hss
%       \usebox0
%       \hss
%       \kern-\rightmargin\relax
%     }^^A
%   }^^A
%   \ifdim\wd0>\lw
%     \sbox0{\small\t}^^A
%     \ifdim\wd0>\linewidth
%       \ifdim\wd0>\lw
%         \sbox0{\footnotesize\t}^^A
%         \ifdim\wd0>\linewidth
%           \ifdim\wd0>\lw
%             \sbox0{\scriptsize\t}^^A
%             \ifdim\wd0>\linewidth
%               \ifdim\wd0>\lw
%                 \sbox0{\tiny\t}^^A
%                 \ifdim\wd0>\linewidth
%                   \lwbox
%                 \else
%                   \usebox0
%                 \fi
%               \else
%                 \lwbox
%               \fi
%             \else
%               \usebox0
%             \fi
%           \else
%             \lwbox
%           \fi
%         \else
%           \usebox0
%         \fi
%       \else
%         \lwbox
%       \fi
%     \else
%       \usebox0
%     \fi
%   \else
%     \lwbox
%   \fi
% \else
%   \usebox0
% \fi
% \end{quote}
% If you have a \xfile{docstrip.cfg} that configures and enables \docstrip's
% TDS installing feature, then some files can already be in the right
% place, see the documentation of \docstrip.
%
% \subsection{Refresh file name databases}
%
% If your \TeX~distribution
% (\teTeX, \mikTeX, \dots) relies on file name databases, you must refresh
% these. For example, \teTeX\ users run \verb|texhash| or
% \verb|mktexlsr|.
%
% \subsection{Some details for the interested}
%
% \paragraph{Attached source.}
%
% The PDF documentation on CTAN also includes the
% \xfile{.dtx} source file. It can be extracted by
% AcrobatReader 6 or higher. Another option is \textsf{pdftk},
% e.g. unpack the file into the current directory:
% \begin{quote}
%   \verb|pdftk hologo.pdf unpack_files output .|
% \end{quote}
%
% \paragraph{Unpacking with \LaTeX.}
% The \xfile{.dtx} chooses its action depending on the format:
% \begin{description}
% \item[\plainTeX:] Run \docstrip\ and extract the files.
% \item[\LaTeX:] Generate the documentation.
% \end{description}
% If you insist on using \LaTeX\ for \docstrip\ (really,
% \docstrip\ does not need \LaTeX), then inform the autodetect routine
% about your intention:
% \begin{quote}
%   \verb|latex \let\install=y\input{hologo.dtx}|
% \end{quote}
% Do not forget to quote the argument according to the demands
% of your shell.
%
% \paragraph{Generating the documentation.}
% You can use both the \xfile{.dtx} or the \xfile{.drv} to generate
% the documentation. The process can be configured by the
% configuration file \xfile{ltxdoc.cfg}. For instance, put this
% line into this file, if you want to have A4 as paper format:
% \begin{quote}
%   \verb|\PassOptionsToClass{a4paper}{article}|
% \end{quote}
% An example follows how to generate the
% documentation with pdf\LaTeX:
% \begin{quote}
%\begin{verbatim}
%pdflatex hologo.dtx
%makeindex -s gind.ist hologo.idx
%pdflatex hologo.dtx
%makeindex -s gind.ist hologo.idx
%pdflatex hologo.dtx
%\end{verbatim}
% \end{quote}
%
% \section{Catalogue}
%
% The following XML file can be used as source for the
% \href{http://mirror.ctan.org/help/Catalogue/catalogue.html}{\TeX\ Catalogue}.
% The elements \texttt{caption} and \texttt{description} are imported
% from the original XML file from the Catalogue.
% The name of the XML file in the Catalogue is \xfile{hologo.xml}.
%    \begin{macrocode}
%<*catalogue>
<?xml version='1.0' encoding='us-ascii'?>
<!DOCTYPE entry SYSTEM 'catalogue.dtd'>
<entry datestamp='$Date$' modifier='$Author$' id='hologo'>
  <name>hologo</name>
  <caption>A collection of logos with bookmark support.</caption>
  <authorref id='auth:oberdiek'/>
  <copyright owner='Heiko Oberdiek' year='2010-2012'/>
  <license type='lppl1.3'/>
  <version number='1.10'/>
  <description>
    The package defines a single command <tt>\hologo</tt>, whose
    argument is the usual case-confused ASCII version of the logo.
    The command is bookmark-enabled, so that every logo becomes
    available in bookmarks without further work.
    <p/>
    The package is part of the <xref refid='oberdiek'>oberdiek</xref>
    bundle.
  </description>
  <documentation details='Package documentation'
      href='ctan:/macros/latex/contrib/oberdiek/hologo.pdf'/>
  <ctan file='true' path='/macros/latex/contrib/oberdiek/hologo.dtx'/>
  <miktex location='oberdiek'/>
  <texlive location='oberdiek'/>
  <install path='/macros/latex/contrib/oberdiek/oberdiek.tds.zip'/>
</entry>
%</catalogue>
%    \end{macrocode}
%
% \begin{thebibliography}{9}
% \raggedright
%
% \bibitem{btxdoc}
% Oren Patashnik,
% \textit{\hologo{BibTeX}ing},
% 1988-02-08.\\
% \CTAN{biblio/bibtex/base/}
%
% \bibitem{dtklogos}
% Gerd Neugebauer, DANTE,
% \textit{Package \xpackage{dtklogos}},
% 2011-04-25.\\
% \CTAN{usergrps/dante/dtk/dtklogos.sty}
%
% \bibitem{etexman}
% The \hologo{NTS} Team,
% \textit{The \hologo{eTeX} manual},
% 1998-02.\\
% \CTAN{systems/e-tex/v2/doc/}
%
% \bibitem{ExTeX-FAQ}
% The \hologo{ExTeX} group,
% \textit{\hologo{ExTeX}: FAQ -- How is \hologo{ExTeX} typeset?},
% 2007-04-14.\\
% \url{http://www.extex.org/documentation/faq.html}
%
% \bibitem{LyX}
% %@MISC{ LyX,
% %  title = {{LyX 2.0.0 -- The Document Processor [Computer software and manual]}},
% %  author = {{The LyX Team}},
% %  howpublished = {Internet: http://www.lyx.org},
% %  year = {2011-05-08},
% %  note = {Retrieved May 10, 2011, from http://www.lyx.org},
% %  url = {http://www.lyx.org/}
% %}
% The \hologo{LyX} Team,
% \textit{\hologo{LyX} -- The Document Processor},
% 2011-05-08.\\
% \url{http://www.lyx.org/}
%
% \bibitem{OzTeX}
% Andrew Trevorrow,
% \hologo{OzTeX} FAQ: What is the correct way to typeset ``\hologo{OzTeX}''?,
% 2011-09-15 (visited).
% \url{http://www.trevorrow.com/oztex/ozfaq.html#oztex-logo}
%
% \bibitem{PiCTeX}
% Michael Wichura,
% \textit{The \hologo{PiCTeX} macro package},
% 1987-09-21.
% \CTAN{graphics/pictex/}
%
% \bibitem{scrlogo}
% Markus Kohm,
% \textit{\hologo{KOMAScript} Datei \xfile{scrlogo.dtx}},
% 2009-01-30.\\
% \CTAN{install/macros/latex/contrib/komascript.tds.zip}
%
% \end{thebibliography}
%
% \begin{History}
%   \begin{Version}{2010/04/08 v1.0}
%   \item
%     The first version.
%   \end{Version}
%   \begin{Version}{2010/04/16 v1.1}
%   \item
%     \cs{Hologo} added for support of logos at start of a sentence.
%   \item
%     \cs{hologoSetup} and \cs{hologoLogoSetup} added.
%   \item
%     Options \xoption{break}, \xoption{hyphenbreak}, \xoption{spacebreak}
%     added.
%   \item
%     Variant support added by option \xoption{variant}.
%   \end{Version}
%   \begin{Version}{2010/04/24 v1.2}
%   \item
%     \hologo{LaTeX3} added.
%   \item
%     \hologo{VTeX} added.
%   \end{Version}
%   \begin{Version}{2010/11/21 v1.3}
%   \item
%     \hologo{iniTeX}, \hologo{virTeX} added.
%   \end{Version}
%   \begin{Version}{2011/03/25 v1.4}
%   \item
%     \hologo{ConTeXt} with variants added.
%   \item
%     Option \xoption{discretionarybreak} added as refinement for
%     option \xoption{break}.
%   \end{Version}
%   \begin{Version}{2011/04/21 v1.5}
%   \item
%     Wrong TDS directory for test files fixed.
%   \end{Version}
%   \begin{Version}{2011/10/01 v1.6}
%   \item
%     Support for package \xpackage{tex4ht} added.
%   \item
%     Support for \cs{csname} added if \cs{ifincsname} is available.
%   \item
%     New logos:
%     \hologo{(La)TeX},
%     \hologo{biber},
%     \hologo{BibTeX} (\xoption{sc}, \xoption{sf}),
%     \hologo{emTeX},
%     \hologo{ExTeX},
%     \hologo{KOMAScript},
%     \hologo{La},
%     \hologo{LyX},
%     \hologo{MiKTeX},
%     \hologo{NTS},
%     \hologo{OzMF},
%     \hologo{OzMP},
%     \hologo{OzTeX},
%     \hologo{OzTtH},
%     \hologo{PCTeX},
%     \hologo{PiC},
%     \hologo{PiCTeX},
%     \hologo{METAFONT},
%     \hologo{MetaFun},
%     \hologo{METAPOST},
%     \hologo{MetaPost},
%     \hologo{SLiTeX} (\xoption{lift}, \xoption{narrow}, \xoption{simple}),
%     \hologo{SliTeX} (\xoption{narrow}, \xoption{simple}, \xoption{lift}),
%     \hologo{teTeX}.
%   \item
%     Fixes:
%     \hologo{iniTeX},
%     \hologo{pdfLaTeX},
%     \hologo{pdfTeX},
%     \hologo{virTeX}.
%   \item
%     \cs{hologoFontSetup} and \cs{hologoLogoFontSetup} added.
%   \item
%     \cs{hologoVariant} and \cs{HologoVariant} added.
%   \end{Version}
%   \begin{Version}{2011/11/22 v1.7}
%   \item
%     New logos:
%     \hologo{BibTeX8},
%     \hologo{LaTeXML},
%     \hologo{SageTeX},
%     \hologo{TeX4ht},
%     \hologo{TTH}.
%   \item
%     \hologo{Xe} and friends: Driver stuff fixed.
%   \item
%     \hologo{Xe} and friends: Support for italic added.
%   \item
%     \hologo{Xe} and friends: Package support for \xpackage{pgf}
%     and \xpackage{pstricks} added.
%   \end{Version}
%   \begin{Version}{2011/11/29 v1.8}
%   \item
%     New logos:
%     \hologo{HanTheThanh}.
%   \end{Version}
%   \begin{Version}{2011/12/21 v1.9}
%   \item
%     Patch for package \xpackage{ifxetex} added for the case that
%     \cs{newif} is undefined in \hologo{iniTeX}.
%   \item
%     Some fixes for \hologo{iniTeX}.
%   \end{Version}
%   \begin{Version}{2012/04/26 v1.10}
%   \item
%     Fix in bookmark version of logo ``\hologo{HanTheThanh}''.
%   \end{Version}
%   \begin{Version}{2016/05/12 v1.11}
%   \item
%     Update HOLOGO@IfCharExists (previously in texlive)
%   \item define pdfliteral in current luatex.
%   \end{Version}
% \end{History}
%
% \PrintIndex
%
% \Finale
\endinput
%
        \else
          \input hologo.cfg\relax
        \fi
      \else
        \@PackageInfoNoLine{hologo}{%
          Empty configuration file `hologo.cfg' ignored%
        }%
      \fi
    \fi
  }%
}
%    \end{macrocode}
%
%    \begin{macrocode}
\def\HOLOGO@temp#1#2{%
  \kv@define@key{HoLogoDriver}{#1}[]{%
    \begingroup
      \def\HOLOGO@temp{##1}%
      \ltx@onelevel@sanitize\HOLOGO@temp
      \ifx\HOLOGO@temp\ltx@empty
      \else
        \@PackageError{hologo}{%
          Value (\HOLOGO@temp) not permitted for option `#1'%
        }%
        \@ehc
      \fi
    \endgroup
    \def\hologoDriver{#2}%
  }%
}%
\def\HOLOGO@@temp#1#2{%
  \ifx\kv@value\relax
    \HOLOGO@temp{#1}{#1}%
  \else
    \HOLOGO@temp{#1}{#2}%
  \fi
}%
\kv@parse@normalized{%
  pdftex,%
  luatex=pdftex,%
  dvipdfm,%
  dvipdfmx=dvipdfm,%
  dvips,%
  dvipsone=dvips,%
  xdvi=dvips,%
  xetex,%
  vtex,%
}\HOLOGO@@temp
%    \end{macrocode}
%
%    \begin{macrocode}
\kv@define@key{HoLogoDriver}{driverfallback}{%
  \def\HOLOGO@DriverFallback{#1}%
}
%    \end{macrocode}
%
%    \begin{macro}{\HOLOGO@DriverFallback}
%    \begin{macrocode}
\def\HOLOGO@DriverFallback{dvips}
%    \end{macrocode}
%    \end{macro}
%
%    \begin{macro}{\hologoDriverSetup}
%    \begin{macrocode}
\def\hologoDriverSetup{%
  \let\hologoDriver\ltx@undefined
  \HOLOGO@DriverSetup
}
%    \end{macrocode}
%    \end{macro}
%
%    \begin{macro}{\HOLOGO@DriverSetup}
%    \begin{macrocode}
\def\HOLOGO@DriverSetup#1{%
  \kvsetkeys{HoLogoDriver}{#1}%
  \HOLOGO@CheckDriver
  \ltx@ifundefined{hologoDriver}{%
    \begingroup
    \edef\x{\endgroup
      \noexpand\kvsetkeys{HoLogoDriver}{\HOLOGO@DriverFallback}%
    }\x
  }{}%
  \@PackageInfoNoLine{hologo}{Using driver `\hologoDriver'}%
}
%    \end{macrocode}
%    \end{macro}
%
%    \begin{macro}{\HOLOGO@CheckDriver}
%    \begin{macrocode}
\def\HOLOGO@CheckDriver{%
  \ifpdf
    \def\hologoDriver{pdftex}%
    \let\HOLOGO@pdfliteral\pdfliteral
    \ifluatex
      \ifx\pdfextension\@undefined\else
        \protected\def\pdfliteral{\pdfextension literal}%
        \let\HOLOGO@pdfliteral\pdfliteral
      \fi
      \ltx@IfUndefined{HOLOGO@pdfliteral}{%
        \ifnum\luatexversion<36 %
        \else
          \begingroup
            \let\HOLOGO@temp\endgroup
            \ifcase0%
                \directlua{%
                  if tex.enableprimitives then %
                    tex.enableprimitives('HOLOGO@', {'pdfliteral'})%
                  else %
                    tex.print('1')%
                  end%
                }%
                \ifx\HOLOGO@pdfliteral\@undefined 1\fi%
                \relax%
              \endgroup
              \let\HOLOGO@temp\relax
              \global\let\HOLOGO@pdfliteral\HOLOGO@pdfliteral
            \fi%
          \HOLOGO@temp
        \fi
      }{}%
    \fi
    \ltx@IfUndefined{HOLOGO@pdfliteral}{%
      \@PackageWarningNoLine{hologo}{%
        Cannot find \string\pdfliteral
      }%
    }{}%
  \else
    \ifxetex
      \def\hologoDriver{xetex}%
    \else
      \ifvtex
        \def\hologoDriver{vtex}%
      \fi
    \fi
  \fi
}
%    \end{macrocode}
%    \end{macro}
%
%    \begin{macro}{\HOLOGO@WarningUnsupportedDriver}
%    \begin{macrocode}
\def\HOLOGO@WarningUnsupportedDriver#1{%
  \@PackageWarningNoLine{hologo}{%
    Logo `#1' needs driver specific macros,\MessageBreak
    but driver `\hologoDriver' is not supported.\MessageBreak
    Use a different driver or\MessageBreak
    load package `graphics' or `pgf'%
  }%
}
%    \end{macrocode}
%    \end{macro}
%
% \subsubsection{Reflect box macros}
%
%    Skip driver part if not needed.
%    \begin{macrocode}
\ltx@IfUndefined{reflectbox}{}{%
  \ltx@IfUndefined{rotatebox}{}{%
    \HOLOGO@AtEnd
  }%
}
\ltx@IfUndefined{pgftext}{}{%
  \HOLOGO@AtEnd
}
\ltx@IfUndefined{psscalebox}{}{%
  \HOLOGO@AtEnd
}
%    \end{macrocode}
%
%    \begin{macrocode}
\def\HOLOGO@temp{LaTeX2e}
\ifx\fmtname\HOLOGO@temp
  \RequirePackage{kvoptions}[2011/06/30]%
  \ProcessKeyvalOptions{HoLogoDriver}%
\fi
\HOLOGO@DriverSetup{}
%    \end{macrocode}
%
%    \begin{macro}{\HOLOGO@ReflectBox}
%    \begin{macrocode}
\def\HOLOGO@ReflectBox#1{%
  \begingroup
    \setbox\ltx@zero\hbox{\begingroup#1\endgroup}%
    \setbox\ltx@two\hbox{%
      \kern\wd\ltx@zero
      \csname HOLOGO@ScaleBox@\hologoDriver\endcsname{-1}{1}{%
        \hbox to 0pt{\copy\ltx@zero\hss}%
      }%
    }%
    \wd\ltx@two=\wd\ltx@zero
    \box\ltx@two
  \endgroup
}
%    \end{macrocode}
%    \end{macro}
%
%    \begin{macro}{\HOLOGO@PointReflectBox}
%    \begin{macrocode}
\def\HOLOGO@PointReflectBox#1{%
  \begingroup
    \setbox\ltx@zero\hbox{\begingroup#1\endgroup}%
    \setbox\ltx@two\hbox{%
      \kern\wd\ltx@zero
      \raise\ht\ltx@zero\hbox{%
        \csname HOLOGO@ScaleBox@\hologoDriver\endcsname{-1}{-1}{%
          \hbox to 0pt{\copy\ltx@zero\hss}%
        }%
      }%
    }%
    \wd\ltx@two=\wd\ltx@zero
    \box\ltx@two
  \endgroup
}
%    \end{macrocode}
%    \end{macro}
%
%    We must define all variants because of dynamic driver setup.
%    \begin{macrocode}
\def\HOLOGO@temp#1#2{#2}
%    \end{macrocode}
%
%    \begin{macro}{\HOLOGO@ScaleBox@pdftex}
%    \begin{macrocode}
\HOLOGO@temp{pdftex}{%
  \def\HOLOGO@ScaleBox@pdftex#1#2#3{%
    \HOLOGO@pdfliteral{%
      q #1 0 0 #2 0 0 cm%
    }%
    #3%
    \HOLOGO@pdfliteral{%
      Q%
    }%
  }%
}
%    \end{macrocode}
%    \end{macro}
%    \begin{macro}{\HOLOGO@ScaleBox@dvips}
%    \begin{macrocode}
\HOLOGO@temp{dvips}{%
  \def\HOLOGO@ScaleBox@dvips#1#2#3{%
    \special{ps:%
      gsave %
      currentpoint %
      currentpoint translate %
      #1 #2 scale %
      neg exch neg exch translate%
    }%
    #3%
    \special{ps:%
      currentpoint %
      grestore %
      moveto%
    }%
  }%
}
%    \end{macrocode}
%    \end{macro}
%    \begin{macro}{\HOLOGO@ScaleBox@dvipdfm}
%    \begin{macrocode}
\HOLOGO@temp{dvipdfm}{%
  \let\HOLOGO@ScaleBox@dvipdfm\HOLOGO@ScaleBox@dvips
}
%    \end{macrocode}
%    \end{macro}
%    Since \hologo{XeTeX} v0.6.
%    \begin{macro}{\HOLOGO@ScaleBox@xetex}
%    \begin{macrocode}
\HOLOGO@temp{xetex}{%
  \def\HOLOGO@ScaleBox@xetex#1#2#3{%
    \special{x:gsave}%
    \special{x:scale #1 #2}%
    #3%
    \special{x:grestore}%
  }%
}
%    \end{macrocode}
%    \end{macro}
%    \begin{macro}{\HOLOGO@ScaleBox@vtex}
%    \begin{macrocode}
\HOLOGO@temp{vtex}{%
  \def\HOLOGO@ScaleBox@vtex#1#2#3{%
    \special{r(#1,0,0,#2,0,0}%
    #3%
    \special{r)}%
  }%
}
%    \end{macrocode}
%    \end{macro}
%
%    \begin{macrocode}
\HOLOGO@AtEnd%
%</package>
%    \end{macrocode}
%
% \section{Test}
%
% \subsection{Catcode checks for loading}
%
%    \begin{macrocode}
%<*test1>
%    \end{macrocode}
%    \begin{macrocode}
\catcode`\{=1 %
\catcode`\}=2 %
\catcode`\#=6 %
\catcode`\@=11 %
\expandafter\ifx\csname count@\endcsname\relax
  \countdef\count@=255 %
\fi
\expandafter\ifx\csname @gobble\endcsname\relax
  \long\def\@gobble#1{}%
\fi
\expandafter\ifx\csname @firstofone\endcsname\relax
  \long\def\@firstofone#1{#1}%
\fi
\expandafter\ifx\csname loop\endcsname\relax
  \expandafter\@firstofone
\else
  \expandafter\@gobble
\fi
{%
  \def\loop#1\repeat{%
    \def\body{#1}%
    \iterate
  }%
  \def\iterate{%
    \body
      \let\next\iterate
    \else
      \let\next\relax
    \fi
    \next
  }%
  \let\repeat=\fi
}%
\def\RestoreCatcodes{}
\count@=0 %
\loop
  \edef\RestoreCatcodes{%
    \RestoreCatcodes
    \catcode\the\count@=\the\catcode\count@\relax
  }%
\ifnum\count@<255 %
  \advance\count@ 1 %
\repeat

\def\RangeCatcodeInvalid#1#2{%
  \count@=#1\relax
  \loop
    \catcode\count@=15 %
  \ifnum\count@<#2\relax
    \advance\count@ 1 %
  \repeat
}
\def\RangeCatcodeCheck#1#2#3{%
  \count@=#1\relax
  \loop
    \ifnum#3=\catcode\count@
    \else
      \errmessage{%
        Character \the\count@\space
        with wrong catcode \the\catcode\count@\space
        instead of \number#3%
      }%
    \fi
  \ifnum\count@<#2\relax
    \advance\count@ 1 %
  \repeat
}
\def\space{ }
\expandafter\ifx\csname LoadCommand\endcsname\relax
  \def\LoadCommand{\input hologo.sty\relax}%
\fi
\def\Test{%
  \RangeCatcodeInvalid{0}{47}%
  \RangeCatcodeInvalid{58}{64}%
  \RangeCatcodeInvalid{91}{96}%
  \RangeCatcodeInvalid{123}{255}%
  \catcode`\@=12 %
  \catcode`\\=0 %
  \catcode`\%=14 %
  \LoadCommand
  \RangeCatcodeCheck{0}{36}{15}%
  \RangeCatcodeCheck{37}{37}{14}%
  \RangeCatcodeCheck{38}{47}{15}%
  \RangeCatcodeCheck{48}{57}{12}%
  \RangeCatcodeCheck{58}{63}{15}%
  \RangeCatcodeCheck{64}{64}{12}%
  \RangeCatcodeCheck{65}{90}{11}%
  \RangeCatcodeCheck{91}{91}{15}%
  \RangeCatcodeCheck{92}{92}{0}%
  \RangeCatcodeCheck{93}{96}{15}%
  \RangeCatcodeCheck{97}{122}{11}%
  \RangeCatcodeCheck{123}{255}{15}%
  \RestoreCatcodes
}
\Test
\csname @@end\endcsname
\end
%    \end{macrocode}
%    \begin{macrocode}
%</test1>
%    \end{macrocode}
%
% \subsection{Spacefactor}
%
%    The space factor must be 1000 after a logo. If it is greater 1000
%    then the following space is a space after a sentence closing point.
%    If the space factor is smaller 1000 then an immediate following
%    dot is interpreted as abbreviation, not sentence closing point.
%
%    \begin{macrocode}
%<*test-spacefactor>
\NeedsTeXFormat{LaTeX2e}
\documentclass{article}
\usepackage{hologo}[2016/05/12]
\usepackage{kvsetkeys}
\usepackage{qstest}
\IncludeTests{*}
\LogTests{log}{*}{*}
\begin{document}
\begin{qstest}{spacefactor}{spacefactor}
\newcommand*{\Test}[1]{%
  \sbox0{%
    \hologo{#1}%
    \Expect*{1000 (#1)}*{\the\spacefactor\space(#1)}%
  }%
}%
\makeatletter
\def\TestList{}
\def\hologoEntry#1#2#3{%
  \edef\TestList{%
    \ifx\TestList\@empty
    \else
      \TestList,%
    \fi
    #1%
    \ifx\\#2\\%
    \else
      ={variant=#2}%
    \fi
  }%
}
\hologoList
\expandafter\kv@parse@normalized\expandafter{%
  \TestList
}{%
  \begingroup
    \let\@logo=\kv@key
    \ifx\kv@value\relax
    \else
      \expandafter\hologoLogoSetup\expandafter\@logo\expandafter{%
        \kv@value
      }%
    \fi
    \Test\@logo
  \endgroup
  \@gobbletwo
}
\end{qstest}
\end{document}
%</test-spacefactor>
%    \end{macrocode}
%
% \subsection{Complete list}
%
%    \begin{macrocode}
%<*test-list>
\NeedsTeXFormat{LaTeX2e}
\documentclass[12pt,a4paper]{article}
\usepackage{hologo}[2016/05/12]
\usepackage[T1]{fontenc}
\usepackage{lmodern}
\usepackage{parskip}
\usepackage[unicode]{hyperref}[2011/09/28]
\usepackage{bookmark}[2011/09/19]
\bookmarksetup{%
  numbered,%
  open,%
  openlevel=2,%
}
\renewcommand*{\contentsname}{List of logos}
\begin{document}
\tableofcontents
\def\TestFont#1#2#3#4#5#6{%
  \begingroup
    \usefont{#3}{#4}{#5}{#6}%
    \HologoVariant{#1}{#2}/\hologoVariant{#1}{#2}%
    \quad
    \begingroup\scriptsize\hologoVariant{#1}{#2}\endgroup
    \quad
  \endgroup
  (#3/#4/#5/#6)%
  \par
}
\makeatletter
\def\hologoEntry#1#2#3{%
  \section{%
    \HologoVariant{#1}{#2}/\hologoVariant{#1}{#2} %
    {[#1\ifx\\#2\\\else\space(#2)\fi]}% hash-ok
  }% braces around [] because of bug in tex4ht
  \begingroup
    \hypersetup{unicode=false}%
    \bookmark[%
      dest=\@currentHref,%
      rellevel=1,%
      keeplevel,%
    ]{%
      \HologoVariant{#1}{#2}/\hologoVariant{#1}{#2} %
      (PDFDocEncoding)%
    }%
  \endgroup
  \TestFont{#1}{#2}{OT1}{cmr}{m}{n}%
  \TestFont{#1}{#2}{OT1}{cmss}{m}{n}%
  \TestFont{#1}{#2}{OT1}{cmr}{b}{n}%
  \TestFont{#1}{#2}{OT1}{cmr}{m}{it}%
  \TestFont{#1}{#2}{OT1}{cmtt}{m}{n}%
  \TestFont{#1}{#2}{T1}{lmr}{m}{n}%
  \TestFont{#1}{#2}{T1}{lmss}{m}{n}%
  \TestFont{#1}{#2}{T1}{lmr}{b}{n}%
  \TestFont{#1}{#2}{T1}{lmr}{m}{it}%
  \TestFont{#1}{#2}{T1}{lmtt}{m}{n}%
  \TestFont{#1}{#2}{T1}{lmvtt}{m}{n}%
  \TestFont{#1}{#2}{T1}{qtm}{m}{n}%
  \TestFont{#1}{#2}{T1}{qhv}{m}{n}%
  \TestFont{#1}{#2}{T1}{qtm}{b}{n}%
  \TestFont{#1}{#2}{T1}{qtm}{m}{it}%
  \TestFont{#1}{#2}{T1}{qcr}{m}{n}%
  \newpage
}
\makeatother
\hologoList
\end{document}
%</test-list>
%    \end{macrocode}
%
% \section{Installation}
%
% \subsection{Download}
%
% \paragraph{Package.} This package is available on
% CTAN\footnote{\url{ftp://ftp.ctan.org/tex-archive/}}:
% \begin{description}
% \item[\CTAN{macros/latex/contrib/oberdiek/hologo.dtx}] The source file.
% \item[\CTAN{macros/latex/contrib/oberdiek/hologo.pdf}] Documentation.
% \end{description}
%
%
% \paragraph{Bundle.} All the packages of the bundle `oberdiek'
% are also available in a TDS compliant ZIP archive. There
% the packages are already unpacked and the documentation files
% are generated. The files and directories obey the TDS standard.
% \begin{description}
% \item[\CTAN{install/macros/latex/contrib/oberdiek.tds.zip}]
% \end{description}
% \emph{TDS} refers to the standard ``A Directory Structure
% for \TeX\ Files'' (\CTAN{tds/tds.pdf}). Directories
% with \xfile{texmf} in their name are usually organized this way.
%
% \subsection{Bundle installation}
%
% \paragraph{Unpacking.} Unpack the \xfile{oberdiek.tds.zip} in the
% TDS tree (also known as \xfile{texmf} tree) of your choice.
% Example (linux):
% \begin{quote}
%   |unzip oberdiek.tds.zip -d ~/texmf|
% \end{quote}
%
% \paragraph{Script installation.}
% Check the directory \xfile{TDS:scripts/oberdiek/} for
% scripts that need further installation steps.
% Package \xpackage{attachfile2} comes with the Perl script
% \xfile{pdfatfi.pl} that should be installed in such a way
% that it can be called as \texttt{pdfatfi}.
% Example (linux):
% \begin{quote}
%   |chmod +x scripts/oberdiek/pdfatfi.pl|\\
%   |cp scripts/oberdiek/pdfatfi.pl /usr/local/bin/|
% \end{quote}
%
% \subsection{Package installation}
%
% \paragraph{Unpacking.} The \xfile{.dtx} file is a self-extracting
% \docstrip\ archive. The files are extracted by running the
% \xfile{.dtx} through \plainTeX:
% \begin{quote}
%   \verb|tex hologo.dtx|
% \end{quote}
%
% \paragraph{TDS.} Now the different files must be moved into
% the different directories in your installation TDS tree
% (also known as \xfile{texmf} tree):
% \begin{quote}
% \def\t{^^A
% \begin{tabular}{@{}>{\ttfamily}l@{ $\rightarrow$ }>{\ttfamily}l@{}}
%   hologo.sty & tex/generic/oberdiek/hologo.sty\\
%   hologo.pdf & doc/latex/oberdiek/hologo.pdf\\
%   example/hologo-example.tex & doc/latex/oberdiek/example/hologo-example.tex\\
%   test/hologo-test1.tex & doc/latex/oberdiek/test/hologo-test1.tex\\
%   test/hologo-test-spacefactor.tex & doc/latex/oberdiek/test/hologo-test-spacefactor.tex\\
%   test/hologo-test-list.tex & doc/latex/oberdiek/test/hologo-test-list.tex\\
%   hologo.dtx & source/latex/oberdiek/hologo.dtx\\
% \end{tabular}^^A
% }^^A
% \sbox0{\t}^^A
% \ifdim\wd0>\linewidth
%   \begingroup
%     \advance\linewidth by\leftmargin
%     \advance\linewidth by\rightmargin
%   \edef\x{\endgroup
%     \def\noexpand\lw{\the\linewidth}^^A
%   }\x
%   \def\lwbox{^^A
%     \leavevmode
%     \hbox to \linewidth{^^A
%       \kern-\leftmargin\relax
%       \hss
%       \usebox0
%       \hss
%       \kern-\rightmargin\relax
%     }^^A
%   }^^A
%   \ifdim\wd0>\lw
%     \sbox0{\small\t}^^A
%     \ifdim\wd0>\linewidth
%       \ifdim\wd0>\lw
%         \sbox0{\footnotesize\t}^^A
%         \ifdim\wd0>\linewidth
%           \ifdim\wd0>\lw
%             \sbox0{\scriptsize\t}^^A
%             \ifdim\wd0>\linewidth
%               \ifdim\wd0>\lw
%                 \sbox0{\tiny\t}^^A
%                 \ifdim\wd0>\linewidth
%                   \lwbox
%                 \else
%                   \usebox0
%                 \fi
%               \else
%                 \lwbox
%               \fi
%             \else
%               \usebox0
%             \fi
%           \else
%             \lwbox
%           \fi
%         \else
%           \usebox0
%         \fi
%       \else
%         \lwbox
%       \fi
%     \else
%       \usebox0
%     \fi
%   \else
%     \lwbox
%   \fi
% \else
%   \usebox0
% \fi
% \end{quote}
% If you have a \xfile{docstrip.cfg} that configures and enables \docstrip's
% TDS installing feature, then some files can already be in the right
% place, see the documentation of \docstrip.
%
% \subsection{Refresh file name databases}
%
% If your \TeX~distribution
% (\teTeX, \mikTeX, \dots) relies on file name databases, you must refresh
% these. For example, \teTeX\ users run \verb|texhash| or
% \verb|mktexlsr|.
%
% \subsection{Some details for the interested}
%
% \paragraph{Attached source.}
%
% The PDF documentation on CTAN also includes the
% \xfile{.dtx} source file. It can be extracted by
% AcrobatReader 6 or higher. Another option is \textsf{pdftk},
% e.g. unpack the file into the current directory:
% \begin{quote}
%   \verb|pdftk hologo.pdf unpack_files output .|
% \end{quote}
%
% \paragraph{Unpacking with \LaTeX.}
% The \xfile{.dtx} chooses its action depending on the format:
% \begin{description}
% \item[\plainTeX:] Run \docstrip\ and extract the files.
% \item[\LaTeX:] Generate the documentation.
% \end{description}
% If you insist on using \LaTeX\ for \docstrip\ (really,
% \docstrip\ does not need \LaTeX), then inform the autodetect routine
% about your intention:
% \begin{quote}
%   \verb|latex \let\install=y% \iffalse meta-comment
%
% File: hologo.dtx
% Version: 2016/05/12 v1.11
% Info: A logo collection with bookmark support
%
% Copyright (C) 2010-2012 by
%    Heiko Oberdiek <heiko.oberdiek at googlemail.com>
%
% This work may be distributed and/or modified under the
% conditions of the LaTeX Project Public License, either
% version 1.3c of this license or (at your option) any later
% version. This version of this license is in
%    http://www.latex-project.org/lppl/lppl-1-3c.txt
% and the latest version of this license is in
%    http://www.latex-project.org/lppl.txt
% and version 1.3 or later is part of all distributions of
% LaTeX version 2005/12/01 or later.
%
% This work has the LPPL maintenance status "maintained".
%
% This Current Maintainer of this work is Heiko Oberdiek.
%
% The Base Interpreter refers to any `TeX-Format',
% because some files are installed in TDS:tex/generic//.
%
% This work consists of the main source file hologo.dtx
% and the derived files
%    hologo.sty, hologo.pdf, hologo.ins, hologo.drv, hologo-example.tex,
%    hologo-test1.tex, hologo-test-spacefactor.tex,
%    hologo-test-list.tex.
%
% Distribution:
%    CTAN:macros/latex/contrib/oberdiek/hologo.dtx
%    CTAN:macros/latex/contrib/oberdiek/hologo.pdf
%
% Unpacking:
%    (a) If hologo.ins is present:
%           tex hologo.ins
%    (b) Without hologo.ins:
%           tex hologo.dtx
%    (c) If you insist on using LaTeX
%           latex \let\install=y\input{hologo.dtx}
%        (quote the arguments according to the demands of your shell)
%
% Documentation:
%    (a) If hologo.drv is present:
%           latex hologo.drv
%    (b) Without hologo.drv:
%           latex hologo.dtx; ...
%    The class ltxdoc loads the configuration file ltxdoc.cfg
%    if available. Here you can specify further options, e.g.
%    use A4 as paper format:
%       \PassOptionsToClass{a4paper}{article}
%
%    Programm calls to get the documentation (example):
%       pdflatex hologo.dtx
%       makeindex -s gind.ist hologo.idx
%       pdflatex hologo.dtx
%       makeindex -s gind.ist hologo.idx
%       pdflatex hologo.dtx
%
% Installation:
%    TDS:tex/generic/oberdiek/hologo.sty
%    TDS:doc/latex/oberdiek/hologo.pdf
%    TDS:doc/latex/oberdiek/example/hologo-example.tex
%    TDS:doc/latex/oberdiek/test/hologo-test1.tex
%    TDS:doc/latex/oberdiek/test/hologo-test-spacefactor.tex
%    TDS:doc/latex/oberdiek/test/hologo-test-list.tex
%    TDS:source/latex/oberdiek/hologo.dtx
%
%<*ignore>
\begingroup
  \catcode123=1 %
  \catcode125=2 %
  \def\x{LaTeX2e}%
\expandafter\endgroup
\ifcase 0\ifx\install y1\fi\expandafter
         \ifx\csname processbatchFile\endcsname\relax\else1\fi
         \ifx\fmtname\x\else 1\fi\relax
\else\csname fi\endcsname
%</ignore>
%<*install>
\input docstrip.tex
\Msg{************************************************************************}
\Msg{* Installation}
\Msg{* Package: hologo 2016/05/12 v1.11 A logo collection with bookmark support (HO)}
\Msg{************************************************************************}

\keepsilent
\askforoverwritefalse

\let\MetaPrefix\relax
\preamble

This is a generated file.

Project: hologo
Version: 2016/05/12 v1.11

Copyright (C) 2010-2012 by
   Heiko Oberdiek <heiko.oberdiek at googlemail.com>

This work may be distributed and/or modified under the
conditions of the LaTeX Project Public License, either
version 1.3c of this license or (at your option) any later
version. This version of this license is in
   http://www.latex-project.org/lppl/lppl-1-3c.txt
and the latest version of this license is in
   http://www.latex-project.org/lppl.txt
and version 1.3 or later is part of all distributions of
LaTeX version 2005/12/01 or later.

This work has the LPPL maintenance status "maintained".

This Current Maintainer of this work is Heiko Oberdiek.

The Base Interpreter refers to any `TeX-Format',
because some files are installed in TDS:tex/generic//.

This work consists of the main source file hologo.dtx
and the derived files
   hologo.sty, hologo.pdf, hologo.ins, hologo.drv, hologo-example.tex,
   hologo-test1.tex, hologo-test-spacefactor.tex,
   hologo-test-list.tex.

\endpreamble
\let\MetaPrefix\DoubleperCent

\generate{%
  \file{hologo.ins}{\from{hologo.dtx}{install}}%
  \file{hologo.drv}{\from{hologo.dtx}{driver}}%
  \usedir{tex/generic/oberdiek}%
  \file{hologo.sty}{\from{hologo.dtx}{package}}%
  \usedir{doc/latex/oberdiek/example}%
  \file{hologo-example.tex}{\from{hologo.dtx}{example}}%
  \usedir{doc/latex/oberdiek/test}%
  \file{hologo-test1.tex}{\from{hologo.dtx}{test1}}%
  \file{hologo-test-spacefactor.tex}{\from{hologo.dtx}{test-spacefactor}}%
  \file{hologo-test-list.tex}{\from{hologo.dtx}{test-list}}%
  \nopreamble
  \nopostamble
  \usedir{source/latex/oberdiek/catalogue}%
  \file{hologo.xml}{\from{hologo.dtx}{catalogue}}%
}

\catcode32=13\relax% active space
\let =\space%
\Msg{************************************************************************}
\Msg{*}
\Msg{* To finish the installation you have to move the following}
\Msg{* file into a directory searched by TeX:}
\Msg{*}
\Msg{*     hologo.sty}
\Msg{*}
\Msg{* To produce the documentation run the file `hologo.drv'}
\Msg{* through LaTeX.}
\Msg{*}
\Msg{* Happy TeXing!}
\Msg{*}
\Msg{************************************************************************}

\endbatchfile
%</install>
%<*ignore>
\fi
%</ignore>
%<*driver>
\NeedsTeXFormat{LaTeX2e}
\ProvidesFile{hologo.drv}%
  [2016/05/12 v1.11 A logo collection with bookmark support (HO)]%
\documentclass{ltxdoc}
\usepackage{holtxdoc}[2011/11/22]
\usepackage{hologo}[2016/05/12]
\usepackage{longtable}
\usepackage{array}
\usepackage{paralist}
%\usepackage[T1]{fontenc}
%\usepackage{lmodern}
\begin{document}
  \DocInput{hologo.dtx}%
\end{document}
%</driver>
% \fi
%
%
% \CharacterTable
%  {Upper-case    \A\B\C\D\E\F\G\H\I\J\K\L\M\N\O\P\Q\R\S\T\U\V\W\X\Y\Z
%   Lower-case    \a\b\c\d\e\f\g\h\i\j\k\l\m\n\o\p\q\r\s\t\u\v\w\x\y\z
%   Digits        \0\1\2\3\4\5\6\7\8\9
%   Exclamation   \!     Double quote  \"     Hash (number) \#
%   Dollar        \$     Percent       \%     Ampersand     \&
%   Acute accent  \'     Left paren    \(     Right paren   \)
%   Asterisk      \*     Plus          \+     Comma         \,
%   Minus         \-     Point         \.     Solidus       \/
%   Colon         \:     Semicolon     \;     Less than     \<
%   Equals        \=     Greater than  \>     Question mark \?
%   Commercial at \@     Left bracket  \[     Backslash     \\
%   Right bracket \]     Circumflex    \^     Underscore    \_
%   Grave accent  \`     Left brace    \{     Vertical bar  \|
%   Right brace   \}     Tilde         \~}
%
% \GetFileInfo{hologo.drv}
%
% \title{The \xpackage{hologo} package}
% \date{2016/05/12 v1.11}
% \author{Heiko Oberdiek\\\xemail{heiko.oberdiek at googlemail.com}}
%
% \maketitle
%
% \begin{abstract}
% This package starts a collection of logos with support for bookmarks
% strings.
% \end{abstract}
%
% \tableofcontents
%
% \section{Documentation}
%
% \subsection{Logo macros}
%
% \begin{declcs}{hologo} \M{name}
% \end{declcs}
% Macro \cs{hologo} sets the logo with name \meta{name}.
% The following table shows the supported names.
%
% \begingroup
%   \def\hologoEntry#1#2#3{^^A
%     #1&#2&\hologoLogoSetup{#1}{variant=#2}\hologo{#1}&#3\tabularnewline
%   }
%   \begin{longtable}{>{\ttfamily}l>{\ttfamily}lll}
%     \rmfamily\bfseries{name} & \rmfamily\bfseries variant
%     & \bfseries logo & \bfseries since\\
%     \hline
%     \endhead
%     \hologoList
%   \end{longtable}
% \endgroup
%
% \begin{declcs}{Hologo} \M{name}
% \end{declcs}
% Macro \cs{Hologo} starts the logo \meta{name} with an uppercase
% letter. As an exception small greek letters are not converted
% to uppercase. Examples, see \hologo{eTeX} and \hologo{ExTeX}.
%
% \subsection{Setup macros}
%
% The package does not support package options, but the following
% setup macros can be used to set options.
%
% \begin{declcs}{hologoSetup} \M{key value list}
% \end{declcs}
% Macro \cs{hologoSetup} sets global options.
%
% \begin{declcs}{hologoLogoSetup} \M{logo} \M{key value list}
% \end{declcs}
% Some options can also be used to configure a logo.
% These settings take precedence over global option settings.
%
% \subsection{Options}\label{sec:options}
%
% There are boolean and string options:
% \begin{description}
% \item[Boolean option:]
% It takes |true| or |false|
% as value. If the value is omitted, then |true| is used.
% \item[String option:]
% A value must be given as string. (But the string might be empty.)
% \end{description}
% The following options can be used both in \cs{hologoSetup}
% and \cs{hologoLogoSetup}:
% \begin{description}
% \def\entry#1{\item[\xoption{#1}:]}
% \entry{break}
%   enables or disables line breaks inside the logo. This setting is
%   refined by options \xoption{hyphenbreak}, \xoption{spacebreak}
%   or \xoption{discretionarybreak}.
%   Default is |false|.
% \entry{hyphenbreak}
%   enables or disables the line break right after the hyphen character.
% \entry{spacebreak}
%   enables or disables line breaks at space characters.
% \entry{discretionarybreak}
%   enables or disables line breaks at hyphenation points
%   (inserted by \cs{-}).
% \end{description}
% Macro \cs{hologoLogoSetup} also knows:
% \begin{description}
% \item[\xoption{variant}:]
%   This is a string option. It specifies a variant of a logo that
%   must exist. An empty string selects the package default variant.
% \end{description}
% Example:
% \begin{quote}
%   |\hologoSetup{break=false}|\\
%   |\hologoLogoSetup{plainTeX}{variant=hyphen,hyphenbreak}|\\
%   Then ``plain-\TeX'' contains one break point after the hyphen.
% \end{quote}
%
% \subsection{Driver options}
%
% Sometimes graphical operations are needed to construct some
% glyphs (e.g.\ \hologo{XeTeX}). If package \xpackage{graphics}
% or package \xpackage{pgf} are found, then the macros are taken
% from there. Otherwise the packge defines its own operations
% and therefore needs the driver information. Many drivers are
% detected automatically (\hologo{pdfTeX}/\hologo{LuaTeX}
% in PDF mode, \hologo{XeTeX}, \hologo{VTeX}). These have precedence
% over a driver option. The driver can be given as package option
% or using \cs{hologoDriverSetup}.
% The following list contains the recognized driver options:
% \begin{itemize}
% \item \xoption{pdftex}, \xoption{luatex}
% \item \xoption{dvipdfm}, \xoption{dvipdfmx}
% \item \xoption{dvips}, \xoption{dvipsone}, \xoption{xdvi}
% \item \xoption{xetex}
% \item \xoption{vtex}
% \end{itemize}
% The left driver of a line is the driver name that is used internally.
% The following names are aliases for drivers that use the
% same method. Therefore the entry in the \xext{log} file for
% the used driver prints the internally used driver name.
% \begin{description}
% \item[\xoption{driverfallback}:]
%   This option expects a driver that is used,
%   if the driver could not be detected automatically.
% \end{description}
%
% \begin{declcs}{hologoDriverSetup} \M{driver option}
% \end{declcs}
% The driver can also be configured after package loading
% using \cs{hologoDriverSetup}, also the way for \hologo{plainTeX}
% to setup the driver.
%
% \subsection{Font setup}
%
% Some logos require a special font, but should also be usable by
% \hologo{plainTeX}. Therefore the package provides some ways
% to influence the font settings. The options below
% take font settings as values. Both font commands
% such as \cs{sffamily} and macros that take one argument
% like \cs{textsf} can be used.
%
% \begin{declcs}{hologoFontSetup} \M{key value list}
% \end{declcs}
% Macro \cs{hologoFontSetup} sets the fonts for all logos.
% Supported keys:
% \begin{description}
% \def\entry#1{\item[\xoption{#1}:]}
% \entry{general}
%   This font is used for all logos. The default is empty.
%   That means no special font is used.
% \entry{bibsf}
%   This font is used for
%   {\hologoLogoSetup{BibTeX}{variant=sf}\hologo{BibTeX}}
%   with variant \xoption{sf}.
% \entry{rm}
%   This font is a serif font. It is used for \hologo{ExTeX}.
% \entry{sc}
%   This font specifies a small caps font. It is used for
%   {\hologoLogoSetup{BibTeX}{variant=sc}\hologo{BibTeX}}
%   with variant \xoption{sc}.
% \entry{sf}
%   This font specifies a sans serif font. The default
%   is \cs{sffamily}, then \cs{sf} is tried. Otherwise
%   a warning is given. It is used by \hologo{KOMAScript}.
% \entry{sy}
%   This is the font for math symbols (e.g. cmsy).
%   It is used by \hologo{AmS}, \hologo{NTS}, \hologo{ExTeX}.
% \entry{logo}
%   \hologo{METAFONT} and \hologo{METAPOST} are using that font.
%   In \hologo{LaTeX} \cs{logofamily} is used and
%   the definitions of package \xpackage{mflogo} are used
%   if the package is not loaded.
%   Otherwise the \cs{tenlogo} is used and defined
%   if it does not already exists.
% \end{description}
%
% \begin{declcs}{hologoLogoFontSetup} \M{logo} \M{key value list}
% \end{declcs}
% Fonts can also be set for a logo or logo component separately,
% see the following list.
% The keys are the same as for \cs{hologoFontSetup}.
%
% \begin{longtable}{>{\ttfamily}l>{\sffamily}ll}
%   \meta{logo} & keys & result\\
%   \hline
%   \endhead
%   BibTeX & bibsf & {\hologoLogoSetup{BibTeX}{variant=sf}\hologo{BibTeX}}\\[.5ex]
%   BibTeX & sc & {\hologoLogoSetup{BibTeX}{variant=sc}\hologo{BibTeX}}\\[.5ex]
%   ExTeX & rm & \hologo{ExTeX}\\
%   SliTeX & rm & \hologo{SliTeX}\\[.5ex]
%   AmS & sy & \hologo{AmS}\\
%   ExTeX & sy & \hologo{ExTeX}\\
%   NTS & sy & \hologo{NTS}\\[.5ex]
%   KOMAScript & sf & \hologo{KOMAScript}\\[.5ex]
%   METAFONT & logo & \hologo{METAFONT}\\
%   METAPOST & logo & \hologo{METAPOST}\\[.5ex]
%   SliTeX & sc \hologo{SliTeX}
% \end{longtable}
%
% \subsubsection{Font order}
%
% For all logos the font \xoption{general} is applied first.
% Example:
%\begin{quote}
%|\hologoFontSetup{general=\color{red}}|
%\end{quote}
% will print red logos.
% Then if the font uses a special font \xoption{sf}, for example,
% the font is applied that is setup by \cs{hologoLogoFontSetup}.
% If this font is not setup, then the common font setup
% by \cs{hologoFontSetup} is used. Otherwise a warning is given,
% that there is no font configured.
%
% \subsection{Additional user macros}
%
% Usually a variant of a logo is configured by using
% \cs{hologoLogoSetup}, because it is bad style to mix
% different variants of the same logo in the same text.
% There the following macros are a convenience for testing.
%
% \begin{declcs}{hologoVariant} \M{name} \M{variant}\\
%   \cs{HologoVariant} \M{name} \M{variant}
% \end{declcs}
% Logo \meta{name} is set using \meta{variant} that specifies
% explicitely which variant of the macro is used. If the argument
% is empty, then the default form of the logo is used
% (configurable by \cs{hologoLogoSetup}).
%
% \cs{HologoVariant} is used if the logo is set in a context
% that needs an uppercase first letter (beginning of a sentence, \dots).
%
% \begin{declcs}{hologoList}\\
%   \cs{hologoEntry} \M{logo} \M{variant} \M{since}
% \end{declcs}
% Macro \cs{hologoList} contains all logos that are provided
% by the package including variants. The list consists of calls
% of \cs{hologoEntry} with three arguments starting with the
% logo name \meta{logo} and its variant \meta{variant}. An empty
% variant means the current default. Argument \meta{since} specifies
% with version of the package \xpackage{hologo} is needed to get
% the logo. If the logo is fixed, then the date gets updated.
% Therefore the date \meta{since} is not exactly the date of
% the first introduction, but rather the date of the latest fix.
%
% Before \cs{hologoList} can be used, macro \cs{hologoEntry} needs
% a definition. The example file in section \ref{sec:example}
% shows applications of \cs{hologoList}.
%
% \subsection{Supported contexts}
%
% Macros \cs{hologo} and friends support special contexts:
% \begin{itemize}
% \item \hologo{LaTeX}'s protection mechanism.
% \item Bookmarks of package \xpackage{hyperref}.
% \item Package \xpackage{tex4ht}.
% \item The macros can be used inside \cs{csname} constructs,
%   if \cs{ifincsname} is available (\hologo{pdfTeX}, \hologo{XeTeX},
%   \hologo{LuaTeX}).
% \end{itemize}
%
% \subsection{Example}
% \label{sec:example}
%
% The following example prints the logos in different fonts.
%    \begin{macrocode}
%<*example>
%<<verbatim
\NeedsTeXFormat{LaTeX2e}
\documentclass[a4paper]{article}
\usepackage[
  hmargin=20mm,
  vmargin=20mm,
]{geometry}
\pagestyle{empty}
\usepackage{hologo}[2016/05/12]
\usepackage{longtable}
\usepackage{array}
\setlength{\extrarowheight}{2pt}
\usepackage[T1]{fontenc}
\usepackage{lmodern}
\usepackage{pdflscape}
\usepackage[
  pdfencoding=auto,
]{hyperref}
\hypersetup{
  pdfauthor={Heiko Oberdiek},
  pdftitle={Example for package `hologo'},
  pdfsubject={Logos with fonts lmr, lmss, qtm, qpl, qhv},
}
\usepackage{bookmark}

% Print the logo list on the console

\begingroup
  \typeout{}%
  \typeout{*** Begin of logo list ***}%
  \newcommand*{\hologoEntry}[3]{%
    \typeout{#1 \ifx\\#2\\\else(#2) \fi[#3]}%
  }%
  \hologoList
  \typeout{*** End of logo list ***}%
  \typeout{}%
\endgroup

\begin{document}
\begin{landscape}

  \section{Example file for package `hologo'}

  % Table for font names

  \begin{longtable}{>{\bfseries}ll}
    \textbf{font} & \textbf{Font name}\\
    \hline
    lmr & Latin Modern Roman\\
    lmss & Latin Modern Sans\\
    qtm & \TeX\ Gyre Termes\\
    qhv & \TeX\ Gyre Heros\\
    qpl & \TeX\ Gyre Pagella\\
  \end{longtable}

  % Logo list with logos in different fonts

  \begingroup
    \newcommand*{\SetVariant}[2]{%
      \ifx\\#2\\%
      \else
        \hologoLogoSetup{#1}{variant=#2}%
      \fi
    }%
    \newcommand*{\hologoEntry}[3]{%
      \SetVariant{#1}{#2}%
      \raisebox{1em}[0pt][0pt]{\hypertarget{#1@#2}{}}%
      \bookmark[%
        dest={#1@#2},%
      ]{%
        #1\ifx\\#2\\\else\space(#2)\fi: \Hologo{#1}, \hologo{#1} %
        [Unicode]%
      }%
      \hypersetup{unicode=false}%
      \bookmark[%
        dest={#1@#2},%
      ]{%
        #1\ifx\\#2\\\else\space(#2)\fi: \Hologo{#1}, \hologo{#1} %
        [PDFDocEncoding]%
      }%
      \texttt{#1}%
      &%
      \texttt{#2}%
      &%
      \Hologo{#1}%
      &%
      \SetVariant{#1}{#2}%
      \hologo{#1}%
      &%
      \SetVariant{#1}{#2}%
      \fontfamily{qtm}\selectfont
      \hologo{#1}%
      &%
      \SetVariant{#1}{#2}%
      \fontfamily{qpl}\selectfont
      \hologo{#1}%
      &%
      \SetVariant{#1}{#2}%
      \textsf{\hologo{#1}}%
      &%
      \SetVariant{#1}{#2}%
      \fontfamily{qhv}\selectfont
      \hologo{#1}%
      \tabularnewline
    }%
    \begin{longtable}{llllllll}%
      \textbf{\textit{logo}} & \textbf{\textit{variant}} &
      \texttt{\string\Hologo} &
      \textbf{lmr} & \textbf{qtm} & \textbf{qpl} &
      \textbf{lmss} & \textbf{qhv}
      \tabularnewline
      \hline
      \endhead
      \hologoList
    \end{longtable}%
  \endgroup

\end{landscape}
\end{document}
%verbatim
%</example>
%    \end{macrocode}
%
% \StopEventually{
% }
%
% \section{Implementation}
%    \begin{macrocode}
%<*package>
%    \end{macrocode}
%    Reload check, especially if the package is not used with \LaTeX.
%    \begin{macrocode}
\begingroup\catcode61\catcode48\catcode32=10\relax%
  \catcode13=5 % ^^M
  \endlinechar=13 %
  \catcode35=6 % #
  \catcode39=12 % '
  \catcode44=12 % ,
  \catcode45=12 % -
  \catcode46=12 % .
  \catcode58=12 % :
  \catcode64=11 % @
  \catcode123=1 % {
  \catcode125=2 % }
  \expandafter\let\expandafter\x\csname ver@hologo.sty\endcsname
  \ifx\x\relax % plain-TeX, first loading
  \else
    \def\empty{}%
    \ifx\x\empty % LaTeX, first loading,
      % variable is initialized, but \ProvidesPackage not yet seen
    \else
      \expandafter\ifx\csname PackageInfo\endcsname\relax
        \def\x#1#2{%
          \immediate\write-1{Package #1 Info: #2.}%
        }%
      \else
        \def\x#1#2{\PackageInfo{#1}{#2, stopped}}%
      \fi
      \x{hologo}{The package is already loaded}%
      \aftergroup\endinput
    \fi
  \fi
\endgroup%
%    \end{macrocode}
%    Package identification:
%    \begin{macrocode}
\begingroup\catcode61\catcode48\catcode32=10\relax%
  \catcode13=5 % ^^M
  \endlinechar=13 %
  \catcode35=6 % #
  \catcode39=12 % '
  \catcode40=12 % (
  \catcode41=12 % )
  \catcode44=12 % ,
  \catcode45=12 % -
  \catcode46=12 % .
  \catcode47=12 % /
  \catcode58=12 % :
  \catcode64=11 % @
  \catcode91=12 % [
  \catcode93=12 % ]
  \catcode123=1 % {
  \catcode125=2 % }
  \expandafter\ifx\csname ProvidesPackage\endcsname\relax
    \def\x#1#2#3[#4]{\endgroup
      \immediate\write-1{Package: #3 #4}%
      \xdef#1{#4}%
    }%
  \else
    \def\x#1#2[#3]{\endgroup
      #2[{#3}]%
      \ifx#1\@undefined
        \xdef#1{#3}%
      \fi
      \ifx#1\relax
        \xdef#1{#3}%
      \fi
    }%
  \fi
\expandafter\x\csname ver@hologo.sty\endcsname
\ProvidesPackage{hologo}%
  [2016/05/12 v1.11 A logo collection with bookmark support (HO)]%
%    \end{macrocode}
%
%    \begin{macrocode}
\begingroup\catcode61\catcode48\catcode32=10\relax%
  \catcode13=5 % ^^M
  \endlinechar=13 %
  \catcode123=1 % {
  \catcode125=2 % }
  \catcode64=11 % @
  \def\x{\endgroup
    \expandafter\edef\csname HOLOGO@AtEnd\endcsname{%
      \endlinechar=\the\endlinechar\relax
      \catcode13=\the\catcode13\relax
      \catcode32=\the\catcode32\relax
      \catcode35=\the\catcode35\relax
      \catcode61=\the\catcode61\relax
      \catcode64=\the\catcode64\relax
      \catcode123=\the\catcode123\relax
      \catcode125=\the\catcode125\relax
    }%
  }%
\x\catcode61\catcode48\catcode32=10\relax%
\catcode13=5 % ^^M
\endlinechar=13 %
\catcode35=6 % #
\catcode64=11 % @
\catcode123=1 % {
\catcode125=2 % }
\def\TMP@EnsureCode#1#2{%
  \edef\HOLOGO@AtEnd{%
    \HOLOGO@AtEnd
    \catcode#1=\the\catcode#1\relax
  }%
  \catcode#1=#2\relax
}
\TMP@EnsureCode{10}{12}% ^^J
\TMP@EnsureCode{33}{12}% !
\TMP@EnsureCode{34}{12}% "
\TMP@EnsureCode{36}{3}% $
\TMP@EnsureCode{38}{4}% &
\TMP@EnsureCode{39}{12}% '
\TMP@EnsureCode{40}{12}% (
\TMP@EnsureCode{41}{12}% )
\TMP@EnsureCode{42}{12}% *
\TMP@EnsureCode{43}{12}% +
\TMP@EnsureCode{44}{12}% ,
\TMP@EnsureCode{45}{12}% -
\TMP@EnsureCode{46}{12}% .
\TMP@EnsureCode{47}{12}% /
\TMP@EnsureCode{58}{12}% :
\TMP@EnsureCode{59}{12}% ;
\TMP@EnsureCode{60}{12}% <
\TMP@EnsureCode{62}{12}% >
\TMP@EnsureCode{63}{12}% ?
\TMP@EnsureCode{91}{12}% [
\TMP@EnsureCode{93}{12}% ]
\TMP@EnsureCode{94}{7}% ^ (superscript)
\TMP@EnsureCode{95}{8}% _ (subscript)
\TMP@EnsureCode{96}{12}% `
\TMP@EnsureCode{124}{12}% |
\edef\HOLOGO@AtEnd{%
  \HOLOGO@AtEnd
  \escapechar\the\escapechar\relax
  \noexpand\endinput
}
\escapechar=92 %
%    \end{macrocode}
%
% \subsection{Logo list}
%
%    \begin{macro}{\hologoList}
%    \begin{macrocode}
\def\hologoList{%
  \hologoEntry{(La)TeX}{}{2011/10/01}%
  \hologoEntry{AmSLaTeX}{}{2010/04/16}%
  \hologoEntry{AmSTeX}{}{2010/04/16}%
  \hologoEntry{biber}{}{2011/10/01}%
  \hologoEntry{BibTeX}{}{2011/10/01}%
  \hologoEntry{BibTeX}{sf}{2011/10/01}%
  \hologoEntry{BibTeX}{sc}{2011/10/01}%
  \hologoEntry{BibTeX8}{}{2011/11/22}%
  \hologoEntry{ConTeXt}{}{2011/03/25}%
  \hologoEntry{ConTeXt}{narrow}{2011/03/25}%
  \hologoEntry{ConTeXt}{simple}{2011/03/25}%
  \hologoEntry{emTeX}{}{2010/04/26}%
  \hologoEntry{eTeX}{}{2010/04/08}%
  \hologoEntry{ExTeX}{}{2011/10/01}%
  \hologoEntry{HanTheThanh}{}{2011/11/29}%
  \hologoEntry{iniTeX}{}{2011/10/01}%
  \hologoEntry{KOMAScript}{}{2011/10/01}%
  \hologoEntry{La}{}{2010/05/08}%
  \hologoEntry{LaTeX}{}{2010/04/08}%
  \hologoEntry{LaTeX2e}{}{2010/04/08}%
  \hologoEntry{LaTeX3}{}{2010/04/24}%
  \hologoEntry{LaTeXe}{}{2010/04/08}%
  \hologoEntry{LaTeXML}{}{2011/11/22}%
  \hologoEntry{LaTeXTeX}{}{2011/10/01}%
  \hologoEntry{LuaLaTeX}{}{2010/04/08}%
  \hologoEntry{LuaTeX}{}{2010/04/08}%
  \hologoEntry{LyX}{}{2011/10/01}%
  \hologoEntry{METAFONT}{}{2011/10/01}%
  \hologoEntry{MetaFun}{}{2011/10/01}%
  \hologoEntry{METAPOST}{}{2011/10/01}%
  \hologoEntry{MetaPost}{}{2011/10/01}%
  \hologoEntry{MiKTeX}{}{2011/10/01}%
  \hologoEntry{NTS}{}{2011/10/01}%
  \hologoEntry{OzMF}{}{2011/10/01}%
  \hologoEntry{OzMP}{}{2011/10/01}%
  \hologoEntry{OzTeX}{}{2011/10/01}%
  \hologoEntry{OzTtH}{}{2011/10/01}%
  \hologoEntry{PCTeX}{}{2011/10/01}%
  \hologoEntry{pdfTeX}{}{2011/10/01}%
  \hologoEntry{pdfLaTeX}{}{2011/10/01}%
  \hologoEntry{PiC}{}{2011/10/01}%
  \hologoEntry{PiCTeX}{}{2011/10/01}%
  \hologoEntry{plainTeX}{}{2010/04/08}%
  \hologoEntry{plainTeX}{space}{2010/04/16}%
  \hologoEntry{plainTeX}{hyphen}{2010/04/16}%
  \hologoEntry{plainTeX}{runtogether}{2010/04/16}%
  \hologoEntry{SageTeX}{}{2011/11/22}%
  \hologoEntry{SLiTeX}{}{2011/10/01}%
  \hologoEntry{SLiTeX}{lift}{2011/10/01}%
  \hologoEntry{SLiTeX}{narrow}{2011/10/01}%
  \hologoEntry{SLiTeX}{simple}{2011/10/01}%
  \hologoEntry{SliTeX}{}{2011/10/01}%
  \hologoEntry{SliTeX}{narrow}{2011/10/01}%
  \hologoEntry{SliTeX}{simple}{2011/10/01}%
  \hologoEntry{SliTeX}{lift}{2011/10/01}%
  \hologoEntry{teTeX}{}{2011/10/01}%
  \hologoEntry{TeX}{}{2010/04/08}%
  \hologoEntry{TeX4ht}{}{2011/11/22}%
  \hologoEntry{TTH}{}{2011/11/22}%
  \hologoEntry{virTeX}{}{2011/10/01}%
  \hologoEntry{VTeX}{}{2010/04/24}%
  \hologoEntry{Xe}{}{2010/04/08}%
  \hologoEntry{XeLaTeX}{}{2010/04/08}%
  \hologoEntry{XeTeX}{}{2010/04/08}%
}
%    \end{macrocode}
%    \end{macro}
%
% \subsection{Load resources}
%
%    \begin{macrocode}
\begingroup\expandafter\expandafter\expandafter\endgroup
\expandafter\ifx\csname RequirePackage\endcsname\relax
  \def\TMP@RequirePackage#1[#2]{%
    \begingroup\expandafter\expandafter\expandafter\endgroup
    \expandafter\ifx\csname ver@#1.sty\endcsname\relax
      \input #1.sty\relax
    \fi
  }%
  \TMP@RequirePackage{ltxcmds}[2011/02/04]%
  \TMP@RequirePackage{infwarerr}[2010/04/08]%
  \TMP@RequirePackage{kvsetkeys}[2010/03/01]%
  \TMP@RequirePackage{kvdefinekeys}[2010/03/01]%
  \TMP@RequirePackage{pdftexcmds}[2010/04/01]%
  \TMP@RequirePackage{ifpdf}[2010/01/28]%
  \TMP@RequirePackage{ifluatex}[2010/03/01]%
  \ltx@IfUndefined{newif}{%
    \expandafter\let\csname newif\endcsname\ltx@newif
  }{}%
  \TMP@RequirePackage{ifxetex}[2009/01/23]%
  \TMP@RequirePackage{ifvtex}[2010/03/01]%
\else
  \RequirePackage{ltxcmds}[2011/02/04]%
  \RequirePackage{infwarerr}[2010/04/08]%
  \RequirePackage{kvsetkeys}[2010/03/01]%
  \RequirePackage{kvdefinekeys}[2010/03/01]%
  \RequirePackage{pdftexcmds}[2010/04/01]%
  \RequirePackage{ifpdf}[2010/01/28]%
  \RequirePackage{ifluatex}[2010/03/01]%
  \RequirePackage{ifxetex}[2009/01/23]%
  \RequirePackage{ifvtex}[2010/03/01]%
\fi
%    \end{macrocode}
%
%    \begin{macro}{\HOLOGO@IfDefined}
%    \begin{macrocode}
\def\HOLOGO@IfExists#1{%
  \ifx\@undefined#1%
    \expandafter\ltx@secondoftwo
  \else
    \ifx\relax#1%
      \expandafter\ltx@secondoftwo
    \else
      \expandafter\expandafter\expandafter\ltx@firstoftwo
    \fi
  \fi
}
%    \end{macrocode}
%    \end{macro}
%
% \subsection{Setup macros}
%
%    \begin{macro}{\hologoSetup}
%    \begin{macrocode}
\def\hologoSetup{%
  \let\HOLOGO@name\relax
  \HOLOGO@Setup
}
%    \end{macrocode}
%    \end{macro}
%
%    \begin{macro}{\hologoLogoSetup}
%    \begin{macrocode}
\def\hologoLogoSetup#1{%
  \edef\HOLOGO@name{#1}%
  \ltx@IfUndefined{HoLogo@\HOLOGO@name}{%
    \@PackageError{hologo}{%
      Unknown logo `\HOLOGO@name'%
    }\@ehc
    \ltx@gobble
  }{%
    \HOLOGO@Setup
  }%
}
%    \end{macrocode}
%    \end{macro}
%
%    \begin{macro}{\HOLOGO@Setup}
%    \begin{macrocode}
\def\HOLOGO@Setup{%
  \kvsetkeys{HoLogo}%
}
%    \end{macrocode}
%    \end{macro}
%
% \subsection{Options}
%
%    \begin{macro}{\HOLOGO@DeclareBoolOption}
%    \begin{macrocode}
\def\HOLOGO@DeclareBoolOption#1{%
  \expandafter\chardef\csname HOLOGOOPT@#1\endcsname\ltx@zero
  \kv@define@key{HoLogo}{#1}[true]{%
    \def\HOLOGO@temp{##1}%
    \ifx\HOLOGO@temp\HOLOGO@true
      \ifx\HOLOGO@name\relax
        \expandafter\chardef\csname HOLOGOOPT@#1\endcsname=\ltx@one
      \else
        \expandafter\chardef\csname
        HoLogoOpt@#1@\HOLOGO@name\endcsname\ltx@one
      \fi
      \HOLOGO@SetBreakAll{#1}%
    \else
      \ifx\HOLOGO@temp\HOLOGO@false
        \ifx\HOLOGO@name\relax
          \expandafter\chardef\csname HOLOGOOPT@#1\endcsname=\ltx@zero
        \else
          \expandafter\chardef\csname
          HoLogoOpt@#1@\HOLOGO@name\endcsname=\ltx@zero
        \fi
        \HOLOGO@SetBreakAll{#1}%
      \else
        \@PackageError{hologo}{%
          Unknown value `##1' for boolean option `#1'.\MessageBreak
          Known values are `true' and `false'%
        }\@ehc
      \fi
    \fi
  }%
}
%    \end{macrocode}
%    \end{macro}
%
%    \begin{macro}{\HOLOGO@SetBreakAll}
%    \begin{macrocode}
\def\HOLOGO@SetBreakAll#1{%
  \def\HOLOGO@temp{#1}%
  \ifx\HOLOGO@temp\HOLOGO@break
    \ifx\HOLOGO@name\relax
      \chardef\HOLOGOOPT@hyphenbreak=\HOLOGOOPT@break
      \chardef\HOLOGOOPT@spacebreak=\HOLOGOOPT@break
      \chardef\HOLOGOOPT@discretionarybreak=\HOLOGOOPT@break
    \else
      \expandafter\chardef
         \csname HoLogoOpt@hyphenbreak@\HOLOGO@name\endcsname=%
         \csname HoLogoOpt@break@\HOLOGO@name\endcsname
      \expandafter\chardef
         \csname HoLogoOpt@spacebreak@\HOLOGO@name\endcsname=%
         \csname HoLogoOpt@break@\HOLOGO@name\endcsname
      \expandafter\chardef
         \csname HoLogoOpt@discretionarybreak@\HOLOGO@name
             \endcsname=%
         \csname HoLogoOpt@break@\HOLOGO@name\endcsname
    \fi
  \fi
}
%    \end{macrocode}
%    \end{macro}
%
%    \begin{macro}{\HOLOGO@true}
%    \begin{macrocode}
\def\HOLOGO@true{true}
%    \end{macrocode}
%    \end{macro}
%    \begin{macro}{\HOLOGO@false}
%    \begin{macrocode}
\def\HOLOGO@false{false}
%    \end{macrocode}
%    \end{macro}
%    \begin{macro}{\HOLOGO@break}
%    \begin{macrocode}
\def\HOLOGO@break{break}
%    \end{macrocode}
%    \end{macro}
%
%    \begin{macrocode}
\HOLOGO@DeclareBoolOption{break}
\HOLOGO@DeclareBoolOption{hyphenbreak}
\HOLOGO@DeclareBoolOption{spacebreak}
\HOLOGO@DeclareBoolOption{discretionarybreak}
%    \end{macrocode}
%
%    \begin{macrocode}
\kv@define@key{HoLogo}{variant}{%
  \ifx\HOLOGO@name\relax
    \@PackageError{hologo}{%
      Option `variant' is not available in \string\hologoSetup,%
      \MessageBreak
      Use \string\hologoLogoSetup\space instead%
    }\@ehc
  \else
    \edef\HOLOGO@temp{#1}%
    \ifx\HOLOGO@temp\ltx@empty
      \expandafter
      \let\csname HoLogoOpt@variant@\HOLOGO@name\endcsname\@undefined
    \else
      \ltx@IfUndefined{HoLogo@\HOLOGO@name @\HOLOGO@temp}{%
        \@PackageError{hologo}{%
          Unknown variant `\HOLOGO@temp' of logo `\HOLOGO@name'%
        }\@ehc
      }{%
        \expandafter
        \let\csname HoLogoOpt@variant@\HOLOGO@name\endcsname
            \HOLOGO@temp
      }%
    \fi
  \fi
}
%    \end{macrocode}
%
%    \begin{macro}{\HOLOGO@Variant}
%    \begin{macrocode}
\def\HOLOGO@Variant#1{%
  #1%
  \ltx@ifundefined{HoLogoOpt@variant@#1}{%
  }{%
    @\csname HoLogoOpt@variant@#1\endcsname
  }%
}
%    \end{macrocode}
%    \end{macro}
%
% \subsection{Break/no-break support}
%
%    \begin{macro}{\HOLOGO@space}
%    \begin{macrocode}
\def\HOLOGO@space{%
  \ltx@ifundefined{HoLogoOpt@spacebreak@\HOLOGO@name}{%
    \ltx@ifundefined{HoLogoOpt@break@\HOLOGO@name}{%
      \chardef\HOLOGO@temp=\HOLOGOOPT@spacebreak
    }{%
      \chardef\HOLOGO@temp=%
        \csname HoLogoOpt@break@\HOLOGO@name\endcsname
    }%
  }{%
    \chardef\HOLOGO@temp=%
      \csname HoLogoOpt@spacebreak@\HOLOGO@name\endcsname
  }%
  \ifcase\HOLOGO@temp
    \penalty10000 %
  \fi
  \ltx@space
}
%    \end{macrocode}
%    \end{macro}
%
%    \begin{macro}{\HOLOGO@hyphen}
%    \begin{macrocode}
\def\HOLOGO@hyphen{%
  \ltx@ifundefined{HoLogoOpt@hyphenbreak@\HOLOGO@name}{%
    \ltx@ifundefined{HoLogoOpt@break@\HOLOGO@name}{%
      \chardef\HOLOGO@temp=\HOLOGOOPT@hyphenbreak
    }{%
      \chardef\HOLOGO@temp=%
        \csname HoLogoOpt@break@\HOLOGO@name\endcsname
    }%
  }{%
    \chardef\HOLOGO@temp=%
      \csname HoLogoOpt@hyphenbreak@\HOLOGO@name\endcsname
  }%
  \ifcase\HOLOGO@temp
    \ltx@mbox{-}%
  \else
    -%
  \fi
}
%    \end{macrocode}
%    \end{macro}
%
%    \begin{macro}{\HOLOGO@discretionary}
%    \begin{macrocode}
\def\HOLOGO@discretionary{%
  \ltx@ifundefined{HoLogoOpt@discretionarybreak@\HOLOGO@name}{%
    \ltx@ifundefined{HoLogoOpt@break@\HOLOGO@name}{%
      \chardef\HOLOGO@temp=\HOLOGOOPT@discretionarybreak
    }{%
      \chardef\HOLOGO@temp=%
        \csname HoLogoOpt@break@\HOLOGO@name\endcsname
    }%
  }{%
    \chardef\HOLOGO@temp=%
      \csname HoLogoOpt@discretionarybreak@\HOLOGO@name\endcsname
  }%
  \ifcase\HOLOGO@temp
  \else
    \-%
  \fi
}
%    \end{macrocode}
%    \end{macro}
%
%    \begin{macro}{\HOLOGO@mbox}
%    \begin{macrocode}
\def\HOLOGO@mbox#1{%
  \ltx@ifundefined{HoLogoOpt@break@\HOLOGO@name}{%
    \chardef\HOLOGO@temp=\HOLOGOOPT@hyphenbreak
  }{%
    \chardef\HOLOGO@temp=%
      \csname HoLogoOpt@break@\HOLOGO@name\endcsname
  }%
  \ifcase\HOLOGO@temp
    \ltx@mbox{#1}%
  \else
    #1%
  \fi
}
%    \end{macrocode}
%    \end{macro}
%
% \subsection{Font support}
%
%    \begin{macro}{\HoLogoFont@font}
%    \begin{tabular}{@{}ll@{}}
%    |#1|:& logo name\\
%    |#2|:& font short name\\
%    |#3|:& text
%    \end{tabular}
%    \begin{macrocode}
\def\HoLogoFont@font#1#2#3{%
  \begingroup
    \ltx@IfUndefined{HoLogoFont@logo@#1.#2}{%
      \ltx@IfUndefined{HoLogoFont@font@#2}{%
        \@PackageWarning{hologo}{%
          Missing font `#2' for logo `#1'%
        }%
        #3%
      }{%
        \csname HoLogoFont@font@#2\endcsname{#3}%
      }%
    }{%
      \csname HoLogoFont@logo@#1.#2\endcsname{#3}%
    }%
  \endgroup
}
%    \end{macrocode}
%    \end{macro}
%
%    \begin{macro}{\HoLogoFont@Def}
%    \begin{macrocode}
\def\HoLogoFont@Def#1{%
  \expandafter\def\csname HoLogoFont@font@#1\endcsname
}
%    \end{macrocode}
%    \end{macro}
%    \begin{macro}{\HoLogoFont@LogoDef}
%    \begin{macrocode}
\def\HoLogoFont@LogoDef#1#2{%
  \expandafter\def\csname HoLogoFont@logo@#1.#2\endcsname
}
%    \end{macrocode}
%    \end{macro}
%
% \subsubsection{Font defaults}
%
%    \begin{macro}{\HoLogoFont@font@general}
%    \begin{macrocode}
\HoLogoFont@Def{general}{}%
%    \end{macrocode}
%    \end{macro}
%
%    \begin{macro}{\HoLogoFont@font@rm}
%    \begin{macrocode}
\ltx@IfUndefined{rmfamily}{%
  \ltx@IfUndefined{rm}{%
  }{%
    \HoLogoFont@Def{rm}{\rm}%
  }%
}{%
  \HoLogoFont@Def{rm}{\rmfamily}%
}
%    \end{macrocode}
%    \end{macro}
%
%    \begin{macro}{\HoLogoFont@font@sf}
%    \begin{macrocode}
\ltx@IfUndefined{sffamily}{%
  \ltx@IfUndefined{sf}{%
  }{%
    \HoLogoFont@Def{sf}{\sf}%
  }%
}{%
  \HoLogoFont@Def{sf}{\sffamily}%
}
%    \end{macrocode}
%    \end{macro}
%
%    \begin{macro}{\HoLogoFont@font@bibsf}
%    In case of \hologo{plainTeX} the original small caps
%    variant is used as default. In \hologo{LaTeX}
%    the definition of package \xpackage{dtklogos} \cite{dtklogos}
%    is used.
%\begin{quote}
%\begin{verbatim}
%\DeclareRobustCommand{\BibTeX}{%
%  B%
%  \kern-.05em%
%  \hbox{%
%    $\m@th$% %% force math size calculations
%    \csname S@\f@size\endcsname
%    \fontsize\sf@size\z@
%    \math@fontsfalse
%    \selectfont
%    I%
%    \kern-.025em%
%    B
%  }%
%  \kern-.08em%
%  \-%
%  \TeX
%}
%\end{verbatim}
%\end{quote}
%    \begin{macrocode}
\ltx@IfUndefined{selectfont}{%
  \ltx@IfUndefined{tensc}{%
    \font\tensc=cmcsc10\relax
  }{}%
  \HoLogoFont@Def{bibsf}{\tensc}%
}{%
  \HoLogoFont@Def{bibsf}{%
    $\mathsurround=0pt$%
    \csname S@\f@size\endcsname
    \fontsize\sf@size{0pt}%
    \math@fontsfalse
    \selectfont
  }%
}
%    \end{macrocode}
%    \end{macro}
%
%    \begin{macro}{\HoLogoFont@font@sc}
%    \begin{macrocode}
\ltx@IfUndefined{scshape}{%
  \ltx@IfUndefined{tensc}{%
    \font\tensc=cmcsc10\relax
  }{}%
  \HoLogoFont@Def{sc}{\tensc}%
}{%
  \HoLogoFont@Def{sc}{\scshape}%
}
%    \end{macrocode}
%    \end{macro}
%
%    \begin{macro}{\HoLogoFont@font@sy}
%    \begin{macrocode}
\ltx@IfUndefined{usefont}{%
  \ltx@IfUndefined{tensy}{%
  }{%
    \HoLogoFont@Def{sy}{\tensy}%
  }%
}{%
  \HoLogoFont@Def{sy}{%
    \usefont{OMS}{cmsy}{m}{n}%
  }%
}
%    \end{macrocode}
%    \end{macro}
%
%    \begin{macro}{\HoLogoFont@font@logo}
%    \begin{macrocode}
\begingroup
  \def\x{LaTeX2e}%
\expandafter\endgroup
\ifx\fmtname\x
  \ltx@IfUndefined{logofamily}{%
    \DeclareRobustCommand\logofamily{%
      \not@math@alphabet\logofamily\relax
      \fontencoding{U}%
      \fontfamily{logo}%
      \selectfont
    }%
  }{}%
  \ltx@IfUndefined{logofamily}{%
  }{%
    \HoLogoFont@Def{logo}{\logofamily}%
  }%
\else
  \ltx@IfUndefined{tenlogo}{%
    \font\tenlogo=logo10\relax
  }{}%
  \HoLogoFont@Def{logo}{\tenlogo}%
\fi
%    \end{macrocode}
%    \end{macro}
%
% \subsubsection{Font setup}
%
%    \begin{macro}{\hologoFontSetup}
%    \begin{macrocode}
\def\hologoFontSetup{%
  \let\HOLOGO@name\relax
  \HOLOGO@FontSetup
}
%    \end{macrocode}
%    \end{macro}
%
%    \begin{macro}{\hologoLogoFontSetup}
%    \begin{macrocode}
\def\hologoLogoFontSetup#1{%
  \edef\HOLOGO@name{#1}%
  \ltx@IfUndefined{HoLogo@\HOLOGO@name}{%
    \@PackageError{hologo}{%
      Unknown logo `\HOLOGO@name'%
    }\@ehc
    \ltx@gobble
  }{%
    \HOLOGO@FontSetup
  }%
}
%    \end{macrocode}
%    \end{macro}
%
%    \begin{macro}{\HOLOGO@FontSetup}
%    \begin{macrocode}
\def\HOLOGO@FontSetup{%
  \kvsetkeys{HoLogoFont}%
}
%    \end{macrocode}
%    \end{macro}
%
%    \begin{macrocode}
\def\HOLOGO@temp#1{%
  \kv@define@key{HoLogoFont}{#1}{%
    \ifx\HOLOGO@name\relax
      \HoLogoFont@Def{#1}{##1}%
    \else
      \HoLogoFont@LogoDef\HOLOGO@name{#1}{##1}%
    \fi
  }%
}
\HOLOGO@temp{general}
\HOLOGO@temp{sf}
%    \end{macrocode}
%
% \subsection{Generic logo commands}
%
%    \begin{macrocode}
\HOLOGO@IfExists\hologo{%
  \@PackageError{hologo}{%
    \string\hologo\ltx@space is already defined.\MessageBreak
    Package loading is aborted%
  }\@ehc
  \HOLOGO@AtEnd
}%
\HOLOGO@IfExists\hologoRobust{%
  \@PackageError{hologo}{%
    \string\hologoRobust\ltx@space is already defined.\MessageBreak
    Package loading is aborted%
  }\@ehc
  \HOLOGO@AtEnd
}%
%    \end{macrocode}
%
% \subsubsection{\cs{hologo} and friends}
%
%    \begin{macrocode}
\ifluatex
  \expandafter\ltx@firstofone
\else
  \expandafter\ltx@gobble
\fi
{%
  \ltx@IfUndefined{ifincsname}{%
    \ifnum\luatexversion<36 %
      \expandafter\ltx@gobble
    \else
      \expandafter\ltx@firstofone
    \fi
    {%
      \begingroup
        \ifcase0%
            \directlua{%
              if tex.enableprimitives then %
                tex.enableprimitives('HOLOGO@', {'ifincsname'})%
              else %
                tex.print('1')%
              end%
            }%
            \ifx\HOLOGO@ifincsname\@undefined 1\fi%
            \relax
          \expandafter\ltx@firstofone
        \else
          \endgroup
          \expandafter\ltx@gobble
        \fi
        {%
          \global\let\ifincsname\HOLOGO@ifincsname
        }%
      \HOLOGO@temp
    }%
  }{}%
}
%    \end{macrocode}
%    \begin{macrocode}
\ltx@IfUndefined{ifincsname}{%
  \catcode`$=14 %
}{%
  \catcode`$=9 %
}
%    \end{macrocode}
%
%    \begin{macro}{\hologo}
%    \begin{macrocode}
\def\hologo#1{%
$ \ifincsname
$   \ltx@ifundefined{HoLogoCs@\HOLOGO@Variant{#1}}{%
$     #1%
$   }{%
$     \csname HoLogoCs@\HOLOGO@Variant{#1}\endcsname\ltx@firstoftwo
$   }%
$ \else
    \HOLOGO@IfExists\texorpdfstring\texorpdfstring\ltx@firstoftwo
    {%
      \hologoRobust{#1}%
    }{%
      \ltx@ifundefined{HoLogoBkm@\HOLOGO@Variant{#1}}{%
        \ltx@ifundefined{HoLogo@#1}{?#1?}{#1}%
      }{%
        \csname HoLogoBkm@\HOLOGO@Variant{#1}\endcsname
        \ltx@firstoftwo
      }%
    }%
$ \fi
}
%    \end{macrocode}
%    \end{macro}
%    \begin{macro}{\Hologo}
%    \begin{macrocode}
\def\Hologo#1{%
$ \ifincsname
$   \ltx@ifundefined{HoLogoCs@\HOLOGO@Variant{#1}}{%
$     #1%
$   }{%
$     \csname HoLogoCs@\HOLOGO@Variant{#1}\endcsname\ltx@secondoftwo
$   }%
$ \else
    \HOLOGO@IfExists\texorpdfstring\texorpdfstring\ltx@firstoftwo
    {%
      \HologoRobust{#1}%
    }{%
      \ltx@ifundefined{HoLogoBkm@\HOLOGO@Variant{#1}}{%
        \ltx@ifundefined{HoLogo@#1}{?#1?}{#1}%
      }{%
        \csname HoLogoBkm@\HOLOGO@Variant{#1}\endcsname
        \ltx@secondoftwo
      }%
    }%
$ \fi
}
%    \end{macrocode}
%    \end{macro}
%
%    \begin{macro}{\hologoVariant}
%    \begin{macrocode}
\def\hologoVariant#1#2{%
  \ifx\relax#2\relax
    \hologo{#1}%
  \else
$   \ifincsname
$     \ltx@ifundefined{HoLogoCs@#1@#2}{%
$       #1%
$     }{%
$       \csname HoLogoCs@#1@#2\endcsname\ltx@firstoftwo
$     }%
$   \else
      \HOLOGO@IfExists\texorpdfstring\texorpdfstring\ltx@firstoftwo
      {%
        \hologoVariantRobust{#1}{#2}%
      }{%
        \ltx@ifundefined{HoLogoBkm@#1@#2}{%
          \ltx@ifundefined{HoLogo@#1}{?#1?}{#1}%
        }{%
          \csname HoLogoBkm@#1@#2\endcsname
          \ltx@firstoftwo
        }%
      }%
$   \fi
  \fi
}
%    \end{macrocode}
%    \end{macro}
%    \begin{macro}{\HologoVariant}
%    \begin{macrocode}
\def\HologoVariant#1#2{%
  \ifx\relax#2\relax
    \Hologo{#1}%
  \else
$   \ifincsname
$     \ltx@ifundefined{HoLogoCs@#1@#2}{%
$       #1%
$     }{%
$       \csname HoLogoCs@#1@#2\endcsname\ltx@secondoftwo
$     }%
$   \else
      \HOLOGO@IfExists\texorpdfstring\texorpdfstring\ltx@firstoftwo
      {%
        \HologoVariantRobust{#1}{#2}%
      }{%
        \ltx@ifundefined{HoLogoBkm@#1@#2}{%
          \ltx@ifundefined{HoLogo@#1}{?#1?}{#1}%
        }{%
          \csname HoLogoBkm@#1@#2\endcsname
          \ltx@secondoftwo
        }%
      }%
$   \fi
  \fi
}
%    \end{macrocode}
%    \end{macro}
%
%    \begin{macrocode}
\catcode`\$=3 %
%    \end{macrocode}
%
% \subsubsection{\cs{hologoRobust} and friends}
%
%    \begin{macro}{\hologoRobust}
%    \begin{macrocode}
\ltx@IfUndefined{protected}{%
  \ltx@IfUndefined{DeclareRobustCommand}{%
    \def\hologoRobust#1%
  }{%
    \DeclareRobustCommand*\hologoRobust[1]%
  }%
}{%
  \protected\def\hologoRobust#1%
}%
{%
  \edef\HOLOGO@name{#1}%
  \ltx@IfUndefined{HoLogo@\HOLOGO@Variant\HOLOGO@name}{%
    \@PackageError{hologo}{%
      Unknown logo `\HOLOGO@name'%
    }\@ehc
    ?\HOLOGO@name?%
  }{%
    \ltx@IfUndefined{ver@tex4ht.sty}{%
      \HoLogoFont@font\HOLOGO@name{general}{%
        \csname HoLogo@\HOLOGO@Variant\HOLOGO@name\endcsname
        \ltx@firstoftwo
      }%
    }{%
      \ltx@IfUndefined{HoLogoHtml@\HOLOGO@Variant\HOLOGO@name}{%
        \HOLOGO@name
      }{%
        \csname HoLogoHtml@\HOLOGO@Variant\HOLOGO@name\endcsname
        \ltx@firstoftwo
      }%
    }%
  }%
}
%    \end{macrocode}
%    \end{macro}
%    \begin{macro}{\HologoRobust}
%    \begin{macrocode}
\ltx@IfUndefined{protected}{%
  \ltx@IfUndefined{DeclareRobustCommand}{%
    \def\HologoRobust#1%
  }{%
    \DeclareRobustCommand*\HologoRobust[1]%
  }%
}{%
  \protected\def\HologoRobust#1%
}%
{%
  \edef\HOLOGO@name{#1}%
  \ltx@IfUndefined{HoLogo@\HOLOGO@Variant\HOLOGO@name}{%
    \@PackageError{hologo}{%
      Unknown logo `\HOLOGO@name'%
    }\@ehc
    ?\HOLOGO@name?%
  }{%
    \ltx@IfUndefined{ver@tex4ht.sty}{%
      \HoLogoFont@font\HOLOGO@name{general}{%
        \csname HoLogo@\HOLOGO@Variant\HOLOGO@name\endcsname
        \ltx@secondoftwo
      }%
    }{%
      \ltx@IfUndefined{HoLogoHtml@\HOLOGO@Variant\HOLOGO@name}{%
        \expandafter\HOLOGO@Uppercase\HOLOGO@name
      }{%
        \csname HoLogoHtml@\HOLOGO@Variant\HOLOGO@name\endcsname
        \ltx@secondoftwo
      }%
    }%
  }%
}
%    \end{macrocode}
%    \end{macro}
%    \begin{macro}{\hologoVariantRobust}
%    \begin{macrocode}
\ltx@IfUndefined{protected}{%
  \ltx@IfUndefined{DeclareRobustCommand}{%
    \def\hologoVariantRobust#1#2%
  }{%
    \DeclareRobustCommand*\hologoVariantRobust[2]%
  }%
}{%
  \protected\def\hologoVariantRobust#1#2%
}%
{%
  \begingroup
    \hologoLogoSetup{#1}{variant={#2}}%
    \hologoRobust{#1}%
  \endgroup
}
%    \end{macrocode}
%    \end{macro}
%    \begin{macro}{\HologoVariantRobust}
%    \begin{macrocode}
\ltx@IfUndefined{protected}{%
  \ltx@IfUndefined{DeclareRobustCommand}{%
    \def\HologoVariantRobust#1#2%
  }{%
    \DeclareRobustCommand*\HologoVariantRobust[2]%
  }%
}{%
  \protected\def\HologoVariantRobust#1#2%
}%
{%
  \begingroup
    \hologoLogoSetup{#1}{variant={#2}}%
    \HologoRobust{#1}%
  \endgroup
}
%    \end{macrocode}
%    \end{macro}
%
%    \begin{macro}{\hologorobust}
%    Macro \cs{hologorobust} is only defined for compatibility.
%    Its use is deprecated.
%    \begin{macrocode}
\def\hologorobust{\hologoRobust}
%    \end{macrocode}
%    \end{macro}
%
% \subsection{Helpers}
%
%    \begin{macro}{\HOLOGO@Uppercase}
%    Macro \cs{HOLOGO@Uppercase} is restricted to \cs{uppercase},
%    because \hologo{plainTeX} or \hologo{iniTeX} do not provide
%    \cs{MakeUppercase}.
%    \begin{macrocode}
\def\HOLOGO@Uppercase#1{\uppercase{#1}}
%    \end{macrocode}
%    \end{macro}
%
%    \begin{macro}{\HOLOGO@PdfdocUnicode}
%    \begin{macrocode}
\def\HOLOGO@PdfdocUnicode{%
  \ifx\ifHy@unicode\iftrue
    \expandafter\ltx@secondoftwo
  \else
    \expandafter\ltx@firstoftwo
  \fi
}
%    \end{macrocode}
%    \end{macro}
%
%    \begin{macro}{\HOLOGO@Math}
%    \begin{macrocode}
\def\HOLOGO@MathSetup{%
  \mathsurround0pt\relax
  \HOLOGO@IfExists\f@series{%
    \if b\expandafter\ltx@car\f@series x\@nil
      \csname boldmath\endcsname
   \fi
  }{}%
}
%    \end{macrocode}
%    \end{macro}
%
%    \begin{macro}{\HOLOGO@TempDimen}
%    \begin{macrocode}
\dimendef\HOLOGO@TempDimen=\ltx@zero
%    \end{macrocode}
%    \end{macro}
%    \begin{macro}{\HOLOGO@NegativeKerning}
%    \begin{macrocode}
\def\HOLOGO@NegativeKerning#1{%
  \begingroup
    \HOLOGO@TempDimen=0pt\relax
    \comma@parse@normalized{#1}{%
      \ifdim\HOLOGO@TempDimen=0pt %
        \expandafter\HOLOGO@@NegativeKerning\comma@entry
      \fi
      \ltx@gobble
    }%
    \ifdim\HOLOGO@TempDimen<0pt %
      \kern\HOLOGO@TempDimen
    \fi
  \endgroup
}
%    \end{macrocode}
%    \end{macro}
%    \begin{macro}{\HOLOGO@@NegativeKerning}
%    \begin{macrocode}
\def\HOLOGO@@NegativeKerning#1#2{%
  \setbox\ltx@zero\hbox{#1#2}%
  \HOLOGO@TempDimen=\wd\ltx@zero
  \setbox\ltx@zero\hbox{#1\kern0pt#2}%
  \advance\HOLOGO@TempDimen by -\wd\ltx@zero
}
%    \end{macrocode}
%    \end{macro}
%
%    \begin{macro}{\HOLOGO@SpaceFactor}
%    \begin{macrocode}
\def\HOLOGO@SpaceFactor{%
  \spacefactor1000 %
}
%    \end{macrocode}
%    \end{macro}
%
%    \begin{macro}{\HOLOGO@Span}
%    \begin{macrocode}
\def\HOLOGO@Span#1#2{%
  \HCode{<span class="HoLogo-#1">}%
  #2%
  \HCode{</span>}%
}
%    \end{macrocode}
%    \end{macro}
%
% \subsubsection{Text subscript}
%
%    \begin{macro}{\HOLOGO@SubScript}%
%    \begin{macrocode}
\def\HOLOGO@SubScript#1{%
  \ltx@IfUndefined{textsubscript}{%
    \ltx@IfUndefined{text}{%
      \ltx@mbox{%
        \mathsurround=0pt\relax
        $%
          _{%
            \ltx@IfUndefined{sf@size}{%
              \mathrm{#1}%
            }{%
              \mbox{%
                \fontsize\sf@size{0pt}\selectfont
                #1%
              }%
            }%
          }%
        $%
      }%
    }{%
      \ltx@mbox{%
        \mathsurround=0pt\relax
        $_{\text{#1}}$%
      }%
    }%
  }{%
    \textsubscript{#1}%
  }%
}
%    \end{macrocode}
%    \end{macro}
%
% \subsection{\hologo{TeX} and friends}
%
% \subsubsection{\hologo{TeX}}
%
%    \begin{macro}{\HoLogo@TeX}
%    Source: \hologo{LaTeX} kernel.
%    \begin{macrocode}
\def\HoLogo@TeX#1{%
  T\kern-.1667em\lower.5ex\hbox{E}\kern-.125emX\HOLOGO@SpaceFactor
}
%    \end{macrocode}
%    \end{macro}
%    \begin{macro}{\HoLogoHtml@TeX}
%    \begin{macrocode}
\def\HoLogoHtml@TeX#1{%
  \HoLogoCss@TeX
  \HOLOGO@Span{TeX}{%
    T%
    \HOLOGO@Span{e}{%
      E%
    }%
    X%
  }%
}
%    \end{macrocode}
%    \end{macro}
%    \begin{macro}{\HoLogoCss@TeX}
%    \begin{macrocode}
\def\HoLogoCss@TeX{%
  \Css{%
    span.HoLogo-TeX span.HoLogo-e{%
      position:relative;%
      top:.5ex;%
      margin-left:-.1667em;%
      margin-right:-.125em;%
    }%
  }%
  \Css{%
    a span.HoLogo-TeX span.HoLogo-e{%
      text-decoration:none;%
    }%
  }%
  \global\let\HoLogoCss@TeX\relax
}
%    \end{macrocode}
%    \end{macro}
%
% \subsubsection{\hologo{plainTeX}}
%
%    \begin{macro}{\HoLogo@plainTeX@space}
%    Source: ``The \hologo{TeX}book''
%    \begin{macrocode}
\def\HoLogo@plainTeX@space#1{%
  \HOLOGO@mbox{#1{p}{P}lain}\HOLOGO@space\hologo{TeX}%
}
%    \end{macrocode}
%    \end{macro}
%    \begin{macro}{\HoLogoCs@plainTeX@space}
%    \begin{macrocode}
\def\HoLogoCs@plainTeX@space#1{#1{p}{P}lain TeX}%
%    \end{macrocode}
%    \end{macro}
%    \begin{macro}{\HoLogoBkm@plainTeX@space}
%    \begin{macrocode}
\def\HoLogoBkm@plainTeX@space#1{%
  #1{p}{P}lain \hologo{TeX}%
}
%    \end{macrocode}
%    \end{macro}
%    \begin{macro}{\HoLogoHtml@plainTeX@space}
%    \begin{macrocode}
\def\HoLogoHtml@plainTeX@space#1{%
  #1{p}{P}lain \hologo{TeX}%
}
%    \end{macrocode}
%    \end{macro}
%
%    \begin{macro}{\HoLogo@plainTeX@hyphen}
%    \begin{macrocode}
\def\HoLogo@plainTeX@hyphen#1{%
  \HOLOGO@mbox{#1{p}{P}lain}\HOLOGO@hyphen\hologo{TeX}%
}
%    \end{macrocode}
%    \end{macro}
%    \begin{macro}{\HoLogoCs@plainTeX@hyphen}
%    \begin{macrocode}
\def\HoLogoCs@plainTeX@hyphen#1{#1{p}{P}lain-TeX}
%    \end{macrocode}
%    \end{macro}
%    \begin{macro}{\HoLogoBkm@plainTeX@hyphen}
%    \begin{macrocode}
\def\HoLogoBkm@plainTeX@hyphen#1{%
  #1{p}{P}lain-\hologo{TeX}%
}
%    \end{macrocode}
%    \end{macro}
%    \begin{macro}{\HoLogoHtml@plainTeX@hyphen}
%    \begin{macrocode}
\def\HoLogoHtml@plainTeX@hyphen#1{%
  #1{p}{P}lain-\hologo{TeX}%
}
%    \end{macrocode}
%    \end{macro}
%
%    \begin{macro}{\HoLogo@plainTeX@runtogether}
%    \begin{macrocode}
\def\HoLogo@plainTeX@runtogether#1{%
  \HOLOGO@mbox{#1{p}{P}lain\hologo{TeX}}%
}
%    \end{macrocode}
%    \end{macro}
%    \begin{macro}{\HoLogoCs@plainTeX@runtogether}
%    \begin{macrocode}
\def\HoLogoCs@plainTeX@runtogether#1{#1{p}{P}lainTeX}
%    \end{macrocode}
%    \end{macro}
%    \begin{macro}{\HoLogoBkm@plainTeX@runtogether}
%    \begin{macrocode}
\def\HoLogoBkm@plainTeX@runtogether#1{%
  #1{p}{P}lain\hologo{TeX}%
}
%    \end{macrocode}
%    \end{macro}
%    \begin{macro}{\HoLogoHtml@plainTeX@runtogether}
%    \begin{macrocode}
\def\HoLogoHtml@plainTeX@runtogether#1{%
  #1{p}{P}lain\hologo{TeX}%
}
%    \end{macrocode}
%    \end{macro}
%
%    \begin{macro}{\HoLogo@plainTeX}
%    \begin{macrocode}
\def\HoLogo@plainTeX{\HoLogo@plainTeX@space}
%    \end{macrocode}
%    \end{macro}
%    \begin{macro}{\HoLogoCs@plainTeX}
%    \begin{macrocode}
\def\HoLogoCs@plainTeX{\HoLogoCs@plainTeX@space}
%    \end{macrocode}
%    \end{macro}
%    \begin{macro}{\HoLogoBkm@plainTeX}
%    \begin{macrocode}
\def\HoLogoBkm@plainTeX{\HoLogoBkm@plainTeX@space}
%    \end{macrocode}
%    \end{macro}
%    \begin{macro}{\HoLogoHtml@plainTeX}
%    \begin{macrocode}
\def\HoLogoHtml@plainTeX{\HoLogoHtml@plainTeX@space}
%    \end{macrocode}
%    \end{macro}
%
% \subsubsection{\hologo{LaTeX}}
%
%    Source: \hologo{LaTeX} kernel.
%\begin{quote}
%\begin{verbatim}
%\DeclareRobustCommand{\LaTeX}{%
%  L%
%  \kern-.36em%
%  {%
%    \sbox\z@ T%
%    \vbox to\ht\z@{%
%      \hbox{%
%        \check@mathfonts
%        \fontsize\sf@size\z@
%        \math@fontsfalse
%        \selectfont
%        A%
%      }%
%      \vss
%    }%
%  }%
%  \kern-.15em%
%  \TeX
%}
%\end{verbatim}
%\end{quote}
%
%    \begin{macro}{\HoLogo@La}
%    \begin{macrocode}
\def\HoLogo@La#1{%
  L%
  \kern-.36em%
  \begingroup
    \setbox\ltx@zero\hbox{T}%
    \vbox to\ht\ltx@zero{%
      \hbox{%
        \ltx@ifundefined{check@mathfonts}{%
          \csname sevenrm\endcsname
        }{%
          \check@mathfonts
          \fontsize\sf@size{0pt}%
          \math@fontsfalse\selectfont
        }%
        A%
      }%
      \vss
    }%
  \endgroup
}
%    \end{macrocode}
%    \end{macro}
%
%    \begin{macro}{\HoLogo@LaTeX}
%    Source: \hologo{LaTeX} kernel.
%    \begin{macrocode}
\def\HoLogo@LaTeX#1{%
  \hologo{La}%
  \kern-.15em%
  \hologo{TeX}%
}
%    \end{macrocode}
%    \end{macro}
%    \begin{macro}{\HoLogoHtml@LaTeX}
%    \begin{macrocode}
\def\HoLogoHtml@LaTeX#1{%
  \HoLogoCss@LaTeX
  \HOLOGO@Span{LaTeX}{%
    L%
    \HOLOGO@Span{a}{%
      A%
    }%
    \hologo{TeX}%
  }%
}
%    \end{macrocode}
%    \end{macro}
%    \begin{macro}{\HoLogoCss@LaTeX}
%    \begin{macrocode}
\def\HoLogoCss@LaTeX{%
  \Css{%
    span.HoLogo-LaTeX span.HoLogo-a{%
      position:relative;%
      top:-.5ex;%
      margin-left:-.36em;%
      margin-right:-.15em;%
      font-size:85\%;%
    }%
  }%
  \global\let\HoLogoCss@LaTeX\relax
}
%    \end{macrocode}
%    \end{macro}
%
% \subsubsection{\hologo{(La)TeX}}
%
%    \begin{macro}{\HoLogo@LaTeXTeX}
%    The kerning around the parentheses is taken
%    from package \xpackage{dtklogos} \cite{dtklogos}.
%\begin{quote}
%\begin{verbatim}
%\DeclareRobustCommand{\LaTeXTeX}{%
%  (%
%  \kern-.15em%
%  L%
%  \kern-.36em%
%  {%
%    \sbox\z@ T%
%    \vbox to\ht0{%
%      \hbox{%
%        $\m@th$%
%        \csname S@\f@size\endcsname
%        \fontsize\sf@size\z@
%        \math@fontsfalse
%        \selectfont
%        A%
%      }%
%      \vss
%    }%
%  }%
%  \kern-.2em%
%  )%
%  \kern-.15em%
%  \TeX
%}
%\end{verbatim}
%\end{quote}
%    \begin{macrocode}
\def\HoLogo@LaTeXTeX#1{%
  (%
  \kern-.15em%
  \hologo{La}%
  \kern-.2em%
  )%
  \kern-.15em%
  \hologo{TeX}%
}
%    \end{macrocode}
%    \end{macro}
%    \begin{macro}{\HoLogoBkm@LaTeXTeX}
%    \begin{macrocode}
\def\HoLogoBkm@LaTeXTeX#1{(La)TeX}
%    \end{macrocode}
%    \end{macro}
%
%    \begin{macro}{\HoLogo@(La)TeX}
%    \begin{macrocode}
\expandafter
\let\csname HoLogo@(La)TeX\endcsname\HoLogo@LaTeXTeX
%    \end{macrocode}
%    \end{macro}
%    \begin{macro}{\HoLogoBkm@(La)TeX}
%    \begin{macrocode}
\expandafter
\let\csname HoLogoBkm@(La)TeX\endcsname\HoLogoBkm@LaTeXTeX
%    \end{macrocode}
%    \end{macro}
%    \begin{macro}{\HoLogoHtml@LaTeXTeX}
%    \begin{macrocode}
\def\HoLogoHtml@LaTeXTeX#1{%
  \HoLogoCss@LaTeXTeX
  \HOLOGO@Span{LaTeXTeX}{%
    (%
    \HOLOGO@Span{L}{L}%
    \HOLOGO@Span{a}{A}%
    \HOLOGO@Span{ParenRight}{)}%
    \hologo{TeX}%
  }%
}
%    \end{macrocode}
%    \end{macro}
%    \begin{macro}{\HoLogoHtml@(La)TeX}
%    Kerning after opening parentheses and before closing parentheses
%    is $-0.1$\,em. The original values $-0.15$\,em
%    looked too ugly for a serif font.
%    \begin{macrocode}
\expandafter
\let\csname HoLogoHtml@(La)TeX\endcsname\HoLogoHtml@LaTeXTeX
%    \end{macrocode}
%    \end{macro}
%    \begin{macro}{\HoLogoCss@LaTeXTeX}
%    \begin{macrocode}
\def\HoLogoCss@LaTeXTeX{%
  \Css{%
    span.HoLogo-LaTeXTeX span.HoLogo-L{%
      margin-left:-.1em;%
    }%
  }%
  \Css{%
    span.HoLogo-LaTeXTeX span.HoLogo-a{%
      position:relative;%
      top:-.5ex;%
      margin-left:-.36em;%
      margin-right:-.1em;%
      font-size:85\%;%
    }%
  }%
  \Css{%
    span.HoLogo-LaTeXTeX span.HoLogo-ParenRight{%
      margin-right:-.15em;%
    }%
  }%
  \global\let\HoLogoCss@LaTeXTeX\relax
}
%    \end{macrocode}
%    \end{macro}
%
% \subsubsection{\hologo{LaTeXe}}
%
%    \begin{macro}{\HoLogo@LaTeXe}
%    Source: \hologo{LaTeX} kernel
%    \begin{macrocode}
\def\HoLogo@LaTeXe#1{%
  \hologo{LaTeX}%
  \kern.15em%
  \hbox{%
    \HOLOGO@MathSetup
    2%
    $_{\textstyle\varepsilon}$%
  }%
}
%    \end{macrocode}
%    \end{macro}
%
%    \begin{macro}{\HoLogoCs@LaTeXe}
%    \begin{macrocode}
\ifnum64=`\^^^^0040\relax % test for big chars of LuaTeX/XeTeX
  \catcode`\$=9 %
  \catcode`\&=14 %
\else
  \catcode`\$=14 %
  \catcode`\&=9 %
\fi
\def\HoLogoCs@LaTeXe#1{%
  LaTeX2%
$ \string ^^^^0395%
& e%
}%
\catcode`\$=3 %
\catcode`\&=4 %
%    \end{macrocode}
%    \end{macro}
%
%    \begin{macro}{\HoLogoBkm@LaTeXe}
%    \begin{macrocode}
\def\HoLogoBkm@LaTeXe#1{%
  \hologo{LaTeX}%
  2%
  \HOLOGO@PdfdocUnicode{e}{\textepsilon}%
}
%    \end{macrocode}
%    \end{macro}
%
%    \begin{macro}{\HoLogoHtml@LaTeXe}
%    \begin{macrocode}
\def\HoLogoHtml@LaTeXe#1{%
  \HoLogoCss@LaTeXe
  \HOLOGO@Span{LaTeX2e}{%
    \hologo{LaTeX}%
    \HOLOGO@Span{2}{2}%
    \HOLOGO@Span{e}{%
      \HOLOGO@MathSetup
      \ensuremath{\textstyle\varepsilon}%
    }%
  }%
}
%    \end{macrocode}
%    \end{macro}
%    \begin{macro}{\HoLogoCss@LaTeXe}
%    \begin{macrocode}
\def\HoLogoCss@LaTeXe{%
  \Css{%
    span.HoLogo-LaTeX2e span.HoLogo-2{%
      padding-left:.15em;%
    }%
  }%
  \Css{%
    span.HoLogo-LaTeX2e span.HoLogo-e{%
      position:relative;%
      top:.35ex;%
      text-decoration:none;%
    }%
  }%
  \global\let\HoLogoCss@LaTeXe\relax
}
%    \end{macrocode}
%    \end{macro}
%
%    \begin{macro}{\HoLogo@LaTeX2e}
%    \begin{macrocode}
\expandafter
\let\csname HoLogo@LaTeX2e\endcsname\HoLogo@LaTeXe
%    \end{macrocode}
%    \end{macro}
%    \begin{macro}{\HoLogoCs@LaTeX2e}
%    \begin{macrocode}
\expandafter
\let\csname HoLogoCs@LaTeX2e\endcsname\HoLogoCs@LaTeXe
%    \end{macrocode}
%    \end{macro}
%    \begin{macro}{\HoLogoBkm@LaTeX2e}
%    \begin{macrocode}
\expandafter
\let\csname HoLogoBkm@LaTeX2e\endcsname\HoLogoBkm@LaTeXe
%    \end{macrocode}
%    \end{macro}
%    \begin{macro}{\HoLogoHtml@LaTeX2e}
%    \begin{macrocode}
\expandafter
\let\csname HoLogoHtml@LaTeX2e\endcsname\HoLogoHtml@LaTeXe
%    \end{macrocode}
%    \end{macro}
%
% \subsubsection{\hologo{LaTeX3}}
%
%    \begin{macro}{\HoLogo@LaTeX3}
%    Source: \hologo{LaTeX} kernel
%    \begin{macrocode}
\expandafter\def\csname HoLogo@LaTeX3\endcsname#1{%
  \hologo{LaTeX}%
  3%
}
%    \end{macrocode}
%    \end{macro}
%
%    \begin{macro}{\HoLogoBkm@LaTeX3}
%    \begin{macrocode}
\expandafter\def\csname HoLogoBkm@LaTeX3\endcsname#1{%
  \hologo{LaTeX}%
  3%
}
%    \end{macrocode}
%    \end{macro}
%    \begin{macro}{\HoLogoHtml@LaTeX3}
%    \begin{macrocode}
\expandafter
\let\csname HoLogoHtml@LaTeX3\expandafter\endcsname
\csname HoLogo@LaTeX3\endcsname
%    \end{macrocode}
%    \end{macro}
%
% \subsubsection{\hologo{LaTeXML}}
%
%    \begin{macro}{\HoLogo@LaTeXML}
%    \begin{macrocode}
\def\HoLogo@LaTeXML#1{%
  \HOLOGO@mbox{%
    \hologo{La}%
    \kern-.15em%
    T%
    \kern-.1667em%
    \lower.5ex\hbox{E}%
    \kern-.125em%
    \HoLogoFont@font{LaTeXML}{sc}{xml}%
  }%
}
%    \end{macrocode}
%    \end{macro}
%    \begin{macro}{\HoLogoHtml@pdfLaTeX}
%    \begin{macrocode}
\def\HoLogoHtml@LaTeXML#1{%
  \HOLOGO@Span{LaTeXML}{%
    \HoLogoCss@LaTeX
    \HoLogoCss@TeX
    \HOLOGO@Span{LaTeX}{%
      L%
      \HOLOGO@Span{a}{%
        A%
      }%
    }%
    \HOLOGO@Span{TeX}{%
      T%
      \HOLOGO@Span{e}{%
        E%
      }%
    }%
    \HCode{<span style="font-variant: small-caps;">}%
    xml%
    \HCode{</span>}%
  }%
}
%    \end{macrocode}
%    \end{macro}
%
% \subsubsection{\hologo{eTeX}}
%
%    \begin{macro}{\HoLogo@eTeX}
%    Source: package \xpackage{etex}
%    \begin{macrocode}
\def\HoLogo@eTeX#1{%
  \ltx@mbox{%
    \HOLOGO@MathSetup
    $\varepsilon$%
    -%
    \HOLOGO@NegativeKerning{-T,T-,To}%
    \hologo{TeX}%
  }%
}
%    \end{macrocode}
%    \end{macro}
%    \begin{macro}{\HoLogoCs@eTeX}
%    \begin{macrocode}
\ifnum64=`\^^^^0040\relax % test for big chars of LuaTeX/XeTeX
  \catcode`\$=9 %
  \catcode`\&=14 %
\else
  \catcode`\$=14 %
  \catcode`\&=9 %
\fi
\def\HoLogoCs@eTeX#1{%
$ #1{\string ^^^^0395}{\string ^^^^03b5}%
& #1{e}{E}%
  TeX%
}%
\catcode`\$=3 %
\catcode`\&=4 %
%    \end{macrocode}
%    \end{macro}
%    \begin{macro}{\HoLogoBkm@eTeX}
%    \begin{macrocode}
\def\HoLogoBkm@eTeX#1{%
  \HOLOGO@PdfdocUnicode{#1{e}{E}}{\textepsilon}%
  -%
  \hologo{TeX}%
}
%    \end{macrocode}
%    \end{macro}
%    \begin{macro}{\HoLogoHtml@eTeX}
%    \begin{macrocode}
\def\HoLogoHtml@eTeX#1{%
  \ltx@mbox{%
    \HOLOGO@MathSetup
    $\varepsilon$%
    -%
    \hologo{TeX}%
  }%
}
%    \end{macrocode}
%    \end{macro}
%
% \subsubsection{\hologo{iniTeX}}
%
%    \begin{macro}{\HoLogo@iniTeX}
%    \begin{macrocode}
\def\HoLogo@iniTeX#1{%
  \HOLOGO@mbox{%
    #1{i}{I}ni\hologo{TeX}%
  }%
}
%    \end{macrocode}
%    \end{macro}
%    \begin{macro}{\HoLogoCs@iniTeX}
%    \begin{macrocode}
\def\HoLogoCs@iniTeX#1{#1{i}{I}niTeX}
%    \end{macrocode}
%    \end{macro}
%    \begin{macro}{\HoLogoBkm@iniTeX}
%    \begin{macrocode}
\def\HoLogoBkm@iniTeX#1{%
  #1{i}{I}ni\hologo{TeX}%
}
%    \end{macrocode}
%    \end{macro}
%    \begin{macro}{\HoLogoHtml@iniTeX}
%    \begin{macrocode}
\let\HoLogoHtml@iniTeX\HoLogo@iniTeX
%    \end{macrocode}
%    \end{macro}
%
% \subsubsection{\hologo{virTeX}}
%
%    \begin{macro}{\HoLogo@virTeX}
%    \begin{macrocode}
\def\HoLogo@virTeX#1{%
  \HOLOGO@mbox{%
    #1{v}{V}ir\hologo{TeX}%
  }%
}
%    \end{macrocode}
%    \end{macro}
%    \begin{macro}{\HoLogoCs@virTeX}
%    \begin{macrocode}
\def\HoLogoCs@virTeX#1{#1{v}{V}irTeX}
%    \end{macrocode}
%    \end{macro}
%    \begin{macro}{\HoLogoBkm@virTeX}
%    \begin{macrocode}
\def\HoLogoBkm@virTeX#1{%
  #1{v}{V}ir\hologo{TeX}%
}
%    \end{macrocode}
%    \end{macro}
%    \begin{macro}{\HoLogoHtml@virTeX}
%    \begin{macrocode}
\let\HoLogoHtml@virTeX\HoLogo@virTeX
%    \end{macrocode}
%    \end{macro}
%
% \subsubsection{\hologo{SliTeX}}
%
% \paragraph{Definitions of the three variants.}
%
%    \begin{macro}{\HoLogo@SLiTeX@lift}
%    \begin{macrocode}
\def\HoLogo@SLiTeX@lift#1{%
  \HoLogoFont@font{SliTeX}{rm}{%
    S%
    \kern-.06em%
    L%
    \kern-.18em%
    \raise.32ex\hbox{\HoLogoFont@font{SliTeX}{sc}{i}}%
    \HOLOGO@discretionary
    \kern-.06em%
    \hologo{TeX}%
  }%
}
%    \end{macrocode}
%    \end{macro}
%    \begin{macro}{\HoLogoBkm@SLiTeX@lift}
%    \begin{macrocode}
\def\HoLogoBkm@SLiTeX@lift#1{SLiTeX}
%    \end{macrocode}
%    \end{macro}
%    \begin{macro}{\HoLogoHtml@SLiTeX@lift}
%    \begin{macrocode}
\def\HoLogoHtml@SLiTeX@lift#1{%
  \HoLogoCss@SLiTeX@lift
  \HOLOGO@Span{SLiTeX-lift}{%
    \HoLogoFont@font{SliTeX}{rm}{%
      S%
      \HOLOGO@Span{L}{L}%
      \HOLOGO@Span{i}{i}%
      \hologo{TeX}%
    }%
  }%
}
%    \end{macrocode}
%    \end{macro}
%    \begin{macro}{\HoLogoCss@SLiTeX@lift}
%    \begin{macrocode}
\def\HoLogoCss@SLiTeX@lift{%
  \Css{%
    span.HoLogo-SLiTeX-lift span.HoLogo-L{%
      margin-left:-.06em;%
      margin-right:-.18em;%
    }%
  }%
  \Css{%
    span.HoLogo-SLiTeX-lift span.HoLogo-i{%
      position:relative;%
      top:-.32ex;%
      margin-right:-.06em;%
      font-variant:small-caps;%
    }%
  }%
  \global\let\HoLogoCss@SLiTeX@lift\relax
}
%    \end{macrocode}
%    \end{macro}
%
%    \begin{macro}{\HoLogo@SliTeX@simple}
%    \begin{macrocode}
\def\HoLogo@SliTeX@simple#1{%
  \HoLogoFont@font{SliTeX}{rm}{%
    \ltx@mbox{%
      \HoLogoFont@font{SliTeX}{sc}{Sli}%
    }%
    \HOLOGO@discretionary
    \hologo{TeX}%
  }%
}
%    \end{macrocode}
%    \end{macro}
%    \begin{macro}{\HoLogoBkm@SliTeX@simple}
%    \begin{macrocode}
\def\HoLogoBkm@SliTeX@simple#1{SliTeX}
%    \end{macrocode}
%    \end{macro}
%    \begin{macro}{\HoLogoHtml@SliTeX@simple}
%    \begin{macrocode}
\let\HoLogoHtml@SliTeX@simple\HoLogo@SliTeX@simple
%    \end{macrocode}
%    \end{macro}
%
%    \begin{macro}{\HoLogo@SliTeX@narrow}
%    \begin{macrocode}
\def\HoLogo@SliTeX@narrow#1{%
  \HoLogoFont@font{SliTeX}{rm}{%
    \ltx@mbox{%
      S%
      \kern-.06em%
      \HoLogoFont@font{SliTeX}{sc}{%
        l%
        \kern-.035em%
        i%
      }%
    }%
    \HOLOGO@discretionary
    \kern-.06em%
    \hologo{TeX}%
  }%
}
%    \end{macrocode}
%    \end{macro}
%    \begin{macro}{\HoLogoBkm@SliTeX@narrow}
%    \begin{macrocode}
\def\HoLogoBkm@SliTeX@narrow#1{SliTeX}
%    \end{macrocode}
%    \end{macro}
%    \begin{macro}{\HoLogoHtml@SliTeX@narrow}
%    \begin{macrocode}
\def\HoLogoHtml@SliTeX@narrow#1{%
  \HoLogoCss@SliTeX@narrow
  \HOLOGO@Span{SliTeX-narrow}{%
    \HoLogoFont@font{SliTeX}{rm}{%
      S%
        \HOLOGO@Span{l}{l}%
        \HOLOGO@Span{i}{i}%
      \hologo{TeX}%
    }%
  }%
}
%    \end{macrocode}
%    \end{macro}
%    \begin{macro}{\HoLogoCss@SliTeX@narrow}
%    \begin{macrocode}
\def\HoLogoCss@SliTeX@narrow{%
  \Css{%
    span.HoLogo-SliTeX-narrow span.HoLogo-l{%
      margin-left:-.06em;%
      margin-right:-.035em;%
      font-variant:small-caps;%
    }%
  }%
  \Css{%
    span.HoLogo-SliTeX-narrow span.HoLogo-i{%
      margin-right:-.06em;%
      font-variant:small-caps;%
    }%
  }%
  \global\let\HoLogoCss@SliTeX@narrow\relax
}
%    \end{macrocode}
%    \end{macro}
%
% \paragraph{Macro set completion.}
%
%    \begin{macro}{\HoLogo@SLiTeX@simple}
%    \begin{macrocode}
\def\HoLogo@SLiTeX@simple{\HoLogo@SliTeX@simple}
%    \end{macrocode}
%    \end{macro}
%    \begin{macro}{\HoLogoBkm@SLiTeX@simple}
%    \begin{macrocode}
\def\HoLogoBkm@SLiTeX@simple{\HoLogoBkm@SliTeX@simple}
%    \end{macrocode}
%    \end{macro}
%    \begin{macro}{\HoLogoHtml@SLiTeX@simple}
%    \begin{macrocode}
\def\HoLogoHtml@SLiTeX@simple{\HoLogoHtml@SliTeX@simple}
%    \end{macrocode}
%    \end{macro}
%
%    \begin{macro}{\HoLogo@SLiTeX@narrow}
%    \begin{macrocode}
\def\HoLogo@SLiTeX@narrow{\HoLogo@SliTeX@narrow}
%    \end{macrocode}
%    \end{macro}
%    \begin{macro}{\HoLogoBkm@SLiTeX@narrow}
%    \begin{macrocode}
\def\HoLogoBkm@SLiTeX@narrow{\HoLogoBkm@SliTeX@narrow}
%    \end{macrocode}
%    \end{macro}
%    \begin{macro}{\HoLogoHtml@SLiTeX@narrow}
%    \begin{macrocode}
\def\HoLogoHtml@SLiTeX@narrow{\HoLogoHtml@SliTeX@narrow}
%    \end{macrocode}
%    \end{macro}
%
%    \begin{macro}{\HoLogo@SliTeX@lift}
%    \begin{macrocode}
\def\HoLogo@SliTeX@lift{\HoLogo@SLiTeX@lift}
%    \end{macrocode}
%    \end{macro}
%    \begin{macro}{\HoLogoBkm@SliTeX@lift}
%    \begin{macrocode}
\def\HoLogoBkm@SliTeX@lift{\HoLogoBkm@SLiTeX@lift}
%    \end{macrocode}
%    \end{macro}
%    \begin{macro}{\HoLogoHtml@SliTeX@lift}
%    \begin{macrocode}
\def\HoLogoHtml@SliTeX@lift{\HoLogoHtml@SLiTeX@lift}
%    \end{macrocode}
%    \end{macro}
%
% \paragraph{Defaults.}
%
%    \begin{macro}{\HoLogo@SLiTeX}
%    \begin{macrocode}
\def\HoLogo@SLiTeX{\HoLogo@SLiTeX@lift}
%    \end{macrocode}
%    \end{macro}
%    \begin{macro}{\HoLogoBkm@SLiTeX}
%    \begin{macrocode}
\def\HoLogoBkm@SLiTeX{\HoLogoBkm@SLiTeX@lift}
%    \end{macrocode}
%    \end{macro}
%    \begin{macro}{\HoLogoHtml@SLiTeX}
%    \begin{macrocode}
\def\HoLogoHtml@SLiTeX{\HoLogoHtml@SLiTeX@lift}
%    \end{macrocode}
%    \end{macro}
%
%    \begin{macro}{\HoLogo@SliTeX}
%    \begin{macrocode}
\def\HoLogo@SliTeX{\HoLogo@SliTeX@narrow}
%    \end{macrocode}
%    \end{macro}
%    \begin{macro}{\HoLogoBkm@SliTeX}
%    \begin{macrocode}
\def\HoLogoBkm@SliTeX{\HoLogoBkm@SliTeX@narrow}
%    \end{macrocode}
%    \end{macro}
%    \begin{macro}{\HoLogoHtml@SliTeX}
%    \begin{macrocode}
\def\HoLogoHtml@SliTeX{\HoLogoHtml@SliTeX@narrow}
%    \end{macrocode}
%    \end{macro}
%
% \subsubsection{\hologo{LuaTeX}}
%
%    \begin{macro}{\HoLogo@LuaTeX}
%    The kerning is an idea of Hans Hagen, see mailing list
%    `luatex at tug dot org' in March 2010.
%    \begin{macrocode}
\def\HoLogo@LuaTeX#1{%
  \HOLOGO@mbox{%
    Lua%
    \HOLOGO@NegativeKerning{aT,oT,To}%
    \hologo{TeX}%
  }%
}
%    \end{macrocode}
%    \end{macro}
%    \begin{macro}{\HoLogoHtml@LuaTeX}
%    \begin{macrocode}
\let\HoLogoHtml@LuaTeX\HoLogo@LuaTeX
%    \end{macrocode}
%    \end{macro}
%
% \subsubsection{\hologo{LuaLaTeX}}
%
%    \begin{macro}{\HoLogo@LuaLaTeX}
%    \begin{macrocode}
\def\HoLogo@LuaLaTeX#1{%
  \HOLOGO@mbox{%
    Lua%
    \hologo{LaTeX}%
  }%
}
%    \end{macrocode}
%    \end{macro}
%    \begin{macro}{\HoLogoHtml@LuaLaTeX}
%    \begin{macrocode}
\let\HoLogoHtml@LuaLaTeX\HoLogo@LuaLaTeX
%    \end{macrocode}
%    \end{macro}
%
% \subsubsection{\hologo{XeTeX}, \hologo{XeLaTeX}}
%
%    \begin{macro}{\HOLOGO@IfCharExists}
%    \begin{macrocode}
\ifluatex
  \ifnum\luatexversion<36 %
  \else
    \def\HOLOGO@IfCharExists#1{%
      \ifnum
        \directlua{%
           if luaotfload and luaotfload.aux then
             if luaotfload.aux.font_has_glyph(%
                    font.current(), \number#1) then % 	 
	       tex.print("1") % 	 
	     end % 	 
	   elseif font and font.fonts and font.current then %
            local f = font.fonts[font.current()]%
            if f.characters and f.characters[\number#1] then %
              tex.print("1")%
            end %
          end%
        }0=\ltx@zero
        \expandafter\ltx@secondoftwo
      \else
        \expandafter\ltx@firstoftwo
      \fi
    }%
  \fi
\fi
\ltx@IfUndefined{HOLOGO@IfCharExists}{%
  \def\HOLOGO@@IfCharExists#1{%
    \begingroup
      \tracinglostchars=\ltx@zero
      \setbox\ltx@zero=\hbox{%
        \kern7sp\char#1\relax
        \ifnum\lastkern>\ltx@zero
          \expandafter\aftergroup\csname iffalse\endcsname
        \else
          \expandafter\aftergroup\csname iftrue\endcsname
        \fi
      }%
      % \if{true|false} from \aftergroup
      \endgroup
      \expandafter\ltx@firstoftwo
    \else
      \endgroup
      \expandafter\ltx@secondoftwo
    \fi
  }%
  \ifxetex
    \ltx@IfUndefined{XeTeXfonttype}{}{%
      \ltx@IfUndefined{XeTeXcharglyph}{}{%
        \def\HOLOGO@IfCharExists#1{%
          \ifnum\XeTeXfonttype\font>\ltx@zero
            \expandafter\ltx@firstofthree
          \else
            \expandafter\ltx@gobble
          \fi
          {%
            \ifnum\XeTeXcharglyph#1>\ltx@zero
              \expandafter\ltx@firstoftwo
            \else
              \expandafter\ltx@secondoftwo
            \fi
          }%
          \HOLOGO@@IfCharExists{#1}%
        }%
      }%
    }%
  \fi
}{}
\ltx@ifundefined{HOLOGO@IfCharExists}{%
  \ifnum64=`\^^^^0040\relax % test for big chars of LuaTeX/XeTeX
    \let\HOLOGO@IfCharExists\HOLOGO@@IfCharExists
  \else
    \def\HOLOGO@IfCharExists#1{%
      \ifnum#1>255 %
        \expandafter\ltx@fourthoffour
      \fi
      \HOLOGO@@IfCharExists{#1}%
    }%
  \fi
}{}
%    \end{macrocode}
%    \end{macro}
%
%    \begin{macro}{\HoLogo@Xe}
%    Source: package \xpackage{dtklogos}
%    \begin{macrocode}
\def\HoLogo@Xe#1{%
  X%
  \kern-.1em\relax
  \HOLOGO@IfCharExists{"018E}{%
    \lower.5ex\hbox{\char"018E}%
  }{%
    \chardef\HOLOGO@choice=\ltx@zero
    \ifdim\fontdimen\ltx@one\font>0pt %
      \ltx@IfUndefined{rotatebox}{%
        \ltx@IfUndefined{pgftext}{%
          \ltx@IfUndefined{psscalebox}{%
            \ltx@IfUndefined{HOLOGO@ScaleBox@\hologoDriver}{%
            }{%
              \chardef\HOLOGO@choice=4 %
            }%
          }{%
            \chardef\HOLOGO@choice=3 %
          }%
        }{%
          \chardef\HOLOGO@choice=2 %
        }%
      }{%
        \chardef\HOLOGO@choice=1 %
      }%
      \ifcase\HOLOGO@choice
        \HOLOGO@WarningUnsupportedDriver{Xe}%
        e%
      \or % 1: \rotatebox
        \begingroup
          \setbox\ltx@zero\hbox{\rotatebox{180}{E}}%
          \ltx@LocDimenA=\dp\ltx@zero
          \advance\ltx@LocDimenA by -.5ex\relax
          \raise\ltx@LocDimenA\box\ltx@zero
        \endgroup
      \or % 2: \pgftext
        \lower.5ex\hbox{%
          \pgfpicture
            \pgftext[rotate=180]{E}%
          \endpgfpicture
        }%
      \or % 3: \psscalebox
        \begingroup
          \setbox\ltx@zero\hbox{\psscalebox{-1 -1}{E}}%
          \ltx@LocDimenA=\dp\ltx@zero
          \advance\ltx@LocDimenA by -.5ex\relax
          \raise\ltx@LocDimenA\box\ltx@zero
        \endgroup
      \or % 4: \HOLOGO@PointReflectBox
        \lower.5ex\hbox{\HOLOGO@PointReflectBox{E}}%
      \else
        \@PackageError{hologo}{Internal error (choice/it}\@ehc
      \fi
    \else
      \ltx@IfUndefined{reflectbox}{%
        \ltx@IfUndefined{pgftext}{%
          \ltx@IfUndefined{psscalebox}{%
            \ltx@IfUndefined{HOLOGO@ScaleBox@\hologoDriver}{%
            }{%
              \chardef\HOLOGO@choice=4 %
            }%
          }{%
            \chardef\HOLOGO@choice=3 %
          }%
        }{%
          \chardef\HOLOGO@choice=2 %
        }%
      }{%
        \chardef\HOLOGO@choice=1 %
      }%
      \ifcase\HOLOGO@choice
        \HOLOGO@WarningUnsupportedDriver{Xe}%
        e%
      \or % 1: reflectbox
        \lower.5ex\hbox{%
          \reflectbox{E}%
        }%
      \or % 2: \pgftext
        \lower.5ex\hbox{%
          \pgfpicture
            \pgftransformxscale{-1}%
            \pgftext{E}%
          \endpgfpicture
        }%
      \or % 3: \psscalebox
        \lower.5ex\hbox{%
          \psscalebox{-1 1}{E}%
        }%
      \or % 4: \HOLOGO@Reflectbox
        \lower.5ex\hbox{%
          \HOLOGO@ReflectBox{E}%
        }%
      \else
        \@PackageError{hologo}{Internal error (choice/up)}\@ehc
      \fi
    \fi
  }%
}
%    \end{macrocode}
%    \end{macro}
%    \begin{macro}{\HoLogoHtml@Xe}
%    \begin{macrocode}
\def\HoLogoHtml@Xe#1{%
  \HoLogoCss@Xe
  \HOLOGO@Span{Xe}{%
    X%
    \HOLOGO@Span{e}{%
      \HCode{&\ltx@hashchar x018e;}%
    }%
  }%
}
%    \end{macrocode}
%    \end{macro}
%    \begin{macro}{\HoLogoCss@Xe}
%    \begin{macrocode}
\def\HoLogoCss@Xe{%
  \Css{%
    span.HoLogo-Xe span.HoLogo-e{%
      position:relative;%
      top:.5ex;%
      left-margin:-.1em;%
    }%
  }%
  \global\let\HoLogoCss@Xe\relax
}
%    \end{macrocode}
%    \end{macro}
%
%    \begin{macro}{\HoLogo@XeTeX}
%    \begin{macrocode}
\def\HoLogo@XeTeX#1{%
  \hologo{Xe}%
  \kern-.15em\relax
  \hologo{TeX}%
}
%    \end{macrocode}
%    \end{macro}
%
%    \begin{macro}{\HoLogoHtml@XeTeX}
%    \begin{macrocode}
\def\HoLogoHtml@XeTeX#1{%
  \HoLogoCss@XeTeX
  \HOLOGO@Span{XeTeX}{%
    \hologo{Xe}%
    \hologo{TeX}%
  }%
}
%    \end{macrocode}
%    \end{macro}
%    \begin{macro}{\HoLogoCss@XeTeX}
%    \begin{macrocode}
\def\HoLogoCss@XeTeX{%
  \Css{%
    span.HoLogo-XeTeX span.HoLogo-TeX{%
      margin-left:-.15em;%
    }%
  }%
  \global\let\HoLogoCss@XeTeX\relax
}
%    \end{macrocode}
%    \end{macro}
%
%    \begin{macro}{\HoLogo@XeLaTeX}
%    \begin{macrocode}
\def\HoLogo@XeLaTeX#1{%
  \hologo{Xe}%
  \kern-.13em%
  \hologo{LaTeX}%
}
%    \end{macrocode}
%    \end{macro}
%    \begin{macro}{\HoLogoHtml@XeLaTeX}
%    \begin{macrocode}
\def\HoLogoHtml@XeLaTeX#1{%
  \HoLogoCss@XeLaTeX
  \HOLOGO@Span{XeLaTeX}{%
    \hologo{Xe}%
    \hologo{LaTeX}%
  }%
}
%    \end{macrocode}
%    \end{macro}
%    \begin{macro}{\HoLogoCss@XeLaTeX}
%    \begin{macrocode}
\def\HoLogoCss@XeLaTeX{%
  \Css{%
    span.HoLogo-XeLaTeX span.HoLogo-Xe{%
      margin-right:-.13em;%
    }%
  }%
  \global\let\HoLogoCss@XeLaTeX\relax
}
%    \end{macrocode}
%    \end{macro}
%
% \subsubsection{\hologo{pdfTeX}, \hologo{pdfLaTeX}}
%
%    \begin{macro}{\HoLogo@pdfTeX}
%    \begin{macrocode}
\def\HoLogo@pdfTeX#1{%
  \HOLOGO@mbox{%
    #1{p}{P}df\hologo{TeX}%
  }%
}
%    \end{macrocode}
%    \end{macro}
%    \begin{macro}{\HoLogoCs@pdfTeX}
%    \begin{macrocode}
\def\HoLogoCs@pdfTeX#1{#1{p}{P}dfTeX}
%    \end{macrocode}
%    \end{macro}
%    \begin{macro}{\HoLogoBkm@pdfTeX}
%    \begin{macrocode}
\def\HoLogoBkm@pdfTeX#1{%
  #1{p}{P}df\hologo{TeX}%
}
%    \end{macrocode}
%    \end{macro}
%    \begin{macro}{\HoLogoHtml@pdfTeX}
%    \begin{macrocode}
\let\HoLogoHtml@pdfTeX\HoLogo@pdfTeX
%    \end{macrocode}
%    \end{macro}
%
%    \begin{macro}{\HoLogo@pdfLaTeX}
%    \begin{macrocode}
\def\HoLogo@pdfLaTeX#1{%
  \HOLOGO@mbox{%
    #1{p}{P}df\hologo{LaTeX}%
  }%
}
%    \end{macrocode}
%    \end{macro}
%    \begin{macro}{\HoLogoCs@pdfLaTeX}
%    \begin{macrocode}
\def\HoLogoCs@pdfLaTeX#1{#1{p}{P}dfLaTeX}
%    \end{macrocode}
%    \end{macro}
%    \begin{macro}{\HoLogoBkm@pdfLaTeX}
%    \begin{macrocode}
\def\HoLogoBkm@pdfLaTeX#1{%
  #1{p}{P}df\hologo{LaTeX}%
}
%    \end{macrocode}
%    \end{macro}
%    \begin{macro}{\HoLogoHtml@pdfLaTeX}
%    \begin{macrocode}
\let\HoLogoHtml@pdfLaTeX\HoLogo@pdfLaTeX
%    \end{macrocode}
%    \end{macro}
%
% \subsubsection{\hologo{VTeX}}
%
%    \begin{macro}{\HoLogo@VTeX}
%    \begin{macrocode}
\def\HoLogo@VTeX#1{%
  \HOLOGO@mbox{%
    V\hologo{TeX}%
  }%
}
%    \end{macrocode}
%    \end{macro}
%    \begin{macro}{\HoLogoHtml@VTeX}
%    \begin{macrocode}
\let\HoLogoHtml@VTeX\HoLogo@VTeX
%    \end{macrocode}
%    \end{macro}
%
% \subsubsection{\hologo{AmS}, \dots}
%
%    Source: class \xclass{amsdtx}
%
%    \begin{macro}{\HoLogo@AmS}
%    \begin{macrocode}
\def\HoLogo@AmS#1{%
  \HoLogoFont@font{AmS}{sy}{%
    A%
    \kern-.1667em%
    \lower.5ex\hbox{M}%
    \kern-.125em%
    S%
  }%
}
%    \end{macrocode}
%    \end{macro}
%    \begin{macro}{\HoLogoBkm@AmS}
%    \begin{macrocode}
\def\HoLogoBkm@AmS#1{AmS}
%    \end{macrocode}
%    \end{macro}
%    \begin{macro}{\HoLogoHtml@AmS}
%    \begin{macrocode}
\def\HoLogoHtml@AmS#1{%
  \HoLogoCss@AmS
%  \HoLogoFont@font{AmS}{sy}{%
    \HOLOGO@Span{AmS}{%
      A%
      \HOLOGO@Span{M}{M}%
      S%
    }%
%   }%
}
%    \end{macrocode}
%    \end{macro}
%    \begin{macro}{\HoLogoCss@AmS}
%    \begin{macrocode}
\def\HoLogoCss@AmS{%
  \Css{%
    span.HoLogo-AmS span.HoLogo-M{%
      position:relative;%
      top:.5ex;%
      margin-left:-.1667em;%
      margin-right:-.125em;%
      text-decoration:none;%
    }%
  }%
  \global\let\HoLogoCss@AmS\relax
}
%    \end{macrocode}
%    \end{macro}
%
%    \begin{macro}{\HoLogo@AmSTeX}
%    \begin{macrocode}
\def\HoLogo@AmSTeX#1{%
  \hologo{AmS}%
  \HOLOGO@hyphen
  \hologo{TeX}%
}
%    \end{macrocode}
%    \end{macro}
%    \begin{macro}{\HoLogoBkm@AmSTeX}
%    \begin{macrocode}
\def\HoLogoBkm@AmSTeX#1{AmS-TeX}%
%    \end{macrocode}
%    \end{macro}
%    \begin{macro}{\HoLogoHtml@AmSTeX}
%    \begin{macrocode}
\let\HoLogoHtml@AmSTeX\HoLogo@AmSTeX
%    \end{macrocode}
%    \end{macro}
%
%    \begin{macro}{\HoLogo@AmSLaTeX}
%    \begin{macrocode}
\def\HoLogo@AmSLaTeX#1{%
  \hologo{AmS}%
  \HOLOGO@hyphen
  \hologo{LaTeX}%
}
%    \end{macrocode}
%    \end{macro}
%    \begin{macro}{\HoLogoBkm@AmSLaTeX}
%    \begin{macrocode}
\def\HoLogoBkm@AmSLaTeX#1{AmS-LaTeX}%
%    \end{macrocode}
%    \end{macro}
%    \begin{macro}{\HoLogoHtml@AmSLaTeX}
%    \begin{macrocode}
\let\HoLogoHtml@AmSLaTeX\HoLogo@AmSLaTeX
%    \end{macrocode}
%    \end{macro}
%
% \subsubsection{\hologo{BibTeX}}
%
%    \begin{macro}{\HoLogo@BibTeX@sc}
%    A definition of \hologo{BibTeX} is provided in
%    the documentation source for the manual of \hologo{BibTeX}
%    \cite{btxdoc}.
%\begin{quote}
%\begin{verbatim}
%\def\BibTeX{%
%  {%
%    \rm
%    B%
%    \kern-.05em%
%    {%
%      \sc
%      i%
%      \kern-.025em %
%      b%
%    }%
%    \kern-.08em
%    T%
%    \kern-.1667em%
%    \lower.7ex\hbox{E}%
%    \kern-.125em%
%    X%
%  }%
%}
%\end{verbatim}
%\end{quote}
%    \begin{macrocode}
\def\HoLogo@BibTeX@sc#1{%
  B%
  \kern-.05em%
  \HoLogoFont@font{BibTeX}{sc}{%
    i%
    \kern-.025em%
    b%
  }%
  \HOLOGO@discretionary
  \kern-.08em%
  \hologo{TeX}%
}
%    \end{macrocode}
%    \end{macro}
%    \begin{macro}{\HoLogoHtml@BibTeX@sc}
%    \begin{macrocode}
\def\HoLogoHtml@BibTeX@sc#1{%
  \HoLogoCss@BibTeX@sc
  \HOLOGO@Span{BibTeX-sc}{%
    B%
    \HOLOGO@Span{i}{i}%
    \HOLOGO@Span{b}{b}%
    \hologo{TeX}%
  }%
}
%    \end{macrocode}
%    \end{macro}
%    \begin{macro}{\HoLogoCss@BibTeX@sc}
%    \begin{macrocode}
\def\HoLogoCss@BibTeX@sc{%
  \Css{%
    span.HoLogo-BibTeX-sc span.HoLogo-i{%
      margin-left:-.05em;%
      margin-right:-.025em;%
      font-variant:small-caps;%
    }%
  }%
  \Css{%
    span.HoLogo-BibTeX-sc span.HoLogo-b{%
      margin-right:-.08em;%
      font-variant:small-caps;%
    }%
  }%
  \global\let\HoLogoCss@BibTeX@sc\relax
}
%    \end{macrocode}
%    \end{macro}
%
%    \begin{macro}{\HoLogo@BibTeX@sf}
%    Variant \xoption{sf} avoids trouble with unavailable
%    small caps fonts (e.g., bold versions of Computer Modern or
%    Latin Modern). The definition is taken from
%    package \xpackage{dtklogos} \cite{dtklogos}.
%\begin{quote}
%\begin{verbatim}
%\DeclareRobustCommand{\BibTeX}{%
%  B%
%  \kern-.05em%
%  \hbox{%
%    $\m@th$% %% force math size calculations
%    \csname S@\f@size\endcsname
%    \fontsize\sf@size\z@
%    \math@fontsfalse
%    \selectfont
%    I%
%    \kern-.025em%
%    B
%  }%
%  \kern-.08em%
%  \-%
%  \TeX
%}
%\end{verbatim}
%\end{quote}
%    \begin{macrocode}
\def\HoLogo@BibTeX@sf#1{%
  B%
  \kern-.05em%
  \HoLogoFont@font{BibTeX}{bibsf}{%
    I%
    \kern-.025em%
    B%
  }%
  \HOLOGO@discretionary
  \kern-.08em%
  \hologo{TeX}%
}
%    \end{macrocode}
%    \end{macro}
%    \begin{macro}{\HoLogoHtml@BibTeX@sf}
%    \begin{macrocode}
\def\HoLogoHtml@BibTeX@sf#1{%
  \HoLogoCss@BibTeX@sf
  \HOLOGO@Span{BibTeX-sf}{%
    B%
    \HoLogoFont@font{BibTeX}{bibsf}{%
      \HOLOGO@Span{i}{I}%
      B%
    }%
    \hologo{TeX}%
  }%
}
%    \end{macrocode}
%    \end{macro}
%    \begin{macro}{\HoLogoCss@BibTeX@sf}
%    \begin{macrocode}
\def\HoLogoCss@BibTeX@sf{%
  \Css{%
    span.HoLogo-BibTeX-sf span.HoLogo-i{%
      margin-left:-.05em;%
      margin-right:-.025em;%
    }%
  }%
  \Css{%
    span.HoLogo-BibTeX-sf span.HoLogo-TeX{%
      margin-left:-.08em;%
    }%
  }%
  \global\let\HoLogoCss@BibTeX@sf\relax
}
%    \end{macrocode}
%    \end{macro}
%
%    \begin{macro}{\HoLogo@BibTeX}
%    \begin{macrocode}
\def\HoLogo@BibTeX{\HoLogo@BibTeX@sf}
%    \end{macrocode}
%    \end{macro}
%    \begin{macro}{\HoLogoHtml@BibTeX}
%    \begin{macrocode}
\def\HoLogoHtml@BibTeX{\HoLogoHtml@BibTeX@sf}
%    \end{macrocode}
%    \end{macro}
%
% \subsubsection{\hologo{BibTeX8}}
%
%    \begin{macro}{\HoLogo@BibTeX8}
%    \begin{macrocode}
\expandafter\def\csname HoLogo@BibTeX8\endcsname#1{%
  \hologo{BibTeX}%
  8%
}
%    \end{macrocode}
%    \end{macro}
%
%    \begin{macro}{\HoLogoBkm@BibTeX8}
%    \begin{macrocode}
\expandafter\def\csname HoLogoBkm@BibTeX8\endcsname#1{%
  \hologo{BibTeX}%
  8%
}
%    \end{macrocode}
%    \end{macro}
%    \begin{macro}{\HoLogoHtml@BibTeX8}
%    \begin{macrocode}
\expandafter
\let\csname HoLogoHtml@BibTeX8\expandafter\endcsname
\csname HoLogo@BibTeX8\endcsname
%    \end{macrocode}
%    \end{macro}
%
% \subsubsection{\hologo{ConTeXt}}
%
%    \begin{macro}{\HoLogo@ConTeXt@simple}
%    \begin{macrocode}
\def\HoLogo@ConTeXt@simple#1{%
  \HOLOGO@mbox{Con}%
  \HOLOGO@discretionary
  \HOLOGO@mbox{\hologo{TeX}t}%
}
%    \end{macrocode}
%    \end{macro}
%    \begin{macro}{\HoLogoHtml@ConTeXt@simple}
%    \begin{macrocode}
\let\HoLogoHtml@ConTeXt@simple\HoLogo@ConTeXt@simple
%    \end{macrocode}
%    \end{macro}
%
%    \begin{macro}{\HoLogo@ConTeXt@narrow}
%    This definition of logo \hologo{ConTeXt} with variant \xoption{narrow}
%    comes from TUGboat's class \xclass{ltugboat} (version 2010/11/15 v2.8).
%    \begin{macrocode}
\def\HoLogo@ConTeXt@narrow#1{%
  \HOLOGO@mbox{C\kern-.0333emon}%
  \HOLOGO@discretionary
  \kern-.0667em%
  \HOLOGO@mbox{\hologo{TeX}\kern-.0333emt}%
}
%    \end{macrocode}
%    \end{macro}
%    \begin{macro}{\HoLogoHtml@ConTeXt@narrow}
%    \begin{macrocode}
\def\HoLogoHtml@ConTeXt@narrow#1{%
  \HoLogoCss@ConTeXt@narrow
  \HOLOGO@Span{ConTeXt-narrow}{%
    \HOLOGO@Span{C}{C}%
    on%
    \hologo{TeX}%
    t%
  }%
}
%    \end{macrocode}
%    \end{macro}
%    \begin{macro}{\HoLogoCss@ConTeXt@narrow}
%    \begin{macrocode}
\def\HoLogoCss@ConTeXt@narrow{%
  \Css{%
    span.HoLogo-ConTeXt-narrow span.HoLogo-C{%
      margin-left:-.0333em;%
    }%
  }%
  \Css{%
    span.HoLogo-ConTeXt-narrow span.HoLogo-TeX{%
      margin-left:-.0667em;%
      margin-right:-.0333em;%
    }%
  }%
  \global\let\HoLogoCss@ConTeXt@narrow\relax
}
%    \end{macrocode}
%    \end{macro}
%
%    \begin{macro}{\HoLogo@ConTeXt}
%    \begin{macrocode}
\def\HoLogo@ConTeXt{\HoLogo@ConTeXt@narrow}
%    \end{macrocode}
%    \end{macro}
%    \begin{macro}{\HoLogoHtml@ConTeXt}
%    \begin{macrocode}
\def\HoLogoHtml@ConTeXt{\HoLogoHtml@ConTeXt@narrow}
%    \end{macrocode}
%    \end{macro}
%
% \subsubsection{\hologo{emTeX}}
%
%    \begin{macro}{\HoLogo@emTeX}
%    \begin{macrocode}
\def\HoLogo@emTeX#1{%
  \HOLOGO@mbox{#1{e}{E}m}%
  \HOLOGO@discretionary
  \hologo{TeX}%
}
%    \end{macrocode}
%    \end{macro}
%    \begin{macro}{\HoLogoCs@emTeX}
%    \begin{macrocode}
\def\HoLogoCs@emTeX#1{#1{e}{E}mTeX}%
%    \end{macrocode}
%    \end{macro}
%    \begin{macro}{\HoLogoBkm@emTeX}
%    \begin{macrocode}
\def\HoLogoBkm@emTeX#1{%
  #1{e}{E}m\hologo{TeX}%
}
%    \end{macrocode}
%    \end{macro}
%    \begin{macro}{\HoLogoHtml@emTeX}
%    \begin{macrocode}
\let\HoLogoHtml@emTeX\HoLogo@emTeX
%    \end{macrocode}
%    \end{macro}
%
% \subsubsection{\hologo{ExTeX}}
%
%    \begin{macro}{\HoLogo@ExTeX}
%    The definition is taken from the FAQ of the
%    project \hologo{ExTeX}
%    \cite{ExTeX-FAQ}.
%\begin{quote}
%\begin{verbatim}
%\def\ExTeX{%
%  \textrm{% Logo always with serifs
%    \ensuremath{%
%      \textstyle
%      \varepsilon_{%
%        \kern-0.15em%
%        \mathcal{X}%
%      }%
%    }%
%    \kern-.15em%
%    \TeX
%  }%
%}
%\end{verbatim}
%\end{quote}
%    \begin{macrocode}
\def\HoLogo@ExTeX#1{%
  \HoLogoFont@font{ExTeX}{rm}{%
    \ltx@mbox{%
      \HOLOGO@MathSetup
      $%
        \textstyle
        \varepsilon_{%
          \kern-0.15em%
          \HoLogoFont@font{ExTeX}{sy}{X}%
        }%
      $%
    }%
    \HOLOGO@discretionary
    \kern-.15em%
    \hologo{TeX}%
  }%
}
%    \end{macrocode}
%    \end{macro}
%    \begin{macro}{\HoLogoHtml@ExTeX}
%    \begin{macrocode}
\def\HoLogoHtml@ExTeX#1{%
  \HoLogoCss@ExTeX
  \HoLogoFont@font{ExTeX}{rm}{%
    \HOLOGO@Span{ExTeX}{%
      \ltx@mbox{%
        \HOLOGO@MathSetup
        $\textstyle\varepsilon$%
        \HOLOGO@Span{X}{$\textstyle\chi$}%
        \hologo{TeX}%
      }%
    }%
  }%
}
%    \end{macrocode}
%    \end{macro}
%    \begin{macro}{\HoLogoBkm@ExTeX}
%    \begin{macrocode}
\def\HoLogoBkm@ExTeX#1{%
  \HOLOGO@PdfdocUnicode{#1{e}{E}x}{\textepsilon\textchi}%
  \hologo{TeX}%
}
%    \end{macrocode}
%    \end{macro}
%    \begin{macro}{\HoLogoCss@ExTeX}
%    \begin{macrocode}
\def\HoLogoCss@ExTeX{%
  \Css{%
    span.HoLogo-ExTeX{%
      font-family:serif;%
    }%
  }%
  \Css{%
    span.HoLogo-ExTeX span.HoLogo-TeX{%
      margin-left:-.15em;%
    }%
  }%
  \global\let\HoLogoCss@ExTeX\relax
}
%    \end{macrocode}
%    \end{macro}
%
% \subsubsection{\hologo{MiKTeX}}
%
%    \begin{macro}{\HoLogo@MiKTeX}
%    \begin{macrocode}
\def\HoLogo@MiKTeX#1{%
  \HOLOGO@mbox{MiK}%
  \HOLOGO@discretionary
  \hologo{TeX}%
}
%    \end{macrocode}
%    \end{macro}
%    \begin{macro}{\HoLogoHtml@MiKTeX}
%    \begin{macrocode}
\let\HoLogoHtml@MiKTeX\HoLogo@MiKTeX
%    \end{macrocode}
%    \end{macro}
%
% \subsubsection{\hologo{OzTeX} and friends}
%
%    Source: \hologo{OzTeX} FAQ \cite{OzTeX}:
%    \begin{quote}
%      |\def\OzTeX{O\kern-.03em z\kern-.15em\TeX}|\\
%      (There is no kerning in OzMF, OzMP and OzTtH.)
%    \end{quote}
%
%    \begin{macro}{\HoLogo@OzTeX}
%    \begin{macrocode}
\def\HoLogo@OzTeX#1{%
  O%
  \kern-.03em %
  z%
  \kern-.15em %
  \hologo{TeX}%
}
%    \end{macrocode}
%    \end{macro}
%    \begin{macro}{\HoLogoHtml@OzTeX}
%    \begin{macrocode}
\def\HoLogoHtml@OzTeX#1{%
  \HoLogoCss@OzTeX
  \HOLOGO@Span{OzTeX}{%
    O%
    \HOLOGO@Span{z}{z}%
    \hologo{TeX}%
  }%
}
%    \end{macrocode}
%    \end{macro}
%    \begin{macro}{\HoLogoCss@OzTeX}
%    \begin{macrocode}
\def\HoLogoCss@OzTeX{%
  \Css{%
    span.HoLogo-OzTeX span.HoLogo-z{%
      margin-left:-.03em;%
      margin-right:-.15em;%
    }%
  }%
  \global\let\HoLogoCss@OzTeX\relax
}
%    \end{macrocode}
%    \end{macro}
%
%    \begin{macro}{\HoLogo@OzMF}
%    \begin{macrocode}
\def\HoLogo@OzMF#1{%
  \HOLOGO@mbox{OzMF}%
}
%    \end{macrocode}
%    \end{macro}
%    \begin{macro}{\HoLogo@OzMP}
%    \begin{macrocode}
\def\HoLogo@OzMP#1{%
  \HOLOGO@mbox{OzMP}%
}
%    \end{macrocode}
%    \end{macro}
%    \begin{macro}{\HoLogo@OzTtH}
%    \begin{macrocode}
\def\HoLogo@OzTtH#1{%
  \HOLOGO@mbox{OzTtH}%
}
%    \end{macrocode}
%    \end{macro}
%
% \subsubsection{\hologo{PCTeX}}
%
%    \begin{macro}{\HoLogo@PCTeX}
%    \begin{macrocode}
\def\HoLogo@PCTeX#1{%
  \HOLOGO@mbox{PC}%
  \hologo{TeX}%
}
%    \end{macrocode}
%    \end{macro}
%    \begin{macro}{\HoLogoHtml@PCTeX}
%    \begin{macrocode}
\let\HoLogoHtml@PCTeX\HoLogo@PCTeX
%    \end{macrocode}
%    \end{macro}
%
% \subsubsection{\hologo{PiCTeX}}
%
%    The original definitions from \xfile{pictex.tex} \cite{PiCTeX}:
%\begin{quote}
%\begin{verbatim}
%\def\PiC{%
%  P%
%  \kern-.12em%
%  \lower.5ex\hbox{I}%
%  \kern-.075em%
%  C%
%}
%\def\PiCTeX{%
%  \PiC
%  \kern-.11em%
%  \TeX
%}
%\end{verbatim}
%\end{quote}
%
%    \begin{macro}{\HoLogo@PiC}
%    \begin{macrocode}
\def\HoLogo@PiC#1{%
  P%
  \kern-.12em%
  \lower.5ex\hbox{I}%
  \kern-.075em%
  C%
  \HOLOGO@SpaceFactor
}
%    \end{macrocode}
%    \end{macro}
%    \begin{macro}{\HoLogoHtml@PiC}
%    \begin{macrocode}
\def\HoLogoHtml@PiC#1{%
  \HoLogoCss@PiC
  \HOLOGO@Span{PiC}{%
    P%
    \HOLOGO@Span{i}{I}%
    C%
  }%
}
%    \end{macrocode}
%    \end{macro}
%    \begin{macro}{\HoLogoCss@PiC}
%    \begin{macrocode}
\def\HoLogoCss@PiC{%
  \Css{%
    span.HoLogo-PiC span.HoLogo-i{%
      position:relative;%
      top:.5ex;%
      margin-left:-.12em;%
      margin-right:-.075em;%
      text-decoration:none;%
    }%
  }%
  \global\let\HoLogoCss@PiC\relax
}
%    \end{macrocode}
%    \end{macro}
%
%    \begin{macro}{\HoLogo@PiCTeX}
%    \begin{macrocode}
\def\HoLogo@PiCTeX#1{%
  \hologo{PiC}%
  \HOLOGO@discretionary
  \kern-.11em%
  \hologo{TeX}%
}
%    \end{macrocode}
%    \end{macro}
%    \begin{macro}{\HoLogoHtml@PiCTeX}
%    \begin{macrocode}
\def\HoLogoHtml@PiCTeX#1{%
  \HoLogoCss@PiCTeX
  \HOLOGO@Span{PiCTeX}{%
    \hologo{PiC}%
    \hologo{TeX}%
  }%
}
%    \end{macrocode}
%    \end{macro}
%    \begin{macro}{\HoLogoCss@PiCTeX}
%    \begin{macrocode}
\def\HoLogoCss@PiCTeX{%
  \Css{%
    span.HoLogo-PiCTeX span.HoLogo-PiC{%
      margin-right:-.11em;%
    }%
  }%
  \global\let\HoLogoCss@PiCTeX\relax
}
%    \end{macrocode}
%    \end{macro}
%
% \subsubsection{\hologo{teTeX}}
%
%    \begin{macro}{\HoLogo@teTeX}
%    \begin{macrocode}
\def\HoLogo@teTeX#1{%
  \HOLOGO@mbox{#1{t}{T}e}%
  \HOLOGO@discretionary
  \hologo{TeX}%
}
%    \end{macrocode}
%    \end{macro}
%    \begin{macro}{\HoLogoCs@teTeX}
%    \begin{macrocode}
\def\HoLogoCs@teTeX#1{#1{t}{T}dfTeX}
%    \end{macrocode}
%    \end{macro}
%    \begin{macro}{\HoLogoBkm@teTeX}
%    \begin{macrocode}
\def\HoLogoBkm@teTeX#1{%
  #1{t}{T}e\hologo{TeX}%
}
%    \end{macrocode}
%    \end{macro}
%    \begin{macro}{\HoLogoHtml@teTeX}
%    \begin{macrocode}
\let\HoLogoHtml@teTeX\HoLogo@teTeX
%    \end{macrocode}
%    \end{macro}
%
% \subsubsection{\hologo{TeX4ht}}
%
%    \begin{macro}{\HoLogo@TeX4ht}
%    \begin{macrocode}
\expandafter\def\csname HoLogo@TeX4ht\endcsname#1{%
  \HOLOGO@mbox{\hologo{TeX}4ht}%
}
%    \end{macrocode}
%    \end{macro}
%    \begin{macro}{\HoLogoHtml@TeX4ht}
%    \begin{macrocode}
\expandafter
\let\csname HoLogoHtml@TeX4ht\expandafter\endcsname
\csname HoLogo@TeX4ht\endcsname
%    \end{macrocode}
%    \end{macro}
%
%
% \subsubsection{\hologo{SageTeX}}
%
%    \begin{macro}{\HoLogo@SageTeX}
%    \begin{macrocode}
\def\HoLogo@SageTeX#1{%
  \HOLOGO@mbox{Sage}%
  \HOLOGO@discretionary
  \HOLOGO@NegativeKerning{eT,oT,To}%
  \hologo{TeX}%
}
%    \end{macrocode}
%    \end{macro}
%    \begin{macro}{\HoLogoHtml@SageTeX}
%    \begin{macrocode}
\let\HoLogoHtml@SageTeX\HoLogo@SageTeX
%    \end{macrocode}
%    \end{macro}
%
% \subsection{\hologo{METAFONT} and friends}
%
%    \begin{macro}{\HoLogo@METAFONT}
%    \begin{macrocode}
\def\HoLogo@METAFONT#1{%
  \HoLogoFont@font{METAFONT}{logo}{%
    \HOLOGO@mbox{META}%
    \HOLOGO@discretionary
    \HOLOGO@mbox{FONT}%
  }%
}
%    \end{macrocode}
%    \end{macro}
%
%    \begin{macro}{\HoLogo@METAPOST}
%    \begin{macrocode}
\def\HoLogo@METAPOST#1{%
  \HoLogoFont@font{METAPOST}{logo}{%
    \HOLOGO@mbox{META}%
    \HOLOGO@discretionary
    \HOLOGO@mbox{POST}%
  }%
}
%    \end{macrocode}
%    \end{macro}
%
%    \begin{macro}{\HoLogo@MetaFun}
%    \begin{macrocode}
\def\HoLogo@MetaFun#1{%
  \HOLOGO@mbox{Meta}%
  \HOLOGO@discretionary
  \HOLOGO@mbox{Fun}%
}
%    \end{macrocode}
%    \end{macro}
%
%    \begin{macro}{\HoLogo@MetaPost}
%    \begin{macrocode}
\def\HoLogo@MetaPost#1{%
  \HOLOGO@mbox{Meta}%
  \HOLOGO@discretionary
  \HOLOGO@mbox{Post}%
}
%    \end{macrocode}
%    \end{macro}
%
% \subsection{Others}
%
% \subsubsection{\hologo{biber}}
%
%    \begin{macro}{\HoLogo@biber}
%    \begin{macrocode}
\def\HoLogo@biber#1{%
  \HOLOGO@mbox{#1{b}{B}i}%
  \HOLOGO@discretionary
  \HOLOGO@mbox{ber}%
}
%    \end{macrocode}
%    \end{macro}
%    \begin{macro}{\HoLogoCs@biber}
%    \begin{macrocode}
\def\HoLogoCs@biber#1{#1{b}{B}iber}
%    \end{macrocode}
%    \end{macro}
%    \begin{macro}{\HoLogoBkm@biber}
%    \begin{macrocode}
\def\HoLogoBkm@biber#1{%
  #1{b}{B}iber%
}
%    \end{macrocode}
%    \end{macro}
%    \begin{macro}{\HoLogoHtml@biber}
%    \begin{macrocode}
\let\HoLogoHtml@biber\HoLogo@biber
%    \end{macrocode}
%    \end{macro}
%
% \subsubsection{\hologo{KOMAScript}}
%
%    \begin{macro}{\HoLogo@KOMAScript}
%    The definition for \hologo{KOMAScript} is taken
%    from \hologo{KOMAScript} (\xfile{scrlogo.dtx}, reformatted) \cite{scrlogo}:
%\begin{quote}
%\begin{verbatim}
%\@ifundefined{KOMAScript}{%
%  \DeclareRobustCommand{\KOMAScript}{%
%    \textsf{%
%      K\kern.05em O\kern.05emM\kern.05em A%
%      \kern.1em-\kern.1em %
%      Script%
%    }%
%  }%
%}{}
%\end{verbatim}
%\end{quote}
%    \begin{macrocode}
\def\HoLogo@KOMAScript#1{%
  \HoLogoFont@font{KOMAScript}{sf}{%
    \HOLOGO@mbox{%
      K\kern.05em%
      O\kern.05em%
      M\kern.05em%
      A%
    }%
    \kern.1em%
    \HOLOGO@hyphen
    \kern.1em%
    \HOLOGO@mbox{Script}%
  }%
}
%    \end{macrocode}
%    \end{macro}
%    \begin{macro}{\HoLogoBkm@KOMAScript}
%    \begin{macrocode}
\def\HoLogoBkm@KOMAScript#1{%
  KOMA-Script%
}
%    \end{macrocode}
%    \end{macro}
%    \begin{macro}{\HoLogoHtml@KOMAScript}
%    \begin{macrocode}
\def\HoLogoHtml@KOMAScript#1{%
  \HoLogoCss@KOMAScript
  \HoLogoFont@font{KOMAScript}{sf}{%
    \HOLOGO@Span{KOMAScript}{%
      K%
      \HOLOGO@Span{O}{O}%
      M%
      \HOLOGO@Span{A}{A}%
      \HOLOGO@Span{hyphen}{-}%
      Script%
    }%
  }%
}
%    \end{macrocode}
%    \end{macro}
%    \begin{macro}{\HoLogoCss@KOMAScript}
%    \begin{macrocode}
\def\HoLogoCss@KOMAScript{%
  \Css{%
    span.HoLogo-KOMAScript{%
      font-family:sans-serif;%
    }%
  }%
  \Css{%
    span.HoLogo-KOMAScript span.HoLogo-O{%
      padding-left:.05em;%
      padding-right:.05em;%
    }%
  }%
  \Css{%
    span.HoLogo-KOMAScript span.HoLogo-A{%
      padding-left:.05em;%
    }%
  }%
  \Css{%
    span.HoLogo-KOMAScript span.HoLogo-hyphen{%
      padding-left:.1em;%
      padding-right:.1em;%
    }%
  }%
  \global\let\HoLogoCss@KOMAScript\relax
}
%    \end{macrocode}
%    \end{macro}
%
% \subsubsection{\hologo{LyX}}
%
%    \begin{macro}{\HoLogo@LyX}
%    The definition is taken from the documentation source files
%    of \hologo{LyX}, \xfile{Intro.lyx} \cite{LyX}:
%\begin{quote}
%\begin{verbatim}
%\def\LyX{%
%  \texorpdfstring{%
%    L\kern-.1667em\lower.25em\hbox{Y}\kern-.125emX\@%
%  }{%
%    LyX%
%  }%
%}
%\end{verbatim}
%\end{quote}
%    \begin{macrocode}
\def\HoLogo@LyX#1{%
  L%
  \kern-.1667em%
  \lower.25em\hbox{Y}%
  \kern-.125em%
  X%
  \HOLOGO@SpaceFactor
}
%    \end{macrocode}
%    \end{macro}
%    \begin{macro}{\HoLogoHtml@LyX}
%    \begin{macrocode}
\def\HoLogoHtml@LyX#1{%
  \HoLogoCss@LyX
  \HOLOGO@Span{LyX}{%
    L%
    \HOLOGO@Span{y}{Y}%
    X%
  }%
}
%    \end{macrocode}
%    \end{macro}
%    \begin{macro}{\HoLogoCss@LyX}
%    \begin{macrocode}
\def\HoLogoCss@LyX{%
  \Css{%
    span.HoLogo-LyX span.HoLogo-y{%
      position:relative;%
      top:.25em;%
      margin-left:-.1667em;%
      margin-right:-.125em;%
      text-decoration:none;%
    }%
  }%
  \global\let\HoLogoCss@LyX\relax
}
%    \end{macrocode}
%    \end{macro}
%
% \subsubsection{\hologo{NTS}}
%
%    \begin{macro}{\HoLogo@NTS}
%    Definition for \hologo{NTS} can be found in
%    package \xpackage{etex\textunderscore man} for the \hologo{eTeX} manual \cite{etexman}
%    and in package \xpackage{dtklogos} \cite{dtklogos}:
%\begin{quote}
%\begin{verbatim}
%\def\NTS{%
%  \leavevmode
%  \hbox{%
%    $%
%      \cal N%
%      \kern-0.35em%
%      \lower0.5ex\hbox{$\cal T$}%
%      \kern-0.2em%
%      S%
%    $%
%  }%
%}
%\end{verbatim}
%\end{quote}
%    \begin{macrocode}
\def\HoLogo@NTS#1{%
  \HoLogoFont@font{NTS}{sy}{%
    N\/%
    \kern-.35em%
    \lower.5ex\hbox{T\/}%
    \kern-.2em%
    S\/%
  }%
  \HOLOGO@SpaceFactor
}
%    \end{macrocode}
%    \end{macro}
%
% \subsubsection{\Hologo{TTH} (\hologo{TeX} to HTML translator)}
%
%    Source: \url{http://hutchinson.belmont.ma.us/tth/}
%    In the HTML source the second `T' is printed as subscript.
%\begin{quote}
%\begin{verbatim}
%T<sub>T</sub>H
%\end{verbatim}
%\end{quote}
%    \begin{macro}{\HoLogo@TTH}
%    \begin{macrocode}
\def\HoLogo@TTH#1{%
  \ltx@mbox{%
    T\HOLOGO@SubScript{T}H%
  }%
  \HOLOGO@SpaceFactor
}
%    \end{macrocode}
%    \end{macro}
%
%    \begin{macro}{\HoLogoHtml@TTH}
%    \begin{macrocode}
\def\HoLogoHtml@TTH#1{%
  T\HCode{<sub>}T\HCode{</sub>}H%
}
%    \end{macrocode}
%    \end{macro}
%
% \subsubsection{\Hologo{HanTheThanh}}
%
%    Partial source: Package \xpackage{dtklogos}.
%    The double accent is U+1EBF (latin small letter e with circumflex
%    and acute).
%    \begin{macro}{\HoLogo@HanTheThanh}
%    \begin{macrocode}
\def\HoLogo@HanTheThanh#1{%
  \ltx@mbox{H\`an}%
  \HOLOGO@space
  \ltx@mbox{%
    Th%
    \HOLOGO@IfCharExists{"1EBF}{%
      \char"1EBF\relax
    }{%
      \^e\hbox to 0pt{\hss\raise .5ex\hbox{\'{}}}%
    }%
  }%
  \HOLOGO@space
  \ltx@mbox{Th\`anh}%
}
%    \end{macrocode}
%    \end{macro}
%    \begin{macro}{\HoLogoBkm@HanTheThanh}
%    \begin{macrocode}
\def\HoLogoBkm@HanTheThanh#1{%
  H\`an %
  Th\HOLOGO@PdfdocUnicode{\^e}{\9036\277} %
  Th\`anh%
}
%    \end{macrocode}
%    \end{macro}
%    \begin{macro}{\HoLogoHtml@HanTheThanh}
%    \begin{macrocode}
\def\HoLogoHtml@HanTheThanh#1{%
  H\`an %
  Th\HCode{&\ltx@hashchar x1ebf;} %
  Th\`anh%
}
%    \end{macrocode}
%    \end{macro}
%
% \subsection{Driver detection}
%
%    \begin{macrocode}
\HOLOGO@IfExists\InputIfFileExists{%
  \InputIfFileExists{hologo.cfg}{}{}%
}{%
  \ltx@IfUndefined{pdf@filesize}{%
    \def\HOLOGO@InputIfExists{%
      \openin\HOLOGO@temp=hologo.cfg\relax
      \ifeof\HOLOGO@temp
        \closein\HOLOGO@temp
      \else
        \closein\HOLOGO@temp
        \begingroup
          \def\x{LaTeX2e}%
        \expandafter\endgroup
        \ifx\fmtname\x
          \input{hologo.cfg}%
        \else
          \input hologo.cfg\relax
        \fi
      \fi
    }%
    \ltx@IfUndefined{newread}{%
      \chardef\HOLOGO@temp=15 %
      \def\HOLOGO@CheckRead{%
        \ifeof\HOLOGO@temp
          \HOLOGO@InputIfExists
        \else
          \ifcase\HOLOGO@temp
            \@PackageWarningNoLine{hologo}{%
              Configuration file ignored, because\MessageBreak
              a free read register could not be found%
            }%
          \else
            \begingroup
              \count\ltx@cclv=\HOLOGO@temp
              \advance\ltx@cclv by \ltx@minusone
              \edef\x{\endgroup
                \chardef\noexpand\HOLOGO@temp=\the\count\ltx@cclv
                \relax
              }%
            \x
          \fi
        \fi
      }%
    }{%
      \csname newread\endcsname\HOLOGO@temp
      \HOLOGO@InputIfExists
    }%
  }{%
    \edef\HOLOGO@temp{\pdf@filesize{hologo.cfg}}%
    \ifx\HOLOGO@temp\ltx@empty
    \else
      \ifnum\HOLOGO@temp>0 %
        \begingroup
          \def\x{LaTeX2e}%
        \expandafter\endgroup
        \ifx\fmtname\x
          \input{hologo.cfg}%
        \else
          \input hologo.cfg\relax
        \fi
      \else
        \@PackageInfoNoLine{hologo}{%
          Empty configuration file `hologo.cfg' ignored%
        }%
      \fi
    \fi
  }%
}
%    \end{macrocode}
%
%    \begin{macrocode}
\def\HOLOGO@temp#1#2{%
  \kv@define@key{HoLogoDriver}{#1}[]{%
    \begingroup
      \def\HOLOGO@temp{##1}%
      \ltx@onelevel@sanitize\HOLOGO@temp
      \ifx\HOLOGO@temp\ltx@empty
      \else
        \@PackageError{hologo}{%
          Value (\HOLOGO@temp) not permitted for option `#1'%
        }%
        \@ehc
      \fi
    \endgroup
    \def\hologoDriver{#2}%
  }%
}%
\def\HOLOGO@@temp#1#2{%
  \ifx\kv@value\relax
    \HOLOGO@temp{#1}{#1}%
  \else
    \HOLOGO@temp{#1}{#2}%
  \fi
}%
\kv@parse@normalized{%
  pdftex,%
  luatex=pdftex,%
  dvipdfm,%
  dvipdfmx=dvipdfm,%
  dvips,%
  dvipsone=dvips,%
  xdvi=dvips,%
  xetex,%
  vtex,%
}\HOLOGO@@temp
%    \end{macrocode}
%
%    \begin{macrocode}
\kv@define@key{HoLogoDriver}{driverfallback}{%
  \def\HOLOGO@DriverFallback{#1}%
}
%    \end{macrocode}
%
%    \begin{macro}{\HOLOGO@DriverFallback}
%    \begin{macrocode}
\def\HOLOGO@DriverFallback{dvips}
%    \end{macrocode}
%    \end{macro}
%
%    \begin{macro}{\hologoDriverSetup}
%    \begin{macrocode}
\def\hologoDriverSetup{%
  \let\hologoDriver\ltx@undefined
  \HOLOGO@DriverSetup
}
%    \end{macrocode}
%    \end{macro}
%
%    \begin{macro}{\HOLOGO@DriverSetup}
%    \begin{macrocode}
\def\HOLOGO@DriverSetup#1{%
  \kvsetkeys{HoLogoDriver}{#1}%
  \HOLOGO@CheckDriver
  \ltx@ifundefined{hologoDriver}{%
    \begingroup
    \edef\x{\endgroup
      \noexpand\kvsetkeys{HoLogoDriver}{\HOLOGO@DriverFallback}%
    }\x
  }{}%
  \@PackageInfoNoLine{hologo}{Using driver `\hologoDriver'}%
}
%    \end{macrocode}
%    \end{macro}
%
%    \begin{macro}{\HOLOGO@CheckDriver}
%    \begin{macrocode}
\def\HOLOGO@CheckDriver{%
  \ifpdf
    \def\hologoDriver{pdftex}%
    \let\HOLOGO@pdfliteral\pdfliteral
    \ifluatex
      \ifx\pdfextension\@undefined\else
        \protected\def\pdfliteral{\pdfextension literal}%
        \let\HOLOGO@pdfliteral\pdfliteral
      \fi
      \ltx@IfUndefined{HOLOGO@pdfliteral}{%
        \ifnum\luatexversion<36 %
        \else
          \begingroup
            \let\HOLOGO@temp\endgroup
            \ifcase0%
                \directlua{%
                  if tex.enableprimitives then %
                    tex.enableprimitives('HOLOGO@', {'pdfliteral'})%
                  else %
                    tex.print('1')%
                  end%
                }%
                \ifx\HOLOGO@pdfliteral\@undefined 1\fi%
                \relax%
              \endgroup
              \let\HOLOGO@temp\relax
              \global\let\HOLOGO@pdfliteral\HOLOGO@pdfliteral
            \fi%
          \HOLOGO@temp
        \fi
      }{}%
    \fi
    \ltx@IfUndefined{HOLOGO@pdfliteral}{%
      \@PackageWarningNoLine{hologo}{%
        Cannot find \string\pdfliteral
      }%
    }{}%
  \else
    \ifxetex
      \def\hologoDriver{xetex}%
    \else
      \ifvtex
        \def\hologoDriver{vtex}%
      \fi
    \fi
  \fi
}
%    \end{macrocode}
%    \end{macro}
%
%    \begin{macro}{\HOLOGO@WarningUnsupportedDriver}
%    \begin{macrocode}
\def\HOLOGO@WarningUnsupportedDriver#1{%
  \@PackageWarningNoLine{hologo}{%
    Logo `#1' needs driver specific macros,\MessageBreak
    but driver `\hologoDriver' is not supported.\MessageBreak
    Use a different driver or\MessageBreak
    load package `graphics' or `pgf'%
  }%
}
%    \end{macrocode}
%    \end{macro}
%
% \subsubsection{Reflect box macros}
%
%    Skip driver part if not needed.
%    \begin{macrocode}
\ltx@IfUndefined{reflectbox}{}{%
  \ltx@IfUndefined{rotatebox}{}{%
    \HOLOGO@AtEnd
  }%
}
\ltx@IfUndefined{pgftext}{}{%
  \HOLOGO@AtEnd
}
\ltx@IfUndefined{psscalebox}{}{%
  \HOLOGO@AtEnd
}
%    \end{macrocode}
%
%    \begin{macrocode}
\def\HOLOGO@temp{LaTeX2e}
\ifx\fmtname\HOLOGO@temp
  \RequirePackage{kvoptions}[2011/06/30]%
  \ProcessKeyvalOptions{HoLogoDriver}%
\fi
\HOLOGO@DriverSetup{}
%    \end{macrocode}
%
%    \begin{macro}{\HOLOGO@ReflectBox}
%    \begin{macrocode}
\def\HOLOGO@ReflectBox#1{%
  \begingroup
    \setbox\ltx@zero\hbox{\begingroup#1\endgroup}%
    \setbox\ltx@two\hbox{%
      \kern\wd\ltx@zero
      \csname HOLOGO@ScaleBox@\hologoDriver\endcsname{-1}{1}{%
        \hbox to 0pt{\copy\ltx@zero\hss}%
      }%
    }%
    \wd\ltx@two=\wd\ltx@zero
    \box\ltx@two
  \endgroup
}
%    \end{macrocode}
%    \end{macro}
%
%    \begin{macro}{\HOLOGO@PointReflectBox}
%    \begin{macrocode}
\def\HOLOGO@PointReflectBox#1{%
  \begingroup
    \setbox\ltx@zero\hbox{\begingroup#1\endgroup}%
    \setbox\ltx@two\hbox{%
      \kern\wd\ltx@zero
      \raise\ht\ltx@zero\hbox{%
        \csname HOLOGO@ScaleBox@\hologoDriver\endcsname{-1}{-1}{%
          \hbox to 0pt{\copy\ltx@zero\hss}%
        }%
      }%
    }%
    \wd\ltx@two=\wd\ltx@zero
    \box\ltx@two
  \endgroup
}
%    \end{macrocode}
%    \end{macro}
%
%    We must define all variants because of dynamic driver setup.
%    \begin{macrocode}
\def\HOLOGO@temp#1#2{#2}
%    \end{macrocode}
%
%    \begin{macro}{\HOLOGO@ScaleBox@pdftex}
%    \begin{macrocode}
\HOLOGO@temp{pdftex}{%
  \def\HOLOGO@ScaleBox@pdftex#1#2#3{%
    \HOLOGO@pdfliteral{%
      q #1 0 0 #2 0 0 cm%
    }%
    #3%
    \HOLOGO@pdfliteral{%
      Q%
    }%
  }%
}
%    \end{macrocode}
%    \end{macro}
%    \begin{macro}{\HOLOGO@ScaleBox@dvips}
%    \begin{macrocode}
\HOLOGO@temp{dvips}{%
  \def\HOLOGO@ScaleBox@dvips#1#2#3{%
    \special{ps:%
      gsave %
      currentpoint %
      currentpoint translate %
      #1 #2 scale %
      neg exch neg exch translate%
    }%
    #3%
    \special{ps:%
      currentpoint %
      grestore %
      moveto%
    }%
  }%
}
%    \end{macrocode}
%    \end{macro}
%    \begin{macro}{\HOLOGO@ScaleBox@dvipdfm}
%    \begin{macrocode}
\HOLOGO@temp{dvipdfm}{%
  \let\HOLOGO@ScaleBox@dvipdfm\HOLOGO@ScaleBox@dvips
}
%    \end{macrocode}
%    \end{macro}
%    Since \hologo{XeTeX} v0.6.
%    \begin{macro}{\HOLOGO@ScaleBox@xetex}
%    \begin{macrocode}
\HOLOGO@temp{xetex}{%
  \def\HOLOGO@ScaleBox@xetex#1#2#3{%
    \special{x:gsave}%
    \special{x:scale #1 #2}%
    #3%
    \special{x:grestore}%
  }%
}
%    \end{macrocode}
%    \end{macro}
%    \begin{macro}{\HOLOGO@ScaleBox@vtex}
%    \begin{macrocode}
\HOLOGO@temp{vtex}{%
  \def\HOLOGO@ScaleBox@vtex#1#2#3{%
    \special{r(#1,0,0,#2,0,0}%
    #3%
    \special{r)}%
  }%
}
%    \end{macrocode}
%    \end{macro}
%
%    \begin{macrocode}
\HOLOGO@AtEnd%
%</package>
%    \end{macrocode}
%
% \section{Test}
%
% \subsection{Catcode checks for loading}
%
%    \begin{macrocode}
%<*test1>
%    \end{macrocode}
%    \begin{macrocode}
\catcode`\{=1 %
\catcode`\}=2 %
\catcode`\#=6 %
\catcode`\@=11 %
\expandafter\ifx\csname count@\endcsname\relax
  \countdef\count@=255 %
\fi
\expandafter\ifx\csname @gobble\endcsname\relax
  \long\def\@gobble#1{}%
\fi
\expandafter\ifx\csname @firstofone\endcsname\relax
  \long\def\@firstofone#1{#1}%
\fi
\expandafter\ifx\csname loop\endcsname\relax
  \expandafter\@firstofone
\else
  \expandafter\@gobble
\fi
{%
  \def\loop#1\repeat{%
    \def\body{#1}%
    \iterate
  }%
  \def\iterate{%
    \body
      \let\next\iterate
    \else
      \let\next\relax
    \fi
    \next
  }%
  \let\repeat=\fi
}%
\def\RestoreCatcodes{}
\count@=0 %
\loop
  \edef\RestoreCatcodes{%
    \RestoreCatcodes
    \catcode\the\count@=\the\catcode\count@\relax
  }%
\ifnum\count@<255 %
  \advance\count@ 1 %
\repeat

\def\RangeCatcodeInvalid#1#2{%
  \count@=#1\relax
  \loop
    \catcode\count@=15 %
  \ifnum\count@<#2\relax
    \advance\count@ 1 %
  \repeat
}
\def\RangeCatcodeCheck#1#2#3{%
  \count@=#1\relax
  \loop
    \ifnum#3=\catcode\count@
    \else
      \errmessage{%
        Character \the\count@\space
        with wrong catcode \the\catcode\count@\space
        instead of \number#3%
      }%
    \fi
  \ifnum\count@<#2\relax
    \advance\count@ 1 %
  \repeat
}
\def\space{ }
\expandafter\ifx\csname LoadCommand\endcsname\relax
  \def\LoadCommand{\input hologo.sty\relax}%
\fi
\def\Test{%
  \RangeCatcodeInvalid{0}{47}%
  \RangeCatcodeInvalid{58}{64}%
  \RangeCatcodeInvalid{91}{96}%
  \RangeCatcodeInvalid{123}{255}%
  \catcode`\@=12 %
  \catcode`\\=0 %
  \catcode`\%=14 %
  \LoadCommand
  \RangeCatcodeCheck{0}{36}{15}%
  \RangeCatcodeCheck{37}{37}{14}%
  \RangeCatcodeCheck{38}{47}{15}%
  \RangeCatcodeCheck{48}{57}{12}%
  \RangeCatcodeCheck{58}{63}{15}%
  \RangeCatcodeCheck{64}{64}{12}%
  \RangeCatcodeCheck{65}{90}{11}%
  \RangeCatcodeCheck{91}{91}{15}%
  \RangeCatcodeCheck{92}{92}{0}%
  \RangeCatcodeCheck{93}{96}{15}%
  \RangeCatcodeCheck{97}{122}{11}%
  \RangeCatcodeCheck{123}{255}{15}%
  \RestoreCatcodes
}
\Test
\csname @@end\endcsname
\end
%    \end{macrocode}
%    \begin{macrocode}
%</test1>
%    \end{macrocode}
%
% \subsection{Spacefactor}
%
%    The space factor must be 1000 after a logo. If it is greater 1000
%    then the following space is a space after a sentence closing point.
%    If the space factor is smaller 1000 then an immediate following
%    dot is interpreted as abbreviation, not sentence closing point.
%
%    \begin{macrocode}
%<*test-spacefactor>
\NeedsTeXFormat{LaTeX2e}
\documentclass{article}
\usepackage{hologo}[2016/05/12]
\usepackage{kvsetkeys}
\usepackage{qstest}
\IncludeTests{*}
\LogTests{log}{*}{*}
\begin{document}
\begin{qstest}{spacefactor}{spacefactor}
\newcommand*{\Test}[1]{%
  \sbox0{%
    \hologo{#1}%
    \Expect*{1000 (#1)}*{\the\spacefactor\space(#1)}%
  }%
}%
\makeatletter
\def\TestList{}
\def\hologoEntry#1#2#3{%
  \edef\TestList{%
    \ifx\TestList\@empty
    \else
      \TestList,%
    \fi
    #1%
    \ifx\\#2\\%
    \else
      ={variant=#2}%
    \fi
  }%
}
\hologoList
\expandafter\kv@parse@normalized\expandafter{%
  \TestList
}{%
  \begingroup
    \let\@logo=\kv@key
    \ifx\kv@value\relax
    \else
      \expandafter\hologoLogoSetup\expandafter\@logo\expandafter{%
        \kv@value
      }%
    \fi
    \Test\@logo
  \endgroup
  \@gobbletwo
}
\end{qstest}
\end{document}
%</test-spacefactor>
%    \end{macrocode}
%
% \subsection{Complete list}
%
%    \begin{macrocode}
%<*test-list>
\NeedsTeXFormat{LaTeX2e}
\documentclass[12pt,a4paper]{article}
\usepackage{hologo}[2016/05/12]
\usepackage[T1]{fontenc}
\usepackage{lmodern}
\usepackage{parskip}
\usepackage[unicode]{hyperref}[2011/09/28]
\usepackage{bookmark}[2011/09/19]
\bookmarksetup{%
  numbered,%
  open,%
  openlevel=2,%
}
\renewcommand*{\contentsname}{List of logos}
\begin{document}
\tableofcontents
\def\TestFont#1#2#3#4#5#6{%
  \begingroup
    \usefont{#3}{#4}{#5}{#6}%
    \HologoVariant{#1}{#2}/\hologoVariant{#1}{#2}%
    \quad
    \begingroup\scriptsize\hologoVariant{#1}{#2}\endgroup
    \quad
  \endgroup
  (#3/#4/#5/#6)%
  \par
}
\makeatletter
\def\hologoEntry#1#2#3{%
  \section{%
    \HologoVariant{#1}{#2}/\hologoVariant{#1}{#2} %
    {[#1\ifx\\#2\\\else\space(#2)\fi]}% hash-ok
  }% braces around [] because of bug in tex4ht
  \begingroup
    \hypersetup{unicode=false}%
    \bookmark[%
      dest=\@currentHref,%
      rellevel=1,%
      keeplevel,%
    ]{%
      \HologoVariant{#1}{#2}/\hologoVariant{#1}{#2} %
      (PDFDocEncoding)%
    }%
  \endgroup
  \TestFont{#1}{#2}{OT1}{cmr}{m}{n}%
  \TestFont{#1}{#2}{OT1}{cmss}{m}{n}%
  \TestFont{#1}{#2}{OT1}{cmr}{b}{n}%
  \TestFont{#1}{#2}{OT1}{cmr}{m}{it}%
  \TestFont{#1}{#2}{OT1}{cmtt}{m}{n}%
  \TestFont{#1}{#2}{T1}{lmr}{m}{n}%
  \TestFont{#1}{#2}{T1}{lmss}{m}{n}%
  \TestFont{#1}{#2}{T1}{lmr}{b}{n}%
  \TestFont{#1}{#2}{T1}{lmr}{m}{it}%
  \TestFont{#1}{#2}{T1}{lmtt}{m}{n}%
  \TestFont{#1}{#2}{T1}{lmvtt}{m}{n}%
  \TestFont{#1}{#2}{T1}{qtm}{m}{n}%
  \TestFont{#1}{#2}{T1}{qhv}{m}{n}%
  \TestFont{#1}{#2}{T1}{qtm}{b}{n}%
  \TestFont{#1}{#2}{T1}{qtm}{m}{it}%
  \TestFont{#1}{#2}{T1}{qcr}{m}{n}%
  \newpage
}
\makeatother
\hologoList
\end{document}
%</test-list>
%    \end{macrocode}
%
% \section{Installation}
%
% \subsection{Download}
%
% \paragraph{Package.} This package is available on
% CTAN\footnote{\url{ftp://ftp.ctan.org/tex-archive/}}:
% \begin{description}
% \item[\CTAN{macros/latex/contrib/oberdiek/hologo.dtx}] The source file.
% \item[\CTAN{macros/latex/contrib/oberdiek/hologo.pdf}] Documentation.
% \end{description}
%
%
% \paragraph{Bundle.} All the packages of the bundle `oberdiek'
% are also available in a TDS compliant ZIP archive. There
% the packages are already unpacked and the documentation files
% are generated. The files and directories obey the TDS standard.
% \begin{description}
% \item[\CTAN{install/macros/latex/contrib/oberdiek.tds.zip}]
% \end{description}
% \emph{TDS} refers to the standard ``A Directory Structure
% for \TeX\ Files'' (\CTAN{tds/tds.pdf}). Directories
% with \xfile{texmf} in their name are usually organized this way.
%
% \subsection{Bundle installation}
%
% \paragraph{Unpacking.} Unpack the \xfile{oberdiek.tds.zip} in the
% TDS tree (also known as \xfile{texmf} tree) of your choice.
% Example (linux):
% \begin{quote}
%   |unzip oberdiek.tds.zip -d ~/texmf|
% \end{quote}
%
% \paragraph{Script installation.}
% Check the directory \xfile{TDS:scripts/oberdiek/} for
% scripts that need further installation steps.
% Package \xpackage{attachfile2} comes with the Perl script
% \xfile{pdfatfi.pl} that should be installed in such a way
% that it can be called as \texttt{pdfatfi}.
% Example (linux):
% \begin{quote}
%   |chmod +x scripts/oberdiek/pdfatfi.pl|\\
%   |cp scripts/oberdiek/pdfatfi.pl /usr/local/bin/|
% \end{quote}
%
% \subsection{Package installation}
%
% \paragraph{Unpacking.} The \xfile{.dtx} file is a self-extracting
% \docstrip\ archive. The files are extracted by running the
% \xfile{.dtx} through \plainTeX:
% \begin{quote}
%   \verb|tex hologo.dtx|
% \end{quote}
%
% \paragraph{TDS.} Now the different files must be moved into
% the different directories in your installation TDS tree
% (also known as \xfile{texmf} tree):
% \begin{quote}
% \def\t{^^A
% \begin{tabular}{@{}>{\ttfamily}l@{ $\rightarrow$ }>{\ttfamily}l@{}}
%   hologo.sty & tex/generic/oberdiek/hologo.sty\\
%   hologo.pdf & doc/latex/oberdiek/hologo.pdf\\
%   example/hologo-example.tex & doc/latex/oberdiek/example/hologo-example.tex\\
%   test/hologo-test1.tex & doc/latex/oberdiek/test/hologo-test1.tex\\
%   test/hologo-test-spacefactor.tex & doc/latex/oberdiek/test/hologo-test-spacefactor.tex\\
%   test/hologo-test-list.tex & doc/latex/oberdiek/test/hologo-test-list.tex\\
%   hologo.dtx & source/latex/oberdiek/hologo.dtx\\
% \end{tabular}^^A
% }^^A
% \sbox0{\t}^^A
% \ifdim\wd0>\linewidth
%   \begingroup
%     \advance\linewidth by\leftmargin
%     \advance\linewidth by\rightmargin
%   \edef\x{\endgroup
%     \def\noexpand\lw{\the\linewidth}^^A
%   }\x
%   \def\lwbox{^^A
%     \leavevmode
%     \hbox to \linewidth{^^A
%       \kern-\leftmargin\relax
%       \hss
%       \usebox0
%       \hss
%       \kern-\rightmargin\relax
%     }^^A
%   }^^A
%   \ifdim\wd0>\lw
%     \sbox0{\small\t}^^A
%     \ifdim\wd0>\linewidth
%       \ifdim\wd0>\lw
%         \sbox0{\footnotesize\t}^^A
%         \ifdim\wd0>\linewidth
%           \ifdim\wd0>\lw
%             \sbox0{\scriptsize\t}^^A
%             \ifdim\wd0>\linewidth
%               \ifdim\wd0>\lw
%                 \sbox0{\tiny\t}^^A
%                 \ifdim\wd0>\linewidth
%                   \lwbox
%                 \else
%                   \usebox0
%                 \fi
%               \else
%                 \lwbox
%               \fi
%             \else
%               \usebox0
%             \fi
%           \else
%             \lwbox
%           \fi
%         \else
%           \usebox0
%         \fi
%       \else
%         \lwbox
%       \fi
%     \else
%       \usebox0
%     \fi
%   \else
%     \lwbox
%   \fi
% \else
%   \usebox0
% \fi
% \end{quote}
% If you have a \xfile{docstrip.cfg} that configures and enables \docstrip's
% TDS installing feature, then some files can already be in the right
% place, see the documentation of \docstrip.
%
% \subsection{Refresh file name databases}
%
% If your \TeX~distribution
% (\teTeX, \mikTeX, \dots) relies on file name databases, you must refresh
% these. For example, \teTeX\ users run \verb|texhash| or
% \verb|mktexlsr|.
%
% \subsection{Some details for the interested}
%
% \paragraph{Attached source.}
%
% The PDF documentation on CTAN also includes the
% \xfile{.dtx} source file. It can be extracted by
% AcrobatReader 6 or higher. Another option is \textsf{pdftk},
% e.g. unpack the file into the current directory:
% \begin{quote}
%   \verb|pdftk hologo.pdf unpack_files output .|
% \end{quote}
%
% \paragraph{Unpacking with \LaTeX.}
% The \xfile{.dtx} chooses its action depending on the format:
% \begin{description}
% \item[\plainTeX:] Run \docstrip\ and extract the files.
% \item[\LaTeX:] Generate the documentation.
% \end{description}
% If you insist on using \LaTeX\ for \docstrip\ (really,
% \docstrip\ does not need \LaTeX), then inform the autodetect routine
% about your intention:
% \begin{quote}
%   \verb|latex \let\install=y\input{hologo.dtx}|
% \end{quote}
% Do not forget to quote the argument according to the demands
% of your shell.
%
% \paragraph{Generating the documentation.}
% You can use both the \xfile{.dtx} or the \xfile{.drv} to generate
% the documentation. The process can be configured by the
% configuration file \xfile{ltxdoc.cfg}. For instance, put this
% line into this file, if you want to have A4 as paper format:
% \begin{quote}
%   \verb|\PassOptionsToClass{a4paper}{article}|
% \end{quote}
% An example follows how to generate the
% documentation with pdf\LaTeX:
% \begin{quote}
%\begin{verbatim}
%pdflatex hologo.dtx
%makeindex -s gind.ist hologo.idx
%pdflatex hologo.dtx
%makeindex -s gind.ist hologo.idx
%pdflatex hologo.dtx
%\end{verbatim}
% \end{quote}
%
% \section{Catalogue}
%
% The following XML file can be used as source for the
% \href{http://mirror.ctan.org/help/Catalogue/catalogue.html}{\TeX\ Catalogue}.
% The elements \texttt{caption} and \texttt{description} are imported
% from the original XML file from the Catalogue.
% The name of the XML file in the Catalogue is \xfile{hologo.xml}.
%    \begin{macrocode}
%<*catalogue>
<?xml version='1.0' encoding='us-ascii'?>
<!DOCTYPE entry SYSTEM 'catalogue.dtd'>
<entry datestamp='$Date$' modifier='$Author$' id='hologo'>
  <name>hologo</name>
  <caption>A collection of logos with bookmark support.</caption>
  <authorref id='auth:oberdiek'/>
  <copyright owner='Heiko Oberdiek' year='2010-2012'/>
  <license type='lppl1.3'/>
  <version number='1.10'/>
  <description>
    The package defines a single command <tt>\hologo</tt>, whose
    argument is the usual case-confused ASCII version of the logo.
    The command is bookmark-enabled, so that every logo becomes
    available in bookmarks without further work.
    <p/>
    The package is part of the <xref refid='oberdiek'>oberdiek</xref>
    bundle.
  </description>
  <documentation details='Package documentation'
      href='ctan:/macros/latex/contrib/oberdiek/hologo.pdf'/>
  <ctan file='true' path='/macros/latex/contrib/oberdiek/hologo.dtx'/>
  <miktex location='oberdiek'/>
  <texlive location='oberdiek'/>
  <install path='/macros/latex/contrib/oberdiek/oberdiek.tds.zip'/>
</entry>
%</catalogue>
%    \end{macrocode}
%
% \begin{thebibliography}{9}
% \raggedright
%
% \bibitem{btxdoc}
% Oren Patashnik,
% \textit{\hologo{BibTeX}ing},
% 1988-02-08.\\
% \CTAN{biblio/bibtex/base/}
%
% \bibitem{dtklogos}
% Gerd Neugebauer, DANTE,
% \textit{Package \xpackage{dtklogos}},
% 2011-04-25.\\
% \CTAN{usergrps/dante/dtk/dtklogos.sty}
%
% \bibitem{etexman}
% The \hologo{NTS} Team,
% \textit{The \hologo{eTeX} manual},
% 1998-02.\\
% \CTAN{systems/e-tex/v2/doc/}
%
% \bibitem{ExTeX-FAQ}
% The \hologo{ExTeX} group,
% \textit{\hologo{ExTeX}: FAQ -- How is \hologo{ExTeX} typeset?},
% 2007-04-14.\\
% \url{http://www.extex.org/documentation/faq.html}
%
% \bibitem{LyX}
% %@MISC{ LyX,
% %  title = {{LyX 2.0.0 -- The Document Processor [Computer software and manual]}},
% %  author = {{The LyX Team}},
% %  howpublished = {Internet: http://www.lyx.org},
% %  year = {2011-05-08},
% %  note = {Retrieved May 10, 2011, from http://www.lyx.org},
% %  url = {http://www.lyx.org/}
% %}
% The \hologo{LyX} Team,
% \textit{\hologo{LyX} -- The Document Processor},
% 2011-05-08.\\
% \url{http://www.lyx.org/}
%
% \bibitem{OzTeX}
% Andrew Trevorrow,
% \hologo{OzTeX} FAQ: What is the correct way to typeset ``\hologo{OzTeX}''?,
% 2011-09-15 (visited).
% \url{http://www.trevorrow.com/oztex/ozfaq.html#oztex-logo}
%
% \bibitem{PiCTeX}
% Michael Wichura,
% \textit{The \hologo{PiCTeX} macro package},
% 1987-09-21.
% \CTAN{graphics/pictex/}
%
% \bibitem{scrlogo}
% Markus Kohm,
% \textit{\hologo{KOMAScript} Datei \xfile{scrlogo.dtx}},
% 2009-01-30.\\
% \CTAN{install/macros/latex/contrib/komascript.tds.zip}
%
% \end{thebibliography}
%
% \begin{History}
%   \begin{Version}{2010/04/08 v1.0}
%   \item
%     The first version.
%   \end{Version}
%   \begin{Version}{2010/04/16 v1.1}
%   \item
%     \cs{Hologo} added for support of logos at start of a sentence.
%   \item
%     \cs{hologoSetup} and \cs{hologoLogoSetup} added.
%   \item
%     Options \xoption{break}, \xoption{hyphenbreak}, \xoption{spacebreak}
%     added.
%   \item
%     Variant support added by option \xoption{variant}.
%   \end{Version}
%   \begin{Version}{2010/04/24 v1.2}
%   \item
%     \hologo{LaTeX3} added.
%   \item
%     \hologo{VTeX} added.
%   \end{Version}
%   \begin{Version}{2010/11/21 v1.3}
%   \item
%     \hologo{iniTeX}, \hologo{virTeX} added.
%   \end{Version}
%   \begin{Version}{2011/03/25 v1.4}
%   \item
%     \hologo{ConTeXt} with variants added.
%   \item
%     Option \xoption{discretionarybreak} added as refinement for
%     option \xoption{break}.
%   \end{Version}
%   \begin{Version}{2011/04/21 v1.5}
%   \item
%     Wrong TDS directory for test files fixed.
%   \end{Version}
%   \begin{Version}{2011/10/01 v1.6}
%   \item
%     Support for package \xpackage{tex4ht} added.
%   \item
%     Support for \cs{csname} added if \cs{ifincsname} is available.
%   \item
%     New logos:
%     \hologo{(La)TeX},
%     \hologo{biber},
%     \hologo{BibTeX} (\xoption{sc}, \xoption{sf}),
%     \hologo{emTeX},
%     \hologo{ExTeX},
%     \hologo{KOMAScript},
%     \hologo{La},
%     \hologo{LyX},
%     \hologo{MiKTeX},
%     \hologo{NTS},
%     \hologo{OzMF},
%     \hologo{OzMP},
%     \hologo{OzTeX},
%     \hologo{OzTtH},
%     \hologo{PCTeX},
%     \hologo{PiC},
%     \hologo{PiCTeX},
%     \hologo{METAFONT},
%     \hologo{MetaFun},
%     \hologo{METAPOST},
%     \hologo{MetaPost},
%     \hologo{SLiTeX} (\xoption{lift}, \xoption{narrow}, \xoption{simple}),
%     \hologo{SliTeX} (\xoption{narrow}, \xoption{simple}, \xoption{lift}),
%     \hologo{teTeX}.
%   \item
%     Fixes:
%     \hologo{iniTeX},
%     \hologo{pdfLaTeX},
%     \hologo{pdfTeX},
%     \hologo{virTeX}.
%   \item
%     \cs{hologoFontSetup} and \cs{hologoLogoFontSetup} added.
%   \item
%     \cs{hologoVariant} and \cs{HologoVariant} added.
%   \end{Version}
%   \begin{Version}{2011/11/22 v1.7}
%   \item
%     New logos:
%     \hologo{BibTeX8},
%     \hologo{LaTeXML},
%     \hologo{SageTeX},
%     \hologo{TeX4ht},
%     \hologo{TTH}.
%   \item
%     \hologo{Xe} and friends: Driver stuff fixed.
%   \item
%     \hologo{Xe} and friends: Support for italic added.
%   \item
%     \hologo{Xe} and friends: Package support for \xpackage{pgf}
%     and \xpackage{pstricks} added.
%   \end{Version}
%   \begin{Version}{2011/11/29 v1.8}
%   \item
%     New logos:
%     \hologo{HanTheThanh}.
%   \end{Version}
%   \begin{Version}{2011/12/21 v1.9}
%   \item
%     Patch for package \xpackage{ifxetex} added for the case that
%     \cs{newif} is undefined in \hologo{iniTeX}.
%   \item
%     Some fixes for \hologo{iniTeX}.
%   \end{Version}
%   \begin{Version}{2012/04/26 v1.10}
%   \item
%     Fix in bookmark version of logo ``\hologo{HanTheThanh}''.
%   \end{Version}
%   \begin{Version}{2016/05/12 v1.11}
%   \item
%     Update HOLOGO@IfCharExists (previously in texlive)
%   \item define pdfliteral in current luatex.
%   \end{Version}
% \end{History}
%
% \PrintIndex
%
% \Finale
\endinput
|
% \end{quote}
% Do not forget to quote the argument according to the demands
% of your shell.
%
% \paragraph{Generating the documentation.}
% You can use both the \xfile{.dtx} or the \xfile{.drv} to generate
% the documentation. The process can be configured by the
% configuration file \xfile{ltxdoc.cfg}. For instance, put this
% line into this file, if you want to have A4 as paper format:
% \begin{quote}
%   \verb|\PassOptionsToClass{a4paper}{article}|
% \end{quote}
% An example follows how to generate the
% documentation with pdf\LaTeX:
% \begin{quote}
%\begin{verbatim}
%pdflatex hologo.dtx
%makeindex -s gind.ist hologo.idx
%pdflatex hologo.dtx
%makeindex -s gind.ist hologo.idx
%pdflatex hologo.dtx
%\end{verbatim}
% \end{quote}
%
% \section{Catalogue}
%
% The following XML file can be used as source for the
% \href{http://mirror.ctan.org/help/Catalogue/catalogue.html}{\TeX\ Catalogue}.
% The elements \texttt{caption} and \texttt{description} are imported
% from the original XML file from the Catalogue.
% The name of the XML file in the Catalogue is \xfile{hologo.xml}.
%    \begin{macrocode}
%<*catalogue>
<?xml version='1.0' encoding='us-ascii'?>
<!DOCTYPE entry SYSTEM 'catalogue.dtd'>
<entry datestamp='$Date$' modifier='$Author$' id='hologo'>
  <name>hologo</name>
  <caption>A collection of logos with bookmark support.</caption>
  <authorref id='auth:oberdiek'/>
  <copyright owner='Heiko Oberdiek' year='2010-2012'/>
  <license type='lppl1.3'/>
  <version number='1.10'/>
  <description>
    The package defines a single command <tt>\hologo</tt>, whose
    argument is the usual case-confused ASCII version of the logo.
    The command is bookmark-enabled, so that every logo becomes
    available in bookmarks without further work.
    <p/>
    The package is part of the <xref refid='oberdiek'>oberdiek</xref>
    bundle.
  </description>
  <documentation details='Package documentation'
      href='ctan:/macros/latex/contrib/oberdiek/hologo.pdf'/>
  <ctan file='true' path='/macros/latex/contrib/oberdiek/hologo.dtx'/>
  <miktex location='oberdiek'/>
  <texlive location='oberdiek'/>
  <install path='/macros/latex/contrib/oberdiek/oberdiek.tds.zip'/>
</entry>
%</catalogue>
%    \end{macrocode}
%
% \begin{thebibliography}{9}
% \raggedright
%
% \bibitem{btxdoc}
% Oren Patashnik,
% \textit{\hologo{BibTeX}ing},
% 1988-02-08.\\
% \CTAN{biblio/bibtex/base/}
%
% \bibitem{dtklogos}
% Gerd Neugebauer, DANTE,
% \textit{Package \xpackage{dtklogos}},
% 2011-04-25.\\
% \CTAN{usergrps/dante/dtk/dtklogos.sty}
%
% \bibitem{etexman}
% The \hologo{NTS} Team,
% \textit{The \hologo{eTeX} manual},
% 1998-02.\\
% \CTAN{systems/e-tex/v2/doc/}
%
% \bibitem{ExTeX-FAQ}
% The \hologo{ExTeX} group,
% \textit{\hologo{ExTeX}: FAQ -- How is \hologo{ExTeX} typeset?},
% 2007-04-14.\\
% \url{http://www.extex.org/documentation/faq.html}
%
% \bibitem{LyX}
% %@MISC{ LyX,
% %  title = {{LyX 2.0.0 -- The Document Processor [Computer software and manual]}},
% %  author = {{The LyX Team}},
% %  howpublished = {Internet: http://www.lyx.org},
% %  year = {2011-05-08},
% %  note = {Retrieved May 10, 2011, from http://www.lyx.org},
% %  url = {http://www.lyx.org/}
% %}
% The \hologo{LyX} Team,
% \textit{\hologo{LyX} -- The Document Processor},
% 2011-05-08.\\
% \url{http://www.lyx.org/}
%
% \bibitem{OzTeX}
% Andrew Trevorrow,
% \hologo{OzTeX} FAQ: What is the correct way to typeset ``\hologo{OzTeX}''?,
% 2011-09-15 (visited).
% \url{http://www.trevorrow.com/oztex/ozfaq.html#oztex-logo}
%
% \bibitem{PiCTeX}
% Michael Wichura,
% \textit{The \hologo{PiCTeX} macro package},
% 1987-09-21.
% \CTAN{graphics/pictex/}
%
% \bibitem{scrlogo}
% Markus Kohm,
% \textit{\hologo{KOMAScript} Datei \xfile{scrlogo.dtx}},
% 2009-01-30.\\
% \CTAN{install/macros/latex/contrib/komascript.tds.zip}
%
% \end{thebibliography}
%
% \begin{History}
%   \begin{Version}{2010/04/08 v1.0}
%   \item
%     The first version.
%   \end{Version}
%   \begin{Version}{2010/04/16 v1.1}
%   \item
%     \cs{Hologo} added for support of logos at start of a sentence.
%   \item
%     \cs{hologoSetup} and \cs{hologoLogoSetup} added.
%   \item
%     Options \xoption{break}, \xoption{hyphenbreak}, \xoption{spacebreak}
%     added.
%   \item
%     Variant support added by option \xoption{variant}.
%   \end{Version}
%   \begin{Version}{2010/04/24 v1.2}
%   \item
%     \hologo{LaTeX3} added.
%   \item
%     \hologo{VTeX} added.
%   \end{Version}
%   \begin{Version}{2010/11/21 v1.3}
%   \item
%     \hologo{iniTeX}, \hologo{virTeX} added.
%   \end{Version}
%   \begin{Version}{2011/03/25 v1.4}
%   \item
%     \hologo{ConTeXt} with variants added.
%   \item
%     Option \xoption{discretionarybreak} added as refinement for
%     option \xoption{break}.
%   \end{Version}
%   \begin{Version}{2011/04/21 v1.5}
%   \item
%     Wrong TDS directory for test files fixed.
%   \end{Version}
%   \begin{Version}{2011/10/01 v1.6}
%   \item
%     Support for package \xpackage{tex4ht} added.
%   \item
%     Support for \cs{csname} added if \cs{ifincsname} is available.
%   \item
%     New logos:
%     \hologo{(La)TeX},
%     \hologo{biber},
%     \hologo{BibTeX} (\xoption{sc}, \xoption{sf}),
%     \hologo{emTeX},
%     \hologo{ExTeX},
%     \hologo{KOMAScript},
%     \hologo{La},
%     \hologo{LyX},
%     \hologo{MiKTeX},
%     \hologo{NTS},
%     \hologo{OzMF},
%     \hologo{OzMP},
%     \hologo{OzTeX},
%     \hologo{OzTtH},
%     \hologo{PCTeX},
%     \hologo{PiC},
%     \hologo{PiCTeX},
%     \hologo{METAFONT},
%     \hologo{MetaFun},
%     \hologo{METAPOST},
%     \hologo{MetaPost},
%     \hologo{SLiTeX} (\xoption{lift}, \xoption{narrow}, \xoption{simple}),
%     \hologo{SliTeX} (\xoption{narrow}, \xoption{simple}, \xoption{lift}),
%     \hologo{teTeX}.
%   \item
%     Fixes:
%     \hologo{iniTeX},
%     \hologo{pdfLaTeX},
%     \hologo{pdfTeX},
%     \hologo{virTeX}.
%   \item
%     \cs{hologoFontSetup} and \cs{hologoLogoFontSetup} added.
%   \item
%     \cs{hologoVariant} and \cs{HologoVariant} added.
%   \end{Version}
%   \begin{Version}{2011/11/22 v1.7}
%   \item
%     New logos:
%     \hologo{BibTeX8},
%     \hologo{LaTeXML},
%     \hologo{SageTeX},
%     \hologo{TeX4ht},
%     \hologo{TTH}.
%   \item
%     \hologo{Xe} and friends: Driver stuff fixed.
%   \item
%     \hologo{Xe} and friends: Support for italic added.
%   \item
%     \hologo{Xe} and friends: Package support for \xpackage{pgf}
%     and \xpackage{pstricks} added.
%   \end{Version}
%   \begin{Version}{2011/11/29 v1.8}
%   \item
%     New logos:
%     \hologo{HanTheThanh}.
%   \end{Version}
%   \begin{Version}{2011/12/21 v1.9}
%   \item
%     Patch for package \xpackage{ifxetex} added for the case that
%     \cs{newif} is undefined in \hologo{iniTeX}.
%   \item
%     Some fixes for \hologo{iniTeX}.
%   \end{Version}
%   \begin{Version}{2012/04/26 v1.10}
%   \item
%     Fix in bookmark version of logo ``\hologo{HanTheThanh}''.
%   \end{Version}
%   \begin{Version}{2016/05/12 v1.11}
%   \item
%     Update HOLOGO@IfCharExists (previously in texlive)
%   \item define pdfliteral in current luatex.
%   \end{Version}
% \end{History}
%
% \PrintIndex
%
% \Finale
\endinput
%
        \else
          \input hologo.cfg\relax
        \fi
      \fi
    }%
    \ltx@IfUndefined{newread}{%
      \chardef\HOLOGO@temp=15 %
      \def\HOLOGO@CheckRead{%
        \ifeof\HOLOGO@temp
          \HOLOGO@InputIfExists
        \else
          \ifcase\HOLOGO@temp
            \@PackageWarningNoLine{hologo}{%
              Configuration file ignored, because\MessageBreak
              a free read register could not be found%
            }%
          \else
            \begingroup
              \count\ltx@cclv=\HOLOGO@temp
              \advance\ltx@cclv by \ltx@minusone
              \edef\x{\endgroup
                \chardef\noexpand\HOLOGO@temp=\the\count\ltx@cclv
                \relax
              }%
            \x
          \fi
        \fi
      }%
    }{%
      \csname newread\endcsname\HOLOGO@temp
      \HOLOGO@InputIfExists
    }%
  }{%
    \edef\HOLOGO@temp{\pdf@filesize{hologo.cfg}}%
    \ifx\HOLOGO@temp\ltx@empty
    \else
      \ifnum\HOLOGO@temp>0 %
        \begingroup
          \def\x{LaTeX2e}%
        \expandafter\endgroup
        \ifx\fmtname\x
          % \iffalse meta-comment
%
% File: hologo.dtx
% Version: 2016/05/12 v1.11
% Info: A logo collection with bookmark support
%
% Copyright (C) 2010-2012 by
%    Heiko Oberdiek <heiko.oberdiek at googlemail.com>
%
% This work may be distributed and/or modified under the
% conditions of the LaTeX Project Public License, either
% version 1.3c of this license or (at your option) any later
% version. This version of this license is in
%    http://www.latex-project.org/lppl/lppl-1-3c.txt
% and the latest version of this license is in
%    http://www.latex-project.org/lppl.txt
% and version 1.3 or later is part of all distributions of
% LaTeX version 2005/12/01 or later.
%
% This work has the LPPL maintenance status "maintained".
%
% This Current Maintainer of this work is Heiko Oberdiek.
%
% The Base Interpreter refers to any `TeX-Format',
% because some files are installed in TDS:tex/generic//.
%
% This work consists of the main source file hologo.dtx
% and the derived files
%    hologo.sty, hologo.pdf, hologo.ins, hologo.drv, hologo-example.tex,
%    hologo-test1.tex, hologo-test-spacefactor.tex,
%    hologo-test-list.tex.
%
% Distribution:
%    CTAN:macros/latex/contrib/oberdiek/hologo.dtx
%    CTAN:macros/latex/contrib/oberdiek/hologo.pdf
%
% Unpacking:
%    (a) If hologo.ins is present:
%           tex hologo.ins
%    (b) Without hologo.ins:
%           tex hologo.dtx
%    (c) If you insist on using LaTeX
%           latex \let\install=y% \iffalse meta-comment
%
% File: hologo.dtx
% Version: 2016/05/12 v1.11
% Info: A logo collection with bookmark support
%
% Copyright (C) 2010-2012 by
%    Heiko Oberdiek <heiko.oberdiek at googlemail.com>
%
% This work may be distributed and/or modified under the
% conditions of the LaTeX Project Public License, either
% version 1.3c of this license or (at your option) any later
% version. This version of this license is in
%    http://www.latex-project.org/lppl/lppl-1-3c.txt
% and the latest version of this license is in
%    http://www.latex-project.org/lppl.txt
% and version 1.3 or later is part of all distributions of
% LaTeX version 2005/12/01 or later.
%
% This work has the LPPL maintenance status "maintained".
%
% This Current Maintainer of this work is Heiko Oberdiek.
%
% The Base Interpreter refers to any `TeX-Format',
% because some files are installed in TDS:tex/generic//.
%
% This work consists of the main source file hologo.dtx
% and the derived files
%    hologo.sty, hologo.pdf, hologo.ins, hologo.drv, hologo-example.tex,
%    hologo-test1.tex, hologo-test-spacefactor.tex,
%    hologo-test-list.tex.
%
% Distribution:
%    CTAN:macros/latex/contrib/oberdiek/hologo.dtx
%    CTAN:macros/latex/contrib/oberdiek/hologo.pdf
%
% Unpacking:
%    (a) If hologo.ins is present:
%           tex hologo.ins
%    (b) Without hologo.ins:
%           tex hologo.dtx
%    (c) If you insist on using LaTeX
%           latex \let\install=y\input{hologo.dtx}
%        (quote the arguments according to the demands of your shell)
%
% Documentation:
%    (a) If hologo.drv is present:
%           latex hologo.drv
%    (b) Without hologo.drv:
%           latex hologo.dtx; ...
%    The class ltxdoc loads the configuration file ltxdoc.cfg
%    if available. Here you can specify further options, e.g.
%    use A4 as paper format:
%       \PassOptionsToClass{a4paper}{article}
%
%    Programm calls to get the documentation (example):
%       pdflatex hologo.dtx
%       makeindex -s gind.ist hologo.idx
%       pdflatex hologo.dtx
%       makeindex -s gind.ist hologo.idx
%       pdflatex hologo.dtx
%
% Installation:
%    TDS:tex/generic/oberdiek/hologo.sty
%    TDS:doc/latex/oberdiek/hologo.pdf
%    TDS:doc/latex/oberdiek/example/hologo-example.tex
%    TDS:doc/latex/oberdiek/test/hologo-test1.tex
%    TDS:doc/latex/oberdiek/test/hologo-test-spacefactor.tex
%    TDS:doc/latex/oberdiek/test/hologo-test-list.tex
%    TDS:source/latex/oberdiek/hologo.dtx
%
%<*ignore>
\begingroup
  \catcode123=1 %
  \catcode125=2 %
  \def\x{LaTeX2e}%
\expandafter\endgroup
\ifcase 0\ifx\install y1\fi\expandafter
         \ifx\csname processbatchFile\endcsname\relax\else1\fi
         \ifx\fmtname\x\else 1\fi\relax
\else\csname fi\endcsname
%</ignore>
%<*install>
\input docstrip.tex
\Msg{************************************************************************}
\Msg{* Installation}
\Msg{* Package: hologo 2016/05/12 v1.11 A logo collection with bookmark support (HO)}
\Msg{************************************************************************}

\keepsilent
\askforoverwritefalse

\let\MetaPrefix\relax
\preamble

This is a generated file.

Project: hologo
Version: 2016/05/12 v1.11

Copyright (C) 2010-2012 by
   Heiko Oberdiek <heiko.oberdiek at googlemail.com>

This work may be distributed and/or modified under the
conditions of the LaTeX Project Public License, either
version 1.3c of this license or (at your option) any later
version. This version of this license is in
   http://www.latex-project.org/lppl/lppl-1-3c.txt
and the latest version of this license is in
   http://www.latex-project.org/lppl.txt
and version 1.3 or later is part of all distributions of
LaTeX version 2005/12/01 or later.

This work has the LPPL maintenance status "maintained".

This Current Maintainer of this work is Heiko Oberdiek.

The Base Interpreter refers to any `TeX-Format',
because some files are installed in TDS:tex/generic//.

This work consists of the main source file hologo.dtx
and the derived files
   hologo.sty, hologo.pdf, hologo.ins, hologo.drv, hologo-example.tex,
   hologo-test1.tex, hologo-test-spacefactor.tex,
   hologo-test-list.tex.

\endpreamble
\let\MetaPrefix\DoubleperCent

\generate{%
  \file{hologo.ins}{\from{hologo.dtx}{install}}%
  \file{hologo.drv}{\from{hologo.dtx}{driver}}%
  \usedir{tex/generic/oberdiek}%
  \file{hologo.sty}{\from{hologo.dtx}{package}}%
  \usedir{doc/latex/oberdiek/example}%
  \file{hologo-example.tex}{\from{hologo.dtx}{example}}%
  \usedir{doc/latex/oberdiek/test}%
  \file{hologo-test1.tex}{\from{hologo.dtx}{test1}}%
  \file{hologo-test-spacefactor.tex}{\from{hologo.dtx}{test-spacefactor}}%
  \file{hologo-test-list.tex}{\from{hologo.dtx}{test-list}}%
  \nopreamble
  \nopostamble
  \usedir{source/latex/oberdiek/catalogue}%
  \file{hologo.xml}{\from{hologo.dtx}{catalogue}}%
}

\catcode32=13\relax% active space
\let =\space%
\Msg{************************************************************************}
\Msg{*}
\Msg{* To finish the installation you have to move the following}
\Msg{* file into a directory searched by TeX:}
\Msg{*}
\Msg{*     hologo.sty}
\Msg{*}
\Msg{* To produce the documentation run the file `hologo.drv'}
\Msg{* through LaTeX.}
\Msg{*}
\Msg{* Happy TeXing!}
\Msg{*}
\Msg{************************************************************************}

\endbatchfile
%</install>
%<*ignore>
\fi
%</ignore>
%<*driver>
\NeedsTeXFormat{LaTeX2e}
\ProvidesFile{hologo.drv}%
  [2016/05/12 v1.11 A logo collection with bookmark support (HO)]%
\documentclass{ltxdoc}
\usepackage{holtxdoc}[2011/11/22]
\usepackage{hologo}[2016/05/12]
\usepackage{longtable}
\usepackage{array}
\usepackage{paralist}
%\usepackage[T1]{fontenc}
%\usepackage{lmodern}
\begin{document}
  \DocInput{hologo.dtx}%
\end{document}
%</driver>
% \fi
%
%
% \CharacterTable
%  {Upper-case    \A\B\C\D\E\F\G\H\I\J\K\L\M\N\O\P\Q\R\S\T\U\V\W\X\Y\Z
%   Lower-case    \a\b\c\d\e\f\g\h\i\j\k\l\m\n\o\p\q\r\s\t\u\v\w\x\y\z
%   Digits        \0\1\2\3\4\5\6\7\8\9
%   Exclamation   \!     Double quote  \"     Hash (number) \#
%   Dollar        \$     Percent       \%     Ampersand     \&
%   Acute accent  \'     Left paren    \(     Right paren   \)
%   Asterisk      \*     Plus          \+     Comma         \,
%   Minus         \-     Point         \.     Solidus       \/
%   Colon         \:     Semicolon     \;     Less than     \<
%   Equals        \=     Greater than  \>     Question mark \?
%   Commercial at \@     Left bracket  \[     Backslash     \\
%   Right bracket \]     Circumflex    \^     Underscore    \_
%   Grave accent  \`     Left brace    \{     Vertical bar  \|
%   Right brace   \}     Tilde         \~}
%
% \GetFileInfo{hologo.drv}
%
% \title{The \xpackage{hologo} package}
% \date{2016/05/12 v1.11}
% \author{Heiko Oberdiek\\\xemail{heiko.oberdiek at googlemail.com}}
%
% \maketitle
%
% \begin{abstract}
% This package starts a collection of logos with support for bookmarks
% strings.
% \end{abstract}
%
% \tableofcontents
%
% \section{Documentation}
%
% \subsection{Logo macros}
%
% \begin{declcs}{hologo} \M{name}
% \end{declcs}
% Macro \cs{hologo} sets the logo with name \meta{name}.
% The following table shows the supported names.
%
% \begingroup
%   \def\hologoEntry#1#2#3{^^A
%     #1&#2&\hologoLogoSetup{#1}{variant=#2}\hologo{#1}&#3\tabularnewline
%   }
%   \begin{longtable}{>{\ttfamily}l>{\ttfamily}lll}
%     \rmfamily\bfseries{name} & \rmfamily\bfseries variant
%     & \bfseries logo & \bfseries since\\
%     \hline
%     \endhead
%     \hologoList
%   \end{longtable}
% \endgroup
%
% \begin{declcs}{Hologo} \M{name}
% \end{declcs}
% Macro \cs{Hologo} starts the logo \meta{name} with an uppercase
% letter. As an exception small greek letters are not converted
% to uppercase. Examples, see \hologo{eTeX} and \hologo{ExTeX}.
%
% \subsection{Setup macros}
%
% The package does not support package options, but the following
% setup macros can be used to set options.
%
% \begin{declcs}{hologoSetup} \M{key value list}
% \end{declcs}
% Macro \cs{hologoSetup} sets global options.
%
% \begin{declcs}{hologoLogoSetup} \M{logo} \M{key value list}
% \end{declcs}
% Some options can also be used to configure a logo.
% These settings take precedence over global option settings.
%
% \subsection{Options}\label{sec:options}
%
% There are boolean and string options:
% \begin{description}
% \item[Boolean option:]
% It takes |true| or |false|
% as value. If the value is omitted, then |true| is used.
% \item[String option:]
% A value must be given as string. (But the string might be empty.)
% \end{description}
% The following options can be used both in \cs{hologoSetup}
% and \cs{hologoLogoSetup}:
% \begin{description}
% \def\entry#1{\item[\xoption{#1}:]}
% \entry{break}
%   enables or disables line breaks inside the logo. This setting is
%   refined by options \xoption{hyphenbreak}, \xoption{spacebreak}
%   or \xoption{discretionarybreak}.
%   Default is |false|.
% \entry{hyphenbreak}
%   enables or disables the line break right after the hyphen character.
% \entry{spacebreak}
%   enables or disables line breaks at space characters.
% \entry{discretionarybreak}
%   enables or disables line breaks at hyphenation points
%   (inserted by \cs{-}).
% \end{description}
% Macro \cs{hologoLogoSetup} also knows:
% \begin{description}
% \item[\xoption{variant}:]
%   This is a string option. It specifies a variant of a logo that
%   must exist. An empty string selects the package default variant.
% \end{description}
% Example:
% \begin{quote}
%   |\hologoSetup{break=false}|\\
%   |\hologoLogoSetup{plainTeX}{variant=hyphen,hyphenbreak}|\\
%   Then ``plain-\TeX'' contains one break point after the hyphen.
% \end{quote}
%
% \subsection{Driver options}
%
% Sometimes graphical operations are needed to construct some
% glyphs (e.g.\ \hologo{XeTeX}). If package \xpackage{graphics}
% or package \xpackage{pgf} are found, then the macros are taken
% from there. Otherwise the packge defines its own operations
% and therefore needs the driver information. Many drivers are
% detected automatically (\hologo{pdfTeX}/\hologo{LuaTeX}
% in PDF mode, \hologo{XeTeX}, \hologo{VTeX}). These have precedence
% over a driver option. The driver can be given as package option
% or using \cs{hologoDriverSetup}.
% The following list contains the recognized driver options:
% \begin{itemize}
% \item \xoption{pdftex}, \xoption{luatex}
% \item \xoption{dvipdfm}, \xoption{dvipdfmx}
% \item \xoption{dvips}, \xoption{dvipsone}, \xoption{xdvi}
% \item \xoption{xetex}
% \item \xoption{vtex}
% \end{itemize}
% The left driver of a line is the driver name that is used internally.
% The following names are aliases for drivers that use the
% same method. Therefore the entry in the \xext{log} file for
% the used driver prints the internally used driver name.
% \begin{description}
% \item[\xoption{driverfallback}:]
%   This option expects a driver that is used,
%   if the driver could not be detected automatically.
% \end{description}
%
% \begin{declcs}{hologoDriverSetup} \M{driver option}
% \end{declcs}
% The driver can also be configured after package loading
% using \cs{hologoDriverSetup}, also the way for \hologo{plainTeX}
% to setup the driver.
%
% \subsection{Font setup}
%
% Some logos require a special font, but should also be usable by
% \hologo{plainTeX}. Therefore the package provides some ways
% to influence the font settings. The options below
% take font settings as values. Both font commands
% such as \cs{sffamily} and macros that take one argument
% like \cs{textsf} can be used.
%
% \begin{declcs}{hologoFontSetup} \M{key value list}
% \end{declcs}
% Macro \cs{hologoFontSetup} sets the fonts for all logos.
% Supported keys:
% \begin{description}
% \def\entry#1{\item[\xoption{#1}:]}
% \entry{general}
%   This font is used for all logos. The default is empty.
%   That means no special font is used.
% \entry{bibsf}
%   This font is used for
%   {\hologoLogoSetup{BibTeX}{variant=sf}\hologo{BibTeX}}
%   with variant \xoption{sf}.
% \entry{rm}
%   This font is a serif font. It is used for \hologo{ExTeX}.
% \entry{sc}
%   This font specifies a small caps font. It is used for
%   {\hologoLogoSetup{BibTeX}{variant=sc}\hologo{BibTeX}}
%   with variant \xoption{sc}.
% \entry{sf}
%   This font specifies a sans serif font. The default
%   is \cs{sffamily}, then \cs{sf} is tried. Otherwise
%   a warning is given. It is used by \hologo{KOMAScript}.
% \entry{sy}
%   This is the font for math symbols (e.g. cmsy).
%   It is used by \hologo{AmS}, \hologo{NTS}, \hologo{ExTeX}.
% \entry{logo}
%   \hologo{METAFONT} and \hologo{METAPOST} are using that font.
%   In \hologo{LaTeX} \cs{logofamily} is used and
%   the definitions of package \xpackage{mflogo} are used
%   if the package is not loaded.
%   Otherwise the \cs{tenlogo} is used and defined
%   if it does not already exists.
% \end{description}
%
% \begin{declcs}{hologoLogoFontSetup} \M{logo} \M{key value list}
% \end{declcs}
% Fonts can also be set for a logo or logo component separately,
% see the following list.
% The keys are the same as for \cs{hologoFontSetup}.
%
% \begin{longtable}{>{\ttfamily}l>{\sffamily}ll}
%   \meta{logo} & keys & result\\
%   \hline
%   \endhead
%   BibTeX & bibsf & {\hologoLogoSetup{BibTeX}{variant=sf}\hologo{BibTeX}}\\[.5ex]
%   BibTeX & sc & {\hologoLogoSetup{BibTeX}{variant=sc}\hologo{BibTeX}}\\[.5ex]
%   ExTeX & rm & \hologo{ExTeX}\\
%   SliTeX & rm & \hologo{SliTeX}\\[.5ex]
%   AmS & sy & \hologo{AmS}\\
%   ExTeX & sy & \hologo{ExTeX}\\
%   NTS & sy & \hologo{NTS}\\[.5ex]
%   KOMAScript & sf & \hologo{KOMAScript}\\[.5ex]
%   METAFONT & logo & \hologo{METAFONT}\\
%   METAPOST & logo & \hologo{METAPOST}\\[.5ex]
%   SliTeX & sc \hologo{SliTeX}
% \end{longtable}
%
% \subsubsection{Font order}
%
% For all logos the font \xoption{general} is applied first.
% Example:
%\begin{quote}
%|\hologoFontSetup{general=\color{red}}|
%\end{quote}
% will print red logos.
% Then if the font uses a special font \xoption{sf}, for example,
% the font is applied that is setup by \cs{hologoLogoFontSetup}.
% If this font is not setup, then the common font setup
% by \cs{hologoFontSetup} is used. Otherwise a warning is given,
% that there is no font configured.
%
% \subsection{Additional user macros}
%
% Usually a variant of a logo is configured by using
% \cs{hologoLogoSetup}, because it is bad style to mix
% different variants of the same logo in the same text.
% There the following macros are a convenience for testing.
%
% \begin{declcs}{hologoVariant} \M{name} \M{variant}\\
%   \cs{HologoVariant} \M{name} \M{variant}
% \end{declcs}
% Logo \meta{name} is set using \meta{variant} that specifies
% explicitely which variant of the macro is used. If the argument
% is empty, then the default form of the logo is used
% (configurable by \cs{hologoLogoSetup}).
%
% \cs{HologoVariant} is used if the logo is set in a context
% that needs an uppercase first letter (beginning of a sentence, \dots).
%
% \begin{declcs}{hologoList}\\
%   \cs{hologoEntry} \M{logo} \M{variant} \M{since}
% \end{declcs}
% Macro \cs{hologoList} contains all logos that are provided
% by the package including variants. The list consists of calls
% of \cs{hologoEntry} with three arguments starting with the
% logo name \meta{logo} and its variant \meta{variant}. An empty
% variant means the current default. Argument \meta{since} specifies
% with version of the package \xpackage{hologo} is needed to get
% the logo. If the logo is fixed, then the date gets updated.
% Therefore the date \meta{since} is not exactly the date of
% the first introduction, but rather the date of the latest fix.
%
% Before \cs{hologoList} can be used, macro \cs{hologoEntry} needs
% a definition. The example file in section \ref{sec:example}
% shows applications of \cs{hologoList}.
%
% \subsection{Supported contexts}
%
% Macros \cs{hologo} and friends support special contexts:
% \begin{itemize}
% \item \hologo{LaTeX}'s protection mechanism.
% \item Bookmarks of package \xpackage{hyperref}.
% \item Package \xpackage{tex4ht}.
% \item The macros can be used inside \cs{csname} constructs,
%   if \cs{ifincsname} is available (\hologo{pdfTeX}, \hologo{XeTeX},
%   \hologo{LuaTeX}).
% \end{itemize}
%
% \subsection{Example}
% \label{sec:example}
%
% The following example prints the logos in different fonts.
%    \begin{macrocode}
%<*example>
%<<verbatim
\NeedsTeXFormat{LaTeX2e}
\documentclass[a4paper]{article}
\usepackage[
  hmargin=20mm,
  vmargin=20mm,
]{geometry}
\pagestyle{empty}
\usepackage{hologo}[2016/05/12]
\usepackage{longtable}
\usepackage{array}
\setlength{\extrarowheight}{2pt}
\usepackage[T1]{fontenc}
\usepackage{lmodern}
\usepackage{pdflscape}
\usepackage[
  pdfencoding=auto,
]{hyperref}
\hypersetup{
  pdfauthor={Heiko Oberdiek},
  pdftitle={Example for package `hologo'},
  pdfsubject={Logos with fonts lmr, lmss, qtm, qpl, qhv},
}
\usepackage{bookmark}

% Print the logo list on the console

\begingroup
  \typeout{}%
  \typeout{*** Begin of logo list ***}%
  \newcommand*{\hologoEntry}[3]{%
    \typeout{#1 \ifx\\#2\\\else(#2) \fi[#3]}%
  }%
  \hologoList
  \typeout{*** End of logo list ***}%
  \typeout{}%
\endgroup

\begin{document}
\begin{landscape}

  \section{Example file for package `hologo'}

  % Table for font names

  \begin{longtable}{>{\bfseries}ll}
    \textbf{font} & \textbf{Font name}\\
    \hline
    lmr & Latin Modern Roman\\
    lmss & Latin Modern Sans\\
    qtm & \TeX\ Gyre Termes\\
    qhv & \TeX\ Gyre Heros\\
    qpl & \TeX\ Gyre Pagella\\
  \end{longtable}

  % Logo list with logos in different fonts

  \begingroup
    \newcommand*{\SetVariant}[2]{%
      \ifx\\#2\\%
      \else
        \hologoLogoSetup{#1}{variant=#2}%
      \fi
    }%
    \newcommand*{\hologoEntry}[3]{%
      \SetVariant{#1}{#2}%
      \raisebox{1em}[0pt][0pt]{\hypertarget{#1@#2}{}}%
      \bookmark[%
        dest={#1@#2},%
      ]{%
        #1\ifx\\#2\\\else\space(#2)\fi: \Hologo{#1}, \hologo{#1} %
        [Unicode]%
      }%
      \hypersetup{unicode=false}%
      \bookmark[%
        dest={#1@#2},%
      ]{%
        #1\ifx\\#2\\\else\space(#2)\fi: \Hologo{#1}, \hologo{#1} %
        [PDFDocEncoding]%
      }%
      \texttt{#1}%
      &%
      \texttt{#2}%
      &%
      \Hologo{#1}%
      &%
      \SetVariant{#1}{#2}%
      \hologo{#1}%
      &%
      \SetVariant{#1}{#2}%
      \fontfamily{qtm}\selectfont
      \hologo{#1}%
      &%
      \SetVariant{#1}{#2}%
      \fontfamily{qpl}\selectfont
      \hologo{#1}%
      &%
      \SetVariant{#1}{#2}%
      \textsf{\hologo{#1}}%
      &%
      \SetVariant{#1}{#2}%
      \fontfamily{qhv}\selectfont
      \hologo{#1}%
      \tabularnewline
    }%
    \begin{longtable}{llllllll}%
      \textbf{\textit{logo}} & \textbf{\textit{variant}} &
      \texttt{\string\Hologo} &
      \textbf{lmr} & \textbf{qtm} & \textbf{qpl} &
      \textbf{lmss} & \textbf{qhv}
      \tabularnewline
      \hline
      \endhead
      \hologoList
    \end{longtable}%
  \endgroup

\end{landscape}
\end{document}
%verbatim
%</example>
%    \end{macrocode}
%
% \StopEventually{
% }
%
% \section{Implementation}
%    \begin{macrocode}
%<*package>
%    \end{macrocode}
%    Reload check, especially if the package is not used with \LaTeX.
%    \begin{macrocode}
\begingroup\catcode61\catcode48\catcode32=10\relax%
  \catcode13=5 % ^^M
  \endlinechar=13 %
  \catcode35=6 % #
  \catcode39=12 % '
  \catcode44=12 % ,
  \catcode45=12 % -
  \catcode46=12 % .
  \catcode58=12 % :
  \catcode64=11 % @
  \catcode123=1 % {
  \catcode125=2 % }
  \expandafter\let\expandafter\x\csname ver@hologo.sty\endcsname
  \ifx\x\relax % plain-TeX, first loading
  \else
    \def\empty{}%
    \ifx\x\empty % LaTeX, first loading,
      % variable is initialized, but \ProvidesPackage not yet seen
    \else
      \expandafter\ifx\csname PackageInfo\endcsname\relax
        \def\x#1#2{%
          \immediate\write-1{Package #1 Info: #2.}%
        }%
      \else
        \def\x#1#2{\PackageInfo{#1}{#2, stopped}}%
      \fi
      \x{hologo}{The package is already loaded}%
      \aftergroup\endinput
    \fi
  \fi
\endgroup%
%    \end{macrocode}
%    Package identification:
%    \begin{macrocode}
\begingroup\catcode61\catcode48\catcode32=10\relax%
  \catcode13=5 % ^^M
  \endlinechar=13 %
  \catcode35=6 % #
  \catcode39=12 % '
  \catcode40=12 % (
  \catcode41=12 % )
  \catcode44=12 % ,
  \catcode45=12 % -
  \catcode46=12 % .
  \catcode47=12 % /
  \catcode58=12 % :
  \catcode64=11 % @
  \catcode91=12 % [
  \catcode93=12 % ]
  \catcode123=1 % {
  \catcode125=2 % }
  \expandafter\ifx\csname ProvidesPackage\endcsname\relax
    \def\x#1#2#3[#4]{\endgroup
      \immediate\write-1{Package: #3 #4}%
      \xdef#1{#4}%
    }%
  \else
    \def\x#1#2[#3]{\endgroup
      #2[{#3}]%
      \ifx#1\@undefined
        \xdef#1{#3}%
      \fi
      \ifx#1\relax
        \xdef#1{#3}%
      \fi
    }%
  \fi
\expandafter\x\csname ver@hologo.sty\endcsname
\ProvidesPackage{hologo}%
  [2016/05/12 v1.11 A logo collection with bookmark support (HO)]%
%    \end{macrocode}
%
%    \begin{macrocode}
\begingroup\catcode61\catcode48\catcode32=10\relax%
  \catcode13=5 % ^^M
  \endlinechar=13 %
  \catcode123=1 % {
  \catcode125=2 % }
  \catcode64=11 % @
  \def\x{\endgroup
    \expandafter\edef\csname HOLOGO@AtEnd\endcsname{%
      \endlinechar=\the\endlinechar\relax
      \catcode13=\the\catcode13\relax
      \catcode32=\the\catcode32\relax
      \catcode35=\the\catcode35\relax
      \catcode61=\the\catcode61\relax
      \catcode64=\the\catcode64\relax
      \catcode123=\the\catcode123\relax
      \catcode125=\the\catcode125\relax
    }%
  }%
\x\catcode61\catcode48\catcode32=10\relax%
\catcode13=5 % ^^M
\endlinechar=13 %
\catcode35=6 % #
\catcode64=11 % @
\catcode123=1 % {
\catcode125=2 % }
\def\TMP@EnsureCode#1#2{%
  \edef\HOLOGO@AtEnd{%
    \HOLOGO@AtEnd
    \catcode#1=\the\catcode#1\relax
  }%
  \catcode#1=#2\relax
}
\TMP@EnsureCode{10}{12}% ^^J
\TMP@EnsureCode{33}{12}% !
\TMP@EnsureCode{34}{12}% "
\TMP@EnsureCode{36}{3}% $
\TMP@EnsureCode{38}{4}% &
\TMP@EnsureCode{39}{12}% '
\TMP@EnsureCode{40}{12}% (
\TMP@EnsureCode{41}{12}% )
\TMP@EnsureCode{42}{12}% *
\TMP@EnsureCode{43}{12}% +
\TMP@EnsureCode{44}{12}% ,
\TMP@EnsureCode{45}{12}% -
\TMP@EnsureCode{46}{12}% .
\TMP@EnsureCode{47}{12}% /
\TMP@EnsureCode{58}{12}% :
\TMP@EnsureCode{59}{12}% ;
\TMP@EnsureCode{60}{12}% <
\TMP@EnsureCode{62}{12}% >
\TMP@EnsureCode{63}{12}% ?
\TMP@EnsureCode{91}{12}% [
\TMP@EnsureCode{93}{12}% ]
\TMP@EnsureCode{94}{7}% ^ (superscript)
\TMP@EnsureCode{95}{8}% _ (subscript)
\TMP@EnsureCode{96}{12}% `
\TMP@EnsureCode{124}{12}% |
\edef\HOLOGO@AtEnd{%
  \HOLOGO@AtEnd
  \escapechar\the\escapechar\relax
  \noexpand\endinput
}
\escapechar=92 %
%    \end{macrocode}
%
% \subsection{Logo list}
%
%    \begin{macro}{\hologoList}
%    \begin{macrocode}
\def\hologoList{%
  \hologoEntry{(La)TeX}{}{2011/10/01}%
  \hologoEntry{AmSLaTeX}{}{2010/04/16}%
  \hologoEntry{AmSTeX}{}{2010/04/16}%
  \hologoEntry{biber}{}{2011/10/01}%
  \hologoEntry{BibTeX}{}{2011/10/01}%
  \hologoEntry{BibTeX}{sf}{2011/10/01}%
  \hologoEntry{BibTeX}{sc}{2011/10/01}%
  \hologoEntry{BibTeX8}{}{2011/11/22}%
  \hologoEntry{ConTeXt}{}{2011/03/25}%
  \hologoEntry{ConTeXt}{narrow}{2011/03/25}%
  \hologoEntry{ConTeXt}{simple}{2011/03/25}%
  \hologoEntry{emTeX}{}{2010/04/26}%
  \hologoEntry{eTeX}{}{2010/04/08}%
  \hologoEntry{ExTeX}{}{2011/10/01}%
  \hologoEntry{HanTheThanh}{}{2011/11/29}%
  \hologoEntry{iniTeX}{}{2011/10/01}%
  \hologoEntry{KOMAScript}{}{2011/10/01}%
  \hologoEntry{La}{}{2010/05/08}%
  \hologoEntry{LaTeX}{}{2010/04/08}%
  \hologoEntry{LaTeX2e}{}{2010/04/08}%
  \hologoEntry{LaTeX3}{}{2010/04/24}%
  \hologoEntry{LaTeXe}{}{2010/04/08}%
  \hologoEntry{LaTeXML}{}{2011/11/22}%
  \hologoEntry{LaTeXTeX}{}{2011/10/01}%
  \hologoEntry{LuaLaTeX}{}{2010/04/08}%
  \hologoEntry{LuaTeX}{}{2010/04/08}%
  \hologoEntry{LyX}{}{2011/10/01}%
  \hologoEntry{METAFONT}{}{2011/10/01}%
  \hologoEntry{MetaFun}{}{2011/10/01}%
  \hologoEntry{METAPOST}{}{2011/10/01}%
  \hologoEntry{MetaPost}{}{2011/10/01}%
  \hologoEntry{MiKTeX}{}{2011/10/01}%
  \hologoEntry{NTS}{}{2011/10/01}%
  \hologoEntry{OzMF}{}{2011/10/01}%
  \hologoEntry{OzMP}{}{2011/10/01}%
  \hologoEntry{OzTeX}{}{2011/10/01}%
  \hologoEntry{OzTtH}{}{2011/10/01}%
  \hologoEntry{PCTeX}{}{2011/10/01}%
  \hologoEntry{pdfTeX}{}{2011/10/01}%
  \hologoEntry{pdfLaTeX}{}{2011/10/01}%
  \hologoEntry{PiC}{}{2011/10/01}%
  \hologoEntry{PiCTeX}{}{2011/10/01}%
  \hologoEntry{plainTeX}{}{2010/04/08}%
  \hologoEntry{plainTeX}{space}{2010/04/16}%
  \hologoEntry{plainTeX}{hyphen}{2010/04/16}%
  \hologoEntry{plainTeX}{runtogether}{2010/04/16}%
  \hologoEntry{SageTeX}{}{2011/11/22}%
  \hologoEntry{SLiTeX}{}{2011/10/01}%
  \hologoEntry{SLiTeX}{lift}{2011/10/01}%
  \hologoEntry{SLiTeX}{narrow}{2011/10/01}%
  \hologoEntry{SLiTeX}{simple}{2011/10/01}%
  \hologoEntry{SliTeX}{}{2011/10/01}%
  \hologoEntry{SliTeX}{narrow}{2011/10/01}%
  \hologoEntry{SliTeX}{simple}{2011/10/01}%
  \hologoEntry{SliTeX}{lift}{2011/10/01}%
  \hologoEntry{teTeX}{}{2011/10/01}%
  \hologoEntry{TeX}{}{2010/04/08}%
  \hologoEntry{TeX4ht}{}{2011/11/22}%
  \hologoEntry{TTH}{}{2011/11/22}%
  \hologoEntry{virTeX}{}{2011/10/01}%
  \hologoEntry{VTeX}{}{2010/04/24}%
  \hologoEntry{Xe}{}{2010/04/08}%
  \hologoEntry{XeLaTeX}{}{2010/04/08}%
  \hologoEntry{XeTeX}{}{2010/04/08}%
}
%    \end{macrocode}
%    \end{macro}
%
% \subsection{Load resources}
%
%    \begin{macrocode}
\begingroup\expandafter\expandafter\expandafter\endgroup
\expandafter\ifx\csname RequirePackage\endcsname\relax
  \def\TMP@RequirePackage#1[#2]{%
    \begingroup\expandafter\expandafter\expandafter\endgroup
    \expandafter\ifx\csname ver@#1.sty\endcsname\relax
      \input #1.sty\relax
    \fi
  }%
  \TMP@RequirePackage{ltxcmds}[2011/02/04]%
  \TMP@RequirePackage{infwarerr}[2010/04/08]%
  \TMP@RequirePackage{kvsetkeys}[2010/03/01]%
  \TMP@RequirePackage{kvdefinekeys}[2010/03/01]%
  \TMP@RequirePackage{pdftexcmds}[2010/04/01]%
  \TMP@RequirePackage{ifpdf}[2010/01/28]%
  \TMP@RequirePackage{ifluatex}[2010/03/01]%
  \ltx@IfUndefined{newif}{%
    \expandafter\let\csname newif\endcsname\ltx@newif
  }{}%
  \TMP@RequirePackage{ifxetex}[2009/01/23]%
  \TMP@RequirePackage{ifvtex}[2010/03/01]%
\else
  \RequirePackage{ltxcmds}[2011/02/04]%
  \RequirePackage{infwarerr}[2010/04/08]%
  \RequirePackage{kvsetkeys}[2010/03/01]%
  \RequirePackage{kvdefinekeys}[2010/03/01]%
  \RequirePackage{pdftexcmds}[2010/04/01]%
  \RequirePackage{ifpdf}[2010/01/28]%
  \RequirePackage{ifluatex}[2010/03/01]%
  \RequirePackage{ifxetex}[2009/01/23]%
  \RequirePackage{ifvtex}[2010/03/01]%
\fi
%    \end{macrocode}
%
%    \begin{macro}{\HOLOGO@IfDefined}
%    \begin{macrocode}
\def\HOLOGO@IfExists#1{%
  \ifx\@undefined#1%
    \expandafter\ltx@secondoftwo
  \else
    \ifx\relax#1%
      \expandafter\ltx@secondoftwo
    \else
      \expandafter\expandafter\expandafter\ltx@firstoftwo
    \fi
  \fi
}
%    \end{macrocode}
%    \end{macro}
%
% \subsection{Setup macros}
%
%    \begin{macro}{\hologoSetup}
%    \begin{macrocode}
\def\hologoSetup{%
  \let\HOLOGO@name\relax
  \HOLOGO@Setup
}
%    \end{macrocode}
%    \end{macro}
%
%    \begin{macro}{\hologoLogoSetup}
%    \begin{macrocode}
\def\hologoLogoSetup#1{%
  \edef\HOLOGO@name{#1}%
  \ltx@IfUndefined{HoLogo@\HOLOGO@name}{%
    \@PackageError{hologo}{%
      Unknown logo `\HOLOGO@name'%
    }\@ehc
    \ltx@gobble
  }{%
    \HOLOGO@Setup
  }%
}
%    \end{macrocode}
%    \end{macro}
%
%    \begin{macro}{\HOLOGO@Setup}
%    \begin{macrocode}
\def\HOLOGO@Setup{%
  \kvsetkeys{HoLogo}%
}
%    \end{macrocode}
%    \end{macro}
%
% \subsection{Options}
%
%    \begin{macro}{\HOLOGO@DeclareBoolOption}
%    \begin{macrocode}
\def\HOLOGO@DeclareBoolOption#1{%
  \expandafter\chardef\csname HOLOGOOPT@#1\endcsname\ltx@zero
  \kv@define@key{HoLogo}{#1}[true]{%
    \def\HOLOGO@temp{##1}%
    \ifx\HOLOGO@temp\HOLOGO@true
      \ifx\HOLOGO@name\relax
        \expandafter\chardef\csname HOLOGOOPT@#1\endcsname=\ltx@one
      \else
        \expandafter\chardef\csname
        HoLogoOpt@#1@\HOLOGO@name\endcsname\ltx@one
      \fi
      \HOLOGO@SetBreakAll{#1}%
    \else
      \ifx\HOLOGO@temp\HOLOGO@false
        \ifx\HOLOGO@name\relax
          \expandafter\chardef\csname HOLOGOOPT@#1\endcsname=\ltx@zero
        \else
          \expandafter\chardef\csname
          HoLogoOpt@#1@\HOLOGO@name\endcsname=\ltx@zero
        \fi
        \HOLOGO@SetBreakAll{#1}%
      \else
        \@PackageError{hologo}{%
          Unknown value `##1' for boolean option `#1'.\MessageBreak
          Known values are `true' and `false'%
        }\@ehc
      \fi
    \fi
  }%
}
%    \end{macrocode}
%    \end{macro}
%
%    \begin{macro}{\HOLOGO@SetBreakAll}
%    \begin{macrocode}
\def\HOLOGO@SetBreakAll#1{%
  \def\HOLOGO@temp{#1}%
  \ifx\HOLOGO@temp\HOLOGO@break
    \ifx\HOLOGO@name\relax
      \chardef\HOLOGOOPT@hyphenbreak=\HOLOGOOPT@break
      \chardef\HOLOGOOPT@spacebreak=\HOLOGOOPT@break
      \chardef\HOLOGOOPT@discretionarybreak=\HOLOGOOPT@break
    \else
      \expandafter\chardef
         \csname HoLogoOpt@hyphenbreak@\HOLOGO@name\endcsname=%
         \csname HoLogoOpt@break@\HOLOGO@name\endcsname
      \expandafter\chardef
         \csname HoLogoOpt@spacebreak@\HOLOGO@name\endcsname=%
         \csname HoLogoOpt@break@\HOLOGO@name\endcsname
      \expandafter\chardef
         \csname HoLogoOpt@discretionarybreak@\HOLOGO@name
             \endcsname=%
         \csname HoLogoOpt@break@\HOLOGO@name\endcsname
    \fi
  \fi
}
%    \end{macrocode}
%    \end{macro}
%
%    \begin{macro}{\HOLOGO@true}
%    \begin{macrocode}
\def\HOLOGO@true{true}
%    \end{macrocode}
%    \end{macro}
%    \begin{macro}{\HOLOGO@false}
%    \begin{macrocode}
\def\HOLOGO@false{false}
%    \end{macrocode}
%    \end{macro}
%    \begin{macro}{\HOLOGO@break}
%    \begin{macrocode}
\def\HOLOGO@break{break}
%    \end{macrocode}
%    \end{macro}
%
%    \begin{macrocode}
\HOLOGO@DeclareBoolOption{break}
\HOLOGO@DeclareBoolOption{hyphenbreak}
\HOLOGO@DeclareBoolOption{spacebreak}
\HOLOGO@DeclareBoolOption{discretionarybreak}
%    \end{macrocode}
%
%    \begin{macrocode}
\kv@define@key{HoLogo}{variant}{%
  \ifx\HOLOGO@name\relax
    \@PackageError{hologo}{%
      Option `variant' is not available in \string\hologoSetup,%
      \MessageBreak
      Use \string\hologoLogoSetup\space instead%
    }\@ehc
  \else
    \edef\HOLOGO@temp{#1}%
    \ifx\HOLOGO@temp\ltx@empty
      \expandafter
      \let\csname HoLogoOpt@variant@\HOLOGO@name\endcsname\@undefined
    \else
      \ltx@IfUndefined{HoLogo@\HOLOGO@name @\HOLOGO@temp}{%
        \@PackageError{hologo}{%
          Unknown variant `\HOLOGO@temp' of logo `\HOLOGO@name'%
        }\@ehc
      }{%
        \expandafter
        \let\csname HoLogoOpt@variant@\HOLOGO@name\endcsname
            \HOLOGO@temp
      }%
    \fi
  \fi
}
%    \end{macrocode}
%
%    \begin{macro}{\HOLOGO@Variant}
%    \begin{macrocode}
\def\HOLOGO@Variant#1{%
  #1%
  \ltx@ifundefined{HoLogoOpt@variant@#1}{%
  }{%
    @\csname HoLogoOpt@variant@#1\endcsname
  }%
}
%    \end{macrocode}
%    \end{macro}
%
% \subsection{Break/no-break support}
%
%    \begin{macro}{\HOLOGO@space}
%    \begin{macrocode}
\def\HOLOGO@space{%
  \ltx@ifundefined{HoLogoOpt@spacebreak@\HOLOGO@name}{%
    \ltx@ifundefined{HoLogoOpt@break@\HOLOGO@name}{%
      \chardef\HOLOGO@temp=\HOLOGOOPT@spacebreak
    }{%
      \chardef\HOLOGO@temp=%
        \csname HoLogoOpt@break@\HOLOGO@name\endcsname
    }%
  }{%
    \chardef\HOLOGO@temp=%
      \csname HoLogoOpt@spacebreak@\HOLOGO@name\endcsname
  }%
  \ifcase\HOLOGO@temp
    \penalty10000 %
  \fi
  \ltx@space
}
%    \end{macrocode}
%    \end{macro}
%
%    \begin{macro}{\HOLOGO@hyphen}
%    \begin{macrocode}
\def\HOLOGO@hyphen{%
  \ltx@ifundefined{HoLogoOpt@hyphenbreak@\HOLOGO@name}{%
    \ltx@ifundefined{HoLogoOpt@break@\HOLOGO@name}{%
      \chardef\HOLOGO@temp=\HOLOGOOPT@hyphenbreak
    }{%
      \chardef\HOLOGO@temp=%
        \csname HoLogoOpt@break@\HOLOGO@name\endcsname
    }%
  }{%
    \chardef\HOLOGO@temp=%
      \csname HoLogoOpt@hyphenbreak@\HOLOGO@name\endcsname
  }%
  \ifcase\HOLOGO@temp
    \ltx@mbox{-}%
  \else
    -%
  \fi
}
%    \end{macrocode}
%    \end{macro}
%
%    \begin{macro}{\HOLOGO@discretionary}
%    \begin{macrocode}
\def\HOLOGO@discretionary{%
  \ltx@ifundefined{HoLogoOpt@discretionarybreak@\HOLOGO@name}{%
    \ltx@ifundefined{HoLogoOpt@break@\HOLOGO@name}{%
      \chardef\HOLOGO@temp=\HOLOGOOPT@discretionarybreak
    }{%
      \chardef\HOLOGO@temp=%
        \csname HoLogoOpt@break@\HOLOGO@name\endcsname
    }%
  }{%
    \chardef\HOLOGO@temp=%
      \csname HoLogoOpt@discretionarybreak@\HOLOGO@name\endcsname
  }%
  \ifcase\HOLOGO@temp
  \else
    \-%
  \fi
}
%    \end{macrocode}
%    \end{macro}
%
%    \begin{macro}{\HOLOGO@mbox}
%    \begin{macrocode}
\def\HOLOGO@mbox#1{%
  \ltx@ifundefined{HoLogoOpt@break@\HOLOGO@name}{%
    \chardef\HOLOGO@temp=\HOLOGOOPT@hyphenbreak
  }{%
    \chardef\HOLOGO@temp=%
      \csname HoLogoOpt@break@\HOLOGO@name\endcsname
  }%
  \ifcase\HOLOGO@temp
    \ltx@mbox{#1}%
  \else
    #1%
  \fi
}
%    \end{macrocode}
%    \end{macro}
%
% \subsection{Font support}
%
%    \begin{macro}{\HoLogoFont@font}
%    \begin{tabular}{@{}ll@{}}
%    |#1|:& logo name\\
%    |#2|:& font short name\\
%    |#3|:& text
%    \end{tabular}
%    \begin{macrocode}
\def\HoLogoFont@font#1#2#3{%
  \begingroup
    \ltx@IfUndefined{HoLogoFont@logo@#1.#2}{%
      \ltx@IfUndefined{HoLogoFont@font@#2}{%
        \@PackageWarning{hologo}{%
          Missing font `#2' for logo `#1'%
        }%
        #3%
      }{%
        \csname HoLogoFont@font@#2\endcsname{#3}%
      }%
    }{%
      \csname HoLogoFont@logo@#1.#2\endcsname{#3}%
    }%
  \endgroup
}
%    \end{macrocode}
%    \end{macro}
%
%    \begin{macro}{\HoLogoFont@Def}
%    \begin{macrocode}
\def\HoLogoFont@Def#1{%
  \expandafter\def\csname HoLogoFont@font@#1\endcsname
}
%    \end{macrocode}
%    \end{macro}
%    \begin{macro}{\HoLogoFont@LogoDef}
%    \begin{macrocode}
\def\HoLogoFont@LogoDef#1#2{%
  \expandafter\def\csname HoLogoFont@logo@#1.#2\endcsname
}
%    \end{macrocode}
%    \end{macro}
%
% \subsubsection{Font defaults}
%
%    \begin{macro}{\HoLogoFont@font@general}
%    \begin{macrocode}
\HoLogoFont@Def{general}{}%
%    \end{macrocode}
%    \end{macro}
%
%    \begin{macro}{\HoLogoFont@font@rm}
%    \begin{macrocode}
\ltx@IfUndefined{rmfamily}{%
  \ltx@IfUndefined{rm}{%
  }{%
    \HoLogoFont@Def{rm}{\rm}%
  }%
}{%
  \HoLogoFont@Def{rm}{\rmfamily}%
}
%    \end{macrocode}
%    \end{macro}
%
%    \begin{macro}{\HoLogoFont@font@sf}
%    \begin{macrocode}
\ltx@IfUndefined{sffamily}{%
  \ltx@IfUndefined{sf}{%
  }{%
    \HoLogoFont@Def{sf}{\sf}%
  }%
}{%
  \HoLogoFont@Def{sf}{\sffamily}%
}
%    \end{macrocode}
%    \end{macro}
%
%    \begin{macro}{\HoLogoFont@font@bibsf}
%    In case of \hologo{plainTeX} the original small caps
%    variant is used as default. In \hologo{LaTeX}
%    the definition of package \xpackage{dtklogos} \cite{dtklogos}
%    is used.
%\begin{quote}
%\begin{verbatim}
%\DeclareRobustCommand{\BibTeX}{%
%  B%
%  \kern-.05em%
%  \hbox{%
%    $\m@th$% %% force math size calculations
%    \csname S@\f@size\endcsname
%    \fontsize\sf@size\z@
%    \math@fontsfalse
%    \selectfont
%    I%
%    \kern-.025em%
%    B
%  }%
%  \kern-.08em%
%  \-%
%  \TeX
%}
%\end{verbatim}
%\end{quote}
%    \begin{macrocode}
\ltx@IfUndefined{selectfont}{%
  \ltx@IfUndefined{tensc}{%
    \font\tensc=cmcsc10\relax
  }{}%
  \HoLogoFont@Def{bibsf}{\tensc}%
}{%
  \HoLogoFont@Def{bibsf}{%
    $\mathsurround=0pt$%
    \csname S@\f@size\endcsname
    \fontsize\sf@size{0pt}%
    \math@fontsfalse
    \selectfont
  }%
}
%    \end{macrocode}
%    \end{macro}
%
%    \begin{macro}{\HoLogoFont@font@sc}
%    \begin{macrocode}
\ltx@IfUndefined{scshape}{%
  \ltx@IfUndefined{tensc}{%
    \font\tensc=cmcsc10\relax
  }{}%
  \HoLogoFont@Def{sc}{\tensc}%
}{%
  \HoLogoFont@Def{sc}{\scshape}%
}
%    \end{macrocode}
%    \end{macro}
%
%    \begin{macro}{\HoLogoFont@font@sy}
%    \begin{macrocode}
\ltx@IfUndefined{usefont}{%
  \ltx@IfUndefined{tensy}{%
  }{%
    \HoLogoFont@Def{sy}{\tensy}%
  }%
}{%
  \HoLogoFont@Def{sy}{%
    \usefont{OMS}{cmsy}{m}{n}%
  }%
}
%    \end{macrocode}
%    \end{macro}
%
%    \begin{macro}{\HoLogoFont@font@logo}
%    \begin{macrocode}
\begingroup
  \def\x{LaTeX2e}%
\expandafter\endgroup
\ifx\fmtname\x
  \ltx@IfUndefined{logofamily}{%
    \DeclareRobustCommand\logofamily{%
      \not@math@alphabet\logofamily\relax
      \fontencoding{U}%
      \fontfamily{logo}%
      \selectfont
    }%
  }{}%
  \ltx@IfUndefined{logofamily}{%
  }{%
    \HoLogoFont@Def{logo}{\logofamily}%
  }%
\else
  \ltx@IfUndefined{tenlogo}{%
    \font\tenlogo=logo10\relax
  }{}%
  \HoLogoFont@Def{logo}{\tenlogo}%
\fi
%    \end{macrocode}
%    \end{macro}
%
% \subsubsection{Font setup}
%
%    \begin{macro}{\hologoFontSetup}
%    \begin{macrocode}
\def\hologoFontSetup{%
  \let\HOLOGO@name\relax
  \HOLOGO@FontSetup
}
%    \end{macrocode}
%    \end{macro}
%
%    \begin{macro}{\hologoLogoFontSetup}
%    \begin{macrocode}
\def\hologoLogoFontSetup#1{%
  \edef\HOLOGO@name{#1}%
  \ltx@IfUndefined{HoLogo@\HOLOGO@name}{%
    \@PackageError{hologo}{%
      Unknown logo `\HOLOGO@name'%
    }\@ehc
    \ltx@gobble
  }{%
    \HOLOGO@FontSetup
  }%
}
%    \end{macrocode}
%    \end{macro}
%
%    \begin{macro}{\HOLOGO@FontSetup}
%    \begin{macrocode}
\def\HOLOGO@FontSetup{%
  \kvsetkeys{HoLogoFont}%
}
%    \end{macrocode}
%    \end{macro}
%
%    \begin{macrocode}
\def\HOLOGO@temp#1{%
  \kv@define@key{HoLogoFont}{#1}{%
    \ifx\HOLOGO@name\relax
      \HoLogoFont@Def{#1}{##1}%
    \else
      \HoLogoFont@LogoDef\HOLOGO@name{#1}{##1}%
    \fi
  }%
}
\HOLOGO@temp{general}
\HOLOGO@temp{sf}
%    \end{macrocode}
%
% \subsection{Generic logo commands}
%
%    \begin{macrocode}
\HOLOGO@IfExists\hologo{%
  \@PackageError{hologo}{%
    \string\hologo\ltx@space is already defined.\MessageBreak
    Package loading is aborted%
  }\@ehc
  \HOLOGO@AtEnd
}%
\HOLOGO@IfExists\hologoRobust{%
  \@PackageError{hologo}{%
    \string\hologoRobust\ltx@space is already defined.\MessageBreak
    Package loading is aborted%
  }\@ehc
  \HOLOGO@AtEnd
}%
%    \end{macrocode}
%
% \subsubsection{\cs{hologo} and friends}
%
%    \begin{macrocode}
\ifluatex
  \expandafter\ltx@firstofone
\else
  \expandafter\ltx@gobble
\fi
{%
  \ltx@IfUndefined{ifincsname}{%
    \ifnum\luatexversion<36 %
      \expandafter\ltx@gobble
    \else
      \expandafter\ltx@firstofone
    \fi
    {%
      \begingroup
        \ifcase0%
            \directlua{%
              if tex.enableprimitives then %
                tex.enableprimitives('HOLOGO@', {'ifincsname'})%
              else %
                tex.print('1')%
              end%
            }%
            \ifx\HOLOGO@ifincsname\@undefined 1\fi%
            \relax
          \expandafter\ltx@firstofone
        \else
          \endgroup
          \expandafter\ltx@gobble
        \fi
        {%
          \global\let\ifincsname\HOLOGO@ifincsname
        }%
      \HOLOGO@temp
    }%
  }{}%
}
%    \end{macrocode}
%    \begin{macrocode}
\ltx@IfUndefined{ifincsname}{%
  \catcode`$=14 %
}{%
  \catcode`$=9 %
}
%    \end{macrocode}
%
%    \begin{macro}{\hologo}
%    \begin{macrocode}
\def\hologo#1{%
$ \ifincsname
$   \ltx@ifundefined{HoLogoCs@\HOLOGO@Variant{#1}}{%
$     #1%
$   }{%
$     \csname HoLogoCs@\HOLOGO@Variant{#1}\endcsname\ltx@firstoftwo
$   }%
$ \else
    \HOLOGO@IfExists\texorpdfstring\texorpdfstring\ltx@firstoftwo
    {%
      \hologoRobust{#1}%
    }{%
      \ltx@ifundefined{HoLogoBkm@\HOLOGO@Variant{#1}}{%
        \ltx@ifundefined{HoLogo@#1}{?#1?}{#1}%
      }{%
        \csname HoLogoBkm@\HOLOGO@Variant{#1}\endcsname
        \ltx@firstoftwo
      }%
    }%
$ \fi
}
%    \end{macrocode}
%    \end{macro}
%    \begin{macro}{\Hologo}
%    \begin{macrocode}
\def\Hologo#1{%
$ \ifincsname
$   \ltx@ifundefined{HoLogoCs@\HOLOGO@Variant{#1}}{%
$     #1%
$   }{%
$     \csname HoLogoCs@\HOLOGO@Variant{#1}\endcsname\ltx@secondoftwo
$   }%
$ \else
    \HOLOGO@IfExists\texorpdfstring\texorpdfstring\ltx@firstoftwo
    {%
      \HologoRobust{#1}%
    }{%
      \ltx@ifundefined{HoLogoBkm@\HOLOGO@Variant{#1}}{%
        \ltx@ifundefined{HoLogo@#1}{?#1?}{#1}%
      }{%
        \csname HoLogoBkm@\HOLOGO@Variant{#1}\endcsname
        \ltx@secondoftwo
      }%
    }%
$ \fi
}
%    \end{macrocode}
%    \end{macro}
%
%    \begin{macro}{\hologoVariant}
%    \begin{macrocode}
\def\hologoVariant#1#2{%
  \ifx\relax#2\relax
    \hologo{#1}%
  \else
$   \ifincsname
$     \ltx@ifundefined{HoLogoCs@#1@#2}{%
$       #1%
$     }{%
$       \csname HoLogoCs@#1@#2\endcsname\ltx@firstoftwo
$     }%
$   \else
      \HOLOGO@IfExists\texorpdfstring\texorpdfstring\ltx@firstoftwo
      {%
        \hologoVariantRobust{#1}{#2}%
      }{%
        \ltx@ifundefined{HoLogoBkm@#1@#2}{%
          \ltx@ifundefined{HoLogo@#1}{?#1?}{#1}%
        }{%
          \csname HoLogoBkm@#1@#2\endcsname
          \ltx@firstoftwo
        }%
      }%
$   \fi
  \fi
}
%    \end{macrocode}
%    \end{macro}
%    \begin{macro}{\HologoVariant}
%    \begin{macrocode}
\def\HologoVariant#1#2{%
  \ifx\relax#2\relax
    \Hologo{#1}%
  \else
$   \ifincsname
$     \ltx@ifundefined{HoLogoCs@#1@#2}{%
$       #1%
$     }{%
$       \csname HoLogoCs@#1@#2\endcsname\ltx@secondoftwo
$     }%
$   \else
      \HOLOGO@IfExists\texorpdfstring\texorpdfstring\ltx@firstoftwo
      {%
        \HologoVariantRobust{#1}{#2}%
      }{%
        \ltx@ifundefined{HoLogoBkm@#1@#2}{%
          \ltx@ifundefined{HoLogo@#1}{?#1?}{#1}%
        }{%
          \csname HoLogoBkm@#1@#2\endcsname
          \ltx@secondoftwo
        }%
      }%
$   \fi
  \fi
}
%    \end{macrocode}
%    \end{macro}
%
%    \begin{macrocode}
\catcode`\$=3 %
%    \end{macrocode}
%
% \subsubsection{\cs{hologoRobust} and friends}
%
%    \begin{macro}{\hologoRobust}
%    \begin{macrocode}
\ltx@IfUndefined{protected}{%
  \ltx@IfUndefined{DeclareRobustCommand}{%
    \def\hologoRobust#1%
  }{%
    \DeclareRobustCommand*\hologoRobust[1]%
  }%
}{%
  \protected\def\hologoRobust#1%
}%
{%
  \edef\HOLOGO@name{#1}%
  \ltx@IfUndefined{HoLogo@\HOLOGO@Variant\HOLOGO@name}{%
    \@PackageError{hologo}{%
      Unknown logo `\HOLOGO@name'%
    }\@ehc
    ?\HOLOGO@name?%
  }{%
    \ltx@IfUndefined{ver@tex4ht.sty}{%
      \HoLogoFont@font\HOLOGO@name{general}{%
        \csname HoLogo@\HOLOGO@Variant\HOLOGO@name\endcsname
        \ltx@firstoftwo
      }%
    }{%
      \ltx@IfUndefined{HoLogoHtml@\HOLOGO@Variant\HOLOGO@name}{%
        \HOLOGO@name
      }{%
        \csname HoLogoHtml@\HOLOGO@Variant\HOLOGO@name\endcsname
        \ltx@firstoftwo
      }%
    }%
  }%
}
%    \end{macrocode}
%    \end{macro}
%    \begin{macro}{\HologoRobust}
%    \begin{macrocode}
\ltx@IfUndefined{protected}{%
  \ltx@IfUndefined{DeclareRobustCommand}{%
    \def\HologoRobust#1%
  }{%
    \DeclareRobustCommand*\HologoRobust[1]%
  }%
}{%
  \protected\def\HologoRobust#1%
}%
{%
  \edef\HOLOGO@name{#1}%
  \ltx@IfUndefined{HoLogo@\HOLOGO@Variant\HOLOGO@name}{%
    \@PackageError{hologo}{%
      Unknown logo `\HOLOGO@name'%
    }\@ehc
    ?\HOLOGO@name?%
  }{%
    \ltx@IfUndefined{ver@tex4ht.sty}{%
      \HoLogoFont@font\HOLOGO@name{general}{%
        \csname HoLogo@\HOLOGO@Variant\HOLOGO@name\endcsname
        \ltx@secondoftwo
      }%
    }{%
      \ltx@IfUndefined{HoLogoHtml@\HOLOGO@Variant\HOLOGO@name}{%
        \expandafter\HOLOGO@Uppercase\HOLOGO@name
      }{%
        \csname HoLogoHtml@\HOLOGO@Variant\HOLOGO@name\endcsname
        \ltx@secondoftwo
      }%
    }%
  }%
}
%    \end{macrocode}
%    \end{macro}
%    \begin{macro}{\hologoVariantRobust}
%    \begin{macrocode}
\ltx@IfUndefined{protected}{%
  \ltx@IfUndefined{DeclareRobustCommand}{%
    \def\hologoVariantRobust#1#2%
  }{%
    \DeclareRobustCommand*\hologoVariantRobust[2]%
  }%
}{%
  \protected\def\hologoVariantRobust#1#2%
}%
{%
  \begingroup
    \hologoLogoSetup{#1}{variant={#2}}%
    \hologoRobust{#1}%
  \endgroup
}
%    \end{macrocode}
%    \end{macro}
%    \begin{macro}{\HologoVariantRobust}
%    \begin{macrocode}
\ltx@IfUndefined{protected}{%
  \ltx@IfUndefined{DeclareRobustCommand}{%
    \def\HologoVariantRobust#1#2%
  }{%
    \DeclareRobustCommand*\HologoVariantRobust[2]%
  }%
}{%
  \protected\def\HologoVariantRobust#1#2%
}%
{%
  \begingroup
    \hologoLogoSetup{#1}{variant={#2}}%
    \HologoRobust{#1}%
  \endgroup
}
%    \end{macrocode}
%    \end{macro}
%
%    \begin{macro}{\hologorobust}
%    Macro \cs{hologorobust} is only defined for compatibility.
%    Its use is deprecated.
%    \begin{macrocode}
\def\hologorobust{\hologoRobust}
%    \end{macrocode}
%    \end{macro}
%
% \subsection{Helpers}
%
%    \begin{macro}{\HOLOGO@Uppercase}
%    Macro \cs{HOLOGO@Uppercase} is restricted to \cs{uppercase},
%    because \hologo{plainTeX} or \hologo{iniTeX} do not provide
%    \cs{MakeUppercase}.
%    \begin{macrocode}
\def\HOLOGO@Uppercase#1{\uppercase{#1}}
%    \end{macrocode}
%    \end{macro}
%
%    \begin{macro}{\HOLOGO@PdfdocUnicode}
%    \begin{macrocode}
\def\HOLOGO@PdfdocUnicode{%
  \ifx\ifHy@unicode\iftrue
    \expandafter\ltx@secondoftwo
  \else
    \expandafter\ltx@firstoftwo
  \fi
}
%    \end{macrocode}
%    \end{macro}
%
%    \begin{macro}{\HOLOGO@Math}
%    \begin{macrocode}
\def\HOLOGO@MathSetup{%
  \mathsurround0pt\relax
  \HOLOGO@IfExists\f@series{%
    \if b\expandafter\ltx@car\f@series x\@nil
      \csname boldmath\endcsname
   \fi
  }{}%
}
%    \end{macrocode}
%    \end{macro}
%
%    \begin{macro}{\HOLOGO@TempDimen}
%    \begin{macrocode}
\dimendef\HOLOGO@TempDimen=\ltx@zero
%    \end{macrocode}
%    \end{macro}
%    \begin{macro}{\HOLOGO@NegativeKerning}
%    \begin{macrocode}
\def\HOLOGO@NegativeKerning#1{%
  \begingroup
    \HOLOGO@TempDimen=0pt\relax
    \comma@parse@normalized{#1}{%
      \ifdim\HOLOGO@TempDimen=0pt %
        \expandafter\HOLOGO@@NegativeKerning\comma@entry
      \fi
      \ltx@gobble
    }%
    \ifdim\HOLOGO@TempDimen<0pt %
      \kern\HOLOGO@TempDimen
    \fi
  \endgroup
}
%    \end{macrocode}
%    \end{macro}
%    \begin{macro}{\HOLOGO@@NegativeKerning}
%    \begin{macrocode}
\def\HOLOGO@@NegativeKerning#1#2{%
  \setbox\ltx@zero\hbox{#1#2}%
  \HOLOGO@TempDimen=\wd\ltx@zero
  \setbox\ltx@zero\hbox{#1\kern0pt#2}%
  \advance\HOLOGO@TempDimen by -\wd\ltx@zero
}
%    \end{macrocode}
%    \end{macro}
%
%    \begin{macro}{\HOLOGO@SpaceFactor}
%    \begin{macrocode}
\def\HOLOGO@SpaceFactor{%
  \spacefactor1000 %
}
%    \end{macrocode}
%    \end{macro}
%
%    \begin{macro}{\HOLOGO@Span}
%    \begin{macrocode}
\def\HOLOGO@Span#1#2{%
  \HCode{<span class="HoLogo-#1">}%
  #2%
  \HCode{</span>}%
}
%    \end{macrocode}
%    \end{macro}
%
% \subsubsection{Text subscript}
%
%    \begin{macro}{\HOLOGO@SubScript}%
%    \begin{macrocode}
\def\HOLOGO@SubScript#1{%
  \ltx@IfUndefined{textsubscript}{%
    \ltx@IfUndefined{text}{%
      \ltx@mbox{%
        \mathsurround=0pt\relax
        $%
          _{%
            \ltx@IfUndefined{sf@size}{%
              \mathrm{#1}%
            }{%
              \mbox{%
                \fontsize\sf@size{0pt}\selectfont
                #1%
              }%
            }%
          }%
        $%
      }%
    }{%
      \ltx@mbox{%
        \mathsurround=0pt\relax
        $_{\text{#1}}$%
      }%
    }%
  }{%
    \textsubscript{#1}%
  }%
}
%    \end{macrocode}
%    \end{macro}
%
% \subsection{\hologo{TeX} and friends}
%
% \subsubsection{\hologo{TeX}}
%
%    \begin{macro}{\HoLogo@TeX}
%    Source: \hologo{LaTeX} kernel.
%    \begin{macrocode}
\def\HoLogo@TeX#1{%
  T\kern-.1667em\lower.5ex\hbox{E}\kern-.125emX\HOLOGO@SpaceFactor
}
%    \end{macrocode}
%    \end{macro}
%    \begin{macro}{\HoLogoHtml@TeX}
%    \begin{macrocode}
\def\HoLogoHtml@TeX#1{%
  \HoLogoCss@TeX
  \HOLOGO@Span{TeX}{%
    T%
    \HOLOGO@Span{e}{%
      E%
    }%
    X%
  }%
}
%    \end{macrocode}
%    \end{macro}
%    \begin{macro}{\HoLogoCss@TeX}
%    \begin{macrocode}
\def\HoLogoCss@TeX{%
  \Css{%
    span.HoLogo-TeX span.HoLogo-e{%
      position:relative;%
      top:.5ex;%
      margin-left:-.1667em;%
      margin-right:-.125em;%
    }%
  }%
  \Css{%
    a span.HoLogo-TeX span.HoLogo-e{%
      text-decoration:none;%
    }%
  }%
  \global\let\HoLogoCss@TeX\relax
}
%    \end{macrocode}
%    \end{macro}
%
% \subsubsection{\hologo{plainTeX}}
%
%    \begin{macro}{\HoLogo@plainTeX@space}
%    Source: ``The \hologo{TeX}book''
%    \begin{macrocode}
\def\HoLogo@plainTeX@space#1{%
  \HOLOGO@mbox{#1{p}{P}lain}\HOLOGO@space\hologo{TeX}%
}
%    \end{macrocode}
%    \end{macro}
%    \begin{macro}{\HoLogoCs@plainTeX@space}
%    \begin{macrocode}
\def\HoLogoCs@plainTeX@space#1{#1{p}{P}lain TeX}%
%    \end{macrocode}
%    \end{macro}
%    \begin{macro}{\HoLogoBkm@plainTeX@space}
%    \begin{macrocode}
\def\HoLogoBkm@plainTeX@space#1{%
  #1{p}{P}lain \hologo{TeX}%
}
%    \end{macrocode}
%    \end{macro}
%    \begin{macro}{\HoLogoHtml@plainTeX@space}
%    \begin{macrocode}
\def\HoLogoHtml@plainTeX@space#1{%
  #1{p}{P}lain \hologo{TeX}%
}
%    \end{macrocode}
%    \end{macro}
%
%    \begin{macro}{\HoLogo@plainTeX@hyphen}
%    \begin{macrocode}
\def\HoLogo@plainTeX@hyphen#1{%
  \HOLOGO@mbox{#1{p}{P}lain}\HOLOGO@hyphen\hologo{TeX}%
}
%    \end{macrocode}
%    \end{macro}
%    \begin{macro}{\HoLogoCs@plainTeX@hyphen}
%    \begin{macrocode}
\def\HoLogoCs@plainTeX@hyphen#1{#1{p}{P}lain-TeX}
%    \end{macrocode}
%    \end{macro}
%    \begin{macro}{\HoLogoBkm@plainTeX@hyphen}
%    \begin{macrocode}
\def\HoLogoBkm@plainTeX@hyphen#1{%
  #1{p}{P}lain-\hologo{TeX}%
}
%    \end{macrocode}
%    \end{macro}
%    \begin{macro}{\HoLogoHtml@plainTeX@hyphen}
%    \begin{macrocode}
\def\HoLogoHtml@plainTeX@hyphen#1{%
  #1{p}{P}lain-\hologo{TeX}%
}
%    \end{macrocode}
%    \end{macro}
%
%    \begin{macro}{\HoLogo@plainTeX@runtogether}
%    \begin{macrocode}
\def\HoLogo@plainTeX@runtogether#1{%
  \HOLOGO@mbox{#1{p}{P}lain\hologo{TeX}}%
}
%    \end{macrocode}
%    \end{macro}
%    \begin{macro}{\HoLogoCs@plainTeX@runtogether}
%    \begin{macrocode}
\def\HoLogoCs@plainTeX@runtogether#1{#1{p}{P}lainTeX}
%    \end{macrocode}
%    \end{macro}
%    \begin{macro}{\HoLogoBkm@plainTeX@runtogether}
%    \begin{macrocode}
\def\HoLogoBkm@plainTeX@runtogether#1{%
  #1{p}{P}lain\hologo{TeX}%
}
%    \end{macrocode}
%    \end{macro}
%    \begin{macro}{\HoLogoHtml@plainTeX@runtogether}
%    \begin{macrocode}
\def\HoLogoHtml@plainTeX@runtogether#1{%
  #1{p}{P}lain\hologo{TeX}%
}
%    \end{macrocode}
%    \end{macro}
%
%    \begin{macro}{\HoLogo@plainTeX}
%    \begin{macrocode}
\def\HoLogo@plainTeX{\HoLogo@plainTeX@space}
%    \end{macrocode}
%    \end{macro}
%    \begin{macro}{\HoLogoCs@plainTeX}
%    \begin{macrocode}
\def\HoLogoCs@plainTeX{\HoLogoCs@plainTeX@space}
%    \end{macrocode}
%    \end{macro}
%    \begin{macro}{\HoLogoBkm@plainTeX}
%    \begin{macrocode}
\def\HoLogoBkm@plainTeX{\HoLogoBkm@plainTeX@space}
%    \end{macrocode}
%    \end{macro}
%    \begin{macro}{\HoLogoHtml@plainTeX}
%    \begin{macrocode}
\def\HoLogoHtml@plainTeX{\HoLogoHtml@plainTeX@space}
%    \end{macrocode}
%    \end{macro}
%
% \subsubsection{\hologo{LaTeX}}
%
%    Source: \hologo{LaTeX} kernel.
%\begin{quote}
%\begin{verbatim}
%\DeclareRobustCommand{\LaTeX}{%
%  L%
%  \kern-.36em%
%  {%
%    \sbox\z@ T%
%    \vbox to\ht\z@{%
%      \hbox{%
%        \check@mathfonts
%        \fontsize\sf@size\z@
%        \math@fontsfalse
%        \selectfont
%        A%
%      }%
%      \vss
%    }%
%  }%
%  \kern-.15em%
%  \TeX
%}
%\end{verbatim}
%\end{quote}
%
%    \begin{macro}{\HoLogo@La}
%    \begin{macrocode}
\def\HoLogo@La#1{%
  L%
  \kern-.36em%
  \begingroup
    \setbox\ltx@zero\hbox{T}%
    \vbox to\ht\ltx@zero{%
      \hbox{%
        \ltx@ifundefined{check@mathfonts}{%
          \csname sevenrm\endcsname
        }{%
          \check@mathfonts
          \fontsize\sf@size{0pt}%
          \math@fontsfalse\selectfont
        }%
        A%
      }%
      \vss
    }%
  \endgroup
}
%    \end{macrocode}
%    \end{macro}
%
%    \begin{macro}{\HoLogo@LaTeX}
%    Source: \hologo{LaTeX} kernel.
%    \begin{macrocode}
\def\HoLogo@LaTeX#1{%
  \hologo{La}%
  \kern-.15em%
  \hologo{TeX}%
}
%    \end{macrocode}
%    \end{macro}
%    \begin{macro}{\HoLogoHtml@LaTeX}
%    \begin{macrocode}
\def\HoLogoHtml@LaTeX#1{%
  \HoLogoCss@LaTeX
  \HOLOGO@Span{LaTeX}{%
    L%
    \HOLOGO@Span{a}{%
      A%
    }%
    \hologo{TeX}%
  }%
}
%    \end{macrocode}
%    \end{macro}
%    \begin{macro}{\HoLogoCss@LaTeX}
%    \begin{macrocode}
\def\HoLogoCss@LaTeX{%
  \Css{%
    span.HoLogo-LaTeX span.HoLogo-a{%
      position:relative;%
      top:-.5ex;%
      margin-left:-.36em;%
      margin-right:-.15em;%
      font-size:85\%;%
    }%
  }%
  \global\let\HoLogoCss@LaTeX\relax
}
%    \end{macrocode}
%    \end{macro}
%
% \subsubsection{\hologo{(La)TeX}}
%
%    \begin{macro}{\HoLogo@LaTeXTeX}
%    The kerning around the parentheses is taken
%    from package \xpackage{dtklogos} \cite{dtklogos}.
%\begin{quote}
%\begin{verbatim}
%\DeclareRobustCommand{\LaTeXTeX}{%
%  (%
%  \kern-.15em%
%  L%
%  \kern-.36em%
%  {%
%    \sbox\z@ T%
%    \vbox to\ht0{%
%      \hbox{%
%        $\m@th$%
%        \csname S@\f@size\endcsname
%        \fontsize\sf@size\z@
%        \math@fontsfalse
%        \selectfont
%        A%
%      }%
%      \vss
%    }%
%  }%
%  \kern-.2em%
%  )%
%  \kern-.15em%
%  \TeX
%}
%\end{verbatim}
%\end{quote}
%    \begin{macrocode}
\def\HoLogo@LaTeXTeX#1{%
  (%
  \kern-.15em%
  \hologo{La}%
  \kern-.2em%
  )%
  \kern-.15em%
  \hologo{TeX}%
}
%    \end{macrocode}
%    \end{macro}
%    \begin{macro}{\HoLogoBkm@LaTeXTeX}
%    \begin{macrocode}
\def\HoLogoBkm@LaTeXTeX#1{(La)TeX}
%    \end{macrocode}
%    \end{macro}
%
%    \begin{macro}{\HoLogo@(La)TeX}
%    \begin{macrocode}
\expandafter
\let\csname HoLogo@(La)TeX\endcsname\HoLogo@LaTeXTeX
%    \end{macrocode}
%    \end{macro}
%    \begin{macro}{\HoLogoBkm@(La)TeX}
%    \begin{macrocode}
\expandafter
\let\csname HoLogoBkm@(La)TeX\endcsname\HoLogoBkm@LaTeXTeX
%    \end{macrocode}
%    \end{macro}
%    \begin{macro}{\HoLogoHtml@LaTeXTeX}
%    \begin{macrocode}
\def\HoLogoHtml@LaTeXTeX#1{%
  \HoLogoCss@LaTeXTeX
  \HOLOGO@Span{LaTeXTeX}{%
    (%
    \HOLOGO@Span{L}{L}%
    \HOLOGO@Span{a}{A}%
    \HOLOGO@Span{ParenRight}{)}%
    \hologo{TeX}%
  }%
}
%    \end{macrocode}
%    \end{macro}
%    \begin{macro}{\HoLogoHtml@(La)TeX}
%    Kerning after opening parentheses and before closing parentheses
%    is $-0.1$\,em. The original values $-0.15$\,em
%    looked too ugly for a serif font.
%    \begin{macrocode}
\expandafter
\let\csname HoLogoHtml@(La)TeX\endcsname\HoLogoHtml@LaTeXTeX
%    \end{macrocode}
%    \end{macro}
%    \begin{macro}{\HoLogoCss@LaTeXTeX}
%    \begin{macrocode}
\def\HoLogoCss@LaTeXTeX{%
  \Css{%
    span.HoLogo-LaTeXTeX span.HoLogo-L{%
      margin-left:-.1em;%
    }%
  }%
  \Css{%
    span.HoLogo-LaTeXTeX span.HoLogo-a{%
      position:relative;%
      top:-.5ex;%
      margin-left:-.36em;%
      margin-right:-.1em;%
      font-size:85\%;%
    }%
  }%
  \Css{%
    span.HoLogo-LaTeXTeX span.HoLogo-ParenRight{%
      margin-right:-.15em;%
    }%
  }%
  \global\let\HoLogoCss@LaTeXTeX\relax
}
%    \end{macrocode}
%    \end{macro}
%
% \subsubsection{\hologo{LaTeXe}}
%
%    \begin{macro}{\HoLogo@LaTeXe}
%    Source: \hologo{LaTeX} kernel
%    \begin{macrocode}
\def\HoLogo@LaTeXe#1{%
  \hologo{LaTeX}%
  \kern.15em%
  \hbox{%
    \HOLOGO@MathSetup
    2%
    $_{\textstyle\varepsilon}$%
  }%
}
%    \end{macrocode}
%    \end{macro}
%
%    \begin{macro}{\HoLogoCs@LaTeXe}
%    \begin{macrocode}
\ifnum64=`\^^^^0040\relax % test for big chars of LuaTeX/XeTeX
  \catcode`\$=9 %
  \catcode`\&=14 %
\else
  \catcode`\$=14 %
  \catcode`\&=9 %
\fi
\def\HoLogoCs@LaTeXe#1{%
  LaTeX2%
$ \string ^^^^0395%
& e%
}%
\catcode`\$=3 %
\catcode`\&=4 %
%    \end{macrocode}
%    \end{macro}
%
%    \begin{macro}{\HoLogoBkm@LaTeXe}
%    \begin{macrocode}
\def\HoLogoBkm@LaTeXe#1{%
  \hologo{LaTeX}%
  2%
  \HOLOGO@PdfdocUnicode{e}{\textepsilon}%
}
%    \end{macrocode}
%    \end{macro}
%
%    \begin{macro}{\HoLogoHtml@LaTeXe}
%    \begin{macrocode}
\def\HoLogoHtml@LaTeXe#1{%
  \HoLogoCss@LaTeXe
  \HOLOGO@Span{LaTeX2e}{%
    \hologo{LaTeX}%
    \HOLOGO@Span{2}{2}%
    \HOLOGO@Span{e}{%
      \HOLOGO@MathSetup
      \ensuremath{\textstyle\varepsilon}%
    }%
  }%
}
%    \end{macrocode}
%    \end{macro}
%    \begin{macro}{\HoLogoCss@LaTeXe}
%    \begin{macrocode}
\def\HoLogoCss@LaTeXe{%
  \Css{%
    span.HoLogo-LaTeX2e span.HoLogo-2{%
      padding-left:.15em;%
    }%
  }%
  \Css{%
    span.HoLogo-LaTeX2e span.HoLogo-e{%
      position:relative;%
      top:.35ex;%
      text-decoration:none;%
    }%
  }%
  \global\let\HoLogoCss@LaTeXe\relax
}
%    \end{macrocode}
%    \end{macro}
%
%    \begin{macro}{\HoLogo@LaTeX2e}
%    \begin{macrocode}
\expandafter
\let\csname HoLogo@LaTeX2e\endcsname\HoLogo@LaTeXe
%    \end{macrocode}
%    \end{macro}
%    \begin{macro}{\HoLogoCs@LaTeX2e}
%    \begin{macrocode}
\expandafter
\let\csname HoLogoCs@LaTeX2e\endcsname\HoLogoCs@LaTeXe
%    \end{macrocode}
%    \end{macro}
%    \begin{macro}{\HoLogoBkm@LaTeX2e}
%    \begin{macrocode}
\expandafter
\let\csname HoLogoBkm@LaTeX2e\endcsname\HoLogoBkm@LaTeXe
%    \end{macrocode}
%    \end{macro}
%    \begin{macro}{\HoLogoHtml@LaTeX2e}
%    \begin{macrocode}
\expandafter
\let\csname HoLogoHtml@LaTeX2e\endcsname\HoLogoHtml@LaTeXe
%    \end{macrocode}
%    \end{macro}
%
% \subsubsection{\hologo{LaTeX3}}
%
%    \begin{macro}{\HoLogo@LaTeX3}
%    Source: \hologo{LaTeX} kernel
%    \begin{macrocode}
\expandafter\def\csname HoLogo@LaTeX3\endcsname#1{%
  \hologo{LaTeX}%
  3%
}
%    \end{macrocode}
%    \end{macro}
%
%    \begin{macro}{\HoLogoBkm@LaTeX3}
%    \begin{macrocode}
\expandafter\def\csname HoLogoBkm@LaTeX3\endcsname#1{%
  \hologo{LaTeX}%
  3%
}
%    \end{macrocode}
%    \end{macro}
%    \begin{macro}{\HoLogoHtml@LaTeX3}
%    \begin{macrocode}
\expandafter
\let\csname HoLogoHtml@LaTeX3\expandafter\endcsname
\csname HoLogo@LaTeX3\endcsname
%    \end{macrocode}
%    \end{macro}
%
% \subsubsection{\hologo{LaTeXML}}
%
%    \begin{macro}{\HoLogo@LaTeXML}
%    \begin{macrocode}
\def\HoLogo@LaTeXML#1{%
  \HOLOGO@mbox{%
    \hologo{La}%
    \kern-.15em%
    T%
    \kern-.1667em%
    \lower.5ex\hbox{E}%
    \kern-.125em%
    \HoLogoFont@font{LaTeXML}{sc}{xml}%
  }%
}
%    \end{macrocode}
%    \end{macro}
%    \begin{macro}{\HoLogoHtml@pdfLaTeX}
%    \begin{macrocode}
\def\HoLogoHtml@LaTeXML#1{%
  \HOLOGO@Span{LaTeXML}{%
    \HoLogoCss@LaTeX
    \HoLogoCss@TeX
    \HOLOGO@Span{LaTeX}{%
      L%
      \HOLOGO@Span{a}{%
        A%
      }%
    }%
    \HOLOGO@Span{TeX}{%
      T%
      \HOLOGO@Span{e}{%
        E%
      }%
    }%
    \HCode{<span style="font-variant: small-caps;">}%
    xml%
    \HCode{</span>}%
  }%
}
%    \end{macrocode}
%    \end{macro}
%
% \subsubsection{\hologo{eTeX}}
%
%    \begin{macro}{\HoLogo@eTeX}
%    Source: package \xpackage{etex}
%    \begin{macrocode}
\def\HoLogo@eTeX#1{%
  \ltx@mbox{%
    \HOLOGO@MathSetup
    $\varepsilon$%
    -%
    \HOLOGO@NegativeKerning{-T,T-,To}%
    \hologo{TeX}%
  }%
}
%    \end{macrocode}
%    \end{macro}
%    \begin{macro}{\HoLogoCs@eTeX}
%    \begin{macrocode}
\ifnum64=`\^^^^0040\relax % test for big chars of LuaTeX/XeTeX
  \catcode`\$=9 %
  \catcode`\&=14 %
\else
  \catcode`\$=14 %
  \catcode`\&=9 %
\fi
\def\HoLogoCs@eTeX#1{%
$ #1{\string ^^^^0395}{\string ^^^^03b5}%
& #1{e}{E}%
  TeX%
}%
\catcode`\$=3 %
\catcode`\&=4 %
%    \end{macrocode}
%    \end{macro}
%    \begin{macro}{\HoLogoBkm@eTeX}
%    \begin{macrocode}
\def\HoLogoBkm@eTeX#1{%
  \HOLOGO@PdfdocUnicode{#1{e}{E}}{\textepsilon}%
  -%
  \hologo{TeX}%
}
%    \end{macrocode}
%    \end{macro}
%    \begin{macro}{\HoLogoHtml@eTeX}
%    \begin{macrocode}
\def\HoLogoHtml@eTeX#1{%
  \ltx@mbox{%
    \HOLOGO@MathSetup
    $\varepsilon$%
    -%
    \hologo{TeX}%
  }%
}
%    \end{macrocode}
%    \end{macro}
%
% \subsubsection{\hologo{iniTeX}}
%
%    \begin{macro}{\HoLogo@iniTeX}
%    \begin{macrocode}
\def\HoLogo@iniTeX#1{%
  \HOLOGO@mbox{%
    #1{i}{I}ni\hologo{TeX}%
  }%
}
%    \end{macrocode}
%    \end{macro}
%    \begin{macro}{\HoLogoCs@iniTeX}
%    \begin{macrocode}
\def\HoLogoCs@iniTeX#1{#1{i}{I}niTeX}
%    \end{macrocode}
%    \end{macro}
%    \begin{macro}{\HoLogoBkm@iniTeX}
%    \begin{macrocode}
\def\HoLogoBkm@iniTeX#1{%
  #1{i}{I}ni\hologo{TeX}%
}
%    \end{macrocode}
%    \end{macro}
%    \begin{macro}{\HoLogoHtml@iniTeX}
%    \begin{macrocode}
\let\HoLogoHtml@iniTeX\HoLogo@iniTeX
%    \end{macrocode}
%    \end{macro}
%
% \subsubsection{\hologo{virTeX}}
%
%    \begin{macro}{\HoLogo@virTeX}
%    \begin{macrocode}
\def\HoLogo@virTeX#1{%
  \HOLOGO@mbox{%
    #1{v}{V}ir\hologo{TeX}%
  }%
}
%    \end{macrocode}
%    \end{macro}
%    \begin{macro}{\HoLogoCs@virTeX}
%    \begin{macrocode}
\def\HoLogoCs@virTeX#1{#1{v}{V}irTeX}
%    \end{macrocode}
%    \end{macro}
%    \begin{macro}{\HoLogoBkm@virTeX}
%    \begin{macrocode}
\def\HoLogoBkm@virTeX#1{%
  #1{v}{V}ir\hologo{TeX}%
}
%    \end{macrocode}
%    \end{macro}
%    \begin{macro}{\HoLogoHtml@virTeX}
%    \begin{macrocode}
\let\HoLogoHtml@virTeX\HoLogo@virTeX
%    \end{macrocode}
%    \end{macro}
%
% \subsubsection{\hologo{SliTeX}}
%
% \paragraph{Definitions of the three variants.}
%
%    \begin{macro}{\HoLogo@SLiTeX@lift}
%    \begin{macrocode}
\def\HoLogo@SLiTeX@lift#1{%
  \HoLogoFont@font{SliTeX}{rm}{%
    S%
    \kern-.06em%
    L%
    \kern-.18em%
    \raise.32ex\hbox{\HoLogoFont@font{SliTeX}{sc}{i}}%
    \HOLOGO@discretionary
    \kern-.06em%
    \hologo{TeX}%
  }%
}
%    \end{macrocode}
%    \end{macro}
%    \begin{macro}{\HoLogoBkm@SLiTeX@lift}
%    \begin{macrocode}
\def\HoLogoBkm@SLiTeX@lift#1{SLiTeX}
%    \end{macrocode}
%    \end{macro}
%    \begin{macro}{\HoLogoHtml@SLiTeX@lift}
%    \begin{macrocode}
\def\HoLogoHtml@SLiTeX@lift#1{%
  \HoLogoCss@SLiTeX@lift
  \HOLOGO@Span{SLiTeX-lift}{%
    \HoLogoFont@font{SliTeX}{rm}{%
      S%
      \HOLOGO@Span{L}{L}%
      \HOLOGO@Span{i}{i}%
      \hologo{TeX}%
    }%
  }%
}
%    \end{macrocode}
%    \end{macro}
%    \begin{macro}{\HoLogoCss@SLiTeX@lift}
%    \begin{macrocode}
\def\HoLogoCss@SLiTeX@lift{%
  \Css{%
    span.HoLogo-SLiTeX-lift span.HoLogo-L{%
      margin-left:-.06em;%
      margin-right:-.18em;%
    }%
  }%
  \Css{%
    span.HoLogo-SLiTeX-lift span.HoLogo-i{%
      position:relative;%
      top:-.32ex;%
      margin-right:-.06em;%
      font-variant:small-caps;%
    }%
  }%
  \global\let\HoLogoCss@SLiTeX@lift\relax
}
%    \end{macrocode}
%    \end{macro}
%
%    \begin{macro}{\HoLogo@SliTeX@simple}
%    \begin{macrocode}
\def\HoLogo@SliTeX@simple#1{%
  \HoLogoFont@font{SliTeX}{rm}{%
    \ltx@mbox{%
      \HoLogoFont@font{SliTeX}{sc}{Sli}%
    }%
    \HOLOGO@discretionary
    \hologo{TeX}%
  }%
}
%    \end{macrocode}
%    \end{macro}
%    \begin{macro}{\HoLogoBkm@SliTeX@simple}
%    \begin{macrocode}
\def\HoLogoBkm@SliTeX@simple#1{SliTeX}
%    \end{macrocode}
%    \end{macro}
%    \begin{macro}{\HoLogoHtml@SliTeX@simple}
%    \begin{macrocode}
\let\HoLogoHtml@SliTeX@simple\HoLogo@SliTeX@simple
%    \end{macrocode}
%    \end{macro}
%
%    \begin{macro}{\HoLogo@SliTeX@narrow}
%    \begin{macrocode}
\def\HoLogo@SliTeX@narrow#1{%
  \HoLogoFont@font{SliTeX}{rm}{%
    \ltx@mbox{%
      S%
      \kern-.06em%
      \HoLogoFont@font{SliTeX}{sc}{%
        l%
        \kern-.035em%
        i%
      }%
    }%
    \HOLOGO@discretionary
    \kern-.06em%
    \hologo{TeX}%
  }%
}
%    \end{macrocode}
%    \end{macro}
%    \begin{macro}{\HoLogoBkm@SliTeX@narrow}
%    \begin{macrocode}
\def\HoLogoBkm@SliTeX@narrow#1{SliTeX}
%    \end{macrocode}
%    \end{macro}
%    \begin{macro}{\HoLogoHtml@SliTeX@narrow}
%    \begin{macrocode}
\def\HoLogoHtml@SliTeX@narrow#1{%
  \HoLogoCss@SliTeX@narrow
  \HOLOGO@Span{SliTeX-narrow}{%
    \HoLogoFont@font{SliTeX}{rm}{%
      S%
        \HOLOGO@Span{l}{l}%
        \HOLOGO@Span{i}{i}%
      \hologo{TeX}%
    }%
  }%
}
%    \end{macrocode}
%    \end{macro}
%    \begin{macro}{\HoLogoCss@SliTeX@narrow}
%    \begin{macrocode}
\def\HoLogoCss@SliTeX@narrow{%
  \Css{%
    span.HoLogo-SliTeX-narrow span.HoLogo-l{%
      margin-left:-.06em;%
      margin-right:-.035em;%
      font-variant:small-caps;%
    }%
  }%
  \Css{%
    span.HoLogo-SliTeX-narrow span.HoLogo-i{%
      margin-right:-.06em;%
      font-variant:small-caps;%
    }%
  }%
  \global\let\HoLogoCss@SliTeX@narrow\relax
}
%    \end{macrocode}
%    \end{macro}
%
% \paragraph{Macro set completion.}
%
%    \begin{macro}{\HoLogo@SLiTeX@simple}
%    \begin{macrocode}
\def\HoLogo@SLiTeX@simple{\HoLogo@SliTeX@simple}
%    \end{macrocode}
%    \end{macro}
%    \begin{macro}{\HoLogoBkm@SLiTeX@simple}
%    \begin{macrocode}
\def\HoLogoBkm@SLiTeX@simple{\HoLogoBkm@SliTeX@simple}
%    \end{macrocode}
%    \end{macro}
%    \begin{macro}{\HoLogoHtml@SLiTeX@simple}
%    \begin{macrocode}
\def\HoLogoHtml@SLiTeX@simple{\HoLogoHtml@SliTeX@simple}
%    \end{macrocode}
%    \end{macro}
%
%    \begin{macro}{\HoLogo@SLiTeX@narrow}
%    \begin{macrocode}
\def\HoLogo@SLiTeX@narrow{\HoLogo@SliTeX@narrow}
%    \end{macrocode}
%    \end{macro}
%    \begin{macro}{\HoLogoBkm@SLiTeX@narrow}
%    \begin{macrocode}
\def\HoLogoBkm@SLiTeX@narrow{\HoLogoBkm@SliTeX@narrow}
%    \end{macrocode}
%    \end{macro}
%    \begin{macro}{\HoLogoHtml@SLiTeX@narrow}
%    \begin{macrocode}
\def\HoLogoHtml@SLiTeX@narrow{\HoLogoHtml@SliTeX@narrow}
%    \end{macrocode}
%    \end{macro}
%
%    \begin{macro}{\HoLogo@SliTeX@lift}
%    \begin{macrocode}
\def\HoLogo@SliTeX@lift{\HoLogo@SLiTeX@lift}
%    \end{macrocode}
%    \end{macro}
%    \begin{macro}{\HoLogoBkm@SliTeX@lift}
%    \begin{macrocode}
\def\HoLogoBkm@SliTeX@lift{\HoLogoBkm@SLiTeX@lift}
%    \end{macrocode}
%    \end{macro}
%    \begin{macro}{\HoLogoHtml@SliTeX@lift}
%    \begin{macrocode}
\def\HoLogoHtml@SliTeX@lift{\HoLogoHtml@SLiTeX@lift}
%    \end{macrocode}
%    \end{macro}
%
% \paragraph{Defaults.}
%
%    \begin{macro}{\HoLogo@SLiTeX}
%    \begin{macrocode}
\def\HoLogo@SLiTeX{\HoLogo@SLiTeX@lift}
%    \end{macrocode}
%    \end{macro}
%    \begin{macro}{\HoLogoBkm@SLiTeX}
%    \begin{macrocode}
\def\HoLogoBkm@SLiTeX{\HoLogoBkm@SLiTeX@lift}
%    \end{macrocode}
%    \end{macro}
%    \begin{macro}{\HoLogoHtml@SLiTeX}
%    \begin{macrocode}
\def\HoLogoHtml@SLiTeX{\HoLogoHtml@SLiTeX@lift}
%    \end{macrocode}
%    \end{macro}
%
%    \begin{macro}{\HoLogo@SliTeX}
%    \begin{macrocode}
\def\HoLogo@SliTeX{\HoLogo@SliTeX@narrow}
%    \end{macrocode}
%    \end{macro}
%    \begin{macro}{\HoLogoBkm@SliTeX}
%    \begin{macrocode}
\def\HoLogoBkm@SliTeX{\HoLogoBkm@SliTeX@narrow}
%    \end{macrocode}
%    \end{macro}
%    \begin{macro}{\HoLogoHtml@SliTeX}
%    \begin{macrocode}
\def\HoLogoHtml@SliTeX{\HoLogoHtml@SliTeX@narrow}
%    \end{macrocode}
%    \end{macro}
%
% \subsubsection{\hologo{LuaTeX}}
%
%    \begin{macro}{\HoLogo@LuaTeX}
%    The kerning is an idea of Hans Hagen, see mailing list
%    `luatex at tug dot org' in March 2010.
%    \begin{macrocode}
\def\HoLogo@LuaTeX#1{%
  \HOLOGO@mbox{%
    Lua%
    \HOLOGO@NegativeKerning{aT,oT,To}%
    \hologo{TeX}%
  }%
}
%    \end{macrocode}
%    \end{macro}
%    \begin{macro}{\HoLogoHtml@LuaTeX}
%    \begin{macrocode}
\let\HoLogoHtml@LuaTeX\HoLogo@LuaTeX
%    \end{macrocode}
%    \end{macro}
%
% \subsubsection{\hologo{LuaLaTeX}}
%
%    \begin{macro}{\HoLogo@LuaLaTeX}
%    \begin{macrocode}
\def\HoLogo@LuaLaTeX#1{%
  \HOLOGO@mbox{%
    Lua%
    \hologo{LaTeX}%
  }%
}
%    \end{macrocode}
%    \end{macro}
%    \begin{macro}{\HoLogoHtml@LuaLaTeX}
%    \begin{macrocode}
\let\HoLogoHtml@LuaLaTeX\HoLogo@LuaLaTeX
%    \end{macrocode}
%    \end{macro}
%
% \subsubsection{\hologo{XeTeX}, \hologo{XeLaTeX}}
%
%    \begin{macro}{\HOLOGO@IfCharExists}
%    \begin{macrocode}
\ifluatex
  \ifnum\luatexversion<36 %
  \else
    \def\HOLOGO@IfCharExists#1{%
      \ifnum
        \directlua{%
           if luaotfload and luaotfload.aux then
             if luaotfload.aux.font_has_glyph(%
                    font.current(), \number#1) then % 	 
	       tex.print("1") % 	 
	     end % 	 
	   elseif font and font.fonts and font.current then %
            local f = font.fonts[font.current()]%
            if f.characters and f.characters[\number#1] then %
              tex.print("1")%
            end %
          end%
        }0=\ltx@zero
        \expandafter\ltx@secondoftwo
      \else
        \expandafter\ltx@firstoftwo
      \fi
    }%
  \fi
\fi
\ltx@IfUndefined{HOLOGO@IfCharExists}{%
  \def\HOLOGO@@IfCharExists#1{%
    \begingroup
      \tracinglostchars=\ltx@zero
      \setbox\ltx@zero=\hbox{%
        \kern7sp\char#1\relax
        \ifnum\lastkern>\ltx@zero
          \expandafter\aftergroup\csname iffalse\endcsname
        \else
          \expandafter\aftergroup\csname iftrue\endcsname
        \fi
      }%
      % \if{true|false} from \aftergroup
      \endgroup
      \expandafter\ltx@firstoftwo
    \else
      \endgroup
      \expandafter\ltx@secondoftwo
    \fi
  }%
  \ifxetex
    \ltx@IfUndefined{XeTeXfonttype}{}{%
      \ltx@IfUndefined{XeTeXcharglyph}{}{%
        \def\HOLOGO@IfCharExists#1{%
          \ifnum\XeTeXfonttype\font>\ltx@zero
            \expandafter\ltx@firstofthree
          \else
            \expandafter\ltx@gobble
          \fi
          {%
            \ifnum\XeTeXcharglyph#1>\ltx@zero
              \expandafter\ltx@firstoftwo
            \else
              \expandafter\ltx@secondoftwo
            \fi
          }%
          \HOLOGO@@IfCharExists{#1}%
        }%
      }%
    }%
  \fi
}{}
\ltx@ifundefined{HOLOGO@IfCharExists}{%
  \ifnum64=`\^^^^0040\relax % test for big chars of LuaTeX/XeTeX
    \let\HOLOGO@IfCharExists\HOLOGO@@IfCharExists
  \else
    \def\HOLOGO@IfCharExists#1{%
      \ifnum#1>255 %
        \expandafter\ltx@fourthoffour
      \fi
      \HOLOGO@@IfCharExists{#1}%
    }%
  \fi
}{}
%    \end{macrocode}
%    \end{macro}
%
%    \begin{macro}{\HoLogo@Xe}
%    Source: package \xpackage{dtklogos}
%    \begin{macrocode}
\def\HoLogo@Xe#1{%
  X%
  \kern-.1em\relax
  \HOLOGO@IfCharExists{"018E}{%
    \lower.5ex\hbox{\char"018E}%
  }{%
    \chardef\HOLOGO@choice=\ltx@zero
    \ifdim\fontdimen\ltx@one\font>0pt %
      \ltx@IfUndefined{rotatebox}{%
        \ltx@IfUndefined{pgftext}{%
          \ltx@IfUndefined{psscalebox}{%
            \ltx@IfUndefined{HOLOGO@ScaleBox@\hologoDriver}{%
            }{%
              \chardef\HOLOGO@choice=4 %
            }%
          }{%
            \chardef\HOLOGO@choice=3 %
          }%
        }{%
          \chardef\HOLOGO@choice=2 %
        }%
      }{%
        \chardef\HOLOGO@choice=1 %
      }%
      \ifcase\HOLOGO@choice
        \HOLOGO@WarningUnsupportedDriver{Xe}%
        e%
      \or % 1: \rotatebox
        \begingroup
          \setbox\ltx@zero\hbox{\rotatebox{180}{E}}%
          \ltx@LocDimenA=\dp\ltx@zero
          \advance\ltx@LocDimenA by -.5ex\relax
          \raise\ltx@LocDimenA\box\ltx@zero
        \endgroup
      \or % 2: \pgftext
        \lower.5ex\hbox{%
          \pgfpicture
            \pgftext[rotate=180]{E}%
          \endpgfpicture
        }%
      \or % 3: \psscalebox
        \begingroup
          \setbox\ltx@zero\hbox{\psscalebox{-1 -1}{E}}%
          \ltx@LocDimenA=\dp\ltx@zero
          \advance\ltx@LocDimenA by -.5ex\relax
          \raise\ltx@LocDimenA\box\ltx@zero
        \endgroup
      \or % 4: \HOLOGO@PointReflectBox
        \lower.5ex\hbox{\HOLOGO@PointReflectBox{E}}%
      \else
        \@PackageError{hologo}{Internal error (choice/it}\@ehc
      \fi
    \else
      \ltx@IfUndefined{reflectbox}{%
        \ltx@IfUndefined{pgftext}{%
          \ltx@IfUndefined{psscalebox}{%
            \ltx@IfUndefined{HOLOGO@ScaleBox@\hologoDriver}{%
            }{%
              \chardef\HOLOGO@choice=4 %
            }%
          }{%
            \chardef\HOLOGO@choice=3 %
          }%
        }{%
          \chardef\HOLOGO@choice=2 %
        }%
      }{%
        \chardef\HOLOGO@choice=1 %
      }%
      \ifcase\HOLOGO@choice
        \HOLOGO@WarningUnsupportedDriver{Xe}%
        e%
      \or % 1: reflectbox
        \lower.5ex\hbox{%
          \reflectbox{E}%
        }%
      \or % 2: \pgftext
        \lower.5ex\hbox{%
          \pgfpicture
            \pgftransformxscale{-1}%
            \pgftext{E}%
          \endpgfpicture
        }%
      \or % 3: \psscalebox
        \lower.5ex\hbox{%
          \psscalebox{-1 1}{E}%
        }%
      \or % 4: \HOLOGO@Reflectbox
        \lower.5ex\hbox{%
          \HOLOGO@ReflectBox{E}%
        }%
      \else
        \@PackageError{hologo}{Internal error (choice/up)}\@ehc
      \fi
    \fi
  }%
}
%    \end{macrocode}
%    \end{macro}
%    \begin{macro}{\HoLogoHtml@Xe}
%    \begin{macrocode}
\def\HoLogoHtml@Xe#1{%
  \HoLogoCss@Xe
  \HOLOGO@Span{Xe}{%
    X%
    \HOLOGO@Span{e}{%
      \HCode{&\ltx@hashchar x018e;}%
    }%
  }%
}
%    \end{macrocode}
%    \end{macro}
%    \begin{macro}{\HoLogoCss@Xe}
%    \begin{macrocode}
\def\HoLogoCss@Xe{%
  \Css{%
    span.HoLogo-Xe span.HoLogo-e{%
      position:relative;%
      top:.5ex;%
      left-margin:-.1em;%
    }%
  }%
  \global\let\HoLogoCss@Xe\relax
}
%    \end{macrocode}
%    \end{macro}
%
%    \begin{macro}{\HoLogo@XeTeX}
%    \begin{macrocode}
\def\HoLogo@XeTeX#1{%
  \hologo{Xe}%
  \kern-.15em\relax
  \hologo{TeX}%
}
%    \end{macrocode}
%    \end{macro}
%
%    \begin{macro}{\HoLogoHtml@XeTeX}
%    \begin{macrocode}
\def\HoLogoHtml@XeTeX#1{%
  \HoLogoCss@XeTeX
  \HOLOGO@Span{XeTeX}{%
    \hologo{Xe}%
    \hologo{TeX}%
  }%
}
%    \end{macrocode}
%    \end{macro}
%    \begin{macro}{\HoLogoCss@XeTeX}
%    \begin{macrocode}
\def\HoLogoCss@XeTeX{%
  \Css{%
    span.HoLogo-XeTeX span.HoLogo-TeX{%
      margin-left:-.15em;%
    }%
  }%
  \global\let\HoLogoCss@XeTeX\relax
}
%    \end{macrocode}
%    \end{macro}
%
%    \begin{macro}{\HoLogo@XeLaTeX}
%    \begin{macrocode}
\def\HoLogo@XeLaTeX#1{%
  \hologo{Xe}%
  \kern-.13em%
  \hologo{LaTeX}%
}
%    \end{macrocode}
%    \end{macro}
%    \begin{macro}{\HoLogoHtml@XeLaTeX}
%    \begin{macrocode}
\def\HoLogoHtml@XeLaTeX#1{%
  \HoLogoCss@XeLaTeX
  \HOLOGO@Span{XeLaTeX}{%
    \hologo{Xe}%
    \hologo{LaTeX}%
  }%
}
%    \end{macrocode}
%    \end{macro}
%    \begin{macro}{\HoLogoCss@XeLaTeX}
%    \begin{macrocode}
\def\HoLogoCss@XeLaTeX{%
  \Css{%
    span.HoLogo-XeLaTeX span.HoLogo-Xe{%
      margin-right:-.13em;%
    }%
  }%
  \global\let\HoLogoCss@XeLaTeX\relax
}
%    \end{macrocode}
%    \end{macro}
%
% \subsubsection{\hologo{pdfTeX}, \hologo{pdfLaTeX}}
%
%    \begin{macro}{\HoLogo@pdfTeX}
%    \begin{macrocode}
\def\HoLogo@pdfTeX#1{%
  \HOLOGO@mbox{%
    #1{p}{P}df\hologo{TeX}%
  }%
}
%    \end{macrocode}
%    \end{macro}
%    \begin{macro}{\HoLogoCs@pdfTeX}
%    \begin{macrocode}
\def\HoLogoCs@pdfTeX#1{#1{p}{P}dfTeX}
%    \end{macrocode}
%    \end{macro}
%    \begin{macro}{\HoLogoBkm@pdfTeX}
%    \begin{macrocode}
\def\HoLogoBkm@pdfTeX#1{%
  #1{p}{P}df\hologo{TeX}%
}
%    \end{macrocode}
%    \end{macro}
%    \begin{macro}{\HoLogoHtml@pdfTeX}
%    \begin{macrocode}
\let\HoLogoHtml@pdfTeX\HoLogo@pdfTeX
%    \end{macrocode}
%    \end{macro}
%
%    \begin{macro}{\HoLogo@pdfLaTeX}
%    \begin{macrocode}
\def\HoLogo@pdfLaTeX#1{%
  \HOLOGO@mbox{%
    #1{p}{P}df\hologo{LaTeX}%
  }%
}
%    \end{macrocode}
%    \end{macro}
%    \begin{macro}{\HoLogoCs@pdfLaTeX}
%    \begin{macrocode}
\def\HoLogoCs@pdfLaTeX#1{#1{p}{P}dfLaTeX}
%    \end{macrocode}
%    \end{macro}
%    \begin{macro}{\HoLogoBkm@pdfLaTeX}
%    \begin{macrocode}
\def\HoLogoBkm@pdfLaTeX#1{%
  #1{p}{P}df\hologo{LaTeX}%
}
%    \end{macrocode}
%    \end{macro}
%    \begin{macro}{\HoLogoHtml@pdfLaTeX}
%    \begin{macrocode}
\let\HoLogoHtml@pdfLaTeX\HoLogo@pdfLaTeX
%    \end{macrocode}
%    \end{macro}
%
% \subsubsection{\hologo{VTeX}}
%
%    \begin{macro}{\HoLogo@VTeX}
%    \begin{macrocode}
\def\HoLogo@VTeX#1{%
  \HOLOGO@mbox{%
    V\hologo{TeX}%
  }%
}
%    \end{macrocode}
%    \end{macro}
%    \begin{macro}{\HoLogoHtml@VTeX}
%    \begin{macrocode}
\let\HoLogoHtml@VTeX\HoLogo@VTeX
%    \end{macrocode}
%    \end{macro}
%
% \subsubsection{\hologo{AmS}, \dots}
%
%    Source: class \xclass{amsdtx}
%
%    \begin{macro}{\HoLogo@AmS}
%    \begin{macrocode}
\def\HoLogo@AmS#1{%
  \HoLogoFont@font{AmS}{sy}{%
    A%
    \kern-.1667em%
    \lower.5ex\hbox{M}%
    \kern-.125em%
    S%
  }%
}
%    \end{macrocode}
%    \end{macro}
%    \begin{macro}{\HoLogoBkm@AmS}
%    \begin{macrocode}
\def\HoLogoBkm@AmS#1{AmS}
%    \end{macrocode}
%    \end{macro}
%    \begin{macro}{\HoLogoHtml@AmS}
%    \begin{macrocode}
\def\HoLogoHtml@AmS#1{%
  \HoLogoCss@AmS
%  \HoLogoFont@font{AmS}{sy}{%
    \HOLOGO@Span{AmS}{%
      A%
      \HOLOGO@Span{M}{M}%
      S%
    }%
%   }%
}
%    \end{macrocode}
%    \end{macro}
%    \begin{macro}{\HoLogoCss@AmS}
%    \begin{macrocode}
\def\HoLogoCss@AmS{%
  \Css{%
    span.HoLogo-AmS span.HoLogo-M{%
      position:relative;%
      top:.5ex;%
      margin-left:-.1667em;%
      margin-right:-.125em;%
      text-decoration:none;%
    }%
  }%
  \global\let\HoLogoCss@AmS\relax
}
%    \end{macrocode}
%    \end{macro}
%
%    \begin{macro}{\HoLogo@AmSTeX}
%    \begin{macrocode}
\def\HoLogo@AmSTeX#1{%
  \hologo{AmS}%
  \HOLOGO@hyphen
  \hologo{TeX}%
}
%    \end{macrocode}
%    \end{macro}
%    \begin{macro}{\HoLogoBkm@AmSTeX}
%    \begin{macrocode}
\def\HoLogoBkm@AmSTeX#1{AmS-TeX}%
%    \end{macrocode}
%    \end{macro}
%    \begin{macro}{\HoLogoHtml@AmSTeX}
%    \begin{macrocode}
\let\HoLogoHtml@AmSTeX\HoLogo@AmSTeX
%    \end{macrocode}
%    \end{macro}
%
%    \begin{macro}{\HoLogo@AmSLaTeX}
%    \begin{macrocode}
\def\HoLogo@AmSLaTeX#1{%
  \hologo{AmS}%
  \HOLOGO@hyphen
  \hologo{LaTeX}%
}
%    \end{macrocode}
%    \end{macro}
%    \begin{macro}{\HoLogoBkm@AmSLaTeX}
%    \begin{macrocode}
\def\HoLogoBkm@AmSLaTeX#1{AmS-LaTeX}%
%    \end{macrocode}
%    \end{macro}
%    \begin{macro}{\HoLogoHtml@AmSLaTeX}
%    \begin{macrocode}
\let\HoLogoHtml@AmSLaTeX\HoLogo@AmSLaTeX
%    \end{macrocode}
%    \end{macro}
%
% \subsubsection{\hologo{BibTeX}}
%
%    \begin{macro}{\HoLogo@BibTeX@sc}
%    A definition of \hologo{BibTeX} is provided in
%    the documentation source for the manual of \hologo{BibTeX}
%    \cite{btxdoc}.
%\begin{quote}
%\begin{verbatim}
%\def\BibTeX{%
%  {%
%    \rm
%    B%
%    \kern-.05em%
%    {%
%      \sc
%      i%
%      \kern-.025em %
%      b%
%    }%
%    \kern-.08em
%    T%
%    \kern-.1667em%
%    \lower.7ex\hbox{E}%
%    \kern-.125em%
%    X%
%  }%
%}
%\end{verbatim}
%\end{quote}
%    \begin{macrocode}
\def\HoLogo@BibTeX@sc#1{%
  B%
  \kern-.05em%
  \HoLogoFont@font{BibTeX}{sc}{%
    i%
    \kern-.025em%
    b%
  }%
  \HOLOGO@discretionary
  \kern-.08em%
  \hologo{TeX}%
}
%    \end{macrocode}
%    \end{macro}
%    \begin{macro}{\HoLogoHtml@BibTeX@sc}
%    \begin{macrocode}
\def\HoLogoHtml@BibTeX@sc#1{%
  \HoLogoCss@BibTeX@sc
  \HOLOGO@Span{BibTeX-sc}{%
    B%
    \HOLOGO@Span{i}{i}%
    \HOLOGO@Span{b}{b}%
    \hologo{TeX}%
  }%
}
%    \end{macrocode}
%    \end{macro}
%    \begin{macro}{\HoLogoCss@BibTeX@sc}
%    \begin{macrocode}
\def\HoLogoCss@BibTeX@sc{%
  \Css{%
    span.HoLogo-BibTeX-sc span.HoLogo-i{%
      margin-left:-.05em;%
      margin-right:-.025em;%
      font-variant:small-caps;%
    }%
  }%
  \Css{%
    span.HoLogo-BibTeX-sc span.HoLogo-b{%
      margin-right:-.08em;%
      font-variant:small-caps;%
    }%
  }%
  \global\let\HoLogoCss@BibTeX@sc\relax
}
%    \end{macrocode}
%    \end{macro}
%
%    \begin{macro}{\HoLogo@BibTeX@sf}
%    Variant \xoption{sf} avoids trouble with unavailable
%    small caps fonts (e.g., bold versions of Computer Modern or
%    Latin Modern). The definition is taken from
%    package \xpackage{dtklogos} \cite{dtklogos}.
%\begin{quote}
%\begin{verbatim}
%\DeclareRobustCommand{\BibTeX}{%
%  B%
%  \kern-.05em%
%  \hbox{%
%    $\m@th$% %% force math size calculations
%    \csname S@\f@size\endcsname
%    \fontsize\sf@size\z@
%    \math@fontsfalse
%    \selectfont
%    I%
%    \kern-.025em%
%    B
%  }%
%  \kern-.08em%
%  \-%
%  \TeX
%}
%\end{verbatim}
%\end{quote}
%    \begin{macrocode}
\def\HoLogo@BibTeX@sf#1{%
  B%
  \kern-.05em%
  \HoLogoFont@font{BibTeX}{bibsf}{%
    I%
    \kern-.025em%
    B%
  }%
  \HOLOGO@discretionary
  \kern-.08em%
  \hologo{TeX}%
}
%    \end{macrocode}
%    \end{macro}
%    \begin{macro}{\HoLogoHtml@BibTeX@sf}
%    \begin{macrocode}
\def\HoLogoHtml@BibTeX@sf#1{%
  \HoLogoCss@BibTeX@sf
  \HOLOGO@Span{BibTeX-sf}{%
    B%
    \HoLogoFont@font{BibTeX}{bibsf}{%
      \HOLOGO@Span{i}{I}%
      B%
    }%
    \hologo{TeX}%
  }%
}
%    \end{macrocode}
%    \end{macro}
%    \begin{macro}{\HoLogoCss@BibTeX@sf}
%    \begin{macrocode}
\def\HoLogoCss@BibTeX@sf{%
  \Css{%
    span.HoLogo-BibTeX-sf span.HoLogo-i{%
      margin-left:-.05em;%
      margin-right:-.025em;%
    }%
  }%
  \Css{%
    span.HoLogo-BibTeX-sf span.HoLogo-TeX{%
      margin-left:-.08em;%
    }%
  }%
  \global\let\HoLogoCss@BibTeX@sf\relax
}
%    \end{macrocode}
%    \end{macro}
%
%    \begin{macro}{\HoLogo@BibTeX}
%    \begin{macrocode}
\def\HoLogo@BibTeX{\HoLogo@BibTeX@sf}
%    \end{macrocode}
%    \end{macro}
%    \begin{macro}{\HoLogoHtml@BibTeX}
%    \begin{macrocode}
\def\HoLogoHtml@BibTeX{\HoLogoHtml@BibTeX@sf}
%    \end{macrocode}
%    \end{macro}
%
% \subsubsection{\hologo{BibTeX8}}
%
%    \begin{macro}{\HoLogo@BibTeX8}
%    \begin{macrocode}
\expandafter\def\csname HoLogo@BibTeX8\endcsname#1{%
  \hologo{BibTeX}%
  8%
}
%    \end{macrocode}
%    \end{macro}
%
%    \begin{macro}{\HoLogoBkm@BibTeX8}
%    \begin{macrocode}
\expandafter\def\csname HoLogoBkm@BibTeX8\endcsname#1{%
  \hologo{BibTeX}%
  8%
}
%    \end{macrocode}
%    \end{macro}
%    \begin{macro}{\HoLogoHtml@BibTeX8}
%    \begin{macrocode}
\expandafter
\let\csname HoLogoHtml@BibTeX8\expandafter\endcsname
\csname HoLogo@BibTeX8\endcsname
%    \end{macrocode}
%    \end{macro}
%
% \subsubsection{\hologo{ConTeXt}}
%
%    \begin{macro}{\HoLogo@ConTeXt@simple}
%    \begin{macrocode}
\def\HoLogo@ConTeXt@simple#1{%
  \HOLOGO@mbox{Con}%
  \HOLOGO@discretionary
  \HOLOGO@mbox{\hologo{TeX}t}%
}
%    \end{macrocode}
%    \end{macro}
%    \begin{macro}{\HoLogoHtml@ConTeXt@simple}
%    \begin{macrocode}
\let\HoLogoHtml@ConTeXt@simple\HoLogo@ConTeXt@simple
%    \end{macrocode}
%    \end{macro}
%
%    \begin{macro}{\HoLogo@ConTeXt@narrow}
%    This definition of logo \hologo{ConTeXt} with variant \xoption{narrow}
%    comes from TUGboat's class \xclass{ltugboat} (version 2010/11/15 v2.8).
%    \begin{macrocode}
\def\HoLogo@ConTeXt@narrow#1{%
  \HOLOGO@mbox{C\kern-.0333emon}%
  \HOLOGO@discretionary
  \kern-.0667em%
  \HOLOGO@mbox{\hologo{TeX}\kern-.0333emt}%
}
%    \end{macrocode}
%    \end{macro}
%    \begin{macro}{\HoLogoHtml@ConTeXt@narrow}
%    \begin{macrocode}
\def\HoLogoHtml@ConTeXt@narrow#1{%
  \HoLogoCss@ConTeXt@narrow
  \HOLOGO@Span{ConTeXt-narrow}{%
    \HOLOGO@Span{C}{C}%
    on%
    \hologo{TeX}%
    t%
  }%
}
%    \end{macrocode}
%    \end{macro}
%    \begin{macro}{\HoLogoCss@ConTeXt@narrow}
%    \begin{macrocode}
\def\HoLogoCss@ConTeXt@narrow{%
  \Css{%
    span.HoLogo-ConTeXt-narrow span.HoLogo-C{%
      margin-left:-.0333em;%
    }%
  }%
  \Css{%
    span.HoLogo-ConTeXt-narrow span.HoLogo-TeX{%
      margin-left:-.0667em;%
      margin-right:-.0333em;%
    }%
  }%
  \global\let\HoLogoCss@ConTeXt@narrow\relax
}
%    \end{macrocode}
%    \end{macro}
%
%    \begin{macro}{\HoLogo@ConTeXt}
%    \begin{macrocode}
\def\HoLogo@ConTeXt{\HoLogo@ConTeXt@narrow}
%    \end{macrocode}
%    \end{macro}
%    \begin{macro}{\HoLogoHtml@ConTeXt}
%    \begin{macrocode}
\def\HoLogoHtml@ConTeXt{\HoLogoHtml@ConTeXt@narrow}
%    \end{macrocode}
%    \end{macro}
%
% \subsubsection{\hologo{emTeX}}
%
%    \begin{macro}{\HoLogo@emTeX}
%    \begin{macrocode}
\def\HoLogo@emTeX#1{%
  \HOLOGO@mbox{#1{e}{E}m}%
  \HOLOGO@discretionary
  \hologo{TeX}%
}
%    \end{macrocode}
%    \end{macro}
%    \begin{macro}{\HoLogoCs@emTeX}
%    \begin{macrocode}
\def\HoLogoCs@emTeX#1{#1{e}{E}mTeX}%
%    \end{macrocode}
%    \end{macro}
%    \begin{macro}{\HoLogoBkm@emTeX}
%    \begin{macrocode}
\def\HoLogoBkm@emTeX#1{%
  #1{e}{E}m\hologo{TeX}%
}
%    \end{macrocode}
%    \end{macro}
%    \begin{macro}{\HoLogoHtml@emTeX}
%    \begin{macrocode}
\let\HoLogoHtml@emTeX\HoLogo@emTeX
%    \end{macrocode}
%    \end{macro}
%
% \subsubsection{\hologo{ExTeX}}
%
%    \begin{macro}{\HoLogo@ExTeX}
%    The definition is taken from the FAQ of the
%    project \hologo{ExTeX}
%    \cite{ExTeX-FAQ}.
%\begin{quote}
%\begin{verbatim}
%\def\ExTeX{%
%  \textrm{% Logo always with serifs
%    \ensuremath{%
%      \textstyle
%      \varepsilon_{%
%        \kern-0.15em%
%        \mathcal{X}%
%      }%
%    }%
%    \kern-.15em%
%    \TeX
%  }%
%}
%\end{verbatim}
%\end{quote}
%    \begin{macrocode}
\def\HoLogo@ExTeX#1{%
  \HoLogoFont@font{ExTeX}{rm}{%
    \ltx@mbox{%
      \HOLOGO@MathSetup
      $%
        \textstyle
        \varepsilon_{%
          \kern-0.15em%
          \HoLogoFont@font{ExTeX}{sy}{X}%
        }%
      $%
    }%
    \HOLOGO@discretionary
    \kern-.15em%
    \hologo{TeX}%
  }%
}
%    \end{macrocode}
%    \end{macro}
%    \begin{macro}{\HoLogoHtml@ExTeX}
%    \begin{macrocode}
\def\HoLogoHtml@ExTeX#1{%
  \HoLogoCss@ExTeX
  \HoLogoFont@font{ExTeX}{rm}{%
    \HOLOGO@Span{ExTeX}{%
      \ltx@mbox{%
        \HOLOGO@MathSetup
        $\textstyle\varepsilon$%
        \HOLOGO@Span{X}{$\textstyle\chi$}%
        \hologo{TeX}%
      }%
    }%
  }%
}
%    \end{macrocode}
%    \end{macro}
%    \begin{macro}{\HoLogoBkm@ExTeX}
%    \begin{macrocode}
\def\HoLogoBkm@ExTeX#1{%
  \HOLOGO@PdfdocUnicode{#1{e}{E}x}{\textepsilon\textchi}%
  \hologo{TeX}%
}
%    \end{macrocode}
%    \end{macro}
%    \begin{macro}{\HoLogoCss@ExTeX}
%    \begin{macrocode}
\def\HoLogoCss@ExTeX{%
  \Css{%
    span.HoLogo-ExTeX{%
      font-family:serif;%
    }%
  }%
  \Css{%
    span.HoLogo-ExTeX span.HoLogo-TeX{%
      margin-left:-.15em;%
    }%
  }%
  \global\let\HoLogoCss@ExTeX\relax
}
%    \end{macrocode}
%    \end{macro}
%
% \subsubsection{\hologo{MiKTeX}}
%
%    \begin{macro}{\HoLogo@MiKTeX}
%    \begin{macrocode}
\def\HoLogo@MiKTeX#1{%
  \HOLOGO@mbox{MiK}%
  \HOLOGO@discretionary
  \hologo{TeX}%
}
%    \end{macrocode}
%    \end{macro}
%    \begin{macro}{\HoLogoHtml@MiKTeX}
%    \begin{macrocode}
\let\HoLogoHtml@MiKTeX\HoLogo@MiKTeX
%    \end{macrocode}
%    \end{macro}
%
% \subsubsection{\hologo{OzTeX} and friends}
%
%    Source: \hologo{OzTeX} FAQ \cite{OzTeX}:
%    \begin{quote}
%      |\def\OzTeX{O\kern-.03em z\kern-.15em\TeX}|\\
%      (There is no kerning in OzMF, OzMP and OzTtH.)
%    \end{quote}
%
%    \begin{macro}{\HoLogo@OzTeX}
%    \begin{macrocode}
\def\HoLogo@OzTeX#1{%
  O%
  \kern-.03em %
  z%
  \kern-.15em %
  \hologo{TeX}%
}
%    \end{macrocode}
%    \end{macro}
%    \begin{macro}{\HoLogoHtml@OzTeX}
%    \begin{macrocode}
\def\HoLogoHtml@OzTeX#1{%
  \HoLogoCss@OzTeX
  \HOLOGO@Span{OzTeX}{%
    O%
    \HOLOGO@Span{z}{z}%
    \hologo{TeX}%
  }%
}
%    \end{macrocode}
%    \end{macro}
%    \begin{macro}{\HoLogoCss@OzTeX}
%    \begin{macrocode}
\def\HoLogoCss@OzTeX{%
  \Css{%
    span.HoLogo-OzTeX span.HoLogo-z{%
      margin-left:-.03em;%
      margin-right:-.15em;%
    }%
  }%
  \global\let\HoLogoCss@OzTeX\relax
}
%    \end{macrocode}
%    \end{macro}
%
%    \begin{macro}{\HoLogo@OzMF}
%    \begin{macrocode}
\def\HoLogo@OzMF#1{%
  \HOLOGO@mbox{OzMF}%
}
%    \end{macrocode}
%    \end{macro}
%    \begin{macro}{\HoLogo@OzMP}
%    \begin{macrocode}
\def\HoLogo@OzMP#1{%
  \HOLOGO@mbox{OzMP}%
}
%    \end{macrocode}
%    \end{macro}
%    \begin{macro}{\HoLogo@OzTtH}
%    \begin{macrocode}
\def\HoLogo@OzTtH#1{%
  \HOLOGO@mbox{OzTtH}%
}
%    \end{macrocode}
%    \end{macro}
%
% \subsubsection{\hologo{PCTeX}}
%
%    \begin{macro}{\HoLogo@PCTeX}
%    \begin{macrocode}
\def\HoLogo@PCTeX#1{%
  \HOLOGO@mbox{PC}%
  \hologo{TeX}%
}
%    \end{macrocode}
%    \end{macro}
%    \begin{macro}{\HoLogoHtml@PCTeX}
%    \begin{macrocode}
\let\HoLogoHtml@PCTeX\HoLogo@PCTeX
%    \end{macrocode}
%    \end{macro}
%
% \subsubsection{\hologo{PiCTeX}}
%
%    The original definitions from \xfile{pictex.tex} \cite{PiCTeX}:
%\begin{quote}
%\begin{verbatim}
%\def\PiC{%
%  P%
%  \kern-.12em%
%  \lower.5ex\hbox{I}%
%  \kern-.075em%
%  C%
%}
%\def\PiCTeX{%
%  \PiC
%  \kern-.11em%
%  \TeX
%}
%\end{verbatim}
%\end{quote}
%
%    \begin{macro}{\HoLogo@PiC}
%    \begin{macrocode}
\def\HoLogo@PiC#1{%
  P%
  \kern-.12em%
  \lower.5ex\hbox{I}%
  \kern-.075em%
  C%
  \HOLOGO@SpaceFactor
}
%    \end{macrocode}
%    \end{macro}
%    \begin{macro}{\HoLogoHtml@PiC}
%    \begin{macrocode}
\def\HoLogoHtml@PiC#1{%
  \HoLogoCss@PiC
  \HOLOGO@Span{PiC}{%
    P%
    \HOLOGO@Span{i}{I}%
    C%
  }%
}
%    \end{macrocode}
%    \end{macro}
%    \begin{macro}{\HoLogoCss@PiC}
%    \begin{macrocode}
\def\HoLogoCss@PiC{%
  \Css{%
    span.HoLogo-PiC span.HoLogo-i{%
      position:relative;%
      top:.5ex;%
      margin-left:-.12em;%
      margin-right:-.075em;%
      text-decoration:none;%
    }%
  }%
  \global\let\HoLogoCss@PiC\relax
}
%    \end{macrocode}
%    \end{macro}
%
%    \begin{macro}{\HoLogo@PiCTeX}
%    \begin{macrocode}
\def\HoLogo@PiCTeX#1{%
  \hologo{PiC}%
  \HOLOGO@discretionary
  \kern-.11em%
  \hologo{TeX}%
}
%    \end{macrocode}
%    \end{macro}
%    \begin{macro}{\HoLogoHtml@PiCTeX}
%    \begin{macrocode}
\def\HoLogoHtml@PiCTeX#1{%
  \HoLogoCss@PiCTeX
  \HOLOGO@Span{PiCTeX}{%
    \hologo{PiC}%
    \hologo{TeX}%
  }%
}
%    \end{macrocode}
%    \end{macro}
%    \begin{macro}{\HoLogoCss@PiCTeX}
%    \begin{macrocode}
\def\HoLogoCss@PiCTeX{%
  \Css{%
    span.HoLogo-PiCTeX span.HoLogo-PiC{%
      margin-right:-.11em;%
    }%
  }%
  \global\let\HoLogoCss@PiCTeX\relax
}
%    \end{macrocode}
%    \end{macro}
%
% \subsubsection{\hologo{teTeX}}
%
%    \begin{macro}{\HoLogo@teTeX}
%    \begin{macrocode}
\def\HoLogo@teTeX#1{%
  \HOLOGO@mbox{#1{t}{T}e}%
  \HOLOGO@discretionary
  \hologo{TeX}%
}
%    \end{macrocode}
%    \end{macro}
%    \begin{macro}{\HoLogoCs@teTeX}
%    \begin{macrocode}
\def\HoLogoCs@teTeX#1{#1{t}{T}dfTeX}
%    \end{macrocode}
%    \end{macro}
%    \begin{macro}{\HoLogoBkm@teTeX}
%    \begin{macrocode}
\def\HoLogoBkm@teTeX#1{%
  #1{t}{T}e\hologo{TeX}%
}
%    \end{macrocode}
%    \end{macro}
%    \begin{macro}{\HoLogoHtml@teTeX}
%    \begin{macrocode}
\let\HoLogoHtml@teTeX\HoLogo@teTeX
%    \end{macrocode}
%    \end{macro}
%
% \subsubsection{\hologo{TeX4ht}}
%
%    \begin{macro}{\HoLogo@TeX4ht}
%    \begin{macrocode}
\expandafter\def\csname HoLogo@TeX4ht\endcsname#1{%
  \HOLOGO@mbox{\hologo{TeX}4ht}%
}
%    \end{macrocode}
%    \end{macro}
%    \begin{macro}{\HoLogoHtml@TeX4ht}
%    \begin{macrocode}
\expandafter
\let\csname HoLogoHtml@TeX4ht\expandafter\endcsname
\csname HoLogo@TeX4ht\endcsname
%    \end{macrocode}
%    \end{macro}
%
%
% \subsubsection{\hologo{SageTeX}}
%
%    \begin{macro}{\HoLogo@SageTeX}
%    \begin{macrocode}
\def\HoLogo@SageTeX#1{%
  \HOLOGO@mbox{Sage}%
  \HOLOGO@discretionary
  \HOLOGO@NegativeKerning{eT,oT,To}%
  \hologo{TeX}%
}
%    \end{macrocode}
%    \end{macro}
%    \begin{macro}{\HoLogoHtml@SageTeX}
%    \begin{macrocode}
\let\HoLogoHtml@SageTeX\HoLogo@SageTeX
%    \end{macrocode}
%    \end{macro}
%
% \subsection{\hologo{METAFONT} and friends}
%
%    \begin{macro}{\HoLogo@METAFONT}
%    \begin{macrocode}
\def\HoLogo@METAFONT#1{%
  \HoLogoFont@font{METAFONT}{logo}{%
    \HOLOGO@mbox{META}%
    \HOLOGO@discretionary
    \HOLOGO@mbox{FONT}%
  }%
}
%    \end{macrocode}
%    \end{macro}
%
%    \begin{macro}{\HoLogo@METAPOST}
%    \begin{macrocode}
\def\HoLogo@METAPOST#1{%
  \HoLogoFont@font{METAPOST}{logo}{%
    \HOLOGO@mbox{META}%
    \HOLOGO@discretionary
    \HOLOGO@mbox{POST}%
  }%
}
%    \end{macrocode}
%    \end{macro}
%
%    \begin{macro}{\HoLogo@MetaFun}
%    \begin{macrocode}
\def\HoLogo@MetaFun#1{%
  \HOLOGO@mbox{Meta}%
  \HOLOGO@discretionary
  \HOLOGO@mbox{Fun}%
}
%    \end{macrocode}
%    \end{macro}
%
%    \begin{macro}{\HoLogo@MetaPost}
%    \begin{macrocode}
\def\HoLogo@MetaPost#1{%
  \HOLOGO@mbox{Meta}%
  \HOLOGO@discretionary
  \HOLOGO@mbox{Post}%
}
%    \end{macrocode}
%    \end{macro}
%
% \subsection{Others}
%
% \subsubsection{\hologo{biber}}
%
%    \begin{macro}{\HoLogo@biber}
%    \begin{macrocode}
\def\HoLogo@biber#1{%
  \HOLOGO@mbox{#1{b}{B}i}%
  \HOLOGO@discretionary
  \HOLOGO@mbox{ber}%
}
%    \end{macrocode}
%    \end{macro}
%    \begin{macro}{\HoLogoCs@biber}
%    \begin{macrocode}
\def\HoLogoCs@biber#1{#1{b}{B}iber}
%    \end{macrocode}
%    \end{macro}
%    \begin{macro}{\HoLogoBkm@biber}
%    \begin{macrocode}
\def\HoLogoBkm@biber#1{%
  #1{b}{B}iber%
}
%    \end{macrocode}
%    \end{macro}
%    \begin{macro}{\HoLogoHtml@biber}
%    \begin{macrocode}
\let\HoLogoHtml@biber\HoLogo@biber
%    \end{macrocode}
%    \end{macro}
%
% \subsubsection{\hologo{KOMAScript}}
%
%    \begin{macro}{\HoLogo@KOMAScript}
%    The definition for \hologo{KOMAScript} is taken
%    from \hologo{KOMAScript} (\xfile{scrlogo.dtx}, reformatted) \cite{scrlogo}:
%\begin{quote}
%\begin{verbatim}
%\@ifundefined{KOMAScript}{%
%  \DeclareRobustCommand{\KOMAScript}{%
%    \textsf{%
%      K\kern.05em O\kern.05emM\kern.05em A%
%      \kern.1em-\kern.1em %
%      Script%
%    }%
%  }%
%}{}
%\end{verbatim}
%\end{quote}
%    \begin{macrocode}
\def\HoLogo@KOMAScript#1{%
  \HoLogoFont@font{KOMAScript}{sf}{%
    \HOLOGO@mbox{%
      K\kern.05em%
      O\kern.05em%
      M\kern.05em%
      A%
    }%
    \kern.1em%
    \HOLOGO@hyphen
    \kern.1em%
    \HOLOGO@mbox{Script}%
  }%
}
%    \end{macrocode}
%    \end{macro}
%    \begin{macro}{\HoLogoBkm@KOMAScript}
%    \begin{macrocode}
\def\HoLogoBkm@KOMAScript#1{%
  KOMA-Script%
}
%    \end{macrocode}
%    \end{macro}
%    \begin{macro}{\HoLogoHtml@KOMAScript}
%    \begin{macrocode}
\def\HoLogoHtml@KOMAScript#1{%
  \HoLogoCss@KOMAScript
  \HoLogoFont@font{KOMAScript}{sf}{%
    \HOLOGO@Span{KOMAScript}{%
      K%
      \HOLOGO@Span{O}{O}%
      M%
      \HOLOGO@Span{A}{A}%
      \HOLOGO@Span{hyphen}{-}%
      Script%
    }%
  }%
}
%    \end{macrocode}
%    \end{macro}
%    \begin{macro}{\HoLogoCss@KOMAScript}
%    \begin{macrocode}
\def\HoLogoCss@KOMAScript{%
  \Css{%
    span.HoLogo-KOMAScript{%
      font-family:sans-serif;%
    }%
  }%
  \Css{%
    span.HoLogo-KOMAScript span.HoLogo-O{%
      padding-left:.05em;%
      padding-right:.05em;%
    }%
  }%
  \Css{%
    span.HoLogo-KOMAScript span.HoLogo-A{%
      padding-left:.05em;%
    }%
  }%
  \Css{%
    span.HoLogo-KOMAScript span.HoLogo-hyphen{%
      padding-left:.1em;%
      padding-right:.1em;%
    }%
  }%
  \global\let\HoLogoCss@KOMAScript\relax
}
%    \end{macrocode}
%    \end{macro}
%
% \subsubsection{\hologo{LyX}}
%
%    \begin{macro}{\HoLogo@LyX}
%    The definition is taken from the documentation source files
%    of \hologo{LyX}, \xfile{Intro.lyx} \cite{LyX}:
%\begin{quote}
%\begin{verbatim}
%\def\LyX{%
%  \texorpdfstring{%
%    L\kern-.1667em\lower.25em\hbox{Y}\kern-.125emX\@%
%  }{%
%    LyX%
%  }%
%}
%\end{verbatim}
%\end{quote}
%    \begin{macrocode}
\def\HoLogo@LyX#1{%
  L%
  \kern-.1667em%
  \lower.25em\hbox{Y}%
  \kern-.125em%
  X%
  \HOLOGO@SpaceFactor
}
%    \end{macrocode}
%    \end{macro}
%    \begin{macro}{\HoLogoHtml@LyX}
%    \begin{macrocode}
\def\HoLogoHtml@LyX#1{%
  \HoLogoCss@LyX
  \HOLOGO@Span{LyX}{%
    L%
    \HOLOGO@Span{y}{Y}%
    X%
  }%
}
%    \end{macrocode}
%    \end{macro}
%    \begin{macro}{\HoLogoCss@LyX}
%    \begin{macrocode}
\def\HoLogoCss@LyX{%
  \Css{%
    span.HoLogo-LyX span.HoLogo-y{%
      position:relative;%
      top:.25em;%
      margin-left:-.1667em;%
      margin-right:-.125em;%
      text-decoration:none;%
    }%
  }%
  \global\let\HoLogoCss@LyX\relax
}
%    \end{macrocode}
%    \end{macro}
%
% \subsubsection{\hologo{NTS}}
%
%    \begin{macro}{\HoLogo@NTS}
%    Definition for \hologo{NTS} can be found in
%    package \xpackage{etex\textunderscore man} for the \hologo{eTeX} manual \cite{etexman}
%    and in package \xpackage{dtklogos} \cite{dtklogos}:
%\begin{quote}
%\begin{verbatim}
%\def\NTS{%
%  \leavevmode
%  \hbox{%
%    $%
%      \cal N%
%      \kern-0.35em%
%      \lower0.5ex\hbox{$\cal T$}%
%      \kern-0.2em%
%      S%
%    $%
%  }%
%}
%\end{verbatim}
%\end{quote}
%    \begin{macrocode}
\def\HoLogo@NTS#1{%
  \HoLogoFont@font{NTS}{sy}{%
    N\/%
    \kern-.35em%
    \lower.5ex\hbox{T\/}%
    \kern-.2em%
    S\/%
  }%
  \HOLOGO@SpaceFactor
}
%    \end{macrocode}
%    \end{macro}
%
% \subsubsection{\Hologo{TTH} (\hologo{TeX} to HTML translator)}
%
%    Source: \url{http://hutchinson.belmont.ma.us/tth/}
%    In the HTML source the second `T' is printed as subscript.
%\begin{quote}
%\begin{verbatim}
%T<sub>T</sub>H
%\end{verbatim}
%\end{quote}
%    \begin{macro}{\HoLogo@TTH}
%    \begin{macrocode}
\def\HoLogo@TTH#1{%
  \ltx@mbox{%
    T\HOLOGO@SubScript{T}H%
  }%
  \HOLOGO@SpaceFactor
}
%    \end{macrocode}
%    \end{macro}
%
%    \begin{macro}{\HoLogoHtml@TTH}
%    \begin{macrocode}
\def\HoLogoHtml@TTH#1{%
  T\HCode{<sub>}T\HCode{</sub>}H%
}
%    \end{macrocode}
%    \end{macro}
%
% \subsubsection{\Hologo{HanTheThanh}}
%
%    Partial source: Package \xpackage{dtklogos}.
%    The double accent is U+1EBF (latin small letter e with circumflex
%    and acute).
%    \begin{macro}{\HoLogo@HanTheThanh}
%    \begin{macrocode}
\def\HoLogo@HanTheThanh#1{%
  \ltx@mbox{H\`an}%
  \HOLOGO@space
  \ltx@mbox{%
    Th%
    \HOLOGO@IfCharExists{"1EBF}{%
      \char"1EBF\relax
    }{%
      \^e\hbox to 0pt{\hss\raise .5ex\hbox{\'{}}}%
    }%
  }%
  \HOLOGO@space
  \ltx@mbox{Th\`anh}%
}
%    \end{macrocode}
%    \end{macro}
%    \begin{macro}{\HoLogoBkm@HanTheThanh}
%    \begin{macrocode}
\def\HoLogoBkm@HanTheThanh#1{%
  H\`an %
  Th\HOLOGO@PdfdocUnicode{\^e}{\9036\277} %
  Th\`anh%
}
%    \end{macrocode}
%    \end{macro}
%    \begin{macro}{\HoLogoHtml@HanTheThanh}
%    \begin{macrocode}
\def\HoLogoHtml@HanTheThanh#1{%
  H\`an %
  Th\HCode{&\ltx@hashchar x1ebf;} %
  Th\`anh%
}
%    \end{macrocode}
%    \end{macro}
%
% \subsection{Driver detection}
%
%    \begin{macrocode}
\HOLOGO@IfExists\InputIfFileExists{%
  \InputIfFileExists{hologo.cfg}{}{}%
}{%
  \ltx@IfUndefined{pdf@filesize}{%
    \def\HOLOGO@InputIfExists{%
      \openin\HOLOGO@temp=hologo.cfg\relax
      \ifeof\HOLOGO@temp
        \closein\HOLOGO@temp
      \else
        \closein\HOLOGO@temp
        \begingroup
          \def\x{LaTeX2e}%
        \expandafter\endgroup
        \ifx\fmtname\x
          \input{hologo.cfg}%
        \else
          \input hologo.cfg\relax
        \fi
      \fi
    }%
    \ltx@IfUndefined{newread}{%
      \chardef\HOLOGO@temp=15 %
      \def\HOLOGO@CheckRead{%
        \ifeof\HOLOGO@temp
          \HOLOGO@InputIfExists
        \else
          \ifcase\HOLOGO@temp
            \@PackageWarningNoLine{hologo}{%
              Configuration file ignored, because\MessageBreak
              a free read register could not be found%
            }%
          \else
            \begingroup
              \count\ltx@cclv=\HOLOGO@temp
              \advance\ltx@cclv by \ltx@minusone
              \edef\x{\endgroup
                \chardef\noexpand\HOLOGO@temp=\the\count\ltx@cclv
                \relax
              }%
            \x
          \fi
        \fi
      }%
    }{%
      \csname newread\endcsname\HOLOGO@temp
      \HOLOGO@InputIfExists
    }%
  }{%
    \edef\HOLOGO@temp{\pdf@filesize{hologo.cfg}}%
    \ifx\HOLOGO@temp\ltx@empty
    \else
      \ifnum\HOLOGO@temp>0 %
        \begingroup
          \def\x{LaTeX2e}%
        \expandafter\endgroup
        \ifx\fmtname\x
          \input{hologo.cfg}%
        \else
          \input hologo.cfg\relax
        \fi
      \else
        \@PackageInfoNoLine{hologo}{%
          Empty configuration file `hologo.cfg' ignored%
        }%
      \fi
    \fi
  }%
}
%    \end{macrocode}
%
%    \begin{macrocode}
\def\HOLOGO@temp#1#2{%
  \kv@define@key{HoLogoDriver}{#1}[]{%
    \begingroup
      \def\HOLOGO@temp{##1}%
      \ltx@onelevel@sanitize\HOLOGO@temp
      \ifx\HOLOGO@temp\ltx@empty
      \else
        \@PackageError{hologo}{%
          Value (\HOLOGO@temp) not permitted for option `#1'%
        }%
        \@ehc
      \fi
    \endgroup
    \def\hologoDriver{#2}%
  }%
}%
\def\HOLOGO@@temp#1#2{%
  \ifx\kv@value\relax
    \HOLOGO@temp{#1}{#1}%
  \else
    \HOLOGO@temp{#1}{#2}%
  \fi
}%
\kv@parse@normalized{%
  pdftex,%
  luatex=pdftex,%
  dvipdfm,%
  dvipdfmx=dvipdfm,%
  dvips,%
  dvipsone=dvips,%
  xdvi=dvips,%
  xetex,%
  vtex,%
}\HOLOGO@@temp
%    \end{macrocode}
%
%    \begin{macrocode}
\kv@define@key{HoLogoDriver}{driverfallback}{%
  \def\HOLOGO@DriverFallback{#1}%
}
%    \end{macrocode}
%
%    \begin{macro}{\HOLOGO@DriverFallback}
%    \begin{macrocode}
\def\HOLOGO@DriverFallback{dvips}
%    \end{macrocode}
%    \end{macro}
%
%    \begin{macro}{\hologoDriverSetup}
%    \begin{macrocode}
\def\hologoDriverSetup{%
  \let\hologoDriver\ltx@undefined
  \HOLOGO@DriverSetup
}
%    \end{macrocode}
%    \end{macro}
%
%    \begin{macro}{\HOLOGO@DriverSetup}
%    \begin{macrocode}
\def\HOLOGO@DriverSetup#1{%
  \kvsetkeys{HoLogoDriver}{#1}%
  \HOLOGO@CheckDriver
  \ltx@ifundefined{hologoDriver}{%
    \begingroup
    \edef\x{\endgroup
      \noexpand\kvsetkeys{HoLogoDriver}{\HOLOGO@DriverFallback}%
    }\x
  }{}%
  \@PackageInfoNoLine{hologo}{Using driver `\hologoDriver'}%
}
%    \end{macrocode}
%    \end{macro}
%
%    \begin{macro}{\HOLOGO@CheckDriver}
%    \begin{macrocode}
\def\HOLOGO@CheckDriver{%
  \ifpdf
    \def\hologoDriver{pdftex}%
    \let\HOLOGO@pdfliteral\pdfliteral
    \ifluatex
      \ifx\pdfextension\@undefined\else
        \protected\def\pdfliteral{\pdfextension literal}%
        \let\HOLOGO@pdfliteral\pdfliteral
      \fi
      \ltx@IfUndefined{HOLOGO@pdfliteral}{%
        \ifnum\luatexversion<36 %
        \else
          \begingroup
            \let\HOLOGO@temp\endgroup
            \ifcase0%
                \directlua{%
                  if tex.enableprimitives then %
                    tex.enableprimitives('HOLOGO@', {'pdfliteral'})%
                  else %
                    tex.print('1')%
                  end%
                }%
                \ifx\HOLOGO@pdfliteral\@undefined 1\fi%
                \relax%
              \endgroup
              \let\HOLOGO@temp\relax
              \global\let\HOLOGO@pdfliteral\HOLOGO@pdfliteral
            \fi%
          \HOLOGO@temp
        \fi
      }{}%
    \fi
    \ltx@IfUndefined{HOLOGO@pdfliteral}{%
      \@PackageWarningNoLine{hologo}{%
        Cannot find \string\pdfliteral
      }%
    }{}%
  \else
    \ifxetex
      \def\hologoDriver{xetex}%
    \else
      \ifvtex
        \def\hologoDriver{vtex}%
      \fi
    \fi
  \fi
}
%    \end{macrocode}
%    \end{macro}
%
%    \begin{macro}{\HOLOGO@WarningUnsupportedDriver}
%    \begin{macrocode}
\def\HOLOGO@WarningUnsupportedDriver#1{%
  \@PackageWarningNoLine{hologo}{%
    Logo `#1' needs driver specific macros,\MessageBreak
    but driver `\hologoDriver' is not supported.\MessageBreak
    Use a different driver or\MessageBreak
    load package `graphics' or `pgf'%
  }%
}
%    \end{macrocode}
%    \end{macro}
%
% \subsubsection{Reflect box macros}
%
%    Skip driver part if not needed.
%    \begin{macrocode}
\ltx@IfUndefined{reflectbox}{}{%
  \ltx@IfUndefined{rotatebox}{}{%
    \HOLOGO@AtEnd
  }%
}
\ltx@IfUndefined{pgftext}{}{%
  \HOLOGO@AtEnd
}
\ltx@IfUndefined{psscalebox}{}{%
  \HOLOGO@AtEnd
}
%    \end{macrocode}
%
%    \begin{macrocode}
\def\HOLOGO@temp{LaTeX2e}
\ifx\fmtname\HOLOGO@temp
  \RequirePackage{kvoptions}[2011/06/30]%
  \ProcessKeyvalOptions{HoLogoDriver}%
\fi
\HOLOGO@DriverSetup{}
%    \end{macrocode}
%
%    \begin{macro}{\HOLOGO@ReflectBox}
%    \begin{macrocode}
\def\HOLOGO@ReflectBox#1{%
  \begingroup
    \setbox\ltx@zero\hbox{\begingroup#1\endgroup}%
    \setbox\ltx@two\hbox{%
      \kern\wd\ltx@zero
      \csname HOLOGO@ScaleBox@\hologoDriver\endcsname{-1}{1}{%
        \hbox to 0pt{\copy\ltx@zero\hss}%
      }%
    }%
    \wd\ltx@two=\wd\ltx@zero
    \box\ltx@two
  \endgroup
}
%    \end{macrocode}
%    \end{macro}
%
%    \begin{macro}{\HOLOGO@PointReflectBox}
%    \begin{macrocode}
\def\HOLOGO@PointReflectBox#1{%
  \begingroup
    \setbox\ltx@zero\hbox{\begingroup#1\endgroup}%
    \setbox\ltx@two\hbox{%
      \kern\wd\ltx@zero
      \raise\ht\ltx@zero\hbox{%
        \csname HOLOGO@ScaleBox@\hologoDriver\endcsname{-1}{-1}{%
          \hbox to 0pt{\copy\ltx@zero\hss}%
        }%
      }%
    }%
    \wd\ltx@two=\wd\ltx@zero
    \box\ltx@two
  \endgroup
}
%    \end{macrocode}
%    \end{macro}
%
%    We must define all variants because of dynamic driver setup.
%    \begin{macrocode}
\def\HOLOGO@temp#1#2{#2}
%    \end{macrocode}
%
%    \begin{macro}{\HOLOGO@ScaleBox@pdftex}
%    \begin{macrocode}
\HOLOGO@temp{pdftex}{%
  \def\HOLOGO@ScaleBox@pdftex#1#2#3{%
    \HOLOGO@pdfliteral{%
      q #1 0 0 #2 0 0 cm%
    }%
    #3%
    \HOLOGO@pdfliteral{%
      Q%
    }%
  }%
}
%    \end{macrocode}
%    \end{macro}
%    \begin{macro}{\HOLOGO@ScaleBox@dvips}
%    \begin{macrocode}
\HOLOGO@temp{dvips}{%
  \def\HOLOGO@ScaleBox@dvips#1#2#3{%
    \special{ps:%
      gsave %
      currentpoint %
      currentpoint translate %
      #1 #2 scale %
      neg exch neg exch translate%
    }%
    #3%
    \special{ps:%
      currentpoint %
      grestore %
      moveto%
    }%
  }%
}
%    \end{macrocode}
%    \end{macro}
%    \begin{macro}{\HOLOGO@ScaleBox@dvipdfm}
%    \begin{macrocode}
\HOLOGO@temp{dvipdfm}{%
  \let\HOLOGO@ScaleBox@dvipdfm\HOLOGO@ScaleBox@dvips
}
%    \end{macrocode}
%    \end{macro}
%    Since \hologo{XeTeX} v0.6.
%    \begin{macro}{\HOLOGO@ScaleBox@xetex}
%    \begin{macrocode}
\HOLOGO@temp{xetex}{%
  \def\HOLOGO@ScaleBox@xetex#1#2#3{%
    \special{x:gsave}%
    \special{x:scale #1 #2}%
    #3%
    \special{x:grestore}%
  }%
}
%    \end{macrocode}
%    \end{macro}
%    \begin{macro}{\HOLOGO@ScaleBox@vtex}
%    \begin{macrocode}
\HOLOGO@temp{vtex}{%
  \def\HOLOGO@ScaleBox@vtex#1#2#3{%
    \special{r(#1,0,0,#2,0,0}%
    #3%
    \special{r)}%
  }%
}
%    \end{macrocode}
%    \end{macro}
%
%    \begin{macrocode}
\HOLOGO@AtEnd%
%</package>
%    \end{macrocode}
%
% \section{Test}
%
% \subsection{Catcode checks for loading}
%
%    \begin{macrocode}
%<*test1>
%    \end{macrocode}
%    \begin{macrocode}
\catcode`\{=1 %
\catcode`\}=2 %
\catcode`\#=6 %
\catcode`\@=11 %
\expandafter\ifx\csname count@\endcsname\relax
  \countdef\count@=255 %
\fi
\expandafter\ifx\csname @gobble\endcsname\relax
  \long\def\@gobble#1{}%
\fi
\expandafter\ifx\csname @firstofone\endcsname\relax
  \long\def\@firstofone#1{#1}%
\fi
\expandafter\ifx\csname loop\endcsname\relax
  \expandafter\@firstofone
\else
  \expandafter\@gobble
\fi
{%
  \def\loop#1\repeat{%
    \def\body{#1}%
    \iterate
  }%
  \def\iterate{%
    \body
      \let\next\iterate
    \else
      \let\next\relax
    \fi
    \next
  }%
  \let\repeat=\fi
}%
\def\RestoreCatcodes{}
\count@=0 %
\loop
  \edef\RestoreCatcodes{%
    \RestoreCatcodes
    \catcode\the\count@=\the\catcode\count@\relax
  }%
\ifnum\count@<255 %
  \advance\count@ 1 %
\repeat

\def\RangeCatcodeInvalid#1#2{%
  \count@=#1\relax
  \loop
    \catcode\count@=15 %
  \ifnum\count@<#2\relax
    \advance\count@ 1 %
  \repeat
}
\def\RangeCatcodeCheck#1#2#3{%
  \count@=#1\relax
  \loop
    \ifnum#3=\catcode\count@
    \else
      \errmessage{%
        Character \the\count@\space
        with wrong catcode \the\catcode\count@\space
        instead of \number#3%
      }%
    \fi
  \ifnum\count@<#2\relax
    \advance\count@ 1 %
  \repeat
}
\def\space{ }
\expandafter\ifx\csname LoadCommand\endcsname\relax
  \def\LoadCommand{\input hologo.sty\relax}%
\fi
\def\Test{%
  \RangeCatcodeInvalid{0}{47}%
  \RangeCatcodeInvalid{58}{64}%
  \RangeCatcodeInvalid{91}{96}%
  \RangeCatcodeInvalid{123}{255}%
  \catcode`\@=12 %
  \catcode`\\=0 %
  \catcode`\%=14 %
  \LoadCommand
  \RangeCatcodeCheck{0}{36}{15}%
  \RangeCatcodeCheck{37}{37}{14}%
  \RangeCatcodeCheck{38}{47}{15}%
  \RangeCatcodeCheck{48}{57}{12}%
  \RangeCatcodeCheck{58}{63}{15}%
  \RangeCatcodeCheck{64}{64}{12}%
  \RangeCatcodeCheck{65}{90}{11}%
  \RangeCatcodeCheck{91}{91}{15}%
  \RangeCatcodeCheck{92}{92}{0}%
  \RangeCatcodeCheck{93}{96}{15}%
  \RangeCatcodeCheck{97}{122}{11}%
  \RangeCatcodeCheck{123}{255}{15}%
  \RestoreCatcodes
}
\Test
\csname @@end\endcsname
\end
%    \end{macrocode}
%    \begin{macrocode}
%</test1>
%    \end{macrocode}
%
% \subsection{Spacefactor}
%
%    The space factor must be 1000 after a logo. If it is greater 1000
%    then the following space is a space after a sentence closing point.
%    If the space factor is smaller 1000 then an immediate following
%    dot is interpreted as abbreviation, not sentence closing point.
%
%    \begin{macrocode}
%<*test-spacefactor>
\NeedsTeXFormat{LaTeX2e}
\documentclass{article}
\usepackage{hologo}[2016/05/12]
\usepackage{kvsetkeys}
\usepackage{qstest}
\IncludeTests{*}
\LogTests{log}{*}{*}
\begin{document}
\begin{qstest}{spacefactor}{spacefactor}
\newcommand*{\Test}[1]{%
  \sbox0{%
    \hologo{#1}%
    \Expect*{1000 (#1)}*{\the\spacefactor\space(#1)}%
  }%
}%
\makeatletter
\def\TestList{}
\def\hologoEntry#1#2#3{%
  \edef\TestList{%
    \ifx\TestList\@empty
    \else
      \TestList,%
    \fi
    #1%
    \ifx\\#2\\%
    \else
      ={variant=#2}%
    \fi
  }%
}
\hologoList
\expandafter\kv@parse@normalized\expandafter{%
  \TestList
}{%
  \begingroup
    \let\@logo=\kv@key
    \ifx\kv@value\relax
    \else
      \expandafter\hologoLogoSetup\expandafter\@logo\expandafter{%
        \kv@value
      }%
    \fi
    \Test\@logo
  \endgroup
  \@gobbletwo
}
\end{qstest}
\end{document}
%</test-spacefactor>
%    \end{macrocode}
%
% \subsection{Complete list}
%
%    \begin{macrocode}
%<*test-list>
\NeedsTeXFormat{LaTeX2e}
\documentclass[12pt,a4paper]{article}
\usepackage{hologo}[2016/05/12]
\usepackage[T1]{fontenc}
\usepackage{lmodern}
\usepackage{parskip}
\usepackage[unicode]{hyperref}[2011/09/28]
\usepackage{bookmark}[2011/09/19]
\bookmarksetup{%
  numbered,%
  open,%
  openlevel=2,%
}
\renewcommand*{\contentsname}{List of logos}
\begin{document}
\tableofcontents
\def\TestFont#1#2#3#4#5#6{%
  \begingroup
    \usefont{#3}{#4}{#5}{#6}%
    \HologoVariant{#1}{#2}/\hologoVariant{#1}{#2}%
    \quad
    \begingroup\scriptsize\hologoVariant{#1}{#2}\endgroup
    \quad
  \endgroup
  (#3/#4/#5/#6)%
  \par
}
\makeatletter
\def\hologoEntry#1#2#3{%
  \section{%
    \HologoVariant{#1}{#2}/\hologoVariant{#1}{#2} %
    {[#1\ifx\\#2\\\else\space(#2)\fi]}% hash-ok
  }% braces around [] because of bug in tex4ht
  \begingroup
    \hypersetup{unicode=false}%
    \bookmark[%
      dest=\@currentHref,%
      rellevel=1,%
      keeplevel,%
    ]{%
      \HologoVariant{#1}{#2}/\hologoVariant{#1}{#2} %
      (PDFDocEncoding)%
    }%
  \endgroup
  \TestFont{#1}{#2}{OT1}{cmr}{m}{n}%
  \TestFont{#1}{#2}{OT1}{cmss}{m}{n}%
  \TestFont{#1}{#2}{OT1}{cmr}{b}{n}%
  \TestFont{#1}{#2}{OT1}{cmr}{m}{it}%
  \TestFont{#1}{#2}{OT1}{cmtt}{m}{n}%
  \TestFont{#1}{#2}{T1}{lmr}{m}{n}%
  \TestFont{#1}{#2}{T1}{lmss}{m}{n}%
  \TestFont{#1}{#2}{T1}{lmr}{b}{n}%
  \TestFont{#1}{#2}{T1}{lmr}{m}{it}%
  \TestFont{#1}{#2}{T1}{lmtt}{m}{n}%
  \TestFont{#1}{#2}{T1}{lmvtt}{m}{n}%
  \TestFont{#1}{#2}{T1}{qtm}{m}{n}%
  \TestFont{#1}{#2}{T1}{qhv}{m}{n}%
  \TestFont{#1}{#2}{T1}{qtm}{b}{n}%
  \TestFont{#1}{#2}{T1}{qtm}{m}{it}%
  \TestFont{#1}{#2}{T1}{qcr}{m}{n}%
  \newpage
}
\makeatother
\hologoList
\end{document}
%</test-list>
%    \end{macrocode}
%
% \section{Installation}
%
% \subsection{Download}
%
% \paragraph{Package.} This package is available on
% CTAN\footnote{\url{ftp://ftp.ctan.org/tex-archive/}}:
% \begin{description}
% \item[\CTAN{macros/latex/contrib/oberdiek/hologo.dtx}] The source file.
% \item[\CTAN{macros/latex/contrib/oberdiek/hologo.pdf}] Documentation.
% \end{description}
%
%
% \paragraph{Bundle.} All the packages of the bundle `oberdiek'
% are also available in a TDS compliant ZIP archive. There
% the packages are already unpacked and the documentation files
% are generated. The files and directories obey the TDS standard.
% \begin{description}
% \item[\CTAN{install/macros/latex/contrib/oberdiek.tds.zip}]
% \end{description}
% \emph{TDS} refers to the standard ``A Directory Structure
% for \TeX\ Files'' (\CTAN{tds/tds.pdf}). Directories
% with \xfile{texmf} in their name are usually organized this way.
%
% \subsection{Bundle installation}
%
% \paragraph{Unpacking.} Unpack the \xfile{oberdiek.tds.zip} in the
% TDS tree (also known as \xfile{texmf} tree) of your choice.
% Example (linux):
% \begin{quote}
%   |unzip oberdiek.tds.zip -d ~/texmf|
% \end{quote}
%
% \paragraph{Script installation.}
% Check the directory \xfile{TDS:scripts/oberdiek/} for
% scripts that need further installation steps.
% Package \xpackage{attachfile2} comes with the Perl script
% \xfile{pdfatfi.pl} that should be installed in such a way
% that it can be called as \texttt{pdfatfi}.
% Example (linux):
% \begin{quote}
%   |chmod +x scripts/oberdiek/pdfatfi.pl|\\
%   |cp scripts/oberdiek/pdfatfi.pl /usr/local/bin/|
% \end{quote}
%
% \subsection{Package installation}
%
% \paragraph{Unpacking.} The \xfile{.dtx} file is a self-extracting
% \docstrip\ archive. The files are extracted by running the
% \xfile{.dtx} through \plainTeX:
% \begin{quote}
%   \verb|tex hologo.dtx|
% \end{quote}
%
% \paragraph{TDS.} Now the different files must be moved into
% the different directories in your installation TDS tree
% (also known as \xfile{texmf} tree):
% \begin{quote}
% \def\t{^^A
% \begin{tabular}{@{}>{\ttfamily}l@{ $\rightarrow$ }>{\ttfamily}l@{}}
%   hologo.sty & tex/generic/oberdiek/hologo.sty\\
%   hologo.pdf & doc/latex/oberdiek/hologo.pdf\\
%   example/hologo-example.tex & doc/latex/oberdiek/example/hologo-example.tex\\
%   test/hologo-test1.tex & doc/latex/oberdiek/test/hologo-test1.tex\\
%   test/hologo-test-spacefactor.tex & doc/latex/oberdiek/test/hologo-test-spacefactor.tex\\
%   test/hologo-test-list.tex & doc/latex/oberdiek/test/hologo-test-list.tex\\
%   hologo.dtx & source/latex/oberdiek/hologo.dtx\\
% \end{tabular}^^A
% }^^A
% \sbox0{\t}^^A
% \ifdim\wd0>\linewidth
%   \begingroup
%     \advance\linewidth by\leftmargin
%     \advance\linewidth by\rightmargin
%   \edef\x{\endgroup
%     \def\noexpand\lw{\the\linewidth}^^A
%   }\x
%   \def\lwbox{^^A
%     \leavevmode
%     \hbox to \linewidth{^^A
%       \kern-\leftmargin\relax
%       \hss
%       \usebox0
%       \hss
%       \kern-\rightmargin\relax
%     }^^A
%   }^^A
%   \ifdim\wd0>\lw
%     \sbox0{\small\t}^^A
%     \ifdim\wd0>\linewidth
%       \ifdim\wd0>\lw
%         \sbox0{\footnotesize\t}^^A
%         \ifdim\wd0>\linewidth
%           \ifdim\wd0>\lw
%             \sbox0{\scriptsize\t}^^A
%             \ifdim\wd0>\linewidth
%               \ifdim\wd0>\lw
%                 \sbox0{\tiny\t}^^A
%                 \ifdim\wd0>\linewidth
%                   \lwbox
%                 \else
%                   \usebox0
%                 \fi
%               \else
%                 \lwbox
%               \fi
%             \else
%               \usebox0
%             \fi
%           \else
%             \lwbox
%           \fi
%         \else
%           \usebox0
%         \fi
%       \else
%         \lwbox
%       \fi
%     \else
%       \usebox0
%     \fi
%   \else
%     \lwbox
%   \fi
% \else
%   \usebox0
% \fi
% \end{quote}
% If you have a \xfile{docstrip.cfg} that configures and enables \docstrip's
% TDS installing feature, then some files can already be in the right
% place, see the documentation of \docstrip.
%
% \subsection{Refresh file name databases}
%
% If your \TeX~distribution
% (\teTeX, \mikTeX, \dots) relies on file name databases, you must refresh
% these. For example, \teTeX\ users run \verb|texhash| or
% \verb|mktexlsr|.
%
% \subsection{Some details for the interested}
%
% \paragraph{Attached source.}
%
% The PDF documentation on CTAN also includes the
% \xfile{.dtx} source file. It can be extracted by
% AcrobatReader 6 or higher. Another option is \textsf{pdftk},
% e.g. unpack the file into the current directory:
% \begin{quote}
%   \verb|pdftk hologo.pdf unpack_files output .|
% \end{quote}
%
% \paragraph{Unpacking with \LaTeX.}
% The \xfile{.dtx} chooses its action depending on the format:
% \begin{description}
% \item[\plainTeX:] Run \docstrip\ and extract the files.
% \item[\LaTeX:] Generate the documentation.
% \end{description}
% If you insist on using \LaTeX\ for \docstrip\ (really,
% \docstrip\ does not need \LaTeX), then inform the autodetect routine
% about your intention:
% \begin{quote}
%   \verb|latex \let\install=y\input{hologo.dtx}|
% \end{quote}
% Do not forget to quote the argument according to the demands
% of your shell.
%
% \paragraph{Generating the documentation.}
% You can use both the \xfile{.dtx} or the \xfile{.drv} to generate
% the documentation. The process can be configured by the
% configuration file \xfile{ltxdoc.cfg}. For instance, put this
% line into this file, if you want to have A4 as paper format:
% \begin{quote}
%   \verb|\PassOptionsToClass{a4paper}{article}|
% \end{quote}
% An example follows how to generate the
% documentation with pdf\LaTeX:
% \begin{quote}
%\begin{verbatim}
%pdflatex hologo.dtx
%makeindex -s gind.ist hologo.idx
%pdflatex hologo.dtx
%makeindex -s gind.ist hologo.idx
%pdflatex hologo.dtx
%\end{verbatim}
% \end{quote}
%
% \section{Catalogue}
%
% The following XML file can be used as source for the
% \href{http://mirror.ctan.org/help/Catalogue/catalogue.html}{\TeX\ Catalogue}.
% The elements \texttt{caption} and \texttt{description} are imported
% from the original XML file from the Catalogue.
% The name of the XML file in the Catalogue is \xfile{hologo.xml}.
%    \begin{macrocode}
%<*catalogue>
<?xml version='1.0' encoding='us-ascii'?>
<!DOCTYPE entry SYSTEM 'catalogue.dtd'>
<entry datestamp='$Date$' modifier='$Author$' id='hologo'>
  <name>hologo</name>
  <caption>A collection of logos with bookmark support.</caption>
  <authorref id='auth:oberdiek'/>
  <copyright owner='Heiko Oberdiek' year='2010-2012'/>
  <license type='lppl1.3'/>
  <version number='1.10'/>
  <description>
    The package defines a single command <tt>\hologo</tt>, whose
    argument is the usual case-confused ASCII version of the logo.
    The command is bookmark-enabled, so that every logo becomes
    available in bookmarks without further work.
    <p/>
    The package is part of the <xref refid='oberdiek'>oberdiek</xref>
    bundle.
  </description>
  <documentation details='Package documentation'
      href='ctan:/macros/latex/contrib/oberdiek/hologo.pdf'/>
  <ctan file='true' path='/macros/latex/contrib/oberdiek/hologo.dtx'/>
  <miktex location='oberdiek'/>
  <texlive location='oberdiek'/>
  <install path='/macros/latex/contrib/oberdiek/oberdiek.tds.zip'/>
</entry>
%</catalogue>
%    \end{macrocode}
%
% \begin{thebibliography}{9}
% \raggedright
%
% \bibitem{btxdoc}
% Oren Patashnik,
% \textit{\hologo{BibTeX}ing},
% 1988-02-08.\\
% \CTAN{biblio/bibtex/base/}
%
% \bibitem{dtklogos}
% Gerd Neugebauer, DANTE,
% \textit{Package \xpackage{dtklogos}},
% 2011-04-25.\\
% \CTAN{usergrps/dante/dtk/dtklogos.sty}
%
% \bibitem{etexman}
% The \hologo{NTS} Team,
% \textit{The \hologo{eTeX} manual},
% 1998-02.\\
% \CTAN{systems/e-tex/v2/doc/}
%
% \bibitem{ExTeX-FAQ}
% The \hologo{ExTeX} group,
% \textit{\hologo{ExTeX}: FAQ -- How is \hologo{ExTeX} typeset?},
% 2007-04-14.\\
% \url{http://www.extex.org/documentation/faq.html}
%
% \bibitem{LyX}
% %@MISC{ LyX,
% %  title = {{LyX 2.0.0 -- The Document Processor [Computer software and manual]}},
% %  author = {{The LyX Team}},
% %  howpublished = {Internet: http://www.lyx.org},
% %  year = {2011-05-08},
% %  note = {Retrieved May 10, 2011, from http://www.lyx.org},
% %  url = {http://www.lyx.org/}
% %}
% The \hologo{LyX} Team,
% \textit{\hologo{LyX} -- The Document Processor},
% 2011-05-08.\\
% \url{http://www.lyx.org/}
%
% \bibitem{OzTeX}
% Andrew Trevorrow,
% \hologo{OzTeX} FAQ: What is the correct way to typeset ``\hologo{OzTeX}''?,
% 2011-09-15 (visited).
% \url{http://www.trevorrow.com/oztex/ozfaq.html#oztex-logo}
%
% \bibitem{PiCTeX}
% Michael Wichura,
% \textit{The \hologo{PiCTeX} macro package},
% 1987-09-21.
% \CTAN{graphics/pictex/}
%
% \bibitem{scrlogo}
% Markus Kohm,
% \textit{\hologo{KOMAScript} Datei \xfile{scrlogo.dtx}},
% 2009-01-30.\\
% \CTAN{install/macros/latex/contrib/komascript.tds.zip}
%
% \end{thebibliography}
%
% \begin{History}
%   \begin{Version}{2010/04/08 v1.0}
%   \item
%     The first version.
%   \end{Version}
%   \begin{Version}{2010/04/16 v1.1}
%   \item
%     \cs{Hologo} added for support of logos at start of a sentence.
%   \item
%     \cs{hologoSetup} and \cs{hologoLogoSetup} added.
%   \item
%     Options \xoption{break}, \xoption{hyphenbreak}, \xoption{spacebreak}
%     added.
%   \item
%     Variant support added by option \xoption{variant}.
%   \end{Version}
%   \begin{Version}{2010/04/24 v1.2}
%   \item
%     \hologo{LaTeX3} added.
%   \item
%     \hologo{VTeX} added.
%   \end{Version}
%   \begin{Version}{2010/11/21 v1.3}
%   \item
%     \hologo{iniTeX}, \hologo{virTeX} added.
%   \end{Version}
%   \begin{Version}{2011/03/25 v1.4}
%   \item
%     \hologo{ConTeXt} with variants added.
%   \item
%     Option \xoption{discretionarybreak} added as refinement for
%     option \xoption{break}.
%   \end{Version}
%   \begin{Version}{2011/04/21 v1.5}
%   \item
%     Wrong TDS directory for test files fixed.
%   \end{Version}
%   \begin{Version}{2011/10/01 v1.6}
%   \item
%     Support for package \xpackage{tex4ht} added.
%   \item
%     Support for \cs{csname} added if \cs{ifincsname} is available.
%   \item
%     New logos:
%     \hologo{(La)TeX},
%     \hologo{biber},
%     \hologo{BibTeX} (\xoption{sc}, \xoption{sf}),
%     \hologo{emTeX},
%     \hologo{ExTeX},
%     \hologo{KOMAScript},
%     \hologo{La},
%     \hologo{LyX},
%     \hologo{MiKTeX},
%     \hologo{NTS},
%     \hologo{OzMF},
%     \hologo{OzMP},
%     \hologo{OzTeX},
%     \hologo{OzTtH},
%     \hologo{PCTeX},
%     \hologo{PiC},
%     \hologo{PiCTeX},
%     \hologo{METAFONT},
%     \hologo{MetaFun},
%     \hologo{METAPOST},
%     \hologo{MetaPost},
%     \hologo{SLiTeX} (\xoption{lift}, \xoption{narrow}, \xoption{simple}),
%     \hologo{SliTeX} (\xoption{narrow}, \xoption{simple}, \xoption{lift}),
%     \hologo{teTeX}.
%   \item
%     Fixes:
%     \hologo{iniTeX},
%     \hologo{pdfLaTeX},
%     \hologo{pdfTeX},
%     \hologo{virTeX}.
%   \item
%     \cs{hologoFontSetup} and \cs{hologoLogoFontSetup} added.
%   \item
%     \cs{hologoVariant} and \cs{HologoVariant} added.
%   \end{Version}
%   \begin{Version}{2011/11/22 v1.7}
%   \item
%     New logos:
%     \hologo{BibTeX8},
%     \hologo{LaTeXML},
%     \hologo{SageTeX},
%     \hologo{TeX4ht},
%     \hologo{TTH}.
%   \item
%     \hologo{Xe} and friends: Driver stuff fixed.
%   \item
%     \hologo{Xe} and friends: Support for italic added.
%   \item
%     \hologo{Xe} and friends: Package support for \xpackage{pgf}
%     and \xpackage{pstricks} added.
%   \end{Version}
%   \begin{Version}{2011/11/29 v1.8}
%   \item
%     New logos:
%     \hologo{HanTheThanh}.
%   \end{Version}
%   \begin{Version}{2011/12/21 v1.9}
%   \item
%     Patch for package \xpackage{ifxetex} added for the case that
%     \cs{newif} is undefined in \hologo{iniTeX}.
%   \item
%     Some fixes for \hologo{iniTeX}.
%   \end{Version}
%   \begin{Version}{2012/04/26 v1.10}
%   \item
%     Fix in bookmark version of logo ``\hologo{HanTheThanh}''.
%   \end{Version}
%   \begin{Version}{2016/05/12 v1.11}
%   \item
%     Update HOLOGO@IfCharExists (previously in texlive)
%   \item define pdfliteral in current luatex.
%   \end{Version}
% \end{History}
%
% \PrintIndex
%
% \Finale
\endinput

%        (quote the arguments according to the demands of your shell)
%
% Documentation:
%    (a) If hologo.drv is present:
%           latex hologo.drv
%    (b) Without hologo.drv:
%           latex hologo.dtx; ...
%    The class ltxdoc loads the configuration file ltxdoc.cfg
%    if available. Here you can specify further options, e.g.
%    use A4 as paper format:
%       \PassOptionsToClass{a4paper}{article}
%
%    Programm calls to get the documentation (example):
%       pdflatex hologo.dtx
%       makeindex -s gind.ist hologo.idx
%       pdflatex hologo.dtx
%       makeindex -s gind.ist hologo.idx
%       pdflatex hologo.dtx
%
% Installation:
%    TDS:tex/generic/oberdiek/hologo.sty
%    TDS:doc/latex/oberdiek/hologo.pdf
%    TDS:doc/latex/oberdiek/example/hologo-example.tex
%    TDS:doc/latex/oberdiek/test/hologo-test1.tex
%    TDS:doc/latex/oberdiek/test/hologo-test-spacefactor.tex
%    TDS:doc/latex/oberdiek/test/hologo-test-list.tex
%    TDS:source/latex/oberdiek/hologo.dtx
%
%<*ignore>
\begingroup
  \catcode123=1 %
  \catcode125=2 %
  \def\x{LaTeX2e}%
\expandafter\endgroup
\ifcase 0\ifx\install y1\fi\expandafter
         \ifx\csname processbatchFile\endcsname\relax\else1\fi
         \ifx\fmtname\x\else 1\fi\relax
\else\csname fi\endcsname
%</ignore>
%<*install>
\input docstrip.tex
\Msg{************************************************************************}
\Msg{* Installation}
\Msg{* Package: hologo 2016/05/12 v1.11 A logo collection with bookmark support (HO)}
\Msg{************************************************************************}

\keepsilent
\askforoverwritefalse

\let\MetaPrefix\relax
\preamble

This is a generated file.

Project: hologo
Version: 2016/05/12 v1.11

Copyright (C) 2010-2012 by
   Heiko Oberdiek <heiko.oberdiek at googlemail.com>

This work may be distributed and/or modified under the
conditions of the LaTeX Project Public License, either
version 1.3c of this license or (at your option) any later
version. This version of this license is in
   http://www.latex-project.org/lppl/lppl-1-3c.txt
and the latest version of this license is in
   http://www.latex-project.org/lppl.txt
and version 1.3 or later is part of all distributions of
LaTeX version 2005/12/01 or later.

This work has the LPPL maintenance status "maintained".

This Current Maintainer of this work is Heiko Oberdiek.

The Base Interpreter refers to any `TeX-Format',
because some files are installed in TDS:tex/generic//.

This work consists of the main source file hologo.dtx
and the derived files
   hologo.sty, hologo.pdf, hologo.ins, hologo.drv, hologo-example.tex,
   hologo-test1.tex, hologo-test-spacefactor.tex,
   hologo-test-list.tex.

\endpreamble
\let\MetaPrefix\DoubleperCent

\generate{%
  \file{hologo.ins}{\from{hologo.dtx}{install}}%
  \file{hologo.drv}{\from{hologo.dtx}{driver}}%
  \usedir{tex/generic/oberdiek}%
  \file{hologo.sty}{\from{hologo.dtx}{package}}%
  \usedir{doc/latex/oberdiek/example}%
  \file{hologo-example.tex}{\from{hologo.dtx}{example}}%
  \usedir{doc/latex/oberdiek/test}%
  \file{hologo-test1.tex}{\from{hologo.dtx}{test1}}%
  \file{hologo-test-spacefactor.tex}{\from{hologo.dtx}{test-spacefactor}}%
  \file{hologo-test-list.tex}{\from{hologo.dtx}{test-list}}%
  \nopreamble
  \nopostamble
  \usedir{source/latex/oberdiek/catalogue}%
  \file{hologo.xml}{\from{hologo.dtx}{catalogue}}%
}

\catcode32=13\relax% active space
\let =\space%
\Msg{************************************************************************}
\Msg{*}
\Msg{* To finish the installation you have to move the following}
\Msg{* file into a directory searched by TeX:}
\Msg{*}
\Msg{*     hologo.sty}
\Msg{*}
\Msg{* To produce the documentation run the file `hologo.drv'}
\Msg{* through LaTeX.}
\Msg{*}
\Msg{* Happy TeXing!}
\Msg{*}
\Msg{************************************************************************}

\endbatchfile
%</install>
%<*ignore>
\fi
%</ignore>
%<*driver>
\NeedsTeXFormat{LaTeX2e}
\ProvidesFile{hologo.drv}%
  [2016/05/12 v1.11 A logo collection with bookmark support (HO)]%
\documentclass{ltxdoc}
\usepackage{holtxdoc}[2011/11/22]
\usepackage{hologo}[2016/05/12]
\usepackage{longtable}
\usepackage{array}
\usepackage{paralist}
%\usepackage[T1]{fontenc}
%\usepackage{lmodern}
\begin{document}
  \DocInput{hologo.dtx}%
\end{document}
%</driver>
% \fi
%
%
% \CharacterTable
%  {Upper-case    \A\B\C\D\E\F\G\H\I\J\K\L\M\N\O\P\Q\R\S\T\U\V\W\X\Y\Z
%   Lower-case    \a\b\c\d\e\f\g\h\i\j\k\l\m\n\o\p\q\r\s\t\u\v\w\x\y\z
%   Digits        \0\1\2\3\4\5\6\7\8\9
%   Exclamation   \!     Double quote  \"     Hash (number) \#
%   Dollar        \$     Percent       \%     Ampersand     \&
%   Acute accent  \'     Left paren    \(     Right paren   \)
%   Asterisk      \*     Plus          \+     Comma         \,
%   Minus         \-     Point         \.     Solidus       \/
%   Colon         \:     Semicolon     \;     Less than     \<
%   Equals        \=     Greater than  \>     Question mark \?
%   Commercial at \@     Left bracket  \[     Backslash     \\
%   Right bracket \]     Circumflex    \^     Underscore    \_
%   Grave accent  \`     Left brace    \{     Vertical bar  \|
%   Right brace   \}     Tilde         \~}
%
% \GetFileInfo{hologo.drv}
%
% \title{The \xpackage{hologo} package}
% \date{2016/05/12 v1.11}
% \author{Heiko Oberdiek\\\xemail{heiko.oberdiek at googlemail.com}}
%
% \maketitle
%
% \begin{abstract}
% This package starts a collection of logos with support for bookmarks
% strings.
% \end{abstract}
%
% \tableofcontents
%
% \section{Documentation}
%
% \subsection{Logo macros}
%
% \begin{declcs}{hologo} \M{name}
% \end{declcs}
% Macro \cs{hologo} sets the logo with name \meta{name}.
% The following table shows the supported names.
%
% \begingroup
%   \def\hologoEntry#1#2#3{^^A
%     #1&#2&\hologoLogoSetup{#1}{variant=#2}\hologo{#1}&#3\tabularnewline
%   }
%   \begin{longtable}{>{\ttfamily}l>{\ttfamily}lll}
%     \rmfamily\bfseries{name} & \rmfamily\bfseries variant
%     & \bfseries logo & \bfseries since\\
%     \hline
%     \endhead
%     \hologoList
%   \end{longtable}
% \endgroup
%
% \begin{declcs}{Hologo} \M{name}
% \end{declcs}
% Macro \cs{Hologo} starts the logo \meta{name} with an uppercase
% letter. As an exception small greek letters are not converted
% to uppercase. Examples, see \hologo{eTeX} and \hologo{ExTeX}.
%
% \subsection{Setup macros}
%
% The package does not support package options, but the following
% setup macros can be used to set options.
%
% \begin{declcs}{hologoSetup} \M{key value list}
% \end{declcs}
% Macro \cs{hologoSetup} sets global options.
%
% \begin{declcs}{hologoLogoSetup} \M{logo} \M{key value list}
% \end{declcs}
% Some options can also be used to configure a logo.
% These settings take precedence over global option settings.
%
% \subsection{Options}\label{sec:options}
%
% There are boolean and string options:
% \begin{description}
% \item[Boolean option:]
% It takes |true| or |false|
% as value. If the value is omitted, then |true| is used.
% \item[String option:]
% A value must be given as string. (But the string might be empty.)
% \end{description}
% The following options can be used both in \cs{hologoSetup}
% and \cs{hologoLogoSetup}:
% \begin{description}
% \def\entry#1{\item[\xoption{#1}:]}
% \entry{break}
%   enables or disables line breaks inside the logo. This setting is
%   refined by options \xoption{hyphenbreak}, \xoption{spacebreak}
%   or \xoption{discretionarybreak}.
%   Default is |false|.
% \entry{hyphenbreak}
%   enables or disables the line break right after the hyphen character.
% \entry{spacebreak}
%   enables or disables line breaks at space characters.
% \entry{discretionarybreak}
%   enables or disables line breaks at hyphenation points
%   (inserted by \cs{-}).
% \end{description}
% Macro \cs{hologoLogoSetup} also knows:
% \begin{description}
% \item[\xoption{variant}:]
%   This is a string option. It specifies a variant of a logo that
%   must exist. An empty string selects the package default variant.
% \end{description}
% Example:
% \begin{quote}
%   |\hologoSetup{break=false}|\\
%   |\hologoLogoSetup{plainTeX}{variant=hyphen,hyphenbreak}|\\
%   Then ``plain-\TeX'' contains one break point after the hyphen.
% \end{quote}
%
% \subsection{Driver options}
%
% Sometimes graphical operations are needed to construct some
% glyphs (e.g.\ \hologo{XeTeX}). If package \xpackage{graphics}
% or package \xpackage{pgf} are found, then the macros are taken
% from there. Otherwise the packge defines its own operations
% and therefore needs the driver information. Many drivers are
% detected automatically (\hologo{pdfTeX}/\hologo{LuaTeX}
% in PDF mode, \hologo{XeTeX}, \hologo{VTeX}). These have precedence
% over a driver option. The driver can be given as package option
% or using \cs{hologoDriverSetup}.
% The following list contains the recognized driver options:
% \begin{itemize}
% \item \xoption{pdftex}, \xoption{luatex}
% \item \xoption{dvipdfm}, \xoption{dvipdfmx}
% \item \xoption{dvips}, \xoption{dvipsone}, \xoption{xdvi}
% \item \xoption{xetex}
% \item \xoption{vtex}
% \end{itemize}
% The left driver of a line is the driver name that is used internally.
% The following names are aliases for drivers that use the
% same method. Therefore the entry in the \xext{log} file for
% the used driver prints the internally used driver name.
% \begin{description}
% \item[\xoption{driverfallback}:]
%   This option expects a driver that is used,
%   if the driver could not be detected automatically.
% \end{description}
%
% \begin{declcs}{hologoDriverSetup} \M{driver option}
% \end{declcs}
% The driver can also be configured after package loading
% using \cs{hologoDriverSetup}, also the way for \hologo{plainTeX}
% to setup the driver.
%
% \subsection{Font setup}
%
% Some logos require a special font, but should also be usable by
% \hologo{plainTeX}. Therefore the package provides some ways
% to influence the font settings. The options below
% take font settings as values. Both font commands
% such as \cs{sffamily} and macros that take one argument
% like \cs{textsf} can be used.
%
% \begin{declcs}{hologoFontSetup} \M{key value list}
% \end{declcs}
% Macro \cs{hologoFontSetup} sets the fonts for all logos.
% Supported keys:
% \begin{description}
% \def\entry#1{\item[\xoption{#1}:]}
% \entry{general}
%   This font is used for all logos. The default is empty.
%   That means no special font is used.
% \entry{bibsf}
%   This font is used for
%   {\hologoLogoSetup{BibTeX}{variant=sf}\hologo{BibTeX}}
%   with variant \xoption{sf}.
% \entry{rm}
%   This font is a serif font. It is used for \hologo{ExTeX}.
% \entry{sc}
%   This font specifies a small caps font. It is used for
%   {\hologoLogoSetup{BibTeX}{variant=sc}\hologo{BibTeX}}
%   with variant \xoption{sc}.
% \entry{sf}
%   This font specifies a sans serif font. The default
%   is \cs{sffamily}, then \cs{sf} is tried. Otherwise
%   a warning is given. It is used by \hologo{KOMAScript}.
% \entry{sy}
%   This is the font for math symbols (e.g. cmsy).
%   It is used by \hologo{AmS}, \hologo{NTS}, \hologo{ExTeX}.
% \entry{logo}
%   \hologo{METAFONT} and \hologo{METAPOST} are using that font.
%   In \hologo{LaTeX} \cs{logofamily} is used and
%   the definitions of package \xpackage{mflogo} are used
%   if the package is not loaded.
%   Otherwise the \cs{tenlogo} is used and defined
%   if it does not already exists.
% \end{description}
%
% \begin{declcs}{hologoLogoFontSetup} \M{logo} \M{key value list}
% \end{declcs}
% Fonts can also be set for a logo or logo component separately,
% see the following list.
% The keys are the same as for \cs{hologoFontSetup}.
%
% \begin{longtable}{>{\ttfamily}l>{\sffamily}ll}
%   \meta{logo} & keys & result\\
%   \hline
%   \endhead
%   BibTeX & bibsf & {\hologoLogoSetup{BibTeX}{variant=sf}\hologo{BibTeX}}\\[.5ex]
%   BibTeX & sc & {\hologoLogoSetup{BibTeX}{variant=sc}\hologo{BibTeX}}\\[.5ex]
%   ExTeX & rm & \hologo{ExTeX}\\
%   SliTeX & rm & \hologo{SliTeX}\\[.5ex]
%   AmS & sy & \hologo{AmS}\\
%   ExTeX & sy & \hologo{ExTeX}\\
%   NTS & sy & \hologo{NTS}\\[.5ex]
%   KOMAScript & sf & \hologo{KOMAScript}\\[.5ex]
%   METAFONT & logo & \hologo{METAFONT}\\
%   METAPOST & logo & \hologo{METAPOST}\\[.5ex]
%   SliTeX & sc \hologo{SliTeX}
% \end{longtable}
%
% \subsubsection{Font order}
%
% For all logos the font \xoption{general} is applied first.
% Example:
%\begin{quote}
%|\hologoFontSetup{general=\color{red}}|
%\end{quote}
% will print red logos.
% Then if the font uses a special font \xoption{sf}, for example,
% the font is applied that is setup by \cs{hologoLogoFontSetup}.
% If this font is not setup, then the common font setup
% by \cs{hologoFontSetup} is used. Otherwise a warning is given,
% that there is no font configured.
%
% \subsection{Additional user macros}
%
% Usually a variant of a logo is configured by using
% \cs{hologoLogoSetup}, because it is bad style to mix
% different variants of the same logo in the same text.
% There the following macros are a convenience for testing.
%
% \begin{declcs}{hologoVariant} \M{name} \M{variant}\\
%   \cs{HologoVariant} \M{name} \M{variant}
% \end{declcs}
% Logo \meta{name} is set using \meta{variant} that specifies
% explicitely which variant of the macro is used. If the argument
% is empty, then the default form of the logo is used
% (configurable by \cs{hologoLogoSetup}).
%
% \cs{HologoVariant} is used if the logo is set in a context
% that needs an uppercase first letter (beginning of a sentence, \dots).
%
% \begin{declcs}{hologoList}\\
%   \cs{hologoEntry} \M{logo} \M{variant} \M{since}
% \end{declcs}
% Macro \cs{hologoList} contains all logos that are provided
% by the package including variants. The list consists of calls
% of \cs{hologoEntry} with three arguments starting with the
% logo name \meta{logo} and its variant \meta{variant}. An empty
% variant means the current default. Argument \meta{since} specifies
% with version of the package \xpackage{hologo} is needed to get
% the logo. If the logo is fixed, then the date gets updated.
% Therefore the date \meta{since} is not exactly the date of
% the first introduction, but rather the date of the latest fix.
%
% Before \cs{hologoList} can be used, macro \cs{hologoEntry} needs
% a definition. The example file in section \ref{sec:example}
% shows applications of \cs{hologoList}.
%
% \subsection{Supported contexts}
%
% Macros \cs{hologo} and friends support special contexts:
% \begin{itemize}
% \item \hologo{LaTeX}'s protection mechanism.
% \item Bookmarks of package \xpackage{hyperref}.
% \item Package \xpackage{tex4ht}.
% \item The macros can be used inside \cs{csname} constructs,
%   if \cs{ifincsname} is available (\hologo{pdfTeX}, \hologo{XeTeX},
%   \hologo{LuaTeX}).
% \end{itemize}
%
% \subsection{Example}
% \label{sec:example}
%
% The following example prints the logos in different fonts.
%    \begin{macrocode}
%<*example>
%<<verbatim
\NeedsTeXFormat{LaTeX2e}
\documentclass[a4paper]{article}
\usepackage[
  hmargin=20mm,
  vmargin=20mm,
]{geometry}
\pagestyle{empty}
\usepackage{hologo}[2016/05/12]
\usepackage{longtable}
\usepackage{array}
\setlength{\extrarowheight}{2pt}
\usepackage[T1]{fontenc}
\usepackage{lmodern}
\usepackage{pdflscape}
\usepackage[
  pdfencoding=auto,
]{hyperref}
\hypersetup{
  pdfauthor={Heiko Oberdiek},
  pdftitle={Example for package `hologo'},
  pdfsubject={Logos with fonts lmr, lmss, qtm, qpl, qhv},
}
\usepackage{bookmark}

% Print the logo list on the console

\begingroup
  \typeout{}%
  \typeout{*** Begin of logo list ***}%
  \newcommand*{\hologoEntry}[3]{%
    \typeout{#1 \ifx\\#2\\\else(#2) \fi[#3]}%
  }%
  \hologoList
  \typeout{*** End of logo list ***}%
  \typeout{}%
\endgroup

\begin{document}
\begin{landscape}

  \section{Example file for package `hologo'}

  % Table for font names

  \begin{longtable}{>{\bfseries}ll}
    \textbf{font} & \textbf{Font name}\\
    \hline
    lmr & Latin Modern Roman\\
    lmss & Latin Modern Sans\\
    qtm & \TeX\ Gyre Termes\\
    qhv & \TeX\ Gyre Heros\\
    qpl & \TeX\ Gyre Pagella\\
  \end{longtable}

  % Logo list with logos in different fonts

  \begingroup
    \newcommand*{\SetVariant}[2]{%
      \ifx\\#2\\%
      \else
        \hologoLogoSetup{#1}{variant=#2}%
      \fi
    }%
    \newcommand*{\hologoEntry}[3]{%
      \SetVariant{#1}{#2}%
      \raisebox{1em}[0pt][0pt]{\hypertarget{#1@#2}{}}%
      \bookmark[%
        dest={#1@#2},%
      ]{%
        #1\ifx\\#2\\\else\space(#2)\fi: \Hologo{#1}, \hologo{#1} %
        [Unicode]%
      }%
      \hypersetup{unicode=false}%
      \bookmark[%
        dest={#1@#2},%
      ]{%
        #1\ifx\\#2\\\else\space(#2)\fi: \Hologo{#1}, \hologo{#1} %
        [PDFDocEncoding]%
      }%
      \texttt{#1}%
      &%
      \texttt{#2}%
      &%
      \Hologo{#1}%
      &%
      \SetVariant{#1}{#2}%
      \hologo{#1}%
      &%
      \SetVariant{#1}{#2}%
      \fontfamily{qtm}\selectfont
      \hologo{#1}%
      &%
      \SetVariant{#1}{#2}%
      \fontfamily{qpl}\selectfont
      \hologo{#1}%
      &%
      \SetVariant{#1}{#2}%
      \textsf{\hologo{#1}}%
      &%
      \SetVariant{#1}{#2}%
      \fontfamily{qhv}\selectfont
      \hologo{#1}%
      \tabularnewline
    }%
    \begin{longtable}{llllllll}%
      \textbf{\textit{logo}} & \textbf{\textit{variant}} &
      \texttt{\string\Hologo} &
      \textbf{lmr} & \textbf{qtm} & \textbf{qpl} &
      \textbf{lmss} & \textbf{qhv}
      \tabularnewline
      \hline
      \endhead
      \hologoList
    \end{longtable}%
  \endgroup

\end{landscape}
\end{document}
%verbatim
%</example>
%    \end{macrocode}
%
% \StopEventually{
% }
%
% \section{Implementation}
%    \begin{macrocode}
%<*package>
%    \end{macrocode}
%    Reload check, especially if the package is not used with \LaTeX.
%    \begin{macrocode}
\begingroup\catcode61\catcode48\catcode32=10\relax%
  \catcode13=5 % ^^M
  \endlinechar=13 %
  \catcode35=6 % #
  \catcode39=12 % '
  \catcode44=12 % ,
  \catcode45=12 % -
  \catcode46=12 % .
  \catcode58=12 % :
  \catcode64=11 % @
  \catcode123=1 % {
  \catcode125=2 % }
  \expandafter\let\expandafter\x\csname ver@hologo.sty\endcsname
  \ifx\x\relax % plain-TeX, first loading
  \else
    \def\empty{}%
    \ifx\x\empty % LaTeX, first loading,
      % variable is initialized, but \ProvidesPackage not yet seen
    \else
      \expandafter\ifx\csname PackageInfo\endcsname\relax
        \def\x#1#2{%
          \immediate\write-1{Package #1 Info: #2.}%
        }%
      \else
        \def\x#1#2{\PackageInfo{#1}{#2, stopped}}%
      \fi
      \x{hologo}{The package is already loaded}%
      \aftergroup\endinput
    \fi
  \fi
\endgroup%
%    \end{macrocode}
%    Package identification:
%    \begin{macrocode}
\begingroup\catcode61\catcode48\catcode32=10\relax%
  \catcode13=5 % ^^M
  \endlinechar=13 %
  \catcode35=6 % #
  \catcode39=12 % '
  \catcode40=12 % (
  \catcode41=12 % )
  \catcode44=12 % ,
  \catcode45=12 % -
  \catcode46=12 % .
  \catcode47=12 % /
  \catcode58=12 % :
  \catcode64=11 % @
  \catcode91=12 % [
  \catcode93=12 % ]
  \catcode123=1 % {
  \catcode125=2 % }
  \expandafter\ifx\csname ProvidesPackage\endcsname\relax
    \def\x#1#2#3[#4]{\endgroup
      \immediate\write-1{Package: #3 #4}%
      \xdef#1{#4}%
    }%
  \else
    \def\x#1#2[#3]{\endgroup
      #2[{#3}]%
      \ifx#1\@undefined
        \xdef#1{#3}%
      \fi
      \ifx#1\relax
        \xdef#1{#3}%
      \fi
    }%
  \fi
\expandafter\x\csname ver@hologo.sty\endcsname
\ProvidesPackage{hologo}%
  [2016/05/12 v1.11 A logo collection with bookmark support (HO)]%
%    \end{macrocode}
%
%    \begin{macrocode}
\begingroup\catcode61\catcode48\catcode32=10\relax%
  \catcode13=5 % ^^M
  \endlinechar=13 %
  \catcode123=1 % {
  \catcode125=2 % }
  \catcode64=11 % @
  \def\x{\endgroup
    \expandafter\edef\csname HOLOGO@AtEnd\endcsname{%
      \endlinechar=\the\endlinechar\relax
      \catcode13=\the\catcode13\relax
      \catcode32=\the\catcode32\relax
      \catcode35=\the\catcode35\relax
      \catcode61=\the\catcode61\relax
      \catcode64=\the\catcode64\relax
      \catcode123=\the\catcode123\relax
      \catcode125=\the\catcode125\relax
    }%
  }%
\x\catcode61\catcode48\catcode32=10\relax%
\catcode13=5 % ^^M
\endlinechar=13 %
\catcode35=6 % #
\catcode64=11 % @
\catcode123=1 % {
\catcode125=2 % }
\def\TMP@EnsureCode#1#2{%
  \edef\HOLOGO@AtEnd{%
    \HOLOGO@AtEnd
    \catcode#1=\the\catcode#1\relax
  }%
  \catcode#1=#2\relax
}
\TMP@EnsureCode{10}{12}% ^^J
\TMP@EnsureCode{33}{12}% !
\TMP@EnsureCode{34}{12}% "
\TMP@EnsureCode{36}{3}% $
\TMP@EnsureCode{38}{4}% &
\TMP@EnsureCode{39}{12}% '
\TMP@EnsureCode{40}{12}% (
\TMP@EnsureCode{41}{12}% )
\TMP@EnsureCode{42}{12}% *
\TMP@EnsureCode{43}{12}% +
\TMP@EnsureCode{44}{12}% ,
\TMP@EnsureCode{45}{12}% -
\TMP@EnsureCode{46}{12}% .
\TMP@EnsureCode{47}{12}% /
\TMP@EnsureCode{58}{12}% :
\TMP@EnsureCode{59}{12}% ;
\TMP@EnsureCode{60}{12}% <
\TMP@EnsureCode{62}{12}% >
\TMP@EnsureCode{63}{12}% ?
\TMP@EnsureCode{91}{12}% [
\TMP@EnsureCode{93}{12}% ]
\TMP@EnsureCode{94}{7}% ^ (superscript)
\TMP@EnsureCode{95}{8}% _ (subscript)
\TMP@EnsureCode{96}{12}% `
\TMP@EnsureCode{124}{12}% |
\edef\HOLOGO@AtEnd{%
  \HOLOGO@AtEnd
  \escapechar\the\escapechar\relax
  \noexpand\endinput
}
\escapechar=92 %
%    \end{macrocode}
%
% \subsection{Logo list}
%
%    \begin{macro}{\hologoList}
%    \begin{macrocode}
\def\hologoList{%
  \hologoEntry{(La)TeX}{}{2011/10/01}%
  \hologoEntry{AmSLaTeX}{}{2010/04/16}%
  \hologoEntry{AmSTeX}{}{2010/04/16}%
  \hologoEntry{biber}{}{2011/10/01}%
  \hologoEntry{BibTeX}{}{2011/10/01}%
  \hologoEntry{BibTeX}{sf}{2011/10/01}%
  \hologoEntry{BibTeX}{sc}{2011/10/01}%
  \hologoEntry{BibTeX8}{}{2011/11/22}%
  \hologoEntry{ConTeXt}{}{2011/03/25}%
  \hologoEntry{ConTeXt}{narrow}{2011/03/25}%
  \hologoEntry{ConTeXt}{simple}{2011/03/25}%
  \hologoEntry{emTeX}{}{2010/04/26}%
  \hologoEntry{eTeX}{}{2010/04/08}%
  \hologoEntry{ExTeX}{}{2011/10/01}%
  \hologoEntry{HanTheThanh}{}{2011/11/29}%
  \hologoEntry{iniTeX}{}{2011/10/01}%
  \hologoEntry{KOMAScript}{}{2011/10/01}%
  \hologoEntry{La}{}{2010/05/08}%
  \hologoEntry{LaTeX}{}{2010/04/08}%
  \hologoEntry{LaTeX2e}{}{2010/04/08}%
  \hologoEntry{LaTeX3}{}{2010/04/24}%
  \hologoEntry{LaTeXe}{}{2010/04/08}%
  \hologoEntry{LaTeXML}{}{2011/11/22}%
  \hologoEntry{LaTeXTeX}{}{2011/10/01}%
  \hologoEntry{LuaLaTeX}{}{2010/04/08}%
  \hologoEntry{LuaTeX}{}{2010/04/08}%
  \hologoEntry{LyX}{}{2011/10/01}%
  \hologoEntry{METAFONT}{}{2011/10/01}%
  \hologoEntry{MetaFun}{}{2011/10/01}%
  \hologoEntry{METAPOST}{}{2011/10/01}%
  \hologoEntry{MetaPost}{}{2011/10/01}%
  \hologoEntry{MiKTeX}{}{2011/10/01}%
  \hologoEntry{NTS}{}{2011/10/01}%
  \hologoEntry{OzMF}{}{2011/10/01}%
  \hologoEntry{OzMP}{}{2011/10/01}%
  \hologoEntry{OzTeX}{}{2011/10/01}%
  \hologoEntry{OzTtH}{}{2011/10/01}%
  \hologoEntry{PCTeX}{}{2011/10/01}%
  \hologoEntry{pdfTeX}{}{2011/10/01}%
  \hologoEntry{pdfLaTeX}{}{2011/10/01}%
  \hologoEntry{PiC}{}{2011/10/01}%
  \hologoEntry{PiCTeX}{}{2011/10/01}%
  \hologoEntry{plainTeX}{}{2010/04/08}%
  \hologoEntry{plainTeX}{space}{2010/04/16}%
  \hologoEntry{plainTeX}{hyphen}{2010/04/16}%
  \hologoEntry{plainTeX}{runtogether}{2010/04/16}%
  \hologoEntry{SageTeX}{}{2011/11/22}%
  \hologoEntry{SLiTeX}{}{2011/10/01}%
  \hologoEntry{SLiTeX}{lift}{2011/10/01}%
  \hologoEntry{SLiTeX}{narrow}{2011/10/01}%
  \hologoEntry{SLiTeX}{simple}{2011/10/01}%
  \hologoEntry{SliTeX}{}{2011/10/01}%
  \hologoEntry{SliTeX}{narrow}{2011/10/01}%
  \hologoEntry{SliTeX}{simple}{2011/10/01}%
  \hologoEntry{SliTeX}{lift}{2011/10/01}%
  \hologoEntry{teTeX}{}{2011/10/01}%
  \hologoEntry{TeX}{}{2010/04/08}%
  \hologoEntry{TeX4ht}{}{2011/11/22}%
  \hologoEntry{TTH}{}{2011/11/22}%
  \hologoEntry{virTeX}{}{2011/10/01}%
  \hologoEntry{VTeX}{}{2010/04/24}%
  \hologoEntry{Xe}{}{2010/04/08}%
  \hologoEntry{XeLaTeX}{}{2010/04/08}%
  \hologoEntry{XeTeX}{}{2010/04/08}%
}
%    \end{macrocode}
%    \end{macro}
%
% \subsection{Load resources}
%
%    \begin{macrocode}
\begingroup\expandafter\expandafter\expandafter\endgroup
\expandafter\ifx\csname RequirePackage\endcsname\relax
  \def\TMP@RequirePackage#1[#2]{%
    \begingroup\expandafter\expandafter\expandafter\endgroup
    \expandafter\ifx\csname ver@#1.sty\endcsname\relax
      \input #1.sty\relax
    \fi
  }%
  \TMP@RequirePackage{ltxcmds}[2011/02/04]%
  \TMP@RequirePackage{infwarerr}[2010/04/08]%
  \TMP@RequirePackage{kvsetkeys}[2010/03/01]%
  \TMP@RequirePackage{kvdefinekeys}[2010/03/01]%
  \TMP@RequirePackage{pdftexcmds}[2010/04/01]%
  \TMP@RequirePackage{ifpdf}[2010/01/28]%
  \TMP@RequirePackage{ifluatex}[2010/03/01]%
  \ltx@IfUndefined{newif}{%
    \expandafter\let\csname newif\endcsname\ltx@newif
  }{}%
  \TMP@RequirePackage{ifxetex}[2009/01/23]%
  \TMP@RequirePackage{ifvtex}[2010/03/01]%
\else
  \RequirePackage{ltxcmds}[2011/02/04]%
  \RequirePackage{infwarerr}[2010/04/08]%
  \RequirePackage{kvsetkeys}[2010/03/01]%
  \RequirePackage{kvdefinekeys}[2010/03/01]%
  \RequirePackage{pdftexcmds}[2010/04/01]%
  \RequirePackage{ifpdf}[2010/01/28]%
  \RequirePackage{ifluatex}[2010/03/01]%
  \RequirePackage{ifxetex}[2009/01/23]%
  \RequirePackage{ifvtex}[2010/03/01]%
\fi
%    \end{macrocode}
%
%    \begin{macro}{\HOLOGO@IfDefined}
%    \begin{macrocode}
\def\HOLOGO@IfExists#1{%
  \ifx\@undefined#1%
    \expandafter\ltx@secondoftwo
  \else
    \ifx\relax#1%
      \expandafter\ltx@secondoftwo
    \else
      \expandafter\expandafter\expandafter\ltx@firstoftwo
    \fi
  \fi
}
%    \end{macrocode}
%    \end{macro}
%
% \subsection{Setup macros}
%
%    \begin{macro}{\hologoSetup}
%    \begin{macrocode}
\def\hologoSetup{%
  \let\HOLOGO@name\relax
  \HOLOGO@Setup
}
%    \end{macrocode}
%    \end{macro}
%
%    \begin{macro}{\hologoLogoSetup}
%    \begin{macrocode}
\def\hologoLogoSetup#1{%
  \edef\HOLOGO@name{#1}%
  \ltx@IfUndefined{HoLogo@\HOLOGO@name}{%
    \@PackageError{hologo}{%
      Unknown logo `\HOLOGO@name'%
    }\@ehc
    \ltx@gobble
  }{%
    \HOLOGO@Setup
  }%
}
%    \end{macrocode}
%    \end{macro}
%
%    \begin{macro}{\HOLOGO@Setup}
%    \begin{macrocode}
\def\HOLOGO@Setup{%
  \kvsetkeys{HoLogo}%
}
%    \end{macrocode}
%    \end{macro}
%
% \subsection{Options}
%
%    \begin{macro}{\HOLOGO@DeclareBoolOption}
%    \begin{macrocode}
\def\HOLOGO@DeclareBoolOption#1{%
  \expandafter\chardef\csname HOLOGOOPT@#1\endcsname\ltx@zero
  \kv@define@key{HoLogo}{#1}[true]{%
    \def\HOLOGO@temp{##1}%
    \ifx\HOLOGO@temp\HOLOGO@true
      \ifx\HOLOGO@name\relax
        \expandafter\chardef\csname HOLOGOOPT@#1\endcsname=\ltx@one
      \else
        \expandafter\chardef\csname
        HoLogoOpt@#1@\HOLOGO@name\endcsname\ltx@one
      \fi
      \HOLOGO@SetBreakAll{#1}%
    \else
      \ifx\HOLOGO@temp\HOLOGO@false
        \ifx\HOLOGO@name\relax
          \expandafter\chardef\csname HOLOGOOPT@#1\endcsname=\ltx@zero
        \else
          \expandafter\chardef\csname
          HoLogoOpt@#1@\HOLOGO@name\endcsname=\ltx@zero
        \fi
        \HOLOGO@SetBreakAll{#1}%
      \else
        \@PackageError{hologo}{%
          Unknown value `##1' for boolean option `#1'.\MessageBreak
          Known values are `true' and `false'%
        }\@ehc
      \fi
    \fi
  }%
}
%    \end{macrocode}
%    \end{macro}
%
%    \begin{macro}{\HOLOGO@SetBreakAll}
%    \begin{macrocode}
\def\HOLOGO@SetBreakAll#1{%
  \def\HOLOGO@temp{#1}%
  \ifx\HOLOGO@temp\HOLOGO@break
    \ifx\HOLOGO@name\relax
      \chardef\HOLOGOOPT@hyphenbreak=\HOLOGOOPT@break
      \chardef\HOLOGOOPT@spacebreak=\HOLOGOOPT@break
      \chardef\HOLOGOOPT@discretionarybreak=\HOLOGOOPT@break
    \else
      \expandafter\chardef
         \csname HoLogoOpt@hyphenbreak@\HOLOGO@name\endcsname=%
         \csname HoLogoOpt@break@\HOLOGO@name\endcsname
      \expandafter\chardef
         \csname HoLogoOpt@spacebreak@\HOLOGO@name\endcsname=%
         \csname HoLogoOpt@break@\HOLOGO@name\endcsname
      \expandafter\chardef
         \csname HoLogoOpt@discretionarybreak@\HOLOGO@name
             \endcsname=%
         \csname HoLogoOpt@break@\HOLOGO@name\endcsname
    \fi
  \fi
}
%    \end{macrocode}
%    \end{macro}
%
%    \begin{macro}{\HOLOGO@true}
%    \begin{macrocode}
\def\HOLOGO@true{true}
%    \end{macrocode}
%    \end{macro}
%    \begin{macro}{\HOLOGO@false}
%    \begin{macrocode}
\def\HOLOGO@false{false}
%    \end{macrocode}
%    \end{macro}
%    \begin{macro}{\HOLOGO@break}
%    \begin{macrocode}
\def\HOLOGO@break{break}
%    \end{macrocode}
%    \end{macro}
%
%    \begin{macrocode}
\HOLOGO@DeclareBoolOption{break}
\HOLOGO@DeclareBoolOption{hyphenbreak}
\HOLOGO@DeclareBoolOption{spacebreak}
\HOLOGO@DeclareBoolOption{discretionarybreak}
%    \end{macrocode}
%
%    \begin{macrocode}
\kv@define@key{HoLogo}{variant}{%
  \ifx\HOLOGO@name\relax
    \@PackageError{hologo}{%
      Option `variant' is not available in \string\hologoSetup,%
      \MessageBreak
      Use \string\hologoLogoSetup\space instead%
    }\@ehc
  \else
    \edef\HOLOGO@temp{#1}%
    \ifx\HOLOGO@temp\ltx@empty
      \expandafter
      \let\csname HoLogoOpt@variant@\HOLOGO@name\endcsname\@undefined
    \else
      \ltx@IfUndefined{HoLogo@\HOLOGO@name @\HOLOGO@temp}{%
        \@PackageError{hologo}{%
          Unknown variant `\HOLOGO@temp' of logo `\HOLOGO@name'%
        }\@ehc
      }{%
        \expandafter
        \let\csname HoLogoOpt@variant@\HOLOGO@name\endcsname
            \HOLOGO@temp
      }%
    \fi
  \fi
}
%    \end{macrocode}
%
%    \begin{macro}{\HOLOGO@Variant}
%    \begin{macrocode}
\def\HOLOGO@Variant#1{%
  #1%
  \ltx@ifundefined{HoLogoOpt@variant@#1}{%
  }{%
    @\csname HoLogoOpt@variant@#1\endcsname
  }%
}
%    \end{macrocode}
%    \end{macro}
%
% \subsection{Break/no-break support}
%
%    \begin{macro}{\HOLOGO@space}
%    \begin{macrocode}
\def\HOLOGO@space{%
  \ltx@ifundefined{HoLogoOpt@spacebreak@\HOLOGO@name}{%
    \ltx@ifundefined{HoLogoOpt@break@\HOLOGO@name}{%
      \chardef\HOLOGO@temp=\HOLOGOOPT@spacebreak
    }{%
      \chardef\HOLOGO@temp=%
        \csname HoLogoOpt@break@\HOLOGO@name\endcsname
    }%
  }{%
    \chardef\HOLOGO@temp=%
      \csname HoLogoOpt@spacebreak@\HOLOGO@name\endcsname
  }%
  \ifcase\HOLOGO@temp
    \penalty10000 %
  \fi
  \ltx@space
}
%    \end{macrocode}
%    \end{macro}
%
%    \begin{macro}{\HOLOGO@hyphen}
%    \begin{macrocode}
\def\HOLOGO@hyphen{%
  \ltx@ifundefined{HoLogoOpt@hyphenbreak@\HOLOGO@name}{%
    \ltx@ifundefined{HoLogoOpt@break@\HOLOGO@name}{%
      \chardef\HOLOGO@temp=\HOLOGOOPT@hyphenbreak
    }{%
      \chardef\HOLOGO@temp=%
        \csname HoLogoOpt@break@\HOLOGO@name\endcsname
    }%
  }{%
    \chardef\HOLOGO@temp=%
      \csname HoLogoOpt@hyphenbreak@\HOLOGO@name\endcsname
  }%
  \ifcase\HOLOGO@temp
    \ltx@mbox{-}%
  \else
    -%
  \fi
}
%    \end{macrocode}
%    \end{macro}
%
%    \begin{macro}{\HOLOGO@discretionary}
%    \begin{macrocode}
\def\HOLOGO@discretionary{%
  \ltx@ifundefined{HoLogoOpt@discretionarybreak@\HOLOGO@name}{%
    \ltx@ifundefined{HoLogoOpt@break@\HOLOGO@name}{%
      \chardef\HOLOGO@temp=\HOLOGOOPT@discretionarybreak
    }{%
      \chardef\HOLOGO@temp=%
        \csname HoLogoOpt@break@\HOLOGO@name\endcsname
    }%
  }{%
    \chardef\HOLOGO@temp=%
      \csname HoLogoOpt@discretionarybreak@\HOLOGO@name\endcsname
  }%
  \ifcase\HOLOGO@temp
  \else
    \-%
  \fi
}
%    \end{macrocode}
%    \end{macro}
%
%    \begin{macro}{\HOLOGO@mbox}
%    \begin{macrocode}
\def\HOLOGO@mbox#1{%
  \ltx@ifundefined{HoLogoOpt@break@\HOLOGO@name}{%
    \chardef\HOLOGO@temp=\HOLOGOOPT@hyphenbreak
  }{%
    \chardef\HOLOGO@temp=%
      \csname HoLogoOpt@break@\HOLOGO@name\endcsname
  }%
  \ifcase\HOLOGO@temp
    \ltx@mbox{#1}%
  \else
    #1%
  \fi
}
%    \end{macrocode}
%    \end{macro}
%
% \subsection{Font support}
%
%    \begin{macro}{\HoLogoFont@font}
%    \begin{tabular}{@{}ll@{}}
%    |#1|:& logo name\\
%    |#2|:& font short name\\
%    |#3|:& text
%    \end{tabular}
%    \begin{macrocode}
\def\HoLogoFont@font#1#2#3{%
  \begingroup
    \ltx@IfUndefined{HoLogoFont@logo@#1.#2}{%
      \ltx@IfUndefined{HoLogoFont@font@#2}{%
        \@PackageWarning{hologo}{%
          Missing font `#2' for logo `#1'%
        }%
        #3%
      }{%
        \csname HoLogoFont@font@#2\endcsname{#3}%
      }%
    }{%
      \csname HoLogoFont@logo@#1.#2\endcsname{#3}%
    }%
  \endgroup
}
%    \end{macrocode}
%    \end{macro}
%
%    \begin{macro}{\HoLogoFont@Def}
%    \begin{macrocode}
\def\HoLogoFont@Def#1{%
  \expandafter\def\csname HoLogoFont@font@#1\endcsname
}
%    \end{macrocode}
%    \end{macro}
%    \begin{macro}{\HoLogoFont@LogoDef}
%    \begin{macrocode}
\def\HoLogoFont@LogoDef#1#2{%
  \expandafter\def\csname HoLogoFont@logo@#1.#2\endcsname
}
%    \end{macrocode}
%    \end{macro}
%
% \subsubsection{Font defaults}
%
%    \begin{macro}{\HoLogoFont@font@general}
%    \begin{macrocode}
\HoLogoFont@Def{general}{}%
%    \end{macrocode}
%    \end{macro}
%
%    \begin{macro}{\HoLogoFont@font@rm}
%    \begin{macrocode}
\ltx@IfUndefined{rmfamily}{%
  \ltx@IfUndefined{rm}{%
  }{%
    \HoLogoFont@Def{rm}{\rm}%
  }%
}{%
  \HoLogoFont@Def{rm}{\rmfamily}%
}
%    \end{macrocode}
%    \end{macro}
%
%    \begin{macro}{\HoLogoFont@font@sf}
%    \begin{macrocode}
\ltx@IfUndefined{sffamily}{%
  \ltx@IfUndefined{sf}{%
  }{%
    \HoLogoFont@Def{sf}{\sf}%
  }%
}{%
  \HoLogoFont@Def{sf}{\sffamily}%
}
%    \end{macrocode}
%    \end{macro}
%
%    \begin{macro}{\HoLogoFont@font@bibsf}
%    In case of \hologo{plainTeX} the original small caps
%    variant is used as default. In \hologo{LaTeX}
%    the definition of package \xpackage{dtklogos} \cite{dtklogos}
%    is used.
%\begin{quote}
%\begin{verbatim}
%\DeclareRobustCommand{\BibTeX}{%
%  B%
%  \kern-.05em%
%  \hbox{%
%    $\m@th$% %% force math size calculations
%    \csname S@\f@size\endcsname
%    \fontsize\sf@size\z@
%    \math@fontsfalse
%    \selectfont
%    I%
%    \kern-.025em%
%    B
%  }%
%  \kern-.08em%
%  \-%
%  \TeX
%}
%\end{verbatim}
%\end{quote}
%    \begin{macrocode}
\ltx@IfUndefined{selectfont}{%
  \ltx@IfUndefined{tensc}{%
    \font\tensc=cmcsc10\relax
  }{}%
  \HoLogoFont@Def{bibsf}{\tensc}%
}{%
  \HoLogoFont@Def{bibsf}{%
    $\mathsurround=0pt$%
    \csname S@\f@size\endcsname
    \fontsize\sf@size{0pt}%
    \math@fontsfalse
    \selectfont
  }%
}
%    \end{macrocode}
%    \end{macro}
%
%    \begin{macro}{\HoLogoFont@font@sc}
%    \begin{macrocode}
\ltx@IfUndefined{scshape}{%
  \ltx@IfUndefined{tensc}{%
    \font\tensc=cmcsc10\relax
  }{}%
  \HoLogoFont@Def{sc}{\tensc}%
}{%
  \HoLogoFont@Def{sc}{\scshape}%
}
%    \end{macrocode}
%    \end{macro}
%
%    \begin{macro}{\HoLogoFont@font@sy}
%    \begin{macrocode}
\ltx@IfUndefined{usefont}{%
  \ltx@IfUndefined{tensy}{%
  }{%
    \HoLogoFont@Def{sy}{\tensy}%
  }%
}{%
  \HoLogoFont@Def{sy}{%
    \usefont{OMS}{cmsy}{m}{n}%
  }%
}
%    \end{macrocode}
%    \end{macro}
%
%    \begin{macro}{\HoLogoFont@font@logo}
%    \begin{macrocode}
\begingroup
  \def\x{LaTeX2e}%
\expandafter\endgroup
\ifx\fmtname\x
  \ltx@IfUndefined{logofamily}{%
    \DeclareRobustCommand\logofamily{%
      \not@math@alphabet\logofamily\relax
      \fontencoding{U}%
      \fontfamily{logo}%
      \selectfont
    }%
  }{}%
  \ltx@IfUndefined{logofamily}{%
  }{%
    \HoLogoFont@Def{logo}{\logofamily}%
  }%
\else
  \ltx@IfUndefined{tenlogo}{%
    \font\tenlogo=logo10\relax
  }{}%
  \HoLogoFont@Def{logo}{\tenlogo}%
\fi
%    \end{macrocode}
%    \end{macro}
%
% \subsubsection{Font setup}
%
%    \begin{macro}{\hologoFontSetup}
%    \begin{macrocode}
\def\hologoFontSetup{%
  \let\HOLOGO@name\relax
  \HOLOGO@FontSetup
}
%    \end{macrocode}
%    \end{macro}
%
%    \begin{macro}{\hologoLogoFontSetup}
%    \begin{macrocode}
\def\hologoLogoFontSetup#1{%
  \edef\HOLOGO@name{#1}%
  \ltx@IfUndefined{HoLogo@\HOLOGO@name}{%
    \@PackageError{hologo}{%
      Unknown logo `\HOLOGO@name'%
    }\@ehc
    \ltx@gobble
  }{%
    \HOLOGO@FontSetup
  }%
}
%    \end{macrocode}
%    \end{macro}
%
%    \begin{macro}{\HOLOGO@FontSetup}
%    \begin{macrocode}
\def\HOLOGO@FontSetup{%
  \kvsetkeys{HoLogoFont}%
}
%    \end{macrocode}
%    \end{macro}
%
%    \begin{macrocode}
\def\HOLOGO@temp#1{%
  \kv@define@key{HoLogoFont}{#1}{%
    \ifx\HOLOGO@name\relax
      \HoLogoFont@Def{#1}{##1}%
    \else
      \HoLogoFont@LogoDef\HOLOGO@name{#1}{##1}%
    \fi
  }%
}
\HOLOGO@temp{general}
\HOLOGO@temp{sf}
%    \end{macrocode}
%
% \subsection{Generic logo commands}
%
%    \begin{macrocode}
\HOLOGO@IfExists\hologo{%
  \@PackageError{hologo}{%
    \string\hologo\ltx@space is already defined.\MessageBreak
    Package loading is aborted%
  }\@ehc
  \HOLOGO@AtEnd
}%
\HOLOGO@IfExists\hologoRobust{%
  \@PackageError{hologo}{%
    \string\hologoRobust\ltx@space is already defined.\MessageBreak
    Package loading is aborted%
  }\@ehc
  \HOLOGO@AtEnd
}%
%    \end{macrocode}
%
% \subsubsection{\cs{hologo} and friends}
%
%    \begin{macrocode}
\ifluatex
  \expandafter\ltx@firstofone
\else
  \expandafter\ltx@gobble
\fi
{%
  \ltx@IfUndefined{ifincsname}{%
    \ifnum\luatexversion<36 %
      \expandafter\ltx@gobble
    \else
      \expandafter\ltx@firstofone
    \fi
    {%
      \begingroup
        \ifcase0%
            \directlua{%
              if tex.enableprimitives then %
                tex.enableprimitives('HOLOGO@', {'ifincsname'})%
              else %
                tex.print('1')%
              end%
            }%
            \ifx\HOLOGO@ifincsname\@undefined 1\fi%
            \relax
          \expandafter\ltx@firstofone
        \else
          \endgroup
          \expandafter\ltx@gobble
        \fi
        {%
          \global\let\ifincsname\HOLOGO@ifincsname
        }%
      \HOLOGO@temp
    }%
  }{}%
}
%    \end{macrocode}
%    \begin{macrocode}
\ltx@IfUndefined{ifincsname}{%
  \catcode`$=14 %
}{%
  \catcode`$=9 %
}
%    \end{macrocode}
%
%    \begin{macro}{\hologo}
%    \begin{macrocode}
\def\hologo#1{%
$ \ifincsname
$   \ltx@ifundefined{HoLogoCs@\HOLOGO@Variant{#1}}{%
$     #1%
$   }{%
$     \csname HoLogoCs@\HOLOGO@Variant{#1}\endcsname\ltx@firstoftwo
$   }%
$ \else
    \HOLOGO@IfExists\texorpdfstring\texorpdfstring\ltx@firstoftwo
    {%
      \hologoRobust{#1}%
    }{%
      \ltx@ifundefined{HoLogoBkm@\HOLOGO@Variant{#1}}{%
        \ltx@ifundefined{HoLogo@#1}{?#1?}{#1}%
      }{%
        \csname HoLogoBkm@\HOLOGO@Variant{#1}\endcsname
        \ltx@firstoftwo
      }%
    }%
$ \fi
}
%    \end{macrocode}
%    \end{macro}
%    \begin{macro}{\Hologo}
%    \begin{macrocode}
\def\Hologo#1{%
$ \ifincsname
$   \ltx@ifundefined{HoLogoCs@\HOLOGO@Variant{#1}}{%
$     #1%
$   }{%
$     \csname HoLogoCs@\HOLOGO@Variant{#1}\endcsname\ltx@secondoftwo
$   }%
$ \else
    \HOLOGO@IfExists\texorpdfstring\texorpdfstring\ltx@firstoftwo
    {%
      \HologoRobust{#1}%
    }{%
      \ltx@ifundefined{HoLogoBkm@\HOLOGO@Variant{#1}}{%
        \ltx@ifundefined{HoLogo@#1}{?#1?}{#1}%
      }{%
        \csname HoLogoBkm@\HOLOGO@Variant{#1}\endcsname
        \ltx@secondoftwo
      }%
    }%
$ \fi
}
%    \end{macrocode}
%    \end{macro}
%
%    \begin{macro}{\hologoVariant}
%    \begin{macrocode}
\def\hologoVariant#1#2{%
  \ifx\relax#2\relax
    \hologo{#1}%
  \else
$   \ifincsname
$     \ltx@ifundefined{HoLogoCs@#1@#2}{%
$       #1%
$     }{%
$       \csname HoLogoCs@#1@#2\endcsname\ltx@firstoftwo
$     }%
$   \else
      \HOLOGO@IfExists\texorpdfstring\texorpdfstring\ltx@firstoftwo
      {%
        \hologoVariantRobust{#1}{#2}%
      }{%
        \ltx@ifundefined{HoLogoBkm@#1@#2}{%
          \ltx@ifundefined{HoLogo@#1}{?#1?}{#1}%
        }{%
          \csname HoLogoBkm@#1@#2\endcsname
          \ltx@firstoftwo
        }%
      }%
$   \fi
  \fi
}
%    \end{macrocode}
%    \end{macro}
%    \begin{macro}{\HologoVariant}
%    \begin{macrocode}
\def\HologoVariant#1#2{%
  \ifx\relax#2\relax
    \Hologo{#1}%
  \else
$   \ifincsname
$     \ltx@ifundefined{HoLogoCs@#1@#2}{%
$       #1%
$     }{%
$       \csname HoLogoCs@#1@#2\endcsname\ltx@secondoftwo
$     }%
$   \else
      \HOLOGO@IfExists\texorpdfstring\texorpdfstring\ltx@firstoftwo
      {%
        \HologoVariantRobust{#1}{#2}%
      }{%
        \ltx@ifundefined{HoLogoBkm@#1@#2}{%
          \ltx@ifundefined{HoLogo@#1}{?#1?}{#1}%
        }{%
          \csname HoLogoBkm@#1@#2\endcsname
          \ltx@secondoftwo
        }%
      }%
$   \fi
  \fi
}
%    \end{macrocode}
%    \end{macro}
%
%    \begin{macrocode}
\catcode`\$=3 %
%    \end{macrocode}
%
% \subsubsection{\cs{hologoRobust} and friends}
%
%    \begin{macro}{\hologoRobust}
%    \begin{macrocode}
\ltx@IfUndefined{protected}{%
  \ltx@IfUndefined{DeclareRobustCommand}{%
    \def\hologoRobust#1%
  }{%
    \DeclareRobustCommand*\hologoRobust[1]%
  }%
}{%
  \protected\def\hologoRobust#1%
}%
{%
  \edef\HOLOGO@name{#1}%
  \ltx@IfUndefined{HoLogo@\HOLOGO@Variant\HOLOGO@name}{%
    \@PackageError{hologo}{%
      Unknown logo `\HOLOGO@name'%
    }\@ehc
    ?\HOLOGO@name?%
  }{%
    \ltx@IfUndefined{ver@tex4ht.sty}{%
      \HoLogoFont@font\HOLOGO@name{general}{%
        \csname HoLogo@\HOLOGO@Variant\HOLOGO@name\endcsname
        \ltx@firstoftwo
      }%
    }{%
      \ltx@IfUndefined{HoLogoHtml@\HOLOGO@Variant\HOLOGO@name}{%
        \HOLOGO@name
      }{%
        \csname HoLogoHtml@\HOLOGO@Variant\HOLOGO@name\endcsname
        \ltx@firstoftwo
      }%
    }%
  }%
}
%    \end{macrocode}
%    \end{macro}
%    \begin{macro}{\HologoRobust}
%    \begin{macrocode}
\ltx@IfUndefined{protected}{%
  \ltx@IfUndefined{DeclareRobustCommand}{%
    \def\HologoRobust#1%
  }{%
    \DeclareRobustCommand*\HologoRobust[1]%
  }%
}{%
  \protected\def\HologoRobust#1%
}%
{%
  \edef\HOLOGO@name{#1}%
  \ltx@IfUndefined{HoLogo@\HOLOGO@Variant\HOLOGO@name}{%
    \@PackageError{hologo}{%
      Unknown logo `\HOLOGO@name'%
    }\@ehc
    ?\HOLOGO@name?%
  }{%
    \ltx@IfUndefined{ver@tex4ht.sty}{%
      \HoLogoFont@font\HOLOGO@name{general}{%
        \csname HoLogo@\HOLOGO@Variant\HOLOGO@name\endcsname
        \ltx@secondoftwo
      }%
    }{%
      \ltx@IfUndefined{HoLogoHtml@\HOLOGO@Variant\HOLOGO@name}{%
        \expandafter\HOLOGO@Uppercase\HOLOGO@name
      }{%
        \csname HoLogoHtml@\HOLOGO@Variant\HOLOGO@name\endcsname
        \ltx@secondoftwo
      }%
    }%
  }%
}
%    \end{macrocode}
%    \end{macro}
%    \begin{macro}{\hologoVariantRobust}
%    \begin{macrocode}
\ltx@IfUndefined{protected}{%
  \ltx@IfUndefined{DeclareRobustCommand}{%
    \def\hologoVariantRobust#1#2%
  }{%
    \DeclareRobustCommand*\hologoVariantRobust[2]%
  }%
}{%
  \protected\def\hologoVariantRobust#1#2%
}%
{%
  \begingroup
    \hologoLogoSetup{#1}{variant={#2}}%
    \hologoRobust{#1}%
  \endgroup
}
%    \end{macrocode}
%    \end{macro}
%    \begin{macro}{\HologoVariantRobust}
%    \begin{macrocode}
\ltx@IfUndefined{protected}{%
  \ltx@IfUndefined{DeclareRobustCommand}{%
    \def\HologoVariantRobust#1#2%
  }{%
    \DeclareRobustCommand*\HologoVariantRobust[2]%
  }%
}{%
  \protected\def\HologoVariantRobust#1#2%
}%
{%
  \begingroup
    \hologoLogoSetup{#1}{variant={#2}}%
    \HologoRobust{#1}%
  \endgroup
}
%    \end{macrocode}
%    \end{macro}
%
%    \begin{macro}{\hologorobust}
%    Macro \cs{hologorobust} is only defined for compatibility.
%    Its use is deprecated.
%    \begin{macrocode}
\def\hologorobust{\hologoRobust}
%    \end{macrocode}
%    \end{macro}
%
% \subsection{Helpers}
%
%    \begin{macro}{\HOLOGO@Uppercase}
%    Macro \cs{HOLOGO@Uppercase} is restricted to \cs{uppercase},
%    because \hologo{plainTeX} or \hologo{iniTeX} do not provide
%    \cs{MakeUppercase}.
%    \begin{macrocode}
\def\HOLOGO@Uppercase#1{\uppercase{#1}}
%    \end{macrocode}
%    \end{macro}
%
%    \begin{macro}{\HOLOGO@PdfdocUnicode}
%    \begin{macrocode}
\def\HOLOGO@PdfdocUnicode{%
  \ifx\ifHy@unicode\iftrue
    \expandafter\ltx@secondoftwo
  \else
    \expandafter\ltx@firstoftwo
  \fi
}
%    \end{macrocode}
%    \end{macro}
%
%    \begin{macro}{\HOLOGO@Math}
%    \begin{macrocode}
\def\HOLOGO@MathSetup{%
  \mathsurround0pt\relax
  \HOLOGO@IfExists\f@series{%
    \if b\expandafter\ltx@car\f@series x\@nil
      \csname boldmath\endcsname
   \fi
  }{}%
}
%    \end{macrocode}
%    \end{macro}
%
%    \begin{macro}{\HOLOGO@TempDimen}
%    \begin{macrocode}
\dimendef\HOLOGO@TempDimen=\ltx@zero
%    \end{macrocode}
%    \end{macro}
%    \begin{macro}{\HOLOGO@NegativeKerning}
%    \begin{macrocode}
\def\HOLOGO@NegativeKerning#1{%
  \begingroup
    \HOLOGO@TempDimen=0pt\relax
    \comma@parse@normalized{#1}{%
      \ifdim\HOLOGO@TempDimen=0pt %
        \expandafter\HOLOGO@@NegativeKerning\comma@entry
      \fi
      \ltx@gobble
    }%
    \ifdim\HOLOGO@TempDimen<0pt %
      \kern\HOLOGO@TempDimen
    \fi
  \endgroup
}
%    \end{macrocode}
%    \end{macro}
%    \begin{macro}{\HOLOGO@@NegativeKerning}
%    \begin{macrocode}
\def\HOLOGO@@NegativeKerning#1#2{%
  \setbox\ltx@zero\hbox{#1#2}%
  \HOLOGO@TempDimen=\wd\ltx@zero
  \setbox\ltx@zero\hbox{#1\kern0pt#2}%
  \advance\HOLOGO@TempDimen by -\wd\ltx@zero
}
%    \end{macrocode}
%    \end{macro}
%
%    \begin{macro}{\HOLOGO@SpaceFactor}
%    \begin{macrocode}
\def\HOLOGO@SpaceFactor{%
  \spacefactor1000 %
}
%    \end{macrocode}
%    \end{macro}
%
%    \begin{macro}{\HOLOGO@Span}
%    \begin{macrocode}
\def\HOLOGO@Span#1#2{%
  \HCode{<span class="HoLogo-#1">}%
  #2%
  \HCode{</span>}%
}
%    \end{macrocode}
%    \end{macro}
%
% \subsubsection{Text subscript}
%
%    \begin{macro}{\HOLOGO@SubScript}%
%    \begin{macrocode}
\def\HOLOGO@SubScript#1{%
  \ltx@IfUndefined{textsubscript}{%
    \ltx@IfUndefined{text}{%
      \ltx@mbox{%
        \mathsurround=0pt\relax
        $%
          _{%
            \ltx@IfUndefined{sf@size}{%
              \mathrm{#1}%
            }{%
              \mbox{%
                \fontsize\sf@size{0pt}\selectfont
                #1%
              }%
            }%
          }%
        $%
      }%
    }{%
      \ltx@mbox{%
        \mathsurround=0pt\relax
        $_{\text{#1}}$%
      }%
    }%
  }{%
    \textsubscript{#1}%
  }%
}
%    \end{macrocode}
%    \end{macro}
%
% \subsection{\hologo{TeX} and friends}
%
% \subsubsection{\hologo{TeX}}
%
%    \begin{macro}{\HoLogo@TeX}
%    Source: \hologo{LaTeX} kernel.
%    \begin{macrocode}
\def\HoLogo@TeX#1{%
  T\kern-.1667em\lower.5ex\hbox{E}\kern-.125emX\HOLOGO@SpaceFactor
}
%    \end{macrocode}
%    \end{macro}
%    \begin{macro}{\HoLogoHtml@TeX}
%    \begin{macrocode}
\def\HoLogoHtml@TeX#1{%
  \HoLogoCss@TeX
  \HOLOGO@Span{TeX}{%
    T%
    \HOLOGO@Span{e}{%
      E%
    }%
    X%
  }%
}
%    \end{macrocode}
%    \end{macro}
%    \begin{macro}{\HoLogoCss@TeX}
%    \begin{macrocode}
\def\HoLogoCss@TeX{%
  \Css{%
    span.HoLogo-TeX span.HoLogo-e{%
      position:relative;%
      top:.5ex;%
      margin-left:-.1667em;%
      margin-right:-.125em;%
    }%
  }%
  \Css{%
    a span.HoLogo-TeX span.HoLogo-e{%
      text-decoration:none;%
    }%
  }%
  \global\let\HoLogoCss@TeX\relax
}
%    \end{macrocode}
%    \end{macro}
%
% \subsubsection{\hologo{plainTeX}}
%
%    \begin{macro}{\HoLogo@plainTeX@space}
%    Source: ``The \hologo{TeX}book''
%    \begin{macrocode}
\def\HoLogo@plainTeX@space#1{%
  \HOLOGO@mbox{#1{p}{P}lain}\HOLOGO@space\hologo{TeX}%
}
%    \end{macrocode}
%    \end{macro}
%    \begin{macro}{\HoLogoCs@plainTeX@space}
%    \begin{macrocode}
\def\HoLogoCs@plainTeX@space#1{#1{p}{P}lain TeX}%
%    \end{macrocode}
%    \end{macro}
%    \begin{macro}{\HoLogoBkm@plainTeX@space}
%    \begin{macrocode}
\def\HoLogoBkm@plainTeX@space#1{%
  #1{p}{P}lain \hologo{TeX}%
}
%    \end{macrocode}
%    \end{macro}
%    \begin{macro}{\HoLogoHtml@plainTeX@space}
%    \begin{macrocode}
\def\HoLogoHtml@plainTeX@space#1{%
  #1{p}{P}lain \hologo{TeX}%
}
%    \end{macrocode}
%    \end{macro}
%
%    \begin{macro}{\HoLogo@plainTeX@hyphen}
%    \begin{macrocode}
\def\HoLogo@plainTeX@hyphen#1{%
  \HOLOGO@mbox{#1{p}{P}lain}\HOLOGO@hyphen\hologo{TeX}%
}
%    \end{macrocode}
%    \end{macro}
%    \begin{macro}{\HoLogoCs@plainTeX@hyphen}
%    \begin{macrocode}
\def\HoLogoCs@plainTeX@hyphen#1{#1{p}{P}lain-TeX}
%    \end{macrocode}
%    \end{macro}
%    \begin{macro}{\HoLogoBkm@plainTeX@hyphen}
%    \begin{macrocode}
\def\HoLogoBkm@plainTeX@hyphen#1{%
  #1{p}{P}lain-\hologo{TeX}%
}
%    \end{macrocode}
%    \end{macro}
%    \begin{macro}{\HoLogoHtml@plainTeX@hyphen}
%    \begin{macrocode}
\def\HoLogoHtml@plainTeX@hyphen#1{%
  #1{p}{P}lain-\hologo{TeX}%
}
%    \end{macrocode}
%    \end{macro}
%
%    \begin{macro}{\HoLogo@plainTeX@runtogether}
%    \begin{macrocode}
\def\HoLogo@plainTeX@runtogether#1{%
  \HOLOGO@mbox{#1{p}{P}lain\hologo{TeX}}%
}
%    \end{macrocode}
%    \end{macro}
%    \begin{macro}{\HoLogoCs@plainTeX@runtogether}
%    \begin{macrocode}
\def\HoLogoCs@plainTeX@runtogether#1{#1{p}{P}lainTeX}
%    \end{macrocode}
%    \end{macro}
%    \begin{macro}{\HoLogoBkm@plainTeX@runtogether}
%    \begin{macrocode}
\def\HoLogoBkm@plainTeX@runtogether#1{%
  #1{p}{P}lain\hologo{TeX}%
}
%    \end{macrocode}
%    \end{macro}
%    \begin{macro}{\HoLogoHtml@plainTeX@runtogether}
%    \begin{macrocode}
\def\HoLogoHtml@plainTeX@runtogether#1{%
  #1{p}{P}lain\hologo{TeX}%
}
%    \end{macrocode}
%    \end{macro}
%
%    \begin{macro}{\HoLogo@plainTeX}
%    \begin{macrocode}
\def\HoLogo@plainTeX{\HoLogo@plainTeX@space}
%    \end{macrocode}
%    \end{macro}
%    \begin{macro}{\HoLogoCs@plainTeX}
%    \begin{macrocode}
\def\HoLogoCs@plainTeX{\HoLogoCs@plainTeX@space}
%    \end{macrocode}
%    \end{macro}
%    \begin{macro}{\HoLogoBkm@plainTeX}
%    \begin{macrocode}
\def\HoLogoBkm@plainTeX{\HoLogoBkm@plainTeX@space}
%    \end{macrocode}
%    \end{macro}
%    \begin{macro}{\HoLogoHtml@plainTeX}
%    \begin{macrocode}
\def\HoLogoHtml@plainTeX{\HoLogoHtml@plainTeX@space}
%    \end{macrocode}
%    \end{macro}
%
% \subsubsection{\hologo{LaTeX}}
%
%    Source: \hologo{LaTeX} kernel.
%\begin{quote}
%\begin{verbatim}
%\DeclareRobustCommand{\LaTeX}{%
%  L%
%  \kern-.36em%
%  {%
%    \sbox\z@ T%
%    \vbox to\ht\z@{%
%      \hbox{%
%        \check@mathfonts
%        \fontsize\sf@size\z@
%        \math@fontsfalse
%        \selectfont
%        A%
%      }%
%      \vss
%    }%
%  }%
%  \kern-.15em%
%  \TeX
%}
%\end{verbatim}
%\end{quote}
%
%    \begin{macro}{\HoLogo@La}
%    \begin{macrocode}
\def\HoLogo@La#1{%
  L%
  \kern-.36em%
  \begingroup
    \setbox\ltx@zero\hbox{T}%
    \vbox to\ht\ltx@zero{%
      \hbox{%
        \ltx@ifundefined{check@mathfonts}{%
          \csname sevenrm\endcsname
        }{%
          \check@mathfonts
          \fontsize\sf@size{0pt}%
          \math@fontsfalse\selectfont
        }%
        A%
      }%
      \vss
    }%
  \endgroup
}
%    \end{macrocode}
%    \end{macro}
%
%    \begin{macro}{\HoLogo@LaTeX}
%    Source: \hologo{LaTeX} kernel.
%    \begin{macrocode}
\def\HoLogo@LaTeX#1{%
  \hologo{La}%
  \kern-.15em%
  \hologo{TeX}%
}
%    \end{macrocode}
%    \end{macro}
%    \begin{macro}{\HoLogoHtml@LaTeX}
%    \begin{macrocode}
\def\HoLogoHtml@LaTeX#1{%
  \HoLogoCss@LaTeX
  \HOLOGO@Span{LaTeX}{%
    L%
    \HOLOGO@Span{a}{%
      A%
    }%
    \hologo{TeX}%
  }%
}
%    \end{macrocode}
%    \end{macro}
%    \begin{macro}{\HoLogoCss@LaTeX}
%    \begin{macrocode}
\def\HoLogoCss@LaTeX{%
  \Css{%
    span.HoLogo-LaTeX span.HoLogo-a{%
      position:relative;%
      top:-.5ex;%
      margin-left:-.36em;%
      margin-right:-.15em;%
      font-size:85\%;%
    }%
  }%
  \global\let\HoLogoCss@LaTeX\relax
}
%    \end{macrocode}
%    \end{macro}
%
% \subsubsection{\hologo{(La)TeX}}
%
%    \begin{macro}{\HoLogo@LaTeXTeX}
%    The kerning around the parentheses is taken
%    from package \xpackage{dtklogos} \cite{dtklogos}.
%\begin{quote}
%\begin{verbatim}
%\DeclareRobustCommand{\LaTeXTeX}{%
%  (%
%  \kern-.15em%
%  L%
%  \kern-.36em%
%  {%
%    \sbox\z@ T%
%    \vbox to\ht0{%
%      \hbox{%
%        $\m@th$%
%        \csname S@\f@size\endcsname
%        \fontsize\sf@size\z@
%        \math@fontsfalse
%        \selectfont
%        A%
%      }%
%      \vss
%    }%
%  }%
%  \kern-.2em%
%  )%
%  \kern-.15em%
%  \TeX
%}
%\end{verbatim}
%\end{quote}
%    \begin{macrocode}
\def\HoLogo@LaTeXTeX#1{%
  (%
  \kern-.15em%
  \hologo{La}%
  \kern-.2em%
  )%
  \kern-.15em%
  \hologo{TeX}%
}
%    \end{macrocode}
%    \end{macro}
%    \begin{macro}{\HoLogoBkm@LaTeXTeX}
%    \begin{macrocode}
\def\HoLogoBkm@LaTeXTeX#1{(La)TeX}
%    \end{macrocode}
%    \end{macro}
%
%    \begin{macro}{\HoLogo@(La)TeX}
%    \begin{macrocode}
\expandafter
\let\csname HoLogo@(La)TeX\endcsname\HoLogo@LaTeXTeX
%    \end{macrocode}
%    \end{macro}
%    \begin{macro}{\HoLogoBkm@(La)TeX}
%    \begin{macrocode}
\expandafter
\let\csname HoLogoBkm@(La)TeX\endcsname\HoLogoBkm@LaTeXTeX
%    \end{macrocode}
%    \end{macro}
%    \begin{macro}{\HoLogoHtml@LaTeXTeX}
%    \begin{macrocode}
\def\HoLogoHtml@LaTeXTeX#1{%
  \HoLogoCss@LaTeXTeX
  \HOLOGO@Span{LaTeXTeX}{%
    (%
    \HOLOGO@Span{L}{L}%
    \HOLOGO@Span{a}{A}%
    \HOLOGO@Span{ParenRight}{)}%
    \hologo{TeX}%
  }%
}
%    \end{macrocode}
%    \end{macro}
%    \begin{macro}{\HoLogoHtml@(La)TeX}
%    Kerning after opening parentheses and before closing parentheses
%    is $-0.1$\,em. The original values $-0.15$\,em
%    looked too ugly for a serif font.
%    \begin{macrocode}
\expandafter
\let\csname HoLogoHtml@(La)TeX\endcsname\HoLogoHtml@LaTeXTeX
%    \end{macrocode}
%    \end{macro}
%    \begin{macro}{\HoLogoCss@LaTeXTeX}
%    \begin{macrocode}
\def\HoLogoCss@LaTeXTeX{%
  \Css{%
    span.HoLogo-LaTeXTeX span.HoLogo-L{%
      margin-left:-.1em;%
    }%
  }%
  \Css{%
    span.HoLogo-LaTeXTeX span.HoLogo-a{%
      position:relative;%
      top:-.5ex;%
      margin-left:-.36em;%
      margin-right:-.1em;%
      font-size:85\%;%
    }%
  }%
  \Css{%
    span.HoLogo-LaTeXTeX span.HoLogo-ParenRight{%
      margin-right:-.15em;%
    }%
  }%
  \global\let\HoLogoCss@LaTeXTeX\relax
}
%    \end{macrocode}
%    \end{macro}
%
% \subsubsection{\hologo{LaTeXe}}
%
%    \begin{macro}{\HoLogo@LaTeXe}
%    Source: \hologo{LaTeX} kernel
%    \begin{macrocode}
\def\HoLogo@LaTeXe#1{%
  \hologo{LaTeX}%
  \kern.15em%
  \hbox{%
    \HOLOGO@MathSetup
    2%
    $_{\textstyle\varepsilon}$%
  }%
}
%    \end{macrocode}
%    \end{macro}
%
%    \begin{macro}{\HoLogoCs@LaTeXe}
%    \begin{macrocode}
\ifnum64=`\^^^^0040\relax % test for big chars of LuaTeX/XeTeX
  \catcode`\$=9 %
  \catcode`\&=14 %
\else
  \catcode`\$=14 %
  \catcode`\&=9 %
\fi
\def\HoLogoCs@LaTeXe#1{%
  LaTeX2%
$ \string ^^^^0395%
& e%
}%
\catcode`\$=3 %
\catcode`\&=4 %
%    \end{macrocode}
%    \end{macro}
%
%    \begin{macro}{\HoLogoBkm@LaTeXe}
%    \begin{macrocode}
\def\HoLogoBkm@LaTeXe#1{%
  \hologo{LaTeX}%
  2%
  \HOLOGO@PdfdocUnicode{e}{\textepsilon}%
}
%    \end{macrocode}
%    \end{macro}
%
%    \begin{macro}{\HoLogoHtml@LaTeXe}
%    \begin{macrocode}
\def\HoLogoHtml@LaTeXe#1{%
  \HoLogoCss@LaTeXe
  \HOLOGO@Span{LaTeX2e}{%
    \hologo{LaTeX}%
    \HOLOGO@Span{2}{2}%
    \HOLOGO@Span{e}{%
      \HOLOGO@MathSetup
      \ensuremath{\textstyle\varepsilon}%
    }%
  }%
}
%    \end{macrocode}
%    \end{macro}
%    \begin{macro}{\HoLogoCss@LaTeXe}
%    \begin{macrocode}
\def\HoLogoCss@LaTeXe{%
  \Css{%
    span.HoLogo-LaTeX2e span.HoLogo-2{%
      padding-left:.15em;%
    }%
  }%
  \Css{%
    span.HoLogo-LaTeX2e span.HoLogo-e{%
      position:relative;%
      top:.35ex;%
      text-decoration:none;%
    }%
  }%
  \global\let\HoLogoCss@LaTeXe\relax
}
%    \end{macrocode}
%    \end{macro}
%
%    \begin{macro}{\HoLogo@LaTeX2e}
%    \begin{macrocode}
\expandafter
\let\csname HoLogo@LaTeX2e\endcsname\HoLogo@LaTeXe
%    \end{macrocode}
%    \end{macro}
%    \begin{macro}{\HoLogoCs@LaTeX2e}
%    \begin{macrocode}
\expandafter
\let\csname HoLogoCs@LaTeX2e\endcsname\HoLogoCs@LaTeXe
%    \end{macrocode}
%    \end{macro}
%    \begin{macro}{\HoLogoBkm@LaTeX2e}
%    \begin{macrocode}
\expandafter
\let\csname HoLogoBkm@LaTeX2e\endcsname\HoLogoBkm@LaTeXe
%    \end{macrocode}
%    \end{macro}
%    \begin{macro}{\HoLogoHtml@LaTeX2e}
%    \begin{macrocode}
\expandafter
\let\csname HoLogoHtml@LaTeX2e\endcsname\HoLogoHtml@LaTeXe
%    \end{macrocode}
%    \end{macro}
%
% \subsubsection{\hologo{LaTeX3}}
%
%    \begin{macro}{\HoLogo@LaTeX3}
%    Source: \hologo{LaTeX} kernel
%    \begin{macrocode}
\expandafter\def\csname HoLogo@LaTeX3\endcsname#1{%
  \hologo{LaTeX}%
  3%
}
%    \end{macrocode}
%    \end{macro}
%
%    \begin{macro}{\HoLogoBkm@LaTeX3}
%    \begin{macrocode}
\expandafter\def\csname HoLogoBkm@LaTeX3\endcsname#1{%
  \hologo{LaTeX}%
  3%
}
%    \end{macrocode}
%    \end{macro}
%    \begin{macro}{\HoLogoHtml@LaTeX3}
%    \begin{macrocode}
\expandafter
\let\csname HoLogoHtml@LaTeX3\expandafter\endcsname
\csname HoLogo@LaTeX3\endcsname
%    \end{macrocode}
%    \end{macro}
%
% \subsubsection{\hologo{LaTeXML}}
%
%    \begin{macro}{\HoLogo@LaTeXML}
%    \begin{macrocode}
\def\HoLogo@LaTeXML#1{%
  \HOLOGO@mbox{%
    \hologo{La}%
    \kern-.15em%
    T%
    \kern-.1667em%
    \lower.5ex\hbox{E}%
    \kern-.125em%
    \HoLogoFont@font{LaTeXML}{sc}{xml}%
  }%
}
%    \end{macrocode}
%    \end{macro}
%    \begin{macro}{\HoLogoHtml@pdfLaTeX}
%    \begin{macrocode}
\def\HoLogoHtml@LaTeXML#1{%
  \HOLOGO@Span{LaTeXML}{%
    \HoLogoCss@LaTeX
    \HoLogoCss@TeX
    \HOLOGO@Span{LaTeX}{%
      L%
      \HOLOGO@Span{a}{%
        A%
      }%
    }%
    \HOLOGO@Span{TeX}{%
      T%
      \HOLOGO@Span{e}{%
        E%
      }%
    }%
    \HCode{<span style="font-variant: small-caps;">}%
    xml%
    \HCode{</span>}%
  }%
}
%    \end{macrocode}
%    \end{macro}
%
% \subsubsection{\hologo{eTeX}}
%
%    \begin{macro}{\HoLogo@eTeX}
%    Source: package \xpackage{etex}
%    \begin{macrocode}
\def\HoLogo@eTeX#1{%
  \ltx@mbox{%
    \HOLOGO@MathSetup
    $\varepsilon$%
    -%
    \HOLOGO@NegativeKerning{-T,T-,To}%
    \hologo{TeX}%
  }%
}
%    \end{macrocode}
%    \end{macro}
%    \begin{macro}{\HoLogoCs@eTeX}
%    \begin{macrocode}
\ifnum64=`\^^^^0040\relax % test for big chars of LuaTeX/XeTeX
  \catcode`\$=9 %
  \catcode`\&=14 %
\else
  \catcode`\$=14 %
  \catcode`\&=9 %
\fi
\def\HoLogoCs@eTeX#1{%
$ #1{\string ^^^^0395}{\string ^^^^03b5}%
& #1{e}{E}%
  TeX%
}%
\catcode`\$=3 %
\catcode`\&=4 %
%    \end{macrocode}
%    \end{macro}
%    \begin{macro}{\HoLogoBkm@eTeX}
%    \begin{macrocode}
\def\HoLogoBkm@eTeX#1{%
  \HOLOGO@PdfdocUnicode{#1{e}{E}}{\textepsilon}%
  -%
  \hologo{TeX}%
}
%    \end{macrocode}
%    \end{macro}
%    \begin{macro}{\HoLogoHtml@eTeX}
%    \begin{macrocode}
\def\HoLogoHtml@eTeX#1{%
  \ltx@mbox{%
    \HOLOGO@MathSetup
    $\varepsilon$%
    -%
    \hologo{TeX}%
  }%
}
%    \end{macrocode}
%    \end{macro}
%
% \subsubsection{\hologo{iniTeX}}
%
%    \begin{macro}{\HoLogo@iniTeX}
%    \begin{macrocode}
\def\HoLogo@iniTeX#1{%
  \HOLOGO@mbox{%
    #1{i}{I}ni\hologo{TeX}%
  }%
}
%    \end{macrocode}
%    \end{macro}
%    \begin{macro}{\HoLogoCs@iniTeX}
%    \begin{macrocode}
\def\HoLogoCs@iniTeX#1{#1{i}{I}niTeX}
%    \end{macrocode}
%    \end{macro}
%    \begin{macro}{\HoLogoBkm@iniTeX}
%    \begin{macrocode}
\def\HoLogoBkm@iniTeX#1{%
  #1{i}{I}ni\hologo{TeX}%
}
%    \end{macrocode}
%    \end{macro}
%    \begin{macro}{\HoLogoHtml@iniTeX}
%    \begin{macrocode}
\let\HoLogoHtml@iniTeX\HoLogo@iniTeX
%    \end{macrocode}
%    \end{macro}
%
% \subsubsection{\hologo{virTeX}}
%
%    \begin{macro}{\HoLogo@virTeX}
%    \begin{macrocode}
\def\HoLogo@virTeX#1{%
  \HOLOGO@mbox{%
    #1{v}{V}ir\hologo{TeX}%
  }%
}
%    \end{macrocode}
%    \end{macro}
%    \begin{macro}{\HoLogoCs@virTeX}
%    \begin{macrocode}
\def\HoLogoCs@virTeX#1{#1{v}{V}irTeX}
%    \end{macrocode}
%    \end{macro}
%    \begin{macro}{\HoLogoBkm@virTeX}
%    \begin{macrocode}
\def\HoLogoBkm@virTeX#1{%
  #1{v}{V}ir\hologo{TeX}%
}
%    \end{macrocode}
%    \end{macro}
%    \begin{macro}{\HoLogoHtml@virTeX}
%    \begin{macrocode}
\let\HoLogoHtml@virTeX\HoLogo@virTeX
%    \end{macrocode}
%    \end{macro}
%
% \subsubsection{\hologo{SliTeX}}
%
% \paragraph{Definitions of the three variants.}
%
%    \begin{macro}{\HoLogo@SLiTeX@lift}
%    \begin{macrocode}
\def\HoLogo@SLiTeX@lift#1{%
  \HoLogoFont@font{SliTeX}{rm}{%
    S%
    \kern-.06em%
    L%
    \kern-.18em%
    \raise.32ex\hbox{\HoLogoFont@font{SliTeX}{sc}{i}}%
    \HOLOGO@discretionary
    \kern-.06em%
    \hologo{TeX}%
  }%
}
%    \end{macrocode}
%    \end{macro}
%    \begin{macro}{\HoLogoBkm@SLiTeX@lift}
%    \begin{macrocode}
\def\HoLogoBkm@SLiTeX@lift#1{SLiTeX}
%    \end{macrocode}
%    \end{macro}
%    \begin{macro}{\HoLogoHtml@SLiTeX@lift}
%    \begin{macrocode}
\def\HoLogoHtml@SLiTeX@lift#1{%
  \HoLogoCss@SLiTeX@lift
  \HOLOGO@Span{SLiTeX-lift}{%
    \HoLogoFont@font{SliTeX}{rm}{%
      S%
      \HOLOGO@Span{L}{L}%
      \HOLOGO@Span{i}{i}%
      \hologo{TeX}%
    }%
  }%
}
%    \end{macrocode}
%    \end{macro}
%    \begin{macro}{\HoLogoCss@SLiTeX@lift}
%    \begin{macrocode}
\def\HoLogoCss@SLiTeX@lift{%
  \Css{%
    span.HoLogo-SLiTeX-lift span.HoLogo-L{%
      margin-left:-.06em;%
      margin-right:-.18em;%
    }%
  }%
  \Css{%
    span.HoLogo-SLiTeX-lift span.HoLogo-i{%
      position:relative;%
      top:-.32ex;%
      margin-right:-.06em;%
      font-variant:small-caps;%
    }%
  }%
  \global\let\HoLogoCss@SLiTeX@lift\relax
}
%    \end{macrocode}
%    \end{macro}
%
%    \begin{macro}{\HoLogo@SliTeX@simple}
%    \begin{macrocode}
\def\HoLogo@SliTeX@simple#1{%
  \HoLogoFont@font{SliTeX}{rm}{%
    \ltx@mbox{%
      \HoLogoFont@font{SliTeX}{sc}{Sli}%
    }%
    \HOLOGO@discretionary
    \hologo{TeX}%
  }%
}
%    \end{macrocode}
%    \end{macro}
%    \begin{macro}{\HoLogoBkm@SliTeX@simple}
%    \begin{macrocode}
\def\HoLogoBkm@SliTeX@simple#1{SliTeX}
%    \end{macrocode}
%    \end{macro}
%    \begin{macro}{\HoLogoHtml@SliTeX@simple}
%    \begin{macrocode}
\let\HoLogoHtml@SliTeX@simple\HoLogo@SliTeX@simple
%    \end{macrocode}
%    \end{macro}
%
%    \begin{macro}{\HoLogo@SliTeX@narrow}
%    \begin{macrocode}
\def\HoLogo@SliTeX@narrow#1{%
  \HoLogoFont@font{SliTeX}{rm}{%
    \ltx@mbox{%
      S%
      \kern-.06em%
      \HoLogoFont@font{SliTeX}{sc}{%
        l%
        \kern-.035em%
        i%
      }%
    }%
    \HOLOGO@discretionary
    \kern-.06em%
    \hologo{TeX}%
  }%
}
%    \end{macrocode}
%    \end{macro}
%    \begin{macro}{\HoLogoBkm@SliTeX@narrow}
%    \begin{macrocode}
\def\HoLogoBkm@SliTeX@narrow#1{SliTeX}
%    \end{macrocode}
%    \end{macro}
%    \begin{macro}{\HoLogoHtml@SliTeX@narrow}
%    \begin{macrocode}
\def\HoLogoHtml@SliTeX@narrow#1{%
  \HoLogoCss@SliTeX@narrow
  \HOLOGO@Span{SliTeX-narrow}{%
    \HoLogoFont@font{SliTeX}{rm}{%
      S%
        \HOLOGO@Span{l}{l}%
        \HOLOGO@Span{i}{i}%
      \hologo{TeX}%
    }%
  }%
}
%    \end{macrocode}
%    \end{macro}
%    \begin{macro}{\HoLogoCss@SliTeX@narrow}
%    \begin{macrocode}
\def\HoLogoCss@SliTeX@narrow{%
  \Css{%
    span.HoLogo-SliTeX-narrow span.HoLogo-l{%
      margin-left:-.06em;%
      margin-right:-.035em;%
      font-variant:small-caps;%
    }%
  }%
  \Css{%
    span.HoLogo-SliTeX-narrow span.HoLogo-i{%
      margin-right:-.06em;%
      font-variant:small-caps;%
    }%
  }%
  \global\let\HoLogoCss@SliTeX@narrow\relax
}
%    \end{macrocode}
%    \end{macro}
%
% \paragraph{Macro set completion.}
%
%    \begin{macro}{\HoLogo@SLiTeX@simple}
%    \begin{macrocode}
\def\HoLogo@SLiTeX@simple{\HoLogo@SliTeX@simple}
%    \end{macrocode}
%    \end{macro}
%    \begin{macro}{\HoLogoBkm@SLiTeX@simple}
%    \begin{macrocode}
\def\HoLogoBkm@SLiTeX@simple{\HoLogoBkm@SliTeX@simple}
%    \end{macrocode}
%    \end{macro}
%    \begin{macro}{\HoLogoHtml@SLiTeX@simple}
%    \begin{macrocode}
\def\HoLogoHtml@SLiTeX@simple{\HoLogoHtml@SliTeX@simple}
%    \end{macrocode}
%    \end{macro}
%
%    \begin{macro}{\HoLogo@SLiTeX@narrow}
%    \begin{macrocode}
\def\HoLogo@SLiTeX@narrow{\HoLogo@SliTeX@narrow}
%    \end{macrocode}
%    \end{macro}
%    \begin{macro}{\HoLogoBkm@SLiTeX@narrow}
%    \begin{macrocode}
\def\HoLogoBkm@SLiTeX@narrow{\HoLogoBkm@SliTeX@narrow}
%    \end{macrocode}
%    \end{macro}
%    \begin{macro}{\HoLogoHtml@SLiTeX@narrow}
%    \begin{macrocode}
\def\HoLogoHtml@SLiTeX@narrow{\HoLogoHtml@SliTeX@narrow}
%    \end{macrocode}
%    \end{macro}
%
%    \begin{macro}{\HoLogo@SliTeX@lift}
%    \begin{macrocode}
\def\HoLogo@SliTeX@lift{\HoLogo@SLiTeX@lift}
%    \end{macrocode}
%    \end{macro}
%    \begin{macro}{\HoLogoBkm@SliTeX@lift}
%    \begin{macrocode}
\def\HoLogoBkm@SliTeX@lift{\HoLogoBkm@SLiTeX@lift}
%    \end{macrocode}
%    \end{macro}
%    \begin{macro}{\HoLogoHtml@SliTeX@lift}
%    \begin{macrocode}
\def\HoLogoHtml@SliTeX@lift{\HoLogoHtml@SLiTeX@lift}
%    \end{macrocode}
%    \end{macro}
%
% \paragraph{Defaults.}
%
%    \begin{macro}{\HoLogo@SLiTeX}
%    \begin{macrocode}
\def\HoLogo@SLiTeX{\HoLogo@SLiTeX@lift}
%    \end{macrocode}
%    \end{macro}
%    \begin{macro}{\HoLogoBkm@SLiTeX}
%    \begin{macrocode}
\def\HoLogoBkm@SLiTeX{\HoLogoBkm@SLiTeX@lift}
%    \end{macrocode}
%    \end{macro}
%    \begin{macro}{\HoLogoHtml@SLiTeX}
%    \begin{macrocode}
\def\HoLogoHtml@SLiTeX{\HoLogoHtml@SLiTeX@lift}
%    \end{macrocode}
%    \end{macro}
%
%    \begin{macro}{\HoLogo@SliTeX}
%    \begin{macrocode}
\def\HoLogo@SliTeX{\HoLogo@SliTeX@narrow}
%    \end{macrocode}
%    \end{macro}
%    \begin{macro}{\HoLogoBkm@SliTeX}
%    \begin{macrocode}
\def\HoLogoBkm@SliTeX{\HoLogoBkm@SliTeX@narrow}
%    \end{macrocode}
%    \end{macro}
%    \begin{macro}{\HoLogoHtml@SliTeX}
%    \begin{macrocode}
\def\HoLogoHtml@SliTeX{\HoLogoHtml@SliTeX@narrow}
%    \end{macrocode}
%    \end{macro}
%
% \subsubsection{\hologo{LuaTeX}}
%
%    \begin{macro}{\HoLogo@LuaTeX}
%    The kerning is an idea of Hans Hagen, see mailing list
%    `luatex at tug dot org' in March 2010.
%    \begin{macrocode}
\def\HoLogo@LuaTeX#1{%
  \HOLOGO@mbox{%
    Lua%
    \HOLOGO@NegativeKerning{aT,oT,To}%
    \hologo{TeX}%
  }%
}
%    \end{macrocode}
%    \end{macro}
%    \begin{macro}{\HoLogoHtml@LuaTeX}
%    \begin{macrocode}
\let\HoLogoHtml@LuaTeX\HoLogo@LuaTeX
%    \end{macrocode}
%    \end{macro}
%
% \subsubsection{\hologo{LuaLaTeX}}
%
%    \begin{macro}{\HoLogo@LuaLaTeX}
%    \begin{macrocode}
\def\HoLogo@LuaLaTeX#1{%
  \HOLOGO@mbox{%
    Lua%
    \hologo{LaTeX}%
  }%
}
%    \end{macrocode}
%    \end{macro}
%    \begin{macro}{\HoLogoHtml@LuaLaTeX}
%    \begin{macrocode}
\let\HoLogoHtml@LuaLaTeX\HoLogo@LuaLaTeX
%    \end{macrocode}
%    \end{macro}
%
% \subsubsection{\hologo{XeTeX}, \hologo{XeLaTeX}}
%
%    \begin{macro}{\HOLOGO@IfCharExists}
%    \begin{macrocode}
\ifluatex
  \ifnum\luatexversion<36 %
  \else
    \def\HOLOGO@IfCharExists#1{%
      \ifnum
        \directlua{%
           if luaotfload and luaotfload.aux then
             if luaotfload.aux.font_has_glyph(%
                    font.current(), \number#1) then % 	 
	       tex.print("1") % 	 
	     end % 	 
	   elseif font and font.fonts and font.current then %
            local f = font.fonts[font.current()]%
            if f.characters and f.characters[\number#1] then %
              tex.print("1")%
            end %
          end%
        }0=\ltx@zero
        \expandafter\ltx@secondoftwo
      \else
        \expandafter\ltx@firstoftwo
      \fi
    }%
  \fi
\fi
\ltx@IfUndefined{HOLOGO@IfCharExists}{%
  \def\HOLOGO@@IfCharExists#1{%
    \begingroup
      \tracinglostchars=\ltx@zero
      \setbox\ltx@zero=\hbox{%
        \kern7sp\char#1\relax
        \ifnum\lastkern>\ltx@zero
          \expandafter\aftergroup\csname iffalse\endcsname
        \else
          \expandafter\aftergroup\csname iftrue\endcsname
        \fi
      }%
      % \if{true|false} from \aftergroup
      \endgroup
      \expandafter\ltx@firstoftwo
    \else
      \endgroup
      \expandafter\ltx@secondoftwo
    \fi
  }%
  \ifxetex
    \ltx@IfUndefined{XeTeXfonttype}{}{%
      \ltx@IfUndefined{XeTeXcharglyph}{}{%
        \def\HOLOGO@IfCharExists#1{%
          \ifnum\XeTeXfonttype\font>\ltx@zero
            \expandafter\ltx@firstofthree
          \else
            \expandafter\ltx@gobble
          \fi
          {%
            \ifnum\XeTeXcharglyph#1>\ltx@zero
              \expandafter\ltx@firstoftwo
            \else
              \expandafter\ltx@secondoftwo
            \fi
          }%
          \HOLOGO@@IfCharExists{#1}%
        }%
      }%
    }%
  \fi
}{}
\ltx@ifundefined{HOLOGO@IfCharExists}{%
  \ifnum64=`\^^^^0040\relax % test for big chars of LuaTeX/XeTeX
    \let\HOLOGO@IfCharExists\HOLOGO@@IfCharExists
  \else
    \def\HOLOGO@IfCharExists#1{%
      \ifnum#1>255 %
        \expandafter\ltx@fourthoffour
      \fi
      \HOLOGO@@IfCharExists{#1}%
    }%
  \fi
}{}
%    \end{macrocode}
%    \end{macro}
%
%    \begin{macro}{\HoLogo@Xe}
%    Source: package \xpackage{dtklogos}
%    \begin{macrocode}
\def\HoLogo@Xe#1{%
  X%
  \kern-.1em\relax
  \HOLOGO@IfCharExists{"018E}{%
    \lower.5ex\hbox{\char"018E}%
  }{%
    \chardef\HOLOGO@choice=\ltx@zero
    \ifdim\fontdimen\ltx@one\font>0pt %
      \ltx@IfUndefined{rotatebox}{%
        \ltx@IfUndefined{pgftext}{%
          \ltx@IfUndefined{psscalebox}{%
            \ltx@IfUndefined{HOLOGO@ScaleBox@\hologoDriver}{%
            }{%
              \chardef\HOLOGO@choice=4 %
            }%
          }{%
            \chardef\HOLOGO@choice=3 %
          }%
        }{%
          \chardef\HOLOGO@choice=2 %
        }%
      }{%
        \chardef\HOLOGO@choice=1 %
      }%
      \ifcase\HOLOGO@choice
        \HOLOGO@WarningUnsupportedDriver{Xe}%
        e%
      \or % 1: \rotatebox
        \begingroup
          \setbox\ltx@zero\hbox{\rotatebox{180}{E}}%
          \ltx@LocDimenA=\dp\ltx@zero
          \advance\ltx@LocDimenA by -.5ex\relax
          \raise\ltx@LocDimenA\box\ltx@zero
        \endgroup
      \or % 2: \pgftext
        \lower.5ex\hbox{%
          \pgfpicture
            \pgftext[rotate=180]{E}%
          \endpgfpicture
        }%
      \or % 3: \psscalebox
        \begingroup
          \setbox\ltx@zero\hbox{\psscalebox{-1 -1}{E}}%
          \ltx@LocDimenA=\dp\ltx@zero
          \advance\ltx@LocDimenA by -.5ex\relax
          \raise\ltx@LocDimenA\box\ltx@zero
        \endgroup
      \or % 4: \HOLOGO@PointReflectBox
        \lower.5ex\hbox{\HOLOGO@PointReflectBox{E}}%
      \else
        \@PackageError{hologo}{Internal error (choice/it}\@ehc
      \fi
    \else
      \ltx@IfUndefined{reflectbox}{%
        \ltx@IfUndefined{pgftext}{%
          \ltx@IfUndefined{psscalebox}{%
            \ltx@IfUndefined{HOLOGO@ScaleBox@\hologoDriver}{%
            }{%
              \chardef\HOLOGO@choice=4 %
            }%
          }{%
            \chardef\HOLOGO@choice=3 %
          }%
        }{%
          \chardef\HOLOGO@choice=2 %
        }%
      }{%
        \chardef\HOLOGO@choice=1 %
      }%
      \ifcase\HOLOGO@choice
        \HOLOGO@WarningUnsupportedDriver{Xe}%
        e%
      \or % 1: reflectbox
        \lower.5ex\hbox{%
          \reflectbox{E}%
        }%
      \or % 2: \pgftext
        \lower.5ex\hbox{%
          \pgfpicture
            \pgftransformxscale{-1}%
            \pgftext{E}%
          \endpgfpicture
        }%
      \or % 3: \psscalebox
        \lower.5ex\hbox{%
          \psscalebox{-1 1}{E}%
        }%
      \or % 4: \HOLOGO@Reflectbox
        \lower.5ex\hbox{%
          \HOLOGO@ReflectBox{E}%
        }%
      \else
        \@PackageError{hologo}{Internal error (choice/up)}\@ehc
      \fi
    \fi
  }%
}
%    \end{macrocode}
%    \end{macro}
%    \begin{macro}{\HoLogoHtml@Xe}
%    \begin{macrocode}
\def\HoLogoHtml@Xe#1{%
  \HoLogoCss@Xe
  \HOLOGO@Span{Xe}{%
    X%
    \HOLOGO@Span{e}{%
      \HCode{&\ltx@hashchar x018e;}%
    }%
  }%
}
%    \end{macrocode}
%    \end{macro}
%    \begin{macro}{\HoLogoCss@Xe}
%    \begin{macrocode}
\def\HoLogoCss@Xe{%
  \Css{%
    span.HoLogo-Xe span.HoLogo-e{%
      position:relative;%
      top:.5ex;%
      left-margin:-.1em;%
    }%
  }%
  \global\let\HoLogoCss@Xe\relax
}
%    \end{macrocode}
%    \end{macro}
%
%    \begin{macro}{\HoLogo@XeTeX}
%    \begin{macrocode}
\def\HoLogo@XeTeX#1{%
  \hologo{Xe}%
  \kern-.15em\relax
  \hologo{TeX}%
}
%    \end{macrocode}
%    \end{macro}
%
%    \begin{macro}{\HoLogoHtml@XeTeX}
%    \begin{macrocode}
\def\HoLogoHtml@XeTeX#1{%
  \HoLogoCss@XeTeX
  \HOLOGO@Span{XeTeX}{%
    \hologo{Xe}%
    \hologo{TeX}%
  }%
}
%    \end{macrocode}
%    \end{macro}
%    \begin{macro}{\HoLogoCss@XeTeX}
%    \begin{macrocode}
\def\HoLogoCss@XeTeX{%
  \Css{%
    span.HoLogo-XeTeX span.HoLogo-TeX{%
      margin-left:-.15em;%
    }%
  }%
  \global\let\HoLogoCss@XeTeX\relax
}
%    \end{macrocode}
%    \end{macro}
%
%    \begin{macro}{\HoLogo@XeLaTeX}
%    \begin{macrocode}
\def\HoLogo@XeLaTeX#1{%
  \hologo{Xe}%
  \kern-.13em%
  \hologo{LaTeX}%
}
%    \end{macrocode}
%    \end{macro}
%    \begin{macro}{\HoLogoHtml@XeLaTeX}
%    \begin{macrocode}
\def\HoLogoHtml@XeLaTeX#1{%
  \HoLogoCss@XeLaTeX
  \HOLOGO@Span{XeLaTeX}{%
    \hologo{Xe}%
    \hologo{LaTeX}%
  }%
}
%    \end{macrocode}
%    \end{macro}
%    \begin{macro}{\HoLogoCss@XeLaTeX}
%    \begin{macrocode}
\def\HoLogoCss@XeLaTeX{%
  \Css{%
    span.HoLogo-XeLaTeX span.HoLogo-Xe{%
      margin-right:-.13em;%
    }%
  }%
  \global\let\HoLogoCss@XeLaTeX\relax
}
%    \end{macrocode}
%    \end{macro}
%
% \subsubsection{\hologo{pdfTeX}, \hologo{pdfLaTeX}}
%
%    \begin{macro}{\HoLogo@pdfTeX}
%    \begin{macrocode}
\def\HoLogo@pdfTeX#1{%
  \HOLOGO@mbox{%
    #1{p}{P}df\hologo{TeX}%
  }%
}
%    \end{macrocode}
%    \end{macro}
%    \begin{macro}{\HoLogoCs@pdfTeX}
%    \begin{macrocode}
\def\HoLogoCs@pdfTeX#1{#1{p}{P}dfTeX}
%    \end{macrocode}
%    \end{macro}
%    \begin{macro}{\HoLogoBkm@pdfTeX}
%    \begin{macrocode}
\def\HoLogoBkm@pdfTeX#1{%
  #1{p}{P}df\hologo{TeX}%
}
%    \end{macrocode}
%    \end{macro}
%    \begin{macro}{\HoLogoHtml@pdfTeX}
%    \begin{macrocode}
\let\HoLogoHtml@pdfTeX\HoLogo@pdfTeX
%    \end{macrocode}
%    \end{macro}
%
%    \begin{macro}{\HoLogo@pdfLaTeX}
%    \begin{macrocode}
\def\HoLogo@pdfLaTeX#1{%
  \HOLOGO@mbox{%
    #1{p}{P}df\hologo{LaTeX}%
  }%
}
%    \end{macrocode}
%    \end{macro}
%    \begin{macro}{\HoLogoCs@pdfLaTeX}
%    \begin{macrocode}
\def\HoLogoCs@pdfLaTeX#1{#1{p}{P}dfLaTeX}
%    \end{macrocode}
%    \end{macro}
%    \begin{macro}{\HoLogoBkm@pdfLaTeX}
%    \begin{macrocode}
\def\HoLogoBkm@pdfLaTeX#1{%
  #1{p}{P}df\hologo{LaTeX}%
}
%    \end{macrocode}
%    \end{macro}
%    \begin{macro}{\HoLogoHtml@pdfLaTeX}
%    \begin{macrocode}
\let\HoLogoHtml@pdfLaTeX\HoLogo@pdfLaTeX
%    \end{macrocode}
%    \end{macro}
%
% \subsubsection{\hologo{VTeX}}
%
%    \begin{macro}{\HoLogo@VTeX}
%    \begin{macrocode}
\def\HoLogo@VTeX#1{%
  \HOLOGO@mbox{%
    V\hologo{TeX}%
  }%
}
%    \end{macrocode}
%    \end{macro}
%    \begin{macro}{\HoLogoHtml@VTeX}
%    \begin{macrocode}
\let\HoLogoHtml@VTeX\HoLogo@VTeX
%    \end{macrocode}
%    \end{macro}
%
% \subsubsection{\hologo{AmS}, \dots}
%
%    Source: class \xclass{amsdtx}
%
%    \begin{macro}{\HoLogo@AmS}
%    \begin{macrocode}
\def\HoLogo@AmS#1{%
  \HoLogoFont@font{AmS}{sy}{%
    A%
    \kern-.1667em%
    \lower.5ex\hbox{M}%
    \kern-.125em%
    S%
  }%
}
%    \end{macrocode}
%    \end{macro}
%    \begin{macro}{\HoLogoBkm@AmS}
%    \begin{macrocode}
\def\HoLogoBkm@AmS#1{AmS}
%    \end{macrocode}
%    \end{macro}
%    \begin{macro}{\HoLogoHtml@AmS}
%    \begin{macrocode}
\def\HoLogoHtml@AmS#1{%
  \HoLogoCss@AmS
%  \HoLogoFont@font{AmS}{sy}{%
    \HOLOGO@Span{AmS}{%
      A%
      \HOLOGO@Span{M}{M}%
      S%
    }%
%   }%
}
%    \end{macrocode}
%    \end{macro}
%    \begin{macro}{\HoLogoCss@AmS}
%    \begin{macrocode}
\def\HoLogoCss@AmS{%
  \Css{%
    span.HoLogo-AmS span.HoLogo-M{%
      position:relative;%
      top:.5ex;%
      margin-left:-.1667em;%
      margin-right:-.125em;%
      text-decoration:none;%
    }%
  }%
  \global\let\HoLogoCss@AmS\relax
}
%    \end{macrocode}
%    \end{macro}
%
%    \begin{macro}{\HoLogo@AmSTeX}
%    \begin{macrocode}
\def\HoLogo@AmSTeX#1{%
  \hologo{AmS}%
  \HOLOGO@hyphen
  \hologo{TeX}%
}
%    \end{macrocode}
%    \end{macro}
%    \begin{macro}{\HoLogoBkm@AmSTeX}
%    \begin{macrocode}
\def\HoLogoBkm@AmSTeX#1{AmS-TeX}%
%    \end{macrocode}
%    \end{macro}
%    \begin{macro}{\HoLogoHtml@AmSTeX}
%    \begin{macrocode}
\let\HoLogoHtml@AmSTeX\HoLogo@AmSTeX
%    \end{macrocode}
%    \end{macro}
%
%    \begin{macro}{\HoLogo@AmSLaTeX}
%    \begin{macrocode}
\def\HoLogo@AmSLaTeX#1{%
  \hologo{AmS}%
  \HOLOGO@hyphen
  \hologo{LaTeX}%
}
%    \end{macrocode}
%    \end{macro}
%    \begin{macro}{\HoLogoBkm@AmSLaTeX}
%    \begin{macrocode}
\def\HoLogoBkm@AmSLaTeX#1{AmS-LaTeX}%
%    \end{macrocode}
%    \end{macro}
%    \begin{macro}{\HoLogoHtml@AmSLaTeX}
%    \begin{macrocode}
\let\HoLogoHtml@AmSLaTeX\HoLogo@AmSLaTeX
%    \end{macrocode}
%    \end{macro}
%
% \subsubsection{\hologo{BibTeX}}
%
%    \begin{macro}{\HoLogo@BibTeX@sc}
%    A definition of \hologo{BibTeX} is provided in
%    the documentation source for the manual of \hologo{BibTeX}
%    \cite{btxdoc}.
%\begin{quote}
%\begin{verbatim}
%\def\BibTeX{%
%  {%
%    \rm
%    B%
%    \kern-.05em%
%    {%
%      \sc
%      i%
%      \kern-.025em %
%      b%
%    }%
%    \kern-.08em
%    T%
%    \kern-.1667em%
%    \lower.7ex\hbox{E}%
%    \kern-.125em%
%    X%
%  }%
%}
%\end{verbatim}
%\end{quote}
%    \begin{macrocode}
\def\HoLogo@BibTeX@sc#1{%
  B%
  \kern-.05em%
  \HoLogoFont@font{BibTeX}{sc}{%
    i%
    \kern-.025em%
    b%
  }%
  \HOLOGO@discretionary
  \kern-.08em%
  \hologo{TeX}%
}
%    \end{macrocode}
%    \end{macro}
%    \begin{macro}{\HoLogoHtml@BibTeX@sc}
%    \begin{macrocode}
\def\HoLogoHtml@BibTeX@sc#1{%
  \HoLogoCss@BibTeX@sc
  \HOLOGO@Span{BibTeX-sc}{%
    B%
    \HOLOGO@Span{i}{i}%
    \HOLOGO@Span{b}{b}%
    \hologo{TeX}%
  }%
}
%    \end{macrocode}
%    \end{macro}
%    \begin{macro}{\HoLogoCss@BibTeX@sc}
%    \begin{macrocode}
\def\HoLogoCss@BibTeX@sc{%
  \Css{%
    span.HoLogo-BibTeX-sc span.HoLogo-i{%
      margin-left:-.05em;%
      margin-right:-.025em;%
      font-variant:small-caps;%
    }%
  }%
  \Css{%
    span.HoLogo-BibTeX-sc span.HoLogo-b{%
      margin-right:-.08em;%
      font-variant:small-caps;%
    }%
  }%
  \global\let\HoLogoCss@BibTeX@sc\relax
}
%    \end{macrocode}
%    \end{macro}
%
%    \begin{macro}{\HoLogo@BibTeX@sf}
%    Variant \xoption{sf} avoids trouble with unavailable
%    small caps fonts (e.g., bold versions of Computer Modern or
%    Latin Modern). The definition is taken from
%    package \xpackage{dtklogos} \cite{dtklogos}.
%\begin{quote}
%\begin{verbatim}
%\DeclareRobustCommand{\BibTeX}{%
%  B%
%  \kern-.05em%
%  \hbox{%
%    $\m@th$% %% force math size calculations
%    \csname S@\f@size\endcsname
%    \fontsize\sf@size\z@
%    \math@fontsfalse
%    \selectfont
%    I%
%    \kern-.025em%
%    B
%  }%
%  \kern-.08em%
%  \-%
%  \TeX
%}
%\end{verbatim}
%\end{quote}
%    \begin{macrocode}
\def\HoLogo@BibTeX@sf#1{%
  B%
  \kern-.05em%
  \HoLogoFont@font{BibTeX}{bibsf}{%
    I%
    \kern-.025em%
    B%
  }%
  \HOLOGO@discretionary
  \kern-.08em%
  \hologo{TeX}%
}
%    \end{macrocode}
%    \end{macro}
%    \begin{macro}{\HoLogoHtml@BibTeX@sf}
%    \begin{macrocode}
\def\HoLogoHtml@BibTeX@sf#1{%
  \HoLogoCss@BibTeX@sf
  \HOLOGO@Span{BibTeX-sf}{%
    B%
    \HoLogoFont@font{BibTeX}{bibsf}{%
      \HOLOGO@Span{i}{I}%
      B%
    }%
    \hologo{TeX}%
  }%
}
%    \end{macrocode}
%    \end{macro}
%    \begin{macro}{\HoLogoCss@BibTeX@sf}
%    \begin{macrocode}
\def\HoLogoCss@BibTeX@sf{%
  \Css{%
    span.HoLogo-BibTeX-sf span.HoLogo-i{%
      margin-left:-.05em;%
      margin-right:-.025em;%
    }%
  }%
  \Css{%
    span.HoLogo-BibTeX-sf span.HoLogo-TeX{%
      margin-left:-.08em;%
    }%
  }%
  \global\let\HoLogoCss@BibTeX@sf\relax
}
%    \end{macrocode}
%    \end{macro}
%
%    \begin{macro}{\HoLogo@BibTeX}
%    \begin{macrocode}
\def\HoLogo@BibTeX{\HoLogo@BibTeX@sf}
%    \end{macrocode}
%    \end{macro}
%    \begin{macro}{\HoLogoHtml@BibTeX}
%    \begin{macrocode}
\def\HoLogoHtml@BibTeX{\HoLogoHtml@BibTeX@sf}
%    \end{macrocode}
%    \end{macro}
%
% \subsubsection{\hologo{BibTeX8}}
%
%    \begin{macro}{\HoLogo@BibTeX8}
%    \begin{macrocode}
\expandafter\def\csname HoLogo@BibTeX8\endcsname#1{%
  \hologo{BibTeX}%
  8%
}
%    \end{macrocode}
%    \end{macro}
%
%    \begin{macro}{\HoLogoBkm@BibTeX8}
%    \begin{macrocode}
\expandafter\def\csname HoLogoBkm@BibTeX8\endcsname#1{%
  \hologo{BibTeX}%
  8%
}
%    \end{macrocode}
%    \end{macro}
%    \begin{macro}{\HoLogoHtml@BibTeX8}
%    \begin{macrocode}
\expandafter
\let\csname HoLogoHtml@BibTeX8\expandafter\endcsname
\csname HoLogo@BibTeX8\endcsname
%    \end{macrocode}
%    \end{macro}
%
% \subsubsection{\hologo{ConTeXt}}
%
%    \begin{macro}{\HoLogo@ConTeXt@simple}
%    \begin{macrocode}
\def\HoLogo@ConTeXt@simple#1{%
  \HOLOGO@mbox{Con}%
  \HOLOGO@discretionary
  \HOLOGO@mbox{\hologo{TeX}t}%
}
%    \end{macrocode}
%    \end{macro}
%    \begin{macro}{\HoLogoHtml@ConTeXt@simple}
%    \begin{macrocode}
\let\HoLogoHtml@ConTeXt@simple\HoLogo@ConTeXt@simple
%    \end{macrocode}
%    \end{macro}
%
%    \begin{macro}{\HoLogo@ConTeXt@narrow}
%    This definition of logo \hologo{ConTeXt} with variant \xoption{narrow}
%    comes from TUGboat's class \xclass{ltugboat} (version 2010/11/15 v2.8).
%    \begin{macrocode}
\def\HoLogo@ConTeXt@narrow#1{%
  \HOLOGO@mbox{C\kern-.0333emon}%
  \HOLOGO@discretionary
  \kern-.0667em%
  \HOLOGO@mbox{\hologo{TeX}\kern-.0333emt}%
}
%    \end{macrocode}
%    \end{macro}
%    \begin{macro}{\HoLogoHtml@ConTeXt@narrow}
%    \begin{macrocode}
\def\HoLogoHtml@ConTeXt@narrow#1{%
  \HoLogoCss@ConTeXt@narrow
  \HOLOGO@Span{ConTeXt-narrow}{%
    \HOLOGO@Span{C}{C}%
    on%
    \hologo{TeX}%
    t%
  }%
}
%    \end{macrocode}
%    \end{macro}
%    \begin{macro}{\HoLogoCss@ConTeXt@narrow}
%    \begin{macrocode}
\def\HoLogoCss@ConTeXt@narrow{%
  \Css{%
    span.HoLogo-ConTeXt-narrow span.HoLogo-C{%
      margin-left:-.0333em;%
    }%
  }%
  \Css{%
    span.HoLogo-ConTeXt-narrow span.HoLogo-TeX{%
      margin-left:-.0667em;%
      margin-right:-.0333em;%
    }%
  }%
  \global\let\HoLogoCss@ConTeXt@narrow\relax
}
%    \end{macrocode}
%    \end{macro}
%
%    \begin{macro}{\HoLogo@ConTeXt}
%    \begin{macrocode}
\def\HoLogo@ConTeXt{\HoLogo@ConTeXt@narrow}
%    \end{macrocode}
%    \end{macro}
%    \begin{macro}{\HoLogoHtml@ConTeXt}
%    \begin{macrocode}
\def\HoLogoHtml@ConTeXt{\HoLogoHtml@ConTeXt@narrow}
%    \end{macrocode}
%    \end{macro}
%
% \subsubsection{\hologo{emTeX}}
%
%    \begin{macro}{\HoLogo@emTeX}
%    \begin{macrocode}
\def\HoLogo@emTeX#1{%
  \HOLOGO@mbox{#1{e}{E}m}%
  \HOLOGO@discretionary
  \hologo{TeX}%
}
%    \end{macrocode}
%    \end{macro}
%    \begin{macro}{\HoLogoCs@emTeX}
%    \begin{macrocode}
\def\HoLogoCs@emTeX#1{#1{e}{E}mTeX}%
%    \end{macrocode}
%    \end{macro}
%    \begin{macro}{\HoLogoBkm@emTeX}
%    \begin{macrocode}
\def\HoLogoBkm@emTeX#1{%
  #1{e}{E}m\hologo{TeX}%
}
%    \end{macrocode}
%    \end{macro}
%    \begin{macro}{\HoLogoHtml@emTeX}
%    \begin{macrocode}
\let\HoLogoHtml@emTeX\HoLogo@emTeX
%    \end{macrocode}
%    \end{macro}
%
% \subsubsection{\hologo{ExTeX}}
%
%    \begin{macro}{\HoLogo@ExTeX}
%    The definition is taken from the FAQ of the
%    project \hologo{ExTeX}
%    \cite{ExTeX-FAQ}.
%\begin{quote}
%\begin{verbatim}
%\def\ExTeX{%
%  \textrm{% Logo always with serifs
%    \ensuremath{%
%      \textstyle
%      \varepsilon_{%
%        \kern-0.15em%
%        \mathcal{X}%
%      }%
%    }%
%    \kern-.15em%
%    \TeX
%  }%
%}
%\end{verbatim}
%\end{quote}
%    \begin{macrocode}
\def\HoLogo@ExTeX#1{%
  \HoLogoFont@font{ExTeX}{rm}{%
    \ltx@mbox{%
      \HOLOGO@MathSetup
      $%
        \textstyle
        \varepsilon_{%
          \kern-0.15em%
          \HoLogoFont@font{ExTeX}{sy}{X}%
        }%
      $%
    }%
    \HOLOGO@discretionary
    \kern-.15em%
    \hologo{TeX}%
  }%
}
%    \end{macrocode}
%    \end{macro}
%    \begin{macro}{\HoLogoHtml@ExTeX}
%    \begin{macrocode}
\def\HoLogoHtml@ExTeX#1{%
  \HoLogoCss@ExTeX
  \HoLogoFont@font{ExTeX}{rm}{%
    \HOLOGO@Span{ExTeX}{%
      \ltx@mbox{%
        \HOLOGO@MathSetup
        $\textstyle\varepsilon$%
        \HOLOGO@Span{X}{$\textstyle\chi$}%
        \hologo{TeX}%
      }%
    }%
  }%
}
%    \end{macrocode}
%    \end{macro}
%    \begin{macro}{\HoLogoBkm@ExTeX}
%    \begin{macrocode}
\def\HoLogoBkm@ExTeX#1{%
  \HOLOGO@PdfdocUnicode{#1{e}{E}x}{\textepsilon\textchi}%
  \hologo{TeX}%
}
%    \end{macrocode}
%    \end{macro}
%    \begin{macro}{\HoLogoCss@ExTeX}
%    \begin{macrocode}
\def\HoLogoCss@ExTeX{%
  \Css{%
    span.HoLogo-ExTeX{%
      font-family:serif;%
    }%
  }%
  \Css{%
    span.HoLogo-ExTeX span.HoLogo-TeX{%
      margin-left:-.15em;%
    }%
  }%
  \global\let\HoLogoCss@ExTeX\relax
}
%    \end{macrocode}
%    \end{macro}
%
% \subsubsection{\hologo{MiKTeX}}
%
%    \begin{macro}{\HoLogo@MiKTeX}
%    \begin{macrocode}
\def\HoLogo@MiKTeX#1{%
  \HOLOGO@mbox{MiK}%
  \HOLOGO@discretionary
  \hologo{TeX}%
}
%    \end{macrocode}
%    \end{macro}
%    \begin{macro}{\HoLogoHtml@MiKTeX}
%    \begin{macrocode}
\let\HoLogoHtml@MiKTeX\HoLogo@MiKTeX
%    \end{macrocode}
%    \end{macro}
%
% \subsubsection{\hologo{OzTeX} and friends}
%
%    Source: \hologo{OzTeX} FAQ \cite{OzTeX}:
%    \begin{quote}
%      |\def\OzTeX{O\kern-.03em z\kern-.15em\TeX}|\\
%      (There is no kerning in OzMF, OzMP and OzTtH.)
%    \end{quote}
%
%    \begin{macro}{\HoLogo@OzTeX}
%    \begin{macrocode}
\def\HoLogo@OzTeX#1{%
  O%
  \kern-.03em %
  z%
  \kern-.15em %
  \hologo{TeX}%
}
%    \end{macrocode}
%    \end{macro}
%    \begin{macro}{\HoLogoHtml@OzTeX}
%    \begin{macrocode}
\def\HoLogoHtml@OzTeX#1{%
  \HoLogoCss@OzTeX
  \HOLOGO@Span{OzTeX}{%
    O%
    \HOLOGO@Span{z}{z}%
    \hologo{TeX}%
  }%
}
%    \end{macrocode}
%    \end{macro}
%    \begin{macro}{\HoLogoCss@OzTeX}
%    \begin{macrocode}
\def\HoLogoCss@OzTeX{%
  \Css{%
    span.HoLogo-OzTeX span.HoLogo-z{%
      margin-left:-.03em;%
      margin-right:-.15em;%
    }%
  }%
  \global\let\HoLogoCss@OzTeX\relax
}
%    \end{macrocode}
%    \end{macro}
%
%    \begin{macro}{\HoLogo@OzMF}
%    \begin{macrocode}
\def\HoLogo@OzMF#1{%
  \HOLOGO@mbox{OzMF}%
}
%    \end{macrocode}
%    \end{macro}
%    \begin{macro}{\HoLogo@OzMP}
%    \begin{macrocode}
\def\HoLogo@OzMP#1{%
  \HOLOGO@mbox{OzMP}%
}
%    \end{macrocode}
%    \end{macro}
%    \begin{macro}{\HoLogo@OzTtH}
%    \begin{macrocode}
\def\HoLogo@OzTtH#1{%
  \HOLOGO@mbox{OzTtH}%
}
%    \end{macrocode}
%    \end{macro}
%
% \subsubsection{\hologo{PCTeX}}
%
%    \begin{macro}{\HoLogo@PCTeX}
%    \begin{macrocode}
\def\HoLogo@PCTeX#1{%
  \HOLOGO@mbox{PC}%
  \hologo{TeX}%
}
%    \end{macrocode}
%    \end{macro}
%    \begin{macro}{\HoLogoHtml@PCTeX}
%    \begin{macrocode}
\let\HoLogoHtml@PCTeX\HoLogo@PCTeX
%    \end{macrocode}
%    \end{macro}
%
% \subsubsection{\hologo{PiCTeX}}
%
%    The original definitions from \xfile{pictex.tex} \cite{PiCTeX}:
%\begin{quote}
%\begin{verbatim}
%\def\PiC{%
%  P%
%  \kern-.12em%
%  \lower.5ex\hbox{I}%
%  \kern-.075em%
%  C%
%}
%\def\PiCTeX{%
%  \PiC
%  \kern-.11em%
%  \TeX
%}
%\end{verbatim}
%\end{quote}
%
%    \begin{macro}{\HoLogo@PiC}
%    \begin{macrocode}
\def\HoLogo@PiC#1{%
  P%
  \kern-.12em%
  \lower.5ex\hbox{I}%
  \kern-.075em%
  C%
  \HOLOGO@SpaceFactor
}
%    \end{macrocode}
%    \end{macro}
%    \begin{macro}{\HoLogoHtml@PiC}
%    \begin{macrocode}
\def\HoLogoHtml@PiC#1{%
  \HoLogoCss@PiC
  \HOLOGO@Span{PiC}{%
    P%
    \HOLOGO@Span{i}{I}%
    C%
  }%
}
%    \end{macrocode}
%    \end{macro}
%    \begin{macro}{\HoLogoCss@PiC}
%    \begin{macrocode}
\def\HoLogoCss@PiC{%
  \Css{%
    span.HoLogo-PiC span.HoLogo-i{%
      position:relative;%
      top:.5ex;%
      margin-left:-.12em;%
      margin-right:-.075em;%
      text-decoration:none;%
    }%
  }%
  \global\let\HoLogoCss@PiC\relax
}
%    \end{macrocode}
%    \end{macro}
%
%    \begin{macro}{\HoLogo@PiCTeX}
%    \begin{macrocode}
\def\HoLogo@PiCTeX#1{%
  \hologo{PiC}%
  \HOLOGO@discretionary
  \kern-.11em%
  \hologo{TeX}%
}
%    \end{macrocode}
%    \end{macro}
%    \begin{macro}{\HoLogoHtml@PiCTeX}
%    \begin{macrocode}
\def\HoLogoHtml@PiCTeX#1{%
  \HoLogoCss@PiCTeX
  \HOLOGO@Span{PiCTeX}{%
    \hologo{PiC}%
    \hologo{TeX}%
  }%
}
%    \end{macrocode}
%    \end{macro}
%    \begin{macro}{\HoLogoCss@PiCTeX}
%    \begin{macrocode}
\def\HoLogoCss@PiCTeX{%
  \Css{%
    span.HoLogo-PiCTeX span.HoLogo-PiC{%
      margin-right:-.11em;%
    }%
  }%
  \global\let\HoLogoCss@PiCTeX\relax
}
%    \end{macrocode}
%    \end{macro}
%
% \subsubsection{\hologo{teTeX}}
%
%    \begin{macro}{\HoLogo@teTeX}
%    \begin{macrocode}
\def\HoLogo@teTeX#1{%
  \HOLOGO@mbox{#1{t}{T}e}%
  \HOLOGO@discretionary
  \hologo{TeX}%
}
%    \end{macrocode}
%    \end{macro}
%    \begin{macro}{\HoLogoCs@teTeX}
%    \begin{macrocode}
\def\HoLogoCs@teTeX#1{#1{t}{T}dfTeX}
%    \end{macrocode}
%    \end{macro}
%    \begin{macro}{\HoLogoBkm@teTeX}
%    \begin{macrocode}
\def\HoLogoBkm@teTeX#1{%
  #1{t}{T}e\hologo{TeX}%
}
%    \end{macrocode}
%    \end{macro}
%    \begin{macro}{\HoLogoHtml@teTeX}
%    \begin{macrocode}
\let\HoLogoHtml@teTeX\HoLogo@teTeX
%    \end{macrocode}
%    \end{macro}
%
% \subsubsection{\hologo{TeX4ht}}
%
%    \begin{macro}{\HoLogo@TeX4ht}
%    \begin{macrocode}
\expandafter\def\csname HoLogo@TeX4ht\endcsname#1{%
  \HOLOGO@mbox{\hologo{TeX}4ht}%
}
%    \end{macrocode}
%    \end{macro}
%    \begin{macro}{\HoLogoHtml@TeX4ht}
%    \begin{macrocode}
\expandafter
\let\csname HoLogoHtml@TeX4ht\expandafter\endcsname
\csname HoLogo@TeX4ht\endcsname
%    \end{macrocode}
%    \end{macro}
%
%
% \subsubsection{\hologo{SageTeX}}
%
%    \begin{macro}{\HoLogo@SageTeX}
%    \begin{macrocode}
\def\HoLogo@SageTeX#1{%
  \HOLOGO@mbox{Sage}%
  \HOLOGO@discretionary
  \HOLOGO@NegativeKerning{eT,oT,To}%
  \hologo{TeX}%
}
%    \end{macrocode}
%    \end{macro}
%    \begin{macro}{\HoLogoHtml@SageTeX}
%    \begin{macrocode}
\let\HoLogoHtml@SageTeX\HoLogo@SageTeX
%    \end{macrocode}
%    \end{macro}
%
% \subsection{\hologo{METAFONT} and friends}
%
%    \begin{macro}{\HoLogo@METAFONT}
%    \begin{macrocode}
\def\HoLogo@METAFONT#1{%
  \HoLogoFont@font{METAFONT}{logo}{%
    \HOLOGO@mbox{META}%
    \HOLOGO@discretionary
    \HOLOGO@mbox{FONT}%
  }%
}
%    \end{macrocode}
%    \end{macro}
%
%    \begin{macro}{\HoLogo@METAPOST}
%    \begin{macrocode}
\def\HoLogo@METAPOST#1{%
  \HoLogoFont@font{METAPOST}{logo}{%
    \HOLOGO@mbox{META}%
    \HOLOGO@discretionary
    \HOLOGO@mbox{POST}%
  }%
}
%    \end{macrocode}
%    \end{macro}
%
%    \begin{macro}{\HoLogo@MetaFun}
%    \begin{macrocode}
\def\HoLogo@MetaFun#1{%
  \HOLOGO@mbox{Meta}%
  \HOLOGO@discretionary
  \HOLOGO@mbox{Fun}%
}
%    \end{macrocode}
%    \end{macro}
%
%    \begin{macro}{\HoLogo@MetaPost}
%    \begin{macrocode}
\def\HoLogo@MetaPost#1{%
  \HOLOGO@mbox{Meta}%
  \HOLOGO@discretionary
  \HOLOGO@mbox{Post}%
}
%    \end{macrocode}
%    \end{macro}
%
% \subsection{Others}
%
% \subsubsection{\hologo{biber}}
%
%    \begin{macro}{\HoLogo@biber}
%    \begin{macrocode}
\def\HoLogo@biber#1{%
  \HOLOGO@mbox{#1{b}{B}i}%
  \HOLOGO@discretionary
  \HOLOGO@mbox{ber}%
}
%    \end{macrocode}
%    \end{macro}
%    \begin{macro}{\HoLogoCs@biber}
%    \begin{macrocode}
\def\HoLogoCs@biber#1{#1{b}{B}iber}
%    \end{macrocode}
%    \end{macro}
%    \begin{macro}{\HoLogoBkm@biber}
%    \begin{macrocode}
\def\HoLogoBkm@biber#1{%
  #1{b}{B}iber%
}
%    \end{macrocode}
%    \end{macro}
%    \begin{macro}{\HoLogoHtml@biber}
%    \begin{macrocode}
\let\HoLogoHtml@biber\HoLogo@biber
%    \end{macrocode}
%    \end{macro}
%
% \subsubsection{\hologo{KOMAScript}}
%
%    \begin{macro}{\HoLogo@KOMAScript}
%    The definition for \hologo{KOMAScript} is taken
%    from \hologo{KOMAScript} (\xfile{scrlogo.dtx}, reformatted) \cite{scrlogo}:
%\begin{quote}
%\begin{verbatim}
%\@ifundefined{KOMAScript}{%
%  \DeclareRobustCommand{\KOMAScript}{%
%    \textsf{%
%      K\kern.05em O\kern.05emM\kern.05em A%
%      \kern.1em-\kern.1em %
%      Script%
%    }%
%  }%
%}{}
%\end{verbatim}
%\end{quote}
%    \begin{macrocode}
\def\HoLogo@KOMAScript#1{%
  \HoLogoFont@font{KOMAScript}{sf}{%
    \HOLOGO@mbox{%
      K\kern.05em%
      O\kern.05em%
      M\kern.05em%
      A%
    }%
    \kern.1em%
    \HOLOGO@hyphen
    \kern.1em%
    \HOLOGO@mbox{Script}%
  }%
}
%    \end{macrocode}
%    \end{macro}
%    \begin{macro}{\HoLogoBkm@KOMAScript}
%    \begin{macrocode}
\def\HoLogoBkm@KOMAScript#1{%
  KOMA-Script%
}
%    \end{macrocode}
%    \end{macro}
%    \begin{macro}{\HoLogoHtml@KOMAScript}
%    \begin{macrocode}
\def\HoLogoHtml@KOMAScript#1{%
  \HoLogoCss@KOMAScript
  \HoLogoFont@font{KOMAScript}{sf}{%
    \HOLOGO@Span{KOMAScript}{%
      K%
      \HOLOGO@Span{O}{O}%
      M%
      \HOLOGO@Span{A}{A}%
      \HOLOGO@Span{hyphen}{-}%
      Script%
    }%
  }%
}
%    \end{macrocode}
%    \end{macro}
%    \begin{macro}{\HoLogoCss@KOMAScript}
%    \begin{macrocode}
\def\HoLogoCss@KOMAScript{%
  \Css{%
    span.HoLogo-KOMAScript{%
      font-family:sans-serif;%
    }%
  }%
  \Css{%
    span.HoLogo-KOMAScript span.HoLogo-O{%
      padding-left:.05em;%
      padding-right:.05em;%
    }%
  }%
  \Css{%
    span.HoLogo-KOMAScript span.HoLogo-A{%
      padding-left:.05em;%
    }%
  }%
  \Css{%
    span.HoLogo-KOMAScript span.HoLogo-hyphen{%
      padding-left:.1em;%
      padding-right:.1em;%
    }%
  }%
  \global\let\HoLogoCss@KOMAScript\relax
}
%    \end{macrocode}
%    \end{macro}
%
% \subsubsection{\hologo{LyX}}
%
%    \begin{macro}{\HoLogo@LyX}
%    The definition is taken from the documentation source files
%    of \hologo{LyX}, \xfile{Intro.lyx} \cite{LyX}:
%\begin{quote}
%\begin{verbatim}
%\def\LyX{%
%  \texorpdfstring{%
%    L\kern-.1667em\lower.25em\hbox{Y}\kern-.125emX\@%
%  }{%
%    LyX%
%  }%
%}
%\end{verbatim}
%\end{quote}
%    \begin{macrocode}
\def\HoLogo@LyX#1{%
  L%
  \kern-.1667em%
  \lower.25em\hbox{Y}%
  \kern-.125em%
  X%
  \HOLOGO@SpaceFactor
}
%    \end{macrocode}
%    \end{macro}
%    \begin{macro}{\HoLogoHtml@LyX}
%    \begin{macrocode}
\def\HoLogoHtml@LyX#1{%
  \HoLogoCss@LyX
  \HOLOGO@Span{LyX}{%
    L%
    \HOLOGO@Span{y}{Y}%
    X%
  }%
}
%    \end{macrocode}
%    \end{macro}
%    \begin{macro}{\HoLogoCss@LyX}
%    \begin{macrocode}
\def\HoLogoCss@LyX{%
  \Css{%
    span.HoLogo-LyX span.HoLogo-y{%
      position:relative;%
      top:.25em;%
      margin-left:-.1667em;%
      margin-right:-.125em;%
      text-decoration:none;%
    }%
  }%
  \global\let\HoLogoCss@LyX\relax
}
%    \end{macrocode}
%    \end{macro}
%
% \subsubsection{\hologo{NTS}}
%
%    \begin{macro}{\HoLogo@NTS}
%    Definition for \hologo{NTS} can be found in
%    package \xpackage{etex\textunderscore man} for the \hologo{eTeX} manual \cite{etexman}
%    and in package \xpackage{dtklogos} \cite{dtklogos}:
%\begin{quote}
%\begin{verbatim}
%\def\NTS{%
%  \leavevmode
%  \hbox{%
%    $%
%      \cal N%
%      \kern-0.35em%
%      \lower0.5ex\hbox{$\cal T$}%
%      \kern-0.2em%
%      S%
%    $%
%  }%
%}
%\end{verbatim}
%\end{quote}
%    \begin{macrocode}
\def\HoLogo@NTS#1{%
  \HoLogoFont@font{NTS}{sy}{%
    N\/%
    \kern-.35em%
    \lower.5ex\hbox{T\/}%
    \kern-.2em%
    S\/%
  }%
  \HOLOGO@SpaceFactor
}
%    \end{macrocode}
%    \end{macro}
%
% \subsubsection{\Hologo{TTH} (\hologo{TeX} to HTML translator)}
%
%    Source: \url{http://hutchinson.belmont.ma.us/tth/}
%    In the HTML source the second `T' is printed as subscript.
%\begin{quote}
%\begin{verbatim}
%T<sub>T</sub>H
%\end{verbatim}
%\end{quote}
%    \begin{macro}{\HoLogo@TTH}
%    \begin{macrocode}
\def\HoLogo@TTH#1{%
  \ltx@mbox{%
    T\HOLOGO@SubScript{T}H%
  }%
  \HOLOGO@SpaceFactor
}
%    \end{macrocode}
%    \end{macro}
%
%    \begin{macro}{\HoLogoHtml@TTH}
%    \begin{macrocode}
\def\HoLogoHtml@TTH#1{%
  T\HCode{<sub>}T\HCode{</sub>}H%
}
%    \end{macrocode}
%    \end{macro}
%
% \subsubsection{\Hologo{HanTheThanh}}
%
%    Partial source: Package \xpackage{dtklogos}.
%    The double accent is U+1EBF (latin small letter e with circumflex
%    and acute).
%    \begin{macro}{\HoLogo@HanTheThanh}
%    \begin{macrocode}
\def\HoLogo@HanTheThanh#1{%
  \ltx@mbox{H\`an}%
  \HOLOGO@space
  \ltx@mbox{%
    Th%
    \HOLOGO@IfCharExists{"1EBF}{%
      \char"1EBF\relax
    }{%
      \^e\hbox to 0pt{\hss\raise .5ex\hbox{\'{}}}%
    }%
  }%
  \HOLOGO@space
  \ltx@mbox{Th\`anh}%
}
%    \end{macrocode}
%    \end{macro}
%    \begin{macro}{\HoLogoBkm@HanTheThanh}
%    \begin{macrocode}
\def\HoLogoBkm@HanTheThanh#1{%
  H\`an %
  Th\HOLOGO@PdfdocUnicode{\^e}{\9036\277} %
  Th\`anh%
}
%    \end{macrocode}
%    \end{macro}
%    \begin{macro}{\HoLogoHtml@HanTheThanh}
%    \begin{macrocode}
\def\HoLogoHtml@HanTheThanh#1{%
  H\`an %
  Th\HCode{&\ltx@hashchar x1ebf;} %
  Th\`anh%
}
%    \end{macrocode}
%    \end{macro}
%
% \subsection{Driver detection}
%
%    \begin{macrocode}
\HOLOGO@IfExists\InputIfFileExists{%
  \InputIfFileExists{hologo.cfg}{}{}%
}{%
  \ltx@IfUndefined{pdf@filesize}{%
    \def\HOLOGO@InputIfExists{%
      \openin\HOLOGO@temp=hologo.cfg\relax
      \ifeof\HOLOGO@temp
        \closein\HOLOGO@temp
      \else
        \closein\HOLOGO@temp
        \begingroup
          \def\x{LaTeX2e}%
        \expandafter\endgroup
        \ifx\fmtname\x
          % \iffalse meta-comment
%
% File: hologo.dtx
% Version: 2016/05/12 v1.11
% Info: A logo collection with bookmark support
%
% Copyright (C) 2010-2012 by
%    Heiko Oberdiek <heiko.oberdiek at googlemail.com>
%
% This work may be distributed and/or modified under the
% conditions of the LaTeX Project Public License, either
% version 1.3c of this license or (at your option) any later
% version. This version of this license is in
%    http://www.latex-project.org/lppl/lppl-1-3c.txt
% and the latest version of this license is in
%    http://www.latex-project.org/lppl.txt
% and version 1.3 or later is part of all distributions of
% LaTeX version 2005/12/01 or later.
%
% This work has the LPPL maintenance status "maintained".
%
% This Current Maintainer of this work is Heiko Oberdiek.
%
% The Base Interpreter refers to any `TeX-Format',
% because some files are installed in TDS:tex/generic//.
%
% This work consists of the main source file hologo.dtx
% and the derived files
%    hologo.sty, hologo.pdf, hologo.ins, hologo.drv, hologo-example.tex,
%    hologo-test1.tex, hologo-test-spacefactor.tex,
%    hologo-test-list.tex.
%
% Distribution:
%    CTAN:macros/latex/contrib/oberdiek/hologo.dtx
%    CTAN:macros/latex/contrib/oberdiek/hologo.pdf
%
% Unpacking:
%    (a) If hologo.ins is present:
%           tex hologo.ins
%    (b) Without hologo.ins:
%           tex hologo.dtx
%    (c) If you insist on using LaTeX
%           latex \let\install=y\input{hologo.dtx}
%        (quote the arguments according to the demands of your shell)
%
% Documentation:
%    (a) If hologo.drv is present:
%           latex hologo.drv
%    (b) Without hologo.drv:
%           latex hologo.dtx; ...
%    The class ltxdoc loads the configuration file ltxdoc.cfg
%    if available. Here you can specify further options, e.g.
%    use A4 as paper format:
%       \PassOptionsToClass{a4paper}{article}
%
%    Programm calls to get the documentation (example):
%       pdflatex hologo.dtx
%       makeindex -s gind.ist hologo.idx
%       pdflatex hologo.dtx
%       makeindex -s gind.ist hologo.idx
%       pdflatex hologo.dtx
%
% Installation:
%    TDS:tex/generic/oberdiek/hologo.sty
%    TDS:doc/latex/oberdiek/hologo.pdf
%    TDS:doc/latex/oberdiek/example/hologo-example.tex
%    TDS:doc/latex/oberdiek/test/hologo-test1.tex
%    TDS:doc/latex/oberdiek/test/hologo-test-spacefactor.tex
%    TDS:doc/latex/oberdiek/test/hologo-test-list.tex
%    TDS:source/latex/oberdiek/hologo.dtx
%
%<*ignore>
\begingroup
  \catcode123=1 %
  \catcode125=2 %
  \def\x{LaTeX2e}%
\expandafter\endgroup
\ifcase 0\ifx\install y1\fi\expandafter
         \ifx\csname processbatchFile\endcsname\relax\else1\fi
         \ifx\fmtname\x\else 1\fi\relax
\else\csname fi\endcsname
%</ignore>
%<*install>
\input docstrip.tex
\Msg{************************************************************************}
\Msg{* Installation}
\Msg{* Package: hologo 2016/05/12 v1.11 A logo collection with bookmark support (HO)}
\Msg{************************************************************************}

\keepsilent
\askforoverwritefalse

\let\MetaPrefix\relax
\preamble

This is a generated file.

Project: hologo
Version: 2016/05/12 v1.11

Copyright (C) 2010-2012 by
   Heiko Oberdiek <heiko.oberdiek at googlemail.com>

This work may be distributed and/or modified under the
conditions of the LaTeX Project Public License, either
version 1.3c of this license or (at your option) any later
version. This version of this license is in
   http://www.latex-project.org/lppl/lppl-1-3c.txt
and the latest version of this license is in
   http://www.latex-project.org/lppl.txt
and version 1.3 or later is part of all distributions of
LaTeX version 2005/12/01 or later.

This work has the LPPL maintenance status "maintained".

This Current Maintainer of this work is Heiko Oberdiek.

The Base Interpreter refers to any `TeX-Format',
because some files are installed in TDS:tex/generic//.

This work consists of the main source file hologo.dtx
and the derived files
   hologo.sty, hologo.pdf, hologo.ins, hologo.drv, hologo-example.tex,
   hologo-test1.tex, hologo-test-spacefactor.tex,
   hologo-test-list.tex.

\endpreamble
\let\MetaPrefix\DoubleperCent

\generate{%
  \file{hologo.ins}{\from{hologo.dtx}{install}}%
  \file{hologo.drv}{\from{hologo.dtx}{driver}}%
  \usedir{tex/generic/oberdiek}%
  \file{hologo.sty}{\from{hologo.dtx}{package}}%
  \usedir{doc/latex/oberdiek/example}%
  \file{hologo-example.tex}{\from{hologo.dtx}{example}}%
  \usedir{doc/latex/oberdiek/test}%
  \file{hologo-test1.tex}{\from{hologo.dtx}{test1}}%
  \file{hologo-test-spacefactor.tex}{\from{hologo.dtx}{test-spacefactor}}%
  \file{hologo-test-list.tex}{\from{hologo.dtx}{test-list}}%
  \nopreamble
  \nopostamble
  \usedir{source/latex/oberdiek/catalogue}%
  \file{hologo.xml}{\from{hologo.dtx}{catalogue}}%
}

\catcode32=13\relax% active space
\let =\space%
\Msg{************************************************************************}
\Msg{*}
\Msg{* To finish the installation you have to move the following}
\Msg{* file into a directory searched by TeX:}
\Msg{*}
\Msg{*     hologo.sty}
\Msg{*}
\Msg{* To produce the documentation run the file `hologo.drv'}
\Msg{* through LaTeX.}
\Msg{*}
\Msg{* Happy TeXing!}
\Msg{*}
\Msg{************************************************************************}

\endbatchfile
%</install>
%<*ignore>
\fi
%</ignore>
%<*driver>
\NeedsTeXFormat{LaTeX2e}
\ProvidesFile{hologo.drv}%
  [2016/05/12 v1.11 A logo collection with bookmark support (HO)]%
\documentclass{ltxdoc}
\usepackage{holtxdoc}[2011/11/22]
\usepackage{hologo}[2016/05/12]
\usepackage{longtable}
\usepackage{array}
\usepackage{paralist}
%\usepackage[T1]{fontenc}
%\usepackage{lmodern}
\begin{document}
  \DocInput{hologo.dtx}%
\end{document}
%</driver>
% \fi
%
%
% \CharacterTable
%  {Upper-case    \A\B\C\D\E\F\G\H\I\J\K\L\M\N\O\P\Q\R\S\T\U\V\W\X\Y\Z
%   Lower-case    \a\b\c\d\e\f\g\h\i\j\k\l\m\n\o\p\q\r\s\t\u\v\w\x\y\z
%   Digits        \0\1\2\3\4\5\6\7\8\9
%   Exclamation   \!     Double quote  \"     Hash (number) \#
%   Dollar        \$     Percent       \%     Ampersand     \&
%   Acute accent  \'     Left paren    \(     Right paren   \)
%   Asterisk      \*     Plus          \+     Comma         \,
%   Minus         \-     Point         \.     Solidus       \/
%   Colon         \:     Semicolon     \;     Less than     \<
%   Equals        \=     Greater than  \>     Question mark \?
%   Commercial at \@     Left bracket  \[     Backslash     \\
%   Right bracket \]     Circumflex    \^     Underscore    \_
%   Grave accent  \`     Left brace    \{     Vertical bar  \|
%   Right brace   \}     Tilde         \~}
%
% \GetFileInfo{hologo.drv}
%
% \title{The \xpackage{hologo} package}
% \date{2016/05/12 v1.11}
% \author{Heiko Oberdiek\\\xemail{heiko.oberdiek at googlemail.com}}
%
% \maketitle
%
% \begin{abstract}
% This package starts a collection of logos with support for bookmarks
% strings.
% \end{abstract}
%
% \tableofcontents
%
% \section{Documentation}
%
% \subsection{Logo macros}
%
% \begin{declcs}{hologo} \M{name}
% \end{declcs}
% Macro \cs{hologo} sets the logo with name \meta{name}.
% The following table shows the supported names.
%
% \begingroup
%   \def\hologoEntry#1#2#3{^^A
%     #1&#2&\hologoLogoSetup{#1}{variant=#2}\hologo{#1}&#3\tabularnewline
%   }
%   \begin{longtable}{>{\ttfamily}l>{\ttfamily}lll}
%     \rmfamily\bfseries{name} & \rmfamily\bfseries variant
%     & \bfseries logo & \bfseries since\\
%     \hline
%     \endhead
%     \hologoList
%   \end{longtable}
% \endgroup
%
% \begin{declcs}{Hologo} \M{name}
% \end{declcs}
% Macro \cs{Hologo} starts the logo \meta{name} with an uppercase
% letter. As an exception small greek letters are not converted
% to uppercase. Examples, see \hologo{eTeX} and \hologo{ExTeX}.
%
% \subsection{Setup macros}
%
% The package does not support package options, but the following
% setup macros can be used to set options.
%
% \begin{declcs}{hologoSetup} \M{key value list}
% \end{declcs}
% Macro \cs{hologoSetup} sets global options.
%
% \begin{declcs}{hologoLogoSetup} \M{logo} \M{key value list}
% \end{declcs}
% Some options can also be used to configure a logo.
% These settings take precedence over global option settings.
%
% \subsection{Options}\label{sec:options}
%
% There are boolean and string options:
% \begin{description}
% \item[Boolean option:]
% It takes |true| or |false|
% as value. If the value is omitted, then |true| is used.
% \item[String option:]
% A value must be given as string. (But the string might be empty.)
% \end{description}
% The following options can be used both in \cs{hologoSetup}
% and \cs{hologoLogoSetup}:
% \begin{description}
% \def\entry#1{\item[\xoption{#1}:]}
% \entry{break}
%   enables or disables line breaks inside the logo. This setting is
%   refined by options \xoption{hyphenbreak}, \xoption{spacebreak}
%   or \xoption{discretionarybreak}.
%   Default is |false|.
% \entry{hyphenbreak}
%   enables or disables the line break right after the hyphen character.
% \entry{spacebreak}
%   enables or disables line breaks at space characters.
% \entry{discretionarybreak}
%   enables or disables line breaks at hyphenation points
%   (inserted by \cs{-}).
% \end{description}
% Macro \cs{hologoLogoSetup} also knows:
% \begin{description}
% \item[\xoption{variant}:]
%   This is a string option. It specifies a variant of a logo that
%   must exist. An empty string selects the package default variant.
% \end{description}
% Example:
% \begin{quote}
%   |\hologoSetup{break=false}|\\
%   |\hologoLogoSetup{plainTeX}{variant=hyphen,hyphenbreak}|\\
%   Then ``plain-\TeX'' contains one break point after the hyphen.
% \end{quote}
%
% \subsection{Driver options}
%
% Sometimes graphical operations are needed to construct some
% glyphs (e.g.\ \hologo{XeTeX}). If package \xpackage{graphics}
% or package \xpackage{pgf} are found, then the macros are taken
% from there. Otherwise the packge defines its own operations
% and therefore needs the driver information. Many drivers are
% detected automatically (\hologo{pdfTeX}/\hologo{LuaTeX}
% in PDF mode, \hologo{XeTeX}, \hologo{VTeX}). These have precedence
% over a driver option. The driver can be given as package option
% or using \cs{hologoDriverSetup}.
% The following list contains the recognized driver options:
% \begin{itemize}
% \item \xoption{pdftex}, \xoption{luatex}
% \item \xoption{dvipdfm}, \xoption{dvipdfmx}
% \item \xoption{dvips}, \xoption{dvipsone}, \xoption{xdvi}
% \item \xoption{xetex}
% \item \xoption{vtex}
% \end{itemize}
% The left driver of a line is the driver name that is used internally.
% The following names are aliases for drivers that use the
% same method. Therefore the entry in the \xext{log} file for
% the used driver prints the internally used driver name.
% \begin{description}
% \item[\xoption{driverfallback}:]
%   This option expects a driver that is used,
%   if the driver could not be detected automatically.
% \end{description}
%
% \begin{declcs}{hologoDriverSetup} \M{driver option}
% \end{declcs}
% The driver can also be configured after package loading
% using \cs{hologoDriverSetup}, also the way for \hologo{plainTeX}
% to setup the driver.
%
% \subsection{Font setup}
%
% Some logos require a special font, but should also be usable by
% \hologo{plainTeX}. Therefore the package provides some ways
% to influence the font settings. The options below
% take font settings as values. Both font commands
% such as \cs{sffamily} and macros that take one argument
% like \cs{textsf} can be used.
%
% \begin{declcs}{hologoFontSetup} \M{key value list}
% \end{declcs}
% Macro \cs{hologoFontSetup} sets the fonts for all logos.
% Supported keys:
% \begin{description}
% \def\entry#1{\item[\xoption{#1}:]}
% \entry{general}
%   This font is used for all logos. The default is empty.
%   That means no special font is used.
% \entry{bibsf}
%   This font is used for
%   {\hologoLogoSetup{BibTeX}{variant=sf}\hologo{BibTeX}}
%   with variant \xoption{sf}.
% \entry{rm}
%   This font is a serif font. It is used for \hologo{ExTeX}.
% \entry{sc}
%   This font specifies a small caps font. It is used for
%   {\hologoLogoSetup{BibTeX}{variant=sc}\hologo{BibTeX}}
%   with variant \xoption{sc}.
% \entry{sf}
%   This font specifies a sans serif font. The default
%   is \cs{sffamily}, then \cs{sf} is tried. Otherwise
%   a warning is given. It is used by \hologo{KOMAScript}.
% \entry{sy}
%   This is the font for math symbols (e.g. cmsy).
%   It is used by \hologo{AmS}, \hologo{NTS}, \hologo{ExTeX}.
% \entry{logo}
%   \hologo{METAFONT} and \hologo{METAPOST} are using that font.
%   In \hologo{LaTeX} \cs{logofamily} is used and
%   the definitions of package \xpackage{mflogo} are used
%   if the package is not loaded.
%   Otherwise the \cs{tenlogo} is used and defined
%   if it does not already exists.
% \end{description}
%
% \begin{declcs}{hologoLogoFontSetup} \M{logo} \M{key value list}
% \end{declcs}
% Fonts can also be set for a logo or logo component separately,
% see the following list.
% The keys are the same as for \cs{hologoFontSetup}.
%
% \begin{longtable}{>{\ttfamily}l>{\sffamily}ll}
%   \meta{logo} & keys & result\\
%   \hline
%   \endhead
%   BibTeX & bibsf & {\hologoLogoSetup{BibTeX}{variant=sf}\hologo{BibTeX}}\\[.5ex]
%   BibTeX & sc & {\hologoLogoSetup{BibTeX}{variant=sc}\hologo{BibTeX}}\\[.5ex]
%   ExTeX & rm & \hologo{ExTeX}\\
%   SliTeX & rm & \hologo{SliTeX}\\[.5ex]
%   AmS & sy & \hologo{AmS}\\
%   ExTeX & sy & \hologo{ExTeX}\\
%   NTS & sy & \hologo{NTS}\\[.5ex]
%   KOMAScript & sf & \hologo{KOMAScript}\\[.5ex]
%   METAFONT & logo & \hologo{METAFONT}\\
%   METAPOST & logo & \hologo{METAPOST}\\[.5ex]
%   SliTeX & sc \hologo{SliTeX}
% \end{longtable}
%
% \subsubsection{Font order}
%
% For all logos the font \xoption{general} is applied first.
% Example:
%\begin{quote}
%|\hologoFontSetup{general=\color{red}}|
%\end{quote}
% will print red logos.
% Then if the font uses a special font \xoption{sf}, for example,
% the font is applied that is setup by \cs{hologoLogoFontSetup}.
% If this font is not setup, then the common font setup
% by \cs{hologoFontSetup} is used. Otherwise a warning is given,
% that there is no font configured.
%
% \subsection{Additional user macros}
%
% Usually a variant of a logo is configured by using
% \cs{hologoLogoSetup}, because it is bad style to mix
% different variants of the same logo in the same text.
% There the following macros are a convenience for testing.
%
% \begin{declcs}{hologoVariant} \M{name} \M{variant}\\
%   \cs{HologoVariant} \M{name} \M{variant}
% \end{declcs}
% Logo \meta{name} is set using \meta{variant} that specifies
% explicitely which variant of the macro is used. If the argument
% is empty, then the default form of the logo is used
% (configurable by \cs{hologoLogoSetup}).
%
% \cs{HologoVariant} is used if the logo is set in a context
% that needs an uppercase first letter (beginning of a sentence, \dots).
%
% \begin{declcs}{hologoList}\\
%   \cs{hologoEntry} \M{logo} \M{variant} \M{since}
% \end{declcs}
% Macro \cs{hologoList} contains all logos that are provided
% by the package including variants. The list consists of calls
% of \cs{hologoEntry} with three arguments starting with the
% logo name \meta{logo} and its variant \meta{variant}. An empty
% variant means the current default. Argument \meta{since} specifies
% with version of the package \xpackage{hologo} is needed to get
% the logo. If the logo is fixed, then the date gets updated.
% Therefore the date \meta{since} is not exactly the date of
% the first introduction, but rather the date of the latest fix.
%
% Before \cs{hologoList} can be used, macro \cs{hologoEntry} needs
% a definition. The example file in section \ref{sec:example}
% shows applications of \cs{hologoList}.
%
% \subsection{Supported contexts}
%
% Macros \cs{hologo} and friends support special contexts:
% \begin{itemize}
% \item \hologo{LaTeX}'s protection mechanism.
% \item Bookmarks of package \xpackage{hyperref}.
% \item Package \xpackage{tex4ht}.
% \item The macros can be used inside \cs{csname} constructs,
%   if \cs{ifincsname} is available (\hologo{pdfTeX}, \hologo{XeTeX},
%   \hologo{LuaTeX}).
% \end{itemize}
%
% \subsection{Example}
% \label{sec:example}
%
% The following example prints the logos in different fonts.
%    \begin{macrocode}
%<*example>
%<<verbatim
\NeedsTeXFormat{LaTeX2e}
\documentclass[a4paper]{article}
\usepackage[
  hmargin=20mm,
  vmargin=20mm,
]{geometry}
\pagestyle{empty}
\usepackage{hologo}[2016/05/12]
\usepackage{longtable}
\usepackage{array}
\setlength{\extrarowheight}{2pt}
\usepackage[T1]{fontenc}
\usepackage{lmodern}
\usepackage{pdflscape}
\usepackage[
  pdfencoding=auto,
]{hyperref}
\hypersetup{
  pdfauthor={Heiko Oberdiek},
  pdftitle={Example for package `hologo'},
  pdfsubject={Logos with fonts lmr, lmss, qtm, qpl, qhv},
}
\usepackage{bookmark}

% Print the logo list on the console

\begingroup
  \typeout{}%
  \typeout{*** Begin of logo list ***}%
  \newcommand*{\hologoEntry}[3]{%
    \typeout{#1 \ifx\\#2\\\else(#2) \fi[#3]}%
  }%
  \hologoList
  \typeout{*** End of logo list ***}%
  \typeout{}%
\endgroup

\begin{document}
\begin{landscape}

  \section{Example file for package `hologo'}

  % Table for font names

  \begin{longtable}{>{\bfseries}ll}
    \textbf{font} & \textbf{Font name}\\
    \hline
    lmr & Latin Modern Roman\\
    lmss & Latin Modern Sans\\
    qtm & \TeX\ Gyre Termes\\
    qhv & \TeX\ Gyre Heros\\
    qpl & \TeX\ Gyre Pagella\\
  \end{longtable}

  % Logo list with logos in different fonts

  \begingroup
    \newcommand*{\SetVariant}[2]{%
      \ifx\\#2\\%
      \else
        \hologoLogoSetup{#1}{variant=#2}%
      \fi
    }%
    \newcommand*{\hologoEntry}[3]{%
      \SetVariant{#1}{#2}%
      \raisebox{1em}[0pt][0pt]{\hypertarget{#1@#2}{}}%
      \bookmark[%
        dest={#1@#2},%
      ]{%
        #1\ifx\\#2\\\else\space(#2)\fi: \Hologo{#1}, \hologo{#1} %
        [Unicode]%
      }%
      \hypersetup{unicode=false}%
      \bookmark[%
        dest={#1@#2},%
      ]{%
        #1\ifx\\#2\\\else\space(#2)\fi: \Hologo{#1}, \hologo{#1} %
        [PDFDocEncoding]%
      }%
      \texttt{#1}%
      &%
      \texttt{#2}%
      &%
      \Hologo{#1}%
      &%
      \SetVariant{#1}{#2}%
      \hologo{#1}%
      &%
      \SetVariant{#1}{#2}%
      \fontfamily{qtm}\selectfont
      \hologo{#1}%
      &%
      \SetVariant{#1}{#2}%
      \fontfamily{qpl}\selectfont
      \hologo{#1}%
      &%
      \SetVariant{#1}{#2}%
      \textsf{\hologo{#1}}%
      &%
      \SetVariant{#1}{#2}%
      \fontfamily{qhv}\selectfont
      \hologo{#1}%
      \tabularnewline
    }%
    \begin{longtable}{llllllll}%
      \textbf{\textit{logo}} & \textbf{\textit{variant}} &
      \texttt{\string\Hologo} &
      \textbf{lmr} & \textbf{qtm} & \textbf{qpl} &
      \textbf{lmss} & \textbf{qhv}
      \tabularnewline
      \hline
      \endhead
      \hologoList
    \end{longtable}%
  \endgroup

\end{landscape}
\end{document}
%verbatim
%</example>
%    \end{macrocode}
%
% \StopEventually{
% }
%
% \section{Implementation}
%    \begin{macrocode}
%<*package>
%    \end{macrocode}
%    Reload check, especially if the package is not used with \LaTeX.
%    \begin{macrocode}
\begingroup\catcode61\catcode48\catcode32=10\relax%
  \catcode13=5 % ^^M
  \endlinechar=13 %
  \catcode35=6 % #
  \catcode39=12 % '
  \catcode44=12 % ,
  \catcode45=12 % -
  \catcode46=12 % .
  \catcode58=12 % :
  \catcode64=11 % @
  \catcode123=1 % {
  \catcode125=2 % }
  \expandafter\let\expandafter\x\csname ver@hologo.sty\endcsname
  \ifx\x\relax % plain-TeX, first loading
  \else
    \def\empty{}%
    \ifx\x\empty % LaTeX, first loading,
      % variable is initialized, but \ProvidesPackage not yet seen
    \else
      \expandafter\ifx\csname PackageInfo\endcsname\relax
        \def\x#1#2{%
          \immediate\write-1{Package #1 Info: #2.}%
        }%
      \else
        \def\x#1#2{\PackageInfo{#1}{#2, stopped}}%
      \fi
      \x{hologo}{The package is already loaded}%
      \aftergroup\endinput
    \fi
  \fi
\endgroup%
%    \end{macrocode}
%    Package identification:
%    \begin{macrocode}
\begingroup\catcode61\catcode48\catcode32=10\relax%
  \catcode13=5 % ^^M
  \endlinechar=13 %
  \catcode35=6 % #
  \catcode39=12 % '
  \catcode40=12 % (
  \catcode41=12 % )
  \catcode44=12 % ,
  \catcode45=12 % -
  \catcode46=12 % .
  \catcode47=12 % /
  \catcode58=12 % :
  \catcode64=11 % @
  \catcode91=12 % [
  \catcode93=12 % ]
  \catcode123=1 % {
  \catcode125=2 % }
  \expandafter\ifx\csname ProvidesPackage\endcsname\relax
    \def\x#1#2#3[#4]{\endgroup
      \immediate\write-1{Package: #3 #4}%
      \xdef#1{#4}%
    }%
  \else
    \def\x#1#2[#3]{\endgroup
      #2[{#3}]%
      \ifx#1\@undefined
        \xdef#1{#3}%
      \fi
      \ifx#1\relax
        \xdef#1{#3}%
      \fi
    }%
  \fi
\expandafter\x\csname ver@hologo.sty\endcsname
\ProvidesPackage{hologo}%
  [2016/05/12 v1.11 A logo collection with bookmark support (HO)]%
%    \end{macrocode}
%
%    \begin{macrocode}
\begingroup\catcode61\catcode48\catcode32=10\relax%
  \catcode13=5 % ^^M
  \endlinechar=13 %
  \catcode123=1 % {
  \catcode125=2 % }
  \catcode64=11 % @
  \def\x{\endgroup
    \expandafter\edef\csname HOLOGO@AtEnd\endcsname{%
      \endlinechar=\the\endlinechar\relax
      \catcode13=\the\catcode13\relax
      \catcode32=\the\catcode32\relax
      \catcode35=\the\catcode35\relax
      \catcode61=\the\catcode61\relax
      \catcode64=\the\catcode64\relax
      \catcode123=\the\catcode123\relax
      \catcode125=\the\catcode125\relax
    }%
  }%
\x\catcode61\catcode48\catcode32=10\relax%
\catcode13=5 % ^^M
\endlinechar=13 %
\catcode35=6 % #
\catcode64=11 % @
\catcode123=1 % {
\catcode125=2 % }
\def\TMP@EnsureCode#1#2{%
  \edef\HOLOGO@AtEnd{%
    \HOLOGO@AtEnd
    \catcode#1=\the\catcode#1\relax
  }%
  \catcode#1=#2\relax
}
\TMP@EnsureCode{10}{12}% ^^J
\TMP@EnsureCode{33}{12}% !
\TMP@EnsureCode{34}{12}% "
\TMP@EnsureCode{36}{3}% $
\TMP@EnsureCode{38}{4}% &
\TMP@EnsureCode{39}{12}% '
\TMP@EnsureCode{40}{12}% (
\TMP@EnsureCode{41}{12}% )
\TMP@EnsureCode{42}{12}% *
\TMP@EnsureCode{43}{12}% +
\TMP@EnsureCode{44}{12}% ,
\TMP@EnsureCode{45}{12}% -
\TMP@EnsureCode{46}{12}% .
\TMP@EnsureCode{47}{12}% /
\TMP@EnsureCode{58}{12}% :
\TMP@EnsureCode{59}{12}% ;
\TMP@EnsureCode{60}{12}% <
\TMP@EnsureCode{62}{12}% >
\TMP@EnsureCode{63}{12}% ?
\TMP@EnsureCode{91}{12}% [
\TMP@EnsureCode{93}{12}% ]
\TMP@EnsureCode{94}{7}% ^ (superscript)
\TMP@EnsureCode{95}{8}% _ (subscript)
\TMP@EnsureCode{96}{12}% `
\TMP@EnsureCode{124}{12}% |
\edef\HOLOGO@AtEnd{%
  \HOLOGO@AtEnd
  \escapechar\the\escapechar\relax
  \noexpand\endinput
}
\escapechar=92 %
%    \end{macrocode}
%
% \subsection{Logo list}
%
%    \begin{macro}{\hologoList}
%    \begin{macrocode}
\def\hologoList{%
  \hologoEntry{(La)TeX}{}{2011/10/01}%
  \hologoEntry{AmSLaTeX}{}{2010/04/16}%
  \hologoEntry{AmSTeX}{}{2010/04/16}%
  \hologoEntry{biber}{}{2011/10/01}%
  \hologoEntry{BibTeX}{}{2011/10/01}%
  \hologoEntry{BibTeX}{sf}{2011/10/01}%
  \hologoEntry{BibTeX}{sc}{2011/10/01}%
  \hologoEntry{BibTeX8}{}{2011/11/22}%
  \hologoEntry{ConTeXt}{}{2011/03/25}%
  \hologoEntry{ConTeXt}{narrow}{2011/03/25}%
  \hologoEntry{ConTeXt}{simple}{2011/03/25}%
  \hologoEntry{emTeX}{}{2010/04/26}%
  \hologoEntry{eTeX}{}{2010/04/08}%
  \hologoEntry{ExTeX}{}{2011/10/01}%
  \hologoEntry{HanTheThanh}{}{2011/11/29}%
  \hologoEntry{iniTeX}{}{2011/10/01}%
  \hologoEntry{KOMAScript}{}{2011/10/01}%
  \hologoEntry{La}{}{2010/05/08}%
  \hologoEntry{LaTeX}{}{2010/04/08}%
  \hologoEntry{LaTeX2e}{}{2010/04/08}%
  \hologoEntry{LaTeX3}{}{2010/04/24}%
  \hologoEntry{LaTeXe}{}{2010/04/08}%
  \hologoEntry{LaTeXML}{}{2011/11/22}%
  \hologoEntry{LaTeXTeX}{}{2011/10/01}%
  \hologoEntry{LuaLaTeX}{}{2010/04/08}%
  \hologoEntry{LuaTeX}{}{2010/04/08}%
  \hologoEntry{LyX}{}{2011/10/01}%
  \hologoEntry{METAFONT}{}{2011/10/01}%
  \hologoEntry{MetaFun}{}{2011/10/01}%
  \hologoEntry{METAPOST}{}{2011/10/01}%
  \hologoEntry{MetaPost}{}{2011/10/01}%
  \hologoEntry{MiKTeX}{}{2011/10/01}%
  \hologoEntry{NTS}{}{2011/10/01}%
  \hologoEntry{OzMF}{}{2011/10/01}%
  \hologoEntry{OzMP}{}{2011/10/01}%
  \hologoEntry{OzTeX}{}{2011/10/01}%
  \hologoEntry{OzTtH}{}{2011/10/01}%
  \hologoEntry{PCTeX}{}{2011/10/01}%
  \hologoEntry{pdfTeX}{}{2011/10/01}%
  \hologoEntry{pdfLaTeX}{}{2011/10/01}%
  \hologoEntry{PiC}{}{2011/10/01}%
  \hologoEntry{PiCTeX}{}{2011/10/01}%
  \hologoEntry{plainTeX}{}{2010/04/08}%
  \hologoEntry{plainTeX}{space}{2010/04/16}%
  \hologoEntry{plainTeX}{hyphen}{2010/04/16}%
  \hologoEntry{plainTeX}{runtogether}{2010/04/16}%
  \hologoEntry{SageTeX}{}{2011/11/22}%
  \hologoEntry{SLiTeX}{}{2011/10/01}%
  \hologoEntry{SLiTeX}{lift}{2011/10/01}%
  \hologoEntry{SLiTeX}{narrow}{2011/10/01}%
  \hologoEntry{SLiTeX}{simple}{2011/10/01}%
  \hologoEntry{SliTeX}{}{2011/10/01}%
  \hologoEntry{SliTeX}{narrow}{2011/10/01}%
  \hologoEntry{SliTeX}{simple}{2011/10/01}%
  \hologoEntry{SliTeX}{lift}{2011/10/01}%
  \hologoEntry{teTeX}{}{2011/10/01}%
  \hologoEntry{TeX}{}{2010/04/08}%
  \hologoEntry{TeX4ht}{}{2011/11/22}%
  \hologoEntry{TTH}{}{2011/11/22}%
  \hologoEntry{virTeX}{}{2011/10/01}%
  \hologoEntry{VTeX}{}{2010/04/24}%
  \hologoEntry{Xe}{}{2010/04/08}%
  \hologoEntry{XeLaTeX}{}{2010/04/08}%
  \hologoEntry{XeTeX}{}{2010/04/08}%
}
%    \end{macrocode}
%    \end{macro}
%
% \subsection{Load resources}
%
%    \begin{macrocode}
\begingroup\expandafter\expandafter\expandafter\endgroup
\expandafter\ifx\csname RequirePackage\endcsname\relax
  \def\TMP@RequirePackage#1[#2]{%
    \begingroup\expandafter\expandafter\expandafter\endgroup
    \expandafter\ifx\csname ver@#1.sty\endcsname\relax
      \input #1.sty\relax
    \fi
  }%
  \TMP@RequirePackage{ltxcmds}[2011/02/04]%
  \TMP@RequirePackage{infwarerr}[2010/04/08]%
  \TMP@RequirePackage{kvsetkeys}[2010/03/01]%
  \TMP@RequirePackage{kvdefinekeys}[2010/03/01]%
  \TMP@RequirePackage{pdftexcmds}[2010/04/01]%
  \TMP@RequirePackage{ifpdf}[2010/01/28]%
  \TMP@RequirePackage{ifluatex}[2010/03/01]%
  \ltx@IfUndefined{newif}{%
    \expandafter\let\csname newif\endcsname\ltx@newif
  }{}%
  \TMP@RequirePackage{ifxetex}[2009/01/23]%
  \TMP@RequirePackage{ifvtex}[2010/03/01]%
\else
  \RequirePackage{ltxcmds}[2011/02/04]%
  \RequirePackage{infwarerr}[2010/04/08]%
  \RequirePackage{kvsetkeys}[2010/03/01]%
  \RequirePackage{kvdefinekeys}[2010/03/01]%
  \RequirePackage{pdftexcmds}[2010/04/01]%
  \RequirePackage{ifpdf}[2010/01/28]%
  \RequirePackage{ifluatex}[2010/03/01]%
  \RequirePackage{ifxetex}[2009/01/23]%
  \RequirePackage{ifvtex}[2010/03/01]%
\fi
%    \end{macrocode}
%
%    \begin{macro}{\HOLOGO@IfDefined}
%    \begin{macrocode}
\def\HOLOGO@IfExists#1{%
  \ifx\@undefined#1%
    \expandafter\ltx@secondoftwo
  \else
    \ifx\relax#1%
      \expandafter\ltx@secondoftwo
    \else
      \expandafter\expandafter\expandafter\ltx@firstoftwo
    \fi
  \fi
}
%    \end{macrocode}
%    \end{macro}
%
% \subsection{Setup macros}
%
%    \begin{macro}{\hologoSetup}
%    \begin{macrocode}
\def\hologoSetup{%
  \let\HOLOGO@name\relax
  \HOLOGO@Setup
}
%    \end{macrocode}
%    \end{macro}
%
%    \begin{macro}{\hologoLogoSetup}
%    \begin{macrocode}
\def\hologoLogoSetup#1{%
  \edef\HOLOGO@name{#1}%
  \ltx@IfUndefined{HoLogo@\HOLOGO@name}{%
    \@PackageError{hologo}{%
      Unknown logo `\HOLOGO@name'%
    }\@ehc
    \ltx@gobble
  }{%
    \HOLOGO@Setup
  }%
}
%    \end{macrocode}
%    \end{macro}
%
%    \begin{macro}{\HOLOGO@Setup}
%    \begin{macrocode}
\def\HOLOGO@Setup{%
  \kvsetkeys{HoLogo}%
}
%    \end{macrocode}
%    \end{macro}
%
% \subsection{Options}
%
%    \begin{macro}{\HOLOGO@DeclareBoolOption}
%    \begin{macrocode}
\def\HOLOGO@DeclareBoolOption#1{%
  \expandafter\chardef\csname HOLOGOOPT@#1\endcsname\ltx@zero
  \kv@define@key{HoLogo}{#1}[true]{%
    \def\HOLOGO@temp{##1}%
    \ifx\HOLOGO@temp\HOLOGO@true
      \ifx\HOLOGO@name\relax
        \expandafter\chardef\csname HOLOGOOPT@#1\endcsname=\ltx@one
      \else
        \expandafter\chardef\csname
        HoLogoOpt@#1@\HOLOGO@name\endcsname\ltx@one
      \fi
      \HOLOGO@SetBreakAll{#1}%
    \else
      \ifx\HOLOGO@temp\HOLOGO@false
        \ifx\HOLOGO@name\relax
          \expandafter\chardef\csname HOLOGOOPT@#1\endcsname=\ltx@zero
        \else
          \expandafter\chardef\csname
          HoLogoOpt@#1@\HOLOGO@name\endcsname=\ltx@zero
        \fi
        \HOLOGO@SetBreakAll{#1}%
      \else
        \@PackageError{hologo}{%
          Unknown value `##1' for boolean option `#1'.\MessageBreak
          Known values are `true' and `false'%
        }\@ehc
      \fi
    \fi
  }%
}
%    \end{macrocode}
%    \end{macro}
%
%    \begin{macro}{\HOLOGO@SetBreakAll}
%    \begin{macrocode}
\def\HOLOGO@SetBreakAll#1{%
  \def\HOLOGO@temp{#1}%
  \ifx\HOLOGO@temp\HOLOGO@break
    \ifx\HOLOGO@name\relax
      \chardef\HOLOGOOPT@hyphenbreak=\HOLOGOOPT@break
      \chardef\HOLOGOOPT@spacebreak=\HOLOGOOPT@break
      \chardef\HOLOGOOPT@discretionarybreak=\HOLOGOOPT@break
    \else
      \expandafter\chardef
         \csname HoLogoOpt@hyphenbreak@\HOLOGO@name\endcsname=%
         \csname HoLogoOpt@break@\HOLOGO@name\endcsname
      \expandafter\chardef
         \csname HoLogoOpt@spacebreak@\HOLOGO@name\endcsname=%
         \csname HoLogoOpt@break@\HOLOGO@name\endcsname
      \expandafter\chardef
         \csname HoLogoOpt@discretionarybreak@\HOLOGO@name
             \endcsname=%
         \csname HoLogoOpt@break@\HOLOGO@name\endcsname
    \fi
  \fi
}
%    \end{macrocode}
%    \end{macro}
%
%    \begin{macro}{\HOLOGO@true}
%    \begin{macrocode}
\def\HOLOGO@true{true}
%    \end{macrocode}
%    \end{macro}
%    \begin{macro}{\HOLOGO@false}
%    \begin{macrocode}
\def\HOLOGO@false{false}
%    \end{macrocode}
%    \end{macro}
%    \begin{macro}{\HOLOGO@break}
%    \begin{macrocode}
\def\HOLOGO@break{break}
%    \end{macrocode}
%    \end{macro}
%
%    \begin{macrocode}
\HOLOGO@DeclareBoolOption{break}
\HOLOGO@DeclareBoolOption{hyphenbreak}
\HOLOGO@DeclareBoolOption{spacebreak}
\HOLOGO@DeclareBoolOption{discretionarybreak}
%    \end{macrocode}
%
%    \begin{macrocode}
\kv@define@key{HoLogo}{variant}{%
  \ifx\HOLOGO@name\relax
    \@PackageError{hologo}{%
      Option `variant' is not available in \string\hologoSetup,%
      \MessageBreak
      Use \string\hologoLogoSetup\space instead%
    }\@ehc
  \else
    \edef\HOLOGO@temp{#1}%
    \ifx\HOLOGO@temp\ltx@empty
      \expandafter
      \let\csname HoLogoOpt@variant@\HOLOGO@name\endcsname\@undefined
    \else
      \ltx@IfUndefined{HoLogo@\HOLOGO@name @\HOLOGO@temp}{%
        \@PackageError{hologo}{%
          Unknown variant `\HOLOGO@temp' of logo `\HOLOGO@name'%
        }\@ehc
      }{%
        \expandafter
        \let\csname HoLogoOpt@variant@\HOLOGO@name\endcsname
            \HOLOGO@temp
      }%
    \fi
  \fi
}
%    \end{macrocode}
%
%    \begin{macro}{\HOLOGO@Variant}
%    \begin{macrocode}
\def\HOLOGO@Variant#1{%
  #1%
  \ltx@ifundefined{HoLogoOpt@variant@#1}{%
  }{%
    @\csname HoLogoOpt@variant@#1\endcsname
  }%
}
%    \end{macrocode}
%    \end{macro}
%
% \subsection{Break/no-break support}
%
%    \begin{macro}{\HOLOGO@space}
%    \begin{macrocode}
\def\HOLOGO@space{%
  \ltx@ifundefined{HoLogoOpt@spacebreak@\HOLOGO@name}{%
    \ltx@ifundefined{HoLogoOpt@break@\HOLOGO@name}{%
      \chardef\HOLOGO@temp=\HOLOGOOPT@spacebreak
    }{%
      \chardef\HOLOGO@temp=%
        \csname HoLogoOpt@break@\HOLOGO@name\endcsname
    }%
  }{%
    \chardef\HOLOGO@temp=%
      \csname HoLogoOpt@spacebreak@\HOLOGO@name\endcsname
  }%
  \ifcase\HOLOGO@temp
    \penalty10000 %
  \fi
  \ltx@space
}
%    \end{macrocode}
%    \end{macro}
%
%    \begin{macro}{\HOLOGO@hyphen}
%    \begin{macrocode}
\def\HOLOGO@hyphen{%
  \ltx@ifundefined{HoLogoOpt@hyphenbreak@\HOLOGO@name}{%
    \ltx@ifundefined{HoLogoOpt@break@\HOLOGO@name}{%
      \chardef\HOLOGO@temp=\HOLOGOOPT@hyphenbreak
    }{%
      \chardef\HOLOGO@temp=%
        \csname HoLogoOpt@break@\HOLOGO@name\endcsname
    }%
  }{%
    \chardef\HOLOGO@temp=%
      \csname HoLogoOpt@hyphenbreak@\HOLOGO@name\endcsname
  }%
  \ifcase\HOLOGO@temp
    \ltx@mbox{-}%
  \else
    -%
  \fi
}
%    \end{macrocode}
%    \end{macro}
%
%    \begin{macro}{\HOLOGO@discretionary}
%    \begin{macrocode}
\def\HOLOGO@discretionary{%
  \ltx@ifundefined{HoLogoOpt@discretionarybreak@\HOLOGO@name}{%
    \ltx@ifundefined{HoLogoOpt@break@\HOLOGO@name}{%
      \chardef\HOLOGO@temp=\HOLOGOOPT@discretionarybreak
    }{%
      \chardef\HOLOGO@temp=%
        \csname HoLogoOpt@break@\HOLOGO@name\endcsname
    }%
  }{%
    \chardef\HOLOGO@temp=%
      \csname HoLogoOpt@discretionarybreak@\HOLOGO@name\endcsname
  }%
  \ifcase\HOLOGO@temp
  \else
    \-%
  \fi
}
%    \end{macrocode}
%    \end{macro}
%
%    \begin{macro}{\HOLOGO@mbox}
%    \begin{macrocode}
\def\HOLOGO@mbox#1{%
  \ltx@ifundefined{HoLogoOpt@break@\HOLOGO@name}{%
    \chardef\HOLOGO@temp=\HOLOGOOPT@hyphenbreak
  }{%
    \chardef\HOLOGO@temp=%
      \csname HoLogoOpt@break@\HOLOGO@name\endcsname
  }%
  \ifcase\HOLOGO@temp
    \ltx@mbox{#1}%
  \else
    #1%
  \fi
}
%    \end{macrocode}
%    \end{macro}
%
% \subsection{Font support}
%
%    \begin{macro}{\HoLogoFont@font}
%    \begin{tabular}{@{}ll@{}}
%    |#1|:& logo name\\
%    |#2|:& font short name\\
%    |#3|:& text
%    \end{tabular}
%    \begin{macrocode}
\def\HoLogoFont@font#1#2#3{%
  \begingroup
    \ltx@IfUndefined{HoLogoFont@logo@#1.#2}{%
      \ltx@IfUndefined{HoLogoFont@font@#2}{%
        \@PackageWarning{hologo}{%
          Missing font `#2' for logo `#1'%
        }%
        #3%
      }{%
        \csname HoLogoFont@font@#2\endcsname{#3}%
      }%
    }{%
      \csname HoLogoFont@logo@#1.#2\endcsname{#3}%
    }%
  \endgroup
}
%    \end{macrocode}
%    \end{macro}
%
%    \begin{macro}{\HoLogoFont@Def}
%    \begin{macrocode}
\def\HoLogoFont@Def#1{%
  \expandafter\def\csname HoLogoFont@font@#1\endcsname
}
%    \end{macrocode}
%    \end{macro}
%    \begin{macro}{\HoLogoFont@LogoDef}
%    \begin{macrocode}
\def\HoLogoFont@LogoDef#1#2{%
  \expandafter\def\csname HoLogoFont@logo@#1.#2\endcsname
}
%    \end{macrocode}
%    \end{macro}
%
% \subsubsection{Font defaults}
%
%    \begin{macro}{\HoLogoFont@font@general}
%    \begin{macrocode}
\HoLogoFont@Def{general}{}%
%    \end{macrocode}
%    \end{macro}
%
%    \begin{macro}{\HoLogoFont@font@rm}
%    \begin{macrocode}
\ltx@IfUndefined{rmfamily}{%
  \ltx@IfUndefined{rm}{%
  }{%
    \HoLogoFont@Def{rm}{\rm}%
  }%
}{%
  \HoLogoFont@Def{rm}{\rmfamily}%
}
%    \end{macrocode}
%    \end{macro}
%
%    \begin{macro}{\HoLogoFont@font@sf}
%    \begin{macrocode}
\ltx@IfUndefined{sffamily}{%
  \ltx@IfUndefined{sf}{%
  }{%
    \HoLogoFont@Def{sf}{\sf}%
  }%
}{%
  \HoLogoFont@Def{sf}{\sffamily}%
}
%    \end{macrocode}
%    \end{macro}
%
%    \begin{macro}{\HoLogoFont@font@bibsf}
%    In case of \hologo{plainTeX} the original small caps
%    variant is used as default. In \hologo{LaTeX}
%    the definition of package \xpackage{dtklogos} \cite{dtklogos}
%    is used.
%\begin{quote}
%\begin{verbatim}
%\DeclareRobustCommand{\BibTeX}{%
%  B%
%  \kern-.05em%
%  \hbox{%
%    $\m@th$% %% force math size calculations
%    \csname S@\f@size\endcsname
%    \fontsize\sf@size\z@
%    \math@fontsfalse
%    \selectfont
%    I%
%    \kern-.025em%
%    B
%  }%
%  \kern-.08em%
%  \-%
%  \TeX
%}
%\end{verbatim}
%\end{quote}
%    \begin{macrocode}
\ltx@IfUndefined{selectfont}{%
  \ltx@IfUndefined{tensc}{%
    \font\tensc=cmcsc10\relax
  }{}%
  \HoLogoFont@Def{bibsf}{\tensc}%
}{%
  \HoLogoFont@Def{bibsf}{%
    $\mathsurround=0pt$%
    \csname S@\f@size\endcsname
    \fontsize\sf@size{0pt}%
    \math@fontsfalse
    \selectfont
  }%
}
%    \end{macrocode}
%    \end{macro}
%
%    \begin{macro}{\HoLogoFont@font@sc}
%    \begin{macrocode}
\ltx@IfUndefined{scshape}{%
  \ltx@IfUndefined{tensc}{%
    \font\tensc=cmcsc10\relax
  }{}%
  \HoLogoFont@Def{sc}{\tensc}%
}{%
  \HoLogoFont@Def{sc}{\scshape}%
}
%    \end{macrocode}
%    \end{macro}
%
%    \begin{macro}{\HoLogoFont@font@sy}
%    \begin{macrocode}
\ltx@IfUndefined{usefont}{%
  \ltx@IfUndefined{tensy}{%
  }{%
    \HoLogoFont@Def{sy}{\tensy}%
  }%
}{%
  \HoLogoFont@Def{sy}{%
    \usefont{OMS}{cmsy}{m}{n}%
  }%
}
%    \end{macrocode}
%    \end{macro}
%
%    \begin{macro}{\HoLogoFont@font@logo}
%    \begin{macrocode}
\begingroup
  \def\x{LaTeX2e}%
\expandafter\endgroup
\ifx\fmtname\x
  \ltx@IfUndefined{logofamily}{%
    \DeclareRobustCommand\logofamily{%
      \not@math@alphabet\logofamily\relax
      \fontencoding{U}%
      \fontfamily{logo}%
      \selectfont
    }%
  }{}%
  \ltx@IfUndefined{logofamily}{%
  }{%
    \HoLogoFont@Def{logo}{\logofamily}%
  }%
\else
  \ltx@IfUndefined{tenlogo}{%
    \font\tenlogo=logo10\relax
  }{}%
  \HoLogoFont@Def{logo}{\tenlogo}%
\fi
%    \end{macrocode}
%    \end{macro}
%
% \subsubsection{Font setup}
%
%    \begin{macro}{\hologoFontSetup}
%    \begin{macrocode}
\def\hologoFontSetup{%
  \let\HOLOGO@name\relax
  \HOLOGO@FontSetup
}
%    \end{macrocode}
%    \end{macro}
%
%    \begin{macro}{\hologoLogoFontSetup}
%    \begin{macrocode}
\def\hologoLogoFontSetup#1{%
  \edef\HOLOGO@name{#1}%
  \ltx@IfUndefined{HoLogo@\HOLOGO@name}{%
    \@PackageError{hologo}{%
      Unknown logo `\HOLOGO@name'%
    }\@ehc
    \ltx@gobble
  }{%
    \HOLOGO@FontSetup
  }%
}
%    \end{macrocode}
%    \end{macro}
%
%    \begin{macro}{\HOLOGO@FontSetup}
%    \begin{macrocode}
\def\HOLOGO@FontSetup{%
  \kvsetkeys{HoLogoFont}%
}
%    \end{macrocode}
%    \end{macro}
%
%    \begin{macrocode}
\def\HOLOGO@temp#1{%
  \kv@define@key{HoLogoFont}{#1}{%
    \ifx\HOLOGO@name\relax
      \HoLogoFont@Def{#1}{##1}%
    \else
      \HoLogoFont@LogoDef\HOLOGO@name{#1}{##1}%
    \fi
  }%
}
\HOLOGO@temp{general}
\HOLOGO@temp{sf}
%    \end{macrocode}
%
% \subsection{Generic logo commands}
%
%    \begin{macrocode}
\HOLOGO@IfExists\hologo{%
  \@PackageError{hologo}{%
    \string\hologo\ltx@space is already defined.\MessageBreak
    Package loading is aborted%
  }\@ehc
  \HOLOGO@AtEnd
}%
\HOLOGO@IfExists\hologoRobust{%
  \@PackageError{hologo}{%
    \string\hologoRobust\ltx@space is already defined.\MessageBreak
    Package loading is aborted%
  }\@ehc
  \HOLOGO@AtEnd
}%
%    \end{macrocode}
%
% \subsubsection{\cs{hologo} and friends}
%
%    \begin{macrocode}
\ifluatex
  \expandafter\ltx@firstofone
\else
  \expandafter\ltx@gobble
\fi
{%
  \ltx@IfUndefined{ifincsname}{%
    \ifnum\luatexversion<36 %
      \expandafter\ltx@gobble
    \else
      \expandafter\ltx@firstofone
    \fi
    {%
      \begingroup
        \ifcase0%
            \directlua{%
              if tex.enableprimitives then %
                tex.enableprimitives('HOLOGO@', {'ifincsname'})%
              else %
                tex.print('1')%
              end%
            }%
            \ifx\HOLOGO@ifincsname\@undefined 1\fi%
            \relax
          \expandafter\ltx@firstofone
        \else
          \endgroup
          \expandafter\ltx@gobble
        \fi
        {%
          \global\let\ifincsname\HOLOGO@ifincsname
        }%
      \HOLOGO@temp
    }%
  }{}%
}
%    \end{macrocode}
%    \begin{macrocode}
\ltx@IfUndefined{ifincsname}{%
  \catcode`$=14 %
}{%
  \catcode`$=9 %
}
%    \end{macrocode}
%
%    \begin{macro}{\hologo}
%    \begin{macrocode}
\def\hologo#1{%
$ \ifincsname
$   \ltx@ifundefined{HoLogoCs@\HOLOGO@Variant{#1}}{%
$     #1%
$   }{%
$     \csname HoLogoCs@\HOLOGO@Variant{#1}\endcsname\ltx@firstoftwo
$   }%
$ \else
    \HOLOGO@IfExists\texorpdfstring\texorpdfstring\ltx@firstoftwo
    {%
      \hologoRobust{#1}%
    }{%
      \ltx@ifundefined{HoLogoBkm@\HOLOGO@Variant{#1}}{%
        \ltx@ifundefined{HoLogo@#1}{?#1?}{#1}%
      }{%
        \csname HoLogoBkm@\HOLOGO@Variant{#1}\endcsname
        \ltx@firstoftwo
      }%
    }%
$ \fi
}
%    \end{macrocode}
%    \end{macro}
%    \begin{macro}{\Hologo}
%    \begin{macrocode}
\def\Hologo#1{%
$ \ifincsname
$   \ltx@ifundefined{HoLogoCs@\HOLOGO@Variant{#1}}{%
$     #1%
$   }{%
$     \csname HoLogoCs@\HOLOGO@Variant{#1}\endcsname\ltx@secondoftwo
$   }%
$ \else
    \HOLOGO@IfExists\texorpdfstring\texorpdfstring\ltx@firstoftwo
    {%
      \HologoRobust{#1}%
    }{%
      \ltx@ifundefined{HoLogoBkm@\HOLOGO@Variant{#1}}{%
        \ltx@ifundefined{HoLogo@#1}{?#1?}{#1}%
      }{%
        \csname HoLogoBkm@\HOLOGO@Variant{#1}\endcsname
        \ltx@secondoftwo
      }%
    }%
$ \fi
}
%    \end{macrocode}
%    \end{macro}
%
%    \begin{macro}{\hologoVariant}
%    \begin{macrocode}
\def\hologoVariant#1#2{%
  \ifx\relax#2\relax
    \hologo{#1}%
  \else
$   \ifincsname
$     \ltx@ifundefined{HoLogoCs@#1@#2}{%
$       #1%
$     }{%
$       \csname HoLogoCs@#1@#2\endcsname\ltx@firstoftwo
$     }%
$   \else
      \HOLOGO@IfExists\texorpdfstring\texorpdfstring\ltx@firstoftwo
      {%
        \hologoVariantRobust{#1}{#2}%
      }{%
        \ltx@ifundefined{HoLogoBkm@#1@#2}{%
          \ltx@ifundefined{HoLogo@#1}{?#1?}{#1}%
        }{%
          \csname HoLogoBkm@#1@#2\endcsname
          \ltx@firstoftwo
        }%
      }%
$   \fi
  \fi
}
%    \end{macrocode}
%    \end{macro}
%    \begin{macro}{\HologoVariant}
%    \begin{macrocode}
\def\HologoVariant#1#2{%
  \ifx\relax#2\relax
    \Hologo{#1}%
  \else
$   \ifincsname
$     \ltx@ifundefined{HoLogoCs@#1@#2}{%
$       #1%
$     }{%
$       \csname HoLogoCs@#1@#2\endcsname\ltx@secondoftwo
$     }%
$   \else
      \HOLOGO@IfExists\texorpdfstring\texorpdfstring\ltx@firstoftwo
      {%
        \HologoVariantRobust{#1}{#2}%
      }{%
        \ltx@ifundefined{HoLogoBkm@#1@#2}{%
          \ltx@ifundefined{HoLogo@#1}{?#1?}{#1}%
        }{%
          \csname HoLogoBkm@#1@#2\endcsname
          \ltx@secondoftwo
        }%
      }%
$   \fi
  \fi
}
%    \end{macrocode}
%    \end{macro}
%
%    \begin{macrocode}
\catcode`\$=3 %
%    \end{macrocode}
%
% \subsubsection{\cs{hologoRobust} and friends}
%
%    \begin{macro}{\hologoRobust}
%    \begin{macrocode}
\ltx@IfUndefined{protected}{%
  \ltx@IfUndefined{DeclareRobustCommand}{%
    \def\hologoRobust#1%
  }{%
    \DeclareRobustCommand*\hologoRobust[1]%
  }%
}{%
  \protected\def\hologoRobust#1%
}%
{%
  \edef\HOLOGO@name{#1}%
  \ltx@IfUndefined{HoLogo@\HOLOGO@Variant\HOLOGO@name}{%
    \@PackageError{hologo}{%
      Unknown logo `\HOLOGO@name'%
    }\@ehc
    ?\HOLOGO@name?%
  }{%
    \ltx@IfUndefined{ver@tex4ht.sty}{%
      \HoLogoFont@font\HOLOGO@name{general}{%
        \csname HoLogo@\HOLOGO@Variant\HOLOGO@name\endcsname
        \ltx@firstoftwo
      }%
    }{%
      \ltx@IfUndefined{HoLogoHtml@\HOLOGO@Variant\HOLOGO@name}{%
        \HOLOGO@name
      }{%
        \csname HoLogoHtml@\HOLOGO@Variant\HOLOGO@name\endcsname
        \ltx@firstoftwo
      }%
    }%
  }%
}
%    \end{macrocode}
%    \end{macro}
%    \begin{macro}{\HologoRobust}
%    \begin{macrocode}
\ltx@IfUndefined{protected}{%
  \ltx@IfUndefined{DeclareRobustCommand}{%
    \def\HologoRobust#1%
  }{%
    \DeclareRobustCommand*\HologoRobust[1]%
  }%
}{%
  \protected\def\HologoRobust#1%
}%
{%
  \edef\HOLOGO@name{#1}%
  \ltx@IfUndefined{HoLogo@\HOLOGO@Variant\HOLOGO@name}{%
    \@PackageError{hologo}{%
      Unknown logo `\HOLOGO@name'%
    }\@ehc
    ?\HOLOGO@name?%
  }{%
    \ltx@IfUndefined{ver@tex4ht.sty}{%
      \HoLogoFont@font\HOLOGO@name{general}{%
        \csname HoLogo@\HOLOGO@Variant\HOLOGO@name\endcsname
        \ltx@secondoftwo
      }%
    }{%
      \ltx@IfUndefined{HoLogoHtml@\HOLOGO@Variant\HOLOGO@name}{%
        \expandafter\HOLOGO@Uppercase\HOLOGO@name
      }{%
        \csname HoLogoHtml@\HOLOGO@Variant\HOLOGO@name\endcsname
        \ltx@secondoftwo
      }%
    }%
  }%
}
%    \end{macrocode}
%    \end{macro}
%    \begin{macro}{\hologoVariantRobust}
%    \begin{macrocode}
\ltx@IfUndefined{protected}{%
  \ltx@IfUndefined{DeclareRobustCommand}{%
    \def\hologoVariantRobust#1#2%
  }{%
    \DeclareRobustCommand*\hologoVariantRobust[2]%
  }%
}{%
  \protected\def\hologoVariantRobust#1#2%
}%
{%
  \begingroup
    \hologoLogoSetup{#1}{variant={#2}}%
    \hologoRobust{#1}%
  \endgroup
}
%    \end{macrocode}
%    \end{macro}
%    \begin{macro}{\HologoVariantRobust}
%    \begin{macrocode}
\ltx@IfUndefined{protected}{%
  \ltx@IfUndefined{DeclareRobustCommand}{%
    \def\HologoVariantRobust#1#2%
  }{%
    \DeclareRobustCommand*\HologoVariantRobust[2]%
  }%
}{%
  \protected\def\HologoVariantRobust#1#2%
}%
{%
  \begingroup
    \hologoLogoSetup{#1}{variant={#2}}%
    \HologoRobust{#1}%
  \endgroup
}
%    \end{macrocode}
%    \end{macro}
%
%    \begin{macro}{\hologorobust}
%    Macro \cs{hologorobust} is only defined for compatibility.
%    Its use is deprecated.
%    \begin{macrocode}
\def\hologorobust{\hologoRobust}
%    \end{macrocode}
%    \end{macro}
%
% \subsection{Helpers}
%
%    \begin{macro}{\HOLOGO@Uppercase}
%    Macro \cs{HOLOGO@Uppercase} is restricted to \cs{uppercase},
%    because \hologo{plainTeX} or \hologo{iniTeX} do not provide
%    \cs{MakeUppercase}.
%    \begin{macrocode}
\def\HOLOGO@Uppercase#1{\uppercase{#1}}
%    \end{macrocode}
%    \end{macro}
%
%    \begin{macro}{\HOLOGO@PdfdocUnicode}
%    \begin{macrocode}
\def\HOLOGO@PdfdocUnicode{%
  \ifx\ifHy@unicode\iftrue
    \expandafter\ltx@secondoftwo
  \else
    \expandafter\ltx@firstoftwo
  \fi
}
%    \end{macrocode}
%    \end{macro}
%
%    \begin{macro}{\HOLOGO@Math}
%    \begin{macrocode}
\def\HOLOGO@MathSetup{%
  \mathsurround0pt\relax
  \HOLOGO@IfExists\f@series{%
    \if b\expandafter\ltx@car\f@series x\@nil
      \csname boldmath\endcsname
   \fi
  }{}%
}
%    \end{macrocode}
%    \end{macro}
%
%    \begin{macro}{\HOLOGO@TempDimen}
%    \begin{macrocode}
\dimendef\HOLOGO@TempDimen=\ltx@zero
%    \end{macrocode}
%    \end{macro}
%    \begin{macro}{\HOLOGO@NegativeKerning}
%    \begin{macrocode}
\def\HOLOGO@NegativeKerning#1{%
  \begingroup
    \HOLOGO@TempDimen=0pt\relax
    \comma@parse@normalized{#1}{%
      \ifdim\HOLOGO@TempDimen=0pt %
        \expandafter\HOLOGO@@NegativeKerning\comma@entry
      \fi
      \ltx@gobble
    }%
    \ifdim\HOLOGO@TempDimen<0pt %
      \kern\HOLOGO@TempDimen
    \fi
  \endgroup
}
%    \end{macrocode}
%    \end{macro}
%    \begin{macro}{\HOLOGO@@NegativeKerning}
%    \begin{macrocode}
\def\HOLOGO@@NegativeKerning#1#2{%
  \setbox\ltx@zero\hbox{#1#2}%
  \HOLOGO@TempDimen=\wd\ltx@zero
  \setbox\ltx@zero\hbox{#1\kern0pt#2}%
  \advance\HOLOGO@TempDimen by -\wd\ltx@zero
}
%    \end{macrocode}
%    \end{macro}
%
%    \begin{macro}{\HOLOGO@SpaceFactor}
%    \begin{macrocode}
\def\HOLOGO@SpaceFactor{%
  \spacefactor1000 %
}
%    \end{macrocode}
%    \end{macro}
%
%    \begin{macro}{\HOLOGO@Span}
%    \begin{macrocode}
\def\HOLOGO@Span#1#2{%
  \HCode{<span class="HoLogo-#1">}%
  #2%
  \HCode{</span>}%
}
%    \end{macrocode}
%    \end{macro}
%
% \subsubsection{Text subscript}
%
%    \begin{macro}{\HOLOGO@SubScript}%
%    \begin{macrocode}
\def\HOLOGO@SubScript#1{%
  \ltx@IfUndefined{textsubscript}{%
    \ltx@IfUndefined{text}{%
      \ltx@mbox{%
        \mathsurround=0pt\relax
        $%
          _{%
            \ltx@IfUndefined{sf@size}{%
              \mathrm{#1}%
            }{%
              \mbox{%
                \fontsize\sf@size{0pt}\selectfont
                #1%
              }%
            }%
          }%
        $%
      }%
    }{%
      \ltx@mbox{%
        \mathsurround=0pt\relax
        $_{\text{#1}}$%
      }%
    }%
  }{%
    \textsubscript{#1}%
  }%
}
%    \end{macrocode}
%    \end{macro}
%
% \subsection{\hologo{TeX} and friends}
%
% \subsubsection{\hologo{TeX}}
%
%    \begin{macro}{\HoLogo@TeX}
%    Source: \hologo{LaTeX} kernel.
%    \begin{macrocode}
\def\HoLogo@TeX#1{%
  T\kern-.1667em\lower.5ex\hbox{E}\kern-.125emX\HOLOGO@SpaceFactor
}
%    \end{macrocode}
%    \end{macro}
%    \begin{macro}{\HoLogoHtml@TeX}
%    \begin{macrocode}
\def\HoLogoHtml@TeX#1{%
  \HoLogoCss@TeX
  \HOLOGO@Span{TeX}{%
    T%
    \HOLOGO@Span{e}{%
      E%
    }%
    X%
  }%
}
%    \end{macrocode}
%    \end{macro}
%    \begin{macro}{\HoLogoCss@TeX}
%    \begin{macrocode}
\def\HoLogoCss@TeX{%
  \Css{%
    span.HoLogo-TeX span.HoLogo-e{%
      position:relative;%
      top:.5ex;%
      margin-left:-.1667em;%
      margin-right:-.125em;%
    }%
  }%
  \Css{%
    a span.HoLogo-TeX span.HoLogo-e{%
      text-decoration:none;%
    }%
  }%
  \global\let\HoLogoCss@TeX\relax
}
%    \end{macrocode}
%    \end{macro}
%
% \subsubsection{\hologo{plainTeX}}
%
%    \begin{macro}{\HoLogo@plainTeX@space}
%    Source: ``The \hologo{TeX}book''
%    \begin{macrocode}
\def\HoLogo@plainTeX@space#1{%
  \HOLOGO@mbox{#1{p}{P}lain}\HOLOGO@space\hologo{TeX}%
}
%    \end{macrocode}
%    \end{macro}
%    \begin{macro}{\HoLogoCs@plainTeX@space}
%    \begin{macrocode}
\def\HoLogoCs@plainTeX@space#1{#1{p}{P}lain TeX}%
%    \end{macrocode}
%    \end{macro}
%    \begin{macro}{\HoLogoBkm@plainTeX@space}
%    \begin{macrocode}
\def\HoLogoBkm@plainTeX@space#1{%
  #1{p}{P}lain \hologo{TeX}%
}
%    \end{macrocode}
%    \end{macro}
%    \begin{macro}{\HoLogoHtml@plainTeX@space}
%    \begin{macrocode}
\def\HoLogoHtml@plainTeX@space#1{%
  #1{p}{P}lain \hologo{TeX}%
}
%    \end{macrocode}
%    \end{macro}
%
%    \begin{macro}{\HoLogo@plainTeX@hyphen}
%    \begin{macrocode}
\def\HoLogo@plainTeX@hyphen#1{%
  \HOLOGO@mbox{#1{p}{P}lain}\HOLOGO@hyphen\hologo{TeX}%
}
%    \end{macrocode}
%    \end{macro}
%    \begin{macro}{\HoLogoCs@plainTeX@hyphen}
%    \begin{macrocode}
\def\HoLogoCs@plainTeX@hyphen#1{#1{p}{P}lain-TeX}
%    \end{macrocode}
%    \end{macro}
%    \begin{macro}{\HoLogoBkm@plainTeX@hyphen}
%    \begin{macrocode}
\def\HoLogoBkm@plainTeX@hyphen#1{%
  #1{p}{P}lain-\hologo{TeX}%
}
%    \end{macrocode}
%    \end{macro}
%    \begin{macro}{\HoLogoHtml@plainTeX@hyphen}
%    \begin{macrocode}
\def\HoLogoHtml@plainTeX@hyphen#1{%
  #1{p}{P}lain-\hologo{TeX}%
}
%    \end{macrocode}
%    \end{macro}
%
%    \begin{macro}{\HoLogo@plainTeX@runtogether}
%    \begin{macrocode}
\def\HoLogo@plainTeX@runtogether#1{%
  \HOLOGO@mbox{#1{p}{P}lain\hologo{TeX}}%
}
%    \end{macrocode}
%    \end{macro}
%    \begin{macro}{\HoLogoCs@plainTeX@runtogether}
%    \begin{macrocode}
\def\HoLogoCs@plainTeX@runtogether#1{#1{p}{P}lainTeX}
%    \end{macrocode}
%    \end{macro}
%    \begin{macro}{\HoLogoBkm@plainTeX@runtogether}
%    \begin{macrocode}
\def\HoLogoBkm@plainTeX@runtogether#1{%
  #1{p}{P}lain\hologo{TeX}%
}
%    \end{macrocode}
%    \end{macro}
%    \begin{macro}{\HoLogoHtml@plainTeX@runtogether}
%    \begin{macrocode}
\def\HoLogoHtml@plainTeX@runtogether#1{%
  #1{p}{P}lain\hologo{TeX}%
}
%    \end{macrocode}
%    \end{macro}
%
%    \begin{macro}{\HoLogo@plainTeX}
%    \begin{macrocode}
\def\HoLogo@plainTeX{\HoLogo@plainTeX@space}
%    \end{macrocode}
%    \end{macro}
%    \begin{macro}{\HoLogoCs@plainTeX}
%    \begin{macrocode}
\def\HoLogoCs@plainTeX{\HoLogoCs@plainTeX@space}
%    \end{macrocode}
%    \end{macro}
%    \begin{macro}{\HoLogoBkm@plainTeX}
%    \begin{macrocode}
\def\HoLogoBkm@plainTeX{\HoLogoBkm@plainTeX@space}
%    \end{macrocode}
%    \end{macro}
%    \begin{macro}{\HoLogoHtml@plainTeX}
%    \begin{macrocode}
\def\HoLogoHtml@plainTeX{\HoLogoHtml@plainTeX@space}
%    \end{macrocode}
%    \end{macro}
%
% \subsubsection{\hologo{LaTeX}}
%
%    Source: \hologo{LaTeX} kernel.
%\begin{quote}
%\begin{verbatim}
%\DeclareRobustCommand{\LaTeX}{%
%  L%
%  \kern-.36em%
%  {%
%    \sbox\z@ T%
%    \vbox to\ht\z@{%
%      \hbox{%
%        \check@mathfonts
%        \fontsize\sf@size\z@
%        \math@fontsfalse
%        \selectfont
%        A%
%      }%
%      \vss
%    }%
%  }%
%  \kern-.15em%
%  \TeX
%}
%\end{verbatim}
%\end{quote}
%
%    \begin{macro}{\HoLogo@La}
%    \begin{macrocode}
\def\HoLogo@La#1{%
  L%
  \kern-.36em%
  \begingroup
    \setbox\ltx@zero\hbox{T}%
    \vbox to\ht\ltx@zero{%
      \hbox{%
        \ltx@ifundefined{check@mathfonts}{%
          \csname sevenrm\endcsname
        }{%
          \check@mathfonts
          \fontsize\sf@size{0pt}%
          \math@fontsfalse\selectfont
        }%
        A%
      }%
      \vss
    }%
  \endgroup
}
%    \end{macrocode}
%    \end{macro}
%
%    \begin{macro}{\HoLogo@LaTeX}
%    Source: \hologo{LaTeX} kernel.
%    \begin{macrocode}
\def\HoLogo@LaTeX#1{%
  \hologo{La}%
  \kern-.15em%
  \hologo{TeX}%
}
%    \end{macrocode}
%    \end{macro}
%    \begin{macro}{\HoLogoHtml@LaTeX}
%    \begin{macrocode}
\def\HoLogoHtml@LaTeX#1{%
  \HoLogoCss@LaTeX
  \HOLOGO@Span{LaTeX}{%
    L%
    \HOLOGO@Span{a}{%
      A%
    }%
    \hologo{TeX}%
  }%
}
%    \end{macrocode}
%    \end{macro}
%    \begin{macro}{\HoLogoCss@LaTeX}
%    \begin{macrocode}
\def\HoLogoCss@LaTeX{%
  \Css{%
    span.HoLogo-LaTeX span.HoLogo-a{%
      position:relative;%
      top:-.5ex;%
      margin-left:-.36em;%
      margin-right:-.15em;%
      font-size:85\%;%
    }%
  }%
  \global\let\HoLogoCss@LaTeX\relax
}
%    \end{macrocode}
%    \end{macro}
%
% \subsubsection{\hologo{(La)TeX}}
%
%    \begin{macro}{\HoLogo@LaTeXTeX}
%    The kerning around the parentheses is taken
%    from package \xpackage{dtklogos} \cite{dtklogos}.
%\begin{quote}
%\begin{verbatim}
%\DeclareRobustCommand{\LaTeXTeX}{%
%  (%
%  \kern-.15em%
%  L%
%  \kern-.36em%
%  {%
%    \sbox\z@ T%
%    \vbox to\ht0{%
%      \hbox{%
%        $\m@th$%
%        \csname S@\f@size\endcsname
%        \fontsize\sf@size\z@
%        \math@fontsfalse
%        \selectfont
%        A%
%      }%
%      \vss
%    }%
%  }%
%  \kern-.2em%
%  )%
%  \kern-.15em%
%  \TeX
%}
%\end{verbatim}
%\end{quote}
%    \begin{macrocode}
\def\HoLogo@LaTeXTeX#1{%
  (%
  \kern-.15em%
  \hologo{La}%
  \kern-.2em%
  )%
  \kern-.15em%
  \hologo{TeX}%
}
%    \end{macrocode}
%    \end{macro}
%    \begin{macro}{\HoLogoBkm@LaTeXTeX}
%    \begin{macrocode}
\def\HoLogoBkm@LaTeXTeX#1{(La)TeX}
%    \end{macrocode}
%    \end{macro}
%
%    \begin{macro}{\HoLogo@(La)TeX}
%    \begin{macrocode}
\expandafter
\let\csname HoLogo@(La)TeX\endcsname\HoLogo@LaTeXTeX
%    \end{macrocode}
%    \end{macro}
%    \begin{macro}{\HoLogoBkm@(La)TeX}
%    \begin{macrocode}
\expandafter
\let\csname HoLogoBkm@(La)TeX\endcsname\HoLogoBkm@LaTeXTeX
%    \end{macrocode}
%    \end{macro}
%    \begin{macro}{\HoLogoHtml@LaTeXTeX}
%    \begin{macrocode}
\def\HoLogoHtml@LaTeXTeX#1{%
  \HoLogoCss@LaTeXTeX
  \HOLOGO@Span{LaTeXTeX}{%
    (%
    \HOLOGO@Span{L}{L}%
    \HOLOGO@Span{a}{A}%
    \HOLOGO@Span{ParenRight}{)}%
    \hologo{TeX}%
  }%
}
%    \end{macrocode}
%    \end{macro}
%    \begin{macro}{\HoLogoHtml@(La)TeX}
%    Kerning after opening parentheses and before closing parentheses
%    is $-0.1$\,em. The original values $-0.15$\,em
%    looked too ugly for a serif font.
%    \begin{macrocode}
\expandafter
\let\csname HoLogoHtml@(La)TeX\endcsname\HoLogoHtml@LaTeXTeX
%    \end{macrocode}
%    \end{macro}
%    \begin{macro}{\HoLogoCss@LaTeXTeX}
%    \begin{macrocode}
\def\HoLogoCss@LaTeXTeX{%
  \Css{%
    span.HoLogo-LaTeXTeX span.HoLogo-L{%
      margin-left:-.1em;%
    }%
  }%
  \Css{%
    span.HoLogo-LaTeXTeX span.HoLogo-a{%
      position:relative;%
      top:-.5ex;%
      margin-left:-.36em;%
      margin-right:-.1em;%
      font-size:85\%;%
    }%
  }%
  \Css{%
    span.HoLogo-LaTeXTeX span.HoLogo-ParenRight{%
      margin-right:-.15em;%
    }%
  }%
  \global\let\HoLogoCss@LaTeXTeX\relax
}
%    \end{macrocode}
%    \end{macro}
%
% \subsubsection{\hologo{LaTeXe}}
%
%    \begin{macro}{\HoLogo@LaTeXe}
%    Source: \hologo{LaTeX} kernel
%    \begin{macrocode}
\def\HoLogo@LaTeXe#1{%
  \hologo{LaTeX}%
  \kern.15em%
  \hbox{%
    \HOLOGO@MathSetup
    2%
    $_{\textstyle\varepsilon}$%
  }%
}
%    \end{macrocode}
%    \end{macro}
%
%    \begin{macro}{\HoLogoCs@LaTeXe}
%    \begin{macrocode}
\ifnum64=`\^^^^0040\relax % test for big chars of LuaTeX/XeTeX
  \catcode`\$=9 %
  \catcode`\&=14 %
\else
  \catcode`\$=14 %
  \catcode`\&=9 %
\fi
\def\HoLogoCs@LaTeXe#1{%
  LaTeX2%
$ \string ^^^^0395%
& e%
}%
\catcode`\$=3 %
\catcode`\&=4 %
%    \end{macrocode}
%    \end{macro}
%
%    \begin{macro}{\HoLogoBkm@LaTeXe}
%    \begin{macrocode}
\def\HoLogoBkm@LaTeXe#1{%
  \hologo{LaTeX}%
  2%
  \HOLOGO@PdfdocUnicode{e}{\textepsilon}%
}
%    \end{macrocode}
%    \end{macro}
%
%    \begin{macro}{\HoLogoHtml@LaTeXe}
%    \begin{macrocode}
\def\HoLogoHtml@LaTeXe#1{%
  \HoLogoCss@LaTeXe
  \HOLOGO@Span{LaTeX2e}{%
    \hologo{LaTeX}%
    \HOLOGO@Span{2}{2}%
    \HOLOGO@Span{e}{%
      \HOLOGO@MathSetup
      \ensuremath{\textstyle\varepsilon}%
    }%
  }%
}
%    \end{macrocode}
%    \end{macro}
%    \begin{macro}{\HoLogoCss@LaTeXe}
%    \begin{macrocode}
\def\HoLogoCss@LaTeXe{%
  \Css{%
    span.HoLogo-LaTeX2e span.HoLogo-2{%
      padding-left:.15em;%
    }%
  }%
  \Css{%
    span.HoLogo-LaTeX2e span.HoLogo-e{%
      position:relative;%
      top:.35ex;%
      text-decoration:none;%
    }%
  }%
  \global\let\HoLogoCss@LaTeXe\relax
}
%    \end{macrocode}
%    \end{macro}
%
%    \begin{macro}{\HoLogo@LaTeX2e}
%    \begin{macrocode}
\expandafter
\let\csname HoLogo@LaTeX2e\endcsname\HoLogo@LaTeXe
%    \end{macrocode}
%    \end{macro}
%    \begin{macro}{\HoLogoCs@LaTeX2e}
%    \begin{macrocode}
\expandafter
\let\csname HoLogoCs@LaTeX2e\endcsname\HoLogoCs@LaTeXe
%    \end{macrocode}
%    \end{macro}
%    \begin{macro}{\HoLogoBkm@LaTeX2e}
%    \begin{macrocode}
\expandafter
\let\csname HoLogoBkm@LaTeX2e\endcsname\HoLogoBkm@LaTeXe
%    \end{macrocode}
%    \end{macro}
%    \begin{macro}{\HoLogoHtml@LaTeX2e}
%    \begin{macrocode}
\expandafter
\let\csname HoLogoHtml@LaTeX2e\endcsname\HoLogoHtml@LaTeXe
%    \end{macrocode}
%    \end{macro}
%
% \subsubsection{\hologo{LaTeX3}}
%
%    \begin{macro}{\HoLogo@LaTeX3}
%    Source: \hologo{LaTeX} kernel
%    \begin{macrocode}
\expandafter\def\csname HoLogo@LaTeX3\endcsname#1{%
  \hologo{LaTeX}%
  3%
}
%    \end{macrocode}
%    \end{macro}
%
%    \begin{macro}{\HoLogoBkm@LaTeX3}
%    \begin{macrocode}
\expandafter\def\csname HoLogoBkm@LaTeX3\endcsname#1{%
  \hologo{LaTeX}%
  3%
}
%    \end{macrocode}
%    \end{macro}
%    \begin{macro}{\HoLogoHtml@LaTeX3}
%    \begin{macrocode}
\expandafter
\let\csname HoLogoHtml@LaTeX3\expandafter\endcsname
\csname HoLogo@LaTeX3\endcsname
%    \end{macrocode}
%    \end{macro}
%
% \subsubsection{\hologo{LaTeXML}}
%
%    \begin{macro}{\HoLogo@LaTeXML}
%    \begin{macrocode}
\def\HoLogo@LaTeXML#1{%
  \HOLOGO@mbox{%
    \hologo{La}%
    \kern-.15em%
    T%
    \kern-.1667em%
    \lower.5ex\hbox{E}%
    \kern-.125em%
    \HoLogoFont@font{LaTeXML}{sc}{xml}%
  }%
}
%    \end{macrocode}
%    \end{macro}
%    \begin{macro}{\HoLogoHtml@pdfLaTeX}
%    \begin{macrocode}
\def\HoLogoHtml@LaTeXML#1{%
  \HOLOGO@Span{LaTeXML}{%
    \HoLogoCss@LaTeX
    \HoLogoCss@TeX
    \HOLOGO@Span{LaTeX}{%
      L%
      \HOLOGO@Span{a}{%
        A%
      }%
    }%
    \HOLOGO@Span{TeX}{%
      T%
      \HOLOGO@Span{e}{%
        E%
      }%
    }%
    \HCode{<span style="font-variant: small-caps;">}%
    xml%
    \HCode{</span>}%
  }%
}
%    \end{macrocode}
%    \end{macro}
%
% \subsubsection{\hologo{eTeX}}
%
%    \begin{macro}{\HoLogo@eTeX}
%    Source: package \xpackage{etex}
%    \begin{macrocode}
\def\HoLogo@eTeX#1{%
  \ltx@mbox{%
    \HOLOGO@MathSetup
    $\varepsilon$%
    -%
    \HOLOGO@NegativeKerning{-T,T-,To}%
    \hologo{TeX}%
  }%
}
%    \end{macrocode}
%    \end{macro}
%    \begin{macro}{\HoLogoCs@eTeX}
%    \begin{macrocode}
\ifnum64=`\^^^^0040\relax % test for big chars of LuaTeX/XeTeX
  \catcode`\$=9 %
  \catcode`\&=14 %
\else
  \catcode`\$=14 %
  \catcode`\&=9 %
\fi
\def\HoLogoCs@eTeX#1{%
$ #1{\string ^^^^0395}{\string ^^^^03b5}%
& #1{e}{E}%
  TeX%
}%
\catcode`\$=3 %
\catcode`\&=4 %
%    \end{macrocode}
%    \end{macro}
%    \begin{macro}{\HoLogoBkm@eTeX}
%    \begin{macrocode}
\def\HoLogoBkm@eTeX#1{%
  \HOLOGO@PdfdocUnicode{#1{e}{E}}{\textepsilon}%
  -%
  \hologo{TeX}%
}
%    \end{macrocode}
%    \end{macro}
%    \begin{macro}{\HoLogoHtml@eTeX}
%    \begin{macrocode}
\def\HoLogoHtml@eTeX#1{%
  \ltx@mbox{%
    \HOLOGO@MathSetup
    $\varepsilon$%
    -%
    \hologo{TeX}%
  }%
}
%    \end{macrocode}
%    \end{macro}
%
% \subsubsection{\hologo{iniTeX}}
%
%    \begin{macro}{\HoLogo@iniTeX}
%    \begin{macrocode}
\def\HoLogo@iniTeX#1{%
  \HOLOGO@mbox{%
    #1{i}{I}ni\hologo{TeX}%
  }%
}
%    \end{macrocode}
%    \end{macro}
%    \begin{macro}{\HoLogoCs@iniTeX}
%    \begin{macrocode}
\def\HoLogoCs@iniTeX#1{#1{i}{I}niTeX}
%    \end{macrocode}
%    \end{macro}
%    \begin{macro}{\HoLogoBkm@iniTeX}
%    \begin{macrocode}
\def\HoLogoBkm@iniTeX#1{%
  #1{i}{I}ni\hologo{TeX}%
}
%    \end{macrocode}
%    \end{macro}
%    \begin{macro}{\HoLogoHtml@iniTeX}
%    \begin{macrocode}
\let\HoLogoHtml@iniTeX\HoLogo@iniTeX
%    \end{macrocode}
%    \end{macro}
%
% \subsubsection{\hologo{virTeX}}
%
%    \begin{macro}{\HoLogo@virTeX}
%    \begin{macrocode}
\def\HoLogo@virTeX#1{%
  \HOLOGO@mbox{%
    #1{v}{V}ir\hologo{TeX}%
  }%
}
%    \end{macrocode}
%    \end{macro}
%    \begin{macro}{\HoLogoCs@virTeX}
%    \begin{macrocode}
\def\HoLogoCs@virTeX#1{#1{v}{V}irTeX}
%    \end{macrocode}
%    \end{macro}
%    \begin{macro}{\HoLogoBkm@virTeX}
%    \begin{macrocode}
\def\HoLogoBkm@virTeX#1{%
  #1{v}{V}ir\hologo{TeX}%
}
%    \end{macrocode}
%    \end{macro}
%    \begin{macro}{\HoLogoHtml@virTeX}
%    \begin{macrocode}
\let\HoLogoHtml@virTeX\HoLogo@virTeX
%    \end{macrocode}
%    \end{macro}
%
% \subsubsection{\hologo{SliTeX}}
%
% \paragraph{Definitions of the three variants.}
%
%    \begin{macro}{\HoLogo@SLiTeX@lift}
%    \begin{macrocode}
\def\HoLogo@SLiTeX@lift#1{%
  \HoLogoFont@font{SliTeX}{rm}{%
    S%
    \kern-.06em%
    L%
    \kern-.18em%
    \raise.32ex\hbox{\HoLogoFont@font{SliTeX}{sc}{i}}%
    \HOLOGO@discretionary
    \kern-.06em%
    \hologo{TeX}%
  }%
}
%    \end{macrocode}
%    \end{macro}
%    \begin{macro}{\HoLogoBkm@SLiTeX@lift}
%    \begin{macrocode}
\def\HoLogoBkm@SLiTeX@lift#1{SLiTeX}
%    \end{macrocode}
%    \end{macro}
%    \begin{macro}{\HoLogoHtml@SLiTeX@lift}
%    \begin{macrocode}
\def\HoLogoHtml@SLiTeX@lift#1{%
  \HoLogoCss@SLiTeX@lift
  \HOLOGO@Span{SLiTeX-lift}{%
    \HoLogoFont@font{SliTeX}{rm}{%
      S%
      \HOLOGO@Span{L}{L}%
      \HOLOGO@Span{i}{i}%
      \hologo{TeX}%
    }%
  }%
}
%    \end{macrocode}
%    \end{macro}
%    \begin{macro}{\HoLogoCss@SLiTeX@lift}
%    \begin{macrocode}
\def\HoLogoCss@SLiTeX@lift{%
  \Css{%
    span.HoLogo-SLiTeX-lift span.HoLogo-L{%
      margin-left:-.06em;%
      margin-right:-.18em;%
    }%
  }%
  \Css{%
    span.HoLogo-SLiTeX-lift span.HoLogo-i{%
      position:relative;%
      top:-.32ex;%
      margin-right:-.06em;%
      font-variant:small-caps;%
    }%
  }%
  \global\let\HoLogoCss@SLiTeX@lift\relax
}
%    \end{macrocode}
%    \end{macro}
%
%    \begin{macro}{\HoLogo@SliTeX@simple}
%    \begin{macrocode}
\def\HoLogo@SliTeX@simple#1{%
  \HoLogoFont@font{SliTeX}{rm}{%
    \ltx@mbox{%
      \HoLogoFont@font{SliTeX}{sc}{Sli}%
    }%
    \HOLOGO@discretionary
    \hologo{TeX}%
  }%
}
%    \end{macrocode}
%    \end{macro}
%    \begin{macro}{\HoLogoBkm@SliTeX@simple}
%    \begin{macrocode}
\def\HoLogoBkm@SliTeX@simple#1{SliTeX}
%    \end{macrocode}
%    \end{macro}
%    \begin{macro}{\HoLogoHtml@SliTeX@simple}
%    \begin{macrocode}
\let\HoLogoHtml@SliTeX@simple\HoLogo@SliTeX@simple
%    \end{macrocode}
%    \end{macro}
%
%    \begin{macro}{\HoLogo@SliTeX@narrow}
%    \begin{macrocode}
\def\HoLogo@SliTeX@narrow#1{%
  \HoLogoFont@font{SliTeX}{rm}{%
    \ltx@mbox{%
      S%
      \kern-.06em%
      \HoLogoFont@font{SliTeX}{sc}{%
        l%
        \kern-.035em%
        i%
      }%
    }%
    \HOLOGO@discretionary
    \kern-.06em%
    \hologo{TeX}%
  }%
}
%    \end{macrocode}
%    \end{macro}
%    \begin{macro}{\HoLogoBkm@SliTeX@narrow}
%    \begin{macrocode}
\def\HoLogoBkm@SliTeX@narrow#1{SliTeX}
%    \end{macrocode}
%    \end{macro}
%    \begin{macro}{\HoLogoHtml@SliTeX@narrow}
%    \begin{macrocode}
\def\HoLogoHtml@SliTeX@narrow#1{%
  \HoLogoCss@SliTeX@narrow
  \HOLOGO@Span{SliTeX-narrow}{%
    \HoLogoFont@font{SliTeX}{rm}{%
      S%
        \HOLOGO@Span{l}{l}%
        \HOLOGO@Span{i}{i}%
      \hologo{TeX}%
    }%
  }%
}
%    \end{macrocode}
%    \end{macro}
%    \begin{macro}{\HoLogoCss@SliTeX@narrow}
%    \begin{macrocode}
\def\HoLogoCss@SliTeX@narrow{%
  \Css{%
    span.HoLogo-SliTeX-narrow span.HoLogo-l{%
      margin-left:-.06em;%
      margin-right:-.035em;%
      font-variant:small-caps;%
    }%
  }%
  \Css{%
    span.HoLogo-SliTeX-narrow span.HoLogo-i{%
      margin-right:-.06em;%
      font-variant:small-caps;%
    }%
  }%
  \global\let\HoLogoCss@SliTeX@narrow\relax
}
%    \end{macrocode}
%    \end{macro}
%
% \paragraph{Macro set completion.}
%
%    \begin{macro}{\HoLogo@SLiTeX@simple}
%    \begin{macrocode}
\def\HoLogo@SLiTeX@simple{\HoLogo@SliTeX@simple}
%    \end{macrocode}
%    \end{macro}
%    \begin{macro}{\HoLogoBkm@SLiTeX@simple}
%    \begin{macrocode}
\def\HoLogoBkm@SLiTeX@simple{\HoLogoBkm@SliTeX@simple}
%    \end{macrocode}
%    \end{macro}
%    \begin{macro}{\HoLogoHtml@SLiTeX@simple}
%    \begin{macrocode}
\def\HoLogoHtml@SLiTeX@simple{\HoLogoHtml@SliTeX@simple}
%    \end{macrocode}
%    \end{macro}
%
%    \begin{macro}{\HoLogo@SLiTeX@narrow}
%    \begin{macrocode}
\def\HoLogo@SLiTeX@narrow{\HoLogo@SliTeX@narrow}
%    \end{macrocode}
%    \end{macro}
%    \begin{macro}{\HoLogoBkm@SLiTeX@narrow}
%    \begin{macrocode}
\def\HoLogoBkm@SLiTeX@narrow{\HoLogoBkm@SliTeX@narrow}
%    \end{macrocode}
%    \end{macro}
%    \begin{macro}{\HoLogoHtml@SLiTeX@narrow}
%    \begin{macrocode}
\def\HoLogoHtml@SLiTeX@narrow{\HoLogoHtml@SliTeX@narrow}
%    \end{macrocode}
%    \end{macro}
%
%    \begin{macro}{\HoLogo@SliTeX@lift}
%    \begin{macrocode}
\def\HoLogo@SliTeX@lift{\HoLogo@SLiTeX@lift}
%    \end{macrocode}
%    \end{macro}
%    \begin{macro}{\HoLogoBkm@SliTeX@lift}
%    \begin{macrocode}
\def\HoLogoBkm@SliTeX@lift{\HoLogoBkm@SLiTeX@lift}
%    \end{macrocode}
%    \end{macro}
%    \begin{macro}{\HoLogoHtml@SliTeX@lift}
%    \begin{macrocode}
\def\HoLogoHtml@SliTeX@lift{\HoLogoHtml@SLiTeX@lift}
%    \end{macrocode}
%    \end{macro}
%
% \paragraph{Defaults.}
%
%    \begin{macro}{\HoLogo@SLiTeX}
%    \begin{macrocode}
\def\HoLogo@SLiTeX{\HoLogo@SLiTeX@lift}
%    \end{macrocode}
%    \end{macro}
%    \begin{macro}{\HoLogoBkm@SLiTeX}
%    \begin{macrocode}
\def\HoLogoBkm@SLiTeX{\HoLogoBkm@SLiTeX@lift}
%    \end{macrocode}
%    \end{macro}
%    \begin{macro}{\HoLogoHtml@SLiTeX}
%    \begin{macrocode}
\def\HoLogoHtml@SLiTeX{\HoLogoHtml@SLiTeX@lift}
%    \end{macrocode}
%    \end{macro}
%
%    \begin{macro}{\HoLogo@SliTeX}
%    \begin{macrocode}
\def\HoLogo@SliTeX{\HoLogo@SliTeX@narrow}
%    \end{macrocode}
%    \end{macro}
%    \begin{macro}{\HoLogoBkm@SliTeX}
%    \begin{macrocode}
\def\HoLogoBkm@SliTeX{\HoLogoBkm@SliTeX@narrow}
%    \end{macrocode}
%    \end{macro}
%    \begin{macro}{\HoLogoHtml@SliTeX}
%    \begin{macrocode}
\def\HoLogoHtml@SliTeX{\HoLogoHtml@SliTeX@narrow}
%    \end{macrocode}
%    \end{macro}
%
% \subsubsection{\hologo{LuaTeX}}
%
%    \begin{macro}{\HoLogo@LuaTeX}
%    The kerning is an idea of Hans Hagen, see mailing list
%    `luatex at tug dot org' in March 2010.
%    \begin{macrocode}
\def\HoLogo@LuaTeX#1{%
  \HOLOGO@mbox{%
    Lua%
    \HOLOGO@NegativeKerning{aT,oT,To}%
    \hologo{TeX}%
  }%
}
%    \end{macrocode}
%    \end{macro}
%    \begin{macro}{\HoLogoHtml@LuaTeX}
%    \begin{macrocode}
\let\HoLogoHtml@LuaTeX\HoLogo@LuaTeX
%    \end{macrocode}
%    \end{macro}
%
% \subsubsection{\hologo{LuaLaTeX}}
%
%    \begin{macro}{\HoLogo@LuaLaTeX}
%    \begin{macrocode}
\def\HoLogo@LuaLaTeX#1{%
  \HOLOGO@mbox{%
    Lua%
    \hologo{LaTeX}%
  }%
}
%    \end{macrocode}
%    \end{macro}
%    \begin{macro}{\HoLogoHtml@LuaLaTeX}
%    \begin{macrocode}
\let\HoLogoHtml@LuaLaTeX\HoLogo@LuaLaTeX
%    \end{macrocode}
%    \end{macro}
%
% \subsubsection{\hologo{XeTeX}, \hologo{XeLaTeX}}
%
%    \begin{macro}{\HOLOGO@IfCharExists}
%    \begin{macrocode}
\ifluatex
  \ifnum\luatexversion<36 %
  \else
    \def\HOLOGO@IfCharExists#1{%
      \ifnum
        \directlua{%
           if luaotfload and luaotfload.aux then
             if luaotfload.aux.font_has_glyph(%
                    font.current(), \number#1) then % 	 
	       tex.print("1") % 	 
	     end % 	 
	   elseif font and font.fonts and font.current then %
            local f = font.fonts[font.current()]%
            if f.characters and f.characters[\number#1] then %
              tex.print("1")%
            end %
          end%
        }0=\ltx@zero
        \expandafter\ltx@secondoftwo
      \else
        \expandafter\ltx@firstoftwo
      \fi
    }%
  \fi
\fi
\ltx@IfUndefined{HOLOGO@IfCharExists}{%
  \def\HOLOGO@@IfCharExists#1{%
    \begingroup
      \tracinglostchars=\ltx@zero
      \setbox\ltx@zero=\hbox{%
        \kern7sp\char#1\relax
        \ifnum\lastkern>\ltx@zero
          \expandafter\aftergroup\csname iffalse\endcsname
        \else
          \expandafter\aftergroup\csname iftrue\endcsname
        \fi
      }%
      % \if{true|false} from \aftergroup
      \endgroup
      \expandafter\ltx@firstoftwo
    \else
      \endgroup
      \expandafter\ltx@secondoftwo
    \fi
  }%
  \ifxetex
    \ltx@IfUndefined{XeTeXfonttype}{}{%
      \ltx@IfUndefined{XeTeXcharglyph}{}{%
        \def\HOLOGO@IfCharExists#1{%
          \ifnum\XeTeXfonttype\font>\ltx@zero
            \expandafter\ltx@firstofthree
          \else
            \expandafter\ltx@gobble
          \fi
          {%
            \ifnum\XeTeXcharglyph#1>\ltx@zero
              \expandafter\ltx@firstoftwo
            \else
              \expandafter\ltx@secondoftwo
            \fi
          }%
          \HOLOGO@@IfCharExists{#1}%
        }%
      }%
    }%
  \fi
}{}
\ltx@ifundefined{HOLOGO@IfCharExists}{%
  \ifnum64=`\^^^^0040\relax % test for big chars of LuaTeX/XeTeX
    \let\HOLOGO@IfCharExists\HOLOGO@@IfCharExists
  \else
    \def\HOLOGO@IfCharExists#1{%
      \ifnum#1>255 %
        \expandafter\ltx@fourthoffour
      \fi
      \HOLOGO@@IfCharExists{#1}%
    }%
  \fi
}{}
%    \end{macrocode}
%    \end{macro}
%
%    \begin{macro}{\HoLogo@Xe}
%    Source: package \xpackage{dtklogos}
%    \begin{macrocode}
\def\HoLogo@Xe#1{%
  X%
  \kern-.1em\relax
  \HOLOGO@IfCharExists{"018E}{%
    \lower.5ex\hbox{\char"018E}%
  }{%
    \chardef\HOLOGO@choice=\ltx@zero
    \ifdim\fontdimen\ltx@one\font>0pt %
      \ltx@IfUndefined{rotatebox}{%
        \ltx@IfUndefined{pgftext}{%
          \ltx@IfUndefined{psscalebox}{%
            \ltx@IfUndefined{HOLOGO@ScaleBox@\hologoDriver}{%
            }{%
              \chardef\HOLOGO@choice=4 %
            }%
          }{%
            \chardef\HOLOGO@choice=3 %
          }%
        }{%
          \chardef\HOLOGO@choice=2 %
        }%
      }{%
        \chardef\HOLOGO@choice=1 %
      }%
      \ifcase\HOLOGO@choice
        \HOLOGO@WarningUnsupportedDriver{Xe}%
        e%
      \or % 1: \rotatebox
        \begingroup
          \setbox\ltx@zero\hbox{\rotatebox{180}{E}}%
          \ltx@LocDimenA=\dp\ltx@zero
          \advance\ltx@LocDimenA by -.5ex\relax
          \raise\ltx@LocDimenA\box\ltx@zero
        \endgroup
      \or % 2: \pgftext
        \lower.5ex\hbox{%
          \pgfpicture
            \pgftext[rotate=180]{E}%
          \endpgfpicture
        }%
      \or % 3: \psscalebox
        \begingroup
          \setbox\ltx@zero\hbox{\psscalebox{-1 -1}{E}}%
          \ltx@LocDimenA=\dp\ltx@zero
          \advance\ltx@LocDimenA by -.5ex\relax
          \raise\ltx@LocDimenA\box\ltx@zero
        \endgroup
      \or % 4: \HOLOGO@PointReflectBox
        \lower.5ex\hbox{\HOLOGO@PointReflectBox{E}}%
      \else
        \@PackageError{hologo}{Internal error (choice/it}\@ehc
      \fi
    \else
      \ltx@IfUndefined{reflectbox}{%
        \ltx@IfUndefined{pgftext}{%
          \ltx@IfUndefined{psscalebox}{%
            \ltx@IfUndefined{HOLOGO@ScaleBox@\hologoDriver}{%
            }{%
              \chardef\HOLOGO@choice=4 %
            }%
          }{%
            \chardef\HOLOGO@choice=3 %
          }%
        }{%
          \chardef\HOLOGO@choice=2 %
        }%
      }{%
        \chardef\HOLOGO@choice=1 %
      }%
      \ifcase\HOLOGO@choice
        \HOLOGO@WarningUnsupportedDriver{Xe}%
        e%
      \or % 1: reflectbox
        \lower.5ex\hbox{%
          \reflectbox{E}%
        }%
      \or % 2: \pgftext
        \lower.5ex\hbox{%
          \pgfpicture
            \pgftransformxscale{-1}%
            \pgftext{E}%
          \endpgfpicture
        }%
      \or % 3: \psscalebox
        \lower.5ex\hbox{%
          \psscalebox{-1 1}{E}%
        }%
      \or % 4: \HOLOGO@Reflectbox
        \lower.5ex\hbox{%
          \HOLOGO@ReflectBox{E}%
        }%
      \else
        \@PackageError{hologo}{Internal error (choice/up)}\@ehc
      \fi
    \fi
  }%
}
%    \end{macrocode}
%    \end{macro}
%    \begin{macro}{\HoLogoHtml@Xe}
%    \begin{macrocode}
\def\HoLogoHtml@Xe#1{%
  \HoLogoCss@Xe
  \HOLOGO@Span{Xe}{%
    X%
    \HOLOGO@Span{e}{%
      \HCode{&\ltx@hashchar x018e;}%
    }%
  }%
}
%    \end{macrocode}
%    \end{macro}
%    \begin{macro}{\HoLogoCss@Xe}
%    \begin{macrocode}
\def\HoLogoCss@Xe{%
  \Css{%
    span.HoLogo-Xe span.HoLogo-e{%
      position:relative;%
      top:.5ex;%
      left-margin:-.1em;%
    }%
  }%
  \global\let\HoLogoCss@Xe\relax
}
%    \end{macrocode}
%    \end{macro}
%
%    \begin{macro}{\HoLogo@XeTeX}
%    \begin{macrocode}
\def\HoLogo@XeTeX#1{%
  \hologo{Xe}%
  \kern-.15em\relax
  \hologo{TeX}%
}
%    \end{macrocode}
%    \end{macro}
%
%    \begin{macro}{\HoLogoHtml@XeTeX}
%    \begin{macrocode}
\def\HoLogoHtml@XeTeX#1{%
  \HoLogoCss@XeTeX
  \HOLOGO@Span{XeTeX}{%
    \hologo{Xe}%
    \hologo{TeX}%
  }%
}
%    \end{macrocode}
%    \end{macro}
%    \begin{macro}{\HoLogoCss@XeTeX}
%    \begin{macrocode}
\def\HoLogoCss@XeTeX{%
  \Css{%
    span.HoLogo-XeTeX span.HoLogo-TeX{%
      margin-left:-.15em;%
    }%
  }%
  \global\let\HoLogoCss@XeTeX\relax
}
%    \end{macrocode}
%    \end{macro}
%
%    \begin{macro}{\HoLogo@XeLaTeX}
%    \begin{macrocode}
\def\HoLogo@XeLaTeX#1{%
  \hologo{Xe}%
  \kern-.13em%
  \hologo{LaTeX}%
}
%    \end{macrocode}
%    \end{macro}
%    \begin{macro}{\HoLogoHtml@XeLaTeX}
%    \begin{macrocode}
\def\HoLogoHtml@XeLaTeX#1{%
  \HoLogoCss@XeLaTeX
  \HOLOGO@Span{XeLaTeX}{%
    \hologo{Xe}%
    \hologo{LaTeX}%
  }%
}
%    \end{macrocode}
%    \end{macro}
%    \begin{macro}{\HoLogoCss@XeLaTeX}
%    \begin{macrocode}
\def\HoLogoCss@XeLaTeX{%
  \Css{%
    span.HoLogo-XeLaTeX span.HoLogo-Xe{%
      margin-right:-.13em;%
    }%
  }%
  \global\let\HoLogoCss@XeLaTeX\relax
}
%    \end{macrocode}
%    \end{macro}
%
% \subsubsection{\hologo{pdfTeX}, \hologo{pdfLaTeX}}
%
%    \begin{macro}{\HoLogo@pdfTeX}
%    \begin{macrocode}
\def\HoLogo@pdfTeX#1{%
  \HOLOGO@mbox{%
    #1{p}{P}df\hologo{TeX}%
  }%
}
%    \end{macrocode}
%    \end{macro}
%    \begin{macro}{\HoLogoCs@pdfTeX}
%    \begin{macrocode}
\def\HoLogoCs@pdfTeX#1{#1{p}{P}dfTeX}
%    \end{macrocode}
%    \end{macro}
%    \begin{macro}{\HoLogoBkm@pdfTeX}
%    \begin{macrocode}
\def\HoLogoBkm@pdfTeX#1{%
  #1{p}{P}df\hologo{TeX}%
}
%    \end{macrocode}
%    \end{macro}
%    \begin{macro}{\HoLogoHtml@pdfTeX}
%    \begin{macrocode}
\let\HoLogoHtml@pdfTeX\HoLogo@pdfTeX
%    \end{macrocode}
%    \end{macro}
%
%    \begin{macro}{\HoLogo@pdfLaTeX}
%    \begin{macrocode}
\def\HoLogo@pdfLaTeX#1{%
  \HOLOGO@mbox{%
    #1{p}{P}df\hologo{LaTeX}%
  }%
}
%    \end{macrocode}
%    \end{macro}
%    \begin{macro}{\HoLogoCs@pdfLaTeX}
%    \begin{macrocode}
\def\HoLogoCs@pdfLaTeX#1{#1{p}{P}dfLaTeX}
%    \end{macrocode}
%    \end{macro}
%    \begin{macro}{\HoLogoBkm@pdfLaTeX}
%    \begin{macrocode}
\def\HoLogoBkm@pdfLaTeX#1{%
  #1{p}{P}df\hologo{LaTeX}%
}
%    \end{macrocode}
%    \end{macro}
%    \begin{macro}{\HoLogoHtml@pdfLaTeX}
%    \begin{macrocode}
\let\HoLogoHtml@pdfLaTeX\HoLogo@pdfLaTeX
%    \end{macrocode}
%    \end{macro}
%
% \subsubsection{\hologo{VTeX}}
%
%    \begin{macro}{\HoLogo@VTeX}
%    \begin{macrocode}
\def\HoLogo@VTeX#1{%
  \HOLOGO@mbox{%
    V\hologo{TeX}%
  }%
}
%    \end{macrocode}
%    \end{macro}
%    \begin{macro}{\HoLogoHtml@VTeX}
%    \begin{macrocode}
\let\HoLogoHtml@VTeX\HoLogo@VTeX
%    \end{macrocode}
%    \end{macro}
%
% \subsubsection{\hologo{AmS}, \dots}
%
%    Source: class \xclass{amsdtx}
%
%    \begin{macro}{\HoLogo@AmS}
%    \begin{macrocode}
\def\HoLogo@AmS#1{%
  \HoLogoFont@font{AmS}{sy}{%
    A%
    \kern-.1667em%
    \lower.5ex\hbox{M}%
    \kern-.125em%
    S%
  }%
}
%    \end{macrocode}
%    \end{macro}
%    \begin{macro}{\HoLogoBkm@AmS}
%    \begin{macrocode}
\def\HoLogoBkm@AmS#1{AmS}
%    \end{macrocode}
%    \end{macro}
%    \begin{macro}{\HoLogoHtml@AmS}
%    \begin{macrocode}
\def\HoLogoHtml@AmS#1{%
  \HoLogoCss@AmS
%  \HoLogoFont@font{AmS}{sy}{%
    \HOLOGO@Span{AmS}{%
      A%
      \HOLOGO@Span{M}{M}%
      S%
    }%
%   }%
}
%    \end{macrocode}
%    \end{macro}
%    \begin{macro}{\HoLogoCss@AmS}
%    \begin{macrocode}
\def\HoLogoCss@AmS{%
  \Css{%
    span.HoLogo-AmS span.HoLogo-M{%
      position:relative;%
      top:.5ex;%
      margin-left:-.1667em;%
      margin-right:-.125em;%
      text-decoration:none;%
    }%
  }%
  \global\let\HoLogoCss@AmS\relax
}
%    \end{macrocode}
%    \end{macro}
%
%    \begin{macro}{\HoLogo@AmSTeX}
%    \begin{macrocode}
\def\HoLogo@AmSTeX#1{%
  \hologo{AmS}%
  \HOLOGO@hyphen
  \hologo{TeX}%
}
%    \end{macrocode}
%    \end{macro}
%    \begin{macro}{\HoLogoBkm@AmSTeX}
%    \begin{macrocode}
\def\HoLogoBkm@AmSTeX#1{AmS-TeX}%
%    \end{macrocode}
%    \end{macro}
%    \begin{macro}{\HoLogoHtml@AmSTeX}
%    \begin{macrocode}
\let\HoLogoHtml@AmSTeX\HoLogo@AmSTeX
%    \end{macrocode}
%    \end{macro}
%
%    \begin{macro}{\HoLogo@AmSLaTeX}
%    \begin{macrocode}
\def\HoLogo@AmSLaTeX#1{%
  \hologo{AmS}%
  \HOLOGO@hyphen
  \hologo{LaTeX}%
}
%    \end{macrocode}
%    \end{macro}
%    \begin{macro}{\HoLogoBkm@AmSLaTeX}
%    \begin{macrocode}
\def\HoLogoBkm@AmSLaTeX#1{AmS-LaTeX}%
%    \end{macrocode}
%    \end{macro}
%    \begin{macro}{\HoLogoHtml@AmSLaTeX}
%    \begin{macrocode}
\let\HoLogoHtml@AmSLaTeX\HoLogo@AmSLaTeX
%    \end{macrocode}
%    \end{macro}
%
% \subsubsection{\hologo{BibTeX}}
%
%    \begin{macro}{\HoLogo@BibTeX@sc}
%    A definition of \hologo{BibTeX} is provided in
%    the documentation source for the manual of \hologo{BibTeX}
%    \cite{btxdoc}.
%\begin{quote}
%\begin{verbatim}
%\def\BibTeX{%
%  {%
%    \rm
%    B%
%    \kern-.05em%
%    {%
%      \sc
%      i%
%      \kern-.025em %
%      b%
%    }%
%    \kern-.08em
%    T%
%    \kern-.1667em%
%    \lower.7ex\hbox{E}%
%    \kern-.125em%
%    X%
%  }%
%}
%\end{verbatim}
%\end{quote}
%    \begin{macrocode}
\def\HoLogo@BibTeX@sc#1{%
  B%
  \kern-.05em%
  \HoLogoFont@font{BibTeX}{sc}{%
    i%
    \kern-.025em%
    b%
  }%
  \HOLOGO@discretionary
  \kern-.08em%
  \hologo{TeX}%
}
%    \end{macrocode}
%    \end{macro}
%    \begin{macro}{\HoLogoHtml@BibTeX@sc}
%    \begin{macrocode}
\def\HoLogoHtml@BibTeX@sc#1{%
  \HoLogoCss@BibTeX@sc
  \HOLOGO@Span{BibTeX-sc}{%
    B%
    \HOLOGO@Span{i}{i}%
    \HOLOGO@Span{b}{b}%
    \hologo{TeX}%
  }%
}
%    \end{macrocode}
%    \end{macro}
%    \begin{macro}{\HoLogoCss@BibTeX@sc}
%    \begin{macrocode}
\def\HoLogoCss@BibTeX@sc{%
  \Css{%
    span.HoLogo-BibTeX-sc span.HoLogo-i{%
      margin-left:-.05em;%
      margin-right:-.025em;%
      font-variant:small-caps;%
    }%
  }%
  \Css{%
    span.HoLogo-BibTeX-sc span.HoLogo-b{%
      margin-right:-.08em;%
      font-variant:small-caps;%
    }%
  }%
  \global\let\HoLogoCss@BibTeX@sc\relax
}
%    \end{macrocode}
%    \end{macro}
%
%    \begin{macro}{\HoLogo@BibTeX@sf}
%    Variant \xoption{sf} avoids trouble with unavailable
%    small caps fonts (e.g., bold versions of Computer Modern or
%    Latin Modern). The definition is taken from
%    package \xpackage{dtklogos} \cite{dtklogos}.
%\begin{quote}
%\begin{verbatim}
%\DeclareRobustCommand{\BibTeX}{%
%  B%
%  \kern-.05em%
%  \hbox{%
%    $\m@th$% %% force math size calculations
%    \csname S@\f@size\endcsname
%    \fontsize\sf@size\z@
%    \math@fontsfalse
%    \selectfont
%    I%
%    \kern-.025em%
%    B
%  }%
%  \kern-.08em%
%  \-%
%  \TeX
%}
%\end{verbatim}
%\end{quote}
%    \begin{macrocode}
\def\HoLogo@BibTeX@sf#1{%
  B%
  \kern-.05em%
  \HoLogoFont@font{BibTeX}{bibsf}{%
    I%
    \kern-.025em%
    B%
  }%
  \HOLOGO@discretionary
  \kern-.08em%
  \hologo{TeX}%
}
%    \end{macrocode}
%    \end{macro}
%    \begin{macro}{\HoLogoHtml@BibTeX@sf}
%    \begin{macrocode}
\def\HoLogoHtml@BibTeX@sf#1{%
  \HoLogoCss@BibTeX@sf
  \HOLOGO@Span{BibTeX-sf}{%
    B%
    \HoLogoFont@font{BibTeX}{bibsf}{%
      \HOLOGO@Span{i}{I}%
      B%
    }%
    \hologo{TeX}%
  }%
}
%    \end{macrocode}
%    \end{macro}
%    \begin{macro}{\HoLogoCss@BibTeX@sf}
%    \begin{macrocode}
\def\HoLogoCss@BibTeX@sf{%
  \Css{%
    span.HoLogo-BibTeX-sf span.HoLogo-i{%
      margin-left:-.05em;%
      margin-right:-.025em;%
    }%
  }%
  \Css{%
    span.HoLogo-BibTeX-sf span.HoLogo-TeX{%
      margin-left:-.08em;%
    }%
  }%
  \global\let\HoLogoCss@BibTeX@sf\relax
}
%    \end{macrocode}
%    \end{macro}
%
%    \begin{macro}{\HoLogo@BibTeX}
%    \begin{macrocode}
\def\HoLogo@BibTeX{\HoLogo@BibTeX@sf}
%    \end{macrocode}
%    \end{macro}
%    \begin{macro}{\HoLogoHtml@BibTeX}
%    \begin{macrocode}
\def\HoLogoHtml@BibTeX{\HoLogoHtml@BibTeX@sf}
%    \end{macrocode}
%    \end{macro}
%
% \subsubsection{\hologo{BibTeX8}}
%
%    \begin{macro}{\HoLogo@BibTeX8}
%    \begin{macrocode}
\expandafter\def\csname HoLogo@BibTeX8\endcsname#1{%
  \hologo{BibTeX}%
  8%
}
%    \end{macrocode}
%    \end{macro}
%
%    \begin{macro}{\HoLogoBkm@BibTeX8}
%    \begin{macrocode}
\expandafter\def\csname HoLogoBkm@BibTeX8\endcsname#1{%
  \hologo{BibTeX}%
  8%
}
%    \end{macrocode}
%    \end{macro}
%    \begin{macro}{\HoLogoHtml@BibTeX8}
%    \begin{macrocode}
\expandafter
\let\csname HoLogoHtml@BibTeX8\expandafter\endcsname
\csname HoLogo@BibTeX8\endcsname
%    \end{macrocode}
%    \end{macro}
%
% \subsubsection{\hologo{ConTeXt}}
%
%    \begin{macro}{\HoLogo@ConTeXt@simple}
%    \begin{macrocode}
\def\HoLogo@ConTeXt@simple#1{%
  \HOLOGO@mbox{Con}%
  \HOLOGO@discretionary
  \HOLOGO@mbox{\hologo{TeX}t}%
}
%    \end{macrocode}
%    \end{macro}
%    \begin{macro}{\HoLogoHtml@ConTeXt@simple}
%    \begin{macrocode}
\let\HoLogoHtml@ConTeXt@simple\HoLogo@ConTeXt@simple
%    \end{macrocode}
%    \end{macro}
%
%    \begin{macro}{\HoLogo@ConTeXt@narrow}
%    This definition of logo \hologo{ConTeXt} with variant \xoption{narrow}
%    comes from TUGboat's class \xclass{ltugboat} (version 2010/11/15 v2.8).
%    \begin{macrocode}
\def\HoLogo@ConTeXt@narrow#1{%
  \HOLOGO@mbox{C\kern-.0333emon}%
  \HOLOGO@discretionary
  \kern-.0667em%
  \HOLOGO@mbox{\hologo{TeX}\kern-.0333emt}%
}
%    \end{macrocode}
%    \end{macro}
%    \begin{macro}{\HoLogoHtml@ConTeXt@narrow}
%    \begin{macrocode}
\def\HoLogoHtml@ConTeXt@narrow#1{%
  \HoLogoCss@ConTeXt@narrow
  \HOLOGO@Span{ConTeXt-narrow}{%
    \HOLOGO@Span{C}{C}%
    on%
    \hologo{TeX}%
    t%
  }%
}
%    \end{macrocode}
%    \end{macro}
%    \begin{macro}{\HoLogoCss@ConTeXt@narrow}
%    \begin{macrocode}
\def\HoLogoCss@ConTeXt@narrow{%
  \Css{%
    span.HoLogo-ConTeXt-narrow span.HoLogo-C{%
      margin-left:-.0333em;%
    }%
  }%
  \Css{%
    span.HoLogo-ConTeXt-narrow span.HoLogo-TeX{%
      margin-left:-.0667em;%
      margin-right:-.0333em;%
    }%
  }%
  \global\let\HoLogoCss@ConTeXt@narrow\relax
}
%    \end{macrocode}
%    \end{macro}
%
%    \begin{macro}{\HoLogo@ConTeXt}
%    \begin{macrocode}
\def\HoLogo@ConTeXt{\HoLogo@ConTeXt@narrow}
%    \end{macrocode}
%    \end{macro}
%    \begin{macro}{\HoLogoHtml@ConTeXt}
%    \begin{macrocode}
\def\HoLogoHtml@ConTeXt{\HoLogoHtml@ConTeXt@narrow}
%    \end{macrocode}
%    \end{macro}
%
% \subsubsection{\hologo{emTeX}}
%
%    \begin{macro}{\HoLogo@emTeX}
%    \begin{macrocode}
\def\HoLogo@emTeX#1{%
  \HOLOGO@mbox{#1{e}{E}m}%
  \HOLOGO@discretionary
  \hologo{TeX}%
}
%    \end{macrocode}
%    \end{macro}
%    \begin{macro}{\HoLogoCs@emTeX}
%    \begin{macrocode}
\def\HoLogoCs@emTeX#1{#1{e}{E}mTeX}%
%    \end{macrocode}
%    \end{macro}
%    \begin{macro}{\HoLogoBkm@emTeX}
%    \begin{macrocode}
\def\HoLogoBkm@emTeX#1{%
  #1{e}{E}m\hologo{TeX}%
}
%    \end{macrocode}
%    \end{macro}
%    \begin{macro}{\HoLogoHtml@emTeX}
%    \begin{macrocode}
\let\HoLogoHtml@emTeX\HoLogo@emTeX
%    \end{macrocode}
%    \end{macro}
%
% \subsubsection{\hologo{ExTeX}}
%
%    \begin{macro}{\HoLogo@ExTeX}
%    The definition is taken from the FAQ of the
%    project \hologo{ExTeX}
%    \cite{ExTeX-FAQ}.
%\begin{quote}
%\begin{verbatim}
%\def\ExTeX{%
%  \textrm{% Logo always with serifs
%    \ensuremath{%
%      \textstyle
%      \varepsilon_{%
%        \kern-0.15em%
%        \mathcal{X}%
%      }%
%    }%
%    \kern-.15em%
%    \TeX
%  }%
%}
%\end{verbatim}
%\end{quote}
%    \begin{macrocode}
\def\HoLogo@ExTeX#1{%
  \HoLogoFont@font{ExTeX}{rm}{%
    \ltx@mbox{%
      \HOLOGO@MathSetup
      $%
        \textstyle
        \varepsilon_{%
          \kern-0.15em%
          \HoLogoFont@font{ExTeX}{sy}{X}%
        }%
      $%
    }%
    \HOLOGO@discretionary
    \kern-.15em%
    \hologo{TeX}%
  }%
}
%    \end{macrocode}
%    \end{macro}
%    \begin{macro}{\HoLogoHtml@ExTeX}
%    \begin{macrocode}
\def\HoLogoHtml@ExTeX#1{%
  \HoLogoCss@ExTeX
  \HoLogoFont@font{ExTeX}{rm}{%
    \HOLOGO@Span{ExTeX}{%
      \ltx@mbox{%
        \HOLOGO@MathSetup
        $\textstyle\varepsilon$%
        \HOLOGO@Span{X}{$\textstyle\chi$}%
        \hologo{TeX}%
      }%
    }%
  }%
}
%    \end{macrocode}
%    \end{macro}
%    \begin{macro}{\HoLogoBkm@ExTeX}
%    \begin{macrocode}
\def\HoLogoBkm@ExTeX#1{%
  \HOLOGO@PdfdocUnicode{#1{e}{E}x}{\textepsilon\textchi}%
  \hologo{TeX}%
}
%    \end{macrocode}
%    \end{macro}
%    \begin{macro}{\HoLogoCss@ExTeX}
%    \begin{macrocode}
\def\HoLogoCss@ExTeX{%
  \Css{%
    span.HoLogo-ExTeX{%
      font-family:serif;%
    }%
  }%
  \Css{%
    span.HoLogo-ExTeX span.HoLogo-TeX{%
      margin-left:-.15em;%
    }%
  }%
  \global\let\HoLogoCss@ExTeX\relax
}
%    \end{macrocode}
%    \end{macro}
%
% \subsubsection{\hologo{MiKTeX}}
%
%    \begin{macro}{\HoLogo@MiKTeX}
%    \begin{macrocode}
\def\HoLogo@MiKTeX#1{%
  \HOLOGO@mbox{MiK}%
  \HOLOGO@discretionary
  \hologo{TeX}%
}
%    \end{macrocode}
%    \end{macro}
%    \begin{macro}{\HoLogoHtml@MiKTeX}
%    \begin{macrocode}
\let\HoLogoHtml@MiKTeX\HoLogo@MiKTeX
%    \end{macrocode}
%    \end{macro}
%
% \subsubsection{\hologo{OzTeX} and friends}
%
%    Source: \hologo{OzTeX} FAQ \cite{OzTeX}:
%    \begin{quote}
%      |\def\OzTeX{O\kern-.03em z\kern-.15em\TeX}|\\
%      (There is no kerning in OzMF, OzMP and OzTtH.)
%    \end{quote}
%
%    \begin{macro}{\HoLogo@OzTeX}
%    \begin{macrocode}
\def\HoLogo@OzTeX#1{%
  O%
  \kern-.03em %
  z%
  \kern-.15em %
  \hologo{TeX}%
}
%    \end{macrocode}
%    \end{macro}
%    \begin{macro}{\HoLogoHtml@OzTeX}
%    \begin{macrocode}
\def\HoLogoHtml@OzTeX#1{%
  \HoLogoCss@OzTeX
  \HOLOGO@Span{OzTeX}{%
    O%
    \HOLOGO@Span{z}{z}%
    \hologo{TeX}%
  }%
}
%    \end{macrocode}
%    \end{macro}
%    \begin{macro}{\HoLogoCss@OzTeX}
%    \begin{macrocode}
\def\HoLogoCss@OzTeX{%
  \Css{%
    span.HoLogo-OzTeX span.HoLogo-z{%
      margin-left:-.03em;%
      margin-right:-.15em;%
    }%
  }%
  \global\let\HoLogoCss@OzTeX\relax
}
%    \end{macrocode}
%    \end{macro}
%
%    \begin{macro}{\HoLogo@OzMF}
%    \begin{macrocode}
\def\HoLogo@OzMF#1{%
  \HOLOGO@mbox{OzMF}%
}
%    \end{macrocode}
%    \end{macro}
%    \begin{macro}{\HoLogo@OzMP}
%    \begin{macrocode}
\def\HoLogo@OzMP#1{%
  \HOLOGO@mbox{OzMP}%
}
%    \end{macrocode}
%    \end{macro}
%    \begin{macro}{\HoLogo@OzTtH}
%    \begin{macrocode}
\def\HoLogo@OzTtH#1{%
  \HOLOGO@mbox{OzTtH}%
}
%    \end{macrocode}
%    \end{macro}
%
% \subsubsection{\hologo{PCTeX}}
%
%    \begin{macro}{\HoLogo@PCTeX}
%    \begin{macrocode}
\def\HoLogo@PCTeX#1{%
  \HOLOGO@mbox{PC}%
  \hologo{TeX}%
}
%    \end{macrocode}
%    \end{macro}
%    \begin{macro}{\HoLogoHtml@PCTeX}
%    \begin{macrocode}
\let\HoLogoHtml@PCTeX\HoLogo@PCTeX
%    \end{macrocode}
%    \end{macro}
%
% \subsubsection{\hologo{PiCTeX}}
%
%    The original definitions from \xfile{pictex.tex} \cite{PiCTeX}:
%\begin{quote}
%\begin{verbatim}
%\def\PiC{%
%  P%
%  \kern-.12em%
%  \lower.5ex\hbox{I}%
%  \kern-.075em%
%  C%
%}
%\def\PiCTeX{%
%  \PiC
%  \kern-.11em%
%  \TeX
%}
%\end{verbatim}
%\end{quote}
%
%    \begin{macro}{\HoLogo@PiC}
%    \begin{macrocode}
\def\HoLogo@PiC#1{%
  P%
  \kern-.12em%
  \lower.5ex\hbox{I}%
  \kern-.075em%
  C%
  \HOLOGO@SpaceFactor
}
%    \end{macrocode}
%    \end{macro}
%    \begin{macro}{\HoLogoHtml@PiC}
%    \begin{macrocode}
\def\HoLogoHtml@PiC#1{%
  \HoLogoCss@PiC
  \HOLOGO@Span{PiC}{%
    P%
    \HOLOGO@Span{i}{I}%
    C%
  }%
}
%    \end{macrocode}
%    \end{macro}
%    \begin{macro}{\HoLogoCss@PiC}
%    \begin{macrocode}
\def\HoLogoCss@PiC{%
  \Css{%
    span.HoLogo-PiC span.HoLogo-i{%
      position:relative;%
      top:.5ex;%
      margin-left:-.12em;%
      margin-right:-.075em;%
      text-decoration:none;%
    }%
  }%
  \global\let\HoLogoCss@PiC\relax
}
%    \end{macrocode}
%    \end{macro}
%
%    \begin{macro}{\HoLogo@PiCTeX}
%    \begin{macrocode}
\def\HoLogo@PiCTeX#1{%
  \hologo{PiC}%
  \HOLOGO@discretionary
  \kern-.11em%
  \hologo{TeX}%
}
%    \end{macrocode}
%    \end{macro}
%    \begin{macro}{\HoLogoHtml@PiCTeX}
%    \begin{macrocode}
\def\HoLogoHtml@PiCTeX#1{%
  \HoLogoCss@PiCTeX
  \HOLOGO@Span{PiCTeX}{%
    \hologo{PiC}%
    \hologo{TeX}%
  }%
}
%    \end{macrocode}
%    \end{macro}
%    \begin{macro}{\HoLogoCss@PiCTeX}
%    \begin{macrocode}
\def\HoLogoCss@PiCTeX{%
  \Css{%
    span.HoLogo-PiCTeX span.HoLogo-PiC{%
      margin-right:-.11em;%
    }%
  }%
  \global\let\HoLogoCss@PiCTeX\relax
}
%    \end{macrocode}
%    \end{macro}
%
% \subsubsection{\hologo{teTeX}}
%
%    \begin{macro}{\HoLogo@teTeX}
%    \begin{macrocode}
\def\HoLogo@teTeX#1{%
  \HOLOGO@mbox{#1{t}{T}e}%
  \HOLOGO@discretionary
  \hologo{TeX}%
}
%    \end{macrocode}
%    \end{macro}
%    \begin{macro}{\HoLogoCs@teTeX}
%    \begin{macrocode}
\def\HoLogoCs@teTeX#1{#1{t}{T}dfTeX}
%    \end{macrocode}
%    \end{macro}
%    \begin{macro}{\HoLogoBkm@teTeX}
%    \begin{macrocode}
\def\HoLogoBkm@teTeX#1{%
  #1{t}{T}e\hologo{TeX}%
}
%    \end{macrocode}
%    \end{macro}
%    \begin{macro}{\HoLogoHtml@teTeX}
%    \begin{macrocode}
\let\HoLogoHtml@teTeX\HoLogo@teTeX
%    \end{macrocode}
%    \end{macro}
%
% \subsubsection{\hologo{TeX4ht}}
%
%    \begin{macro}{\HoLogo@TeX4ht}
%    \begin{macrocode}
\expandafter\def\csname HoLogo@TeX4ht\endcsname#1{%
  \HOLOGO@mbox{\hologo{TeX}4ht}%
}
%    \end{macrocode}
%    \end{macro}
%    \begin{macro}{\HoLogoHtml@TeX4ht}
%    \begin{macrocode}
\expandafter
\let\csname HoLogoHtml@TeX4ht\expandafter\endcsname
\csname HoLogo@TeX4ht\endcsname
%    \end{macrocode}
%    \end{macro}
%
%
% \subsubsection{\hologo{SageTeX}}
%
%    \begin{macro}{\HoLogo@SageTeX}
%    \begin{macrocode}
\def\HoLogo@SageTeX#1{%
  \HOLOGO@mbox{Sage}%
  \HOLOGO@discretionary
  \HOLOGO@NegativeKerning{eT,oT,To}%
  \hologo{TeX}%
}
%    \end{macrocode}
%    \end{macro}
%    \begin{macro}{\HoLogoHtml@SageTeX}
%    \begin{macrocode}
\let\HoLogoHtml@SageTeX\HoLogo@SageTeX
%    \end{macrocode}
%    \end{macro}
%
% \subsection{\hologo{METAFONT} and friends}
%
%    \begin{macro}{\HoLogo@METAFONT}
%    \begin{macrocode}
\def\HoLogo@METAFONT#1{%
  \HoLogoFont@font{METAFONT}{logo}{%
    \HOLOGO@mbox{META}%
    \HOLOGO@discretionary
    \HOLOGO@mbox{FONT}%
  }%
}
%    \end{macrocode}
%    \end{macro}
%
%    \begin{macro}{\HoLogo@METAPOST}
%    \begin{macrocode}
\def\HoLogo@METAPOST#1{%
  \HoLogoFont@font{METAPOST}{logo}{%
    \HOLOGO@mbox{META}%
    \HOLOGO@discretionary
    \HOLOGO@mbox{POST}%
  }%
}
%    \end{macrocode}
%    \end{macro}
%
%    \begin{macro}{\HoLogo@MetaFun}
%    \begin{macrocode}
\def\HoLogo@MetaFun#1{%
  \HOLOGO@mbox{Meta}%
  \HOLOGO@discretionary
  \HOLOGO@mbox{Fun}%
}
%    \end{macrocode}
%    \end{macro}
%
%    \begin{macro}{\HoLogo@MetaPost}
%    \begin{macrocode}
\def\HoLogo@MetaPost#1{%
  \HOLOGO@mbox{Meta}%
  \HOLOGO@discretionary
  \HOLOGO@mbox{Post}%
}
%    \end{macrocode}
%    \end{macro}
%
% \subsection{Others}
%
% \subsubsection{\hologo{biber}}
%
%    \begin{macro}{\HoLogo@biber}
%    \begin{macrocode}
\def\HoLogo@biber#1{%
  \HOLOGO@mbox{#1{b}{B}i}%
  \HOLOGO@discretionary
  \HOLOGO@mbox{ber}%
}
%    \end{macrocode}
%    \end{macro}
%    \begin{macro}{\HoLogoCs@biber}
%    \begin{macrocode}
\def\HoLogoCs@biber#1{#1{b}{B}iber}
%    \end{macrocode}
%    \end{macro}
%    \begin{macro}{\HoLogoBkm@biber}
%    \begin{macrocode}
\def\HoLogoBkm@biber#1{%
  #1{b}{B}iber%
}
%    \end{macrocode}
%    \end{macro}
%    \begin{macro}{\HoLogoHtml@biber}
%    \begin{macrocode}
\let\HoLogoHtml@biber\HoLogo@biber
%    \end{macrocode}
%    \end{macro}
%
% \subsubsection{\hologo{KOMAScript}}
%
%    \begin{macro}{\HoLogo@KOMAScript}
%    The definition for \hologo{KOMAScript} is taken
%    from \hologo{KOMAScript} (\xfile{scrlogo.dtx}, reformatted) \cite{scrlogo}:
%\begin{quote}
%\begin{verbatim}
%\@ifundefined{KOMAScript}{%
%  \DeclareRobustCommand{\KOMAScript}{%
%    \textsf{%
%      K\kern.05em O\kern.05emM\kern.05em A%
%      \kern.1em-\kern.1em %
%      Script%
%    }%
%  }%
%}{}
%\end{verbatim}
%\end{quote}
%    \begin{macrocode}
\def\HoLogo@KOMAScript#1{%
  \HoLogoFont@font{KOMAScript}{sf}{%
    \HOLOGO@mbox{%
      K\kern.05em%
      O\kern.05em%
      M\kern.05em%
      A%
    }%
    \kern.1em%
    \HOLOGO@hyphen
    \kern.1em%
    \HOLOGO@mbox{Script}%
  }%
}
%    \end{macrocode}
%    \end{macro}
%    \begin{macro}{\HoLogoBkm@KOMAScript}
%    \begin{macrocode}
\def\HoLogoBkm@KOMAScript#1{%
  KOMA-Script%
}
%    \end{macrocode}
%    \end{macro}
%    \begin{macro}{\HoLogoHtml@KOMAScript}
%    \begin{macrocode}
\def\HoLogoHtml@KOMAScript#1{%
  \HoLogoCss@KOMAScript
  \HoLogoFont@font{KOMAScript}{sf}{%
    \HOLOGO@Span{KOMAScript}{%
      K%
      \HOLOGO@Span{O}{O}%
      M%
      \HOLOGO@Span{A}{A}%
      \HOLOGO@Span{hyphen}{-}%
      Script%
    }%
  }%
}
%    \end{macrocode}
%    \end{macro}
%    \begin{macro}{\HoLogoCss@KOMAScript}
%    \begin{macrocode}
\def\HoLogoCss@KOMAScript{%
  \Css{%
    span.HoLogo-KOMAScript{%
      font-family:sans-serif;%
    }%
  }%
  \Css{%
    span.HoLogo-KOMAScript span.HoLogo-O{%
      padding-left:.05em;%
      padding-right:.05em;%
    }%
  }%
  \Css{%
    span.HoLogo-KOMAScript span.HoLogo-A{%
      padding-left:.05em;%
    }%
  }%
  \Css{%
    span.HoLogo-KOMAScript span.HoLogo-hyphen{%
      padding-left:.1em;%
      padding-right:.1em;%
    }%
  }%
  \global\let\HoLogoCss@KOMAScript\relax
}
%    \end{macrocode}
%    \end{macro}
%
% \subsubsection{\hologo{LyX}}
%
%    \begin{macro}{\HoLogo@LyX}
%    The definition is taken from the documentation source files
%    of \hologo{LyX}, \xfile{Intro.lyx} \cite{LyX}:
%\begin{quote}
%\begin{verbatim}
%\def\LyX{%
%  \texorpdfstring{%
%    L\kern-.1667em\lower.25em\hbox{Y}\kern-.125emX\@%
%  }{%
%    LyX%
%  }%
%}
%\end{verbatim}
%\end{quote}
%    \begin{macrocode}
\def\HoLogo@LyX#1{%
  L%
  \kern-.1667em%
  \lower.25em\hbox{Y}%
  \kern-.125em%
  X%
  \HOLOGO@SpaceFactor
}
%    \end{macrocode}
%    \end{macro}
%    \begin{macro}{\HoLogoHtml@LyX}
%    \begin{macrocode}
\def\HoLogoHtml@LyX#1{%
  \HoLogoCss@LyX
  \HOLOGO@Span{LyX}{%
    L%
    \HOLOGO@Span{y}{Y}%
    X%
  }%
}
%    \end{macrocode}
%    \end{macro}
%    \begin{macro}{\HoLogoCss@LyX}
%    \begin{macrocode}
\def\HoLogoCss@LyX{%
  \Css{%
    span.HoLogo-LyX span.HoLogo-y{%
      position:relative;%
      top:.25em;%
      margin-left:-.1667em;%
      margin-right:-.125em;%
      text-decoration:none;%
    }%
  }%
  \global\let\HoLogoCss@LyX\relax
}
%    \end{macrocode}
%    \end{macro}
%
% \subsubsection{\hologo{NTS}}
%
%    \begin{macro}{\HoLogo@NTS}
%    Definition for \hologo{NTS} can be found in
%    package \xpackage{etex\textunderscore man} for the \hologo{eTeX} manual \cite{etexman}
%    and in package \xpackage{dtklogos} \cite{dtklogos}:
%\begin{quote}
%\begin{verbatim}
%\def\NTS{%
%  \leavevmode
%  \hbox{%
%    $%
%      \cal N%
%      \kern-0.35em%
%      \lower0.5ex\hbox{$\cal T$}%
%      \kern-0.2em%
%      S%
%    $%
%  }%
%}
%\end{verbatim}
%\end{quote}
%    \begin{macrocode}
\def\HoLogo@NTS#1{%
  \HoLogoFont@font{NTS}{sy}{%
    N\/%
    \kern-.35em%
    \lower.5ex\hbox{T\/}%
    \kern-.2em%
    S\/%
  }%
  \HOLOGO@SpaceFactor
}
%    \end{macrocode}
%    \end{macro}
%
% \subsubsection{\Hologo{TTH} (\hologo{TeX} to HTML translator)}
%
%    Source: \url{http://hutchinson.belmont.ma.us/tth/}
%    In the HTML source the second `T' is printed as subscript.
%\begin{quote}
%\begin{verbatim}
%T<sub>T</sub>H
%\end{verbatim}
%\end{quote}
%    \begin{macro}{\HoLogo@TTH}
%    \begin{macrocode}
\def\HoLogo@TTH#1{%
  \ltx@mbox{%
    T\HOLOGO@SubScript{T}H%
  }%
  \HOLOGO@SpaceFactor
}
%    \end{macrocode}
%    \end{macro}
%
%    \begin{macro}{\HoLogoHtml@TTH}
%    \begin{macrocode}
\def\HoLogoHtml@TTH#1{%
  T\HCode{<sub>}T\HCode{</sub>}H%
}
%    \end{macrocode}
%    \end{macro}
%
% \subsubsection{\Hologo{HanTheThanh}}
%
%    Partial source: Package \xpackage{dtklogos}.
%    The double accent is U+1EBF (latin small letter e with circumflex
%    and acute).
%    \begin{macro}{\HoLogo@HanTheThanh}
%    \begin{macrocode}
\def\HoLogo@HanTheThanh#1{%
  \ltx@mbox{H\`an}%
  \HOLOGO@space
  \ltx@mbox{%
    Th%
    \HOLOGO@IfCharExists{"1EBF}{%
      \char"1EBF\relax
    }{%
      \^e\hbox to 0pt{\hss\raise .5ex\hbox{\'{}}}%
    }%
  }%
  \HOLOGO@space
  \ltx@mbox{Th\`anh}%
}
%    \end{macrocode}
%    \end{macro}
%    \begin{macro}{\HoLogoBkm@HanTheThanh}
%    \begin{macrocode}
\def\HoLogoBkm@HanTheThanh#1{%
  H\`an %
  Th\HOLOGO@PdfdocUnicode{\^e}{\9036\277} %
  Th\`anh%
}
%    \end{macrocode}
%    \end{macro}
%    \begin{macro}{\HoLogoHtml@HanTheThanh}
%    \begin{macrocode}
\def\HoLogoHtml@HanTheThanh#1{%
  H\`an %
  Th\HCode{&\ltx@hashchar x1ebf;} %
  Th\`anh%
}
%    \end{macrocode}
%    \end{macro}
%
% \subsection{Driver detection}
%
%    \begin{macrocode}
\HOLOGO@IfExists\InputIfFileExists{%
  \InputIfFileExists{hologo.cfg}{}{}%
}{%
  \ltx@IfUndefined{pdf@filesize}{%
    \def\HOLOGO@InputIfExists{%
      \openin\HOLOGO@temp=hologo.cfg\relax
      \ifeof\HOLOGO@temp
        \closein\HOLOGO@temp
      \else
        \closein\HOLOGO@temp
        \begingroup
          \def\x{LaTeX2e}%
        \expandafter\endgroup
        \ifx\fmtname\x
          \input{hologo.cfg}%
        \else
          \input hologo.cfg\relax
        \fi
      \fi
    }%
    \ltx@IfUndefined{newread}{%
      \chardef\HOLOGO@temp=15 %
      \def\HOLOGO@CheckRead{%
        \ifeof\HOLOGO@temp
          \HOLOGO@InputIfExists
        \else
          \ifcase\HOLOGO@temp
            \@PackageWarningNoLine{hologo}{%
              Configuration file ignored, because\MessageBreak
              a free read register could not be found%
            }%
          \else
            \begingroup
              \count\ltx@cclv=\HOLOGO@temp
              \advance\ltx@cclv by \ltx@minusone
              \edef\x{\endgroup
                \chardef\noexpand\HOLOGO@temp=\the\count\ltx@cclv
                \relax
              }%
            \x
          \fi
        \fi
      }%
    }{%
      \csname newread\endcsname\HOLOGO@temp
      \HOLOGO@InputIfExists
    }%
  }{%
    \edef\HOLOGO@temp{\pdf@filesize{hologo.cfg}}%
    \ifx\HOLOGO@temp\ltx@empty
    \else
      \ifnum\HOLOGO@temp>0 %
        \begingroup
          \def\x{LaTeX2e}%
        \expandafter\endgroup
        \ifx\fmtname\x
          \input{hologo.cfg}%
        \else
          \input hologo.cfg\relax
        \fi
      \else
        \@PackageInfoNoLine{hologo}{%
          Empty configuration file `hologo.cfg' ignored%
        }%
      \fi
    \fi
  }%
}
%    \end{macrocode}
%
%    \begin{macrocode}
\def\HOLOGO@temp#1#2{%
  \kv@define@key{HoLogoDriver}{#1}[]{%
    \begingroup
      \def\HOLOGO@temp{##1}%
      \ltx@onelevel@sanitize\HOLOGO@temp
      \ifx\HOLOGO@temp\ltx@empty
      \else
        \@PackageError{hologo}{%
          Value (\HOLOGO@temp) not permitted for option `#1'%
        }%
        \@ehc
      \fi
    \endgroup
    \def\hologoDriver{#2}%
  }%
}%
\def\HOLOGO@@temp#1#2{%
  \ifx\kv@value\relax
    \HOLOGO@temp{#1}{#1}%
  \else
    \HOLOGO@temp{#1}{#2}%
  \fi
}%
\kv@parse@normalized{%
  pdftex,%
  luatex=pdftex,%
  dvipdfm,%
  dvipdfmx=dvipdfm,%
  dvips,%
  dvipsone=dvips,%
  xdvi=dvips,%
  xetex,%
  vtex,%
}\HOLOGO@@temp
%    \end{macrocode}
%
%    \begin{macrocode}
\kv@define@key{HoLogoDriver}{driverfallback}{%
  \def\HOLOGO@DriverFallback{#1}%
}
%    \end{macrocode}
%
%    \begin{macro}{\HOLOGO@DriverFallback}
%    \begin{macrocode}
\def\HOLOGO@DriverFallback{dvips}
%    \end{macrocode}
%    \end{macro}
%
%    \begin{macro}{\hologoDriverSetup}
%    \begin{macrocode}
\def\hologoDriverSetup{%
  \let\hologoDriver\ltx@undefined
  \HOLOGO@DriverSetup
}
%    \end{macrocode}
%    \end{macro}
%
%    \begin{macro}{\HOLOGO@DriverSetup}
%    \begin{macrocode}
\def\HOLOGO@DriverSetup#1{%
  \kvsetkeys{HoLogoDriver}{#1}%
  \HOLOGO@CheckDriver
  \ltx@ifundefined{hologoDriver}{%
    \begingroup
    \edef\x{\endgroup
      \noexpand\kvsetkeys{HoLogoDriver}{\HOLOGO@DriverFallback}%
    }\x
  }{}%
  \@PackageInfoNoLine{hologo}{Using driver `\hologoDriver'}%
}
%    \end{macrocode}
%    \end{macro}
%
%    \begin{macro}{\HOLOGO@CheckDriver}
%    \begin{macrocode}
\def\HOLOGO@CheckDriver{%
  \ifpdf
    \def\hologoDriver{pdftex}%
    \let\HOLOGO@pdfliteral\pdfliteral
    \ifluatex
      \ifx\pdfextension\@undefined\else
        \protected\def\pdfliteral{\pdfextension literal}%
        \let\HOLOGO@pdfliteral\pdfliteral
      \fi
      \ltx@IfUndefined{HOLOGO@pdfliteral}{%
        \ifnum\luatexversion<36 %
        \else
          \begingroup
            \let\HOLOGO@temp\endgroup
            \ifcase0%
                \directlua{%
                  if tex.enableprimitives then %
                    tex.enableprimitives('HOLOGO@', {'pdfliteral'})%
                  else %
                    tex.print('1')%
                  end%
                }%
                \ifx\HOLOGO@pdfliteral\@undefined 1\fi%
                \relax%
              \endgroup
              \let\HOLOGO@temp\relax
              \global\let\HOLOGO@pdfliteral\HOLOGO@pdfliteral
            \fi%
          \HOLOGO@temp
        \fi
      }{}%
    \fi
    \ltx@IfUndefined{HOLOGO@pdfliteral}{%
      \@PackageWarningNoLine{hologo}{%
        Cannot find \string\pdfliteral
      }%
    }{}%
  \else
    \ifxetex
      \def\hologoDriver{xetex}%
    \else
      \ifvtex
        \def\hologoDriver{vtex}%
      \fi
    \fi
  \fi
}
%    \end{macrocode}
%    \end{macro}
%
%    \begin{macro}{\HOLOGO@WarningUnsupportedDriver}
%    \begin{macrocode}
\def\HOLOGO@WarningUnsupportedDriver#1{%
  \@PackageWarningNoLine{hologo}{%
    Logo `#1' needs driver specific macros,\MessageBreak
    but driver `\hologoDriver' is not supported.\MessageBreak
    Use a different driver or\MessageBreak
    load package `graphics' or `pgf'%
  }%
}
%    \end{macrocode}
%    \end{macro}
%
% \subsubsection{Reflect box macros}
%
%    Skip driver part if not needed.
%    \begin{macrocode}
\ltx@IfUndefined{reflectbox}{}{%
  \ltx@IfUndefined{rotatebox}{}{%
    \HOLOGO@AtEnd
  }%
}
\ltx@IfUndefined{pgftext}{}{%
  \HOLOGO@AtEnd
}
\ltx@IfUndefined{psscalebox}{}{%
  \HOLOGO@AtEnd
}
%    \end{macrocode}
%
%    \begin{macrocode}
\def\HOLOGO@temp{LaTeX2e}
\ifx\fmtname\HOLOGO@temp
  \RequirePackage{kvoptions}[2011/06/30]%
  \ProcessKeyvalOptions{HoLogoDriver}%
\fi
\HOLOGO@DriverSetup{}
%    \end{macrocode}
%
%    \begin{macro}{\HOLOGO@ReflectBox}
%    \begin{macrocode}
\def\HOLOGO@ReflectBox#1{%
  \begingroup
    \setbox\ltx@zero\hbox{\begingroup#1\endgroup}%
    \setbox\ltx@two\hbox{%
      \kern\wd\ltx@zero
      \csname HOLOGO@ScaleBox@\hologoDriver\endcsname{-1}{1}{%
        \hbox to 0pt{\copy\ltx@zero\hss}%
      }%
    }%
    \wd\ltx@two=\wd\ltx@zero
    \box\ltx@two
  \endgroup
}
%    \end{macrocode}
%    \end{macro}
%
%    \begin{macro}{\HOLOGO@PointReflectBox}
%    \begin{macrocode}
\def\HOLOGO@PointReflectBox#1{%
  \begingroup
    \setbox\ltx@zero\hbox{\begingroup#1\endgroup}%
    \setbox\ltx@two\hbox{%
      \kern\wd\ltx@zero
      \raise\ht\ltx@zero\hbox{%
        \csname HOLOGO@ScaleBox@\hologoDriver\endcsname{-1}{-1}{%
          \hbox to 0pt{\copy\ltx@zero\hss}%
        }%
      }%
    }%
    \wd\ltx@two=\wd\ltx@zero
    \box\ltx@two
  \endgroup
}
%    \end{macrocode}
%    \end{macro}
%
%    We must define all variants because of dynamic driver setup.
%    \begin{macrocode}
\def\HOLOGO@temp#1#2{#2}
%    \end{macrocode}
%
%    \begin{macro}{\HOLOGO@ScaleBox@pdftex}
%    \begin{macrocode}
\HOLOGO@temp{pdftex}{%
  \def\HOLOGO@ScaleBox@pdftex#1#2#3{%
    \HOLOGO@pdfliteral{%
      q #1 0 0 #2 0 0 cm%
    }%
    #3%
    \HOLOGO@pdfliteral{%
      Q%
    }%
  }%
}
%    \end{macrocode}
%    \end{macro}
%    \begin{macro}{\HOLOGO@ScaleBox@dvips}
%    \begin{macrocode}
\HOLOGO@temp{dvips}{%
  \def\HOLOGO@ScaleBox@dvips#1#2#3{%
    \special{ps:%
      gsave %
      currentpoint %
      currentpoint translate %
      #1 #2 scale %
      neg exch neg exch translate%
    }%
    #3%
    \special{ps:%
      currentpoint %
      grestore %
      moveto%
    }%
  }%
}
%    \end{macrocode}
%    \end{macro}
%    \begin{macro}{\HOLOGO@ScaleBox@dvipdfm}
%    \begin{macrocode}
\HOLOGO@temp{dvipdfm}{%
  \let\HOLOGO@ScaleBox@dvipdfm\HOLOGO@ScaleBox@dvips
}
%    \end{macrocode}
%    \end{macro}
%    Since \hologo{XeTeX} v0.6.
%    \begin{macro}{\HOLOGO@ScaleBox@xetex}
%    \begin{macrocode}
\HOLOGO@temp{xetex}{%
  \def\HOLOGO@ScaleBox@xetex#1#2#3{%
    \special{x:gsave}%
    \special{x:scale #1 #2}%
    #3%
    \special{x:grestore}%
  }%
}
%    \end{macrocode}
%    \end{macro}
%    \begin{macro}{\HOLOGO@ScaleBox@vtex}
%    \begin{macrocode}
\HOLOGO@temp{vtex}{%
  \def\HOLOGO@ScaleBox@vtex#1#2#3{%
    \special{r(#1,0,0,#2,0,0}%
    #3%
    \special{r)}%
  }%
}
%    \end{macrocode}
%    \end{macro}
%
%    \begin{macrocode}
\HOLOGO@AtEnd%
%</package>
%    \end{macrocode}
%
% \section{Test}
%
% \subsection{Catcode checks for loading}
%
%    \begin{macrocode}
%<*test1>
%    \end{macrocode}
%    \begin{macrocode}
\catcode`\{=1 %
\catcode`\}=2 %
\catcode`\#=6 %
\catcode`\@=11 %
\expandafter\ifx\csname count@\endcsname\relax
  \countdef\count@=255 %
\fi
\expandafter\ifx\csname @gobble\endcsname\relax
  \long\def\@gobble#1{}%
\fi
\expandafter\ifx\csname @firstofone\endcsname\relax
  \long\def\@firstofone#1{#1}%
\fi
\expandafter\ifx\csname loop\endcsname\relax
  \expandafter\@firstofone
\else
  \expandafter\@gobble
\fi
{%
  \def\loop#1\repeat{%
    \def\body{#1}%
    \iterate
  }%
  \def\iterate{%
    \body
      \let\next\iterate
    \else
      \let\next\relax
    \fi
    \next
  }%
  \let\repeat=\fi
}%
\def\RestoreCatcodes{}
\count@=0 %
\loop
  \edef\RestoreCatcodes{%
    \RestoreCatcodes
    \catcode\the\count@=\the\catcode\count@\relax
  }%
\ifnum\count@<255 %
  \advance\count@ 1 %
\repeat

\def\RangeCatcodeInvalid#1#2{%
  \count@=#1\relax
  \loop
    \catcode\count@=15 %
  \ifnum\count@<#2\relax
    \advance\count@ 1 %
  \repeat
}
\def\RangeCatcodeCheck#1#2#3{%
  \count@=#1\relax
  \loop
    \ifnum#3=\catcode\count@
    \else
      \errmessage{%
        Character \the\count@\space
        with wrong catcode \the\catcode\count@\space
        instead of \number#3%
      }%
    \fi
  \ifnum\count@<#2\relax
    \advance\count@ 1 %
  \repeat
}
\def\space{ }
\expandafter\ifx\csname LoadCommand\endcsname\relax
  \def\LoadCommand{\input hologo.sty\relax}%
\fi
\def\Test{%
  \RangeCatcodeInvalid{0}{47}%
  \RangeCatcodeInvalid{58}{64}%
  \RangeCatcodeInvalid{91}{96}%
  \RangeCatcodeInvalid{123}{255}%
  \catcode`\@=12 %
  \catcode`\\=0 %
  \catcode`\%=14 %
  \LoadCommand
  \RangeCatcodeCheck{0}{36}{15}%
  \RangeCatcodeCheck{37}{37}{14}%
  \RangeCatcodeCheck{38}{47}{15}%
  \RangeCatcodeCheck{48}{57}{12}%
  \RangeCatcodeCheck{58}{63}{15}%
  \RangeCatcodeCheck{64}{64}{12}%
  \RangeCatcodeCheck{65}{90}{11}%
  \RangeCatcodeCheck{91}{91}{15}%
  \RangeCatcodeCheck{92}{92}{0}%
  \RangeCatcodeCheck{93}{96}{15}%
  \RangeCatcodeCheck{97}{122}{11}%
  \RangeCatcodeCheck{123}{255}{15}%
  \RestoreCatcodes
}
\Test
\csname @@end\endcsname
\end
%    \end{macrocode}
%    \begin{macrocode}
%</test1>
%    \end{macrocode}
%
% \subsection{Spacefactor}
%
%    The space factor must be 1000 after a logo. If it is greater 1000
%    then the following space is a space after a sentence closing point.
%    If the space factor is smaller 1000 then an immediate following
%    dot is interpreted as abbreviation, not sentence closing point.
%
%    \begin{macrocode}
%<*test-spacefactor>
\NeedsTeXFormat{LaTeX2e}
\documentclass{article}
\usepackage{hologo}[2016/05/12]
\usepackage{kvsetkeys}
\usepackage{qstest}
\IncludeTests{*}
\LogTests{log}{*}{*}
\begin{document}
\begin{qstest}{spacefactor}{spacefactor}
\newcommand*{\Test}[1]{%
  \sbox0{%
    \hologo{#1}%
    \Expect*{1000 (#1)}*{\the\spacefactor\space(#1)}%
  }%
}%
\makeatletter
\def\TestList{}
\def\hologoEntry#1#2#3{%
  \edef\TestList{%
    \ifx\TestList\@empty
    \else
      \TestList,%
    \fi
    #1%
    \ifx\\#2\\%
    \else
      ={variant=#2}%
    \fi
  }%
}
\hologoList
\expandafter\kv@parse@normalized\expandafter{%
  \TestList
}{%
  \begingroup
    \let\@logo=\kv@key
    \ifx\kv@value\relax
    \else
      \expandafter\hologoLogoSetup\expandafter\@logo\expandafter{%
        \kv@value
      }%
    \fi
    \Test\@logo
  \endgroup
  \@gobbletwo
}
\end{qstest}
\end{document}
%</test-spacefactor>
%    \end{macrocode}
%
% \subsection{Complete list}
%
%    \begin{macrocode}
%<*test-list>
\NeedsTeXFormat{LaTeX2e}
\documentclass[12pt,a4paper]{article}
\usepackage{hologo}[2016/05/12]
\usepackage[T1]{fontenc}
\usepackage{lmodern}
\usepackage{parskip}
\usepackage[unicode]{hyperref}[2011/09/28]
\usepackage{bookmark}[2011/09/19]
\bookmarksetup{%
  numbered,%
  open,%
  openlevel=2,%
}
\renewcommand*{\contentsname}{List of logos}
\begin{document}
\tableofcontents
\def\TestFont#1#2#3#4#5#6{%
  \begingroup
    \usefont{#3}{#4}{#5}{#6}%
    \HologoVariant{#1}{#2}/\hologoVariant{#1}{#2}%
    \quad
    \begingroup\scriptsize\hologoVariant{#1}{#2}\endgroup
    \quad
  \endgroup
  (#3/#4/#5/#6)%
  \par
}
\makeatletter
\def\hologoEntry#1#2#3{%
  \section{%
    \HologoVariant{#1}{#2}/\hologoVariant{#1}{#2} %
    {[#1\ifx\\#2\\\else\space(#2)\fi]}% hash-ok
  }% braces around [] because of bug in tex4ht
  \begingroup
    \hypersetup{unicode=false}%
    \bookmark[%
      dest=\@currentHref,%
      rellevel=1,%
      keeplevel,%
    ]{%
      \HologoVariant{#1}{#2}/\hologoVariant{#1}{#2} %
      (PDFDocEncoding)%
    }%
  \endgroup
  \TestFont{#1}{#2}{OT1}{cmr}{m}{n}%
  \TestFont{#1}{#2}{OT1}{cmss}{m}{n}%
  \TestFont{#1}{#2}{OT1}{cmr}{b}{n}%
  \TestFont{#1}{#2}{OT1}{cmr}{m}{it}%
  \TestFont{#1}{#2}{OT1}{cmtt}{m}{n}%
  \TestFont{#1}{#2}{T1}{lmr}{m}{n}%
  \TestFont{#1}{#2}{T1}{lmss}{m}{n}%
  \TestFont{#1}{#2}{T1}{lmr}{b}{n}%
  \TestFont{#1}{#2}{T1}{lmr}{m}{it}%
  \TestFont{#1}{#2}{T1}{lmtt}{m}{n}%
  \TestFont{#1}{#2}{T1}{lmvtt}{m}{n}%
  \TestFont{#1}{#2}{T1}{qtm}{m}{n}%
  \TestFont{#1}{#2}{T1}{qhv}{m}{n}%
  \TestFont{#1}{#2}{T1}{qtm}{b}{n}%
  \TestFont{#1}{#2}{T1}{qtm}{m}{it}%
  \TestFont{#1}{#2}{T1}{qcr}{m}{n}%
  \newpage
}
\makeatother
\hologoList
\end{document}
%</test-list>
%    \end{macrocode}
%
% \section{Installation}
%
% \subsection{Download}
%
% \paragraph{Package.} This package is available on
% CTAN\footnote{\url{ftp://ftp.ctan.org/tex-archive/}}:
% \begin{description}
% \item[\CTAN{macros/latex/contrib/oberdiek/hologo.dtx}] The source file.
% \item[\CTAN{macros/latex/contrib/oberdiek/hologo.pdf}] Documentation.
% \end{description}
%
%
% \paragraph{Bundle.} All the packages of the bundle `oberdiek'
% are also available in a TDS compliant ZIP archive. There
% the packages are already unpacked and the documentation files
% are generated. The files and directories obey the TDS standard.
% \begin{description}
% \item[\CTAN{install/macros/latex/contrib/oberdiek.tds.zip}]
% \end{description}
% \emph{TDS} refers to the standard ``A Directory Structure
% for \TeX\ Files'' (\CTAN{tds/tds.pdf}). Directories
% with \xfile{texmf} in their name are usually organized this way.
%
% \subsection{Bundle installation}
%
% \paragraph{Unpacking.} Unpack the \xfile{oberdiek.tds.zip} in the
% TDS tree (also known as \xfile{texmf} tree) of your choice.
% Example (linux):
% \begin{quote}
%   |unzip oberdiek.tds.zip -d ~/texmf|
% \end{quote}
%
% \paragraph{Script installation.}
% Check the directory \xfile{TDS:scripts/oberdiek/} for
% scripts that need further installation steps.
% Package \xpackage{attachfile2} comes with the Perl script
% \xfile{pdfatfi.pl} that should be installed in such a way
% that it can be called as \texttt{pdfatfi}.
% Example (linux):
% \begin{quote}
%   |chmod +x scripts/oberdiek/pdfatfi.pl|\\
%   |cp scripts/oberdiek/pdfatfi.pl /usr/local/bin/|
% \end{quote}
%
% \subsection{Package installation}
%
% \paragraph{Unpacking.} The \xfile{.dtx} file is a self-extracting
% \docstrip\ archive. The files are extracted by running the
% \xfile{.dtx} through \plainTeX:
% \begin{quote}
%   \verb|tex hologo.dtx|
% \end{quote}
%
% \paragraph{TDS.} Now the different files must be moved into
% the different directories in your installation TDS tree
% (also known as \xfile{texmf} tree):
% \begin{quote}
% \def\t{^^A
% \begin{tabular}{@{}>{\ttfamily}l@{ $\rightarrow$ }>{\ttfamily}l@{}}
%   hologo.sty & tex/generic/oberdiek/hologo.sty\\
%   hologo.pdf & doc/latex/oberdiek/hologo.pdf\\
%   example/hologo-example.tex & doc/latex/oberdiek/example/hologo-example.tex\\
%   test/hologo-test1.tex & doc/latex/oberdiek/test/hologo-test1.tex\\
%   test/hologo-test-spacefactor.tex & doc/latex/oberdiek/test/hologo-test-spacefactor.tex\\
%   test/hologo-test-list.tex & doc/latex/oberdiek/test/hologo-test-list.tex\\
%   hologo.dtx & source/latex/oberdiek/hologo.dtx\\
% \end{tabular}^^A
% }^^A
% \sbox0{\t}^^A
% \ifdim\wd0>\linewidth
%   \begingroup
%     \advance\linewidth by\leftmargin
%     \advance\linewidth by\rightmargin
%   \edef\x{\endgroup
%     \def\noexpand\lw{\the\linewidth}^^A
%   }\x
%   \def\lwbox{^^A
%     \leavevmode
%     \hbox to \linewidth{^^A
%       \kern-\leftmargin\relax
%       \hss
%       \usebox0
%       \hss
%       \kern-\rightmargin\relax
%     }^^A
%   }^^A
%   \ifdim\wd0>\lw
%     \sbox0{\small\t}^^A
%     \ifdim\wd0>\linewidth
%       \ifdim\wd0>\lw
%         \sbox0{\footnotesize\t}^^A
%         \ifdim\wd0>\linewidth
%           \ifdim\wd0>\lw
%             \sbox0{\scriptsize\t}^^A
%             \ifdim\wd0>\linewidth
%               \ifdim\wd0>\lw
%                 \sbox0{\tiny\t}^^A
%                 \ifdim\wd0>\linewidth
%                   \lwbox
%                 \else
%                   \usebox0
%                 \fi
%               \else
%                 \lwbox
%               \fi
%             \else
%               \usebox0
%             \fi
%           \else
%             \lwbox
%           \fi
%         \else
%           \usebox0
%         \fi
%       \else
%         \lwbox
%       \fi
%     \else
%       \usebox0
%     \fi
%   \else
%     \lwbox
%   \fi
% \else
%   \usebox0
% \fi
% \end{quote}
% If you have a \xfile{docstrip.cfg} that configures and enables \docstrip's
% TDS installing feature, then some files can already be in the right
% place, see the documentation of \docstrip.
%
% \subsection{Refresh file name databases}
%
% If your \TeX~distribution
% (\teTeX, \mikTeX, \dots) relies on file name databases, you must refresh
% these. For example, \teTeX\ users run \verb|texhash| or
% \verb|mktexlsr|.
%
% \subsection{Some details for the interested}
%
% \paragraph{Attached source.}
%
% The PDF documentation on CTAN also includes the
% \xfile{.dtx} source file. It can be extracted by
% AcrobatReader 6 or higher. Another option is \textsf{pdftk},
% e.g. unpack the file into the current directory:
% \begin{quote}
%   \verb|pdftk hologo.pdf unpack_files output .|
% \end{quote}
%
% \paragraph{Unpacking with \LaTeX.}
% The \xfile{.dtx} chooses its action depending on the format:
% \begin{description}
% \item[\plainTeX:] Run \docstrip\ and extract the files.
% \item[\LaTeX:] Generate the documentation.
% \end{description}
% If you insist on using \LaTeX\ for \docstrip\ (really,
% \docstrip\ does not need \LaTeX), then inform the autodetect routine
% about your intention:
% \begin{quote}
%   \verb|latex \let\install=y\input{hologo.dtx}|
% \end{quote}
% Do not forget to quote the argument according to the demands
% of your shell.
%
% \paragraph{Generating the documentation.}
% You can use both the \xfile{.dtx} or the \xfile{.drv} to generate
% the documentation. The process can be configured by the
% configuration file \xfile{ltxdoc.cfg}. For instance, put this
% line into this file, if you want to have A4 as paper format:
% \begin{quote}
%   \verb|\PassOptionsToClass{a4paper}{article}|
% \end{quote}
% An example follows how to generate the
% documentation with pdf\LaTeX:
% \begin{quote}
%\begin{verbatim}
%pdflatex hologo.dtx
%makeindex -s gind.ist hologo.idx
%pdflatex hologo.dtx
%makeindex -s gind.ist hologo.idx
%pdflatex hologo.dtx
%\end{verbatim}
% \end{quote}
%
% \section{Catalogue}
%
% The following XML file can be used as source for the
% \href{http://mirror.ctan.org/help/Catalogue/catalogue.html}{\TeX\ Catalogue}.
% The elements \texttt{caption} and \texttt{description} are imported
% from the original XML file from the Catalogue.
% The name of the XML file in the Catalogue is \xfile{hologo.xml}.
%    \begin{macrocode}
%<*catalogue>
<?xml version='1.0' encoding='us-ascii'?>
<!DOCTYPE entry SYSTEM 'catalogue.dtd'>
<entry datestamp='$Date$' modifier='$Author$' id='hologo'>
  <name>hologo</name>
  <caption>A collection of logos with bookmark support.</caption>
  <authorref id='auth:oberdiek'/>
  <copyright owner='Heiko Oberdiek' year='2010-2012'/>
  <license type='lppl1.3'/>
  <version number='1.10'/>
  <description>
    The package defines a single command <tt>\hologo</tt>, whose
    argument is the usual case-confused ASCII version of the logo.
    The command is bookmark-enabled, so that every logo becomes
    available in bookmarks without further work.
    <p/>
    The package is part of the <xref refid='oberdiek'>oberdiek</xref>
    bundle.
  </description>
  <documentation details='Package documentation'
      href='ctan:/macros/latex/contrib/oberdiek/hologo.pdf'/>
  <ctan file='true' path='/macros/latex/contrib/oberdiek/hologo.dtx'/>
  <miktex location='oberdiek'/>
  <texlive location='oberdiek'/>
  <install path='/macros/latex/contrib/oberdiek/oberdiek.tds.zip'/>
</entry>
%</catalogue>
%    \end{macrocode}
%
% \begin{thebibliography}{9}
% \raggedright
%
% \bibitem{btxdoc}
% Oren Patashnik,
% \textit{\hologo{BibTeX}ing},
% 1988-02-08.\\
% \CTAN{biblio/bibtex/base/}
%
% \bibitem{dtklogos}
% Gerd Neugebauer, DANTE,
% \textit{Package \xpackage{dtklogos}},
% 2011-04-25.\\
% \CTAN{usergrps/dante/dtk/dtklogos.sty}
%
% \bibitem{etexman}
% The \hologo{NTS} Team,
% \textit{The \hologo{eTeX} manual},
% 1998-02.\\
% \CTAN{systems/e-tex/v2/doc/}
%
% \bibitem{ExTeX-FAQ}
% The \hologo{ExTeX} group,
% \textit{\hologo{ExTeX}: FAQ -- How is \hologo{ExTeX} typeset?},
% 2007-04-14.\\
% \url{http://www.extex.org/documentation/faq.html}
%
% \bibitem{LyX}
% %@MISC{ LyX,
% %  title = {{LyX 2.0.0 -- The Document Processor [Computer software and manual]}},
% %  author = {{The LyX Team}},
% %  howpublished = {Internet: http://www.lyx.org},
% %  year = {2011-05-08},
% %  note = {Retrieved May 10, 2011, from http://www.lyx.org},
% %  url = {http://www.lyx.org/}
% %}
% The \hologo{LyX} Team,
% \textit{\hologo{LyX} -- The Document Processor},
% 2011-05-08.\\
% \url{http://www.lyx.org/}
%
% \bibitem{OzTeX}
% Andrew Trevorrow,
% \hologo{OzTeX} FAQ: What is the correct way to typeset ``\hologo{OzTeX}''?,
% 2011-09-15 (visited).
% \url{http://www.trevorrow.com/oztex/ozfaq.html#oztex-logo}
%
% \bibitem{PiCTeX}
% Michael Wichura,
% \textit{The \hologo{PiCTeX} macro package},
% 1987-09-21.
% \CTAN{graphics/pictex/}
%
% \bibitem{scrlogo}
% Markus Kohm,
% \textit{\hologo{KOMAScript} Datei \xfile{scrlogo.dtx}},
% 2009-01-30.\\
% \CTAN{install/macros/latex/contrib/komascript.tds.zip}
%
% \end{thebibliography}
%
% \begin{History}
%   \begin{Version}{2010/04/08 v1.0}
%   \item
%     The first version.
%   \end{Version}
%   \begin{Version}{2010/04/16 v1.1}
%   \item
%     \cs{Hologo} added for support of logos at start of a sentence.
%   \item
%     \cs{hologoSetup} and \cs{hologoLogoSetup} added.
%   \item
%     Options \xoption{break}, \xoption{hyphenbreak}, \xoption{spacebreak}
%     added.
%   \item
%     Variant support added by option \xoption{variant}.
%   \end{Version}
%   \begin{Version}{2010/04/24 v1.2}
%   \item
%     \hologo{LaTeX3} added.
%   \item
%     \hologo{VTeX} added.
%   \end{Version}
%   \begin{Version}{2010/11/21 v1.3}
%   \item
%     \hologo{iniTeX}, \hologo{virTeX} added.
%   \end{Version}
%   \begin{Version}{2011/03/25 v1.4}
%   \item
%     \hologo{ConTeXt} with variants added.
%   \item
%     Option \xoption{discretionarybreak} added as refinement for
%     option \xoption{break}.
%   \end{Version}
%   \begin{Version}{2011/04/21 v1.5}
%   \item
%     Wrong TDS directory for test files fixed.
%   \end{Version}
%   \begin{Version}{2011/10/01 v1.6}
%   \item
%     Support for package \xpackage{tex4ht} added.
%   \item
%     Support for \cs{csname} added if \cs{ifincsname} is available.
%   \item
%     New logos:
%     \hologo{(La)TeX},
%     \hologo{biber},
%     \hologo{BibTeX} (\xoption{sc}, \xoption{sf}),
%     \hologo{emTeX},
%     \hologo{ExTeX},
%     \hologo{KOMAScript},
%     \hologo{La},
%     \hologo{LyX},
%     \hologo{MiKTeX},
%     \hologo{NTS},
%     \hologo{OzMF},
%     \hologo{OzMP},
%     \hologo{OzTeX},
%     \hologo{OzTtH},
%     \hologo{PCTeX},
%     \hologo{PiC},
%     \hologo{PiCTeX},
%     \hologo{METAFONT},
%     \hologo{MetaFun},
%     \hologo{METAPOST},
%     \hologo{MetaPost},
%     \hologo{SLiTeX} (\xoption{lift}, \xoption{narrow}, \xoption{simple}),
%     \hologo{SliTeX} (\xoption{narrow}, \xoption{simple}, \xoption{lift}),
%     \hologo{teTeX}.
%   \item
%     Fixes:
%     \hologo{iniTeX},
%     \hologo{pdfLaTeX},
%     \hologo{pdfTeX},
%     \hologo{virTeX}.
%   \item
%     \cs{hologoFontSetup} and \cs{hologoLogoFontSetup} added.
%   \item
%     \cs{hologoVariant} and \cs{HologoVariant} added.
%   \end{Version}
%   \begin{Version}{2011/11/22 v1.7}
%   \item
%     New logos:
%     \hologo{BibTeX8},
%     \hologo{LaTeXML},
%     \hologo{SageTeX},
%     \hologo{TeX4ht},
%     \hologo{TTH}.
%   \item
%     \hologo{Xe} and friends: Driver stuff fixed.
%   \item
%     \hologo{Xe} and friends: Support for italic added.
%   \item
%     \hologo{Xe} and friends: Package support for \xpackage{pgf}
%     and \xpackage{pstricks} added.
%   \end{Version}
%   \begin{Version}{2011/11/29 v1.8}
%   \item
%     New logos:
%     \hologo{HanTheThanh}.
%   \end{Version}
%   \begin{Version}{2011/12/21 v1.9}
%   \item
%     Patch for package \xpackage{ifxetex} added for the case that
%     \cs{newif} is undefined in \hologo{iniTeX}.
%   \item
%     Some fixes for \hologo{iniTeX}.
%   \end{Version}
%   \begin{Version}{2012/04/26 v1.10}
%   \item
%     Fix in bookmark version of logo ``\hologo{HanTheThanh}''.
%   \end{Version}
%   \begin{Version}{2016/05/12 v1.11}
%   \item
%     Update HOLOGO@IfCharExists (previously in texlive)
%   \item define pdfliteral in current luatex.
%   \end{Version}
% \end{History}
%
% \PrintIndex
%
% \Finale
\endinput
%
        \else
          \input hologo.cfg\relax
        \fi
      \fi
    }%
    \ltx@IfUndefined{newread}{%
      \chardef\HOLOGO@temp=15 %
      \def\HOLOGO@CheckRead{%
        \ifeof\HOLOGO@temp
          \HOLOGO@InputIfExists
        \else
          \ifcase\HOLOGO@temp
            \@PackageWarningNoLine{hologo}{%
              Configuration file ignored, because\MessageBreak
              a free read register could not be found%
            }%
          \else
            \begingroup
              \count\ltx@cclv=\HOLOGO@temp
              \advance\ltx@cclv by \ltx@minusone
              \edef\x{\endgroup
                \chardef\noexpand\HOLOGO@temp=\the\count\ltx@cclv
                \relax
              }%
            \x
          \fi
        \fi
      }%
    }{%
      \csname newread\endcsname\HOLOGO@temp
      \HOLOGO@InputIfExists
    }%
  }{%
    \edef\HOLOGO@temp{\pdf@filesize{hologo.cfg}}%
    \ifx\HOLOGO@temp\ltx@empty
    \else
      \ifnum\HOLOGO@temp>0 %
        \begingroup
          \def\x{LaTeX2e}%
        \expandafter\endgroup
        \ifx\fmtname\x
          % \iffalse meta-comment
%
% File: hologo.dtx
% Version: 2016/05/12 v1.11
% Info: A logo collection with bookmark support
%
% Copyright (C) 2010-2012 by
%    Heiko Oberdiek <heiko.oberdiek at googlemail.com>
%
% This work may be distributed and/or modified under the
% conditions of the LaTeX Project Public License, either
% version 1.3c of this license or (at your option) any later
% version. This version of this license is in
%    http://www.latex-project.org/lppl/lppl-1-3c.txt
% and the latest version of this license is in
%    http://www.latex-project.org/lppl.txt
% and version 1.3 or later is part of all distributions of
% LaTeX version 2005/12/01 or later.
%
% This work has the LPPL maintenance status "maintained".
%
% This Current Maintainer of this work is Heiko Oberdiek.
%
% The Base Interpreter refers to any `TeX-Format',
% because some files are installed in TDS:tex/generic//.
%
% This work consists of the main source file hologo.dtx
% and the derived files
%    hologo.sty, hologo.pdf, hologo.ins, hologo.drv, hologo-example.tex,
%    hologo-test1.tex, hologo-test-spacefactor.tex,
%    hologo-test-list.tex.
%
% Distribution:
%    CTAN:macros/latex/contrib/oberdiek/hologo.dtx
%    CTAN:macros/latex/contrib/oberdiek/hologo.pdf
%
% Unpacking:
%    (a) If hologo.ins is present:
%           tex hologo.ins
%    (b) Without hologo.ins:
%           tex hologo.dtx
%    (c) If you insist on using LaTeX
%           latex \let\install=y\input{hologo.dtx}
%        (quote the arguments according to the demands of your shell)
%
% Documentation:
%    (a) If hologo.drv is present:
%           latex hologo.drv
%    (b) Without hologo.drv:
%           latex hologo.dtx; ...
%    The class ltxdoc loads the configuration file ltxdoc.cfg
%    if available. Here you can specify further options, e.g.
%    use A4 as paper format:
%       \PassOptionsToClass{a4paper}{article}
%
%    Programm calls to get the documentation (example):
%       pdflatex hologo.dtx
%       makeindex -s gind.ist hologo.idx
%       pdflatex hologo.dtx
%       makeindex -s gind.ist hologo.idx
%       pdflatex hologo.dtx
%
% Installation:
%    TDS:tex/generic/oberdiek/hologo.sty
%    TDS:doc/latex/oberdiek/hologo.pdf
%    TDS:doc/latex/oberdiek/example/hologo-example.tex
%    TDS:doc/latex/oberdiek/test/hologo-test1.tex
%    TDS:doc/latex/oberdiek/test/hologo-test-spacefactor.tex
%    TDS:doc/latex/oberdiek/test/hologo-test-list.tex
%    TDS:source/latex/oberdiek/hologo.dtx
%
%<*ignore>
\begingroup
  \catcode123=1 %
  \catcode125=2 %
  \def\x{LaTeX2e}%
\expandafter\endgroup
\ifcase 0\ifx\install y1\fi\expandafter
         \ifx\csname processbatchFile\endcsname\relax\else1\fi
         \ifx\fmtname\x\else 1\fi\relax
\else\csname fi\endcsname
%</ignore>
%<*install>
\input docstrip.tex
\Msg{************************************************************************}
\Msg{* Installation}
\Msg{* Package: hologo 2016/05/12 v1.11 A logo collection with bookmark support (HO)}
\Msg{************************************************************************}

\keepsilent
\askforoverwritefalse

\let\MetaPrefix\relax
\preamble

This is a generated file.

Project: hologo
Version: 2016/05/12 v1.11

Copyright (C) 2010-2012 by
   Heiko Oberdiek <heiko.oberdiek at googlemail.com>

This work may be distributed and/or modified under the
conditions of the LaTeX Project Public License, either
version 1.3c of this license or (at your option) any later
version. This version of this license is in
   http://www.latex-project.org/lppl/lppl-1-3c.txt
and the latest version of this license is in
   http://www.latex-project.org/lppl.txt
and version 1.3 or later is part of all distributions of
LaTeX version 2005/12/01 or later.

This work has the LPPL maintenance status "maintained".

This Current Maintainer of this work is Heiko Oberdiek.

The Base Interpreter refers to any `TeX-Format',
because some files are installed in TDS:tex/generic//.

This work consists of the main source file hologo.dtx
and the derived files
   hologo.sty, hologo.pdf, hologo.ins, hologo.drv, hologo-example.tex,
   hologo-test1.tex, hologo-test-spacefactor.tex,
   hologo-test-list.tex.

\endpreamble
\let\MetaPrefix\DoubleperCent

\generate{%
  \file{hologo.ins}{\from{hologo.dtx}{install}}%
  \file{hologo.drv}{\from{hologo.dtx}{driver}}%
  \usedir{tex/generic/oberdiek}%
  \file{hologo.sty}{\from{hologo.dtx}{package}}%
  \usedir{doc/latex/oberdiek/example}%
  \file{hologo-example.tex}{\from{hologo.dtx}{example}}%
  \usedir{doc/latex/oberdiek/test}%
  \file{hologo-test1.tex}{\from{hologo.dtx}{test1}}%
  \file{hologo-test-spacefactor.tex}{\from{hologo.dtx}{test-spacefactor}}%
  \file{hologo-test-list.tex}{\from{hologo.dtx}{test-list}}%
  \nopreamble
  \nopostamble
  \usedir{source/latex/oberdiek/catalogue}%
  \file{hologo.xml}{\from{hologo.dtx}{catalogue}}%
}

\catcode32=13\relax% active space
\let =\space%
\Msg{************************************************************************}
\Msg{*}
\Msg{* To finish the installation you have to move the following}
\Msg{* file into a directory searched by TeX:}
\Msg{*}
\Msg{*     hologo.sty}
\Msg{*}
\Msg{* To produce the documentation run the file `hologo.drv'}
\Msg{* through LaTeX.}
\Msg{*}
\Msg{* Happy TeXing!}
\Msg{*}
\Msg{************************************************************************}

\endbatchfile
%</install>
%<*ignore>
\fi
%</ignore>
%<*driver>
\NeedsTeXFormat{LaTeX2e}
\ProvidesFile{hologo.drv}%
  [2016/05/12 v1.11 A logo collection with bookmark support (HO)]%
\documentclass{ltxdoc}
\usepackage{holtxdoc}[2011/11/22]
\usepackage{hologo}[2016/05/12]
\usepackage{longtable}
\usepackage{array}
\usepackage{paralist}
%\usepackage[T1]{fontenc}
%\usepackage{lmodern}
\begin{document}
  \DocInput{hologo.dtx}%
\end{document}
%</driver>
% \fi
%
%
% \CharacterTable
%  {Upper-case    \A\B\C\D\E\F\G\H\I\J\K\L\M\N\O\P\Q\R\S\T\U\V\W\X\Y\Z
%   Lower-case    \a\b\c\d\e\f\g\h\i\j\k\l\m\n\o\p\q\r\s\t\u\v\w\x\y\z
%   Digits        \0\1\2\3\4\5\6\7\8\9
%   Exclamation   \!     Double quote  \"     Hash (number) \#
%   Dollar        \$     Percent       \%     Ampersand     \&
%   Acute accent  \'     Left paren    \(     Right paren   \)
%   Asterisk      \*     Plus          \+     Comma         \,
%   Minus         \-     Point         \.     Solidus       \/
%   Colon         \:     Semicolon     \;     Less than     \<
%   Equals        \=     Greater than  \>     Question mark \?
%   Commercial at \@     Left bracket  \[     Backslash     \\
%   Right bracket \]     Circumflex    \^     Underscore    \_
%   Grave accent  \`     Left brace    \{     Vertical bar  \|
%   Right brace   \}     Tilde         \~}
%
% \GetFileInfo{hologo.drv}
%
% \title{The \xpackage{hologo} package}
% \date{2016/05/12 v1.11}
% \author{Heiko Oberdiek\\\xemail{heiko.oberdiek at googlemail.com}}
%
% \maketitle
%
% \begin{abstract}
% This package starts a collection of logos with support for bookmarks
% strings.
% \end{abstract}
%
% \tableofcontents
%
% \section{Documentation}
%
% \subsection{Logo macros}
%
% \begin{declcs}{hologo} \M{name}
% \end{declcs}
% Macro \cs{hologo} sets the logo with name \meta{name}.
% The following table shows the supported names.
%
% \begingroup
%   \def\hologoEntry#1#2#3{^^A
%     #1&#2&\hologoLogoSetup{#1}{variant=#2}\hologo{#1}&#3\tabularnewline
%   }
%   \begin{longtable}{>{\ttfamily}l>{\ttfamily}lll}
%     \rmfamily\bfseries{name} & \rmfamily\bfseries variant
%     & \bfseries logo & \bfseries since\\
%     \hline
%     \endhead
%     \hologoList
%   \end{longtable}
% \endgroup
%
% \begin{declcs}{Hologo} \M{name}
% \end{declcs}
% Macro \cs{Hologo} starts the logo \meta{name} with an uppercase
% letter. As an exception small greek letters are not converted
% to uppercase. Examples, see \hologo{eTeX} and \hologo{ExTeX}.
%
% \subsection{Setup macros}
%
% The package does not support package options, but the following
% setup macros can be used to set options.
%
% \begin{declcs}{hologoSetup} \M{key value list}
% \end{declcs}
% Macro \cs{hologoSetup} sets global options.
%
% \begin{declcs}{hologoLogoSetup} \M{logo} \M{key value list}
% \end{declcs}
% Some options can also be used to configure a logo.
% These settings take precedence over global option settings.
%
% \subsection{Options}\label{sec:options}
%
% There are boolean and string options:
% \begin{description}
% \item[Boolean option:]
% It takes |true| or |false|
% as value. If the value is omitted, then |true| is used.
% \item[String option:]
% A value must be given as string. (But the string might be empty.)
% \end{description}
% The following options can be used both in \cs{hologoSetup}
% and \cs{hologoLogoSetup}:
% \begin{description}
% \def\entry#1{\item[\xoption{#1}:]}
% \entry{break}
%   enables or disables line breaks inside the logo. This setting is
%   refined by options \xoption{hyphenbreak}, \xoption{spacebreak}
%   or \xoption{discretionarybreak}.
%   Default is |false|.
% \entry{hyphenbreak}
%   enables or disables the line break right after the hyphen character.
% \entry{spacebreak}
%   enables or disables line breaks at space characters.
% \entry{discretionarybreak}
%   enables or disables line breaks at hyphenation points
%   (inserted by \cs{-}).
% \end{description}
% Macro \cs{hologoLogoSetup} also knows:
% \begin{description}
% \item[\xoption{variant}:]
%   This is a string option. It specifies a variant of a logo that
%   must exist. An empty string selects the package default variant.
% \end{description}
% Example:
% \begin{quote}
%   |\hologoSetup{break=false}|\\
%   |\hologoLogoSetup{plainTeX}{variant=hyphen,hyphenbreak}|\\
%   Then ``plain-\TeX'' contains one break point after the hyphen.
% \end{quote}
%
% \subsection{Driver options}
%
% Sometimes graphical operations are needed to construct some
% glyphs (e.g.\ \hologo{XeTeX}). If package \xpackage{graphics}
% or package \xpackage{pgf} are found, then the macros are taken
% from there. Otherwise the packge defines its own operations
% and therefore needs the driver information. Many drivers are
% detected automatically (\hologo{pdfTeX}/\hologo{LuaTeX}
% in PDF mode, \hologo{XeTeX}, \hologo{VTeX}). These have precedence
% over a driver option. The driver can be given as package option
% or using \cs{hologoDriverSetup}.
% The following list contains the recognized driver options:
% \begin{itemize}
% \item \xoption{pdftex}, \xoption{luatex}
% \item \xoption{dvipdfm}, \xoption{dvipdfmx}
% \item \xoption{dvips}, \xoption{dvipsone}, \xoption{xdvi}
% \item \xoption{xetex}
% \item \xoption{vtex}
% \end{itemize}
% The left driver of a line is the driver name that is used internally.
% The following names are aliases for drivers that use the
% same method. Therefore the entry in the \xext{log} file for
% the used driver prints the internally used driver name.
% \begin{description}
% \item[\xoption{driverfallback}:]
%   This option expects a driver that is used,
%   if the driver could not be detected automatically.
% \end{description}
%
% \begin{declcs}{hologoDriverSetup} \M{driver option}
% \end{declcs}
% The driver can also be configured after package loading
% using \cs{hologoDriverSetup}, also the way for \hologo{plainTeX}
% to setup the driver.
%
% \subsection{Font setup}
%
% Some logos require a special font, but should also be usable by
% \hologo{plainTeX}. Therefore the package provides some ways
% to influence the font settings. The options below
% take font settings as values. Both font commands
% such as \cs{sffamily} and macros that take one argument
% like \cs{textsf} can be used.
%
% \begin{declcs}{hologoFontSetup} \M{key value list}
% \end{declcs}
% Macro \cs{hologoFontSetup} sets the fonts for all logos.
% Supported keys:
% \begin{description}
% \def\entry#1{\item[\xoption{#1}:]}
% \entry{general}
%   This font is used for all logos. The default is empty.
%   That means no special font is used.
% \entry{bibsf}
%   This font is used for
%   {\hologoLogoSetup{BibTeX}{variant=sf}\hologo{BibTeX}}
%   with variant \xoption{sf}.
% \entry{rm}
%   This font is a serif font. It is used for \hologo{ExTeX}.
% \entry{sc}
%   This font specifies a small caps font. It is used for
%   {\hologoLogoSetup{BibTeX}{variant=sc}\hologo{BibTeX}}
%   with variant \xoption{sc}.
% \entry{sf}
%   This font specifies a sans serif font. The default
%   is \cs{sffamily}, then \cs{sf} is tried. Otherwise
%   a warning is given. It is used by \hologo{KOMAScript}.
% \entry{sy}
%   This is the font for math symbols (e.g. cmsy).
%   It is used by \hologo{AmS}, \hologo{NTS}, \hologo{ExTeX}.
% \entry{logo}
%   \hologo{METAFONT} and \hologo{METAPOST} are using that font.
%   In \hologo{LaTeX} \cs{logofamily} is used and
%   the definitions of package \xpackage{mflogo} are used
%   if the package is not loaded.
%   Otherwise the \cs{tenlogo} is used and defined
%   if it does not already exists.
% \end{description}
%
% \begin{declcs}{hologoLogoFontSetup} \M{logo} \M{key value list}
% \end{declcs}
% Fonts can also be set for a logo or logo component separately,
% see the following list.
% The keys are the same as for \cs{hologoFontSetup}.
%
% \begin{longtable}{>{\ttfamily}l>{\sffamily}ll}
%   \meta{logo} & keys & result\\
%   \hline
%   \endhead
%   BibTeX & bibsf & {\hologoLogoSetup{BibTeX}{variant=sf}\hologo{BibTeX}}\\[.5ex]
%   BibTeX & sc & {\hologoLogoSetup{BibTeX}{variant=sc}\hologo{BibTeX}}\\[.5ex]
%   ExTeX & rm & \hologo{ExTeX}\\
%   SliTeX & rm & \hologo{SliTeX}\\[.5ex]
%   AmS & sy & \hologo{AmS}\\
%   ExTeX & sy & \hologo{ExTeX}\\
%   NTS & sy & \hologo{NTS}\\[.5ex]
%   KOMAScript & sf & \hologo{KOMAScript}\\[.5ex]
%   METAFONT & logo & \hologo{METAFONT}\\
%   METAPOST & logo & \hologo{METAPOST}\\[.5ex]
%   SliTeX & sc \hologo{SliTeX}
% \end{longtable}
%
% \subsubsection{Font order}
%
% For all logos the font \xoption{general} is applied first.
% Example:
%\begin{quote}
%|\hologoFontSetup{general=\color{red}}|
%\end{quote}
% will print red logos.
% Then if the font uses a special font \xoption{sf}, for example,
% the font is applied that is setup by \cs{hologoLogoFontSetup}.
% If this font is not setup, then the common font setup
% by \cs{hologoFontSetup} is used. Otherwise a warning is given,
% that there is no font configured.
%
% \subsection{Additional user macros}
%
% Usually a variant of a logo is configured by using
% \cs{hologoLogoSetup}, because it is bad style to mix
% different variants of the same logo in the same text.
% There the following macros are a convenience for testing.
%
% \begin{declcs}{hologoVariant} \M{name} \M{variant}\\
%   \cs{HologoVariant} \M{name} \M{variant}
% \end{declcs}
% Logo \meta{name} is set using \meta{variant} that specifies
% explicitely which variant of the macro is used. If the argument
% is empty, then the default form of the logo is used
% (configurable by \cs{hologoLogoSetup}).
%
% \cs{HologoVariant} is used if the logo is set in a context
% that needs an uppercase first letter (beginning of a sentence, \dots).
%
% \begin{declcs}{hologoList}\\
%   \cs{hologoEntry} \M{logo} \M{variant} \M{since}
% \end{declcs}
% Macro \cs{hologoList} contains all logos that are provided
% by the package including variants. The list consists of calls
% of \cs{hologoEntry} with three arguments starting with the
% logo name \meta{logo} and its variant \meta{variant}. An empty
% variant means the current default. Argument \meta{since} specifies
% with version of the package \xpackage{hologo} is needed to get
% the logo. If the logo is fixed, then the date gets updated.
% Therefore the date \meta{since} is not exactly the date of
% the first introduction, but rather the date of the latest fix.
%
% Before \cs{hologoList} can be used, macro \cs{hologoEntry} needs
% a definition. The example file in section \ref{sec:example}
% shows applications of \cs{hologoList}.
%
% \subsection{Supported contexts}
%
% Macros \cs{hologo} and friends support special contexts:
% \begin{itemize}
% \item \hologo{LaTeX}'s protection mechanism.
% \item Bookmarks of package \xpackage{hyperref}.
% \item Package \xpackage{tex4ht}.
% \item The macros can be used inside \cs{csname} constructs,
%   if \cs{ifincsname} is available (\hologo{pdfTeX}, \hologo{XeTeX},
%   \hologo{LuaTeX}).
% \end{itemize}
%
% \subsection{Example}
% \label{sec:example}
%
% The following example prints the logos in different fonts.
%    \begin{macrocode}
%<*example>
%<<verbatim
\NeedsTeXFormat{LaTeX2e}
\documentclass[a4paper]{article}
\usepackage[
  hmargin=20mm,
  vmargin=20mm,
]{geometry}
\pagestyle{empty}
\usepackage{hologo}[2016/05/12]
\usepackage{longtable}
\usepackage{array}
\setlength{\extrarowheight}{2pt}
\usepackage[T1]{fontenc}
\usepackage{lmodern}
\usepackage{pdflscape}
\usepackage[
  pdfencoding=auto,
]{hyperref}
\hypersetup{
  pdfauthor={Heiko Oberdiek},
  pdftitle={Example for package `hologo'},
  pdfsubject={Logos with fonts lmr, lmss, qtm, qpl, qhv},
}
\usepackage{bookmark}

% Print the logo list on the console

\begingroup
  \typeout{}%
  \typeout{*** Begin of logo list ***}%
  \newcommand*{\hologoEntry}[3]{%
    \typeout{#1 \ifx\\#2\\\else(#2) \fi[#3]}%
  }%
  \hologoList
  \typeout{*** End of logo list ***}%
  \typeout{}%
\endgroup

\begin{document}
\begin{landscape}

  \section{Example file for package `hologo'}

  % Table for font names

  \begin{longtable}{>{\bfseries}ll}
    \textbf{font} & \textbf{Font name}\\
    \hline
    lmr & Latin Modern Roman\\
    lmss & Latin Modern Sans\\
    qtm & \TeX\ Gyre Termes\\
    qhv & \TeX\ Gyre Heros\\
    qpl & \TeX\ Gyre Pagella\\
  \end{longtable}

  % Logo list with logos in different fonts

  \begingroup
    \newcommand*{\SetVariant}[2]{%
      \ifx\\#2\\%
      \else
        \hologoLogoSetup{#1}{variant=#2}%
      \fi
    }%
    \newcommand*{\hologoEntry}[3]{%
      \SetVariant{#1}{#2}%
      \raisebox{1em}[0pt][0pt]{\hypertarget{#1@#2}{}}%
      \bookmark[%
        dest={#1@#2},%
      ]{%
        #1\ifx\\#2\\\else\space(#2)\fi: \Hologo{#1}, \hologo{#1} %
        [Unicode]%
      }%
      \hypersetup{unicode=false}%
      \bookmark[%
        dest={#1@#2},%
      ]{%
        #1\ifx\\#2\\\else\space(#2)\fi: \Hologo{#1}, \hologo{#1} %
        [PDFDocEncoding]%
      }%
      \texttt{#1}%
      &%
      \texttt{#2}%
      &%
      \Hologo{#1}%
      &%
      \SetVariant{#1}{#2}%
      \hologo{#1}%
      &%
      \SetVariant{#1}{#2}%
      \fontfamily{qtm}\selectfont
      \hologo{#1}%
      &%
      \SetVariant{#1}{#2}%
      \fontfamily{qpl}\selectfont
      \hologo{#1}%
      &%
      \SetVariant{#1}{#2}%
      \textsf{\hologo{#1}}%
      &%
      \SetVariant{#1}{#2}%
      \fontfamily{qhv}\selectfont
      \hologo{#1}%
      \tabularnewline
    }%
    \begin{longtable}{llllllll}%
      \textbf{\textit{logo}} & \textbf{\textit{variant}} &
      \texttt{\string\Hologo} &
      \textbf{lmr} & \textbf{qtm} & \textbf{qpl} &
      \textbf{lmss} & \textbf{qhv}
      \tabularnewline
      \hline
      \endhead
      \hologoList
    \end{longtable}%
  \endgroup

\end{landscape}
\end{document}
%verbatim
%</example>
%    \end{macrocode}
%
% \StopEventually{
% }
%
% \section{Implementation}
%    \begin{macrocode}
%<*package>
%    \end{macrocode}
%    Reload check, especially if the package is not used with \LaTeX.
%    \begin{macrocode}
\begingroup\catcode61\catcode48\catcode32=10\relax%
  \catcode13=5 % ^^M
  \endlinechar=13 %
  \catcode35=6 % #
  \catcode39=12 % '
  \catcode44=12 % ,
  \catcode45=12 % -
  \catcode46=12 % .
  \catcode58=12 % :
  \catcode64=11 % @
  \catcode123=1 % {
  \catcode125=2 % }
  \expandafter\let\expandafter\x\csname ver@hologo.sty\endcsname
  \ifx\x\relax % plain-TeX, first loading
  \else
    \def\empty{}%
    \ifx\x\empty % LaTeX, first loading,
      % variable is initialized, but \ProvidesPackage not yet seen
    \else
      \expandafter\ifx\csname PackageInfo\endcsname\relax
        \def\x#1#2{%
          \immediate\write-1{Package #1 Info: #2.}%
        }%
      \else
        \def\x#1#2{\PackageInfo{#1}{#2, stopped}}%
      \fi
      \x{hologo}{The package is already loaded}%
      \aftergroup\endinput
    \fi
  \fi
\endgroup%
%    \end{macrocode}
%    Package identification:
%    \begin{macrocode}
\begingroup\catcode61\catcode48\catcode32=10\relax%
  \catcode13=5 % ^^M
  \endlinechar=13 %
  \catcode35=6 % #
  \catcode39=12 % '
  \catcode40=12 % (
  \catcode41=12 % )
  \catcode44=12 % ,
  \catcode45=12 % -
  \catcode46=12 % .
  \catcode47=12 % /
  \catcode58=12 % :
  \catcode64=11 % @
  \catcode91=12 % [
  \catcode93=12 % ]
  \catcode123=1 % {
  \catcode125=2 % }
  \expandafter\ifx\csname ProvidesPackage\endcsname\relax
    \def\x#1#2#3[#4]{\endgroup
      \immediate\write-1{Package: #3 #4}%
      \xdef#1{#4}%
    }%
  \else
    \def\x#1#2[#3]{\endgroup
      #2[{#3}]%
      \ifx#1\@undefined
        \xdef#1{#3}%
      \fi
      \ifx#1\relax
        \xdef#1{#3}%
      \fi
    }%
  \fi
\expandafter\x\csname ver@hologo.sty\endcsname
\ProvidesPackage{hologo}%
  [2016/05/12 v1.11 A logo collection with bookmark support (HO)]%
%    \end{macrocode}
%
%    \begin{macrocode}
\begingroup\catcode61\catcode48\catcode32=10\relax%
  \catcode13=5 % ^^M
  \endlinechar=13 %
  \catcode123=1 % {
  \catcode125=2 % }
  \catcode64=11 % @
  \def\x{\endgroup
    \expandafter\edef\csname HOLOGO@AtEnd\endcsname{%
      \endlinechar=\the\endlinechar\relax
      \catcode13=\the\catcode13\relax
      \catcode32=\the\catcode32\relax
      \catcode35=\the\catcode35\relax
      \catcode61=\the\catcode61\relax
      \catcode64=\the\catcode64\relax
      \catcode123=\the\catcode123\relax
      \catcode125=\the\catcode125\relax
    }%
  }%
\x\catcode61\catcode48\catcode32=10\relax%
\catcode13=5 % ^^M
\endlinechar=13 %
\catcode35=6 % #
\catcode64=11 % @
\catcode123=1 % {
\catcode125=2 % }
\def\TMP@EnsureCode#1#2{%
  \edef\HOLOGO@AtEnd{%
    \HOLOGO@AtEnd
    \catcode#1=\the\catcode#1\relax
  }%
  \catcode#1=#2\relax
}
\TMP@EnsureCode{10}{12}% ^^J
\TMP@EnsureCode{33}{12}% !
\TMP@EnsureCode{34}{12}% "
\TMP@EnsureCode{36}{3}% $
\TMP@EnsureCode{38}{4}% &
\TMP@EnsureCode{39}{12}% '
\TMP@EnsureCode{40}{12}% (
\TMP@EnsureCode{41}{12}% )
\TMP@EnsureCode{42}{12}% *
\TMP@EnsureCode{43}{12}% +
\TMP@EnsureCode{44}{12}% ,
\TMP@EnsureCode{45}{12}% -
\TMP@EnsureCode{46}{12}% .
\TMP@EnsureCode{47}{12}% /
\TMP@EnsureCode{58}{12}% :
\TMP@EnsureCode{59}{12}% ;
\TMP@EnsureCode{60}{12}% <
\TMP@EnsureCode{62}{12}% >
\TMP@EnsureCode{63}{12}% ?
\TMP@EnsureCode{91}{12}% [
\TMP@EnsureCode{93}{12}% ]
\TMP@EnsureCode{94}{7}% ^ (superscript)
\TMP@EnsureCode{95}{8}% _ (subscript)
\TMP@EnsureCode{96}{12}% `
\TMP@EnsureCode{124}{12}% |
\edef\HOLOGO@AtEnd{%
  \HOLOGO@AtEnd
  \escapechar\the\escapechar\relax
  \noexpand\endinput
}
\escapechar=92 %
%    \end{macrocode}
%
% \subsection{Logo list}
%
%    \begin{macro}{\hologoList}
%    \begin{macrocode}
\def\hologoList{%
  \hologoEntry{(La)TeX}{}{2011/10/01}%
  \hologoEntry{AmSLaTeX}{}{2010/04/16}%
  \hologoEntry{AmSTeX}{}{2010/04/16}%
  \hologoEntry{biber}{}{2011/10/01}%
  \hologoEntry{BibTeX}{}{2011/10/01}%
  \hologoEntry{BibTeX}{sf}{2011/10/01}%
  \hologoEntry{BibTeX}{sc}{2011/10/01}%
  \hologoEntry{BibTeX8}{}{2011/11/22}%
  \hologoEntry{ConTeXt}{}{2011/03/25}%
  \hologoEntry{ConTeXt}{narrow}{2011/03/25}%
  \hologoEntry{ConTeXt}{simple}{2011/03/25}%
  \hologoEntry{emTeX}{}{2010/04/26}%
  \hologoEntry{eTeX}{}{2010/04/08}%
  \hologoEntry{ExTeX}{}{2011/10/01}%
  \hologoEntry{HanTheThanh}{}{2011/11/29}%
  \hologoEntry{iniTeX}{}{2011/10/01}%
  \hologoEntry{KOMAScript}{}{2011/10/01}%
  \hologoEntry{La}{}{2010/05/08}%
  \hologoEntry{LaTeX}{}{2010/04/08}%
  \hologoEntry{LaTeX2e}{}{2010/04/08}%
  \hologoEntry{LaTeX3}{}{2010/04/24}%
  \hologoEntry{LaTeXe}{}{2010/04/08}%
  \hologoEntry{LaTeXML}{}{2011/11/22}%
  \hologoEntry{LaTeXTeX}{}{2011/10/01}%
  \hologoEntry{LuaLaTeX}{}{2010/04/08}%
  \hologoEntry{LuaTeX}{}{2010/04/08}%
  \hologoEntry{LyX}{}{2011/10/01}%
  \hologoEntry{METAFONT}{}{2011/10/01}%
  \hologoEntry{MetaFun}{}{2011/10/01}%
  \hologoEntry{METAPOST}{}{2011/10/01}%
  \hologoEntry{MetaPost}{}{2011/10/01}%
  \hologoEntry{MiKTeX}{}{2011/10/01}%
  \hologoEntry{NTS}{}{2011/10/01}%
  \hologoEntry{OzMF}{}{2011/10/01}%
  \hologoEntry{OzMP}{}{2011/10/01}%
  \hologoEntry{OzTeX}{}{2011/10/01}%
  \hologoEntry{OzTtH}{}{2011/10/01}%
  \hologoEntry{PCTeX}{}{2011/10/01}%
  \hologoEntry{pdfTeX}{}{2011/10/01}%
  \hologoEntry{pdfLaTeX}{}{2011/10/01}%
  \hologoEntry{PiC}{}{2011/10/01}%
  \hologoEntry{PiCTeX}{}{2011/10/01}%
  \hologoEntry{plainTeX}{}{2010/04/08}%
  \hologoEntry{plainTeX}{space}{2010/04/16}%
  \hologoEntry{plainTeX}{hyphen}{2010/04/16}%
  \hologoEntry{plainTeX}{runtogether}{2010/04/16}%
  \hologoEntry{SageTeX}{}{2011/11/22}%
  \hologoEntry{SLiTeX}{}{2011/10/01}%
  \hologoEntry{SLiTeX}{lift}{2011/10/01}%
  \hologoEntry{SLiTeX}{narrow}{2011/10/01}%
  \hologoEntry{SLiTeX}{simple}{2011/10/01}%
  \hologoEntry{SliTeX}{}{2011/10/01}%
  \hologoEntry{SliTeX}{narrow}{2011/10/01}%
  \hologoEntry{SliTeX}{simple}{2011/10/01}%
  \hologoEntry{SliTeX}{lift}{2011/10/01}%
  \hologoEntry{teTeX}{}{2011/10/01}%
  \hologoEntry{TeX}{}{2010/04/08}%
  \hologoEntry{TeX4ht}{}{2011/11/22}%
  \hologoEntry{TTH}{}{2011/11/22}%
  \hologoEntry{virTeX}{}{2011/10/01}%
  \hologoEntry{VTeX}{}{2010/04/24}%
  \hologoEntry{Xe}{}{2010/04/08}%
  \hologoEntry{XeLaTeX}{}{2010/04/08}%
  \hologoEntry{XeTeX}{}{2010/04/08}%
}
%    \end{macrocode}
%    \end{macro}
%
% \subsection{Load resources}
%
%    \begin{macrocode}
\begingroup\expandafter\expandafter\expandafter\endgroup
\expandafter\ifx\csname RequirePackage\endcsname\relax
  \def\TMP@RequirePackage#1[#2]{%
    \begingroup\expandafter\expandafter\expandafter\endgroup
    \expandafter\ifx\csname ver@#1.sty\endcsname\relax
      \input #1.sty\relax
    \fi
  }%
  \TMP@RequirePackage{ltxcmds}[2011/02/04]%
  \TMP@RequirePackage{infwarerr}[2010/04/08]%
  \TMP@RequirePackage{kvsetkeys}[2010/03/01]%
  \TMP@RequirePackage{kvdefinekeys}[2010/03/01]%
  \TMP@RequirePackage{pdftexcmds}[2010/04/01]%
  \TMP@RequirePackage{ifpdf}[2010/01/28]%
  \TMP@RequirePackage{ifluatex}[2010/03/01]%
  \ltx@IfUndefined{newif}{%
    \expandafter\let\csname newif\endcsname\ltx@newif
  }{}%
  \TMP@RequirePackage{ifxetex}[2009/01/23]%
  \TMP@RequirePackage{ifvtex}[2010/03/01]%
\else
  \RequirePackage{ltxcmds}[2011/02/04]%
  \RequirePackage{infwarerr}[2010/04/08]%
  \RequirePackage{kvsetkeys}[2010/03/01]%
  \RequirePackage{kvdefinekeys}[2010/03/01]%
  \RequirePackage{pdftexcmds}[2010/04/01]%
  \RequirePackage{ifpdf}[2010/01/28]%
  \RequirePackage{ifluatex}[2010/03/01]%
  \RequirePackage{ifxetex}[2009/01/23]%
  \RequirePackage{ifvtex}[2010/03/01]%
\fi
%    \end{macrocode}
%
%    \begin{macro}{\HOLOGO@IfDefined}
%    \begin{macrocode}
\def\HOLOGO@IfExists#1{%
  \ifx\@undefined#1%
    \expandafter\ltx@secondoftwo
  \else
    \ifx\relax#1%
      \expandafter\ltx@secondoftwo
    \else
      \expandafter\expandafter\expandafter\ltx@firstoftwo
    \fi
  \fi
}
%    \end{macrocode}
%    \end{macro}
%
% \subsection{Setup macros}
%
%    \begin{macro}{\hologoSetup}
%    \begin{macrocode}
\def\hologoSetup{%
  \let\HOLOGO@name\relax
  \HOLOGO@Setup
}
%    \end{macrocode}
%    \end{macro}
%
%    \begin{macro}{\hologoLogoSetup}
%    \begin{macrocode}
\def\hologoLogoSetup#1{%
  \edef\HOLOGO@name{#1}%
  \ltx@IfUndefined{HoLogo@\HOLOGO@name}{%
    \@PackageError{hologo}{%
      Unknown logo `\HOLOGO@name'%
    }\@ehc
    \ltx@gobble
  }{%
    \HOLOGO@Setup
  }%
}
%    \end{macrocode}
%    \end{macro}
%
%    \begin{macro}{\HOLOGO@Setup}
%    \begin{macrocode}
\def\HOLOGO@Setup{%
  \kvsetkeys{HoLogo}%
}
%    \end{macrocode}
%    \end{macro}
%
% \subsection{Options}
%
%    \begin{macro}{\HOLOGO@DeclareBoolOption}
%    \begin{macrocode}
\def\HOLOGO@DeclareBoolOption#1{%
  \expandafter\chardef\csname HOLOGOOPT@#1\endcsname\ltx@zero
  \kv@define@key{HoLogo}{#1}[true]{%
    \def\HOLOGO@temp{##1}%
    \ifx\HOLOGO@temp\HOLOGO@true
      \ifx\HOLOGO@name\relax
        \expandafter\chardef\csname HOLOGOOPT@#1\endcsname=\ltx@one
      \else
        \expandafter\chardef\csname
        HoLogoOpt@#1@\HOLOGO@name\endcsname\ltx@one
      \fi
      \HOLOGO@SetBreakAll{#1}%
    \else
      \ifx\HOLOGO@temp\HOLOGO@false
        \ifx\HOLOGO@name\relax
          \expandafter\chardef\csname HOLOGOOPT@#1\endcsname=\ltx@zero
        \else
          \expandafter\chardef\csname
          HoLogoOpt@#1@\HOLOGO@name\endcsname=\ltx@zero
        \fi
        \HOLOGO@SetBreakAll{#1}%
      \else
        \@PackageError{hologo}{%
          Unknown value `##1' for boolean option `#1'.\MessageBreak
          Known values are `true' and `false'%
        }\@ehc
      \fi
    \fi
  }%
}
%    \end{macrocode}
%    \end{macro}
%
%    \begin{macro}{\HOLOGO@SetBreakAll}
%    \begin{macrocode}
\def\HOLOGO@SetBreakAll#1{%
  \def\HOLOGO@temp{#1}%
  \ifx\HOLOGO@temp\HOLOGO@break
    \ifx\HOLOGO@name\relax
      \chardef\HOLOGOOPT@hyphenbreak=\HOLOGOOPT@break
      \chardef\HOLOGOOPT@spacebreak=\HOLOGOOPT@break
      \chardef\HOLOGOOPT@discretionarybreak=\HOLOGOOPT@break
    \else
      \expandafter\chardef
         \csname HoLogoOpt@hyphenbreak@\HOLOGO@name\endcsname=%
         \csname HoLogoOpt@break@\HOLOGO@name\endcsname
      \expandafter\chardef
         \csname HoLogoOpt@spacebreak@\HOLOGO@name\endcsname=%
         \csname HoLogoOpt@break@\HOLOGO@name\endcsname
      \expandafter\chardef
         \csname HoLogoOpt@discretionarybreak@\HOLOGO@name
             \endcsname=%
         \csname HoLogoOpt@break@\HOLOGO@name\endcsname
    \fi
  \fi
}
%    \end{macrocode}
%    \end{macro}
%
%    \begin{macro}{\HOLOGO@true}
%    \begin{macrocode}
\def\HOLOGO@true{true}
%    \end{macrocode}
%    \end{macro}
%    \begin{macro}{\HOLOGO@false}
%    \begin{macrocode}
\def\HOLOGO@false{false}
%    \end{macrocode}
%    \end{macro}
%    \begin{macro}{\HOLOGO@break}
%    \begin{macrocode}
\def\HOLOGO@break{break}
%    \end{macrocode}
%    \end{macro}
%
%    \begin{macrocode}
\HOLOGO@DeclareBoolOption{break}
\HOLOGO@DeclareBoolOption{hyphenbreak}
\HOLOGO@DeclareBoolOption{spacebreak}
\HOLOGO@DeclareBoolOption{discretionarybreak}
%    \end{macrocode}
%
%    \begin{macrocode}
\kv@define@key{HoLogo}{variant}{%
  \ifx\HOLOGO@name\relax
    \@PackageError{hologo}{%
      Option `variant' is not available in \string\hologoSetup,%
      \MessageBreak
      Use \string\hologoLogoSetup\space instead%
    }\@ehc
  \else
    \edef\HOLOGO@temp{#1}%
    \ifx\HOLOGO@temp\ltx@empty
      \expandafter
      \let\csname HoLogoOpt@variant@\HOLOGO@name\endcsname\@undefined
    \else
      \ltx@IfUndefined{HoLogo@\HOLOGO@name @\HOLOGO@temp}{%
        \@PackageError{hologo}{%
          Unknown variant `\HOLOGO@temp' of logo `\HOLOGO@name'%
        }\@ehc
      }{%
        \expandafter
        \let\csname HoLogoOpt@variant@\HOLOGO@name\endcsname
            \HOLOGO@temp
      }%
    \fi
  \fi
}
%    \end{macrocode}
%
%    \begin{macro}{\HOLOGO@Variant}
%    \begin{macrocode}
\def\HOLOGO@Variant#1{%
  #1%
  \ltx@ifundefined{HoLogoOpt@variant@#1}{%
  }{%
    @\csname HoLogoOpt@variant@#1\endcsname
  }%
}
%    \end{macrocode}
%    \end{macro}
%
% \subsection{Break/no-break support}
%
%    \begin{macro}{\HOLOGO@space}
%    \begin{macrocode}
\def\HOLOGO@space{%
  \ltx@ifundefined{HoLogoOpt@spacebreak@\HOLOGO@name}{%
    \ltx@ifundefined{HoLogoOpt@break@\HOLOGO@name}{%
      \chardef\HOLOGO@temp=\HOLOGOOPT@spacebreak
    }{%
      \chardef\HOLOGO@temp=%
        \csname HoLogoOpt@break@\HOLOGO@name\endcsname
    }%
  }{%
    \chardef\HOLOGO@temp=%
      \csname HoLogoOpt@spacebreak@\HOLOGO@name\endcsname
  }%
  \ifcase\HOLOGO@temp
    \penalty10000 %
  \fi
  \ltx@space
}
%    \end{macrocode}
%    \end{macro}
%
%    \begin{macro}{\HOLOGO@hyphen}
%    \begin{macrocode}
\def\HOLOGO@hyphen{%
  \ltx@ifundefined{HoLogoOpt@hyphenbreak@\HOLOGO@name}{%
    \ltx@ifundefined{HoLogoOpt@break@\HOLOGO@name}{%
      \chardef\HOLOGO@temp=\HOLOGOOPT@hyphenbreak
    }{%
      \chardef\HOLOGO@temp=%
        \csname HoLogoOpt@break@\HOLOGO@name\endcsname
    }%
  }{%
    \chardef\HOLOGO@temp=%
      \csname HoLogoOpt@hyphenbreak@\HOLOGO@name\endcsname
  }%
  \ifcase\HOLOGO@temp
    \ltx@mbox{-}%
  \else
    -%
  \fi
}
%    \end{macrocode}
%    \end{macro}
%
%    \begin{macro}{\HOLOGO@discretionary}
%    \begin{macrocode}
\def\HOLOGO@discretionary{%
  \ltx@ifundefined{HoLogoOpt@discretionarybreak@\HOLOGO@name}{%
    \ltx@ifundefined{HoLogoOpt@break@\HOLOGO@name}{%
      \chardef\HOLOGO@temp=\HOLOGOOPT@discretionarybreak
    }{%
      \chardef\HOLOGO@temp=%
        \csname HoLogoOpt@break@\HOLOGO@name\endcsname
    }%
  }{%
    \chardef\HOLOGO@temp=%
      \csname HoLogoOpt@discretionarybreak@\HOLOGO@name\endcsname
  }%
  \ifcase\HOLOGO@temp
  \else
    \-%
  \fi
}
%    \end{macrocode}
%    \end{macro}
%
%    \begin{macro}{\HOLOGO@mbox}
%    \begin{macrocode}
\def\HOLOGO@mbox#1{%
  \ltx@ifundefined{HoLogoOpt@break@\HOLOGO@name}{%
    \chardef\HOLOGO@temp=\HOLOGOOPT@hyphenbreak
  }{%
    \chardef\HOLOGO@temp=%
      \csname HoLogoOpt@break@\HOLOGO@name\endcsname
  }%
  \ifcase\HOLOGO@temp
    \ltx@mbox{#1}%
  \else
    #1%
  \fi
}
%    \end{macrocode}
%    \end{macro}
%
% \subsection{Font support}
%
%    \begin{macro}{\HoLogoFont@font}
%    \begin{tabular}{@{}ll@{}}
%    |#1|:& logo name\\
%    |#2|:& font short name\\
%    |#3|:& text
%    \end{tabular}
%    \begin{macrocode}
\def\HoLogoFont@font#1#2#3{%
  \begingroup
    \ltx@IfUndefined{HoLogoFont@logo@#1.#2}{%
      \ltx@IfUndefined{HoLogoFont@font@#2}{%
        \@PackageWarning{hologo}{%
          Missing font `#2' for logo `#1'%
        }%
        #3%
      }{%
        \csname HoLogoFont@font@#2\endcsname{#3}%
      }%
    }{%
      \csname HoLogoFont@logo@#1.#2\endcsname{#3}%
    }%
  \endgroup
}
%    \end{macrocode}
%    \end{macro}
%
%    \begin{macro}{\HoLogoFont@Def}
%    \begin{macrocode}
\def\HoLogoFont@Def#1{%
  \expandafter\def\csname HoLogoFont@font@#1\endcsname
}
%    \end{macrocode}
%    \end{macro}
%    \begin{macro}{\HoLogoFont@LogoDef}
%    \begin{macrocode}
\def\HoLogoFont@LogoDef#1#2{%
  \expandafter\def\csname HoLogoFont@logo@#1.#2\endcsname
}
%    \end{macrocode}
%    \end{macro}
%
% \subsubsection{Font defaults}
%
%    \begin{macro}{\HoLogoFont@font@general}
%    \begin{macrocode}
\HoLogoFont@Def{general}{}%
%    \end{macrocode}
%    \end{macro}
%
%    \begin{macro}{\HoLogoFont@font@rm}
%    \begin{macrocode}
\ltx@IfUndefined{rmfamily}{%
  \ltx@IfUndefined{rm}{%
  }{%
    \HoLogoFont@Def{rm}{\rm}%
  }%
}{%
  \HoLogoFont@Def{rm}{\rmfamily}%
}
%    \end{macrocode}
%    \end{macro}
%
%    \begin{macro}{\HoLogoFont@font@sf}
%    \begin{macrocode}
\ltx@IfUndefined{sffamily}{%
  \ltx@IfUndefined{sf}{%
  }{%
    \HoLogoFont@Def{sf}{\sf}%
  }%
}{%
  \HoLogoFont@Def{sf}{\sffamily}%
}
%    \end{macrocode}
%    \end{macro}
%
%    \begin{macro}{\HoLogoFont@font@bibsf}
%    In case of \hologo{plainTeX} the original small caps
%    variant is used as default. In \hologo{LaTeX}
%    the definition of package \xpackage{dtklogos} \cite{dtklogos}
%    is used.
%\begin{quote}
%\begin{verbatim}
%\DeclareRobustCommand{\BibTeX}{%
%  B%
%  \kern-.05em%
%  \hbox{%
%    $\m@th$% %% force math size calculations
%    \csname S@\f@size\endcsname
%    \fontsize\sf@size\z@
%    \math@fontsfalse
%    \selectfont
%    I%
%    \kern-.025em%
%    B
%  }%
%  \kern-.08em%
%  \-%
%  \TeX
%}
%\end{verbatim}
%\end{quote}
%    \begin{macrocode}
\ltx@IfUndefined{selectfont}{%
  \ltx@IfUndefined{tensc}{%
    \font\tensc=cmcsc10\relax
  }{}%
  \HoLogoFont@Def{bibsf}{\tensc}%
}{%
  \HoLogoFont@Def{bibsf}{%
    $\mathsurround=0pt$%
    \csname S@\f@size\endcsname
    \fontsize\sf@size{0pt}%
    \math@fontsfalse
    \selectfont
  }%
}
%    \end{macrocode}
%    \end{macro}
%
%    \begin{macro}{\HoLogoFont@font@sc}
%    \begin{macrocode}
\ltx@IfUndefined{scshape}{%
  \ltx@IfUndefined{tensc}{%
    \font\tensc=cmcsc10\relax
  }{}%
  \HoLogoFont@Def{sc}{\tensc}%
}{%
  \HoLogoFont@Def{sc}{\scshape}%
}
%    \end{macrocode}
%    \end{macro}
%
%    \begin{macro}{\HoLogoFont@font@sy}
%    \begin{macrocode}
\ltx@IfUndefined{usefont}{%
  \ltx@IfUndefined{tensy}{%
  }{%
    \HoLogoFont@Def{sy}{\tensy}%
  }%
}{%
  \HoLogoFont@Def{sy}{%
    \usefont{OMS}{cmsy}{m}{n}%
  }%
}
%    \end{macrocode}
%    \end{macro}
%
%    \begin{macro}{\HoLogoFont@font@logo}
%    \begin{macrocode}
\begingroup
  \def\x{LaTeX2e}%
\expandafter\endgroup
\ifx\fmtname\x
  \ltx@IfUndefined{logofamily}{%
    \DeclareRobustCommand\logofamily{%
      \not@math@alphabet\logofamily\relax
      \fontencoding{U}%
      \fontfamily{logo}%
      \selectfont
    }%
  }{}%
  \ltx@IfUndefined{logofamily}{%
  }{%
    \HoLogoFont@Def{logo}{\logofamily}%
  }%
\else
  \ltx@IfUndefined{tenlogo}{%
    \font\tenlogo=logo10\relax
  }{}%
  \HoLogoFont@Def{logo}{\tenlogo}%
\fi
%    \end{macrocode}
%    \end{macro}
%
% \subsubsection{Font setup}
%
%    \begin{macro}{\hologoFontSetup}
%    \begin{macrocode}
\def\hologoFontSetup{%
  \let\HOLOGO@name\relax
  \HOLOGO@FontSetup
}
%    \end{macrocode}
%    \end{macro}
%
%    \begin{macro}{\hologoLogoFontSetup}
%    \begin{macrocode}
\def\hologoLogoFontSetup#1{%
  \edef\HOLOGO@name{#1}%
  \ltx@IfUndefined{HoLogo@\HOLOGO@name}{%
    \@PackageError{hologo}{%
      Unknown logo `\HOLOGO@name'%
    }\@ehc
    \ltx@gobble
  }{%
    \HOLOGO@FontSetup
  }%
}
%    \end{macrocode}
%    \end{macro}
%
%    \begin{macro}{\HOLOGO@FontSetup}
%    \begin{macrocode}
\def\HOLOGO@FontSetup{%
  \kvsetkeys{HoLogoFont}%
}
%    \end{macrocode}
%    \end{macro}
%
%    \begin{macrocode}
\def\HOLOGO@temp#1{%
  \kv@define@key{HoLogoFont}{#1}{%
    \ifx\HOLOGO@name\relax
      \HoLogoFont@Def{#1}{##1}%
    \else
      \HoLogoFont@LogoDef\HOLOGO@name{#1}{##1}%
    \fi
  }%
}
\HOLOGO@temp{general}
\HOLOGO@temp{sf}
%    \end{macrocode}
%
% \subsection{Generic logo commands}
%
%    \begin{macrocode}
\HOLOGO@IfExists\hologo{%
  \@PackageError{hologo}{%
    \string\hologo\ltx@space is already defined.\MessageBreak
    Package loading is aborted%
  }\@ehc
  \HOLOGO@AtEnd
}%
\HOLOGO@IfExists\hologoRobust{%
  \@PackageError{hologo}{%
    \string\hologoRobust\ltx@space is already defined.\MessageBreak
    Package loading is aborted%
  }\@ehc
  \HOLOGO@AtEnd
}%
%    \end{macrocode}
%
% \subsubsection{\cs{hologo} and friends}
%
%    \begin{macrocode}
\ifluatex
  \expandafter\ltx@firstofone
\else
  \expandafter\ltx@gobble
\fi
{%
  \ltx@IfUndefined{ifincsname}{%
    \ifnum\luatexversion<36 %
      \expandafter\ltx@gobble
    \else
      \expandafter\ltx@firstofone
    \fi
    {%
      \begingroup
        \ifcase0%
            \directlua{%
              if tex.enableprimitives then %
                tex.enableprimitives('HOLOGO@', {'ifincsname'})%
              else %
                tex.print('1')%
              end%
            }%
            \ifx\HOLOGO@ifincsname\@undefined 1\fi%
            \relax
          \expandafter\ltx@firstofone
        \else
          \endgroup
          \expandafter\ltx@gobble
        \fi
        {%
          \global\let\ifincsname\HOLOGO@ifincsname
        }%
      \HOLOGO@temp
    }%
  }{}%
}
%    \end{macrocode}
%    \begin{macrocode}
\ltx@IfUndefined{ifincsname}{%
  \catcode`$=14 %
}{%
  \catcode`$=9 %
}
%    \end{macrocode}
%
%    \begin{macro}{\hologo}
%    \begin{macrocode}
\def\hologo#1{%
$ \ifincsname
$   \ltx@ifundefined{HoLogoCs@\HOLOGO@Variant{#1}}{%
$     #1%
$   }{%
$     \csname HoLogoCs@\HOLOGO@Variant{#1}\endcsname\ltx@firstoftwo
$   }%
$ \else
    \HOLOGO@IfExists\texorpdfstring\texorpdfstring\ltx@firstoftwo
    {%
      \hologoRobust{#1}%
    }{%
      \ltx@ifundefined{HoLogoBkm@\HOLOGO@Variant{#1}}{%
        \ltx@ifundefined{HoLogo@#1}{?#1?}{#1}%
      }{%
        \csname HoLogoBkm@\HOLOGO@Variant{#1}\endcsname
        \ltx@firstoftwo
      }%
    }%
$ \fi
}
%    \end{macrocode}
%    \end{macro}
%    \begin{macro}{\Hologo}
%    \begin{macrocode}
\def\Hologo#1{%
$ \ifincsname
$   \ltx@ifundefined{HoLogoCs@\HOLOGO@Variant{#1}}{%
$     #1%
$   }{%
$     \csname HoLogoCs@\HOLOGO@Variant{#1}\endcsname\ltx@secondoftwo
$   }%
$ \else
    \HOLOGO@IfExists\texorpdfstring\texorpdfstring\ltx@firstoftwo
    {%
      \HologoRobust{#1}%
    }{%
      \ltx@ifundefined{HoLogoBkm@\HOLOGO@Variant{#1}}{%
        \ltx@ifundefined{HoLogo@#1}{?#1?}{#1}%
      }{%
        \csname HoLogoBkm@\HOLOGO@Variant{#1}\endcsname
        \ltx@secondoftwo
      }%
    }%
$ \fi
}
%    \end{macrocode}
%    \end{macro}
%
%    \begin{macro}{\hologoVariant}
%    \begin{macrocode}
\def\hologoVariant#1#2{%
  \ifx\relax#2\relax
    \hologo{#1}%
  \else
$   \ifincsname
$     \ltx@ifundefined{HoLogoCs@#1@#2}{%
$       #1%
$     }{%
$       \csname HoLogoCs@#1@#2\endcsname\ltx@firstoftwo
$     }%
$   \else
      \HOLOGO@IfExists\texorpdfstring\texorpdfstring\ltx@firstoftwo
      {%
        \hologoVariantRobust{#1}{#2}%
      }{%
        \ltx@ifundefined{HoLogoBkm@#1@#2}{%
          \ltx@ifundefined{HoLogo@#1}{?#1?}{#1}%
        }{%
          \csname HoLogoBkm@#1@#2\endcsname
          \ltx@firstoftwo
        }%
      }%
$   \fi
  \fi
}
%    \end{macrocode}
%    \end{macro}
%    \begin{macro}{\HologoVariant}
%    \begin{macrocode}
\def\HologoVariant#1#2{%
  \ifx\relax#2\relax
    \Hologo{#1}%
  \else
$   \ifincsname
$     \ltx@ifundefined{HoLogoCs@#1@#2}{%
$       #1%
$     }{%
$       \csname HoLogoCs@#1@#2\endcsname\ltx@secondoftwo
$     }%
$   \else
      \HOLOGO@IfExists\texorpdfstring\texorpdfstring\ltx@firstoftwo
      {%
        \HologoVariantRobust{#1}{#2}%
      }{%
        \ltx@ifundefined{HoLogoBkm@#1@#2}{%
          \ltx@ifundefined{HoLogo@#1}{?#1?}{#1}%
        }{%
          \csname HoLogoBkm@#1@#2\endcsname
          \ltx@secondoftwo
        }%
      }%
$   \fi
  \fi
}
%    \end{macrocode}
%    \end{macro}
%
%    \begin{macrocode}
\catcode`\$=3 %
%    \end{macrocode}
%
% \subsubsection{\cs{hologoRobust} and friends}
%
%    \begin{macro}{\hologoRobust}
%    \begin{macrocode}
\ltx@IfUndefined{protected}{%
  \ltx@IfUndefined{DeclareRobustCommand}{%
    \def\hologoRobust#1%
  }{%
    \DeclareRobustCommand*\hologoRobust[1]%
  }%
}{%
  \protected\def\hologoRobust#1%
}%
{%
  \edef\HOLOGO@name{#1}%
  \ltx@IfUndefined{HoLogo@\HOLOGO@Variant\HOLOGO@name}{%
    \@PackageError{hologo}{%
      Unknown logo `\HOLOGO@name'%
    }\@ehc
    ?\HOLOGO@name?%
  }{%
    \ltx@IfUndefined{ver@tex4ht.sty}{%
      \HoLogoFont@font\HOLOGO@name{general}{%
        \csname HoLogo@\HOLOGO@Variant\HOLOGO@name\endcsname
        \ltx@firstoftwo
      }%
    }{%
      \ltx@IfUndefined{HoLogoHtml@\HOLOGO@Variant\HOLOGO@name}{%
        \HOLOGO@name
      }{%
        \csname HoLogoHtml@\HOLOGO@Variant\HOLOGO@name\endcsname
        \ltx@firstoftwo
      }%
    }%
  }%
}
%    \end{macrocode}
%    \end{macro}
%    \begin{macro}{\HologoRobust}
%    \begin{macrocode}
\ltx@IfUndefined{protected}{%
  \ltx@IfUndefined{DeclareRobustCommand}{%
    \def\HologoRobust#1%
  }{%
    \DeclareRobustCommand*\HologoRobust[1]%
  }%
}{%
  \protected\def\HologoRobust#1%
}%
{%
  \edef\HOLOGO@name{#1}%
  \ltx@IfUndefined{HoLogo@\HOLOGO@Variant\HOLOGO@name}{%
    \@PackageError{hologo}{%
      Unknown logo `\HOLOGO@name'%
    }\@ehc
    ?\HOLOGO@name?%
  }{%
    \ltx@IfUndefined{ver@tex4ht.sty}{%
      \HoLogoFont@font\HOLOGO@name{general}{%
        \csname HoLogo@\HOLOGO@Variant\HOLOGO@name\endcsname
        \ltx@secondoftwo
      }%
    }{%
      \ltx@IfUndefined{HoLogoHtml@\HOLOGO@Variant\HOLOGO@name}{%
        \expandafter\HOLOGO@Uppercase\HOLOGO@name
      }{%
        \csname HoLogoHtml@\HOLOGO@Variant\HOLOGO@name\endcsname
        \ltx@secondoftwo
      }%
    }%
  }%
}
%    \end{macrocode}
%    \end{macro}
%    \begin{macro}{\hologoVariantRobust}
%    \begin{macrocode}
\ltx@IfUndefined{protected}{%
  \ltx@IfUndefined{DeclareRobustCommand}{%
    \def\hologoVariantRobust#1#2%
  }{%
    \DeclareRobustCommand*\hologoVariantRobust[2]%
  }%
}{%
  \protected\def\hologoVariantRobust#1#2%
}%
{%
  \begingroup
    \hologoLogoSetup{#1}{variant={#2}}%
    \hologoRobust{#1}%
  \endgroup
}
%    \end{macrocode}
%    \end{macro}
%    \begin{macro}{\HologoVariantRobust}
%    \begin{macrocode}
\ltx@IfUndefined{protected}{%
  \ltx@IfUndefined{DeclareRobustCommand}{%
    \def\HologoVariantRobust#1#2%
  }{%
    \DeclareRobustCommand*\HologoVariantRobust[2]%
  }%
}{%
  \protected\def\HologoVariantRobust#1#2%
}%
{%
  \begingroup
    \hologoLogoSetup{#1}{variant={#2}}%
    \HologoRobust{#1}%
  \endgroup
}
%    \end{macrocode}
%    \end{macro}
%
%    \begin{macro}{\hologorobust}
%    Macro \cs{hologorobust} is only defined for compatibility.
%    Its use is deprecated.
%    \begin{macrocode}
\def\hologorobust{\hologoRobust}
%    \end{macrocode}
%    \end{macro}
%
% \subsection{Helpers}
%
%    \begin{macro}{\HOLOGO@Uppercase}
%    Macro \cs{HOLOGO@Uppercase} is restricted to \cs{uppercase},
%    because \hologo{plainTeX} or \hologo{iniTeX} do not provide
%    \cs{MakeUppercase}.
%    \begin{macrocode}
\def\HOLOGO@Uppercase#1{\uppercase{#1}}
%    \end{macrocode}
%    \end{macro}
%
%    \begin{macro}{\HOLOGO@PdfdocUnicode}
%    \begin{macrocode}
\def\HOLOGO@PdfdocUnicode{%
  \ifx\ifHy@unicode\iftrue
    \expandafter\ltx@secondoftwo
  \else
    \expandafter\ltx@firstoftwo
  \fi
}
%    \end{macrocode}
%    \end{macro}
%
%    \begin{macro}{\HOLOGO@Math}
%    \begin{macrocode}
\def\HOLOGO@MathSetup{%
  \mathsurround0pt\relax
  \HOLOGO@IfExists\f@series{%
    \if b\expandafter\ltx@car\f@series x\@nil
      \csname boldmath\endcsname
   \fi
  }{}%
}
%    \end{macrocode}
%    \end{macro}
%
%    \begin{macro}{\HOLOGO@TempDimen}
%    \begin{macrocode}
\dimendef\HOLOGO@TempDimen=\ltx@zero
%    \end{macrocode}
%    \end{macro}
%    \begin{macro}{\HOLOGO@NegativeKerning}
%    \begin{macrocode}
\def\HOLOGO@NegativeKerning#1{%
  \begingroup
    \HOLOGO@TempDimen=0pt\relax
    \comma@parse@normalized{#1}{%
      \ifdim\HOLOGO@TempDimen=0pt %
        \expandafter\HOLOGO@@NegativeKerning\comma@entry
      \fi
      \ltx@gobble
    }%
    \ifdim\HOLOGO@TempDimen<0pt %
      \kern\HOLOGO@TempDimen
    \fi
  \endgroup
}
%    \end{macrocode}
%    \end{macro}
%    \begin{macro}{\HOLOGO@@NegativeKerning}
%    \begin{macrocode}
\def\HOLOGO@@NegativeKerning#1#2{%
  \setbox\ltx@zero\hbox{#1#2}%
  \HOLOGO@TempDimen=\wd\ltx@zero
  \setbox\ltx@zero\hbox{#1\kern0pt#2}%
  \advance\HOLOGO@TempDimen by -\wd\ltx@zero
}
%    \end{macrocode}
%    \end{macro}
%
%    \begin{macro}{\HOLOGO@SpaceFactor}
%    \begin{macrocode}
\def\HOLOGO@SpaceFactor{%
  \spacefactor1000 %
}
%    \end{macrocode}
%    \end{macro}
%
%    \begin{macro}{\HOLOGO@Span}
%    \begin{macrocode}
\def\HOLOGO@Span#1#2{%
  \HCode{<span class="HoLogo-#1">}%
  #2%
  \HCode{</span>}%
}
%    \end{macrocode}
%    \end{macro}
%
% \subsubsection{Text subscript}
%
%    \begin{macro}{\HOLOGO@SubScript}%
%    \begin{macrocode}
\def\HOLOGO@SubScript#1{%
  \ltx@IfUndefined{textsubscript}{%
    \ltx@IfUndefined{text}{%
      \ltx@mbox{%
        \mathsurround=0pt\relax
        $%
          _{%
            \ltx@IfUndefined{sf@size}{%
              \mathrm{#1}%
            }{%
              \mbox{%
                \fontsize\sf@size{0pt}\selectfont
                #1%
              }%
            }%
          }%
        $%
      }%
    }{%
      \ltx@mbox{%
        \mathsurround=0pt\relax
        $_{\text{#1}}$%
      }%
    }%
  }{%
    \textsubscript{#1}%
  }%
}
%    \end{macrocode}
%    \end{macro}
%
% \subsection{\hologo{TeX} and friends}
%
% \subsubsection{\hologo{TeX}}
%
%    \begin{macro}{\HoLogo@TeX}
%    Source: \hologo{LaTeX} kernel.
%    \begin{macrocode}
\def\HoLogo@TeX#1{%
  T\kern-.1667em\lower.5ex\hbox{E}\kern-.125emX\HOLOGO@SpaceFactor
}
%    \end{macrocode}
%    \end{macro}
%    \begin{macro}{\HoLogoHtml@TeX}
%    \begin{macrocode}
\def\HoLogoHtml@TeX#1{%
  \HoLogoCss@TeX
  \HOLOGO@Span{TeX}{%
    T%
    \HOLOGO@Span{e}{%
      E%
    }%
    X%
  }%
}
%    \end{macrocode}
%    \end{macro}
%    \begin{macro}{\HoLogoCss@TeX}
%    \begin{macrocode}
\def\HoLogoCss@TeX{%
  \Css{%
    span.HoLogo-TeX span.HoLogo-e{%
      position:relative;%
      top:.5ex;%
      margin-left:-.1667em;%
      margin-right:-.125em;%
    }%
  }%
  \Css{%
    a span.HoLogo-TeX span.HoLogo-e{%
      text-decoration:none;%
    }%
  }%
  \global\let\HoLogoCss@TeX\relax
}
%    \end{macrocode}
%    \end{macro}
%
% \subsubsection{\hologo{plainTeX}}
%
%    \begin{macro}{\HoLogo@plainTeX@space}
%    Source: ``The \hologo{TeX}book''
%    \begin{macrocode}
\def\HoLogo@plainTeX@space#1{%
  \HOLOGO@mbox{#1{p}{P}lain}\HOLOGO@space\hologo{TeX}%
}
%    \end{macrocode}
%    \end{macro}
%    \begin{macro}{\HoLogoCs@plainTeX@space}
%    \begin{macrocode}
\def\HoLogoCs@plainTeX@space#1{#1{p}{P}lain TeX}%
%    \end{macrocode}
%    \end{macro}
%    \begin{macro}{\HoLogoBkm@plainTeX@space}
%    \begin{macrocode}
\def\HoLogoBkm@plainTeX@space#1{%
  #1{p}{P}lain \hologo{TeX}%
}
%    \end{macrocode}
%    \end{macro}
%    \begin{macro}{\HoLogoHtml@plainTeX@space}
%    \begin{macrocode}
\def\HoLogoHtml@plainTeX@space#1{%
  #1{p}{P}lain \hologo{TeX}%
}
%    \end{macrocode}
%    \end{macro}
%
%    \begin{macro}{\HoLogo@plainTeX@hyphen}
%    \begin{macrocode}
\def\HoLogo@plainTeX@hyphen#1{%
  \HOLOGO@mbox{#1{p}{P}lain}\HOLOGO@hyphen\hologo{TeX}%
}
%    \end{macrocode}
%    \end{macro}
%    \begin{macro}{\HoLogoCs@plainTeX@hyphen}
%    \begin{macrocode}
\def\HoLogoCs@plainTeX@hyphen#1{#1{p}{P}lain-TeX}
%    \end{macrocode}
%    \end{macro}
%    \begin{macro}{\HoLogoBkm@plainTeX@hyphen}
%    \begin{macrocode}
\def\HoLogoBkm@plainTeX@hyphen#1{%
  #1{p}{P}lain-\hologo{TeX}%
}
%    \end{macrocode}
%    \end{macro}
%    \begin{macro}{\HoLogoHtml@plainTeX@hyphen}
%    \begin{macrocode}
\def\HoLogoHtml@plainTeX@hyphen#1{%
  #1{p}{P}lain-\hologo{TeX}%
}
%    \end{macrocode}
%    \end{macro}
%
%    \begin{macro}{\HoLogo@plainTeX@runtogether}
%    \begin{macrocode}
\def\HoLogo@plainTeX@runtogether#1{%
  \HOLOGO@mbox{#1{p}{P}lain\hologo{TeX}}%
}
%    \end{macrocode}
%    \end{macro}
%    \begin{macro}{\HoLogoCs@plainTeX@runtogether}
%    \begin{macrocode}
\def\HoLogoCs@plainTeX@runtogether#1{#1{p}{P}lainTeX}
%    \end{macrocode}
%    \end{macro}
%    \begin{macro}{\HoLogoBkm@plainTeX@runtogether}
%    \begin{macrocode}
\def\HoLogoBkm@plainTeX@runtogether#1{%
  #1{p}{P}lain\hologo{TeX}%
}
%    \end{macrocode}
%    \end{macro}
%    \begin{macro}{\HoLogoHtml@plainTeX@runtogether}
%    \begin{macrocode}
\def\HoLogoHtml@plainTeX@runtogether#1{%
  #1{p}{P}lain\hologo{TeX}%
}
%    \end{macrocode}
%    \end{macro}
%
%    \begin{macro}{\HoLogo@plainTeX}
%    \begin{macrocode}
\def\HoLogo@plainTeX{\HoLogo@plainTeX@space}
%    \end{macrocode}
%    \end{macro}
%    \begin{macro}{\HoLogoCs@plainTeX}
%    \begin{macrocode}
\def\HoLogoCs@plainTeX{\HoLogoCs@plainTeX@space}
%    \end{macrocode}
%    \end{macro}
%    \begin{macro}{\HoLogoBkm@plainTeX}
%    \begin{macrocode}
\def\HoLogoBkm@plainTeX{\HoLogoBkm@plainTeX@space}
%    \end{macrocode}
%    \end{macro}
%    \begin{macro}{\HoLogoHtml@plainTeX}
%    \begin{macrocode}
\def\HoLogoHtml@plainTeX{\HoLogoHtml@plainTeX@space}
%    \end{macrocode}
%    \end{macro}
%
% \subsubsection{\hologo{LaTeX}}
%
%    Source: \hologo{LaTeX} kernel.
%\begin{quote}
%\begin{verbatim}
%\DeclareRobustCommand{\LaTeX}{%
%  L%
%  \kern-.36em%
%  {%
%    \sbox\z@ T%
%    \vbox to\ht\z@{%
%      \hbox{%
%        \check@mathfonts
%        \fontsize\sf@size\z@
%        \math@fontsfalse
%        \selectfont
%        A%
%      }%
%      \vss
%    }%
%  }%
%  \kern-.15em%
%  \TeX
%}
%\end{verbatim}
%\end{quote}
%
%    \begin{macro}{\HoLogo@La}
%    \begin{macrocode}
\def\HoLogo@La#1{%
  L%
  \kern-.36em%
  \begingroup
    \setbox\ltx@zero\hbox{T}%
    \vbox to\ht\ltx@zero{%
      \hbox{%
        \ltx@ifundefined{check@mathfonts}{%
          \csname sevenrm\endcsname
        }{%
          \check@mathfonts
          \fontsize\sf@size{0pt}%
          \math@fontsfalse\selectfont
        }%
        A%
      }%
      \vss
    }%
  \endgroup
}
%    \end{macrocode}
%    \end{macro}
%
%    \begin{macro}{\HoLogo@LaTeX}
%    Source: \hologo{LaTeX} kernel.
%    \begin{macrocode}
\def\HoLogo@LaTeX#1{%
  \hologo{La}%
  \kern-.15em%
  \hologo{TeX}%
}
%    \end{macrocode}
%    \end{macro}
%    \begin{macro}{\HoLogoHtml@LaTeX}
%    \begin{macrocode}
\def\HoLogoHtml@LaTeX#1{%
  \HoLogoCss@LaTeX
  \HOLOGO@Span{LaTeX}{%
    L%
    \HOLOGO@Span{a}{%
      A%
    }%
    \hologo{TeX}%
  }%
}
%    \end{macrocode}
%    \end{macro}
%    \begin{macro}{\HoLogoCss@LaTeX}
%    \begin{macrocode}
\def\HoLogoCss@LaTeX{%
  \Css{%
    span.HoLogo-LaTeX span.HoLogo-a{%
      position:relative;%
      top:-.5ex;%
      margin-left:-.36em;%
      margin-right:-.15em;%
      font-size:85\%;%
    }%
  }%
  \global\let\HoLogoCss@LaTeX\relax
}
%    \end{macrocode}
%    \end{macro}
%
% \subsubsection{\hologo{(La)TeX}}
%
%    \begin{macro}{\HoLogo@LaTeXTeX}
%    The kerning around the parentheses is taken
%    from package \xpackage{dtklogos} \cite{dtklogos}.
%\begin{quote}
%\begin{verbatim}
%\DeclareRobustCommand{\LaTeXTeX}{%
%  (%
%  \kern-.15em%
%  L%
%  \kern-.36em%
%  {%
%    \sbox\z@ T%
%    \vbox to\ht0{%
%      \hbox{%
%        $\m@th$%
%        \csname S@\f@size\endcsname
%        \fontsize\sf@size\z@
%        \math@fontsfalse
%        \selectfont
%        A%
%      }%
%      \vss
%    }%
%  }%
%  \kern-.2em%
%  )%
%  \kern-.15em%
%  \TeX
%}
%\end{verbatim}
%\end{quote}
%    \begin{macrocode}
\def\HoLogo@LaTeXTeX#1{%
  (%
  \kern-.15em%
  \hologo{La}%
  \kern-.2em%
  )%
  \kern-.15em%
  \hologo{TeX}%
}
%    \end{macrocode}
%    \end{macro}
%    \begin{macro}{\HoLogoBkm@LaTeXTeX}
%    \begin{macrocode}
\def\HoLogoBkm@LaTeXTeX#1{(La)TeX}
%    \end{macrocode}
%    \end{macro}
%
%    \begin{macro}{\HoLogo@(La)TeX}
%    \begin{macrocode}
\expandafter
\let\csname HoLogo@(La)TeX\endcsname\HoLogo@LaTeXTeX
%    \end{macrocode}
%    \end{macro}
%    \begin{macro}{\HoLogoBkm@(La)TeX}
%    \begin{macrocode}
\expandafter
\let\csname HoLogoBkm@(La)TeX\endcsname\HoLogoBkm@LaTeXTeX
%    \end{macrocode}
%    \end{macro}
%    \begin{macro}{\HoLogoHtml@LaTeXTeX}
%    \begin{macrocode}
\def\HoLogoHtml@LaTeXTeX#1{%
  \HoLogoCss@LaTeXTeX
  \HOLOGO@Span{LaTeXTeX}{%
    (%
    \HOLOGO@Span{L}{L}%
    \HOLOGO@Span{a}{A}%
    \HOLOGO@Span{ParenRight}{)}%
    \hologo{TeX}%
  }%
}
%    \end{macrocode}
%    \end{macro}
%    \begin{macro}{\HoLogoHtml@(La)TeX}
%    Kerning after opening parentheses and before closing parentheses
%    is $-0.1$\,em. The original values $-0.15$\,em
%    looked too ugly for a serif font.
%    \begin{macrocode}
\expandafter
\let\csname HoLogoHtml@(La)TeX\endcsname\HoLogoHtml@LaTeXTeX
%    \end{macrocode}
%    \end{macro}
%    \begin{macro}{\HoLogoCss@LaTeXTeX}
%    \begin{macrocode}
\def\HoLogoCss@LaTeXTeX{%
  \Css{%
    span.HoLogo-LaTeXTeX span.HoLogo-L{%
      margin-left:-.1em;%
    }%
  }%
  \Css{%
    span.HoLogo-LaTeXTeX span.HoLogo-a{%
      position:relative;%
      top:-.5ex;%
      margin-left:-.36em;%
      margin-right:-.1em;%
      font-size:85\%;%
    }%
  }%
  \Css{%
    span.HoLogo-LaTeXTeX span.HoLogo-ParenRight{%
      margin-right:-.15em;%
    }%
  }%
  \global\let\HoLogoCss@LaTeXTeX\relax
}
%    \end{macrocode}
%    \end{macro}
%
% \subsubsection{\hologo{LaTeXe}}
%
%    \begin{macro}{\HoLogo@LaTeXe}
%    Source: \hologo{LaTeX} kernel
%    \begin{macrocode}
\def\HoLogo@LaTeXe#1{%
  \hologo{LaTeX}%
  \kern.15em%
  \hbox{%
    \HOLOGO@MathSetup
    2%
    $_{\textstyle\varepsilon}$%
  }%
}
%    \end{macrocode}
%    \end{macro}
%
%    \begin{macro}{\HoLogoCs@LaTeXe}
%    \begin{macrocode}
\ifnum64=`\^^^^0040\relax % test for big chars of LuaTeX/XeTeX
  \catcode`\$=9 %
  \catcode`\&=14 %
\else
  \catcode`\$=14 %
  \catcode`\&=9 %
\fi
\def\HoLogoCs@LaTeXe#1{%
  LaTeX2%
$ \string ^^^^0395%
& e%
}%
\catcode`\$=3 %
\catcode`\&=4 %
%    \end{macrocode}
%    \end{macro}
%
%    \begin{macro}{\HoLogoBkm@LaTeXe}
%    \begin{macrocode}
\def\HoLogoBkm@LaTeXe#1{%
  \hologo{LaTeX}%
  2%
  \HOLOGO@PdfdocUnicode{e}{\textepsilon}%
}
%    \end{macrocode}
%    \end{macro}
%
%    \begin{macro}{\HoLogoHtml@LaTeXe}
%    \begin{macrocode}
\def\HoLogoHtml@LaTeXe#1{%
  \HoLogoCss@LaTeXe
  \HOLOGO@Span{LaTeX2e}{%
    \hologo{LaTeX}%
    \HOLOGO@Span{2}{2}%
    \HOLOGO@Span{e}{%
      \HOLOGO@MathSetup
      \ensuremath{\textstyle\varepsilon}%
    }%
  }%
}
%    \end{macrocode}
%    \end{macro}
%    \begin{macro}{\HoLogoCss@LaTeXe}
%    \begin{macrocode}
\def\HoLogoCss@LaTeXe{%
  \Css{%
    span.HoLogo-LaTeX2e span.HoLogo-2{%
      padding-left:.15em;%
    }%
  }%
  \Css{%
    span.HoLogo-LaTeX2e span.HoLogo-e{%
      position:relative;%
      top:.35ex;%
      text-decoration:none;%
    }%
  }%
  \global\let\HoLogoCss@LaTeXe\relax
}
%    \end{macrocode}
%    \end{macro}
%
%    \begin{macro}{\HoLogo@LaTeX2e}
%    \begin{macrocode}
\expandafter
\let\csname HoLogo@LaTeX2e\endcsname\HoLogo@LaTeXe
%    \end{macrocode}
%    \end{macro}
%    \begin{macro}{\HoLogoCs@LaTeX2e}
%    \begin{macrocode}
\expandafter
\let\csname HoLogoCs@LaTeX2e\endcsname\HoLogoCs@LaTeXe
%    \end{macrocode}
%    \end{macro}
%    \begin{macro}{\HoLogoBkm@LaTeX2e}
%    \begin{macrocode}
\expandafter
\let\csname HoLogoBkm@LaTeX2e\endcsname\HoLogoBkm@LaTeXe
%    \end{macrocode}
%    \end{macro}
%    \begin{macro}{\HoLogoHtml@LaTeX2e}
%    \begin{macrocode}
\expandafter
\let\csname HoLogoHtml@LaTeX2e\endcsname\HoLogoHtml@LaTeXe
%    \end{macrocode}
%    \end{macro}
%
% \subsubsection{\hologo{LaTeX3}}
%
%    \begin{macro}{\HoLogo@LaTeX3}
%    Source: \hologo{LaTeX} kernel
%    \begin{macrocode}
\expandafter\def\csname HoLogo@LaTeX3\endcsname#1{%
  \hologo{LaTeX}%
  3%
}
%    \end{macrocode}
%    \end{macro}
%
%    \begin{macro}{\HoLogoBkm@LaTeX3}
%    \begin{macrocode}
\expandafter\def\csname HoLogoBkm@LaTeX3\endcsname#1{%
  \hologo{LaTeX}%
  3%
}
%    \end{macrocode}
%    \end{macro}
%    \begin{macro}{\HoLogoHtml@LaTeX3}
%    \begin{macrocode}
\expandafter
\let\csname HoLogoHtml@LaTeX3\expandafter\endcsname
\csname HoLogo@LaTeX3\endcsname
%    \end{macrocode}
%    \end{macro}
%
% \subsubsection{\hologo{LaTeXML}}
%
%    \begin{macro}{\HoLogo@LaTeXML}
%    \begin{macrocode}
\def\HoLogo@LaTeXML#1{%
  \HOLOGO@mbox{%
    \hologo{La}%
    \kern-.15em%
    T%
    \kern-.1667em%
    \lower.5ex\hbox{E}%
    \kern-.125em%
    \HoLogoFont@font{LaTeXML}{sc}{xml}%
  }%
}
%    \end{macrocode}
%    \end{macro}
%    \begin{macro}{\HoLogoHtml@pdfLaTeX}
%    \begin{macrocode}
\def\HoLogoHtml@LaTeXML#1{%
  \HOLOGO@Span{LaTeXML}{%
    \HoLogoCss@LaTeX
    \HoLogoCss@TeX
    \HOLOGO@Span{LaTeX}{%
      L%
      \HOLOGO@Span{a}{%
        A%
      }%
    }%
    \HOLOGO@Span{TeX}{%
      T%
      \HOLOGO@Span{e}{%
        E%
      }%
    }%
    \HCode{<span style="font-variant: small-caps;">}%
    xml%
    \HCode{</span>}%
  }%
}
%    \end{macrocode}
%    \end{macro}
%
% \subsubsection{\hologo{eTeX}}
%
%    \begin{macro}{\HoLogo@eTeX}
%    Source: package \xpackage{etex}
%    \begin{macrocode}
\def\HoLogo@eTeX#1{%
  \ltx@mbox{%
    \HOLOGO@MathSetup
    $\varepsilon$%
    -%
    \HOLOGO@NegativeKerning{-T,T-,To}%
    \hologo{TeX}%
  }%
}
%    \end{macrocode}
%    \end{macro}
%    \begin{macro}{\HoLogoCs@eTeX}
%    \begin{macrocode}
\ifnum64=`\^^^^0040\relax % test for big chars of LuaTeX/XeTeX
  \catcode`\$=9 %
  \catcode`\&=14 %
\else
  \catcode`\$=14 %
  \catcode`\&=9 %
\fi
\def\HoLogoCs@eTeX#1{%
$ #1{\string ^^^^0395}{\string ^^^^03b5}%
& #1{e}{E}%
  TeX%
}%
\catcode`\$=3 %
\catcode`\&=4 %
%    \end{macrocode}
%    \end{macro}
%    \begin{macro}{\HoLogoBkm@eTeX}
%    \begin{macrocode}
\def\HoLogoBkm@eTeX#1{%
  \HOLOGO@PdfdocUnicode{#1{e}{E}}{\textepsilon}%
  -%
  \hologo{TeX}%
}
%    \end{macrocode}
%    \end{macro}
%    \begin{macro}{\HoLogoHtml@eTeX}
%    \begin{macrocode}
\def\HoLogoHtml@eTeX#1{%
  \ltx@mbox{%
    \HOLOGO@MathSetup
    $\varepsilon$%
    -%
    \hologo{TeX}%
  }%
}
%    \end{macrocode}
%    \end{macro}
%
% \subsubsection{\hologo{iniTeX}}
%
%    \begin{macro}{\HoLogo@iniTeX}
%    \begin{macrocode}
\def\HoLogo@iniTeX#1{%
  \HOLOGO@mbox{%
    #1{i}{I}ni\hologo{TeX}%
  }%
}
%    \end{macrocode}
%    \end{macro}
%    \begin{macro}{\HoLogoCs@iniTeX}
%    \begin{macrocode}
\def\HoLogoCs@iniTeX#1{#1{i}{I}niTeX}
%    \end{macrocode}
%    \end{macro}
%    \begin{macro}{\HoLogoBkm@iniTeX}
%    \begin{macrocode}
\def\HoLogoBkm@iniTeX#1{%
  #1{i}{I}ni\hologo{TeX}%
}
%    \end{macrocode}
%    \end{macro}
%    \begin{macro}{\HoLogoHtml@iniTeX}
%    \begin{macrocode}
\let\HoLogoHtml@iniTeX\HoLogo@iniTeX
%    \end{macrocode}
%    \end{macro}
%
% \subsubsection{\hologo{virTeX}}
%
%    \begin{macro}{\HoLogo@virTeX}
%    \begin{macrocode}
\def\HoLogo@virTeX#1{%
  \HOLOGO@mbox{%
    #1{v}{V}ir\hologo{TeX}%
  }%
}
%    \end{macrocode}
%    \end{macro}
%    \begin{macro}{\HoLogoCs@virTeX}
%    \begin{macrocode}
\def\HoLogoCs@virTeX#1{#1{v}{V}irTeX}
%    \end{macrocode}
%    \end{macro}
%    \begin{macro}{\HoLogoBkm@virTeX}
%    \begin{macrocode}
\def\HoLogoBkm@virTeX#1{%
  #1{v}{V}ir\hologo{TeX}%
}
%    \end{macrocode}
%    \end{macro}
%    \begin{macro}{\HoLogoHtml@virTeX}
%    \begin{macrocode}
\let\HoLogoHtml@virTeX\HoLogo@virTeX
%    \end{macrocode}
%    \end{macro}
%
% \subsubsection{\hologo{SliTeX}}
%
% \paragraph{Definitions of the three variants.}
%
%    \begin{macro}{\HoLogo@SLiTeX@lift}
%    \begin{macrocode}
\def\HoLogo@SLiTeX@lift#1{%
  \HoLogoFont@font{SliTeX}{rm}{%
    S%
    \kern-.06em%
    L%
    \kern-.18em%
    \raise.32ex\hbox{\HoLogoFont@font{SliTeX}{sc}{i}}%
    \HOLOGO@discretionary
    \kern-.06em%
    \hologo{TeX}%
  }%
}
%    \end{macrocode}
%    \end{macro}
%    \begin{macro}{\HoLogoBkm@SLiTeX@lift}
%    \begin{macrocode}
\def\HoLogoBkm@SLiTeX@lift#1{SLiTeX}
%    \end{macrocode}
%    \end{macro}
%    \begin{macro}{\HoLogoHtml@SLiTeX@lift}
%    \begin{macrocode}
\def\HoLogoHtml@SLiTeX@lift#1{%
  \HoLogoCss@SLiTeX@lift
  \HOLOGO@Span{SLiTeX-lift}{%
    \HoLogoFont@font{SliTeX}{rm}{%
      S%
      \HOLOGO@Span{L}{L}%
      \HOLOGO@Span{i}{i}%
      \hologo{TeX}%
    }%
  }%
}
%    \end{macrocode}
%    \end{macro}
%    \begin{macro}{\HoLogoCss@SLiTeX@lift}
%    \begin{macrocode}
\def\HoLogoCss@SLiTeX@lift{%
  \Css{%
    span.HoLogo-SLiTeX-lift span.HoLogo-L{%
      margin-left:-.06em;%
      margin-right:-.18em;%
    }%
  }%
  \Css{%
    span.HoLogo-SLiTeX-lift span.HoLogo-i{%
      position:relative;%
      top:-.32ex;%
      margin-right:-.06em;%
      font-variant:small-caps;%
    }%
  }%
  \global\let\HoLogoCss@SLiTeX@lift\relax
}
%    \end{macrocode}
%    \end{macro}
%
%    \begin{macro}{\HoLogo@SliTeX@simple}
%    \begin{macrocode}
\def\HoLogo@SliTeX@simple#1{%
  \HoLogoFont@font{SliTeX}{rm}{%
    \ltx@mbox{%
      \HoLogoFont@font{SliTeX}{sc}{Sli}%
    }%
    \HOLOGO@discretionary
    \hologo{TeX}%
  }%
}
%    \end{macrocode}
%    \end{macro}
%    \begin{macro}{\HoLogoBkm@SliTeX@simple}
%    \begin{macrocode}
\def\HoLogoBkm@SliTeX@simple#1{SliTeX}
%    \end{macrocode}
%    \end{macro}
%    \begin{macro}{\HoLogoHtml@SliTeX@simple}
%    \begin{macrocode}
\let\HoLogoHtml@SliTeX@simple\HoLogo@SliTeX@simple
%    \end{macrocode}
%    \end{macro}
%
%    \begin{macro}{\HoLogo@SliTeX@narrow}
%    \begin{macrocode}
\def\HoLogo@SliTeX@narrow#1{%
  \HoLogoFont@font{SliTeX}{rm}{%
    \ltx@mbox{%
      S%
      \kern-.06em%
      \HoLogoFont@font{SliTeX}{sc}{%
        l%
        \kern-.035em%
        i%
      }%
    }%
    \HOLOGO@discretionary
    \kern-.06em%
    \hologo{TeX}%
  }%
}
%    \end{macrocode}
%    \end{macro}
%    \begin{macro}{\HoLogoBkm@SliTeX@narrow}
%    \begin{macrocode}
\def\HoLogoBkm@SliTeX@narrow#1{SliTeX}
%    \end{macrocode}
%    \end{macro}
%    \begin{macro}{\HoLogoHtml@SliTeX@narrow}
%    \begin{macrocode}
\def\HoLogoHtml@SliTeX@narrow#1{%
  \HoLogoCss@SliTeX@narrow
  \HOLOGO@Span{SliTeX-narrow}{%
    \HoLogoFont@font{SliTeX}{rm}{%
      S%
        \HOLOGO@Span{l}{l}%
        \HOLOGO@Span{i}{i}%
      \hologo{TeX}%
    }%
  }%
}
%    \end{macrocode}
%    \end{macro}
%    \begin{macro}{\HoLogoCss@SliTeX@narrow}
%    \begin{macrocode}
\def\HoLogoCss@SliTeX@narrow{%
  \Css{%
    span.HoLogo-SliTeX-narrow span.HoLogo-l{%
      margin-left:-.06em;%
      margin-right:-.035em;%
      font-variant:small-caps;%
    }%
  }%
  \Css{%
    span.HoLogo-SliTeX-narrow span.HoLogo-i{%
      margin-right:-.06em;%
      font-variant:small-caps;%
    }%
  }%
  \global\let\HoLogoCss@SliTeX@narrow\relax
}
%    \end{macrocode}
%    \end{macro}
%
% \paragraph{Macro set completion.}
%
%    \begin{macro}{\HoLogo@SLiTeX@simple}
%    \begin{macrocode}
\def\HoLogo@SLiTeX@simple{\HoLogo@SliTeX@simple}
%    \end{macrocode}
%    \end{macro}
%    \begin{macro}{\HoLogoBkm@SLiTeX@simple}
%    \begin{macrocode}
\def\HoLogoBkm@SLiTeX@simple{\HoLogoBkm@SliTeX@simple}
%    \end{macrocode}
%    \end{macro}
%    \begin{macro}{\HoLogoHtml@SLiTeX@simple}
%    \begin{macrocode}
\def\HoLogoHtml@SLiTeX@simple{\HoLogoHtml@SliTeX@simple}
%    \end{macrocode}
%    \end{macro}
%
%    \begin{macro}{\HoLogo@SLiTeX@narrow}
%    \begin{macrocode}
\def\HoLogo@SLiTeX@narrow{\HoLogo@SliTeX@narrow}
%    \end{macrocode}
%    \end{macro}
%    \begin{macro}{\HoLogoBkm@SLiTeX@narrow}
%    \begin{macrocode}
\def\HoLogoBkm@SLiTeX@narrow{\HoLogoBkm@SliTeX@narrow}
%    \end{macrocode}
%    \end{macro}
%    \begin{macro}{\HoLogoHtml@SLiTeX@narrow}
%    \begin{macrocode}
\def\HoLogoHtml@SLiTeX@narrow{\HoLogoHtml@SliTeX@narrow}
%    \end{macrocode}
%    \end{macro}
%
%    \begin{macro}{\HoLogo@SliTeX@lift}
%    \begin{macrocode}
\def\HoLogo@SliTeX@lift{\HoLogo@SLiTeX@lift}
%    \end{macrocode}
%    \end{macro}
%    \begin{macro}{\HoLogoBkm@SliTeX@lift}
%    \begin{macrocode}
\def\HoLogoBkm@SliTeX@lift{\HoLogoBkm@SLiTeX@lift}
%    \end{macrocode}
%    \end{macro}
%    \begin{macro}{\HoLogoHtml@SliTeX@lift}
%    \begin{macrocode}
\def\HoLogoHtml@SliTeX@lift{\HoLogoHtml@SLiTeX@lift}
%    \end{macrocode}
%    \end{macro}
%
% \paragraph{Defaults.}
%
%    \begin{macro}{\HoLogo@SLiTeX}
%    \begin{macrocode}
\def\HoLogo@SLiTeX{\HoLogo@SLiTeX@lift}
%    \end{macrocode}
%    \end{macro}
%    \begin{macro}{\HoLogoBkm@SLiTeX}
%    \begin{macrocode}
\def\HoLogoBkm@SLiTeX{\HoLogoBkm@SLiTeX@lift}
%    \end{macrocode}
%    \end{macro}
%    \begin{macro}{\HoLogoHtml@SLiTeX}
%    \begin{macrocode}
\def\HoLogoHtml@SLiTeX{\HoLogoHtml@SLiTeX@lift}
%    \end{macrocode}
%    \end{macro}
%
%    \begin{macro}{\HoLogo@SliTeX}
%    \begin{macrocode}
\def\HoLogo@SliTeX{\HoLogo@SliTeX@narrow}
%    \end{macrocode}
%    \end{macro}
%    \begin{macro}{\HoLogoBkm@SliTeX}
%    \begin{macrocode}
\def\HoLogoBkm@SliTeX{\HoLogoBkm@SliTeX@narrow}
%    \end{macrocode}
%    \end{macro}
%    \begin{macro}{\HoLogoHtml@SliTeX}
%    \begin{macrocode}
\def\HoLogoHtml@SliTeX{\HoLogoHtml@SliTeX@narrow}
%    \end{macrocode}
%    \end{macro}
%
% \subsubsection{\hologo{LuaTeX}}
%
%    \begin{macro}{\HoLogo@LuaTeX}
%    The kerning is an idea of Hans Hagen, see mailing list
%    `luatex at tug dot org' in March 2010.
%    \begin{macrocode}
\def\HoLogo@LuaTeX#1{%
  \HOLOGO@mbox{%
    Lua%
    \HOLOGO@NegativeKerning{aT,oT,To}%
    \hologo{TeX}%
  }%
}
%    \end{macrocode}
%    \end{macro}
%    \begin{macro}{\HoLogoHtml@LuaTeX}
%    \begin{macrocode}
\let\HoLogoHtml@LuaTeX\HoLogo@LuaTeX
%    \end{macrocode}
%    \end{macro}
%
% \subsubsection{\hologo{LuaLaTeX}}
%
%    \begin{macro}{\HoLogo@LuaLaTeX}
%    \begin{macrocode}
\def\HoLogo@LuaLaTeX#1{%
  \HOLOGO@mbox{%
    Lua%
    \hologo{LaTeX}%
  }%
}
%    \end{macrocode}
%    \end{macro}
%    \begin{macro}{\HoLogoHtml@LuaLaTeX}
%    \begin{macrocode}
\let\HoLogoHtml@LuaLaTeX\HoLogo@LuaLaTeX
%    \end{macrocode}
%    \end{macro}
%
% \subsubsection{\hologo{XeTeX}, \hologo{XeLaTeX}}
%
%    \begin{macro}{\HOLOGO@IfCharExists}
%    \begin{macrocode}
\ifluatex
  \ifnum\luatexversion<36 %
  \else
    \def\HOLOGO@IfCharExists#1{%
      \ifnum
        \directlua{%
           if luaotfload and luaotfload.aux then
             if luaotfload.aux.font_has_glyph(%
                    font.current(), \number#1) then % 	 
	       tex.print("1") % 	 
	     end % 	 
	   elseif font and font.fonts and font.current then %
            local f = font.fonts[font.current()]%
            if f.characters and f.characters[\number#1] then %
              tex.print("1")%
            end %
          end%
        }0=\ltx@zero
        \expandafter\ltx@secondoftwo
      \else
        \expandafter\ltx@firstoftwo
      \fi
    }%
  \fi
\fi
\ltx@IfUndefined{HOLOGO@IfCharExists}{%
  \def\HOLOGO@@IfCharExists#1{%
    \begingroup
      \tracinglostchars=\ltx@zero
      \setbox\ltx@zero=\hbox{%
        \kern7sp\char#1\relax
        \ifnum\lastkern>\ltx@zero
          \expandafter\aftergroup\csname iffalse\endcsname
        \else
          \expandafter\aftergroup\csname iftrue\endcsname
        \fi
      }%
      % \if{true|false} from \aftergroup
      \endgroup
      \expandafter\ltx@firstoftwo
    \else
      \endgroup
      \expandafter\ltx@secondoftwo
    \fi
  }%
  \ifxetex
    \ltx@IfUndefined{XeTeXfonttype}{}{%
      \ltx@IfUndefined{XeTeXcharglyph}{}{%
        \def\HOLOGO@IfCharExists#1{%
          \ifnum\XeTeXfonttype\font>\ltx@zero
            \expandafter\ltx@firstofthree
          \else
            \expandafter\ltx@gobble
          \fi
          {%
            \ifnum\XeTeXcharglyph#1>\ltx@zero
              \expandafter\ltx@firstoftwo
            \else
              \expandafter\ltx@secondoftwo
            \fi
          }%
          \HOLOGO@@IfCharExists{#1}%
        }%
      }%
    }%
  \fi
}{}
\ltx@ifundefined{HOLOGO@IfCharExists}{%
  \ifnum64=`\^^^^0040\relax % test for big chars of LuaTeX/XeTeX
    \let\HOLOGO@IfCharExists\HOLOGO@@IfCharExists
  \else
    \def\HOLOGO@IfCharExists#1{%
      \ifnum#1>255 %
        \expandafter\ltx@fourthoffour
      \fi
      \HOLOGO@@IfCharExists{#1}%
    }%
  \fi
}{}
%    \end{macrocode}
%    \end{macro}
%
%    \begin{macro}{\HoLogo@Xe}
%    Source: package \xpackage{dtklogos}
%    \begin{macrocode}
\def\HoLogo@Xe#1{%
  X%
  \kern-.1em\relax
  \HOLOGO@IfCharExists{"018E}{%
    \lower.5ex\hbox{\char"018E}%
  }{%
    \chardef\HOLOGO@choice=\ltx@zero
    \ifdim\fontdimen\ltx@one\font>0pt %
      \ltx@IfUndefined{rotatebox}{%
        \ltx@IfUndefined{pgftext}{%
          \ltx@IfUndefined{psscalebox}{%
            \ltx@IfUndefined{HOLOGO@ScaleBox@\hologoDriver}{%
            }{%
              \chardef\HOLOGO@choice=4 %
            }%
          }{%
            \chardef\HOLOGO@choice=3 %
          }%
        }{%
          \chardef\HOLOGO@choice=2 %
        }%
      }{%
        \chardef\HOLOGO@choice=1 %
      }%
      \ifcase\HOLOGO@choice
        \HOLOGO@WarningUnsupportedDriver{Xe}%
        e%
      \or % 1: \rotatebox
        \begingroup
          \setbox\ltx@zero\hbox{\rotatebox{180}{E}}%
          \ltx@LocDimenA=\dp\ltx@zero
          \advance\ltx@LocDimenA by -.5ex\relax
          \raise\ltx@LocDimenA\box\ltx@zero
        \endgroup
      \or % 2: \pgftext
        \lower.5ex\hbox{%
          \pgfpicture
            \pgftext[rotate=180]{E}%
          \endpgfpicture
        }%
      \or % 3: \psscalebox
        \begingroup
          \setbox\ltx@zero\hbox{\psscalebox{-1 -1}{E}}%
          \ltx@LocDimenA=\dp\ltx@zero
          \advance\ltx@LocDimenA by -.5ex\relax
          \raise\ltx@LocDimenA\box\ltx@zero
        \endgroup
      \or % 4: \HOLOGO@PointReflectBox
        \lower.5ex\hbox{\HOLOGO@PointReflectBox{E}}%
      \else
        \@PackageError{hologo}{Internal error (choice/it}\@ehc
      \fi
    \else
      \ltx@IfUndefined{reflectbox}{%
        \ltx@IfUndefined{pgftext}{%
          \ltx@IfUndefined{psscalebox}{%
            \ltx@IfUndefined{HOLOGO@ScaleBox@\hologoDriver}{%
            }{%
              \chardef\HOLOGO@choice=4 %
            }%
          }{%
            \chardef\HOLOGO@choice=3 %
          }%
        }{%
          \chardef\HOLOGO@choice=2 %
        }%
      }{%
        \chardef\HOLOGO@choice=1 %
      }%
      \ifcase\HOLOGO@choice
        \HOLOGO@WarningUnsupportedDriver{Xe}%
        e%
      \or % 1: reflectbox
        \lower.5ex\hbox{%
          \reflectbox{E}%
        }%
      \or % 2: \pgftext
        \lower.5ex\hbox{%
          \pgfpicture
            \pgftransformxscale{-1}%
            \pgftext{E}%
          \endpgfpicture
        }%
      \or % 3: \psscalebox
        \lower.5ex\hbox{%
          \psscalebox{-1 1}{E}%
        }%
      \or % 4: \HOLOGO@Reflectbox
        \lower.5ex\hbox{%
          \HOLOGO@ReflectBox{E}%
        }%
      \else
        \@PackageError{hologo}{Internal error (choice/up)}\@ehc
      \fi
    \fi
  }%
}
%    \end{macrocode}
%    \end{macro}
%    \begin{macro}{\HoLogoHtml@Xe}
%    \begin{macrocode}
\def\HoLogoHtml@Xe#1{%
  \HoLogoCss@Xe
  \HOLOGO@Span{Xe}{%
    X%
    \HOLOGO@Span{e}{%
      \HCode{&\ltx@hashchar x018e;}%
    }%
  }%
}
%    \end{macrocode}
%    \end{macro}
%    \begin{macro}{\HoLogoCss@Xe}
%    \begin{macrocode}
\def\HoLogoCss@Xe{%
  \Css{%
    span.HoLogo-Xe span.HoLogo-e{%
      position:relative;%
      top:.5ex;%
      left-margin:-.1em;%
    }%
  }%
  \global\let\HoLogoCss@Xe\relax
}
%    \end{macrocode}
%    \end{macro}
%
%    \begin{macro}{\HoLogo@XeTeX}
%    \begin{macrocode}
\def\HoLogo@XeTeX#1{%
  \hologo{Xe}%
  \kern-.15em\relax
  \hologo{TeX}%
}
%    \end{macrocode}
%    \end{macro}
%
%    \begin{macro}{\HoLogoHtml@XeTeX}
%    \begin{macrocode}
\def\HoLogoHtml@XeTeX#1{%
  \HoLogoCss@XeTeX
  \HOLOGO@Span{XeTeX}{%
    \hologo{Xe}%
    \hologo{TeX}%
  }%
}
%    \end{macrocode}
%    \end{macro}
%    \begin{macro}{\HoLogoCss@XeTeX}
%    \begin{macrocode}
\def\HoLogoCss@XeTeX{%
  \Css{%
    span.HoLogo-XeTeX span.HoLogo-TeX{%
      margin-left:-.15em;%
    }%
  }%
  \global\let\HoLogoCss@XeTeX\relax
}
%    \end{macrocode}
%    \end{macro}
%
%    \begin{macro}{\HoLogo@XeLaTeX}
%    \begin{macrocode}
\def\HoLogo@XeLaTeX#1{%
  \hologo{Xe}%
  \kern-.13em%
  \hologo{LaTeX}%
}
%    \end{macrocode}
%    \end{macro}
%    \begin{macro}{\HoLogoHtml@XeLaTeX}
%    \begin{macrocode}
\def\HoLogoHtml@XeLaTeX#1{%
  \HoLogoCss@XeLaTeX
  \HOLOGO@Span{XeLaTeX}{%
    \hologo{Xe}%
    \hologo{LaTeX}%
  }%
}
%    \end{macrocode}
%    \end{macro}
%    \begin{macro}{\HoLogoCss@XeLaTeX}
%    \begin{macrocode}
\def\HoLogoCss@XeLaTeX{%
  \Css{%
    span.HoLogo-XeLaTeX span.HoLogo-Xe{%
      margin-right:-.13em;%
    }%
  }%
  \global\let\HoLogoCss@XeLaTeX\relax
}
%    \end{macrocode}
%    \end{macro}
%
% \subsubsection{\hologo{pdfTeX}, \hologo{pdfLaTeX}}
%
%    \begin{macro}{\HoLogo@pdfTeX}
%    \begin{macrocode}
\def\HoLogo@pdfTeX#1{%
  \HOLOGO@mbox{%
    #1{p}{P}df\hologo{TeX}%
  }%
}
%    \end{macrocode}
%    \end{macro}
%    \begin{macro}{\HoLogoCs@pdfTeX}
%    \begin{macrocode}
\def\HoLogoCs@pdfTeX#1{#1{p}{P}dfTeX}
%    \end{macrocode}
%    \end{macro}
%    \begin{macro}{\HoLogoBkm@pdfTeX}
%    \begin{macrocode}
\def\HoLogoBkm@pdfTeX#1{%
  #1{p}{P}df\hologo{TeX}%
}
%    \end{macrocode}
%    \end{macro}
%    \begin{macro}{\HoLogoHtml@pdfTeX}
%    \begin{macrocode}
\let\HoLogoHtml@pdfTeX\HoLogo@pdfTeX
%    \end{macrocode}
%    \end{macro}
%
%    \begin{macro}{\HoLogo@pdfLaTeX}
%    \begin{macrocode}
\def\HoLogo@pdfLaTeX#1{%
  \HOLOGO@mbox{%
    #1{p}{P}df\hologo{LaTeX}%
  }%
}
%    \end{macrocode}
%    \end{macro}
%    \begin{macro}{\HoLogoCs@pdfLaTeX}
%    \begin{macrocode}
\def\HoLogoCs@pdfLaTeX#1{#1{p}{P}dfLaTeX}
%    \end{macrocode}
%    \end{macro}
%    \begin{macro}{\HoLogoBkm@pdfLaTeX}
%    \begin{macrocode}
\def\HoLogoBkm@pdfLaTeX#1{%
  #1{p}{P}df\hologo{LaTeX}%
}
%    \end{macrocode}
%    \end{macro}
%    \begin{macro}{\HoLogoHtml@pdfLaTeX}
%    \begin{macrocode}
\let\HoLogoHtml@pdfLaTeX\HoLogo@pdfLaTeX
%    \end{macrocode}
%    \end{macro}
%
% \subsubsection{\hologo{VTeX}}
%
%    \begin{macro}{\HoLogo@VTeX}
%    \begin{macrocode}
\def\HoLogo@VTeX#1{%
  \HOLOGO@mbox{%
    V\hologo{TeX}%
  }%
}
%    \end{macrocode}
%    \end{macro}
%    \begin{macro}{\HoLogoHtml@VTeX}
%    \begin{macrocode}
\let\HoLogoHtml@VTeX\HoLogo@VTeX
%    \end{macrocode}
%    \end{macro}
%
% \subsubsection{\hologo{AmS}, \dots}
%
%    Source: class \xclass{amsdtx}
%
%    \begin{macro}{\HoLogo@AmS}
%    \begin{macrocode}
\def\HoLogo@AmS#1{%
  \HoLogoFont@font{AmS}{sy}{%
    A%
    \kern-.1667em%
    \lower.5ex\hbox{M}%
    \kern-.125em%
    S%
  }%
}
%    \end{macrocode}
%    \end{macro}
%    \begin{macro}{\HoLogoBkm@AmS}
%    \begin{macrocode}
\def\HoLogoBkm@AmS#1{AmS}
%    \end{macrocode}
%    \end{macro}
%    \begin{macro}{\HoLogoHtml@AmS}
%    \begin{macrocode}
\def\HoLogoHtml@AmS#1{%
  \HoLogoCss@AmS
%  \HoLogoFont@font{AmS}{sy}{%
    \HOLOGO@Span{AmS}{%
      A%
      \HOLOGO@Span{M}{M}%
      S%
    }%
%   }%
}
%    \end{macrocode}
%    \end{macro}
%    \begin{macro}{\HoLogoCss@AmS}
%    \begin{macrocode}
\def\HoLogoCss@AmS{%
  \Css{%
    span.HoLogo-AmS span.HoLogo-M{%
      position:relative;%
      top:.5ex;%
      margin-left:-.1667em;%
      margin-right:-.125em;%
      text-decoration:none;%
    }%
  }%
  \global\let\HoLogoCss@AmS\relax
}
%    \end{macrocode}
%    \end{macro}
%
%    \begin{macro}{\HoLogo@AmSTeX}
%    \begin{macrocode}
\def\HoLogo@AmSTeX#1{%
  \hologo{AmS}%
  \HOLOGO@hyphen
  \hologo{TeX}%
}
%    \end{macrocode}
%    \end{macro}
%    \begin{macro}{\HoLogoBkm@AmSTeX}
%    \begin{macrocode}
\def\HoLogoBkm@AmSTeX#1{AmS-TeX}%
%    \end{macrocode}
%    \end{macro}
%    \begin{macro}{\HoLogoHtml@AmSTeX}
%    \begin{macrocode}
\let\HoLogoHtml@AmSTeX\HoLogo@AmSTeX
%    \end{macrocode}
%    \end{macro}
%
%    \begin{macro}{\HoLogo@AmSLaTeX}
%    \begin{macrocode}
\def\HoLogo@AmSLaTeX#1{%
  \hologo{AmS}%
  \HOLOGO@hyphen
  \hologo{LaTeX}%
}
%    \end{macrocode}
%    \end{macro}
%    \begin{macro}{\HoLogoBkm@AmSLaTeX}
%    \begin{macrocode}
\def\HoLogoBkm@AmSLaTeX#1{AmS-LaTeX}%
%    \end{macrocode}
%    \end{macro}
%    \begin{macro}{\HoLogoHtml@AmSLaTeX}
%    \begin{macrocode}
\let\HoLogoHtml@AmSLaTeX\HoLogo@AmSLaTeX
%    \end{macrocode}
%    \end{macro}
%
% \subsubsection{\hologo{BibTeX}}
%
%    \begin{macro}{\HoLogo@BibTeX@sc}
%    A definition of \hologo{BibTeX} is provided in
%    the documentation source for the manual of \hologo{BibTeX}
%    \cite{btxdoc}.
%\begin{quote}
%\begin{verbatim}
%\def\BibTeX{%
%  {%
%    \rm
%    B%
%    \kern-.05em%
%    {%
%      \sc
%      i%
%      \kern-.025em %
%      b%
%    }%
%    \kern-.08em
%    T%
%    \kern-.1667em%
%    \lower.7ex\hbox{E}%
%    \kern-.125em%
%    X%
%  }%
%}
%\end{verbatim}
%\end{quote}
%    \begin{macrocode}
\def\HoLogo@BibTeX@sc#1{%
  B%
  \kern-.05em%
  \HoLogoFont@font{BibTeX}{sc}{%
    i%
    \kern-.025em%
    b%
  }%
  \HOLOGO@discretionary
  \kern-.08em%
  \hologo{TeX}%
}
%    \end{macrocode}
%    \end{macro}
%    \begin{macro}{\HoLogoHtml@BibTeX@sc}
%    \begin{macrocode}
\def\HoLogoHtml@BibTeX@sc#1{%
  \HoLogoCss@BibTeX@sc
  \HOLOGO@Span{BibTeX-sc}{%
    B%
    \HOLOGO@Span{i}{i}%
    \HOLOGO@Span{b}{b}%
    \hologo{TeX}%
  }%
}
%    \end{macrocode}
%    \end{macro}
%    \begin{macro}{\HoLogoCss@BibTeX@sc}
%    \begin{macrocode}
\def\HoLogoCss@BibTeX@sc{%
  \Css{%
    span.HoLogo-BibTeX-sc span.HoLogo-i{%
      margin-left:-.05em;%
      margin-right:-.025em;%
      font-variant:small-caps;%
    }%
  }%
  \Css{%
    span.HoLogo-BibTeX-sc span.HoLogo-b{%
      margin-right:-.08em;%
      font-variant:small-caps;%
    }%
  }%
  \global\let\HoLogoCss@BibTeX@sc\relax
}
%    \end{macrocode}
%    \end{macro}
%
%    \begin{macro}{\HoLogo@BibTeX@sf}
%    Variant \xoption{sf} avoids trouble with unavailable
%    small caps fonts (e.g., bold versions of Computer Modern or
%    Latin Modern). The definition is taken from
%    package \xpackage{dtklogos} \cite{dtklogos}.
%\begin{quote}
%\begin{verbatim}
%\DeclareRobustCommand{\BibTeX}{%
%  B%
%  \kern-.05em%
%  \hbox{%
%    $\m@th$% %% force math size calculations
%    \csname S@\f@size\endcsname
%    \fontsize\sf@size\z@
%    \math@fontsfalse
%    \selectfont
%    I%
%    \kern-.025em%
%    B
%  }%
%  \kern-.08em%
%  \-%
%  \TeX
%}
%\end{verbatim}
%\end{quote}
%    \begin{macrocode}
\def\HoLogo@BibTeX@sf#1{%
  B%
  \kern-.05em%
  \HoLogoFont@font{BibTeX}{bibsf}{%
    I%
    \kern-.025em%
    B%
  }%
  \HOLOGO@discretionary
  \kern-.08em%
  \hologo{TeX}%
}
%    \end{macrocode}
%    \end{macro}
%    \begin{macro}{\HoLogoHtml@BibTeX@sf}
%    \begin{macrocode}
\def\HoLogoHtml@BibTeX@sf#1{%
  \HoLogoCss@BibTeX@sf
  \HOLOGO@Span{BibTeX-sf}{%
    B%
    \HoLogoFont@font{BibTeX}{bibsf}{%
      \HOLOGO@Span{i}{I}%
      B%
    }%
    \hologo{TeX}%
  }%
}
%    \end{macrocode}
%    \end{macro}
%    \begin{macro}{\HoLogoCss@BibTeX@sf}
%    \begin{macrocode}
\def\HoLogoCss@BibTeX@sf{%
  \Css{%
    span.HoLogo-BibTeX-sf span.HoLogo-i{%
      margin-left:-.05em;%
      margin-right:-.025em;%
    }%
  }%
  \Css{%
    span.HoLogo-BibTeX-sf span.HoLogo-TeX{%
      margin-left:-.08em;%
    }%
  }%
  \global\let\HoLogoCss@BibTeX@sf\relax
}
%    \end{macrocode}
%    \end{macro}
%
%    \begin{macro}{\HoLogo@BibTeX}
%    \begin{macrocode}
\def\HoLogo@BibTeX{\HoLogo@BibTeX@sf}
%    \end{macrocode}
%    \end{macro}
%    \begin{macro}{\HoLogoHtml@BibTeX}
%    \begin{macrocode}
\def\HoLogoHtml@BibTeX{\HoLogoHtml@BibTeX@sf}
%    \end{macrocode}
%    \end{macro}
%
% \subsubsection{\hologo{BibTeX8}}
%
%    \begin{macro}{\HoLogo@BibTeX8}
%    \begin{macrocode}
\expandafter\def\csname HoLogo@BibTeX8\endcsname#1{%
  \hologo{BibTeX}%
  8%
}
%    \end{macrocode}
%    \end{macro}
%
%    \begin{macro}{\HoLogoBkm@BibTeX8}
%    \begin{macrocode}
\expandafter\def\csname HoLogoBkm@BibTeX8\endcsname#1{%
  \hologo{BibTeX}%
  8%
}
%    \end{macrocode}
%    \end{macro}
%    \begin{macro}{\HoLogoHtml@BibTeX8}
%    \begin{macrocode}
\expandafter
\let\csname HoLogoHtml@BibTeX8\expandafter\endcsname
\csname HoLogo@BibTeX8\endcsname
%    \end{macrocode}
%    \end{macro}
%
% \subsubsection{\hologo{ConTeXt}}
%
%    \begin{macro}{\HoLogo@ConTeXt@simple}
%    \begin{macrocode}
\def\HoLogo@ConTeXt@simple#1{%
  \HOLOGO@mbox{Con}%
  \HOLOGO@discretionary
  \HOLOGO@mbox{\hologo{TeX}t}%
}
%    \end{macrocode}
%    \end{macro}
%    \begin{macro}{\HoLogoHtml@ConTeXt@simple}
%    \begin{macrocode}
\let\HoLogoHtml@ConTeXt@simple\HoLogo@ConTeXt@simple
%    \end{macrocode}
%    \end{macro}
%
%    \begin{macro}{\HoLogo@ConTeXt@narrow}
%    This definition of logo \hologo{ConTeXt} with variant \xoption{narrow}
%    comes from TUGboat's class \xclass{ltugboat} (version 2010/11/15 v2.8).
%    \begin{macrocode}
\def\HoLogo@ConTeXt@narrow#1{%
  \HOLOGO@mbox{C\kern-.0333emon}%
  \HOLOGO@discretionary
  \kern-.0667em%
  \HOLOGO@mbox{\hologo{TeX}\kern-.0333emt}%
}
%    \end{macrocode}
%    \end{macro}
%    \begin{macro}{\HoLogoHtml@ConTeXt@narrow}
%    \begin{macrocode}
\def\HoLogoHtml@ConTeXt@narrow#1{%
  \HoLogoCss@ConTeXt@narrow
  \HOLOGO@Span{ConTeXt-narrow}{%
    \HOLOGO@Span{C}{C}%
    on%
    \hologo{TeX}%
    t%
  }%
}
%    \end{macrocode}
%    \end{macro}
%    \begin{macro}{\HoLogoCss@ConTeXt@narrow}
%    \begin{macrocode}
\def\HoLogoCss@ConTeXt@narrow{%
  \Css{%
    span.HoLogo-ConTeXt-narrow span.HoLogo-C{%
      margin-left:-.0333em;%
    }%
  }%
  \Css{%
    span.HoLogo-ConTeXt-narrow span.HoLogo-TeX{%
      margin-left:-.0667em;%
      margin-right:-.0333em;%
    }%
  }%
  \global\let\HoLogoCss@ConTeXt@narrow\relax
}
%    \end{macrocode}
%    \end{macro}
%
%    \begin{macro}{\HoLogo@ConTeXt}
%    \begin{macrocode}
\def\HoLogo@ConTeXt{\HoLogo@ConTeXt@narrow}
%    \end{macrocode}
%    \end{macro}
%    \begin{macro}{\HoLogoHtml@ConTeXt}
%    \begin{macrocode}
\def\HoLogoHtml@ConTeXt{\HoLogoHtml@ConTeXt@narrow}
%    \end{macrocode}
%    \end{macro}
%
% \subsubsection{\hologo{emTeX}}
%
%    \begin{macro}{\HoLogo@emTeX}
%    \begin{macrocode}
\def\HoLogo@emTeX#1{%
  \HOLOGO@mbox{#1{e}{E}m}%
  \HOLOGO@discretionary
  \hologo{TeX}%
}
%    \end{macrocode}
%    \end{macro}
%    \begin{macro}{\HoLogoCs@emTeX}
%    \begin{macrocode}
\def\HoLogoCs@emTeX#1{#1{e}{E}mTeX}%
%    \end{macrocode}
%    \end{macro}
%    \begin{macro}{\HoLogoBkm@emTeX}
%    \begin{macrocode}
\def\HoLogoBkm@emTeX#1{%
  #1{e}{E}m\hologo{TeX}%
}
%    \end{macrocode}
%    \end{macro}
%    \begin{macro}{\HoLogoHtml@emTeX}
%    \begin{macrocode}
\let\HoLogoHtml@emTeX\HoLogo@emTeX
%    \end{macrocode}
%    \end{macro}
%
% \subsubsection{\hologo{ExTeX}}
%
%    \begin{macro}{\HoLogo@ExTeX}
%    The definition is taken from the FAQ of the
%    project \hologo{ExTeX}
%    \cite{ExTeX-FAQ}.
%\begin{quote}
%\begin{verbatim}
%\def\ExTeX{%
%  \textrm{% Logo always with serifs
%    \ensuremath{%
%      \textstyle
%      \varepsilon_{%
%        \kern-0.15em%
%        \mathcal{X}%
%      }%
%    }%
%    \kern-.15em%
%    \TeX
%  }%
%}
%\end{verbatim}
%\end{quote}
%    \begin{macrocode}
\def\HoLogo@ExTeX#1{%
  \HoLogoFont@font{ExTeX}{rm}{%
    \ltx@mbox{%
      \HOLOGO@MathSetup
      $%
        \textstyle
        \varepsilon_{%
          \kern-0.15em%
          \HoLogoFont@font{ExTeX}{sy}{X}%
        }%
      $%
    }%
    \HOLOGO@discretionary
    \kern-.15em%
    \hologo{TeX}%
  }%
}
%    \end{macrocode}
%    \end{macro}
%    \begin{macro}{\HoLogoHtml@ExTeX}
%    \begin{macrocode}
\def\HoLogoHtml@ExTeX#1{%
  \HoLogoCss@ExTeX
  \HoLogoFont@font{ExTeX}{rm}{%
    \HOLOGO@Span{ExTeX}{%
      \ltx@mbox{%
        \HOLOGO@MathSetup
        $\textstyle\varepsilon$%
        \HOLOGO@Span{X}{$\textstyle\chi$}%
        \hologo{TeX}%
      }%
    }%
  }%
}
%    \end{macrocode}
%    \end{macro}
%    \begin{macro}{\HoLogoBkm@ExTeX}
%    \begin{macrocode}
\def\HoLogoBkm@ExTeX#1{%
  \HOLOGO@PdfdocUnicode{#1{e}{E}x}{\textepsilon\textchi}%
  \hologo{TeX}%
}
%    \end{macrocode}
%    \end{macro}
%    \begin{macro}{\HoLogoCss@ExTeX}
%    \begin{macrocode}
\def\HoLogoCss@ExTeX{%
  \Css{%
    span.HoLogo-ExTeX{%
      font-family:serif;%
    }%
  }%
  \Css{%
    span.HoLogo-ExTeX span.HoLogo-TeX{%
      margin-left:-.15em;%
    }%
  }%
  \global\let\HoLogoCss@ExTeX\relax
}
%    \end{macrocode}
%    \end{macro}
%
% \subsubsection{\hologo{MiKTeX}}
%
%    \begin{macro}{\HoLogo@MiKTeX}
%    \begin{macrocode}
\def\HoLogo@MiKTeX#1{%
  \HOLOGO@mbox{MiK}%
  \HOLOGO@discretionary
  \hologo{TeX}%
}
%    \end{macrocode}
%    \end{macro}
%    \begin{macro}{\HoLogoHtml@MiKTeX}
%    \begin{macrocode}
\let\HoLogoHtml@MiKTeX\HoLogo@MiKTeX
%    \end{macrocode}
%    \end{macro}
%
% \subsubsection{\hologo{OzTeX} and friends}
%
%    Source: \hologo{OzTeX} FAQ \cite{OzTeX}:
%    \begin{quote}
%      |\def\OzTeX{O\kern-.03em z\kern-.15em\TeX}|\\
%      (There is no kerning in OzMF, OzMP and OzTtH.)
%    \end{quote}
%
%    \begin{macro}{\HoLogo@OzTeX}
%    \begin{macrocode}
\def\HoLogo@OzTeX#1{%
  O%
  \kern-.03em %
  z%
  \kern-.15em %
  \hologo{TeX}%
}
%    \end{macrocode}
%    \end{macro}
%    \begin{macro}{\HoLogoHtml@OzTeX}
%    \begin{macrocode}
\def\HoLogoHtml@OzTeX#1{%
  \HoLogoCss@OzTeX
  \HOLOGO@Span{OzTeX}{%
    O%
    \HOLOGO@Span{z}{z}%
    \hologo{TeX}%
  }%
}
%    \end{macrocode}
%    \end{macro}
%    \begin{macro}{\HoLogoCss@OzTeX}
%    \begin{macrocode}
\def\HoLogoCss@OzTeX{%
  \Css{%
    span.HoLogo-OzTeX span.HoLogo-z{%
      margin-left:-.03em;%
      margin-right:-.15em;%
    }%
  }%
  \global\let\HoLogoCss@OzTeX\relax
}
%    \end{macrocode}
%    \end{macro}
%
%    \begin{macro}{\HoLogo@OzMF}
%    \begin{macrocode}
\def\HoLogo@OzMF#1{%
  \HOLOGO@mbox{OzMF}%
}
%    \end{macrocode}
%    \end{macro}
%    \begin{macro}{\HoLogo@OzMP}
%    \begin{macrocode}
\def\HoLogo@OzMP#1{%
  \HOLOGO@mbox{OzMP}%
}
%    \end{macrocode}
%    \end{macro}
%    \begin{macro}{\HoLogo@OzTtH}
%    \begin{macrocode}
\def\HoLogo@OzTtH#1{%
  \HOLOGO@mbox{OzTtH}%
}
%    \end{macrocode}
%    \end{macro}
%
% \subsubsection{\hologo{PCTeX}}
%
%    \begin{macro}{\HoLogo@PCTeX}
%    \begin{macrocode}
\def\HoLogo@PCTeX#1{%
  \HOLOGO@mbox{PC}%
  \hologo{TeX}%
}
%    \end{macrocode}
%    \end{macro}
%    \begin{macro}{\HoLogoHtml@PCTeX}
%    \begin{macrocode}
\let\HoLogoHtml@PCTeX\HoLogo@PCTeX
%    \end{macrocode}
%    \end{macro}
%
% \subsubsection{\hologo{PiCTeX}}
%
%    The original definitions from \xfile{pictex.tex} \cite{PiCTeX}:
%\begin{quote}
%\begin{verbatim}
%\def\PiC{%
%  P%
%  \kern-.12em%
%  \lower.5ex\hbox{I}%
%  \kern-.075em%
%  C%
%}
%\def\PiCTeX{%
%  \PiC
%  \kern-.11em%
%  \TeX
%}
%\end{verbatim}
%\end{quote}
%
%    \begin{macro}{\HoLogo@PiC}
%    \begin{macrocode}
\def\HoLogo@PiC#1{%
  P%
  \kern-.12em%
  \lower.5ex\hbox{I}%
  \kern-.075em%
  C%
  \HOLOGO@SpaceFactor
}
%    \end{macrocode}
%    \end{macro}
%    \begin{macro}{\HoLogoHtml@PiC}
%    \begin{macrocode}
\def\HoLogoHtml@PiC#1{%
  \HoLogoCss@PiC
  \HOLOGO@Span{PiC}{%
    P%
    \HOLOGO@Span{i}{I}%
    C%
  }%
}
%    \end{macrocode}
%    \end{macro}
%    \begin{macro}{\HoLogoCss@PiC}
%    \begin{macrocode}
\def\HoLogoCss@PiC{%
  \Css{%
    span.HoLogo-PiC span.HoLogo-i{%
      position:relative;%
      top:.5ex;%
      margin-left:-.12em;%
      margin-right:-.075em;%
      text-decoration:none;%
    }%
  }%
  \global\let\HoLogoCss@PiC\relax
}
%    \end{macrocode}
%    \end{macro}
%
%    \begin{macro}{\HoLogo@PiCTeX}
%    \begin{macrocode}
\def\HoLogo@PiCTeX#1{%
  \hologo{PiC}%
  \HOLOGO@discretionary
  \kern-.11em%
  \hologo{TeX}%
}
%    \end{macrocode}
%    \end{macro}
%    \begin{macro}{\HoLogoHtml@PiCTeX}
%    \begin{macrocode}
\def\HoLogoHtml@PiCTeX#1{%
  \HoLogoCss@PiCTeX
  \HOLOGO@Span{PiCTeX}{%
    \hologo{PiC}%
    \hologo{TeX}%
  }%
}
%    \end{macrocode}
%    \end{macro}
%    \begin{macro}{\HoLogoCss@PiCTeX}
%    \begin{macrocode}
\def\HoLogoCss@PiCTeX{%
  \Css{%
    span.HoLogo-PiCTeX span.HoLogo-PiC{%
      margin-right:-.11em;%
    }%
  }%
  \global\let\HoLogoCss@PiCTeX\relax
}
%    \end{macrocode}
%    \end{macro}
%
% \subsubsection{\hologo{teTeX}}
%
%    \begin{macro}{\HoLogo@teTeX}
%    \begin{macrocode}
\def\HoLogo@teTeX#1{%
  \HOLOGO@mbox{#1{t}{T}e}%
  \HOLOGO@discretionary
  \hologo{TeX}%
}
%    \end{macrocode}
%    \end{macro}
%    \begin{macro}{\HoLogoCs@teTeX}
%    \begin{macrocode}
\def\HoLogoCs@teTeX#1{#1{t}{T}dfTeX}
%    \end{macrocode}
%    \end{macro}
%    \begin{macro}{\HoLogoBkm@teTeX}
%    \begin{macrocode}
\def\HoLogoBkm@teTeX#1{%
  #1{t}{T}e\hologo{TeX}%
}
%    \end{macrocode}
%    \end{macro}
%    \begin{macro}{\HoLogoHtml@teTeX}
%    \begin{macrocode}
\let\HoLogoHtml@teTeX\HoLogo@teTeX
%    \end{macrocode}
%    \end{macro}
%
% \subsubsection{\hologo{TeX4ht}}
%
%    \begin{macro}{\HoLogo@TeX4ht}
%    \begin{macrocode}
\expandafter\def\csname HoLogo@TeX4ht\endcsname#1{%
  \HOLOGO@mbox{\hologo{TeX}4ht}%
}
%    \end{macrocode}
%    \end{macro}
%    \begin{macro}{\HoLogoHtml@TeX4ht}
%    \begin{macrocode}
\expandafter
\let\csname HoLogoHtml@TeX4ht\expandafter\endcsname
\csname HoLogo@TeX4ht\endcsname
%    \end{macrocode}
%    \end{macro}
%
%
% \subsubsection{\hologo{SageTeX}}
%
%    \begin{macro}{\HoLogo@SageTeX}
%    \begin{macrocode}
\def\HoLogo@SageTeX#1{%
  \HOLOGO@mbox{Sage}%
  \HOLOGO@discretionary
  \HOLOGO@NegativeKerning{eT,oT,To}%
  \hologo{TeX}%
}
%    \end{macrocode}
%    \end{macro}
%    \begin{macro}{\HoLogoHtml@SageTeX}
%    \begin{macrocode}
\let\HoLogoHtml@SageTeX\HoLogo@SageTeX
%    \end{macrocode}
%    \end{macro}
%
% \subsection{\hologo{METAFONT} and friends}
%
%    \begin{macro}{\HoLogo@METAFONT}
%    \begin{macrocode}
\def\HoLogo@METAFONT#1{%
  \HoLogoFont@font{METAFONT}{logo}{%
    \HOLOGO@mbox{META}%
    \HOLOGO@discretionary
    \HOLOGO@mbox{FONT}%
  }%
}
%    \end{macrocode}
%    \end{macro}
%
%    \begin{macro}{\HoLogo@METAPOST}
%    \begin{macrocode}
\def\HoLogo@METAPOST#1{%
  \HoLogoFont@font{METAPOST}{logo}{%
    \HOLOGO@mbox{META}%
    \HOLOGO@discretionary
    \HOLOGO@mbox{POST}%
  }%
}
%    \end{macrocode}
%    \end{macro}
%
%    \begin{macro}{\HoLogo@MetaFun}
%    \begin{macrocode}
\def\HoLogo@MetaFun#1{%
  \HOLOGO@mbox{Meta}%
  \HOLOGO@discretionary
  \HOLOGO@mbox{Fun}%
}
%    \end{macrocode}
%    \end{macro}
%
%    \begin{macro}{\HoLogo@MetaPost}
%    \begin{macrocode}
\def\HoLogo@MetaPost#1{%
  \HOLOGO@mbox{Meta}%
  \HOLOGO@discretionary
  \HOLOGO@mbox{Post}%
}
%    \end{macrocode}
%    \end{macro}
%
% \subsection{Others}
%
% \subsubsection{\hologo{biber}}
%
%    \begin{macro}{\HoLogo@biber}
%    \begin{macrocode}
\def\HoLogo@biber#1{%
  \HOLOGO@mbox{#1{b}{B}i}%
  \HOLOGO@discretionary
  \HOLOGO@mbox{ber}%
}
%    \end{macrocode}
%    \end{macro}
%    \begin{macro}{\HoLogoCs@biber}
%    \begin{macrocode}
\def\HoLogoCs@biber#1{#1{b}{B}iber}
%    \end{macrocode}
%    \end{macro}
%    \begin{macro}{\HoLogoBkm@biber}
%    \begin{macrocode}
\def\HoLogoBkm@biber#1{%
  #1{b}{B}iber%
}
%    \end{macrocode}
%    \end{macro}
%    \begin{macro}{\HoLogoHtml@biber}
%    \begin{macrocode}
\let\HoLogoHtml@biber\HoLogo@biber
%    \end{macrocode}
%    \end{macro}
%
% \subsubsection{\hologo{KOMAScript}}
%
%    \begin{macro}{\HoLogo@KOMAScript}
%    The definition for \hologo{KOMAScript} is taken
%    from \hologo{KOMAScript} (\xfile{scrlogo.dtx}, reformatted) \cite{scrlogo}:
%\begin{quote}
%\begin{verbatim}
%\@ifundefined{KOMAScript}{%
%  \DeclareRobustCommand{\KOMAScript}{%
%    \textsf{%
%      K\kern.05em O\kern.05emM\kern.05em A%
%      \kern.1em-\kern.1em %
%      Script%
%    }%
%  }%
%}{}
%\end{verbatim}
%\end{quote}
%    \begin{macrocode}
\def\HoLogo@KOMAScript#1{%
  \HoLogoFont@font{KOMAScript}{sf}{%
    \HOLOGO@mbox{%
      K\kern.05em%
      O\kern.05em%
      M\kern.05em%
      A%
    }%
    \kern.1em%
    \HOLOGO@hyphen
    \kern.1em%
    \HOLOGO@mbox{Script}%
  }%
}
%    \end{macrocode}
%    \end{macro}
%    \begin{macro}{\HoLogoBkm@KOMAScript}
%    \begin{macrocode}
\def\HoLogoBkm@KOMAScript#1{%
  KOMA-Script%
}
%    \end{macrocode}
%    \end{macro}
%    \begin{macro}{\HoLogoHtml@KOMAScript}
%    \begin{macrocode}
\def\HoLogoHtml@KOMAScript#1{%
  \HoLogoCss@KOMAScript
  \HoLogoFont@font{KOMAScript}{sf}{%
    \HOLOGO@Span{KOMAScript}{%
      K%
      \HOLOGO@Span{O}{O}%
      M%
      \HOLOGO@Span{A}{A}%
      \HOLOGO@Span{hyphen}{-}%
      Script%
    }%
  }%
}
%    \end{macrocode}
%    \end{macro}
%    \begin{macro}{\HoLogoCss@KOMAScript}
%    \begin{macrocode}
\def\HoLogoCss@KOMAScript{%
  \Css{%
    span.HoLogo-KOMAScript{%
      font-family:sans-serif;%
    }%
  }%
  \Css{%
    span.HoLogo-KOMAScript span.HoLogo-O{%
      padding-left:.05em;%
      padding-right:.05em;%
    }%
  }%
  \Css{%
    span.HoLogo-KOMAScript span.HoLogo-A{%
      padding-left:.05em;%
    }%
  }%
  \Css{%
    span.HoLogo-KOMAScript span.HoLogo-hyphen{%
      padding-left:.1em;%
      padding-right:.1em;%
    }%
  }%
  \global\let\HoLogoCss@KOMAScript\relax
}
%    \end{macrocode}
%    \end{macro}
%
% \subsubsection{\hologo{LyX}}
%
%    \begin{macro}{\HoLogo@LyX}
%    The definition is taken from the documentation source files
%    of \hologo{LyX}, \xfile{Intro.lyx} \cite{LyX}:
%\begin{quote}
%\begin{verbatim}
%\def\LyX{%
%  \texorpdfstring{%
%    L\kern-.1667em\lower.25em\hbox{Y}\kern-.125emX\@%
%  }{%
%    LyX%
%  }%
%}
%\end{verbatim}
%\end{quote}
%    \begin{macrocode}
\def\HoLogo@LyX#1{%
  L%
  \kern-.1667em%
  \lower.25em\hbox{Y}%
  \kern-.125em%
  X%
  \HOLOGO@SpaceFactor
}
%    \end{macrocode}
%    \end{macro}
%    \begin{macro}{\HoLogoHtml@LyX}
%    \begin{macrocode}
\def\HoLogoHtml@LyX#1{%
  \HoLogoCss@LyX
  \HOLOGO@Span{LyX}{%
    L%
    \HOLOGO@Span{y}{Y}%
    X%
  }%
}
%    \end{macrocode}
%    \end{macro}
%    \begin{macro}{\HoLogoCss@LyX}
%    \begin{macrocode}
\def\HoLogoCss@LyX{%
  \Css{%
    span.HoLogo-LyX span.HoLogo-y{%
      position:relative;%
      top:.25em;%
      margin-left:-.1667em;%
      margin-right:-.125em;%
      text-decoration:none;%
    }%
  }%
  \global\let\HoLogoCss@LyX\relax
}
%    \end{macrocode}
%    \end{macro}
%
% \subsubsection{\hologo{NTS}}
%
%    \begin{macro}{\HoLogo@NTS}
%    Definition for \hologo{NTS} can be found in
%    package \xpackage{etex\textunderscore man} for the \hologo{eTeX} manual \cite{etexman}
%    and in package \xpackage{dtklogos} \cite{dtklogos}:
%\begin{quote}
%\begin{verbatim}
%\def\NTS{%
%  \leavevmode
%  \hbox{%
%    $%
%      \cal N%
%      \kern-0.35em%
%      \lower0.5ex\hbox{$\cal T$}%
%      \kern-0.2em%
%      S%
%    $%
%  }%
%}
%\end{verbatim}
%\end{quote}
%    \begin{macrocode}
\def\HoLogo@NTS#1{%
  \HoLogoFont@font{NTS}{sy}{%
    N\/%
    \kern-.35em%
    \lower.5ex\hbox{T\/}%
    \kern-.2em%
    S\/%
  }%
  \HOLOGO@SpaceFactor
}
%    \end{macrocode}
%    \end{macro}
%
% \subsubsection{\Hologo{TTH} (\hologo{TeX} to HTML translator)}
%
%    Source: \url{http://hutchinson.belmont.ma.us/tth/}
%    In the HTML source the second `T' is printed as subscript.
%\begin{quote}
%\begin{verbatim}
%T<sub>T</sub>H
%\end{verbatim}
%\end{quote}
%    \begin{macro}{\HoLogo@TTH}
%    \begin{macrocode}
\def\HoLogo@TTH#1{%
  \ltx@mbox{%
    T\HOLOGO@SubScript{T}H%
  }%
  \HOLOGO@SpaceFactor
}
%    \end{macrocode}
%    \end{macro}
%
%    \begin{macro}{\HoLogoHtml@TTH}
%    \begin{macrocode}
\def\HoLogoHtml@TTH#1{%
  T\HCode{<sub>}T\HCode{</sub>}H%
}
%    \end{macrocode}
%    \end{macro}
%
% \subsubsection{\Hologo{HanTheThanh}}
%
%    Partial source: Package \xpackage{dtklogos}.
%    The double accent is U+1EBF (latin small letter e with circumflex
%    and acute).
%    \begin{macro}{\HoLogo@HanTheThanh}
%    \begin{macrocode}
\def\HoLogo@HanTheThanh#1{%
  \ltx@mbox{H\`an}%
  \HOLOGO@space
  \ltx@mbox{%
    Th%
    \HOLOGO@IfCharExists{"1EBF}{%
      \char"1EBF\relax
    }{%
      \^e\hbox to 0pt{\hss\raise .5ex\hbox{\'{}}}%
    }%
  }%
  \HOLOGO@space
  \ltx@mbox{Th\`anh}%
}
%    \end{macrocode}
%    \end{macro}
%    \begin{macro}{\HoLogoBkm@HanTheThanh}
%    \begin{macrocode}
\def\HoLogoBkm@HanTheThanh#1{%
  H\`an %
  Th\HOLOGO@PdfdocUnicode{\^e}{\9036\277} %
  Th\`anh%
}
%    \end{macrocode}
%    \end{macro}
%    \begin{macro}{\HoLogoHtml@HanTheThanh}
%    \begin{macrocode}
\def\HoLogoHtml@HanTheThanh#1{%
  H\`an %
  Th\HCode{&\ltx@hashchar x1ebf;} %
  Th\`anh%
}
%    \end{macrocode}
%    \end{macro}
%
% \subsection{Driver detection}
%
%    \begin{macrocode}
\HOLOGO@IfExists\InputIfFileExists{%
  \InputIfFileExists{hologo.cfg}{}{}%
}{%
  \ltx@IfUndefined{pdf@filesize}{%
    \def\HOLOGO@InputIfExists{%
      \openin\HOLOGO@temp=hologo.cfg\relax
      \ifeof\HOLOGO@temp
        \closein\HOLOGO@temp
      \else
        \closein\HOLOGO@temp
        \begingroup
          \def\x{LaTeX2e}%
        \expandafter\endgroup
        \ifx\fmtname\x
          \input{hologo.cfg}%
        \else
          \input hologo.cfg\relax
        \fi
      \fi
    }%
    \ltx@IfUndefined{newread}{%
      \chardef\HOLOGO@temp=15 %
      \def\HOLOGO@CheckRead{%
        \ifeof\HOLOGO@temp
          \HOLOGO@InputIfExists
        \else
          \ifcase\HOLOGO@temp
            \@PackageWarningNoLine{hologo}{%
              Configuration file ignored, because\MessageBreak
              a free read register could not be found%
            }%
          \else
            \begingroup
              \count\ltx@cclv=\HOLOGO@temp
              \advance\ltx@cclv by \ltx@minusone
              \edef\x{\endgroup
                \chardef\noexpand\HOLOGO@temp=\the\count\ltx@cclv
                \relax
              }%
            \x
          \fi
        \fi
      }%
    }{%
      \csname newread\endcsname\HOLOGO@temp
      \HOLOGO@InputIfExists
    }%
  }{%
    \edef\HOLOGO@temp{\pdf@filesize{hologo.cfg}}%
    \ifx\HOLOGO@temp\ltx@empty
    \else
      \ifnum\HOLOGO@temp>0 %
        \begingroup
          \def\x{LaTeX2e}%
        \expandafter\endgroup
        \ifx\fmtname\x
          \input{hologo.cfg}%
        \else
          \input hologo.cfg\relax
        \fi
      \else
        \@PackageInfoNoLine{hologo}{%
          Empty configuration file `hologo.cfg' ignored%
        }%
      \fi
    \fi
  }%
}
%    \end{macrocode}
%
%    \begin{macrocode}
\def\HOLOGO@temp#1#2{%
  \kv@define@key{HoLogoDriver}{#1}[]{%
    \begingroup
      \def\HOLOGO@temp{##1}%
      \ltx@onelevel@sanitize\HOLOGO@temp
      \ifx\HOLOGO@temp\ltx@empty
      \else
        \@PackageError{hologo}{%
          Value (\HOLOGO@temp) not permitted for option `#1'%
        }%
        \@ehc
      \fi
    \endgroup
    \def\hologoDriver{#2}%
  }%
}%
\def\HOLOGO@@temp#1#2{%
  \ifx\kv@value\relax
    \HOLOGO@temp{#1}{#1}%
  \else
    \HOLOGO@temp{#1}{#2}%
  \fi
}%
\kv@parse@normalized{%
  pdftex,%
  luatex=pdftex,%
  dvipdfm,%
  dvipdfmx=dvipdfm,%
  dvips,%
  dvipsone=dvips,%
  xdvi=dvips,%
  xetex,%
  vtex,%
}\HOLOGO@@temp
%    \end{macrocode}
%
%    \begin{macrocode}
\kv@define@key{HoLogoDriver}{driverfallback}{%
  \def\HOLOGO@DriverFallback{#1}%
}
%    \end{macrocode}
%
%    \begin{macro}{\HOLOGO@DriverFallback}
%    \begin{macrocode}
\def\HOLOGO@DriverFallback{dvips}
%    \end{macrocode}
%    \end{macro}
%
%    \begin{macro}{\hologoDriverSetup}
%    \begin{macrocode}
\def\hologoDriverSetup{%
  \let\hologoDriver\ltx@undefined
  \HOLOGO@DriverSetup
}
%    \end{macrocode}
%    \end{macro}
%
%    \begin{macro}{\HOLOGO@DriverSetup}
%    \begin{macrocode}
\def\HOLOGO@DriverSetup#1{%
  \kvsetkeys{HoLogoDriver}{#1}%
  \HOLOGO@CheckDriver
  \ltx@ifundefined{hologoDriver}{%
    \begingroup
    \edef\x{\endgroup
      \noexpand\kvsetkeys{HoLogoDriver}{\HOLOGO@DriverFallback}%
    }\x
  }{}%
  \@PackageInfoNoLine{hologo}{Using driver `\hologoDriver'}%
}
%    \end{macrocode}
%    \end{macro}
%
%    \begin{macro}{\HOLOGO@CheckDriver}
%    \begin{macrocode}
\def\HOLOGO@CheckDriver{%
  \ifpdf
    \def\hologoDriver{pdftex}%
    \let\HOLOGO@pdfliteral\pdfliteral
    \ifluatex
      \ifx\pdfextension\@undefined\else
        \protected\def\pdfliteral{\pdfextension literal}%
        \let\HOLOGO@pdfliteral\pdfliteral
      \fi
      \ltx@IfUndefined{HOLOGO@pdfliteral}{%
        \ifnum\luatexversion<36 %
        \else
          \begingroup
            \let\HOLOGO@temp\endgroup
            \ifcase0%
                \directlua{%
                  if tex.enableprimitives then %
                    tex.enableprimitives('HOLOGO@', {'pdfliteral'})%
                  else %
                    tex.print('1')%
                  end%
                }%
                \ifx\HOLOGO@pdfliteral\@undefined 1\fi%
                \relax%
              \endgroup
              \let\HOLOGO@temp\relax
              \global\let\HOLOGO@pdfliteral\HOLOGO@pdfliteral
            \fi%
          \HOLOGO@temp
        \fi
      }{}%
    \fi
    \ltx@IfUndefined{HOLOGO@pdfliteral}{%
      \@PackageWarningNoLine{hologo}{%
        Cannot find \string\pdfliteral
      }%
    }{}%
  \else
    \ifxetex
      \def\hologoDriver{xetex}%
    \else
      \ifvtex
        \def\hologoDriver{vtex}%
      \fi
    \fi
  \fi
}
%    \end{macrocode}
%    \end{macro}
%
%    \begin{macro}{\HOLOGO@WarningUnsupportedDriver}
%    \begin{macrocode}
\def\HOLOGO@WarningUnsupportedDriver#1{%
  \@PackageWarningNoLine{hologo}{%
    Logo `#1' needs driver specific macros,\MessageBreak
    but driver `\hologoDriver' is not supported.\MessageBreak
    Use a different driver or\MessageBreak
    load package `graphics' or `pgf'%
  }%
}
%    \end{macrocode}
%    \end{macro}
%
% \subsubsection{Reflect box macros}
%
%    Skip driver part if not needed.
%    \begin{macrocode}
\ltx@IfUndefined{reflectbox}{}{%
  \ltx@IfUndefined{rotatebox}{}{%
    \HOLOGO@AtEnd
  }%
}
\ltx@IfUndefined{pgftext}{}{%
  \HOLOGO@AtEnd
}
\ltx@IfUndefined{psscalebox}{}{%
  \HOLOGO@AtEnd
}
%    \end{macrocode}
%
%    \begin{macrocode}
\def\HOLOGO@temp{LaTeX2e}
\ifx\fmtname\HOLOGO@temp
  \RequirePackage{kvoptions}[2011/06/30]%
  \ProcessKeyvalOptions{HoLogoDriver}%
\fi
\HOLOGO@DriverSetup{}
%    \end{macrocode}
%
%    \begin{macro}{\HOLOGO@ReflectBox}
%    \begin{macrocode}
\def\HOLOGO@ReflectBox#1{%
  \begingroup
    \setbox\ltx@zero\hbox{\begingroup#1\endgroup}%
    \setbox\ltx@two\hbox{%
      \kern\wd\ltx@zero
      \csname HOLOGO@ScaleBox@\hologoDriver\endcsname{-1}{1}{%
        \hbox to 0pt{\copy\ltx@zero\hss}%
      }%
    }%
    \wd\ltx@two=\wd\ltx@zero
    \box\ltx@two
  \endgroup
}
%    \end{macrocode}
%    \end{macro}
%
%    \begin{macro}{\HOLOGO@PointReflectBox}
%    \begin{macrocode}
\def\HOLOGO@PointReflectBox#1{%
  \begingroup
    \setbox\ltx@zero\hbox{\begingroup#1\endgroup}%
    \setbox\ltx@two\hbox{%
      \kern\wd\ltx@zero
      \raise\ht\ltx@zero\hbox{%
        \csname HOLOGO@ScaleBox@\hologoDriver\endcsname{-1}{-1}{%
          \hbox to 0pt{\copy\ltx@zero\hss}%
        }%
      }%
    }%
    \wd\ltx@two=\wd\ltx@zero
    \box\ltx@two
  \endgroup
}
%    \end{macrocode}
%    \end{macro}
%
%    We must define all variants because of dynamic driver setup.
%    \begin{macrocode}
\def\HOLOGO@temp#1#2{#2}
%    \end{macrocode}
%
%    \begin{macro}{\HOLOGO@ScaleBox@pdftex}
%    \begin{macrocode}
\HOLOGO@temp{pdftex}{%
  \def\HOLOGO@ScaleBox@pdftex#1#2#3{%
    \HOLOGO@pdfliteral{%
      q #1 0 0 #2 0 0 cm%
    }%
    #3%
    \HOLOGO@pdfliteral{%
      Q%
    }%
  }%
}
%    \end{macrocode}
%    \end{macro}
%    \begin{macro}{\HOLOGO@ScaleBox@dvips}
%    \begin{macrocode}
\HOLOGO@temp{dvips}{%
  \def\HOLOGO@ScaleBox@dvips#1#2#3{%
    \special{ps:%
      gsave %
      currentpoint %
      currentpoint translate %
      #1 #2 scale %
      neg exch neg exch translate%
    }%
    #3%
    \special{ps:%
      currentpoint %
      grestore %
      moveto%
    }%
  }%
}
%    \end{macrocode}
%    \end{macro}
%    \begin{macro}{\HOLOGO@ScaleBox@dvipdfm}
%    \begin{macrocode}
\HOLOGO@temp{dvipdfm}{%
  \let\HOLOGO@ScaleBox@dvipdfm\HOLOGO@ScaleBox@dvips
}
%    \end{macrocode}
%    \end{macro}
%    Since \hologo{XeTeX} v0.6.
%    \begin{macro}{\HOLOGO@ScaleBox@xetex}
%    \begin{macrocode}
\HOLOGO@temp{xetex}{%
  \def\HOLOGO@ScaleBox@xetex#1#2#3{%
    \special{x:gsave}%
    \special{x:scale #1 #2}%
    #3%
    \special{x:grestore}%
  }%
}
%    \end{macrocode}
%    \end{macro}
%    \begin{macro}{\HOLOGO@ScaleBox@vtex}
%    \begin{macrocode}
\HOLOGO@temp{vtex}{%
  \def\HOLOGO@ScaleBox@vtex#1#2#3{%
    \special{r(#1,0,0,#2,0,0}%
    #3%
    \special{r)}%
  }%
}
%    \end{macrocode}
%    \end{macro}
%
%    \begin{macrocode}
\HOLOGO@AtEnd%
%</package>
%    \end{macrocode}
%
% \section{Test}
%
% \subsection{Catcode checks for loading}
%
%    \begin{macrocode}
%<*test1>
%    \end{macrocode}
%    \begin{macrocode}
\catcode`\{=1 %
\catcode`\}=2 %
\catcode`\#=6 %
\catcode`\@=11 %
\expandafter\ifx\csname count@\endcsname\relax
  \countdef\count@=255 %
\fi
\expandafter\ifx\csname @gobble\endcsname\relax
  \long\def\@gobble#1{}%
\fi
\expandafter\ifx\csname @firstofone\endcsname\relax
  \long\def\@firstofone#1{#1}%
\fi
\expandafter\ifx\csname loop\endcsname\relax
  \expandafter\@firstofone
\else
  \expandafter\@gobble
\fi
{%
  \def\loop#1\repeat{%
    \def\body{#1}%
    \iterate
  }%
  \def\iterate{%
    \body
      \let\next\iterate
    \else
      \let\next\relax
    \fi
    \next
  }%
  \let\repeat=\fi
}%
\def\RestoreCatcodes{}
\count@=0 %
\loop
  \edef\RestoreCatcodes{%
    \RestoreCatcodes
    \catcode\the\count@=\the\catcode\count@\relax
  }%
\ifnum\count@<255 %
  \advance\count@ 1 %
\repeat

\def\RangeCatcodeInvalid#1#2{%
  \count@=#1\relax
  \loop
    \catcode\count@=15 %
  \ifnum\count@<#2\relax
    \advance\count@ 1 %
  \repeat
}
\def\RangeCatcodeCheck#1#2#3{%
  \count@=#1\relax
  \loop
    \ifnum#3=\catcode\count@
    \else
      \errmessage{%
        Character \the\count@\space
        with wrong catcode \the\catcode\count@\space
        instead of \number#3%
      }%
    \fi
  \ifnum\count@<#2\relax
    \advance\count@ 1 %
  \repeat
}
\def\space{ }
\expandafter\ifx\csname LoadCommand\endcsname\relax
  \def\LoadCommand{\input hologo.sty\relax}%
\fi
\def\Test{%
  \RangeCatcodeInvalid{0}{47}%
  \RangeCatcodeInvalid{58}{64}%
  \RangeCatcodeInvalid{91}{96}%
  \RangeCatcodeInvalid{123}{255}%
  \catcode`\@=12 %
  \catcode`\\=0 %
  \catcode`\%=14 %
  \LoadCommand
  \RangeCatcodeCheck{0}{36}{15}%
  \RangeCatcodeCheck{37}{37}{14}%
  \RangeCatcodeCheck{38}{47}{15}%
  \RangeCatcodeCheck{48}{57}{12}%
  \RangeCatcodeCheck{58}{63}{15}%
  \RangeCatcodeCheck{64}{64}{12}%
  \RangeCatcodeCheck{65}{90}{11}%
  \RangeCatcodeCheck{91}{91}{15}%
  \RangeCatcodeCheck{92}{92}{0}%
  \RangeCatcodeCheck{93}{96}{15}%
  \RangeCatcodeCheck{97}{122}{11}%
  \RangeCatcodeCheck{123}{255}{15}%
  \RestoreCatcodes
}
\Test
\csname @@end\endcsname
\end
%    \end{macrocode}
%    \begin{macrocode}
%</test1>
%    \end{macrocode}
%
% \subsection{Spacefactor}
%
%    The space factor must be 1000 after a logo. If it is greater 1000
%    then the following space is a space after a sentence closing point.
%    If the space factor is smaller 1000 then an immediate following
%    dot is interpreted as abbreviation, not sentence closing point.
%
%    \begin{macrocode}
%<*test-spacefactor>
\NeedsTeXFormat{LaTeX2e}
\documentclass{article}
\usepackage{hologo}[2016/05/12]
\usepackage{kvsetkeys}
\usepackage{qstest}
\IncludeTests{*}
\LogTests{log}{*}{*}
\begin{document}
\begin{qstest}{spacefactor}{spacefactor}
\newcommand*{\Test}[1]{%
  \sbox0{%
    \hologo{#1}%
    \Expect*{1000 (#1)}*{\the\spacefactor\space(#1)}%
  }%
}%
\makeatletter
\def\TestList{}
\def\hologoEntry#1#2#3{%
  \edef\TestList{%
    \ifx\TestList\@empty
    \else
      \TestList,%
    \fi
    #1%
    \ifx\\#2\\%
    \else
      ={variant=#2}%
    \fi
  }%
}
\hologoList
\expandafter\kv@parse@normalized\expandafter{%
  \TestList
}{%
  \begingroup
    \let\@logo=\kv@key
    \ifx\kv@value\relax
    \else
      \expandafter\hologoLogoSetup\expandafter\@logo\expandafter{%
        \kv@value
      }%
    \fi
    \Test\@logo
  \endgroup
  \@gobbletwo
}
\end{qstest}
\end{document}
%</test-spacefactor>
%    \end{macrocode}
%
% \subsection{Complete list}
%
%    \begin{macrocode}
%<*test-list>
\NeedsTeXFormat{LaTeX2e}
\documentclass[12pt,a4paper]{article}
\usepackage{hologo}[2016/05/12]
\usepackage[T1]{fontenc}
\usepackage{lmodern}
\usepackage{parskip}
\usepackage[unicode]{hyperref}[2011/09/28]
\usepackage{bookmark}[2011/09/19]
\bookmarksetup{%
  numbered,%
  open,%
  openlevel=2,%
}
\renewcommand*{\contentsname}{List of logos}
\begin{document}
\tableofcontents
\def\TestFont#1#2#3#4#5#6{%
  \begingroup
    \usefont{#3}{#4}{#5}{#6}%
    \HologoVariant{#1}{#2}/\hologoVariant{#1}{#2}%
    \quad
    \begingroup\scriptsize\hologoVariant{#1}{#2}\endgroup
    \quad
  \endgroup
  (#3/#4/#5/#6)%
  \par
}
\makeatletter
\def\hologoEntry#1#2#3{%
  \section{%
    \HologoVariant{#1}{#2}/\hologoVariant{#1}{#2} %
    {[#1\ifx\\#2\\\else\space(#2)\fi]}% hash-ok
  }% braces around [] because of bug in tex4ht
  \begingroup
    \hypersetup{unicode=false}%
    \bookmark[%
      dest=\@currentHref,%
      rellevel=1,%
      keeplevel,%
    ]{%
      \HologoVariant{#1}{#2}/\hologoVariant{#1}{#2} %
      (PDFDocEncoding)%
    }%
  \endgroup
  \TestFont{#1}{#2}{OT1}{cmr}{m}{n}%
  \TestFont{#1}{#2}{OT1}{cmss}{m}{n}%
  \TestFont{#1}{#2}{OT1}{cmr}{b}{n}%
  \TestFont{#1}{#2}{OT1}{cmr}{m}{it}%
  \TestFont{#1}{#2}{OT1}{cmtt}{m}{n}%
  \TestFont{#1}{#2}{T1}{lmr}{m}{n}%
  \TestFont{#1}{#2}{T1}{lmss}{m}{n}%
  \TestFont{#1}{#2}{T1}{lmr}{b}{n}%
  \TestFont{#1}{#2}{T1}{lmr}{m}{it}%
  \TestFont{#1}{#2}{T1}{lmtt}{m}{n}%
  \TestFont{#1}{#2}{T1}{lmvtt}{m}{n}%
  \TestFont{#1}{#2}{T1}{qtm}{m}{n}%
  \TestFont{#1}{#2}{T1}{qhv}{m}{n}%
  \TestFont{#1}{#2}{T1}{qtm}{b}{n}%
  \TestFont{#1}{#2}{T1}{qtm}{m}{it}%
  \TestFont{#1}{#2}{T1}{qcr}{m}{n}%
  \newpage
}
\makeatother
\hologoList
\end{document}
%</test-list>
%    \end{macrocode}
%
% \section{Installation}
%
% \subsection{Download}
%
% \paragraph{Package.} This package is available on
% CTAN\footnote{\url{ftp://ftp.ctan.org/tex-archive/}}:
% \begin{description}
% \item[\CTAN{macros/latex/contrib/oberdiek/hologo.dtx}] The source file.
% \item[\CTAN{macros/latex/contrib/oberdiek/hologo.pdf}] Documentation.
% \end{description}
%
%
% \paragraph{Bundle.} All the packages of the bundle `oberdiek'
% are also available in a TDS compliant ZIP archive. There
% the packages are already unpacked and the documentation files
% are generated. The files and directories obey the TDS standard.
% \begin{description}
% \item[\CTAN{install/macros/latex/contrib/oberdiek.tds.zip}]
% \end{description}
% \emph{TDS} refers to the standard ``A Directory Structure
% for \TeX\ Files'' (\CTAN{tds/tds.pdf}). Directories
% with \xfile{texmf} in their name are usually organized this way.
%
% \subsection{Bundle installation}
%
% \paragraph{Unpacking.} Unpack the \xfile{oberdiek.tds.zip} in the
% TDS tree (also known as \xfile{texmf} tree) of your choice.
% Example (linux):
% \begin{quote}
%   |unzip oberdiek.tds.zip -d ~/texmf|
% \end{quote}
%
% \paragraph{Script installation.}
% Check the directory \xfile{TDS:scripts/oberdiek/} for
% scripts that need further installation steps.
% Package \xpackage{attachfile2} comes with the Perl script
% \xfile{pdfatfi.pl} that should be installed in such a way
% that it can be called as \texttt{pdfatfi}.
% Example (linux):
% \begin{quote}
%   |chmod +x scripts/oberdiek/pdfatfi.pl|\\
%   |cp scripts/oberdiek/pdfatfi.pl /usr/local/bin/|
% \end{quote}
%
% \subsection{Package installation}
%
% \paragraph{Unpacking.} The \xfile{.dtx} file is a self-extracting
% \docstrip\ archive. The files are extracted by running the
% \xfile{.dtx} through \plainTeX:
% \begin{quote}
%   \verb|tex hologo.dtx|
% \end{quote}
%
% \paragraph{TDS.} Now the different files must be moved into
% the different directories in your installation TDS tree
% (also known as \xfile{texmf} tree):
% \begin{quote}
% \def\t{^^A
% \begin{tabular}{@{}>{\ttfamily}l@{ $\rightarrow$ }>{\ttfamily}l@{}}
%   hologo.sty & tex/generic/oberdiek/hologo.sty\\
%   hologo.pdf & doc/latex/oberdiek/hologo.pdf\\
%   example/hologo-example.tex & doc/latex/oberdiek/example/hologo-example.tex\\
%   test/hologo-test1.tex & doc/latex/oberdiek/test/hologo-test1.tex\\
%   test/hologo-test-spacefactor.tex & doc/latex/oberdiek/test/hologo-test-spacefactor.tex\\
%   test/hologo-test-list.tex & doc/latex/oberdiek/test/hologo-test-list.tex\\
%   hologo.dtx & source/latex/oberdiek/hologo.dtx\\
% \end{tabular}^^A
% }^^A
% \sbox0{\t}^^A
% \ifdim\wd0>\linewidth
%   \begingroup
%     \advance\linewidth by\leftmargin
%     \advance\linewidth by\rightmargin
%   \edef\x{\endgroup
%     \def\noexpand\lw{\the\linewidth}^^A
%   }\x
%   \def\lwbox{^^A
%     \leavevmode
%     \hbox to \linewidth{^^A
%       \kern-\leftmargin\relax
%       \hss
%       \usebox0
%       \hss
%       \kern-\rightmargin\relax
%     }^^A
%   }^^A
%   \ifdim\wd0>\lw
%     \sbox0{\small\t}^^A
%     \ifdim\wd0>\linewidth
%       \ifdim\wd0>\lw
%         \sbox0{\footnotesize\t}^^A
%         \ifdim\wd0>\linewidth
%           \ifdim\wd0>\lw
%             \sbox0{\scriptsize\t}^^A
%             \ifdim\wd0>\linewidth
%               \ifdim\wd0>\lw
%                 \sbox0{\tiny\t}^^A
%                 \ifdim\wd0>\linewidth
%                   \lwbox
%                 \else
%                   \usebox0
%                 \fi
%               \else
%                 \lwbox
%               \fi
%             \else
%               \usebox0
%             \fi
%           \else
%             \lwbox
%           \fi
%         \else
%           \usebox0
%         \fi
%       \else
%         \lwbox
%       \fi
%     \else
%       \usebox0
%     \fi
%   \else
%     \lwbox
%   \fi
% \else
%   \usebox0
% \fi
% \end{quote}
% If you have a \xfile{docstrip.cfg} that configures and enables \docstrip's
% TDS installing feature, then some files can already be in the right
% place, see the documentation of \docstrip.
%
% \subsection{Refresh file name databases}
%
% If your \TeX~distribution
% (\teTeX, \mikTeX, \dots) relies on file name databases, you must refresh
% these. For example, \teTeX\ users run \verb|texhash| or
% \verb|mktexlsr|.
%
% \subsection{Some details for the interested}
%
% \paragraph{Attached source.}
%
% The PDF documentation on CTAN also includes the
% \xfile{.dtx} source file. It can be extracted by
% AcrobatReader 6 or higher. Another option is \textsf{pdftk},
% e.g. unpack the file into the current directory:
% \begin{quote}
%   \verb|pdftk hologo.pdf unpack_files output .|
% \end{quote}
%
% \paragraph{Unpacking with \LaTeX.}
% The \xfile{.dtx} chooses its action depending on the format:
% \begin{description}
% \item[\plainTeX:] Run \docstrip\ and extract the files.
% \item[\LaTeX:] Generate the documentation.
% \end{description}
% If you insist on using \LaTeX\ for \docstrip\ (really,
% \docstrip\ does not need \LaTeX), then inform the autodetect routine
% about your intention:
% \begin{quote}
%   \verb|latex \let\install=y\input{hologo.dtx}|
% \end{quote}
% Do not forget to quote the argument according to the demands
% of your shell.
%
% \paragraph{Generating the documentation.}
% You can use both the \xfile{.dtx} or the \xfile{.drv} to generate
% the documentation. The process can be configured by the
% configuration file \xfile{ltxdoc.cfg}. For instance, put this
% line into this file, if you want to have A4 as paper format:
% \begin{quote}
%   \verb|\PassOptionsToClass{a4paper}{article}|
% \end{quote}
% An example follows how to generate the
% documentation with pdf\LaTeX:
% \begin{quote}
%\begin{verbatim}
%pdflatex hologo.dtx
%makeindex -s gind.ist hologo.idx
%pdflatex hologo.dtx
%makeindex -s gind.ist hologo.idx
%pdflatex hologo.dtx
%\end{verbatim}
% \end{quote}
%
% \section{Catalogue}
%
% The following XML file can be used as source for the
% \href{http://mirror.ctan.org/help/Catalogue/catalogue.html}{\TeX\ Catalogue}.
% The elements \texttt{caption} and \texttt{description} are imported
% from the original XML file from the Catalogue.
% The name of the XML file in the Catalogue is \xfile{hologo.xml}.
%    \begin{macrocode}
%<*catalogue>
<?xml version='1.0' encoding='us-ascii'?>
<!DOCTYPE entry SYSTEM 'catalogue.dtd'>
<entry datestamp='$Date$' modifier='$Author$' id='hologo'>
  <name>hologo</name>
  <caption>A collection of logos with bookmark support.</caption>
  <authorref id='auth:oberdiek'/>
  <copyright owner='Heiko Oberdiek' year='2010-2012'/>
  <license type='lppl1.3'/>
  <version number='1.10'/>
  <description>
    The package defines a single command <tt>\hologo</tt>, whose
    argument is the usual case-confused ASCII version of the logo.
    The command is bookmark-enabled, so that every logo becomes
    available in bookmarks without further work.
    <p/>
    The package is part of the <xref refid='oberdiek'>oberdiek</xref>
    bundle.
  </description>
  <documentation details='Package documentation'
      href='ctan:/macros/latex/contrib/oberdiek/hologo.pdf'/>
  <ctan file='true' path='/macros/latex/contrib/oberdiek/hologo.dtx'/>
  <miktex location='oberdiek'/>
  <texlive location='oberdiek'/>
  <install path='/macros/latex/contrib/oberdiek/oberdiek.tds.zip'/>
</entry>
%</catalogue>
%    \end{macrocode}
%
% \begin{thebibliography}{9}
% \raggedright
%
% \bibitem{btxdoc}
% Oren Patashnik,
% \textit{\hologo{BibTeX}ing},
% 1988-02-08.\\
% \CTAN{biblio/bibtex/base/}
%
% \bibitem{dtklogos}
% Gerd Neugebauer, DANTE,
% \textit{Package \xpackage{dtklogos}},
% 2011-04-25.\\
% \CTAN{usergrps/dante/dtk/dtklogos.sty}
%
% \bibitem{etexman}
% The \hologo{NTS} Team,
% \textit{The \hologo{eTeX} manual},
% 1998-02.\\
% \CTAN{systems/e-tex/v2/doc/}
%
% \bibitem{ExTeX-FAQ}
% The \hologo{ExTeX} group,
% \textit{\hologo{ExTeX}: FAQ -- How is \hologo{ExTeX} typeset?},
% 2007-04-14.\\
% \url{http://www.extex.org/documentation/faq.html}
%
% \bibitem{LyX}
% %@MISC{ LyX,
% %  title = {{LyX 2.0.0 -- The Document Processor [Computer software and manual]}},
% %  author = {{The LyX Team}},
% %  howpublished = {Internet: http://www.lyx.org},
% %  year = {2011-05-08},
% %  note = {Retrieved May 10, 2011, from http://www.lyx.org},
% %  url = {http://www.lyx.org/}
% %}
% The \hologo{LyX} Team,
% \textit{\hologo{LyX} -- The Document Processor},
% 2011-05-08.\\
% \url{http://www.lyx.org/}
%
% \bibitem{OzTeX}
% Andrew Trevorrow,
% \hologo{OzTeX} FAQ: What is the correct way to typeset ``\hologo{OzTeX}''?,
% 2011-09-15 (visited).
% \url{http://www.trevorrow.com/oztex/ozfaq.html#oztex-logo}
%
% \bibitem{PiCTeX}
% Michael Wichura,
% \textit{The \hologo{PiCTeX} macro package},
% 1987-09-21.
% \CTAN{graphics/pictex/}
%
% \bibitem{scrlogo}
% Markus Kohm,
% \textit{\hologo{KOMAScript} Datei \xfile{scrlogo.dtx}},
% 2009-01-30.\\
% \CTAN{install/macros/latex/contrib/komascript.tds.zip}
%
% \end{thebibliography}
%
% \begin{History}
%   \begin{Version}{2010/04/08 v1.0}
%   \item
%     The first version.
%   \end{Version}
%   \begin{Version}{2010/04/16 v1.1}
%   \item
%     \cs{Hologo} added for support of logos at start of a sentence.
%   \item
%     \cs{hologoSetup} and \cs{hologoLogoSetup} added.
%   \item
%     Options \xoption{break}, \xoption{hyphenbreak}, \xoption{spacebreak}
%     added.
%   \item
%     Variant support added by option \xoption{variant}.
%   \end{Version}
%   \begin{Version}{2010/04/24 v1.2}
%   \item
%     \hologo{LaTeX3} added.
%   \item
%     \hologo{VTeX} added.
%   \end{Version}
%   \begin{Version}{2010/11/21 v1.3}
%   \item
%     \hologo{iniTeX}, \hologo{virTeX} added.
%   \end{Version}
%   \begin{Version}{2011/03/25 v1.4}
%   \item
%     \hologo{ConTeXt} with variants added.
%   \item
%     Option \xoption{discretionarybreak} added as refinement for
%     option \xoption{break}.
%   \end{Version}
%   \begin{Version}{2011/04/21 v1.5}
%   \item
%     Wrong TDS directory for test files fixed.
%   \end{Version}
%   \begin{Version}{2011/10/01 v1.6}
%   \item
%     Support for package \xpackage{tex4ht} added.
%   \item
%     Support for \cs{csname} added if \cs{ifincsname} is available.
%   \item
%     New logos:
%     \hologo{(La)TeX},
%     \hologo{biber},
%     \hologo{BibTeX} (\xoption{sc}, \xoption{sf}),
%     \hologo{emTeX},
%     \hologo{ExTeX},
%     \hologo{KOMAScript},
%     \hologo{La},
%     \hologo{LyX},
%     \hologo{MiKTeX},
%     \hologo{NTS},
%     \hologo{OzMF},
%     \hologo{OzMP},
%     \hologo{OzTeX},
%     \hologo{OzTtH},
%     \hologo{PCTeX},
%     \hologo{PiC},
%     \hologo{PiCTeX},
%     \hologo{METAFONT},
%     \hologo{MetaFun},
%     \hologo{METAPOST},
%     \hologo{MetaPost},
%     \hologo{SLiTeX} (\xoption{lift}, \xoption{narrow}, \xoption{simple}),
%     \hologo{SliTeX} (\xoption{narrow}, \xoption{simple}, \xoption{lift}),
%     \hologo{teTeX}.
%   \item
%     Fixes:
%     \hologo{iniTeX},
%     \hologo{pdfLaTeX},
%     \hologo{pdfTeX},
%     \hologo{virTeX}.
%   \item
%     \cs{hologoFontSetup} and \cs{hologoLogoFontSetup} added.
%   \item
%     \cs{hologoVariant} and \cs{HologoVariant} added.
%   \end{Version}
%   \begin{Version}{2011/11/22 v1.7}
%   \item
%     New logos:
%     \hologo{BibTeX8},
%     \hologo{LaTeXML},
%     \hologo{SageTeX},
%     \hologo{TeX4ht},
%     \hologo{TTH}.
%   \item
%     \hologo{Xe} and friends: Driver stuff fixed.
%   \item
%     \hologo{Xe} and friends: Support for italic added.
%   \item
%     \hologo{Xe} and friends: Package support for \xpackage{pgf}
%     and \xpackage{pstricks} added.
%   \end{Version}
%   \begin{Version}{2011/11/29 v1.8}
%   \item
%     New logos:
%     \hologo{HanTheThanh}.
%   \end{Version}
%   \begin{Version}{2011/12/21 v1.9}
%   \item
%     Patch for package \xpackage{ifxetex} added for the case that
%     \cs{newif} is undefined in \hologo{iniTeX}.
%   \item
%     Some fixes for \hologo{iniTeX}.
%   \end{Version}
%   \begin{Version}{2012/04/26 v1.10}
%   \item
%     Fix in bookmark version of logo ``\hologo{HanTheThanh}''.
%   \end{Version}
%   \begin{Version}{2016/05/12 v1.11}
%   \item
%     Update HOLOGO@IfCharExists (previously in texlive)
%   \item define pdfliteral in current luatex.
%   \end{Version}
% \end{History}
%
% \PrintIndex
%
% \Finale
\endinput
%
        \else
          \input hologo.cfg\relax
        \fi
      \else
        \@PackageInfoNoLine{hologo}{%
          Empty configuration file `hologo.cfg' ignored%
        }%
      \fi
    \fi
  }%
}
%    \end{macrocode}
%
%    \begin{macrocode}
\def\HOLOGO@temp#1#2{%
  \kv@define@key{HoLogoDriver}{#1}[]{%
    \begingroup
      \def\HOLOGO@temp{##1}%
      \ltx@onelevel@sanitize\HOLOGO@temp
      \ifx\HOLOGO@temp\ltx@empty
      \else
        \@PackageError{hologo}{%
          Value (\HOLOGO@temp) not permitted for option `#1'%
        }%
        \@ehc
      \fi
    \endgroup
    \def\hologoDriver{#2}%
  }%
}%
\def\HOLOGO@@temp#1#2{%
  \ifx\kv@value\relax
    \HOLOGO@temp{#1}{#1}%
  \else
    \HOLOGO@temp{#1}{#2}%
  \fi
}%
\kv@parse@normalized{%
  pdftex,%
  luatex=pdftex,%
  dvipdfm,%
  dvipdfmx=dvipdfm,%
  dvips,%
  dvipsone=dvips,%
  xdvi=dvips,%
  xetex,%
  vtex,%
}\HOLOGO@@temp
%    \end{macrocode}
%
%    \begin{macrocode}
\kv@define@key{HoLogoDriver}{driverfallback}{%
  \def\HOLOGO@DriverFallback{#1}%
}
%    \end{macrocode}
%
%    \begin{macro}{\HOLOGO@DriverFallback}
%    \begin{macrocode}
\def\HOLOGO@DriverFallback{dvips}
%    \end{macrocode}
%    \end{macro}
%
%    \begin{macro}{\hologoDriverSetup}
%    \begin{macrocode}
\def\hologoDriverSetup{%
  \let\hologoDriver\ltx@undefined
  \HOLOGO@DriverSetup
}
%    \end{macrocode}
%    \end{macro}
%
%    \begin{macro}{\HOLOGO@DriverSetup}
%    \begin{macrocode}
\def\HOLOGO@DriverSetup#1{%
  \kvsetkeys{HoLogoDriver}{#1}%
  \HOLOGO@CheckDriver
  \ltx@ifundefined{hologoDriver}{%
    \begingroup
    \edef\x{\endgroup
      \noexpand\kvsetkeys{HoLogoDriver}{\HOLOGO@DriverFallback}%
    }\x
  }{}%
  \@PackageInfoNoLine{hologo}{Using driver `\hologoDriver'}%
}
%    \end{macrocode}
%    \end{macro}
%
%    \begin{macro}{\HOLOGO@CheckDriver}
%    \begin{macrocode}
\def\HOLOGO@CheckDriver{%
  \ifpdf
    \def\hologoDriver{pdftex}%
    \let\HOLOGO@pdfliteral\pdfliteral
    \ifluatex
      \ifx\pdfextension\@undefined\else
        \protected\def\pdfliteral{\pdfextension literal}%
        \let\HOLOGO@pdfliteral\pdfliteral
      \fi
      \ltx@IfUndefined{HOLOGO@pdfliteral}{%
        \ifnum\luatexversion<36 %
        \else
          \begingroup
            \let\HOLOGO@temp\endgroup
            \ifcase0%
                \directlua{%
                  if tex.enableprimitives then %
                    tex.enableprimitives('HOLOGO@', {'pdfliteral'})%
                  else %
                    tex.print('1')%
                  end%
                }%
                \ifx\HOLOGO@pdfliteral\@undefined 1\fi%
                \relax%
              \endgroup
              \let\HOLOGO@temp\relax
              \global\let\HOLOGO@pdfliteral\HOLOGO@pdfliteral
            \fi%
          \HOLOGO@temp
        \fi
      }{}%
    \fi
    \ltx@IfUndefined{HOLOGO@pdfliteral}{%
      \@PackageWarningNoLine{hologo}{%
        Cannot find \string\pdfliteral
      }%
    }{}%
  \else
    \ifxetex
      \def\hologoDriver{xetex}%
    \else
      \ifvtex
        \def\hologoDriver{vtex}%
      \fi
    \fi
  \fi
}
%    \end{macrocode}
%    \end{macro}
%
%    \begin{macro}{\HOLOGO@WarningUnsupportedDriver}
%    \begin{macrocode}
\def\HOLOGO@WarningUnsupportedDriver#1{%
  \@PackageWarningNoLine{hologo}{%
    Logo `#1' needs driver specific macros,\MessageBreak
    but driver `\hologoDriver' is not supported.\MessageBreak
    Use a different driver or\MessageBreak
    load package `graphics' or `pgf'%
  }%
}
%    \end{macrocode}
%    \end{macro}
%
% \subsubsection{Reflect box macros}
%
%    Skip driver part if not needed.
%    \begin{macrocode}
\ltx@IfUndefined{reflectbox}{}{%
  \ltx@IfUndefined{rotatebox}{}{%
    \HOLOGO@AtEnd
  }%
}
\ltx@IfUndefined{pgftext}{}{%
  \HOLOGO@AtEnd
}
\ltx@IfUndefined{psscalebox}{}{%
  \HOLOGO@AtEnd
}
%    \end{macrocode}
%
%    \begin{macrocode}
\def\HOLOGO@temp{LaTeX2e}
\ifx\fmtname\HOLOGO@temp
  \RequirePackage{kvoptions}[2011/06/30]%
  \ProcessKeyvalOptions{HoLogoDriver}%
\fi
\HOLOGO@DriverSetup{}
%    \end{macrocode}
%
%    \begin{macro}{\HOLOGO@ReflectBox}
%    \begin{macrocode}
\def\HOLOGO@ReflectBox#1{%
  \begingroup
    \setbox\ltx@zero\hbox{\begingroup#1\endgroup}%
    \setbox\ltx@two\hbox{%
      \kern\wd\ltx@zero
      \csname HOLOGO@ScaleBox@\hologoDriver\endcsname{-1}{1}{%
        \hbox to 0pt{\copy\ltx@zero\hss}%
      }%
    }%
    \wd\ltx@two=\wd\ltx@zero
    \box\ltx@two
  \endgroup
}
%    \end{macrocode}
%    \end{macro}
%
%    \begin{macro}{\HOLOGO@PointReflectBox}
%    \begin{macrocode}
\def\HOLOGO@PointReflectBox#1{%
  \begingroup
    \setbox\ltx@zero\hbox{\begingroup#1\endgroup}%
    \setbox\ltx@two\hbox{%
      \kern\wd\ltx@zero
      \raise\ht\ltx@zero\hbox{%
        \csname HOLOGO@ScaleBox@\hologoDriver\endcsname{-1}{-1}{%
          \hbox to 0pt{\copy\ltx@zero\hss}%
        }%
      }%
    }%
    \wd\ltx@two=\wd\ltx@zero
    \box\ltx@two
  \endgroup
}
%    \end{macrocode}
%    \end{macro}
%
%    We must define all variants because of dynamic driver setup.
%    \begin{macrocode}
\def\HOLOGO@temp#1#2{#2}
%    \end{macrocode}
%
%    \begin{macro}{\HOLOGO@ScaleBox@pdftex}
%    \begin{macrocode}
\HOLOGO@temp{pdftex}{%
  \def\HOLOGO@ScaleBox@pdftex#1#2#3{%
    \HOLOGO@pdfliteral{%
      q #1 0 0 #2 0 0 cm%
    }%
    #3%
    \HOLOGO@pdfliteral{%
      Q%
    }%
  }%
}
%    \end{macrocode}
%    \end{macro}
%    \begin{macro}{\HOLOGO@ScaleBox@dvips}
%    \begin{macrocode}
\HOLOGO@temp{dvips}{%
  \def\HOLOGO@ScaleBox@dvips#1#2#3{%
    \special{ps:%
      gsave %
      currentpoint %
      currentpoint translate %
      #1 #2 scale %
      neg exch neg exch translate%
    }%
    #3%
    \special{ps:%
      currentpoint %
      grestore %
      moveto%
    }%
  }%
}
%    \end{macrocode}
%    \end{macro}
%    \begin{macro}{\HOLOGO@ScaleBox@dvipdfm}
%    \begin{macrocode}
\HOLOGO@temp{dvipdfm}{%
  \let\HOLOGO@ScaleBox@dvipdfm\HOLOGO@ScaleBox@dvips
}
%    \end{macrocode}
%    \end{macro}
%    Since \hologo{XeTeX} v0.6.
%    \begin{macro}{\HOLOGO@ScaleBox@xetex}
%    \begin{macrocode}
\HOLOGO@temp{xetex}{%
  \def\HOLOGO@ScaleBox@xetex#1#2#3{%
    \special{x:gsave}%
    \special{x:scale #1 #2}%
    #3%
    \special{x:grestore}%
  }%
}
%    \end{macrocode}
%    \end{macro}
%    \begin{macro}{\HOLOGO@ScaleBox@vtex}
%    \begin{macrocode}
\HOLOGO@temp{vtex}{%
  \def\HOLOGO@ScaleBox@vtex#1#2#3{%
    \special{r(#1,0,0,#2,0,0}%
    #3%
    \special{r)}%
  }%
}
%    \end{macrocode}
%    \end{macro}
%
%    \begin{macrocode}
\HOLOGO@AtEnd%
%</package>
%    \end{macrocode}
%
% \section{Test}
%
% \subsection{Catcode checks for loading}
%
%    \begin{macrocode}
%<*test1>
%    \end{macrocode}
%    \begin{macrocode}
\catcode`\{=1 %
\catcode`\}=2 %
\catcode`\#=6 %
\catcode`\@=11 %
\expandafter\ifx\csname count@\endcsname\relax
  \countdef\count@=255 %
\fi
\expandafter\ifx\csname @gobble\endcsname\relax
  \long\def\@gobble#1{}%
\fi
\expandafter\ifx\csname @firstofone\endcsname\relax
  \long\def\@firstofone#1{#1}%
\fi
\expandafter\ifx\csname loop\endcsname\relax
  \expandafter\@firstofone
\else
  \expandafter\@gobble
\fi
{%
  \def\loop#1\repeat{%
    \def\body{#1}%
    \iterate
  }%
  \def\iterate{%
    \body
      \let\next\iterate
    \else
      \let\next\relax
    \fi
    \next
  }%
  \let\repeat=\fi
}%
\def\RestoreCatcodes{}
\count@=0 %
\loop
  \edef\RestoreCatcodes{%
    \RestoreCatcodes
    \catcode\the\count@=\the\catcode\count@\relax
  }%
\ifnum\count@<255 %
  \advance\count@ 1 %
\repeat

\def\RangeCatcodeInvalid#1#2{%
  \count@=#1\relax
  \loop
    \catcode\count@=15 %
  \ifnum\count@<#2\relax
    \advance\count@ 1 %
  \repeat
}
\def\RangeCatcodeCheck#1#2#3{%
  \count@=#1\relax
  \loop
    \ifnum#3=\catcode\count@
    \else
      \errmessage{%
        Character \the\count@\space
        with wrong catcode \the\catcode\count@\space
        instead of \number#3%
      }%
    \fi
  \ifnum\count@<#2\relax
    \advance\count@ 1 %
  \repeat
}
\def\space{ }
\expandafter\ifx\csname LoadCommand\endcsname\relax
  \def\LoadCommand{\input hologo.sty\relax}%
\fi
\def\Test{%
  \RangeCatcodeInvalid{0}{47}%
  \RangeCatcodeInvalid{58}{64}%
  \RangeCatcodeInvalid{91}{96}%
  \RangeCatcodeInvalid{123}{255}%
  \catcode`\@=12 %
  \catcode`\\=0 %
  \catcode`\%=14 %
  \LoadCommand
  \RangeCatcodeCheck{0}{36}{15}%
  \RangeCatcodeCheck{37}{37}{14}%
  \RangeCatcodeCheck{38}{47}{15}%
  \RangeCatcodeCheck{48}{57}{12}%
  \RangeCatcodeCheck{58}{63}{15}%
  \RangeCatcodeCheck{64}{64}{12}%
  \RangeCatcodeCheck{65}{90}{11}%
  \RangeCatcodeCheck{91}{91}{15}%
  \RangeCatcodeCheck{92}{92}{0}%
  \RangeCatcodeCheck{93}{96}{15}%
  \RangeCatcodeCheck{97}{122}{11}%
  \RangeCatcodeCheck{123}{255}{15}%
  \RestoreCatcodes
}
\Test
\csname @@end\endcsname
\end
%    \end{macrocode}
%    \begin{macrocode}
%</test1>
%    \end{macrocode}
%
% \subsection{Spacefactor}
%
%    The space factor must be 1000 after a logo. If it is greater 1000
%    then the following space is a space after a sentence closing point.
%    If the space factor is smaller 1000 then an immediate following
%    dot is interpreted as abbreviation, not sentence closing point.
%
%    \begin{macrocode}
%<*test-spacefactor>
\NeedsTeXFormat{LaTeX2e}
\documentclass{article}
\usepackage{hologo}[2016/05/12]
\usepackage{kvsetkeys}
\usepackage{qstest}
\IncludeTests{*}
\LogTests{log}{*}{*}
\begin{document}
\begin{qstest}{spacefactor}{spacefactor}
\newcommand*{\Test}[1]{%
  \sbox0{%
    \hologo{#1}%
    \Expect*{1000 (#1)}*{\the\spacefactor\space(#1)}%
  }%
}%
\makeatletter
\def\TestList{}
\def\hologoEntry#1#2#3{%
  \edef\TestList{%
    \ifx\TestList\@empty
    \else
      \TestList,%
    \fi
    #1%
    \ifx\\#2\\%
    \else
      ={variant=#2}%
    \fi
  }%
}
\hologoList
\expandafter\kv@parse@normalized\expandafter{%
  \TestList
}{%
  \begingroup
    \let\@logo=\kv@key
    \ifx\kv@value\relax
    \else
      \expandafter\hologoLogoSetup\expandafter\@logo\expandafter{%
        \kv@value
      }%
    \fi
    \Test\@logo
  \endgroup
  \@gobbletwo
}
\end{qstest}
\end{document}
%</test-spacefactor>
%    \end{macrocode}
%
% \subsection{Complete list}
%
%    \begin{macrocode}
%<*test-list>
\NeedsTeXFormat{LaTeX2e}
\documentclass[12pt,a4paper]{article}
\usepackage{hologo}[2016/05/12]
\usepackage[T1]{fontenc}
\usepackage{lmodern}
\usepackage{parskip}
\usepackage[unicode]{hyperref}[2011/09/28]
\usepackage{bookmark}[2011/09/19]
\bookmarksetup{%
  numbered,%
  open,%
  openlevel=2,%
}
\renewcommand*{\contentsname}{List of logos}
\begin{document}
\tableofcontents
\def\TestFont#1#2#3#4#5#6{%
  \begingroup
    \usefont{#3}{#4}{#5}{#6}%
    \HologoVariant{#1}{#2}/\hologoVariant{#1}{#2}%
    \quad
    \begingroup\scriptsize\hologoVariant{#1}{#2}\endgroup
    \quad
  \endgroup
  (#3/#4/#5/#6)%
  \par
}
\makeatletter
\def\hologoEntry#1#2#3{%
  \section{%
    \HologoVariant{#1}{#2}/\hologoVariant{#1}{#2} %
    {[#1\ifx\\#2\\\else\space(#2)\fi]}% hash-ok
  }% braces around [] because of bug in tex4ht
  \begingroup
    \hypersetup{unicode=false}%
    \bookmark[%
      dest=\@currentHref,%
      rellevel=1,%
      keeplevel,%
    ]{%
      \HologoVariant{#1}{#2}/\hologoVariant{#1}{#2} %
      (PDFDocEncoding)%
    }%
  \endgroup
  \TestFont{#1}{#2}{OT1}{cmr}{m}{n}%
  \TestFont{#1}{#2}{OT1}{cmss}{m}{n}%
  \TestFont{#1}{#2}{OT1}{cmr}{b}{n}%
  \TestFont{#1}{#2}{OT1}{cmr}{m}{it}%
  \TestFont{#1}{#2}{OT1}{cmtt}{m}{n}%
  \TestFont{#1}{#2}{T1}{lmr}{m}{n}%
  \TestFont{#1}{#2}{T1}{lmss}{m}{n}%
  \TestFont{#1}{#2}{T1}{lmr}{b}{n}%
  \TestFont{#1}{#2}{T1}{lmr}{m}{it}%
  \TestFont{#1}{#2}{T1}{lmtt}{m}{n}%
  \TestFont{#1}{#2}{T1}{lmvtt}{m}{n}%
  \TestFont{#1}{#2}{T1}{qtm}{m}{n}%
  \TestFont{#1}{#2}{T1}{qhv}{m}{n}%
  \TestFont{#1}{#2}{T1}{qtm}{b}{n}%
  \TestFont{#1}{#2}{T1}{qtm}{m}{it}%
  \TestFont{#1}{#2}{T1}{qcr}{m}{n}%
  \newpage
}
\makeatother
\hologoList
\end{document}
%</test-list>
%    \end{macrocode}
%
% \section{Installation}
%
% \subsection{Download}
%
% \paragraph{Package.} This package is available on
% CTAN\footnote{\url{ftp://ftp.ctan.org/tex-archive/}}:
% \begin{description}
% \item[\CTAN{macros/latex/contrib/oberdiek/hologo.dtx}] The source file.
% \item[\CTAN{macros/latex/contrib/oberdiek/hologo.pdf}] Documentation.
% \end{description}
%
%
% \paragraph{Bundle.} All the packages of the bundle `oberdiek'
% are also available in a TDS compliant ZIP archive. There
% the packages are already unpacked and the documentation files
% are generated. The files and directories obey the TDS standard.
% \begin{description}
% \item[\CTAN{install/macros/latex/contrib/oberdiek.tds.zip}]
% \end{description}
% \emph{TDS} refers to the standard ``A Directory Structure
% for \TeX\ Files'' (\CTAN{tds/tds.pdf}). Directories
% with \xfile{texmf} in their name are usually organized this way.
%
% \subsection{Bundle installation}
%
% \paragraph{Unpacking.} Unpack the \xfile{oberdiek.tds.zip} in the
% TDS tree (also known as \xfile{texmf} tree) of your choice.
% Example (linux):
% \begin{quote}
%   |unzip oberdiek.tds.zip -d ~/texmf|
% \end{quote}
%
% \paragraph{Script installation.}
% Check the directory \xfile{TDS:scripts/oberdiek/} for
% scripts that need further installation steps.
% Package \xpackage{attachfile2} comes with the Perl script
% \xfile{pdfatfi.pl} that should be installed in such a way
% that it can be called as \texttt{pdfatfi}.
% Example (linux):
% \begin{quote}
%   |chmod +x scripts/oberdiek/pdfatfi.pl|\\
%   |cp scripts/oberdiek/pdfatfi.pl /usr/local/bin/|
% \end{quote}
%
% \subsection{Package installation}
%
% \paragraph{Unpacking.} The \xfile{.dtx} file is a self-extracting
% \docstrip\ archive. The files are extracted by running the
% \xfile{.dtx} through \plainTeX:
% \begin{quote}
%   \verb|tex hologo.dtx|
% \end{quote}
%
% \paragraph{TDS.} Now the different files must be moved into
% the different directories in your installation TDS tree
% (also known as \xfile{texmf} tree):
% \begin{quote}
% \def\t{^^A
% \begin{tabular}{@{}>{\ttfamily}l@{ $\rightarrow$ }>{\ttfamily}l@{}}
%   hologo.sty & tex/generic/oberdiek/hologo.sty\\
%   hologo.pdf & doc/latex/oberdiek/hologo.pdf\\
%   example/hologo-example.tex & doc/latex/oberdiek/example/hologo-example.tex\\
%   test/hologo-test1.tex & doc/latex/oberdiek/test/hologo-test1.tex\\
%   test/hologo-test-spacefactor.tex & doc/latex/oberdiek/test/hologo-test-spacefactor.tex\\
%   test/hologo-test-list.tex & doc/latex/oberdiek/test/hologo-test-list.tex\\
%   hologo.dtx & source/latex/oberdiek/hologo.dtx\\
% \end{tabular}^^A
% }^^A
% \sbox0{\t}^^A
% \ifdim\wd0>\linewidth
%   \begingroup
%     \advance\linewidth by\leftmargin
%     \advance\linewidth by\rightmargin
%   \edef\x{\endgroup
%     \def\noexpand\lw{\the\linewidth}^^A
%   }\x
%   \def\lwbox{^^A
%     \leavevmode
%     \hbox to \linewidth{^^A
%       \kern-\leftmargin\relax
%       \hss
%       \usebox0
%       \hss
%       \kern-\rightmargin\relax
%     }^^A
%   }^^A
%   \ifdim\wd0>\lw
%     \sbox0{\small\t}^^A
%     \ifdim\wd0>\linewidth
%       \ifdim\wd0>\lw
%         \sbox0{\footnotesize\t}^^A
%         \ifdim\wd0>\linewidth
%           \ifdim\wd0>\lw
%             \sbox0{\scriptsize\t}^^A
%             \ifdim\wd0>\linewidth
%               \ifdim\wd0>\lw
%                 \sbox0{\tiny\t}^^A
%                 \ifdim\wd0>\linewidth
%                   \lwbox
%                 \else
%                   \usebox0
%                 \fi
%               \else
%                 \lwbox
%               \fi
%             \else
%               \usebox0
%             \fi
%           \else
%             \lwbox
%           \fi
%         \else
%           \usebox0
%         \fi
%       \else
%         \lwbox
%       \fi
%     \else
%       \usebox0
%     \fi
%   \else
%     \lwbox
%   \fi
% \else
%   \usebox0
% \fi
% \end{quote}
% If you have a \xfile{docstrip.cfg} that configures and enables \docstrip's
% TDS installing feature, then some files can already be in the right
% place, see the documentation of \docstrip.
%
% \subsection{Refresh file name databases}
%
% If your \TeX~distribution
% (\teTeX, \mikTeX, \dots) relies on file name databases, you must refresh
% these. For example, \teTeX\ users run \verb|texhash| or
% \verb|mktexlsr|.
%
% \subsection{Some details for the interested}
%
% \paragraph{Attached source.}
%
% The PDF documentation on CTAN also includes the
% \xfile{.dtx} source file. It can be extracted by
% AcrobatReader 6 or higher. Another option is \textsf{pdftk},
% e.g. unpack the file into the current directory:
% \begin{quote}
%   \verb|pdftk hologo.pdf unpack_files output .|
% \end{quote}
%
% \paragraph{Unpacking with \LaTeX.}
% The \xfile{.dtx} chooses its action depending on the format:
% \begin{description}
% \item[\plainTeX:] Run \docstrip\ and extract the files.
% \item[\LaTeX:] Generate the documentation.
% \end{description}
% If you insist on using \LaTeX\ for \docstrip\ (really,
% \docstrip\ does not need \LaTeX), then inform the autodetect routine
% about your intention:
% \begin{quote}
%   \verb|latex \let\install=y% \iffalse meta-comment
%
% File: hologo.dtx
% Version: 2016/05/12 v1.11
% Info: A logo collection with bookmark support
%
% Copyright (C) 2010-2012 by
%    Heiko Oberdiek <heiko.oberdiek at googlemail.com>
%
% This work may be distributed and/or modified under the
% conditions of the LaTeX Project Public License, either
% version 1.3c of this license or (at your option) any later
% version. This version of this license is in
%    http://www.latex-project.org/lppl/lppl-1-3c.txt
% and the latest version of this license is in
%    http://www.latex-project.org/lppl.txt
% and version 1.3 or later is part of all distributions of
% LaTeX version 2005/12/01 or later.
%
% This work has the LPPL maintenance status "maintained".
%
% This Current Maintainer of this work is Heiko Oberdiek.
%
% The Base Interpreter refers to any `TeX-Format',
% because some files are installed in TDS:tex/generic//.
%
% This work consists of the main source file hologo.dtx
% and the derived files
%    hologo.sty, hologo.pdf, hologo.ins, hologo.drv, hologo-example.tex,
%    hologo-test1.tex, hologo-test-spacefactor.tex,
%    hologo-test-list.tex.
%
% Distribution:
%    CTAN:macros/latex/contrib/oberdiek/hologo.dtx
%    CTAN:macros/latex/contrib/oberdiek/hologo.pdf
%
% Unpacking:
%    (a) If hologo.ins is present:
%           tex hologo.ins
%    (b) Without hologo.ins:
%           tex hologo.dtx
%    (c) If you insist on using LaTeX
%           latex \let\install=y\input{hologo.dtx}
%        (quote the arguments according to the demands of your shell)
%
% Documentation:
%    (a) If hologo.drv is present:
%           latex hologo.drv
%    (b) Without hologo.drv:
%           latex hologo.dtx; ...
%    The class ltxdoc loads the configuration file ltxdoc.cfg
%    if available. Here you can specify further options, e.g.
%    use A4 as paper format:
%       \PassOptionsToClass{a4paper}{article}
%
%    Programm calls to get the documentation (example):
%       pdflatex hologo.dtx
%       makeindex -s gind.ist hologo.idx
%       pdflatex hologo.dtx
%       makeindex -s gind.ist hologo.idx
%       pdflatex hologo.dtx
%
% Installation:
%    TDS:tex/generic/oberdiek/hologo.sty
%    TDS:doc/latex/oberdiek/hologo.pdf
%    TDS:doc/latex/oberdiek/example/hologo-example.tex
%    TDS:doc/latex/oberdiek/test/hologo-test1.tex
%    TDS:doc/latex/oberdiek/test/hologo-test-spacefactor.tex
%    TDS:doc/latex/oberdiek/test/hologo-test-list.tex
%    TDS:source/latex/oberdiek/hologo.dtx
%
%<*ignore>
\begingroup
  \catcode123=1 %
  \catcode125=2 %
  \def\x{LaTeX2e}%
\expandafter\endgroup
\ifcase 0\ifx\install y1\fi\expandafter
         \ifx\csname processbatchFile\endcsname\relax\else1\fi
         \ifx\fmtname\x\else 1\fi\relax
\else\csname fi\endcsname
%</ignore>
%<*install>
\input docstrip.tex
\Msg{************************************************************************}
\Msg{* Installation}
\Msg{* Package: hologo 2016/05/12 v1.11 A logo collection with bookmark support (HO)}
\Msg{************************************************************************}

\keepsilent
\askforoverwritefalse

\let\MetaPrefix\relax
\preamble

This is a generated file.

Project: hologo
Version: 2016/05/12 v1.11

Copyright (C) 2010-2012 by
   Heiko Oberdiek <heiko.oberdiek at googlemail.com>

This work may be distributed and/or modified under the
conditions of the LaTeX Project Public License, either
version 1.3c of this license or (at your option) any later
version. This version of this license is in
   http://www.latex-project.org/lppl/lppl-1-3c.txt
and the latest version of this license is in
   http://www.latex-project.org/lppl.txt
and version 1.3 or later is part of all distributions of
LaTeX version 2005/12/01 or later.

This work has the LPPL maintenance status "maintained".

This Current Maintainer of this work is Heiko Oberdiek.

The Base Interpreter refers to any `TeX-Format',
because some files are installed in TDS:tex/generic//.

This work consists of the main source file hologo.dtx
and the derived files
   hologo.sty, hologo.pdf, hologo.ins, hologo.drv, hologo-example.tex,
   hologo-test1.tex, hologo-test-spacefactor.tex,
   hologo-test-list.tex.

\endpreamble
\let\MetaPrefix\DoubleperCent

\generate{%
  \file{hologo.ins}{\from{hologo.dtx}{install}}%
  \file{hologo.drv}{\from{hologo.dtx}{driver}}%
  \usedir{tex/generic/oberdiek}%
  \file{hologo.sty}{\from{hologo.dtx}{package}}%
  \usedir{doc/latex/oberdiek/example}%
  \file{hologo-example.tex}{\from{hologo.dtx}{example}}%
  \usedir{doc/latex/oberdiek/test}%
  \file{hologo-test1.tex}{\from{hologo.dtx}{test1}}%
  \file{hologo-test-spacefactor.tex}{\from{hologo.dtx}{test-spacefactor}}%
  \file{hologo-test-list.tex}{\from{hologo.dtx}{test-list}}%
  \nopreamble
  \nopostamble
  \usedir{source/latex/oberdiek/catalogue}%
  \file{hologo.xml}{\from{hologo.dtx}{catalogue}}%
}

\catcode32=13\relax% active space
\let =\space%
\Msg{************************************************************************}
\Msg{*}
\Msg{* To finish the installation you have to move the following}
\Msg{* file into a directory searched by TeX:}
\Msg{*}
\Msg{*     hologo.sty}
\Msg{*}
\Msg{* To produce the documentation run the file `hologo.drv'}
\Msg{* through LaTeX.}
\Msg{*}
\Msg{* Happy TeXing!}
\Msg{*}
\Msg{************************************************************************}

\endbatchfile
%</install>
%<*ignore>
\fi
%</ignore>
%<*driver>
\NeedsTeXFormat{LaTeX2e}
\ProvidesFile{hologo.drv}%
  [2016/05/12 v1.11 A logo collection with bookmark support (HO)]%
\documentclass{ltxdoc}
\usepackage{holtxdoc}[2011/11/22]
\usepackage{hologo}[2016/05/12]
\usepackage{longtable}
\usepackage{array}
\usepackage{paralist}
%\usepackage[T1]{fontenc}
%\usepackage{lmodern}
\begin{document}
  \DocInput{hologo.dtx}%
\end{document}
%</driver>
% \fi
%
%
% \CharacterTable
%  {Upper-case    \A\B\C\D\E\F\G\H\I\J\K\L\M\N\O\P\Q\R\S\T\U\V\W\X\Y\Z
%   Lower-case    \a\b\c\d\e\f\g\h\i\j\k\l\m\n\o\p\q\r\s\t\u\v\w\x\y\z
%   Digits        \0\1\2\3\4\5\6\7\8\9
%   Exclamation   \!     Double quote  \"     Hash (number) \#
%   Dollar        \$     Percent       \%     Ampersand     \&
%   Acute accent  \'     Left paren    \(     Right paren   \)
%   Asterisk      \*     Plus          \+     Comma         \,
%   Minus         \-     Point         \.     Solidus       \/
%   Colon         \:     Semicolon     \;     Less than     \<
%   Equals        \=     Greater than  \>     Question mark \?
%   Commercial at \@     Left bracket  \[     Backslash     \\
%   Right bracket \]     Circumflex    \^     Underscore    \_
%   Grave accent  \`     Left brace    \{     Vertical bar  \|
%   Right brace   \}     Tilde         \~}
%
% \GetFileInfo{hologo.drv}
%
% \title{The \xpackage{hologo} package}
% \date{2016/05/12 v1.11}
% \author{Heiko Oberdiek\\\xemail{heiko.oberdiek at googlemail.com}}
%
% \maketitle
%
% \begin{abstract}
% This package starts a collection of logos with support for bookmarks
% strings.
% \end{abstract}
%
% \tableofcontents
%
% \section{Documentation}
%
% \subsection{Logo macros}
%
% \begin{declcs}{hologo} \M{name}
% \end{declcs}
% Macro \cs{hologo} sets the logo with name \meta{name}.
% The following table shows the supported names.
%
% \begingroup
%   \def\hologoEntry#1#2#3{^^A
%     #1&#2&\hologoLogoSetup{#1}{variant=#2}\hologo{#1}&#3\tabularnewline
%   }
%   \begin{longtable}{>{\ttfamily}l>{\ttfamily}lll}
%     \rmfamily\bfseries{name} & \rmfamily\bfseries variant
%     & \bfseries logo & \bfseries since\\
%     \hline
%     \endhead
%     \hologoList
%   \end{longtable}
% \endgroup
%
% \begin{declcs}{Hologo} \M{name}
% \end{declcs}
% Macro \cs{Hologo} starts the logo \meta{name} with an uppercase
% letter. As an exception small greek letters are not converted
% to uppercase. Examples, see \hologo{eTeX} and \hologo{ExTeX}.
%
% \subsection{Setup macros}
%
% The package does not support package options, but the following
% setup macros can be used to set options.
%
% \begin{declcs}{hologoSetup} \M{key value list}
% \end{declcs}
% Macro \cs{hologoSetup} sets global options.
%
% \begin{declcs}{hologoLogoSetup} \M{logo} \M{key value list}
% \end{declcs}
% Some options can also be used to configure a logo.
% These settings take precedence over global option settings.
%
% \subsection{Options}\label{sec:options}
%
% There are boolean and string options:
% \begin{description}
% \item[Boolean option:]
% It takes |true| or |false|
% as value. If the value is omitted, then |true| is used.
% \item[String option:]
% A value must be given as string. (But the string might be empty.)
% \end{description}
% The following options can be used both in \cs{hologoSetup}
% and \cs{hologoLogoSetup}:
% \begin{description}
% \def\entry#1{\item[\xoption{#1}:]}
% \entry{break}
%   enables or disables line breaks inside the logo. This setting is
%   refined by options \xoption{hyphenbreak}, \xoption{spacebreak}
%   or \xoption{discretionarybreak}.
%   Default is |false|.
% \entry{hyphenbreak}
%   enables or disables the line break right after the hyphen character.
% \entry{spacebreak}
%   enables or disables line breaks at space characters.
% \entry{discretionarybreak}
%   enables or disables line breaks at hyphenation points
%   (inserted by \cs{-}).
% \end{description}
% Macro \cs{hologoLogoSetup} also knows:
% \begin{description}
% \item[\xoption{variant}:]
%   This is a string option. It specifies a variant of a logo that
%   must exist. An empty string selects the package default variant.
% \end{description}
% Example:
% \begin{quote}
%   |\hologoSetup{break=false}|\\
%   |\hologoLogoSetup{plainTeX}{variant=hyphen,hyphenbreak}|\\
%   Then ``plain-\TeX'' contains one break point after the hyphen.
% \end{quote}
%
% \subsection{Driver options}
%
% Sometimes graphical operations are needed to construct some
% glyphs (e.g.\ \hologo{XeTeX}). If package \xpackage{graphics}
% or package \xpackage{pgf} are found, then the macros are taken
% from there. Otherwise the packge defines its own operations
% and therefore needs the driver information. Many drivers are
% detected automatically (\hologo{pdfTeX}/\hologo{LuaTeX}
% in PDF mode, \hologo{XeTeX}, \hologo{VTeX}). These have precedence
% over a driver option. The driver can be given as package option
% or using \cs{hologoDriverSetup}.
% The following list contains the recognized driver options:
% \begin{itemize}
% \item \xoption{pdftex}, \xoption{luatex}
% \item \xoption{dvipdfm}, \xoption{dvipdfmx}
% \item \xoption{dvips}, \xoption{dvipsone}, \xoption{xdvi}
% \item \xoption{xetex}
% \item \xoption{vtex}
% \end{itemize}
% The left driver of a line is the driver name that is used internally.
% The following names are aliases for drivers that use the
% same method. Therefore the entry in the \xext{log} file for
% the used driver prints the internally used driver name.
% \begin{description}
% \item[\xoption{driverfallback}:]
%   This option expects a driver that is used,
%   if the driver could not be detected automatically.
% \end{description}
%
% \begin{declcs}{hologoDriverSetup} \M{driver option}
% \end{declcs}
% The driver can also be configured after package loading
% using \cs{hologoDriverSetup}, also the way for \hologo{plainTeX}
% to setup the driver.
%
% \subsection{Font setup}
%
% Some logos require a special font, but should also be usable by
% \hologo{plainTeX}. Therefore the package provides some ways
% to influence the font settings. The options below
% take font settings as values. Both font commands
% such as \cs{sffamily} and macros that take one argument
% like \cs{textsf} can be used.
%
% \begin{declcs}{hologoFontSetup} \M{key value list}
% \end{declcs}
% Macro \cs{hologoFontSetup} sets the fonts for all logos.
% Supported keys:
% \begin{description}
% \def\entry#1{\item[\xoption{#1}:]}
% \entry{general}
%   This font is used for all logos. The default is empty.
%   That means no special font is used.
% \entry{bibsf}
%   This font is used for
%   {\hologoLogoSetup{BibTeX}{variant=sf}\hologo{BibTeX}}
%   with variant \xoption{sf}.
% \entry{rm}
%   This font is a serif font. It is used for \hologo{ExTeX}.
% \entry{sc}
%   This font specifies a small caps font. It is used for
%   {\hologoLogoSetup{BibTeX}{variant=sc}\hologo{BibTeX}}
%   with variant \xoption{sc}.
% \entry{sf}
%   This font specifies a sans serif font. The default
%   is \cs{sffamily}, then \cs{sf} is tried. Otherwise
%   a warning is given. It is used by \hologo{KOMAScript}.
% \entry{sy}
%   This is the font for math symbols (e.g. cmsy).
%   It is used by \hologo{AmS}, \hologo{NTS}, \hologo{ExTeX}.
% \entry{logo}
%   \hologo{METAFONT} and \hologo{METAPOST} are using that font.
%   In \hologo{LaTeX} \cs{logofamily} is used and
%   the definitions of package \xpackage{mflogo} are used
%   if the package is not loaded.
%   Otherwise the \cs{tenlogo} is used and defined
%   if it does not already exists.
% \end{description}
%
% \begin{declcs}{hologoLogoFontSetup} \M{logo} \M{key value list}
% \end{declcs}
% Fonts can also be set for a logo or logo component separately,
% see the following list.
% The keys are the same as for \cs{hologoFontSetup}.
%
% \begin{longtable}{>{\ttfamily}l>{\sffamily}ll}
%   \meta{logo} & keys & result\\
%   \hline
%   \endhead
%   BibTeX & bibsf & {\hologoLogoSetup{BibTeX}{variant=sf}\hologo{BibTeX}}\\[.5ex]
%   BibTeX & sc & {\hologoLogoSetup{BibTeX}{variant=sc}\hologo{BibTeX}}\\[.5ex]
%   ExTeX & rm & \hologo{ExTeX}\\
%   SliTeX & rm & \hologo{SliTeX}\\[.5ex]
%   AmS & sy & \hologo{AmS}\\
%   ExTeX & sy & \hologo{ExTeX}\\
%   NTS & sy & \hologo{NTS}\\[.5ex]
%   KOMAScript & sf & \hologo{KOMAScript}\\[.5ex]
%   METAFONT & logo & \hologo{METAFONT}\\
%   METAPOST & logo & \hologo{METAPOST}\\[.5ex]
%   SliTeX & sc \hologo{SliTeX}
% \end{longtable}
%
% \subsubsection{Font order}
%
% For all logos the font \xoption{general} is applied first.
% Example:
%\begin{quote}
%|\hologoFontSetup{general=\color{red}}|
%\end{quote}
% will print red logos.
% Then if the font uses a special font \xoption{sf}, for example,
% the font is applied that is setup by \cs{hologoLogoFontSetup}.
% If this font is not setup, then the common font setup
% by \cs{hologoFontSetup} is used. Otherwise a warning is given,
% that there is no font configured.
%
% \subsection{Additional user macros}
%
% Usually a variant of a logo is configured by using
% \cs{hologoLogoSetup}, because it is bad style to mix
% different variants of the same logo in the same text.
% There the following macros are a convenience for testing.
%
% \begin{declcs}{hologoVariant} \M{name} \M{variant}\\
%   \cs{HologoVariant} \M{name} \M{variant}
% \end{declcs}
% Logo \meta{name} is set using \meta{variant} that specifies
% explicitely which variant of the macro is used. If the argument
% is empty, then the default form of the logo is used
% (configurable by \cs{hologoLogoSetup}).
%
% \cs{HologoVariant} is used if the logo is set in a context
% that needs an uppercase first letter (beginning of a sentence, \dots).
%
% \begin{declcs}{hologoList}\\
%   \cs{hologoEntry} \M{logo} \M{variant} \M{since}
% \end{declcs}
% Macro \cs{hologoList} contains all logos that are provided
% by the package including variants. The list consists of calls
% of \cs{hologoEntry} with three arguments starting with the
% logo name \meta{logo} and its variant \meta{variant}. An empty
% variant means the current default. Argument \meta{since} specifies
% with version of the package \xpackage{hologo} is needed to get
% the logo. If the logo is fixed, then the date gets updated.
% Therefore the date \meta{since} is not exactly the date of
% the first introduction, but rather the date of the latest fix.
%
% Before \cs{hologoList} can be used, macro \cs{hologoEntry} needs
% a definition. The example file in section \ref{sec:example}
% shows applications of \cs{hologoList}.
%
% \subsection{Supported contexts}
%
% Macros \cs{hologo} and friends support special contexts:
% \begin{itemize}
% \item \hologo{LaTeX}'s protection mechanism.
% \item Bookmarks of package \xpackage{hyperref}.
% \item Package \xpackage{tex4ht}.
% \item The macros can be used inside \cs{csname} constructs,
%   if \cs{ifincsname} is available (\hologo{pdfTeX}, \hologo{XeTeX},
%   \hologo{LuaTeX}).
% \end{itemize}
%
% \subsection{Example}
% \label{sec:example}
%
% The following example prints the logos in different fonts.
%    \begin{macrocode}
%<*example>
%<<verbatim
\NeedsTeXFormat{LaTeX2e}
\documentclass[a4paper]{article}
\usepackage[
  hmargin=20mm,
  vmargin=20mm,
]{geometry}
\pagestyle{empty}
\usepackage{hologo}[2016/05/12]
\usepackage{longtable}
\usepackage{array}
\setlength{\extrarowheight}{2pt}
\usepackage[T1]{fontenc}
\usepackage{lmodern}
\usepackage{pdflscape}
\usepackage[
  pdfencoding=auto,
]{hyperref}
\hypersetup{
  pdfauthor={Heiko Oberdiek},
  pdftitle={Example for package `hologo'},
  pdfsubject={Logos with fonts lmr, lmss, qtm, qpl, qhv},
}
\usepackage{bookmark}

% Print the logo list on the console

\begingroup
  \typeout{}%
  \typeout{*** Begin of logo list ***}%
  \newcommand*{\hologoEntry}[3]{%
    \typeout{#1 \ifx\\#2\\\else(#2) \fi[#3]}%
  }%
  \hologoList
  \typeout{*** End of logo list ***}%
  \typeout{}%
\endgroup

\begin{document}
\begin{landscape}

  \section{Example file for package `hologo'}

  % Table for font names

  \begin{longtable}{>{\bfseries}ll}
    \textbf{font} & \textbf{Font name}\\
    \hline
    lmr & Latin Modern Roman\\
    lmss & Latin Modern Sans\\
    qtm & \TeX\ Gyre Termes\\
    qhv & \TeX\ Gyre Heros\\
    qpl & \TeX\ Gyre Pagella\\
  \end{longtable}

  % Logo list with logos in different fonts

  \begingroup
    \newcommand*{\SetVariant}[2]{%
      \ifx\\#2\\%
      \else
        \hologoLogoSetup{#1}{variant=#2}%
      \fi
    }%
    \newcommand*{\hologoEntry}[3]{%
      \SetVariant{#1}{#2}%
      \raisebox{1em}[0pt][0pt]{\hypertarget{#1@#2}{}}%
      \bookmark[%
        dest={#1@#2},%
      ]{%
        #1\ifx\\#2\\\else\space(#2)\fi: \Hologo{#1}, \hologo{#1} %
        [Unicode]%
      }%
      \hypersetup{unicode=false}%
      \bookmark[%
        dest={#1@#2},%
      ]{%
        #1\ifx\\#2\\\else\space(#2)\fi: \Hologo{#1}, \hologo{#1} %
        [PDFDocEncoding]%
      }%
      \texttt{#1}%
      &%
      \texttt{#2}%
      &%
      \Hologo{#1}%
      &%
      \SetVariant{#1}{#2}%
      \hologo{#1}%
      &%
      \SetVariant{#1}{#2}%
      \fontfamily{qtm}\selectfont
      \hologo{#1}%
      &%
      \SetVariant{#1}{#2}%
      \fontfamily{qpl}\selectfont
      \hologo{#1}%
      &%
      \SetVariant{#1}{#2}%
      \textsf{\hologo{#1}}%
      &%
      \SetVariant{#1}{#2}%
      \fontfamily{qhv}\selectfont
      \hologo{#1}%
      \tabularnewline
    }%
    \begin{longtable}{llllllll}%
      \textbf{\textit{logo}} & \textbf{\textit{variant}} &
      \texttt{\string\Hologo} &
      \textbf{lmr} & \textbf{qtm} & \textbf{qpl} &
      \textbf{lmss} & \textbf{qhv}
      \tabularnewline
      \hline
      \endhead
      \hologoList
    \end{longtable}%
  \endgroup

\end{landscape}
\end{document}
%verbatim
%</example>
%    \end{macrocode}
%
% \StopEventually{
% }
%
% \section{Implementation}
%    \begin{macrocode}
%<*package>
%    \end{macrocode}
%    Reload check, especially if the package is not used with \LaTeX.
%    \begin{macrocode}
\begingroup\catcode61\catcode48\catcode32=10\relax%
  \catcode13=5 % ^^M
  \endlinechar=13 %
  \catcode35=6 % #
  \catcode39=12 % '
  \catcode44=12 % ,
  \catcode45=12 % -
  \catcode46=12 % .
  \catcode58=12 % :
  \catcode64=11 % @
  \catcode123=1 % {
  \catcode125=2 % }
  \expandafter\let\expandafter\x\csname ver@hologo.sty\endcsname
  \ifx\x\relax % plain-TeX, first loading
  \else
    \def\empty{}%
    \ifx\x\empty % LaTeX, first loading,
      % variable is initialized, but \ProvidesPackage not yet seen
    \else
      \expandafter\ifx\csname PackageInfo\endcsname\relax
        \def\x#1#2{%
          \immediate\write-1{Package #1 Info: #2.}%
        }%
      \else
        \def\x#1#2{\PackageInfo{#1}{#2, stopped}}%
      \fi
      \x{hologo}{The package is already loaded}%
      \aftergroup\endinput
    \fi
  \fi
\endgroup%
%    \end{macrocode}
%    Package identification:
%    \begin{macrocode}
\begingroup\catcode61\catcode48\catcode32=10\relax%
  \catcode13=5 % ^^M
  \endlinechar=13 %
  \catcode35=6 % #
  \catcode39=12 % '
  \catcode40=12 % (
  \catcode41=12 % )
  \catcode44=12 % ,
  \catcode45=12 % -
  \catcode46=12 % .
  \catcode47=12 % /
  \catcode58=12 % :
  \catcode64=11 % @
  \catcode91=12 % [
  \catcode93=12 % ]
  \catcode123=1 % {
  \catcode125=2 % }
  \expandafter\ifx\csname ProvidesPackage\endcsname\relax
    \def\x#1#2#3[#4]{\endgroup
      \immediate\write-1{Package: #3 #4}%
      \xdef#1{#4}%
    }%
  \else
    \def\x#1#2[#3]{\endgroup
      #2[{#3}]%
      \ifx#1\@undefined
        \xdef#1{#3}%
      \fi
      \ifx#1\relax
        \xdef#1{#3}%
      \fi
    }%
  \fi
\expandafter\x\csname ver@hologo.sty\endcsname
\ProvidesPackage{hologo}%
  [2016/05/12 v1.11 A logo collection with bookmark support (HO)]%
%    \end{macrocode}
%
%    \begin{macrocode}
\begingroup\catcode61\catcode48\catcode32=10\relax%
  \catcode13=5 % ^^M
  \endlinechar=13 %
  \catcode123=1 % {
  \catcode125=2 % }
  \catcode64=11 % @
  \def\x{\endgroup
    \expandafter\edef\csname HOLOGO@AtEnd\endcsname{%
      \endlinechar=\the\endlinechar\relax
      \catcode13=\the\catcode13\relax
      \catcode32=\the\catcode32\relax
      \catcode35=\the\catcode35\relax
      \catcode61=\the\catcode61\relax
      \catcode64=\the\catcode64\relax
      \catcode123=\the\catcode123\relax
      \catcode125=\the\catcode125\relax
    }%
  }%
\x\catcode61\catcode48\catcode32=10\relax%
\catcode13=5 % ^^M
\endlinechar=13 %
\catcode35=6 % #
\catcode64=11 % @
\catcode123=1 % {
\catcode125=2 % }
\def\TMP@EnsureCode#1#2{%
  \edef\HOLOGO@AtEnd{%
    \HOLOGO@AtEnd
    \catcode#1=\the\catcode#1\relax
  }%
  \catcode#1=#2\relax
}
\TMP@EnsureCode{10}{12}% ^^J
\TMP@EnsureCode{33}{12}% !
\TMP@EnsureCode{34}{12}% "
\TMP@EnsureCode{36}{3}% $
\TMP@EnsureCode{38}{4}% &
\TMP@EnsureCode{39}{12}% '
\TMP@EnsureCode{40}{12}% (
\TMP@EnsureCode{41}{12}% )
\TMP@EnsureCode{42}{12}% *
\TMP@EnsureCode{43}{12}% +
\TMP@EnsureCode{44}{12}% ,
\TMP@EnsureCode{45}{12}% -
\TMP@EnsureCode{46}{12}% .
\TMP@EnsureCode{47}{12}% /
\TMP@EnsureCode{58}{12}% :
\TMP@EnsureCode{59}{12}% ;
\TMP@EnsureCode{60}{12}% <
\TMP@EnsureCode{62}{12}% >
\TMP@EnsureCode{63}{12}% ?
\TMP@EnsureCode{91}{12}% [
\TMP@EnsureCode{93}{12}% ]
\TMP@EnsureCode{94}{7}% ^ (superscript)
\TMP@EnsureCode{95}{8}% _ (subscript)
\TMP@EnsureCode{96}{12}% `
\TMP@EnsureCode{124}{12}% |
\edef\HOLOGO@AtEnd{%
  \HOLOGO@AtEnd
  \escapechar\the\escapechar\relax
  \noexpand\endinput
}
\escapechar=92 %
%    \end{macrocode}
%
% \subsection{Logo list}
%
%    \begin{macro}{\hologoList}
%    \begin{macrocode}
\def\hologoList{%
  \hologoEntry{(La)TeX}{}{2011/10/01}%
  \hologoEntry{AmSLaTeX}{}{2010/04/16}%
  \hologoEntry{AmSTeX}{}{2010/04/16}%
  \hologoEntry{biber}{}{2011/10/01}%
  \hologoEntry{BibTeX}{}{2011/10/01}%
  \hologoEntry{BibTeX}{sf}{2011/10/01}%
  \hologoEntry{BibTeX}{sc}{2011/10/01}%
  \hologoEntry{BibTeX8}{}{2011/11/22}%
  \hologoEntry{ConTeXt}{}{2011/03/25}%
  \hologoEntry{ConTeXt}{narrow}{2011/03/25}%
  \hologoEntry{ConTeXt}{simple}{2011/03/25}%
  \hologoEntry{emTeX}{}{2010/04/26}%
  \hologoEntry{eTeX}{}{2010/04/08}%
  \hologoEntry{ExTeX}{}{2011/10/01}%
  \hologoEntry{HanTheThanh}{}{2011/11/29}%
  \hologoEntry{iniTeX}{}{2011/10/01}%
  \hologoEntry{KOMAScript}{}{2011/10/01}%
  \hologoEntry{La}{}{2010/05/08}%
  \hologoEntry{LaTeX}{}{2010/04/08}%
  \hologoEntry{LaTeX2e}{}{2010/04/08}%
  \hologoEntry{LaTeX3}{}{2010/04/24}%
  \hologoEntry{LaTeXe}{}{2010/04/08}%
  \hologoEntry{LaTeXML}{}{2011/11/22}%
  \hologoEntry{LaTeXTeX}{}{2011/10/01}%
  \hologoEntry{LuaLaTeX}{}{2010/04/08}%
  \hologoEntry{LuaTeX}{}{2010/04/08}%
  \hologoEntry{LyX}{}{2011/10/01}%
  \hologoEntry{METAFONT}{}{2011/10/01}%
  \hologoEntry{MetaFun}{}{2011/10/01}%
  \hologoEntry{METAPOST}{}{2011/10/01}%
  \hologoEntry{MetaPost}{}{2011/10/01}%
  \hologoEntry{MiKTeX}{}{2011/10/01}%
  \hologoEntry{NTS}{}{2011/10/01}%
  \hologoEntry{OzMF}{}{2011/10/01}%
  \hologoEntry{OzMP}{}{2011/10/01}%
  \hologoEntry{OzTeX}{}{2011/10/01}%
  \hologoEntry{OzTtH}{}{2011/10/01}%
  \hologoEntry{PCTeX}{}{2011/10/01}%
  \hologoEntry{pdfTeX}{}{2011/10/01}%
  \hologoEntry{pdfLaTeX}{}{2011/10/01}%
  \hologoEntry{PiC}{}{2011/10/01}%
  \hologoEntry{PiCTeX}{}{2011/10/01}%
  \hologoEntry{plainTeX}{}{2010/04/08}%
  \hologoEntry{plainTeX}{space}{2010/04/16}%
  \hologoEntry{plainTeX}{hyphen}{2010/04/16}%
  \hologoEntry{plainTeX}{runtogether}{2010/04/16}%
  \hologoEntry{SageTeX}{}{2011/11/22}%
  \hologoEntry{SLiTeX}{}{2011/10/01}%
  \hologoEntry{SLiTeX}{lift}{2011/10/01}%
  \hologoEntry{SLiTeX}{narrow}{2011/10/01}%
  \hologoEntry{SLiTeX}{simple}{2011/10/01}%
  \hologoEntry{SliTeX}{}{2011/10/01}%
  \hologoEntry{SliTeX}{narrow}{2011/10/01}%
  \hologoEntry{SliTeX}{simple}{2011/10/01}%
  \hologoEntry{SliTeX}{lift}{2011/10/01}%
  \hologoEntry{teTeX}{}{2011/10/01}%
  \hologoEntry{TeX}{}{2010/04/08}%
  \hologoEntry{TeX4ht}{}{2011/11/22}%
  \hologoEntry{TTH}{}{2011/11/22}%
  \hologoEntry{virTeX}{}{2011/10/01}%
  \hologoEntry{VTeX}{}{2010/04/24}%
  \hologoEntry{Xe}{}{2010/04/08}%
  \hologoEntry{XeLaTeX}{}{2010/04/08}%
  \hologoEntry{XeTeX}{}{2010/04/08}%
}
%    \end{macrocode}
%    \end{macro}
%
% \subsection{Load resources}
%
%    \begin{macrocode}
\begingroup\expandafter\expandafter\expandafter\endgroup
\expandafter\ifx\csname RequirePackage\endcsname\relax
  \def\TMP@RequirePackage#1[#2]{%
    \begingroup\expandafter\expandafter\expandafter\endgroup
    \expandafter\ifx\csname ver@#1.sty\endcsname\relax
      \input #1.sty\relax
    \fi
  }%
  \TMP@RequirePackage{ltxcmds}[2011/02/04]%
  \TMP@RequirePackage{infwarerr}[2010/04/08]%
  \TMP@RequirePackage{kvsetkeys}[2010/03/01]%
  \TMP@RequirePackage{kvdefinekeys}[2010/03/01]%
  \TMP@RequirePackage{pdftexcmds}[2010/04/01]%
  \TMP@RequirePackage{ifpdf}[2010/01/28]%
  \TMP@RequirePackage{ifluatex}[2010/03/01]%
  \ltx@IfUndefined{newif}{%
    \expandafter\let\csname newif\endcsname\ltx@newif
  }{}%
  \TMP@RequirePackage{ifxetex}[2009/01/23]%
  \TMP@RequirePackage{ifvtex}[2010/03/01]%
\else
  \RequirePackage{ltxcmds}[2011/02/04]%
  \RequirePackage{infwarerr}[2010/04/08]%
  \RequirePackage{kvsetkeys}[2010/03/01]%
  \RequirePackage{kvdefinekeys}[2010/03/01]%
  \RequirePackage{pdftexcmds}[2010/04/01]%
  \RequirePackage{ifpdf}[2010/01/28]%
  \RequirePackage{ifluatex}[2010/03/01]%
  \RequirePackage{ifxetex}[2009/01/23]%
  \RequirePackage{ifvtex}[2010/03/01]%
\fi
%    \end{macrocode}
%
%    \begin{macro}{\HOLOGO@IfDefined}
%    \begin{macrocode}
\def\HOLOGO@IfExists#1{%
  \ifx\@undefined#1%
    \expandafter\ltx@secondoftwo
  \else
    \ifx\relax#1%
      \expandafter\ltx@secondoftwo
    \else
      \expandafter\expandafter\expandafter\ltx@firstoftwo
    \fi
  \fi
}
%    \end{macrocode}
%    \end{macro}
%
% \subsection{Setup macros}
%
%    \begin{macro}{\hologoSetup}
%    \begin{macrocode}
\def\hologoSetup{%
  \let\HOLOGO@name\relax
  \HOLOGO@Setup
}
%    \end{macrocode}
%    \end{macro}
%
%    \begin{macro}{\hologoLogoSetup}
%    \begin{macrocode}
\def\hologoLogoSetup#1{%
  \edef\HOLOGO@name{#1}%
  \ltx@IfUndefined{HoLogo@\HOLOGO@name}{%
    \@PackageError{hologo}{%
      Unknown logo `\HOLOGO@name'%
    }\@ehc
    \ltx@gobble
  }{%
    \HOLOGO@Setup
  }%
}
%    \end{macrocode}
%    \end{macro}
%
%    \begin{macro}{\HOLOGO@Setup}
%    \begin{macrocode}
\def\HOLOGO@Setup{%
  \kvsetkeys{HoLogo}%
}
%    \end{macrocode}
%    \end{macro}
%
% \subsection{Options}
%
%    \begin{macro}{\HOLOGO@DeclareBoolOption}
%    \begin{macrocode}
\def\HOLOGO@DeclareBoolOption#1{%
  \expandafter\chardef\csname HOLOGOOPT@#1\endcsname\ltx@zero
  \kv@define@key{HoLogo}{#1}[true]{%
    \def\HOLOGO@temp{##1}%
    \ifx\HOLOGO@temp\HOLOGO@true
      \ifx\HOLOGO@name\relax
        \expandafter\chardef\csname HOLOGOOPT@#1\endcsname=\ltx@one
      \else
        \expandafter\chardef\csname
        HoLogoOpt@#1@\HOLOGO@name\endcsname\ltx@one
      \fi
      \HOLOGO@SetBreakAll{#1}%
    \else
      \ifx\HOLOGO@temp\HOLOGO@false
        \ifx\HOLOGO@name\relax
          \expandafter\chardef\csname HOLOGOOPT@#1\endcsname=\ltx@zero
        \else
          \expandafter\chardef\csname
          HoLogoOpt@#1@\HOLOGO@name\endcsname=\ltx@zero
        \fi
        \HOLOGO@SetBreakAll{#1}%
      \else
        \@PackageError{hologo}{%
          Unknown value `##1' for boolean option `#1'.\MessageBreak
          Known values are `true' and `false'%
        }\@ehc
      \fi
    \fi
  }%
}
%    \end{macrocode}
%    \end{macro}
%
%    \begin{macro}{\HOLOGO@SetBreakAll}
%    \begin{macrocode}
\def\HOLOGO@SetBreakAll#1{%
  \def\HOLOGO@temp{#1}%
  \ifx\HOLOGO@temp\HOLOGO@break
    \ifx\HOLOGO@name\relax
      \chardef\HOLOGOOPT@hyphenbreak=\HOLOGOOPT@break
      \chardef\HOLOGOOPT@spacebreak=\HOLOGOOPT@break
      \chardef\HOLOGOOPT@discretionarybreak=\HOLOGOOPT@break
    \else
      \expandafter\chardef
         \csname HoLogoOpt@hyphenbreak@\HOLOGO@name\endcsname=%
         \csname HoLogoOpt@break@\HOLOGO@name\endcsname
      \expandafter\chardef
         \csname HoLogoOpt@spacebreak@\HOLOGO@name\endcsname=%
         \csname HoLogoOpt@break@\HOLOGO@name\endcsname
      \expandafter\chardef
         \csname HoLogoOpt@discretionarybreak@\HOLOGO@name
             \endcsname=%
         \csname HoLogoOpt@break@\HOLOGO@name\endcsname
    \fi
  \fi
}
%    \end{macrocode}
%    \end{macro}
%
%    \begin{macro}{\HOLOGO@true}
%    \begin{macrocode}
\def\HOLOGO@true{true}
%    \end{macrocode}
%    \end{macro}
%    \begin{macro}{\HOLOGO@false}
%    \begin{macrocode}
\def\HOLOGO@false{false}
%    \end{macrocode}
%    \end{macro}
%    \begin{macro}{\HOLOGO@break}
%    \begin{macrocode}
\def\HOLOGO@break{break}
%    \end{macrocode}
%    \end{macro}
%
%    \begin{macrocode}
\HOLOGO@DeclareBoolOption{break}
\HOLOGO@DeclareBoolOption{hyphenbreak}
\HOLOGO@DeclareBoolOption{spacebreak}
\HOLOGO@DeclareBoolOption{discretionarybreak}
%    \end{macrocode}
%
%    \begin{macrocode}
\kv@define@key{HoLogo}{variant}{%
  \ifx\HOLOGO@name\relax
    \@PackageError{hologo}{%
      Option `variant' is not available in \string\hologoSetup,%
      \MessageBreak
      Use \string\hologoLogoSetup\space instead%
    }\@ehc
  \else
    \edef\HOLOGO@temp{#1}%
    \ifx\HOLOGO@temp\ltx@empty
      \expandafter
      \let\csname HoLogoOpt@variant@\HOLOGO@name\endcsname\@undefined
    \else
      \ltx@IfUndefined{HoLogo@\HOLOGO@name @\HOLOGO@temp}{%
        \@PackageError{hologo}{%
          Unknown variant `\HOLOGO@temp' of logo `\HOLOGO@name'%
        }\@ehc
      }{%
        \expandafter
        \let\csname HoLogoOpt@variant@\HOLOGO@name\endcsname
            \HOLOGO@temp
      }%
    \fi
  \fi
}
%    \end{macrocode}
%
%    \begin{macro}{\HOLOGO@Variant}
%    \begin{macrocode}
\def\HOLOGO@Variant#1{%
  #1%
  \ltx@ifundefined{HoLogoOpt@variant@#1}{%
  }{%
    @\csname HoLogoOpt@variant@#1\endcsname
  }%
}
%    \end{macrocode}
%    \end{macro}
%
% \subsection{Break/no-break support}
%
%    \begin{macro}{\HOLOGO@space}
%    \begin{macrocode}
\def\HOLOGO@space{%
  \ltx@ifundefined{HoLogoOpt@spacebreak@\HOLOGO@name}{%
    \ltx@ifundefined{HoLogoOpt@break@\HOLOGO@name}{%
      \chardef\HOLOGO@temp=\HOLOGOOPT@spacebreak
    }{%
      \chardef\HOLOGO@temp=%
        \csname HoLogoOpt@break@\HOLOGO@name\endcsname
    }%
  }{%
    \chardef\HOLOGO@temp=%
      \csname HoLogoOpt@spacebreak@\HOLOGO@name\endcsname
  }%
  \ifcase\HOLOGO@temp
    \penalty10000 %
  \fi
  \ltx@space
}
%    \end{macrocode}
%    \end{macro}
%
%    \begin{macro}{\HOLOGO@hyphen}
%    \begin{macrocode}
\def\HOLOGO@hyphen{%
  \ltx@ifundefined{HoLogoOpt@hyphenbreak@\HOLOGO@name}{%
    \ltx@ifundefined{HoLogoOpt@break@\HOLOGO@name}{%
      \chardef\HOLOGO@temp=\HOLOGOOPT@hyphenbreak
    }{%
      \chardef\HOLOGO@temp=%
        \csname HoLogoOpt@break@\HOLOGO@name\endcsname
    }%
  }{%
    \chardef\HOLOGO@temp=%
      \csname HoLogoOpt@hyphenbreak@\HOLOGO@name\endcsname
  }%
  \ifcase\HOLOGO@temp
    \ltx@mbox{-}%
  \else
    -%
  \fi
}
%    \end{macrocode}
%    \end{macro}
%
%    \begin{macro}{\HOLOGO@discretionary}
%    \begin{macrocode}
\def\HOLOGO@discretionary{%
  \ltx@ifundefined{HoLogoOpt@discretionarybreak@\HOLOGO@name}{%
    \ltx@ifundefined{HoLogoOpt@break@\HOLOGO@name}{%
      \chardef\HOLOGO@temp=\HOLOGOOPT@discretionarybreak
    }{%
      \chardef\HOLOGO@temp=%
        \csname HoLogoOpt@break@\HOLOGO@name\endcsname
    }%
  }{%
    \chardef\HOLOGO@temp=%
      \csname HoLogoOpt@discretionarybreak@\HOLOGO@name\endcsname
  }%
  \ifcase\HOLOGO@temp
  \else
    \-%
  \fi
}
%    \end{macrocode}
%    \end{macro}
%
%    \begin{macro}{\HOLOGO@mbox}
%    \begin{macrocode}
\def\HOLOGO@mbox#1{%
  \ltx@ifundefined{HoLogoOpt@break@\HOLOGO@name}{%
    \chardef\HOLOGO@temp=\HOLOGOOPT@hyphenbreak
  }{%
    \chardef\HOLOGO@temp=%
      \csname HoLogoOpt@break@\HOLOGO@name\endcsname
  }%
  \ifcase\HOLOGO@temp
    \ltx@mbox{#1}%
  \else
    #1%
  \fi
}
%    \end{macrocode}
%    \end{macro}
%
% \subsection{Font support}
%
%    \begin{macro}{\HoLogoFont@font}
%    \begin{tabular}{@{}ll@{}}
%    |#1|:& logo name\\
%    |#2|:& font short name\\
%    |#3|:& text
%    \end{tabular}
%    \begin{macrocode}
\def\HoLogoFont@font#1#2#3{%
  \begingroup
    \ltx@IfUndefined{HoLogoFont@logo@#1.#2}{%
      \ltx@IfUndefined{HoLogoFont@font@#2}{%
        \@PackageWarning{hologo}{%
          Missing font `#2' for logo `#1'%
        }%
        #3%
      }{%
        \csname HoLogoFont@font@#2\endcsname{#3}%
      }%
    }{%
      \csname HoLogoFont@logo@#1.#2\endcsname{#3}%
    }%
  \endgroup
}
%    \end{macrocode}
%    \end{macro}
%
%    \begin{macro}{\HoLogoFont@Def}
%    \begin{macrocode}
\def\HoLogoFont@Def#1{%
  \expandafter\def\csname HoLogoFont@font@#1\endcsname
}
%    \end{macrocode}
%    \end{macro}
%    \begin{macro}{\HoLogoFont@LogoDef}
%    \begin{macrocode}
\def\HoLogoFont@LogoDef#1#2{%
  \expandafter\def\csname HoLogoFont@logo@#1.#2\endcsname
}
%    \end{macrocode}
%    \end{macro}
%
% \subsubsection{Font defaults}
%
%    \begin{macro}{\HoLogoFont@font@general}
%    \begin{macrocode}
\HoLogoFont@Def{general}{}%
%    \end{macrocode}
%    \end{macro}
%
%    \begin{macro}{\HoLogoFont@font@rm}
%    \begin{macrocode}
\ltx@IfUndefined{rmfamily}{%
  \ltx@IfUndefined{rm}{%
  }{%
    \HoLogoFont@Def{rm}{\rm}%
  }%
}{%
  \HoLogoFont@Def{rm}{\rmfamily}%
}
%    \end{macrocode}
%    \end{macro}
%
%    \begin{macro}{\HoLogoFont@font@sf}
%    \begin{macrocode}
\ltx@IfUndefined{sffamily}{%
  \ltx@IfUndefined{sf}{%
  }{%
    \HoLogoFont@Def{sf}{\sf}%
  }%
}{%
  \HoLogoFont@Def{sf}{\sffamily}%
}
%    \end{macrocode}
%    \end{macro}
%
%    \begin{macro}{\HoLogoFont@font@bibsf}
%    In case of \hologo{plainTeX} the original small caps
%    variant is used as default. In \hologo{LaTeX}
%    the definition of package \xpackage{dtklogos} \cite{dtklogos}
%    is used.
%\begin{quote}
%\begin{verbatim}
%\DeclareRobustCommand{\BibTeX}{%
%  B%
%  \kern-.05em%
%  \hbox{%
%    $\m@th$% %% force math size calculations
%    \csname S@\f@size\endcsname
%    \fontsize\sf@size\z@
%    \math@fontsfalse
%    \selectfont
%    I%
%    \kern-.025em%
%    B
%  }%
%  \kern-.08em%
%  \-%
%  \TeX
%}
%\end{verbatim}
%\end{quote}
%    \begin{macrocode}
\ltx@IfUndefined{selectfont}{%
  \ltx@IfUndefined{tensc}{%
    \font\tensc=cmcsc10\relax
  }{}%
  \HoLogoFont@Def{bibsf}{\tensc}%
}{%
  \HoLogoFont@Def{bibsf}{%
    $\mathsurround=0pt$%
    \csname S@\f@size\endcsname
    \fontsize\sf@size{0pt}%
    \math@fontsfalse
    \selectfont
  }%
}
%    \end{macrocode}
%    \end{macro}
%
%    \begin{macro}{\HoLogoFont@font@sc}
%    \begin{macrocode}
\ltx@IfUndefined{scshape}{%
  \ltx@IfUndefined{tensc}{%
    \font\tensc=cmcsc10\relax
  }{}%
  \HoLogoFont@Def{sc}{\tensc}%
}{%
  \HoLogoFont@Def{sc}{\scshape}%
}
%    \end{macrocode}
%    \end{macro}
%
%    \begin{macro}{\HoLogoFont@font@sy}
%    \begin{macrocode}
\ltx@IfUndefined{usefont}{%
  \ltx@IfUndefined{tensy}{%
  }{%
    \HoLogoFont@Def{sy}{\tensy}%
  }%
}{%
  \HoLogoFont@Def{sy}{%
    \usefont{OMS}{cmsy}{m}{n}%
  }%
}
%    \end{macrocode}
%    \end{macro}
%
%    \begin{macro}{\HoLogoFont@font@logo}
%    \begin{macrocode}
\begingroup
  \def\x{LaTeX2e}%
\expandafter\endgroup
\ifx\fmtname\x
  \ltx@IfUndefined{logofamily}{%
    \DeclareRobustCommand\logofamily{%
      \not@math@alphabet\logofamily\relax
      \fontencoding{U}%
      \fontfamily{logo}%
      \selectfont
    }%
  }{}%
  \ltx@IfUndefined{logofamily}{%
  }{%
    \HoLogoFont@Def{logo}{\logofamily}%
  }%
\else
  \ltx@IfUndefined{tenlogo}{%
    \font\tenlogo=logo10\relax
  }{}%
  \HoLogoFont@Def{logo}{\tenlogo}%
\fi
%    \end{macrocode}
%    \end{macro}
%
% \subsubsection{Font setup}
%
%    \begin{macro}{\hologoFontSetup}
%    \begin{macrocode}
\def\hologoFontSetup{%
  \let\HOLOGO@name\relax
  \HOLOGO@FontSetup
}
%    \end{macrocode}
%    \end{macro}
%
%    \begin{macro}{\hologoLogoFontSetup}
%    \begin{macrocode}
\def\hologoLogoFontSetup#1{%
  \edef\HOLOGO@name{#1}%
  \ltx@IfUndefined{HoLogo@\HOLOGO@name}{%
    \@PackageError{hologo}{%
      Unknown logo `\HOLOGO@name'%
    }\@ehc
    \ltx@gobble
  }{%
    \HOLOGO@FontSetup
  }%
}
%    \end{macrocode}
%    \end{macro}
%
%    \begin{macro}{\HOLOGO@FontSetup}
%    \begin{macrocode}
\def\HOLOGO@FontSetup{%
  \kvsetkeys{HoLogoFont}%
}
%    \end{macrocode}
%    \end{macro}
%
%    \begin{macrocode}
\def\HOLOGO@temp#1{%
  \kv@define@key{HoLogoFont}{#1}{%
    \ifx\HOLOGO@name\relax
      \HoLogoFont@Def{#1}{##1}%
    \else
      \HoLogoFont@LogoDef\HOLOGO@name{#1}{##1}%
    \fi
  }%
}
\HOLOGO@temp{general}
\HOLOGO@temp{sf}
%    \end{macrocode}
%
% \subsection{Generic logo commands}
%
%    \begin{macrocode}
\HOLOGO@IfExists\hologo{%
  \@PackageError{hologo}{%
    \string\hologo\ltx@space is already defined.\MessageBreak
    Package loading is aborted%
  }\@ehc
  \HOLOGO@AtEnd
}%
\HOLOGO@IfExists\hologoRobust{%
  \@PackageError{hologo}{%
    \string\hologoRobust\ltx@space is already defined.\MessageBreak
    Package loading is aborted%
  }\@ehc
  \HOLOGO@AtEnd
}%
%    \end{macrocode}
%
% \subsubsection{\cs{hologo} and friends}
%
%    \begin{macrocode}
\ifluatex
  \expandafter\ltx@firstofone
\else
  \expandafter\ltx@gobble
\fi
{%
  \ltx@IfUndefined{ifincsname}{%
    \ifnum\luatexversion<36 %
      \expandafter\ltx@gobble
    \else
      \expandafter\ltx@firstofone
    \fi
    {%
      \begingroup
        \ifcase0%
            \directlua{%
              if tex.enableprimitives then %
                tex.enableprimitives('HOLOGO@', {'ifincsname'})%
              else %
                tex.print('1')%
              end%
            }%
            \ifx\HOLOGO@ifincsname\@undefined 1\fi%
            \relax
          \expandafter\ltx@firstofone
        \else
          \endgroup
          \expandafter\ltx@gobble
        \fi
        {%
          \global\let\ifincsname\HOLOGO@ifincsname
        }%
      \HOLOGO@temp
    }%
  }{}%
}
%    \end{macrocode}
%    \begin{macrocode}
\ltx@IfUndefined{ifincsname}{%
  \catcode`$=14 %
}{%
  \catcode`$=9 %
}
%    \end{macrocode}
%
%    \begin{macro}{\hologo}
%    \begin{macrocode}
\def\hologo#1{%
$ \ifincsname
$   \ltx@ifundefined{HoLogoCs@\HOLOGO@Variant{#1}}{%
$     #1%
$   }{%
$     \csname HoLogoCs@\HOLOGO@Variant{#1}\endcsname\ltx@firstoftwo
$   }%
$ \else
    \HOLOGO@IfExists\texorpdfstring\texorpdfstring\ltx@firstoftwo
    {%
      \hologoRobust{#1}%
    }{%
      \ltx@ifundefined{HoLogoBkm@\HOLOGO@Variant{#1}}{%
        \ltx@ifundefined{HoLogo@#1}{?#1?}{#1}%
      }{%
        \csname HoLogoBkm@\HOLOGO@Variant{#1}\endcsname
        \ltx@firstoftwo
      }%
    }%
$ \fi
}
%    \end{macrocode}
%    \end{macro}
%    \begin{macro}{\Hologo}
%    \begin{macrocode}
\def\Hologo#1{%
$ \ifincsname
$   \ltx@ifundefined{HoLogoCs@\HOLOGO@Variant{#1}}{%
$     #1%
$   }{%
$     \csname HoLogoCs@\HOLOGO@Variant{#1}\endcsname\ltx@secondoftwo
$   }%
$ \else
    \HOLOGO@IfExists\texorpdfstring\texorpdfstring\ltx@firstoftwo
    {%
      \HologoRobust{#1}%
    }{%
      \ltx@ifundefined{HoLogoBkm@\HOLOGO@Variant{#1}}{%
        \ltx@ifundefined{HoLogo@#1}{?#1?}{#1}%
      }{%
        \csname HoLogoBkm@\HOLOGO@Variant{#1}\endcsname
        \ltx@secondoftwo
      }%
    }%
$ \fi
}
%    \end{macrocode}
%    \end{macro}
%
%    \begin{macro}{\hologoVariant}
%    \begin{macrocode}
\def\hologoVariant#1#2{%
  \ifx\relax#2\relax
    \hologo{#1}%
  \else
$   \ifincsname
$     \ltx@ifundefined{HoLogoCs@#1@#2}{%
$       #1%
$     }{%
$       \csname HoLogoCs@#1@#2\endcsname\ltx@firstoftwo
$     }%
$   \else
      \HOLOGO@IfExists\texorpdfstring\texorpdfstring\ltx@firstoftwo
      {%
        \hologoVariantRobust{#1}{#2}%
      }{%
        \ltx@ifundefined{HoLogoBkm@#1@#2}{%
          \ltx@ifundefined{HoLogo@#1}{?#1?}{#1}%
        }{%
          \csname HoLogoBkm@#1@#2\endcsname
          \ltx@firstoftwo
        }%
      }%
$   \fi
  \fi
}
%    \end{macrocode}
%    \end{macro}
%    \begin{macro}{\HologoVariant}
%    \begin{macrocode}
\def\HologoVariant#1#2{%
  \ifx\relax#2\relax
    \Hologo{#1}%
  \else
$   \ifincsname
$     \ltx@ifundefined{HoLogoCs@#1@#2}{%
$       #1%
$     }{%
$       \csname HoLogoCs@#1@#2\endcsname\ltx@secondoftwo
$     }%
$   \else
      \HOLOGO@IfExists\texorpdfstring\texorpdfstring\ltx@firstoftwo
      {%
        \HologoVariantRobust{#1}{#2}%
      }{%
        \ltx@ifundefined{HoLogoBkm@#1@#2}{%
          \ltx@ifundefined{HoLogo@#1}{?#1?}{#1}%
        }{%
          \csname HoLogoBkm@#1@#2\endcsname
          \ltx@secondoftwo
        }%
      }%
$   \fi
  \fi
}
%    \end{macrocode}
%    \end{macro}
%
%    \begin{macrocode}
\catcode`\$=3 %
%    \end{macrocode}
%
% \subsubsection{\cs{hologoRobust} and friends}
%
%    \begin{macro}{\hologoRobust}
%    \begin{macrocode}
\ltx@IfUndefined{protected}{%
  \ltx@IfUndefined{DeclareRobustCommand}{%
    \def\hologoRobust#1%
  }{%
    \DeclareRobustCommand*\hologoRobust[1]%
  }%
}{%
  \protected\def\hologoRobust#1%
}%
{%
  \edef\HOLOGO@name{#1}%
  \ltx@IfUndefined{HoLogo@\HOLOGO@Variant\HOLOGO@name}{%
    \@PackageError{hologo}{%
      Unknown logo `\HOLOGO@name'%
    }\@ehc
    ?\HOLOGO@name?%
  }{%
    \ltx@IfUndefined{ver@tex4ht.sty}{%
      \HoLogoFont@font\HOLOGO@name{general}{%
        \csname HoLogo@\HOLOGO@Variant\HOLOGO@name\endcsname
        \ltx@firstoftwo
      }%
    }{%
      \ltx@IfUndefined{HoLogoHtml@\HOLOGO@Variant\HOLOGO@name}{%
        \HOLOGO@name
      }{%
        \csname HoLogoHtml@\HOLOGO@Variant\HOLOGO@name\endcsname
        \ltx@firstoftwo
      }%
    }%
  }%
}
%    \end{macrocode}
%    \end{macro}
%    \begin{macro}{\HologoRobust}
%    \begin{macrocode}
\ltx@IfUndefined{protected}{%
  \ltx@IfUndefined{DeclareRobustCommand}{%
    \def\HologoRobust#1%
  }{%
    \DeclareRobustCommand*\HologoRobust[1]%
  }%
}{%
  \protected\def\HologoRobust#1%
}%
{%
  \edef\HOLOGO@name{#1}%
  \ltx@IfUndefined{HoLogo@\HOLOGO@Variant\HOLOGO@name}{%
    \@PackageError{hologo}{%
      Unknown logo `\HOLOGO@name'%
    }\@ehc
    ?\HOLOGO@name?%
  }{%
    \ltx@IfUndefined{ver@tex4ht.sty}{%
      \HoLogoFont@font\HOLOGO@name{general}{%
        \csname HoLogo@\HOLOGO@Variant\HOLOGO@name\endcsname
        \ltx@secondoftwo
      }%
    }{%
      \ltx@IfUndefined{HoLogoHtml@\HOLOGO@Variant\HOLOGO@name}{%
        \expandafter\HOLOGO@Uppercase\HOLOGO@name
      }{%
        \csname HoLogoHtml@\HOLOGO@Variant\HOLOGO@name\endcsname
        \ltx@secondoftwo
      }%
    }%
  }%
}
%    \end{macrocode}
%    \end{macro}
%    \begin{macro}{\hologoVariantRobust}
%    \begin{macrocode}
\ltx@IfUndefined{protected}{%
  \ltx@IfUndefined{DeclareRobustCommand}{%
    \def\hologoVariantRobust#1#2%
  }{%
    \DeclareRobustCommand*\hologoVariantRobust[2]%
  }%
}{%
  \protected\def\hologoVariantRobust#1#2%
}%
{%
  \begingroup
    \hologoLogoSetup{#1}{variant={#2}}%
    \hologoRobust{#1}%
  \endgroup
}
%    \end{macrocode}
%    \end{macro}
%    \begin{macro}{\HologoVariantRobust}
%    \begin{macrocode}
\ltx@IfUndefined{protected}{%
  \ltx@IfUndefined{DeclareRobustCommand}{%
    \def\HologoVariantRobust#1#2%
  }{%
    \DeclareRobustCommand*\HologoVariantRobust[2]%
  }%
}{%
  \protected\def\HologoVariantRobust#1#2%
}%
{%
  \begingroup
    \hologoLogoSetup{#1}{variant={#2}}%
    \HologoRobust{#1}%
  \endgroup
}
%    \end{macrocode}
%    \end{macro}
%
%    \begin{macro}{\hologorobust}
%    Macro \cs{hologorobust} is only defined for compatibility.
%    Its use is deprecated.
%    \begin{macrocode}
\def\hologorobust{\hologoRobust}
%    \end{macrocode}
%    \end{macro}
%
% \subsection{Helpers}
%
%    \begin{macro}{\HOLOGO@Uppercase}
%    Macro \cs{HOLOGO@Uppercase} is restricted to \cs{uppercase},
%    because \hologo{plainTeX} or \hologo{iniTeX} do not provide
%    \cs{MakeUppercase}.
%    \begin{macrocode}
\def\HOLOGO@Uppercase#1{\uppercase{#1}}
%    \end{macrocode}
%    \end{macro}
%
%    \begin{macro}{\HOLOGO@PdfdocUnicode}
%    \begin{macrocode}
\def\HOLOGO@PdfdocUnicode{%
  \ifx\ifHy@unicode\iftrue
    \expandafter\ltx@secondoftwo
  \else
    \expandafter\ltx@firstoftwo
  \fi
}
%    \end{macrocode}
%    \end{macro}
%
%    \begin{macro}{\HOLOGO@Math}
%    \begin{macrocode}
\def\HOLOGO@MathSetup{%
  \mathsurround0pt\relax
  \HOLOGO@IfExists\f@series{%
    \if b\expandafter\ltx@car\f@series x\@nil
      \csname boldmath\endcsname
   \fi
  }{}%
}
%    \end{macrocode}
%    \end{macro}
%
%    \begin{macro}{\HOLOGO@TempDimen}
%    \begin{macrocode}
\dimendef\HOLOGO@TempDimen=\ltx@zero
%    \end{macrocode}
%    \end{macro}
%    \begin{macro}{\HOLOGO@NegativeKerning}
%    \begin{macrocode}
\def\HOLOGO@NegativeKerning#1{%
  \begingroup
    \HOLOGO@TempDimen=0pt\relax
    \comma@parse@normalized{#1}{%
      \ifdim\HOLOGO@TempDimen=0pt %
        \expandafter\HOLOGO@@NegativeKerning\comma@entry
      \fi
      \ltx@gobble
    }%
    \ifdim\HOLOGO@TempDimen<0pt %
      \kern\HOLOGO@TempDimen
    \fi
  \endgroup
}
%    \end{macrocode}
%    \end{macro}
%    \begin{macro}{\HOLOGO@@NegativeKerning}
%    \begin{macrocode}
\def\HOLOGO@@NegativeKerning#1#2{%
  \setbox\ltx@zero\hbox{#1#2}%
  \HOLOGO@TempDimen=\wd\ltx@zero
  \setbox\ltx@zero\hbox{#1\kern0pt#2}%
  \advance\HOLOGO@TempDimen by -\wd\ltx@zero
}
%    \end{macrocode}
%    \end{macro}
%
%    \begin{macro}{\HOLOGO@SpaceFactor}
%    \begin{macrocode}
\def\HOLOGO@SpaceFactor{%
  \spacefactor1000 %
}
%    \end{macrocode}
%    \end{macro}
%
%    \begin{macro}{\HOLOGO@Span}
%    \begin{macrocode}
\def\HOLOGO@Span#1#2{%
  \HCode{<span class="HoLogo-#1">}%
  #2%
  \HCode{</span>}%
}
%    \end{macrocode}
%    \end{macro}
%
% \subsubsection{Text subscript}
%
%    \begin{macro}{\HOLOGO@SubScript}%
%    \begin{macrocode}
\def\HOLOGO@SubScript#1{%
  \ltx@IfUndefined{textsubscript}{%
    \ltx@IfUndefined{text}{%
      \ltx@mbox{%
        \mathsurround=0pt\relax
        $%
          _{%
            \ltx@IfUndefined{sf@size}{%
              \mathrm{#1}%
            }{%
              \mbox{%
                \fontsize\sf@size{0pt}\selectfont
                #1%
              }%
            }%
          }%
        $%
      }%
    }{%
      \ltx@mbox{%
        \mathsurround=0pt\relax
        $_{\text{#1}}$%
      }%
    }%
  }{%
    \textsubscript{#1}%
  }%
}
%    \end{macrocode}
%    \end{macro}
%
% \subsection{\hologo{TeX} and friends}
%
% \subsubsection{\hologo{TeX}}
%
%    \begin{macro}{\HoLogo@TeX}
%    Source: \hologo{LaTeX} kernel.
%    \begin{macrocode}
\def\HoLogo@TeX#1{%
  T\kern-.1667em\lower.5ex\hbox{E}\kern-.125emX\HOLOGO@SpaceFactor
}
%    \end{macrocode}
%    \end{macro}
%    \begin{macro}{\HoLogoHtml@TeX}
%    \begin{macrocode}
\def\HoLogoHtml@TeX#1{%
  \HoLogoCss@TeX
  \HOLOGO@Span{TeX}{%
    T%
    \HOLOGO@Span{e}{%
      E%
    }%
    X%
  }%
}
%    \end{macrocode}
%    \end{macro}
%    \begin{macro}{\HoLogoCss@TeX}
%    \begin{macrocode}
\def\HoLogoCss@TeX{%
  \Css{%
    span.HoLogo-TeX span.HoLogo-e{%
      position:relative;%
      top:.5ex;%
      margin-left:-.1667em;%
      margin-right:-.125em;%
    }%
  }%
  \Css{%
    a span.HoLogo-TeX span.HoLogo-e{%
      text-decoration:none;%
    }%
  }%
  \global\let\HoLogoCss@TeX\relax
}
%    \end{macrocode}
%    \end{macro}
%
% \subsubsection{\hologo{plainTeX}}
%
%    \begin{macro}{\HoLogo@plainTeX@space}
%    Source: ``The \hologo{TeX}book''
%    \begin{macrocode}
\def\HoLogo@plainTeX@space#1{%
  \HOLOGO@mbox{#1{p}{P}lain}\HOLOGO@space\hologo{TeX}%
}
%    \end{macrocode}
%    \end{macro}
%    \begin{macro}{\HoLogoCs@plainTeX@space}
%    \begin{macrocode}
\def\HoLogoCs@plainTeX@space#1{#1{p}{P}lain TeX}%
%    \end{macrocode}
%    \end{macro}
%    \begin{macro}{\HoLogoBkm@plainTeX@space}
%    \begin{macrocode}
\def\HoLogoBkm@plainTeX@space#1{%
  #1{p}{P}lain \hologo{TeX}%
}
%    \end{macrocode}
%    \end{macro}
%    \begin{macro}{\HoLogoHtml@plainTeX@space}
%    \begin{macrocode}
\def\HoLogoHtml@plainTeX@space#1{%
  #1{p}{P}lain \hologo{TeX}%
}
%    \end{macrocode}
%    \end{macro}
%
%    \begin{macro}{\HoLogo@plainTeX@hyphen}
%    \begin{macrocode}
\def\HoLogo@plainTeX@hyphen#1{%
  \HOLOGO@mbox{#1{p}{P}lain}\HOLOGO@hyphen\hologo{TeX}%
}
%    \end{macrocode}
%    \end{macro}
%    \begin{macro}{\HoLogoCs@plainTeX@hyphen}
%    \begin{macrocode}
\def\HoLogoCs@plainTeX@hyphen#1{#1{p}{P}lain-TeX}
%    \end{macrocode}
%    \end{macro}
%    \begin{macro}{\HoLogoBkm@plainTeX@hyphen}
%    \begin{macrocode}
\def\HoLogoBkm@plainTeX@hyphen#1{%
  #1{p}{P}lain-\hologo{TeX}%
}
%    \end{macrocode}
%    \end{macro}
%    \begin{macro}{\HoLogoHtml@plainTeX@hyphen}
%    \begin{macrocode}
\def\HoLogoHtml@plainTeX@hyphen#1{%
  #1{p}{P}lain-\hologo{TeX}%
}
%    \end{macrocode}
%    \end{macro}
%
%    \begin{macro}{\HoLogo@plainTeX@runtogether}
%    \begin{macrocode}
\def\HoLogo@plainTeX@runtogether#1{%
  \HOLOGO@mbox{#1{p}{P}lain\hologo{TeX}}%
}
%    \end{macrocode}
%    \end{macro}
%    \begin{macro}{\HoLogoCs@plainTeX@runtogether}
%    \begin{macrocode}
\def\HoLogoCs@plainTeX@runtogether#1{#1{p}{P}lainTeX}
%    \end{macrocode}
%    \end{macro}
%    \begin{macro}{\HoLogoBkm@plainTeX@runtogether}
%    \begin{macrocode}
\def\HoLogoBkm@plainTeX@runtogether#1{%
  #1{p}{P}lain\hologo{TeX}%
}
%    \end{macrocode}
%    \end{macro}
%    \begin{macro}{\HoLogoHtml@plainTeX@runtogether}
%    \begin{macrocode}
\def\HoLogoHtml@plainTeX@runtogether#1{%
  #1{p}{P}lain\hologo{TeX}%
}
%    \end{macrocode}
%    \end{macro}
%
%    \begin{macro}{\HoLogo@plainTeX}
%    \begin{macrocode}
\def\HoLogo@plainTeX{\HoLogo@plainTeX@space}
%    \end{macrocode}
%    \end{macro}
%    \begin{macro}{\HoLogoCs@plainTeX}
%    \begin{macrocode}
\def\HoLogoCs@plainTeX{\HoLogoCs@plainTeX@space}
%    \end{macrocode}
%    \end{macro}
%    \begin{macro}{\HoLogoBkm@plainTeX}
%    \begin{macrocode}
\def\HoLogoBkm@plainTeX{\HoLogoBkm@plainTeX@space}
%    \end{macrocode}
%    \end{macro}
%    \begin{macro}{\HoLogoHtml@plainTeX}
%    \begin{macrocode}
\def\HoLogoHtml@plainTeX{\HoLogoHtml@plainTeX@space}
%    \end{macrocode}
%    \end{macro}
%
% \subsubsection{\hologo{LaTeX}}
%
%    Source: \hologo{LaTeX} kernel.
%\begin{quote}
%\begin{verbatim}
%\DeclareRobustCommand{\LaTeX}{%
%  L%
%  \kern-.36em%
%  {%
%    \sbox\z@ T%
%    \vbox to\ht\z@{%
%      \hbox{%
%        \check@mathfonts
%        \fontsize\sf@size\z@
%        \math@fontsfalse
%        \selectfont
%        A%
%      }%
%      \vss
%    }%
%  }%
%  \kern-.15em%
%  \TeX
%}
%\end{verbatim}
%\end{quote}
%
%    \begin{macro}{\HoLogo@La}
%    \begin{macrocode}
\def\HoLogo@La#1{%
  L%
  \kern-.36em%
  \begingroup
    \setbox\ltx@zero\hbox{T}%
    \vbox to\ht\ltx@zero{%
      \hbox{%
        \ltx@ifundefined{check@mathfonts}{%
          \csname sevenrm\endcsname
        }{%
          \check@mathfonts
          \fontsize\sf@size{0pt}%
          \math@fontsfalse\selectfont
        }%
        A%
      }%
      \vss
    }%
  \endgroup
}
%    \end{macrocode}
%    \end{macro}
%
%    \begin{macro}{\HoLogo@LaTeX}
%    Source: \hologo{LaTeX} kernel.
%    \begin{macrocode}
\def\HoLogo@LaTeX#1{%
  \hologo{La}%
  \kern-.15em%
  \hologo{TeX}%
}
%    \end{macrocode}
%    \end{macro}
%    \begin{macro}{\HoLogoHtml@LaTeX}
%    \begin{macrocode}
\def\HoLogoHtml@LaTeX#1{%
  \HoLogoCss@LaTeX
  \HOLOGO@Span{LaTeX}{%
    L%
    \HOLOGO@Span{a}{%
      A%
    }%
    \hologo{TeX}%
  }%
}
%    \end{macrocode}
%    \end{macro}
%    \begin{macro}{\HoLogoCss@LaTeX}
%    \begin{macrocode}
\def\HoLogoCss@LaTeX{%
  \Css{%
    span.HoLogo-LaTeX span.HoLogo-a{%
      position:relative;%
      top:-.5ex;%
      margin-left:-.36em;%
      margin-right:-.15em;%
      font-size:85\%;%
    }%
  }%
  \global\let\HoLogoCss@LaTeX\relax
}
%    \end{macrocode}
%    \end{macro}
%
% \subsubsection{\hologo{(La)TeX}}
%
%    \begin{macro}{\HoLogo@LaTeXTeX}
%    The kerning around the parentheses is taken
%    from package \xpackage{dtklogos} \cite{dtklogos}.
%\begin{quote}
%\begin{verbatim}
%\DeclareRobustCommand{\LaTeXTeX}{%
%  (%
%  \kern-.15em%
%  L%
%  \kern-.36em%
%  {%
%    \sbox\z@ T%
%    \vbox to\ht0{%
%      \hbox{%
%        $\m@th$%
%        \csname S@\f@size\endcsname
%        \fontsize\sf@size\z@
%        \math@fontsfalse
%        \selectfont
%        A%
%      }%
%      \vss
%    }%
%  }%
%  \kern-.2em%
%  )%
%  \kern-.15em%
%  \TeX
%}
%\end{verbatim}
%\end{quote}
%    \begin{macrocode}
\def\HoLogo@LaTeXTeX#1{%
  (%
  \kern-.15em%
  \hologo{La}%
  \kern-.2em%
  )%
  \kern-.15em%
  \hologo{TeX}%
}
%    \end{macrocode}
%    \end{macro}
%    \begin{macro}{\HoLogoBkm@LaTeXTeX}
%    \begin{macrocode}
\def\HoLogoBkm@LaTeXTeX#1{(La)TeX}
%    \end{macrocode}
%    \end{macro}
%
%    \begin{macro}{\HoLogo@(La)TeX}
%    \begin{macrocode}
\expandafter
\let\csname HoLogo@(La)TeX\endcsname\HoLogo@LaTeXTeX
%    \end{macrocode}
%    \end{macro}
%    \begin{macro}{\HoLogoBkm@(La)TeX}
%    \begin{macrocode}
\expandafter
\let\csname HoLogoBkm@(La)TeX\endcsname\HoLogoBkm@LaTeXTeX
%    \end{macrocode}
%    \end{macro}
%    \begin{macro}{\HoLogoHtml@LaTeXTeX}
%    \begin{macrocode}
\def\HoLogoHtml@LaTeXTeX#1{%
  \HoLogoCss@LaTeXTeX
  \HOLOGO@Span{LaTeXTeX}{%
    (%
    \HOLOGO@Span{L}{L}%
    \HOLOGO@Span{a}{A}%
    \HOLOGO@Span{ParenRight}{)}%
    \hologo{TeX}%
  }%
}
%    \end{macrocode}
%    \end{macro}
%    \begin{macro}{\HoLogoHtml@(La)TeX}
%    Kerning after opening parentheses and before closing parentheses
%    is $-0.1$\,em. The original values $-0.15$\,em
%    looked too ugly for a serif font.
%    \begin{macrocode}
\expandafter
\let\csname HoLogoHtml@(La)TeX\endcsname\HoLogoHtml@LaTeXTeX
%    \end{macrocode}
%    \end{macro}
%    \begin{macro}{\HoLogoCss@LaTeXTeX}
%    \begin{macrocode}
\def\HoLogoCss@LaTeXTeX{%
  \Css{%
    span.HoLogo-LaTeXTeX span.HoLogo-L{%
      margin-left:-.1em;%
    }%
  }%
  \Css{%
    span.HoLogo-LaTeXTeX span.HoLogo-a{%
      position:relative;%
      top:-.5ex;%
      margin-left:-.36em;%
      margin-right:-.1em;%
      font-size:85\%;%
    }%
  }%
  \Css{%
    span.HoLogo-LaTeXTeX span.HoLogo-ParenRight{%
      margin-right:-.15em;%
    }%
  }%
  \global\let\HoLogoCss@LaTeXTeX\relax
}
%    \end{macrocode}
%    \end{macro}
%
% \subsubsection{\hologo{LaTeXe}}
%
%    \begin{macro}{\HoLogo@LaTeXe}
%    Source: \hologo{LaTeX} kernel
%    \begin{macrocode}
\def\HoLogo@LaTeXe#1{%
  \hologo{LaTeX}%
  \kern.15em%
  \hbox{%
    \HOLOGO@MathSetup
    2%
    $_{\textstyle\varepsilon}$%
  }%
}
%    \end{macrocode}
%    \end{macro}
%
%    \begin{macro}{\HoLogoCs@LaTeXe}
%    \begin{macrocode}
\ifnum64=`\^^^^0040\relax % test for big chars of LuaTeX/XeTeX
  \catcode`\$=9 %
  \catcode`\&=14 %
\else
  \catcode`\$=14 %
  \catcode`\&=9 %
\fi
\def\HoLogoCs@LaTeXe#1{%
  LaTeX2%
$ \string ^^^^0395%
& e%
}%
\catcode`\$=3 %
\catcode`\&=4 %
%    \end{macrocode}
%    \end{macro}
%
%    \begin{macro}{\HoLogoBkm@LaTeXe}
%    \begin{macrocode}
\def\HoLogoBkm@LaTeXe#1{%
  \hologo{LaTeX}%
  2%
  \HOLOGO@PdfdocUnicode{e}{\textepsilon}%
}
%    \end{macrocode}
%    \end{macro}
%
%    \begin{macro}{\HoLogoHtml@LaTeXe}
%    \begin{macrocode}
\def\HoLogoHtml@LaTeXe#1{%
  \HoLogoCss@LaTeXe
  \HOLOGO@Span{LaTeX2e}{%
    \hologo{LaTeX}%
    \HOLOGO@Span{2}{2}%
    \HOLOGO@Span{e}{%
      \HOLOGO@MathSetup
      \ensuremath{\textstyle\varepsilon}%
    }%
  }%
}
%    \end{macrocode}
%    \end{macro}
%    \begin{macro}{\HoLogoCss@LaTeXe}
%    \begin{macrocode}
\def\HoLogoCss@LaTeXe{%
  \Css{%
    span.HoLogo-LaTeX2e span.HoLogo-2{%
      padding-left:.15em;%
    }%
  }%
  \Css{%
    span.HoLogo-LaTeX2e span.HoLogo-e{%
      position:relative;%
      top:.35ex;%
      text-decoration:none;%
    }%
  }%
  \global\let\HoLogoCss@LaTeXe\relax
}
%    \end{macrocode}
%    \end{macro}
%
%    \begin{macro}{\HoLogo@LaTeX2e}
%    \begin{macrocode}
\expandafter
\let\csname HoLogo@LaTeX2e\endcsname\HoLogo@LaTeXe
%    \end{macrocode}
%    \end{macro}
%    \begin{macro}{\HoLogoCs@LaTeX2e}
%    \begin{macrocode}
\expandafter
\let\csname HoLogoCs@LaTeX2e\endcsname\HoLogoCs@LaTeXe
%    \end{macrocode}
%    \end{macro}
%    \begin{macro}{\HoLogoBkm@LaTeX2e}
%    \begin{macrocode}
\expandafter
\let\csname HoLogoBkm@LaTeX2e\endcsname\HoLogoBkm@LaTeXe
%    \end{macrocode}
%    \end{macro}
%    \begin{macro}{\HoLogoHtml@LaTeX2e}
%    \begin{macrocode}
\expandafter
\let\csname HoLogoHtml@LaTeX2e\endcsname\HoLogoHtml@LaTeXe
%    \end{macrocode}
%    \end{macro}
%
% \subsubsection{\hologo{LaTeX3}}
%
%    \begin{macro}{\HoLogo@LaTeX3}
%    Source: \hologo{LaTeX} kernel
%    \begin{macrocode}
\expandafter\def\csname HoLogo@LaTeX3\endcsname#1{%
  \hologo{LaTeX}%
  3%
}
%    \end{macrocode}
%    \end{macro}
%
%    \begin{macro}{\HoLogoBkm@LaTeX3}
%    \begin{macrocode}
\expandafter\def\csname HoLogoBkm@LaTeX3\endcsname#1{%
  \hologo{LaTeX}%
  3%
}
%    \end{macrocode}
%    \end{macro}
%    \begin{macro}{\HoLogoHtml@LaTeX3}
%    \begin{macrocode}
\expandafter
\let\csname HoLogoHtml@LaTeX3\expandafter\endcsname
\csname HoLogo@LaTeX3\endcsname
%    \end{macrocode}
%    \end{macro}
%
% \subsubsection{\hologo{LaTeXML}}
%
%    \begin{macro}{\HoLogo@LaTeXML}
%    \begin{macrocode}
\def\HoLogo@LaTeXML#1{%
  \HOLOGO@mbox{%
    \hologo{La}%
    \kern-.15em%
    T%
    \kern-.1667em%
    \lower.5ex\hbox{E}%
    \kern-.125em%
    \HoLogoFont@font{LaTeXML}{sc}{xml}%
  }%
}
%    \end{macrocode}
%    \end{macro}
%    \begin{macro}{\HoLogoHtml@pdfLaTeX}
%    \begin{macrocode}
\def\HoLogoHtml@LaTeXML#1{%
  \HOLOGO@Span{LaTeXML}{%
    \HoLogoCss@LaTeX
    \HoLogoCss@TeX
    \HOLOGO@Span{LaTeX}{%
      L%
      \HOLOGO@Span{a}{%
        A%
      }%
    }%
    \HOLOGO@Span{TeX}{%
      T%
      \HOLOGO@Span{e}{%
        E%
      }%
    }%
    \HCode{<span style="font-variant: small-caps;">}%
    xml%
    \HCode{</span>}%
  }%
}
%    \end{macrocode}
%    \end{macro}
%
% \subsubsection{\hologo{eTeX}}
%
%    \begin{macro}{\HoLogo@eTeX}
%    Source: package \xpackage{etex}
%    \begin{macrocode}
\def\HoLogo@eTeX#1{%
  \ltx@mbox{%
    \HOLOGO@MathSetup
    $\varepsilon$%
    -%
    \HOLOGO@NegativeKerning{-T,T-,To}%
    \hologo{TeX}%
  }%
}
%    \end{macrocode}
%    \end{macro}
%    \begin{macro}{\HoLogoCs@eTeX}
%    \begin{macrocode}
\ifnum64=`\^^^^0040\relax % test for big chars of LuaTeX/XeTeX
  \catcode`\$=9 %
  \catcode`\&=14 %
\else
  \catcode`\$=14 %
  \catcode`\&=9 %
\fi
\def\HoLogoCs@eTeX#1{%
$ #1{\string ^^^^0395}{\string ^^^^03b5}%
& #1{e}{E}%
  TeX%
}%
\catcode`\$=3 %
\catcode`\&=4 %
%    \end{macrocode}
%    \end{macro}
%    \begin{macro}{\HoLogoBkm@eTeX}
%    \begin{macrocode}
\def\HoLogoBkm@eTeX#1{%
  \HOLOGO@PdfdocUnicode{#1{e}{E}}{\textepsilon}%
  -%
  \hologo{TeX}%
}
%    \end{macrocode}
%    \end{macro}
%    \begin{macro}{\HoLogoHtml@eTeX}
%    \begin{macrocode}
\def\HoLogoHtml@eTeX#1{%
  \ltx@mbox{%
    \HOLOGO@MathSetup
    $\varepsilon$%
    -%
    \hologo{TeX}%
  }%
}
%    \end{macrocode}
%    \end{macro}
%
% \subsubsection{\hologo{iniTeX}}
%
%    \begin{macro}{\HoLogo@iniTeX}
%    \begin{macrocode}
\def\HoLogo@iniTeX#1{%
  \HOLOGO@mbox{%
    #1{i}{I}ni\hologo{TeX}%
  }%
}
%    \end{macrocode}
%    \end{macro}
%    \begin{macro}{\HoLogoCs@iniTeX}
%    \begin{macrocode}
\def\HoLogoCs@iniTeX#1{#1{i}{I}niTeX}
%    \end{macrocode}
%    \end{macro}
%    \begin{macro}{\HoLogoBkm@iniTeX}
%    \begin{macrocode}
\def\HoLogoBkm@iniTeX#1{%
  #1{i}{I}ni\hologo{TeX}%
}
%    \end{macrocode}
%    \end{macro}
%    \begin{macro}{\HoLogoHtml@iniTeX}
%    \begin{macrocode}
\let\HoLogoHtml@iniTeX\HoLogo@iniTeX
%    \end{macrocode}
%    \end{macro}
%
% \subsubsection{\hologo{virTeX}}
%
%    \begin{macro}{\HoLogo@virTeX}
%    \begin{macrocode}
\def\HoLogo@virTeX#1{%
  \HOLOGO@mbox{%
    #1{v}{V}ir\hologo{TeX}%
  }%
}
%    \end{macrocode}
%    \end{macro}
%    \begin{macro}{\HoLogoCs@virTeX}
%    \begin{macrocode}
\def\HoLogoCs@virTeX#1{#1{v}{V}irTeX}
%    \end{macrocode}
%    \end{macro}
%    \begin{macro}{\HoLogoBkm@virTeX}
%    \begin{macrocode}
\def\HoLogoBkm@virTeX#1{%
  #1{v}{V}ir\hologo{TeX}%
}
%    \end{macrocode}
%    \end{macro}
%    \begin{macro}{\HoLogoHtml@virTeX}
%    \begin{macrocode}
\let\HoLogoHtml@virTeX\HoLogo@virTeX
%    \end{macrocode}
%    \end{macro}
%
% \subsubsection{\hologo{SliTeX}}
%
% \paragraph{Definitions of the three variants.}
%
%    \begin{macro}{\HoLogo@SLiTeX@lift}
%    \begin{macrocode}
\def\HoLogo@SLiTeX@lift#1{%
  \HoLogoFont@font{SliTeX}{rm}{%
    S%
    \kern-.06em%
    L%
    \kern-.18em%
    \raise.32ex\hbox{\HoLogoFont@font{SliTeX}{sc}{i}}%
    \HOLOGO@discretionary
    \kern-.06em%
    \hologo{TeX}%
  }%
}
%    \end{macrocode}
%    \end{macro}
%    \begin{macro}{\HoLogoBkm@SLiTeX@lift}
%    \begin{macrocode}
\def\HoLogoBkm@SLiTeX@lift#1{SLiTeX}
%    \end{macrocode}
%    \end{macro}
%    \begin{macro}{\HoLogoHtml@SLiTeX@lift}
%    \begin{macrocode}
\def\HoLogoHtml@SLiTeX@lift#1{%
  \HoLogoCss@SLiTeX@lift
  \HOLOGO@Span{SLiTeX-lift}{%
    \HoLogoFont@font{SliTeX}{rm}{%
      S%
      \HOLOGO@Span{L}{L}%
      \HOLOGO@Span{i}{i}%
      \hologo{TeX}%
    }%
  }%
}
%    \end{macrocode}
%    \end{macro}
%    \begin{macro}{\HoLogoCss@SLiTeX@lift}
%    \begin{macrocode}
\def\HoLogoCss@SLiTeX@lift{%
  \Css{%
    span.HoLogo-SLiTeX-lift span.HoLogo-L{%
      margin-left:-.06em;%
      margin-right:-.18em;%
    }%
  }%
  \Css{%
    span.HoLogo-SLiTeX-lift span.HoLogo-i{%
      position:relative;%
      top:-.32ex;%
      margin-right:-.06em;%
      font-variant:small-caps;%
    }%
  }%
  \global\let\HoLogoCss@SLiTeX@lift\relax
}
%    \end{macrocode}
%    \end{macro}
%
%    \begin{macro}{\HoLogo@SliTeX@simple}
%    \begin{macrocode}
\def\HoLogo@SliTeX@simple#1{%
  \HoLogoFont@font{SliTeX}{rm}{%
    \ltx@mbox{%
      \HoLogoFont@font{SliTeX}{sc}{Sli}%
    }%
    \HOLOGO@discretionary
    \hologo{TeX}%
  }%
}
%    \end{macrocode}
%    \end{macro}
%    \begin{macro}{\HoLogoBkm@SliTeX@simple}
%    \begin{macrocode}
\def\HoLogoBkm@SliTeX@simple#1{SliTeX}
%    \end{macrocode}
%    \end{macro}
%    \begin{macro}{\HoLogoHtml@SliTeX@simple}
%    \begin{macrocode}
\let\HoLogoHtml@SliTeX@simple\HoLogo@SliTeX@simple
%    \end{macrocode}
%    \end{macro}
%
%    \begin{macro}{\HoLogo@SliTeX@narrow}
%    \begin{macrocode}
\def\HoLogo@SliTeX@narrow#1{%
  \HoLogoFont@font{SliTeX}{rm}{%
    \ltx@mbox{%
      S%
      \kern-.06em%
      \HoLogoFont@font{SliTeX}{sc}{%
        l%
        \kern-.035em%
        i%
      }%
    }%
    \HOLOGO@discretionary
    \kern-.06em%
    \hologo{TeX}%
  }%
}
%    \end{macrocode}
%    \end{macro}
%    \begin{macro}{\HoLogoBkm@SliTeX@narrow}
%    \begin{macrocode}
\def\HoLogoBkm@SliTeX@narrow#1{SliTeX}
%    \end{macrocode}
%    \end{macro}
%    \begin{macro}{\HoLogoHtml@SliTeX@narrow}
%    \begin{macrocode}
\def\HoLogoHtml@SliTeX@narrow#1{%
  \HoLogoCss@SliTeX@narrow
  \HOLOGO@Span{SliTeX-narrow}{%
    \HoLogoFont@font{SliTeX}{rm}{%
      S%
        \HOLOGO@Span{l}{l}%
        \HOLOGO@Span{i}{i}%
      \hologo{TeX}%
    }%
  }%
}
%    \end{macrocode}
%    \end{macro}
%    \begin{macro}{\HoLogoCss@SliTeX@narrow}
%    \begin{macrocode}
\def\HoLogoCss@SliTeX@narrow{%
  \Css{%
    span.HoLogo-SliTeX-narrow span.HoLogo-l{%
      margin-left:-.06em;%
      margin-right:-.035em;%
      font-variant:small-caps;%
    }%
  }%
  \Css{%
    span.HoLogo-SliTeX-narrow span.HoLogo-i{%
      margin-right:-.06em;%
      font-variant:small-caps;%
    }%
  }%
  \global\let\HoLogoCss@SliTeX@narrow\relax
}
%    \end{macrocode}
%    \end{macro}
%
% \paragraph{Macro set completion.}
%
%    \begin{macro}{\HoLogo@SLiTeX@simple}
%    \begin{macrocode}
\def\HoLogo@SLiTeX@simple{\HoLogo@SliTeX@simple}
%    \end{macrocode}
%    \end{macro}
%    \begin{macro}{\HoLogoBkm@SLiTeX@simple}
%    \begin{macrocode}
\def\HoLogoBkm@SLiTeX@simple{\HoLogoBkm@SliTeX@simple}
%    \end{macrocode}
%    \end{macro}
%    \begin{macro}{\HoLogoHtml@SLiTeX@simple}
%    \begin{macrocode}
\def\HoLogoHtml@SLiTeX@simple{\HoLogoHtml@SliTeX@simple}
%    \end{macrocode}
%    \end{macro}
%
%    \begin{macro}{\HoLogo@SLiTeX@narrow}
%    \begin{macrocode}
\def\HoLogo@SLiTeX@narrow{\HoLogo@SliTeX@narrow}
%    \end{macrocode}
%    \end{macro}
%    \begin{macro}{\HoLogoBkm@SLiTeX@narrow}
%    \begin{macrocode}
\def\HoLogoBkm@SLiTeX@narrow{\HoLogoBkm@SliTeX@narrow}
%    \end{macrocode}
%    \end{macro}
%    \begin{macro}{\HoLogoHtml@SLiTeX@narrow}
%    \begin{macrocode}
\def\HoLogoHtml@SLiTeX@narrow{\HoLogoHtml@SliTeX@narrow}
%    \end{macrocode}
%    \end{macro}
%
%    \begin{macro}{\HoLogo@SliTeX@lift}
%    \begin{macrocode}
\def\HoLogo@SliTeX@lift{\HoLogo@SLiTeX@lift}
%    \end{macrocode}
%    \end{macro}
%    \begin{macro}{\HoLogoBkm@SliTeX@lift}
%    \begin{macrocode}
\def\HoLogoBkm@SliTeX@lift{\HoLogoBkm@SLiTeX@lift}
%    \end{macrocode}
%    \end{macro}
%    \begin{macro}{\HoLogoHtml@SliTeX@lift}
%    \begin{macrocode}
\def\HoLogoHtml@SliTeX@lift{\HoLogoHtml@SLiTeX@lift}
%    \end{macrocode}
%    \end{macro}
%
% \paragraph{Defaults.}
%
%    \begin{macro}{\HoLogo@SLiTeX}
%    \begin{macrocode}
\def\HoLogo@SLiTeX{\HoLogo@SLiTeX@lift}
%    \end{macrocode}
%    \end{macro}
%    \begin{macro}{\HoLogoBkm@SLiTeX}
%    \begin{macrocode}
\def\HoLogoBkm@SLiTeX{\HoLogoBkm@SLiTeX@lift}
%    \end{macrocode}
%    \end{macro}
%    \begin{macro}{\HoLogoHtml@SLiTeX}
%    \begin{macrocode}
\def\HoLogoHtml@SLiTeX{\HoLogoHtml@SLiTeX@lift}
%    \end{macrocode}
%    \end{macro}
%
%    \begin{macro}{\HoLogo@SliTeX}
%    \begin{macrocode}
\def\HoLogo@SliTeX{\HoLogo@SliTeX@narrow}
%    \end{macrocode}
%    \end{macro}
%    \begin{macro}{\HoLogoBkm@SliTeX}
%    \begin{macrocode}
\def\HoLogoBkm@SliTeX{\HoLogoBkm@SliTeX@narrow}
%    \end{macrocode}
%    \end{macro}
%    \begin{macro}{\HoLogoHtml@SliTeX}
%    \begin{macrocode}
\def\HoLogoHtml@SliTeX{\HoLogoHtml@SliTeX@narrow}
%    \end{macrocode}
%    \end{macro}
%
% \subsubsection{\hologo{LuaTeX}}
%
%    \begin{macro}{\HoLogo@LuaTeX}
%    The kerning is an idea of Hans Hagen, see mailing list
%    `luatex at tug dot org' in March 2010.
%    \begin{macrocode}
\def\HoLogo@LuaTeX#1{%
  \HOLOGO@mbox{%
    Lua%
    \HOLOGO@NegativeKerning{aT,oT,To}%
    \hologo{TeX}%
  }%
}
%    \end{macrocode}
%    \end{macro}
%    \begin{macro}{\HoLogoHtml@LuaTeX}
%    \begin{macrocode}
\let\HoLogoHtml@LuaTeX\HoLogo@LuaTeX
%    \end{macrocode}
%    \end{macro}
%
% \subsubsection{\hologo{LuaLaTeX}}
%
%    \begin{macro}{\HoLogo@LuaLaTeX}
%    \begin{macrocode}
\def\HoLogo@LuaLaTeX#1{%
  \HOLOGO@mbox{%
    Lua%
    \hologo{LaTeX}%
  }%
}
%    \end{macrocode}
%    \end{macro}
%    \begin{macro}{\HoLogoHtml@LuaLaTeX}
%    \begin{macrocode}
\let\HoLogoHtml@LuaLaTeX\HoLogo@LuaLaTeX
%    \end{macrocode}
%    \end{macro}
%
% \subsubsection{\hologo{XeTeX}, \hologo{XeLaTeX}}
%
%    \begin{macro}{\HOLOGO@IfCharExists}
%    \begin{macrocode}
\ifluatex
  \ifnum\luatexversion<36 %
  \else
    \def\HOLOGO@IfCharExists#1{%
      \ifnum
        \directlua{%
           if luaotfload and luaotfload.aux then
             if luaotfload.aux.font_has_glyph(%
                    font.current(), \number#1) then % 	 
	       tex.print("1") % 	 
	     end % 	 
	   elseif font and font.fonts and font.current then %
            local f = font.fonts[font.current()]%
            if f.characters and f.characters[\number#1] then %
              tex.print("1")%
            end %
          end%
        }0=\ltx@zero
        \expandafter\ltx@secondoftwo
      \else
        \expandafter\ltx@firstoftwo
      \fi
    }%
  \fi
\fi
\ltx@IfUndefined{HOLOGO@IfCharExists}{%
  \def\HOLOGO@@IfCharExists#1{%
    \begingroup
      \tracinglostchars=\ltx@zero
      \setbox\ltx@zero=\hbox{%
        \kern7sp\char#1\relax
        \ifnum\lastkern>\ltx@zero
          \expandafter\aftergroup\csname iffalse\endcsname
        \else
          \expandafter\aftergroup\csname iftrue\endcsname
        \fi
      }%
      % \if{true|false} from \aftergroup
      \endgroup
      \expandafter\ltx@firstoftwo
    \else
      \endgroup
      \expandafter\ltx@secondoftwo
    \fi
  }%
  \ifxetex
    \ltx@IfUndefined{XeTeXfonttype}{}{%
      \ltx@IfUndefined{XeTeXcharglyph}{}{%
        \def\HOLOGO@IfCharExists#1{%
          \ifnum\XeTeXfonttype\font>\ltx@zero
            \expandafter\ltx@firstofthree
          \else
            \expandafter\ltx@gobble
          \fi
          {%
            \ifnum\XeTeXcharglyph#1>\ltx@zero
              \expandafter\ltx@firstoftwo
            \else
              \expandafter\ltx@secondoftwo
            \fi
          }%
          \HOLOGO@@IfCharExists{#1}%
        }%
      }%
    }%
  \fi
}{}
\ltx@ifundefined{HOLOGO@IfCharExists}{%
  \ifnum64=`\^^^^0040\relax % test for big chars of LuaTeX/XeTeX
    \let\HOLOGO@IfCharExists\HOLOGO@@IfCharExists
  \else
    \def\HOLOGO@IfCharExists#1{%
      \ifnum#1>255 %
        \expandafter\ltx@fourthoffour
      \fi
      \HOLOGO@@IfCharExists{#1}%
    }%
  \fi
}{}
%    \end{macrocode}
%    \end{macro}
%
%    \begin{macro}{\HoLogo@Xe}
%    Source: package \xpackage{dtklogos}
%    \begin{macrocode}
\def\HoLogo@Xe#1{%
  X%
  \kern-.1em\relax
  \HOLOGO@IfCharExists{"018E}{%
    \lower.5ex\hbox{\char"018E}%
  }{%
    \chardef\HOLOGO@choice=\ltx@zero
    \ifdim\fontdimen\ltx@one\font>0pt %
      \ltx@IfUndefined{rotatebox}{%
        \ltx@IfUndefined{pgftext}{%
          \ltx@IfUndefined{psscalebox}{%
            \ltx@IfUndefined{HOLOGO@ScaleBox@\hologoDriver}{%
            }{%
              \chardef\HOLOGO@choice=4 %
            }%
          }{%
            \chardef\HOLOGO@choice=3 %
          }%
        }{%
          \chardef\HOLOGO@choice=2 %
        }%
      }{%
        \chardef\HOLOGO@choice=1 %
      }%
      \ifcase\HOLOGO@choice
        \HOLOGO@WarningUnsupportedDriver{Xe}%
        e%
      \or % 1: \rotatebox
        \begingroup
          \setbox\ltx@zero\hbox{\rotatebox{180}{E}}%
          \ltx@LocDimenA=\dp\ltx@zero
          \advance\ltx@LocDimenA by -.5ex\relax
          \raise\ltx@LocDimenA\box\ltx@zero
        \endgroup
      \or % 2: \pgftext
        \lower.5ex\hbox{%
          \pgfpicture
            \pgftext[rotate=180]{E}%
          \endpgfpicture
        }%
      \or % 3: \psscalebox
        \begingroup
          \setbox\ltx@zero\hbox{\psscalebox{-1 -1}{E}}%
          \ltx@LocDimenA=\dp\ltx@zero
          \advance\ltx@LocDimenA by -.5ex\relax
          \raise\ltx@LocDimenA\box\ltx@zero
        \endgroup
      \or % 4: \HOLOGO@PointReflectBox
        \lower.5ex\hbox{\HOLOGO@PointReflectBox{E}}%
      \else
        \@PackageError{hologo}{Internal error (choice/it}\@ehc
      \fi
    \else
      \ltx@IfUndefined{reflectbox}{%
        \ltx@IfUndefined{pgftext}{%
          \ltx@IfUndefined{psscalebox}{%
            \ltx@IfUndefined{HOLOGO@ScaleBox@\hologoDriver}{%
            }{%
              \chardef\HOLOGO@choice=4 %
            }%
          }{%
            \chardef\HOLOGO@choice=3 %
          }%
        }{%
          \chardef\HOLOGO@choice=2 %
        }%
      }{%
        \chardef\HOLOGO@choice=1 %
      }%
      \ifcase\HOLOGO@choice
        \HOLOGO@WarningUnsupportedDriver{Xe}%
        e%
      \or % 1: reflectbox
        \lower.5ex\hbox{%
          \reflectbox{E}%
        }%
      \or % 2: \pgftext
        \lower.5ex\hbox{%
          \pgfpicture
            \pgftransformxscale{-1}%
            \pgftext{E}%
          \endpgfpicture
        }%
      \or % 3: \psscalebox
        \lower.5ex\hbox{%
          \psscalebox{-1 1}{E}%
        }%
      \or % 4: \HOLOGO@Reflectbox
        \lower.5ex\hbox{%
          \HOLOGO@ReflectBox{E}%
        }%
      \else
        \@PackageError{hologo}{Internal error (choice/up)}\@ehc
      \fi
    \fi
  }%
}
%    \end{macrocode}
%    \end{macro}
%    \begin{macro}{\HoLogoHtml@Xe}
%    \begin{macrocode}
\def\HoLogoHtml@Xe#1{%
  \HoLogoCss@Xe
  \HOLOGO@Span{Xe}{%
    X%
    \HOLOGO@Span{e}{%
      \HCode{&\ltx@hashchar x018e;}%
    }%
  }%
}
%    \end{macrocode}
%    \end{macro}
%    \begin{macro}{\HoLogoCss@Xe}
%    \begin{macrocode}
\def\HoLogoCss@Xe{%
  \Css{%
    span.HoLogo-Xe span.HoLogo-e{%
      position:relative;%
      top:.5ex;%
      left-margin:-.1em;%
    }%
  }%
  \global\let\HoLogoCss@Xe\relax
}
%    \end{macrocode}
%    \end{macro}
%
%    \begin{macro}{\HoLogo@XeTeX}
%    \begin{macrocode}
\def\HoLogo@XeTeX#1{%
  \hologo{Xe}%
  \kern-.15em\relax
  \hologo{TeX}%
}
%    \end{macrocode}
%    \end{macro}
%
%    \begin{macro}{\HoLogoHtml@XeTeX}
%    \begin{macrocode}
\def\HoLogoHtml@XeTeX#1{%
  \HoLogoCss@XeTeX
  \HOLOGO@Span{XeTeX}{%
    \hologo{Xe}%
    \hologo{TeX}%
  }%
}
%    \end{macrocode}
%    \end{macro}
%    \begin{macro}{\HoLogoCss@XeTeX}
%    \begin{macrocode}
\def\HoLogoCss@XeTeX{%
  \Css{%
    span.HoLogo-XeTeX span.HoLogo-TeX{%
      margin-left:-.15em;%
    }%
  }%
  \global\let\HoLogoCss@XeTeX\relax
}
%    \end{macrocode}
%    \end{macro}
%
%    \begin{macro}{\HoLogo@XeLaTeX}
%    \begin{macrocode}
\def\HoLogo@XeLaTeX#1{%
  \hologo{Xe}%
  \kern-.13em%
  \hologo{LaTeX}%
}
%    \end{macrocode}
%    \end{macro}
%    \begin{macro}{\HoLogoHtml@XeLaTeX}
%    \begin{macrocode}
\def\HoLogoHtml@XeLaTeX#1{%
  \HoLogoCss@XeLaTeX
  \HOLOGO@Span{XeLaTeX}{%
    \hologo{Xe}%
    \hologo{LaTeX}%
  }%
}
%    \end{macrocode}
%    \end{macro}
%    \begin{macro}{\HoLogoCss@XeLaTeX}
%    \begin{macrocode}
\def\HoLogoCss@XeLaTeX{%
  \Css{%
    span.HoLogo-XeLaTeX span.HoLogo-Xe{%
      margin-right:-.13em;%
    }%
  }%
  \global\let\HoLogoCss@XeLaTeX\relax
}
%    \end{macrocode}
%    \end{macro}
%
% \subsubsection{\hologo{pdfTeX}, \hologo{pdfLaTeX}}
%
%    \begin{macro}{\HoLogo@pdfTeX}
%    \begin{macrocode}
\def\HoLogo@pdfTeX#1{%
  \HOLOGO@mbox{%
    #1{p}{P}df\hologo{TeX}%
  }%
}
%    \end{macrocode}
%    \end{macro}
%    \begin{macro}{\HoLogoCs@pdfTeX}
%    \begin{macrocode}
\def\HoLogoCs@pdfTeX#1{#1{p}{P}dfTeX}
%    \end{macrocode}
%    \end{macro}
%    \begin{macro}{\HoLogoBkm@pdfTeX}
%    \begin{macrocode}
\def\HoLogoBkm@pdfTeX#1{%
  #1{p}{P}df\hologo{TeX}%
}
%    \end{macrocode}
%    \end{macro}
%    \begin{macro}{\HoLogoHtml@pdfTeX}
%    \begin{macrocode}
\let\HoLogoHtml@pdfTeX\HoLogo@pdfTeX
%    \end{macrocode}
%    \end{macro}
%
%    \begin{macro}{\HoLogo@pdfLaTeX}
%    \begin{macrocode}
\def\HoLogo@pdfLaTeX#1{%
  \HOLOGO@mbox{%
    #1{p}{P}df\hologo{LaTeX}%
  }%
}
%    \end{macrocode}
%    \end{macro}
%    \begin{macro}{\HoLogoCs@pdfLaTeX}
%    \begin{macrocode}
\def\HoLogoCs@pdfLaTeX#1{#1{p}{P}dfLaTeX}
%    \end{macrocode}
%    \end{macro}
%    \begin{macro}{\HoLogoBkm@pdfLaTeX}
%    \begin{macrocode}
\def\HoLogoBkm@pdfLaTeX#1{%
  #1{p}{P}df\hologo{LaTeX}%
}
%    \end{macrocode}
%    \end{macro}
%    \begin{macro}{\HoLogoHtml@pdfLaTeX}
%    \begin{macrocode}
\let\HoLogoHtml@pdfLaTeX\HoLogo@pdfLaTeX
%    \end{macrocode}
%    \end{macro}
%
% \subsubsection{\hologo{VTeX}}
%
%    \begin{macro}{\HoLogo@VTeX}
%    \begin{macrocode}
\def\HoLogo@VTeX#1{%
  \HOLOGO@mbox{%
    V\hologo{TeX}%
  }%
}
%    \end{macrocode}
%    \end{macro}
%    \begin{macro}{\HoLogoHtml@VTeX}
%    \begin{macrocode}
\let\HoLogoHtml@VTeX\HoLogo@VTeX
%    \end{macrocode}
%    \end{macro}
%
% \subsubsection{\hologo{AmS}, \dots}
%
%    Source: class \xclass{amsdtx}
%
%    \begin{macro}{\HoLogo@AmS}
%    \begin{macrocode}
\def\HoLogo@AmS#1{%
  \HoLogoFont@font{AmS}{sy}{%
    A%
    \kern-.1667em%
    \lower.5ex\hbox{M}%
    \kern-.125em%
    S%
  }%
}
%    \end{macrocode}
%    \end{macro}
%    \begin{macro}{\HoLogoBkm@AmS}
%    \begin{macrocode}
\def\HoLogoBkm@AmS#1{AmS}
%    \end{macrocode}
%    \end{macro}
%    \begin{macro}{\HoLogoHtml@AmS}
%    \begin{macrocode}
\def\HoLogoHtml@AmS#1{%
  \HoLogoCss@AmS
%  \HoLogoFont@font{AmS}{sy}{%
    \HOLOGO@Span{AmS}{%
      A%
      \HOLOGO@Span{M}{M}%
      S%
    }%
%   }%
}
%    \end{macrocode}
%    \end{macro}
%    \begin{macro}{\HoLogoCss@AmS}
%    \begin{macrocode}
\def\HoLogoCss@AmS{%
  \Css{%
    span.HoLogo-AmS span.HoLogo-M{%
      position:relative;%
      top:.5ex;%
      margin-left:-.1667em;%
      margin-right:-.125em;%
      text-decoration:none;%
    }%
  }%
  \global\let\HoLogoCss@AmS\relax
}
%    \end{macrocode}
%    \end{macro}
%
%    \begin{macro}{\HoLogo@AmSTeX}
%    \begin{macrocode}
\def\HoLogo@AmSTeX#1{%
  \hologo{AmS}%
  \HOLOGO@hyphen
  \hologo{TeX}%
}
%    \end{macrocode}
%    \end{macro}
%    \begin{macro}{\HoLogoBkm@AmSTeX}
%    \begin{macrocode}
\def\HoLogoBkm@AmSTeX#1{AmS-TeX}%
%    \end{macrocode}
%    \end{macro}
%    \begin{macro}{\HoLogoHtml@AmSTeX}
%    \begin{macrocode}
\let\HoLogoHtml@AmSTeX\HoLogo@AmSTeX
%    \end{macrocode}
%    \end{macro}
%
%    \begin{macro}{\HoLogo@AmSLaTeX}
%    \begin{macrocode}
\def\HoLogo@AmSLaTeX#1{%
  \hologo{AmS}%
  \HOLOGO@hyphen
  \hologo{LaTeX}%
}
%    \end{macrocode}
%    \end{macro}
%    \begin{macro}{\HoLogoBkm@AmSLaTeX}
%    \begin{macrocode}
\def\HoLogoBkm@AmSLaTeX#1{AmS-LaTeX}%
%    \end{macrocode}
%    \end{macro}
%    \begin{macro}{\HoLogoHtml@AmSLaTeX}
%    \begin{macrocode}
\let\HoLogoHtml@AmSLaTeX\HoLogo@AmSLaTeX
%    \end{macrocode}
%    \end{macro}
%
% \subsubsection{\hologo{BibTeX}}
%
%    \begin{macro}{\HoLogo@BibTeX@sc}
%    A definition of \hologo{BibTeX} is provided in
%    the documentation source for the manual of \hologo{BibTeX}
%    \cite{btxdoc}.
%\begin{quote}
%\begin{verbatim}
%\def\BibTeX{%
%  {%
%    \rm
%    B%
%    \kern-.05em%
%    {%
%      \sc
%      i%
%      \kern-.025em %
%      b%
%    }%
%    \kern-.08em
%    T%
%    \kern-.1667em%
%    \lower.7ex\hbox{E}%
%    \kern-.125em%
%    X%
%  }%
%}
%\end{verbatim}
%\end{quote}
%    \begin{macrocode}
\def\HoLogo@BibTeX@sc#1{%
  B%
  \kern-.05em%
  \HoLogoFont@font{BibTeX}{sc}{%
    i%
    \kern-.025em%
    b%
  }%
  \HOLOGO@discretionary
  \kern-.08em%
  \hologo{TeX}%
}
%    \end{macrocode}
%    \end{macro}
%    \begin{macro}{\HoLogoHtml@BibTeX@sc}
%    \begin{macrocode}
\def\HoLogoHtml@BibTeX@sc#1{%
  \HoLogoCss@BibTeX@sc
  \HOLOGO@Span{BibTeX-sc}{%
    B%
    \HOLOGO@Span{i}{i}%
    \HOLOGO@Span{b}{b}%
    \hologo{TeX}%
  }%
}
%    \end{macrocode}
%    \end{macro}
%    \begin{macro}{\HoLogoCss@BibTeX@sc}
%    \begin{macrocode}
\def\HoLogoCss@BibTeX@sc{%
  \Css{%
    span.HoLogo-BibTeX-sc span.HoLogo-i{%
      margin-left:-.05em;%
      margin-right:-.025em;%
      font-variant:small-caps;%
    }%
  }%
  \Css{%
    span.HoLogo-BibTeX-sc span.HoLogo-b{%
      margin-right:-.08em;%
      font-variant:small-caps;%
    }%
  }%
  \global\let\HoLogoCss@BibTeX@sc\relax
}
%    \end{macrocode}
%    \end{macro}
%
%    \begin{macro}{\HoLogo@BibTeX@sf}
%    Variant \xoption{sf} avoids trouble with unavailable
%    small caps fonts (e.g., bold versions of Computer Modern or
%    Latin Modern). The definition is taken from
%    package \xpackage{dtklogos} \cite{dtklogos}.
%\begin{quote}
%\begin{verbatim}
%\DeclareRobustCommand{\BibTeX}{%
%  B%
%  \kern-.05em%
%  \hbox{%
%    $\m@th$% %% force math size calculations
%    \csname S@\f@size\endcsname
%    \fontsize\sf@size\z@
%    \math@fontsfalse
%    \selectfont
%    I%
%    \kern-.025em%
%    B
%  }%
%  \kern-.08em%
%  \-%
%  \TeX
%}
%\end{verbatim}
%\end{quote}
%    \begin{macrocode}
\def\HoLogo@BibTeX@sf#1{%
  B%
  \kern-.05em%
  \HoLogoFont@font{BibTeX}{bibsf}{%
    I%
    \kern-.025em%
    B%
  }%
  \HOLOGO@discretionary
  \kern-.08em%
  \hologo{TeX}%
}
%    \end{macrocode}
%    \end{macro}
%    \begin{macro}{\HoLogoHtml@BibTeX@sf}
%    \begin{macrocode}
\def\HoLogoHtml@BibTeX@sf#1{%
  \HoLogoCss@BibTeX@sf
  \HOLOGO@Span{BibTeX-sf}{%
    B%
    \HoLogoFont@font{BibTeX}{bibsf}{%
      \HOLOGO@Span{i}{I}%
      B%
    }%
    \hologo{TeX}%
  }%
}
%    \end{macrocode}
%    \end{macro}
%    \begin{macro}{\HoLogoCss@BibTeX@sf}
%    \begin{macrocode}
\def\HoLogoCss@BibTeX@sf{%
  \Css{%
    span.HoLogo-BibTeX-sf span.HoLogo-i{%
      margin-left:-.05em;%
      margin-right:-.025em;%
    }%
  }%
  \Css{%
    span.HoLogo-BibTeX-sf span.HoLogo-TeX{%
      margin-left:-.08em;%
    }%
  }%
  \global\let\HoLogoCss@BibTeX@sf\relax
}
%    \end{macrocode}
%    \end{macro}
%
%    \begin{macro}{\HoLogo@BibTeX}
%    \begin{macrocode}
\def\HoLogo@BibTeX{\HoLogo@BibTeX@sf}
%    \end{macrocode}
%    \end{macro}
%    \begin{macro}{\HoLogoHtml@BibTeX}
%    \begin{macrocode}
\def\HoLogoHtml@BibTeX{\HoLogoHtml@BibTeX@sf}
%    \end{macrocode}
%    \end{macro}
%
% \subsubsection{\hologo{BibTeX8}}
%
%    \begin{macro}{\HoLogo@BibTeX8}
%    \begin{macrocode}
\expandafter\def\csname HoLogo@BibTeX8\endcsname#1{%
  \hologo{BibTeX}%
  8%
}
%    \end{macrocode}
%    \end{macro}
%
%    \begin{macro}{\HoLogoBkm@BibTeX8}
%    \begin{macrocode}
\expandafter\def\csname HoLogoBkm@BibTeX8\endcsname#1{%
  \hologo{BibTeX}%
  8%
}
%    \end{macrocode}
%    \end{macro}
%    \begin{macro}{\HoLogoHtml@BibTeX8}
%    \begin{macrocode}
\expandafter
\let\csname HoLogoHtml@BibTeX8\expandafter\endcsname
\csname HoLogo@BibTeX8\endcsname
%    \end{macrocode}
%    \end{macro}
%
% \subsubsection{\hologo{ConTeXt}}
%
%    \begin{macro}{\HoLogo@ConTeXt@simple}
%    \begin{macrocode}
\def\HoLogo@ConTeXt@simple#1{%
  \HOLOGO@mbox{Con}%
  \HOLOGO@discretionary
  \HOLOGO@mbox{\hologo{TeX}t}%
}
%    \end{macrocode}
%    \end{macro}
%    \begin{macro}{\HoLogoHtml@ConTeXt@simple}
%    \begin{macrocode}
\let\HoLogoHtml@ConTeXt@simple\HoLogo@ConTeXt@simple
%    \end{macrocode}
%    \end{macro}
%
%    \begin{macro}{\HoLogo@ConTeXt@narrow}
%    This definition of logo \hologo{ConTeXt} with variant \xoption{narrow}
%    comes from TUGboat's class \xclass{ltugboat} (version 2010/11/15 v2.8).
%    \begin{macrocode}
\def\HoLogo@ConTeXt@narrow#1{%
  \HOLOGO@mbox{C\kern-.0333emon}%
  \HOLOGO@discretionary
  \kern-.0667em%
  \HOLOGO@mbox{\hologo{TeX}\kern-.0333emt}%
}
%    \end{macrocode}
%    \end{macro}
%    \begin{macro}{\HoLogoHtml@ConTeXt@narrow}
%    \begin{macrocode}
\def\HoLogoHtml@ConTeXt@narrow#1{%
  \HoLogoCss@ConTeXt@narrow
  \HOLOGO@Span{ConTeXt-narrow}{%
    \HOLOGO@Span{C}{C}%
    on%
    \hologo{TeX}%
    t%
  }%
}
%    \end{macrocode}
%    \end{macro}
%    \begin{macro}{\HoLogoCss@ConTeXt@narrow}
%    \begin{macrocode}
\def\HoLogoCss@ConTeXt@narrow{%
  \Css{%
    span.HoLogo-ConTeXt-narrow span.HoLogo-C{%
      margin-left:-.0333em;%
    }%
  }%
  \Css{%
    span.HoLogo-ConTeXt-narrow span.HoLogo-TeX{%
      margin-left:-.0667em;%
      margin-right:-.0333em;%
    }%
  }%
  \global\let\HoLogoCss@ConTeXt@narrow\relax
}
%    \end{macrocode}
%    \end{macro}
%
%    \begin{macro}{\HoLogo@ConTeXt}
%    \begin{macrocode}
\def\HoLogo@ConTeXt{\HoLogo@ConTeXt@narrow}
%    \end{macrocode}
%    \end{macro}
%    \begin{macro}{\HoLogoHtml@ConTeXt}
%    \begin{macrocode}
\def\HoLogoHtml@ConTeXt{\HoLogoHtml@ConTeXt@narrow}
%    \end{macrocode}
%    \end{macro}
%
% \subsubsection{\hologo{emTeX}}
%
%    \begin{macro}{\HoLogo@emTeX}
%    \begin{macrocode}
\def\HoLogo@emTeX#1{%
  \HOLOGO@mbox{#1{e}{E}m}%
  \HOLOGO@discretionary
  \hologo{TeX}%
}
%    \end{macrocode}
%    \end{macro}
%    \begin{macro}{\HoLogoCs@emTeX}
%    \begin{macrocode}
\def\HoLogoCs@emTeX#1{#1{e}{E}mTeX}%
%    \end{macrocode}
%    \end{macro}
%    \begin{macro}{\HoLogoBkm@emTeX}
%    \begin{macrocode}
\def\HoLogoBkm@emTeX#1{%
  #1{e}{E}m\hologo{TeX}%
}
%    \end{macrocode}
%    \end{macro}
%    \begin{macro}{\HoLogoHtml@emTeX}
%    \begin{macrocode}
\let\HoLogoHtml@emTeX\HoLogo@emTeX
%    \end{macrocode}
%    \end{macro}
%
% \subsubsection{\hologo{ExTeX}}
%
%    \begin{macro}{\HoLogo@ExTeX}
%    The definition is taken from the FAQ of the
%    project \hologo{ExTeX}
%    \cite{ExTeX-FAQ}.
%\begin{quote}
%\begin{verbatim}
%\def\ExTeX{%
%  \textrm{% Logo always with serifs
%    \ensuremath{%
%      \textstyle
%      \varepsilon_{%
%        \kern-0.15em%
%        \mathcal{X}%
%      }%
%    }%
%    \kern-.15em%
%    \TeX
%  }%
%}
%\end{verbatim}
%\end{quote}
%    \begin{macrocode}
\def\HoLogo@ExTeX#1{%
  \HoLogoFont@font{ExTeX}{rm}{%
    \ltx@mbox{%
      \HOLOGO@MathSetup
      $%
        \textstyle
        \varepsilon_{%
          \kern-0.15em%
          \HoLogoFont@font{ExTeX}{sy}{X}%
        }%
      $%
    }%
    \HOLOGO@discretionary
    \kern-.15em%
    \hologo{TeX}%
  }%
}
%    \end{macrocode}
%    \end{macro}
%    \begin{macro}{\HoLogoHtml@ExTeX}
%    \begin{macrocode}
\def\HoLogoHtml@ExTeX#1{%
  \HoLogoCss@ExTeX
  \HoLogoFont@font{ExTeX}{rm}{%
    \HOLOGO@Span{ExTeX}{%
      \ltx@mbox{%
        \HOLOGO@MathSetup
        $\textstyle\varepsilon$%
        \HOLOGO@Span{X}{$\textstyle\chi$}%
        \hologo{TeX}%
      }%
    }%
  }%
}
%    \end{macrocode}
%    \end{macro}
%    \begin{macro}{\HoLogoBkm@ExTeX}
%    \begin{macrocode}
\def\HoLogoBkm@ExTeX#1{%
  \HOLOGO@PdfdocUnicode{#1{e}{E}x}{\textepsilon\textchi}%
  \hologo{TeX}%
}
%    \end{macrocode}
%    \end{macro}
%    \begin{macro}{\HoLogoCss@ExTeX}
%    \begin{macrocode}
\def\HoLogoCss@ExTeX{%
  \Css{%
    span.HoLogo-ExTeX{%
      font-family:serif;%
    }%
  }%
  \Css{%
    span.HoLogo-ExTeX span.HoLogo-TeX{%
      margin-left:-.15em;%
    }%
  }%
  \global\let\HoLogoCss@ExTeX\relax
}
%    \end{macrocode}
%    \end{macro}
%
% \subsubsection{\hologo{MiKTeX}}
%
%    \begin{macro}{\HoLogo@MiKTeX}
%    \begin{macrocode}
\def\HoLogo@MiKTeX#1{%
  \HOLOGO@mbox{MiK}%
  \HOLOGO@discretionary
  \hologo{TeX}%
}
%    \end{macrocode}
%    \end{macro}
%    \begin{macro}{\HoLogoHtml@MiKTeX}
%    \begin{macrocode}
\let\HoLogoHtml@MiKTeX\HoLogo@MiKTeX
%    \end{macrocode}
%    \end{macro}
%
% \subsubsection{\hologo{OzTeX} and friends}
%
%    Source: \hologo{OzTeX} FAQ \cite{OzTeX}:
%    \begin{quote}
%      |\def\OzTeX{O\kern-.03em z\kern-.15em\TeX}|\\
%      (There is no kerning in OzMF, OzMP and OzTtH.)
%    \end{quote}
%
%    \begin{macro}{\HoLogo@OzTeX}
%    \begin{macrocode}
\def\HoLogo@OzTeX#1{%
  O%
  \kern-.03em %
  z%
  \kern-.15em %
  \hologo{TeX}%
}
%    \end{macrocode}
%    \end{macro}
%    \begin{macro}{\HoLogoHtml@OzTeX}
%    \begin{macrocode}
\def\HoLogoHtml@OzTeX#1{%
  \HoLogoCss@OzTeX
  \HOLOGO@Span{OzTeX}{%
    O%
    \HOLOGO@Span{z}{z}%
    \hologo{TeX}%
  }%
}
%    \end{macrocode}
%    \end{macro}
%    \begin{macro}{\HoLogoCss@OzTeX}
%    \begin{macrocode}
\def\HoLogoCss@OzTeX{%
  \Css{%
    span.HoLogo-OzTeX span.HoLogo-z{%
      margin-left:-.03em;%
      margin-right:-.15em;%
    }%
  }%
  \global\let\HoLogoCss@OzTeX\relax
}
%    \end{macrocode}
%    \end{macro}
%
%    \begin{macro}{\HoLogo@OzMF}
%    \begin{macrocode}
\def\HoLogo@OzMF#1{%
  \HOLOGO@mbox{OzMF}%
}
%    \end{macrocode}
%    \end{macro}
%    \begin{macro}{\HoLogo@OzMP}
%    \begin{macrocode}
\def\HoLogo@OzMP#1{%
  \HOLOGO@mbox{OzMP}%
}
%    \end{macrocode}
%    \end{macro}
%    \begin{macro}{\HoLogo@OzTtH}
%    \begin{macrocode}
\def\HoLogo@OzTtH#1{%
  \HOLOGO@mbox{OzTtH}%
}
%    \end{macrocode}
%    \end{macro}
%
% \subsubsection{\hologo{PCTeX}}
%
%    \begin{macro}{\HoLogo@PCTeX}
%    \begin{macrocode}
\def\HoLogo@PCTeX#1{%
  \HOLOGO@mbox{PC}%
  \hologo{TeX}%
}
%    \end{macrocode}
%    \end{macro}
%    \begin{macro}{\HoLogoHtml@PCTeX}
%    \begin{macrocode}
\let\HoLogoHtml@PCTeX\HoLogo@PCTeX
%    \end{macrocode}
%    \end{macro}
%
% \subsubsection{\hologo{PiCTeX}}
%
%    The original definitions from \xfile{pictex.tex} \cite{PiCTeX}:
%\begin{quote}
%\begin{verbatim}
%\def\PiC{%
%  P%
%  \kern-.12em%
%  \lower.5ex\hbox{I}%
%  \kern-.075em%
%  C%
%}
%\def\PiCTeX{%
%  \PiC
%  \kern-.11em%
%  \TeX
%}
%\end{verbatim}
%\end{quote}
%
%    \begin{macro}{\HoLogo@PiC}
%    \begin{macrocode}
\def\HoLogo@PiC#1{%
  P%
  \kern-.12em%
  \lower.5ex\hbox{I}%
  \kern-.075em%
  C%
  \HOLOGO@SpaceFactor
}
%    \end{macrocode}
%    \end{macro}
%    \begin{macro}{\HoLogoHtml@PiC}
%    \begin{macrocode}
\def\HoLogoHtml@PiC#1{%
  \HoLogoCss@PiC
  \HOLOGO@Span{PiC}{%
    P%
    \HOLOGO@Span{i}{I}%
    C%
  }%
}
%    \end{macrocode}
%    \end{macro}
%    \begin{macro}{\HoLogoCss@PiC}
%    \begin{macrocode}
\def\HoLogoCss@PiC{%
  \Css{%
    span.HoLogo-PiC span.HoLogo-i{%
      position:relative;%
      top:.5ex;%
      margin-left:-.12em;%
      margin-right:-.075em;%
      text-decoration:none;%
    }%
  }%
  \global\let\HoLogoCss@PiC\relax
}
%    \end{macrocode}
%    \end{macro}
%
%    \begin{macro}{\HoLogo@PiCTeX}
%    \begin{macrocode}
\def\HoLogo@PiCTeX#1{%
  \hologo{PiC}%
  \HOLOGO@discretionary
  \kern-.11em%
  \hologo{TeX}%
}
%    \end{macrocode}
%    \end{macro}
%    \begin{macro}{\HoLogoHtml@PiCTeX}
%    \begin{macrocode}
\def\HoLogoHtml@PiCTeX#1{%
  \HoLogoCss@PiCTeX
  \HOLOGO@Span{PiCTeX}{%
    \hologo{PiC}%
    \hologo{TeX}%
  }%
}
%    \end{macrocode}
%    \end{macro}
%    \begin{macro}{\HoLogoCss@PiCTeX}
%    \begin{macrocode}
\def\HoLogoCss@PiCTeX{%
  \Css{%
    span.HoLogo-PiCTeX span.HoLogo-PiC{%
      margin-right:-.11em;%
    }%
  }%
  \global\let\HoLogoCss@PiCTeX\relax
}
%    \end{macrocode}
%    \end{macro}
%
% \subsubsection{\hologo{teTeX}}
%
%    \begin{macro}{\HoLogo@teTeX}
%    \begin{macrocode}
\def\HoLogo@teTeX#1{%
  \HOLOGO@mbox{#1{t}{T}e}%
  \HOLOGO@discretionary
  \hologo{TeX}%
}
%    \end{macrocode}
%    \end{macro}
%    \begin{macro}{\HoLogoCs@teTeX}
%    \begin{macrocode}
\def\HoLogoCs@teTeX#1{#1{t}{T}dfTeX}
%    \end{macrocode}
%    \end{macro}
%    \begin{macro}{\HoLogoBkm@teTeX}
%    \begin{macrocode}
\def\HoLogoBkm@teTeX#1{%
  #1{t}{T}e\hologo{TeX}%
}
%    \end{macrocode}
%    \end{macro}
%    \begin{macro}{\HoLogoHtml@teTeX}
%    \begin{macrocode}
\let\HoLogoHtml@teTeX\HoLogo@teTeX
%    \end{macrocode}
%    \end{macro}
%
% \subsubsection{\hologo{TeX4ht}}
%
%    \begin{macro}{\HoLogo@TeX4ht}
%    \begin{macrocode}
\expandafter\def\csname HoLogo@TeX4ht\endcsname#1{%
  \HOLOGO@mbox{\hologo{TeX}4ht}%
}
%    \end{macrocode}
%    \end{macro}
%    \begin{macro}{\HoLogoHtml@TeX4ht}
%    \begin{macrocode}
\expandafter
\let\csname HoLogoHtml@TeX4ht\expandafter\endcsname
\csname HoLogo@TeX4ht\endcsname
%    \end{macrocode}
%    \end{macro}
%
%
% \subsubsection{\hologo{SageTeX}}
%
%    \begin{macro}{\HoLogo@SageTeX}
%    \begin{macrocode}
\def\HoLogo@SageTeX#1{%
  \HOLOGO@mbox{Sage}%
  \HOLOGO@discretionary
  \HOLOGO@NegativeKerning{eT,oT,To}%
  \hologo{TeX}%
}
%    \end{macrocode}
%    \end{macro}
%    \begin{macro}{\HoLogoHtml@SageTeX}
%    \begin{macrocode}
\let\HoLogoHtml@SageTeX\HoLogo@SageTeX
%    \end{macrocode}
%    \end{macro}
%
% \subsection{\hologo{METAFONT} and friends}
%
%    \begin{macro}{\HoLogo@METAFONT}
%    \begin{macrocode}
\def\HoLogo@METAFONT#1{%
  \HoLogoFont@font{METAFONT}{logo}{%
    \HOLOGO@mbox{META}%
    \HOLOGO@discretionary
    \HOLOGO@mbox{FONT}%
  }%
}
%    \end{macrocode}
%    \end{macro}
%
%    \begin{macro}{\HoLogo@METAPOST}
%    \begin{macrocode}
\def\HoLogo@METAPOST#1{%
  \HoLogoFont@font{METAPOST}{logo}{%
    \HOLOGO@mbox{META}%
    \HOLOGO@discretionary
    \HOLOGO@mbox{POST}%
  }%
}
%    \end{macrocode}
%    \end{macro}
%
%    \begin{macro}{\HoLogo@MetaFun}
%    \begin{macrocode}
\def\HoLogo@MetaFun#1{%
  \HOLOGO@mbox{Meta}%
  \HOLOGO@discretionary
  \HOLOGO@mbox{Fun}%
}
%    \end{macrocode}
%    \end{macro}
%
%    \begin{macro}{\HoLogo@MetaPost}
%    \begin{macrocode}
\def\HoLogo@MetaPost#1{%
  \HOLOGO@mbox{Meta}%
  \HOLOGO@discretionary
  \HOLOGO@mbox{Post}%
}
%    \end{macrocode}
%    \end{macro}
%
% \subsection{Others}
%
% \subsubsection{\hologo{biber}}
%
%    \begin{macro}{\HoLogo@biber}
%    \begin{macrocode}
\def\HoLogo@biber#1{%
  \HOLOGO@mbox{#1{b}{B}i}%
  \HOLOGO@discretionary
  \HOLOGO@mbox{ber}%
}
%    \end{macrocode}
%    \end{macro}
%    \begin{macro}{\HoLogoCs@biber}
%    \begin{macrocode}
\def\HoLogoCs@biber#1{#1{b}{B}iber}
%    \end{macrocode}
%    \end{macro}
%    \begin{macro}{\HoLogoBkm@biber}
%    \begin{macrocode}
\def\HoLogoBkm@biber#1{%
  #1{b}{B}iber%
}
%    \end{macrocode}
%    \end{macro}
%    \begin{macro}{\HoLogoHtml@biber}
%    \begin{macrocode}
\let\HoLogoHtml@biber\HoLogo@biber
%    \end{macrocode}
%    \end{macro}
%
% \subsubsection{\hologo{KOMAScript}}
%
%    \begin{macro}{\HoLogo@KOMAScript}
%    The definition for \hologo{KOMAScript} is taken
%    from \hologo{KOMAScript} (\xfile{scrlogo.dtx}, reformatted) \cite{scrlogo}:
%\begin{quote}
%\begin{verbatim}
%\@ifundefined{KOMAScript}{%
%  \DeclareRobustCommand{\KOMAScript}{%
%    \textsf{%
%      K\kern.05em O\kern.05emM\kern.05em A%
%      \kern.1em-\kern.1em %
%      Script%
%    }%
%  }%
%}{}
%\end{verbatim}
%\end{quote}
%    \begin{macrocode}
\def\HoLogo@KOMAScript#1{%
  \HoLogoFont@font{KOMAScript}{sf}{%
    \HOLOGO@mbox{%
      K\kern.05em%
      O\kern.05em%
      M\kern.05em%
      A%
    }%
    \kern.1em%
    \HOLOGO@hyphen
    \kern.1em%
    \HOLOGO@mbox{Script}%
  }%
}
%    \end{macrocode}
%    \end{macro}
%    \begin{macro}{\HoLogoBkm@KOMAScript}
%    \begin{macrocode}
\def\HoLogoBkm@KOMAScript#1{%
  KOMA-Script%
}
%    \end{macrocode}
%    \end{macro}
%    \begin{macro}{\HoLogoHtml@KOMAScript}
%    \begin{macrocode}
\def\HoLogoHtml@KOMAScript#1{%
  \HoLogoCss@KOMAScript
  \HoLogoFont@font{KOMAScript}{sf}{%
    \HOLOGO@Span{KOMAScript}{%
      K%
      \HOLOGO@Span{O}{O}%
      M%
      \HOLOGO@Span{A}{A}%
      \HOLOGO@Span{hyphen}{-}%
      Script%
    }%
  }%
}
%    \end{macrocode}
%    \end{macro}
%    \begin{macro}{\HoLogoCss@KOMAScript}
%    \begin{macrocode}
\def\HoLogoCss@KOMAScript{%
  \Css{%
    span.HoLogo-KOMAScript{%
      font-family:sans-serif;%
    }%
  }%
  \Css{%
    span.HoLogo-KOMAScript span.HoLogo-O{%
      padding-left:.05em;%
      padding-right:.05em;%
    }%
  }%
  \Css{%
    span.HoLogo-KOMAScript span.HoLogo-A{%
      padding-left:.05em;%
    }%
  }%
  \Css{%
    span.HoLogo-KOMAScript span.HoLogo-hyphen{%
      padding-left:.1em;%
      padding-right:.1em;%
    }%
  }%
  \global\let\HoLogoCss@KOMAScript\relax
}
%    \end{macrocode}
%    \end{macro}
%
% \subsubsection{\hologo{LyX}}
%
%    \begin{macro}{\HoLogo@LyX}
%    The definition is taken from the documentation source files
%    of \hologo{LyX}, \xfile{Intro.lyx} \cite{LyX}:
%\begin{quote}
%\begin{verbatim}
%\def\LyX{%
%  \texorpdfstring{%
%    L\kern-.1667em\lower.25em\hbox{Y}\kern-.125emX\@%
%  }{%
%    LyX%
%  }%
%}
%\end{verbatim}
%\end{quote}
%    \begin{macrocode}
\def\HoLogo@LyX#1{%
  L%
  \kern-.1667em%
  \lower.25em\hbox{Y}%
  \kern-.125em%
  X%
  \HOLOGO@SpaceFactor
}
%    \end{macrocode}
%    \end{macro}
%    \begin{macro}{\HoLogoHtml@LyX}
%    \begin{macrocode}
\def\HoLogoHtml@LyX#1{%
  \HoLogoCss@LyX
  \HOLOGO@Span{LyX}{%
    L%
    \HOLOGO@Span{y}{Y}%
    X%
  }%
}
%    \end{macrocode}
%    \end{macro}
%    \begin{macro}{\HoLogoCss@LyX}
%    \begin{macrocode}
\def\HoLogoCss@LyX{%
  \Css{%
    span.HoLogo-LyX span.HoLogo-y{%
      position:relative;%
      top:.25em;%
      margin-left:-.1667em;%
      margin-right:-.125em;%
      text-decoration:none;%
    }%
  }%
  \global\let\HoLogoCss@LyX\relax
}
%    \end{macrocode}
%    \end{macro}
%
% \subsubsection{\hologo{NTS}}
%
%    \begin{macro}{\HoLogo@NTS}
%    Definition for \hologo{NTS} can be found in
%    package \xpackage{etex\textunderscore man} for the \hologo{eTeX} manual \cite{etexman}
%    and in package \xpackage{dtklogos} \cite{dtklogos}:
%\begin{quote}
%\begin{verbatim}
%\def\NTS{%
%  \leavevmode
%  \hbox{%
%    $%
%      \cal N%
%      \kern-0.35em%
%      \lower0.5ex\hbox{$\cal T$}%
%      \kern-0.2em%
%      S%
%    $%
%  }%
%}
%\end{verbatim}
%\end{quote}
%    \begin{macrocode}
\def\HoLogo@NTS#1{%
  \HoLogoFont@font{NTS}{sy}{%
    N\/%
    \kern-.35em%
    \lower.5ex\hbox{T\/}%
    \kern-.2em%
    S\/%
  }%
  \HOLOGO@SpaceFactor
}
%    \end{macrocode}
%    \end{macro}
%
% \subsubsection{\Hologo{TTH} (\hologo{TeX} to HTML translator)}
%
%    Source: \url{http://hutchinson.belmont.ma.us/tth/}
%    In the HTML source the second `T' is printed as subscript.
%\begin{quote}
%\begin{verbatim}
%T<sub>T</sub>H
%\end{verbatim}
%\end{quote}
%    \begin{macro}{\HoLogo@TTH}
%    \begin{macrocode}
\def\HoLogo@TTH#1{%
  \ltx@mbox{%
    T\HOLOGO@SubScript{T}H%
  }%
  \HOLOGO@SpaceFactor
}
%    \end{macrocode}
%    \end{macro}
%
%    \begin{macro}{\HoLogoHtml@TTH}
%    \begin{macrocode}
\def\HoLogoHtml@TTH#1{%
  T\HCode{<sub>}T\HCode{</sub>}H%
}
%    \end{macrocode}
%    \end{macro}
%
% \subsubsection{\Hologo{HanTheThanh}}
%
%    Partial source: Package \xpackage{dtklogos}.
%    The double accent is U+1EBF (latin small letter e with circumflex
%    and acute).
%    \begin{macro}{\HoLogo@HanTheThanh}
%    \begin{macrocode}
\def\HoLogo@HanTheThanh#1{%
  \ltx@mbox{H\`an}%
  \HOLOGO@space
  \ltx@mbox{%
    Th%
    \HOLOGO@IfCharExists{"1EBF}{%
      \char"1EBF\relax
    }{%
      \^e\hbox to 0pt{\hss\raise .5ex\hbox{\'{}}}%
    }%
  }%
  \HOLOGO@space
  \ltx@mbox{Th\`anh}%
}
%    \end{macrocode}
%    \end{macro}
%    \begin{macro}{\HoLogoBkm@HanTheThanh}
%    \begin{macrocode}
\def\HoLogoBkm@HanTheThanh#1{%
  H\`an %
  Th\HOLOGO@PdfdocUnicode{\^e}{\9036\277} %
  Th\`anh%
}
%    \end{macrocode}
%    \end{macro}
%    \begin{macro}{\HoLogoHtml@HanTheThanh}
%    \begin{macrocode}
\def\HoLogoHtml@HanTheThanh#1{%
  H\`an %
  Th\HCode{&\ltx@hashchar x1ebf;} %
  Th\`anh%
}
%    \end{macrocode}
%    \end{macro}
%
% \subsection{Driver detection}
%
%    \begin{macrocode}
\HOLOGO@IfExists\InputIfFileExists{%
  \InputIfFileExists{hologo.cfg}{}{}%
}{%
  \ltx@IfUndefined{pdf@filesize}{%
    \def\HOLOGO@InputIfExists{%
      \openin\HOLOGO@temp=hologo.cfg\relax
      \ifeof\HOLOGO@temp
        \closein\HOLOGO@temp
      \else
        \closein\HOLOGO@temp
        \begingroup
          \def\x{LaTeX2e}%
        \expandafter\endgroup
        \ifx\fmtname\x
          \input{hologo.cfg}%
        \else
          \input hologo.cfg\relax
        \fi
      \fi
    }%
    \ltx@IfUndefined{newread}{%
      \chardef\HOLOGO@temp=15 %
      \def\HOLOGO@CheckRead{%
        \ifeof\HOLOGO@temp
          \HOLOGO@InputIfExists
        \else
          \ifcase\HOLOGO@temp
            \@PackageWarningNoLine{hologo}{%
              Configuration file ignored, because\MessageBreak
              a free read register could not be found%
            }%
          \else
            \begingroup
              \count\ltx@cclv=\HOLOGO@temp
              \advance\ltx@cclv by \ltx@minusone
              \edef\x{\endgroup
                \chardef\noexpand\HOLOGO@temp=\the\count\ltx@cclv
                \relax
              }%
            \x
          \fi
        \fi
      }%
    }{%
      \csname newread\endcsname\HOLOGO@temp
      \HOLOGO@InputIfExists
    }%
  }{%
    \edef\HOLOGO@temp{\pdf@filesize{hologo.cfg}}%
    \ifx\HOLOGO@temp\ltx@empty
    \else
      \ifnum\HOLOGO@temp>0 %
        \begingroup
          \def\x{LaTeX2e}%
        \expandafter\endgroup
        \ifx\fmtname\x
          \input{hologo.cfg}%
        \else
          \input hologo.cfg\relax
        \fi
      \else
        \@PackageInfoNoLine{hologo}{%
          Empty configuration file `hologo.cfg' ignored%
        }%
      \fi
    \fi
  }%
}
%    \end{macrocode}
%
%    \begin{macrocode}
\def\HOLOGO@temp#1#2{%
  \kv@define@key{HoLogoDriver}{#1}[]{%
    \begingroup
      \def\HOLOGO@temp{##1}%
      \ltx@onelevel@sanitize\HOLOGO@temp
      \ifx\HOLOGO@temp\ltx@empty
      \else
        \@PackageError{hologo}{%
          Value (\HOLOGO@temp) not permitted for option `#1'%
        }%
        \@ehc
      \fi
    \endgroup
    \def\hologoDriver{#2}%
  }%
}%
\def\HOLOGO@@temp#1#2{%
  \ifx\kv@value\relax
    \HOLOGO@temp{#1}{#1}%
  \else
    \HOLOGO@temp{#1}{#2}%
  \fi
}%
\kv@parse@normalized{%
  pdftex,%
  luatex=pdftex,%
  dvipdfm,%
  dvipdfmx=dvipdfm,%
  dvips,%
  dvipsone=dvips,%
  xdvi=dvips,%
  xetex,%
  vtex,%
}\HOLOGO@@temp
%    \end{macrocode}
%
%    \begin{macrocode}
\kv@define@key{HoLogoDriver}{driverfallback}{%
  \def\HOLOGO@DriverFallback{#1}%
}
%    \end{macrocode}
%
%    \begin{macro}{\HOLOGO@DriverFallback}
%    \begin{macrocode}
\def\HOLOGO@DriverFallback{dvips}
%    \end{macrocode}
%    \end{macro}
%
%    \begin{macro}{\hologoDriverSetup}
%    \begin{macrocode}
\def\hologoDriverSetup{%
  \let\hologoDriver\ltx@undefined
  \HOLOGO@DriverSetup
}
%    \end{macrocode}
%    \end{macro}
%
%    \begin{macro}{\HOLOGO@DriverSetup}
%    \begin{macrocode}
\def\HOLOGO@DriverSetup#1{%
  \kvsetkeys{HoLogoDriver}{#1}%
  \HOLOGO@CheckDriver
  \ltx@ifundefined{hologoDriver}{%
    \begingroup
    \edef\x{\endgroup
      \noexpand\kvsetkeys{HoLogoDriver}{\HOLOGO@DriverFallback}%
    }\x
  }{}%
  \@PackageInfoNoLine{hologo}{Using driver `\hologoDriver'}%
}
%    \end{macrocode}
%    \end{macro}
%
%    \begin{macro}{\HOLOGO@CheckDriver}
%    \begin{macrocode}
\def\HOLOGO@CheckDriver{%
  \ifpdf
    \def\hologoDriver{pdftex}%
    \let\HOLOGO@pdfliteral\pdfliteral
    \ifluatex
      \ifx\pdfextension\@undefined\else
        \protected\def\pdfliteral{\pdfextension literal}%
        \let\HOLOGO@pdfliteral\pdfliteral
      \fi
      \ltx@IfUndefined{HOLOGO@pdfliteral}{%
        \ifnum\luatexversion<36 %
        \else
          \begingroup
            \let\HOLOGO@temp\endgroup
            \ifcase0%
                \directlua{%
                  if tex.enableprimitives then %
                    tex.enableprimitives('HOLOGO@', {'pdfliteral'})%
                  else %
                    tex.print('1')%
                  end%
                }%
                \ifx\HOLOGO@pdfliteral\@undefined 1\fi%
                \relax%
              \endgroup
              \let\HOLOGO@temp\relax
              \global\let\HOLOGO@pdfliteral\HOLOGO@pdfliteral
            \fi%
          \HOLOGO@temp
        \fi
      }{}%
    \fi
    \ltx@IfUndefined{HOLOGO@pdfliteral}{%
      \@PackageWarningNoLine{hologo}{%
        Cannot find \string\pdfliteral
      }%
    }{}%
  \else
    \ifxetex
      \def\hologoDriver{xetex}%
    \else
      \ifvtex
        \def\hologoDriver{vtex}%
      \fi
    \fi
  \fi
}
%    \end{macrocode}
%    \end{macro}
%
%    \begin{macro}{\HOLOGO@WarningUnsupportedDriver}
%    \begin{macrocode}
\def\HOLOGO@WarningUnsupportedDriver#1{%
  \@PackageWarningNoLine{hologo}{%
    Logo `#1' needs driver specific macros,\MessageBreak
    but driver `\hologoDriver' is not supported.\MessageBreak
    Use a different driver or\MessageBreak
    load package `graphics' or `pgf'%
  }%
}
%    \end{macrocode}
%    \end{macro}
%
% \subsubsection{Reflect box macros}
%
%    Skip driver part if not needed.
%    \begin{macrocode}
\ltx@IfUndefined{reflectbox}{}{%
  \ltx@IfUndefined{rotatebox}{}{%
    \HOLOGO@AtEnd
  }%
}
\ltx@IfUndefined{pgftext}{}{%
  \HOLOGO@AtEnd
}
\ltx@IfUndefined{psscalebox}{}{%
  \HOLOGO@AtEnd
}
%    \end{macrocode}
%
%    \begin{macrocode}
\def\HOLOGO@temp{LaTeX2e}
\ifx\fmtname\HOLOGO@temp
  \RequirePackage{kvoptions}[2011/06/30]%
  \ProcessKeyvalOptions{HoLogoDriver}%
\fi
\HOLOGO@DriverSetup{}
%    \end{macrocode}
%
%    \begin{macro}{\HOLOGO@ReflectBox}
%    \begin{macrocode}
\def\HOLOGO@ReflectBox#1{%
  \begingroup
    \setbox\ltx@zero\hbox{\begingroup#1\endgroup}%
    \setbox\ltx@two\hbox{%
      \kern\wd\ltx@zero
      \csname HOLOGO@ScaleBox@\hologoDriver\endcsname{-1}{1}{%
        \hbox to 0pt{\copy\ltx@zero\hss}%
      }%
    }%
    \wd\ltx@two=\wd\ltx@zero
    \box\ltx@two
  \endgroup
}
%    \end{macrocode}
%    \end{macro}
%
%    \begin{macro}{\HOLOGO@PointReflectBox}
%    \begin{macrocode}
\def\HOLOGO@PointReflectBox#1{%
  \begingroup
    \setbox\ltx@zero\hbox{\begingroup#1\endgroup}%
    \setbox\ltx@two\hbox{%
      \kern\wd\ltx@zero
      \raise\ht\ltx@zero\hbox{%
        \csname HOLOGO@ScaleBox@\hologoDriver\endcsname{-1}{-1}{%
          \hbox to 0pt{\copy\ltx@zero\hss}%
        }%
      }%
    }%
    \wd\ltx@two=\wd\ltx@zero
    \box\ltx@two
  \endgroup
}
%    \end{macrocode}
%    \end{macro}
%
%    We must define all variants because of dynamic driver setup.
%    \begin{macrocode}
\def\HOLOGO@temp#1#2{#2}
%    \end{macrocode}
%
%    \begin{macro}{\HOLOGO@ScaleBox@pdftex}
%    \begin{macrocode}
\HOLOGO@temp{pdftex}{%
  \def\HOLOGO@ScaleBox@pdftex#1#2#3{%
    \HOLOGO@pdfliteral{%
      q #1 0 0 #2 0 0 cm%
    }%
    #3%
    \HOLOGO@pdfliteral{%
      Q%
    }%
  }%
}
%    \end{macrocode}
%    \end{macro}
%    \begin{macro}{\HOLOGO@ScaleBox@dvips}
%    \begin{macrocode}
\HOLOGO@temp{dvips}{%
  \def\HOLOGO@ScaleBox@dvips#1#2#3{%
    \special{ps:%
      gsave %
      currentpoint %
      currentpoint translate %
      #1 #2 scale %
      neg exch neg exch translate%
    }%
    #3%
    \special{ps:%
      currentpoint %
      grestore %
      moveto%
    }%
  }%
}
%    \end{macrocode}
%    \end{macro}
%    \begin{macro}{\HOLOGO@ScaleBox@dvipdfm}
%    \begin{macrocode}
\HOLOGO@temp{dvipdfm}{%
  \let\HOLOGO@ScaleBox@dvipdfm\HOLOGO@ScaleBox@dvips
}
%    \end{macrocode}
%    \end{macro}
%    Since \hologo{XeTeX} v0.6.
%    \begin{macro}{\HOLOGO@ScaleBox@xetex}
%    \begin{macrocode}
\HOLOGO@temp{xetex}{%
  \def\HOLOGO@ScaleBox@xetex#1#2#3{%
    \special{x:gsave}%
    \special{x:scale #1 #2}%
    #3%
    \special{x:grestore}%
  }%
}
%    \end{macrocode}
%    \end{macro}
%    \begin{macro}{\HOLOGO@ScaleBox@vtex}
%    \begin{macrocode}
\HOLOGO@temp{vtex}{%
  \def\HOLOGO@ScaleBox@vtex#1#2#3{%
    \special{r(#1,0,0,#2,0,0}%
    #3%
    \special{r)}%
  }%
}
%    \end{macrocode}
%    \end{macro}
%
%    \begin{macrocode}
\HOLOGO@AtEnd%
%</package>
%    \end{macrocode}
%
% \section{Test}
%
% \subsection{Catcode checks for loading}
%
%    \begin{macrocode}
%<*test1>
%    \end{macrocode}
%    \begin{macrocode}
\catcode`\{=1 %
\catcode`\}=2 %
\catcode`\#=6 %
\catcode`\@=11 %
\expandafter\ifx\csname count@\endcsname\relax
  \countdef\count@=255 %
\fi
\expandafter\ifx\csname @gobble\endcsname\relax
  \long\def\@gobble#1{}%
\fi
\expandafter\ifx\csname @firstofone\endcsname\relax
  \long\def\@firstofone#1{#1}%
\fi
\expandafter\ifx\csname loop\endcsname\relax
  \expandafter\@firstofone
\else
  \expandafter\@gobble
\fi
{%
  \def\loop#1\repeat{%
    \def\body{#1}%
    \iterate
  }%
  \def\iterate{%
    \body
      \let\next\iterate
    \else
      \let\next\relax
    \fi
    \next
  }%
  \let\repeat=\fi
}%
\def\RestoreCatcodes{}
\count@=0 %
\loop
  \edef\RestoreCatcodes{%
    \RestoreCatcodes
    \catcode\the\count@=\the\catcode\count@\relax
  }%
\ifnum\count@<255 %
  \advance\count@ 1 %
\repeat

\def\RangeCatcodeInvalid#1#2{%
  \count@=#1\relax
  \loop
    \catcode\count@=15 %
  \ifnum\count@<#2\relax
    \advance\count@ 1 %
  \repeat
}
\def\RangeCatcodeCheck#1#2#3{%
  \count@=#1\relax
  \loop
    \ifnum#3=\catcode\count@
    \else
      \errmessage{%
        Character \the\count@\space
        with wrong catcode \the\catcode\count@\space
        instead of \number#3%
      }%
    \fi
  \ifnum\count@<#2\relax
    \advance\count@ 1 %
  \repeat
}
\def\space{ }
\expandafter\ifx\csname LoadCommand\endcsname\relax
  \def\LoadCommand{\input hologo.sty\relax}%
\fi
\def\Test{%
  \RangeCatcodeInvalid{0}{47}%
  \RangeCatcodeInvalid{58}{64}%
  \RangeCatcodeInvalid{91}{96}%
  \RangeCatcodeInvalid{123}{255}%
  \catcode`\@=12 %
  \catcode`\\=0 %
  \catcode`\%=14 %
  \LoadCommand
  \RangeCatcodeCheck{0}{36}{15}%
  \RangeCatcodeCheck{37}{37}{14}%
  \RangeCatcodeCheck{38}{47}{15}%
  \RangeCatcodeCheck{48}{57}{12}%
  \RangeCatcodeCheck{58}{63}{15}%
  \RangeCatcodeCheck{64}{64}{12}%
  \RangeCatcodeCheck{65}{90}{11}%
  \RangeCatcodeCheck{91}{91}{15}%
  \RangeCatcodeCheck{92}{92}{0}%
  \RangeCatcodeCheck{93}{96}{15}%
  \RangeCatcodeCheck{97}{122}{11}%
  \RangeCatcodeCheck{123}{255}{15}%
  \RestoreCatcodes
}
\Test
\csname @@end\endcsname
\end
%    \end{macrocode}
%    \begin{macrocode}
%</test1>
%    \end{macrocode}
%
% \subsection{Spacefactor}
%
%    The space factor must be 1000 after a logo. If it is greater 1000
%    then the following space is a space after a sentence closing point.
%    If the space factor is smaller 1000 then an immediate following
%    dot is interpreted as abbreviation, not sentence closing point.
%
%    \begin{macrocode}
%<*test-spacefactor>
\NeedsTeXFormat{LaTeX2e}
\documentclass{article}
\usepackage{hologo}[2016/05/12]
\usepackage{kvsetkeys}
\usepackage{qstest}
\IncludeTests{*}
\LogTests{log}{*}{*}
\begin{document}
\begin{qstest}{spacefactor}{spacefactor}
\newcommand*{\Test}[1]{%
  \sbox0{%
    \hologo{#1}%
    \Expect*{1000 (#1)}*{\the\spacefactor\space(#1)}%
  }%
}%
\makeatletter
\def\TestList{}
\def\hologoEntry#1#2#3{%
  \edef\TestList{%
    \ifx\TestList\@empty
    \else
      \TestList,%
    \fi
    #1%
    \ifx\\#2\\%
    \else
      ={variant=#2}%
    \fi
  }%
}
\hologoList
\expandafter\kv@parse@normalized\expandafter{%
  \TestList
}{%
  \begingroup
    \let\@logo=\kv@key
    \ifx\kv@value\relax
    \else
      \expandafter\hologoLogoSetup\expandafter\@logo\expandafter{%
        \kv@value
      }%
    \fi
    \Test\@logo
  \endgroup
  \@gobbletwo
}
\end{qstest}
\end{document}
%</test-spacefactor>
%    \end{macrocode}
%
% \subsection{Complete list}
%
%    \begin{macrocode}
%<*test-list>
\NeedsTeXFormat{LaTeX2e}
\documentclass[12pt,a4paper]{article}
\usepackage{hologo}[2016/05/12]
\usepackage[T1]{fontenc}
\usepackage{lmodern}
\usepackage{parskip}
\usepackage[unicode]{hyperref}[2011/09/28]
\usepackage{bookmark}[2011/09/19]
\bookmarksetup{%
  numbered,%
  open,%
  openlevel=2,%
}
\renewcommand*{\contentsname}{List of logos}
\begin{document}
\tableofcontents
\def\TestFont#1#2#3#4#5#6{%
  \begingroup
    \usefont{#3}{#4}{#5}{#6}%
    \HologoVariant{#1}{#2}/\hologoVariant{#1}{#2}%
    \quad
    \begingroup\scriptsize\hologoVariant{#1}{#2}\endgroup
    \quad
  \endgroup
  (#3/#4/#5/#6)%
  \par
}
\makeatletter
\def\hologoEntry#1#2#3{%
  \section{%
    \HologoVariant{#1}{#2}/\hologoVariant{#1}{#2} %
    {[#1\ifx\\#2\\\else\space(#2)\fi]}% hash-ok
  }% braces around [] because of bug in tex4ht
  \begingroup
    \hypersetup{unicode=false}%
    \bookmark[%
      dest=\@currentHref,%
      rellevel=1,%
      keeplevel,%
    ]{%
      \HologoVariant{#1}{#2}/\hologoVariant{#1}{#2} %
      (PDFDocEncoding)%
    }%
  \endgroup
  \TestFont{#1}{#2}{OT1}{cmr}{m}{n}%
  \TestFont{#1}{#2}{OT1}{cmss}{m}{n}%
  \TestFont{#1}{#2}{OT1}{cmr}{b}{n}%
  \TestFont{#1}{#2}{OT1}{cmr}{m}{it}%
  \TestFont{#1}{#2}{OT1}{cmtt}{m}{n}%
  \TestFont{#1}{#2}{T1}{lmr}{m}{n}%
  \TestFont{#1}{#2}{T1}{lmss}{m}{n}%
  \TestFont{#1}{#2}{T1}{lmr}{b}{n}%
  \TestFont{#1}{#2}{T1}{lmr}{m}{it}%
  \TestFont{#1}{#2}{T1}{lmtt}{m}{n}%
  \TestFont{#1}{#2}{T1}{lmvtt}{m}{n}%
  \TestFont{#1}{#2}{T1}{qtm}{m}{n}%
  \TestFont{#1}{#2}{T1}{qhv}{m}{n}%
  \TestFont{#1}{#2}{T1}{qtm}{b}{n}%
  \TestFont{#1}{#2}{T1}{qtm}{m}{it}%
  \TestFont{#1}{#2}{T1}{qcr}{m}{n}%
  \newpage
}
\makeatother
\hologoList
\end{document}
%</test-list>
%    \end{macrocode}
%
% \section{Installation}
%
% \subsection{Download}
%
% \paragraph{Package.} This package is available on
% CTAN\footnote{\url{ftp://ftp.ctan.org/tex-archive/}}:
% \begin{description}
% \item[\CTAN{macros/latex/contrib/oberdiek/hologo.dtx}] The source file.
% \item[\CTAN{macros/latex/contrib/oberdiek/hologo.pdf}] Documentation.
% \end{description}
%
%
% \paragraph{Bundle.} All the packages of the bundle `oberdiek'
% are also available in a TDS compliant ZIP archive. There
% the packages are already unpacked and the documentation files
% are generated. The files and directories obey the TDS standard.
% \begin{description}
% \item[\CTAN{install/macros/latex/contrib/oberdiek.tds.zip}]
% \end{description}
% \emph{TDS} refers to the standard ``A Directory Structure
% for \TeX\ Files'' (\CTAN{tds/tds.pdf}). Directories
% with \xfile{texmf} in their name are usually organized this way.
%
% \subsection{Bundle installation}
%
% \paragraph{Unpacking.} Unpack the \xfile{oberdiek.tds.zip} in the
% TDS tree (also known as \xfile{texmf} tree) of your choice.
% Example (linux):
% \begin{quote}
%   |unzip oberdiek.tds.zip -d ~/texmf|
% \end{quote}
%
% \paragraph{Script installation.}
% Check the directory \xfile{TDS:scripts/oberdiek/} for
% scripts that need further installation steps.
% Package \xpackage{attachfile2} comes with the Perl script
% \xfile{pdfatfi.pl} that should be installed in such a way
% that it can be called as \texttt{pdfatfi}.
% Example (linux):
% \begin{quote}
%   |chmod +x scripts/oberdiek/pdfatfi.pl|\\
%   |cp scripts/oberdiek/pdfatfi.pl /usr/local/bin/|
% \end{quote}
%
% \subsection{Package installation}
%
% \paragraph{Unpacking.} The \xfile{.dtx} file is a self-extracting
% \docstrip\ archive. The files are extracted by running the
% \xfile{.dtx} through \plainTeX:
% \begin{quote}
%   \verb|tex hologo.dtx|
% \end{quote}
%
% \paragraph{TDS.} Now the different files must be moved into
% the different directories in your installation TDS tree
% (also known as \xfile{texmf} tree):
% \begin{quote}
% \def\t{^^A
% \begin{tabular}{@{}>{\ttfamily}l@{ $\rightarrow$ }>{\ttfamily}l@{}}
%   hologo.sty & tex/generic/oberdiek/hologo.sty\\
%   hologo.pdf & doc/latex/oberdiek/hologo.pdf\\
%   example/hologo-example.tex & doc/latex/oberdiek/example/hologo-example.tex\\
%   test/hologo-test1.tex & doc/latex/oberdiek/test/hologo-test1.tex\\
%   test/hologo-test-spacefactor.tex & doc/latex/oberdiek/test/hologo-test-spacefactor.tex\\
%   test/hologo-test-list.tex & doc/latex/oberdiek/test/hologo-test-list.tex\\
%   hologo.dtx & source/latex/oberdiek/hologo.dtx\\
% \end{tabular}^^A
% }^^A
% \sbox0{\t}^^A
% \ifdim\wd0>\linewidth
%   \begingroup
%     \advance\linewidth by\leftmargin
%     \advance\linewidth by\rightmargin
%   \edef\x{\endgroup
%     \def\noexpand\lw{\the\linewidth}^^A
%   }\x
%   \def\lwbox{^^A
%     \leavevmode
%     \hbox to \linewidth{^^A
%       \kern-\leftmargin\relax
%       \hss
%       \usebox0
%       \hss
%       \kern-\rightmargin\relax
%     }^^A
%   }^^A
%   \ifdim\wd0>\lw
%     \sbox0{\small\t}^^A
%     \ifdim\wd0>\linewidth
%       \ifdim\wd0>\lw
%         \sbox0{\footnotesize\t}^^A
%         \ifdim\wd0>\linewidth
%           \ifdim\wd0>\lw
%             \sbox0{\scriptsize\t}^^A
%             \ifdim\wd0>\linewidth
%               \ifdim\wd0>\lw
%                 \sbox0{\tiny\t}^^A
%                 \ifdim\wd0>\linewidth
%                   \lwbox
%                 \else
%                   \usebox0
%                 \fi
%               \else
%                 \lwbox
%               \fi
%             \else
%               \usebox0
%             \fi
%           \else
%             \lwbox
%           \fi
%         \else
%           \usebox0
%         \fi
%       \else
%         \lwbox
%       \fi
%     \else
%       \usebox0
%     \fi
%   \else
%     \lwbox
%   \fi
% \else
%   \usebox0
% \fi
% \end{quote}
% If you have a \xfile{docstrip.cfg} that configures and enables \docstrip's
% TDS installing feature, then some files can already be in the right
% place, see the documentation of \docstrip.
%
% \subsection{Refresh file name databases}
%
% If your \TeX~distribution
% (\teTeX, \mikTeX, \dots) relies on file name databases, you must refresh
% these. For example, \teTeX\ users run \verb|texhash| or
% \verb|mktexlsr|.
%
% \subsection{Some details for the interested}
%
% \paragraph{Attached source.}
%
% The PDF documentation on CTAN also includes the
% \xfile{.dtx} source file. It can be extracted by
% AcrobatReader 6 or higher. Another option is \textsf{pdftk},
% e.g. unpack the file into the current directory:
% \begin{quote}
%   \verb|pdftk hologo.pdf unpack_files output .|
% \end{quote}
%
% \paragraph{Unpacking with \LaTeX.}
% The \xfile{.dtx} chooses its action depending on the format:
% \begin{description}
% \item[\plainTeX:] Run \docstrip\ and extract the files.
% \item[\LaTeX:] Generate the documentation.
% \end{description}
% If you insist on using \LaTeX\ for \docstrip\ (really,
% \docstrip\ does not need \LaTeX), then inform the autodetect routine
% about your intention:
% \begin{quote}
%   \verb|latex \let\install=y\input{hologo.dtx}|
% \end{quote}
% Do not forget to quote the argument according to the demands
% of your shell.
%
% \paragraph{Generating the documentation.}
% You can use both the \xfile{.dtx} or the \xfile{.drv} to generate
% the documentation. The process can be configured by the
% configuration file \xfile{ltxdoc.cfg}. For instance, put this
% line into this file, if you want to have A4 as paper format:
% \begin{quote}
%   \verb|\PassOptionsToClass{a4paper}{article}|
% \end{quote}
% An example follows how to generate the
% documentation with pdf\LaTeX:
% \begin{quote}
%\begin{verbatim}
%pdflatex hologo.dtx
%makeindex -s gind.ist hologo.idx
%pdflatex hologo.dtx
%makeindex -s gind.ist hologo.idx
%pdflatex hologo.dtx
%\end{verbatim}
% \end{quote}
%
% \section{Catalogue}
%
% The following XML file can be used as source for the
% \href{http://mirror.ctan.org/help/Catalogue/catalogue.html}{\TeX\ Catalogue}.
% The elements \texttt{caption} and \texttt{description} are imported
% from the original XML file from the Catalogue.
% The name of the XML file in the Catalogue is \xfile{hologo.xml}.
%    \begin{macrocode}
%<*catalogue>
<?xml version='1.0' encoding='us-ascii'?>
<!DOCTYPE entry SYSTEM 'catalogue.dtd'>
<entry datestamp='$Date$' modifier='$Author$' id='hologo'>
  <name>hologo</name>
  <caption>A collection of logos with bookmark support.</caption>
  <authorref id='auth:oberdiek'/>
  <copyright owner='Heiko Oberdiek' year='2010-2012'/>
  <license type='lppl1.3'/>
  <version number='1.10'/>
  <description>
    The package defines a single command <tt>\hologo</tt>, whose
    argument is the usual case-confused ASCII version of the logo.
    The command is bookmark-enabled, so that every logo becomes
    available in bookmarks without further work.
    <p/>
    The package is part of the <xref refid='oberdiek'>oberdiek</xref>
    bundle.
  </description>
  <documentation details='Package documentation'
      href='ctan:/macros/latex/contrib/oberdiek/hologo.pdf'/>
  <ctan file='true' path='/macros/latex/contrib/oberdiek/hologo.dtx'/>
  <miktex location='oberdiek'/>
  <texlive location='oberdiek'/>
  <install path='/macros/latex/contrib/oberdiek/oberdiek.tds.zip'/>
</entry>
%</catalogue>
%    \end{macrocode}
%
% \begin{thebibliography}{9}
% \raggedright
%
% \bibitem{btxdoc}
% Oren Patashnik,
% \textit{\hologo{BibTeX}ing},
% 1988-02-08.\\
% \CTAN{biblio/bibtex/base/}
%
% \bibitem{dtklogos}
% Gerd Neugebauer, DANTE,
% \textit{Package \xpackage{dtklogos}},
% 2011-04-25.\\
% \CTAN{usergrps/dante/dtk/dtklogos.sty}
%
% \bibitem{etexman}
% The \hologo{NTS} Team,
% \textit{The \hologo{eTeX} manual},
% 1998-02.\\
% \CTAN{systems/e-tex/v2/doc/}
%
% \bibitem{ExTeX-FAQ}
% The \hologo{ExTeX} group,
% \textit{\hologo{ExTeX}: FAQ -- How is \hologo{ExTeX} typeset?},
% 2007-04-14.\\
% \url{http://www.extex.org/documentation/faq.html}
%
% \bibitem{LyX}
% %@MISC{ LyX,
% %  title = {{LyX 2.0.0 -- The Document Processor [Computer software and manual]}},
% %  author = {{The LyX Team}},
% %  howpublished = {Internet: http://www.lyx.org},
% %  year = {2011-05-08},
% %  note = {Retrieved May 10, 2011, from http://www.lyx.org},
% %  url = {http://www.lyx.org/}
% %}
% The \hologo{LyX} Team,
% \textit{\hologo{LyX} -- The Document Processor},
% 2011-05-08.\\
% \url{http://www.lyx.org/}
%
% \bibitem{OzTeX}
% Andrew Trevorrow,
% \hologo{OzTeX} FAQ: What is the correct way to typeset ``\hologo{OzTeX}''?,
% 2011-09-15 (visited).
% \url{http://www.trevorrow.com/oztex/ozfaq.html#oztex-logo}
%
% \bibitem{PiCTeX}
% Michael Wichura,
% \textit{The \hologo{PiCTeX} macro package},
% 1987-09-21.
% \CTAN{graphics/pictex/}
%
% \bibitem{scrlogo}
% Markus Kohm,
% \textit{\hologo{KOMAScript} Datei \xfile{scrlogo.dtx}},
% 2009-01-30.\\
% \CTAN{install/macros/latex/contrib/komascript.tds.zip}
%
% \end{thebibliography}
%
% \begin{History}
%   \begin{Version}{2010/04/08 v1.0}
%   \item
%     The first version.
%   \end{Version}
%   \begin{Version}{2010/04/16 v1.1}
%   \item
%     \cs{Hologo} added for support of logos at start of a sentence.
%   \item
%     \cs{hologoSetup} and \cs{hologoLogoSetup} added.
%   \item
%     Options \xoption{break}, \xoption{hyphenbreak}, \xoption{spacebreak}
%     added.
%   \item
%     Variant support added by option \xoption{variant}.
%   \end{Version}
%   \begin{Version}{2010/04/24 v1.2}
%   \item
%     \hologo{LaTeX3} added.
%   \item
%     \hologo{VTeX} added.
%   \end{Version}
%   \begin{Version}{2010/11/21 v1.3}
%   \item
%     \hologo{iniTeX}, \hologo{virTeX} added.
%   \end{Version}
%   \begin{Version}{2011/03/25 v1.4}
%   \item
%     \hologo{ConTeXt} with variants added.
%   \item
%     Option \xoption{discretionarybreak} added as refinement for
%     option \xoption{break}.
%   \end{Version}
%   \begin{Version}{2011/04/21 v1.5}
%   \item
%     Wrong TDS directory for test files fixed.
%   \end{Version}
%   \begin{Version}{2011/10/01 v1.6}
%   \item
%     Support for package \xpackage{tex4ht} added.
%   \item
%     Support for \cs{csname} added if \cs{ifincsname} is available.
%   \item
%     New logos:
%     \hologo{(La)TeX},
%     \hologo{biber},
%     \hologo{BibTeX} (\xoption{sc}, \xoption{sf}),
%     \hologo{emTeX},
%     \hologo{ExTeX},
%     \hologo{KOMAScript},
%     \hologo{La},
%     \hologo{LyX},
%     \hologo{MiKTeX},
%     \hologo{NTS},
%     \hologo{OzMF},
%     \hologo{OzMP},
%     \hologo{OzTeX},
%     \hologo{OzTtH},
%     \hologo{PCTeX},
%     \hologo{PiC},
%     \hologo{PiCTeX},
%     \hologo{METAFONT},
%     \hologo{MetaFun},
%     \hologo{METAPOST},
%     \hologo{MetaPost},
%     \hologo{SLiTeX} (\xoption{lift}, \xoption{narrow}, \xoption{simple}),
%     \hologo{SliTeX} (\xoption{narrow}, \xoption{simple}, \xoption{lift}),
%     \hologo{teTeX}.
%   \item
%     Fixes:
%     \hologo{iniTeX},
%     \hologo{pdfLaTeX},
%     \hologo{pdfTeX},
%     \hologo{virTeX}.
%   \item
%     \cs{hologoFontSetup} and \cs{hologoLogoFontSetup} added.
%   \item
%     \cs{hologoVariant} and \cs{HologoVariant} added.
%   \end{Version}
%   \begin{Version}{2011/11/22 v1.7}
%   \item
%     New logos:
%     \hologo{BibTeX8},
%     \hologo{LaTeXML},
%     \hologo{SageTeX},
%     \hologo{TeX4ht},
%     \hologo{TTH}.
%   \item
%     \hologo{Xe} and friends: Driver stuff fixed.
%   \item
%     \hologo{Xe} and friends: Support for italic added.
%   \item
%     \hologo{Xe} and friends: Package support for \xpackage{pgf}
%     and \xpackage{pstricks} added.
%   \end{Version}
%   \begin{Version}{2011/11/29 v1.8}
%   \item
%     New logos:
%     \hologo{HanTheThanh}.
%   \end{Version}
%   \begin{Version}{2011/12/21 v1.9}
%   \item
%     Patch for package \xpackage{ifxetex} added for the case that
%     \cs{newif} is undefined in \hologo{iniTeX}.
%   \item
%     Some fixes for \hologo{iniTeX}.
%   \end{Version}
%   \begin{Version}{2012/04/26 v1.10}
%   \item
%     Fix in bookmark version of logo ``\hologo{HanTheThanh}''.
%   \end{Version}
%   \begin{Version}{2016/05/12 v1.11}
%   \item
%     Update HOLOGO@IfCharExists (previously in texlive)
%   \item define pdfliteral in current luatex.
%   \end{Version}
% \end{History}
%
% \PrintIndex
%
% \Finale
\endinput
|
% \end{quote}
% Do not forget to quote the argument according to the demands
% of your shell.
%
% \paragraph{Generating the documentation.}
% You can use both the \xfile{.dtx} or the \xfile{.drv} to generate
% the documentation. The process can be configured by the
% configuration file \xfile{ltxdoc.cfg}. For instance, put this
% line into this file, if you want to have A4 as paper format:
% \begin{quote}
%   \verb|\PassOptionsToClass{a4paper}{article}|
% \end{quote}
% An example follows how to generate the
% documentation with pdf\LaTeX:
% \begin{quote}
%\begin{verbatim}
%pdflatex hologo.dtx
%makeindex -s gind.ist hologo.idx
%pdflatex hologo.dtx
%makeindex -s gind.ist hologo.idx
%pdflatex hologo.dtx
%\end{verbatim}
% \end{quote}
%
% \section{Catalogue}
%
% The following XML file can be used as source for the
% \href{http://mirror.ctan.org/help/Catalogue/catalogue.html}{\TeX\ Catalogue}.
% The elements \texttt{caption} and \texttt{description} are imported
% from the original XML file from the Catalogue.
% The name of the XML file in the Catalogue is \xfile{hologo.xml}.
%    \begin{macrocode}
%<*catalogue>
<?xml version='1.0' encoding='us-ascii'?>
<!DOCTYPE entry SYSTEM 'catalogue.dtd'>
<entry datestamp='$Date$' modifier='$Author$' id='hologo'>
  <name>hologo</name>
  <caption>A collection of logos with bookmark support.</caption>
  <authorref id='auth:oberdiek'/>
  <copyright owner='Heiko Oberdiek' year='2010-2012'/>
  <license type='lppl1.3'/>
  <version number='1.10'/>
  <description>
    The package defines a single command <tt>\hologo</tt>, whose
    argument is the usual case-confused ASCII version of the logo.
    The command is bookmark-enabled, so that every logo becomes
    available in bookmarks without further work.
    <p/>
    The package is part of the <xref refid='oberdiek'>oberdiek</xref>
    bundle.
  </description>
  <documentation details='Package documentation'
      href='ctan:/macros/latex/contrib/oberdiek/hologo.pdf'/>
  <ctan file='true' path='/macros/latex/contrib/oberdiek/hologo.dtx'/>
  <miktex location='oberdiek'/>
  <texlive location='oberdiek'/>
  <install path='/macros/latex/contrib/oberdiek/oberdiek.tds.zip'/>
</entry>
%</catalogue>
%    \end{macrocode}
%
% \begin{thebibliography}{9}
% \raggedright
%
% \bibitem{btxdoc}
% Oren Patashnik,
% \textit{\hologo{BibTeX}ing},
% 1988-02-08.\\
% \CTAN{biblio/bibtex/base/}
%
% \bibitem{dtklogos}
% Gerd Neugebauer, DANTE,
% \textit{Package \xpackage{dtklogos}},
% 2011-04-25.\\
% \CTAN{usergrps/dante/dtk/dtklogos.sty}
%
% \bibitem{etexman}
% The \hologo{NTS} Team,
% \textit{The \hologo{eTeX} manual},
% 1998-02.\\
% \CTAN{systems/e-tex/v2/doc/}
%
% \bibitem{ExTeX-FAQ}
% The \hologo{ExTeX} group,
% \textit{\hologo{ExTeX}: FAQ -- How is \hologo{ExTeX} typeset?},
% 2007-04-14.\\
% \url{http://www.extex.org/documentation/faq.html}
%
% \bibitem{LyX}
% %@MISC{ LyX,
% %  title = {{LyX 2.0.0 -- The Document Processor [Computer software and manual]}},
% %  author = {{The LyX Team}},
% %  howpublished = {Internet: http://www.lyx.org},
% %  year = {2011-05-08},
% %  note = {Retrieved May 10, 2011, from http://www.lyx.org},
% %  url = {http://www.lyx.org/}
% %}
% The \hologo{LyX} Team,
% \textit{\hologo{LyX} -- The Document Processor},
% 2011-05-08.\\
% \url{http://www.lyx.org/}
%
% \bibitem{OzTeX}
% Andrew Trevorrow,
% \hologo{OzTeX} FAQ: What is the correct way to typeset ``\hologo{OzTeX}''?,
% 2011-09-15 (visited).
% \url{http://www.trevorrow.com/oztex/ozfaq.html#oztex-logo}
%
% \bibitem{PiCTeX}
% Michael Wichura,
% \textit{The \hologo{PiCTeX} macro package},
% 1987-09-21.
% \CTAN{graphics/pictex/}
%
% \bibitem{scrlogo}
% Markus Kohm,
% \textit{\hologo{KOMAScript} Datei \xfile{scrlogo.dtx}},
% 2009-01-30.\\
% \CTAN{install/macros/latex/contrib/komascript.tds.zip}
%
% \end{thebibliography}
%
% \begin{History}
%   \begin{Version}{2010/04/08 v1.0}
%   \item
%     The first version.
%   \end{Version}
%   \begin{Version}{2010/04/16 v1.1}
%   \item
%     \cs{Hologo} added for support of logos at start of a sentence.
%   \item
%     \cs{hologoSetup} and \cs{hologoLogoSetup} added.
%   \item
%     Options \xoption{break}, \xoption{hyphenbreak}, \xoption{spacebreak}
%     added.
%   \item
%     Variant support added by option \xoption{variant}.
%   \end{Version}
%   \begin{Version}{2010/04/24 v1.2}
%   \item
%     \hologo{LaTeX3} added.
%   \item
%     \hologo{VTeX} added.
%   \end{Version}
%   \begin{Version}{2010/11/21 v1.3}
%   \item
%     \hologo{iniTeX}, \hologo{virTeX} added.
%   \end{Version}
%   \begin{Version}{2011/03/25 v1.4}
%   \item
%     \hologo{ConTeXt} with variants added.
%   \item
%     Option \xoption{discretionarybreak} added as refinement for
%     option \xoption{break}.
%   \end{Version}
%   \begin{Version}{2011/04/21 v1.5}
%   \item
%     Wrong TDS directory for test files fixed.
%   \end{Version}
%   \begin{Version}{2011/10/01 v1.6}
%   \item
%     Support for package \xpackage{tex4ht} added.
%   \item
%     Support for \cs{csname} added if \cs{ifincsname} is available.
%   \item
%     New logos:
%     \hologo{(La)TeX},
%     \hologo{biber},
%     \hologo{BibTeX} (\xoption{sc}, \xoption{sf}),
%     \hologo{emTeX},
%     \hologo{ExTeX},
%     \hologo{KOMAScript},
%     \hologo{La},
%     \hologo{LyX},
%     \hologo{MiKTeX},
%     \hologo{NTS},
%     \hologo{OzMF},
%     \hologo{OzMP},
%     \hologo{OzTeX},
%     \hologo{OzTtH},
%     \hologo{PCTeX},
%     \hologo{PiC},
%     \hologo{PiCTeX},
%     \hologo{METAFONT},
%     \hologo{MetaFun},
%     \hologo{METAPOST},
%     \hologo{MetaPost},
%     \hologo{SLiTeX} (\xoption{lift}, \xoption{narrow}, \xoption{simple}),
%     \hologo{SliTeX} (\xoption{narrow}, \xoption{simple}, \xoption{lift}),
%     \hologo{teTeX}.
%   \item
%     Fixes:
%     \hologo{iniTeX},
%     \hologo{pdfLaTeX},
%     \hologo{pdfTeX},
%     \hologo{virTeX}.
%   \item
%     \cs{hologoFontSetup} and \cs{hologoLogoFontSetup} added.
%   \item
%     \cs{hologoVariant} and \cs{HologoVariant} added.
%   \end{Version}
%   \begin{Version}{2011/11/22 v1.7}
%   \item
%     New logos:
%     \hologo{BibTeX8},
%     \hologo{LaTeXML},
%     \hologo{SageTeX},
%     \hologo{TeX4ht},
%     \hologo{TTH}.
%   \item
%     \hologo{Xe} and friends: Driver stuff fixed.
%   \item
%     \hologo{Xe} and friends: Support for italic added.
%   \item
%     \hologo{Xe} and friends: Package support for \xpackage{pgf}
%     and \xpackage{pstricks} added.
%   \end{Version}
%   \begin{Version}{2011/11/29 v1.8}
%   \item
%     New logos:
%     \hologo{HanTheThanh}.
%   \end{Version}
%   \begin{Version}{2011/12/21 v1.9}
%   \item
%     Patch for package \xpackage{ifxetex} added for the case that
%     \cs{newif} is undefined in \hologo{iniTeX}.
%   \item
%     Some fixes for \hologo{iniTeX}.
%   \end{Version}
%   \begin{Version}{2012/04/26 v1.10}
%   \item
%     Fix in bookmark version of logo ``\hologo{HanTheThanh}''.
%   \end{Version}
%   \begin{Version}{2016/05/12 v1.11}
%   \item
%     Update HOLOGO@IfCharExists (previously in texlive)
%   \item define pdfliteral in current luatex.
%   \end{Version}
% \end{History}
%
% \PrintIndex
%
% \Finale
\endinput
%
        \else
          \input hologo.cfg\relax
        \fi
      \else
        \@PackageInfoNoLine{hologo}{%
          Empty configuration file `hologo.cfg' ignored%
        }%
      \fi
    \fi
  }%
}
%    \end{macrocode}
%
%    \begin{macrocode}
\def\HOLOGO@temp#1#2{%
  \kv@define@key{HoLogoDriver}{#1}[]{%
    \begingroup
      \def\HOLOGO@temp{##1}%
      \ltx@onelevel@sanitize\HOLOGO@temp
      \ifx\HOLOGO@temp\ltx@empty
      \else
        \@PackageError{hologo}{%
          Value (\HOLOGO@temp) not permitted for option `#1'%
        }%
        \@ehc
      \fi
    \endgroup
    \def\hologoDriver{#2}%
  }%
}%
\def\HOLOGO@@temp#1#2{%
  \ifx\kv@value\relax
    \HOLOGO@temp{#1}{#1}%
  \else
    \HOLOGO@temp{#1}{#2}%
  \fi
}%
\kv@parse@normalized{%
  pdftex,%
  luatex=pdftex,%
  dvipdfm,%
  dvipdfmx=dvipdfm,%
  dvips,%
  dvipsone=dvips,%
  xdvi=dvips,%
  xetex,%
  vtex,%
}\HOLOGO@@temp
%    \end{macrocode}
%
%    \begin{macrocode}
\kv@define@key{HoLogoDriver}{driverfallback}{%
  \def\HOLOGO@DriverFallback{#1}%
}
%    \end{macrocode}
%
%    \begin{macro}{\HOLOGO@DriverFallback}
%    \begin{macrocode}
\def\HOLOGO@DriverFallback{dvips}
%    \end{macrocode}
%    \end{macro}
%
%    \begin{macro}{\hologoDriverSetup}
%    \begin{macrocode}
\def\hologoDriverSetup{%
  \let\hologoDriver\ltx@undefined
  \HOLOGO@DriverSetup
}
%    \end{macrocode}
%    \end{macro}
%
%    \begin{macro}{\HOLOGO@DriverSetup}
%    \begin{macrocode}
\def\HOLOGO@DriverSetup#1{%
  \kvsetkeys{HoLogoDriver}{#1}%
  \HOLOGO@CheckDriver
  \ltx@ifundefined{hologoDriver}{%
    \begingroup
    \edef\x{\endgroup
      \noexpand\kvsetkeys{HoLogoDriver}{\HOLOGO@DriverFallback}%
    }\x
  }{}%
  \@PackageInfoNoLine{hologo}{Using driver `\hologoDriver'}%
}
%    \end{macrocode}
%    \end{macro}
%
%    \begin{macro}{\HOLOGO@CheckDriver}
%    \begin{macrocode}
\def\HOLOGO@CheckDriver{%
  \ifpdf
    \def\hologoDriver{pdftex}%
    \let\HOLOGO@pdfliteral\pdfliteral
    \ifluatex
      \ifx\pdfextension\@undefined\else
        \protected\def\pdfliteral{\pdfextension literal}%
        \let\HOLOGO@pdfliteral\pdfliteral
      \fi
      \ltx@IfUndefined{HOLOGO@pdfliteral}{%
        \ifnum\luatexversion<36 %
        \else
          \begingroup
            \let\HOLOGO@temp\endgroup
            \ifcase0%
                \directlua{%
                  if tex.enableprimitives then %
                    tex.enableprimitives('HOLOGO@', {'pdfliteral'})%
                  else %
                    tex.print('1')%
                  end%
                }%
                \ifx\HOLOGO@pdfliteral\@undefined 1\fi%
                \relax%
              \endgroup
              \let\HOLOGO@temp\relax
              \global\let\HOLOGO@pdfliteral\HOLOGO@pdfliteral
            \fi%
          \HOLOGO@temp
        \fi
      }{}%
    \fi
    \ltx@IfUndefined{HOLOGO@pdfliteral}{%
      \@PackageWarningNoLine{hologo}{%
        Cannot find \string\pdfliteral
      }%
    }{}%
  \else
    \ifxetex
      \def\hologoDriver{xetex}%
    \else
      \ifvtex
        \def\hologoDriver{vtex}%
      \fi
    \fi
  \fi
}
%    \end{macrocode}
%    \end{macro}
%
%    \begin{macro}{\HOLOGO@WarningUnsupportedDriver}
%    \begin{macrocode}
\def\HOLOGO@WarningUnsupportedDriver#1{%
  \@PackageWarningNoLine{hologo}{%
    Logo `#1' needs driver specific macros,\MessageBreak
    but driver `\hologoDriver' is not supported.\MessageBreak
    Use a different driver or\MessageBreak
    load package `graphics' or `pgf'%
  }%
}
%    \end{macrocode}
%    \end{macro}
%
% \subsubsection{Reflect box macros}
%
%    Skip driver part if not needed.
%    \begin{macrocode}
\ltx@IfUndefined{reflectbox}{}{%
  \ltx@IfUndefined{rotatebox}{}{%
    \HOLOGO@AtEnd
  }%
}
\ltx@IfUndefined{pgftext}{}{%
  \HOLOGO@AtEnd
}
\ltx@IfUndefined{psscalebox}{}{%
  \HOLOGO@AtEnd
}
%    \end{macrocode}
%
%    \begin{macrocode}
\def\HOLOGO@temp{LaTeX2e}
\ifx\fmtname\HOLOGO@temp
  \RequirePackage{kvoptions}[2011/06/30]%
  \ProcessKeyvalOptions{HoLogoDriver}%
\fi
\HOLOGO@DriverSetup{}
%    \end{macrocode}
%
%    \begin{macro}{\HOLOGO@ReflectBox}
%    \begin{macrocode}
\def\HOLOGO@ReflectBox#1{%
  \begingroup
    \setbox\ltx@zero\hbox{\begingroup#1\endgroup}%
    \setbox\ltx@two\hbox{%
      \kern\wd\ltx@zero
      \csname HOLOGO@ScaleBox@\hologoDriver\endcsname{-1}{1}{%
        \hbox to 0pt{\copy\ltx@zero\hss}%
      }%
    }%
    \wd\ltx@two=\wd\ltx@zero
    \box\ltx@two
  \endgroup
}
%    \end{macrocode}
%    \end{macro}
%
%    \begin{macro}{\HOLOGO@PointReflectBox}
%    \begin{macrocode}
\def\HOLOGO@PointReflectBox#1{%
  \begingroup
    \setbox\ltx@zero\hbox{\begingroup#1\endgroup}%
    \setbox\ltx@two\hbox{%
      \kern\wd\ltx@zero
      \raise\ht\ltx@zero\hbox{%
        \csname HOLOGO@ScaleBox@\hologoDriver\endcsname{-1}{-1}{%
          \hbox to 0pt{\copy\ltx@zero\hss}%
        }%
      }%
    }%
    \wd\ltx@two=\wd\ltx@zero
    \box\ltx@two
  \endgroup
}
%    \end{macrocode}
%    \end{macro}
%
%    We must define all variants because of dynamic driver setup.
%    \begin{macrocode}
\def\HOLOGO@temp#1#2{#2}
%    \end{macrocode}
%
%    \begin{macro}{\HOLOGO@ScaleBox@pdftex}
%    \begin{macrocode}
\HOLOGO@temp{pdftex}{%
  \def\HOLOGO@ScaleBox@pdftex#1#2#3{%
    \HOLOGO@pdfliteral{%
      q #1 0 0 #2 0 0 cm%
    }%
    #3%
    \HOLOGO@pdfliteral{%
      Q%
    }%
  }%
}
%    \end{macrocode}
%    \end{macro}
%    \begin{macro}{\HOLOGO@ScaleBox@dvips}
%    \begin{macrocode}
\HOLOGO@temp{dvips}{%
  \def\HOLOGO@ScaleBox@dvips#1#2#3{%
    \special{ps:%
      gsave %
      currentpoint %
      currentpoint translate %
      #1 #2 scale %
      neg exch neg exch translate%
    }%
    #3%
    \special{ps:%
      currentpoint %
      grestore %
      moveto%
    }%
  }%
}
%    \end{macrocode}
%    \end{macro}
%    \begin{macro}{\HOLOGO@ScaleBox@dvipdfm}
%    \begin{macrocode}
\HOLOGO@temp{dvipdfm}{%
  \let\HOLOGO@ScaleBox@dvipdfm\HOLOGO@ScaleBox@dvips
}
%    \end{macrocode}
%    \end{macro}
%    Since \hologo{XeTeX} v0.6.
%    \begin{macro}{\HOLOGO@ScaleBox@xetex}
%    \begin{macrocode}
\HOLOGO@temp{xetex}{%
  \def\HOLOGO@ScaleBox@xetex#1#2#3{%
    \special{x:gsave}%
    \special{x:scale #1 #2}%
    #3%
    \special{x:grestore}%
  }%
}
%    \end{macrocode}
%    \end{macro}
%    \begin{macro}{\HOLOGO@ScaleBox@vtex}
%    \begin{macrocode}
\HOLOGO@temp{vtex}{%
  \def\HOLOGO@ScaleBox@vtex#1#2#3{%
    \special{r(#1,0,0,#2,0,0}%
    #3%
    \special{r)}%
  }%
}
%    \end{macrocode}
%    \end{macro}
%
%    \begin{macrocode}
\HOLOGO@AtEnd%
%</package>
%    \end{macrocode}
%
% \section{Test}
%
% \subsection{Catcode checks for loading}
%
%    \begin{macrocode}
%<*test1>
%    \end{macrocode}
%    \begin{macrocode}
\catcode`\{=1 %
\catcode`\}=2 %
\catcode`\#=6 %
\catcode`\@=11 %
\expandafter\ifx\csname count@\endcsname\relax
  \countdef\count@=255 %
\fi
\expandafter\ifx\csname @gobble\endcsname\relax
  \long\def\@gobble#1{}%
\fi
\expandafter\ifx\csname @firstofone\endcsname\relax
  \long\def\@firstofone#1{#1}%
\fi
\expandafter\ifx\csname loop\endcsname\relax
  \expandafter\@firstofone
\else
  \expandafter\@gobble
\fi
{%
  \def\loop#1\repeat{%
    \def\body{#1}%
    \iterate
  }%
  \def\iterate{%
    \body
      \let\next\iterate
    \else
      \let\next\relax
    \fi
    \next
  }%
  \let\repeat=\fi
}%
\def\RestoreCatcodes{}
\count@=0 %
\loop
  \edef\RestoreCatcodes{%
    \RestoreCatcodes
    \catcode\the\count@=\the\catcode\count@\relax
  }%
\ifnum\count@<255 %
  \advance\count@ 1 %
\repeat

\def\RangeCatcodeInvalid#1#2{%
  \count@=#1\relax
  \loop
    \catcode\count@=15 %
  \ifnum\count@<#2\relax
    \advance\count@ 1 %
  \repeat
}
\def\RangeCatcodeCheck#1#2#3{%
  \count@=#1\relax
  \loop
    \ifnum#3=\catcode\count@
    \else
      \errmessage{%
        Character \the\count@\space
        with wrong catcode \the\catcode\count@\space
        instead of \number#3%
      }%
    \fi
  \ifnum\count@<#2\relax
    \advance\count@ 1 %
  \repeat
}
\def\space{ }
\expandafter\ifx\csname LoadCommand\endcsname\relax
  \def\LoadCommand{\input hologo.sty\relax}%
\fi
\def\Test{%
  \RangeCatcodeInvalid{0}{47}%
  \RangeCatcodeInvalid{58}{64}%
  \RangeCatcodeInvalid{91}{96}%
  \RangeCatcodeInvalid{123}{255}%
  \catcode`\@=12 %
  \catcode`\\=0 %
  \catcode`\%=14 %
  \LoadCommand
  \RangeCatcodeCheck{0}{36}{15}%
  \RangeCatcodeCheck{37}{37}{14}%
  \RangeCatcodeCheck{38}{47}{15}%
  \RangeCatcodeCheck{48}{57}{12}%
  \RangeCatcodeCheck{58}{63}{15}%
  \RangeCatcodeCheck{64}{64}{12}%
  \RangeCatcodeCheck{65}{90}{11}%
  \RangeCatcodeCheck{91}{91}{15}%
  \RangeCatcodeCheck{92}{92}{0}%
  \RangeCatcodeCheck{93}{96}{15}%
  \RangeCatcodeCheck{97}{122}{11}%
  \RangeCatcodeCheck{123}{255}{15}%
  \RestoreCatcodes
}
\Test
\csname @@end\endcsname
\end
%    \end{macrocode}
%    \begin{macrocode}
%</test1>
%    \end{macrocode}
%
% \subsection{Spacefactor}
%
%    The space factor must be 1000 after a logo. If it is greater 1000
%    then the following space is a space after a sentence closing point.
%    If the space factor is smaller 1000 then an immediate following
%    dot is interpreted as abbreviation, not sentence closing point.
%
%    \begin{macrocode}
%<*test-spacefactor>
\NeedsTeXFormat{LaTeX2e}
\documentclass{article}
\usepackage{hologo}[2016/05/12]
\usepackage{kvsetkeys}
\usepackage{qstest}
\IncludeTests{*}
\LogTests{log}{*}{*}
\begin{document}
\begin{qstest}{spacefactor}{spacefactor}
\newcommand*{\Test}[1]{%
  \sbox0{%
    \hologo{#1}%
    \Expect*{1000 (#1)}*{\the\spacefactor\space(#1)}%
  }%
}%
\makeatletter
\def\TestList{}
\def\hologoEntry#1#2#3{%
  \edef\TestList{%
    \ifx\TestList\@empty
    \else
      \TestList,%
    \fi
    #1%
    \ifx\\#2\\%
    \else
      ={variant=#2}%
    \fi
  }%
}
\hologoList
\expandafter\kv@parse@normalized\expandafter{%
  \TestList
}{%
  \begingroup
    \let\@logo=\kv@key
    \ifx\kv@value\relax
    \else
      \expandafter\hologoLogoSetup\expandafter\@logo\expandafter{%
        \kv@value
      }%
    \fi
    \Test\@logo
  \endgroup
  \@gobbletwo
}
\end{qstest}
\end{document}
%</test-spacefactor>
%    \end{macrocode}
%
% \subsection{Complete list}
%
%    \begin{macrocode}
%<*test-list>
\NeedsTeXFormat{LaTeX2e}
\documentclass[12pt,a4paper]{article}
\usepackage{hologo}[2016/05/12]
\usepackage[T1]{fontenc}
\usepackage{lmodern}
\usepackage{parskip}
\usepackage[unicode]{hyperref}[2011/09/28]
\usepackage{bookmark}[2011/09/19]
\bookmarksetup{%
  numbered,%
  open,%
  openlevel=2,%
}
\renewcommand*{\contentsname}{List of logos}
\begin{document}
\tableofcontents
\def\TestFont#1#2#3#4#5#6{%
  \begingroup
    \usefont{#3}{#4}{#5}{#6}%
    \HologoVariant{#1}{#2}/\hologoVariant{#1}{#2}%
    \quad
    \begingroup\scriptsize\hologoVariant{#1}{#2}\endgroup
    \quad
  \endgroup
  (#3/#4/#5/#6)%
  \par
}
\makeatletter
\def\hologoEntry#1#2#3{%
  \section{%
    \HologoVariant{#1}{#2}/\hologoVariant{#1}{#2} %
    {[#1\ifx\\#2\\\else\space(#2)\fi]}% hash-ok
  }% braces around [] because of bug in tex4ht
  \begingroup
    \hypersetup{unicode=false}%
    \bookmark[%
      dest=\@currentHref,%
      rellevel=1,%
      keeplevel,%
    ]{%
      \HologoVariant{#1}{#2}/\hologoVariant{#1}{#2} %
      (PDFDocEncoding)%
    }%
  \endgroup
  \TestFont{#1}{#2}{OT1}{cmr}{m}{n}%
  \TestFont{#1}{#2}{OT1}{cmss}{m}{n}%
  \TestFont{#1}{#2}{OT1}{cmr}{b}{n}%
  \TestFont{#1}{#2}{OT1}{cmr}{m}{it}%
  \TestFont{#1}{#2}{OT1}{cmtt}{m}{n}%
  \TestFont{#1}{#2}{T1}{lmr}{m}{n}%
  \TestFont{#1}{#2}{T1}{lmss}{m}{n}%
  \TestFont{#1}{#2}{T1}{lmr}{b}{n}%
  \TestFont{#1}{#2}{T1}{lmr}{m}{it}%
  \TestFont{#1}{#2}{T1}{lmtt}{m}{n}%
  \TestFont{#1}{#2}{T1}{lmvtt}{m}{n}%
  \TestFont{#1}{#2}{T1}{qtm}{m}{n}%
  \TestFont{#1}{#2}{T1}{qhv}{m}{n}%
  \TestFont{#1}{#2}{T1}{qtm}{b}{n}%
  \TestFont{#1}{#2}{T1}{qtm}{m}{it}%
  \TestFont{#1}{#2}{T1}{qcr}{m}{n}%
  \newpage
}
\makeatother
\hologoList
\end{document}
%</test-list>
%    \end{macrocode}
%
% \section{Installation}
%
% \subsection{Download}
%
% \paragraph{Package.} This package is available on
% CTAN\footnote{\url{ftp://ftp.ctan.org/tex-archive/}}:
% \begin{description}
% \item[\CTAN{macros/latex/contrib/oberdiek/hologo.dtx}] The source file.
% \item[\CTAN{macros/latex/contrib/oberdiek/hologo.pdf}] Documentation.
% \end{description}
%
%
% \paragraph{Bundle.} All the packages of the bundle `oberdiek'
% are also available in a TDS compliant ZIP archive. There
% the packages are already unpacked and the documentation files
% are generated. The files and directories obey the TDS standard.
% \begin{description}
% \item[\CTAN{install/macros/latex/contrib/oberdiek.tds.zip}]
% \end{description}
% \emph{TDS} refers to the standard ``A Directory Structure
% for \TeX\ Files'' (\CTAN{tds/tds.pdf}). Directories
% with \xfile{texmf} in their name are usually organized this way.
%
% \subsection{Bundle installation}
%
% \paragraph{Unpacking.} Unpack the \xfile{oberdiek.tds.zip} in the
% TDS tree (also known as \xfile{texmf} tree) of your choice.
% Example (linux):
% \begin{quote}
%   |unzip oberdiek.tds.zip -d ~/texmf|
% \end{quote}
%
% \paragraph{Script installation.}
% Check the directory \xfile{TDS:scripts/oberdiek/} for
% scripts that need further installation steps.
% Package \xpackage{attachfile2} comes with the Perl script
% \xfile{pdfatfi.pl} that should be installed in such a way
% that it can be called as \texttt{pdfatfi}.
% Example (linux):
% \begin{quote}
%   |chmod +x scripts/oberdiek/pdfatfi.pl|\\
%   |cp scripts/oberdiek/pdfatfi.pl /usr/local/bin/|
% \end{quote}
%
% \subsection{Package installation}
%
% \paragraph{Unpacking.} The \xfile{.dtx} file is a self-extracting
% \docstrip\ archive. The files are extracted by running the
% \xfile{.dtx} through \plainTeX:
% \begin{quote}
%   \verb|tex hologo.dtx|
% \end{quote}
%
% \paragraph{TDS.} Now the different files must be moved into
% the different directories in your installation TDS tree
% (also known as \xfile{texmf} tree):
% \begin{quote}
% \def\t{^^A
% \begin{tabular}{@{}>{\ttfamily}l@{ $\rightarrow$ }>{\ttfamily}l@{}}
%   hologo.sty & tex/generic/oberdiek/hologo.sty\\
%   hologo.pdf & doc/latex/oberdiek/hologo.pdf\\
%   example/hologo-example.tex & doc/latex/oberdiek/example/hologo-example.tex\\
%   test/hologo-test1.tex & doc/latex/oberdiek/test/hologo-test1.tex\\
%   test/hologo-test-spacefactor.tex & doc/latex/oberdiek/test/hologo-test-spacefactor.tex\\
%   test/hologo-test-list.tex & doc/latex/oberdiek/test/hologo-test-list.tex\\
%   hologo.dtx & source/latex/oberdiek/hologo.dtx\\
% \end{tabular}^^A
% }^^A
% \sbox0{\t}^^A
% \ifdim\wd0>\linewidth
%   \begingroup
%     \advance\linewidth by\leftmargin
%     \advance\linewidth by\rightmargin
%   \edef\x{\endgroup
%     \def\noexpand\lw{\the\linewidth}^^A
%   }\x
%   \def\lwbox{^^A
%     \leavevmode
%     \hbox to \linewidth{^^A
%       \kern-\leftmargin\relax
%       \hss
%       \usebox0
%       \hss
%       \kern-\rightmargin\relax
%     }^^A
%   }^^A
%   \ifdim\wd0>\lw
%     \sbox0{\small\t}^^A
%     \ifdim\wd0>\linewidth
%       \ifdim\wd0>\lw
%         \sbox0{\footnotesize\t}^^A
%         \ifdim\wd0>\linewidth
%           \ifdim\wd0>\lw
%             \sbox0{\scriptsize\t}^^A
%             \ifdim\wd0>\linewidth
%               \ifdim\wd0>\lw
%                 \sbox0{\tiny\t}^^A
%                 \ifdim\wd0>\linewidth
%                   \lwbox
%                 \else
%                   \usebox0
%                 \fi
%               \else
%                 \lwbox
%               \fi
%             \else
%               \usebox0
%             \fi
%           \else
%             \lwbox
%           \fi
%         \else
%           \usebox0
%         \fi
%       \else
%         \lwbox
%       \fi
%     \else
%       \usebox0
%     \fi
%   \else
%     \lwbox
%   \fi
% \else
%   \usebox0
% \fi
% \end{quote}
% If you have a \xfile{docstrip.cfg} that configures and enables \docstrip's
% TDS installing feature, then some files can already be in the right
% place, see the documentation of \docstrip.
%
% \subsection{Refresh file name databases}
%
% If your \TeX~distribution
% (\teTeX, \mikTeX, \dots) relies on file name databases, you must refresh
% these. For example, \teTeX\ users run \verb|texhash| or
% \verb|mktexlsr|.
%
% \subsection{Some details for the interested}
%
% \paragraph{Attached source.}
%
% The PDF documentation on CTAN also includes the
% \xfile{.dtx} source file. It can be extracted by
% AcrobatReader 6 or higher. Another option is \textsf{pdftk},
% e.g. unpack the file into the current directory:
% \begin{quote}
%   \verb|pdftk hologo.pdf unpack_files output .|
% \end{quote}
%
% \paragraph{Unpacking with \LaTeX.}
% The \xfile{.dtx} chooses its action depending on the format:
% \begin{description}
% \item[\plainTeX:] Run \docstrip\ and extract the files.
% \item[\LaTeX:] Generate the documentation.
% \end{description}
% If you insist on using \LaTeX\ for \docstrip\ (really,
% \docstrip\ does not need \LaTeX), then inform the autodetect routine
% about your intention:
% \begin{quote}
%   \verb|latex \let\install=y% \iffalse meta-comment
%
% File: hologo.dtx
% Version: 2016/05/12 v1.11
% Info: A logo collection with bookmark support
%
% Copyright (C) 2010-2012 by
%    Heiko Oberdiek <heiko.oberdiek at googlemail.com>
%
% This work may be distributed and/or modified under the
% conditions of the LaTeX Project Public License, either
% version 1.3c of this license or (at your option) any later
% version. This version of this license is in
%    http://www.latex-project.org/lppl/lppl-1-3c.txt
% and the latest version of this license is in
%    http://www.latex-project.org/lppl.txt
% and version 1.3 or later is part of all distributions of
% LaTeX version 2005/12/01 or later.
%
% This work has the LPPL maintenance status "maintained".
%
% This Current Maintainer of this work is Heiko Oberdiek.
%
% The Base Interpreter refers to any `TeX-Format',
% because some files are installed in TDS:tex/generic//.
%
% This work consists of the main source file hologo.dtx
% and the derived files
%    hologo.sty, hologo.pdf, hologo.ins, hologo.drv, hologo-example.tex,
%    hologo-test1.tex, hologo-test-spacefactor.tex,
%    hologo-test-list.tex.
%
% Distribution:
%    CTAN:macros/latex/contrib/oberdiek/hologo.dtx
%    CTAN:macros/latex/contrib/oberdiek/hologo.pdf
%
% Unpacking:
%    (a) If hologo.ins is present:
%           tex hologo.ins
%    (b) Without hologo.ins:
%           tex hologo.dtx
%    (c) If you insist on using LaTeX
%           latex \let\install=y% \iffalse meta-comment
%
% File: hologo.dtx
% Version: 2016/05/12 v1.11
% Info: A logo collection with bookmark support
%
% Copyright (C) 2010-2012 by
%    Heiko Oberdiek <heiko.oberdiek at googlemail.com>
%
% This work may be distributed and/or modified under the
% conditions of the LaTeX Project Public License, either
% version 1.3c of this license or (at your option) any later
% version. This version of this license is in
%    http://www.latex-project.org/lppl/lppl-1-3c.txt
% and the latest version of this license is in
%    http://www.latex-project.org/lppl.txt
% and version 1.3 or later is part of all distributions of
% LaTeX version 2005/12/01 or later.
%
% This work has the LPPL maintenance status "maintained".
%
% This Current Maintainer of this work is Heiko Oberdiek.
%
% The Base Interpreter refers to any `TeX-Format',
% because some files are installed in TDS:tex/generic//.
%
% This work consists of the main source file hologo.dtx
% and the derived files
%    hologo.sty, hologo.pdf, hologo.ins, hologo.drv, hologo-example.tex,
%    hologo-test1.tex, hologo-test-spacefactor.tex,
%    hologo-test-list.tex.
%
% Distribution:
%    CTAN:macros/latex/contrib/oberdiek/hologo.dtx
%    CTAN:macros/latex/contrib/oberdiek/hologo.pdf
%
% Unpacking:
%    (a) If hologo.ins is present:
%           tex hologo.ins
%    (b) Without hologo.ins:
%           tex hologo.dtx
%    (c) If you insist on using LaTeX
%           latex \let\install=y\input{hologo.dtx}
%        (quote the arguments according to the demands of your shell)
%
% Documentation:
%    (a) If hologo.drv is present:
%           latex hologo.drv
%    (b) Without hologo.drv:
%           latex hologo.dtx; ...
%    The class ltxdoc loads the configuration file ltxdoc.cfg
%    if available. Here you can specify further options, e.g.
%    use A4 as paper format:
%       \PassOptionsToClass{a4paper}{article}
%
%    Programm calls to get the documentation (example):
%       pdflatex hologo.dtx
%       makeindex -s gind.ist hologo.idx
%       pdflatex hologo.dtx
%       makeindex -s gind.ist hologo.idx
%       pdflatex hologo.dtx
%
% Installation:
%    TDS:tex/generic/oberdiek/hologo.sty
%    TDS:doc/latex/oberdiek/hologo.pdf
%    TDS:doc/latex/oberdiek/example/hologo-example.tex
%    TDS:doc/latex/oberdiek/test/hologo-test1.tex
%    TDS:doc/latex/oberdiek/test/hologo-test-spacefactor.tex
%    TDS:doc/latex/oberdiek/test/hologo-test-list.tex
%    TDS:source/latex/oberdiek/hologo.dtx
%
%<*ignore>
\begingroup
  \catcode123=1 %
  \catcode125=2 %
  \def\x{LaTeX2e}%
\expandafter\endgroup
\ifcase 0\ifx\install y1\fi\expandafter
         \ifx\csname processbatchFile\endcsname\relax\else1\fi
         \ifx\fmtname\x\else 1\fi\relax
\else\csname fi\endcsname
%</ignore>
%<*install>
\input docstrip.tex
\Msg{************************************************************************}
\Msg{* Installation}
\Msg{* Package: hologo 2016/05/12 v1.11 A logo collection with bookmark support (HO)}
\Msg{************************************************************************}

\keepsilent
\askforoverwritefalse

\let\MetaPrefix\relax
\preamble

This is a generated file.

Project: hologo
Version: 2016/05/12 v1.11

Copyright (C) 2010-2012 by
   Heiko Oberdiek <heiko.oberdiek at googlemail.com>

This work may be distributed and/or modified under the
conditions of the LaTeX Project Public License, either
version 1.3c of this license or (at your option) any later
version. This version of this license is in
   http://www.latex-project.org/lppl/lppl-1-3c.txt
and the latest version of this license is in
   http://www.latex-project.org/lppl.txt
and version 1.3 or later is part of all distributions of
LaTeX version 2005/12/01 or later.

This work has the LPPL maintenance status "maintained".

This Current Maintainer of this work is Heiko Oberdiek.

The Base Interpreter refers to any `TeX-Format',
because some files are installed in TDS:tex/generic//.

This work consists of the main source file hologo.dtx
and the derived files
   hologo.sty, hologo.pdf, hologo.ins, hologo.drv, hologo-example.tex,
   hologo-test1.tex, hologo-test-spacefactor.tex,
   hologo-test-list.tex.

\endpreamble
\let\MetaPrefix\DoubleperCent

\generate{%
  \file{hologo.ins}{\from{hologo.dtx}{install}}%
  \file{hologo.drv}{\from{hologo.dtx}{driver}}%
  \usedir{tex/generic/oberdiek}%
  \file{hologo.sty}{\from{hologo.dtx}{package}}%
  \usedir{doc/latex/oberdiek/example}%
  \file{hologo-example.tex}{\from{hologo.dtx}{example}}%
  \usedir{doc/latex/oberdiek/test}%
  \file{hologo-test1.tex}{\from{hologo.dtx}{test1}}%
  \file{hologo-test-spacefactor.tex}{\from{hologo.dtx}{test-spacefactor}}%
  \file{hologo-test-list.tex}{\from{hologo.dtx}{test-list}}%
  \nopreamble
  \nopostamble
  \usedir{source/latex/oberdiek/catalogue}%
  \file{hologo.xml}{\from{hologo.dtx}{catalogue}}%
}

\catcode32=13\relax% active space
\let =\space%
\Msg{************************************************************************}
\Msg{*}
\Msg{* To finish the installation you have to move the following}
\Msg{* file into a directory searched by TeX:}
\Msg{*}
\Msg{*     hologo.sty}
\Msg{*}
\Msg{* To produce the documentation run the file `hologo.drv'}
\Msg{* through LaTeX.}
\Msg{*}
\Msg{* Happy TeXing!}
\Msg{*}
\Msg{************************************************************************}

\endbatchfile
%</install>
%<*ignore>
\fi
%</ignore>
%<*driver>
\NeedsTeXFormat{LaTeX2e}
\ProvidesFile{hologo.drv}%
  [2016/05/12 v1.11 A logo collection with bookmark support (HO)]%
\documentclass{ltxdoc}
\usepackage{holtxdoc}[2011/11/22]
\usepackage{hologo}[2016/05/12]
\usepackage{longtable}
\usepackage{array}
\usepackage{paralist}
%\usepackage[T1]{fontenc}
%\usepackage{lmodern}
\begin{document}
  \DocInput{hologo.dtx}%
\end{document}
%</driver>
% \fi
%
%
% \CharacterTable
%  {Upper-case    \A\B\C\D\E\F\G\H\I\J\K\L\M\N\O\P\Q\R\S\T\U\V\W\X\Y\Z
%   Lower-case    \a\b\c\d\e\f\g\h\i\j\k\l\m\n\o\p\q\r\s\t\u\v\w\x\y\z
%   Digits        \0\1\2\3\4\5\6\7\8\9
%   Exclamation   \!     Double quote  \"     Hash (number) \#
%   Dollar        \$     Percent       \%     Ampersand     \&
%   Acute accent  \'     Left paren    \(     Right paren   \)
%   Asterisk      \*     Plus          \+     Comma         \,
%   Minus         \-     Point         \.     Solidus       \/
%   Colon         \:     Semicolon     \;     Less than     \<
%   Equals        \=     Greater than  \>     Question mark \?
%   Commercial at \@     Left bracket  \[     Backslash     \\
%   Right bracket \]     Circumflex    \^     Underscore    \_
%   Grave accent  \`     Left brace    \{     Vertical bar  \|
%   Right brace   \}     Tilde         \~}
%
% \GetFileInfo{hologo.drv}
%
% \title{The \xpackage{hologo} package}
% \date{2016/05/12 v1.11}
% \author{Heiko Oberdiek\\\xemail{heiko.oberdiek at googlemail.com}}
%
% \maketitle
%
% \begin{abstract}
% This package starts a collection of logos with support for bookmarks
% strings.
% \end{abstract}
%
% \tableofcontents
%
% \section{Documentation}
%
% \subsection{Logo macros}
%
% \begin{declcs}{hologo} \M{name}
% \end{declcs}
% Macro \cs{hologo} sets the logo with name \meta{name}.
% The following table shows the supported names.
%
% \begingroup
%   \def\hologoEntry#1#2#3{^^A
%     #1&#2&\hologoLogoSetup{#1}{variant=#2}\hologo{#1}&#3\tabularnewline
%   }
%   \begin{longtable}{>{\ttfamily}l>{\ttfamily}lll}
%     \rmfamily\bfseries{name} & \rmfamily\bfseries variant
%     & \bfseries logo & \bfseries since\\
%     \hline
%     \endhead
%     \hologoList
%   \end{longtable}
% \endgroup
%
% \begin{declcs}{Hologo} \M{name}
% \end{declcs}
% Macro \cs{Hologo} starts the logo \meta{name} with an uppercase
% letter. As an exception small greek letters are not converted
% to uppercase. Examples, see \hologo{eTeX} and \hologo{ExTeX}.
%
% \subsection{Setup macros}
%
% The package does not support package options, but the following
% setup macros can be used to set options.
%
% \begin{declcs}{hologoSetup} \M{key value list}
% \end{declcs}
% Macro \cs{hologoSetup} sets global options.
%
% \begin{declcs}{hologoLogoSetup} \M{logo} \M{key value list}
% \end{declcs}
% Some options can also be used to configure a logo.
% These settings take precedence over global option settings.
%
% \subsection{Options}\label{sec:options}
%
% There are boolean and string options:
% \begin{description}
% \item[Boolean option:]
% It takes |true| or |false|
% as value. If the value is omitted, then |true| is used.
% \item[String option:]
% A value must be given as string. (But the string might be empty.)
% \end{description}
% The following options can be used both in \cs{hologoSetup}
% and \cs{hologoLogoSetup}:
% \begin{description}
% \def\entry#1{\item[\xoption{#1}:]}
% \entry{break}
%   enables or disables line breaks inside the logo. This setting is
%   refined by options \xoption{hyphenbreak}, \xoption{spacebreak}
%   or \xoption{discretionarybreak}.
%   Default is |false|.
% \entry{hyphenbreak}
%   enables or disables the line break right after the hyphen character.
% \entry{spacebreak}
%   enables or disables line breaks at space characters.
% \entry{discretionarybreak}
%   enables or disables line breaks at hyphenation points
%   (inserted by \cs{-}).
% \end{description}
% Macro \cs{hologoLogoSetup} also knows:
% \begin{description}
% \item[\xoption{variant}:]
%   This is a string option. It specifies a variant of a logo that
%   must exist. An empty string selects the package default variant.
% \end{description}
% Example:
% \begin{quote}
%   |\hologoSetup{break=false}|\\
%   |\hologoLogoSetup{plainTeX}{variant=hyphen,hyphenbreak}|\\
%   Then ``plain-\TeX'' contains one break point after the hyphen.
% \end{quote}
%
% \subsection{Driver options}
%
% Sometimes graphical operations are needed to construct some
% glyphs (e.g.\ \hologo{XeTeX}). If package \xpackage{graphics}
% or package \xpackage{pgf} are found, then the macros are taken
% from there. Otherwise the packge defines its own operations
% and therefore needs the driver information. Many drivers are
% detected automatically (\hologo{pdfTeX}/\hologo{LuaTeX}
% in PDF mode, \hologo{XeTeX}, \hologo{VTeX}). These have precedence
% over a driver option. The driver can be given as package option
% or using \cs{hologoDriverSetup}.
% The following list contains the recognized driver options:
% \begin{itemize}
% \item \xoption{pdftex}, \xoption{luatex}
% \item \xoption{dvipdfm}, \xoption{dvipdfmx}
% \item \xoption{dvips}, \xoption{dvipsone}, \xoption{xdvi}
% \item \xoption{xetex}
% \item \xoption{vtex}
% \end{itemize}
% The left driver of a line is the driver name that is used internally.
% The following names are aliases for drivers that use the
% same method. Therefore the entry in the \xext{log} file for
% the used driver prints the internally used driver name.
% \begin{description}
% \item[\xoption{driverfallback}:]
%   This option expects a driver that is used,
%   if the driver could not be detected automatically.
% \end{description}
%
% \begin{declcs}{hologoDriverSetup} \M{driver option}
% \end{declcs}
% The driver can also be configured after package loading
% using \cs{hologoDriverSetup}, also the way for \hologo{plainTeX}
% to setup the driver.
%
% \subsection{Font setup}
%
% Some logos require a special font, but should also be usable by
% \hologo{plainTeX}. Therefore the package provides some ways
% to influence the font settings. The options below
% take font settings as values. Both font commands
% such as \cs{sffamily} and macros that take one argument
% like \cs{textsf} can be used.
%
% \begin{declcs}{hologoFontSetup} \M{key value list}
% \end{declcs}
% Macro \cs{hologoFontSetup} sets the fonts for all logos.
% Supported keys:
% \begin{description}
% \def\entry#1{\item[\xoption{#1}:]}
% \entry{general}
%   This font is used for all logos. The default is empty.
%   That means no special font is used.
% \entry{bibsf}
%   This font is used for
%   {\hologoLogoSetup{BibTeX}{variant=sf}\hologo{BibTeX}}
%   with variant \xoption{sf}.
% \entry{rm}
%   This font is a serif font. It is used for \hologo{ExTeX}.
% \entry{sc}
%   This font specifies a small caps font. It is used for
%   {\hologoLogoSetup{BibTeX}{variant=sc}\hologo{BibTeX}}
%   with variant \xoption{sc}.
% \entry{sf}
%   This font specifies a sans serif font. The default
%   is \cs{sffamily}, then \cs{sf} is tried. Otherwise
%   a warning is given. It is used by \hologo{KOMAScript}.
% \entry{sy}
%   This is the font for math symbols (e.g. cmsy).
%   It is used by \hologo{AmS}, \hologo{NTS}, \hologo{ExTeX}.
% \entry{logo}
%   \hologo{METAFONT} and \hologo{METAPOST} are using that font.
%   In \hologo{LaTeX} \cs{logofamily} is used and
%   the definitions of package \xpackage{mflogo} are used
%   if the package is not loaded.
%   Otherwise the \cs{tenlogo} is used and defined
%   if it does not already exists.
% \end{description}
%
% \begin{declcs}{hologoLogoFontSetup} \M{logo} \M{key value list}
% \end{declcs}
% Fonts can also be set for a logo or logo component separately,
% see the following list.
% The keys are the same as for \cs{hologoFontSetup}.
%
% \begin{longtable}{>{\ttfamily}l>{\sffamily}ll}
%   \meta{logo} & keys & result\\
%   \hline
%   \endhead
%   BibTeX & bibsf & {\hologoLogoSetup{BibTeX}{variant=sf}\hologo{BibTeX}}\\[.5ex]
%   BibTeX & sc & {\hologoLogoSetup{BibTeX}{variant=sc}\hologo{BibTeX}}\\[.5ex]
%   ExTeX & rm & \hologo{ExTeX}\\
%   SliTeX & rm & \hologo{SliTeX}\\[.5ex]
%   AmS & sy & \hologo{AmS}\\
%   ExTeX & sy & \hologo{ExTeX}\\
%   NTS & sy & \hologo{NTS}\\[.5ex]
%   KOMAScript & sf & \hologo{KOMAScript}\\[.5ex]
%   METAFONT & logo & \hologo{METAFONT}\\
%   METAPOST & logo & \hologo{METAPOST}\\[.5ex]
%   SliTeX & sc \hologo{SliTeX}
% \end{longtable}
%
% \subsubsection{Font order}
%
% For all logos the font \xoption{general} is applied first.
% Example:
%\begin{quote}
%|\hologoFontSetup{general=\color{red}}|
%\end{quote}
% will print red logos.
% Then if the font uses a special font \xoption{sf}, for example,
% the font is applied that is setup by \cs{hologoLogoFontSetup}.
% If this font is not setup, then the common font setup
% by \cs{hologoFontSetup} is used. Otherwise a warning is given,
% that there is no font configured.
%
% \subsection{Additional user macros}
%
% Usually a variant of a logo is configured by using
% \cs{hologoLogoSetup}, because it is bad style to mix
% different variants of the same logo in the same text.
% There the following macros are a convenience for testing.
%
% \begin{declcs}{hologoVariant} \M{name} \M{variant}\\
%   \cs{HologoVariant} \M{name} \M{variant}
% \end{declcs}
% Logo \meta{name} is set using \meta{variant} that specifies
% explicitely which variant of the macro is used. If the argument
% is empty, then the default form of the logo is used
% (configurable by \cs{hologoLogoSetup}).
%
% \cs{HologoVariant} is used if the logo is set in a context
% that needs an uppercase first letter (beginning of a sentence, \dots).
%
% \begin{declcs}{hologoList}\\
%   \cs{hologoEntry} \M{logo} \M{variant} \M{since}
% \end{declcs}
% Macro \cs{hologoList} contains all logos that are provided
% by the package including variants. The list consists of calls
% of \cs{hologoEntry} with three arguments starting with the
% logo name \meta{logo} and its variant \meta{variant}. An empty
% variant means the current default. Argument \meta{since} specifies
% with version of the package \xpackage{hologo} is needed to get
% the logo. If the logo is fixed, then the date gets updated.
% Therefore the date \meta{since} is not exactly the date of
% the first introduction, but rather the date of the latest fix.
%
% Before \cs{hologoList} can be used, macro \cs{hologoEntry} needs
% a definition. The example file in section \ref{sec:example}
% shows applications of \cs{hologoList}.
%
% \subsection{Supported contexts}
%
% Macros \cs{hologo} and friends support special contexts:
% \begin{itemize}
% \item \hologo{LaTeX}'s protection mechanism.
% \item Bookmarks of package \xpackage{hyperref}.
% \item Package \xpackage{tex4ht}.
% \item The macros can be used inside \cs{csname} constructs,
%   if \cs{ifincsname} is available (\hologo{pdfTeX}, \hologo{XeTeX},
%   \hologo{LuaTeX}).
% \end{itemize}
%
% \subsection{Example}
% \label{sec:example}
%
% The following example prints the logos in different fonts.
%    \begin{macrocode}
%<*example>
%<<verbatim
\NeedsTeXFormat{LaTeX2e}
\documentclass[a4paper]{article}
\usepackage[
  hmargin=20mm,
  vmargin=20mm,
]{geometry}
\pagestyle{empty}
\usepackage{hologo}[2016/05/12]
\usepackage{longtable}
\usepackage{array}
\setlength{\extrarowheight}{2pt}
\usepackage[T1]{fontenc}
\usepackage{lmodern}
\usepackage{pdflscape}
\usepackage[
  pdfencoding=auto,
]{hyperref}
\hypersetup{
  pdfauthor={Heiko Oberdiek},
  pdftitle={Example for package `hologo'},
  pdfsubject={Logos with fonts lmr, lmss, qtm, qpl, qhv},
}
\usepackage{bookmark}

% Print the logo list on the console

\begingroup
  \typeout{}%
  \typeout{*** Begin of logo list ***}%
  \newcommand*{\hologoEntry}[3]{%
    \typeout{#1 \ifx\\#2\\\else(#2) \fi[#3]}%
  }%
  \hologoList
  \typeout{*** End of logo list ***}%
  \typeout{}%
\endgroup

\begin{document}
\begin{landscape}

  \section{Example file for package `hologo'}

  % Table for font names

  \begin{longtable}{>{\bfseries}ll}
    \textbf{font} & \textbf{Font name}\\
    \hline
    lmr & Latin Modern Roman\\
    lmss & Latin Modern Sans\\
    qtm & \TeX\ Gyre Termes\\
    qhv & \TeX\ Gyre Heros\\
    qpl & \TeX\ Gyre Pagella\\
  \end{longtable}

  % Logo list with logos in different fonts

  \begingroup
    \newcommand*{\SetVariant}[2]{%
      \ifx\\#2\\%
      \else
        \hologoLogoSetup{#1}{variant=#2}%
      \fi
    }%
    \newcommand*{\hologoEntry}[3]{%
      \SetVariant{#1}{#2}%
      \raisebox{1em}[0pt][0pt]{\hypertarget{#1@#2}{}}%
      \bookmark[%
        dest={#1@#2},%
      ]{%
        #1\ifx\\#2\\\else\space(#2)\fi: \Hologo{#1}, \hologo{#1} %
        [Unicode]%
      }%
      \hypersetup{unicode=false}%
      \bookmark[%
        dest={#1@#2},%
      ]{%
        #1\ifx\\#2\\\else\space(#2)\fi: \Hologo{#1}, \hologo{#1} %
        [PDFDocEncoding]%
      }%
      \texttt{#1}%
      &%
      \texttt{#2}%
      &%
      \Hologo{#1}%
      &%
      \SetVariant{#1}{#2}%
      \hologo{#1}%
      &%
      \SetVariant{#1}{#2}%
      \fontfamily{qtm}\selectfont
      \hologo{#1}%
      &%
      \SetVariant{#1}{#2}%
      \fontfamily{qpl}\selectfont
      \hologo{#1}%
      &%
      \SetVariant{#1}{#2}%
      \textsf{\hologo{#1}}%
      &%
      \SetVariant{#1}{#2}%
      \fontfamily{qhv}\selectfont
      \hologo{#1}%
      \tabularnewline
    }%
    \begin{longtable}{llllllll}%
      \textbf{\textit{logo}} & \textbf{\textit{variant}} &
      \texttt{\string\Hologo} &
      \textbf{lmr} & \textbf{qtm} & \textbf{qpl} &
      \textbf{lmss} & \textbf{qhv}
      \tabularnewline
      \hline
      \endhead
      \hologoList
    \end{longtable}%
  \endgroup

\end{landscape}
\end{document}
%verbatim
%</example>
%    \end{macrocode}
%
% \StopEventually{
% }
%
% \section{Implementation}
%    \begin{macrocode}
%<*package>
%    \end{macrocode}
%    Reload check, especially if the package is not used with \LaTeX.
%    \begin{macrocode}
\begingroup\catcode61\catcode48\catcode32=10\relax%
  \catcode13=5 % ^^M
  \endlinechar=13 %
  \catcode35=6 % #
  \catcode39=12 % '
  \catcode44=12 % ,
  \catcode45=12 % -
  \catcode46=12 % .
  \catcode58=12 % :
  \catcode64=11 % @
  \catcode123=1 % {
  \catcode125=2 % }
  \expandafter\let\expandafter\x\csname ver@hologo.sty\endcsname
  \ifx\x\relax % plain-TeX, first loading
  \else
    \def\empty{}%
    \ifx\x\empty % LaTeX, first loading,
      % variable is initialized, but \ProvidesPackage not yet seen
    \else
      \expandafter\ifx\csname PackageInfo\endcsname\relax
        \def\x#1#2{%
          \immediate\write-1{Package #1 Info: #2.}%
        }%
      \else
        \def\x#1#2{\PackageInfo{#1}{#2, stopped}}%
      \fi
      \x{hologo}{The package is already loaded}%
      \aftergroup\endinput
    \fi
  \fi
\endgroup%
%    \end{macrocode}
%    Package identification:
%    \begin{macrocode}
\begingroup\catcode61\catcode48\catcode32=10\relax%
  \catcode13=5 % ^^M
  \endlinechar=13 %
  \catcode35=6 % #
  \catcode39=12 % '
  \catcode40=12 % (
  \catcode41=12 % )
  \catcode44=12 % ,
  \catcode45=12 % -
  \catcode46=12 % .
  \catcode47=12 % /
  \catcode58=12 % :
  \catcode64=11 % @
  \catcode91=12 % [
  \catcode93=12 % ]
  \catcode123=1 % {
  \catcode125=2 % }
  \expandafter\ifx\csname ProvidesPackage\endcsname\relax
    \def\x#1#2#3[#4]{\endgroup
      \immediate\write-1{Package: #3 #4}%
      \xdef#1{#4}%
    }%
  \else
    \def\x#1#2[#3]{\endgroup
      #2[{#3}]%
      \ifx#1\@undefined
        \xdef#1{#3}%
      \fi
      \ifx#1\relax
        \xdef#1{#3}%
      \fi
    }%
  \fi
\expandafter\x\csname ver@hologo.sty\endcsname
\ProvidesPackage{hologo}%
  [2016/05/12 v1.11 A logo collection with bookmark support (HO)]%
%    \end{macrocode}
%
%    \begin{macrocode}
\begingroup\catcode61\catcode48\catcode32=10\relax%
  \catcode13=5 % ^^M
  \endlinechar=13 %
  \catcode123=1 % {
  \catcode125=2 % }
  \catcode64=11 % @
  \def\x{\endgroup
    \expandafter\edef\csname HOLOGO@AtEnd\endcsname{%
      \endlinechar=\the\endlinechar\relax
      \catcode13=\the\catcode13\relax
      \catcode32=\the\catcode32\relax
      \catcode35=\the\catcode35\relax
      \catcode61=\the\catcode61\relax
      \catcode64=\the\catcode64\relax
      \catcode123=\the\catcode123\relax
      \catcode125=\the\catcode125\relax
    }%
  }%
\x\catcode61\catcode48\catcode32=10\relax%
\catcode13=5 % ^^M
\endlinechar=13 %
\catcode35=6 % #
\catcode64=11 % @
\catcode123=1 % {
\catcode125=2 % }
\def\TMP@EnsureCode#1#2{%
  \edef\HOLOGO@AtEnd{%
    \HOLOGO@AtEnd
    \catcode#1=\the\catcode#1\relax
  }%
  \catcode#1=#2\relax
}
\TMP@EnsureCode{10}{12}% ^^J
\TMP@EnsureCode{33}{12}% !
\TMP@EnsureCode{34}{12}% "
\TMP@EnsureCode{36}{3}% $
\TMP@EnsureCode{38}{4}% &
\TMP@EnsureCode{39}{12}% '
\TMP@EnsureCode{40}{12}% (
\TMP@EnsureCode{41}{12}% )
\TMP@EnsureCode{42}{12}% *
\TMP@EnsureCode{43}{12}% +
\TMP@EnsureCode{44}{12}% ,
\TMP@EnsureCode{45}{12}% -
\TMP@EnsureCode{46}{12}% .
\TMP@EnsureCode{47}{12}% /
\TMP@EnsureCode{58}{12}% :
\TMP@EnsureCode{59}{12}% ;
\TMP@EnsureCode{60}{12}% <
\TMP@EnsureCode{62}{12}% >
\TMP@EnsureCode{63}{12}% ?
\TMP@EnsureCode{91}{12}% [
\TMP@EnsureCode{93}{12}% ]
\TMP@EnsureCode{94}{7}% ^ (superscript)
\TMP@EnsureCode{95}{8}% _ (subscript)
\TMP@EnsureCode{96}{12}% `
\TMP@EnsureCode{124}{12}% |
\edef\HOLOGO@AtEnd{%
  \HOLOGO@AtEnd
  \escapechar\the\escapechar\relax
  \noexpand\endinput
}
\escapechar=92 %
%    \end{macrocode}
%
% \subsection{Logo list}
%
%    \begin{macro}{\hologoList}
%    \begin{macrocode}
\def\hologoList{%
  \hologoEntry{(La)TeX}{}{2011/10/01}%
  \hologoEntry{AmSLaTeX}{}{2010/04/16}%
  \hologoEntry{AmSTeX}{}{2010/04/16}%
  \hologoEntry{biber}{}{2011/10/01}%
  \hologoEntry{BibTeX}{}{2011/10/01}%
  \hologoEntry{BibTeX}{sf}{2011/10/01}%
  \hologoEntry{BibTeX}{sc}{2011/10/01}%
  \hologoEntry{BibTeX8}{}{2011/11/22}%
  \hologoEntry{ConTeXt}{}{2011/03/25}%
  \hologoEntry{ConTeXt}{narrow}{2011/03/25}%
  \hologoEntry{ConTeXt}{simple}{2011/03/25}%
  \hologoEntry{emTeX}{}{2010/04/26}%
  \hologoEntry{eTeX}{}{2010/04/08}%
  \hologoEntry{ExTeX}{}{2011/10/01}%
  \hologoEntry{HanTheThanh}{}{2011/11/29}%
  \hologoEntry{iniTeX}{}{2011/10/01}%
  \hologoEntry{KOMAScript}{}{2011/10/01}%
  \hologoEntry{La}{}{2010/05/08}%
  \hologoEntry{LaTeX}{}{2010/04/08}%
  \hologoEntry{LaTeX2e}{}{2010/04/08}%
  \hologoEntry{LaTeX3}{}{2010/04/24}%
  \hologoEntry{LaTeXe}{}{2010/04/08}%
  \hologoEntry{LaTeXML}{}{2011/11/22}%
  \hologoEntry{LaTeXTeX}{}{2011/10/01}%
  \hologoEntry{LuaLaTeX}{}{2010/04/08}%
  \hologoEntry{LuaTeX}{}{2010/04/08}%
  \hologoEntry{LyX}{}{2011/10/01}%
  \hologoEntry{METAFONT}{}{2011/10/01}%
  \hologoEntry{MetaFun}{}{2011/10/01}%
  \hologoEntry{METAPOST}{}{2011/10/01}%
  \hologoEntry{MetaPost}{}{2011/10/01}%
  \hologoEntry{MiKTeX}{}{2011/10/01}%
  \hologoEntry{NTS}{}{2011/10/01}%
  \hologoEntry{OzMF}{}{2011/10/01}%
  \hologoEntry{OzMP}{}{2011/10/01}%
  \hologoEntry{OzTeX}{}{2011/10/01}%
  \hologoEntry{OzTtH}{}{2011/10/01}%
  \hologoEntry{PCTeX}{}{2011/10/01}%
  \hologoEntry{pdfTeX}{}{2011/10/01}%
  \hologoEntry{pdfLaTeX}{}{2011/10/01}%
  \hologoEntry{PiC}{}{2011/10/01}%
  \hologoEntry{PiCTeX}{}{2011/10/01}%
  \hologoEntry{plainTeX}{}{2010/04/08}%
  \hologoEntry{plainTeX}{space}{2010/04/16}%
  \hologoEntry{plainTeX}{hyphen}{2010/04/16}%
  \hologoEntry{plainTeX}{runtogether}{2010/04/16}%
  \hologoEntry{SageTeX}{}{2011/11/22}%
  \hologoEntry{SLiTeX}{}{2011/10/01}%
  \hologoEntry{SLiTeX}{lift}{2011/10/01}%
  \hologoEntry{SLiTeX}{narrow}{2011/10/01}%
  \hologoEntry{SLiTeX}{simple}{2011/10/01}%
  \hologoEntry{SliTeX}{}{2011/10/01}%
  \hologoEntry{SliTeX}{narrow}{2011/10/01}%
  \hologoEntry{SliTeX}{simple}{2011/10/01}%
  \hologoEntry{SliTeX}{lift}{2011/10/01}%
  \hologoEntry{teTeX}{}{2011/10/01}%
  \hologoEntry{TeX}{}{2010/04/08}%
  \hologoEntry{TeX4ht}{}{2011/11/22}%
  \hologoEntry{TTH}{}{2011/11/22}%
  \hologoEntry{virTeX}{}{2011/10/01}%
  \hologoEntry{VTeX}{}{2010/04/24}%
  \hologoEntry{Xe}{}{2010/04/08}%
  \hologoEntry{XeLaTeX}{}{2010/04/08}%
  \hologoEntry{XeTeX}{}{2010/04/08}%
}
%    \end{macrocode}
%    \end{macro}
%
% \subsection{Load resources}
%
%    \begin{macrocode}
\begingroup\expandafter\expandafter\expandafter\endgroup
\expandafter\ifx\csname RequirePackage\endcsname\relax
  \def\TMP@RequirePackage#1[#2]{%
    \begingroup\expandafter\expandafter\expandafter\endgroup
    \expandafter\ifx\csname ver@#1.sty\endcsname\relax
      \input #1.sty\relax
    \fi
  }%
  \TMP@RequirePackage{ltxcmds}[2011/02/04]%
  \TMP@RequirePackage{infwarerr}[2010/04/08]%
  \TMP@RequirePackage{kvsetkeys}[2010/03/01]%
  \TMP@RequirePackage{kvdefinekeys}[2010/03/01]%
  \TMP@RequirePackage{pdftexcmds}[2010/04/01]%
  \TMP@RequirePackage{ifpdf}[2010/01/28]%
  \TMP@RequirePackage{ifluatex}[2010/03/01]%
  \ltx@IfUndefined{newif}{%
    \expandafter\let\csname newif\endcsname\ltx@newif
  }{}%
  \TMP@RequirePackage{ifxetex}[2009/01/23]%
  \TMP@RequirePackage{ifvtex}[2010/03/01]%
\else
  \RequirePackage{ltxcmds}[2011/02/04]%
  \RequirePackage{infwarerr}[2010/04/08]%
  \RequirePackage{kvsetkeys}[2010/03/01]%
  \RequirePackage{kvdefinekeys}[2010/03/01]%
  \RequirePackage{pdftexcmds}[2010/04/01]%
  \RequirePackage{ifpdf}[2010/01/28]%
  \RequirePackage{ifluatex}[2010/03/01]%
  \RequirePackage{ifxetex}[2009/01/23]%
  \RequirePackage{ifvtex}[2010/03/01]%
\fi
%    \end{macrocode}
%
%    \begin{macro}{\HOLOGO@IfDefined}
%    \begin{macrocode}
\def\HOLOGO@IfExists#1{%
  \ifx\@undefined#1%
    \expandafter\ltx@secondoftwo
  \else
    \ifx\relax#1%
      \expandafter\ltx@secondoftwo
    \else
      \expandafter\expandafter\expandafter\ltx@firstoftwo
    \fi
  \fi
}
%    \end{macrocode}
%    \end{macro}
%
% \subsection{Setup macros}
%
%    \begin{macro}{\hologoSetup}
%    \begin{macrocode}
\def\hologoSetup{%
  \let\HOLOGO@name\relax
  \HOLOGO@Setup
}
%    \end{macrocode}
%    \end{macro}
%
%    \begin{macro}{\hologoLogoSetup}
%    \begin{macrocode}
\def\hologoLogoSetup#1{%
  \edef\HOLOGO@name{#1}%
  \ltx@IfUndefined{HoLogo@\HOLOGO@name}{%
    \@PackageError{hologo}{%
      Unknown logo `\HOLOGO@name'%
    }\@ehc
    \ltx@gobble
  }{%
    \HOLOGO@Setup
  }%
}
%    \end{macrocode}
%    \end{macro}
%
%    \begin{macro}{\HOLOGO@Setup}
%    \begin{macrocode}
\def\HOLOGO@Setup{%
  \kvsetkeys{HoLogo}%
}
%    \end{macrocode}
%    \end{macro}
%
% \subsection{Options}
%
%    \begin{macro}{\HOLOGO@DeclareBoolOption}
%    \begin{macrocode}
\def\HOLOGO@DeclareBoolOption#1{%
  \expandafter\chardef\csname HOLOGOOPT@#1\endcsname\ltx@zero
  \kv@define@key{HoLogo}{#1}[true]{%
    \def\HOLOGO@temp{##1}%
    \ifx\HOLOGO@temp\HOLOGO@true
      \ifx\HOLOGO@name\relax
        \expandafter\chardef\csname HOLOGOOPT@#1\endcsname=\ltx@one
      \else
        \expandafter\chardef\csname
        HoLogoOpt@#1@\HOLOGO@name\endcsname\ltx@one
      \fi
      \HOLOGO@SetBreakAll{#1}%
    \else
      \ifx\HOLOGO@temp\HOLOGO@false
        \ifx\HOLOGO@name\relax
          \expandafter\chardef\csname HOLOGOOPT@#1\endcsname=\ltx@zero
        \else
          \expandafter\chardef\csname
          HoLogoOpt@#1@\HOLOGO@name\endcsname=\ltx@zero
        \fi
        \HOLOGO@SetBreakAll{#1}%
      \else
        \@PackageError{hologo}{%
          Unknown value `##1' for boolean option `#1'.\MessageBreak
          Known values are `true' and `false'%
        }\@ehc
      \fi
    \fi
  }%
}
%    \end{macrocode}
%    \end{macro}
%
%    \begin{macro}{\HOLOGO@SetBreakAll}
%    \begin{macrocode}
\def\HOLOGO@SetBreakAll#1{%
  \def\HOLOGO@temp{#1}%
  \ifx\HOLOGO@temp\HOLOGO@break
    \ifx\HOLOGO@name\relax
      \chardef\HOLOGOOPT@hyphenbreak=\HOLOGOOPT@break
      \chardef\HOLOGOOPT@spacebreak=\HOLOGOOPT@break
      \chardef\HOLOGOOPT@discretionarybreak=\HOLOGOOPT@break
    \else
      \expandafter\chardef
         \csname HoLogoOpt@hyphenbreak@\HOLOGO@name\endcsname=%
         \csname HoLogoOpt@break@\HOLOGO@name\endcsname
      \expandafter\chardef
         \csname HoLogoOpt@spacebreak@\HOLOGO@name\endcsname=%
         \csname HoLogoOpt@break@\HOLOGO@name\endcsname
      \expandafter\chardef
         \csname HoLogoOpt@discretionarybreak@\HOLOGO@name
             \endcsname=%
         \csname HoLogoOpt@break@\HOLOGO@name\endcsname
    \fi
  \fi
}
%    \end{macrocode}
%    \end{macro}
%
%    \begin{macro}{\HOLOGO@true}
%    \begin{macrocode}
\def\HOLOGO@true{true}
%    \end{macrocode}
%    \end{macro}
%    \begin{macro}{\HOLOGO@false}
%    \begin{macrocode}
\def\HOLOGO@false{false}
%    \end{macrocode}
%    \end{macro}
%    \begin{macro}{\HOLOGO@break}
%    \begin{macrocode}
\def\HOLOGO@break{break}
%    \end{macrocode}
%    \end{macro}
%
%    \begin{macrocode}
\HOLOGO@DeclareBoolOption{break}
\HOLOGO@DeclareBoolOption{hyphenbreak}
\HOLOGO@DeclareBoolOption{spacebreak}
\HOLOGO@DeclareBoolOption{discretionarybreak}
%    \end{macrocode}
%
%    \begin{macrocode}
\kv@define@key{HoLogo}{variant}{%
  \ifx\HOLOGO@name\relax
    \@PackageError{hologo}{%
      Option `variant' is not available in \string\hologoSetup,%
      \MessageBreak
      Use \string\hologoLogoSetup\space instead%
    }\@ehc
  \else
    \edef\HOLOGO@temp{#1}%
    \ifx\HOLOGO@temp\ltx@empty
      \expandafter
      \let\csname HoLogoOpt@variant@\HOLOGO@name\endcsname\@undefined
    \else
      \ltx@IfUndefined{HoLogo@\HOLOGO@name @\HOLOGO@temp}{%
        \@PackageError{hologo}{%
          Unknown variant `\HOLOGO@temp' of logo `\HOLOGO@name'%
        }\@ehc
      }{%
        \expandafter
        \let\csname HoLogoOpt@variant@\HOLOGO@name\endcsname
            \HOLOGO@temp
      }%
    \fi
  \fi
}
%    \end{macrocode}
%
%    \begin{macro}{\HOLOGO@Variant}
%    \begin{macrocode}
\def\HOLOGO@Variant#1{%
  #1%
  \ltx@ifundefined{HoLogoOpt@variant@#1}{%
  }{%
    @\csname HoLogoOpt@variant@#1\endcsname
  }%
}
%    \end{macrocode}
%    \end{macro}
%
% \subsection{Break/no-break support}
%
%    \begin{macro}{\HOLOGO@space}
%    \begin{macrocode}
\def\HOLOGO@space{%
  \ltx@ifundefined{HoLogoOpt@spacebreak@\HOLOGO@name}{%
    \ltx@ifundefined{HoLogoOpt@break@\HOLOGO@name}{%
      \chardef\HOLOGO@temp=\HOLOGOOPT@spacebreak
    }{%
      \chardef\HOLOGO@temp=%
        \csname HoLogoOpt@break@\HOLOGO@name\endcsname
    }%
  }{%
    \chardef\HOLOGO@temp=%
      \csname HoLogoOpt@spacebreak@\HOLOGO@name\endcsname
  }%
  \ifcase\HOLOGO@temp
    \penalty10000 %
  \fi
  \ltx@space
}
%    \end{macrocode}
%    \end{macro}
%
%    \begin{macro}{\HOLOGO@hyphen}
%    \begin{macrocode}
\def\HOLOGO@hyphen{%
  \ltx@ifundefined{HoLogoOpt@hyphenbreak@\HOLOGO@name}{%
    \ltx@ifundefined{HoLogoOpt@break@\HOLOGO@name}{%
      \chardef\HOLOGO@temp=\HOLOGOOPT@hyphenbreak
    }{%
      \chardef\HOLOGO@temp=%
        \csname HoLogoOpt@break@\HOLOGO@name\endcsname
    }%
  }{%
    \chardef\HOLOGO@temp=%
      \csname HoLogoOpt@hyphenbreak@\HOLOGO@name\endcsname
  }%
  \ifcase\HOLOGO@temp
    \ltx@mbox{-}%
  \else
    -%
  \fi
}
%    \end{macrocode}
%    \end{macro}
%
%    \begin{macro}{\HOLOGO@discretionary}
%    \begin{macrocode}
\def\HOLOGO@discretionary{%
  \ltx@ifundefined{HoLogoOpt@discretionarybreak@\HOLOGO@name}{%
    \ltx@ifundefined{HoLogoOpt@break@\HOLOGO@name}{%
      \chardef\HOLOGO@temp=\HOLOGOOPT@discretionarybreak
    }{%
      \chardef\HOLOGO@temp=%
        \csname HoLogoOpt@break@\HOLOGO@name\endcsname
    }%
  }{%
    \chardef\HOLOGO@temp=%
      \csname HoLogoOpt@discretionarybreak@\HOLOGO@name\endcsname
  }%
  \ifcase\HOLOGO@temp
  \else
    \-%
  \fi
}
%    \end{macrocode}
%    \end{macro}
%
%    \begin{macro}{\HOLOGO@mbox}
%    \begin{macrocode}
\def\HOLOGO@mbox#1{%
  \ltx@ifundefined{HoLogoOpt@break@\HOLOGO@name}{%
    \chardef\HOLOGO@temp=\HOLOGOOPT@hyphenbreak
  }{%
    \chardef\HOLOGO@temp=%
      \csname HoLogoOpt@break@\HOLOGO@name\endcsname
  }%
  \ifcase\HOLOGO@temp
    \ltx@mbox{#1}%
  \else
    #1%
  \fi
}
%    \end{macrocode}
%    \end{macro}
%
% \subsection{Font support}
%
%    \begin{macro}{\HoLogoFont@font}
%    \begin{tabular}{@{}ll@{}}
%    |#1|:& logo name\\
%    |#2|:& font short name\\
%    |#3|:& text
%    \end{tabular}
%    \begin{macrocode}
\def\HoLogoFont@font#1#2#3{%
  \begingroup
    \ltx@IfUndefined{HoLogoFont@logo@#1.#2}{%
      \ltx@IfUndefined{HoLogoFont@font@#2}{%
        \@PackageWarning{hologo}{%
          Missing font `#2' for logo `#1'%
        }%
        #3%
      }{%
        \csname HoLogoFont@font@#2\endcsname{#3}%
      }%
    }{%
      \csname HoLogoFont@logo@#1.#2\endcsname{#3}%
    }%
  \endgroup
}
%    \end{macrocode}
%    \end{macro}
%
%    \begin{macro}{\HoLogoFont@Def}
%    \begin{macrocode}
\def\HoLogoFont@Def#1{%
  \expandafter\def\csname HoLogoFont@font@#1\endcsname
}
%    \end{macrocode}
%    \end{macro}
%    \begin{macro}{\HoLogoFont@LogoDef}
%    \begin{macrocode}
\def\HoLogoFont@LogoDef#1#2{%
  \expandafter\def\csname HoLogoFont@logo@#1.#2\endcsname
}
%    \end{macrocode}
%    \end{macro}
%
% \subsubsection{Font defaults}
%
%    \begin{macro}{\HoLogoFont@font@general}
%    \begin{macrocode}
\HoLogoFont@Def{general}{}%
%    \end{macrocode}
%    \end{macro}
%
%    \begin{macro}{\HoLogoFont@font@rm}
%    \begin{macrocode}
\ltx@IfUndefined{rmfamily}{%
  \ltx@IfUndefined{rm}{%
  }{%
    \HoLogoFont@Def{rm}{\rm}%
  }%
}{%
  \HoLogoFont@Def{rm}{\rmfamily}%
}
%    \end{macrocode}
%    \end{macro}
%
%    \begin{macro}{\HoLogoFont@font@sf}
%    \begin{macrocode}
\ltx@IfUndefined{sffamily}{%
  \ltx@IfUndefined{sf}{%
  }{%
    \HoLogoFont@Def{sf}{\sf}%
  }%
}{%
  \HoLogoFont@Def{sf}{\sffamily}%
}
%    \end{macrocode}
%    \end{macro}
%
%    \begin{macro}{\HoLogoFont@font@bibsf}
%    In case of \hologo{plainTeX} the original small caps
%    variant is used as default. In \hologo{LaTeX}
%    the definition of package \xpackage{dtklogos} \cite{dtklogos}
%    is used.
%\begin{quote}
%\begin{verbatim}
%\DeclareRobustCommand{\BibTeX}{%
%  B%
%  \kern-.05em%
%  \hbox{%
%    $\m@th$% %% force math size calculations
%    \csname S@\f@size\endcsname
%    \fontsize\sf@size\z@
%    \math@fontsfalse
%    \selectfont
%    I%
%    \kern-.025em%
%    B
%  }%
%  \kern-.08em%
%  \-%
%  \TeX
%}
%\end{verbatim}
%\end{quote}
%    \begin{macrocode}
\ltx@IfUndefined{selectfont}{%
  \ltx@IfUndefined{tensc}{%
    \font\tensc=cmcsc10\relax
  }{}%
  \HoLogoFont@Def{bibsf}{\tensc}%
}{%
  \HoLogoFont@Def{bibsf}{%
    $\mathsurround=0pt$%
    \csname S@\f@size\endcsname
    \fontsize\sf@size{0pt}%
    \math@fontsfalse
    \selectfont
  }%
}
%    \end{macrocode}
%    \end{macro}
%
%    \begin{macro}{\HoLogoFont@font@sc}
%    \begin{macrocode}
\ltx@IfUndefined{scshape}{%
  \ltx@IfUndefined{tensc}{%
    \font\tensc=cmcsc10\relax
  }{}%
  \HoLogoFont@Def{sc}{\tensc}%
}{%
  \HoLogoFont@Def{sc}{\scshape}%
}
%    \end{macrocode}
%    \end{macro}
%
%    \begin{macro}{\HoLogoFont@font@sy}
%    \begin{macrocode}
\ltx@IfUndefined{usefont}{%
  \ltx@IfUndefined{tensy}{%
  }{%
    \HoLogoFont@Def{sy}{\tensy}%
  }%
}{%
  \HoLogoFont@Def{sy}{%
    \usefont{OMS}{cmsy}{m}{n}%
  }%
}
%    \end{macrocode}
%    \end{macro}
%
%    \begin{macro}{\HoLogoFont@font@logo}
%    \begin{macrocode}
\begingroup
  \def\x{LaTeX2e}%
\expandafter\endgroup
\ifx\fmtname\x
  \ltx@IfUndefined{logofamily}{%
    \DeclareRobustCommand\logofamily{%
      \not@math@alphabet\logofamily\relax
      \fontencoding{U}%
      \fontfamily{logo}%
      \selectfont
    }%
  }{}%
  \ltx@IfUndefined{logofamily}{%
  }{%
    \HoLogoFont@Def{logo}{\logofamily}%
  }%
\else
  \ltx@IfUndefined{tenlogo}{%
    \font\tenlogo=logo10\relax
  }{}%
  \HoLogoFont@Def{logo}{\tenlogo}%
\fi
%    \end{macrocode}
%    \end{macro}
%
% \subsubsection{Font setup}
%
%    \begin{macro}{\hologoFontSetup}
%    \begin{macrocode}
\def\hologoFontSetup{%
  \let\HOLOGO@name\relax
  \HOLOGO@FontSetup
}
%    \end{macrocode}
%    \end{macro}
%
%    \begin{macro}{\hologoLogoFontSetup}
%    \begin{macrocode}
\def\hologoLogoFontSetup#1{%
  \edef\HOLOGO@name{#1}%
  \ltx@IfUndefined{HoLogo@\HOLOGO@name}{%
    \@PackageError{hologo}{%
      Unknown logo `\HOLOGO@name'%
    }\@ehc
    \ltx@gobble
  }{%
    \HOLOGO@FontSetup
  }%
}
%    \end{macrocode}
%    \end{macro}
%
%    \begin{macro}{\HOLOGO@FontSetup}
%    \begin{macrocode}
\def\HOLOGO@FontSetup{%
  \kvsetkeys{HoLogoFont}%
}
%    \end{macrocode}
%    \end{macro}
%
%    \begin{macrocode}
\def\HOLOGO@temp#1{%
  \kv@define@key{HoLogoFont}{#1}{%
    \ifx\HOLOGO@name\relax
      \HoLogoFont@Def{#1}{##1}%
    \else
      \HoLogoFont@LogoDef\HOLOGO@name{#1}{##1}%
    \fi
  }%
}
\HOLOGO@temp{general}
\HOLOGO@temp{sf}
%    \end{macrocode}
%
% \subsection{Generic logo commands}
%
%    \begin{macrocode}
\HOLOGO@IfExists\hologo{%
  \@PackageError{hologo}{%
    \string\hologo\ltx@space is already defined.\MessageBreak
    Package loading is aborted%
  }\@ehc
  \HOLOGO@AtEnd
}%
\HOLOGO@IfExists\hologoRobust{%
  \@PackageError{hologo}{%
    \string\hologoRobust\ltx@space is already defined.\MessageBreak
    Package loading is aborted%
  }\@ehc
  \HOLOGO@AtEnd
}%
%    \end{macrocode}
%
% \subsubsection{\cs{hologo} and friends}
%
%    \begin{macrocode}
\ifluatex
  \expandafter\ltx@firstofone
\else
  \expandafter\ltx@gobble
\fi
{%
  \ltx@IfUndefined{ifincsname}{%
    \ifnum\luatexversion<36 %
      \expandafter\ltx@gobble
    \else
      \expandafter\ltx@firstofone
    \fi
    {%
      \begingroup
        \ifcase0%
            \directlua{%
              if tex.enableprimitives then %
                tex.enableprimitives('HOLOGO@', {'ifincsname'})%
              else %
                tex.print('1')%
              end%
            }%
            \ifx\HOLOGO@ifincsname\@undefined 1\fi%
            \relax
          \expandafter\ltx@firstofone
        \else
          \endgroup
          \expandafter\ltx@gobble
        \fi
        {%
          \global\let\ifincsname\HOLOGO@ifincsname
        }%
      \HOLOGO@temp
    }%
  }{}%
}
%    \end{macrocode}
%    \begin{macrocode}
\ltx@IfUndefined{ifincsname}{%
  \catcode`$=14 %
}{%
  \catcode`$=9 %
}
%    \end{macrocode}
%
%    \begin{macro}{\hologo}
%    \begin{macrocode}
\def\hologo#1{%
$ \ifincsname
$   \ltx@ifundefined{HoLogoCs@\HOLOGO@Variant{#1}}{%
$     #1%
$   }{%
$     \csname HoLogoCs@\HOLOGO@Variant{#1}\endcsname\ltx@firstoftwo
$   }%
$ \else
    \HOLOGO@IfExists\texorpdfstring\texorpdfstring\ltx@firstoftwo
    {%
      \hologoRobust{#1}%
    }{%
      \ltx@ifundefined{HoLogoBkm@\HOLOGO@Variant{#1}}{%
        \ltx@ifundefined{HoLogo@#1}{?#1?}{#1}%
      }{%
        \csname HoLogoBkm@\HOLOGO@Variant{#1}\endcsname
        \ltx@firstoftwo
      }%
    }%
$ \fi
}
%    \end{macrocode}
%    \end{macro}
%    \begin{macro}{\Hologo}
%    \begin{macrocode}
\def\Hologo#1{%
$ \ifincsname
$   \ltx@ifundefined{HoLogoCs@\HOLOGO@Variant{#1}}{%
$     #1%
$   }{%
$     \csname HoLogoCs@\HOLOGO@Variant{#1}\endcsname\ltx@secondoftwo
$   }%
$ \else
    \HOLOGO@IfExists\texorpdfstring\texorpdfstring\ltx@firstoftwo
    {%
      \HologoRobust{#1}%
    }{%
      \ltx@ifundefined{HoLogoBkm@\HOLOGO@Variant{#1}}{%
        \ltx@ifundefined{HoLogo@#1}{?#1?}{#1}%
      }{%
        \csname HoLogoBkm@\HOLOGO@Variant{#1}\endcsname
        \ltx@secondoftwo
      }%
    }%
$ \fi
}
%    \end{macrocode}
%    \end{macro}
%
%    \begin{macro}{\hologoVariant}
%    \begin{macrocode}
\def\hologoVariant#1#2{%
  \ifx\relax#2\relax
    \hologo{#1}%
  \else
$   \ifincsname
$     \ltx@ifundefined{HoLogoCs@#1@#2}{%
$       #1%
$     }{%
$       \csname HoLogoCs@#1@#2\endcsname\ltx@firstoftwo
$     }%
$   \else
      \HOLOGO@IfExists\texorpdfstring\texorpdfstring\ltx@firstoftwo
      {%
        \hologoVariantRobust{#1}{#2}%
      }{%
        \ltx@ifundefined{HoLogoBkm@#1@#2}{%
          \ltx@ifundefined{HoLogo@#1}{?#1?}{#1}%
        }{%
          \csname HoLogoBkm@#1@#2\endcsname
          \ltx@firstoftwo
        }%
      }%
$   \fi
  \fi
}
%    \end{macrocode}
%    \end{macro}
%    \begin{macro}{\HologoVariant}
%    \begin{macrocode}
\def\HologoVariant#1#2{%
  \ifx\relax#2\relax
    \Hologo{#1}%
  \else
$   \ifincsname
$     \ltx@ifundefined{HoLogoCs@#1@#2}{%
$       #1%
$     }{%
$       \csname HoLogoCs@#1@#2\endcsname\ltx@secondoftwo
$     }%
$   \else
      \HOLOGO@IfExists\texorpdfstring\texorpdfstring\ltx@firstoftwo
      {%
        \HologoVariantRobust{#1}{#2}%
      }{%
        \ltx@ifundefined{HoLogoBkm@#1@#2}{%
          \ltx@ifundefined{HoLogo@#1}{?#1?}{#1}%
        }{%
          \csname HoLogoBkm@#1@#2\endcsname
          \ltx@secondoftwo
        }%
      }%
$   \fi
  \fi
}
%    \end{macrocode}
%    \end{macro}
%
%    \begin{macrocode}
\catcode`\$=3 %
%    \end{macrocode}
%
% \subsubsection{\cs{hologoRobust} and friends}
%
%    \begin{macro}{\hologoRobust}
%    \begin{macrocode}
\ltx@IfUndefined{protected}{%
  \ltx@IfUndefined{DeclareRobustCommand}{%
    \def\hologoRobust#1%
  }{%
    \DeclareRobustCommand*\hologoRobust[1]%
  }%
}{%
  \protected\def\hologoRobust#1%
}%
{%
  \edef\HOLOGO@name{#1}%
  \ltx@IfUndefined{HoLogo@\HOLOGO@Variant\HOLOGO@name}{%
    \@PackageError{hologo}{%
      Unknown logo `\HOLOGO@name'%
    }\@ehc
    ?\HOLOGO@name?%
  }{%
    \ltx@IfUndefined{ver@tex4ht.sty}{%
      \HoLogoFont@font\HOLOGO@name{general}{%
        \csname HoLogo@\HOLOGO@Variant\HOLOGO@name\endcsname
        \ltx@firstoftwo
      }%
    }{%
      \ltx@IfUndefined{HoLogoHtml@\HOLOGO@Variant\HOLOGO@name}{%
        \HOLOGO@name
      }{%
        \csname HoLogoHtml@\HOLOGO@Variant\HOLOGO@name\endcsname
        \ltx@firstoftwo
      }%
    }%
  }%
}
%    \end{macrocode}
%    \end{macro}
%    \begin{macro}{\HologoRobust}
%    \begin{macrocode}
\ltx@IfUndefined{protected}{%
  \ltx@IfUndefined{DeclareRobustCommand}{%
    \def\HologoRobust#1%
  }{%
    \DeclareRobustCommand*\HologoRobust[1]%
  }%
}{%
  \protected\def\HologoRobust#1%
}%
{%
  \edef\HOLOGO@name{#1}%
  \ltx@IfUndefined{HoLogo@\HOLOGO@Variant\HOLOGO@name}{%
    \@PackageError{hologo}{%
      Unknown logo `\HOLOGO@name'%
    }\@ehc
    ?\HOLOGO@name?%
  }{%
    \ltx@IfUndefined{ver@tex4ht.sty}{%
      \HoLogoFont@font\HOLOGO@name{general}{%
        \csname HoLogo@\HOLOGO@Variant\HOLOGO@name\endcsname
        \ltx@secondoftwo
      }%
    }{%
      \ltx@IfUndefined{HoLogoHtml@\HOLOGO@Variant\HOLOGO@name}{%
        \expandafter\HOLOGO@Uppercase\HOLOGO@name
      }{%
        \csname HoLogoHtml@\HOLOGO@Variant\HOLOGO@name\endcsname
        \ltx@secondoftwo
      }%
    }%
  }%
}
%    \end{macrocode}
%    \end{macro}
%    \begin{macro}{\hologoVariantRobust}
%    \begin{macrocode}
\ltx@IfUndefined{protected}{%
  \ltx@IfUndefined{DeclareRobustCommand}{%
    \def\hologoVariantRobust#1#2%
  }{%
    \DeclareRobustCommand*\hologoVariantRobust[2]%
  }%
}{%
  \protected\def\hologoVariantRobust#1#2%
}%
{%
  \begingroup
    \hologoLogoSetup{#1}{variant={#2}}%
    \hologoRobust{#1}%
  \endgroup
}
%    \end{macrocode}
%    \end{macro}
%    \begin{macro}{\HologoVariantRobust}
%    \begin{macrocode}
\ltx@IfUndefined{protected}{%
  \ltx@IfUndefined{DeclareRobustCommand}{%
    \def\HologoVariantRobust#1#2%
  }{%
    \DeclareRobustCommand*\HologoVariantRobust[2]%
  }%
}{%
  \protected\def\HologoVariantRobust#1#2%
}%
{%
  \begingroup
    \hologoLogoSetup{#1}{variant={#2}}%
    \HologoRobust{#1}%
  \endgroup
}
%    \end{macrocode}
%    \end{macro}
%
%    \begin{macro}{\hologorobust}
%    Macro \cs{hologorobust} is only defined for compatibility.
%    Its use is deprecated.
%    \begin{macrocode}
\def\hologorobust{\hologoRobust}
%    \end{macrocode}
%    \end{macro}
%
% \subsection{Helpers}
%
%    \begin{macro}{\HOLOGO@Uppercase}
%    Macro \cs{HOLOGO@Uppercase} is restricted to \cs{uppercase},
%    because \hologo{plainTeX} or \hologo{iniTeX} do not provide
%    \cs{MakeUppercase}.
%    \begin{macrocode}
\def\HOLOGO@Uppercase#1{\uppercase{#1}}
%    \end{macrocode}
%    \end{macro}
%
%    \begin{macro}{\HOLOGO@PdfdocUnicode}
%    \begin{macrocode}
\def\HOLOGO@PdfdocUnicode{%
  \ifx\ifHy@unicode\iftrue
    \expandafter\ltx@secondoftwo
  \else
    \expandafter\ltx@firstoftwo
  \fi
}
%    \end{macrocode}
%    \end{macro}
%
%    \begin{macro}{\HOLOGO@Math}
%    \begin{macrocode}
\def\HOLOGO@MathSetup{%
  \mathsurround0pt\relax
  \HOLOGO@IfExists\f@series{%
    \if b\expandafter\ltx@car\f@series x\@nil
      \csname boldmath\endcsname
   \fi
  }{}%
}
%    \end{macrocode}
%    \end{macro}
%
%    \begin{macro}{\HOLOGO@TempDimen}
%    \begin{macrocode}
\dimendef\HOLOGO@TempDimen=\ltx@zero
%    \end{macrocode}
%    \end{macro}
%    \begin{macro}{\HOLOGO@NegativeKerning}
%    \begin{macrocode}
\def\HOLOGO@NegativeKerning#1{%
  \begingroup
    \HOLOGO@TempDimen=0pt\relax
    \comma@parse@normalized{#1}{%
      \ifdim\HOLOGO@TempDimen=0pt %
        \expandafter\HOLOGO@@NegativeKerning\comma@entry
      \fi
      \ltx@gobble
    }%
    \ifdim\HOLOGO@TempDimen<0pt %
      \kern\HOLOGO@TempDimen
    \fi
  \endgroup
}
%    \end{macrocode}
%    \end{macro}
%    \begin{macro}{\HOLOGO@@NegativeKerning}
%    \begin{macrocode}
\def\HOLOGO@@NegativeKerning#1#2{%
  \setbox\ltx@zero\hbox{#1#2}%
  \HOLOGO@TempDimen=\wd\ltx@zero
  \setbox\ltx@zero\hbox{#1\kern0pt#2}%
  \advance\HOLOGO@TempDimen by -\wd\ltx@zero
}
%    \end{macrocode}
%    \end{macro}
%
%    \begin{macro}{\HOLOGO@SpaceFactor}
%    \begin{macrocode}
\def\HOLOGO@SpaceFactor{%
  \spacefactor1000 %
}
%    \end{macrocode}
%    \end{macro}
%
%    \begin{macro}{\HOLOGO@Span}
%    \begin{macrocode}
\def\HOLOGO@Span#1#2{%
  \HCode{<span class="HoLogo-#1">}%
  #2%
  \HCode{</span>}%
}
%    \end{macrocode}
%    \end{macro}
%
% \subsubsection{Text subscript}
%
%    \begin{macro}{\HOLOGO@SubScript}%
%    \begin{macrocode}
\def\HOLOGO@SubScript#1{%
  \ltx@IfUndefined{textsubscript}{%
    \ltx@IfUndefined{text}{%
      \ltx@mbox{%
        \mathsurround=0pt\relax
        $%
          _{%
            \ltx@IfUndefined{sf@size}{%
              \mathrm{#1}%
            }{%
              \mbox{%
                \fontsize\sf@size{0pt}\selectfont
                #1%
              }%
            }%
          }%
        $%
      }%
    }{%
      \ltx@mbox{%
        \mathsurround=0pt\relax
        $_{\text{#1}}$%
      }%
    }%
  }{%
    \textsubscript{#1}%
  }%
}
%    \end{macrocode}
%    \end{macro}
%
% \subsection{\hologo{TeX} and friends}
%
% \subsubsection{\hologo{TeX}}
%
%    \begin{macro}{\HoLogo@TeX}
%    Source: \hologo{LaTeX} kernel.
%    \begin{macrocode}
\def\HoLogo@TeX#1{%
  T\kern-.1667em\lower.5ex\hbox{E}\kern-.125emX\HOLOGO@SpaceFactor
}
%    \end{macrocode}
%    \end{macro}
%    \begin{macro}{\HoLogoHtml@TeX}
%    \begin{macrocode}
\def\HoLogoHtml@TeX#1{%
  \HoLogoCss@TeX
  \HOLOGO@Span{TeX}{%
    T%
    \HOLOGO@Span{e}{%
      E%
    }%
    X%
  }%
}
%    \end{macrocode}
%    \end{macro}
%    \begin{macro}{\HoLogoCss@TeX}
%    \begin{macrocode}
\def\HoLogoCss@TeX{%
  \Css{%
    span.HoLogo-TeX span.HoLogo-e{%
      position:relative;%
      top:.5ex;%
      margin-left:-.1667em;%
      margin-right:-.125em;%
    }%
  }%
  \Css{%
    a span.HoLogo-TeX span.HoLogo-e{%
      text-decoration:none;%
    }%
  }%
  \global\let\HoLogoCss@TeX\relax
}
%    \end{macrocode}
%    \end{macro}
%
% \subsubsection{\hologo{plainTeX}}
%
%    \begin{macro}{\HoLogo@plainTeX@space}
%    Source: ``The \hologo{TeX}book''
%    \begin{macrocode}
\def\HoLogo@plainTeX@space#1{%
  \HOLOGO@mbox{#1{p}{P}lain}\HOLOGO@space\hologo{TeX}%
}
%    \end{macrocode}
%    \end{macro}
%    \begin{macro}{\HoLogoCs@plainTeX@space}
%    \begin{macrocode}
\def\HoLogoCs@plainTeX@space#1{#1{p}{P}lain TeX}%
%    \end{macrocode}
%    \end{macro}
%    \begin{macro}{\HoLogoBkm@plainTeX@space}
%    \begin{macrocode}
\def\HoLogoBkm@plainTeX@space#1{%
  #1{p}{P}lain \hologo{TeX}%
}
%    \end{macrocode}
%    \end{macro}
%    \begin{macro}{\HoLogoHtml@plainTeX@space}
%    \begin{macrocode}
\def\HoLogoHtml@plainTeX@space#1{%
  #1{p}{P}lain \hologo{TeX}%
}
%    \end{macrocode}
%    \end{macro}
%
%    \begin{macro}{\HoLogo@plainTeX@hyphen}
%    \begin{macrocode}
\def\HoLogo@plainTeX@hyphen#1{%
  \HOLOGO@mbox{#1{p}{P}lain}\HOLOGO@hyphen\hologo{TeX}%
}
%    \end{macrocode}
%    \end{macro}
%    \begin{macro}{\HoLogoCs@plainTeX@hyphen}
%    \begin{macrocode}
\def\HoLogoCs@plainTeX@hyphen#1{#1{p}{P}lain-TeX}
%    \end{macrocode}
%    \end{macro}
%    \begin{macro}{\HoLogoBkm@plainTeX@hyphen}
%    \begin{macrocode}
\def\HoLogoBkm@plainTeX@hyphen#1{%
  #1{p}{P}lain-\hologo{TeX}%
}
%    \end{macrocode}
%    \end{macro}
%    \begin{macro}{\HoLogoHtml@plainTeX@hyphen}
%    \begin{macrocode}
\def\HoLogoHtml@plainTeX@hyphen#1{%
  #1{p}{P}lain-\hologo{TeX}%
}
%    \end{macrocode}
%    \end{macro}
%
%    \begin{macro}{\HoLogo@plainTeX@runtogether}
%    \begin{macrocode}
\def\HoLogo@plainTeX@runtogether#1{%
  \HOLOGO@mbox{#1{p}{P}lain\hologo{TeX}}%
}
%    \end{macrocode}
%    \end{macro}
%    \begin{macro}{\HoLogoCs@plainTeX@runtogether}
%    \begin{macrocode}
\def\HoLogoCs@plainTeX@runtogether#1{#1{p}{P}lainTeX}
%    \end{macrocode}
%    \end{macro}
%    \begin{macro}{\HoLogoBkm@plainTeX@runtogether}
%    \begin{macrocode}
\def\HoLogoBkm@plainTeX@runtogether#1{%
  #1{p}{P}lain\hologo{TeX}%
}
%    \end{macrocode}
%    \end{macro}
%    \begin{macro}{\HoLogoHtml@plainTeX@runtogether}
%    \begin{macrocode}
\def\HoLogoHtml@plainTeX@runtogether#1{%
  #1{p}{P}lain\hologo{TeX}%
}
%    \end{macrocode}
%    \end{macro}
%
%    \begin{macro}{\HoLogo@plainTeX}
%    \begin{macrocode}
\def\HoLogo@plainTeX{\HoLogo@plainTeX@space}
%    \end{macrocode}
%    \end{macro}
%    \begin{macro}{\HoLogoCs@plainTeX}
%    \begin{macrocode}
\def\HoLogoCs@plainTeX{\HoLogoCs@plainTeX@space}
%    \end{macrocode}
%    \end{macro}
%    \begin{macro}{\HoLogoBkm@plainTeX}
%    \begin{macrocode}
\def\HoLogoBkm@plainTeX{\HoLogoBkm@plainTeX@space}
%    \end{macrocode}
%    \end{macro}
%    \begin{macro}{\HoLogoHtml@plainTeX}
%    \begin{macrocode}
\def\HoLogoHtml@plainTeX{\HoLogoHtml@plainTeX@space}
%    \end{macrocode}
%    \end{macro}
%
% \subsubsection{\hologo{LaTeX}}
%
%    Source: \hologo{LaTeX} kernel.
%\begin{quote}
%\begin{verbatim}
%\DeclareRobustCommand{\LaTeX}{%
%  L%
%  \kern-.36em%
%  {%
%    \sbox\z@ T%
%    \vbox to\ht\z@{%
%      \hbox{%
%        \check@mathfonts
%        \fontsize\sf@size\z@
%        \math@fontsfalse
%        \selectfont
%        A%
%      }%
%      \vss
%    }%
%  }%
%  \kern-.15em%
%  \TeX
%}
%\end{verbatim}
%\end{quote}
%
%    \begin{macro}{\HoLogo@La}
%    \begin{macrocode}
\def\HoLogo@La#1{%
  L%
  \kern-.36em%
  \begingroup
    \setbox\ltx@zero\hbox{T}%
    \vbox to\ht\ltx@zero{%
      \hbox{%
        \ltx@ifundefined{check@mathfonts}{%
          \csname sevenrm\endcsname
        }{%
          \check@mathfonts
          \fontsize\sf@size{0pt}%
          \math@fontsfalse\selectfont
        }%
        A%
      }%
      \vss
    }%
  \endgroup
}
%    \end{macrocode}
%    \end{macro}
%
%    \begin{macro}{\HoLogo@LaTeX}
%    Source: \hologo{LaTeX} kernel.
%    \begin{macrocode}
\def\HoLogo@LaTeX#1{%
  \hologo{La}%
  \kern-.15em%
  \hologo{TeX}%
}
%    \end{macrocode}
%    \end{macro}
%    \begin{macro}{\HoLogoHtml@LaTeX}
%    \begin{macrocode}
\def\HoLogoHtml@LaTeX#1{%
  \HoLogoCss@LaTeX
  \HOLOGO@Span{LaTeX}{%
    L%
    \HOLOGO@Span{a}{%
      A%
    }%
    \hologo{TeX}%
  }%
}
%    \end{macrocode}
%    \end{macro}
%    \begin{macro}{\HoLogoCss@LaTeX}
%    \begin{macrocode}
\def\HoLogoCss@LaTeX{%
  \Css{%
    span.HoLogo-LaTeX span.HoLogo-a{%
      position:relative;%
      top:-.5ex;%
      margin-left:-.36em;%
      margin-right:-.15em;%
      font-size:85\%;%
    }%
  }%
  \global\let\HoLogoCss@LaTeX\relax
}
%    \end{macrocode}
%    \end{macro}
%
% \subsubsection{\hologo{(La)TeX}}
%
%    \begin{macro}{\HoLogo@LaTeXTeX}
%    The kerning around the parentheses is taken
%    from package \xpackage{dtklogos} \cite{dtklogos}.
%\begin{quote}
%\begin{verbatim}
%\DeclareRobustCommand{\LaTeXTeX}{%
%  (%
%  \kern-.15em%
%  L%
%  \kern-.36em%
%  {%
%    \sbox\z@ T%
%    \vbox to\ht0{%
%      \hbox{%
%        $\m@th$%
%        \csname S@\f@size\endcsname
%        \fontsize\sf@size\z@
%        \math@fontsfalse
%        \selectfont
%        A%
%      }%
%      \vss
%    }%
%  }%
%  \kern-.2em%
%  )%
%  \kern-.15em%
%  \TeX
%}
%\end{verbatim}
%\end{quote}
%    \begin{macrocode}
\def\HoLogo@LaTeXTeX#1{%
  (%
  \kern-.15em%
  \hologo{La}%
  \kern-.2em%
  )%
  \kern-.15em%
  \hologo{TeX}%
}
%    \end{macrocode}
%    \end{macro}
%    \begin{macro}{\HoLogoBkm@LaTeXTeX}
%    \begin{macrocode}
\def\HoLogoBkm@LaTeXTeX#1{(La)TeX}
%    \end{macrocode}
%    \end{macro}
%
%    \begin{macro}{\HoLogo@(La)TeX}
%    \begin{macrocode}
\expandafter
\let\csname HoLogo@(La)TeX\endcsname\HoLogo@LaTeXTeX
%    \end{macrocode}
%    \end{macro}
%    \begin{macro}{\HoLogoBkm@(La)TeX}
%    \begin{macrocode}
\expandafter
\let\csname HoLogoBkm@(La)TeX\endcsname\HoLogoBkm@LaTeXTeX
%    \end{macrocode}
%    \end{macro}
%    \begin{macro}{\HoLogoHtml@LaTeXTeX}
%    \begin{macrocode}
\def\HoLogoHtml@LaTeXTeX#1{%
  \HoLogoCss@LaTeXTeX
  \HOLOGO@Span{LaTeXTeX}{%
    (%
    \HOLOGO@Span{L}{L}%
    \HOLOGO@Span{a}{A}%
    \HOLOGO@Span{ParenRight}{)}%
    \hologo{TeX}%
  }%
}
%    \end{macrocode}
%    \end{macro}
%    \begin{macro}{\HoLogoHtml@(La)TeX}
%    Kerning after opening parentheses and before closing parentheses
%    is $-0.1$\,em. The original values $-0.15$\,em
%    looked too ugly for a serif font.
%    \begin{macrocode}
\expandafter
\let\csname HoLogoHtml@(La)TeX\endcsname\HoLogoHtml@LaTeXTeX
%    \end{macrocode}
%    \end{macro}
%    \begin{macro}{\HoLogoCss@LaTeXTeX}
%    \begin{macrocode}
\def\HoLogoCss@LaTeXTeX{%
  \Css{%
    span.HoLogo-LaTeXTeX span.HoLogo-L{%
      margin-left:-.1em;%
    }%
  }%
  \Css{%
    span.HoLogo-LaTeXTeX span.HoLogo-a{%
      position:relative;%
      top:-.5ex;%
      margin-left:-.36em;%
      margin-right:-.1em;%
      font-size:85\%;%
    }%
  }%
  \Css{%
    span.HoLogo-LaTeXTeX span.HoLogo-ParenRight{%
      margin-right:-.15em;%
    }%
  }%
  \global\let\HoLogoCss@LaTeXTeX\relax
}
%    \end{macrocode}
%    \end{macro}
%
% \subsubsection{\hologo{LaTeXe}}
%
%    \begin{macro}{\HoLogo@LaTeXe}
%    Source: \hologo{LaTeX} kernel
%    \begin{macrocode}
\def\HoLogo@LaTeXe#1{%
  \hologo{LaTeX}%
  \kern.15em%
  \hbox{%
    \HOLOGO@MathSetup
    2%
    $_{\textstyle\varepsilon}$%
  }%
}
%    \end{macrocode}
%    \end{macro}
%
%    \begin{macro}{\HoLogoCs@LaTeXe}
%    \begin{macrocode}
\ifnum64=`\^^^^0040\relax % test for big chars of LuaTeX/XeTeX
  \catcode`\$=9 %
  \catcode`\&=14 %
\else
  \catcode`\$=14 %
  \catcode`\&=9 %
\fi
\def\HoLogoCs@LaTeXe#1{%
  LaTeX2%
$ \string ^^^^0395%
& e%
}%
\catcode`\$=3 %
\catcode`\&=4 %
%    \end{macrocode}
%    \end{macro}
%
%    \begin{macro}{\HoLogoBkm@LaTeXe}
%    \begin{macrocode}
\def\HoLogoBkm@LaTeXe#1{%
  \hologo{LaTeX}%
  2%
  \HOLOGO@PdfdocUnicode{e}{\textepsilon}%
}
%    \end{macrocode}
%    \end{macro}
%
%    \begin{macro}{\HoLogoHtml@LaTeXe}
%    \begin{macrocode}
\def\HoLogoHtml@LaTeXe#1{%
  \HoLogoCss@LaTeXe
  \HOLOGO@Span{LaTeX2e}{%
    \hologo{LaTeX}%
    \HOLOGO@Span{2}{2}%
    \HOLOGO@Span{e}{%
      \HOLOGO@MathSetup
      \ensuremath{\textstyle\varepsilon}%
    }%
  }%
}
%    \end{macrocode}
%    \end{macro}
%    \begin{macro}{\HoLogoCss@LaTeXe}
%    \begin{macrocode}
\def\HoLogoCss@LaTeXe{%
  \Css{%
    span.HoLogo-LaTeX2e span.HoLogo-2{%
      padding-left:.15em;%
    }%
  }%
  \Css{%
    span.HoLogo-LaTeX2e span.HoLogo-e{%
      position:relative;%
      top:.35ex;%
      text-decoration:none;%
    }%
  }%
  \global\let\HoLogoCss@LaTeXe\relax
}
%    \end{macrocode}
%    \end{macro}
%
%    \begin{macro}{\HoLogo@LaTeX2e}
%    \begin{macrocode}
\expandafter
\let\csname HoLogo@LaTeX2e\endcsname\HoLogo@LaTeXe
%    \end{macrocode}
%    \end{macro}
%    \begin{macro}{\HoLogoCs@LaTeX2e}
%    \begin{macrocode}
\expandafter
\let\csname HoLogoCs@LaTeX2e\endcsname\HoLogoCs@LaTeXe
%    \end{macrocode}
%    \end{macro}
%    \begin{macro}{\HoLogoBkm@LaTeX2e}
%    \begin{macrocode}
\expandafter
\let\csname HoLogoBkm@LaTeX2e\endcsname\HoLogoBkm@LaTeXe
%    \end{macrocode}
%    \end{macro}
%    \begin{macro}{\HoLogoHtml@LaTeX2e}
%    \begin{macrocode}
\expandafter
\let\csname HoLogoHtml@LaTeX2e\endcsname\HoLogoHtml@LaTeXe
%    \end{macrocode}
%    \end{macro}
%
% \subsubsection{\hologo{LaTeX3}}
%
%    \begin{macro}{\HoLogo@LaTeX3}
%    Source: \hologo{LaTeX} kernel
%    \begin{macrocode}
\expandafter\def\csname HoLogo@LaTeX3\endcsname#1{%
  \hologo{LaTeX}%
  3%
}
%    \end{macrocode}
%    \end{macro}
%
%    \begin{macro}{\HoLogoBkm@LaTeX3}
%    \begin{macrocode}
\expandafter\def\csname HoLogoBkm@LaTeX3\endcsname#1{%
  \hologo{LaTeX}%
  3%
}
%    \end{macrocode}
%    \end{macro}
%    \begin{macro}{\HoLogoHtml@LaTeX3}
%    \begin{macrocode}
\expandafter
\let\csname HoLogoHtml@LaTeX3\expandafter\endcsname
\csname HoLogo@LaTeX3\endcsname
%    \end{macrocode}
%    \end{macro}
%
% \subsubsection{\hologo{LaTeXML}}
%
%    \begin{macro}{\HoLogo@LaTeXML}
%    \begin{macrocode}
\def\HoLogo@LaTeXML#1{%
  \HOLOGO@mbox{%
    \hologo{La}%
    \kern-.15em%
    T%
    \kern-.1667em%
    \lower.5ex\hbox{E}%
    \kern-.125em%
    \HoLogoFont@font{LaTeXML}{sc}{xml}%
  }%
}
%    \end{macrocode}
%    \end{macro}
%    \begin{macro}{\HoLogoHtml@pdfLaTeX}
%    \begin{macrocode}
\def\HoLogoHtml@LaTeXML#1{%
  \HOLOGO@Span{LaTeXML}{%
    \HoLogoCss@LaTeX
    \HoLogoCss@TeX
    \HOLOGO@Span{LaTeX}{%
      L%
      \HOLOGO@Span{a}{%
        A%
      }%
    }%
    \HOLOGO@Span{TeX}{%
      T%
      \HOLOGO@Span{e}{%
        E%
      }%
    }%
    \HCode{<span style="font-variant: small-caps;">}%
    xml%
    \HCode{</span>}%
  }%
}
%    \end{macrocode}
%    \end{macro}
%
% \subsubsection{\hologo{eTeX}}
%
%    \begin{macro}{\HoLogo@eTeX}
%    Source: package \xpackage{etex}
%    \begin{macrocode}
\def\HoLogo@eTeX#1{%
  \ltx@mbox{%
    \HOLOGO@MathSetup
    $\varepsilon$%
    -%
    \HOLOGO@NegativeKerning{-T,T-,To}%
    \hologo{TeX}%
  }%
}
%    \end{macrocode}
%    \end{macro}
%    \begin{macro}{\HoLogoCs@eTeX}
%    \begin{macrocode}
\ifnum64=`\^^^^0040\relax % test for big chars of LuaTeX/XeTeX
  \catcode`\$=9 %
  \catcode`\&=14 %
\else
  \catcode`\$=14 %
  \catcode`\&=9 %
\fi
\def\HoLogoCs@eTeX#1{%
$ #1{\string ^^^^0395}{\string ^^^^03b5}%
& #1{e}{E}%
  TeX%
}%
\catcode`\$=3 %
\catcode`\&=4 %
%    \end{macrocode}
%    \end{macro}
%    \begin{macro}{\HoLogoBkm@eTeX}
%    \begin{macrocode}
\def\HoLogoBkm@eTeX#1{%
  \HOLOGO@PdfdocUnicode{#1{e}{E}}{\textepsilon}%
  -%
  \hologo{TeX}%
}
%    \end{macrocode}
%    \end{macro}
%    \begin{macro}{\HoLogoHtml@eTeX}
%    \begin{macrocode}
\def\HoLogoHtml@eTeX#1{%
  \ltx@mbox{%
    \HOLOGO@MathSetup
    $\varepsilon$%
    -%
    \hologo{TeX}%
  }%
}
%    \end{macrocode}
%    \end{macro}
%
% \subsubsection{\hologo{iniTeX}}
%
%    \begin{macro}{\HoLogo@iniTeX}
%    \begin{macrocode}
\def\HoLogo@iniTeX#1{%
  \HOLOGO@mbox{%
    #1{i}{I}ni\hologo{TeX}%
  }%
}
%    \end{macrocode}
%    \end{macro}
%    \begin{macro}{\HoLogoCs@iniTeX}
%    \begin{macrocode}
\def\HoLogoCs@iniTeX#1{#1{i}{I}niTeX}
%    \end{macrocode}
%    \end{macro}
%    \begin{macro}{\HoLogoBkm@iniTeX}
%    \begin{macrocode}
\def\HoLogoBkm@iniTeX#1{%
  #1{i}{I}ni\hologo{TeX}%
}
%    \end{macrocode}
%    \end{macro}
%    \begin{macro}{\HoLogoHtml@iniTeX}
%    \begin{macrocode}
\let\HoLogoHtml@iniTeX\HoLogo@iniTeX
%    \end{macrocode}
%    \end{macro}
%
% \subsubsection{\hologo{virTeX}}
%
%    \begin{macro}{\HoLogo@virTeX}
%    \begin{macrocode}
\def\HoLogo@virTeX#1{%
  \HOLOGO@mbox{%
    #1{v}{V}ir\hologo{TeX}%
  }%
}
%    \end{macrocode}
%    \end{macro}
%    \begin{macro}{\HoLogoCs@virTeX}
%    \begin{macrocode}
\def\HoLogoCs@virTeX#1{#1{v}{V}irTeX}
%    \end{macrocode}
%    \end{macro}
%    \begin{macro}{\HoLogoBkm@virTeX}
%    \begin{macrocode}
\def\HoLogoBkm@virTeX#1{%
  #1{v}{V}ir\hologo{TeX}%
}
%    \end{macrocode}
%    \end{macro}
%    \begin{macro}{\HoLogoHtml@virTeX}
%    \begin{macrocode}
\let\HoLogoHtml@virTeX\HoLogo@virTeX
%    \end{macrocode}
%    \end{macro}
%
% \subsubsection{\hologo{SliTeX}}
%
% \paragraph{Definitions of the three variants.}
%
%    \begin{macro}{\HoLogo@SLiTeX@lift}
%    \begin{macrocode}
\def\HoLogo@SLiTeX@lift#1{%
  \HoLogoFont@font{SliTeX}{rm}{%
    S%
    \kern-.06em%
    L%
    \kern-.18em%
    \raise.32ex\hbox{\HoLogoFont@font{SliTeX}{sc}{i}}%
    \HOLOGO@discretionary
    \kern-.06em%
    \hologo{TeX}%
  }%
}
%    \end{macrocode}
%    \end{macro}
%    \begin{macro}{\HoLogoBkm@SLiTeX@lift}
%    \begin{macrocode}
\def\HoLogoBkm@SLiTeX@lift#1{SLiTeX}
%    \end{macrocode}
%    \end{macro}
%    \begin{macro}{\HoLogoHtml@SLiTeX@lift}
%    \begin{macrocode}
\def\HoLogoHtml@SLiTeX@lift#1{%
  \HoLogoCss@SLiTeX@lift
  \HOLOGO@Span{SLiTeX-lift}{%
    \HoLogoFont@font{SliTeX}{rm}{%
      S%
      \HOLOGO@Span{L}{L}%
      \HOLOGO@Span{i}{i}%
      \hologo{TeX}%
    }%
  }%
}
%    \end{macrocode}
%    \end{macro}
%    \begin{macro}{\HoLogoCss@SLiTeX@lift}
%    \begin{macrocode}
\def\HoLogoCss@SLiTeX@lift{%
  \Css{%
    span.HoLogo-SLiTeX-lift span.HoLogo-L{%
      margin-left:-.06em;%
      margin-right:-.18em;%
    }%
  }%
  \Css{%
    span.HoLogo-SLiTeX-lift span.HoLogo-i{%
      position:relative;%
      top:-.32ex;%
      margin-right:-.06em;%
      font-variant:small-caps;%
    }%
  }%
  \global\let\HoLogoCss@SLiTeX@lift\relax
}
%    \end{macrocode}
%    \end{macro}
%
%    \begin{macro}{\HoLogo@SliTeX@simple}
%    \begin{macrocode}
\def\HoLogo@SliTeX@simple#1{%
  \HoLogoFont@font{SliTeX}{rm}{%
    \ltx@mbox{%
      \HoLogoFont@font{SliTeX}{sc}{Sli}%
    }%
    \HOLOGO@discretionary
    \hologo{TeX}%
  }%
}
%    \end{macrocode}
%    \end{macro}
%    \begin{macro}{\HoLogoBkm@SliTeX@simple}
%    \begin{macrocode}
\def\HoLogoBkm@SliTeX@simple#1{SliTeX}
%    \end{macrocode}
%    \end{macro}
%    \begin{macro}{\HoLogoHtml@SliTeX@simple}
%    \begin{macrocode}
\let\HoLogoHtml@SliTeX@simple\HoLogo@SliTeX@simple
%    \end{macrocode}
%    \end{macro}
%
%    \begin{macro}{\HoLogo@SliTeX@narrow}
%    \begin{macrocode}
\def\HoLogo@SliTeX@narrow#1{%
  \HoLogoFont@font{SliTeX}{rm}{%
    \ltx@mbox{%
      S%
      \kern-.06em%
      \HoLogoFont@font{SliTeX}{sc}{%
        l%
        \kern-.035em%
        i%
      }%
    }%
    \HOLOGO@discretionary
    \kern-.06em%
    \hologo{TeX}%
  }%
}
%    \end{macrocode}
%    \end{macro}
%    \begin{macro}{\HoLogoBkm@SliTeX@narrow}
%    \begin{macrocode}
\def\HoLogoBkm@SliTeX@narrow#1{SliTeX}
%    \end{macrocode}
%    \end{macro}
%    \begin{macro}{\HoLogoHtml@SliTeX@narrow}
%    \begin{macrocode}
\def\HoLogoHtml@SliTeX@narrow#1{%
  \HoLogoCss@SliTeX@narrow
  \HOLOGO@Span{SliTeX-narrow}{%
    \HoLogoFont@font{SliTeX}{rm}{%
      S%
        \HOLOGO@Span{l}{l}%
        \HOLOGO@Span{i}{i}%
      \hologo{TeX}%
    }%
  }%
}
%    \end{macrocode}
%    \end{macro}
%    \begin{macro}{\HoLogoCss@SliTeX@narrow}
%    \begin{macrocode}
\def\HoLogoCss@SliTeX@narrow{%
  \Css{%
    span.HoLogo-SliTeX-narrow span.HoLogo-l{%
      margin-left:-.06em;%
      margin-right:-.035em;%
      font-variant:small-caps;%
    }%
  }%
  \Css{%
    span.HoLogo-SliTeX-narrow span.HoLogo-i{%
      margin-right:-.06em;%
      font-variant:small-caps;%
    }%
  }%
  \global\let\HoLogoCss@SliTeX@narrow\relax
}
%    \end{macrocode}
%    \end{macro}
%
% \paragraph{Macro set completion.}
%
%    \begin{macro}{\HoLogo@SLiTeX@simple}
%    \begin{macrocode}
\def\HoLogo@SLiTeX@simple{\HoLogo@SliTeX@simple}
%    \end{macrocode}
%    \end{macro}
%    \begin{macro}{\HoLogoBkm@SLiTeX@simple}
%    \begin{macrocode}
\def\HoLogoBkm@SLiTeX@simple{\HoLogoBkm@SliTeX@simple}
%    \end{macrocode}
%    \end{macro}
%    \begin{macro}{\HoLogoHtml@SLiTeX@simple}
%    \begin{macrocode}
\def\HoLogoHtml@SLiTeX@simple{\HoLogoHtml@SliTeX@simple}
%    \end{macrocode}
%    \end{macro}
%
%    \begin{macro}{\HoLogo@SLiTeX@narrow}
%    \begin{macrocode}
\def\HoLogo@SLiTeX@narrow{\HoLogo@SliTeX@narrow}
%    \end{macrocode}
%    \end{macro}
%    \begin{macro}{\HoLogoBkm@SLiTeX@narrow}
%    \begin{macrocode}
\def\HoLogoBkm@SLiTeX@narrow{\HoLogoBkm@SliTeX@narrow}
%    \end{macrocode}
%    \end{macro}
%    \begin{macro}{\HoLogoHtml@SLiTeX@narrow}
%    \begin{macrocode}
\def\HoLogoHtml@SLiTeX@narrow{\HoLogoHtml@SliTeX@narrow}
%    \end{macrocode}
%    \end{macro}
%
%    \begin{macro}{\HoLogo@SliTeX@lift}
%    \begin{macrocode}
\def\HoLogo@SliTeX@lift{\HoLogo@SLiTeX@lift}
%    \end{macrocode}
%    \end{macro}
%    \begin{macro}{\HoLogoBkm@SliTeX@lift}
%    \begin{macrocode}
\def\HoLogoBkm@SliTeX@lift{\HoLogoBkm@SLiTeX@lift}
%    \end{macrocode}
%    \end{macro}
%    \begin{macro}{\HoLogoHtml@SliTeX@lift}
%    \begin{macrocode}
\def\HoLogoHtml@SliTeX@lift{\HoLogoHtml@SLiTeX@lift}
%    \end{macrocode}
%    \end{macro}
%
% \paragraph{Defaults.}
%
%    \begin{macro}{\HoLogo@SLiTeX}
%    \begin{macrocode}
\def\HoLogo@SLiTeX{\HoLogo@SLiTeX@lift}
%    \end{macrocode}
%    \end{macro}
%    \begin{macro}{\HoLogoBkm@SLiTeX}
%    \begin{macrocode}
\def\HoLogoBkm@SLiTeX{\HoLogoBkm@SLiTeX@lift}
%    \end{macrocode}
%    \end{macro}
%    \begin{macro}{\HoLogoHtml@SLiTeX}
%    \begin{macrocode}
\def\HoLogoHtml@SLiTeX{\HoLogoHtml@SLiTeX@lift}
%    \end{macrocode}
%    \end{macro}
%
%    \begin{macro}{\HoLogo@SliTeX}
%    \begin{macrocode}
\def\HoLogo@SliTeX{\HoLogo@SliTeX@narrow}
%    \end{macrocode}
%    \end{macro}
%    \begin{macro}{\HoLogoBkm@SliTeX}
%    \begin{macrocode}
\def\HoLogoBkm@SliTeX{\HoLogoBkm@SliTeX@narrow}
%    \end{macrocode}
%    \end{macro}
%    \begin{macro}{\HoLogoHtml@SliTeX}
%    \begin{macrocode}
\def\HoLogoHtml@SliTeX{\HoLogoHtml@SliTeX@narrow}
%    \end{macrocode}
%    \end{macro}
%
% \subsubsection{\hologo{LuaTeX}}
%
%    \begin{macro}{\HoLogo@LuaTeX}
%    The kerning is an idea of Hans Hagen, see mailing list
%    `luatex at tug dot org' in March 2010.
%    \begin{macrocode}
\def\HoLogo@LuaTeX#1{%
  \HOLOGO@mbox{%
    Lua%
    \HOLOGO@NegativeKerning{aT,oT,To}%
    \hologo{TeX}%
  }%
}
%    \end{macrocode}
%    \end{macro}
%    \begin{macro}{\HoLogoHtml@LuaTeX}
%    \begin{macrocode}
\let\HoLogoHtml@LuaTeX\HoLogo@LuaTeX
%    \end{macrocode}
%    \end{macro}
%
% \subsubsection{\hologo{LuaLaTeX}}
%
%    \begin{macro}{\HoLogo@LuaLaTeX}
%    \begin{macrocode}
\def\HoLogo@LuaLaTeX#1{%
  \HOLOGO@mbox{%
    Lua%
    \hologo{LaTeX}%
  }%
}
%    \end{macrocode}
%    \end{macro}
%    \begin{macro}{\HoLogoHtml@LuaLaTeX}
%    \begin{macrocode}
\let\HoLogoHtml@LuaLaTeX\HoLogo@LuaLaTeX
%    \end{macrocode}
%    \end{macro}
%
% \subsubsection{\hologo{XeTeX}, \hologo{XeLaTeX}}
%
%    \begin{macro}{\HOLOGO@IfCharExists}
%    \begin{macrocode}
\ifluatex
  \ifnum\luatexversion<36 %
  \else
    \def\HOLOGO@IfCharExists#1{%
      \ifnum
        \directlua{%
           if luaotfload and luaotfload.aux then
             if luaotfload.aux.font_has_glyph(%
                    font.current(), \number#1) then % 	 
	       tex.print("1") % 	 
	     end % 	 
	   elseif font and font.fonts and font.current then %
            local f = font.fonts[font.current()]%
            if f.characters and f.characters[\number#1] then %
              tex.print("1")%
            end %
          end%
        }0=\ltx@zero
        \expandafter\ltx@secondoftwo
      \else
        \expandafter\ltx@firstoftwo
      \fi
    }%
  \fi
\fi
\ltx@IfUndefined{HOLOGO@IfCharExists}{%
  \def\HOLOGO@@IfCharExists#1{%
    \begingroup
      \tracinglostchars=\ltx@zero
      \setbox\ltx@zero=\hbox{%
        \kern7sp\char#1\relax
        \ifnum\lastkern>\ltx@zero
          \expandafter\aftergroup\csname iffalse\endcsname
        \else
          \expandafter\aftergroup\csname iftrue\endcsname
        \fi
      }%
      % \if{true|false} from \aftergroup
      \endgroup
      \expandafter\ltx@firstoftwo
    \else
      \endgroup
      \expandafter\ltx@secondoftwo
    \fi
  }%
  \ifxetex
    \ltx@IfUndefined{XeTeXfonttype}{}{%
      \ltx@IfUndefined{XeTeXcharglyph}{}{%
        \def\HOLOGO@IfCharExists#1{%
          \ifnum\XeTeXfonttype\font>\ltx@zero
            \expandafter\ltx@firstofthree
          \else
            \expandafter\ltx@gobble
          \fi
          {%
            \ifnum\XeTeXcharglyph#1>\ltx@zero
              \expandafter\ltx@firstoftwo
            \else
              \expandafter\ltx@secondoftwo
            \fi
          }%
          \HOLOGO@@IfCharExists{#1}%
        }%
      }%
    }%
  \fi
}{}
\ltx@ifundefined{HOLOGO@IfCharExists}{%
  \ifnum64=`\^^^^0040\relax % test for big chars of LuaTeX/XeTeX
    \let\HOLOGO@IfCharExists\HOLOGO@@IfCharExists
  \else
    \def\HOLOGO@IfCharExists#1{%
      \ifnum#1>255 %
        \expandafter\ltx@fourthoffour
      \fi
      \HOLOGO@@IfCharExists{#1}%
    }%
  \fi
}{}
%    \end{macrocode}
%    \end{macro}
%
%    \begin{macro}{\HoLogo@Xe}
%    Source: package \xpackage{dtklogos}
%    \begin{macrocode}
\def\HoLogo@Xe#1{%
  X%
  \kern-.1em\relax
  \HOLOGO@IfCharExists{"018E}{%
    \lower.5ex\hbox{\char"018E}%
  }{%
    \chardef\HOLOGO@choice=\ltx@zero
    \ifdim\fontdimen\ltx@one\font>0pt %
      \ltx@IfUndefined{rotatebox}{%
        \ltx@IfUndefined{pgftext}{%
          \ltx@IfUndefined{psscalebox}{%
            \ltx@IfUndefined{HOLOGO@ScaleBox@\hologoDriver}{%
            }{%
              \chardef\HOLOGO@choice=4 %
            }%
          }{%
            \chardef\HOLOGO@choice=3 %
          }%
        }{%
          \chardef\HOLOGO@choice=2 %
        }%
      }{%
        \chardef\HOLOGO@choice=1 %
      }%
      \ifcase\HOLOGO@choice
        \HOLOGO@WarningUnsupportedDriver{Xe}%
        e%
      \or % 1: \rotatebox
        \begingroup
          \setbox\ltx@zero\hbox{\rotatebox{180}{E}}%
          \ltx@LocDimenA=\dp\ltx@zero
          \advance\ltx@LocDimenA by -.5ex\relax
          \raise\ltx@LocDimenA\box\ltx@zero
        \endgroup
      \or % 2: \pgftext
        \lower.5ex\hbox{%
          \pgfpicture
            \pgftext[rotate=180]{E}%
          \endpgfpicture
        }%
      \or % 3: \psscalebox
        \begingroup
          \setbox\ltx@zero\hbox{\psscalebox{-1 -1}{E}}%
          \ltx@LocDimenA=\dp\ltx@zero
          \advance\ltx@LocDimenA by -.5ex\relax
          \raise\ltx@LocDimenA\box\ltx@zero
        \endgroup
      \or % 4: \HOLOGO@PointReflectBox
        \lower.5ex\hbox{\HOLOGO@PointReflectBox{E}}%
      \else
        \@PackageError{hologo}{Internal error (choice/it}\@ehc
      \fi
    \else
      \ltx@IfUndefined{reflectbox}{%
        \ltx@IfUndefined{pgftext}{%
          \ltx@IfUndefined{psscalebox}{%
            \ltx@IfUndefined{HOLOGO@ScaleBox@\hologoDriver}{%
            }{%
              \chardef\HOLOGO@choice=4 %
            }%
          }{%
            \chardef\HOLOGO@choice=3 %
          }%
        }{%
          \chardef\HOLOGO@choice=2 %
        }%
      }{%
        \chardef\HOLOGO@choice=1 %
      }%
      \ifcase\HOLOGO@choice
        \HOLOGO@WarningUnsupportedDriver{Xe}%
        e%
      \or % 1: reflectbox
        \lower.5ex\hbox{%
          \reflectbox{E}%
        }%
      \or % 2: \pgftext
        \lower.5ex\hbox{%
          \pgfpicture
            \pgftransformxscale{-1}%
            \pgftext{E}%
          \endpgfpicture
        }%
      \or % 3: \psscalebox
        \lower.5ex\hbox{%
          \psscalebox{-1 1}{E}%
        }%
      \or % 4: \HOLOGO@Reflectbox
        \lower.5ex\hbox{%
          \HOLOGO@ReflectBox{E}%
        }%
      \else
        \@PackageError{hologo}{Internal error (choice/up)}\@ehc
      \fi
    \fi
  }%
}
%    \end{macrocode}
%    \end{macro}
%    \begin{macro}{\HoLogoHtml@Xe}
%    \begin{macrocode}
\def\HoLogoHtml@Xe#1{%
  \HoLogoCss@Xe
  \HOLOGO@Span{Xe}{%
    X%
    \HOLOGO@Span{e}{%
      \HCode{&\ltx@hashchar x018e;}%
    }%
  }%
}
%    \end{macrocode}
%    \end{macro}
%    \begin{macro}{\HoLogoCss@Xe}
%    \begin{macrocode}
\def\HoLogoCss@Xe{%
  \Css{%
    span.HoLogo-Xe span.HoLogo-e{%
      position:relative;%
      top:.5ex;%
      left-margin:-.1em;%
    }%
  }%
  \global\let\HoLogoCss@Xe\relax
}
%    \end{macrocode}
%    \end{macro}
%
%    \begin{macro}{\HoLogo@XeTeX}
%    \begin{macrocode}
\def\HoLogo@XeTeX#1{%
  \hologo{Xe}%
  \kern-.15em\relax
  \hologo{TeX}%
}
%    \end{macrocode}
%    \end{macro}
%
%    \begin{macro}{\HoLogoHtml@XeTeX}
%    \begin{macrocode}
\def\HoLogoHtml@XeTeX#1{%
  \HoLogoCss@XeTeX
  \HOLOGO@Span{XeTeX}{%
    \hologo{Xe}%
    \hologo{TeX}%
  }%
}
%    \end{macrocode}
%    \end{macro}
%    \begin{macro}{\HoLogoCss@XeTeX}
%    \begin{macrocode}
\def\HoLogoCss@XeTeX{%
  \Css{%
    span.HoLogo-XeTeX span.HoLogo-TeX{%
      margin-left:-.15em;%
    }%
  }%
  \global\let\HoLogoCss@XeTeX\relax
}
%    \end{macrocode}
%    \end{macro}
%
%    \begin{macro}{\HoLogo@XeLaTeX}
%    \begin{macrocode}
\def\HoLogo@XeLaTeX#1{%
  \hologo{Xe}%
  \kern-.13em%
  \hologo{LaTeX}%
}
%    \end{macrocode}
%    \end{macro}
%    \begin{macro}{\HoLogoHtml@XeLaTeX}
%    \begin{macrocode}
\def\HoLogoHtml@XeLaTeX#1{%
  \HoLogoCss@XeLaTeX
  \HOLOGO@Span{XeLaTeX}{%
    \hologo{Xe}%
    \hologo{LaTeX}%
  }%
}
%    \end{macrocode}
%    \end{macro}
%    \begin{macro}{\HoLogoCss@XeLaTeX}
%    \begin{macrocode}
\def\HoLogoCss@XeLaTeX{%
  \Css{%
    span.HoLogo-XeLaTeX span.HoLogo-Xe{%
      margin-right:-.13em;%
    }%
  }%
  \global\let\HoLogoCss@XeLaTeX\relax
}
%    \end{macrocode}
%    \end{macro}
%
% \subsubsection{\hologo{pdfTeX}, \hologo{pdfLaTeX}}
%
%    \begin{macro}{\HoLogo@pdfTeX}
%    \begin{macrocode}
\def\HoLogo@pdfTeX#1{%
  \HOLOGO@mbox{%
    #1{p}{P}df\hologo{TeX}%
  }%
}
%    \end{macrocode}
%    \end{macro}
%    \begin{macro}{\HoLogoCs@pdfTeX}
%    \begin{macrocode}
\def\HoLogoCs@pdfTeX#1{#1{p}{P}dfTeX}
%    \end{macrocode}
%    \end{macro}
%    \begin{macro}{\HoLogoBkm@pdfTeX}
%    \begin{macrocode}
\def\HoLogoBkm@pdfTeX#1{%
  #1{p}{P}df\hologo{TeX}%
}
%    \end{macrocode}
%    \end{macro}
%    \begin{macro}{\HoLogoHtml@pdfTeX}
%    \begin{macrocode}
\let\HoLogoHtml@pdfTeX\HoLogo@pdfTeX
%    \end{macrocode}
%    \end{macro}
%
%    \begin{macro}{\HoLogo@pdfLaTeX}
%    \begin{macrocode}
\def\HoLogo@pdfLaTeX#1{%
  \HOLOGO@mbox{%
    #1{p}{P}df\hologo{LaTeX}%
  }%
}
%    \end{macrocode}
%    \end{macro}
%    \begin{macro}{\HoLogoCs@pdfLaTeX}
%    \begin{macrocode}
\def\HoLogoCs@pdfLaTeX#1{#1{p}{P}dfLaTeX}
%    \end{macrocode}
%    \end{macro}
%    \begin{macro}{\HoLogoBkm@pdfLaTeX}
%    \begin{macrocode}
\def\HoLogoBkm@pdfLaTeX#1{%
  #1{p}{P}df\hologo{LaTeX}%
}
%    \end{macrocode}
%    \end{macro}
%    \begin{macro}{\HoLogoHtml@pdfLaTeX}
%    \begin{macrocode}
\let\HoLogoHtml@pdfLaTeX\HoLogo@pdfLaTeX
%    \end{macrocode}
%    \end{macro}
%
% \subsubsection{\hologo{VTeX}}
%
%    \begin{macro}{\HoLogo@VTeX}
%    \begin{macrocode}
\def\HoLogo@VTeX#1{%
  \HOLOGO@mbox{%
    V\hologo{TeX}%
  }%
}
%    \end{macrocode}
%    \end{macro}
%    \begin{macro}{\HoLogoHtml@VTeX}
%    \begin{macrocode}
\let\HoLogoHtml@VTeX\HoLogo@VTeX
%    \end{macrocode}
%    \end{macro}
%
% \subsubsection{\hologo{AmS}, \dots}
%
%    Source: class \xclass{amsdtx}
%
%    \begin{macro}{\HoLogo@AmS}
%    \begin{macrocode}
\def\HoLogo@AmS#1{%
  \HoLogoFont@font{AmS}{sy}{%
    A%
    \kern-.1667em%
    \lower.5ex\hbox{M}%
    \kern-.125em%
    S%
  }%
}
%    \end{macrocode}
%    \end{macro}
%    \begin{macro}{\HoLogoBkm@AmS}
%    \begin{macrocode}
\def\HoLogoBkm@AmS#1{AmS}
%    \end{macrocode}
%    \end{macro}
%    \begin{macro}{\HoLogoHtml@AmS}
%    \begin{macrocode}
\def\HoLogoHtml@AmS#1{%
  \HoLogoCss@AmS
%  \HoLogoFont@font{AmS}{sy}{%
    \HOLOGO@Span{AmS}{%
      A%
      \HOLOGO@Span{M}{M}%
      S%
    }%
%   }%
}
%    \end{macrocode}
%    \end{macro}
%    \begin{macro}{\HoLogoCss@AmS}
%    \begin{macrocode}
\def\HoLogoCss@AmS{%
  \Css{%
    span.HoLogo-AmS span.HoLogo-M{%
      position:relative;%
      top:.5ex;%
      margin-left:-.1667em;%
      margin-right:-.125em;%
      text-decoration:none;%
    }%
  }%
  \global\let\HoLogoCss@AmS\relax
}
%    \end{macrocode}
%    \end{macro}
%
%    \begin{macro}{\HoLogo@AmSTeX}
%    \begin{macrocode}
\def\HoLogo@AmSTeX#1{%
  \hologo{AmS}%
  \HOLOGO@hyphen
  \hologo{TeX}%
}
%    \end{macrocode}
%    \end{macro}
%    \begin{macro}{\HoLogoBkm@AmSTeX}
%    \begin{macrocode}
\def\HoLogoBkm@AmSTeX#1{AmS-TeX}%
%    \end{macrocode}
%    \end{macro}
%    \begin{macro}{\HoLogoHtml@AmSTeX}
%    \begin{macrocode}
\let\HoLogoHtml@AmSTeX\HoLogo@AmSTeX
%    \end{macrocode}
%    \end{macro}
%
%    \begin{macro}{\HoLogo@AmSLaTeX}
%    \begin{macrocode}
\def\HoLogo@AmSLaTeX#1{%
  \hologo{AmS}%
  \HOLOGO@hyphen
  \hologo{LaTeX}%
}
%    \end{macrocode}
%    \end{macro}
%    \begin{macro}{\HoLogoBkm@AmSLaTeX}
%    \begin{macrocode}
\def\HoLogoBkm@AmSLaTeX#1{AmS-LaTeX}%
%    \end{macrocode}
%    \end{macro}
%    \begin{macro}{\HoLogoHtml@AmSLaTeX}
%    \begin{macrocode}
\let\HoLogoHtml@AmSLaTeX\HoLogo@AmSLaTeX
%    \end{macrocode}
%    \end{macro}
%
% \subsubsection{\hologo{BibTeX}}
%
%    \begin{macro}{\HoLogo@BibTeX@sc}
%    A definition of \hologo{BibTeX} is provided in
%    the documentation source for the manual of \hologo{BibTeX}
%    \cite{btxdoc}.
%\begin{quote}
%\begin{verbatim}
%\def\BibTeX{%
%  {%
%    \rm
%    B%
%    \kern-.05em%
%    {%
%      \sc
%      i%
%      \kern-.025em %
%      b%
%    }%
%    \kern-.08em
%    T%
%    \kern-.1667em%
%    \lower.7ex\hbox{E}%
%    \kern-.125em%
%    X%
%  }%
%}
%\end{verbatim}
%\end{quote}
%    \begin{macrocode}
\def\HoLogo@BibTeX@sc#1{%
  B%
  \kern-.05em%
  \HoLogoFont@font{BibTeX}{sc}{%
    i%
    \kern-.025em%
    b%
  }%
  \HOLOGO@discretionary
  \kern-.08em%
  \hologo{TeX}%
}
%    \end{macrocode}
%    \end{macro}
%    \begin{macro}{\HoLogoHtml@BibTeX@sc}
%    \begin{macrocode}
\def\HoLogoHtml@BibTeX@sc#1{%
  \HoLogoCss@BibTeX@sc
  \HOLOGO@Span{BibTeX-sc}{%
    B%
    \HOLOGO@Span{i}{i}%
    \HOLOGO@Span{b}{b}%
    \hologo{TeX}%
  }%
}
%    \end{macrocode}
%    \end{macro}
%    \begin{macro}{\HoLogoCss@BibTeX@sc}
%    \begin{macrocode}
\def\HoLogoCss@BibTeX@sc{%
  \Css{%
    span.HoLogo-BibTeX-sc span.HoLogo-i{%
      margin-left:-.05em;%
      margin-right:-.025em;%
      font-variant:small-caps;%
    }%
  }%
  \Css{%
    span.HoLogo-BibTeX-sc span.HoLogo-b{%
      margin-right:-.08em;%
      font-variant:small-caps;%
    }%
  }%
  \global\let\HoLogoCss@BibTeX@sc\relax
}
%    \end{macrocode}
%    \end{macro}
%
%    \begin{macro}{\HoLogo@BibTeX@sf}
%    Variant \xoption{sf} avoids trouble with unavailable
%    small caps fonts (e.g., bold versions of Computer Modern or
%    Latin Modern). The definition is taken from
%    package \xpackage{dtklogos} \cite{dtklogos}.
%\begin{quote}
%\begin{verbatim}
%\DeclareRobustCommand{\BibTeX}{%
%  B%
%  \kern-.05em%
%  \hbox{%
%    $\m@th$% %% force math size calculations
%    \csname S@\f@size\endcsname
%    \fontsize\sf@size\z@
%    \math@fontsfalse
%    \selectfont
%    I%
%    \kern-.025em%
%    B
%  }%
%  \kern-.08em%
%  \-%
%  \TeX
%}
%\end{verbatim}
%\end{quote}
%    \begin{macrocode}
\def\HoLogo@BibTeX@sf#1{%
  B%
  \kern-.05em%
  \HoLogoFont@font{BibTeX}{bibsf}{%
    I%
    \kern-.025em%
    B%
  }%
  \HOLOGO@discretionary
  \kern-.08em%
  \hologo{TeX}%
}
%    \end{macrocode}
%    \end{macro}
%    \begin{macro}{\HoLogoHtml@BibTeX@sf}
%    \begin{macrocode}
\def\HoLogoHtml@BibTeX@sf#1{%
  \HoLogoCss@BibTeX@sf
  \HOLOGO@Span{BibTeX-sf}{%
    B%
    \HoLogoFont@font{BibTeX}{bibsf}{%
      \HOLOGO@Span{i}{I}%
      B%
    }%
    \hologo{TeX}%
  }%
}
%    \end{macrocode}
%    \end{macro}
%    \begin{macro}{\HoLogoCss@BibTeX@sf}
%    \begin{macrocode}
\def\HoLogoCss@BibTeX@sf{%
  \Css{%
    span.HoLogo-BibTeX-sf span.HoLogo-i{%
      margin-left:-.05em;%
      margin-right:-.025em;%
    }%
  }%
  \Css{%
    span.HoLogo-BibTeX-sf span.HoLogo-TeX{%
      margin-left:-.08em;%
    }%
  }%
  \global\let\HoLogoCss@BibTeX@sf\relax
}
%    \end{macrocode}
%    \end{macro}
%
%    \begin{macro}{\HoLogo@BibTeX}
%    \begin{macrocode}
\def\HoLogo@BibTeX{\HoLogo@BibTeX@sf}
%    \end{macrocode}
%    \end{macro}
%    \begin{macro}{\HoLogoHtml@BibTeX}
%    \begin{macrocode}
\def\HoLogoHtml@BibTeX{\HoLogoHtml@BibTeX@sf}
%    \end{macrocode}
%    \end{macro}
%
% \subsubsection{\hologo{BibTeX8}}
%
%    \begin{macro}{\HoLogo@BibTeX8}
%    \begin{macrocode}
\expandafter\def\csname HoLogo@BibTeX8\endcsname#1{%
  \hologo{BibTeX}%
  8%
}
%    \end{macrocode}
%    \end{macro}
%
%    \begin{macro}{\HoLogoBkm@BibTeX8}
%    \begin{macrocode}
\expandafter\def\csname HoLogoBkm@BibTeX8\endcsname#1{%
  \hologo{BibTeX}%
  8%
}
%    \end{macrocode}
%    \end{macro}
%    \begin{macro}{\HoLogoHtml@BibTeX8}
%    \begin{macrocode}
\expandafter
\let\csname HoLogoHtml@BibTeX8\expandafter\endcsname
\csname HoLogo@BibTeX8\endcsname
%    \end{macrocode}
%    \end{macro}
%
% \subsubsection{\hologo{ConTeXt}}
%
%    \begin{macro}{\HoLogo@ConTeXt@simple}
%    \begin{macrocode}
\def\HoLogo@ConTeXt@simple#1{%
  \HOLOGO@mbox{Con}%
  \HOLOGO@discretionary
  \HOLOGO@mbox{\hologo{TeX}t}%
}
%    \end{macrocode}
%    \end{macro}
%    \begin{macro}{\HoLogoHtml@ConTeXt@simple}
%    \begin{macrocode}
\let\HoLogoHtml@ConTeXt@simple\HoLogo@ConTeXt@simple
%    \end{macrocode}
%    \end{macro}
%
%    \begin{macro}{\HoLogo@ConTeXt@narrow}
%    This definition of logo \hologo{ConTeXt} with variant \xoption{narrow}
%    comes from TUGboat's class \xclass{ltugboat} (version 2010/11/15 v2.8).
%    \begin{macrocode}
\def\HoLogo@ConTeXt@narrow#1{%
  \HOLOGO@mbox{C\kern-.0333emon}%
  \HOLOGO@discretionary
  \kern-.0667em%
  \HOLOGO@mbox{\hologo{TeX}\kern-.0333emt}%
}
%    \end{macrocode}
%    \end{macro}
%    \begin{macro}{\HoLogoHtml@ConTeXt@narrow}
%    \begin{macrocode}
\def\HoLogoHtml@ConTeXt@narrow#1{%
  \HoLogoCss@ConTeXt@narrow
  \HOLOGO@Span{ConTeXt-narrow}{%
    \HOLOGO@Span{C}{C}%
    on%
    \hologo{TeX}%
    t%
  }%
}
%    \end{macrocode}
%    \end{macro}
%    \begin{macro}{\HoLogoCss@ConTeXt@narrow}
%    \begin{macrocode}
\def\HoLogoCss@ConTeXt@narrow{%
  \Css{%
    span.HoLogo-ConTeXt-narrow span.HoLogo-C{%
      margin-left:-.0333em;%
    }%
  }%
  \Css{%
    span.HoLogo-ConTeXt-narrow span.HoLogo-TeX{%
      margin-left:-.0667em;%
      margin-right:-.0333em;%
    }%
  }%
  \global\let\HoLogoCss@ConTeXt@narrow\relax
}
%    \end{macrocode}
%    \end{macro}
%
%    \begin{macro}{\HoLogo@ConTeXt}
%    \begin{macrocode}
\def\HoLogo@ConTeXt{\HoLogo@ConTeXt@narrow}
%    \end{macrocode}
%    \end{macro}
%    \begin{macro}{\HoLogoHtml@ConTeXt}
%    \begin{macrocode}
\def\HoLogoHtml@ConTeXt{\HoLogoHtml@ConTeXt@narrow}
%    \end{macrocode}
%    \end{macro}
%
% \subsubsection{\hologo{emTeX}}
%
%    \begin{macro}{\HoLogo@emTeX}
%    \begin{macrocode}
\def\HoLogo@emTeX#1{%
  \HOLOGO@mbox{#1{e}{E}m}%
  \HOLOGO@discretionary
  \hologo{TeX}%
}
%    \end{macrocode}
%    \end{macro}
%    \begin{macro}{\HoLogoCs@emTeX}
%    \begin{macrocode}
\def\HoLogoCs@emTeX#1{#1{e}{E}mTeX}%
%    \end{macrocode}
%    \end{macro}
%    \begin{macro}{\HoLogoBkm@emTeX}
%    \begin{macrocode}
\def\HoLogoBkm@emTeX#1{%
  #1{e}{E}m\hologo{TeX}%
}
%    \end{macrocode}
%    \end{macro}
%    \begin{macro}{\HoLogoHtml@emTeX}
%    \begin{macrocode}
\let\HoLogoHtml@emTeX\HoLogo@emTeX
%    \end{macrocode}
%    \end{macro}
%
% \subsubsection{\hologo{ExTeX}}
%
%    \begin{macro}{\HoLogo@ExTeX}
%    The definition is taken from the FAQ of the
%    project \hologo{ExTeX}
%    \cite{ExTeX-FAQ}.
%\begin{quote}
%\begin{verbatim}
%\def\ExTeX{%
%  \textrm{% Logo always with serifs
%    \ensuremath{%
%      \textstyle
%      \varepsilon_{%
%        \kern-0.15em%
%        \mathcal{X}%
%      }%
%    }%
%    \kern-.15em%
%    \TeX
%  }%
%}
%\end{verbatim}
%\end{quote}
%    \begin{macrocode}
\def\HoLogo@ExTeX#1{%
  \HoLogoFont@font{ExTeX}{rm}{%
    \ltx@mbox{%
      \HOLOGO@MathSetup
      $%
        \textstyle
        \varepsilon_{%
          \kern-0.15em%
          \HoLogoFont@font{ExTeX}{sy}{X}%
        }%
      $%
    }%
    \HOLOGO@discretionary
    \kern-.15em%
    \hologo{TeX}%
  }%
}
%    \end{macrocode}
%    \end{macro}
%    \begin{macro}{\HoLogoHtml@ExTeX}
%    \begin{macrocode}
\def\HoLogoHtml@ExTeX#1{%
  \HoLogoCss@ExTeX
  \HoLogoFont@font{ExTeX}{rm}{%
    \HOLOGO@Span{ExTeX}{%
      \ltx@mbox{%
        \HOLOGO@MathSetup
        $\textstyle\varepsilon$%
        \HOLOGO@Span{X}{$\textstyle\chi$}%
        \hologo{TeX}%
      }%
    }%
  }%
}
%    \end{macrocode}
%    \end{macro}
%    \begin{macro}{\HoLogoBkm@ExTeX}
%    \begin{macrocode}
\def\HoLogoBkm@ExTeX#1{%
  \HOLOGO@PdfdocUnicode{#1{e}{E}x}{\textepsilon\textchi}%
  \hologo{TeX}%
}
%    \end{macrocode}
%    \end{macro}
%    \begin{macro}{\HoLogoCss@ExTeX}
%    \begin{macrocode}
\def\HoLogoCss@ExTeX{%
  \Css{%
    span.HoLogo-ExTeX{%
      font-family:serif;%
    }%
  }%
  \Css{%
    span.HoLogo-ExTeX span.HoLogo-TeX{%
      margin-left:-.15em;%
    }%
  }%
  \global\let\HoLogoCss@ExTeX\relax
}
%    \end{macrocode}
%    \end{macro}
%
% \subsubsection{\hologo{MiKTeX}}
%
%    \begin{macro}{\HoLogo@MiKTeX}
%    \begin{macrocode}
\def\HoLogo@MiKTeX#1{%
  \HOLOGO@mbox{MiK}%
  \HOLOGO@discretionary
  \hologo{TeX}%
}
%    \end{macrocode}
%    \end{macro}
%    \begin{macro}{\HoLogoHtml@MiKTeX}
%    \begin{macrocode}
\let\HoLogoHtml@MiKTeX\HoLogo@MiKTeX
%    \end{macrocode}
%    \end{macro}
%
% \subsubsection{\hologo{OzTeX} and friends}
%
%    Source: \hologo{OzTeX} FAQ \cite{OzTeX}:
%    \begin{quote}
%      |\def\OzTeX{O\kern-.03em z\kern-.15em\TeX}|\\
%      (There is no kerning in OzMF, OzMP and OzTtH.)
%    \end{quote}
%
%    \begin{macro}{\HoLogo@OzTeX}
%    \begin{macrocode}
\def\HoLogo@OzTeX#1{%
  O%
  \kern-.03em %
  z%
  \kern-.15em %
  \hologo{TeX}%
}
%    \end{macrocode}
%    \end{macro}
%    \begin{macro}{\HoLogoHtml@OzTeX}
%    \begin{macrocode}
\def\HoLogoHtml@OzTeX#1{%
  \HoLogoCss@OzTeX
  \HOLOGO@Span{OzTeX}{%
    O%
    \HOLOGO@Span{z}{z}%
    \hologo{TeX}%
  }%
}
%    \end{macrocode}
%    \end{macro}
%    \begin{macro}{\HoLogoCss@OzTeX}
%    \begin{macrocode}
\def\HoLogoCss@OzTeX{%
  \Css{%
    span.HoLogo-OzTeX span.HoLogo-z{%
      margin-left:-.03em;%
      margin-right:-.15em;%
    }%
  }%
  \global\let\HoLogoCss@OzTeX\relax
}
%    \end{macrocode}
%    \end{macro}
%
%    \begin{macro}{\HoLogo@OzMF}
%    \begin{macrocode}
\def\HoLogo@OzMF#1{%
  \HOLOGO@mbox{OzMF}%
}
%    \end{macrocode}
%    \end{macro}
%    \begin{macro}{\HoLogo@OzMP}
%    \begin{macrocode}
\def\HoLogo@OzMP#1{%
  \HOLOGO@mbox{OzMP}%
}
%    \end{macrocode}
%    \end{macro}
%    \begin{macro}{\HoLogo@OzTtH}
%    \begin{macrocode}
\def\HoLogo@OzTtH#1{%
  \HOLOGO@mbox{OzTtH}%
}
%    \end{macrocode}
%    \end{macro}
%
% \subsubsection{\hologo{PCTeX}}
%
%    \begin{macro}{\HoLogo@PCTeX}
%    \begin{macrocode}
\def\HoLogo@PCTeX#1{%
  \HOLOGO@mbox{PC}%
  \hologo{TeX}%
}
%    \end{macrocode}
%    \end{macro}
%    \begin{macro}{\HoLogoHtml@PCTeX}
%    \begin{macrocode}
\let\HoLogoHtml@PCTeX\HoLogo@PCTeX
%    \end{macrocode}
%    \end{macro}
%
% \subsubsection{\hologo{PiCTeX}}
%
%    The original definitions from \xfile{pictex.tex} \cite{PiCTeX}:
%\begin{quote}
%\begin{verbatim}
%\def\PiC{%
%  P%
%  \kern-.12em%
%  \lower.5ex\hbox{I}%
%  \kern-.075em%
%  C%
%}
%\def\PiCTeX{%
%  \PiC
%  \kern-.11em%
%  \TeX
%}
%\end{verbatim}
%\end{quote}
%
%    \begin{macro}{\HoLogo@PiC}
%    \begin{macrocode}
\def\HoLogo@PiC#1{%
  P%
  \kern-.12em%
  \lower.5ex\hbox{I}%
  \kern-.075em%
  C%
  \HOLOGO@SpaceFactor
}
%    \end{macrocode}
%    \end{macro}
%    \begin{macro}{\HoLogoHtml@PiC}
%    \begin{macrocode}
\def\HoLogoHtml@PiC#1{%
  \HoLogoCss@PiC
  \HOLOGO@Span{PiC}{%
    P%
    \HOLOGO@Span{i}{I}%
    C%
  }%
}
%    \end{macrocode}
%    \end{macro}
%    \begin{macro}{\HoLogoCss@PiC}
%    \begin{macrocode}
\def\HoLogoCss@PiC{%
  \Css{%
    span.HoLogo-PiC span.HoLogo-i{%
      position:relative;%
      top:.5ex;%
      margin-left:-.12em;%
      margin-right:-.075em;%
      text-decoration:none;%
    }%
  }%
  \global\let\HoLogoCss@PiC\relax
}
%    \end{macrocode}
%    \end{macro}
%
%    \begin{macro}{\HoLogo@PiCTeX}
%    \begin{macrocode}
\def\HoLogo@PiCTeX#1{%
  \hologo{PiC}%
  \HOLOGO@discretionary
  \kern-.11em%
  \hologo{TeX}%
}
%    \end{macrocode}
%    \end{macro}
%    \begin{macro}{\HoLogoHtml@PiCTeX}
%    \begin{macrocode}
\def\HoLogoHtml@PiCTeX#1{%
  \HoLogoCss@PiCTeX
  \HOLOGO@Span{PiCTeX}{%
    \hologo{PiC}%
    \hologo{TeX}%
  }%
}
%    \end{macrocode}
%    \end{macro}
%    \begin{macro}{\HoLogoCss@PiCTeX}
%    \begin{macrocode}
\def\HoLogoCss@PiCTeX{%
  \Css{%
    span.HoLogo-PiCTeX span.HoLogo-PiC{%
      margin-right:-.11em;%
    }%
  }%
  \global\let\HoLogoCss@PiCTeX\relax
}
%    \end{macrocode}
%    \end{macro}
%
% \subsubsection{\hologo{teTeX}}
%
%    \begin{macro}{\HoLogo@teTeX}
%    \begin{macrocode}
\def\HoLogo@teTeX#1{%
  \HOLOGO@mbox{#1{t}{T}e}%
  \HOLOGO@discretionary
  \hologo{TeX}%
}
%    \end{macrocode}
%    \end{macro}
%    \begin{macro}{\HoLogoCs@teTeX}
%    \begin{macrocode}
\def\HoLogoCs@teTeX#1{#1{t}{T}dfTeX}
%    \end{macrocode}
%    \end{macro}
%    \begin{macro}{\HoLogoBkm@teTeX}
%    \begin{macrocode}
\def\HoLogoBkm@teTeX#1{%
  #1{t}{T}e\hologo{TeX}%
}
%    \end{macrocode}
%    \end{macro}
%    \begin{macro}{\HoLogoHtml@teTeX}
%    \begin{macrocode}
\let\HoLogoHtml@teTeX\HoLogo@teTeX
%    \end{macrocode}
%    \end{macro}
%
% \subsubsection{\hologo{TeX4ht}}
%
%    \begin{macro}{\HoLogo@TeX4ht}
%    \begin{macrocode}
\expandafter\def\csname HoLogo@TeX4ht\endcsname#1{%
  \HOLOGO@mbox{\hologo{TeX}4ht}%
}
%    \end{macrocode}
%    \end{macro}
%    \begin{macro}{\HoLogoHtml@TeX4ht}
%    \begin{macrocode}
\expandafter
\let\csname HoLogoHtml@TeX4ht\expandafter\endcsname
\csname HoLogo@TeX4ht\endcsname
%    \end{macrocode}
%    \end{macro}
%
%
% \subsubsection{\hologo{SageTeX}}
%
%    \begin{macro}{\HoLogo@SageTeX}
%    \begin{macrocode}
\def\HoLogo@SageTeX#1{%
  \HOLOGO@mbox{Sage}%
  \HOLOGO@discretionary
  \HOLOGO@NegativeKerning{eT,oT,To}%
  \hologo{TeX}%
}
%    \end{macrocode}
%    \end{macro}
%    \begin{macro}{\HoLogoHtml@SageTeX}
%    \begin{macrocode}
\let\HoLogoHtml@SageTeX\HoLogo@SageTeX
%    \end{macrocode}
%    \end{macro}
%
% \subsection{\hologo{METAFONT} and friends}
%
%    \begin{macro}{\HoLogo@METAFONT}
%    \begin{macrocode}
\def\HoLogo@METAFONT#1{%
  \HoLogoFont@font{METAFONT}{logo}{%
    \HOLOGO@mbox{META}%
    \HOLOGO@discretionary
    \HOLOGO@mbox{FONT}%
  }%
}
%    \end{macrocode}
%    \end{macro}
%
%    \begin{macro}{\HoLogo@METAPOST}
%    \begin{macrocode}
\def\HoLogo@METAPOST#1{%
  \HoLogoFont@font{METAPOST}{logo}{%
    \HOLOGO@mbox{META}%
    \HOLOGO@discretionary
    \HOLOGO@mbox{POST}%
  }%
}
%    \end{macrocode}
%    \end{macro}
%
%    \begin{macro}{\HoLogo@MetaFun}
%    \begin{macrocode}
\def\HoLogo@MetaFun#1{%
  \HOLOGO@mbox{Meta}%
  \HOLOGO@discretionary
  \HOLOGO@mbox{Fun}%
}
%    \end{macrocode}
%    \end{macro}
%
%    \begin{macro}{\HoLogo@MetaPost}
%    \begin{macrocode}
\def\HoLogo@MetaPost#1{%
  \HOLOGO@mbox{Meta}%
  \HOLOGO@discretionary
  \HOLOGO@mbox{Post}%
}
%    \end{macrocode}
%    \end{macro}
%
% \subsection{Others}
%
% \subsubsection{\hologo{biber}}
%
%    \begin{macro}{\HoLogo@biber}
%    \begin{macrocode}
\def\HoLogo@biber#1{%
  \HOLOGO@mbox{#1{b}{B}i}%
  \HOLOGO@discretionary
  \HOLOGO@mbox{ber}%
}
%    \end{macrocode}
%    \end{macro}
%    \begin{macro}{\HoLogoCs@biber}
%    \begin{macrocode}
\def\HoLogoCs@biber#1{#1{b}{B}iber}
%    \end{macrocode}
%    \end{macro}
%    \begin{macro}{\HoLogoBkm@biber}
%    \begin{macrocode}
\def\HoLogoBkm@biber#1{%
  #1{b}{B}iber%
}
%    \end{macrocode}
%    \end{macro}
%    \begin{macro}{\HoLogoHtml@biber}
%    \begin{macrocode}
\let\HoLogoHtml@biber\HoLogo@biber
%    \end{macrocode}
%    \end{macro}
%
% \subsubsection{\hologo{KOMAScript}}
%
%    \begin{macro}{\HoLogo@KOMAScript}
%    The definition for \hologo{KOMAScript} is taken
%    from \hologo{KOMAScript} (\xfile{scrlogo.dtx}, reformatted) \cite{scrlogo}:
%\begin{quote}
%\begin{verbatim}
%\@ifundefined{KOMAScript}{%
%  \DeclareRobustCommand{\KOMAScript}{%
%    \textsf{%
%      K\kern.05em O\kern.05emM\kern.05em A%
%      \kern.1em-\kern.1em %
%      Script%
%    }%
%  }%
%}{}
%\end{verbatim}
%\end{quote}
%    \begin{macrocode}
\def\HoLogo@KOMAScript#1{%
  \HoLogoFont@font{KOMAScript}{sf}{%
    \HOLOGO@mbox{%
      K\kern.05em%
      O\kern.05em%
      M\kern.05em%
      A%
    }%
    \kern.1em%
    \HOLOGO@hyphen
    \kern.1em%
    \HOLOGO@mbox{Script}%
  }%
}
%    \end{macrocode}
%    \end{macro}
%    \begin{macro}{\HoLogoBkm@KOMAScript}
%    \begin{macrocode}
\def\HoLogoBkm@KOMAScript#1{%
  KOMA-Script%
}
%    \end{macrocode}
%    \end{macro}
%    \begin{macro}{\HoLogoHtml@KOMAScript}
%    \begin{macrocode}
\def\HoLogoHtml@KOMAScript#1{%
  \HoLogoCss@KOMAScript
  \HoLogoFont@font{KOMAScript}{sf}{%
    \HOLOGO@Span{KOMAScript}{%
      K%
      \HOLOGO@Span{O}{O}%
      M%
      \HOLOGO@Span{A}{A}%
      \HOLOGO@Span{hyphen}{-}%
      Script%
    }%
  }%
}
%    \end{macrocode}
%    \end{macro}
%    \begin{macro}{\HoLogoCss@KOMAScript}
%    \begin{macrocode}
\def\HoLogoCss@KOMAScript{%
  \Css{%
    span.HoLogo-KOMAScript{%
      font-family:sans-serif;%
    }%
  }%
  \Css{%
    span.HoLogo-KOMAScript span.HoLogo-O{%
      padding-left:.05em;%
      padding-right:.05em;%
    }%
  }%
  \Css{%
    span.HoLogo-KOMAScript span.HoLogo-A{%
      padding-left:.05em;%
    }%
  }%
  \Css{%
    span.HoLogo-KOMAScript span.HoLogo-hyphen{%
      padding-left:.1em;%
      padding-right:.1em;%
    }%
  }%
  \global\let\HoLogoCss@KOMAScript\relax
}
%    \end{macrocode}
%    \end{macro}
%
% \subsubsection{\hologo{LyX}}
%
%    \begin{macro}{\HoLogo@LyX}
%    The definition is taken from the documentation source files
%    of \hologo{LyX}, \xfile{Intro.lyx} \cite{LyX}:
%\begin{quote}
%\begin{verbatim}
%\def\LyX{%
%  \texorpdfstring{%
%    L\kern-.1667em\lower.25em\hbox{Y}\kern-.125emX\@%
%  }{%
%    LyX%
%  }%
%}
%\end{verbatim}
%\end{quote}
%    \begin{macrocode}
\def\HoLogo@LyX#1{%
  L%
  \kern-.1667em%
  \lower.25em\hbox{Y}%
  \kern-.125em%
  X%
  \HOLOGO@SpaceFactor
}
%    \end{macrocode}
%    \end{macro}
%    \begin{macro}{\HoLogoHtml@LyX}
%    \begin{macrocode}
\def\HoLogoHtml@LyX#1{%
  \HoLogoCss@LyX
  \HOLOGO@Span{LyX}{%
    L%
    \HOLOGO@Span{y}{Y}%
    X%
  }%
}
%    \end{macrocode}
%    \end{macro}
%    \begin{macro}{\HoLogoCss@LyX}
%    \begin{macrocode}
\def\HoLogoCss@LyX{%
  \Css{%
    span.HoLogo-LyX span.HoLogo-y{%
      position:relative;%
      top:.25em;%
      margin-left:-.1667em;%
      margin-right:-.125em;%
      text-decoration:none;%
    }%
  }%
  \global\let\HoLogoCss@LyX\relax
}
%    \end{macrocode}
%    \end{macro}
%
% \subsubsection{\hologo{NTS}}
%
%    \begin{macro}{\HoLogo@NTS}
%    Definition for \hologo{NTS} can be found in
%    package \xpackage{etex\textunderscore man} for the \hologo{eTeX} manual \cite{etexman}
%    and in package \xpackage{dtklogos} \cite{dtklogos}:
%\begin{quote}
%\begin{verbatim}
%\def\NTS{%
%  \leavevmode
%  \hbox{%
%    $%
%      \cal N%
%      \kern-0.35em%
%      \lower0.5ex\hbox{$\cal T$}%
%      \kern-0.2em%
%      S%
%    $%
%  }%
%}
%\end{verbatim}
%\end{quote}
%    \begin{macrocode}
\def\HoLogo@NTS#1{%
  \HoLogoFont@font{NTS}{sy}{%
    N\/%
    \kern-.35em%
    \lower.5ex\hbox{T\/}%
    \kern-.2em%
    S\/%
  }%
  \HOLOGO@SpaceFactor
}
%    \end{macrocode}
%    \end{macro}
%
% \subsubsection{\Hologo{TTH} (\hologo{TeX} to HTML translator)}
%
%    Source: \url{http://hutchinson.belmont.ma.us/tth/}
%    In the HTML source the second `T' is printed as subscript.
%\begin{quote}
%\begin{verbatim}
%T<sub>T</sub>H
%\end{verbatim}
%\end{quote}
%    \begin{macro}{\HoLogo@TTH}
%    \begin{macrocode}
\def\HoLogo@TTH#1{%
  \ltx@mbox{%
    T\HOLOGO@SubScript{T}H%
  }%
  \HOLOGO@SpaceFactor
}
%    \end{macrocode}
%    \end{macro}
%
%    \begin{macro}{\HoLogoHtml@TTH}
%    \begin{macrocode}
\def\HoLogoHtml@TTH#1{%
  T\HCode{<sub>}T\HCode{</sub>}H%
}
%    \end{macrocode}
%    \end{macro}
%
% \subsubsection{\Hologo{HanTheThanh}}
%
%    Partial source: Package \xpackage{dtklogos}.
%    The double accent is U+1EBF (latin small letter e with circumflex
%    and acute).
%    \begin{macro}{\HoLogo@HanTheThanh}
%    \begin{macrocode}
\def\HoLogo@HanTheThanh#1{%
  \ltx@mbox{H\`an}%
  \HOLOGO@space
  \ltx@mbox{%
    Th%
    \HOLOGO@IfCharExists{"1EBF}{%
      \char"1EBF\relax
    }{%
      \^e\hbox to 0pt{\hss\raise .5ex\hbox{\'{}}}%
    }%
  }%
  \HOLOGO@space
  \ltx@mbox{Th\`anh}%
}
%    \end{macrocode}
%    \end{macro}
%    \begin{macro}{\HoLogoBkm@HanTheThanh}
%    \begin{macrocode}
\def\HoLogoBkm@HanTheThanh#1{%
  H\`an %
  Th\HOLOGO@PdfdocUnicode{\^e}{\9036\277} %
  Th\`anh%
}
%    \end{macrocode}
%    \end{macro}
%    \begin{macro}{\HoLogoHtml@HanTheThanh}
%    \begin{macrocode}
\def\HoLogoHtml@HanTheThanh#1{%
  H\`an %
  Th\HCode{&\ltx@hashchar x1ebf;} %
  Th\`anh%
}
%    \end{macrocode}
%    \end{macro}
%
% \subsection{Driver detection}
%
%    \begin{macrocode}
\HOLOGO@IfExists\InputIfFileExists{%
  \InputIfFileExists{hologo.cfg}{}{}%
}{%
  \ltx@IfUndefined{pdf@filesize}{%
    \def\HOLOGO@InputIfExists{%
      \openin\HOLOGO@temp=hologo.cfg\relax
      \ifeof\HOLOGO@temp
        \closein\HOLOGO@temp
      \else
        \closein\HOLOGO@temp
        \begingroup
          \def\x{LaTeX2e}%
        \expandafter\endgroup
        \ifx\fmtname\x
          \input{hologo.cfg}%
        \else
          \input hologo.cfg\relax
        \fi
      \fi
    }%
    \ltx@IfUndefined{newread}{%
      \chardef\HOLOGO@temp=15 %
      \def\HOLOGO@CheckRead{%
        \ifeof\HOLOGO@temp
          \HOLOGO@InputIfExists
        \else
          \ifcase\HOLOGO@temp
            \@PackageWarningNoLine{hologo}{%
              Configuration file ignored, because\MessageBreak
              a free read register could not be found%
            }%
          \else
            \begingroup
              \count\ltx@cclv=\HOLOGO@temp
              \advance\ltx@cclv by \ltx@minusone
              \edef\x{\endgroup
                \chardef\noexpand\HOLOGO@temp=\the\count\ltx@cclv
                \relax
              }%
            \x
          \fi
        \fi
      }%
    }{%
      \csname newread\endcsname\HOLOGO@temp
      \HOLOGO@InputIfExists
    }%
  }{%
    \edef\HOLOGO@temp{\pdf@filesize{hologo.cfg}}%
    \ifx\HOLOGO@temp\ltx@empty
    \else
      \ifnum\HOLOGO@temp>0 %
        \begingroup
          \def\x{LaTeX2e}%
        \expandafter\endgroup
        \ifx\fmtname\x
          \input{hologo.cfg}%
        \else
          \input hologo.cfg\relax
        \fi
      \else
        \@PackageInfoNoLine{hologo}{%
          Empty configuration file `hologo.cfg' ignored%
        }%
      \fi
    \fi
  }%
}
%    \end{macrocode}
%
%    \begin{macrocode}
\def\HOLOGO@temp#1#2{%
  \kv@define@key{HoLogoDriver}{#1}[]{%
    \begingroup
      \def\HOLOGO@temp{##1}%
      \ltx@onelevel@sanitize\HOLOGO@temp
      \ifx\HOLOGO@temp\ltx@empty
      \else
        \@PackageError{hologo}{%
          Value (\HOLOGO@temp) not permitted for option `#1'%
        }%
        \@ehc
      \fi
    \endgroup
    \def\hologoDriver{#2}%
  }%
}%
\def\HOLOGO@@temp#1#2{%
  \ifx\kv@value\relax
    \HOLOGO@temp{#1}{#1}%
  \else
    \HOLOGO@temp{#1}{#2}%
  \fi
}%
\kv@parse@normalized{%
  pdftex,%
  luatex=pdftex,%
  dvipdfm,%
  dvipdfmx=dvipdfm,%
  dvips,%
  dvipsone=dvips,%
  xdvi=dvips,%
  xetex,%
  vtex,%
}\HOLOGO@@temp
%    \end{macrocode}
%
%    \begin{macrocode}
\kv@define@key{HoLogoDriver}{driverfallback}{%
  \def\HOLOGO@DriverFallback{#1}%
}
%    \end{macrocode}
%
%    \begin{macro}{\HOLOGO@DriverFallback}
%    \begin{macrocode}
\def\HOLOGO@DriverFallback{dvips}
%    \end{macrocode}
%    \end{macro}
%
%    \begin{macro}{\hologoDriverSetup}
%    \begin{macrocode}
\def\hologoDriverSetup{%
  \let\hologoDriver\ltx@undefined
  \HOLOGO@DriverSetup
}
%    \end{macrocode}
%    \end{macro}
%
%    \begin{macro}{\HOLOGO@DriverSetup}
%    \begin{macrocode}
\def\HOLOGO@DriverSetup#1{%
  \kvsetkeys{HoLogoDriver}{#1}%
  \HOLOGO@CheckDriver
  \ltx@ifundefined{hologoDriver}{%
    \begingroup
    \edef\x{\endgroup
      \noexpand\kvsetkeys{HoLogoDriver}{\HOLOGO@DriverFallback}%
    }\x
  }{}%
  \@PackageInfoNoLine{hologo}{Using driver `\hologoDriver'}%
}
%    \end{macrocode}
%    \end{macro}
%
%    \begin{macro}{\HOLOGO@CheckDriver}
%    \begin{macrocode}
\def\HOLOGO@CheckDriver{%
  \ifpdf
    \def\hologoDriver{pdftex}%
    \let\HOLOGO@pdfliteral\pdfliteral
    \ifluatex
      \ifx\pdfextension\@undefined\else
        \protected\def\pdfliteral{\pdfextension literal}%
        \let\HOLOGO@pdfliteral\pdfliteral
      \fi
      \ltx@IfUndefined{HOLOGO@pdfliteral}{%
        \ifnum\luatexversion<36 %
        \else
          \begingroup
            \let\HOLOGO@temp\endgroup
            \ifcase0%
                \directlua{%
                  if tex.enableprimitives then %
                    tex.enableprimitives('HOLOGO@', {'pdfliteral'})%
                  else %
                    tex.print('1')%
                  end%
                }%
                \ifx\HOLOGO@pdfliteral\@undefined 1\fi%
                \relax%
              \endgroup
              \let\HOLOGO@temp\relax
              \global\let\HOLOGO@pdfliteral\HOLOGO@pdfliteral
            \fi%
          \HOLOGO@temp
        \fi
      }{}%
    \fi
    \ltx@IfUndefined{HOLOGO@pdfliteral}{%
      \@PackageWarningNoLine{hologo}{%
        Cannot find \string\pdfliteral
      }%
    }{}%
  \else
    \ifxetex
      \def\hologoDriver{xetex}%
    \else
      \ifvtex
        \def\hologoDriver{vtex}%
      \fi
    \fi
  \fi
}
%    \end{macrocode}
%    \end{macro}
%
%    \begin{macro}{\HOLOGO@WarningUnsupportedDriver}
%    \begin{macrocode}
\def\HOLOGO@WarningUnsupportedDriver#1{%
  \@PackageWarningNoLine{hologo}{%
    Logo `#1' needs driver specific macros,\MessageBreak
    but driver `\hologoDriver' is not supported.\MessageBreak
    Use a different driver or\MessageBreak
    load package `graphics' or `pgf'%
  }%
}
%    \end{macrocode}
%    \end{macro}
%
% \subsubsection{Reflect box macros}
%
%    Skip driver part if not needed.
%    \begin{macrocode}
\ltx@IfUndefined{reflectbox}{}{%
  \ltx@IfUndefined{rotatebox}{}{%
    \HOLOGO@AtEnd
  }%
}
\ltx@IfUndefined{pgftext}{}{%
  \HOLOGO@AtEnd
}
\ltx@IfUndefined{psscalebox}{}{%
  \HOLOGO@AtEnd
}
%    \end{macrocode}
%
%    \begin{macrocode}
\def\HOLOGO@temp{LaTeX2e}
\ifx\fmtname\HOLOGO@temp
  \RequirePackage{kvoptions}[2011/06/30]%
  \ProcessKeyvalOptions{HoLogoDriver}%
\fi
\HOLOGO@DriverSetup{}
%    \end{macrocode}
%
%    \begin{macro}{\HOLOGO@ReflectBox}
%    \begin{macrocode}
\def\HOLOGO@ReflectBox#1{%
  \begingroup
    \setbox\ltx@zero\hbox{\begingroup#1\endgroup}%
    \setbox\ltx@two\hbox{%
      \kern\wd\ltx@zero
      \csname HOLOGO@ScaleBox@\hologoDriver\endcsname{-1}{1}{%
        \hbox to 0pt{\copy\ltx@zero\hss}%
      }%
    }%
    \wd\ltx@two=\wd\ltx@zero
    \box\ltx@two
  \endgroup
}
%    \end{macrocode}
%    \end{macro}
%
%    \begin{macro}{\HOLOGO@PointReflectBox}
%    \begin{macrocode}
\def\HOLOGO@PointReflectBox#1{%
  \begingroup
    \setbox\ltx@zero\hbox{\begingroup#1\endgroup}%
    \setbox\ltx@two\hbox{%
      \kern\wd\ltx@zero
      \raise\ht\ltx@zero\hbox{%
        \csname HOLOGO@ScaleBox@\hologoDriver\endcsname{-1}{-1}{%
          \hbox to 0pt{\copy\ltx@zero\hss}%
        }%
      }%
    }%
    \wd\ltx@two=\wd\ltx@zero
    \box\ltx@two
  \endgroup
}
%    \end{macrocode}
%    \end{macro}
%
%    We must define all variants because of dynamic driver setup.
%    \begin{macrocode}
\def\HOLOGO@temp#1#2{#2}
%    \end{macrocode}
%
%    \begin{macro}{\HOLOGO@ScaleBox@pdftex}
%    \begin{macrocode}
\HOLOGO@temp{pdftex}{%
  \def\HOLOGO@ScaleBox@pdftex#1#2#3{%
    \HOLOGO@pdfliteral{%
      q #1 0 0 #2 0 0 cm%
    }%
    #3%
    \HOLOGO@pdfliteral{%
      Q%
    }%
  }%
}
%    \end{macrocode}
%    \end{macro}
%    \begin{macro}{\HOLOGO@ScaleBox@dvips}
%    \begin{macrocode}
\HOLOGO@temp{dvips}{%
  \def\HOLOGO@ScaleBox@dvips#1#2#3{%
    \special{ps:%
      gsave %
      currentpoint %
      currentpoint translate %
      #1 #2 scale %
      neg exch neg exch translate%
    }%
    #3%
    \special{ps:%
      currentpoint %
      grestore %
      moveto%
    }%
  }%
}
%    \end{macrocode}
%    \end{macro}
%    \begin{macro}{\HOLOGO@ScaleBox@dvipdfm}
%    \begin{macrocode}
\HOLOGO@temp{dvipdfm}{%
  \let\HOLOGO@ScaleBox@dvipdfm\HOLOGO@ScaleBox@dvips
}
%    \end{macrocode}
%    \end{macro}
%    Since \hologo{XeTeX} v0.6.
%    \begin{macro}{\HOLOGO@ScaleBox@xetex}
%    \begin{macrocode}
\HOLOGO@temp{xetex}{%
  \def\HOLOGO@ScaleBox@xetex#1#2#3{%
    \special{x:gsave}%
    \special{x:scale #1 #2}%
    #3%
    \special{x:grestore}%
  }%
}
%    \end{macrocode}
%    \end{macro}
%    \begin{macro}{\HOLOGO@ScaleBox@vtex}
%    \begin{macrocode}
\HOLOGO@temp{vtex}{%
  \def\HOLOGO@ScaleBox@vtex#1#2#3{%
    \special{r(#1,0,0,#2,0,0}%
    #3%
    \special{r)}%
  }%
}
%    \end{macrocode}
%    \end{macro}
%
%    \begin{macrocode}
\HOLOGO@AtEnd%
%</package>
%    \end{macrocode}
%
% \section{Test}
%
% \subsection{Catcode checks for loading}
%
%    \begin{macrocode}
%<*test1>
%    \end{macrocode}
%    \begin{macrocode}
\catcode`\{=1 %
\catcode`\}=2 %
\catcode`\#=6 %
\catcode`\@=11 %
\expandafter\ifx\csname count@\endcsname\relax
  \countdef\count@=255 %
\fi
\expandafter\ifx\csname @gobble\endcsname\relax
  \long\def\@gobble#1{}%
\fi
\expandafter\ifx\csname @firstofone\endcsname\relax
  \long\def\@firstofone#1{#1}%
\fi
\expandafter\ifx\csname loop\endcsname\relax
  \expandafter\@firstofone
\else
  \expandafter\@gobble
\fi
{%
  \def\loop#1\repeat{%
    \def\body{#1}%
    \iterate
  }%
  \def\iterate{%
    \body
      \let\next\iterate
    \else
      \let\next\relax
    \fi
    \next
  }%
  \let\repeat=\fi
}%
\def\RestoreCatcodes{}
\count@=0 %
\loop
  \edef\RestoreCatcodes{%
    \RestoreCatcodes
    \catcode\the\count@=\the\catcode\count@\relax
  }%
\ifnum\count@<255 %
  \advance\count@ 1 %
\repeat

\def\RangeCatcodeInvalid#1#2{%
  \count@=#1\relax
  \loop
    \catcode\count@=15 %
  \ifnum\count@<#2\relax
    \advance\count@ 1 %
  \repeat
}
\def\RangeCatcodeCheck#1#2#3{%
  \count@=#1\relax
  \loop
    \ifnum#3=\catcode\count@
    \else
      \errmessage{%
        Character \the\count@\space
        with wrong catcode \the\catcode\count@\space
        instead of \number#3%
      }%
    \fi
  \ifnum\count@<#2\relax
    \advance\count@ 1 %
  \repeat
}
\def\space{ }
\expandafter\ifx\csname LoadCommand\endcsname\relax
  \def\LoadCommand{\input hologo.sty\relax}%
\fi
\def\Test{%
  \RangeCatcodeInvalid{0}{47}%
  \RangeCatcodeInvalid{58}{64}%
  \RangeCatcodeInvalid{91}{96}%
  \RangeCatcodeInvalid{123}{255}%
  \catcode`\@=12 %
  \catcode`\\=0 %
  \catcode`\%=14 %
  \LoadCommand
  \RangeCatcodeCheck{0}{36}{15}%
  \RangeCatcodeCheck{37}{37}{14}%
  \RangeCatcodeCheck{38}{47}{15}%
  \RangeCatcodeCheck{48}{57}{12}%
  \RangeCatcodeCheck{58}{63}{15}%
  \RangeCatcodeCheck{64}{64}{12}%
  \RangeCatcodeCheck{65}{90}{11}%
  \RangeCatcodeCheck{91}{91}{15}%
  \RangeCatcodeCheck{92}{92}{0}%
  \RangeCatcodeCheck{93}{96}{15}%
  \RangeCatcodeCheck{97}{122}{11}%
  \RangeCatcodeCheck{123}{255}{15}%
  \RestoreCatcodes
}
\Test
\csname @@end\endcsname
\end
%    \end{macrocode}
%    \begin{macrocode}
%</test1>
%    \end{macrocode}
%
% \subsection{Spacefactor}
%
%    The space factor must be 1000 after a logo. If it is greater 1000
%    then the following space is a space after a sentence closing point.
%    If the space factor is smaller 1000 then an immediate following
%    dot is interpreted as abbreviation, not sentence closing point.
%
%    \begin{macrocode}
%<*test-spacefactor>
\NeedsTeXFormat{LaTeX2e}
\documentclass{article}
\usepackage{hologo}[2016/05/12]
\usepackage{kvsetkeys}
\usepackage{qstest}
\IncludeTests{*}
\LogTests{log}{*}{*}
\begin{document}
\begin{qstest}{spacefactor}{spacefactor}
\newcommand*{\Test}[1]{%
  \sbox0{%
    \hologo{#1}%
    \Expect*{1000 (#1)}*{\the\spacefactor\space(#1)}%
  }%
}%
\makeatletter
\def\TestList{}
\def\hologoEntry#1#2#3{%
  \edef\TestList{%
    \ifx\TestList\@empty
    \else
      \TestList,%
    \fi
    #1%
    \ifx\\#2\\%
    \else
      ={variant=#2}%
    \fi
  }%
}
\hologoList
\expandafter\kv@parse@normalized\expandafter{%
  \TestList
}{%
  \begingroup
    \let\@logo=\kv@key
    \ifx\kv@value\relax
    \else
      \expandafter\hologoLogoSetup\expandafter\@logo\expandafter{%
        \kv@value
      }%
    \fi
    \Test\@logo
  \endgroup
  \@gobbletwo
}
\end{qstest}
\end{document}
%</test-spacefactor>
%    \end{macrocode}
%
% \subsection{Complete list}
%
%    \begin{macrocode}
%<*test-list>
\NeedsTeXFormat{LaTeX2e}
\documentclass[12pt,a4paper]{article}
\usepackage{hologo}[2016/05/12]
\usepackage[T1]{fontenc}
\usepackage{lmodern}
\usepackage{parskip}
\usepackage[unicode]{hyperref}[2011/09/28]
\usepackage{bookmark}[2011/09/19]
\bookmarksetup{%
  numbered,%
  open,%
  openlevel=2,%
}
\renewcommand*{\contentsname}{List of logos}
\begin{document}
\tableofcontents
\def\TestFont#1#2#3#4#5#6{%
  \begingroup
    \usefont{#3}{#4}{#5}{#6}%
    \HologoVariant{#1}{#2}/\hologoVariant{#1}{#2}%
    \quad
    \begingroup\scriptsize\hologoVariant{#1}{#2}\endgroup
    \quad
  \endgroup
  (#3/#4/#5/#6)%
  \par
}
\makeatletter
\def\hologoEntry#1#2#3{%
  \section{%
    \HologoVariant{#1}{#2}/\hologoVariant{#1}{#2} %
    {[#1\ifx\\#2\\\else\space(#2)\fi]}% hash-ok
  }% braces around [] because of bug in tex4ht
  \begingroup
    \hypersetup{unicode=false}%
    \bookmark[%
      dest=\@currentHref,%
      rellevel=1,%
      keeplevel,%
    ]{%
      \HologoVariant{#1}{#2}/\hologoVariant{#1}{#2} %
      (PDFDocEncoding)%
    }%
  \endgroup
  \TestFont{#1}{#2}{OT1}{cmr}{m}{n}%
  \TestFont{#1}{#2}{OT1}{cmss}{m}{n}%
  \TestFont{#1}{#2}{OT1}{cmr}{b}{n}%
  \TestFont{#1}{#2}{OT1}{cmr}{m}{it}%
  \TestFont{#1}{#2}{OT1}{cmtt}{m}{n}%
  \TestFont{#1}{#2}{T1}{lmr}{m}{n}%
  \TestFont{#1}{#2}{T1}{lmss}{m}{n}%
  \TestFont{#1}{#2}{T1}{lmr}{b}{n}%
  \TestFont{#1}{#2}{T1}{lmr}{m}{it}%
  \TestFont{#1}{#2}{T1}{lmtt}{m}{n}%
  \TestFont{#1}{#2}{T1}{lmvtt}{m}{n}%
  \TestFont{#1}{#2}{T1}{qtm}{m}{n}%
  \TestFont{#1}{#2}{T1}{qhv}{m}{n}%
  \TestFont{#1}{#2}{T1}{qtm}{b}{n}%
  \TestFont{#1}{#2}{T1}{qtm}{m}{it}%
  \TestFont{#1}{#2}{T1}{qcr}{m}{n}%
  \newpage
}
\makeatother
\hologoList
\end{document}
%</test-list>
%    \end{macrocode}
%
% \section{Installation}
%
% \subsection{Download}
%
% \paragraph{Package.} This package is available on
% CTAN\footnote{\url{ftp://ftp.ctan.org/tex-archive/}}:
% \begin{description}
% \item[\CTAN{macros/latex/contrib/oberdiek/hologo.dtx}] The source file.
% \item[\CTAN{macros/latex/contrib/oberdiek/hologo.pdf}] Documentation.
% \end{description}
%
%
% \paragraph{Bundle.} All the packages of the bundle `oberdiek'
% are also available in a TDS compliant ZIP archive. There
% the packages are already unpacked and the documentation files
% are generated. The files and directories obey the TDS standard.
% \begin{description}
% \item[\CTAN{install/macros/latex/contrib/oberdiek.tds.zip}]
% \end{description}
% \emph{TDS} refers to the standard ``A Directory Structure
% for \TeX\ Files'' (\CTAN{tds/tds.pdf}). Directories
% with \xfile{texmf} in their name are usually organized this way.
%
% \subsection{Bundle installation}
%
% \paragraph{Unpacking.} Unpack the \xfile{oberdiek.tds.zip} in the
% TDS tree (also known as \xfile{texmf} tree) of your choice.
% Example (linux):
% \begin{quote}
%   |unzip oberdiek.tds.zip -d ~/texmf|
% \end{quote}
%
% \paragraph{Script installation.}
% Check the directory \xfile{TDS:scripts/oberdiek/} for
% scripts that need further installation steps.
% Package \xpackage{attachfile2} comes with the Perl script
% \xfile{pdfatfi.pl} that should be installed in such a way
% that it can be called as \texttt{pdfatfi}.
% Example (linux):
% \begin{quote}
%   |chmod +x scripts/oberdiek/pdfatfi.pl|\\
%   |cp scripts/oberdiek/pdfatfi.pl /usr/local/bin/|
% \end{quote}
%
% \subsection{Package installation}
%
% \paragraph{Unpacking.} The \xfile{.dtx} file is a self-extracting
% \docstrip\ archive. The files are extracted by running the
% \xfile{.dtx} through \plainTeX:
% \begin{quote}
%   \verb|tex hologo.dtx|
% \end{quote}
%
% \paragraph{TDS.} Now the different files must be moved into
% the different directories in your installation TDS tree
% (also known as \xfile{texmf} tree):
% \begin{quote}
% \def\t{^^A
% \begin{tabular}{@{}>{\ttfamily}l@{ $\rightarrow$ }>{\ttfamily}l@{}}
%   hologo.sty & tex/generic/oberdiek/hologo.sty\\
%   hologo.pdf & doc/latex/oberdiek/hologo.pdf\\
%   example/hologo-example.tex & doc/latex/oberdiek/example/hologo-example.tex\\
%   test/hologo-test1.tex & doc/latex/oberdiek/test/hologo-test1.tex\\
%   test/hologo-test-spacefactor.tex & doc/latex/oberdiek/test/hologo-test-spacefactor.tex\\
%   test/hologo-test-list.tex & doc/latex/oberdiek/test/hologo-test-list.tex\\
%   hologo.dtx & source/latex/oberdiek/hologo.dtx\\
% \end{tabular}^^A
% }^^A
% \sbox0{\t}^^A
% \ifdim\wd0>\linewidth
%   \begingroup
%     \advance\linewidth by\leftmargin
%     \advance\linewidth by\rightmargin
%   \edef\x{\endgroup
%     \def\noexpand\lw{\the\linewidth}^^A
%   }\x
%   \def\lwbox{^^A
%     \leavevmode
%     \hbox to \linewidth{^^A
%       \kern-\leftmargin\relax
%       \hss
%       \usebox0
%       \hss
%       \kern-\rightmargin\relax
%     }^^A
%   }^^A
%   \ifdim\wd0>\lw
%     \sbox0{\small\t}^^A
%     \ifdim\wd0>\linewidth
%       \ifdim\wd0>\lw
%         \sbox0{\footnotesize\t}^^A
%         \ifdim\wd0>\linewidth
%           \ifdim\wd0>\lw
%             \sbox0{\scriptsize\t}^^A
%             \ifdim\wd0>\linewidth
%               \ifdim\wd0>\lw
%                 \sbox0{\tiny\t}^^A
%                 \ifdim\wd0>\linewidth
%                   \lwbox
%                 \else
%                   \usebox0
%                 \fi
%               \else
%                 \lwbox
%               \fi
%             \else
%               \usebox0
%             \fi
%           \else
%             \lwbox
%           \fi
%         \else
%           \usebox0
%         \fi
%       \else
%         \lwbox
%       \fi
%     \else
%       \usebox0
%     \fi
%   \else
%     \lwbox
%   \fi
% \else
%   \usebox0
% \fi
% \end{quote}
% If you have a \xfile{docstrip.cfg} that configures and enables \docstrip's
% TDS installing feature, then some files can already be in the right
% place, see the documentation of \docstrip.
%
% \subsection{Refresh file name databases}
%
% If your \TeX~distribution
% (\teTeX, \mikTeX, \dots) relies on file name databases, you must refresh
% these. For example, \teTeX\ users run \verb|texhash| or
% \verb|mktexlsr|.
%
% \subsection{Some details for the interested}
%
% \paragraph{Attached source.}
%
% The PDF documentation on CTAN also includes the
% \xfile{.dtx} source file. It can be extracted by
% AcrobatReader 6 or higher. Another option is \textsf{pdftk},
% e.g. unpack the file into the current directory:
% \begin{quote}
%   \verb|pdftk hologo.pdf unpack_files output .|
% \end{quote}
%
% \paragraph{Unpacking with \LaTeX.}
% The \xfile{.dtx} chooses its action depending on the format:
% \begin{description}
% \item[\plainTeX:] Run \docstrip\ and extract the files.
% \item[\LaTeX:] Generate the documentation.
% \end{description}
% If you insist on using \LaTeX\ for \docstrip\ (really,
% \docstrip\ does not need \LaTeX), then inform the autodetect routine
% about your intention:
% \begin{quote}
%   \verb|latex \let\install=y\input{hologo.dtx}|
% \end{quote}
% Do not forget to quote the argument according to the demands
% of your shell.
%
% \paragraph{Generating the documentation.}
% You can use both the \xfile{.dtx} or the \xfile{.drv} to generate
% the documentation. The process can be configured by the
% configuration file \xfile{ltxdoc.cfg}. For instance, put this
% line into this file, if you want to have A4 as paper format:
% \begin{quote}
%   \verb|\PassOptionsToClass{a4paper}{article}|
% \end{quote}
% An example follows how to generate the
% documentation with pdf\LaTeX:
% \begin{quote}
%\begin{verbatim}
%pdflatex hologo.dtx
%makeindex -s gind.ist hologo.idx
%pdflatex hologo.dtx
%makeindex -s gind.ist hologo.idx
%pdflatex hologo.dtx
%\end{verbatim}
% \end{quote}
%
% \section{Catalogue}
%
% The following XML file can be used as source for the
% \href{http://mirror.ctan.org/help/Catalogue/catalogue.html}{\TeX\ Catalogue}.
% The elements \texttt{caption} and \texttt{description} are imported
% from the original XML file from the Catalogue.
% The name of the XML file in the Catalogue is \xfile{hologo.xml}.
%    \begin{macrocode}
%<*catalogue>
<?xml version='1.0' encoding='us-ascii'?>
<!DOCTYPE entry SYSTEM 'catalogue.dtd'>
<entry datestamp='$Date$' modifier='$Author$' id='hologo'>
  <name>hologo</name>
  <caption>A collection of logos with bookmark support.</caption>
  <authorref id='auth:oberdiek'/>
  <copyright owner='Heiko Oberdiek' year='2010-2012'/>
  <license type='lppl1.3'/>
  <version number='1.10'/>
  <description>
    The package defines a single command <tt>\hologo</tt>, whose
    argument is the usual case-confused ASCII version of the logo.
    The command is bookmark-enabled, so that every logo becomes
    available in bookmarks without further work.
    <p/>
    The package is part of the <xref refid='oberdiek'>oberdiek</xref>
    bundle.
  </description>
  <documentation details='Package documentation'
      href='ctan:/macros/latex/contrib/oberdiek/hologo.pdf'/>
  <ctan file='true' path='/macros/latex/contrib/oberdiek/hologo.dtx'/>
  <miktex location='oberdiek'/>
  <texlive location='oberdiek'/>
  <install path='/macros/latex/contrib/oberdiek/oberdiek.tds.zip'/>
</entry>
%</catalogue>
%    \end{macrocode}
%
% \begin{thebibliography}{9}
% \raggedright
%
% \bibitem{btxdoc}
% Oren Patashnik,
% \textit{\hologo{BibTeX}ing},
% 1988-02-08.\\
% \CTAN{biblio/bibtex/base/}
%
% \bibitem{dtklogos}
% Gerd Neugebauer, DANTE,
% \textit{Package \xpackage{dtklogos}},
% 2011-04-25.\\
% \CTAN{usergrps/dante/dtk/dtklogos.sty}
%
% \bibitem{etexman}
% The \hologo{NTS} Team,
% \textit{The \hologo{eTeX} manual},
% 1998-02.\\
% \CTAN{systems/e-tex/v2/doc/}
%
% \bibitem{ExTeX-FAQ}
% The \hologo{ExTeX} group,
% \textit{\hologo{ExTeX}: FAQ -- How is \hologo{ExTeX} typeset?},
% 2007-04-14.\\
% \url{http://www.extex.org/documentation/faq.html}
%
% \bibitem{LyX}
% %@MISC{ LyX,
% %  title = {{LyX 2.0.0 -- The Document Processor [Computer software and manual]}},
% %  author = {{The LyX Team}},
% %  howpublished = {Internet: http://www.lyx.org},
% %  year = {2011-05-08},
% %  note = {Retrieved May 10, 2011, from http://www.lyx.org},
% %  url = {http://www.lyx.org/}
% %}
% The \hologo{LyX} Team,
% \textit{\hologo{LyX} -- The Document Processor},
% 2011-05-08.\\
% \url{http://www.lyx.org/}
%
% \bibitem{OzTeX}
% Andrew Trevorrow,
% \hologo{OzTeX} FAQ: What is the correct way to typeset ``\hologo{OzTeX}''?,
% 2011-09-15 (visited).
% \url{http://www.trevorrow.com/oztex/ozfaq.html#oztex-logo}
%
% \bibitem{PiCTeX}
% Michael Wichura,
% \textit{The \hologo{PiCTeX} macro package},
% 1987-09-21.
% \CTAN{graphics/pictex/}
%
% \bibitem{scrlogo}
% Markus Kohm,
% \textit{\hologo{KOMAScript} Datei \xfile{scrlogo.dtx}},
% 2009-01-30.\\
% \CTAN{install/macros/latex/contrib/komascript.tds.zip}
%
% \end{thebibliography}
%
% \begin{History}
%   \begin{Version}{2010/04/08 v1.0}
%   \item
%     The first version.
%   \end{Version}
%   \begin{Version}{2010/04/16 v1.1}
%   \item
%     \cs{Hologo} added for support of logos at start of a sentence.
%   \item
%     \cs{hologoSetup} and \cs{hologoLogoSetup} added.
%   \item
%     Options \xoption{break}, \xoption{hyphenbreak}, \xoption{spacebreak}
%     added.
%   \item
%     Variant support added by option \xoption{variant}.
%   \end{Version}
%   \begin{Version}{2010/04/24 v1.2}
%   \item
%     \hologo{LaTeX3} added.
%   \item
%     \hologo{VTeX} added.
%   \end{Version}
%   \begin{Version}{2010/11/21 v1.3}
%   \item
%     \hologo{iniTeX}, \hologo{virTeX} added.
%   \end{Version}
%   \begin{Version}{2011/03/25 v1.4}
%   \item
%     \hologo{ConTeXt} with variants added.
%   \item
%     Option \xoption{discretionarybreak} added as refinement for
%     option \xoption{break}.
%   \end{Version}
%   \begin{Version}{2011/04/21 v1.5}
%   \item
%     Wrong TDS directory for test files fixed.
%   \end{Version}
%   \begin{Version}{2011/10/01 v1.6}
%   \item
%     Support for package \xpackage{tex4ht} added.
%   \item
%     Support for \cs{csname} added if \cs{ifincsname} is available.
%   \item
%     New logos:
%     \hologo{(La)TeX},
%     \hologo{biber},
%     \hologo{BibTeX} (\xoption{sc}, \xoption{sf}),
%     \hologo{emTeX},
%     \hologo{ExTeX},
%     \hologo{KOMAScript},
%     \hologo{La},
%     \hologo{LyX},
%     \hologo{MiKTeX},
%     \hologo{NTS},
%     \hologo{OzMF},
%     \hologo{OzMP},
%     \hologo{OzTeX},
%     \hologo{OzTtH},
%     \hologo{PCTeX},
%     \hologo{PiC},
%     \hologo{PiCTeX},
%     \hologo{METAFONT},
%     \hologo{MetaFun},
%     \hologo{METAPOST},
%     \hologo{MetaPost},
%     \hologo{SLiTeX} (\xoption{lift}, \xoption{narrow}, \xoption{simple}),
%     \hologo{SliTeX} (\xoption{narrow}, \xoption{simple}, \xoption{lift}),
%     \hologo{teTeX}.
%   \item
%     Fixes:
%     \hologo{iniTeX},
%     \hologo{pdfLaTeX},
%     \hologo{pdfTeX},
%     \hologo{virTeX}.
%   \item
%     \cs{hologoFontSetup} and \cs{hologoLogoFontSetup} added.
%   \item
%     \cs{hologoVariant} and \cs{HologoVariant} added.
%   \end{Version}
%   \begin{Version}{2011/11/22 v1.7}
%   \item
%     New logos:
%     \hologo{BibTeX8},
%     \hologo{LaTeXML},
%     \hologo{SageTeX},
%     \hologo{TeX4ht},
%     \hologo{TTH}.
%   \item
%     \hologo{Xe} and friends: Driver stuff fixed.
%   \item
%     \hologo{Xe} and friends: Support for italic added.
%   \item
%     \hologo{Xe} and friends: Package support for \xpackage{pgf}
%     and \xpackage{pstricks} added.
%   \end{Version}
%   \begin{Version}{2011/11/29 v1.8}
%   \item
%     New logos:
%     \hologo{HanTheThanh}.
%   \end{Version}
%   \begin{Version}{2011/12/21 v1.9}
%   \item
%     Patch for package \xpackage{ifxetex} added for the case that
%     \cs{newif} is undefined in \hologo{iniTeX}.
%   \item
%     Some fixes for \hologo{iniTeX}.
%   \end{Version}
%   \begin{Version}{2012/04/26 v1.10}
%   \item
%     Fix in bookmark version of logo ``\hologo{HanTheThanh}''.
%   \end{Version}
%   \begin{Version}{2016/05/12 v1.11}
%   \item
%     Update HOLOGO@IfCharExists (previously in texlive)
%   \item define pdfliteral in current luatex.
%   \end{Version}
% \end{History}
%
% \PrintIndex
%
% \Finale
\endinput

%        (quote the arguments according to the demands of your shell)
%
% Documentation:
%    (a) If hologo.drv is present:
%           latex hologo.drv
%    (b) Without hologo.drv:
%           latex hologo.dtx; ...
%    The class ltxdoc loads the configuration file ltxdoc.cfg
%    if available. Here you can specify further options, e.g.
%    use A4 as paper format:
%       \PassOptionsToClass{a4paper}{article}
%
%    Programm calls to get the documentation (example):
%       pdflatex hologo.dtx
%       makeindex -s gind.ist hologo.idx
%       pdflatex hologo.dtx
%       makeindex -s gind.ist hologo.idx
%       pdflatex hologo.dtx
%
% Installation:
%    TDS:tex/generic/oberdiek/hologo.sty
%    TDS:doc/latex/oberdiek/hologo.pdf
%    TDS:doc/latex/oberdiek/example/hologo-example.tex
%    TDS:doc/latex/oberdiek/test/hologo-test1.tex
%    TDS:doc/latex/oberdiek/test/hologo-test-spacefactor.tex
%    TDS:doc/latex/oberdiek/test/hologo-test-list.tex
%    TDS:source/latex/oberdiek/hologo.dtx
%
%<*ignore>
\begingroup
  \catcode123=1 %
  \catcode125=2 %
  \def\x{LaTeX2e}%
\expandafter\endgroup
\ifcase 0\ifx\install y1\fi\expandafter
         \ifx\csname processbatchFile\endcsname\relax\else1\fi
         \ifx\fmtname\x\else 1\fi\relax
\else\csname fi\endcsname
%</ignore>
%<*install>
\input docstrip.tex
\Msg{************************************************************************}
\Msg{* Installation}
\Msg{* Package: hologo 2016/05/12 v1.11 A logo collection with bookmark support (HO)}
\Msg{************************************************************************}

\keepsilent
\askforoverwritefalse

\let\MetaPrefix\relax
\preamble

This is a generated file.

Project: hologo
Version: 2016/05/12 v1.11

Copyright (C) 2010-2012 by
   Heiko Oberdiek <heiko.oberdiek at googlemail.com>

This work may be distributed and/or modified under the
conditions of the LaTeX Project Public License, either
version 1.3c of this license or (at your option) any later
version. This version of this license is in
   http://www.latex-project.org/lppl/lppl-1-3c.txt
and the latest version of this license is in
   http://www.latex-project.org/lppl.txt
and version 1.3 or later is part of all distributions of
LaTeX version 2005/12/01 or later.

This work has the LPPL maintenance status "maintained".

This Current Maintainer of this work is Heiko Oberdiek.

The Base Interpreter refers to any `TeX-Format',
because some files are installed in TDS:tex/generic//.

This work consists of the main source file hologo.dtx
and the derived files
   hologo.sty, hologo.pdf, hologo.ins, hologo.drv, hologo-example.tex,
   hologo-test1.tex, hologo-test-spacefactor.tex,
   hologo-test-list.tex.

\endpreamble
\let\MetaPrefix\DoubleperCent

\generate{%
  \file{hologo.ins}{\from{hologo.dtx}{install}}%
  \file{hologo.drv}{\from{hologo.dtx}{driver}}%
  \usedir{tex/generic/oberdiek}%
  \file{hologo.sty}{\from{hologo.dtx}{package}}%
  \usedir{doc/latex/oberdiek/example}%
  \file{hologo-example.tex}{\from{hologo.dtx}{example}}%
  \usedir{doc/latex/oberdiek/test}%
  \file{hologo-test1.tex}{\from{hologo.dtx}{test1}}%
  \file{hologo-test-spacefactor.tex}{\from{hologo.dtx}{test-spacefactor}}%
  \file{hologo-test-list.tex}{\from{hologo.dtx}{test-list}}%
  \nopreamble
  \nopostamble
  \usedir{source/latex/oberdiek/catalogue}%
  \file{hologo.xml}{\from{hologo.dtx}{catalogue}}%
}

\catcode32=13\relax% active space
\let =\space%
\Msg{************************************************************************}
\Msg{*}
\Msg{* To finish the installation you have to move the following}
\Msg{* file into a directory searched by TeX:}
\Msg{*}
\Msg{*     hologo.sty}
\Msg{*}
\Msg{* To produce the documentation run the file `hologo.drv'}
\Msg{* through LaTeX.}
\Msg{*}
\Msg{* Happy TeXing!}
\Msg{*}
\Msg{************************************************************************}

\endbatchfile
%</install>
%<*ignore>
\fi
%</ignore>
%<*driver>
\NeedsTeXFormat{LaTeX2e}
\ProvidesFile{hologo.drv}%
  [2016/05/12 v1.11 A logo collection with bookmark support (HO)]%
\documentclass{ltxdoc}
\usepackage{holtxdoc}[2011/11/22]
\usepackage{hologo}[2016/05/12]
\usepackage{longtable}
\usepackage{array}
\usepackage{paralist}
%\usepackage[T1]{fontenc}
%\usepackage{lmodern}
\begin{document}
  \DocInput{hologo.dtx}%
\end{document}
%</driver>
% \fi
%
%
% \CharacterTable
%  {Upper-case    \A\B\C\D\E\F\G\H\I\J\K\L\M\N\O\P\Q\R\S\T\U\V\W\X\Y\Z
%   Lower-case    \a\b\c\d\e\f\g\h\i\j\k\l\m\n\o\p\q\r\s\t\u\v\w\x\y\z
%   Digits        \0\1\2\3\4\5\6\7\8\9
%   Exclamation   \!     Double quote  \"     Hash (number) \#
%   Dollar        \$     Percent       \%     Ampersand     \&
%   Acute accent  \'     Left paren    \(     Right paren   \)
%   Asterisk      \*     Plus          \+     Comma         \,
%   Minus         \-     Point         \.     Solidus       \/
%   Colon         \:     Semicolon     \;     Less than     \<
%   Equals        \=     Greater than  \>     Question mark \?
%   Commercial at \@     Left bracket  \[     Backslash     \\
%   Right bracket \]     Circumflex    \^     Underscore    \_
%   Grave accent  \`     Left brace    \{     Vertical bar  \|
%   Right brace   \}     Tilde         \~}
%
% \GetFileInfo{hologo.drv}
%
% \title{The \xpackage{hologo} package}
% \date{2016/05/12 v1.11}
% \author{Heiko Oberdiek\\\xemail{heiko.oberdiek at googlemail.com}}
%
% \maketitle
%
% \begin{abstract}
% This package starts a collection of logos with support for bookmarks
% strings.
% \end{abstract}
%
% \tableofcontents
%
% \section{Documentation}
%
% \subsection{Logo macros}
%
% \begin{declcs}{hologo} \M{name}
% \end{declcs}
% Macro \cs{hologo} sets the logo with name \meta{name}.
% The following table shows the supported names.
%
% \begingroup
%   \def\hologoEntry#1#2#3{^^A
%     #1&#2&\hologoLogoSetup{#1}{variant=#2}\hologo{#1}&#3\tabularnewline
%   }
%   \begin{longtable}{>{\ttfamily}l>{\ttfamily}lll}
%     \rmfamily\bfseries{name} & \rmfamily\bfseries variant
%     & \bfseries logo & \bfseries since\\
%     \hline
%     \endhead
%     \hologoList
%   \end{longtable}
% \endgroup
%
% \begin{declcs}{Hologo} \M{name}
% \end{declcs}
% Macro \cs{Hologo} starts the logo \meta{name} with an uppercase
% letter. As an exception small greek letters are not converted
% to uppercase. Examples, see \hologo{eTeX} and \hologo{ExTeX}.
%
% \subsection{Setup macros}
%
% The package does not support package options, but the following
% setup macros can be used to set options.
%
% \begin{declcs}{hologoSetup} \M{key value list}
% \end{declcs}
% Macro \cs{hologoSetup} sets global options.
%
% \begin{declcs}{hologoLogoSetup} \M{logo} \M{key value list}
% \end{declcs}
% Some options can also be used to configure a logo.
% These settings take precedence over global option settings.
%
% \subsection{Options}\label{sec:options}
%
% There are boolean and string options:
% \begin{description}
% \item[Boolean option:]
% It takes |true| or |false|
% as value. If the value is omitted, then |true| is used.
% \item[String option:]
% A value must be given as string. (But the string might be empty.)
% \end{description}
% The following options can be used both in \cs{hologoSetup}
% and \cs{hologoLogoSetup}:
% \begin{description}
% \def\entry#1{\item[\xoption{#1}:]}
% \entry{break}
%   enables or disables line breaks inside the logo. This setting is
%   refined by options \xoption{hyphenbreak}, \xoption{spacebreak}
%   or \xoption{discretionarybreak}.
%   Default is |false|.
% \entry{hyphenbreak}
%   enables or disables the line break right after the hyphen character.
% \entry{spacebreak}
%   enables or disables line breaks at space characters.
% \entry{discretionarybreak}
%   enables or disables line breaks at hyphenation points
%   (inserted by \cs{-}).
% \end{description}
% Macro \cs{hologoLogoSetup} also knows:
% \begin{description}
% \item[\xoption{variant}:]
%   This is a string option. It specifies a variant of a logo that
%   must exist. An empty string selects the package default variant.
% \end{description}
% Example:
% \begin{quote}
%   |\hologoSetup{break=false}|\\
%   |\hologoLogoSetup{plainTeX}{variant=hyphen,hyphenbreak}|\\
%   Then ``plain-\TeX'' contains one break point after the hyphen.
% \end{quote}
%
% \subsection{Driver options}
%
% Sometimes graphical operations are needed to construct some
% glyphs (e.g.\ \hologo{XeTeX}). If package \xpackage{graphics}
% or package \xpackage{pgf} are found, then the macros are taken
% from there. Otherwise the packge defines its own operations
% and therefore needs the driver information. Many drivers are
% detected automatically (\hologo{pdfTeX}/\hologo{LuaTeX}
% in PDF mode, \hologo{XeTeX}, \hologo{VTeX}). These have precedence
% over a driver option. The driver can be given as package option
% or using \cs{hologoDriverSetup}.
% The following list contains the recognized driver options:
% \begin{itemize}
% \item \xoption{pdftex}, \xoption{luatex}
% \item \xoption{dvipdfm}, \xoption{dvipdfmx}
% \item \xoption{dvips}, \xoption{dvipsone}, \xoption{xdvi}
% \item \xoption{xetex}
% \item \xoption{vtex}
% \end{itemize}
% The left driver of a line is the driver name that is used internally.
% The following names are aliases for drivers that use the
% same method. Therefore the entry in the \xext{log} file for
% the used driver prints the internally used driver name.
% \begin{description}
% \item[\xoption{driverfallback}:]
%   This option expects a driver that is used,
%   if the driver could not be detected automatically.
% \end{description}
%
% \begin{declcs}{hologoDriverSetup} \M{driver option}
% \end{declcs}
% The driver can also be configured after package loading
% using \cs{hologoDriverSetup}, also the way for \hologo{plainTeX}
% to setup the driver.
%
% \subsection{Font setup}
%
% Some logos require a special font, but should also be usable by
% \hologo{plainTeX}. Therefore the package provides some ways
% to influence the font settings. The options below
% take font settings as values. Both font commands
% such as \cs{sffamily} and macros that take one argument
% like \cs{textsf} can be used.
%
% \begin{declcs}{hologoFontSetup} \M{key value list}
% \end{declcs}
% Macro \cs{hologoFontSetup} sets the fonts for all logos.
% Supported keys:
% \begin{description}
% \def\entry#1{\item[\xoption{#1}:]}
% \entry{general}
%   This font is used for all logos. The default is empty.
%   That means no special font is used.
% \entry{bibsf}
%   This font is used for
%   {\hologoLogoSetup{BibTeX}{variant=sf}\hologo{BibTeX}}
%   with variant \xoption{sf}.
% \entry{rm}
%   This font is a serif font. It is used for \hologo{ExTeX}.
% \entry{sc}
%   This font specifies a small caps font. It is used for
%   {\hologoLogoSetup{BibTeX}{variant=sc}\hologo{BibTeX}}
%   with variant \xoption{sc}.
% \entry{sf}
%   This font specifies a sans serif font. The default
%   is \cs{sffamily}, then \cs{sf} is tried. Otherwise
%   a warning is given. It is used by \hologo{KOMAScript}.
% \entry{sy}
%   This is the font for math symbols (e.g. cmsy).
%   It is used by \hologo{AmS}, \hologo{NTS}, \hologo{ExTeX}.
% \entry{logo}
%   \hologo{METAFONT} and \hologo{METAPOST} are using that font.
%   In \hologo{LaTeX} \cs{logofamily} is used and
%   the definitions of package \xpackage{mflogo} are used
%   if the package is not loaded.
%   Otherwise the \cs{tenlogo} is used and defined
%   if it does not already exists.
% \end{description}
%
% \begin{declcs}{hologoLogoFontSetup} \M{logo} \M{key value list}
% \end{declcs}
% Fonts can also be set for a logo or logo component separately,
% see the following list.
% The keys are the same as for \cs{hologoFontSetup}.
%
% \begin{longtable}{>{\ttfamily}l>{\sffamily}ll}
%   \meta{logo} & keys & result\\
%   \hline
%   \endhead
%   BibTeX & bibsf & {\hologoLogoSetup{BibTeX}{variant=sf}\hologo{BibTeX}}\\[.5ex]
%   BibTeX & sc & {\hologoLogoSetup{BibTeX}{variant=sc}\hologo{BibTeX}}\\[.5ex]
%   ExTeX & rm & \hologo{ExTeX}\\
%   SliTeX & rm & \hologo{SliTeX}\\[.5ex]
%   AmS & sy & \hologo{AmS}\\
%   ExTeX & sy & \hologo{ExTeX}\\
%   NTS & sy & \hologo{NTS}\\[.5ex]
%   KOMAScript & sf & \hologo{KOMAScript}\\[.5ex]
%   METAFONT & logo & \hologo{METAFONT}\\
%   METAPOST & logo & \hologo{METAPOST}\\[.5ex]
%   SliTeX & sc \hologo{SliTeX}
% \end{longtable}
%
% \subsubsection{Font order}
%
% For all logos the font \xoption{general} is applied first.
% Example:
%\begin{quote}
%|\hologoFontSetup{general=\color{red}}|
%\end{quote}
% will print red logos.
% Then if the font uses a special font \xoption{sf}, for example,
% the font is applied that is setup by \cs{hologoLogoFontSetup}.
% If this font is not setup, then the common font setup
% by \cs{hologoFontSetup} is used. Otherwise a warning is given,
% that there is no font configured.
%
% \subsection{Additional user macros}
%
% Usually a variant of a logo is configured by using
% \cs{hologoLogoSetup}, because it is bad style to mix
% different variants of the same logo in the same text.
% There the following macros are a convenience for testing.
%
% \begin{declcs}{hologoVariant} \M{name} \M{variant}\\
%   \cs{HologoVariant} \M{name} \M{variant}
% \end{declcs}
% Logo \meta{name} is set using \meta{variant} that specifies
% explicitely which variant of the macro is used. If the argument
% is empty, then the default form of the logo is used
% (configurable by \cs{hologoLogoSetup}).
%
% \cs{HologoVariant} is used if the logo is set in a context
% that needs an uppercase first letter (beginning of a sentence, \dots).
%
% \begin{declcs}{hologoList}\\
%   \cs{hologoEntry} \M{logo} \M{variant} \M{since}
% \end{declcs}
% Macro \cs{hologoList} contains all logos that are provided
% by the package including variants. The list consists of calls
% of \cs{hologoEntry} with three arguments starting with the
% logo name \meta{logo} and its variant \meta{variant}. An empty
% variant means the current default. Argument \meta{since} specifies
% with version of the package \xpackage{hologo} is needed to get
% the logo. If the logo is fixed, then the date gets updated.
% Therefore the date \meta{since} is not exactly the date of
% the first introduction, but rather the date of the latest fix.
%
% Before \cs{hologoList} can be used, macro \cs{hologoEntry} needs
% a definition. The example file in section \ref{sec:example}
% shows applications of \cs{hologoList}.
%
% \subsection{Supported contexts}
%
% Macros \cs{hologo} and friends support special contexts:
% \begin{itemize}
% \item \hologo{LaTeX}'s protection mechanism.
% \item Bookmarks of package \xpackage{hyperref}.
% \item Package \xpackage{tex4ht}.
% \item The macros can be used inside \cs{csname} constructs,
%   if \cs{ifincsname} is available (\hologo{pdfTeX}, \hologo{XeTeX},
%   \hologo{LuaTeX}).
% \end{itemize}
%
% \subsection{Example}
% \label{sec:example}
%
% The following example prints the logos in different fonts.
%    \begin{macrocode}
%<*example>
%<<verbatim
\NeedsTeXFormat{LaTeX2e}
\documentclass[a4paper]{article}
\usepackage[
  hmargin=20mm,
  vmargin=20mm,
]{geometry}
\pagestyle{empty}
\usepackage{hologo}[2016/05/12]
\usepackage{longtable}
\usepackage{array}
\setlength{\extrarowheight}{2pt}
\usepackage[T1]{fontenc}
\usepackage{lmodern}
\usepackage{pdflscape}
\usepackage[
  pdfencoding=auto,
]{hyperref}
\hypersetup{
  pdfauthor={Heiko Oberdiek},
  pdftitle={Example for package `hologo'},
  pdfsubject={Logos with fonts lmr, lmss, qtm, qpl, qhv},
}
\usepackage{bookmark}

% Print the logo list on the console

\begingroup
  \typeout{}%
  \typeout{*** Begin of logo list ***}%
  \newcommand*{\hologoEntry}[3]{%
    \typeout{#1 \ifx\\#2\\\else(#2) \fi[#3]}%
  }%
  \hologoList
  \typeout{*** End of logo list ***}%
  \typeout{}%
\endgroup

\begin{document}
\begin{landscape}

  \section{Example file for package `hologo'}

  % Table for font names

  \begin{longtable}{>{\bfseries}ll}
    \textbf{font} & \textbf{Font name}\\
    \hline
    lmr & Latin Modern Roman\\
    lmss & Latin Modern Sans\\
    qtm & \TeX\ Gyre Termes\\
    qhv & \TeX\ Gyre Heros\\
    qpl & \TeX\ Gyre Pagella\\
  \end{longtable}

  % Logo list with logos in different fonts

  \begingroup
    \newcommand*{\SetVariant}[2]{%
      \ifx\\#2\\%
      \else
        \hologoLogoSetup{#1}{variant=#2}%
      \fi
    }%
    \newcommand*{\hologoEntry}[3]{%
      \SetVariant{#1}{#2}%
      \raisebox{1em}[0pt][0pt]{\hypertarget{#1@#2}{}}%
      \bookmark[%
        dest={#1@#2},%
      ]{%
        #1\ifx\\#2\\\else\space(#2)\fi: \Hologo{#1}, \hologo{#1} %
        [Unicode]%
      }%
      \hypersetup{unicode=false}%
      \bookmark[%
        dest={#1@#2},%
      ]{%
        #1\ifx\\#2\\\else\space(#2)\fi: \Hologo{#1}, \hologo{#1} %
        [PDFDocEncoding]%
      }%
      \texttt{#1}%
      &%
      \texttt{#2}%
      &%
      \Hologo{#1}%
      &%
      \SetVariant{#1}{#2}%
      \hologo{#1}%
      &%
      \SetVariant{#1}{#2}%
      \fontfamily{qtm}\selectfont
      \hologo{#1}%
      &%
      \SetVariant{#1}{#2}%
      \fontfamily{qpl}\selectfont
      \hologo{#1}%
      &%
      \SetVariant{#1}{#2}%
      \textsf{\hologo{#1}}%
      &%
      \SetVariant{#1}{#2}%
      \fontfamily{qhv}\selectfont
      \hologo{#1}%
      \tabularnewline
    }%
    \begin{longtable}{llllllll}%
      \textbf{\textit{logo}} & \textbf{\textit{variant}} &
      \texttt{\string\Hologo} &
      \textbf{lmr} & \textbf{qtm} & \textbf{qpl} &
      \textbf{lmss} & \textbf{qhv}
      \tabularnewline
      \hline
      \endhead
      \hologoList
    \end{longtable}%
  \endgroup

\end{landscape}
\end{document}
%verbatim
%</example>
%    \end{macrocode}
%
% \StopEventually{
% }
%
% \section{Implementation}
%    \begin{macrocode}
%<*package>
%    \end{macrocode}
%    Reload check, especially if the package is not used with \LaTeX.
%    \begin{macrocode}
\begingroup\catcode61\catcode48\catcode32=10\relax%
  \catcode13=5 % ^^M
  \endlinechar=13 %
  \catcode35=6 % #
  \catcode39=12 % '
  \catcode44=12 % ,
  \catcode45=12 % -
  \catcode46=12 % .
  \catcode58=12 % :
  \catcode64=11 % @
  \catcode123=1 % {
  \catcode125=2 % }
  \expandafter\let\expandafter\x\csname ver@hologo.sty\endcsname
  \ifx\x\relax % plain-TeX, first loading
  \else
    \def\empty{}%
    \ifx\x\empty % LaTeX, first loading,
      % variable is initialized, but \ProvidesPackage not yet seen
    \else
      \expandafter\ifx\csname PackageInfo\endcsname\relax
        \def\x#1#2{%
          \immediate\write-1{Package #1 Info: #2.}%
        }%
      \else
        \def\x#1#2{\PackageInfo{#1}{#2, stopped}}%
      \fi
      \x{hologo}{The package is already loaded}%
      \aftergroup\endinput
    \fi
  \fi
\endgroup%
%    \end{macrocode}
%    Package identification:
%    \begin{macrocode}
\begingroup\catcode61\catcode48\catcode32=10\relax%
  \catcode13=5 % ^^M
  \endlinechar=13 %
  \catcode35=6 % #
  \catcode39=12 % '
  \catcode40=12 % (
  \catcode41=12 % )
  \catcode44=12 % ,
  \catcode45=12 % -
  \catcode46=12 % .
  \catcode47=12 % /
  \catcode58=12 % :
  \catcode64=11 % @
  \catcode91=12 % [
  \catcode93=12 % ]
  \catcode123=1 % {
  \catcode125=2 % }
  \expandafter\ifx\csname ProvidesPackage\endcsname\relax
    \def\x#1#2#3[#4]{\endgroup
      \immediate\write-1{Package: #3 #4}%
      \xdef#1{#4}%
    }%
  \else
    \def\x#1#2[#3]{\endgroup
      #2[{#3}]%
      \ifx#1\@undefined
        \xdef#1{#3}%
      \fi
      \ifx#1\relax
        \xdef#1{#3}%
      \fi
    }%
  \fi
\expandafter\x\csname ver@hologo.sty\endcsname
\ProvidesPackage{hologo}%
  [2016/05/12 v1.11 A logo collection with bookmark support (HO)]%
%    \end{macrocode}
%
%    \begin{macrocode}
\begingroup\catcode61\catcode48\catcode32=10\relax%
  \catcode13=5 % ^^M
  \endlinechar=13 %
  \catcode123=1 % {
  \catcode125=2 % }
  \catcode64=11 % @
  \def\x{\endgroup
    \expandafter\edef\csname HOLOGO@AtEnd\endcsname{%
      \endlinechar=\the\endlinechar\relax
      \catcode13=\the\catcode13\relax
      \catcode32=\the\catcode32\relax
      \catcode35=\the\catcode35\relax
      \catcode61=\the\catcode61\relax
      \catcode64=\the\catcode64\relax
      \catcode123=\the\catcode123\relax
      \catcode125=\the\catcode125\relax
    }%
  }%
\x\catcode61\catcode48\catcode32=10\relax%
\catcode13=5 % ^^M
\endlinechar=13 %
\catcode35=6 % #
\catcode64=11 % @
\catcode123=1 % {
\catcode125=2 % }
\def\TMP@EnsureCode#1#2{%
  \edef\HOLOGO@AtEnd{%
    \HOLOGO@AtEnd
    \catcode#1=\the\catcode#1\relax
  }%
  \catcode#1=#2\relax
}
\TMP@EnsureCode{10}{12}% ^^J
\TMP@EnsureCode{33}{12}% !
\TMP@EnsureCode{34}{12}% "
\TMP@EnsureCode{36}{3}% $
\TMP@EnsureCode{38}{4}% &
\TMP@EnsureCode{39}{12}% '
\TMP@EnsureCode{40}{12}% (
\TMP@EnsureCode{41}{12}% )
\TMP@EnsureCode{42}{12}% *
\TMP@EnsureCode{43}{12}% +
\TMP@EnsureCode{44}{12}% ,
\TMP@EnsureCode{45}{12}% -
\TMP@EnsureCode{46}{12}% .
\TMP@EnsureCode{47}{12}% /
\TMP@EnsureCode{58}{12}% :
\TMP@EnsureCode{59}{12}% ;
\TMP@EnsureCode{60}{12}% <
\TMP@EnsureCode{62}{12}% >
\TMP@EnsureCode{63}{12}% ?
\TMP@EnsureCode{91}{12}% [
\TMP@EnsureCode{93}{12}% ]
\TMP@EnsureCode{94}{7}% ^ (superscript)
\TMP@EnsureCode{95}{8}% _ (subscript)
\TMP@EnsureCode{96}{12}% `
\TMP@EnsureCode{124}{12}% |
\edef\HOLOGO@AtEnd{%
  \HOLOGO@AtEnd
  \escapechar\the\escapechar\relax
  \noexpand\endinput
}
\escapechar=92 %
%    \end{macrocode}
%
% \subsection{Logo list}
%
%    \begin{macro}{\hologoList}
%    \begin{macrocode}
\def\hologoList{%
  \hologoEntry{(La)TeX}{}{2011/10/01}%
  \hologoEntry{AmSLaTeX}{}{2010/04/16}%
  \hologoEntry{AmSTeX}{}{2010/04/16}%
  \hologoEntry{biber}{}{2011/10/01}%
  \hologoEntry{BibTeX}{}{2011/10/01}%
  \hologoEntry{BibTeX}{sf}{2011/10/01}%
  \hologoEntry{BibTeX}{sc}{2011/10/01}%
  \hologoEntry{BibTeX8}{}{2011/11/22}%
  \hologoEntry{ConTeXt}{}{2011/03/25}%
  \hologoEntry{ConTeXt}{narrow}{2011/03/25}%
  \hologoEntry{ConTeXt}{simple}{2011/03/25}%
  \hologoEntry{emTeX}{}{2010/04/26}%
  \hologoEntry{eTeX}{}{2010/04/08}%
  \hologoEntry{ExTeX}{}{2011/10/01}%
  \hologoEntry{HanTheThanh}{}{2011/11/29}%
  \hologoEntry{iniTeX}{}{2011/10/01}%
  \hologoEntry{KOMAScript}{}{2011/10/01}%
  \hologoEntry{La}{}{2010/05/08}%
  \hologoEntry{LaTeX}{}{2010/04/08}%
  \hologoEntry{LaTeX2e}{}{2010/04/08}%
  \hologoEntry{LaTeX3}{}{2010/04/24}%
  \hologoEntry{LaTeXe}{}{2010/04/08}%
  \hologoEntry{LaTeXML}{}{2011/11/22}%
  \hologoEntry{LaTeXTeX}{}{2011/10/01}%
  \hologoEntry{LuaLaTeX}{}{2010/04/08}%
  \hologoEntry{LuaTeX}{}{2010/04/08}%
  \hologoEntry{LyX}{}{2011/10/01}%
  \hologoEntry{METAFONT}{}{2011/10/01}%
  \hologoEntry{MetaFun}{}{2011/10/01}%
  \hologoEntry{METAPOST}{}{2011/10/01}%
  \hologoEntry{MetaPost}{}{2011/10/01}%
  \hologoEntry{MiKTeX}{}{2011/10/01}%
  \hologoEntry{NTS}{}{2011/10/01}%
  \hologoEntry{OzMF}{}{2011/10/01}%
  \hologoEntry{OzMP}{}{2011/10/01}%
  \hologoEntry{OzTeX}{}{2011/10/01}%
  \hologoEntry{OzTtH}{}{2011/10/01}%
  \hologoEntry{PCTeX}{}{2011/10/01}%
  \hologoEntry{pdfTeX}{}{2011/10/01}%
  \hologoEntry{pdfLaTeX}{}{2011/10/01}%
  \hologoEntry{PiC}{}{2011/10/01}%
  \hologoEntry{PiCTeX}{}{2011/10/01}%
  \hologoEntry{plainTeX}{}{2010/04/08}%
  \hologoEntry{plainTeX}{space}{2010/04/16}%
  \hologoEntry{plainTeX}{hyphen}{2010/04/16}%
  \hologoEntry{plainTeX}{runtogether}{2010/04/16}%
  \hologoEntry{SageTeX}{}{2011/11/22}%
  \hologoEntry{SLiTeX}{}{2011/10/01}%
  \hologoEntry{SLiTeX}{lift}{2011/10/01}%
  \hologoEntry{SLiTeX}{narrow}{2011/10/01}%
  \hologoEntry{SLiTeX}{simple}{2011/10/01}%
  \hologoEntry{SliTeX}{}{2011/10/01}%
  \hologoEntry{SliTeX}{narrow}{2011/10/01}%
  \hologoEntry{SliTeX}{simple}{2011/10/01}%
  \hologoEntry{SliTeX}{lift}{2011/10/01}%
  \hologoEntry{teTeX}{}{2011/10/01}%
  \hologoEntry{TeX}{}{2010/04/08}%
  \hologoEntry{TeX4ht}{}{2011/11/22}%
  \hologoEntry{TTH}{}{2011/11/22}%
  \hologoEntry{virTeX}{}{2011/10/01}%
  \hologoEntry{VTeX}{}{2010/04/24}%
  \hologoEntry{Xe}{}{2010/04/08}%
  \hologoEntry{XeLaTeX}{}{2010/04/08}%
  \hologoEntry{XeTeX}{}{2010/04/08}%
}
%    \end{macrocode}
%    \end{macro}
%
% \subsection{Load resources}
%
%    \begin{macrocode}
\begingroup\expandafter\expandafter\expandafter\endgroup
\expandafter\ifx\csname RequirePackage\endcsname\relax
  \def\TMP@RequirePackage#1[#2]{%
    \begingroup\expandafter\expandafter\expandafter\endgroup
    \expandafter\ifx\csname ver@#1.sty\endcsname\relax
      \input #1.sty\relax
    \fi
  }%
  \TMP@RequirePackage{ltxcmds}[2011/02/04]%
  \TMP@RequirePackage{infwarerr}[2010/04/08]%
  \TMP@RequirePackage{kvsetkeys}[2010/03/01]%
  \TMP@RequirePackage{kvdefinekeys}[2010/03/01]%
  \TMP@RequirePackage{pdftexcmds}[2010/04/01]%
  \TMP@RequirePackage{ifpdf}[2010/01/28]%
  \TMP@RequirePackage{ifluatex}[2010/03/01]%
  \ltx@IfUndefined{newif}{%
    \expandafter\let\csname newif\endcsname\ltx@newif
  }{}%
  \TMP@RequirePackage{ifxetex}[2009/01/23]%
  \TMP@RequirePackage{ifvtex}[2010/03/01]%
\else
  \RequirePackage{ltxcmds}[2011/02/04]%
  \RequirePackage{infwarerr}[2010/04/08]%
  \RequirePackage{kvsetkeys}[2010/03/01]%
  \RequirePackage{kvdefinekeys}[2010/03/01]%
  \RequirePackage{pdftexcmds}[2010/04/01]%
  \RequirePackage{ifpdf}[2010/01/28]%
  \RequirePackage{ifluatex}[2010/03/01]%
  \RequirePackage{ifxetex}[2009/01/23]%
  \RequirePackage{ifvtex}[2010/03/01]%
\fi
%    \end{macrocode}
%
%    \begin{macro}{\HOLOGO@IfDefined}
%    \begin{macrocode}
\def\HOLOGO@IfExists#1{%
  \ifx\@undefined#1%
    \expandafter\ltx@secondoftwo
  \else
    \ifx\relax#1%
      \expandafter\ltx@secondoftwo
    \else
      \expandafter\expandafter\expandafter\ltx@firstoftwo
    \fi
  \fi
}
%    \end{macrocode}
%    \end{macro}
%
% \subsection{Setup macros}
%
%    \begin{macro}{\hologoSetup}
%    \begin{macrocode}
\def\hologoSetup{%
  \let\HOLOGO@name\relax
  \HOLOGO@Setup
}
%    \end{macrocode}
%    \end{macro}
%
%    \begin{macro}{\hologoLogoSetup}
%    \begin{macrocode}
\def\hologoLogoSetup#1{%
  \edef\HOLOGO@name{#1}%
  \ltx@IfUndefined{HoLogo@\HOLOGO@name}{%
    \@PackageError{hologo}{%
      Unknown logo `\HOLOGO@name'%
    }\@ehc
    \ltx@gobble
  }{%
    \HOLOGO@Setup
  }%
}
%    \end{macrocode}
%    \end{macro}
%
%    \begin{macro}{\HOLOGO@Setup}
%    \begin{macrocode}
\def\HOLOGO@Setup{%
  \kvsetkeys{HoLogo}%
}
%    \end{macrocode}
%    \end{macro}
%
% \subsection{Options}
%
%    \begin{macro}{\HOLOGO@DeclareBoolOption}
%    \begin{macrocode}
\def\HOLOGO@DeclareBoolOption#1{%
  \expandafter\chardef\csname HOLOGOOPT@#1\endcsname\ltx@zero
  \kv@define@key{HoLogo}{#1}[true]{%
    \def\HOLOGO@temp{##1}%
    \ifx\HOLOGO@temp\HOLOGO@true
      \ifx\HOLOGO@name\relax
        \expandafter\chardef\csname HOLOGOOPT@#1\endcsname=\ltx@one
      \else
        \expandafter\chardef\csname
        HoLogoOpt@#1@\HOLOGO@name\endcsname\ltx@one
      \fi
      \HOLOGO@SetBreakAll{#1}%
    \else
      \ifx\HOLOGO@temp\HOLOGO@false
        \ifx\HOLOGO@name\relax
          \expandafter\chardef\csname HOLOGOOPT@#1\endcsname=\ltx@zero
        \else
          \expandafter\chardef\csname
          HoLogoOpt@#1@\HOLOGO@name\endcsname=\ltx@zero
        \fi
        \HOLOGO@SetBreakAll{#1}%
      \else
        \@PackageError{hologo}{%
          Unknown value `##1' for boolean option `#1'.\MessageBreak
          Known values are `true' and `false'%
        }\@ehc
      \fi
    \fi
  }%
}
%    \end{macrocode}
%    \end{macro}
%
%    \begin{macro}{\HOLOGO@SetBreakAll}
%    \begin{macrocode}
\def\HOLOGO@SetBreakAll#1{%
  \def\HOLOGO@temp{#1}%
  \ifx\HOLOGO@temp\HOLOGO@break
    \ifx\HOLOGO@name\relax
      \chardef\HOLOGOOPT@hyphenbreak=\HOLOGOOPT@break
      \chardef\HOLOGOOPT@spacebreak=\HOLOGOOPT@break
      \chardef\HOLOGOOPT@discretionarybreak=\HOLOGOOPT@break
    \else
      \expandafter\chardef
         \csname HoLogoOpt@hyphenbreak@\HOLOGO@name\endcsname=%
         \csname HoLogoOpt@break@\HOLOGO@name\endcsname
      \expandafter\chardef
         \csname HoLogoOpt@spacebreak@\HOLOGO@name\endcsname=%
         \csname HoLogoOpt@break@\HOLOGO@name\endcsname
      \expandafter\chardef
         \csname HoLogoOpt@discretionarybreak@\HOLOGO@name
             \endcsname=%
         \csname HoLogoOpt@break@\HOLOGO@name\endcsname
    \fi
  \fi
}
%    \end{macrocode}
%    \end{macro}
%
%    \begin{macro}{\HOLOGO@true}
%    \begin{macrocode}
\def\HOLOGO@true{true}
%    \end{macrocode}
%    \end{macro}
%    \begin{macro}{\HOLOGO@false}
%    \begin{macrocode}
\def\HOLOGO@false{false}
%    \end{macrocode}
%    \end{macro}
%    \begin{macro}{\HOLOGO@break}
%    \begin{macrocode}
\def\HOLOGO@break{break}
%    \end{macrocode}
%    \end{macro}
%
%    \begin{macrocode}
\HOLOGO@DeclareBoolOption{break}
\HOLOGO@DeclareBoolOption{hyphenbreak}
\HOLOGO@DeclareBoolOption{spacebreak}
\HOLOGO@DeclareBoolOption{discretionarybreak}
%    \end{macrocode}
%
%    \begin{macrocode}
\kv@define@key{HoLogo}{variant}{%
  \ifx\HOLOGO@name\relax
    \@PackageError{hologo}{%
      Option `variant' is not available in \string\hologoSetup,%
      \MessageBreak
      Use \string\hologoLogoSetup\space instead%
    }\@ehc
  \else
    \edef\HOLOGO@temp{#1}%
    \ifx\HOLOGO@temp\ltx@empty
      \expandafter
      \let\csname HoLogoOpt@variant@\HOLOGO@name\endcsname\@undefined
    \else
      \ltx@IfUndefined{HoLogo@\HOLOGO@name @\HOLOGO@temp}{%
        \@PackageError{hologo}{%
          Unknown variant `\HOLOGO@temp' of logo `\HOLOGO@name'%
        }\@ehc
      }{%
        \expandafter
        \let\csname HoLogoOpt@variant@\HOLOGO@name\endcsname
            \HOLOGO@temp
      }%
    \fi
  \fi
}
%    \end{macrocode}
%
%    \begin{macro}{\HOLOGO@Variant}
%    \begin{macrocode}
\def\HOLOGO@Variant#1{%
  #1%
  \ltx@ifundefined{HoLogoOpt@variant@#1}{%
  }{%
    @\csname HoLogoOpt@variant@#1\endcsname
  }%
}
%    \end{macrocode}
%    \end{macro}
%
% \subsection{Break/no-break support}
%
%    \begin{macro}{\HOLOGO@space}
%    \begin{macrocode}
\def\HOLOGO@space{%
  \ltx@ifundefined{HoLogoOpt@spacebreak@\HOLOGO@name}{%
    \ltx@ifundefined{HoLogoOpt@break@\HOLOGO@name}{%
      \chardef\HOLOGO@temp=\HOLOGOOPT@spacebreak
    }{%
      \chardef\HOLOGO@temp=%
        \csname HoLogoOpt@break@\HOLOGO@name\endcsname
    }%
  }{%
    \chardef\HOLOGO@temp=%
      \csname HoLogoOpt@spacebreak@\HOLOGO@name\endcsname
  }%
  \ifcase\HOLOGO@temp
    \penalty10000 %
  \fi
  \ltx@space
}
%    \end{macrocode}
%    \end{macro}
%
%    \begin{macro}{\HOLOGO@hyphen}
%    \begin{macrocode}
\def\HOLOGO@hyphen{%
  \ltx@ifundefined{HoLogoOpt@hyphenbreak@\HOLOGO@name}{%
    \ltx@ifundefined{HoLogoOpt@break@\HOLOGO@name}{%
      \chardef\HOLOGO@temp=\HOLOGOOPT@hyphenbreak
    }{%
      \chardef\HOLOGO@temp=%
        \csname HoLogoOpt@break@\HOLOGO@name\endcsname
    }%
  }{%
    \chardef\HOLOGO@temp=%
      \csname HoLogoOpt@hyphenbreak@\HOLOGO@name\endcsname
  }%
  \ifcase\HOLOGO@temp
    \ltx@mbox{-}%
  \else
    -%
  \fi
}
%    \end{macrocode}
%    \end{macro}
%
%    \begin{macro}{\HOLOGO@discretionary}
%    \begin{macrocode}
\def\HOLOGO@discretionary{%
  \ltx@ifundefined{HoLogoOpt@discretionarybreak@\HOLOGO@name}{%
    \ltx@ifundefined{HoLogoOpt@break@\HOLOGO@name}{%
      \chardef\HOLOGO@temp=\HOLOGOOPT@discretionarybreak
    }{%
      \chardef\HOLOGO@temp=%
        \csname HoLogoOpt@break@\HOLOGO@name\endcsname
    }%
  }{%
    \chardef\HOLOGO@temp=%
      \csname HoLogoOpt@discretionarybreak@\HOLOGO@name\endcsname
  }%
  \ifcase\HOLOGO@temp
  \else
    \-%
  \fi
}
%    \end{macrocode}
%    \end{macro}
%
%    \begin{macro}{\HOLOGO@mbox}
%    \begin{macrocode}
\def\HOLOGO@mbox#1{%
  \ltx@ifundefined{HoLogoOpt@break@\HOLOGO@name}{%
    \chardef\HOLOGO@temp=\HOLOGOOPT@hyphenbreak
  }{%
    \chardef\HOLOGO@temp=%
      \csname HoLogoOpt@break@\HOLOGO@name\endcsname
  }%
  \ifcase\HOLOGO@temp
    \ltx@mbox{#1}%
  \else
    #1%
  \fi
}
%    \end{macrocode}
%    \end{macro}
%
% \subsection{Font support}
%
%    \begin{macro}{\HoLogoFont@font}
%    \begin{tabular}{@{}ll@{}}
%    |#1|:& logo name\\
%    |#2|:& font short name\\
%    |#3|:& text
%    \end{tabular}
%    \begin{macrocode}
\def\HoLogoFont@font#1#2#3{%
  \begingroup
    \ltx@IfUndefined{HoLogoFont@logo@#1.#2}{%
      \ltx@IfUndefined{HoLogoFont@font@#2}{%
        \@PackageWarning{hologo}{%
          Missing font `#2' for logo `#1'%
        }%
        #3%
      }{%
        \csname HoLogoFont@font@#2\endcsname{#3}%
      }%
    }{%
      \csname HoLogoFont@logo@#1.#2\endcsname{#3}%
    }%
  \endgroup
}
%    \end{macrocode}
%    \end{macro}
%
%    \begin{macro}{\HoLogoFont@Def}
%    \begin{macrocode}
\def\HoLogoFont@Def#1{%
  \expandafter\def\csname HoLogoFont@font@#1\endcsname
}
%    \end{macrocode}
%    \end{macro}
%    \begin{macro}{\HoLogoFont@LogoDef}
%    \begin{macrocode}
\def\HoLogoFont@LogoDef#1#2{%
  \expandafter\def\csname HoLogoFont@logo@#1.#2\endcsname
}
%    \end{macrocode}
%    \end{macro}
%
% \subsubsection{Font defaults}
%
%    \begin{macro}{\HoLogoFont@font@general}
%    \begin{macrocode}
\HoLogoFont@Def{general}{}%
%    \end{macrocode}
%    \end{macro}
%
%    \begin{macro}{\HoLogoFont@font@rm}
%    \begin{macrocode}
\ltx@IfUndefined{rmfamily}{%
  \ltx@IfUndefined{rm}{%
  }{%
    \HoLogoFont@Def{rm}{\rm}%
  }%
}{%
  \HoLogoFont@Def{rm}{\rmfamily}%
}
%    \end{macrocode}
%    \end{macro}
%
%    \begin{macro}{\HoLogoFont@font@sf}
%    \begin{macrocode}
\ltx@IfUndefined{sffamily}{%
  \ltx@IfUndefined{sf}{%
  }{%
    \HoLogoFont@Def{sf}{\sf}%
  }%
}{%
  \HoLogoFont@Def{sf}{\sffamily}%
}
%    \end{macrocode}
%    \end{macro}
%
%    \begin{macro}{\HoLogoFont@font@bibsf}
%    In case of \hologo{plainTeX} the original small caps
%    variant is used as default. In \hologo{LaTeX}
%    the definition of package \xpackage{dtklogos} \cite{dtklogos}
%    is used.
%\begin{quote}
%\begin{verbatim}
%\DeclareRobustCommand{\BibTeX}{%
%  B%
%  \kern-.05em%
%  \hbox{%
%    $\m@th$% %% force math size calculations
%    \csname S@\f@size\endcsname
%    \fontsize\sf@size\z@
%    \math@fontsfalse
%    \selectfont
%    I%
%    \kern-.025em%
%    B
%  }%
%  \kern-.08em%
%  \-%
%  \TeX
%}
%\end{verbatim}
%\end{quote}
%    \begin{macrocode}
\ltx@IfUndefined{selectfont}{%
  \ltx@IfUndefined{tensc}{%
    \font\tensc=cmcsc10\relax
  }{}%
  \HoLogoFont@Def{bibsf}{\tensc}%
}{%
  \HoLogoFont@Def{bibsf}{%
    $\mathsurround=0pt$%
    \csname S@\f@size\endcsname
    \fontsize\sf@size{0pt}%
    \math@fontsfalse
    \selectfont
  }%
}
%    \end{macrocode}
%    \end{macro}
%
%    \begin{macro}{\HoLogoFont@font@sc}
%    \begin{macrocode}
\ltx@IfUndefined{scshape}{%
  \ltx@IfUndefined{tensc}{%
    \font\tensc=cmcsc10\relax
  }{}%
  \HoLogoFont@Def{sc}{\tensc}%
}{%
  \HoLogoFont@Def{sc}{\scshape}%
}
%    \end{macrocode}
%    \end{macro}
%
%    \begin{macro}{\HoLogoFont@font@sy}
%    \begin{macrocode}
\ltx@IfUndefined{usefont}{%
  \ltx@IfUndefined{tensy}{%
  }{%
    \HoLogoFont@Def{sy}{\tensy}%
  }%
}{%
  \HoLogoFont@Def{sy}{%
    \usefont{OMS}{cmsy}{m}{n}%
  }%
}
%    \end{macrocode}
%    \end{macro}
%
%    \begin{macro}{\HoLogoFont@font@logo}
%    \begin{macrocode}
\begingroup
  \def\x{LaTeX2e}%
\expandafter\endgroup
\ifx\fmtname\x
  \ltx@IfUndefined{logofamily}{%
    \DeclareRobustCommand\logofamily{%
      \not@math@alphabet\logofamily\relax
      \fontencoding{U}%
      \fontfamily{logo}%
      \selectfont
    }%
  }{}%
  \ltx@IfUndefined{logofamily}{%
  }{%
    \HoLogoFont@Def{logo}{\logofamily}%
  }%
\else
  \ltx@IfUndefined{tenlogo}{%
    \font\tenlogo=logo10\relax
  }{}%
  \HoLogoFont@Def{logo}{\tenlogo}%
\fi
%    \end{macrocode}
%    \end{macro}
%
% \subsubsection{Font setup}
%
%    \begin{macro}{\hologoFontSetup}
%    \begin{macrocode}
\def\hologoFontSetup{%
  \let\HOLOGO@name\relax
  \HOLOGO@FontSetup
}
%    \end{macrocode}
%    \end{macro}
%
%    \begin{macro}{\hologoLogoFontSetup}
%    \begin{macrocode}
\def\hologoLogoFontSetup#1{%
  \edef\HOLOGO@name{#1}%
  \ltx@IfUndefined{HoLogo@\HOLOGO@name}{%
    \@PackageError{hologo}{%
      Unknown logo `\HOLOGO@name'%
    }\@ehc
    \ltx@gobble
  }{%
    \HOLOGO@FontSetup
  }%
}
%    \end{macrocode}
%    \end{macro}
%
%    \begin{macro}{\HOLOGO@FontSetup}
%    \begin{macrocode}
\def\HOLOGO@FontSetup{%
  \kvsetkeys{HoLogoFont}%
}
%    \end{macrocode}
%    \end{macro}
%
%    \begin{macrocode}
\def\HOLOGO@temp#1{%
  \kv@define@key{HoLogoFont}{#1}{%
    \ifx\HOLOGO@name\relax
      \HoLogoFont@Def{#1}{##1}%
    \else
      \HoLogoFont@LogoDef\HOLOGO@name{#1}{##1}%
    \fi
  }%
}
\HOLOGO@temp{general}
\HOLOGO@temp{sf}
%    \end{macrocode}
%
% \subsection{Generic logo commands}
%
%    \begin{macrocode}
\HOLOGO@IfExists\hologo{%
  \@PackageError{hologo}{%
    \string\hologo\ltx@space is already defined.\MessageBreak
    Package loading is aborted%
  }\@ehc
  \HOLOGO@AtEnd
}%
\HOLOGO@IfExists\hologoRobust{%
  \@PackageError{hologo}{%
    \string\hologoRobust\ltx@space is already defined.\MessageBreak
    Package loading is aborted%
  }\@ehc
  \HOLOGO@AtEnd
}%
%    \end{macrocode}
%
% \subsubsection{\cs{hologo} and friends}
%
%    \begin{macrocode}
\ifluatex
  \expandafter\ltx@firstofone
\else
  \expandafter\ltx@gobble
\fi
{%
  \ltx@IfUndefined{ifincsname}{%
    \ifnum\luatexversion<36 %
      \expandafter\ltx@gobble
    \else
      \expandafter\ltx@firstofone
    \fi
    {%
      \begingroup
        \ifcase0%
            \directlua{%
              if tex.enableprimitives then %
                tex.enableprimitives('HOLOGO@', {'ifincsname'})%
              else %
                tex.print('1')%
              end%
            }%
            \ifx\HOLOGO@ifincsname\@undefined 1\fi%
            \relax
          \expandafter\ltx@firstofone
        \else
          \endgroup
          \expandafter\ltx@gobble
        \fi
        {%
          \global\let\ifincsname\HOLOGO@ifincsname
        }%
      \HOLOGO@temp
    }%
  }{}%
}
%    \end{macrocode}
%    \begin{macrocode}
\ltx@IfUndefined{ifincsname}{%
  \catcode`$=14 %
}{%
  \catcode`$=9 %
}
%    \end{macrocode}
%
%    \begin{macro}{\hologo}
%    \begin{macrocode}
\def\hologo#1{%
$ \ifincsname
$   \ltx@ifundefined{HoLogoCs@\HOLOGO@Variant{#1}}{%
$     #1%
$   }{%
$     \csname HoLogoCs@\HOLOGO@Variant{#1}\endcsname\ltx@firstoftwo
$   }%
$ \else
    \HOLOGO@IfExists\texorpdfstring\texorpdfstring\ltx@firstoftwo
    {%
      \hologoRobust{#1}%
    }{%
      \ltx@ifundefined{HoLogoBkm@\HOLOGO@Variant{#1}}{%
        \ltx@ifundefined{HoLogo@#1}{?#1?}{#1}%
      }{%
        \csname HoLogoBkm@\HOLOGO@Variant{#1}\endcsname
        \ltx@firstoftwo
      }%
    }%
$ \fi
}
%    \end{macrocode}
%    \end{macro}
%    \begin{macro}{\Hologo}
%    \begin{macrocode}
\def\Hologo#1{%
$ \ifincsname
$   \ltx@ifundefined{HoLogoCs@\HOLOGO@Variant{#1}}{%
$     #1%
$   }{%
$     \csname HoLogoCs@\HOLOGO@Variant{#1}\endcsname\ltx@secondoftwo
$   }%
$ \else
    \HOLOGO@IfExists\texorpdfstring\texorpdfstring\ltx@firstoftwo
    {%
      \HologoRobust{#1}%
    }{%
      \ltx@ifundefined{HoLogoBkm@\HOLOGO@Variant{#1}}{%
        \ltx@ifundefined{HoLogo@#1}{?#1?}{#1}%
      }{%
        \csname HoLogoBkm@\HOLOGO@Variant{#1}\endcsname
        \ltx@secondoftwo
      }%
    }%
$ \fi
}
%    \end{macrocode}
%    \end{macro}
%
%    \begin{macro}{\hologoVariant}
%    \begin{macrocode}
\def\hologoVariant#1#2{%
  \ifx\relax#2\relax
    \hologo{#1}%
  \else
$   \ifincsname
$     \ltx@ifundefined{HoLogoCs@#1@#2}{%
$       #1%
$     }{%
$       \csname HoLogoCs@#1@#2\endcsname\ltx@firstoftwo
$     }%
$   \else
      \HOLOGO@IfExists\texorpdfstring\texorpdfstring\ltx@firstoftwo
      {%
        \hologoVariantRobust{#1}{#2}%
      }{%
        \ltx@ifundefined{HoLogoBkm@#1@#2}{%
          \ltx@ifundefined{HoLogo@#1}{?#1?}{#1}%
        }{%
          \csname HoLogoBkm@#1@#2\endcsname
          \ltx@firstoftwo
        }%
      }%
$   \fi
  \fi
}
%    \end{macrocode}
%    \end{macro}
%    \begin{macro}{\HologoVariant}
%    \begin{macrocode}
\def\HologoVariant#1#2{%
  \ifx\relax#2\relax
    \Hologo{#1}%
  \else
$   \ifincsname
$     \ltx@ifundefined{HoLogoCs@#1@#2}{%
$       #1%
$     }{%
$       \csname HoLogoCs@#1@#2\endcsname\ltx@secondoftwo
$     }%
$   \else
      \HOLOGO@IfExists\texorpdfstring\texorpdfstring\ltx@firstoftwo
      {%
        \HologoVariantRobust{#1}{#2}%
      }{%
        \ltx@ifundefined{HoLogoBkm@#1@#2}{%
          \ltx@ifundefined{HoLogo@#1}{?#1?}{#1}%
        }{%
          \csname HoLogoBkm@#1@#2\endcsname
          \ltx@secondoftwo
        }%
      }%
$   \fi
  \fi
}
%    \end{macrocode}
%    \end{macro}
%
%    \begin{macrocode}
\catcode`\$=3 %
%    \end{macrocode}
%
% \subsubsection{\cs{hologoRobust} and friends}
%
%    \begin{macro}{\hologoRobust}
%    \begin{macrocode}
\ltx@IfUndefined{protected}{%
  \ltx@IfUndefined{DeclareRobustCommand}{%
    \def\hologoRobust#1%
  }{%
    \DeclareRobustCommand*\hologoRobust[1]%
  }%
}{%
  \protected\def\hologoRobust#1%
}%
{%
  \edef\HOLOGO@name{#1}%
  \ltx@IfUndefined{HoLogo@\HOLOGO@Variant\HOLOGO@name}{%
    \@PackageError{hologo}{%
      Unknown logo `\HOLOGO@name'%
    }\@ehc
    ?\HOLOGO@name?%
  }{%
    \ltx@IfUndefined{ver@tex4ht.sty}{%
      \HoLogoFont@font\HOLOGO@name{general}{%
        \csname HoLogo@\HOLOGO@Variant\HOLOGO@name\endcsname
        \ltx@firstoftwo
      }%
    }{%
      \ltx@IfUndefined{HoLogoHtml@\HOLOGO@Variant\HOLOGO@name}{%
        \HOLOGO@name
      }{%
        \csname HoLogoHtml@\HOLOGO@Variant\HOLOGO@name\endcsname
        \ltx@firstoftwo
      }%
    }%
  }%
}
%    \end{macrocode}
%    \end{macro}
%    \begin{macro}{\HologoRobust}
%    \begin{macrocode}
\ltx@IfUndefined{protected}{%
  \ltx@IfUndefined{DeclareRobustCommand}{%
    \def\HologoRobust#1%
  }{%
    \DeclareRobustCommand*\HologoRobust[1]%
  }%
}{%
  \protected\def\HologoRobust#1%
}%
{%
  \edef\HOLOGO@name{#1}%
  \ltx@IfUndefined{HoLogo@\HOLOGO@Variant\HOLOGO@name}{%
    \@PackageError{hologo}{%
      Unknown logo `\HOLOGO@name'%
    }\@ehc
    ?\HOLOGO@name?%
  }{%
    \ltx@IfUndefined{ver@tex4ht.sty}{%
      \HoLogoFont@font\HOLOGO@name{general}{%
        \csname HoLogo@\HOLOGO@Variant\HOLOGO@name\endcsname
        \ltx@secondoftwo
      }%
    }{%
      \ltx@IfUndefined{HoLogoHtml@\HOLOGO@Variant\HOLOGO@name}{%
        \expandafter\HOLOGO@Uppercase\HOLOGO@name
      }{%
        \csname HoLogoHtml@\HOLOGO@Variant\HOLOGO@name\endcsname
        \ltx@secondoftwo
      }%
    }%
  }%
}
%    \end{macrocode}
%    \end{macro}
%    \begin{macro}{\hologoVariantRobust}
%    \begin{macrocode}
\ltx@IfUndefined{protected}{%
  \ltx@IfUndefined{DeclareRobustCommand}{%
    \def\hologoVariantRobust#1#2%
  }{%
    \DeclareRobustCommand*\hologoVariantRobust[2]%
  }%
}{%
  \protected\def\hologoVariantRobust#1#2%
}%
{%
  \begingroup
    \hologoLogoSetup{#1}{variant={#2}}%
    \hologoRobust{#1}%
  \endgroup
}
%    \end{macrocode}
%    \end{macro}
%    \begin{macro}{\HologoVariantRobust}
%    \begin{macrocode}
\ltx@IfUndefined{protected}{%
  \ltx@IfUndefined{DeclareRobustCommand}{%
    \def\HologoVariantRobust#1#2%
  }{%
    \DeclareRobustCommand*\HologoVariantRobust[2]%
  }%
}{%
  \protected\def\HologoVariantRobust#1#2%
}%
{%
  \begingroup
    \hologoLogoSetup{#1}{variant={#2}}%
    \HologoRobust{#1}%
  \endgroup
}
%    \end{macrocode}
%    \end{macro}
%
%    \begin{macro}{\hologorobust}
%    Macro \cs{hologorobust} is only defined for compatibility.
%    Its use is deprecated.
%    \begin{macrocode}
\def\hologorobust{\hologoRobust}
%    \end{macrocode}
%    \end{macro}
%
% \subsection{Helpers}
%
%    \begin{macro}{\HOLOGO@Uppercase}
%    Macro \cs{HOLOGO@Uppercase} is restricted to \cs{uppercase},
%    because \hologo{plainTeX} or \hologo{iniTeX} do not provide
%    \cs{MakeUppercase}.
%    \begin{macrocode}
\def\HOLOGO@Uppercase#1{\uppercase{#1}}
%    \end{macrocode}
%    \end{macro}
%
%    \begin{macro}{\HOLOGO@PdfdocUnicode}
%    \begin{macrocode}
\def\HOLOGO@PdfdocUnicode{%
  \ifx\ifHy@unicode\iftrue
    \expandafter\ltx@secondoftwo
  \else
    \expandafter\ltx@firstoftwo
  \fi
}
%    \end{macrocode}
%    \end{macro}
%
%    \begin{macro}{\HOLOGO@Math}
%    \begin{macrocode}
\def\HOLOGO@MathSetup{%
  \mathsurround0pt\relax
  \HOLOGO@IfExists\f@series{%
    \if b\expandafter\ltx@car\f@series x\@nil
      \csname boldmath\endcsname
   \fi
  }{}%
}
%    \end{macrocode}
%    \end{macro}
%
%    \begin{macro}{\HOLOGO@TempDimen}
%    \begin{macrocode}
\dimendef\HOLOGO@TempDimen=\ltx@zero
%    \end{macrocode}
%    \end{macro}
%    \begin{macro}{\HOLOGO@NegativeKerning}
%    \begin{macrocode}
\def\HOLOGO@NegativeKerning#1{%
  \begingroup
    \HOLOGO@TempDimen=0pt\relax
    \comma@parse@normalized{#1}{%
      \ifdim\HOLOGO@TempDimen=0pt %
        \expandafter\HOLOGO@@NegativeKerning\comma@entry
      \fi
      \ltx@gobble
    }%
    \ifdim\HOLOGO@TempDimen<0pt %
      \kern\HOLOGO@TempDimen
    \fi
  \endgroup
}
%    \end{macrocode}
%    \end{macro}
%    \begin{macro}{\HOLOGO@@NegativeKerning}
%    \begin{macrocode}
\def\HOLOGO@@NegativeKerning#1#2{%
  \setbox\ltx@zero\hbox{#1#2}%
  \HOLOGO@TempDimen=\wd\ltx@zero
  \setbox\ltx@zero\hbox{#1\kern0pt#2}%
  \advance\HOLOGO@TempDimen by -\wd\ltx@zero
}
%    \end{macrocode}
%    \end{macro}
%
%    \begin{macro}{\HOLOGO@SpaceFactor}
%    \begin{macrocode}
\def\HOLOGO@SpaceFactor{%
  \spacefactor1000 %
}
%    \end{macrocode}
%    \end{macro}
%
%    \begin{macro}{\HOLOGO@Span}
%    \begin{macrocode}
\def\HOLOGO@Span#1#2{%
  \HCode{<span class="HoLogo-#1">}%
  #2%
  \HCode{</span>}%
}
%    \end{macrocode}
%    \end{macro}
%
% \subsubsection{Text subscript}
%
%    \begin{macro}{\HOLOGO@SubScript}%
%    \begin{macrocode}
\def\HOLOGO@SubScript#1{%
  \ltx@IfUndefined{textsubscript}{%
    \ltx@IfUndefined{text}{%
      \ltx@mbox{%
        \mathsurround=0pt\relax
        $%
          _{%
            \ltx@IfUndefined{sf@size}{%
              \mathrm{#1}%
            }{%
              \mbox{%
                \fontsize\sf@size{0pt}\selectfont
                #1%
              }%
            }%
          }%
        $%
      }%
    }{%
      \ltx@mbox{%
        \mathsurround=0pt\relax
        $_{\text{#1}}$%
      }%
    }%
  }{%
    \textsubscript{#1}%
  }%
}
%    \end{macrocode}
%    \end{macro}
%
% \subsection{\hologo{TeX} and friends}
%
% \subsubsection{\hologo{TeX}}
%
%    \begin{macro}{\HoLogo@TeX}
%    Source: \hologo{LaTeX} kernel.
%    \begin{macrocode}
\def\HoLogo@TeX#1{%
  T\kern-.1667em\lower.5ex\hbox{E}\kern-.125emX\HOLOGO@SpaceFactor
}
%    \end{macrocode}
%    \end{macro}
%    \begin{macro}{\HoLogoHtml@TeX}
%    \begin{macrocode}
\def\HoLogoHtml@TeX#1{%
  \HoLogoCss@TeX
  \HOLOGO@Span{TeX}{%
    T%
    \HOLOGO@Span{e}{%
      E%
    }%
    X%
  }%
}
%    \end{macrocode}
%    \end{macro}
%    \begin{macro}{\HoLogoCss@TeX}
%    \begin{macrocode}
\def\HoLogoCss@TeX{%
  \Css{%
    span.HoLogo-TeX span.HoLogo-e{%
      position:relative;%
      top:.5ex;%
      margin-left:-.1667em;%
      margin-right:-.125em;%
    }%
  }%
  \Css{%
    a span.HoLogo-TeX span.HoLogo-e{%
      text-decoration:none;%
    }%
  }%
  \global\let\HoLogoCss@TeX\relax
}
%    \end{macrocode}
%    \end{macro}
%
% \subsubsection{\hologo{plainTeX}}
%
%    \begin{macro}{\HoLogo@plainTeX@space}
%    Source: ``The \hologo{TeX}book''
%    \begin{macrocode}
\def\HoLogo@plainTeX@space#1{%
  \HOLOGO@mbox{#1{p}{P}lain}\HOLOGO@space\hologo{TeX}%
}
%    \end{macrocode}
%    \end{macro}
%    \begin{macro}{\HoLogoCs@plainTeX@space}
%    \begin{macrocode}
\def\HoLogoCs@plainTeX@space#1{#1{p}{P}lain TeX}%
%    \end{macrocode}
%    \end{macro}
%    \begin{macro}{\HoLogoBkm@plainTeX@space}
%    \begin{macrocode}
\def\HoLogoBkm@plainTeX@space#1{%
  #1{p}{P}lain \hologo{TeX}%
}
%    \end{macrocode}
%    \end{macro}
%    \begin{macro}{\HoLogoHtml@plainTeX@space}
%    \begin{macrocode}
\def\HoLogoHtml@plainTeX@space#1{%
  #1{p}{P}lain \hologo{TeX}%
}
%    \end{macrocode}
%    \end{macro}
%
%    \begin{macro}{\HoLogo@plainTeX@hyphen}
%    \begin{macrocode}
\def\HoLogo@plainTeX@hyphen#1{%
  \HOLOGO@mbox{#1{p}{P}lain}\HOLOGO@hyphen\hologo{TeX}%
}
%    \end{macrocode}
%    \end{macro}
%    \begin{macro}{\HoLogoCs@plainTeX@hyphen}
%    \begin{macrocode}
\def\HoLogoCs@plainTeX@hyphen#1{#1{p}{P}lain-TeX}
%    \end{macrocode}
%    \end{macro}
%    \begin{macro}{\HoLogoBkm@plainTeX@hyphen}
%    \begin{macrocode}
\def\HoLogoBkm@plainTeX@hyphen#1{%
  #1{p}{P}lain-\hologo{TeX}%
}
%    \end{macrocode}
%    \end{macro}
%    \begin{macro}{\HoLogoHtml@plainTeX@hyphen}
%    \begin{macrocode}
\def\HoLogoHtml@plainTeX@hyphen#1{%
  #1{p}{P}lain-\hologo{TeX}%
}
%    \end{macrocode}
%    \end{macro}
%
%    \begin{macro}{\HoLogo@plainTeX@runtogether}
%    \begin{macrocode}
\def\HoLogo@plainTeX@runtogether#1{%
  \HOLOGO@mbox{#1{p}{P}lain\hologo{TeX}}%
}
%    \end{macrocode}
%    \end{macro}
%    \begin{macro}{\HoLogoCs@plainTeX@runtogether}
%    \begin{macrocode}
\def\HoLogoCs@plainTeX@runtogether#1{#1{p}{P}lainTeX}
%    \end{macrocode}
%    \end{macro}
%    \begin{macro}{\HoLogoBkm@plainTeX@runtogether}
%    \begin{macrocode}
\def\HoLogoBkm@plainTeX@runtogether#1{%
  #1{p}{P}lain\hologo{TeX}%
}
%    \end{macrocode}
%    \end{macro}
%    \begin{macro}{\HoLogoHtml@plainTeX@runtogether}
%    \begin{macrocode}
\def\HoLogoHtml@plainTeX@runtogether#1{%
  #1{p}{P}lain\hologo{TeX}%
}
%    \end{macrocode}
%    \end{macro}
%
%    \begin{macro}{\HoLogo@plainTeX}
%    \begin{macrocode}
\def\HoLogo@plainTeX{\HoLogo@plainTeX@space}
%    \end{macrocode}
%    \end{macro}
%    \begin{macro}{\HoLogoCs@plainTeX}
%    \begin{macrocode}
\def\HoLogoCs@plainTeX{\HoLogoCs@plainTeX@space}
%    \end{macrocode}
%    \end{macro}
%    \begin{macro}{\HoLogoBkm@plainTeX}
%    \begin{macrocode}
\def\HoLogoBkm@plainTeX{\HoLogoBkm@plainTeX@space}
%    \end{macrocode}
%    \end{macro}
%    \begin{macro}{\HoLogoHtml@plainTeX}
%    \begin{macrocode}
\def\HoLogoHtml@plainTeX{\HoLogoHtml@plainTeX@space}
%    \end{macrocode}
%    \end{macro}
%
% \subsubsection{\hologo{LaTeX}}
%
%    Source: \hologo{LaTeX} kernel.
%\begin{quote}
%\begin{verbatim}
%\DeclareRobustCommand{\LaTeX}{%
%  L%
%  \kern-.36em%
%  {%
%    \sbox\z@ T%
%    \vbox to\ht\z@{%
%      \hbox{%
%        \check@mathfonts
%        \fontsize\sf@size\z@
%        \math@fontsfalse
%        \selectfont
%        A%
%      }%
%      \vss
%    }%
%  }%
%  \kern-.15em%
%  \TeX
%}
%\end{verbatim}
%\end{quote}
%
%    \begin{macro}{\HoLogo@La}
%    \begin{macrocode}
\def\HoLogo@La#1{%
  L%
  \kern-.36em%
  \begingroup
    \setbox\ltx@zero\hbox{T}%
    \vbox to\ht\ltx@zero{%
      \hbox{%
        \ltx@ifundefined{check@mathfonts}{%
          \csname sevenrm\endcsname
        }{%
          \check@mathfonts
          \fontsize\sf@size{0pt}%
          \math@fontsfalse\selectfont
        }%
        A%
      }%
      \vss
    }%
  \endgroup
}
%    \end{macrocode}
%    \end{macro}
%
%    \begin{macro}{\HoLogo@LaTeX}
%    Source: \hologo{LaTeX} kernel.
%    \begin{macrocode}
\def\HoLogo@LaTeX#1{%
  \hologo{La}%
  \kern-.15em%
  \hologo{TeX}%
}
%    \end{macrocode}
%    \end{macro}
%    \begin{macro}{\HoLogoHtml@LaTeX}
%    \begin{macrocode}
\def\HoLogoHtml@LaTeX#1{%
  \HoLogoCss@LaTeX
  \HOLOGO@Span{LaTeX}{%
    L%
    \HOLOGO@Span{a}{%
      A%
    }%
    \hologo{TeX}%
  }%
}
%    \end{macrocode}
%    \end{macro}
%    \begin{macro}{\HoLogoCss@LaTeX}
%    \begin{macrocode}
\def\HoLogoCss@LaTeX{%
  \Css{%
    span.HoLogo-LaTeX span.HoLogo-a{%
      position:relative;%
      top:-.5ex;%
      margin-left:-.36em;%
      margin-right:-.15em;%
      font-size:85\%;%
    }%
  }%
  \global\let\HoLogoCss@LaTeX\relax
}
%    \end{macrocode}
%    \end{macro}
%
% \subsubsection{\hologo{(La)TeX}}
%
%    \begin{macro}{\HoLogo@LaTeXTeX}
%    The kerning around the parentheses is taken
%    from package \xpackage{dtklogos} \cite{dtklogos}.
%\begin{quote}
%\begin{verbatim}
%\DeclareRobustCommand{\LaTeXTeX}{%
%  (%
%  \kern-.15em%
%  L%
%  \kern-.36em%
%  {%
%    \sbox\z@ T%
%    \vbox to\ht0{%
%      \hbox{%
%        $\m@th$%
%        \csname S@\f@size\endcsname
%        \fontsize\sf@size\z@
%        \math@fontsfalse
%        \selectfont
%        A%
%      }%
%      \vss
%    }%
%  }%
%  \kern-.2em%
%  )%
%  \kern-.15em%
%  \TeX
%}
%\end{verbatim}
%\end{quote}
%    \begin{macrocode}
\def\HoLogo@LaTeXTeX#1{%
  (%
  \kern-.15em%
  \hologo{La}%
  \kern-.2em%
  )%
  \kern-.15em%
  \hologo{TeX}%
}
%    \end{macrocode}
%    \end{macro}
%    \begin{macro}{\HoLogoBkm@LaTeXTeX}
%    \begin{macrocode}
\def\HoLogoBkm@LaTeXTeX#1{(La)TeX}
%    \end{macrocode}
%    \end{macro}
%
%    \begin{macro}{\HoLogo@(La)TeX}
%    \begin{macrocode}
\expandafter
\let\csname HoLogo@(La)TeX\endcsname\HoLogo@LaTeXTeX
%    \end{macrocode}
%    \end{macro}
%    \begin{macro}{\HoLogoBkm@(La)TeX}
%    \begin{macrocode}
\expandafter
\let\csname HoLogoBkm@(La)TeX\endcsname\HoLogoBkm@LaTeXTeX
%    \end{macrocode}
%    \end{macro}
%    \begin{macro}{\HoLogoHtml@LaTeXTeX}
%    \begin{macrocode}
\def\HoLogoHtml@LaTeXTeX#1{%
  \HoLogoCss@LaTeXTeX
  \HOLOGO@Span{LaTeXTeX}{%
    (%
    \HOLOGO@Span{L}{L}%
    \HOLOGO@Span{a}{A}%
    \HOLOGO@Span{ParenRight}{)}%
    \hologo{TeX}%
  }%
}
%    \end{macrocode}
%    \end{macro}
%    \begin{macro}{\HoLogoHtml@(La)TeX}
%    Kerning after opening parentheses and before closing parentheses
%    is $-0.1$\,em. The original values $-0.15$\,em
%    looked too ugly for a serif font.
%    \begin{macrocode}
\expandafter
\let\csname HoLogoHtml@(La)TeX\endcsname\HoLogoHtml@LaTeXTeX
%    \end{macrocode}
%    \end{macro}
%    \begin{macro}{\HoLogoCss@LaTeXTeX}
%    \begin{macrocode}
\def\HoLogoCss@LaTeXTeX{%
  \Css{%
    span.HoLogo-LaTeXTeX span.HoLogo-L{%
      margin-left:-.1em;%
    }%
  }%
  \Css{%
    span.HoLogo-LaTeXTeX span.HoLogo-a{%
      position:relative;%
      top:-.5ex;%
      margin-left:-.36em;%
      margin-right:-.1em;%
      font-size:85\%;%
    }%
  }%
  \Css{%
    span.HoLogo-LaTeXTeX span.HoLogo-ParenRight{%
      margin-right:-.15em;%
    }%
  }%
  \global\let\HoLogoCss@LaTeXTeX\relax
}
%    \end{macrocode}
%    \end{macro}
%
% \subsubsection{\hologo{LaTeXe}}
%
%    \begin{macro}{\HoLogo@LaTeXe}
%    Source: \hologo{LaTeX} kernel
%    \begin{macrocode}
\def\HoLogo@LaTeXe#1{%
  \hologo{LaTeX}%
  \kern.15em%
  \hbox{%
    \HOLOGO@MathSetup
    2%
    $_{\textstyle\varepsilon}$%
  }%
}
%    \end{macrocode}
%    \end{macro}
%
%    \begin{macro}{\HoLogoCs@LaTeXe}
%    \begin{macrocode}
\ifnum64=`\^^^^0040\relax % test for big chars of LuaTeX/XeTeX
  \catcode`\$=9 %
  \catcode`\&=14 %
\else
  \catcode`\$=14 %
  \catcode`\&=9 %
\fi
\def\HoLogoCs@LaTeXe#1{%
  LaTeX2%
$ \string ^^^^0395%
& e%
}%
\catcode`\$=3 %
\catcode`\&=4 %
%    \end{macrocode}
%    \end{macro}
%
%    \begin{macro}{\HoLogoBkm@LaTeXe}
%    \begin{macrocode}
\def\HoLogoBkm@LaTeXe#1{%
  \hologo{LaTeX}%
  2%
  \HOLOGO@PdfdocUnicode{e}{\textepsilon}%
}
%    \end{macrocode}
%    \end{macro}
%
%    \begin{macro}{\HoLogoHtml@LaTeXe}
%    \begin{macrocode}
\def\HoLogoHtml@LaTeXe#1{%
  \HoLogoCss@LaTeXe
  \HOLOGO@Span{LaTeX2e}{%
    \hologo{LaTeX}%
    \HOLOGO@Span{2}{2}%
    \HOLOGO@Span{e}{%
      \HOLOGO@MathSetup
      \ensuremath{\textstyle\varepsilon}%
    }%
  }%
}
%    \end{macrocode}
%    \end{macro}
%    \begin{macro}{\HoLogoCss@LaTeXe}
%    \begin{macrocode}
\def\HoLogoCss@LaTeXe{%
  \Css{%
    span.HoLogo-LaTeX2e span.HoLogo-2{%
      padding-left:.15em;%
    }%
  }%
  \Css{%
    span.HoLogo-LaTeX2e span.HoLogo-e{%
      position:relative;%
      top:.35ex;%
      text-decoration:none;%
    }%
  }%
  \global\let\HoLogoCss@LaTeXe\relax
}
%    \end{macrocode}
%    \end{macro}
%
%    \begin{macro}{\HoLogo@LaTeX2e}
%    \begin{macrocode}
\expandafter
\let\csname HoLogo@LaTeX2e\endcsname\HoLogo@LaTeXe
%    \end{macrocode}
%    \end{macro}
%    \begin{macro}{\HoLogoCs@LaTeX2e}
%    \begin{macrocode}
\expandafter
\let\csname HoLogoCs@LaTeX2e\endcsname\HoLogoCs@LaTeXe
%    \end{macrocode}
%    \end{macro}
%    \begin{macro}{\HoLogoBkm@LaTeX2e}
%    \begin{macrocode}
\expandafter
\let\csname HoLogoBkm@LaTeX2e\endcsname\HoLogoBkm@LaTeXe
%    \end{macrocode}
%    \end{macro}
%    \begin{macro}{\HoLogoHtml@LaTeX2e}
%    \begin{macrocode}
\expandafter
\let\csname HoLogoHtml@LaTeX2e\endcsname\HoLogoHtml@LaTeXe
%    \end{macrocode}
%    \end{macro}
%
% \subsubsection{\hologo{LaTeX3}}
%
%    \begin{macro}{\HoLogo@LaTeX3}
%    Source: \hologo{LaTeX} kernel
%    \begin{macrocode}
\expandafter\def\csname HoLogo@LaTeX3\endcsname#1{%
  \hologo{LaTeX}%
  3%
}
%    \end{macrocode}
%    \end{macro}
%
%    \begin{macro}{\HoLogoBkm@LaTeX3}
%    \begin{macrocode}
\expandafter\def\csname HoLogoBkm@LaTeX3\endcsname#1{%
  \hologo{LaTeX}%
  3%
}
%    \end{macrocode}
%    \end{macro}
%    \begin{macro}{\HoLogoHtml@LaTeX3}
%    \begin{macrocode}
\expandafter
\let\csname HoLogoHtml@LaTeX3\expandafter\endcsname
\csname HoLogo@LaTeX3\endcsname
%    \end{macrocode}
%    \end{macro}
%
% \subsubsection{\hologo{LaTeXML}}
%
%    \begin{macro}{\HoLogo@LaTeXML}
%    \begin{macrocode}
\def\HoLogo@LaTeXML#1{%
  \HOLOGO@mbox{%
    \hologo{La}%
    \kern-.15em%
    T%
    \kern-.1667em%
    \lower.5ex\hbox{E}%
    \kern-.125em%
    \HoLogoFont@font{LaTeXML}{sc}{xml}%
  }%
}
%    \end{macrocode}
%    \end{macro}
%    \begin{macro}{\HoLogoHtml@pdfLaTeX}
%    \begin{macrocode}
\def\HoLogoHtml@LaTeXML#1{%
  \HOLOGO@Span{LaTeXML}{%
    \HoLogoCss@LaTeX
    \HoLogoCss@TeX
    \HOLOGO@Span{LaTeX}{%
      L%
      \HOLOGO@Span{a}{%
        A%
      }%
    }%
    \HOLOGO@Span{TeX}{%
      T%
      \HOLOGO@Span{e}{%
        E%
      }%
    }%
    \HCode{<span style="font-variant: small-caps;">}%
    xml%
    \HCode{</span>}%
  }%
}
%    \end{macrocode}
%    \end{macro}
%
% \subsubsection{\hologo{eTeX}}
%
%    \begin{macro}{\HoLogo@eTeX}
%    Source: package \xpackage{etex}
%    \begin{macrocode}
\def\HoLogo@eTeX#1{%
  \ltx@mbox{%
    \HOLOGO@MathSetup
    $\varepsilon$%
    -%
    \HOLOGO@NegativeKerning{-T,T-,To}%
    \hologo{TeX}%
  }%
}
%    \end{macrocode}
%    \end{macro}
%    \begin{macro}{\HoLogoCs@eTeX}
%    \begin{macrocode}
\ifnum64=`\^^^^0040\relax % test for big chars of LuaTeX/XeTeX
  \catcode`\$=9 %
  \catcode`\&=14 %
\else
  \catcode`\$=14 %
  \catcode`\&=9 %
\fi
\def\HoLogoCs@eTeX#1{%
$ #1{\string ^^^^0395}{\string ^^^^03b5}%
& #1{e}{E}%
  TeX%
}%
\catcode`\$=3 %
\catcode`\&=4 %
%    \end{macrocode}
%    \end{macro}
%    \begin{macro}{\HoLogoBkm@eTeX}
%    \begin{macrocode}
\def\HoLogoBkm@eTeX#1{%
  \HOLOGO@PdfdocUnicode{#1{e}{E}}{\textepsilon}%
  -%
  \hologo{TeX}%
}
%    \end{macrocode}
%    \end{macro}
%    \begin{macro}{\HoLogoHtml@eTeX}
%    \begin{macrocode}
\def\HoLogoHtml@eTeX#1{%
  \ltx@mbox{%
    \HOLOGO@MathSetup
    $\varepsilon$%
    -%
    \hologo{TeX}%
  }%
}
%    \end{macrocode}
%    \end{macro}
%
% \subsubsection{\hologo{iniTeX}}
%
%    \begin{macro}{\HoLogo@iniTeX}
%    \begin{macrocode}
\def\HoLogo@iniTeX#1{%
  \HOLOGO@mbox{%
    #1{i}{I}ni\hologo{TeX}%
  }%
}
%    \end{macrocode}
%    \end{macro}
%    \begin{macro}{\HoLogoCs@iniTeX}
%    \begin{macrocode}
\def\HoLogoCs@iniTeX#1{#1{i}{I}niTeX}
%    \end{macrocode}
%    \end{macro}
%    \begin{macro}{\HoLogoBkm@iniTeX}
%    \begin{macrocode}
\def\HoLogoBkm@iniTeX#1{%
  #1{i}{I}ni\hologo{TeX}%
}
%    \end{macrocode}
%    \end{macro}
%    \begin{macro}{\HoLogoHtml@iniTeX}
%    \begin{macrocode}
\let\HoLogoHtml@iniTeX\HoLogo@iniTeX
%    \end{macrocode}
%    \end{macro}
%
% \subsubsection{\hologo{virTeX}}
%
%    \begin{macro}{\HoLogo@virTeX}
%    \begin{macrocode}
\def\HoLogo@virTeX#1{%
  \HOLOGO@mbox{%
    #1{v}{V}ir\hologo{TeX}%
  }%
}
%    \end{macrocode}
%    \end{macro}
%    \begin{macro}{\HoLogoCs@virTeX}
%    \begin{macrocode}
\def\HoLogoCs@virTeX#1{#1{v}{V}irTeX}
%    \end{macrocode}
%    \end{macro}
%    \begin{macro}{\HoLogoBkm@virTeX}
%    \begin{macrocode}
\def\HoLogoBkm@virTeX#1{%
  #1{v}{V}ir\hologo{TeX}%
}
%    \end{macrocode}
%    \end{macro}
%    \begin{macro}{\HoLogoHtml@virTeX}
%    \begin{macrocode}
\let\HoLogoHtml@virTeX\HoLogo@virTeX
%    \end{macrocode}
%    \end{macro}
%
% \subsubsection{\hologo{SliTeX}}
%
% \paragraph{Definitions of the three variants.}
%
%    \begin{macro}{\HoLogo@SLiTeX@lift}
%    \begin{macrocode}
\def\HoLogo@SLiTeX@lift#1{%
  \HoLogoFont@font{SliTeX}{rm}{%
    S%
    \kern-.06em%
    L%
    \kern-.18em%
    \raise.32ex\hbox{\HoLogoFont@font{SliTeX}{sc}{i}}%
    \HOLOGO@discretionary
    \kern-.06em%
    \hologo{TeX}%
  }%
}
%    \end{macrocode}
%    \end{macro}
%    \begin{macro}{\HoLogoBkm@SLiTeX@lift}
%    \begin{macrocode}
\def\HoLogoBkm@SLiTeX@lift#1{SLiTeX}
%    \end{macrocode}
%    \end{macro}
%    \begin{macro}{\HoLogoHtml@SLiTeX@lift}
%    \begin{macrocode}
\def\HoLogoHtml@SLiTeX@lift#1{%
  \HoLogoCss@SLiTeX@lift
  \HOLOGO@Span{SLiTeX-lift}{%
    \HoLogoFont@font{SliTeX}{rm}{%
      S%
      \HOLOGO@Span{L}{L}%
      \HOLOGO@Span{i}{i}%
      \hologo{TeX}%
    }%
  }%
}
%    \end{macrocode}
%    \end{macro}
%    \begin{macro}{\HoLogoCss@SLiTeX@lift}
%    \begin{macrocode}
\def\HoLogoCss@SLiTeX@lift{%
  \Css{%
    span.HoLogo-SLiTeX-lift span.HoLogo-L{%
      margin-left:-.06em;%
      margin-right:-.18em;%
    }%
  }%
  \Css{%
    span.HoLogo-SLiTeX-lift span.HoLogo-i{%
      position:relative;%
      top:-.32ex;%
      margin-right:-.06em;%
      font-variant:small-caps;%
    }%
  }%
  \global\let\HoLogoCss@SLiTeX@lift\relax
}
%    \end{macrocode}
%    \end{macro}
%
%    \begin{macro}{\HoLogo@SliTeX@simple}
%    \begin{macrocode}
\def\HoLogo@SliTeX@simple#1{%
  \HoLogoFont@font{SliTeX}{rm}{%
    \ltx@mbox{%
      \HoLogoFont@font{SliTeX}{sc}{Sli}%
    }%
    \HOLOGO@discretionary
    \hologo{TeX}%
  }%
}
%    \end{macrocode}
%    \end{macro}
%    \begin{macro}{\HoLogoBkm@SliTeX@simple}
%    \begin{macrocode}
\def\HoLogoBkm@SliTeX@simple#1{SliTeX}
%    \end{macrocode}
%    \end{macro}
%    \begin{macro}{\HoLogoHtml@SliTeX@simple}
%    \begin{macrocode}
\let\HoLogoHtml@SliTeX@simple\HoLogo@SliTeX@simple
%    \end{macrocode}
%    \end{macro}
%
%    \begin{macro}{\HoLogo@SliTeX@narrow}
%    \begin{macrocode}
\def\HoLogo@SliTeX@narrow#1{%
  \HoLogoFont@font{SliTeX}{rm}{%
    \ltx@mbox{%
      S%
      \kern-.06em%
      \HoLogoFont@font{SliTeX}{sc}{%
        l%
        \kern-.035em%
        i%
      }%
    }%
    \HOLOGO@discretionary
    \kern-.06em%
    \hologo{TeX}%
  }%
}
%    \end{macrocode}
%    \end{macro}
%    \begin{macro}{\HoLogoBkm@SliTeX@narrow}
%    \begin{macrocode}
\def\HoLogoBkm@SliTeX@narrow#1{SliTeX}
%    \end{macrocode}
%    \end{macro}
%    \begin{macro}{\HoLogoHtml@SliTeX@narrow}
%    \begin{macrocode}
\def\HoLogoHtml@SliTeX@narrow#1{%
  \HoLogoCss@SliTeX@narrow
  \HOLOGO@Span{SliTeX-narrow}{%
    \HoLogoFont@font{SliTeX}{rm}{%
      S%
        \HOLOGO@Span{l}{l}%
        \HOLOGO@Span{i}{i}%
      \hologo{TeX}%
    }%
  }%
}
%    \end{macrocode}
%    \end{macro}
%    \begin{macro}{\HoLogoCss@SliTeX@narrow}
%    \begin{macrocode}
\def\HoLogoCss@SliTeX@narrow{%
  \Css{%
    span.HoLogo-SliTeX-narrow span.HoLogo-l{%
      margin-left:-.06em;%
      margin-right:-.035em;%
      font-variant:small-caps;%
    }%
  }%
  \Css{%
    span.HoLogo-SliTeX-narrow span.HoLogo-i{%
      margin-right:-.06em;%
      font-variant:small-caps;%
    }%
  }%
  \global\let\HoLogoCss@SliTeX@narrow\relax
}
%    \end{macrocode}
%    \end{macro}
%
% \paragraph{Macro set completion.}
%
%    \begin{macro}{\HoLogo@SLiTeX@simple}
%    \begin{macrocode}
\def\HoLogo@SLiTeX@simple{\HoLogo@SliTeX@simple}
%    \end{macrocode}
%    \end{macro}
%    \begin{macro}{\HoLogoBkm@SLiTeX@simple}
%    \begin{macrocode}
\def\HoLogoBkm@SLiTeX@simple{\HoLogoBkm@SliTeX@simple}
%    \end{macrocode}
%    \end{macro}
%    \begin{macro}{\HoLogoHtml@SLiTeX@simple}
%    \begin{macrocode}
\def\HoLogoHtml@SLiTeX@simple{\HoLogoHtml@SliTeX@simple}
%    \end{macrocode}
%    \end{macro}
%
%    \begin{macro}{\HoLogo@SLiTeX@narrow}
%    \begin{macrocode}
\def\HoLogo@SLiTeX@narrow{\HoLogo@SliTeX@narrow}
%    \end{macrocode}
%    \end{macro}
%    \begin{macro}{\HoLogoBkm@SLiTeX@narrow}
%    \begin{macrocode}
\def\HoLogoBkm@SLiTeX@narrow{\HoLogoBkm@SliTeX@narrow}
%    \end{macrocode}
%    \end{macro}
%    \begin{macro}{\HoLogoHtml@SLiTeX@narrow}
%    \begin{macrocode}
\def\HoLogoHtml@SLiTeX@narrow{\HoLogoHtml@SliTeX@narrow}
%    \end{macrocode}
%    \end{macro}
%
%    \begin{macro}{\HoLogo@SliTeX@lift}
%    \begin{macrocode}
\def\HoLogo@SliTeX@lift{\HoLogo@SLiTeX@lift}
%    \end{macrocode}
%    \end{macro}
%    \begin{macro}{\HoLogoBkm@SliTeX@lift}
%    \begin{macrocode}
\def\HoLogoBkm@SliTeX@lift{\HoLogoBkm@SLiTeX@lift}
%    \end{macrocode}
%    \end{macro}
%    \begin{macro}{\HoLogoHtml@SliTeX@lift}
%    \begin{macrocode}
\def\HoLogoHtml@SliTeX@lift{\HoLogoHtml@SLiTeX@lift}
%    \end{macrocode}
%    \end{macro}
%
% \paragraph{Defaults.}
%
%    \begin{macro}{\HoLogo@SLiTeX}
%    \begin{macrocode}
\def\HoLogo@SLiTeX{\HoLogo@SLiTeX@lift}
%    \end{macrocode}
%    \end{macro}
%    \begin{macro}{\HoLogoBkm@SLiTeX}
%    \begin{macrocode}
\def\HoLogoBkm@SLiTeX{\HoLogoBkm@SLiTeX@lift}
%    \end{macrocode}
%    \end{macro}
%    \begin{macro}{\HoLogoHtml@SLiTeX}
%    \begin{macrocode}
\def\HoLogoHtml@SLiTeX{\HoLogoHtml@SLiTeX@lift}
%    \end{macrocode}
%    \end{macro}
%
%    \begin{macro}{\HoLogo@SliTeX}
%    \begin{macrocode}
\def\HoLogo@SliTeX{\HoLogo@SliTeX@narrow}
%    \end{macrocode}
%    \end{macro}
%    \begin{macro}{\HoLogoBkm@SliTeX}
%    \begin{macrocode}
\def\HoLogoBkm@SliTeX{\HoLogoBkm@SliTeX@narrow}
%    \end{macrocode}
%    \end{macro}
%    \begin{macro}{\HoLogoHtml@SliTeX}
%    \begin{macrocode}
\def\HoLogoHtml@SliTeX{\HoLogoHtml@SliTeX@narrow}
%    \end{macrocode}
%    \end{macro}
%
% \subsubsection{\hologo{LuaTeX}}
%
%    \begin{macro}{\HoLogo@LuaTeX}
%    The kerning is an idea of Hans Hagen, see mailing list
%    `luatex at tug dot org' in March 2010.
%    \begin{macrocode}
\def\HoLogo@LuaTeX#1{%
  \HOLOGO@mbox{%
    Lua%
    \HOLOGO@NegativeKerning{aT,oT,To}%
    \hologo{TeX}%
  }%
}
%    \end{macrocode}
%    \end{macro}
%    \begin{macro}{\HoLogoHtml@LuaTeX}
%    \begin{macrocode}
\let\HoLogoHtml@LuaTeX\HoLogo@LuaTeX
%    \end{macrocode}
%    \end{macro}
%
% \subsubsection{\hologo{LuaLaTeX}}
%
%    \begin{macro}{\HoLogo@LuaLaTeX}
%    \begin{macrocode}
\def\HoLogo@LuaLaTeX#1{%
  \HOLOGO@mbox{%
    Lua%
    \hologo{LaTeX}%
  }%
}
%    \end{macrocode}
%    \end{macro}
%    \begin{macro}{\HoLogoHtml@LuaLaTeX}
%    \begin{macrocode}
\let\HoLogoHtml@LuaLaTeX\HoLogo@LuaLaTeX
%    \end{macrocode}
%    \end{macro}
%
% \subsubsection{\hologo{XeTeX}, \hologo{XeLaTeX}}
%
%    \begin{macro}{\HOLOGO@IfCharExists}
%    \begin{macrocode}
\ifluatex
  \ifnum\luatexversion<36 %
  \else
    \def\HOLOGO@IfCharExists#1{%
      \ifnum
        \directlua{%
           if luaotfload and luaotfload.aux then
             if luaotfload.aux.font_has_glyph(%
                    font.current(), \number#1) then % 	 
	       tex.print("1") % 	 
	     end % 	 
	   elseif font and font.fonts and font.current then %
            local f = font.fonts[font.current()]%
            if f.characters and f.characters[\number#1] then %
              tex.print("1")%
            end %
          end%
        }0=\ltx@zero
        \expandafter\ltx@secondoftwo
      \else
        \expandafter\ltx@firstoftwo
      \fi
    }%
  \fi
\fi
\ltx@IfUndefined{HOLOGO@IfCharExists}{%
  \def\HOLOGO@@IfCharExists#1{%
    \begingroup
      \tracinglostchars=\ltx@zero
      \setbox\ltx@zero=\hbox{%
        \kern7sp\char#1\relax
        \ifnum\lastkern>\ltx@zero
          \expandafter\aftergroup\csname iffalse\endcsname
        \else
          \expandafter\aftergroup\csname iftrue\endcsname
        \fi
      }%
      % \if{true|false} from \aftergroup
      \endgroup
      \expandafter\ltx@firstoftwo
    \else
      \endgroup
      \expandafter\ltx@secondoftwo
    \fi
  }%
  \ifxetex
    \ltx@IfUndefined{XeTeXfonttype}{}{%
      \ltx@IfUndefined{XeTeXcharglyph}{}{%
        \def\HOLOGO@IfCharExists#1{%
          \ifnum\XeTeXfonttype\font>\ltx@zero
            \expandafter\ltx@firstofthree
          \else
            \expandafter\ltx@gobble
          \fi
          {%
            \ifnum\XeTeXcharglyph#1>\ltx@zero
              \expandafter\ltx@firstoftwo
            \else
              \expandafter\ltx@secondoftwo
            \fi
          }%
          \HOLOGO@@IfCharExists{#1}%
        }%
      }%
    }%
  \fi
}{}
\ltx@ifundefined{HOLOGO@IfCharExists}{%
  \ifnum64=`\^^^^0040\relax % test for big chars of LuaTeX/XeTeX
    \let\HOLOGO@IfCharExists\HOLOGO@@IfCharExists
  \else
    \def\HOLOGO@IfCharExists#1{%
      \ifnum#1>255 %
        \expandafter\ltx@fourthoffour
      \fi
      \HOLOGO@@IfCharExists{#1}%
    }%
  \fi
}{}
%    \end{macrocode}
%    \end{macro}
%
%    \begin{macro}{\HoLogo@Xe}
%    Source: package \xpackage{dtklogos}
%    \begin{macrocode}
\def\HoLogo@Xe#1{%
  X%
  \kern-.1em\relax
  \HOLOGO@IfCharExists{"018E}{%
    \lower.5ex\hbox{\char"018E}%
  }{%
    \chardef\HOLOGO@choice=\ltx@zero
    \ifdim\fontdimen\ltx@one\font>0pt %
      \ltx@IfUndefined{rotatebox}{%
        \ltx@IfUndefined{pgftext}{%
          \ltx@IfUndefined{psscalebox}{%
            \ltx@IfUndefined{HOLOGO@ScaleBox@\hologoDriver}{%
            }{%
              \chardef\HOLOGO@choice=4 %
            }%
          }{%
            \chardef\HOLOGO@choice=3 %
          }%
        }{%
          \chardef\HOLOGO@choice=2 %
        }%
      }{%
        \chardef\HOLOGO@choice=1 %
      }%
      \ifcase\HOLOGO@choice
        \HOLOGO@WarningUnsupportedDriver{Xe}%
        e%
      \or % 1: \rotatebox
        \begingroup
          \setbox\ltx@zero\hbox{\rotatebox{180}{E}}%
          \ltx@LocDimenA=\dp\ltx@zero
          \advance\ltx@LocDimenA by -.5ex\relax
          \raise\ltx@LocDimenA\box\ltx@zero
        \endgroup
      \or % 2: \pgftext
        \lower.5ex\hbox{%
          \pgfpicture
            \pgftext[rotate=180]{E}%
          \endpgfpicture
        }%
      \or % 3: \psscalebox
        \begingroup
          \setbox\ltx@zero\hbox{\psscalebox{-1 -1}{E}}%
          \ltx@LocDimenA=\dp\ltx@zero
          \advance\ltx@LocDimenA by -.5ex\relax
          \raise\ltx@LocDimenA\box\ltx@zero
        \endgroup
      \or % 4: \HOLOGO@PointReflectBox
        \lower.5ex\hbox{\HOLOGO@PointReflectBox{E}}%
      \else
        \@PackageError{hologo}{Internal error (choice/it}\@ehc
      \fi
    \else
      \ltx@IfUndefined{reflectbox}{%
        \ltx@IfUndefined{pgftext}{%
          \ltx@IfUndefined{psscalebox}{%
            \ltx@IfUndefined{HOLOGO@ScaleBox@\hologoDriver}{%
            }{%
              \chardef\HOLOGO@choice=4 %
            }%
          }{%
            \chardef\HOLOGO@choice=3 %
          }%
        }{%
          \chardef\HOLOGO@choice=2 %
        }%
      }{%
        \chardef\HOLOGO@choice=1 %
      }%
      \ifcase\HOLOGO@choice
        \HOLOGO@WarningUnsupportedDriver{Xe}%
        e%
      \or % 1: reflectbox
        \lower.5ex\hbox{%
          \reflectbox{E}%
        }%
      \or % 2: \pgftext
        \lower.5ex\hbox{%
          \pgfpicture
            \pgftransformxscale{-1}%
            \pgftext{E}%
          \endpgfpicture
        }%
      \or % 3: \psscalebox
        \lower.5ex\hbox{%
          \psscalebox{-1 1}{E}%
        }%
      \or % 4: \HOLOGO@Reflectbox
        \lower.5ex\hbox{%
          \HOLOGO@ReflectBox{E}%
        }%
      \else
        \@PackageError{hologo}{Internal error (choice/up)}\@ehc
      \fi
    \fi
  }%
}
%    \end{macrocode}
%    \end{macro}
%    \begin{macro}{\HoLogoHtml@Xe}
%    \begin{macrocode}
\def\HoLogoHtml@Xe#1{%
  \HoLogoCss@Xe
  \HOLOGO@Span{Xe}{%
    X%
    \HOLOGO@Span{e}{%
      \HCode{&\ltx@hashchar x018e;}%
    }%
  }%
}
%    \end{macrocode}
%    \end{macro}
%    \begin{macro}{\HoLogoCss@Xe}
%    \begin{macrocode}
\def\HoLogoCss@Xe{%
  \Css{%
    span.HoLogo-Xe span.HoLogo-e{%
      position:relative;%
      top:.5ex;%
      left-margin:-.1em;%
    }%
  }%
  \global\let\HoLogoCss@Xe\relax
}
%    \end{macrocode}
%    \end{macro}
%
%    \begin{macro}{\HoLogo@XeTeX}
%    \begin{macrocode}
\def\HoLogo@XeTeX#1{%
  \hologo{Xe}%
  \kern-.15em\relax
  \hologo{TeX}%
}
%    \end{macrocode}
%    \end{macro}
%
%    \begin{macro}{\HoLogoHtml@XeTeX}
%    \begin{macrocode}
\def\HoLogoHtml@XeTeX#1{%
  \HoLogoCss@XeTeX
  \HOLOGO@Span{XeTeX}{%
    \hologo{Xe}%
    \hologo{TeX}%
  }%
}
%    \end{macrocode}
%    \end{macro}
%    \begin{macro}{\HoLogoCss@XeTeX}
%    \begin{macrocode}
\def\HoLogoCss@XeTeX{%
  \Css{%
    span.HoLogo-XeTeX span.HoLogo-TeX{%
      margin-left:-.15em;%
    }%
  }%
  \global\let\HoLogoCss@XeTeX\relax
}
%    \end{macrocode}
%    \end{macro}
%
%    \begin{macro}{\HoLogo@XeLaTeX}
%    \begin{macrocode}
\def\HoLogo@XeLaTeX#1{%
  \hologo{Xe}%
  \kern-.13em%
  \hologo{LaTeX}%
}
%    \end{macrocode}
%    \end{macro}
%    \begin{macro}{\HoLogoHtml@XeLaTeX}
%    \begin{macrocode}
\def\HoLogoHtml@XeLaTeX#1{%
  \HoLogoCss@XeLaTeX
  \HOLOGO@Span{XeLaTeX}{%
    \hologo{Xe}%
    \hologo{LaTeX}%
  }%
}
%    \end{macrocode}
%    \end{macro}
%    \begin{macro}{\HoLogoCss@XeLaTeX}
%    \begin{macrocode}
\def\HoLogoCss@XeLaTeX{%
  \Css{%
    span.HoLogo-XeLaTeX span.HoLogo-Xe{%
      margin-right:-.13em;%
    }%
  }%
  \global\let\HoLogoCss@XeLaTeX\relax
}
%    \end{macrocode}
%    \end{macro}
%
% \subsubsection{\hologo{pdfTeX}, \hologo{pdfLaTeX}}
%
%    \begin{macro}{\HoLogo@pdfTeX}
%    \begin{macrocode}
\def\HoLogo@pdfTeX#1{%
  \HOLOGO@mbox{%
    #1{p}{P}df\hologo{TeX}%
  }%
}
%    \end{macrocode}
%    \end{macro}
%    \begin{macro}{\HoLogoCs@pdfTeX}
%    \begin{macrocode}
\def\HoLogoCs@pdfTeX#1{#1{p}{P}dfTeX}
%    \end{macrocode}
%    \end{macro}
%    \begin{macro}{\HoLogoBkm@pdfTeX}
%    \begin{macrocode}
\def\HoLogoBkm@pdfTeX#1{%
  #1{p}{P}df\hologo{TeX}%
}
%    \end{macrocode}
%    \end{macro}
%    \begin{macro}{\HoLogoHtml@pdfTeX}
%    \begin{macrocode}
\let\HoLogoHtml@pdfTeX\HoLogo@pdfTeX
%    \end{macrocode}
%    \end{macro}
%
%    \begin{macro}{\HoLogo@pdfLaTeX}
%    \begin{macrocode}
\def\HoLogo@pdfLaTeX#1{%
  \HOLOGO@mbox{%
    #1{p}{P}df\hologo{LaTeX}%
  }%
}
%    \end{macrocode}
%    \end{macro}
%    \begin{macro}{\HoLogoCs@pdfLaTeX}
%    \begin{macrocode}
\def\HoLogoCs@pdfLaTeX#1{#1{p}{P}dfLaTeX}
%    \end{macrocode}
%    \end{macro}
%    \begin{macro}{\HoLogoBkm@pdfLaTeX}
%    \begin{macrocode}
\def\HoLogoBkm@pdfLaTeX#1{%
  #1{p}{P}df\hologo{LaTeX}%
}
%    \end{macrocode}
%    \end{macro}
%    \begin{macro}{\HoLogoHtml@pdfLaTeX}
%    \begin{macrocode}
\let\HoLogoHtml@pdfLaTeX\HoLogo@pdfLaTeX
%    \end{macrocode}
%    \end{macro}
%
% \subsubsection{\hologo{VTeX}}
%
%    \begin{macro}{\HoLogo@VTeX}
%    \begin{macrocode}
\def\HoLogo@VTeX#1{%
  \HOLOGO@mbox{%
    V\hologo{TeX}%
  }%
}
%    \end{macrocode}
%    \end{macro}
%    \begin{macro}{\HoLogoHtml@VTeX}
%    \begin{macrocode}
\let\HoLogoHtml@VTeX\HoLogo@VTeX
%    \end{macrocode}
%    \end{macro}
%
% \subsubsection{\hologo{AmS}, \dots}
%
%    Source: class \xclass{amsdtx}
%
%    \begin{macro}{\HoLogo@AmS}
%    \begin{macrocode}
\def\HoLogo@AmS#1{%
  \HoLogoFont@font{AmS}{sy}{%
    A%
    \kern-.1667em%
    \lower.5ex\hbox{M}%
    \kern-.125em%
    S%
  }%
}
%    \end{macrocode}
%    \end{macro}
%    \begin{macro}{\HoLogoBkm@AmS}
%    \begin{macrocode}
\def\HoLogoBkm@AmS#1{AmS}
%    \end{macrocode}
%    \end{macro}
%    \begin{macro}{\HoLogoHtml@AmS}
%    \begin{macrocode}
\def\HoLogoHtml@AmS#1{%
  \HoLogoCss@AmS
%  \HoLogoFont@font{AmS}{sy}{%
    \HOLOGO@Span{AmS}{%
      A%
      \HOLOGO@Span{M}{M}%
      S%
    }%
%   }%
}
%    \end{macrocode}
%    \end{macro}
%    \begin{macro}{\HoLogoCss@AmS}
%    \begin{macrocode}
\def\HoLogoCss@AmS{%
  \Css{%
    span.HoLogo-AmS span.HoLogo-M{%
      position:relative;%
      top:.5ex;%
      margin-left:-.1667em;%
      margin-right:-.125em;%
      text-decoration:none;%
    }%
  }%
  \global\let\HoLogoCss@AmS\relax
}
%    \end{macrocode}
%    \end{macro}
%
%    \begin{macro}{\HoLogo@AmSTeX}
%    \begin{macrocode}
\def\HoLogo@AmSTeX#1{%
  \hologo{AmS}%
  \HOLOGO@hyphen
  \hologo{TeX}%
}
%    \end{macrocode}
%    \end{macro}
%    \begin{macro}{\HoLogoBkm@AmSTeX}
%    \begin{macrocode}
\def\HoLogoBkm@AmSTeX#1{AmS-TeX}%
%    \end{macrocode}
%    \end{macro}
%    \begin{macro}{\HoLogoHtml@AmSTeX}
%    \begin{macrocode}
\let\HoLogoHtml@AmSTeX\HoLogo@AmSTeX
%    \end{macrocode}
%    \end{macro}
%
%    \begin{macro}{\HoLogo@AmSLaTeX}
%    \begin{macrocode}
\def\HoLogo@AmSLaTeX#1{%
  \hologo{AmS}%
  \HOLOGO@hyphen
  \hologo{LaTeX}%
}
%    \end{macrocode}
%    \end{macro}
%    \begin{macro}{\HoLogoBkm@AmSLaTeX}
%    \begin{macrocode}
\def\HoLogoBkm@AmSLaTeX#1{AmS-LaTeX}%
%    \end{macrocode}
%    \end{macro}
%    \begin{macro}{\HoLogoHtml@AmSLaTeX}
%    \begin{macrocode}
\let\HoLogoHtml@AmSLaTeX\HoLogo@AmSLaTeX
%    \end{macrocode}
%    \end{macro}
%
% \subsubsection{\hologo{BibTeX}}
%
%    \begin{macro}{\HoLogo@BibTeX@sc}
%    A definition of \hologo{BibTeX} is provided in
%    the documentation source for the manual of \hologo{BibTeX}
%    \cite{btxdoc}.
%\begin{quote}
%\begin{verbatim}
%\def\BibTeX{%
%  {%
%    \rm
%    B%
%    \kern-.05em%
%    {%
%      \sc
%      i%
%      \kern-.025em %
%      b%
%    }%
%    \kern-.08em
%    T%
%    \kern-.1667em%
%    \lower.7ex\hbox{E}%
%    \kern-.125em%
%    X%
%  }%
%}
%\end{verbatim}
%\end{quote}
%    \begin{macrocode}
\def\HoLogo@BibTeX@sc#1{%
  B%
  \kern-.05em%
  \HoLogoFont@font{BibTeX}{sc}{%
    i%
    \kern-.025em%
    b%
  }%
  \HOLOGO@discretionary
  \kern-.08em%
  \hologo{TeX}%
}
%    \end{macrocode}
%    \end{macro}
%    \begin{macro}{\HoLogoHtml@BibTeX@sc}
%    \begin{macrocode}
\def\HoLogoHtml@BibTeX@sc#1{%
  \HoLogoCss@BibTeX@sc
  \HOLOGO@Span{BibTeX-sc}{%
    B%
    \HOLOGO@Span{i}{i}%
    \HOLOGO@Span{b}{b}%
    \hologo{TeX}%
  }%
}
%    \end{macrocode}
%    \end{macro}
%    \begin{macro}{\HoLogoCss@BibTeX@sc}
%    \begin{macrocode}
\def\HoLogoCss@BibTeX@sc{%
  \Css{%
    span.HoLogo-BibTeX-sc span.HoLogo-i{%
      margin-left:-.05em;%
      margin-right:-.025em;%
      font-variant:small-caps;%
    }%
  }%
  \Css{%
    span.HoLogo-BibTeX-sc span.HoLogo-b{%
      margin-right:-.08em;%
      font-variant:small-caps;%
    }%
  }%
  \global\let\HoLogoCss@BibTeX@sc\relax
}
%    \end{macrocode}
%    \end{macro}
%
%    \begin{macro}{\HoLogo@BibTeX@sf}
%    Variant \xoption{sf} avoids trouble with unavailable
%    small caps fonts (e.g., bold versions of Computer Modern or
%    Latin Modern). The definition is taken from
%    package \xpackage{dtklogos} \cite{dtklogos}.
%\begin{quote}
%\begin{verbatim}
%\DeclareRobustCommand{\BibTeX}{%
%  B%
%  \kern-.05em%
%  \hbox{%
%    $\m@th$% %% force math size calculations
%    \csname S@\f@size\endcsname
%    \fontsize\sf@size\z@
%    \math@fontsfalse
%    \selectfont
%    I%
%    \kern-.025em%
%    B
%  }%
%  \kern-.08em%
%  \-%
%  \TeX
%}
%\end{verbatim}
%\end{quote}
%    \begin{macrocode}
\def\HoLogo@BibTeX@sf#1{%
  B%
  \kern-.05em%
  \HoLogoFont@font{BibTeX}{bibsf}{%
    I%
    \kern-.025em%
    B%
  }%
  \HOLOGO@discretionary
  \kern-.08em%
  \hologo{TeX}%
}
%    \end{macrocode}
%    \end{macro}
%    \begin{macro}{\HoLogoHtml@BibTeX@sf}
%    \begin{macrocode}
\def\HoLogoHtml@BibTeX@sf#1{%
  \HoLogoCss@BibTeX@sf
  \HOLOGO@Span{BibTeX-sf}{%
    B%
    \HoLogoFont@font{BibTeX}{bibsf}{%
      \HOLOGO@Span{i}{I}%
      B%
    }%
    \hologo{TeX}%
  }%
}
%    \end{macrocode}
%    \end{macro}
%    \begin{macro}{\HoLogoCss@BibTeX@sf}
%    \begin{macrocode}
\def\HoLogoCss@BibTeX@sf{%
  \Css{%
    span.HoLogo-BibTeX-sf span.HoLogo-i{%
      margin-left:-.05em;%
      margin-right:-.025em;%
    }%
  }%
  \Css{%
    span.HoLogo-BibTeX-sf span.HoLogo-TeX{%
      margin-left:-.08em;%
    }%
  }%
  \global\let\HoLogoCss@BibTeX@sf\relax
}
%    \end{macrocode}
%    \end{macro}
%
%    \begin{macro}{\HoLogo@BibTeX}
%    \begin{macrocode}
\def\HoLogo@BibTeX{\HoLogo@BibTeX@sf}
%    \end{macrocode}
%    \end{macro}
%    \begin{macro}{\HoLogoHtml@BibTeX}
%    \begin{macrocode}
\def\HoLogoHtml@BibTeX{\HoLogoHtml@BibTeX@sf}
%    \end{macrocode}
%    \end{macro}
%
% \subsubsection{\hologo{BibTeX8}}
%
%    \begin{macro}{\HoLogo@BibTeX8}
%    \begin{macrocode}
\expandafter\def\csname HoLogo@BibTeX8\endcsname#1{%
  \hologo{BibTeX}%
  8%
}
%    \end{macrocode}
%    \end{macro}
%
%    \begin{macro}{\HoLogoBkm@BibTeX8}
%    \begin{macrocode}
\expandafter\def\csname HoLogoBkm@BibTeX8\endcsname#1{%
  \hologo{BibTeX}%
  8%
}
%    \end{macrocode}
%    \end{macro}
%    \begin{macro}{\HoLogoHtml@BibTeX8}
%    \begin{macrocode}
\expandafter
\let\csname HoLogoHtml@BibTeX8\expandafter\endcsname
\csname HoLogo@BibTeX8\endcsname
%    \end{macrocode}
%    \end{macro}
%
% \subsubsection{\hologo{ConTeXt}}
%
%    \begin{macro}{\HoLogo@ConTeXt@simple}
%    \begin{macrocode}
\def\HoLogo@ConTeXt@simple#1{%
  \HOLOGO@mbox{Con}%
  \HOLOGO@discretionary
  \HOLOGO@mbox{\hologo{TeX}t}%
}
%    \end{macrocode}
%    \end{macro}
%    \begin{macro}{\HoLogoHtml@ConTeXt@simple}
%    \begin{macrocode}
\let\HoLogoHtml@ConTeXt@simple\HoLogo@ConTeXt@simple
%    \end{macrocode}
%    \end{macro}
%
%    \begin{macro}{\HoLogo@ConTeXt@narrow}
%    This definition of logo \hologo{ConTeXt} with variant \xoption{narrow}
%    comes from TUGboat's class \xclass{ltugboat} (version 2010/11/15 v2.8).
%    \begin{macrocode}
\def\HoLogo@ConTeXt@narrow#1{%
  \HOLOGO@mbox{C\kern-.0333emon}%
  \HOLOGO@discretionary
  \kern-.0667em%
  \HOLOGO@mbox{\hologo{TeX}\kern-.0333emt}%
}
%    \end{macrocode}
%    \end{macro}
%    \begin{macro}{\HoLogoHtml@ConTeXt@narrow}
%    \begin{macrocode}
\def\HoLogoHtml@ConTeXt@narrow#1{%
  \HoLogoCss@ConTeXt@narrow
  \HOLOGO@Span{ConTeXt-narrow}{%
    \HOLOGO@Span{C}{C}%
    on%
    \hologo{TeX}%
    t%
  }%
}
%    \end{macrocode}
%    \end{macro}
%    \begin{macro}{\HoLogoCss@ConTeXt@narrow}
%    \begin{macrocode}
\def\HoLogoCss@ConTeXt@narrow{%
  \Css{%
    span.HoLogo-ConTeXt-narrow span.HoLogo-C{%
      margin-left:-.0333em;%
    }%
  }%
  \Css{%
    span.HoLogo-ConTeXt-narrow span.HoLogo-TeX{%
      margin-left:-.0667em;%
      margin-right:-.0333em;%
    }%
  }%
  \global\let\HoLogoCss@ConTeXt@narrow\relax
}
%    \end{macrocode}
%    \end{macro}
%
%    \begin{macro}{\HoLogo@ConTeXt}
%    \begin{macrocode}
\def\HoLogo@ConTeXt{\HoLogo@ConTeXt@narrow}
%    \end{macrocode}
%    \end{macro}
%    \begin{macro}{\HoLogoHtml@ConTeXt}
%    \begin{macrocode}
\def\HoLogoHtml@ConTeXt{\HoLogoHtml@ConTeXt@narrow}
%    \end{macrocode}
%    \end{macro}
%
% \subsubsection{\hologo{emTeX}}
%
%    \begin{macro}{\HoLogo@emTeX}
%    \begin{macrocode}
\def\HoLogo@emTeX#1{%
  \HOLOGO@mbox{#1{e}{E}m}%
  \HOLOGO@discretionary
  \hologo{TeX}%
}
%    \end{macrocode}
%    \end{macro}
%    \begin{macro}{\HoLogoCs@emTeX}
%    \begin{macrocode}
\def\HoLogoCs@emTeX#1{#1{e}{E}mTeX}%
%    \end{macrocode}
%    \end{macro}
%    \begin{macro}{\HoLogoBkm@emTeX}
%    \begin{macrocode}
\def\HoLogoBkm@emTeX#1{%
  #1{e}{E}m\hologo{TeX}%
}
%    \end{macrocode}
%    \end{macro}
%    \begin{macro}{\HoLogoHtml@emTeX}
%    \begin{macrocode}
\let\HoLogoHtml@emTeX\HoLogo@emTeX
%    \end{macrocode}
%    \end{macro}
%
% \subsubsection{\hologo{ExTeX}}
%
%    \begin{macro}{\HoLogo@ExTeX}
%    The definition is taken from the FAQ of the
%    project \hologo{ExTeX}
%    \cite{ExTeX-FAQ}.
%\begin{quote}
%\begin{verbatim}
%\def\ExTeX{%
%  \textrm{% Logo always with serifs
%    \ensuremath{%
%      \textstyle
%      \varepsilon_{%
%        \kern-0.15em%
%        \mathcal{X}%
%      }%
%    }%
%    \kern-.15em%
%    \TeX
%  }%
%}
%\end{verbatim}
%\end{quote}
%    \begin{macrocode}
\def\HoLogo@ExTeX#1{%
  \HoLogoFont@font{ExTeX}{rm}{%
    \ltx@mbox{%
      \HOLOGO@MathSetup
      $%
        \textstyle
        \varepsilon_{%
          \kern-0.15em%
          \HoLogoFont@font{ExTeX}{sy}{X}%
        }%
      $%
    }%
    \HOLOGO@discretionary
    \kern-.15em%
    \hologo{TeX}%
  }%
}
%    \end{macrocode}
%    \end{macro}
%    \begin{macro}{\HoLogoHtml@ExTeX}
%    \begin{macrocode}
\def\HoLogoHtml@ExTeX#1{%
  \HoLogoCss@ExTeX
  \HoLogoFont@font{ExTeX}{rm}{%
    \HOLOGO@Span{ExTeX}{%
      \ltx@mbox{%
        \HOLOGO@MathSetup
        $\textstyle\varepsilon$%
        \HOLOGO@Span{X}{$\textstyle\chi$}%
        \hologo{TeX}%
      }%
    }%
  }%
}
%    \end{macrocode}
%    \end{macro}
%    \begin{macro}{\HoLogoBkm@ExTeX}
%    \begin{macrocode}
\def\HoLogoBkm@ExTeX#1{%
  \HOLOGO@PdfdocUnicode{#1{e}{E}x}{\textepsilon\textchi}%
  \hologo{TeX}%
}
%    \end{macrocode}
%    \end{macro}
%    \begin{macro}{\HoLogoCss@ExTeX}
%    \begin{macrocode}
\def\HoLogoCss@ExTeX{%
  \Css{%
    span.HoLogo-ExTeX{%
      font-family:serif;%
    }%
  }%
  \Css{%
    span.HoLogo-ExTeX span.HoLogo-TeX{%
      margin-left:-.15em;%
    }%
  }%
  \global\let\HoLogoCss@ExTeX\relax
}
%    \end{macrocode}
%    \end{macro}
%
% \subsubsection{\hologo{MiKTeX}}
%
%    \begin{macro}{\HoLogo@MiKTeX}
%    \begin{macrocode}
\def\HoLogo@MiKTeX#1{%
  \HOLOGO@mbox{MiK}%
  \HOLOGO@discretionary
  \hologo{TeX}%
}
%    \end{macrocode}
%    \end{macro}
%    \begin{macro}{\HoLogoHtml@MiKTeX}
%    \begin{macrocode}
\let\HoLogoHtml@MiKTeX\HoLogo@MiKTeX
%    \end{macrocode}
%    \end{macro}
%
% \subsubsection{\hologo{OzTeX} and friends}
%
%    Source: \hologo{OzTeX} FAQ \cite{OzTeX}:
%    \begin{quote}
%      |\def\OzTeX{O\kern-.03em z\kern-.15em\TeX}|\\
%      (There is no kerning in OzMF, OzMP and OzTtH.)
%    \end{quote}
%
%    \begin{macro}{\HoLogo@OzTeX}
%    \begin{macrocode}
\def\HoLogo@OzTeX#1{%
  O%
  \kern-.03em %
  z%
  \kern-.15em %
  \hologo{TeX}%
}
%    \end{macrocode}
%    \end{macro}
%    \begin{macro}{\HoLogoHtml@OzTeX}
%    \begin{macrocode}
\def\HoLogoHtml@OzTeX#1{%
  \HoLogoCss@OzTeX
  \HOLOGO@Span{OzTeX}{%
    O%
    \HOLOGO@Span{z}{z}%
    \hologo{TeX}%
  }%
}
%    \end{macrocode}
%    \end{macro}
%    \begin{macro}{\HoLogoCss@OzTeX}
%    \begin{macrocode}
\def\HoLogoCss@OzTeX{%
  \Css{%
    span.HoLogo-OzTeX span.HoLogo-z{%
      margin-left:-.03em;%
      margin-right:-.15em;%
    }%
  }%
  \global\let\HoLogoCss@OzTeX\relax
}
%    \end{macrocode}
%    \end{macro}
%
%    \begin{macro}{\HoLogo@OzMF}
%    \begin{macrocode}
\def\HoLogo@OzMF#1{%
  \HOLOGO@mbox{OzMF}%
}
%    \end{macrocode}
%    \end{macro}
%    \begin{macro}{\HoLogo@OzMP}
%    \begin{macrocode}
\def\HoLogo@OzMP#1{%
  \HOLOGO@mbox{OzMP}%
}
%    \end{macrocode}
%    \end{macro}
%    \begin{macro}{\HoLogo@OzTtH}
%    \begin{macrocode}
\def\HoLogo@OzTtH#1{%
  \HOLOGO@mbox{OzTtH}%
}
%    \end{macrocode}
%    \end{macro}
%
% \subsubsection{\hologo{PCTeX}}
%
%    \begin{macro}{\HoLogo@PCTeX}
%    \begin{macrocode}
\def\HoLogo@PCTeX#1{%
  \HOLOGO@mbox{PC}%
  \hologo{TeX}%
}
%    \end{macrocode}
%    \end{macro}
%    \begin{macro}{\HoLogoHtml@PCTeX}
%    \begin{macrocode}
\let\HoLogoHtml@PCTeX\HoLogo@PCTeX
%    \end{macrocode}
%    \end{macro}
%
% \subsubsection{\hologo{PiCTeX}}
%
%    The original definitions from \xfile{pictex.tex} \cite{PiCTeX}:
%\begin{quote}
%\begin{verbatim}
%\def\PiC{%
%  P%
%  \kern-.12em%
%  \lower.5ex\hbox{I}%
%  \kern-.075em%
%  C%
%}
%\def\PiCTeX{%
%  \PiC
%  \kern-.11em%
%  \TeX
%}
%\end{verbatim}
%\end{quote}
%
%    \begin{macro}{\HoLogo@PiC}
%    \begin{macrocode}
\def\HoLogo@PiC#1{%
  P%
  \kern-.12em%
  \lower.5ex\hbox{I}%
  \kern-.075em%
  C%
  \HOLOGO@SpaceFactor
}
%    \end{macrocode}
%    \end{macro}
%    \begin{macro}{\HoLogoHtml@PiC}
%    \begin{macrocode}
\def\HoLogoHtml@PiC#1{%
  \HoLogoCss@PiC
  \HOLOGO@Span{PiC}{%
    P%
    \HOLOGO@Span{i}{I}%
    C%
  }%
}
%    \end{macrocode}
%    \end{macro}
%    \begin{macro}{\HoLogoCss@PiC}
%    \begin{macrocode}
\def\HoLogoCss@PiC{%
  \Css{%
    span.HoLogo-PiC span.HoLogo-i{%
      position:relative;%
      top:.5ex;%
      margin-left:-.12em;%
      margin-right:-.075em;%
      text-decoration:none;%
    }%
  }%
  \global\let\HoLogoCss@PiC\relax
}
%    \end{macrocode}
%    \end{macro}
%
%    \begin{macro}{\HoLogo@PiCTeX}
%    \begin{macrocode}
\def\HoLogo@PiCTeX#1{%
  \hologo{PiC}%
  \HOLOGO@discretionary
  \kern-.11em%
  \hologo{TeX}%
}
%    \end{macrocode}
%    \end{macro}
%    \begin{macro}{\HoLogoHtml@PiCTeX}
%    \begin{macrocode}
\def\HoLogoHtml@PiCTeX#1{%
  \HoLogoCss@PiCTeX
  \HOLOGO@Span{PiCTeX}{%
    \hologo{PiC}%
    \hologo{TeX}%
  }%
}
%    \end{macrocode}
%    \end{macro}
%    \begin{macro}{\HoLogoCss@PiCTeX}
%    \begin{macrocode}
\def\HoLogoCss@PiCTeX{%
  \Css{%
    span.HoLogo-PiCTeX span.HoLogo-PiC{%
      margin-right:-.11em;%
    }%
  }%
  \global\let\HoLogoCss@PiCTeX\relax
}
%    \end{macrocode}
%    \end{macro}
%
% \subsubsection{\hologo{teTeX}}
%
%    \begin{macro}{\HoLogo@teTeX}
%    \begin{macrocode}
\def\HoLogo@teTeX#1{%
  \HOLOGO@mbox{#1{t}{T}e}%
  \HOLOGO@discretionary
  \hologo{TeX}%
}
%    \end{macrocode}
%    \end{macro}
%    \begin{macro}{\HoLogoCs@teTeX}
%    \begin{macrocode}
\def\HoLogoCs@teTeX#1{#1{t}{T}dfTeX}
%    \end{macrocode}
%    \end{macro}
%    \begin{macro}{\HoLogoBkm@teTeX}
%    \begin{macrocode}
\def\HoLogoBkm@teTeX#1{%
  #1{t}{T}e\hologo{TeX}%
}
%    \end{macrocode}
%    \end{macro}
%    \begin{macro}{\HoLogoHtml@teTeX}
%    \begin{macrocode}
\let\HoLogoHtml@teTeX\HoLogo@teTeX
%    \end{macrocode}
%    \end{macro}
%
% \subsubsection{\hologo{TeX4ht}}
%
%    \begin{macro}{\HoLogo@TeX4ht}
%    \begin{macrocode}
\expandafter\def\csname HoLogo@TeX4ht\endcsname#1{%
  \HOLOGO@mbox{\hologo{TeX}4ht}%
}
%    \end{macrocode}
%    \end{macro}
%    \begin{macro}{\HoLogoHtml@TeX4ht}
%    \begin{macrocode}
\expandafter
\let\csname HoLogoHtml@TeX4ht\expandafter\endcsname
\csname HoLogo@TeX4ht\endcsname
%    \end{macrocode}
%    \end{macro}
%
%
% \subsubsection{\hologo{SageTeX}}
%
%    \begin{macro}{\HoLogo@SageTeX}
%    \begin{macrocode}
\def\HoLogo@SageTeX#1{%
  \HOLOGO@mbox{Sage}%
  \HOLOGO@discretionary
  \HOLOGO@NegativeKerning{eT,oT,To}%
  \hologo{TeX}%
}
%    \end{macrocode}
%    \end{macro}
%    \begin{macro}{\HoLogoHtml@SageTeX}
%    \begin{macrocode}
\let\HoLogoHtml@SageTeX\HoLogo@SageTeX
%    \end{macrocode}
%    \end{macro}
%
% \subsection{\hologo{METAFONT} and friends}
%
%    \begin{macro}{\HoLogo@METAFONT}
%    \begin{macrocode}
\def\HoLogo@METAFONT#1{%
  \HoLogoFont@font{METAFONT}{logo}{%
    \HOLOGO@mbox{META}%
    \HOLOGO@discretionary
    \HOLOGO@mbox{FONT}%
  }%
}
%    \end{macrocode}
%    \end{macro}
%
%    \begin{macro}{\HoLogo@METAPOST}
%    \begin{macrocode}
\def\HoLogo@METAPOST#1{%
  \HoLogoFont@font{METAPOST}{logo}{%
    \HOLOGO@mbox{META}%
    \HOLOGO@discretionary
    \HOLOGO@mbox{POST}%
  }%
}
%    \end{macrocode}
%    \end{macro}
%
%    \begin{macro}{\HoLogo@MetaFun}
%    \begin{macrocode}
\def\HoLogo@MetaFun#1{%
  \HOLOGO@mbox{Meta}%
  \HOLOGO@discretionary
  \HOLOGO@mbox{Fun}%
}
%    \end{macrocode}
%    \end{macro}
%
%    \begin{macro}{\HoLogo@MetaPost}
%    \begin{macrocode}
\def\HoLogo@MetaPost#1{%
  \HOLOGO@mbox{Meta}%
  \HOLOGO@discretionary
  \HOLOGO@mbox{Post}%
}
%    \end{macrocode}
%    \end{macro}
%
% \subsection{Others}
%
% \subsubsection{\hologo{biber}}
%
%    \begin{macro}{\HoLogo@biber}
%    \begin{macrocode}
\def\HoLogo@biber#1{%
  \HOLOGO@mbox{#1{b}{B}i}%
  \HOLOGO@discretionary
  \HOLOGO@mbox{ber}%
}
%    \end{macrocode}
%    \end{macro}
%    \begin{macro}{\HoLogoCs@biber}
%    \begin{macrocode}
\def\HoLogoCs@biber#1{#1{b}{B}iber}
%    \end{macrocode}
%    \end{macro}
%    \begin{macro}{\HoLogoBkm@biber}
%    \begin{macrocode}
\def\HoLogoBkm@biber#1{%
  #1{b}{B}iber%
}
%    \end{macrocode}
%    \end{macro}
%    \begin{macro}{\HoLogoHtml@biber}
%    \begin{macrocode}
\let\HoLogoHtml@biber\HoLogo@biber
%    \end{macrocode}
%    \end{macro}
%
% \subsubsection{\hologo{KOMAScript}}
%
%    \begin{macro}{\HoLogo@KOMAScript}
%    The definition for \hologo{KOMAScript} is taken
%    from \hologo{KOMAScript} (\xfile{scrlogo.dtx}, reformatted) \cite{scrlogo}:
%\begin{quote}
%\begin{verbatim}
%\@ifundefined{KOMAScript}{%
%  \DeclareRobustCommand{\KOMAScript}{%
%    \textsf{%
%      K\kern.05em O\kern.05emM\kern.05em A%
%      \kern.1em-\kern.1em %
%      Script%
%    }%
%  }%
%}{}
%\end{verbatim}
%\end{quote}
%    \begin{macrocode}
\def\HoLogo@KOMAScript#1{%
  \HoLogoFont@font{KOMAScript}{sf}{%
    \HOLOGO@mbox{%
      K\kern.05em%
      O\kern.05em%
      M\kern.05em%
      A%
    }%
    \kern.1em%
    \HOLOGO@hyphen
    \kern.1em%
    \HOLOGO@mbox{Script}%
  }%
}
%    \end{macrocode}
%    \end{macro}
%    \begin{macro}{\HoLogoBkm@KOMAScript}
%    \begin{macrocode}
\def\HoLogoBkm@KOMAScript#1{%
  KOMA-Script%
}
%    \end{macrocode}
%    \end{macro}
%    \begin{macro}{\HoLogoHtml@KOMAScript}
%    \begin{macrocode}
\def\HoLogoHtml@KOMAScript#1{%
  \HoLogoCss@KOMAScript
  \HoLogoFont@font{KOMAScript}{sf}{%
    \HOLOGO@Span{KOMAScript}{%
      K%
      \HOLOGO@Span{O}{O}%
      M%
      \HOLOGO@Span{A}{A}%
      \HOLOGO@Span{hyphen}{-}%
      Script%
    }%
  }%
}
%    \end{macrocode}
%    \end{macro}
%    \begin{macro}{\HoLogoCss@KOMAScript}
%    \begin{macrocode}
\def\HoLogoCss@KOMAScript{%
  \Css{%
    span.HoLogo-KOMAScript{%
      font-family:sans-serif;%
    }%
  }%
  \Css{%
    span.HoLogo-KOMAScript span.HoLogo-O{%
      padding-left:.05em;%
      padding-right:.05em;%
    }%
  }%
  \Css{%
    span.HoLogo-KOMAScript span.HoLogo-A{%
      padding-left:.05em;%
    }%
  }%
  \Css{%
    span.HoLogo-KOMAScript span.HoLogo-hyphen{%
      padding-left:.1em;%
      padding-right:.1em;%
    }%
  }%
  \global\let\HoLogoCss@KOMAScript\relax
}
%    \end{macrocode}
%    \end{macro}
%
% \subsubsection{\hologo{LyX}}
%
%    \begin{macro}{\HoLogo@LyX}
%    The definition is taken from the documentation source files
%    of \hologo{LyX}, \xfile{Intro.lyx} \cite{LyX}:
%\begin{quote}
%\begin{verbatim}
%\def\LyX{%
%  \texorpdfstring{%
%    L\kern-.1667em\lower.25em\hbox{Y}\kern-.125emX\@%
%  }{%
%    LyX%
%  }%
%}
%\end{verbatim}
%\end{quote}
%    \begin{macrocode}
\def\HoLogo@LyX#1{%
  L%
  \kern-.1667em%
  \lower.25em\hbox{Y}%
  \kern-.125em%
  X%
  \HOLOGO@SpaceFactor
}
%    \end{macrocode}
%    \end{macro}
%    \begin{macro}{\HoLogoHtml@LyX}
%    \begin{macrocode}
\def\HoLogoHtml@LyX#1{%
  \HoLogoCss@LyX
  \HOLOGO@Span{LyX}{%
    L%
    \HOLOGO@Span{y}{Y}%
    X%
  }%
}
%    \end{macrocode}
%    \end{macro}
%    \begin{macro}{\HoLogoCss@LyX}
%    \begin{macrocode}
\def\HoLogoCss@LyX{%
  \Css{%
    span.HoLogo-LyX span.HoLogo-y{%
      position:relative;%
      top:.25em;%
      margin-left:-.1667em;%
      margin-right:-.125em;%
      text-decoration:none;%
    }%
  }%
  \global\let\HoLogoCss@LyX\relax
}
%    \end{macrocode}
%    \end{macro}
%
% \subsubsection{\hologo{NTS}}
%
%    \begin{macro}{\HoLogo@NTS}
%    Definition for \hologo{NTS} can be found in
%    package \xpackage{etex\textunderscore man} for the \hologo{eTeX} manual \cite{etexman}
%    and in package \xpackage{dtklogos} \cite{dtklogos}:
%\begin{quote}
%\begin{verbatim}
%\def\NTS{%
%  \leavevmode
%  \hbox{%
%    $%
%      \cal N%
%      \kern-0.35em%
%      \lower0.5ex\hbox{$\cal T$}%
%      \kern-0.2em%
%      S%
%    $%
%  }%
%}
%\end{verbatim}
%\end{quote}
%    \begin{macrocode}
\def\HoLogo@NTS#1{%
  \HoLogoFont@font{NTS}{sy}{%
    N\/%
    \kern-.35em%
    \lower.5ex\hbox{T\/}%
    \kern-.2em%
    S\/%
  }%
  \HOLOGO@SpaceFactor
}
%    \end{macrocode}
%    \end{macro}
%
% \subsubsection{\Hologo{TTH} (\hologo{TeX} to HTML translator)}
%
%    Source: \url{http://hutchinson.belmont.ma.us/tth/}
%    In the HTML source the second `T' is printed as subscript.
%\begin{quote}
%\begin{verbatim}
%T<sub>T</sub>H
%\end{verbatim}
%\end{quote}
%    \begin{macro}{\HoLogo@TTH}
%    \begin{macrocode}
\def\HoLogo@TTH#1{%
  \ltx@mbox{%
    T\HOLOGO@SubScript{T}H%
  }%
  \HOLOGO@SpaceFactor
}
%    \end{macrocode}
%    \end{macro}
%
%    \begin{macro}{\HoLogoHtml@TTH}
%    \begin{macrocode}
\def\HoLogoHtml@TTH#1{%
  T\HCode{<sub>}T\HCode{</sub>}H%
}
%    \end{macrocode}
%    \end{macro}
%
% \subsubsection{\Hologo{HanTheThanh}}
%
%    Partial source: Package \xpackage{dtklogos}.
%    The double accent is U+1EBF (latin small letter e with circumflex
%    and acute).
%    \begin{macro}{\HoLogo@HanTheThanh}
%    \begin{macrocode}
\def\HoLogo@HanTheThanh#1{%
  \ltx@mbox{H\`an}%
  \HOLOGO@space
  \ltx@mbox{%
    Th%
    \HOLOGO@IfCharExists{"1EBF}{%
      \char"1EBF\relax
    }{%
      \^e\hbox to 0pt{\hss\raise .5ex\hbox{\'{}}}%
    }%
  }%
  \HOLOGO@space
  \ltx@mbox{Th\`anh}%
}
%    \end{macrocode}
%    \end{macro}
%    \begin{macro}{\HoLogoBkm@HanTheThanh}
%    \begin{macrocode}
\def\HoLogoBkm@HanTheThanh#1{%
  H\`an %
  Th\HOLOGO@PdfdocUnicode{\^e}{\9036\277} %
  Th\`anh%
}
%    \end{macrocode}
%    \end{macro}
%    \begin{macro}{\HoLogoHtml@HanTheThanh}
%    \begin{macrocode}
\def\HoLogoHtml@HanTheThanh#1{%
  H\`an %
  Th\HCode{&\ltx@hashchar x1ebf;} %
  Th\`anh%
}
%    \end{macrocode}
%    \end{macro}
%
% \subsection{Driver detection}
%
%    \begin{macrocode}
\HOLOGO@IfExists\InputIfFileExists{%
  \InputIfFileExists{hologo.cfg}{}{}%
}{%
  \ltx@IfUndefined{pdf@filesize}{%
    \def\HOLOGO@InputIfExists{%
      \openin\HOLOGO@temp=hologo.cfg\relax
      \ifeof\HOLOGO@temp
        \closein\HOLOGO@temp
      \else
        \closein\HOLOGO@temp
        \begingroup
          \def\x{LaTeX2e}%
        \expandafter\endgroup
        \ifx\fmtname\x
          % \iffalse meta-comment
%
% File: hologo.dtx
% Version: 2016/05/12 v1.11
% Info: A logo collection with bookmark support
%
% Copyright (C) 2010-2012 by
%    Heiko Oberdiek <heiko.oberdiek at googlemail.com>
%
% This work may be distributed and/or modified under the
% conditions of the LaTeX Project Public License, either
% version 1.3c of this license or (at your option) any later
% version. This version of this license is in
%    http://www.latex-project.org/lppl/lppl-1-3c.txt
% and the latest version of this license is in
%    http://www.latex-project.org/lppl.txt
% and version 1.3 or later is part of all distributions of
% LaTeX version 2005/12/01 or later.
%
% This work has the LPPL maintenance status "maintained".
%
% This Current Maintainer of this work is Heiko Oberdiek.
%
% The Base Interpreter refers to any `TeX-Format',
% because some files are installed in TDS:tex/generic//.
%
% This work consists of the main source file hologo.dtx
% and the derived files
%    hologo.sty, hologo.pdf, hologo.ins, hologo.drv, hologo-example.tex,
%    hologo-test1.tex, hologo-test-spacefactor.tex,
%    hologo-test-list.tex.
%
% Distribution:
%    CTAN:macros/latex/contrib/oberdiek/hologo.dtx
%    CTAN:macros/latex/contrib/oberdiek/hologo.pdf
%
% Unpacking:
%    (a) If hologo.ins is present:
%           tex hologo.ins
%    (b) Without hologo.ins:
%           tex hologo.dtx
%    (c) If you insist on using LaTeX
%           latex \let\install=y\input{hologo.dtx}
%        (quote the arguments according to the demands of your shell)
%
% Documentation:
%    (a) If hologo.drv is present:
%           latex hologo.drv
%    (b) Without hologo.drv:
%           latex hologo.dtx; ...
%    The class ltxdoc loads the configuration file ltxdoc.cfg
%    if available. Here you can specify further options, e.g.
%    use A4 as paper format:
%       \PassOptionsToClass{a4paper}{article}
%
%    Programm calls to get the documentation (example):
%       pdflatex hologo.dtx
%       makeindex -s gind.ist hologo.idx
%       pdflatex hologo.dtx
%       makeindex -s gind.ist hologo.idx
%       pdflatex hologo.dtx
%
% Installation:
%    TDS:tex/generic/oberdiek/hologo.sty
%    TDS:doc/latex/oberdiek/hologo.pdf
%    TDS:doc/latex/oberdiek/example/hologo-example.tex
%    TDS:doc/latex/oberdiek/test/hologo-test1.tex
%    TDS:doc/latex/oberdiek/test/hologo-test-spacefactor.tex
%    TDS:doc/latex/oberdiek/test/hologo-test-list.tex
%    TDS:source/latex/oberdiek/hologo.dtx
%
%<*ignore>
\begingroup
  \catcode123=1 %
  \catcode125=2 %
  \def\x{LaTeX2e}%
\expandafter\endgroup
\ifcase 0\ifx\install y1\fi\expandafter
         \ifx\csname processbatchFile\endcsname\relax\else1\fi
         \ifx\fmtname\x\else 1\fi\relax
\else\csname fi\endcsname
%</ignore>
%<*install>
\input docstrip.tex
\Msg{************************************************************************}
\Msg{* Installation}
\Msg{* Package: hologo 2016/05/12 v1.11 A logo collection with bookmark support (HO)}
\Msg{************************************************************************}

\keepsilent
\askforoverwritefalse

\let\MetaPrefix\relax
\preamble

This is a generated file.

Project: hologo
Version: 2016/05/12 v1.11

Copyright (C) 2010-2012 by
   Heiko Oberdiek <heiko.oberdiek at googlemail.com>

This work may be distributed and/or modified under the
conditions of the LaTeX Project Public License, either
version 1.3c of this license or (at your option) any later
version. This version of this license is in
   http://www.latex-project.org/lppl/lppl-1-3c.txt
and the latest version of this license is in
   http://www.latex-project.org/lppl.txt
and version 1.3 or later is part of all distributions of
LaTeX version 2005/12/01 or later.

This work has the LPPL maintenance status "maintained".

This Current Maintainer of this work is Heiko Oberdiek.

The Base Interpreter refers to any `TeX-Format',
because some files are installed in TDS:tex/generic//.

This work consists of the main source file hologo.dtx
and the derived files
   hologo.sty, hologo.pdf, hologo.ins, hologo.drv, hologo-example.tex,
   hologo-test1.tex, hologo-test-spacefactor.tex,
   hologo-test-list.tex.

\endpreamble
\let\MetaPrefix\DoubleperCent

\generate{%
  \file{hologo.ins}{\from{hologo.dtx}{install}}%
  \file{hologo.drv}{\from{hologo.dtx}{driver}}%
  \usedir{tex/generic/oberdiek}%
  \file{hologo.sty}{\from{hologo.dtx}{package}}%
  \usedir{doc/latex/oberdiek/example}%
  \file{hologo-example.tex}{\from{hologo.dtx}{example}}%
  \usedir{doc/latex/oberdiek/test}%
  \file{hologo-test1.tex}{\from{hologo.dtx}{test1}}%
  \file{hologo-test-spacefactor.tex}{\from{hologo.dtx}{test-spacefactor}}%
  \file{hologo-test-list.tex}{\from{hologo.dtx}{test-list}}%
  \nopreamble
  \nopostamble
  \usedir{source/latex/oberdiek/catalogue}%
  \file{hologo.xml}{\from{hologo.dtx}{catalogue}}%
}

\catcode32=13\relax% active space
\let =\space%
\Msg{************************************************************************}
\Msg{*}
\Msg{* To finish the installation you have to move the following}
\Msg{* file into a directory searched by TeX:}
\Msg{*}
\Msg{*     hologo.sty}
\Msg{*}
\Msg{* To produce the documentation run the file `hologo.drv'}
\Msg{* through LaTeX.}
\Msg{*}
\Msg{* Happy TeXing!}
\Msg{*}
\Msg{************************************************************************}

\endbatchfile
%</install>
%<*ignore>
\fi
%</ignore>
%<*driver>
\NeedsTeXFormat{LaTeX2e}
\ProvidesFile{hologo.drv}%
  [2016/05/12 v1.11 A logo collection with bookmark support (HO)]%
\documentclass{ltxdoc}
\usepackage{holtxdoc}[2011/11/22]
\usepackage{hologo}[2016/05/12]
\usepackage{longtable}
\usepackage{array}
\usepackage{paralist}
%\usepackage[T1]{fontenc}
%\usepackage{lmodern}
\begin{document}
  \DocInput{hologo.dtx}%
\end{document}
%</driver>
% \fi
%
%
% \CharacterTable
%  {Upper-case    \A\B\C\D\E\F\G\H\I\J\K\L\M\N\O\P\Q\R\S\T\U\V\W\X\Y\Z
%   Lower-case    \a\b\c\d\e\f\g\h\i\j\k\l\m\n\o\p\q\r\s\t\u\v\w\x\y\z
%   Digits        \0\1\2\3\4\5\6\7\8\9
%   Exclamation   \!     Double quote  \"     Hash (number) \#
%   Dollar        \$     Percent       \%     Ampersand     \&
%   Acute accent  \'     Left paren    \(     Right paren   \)
%   Asterisk      \*     Plus          \+     Comma         \,
%   Minus         \-     Point         \.     Solidus       \/
%   Colon         \:     Semicolon     \;     Less than     \<
%   Equals        \=     Greater than  \>     Question mark \?
%   Commercial at \@     Left bracket  \[     Backslash     \\
%   Right bracket \]     Circumflex    \^     Underscore    \_
%   Grave accent  \`     Left brace    \{     Vertical bar  \|
%   Right brace   \}     Tilde         \~}
%
% \GetFileInfo{hologo.drv}
%
% \title{The \xpackage{hologo} package}
% \date{2016/05/12 v1.11}
% \author{Heiko Oberdiek\\\xemail{heiko.oberdiek at googlemail.com}}
%
% \maketitle
%
% \begin{abstract}
% This package starts a collection of logos with support for bookmarks
% strings.
% \end{abstract}
%
% \tableofcontents
%
% \section{Documentation}
%
% \subsection{Logo macros}
%
% \begin{declcs}{hologo} \M{name}
% \end{declcs}
% Macro \cs{hologo} sets the logo with name \meta{name}.
% The following table shows the supported names.
%
% \begingroup
%   \def\hologoEntry#1#2#3{^^A
%     #1&#2&\hologoLogoSetup{#1}{variant=#2}\hologo{#1}&#3\tabularnewline
%   }
%   \begin{longtable}{>{\ttfamily}l>{\ttfamily}lll}
%     \rmfamily\bfseries{name} & \rmfamily\bfseries variant
%     & \bfseries logo & \bfseries since\\
%     \hline
%     \endhead
%     \hologoList
%   \end{longtable}
% \endgroup
%
% \begin{declcs}{Hologo} \M{name}
% \end{declcs}
% Macro \cs{Hologo} starts the logo \meta{name} with an uppercase
% letter. As an exception small greek letters are not converted
% to uppercase. Examples, see \hologo{eTeX} and \hologo{ExTeX}.
%
% \subsection{Setup macros}
%
% The package does not support package options, but the following
% setup macros can be used to set options.
%
% \begin{declcs}{hologoSetup} \M{key value list}
% \end{declcs}
% Macro \cs{hologoSetup} sets global options.
%
% \begin{declcs}{hologoLogoSetup} \M{logo} \M{key value list}
% \end{declcs}
% Some options can also be used to configure a logo.
% These settings take precedence over global option settings.
%
% \subsection{Options}\label{sec:options}
%
% There are boolean and string options:
% \begin{description}
% \item[Boolean option:]
% It takes |true| or |false|
% as value. If the value is omitted, then |true| is used.
% \item[String option:]
% A value must be given as string. (But the string might be empty.)
% \end{description}
% The following options can be used both in \cs{hologoSetup}
% and \cs{hologoLogoSetup}:
% \begin{description}
% \def\entry#1{\item[\xoption{#1}:]}
% \entry{break}
%   enables or disables line breaks inside the logo. This setting is
%   refined by options \xoption{hyphenbreak}, \xoption{spacebreak}
%   or \xoption{discretionarybreak}.
%   Default is |false|.
% \entry{hyphenbreak}
%   enables or disables the line break right after the hyphen character.
% \entry{spacebreak}
%   enables or disables line breaks at space characters.
% \entry{discretionarybreak}
%   enables or disables line breaks at hyphenation points
%   (inserted by \cs{-}).
% \end{description}
% Macro \cs{hologoLogoSetup} also knows:
% \begin{description}
% \item[\xoption{variant}:]
%   This is a string option. It specifies a variant of a logo that
%   must exist. An empty string selects the package default variant.
% \end{description}
% Example:
% \begin{quote}
%   |\hologoSetup{break=false}|\\
%   |\hologoLogoSetup{plainTeX}{variant=hyphen,hyphenbreak}|\\
%   Then ``plain-\TeX'' contains one break point after the hyphen.
% \end{quote}
%
% \subsection{Driver options}
%
% Sometimes graphical operations are needed to construct some
% glyphs (e.g.\ \hologo{XeTeX}). If package \xpackage{graphics}
% or package \xpackage{pgf} are found, then the macros are taken
% from there. Otherwise the packge defines its own operations
% and therefore needs the driver information. Many drivers are
% detected automatically (\hologo{pdfTeX}/\hologo{LuaTeX}
% in PDF mode, \hologo{XeTeX}, \hologo{VTeX}). These have precedence
% over a driver option. The driver can be given as package option
% or using \cs{hologoDriverSetup}.
% The following list contains the recognized driver options:
% \begin{itemize}
% \item \xoption{pdftex}, \xoption{luatex}
% \item \xoption{dvipdfm}, \xoption{dvipdfmx}
% \item \xoption{dvips}, \xoption{dvipsone}, \xoption{xdvi}
% \item \xoption{xetex}
% \item \xoption{vtex}
% \end{itemize}
% The left driver of a line is the driver name that is used internally.
% The following names are aliases for drivers that use the
% same method. Therefore the entry in the \xext{log} file for
% the used driver prints the internally used driver name.
% \begin{description}
% \item[\xoption{driverfallback}:]
%   This option expects a driver that is used,
%   if the driver could not be detected automatically.
% \end{description}
%
% \begin{declcs}{hologoDriverSetup} \M{driver option}
% \end{declcs}
% The driver can also be configured after package loading
% using \cs{hologoDriverSetup}, also the way for \hologo{plainTeX}
% to setup the driver.
%
% \subsection{Font setup}
%
% Some logos require a special font, but should also be usable by
% \hologo{plainTeX}. Therefore the package provides some ways
% to influence the font settings. The options below
% take font settings as values. Both font commands
% such as \cs{sffamily} and macros that take one argument
% like \cs{textsf} can be used.
%
% \begin{declcs}{hologoFontSetup} \M{key value list}
% \end{declcs}
% Macro \cs{hologoFontSetup} sets the fonts for all logos.
% Supported keys:
% \begin{description}
% \def\entry#1{\item[\xoption{#1}:]}
% \entry{general}
%   This font is used for all logos. The default is empty.
%   That means no special font is used.
% \entry{bibsf}
%   This font is used for
%   {\hologoLogoSetup{BibTeX}{variant=sf}\hologo{BibTeX}}
%   with variant \xoption{sf}.
% \entry{rm}
%   This font is a serif font. It is used for \hologo{ExTeX}.
% \entry{sc}
%   This font specifies a small caps font. It is used for
%   {\hologoLogoSetup{BibTeX}{variant=sc}\hologo{BibTeX}}
%   with variant \xoption{sc}.
% \entry{sf}
%   This font specifies a sans serif font. The default
%   is \cs{sffamily}, then \cs{sf} is tried. Otherwise
%   a warning is given. It is used by \hologo{KOMAScript}.
% \entry{sy}
%   This is the font for math symbols (e.g. cmsy).
%   It is used by \hologo{AmS}, \hologo{NTS}, \hologo{ExTeX}.
% \entry{logo}
%   \hologo{METAFONT} and \hologo{METAPOST} are using that font.
%   In \hologo{LaTeX} \cs{logofamily} is used and
%   the definitions of package \xpackage{mflogo} are used
%   if the package is not loaded.
%   Otherwise the \cs{tenlogo} is used and defined
%   if it does not already exists.
% \end{description}
%
% \begin{declcs}{hologoLogoFontSetup} \M{logo} \M{key value list}
% \end{declcs}
% Fonts can also be set for a logo or logo component separately,
% see the following list.
% The keys are the same as for \cs{hologoFontSetup}.
%
% \begin{longtable}{>{\ttfamily}l>{\sffamily}ll}
%   \meta{logo} & keys & result\\
%   \hline
%   \endhead
%   BibTeX & bibsf & {\hologoLogoSetup{BibTeX}{variant=sf}\hologo{BibTeX}}\\[.5ex]
%   BibTeX & sc & {\hologoLogoSetup{BibTeX}{variant=sc}\hologo{BibTeX}}\\[.5ex]
%   ExTeX & rm & \hologo{ExTeX}\\
%   SliTeX & rm & \hologo{SliTeX}\\[.5ex]
%   AmS & sy & \hologo{AmS}\\
%   ExTeX & sy & \hologo{ExTeX}\\
%   NTS & sy & \hologo{NTS}\\[.5ex]
%   KOMAScript & sf & \hologo{KOMAScript}\\[.5ex]
%   METAFONT & logo & \hologo{METAFONT}\\
%   METAPOST & logo & \hologo{METAPOST}\\[.5ex]
%   SliTeX & sc \hologo{SliTeX}
% \end{longtable}
%
% \subsubsection{Font order}
%
% For all logos the font \xoption{general} is applied first.
% Example:
%\begin{quote}
%|\hologoFontSetup{general=\color{red}}|
%\end{quote}
% will print red logos.
% Then if the font uses a special font \xoption{sf}, for example,
% the font is applied that is setup by \cs{hologoLogoFontSetup}.
% If this font is not setup, then the common font setup
% by \cs{hologoFontSetup} is used. Otherwise a warning is given,
% that there is no font configured.
%
% \subsection{Additional user macros}
%
% Usually a variant of a logo is configured by using
% \cs{hologoLogoSetup}, because it is bad style to mix
% different variants of the same logo in the same text.
% There the following macros are a convenience for testing.
%
% \begin{declcs}{hologoVariant} \M{name} \M{variant}\\
%   \cs{HologoVariant} \M{name} \M{variant}
% \end{declcs}
% Logo \meta{name} is set using \meta{variant} that specifies
% explicitely which variant of the macro is used. If the argument
% is empty, then the default form of the logo is used
% (configurable by \cs{hologoLogoSetup}).
%
% \cs{HologoVariant} is used if the logo is set in a context
% that needs an uppercase first letter (beginning of a sentence, \dots).
%
% \begin{declcs}{hologoList}\\
%   \cs{hologoEntry} \M{logo} \M{variant} \M{since}
% \end{declcs}
% Macro \cs{hologoList} contains all logos that are provided
% by the package including variants. The list consists of calls
% of \cs{hologoEntry} with three arguments starting with the
% logo name \meta{logo} and its variant \meta{variant}. An empty
% variant means the current default. Argument \meta{since} specifies
% with version of the package \xpackage{hologo} is needed to get
% the logo. If the logo is fixed, then the date gets updated.
% Therefore the date \meta{since} is not exactly the date of
% the first introduction, but rather the date of the latest fix.
%
% Before \cs{hologoList} can be used, macro \cs{hologoEntry} needs
% a definition. The example file in section \ref{sec:example}
% shows applications of \cs{hologoList}.
%
% \subsection{Supported contexts}
%
% Macros \cs{hologo} and friends support special contexts:
% \begin{itemize}
% \item \hologo{LaTeX}'s protection mechanism.
% \item Bookmarks of package \xpackage{hyperref}.
% \item Package \xpackage{tex4ht}.
% \item The macros can be used inside \cs{csname} constructs,
%   if \cs{ifincsname} is available (\hologo{pdfTeX}, \hologo{XeTeX},
%   \hologo{LuaTeX}).
% \end{itemize}
%
% \subsection{Example}
% \label{sec:example}
%
% The following example prints the logos in different fonts.
%    \begin{macrocode}
%<*example>
%<<verbatim
\NeedsTeXFormat{LaTeX2e}
\documentclass[a4paper]{article}
\usepackage[
  hmargin=20mm,
  vmargin=20mm,
]{geometry}
\pagestyle{empty}
\usepackage{hologo}[2016/05/12]
\usepackage{longtable}
\usepackage{array}
\setlength{\extrarowheight}{2pt}
\usepackage[T1]{fontenc}
\usepackage{lmodern}
\usepackage{pdflscape}
\usepackage[
  pdfencoding=auto,
]{hyperref}
\hypersetup{
  pdfauthor={Heiko Oberdiek},
  pdftitle={Example for package `hologo'},
  pdfsubject={Logos with fonts lmr, lmss, qtm, qpl, qhv},
}
\usepackage{bookmark}

% Print the logo list on the console

\begingroup
  \typeout{}%
  \typeout{*** Begin of logo list ***}%
  \newcommand*{\hologoEntry}[3]{%
    \typeout{#1 \ifx\\#2\\\else(#2) \fi[#3]}%
  }%
  \hologoList
  \typeout{*** End of logo list ***}%
  \typeout{}%
\endgroup

\begin{document}
\begin{landscape}

  \section{Example file for package `hologo'}

  % Table for font names

  \begin{longtable}{>{\bfseries}ll}
    \textbf{font} & \textbf{Font name}\\
    \hline
    lmr & Latin Modern Roman\\
    lmss & Latin Modern Sans\\
    qtm & \TeX\ Gyre Termes\\
    qhv & \TeX\ Gyre Heros\\
    qpl & \TeX\ Gyre Pagella\\
  \end{longtable}

  % Logo list with logos in different fonts

  \begingroup
    \newcommand*{\SetVariant}[2]{%
      \ifx\\#2\\%
      \else
        \hologoLogoSetup{#1}{variant=#2}%
      \fi
    }%
    \newcommand*{\hologoEntry}[3]{%
      \SetVariant{#1}{#2}%
      \raisebox{1em}[0pt][0pt]{\hypertarget{#1@#2}{}}%
      \bookmark[%
        dest={#1@#2},%
      ]{%
        #1\ifx\\#2\\\else\space(#2)\fi: \Hologo{#1}, \hologo{#1} %
        [Unicode]%
      }%
      \hypersetup{unicode=false}%
      \bookmark[%
        dest={#1@#2},%
      ]{%
        #1\ifx\\#2\\\else\space(#2)\fi: \Hologo{#1}, \hologo{#1} %
        [PDFDocEncoding]%
      }%
      \texttt{#1}%
      &%
      \texttt{#2}%
      &%
      \Hologo{#1}%
      &%
      \SetVariant{#1}{#2}%
      \hologo{#1}%
      &%
      \SetVariant{#1}{#2}%
      \fontfamily{qtm}\selectfont
      \hologo{#1}%
      &%
      \SetVariant{#1}{#2}%
      \fontfamily{qpl}\selectfont
      \hologo{#1}%
      &%
      \SetVariant{#1}{#2}%
      \textsf{\hologo{#1}}%
      &%
      \SetVariant{#1}{#2}%
      \fontfamily{qhv}\selectfont
      \hologo{#1}%
      \tabularnewline
    }%
    \begin{longtable}{llllllll}%
      \textbf{\textit{logo}} & \textbf{\textit{variant}} &
      \texttt{\string\Hologo} &
      \textbf{lmr} & \textbf{qtm} & \textbf{qpl} &
      \textbf{lmss} & \textbf{qhv}
      \tabularnewline
      \hline
      \endhead
      \hologoList
    \end{longtable}%
  \endgroup

\end{landscape}
\end{document}
%verbatim
%</example>
%    \end{macrocode}
%
% \StopEventually{
% }
%
% \section{Implementation}
%    \begin{macrocode}
%<*package>
%    \end{macrocode}
%    Reload check, especially if the package is not used with \LaTeX.
%    \begin{macrocode}
\begingroup\catcode61\catcode48\catcode32=10\relax%
  \catcode13=5 % ^^M
  \endlinechar=13 %
  \catcode35=6 % #
  \catcode39=12 % '
  \catcode44=12 % ,
  \catcode45=12 % -
  \catcode46=12 % .
  \catcode58=12 % :
  \catcode64=11 % @
  \catcode123=1 % {
  \catcode125=2 % }
  \expandafter\let\expandafter\x\csname ver@hologo.sty\endcsname
  \ifx\x\relax % plain-TeX, first loading
  \else
    \def\empty{}%
    \ifx\x\empty % LaTeX, first loading,
      % variable is initialized, but \ProvidesPackage not yet seen
    \else
      \expandafter\ifx\csname PackageInfo\endcsname\relax
        \def\x#1#2{%
          \immediate\write-1{Package #1 Info: #2.}%
        }%
      \else
        \def\x#1#2{\PackageInfo{#1}{#2, stopped}}%
      \fi
      \x{hologo}{The package is already loaded}%
      \aftergroup\endinput
    \fi
  \fi
\endgroup%
%    \end{macrocode}
%    Package identification:
%    \begin{macrocode}
\begingroup\catcode61\catcode48\catcode32=10\relax%
  \catcode13=5 % ^^M
  \endlinechar=13 %
  \catcode35=6 % #
  \catcode39=12 % '
  \catcode40=12 % (
  \catcode41=12 % )
  \catcode44=12 % ,
  \catcode45=12 % -
  \catcode46=12 % .
  \catcode47=12 % /
  \catcode58=12 % :
  \catcode64=11 % @
  \catcode91=12 % [
  \catcode93=12 % ]
  \catcode123=1 % {
  \catcode125=2 % }
  \expandafter\ifx\csname ProvidesPackage\endcsname\relax
    \def\x#1#2#3[#4]{\endgroup
      \immediate\write-1{Package: #3 #4}%
      \xdef#1{#4}%
    }%
  \else
    \def\x#1#2[#3]{\endgroup
      #2[{#3}]%
      \ifx#1\@undefined
        \xdef#1{#3}%
      \fi
      \ifx#1\relax
        \xdef#1{#3}%
      \fi
    }%
  \fi
\expandafter\x\csname ver@hologo.sty\endcsname
\ProvidesPackage{hologo}%
  [2016/05/12 v1.11 A logo collection with bookmark support (HO)]%
%    \end{macrocode}
%
%    \begin{macrocode}
\begingroup\catcode61\catcode48\catcode32=10\relax%
  \catcode13=5 % ^^M
  \endlinechar=13 %
  \catcode123=1 % {
  \catcode125=2 % }
  \catcode64=11 % @
  \def\x{\endgroup
    \expandafter\edef\csname HOLOGO@AtEnd\endcsname{%
      \endlinechar=\the\endlinechar\relax
      \catcode13=\the\catcode13\relax
      \catcode32=\the\catcode32\relax
      \catcode35=\the\catcode35\relax
      \catcode61=\the\catcode61\relax
      \catcode64=\the\catcode64\relax
      \catcode123=\the\catcode123\relax
      \catcode125=\the\catcode125\relax
    }%
  }%
\x\catcode61\catcode48\catcode32=10\relax%
\catcode13=5 % ^^M
\endlinechar=13 %
\catcode35=6 % #
\catcode64=11 % @
\catcode123=1 % {
\catcode125=2 % }
\def\TMP@EnsureCode#1#2{%
  \edef\HOLOGO@AtEnd{%
    \HOLOGO@AtEnd
    \catcode#1=\the\catcode#1\relax
  }%
  \catcode#1=#2\relax
}
\TMP@EnsureCode{10}{12}% ^^J
\TMP@EnsureCode{33}{12}% !
\TMP@EnsureCode{34}{12}% "
\TMP@EnsureCode{36}{3}% $
\TMP@EnsureCode{38}{4}% &
\TMP@EnsureCode{39}{12}% '
\TMP@EnsureCode{40}{12}% (
\TMP@EnsureCode{41}{12}% )
\TMP@EnsureCode{42}{12}% *
\TMP@EnsureCode{43}{12}% +
\TMP@EnsureCode{44}{12}% ,
\TMP@EnsureCode{45}{12}% -
\TMP@EnsureCode{46}{12}% .
\TMP@EnsureCode{47}{12}% /
\TMP@EnsureCode{58}{12}% :
\TMP@EnsureCode{59}{12}% ;
\TMP@EnsureCode{60}{12}% <
\TMP@EnsureCode{62}{12}% >
\TMP@EnsureCode{63}{12}% ?
\TMP@EnsureCode{91}{12}% [
\TMP@EnsureCode{93}{12}% ]
\TMP@EnsureCode{94}{7}% ^ (superscript)
\TMP@EnsureCode{95}{8}% _ (subscript)
\TMP@EnsureCode{96}{12}% `
\TMP@EnsureCode{124}{12}% |
\edef\HOLOGO@AtEnd{%
  \HOLOGO@AtEnd
  \escapechar\the\escapechar\relax
  \noexpand\endinput
}
\escapechar=92 %
%    \end{macrocode}
%
% \subsection{Logo list}
%
%    \begin{macro}{\hologoList}
%    \begin{macrocode}
\def\hologoList{%
  \hologoEntry{(La)TeX}{}{2011/10/01}%
  \hologoEntry{AmSLaTeX}{}{2010/04/16}%
  \hologoEntry{AmSTeX}{}{2010/04/16}%
  \hologoEntry{biber}{}{2011/10/01}%
  \hologoEntry{BibTeX}{}{2011/10/01}%
  \hologoEntry{BibTeX}{sf}{2011/10/01}%
  \hologoEntry{BibTeX}{sc}{2011/10/01}%
  \hologoEntry{BibTeX8}{}{2011/11/22}%
  \hologoEntry{ConTeXt}{}{2011/03/25}%
  \hologoEntry{ConTeXt}{narrow}{2011/03/25}%
  \hologoEntry{ConTeXt}{simple}{2011/03/25}%
  \hologoEntry{emTeX}{}{2010/04/26}%
  \hologoEntry{eTeX}{}{2010/04/08}%
  \hologoEntry{ExTeX}{}{2011/10/01}%
  \hologoEntry{HanTheThanh}{}{2011/11/29}%
  \hologoEntry{iniTeX}{}{2011/10/01}%
  \hologoEntry{KOMAScript}{}{2011/10/01}%
  \hologoEntry{La}{}{2010/05/08}%
  \hologoEntry{LaTeX}{}{2010/04/08}%
  \hologoEntry{LaTeX2e}{}{2010/04/08}%
  \hologoEntry{LaTeX3}{}{2010/04/24}%
  \hologoEntry{LaTeXe}{}{2010/04/08}%
  \hologoEntry{LaTeXML}{}{2011/11/22}%
  \hologoEntry{LaTeXTeX}{}{2011/10/01}%
  \hologoEntry{LuaLaTeX}{}{2010/04/08}%
  \hologoEntry{LuaTeX}{}{2010/04/08}%
  \hologoEntry{LyX}{}{2011/10/01}%
  \hologoEntry{METAFONT}{}{2011/10/01}%
  \hologoEntry{MetaFun}{}{2011/10/01}%
  \hologoEntry{METAPOST}{}{2011/10/01}%
  \hologoEntry{MetaPost}{}{2011/10/01}%
  \hologoEntry{MiKTeX}{}{2011/10/01}%
  \hologoEntry{NTS}{}{2011/10/01}%
  \hologoEntry{OzMF}{}{2011/10/01}%
  \hologoEntry{OzMP}{}{2011/10/01}%
  \hologoEntry{OzTeX}{}{2011/10/01}%
  \hologoEntry{OzTtH}{}{2011/10/01}%
  \hologoEntry{PCTeX}{}{2011/10/01}%
  \hologoEntry{pdfTeX}{}{2011/10/01}%
  \hologoEntry{pdfLaTeX}{}{2011/10/01}%
  \hologoEntry{PiC}{}{2011/10/01}%
  \hologoEntry{PiCTeX}{}{2011/10/01}%
  \hologoEntry{plainTeX}{}{2010/04/08}%
  \hologoEntry{plainTeX}{space}{2010/04/16}%
  \hologoEntry{plainTeX}{hyphen}{2010/04/16}%
  \hologoEntry{plainTeX}{runtogether}{2010/04/16}%
  \hologoEntry{SageTeX}{}{2011/11/22}%
  \hologoEntry{SLiTeX}{}{2011/10/01}%
  \hologoEntry{SLiTeX}{lift}{2011/10/01}%
  \hologoEntry{SLiTeX}{narrow}{2011/10/01}%
  \hologoEntry{SLiTeX}{simple}{2011/10/01}%
  \hologoEntry{SliTeX}{}{2011/10/01}%
  \hologoEntry{SliTeX}{narrow}{2011/10/01}%
  \hologoEntry{SliTeX}{simple}{2011/10/01}%
  \hologoEntry{SliTeX}{lift}{2011/10/01}%
  \hologoEntry{teTeX}{}{2011/10/01}%
  \hologoEntry{TeX}{}{2010/04/08}%
  \hologoEntry{TeX4ht}{}{2011/11/22}%
  \hologoEntry{TTH}{}{2011/11/22}%
  \hologoEntry{virTeX}{}{2011/10/01}%
  \hologoEntry{VTeX}{}{2010/04/24}%
  \hologoEntry{Xe}{}{2010/04/08}%
  \hologoEntry{XeLaTeX}{}{2010/04/08}%
  \hologoEntry{XeTeX}{}{2010/04/08}%
}
%    \end{macrocode}
%    \end{macro}
%
% \subsection{Load resources}
%
%    \begin{macrocode}
\begingroup\expandafter\expandafter\expandafter\endgroup
\expandafter\ifx\csname RequirePackage\endcsname\relax
  \def\TMP@RequirePackage#1[#2]{%
    \begingroup\expandafter\expandafter\expandafter\endgroup
    \expandafter\ifx\csname ver@#1.sty\endcsname\relax
      \input #1.sty\relax
    \fi
  }%
  \TMP@RequirePackage{ltxcmds}[2011/02/04]%
  \TMP@RequirePackage{infwarerr}[2010/04/08]%
  \TMP@RequirePackage{kvsetkeys}[2010/03/01]%
  \TMP@RequirePackage{kvdefinekeys}[2010/03/01]%
  \TMP@RequirePackage{pdftexcmds}[2010/04/01]%
  \TMP@RequirePackage{ifpdf}[2010/01/28]%
  \TMP@RequirePackage{ifluatex}[2010/03/01]%
  \ltx@IfUndefined{newif}{%
    \expandafter\let\csname newif\endcsname\ltx@newif
  }{}%
  \TMP@RequirePackage{ifxetex}[2009/01/23]%
  \TMP@RequirePackage{ifvtex}[2010/03/01]%
\else
  \RequirePackage{ltxcmds}[2011/02/04]%
  \RequirePackage{infwarerr}[2010/04/08]%
  \RequirePackage{kvsetkeys}[2010/03/01]%
  \RequirePackage{kvdefinekeys}[2010/03/01]%
  \RequirePackage{pdftexcmds}[2010/04/01]%
  \RequirePackage{ifpdf}[2010/01/28]%
  \RequirePackage{ifluatex}[2010/03/01]%
  \RequirePackage{ifxetex}[2009/01/23]%
  \RequirePackage{ifvtex}[2010/03/01]%
\fi
%    \end{macrocode}
%
%    \begin{macro}{\HOLOGO@IfDefined}
%    \begin{macrocode}
\def\HOLOGO@IfExists#1{%
  \ifx\@undefined#1%
    \expandafter\ltx@secondoftwo
  \else
    \ifx\relax#1%
      \expandafter\ltx@secondoftwo
    \else
      \expandafter\expandafter\expandafter\ltx@firstoftwo
    \fi
  \fi
}
%    \end{macrocode}
%    \end{macro}
%
% \subsection{Setup macros}
%
%    \begin{macro}{\hologoSetup}
%    \begin{macrocode}
\def\hologoSetup{%
  \let\HOLOGO@name\relax
  \HOLOGO@Setup
}
%    \end{macrocode}
%    \end{macro}
%
%    \begin{macro}{\hologoLogoSetup}
%    \begin{macrocode}
\def\hologoLogoSetup#1{%
  \edef\HOLOGO@name{#1}%
  \ltx@IfUndefined{HoLogo@\HOLOGO@name}{%
    \@PackageError{hologo}{%
      Unknown logo `\HOLOGO@name'%
    }\@ehc
    \ltx@gobble
  }{%
    \HOLOGO@Setup
  }%
}
%    \end{macrocode}
%    \end{macro}
%
%    \begin{macro}{\HOLOGO@Setup}
%    \begin{macrocode}
\def\HOLOGO@Setup{%
  \kvsetkeys{HoLogo}%
}
%    \end{macrocode}
%    \end{macro}
%
% \subsection{Options}
%
%    \begin{macro}{\HOLOGO@DeclareBoolOption}
%    \begin{macrocode}
\def\HOLOGO@DeclareBoolOption#1{%
  \expandafter\chardef\csname HOLOGOOPT@#1\endcsname\ltx@zero
  \kv@define@key{HoLogo}{#1}[true]{%
    \def\HOLOGO@temp{##1}%
    \ifx\HOLOGO@temp\HOLOGO@true
      \ifx\HOLOGO@name\relax
        \expandafter\chardef\csname HOLOGOOPT@#1\endcsname=\ltx@one
      \else
        \expandafter\chardef\csname
        HoLogoOpt@#1@\HOLOGO@name\endcsname\ltx@one
      \fi
      \HOLOGO@SetBreakAll{#1}%
    \else
      \ifx\HOLOGO@temp\HOLOGO@false
        \ifx\HOLOGO@name\relax
          \expandafter\chardef\csname HOLOGOOPT@#1\endcsname=\ltx@zero
        \else
          \expandafter\chardef\csname
          HoLogoOpt@#1@\HOLOGO@name\endcsname=\ltx@zero
        \fi
        \HOLOGO@SetBreakAll{#1}%
      \else
        \@PackageError{hologo}{%
          Unknown value `##1' for boolean option `#1'.\MessageBreak
          Known values are `true' and `false'%
        }\@ehc
      \fi
    \fi
  }%
}
%    \end{macrocode}
%    \end{macro}
%
%    \begin{macro}{\HOLOGO@SetBreakAll}
%    \begin{macrocode}
\def\HOLOGO@SetBreakAll#1{%
  \def\HOLOGO@temp{#1}%
  \ifx\HOLOGO@temp\HOLOGO@break
    \ifx\HOLOGO@name\relax
      \chardef\HOLOGOOPT@hyphenbreak=\HOLOGOOPT@break
      \chardef\HOLOGOOPT@spacebreak=\HOLOGOOPT@break
      \chardef\HOLOGOOPT@discretionarybreak=\HOLOGOOPT@break
    \else
      \expandafter\chardef
         \csname HoLogoOpt@hyphenbreak@\HOLOGO@name\endcsname=%
         \csname HoLogoOpt@break@\HOLOGO@name\endcsname
      \expandafter\chardef
         \csname HoLogoOpt@spacebreak@\HOLOGO@name\endcsname=%
         \csname HoLogoOpt@break@\HOLOGO@name\endcsname
      \expandafter\chardef
         \csname HoLogoOpt@discretionarybreak@\HOLOGO@name
             \endcsname=%
         \csname HoLogoOpt@break@\HOLOGO@name\endcsname
    \fi
  \fi
}
%    \end{macrocode}
%    \end{macro}
%
%    \begin{macro}{\HOLOGO@true}
%    \begin{macrocode}
\def\HOLOGO@true{true}
%    \end{macrocode}
%    \end{macro}
%    \begin{macro}{\HOLOGO@false}
%    \begin{macrocode}
\def\HOLOGO@false{false}
%    \end{macrocode}
%    \end{macro}
%    \begin{macro}{\HOLOGO@break}
%    \begin{macrocode}
\def\HOLOGO@break{break}
%    \end{macrocode}
%    \end{macro}
%
%    \begin{macrocode}
\HOLOGO@DeclareBoolOption{break}
\HOLOGO@DeclareBoolOption{hyphenbreak}
\HOLOGO@DeclareBoolOption{spacebreak}
\HOLOGO@DeclareBoolOption{discretionarybreak}
%    \end{macrocode}
%
%    \begin{macrocode}
\kv@define@key{HoLogo}{variant}{%
  \ifx\HOLOGO@name\relax
    \@PackageError{hologo}{%
      Option `variant' is not available in \string\hologoSetup,%
      \MessageBreak
      Use \string\hologoLogoSetup\space instead%
    }\@ehc
  \else
    \edef\HOLOGO@temp{#1}%
    \ifx\HOLOGO@temp\ltx@empty
      \expandafter
      \let\csname HoLogoOpt@variant@\HOLOGO@name\endcsname\@undefined
    \else
      \ltx@IfUndefined{HoLogo@\HOLOGO@name @\HOLOGO@temp}{%
        \@PackageError{hologo}{%
          Unknown variant `\HOLOGO@temp' of logo `\HOLOGO@name'%
        }\@ehc
      }{%
        \expandafter
        \let\csname HoLogoOpt@variant@\HOLOGO@name\endcsname
            \HOLOGO@temp
      }%
    \fi
  \fi
}
%    \end{macrocode}
%
%    \begin{macro}{\HOLOGO@Variant}
%    \begin{macrocode}
\def\HOLOGO@Variant#1{%
  #1%
  \ltx@ifundefined{HoLogoOpt@variant@#1}{%
  }{%
    @\csname HoLogoOpt@variant@#1\endcsname
  }%
}
%    \end{macrocode}
%    \end{macro}
%
% \subsection{Break/no-break support}
%
%    \begin{macro}{\HOLOGO@space}
%    \begin{macrocode}
\def\HOLOGO@space{%
  \ltx@ifundefined{HoLogoOpt@spacebreak@\HOLOGO@name}{%
    \ltx@ifundefined{HoLogoOpt@break@\HOLOGO@name}{%
      \chardef\HOLOGO@temp=\HOLOGOOPT@spacebreak
    }{%
      \chardef\HOLOGO@temp=%
        \csname HoLogoOpt@break@\HOLOGO@name\endcsname
    }%
  }{%
    \chardef\HOLOGO@temp=%
      \csname HoLogoOpt@spacebreak@\HOLOGO@name\endcsname
  }%
  \ifcase\HOLOGO@temp
    \penalty10000 %
  \fi
  \ltx@space
}
%    \end{macrocode}
%    \end{macro}
%
%    \begin{macro}{\HOLOGO@hyphen}
%    \begin{macrocode}
\def\HOLOGO@hyphen{%
  \ltx@ifundefined{HoLogoOpt@hyphenbreak@\HOLOGO@name}{%
    \ltx@ifundefined{HoLogoOpt@break@\HOLOGO@name}{%
      \chardef\HOLOGO@temp=\HOLOGOOPT@hyphenbreak
    }{%
      \chardef\HOLOGO@temp=%
        \csname HoLogoOpt@break@\HOLOGO@name\endcsname
    }%
  }{%
    \chardef\HOLOGO@temp=%
      \csname HoLogoOpt@hyphenbreak@\HOLOGO@name\endcsname
  }%
  \ifcase\HOLOGO@temp
    \ltx@mbox{-}%
  \else
    -%
  \fi
}
%    \end{macrocode}
%    \end{macro}
%
%    \begin{macro}{\HOLOGO@discretionary}
%    \begin{macrocode}
\def\HOLOGO@discretionary{%
  \ltx@ifundefined{HoLogoOpt@discretionarybreak@\HOLOGO@name}{%
    \ltx@ifundefined{HoLogoOpt@break@\HOLOGO@name}{%
      \chardef\HOLOGO@temp=\HOLOGOOPT@discretionarybreak
    }{%
      \chardef\HOLOGO@temp=%
        \csname HoLogoOpt@break@\HOLOGO@name\endcsname
    }%
  }{%
    \chardef\HOLOGO@temp=%
      \csname HoLogoOpt@discretionarybreak@\HOLOGO@name\endcsname
  }%
  \ifcase\HOLOGO@temp
  \else
    \-%
  \fi
}
%    \end{macrocode}
%    \end{macro}
%
%    \begin{macro}{\HOLOGO@mbox}
%    \begin{macrocode}
\def\HOLOGO@mbox#1{%
  \ltx@ifundefined{HoLogoOpt@break@\HOLOGO@name}{%
    \chardef\HOLOGO@temp=\HOLOGOOPT@hyphenbreak
  }{%
    \chardef\HOLOGO@temp=%
      \csname HoLogoOpt@break@\HOLOGO@name\endcsname
  }%
  \ifcase\HOLOGO@temp
    \ltx@mbox{#1}%
  \else
    #1%
  \fi
}
%    \end{macrocode}
%    \end{macro}
%
% \subsection{Font support}
%
%    \begin{macro}{\HoLogoFont@font}
%    \begin{tabular}{@{}ll@{}}
%    |#1|:& logo name\\
%    |#2|:& font short name\\
%    |#3|:& text
%    \end{tabular}
%    \begin{macrocode}
\def\HoLogoFont@font#1#2#3{%
  \begingroup
    \ltx@IfUndefined{HoLogoFont@logo@#1.#2}{%
      \ltx@IfUndefined{HoLogoFont@font@#2}{%
        \@PackageWarning{hologo}{%
          Missing font `#2' for logo `#1'%
        }%
        #3%
      }{%
        \csname HoLogoFont@font@#2\endcsname{#3}%
      }%
    }{%
      \csname HoLogoFont@logo@#1.#2\endcsname{#3}%
    }%
  \endgroup
}
%    \end{macrocode}
%    \end{macro}
%
%    \begin{macro}{\HoLogoFont@Def}
%    \begin{macrocode}
\def\HoLogoFont@Def#1{%
  \expandafter\def\csname HoLogoFont@font@#1\endcsname
}
%    \end{macrocode}
%    \end{macro}
%    \begin{macro}{\HoLogoFont@LogoDef}
%    \begin{macrocode}
\def\HoLogoFont@LogoDef#1#2{%
  \expandafter\def\csname HoLogoFont@logo@#1.#2\endcsname
}
%    \end{macrocode}
%    \end{macro}
%
% \subsubsection{Font defaults}
%
%    \begin{macro}{\HoLogoFont@font@general}
%    \begin{macrocode}
\HoLogoFont@Def{general}{}%
%    \end{macrocode}
%    \end{macro}
%
%    \begin{macro}{\HoLogoFont@font@rm}
%    \begin{macrocode}
\ltx@IfUndefined{rmfamily}{%
  \ltx@IfUndefined{rm}{%
  }{%
    \HoLogoFont@Def{rm}{\rm}%
  }%
}{%
  \HoLogoFont@Def{rm}{\rmfamily}%
}
%    \end{macrocode}
%    \end{macro}
%
%    \begin{macro}{\HoLogoFont@font@sf}
%    \begin{macrocode}
\ltx@IfUndefined{sffamily}{%
  \ltx@IfUndefined{sf}{%
  }{%
    \HoLogoFont@Def{sf}{\sf}%
  }%
}{%
  \HoLogoFont@Def{sf}{\sffamily}%
}
%    \end{macrocode}
%    \end{macro}
%
%    \begin{macro}{\HoLogoFont@font@bibsf}
%    In case of \hologo{plainTeX} the original small caps
%    variant is used as default. In \hologo{LaTeX}
%    the definition of package \xpackage{dtklogos} \cite{dtklogos}
%    is used.
%\begin{quote}
%\begin{verbatim}
%\DeclareRobustCommand{\BibTeX}{%
%  B%
%  \kern-.05em%
%  \hbox{%
%    $\m@th$% %% force math size calculations
%    \csname S@\f@size\endcsname
%    \fontsize\sf@size\z@
%    \math@fontsfalse
%    \selectfont
%    I%
%    \kern-.025em%
%    B
%  }%
%  \kern-.08em%
%  \-%
%  \TeX
%}
%\end{verbatim}
%\end{quote}
%    \begin{macrocode}
\ltx@IfUndefined{selectfont}{%
  \ltx@IfUndefined{tensc}{%
    \font\tensc=cmcsc10\relax
  }{}%
  \HoLogoFont@Def{bibsf}{\tensc}%
}{%
  \HoLogoFont@Def{bibsf}{%
    $\mathsurround=0pt$%
    \csname S@\f@size\endcsname
    \fontsize\sf@size{0pt}%
    \math@fontsfalse
    \selectfont
  }%
}
%    \end{macrocode}
%    \end{macro}
%
%    \begin{macro}{\HoLogoFont@font@sc}
%    \begin{macrocode}
\ltx@IfUndefined{scshape}{%
  \ltx@IfUndefined{tensc}{%
    \font\tensc=cmcsc10\relax
  }{}%
  \HoLogoFont@Def{sc}{\tensc}%
}{%
  \HoLogoFont@Def{sc}{\scshape}%
}
%    \end{macrocode}
%    \end{macro}
%
%    \begin{macro}{\HoLogoFont@font@sy}
%    \begin{macrocode}
\ltx@IfUndefined{usefont}{%
  \ltx@IfUndefined{tensy}{%
  }{%
    \HoLogoFont@Def{sy}{\tensy}%
  }%
}{%
  \HoLogoFont@Def{sy}{%
    \usefont{OMS}{cmsy}{m}{n}%
  }%
}
%    \end{macrocode}
%    \end{macro}
%
%    \begin{macro}{\HoLogoFont@font@logo}
%    \begin{macrocode}
\begingroup
  \def\x{LaTeX2e}%
\expandafter\endgroup
\ifx\fmtname\x
  \ltx@IfUndefined{logofamily}{%
    \DeclareRobustCommand\logofamily{%
      \not@math@alphabet\logofamily\relax
      \fontencoding{U}%
      \fontfamily{logo}%
      \selectfont
    }%
  }{}%
  \ltx@IfUndefined{logofamily}{%
  }{%
    \HoLogoFont@Def{logo}{\logofamily}%
  }%
\else
  \ltx@IfUndefined{tenlogo}{%
    \font\tenlogo=logo10\relax
  }{}%
  \HoLogoFont@Def{logo}{\tenlogo}%
\fi
%    \end{macrocode}
%    \end{macro}
%
% \subsubsection{Font setup}
%
%    \begin{macro}{\hologoFontSetup}
%    \begin{macrocode}
\def\hologoFontSetup{%
  \let\HOLOGO@name\relax
  \HOLOGO@FontSetup
}
%    \end{macrocode}
%    \end{macro}
%
%    \begin{macro}{\hologoLogoFontSetup}
%    \begin{macrocode}
\def\hologoLogoFontSetup#1{%
  \edef\HOLOGO@name{#1}%
  \ltx@IfUndefined{HoLogo@\HOLOGO@name}{%
    \@PackageError{hologo}{%
      Unknown logo `\HOLOGO@name'%
    }\@ehc
    \ltx@gobble
  }{%
    \HOLOGO@FontSetup
  }%
}
%    \end{macrocode}
%    \end{macro}
%
%    \begin{macro}{\HOLOGO@FontSetup}
%    \begin{macrocode}
\def\HOLOGO@FontSetup{%
  \kvsetkeys{HoLogoFont}%
}
%    \end{macrocode}
%    \end{macro}
%
%    \begin{macrocode}
\def\HOLOGO@temp#1{%
  \kv@define@key{HoLogoFont}{#1}{%
    \ifx\HOLOGO@name\relax
      \HoLogoFont@Def{#1}{##1}%
    \else
      \HoLogoFont@LogoDef\HOLOGO@name{#1}{##1}%
    \fi
  }%
}
\HOLOGO@temp{general}
\HOLOGO@temp{sf}
%    \end{macrocode}
%
% \subsection{Generic logo commands}
%
%    \begin{macrocode}
\HOLOGO@IfExists\hologo{%
  \@PackageError{hologo}{%
    \string\hologo\ltx@space is already defined.\MessageBreak
    Package loading is aborted%
  }\@ehc
  \HOLOGO@AtEnd
}%
\HOLOGO@IfExists\hologoRobust{%
  \@PackageError{hologo}{%
    \string\hologoRobust\ltx@space is already defined.\MessageBreak
    Package loading is aborted%
  }\@ehc
  \HOLOGO@AtEnd
}%
%    \end{macrocode}
%
% \subsubsection{\cs{hologo} and friends}
%
%    \begin{macrocode}
\ifluatex
  \expandafter\ltx@firstofone
\else
  \expandafter\ltx@gobble
\fi
{%
  \ltx@IfUndefined{ifincsname}{%
    \ifnum\luatexversion<36 %
      \expandafter\ltx@gobble
    \else
      \expandafter\ltx@firstofone
    \fi
    {%
      \begingroup
        \ifcase0%
            \directlua{%
              if tex.enableprimitives then %
                tex.enableprimitives('HOLOGO@', {'ifincsname'})%
              else %
                tex.print('1')%
              end%
            }%
            \ifx\HOLOGO@ifincsname\@undefined 1\fi%
            \relax
          \expandafter\ltx@firstofone
        \else
          \endgroup
          \expandafter\ltx@gobble
        \fi
        {%
          \global\let\ifincsname\HOLOGO@ifincsname
        }%
      \HOLOGO@temp
    }%
  }{}%
}
%    \end{macrocode}
%    \begin{macrocode}
\ltx@IfUndefined{ifincsname}{%
  \catcode`$=14 %
}{%
  \catcode`$=9 %
}
%    \end{macrocode}
%
%    \begin{macro}{\hologo}
%    \begin{macrocode}
\def\hologo#1{%
$ \ifincsname
$   \ltx@ifundefined{HoLogoCs@\HOLOGO@Variant{#1}}{%
$     #1%
$   }{%
$     \csname HoLogoCs@\HOLOGO@Variant{#1}\endcsname\ltx@firstoftwo
$   }%
$ \else
    \HOLOGO@IfExists\texorpdfstring\texorpdfstring\ltx@firstoftwo
    {%
      \hologoRobust{#1}%
    }{%
      \ltx@ifundefined{HoLogoBkm@\HOLOGO@Variant{#1}}{%
        \ltx@ifundefined{HoLogo@#1}{?#1?}{#1}%
      }{%
        \csname HoLogoBkm@\HOLOGO@Variant{#1}\endcsname
        \ltx@firstoftwo
      }%
    }%
$ \fi
}
%    \end{macrocode}
%    \end{macro}
%    \begin{macro}{\Hologo}
%    \begin{macrocode}
\def\Hologo#1{%
$ \ifincsname
$   \ltx@ifundefined{HoLogoCs@\HOLOGO@Variant{#1}}{%
$     #1%
$   }{%
$     \csname HoLogoCs@\HOLOGO@Variant{#1}\endcsname\ltx@secondoftwo
$   }%
$ \else
    \HOLOGO@IfExists\texorpdfstring\texorpdfstring\ltx@firstoftwo
    {%
      \HologoRobust{#1}%
    }{%
      \ltx@ifundefined{HoLogoBkm@\HOLOGO@Variant{#1}}{%
        \ltx@ifundefined{HoLogo@#1}{?#1?}{#1}%
      }{%
        \csname HoLogoBkm@\HOLOGO@Variant{#1}\endcsname
        \ltx@secondoftwo
      }%
    }%
$ \fi
}
%    \end{macrocode}
%    \end{macro}
%
%    \begin{macro}{\hologoVariant}
%    \begin{macrocode}
\def\hologoVariant#1#2{%
  \ifx\relax#2\relax
    \hologo{#1}%
  \else
$   \ifincsname
$     \ltx@ifundefined{HoLogoCs@#1@#2}{%
$       #1%
$     }{%
$       \csname HoLogoCs@#1@#2\endcsname\ltx@firstoftwo
$     }%
$   \else
      \HOLOGO@IfExists\texorpdfstring\texorpdfstring\ltx@firstoftwo
      {%
        \hologoVariantRobust{#1}{#2}%
      }{%
        \ltx@ifundefined{HoLogoBkm@#1@#2}{%
          \ltx@ifundefined{HoLogo@#1}{?#1?}{#1}%
        }{%
          \csname HoLogoBkm@#1@#2\endcsname
          \ltx@firstoftwo
        }%
      }%
$   \fi
  \fi
}
%    \end{macrocode}
%    \end{macro}
%    \begin{macro}{\HologoVariant}
%    \begin{macrocode}
\def\HologoVariant#1#2{%
  \ifx\relax#2\relax
    \Hologo{#1}%
  \else
$   \ifincsname
$     \ltx@ifundefined{HoLogoCs@#1@#2}{%
$       #1%
$     }{%
$       \csname HoLogoCs@#1@#2\endcsname\ltx@secondoftwo
$     }%
$   \else
      \HOLOGO@IfExists\texorpdfstring\texorpdfstring\ltx@firstoftwo
      {%
        \HologoVariantRobust{#1}{#2}%
      }{%
        \ltx@ifundefined{HoLogoBkm@#1@#2}{%
          \ltx@ifundefined{HoLogo@#1}{?#1?}{#1}%
        }{%
          \csname HoLogoBkm@#1@#2\endcsname
          \ltx@secondoftwo
        }%
      }%
$   \fi
  \fi
}
%    \end{macrocode}
%    \end{macro}
%
%    \begin{macrocode}
\catcode`\$=3 %
%    \end{macrocode}
%
% \subsubsection{\cs{hologoRobust} and friends}
%
%    \begin{macro}{\hologoRobust}
%    \begin{macrocode}
\ltx@IfUndefined{protected}{%
  \ltx@IfUndefined{DeclareRobustCommand}{%
    \def\hologoRobust#1%
  }{%
    \DeclareRobustCommand*\hologoRobust[1]%
  }%
}{%
  \protected\def\hologoRobust#1%
}%
{%
  \edef\HOLOGO@name{#1}%
  \ltx@IfUndefined{HoLogo@\HOLOGO@Variant\HOLOGO@name}{%
    \@PackageError{hologo}{%
      Unknown logo `\HOLOGO@name'%
    }\@ehc
    ?\HOLOGO@name?%
  }{%
    \ltx@IfUndefined{ver@tex4ht.sty}{%
      \HoLogoFont@font\HOLOGO@name{general}{%
        \csname HoLogo@\HOLOGO@Variant\HOLOGO@name\endcsname
        \ltx@firstoftwo
      }%
    }{%
      \ltx@IfUndefined{HoLogoHtml@\HOLOGO@Variant\HOLOGO@name}{%
        \HOLOGO@name
      }{%
        \csname HoLogoHtml@\HOLOGO@Variant\HOLOGO@name\endcsname
        \ltx@firstoftwo
      }%
    }%
  }%
}
%    \end{macrocode}
%    \end{macro}
%    \begin{macro}{\HologoRobust}
%    \begin{macrocode}
\ltx@IfUndefined{protected}{%
  \ltx@IfUndefined{DeclareRobustCommand}{%
    \def\HologoRobust#1%
  }{%
    \DeclareRobustCommand*\HologoRobust[1]%
  }%
}{%
  \protected\def\HologoRobust#1%
}%
{%
  \edef\HOLOGO@name{#1}%
  \ltx@IfUndefined{HoLogo@\HOLOGO@Variant\HOLOGO@name}{%
    \@PackageError{hologo}{%
      Unknown logo `\HOLOGO@name'%
    }\@ehc
    ?\HOLOGO@name?%
  }{%
    \ltx@IfUndefined{ver@tex4ht.sty}{%
      \HoLogoFont@font\HOLOGO@name{general}{%
        \csname HoLogo@\HOLOGO@Variant\HOLOGO@name\endcsname
        \ltx@secondoftwo
      }%
    }{%
      \ltx@IfUndefined{HoLogoHtml@\HOLOGO@Variant\HOLOGO@name}{%
        \expandafter\HOLOGO@Uppercase\HOLOGO@name
      }{%
        \csname HoLogoHtml@\HOLOGO@Variant\HOLOGO@name\endcsname
        \ltx@secondoftwo
      }%
    }%
  }%
}
%    \end{macrocode}
%    \end{macro}
%    \begin{macro}{\hologoVariantRobust}
%    \begin{macrocode}
\ltx@IfUndefined{protected}{%
  \ltx@IfUndefined{DeclareRobustCommand}{%
    \def\hologoVariantRobust#1#2%
  }{%
    \DeclareRobustCommand*\hologoVariantRobust[2]%
  }%
}{%
  \protected\def\hologoVariantRobust#1#2%
}%
{%
  \begingroup
    \hologoLogoSetup{#1}{variant={#2}}%
    \hologoRobust{#1}%
  \endgroup
}
%    \end{macrocode}
%    \end{macro}
%    \begin{macro}{\HologoVariantRobust}
%    \begin{macrocode}
\ltx@IfUndefined{protected}{%
  \ltx@IfUndefined{DeclareRobustCommand}{%
    \def\HologoVariantRobust#1#2%
  }{%
    \DeclareRobustCommand*\HologoVariantRobust[2]%
  }%
}{%
  \protected\def\HologoVariantRobust#1#2%
}%
{%
  \begingroup
    \hologoLogoSetup{#1}{variant={#2}}%
    \HologoRobust{#1}%
  \endgroup
}
%    \end{macrocode}
%    \end{macro}
%
%    \begin{macro}{\hologorobust}
%    Macro \cs{hologorobust} is only defined for compatibility.
%    Its use is deprecated.
%    \begin{macrocode}
\def\hologorobust{\hologoRobust}
%    \end{macrocode}
%    \end{macro}
%
% \subsection{Helpers}
%
%    \begin{macro}{\HOLOGO@Uppercase}
%    Macro \cs{HOLOGO@Uppercase} is restricted to \cs{uppercase},
%    because \hologo{plainTeX} or \hologo{iniTeX} do not provide
%    \cs{MakeUppercase}.
%    \begin{macrocode}
\def\HOLOGO@Uppercase#1{\uppercase{#1}}
%    \end{macrocode}
%    \end{macro}
%
%    \begin{macro}{\HOLOGO@PdfdocUnicode}
%    \begin{macrocode}
\def\HOLOGO@PdfdocUnicode{%
  \ifx\ifHy@unicode\iftrue
    \expandafter\ltx@secondoftwo
  \else
    \expandafter\ltx@firstoftwo
  \fi
}
%    \end{macrocode}
%    \end{macro}
%
%    \begin{macro}{\HOLOGO@Math}
%    \begin{macrocode}
\def\HOLOGO@MathSetup{%
  \mathsurround0pt\relax
  \HOLOGO@IfExists\f@series{%
    \if b\expandafter\ltx@car\f@series x\@nil
      \csname boldmath\endcsname
   \fi
  }{}%
}
%    \end{macrocode}
%    \end{macro}
%
%    \begin{macro}{\HOLOGO@TempDimen}
%    \begin{macrocode}
\dimendef\HOLOGO@TempDimen=\ltx@zero
%    \end{macrocode}
%    \end{macro}
%    \begin{macro}{\HOLOGO@NegativeKerning}
%    \begin{macrocode}
\def\HOLOGO@NegativeKerning#1{%
  \begingroup
    \HOLOGO@TempDimen=0pt\relax
    \comma@parse@normalized{#1}{%
      \ifdim\HOLOGO@TempDimen=0pt %
        \expandafter\HOLOGO@@NegativeKerning\comma@entry
      \fi
      \ltx@gobble
    }%
    \ifdim\HOLOGO@TempDimen<0pt %
      \kern\HOLOGO@TempDimen
    \fi
  \endgroup
}
%    \end{macrocode}
%    \end{macro}
%    \begin{macro}{\HOLOGO@@NegativeKerning}
%    \begin{macrocode}
\def\HOLOGO@@NegativeKerning#1#2{%
  \setbox\ltx@zero\hbox{#1#2}%
  \HOLOGO@TempDimen=\wd\ltx@zero
  \setbox\ltx@zero\hbox{#1\kern0pt#2}%
  \advance\HOLOGO@TempDimen by -\wd\ltx@zero
}
%    \end{macrocode}
%    \end{macro}
%
%    \begin{macro}{\HOLOGO@SpaceFactor}
%    \begin{macrocode}
\def\HOLOGO@SpaceFactor{%
  \spacefactor1000 %
}
%    \end{macrocode}
%    \end{macro}
%
%    \begin{macro}{\HOLOGO@Span}
%    \begin{macrocode}
\def\HOLOGO@Span#1#2{%
  \HCode{<span class="HoLogo-#1">}%
  #2%
  \HCode{</span>}%
}
%    \end{macrocode}
%    \end{macro}
%
% \subsubsection{Text subscript}
%
%    \begin{macro}{\HOLOGO@SubScript}%
%    \begin{macrocode}
\def\HOLOGO@SubScript#1{%
  \ltx@IfUndefined{textsubscript}{%
    \ltx@IfUndefined{text}{%
      \ltx@mbox{%
        \mathsurround=0pt\relax
        $%
          _{%
            \ltx@IfUndefined{sf@size}{%
              \mathrm{#1}%
            }{%
              \mbox{%
                \fontsize\sf@size{0pt}\selectfont
                #1%
              }%
            }%
          }%
        $%
      }%
    }{%
      \ltx@mbox{%
        \mathsurround=0pt\relax
        $_{\text{#1}}$%
      }%
    }%
  }{%
    \textsubscript{#1}%
  }%
}
%    \end{macrocode}
%    \end{macro}
%
% \subsection{\hologo{TeX} and friends}
%
% \subsubsection{\hologo{TeX}}
%
%    \begin{macro}{\HoLogo@TeX}
%    Source: \hologo{LaTeX} kernel.
%    \begin{macrocode}
\def\HoLogo@TeX#1{%
  T\kern-.1667em\lower.5ex\hbox{E}\kern-.125emX\HOLOGO@SpaceFactor
}
%    \end{macrocode}
%    \end{macro}
%    \begin{macro}{\HoLogoHtml@TeX}
%    \begin{macrocode}
\def\HoLogoHtml@TeX#1{%
  \HoLogoCss@TeX
  \HOLOGO@Span{TeX}{%
    T%
    \HOLOGO@Span{e}{%
      E%
    }%
    X%
  }%
}
%    \end{macrocode}
%    \end{macro}
%    \begin{macro}{\HoLogoCss@TeX}
%    \begin{macrocode}
\def\HoLogoCss@TeX{%
  \Css{%
    span.HoLogo-TeX span.HoLogo-e{%
      position:relative;%
      top:.5ex;%
      margin-left:-.1667em;%
      margin-right:-.125em;%
    }%
  }%
  \Css{%
    a span.HoLogo-TeX span.HoLogo-e{%
      text-decoration:none;%
    }%
  }%
  \global\let\HoLogoCss@TeX\relax
}
%    \end{macrocode}
%    \end{macro}
%
% \subsubsection{\hologo{plainTeX}}
%
%    \begin{macro}{\HoLogo@plainTeX@space}
%    Source: ``The \hologo{TeX}book''
%    \begin{macrocode}
\def\HoLogo@plainTeX@space#1{%
  \HOLOGO@mbox{#1{p}{P}lain}\HOLOGO@space\hologo{TeX}%
}
%    \end{macrocode}
%    \end{macro}
%    \begin{macro}{\HoLogoCs@plainTeX@space}
%    \begin{macrocode}
\def\HoLogoCs@plainTeX@space#1{#1{p}{P}lain TeX}%
%    \end{macrocode}
%    \end{macro}
%    \begin{macro}{\HoLogoBkm@plainTeX@space}
%    \begin{macrocode}
\def\HoLogoBkm@plainTeX@space#1{%
  #1{p}{P}lain \hologo{TeX}%
}
%    \end{macrocode}
%    \end{macro}
%    \begin{macro}{\HoLogoHtml@plainTeX@space}
%    \begin{macrocode}
\def\HoLogoHtml@plainTeX@space#1{%
  #1{p}{P}lain \hologo{TeX}%
}
%    \end{macrocode}
%    \end{macro}
%
%    \begin{macro}{\HoLogo@plainTeX@hyphen}
%    \begin{macrocode}
\def\HoLogo@plainTeX@hyphen#1{%
  \HOLOGO@mbox{#1{p}{P}lain}\HOLOGO@hyphen\hologo{TeX}%
}
%    \end{macrocode}
%    \end{macro}
%    \begin{macro}{\HoLogoCs@plainTeX@hyphen}
%    \begin{macrocode}
\def\HoLogoCs@plainTeX@hyphen#1{#1{p}{P}lain-TeX}
%    \end{macrocode}
%    \end{macro}
%    \begin{macro}{\HoLogoBkm@plainTeX@hyphen}
%    \begin{macrocode}
\def\HoLogoBkm@plainTeX@hyphen#1{%
  #1{p}{P}lain-\hologo{TeX}%
}
%    \end{macrocode}
%    \end{macro}
%    \begin{macro}{\HoLogoHtml@plainTeX@hyphen}
%    \begin{macrocode}
\def\HoLogoHtml@plainTeX@hyphen#1{%
  #1{p}{P}lain-\hologo{TeX}%
}
%    \end{macrocode}
%    \end{macro}
%
%    \begin{macro}{\HoLogo@plainTeX@runtogether}
%    \begin{macrocode}
\def\HoLogo@plainTeX@runtogether#1{%
  \HOLOGO@mbox{#1{p}{P}lain\hologo{TeX}}%
}
%    \end{macrocode}
%    \end{macro}
%    \begin{macro}{\HoLogoCs@plainTeX@runtogether}
%    \begin{macrocode}
\def\HoLogoCs@plainTeX@runtogether#1{#1{p}{P}lainTeX}
%    \end{macrocode}
%    \end{macro}
%    \begin{macro}{\HoLogoBkm@plainTeX@runtogether}
%    \begin{macrocode}
\def\HoLogoBkm@plainTeX@runtogether#1{%
  #1{p}{P}lain\hologo{TeX}%
}
%    \end{macrocode}
%    \end{macro}
%    \begin{macro}{\HoLogoHtml@plainTeX@runtogether}
%    \begin{macrocode}
\def\HoLogoHtml@plainTeX@runtogether#1{%
  #1{p}{P}lain\hologo{TeX}%
}
%    \end{macrocode}
%    \end{macro}
%
%    \begin{macro}{\HoLogo@plainTeX}
%    \begin{macrocode}
\def\HoLogo@plainTeX{\HoLogo@plainTeX@space}
%    \end{macrocode}
%    \end{macro}
%    \begin{macro}{\HoLogoCs@plainTeX}
%    \begin{macrocode}
\def\HoLogoCs@plainTeX{\HoLogoCs@plainTeX@space}
%    \end{macrocode}
%    \end{macro}
%    \begin{macro}{\HoLogoBkm@plainTeX}
%    \begin{macrocode}
\def\HoLogoBkm@plainTeX{\HoLogoBkm@plainTeX@space}
%    \end{macrocode}
%    \end{macro}
%    \begin{macro}{\HoLogoHtml@plainTeX}
%    \begin{macrocode}
\def\HoLogoHtml@plainTeX{\HoLogoHtml@plainTeX@space}
%    \end{macrocode}
%    \end{macro}
%
% \subsubsection{\hologo{LaTeX}}
%
%    Source: \hologo{LaTeX} kernel.
%\begin{quote}
%\begin{verbatim}
%\DeclareRobustCommand{\LaTeX}{%
%  L%
%  \kern-.36em%
%  {%
%    \sbox\z@ T%
%    \vbox to\ht\z@{%
%      \hbox{%
%        \check@mathfonts
%        \fontsize\sf@size\z@
%        \math@fontsfalse
%        \selectfont
%        A%
%      }%
%      \vss
%    }%
%  }%
%  \kern-.15em%
%  \TeX
%}
%\end{verbatim}
%\end{quote}
%
%    \begin{macro}{\HoLogo@La}
%    \begin{macrocode}
\def\HoLogo@La#1{%
  L%
  \kern-.36em%
  \begingroup
    \setbox\ltx@zero\hbox{T}%
    \vbox to\ht\ltx@zero{%
      \hbox{%
        \ltx@ifundefined{check@mathfonts}{%
          \csname sevenrm\endcsname
        }{%
          \check@mathfonts
          \fontsize\sf@size{0pt}%
          \math@fontsfalse\selectfont
        }%
        A%
      }%
      \vss
    }%
  \endgroup
}
%    \end{macrocode}
%    \end{macro}
%
%    \begin{macro}{\HoLogo@LaTeX}
%    Source: \hologo{LaTeX} kernel.
%    \begin{macrocode}
\def\HoLogo@LaTeX#1{%
  \hologo{La}%
  \kern-.15em%
  \hologo{TeX}%
}
%    \end{macrocode}
%    \end{macro}
%    \begin{macro}{\HoLogoHtml@LaTeX}
%    \begin{macrocode}
\def\HoLogoHtml@LaTeX#1{%
  \HoLogoCss@LaTeX
  \HOLOGO@Span{LaTeX}{%
    L%
    \HOLOGO@Span{a}{%
      A%
    }%
    \hologo{TeX}%
  }%
}
%    \end{macrocode}
%    \end{macro}
%    \begin{macro}{\HoLogoCss@LaTeX}
%    \begin{macrocode}
\def\HoLogoCss@LaTeX{%
  \Css{%
    span.HoLogo-LaTeX span.HoLogo-a{%
      position:relative;%
      top:-.5ex;%
      margin-left:-.36em;%
      margin-right:-.15em;%
      font-size:85\%;%
    }%
  }%
  \global\let\HoLogoCss@LaTeX\relax
}
%    \end{macrocode}
%    \end{macro}
%
% \subsubsection{\hologo{(La)TeX}}
%
%    \begin{macro}{\HoLogo@LaTeXTeX}
%    The kerning around the parentheses is taken
%    from package \xpackage{dtklogos} \cite{dtklogos}.
%\begin{quote}
%\begin{verbatim}
%\DeclareRobustCommand{\LaTeXTeX}{%
%  (%
%  \kern-.15em%
%  L%
%  \kern-.36em%
%  {%
%    \sbox\z@ T%
%    \vbox to\ht0{%
%      \hbox{%
%        $\m@th$%
%        \csname S@\f@size\endcsname
%        \fontsize\sf@size\z@
%        \math@fontsfalse
%        \selectfont
%        A%
%      }%
%      \vss
%    }%
%  }%
%  \kern-.2em%
%  )%
%  \kern-.15em%
%  \TeX
%}
%\end{verbatim}
%\end{quote}
%    \begin{macrocode}
\def\HoLogo@LaTeXTeX#1{%
  (%
  \kern-.15em%
  \hologo{La}%
  \kern-.2em%
  )%
  \kern-.15em%
  \hologo{TeX}%
}
%    \end{macrocode}
%    \end{macro}
%    \begin{macro}{\HoLogoBkm@LaTeXTeX}
%    \begin{macrocode}
\def\HoLogoBkm@LaTeXTeX#1{(La)TeX}
%    \end{macrocode}
%    \end{macro}
%
%    \begin{macro}{\HoLogo@(La)TeX}
%    \begin{macrocode}
\expandafter
\let\csname HoLogo@(La)TeX\endcsname\HoLogo@LaTeXTeX
%    \end{macrocode}
%    \end{macro}
%    \begin{macro}{\HoLogoBkm@(La)TeX}
%    \begin{macrocode}
\expandafter
\let\csname HoLogoBkm@(La)TeX\endcsname\HoLogoBkm@LaTeXTeX
%    \end{macrocode}
%    \end{macro}
%    \begin{macro}{\HoLogoHtml@LaTeXTeX}
%    \begin{macrocode}
\def\HoLogoHtml@LaTeXTeX#1{%
  \HoLogoCss@LaTeXTeX
  \HOLOGO@Span{LaTeXTeX}{%
    (%
    \HOLOGO@Span{L}{L}%
    \HOLOGO@Span{a}{A}%
    \HOLOGO@Span{ParenRight}{)}%
    \hologo{TeX}%
  }%
}
%    \end{macrocode}
%    \end{macro}
%    \begin{macro}{\HoLogoHtml@(La)TeX}
%    Kerning after opening parentheses and before closing parentheses
%    is $-0.1$\,em. The original values $-0.15$\,em
%    looked too ugly for a serif font.
%    \begin{macrocode}
\expandafter
\let\csname HoLogoHtml@(La)TeX\endcsname\HoLogoHtml@LaTeXTeX
%    \end{macrocode}
%    \end{macro}
%    \begin{macro}{\HoLogoCss@LaTeXTeX}
%    \begin{macrocode}
\def\HoLogoCss@LaTeXTeX{%
  \Css{%
    span.HoLogo-LaTeXTeX span.HoLogo-L{%
      margin-left:-.1em;%
    }%
  }%
  \Css{%
    span.HoLogo-LaTeXTeX span.HoLogo-a{%
      position:relative;%
      top:-.5ex;%
      margin-left:-.36em;%
      margin-right:-.1em;%
      font-size:85\%;%
    }%
  }%
  \Css{%
    span.HoLogo-LaTeXTeX span.HoLogo-ParenRight{%
      margin-right:-.15em;%
    }%
  }%
  \global\let\HoLogoCss@LaTeXTeX\relax
}
%    \end{macrocode}
%    \end{macro}
%
% \subsubsection{\hologo{LaTeXe}}
%
%    \begin{macro}{\HoLogo@LaTeXe}
%    Source: \hologo{LaTeX} kernel
%    \begin{macrocode}
\def\HoLogo@LaTeXe#1{%
  \hologo{LaTeX}%
  \kern.15em%
  \hbox{%
    \HOLOGO@MathSetup
    2%
    $_{\textstyle\varepsilon}$%
  }%
}
%    \end{macrocode}
%    \end{macro}
%
%    \begin{macro}{\HoLogoCs@LaTeXe}
%    \begin{macrocode}
\ifnum64=`\^^^^0040\relax % test for big chars of LuaTeX/XeTeX
  \catcode`\$=9 %
  \catcode`\&=14 %
\else
  \catcode`\$=14 %
  \catcode`\&=9 %
\fi
\def\HoLogoCs@LaTeXe#1{%
  LaTeX2%
$ \string ^^^^0395%
& e%
}%
\catcode`\$=3 %
\catcode`\&=4 %
%    \end{macrocode}
%    \end{macro}
%
%    \begin{macro}{\HoLogoBkm@LaTeXe}
%    \begin{macrocode}
\def\HoLogoBkm@LaTeXe#1{%
  \hologo{LaTeX}%
  2%
  \HOLOGO@PdfdocUnicode{e}{\textepsilon}%
}
%    \end{macrocode}
%    \end{macro}
%
%    \begin{macro}{\HoLogoHtml@LaTeXe}
%    \begin{macrocode}
\def\HoLogoHtml@LaTeXe#1{%
  \HoLogoCss@LaTeXe
  \HOLOGO@Span{LaTeX2e}{%
    \hologo{LaTeX}%
    \HOLOGO@Span{2}{2}%
    \HOLOGO@Span{e}{%
      \HOLOGO@MathSetup
      \ensuremath{\textstyle\varepsilon}%
    }%
  }%
}
%    \end{macrocode}
%    \end{macro}
%    \begin{macro}{\HoLogoCss@LaTeXe}
%    \begin{macrocode}
\def\HoLogoCss@LaTeXe{%
  \Css{%
    span.HoLogo-LaTeX2e span.HoLogo-2{%
      padding-left:.15em;%
    }%
  }%
  \Css{%
    span.HoLogo-LaTeX2e span.HoLogo-e{%
      position:relative;%
      top:.35ex;%
      text-decoration:none;%
    }%
  }%
  \global\let\HoLogoCss@LaTeXe\relax
}
%    \end{macrocode}
%    \end{macro}
%
%    \begin{macro}{\HoLogo@LaTeX2e}
%    \begin{macrocode}
\expandafter
\let\csname HoLogo@LaTeX2e\endcsname\HoLogo@LaTeXe
%    \end{macrocode}
%    \end{macro}
%    \begin{macro}{\HoLogoCs@LaTeX2e}
%    \begin{macrocode}
\expandafter
\let\csname HoLogoCs@LaTeX2e\endcsname\HoLogoCs@LaTeXe
%    \end{macrocode}
%    \end{macro}
%    \begin{macro}{\HoLogoBkm@LaTeX2e}
%    \begin{macrocode}
\expandafter
\let\csname HoLogoBkm@LaTeX2e\endcsname\HoLogoBkm@LaTeXe
%    \end{macrocode}
%    \end{macro}
%    \begin{macro}{\HoLogoHtml@LaTeX2e}
%    \begin{macrocode}
\expandafter
\let\csname HoLogoHtml@LaTeX2e\endcsname\HoLogoHtml@LaTeXe
%    \end{macrocode}
%    \end{macro}
%
% \subsubsection{\hologo{LaTeX3}}
%
%    \begin{macro}{\HoLogo@LaTeX3}
%    Source: \hologo{LaTeX} kernel
%    \begin{macrocode}
\expandafter\def\csname HoLogo@LaTeX3\endcsname#1{%
  \hologo{LaTeX}%
  3%
}
%    \end{macrocode}
%    \end{macro}
%
%    \begin{macro}{\HoLogoBkm@LaTeX3}
%    \begin{macrocode}
\expandafter\def\csname HoLogoBkm@LaTeX3\endcsname#1{%
  \hologo{LaTeX}%
  3%
}
%    \end{macrocode}
%    \end{macro}
%    \begin{macro}{\HoLogoHtml@LaTeX3}
%    \begin{macrocode}
\expandafter
\let\csname HoLogoHtml@LaTeX3\expandafter\endcsname
\csname HoLogo@LaTeX3\endcsname
%    \end{macrocode}
%    \end{macro}
%
% \subsubsection{\hologo{LaTeXML}}
%
%    \begin{macro}{\HoLogo@LaTeXML}
%    \begin{macrocode}
\def\HoLogo@LaTeXML#1{%
  \HOLOGO@mbox{%
    \hologo{La}%
    \kern-.15em%
    T%
    \kern-.1667em%
    \lower.5ex\hbox{E}%
    \kern-.125em%
    \HoLogoFont@font{LaTeXML}{sc}{xml}%
  }%
}
%    \end{macrocode}
%    \end{macro}
%    \begin{macro}{\HoLogoHtml@pdfLaTeX}
%    \begin{macrocode}
\def\HoLogoHtml@LaTeXML#1{%
  \HOLOGO@Span{LaTeXML}{%
    \HoLogoCss@LaTeX
    \HoLogoCss@TeX
    \HOLOGO@Span{LaTeX}{%
      L%
      \HOLOGO@Span{a}{%
        A%
      }%
    }%
    \HOLOGO@Span{TeX}{%
      T%
      \HOLOGO@Span{e}{%
        E%
      }%
    }%
    \HCode{<span style="font-variant: small-caps;">}%
    xml%
    \HCode{</span>}%
  }%
}
%    \end{macrocode}
%    \end{macro}
%
% \subsubsection{\hologo{eTeX}}
%
%    \begin{macro}{\HoLogo@eTeX}
%    Source: package \xpackage{etex}
%    \begin{macrocode}
\def\HoLogo@eTeX#1{%
  \ltx@mbox{%
    \HOLOGO@MathSetup
    $\varepsilon$%
    -%
    \HOLOGO@NegativeKerning{-T,T-,To}%
    \hologo{TeX}%
  }%
}
%    \end{macrocode}
%    \end{macro}
%    \begin{macro}{\HoLogoCs@eTeX}
%    \begin{macrocode}
\ifnum64=`\^^^^0040\relax % test for big chars of LuaTeX/XeTeX
  \catcode`\$=9 %
  \catcode`\&=14 %
\else
  \catcode`\$=14 %
  \catcode`\&=9 %
\fi
\def\HoLogoCs@eTeX#1{%
$ #1{\string ^^^^0395}{\string ^^^^03b5}%
& #1{e}{E}%
  TeX%
}%
\catcode`\$=3 %
\catcode`\&=4 %
%    \end{macrocode}
%    \end{macro}
%    \begin{macro}{\HoLogoBkm@eTeX}
%    \begin{macrocode}
\def\HoLogoBkm@eTeX#1{%
  \HOLOGO@PdfdocUnicode{#1{e}{E}}{\textepsilon}%
  -%
  \hologo{TeX}%
}
%    \end{macrocode}
%    \end{macro}
%    \begin{macro}{\HoLogoHtml@eTeX}
%    \begin{macrocode}
\def\HoLogoHtml@eTeX#1{%
  \ltx@mbox{%
    \HOLOGO@MathSetup
    $\varepsilon$%
    -%
    \hologo{TeX}%
  }%
}
%    \end{macrocode}
%    \end{macro}
%
% \subsubsection{\hologo{iniTeX}}
%
%    \begin{macro}{\HoLogo@iniTeX}
%    \begin{macrocode}
\def\HoLogo@iniTeX#1{%
  \HOLOGO@mbox{%
    #1{i}{I}ni\hologo{TeX}%
  }%
}
%    \end{macrocode}
%    \end{macro}
%    \begin{macro}{\HoLogoCs@iniTeX}
%    \begin{macrocode}
\def\HoLogoCs@iniTeX#1{#1{i}{I}niTeX}
%    \end{macrocode}
%    \end{macro}
%    \begin{macro}{\HoLogoBkm@iniTeX}
%    \begin{macrocode}
\def\HoLogoBkm@iniTeX#1{%
  #1{i}{I}ni\hologo{TeX}%
}
%    \end{macrocode}
%    \end{macro}
%    \begin{macro}{\HoLogoHtml@iniTeX}
%    \begin{macrocode}
\let\HoLogoHtml@iniTeX\HoLogo@iniTeX
%    \end{macrocode}
%    \end{macro}
%
% \subsubsection{\hologo{virTeX}}
%
%    \begin{macro}{\HoLogo@virTeX}
%    \begin{macrocode}
\def\HoLogo@virTeX#1{%
  \HOLOGO@mbox{%
    #1{v}{V}ir\hologo{TeX}%
  }%
}
%    \end{macrocode}
%    \end{macro}
%    \begin{macro}{\HoLogoCs@virTeX}
%    \begin{macrocode}
\def\HoLogoCs@virTeX#1{#1{v}{V}irTeX}
%    \end{macrocode}
%    \end{macro}
%    \begin{macro}{\HoLogoBkm@virTeX}
%    \begin{macrocode}
\def\HoLogoBkm@virTeX#1{%
  #1{v}{V}ir\hologo{TeX}%
}
%    \end{macrocode}
%    \end{macro}
%    \begin{macro}{\HoLogoHtml@virTeX}
%    \begin{macrocode}
\let\HoLogoHtml@virTeX\HoLogo@virTeX
%    \end{macrocode}
%    \end{macro}
%
% \subsubsection{\hologo{SliTeX}}
%
% \paragraph{Definitions of the three variants.}
%
%    \begin{macro}{\HoLogo@SLiTeX@lift}
%    \begin{macrocode}
\def\HoLogo@SLiTeX@lift#1{%
  \HoLogoFont@font{SliTeX}{rm}{%
    S%
    \kern-.06em%
    L%
    \kern-.18em%
    \raise.32ex\hbox{\HoLogoFont@font{SliTeX}{sc}{i}}%
    \HOLOGO@discretionary
    \kern-.06em%
    \hologo{TeX}%
  }%
}
%    \end{macrocode}
%    \end{macro}
%    \begin{macro}{\HoLogoBkm@SLiTeX@lift}
%    \begin{macrocode}
\def\HoLogoBkm@SLiTeX@lift#1{SLiTeX}
%    \end{macrocode}
%    \end{macro}
%    \begin{macro}{\HoLogoHtml@SLiTeX@lift}
%    \begin{macrocode}
\def\HoLogoHtml@SLiTeX@lift#1{%
  \HoLogoCss@SLiTeX@lift
  \HOLOGO@Span{SLiTeX-lift}{%
    \HoLogoFont@font{SliTeX}{rm}{%
      S%
      \HOLOGO@Span{L}{L}%
      \HOLOGO@Span{i}{i}%
      \hologo{TeX}%
    }%
  }%
}
%    \end{macrocode}
%    \end{macro}
%    \begin{macro}{\HoLogoCss@SLiTeX@lift}
%    \begin{macrocode}
\def\HoLogoCss@SLiTeX@lift{%
  \Css{%
    span.HoLogo-SLiTeX-lift span.HoLogo-L{%
      margin-left:-.06em;%
      margin-right:-.18em;%
    }%
  }%
  \Css{%
    span.HoLogo-SLiTeX-lift span.HoLogo-i{%
      position:relative;%
      top:-.32ex;%
      margin-right:-.06em;%
      font-variant:small-caps;%
    }%
  }%
  \global\let\HoLogoCss@SLiTeX@lift\relax
}
%    \end{macrocode}
%    \end{macro}
%
%    \begin{macro}{\HoLogo@SliTeX@simple}
%    \begin{macrocode}
\def\HoLogo@SliTeX@simple#1{%
  \HoLogoFont@font{SliTeX}{rm}{%
    \ltx@mbox{%
      \HoLogoFont@font{SliTeX}{sc}{Sli}%
    }%
    \HOLOGO@discretionary
    \hologo{TeX}%
  }%
}
%    \end{macrocode}
%    \end{macro}
%    \begin{macro}{\HoLogoBkm@SliTeX@simple}
%    \begin{macrocode}
\def\HoLogoBkm@SliTeX@simple#1{SliTeX}
%    \end{macrocode}
%    \end{macro}
%    \begin{macro}{\HoLogoHtml@SliTeX@simple}
%    \begin{macrocode}
\let\HoLogoHtml@SliTeX@simple\HoLogo@SliTeX@simple
%    \end{macrocode}
%    \end{macro}
%
%    \begin{macro}{\HoLogo@SliTeX@narrow}
%    \begin{macrocode}
\def\HoLogo@SliTeX@narrow#1{%
  \HoLogoFont@font{SliTeX}{rm}{%
    \ltx@mbox{%
      S%
      \kern-.06em%
      \HoLogoFont@font{SliTeX}{sc}{%
        l%
        \kern-.035em%
        i%
      }%
    }%
    \HOLOGO@discretionary
    \kern-.06em%
    \hologo{TeX}%
  }%
}
%    \end{macrocode}
%    \end{macro}
%    \begin{macro}{\HoLogoBkm@SliTeX@narrow}
%    \begin{macrocode}
\def\HoLogoBkm@SliTeX@narrow#1{SliTeX}
%    \end{macrocode}
%    \end{macro}
%    \begin{macro}{\HoLogoHtml@SliTeX@narrow}
%    \begin{macrocode}
\def\HoLogoHtml@SliTeX@narrow#1{%
  \HoLogoCss@SliTeX@narrow
  \HOLOGO@Span{SliTeX-narrow}{%
    \HoLogoFont@font{SliTeX}{rm}{%
      S%
        \HOLOGO@Span{l}{l}%
        \HOLOGO@Span{i}{i}%
      \hologo{TeX}%
    }%
  }%
}
%    \end{macrocode}
%    \end{macro}
%    \begin{macro}{\HoLogoCss@SliTeX@narrow}
%    \begin{macrocode}
\def\HoLogoCss@SliTeX@narrow{%
  \Css{%
    span.HoLogo-SliTeX-narrow span.HoLogo-l{%
      margin-left:-.06em;%
      margin-right:-.035em;%
      font-variant:small-caps;%
    }%
  }%
  \Css{%
    span.HoLogo-SliTeX-narrow span.HoLogo-i{%
      margin-right:-.06em;%
      font-variant:small-caps;%
    }%
  }%
  \global\let\HoLogoCss@SliTeX@narrow\relax
}
%    \end{macrocode}
%    \end{macro}
%
% \paragraph{Macro set completion.}
%
%    \begin{macro}{\HoLogo@SLiTeX@simple}
%    \begin{macrocode}
\def\HoLogo@SLiTeX@simple{\HoLogo@SliTeX@simple}
%    \end{macrocode}
%    \end{macro}
%    \begin{macro}{\HoLogoBkm@SLiTeX@simple}
%    \begin{macrocode}
\def\HoLogoBkm@SLiTeX@simple{\HoLogoBkm@SliTeX@simple}
%    \end{macrocode}
%    \end{macro}
%    \begin{macro}{\HoLogoHtml@SLiTeX@simple}
%    \begin{macrocode}
\def\HoLogoHtml@SLiTeX@simple{\HoLogoHtml@SliTeX@simple}
%    \end{macrocode}
%    \end{macro}
%
%    \begin{macro}{\HoLogo@SLiTeX@narrow}
%    \begin{macrocode}
\def\HoLogo@SLiTeX@narrow{\HoLogo@SliTeX@narrow}
%    \end{macrocode}
%    \end{macro}
%    \begin{macro}{\HoLogoBkm@SLiTeX@narrow}
%    \begin{macrocode}
\def\HoLogoBkm@SLiTeX@narrow{\HoLogoBkm@SliTeX@narrow}
%    \end{macrocode}
%    \end{macro}
%    \begin{macro}{\HoLogoHtml@SLiTeX@narrow}
%    \begin{macrocode}
\def\HoLogoHtml@SLiTeX@narrow{\HoLogoHtml@SliTeX@narrow}
%    \end{macrocode}
%    \end{macro}
%
%    \begin{macro}{\HoLogo@SliTeX@lift}
%    \begin{macrocode}
\def\HoLogo@SliTeX@lift{\HoLogo@SLiTeX@lift}
%    \end{macrocode}
%    \end{macro}
%    \begin{macro}{\HoLogoBkm@SliTeX@lift}
%    \begin{macrocode}
\def\HoLogoBkm@SliTeX@lift{\HoLogoBkm@SLiTeX@lift}
%    \end{macrocode}
%    \end{macro}
%    \begin{macro}{\HoLogoHtml@SliTeX@lift}
%    \begin{macrocode}
\def\HoLogoHtml@SliTeX@lift{\HoLogoHtml@SLiTeX@lift}
%    \end{macrocode}
%    \end{macro}
%
% \paragraph{Defaults.}
%
%    \begin{macro}{\HoLogo@SLiTeX}
%    \begin{macrocode}
\def\HoLogo@SLiTeX{\HoLogo@SLiTeX@lift}
%    \end{macrocode}
%    \end{macro}
%    \begin{macro}{\HoLogoBkm@SLiTeX}
%    \begin{macrocode}
\def\HoLogoBkm@SLiTeX{\HoLogoBkm@SLiTeX@lift}
%    \end{macrocode}
%    \end{macro}
%    \begin{macro}{\HoLogoHtml@SLiTeX}
%    \begin{macrocode}
\def\HoLogoHtml@SLiTeX{\HoLogoHtml@SLiTeX@lift}
%    \end{macrocode}
%    \end{macro}
%
%    \begin{macro}{\HoLogo@SliTeX}
%    \begin{macrocode}
\def\HoLogo@SliTeX{\HoLogo@SliTeX@narrow}
%    \end{macrocode}
%    \end{macro}
%    \begin{macro}{\HoLogoBkm@SliTeX}
%    \begin{macrocode}
\def\HoLogoBkm@SliTeX{\HoLogoBkm@SliTeX@narrow}
%    \end{macrocode}
%    \end{macro}
%    \begin{macro}{\HoLogoHtml@SliTeX}
%    \begin{macrocode}
\def\HoLogoHtml@SliTeX{\HoLogoHtml@SliTeX@narrow}
%    \end{macrocode}
%    \end{macro}
%
% \subsubsection{\hologo{LuaTeX}}
%
%    \begin{macro}{\HoLogo@LuaTeX}
%    The kerning is an idea of Hans Hagen, see mailing list
%    `luatex at tug dot org' in March 2010.
%    \begin{macrocode}
\def\HoLogo@LuaTeX#1{%
  \HOLOGO@mbox{%
    Lua%
    \HOLOGO@NegativeKerning{aT,oT,To}%
    \hologo{TeX}%
  }%
}
%    \end{macrocode}
%    \end{macro}
%    \begin{macro}{\HoLogoHtml@LuaTeX}
%    \begin{macrocode}
\let\HoLogoHtml@LuaTeX\HoLogo@LuaTeX
%    \end{macrocode}
%    \end{macro}
%
% \subsubsection{\hologo{LuaLaTeX}}
%
%    \begin{macro}{\HoLogo@LuaLaTeX}
%    \begin{macrocode}
\def\HoLogo@LuaLaTeX#1{%
  \HOLOGO@mbox{%
    Lua%
    \hologo{LaTeX}%
  }%
}
%    \end{macrocode}
%    \end{macro}
%    \begin{macro}{\HoLogoHtml@LuaLaTeX}
%    \begin{macrocode}
\let\HoLogoHtml@LuaLaTeX\HoLogo@LuaLaTeX
%    \end{macrocode}
%    \end{macro}
%
% \subsubsection{\hologo{XeTeX}, \hologo{XeLaTeX}}
%
%    \begin{macro}{\HOLOGO@IfCharExists}
%    \begin{macrocode}
\ifluatex
  \ifnum\luatexversion<36 %
  \else
    \def\HOLOGO@IfCharExists#1{%
      \ifnum
        \directlua{%
           if luaotfload and luaotfload.aux then
             if luaotfload.aux.font_has_glyph(%
                    font.current(), \number#1) then % 	 
	       tex.print("1") % 	 
	     end % 	 
	   elseif font and font.fonts and font.current then %
            local f = font.fonts[font.current()]%
            if f.characters and f.characters[\number#1] then %
              tex.print("1")%
            end %
          end%
        }0=\ltx@zero
        \expandafter\ltx@secondoftwo
      \else
        \expandafter\ltx@firstoftwo
      \fi
    }%
  \fi
\fi
\ltx@IfUndefined{HOLOGO@IfCharExists}{%
  \def\HOLOGO@@IfCharExists#1{%
    \begingroup
      \tracinglostchars=\ltx@zero
      \setbox\ltx@zero=\hbox{%
        \kern7sp\char#1\relax
        \ifnum\lastkern>\ltx@zero
          \expandafter\aftergroup\csname iffalse\endcsname
        \else
          \expandafter\aftergroup\csname iftrue\endcsname
        \fi
      }%
      % \if{true|false} from \aftergroup
      \endgroup
      \expandafter\ltx@firstoftwo
    \else
      \endgroup
      \expandafter\ltx@secondoftwo
    \fi
  }%
  \ifxetex
    \ltx@IfUndefined{XeTeXfonttype}{}{%
      \ltx@IfUndefined{XeTeXcharglyph}{}{%
        \def\HOLOGO@IfCharExists#1{%
          \ifnum\XeTeXfonttype\font>\ltx@zero
            \expandafter\ltx@firstofthree
          \else
            \expandafter\ltx@gobble
          \fi
          {%
            \ifnum\XeTeXcharglyph#1>\ltx@zero
              \expandafter\ltx@firstoftwo
            \else
              \expandafter\ltx@secondoftwo
            \fi
          }%
          \HOLOGO@@IfCharExists{#1}%
        }%
      }%
    }%
  \fi
}{}
\ltx@ifundefined{HOLOGO@IfCharExists}{%
  \ifnum64=`\^^^^0040\relax % test for big chars of LuaTeX/XeTeX
    \let\HOLOGO@IfCharExists\HOLOGO@@IfCharExists
  \else
    \def\HOLOGO@IfCharExists#1{%
      \ifnum#1>255 %
        \expandafter\ltx@fourthoffour
      \fi
      \HOLOGO@@IfCharExists{#1}%
    }%
  \fi
}{}
%    \end{macrocode}
%    \end{macro}
%
%    \begin{macro}{\HoLogo@Xe}
%    Source: package \xpackage{dtklogos}
%    \begin{macrocode}
\def\HoLogo@Xe#1{%
  X%
  \kern-.1em\relax
  \HOLOGO@IfCharExists{"018E}{%
    \lower.5ex\hbox{\char"018E}%
  }{%
    \chardef\HOLOGO@choice=\ltx@zero
    \ifdim\fontdimen\ltx@one\font>0pt %
      \ltx@IfUndefined{rotatebox}{%
        \ltx@IfUndefined{pgftext}{%
          \ltx@IfUndefined{psscalebox}{%
            \ltx@IfUndefined{HOLOGO@ScaleBox@\hologoDriver}{%
            }{%
              \chardef\HOLOGO@choice=4 %
            }%
          }{%
            \chardef\HOLOGO@choice=3 %
          }%
        }{%
          \chardef\HOLOGO@choice=2 %
        }%
      }{%
        \chardef\HOLOGO@choice=1 %
      }%
      \ifcase\HOLOGO@choice
        \HOLOGO@WarningUnsupportedDriver{Xe}%
        e%
      \or % 1: \rotatebox
        \begingroup
          \setbox\ltx@zero\hbox{\rotatebox{180}{E}}%
          \ltx@LocDimenA=\dp\ltx@zero
          \advance\ltx@LocDimenA by -.5ex\relax
          \raise\ltx@LocDimenA\box\ltx@zero
        \endgroup
      \or % 2: \pgftext
        \lower.5ex\hbox{%
          \pgfpicture
            \pgftext[rotate=180]{E}%
          \endpgfpicture
        }%
      \or % 3: \psscalebox
        \begingroup
          \setbox\ltx@zero\hbox{\psscalebox{-1 -1}{E}}%
          \ltx@LocDimenA=\dp\ltx@zero
          \advance\ltx@LocDimenA by -.5ex\relax
          \raise\ltx@LocDimenA\box\ltx@zero
        \endgroup
      \or % 4: \HOLOGO@PointReflectBox
        \lower.5ex\hbox{\HOLOGO@PointReflectBox{E}}%
      \else
        \@PackageError{hologo}{Internal error (choice/it}\@ehc
      \fi
    \else
      \ltx@IfUndefined{reflectbox}{%
        \ltx@IfUndefined{pgftext}{%
          \ltx@IfUndefined{psscalebox}{%
            \ltx@IfUndefined{HOLOGO@ScaleBox@\hologoDriver}{%
            }{%
              \chardef\HOLOGO@choice=4 %
            }%
          }{%
            \chardef\HOLOGO@choice=3 %
          }%
        }{%
          \chardef\HOLOGO@choice=2 %
        }%
      }{%
        \chardef\HOLOGO@choice=1 %
      }%
      \ifcase\HOLOGO@choice
        \HOLOGO@WarningUnsupportedDriver{Xe}%
        e%
      \or % 1: reflectbox
        \lower.5ex\hbox{%
          \reflectbox{E}%
        }%
      \or % 2: \pgftext
        \lower.5ex\hbox{%
          \pgfpicture
            \pgftransformxscale{-1}%
            \pgftext{E}%
          \endpgfpicture
        }%
      \or % 3: \psscalebox
        \lower.5ex\hbox{%
          \psscalebox{-1 1}{E}%
        }%
      \or % 4: \HOLOGO@Reflectbox
        \lower.5ex\hbox{%
          \HOLOGO@ReflectBox{E}%
        }%
      \else
        \@PackageError{hologo}{Internal error (choice/up)}\@ehc
      \fi
    \fi
  }%
}
%    \end{macrocode}
%    \end{macro}
%    \begin{macro}{\HoLogoHtml@Xe}
%    \begin{macrocode}
\def\HoLogoHtml@Xe#1{%
  \HoLogoCss@Xe
  \HOLOGO@Span{Xe}{%
    X%
    \HOLOGO@Span{e}{%
      \HCode{&\ltx@hashchar x018e;}%
    }%
  }%
}
%    \end{macrocode}
%    \end{macro}
%    \begin{macro}{\HoLogoCss@Xe}
%    \begin{macrocode}
\def\HoLogoCss@Xe{%
  \Css{%
    span.HoLogo-Xe span.HoLogo-e{%
      position:relative;%
      top:.5ex;%
      left-margin:-.1em;%
    }%
  }%
  \global\let\HoLogoCss@Xe\relax
}
%    \end{macrocode}
%    \end{macro}
%
%    \begin{macro}{\HoLogo@XeTeX}
%    \begin{macrocode}
\def\HoLogo@XeTeX#1{%
  \hologo{Xe}%
  \kern-.15em\relax
  \hologo{TeX}%
}
%    \end{macrocode}
%    \end{macro}
%
%    \begin{macro}{\HoLogoHtml@XeTeX}
%    \begin{macrocode}
\def\HoLogoHtml@XeTeX#1{%
  \HoLogoCss@XeTeX
  \HOLOGO@Span{XeTeX}{%
    \hologo{Xe}%
    \hologo{TeX}%
  }%
}
%    \end{macrocode}
%    \end{macro}
%    \begin{macro}{\HoLogoCss@XeTeX}
%    \begin{macrocode}
\def\HoLogoCss@XeTeX{%
  \Css{%
    span.HoLogo-XeTeX span.HoLogo-TeX{%
      margin-left:-.15em;%
    }%
  }%
  \global\let\HoLogoCss@XeTeX\relax
}
%    \end{macrocode}
%    \end{macro}
%
%    \begin{macro}{\HoLogo@XeLaTeX}
%    \begin{macrocode}
\def\HoLogo@XeLaTeX#1{%
  \hologo{Xe}%
  \kern-.13em%
  \hologo{LaTeX}%
}
%    \end{macrocode}
%    \end{macro}
%    \begin{macro}{\HoLogoHtml@XeLaTeX}
%    \begin{macrocode}
\def\HoLogoHtml@XeLaTeX#1{%
  \HoLogoCss@XeLaTeX
  \HOLOGO@Span{XeLaTeX}{%
    \hologo{Xe}%
    \hologo{LaTeX}%
  }%
}
%    \end{macrocode}
%    \end{macro}
%    \begin{macro}{\HoLogoCss@XeLaTeX}
%    \begin{macrocode}
\def\HoLogoCss@XeLaTeX{%
  \Css{%
    span.HoLogo-XeLaTeX span.HoLogo-Xe{%
      margin-right:-.13em;%
    }%
  }%
  \global\let\HoLogoCss@XeLaTeX\relax
}
%    \end{macrocode}
%    \end{macro}
%
% \subsubsection{\hologo{pdfTeX}, \hologo{pdfLaTeX}}
%
%    \begin{macro}{\HoLogo@pdfTeX}
%    \begin{macrocode}
\def\HoLogo@pdfTeX#1{%
  \HOLOGO@mbox{%
    #1{p}{P}df\hologo{TeX}%
  }%
}
%    \end{macrocode}
%    \end{macro}
%    \begin{macro}{\HoLogoCs@pdfTeX}
%    \begin{macrocode}
\def\HoLogoCs@pdfTeX#1{#1{p}{P}dfTeX}
%    \end{macrocode}
%    \end{macro}
%    \begin{macro}{\HoLogoBkm@pdfTeX}
%    \begin{macrocode}
\def\HoLogoBkm@pdfTeX#1{%
  #1{p}{P}df\hologo{TeX}%
}
%    \end{macrocode}
%    \end{macro}
%    \begin{macro}{\HoLogoHtml@pdfTeX}
%    \begin{macrocode}
\let\HoLogoHtml@pdfTeX\HoLogo@pdfTeX
%    \end{macrocode}
%    \end{macro}
%
%    \begin{macro}{\HoLogo@pdfLaTeX}
%    \begin{macrocode}
\def\HoLogo@pdfLaTeX#1{%
  \HOLOGO@mbox{%
    #1{p}{P}df\hologo{LaTeX}%
  }%
}
%    \end{macrocode}
%    \end{macro}
%    \begin{macro}{\HoLogoCs@pdfLaTeX}
%    \begin{macrocode}
\def\HoLogoCs@pdfLaTeX#1{#1{p}{P}dfLaTeX}
%    \end{macrocode}
%    \end{macro}
%    \begin{macro}{\HoLogoBkm@pdfLaTeX}
%    \begin{macrocode}
\def\HoLogoBkm@pdfLaTeX#1{%
  #1{p}{P}df\hologo{LaTeX}%
}
%    \end{macrocode}
%    \end{macro}
%    \begin{macro}{\HoLogoHtml@pdfLaTeX}
%    \begin{macrocode}
\let\HoLogoHtml@pdfLaTeX\HoLogo@pdfLaTeX
%    \end{macrocode}
%    \end{macro}
%
% \subsubsection{\hologo{VTeX}}
%
%    \begin{macro}{\HoLogo@VTeX}
%    \begin{macrocode}
\def\HoLogo@VTeX#1{%
  \HOLOGO@mbox{%
    V\hologo{TeX}%
  }%
}
%    \end{macrocode}
%    \end{macro}
%    \begin{macro}{\HoLogoHtml@VTeX}
%    \begin{macrocode}
\let\HoLogoHtml@VTeX\HoLogo@VTeX
%    \end{macrocode}
%    \end{macro}
%
% \subsubsection{\hologo{AmS}, \dots}
%
%    Source: class \xclass{amsdtx}
%
%    \begin{macro}{\HoLogo@AmS}
%    \begin{macrocode}
\def\HoLogo@AmS#1{%
  \HoLogoFont@font{AmS}{sy}{%
    A%
    \kern-.1667em%
    \lower.5ex\hbox{M}%
    \kern-.125em%
    S%
  }%
}
%    \end{macrocode}
%    \end{macro}
%    \begin{macro}{\HoLogoBkm@AmS}
%    \begin{macrocode}
\def\HoLogoBkm@AmS#1{AmS}
%    \end{macrocode}
%    \end{macro}
%    \begin{macro}{\HoLogoHtml@AmS}
%    \begin{macrocode}
\def\HoLogoHtml@AmS#1{%
  \HoLogoCss@AmS
%  \HoLogoFont@font{AmS}{sy}{%
    \HOLOGO@Span{AmS}{%
      A%
      \HOLOGO@Span{M}{M}%
      S%
    }%
%   }%
}
%    \end{macrocode}
%    \end{macro}
%    \begin{macro}{\HoLogoCss@AmS}
%    \begin{macrocode}
\def\HoLogoCss@AmS{%
  \Css{%
    span.HoLogo-AmS span.HoLogo-M{%
      position:relative;%
      top:.5ex;%
      margin-left:-.1667em;%
      margin-right:-.125em;%
      text-decoration:none;%
    }%
  }%
  \global\let\HoLogoCss@AmS\relax
}
%    \end{macrocode}
%    \end{macro}
%
%    \begin{macro}{\HoLogo@AmSTeX}
%    \begin{macrocode}
\def\HoLogo@AmSTeX#1{%
  \hologo{AmS}%
  \HOLOGO@hyphen
  \hologo{TeX}%
}
%    \end{macrocode}
%    \end{macro}
%    \begin{macro}{\HoLogoBkm@AmSTeX}
%    \begin{macrocode}
\def\HoLogoBkm@AmSTeX#1{AmS-TeX}%
%    \end{macrocode}
%    \end{macro}
%    \begin{macro}{\HoLogoHtml@AmSTeX}
%    \begin{macrocode}
\let\HoLogoHtml@AmSTeX\HoLogo@AmSTeX
%    \end{macrocode}
%    \end{macro}
%
%    \begin{macro}{\HoLogo@AmSLaTeX}
%    \begin{macrocode}
\def\HoLogo@AmSLaTeX#1{%
  \hologo{AmS}%
  \HOLOGO@hyphen
  \hologo{LaTeX}%
}
%    \end{macrocode}
%    \end{macro}
%    \begin{macro}{\HoLogoBkm@AmSLaTeX}
%    \begin{macrocode}
\def\HoLogoBkm@AmSLaTeX#1{AmS-LaTeX}%
%    \end{macrocode}
%    \end{macro}
%    \begin{macro}{\HoLogoHtml@AmSLaTeX}
%    \begin{macrocode}
\let\HoLogoHtml@AmSLaTeX\HoLogo@AmSLaTeX
%    \end{macrocode}
%    \end{macro}
%
% \subsubsection{\hologo{BibTeX}}
%
%    \begin{macro}{\HoLogo@BibTeX@sc}
%    A definition of \hologo{BibTeX} is provided in
%    the documentation source for the manual of \hologo{BibTeX}
%    \cite{btxdoc}.
%\begin{quote}
%\begin{verbatim}
%\def\BibTeX{%
%  {%
%    \rm
%    B%
%    \kern-.05em%
%    {%
%      \sc
%      i%
%      \kern-.025em %
%      b%
%    }%
%    \kern-.08em
%    T%
%    \kern-.1667em%
%    \lower.7ex\hbox{E}%
%    \kern-.125em%
%    X%
%  }%
%}
%\end{verbatim}
%\end{quote}
%    \begin{macrocode}
\def\HoLogo@BibTeX@sc#1{%
  B%
  \kern-.05em%
  \HoLogoFont@font{BibTeX}{sc}{%
    i%
    \kern-.025em%
    b%
  }%
  \HOLOGO@discretionary
  \kern-.08em%
  \hologo{TeX}%
}
%    \end{macrocode}
%    \end{macro}
%    \begin{macro}{\HoLogoHtml@BibTeX@sc}
%    \begin{macrocode}
\def\HoLogoHtml@BibTeX@sc#1{%
  \HoLogoCss@BibTeX@sc
  \HOLOGO@Span{BibTeX-sc}{%
    B%
    \HOLOGO@Span{i}{i}%
    \HOLOGO@Span{b}{b}%
    \hologo{TeX}%
  }%
}
%    \end{macrocode}
%    \end{macro}
%    \begin{macro}{\HoLogoCss@BibTeX@sc}
%    \begin{macrocode}
\def\HoLogoCss@BibTeX@sc{%
  \Css{%
    span.HoLogo-BibTeX-sc span.HoLogo-i{%
      margin-left:-.05em;%
      margin-right:-.025em;%
      font-variant:small-caps;%
    }%
  }%
  \Css{%
    span.HoLogo-BibTeX-sc span.HoLogo-b{%
      margin-right:-.08em;%
      font-variant:small-caps;%
    }%
  }%
  \global\let\HoLogoCss@BibTeX@sc\relax
}
%    \end{macrocode}
%    \end{macro}
%
%    \begin{macro}{\HoLogo@BibTeX@sf}
%    Variant \xoption{sf} avoids trouble with unavailable
%    small caps fonts (e.g., bold versions of Computer Modern or
%    Latin Modern). The definition is taken from
%    package \xpackage{dtklogos} \cite{dtklogos}.
%\begin{quote}
%\begin{verbatim}
%\DeclareRobustCommand{\BibTeX}{%
%  B%
%  \kern-.05em%
%  \hbox{%
%    $\m@th$% %% force math size calculations
%    \csname S@\f@size\endcsname
%    \fontsize\sf@size\z@
%    \math@fontsfalse
%    \selectfont
%    I%
%    \kern-.025em%
%    B
%  }%
%  \kern-.08em%
%  \-%
%  \TeX
%}
%\end{verbatim}
%\end{quote}
%    \begin{macrocode}
\def\HoLogo@BibTeX@sf#1{%
  B%
  \kern-.05em%
  \HoLogoFont@font{BibTeX}{bibsf}{%
    I%
    \kern-.025em%
    B%
  }%
  \HOLOGO@discretionary
  \kern-.08em%
  \hologo{TeX}%
}
%    \end{macrocode}
%    \end{macro}
%    \begin{macro}{\HoLogoHtml@BibTeX@sf}
%    \begin{macrocode}
\def\HoLogoHtml@BibTeX@sf#1{%
  \HoLogoCss@BibTeX@sf
  \HOLOGO@Span{BibTeX-sf}{%
    B%
    \HoLogoFont@font{BibTeX}{bibsf}{%
      \HOLOGO@Span{i}{I}%
      B%
    }%
    \hologo{TeX}%
  }%
}
%    \end{macrocode}
%    \end{macro}
%    \begin{macro}{\HoLogoCss@BibTeX@sf}
%    \begin{macrocode}
\def\HoLogoCss@BibTeX@sf{%
  \Css{%
    span.HoLogo-BibTeX-sf span.HoLogo-i{%
      margin-left:-.05em;%
      margin-right:-.025em;%
    }%
  }%
  \Css{%
    span.HoLogo-BibTeX-sf span.HoLogo-TeX{%
      margin-left:-.08em;%
    }%
  }%
  \global\let\HoLogoCss@BibTeX@sf\relax
}
%    \end{macrocode}
%    \end{macro}
%
%    \begin{macro}{\HoLogo@BibTeX}
%    \begin{macrocode}
\def\HoLogo@BibTeX{\HoLogo@BibTeX@sf}
%    \end{macrocode}
%    \end{macro}
%    \begin{macro}{\HoLogoHtml@BibTeX}
%    \begin{macrocode}
\def\HoLogoHtml@BibTeX{\HoLogoHtml@BibTeX@sf}
%    \end{macrocode}
%    \end{macro}
%
% \subsubsection{\hologo{BibTeX8}}
%
%    \begin{macro}{\HoLogo@BibTeX8}
%    \begin{macrocode}
\expandafter\def\csname HoLogo@BibTeX8\endcsname#1{%
  \hologo{BibTeX}%
  8%
}
%    \end{macrocode}
%    \end{macro}
%
%    \begin{macro}{\HoLogoBkm@BibTeX8}
%    \begin{macrocode}
\expandafter\def\csname HoLogoBkm@BibTeX8\endcsname#1{%
  \hologo{BibTeX}%
  8%
}
%    \end{macrocode}
%    \end{macro}
%    \begin{macro}{\HoLogoHtml@BibTeX8}
%    \begin{macrocode}
\expandafter
\let\csname HoLogoHtml@BibTeX8\expandafter\endcsname
\csname HoLogo@BibTeX8\endcsname
%    \end{macrocode}
%    \end{macro}
%
% \subsubsection{\hologo{ConTeXt}}
%
%    \begin{macro}{\HoLogo@ConTeXt@simple}
%    \begin{macrocode}
\def\HoLogo@ConTeXt@simple#1{%
  \HOLOGO@mbox{Con}%
  \HOLOGO@discretionary
  \HOLOGO@mbox{\hologo{TeX}t}%
}
%    \end{macrocode}
%    \end{macro}
%    \begin{macro}{\HoLogoHtml@ConTeXt@simple}
%    \begin{macrocode}
\let\HoLogoHtml@ConTeXt@simple\HoLogo@ConTeXt@simple
%    \end{macrocode}
%    \end{macro}
%
%    \begin{macro}{\HoLogo@ConTeXt@narrow}
%    This definition of logo \hologo{ConTeXt} with variant \xoption{narrow}
%    comes from TUGboat's class \xclass{ltugboat} (version 2010/11/15 v2.8).
%    \begin{macrocode}
\def\HoLogo@ConTeXt@narrow#1{%
  \HOLOGO@mbox{C\kern-.0333emon}%
  \HOLOGO@discretionary
  \kern-.0667em%
  \HOLOGO@mbox{\hologo{TeX}\kern-.0333emt}%
}
%    \end{macrocode}
%    \end{macro}
%    \begin{macro}{\HoLogoHtml@ConTeXt@narrow}
%    \begin{macrocode}
\def\HoLogoHtml@ConTeXt@narrow#1{%
  \HoLogoCss@ConTeXt@narrow
  \HOLOGO@Span{ConTeXt-narrow}{%
    \HOLOGO@Span{C}{C}%
    on%
    \hologo{TeX}%
    t%
  }%
}
%    \end{macrocode}
%    \end{macro}
%    \begin{macro}{\HoLogoCss@ConTeXt@narrow}
%    \begin{macrocode}
\def\HoLogoCss@ConTeXt@narrow{%
  \Css{%
    span.HoLogo-ConTeXt-narrow span.HoLogo-C{%
      margin-left:-.0333em;%
    }%
  }%
  \Css{%
    span.HoLogo-ConTeXt-narrow span.HoLogo-TeX{%
      margin-left:-.0667em;%
      margin-right:-.0333em;%
    }%
  }%
  \global\let\HoLogoCss@ConTeXt@narrow\relax
}
%    \end{macrocode}
%    \end{macro}
%
%    \begin{macro}{\HoLogo@ConTeXt}
%    \begin{macrocode}
\def\HoLogo@ConTeXt{\HoLogo@ConTeXt@narrow}
%    \end{macrocode}
%    \end{macro}
%    \begin{macro}{\HoLogoHtml@ConTeXt}
%    \begin{macrocode}
\def\HoLogoHtml@ConTeXt{\HoLogoHtml@ConTeXt@narrow}
%    \end{macrocode}
%    \end{macro}
%
% \subsubsection{\hologo{emTeX}}
%
%    \begin{macro}{\HoLogo@emTeX}
%    \begin{macrocode}
\def\HoLogo@emTeX#1{%
  \HOLOGO@mbox{#1{e}{E}m}%
  \HOLOGO@discretionary
  \hologo{TeX}%
}
%    \end{macrocode}
%    \end{macro}
%    \begin{macro}{\HoLogoCs@emTeX}
%    \begin{macrocode}
\def\HoLogoCs@emTeX#1{#1{e}{E}mTeX}%
%    \end{macrocode}
%    \end{macro}
%    \begin{macro}{\HoLogoBkm@emTeX}
%    \begin{macrocode}
\def\HoLogoBkm@emTeX#1{%
  #1{e}{E}m\hologo{TeX}%
}
%    \end{macrocode}
%    \end{macro}
%    \begin{macro}{\HoLogoHtml@emTeX}
%    \begin{macrocode}
\let\HoLogoHtml@emTeX\HoLogo@emTeX
%    \end{macrocode}
%    \end{macro}
%
% \subsubsection{\hologo{ExTeX}}
%
%    \begin{macro}{\HoLogo@ExTeX}
%    The definition is taken from the FAQ of the
%    project \hologo{ExTeX}
%    \cite{ExTeX-FAQ}.
%\begin{quote}
%\begin{verbatim}
%\def\ExTeX{%
%  \textrm{% Logo always with serifs
%    \ensuremath{%
%      \textstyle
%      \varepsilon_{%
%        \kern-0.15em%
%        \mathcal{X}%
%      }%
%    }%
%    \kern-.15em%
%    \TeX
%  }%
%}
%\end{verbatim}
%\end{quote}
%    \begin{macrocode}
\def\HoLogo@ExTeX#1{%
  \HoLogoFont@font{ExTeX}{rm}{%
    \ltx@mbox{%
      \HOLOGO@MathSetup
      $%
        \textstyle
        \varepsilon_{%
          \kern-0.15em%
          \HoLogoFont@font{ExTeX}{sy}{X}%
        }%
      $%
    }%
    \HOLOGO@discretionary
    \kern-.15em%
    \hologo{TeX}%
  }%
}
%    \end{macrocode}
%    \end{macro}
%    \begin{macro}{\HoLogoHtml@ExTeX}
%    \begin{macrocode}
\def\HoLogoHtml@ExTeX#1{%
  \HoLogoCss@ExTeX
  \HoLogoFont@font{ExTeX}{rm}{%
    \HOLOGO@Span{ExTeX}{%
      \ltx@mbox{%
        \HOLOGO@MathSetup
        $\textstyle\varepsilon$%
        \HOLOGO@Span{X}{$\textstyle\chi$}%
        \hologo{TeX}%
      }%
    }%
  }%
}
%    \end{macrocode}
%    \end{macro}
%    \begin{macro}{\HoLogoBkm@ExTeX}
%    \begin{macrocode}
\def\HoLogoBkm@ExTeX#1{%
  \HOLOGO@PdfdocUnicode{#1{e}{E}x}{\textepsilon\textchi}%
  \hologo{TeX}%
}
%    \end{macrocode}
%    \end{macro}
%    \begin{macro}{\HoLogoCss@ExTeX}
%    \begin{macrocode}
\def\HoLogoCss@ExTeX{%
  \Css{%
    span.HoLogo-ExTeX{%
      font-family:serif;%
    }%
  }%
  \Css{%
    span.HoLogo-ExTeX span.HoLogo-TeX{%
      margin-left:-.15em;%
    }%
  }%
  \global\let\HoLogoCss@ExTeX\relax
}
%    \end{macrocode}
%    \end{macro}
%
% \subsubsection{\hologo{MiKTeX}}
%
%    \begin{macro}{\HoLogo@MiKTeX}
%    \begin{macrocode}
\def\HoLogo@MiKTeX#1{%
  \HOLOGO@mbox{MiK}%
  \HOLOGO@discretionary
  \hologo{TeX}%
}
%    \end{macrocode}
%    \end{macro}
%    \begin{macro}{\HoLogoHtml@MiKTeX}
%    \begin{macrocode}
\let\HoLogoHtml@MiKTeX\HoLogo@MiKTeX
%    \end{macrocode}
%    \end{macro}
%
% \subsubsection{\hologo{OzTeX} and friends}
%
%    Source: \hologo{OzTeX} FAQ \cite{OzTeX}:
%    \begin{quote}
%      |\def\OzTeX{O\kern-.03em z\kern-.15em\TeX}|\\
%      (There is no kerning in OzMF, OzMP and OzTtH.)
%    \end{quote}
%
%    \begin{macro}{\HoLogo@OzTeX}
%    \begin{macrocode}
\def\HoLogo@OzTeX#1{%
  O%
  \kern-.03em %
  z%
  \kern-.15em %
  \hologo{TeX}%
}
%    \end{macrocode}
%    \end{macro}
%    \begin{macro}{\HoLogoHtml@OzTeX}
%    \begin{macrocode}
\def\HoLogoHtml@OzTeX#1{%
  \HoLogoCss@OzTeX
  \HOLOGO@Span{OzTeX}{%
    O%
    \HOLOGO@Span{z}{z}%
    \hologo{TeX}%
  }%
}
%    \end{macrocode}
%    \end{macro}
%    \begin{macro}{\HoLogoCss@OzTeX}
%    \begin{macrocode}
\def\HoLogoCss@OzTeX{%
  \Css{%
    span.HoLogo-OzTeX span.HoLogo-z{%
      margin-left:-.03em;%
      margin-right:-.15em;%
    }%
  }%
  \global\let\HoLogoCss@OzTeX\relax
}
%    \end{macrocode}
%    \end{macro}
%
%    \begin{macro}{\HoLogo@OzMF}
%    \begin{macrocode}
\def\HoLogo@OzMF#1{%
  \HOLOGO@mbox{OzMF}%
}
%    \end{macrocode}
%    \end{macro}
%    \begin{macro}{\HoLogo@OzMP}
%    \begin{macrocode}
\def\HoLogo@OzMP#1{%
  \HOLOGO@mbox{OzMP}%
}
%    \end{macrocode}
%    \end{macro}
%    \begin{macro}{\HoLogo@OzTtH}
%    \begin{macrocode}
\def\HoLogo@OzTtH#1{%
  \HOLOGO@mbox{OzTtH}%
}
%    \end{macrocode}
%    \end{macro}
%
% \subsubsection{\hologo{PCTeX}}
%
%    \begin{macro}{\HoLogo@PCTeX}
%    \begin{macrocode}
\def\HoLogo@PCTeX#1{%
  \HOLOGO@mbox{PC}%
  \hologo{TeX}%
}
%    \end{macrocode}
%    \end{macro}
%    \begin{macro}{\HoLogoHtml@PCTeX}
%    \begin{macrocode}
\let\HoLogoHtml@PCTeX\HoLogo@PCTeX
%    \end{macrocode}
%    \end{macro}
%
% \subsubsection{\hologo{PiCTeX}}
%
%    The original definitions from \xfile{pictex.tex} \cite{PiCTeX}:
%\begin{quote}
%\begin{verbatim}
%\def\PiC{%
%  P%
%  \kern-.12em%
%  \lower.5ex\hbox{I}%
%  \kern-.075em%
%  C%
%}
%\def\PiCTeX{%
%  \PiC
%  \kern-.11em%
%  \TeX
%}
%\end{verbatim}
%\end{quote}
%
%    \begin{macro}{\HoLogo@PiC}
%    \begin{macrocode}
\def\HoLogo@PiC#1{%
  P%
  \kern-.12em%
  \lower.5ex\hbox{I}%
  \kern-.075em%
  C%
  \HOLOGO@SpaceFactor
}
%    \end{macrocode}
%    \end{macro}
%    \begin{macro}{\HoLogoHtml@PiC}
%    \begin{macrocode}
\def\HoLogoHtml@PiC#1{%
  \HoLogoCss@PiC
  \HOLOGO@Span{PiC}{%
    P%
    \HOLOGO@Span{i}{I}%
    C%
  }%
}
%    \end{macrocode}
%    \end{macro}
%    \begin{macro}{\HoLogoCss@PiC}
%    \begin{macrocode}
\def\HoLogoCss@PiC{%
  \Css{%
    span.HoLogo-PiC span.HoLogo-i{%
      position:relative;%
      top:.5ex;%
      margin-left:-.12em;%
      margin-right:-.075em;%
      text-decoration:none;%
    }%
  }%
  \global\let\HoLogoCss@PiC\relax
}
%    \end{macrocode}
%    \end{macro}
%
%    \begin{macro}{\HoLogo@PiCTeX}
%    \begin{macrocode}
\def\HoLogo@PiCTeX#1{%
  \hologo{PiC}%
  \HOLOGO@discretionary
  \kern-.11em%
  \hologo{TeX}%
}
%    \end{macrocode}
%    \end{macro}
%    \begin{macro}{\HoLogoHtml@PiCTeX}
%    \begin{macrocode}
\def\HoLogoHtml@PiCTeX#1{%
  \HoLogoCss@PiCTeX
  \HOLOGO@Span{PiCTeX}{%
    \hologo{PiC}%
    \hologo{TeX}%
  }%
}
%    \end{macrocode}
%    \end{macro}
%    \begin{macro}{\HoLogoCss@PiCTeX}
%    \begin{macrocode}
\def\HoLogoCss@PiCTeX{%
  \Css{%
    span.HoLogo-PiCTeX span.HoLogo-PiC{%
      margin-right:-.11em;%
    }%
  }%
  \global\let\HoLogoCss@PiCTeX\relax
}
%    \end{macrocode}
%    \end{macro}
%
% \subsubsection{\hologo{teTeX}}
%
%    \begin{macro}{\HoLogo@teTeX}
%    \begin{macrocode}
\def\HoLogo@teTeX#1{%
  \HOLOGO@mbox{#1{t}{T}e}%
  \HOLOGO@discretionary
  \hologo{TeX}%
}
%    \end{macrocode}
%    \end{macro}
%    \begin{macro}{\HoLogoCs@teTeX}
%    \begin{macrocode}
\def\HoLogoCs@teTeX#1{#1{t}{T}dfTeX}
%    \end{macrocode}
%    \end{macro}
%    \begin{macro}{\HoLogoBkm@teTeX}
%    \begin{macrocode}
\def\HoLogoBkm@teTeX#1{%
  #1{t}{T}e\hologo{TeX}%
}
%    \end{macrocode}
%    \end{macro}
%    \begin{macro}{\HoLogoHtml@teTeX}
%    \begin{macrocode}
\let\HoLogoHtml@teTeX\HoLogo@teTeX
%    \end{macrocode}
%    \end{macro}
%
% \subsubsection{\hologo{TeX4ht}}
%
%    \begin{macro}{\HoLogo@TeX4ht}
%    \begin{macrocode}
\expandafter\def\csname HoLogo@TeX4ht\endcsname#1{%
  \HOLOGO@mbox{\hologo{TeX}4ht}%
}
%    \end{macrocode}
%    \end{macro}
%    \begin{macro}{\HoLogoHtml@TeX4ht}
%    \begin{macrocode}
\expandafter
\let\csname HoLogoHtml@TeX4ht\expandafter\endcsname
\csname HoLogo@TeX4ht\endcsname
%    \end{macrocode}
%    \end{macro}
%
%
% \subsubsection{\hologo{SageTeX}}
%
%    \begin{macro}{\HoLogo@SageTeX}
%    \begin{macrocode}
\def\HoLogo@SageTeX#1{%
  \HOLOGO@mbox{Sage}%
  \HOLOGO@discretionary
  \HOLOGO@NegativeKerning{eT,oT,To}%
  \hologo{TeX}%
}
%    \end{macrocode}
%    \end{macro}
%    \begin{macro}{\HoLogoHtml@SageTeX}
%    \begin{macrocode}
\let\HoLogoHtml@SageTeX\HoLogo@SageTeX
%    \end{macrocode}
%    \end{macro}
%
% \subsection{\hologo{METAFONT} and friends}
%
%    \begin{macro}{\HoLogo@METAFONT}
%    \begin{macrocode}
\def\HoLogo@METAFONT#1{%
  \HoLogoFont@font{METAFONT}{logo}{%
    \HOLOGO@mbox{META}%
    \HOLOGO@discretionary
    \HOLOGO@mbox{FONT}%
  }%
}
%    \end{macrocode}
%    \end{macro}
%
%    \begin{macro}{\HoLogo@METAPOST}
%    \begin{macrocode}
\def\HoLogo@METAPOST#1{%
  \HoLogoFont@font{METAPOST}{logo}{%
    \HOLOGO@mbox{META}%
    \HOLOGO@discretionary
    \HOLOGO@mbox{POST}%
  }%
}
%    \end{macrocode}
%    \end{macro}
%
%    \begin{macro}{\HoLogo@MetaFun}
%    \begin{macrocode}
\def\HoLogo@MetaFun#1{%
  \HOLOGO@mbox{Meta}%
  \HOLOGO@discretionary
  \HOLOGO@mbox{Fun}%
}
%    \end{macrocode}
%    \end{macro}
%
%    \begin{macro}{\HoLogo@MetaPost}
%    \begin{macrocode}
\def\HoLogo@MetaPost#1{%
  \HOLOGO@mbox{Meta}%
  \HOLOGO@discretionary
  \HOLOGO@mbox{Post}%
}
%    \end{macrocode}
%    \end{macro}
%
% \subsection{Others}
%
% \subsubsection{\hologo{biber}}
%
%    \begin{macro}{\HoLogo@biber}
%    \begin{macrocode}
\def\HoLogo@biber#1{%
  \HOLOGO@mbox{#1{b}{B}i}%
  \HOLOGO@discretionary
  \HOLOGO@mbox{ber}%
}
%    \end{macrocode}
%    \end{macro}
%    \begin{macro}{\HoLogoCs@biber}
%    \begin{macrocode}
\def\HoLogoCs@biber#1{#1{b}{B}iber}
%    \end{macrocode}
%    \end{macro}
%    \begin{macro}{\HoLogoBkm@biber}
%    \begin{macrocode}
\def\HoLogoBkm@biber#1{%
  #1{b}{B}iber%
}
%    \end{macrocode}
%    \end{macro}
%    \begin{macro}{\HoLogoHtml@biber}
%    \begin{macrocode}
\let\HoLogoHtml@biber\HoLogo@biber
%    \end{macrocode}
%    \end{macro}
%
% \subsubsection{\hologo{KOMAScript}}
%
%    \begin{macro}{\HoLogo@KOMAScript}
%    The definition for \hologo{KOMAScript} is taken
%    from \hologo{KOMAScript} (\xfile{scrlogo.dtx}, reformatted) \cite{scrlogo}:
%\begin{quote}
%\begin{verbatim}
%\@ifundefined{KOMAScript}{%
%  \DeclareRobustCommand{\KOMAScript}{%
%    \textsf{%
%      K\kern.05em O\kern.05emM\kern.05em A%
%      \kern.1em-\kern.1em %
%      Script%
%    }%
%  }%
%}{}
%\end{verbatim}
%\end{quote}
%    \begin{macrocode}
\def\HoLogo@KOMAScript#1{%
  \HoLogoFont@font{KOMAScript}{sf}{%
    \HOLOGO@mbox{%
      K\kern.05em%
      O\kern.05em%
      M\kern.05em%
      A%
    }%
    \kern.1em%
    \HOLOGO@hyphen
    \kern.1em%
    \HOLOGO@mbox{Script}%
  }%
}
%    \end{macrocode}
%    \end{macro}
%    \begin{macro}{\HoLogoBkm@KOMAScript}
%    \begin{macrocode}
\def\HoLogoBkm@KOMAScript#1{%
  KOMA-Script%
}
%    \end{macrocode}
%    \end{macro}
%    \begin{macro}{\HoLogoHtml@KOMAScript}
%    \begin{macrocode}
\def\HoLogoHtml@KOMAScript#1{%
  \HoLogoCss@KOMAScript
  \HoLogoFont@font{KOMAScript}{sf}{%
    \HOLOGO@Span{KOMAScript}{%
      K%
      \HOLOGO@Span{O}{O}%
      M%
      \HOLOGO@Span{A}{A}%
      \HOLOGO@Span{hyphen}{-}%
      Script%
    }%
  }%
}
%    \end{macrocode}
%    \end{macro}
%    \begin{macro}{\HoLogoCss@KOMAScript}
%    \begin{macrocode}
\def\HoLogoCss@KOMAScript{%
  \Css{%
    span.HoLogo-KOMAScript{%
      font-family:sans-serif;%
    }%
  }%
  \Css{%
    span.HoLogo-KOMAScript span.HoLogo-O{%
      padding-left:.05em;%
      padding-right:.05em;%
    }%
  }%
  \Css{%
    span.HoLogo-KOMAScript span.HoLogo-A{%
      padding-left:.05em;%
    }%
  }%
  \Css{%
    span.HoLogo-KOMAScript span.HoLogo-hyphen{%
      padding-left:.1em;%
      padding-right:.1em;%
    }%
  }%
  \global\let\HoLogoCss@KOMAScript\relax
}
%    \end{macrocode}
%    \end{macro}
%
% \subsubsection{\hologo{LyX}}
%
%    \begin{macro}{\HoLogo@LyX}
%    The definition is taken from the documentation source files
%    of \hologo{LyX}, \xfile{Intro.lyx} \cite{LyX}:
%\begin{quote}
%\begin{verbatim}
%\def\LyX{%
%  \texorpdfstring{%
%    L\kern-.1667em\lower.25em\hbox{Y}\kern-.125emX\@%
%  }{%
%    LyX%
%  }%
%}
%\end{verbatim}
%\end{quote}
%    \begin{macrocode}
\def\HoLogo@LyX#1{%
  L%
  \kern-.1667em%
  \lower.25em\hbox{Y}%
  \kern-.125em%
  X%
  \HOLOGO@SpaceFactor
}
%    \end{macrocode}
%    \end{macro}
%    \begin{macro}{\HoLogoHtml@LyX}
%    \begin{macrocode}
\def\HoLogoHtml@LyX#1{%
  \HoLogoCss@LyX
  \HOLOGO@Span{LyX}{%
    L%
    \HOLOGO@Span{y}{Y}%
    X%
  }%
}
%    \end{macrocode}
%    \end{macro}
%    \begin{macro}{\HoLogoCss@LyX}
%    \begin{macrocode}
\def\HoLogoCss@LyX{%
  \Css{%
    span.HoLogo-LyX span.HoLogo-y{%
      position:relative;%
      top:.25em;%
      margin-left:-.1667em;%
      margin-right:-.125em;%
      text-decoration:none;%
    }%
  }%
  \global\let\HoLogoCss@LyX\relax
}
%    \end{macrocode}
%    \end{macro}
%
% \subsubsection{\hologo{NTS}}
%
%    \begin{macro}{\HoLogo@NTS}
%    Definition for \hologo{NTS} can be found in
%    package \xpackage{etex\textunderscore man} for the \hologo{eTeX} manual \cite{etexman}
%    and in package \xpackage{dtklogos} \cite{dtklogos}:
%\begin{quote}
%\begin{verbatim}
%\def\NTS{%
%  \leavevmode
%  \hbox{%
%    $%
%      \cal N%
%      \kern-0.35em%
%      \lower0.5ex\hbox{$\cal T$}%
%      \kern-0.2em%
%      S%
%    $%
%  }%
%}
%\end{verbatim}
%\end{quote}
%    \begin{macrocode}
\def\HoLogo@NTS#1{%
  \HoLogoFont@font{NTS}{sy}{%
    N\/%
    \kern-.35em%
    \lower.5ex\hbox{T\/}%
    \kern-.2em%
    S\/%
  }%
  \HOLOGO@SpaceFactor
}
%    \end{macrocode}
%    \end{macro}
%
% \subsubsection{\Hologo{TTH} (\hologo{TeX} to HTML translator)}
%
%    Source: \url{http://hutchinson.belmont.ma.us/tth/}
%    In the HTML source the second `T' is printed as subscript.
%\begin{quote}
%\begin{verbatim}
%T<sub>T</sub>H
%\end{verbatim}
%\end{quote}
%    \begin{macro}{\HoLogo@TTH}
%    \begin{macrocode}
\def\HoLogo@TTH#1{%
  \ltx@mbox{%
    T\HOLOGO@SubScript{T}H%
  }%
  \HOLOGO@SpaceFactor
}
%    \end{macrocode}
%    \end{macro}
%
%    \begin{macro}{\HoLogoHtml@TTH}
%    \begin{macrocode}
\def\HoLogoHtml@TTH#1{%
  T\HCode{<sub>}T\HCode{</sub>}H%
}
%    \end{macrocode}
%    \end{macro}
%
% \subsubsection{\Hologo{HanTheThanh}}
%
%    Partial source: Package \xpackage{dtklogos}.
%    The double accent is U+1EBF (latin small letter e with circumflex
%    and acute).
%    \begin{macro}{\HoLogo@HanTheThanh}
%    \begin{macrocode}
\def\HoLogo@HanTheThanh#1{%
  \ltx@mbox{H\`an}%
  \HOLOGO@space
  \ltx@mbox{%
    Th%
    \HOLOGO@IfCharExists{"1EBF}{%
      \char"1EBF\relax
    }{%
      \^e\hbox to 0pt{\hss\raise .5ex\hbox{\'{}}}%
    }%
  }%
  \HOLOGO@space
  \ltx@mbox{Th\`anh}%
}
%    \end{macrocode}
%    \end{macro}
%    \begin{macro}{\HoLogoBkm@HanTheThanh}
%    \begin{macrocode}
\def\HoLogoBkm@HanTheThanh#1{%
  H\`an %
  Th\HOLOGO@PdfdocUnicode{\^e}{\9036\277} %
  Th\`anh%
}
%    \end{macrocode}
%    \end{macro}
%    \begin{macro}{\HoLogoHtml@HanTheThanh}
%    \begin{macrocode}
\def\HoLogoHtml@HanTheThanh#1{%
  H\`an %
  Th\HCode{&\ltx@hashchar x1ebf;} %
  Th\`anh%
}
%    \end{macrocode}
%    \end{macro}
%
% \subsection{Driver detection}
%
%    \begin{macrocode}
\HOLOGO@IfExists\InputIfFileExists{%
  \InputIfFileExists{hologo.cfg}{}{}%
}{%
  \ltx@IfUndefined{pdf@filesize}{%
    \def\HOLOGO@InputIfExists{%
      \openin\HOLOGO@temp=hologo.cfg\relax
      \ifeof\HOLOGO@temp
        \closein\HOLOGO@temp
      \else
        \closein\HOLOGO@temp
        \begingroup
          \def\x{LaTeX2e}%
        \expandafter\endgroup
        \ifx\fmtname\x
          \input{hologo.cfg}%
        \else
          \input hologo.cfg\relax
        \fi
      \fi
    }%
    \ltx@IfUndefined{newread}{%
      \chardef\HOLOGO@temp=15 %
      \def\HOLOGO@CheckRead{%
        \ifeof\HOLOGO@temp
          \HOLOGO@InputIfExists
        \else
          \ifcase\HOLOGO@temp
            \@PackageWarningNoLine{hologo}{%
              Configuration file ignored, because\MessageBreak
              a free read register could not be found%
            }%
          \else
            \begingroup
              \count\ltx@cclv=\HOLOGO@temp
              \advance\ltx@cclv by \ltx@minusone
              \edef\x{\endgroup
                \chardef\noexpand\HOLOGO@temp=\the\count\ltx@cclv
                \relax
              }%
            \x
          \fi
        \fi
      }%
    }{%
      \csname newread\endcsname\HOLOGO@temp
      \HOLOGO@InputIfExists
    }%
  }{%
    \edef\HOLOGO@temp{\pdf@filesize{hologo.cfg}}%
    \ifx\HOLOGO@temp\ltx@empty
    \else
      \ifnum\HOLOGO@temp>0 %
        \begingroup
          \def\x{LaTeX2e}%
        \expandafter\endgroup
        \ifx\fmtname\x
          \input{hologo.cfg}%
        \else
          \input hologo.cfg\relax
        \fi
      \else
        \@PackageInfoNoLine{hologo}{%
          Empty configuration file `hologo.cfg' ignored%
        }%
      \fi
    \fi
  }%
}
%    \end{macrocode}
%
%    \begin{macrocode}
\def\HOLOGO@temp#1#2{%
  \kv@define@key{HoLogoDriver}{#1}[]{%
    \begingroup
      \def\HOLOGO@temp{##1}%
      \ltx@onelevel@sanitize\HOLOGO@temp
      \ifx\HOLOGO@temp\ltx@empty
      \else
        \@PackageError{hologo}{%
          Value (\HOLOGO@temp) not permitted for option `#1'%
        }%
        \@ehc
      \fi
    \endgroup
    \def\hologoDriver{#2}%
  }%
}%
\def\HOLOGO@@temp#1#2{%
  \ifx\kv@value\relax
    \HOLOGO@temp{#1}{#1}%
  \else
    \HOLOGO@temp{#1}{#2}%
  \fi
}%
\kv@parse@normalized{%
  pdftex,%
  luatex=pdftex,%
  dvipdfm,%
  dvipdfmx=dvipdfm,%
  dvips,%
  dvipsone=dvips,%
  xdvi=dvips,%
  xetex,%
  vtex,%
}\HOLOGO@@temp
%    \end{macrocode}
%
%    \begin{macrocode}
\kv@define@key{HoLogoDriver}{driverfallback}{%
  \def\HOLOGO@DriverFallback{#1}%
}
%    \end{macrocode}
%
%    \begin{macro}{\HOLOGO@DriverFallback}
%    \begin{macrocode}
\def\HOLOGO@DriverFallback{dvips}
%    \end{macrocode}
%    \end{macro}
%
%    \begin{macro}{\hologoDriverSetup}
%    \begin{macrocode}
\def\hologoDriverSetup{%
  \let\hologoDriver\ltx@undefined
  \HOLOGO@DriverSetup
}
%    \end{macrocode}
%    \end{macro}
%
%    \begin{macro}{\HOLOGO@DriverSetup}
%    \begin{macrocode}
\def\HOLOGO@DriverSetup#1{%
  \kvsetkeys{HoLogoDriver}{#1}%
  \HOLOGO@CheckDriver
  \ltx@ifundefined{hologoDriver}{%
    \begingroup
    \edef\x{\endgroup
      \noexpand\kvsetkeys{HoLogoDriver}{\HOLOGO@DriverFallback}%
    }\x
  }{}%
  \@PackageInfoNoLine{hologo}{Using driver `\hologoDriver'}%
}
%    \end{macrocode}
%    \end{macro}
%
%    \begin{macro}{\HOLOGO@CheckDriver}
%    \begin{macrocode}
\def\HOLOGO@CheckDriver{%
  \ifpdf
    \def\hologoDriver{pdftex}%
    \let\HOLOGO@pdfliteral\pdfliteral
    \ifluatex
      \ifx\pdfextension\@undefined\else
        \protected\def\pdfliteral{\pdfextension literal}%
        \let\HOLOGO@pdfliteral\pdfliteral
      \fi
      \ltx@IfUndefined{HOLOGO@pdfliteral}{%
        \ifnum\luatexversion<36 %
        \else
          \begingroup
            \let\HOLOGO@temp\endgroup
            \ifcase0%
                \directlua{%
                  if tex.enableprimitives then %
                    tex.enableprimitives('HOLOGO@', {'pdfliteral'})%
                  else %
                    tex.print('1')%
                  end%
                }%
                \ifx\HOLOGO@pdfliteral\@undefined 1\fi%
                \relax%
              \endgroup
              \let\HOLOGO@temp\relax
              \global\let\HOLOGO@pdfliteral\HOLOGO@pdfliteral
            \fi%
          \HOLOGO@temp
        \fi
      }{}%
    \fi
    \ltx@IfUndefined{HOLOGO@pdfliteral}{%
      \@PackageWarningNoLine{hologo}{%
        Cannot find \string\pdfliteral
      }%
    }{}%
  \else
    \ifxetex
      \def\hologoDriver{xetex}%
    \else
      \ifvtex
        \def\hologoDriver{vtex}%
      \fi
    \fi
  \fi
}
%    \end{macrocode}
%    \end{macro}
%
%    \begin{macro}{\HOLOGO@WarningUnsupportedDriver}
%    \begin{macrocode}
\def\HOLOGO@WarningUnsupportedDriver#1{%
  \@PackageWarningNoLine{hologo}{%
    Logo `#1' needs driver specific macros,\MessageBreak
    but driver `\hologoDriver' is not supported.\MessageBreak
    Use a different driver or\MessageBreak
    load package `graphics' or `pgf'%
  }%
}
%    \end{macrocode}
%    \end{macro}
%
% \subsubsection{Reflect box macros}
%
%    Skip driver part if not needed.
%    \begin{macrocode}
\ltx@IfUndefined{reflectbox}{}{%
  \ltx@IfUndefined{rotatebox}{}{%
    \HOLOGO@AtEnd
  }%
}
\ltx@IfUndefined{pgftext}{}{%
  \HOLOGO@AtEnd
}
\ltx@IfUndefined{psscalebox}{}{%
  \HOLOGO@AtEnd
}
%    \end{macrocode}
%
%    \begin{macrocode}
\def\HOLOGO@temp{LaTeX2e}
\ifx\fmtname\HOLOGO@temp
  \RequirePackage{kvoptions}[2011/06/30]%
  \ProcessKeyvalOptions{HoLogoDriver}%
\fi
\HOLOGO@DriverSetup{}
%    \end{macrocode}
%
%    \begin{macro}{\HOLOGO@ReflectBox}
%    \begin{macrocode}
\def\HOLOGO@ReflectBox#1{%
  \begingroup
    \setbox\ltx@zero\hbox{\begingroup#1\endgroup}%
    \setbox\ltx@two\hbox{%
      \kern\wd\ltx@zero
      \csname HOLOGO@ScaleBox@\hologoDriver\endcsname{-1}{1}{%
        \hbox to 0pt{\copy\ltx@zero\hss}%
      }%
    }%
    \wd\ltx@two=\wd\ltx@zero
    \box\ltx@two
  \endgroup
}
%    \end{macrocode}
%    \end{macro}
%
%    \begin{macro}{\HOLOGO@PointReflectBox}
%    \begin{macrocode}
\def\HOLOGO@PointReflectBox#1{%
  \begingroup
    \setbox\ltx@zero\hbox{\begingroup#1\endgroup}%
    \setbox\ltx@two\hbox{%
      \kern\wd\ltx@zero
      \raise\ht\ltx@zero\hbox{%
        \csname HOLOGO@ScaleBox@\hologoDriver\endcsname{-1}{-1}{%
          \hbox to 0pt{\copy\ltx@zero\hss}%
        }%
      }%
    }%
    \wd\ltx@two=\wd\ltx@zero
    \box\ltx@two
  \endgroup
}
%    \end{macrocode}
%    \end{macro}
%
%    We must define all variants because of dynamic driver setup.
%    \begin{macrocode}
\def\HOLOGO@temp#1#2{#2}
%    \end{macrocode}
%
%    \begin{macro}{\HOLOGO@ScaleBox@pdftex}
%    \begin{macrocode}
\HOLOGO@temp{pdftex}{%
  \def\HOLOGO@ScaleBox@pdftex#1#2#3{%
    \HOLOGO@pdfliteral{%
      q #1 0 0 #2 0 0 cm%
    }%
    #3%
    \HOLOGO@pdfliteral{%
      Q%
    }%
  }%
}
%    \end{macrocode}
%    \end{macro}
%    \begin{macro}{\HOLOGO@ScaleBox@dvips}
%    \begin{macrocode}
\HOLOGO@temp{dvips}{%
  \def\HOLOGO@ScaleBox@dvips#1#2#3{%
    \special{ps:%
      gsave %
      currentpoint %
      currentpoint translate %
      #1 #2 scale %
      neg exch neg exch translate%
    }%
    #3%
    \special{ps:%
      currentpoint %
      grestore %
      moveto%
    }%
  }%
}
%    \end{macrocode}
%    \end{macro}
%    \begin{macro}{\HOLOGO@ScaleBox@dvipdfm}
%    \begin{macrocode}
\HOLOGO@temp{dvipdfm}{%
  \let\HOLOGO@ScaleBox@dvipdfm\HOLOGO@ScaleBox@dvips
}
%    \end{macrocode}
%    \end{macro}
%    Since \hologo{XeTeX} v0.6.
%    \begin{macro}{\HOLOGO@ScaleBox@xetex}
%    \begin{macrocode}
\HOLOGO@temp{xetex}{%
  \def\HOLOGO@ScaleBox@xetex#1#2#3{%
    \special{x:gsave}%
    \special{x:scale #1 #2}%
    #3%
    \special{x:grestore}%
  }%
}
%    \end{macrocode}
%    \end{macro}
%    \begin{macro}{\HOLOGO@ScaleBox@vtex}
%    \begin{macrocode}
\HOLOGO@temp{vtex}{%
  \def\HOLOGO@ScaleBox@vtex#1#2#3{%
    \special{r(#1,0,0,#2,0,0}%
    #3%
    \special{r)}%
  }%
}
%    \end{macrocode}
%    \end{macro}
%
%    \begin{macrocode}
\HOLOGO@AtEnd%
%</package>
%    \end{macrocode}
%
% \section{Test}
%
% \subsection{Catcode checks for loading}
%
%    \begin{macrocode}
%<*test1>
%    \end{macrocode}
%    \begin{macrocode}
\catcode`\{=1 %
\catcode`\}=2 %
\catcode`\#=6 %
\catcode`\@=11 %
\expandafter\ifx\csname count@\endcsname\relax
  \countdef\count@=255 %
\fi
\expandafter\ifx\csname @gobble\endcsname\relax
  \long\def\@gobble#1{}%
\fi
\expandafter\ifx\csname @firstofone\endcsname\relax
  \long\def\@firstofone#1{#1}%
\fi
\expandafter\ifx\csname loop\endcsname\relax
  \expandafter\@firstofone
\else
  \expandafter\@gobble
\fi
{%
  \def\loop#1\repeat{%
    \def\body{#1}%
    \iterate
  }%
  \def\iterate{%
    \body
      \let\next\iterate
    \else
      \let\next\relax
    \fi
    \next
  }%
  \let\repeat=\fi
}%
\def\RestoreCatcodes{}
\count@=0 %
\loop
  \edef\RestoreCatcodes{%
    \RestoreCatcodes
    \catcode\the\count@=\the\catcode\count@\relax
  }%
\ifnum\count@<255 %
  \advance\count@ 1 %
\repeat

\def\RangeCatcodeInvalid#1#2{%
  \count@=#1\relax
  \loop
    \catcode\count@=15 %
  \ifnum\count@<#2\relax
    \advance\count@ 1 %
  \repeat
}
\def\RangeCatcodeCheck#1#2#3{%
  \count@=#1\relax
  \loop
    \ifnum#3=\catcode\count@
    \else
      \errmessage{%
        Character \the\count@\space
        with wrong catcode \the\catcode\count@\space
        instead of \number#3%
      }%
    \fi
  \ifnum\count@<#2\relax
    \advance\count@ 1 %
  \repeat
}
\def\space{ }
\expandafter\ifx\csname LoadCommand\endcsname\relax
  \def\LoadCommand{\input hologo.sty\relax}%
\fi
\def\Test{%
  \RangeCatcodeInvalid{0}{47}%
  \RangeCatcodeInvalid{58}{64}%
  \RangeCatcodeInvalid{91}{96}%
  \RangeCatcodeInvalid{123}{255}%
  \catcode`\@=12 %
  \catcode`\\=0 %
  \catcode`\%=14 %
  \LoadCommand
  \RangeCatcodeCheck{0}{36}{15}%
  \RangeCatcodeCheck{37}{37}{14}%
  \RangeCatcodeCheck{38}{47}{15}%
  \RangeCatcodeCheck{48}{57}{12}%
  \RangeCatcodeCheck{58}{63}{15}%
  \RangeCatcodeCheck{64}{64}{12}%
  \RangeCatcodeCheck{65}{90}{11}%
  \RangeCatcodeCheck{91}{91}{15}%
  \RangeCatcodeCheck{92}{92}{0}%
  \RangeCatcodeCheck{93}{96}{15}%
  \RangeCatcodeCheck{97}{122}{11}%
  \RangeCatcodeCheck{123}{255}{15}%
  \RestoreCatcodes
}
\Test
\csname @@end\endcsname
\end
%    \end{macrocode}
%    \begin{macrocode}
%</test1>
%    \end{macrocode}
%
% \subsection{Spacefactor}
%
%    The space factor must be 1000 after a logo. If it is greater 1000
%    then the following space is a space after a sentence closing point.
%    If the space factor is smaller 1000 then an immediate following
%    dot is interpreted as abbreviation, not sentence closing point.
%
%    \begin{macrocode}
%<*test-spacefactor>
\NeedsTeXFormat{LaTeX2e}
\documentclass{article}
\usepackage{hologo}[2016/05/12]
\usepackage{kvsetkeys}
\usepackage{qstest}
\IncludeTests{*}
\LogTests{log}{*}{*}
\begin{document}
\begin{qstest}{spacefactor}{spacefactor}
\newcommand*{\Test}[1]{%
  \sbox0{%
    \hologo{#1}%
    \Expect*{1000 (#1)}*{\the\spacefactor\space(#1)}%
  }%
}%
\makeatletter
\def\TestList{}
\def\hologoEntry#1#2#3{%
  \edef\TestList{%
    \ifx\TestList\@empty
    \else
      \TestList,%
    \fi
    #1%
    \ifx\\#2\\%
    \else
      ={variant=#2}%
    \fi
  }%
}
\hologoList
\expandafter\kv@parse@normalized\expandafter{%
  \TestList
}{%
  \begingroup
    \let\@logo=\kv@key
    \ifx\kv@value\relax
    \else
      \expandafter\hologoLogoSetup\expandafter\@logo\expandafter{%
        \kv@value
      }%
    \fi
    \Test\@logo
  \endgroup
  \@gobbletwo
}
\end{qstest}
\end{document}
%</test-spacefactor>
%    \end{macrocode}
%
% \subsection{Complete list}
%
%    \begin{macrocode}
%<*test-list>
\NeedsTeXFormat{LaTeX2e}
\documentclass[12pt,a4paper]{article}
\usepackage{hologo}[2016/05/12]
\usepackage[T1]{fontenc}
\usepackage{lmodern}
\usepackage{parskip}
\usepackage[unicode]{hyperref}[2011/09/28]
\usepackage{bookmark}[2011/09/19]
\bookmarksetup{%
  numbered,%
  open,%
  openlevel=2,%
}
\renewcommand*{\contentsname}{List of logos}
\begin{document}
\tableofcontents
\def\TestFont#1#2#3#4#5#6{%
  \begingroup
    \usefont{#3}{#4}{#5}{#6}%
    \HologoVariant{#1}{#2}/\hologoVariant{#1}{#2}%
    \quad
    \begingroup\scriptsize\hologoVariant{#1}{#2}\endgroup
    \quad
  \endgroup
  (#3/#4/#5/#6)%
  \par
}
\makeatletter
\def\hologoEntry#1#2#3{%
  \section{%
    \HologoVariant{#1}{#2}/\hologoVariant{#1}{#2} %
    {[#1\ifx\\#2\\\else\space(#2)\fi]}% hash-ok
  }% braces around [] because of bug in tex4ht
  \begingroup
    \hypersetup{unicode=false}%
    \bookmark[%
      dest=\@currentHref,%
      rellevel=1,%
      keeplevel,%
    ]{%
      \HologoVariant{#1}{#2}/\hologoVariant{#1}{#2} %
      (PDFDocEncoding)%
    }%
  \endgroup
  \TestFont{#1}{#2}{OT1}{cmr}{m}{n}%
  \TestFont{#1}{#2}{OT1}{cmss}{m}{n}%
  \TestFont{#1}{#2}{OT1}{cmr}{b}{n}%
  \TestFont{#1}{#2}{OT1}{cmr}{m}{it}%
  \TestFont{#1}{#2}{OT1}{cmtt}{m}{n}%
  \TestFont{#1}{#2}{T1}{lmr}{m}{n}%
  \TestFont{#1}{#2}{T1}{lmss}{m}{n}%
  \TestFont{#1}{#2}{T1}{lmr}{b}{n}%
  \TestFont{#1}{#2}{T1}{lmr}{m}{it}%
  \TestFont{#1}{#2}{T1}{lmtt}{m}{n}%
  \TestFont{#1}{#2}{T1}{lmvtt}{m}{n}%
  \TestFont{#1}{#2}{T1}{qtm}{m}{n}%
  \TestFont{#1}{#2}{T1}{qhv}{m}{n}%
  \TestFont{#1}{#2}{T1}{qtm}{b}{n}%
  \TestFont{#1}{#2}{T1}{qtm}{m}{it}%
  \TestFont{#1}{#2}{T1}{qcr}{m}{n}%
  \newpage
}
\makeatother
\hologoList
\end{document}
%</test-list>
%    \end{macrocode}
%
% \section{Installation}
%
% \subsection{Download}
%
% \paragraph{Package.} This package is available on
% CTAN\footnote{\url{ftp://ftp.ctan.org/tex-archive/}}:
% \begin{description}
% \item[\CTAN{macros/latex/contrib/oberdiek/hologo.dtx}] The source file.
% \item[\CTAN{macros/latex/contrib/oberdiek/hologo.pdf}] Documentation.
% \end{description}
%
%
% \paragraph{Bundle.} All the packages of the bundle `oberdiek'
% are also available in a TDS compliant ZIP archive. There
% the packages are already unpacked and the documentation files
% are generated. The files and directories obey the TDS standard.
% \begin{description}
% \item[\CTAN{install/macros/latex/contrib/oberdiek.tds.zip}]
% \end{description}
% \emph{TDS} refers to the standard ``A Directory Structure
% for \TeX\ Files'' (\CTAN{tds/tds.pdf}). Directories
% with \xfile{texmf} in their name are usually organized this way.
%
% \subsection{Bundle installation}
%
% \paragraph{Unpacking.} Unpack the \xfile{oberdiek.tds.zip} in the
% TDS tree (also known as \xfile{texmf} tree) of your choice.
% Example (linux):
% \begin{quote}
%   |unzip oberdiek.tds.zip -d ~/texmf|
% \end{quote}
%
% \paragraph{Script installation.}
% Check the directory \xfile{TDS:scripts/oberdiek/} for
% scripts that need further installation steps.
% Package \xpackage{attachfile2} comes with the Perl script
% \xfile{pdfatfi.pl} that should be installed in such a way
% that it can be called as \texttt{pdfatfi}.
% Example (linux):
% \begin{quote}
%   |chmod +x scripts/oberdiek/pdfatfi.pl|\\
%   |cp scripts/oberdiek/pdfatfi.pl /usr/local/bin/|
% \end{quote}
%
% \subsection{Package installation}
%
% \paragraph{Unpacking.} The \xfile{.dtx} file is a self-extracting
% \docstrip\ archive. The files are extracted by running the
% \xfile{.dtx} through \plainTeX:
% \begin{quote}
%   \verb|tex hologo.dtx|
% \end{quote}
%
% \paragraph{TDS.} Now the different files must be moved into
% the different directories in your installation TDS tree
% (also known as \xfile{texmf} tree):
% \begin{quote}
% \def\t{^^A
% \begin{tabular}{@{}>{\ttfamily}l@{ $\rightarrow$ }>{\ttfamily}l@{}}
%   hologo.sty & tex/generic/oberdiek/hologo.sty\\
%   hologo.pdf & doc/latex/oberdiek/hologo.pdf\\
%   example/hologo-example.tex & doc/latex/oberdiek/example/hologo-example.tex\\
%   test/hologo-test1.tex & doc/latex/oberdiek/test/hologo-test1.tex\\
%   test/hologo-test-spacefactor.tex & doc/latex/oberdiek/test/hologo-test-spacefactor.tex\\
%   test/hologo-test-list.tex & doc/latex/oberdiek/test/hologo-test-list.tex\\
%   hologo.dtx & source/latex/oberdiek/hologo.dtx\\
% \end{tabular}^^A
% }^^A
% \sbox0{\t}^^A
% \ifdim\wd0>\linewidth
%   \begingroup
%     \advance\linewidth by\leftmargin
%     \advance\linewidth by\rightmargin
%   \edef\x{\endgroup
%     \def\noexpand\lw{\the\linewidth}^^A
%   }\x
%   \def\lwbox{^^A
%     \leavevmode
%     \hbox to \linewidth{^^A
%       \kern-\leftmargin\relax
%       \hss
%       \usebox0
%       \hss
%       \kern-\rightmargin\relax
%     }^^A
%   }^^A
%   \ifdim\wd0>\lw
%     \sbox0{\small\t}^^A
%     \ifdim\wd0>\linewidth
%       \ifdim\wd0>\lw
%         \sbox0{\footnotesize\t}^^A
%         \ifdim\wd0>\linewidth
%           \ifdim\wd0>\lw
%             \sbox0{\scriptsize\t}^^A
%             \ifdim\wd0>\linewidth
%               \ifdim\wd0>\lw
%                 \sbox0{\tiny\t}^^A
%                 \ifdim\wd0>\linewidth
%                   \lwbox
%                 \else
%                   \usebox0
%                 \fi
%               \else
%                 \lwbox
%               \fi
%             \else
%               \usebox0
%             \fi
%           \else
%             \lwbox
%           \fi
%         \else
%           \usebox0
%         \fi
%       \else
%         \lwbox
%       \fi
%     \else
%       \usebox0
%     \fi
%   \else
%     \lwbox
%   \fi
% \else
%   \usebox0
% \fi
% \end{quote}
% If you have a \xfile{docstrip.cfg} that configures and enables \docstrip's
% TDS installing feature, then some files can already be in the right
% place, see the documentation of \docstrip.
%
% \subsection{Refresh file name databases}
%
% If your \TeX~distribution
% (\teTeX, \mikTeX, \dots) relies on file name databases, you must refresh
% these. For example, \teTeX\ users run \verb|texhash| or
% \verb|mktexlsr|.
%
% \subsection{Some details for the interested}
%
% \paragraph{Attached source.}
%
% The PDF documentation on CTAN also includes the
% \xfile{.dtx} source file. It can be extracted by
% AcrobatReader 6 or higher. Another option is \textsf{pdftk},
% e.g. unpack the file into the current directory:
% \begin{quote}
%   \verb|pdftk hologo.pdf unpack_files output .|
% \end{quote}
%
% \paragraph{Unpacking with \LaTeX.}
% The \xfile{.dtx} chooses its action depending on the format:
% \begin{description}
% \item[\plainTeX:] Run \docstrip\ and extract the files.
% \item[\LaTeX:] Generate the documentation.
% \end{description}
% If you insist on using \LaTeX\ for \docstrip\ (really,
% \docstrip\ does not need \LaTeX), then inform the autodetect routine
% about your intention:
% \begin{quote}
%   \verb|latex \let\install=y\input{hologo.dtx}|
% \end{quote}
% Do not forget to quote the argument according to the demands
% of your shell.
%
% \paragraph{Generating the documentation.}
% You can use both the \xfile{.dtx} or the \xfile{.drv} to generate
% the documentation. The process can be configured by the
% configuration file \xfile{ltxdoc.cfg}. For instance, put this
% line into this file, if you want to have A4 as paper format:
% \begin{quote}
%   \verb|\PassOptionsToClass{a4paper}{article}|
% \end{quote}
% An example follows how to generate the
% documentation with pdf\LaTeX:
% \begin{quote}
%\begin{verbatim}
%pdflatex hologo.dtx
%makeindex -s gind.ist hologo.idx
%pdflatex hologo.dtx
%makeindex -s gind.ist hologo.idx
%pdflatex hologo.dtx
%\end{verbatim}
% \end{quote}
%
% \section{Catalogue}
%
% The following XML file can be used as source for the
% \href{http://mirror.ctan.org/help/Catalogue/catalogue.html}{\TeX\ Catalogue}.
% The elements \texttt{caption} and \texttt{description} are imported
% from the original XML file from the Catalogue.
% The name of the XML file in the Catalogue is \xfile{hologo.xml}.
%    \begin{macrocode}
%<*catalogue>
<?xml version='1.0' encoding='us-ascii'?>
<!DOCTYPE entry SYSTEM 'catalogue.dtd'>
<entry datestamp='$Date$' modifier='$Author$' id='hologo'>
  <name>hologo</name>
  <caption>A collection of logos with bookmark support.</caption>
  <authorref id='auth:oberdiek'/>
  <copyright owner='Heiko Oberdiek' year='2010-2012'/>
  <license type='lppl1.3'/>
  <version number='1.10'/>
  <description>
    The package defines a single command <tt>\hologo</tt>, whose
    argument is the usual case-confused ASCII version of the logo.
    The command is bookmark-enabled, so that every logo becomes
    available in bookmarks without further work.
    <p/>
    The package is part of the <xref refid='oberdiek'>oberdiek</xref>
    bundle.
  </description>
  <documentation details='Package documentation'
      href='ctan:/macros/latex/contrib/oberdiek/hologo.pdf'/>
  <ctan file='true' path='/macros/latex/contrib/oberdiek/hologo.dtx'/>
  <miktex location='oberdiek'/>
  <texlive location='oberdiek'/>
  <install path='/macros/latex/contrib/oberdiek/oberdiek.tds.zip'/>
</entry>
%</catalogue>
%    \end{macrocode}
%
% \begin{thebibliography}{9}
% \raggedright
%
% \bibitem{btxdoc}
% Oren Patashnik,
% \textit{\hologo{BibTeX}ing},
% 1988-02-08.\\
% \CTAN{biblio/bibtex/base/}
%
% \bibitem{dtklogos}
% Gerd Neugebauer, DANTE,
% \textit{Package \xpackage{dtklogos}},
% 2011-04-25.\\
% \CTAN{usergrps/dante/dtk/dtklogos.sty}
%
% \bibitem{etexman}
% The \hologo{NTS} Team,
% \textit{The \hologo{eTeX} manual},
% 1998-02.\\
% \CTAN{systems/e-tex/v2/doc/}
%
% \bibitem{ExTeX-FAQ}
% The \hologo{ExTeX} group,
% \textit{\hologo{ExTeX}: FAQ -- How is \hologo{ExTeX} typeset?},
% 2007-04-14.\\
% \url{http://www.extex.org/documentation/faq.html}
%
% \bibitem{LyX}
% %@MISC{ LyX,
% %  title = {{LyX 2.0.0 -- The Document Processor [Computer software and manual]}},
% %  author = {{The LyX Team}},
% %  howpublished = {Internet: http://www.lyx.org},
% %  year = {2011-05-08},
% %  note = {Retrieved May 10, 2011, from http://www.lyx.org},
% %  url = {http://www.lyx.org/}
% %}
% The \hologo{LyX} Team,
% \textit{\hologo{LyX} -- The Document Processor},
% 2011-05-08.\\
% \url{http://www.lyx.org/}
%
% \bibitem{OzTeX}
% Andrew Trevorrow,
% \hologo{OzTeX} FAQ: What is the correct way to typeset ``\hologo{OzTeX}''?,
% 2011-09-15 (visited).
% \url{http://www.trevorrow.com/oztex/ozfaq.html#oztex-logo}
%
% \bibitem{PiCTeX}
% Michael Wichura,
% \textit{The \hologo{PiCTeX} macro package},
% 1987-09-21.
% \CTAN{graphics/pictex/}
%
% \bibitem{scrlogo}
% Markus Kohm,
% \textit{\hologo{KOMAScript} Datei \xfile{scrlogo.dtx}},
% 2009-01-30.\\
% \CTAN{install/macros/latex/contrib/komascript.tds.zip}
%
% \end{thebibliography}
%
% \begin{History}
%   \begin{Version}{2010/04/08 v1.0}
%   \item
%     The first version.
%   \end{Version}
%   \begin{Version}{2010/04/16 v1.1}
%   \item
%     \cs{Hologo} added for support of logos at start of a sentence.
%   \item
%     \cs{hologoSetup} and \cs{hologoLogoSetup} added.
%   \item
%     Options \xoption{break}, \xoption{hyphenbreak}, \xoption{spacebreak}
%     added.
%   \item
%     Variant support added by option \xoption{variant}.
%   \end{Version}
%   \begin{Version}{2010/04/24 v1.2}
%   \item
%     \hologo{LaTeX3} added.
%   \item
%     \hologo{VTeX} added.
%   \end{Version}
%   \begin{Version}{2010/11/21 v1.3}
%   \item
%     \hologo{iniTeX}, \hologo{virTeX} added.
%   \end{Version}
%   \begin{Version}{2011/03/25 v1.4}
%   \item
%     \hologo{ConTeXt} with variants added.
%   \item
%     Option \xoption{discretionarybreak} added as refinement for
%     option \xoption{break}.
%   \end{Version}
%   \begin{Version}{2011/04/21 v1.5}
%   \item
%     Wrong TDS directory for test files fixed.
%   \end{Version}
%   \begin{Version}{2011/10/01 v1.6}
%   \item
%     Support for package \xpackage{tex4ht} added.
%   \item
%     Support for \cs{csname} added if \cs{ifincsname} is available.
%   \item
%     New logos:
%     \hologo{(La)TeX},
%     \hologo{biber},
%     \hologo{BibTeX} (\xoption{sc}, \xoption{sf}),
%     \hologo{emTeX},
%     \hologo{ExTeX},
%     \hologo{KOMAScript},
%     \hologo{La},
%     \hologo{LyX},
%     \hologo{MiKTeX},
%     \hologo{NTS},
%     \hologo{OzMF},
%     \hologo{OzMP},
%     \hologo{OzTeX},
%     \hologo{OzTtH},
%     \hologo{PCTeX},
%     \hologo{PiC},
%     \hologo{PiCTeX},
%     \hologo{METAFONT},
%     \hologo{MetaFun},
%     \hologo{METAPOST},
%     \hologo{MetaPost},
%     \hologo{SLiTeX} (\xoption{lift}, \xoption{narrow}, \xoption{simple}),
%     \hologo{SliTeX} (\xoption{narrow}, \xoption{simple}, \xoption{lift}),
%     \hologo{teTeX}.
%   \item
%     Fixes:
%     \hologo{iniTeX},
%     \hologo{pdfLaTeX},
%     \hologo{pdfTeX},
%     \hologo{virTeX}.
%   \item
%     \cs{hologoFontSetup} and \cs{hologoLogoFontSetup} added.
%   \item
%     \cs{hologoVariant} and \cs{HologoVariant} added.
%   \end{Version}
%   \begin{Version}{2011/11/22 v1.7}
%   \item
%     New logos:
%     \hologo{BibTeX8},
%     \hologo{LaTeXML},
%     \hologo{SageTeX},
%     \hologo{TeX4ht},
%     \hologo{TTH}.
%   \item
%     \hologo{Xe} and friends: Driver stuff fixed.
%   \item
%     \hologo{Xe} and friends: Support for italic added.
%   \item
%     \hologo{Xe} and friends: Package support for \xpackage{pgf}
%     and \xpackage{pstricks} added.
%   \end{Version}
%   \begin{Version}{2011/11/29 v1.8}
%   \item
%     New logos:
%     \hologo{HanTheThanh}.
%   \end{Version}
%   \begin{Version}{2011/12/21 v1.9}
%   \item
%     Patch for package \xpackage{ifxetex} added for the case that
%     \cs{newif} is undefined in \hologo{iniTeX}.
%   \item
%     Some fixes for \hologo{iniTeX}.
%   \end{Version}
%   \begin{Version}{2012/04/26 v1.10}
%   \item
%     Fix in bookmark version of logo ``\hologo{HanTheThanh}''.
%   \end{Version}
%   \begin{Version}{2016/05/12 v1.11}
%   \item
%     Update HOLOGO@IfCharExists (previously in texlive)
%   \item define pdfliteral in current luatex.
%   \end{Version}
% \end{History}
%
% \PrintIndex
%
% \Finale
\endinput
%
        \else
          \input hologo.cfg\relax
        \fi
      \fi
    }%
    \ltx@IfUndefined{newread}{%
      \chardef\HOLOGO@temp=15 %
      \def\HOLOGO@CheckRead{%
        \ifeof\HOLOGO@temp
          \HOLOGO@InputIfExists
        \else
          \ifcase\HOLOGO@temp
            \@PackageWarningNoLine{hologo}{%
              Configuration file ignored, because\MessageBreak
              a free read register could not be found%
            }%
          \else
            \begingroup
              \count\ltx@cclv=\HOLOGO@temp
              \advance\ltx@cclv by \ltx@minusone
              \edef\x{\endgroup
                \chardef\noexpand\HOLOGO@temp=\the\count\ltx@cclv
                \relax
              }%
            \x
          \fi
        \fi
      }%
    }{%
      \csname newread\endcsname\HOLOGO@temp
      \HOLOGO@InputIfExists
    }%
  }{%
    \edef\HOLOGO@temp{\pdf@filesize{hologo.cfg}}%
    \ifx\HOLOGO@temp\ltx@empty
    \else
      \ifnum\HOLOGO@temp>0 %
        \begingroup
          \def\x{LaTeX2e}%
        \expandafter\endgroup
        \ifx\fmtname\x
          % \iffalse meta-comment
%
% File: hologo.dtx
% Version: 2016/05/12 v1.11
% Info: A logo collection with bookmark support
%
% Copyright (C) 2010-2012 by
%    Heiko Oberdiek <heiko.oberdiek at googlemail.com>
%
% This work may be distributed and/or modified under the
% conditions of the LaTeX Project Public License, either
% version 1.3c of this license or (at your option) any later
% version. This version of this license is in
%    http://www.latex-project.org/lppl/lppl-1-3c.txt
% and the latest version of this license is in
%    http://www.latex-project.org/lppl.txt
% and version 1.3 or later is part of all distributions of
% LaTeX version 2005/12/01 or later.
%
% This work has the LPPL maintenance status "maintained".
%
% This Current Maintainer of this work is Heiko Oberdiek.
%
% The Base Interpreter refers to any `TeX-Format',
% because some files are installed in TDS:tex/generic//.
%
% This work consists of the main source file hologo.dtx
% and the derived files
%    hologo.sty, hologo.pdf, hologo.ins, hologo.drv, hologo-example.tex,
%    hologo-test1.tex, hologo-test-spacefactor.tex,
%    hologo-test-list.tex.
%
% Distribution:
%    CTAN:macros/latex/contrib/oberdiek/hologo.dtx
%    CTAN:macros/latex/contrib/oberdiek/hologo.pdf
%
% Unpacking:
%    (a) If hologo.ins is present:
%           tex hologo.ins
%    (b) Without hologo.ins:
%           tex hologo.dtx
%    (c) If you insist on using LaTeX
%           latex \let\install=y\input{hologo.dtx}
%        (quote the arguments according to the demands of your shell)
%
% Documentation:
%    (a) If hologo.drv is present:
%           latex hologo.drv
%    (b) Without hologo.drv:
%           latex hologo.dtx; ...
%    The class ltxdoc loads the configuration file ltxdoc.cfg
%    if available. Here you can specify further options, e.g.
%    use A4 as paper format:
%       \PassOptionsToClass{a4paper}{article}
%
%    Programm calls to get the documentation (example):
%       pdflatex hologo.dtx
%       makeindex -s gind.ist hologo.idx
%       pdflatex hologo.dtx
%       makeindex -s gind.ist hologo.idx
%       pdflatex hologo.dtx
%
% Installation:
%    TDS:tex/generic/oberdiek/hologo.sty
%    TDS:doc/latex/oberdiek/hologo.pdf
%    TDS:doc/latex/oberdiek/example/hologo-example.tex
%    TDS:doc/latex/oberdiek/test/hologo-test1.tex
%    TDS:doc/latex/oberdiek/test/hologo-test-spacefactor.tex
%    TDS:doc/latex/oberdiek/test/hologo-test-list.tex
%    TDS:source/latex/oberdiek/hologo.dtx
%
%<*ignore>
\begingroup
  \catcode123=1 %
  \catcode125=2 %
  \def\x{LaTeX2e}%
\expandafter\endgroup
\ifcase 0\ifx\install y1\fi\expandafter
         \ifx\csname processbatchFile\endcsname\relax\else1\fi
         \ifx\fmtname\x\else 1\fi\relax
\else\csname fi\endcsname
%</ignore>
%<*install>
\input docstrip.tex
\Msg{************************************************************************}
\Msg{* Installation}
\Msg{* Package: hologo 2016/05/12 v1.11 A logo collection with bookmark support (HO)}
\Msg{************************************************************************}

\keepsilent
\askforoverwritefalse

\let\MetaPrefix\relax
\preamble

This is a generated file.

Project: hologo
Version: 2016/05/12 v1.11

Copyright (C) 2010-2012 by
   Heiko Oberdiek <heiko.oberdiek at googlemail.com>

This work may be distributed and/or modified under the
conditions of the LaTeX Project Public License, either
version 1.3c of this license or (at your option) any later
version. This version of this license is in
   http://www.latex-project.org/lppl/lppl-1-3c.txt
and the latest version of this license is in
   http://www.latex-project.org/lppl.txt
and version 1.3 or later is part of all distributions of
LaTeX version 2005/12/01 or later.

This work has the LPPL maintenance status "maintained".

This Current Maintainer of this work is Heiko Oberdiek.

The Base Interpreter refers to any `TeX-Format',
because some files are installed in TDS:tex/generic//.

This work consists of the main source file hologo.dtx
and the derived files
   hologo.sty, hologo.pdf, hologo.ins, hologo.drv, hologo-example.tex,
   hologo-test1.tex, hologo-test-spacefactor.tex,
   hologo-test-list.tex.

\endpreamble
\let\MetaPrefix\DoubleperCent

\generate{%
  \file{hologo.ins}{\from{hologo.dtx}{install}}%
  \file{hologo.drv}{\from{hologo.dtx}{driver}}%
  \usedir{tex/generic/oberdiek}%
  \file{hologo.sty}{\from{hologo.dtx}{package}}%
  \usedir{doc/latex/oberdiek/example}%
  \file{hologo-example.tex}{\from{hologo.dtx}{example}}%
  \usedir{doc/latex/oberdiek/test}%
  \file{hologo-test1.tex}{\from{hologo.dtx}{test1}}%
  \file{hologo-test-spacefactor.tex}{\from{hologo.dtx}{test-spacefactor}}%
  \file{hologo-test-list.tex}{\from{hologo.dtx}{test-list}}%
  \nopreamble
  \nopostamble
  \usedir{source/latex/oberdiek/catalogue}%
  \file{hologo.xml}{\from{hologo.dtx}{catalogue}}%
}

\catcode32=13\relax% active space
\let =\space%
\Msg{************************************************************************}
\Msg{*}
\Msg{* To finish the installation you have to move the following}
\Msg{* file into a directory searched by TeX:}
\Msg{*}
\Msg{*     hologo.sty}
\Msg{*}
\Msg{* To produce the documentation run the file `hologo.drv'}
\Msg{* through LaTeX.}
\Msg{*}
\Msg{* Happy TeXing!}
\Msg{*}
\Msg{************************************************************************}

\endbatchfile
%</install>
%<*ignore>
\fi
%</ignore>
%<*driver>
\NeedsTeXFormat{LaTeX2e}
\ProvidesFile{hologo.drv}%
  [2016/05/12 v1.11 A logo collection with bookmark support (HO)]%
\documentclass{ltxdoc}
\usepackage{holtxdoc}[2011/11/22]
\usepackage{hologo}[2016/05/12]
\usepackage{longtable}
\usepackage{array}
\usepackage{paralist}
%\usepackage[T1]{fontenc}
%\usepackage{lmodern}
\begin{document}
  \DocInput{hologo.dtx}%
\end{document}
%</driver>
% \fi
%
%
% \CharacterTable
%  {Upper-case    \A\B\C\D\E\F\G\H\I\J\K\L\M\N\O\P\Q\R\S\T\U\V\W\X\Y\Z
%   Lower-case    \a\b\c\d\e\f\g\h\i\j\k\l\m\n\o\p\q\r\s\t\u\v\w\x\y\z
%   Digits        \0\1\2\3\4\5\6\7\8\9
%   Exclamation   \!     Double quote  \"     Hash (number) \#
%   Dollar        \$     Percent       \%     Ampersand     \&
%   Acute accent  \'     Left paren    \(     Right paren   \)
%   Asterisk      \*     Plus          \+     Comma         \,
%   Minus         \-     Point         \.     Solidus       \/
%   Colon         \:     Semicolon     \;     Less than     \<
%   Equals        \=     Greater than  \>     Question mark \?
%   Commercial at \@     Left bracket  \[     Backslash     \\
%   Right bracket \]     Circumflex    \^     Underscore    \_
%   Grave accent  \`     Left brace    \{     Vertical bar  \|
%   Right brace   \}     Tilde         \~}
%
% \GetFileInfo{hologo.drv}
%
% \title{The \xpackage{hologo} package}
% \date{2016/05/12 v1.11}
% \author{Heiko Oberdiek\\\xemail{heiko.oberdiek at googlemail.com}}
%
% \maketitle
%
% \begin{abstract}
% This package starts a collection of logos with support for bookmarks
% strings.
% \end{abstract}
%
% \tableofcontents
%
% \section{Documentation}
%
% \subsection{Logo macros}
%
% \begin{declcs}{hologo} \M{name}
% \end{declcs}
% Macro \cs{hologo} sets the logo with name \meta{name}.
% The following table shows the supported names.
%
% \begingroup
%   \def\hologoEntry#1#2#3{^^A
%     #1&#2&\hologoLogoSetup{#1}{variant=#2}\hologo{#1}&#3\tabularnewline
%   }
%   \begin{longtable}{>{\ttfamily}l>{\ttfamily}lll}
%     \rmfamily\bfseries{name} & \rmfamily\bfseries variant
%     & \bfseries logo & \bfseries since\\
%     \hline
%     \endhead
%     \hologoList
%   \end{longtable}
% \endgroup
%
% \begin{declcs}{Hologo} \M{name}
% \end{declcs}
% Macro \cs{Hologo} starts the logo \meta{name} with an uppercase
% letter. As an exception small greek letters are not converted
% to uppercase. Examples, see \hologo{eTeX} and \hologo{ExTeX}.
%
% \subsection{Setup macros}
%
% The package does not support package options, but the following
% setup macros can be used to set options.
%
% \begin{declcs}{hologoSetup} \M{key value list}
% \end{declcs}
% Macro \cs{hologoSetup} sets global options.
%
% \begin{declcs}{hologoLogoSetup} \M{logo} \M{key value list}
% \end{declcs}
% Some options can also be used to configure a logo.
% These settings take precedence over global option settings.
%
% \subsection{Options}\label{sec:options}
%
% There are boolean and string options:
% \begin{description}
% \item[Boolean option:]
% It takes |true| or |false|
% as value. If the value is omitted, then |true| is used.
% \item[String option:]
% A value must be given as string. (But the string might be empty.)
% \end{description}
% The following options can be used both in \cs{hologoSetup}
% and \cs{hologoLogoSetup}:
% \begin{description}
% \def\entry#1{\item[\xoption{#1}:]}
% \entry{break}
%   enables or disables line breaks inside the logo. This setting is
%   refined by options \xoption{hyphenbreak}, \xoption{spacebreak}
%   or \xoption{discretionarybreak}.
%   Default is |false|.
% \entry{hyphenbreak}
%   enables or disables the line break right after the hyphen character.
% \entry{spacebreak}
%   enables or disables line breaks at space characters.
% \entry{discretionarybreak}
%   enables or disables line breaks at hyphenation points
%   (inserted by \cs{-}).
% \end{description}
% Macro \cs{hologoLogoSetup} also knows:
% \begin{description}
% \item[\xoption{variant}:]
%   This is a string option. It specifies a variant of a logo that
%   must exist. An empty string selects the package default variant.
% \end{description}
% Example:
% \begin{quote}
%   |\hologoSetup{break=false}|\\
%   |\hologoLogoSetup{plainTeX}{variant=hyphen,hyphenbreak}|\\
%   Then ``plain-\TeX'' contains one break point after the hyphen.
% \end{quote}
%
% \subsection{Driver options}
%
% Sometimes graphical operations are needed to construct some
% glyphs (e.g.\ \hologo{XeTeX}). If package \xpackage{graphics}
% or package \xpackage{pgf} are found, then the macros are taken
% from there. Otherwise the packge defines its own operations
% and therefore needs the driver information. Many drivers are
% detected automatically (\hologo{pdfTeX}/\hologo{LuaTeX}
% in PDF mode, \hologo{XeTeX}, \hologo{VTeX}). These have precedence
% over a driver option. The driver can be given as package option
% or using \cs{hologoDriverSetup}.
% The following list contains the recognized driver options:
% \begin{itemize}
% \item \xoption{pdftex}, \xoption{luatex}
% \item \xoption{dvipdfm}, \xoption{dvipdfmx}
% \item \xoption{dvips}, \xoption{dvipsone}, \xoption{xdvi}
% \item \xoption{xetex}
% \item \xoption{vtex}
% \end{itemize}
% The left driver of a line is the driver name that is used internally.
% The following names are aliases for drivers that use the
% same method. Therefore the entry in the \xext{log} file for
% the used driver prints the internally used driver name.
% \begin{description}
% \item[\xoption{driverfallback}:]
%   This option expects a driver that is used,
%   if the driver could not be detected automatically.
% \end{description}
%
% \begin{declcs}{hologoDriverSetup} \M{driver option}
% \end{declcs}
% The driver can also be configured after package loading
% using \cs{hologoDriverSetup}, also the way for \hologo{plainTeX}
% to setup the driver.
%
% \subsection{Font setup}
%
% Some logos require a special font, but should also be usable by
% \hologo{plainTeX}. Therefore the package provides some ways
% to influence the font settings. The options below
% take font settings as values. Both font commands
% such as \cs{sffamily} and macros that take one argument
% like \cs{textsf} can be used.
%
% \begin{declcs}{hologoFontSetup} \M{key value list}
% \end{declcs}
% Macro \cs{hologoFontSetup} sets the fonts for all logos.
% Supported keys:
% \begin{description}
% \def\entry#1{\item[\xoption{#1}:]}
% \entry{general}
%   This font is used for all logos. The default is empty.
%   That means no special font is used.
% \entry{bibsf}
%   This font is used for
%   {\hologoLogoSetup{BibTeX}{variant=sf}\hologo{BibTeX}}
%   with variant \xoption{sf}.
% \entry{rm}
%   This font is a serif font. It is used for \hologo{ExTeX}.
% \entry{sc}
%   This font specifies a small caps font. It is used for
%   {\hologoLogoSetup{BibTeX}{variant=sc}\hologo{BibTeX}}
%   with variant \xoption{sc}.
% \entry{sf}
%   This font specifies a sans serif font. The default
%   is \cs{sffamily}, then \cs{sf} is tried. Otherwise
%   a warning is given. It is used by \hologo{KOMAScript}.
% \entry{sy}
%   This is the font for math symbols (e.g. cmsy).
%   It is used by \hologo{AmS}, \hologo{NTS}, \hologo{ExTeX}.
% \entry{logo}
%   \hologo{METAFONT} and \hologo{METAPOST} are using that font.
%   In \hologo{LaTeX} \cs{logofamily} is used and
%   the definitions of package \xpackage{mflogo} are used
%   if the package is not loaded.
%   Otherwise the \cs{tenlogo} is used and defined
%   if it does not already exists.
% \end{description}
%
% \begin{declcs}{hologoLogoFontSetup} \M{logo} \M{key value list}
% \end{declcs}
% Fonts can also be set for a logo or logo component separately,
% see the following list.
% The keys are the same as for \cs{hologoFontSetup}.
%
% \begin{longtable}{>{\ttfamily}l>{\sffamily}ll}
%   \meta{logo} & keys & result\\
%   \hline
%   \endhead
%   BibTeX & bibsf & {\hologoLogoSetup{BibTeX}{variant=sf}\hologo{BibTeX}}\\[.5ex]
%   BibTeX & sc & {\hologoLogoSetup{BibTeX}{variant=sc}\hologo{BibTeX}}\\[.5ex]
%   ExTeX & rm & \hologo{ExTeX}\\
%   SliTeX & rm & \hologo{SliTeX}\\[.5ex]
%   AmS & sy & \hologo{AmS}\\
%   ExTeX & sy & \hologo{ExTeX}\\
%   NTS & sy & \hologo{NTS}\\[.5ex]
%   KOMAScript & sf & \hologo{KOMAScript}\\[.5ex]
%   METAFONT & logo & \hologo{METAFONT}\\
%   METAPOST & logo & \hologo{METAPOST}\\[.5ex]
%   SliTeX & sc \hologo{SliTeX}
% \end{longtable}
%
% \subsubsection{Font order}
%
% For all logos the font \xoption{general} is applied first.
% Example:
%\begin{quote}
%|\hologoFontSetup{general=\color{red}}|
%\end{quote}
% will print red logos.
% Then if the font uses a special font \xoption{sf}, for example,
% the font is applied that is setup by \cs{hologoLogoFontSetup}.
% If this font is not setup, then the common font setup
% by \cs{hologoFontSetup} is used. Otherwise a warning is given,
% that there is no font configured.
%
% \subsection{Additional user macros}
%
% Usually a variant of a logo is configured by using
% \cs{hologoLogoSetup}, because it is bad style to mix
% different variants of the same logo in the same text.
% There the following macros are a convenience for testing.
%
% \begin{declcs}{hologoVariant} \M{name} \M{variant}\\
%   \cs{HologoVariant} \M{name} \M{variant}
% \end{declcs}
% Logo \meta{name} is set using \meta{variant} that specifies
% explicitely which variant of the macro is used. If the argument
% is empty, then the default form of the logo is used
% (configurable by \cs{hologoLogoSetup}).
%
% \cs{HologoVariant} is used if the logo is set in a context
% that needs an uppercase first letter (beginning of a sentence, \dots).
%
% \begin{declcs}{hologoList}\\
%   \cs{hologoEntry} \M{logo} \M{variant} \M{since}
% \end{declcs}
% Macro \cs{hologoList} contains all logos that are provided
% by the package including variants. The list consists of calls
% of \cs{hologoEntry} with three arguments starting with the
% logo name \meta{logo} and its variant \meta{variant}. An empty
% variant means the current default. Argument \meta{since} specifies
% with version of the package \xpackage{hologo} is needed to get
% the logo. If the logo is fixed, then the date gets updated.
% Therefore the date \meta{since} is not exactly the date of
% the first introduction, but rather the date of the latest fix.
%
% Before \cs{hologoList} can be used, macro \cs{hologoEntry} needs
% a definition. The example file in section \ref{sec:example}
% shows applications of \cs{hologoList}.
%
% \subsection{Supported contexts}
%
% Macros \cs{hologo} and friends support special contexts:
% \begin{itemize}
% \item \hologo{LaTeX}'s protection mechanism.
% \item Bookmarks of package \xpackage{hyperref}.
% \item Package \xpackage{tex4ht}.
% \item The macros can be used inside \cs{csname} constructs,
%   if \cs{ifincsname} is available (\hologo{pdfTeX}, \hologo{XeTeX},
%   \hologo{LuaTeX}).
% \end{itemize}
%
% \subsection{Example}
% \label{sec:example}
%
% The following example prints the logos in different fonts.
%    \begin{macrocode}
%<*example>
%<<verbatim
\NeedsTeXFormat{LaTeX2e}
\documentclass[a4paper]{article}
\usepackage[
  hmargin=20mm,
  vmargin=20mm,
]{geometry}
\pagestyle{empty}
\usepackage{hologo}[2016/05/12]
\usepackage{longtable}
\usepackage{array}
\setlength{\extrarowheight}{2pt}
\usepackage[T1]{fontenc}
\usepackage{lmodern}
\usepackage{pdflscape}
\usepackage[
  pdfencoding=auto,
]{hyperref}
\hypersetup{
  pdfauthor={Heiko Oberdiek},
  pdftitle={Example for package `hologo'},
  pdfsubject={Logos with fonts lmr, lmss, qtm, qpl, qhv},
}
\usepackage{bookmark}

% Print the logo list on the console

\begingroup
  \typeout{}%
  \typeout{*** Begin of logo list ***}%
  \newcommand*{\hologoEntry}[3]{%
    \typeout{#1 \ifx\\#2\\\else(#2) \fi[#3]}%
  }%
  \hologoList
  \typeout{*** End of logo list ***}%
  \typeout{}%
\endgroup

\begin{document}
\begin{landscape}

  \section{Example file for package `hologo'}

  % Table for font names

  \begin{longtable}{>{\bfseries}ll}
    \textbf{font} & \textbf{Font name}\\
    \hline
    lmr & Latin Modern Roman\\
    lmss & Latin Modern Sans\\
    qtm & \TeX\ Gyre Termes\\
    qhv & \TeX\ Gyre Heros\\
    qpl & \TeX\ Gyre Pagella\\
  \end{longtable}

  % Logo list with logos in different fonts

  \begingroup
    \newcommand*{\SetVariant}[2]{%
      \ifx\\#2\\%
      \else
        \hologoLogoSetup{#1}{variant=#2}%
      \fi
    }%
    \newcommand*{\hologoEntry}[3]{%
      \SetVariant{#1}{#2}%
      \raisebox{1em}[0pt][0pt]{\hypertarget{#1@#2}{}}%
      \bookmark[%
        dest={#1@#2},%
      ]{%
        #1\ifx\\#2\\\else\space(#2)\fi: \Hologo{#1}, \hologo{#1} %
        [Unicode]%
      }%
      \hypersetup{unicode=false}%
      \bookmark[%
        dest={#1@#2},%
      ]{%
        #1\ifx\\#2\\\else\space(#2)\fi: \Hologo{#1}, \hologo{#1} %
        [PDFDocEncoding]%
      }%
      \texttt{#1}%
      &%
      \texttt{#2}%
      &%
      \Hologo{#1}%
      &%
      \SetVariant{#1}{#2}%
      \hologo{#1}%
      &%
      \SetVariant{#1}{#2}%
      \fontfamily{qtm}\selectfont
      \hologo{#1}%
      &%
      \SetVariant{#1}{#2}%
      \fontfamily{qpl}\selectfont
      \hologo{#1}%
      &%
      \SetVariant{#1}{#2}%
      \textsf{\hologo{#1}}%
      &%
      \SetVariant{#1}{#2}%
      \fontfamily{qhv}\selectfont
      \hologo{#1}%
      \tabularnewline
    }%
    \begin{longtable}{llllllll}%
      \textbf{\textit{logo}} & \textbf{\textit{variant}} &
      \texttt{\string\Hologo} &
      \textbf{lmr} & \textbf{qtm} & \textbf{qpl} &
      \textbf{lmss} & \textbf{qhv}
      \tabularnewline
      \hline
      \endhead
      \hologoList
    \end{longtable}%
  \endgroup

\end{landscape}
\end{document}
%verbatim
%</example>
%    \end{macrocode}
%
% \StopEventually{
% }
%
% \section{Implementation}
%    \begin{macrocode}
%<*package>
%    \end{macrocode}
%    Reload check, especially if the package is not used with \LaTeX.
%    \begin{macrocode}
\begingroup\catcode61\catcode48\catcode32=10\relax%
  \catcode13=5 % ^^M
  \endlinechar=13 %
  \catcode35=6 % #
  \catcode39=12 % '
  \catcode44=12 % ,
  \catcode45=12 % -
  \catcode46=12 % .
  \catcode58=12 % :
  \catcode64=11 % @
  \catcode123=1 % {
  \catcode125=2 % }
  \expandafter\let\expandafter\x\csname ver@hologo.sty\endcsname
  \ifx\x\relax % plain-TeX, first loading
  \else
    \def\empty{}%
    \ifx\x\empty % LaTeX, first loading,
      % variable is initialized, but \ProvidesPackage not yet seen
    \else
      \expandafter\ifx\csname PackageInfo\endcsname\relax
        \def\x#1#2{%
          \immediate\write-1{Package #1 Info: #2.}%
        }%
      \else
        \def\x#1#2{\PackageInfo{#1}{#2, stopped}}%
      \fi
      \x{hologo}{The package is already loaded}%
      \aftergroup\endinput
    \fi
  \fi
\endgroup%
%    \end{macrocode}
%    Package identification:
%    \begin{macrocode}
\begingroup\catcode61\catcode48\catcode32=10\relax%
  \catcode13=5 % ^^M
  \endlinechar=13 %
  \catcode35=6 % #
  \catcode39=12 % '
  \catcode40=12 % (
  \catcode41=12 % )
  \catcode44=12 % ,
  \catcode45=12 % -
  \catcode46=12 % .
  \catcode47=12 % /
  \catcode58=12 % :
  \catcode64=11 % @
  \catcode91=12 % [
  \catcode93=12 % ]
  \catcode123=1 % {
  \catcode125=2 % }
  \expandafter\ifx\csname ProvidesPackage\endcsname\relax
    \def\x#1#2#3[#4]{\endgroup
      \immediate\write-1{Package: #3 #4}%
      \xdef#1{#4}%
    }%
  \else
    \def\x#1#2[#3]{\endgroup
      #2[{#3}]%
      \ifx#1\@undefined
        \xdef#1{#3}%
      \fi
      \ifx#1\relax
        \xdef#1{#3}%
      \fi
    }%
  \fi
\expandafter\x\csname ver@hologo.sty\endcsname
\ProvidesPackage{hologo}%
  [2016/05/12 v1.11 A logo collection with bookmark support (HO)]%
%    \end{macrocode}
%
%    \begin{macrocode}
\begingroup\catcode61\catcode48\catcode32=10\relax%
  \catcode13=5 % ^^M
  \endlinechar=13 %
  \catcode123=1 % {
  \catcode125=2 % }
  \catcode64=11 % @
  \def\x{\endgroup
    \expandafter\edef\csname HOLOGO@AtEnd\endcsname{%
      \endlinechar=\the\endlinechar\relax
      \catcode13=\the\catcode13\relax
      \catcode32=\the\catcode32\relax
      \catcode35=\the\catcode35\relax
      \catcode61=\the\catcode61\relax
      \catcode64=\the\catcode64\relax
      \catcode123=\the\catcode123\relax
      \catcode125=\the\catcode125\relax
    }%
  }%
\x\catcode61\catcode48\catcode32=10\relax%
\catcode13=5 % ^^M
\endlinechar=13 %
\catcode35=6 % #
\catcode64=11 % @
\catcode123=1 % {
\catcode125=2 % }
\def\TMP@EnsureCode#1#2{%
  \edef\HOLOGO@AtEnd{%
    \HOLOGO@AtEnd
    \catcode#1=\the\catcode#1\relax
  }%
  \catcode#1=#2\relax
}
\TMP@EnsureCode{10}{12}% ^^J
\TMP@EnsureCode{33}{12}% !
\TMP@EnsureCode{34}{12}% "
\TMP@EnsureCode{36}{3}% $
\TMP@EnsureCode{38}{4}% &
\TMP@EnsureCode{39}{12}% '
\TMP@EnsureCode{40}{12}% (
\TMP@EnsureCode{41}{12}% )
\TMP@EnsureCode{42}{12}% *
\TMP@EnsureCode{43}{12}% +
\TMP@EnsureCode{44}{12}% ,
\TMP@EnsureCode{45}{12}% -
\TMP@EnsureCode{46}{12}% .
\TMP@EnsureCode{47}{12}% /
\TMP@EnsureCode{58}{12}% :
\TMP@EnsureCode{59}{12}% ;
\TMP@EnsureCode{60}{12}% <
\TMP@EnsureCode{62}{12}% >
\TMP@EnsureCode{63}{12}% ?
\TMP@EnsureCode{91}{12}% [
\TMP@EnsureCode{93}{12}% ]
\TMP@EnsureCode{94}{7}% ^ (superscript)
\TMP@EnsureCode{95}{8}% _ (subscript)
\TMP@EnsureCode{96}{12}% `
\TMP@EnsureCode{124}{12}% |
\edef\HOLOGO@AtEnd{%
  \HOLOGO@AtEnd
  \escapechar\the\escapechar\relax
  \noexpand\endinput
}
\escapechar=92 %
%    \end{macrocode}
%
% \subsection{Logo list}
%
%    \begin{macro}{\hologoList}
%    \begin{macrocode}
\def\hologoList{%
  \hologoEntry{(La)TeX}{}{2011/10/01}%
  \hologoEntry{AmSLaTeX}{}{2010/04/16}%
  \hologoEntry{AmSTeX}{}{2010/04/16}%
  \hologoEntry{biber}{}{2011/10/01}%
  \hologoEntry{BibTeX}{}{2011/10/01}%
  \hologoEntry{BibTeX}{sf}{2011/10/01}%
  \hologoEntry{BibTeX}{sc}{2011/10/01}%
  \hologoEntry{BibTeX8}{}{2011/11/22}%
  \hologoEntry{ConTeXt}{}{2011/03/25}%
  \hologoEntry{ConTeXt}{narrow}{2011/03/25}%
  \hologoEntry{ConTeXt}{simple}{2011/03/25}%
  \hologoEntry{emTeX}{}{2010/04/26}%
  \hologoEntry{eTeX}{}{2010/04/08}%
  \hologoEntry{ExTeX}{}{2011/10/01}%
  \hologoEntry{HanTheThanh}{}{2011/11/29}%
  \hologoEntry{iniTeX}{}{2011/10/01}%
  \hologoEntry{KOMAScript}{}{2011/10/01}%
  \hologoEntry{La}{}{2010/05/08}%
  \hologoEntry{LaTeX}{}{2010/04/08}%
  \hologoEntry{LaTeX2e}{}{2010/04/08}%
  \hologoEntry{LaTeX3}{}{2010/04/24}%
  \hologoEntry{LaTeXe}{}{2010/04/08}%
  \hologoEntry{LaTeXML}{}{2011/11/22}%
  \hologoEntry{LaTeXTeX}{}{2011/10/01}%
  \hologoEntry{LuaLaTeX}{}{2010/04/08}%
  \hologoEntry{LuaTeX}{}{2010/04/08}%
  \hologoEntry{LyX}{}{2011/10/01}%
  \hologoEntry{METAFONT}{}{2011/10/01}%
  \hologoEntry{MetaFun}{}{2011/10/01}%
  \hologoEntry{METAPOST}{}{2011/10/01}%
  \hologoEntry{MetaPost}{}{2011/10/01}%
  \hologoEntry{MiKTeX}{}{2011/10/01}%
  \hologoEntry{NTS}{}{2011/10/01}%
  \hologoEntry{OzMF}{}{2011/10/01}%
  \hologoEntry{OzMP}{}{2011/10/01}%
  \hologoEntry{OzTeX}{}{2011/10/01}%
  \hologoEntry{OzTtH}{}{2011/10/01}%
  \hologoEntry{PCTeX}{}{2011/10/01}%
  \hologoEntry{pdfTeX}{}{2011/10/01}%
  \hologoEntry{pdfLaTeX}{}{2011/10/01}%
  \hologoEntry{PiC}{}{2011/10/01}%
  \hologoEntry{PiCTeX}{}{2011/10/01}%
  \hologoEntry{plainTeX}{}{2010/04/08}%
  \hologoEntry{plainTeX}{space}{2010/04/16}%
  \hologoEntry{plainTeX}{hyphen}{2010/04/16}%
  \hologoEntry{plainTeX}{runtogether}{2010/04/16}%
  \hologoEntry{SageTeX}{}{2011/11/22}%
  \hologoEntry{SLiTeX}{}{2011/10/01}%
  \hologoEntry{SLiTeX}{lift}{2011/10/01}%
  \hologoEntry{SLiTeX}{narrow}{2011/10/01}%
  \hologoEntry{SLiTeX}{simple}{2011/10/01}%
  \hologoEntry{SliTeX}{}{2011/10/01}%
  \hologoEntry{SliTeX}{narrow}{2011/10/01}%
  \hologoEntry{SliTeX}{simple}{2011/10/01}%
  \hologoEntry{SliTeX}{lift}{2011/10/01}%
  \hologoEntry{teTeX}{}{2011/10/01}%
  \hologoEntry{TeX}{}{2010/04/08}%
  \hologoEntry{TeX4ht}{}{2011/11/22}%
  \hologoEntry{TTH}{}{2011/11/22}%
  \hologoEntry{virTeX}{}{2011/10/01}%
  \hologoEntry{VTeX}{}{2010/04/24}%
  \hologoEntry{Xe}{}{2010/04/08}%
  \hologoEntry{XeLaTeX}{}{2010/04/08}%
  \hologoEntry{XeTeX}{}{2010/04/08}%
}
%    \end{macrocode}
%    \end{macro}
%
% \subsection{Load resources}
%
%    \begin{macrocode}
\begingroup\expandafter\expandafter\expandafter\endgroup
\expandafter\ifx\csname RequirePackage\endcsname\relax
  \def\TMP@RequirePackage#1[#2]{%
    \begingroup\expandafter\expandafter\expandafter\endgroup
    \expandafter\ifx\csname ver@#1.sty\endcsname\relax
      \input #1.sty\relax
    \fi
  }%
  \TMP@RequirePackage{ltxcmds}[2011/02/04]%
  \TMP@RequirePackage{infwarerr}[2010/04/08]%
  \TMP@RequirePackage{kvsetkeys}[2010/03/01]%
  \TMP@RequirePackage{kvdefinekeys}[2010/03/01]%
  \TMP@RequirePackage{pdftexcmds}[2010/04/01]%
  \TMP@RequirePackage{ifpdf}[2010/01/28]%
  \TMP@RequirePackage{ifluatex}[2010/03/01]%
  \ltx@IfUndefined{newif}{%
    \expandafter\let\csname newif\endcsname\ltx@newif
  }{}%
  \TMP@RequirePackage{ifxetex}[2009/01/23]%
  \TMP@RequirePackage{ifvtex}[2010/03/01]%
\else
  \RequirePackage{ltxcmds}[2011/02/04]%
  \RequirePackage{infwarerr}[2010/04/08]%
  \RequirePackage{kvsetkeys}[2010/03/01]%
  \RequirePackage{kvdefinekeys}[2010/03/01]%
  \RequirePackage{pdftexcmds}[2010/04/01]%
  \RequirePackage{ifpdf}[2010/01/28]%
  \RequirePackage{ifluatex}[2010/03/01]%
  \RequirePackage{ifxetex}[2009/01/23]%
  \RequirePackage{ifvtex}[2010/03/01]%
\fi
%    \end{macrocode}
%
%    \begin{macro}{\HOLOGO@IfDefined}
%    \begin{macrocode}
\def\HOLOGO@IfExists#1{%
  \ifx\@undefined#1%
    \expandafter\ltx@secondoftwo
  \else
    \ifx\relax#1%
      \expandafter\ltx@secondoftwo
    \else
      \expandafter\expandafter\expandafter\ltx@firstoftwo
    \fi
  \fi
}
%    \end{macrocode}
%    \end{macro}
%
% \subsection{Setup macros}
%
%    \begin{macro}{\hologoSetup}
%    \begin{macrocode}
\def\hologoSetup{%
  \let\HOLOGO@name\relax
  \HOLOGO@Setup
}
%    \end{macrocode}
%    \end{macro}
%
%    \begin{macro}{\hologoLogoSetup}
%    \begin{macrocode}
\def\hologoLogoSetup#1{%
  \edef\HOLOGO@name{#1}%
  \ltx@IfUndefined{HoLogo@\HOLOGO@name}{%
    \@PackageError{hologo}{%
      Unknown logo `\HOLOGO@name'%
    }\@ehc
    \ltx@gobble
  }{%
    \HOLOGO@Setup
  }%
}
%    \end{macrocode}
%    \end{macro}
%
%    \begin{macro}{\HOLOGO@Setup}
%    \begin{macrocode}
\def\HOLOGO@Setup{%
  \kvsetkeys{HoLogo}%
}
%    \end{macrocode}
%    \end{macro}
%
% \subsection{Options}
%
%    \begin{macro}{\HOLOGO@DeclareBoolOption}
%    \begin{macrocode}
\def\HOLOGO@DeclareBoolOption#1{%
  \expandafter\chardef\csname HOLOGOOPT@#1\endcsname\ltx@zero
  \kv@define@key{HoLogo}{#1}[true]{%
    \def\HOLOGO@temp{##1}%
    \ifx\HOLOGO@temp\HOLOGO@true
      \ifx\HOLOGO@name\relax
        \expandafter\chardef\csname HOLOGOOPT@#1\endcsname=\ltx@one
      \else
        \expandafter\chardef\csname
        HoLogoOpt@#1@\HOLOGO@name\endcsname\ltx@one
      \fi
      \HOLOGO@SetBreakAll{#1}%
    \else
      \ifx\HOLOGO@temp\HOLOGO@false
        \ifx\HOLOGO@name\relax
          \expandafter\chardef\csname HOLOGOOPT@#1\endcsname=\ltx@zero
        \else
          \expandafter\chardef\csname
          HoLogoOpt@#1@\HOLOGO@name\endcsname=\ltx@zero
        \fi
        \HOLOGO@SetBreakAll{#1}%
      \else
        \@PackageError{hologo}{%
          Unknown value `##1' for boolean option `#1'.\MessageBreak
          Known values are `true' and `false'%
        }\@ehc
      \fi
    \fi
  }%
}
%    \end{macrocode}
%    \end{macro}
%
%    \begin{macro}{\HOLOGO@SetBreakAll}
%    \begin{macrocode}
\def\HOLOGO@SetBreakAll#1{%
  \def\HOLOGO@temp{#1}%
  \ifx\HOLOGO@temp\HOLOGO@break
    \ifx\HOLOGO@name\relax
      \chardef\HOLOGOOPT@hyphenbreak=\HOLOGOOPT@break
      \chardef\HOLOGOOPT@spacebreak=\HOLOGOOPT@break
      \chardef\HOLOGOOPT@discretionarybreak=\HOLOGOOPT@break
    \else
      \expandafter\chardef
         \csname HoLogoOpt@hyphenbreak@\HOLOGO@name\endcsname=%
         \csname HoLogoOpt@break@\HOLOGO@name\endcsname
      \expandafter\chardef
         \csname HoLogoOpt@spacebreak@\HOLOGO@name\endcsname=%
         \csname HoLogoOpt@break@\HOLOGO@name\endcsname
      \expandafter\chardef
         \csname HoLogoOpt@discretionarybreak@\HOLOGO@name
             \endcsname=%
         \csname HoLogoOpt@break@\HOLOGO@name\endcsname
    \fi
  \fi
}
%    \end{macrocode}
%    \end{macro}
%
%    \begin{macro}{\HOLOGO@true}
%    \begin{macrocode}
\def\HOLOGO@true{true}
%    \end{macrocode}
%    \end{macro}
%    \begin{macro}{\HOLOGO@false}
%    \begin{macrocode}
\def\HOLOGO@false{false}
%    \end{macrocode}
%    \end{macro}
%    \begin{macro}{\HOLOGO@break}
%    \begin{macrocode}
\def\HOLOGO@break{break}
%    \end{macrocode}
%    \end{macro}
%
%    \begin{macrocode}
\HOLOGO@DeclareBoolOption{break}
\HOLOGO@DeclareBoolOption{hyphenbreak}
\HOLOGO@DeclareBoolOption{spacebreak}
\HOLOGO@DeclareBoolOption{discretionarybreak}
%    \end{macrocode}
%
%    \begin{macrocode}
\kv@define@key{HoLogo}{variant}{%
  \ifx\HOLOGO@name\relax
    \@PackageError{hologo}{%
      Option `variant' is not available in \string\hologoSetup,%
      \MessageBreak
      Use \string\hologoLogoSetup\space instead%
    }\@ehc
  \else
    \edef\HOLOGO@temp{#1}%
    \ifx\HOLOGO@temp\ltx@empty
      \expandafter
      \let\csname HoLogoOpt@variant@\HOLOGO@name\endcsname\@undefined
    \else
      \ltx@IfUndefined{HoLogo@\HOLOGO@name @\HOLOGO@temp}{%
        \@PackageError{hologo}{%
          Unknown variant `\HOLOGO@temp' of logo `\HOLOGO@name'%
        }\@ehc
      }{%
        \expandafter
        \let\csname HoLogoOpt@variant@\HOLOGO@name\endcsname
            \HOLOGO@temp
      }%
    \fi
  \fi
}
%    \end{macrocode}
%
%    \begin{macro}{\HOLOGO@Variant}
%    \begin{macrocode}
\def\HOLOGO@Variant#1{%
  #1%
  \ltx@ifundefined{HoLogoOpt@variant@#1}{%
  }{%
    @\csname HoLogoOpt@variant@#1\endcsname
  }%
}
%    \end{macrocode}
%    \end{macro}
%
% \subsection{Break/no-break support}
%
%    \begin{macro}{\HOLOGO@space}
%    \begin{macrocode}
\def\HOLOGO@space{%
  \ltx@ifundefined{HoLogoOpt@spacebreak@\HOLOGO@name}{%
    \ltx@ifundefined{HoLogoOpt@break@\HOLOGO@name}{%
      \chardef\HOLOGO@temp=\HOLOGOOPT@spacebreak
    }{%
      \chardef\HOLOGO@temp=%
        \csname HoLogoOpt@break@\HOLOGO@name\endcsname
    }%
  }{%
    \chardef\HOLOGO@temp=%
      \csname HoLogoOpt@spacebreak@\HOLOGO@name\endcsname
  }%
  \ifcase\HOLOGO@temp
    \penalty10000 %
  \fi
  \ltx@space
}
%    \end{macrocode}
%    \end{macro}
%
%    \begin{macro}{\HOLOGO@hyphen}
%    \begin{macrocode}
\def\HOLOGO@hyphen{%
  \ltx@ifundefined{HoLogoOpt@hyphenbreak@\HOLOGO@name}{%
    \ltx@ifundefined{HoLogoOpt@break@\HOLOGO@name}{%
      \chardef\HOLOGO@temp=\HOLOGOOPT@hyphenbreak
    }{%
      \chardef\HOLOGO@temp=%
        \csname HoLogoOpt@break@\HOLOGO@name\endcsname
    }%
  }{%
    \chardef\HOLOGO@temp=%
      \csname HoLogoOpt@hyphenbreak@\HOLOGO@name\endcsname
  }%
  \ifcase\HOLOGO@temp
    \ltx@mbox{-}%
  \else
    -%
  \fi
}
%    \end{macrocode}
%    \end{macro}
%
%    \begin{macro}{\HOLOGO@discretionary}
%    \begin{macrocode}
\def\HOLOGO@discretionary{%
  \ltx@ifundefined{HoLogoOpt@discretionarybreak@\HOLOGO@name}{%
    \ltx@ifundefined{HoLogoOpt@break@\HOLOGO@name}{%
      \chardef\HOLOGO@temp=\HOLOGOOPT@discretionarybreak
    }{%
      \chardef\HOLOGO@temp=%
        \csname HoLogoOpt@break@\HOLOGO@name\endcsname
    }%
  }{%
    \chardef\HOLOGO@temp=%
      \csname HoLogoOpt@discretionarybreak@\HOLOGO@name\endcsname
  }%
  \ifcase\HOLOGO@temp
  \else
    \-%
  \fi
}
%    \end{macrocode}
%    \end{macro}
%
%    \begin{macro}{\HOLOGO@mbox}
%    \begin{macrocode}
\def\HOLOGO@mbox#1{%
  \ltx@ifundefined{HoLogoOpt@break@\HOLOGO@name}{%
    \chardef\HOLOGO@temp=\HOLOGOOPT@hyphenbreak
  }{%
    \chardef\HOLOGO@temp=%
      \csname HoLogoOpt@break@\HOLOGO@name\endcsname
  }%
  \ifcase\HOLOGO@temp
    \ltx@mbox{#1}%
  \else
    #1%
  \fi
}
%    \end{macrocode}
%    \end{macro}
%
% \subsection{Font support}
%
%    \begin{macro}{\HoLogoFont@font}
%    \begin{tabular}{@{}ll@{}}
%    |#1|:& logo name\\
%    |#2|:& font short name\\
%    |#3|:& text
%    \end{tabular}
%    \begin{macrocode}
\def\HoLogoFont@font#1#2#3{%
  \begingroup
    \ltx@IfUndefined{HoLogoFont@logo@#1.#2}{%
      \ltx@IfUndefined{HoLogoFont@font@#2}{%
        \@PackageWarning{hologo}{%
          Missing font `#2' for logo `#1'%
        }%
        #3%
      }{%
        \csname HoLogoFont@font@#2\endcsname{#3}%
      }%
    }{%
      \csname HoLogoFont@logo@#1.#2\endcsname{#3}%
    }%
  \endgroup
}
%    \end{macrocode}
%    \end{macro}
%
%    \begin{macro}{\HoLogoFont@Def}
%    \begin{macrocode}
\def\HoLogoFont@Def#1{%
  \expandafter\def\csname HoLogoFont@font@#1\endcsname
}
%    \end{macrocode}
%    \end{macro}
%    \begin{macro}{\HoLogoFont@LogoDef}
%    \begin{macrocode}
\def\HoLogoFont@LogoDef#1#2{%
  \expandafter\def\csname HoLogoFont@logo@#1.#2\endcsname
}
%    \end{macrocode}
%    \end{macro}
%
% \subsubsection{Font defaults}
%
%    \begin{macro}{\HoLogoFont@font@general}
%    \begin{macrocode}
\HoLogoFont@Def{general}{}%
%    \end{macrocode}
%    \end{macro}
%
%    \begin{macro}{\HoLogoFont@font@rm}
%    \begin{macrocode}
\ltx@IfUndefined{rmfamily}{%
  \ltx@IfUndefined{rm}{%
  }{%
    \HoLogoFont@Def{rm}{\rm}%
  }%
}{%
  \HoLogoFont@Def{rm}{\rmfamily}%
}
%    \end{macrocode}
%    \end{macro}
%
%    \begin{macro}{\HoLogoFont@font@sf}
%    \begin{macrocode}
\ltx@IfUndefined{sffamily}{%
  \ltx@IfUndefined{sf}{%
  }{%
    \HoLogoFont@Def{sf}{\sf}%
  }%
}{%
  \HoLogoFont@Def{sf}{\sffamily}%
}
%    \end{macrocode}
%    \end{macro}
%
%    \begin{macro}{\HoLogoFont@font@bibsf}
%    In case of \hologo{plainTeX} the original small caps
%    variant is used as default. In \hologo{LaTeX}
%    the definition of package \xpackage{dtklogos} \cite{dtklogos}
%    is used.
%\begin{quote}
%\begin{verbatim}
%\DeclareRobustCommand{\BibTeX}{%
%  B%
%  \kern-.05em%
%  \hbox{%
%    $\m@th$% %% force math size calculations
%    \csname S@\f@size\endcsname
%    \fontsize\sf@size\z@
%    \math@fontsfalse
%    \selectfont
%    I%
%    \kern-.025em%
%    B
%  }%
%  \kern-.08em%
%  \-%
%  \TeX
%}
%\end{verbatim}
%\end{quote}
%    \begin{macrocode}
\ltx@IfUndefined{selectfont}{%
  \ltx@IfUndefined{tensc}{%
    \font\tensc=cmcsc10\relax
  }{}%
  \HoLogoFont@Def{bibsf}{\tensc}%
}{%
  \HoLogoFont@Def{bibsf}{%
    $\mathsurround=0pt$%
    \csname S@\f@size\endcsname
    \fontsize\sf@size{0pt}%
    \math@fontsfalse
    \selectfont
  }%
}
%    \end{macrocode}
%    \end{macro}
%
%    \begin{macro}{\HoLogoFont@font@sc}
%    \begin{macrocode}
\ltx@IfUndefined{scshape}{%
  \ltx@IfUndefined{tensc}{%
    \font\tensc=cmcsc10\relax
  }{}%
  \HoLogoFont@Def{sc}{\tensc}%
}{%
  \HoLogoFont@Def{sc}{\scshape}%
}
%    \end{macrocode}
%    \end{macro}
%
%    \begin{macro}{\HoLogoFont@font@sy}
%    \begin{macrocode}
\ltx@IfUndefined{usefont}{%
  \ltx@IfUndefined{tensy}{%
  }{%
    \HoLogoFont@Def{sy}{\tensy}%
  }%
}{%
  \HoLogoFont@Def{sy}{%
    \usefont{OMS}{cmsy}{m}{n}%
  }%
}
%    \end{macrocode}
%    \end{macro}
%
%    \begin{macro}{\HoLogoFont@font@logo}
%    \begin{macrocode}
\begingroup
  \def\x{LaTeX2e}%
\expandafter\endgroup
\ifx\fmtname\x
  \ltx@IfUndefined{logofamily}{%
    \DeclareRobustCommand\logofamily{%
      \not@math@alphabet\logofamily\relax
      \fontencoding{U}%
      \fontfamily{logo}%
      \selectfont
    }%
  }{}%
  \ltx@IfUndefined{logofamily}{%
  }{%
    \HoLogoFont@Def{logo}{\logofamily}%
  }%
\else
  \ltx@IfUndefined{tenlogo}{%
    \font\tenlogo=logo10\relax
  }{}%
  \HoLogoFont@Def{logo}{\tenlogo}%
\fi
%    \end{macrocode}
%    \end{macro}
%
% \subsubsection{Font setup}
%
%    \begin{macro}{\hologoFontSetup}
%    \begin{macrocode}
\def\hologoFontSetup{%
  \let\HOLOGO@name\relax
  \HOLOGO@FontSetup
}
%    \end{macrocode}
%    \end{macro}
%
%    \begin{macro}{\hologoLogoFontSetup}
%    \begin{macrocode}
\def\hologoLogoFontSetup#1{%
  \edef\HOLOGO@name{#1}%
  \ltx@IfUndefined{HoLogo@\HOLOGO@name}{%
    \@PackageError{hologo}{%
      Unknown logo `\HOLOGO@name'%
    }\@ehc
    \ltx@gobble
  }{%
    \HOLOGO@FontSetup
  }%
}
%    \end{macrocode}
%    \end{macro}
%
%    \begin{macro}{\HOLOGO@FontSetup}
%    \begin{macrocode}
\def\HOLOGO@FontSetup{%
  \kvsetkeys{HoLogoFont}%
}
%    \end{macrocode}
%    \end{macro}
%
%    \begin{macrocode}
\def\HOLOGO@temp#1{%
  \kv@define@key{HoLogoFont}{#1}{%
    \ifx\HOLOGO@name\relax
      \HoLogoFont@Def{#1}{##1}%
    \else
      \HoLogoFont@LogoDef\HOLOGO@name{#1}{##1}%
    \fi
  }%
}
\HOLOGO@temp{general}
\HOLOGO@temp{sf}
%    \end{macrocode}
%
% \subsection{Generic logo commands}
%
%    \begin{macrocode}
\HOLOGO@IfExists\hologo{%
  \@PackageError{hologo}{%
    \string\hologo\ltx@space is already defined.\MessageBreak
    Package loading is aborted%
  }\@ehc
  \HOLOGO@AtEnd
}%
\HOLOGO@IfExists\hologoRobust{%
  \@PackageError{hologo}{%
    \string\hologoRobust\ltx@space is already defined.\MessageBreak
    Package loading is aborted%
  }\@ehc
  \HOLOGO@AtEnd
}%
%    \end{macrocode}
%
% \subsubsection{\cs{hologo} and friends}
%
%    \begin{macrocode}
\ifluatex
  \expandafter\ltx@firstofone
\else
  \expandafter\ltx@gobble
\fi
{%
  \ltx@IfUndefined{ifincsname}{%
    \ifnum\luatexversion<36 %
      \expandafter\ltx@gobble
    \else
      \expandafter\ltx@firstofone
    \fi
    {%
      \begingroup
        \ifcase0%
            \directlua{%
              if tex.enableprimitives then %
                tex.enableprimitives('HOLOGO@', {'ifincsname'})%
              else %
                tex.print('1')%
              end%
            }%
            \ifx\HOLOGO@ifincsname\@undefined 1\fi%
            \relax
          \expandafter\ltx@firstofone
        \else
          \endgroup
          \expandafter\ltx@gobble
        \fi
        {%
          \global\let\ifincsname\HOLOGO@ifincsname
        }%
      \HOLOGO@temp
    }%
  }{}%
}
%    \end{macrocode}
%    \begin{macrocode}
\ltx@IfUndefined{ifincsname}{%
  \catcode`$=14 %
}{%
  \catcode`$=9 %
}
%    \end{macrocode}
%
%    \begin{macro}{\hologo}
%    \begin{macrocode}
\def\hologo#1{%
$ \ifincsname
$   \ltx@ifundefined{HoLogoCs@\HOLOGO@Variant{#1}}{%
$     #1%
$   }{%
$     \csname HoLogoCs@\HOLOGO@Variant{#1}\endcsname\ltx@firstoftwo
$   }%
$ \else
    \HOLOGO@IfExists\texorpdfstring\texorpdfstring\ltx@firstoftwo
    {%
      \hologoRobust{#1}%
    }{%
      \ltx@ifundefined{HoLogoBkm@\HOLOGO@Variant{#1}}{%
        \ltx@ifundefined{HoLogo@#1}{?#1?}{#1}%
      }{%
        \csname HoLogoBkm@\HOLOGO@Variant{#1}\endcsname
        \ltx@firstoftwo
      }%
    }%
$ \fi
}
%    \end{macrocode}
%    \end{macro}
%    \begin{macro}{\Hologo}
%    \begin{macrocode}
\def\Hologo#1{%
$ \ifincsname
$   \ltx@ifundefined{HoLogoCs@\HOLOGO@Variant{#1}}{%
$     #1%
$   }{%
$     \csname HoLogoCs@\HOLOGO@Variant{#1}\endcsname\ltx@secondoftwo
$   }%
$ \else
    \HOLOGO@IfExists\texorpdfstring\texorpdfstring\ltx@firstoftwo
    {%
      \HologoRobust{#1}%
    }{%
      \ltx@ifundefined{HoLogoBkm@\HOLOGO@Variant{#1}}{%
        \ltx@ifundefined{HoLogo@#1}{?#1?}{#1}%
      }{%
        \csname HoLogoBkm@\HOLOGO@Variant{#1}\endcsname
        \ltx@secondoftwo
      }%
    }%
$ \fi
}
%    \end{macrocode}
%    \end{macro}
%
%    \begin{macro}{\hologoVariant}
%    \begin{macrocode}
\def\hologoVariant#1#2{%
  \ifx\relax#2\relax
    \hologo{#1}%
  \else
$   \ifincsname
$     \ltx@ifundefined{HoLogoCs@#1@#2}{%
$       #1%
$     }{%
$       \csname HoLogoCs@#1@#2\endcsname\ltx@firstoftwo
$     }%
$   \else
      \HOLOGO@IfExists\texorpdfstring\texorpdfstring\ltx@firstoftwo
      {%
        \hologoVariantRobust{#1}{#2}%
      }{%
        \ltx@ifundefined{HoLogoBkm@#1@#2}{%
          \ltx@ifundefined{HoLogo@#1}{?#1?}{#1}%
        }{%
          \csname HoLogoBkm@#1@#2\endcsname
          \ltx@firstoftwo
        }%
      }%
$   \fi
  \fi
}
%    \end{macrocode}
%    \end{macro}
%    \begin{macro}{\HologoVariant}
%    \begin{macrocode}
\def\HologoVariant#1#2{%
  \ifx\relax#2\relax
    \Hologo{#1}%
  \else
$   \ifincsname
$     \ltx@ifundefined{HoLogoCs@#1@#2}{%
$       #1%
$     }{%
$       \csname HoLogoCs@#1@#2\endcsname\ltx@secondoftwo
$     }%
$   \else
      \HOLOGO@IfExists\texorpdfstring\texorpdfstring\ltx@firstoftwo
      {%
        \HologoVariantRobust{#1}{#2}%
      }{%
        \ltx@ifundefined{HoLogoBkm@#1@#2}{%
          \ltx@ifundefined{HoLogo@#1}{?#1?}{#1}%
        }{%
          \csname HoLogoBkm@#1@#2\endcsname
          \ltx@secondoftwo
        }%
      }%
$   \fi
  \fi
}
%    \end{macrocode}
%    \end{macro}
%
%    \begin{macrocode}
\catcode`\$=3 %
%    \end{macrocode}
%
% \subsubsection{\cs{hologoRobust} and friends}
%
%    \begin{macro}{\hologoRobust}
%    \begin{macrocode}
\ltx@IfUndefined{protected}{%
  \ltx@IfUndefined{DeclareRobustCommand}{%
    \def\hologoRobust#1%
  }{%
    \DeclareRobustCommand*\hologoRobust[1]%
  }%
}{%
  \protected\def\hologoRobust#1%
}%
{%
  \edef\HOLOGO@name{#1}%
  \ltx@IfUndefined{HoLogo@\HOLOGO@Variant\HOLOGO@name}{%
    \@PackageError{hologo}{%
      Unknown logo `\HOLOGO@name'%
    }\@ehc
    ?\HOLOGO@name?%
  }{%
    \ltx@IfUndefined{ver@tex4ht.sty}{%
      \HoLogoFont@font\HOLOGO@name{general}{%
        \csname HoLogo@\HOLOGO@Variant\HOLOGO@name\endcsname
        \ltx@firstoftwo
      }%
    }{%
      \ltx@IfUndefined{HoLogoHtml@\HOLOGO@Variant\HOLOGO@name}{%
        \HOLOGO@name
      }{%
        \csname HoLogoHtml@\HOLOGO@Variant\HOLOGO@name\endcsname
        \ltx@firstoftwo
      }%
    }%
  }%
}
%    \end{macrocode}
%    \end{macro}
%    \begin{macro}{\HologoRobust}
%    \begin{macrocode}
\ltx@IfUndefined{protected}{%
  \ltx@IfUndefined{DeclareRobustCommand}{%
    \def\HologoRobust#1%
  }{%
    \DeclareRobustCommand*\HologoRobust[1]%
  }%
}{%
  \protected\def\HologoRobust#1%
}%
{%
  \edef\HOLOGO@name{#1}%
  \ltx@IfUndefined{HoLogo@\HOLOGO@Variant\HOLOGO@name}{%
    \@PackageError{hologo}{%
      Unknown logo `\HOLOGO@name'%
    }\@ehc
    ?\HOLOGO@name?%
  }{%
    \ltx@IfUndefined{ver@tex4ht.sty}{%
      \HoLogoFont@font\HOLOGO@name{general}{%
        \csname HoLogo@\HOLOGO@Variant\HOLOGO@name\endcsname
        \ltx@secondoftwo
      }%
    }{%
      \ltx@IfUndefined{HoLogoHtml@\HOLOGO@Variant\HOLOGO@name}{%
        \expandafter\HOLOGO@Uppercase\HOLOGO@name
      }{%
        \csname HoLogoHtml@\HOLOGO@Variant\HOLOGO@name\endcsname
        \ltx@secondoftwo
      }%
    }%
  }%
}
%    \end{macrocode}
%    \end{macro}
%    \begin{macro}{\hologoVariantRobust}
%    \begin{macrocode}
\ltx@IfUndefined{protected}{%
  \ltx@IfUndefined{DeclareRobustCommand}{%
    \def\hologoVariantRobust#1#2%
  }{%
    \DeclareRobustCommand*\hologoVariantRobust[2]%
  }%
}{%
  \protected\def\hologoVariantRobust#1#2%
}%
{%
  \begingroup
    \hologoLogoSetup{#1}{variant={#2}}%
    \hologoRobust{#1}%
  \endgroup
}
%    \end{macrocode}
%    \end{macro}
%    \begin{macro}{\HologoVariantRobust}
%    \begin{macrocode}
\ltx@IfUndefined{protected}{%
  \ltx@IfUndefined{DeclareRobustCommand}{%
    \def\HologoVariantRobust#1#2%
  }{%
    \DeclareRobustCommand*\HologoVariantRobust[2]%
  }%
}{%
  \protected\def\HologoVariantRobust#1#2%
}%
{%
  \begingroup
    \hologoLogoSetup{#1}{variant={#2}}%
    \HologoRobust{#1}%
  \endgroup
}
%    \end{macrocode}
%    \end{macro}
%
%    \begin{macro}{\hologorobust}
%    Macro \cs{hologorobust} is only defined for compatibility.
%    Its use is deprecated.
%    \begin{macrocode}
\def\hologorobust{\hologoRobust}
%    \end{macrocode}
%    \end{macro}
%
% \subsection{Helpers}
%
%    \begin{macro}{\HOLOGO@Uppercase}
%    Macro \cs{HOLOGO@Uppercase} is restricted to \cs{uppercase},
%    because \hologo{plainTeX} or \hologo{iniTeX} do not provide
%    \cs{MakeUppercase}.
%    \begin{macrocode}
\def\HOLOGO@Uppercase#1{\uppercase{#1}}
%    \end{macrocode}
%    \end{macro}
%
%    \begin{macro}{\HOLOGO@PdfdocUnicode}
%    \begin{macrocode}
\def\HOLOGO@PdfdocUnicode{%
  \ifx\ifHy@unicode\iftrue
    \expandafter\ltx@secondoftwo
  \else
    \expandafter\ltx@firstoftwo
  \fi
}
%    \end{macrocode}
%    \end{macro}
%
%    \begin{macro}{\HOLOGO@Math}
%    \begin{macrocode}
\def\HOLOGO@MathSetup{%
  \mathsurround0pt\relax
  \HOLOGO@IfExists\f@series{%
    \if b\expandafter\ltx@car\f@series x\@nil
      \csname boldmath\endcsname
   \fi
  }{}%
}
%    \end{macrocode}
%    \end{macro}
%
%    \begin{macro}{\HOLOGO@TempDimen}
%    \begin{macrocode}
\dimendef\HOLOGO@TempDimen=\ltx@zero
%    \end{macrocode}
%    \end{macro}
%    \begin{macro}{\HOLOGO@NegativeKerning}
%    \begin{macrocode}
\def\HOLOGO@NegativeKerning#1{%
  \begingroup
    \HOLOGO@TempDimen=0pt\relax
    \comma@parse@normalized{#1}{%
      \ifdim\HOLOGO@TempDimen=0pt %
        \expandafter\HOLOGO@@NegativeKerning\comma@entry
      \fi
      \ltx@gobble
    }%
    \ifdim\HOLOGO@TempDimen<0pt %
      \kern\HOLOGO@TempDimen
    \fi
  \endgroup
}
%    \end{macrocode}
%    \end{macro}
%    \begin{macro}{\HOLOGO@@NegativeKerning}
%    \begin{macrocode}
\def\HOLOGO@@NegativeKerning#1#2{%
  \setbox\ltx@zero\hbox{#1#2}%
  \HOLOGO@TempDimen=\wd\ltx@zero
  \setbox\ltx@zero\hbox{#1\kern0pt#2}%
  \advance\HOLOGO@TempDimen by -\wd\ltx@zero
}
%    \end{macrocode}
%    \end{macro}
%
%    \begin{macro}{\HOLOGO@SpaceFactor}
%    \begin{macrocode}
\def\HOLOGO@SpaceFactor{%
  \spacefactor1000 %
}
%    \end{macrocode}
%    \end{macro}
%
%    \begin{macro}{\HOLOGO@Span}
%    \begin{macrocode}
\def\HOLOGO@Span#1#2{%
  \HCode{<span class="HoLogo-#1">}%
  #2%
  \HCode{</span>}%
}
%    \end{macrocode}
%    \end{macro}
%
% \subsubsection{Text subscript}
%
%    \begin{macro}{\HOLOGO@SubScript}%
%    \begin{macrocode}
\def\HOLOGO@SubScript#1{%
  \ltx@IfUndefined{textsubscript}{%
    \ltx@IfUndefined{text}{%
      \ltx@mbox{%
        \mathsurround=0pt\relax
        $%
          _{%
            \ltx@IfUndefined{sf@size}{%
              \mathrm{#1}%
            }{%
              \mbox{%
                \fontsize\sf@size{0pt}\selectfont
                #1%
              }%
            }%
          }%
        $%
      }%
    }{%
      \ltx@mbox{%
        \mathsurround=0pt\relax
        $_{\text{#1}}$%
      }%
    }%
  }{%
    \textsubscript{#1}%
  }%
}
%    \end{macrocode}
%    \end{macro}
%
% \subsection{\hologo{TeX} and friends}
%
% \subsubsection{\hologo{TeX}}
%
%    \begin{macro}{\HoLogo@TeX}
%    Source: \hologo{LaTeX} kernel.
%    \begin{macrocode}
\def\HoLogo@TeX#1{%
  T\kern-.1667em\lower.5ex\hbox{E}\kern-.125emX\HOLOGO@SpaceFactor
}
%    \end{macrocode}
%    \end{macro}
%    \begin{macro}{\HoLogoHtml@TeX}
%    \begin{macrocode}
\def\HoLogoHtml@TeX#1{%
  \HoLogoCss@TeX
  \HOLOGO@Span{TeX}{%
    T%
    \HOLOGO@Span{e}{%
      E%
    }%
    X%
  }%
}
%    \end{macrocode}
%    \end{macro}
%    \begin{macro}{\HoLogoCss@TeX}
%    \begin{macrocode}
\def\HoLogoCss@TeX{%
  \Css{%
    span.HoLogo-TeX span.HoLogo-e{%
      position:relative;%
      top:.5ex;%
      margin-left:-.1667em;%
      margin-right:-.125em;%
    }%
  }%
  \Css{%
    a span.HoLogo-TeX span.HoLogo-e{%
      text-decoration:none;%
    }%
  }%
  \global\let\HoLogoCss@TeX\relax
}
%    \end{macrocode}
%    \end{macro}
%
% \subsubsection{\hologo{plainTeX}}
%
%    \begin{macro}{\HoLogo@plainTeX@space}
%    Source: ``The \hologo{TeX}book''
%    \begin{macrocode}
\def\HoLogo@plainTeX@space#1{%
  \HOLOGO@mbox{#1{p}{P}lain}\HOLOGO@space\hologo{TeX}%
}
%    \end{macrocode}
%    \end{macro}
%    \begin{macro}{\HoLogoCs@plainTeX@space}
%    \begin{macrocode}
\def\HoLogoCs@plainTeX@space#1{#1{p}{P}lain TeX}%
%    \end{macrocode}
%    \end{macro}
%    \begin{macro}{\HoLogoBkm@plainTeX@space}
%    \begin{macrocode}
\def\HoLogoBkm@plainTeX@space#1{%
  #1{p}{P}lain \hologo{TeX}%
}
%    \end{macrocode}
%    \end{macro}
%    \begin{macro}{\HoLogoHtml@plainTeX@space}
%    \begin{macrocode}
\def\HoLogoHtml@plainTeX@space#1{%
  #1{p}{P}lain \hologo{TeX}%
}
%    \end{macrocode}
%    \end{macro}
%
%    \begin{macro}{\HoLogo@plainTeX@hyphen}
%    \begin{macrocode}
\def\HoLogo@plainTeX@hyphen#1{%
  \HOLOGO@mbox{#1{p}{P}lain}\HOLOGO@hyphen\hologo{TeX}%
}
%    \end{macrocode}
%    \end{macro}
%    \begin{macro}{\HoLogoCs@plainTeX@hyphen}
%    \begin{macrocode}
\def\HoLogoCs@plainTeX@hyphen#1{#1{p}{P}lain-TeX}
%    \end{macrocode}
%    \end{macro}
%    \begin{macro}{\HoLogoBkm@plainTeX@hyphen}
%    \begin{macrocode}
\def\HoLogoBkm@plainTeX@hyphen#1{%
  #1{p}{P}lain-\hologo{TeX}%
}
%    \end{macrocode}
%    \end{macro}
%    \begin{macro}{\HoLogoHtml@plainTeX@hyphen}
%    \begin{macrocode}
\def\HoLogoHtml@plainTeX@hyphen#1{%
  #1{p}{P}lain-\hologo{TeX}%
}
%    \end{macrocode}
%    \end{macro}
%
%    \begin{macro}{\HoLogo@plainTeX@runtogether}
%    \begin{macrocode}
\def\HoLogo@plainTeX@runtogether#1{%
  \HOLOGO@mbox{#1{p}{P}lain\hologo{TeX}}%
}
%    \end{macrocode}
%    \end{macro}
%    \begin{macro}{\HoLogoCs@plainTeX@runtogether}
%    \begin{macrocode}
\def\HoLogoCs@plainTeX@runtogether#1{#1{p}{P}lainTeX}
%    \end{macrocode}
%    \end{macro}
%    \begin{macro}{\HoLogoBkm@plainTeX@runtogether}
%    \begin{macrocode}
\def\HoLogoBkm@plainTeX@runtogether#1{%
  #1{p}{P}lain\hologo{TeX}%
}
%    \end{macrocode}
%    \end{macro}
%    \begin{macro}{\HoLogoHtml@plainTeX@runtogether}
%    \begin{macrocode}
\def\HoLogoHtml@plainTeX@runtogether#1{%
  #1{p}{P}lain\hologo{TeX}%
}
%    \end{macrocode}
%    \end{macro}
%
%    \begin{macro}{\HoLogo@plainTeX}
%    \begin{macrocode}
\def\HoLogo@plainTeX{\HoLogo@plainTeX@space}
%    \end{macrocode}
%    \end{macro}
%    \begin{macro}{\HoLogoCs@plainTeX}
%    \begin{macrocode}
\def\HoLogoCs@plainTeX{\HoLogoCs@plainTeX@space}
%    \end{macrocode}
%    \end{macro}
%    \begin{macro}{\HoLogoBkm@plainTeX}
%    \begin{macrocode}
\def\HoLogoBkm@plainTeX{\HoLogoBkm@plainTeX@space}
%    \end{macrocode}
%    \end{macro}
%    \begin{macro}{\HoLogoHtml@plainTeX}
%    \begin{macrocode}
\def\HoLogoHtml@plainTeX{\HoLogoHtml@plainTeX@space}
%    \end{macrocode}
%    \end{macro}
%
% \subsubsection{\hologo{LaTeX}}
%
%    Source: \hologo{LaTeX} kernel.
%\begin{quote}
%\begin{verbatim}
%\DeclareRobustCommand{\LaTeX}{%
%  L%
%  \kern-.36em%
%  {%
%    \sbox\z@ T%
%    \vbox to\ht\z@{%
%      \hbox{%
%        \check@mathfonts
%        \fontsize\sf@size\z@
%        \math@fontsfalse
%        \selectfont
%        A%
%      }%
%      \vss
%    }%
%  }%
%  \kern-.15em%
%  \TeX
%}
%\end{verbatim}
%\end{quote}
%
%    \begin{macro}{\HoLogo@La}
%    \begin{macrocode}
\def\HoLogo@La#1{%
  L%
  \kern-.36em%
  \begingroup
    \setbox\ltx@zero\hbox{T}%
    \vbox to\ht\ltx@zero{%
      \hbox{%
        \ltx@ifundefined{check@mathfonts}{%
          \csname sevenrm\endcsname
        }{%
          \check@mathfonts
          \fontsize\sf@size{0pt}%
          \math@fontsfalse\selectfont
        }%
        A%
      }%
      \vss
    }%
  \endgroup
}
%    \end{macrocode}
%    \end{macro}
%
%    \begin{macro}{\HoLogo@LaTeX}
%    Source: \hologo{LaTeX} kernel.
%    \begin{macrocode}
\def\HoLogo@LaTeX#1{%
  \hologo{La}%
  \kern-.15em%
  \hologo{TeX}%
}
%    \end{macrocode}
%    \end{macro}
%    \begin{macro}{\HoLogoHtml@LaTeX}
%    \begin{macrocode}
\def\HoLogoHtml@LaTeX#1{%
  \HoLogoCss@LaTeX
  \HOLOGO@Span{LaTeX}{%
    L%
    \HOLOGO@Span{a}{%
      A%
    }%
    \hologo{TeX}%
  }%
}
%    \end{macrocode}
%    \end{macro}
%    \begin{macro}{\HoLogoCss@LaTeX}
%    \begin{macrocode}
\def\HoLogoCss@LaTeX{%
  \Css{%
    span.HoLogo-LaTeX span.HoLogo-a{%
      position:relative;%
      top:-.5ex;%
      margin-left:-.36em;%
      margin-right:-.15em;%
      font-size:85\%;%
    }%
  }%
  \global\let\HoLogoCss@LaTeX\relax
}
%    \end{macrocode}
%    \end{macro}
%
% \subsubsection{\hologo{(La)TeX}}
%
%    \begin{macro}{\HoLogo@LaTeXTeX}
%    The kerning around the parentheses is taken
%    from package \xpackage{dtklogos} \cite{dtklogos}.
%\begin{quote}
%\begin{verbatim}
%\DeclareRobustCommand{\LaTeXTeX}{%
%  (%
%  \kern-.15em%
%  L%
%  \kern-.36em%
%  {%
%    \sbox\z@ T%
%    \vbox to\ht0{%
%      \hbox{%
%        $\m@th$%
%        \csname S@\f@size\endcsname
%        \fontsize\sf@size\z@
%        \math@fontsfalse
%        \selectfont
%        A%
%      }%
%      \vss
%    }%
%  }%
%  \kern-.2em%
%  )%
%  \kern-.15em%
%  \TeX
%}
%\end{verbatim}
%\end{quote}
%    \begin{macrocode}
\def\HoLogo@LaTeXTeX#1{%
  (%
  \kern-.15em%
  \hologo{La}%
  \kern-.2em%
  )%
  \kern-.15em%
  \hologo{TeX}%
}
%    \end{macrocode}
%    \end{macro}
%    \begin{macro}{\HoLogoBkm@LaTeXTeX}
%    \begin{macrocode}
\def\HoLogoBkm@LaTeXTeX#1{(La)TeX}
%    \end{macrocode}
%    \end{macro}
%
%    \begin{macro}{\HoLogo@(La)TeX}
%    \begin{macrocode}
\expandafter
\let\csname HoLogo@(La)TeX\endcsname\HoLogo@LaTeXTeX
%    \end{macrocode}
%    \end{macro}
%    \begin{macro}{\HoLogoBkm@(La)TeX}
%    \begin{macrocode}
\expandafter
\let\csname HoLogoBkm@(La)TeX\endcsname\HoLogoBkm@LaTeXTeX
%    \end{macrocode}
%    \end{macro}
%    \begin{macro}{\HoLogoHtml@LaTeXTeX}
%    \begin{macrocode}
\def\HoLogoHtml@LaTeXTeX#1{%
  \HoLogoCss@LaTeXTeX
  \HOLOGO@Span{LaTeXTeX}{%
    (%
    \HOLOGO@Span{L}{L}%
    \HOLOGO@Span{a}{A}%
    \HOLOGO@Span{ParenRight}{)}%
    \hologo{TeX}%
  }%
}
%    \end{macrocode}
%    \end{macro}
%    \begin{macro}{\HoLogoHtml@(La)TeX}
%    Kerning after opening parentheses and before closing parentheses
%    is $-0.1$\,em. The original values $-0.15$\,em
%    looked too ugly for a serif font.
%    \begin{macrocode}
\expandafter
\let\csname HoLogoHtml@(La)TeX\endcsname\HoLogoHtml@LaTeXTeX
%    \end{macrocode}
%    \end{macro}
%    \begin{macro}{\HoLogoCss@LaTeXTeX}
%    \begin{macrocode}
\def\HoLogoCss@LaTeXTeX{%
  \Css{%
    span.HoLogo-LaTeXTeX span.HoLogo-L{%
      margin-left:-.1em;%
    }%
  }%
  \Css{%
    span.HoLogo-LaTeXTeX span.HoLogo-a{%
      position:relative;%
      top:-.5ex;%
      margin-left:-.36em;%
      margin-right:-.1em;%
      font-size:85\%;%
    }%
  }%
  \Css{%
    span.HoLogo-LaTeXTeX span.HoLogo-ParenRight{%
      margin-right:-.15em;%
    }%
  }%
  \global\let\HoLogoCss@LaTeXTeX\relax
}
%    \end{macrocode}
%    \end{macro}
%
% \subsubsection{\hologo{LaTeXe}}
%
%    \begin{macro}{\HoLogo@LaTeXe}
%    Source: \hologo{LaTeX} kernel
%    \begin{macrocode}
\def\HoLogo@LaTeXe#1{%
  \hologo{LaTeX}%
  \kern.15em%
  \hbox{%
    \HOLOGO@MathSetup
    2%
    $_{\textstyle\varepsilon}$%
  }%
}
%    \end{macrocode}
%    \end{macro}
%
%    \begin{macro}{\HoLogoCs@LaTeXe}
%    \begin{macrocode}
\ifnum64=`\^^^^0040\relax % test for big chars of LuaTeX/XeTeX
  \catcode`\$=9 %
  \catcode`\&=14 %
\else
  \catcode`\$=14 %
  \catcode`\&=9 %
\fi
\def\HoLogoCs@LaTeXe#1{%
  LaTeX2%
$ \string ^^^^0395%
& e%
}%
\catcode`\$=3 %
\catcode`\&=4 %
%    \end{macrocode}
%    \end{macro}
%
%    \begin{macro}{\HoLogoBkm@LaTeXe}
%    \begin{macrocode}
\def\HoLogoBkm@LaTeXe#1{%
  \hologo{LaTeX}%
  2%
  \HOLOGO@PdfdocUnicode{e}{\textepsilon}%
}
%    \end{macrocode}
%    \end{macro}
%
%    \begin{macro}{\HoLogoHtml@LaTeXe}
%    \begin{macrocode}
\def\HoLogoHtml@LaTeXe#1{%
  \HoLogoCss@LaTeXe
  \HOLOGO@Span{LaTeX2e}{%
    \hologo{LaTeX}%
    \HOLOGO@Span{2}{2}%
    \HOLOGO@Span{e}{%
      \HOLOGO@MathSetup
      \ensuremath{\textstyle\varepsilon}%
    }%
  }%
}
%    \end{macrocode}
%    \end{macro}
%    \begin{macro}{\HoLogoCss@LaTeXe}
%    \begin{macrocode}
\def\HoLogoCss@LaTeXe{%
  \Css{%
    span.HoLogo-LaTeX2e span.HoLogo-2{%
      padding-left:.15em;%
    }%
  }%
  \Css{%
    span.HoLogo-LaTeX2e span.HoLogo-e{%
      position:relative;%
      top:.35ex;%
      text-decoration:none;%
    }%
  }%
  \global\let\HoLogoCss@LaTeXe\relax
}
%    \end{macrocode}
%    \end{macro}
%
%    \begin{macro}{\HoLogo@LaTeX2e}
%    \begin{macrocode}
\expandafter
\let\csname HoLogo@LaTeX2e\endcsname\HoLogo@LaTeXe
%    \end{macrocode}
%    \end{macro}
%    \begin{macro}{\HoLogoCs@LaTeX2e}
%    \begin{macrocode}
\expandafter
\let\csname HoLogoCs@LaTeX2e\endcsname\HoLogoCs@LaTeXe
%    \end{macrocode}
%    \end{macro}
%    \begin{macro}{\HoLogoBkm@LaTeX2e}
%    \begin{macrocode}
\expandafter
\let\csname HoLogoBkm@LaTeX2e\endcsname\HoLogoBkm@LaTeXe
%    \end{macrocode}
%    \end{macro}
%    \begin{macro}{\HoLogoHtml@LaTeX2e}
%    \begin{macrocode}
\expandafter
\let\csname HoLogoHtml@LaTeX2e\endcsname\HoLogoHtml@LaTeXe
%    \end{macrocode}
%    \end{macro}
%
% \subsubsection{\hologo{LaTeX3}}
%
%    \begin{macro}{\HoLogo@LaTeX3}
%    Source: \hologo{LaTeX} kernel
%    \begin{macrocode}
\expandafter\def\csname HoLogo@LaTeX3\endcsname#1{%
  \hologo{LaTeX}%
  3%
}
%    \end{macrocode}
%    \end{macro}
%
%    \begin{macro}{\HoLogoBkm@LaTeX3}
%    \begin{macrocode}
\expandafter\def\csname HoLogoBkm@LaTeX3\endcsname#1{%
  \hologo{LaTeX}%
  3%
}
%    \end{macrocode}
%    \end{macro}
%    \begin{macro}{\HoLogoHtml@LaTeX3}
%    \begin{macrocode}
\expandafter
\let\csname HoLogoHtml@LaTeX3\expandafter\endcsname
\csname HoLogo@LaTeX3\endcsname
%    \end{macrocode}
%    \end{macro}
%
% \subsubsection{\hologo{LaTeXML}}
%
%    \begin{macro}{\HoLogo@LaTeXML}
%    \begin{macrocode}
\def\HoLogo@LaTeXML#1{%
  \HOLOGO@mbox{%
    \hologo{La}%
    \kern-.15em%
    T%
    \kern-.1667em%
    \lower.5ex\hbox{E}%
    \kern-.125em%
    \HoLogoFont@font{LaTeXML}{sc}{xml}%
  }%
}
%    \end{macrocode}
%    \end{macro}
%    \begin{macro}{\HoLogoHtml@pdfLaTeX}
%    \begin{macrocode}
\def\HoLogoHtml@LaTeXML#1{%
  \HOLOGO@Span{LaTeXML}{%
    \HoLogoCss@LaTeX
    \HoLogoCss@TeX
    \HOLOGO@Span{LaTeX}{%
      L%
      \HOLOGO@Span{a}{%
        A%
      }%
    }%
    \HOLOGO@Span{TeX}{%
      T%
      \HOLOGO@Span{e}{%
        E%
      }%
    }%
    \HCode{<span style="font-variant: small-caps;">}%
    xml%
    \HCode{</span>}%
  }%
}
%    \end{macrocode}
%    \end{macro}
%
% \subsubsection{\hologo{eTeX}}
%
%    \begin{macro}{\HoLogo@eTeX}
%    Source: package \xpackage{etex}
%    \begin{macrocode}
\def\HoLogo@eTeX#1{%
  \ltx@mbox{%
    \HOLOGO@MathSetup
    $\varepsilon$%
    -%
    \HOLOGO@NegativeKerning{-T,T-,To}%
    \hologo{TeX}%
  }%
}
%    \end{macrocode}
%    \end{macro}
%    \begin{macro}{\HoLogoCs@eTeX}
%    \begin{macrocode}
\ifnum64=`\^^^^0040\relax % test for big chars of LuaTeX/XeTeX
  \catcode`\$=9 %
  \catcode`\&=14 %
\else
  \catcode`\$=14 %
  \catcode`\&=9 %
\fi
\def\HoLogoCs@eTeX#1{%
$ #1{\string ^^^^0395}{\string ^^^^03b5}%
& #1{e}{E}%
  TeX%
}%
\catcode`\$=3 %
\catcode`\&=4 %
%    \end{macrocode}
%    \end{macro}
%    \begin{macro}{\HoLogoBkm@eTeX}
%    \begin{macrocode}
\def\HoLogoBkm@eTeX#1{%
  \HOLOGO@PdfdocUnicode{#1{e}{E}}{\textepsilon}%
  -%
  \hologo{TeX}%
}
%    \end{macrocode}
%    \end{macro}
%    \begin{macro}{\HoLogoHtml@eTeX}
%    \begin{macrocode}
\def\HoLogoHtml@eTeX#1{%
  \ltx@mbox{%
    \HOLOGO@MathSetup
    $\varepsilon$%
    -%
    \hologo{TeX}%
  }%
}
%    \end{macrocode}
%    \end{macro}
%
% \subsubsection{\hologo{iniTeX}}
%
%    \begin{macro}{\HoLogo@iniTeX}
%    \begin{macrocode}
\def\HoLogo@iniTeX#1{%
  \HOLOGO@mbox{%
    #1{i}{I}ni\hologo{TeX}%
  }%
}
%    \end{macrocode}
%    \end{macro}
%    \begin{macro}{\HoLogoCs@iniTeX}
%    \begin{macrocode}
\def\HoLogoCs@iniTeX#1{#1{i}{I}niTeX}
%    \end{macrocode}
%    \end{macro}
%    \begin{macro}{\HoLogoBkm@iniTeX}
%    \begin{macrocode}
\def\HoLogoBkm@iniTeX#1{%
  #1{i}{I}ni\hologo{TeX}%
}
%    \end{macrocode}
%    \end{macro}
%    \begin{macro}{\HoLogoHtml@iniTeX}
%    \begin{macrocode}
\let\HoLogoHtml@iniTeX\HoLogo@iniTeX
%    \end{macrocode}
%    \end{macro}
%
% \subsubsection{\hologo{virTeX}}
%
%    \begin{macro}{\HoLogo@virTeX}
%    \begin{macrocode}
\def\HoLogo@virTeX#1{%
  \HOLOGO@mbox{%
    #1{v}{V}ir\hologo{TeX}%
  }%
}
%    \end{macrocode}
%    \end{macro}
%    \begin{macro}{\HoLogoCs@virTeX}
%    \begin{macrocode}
\def\HoLogoCs@virTeX#1{#1{v}{V}irTeX}
%    \end{macrocode}
%    \end{macro}
%    \begin{macro}{\HoLogoBkm@virTeX}
%    \begin{macrocode}
\def\HoLogoBkm@virTeX#1{%
  #1{v}{V}ir\hologo{TeX}%
}
%    \end{macrocode}
%    \end{macro}
%    \begin{macro}{\HoLogoHtml@virTeX}
%    \begin{macrocode}
\let\HoLogoHtml@virTeX\HoLogo@virTeX
%    \end{macrocode}
%    \end{macro}
%
% \subsubsection{\hologo{SliTeX}}
%
% \paragraph{Definitions of the three variants.}
%
%    \begin{macro}{\HoLogo@SLiTeX@lift}
%    \begin{macrocode}
\def\HoLogo@SLiTeX@lift#1{%
  \HoLogoFont@font{SliTeX}{rm}{%
    S%
    \kern-.06em%
    L%
    \kern-.18em%
    \raise.32ex\hbox{\HoLogoFont@font{SliTeX}{sc}{i}}%
    \HOLOGO@discretionary
    \kern-.06em%
    \hologo{TeX}%
  }%
}
%    \end{macrocode}
%    \end{macro}
%    \begin{macro}{\HoLogoBkm@SLiTeX@lift}
%    \begin{macrocode}
\def\HoLogoBkm@SLiTeX@lift#1{SLiTeX}
%    \end{macrocode}
%    \end{macro}
%    \begin{macro}{\HoLogoHtml@SLiTeX@lift}
%    \begin{macrocode}
\def\HoLogoHtml@SLiTeX@lift#1{%
  \HoLogoCss@SLiTeX@lift
  \HOLOGO@Span{SLiTeX-lift}{%
    \HoLogoFont@font{SliTeX}{rm}{%
      S%
      \HOLOGO@Span{L}{L}%
      \HOLOGO@Span{i}{i}%
      \hologo{TeX}%
    }%
  }%
}
%    \end{macrocode}
%    \end{macro}
%    \begin{macro}{\HoLogoCss@SLiTeX@lift}
%    \begin{macrocode}
\def\HoLogoCss@SLiTeX@lift{%
  \Css{%
    span.HoLogo-SLiTeX-lift span.HoLogo-L{%
      margin-left:-.06em;%
      margin-right:-.18em;%
    }%
  }%
  \Css{%
    span.HoLogo-SLiTeX-lift span.HoLogo-i{%
      position:relative;%
      top:-.32ex;%
      margin-right:-.06em;%
      font-variant:small-caps;%
    }%
  }%
  \global\let\HoLogoCss@SLiTeX@lift\relax
}
%    \end{macrocode}
%    \end{macro}
%
%    \begin{macro}{\HoLogo@SliTeX@simple}
%    \begin{macrocode}
\def\HoLogo@SliTeX@simple#1{%
  \HoLogoFont@font{SliTeX}{rm}{%
    \ltx@mbox{%
      \HoLogoFont@font{SliTeX}{sc}{Sli}%
    }%
    \HOLOGO@discretionary
    \hologo{TeX}%
  }%
}
%    \end{macrocode}
%    \end{macro}
%    \begin{macro}{\HoLogoBkm@SliTeX@simple}
%    \begin{macrocode}
\def\HoLogoBkm@SliTeX@simple#1{SliTeX}
%    \end{macrocode}
%    \end{macro}
%    \begin{macro}{\HoLogoHtml@SliTeX@simple}
%    \begin{macrocode}
\let\HoLogoHtml@SliTeX@simple\HoLogo@SliTeX@simple
%    \end{macrocode}
%    \end{macro}
%
%    \begin{macro}{\HoLogo@SliTeX@narrow}
%    \begin{macrocode}
\def\HoLogo@SliTeX@narrow#1{%
  \HoLogoFont@font{SliTeX}{rm}{%
    \ltx@mbox{%
      S%
      \kern-.06em%
      \HoLogoFont@font{SliTeX}{sc}{%
        l%
        \kern-.035em%
        i%
      }%
    }%
    \HOLOGO@discretionary
    \kern-.06em%
    \hologo{TeX}%
  }%
}
%    \end{macrocode}
%    \end{macro}
%    \begin{macro}{\HoLogoBkm@SliTeX@narrow}
%    \begin{macrocode}
\def\HoLogoBkm@SliTeX@narrow#1{SliTeX}
%    \end{macrocode}
%    \end{macro}
%    \begin{macro}{\HoLogoHtml@SliTeX@narrow}
%    \begin{macrocode}
\def\HoLogoHtml@SliTeX@narrow#1{%
  \HoLogoCss@SliTeX@narrow
  \HOLOGO@Span{SliTeX-narrow}{%
    \HoLogoFont@font{SliTeX}{rm}{%
      S%
        \HOLOGO@Span{l}{l}%
        \HOLOGO@Span{i}{i}%
      \hologo{TeX}%
    }%
  }%
}
%    \end{macrocode}
%    \end{macro}
%    \begin{macro}{\HoLogoCss@SliTeX@narrow}
%    \begin{macrocode}
\def\HoLogoCss@SliTeX@narrow{%
  \Css{%
    span.HoLogo-SliTeX-narrow span.HoLogo-l{%
      margin-left:-.06em;%
      margin-right:-.035em;%
      font-variant:small-caps;%
    }%
  }%
  \Css{%
    span.HoLogo-SliTeX-narrow span.HoLogo-i{%
      margin-right:-.06em;%
      font-variant:small-caps;%
    }%
  }%
  \global\let\HoLogoCss@SliTeX@narrow\relax
}
%    \end{macrocode}
%    \end{macro}
%
% \paragraph{Macro set completion.}
%
%    \begin{macro}{\HoLogo@SLiTeX@simple}
%    \begin{macrocode}
\def\HoLogo@SLiTeX@simple{\HoLogo@SliTeX@simple}
%    \end{macrocode}
%    \end{macro}
%    \begin{macro}{\HoLogoBkm@SLiTeX@simple}
%    \begin{macrocode}
\def\HoLogoBkm@SLiTeX@simple{\HoLogoBkm@SliTeX@simple}
%    \end{macrocode}
%    \end{macro}
%    \begin{macro}{\HoLogoHtml@SLiTeX@simple}
%    \begin{macrocode}
\def\HoLogoHtml@SLiTeX@simple{\HoLogoHtml@SliTeX@simple}
%    \end{macrocode}
%    \end{macro}
%
%    \begin{macro}{\HoLogo@SLiTeX@narrow}
%    \begin{macrocode}
\def\HoLogo@SLiTeX@narrow{\HoLogo@SliTeX@narrow}
%    \end{macrocode}
%    \end{macro}
%    \begin{macro}{\HoLogoBkm@SLiTeX@narrow}
%    \begin{macrocode}
\def\HoLogoBkm@SLiTeX@narrow{\HoLogoBkm@SliTeX@narrow}
%    \end{macrocode}
%    \end{macro}
%    \begin{macro}{\HoLogoHtml@SLiTeX@narrow}
%    \begin{macrocode}
\def\HoLogoHtml@SLiTeX@narrow{\HoLogoHtml@SliTeX@narrow}
%    \end{macrocode}
%    \end{macro}
%
%    \begin{macro}{\HoLogo@SliTeX@lift}
%    \begin{macrocode}
\def\HoLogo@SliTeX@lift{\HoLogo@SLiTeX@lift}
%    \end{macrocode}
%    \end{macro}
%    \begin{macro}{\HoLogoBkm@SliTeX@lift}
%    \begin{macrocode}
\def\HoLogoBkm@SliTeX@lift{\HoLogoBkm@SLiTeX@lift}
%    \end{macrocode}
%    \end{macro}
%    \begin{macro}{\HoLogoHtml@SliTeX@lift}
%    \begin{macrocode}
\def\HoLogoHtml@SliTeX@lift{\HoLogoHtml@SLiTeX@lift}
%    \end{macrocode}
%    \end{macro}
%
% \paragraph{Defaults.}
%
%    \begin{macro}{\HoLogo@SLiTeX}
%    \begin{macrocode}
\def\HoLogo@SLiTeX{\HoLogo@SLiTeX@lift}
%    \end{macrocode}
%    \end{macro}
%    \begin{macro}{\HoLogoBkm@SLiTeX}
%    \begin{macrocode}
\def\HoLogoBkm@SLiTeX{\HoLogoBkm@SLiTeX@lift}
%    \end{macrocode}
%    \end{macro}
%    \begin{macro}{\HoLogoHtml@SLiTeX}
%    \begin{macrocode}
\def\HoLogoHtml@SLiTeX{\HoLogoHtml@SLiTeX@lift}
%    \end{macrocode}
%    \end{macro}
%
%    \begin{macro}{\HoLogo@SliTeX}
%    \begin{macrocode}
\def\HoLogo@SliTeX{\HoLogo@SliTeX@narrow}
%    \end{macrocode}
%    \end{macro}
%    \begin{macro}{\HoLogoBkm@SliTeX}
%    \begin{macrocode}
\def\HoLogoBkm@SliTeX{\HoLogoBkm@SliTeX@narrow}
%    \end{macrocode}
%    \end{macro}
%    \begin{macro}{\HoLogoHtml@SliTeX}
%    \begin{macrocode}
\def\HoLogoHtml@SliTeX{\HoLogoHtml@SliTeX@narrow}
%    \end{macrocode}
%    \end{macro}
%
% \subsubsection{\hologo{LuaTeX}}
%
%    \begin{macro}{\HoLogo@LuaTeX}
%    The kerning is an idea of Hans Hagen, see mailing list
%    `luatex at tug dot org' in March 2010.
%    \begin{macrocode}
\def\HoLogo@LuaTeX#1{%
  \HOLOGO@mbox{%
    Lua%
    \HOLOGO@NegativeKerning{aT,oT,To}%
    \hologo{TeX}%
  }%
}
%    \end{macrocode}
%    \end{macro}
%    \begin{macro}{\HoLogoHtml@LuaTeX}
%    \begin{macrocode}
\let\HoLogoHtml@LuaTeX\HoLogo@LuaTeX
%    \end{macrocode}
%    \end{macro}
%
% \subsubsection{\hologo{LuaLaTeX}}
%
%    \begin{macro}{\HoLogo@LuaLaTeX}
%    \begin{macrocode}
\def\HoLogo@LuaLaTeX#1{%
  \HOLOGO@mbox{%
    Lua%
    \hologo{LaTeX}%
  }%
}
%    \end{macrocode}
%    \end{macro}
%    \begin{macro}{\HoLogoHtml@LuaLaTeX}
%    \begin{macrocode}
\let\HoLogoHtml@LuaLaTeX\HoLogo@LuaLaTeX
%    \end{macrocode}
%    \end{macro}
%
% \subsubsection{\hologo{XeTeX}, \hologo{XeLaTeX}}
%
%    \begin{macro}{\HOLOGO@IfCharExists}
%    \begin{macrocode}
\ifluatex
  \ifnum\luatexversion<36 %
  \else
    \def\HOLOGO@IfCharExists#1{%
      \ifnum
        \directlua{%
           if luaotfload and luaotfload.aux then
             if luaotfload.aux.font_has_glyph(%
                    font.current(), \number#1) then % 	 
	       tex.print("1") % 	 
	     end % 	 
	   elseif font and font.fonts and font.current then %
            local f = font.fonts[font.current()]%
            if f.characters and f.characters[\number#1] then %
              tex.print("1")%
            end %
          end%
        }0=\ltx@zero
        \expandafter\ltx@secondoftwo
      \else
        \expandafter\ltx@firstoftwo
      \fi
    }%
  \fi
\fi
\ltx@IfUndefined{HOLOGO@IfCharExists}{%
  \def\HOLOGO@@IfCharExists#1{%
    \begingroup
      \tracinglostchars=\ltx@zero
      \setbox\ltx@zero=\hbox{%
        \kern7sp\char#1\relax
        \ifnum\lastkern>\ltx@zero
          \expandafter\aftergroup\csname iffalse\endcsname
        \else
          \expandafter\aftergroup\csname iftrue\endcsname
        \fi
      }%
      % \if{true|false} from \aftergroup
      \endgroup
      \expandafter\ltx@firstoftwo
    \else
      \endgroup
      \expandafter\ltx@secondoftwo
    \fi
  }%
  \ifxetex
    \ltx@IfUndefined{XeTeXfonttype}{}{%
      \ltx@IfUndefined{XeTeXcharglyph}{}{%
        \def\HOLOGO@IfCharExists#1{%
          \ifnum\XeTeXfonttype\font>\ltx@zero
            \expandafter\ltx@firstofthree
          \else
            \expandafter\ltx@gobble
          \fi
          {%
            \ifnum\XeTeXcharglyph#1>\ltx@zero
              \expandafter\ltx@firstoftwo
            \else
              \expandafter\ltx@secondoftwo
            \fi
          }%
          \HOLOGO@@IfCharExists{#1}%
        }%
      }%
    }%
  \fi
}{}
\ltx@ifundefined{HOLOGO@IfCharExists}{%
  \ifnum64=`\^^^^0040\relax % test for big chars of LuaTeX/XeTeX
    \let\HOLOGO@IfCharExists\HOLOGO@@IfCharExists
  \else
    \def\HOLOGO@IfCharExists#1{%
      \ifnum#1>255 %
        \expandafter\ltx@fourthoffour
      \fi
      \HOLOGO@@IfCharExists{#1}%
    }%
  \fi
}{}
%    \end{macrocode}
%    \end{macro}
%
%    \begin{macro}{\HoLogo@Xe}
%    Source: package \xpackage{dtklogos}
%    \begin{macrocode}
\def\HoLogo@Xe#1{%
  X%
  \kern-.1em\relax
  \HOLOGO@IfCharExists{"018E}{%
    \lower.5ex\hbox{\char"018E}%
  }{%
    \chardef\HOLOGO@choice=\ltx@zero
    \ifdim\fontdimen\ltx@one\font>0pt %
      \ltx@IfUndefined{rotatebox}{%
        \ltx@IfUndefined{pgftext}{%
          \ltx@IfUndefined{psscalebox}{%
            \ltx@IfUndefined{HOLOGO@ScaleBox@\hologoDriver}{%
            }{%
              \chardef\HOLOGO@choice=4 %
            }%
          }{%
            \chardef\HOLOGO@choice=3 %
          }%
        }{%
          \chardef\HOLOGO@choice=2 %
        }%
      }{%
        \chardef\HOLOGO@choice=1 %
      }%
      \ifcase\HOLOGO@choice
        \HOLOGO@WarningUnsupportedDriver{Xe}%
        e%
      \or % 1: \rotatebox
        \begingroup
          \setbox\ltx@zero\hbox{\rotatebox{180}{E}}%
          \ltx@LocDimenA=\dp\ltx@zero
          \advance\ltx@LocDimenA by -.5ex\relax
          \raise\ltx@LocDimenA\box\ltx@zero
        \endgroup
      \or % 2: \pgftext
        \lower.5ex\hbox{%
          \pgfpicture
            \pgftext[rotate=180]{E}%
          \endpgfpicture
        }%
      \or % 3: \psscalebox
        \begingroup
          \setbox\ltx@zero\hbox{\psscalebox{-1 -1}{E}}%
          \ltx@LocDimenA=\dp\ltx@zero
          \advance\ltx@LocDimenA by -.5ex\relax
          \raise\ltx@LocDimenA\box\ltx@zero
        \endgroup
      \or % 4: \HOLOGO@PointReflectBox
        \lower.5ex\hbox{\HOLOGO@PointReflectBox{E}}%
      \else
        \@PackageError{hologo}{Internal error (choice/it}\@ehc
      \fi
    \else
      \ltx@IfUndefined{reflectbox}{%
        \ltx@IfUndefined{pgftext}{%
          \ltx@IfUndefined{psscalebox}{%
            \ltx@IfUndefined{HOLOGO@ScaleBox@\hologoDriver}{%
            }{%
              \chardef\HOLOGO@choice=4 %
            }%
          }{%
            \chardef\HOLOGO@choice=3 %
          }%
        }{%
          \chardef\HOLOGO@choice=2 %
        }%
      }{%
        \chardef\HOLOGO@choice=1 %
      }%
      \ifcase\HOLOGO@choice
        \HOLOGO@WarningUnsupportedDriver{Xe}%
        e%
      \or % 1: reflectbox
        \lower.5ex\hbox{%
          \reflectbox{E}%
        }%
      \or % 2: \pgftext
        \lower.5ex\hbox{%
          \pgfpicture
            \pgftransformxscale{-1}%
            \pgftext{E}%
          \endpgfpicture
        }%
      \or % 3: \psscalebox
        \lower.5ex\hbox{%
          \psscalebox{-1 1}{E}%
        }%
      \or % 4: \HOLOGO@Reflectbox
        \lower.5ex\hbox{%
          \HOLOGO@ReflectBox{E}%
        }%
      \else
        \@PackageError{hologo}{Internal error (choice/up)}\@ehc
      \fi
    \fi
  }%
}
%    \end{macrocode}
%    \end{macro}
%    \begin{macro}{\HoLogoHtml@Xe}
%    \begin{macrocode}
\def\HoLogoHtml@Xe#1{%
  \HoLogoCss@Xe
  \HOLOGO@Span{Xe}{%
    X%
    \HOLOGO@Span{e}{%
      \HCode{&\ltx@hashchar x018e;}%
    }%
  }%
}
%    \end{macrocode}
%    \end{macro}
%    \begin{macro}{\HoLogoCss@Xe}
%    \begin{macrocode}
\def\HoLogoCss@Xe{%
  \Css{%
    span.HoLogo-Xe span.HoLogo-e{%
      position:relative;%
      top:.5ex;%
      left-margin:-.1em;%
    }%
  }%
  \global\let\HoLogoCss@Xe\relax
}
%    \end{macrocode}
%    \end{macro}
%
%    \begin{macro}{\HoLogo@XeTeX}
%    \begin{macrocode}
\def\HoLogo@XeTeX#1{%
  \hologo{Xe}%
  \kern-.15em\relax
  \hologo{TeX}%
}
%    \end{macrocode}
%    \end{macro}
%
%    \begin{macro}{\HoLogoHtml@XeTeX}
%    \begin{macrocode}
\def\HoLogoHtml@XeTeX#1{%
  \HoLogoCss@XeTeX
  \HOLOGO@Span{XeTeX}{%
    \hologo{Xe}%
    \hologo{TeX}%
  }%
}
%    \end{macrocode}
%    \end{macro}
%    \begin{macro}{\HoLogoCss@XeTeX}
%    \begin{macrocode}
\def\HoLogoCss@XeTeX{%
  \Css{%
    span.HoLogo-XeTeX span.HoLogo-TeX{%
      margin-left:-.15em;%
    }%
  }%
  \global\let\HoLogoCss@XeTeX\relax
}
%    \end{macrocode}
%    \end{macro}
%
%    \begin{macro}{\HoLogo@XeLaTeX}
%    \begin{macrocode}
\def\HoLogo@XeLaTeX#1{%
  \hologo{Xe}%
  \kern-.13em%
  \hologo{LaTeX}%
}
%    \end{macrocode}
%    \end{macro}
%    \begin{macro}{\HoLogoHtml@XeLaTeX}
%    \begin{macrocode}
\def\HoLogoHtml@XeLaTeX#1{%
  \HoLogoCss@XeLaTeX
  \HOLOGO@Span{XeLaTeX}{%
    \hologo{Xe}%
    \hologo{LaTeX}%
  }%
}
%    \end{macrocode}
%    \end{macro}
%    \begin{macro}{\HoLogoCss@XeLaTeX}
%    \begin{macrocode}
\def\HoLogoCss@XeLaTeX{%
  \Css{%
    span.HoLogo-XeLaTeX span.HoLogo-Xe{%
      margin-right:-.13em;%
    }%
  }%
  \global\let\HoLogoCss@XeLaTeX\relax
}
%    \end{macrocode}
%    \end{macro}
%
% \subsubsection{\hologo{pdfTeX}, \hologo{pdfLaTeX}}
%
%    \begin{macro}{\HoLogo@pdfTeX}
%    \begin{macrocode}
\def\HoLogo@pdfTeX#1{%
  \HOLOGO@mbox{%
    #1{p}{P}df\hologo{TeX}%
  }%
}
%    \end{macrocode}
%    \end{macro}
%    \begin{macro}{\HoLogoCs@pdfTeX}
%    \begin{macrocode}
\def\HoLogoCs@pdfTeX#1{#1{p}{P}dfTeX}
%    \end{macrocode}
%    \end{macro}
%    \begin{macro}{\HoLogoBkm@pdfTeX}
%    \begin{macrocode}
\def\HoLogoBkm@pdfTeX#1{%
  #1{p}{P}df\hologo{TeX}%
}
%    \end{macrocode}
%    \end{macro}
%    \begin{macro}{\HoLogoHtml@pdfTeX}
%    \begin{macrocode}
\let\HoLogoHtml@pdfTeX\HoLogo@pdfTeX
%    \end{macrocode}
%    \end{macro}
%
%    \begin{macro}{\HoLogo@pdfLaTeX}
%    \begin{macrocode}
\def\HoLogo@pdfLaTeX#1{%
  \HOLOGO@mbox{%
    #1{p}{P}df\hologo{LaTeX}%
  }%
}
%    \end{macrocode}
%    \end{macro}
%    \begin{macro}{\HoLogoCs@pdfLaTeX}
%    \begin{macrocode}
\def\HoLogoCs@pdfLaTeX#1{#1{p}{P}dfLaTeX}
%    \end{macrocode}
%    \end{macro}
%    \begin{macro}{\HoLogoBkm@pdfLaTeX}
%    \begin{macrocode}
\def\HoLogoBkm@pdfLaTeX#1{%
  #1{p}{P}df\hologo{LaTeX}%
}
%    \end{macrocode}
%    \end{macro}
%    \begin{macro}{\HoLogoHtml@pdfLaTeX}
%    \begin{macrocode}
\let\HoLogoHtml@pdfLaTeX\HoLogo@pdfLaTeX
%    \end{macrocode}
%    \end{macro}
%
% \subsubsection{\hologo{VTeX}}
%
%    \begin{macro}{\HoLogo@VTeX}
%    \begin{macrocode}
\def\HoLogo@VTeX#1{%
  \HOLOGO@mbox{%
    V\hologo{TeX}%
  }%
}
%    \end{macrocode}
%    \end{macro}
%    \begin{macro}{\HoLogoHtml@VTeX}
%    \begin{macrocode}
\let\HoLogoHtml@VTeX\HoLogo@VTeX
%    \end{macrocode}
%    \end{macro}
%
% \subsubsection{\hologo{AmS}, \dots}
%
%    Source: class \xclass{amsdtx}
%
%    \begin{macro}{\HoLogo@AmS}
%    \begin{macrocode}
\def\HoLogo@AmS#1{%
  \HoLogoFont@font{AmS}{sy}{%
    A%
    \kern-.1667em%
    \lower.5ex\hbox{M}%
    \kern-.125em%
    S%
  }%
}
%    \end{macrocode}
%    \end{macro}
%    \begin{macro}{\HoLogoBkm@AmS}
%    \begin{macrocode}
\def\HoLogoBkm@AmS#1{AmS}
%    \end{macrocode}
%    \end{macro}
%    \begin{macro}{\HoLogoHtml@AmS}
%    \begin{macrocode}
\def\HoLogoHtml@AmS#1{%
  \HoLogoCss@AmS
%  \HoLogoFont@font{AmS}{sy}{%
    \HOLOGO@Span{AmS}{%
      A%
      \HOLOGO@Span{M}{M}%
      S%
    }%
%   }%
}
%    \end{macrocode}
%    \end{macro}
%    \begin{macro}{\HoLogoCss@AmS}
%    \begin{macrocode}
\def\HoLogoCss@AmS{%
  \Css{%
    span.HoLogo-AmS span.HoLogo-M{%
      position:relative;%
      top:.5ex;%
      margin-left:-.1667em;%
      margin-right:-.125em;%
      text-decoration:none;%
    }%
  }%
  \global\let\HoLogoCss@AmS\relax
}
%    \end{macrocode}
%    \end{macro}
%
%    \begin{macro}{\HoLogo@AmSTeX}
%    \begin{macrocode}
\def\HoLogo@AmSTeX#1{%
  \hologo{AmS}%
  \HOLOGO@hyphen
  \hologo{TeX}%
}
%    \end{macrocode}
%    \end{macro}
%    \begin{macro}{\HoLogoBkm@AmSTeX}
%    \begin{macrocode}
\def\HoLogoBkm@AmSTeX#1{AmS-TeX}%
%    \end{macrocode}
%    \end{macro}
%    \begin{macro}{\HoLogoHtml@AmSTeX}
%    \begin{macrocode}
\let\HoLogoHtml@AmSTeX\HoLogo@AmSTeX
%    \end{macrocode}
%    \end{macro}
%
%    \begin{macro}{\HoLogo@AmSLaTeX}
%    \begin{macrocode}
\def\HoLogo@AmSLaTeX#1{%
  \hologo{AmS}%
  \HOLOGO@hyphen
  \hologo{LaTeX}%
}
%    \end{macrocode}
%    \end{macro}
%    \begin{macro}{\HoLogoBkm@AmSLaTeX}
%    \begin{macrocode}
\def\HoLogoBkm@AmSLaTeX#1{AmS-LaTeX}%
%    \end{macrocode}
%    \end{macro}
%    \begin{macro}{\HoLogoHtml@AmSLaTeX}
%    \begin{macrocode}
\let\HoLogoHtml@AmSLaTeX\HoLogo@AmSLaTeX
%    \end{macrocode}
%    \end{macro}
%
% \subsubsection{\hologo{BibTeX}}
%
%    \begin{macro}{\HoLogo@BibTeX@sc}
%    A definition of \hologo{BibTeX} is provided in
%    the documentation source for the manual of \hologo{BibTeX}
%    \cite{btxdoc}.
%\begin{quote}
%\begin{verbatim}
%\def\BibTeX{%
%  {%
%    \rm
%    B%
%    \kern-.05em%
%    {%
%      \sc
%      i%
%      \kern-.025em %
%      b%
%    }%
%    \kern-.08em
%    T%
%    \kern-.1667em%
%    \lower.7ex\hbox{E}%
%    \kern-.125em%
%    X%
%  }%
%}
%\end{verbatim}
%\end{quote}
%    \begin{macrocode}
\def\HoLogo@BibTeX@sc#1{%
  B%
  \kern-.05em%
  \HoLogoFont@font{BibTeX}{sc}{%
    i%
    \kern-.025em%
    b%
  }%
  \HOLOGO@discretionary
  \kern-.08em%
  \hologo{TeX}%
}
%    \end{macrocode}
%    \end{macro}
%    \begin{macro}{\HoLogoHtml@BibTeX@sc}
%    \begin{macrocode}
\def\HoLogoHtml@BibTeX@sc#1{%
  \HoLogoCss@BibTeX@sc
  \HOLOGO@Span{BibTeX-sc}{%
    B%
    \HOLOGO@Span{i}{i}%
    \HOLOGO@Span{b}{b}%
    \hologo{TeX}%
  }%
}
%    \end{macrocode}
%    \end{macro}
%    \begin{macro}{\HoLogoCss@BibTeX@sc}
%    \begin{macrocode}
\def\HoLogoCss@BibTeX@sc{%
  \Css{%
    span.HoLogo-BibTeX-sc span.HoLogo-i{%
      margin-left:-.05em;%
      margin-right:-.025em;%
      font-variant:small-caps;%
    }%
  }%
  \Css{%
    span.HoLogo-BibTeX-sc span.HoLogo-b{%
      margin-right:-.08em;%
      font-variant:small-caps;%
    }%
  }%
  \global\let\HoLogoCss@BibTeX@sc\relax
}
%    \end{macrocode}
%    \end{macro}
%
%    \begin{macro}{\HoLogo@BibTeX@sf}
%    Variant \xoption{sf} avoids trouble with unavailable
%    small caps fonts (e.g., bold versions of Computer Modern or
%    Latin Modern). The definition is taken from
%    package \xpackage{dtklogos} \cite{dtklogos}.
%\begin{quote}
%\begin{verbatim}
%\DeclareRobustCommand{\BibTeX}{%
%  B%
%  \kern-.05em%
%  \hbox{%
%    $\m@th$% %% force math size calculations
%    \csname S@\f@size\endcsname
%    \fontsize\sf@size\z@
%    \math@fontsfalse
%    \selectfont
%    I%
%    \kern-.025em%
%    B
%  }%
%  \kern-.08em%
%  \-%
%  \TeX
%}
%\end{verbatim}
%\end{quote}
%    \begin{macrocode}
\def\HoLogo@BibTeX@sf#1{%
  B%
  \kern-.05em%
  \HoLogoFont@font{BibTeX}{bibsf}{%
    I%
    \kern-.025em%
    B%
  }%
  \HOLOGO@discretionary
  \kern-.08em%
  \hologo{TeX}%
}
%    \end{macrocode}
%    \end{macro}
%    \begin{macro}{\HoLogoHtml@BibTeX@sf}
%    \begin{macrocode}
\def\HoLogoHtml@BibTeX@sf#1{%
  \HoLogoCss@BibTeX@sf
  \HOLOGO@Span{BibTeX-sf}{%
    B%
    \HoLogoFont@font{BibTeX}{bibsf}{%
      \HOLOGO@Span{i}{I}%
      B%
    }%
    \hologo{TeX}%
  }%
}
%    \end{macrocode}
%    \end{macro}
%    \begin{macro}{\HoLogoCss@BibTeX@sf}
%    \begin{macrocode}
\def\HoLogoCss@BibTeX@sf{%
  \Css{%
    span.HoLogo-BibTeX-sf span.HoLogo-i{%
      margin-left:-.05em;%
      margin-right:-.025em;%
    }%
  }%
  \Css{%
    span.HoLogo-BibTeX-sf span.HoLogo-TeX{%
      margin-left:-.08em;%
    }%
  }%
  \global\let\HoLogoCss@BibTeX@sf\relax
}
%    \end{macrocode}
%    \end{macro}
%
%    \begin{macro}{\HoLogo@BibTeX}
%    \begin{macrocode}
\def\HoLogo@BibTeX{\HoLogo@BibTeX@sf}
%    \end{macrocode}
%    \end{macro}
%    \begin{macro}{\HoLogoHtml@BibTeX}
%    \begin{macrocode}
\def\HoLogoHtml@BibTeX{\HoLogoHtml@BibTeX@sf}
%    \end{macrocode}
%    \end{macro}
%
% \subsubsection{\hologo{BibTeX8}}
%
%    \begin{macro}{\HoLogo@BibTeX8}
%    \begin{macrocode}
\expandafter\def\csname HoLogo@BibTeX8\endcsname#1{%
  \hologo{BibTeX}%
  8%
}
%    \end{macrocode}
%    \end{macro}
%
%    \begin{macro}{\HoLogoBkm@BibTeX8}
%    \begin{macrocode}
\expandafter\def\csname HoLogoBkm@BibTeX8\endcsname#1{%
  \hologo{BibTeX}%
  8%
}
%    \end{macrocode}
%    \end{macro}
%    \begin{macro}{\HoLogoHtml@BibTeX8}
%    \begin{macrocode}
\expandafter
\let\csname HoLogoHtml@BibTeX8\expandafter\endcsname
\csname HoLogo@BibTeX8\endcsname
%    \end{macrocode}
%    \end{macro}
%
% \subsubsection{\hologo{ConTeXt}}
%
%    \begin{macro}{\HoLogo@ConTeXt@simple}
%    \begin{macrocode}
\def\HoLogo@ConTeXt@simple#1{%
  \HOLOGO@mbox{Con}%
  \HOLOGO@discretionary
  \HOLOGO@mbox{\hologo{TeX}t}%
}
%    \end{macrocode}
%    \end{macro}
%    \begin{macro}{\HoLogoHtml@ConTeXt@simple}
%    \begin{macrocode}
\let\HoLogoHtml@ConTeXt@simple\HoLogo@ConTeXt@simple
%    \end{macrocode}
%    \end{macro}
%
%    \begin{macro}{\HoLogo@ConTeXt@narrow}
%    This definition of logo \hologo{ConTeXt} with variant \xoption{narrow}
%    comes from TUGboat's class \xclass{ltugboat} (version 2010/11/15 v2.8).
%    \begin{macrocode}
\def\HoLogo@ConTeXt@narrow#1{%
  \HOLOGO@mbox{C\kern-.0333emon}%
  \HOLOGO@discretionary
  \kern-.0667em%
  \HOLOGO@mbox{\hologo{TeX}\kern-.0333emt}%
}
%    \end{macrocode}
%    \end{macro}
%    \begin{macro}{\HoLogoHtml@ConTeXt@narrow}
%    \begin{macrocode}
\def\HoLogoHtml@ConTeXt@narrow#1{%
  \HoLogoCss@ConTeXt@narrow
  \HOLOGO@Span{ConTeXt-narrow}{%
    \HOLOGO@Span{C}{C}%
    on%
    \hologo{TeX}%
    t%
  }%
}
%    \end{macrocode}
%    \end{macro}
%    \begin{macro}{\HoLogoCss@ConTeXt@narrow}
%    \begin{macrocode}
\def\HoLogoCss@ConTeXt@narrow{%
  \Css{%
    span.HoLogo-ConTeXt-narrow span.HoLogo-C{%
      margin-left:-.0333em;%
    }%
  }%
  \Css{%
    span.HoLogo-ConTeXt-narrow span.HoLogo-TeX{%
      margin-left:-.0667em;%
      margin-right:-.0333em;%
    }%
  }%
  \global\let\HoLogoCss@ConTeXt@narrow\relax
}
%    \end{macrocode}
%    \end{macro}
%
%    \begin{macro}{\HoLogo@ConTeXt}
%    \begin{macrocode}
\def\HoLogo@ConTeXt{\HoLogo@ConTeXt@narrow}
%    \end{macrocode}
%    \end{macro}
%    \begin{macro}{\HoLogoHtml@ConTeXt}
%    \begin{macrocode}
\def\HoLogoHtml@ConTeXt{\HoLogoHtml@ConTeXt@narrow}
%    \end{macrocode}
%    \end{macro}
%
% \subsubsection{\hologo{emTeX}}
%
%    \begin{macro}{\HoLogo@emTeX}
%    \begin{macrocode}
\def\HoLogo@emTeX#1{%
  \HOLOGO@mbox{#1{e}{E}m}%
  \HOLOGO@discretionary
  \hologo{TeX}%
}
%    \end{macrocode}
%    \end{macro}
%    \begin{macro}{\HoLogoCs@emTeX}
%    \begin{macrocode}
\def\HoLogoCs@emTeX#1{#1{e}{E}mTeX}%
%    \end{macrocode}
%    \end{macro}
%    \begin{macro}{\HoLogoBkm@emTeX}
%    \begin{macrocode}
\def\HoLogoBkm@emTeX#1{%
  #1{e}{E}m\hologo{TeX}%
}
%    \end{macrocode}
%    \end{macro}
%    \begin{macro}{\HoLogoHtml@emTeX}
%    \begin{macrocode}
\let\HoLogoHtml@emTeX\HoLogo@emTeX
%    \end{macrocode}
%    \end{macro}
%
% \subsubsection{\hologo{ExTeX}}
%
%    \begin{macro}{\HoLogo@ExTeX}
%    The definition is taken from the FAQ of the
%    project \hologo{ExTeX}
%    \cite{ExTeX-FAQ}.
%\begin{quote}
%\begin{verbatim}
%\def\ExTeX{%
%  \textrm{% Logo always with serifs
%    \ensuremath{%
%      \textstyle
%      \varepsilon_{%
%        \kern-0.15em%
%        \mathcal{X}%
%      }%
%    }%
%    \kern-.15em%
%    \TeX
%  }%
%}
%\end{verbatim}
%\end{quote}
%    \begin{macrocode}
\def\HoLogo@ExTeX#1{%
  \HoLogoFont@font{ExTeX}{rm}{%
    \ltx@mbox{%
      \HOLOGO@MathSetup
      $%
        \textstyle
        \varepsilon_{%
          \kern-0.15em%
          \HoLogoFont@font{ExTeX}{sy}{X}%
        }%
      $%
    }%
    \HOLOGO@discretionary
    \kern-.15em%
    \hologo{TeX}%
  }%
}
%    \end{macrocode}
%    \end{macro}
%    \begin{macro}{\HoLogoHtml@ExTeX}
%    \begin{macrocode}
\def\HoLogoHtml@ExTeX#1{%
  \HoLogoCss@ExTeX
  \HoLogoFont@font{ExTeX}{rm}{%
    \HOLOGO@Span{ExTeX}{%
      \ltx@mbox{%
        \HOLOGO@MathSetup
        $\textstyle\varepsilon$%
        \HOLOGO@Span{X}{$\textstyle\chi$}%
        \hologo{TeX}%
      }%
    }%
  }%
}
%    \end{macrocode}
%    \end{macro}
%    \begin{macro}{\HoLogoBkm@ExTeX}
%    \begin{macrocode}
\def\HoLogoBkm@ExTeX#1{%
  \HOLOGO@PdfdocUnicode{#1{e}{E}x}{\textepsilon\textchi}%
  \hologo{TeX}%
}
%    \end{macrocode}
%    \end{macro}
%    \begin{macro}{\HoLogoCss@ExTeX}
%    \begin{macrocode}
\def\HoLogoCss@ExTeX{%
  \Css{%
    span.HoLogo-ExTeX{%
      font-family:serif;%
    }%
  }%
  \Css{%
    span.HoLogo-ExTeX span.HoLogo-TeX{%
      margin-left:-.15em;%
    }%
  }%
  \global\let\HoLogoCss@ExTeX\relax
}
%    \end{macrocode}
%    \end{macro}
%
% \subsubsection{\hologo{MiKTeX}}
%
%    \begin{macro}{\HoLogo@MiKTeX}
%    \begin{macrocode}
\def\HoLogo@MiKTeX#1{%
  \HOLOGO@mbox{MiK}%
  \HOLOGO@discretionary
  \hologo{TeX}%
}
%    \end{macrocode}
%    \end{macro}
%    \begin{macro}{\HoLogoHtml@MiKTeX}
%    \begin{macrocode}
\let\HoLogoHtml@MiKTeX\HoLogo@MiKTeX
%    \end{macrocode}
%    \end{macro}
%
% \subsubsection{\hologo{OzTeX} and friends}
%
%    Source: \hologo{OzTeX} FAQ \cite{OzTeX}:
%    \begin{quote}
%      |\def\OzTeX{O\kern-.03em z\kern-.15em\TeX}|\\
%      (There is no kerning in OzMF, OzMP and OzTtH.)
%    \end{quote}
%
%    \begin{macro}{\HoLogo@OzTeX}
%    \begin{macrocode}
\def\HoLogo@OzTeX#1{%
  O%
  \kern-.03em %
  z%
  \kern-.15em %
  \hologo{TeX}%
}
%    \end{macrocode}
%    \end{macro}
%    \begin{macro}{\HoLogoHtml@OzTeX}
%    \begin{macrocode}
\def\HoLogoHtml@OzTeX#1{%
  \HoLogoCss@OzTeX
  \HOLOGO@Span{OzTeX}{%
    O%
    \HOLOGO@Span{z}{z}%
    \hologo{TeX}%
  }%
}
%    \end{macrocode}
%    \end{macro}
%    \begin{macro}{\HoLogoCss@OzTeX}
%    \begin{macrocode}
\def\HoLogoCss@OzTeX{%
  \Css{%
    span.HoLogo-OzTeX span.HoLogo-z{%
      margin-left:-.03em;%
      margin-right:-.15em;%
    }%
  }%
  \global\let\HoLogoCss@OzTeX\relax
}
%    \end{macrocode}
%    \end{macro}
%
%    \begin{macro}{\HoLogo@OzMF}
%    \begin{macrocode}
\def\HoLogo@OzMF#1{%
  \HOLOGO@mbox{OzMF}%
}
%    \end{macrocode}
%    \end{macro}
%    \begin{macro}{\HoLogo@OzMP}
%    \begin{macrocode}
\def\HoLogo@OzMP#1{%
  \HOLOGO@mbox{OzMP}%
}
%    \end{macrocode}
%    \end{macro}
%    \begin{macro}{\HoLogo@OzTtH}
%    \begin{macrocode}
\def\HoLogo@OzTtH#1{%
  \HOLOGO@mbox{OzTtH}%
}
%    \end{macrocode}
%    \end{macro}
%
% \subsubsection{\hologo{PCTeX}}
%
%    \begin{macro}{\HoLogo@PCTeX}
%    \begin{macrocode}
\def\HoLogo@PCTeX#1{%
  \HOLOGO@mbox{PC}%
  \hologo{TeX}%
}
%    \end{macrocode}
%    \end{macro}
%    \begin{macro}{\HoLogoHtml@PCTeX}
%    \begin{macrocode}
\let\HoLogoHtml@PCTeX\HoLogo@PCTeX
%    \end{macrocode}
%    \end{macro}
%
% \subsubsection{\hologo{PiCTeX}}
%
%    The original definitions from \xfile{pictex.tex} \cite{PiCTeX}:
%\begin{quote}
%\begin{verbatim}
%\def\PiC{%
%  P%
%  \kern-.12em%
%  \lower.5ex\hbox{I}%
%  \kern-.075em%
%  C%
%}
%\def\PiCTeX{%
%  \PiC
%  \kern-.11em%
%  \TeX
%}
%\end{verbatim}
%\end{quote}
%
%    \begin{macro}{\HoLogo@PiC}
%    \begin{macrocode}
\def\HoLogo@PiC#1{%
  P%
  \kern-.12em%
  \lower.5ex\hbox{I}%
  \kern-.075em%
  C%
  \HOLOGO@SpaceFactor
}
%    \end{macrocode}
%    \end{macro}
%    \begin{macro}{\HoLogoHtml@PiC}
%    \begin{macrocode}
\def\HoLogoHtml@PiC#1{%
  \HoLogoCss@PiC
  \HOLOGO@Span{PiC}{%
    P%
    \HOLOGO@Span{i}{I}%
    C%
  }%
}
%    \end{macrocode}
%    \end{macro}
%    \begin{macro}{\HoLogoCss@PiC}
%    \begin{macrocode}
\def\HoLogoCss@PiC{%
  \Css{%
    span.HoLogo-PiC span.HoLogo-i{%
      position:relative;%
      top:.5ex;%
      margin-left:-.12em;%
      margin-right:-.075em;%
      text-decoration:none;%
    }%
  }%
  \global\let\HoLogoCss@PiC\relax
}
%    \end{macrocode}
%    \end{macro}
%
%    \begin{macro}{\HoLogo@PiCTeX}
%    \begin{macrocode}
\def\HoLogo@PiCTeX#1{%
  \hologo{PiC}%
  \HOLOGO@discretionary
  \kern-.11em%
  \hologo{TeX}%
}
%    \end{macrocode}
%    \end{macro}
%    \begin{macro}{\HoLogoHtml@PiCTeX}
%    \begin{macrocode}
\def\HoLogoHtml@PiCTeX#1{%
  \HoLogoCss@PiCTeX
  \HOLOGO@Span{PiCTeX}{%
    \hologo{PiC}%
    \hologo{TeX}%
  }%
}
%    \end{macrocode}
%    \end{macro}
%    \begin{macro}{\HoLogoCss@PiCTeX}
%    \begin{macrocode}
\def\HoLogoCss@PiCTeX{%
  \Css{%
    span.HoLogo-PiCTeX span.HoLogo-PiC{%
      margin-right:-.11em;%
    }%
  }%
  \global\let\HoLogoCss@PiCTeX\relax
}
%    \end{macrocode}
%    \end{macro}
%
% \subsubsection{\hologo{teTeX}}
%
%    \begin{macro}{\HoLogo@teTeX}
%    \begin{macrocode}
\def\HoLogo@teTeX#1{%
  \HOLOGO@mbox{#1{t}{T}e}%
  \HOLOGO@discretionary
  \hologo{TeX}%
}
%    \end{macrocode}
%    \end{macro}
%    \begin{macro}{\HoLogoCs@teTeX}
%    \begin{macrocode}
\def\HoLogoCs@teTeX#1{#1{t}{T}dfTeX}
%    \end{macrocode}
%    \end{macro}
%    \begin{macro}{\HoLogoBkm@teTeX}
%    \begin{macrocode}
\def\HoLogoBkm@teTeX#1{%
  #1{t}{T}e\hologo{TeX}%
}
%    \end{macrocode}
%    \end{macro}
%    \begin{macro}{\HoLogoHtml@teTeX}
%    \begin{macrocode}
\let\HoLogoHtml@teTeX\HoLogo@teTeX
%    \end{macrocode}
%    \end{macro}
%
% \subsubsection{\hologo{TeX4ht}}
%
%    \begin{macro}{\HoLogo@TeX4ht}
%    \begin{macrocode}
\expandafter\def\csname HoLogo@TeX4ht\endcsname#1{%
  \HOLOGO@mbox{\hologo{TeX}4ht}%
}
%    \end{macrocode}
%    \end{macro}
%    \begin{macro}{\HoLogoHtml@TeX4ht}
%    \begin{macrocode}
\expandafter
\let\csname HoLogoHtml@TeX4ht\expandafter\endcsname
\csname HoLogo@TeX4ht\endcsname
%    \end{macrocode}
%    \end{macro}
%
%
% \subsubsection{\hologo{SageTeX}}
%
%    \begin{macro}{\HoLogo@SageTeX}
%    \begin{macrocode}
\def\HoLogo@SageTeX#1{%
  \HOLOGO@mbox{Sage}%
  \HOLOGO@discretionary
  \HOLOGO@NegativeKerning{eT,oT,To}%
  \hologo{TeX}%
}
%    \end{macrocode}
%    \end{macro}
%    \begin{macro}{\HoLogoHtml@SageTeX}
%    \begin{macrocode}
\let\HoLogoHtml@SageTeX\HoLogo@SageTeX
%    \end{macrocode}
%    \end{macro}
%
% \subsection{\hologo{METAFONT} and friends}
%
%    \begin{macro}{\HoLogo@METAFONT}
%    \begin{macrocode}
\def\HoLogo@METAFONT#1{%
  \HoLogoFont@font{METAFONT}{logo}{%
    \HOLOGO@mbox{META}%
    \HOLOGO@discretionary
    \HOLOGO@mbox{FONT}%
  }%
}
%    \end{macrocode}
%    \end{macro}
%
%    \begin{macro}{\HoLogo@METAPOST}
%    \begin{macrocode}
\def\HoLogo@METAPOST#1{%
  \HoLogoFont@font{METAPOST}{logo}{%
    \HOLOGO@mbox{META}%
    \HOLOGO@discretionary
    \HOLOGO@mbox{POST}%
  }%
}
%    \end{macrocode}
%    \end{macro}
%
%    \begin{macro}{\HoLogo@MetaFun}
%    \begin{macrocode}
\def\HoLogo@MetaFun#1{%
  \HOLOGO@mbox{Meta}%
  \HOLOGO@discretionary
  \HOLOGO@mbox{Fun}%
}
%    \end{macrocode}
%    \end{macro}
%
%    \begin{macro}{\HoLogo@MetaPost}
%    \begin{macrocode}
\def\HoLogo@MetaPost#1{%
  \HOLOGO@mbox{Meta}%
  \HOLOGO@discretionary
  \HOLOGO@mbox{Post}%
}
%    \end{macrocode}
%    \end{macro}
%
% \subsection{Others}
%
% \subsubsection{\hologo{biber}}
%
%    \begin{macro}{\HoLogo@biber}
%    \begin{macrocode}
\def\HoLogo@biber#1{%
  \HOLOGO@mbox{#1{b}{B}i}%
  \HOLOGO@discretionary
  \HOLOGO@mbox{ber}%
}
%    \end{macrocode}
%    \end{macro}
%    \begin{macro}{\HoLogoCs@biber}
%    \begin{macrocode}
\def\HoLogoCs@biber#1{#1{b}{B}iber}
%    \end{macrocode}
%    \end{macro}
%    \begin{macro}{\HoLogoBkm@biber}
%    \begin{macrocode}
\def\HoLogoBkm@biber#1{%
  #1{b}{B}iber%
}
%    \end{macrocode}
%    \end{macro}
%    \begin{macro}{\HoLogoHtml@biber}
%    \begin{macrocode}
\let\HoLogoHtml@biber\HoLogo@biber
%    \end{macrocode}
%    \end{macro}
%
% \subsubsection{\hologo{KOMAScript}}
%
%    \begin{macro}{\HoLogo@KOMAScript}
%    The definition for \hologo{KOMAScript} is taken
%    from \hologo{KOMAScript} (\xfile{scrlogo.dtx}, reformatted) \cite{scrlogo}:
%\begin{quote}
%\begin{verbatim}
%\@ifundefined{KOMAScript}{%
%  \DeclareRobustCommand{\KOMAScript}{%
%    \textsf{%
%      K\kern.05em O\kern.05emM\kern.05em A%
%      \kern.1em-\kern.1em %
%      Script%
%    }%
%  }%
%}{}
%\end{verbatim}
%\end{quote}
%    \begin{macrocode}
\def\HoLogo@KOMAScript#1{%
  \HoLogoFont@font{KOMAScript}{sf}{%
    \HOLOGO@mbox{%
      K\kern.05em%
      O\kern.05em%
      M\kern.05em%
      A%
    }%
    \kern.1em%
    \HOLOGO@hyphen
    \kern.1em%
    \HOLOGO@mbox{Script}%
  }%
}
%    \end{macrocode}
%    \end{macro}
%    \begin{macro}{\HoLogoBkm@KOMAScript}
%    \begin{macrocode}
\def\HoLogoBkm@KOMAScript#1{%
  KOMA-Script%
}
%    \end{macrocode}
%    \end{macro}
%    \begin{macro}{\HoLogoHtml@KOMAScript}
%    \begin{macrocode}
\def\HoLogoHtml@KOMAScript#1{%
  \HoLogoCss@KOMAScript
  \HoLogoFont@font{KOMAScript}{sf}{%
    \HOLOGO@Span{KOMAScript}{%
      K%
      \HOLOGO@Span{O}{O}%
      M%
      \HOLOGO@Span{A}{A}%
      \HOLOGO@Span{hyphen}{-}%
      Script%
    }%
  }%
}
%    \end{macrocode}
%    \end{macro}
%    \begin{macro}{\HoLogoCss@KOMAScript}
%    \begin{macrocode}
\def\HoLogoCss@KOMAScript{%
  \Css{%
    span.HoLogo-KOMAScript{%
      font-family:sans-serif;%
    }%
  }%
  \Css{%
    span.HoLogo-KOMAScript span.HoLogo-O{%
      padding-left:.05em;%
      padding-right:.05em;%
    }%
  }%
  \Css{%
    span.HoLogo-KOMAScript span.HoLogo-A{%
      padding-left:.05em;%
    }%
  }%
  \Css{%
    span.HoLogo-KOMAScript span.HoLogo-hyphen{%
      padding-left:.1em;%
      padding-right:.1em;%
    }%
  }%
  \global\let\HoLogoCss@KOMAScript\relax
}
%    \end{macrocode}
%    \end{macro}
%
% \subsubsection{\hologo{LyX}}
%
%    \begin{macro}{\HoLogo@LyX}
%    The definition is taken from the documentation source files
%    of \hologo{LyX}, \xfile{Intro.lyx} \cite{LyX}:
%\begin{quote}
%\begin{verbatim}
%\def\LyX{%
%  \texorpdfstring{%
%    L\kern-.1667em\lower.25em\hbox{Y}\kern-.125emX\@%
%  }{%
%    LyX%
%  }%
%}
%\end{verbatim}
%\end{quote}
%    \begin{macrocode}
\def\HoLogo@LyX#1{%
  L%
  \kern-.1667em%
  \lower.25em\hbox{Y}%
  \kern-.125em%
  X%
  \HOLOGO@SpaceFactor
}
%    \end{macrocode}
%    \end{macro}
%    \begin{macro}{\HoLogoHtml@LyX}
%    \begin{macrocode}
\def\HoLogoHtml@LyX#1{%
  \HoLogoCss@LyX
  \HOLOGO@Span{LyX}{%
    L%
    \HOLOGO@Span{y}{Y}%
    X%
  }%
}
%    \end{macrocode}
%    \end{macro}
%    \begin{macro}{\HoLogoCss@LyX}
%    \begin{macrocode}
\def\HoLogoCss@LyX{%
  \Css{%
    span.HoLogo-LyX span.HoLogo-y{%
      position:relative;%
      top:.25em;%
      margin-left:-.1667em;%
      margin-right:-.125em;%
      text-decoration:none;%
    }%
  }%
  \global\let\HoLogoCss@LyX\relax
}
%    \end{macrocode}
%    \end{macro}
%
% \subsubsection{\hologo{NTS}}
%
%    \begin{macro}{\HoLogo@NTS}
%    Definition for \hologo{NTS} can be found in
%    package \xpackage{etex\textunderscore man} for the \hologo{eTeX} manual \cite{etexman}
%    and in package \xpackage{dtklogos} \cite{dtklogos}:
%\begin{quote}
%\begin{verbatim}
%\def\NTS{%
%  \leavevmode
%  \hbox{%
%    $%
%      \cal N%
%      \kern-0.35em%
%      \lower0.5ex\hbox{$\cal T$}%
%      \kern-0.2em%
%      S%
%    $%
%  }%
%}
%\end{verbatim}
%\end{quote}
%    \begin{macrocode}
\def\HoLogo@NTS#1{%
  \HoLogoFont@font{NTS}{sy}{%
    N\/%
    \kern-.35em%
    \lower.5ex\hbox{T\/}%
    \kern-.2em%
    S\/%
  }%
  \HOLOGO@SpaceFactor
}
%    \end{macrocode}
%    \end{macro}
%
% \subsubsection{\Hologo{TTH} (\hologo{TeX} to HTML translator)}
%
%    Source: \url{http://hutchinson.belmont.ma.us/tth/}
%    In the HTML source the second `T' is printed as subscript.
%\begin{quote}
%\begin{verbatim}
%T<sub>T</sub>H
%\end{verbatim}
%\end{quote}
%    \begin{macro}{\HoLogo@TTH}
%    \begin{macrocode}
\def\HoLogo@TTH#1{%
  \ltx@mbox{%
    T\HOLOGO@SubScript{T}H%
  }%
  \HOLOGO@SpaceFactor
}
%    \end{macrocode}
%    \end{macro}
%
%    \begin{macro}{\HoLogoHtml@TTH}
%    \begin{macrocode}
\def\HoLogoHtml@TTH#1{%
  T\HCode{<sub>}T\HCode{</sub>}H%
}
%    \end{macrocode}
%    \end{macro}
%
% \subsubsection{\Hologo{HanTheThanh}}
%
%    Partial source: Package \xpackage{dtklogos}.
%    The double accent is U+1EBF (latin small letter e with circumflex
%    and acute).
%    \begin{macro}{\HoLogo@HanTheThanh}
%    \begin{macrocode}
\def\HoLogo@HanTheThanh#1{%
  \ltx@mbox{H\`an}%
  \HOLOGO@space
  \ltx@mbox{%
    Th%
    \HOLOGO@IfCharExists{"1EBF}{%
      \char"1EBF\relax
    }{%
      \^e\hbox to 0pt{\hss\raise .5ex\hbox{\'{}}}%
    }%
  }%
  \HOLOGO@space
  \ltx@mbox{Th\`anh}%
}
%    \end{macrocode}
%    \end{macro}
%    \begin{macro}{\HoLogoBkm@HanTheThanh}
%    \begin{macrocode}
\def\HoLogoBkm@HanTheThanh#1{%
  H\`an %
  Th\HOLOGO@PdfdocUnicode{\^e}{\9036\277} %
  Th\`anh%
}
%    \end{macrocode}
%    \end{macro}
%    \begin{macro}{\HoLogoHtml@HanTheThanh}
%    \begin{macrocode}
\def\HoLogoHtml@HanTheThanh#1{%
  H\`an %
  Th\HCode{&\ltx@hashchar x1ebf;} %
  Th\`anh%
}
%    \end{macrocode}
%    \end{macro}
%
% \subsection{Driver detection}
%
%    \begin{macrocode}
\HOLOGO@IfExists\InputIfFileExists{%
  \InputIfFileExists{hologo.cfg}{}{}%
}{%
  \ltx@IfUndefined{pdf@filesize}{%
    \def\HOLOGO@InputIfExists{%
      \openin\HOLOGO@temp=hologo.cfg\relax
      \ifeof\HOLOGO@temp
        \closein\HOLOGO@temp
      \else
        \closein\HOLOGO@temp
        \begingroup
          \def\x{LaTeX2e}%
        \expandafter\endgroup
        \ifx\fmtname\x
          \input{hologo.cfg}%
        \else
          \input hologo.cfg\relax
        \fi
      \fi
    }%
    \ltx@IfUndefined{newread}{%
      \chardef\HOLOGO@temp=15 %
      \def\HOLOGO@CheckRead{%
        \ifeof\HOLOGO@temp
          \HOLOGO@InputIfExists
        \else
          \ifcase\HOLOGO@temp
            \@PackageWarningNoLine{hologo}{%
              Configuration file ignored, because\MessageBreak
              a free read register could not be found%
            }%
          \else
            \begingroup
              \count\ltx@cclv=\HOLOGO@temp
              \advance\ltx@cclv by \ltx@minusone
              \edef\x{\endgroup
                \chardef\noexpand\HOLOGO@temp=\the\count\ltx@cclv
                \relax
              }%
            \x
          \fi
        \fi
      }%
    }{%
      \csname newread\endcsname\HOLOGO@temp
      \HOLOGO@InputIfExists
    }%
  }{%
    \edef\HOLOGO@temp{\pdf@filesize{hologo.cfg}}%
    \ifx\HOLOGO@temp\ltx@empty
    \else
      \ifnum\HOLOGO@temp>0 %
        \begingroup
          \def\x{LaTeX2e}%
        \expandafter\endgroup
        \ifx\fmtname\x
          \input{hologo.cfg}%
        \else
          \input hologo.cfg\relax
        \fi
      \else
        \@PackageInfoNoLine{hologo}{%
          Empty configuration file `hologo.cfg' ignored%
        }%
      \fi
    \fi
  }%
}
%    \end{macrocode}
%
%    \begin{macrocode}
\def\HOLOGO@temp#1#2{%
  \kv@define@key{HoLogoDriver}{#1}[]{%
    \begingroup
      \def\HOLOGO@temp{##1}%
      \ltx@onelevel@sanitize\HOLOGO@temp
      \ifx\HOLOGO@temp\ltx@empty
      \else
        \@PackageError{hologo}{%
          Value (\HOLOGO@temp) not permitted for option `#1'%
        }%
        \@ehc
      \fi
    \endgroup
    \def\hologoDriver{#2}%
  }%
}%
\def\HOLOGO@@temp#1#2{%
  \ifx\kv@value\relax
    \HOLOGO@temp{#1}{#1}%
  \else
    \HOLOGO@temp{#1}{#2}%
  \fi
}%
\kv@parse@normalized{%
  pdftex,%
  luatex=pdftex,%
  dvipdfm,%
  dvipdfmx=dvipdfm,%
  dvips,%
  dvipsone=dvips,%
  xdvi=dvips,%
  xetex,%
  vtex,%
}\HOLOGO@@temp
%    \end{macrocode}
%
%    \begin{macrocode}
\kv@define@key{HoLogoDriver}{driverfallback}{%
  \def\HOLOGO@DriverFallback{#1}%
}
%    \end{macrocode}
%
%    \begin{macro}{\HOLOGO@DriverFallback}
%    \begin{macrocode}
\def\HOLOGO@DriverFallback{dvips}
%    \end{macrocode}
%    \end{macro}
%
%    \begin{macro}{\hologoDriverSetup}
%    \begin{macrocode}
\def\hologoDriverSetup{%
  \let\hologoDriver\ltx@undefined
  \HOLOGO@DriverSetup
}
%    \end{macrocode}
%    \end{macro}
%
%    \begin{macro}{\HOLOGO@DriverSetup}
%    \begin{macrocode}
\def\HOLOGO@DriverSetup#1{%
  \kvsetkeys{HoLogoDriver}{#1}%
  \HOLOGO@CheckDriver
  \ltx@ifundefined{hologoDriver}{%
    \begingroup
    \edef\x{\endgroup
      \noexpand\kvsetkeys{HoLogoDriver}{\HOLOGO@DriverFallback}%
    }\x
  }{}%
  \@PackageInfoNoLine{hologo}{Using driver `\hologoDriver'}%
}
%    \end{macrocode}
%    \end{macro}
%
%    \begin{macro}{\HOLOGO@CheckDriver}
%    \begin{macrocode}
\def\HOLOGO@CheckDriver{%
  \ifpdf
    \def\hologoDriver{pdftex}%
    \let\HOLOGO@pdfliteral\pdfliteral
    \ifluatex
      \ifx\pdfextension\@undefined\else
        \protected\def\pdfliteral{\pdfextension literal}%
        \let\HOLOGO@pdfliteral\pdfliteral
      \fi
      \ltx@IfUndefined{HOLOGO@pdfliteral}{%
        \ifnum\luatexversion<36 %
        \else
          \begingroup
            \let\HOLOGO@temp\endgroup
            \ifcase0%
                \directlua{%
                  if tex.enableprimitives then %
                    tex.enableprimitives('HOLOGO@', {'pdfliteral'})%
                  else %
                    tex.print('1')%
                  end%
                }%
                \ifx\HOLOGO@pdfliteral\@undefined 1\fi%
                \relax%
              \endgroup
              \let\HOLOGO@temp\relax
              \global\let\HOLOGO@pdfliteral\HOLOGO@pdfliteral
            \fi%
          \HOLOGO@temp
        \fi
      }{}%
    \fi
    \ltx@IfUndefined{HOLOGO@pdfliteral}{%
      \@PackageWarningNoLine{hologo}{%
        Cannot find \string\pdfliteral
      }%
    }{}%
  \else
    \ifxetex
      \def\hologoDriver{xetex}%
    \else
      \ifvtex
        \def\hologoDriver{vtex}%
      \fi
    \fi
  \fi
}
%    \end{macrocode}
%    \end{macro}
%
%    \begin{macro}{\HOLOGO@WarningUnsupportedDriver}
%    \begin{macrocode}
\def\HOLOGO@WarningUnsupportedDriver#1{%
  \@PackageWarningNoLine{hologo}{%
    Logo `#1' needs driver specific macros,\MessageBreak
    but driver `\hologoDriver' is not supported.\MessageBreak
    Use a different driver or\MessageBreak
    load package `graphics' or `pgf'%
  }%
}
%    \end{macrocode}
%    \end{macro}
%
% \subsubsection{Reflect box macros}
%
%    Skip driver part if not needed.
%    \begin{macrocode}
\ltx@IfUndefined{reflectbox}{}{%
  \ltx@IfUndefined{rotatebox}{}{%
    \HOLOGO@AtEnd
  }%
}
\ltx@IfUndefined{pgftext}{}{%
  \HOLOGO@AtEnd
}
\ltx@IfUndefined{psscalebox}{}{%
  \HOLOGO@AtEnd
}
%    \end{macrocode}
%
%    \begin{macrocode}
\def\HOLOGO@temp{LaTeX2e}
\ifx\fmtname\HOLOGO@temp
  \RequirePackage{kvoptions}[2011/06/30]%
  \ProcessKeyvalOptions{HoLogoDriver}%
\fi
\HOLOGO@DriverSetup{}
%    \end{macrocode}
%
%    \begin{macro}{\HOLOGO@ReflectBox}
%    \begin{macrocode}
\def\HOLOGO@ReflectBox#1{%
  \begingroup
    \setbox\ltx@zero\hbox{\begingroup#1\endgroup}%
    \setbox\ltx@two\hbox{%
      \kern\wd\ltx@zero
      \csname HOLOGO@ScaleBox@\hologoDriver\endcsname{-1}{1}{%
        \hbox to 0pt{\copy\ltx@zero\hss}%
      }%
    }%
    \wd\ltx@two=\wd\ltx@zero
    \box\ltx@two
  \endgroup
}
%    \end{macrocode}
%    \end{macro}
%
%    \begin{macro}{\HOLOGO@PointReflectBox}
%    \begin{macrocode}
\def\HOLOGO@PointReflectBox#1{%
  \begingroup
    \setbox\ltx@zero\hbox{\begingroup#1\endgroup}%
    \setbox\ltx@two\hbox{%
      \kern\wd\ltx@zero
      \raise\ht\ltx@zero\hbox{%
        \csname HOLOGO@ScaleBox@\hologoDriver\endcsname{-1}{-1}{%
          \hbox to 0pt{\copy\ltx@zero\hss}%
        }%
      }%
    }%
    \wd\ltx@two=\wd\ltx@zero
    \box\ltx@two
  \endgroup
}
%    \end{macrocode}
%    \end{macro}
%
%    We must define all variants because of dynamic driver setup.
%    \begin{macrocode}
\def\HOLOGO@temp#1#2{#2}
%    \end{macrocode}
%
%    \begin{macro}{\HOLOGO@ScaleBox@pdftex}
%    \begin{macrocode}
\HOLOGO@temp{pdftex}{%
  \def\HOLOGO@ScaleBox@pdftex#1#2#3{%
    \HOLOGO@pdfliteral{%
      q #1 0 0 #2 0 0 cm%
    }%
    #3%
    \HOLOGO@pdfliteral{%
      Q%
    }%
  }%
}
%    \end{macrocode}
%    \end{macro}
%    \begin{macro}{\HOLOGO@ScaleBox@dvips}
%    \begin{macrocode}
\HOLOGO@temp{dvips}{%
  \def\HOLOGO@ScaleBox@dvips#1#2#3{%
    \special{ps:%
      gsave %
      currentpoint %
      currentpoint translate %
      #1 #2 scale %
      neg exch neg exch translate%
    }%
    #3%
    \special{ps:%
      currentpoint %
      grestore %
      moveto%
    }%
  }%
}
%    \end{macrocode}
%    \end{macro}
%    \begin{macro}{\HOLOGO@ScaleBox@dvipdfm}
%    \begin{macrocode}
\HOLOGO@temp{dvipdfm}{%
  \let\HOLOGO@ScaleBox@dvipdfm\HOLOGO@ScaleBox@dvips
}
%    \end{macrocode}
%    \end{macro}
%    Since \hologo{XeTeX} v0.6.
%    \begin{macro}{\HOLOGO@ScaleBox@xetex}
%    \begin{macrocode}
\HOLOGO@temp{xetex}{%
  \def\HOLOGO@ScaleBox@xetex#1#2#3{%
    \special{x:gsave}%
    \special{x:scale #1 #2}%
    #3%
    \special{x:grestore}%
  }%
}
%    \end{macrocode}
%    \end{macro}
%    \begin{macro}{\HOLOGO@ScaleBox@vtex}
%    \begin{macrocode}
\HOLOGO@temp{vtex}{%
  \def\HOLOGO@ScaleBox@vtex#1#2#3{%
    \special{r(#1,0,0,#2,0,0}%
    #3%
    \special{r)}%
  }%
}
%    \end{macrocode}
%    \end{macro}
%
%    \begin{macrocode}
\HOLOGO@AtEnd%
%</package>
%    \end{macrocode}
%
% \section{Test}
%
% \subsection{Catcode checks for loading}
%
%    \begin{macrocode}
%<*test1>
%    \end{macrocode}
%    \begin{macrocode}
\catcode`\{=1 %
\catcode`\}=2 %
\catcode`\#=6 %
\catcode`\@=11 %
\expandafter\ifx\csname count@\endcsname\relax
  \countdef\count@=255 %
\fi
\expandafter\ifx\csname @gobble\endcsname\relax
  \long\def\@gobble#1{}%
\fi
\expandafter\ifx\csname @firstofone\endcsname\relax
  \long\def\@firstofone#1{#1}%
\fi
\expandafter\ifx\csname loop\endcsname\relax
  \expandafter\@firstofone
\else
  \expandafter\@gobble
\fi
{%
  \def\loop#1\repeat{%
    \def\body{#1}%
    \iterate
  }%
  \def\iterate{%
    \body
      \let\next\iterate
    \else
      \let\next\relax
    \fi
    \next
  }%
  \let\repeat=\fi
}%
\def\RestoreCatcodes{}
\count@=0 %
\loop
  \edef\RestoreCatcodes{%
    \RestoreCatcodes
    \catcode\the\count@=\the\catcode\count@\relax
  }%
\ifnum\count@<255 %
  \advance\count@ 1 %
\repeat

\def\RangeCatcodeInvalid#1#2{%
  \count@=#1\relax
  \loop
    \catcode\count@=15 %
  \ifnum\count@<#2\relax
    \advance\count@ 1 %
  \repeat
}
\def\RangeCatcodeCheck#1#2#3{%
  \count@=#1\relax
  \loop
    \ifnum#3=\catcode\count@
    \else
      \errmessage{%
        Character \the\count@\space
        with wrong catcode \the\catcode\count@\space
        instead of \number#3%
      }%
    \fi
  \ifnum\count@<#2\relax
    \advance\count@ 1 %
  \repeat
}
\def\space{ }
\expandafter\ifx\csname LoadCommand\endcsname\relax
  \def\LoadCommand{\input hologo.sty\relax}%
\fi
\def\Test{%
  \RangeCatcodeInvalid{0}{47}%
  \RangeCatcodeInvalid{58}{64}%
  \RangeCatcodeInvalid{91}{96}%
  \RangeCatcodeInvalid{123}{255}%
  \catcode`\@=12 %
  \catcode`\\=0 %
  \catcode`\%=14 %
  \LoadCommand
  \RangeCatcodeCheck{0}{36}{15}%
  \RangeCatcodeCheck{37}{37}{14}%
  \RangeCatcodeCheck{38}{47}{15}%
  \RangeCatcodeCheck{48}{57}{12}%
  \RangeCatcodeCheck{58}{63}{15}%
  \RangeCatcodeCheck{64}{64}{12}%
  \RangeCatcodeCheck{65}{90}{11}%
  \RangeCatcodeCheck{91}{91}{15}%
  \RangeCatcodeCheck{92}{92}{0}%
  \RangeCatcodeCheck{93}{96}{15}%
  \RangeCatcodeCheck{97}{122}{11}%
  \RangeCatcodeCheck{123}{255}{15}%
  \RestoreCatcodes
}
\Test
\csname @@end\endcsname
\end
%    \end{macrocode}
%    \begin{macrocode}
%</test1>
%    \end{macrocode}
%
% \subsection{Spacefactor}
%
%    The space factor must be 1000 after a logo. If it is greater 1000
%    then the following space is a space after a sentence closing point.
%    If the space factor is smaller 1000 then an immediate following
%    dot is interpreted as abbreviation, not sentence closing point.
%
%    \begin{macrocode}
%<*test-spacefactor>
\NeedsTeXFormat{LaTeX2e}
\documentclass{article}
\usepackage{hologo}[2016/05/12]
\usepackage{kvsetkeys}
\usepackage{qstest}
\IncludeTests{*}
\LogTests{log}{*}{*}
\begin{document}
\begin{qstest}{spacefactor}{spacefactor}
\newcommand*{\Test}[1]{%
  \sbox0{%
    \hologo{#1}%
    \Expect*{1000 (#1)}*{\the\spacefactor\space(#1)}%
  }%
}%
\makeatletter
\def\TestList{}
\def\hologoEntry#1#2#3{%
  \edef\TestList{%
    \ifx\TestList\@empty
    \else
      \TestList,%
    \fi
    #1%
    \ifx\\#2\\%
    \else
      ={variant=#2}%
    \fi
  }%
}
\hologoList
\expandafter\kv@parse@normalized\expandafter{%
  \TestList
}{%
  \begingroup
    \let\@logo=\kv@key
    \ifx\kv@value\relax
    \else
      \expandafter\hologoLogoSetup\expandafter\@logo\expandafter{%
        \kv@value
      }%
    \fi
    \Test\@logo
  \endgroup
  \@gobbletwo
}
\end{qstest}
\end{document}
%</test-spacefactor>
%    \end{macrocode}
%
% \subsection{Complete list}
%
%    \begin{macrocode}
%<*test-list>
\NeedsTeXFormat{LaTeX2e}
\documentclass[12pt,a4paper]{article}
\usepackage{hologo}[2016/05/12]
\usepackage[T1]{fontenc}
\usepackage{lmodern}
\usepackage{parskip}
\usepackage[unicode]{hyperref}[2011/09/28]
\usepackage{bookmark}[2011/09/19]
\bookmarksetup{%
  numbered,%
  open,%
  openlevel=2,%
}
\renewcommand*{\contentsname}{List of logos}
\begin{document}
\tableofcontents
\def\TestFont#1#2#3#4#5#6{%
  \begingroup
    \usefont{#3}{#4}{#5}{#6}%
    \HologoVariant{#1}{#2}/\hologoVariant{#1}{#2}%
    \quad
    \begingroup\scriptsize\hologoVariant{#1}{#2}\endgroup
    \quad
  \endgroup
  (#3/#4/#5/#6)%
  \par
}
\makeatletter
\def\hologoEntry#1#2#3{%
  \section{%
    \HologoVariant{#1}{#2}/\hologoVariant{#1}{#2} %
    {[#1\ifx\\#2\\\else\space(#2)\fi]}% hash-ok
  }% braces around [] because of bug in tex4ht
  \begingroup
    \hypersetup{unicode=false}%
    \bookmark[%
      dest=\@currentHref,%
      rellevel=1,%
      keeplevel,%
    ]{%
      \HologoVariant{#1}{#2}/\hologoVariant{#1}{#2} %
      (PDFDocEncoding)%
    }%
  \endgroup
  \TestFont{#1}{#2}{OT1}{cmr}{m}{n}%
  \TestFont{#1}{#2}{OT1}{cmss}{m}{n}%
  \TestFont{#1}{#2}{OT1}{cmr}{b}{n}%
  \TestFont{#1}{#2}{OT1}{cmr}{m}{it}%
  \TestFont{#1}{#2}{OT1}{cmtt}{m}{n}%
  \TestFont{#1}{#2}{T1}{lmr}{m}{n}%
  \TestFont{#1}{#2}{T1}{lmss}{m}{n}%
  \TestFont{#1}{#2}{T1}{lmr}{b}{n}%
  \TestFont{#1}{#2}{T1}{lmr}{m}{it}%
  \TestFont{#1}{#2}{T1}{lmtt}{m}{n}%
  \TestFont{#1}{#2}{T1}{lmvtt}{m}{n}%
  \TestFont{#1}{#2}{T1}{qtm}{m}{n}%
  \TestFont{#1}{#2}{T1}{qhv}{m}{n}%
  \TestFont{#1}{#2}{T1}{qtm}{b}{n}%
  \TestFont{#1}{#2}{T1}{qtm}{m}{it}%
  \TestFont{#1}{#2}{T1}{qcr}{m}{n}%
  \newpage
}
\makeatother
\hologoList
\end{document}
%</test-list>
%    \end{macrocode}
%
% \section{Installation}
%
% \subsection{Download}
%
% \paragraph{Package.} This package is available on
% CTAN\footnote{\url{ftp://ftp.ctan.org/tex-archive/}}:
% \begin{description}
% \item[\CTAN{macros/latex/contrib/oberdiek/hologo.dtx}] The source file.
% \item[\CTAN{macros/latex/contrib/oberdiek/hologo.pdf}] Documentation.
% \end{description}
%
%
% \paragraph{Bundle.} All the packages of the bundle `oberdiek'
% are also available in a TDS compliant ZIP archive. There
% the packages are already unpacked and the documentation files
% are generated. The files and directories obey the TDS standard.
% \begin{description}
% \item[\CTAN{install/macros/latex/contrib/oberdiek.tds.zip}]
% \end{description}
% \emph{TDS} refers to the standard ``A Directory Structure
% for \TeX\ Files'' (\CTAN{tds/tds.pdf}). Directories
% with \xfile{texmf} in their name are usually organized this way.
%
% \subsection{Bundle installation}
%
% \paragraph{Unpacking.} Unpack the \xfile{oberdiek.tds.zip} in the
% TDS tree (also known as \xfile{texmf} tree) of your choice.
% Example (linux):
% \begin{quote}
%   |unzip oberdiek.tds.zip -d ~/texmf|
% \end{quote}
%
% \paragraph{Script installation.}
% Check the directory \xfile{TDS:scripts/oberdiek/} for
% scripts that need further installation steps.
% Package \xpackage{attachfile2} comes with the Perl script
% \xfile{pdfatfi.pl} that should be installed in such a way
% that it can be called as \texttt{pdfatfi}.
% Example (linux):
% \begin{quote}
%   |chmod +x scripts/oberdiek/pdfatfi.pl|\\
%   |cp scripts/oberdiek/pdfatfi.pl /usr/local/bin/|
% \end{quote}
%
% \subsection{Package installation}
%
% \paragraph{Unpacking.} The \xfile{.dtx} file is a self-extracting
% \docstrip\ archive. The files are extracted by running the
% \xfile{.dtx} through \plainTeX:
% \begin{quote}
%   \verb|tex hologo.dtx|
% \end{quote}
%
% \paragraph{TDS.} Now the different files must be moved into
% the different directories in your installation TDS tree
% (also known as \xfile{texmf} tree):
% \begin{quote}
% \def\t{^^A
% \begin{tabular}{@{}>{\ttfamily}l@{ $\rightarrow$ }>{\ttfamily}l@{}}
%   hologo.sty & tex/generic/oberdiek/hologo.sty\\
%   hologo.pdf & doc/latex/oberdiek/hologo.pdf\\
%   example/hologo-example.tex & doc/latex/oberdiek/example/hologo-example.tex\\
%   test/hologo-test1.tex & doc/latex/oberdiek/test/hologo-test1.tex\\
%   test/hologo-test-spacefactor.tex & doc/latex/oberdiek/test/hologo-test-spacefactor.tex\\
%   test/hologo-test-list.tex & doc/latex/oberdiek/test/hologo-test-list.tex\\
%   hologo.dtx & source/latex/oberdiek/hologo.dtx\\
% \end{tabular}^^A
% }^^A
% \sbox0{\t}^^A
% \ifdim\wd0>\linewidth
%   \begingroup
%     \advance\linewidth by\leftmargin
%     \advance\linewidth by\rightmargin
%   \edef\x{\endgroup
%     \def\noexpand\lw{\the\linewidth}^^A
%   }\x
%   \def\lwbox{^^A
%     \leavevmode
%     \hbox to \linewidth{^^A
%       \kern-\leftmargin\relax
%       \hss
%       \usebox0
%       \hss
%       \kern-\rightmargin\relax
%     }^^A
%   }^^A
%   \ifdim\wd0>\lw
%     \sbox0{\small\t}^^A
%     \ifdim\wd0>\linewidth
%       \ifdim\wd0>\lw
%         \sbox0{\footnotesize\t}^^A
%         \ifdim\wd0>\linewidth
%           \ifdim\wd0>\lw
%             \sbox0{\scriptsize\t}^^A
%             \ifdim\wd0>\linewidth
%               \ifdim\wd0>\lw
%                 \sbox0{\tiny\t}^^A
%                 \ifdim\wd0>\linewidth
%                   \lwbox
%                 \else
%                   \usebox0
%                 \fi
%               \else
%                 \lwbox
%               \fi
%             \else
%               \usebox0
%             \fi
%           \else
%             \lwbox
%           \fi
%         \else
%           \usebox0
%         \fi
%       \else
%         \lwbox
%       \fi
%     \else
%       \usebox0
%     \fi
%   \else
%     \lwbox
%   \fi
% \else
%   \usebox0
% \fi
% \end{quote}
% If you have a \xfile{docstrip.cfg} that configures and enables \docstrip's
% TDS installing feature, then some files can already be in the right
% place, see the documentation of \docstrip.
%
% \subsection{Refresh file name databases}
%
% If your \TeX~distribution
% (\teTeX, \mikTeX, \dots) relies on file name databases, you must refresh
% these. For example, \teTeX\ users run \verb|texhash| or
% \verb|mktexlsr|.
%
% \subsection{Some details for the interested}
%
% \paragraph{Attached source.}
%
% The PDF documentation on CTAN also includes the
% \xfile{.dtx} source file. It can be extracted by
% AcrobatReader 6 or higher. Another option is \textsf{pdftk},
% e.g. unpack the file into the current directory:
% \begin{quote}
%   \verb|pdftk hologo.pdf unpack_files output .|
% \end{quote}
%
% \paragraph{Unpacking with \LaTeX.}
% The \xfile{.dtx} chooses its action depending on the format:
% \begin{description}
% \item[\plainTeX:] Run \docstrip\ and extract the files.
% \item[\LaTeX:] Generate the documentation.
% \end{description}
% If you insist on using \LaTeX\ for \docstrip\ (really,
% \docstrip\ does not need \LaTeX), then inform the autodetect routine
% about your intention:
% \begin{quote}
%   \verb|latex \let\install=y\input{hologo.dtx}|
% \end{quote}
% Do not forget to quote the argument according to the demands
% of your shell.
%
% \paragraph{Generating the documentation.}
% You can use both the \xfile{.dtx} or the \xfile{.drv} to generate
% the documentation. The process can be configured by the
% configuration file \xfile{ltxdoc.cfg}. For instance, put this
% line into this file, if you want to have A4 as paper format:
% \begin{quote}
%   \verb|\PassOptionsToClass{a4paper}{article}|
% \end{quote}
% An example follows how to generate the
% documentation with pdf\LaTeX:
% \begin{quote}
%\begin{verbatim}
%pdflatex hologo.dtx
%makeindex -s gind.ist hologo.idx
%pdflatex hologo.dtx
%makeindex -s gind.ist hologo.idx
%pdflatex hologo.dtx
%\end{verbatim}
% \end{quote}
%
% \section{Catalogue}
%
% The following XML file can be used as source for the
% \href{http://mirror.ctan.org/help/Catalogue/catalogue.html}{\TeX\ Catalogue}.
% The elements \texttt{caption} and \texttt{description} are imported
% from the original XML file from the Catalogue.
% The name of the XML file in the Catalogue is \xfile{hologo.xml}.
%    \begin{macrocode}
%<*catalogue>
<?xml version='1.0' encoding='us-ascii'?>
<!DOCTYPE entry SYSTEM 'catalogue.dtd'>
<entry datestamp='$Date$' modifier='$Author$' id='hologo'>
  <name>hologo</name>
  <caption>A collection of logos with bookmark support.</caption>
  <authorref id='auth:oberdiek'/>
  <copyright owner='Heiko Oberdiek' year='2010-2012'/>
  <license type='lppl1.3'/>
  <version number='1.10'/>
  <description>
    The package defines a single command <tt>\hologo</tt>, whose
    argument is the usual case-confused ASCII version of the logo.
    The command is bookmark-enabled, so that every logo becomes
    available in bookmarks without further work.
    <p/>
    The package is part of the <xref refid='oberdiek'>oberdiek</xref>
    bundle.
  </description>
  <documentation details='Package documentation'
      href='ctan:/macros/latex/contrib/oberdiek/hologo.pdf'/>
  <ctan file='true' path='/macros/latex/contrib/oberdiek/hologo.dtx'/>
  <miktex location='oberdiek'/>
  <texlive location='oberdiek'/>
  <install path='/macros/latex/contrib/oberdiek/oberdiek.tds.zip'/>
</entry>
%</catalogue>
%    \end{macrocode}
%
% \begin{thebibliography}{9}
% \raggedright
%
% \bibitem{btxdoc}
% Oren Patashnik,
% \textit{\hologo{BibTeX}ing},
% 1988-02-08.\\
% \CTAN{biblio/bibtex/base/}
%
% \bibitem{dtklogos}
% Gerd Neugebauer, DANTE,
% \textit{Package \xpackage{dtklogos}},
% 2011-04-25.\\
% \CTAN{usergrps/dante/dtk/dtklogos.sty}
%
% \bibitem{etexman}
% The \hologo{NTS} Team,
% \textit{The \hologo{eTeX} manual},
% 1998-02.\\
% \CTAN{systems/e-tex/v2/doc/}
%
% \bibitem{ExTeX-FAQ}
% The \hologo{ExTeX} group,
% \textit{\hologo{ExTeX}: FAQ -- How is \hologo{ExTeX} typeset?},
% 2007-04-14.\\
% \url{http://www.extex.org/documentation/faq.html}
%
% \bibitem{LyX}
% %@MISC{ LyX,
% %  title = {{LyX 2.0.0 -- The Document Processor [Computer software and manual]}},
% %  author = {{The LyX Team}},
% %  howpublished = {Internet: http://www.lyx.org},
% %  year = {2011-05-08},
% %  note = {Retrieved May 10, 2011, from http://www.lyx.org},
% %  url = {http://www.lyx.org/}
% %}
% The \hologo{LyX} Team,
% \textit{\hologo{LyX} -- The Document Processor},
% 2011-05-08.\\
% \url{http://www.lyx.org/}
%
% \bibitem{OzTeX}
% Andrew Trevorrow,
% \hologo{OzTeX} FAQ: What is the correct way to typeset ``\hologo{OzTeX}''?,
% 2011-09-15 (visited).
% \url{http://www.trevorrow.com/oztex/ozfaq.html#oztex-logo}
%
% \bibitem{PiCTeX}
% Michael Wichura,
% \textit{The \hologo{PiCTeX} macro package},
% 1987-09-21.
% \CTAN{graphics/pictex/}
%
% \bibitem{scrlogo}
% Markus Kohm,
% \textit{\hologo{KOMAScript} Datei \xfile{scrlogo.dtx}},
% 2009-01-30.\\
% \CTAN{install/macros/latex/contrib/komascript.tds.zip}
%
% \end{thebibliography}
%
% \begin{History}
%   \begin{Version}{2010/04/08 v1.0}
%   \item
%     The first version.
%   \end{Version}
%   \begin{Version}{2010/04/16 v1.1}
%   \item
%     \cs{Hologo} added for support of logos at start of a sentence.
%   \item
%     \cs{hologoSetup} and \cs{hologoLogoSetup} added.
%   \item
%     Options \xoption{break}, \xoption{hyphenbreak}, \xoption{spacebreak}
%     added.
%   \item
%     Variant support added by option \xoption{variant}.
%   \end{Version}
%   \begin{Version}{2010/04/24 v1.2}
%   \item
%     \hologo{LaTeX3} added.
%   \item
%     \hologo{VTeX} added.
%   \end{Version}
%   \begin{Version}{2010/11/21 v1.3}
%   \item
%     \hologo{iniTeX}, \hologo{virTeX} added.
%   \end{Version}
%   \begin{Version}{2011/03/25 v1.4}
%   \item
%     \hologo{ConTeXt} with variants added.
%   \item
%     Option \xoption{discretionarybreak} added as refinement for
%     option \xoption{break}.
%   \end{Version}
%   \begin{Version}{2011/04/21 v1.5}
%   \item
%     Wrong TDS directory for test files fixed.
%   \end{Version}
%   \begin{Version}{2011/10/01 v1.6}
%   \item
%     Support for package \xpackage{tex4ht} added.
%   \item
%     Support for \cs{csname} added if \cs{ifincsname} is available.
%   \item
%     New logos:
%     \hologo{(La)TeX},
%     \hologo{biber},
%     \hologo{BibTeX} (\xoption{sc}, \xoption{sf}),
%     \hologo{emTeX},
%     \hologo{ExTeX},
%     \hologo{KOMAScript},
%     \hologo{La},
%     \hologo{LyX},
%     \hologo{MiKTeX},
%     \hologo{NTS},
%     \hologo{OzMF},
%     \hologo{OzMP},
%     \hologo{OzTeX},
%     \hologo{OzTtH},
%     \hologo{PCTeX},
%     \hologo{PiC},
%     \hologo{PiCTeX},
%     \hologo{METAFONT},
%     \hologo{MetaFun},
%     \hologo{METAPOST},
%     \hologo{MetaPost},
%     \hologo{SLiTeX} (\xoption{lift}, \xoption{narrow}, \xoption{simple}),
%     \hologo{SliTeX} (\xoption{narrow}, \xoption{simple}, \xoption{lift}),
%     \hologo{teTeX}.
%   \item
%     Fixes:
%     \hologo{iniTeX},
%     \hologo{pdfLaTeX},
%     \hologo{pdfTeX},
%     \hologo{virTeX}.
%   \item
%     \cs{hologoFontSetup} and \cs{hologoLogoFontSetup} added.
%   \item
%     \cs{hologoVariant} and \cs{HologoVariant} added.
%   \end{Version}
%   \begin{Version}{2011/11/22 v1.7}
%   \item
%     New logos:
%     \hologo{BibTeX8},
%     \hologo{LaTeXML},
%     \hologo{SageTeX},
%     \hologo{TeX4ht},
%     \hologo{TTH}.
%   \item
%     \hologo{Xe} and friends: Driver stuff fixed.
%   \item
%     \hologo{Xe} and friends: Support for italic added.
%   \item
%     \hologo{Xe} and friends: Package support for \xpackage{pgf}
%     and \xpackage{pstricks} added.
%   \end{Version}
%   \begin{Version}{2011/11/29 v1.8}
%   \item
%     New logos:
%     \hologo{HanTheThanh}.
%   \end{Version}
%   \begin{Version}{2011/12/21 v1.9}
%   \item
%     Patch for package \xpackage{ifxetex} added for the case that
%     \cs{newif} is undefined in \hologo{iniTeX}.
%   \item
%     Some fixes for \hologo{iniTeX}.
%   \end{Version}
%   \begin{Version}{2012/04/26 v1.10}
%   \item
%     Fix in bookmark version of logo ``\hologo{HanTheThanh}''.
%   \end{Version}
%   \begin{Version}{2016/05/12 v1.11}
%   \item
%     Update HOLOGO@IfCharExists (previously in texlive)
%   \item define pdfliteral in current luatex.
%   \end{Version}
% \end{History}
%
% \PrintIndex
%
% \Finale
\endinput
%
        \else
          \input hologo.cfg\relax
        \fi
      \else
        \@PackageInfoNoLine{hologo}{%
          Empty configuration file `hologo.cfg' ignored%
        }%
      \fi
    \fi
  }%
}
%    \end{macrocode}
%
%    \begin{macrocode}
\def\HOLOGO@temp#1#2{%
  \kv@define@key{HoLogoDriver}{#1}[]{%
    \begingroup
      \def\HOLOGO@temp{##1}%
      \ltx@onelevel@sanitize\HOLOGO@temp
      \ifx\HOLOGO@temp\ltx@empty
      \else
        \@PackageError{hologo}{%
          Value (\HOLOGO@temp) not permitted for option `#1'%
        }%
        \@ehc
      \fi
    \endgroup
    \def\hologoDriver{#2}%
  }%
}%
\def\HOLOGO@@temp#1#2{%
  \ifx\kv@value\relax
    \HOLOGO@temp{#1}{#1}%
  \else
    \HOLOGO@temp{#1}{#2}%
  \fi
}%
\kv@parse@normalized{%
  pdftex,%
  luatex=pdftex,%
  dvipdfm,%
  dvipdfmx=dvipdfm,%
  dvips,%
  dvipsone=dvips,%
  xdvi=dvips,%
  xetex,%
  vtex,%
}\HOLOGO@@temp
%    \end{macrocode}
%
%    \begin{macrocode}
\kv@define@key{HoLogoDriver}{driverfallback}{%
  \def\HOLOGO@DriverFallback{#1}%
}
%    \end{macrocode}
%
%    \begin{macro}{\HOLOGO@DriverFallback}
%    \begin{macrocode}
\def\HOLOGO@DriverFallback{dvips}
%    \end{macrocode}
%    \end{macro}
%
%    \begin{macro}{\hologoDriverSetup}
%    \begin{macrocode}
\def\hologoDriverSetup{%
  \let\hologoDriver\ltx@undefined
  \HOLOGO@DriverSetup
}
%    \end{macrocode}
%    \end{macro}
%
%    \begin{macro}{\HOLOGO@DriverSetup}
%    \begin{macrocode}
\def\HOLOGO@DriverSetup#1{%
  \kvsetkeys{HoLogoDriver}{#1}%
  \HOLOGO@CheckDriver
  \ltx@ifundefined{hologoDriver}{%
    \begingroup
    \edef\x{\endgroup
      \noexpand\kvsetkeys{HoLogoDriver}{\HOLOGO@DriverFallback}%
    }\x
  }{}%
  \@PackageInfoNoLine{hologo}{Using driver `\hologoDriver'}%
}
%    \end{macrocode}
%    \end{macro}
%
%    \begin{macro}{\HOLOGO@CheckDriver}
%    \begin{macrocode}
\def\HOLOGO@CheckDriver{%
  \ifpdf
    \def\hologoDriver{pdftex}%
    \let\HOLOGO@pdfliteral\pdfliteral
    \ifluatex
      \ifx\pdfextension\@undefined\else
        \protected\def\pdfliteral{\pdfextension literal}%
        \let\HOLOGO@pdfliteral\pdfliteral
      \fi
      \ltx@IfUndefined{HOLOGO@pdfliteral}{%
        \ifnum\luatexversion<36 %
        \else
          \begingroup
            \let\HOLOGO@temp\endgroup
            \ifcase0%
                \directlua{%
                  if tex.enableprimitives then %
                    tex.enableprimitives('HOLOGO@', {'pdfliteral'})%
                  else %
                    tex.print('1')%
                  end%
                }%
                \ifx\HOLOGO@pdfliteral\@undefined 1\fi%
                \relax%
              \endgroup
              \let\HOLOGO@temp\relax
              \global\let\HOLOGO@pdfliteral\HOLOGO@pdfliteral
            \fi%
          \HOLOGO@temp
        \fi
      }{}%
    \fi
    \ltx@IfUndefined{HOLOGO@pdfliteral}{%
      \@PackageWarningNoLine{hologo}{%
        Cannot find \string\pdfliteral
      }%
    }{}%
  \else
    \ifxetex
      \def\hologoDriver{xetex}%
    \else
      \ifvtex
        \def\hologoDriver{vtex}%
      \fi
    \fi
  \fi
}
%    \end{macrocode}
%    \end{macro}
%
%    \begin{macro}{\HOLOGO@WarningUnsupportedDriver}
%    \begin{macrocode}
\def\HOLOGO@WarningUnsupportedDriver#1{%
  \@PackageWarningNoLine{hologo}{%
    Logo `#1' needs driver specific macros,\MessageBreak
    but driver `\hologoDriver' is not supported.\MessageBreak
    Use a different driver or\MessageBreak
    load package `graphics' or `pgf'%
  }%
}
%    \end{macrocode}
%    \end{macro}
%
% \subsubsection{Reflect box macros}
%
%    Skip driver part if not needed.
%    \begin{macrocode}
\ltx@IfUndefined{reflectbox}{}{%
  \ltx@IfUndefined{rotatebox}{}{%
    \HOLOGO@AtEnd
  }%
}
\ltx@IfUndefined{pgftext}{}{%
  \HOLOGO@AtEnd
}
\ltx@IfUndefined{psscalebox}{}{%
  \HOLOGO@AtEnd
}
%    \end{macrocode}
%
%    \begin{macrocode}
\def\HOLOGO@temp{LaTeX2e}
\ifx\fmtname\HOLOGO@temp
  \RequirePackage{kvoptions}[2011/06/30]%
  \ProcessKeyvalOptions{HoLogoDriver}%
\fi
\HOLOGO@DriverSetup{}
%    \end{macrocode}
%
%    \begin{macro}{\HOLOGO@ReflectBox}
%    \begin{macrocode}
\def\HOLOGO@ReflectBox#1{%
  \begingroup
    \setbox\ltx@zero\hbox{\begingroup#1\endgroup}%
    \setbox\ltx@two\hbox{%
      \kern\wd\ltx@zero
      \csname HOLOGO@ScaleBox@\hologoDriver\endcsname{-1}{1}{%
        \hbox to 0pt{\copy\ltx@zero\hss}%
      }%
    }%
    \wd\ltx@two=\wd\ltx@zero
    \box\ltx@two
  \endgroup
}
%    \end{macrocode}
%    \end{macro}
%
%    \begin{macro}{\HOLOGO@PointReflectBox}
%    \begin{macrocode}
\def\HOLOGO@PointReflectBox#1{%
  \begingroup
    \setbox\ltx@zero\hbox{\begingroup#1\endgroup}%
    \setbox\ltx@two\hbox{%
      \kern\wd\ltx@zero
      \raise\ht\ltx@zero\hbox{%
        \csname HOLOGO@ScaleBox@\hologoDriver\endcsname{-1}{-1}{%
          \hbox to 0pt{\copy\ltx@zero\hss}%
        }%
      }%
    }%
    \wd\ltx@two=\wd\ltx@zero
    \box\ltx@two
  \endgroup
}
%    \end{macrocode}
%    \end{macro}
%
%    We must define all variants because of dynamic driver setup.
%    \begin{macrocode}
\def\HOLOGO@temp#1#2{#2}
%    \end{macrocode}
%
%    \begin{macro}{\HOLOGO@ScaleBox@pdftex}
%    \begin{macrocode}
\HOLOGO@temp{pdftex}{%
  \def\HOLOGO@ScaleBox@pdftex#1#2#3{%
    \HOLOGO@pdfliteral{%
      q #1 0 0 #2 0 0 cm%
    }%
    #3%
    \HOLOGO@pdfliteral{%
      Q%
    }%
  }%
}
%    \end{macrocode}
%    \end{macro}
%    \begin{macro}{\HOLOGO@ScaleBox@dvips}
%    \begin{macrocode}
\HOLOGO@temp{dvips}{%
  \def\HOLOGO@ScaleBox@dvips#1#2#3{%
    \special{ps:%
      gsave %
      currentpoint %
      currentpoint translate %
      #1 #2 scale %
      neg exch neg exch translate%
    }%
    #3%
    \special{ps:%
      currentpoint %
      grestore %
      moveto%
    }%
  }%
}
%    \end{macrocode}
%    \end{macro}
%    \begin{macro}{\HOLOGO@ScaleBox@dvipdfm}
%    \begin{macrocode}
\HOLOGO@temp{dvipdfm}{%
  \let\HOLOGO@ScaleBox@dvipdfm\HOLOGO@ScaleBox@dvips
}
%    \end{macrocode}
%    \end{macro}
%    Since \hologo{XeTeX} v0.6.
%    \begin{macro}{\HOLOGO@ScaleBox@xetex}
%    \begin{macrocode}
\HOLOGO@temp{xetex}{%
  \def\HOLOGO@ScaleBox@xetex#1#2#3{%
    \special{x:gsave}%
    \special{x:scale #1 #2}%
    #3%
    \special{x:grestore}%
  }%
}
%    \end{macrocode}
%    \end{macro}
%    \begin{macro}{\HOLOGO@ScaleBox@vtex}
%    \begin{macrocode}
\HOLOGO@temp{vtex}{%
  \def\HOLOGO@ScaleBox@vtex#1#2#3{%
    \special{r(#1,0,0,#2,0,0}%
    #3%
    \special{r)}%
  }%
}
%    \end{macrocode}
%    \end{macro}
%
%    \begin{macrocode}
\HOLOGO@AtEnd%
%</package>
%    \end{macrocode}
%
% \section{Test}
%
% \subsection{Catcode checks for loading}
%
%    \begin{macrocode}
%<*test1>
%    \end{macrocode}
%    \begin{macrocode}
\catcode`\{=1 %
\catcode`\}=2 %
\catcode`\#=6 %
\catcode`\@=11 %
\expandafter\ifx\csname count@\endcsname\relax
  \countdef\count@=255 %
\fi
\expandafter\ifx\csname @gobble\endcsname\relax
  \long\def\@gobble#1{}%
\fi
\expandafter\ifx\csname @firstofone\endcsname\relax
  \long\def\@firstofone#1{#1}%
\fi
\expandafter\ifx\csname loop\endcsname\relax
  \expandafter\@firstofone
\else
  \expandafter\@gobble
\fi
{%
  \def\loop#1\repeat{%
    \def\body{#1}%
    \iterate
  }%
  \def\iterate{%
    \body
      \let\next\iterate
    \else
      \let\next\relax
    \fi
    \next
  }%
  \let\repeat=\fi
}%
\def\RestoreCatcodes{}
\count@=0 %
\loop
  \edef\RestoreCatcodes{%
    \RestoreCatcodes
    \catcode\the\count@=\the\catcode\count@\relax
  }%
\ifnum\count@<255 %
  \advance\count@ 1 %
\repeat

\def\RangeCatcodeInvalid#1#2{%
  \count@=#1\relax
  \loop
    \catcode\count@=15 %
  \ifnum\count@<#2\relax
    \advance\count@ 1 %
  \repeat
}
\def\RangeCatcodeCheck#1#2#3{%
  \count@=#1\relax
  \loop
    \ifnum#3=\catcode\count@
    \else
      \errmessage{%
        Character \the\count@\space
        with wrong catcode \the\catcode\count@\space
        instead of \number#3%
      }%
    \fi
  \ifnum\count@<#2\relax
    \advance\count@ 1 %
  \repeat
}
\def\space{ }
\expandafter\ifx\csname LoadCommand\endcsname\relax
  \def\LoadCommand{\input hologo.sty\relax}%
\fi
\def\Test{%
  \RangeCatcodeInvalid{0}{47}%
  \RangeCatcodeInvalid{58}{64}%
  \RangeCatcodeInvalid{91}{96}%
  \RangeCatcodeInvalid{123}{255}%
  \catcode`\@=12 %
  \catcode`\\=0 %
  \catcode`\%=14 %
  \LoadCommand
  \RangeCatcodeCheck{0}{36}{15}%
  \RangeCatcodeCheck{37}{37}{14}%
  \RangeCatcodeCheck{38}{47}{15}%
  \RangeCatcodeCheck{48}{57}{12}%
  \RangeCatcodeCheck{58}{63}{15}%
  \RangeCatcodeCheck{64}{64}{12}%
  \RangeCatcodeCheck{65}{90}{11}%
  \RangeCatcodeCheck{91}{91}{15}%
  \RangeCatcodeCheck{92}{92}{0}%
  \RangeCatcodeCheck{93}{96}{15}%
  \RangeCatcodeCheck{97}{122}{11}%
  \RangeCatcodeCheck{123}{255}{15}%
  \RestoreCatcodes
}
\Test
\csname @@end\endcsname
\end
%    \end{macrocode}
%    \begin{macrocode}
%</test1>
%    \end{macrocode}
%
% \subsection{Spacefactor}
%
%    The space factor must be 1000 after a logo. If it is greater 1000
%    then the following space is a space after a sentence closing point.
%    If the space factor is smaller 1000 then an immediate following
%    dot is interpreted as abbreviation, not sentence closing point.
%
%    \begin{macrocode}
%<*test-spacefactor>
\NeedsTeXFormat{LaTeX2e}
\documentclass{article}
\usepackage{hologo}[2016/05/12]
\usepackage{kvsetkeys}
\usepackage{qstest}
\IncludeTests{*}
\LogTests{log}{*}{*}
\begin{document}
\begin{qstest}{spacefactor}{spacefactor}
\newcommand*{\Test}[1]{%
  \sbox0{%
    \hologo{#1}%
    \Expect*{1000 (#1)}*{\the\spacefactor\space(#1)}%
  }%
}%
\makeatletter
\def\TestList{}
\def\hologoEntry#1#2#3{%
  \edef\TestList{%
    \ifx\TestList\@empty
    \else
      \TestList,%
    \fi
    #1%
    \ifx\\#2\\%
    \else
      ={variant=#2}%
    \fi
  }%
}
\hologoList
\expandafter\kv@parse@normalized\expandafter{%
  \TestList
}{%
  \begingroup
    \let\@logo=\kv@key
    \ifx\kv@value\relax
    \else
      \expandafter\hologoLogoSetup\expandafter\@logo\expandafter{%
        \kv@value
      }%
    \fi
    \Test\@logo
  \endgroup
  \@gobbletwo
}
\end{qstest}
\end{document}
%</test-spacefactor>
%    \end{macrocode}
%
% \subsection{Complete list}
%
%    \begin{macrocode}
%<*test-list>
\NeedsTeXFormat{LaTeX2e}
\documentclass[12pt,a4paper]{article}
\usepackage{hologo}[2016/05/12]
\usepackage[T1]{fontenc}
\usepackage{lmodern}
\usepackage{parskip}
\usepackage[unicode]{hyperref}[2011/09/28]
\usepackage{bookmark}[2011/09/19]
\bookmarksetup{%
  numbered,%
  open,%
  openlevel=2,%
}
\renewcommand*{\contentsname}{List of logos}
\begin{document}
\tableofcontents
\def\TestFont#1#2#3#4#5#6{%
  \begingroup
    \usefont{#3}{#4}{#5}{#6}%
    \HologoVariant{#1}{#2}/\hologoVariant{#1}{#2}%
    \quad
    \begingroup\scriptsize\hologoVariant{#1}{#2}\endgroup
    \quad
  \endgroup
  (#3/#4/#5/#6)%
  \par
}
\makeatletter
\def\hologoEntry#1#2#3{%
  \section{%
    \HologoVariant{#1}{#2}/\hologoVariant{#1}{#2} %
    {[#1\ifx\\#2\\\else\space(#2)\fi]}% hash-ok
  }% braces around [] because of bug in tex4ht
  \begingroup
    \hypersetup{unicode=false}%
    \bookmark[%
      dest=\@currentHref,%
      rellevel=1,%
      keeplevel,%
    ]{%
      \HologoVariant{#1}{#2}/\hologoVariant{#1}{#2} %
      (PDFDocEncoding)%
    }%
  \endgroup
  \TestFont{#1}{#2}{OT1}{cmr}{m}{n}%
  \TestFont{#1}{#2}{OT1}{cmss}{m}{n}%
  \TestFont{#1}{#2}{OT1}{cmr}{b}{n}%
  \TestFont{#1}{#2}{OT1}{cmr}{m}{it}%
  \TestFont{#1}{#2}{OT1}{cmtt}{m}{n}%
  \TestFont{#1}{#2}{T1}{lmr}{m}{n}%
  \TestFont{#1}{#2}{T1}{lmss}{m}{n}%
  \TestFont{#1}{#2}{T1}{lmr}{b}{n}%
  \TestFont{#1}{#2}{T1}{lmr}{m}{it}%
  \TestFont{#1}{#2}{T1}{lmtt}{m}{n}%
  \TestFont{#1}{#2}{T1}{lmvtt}{m}{n}%
  \TestFont{#1}{#2}{T1}{qtm}{m}{n}%
  \TestFont{#1}{#2}{T1}{qhv}{m}{n}%
  \TestFont{#1}{#2}{T1}{qtm}{b}{n}%
  \TestFont{#1}{#2}{T1}{qtm}{m}{it}%
  \TestFont{#1}{#2}{T1}{qcr}{m}{n}%
  \newpage
}
\makeatother
\hologoList
\end{document}
%</test-list>
%    \end{macrocode}
%
% \section{Installation}
%
% \subsection{Download}
%
% \paragraph{Package.} This package is available on
% CTAN\footnote{\url{ftp://ftp.ctan.org/tex-archive/}}:
% \begin{description}
% \item[\CTAN{macros/latex/contrib/oberdiek/hologo.dtx}] The source file.
% \item[\CTAN{macros/latex/contrib/oberdiek/hologo.pdf}] Documentation.
% \end{description}
%
%
% \paragraph{Bundle.} All the packages of the bundle `oberdiek'
% are also available in a TDS compliant ZIP archive. There
% the packages are already unpacked and the documentation files
% are generated. The files and directories obey the TDS standard.
% \begin{description}
% \item[\CTAN{install/macros/latex/contrib/oberdiek.tds.zip}]
% \end{description}
% \emph{TDS} refers to the standard ``A Directory Structure
% for \TeX\ Files'' (\CTAN{tds/tds.pdf}). Directories
% with \xfile{texmf} in their name are usually organized this way.
%
% \subsection{Bundle installation}
%
% \paragraph{Unpacking.} Unpack the \xfile{oberdiek.tds.zip} in the
% TDS tree (also known as \xfile{texmf} tree) of your choice.
% Example (linux):
% \begin{quote}
%   |unzip oberdiek.tds.zip -d ~/texmf|
% \end{quote}
%
% \paragraph{Script installation.}
% Check the directory \xfile{TDS:scripts/oberdiek/} for
% scripts that need further installation steps.
% Package \xpackage{attachfile2} comes with the Perl script
% \xfile{pdfatfi.pl} that should be installed in such a way
% that it can be called as \texttt{pdfatfi}.
% Example (linux):
% \begin{quote}
%   |chmod +x scripts/oberdiek/pdfatfi.pl|\\
%   |cp scripts/oberdiek/pdfatfi.pl /usr/local/bin/|
% \end{quote}
%
% \subsection{Package installation}
%
% \paragraph{Unpacking.} The \xfile{.dtx} file is a self-extracting
% \docstrip\ archive. The files are extracted by running the
% \xfile{.dtx} through \plainTeX:
% \begin{quote}
%   \verb|tex hologo.dtx|
% \end{quote}
%
% \paragraph{TDS.} Now the different files must be moved into
% the different directories in your installation TDS tree
% (also known as \xfile{texmf} tree):
% \begin{quote}
% \def\t{^^A
% \begin{tabular}{@{}>{\ttfamily}l@{ $\rightarrow$ }>{\ttfamily}l@{}}
%   hologo.sty & tex/generic/oberdiek/hologo.sty\\
%   hologo.pdf & doc/latex/oberdiek/hologo.pdf\\
%   example/hologo-example.tex & doc/latex/oberdiek/example/hologo-example.tex\\
%   test/hologo-test1.tex & doc/latex/oberdiek/test/hologo-test1.tex\\
%   test/hologo-test-spacefactor.tex & doc/latex/oberdiek/test/hologo-test-spacefactor.tex\\
%   test/hologo-test-list.tex & doc/latex/oberdiek/test/hologo-test-list.tex\\
%   hologo.dtx & source/latex/oberdiek/hologo.dtx\\
% \end{tabular}^^A
% }^^A
% \sbox0{\t}^^A
% \ifdim\wd0>\linewidth
%   \begingroup
%     \advance\linewidth by\leftmargin
%     \advance\linewidth by\rightmargin
%   \edef\x{\endgroup
%     \def\noexpand\lw{\the\linewidth}^^A
%   }\x
%   \def\lwbox{^^A
%     \leavevmode
%     \hbox to \linewidth{^^A
%       \kern-\leftmargin\relax
%       \hss
%       \usebox0
%       \hss
%       \kern-\rightmargin\relax
%     }^^A
%   }^^A
%   \ifdim\wd0>\lw
%     \sbox0{\small\t}^^A
%     \ifdim\wd0>\linewidth
%       \ifdim\wd0>\lw
%         \sbox0{\footnotesize\t}^^A
%         \ifdim\wd0>\linewidth
%           \ifdim\wd0>\lw
%             \sbox0{\scriptsize\t}^^A
%             \ifdim\wd0>\linewidth
%               \ifdim\wd0>\lw
%                 \sbox0{\tiny\t}^^A
%                 \ifdim\wd0>\linewidth
%                   \lwbox
%                 \else
%                   \usebox0
%                 \fi
%               \else
%                 \lwbox
%               \fi
%             \else
%               \usebox0
%             \fi
%           \else
%             \lwbox
%           \fi
%         \else
%           \usebox0
%         \fi
%       \else
%         \lwbox
%       \fi
%     \else
%       \usebox0
%     \fi
%   \else
%     \lwbox
%   \fi
% \else
%   \usebox0
% \fi
% \end{quote}
% If you have a \xfile{docstrip.cfg} that configures and enables \docstrip's
% TDS installing feature, then some files can already be in the right
% place, see the documentation of \docstrip.
%
% \subsection{Refresh file name databases}
%
% If your \TeX~distribution
% (\teTeX, \mikTeX, \dots) relies on file name databases, you must refresh
% these. For example, \teTeX\ users run \verb|texhash| or
% \verb|mktexlsr|.
%
% \subsection{Some details for the interested}
%
% \paragraph{Attached source.}
%
% The PDF documentation on CTAN also includes the
% \xfile{.dtx} source file. It can be extracted by
% AcrobatReader 6 or higher. Another option is \textsf{pdftk},
% e.g. unpack the file into the current directory:
% \begin{quote}
%   \verb|pdftk hologo.pdf unpack_files output .|
% \end{quote}
%
% \paragraph{Unpacking with \LaTeX.}
% The \xfile{.dtx} chooses its action depending on the format:
% \begin{description}
% \item[\plainTeX:] Run \docstrip\ and extract the files.
% \item[\LaTeX:] Generate the documentation.
% \end{description}
% If you insist on using \LaTeX\ for \docstrip\ (really,
% \docstrip\ does not need \LaTeX), then inform the autodetect routine
% about your intention:
% \begin{quote}
%   \verb|latex \let\install=y% \iffalse meta-comment
%
% File: hologo.dtx
% Version: 2016/05/12 v1.11
% Info: A logo collection with bookmark support
%
% Copyright (C) 2010-2012 by
%    Heiko Oberdiek <heiko.oberdiek at googlemail.com>
%
% This work may be distributed and/or modified under the
% conditions of the LaTeX Project Public License, either
% version 1.3c of this license or (at your option) any later
% version. This version of this license is in
%    http://www.latex-project.org/lppl/lppl-1-3c.txt
% and the latest version of this license is in
%    http://www.latex-project.org/lppl.txt
% and version 1.3 or later is part of all distributions of
% LaTeX version 2005/12/01 or later.
%
% This work has the LPPL maintenance status "maintained".
%
% This Current Maintainer of this work is Heiko Oberdiek.
%
% The Base Interpreter refers to any `TeX-Format',
% because some files are installed in TDS:tex/generic//.
%
% This work consists of the main source file hologo.dtx
% and the derived files
%    hologo.sty, hologo.pdf, hologo.ins, hologo.drv, hologo-example.tex,
%    hologo-test1.tex, hologo-test-spacefactor.tex,
%    hologo-test-list.tex.
%
% Distribution:
%    CTAN:macros/latex/contrib/oberdiek/hologo.dtx
%    CTAN:macros/latex/contrib/oberdiek/hologo.pdf
%
% Unpacking:
%    (a) If hologo.ins is present:
%           tex hologo.ins
%    (b) Without hologo.ins:
%           tex hologo.dtx
%    (c) If you insist on using LaTeX
%           latex \let\install=y\input{hologo.dtx}
%        (quote the arguments according to the demands of your shell)
%
% Documentation:
%    (a) If hologo.drv is present:
%           latex hologo.drv
%    (b) Without hologo.drv:
%           latex hologo.dtx; ...
%    The class ltxdoc loads the configuration file ltxdoc.cfg
%    if available. Here you can specify further options, e.g.
%    use A4 as paper format:
%       \PassOptionsToClass{a4paper}{article}
%
%    Programm calls to get the documentation (example):
%       pdflatex hologo.dtx
%       makeindex -s gind.ist hologo.idx
%       pdflatex hologo.dtx
%       makeindex -s gind.ist hologo.idx
%       pdflatex hologo.dtx
%
% Installation:
%    TDS:tex/generic/oberdiek/hologo.sty
%    TDS:doc/latex/oberdiek/hologo.pdf
%    TDS:doc/latex/oberdiek/example/hologo-example.tex
%    TDS:doc/latex/oberdiek/test/hologo-test1.tex
%    TDS:doc/latex/oberdiek/test/hologo-test-spacefactor.tex
%    TDS:doc/latex/oberdiek/test/hologo-test-list.tex
%    TDS:source/latex/oberdiek/hologo.dtx
%
%<*ignore>
\begingroup
  \catcode123=1 %
  \catcode125=2 %
  \def\x{LaTeX2e}%
\expandafter\endgroup
\ifcase 0\ifx\install y1\fi\expandafter
         \ifx\csname processbatchFile\endcsname\relax\else1\fi
         \ifx\fmtname\x\else 1\fi\relax
\else\csname fi\endcsname
%</ignore>
%<*install>
\input docstrip.tex
\Msg{************************************************************************}
\Msg{* Installation}
\Msg{* Package: hologo 2016/05/12 v1.11 A logo collection with bookmark support (HO)}
\Msg{************************************************************************}

\keepsilent
\askforoverwritefalse

\let\MetaPrefix\relax
\preamble

This is a generated file.

Project: hologo
Version: 2016/05/12 v1.11

Copyright (C) 2010-2012 by
   Heiko Oberdiek <heiko.oberdiek at googlemail.com>

This work may be distributed and/or modified under the
conditions of the LaTeX Project Public License, either
version 1.3c of this license or (at your option) any later
version. This version of this license is in
   http://www.latex-project.org/lppl/lppl-1-3c.txt
and the latest version of this license is in
   http://www.latex-project.org/lppl.txt
and version 1.3 or later is part of all distributions of
LaTeX version 2005/12/01 or later.

This work has the LPPL maintenance status "maintained".

This Current Maintainer of this work is Heiko Oberdiek.

The Base Interpreter refers to any `TeX-Format',
because some files are installed in TDS:tex/generic//.

This work consists of the main source file hologo.dtx
and the derived files
   hologo.sty, hologo.pdf, hologo.ins, hologo.drv, hologo-example.tex,
   hologo-test1.tex, hologo-test-spacefactor.tex,
   hologo-test-list.tex.

\endpreamble
\let\MetaPrefix\DoubleperCent

\generate{%
  \file{hologo.ins}{\from{hologo.dtx}{install}}%
  \file{hologo.drv}{\from{hologo.dtx}{driver}}%
  \usedir{tex/generic/oberdiek}%
  \file{hologo.sty}{\from{hologo.dtx}{package}}%
  \usedir{doc/latex/oberdiek/example}%
  \file{hologo-example.tex}{\from{hologo.dtx}{example}}%
  \usedir{doc/latex/oberdiek/test}%
  \file{hologo-test1.tex}{\from{hologo.dtx}{test1}}%
  \file{hologo-test-spacefactor.tex}{\from{hologo.dtx}{test-spacefactor}}%
  \file{hologo-test-list.tex}{\from{hologo.dtx}{test-list}}%
  \nopreamble
  \nopostamble
  \usedir{source/latex/oberdiek/catalogue}%
  \file{hologo.xml}{\from{hologo.dtx}{catalogue}}%
}

\catcode32=13\relax% active space
\let =\space%
\Msg{************************************************************************}
\Msg{*}
\Msg{* To finish the installation you have to move the following}
\Msg{* file into a directory searched by TeX:}
\Msg{*}
\Msg{*     hologo.sty}
\Msg{*}
\Msg{* To produce the documentation run the file `hologo.drv'}
\Msg{* through LaTeX.}
\Msg{*}
\Msg{* Happy TeXing!}
\Msg{*}
\Msg{************************************************************************}

\endbatchfile
%</install>
%<*ignore>
\fi
%</ignore>
%<*driver>
\NeedsTeXFormat{LaTeX2e}
\ProvidesFile{hologo.drv}%
  [2016/05/12 v1.11 A logo collection with bookmark support (HO)]%
\documentclass{ltxdoc}
\usepackage{holtxdoc}[2011/11/22]
\usepackage{hologo}[2016/05/12]
\usepackage{longtable}
\usepackage{array}
\usepackage{paralist}
%\usepackage[T1]{fontenc}
%\usepackage{lmodern}
\begin{document}
  \DocInput{hologo.dtx}%
\end{document}
%</driver>
% \fi
%
%
% \CharacterTable
%  {Upper-case    \A\B\C\D\E\F\G\H\I\J\K\L\M\N\O\P\Q\R\S\T\U\V\W\X\Y\Z
%   Lower-case    \a\b\c\d\e\f\g\h\i\j\k\l\m\n\o\p\q\r\s\t\u\v\w\x\y\z
%   Digits        \0\1\2\3\4\5\6\7\8\9
%   Exclamation   \!     Double quote  \"     Hash (number) \#
%   Dollar        \$     Percent       \%     Ampersand     \&
%   Acute accent  \'     Left paren    \(     Right paren   \)
%   Asterisk      \*     Plus          \+     Comma         \,
%   Minus         \-     Point         \.     Solidus       \/
%   Colon         \:     Semicolon     \;     Less than     \<
%   Equals        \=     Greater than  \>     Question mark \?
%   Commercial at \@     Left bracket  \[     Backslash     \\
%   Right bracket \]     Circumflex    \^     Underscore    \_
%   Grave accent  \`     Left brace    \{     Vertical bar  \|
%   Right brace   \}     Tilde         \~}
%
% \GetFileInfo{hologo.drv}
%
% \title{The \xpackage{hologo} package}
% \date{2016/05/12 v1.11}
% \author{Heiko Oberdiek\\\xemail{heiko.oberdiek at googlemail.com}}
%
% \maketitle
%
% \begin{abstract}
% This package starts a collection of logos with support for bookmarks
% strings.
% \end{abstract}
%
% \tableofcontents
%
% \section{Documentation}
%
% \subsection{Logo macros}
%
% \begin{declcs}{hologo} \M{name}
% \end{declcs}
% Macro \cs{hologo} sets the logo with name \meta{name}.
% The following table shows the supported names.
%
% \begingroup
%   \def\hologoEntry#1#2#3{^^A
%     #1&#2&\hologoLogoSetup{#1}{variant=#2}\hologo{#1}&#3\tabularnewline
%   }
%   \begin{longtable}{>{\ttfamily}l>{\ttfamily}lll}
%     \rmfamily\bfseries{name} & \rmfamily\bfseries variant
%     & \bfseries logo & \bfseries since\\
%     \hline
%     \endhead
%     \hologoList
%   \end{longtable}
% \endgroup
%
% \begin{declcs}{Hologo} \M{name}
% \end{declcs}
% Macro \cs{Hologo} starts the logo \meta{name} with an uppercase
% letter. As an exception small greek letters are not converted
% to uppercase. Examples, see \hologo{eTeX} and \hologo{ExTeX}.
%
% \subsection{Setup macros}
%
% The package does not support package options, but the following
% setup macros can be used to set options.
%
% \begin{declcs}{hologoSetup} \M{key value list}
% \end{declcs}
% Macro \cs{hologoSetup} sets global options.
%
% \begin{declcs}{hologoLogoSetup} \M{logo} \M{key value list}
% \end{declcs}
% Some options can also be used to configure a logo.
% These settings take precedence over global option settings.
%
% \subsection{Options}\label{sec:options}
%
% There are boolean and string options:
% \begin{description}
% \item[Boolean option:]
% It takes |true| or |false|
% as value. If the value is omitted, then |true| is used.
% \item[String option:]
% A value must be given as string. (But the string might be empty.)
% \end{description}
% The following options can be used both in \cs{hologoSetup}
% and \cs{hologoLogoSetup}:
% \begin{description}
% \def\entry#1{\item[\xoption{#1}:]}
% \entry{break}
%   enables or disables line breaks inside the logo. This setting is
%   refined by options \xoption{hyphenbreak}, \xoption{spacebreak}
%   or \xoption{discretionarybreak}.
%   Default is |false|.
% \entry{hyphenbreak}
%   enables or disables the line break right after the hyphen character.
% \entry{spacebreak}
%   enables or disables line breaks at space characters.
% \entry{discretionarybreak}
%   enables or disables line breaks at hyphenation points
%   (inserted by \cs{-}).
% \end{description}
% Macro \cs{hologoLogoSetup} also knows:
% \begin{description}
% \item[\xoption{variant}:]
%   This is a string option. It specifies a variant of a logo that
%   must exist. An empty string selects the package default variant.
% \end{description}
% Example:
% \begin{quote}
%   |\hologoSetup{break=false}|\\
%   |\hologoLogoSetup{plainTeX}{variant=hyphen,hyphenbreak}|\\
%   Then ``plain-\TeX'' contains one break point after the hyphen.
% \end{quote}
%
% \subsection{Driver options}
%
% Sometimes graphical operations are needed to construct some
% glyphs (e.g.\ \hologo{XeTeX}). If package \xpackage{graphics}
% or package \xpackage{pgf} are found, then the macros are taken
% from there. Otherwise the packge defines its own operations
% and therefore needs the driver information. Many drivers are
% detected automatically (\hologo{pdfTeX}/\hologo{LuaTeX}
% in PDF mode, \hologo{XeTeX}, \hologo{VTeX}). These have precedence
% over a driver option. The driver can be given as package option
% or using \cs{hologoDriverSetup}.
% The following list contains the recognized driver options:
% \begin{itemize}
% \item \xoption{pdftex}, \xoption{luatex}
% \item \xoption{dvipdfm}, \xoption{dvipdfmx}
% \item \xoption{dvips}, \xoption{dvipsone}, \xoption{xdvi}
% \item \xoption{xetex}
% \item \xoption{vtex}
% \end{itemize}
% The left driver of a line is the driver name that is used internally.
% The following names are aliases for drivers that use the
% same method. Therefore the entry in the \xext{log} file for
% the used driver prints the internally used driver name.
% \begin{description}
% \item[\xoption{driverfallback}:]
%   This option expects a driver that is used,
%   if the driver could not be detected automatically.
% \end{description}
%
% \begin{declcs}{hologoDriverSetup} \M{driver option}
% \end{declcs}
% The driver can also be configured after package loading
% using \cs{hologoDriverSetup}, also the way for \hologo{plainTeX}
% to setup the driver.
%
% \subsection{Font setup}
%
% Some logos require a special font, but should also be usable by
% \hologo{plainTeX}. Therefore the package provides some ways
% to influence the font settings. The options below
% take font settings as values. Both font commands
% such as \cs{sffamily} and macros that take one argument
% like \cs{textsf} can be used.
%
% \begin{declcs}{hologoFontSetup} \M{key value list}
% \end{declcs}
% Macro \cs{hologoFontSetup} sets the fonts for all logos.
% Supported keys:
% \begin{description}
% \def\entry#1{\item[\xoption{#1}:]}
% \entry{general}
%   This font is used for all logos. The default is empty.
%   That means no special font is used.
% \entry{bibsf}
%   This font is used for
%   {\hologoLogoSetup{BibTeX}{variant=sf}\hologo{BibTeX}}
%   with variant \xoption{sf}.
% \entry{rm}
%   This font is a serif font. It is used for \hologo{ExTeX}.
% \entry{sc}
%   This font specifies a small caps font. It is used for
%   {\hologoLogoSetup{BibTeX}{variant=sc}\hologo{BibTeX}}
%   with variant \xoption{sc}.
% \entry{sf}
%   This font specifies a sans serif font. The default
%   is \cs{sffamily}, then \cs{sf} is tried. Otherwise
%   a warning is given. It is used by \hologo{KOMAScript}.
% \entry{sy}
%   This is the font for math symbols (e.g. cmsy).
%   It is used by \hologo{AmS}, \hologo{NTS}, \hologo{ExTeX}.
% \entry{logo}
%   \hologo{METAFONT} and \hologo{METAPOST} are using that font.
%   In \hologo{LaTeX} \cs{logofamily} is used and
%   the definitions of package \xpackage{mflogo} are used
%   if the package is not loaded.
%   Otherwise the \cs{tenlogo} is used and defined
%   if it does not already exists.
% \end{description}
%
% \begin{declcs}{hologoLogoFontSetup} \M{logo} \M{key value list}
% \end{declcs}
% Fonts can also be set for a logo or logo component separately,
% see the following list.
% The keys are the same as for \cs{hologoFontSetup}.
%
% \begin{longtable}{>{\ttfamily}l>{\sffamily}ll}
%   \meta{logo} & keys & result\\
%   \hline
%   \endhead
%   BibTeX & bibsf & {\hologoLogoSetup{BibTeX}{variant=sf}\hologo{BibTeX}}\\[.5ex]
%   BibTeX & sc & {\hologoLogoSetup{BibTeX}{variant=sc}\hologo{BibTeX}}\\[.5ex]
%   ExTeX & rm & \hologo{ExTeX}\\
%   SliTeX & rm & \hologo{SliTeX}\\[.5ex]
%   AmS & sy & \hologo{AmS}\\
%   ExTeX & sy & \hologo{ExTeX}\\
%   NTS & sy & \hologo{NTS}\\[.5ex]
%   KOMAScript & sf & \hologo{KOMAScript}\\[.5ex]
%   METAFONT & logo & \hologo{METAFONT}\\
%   METAPOST & logo & \hologo{METAPOST}\\[.5ex]
%   SliTeX & sc \hologo{SliTeX}
% \end{longtable}
%
% \subsubsection{Font order}
%
% For all logos the font \xoption{general} is applied first.
% Example:
%\begin{quote}
%|\hologoFontSetup{general=\color{red}}|
%\end{quote}
% will print red logos.
% Then if the font uses a special font \xoption{sf}, for example,
% the font is applied that is setup by \cs{hologoLogoFontSetup}.
% If this font is not setup, then the common font setup
% by \cs{hologoFontSetup} is used. Otherwise a warning is given,
% that there is no font configured.
%
% \subsection{Additional user macros}
%
% Usually a variant of a logo is configured by using
% \cs{hologoLogoSetup}, because it is bad style to mix
% different variants of the same logo in the same text.
% There the following macros are a convenience for testing.
%
% \begin{declcs}{hologoVariant} \M{name} \M{variant}\\
%   \cs{HologoVariant} \M{name} \M{variant}
% \end{declcs}
% Logo \meta{name} is set using \meta{variant} that specifies
% explicitely which variant of the macro is used. If the argument
% is empty, then the default form of the logo is used
% (configurable by \cs{hologoLogoSetup}).
%
% \cs{HologoVariant} is used if the logo is set in a context
% that needs an uppercase first letter (beginning of a sentence, \dots).
%
% \begin{declcs}{hologoList}\\
%   \cs{hologoEntry} \M{logo} \M{variant} \M{since}
% \end{declcs}
% Macro \cs{hologoList} contains all logos that are provided
% by the package including variants. The list consists of calls
% of \cs{hologoEntry} with three arguments starting with the
% logo name \meta{logo} and its variant \meta{variant}. An empty
% variant means the current default. Argument \meta{since} specifies
% with version of the package \xpackage{hologo} is needed to get
% the logo. If the logo is fixed, then the date gets updated.
% Therefore the date \meta{since} is not exactly the date of
% the first introduction, but rather the date of the latest fix.
%
% Before \cs{hologoList} can be used, macro \cs{hologoEntry} needs
% a definition. The example file in section \ref{sec:example}
% shows applications of \cs{hologoList}.
%
% \subsection{Supported contexts}
%
% Macros \cs{hologo} and friends support special contexts:
% \begin{itemize}
% \item \hologo{LaTeX}'s protection mechanism.
% \item Bookmarks of package \xpackage{hyperref}.
% \item Package \xpackage{tex4ht}.
% \item The macros can be used inside \cs{csname} constructs,
%   if \cs{ifincsname} is available (\hologo{pdfTeX}, \hologo{XeTeX},
%   \hologo{LuaTeX}).
% \end{itemize}
%
% \subsection{Example}
% \label{sec:example}
%
% The following example prints the logos in different fonts.
%    \begin{macrocode}
%<*example>
%<<verbatim
\NeedsTeXFormat{LaTeX2e}
\documentclass[a4paper]{article}
\usepackage[
  hmargin=20mm,
  vmargin=20mm,
]{geometry}
\pagestyle{empty}
\usepackage{hologo}[2016/05/12]
\usepackage{longtable}
\usepackage{array}
\setlength{\extrarowheight}{2pt}
\usepackage[T1]{fontenc}
\usepackage{lmodern}
\usepackage{pdflscape}
\usepackage[
  pdfencoding=auto,
]{hyperref}
\hypersetup{
  pdfauthor={Heiko Oberdiek},
  pdftitle={Example for package `hologo'},
  pdfsubject={Logos with fonts lmr, lmss, qtm, qpl, qhv},
}
\usepackage{bookmark}

% Print the logo list on the console

\begingroup
  \typeout{}%
  \typeout{*** Begin of logo list ***}%
  \newcommand*{\hologoEntry}[3]{%
    \typeout{#1 \ifx\\#2\\\else(#2) \fi[#3]}%
  }%
  \hologoList
  \typeout{*** End of logo list ***}%
  \typeout{}%
\endgroup

\begin{document}
\begin{landscape}

  \section{Example file for package `hologo'}

  % Table for font names

  \begin{longtable}{>{\bfseries}ll}
    \textbf{font} & \textbf{Font name}\\
    \hline
    lmr & Latin Modern Roman\\
    lmss & Latin Modern Sans\\
    qtm & \TeX\ Gyre Termes\\
    qhv & \TeX\ Gyre Heros\\
    qpl & \TeX\ Gyre Pagella\\
  \end{longtable}

  % Logo list with logos in different fonts

  \begingroup
    \newcommand*{\SetVariant}[2]{%
      \ifx\\#2\\%
      \else
        \hologoLogoSetup{#1}{variant=#2}%
      \fi
    }%
    \newcommand*{\hologoEntry}[3]{%
      \SetVariant{#1}{#2}%
      \raisebox{1em}[0pt][0pt]{\hypertarget{#1@#2}{}}%
      \bookmark[%
        dest={#1@#2},%
      ]{%
        #1\ifx\\#2\\\else\space(#2)\fi: \Hologo{#1}, \hologo{#1} %
        [Unicode]%
      }%
      \hypersetup{unicode=false}%
      \bookmark[%
        dest={#1@#2},%
      ]{%
        #1\ifx\\#2\\\else\space(#2)\fi: \Hologo{#1}, \hologo{#1} %
        [PDFDocEncoding]%
      }%
      \texttt{#1}%
      &%
      \texttt{#2}%
      &%
      \Hologo{#1}%
      &%
      \SetVariant{#1}{#2}%
      \hologo{#1}%
      &%
      \SetVariant{#1}{#2}%
      \fontfamily{qtm}\selectfont
      \hologo{#1}%
      &%
      \SetVariant{#1}{#2}%
      \fontfamily{qpl}\selectfont
      \hologo{#1}%
      &%
      \SetVariant{#1}{#2}%
      \textsf{\hologo{#1}}%
      &%
      \SetVariant{#1}{#2}%
      \fontfamily{qhv}\selectfont
      \hologo{#1}%
      \tabularnewline
    }%
    \begin{longtable}{llllllll}%
      \textbf{\textit{logo}} & \textbf{\textit{variant}} &
      \texttt{\string\Hologo} &
      \textbf{lmr} & \textbf{qtm} & \textbf{qpl} &
      \textbf{lmss} & \textbf{qhv}
      \tabularnewline
      \hline
      \endhead
      \hologoList
    \end{longtable}%
  \endgroup

\end{landscape}
\end{document}
%verbatim
%</example>
%    \end{macrocode}
%
% \StopEventually{
% }
%
% \section{Implementation}
%    \begin{macrocode}
%<*package>
%    \end{macrocode}
%    Reload check, especially if the package is not used with \LaTeX.
%    \begin{macrocode}
\begingroup\catcode61\catcode48\catcode32=10\relax%
  \catcode13=5 % ^^M
  \endlinechar=13 %
  \catcode35=6 % #
  \catcode39=12 % '
  \catcode44=12 % ,
  \catcode45=12 % -
  \catcode46=12 % .
  \catcode58=12 % :
  \catcode64=11 % @
  \catcode123=1 % {
  \catcode125=2 % }
  \expandafter\let\expandafter\x\csname ver@hologo.sty\endcsname
  \ifx\x\relax % plain-TeX, first loading
  \else
    \def\empty{}%
    \ifx\x\empty % LaTeX, first loading,
      % variable is initialized, but \ProvidesPackage not yet seen
    \else
      \expandafter\ifx\csname PackageInfo\endcsname\relax
        \def\x#1#2{%
          \immediate\write-1{Package #1 Info: #2.}%
        }%
      \else
        \def\x#1#2{\PackageInfo{#1}{#2, stopped}}%
      \fi
      \x{hologo}{The package is already loaded}%
      \aftergroup\endinput
    \fi
  \fi
\endgroup%
%    \end{macrocode}
%    Package identification:
%    \begin{macrocode}
\begingroup\catcode61\catcode48\catcode32=10\relax%
  \catcode13=5 % ^^M
  \endlinechar=13 %
  \catcode35=6 % #
  \catcode39=12 % '
  \catcode40=12 % (
  \catcode41=12 % )
  \catcode44=12 % ,
  \catcode45=12 % -
  \catcode46=12 % .
  \catcode47=12 % /
  \catcode58=12 % :
  \catcode64=11 % @
  \catcode91=12 % [
  \catcode93=12 % ]
  \catcode123=1 % {
  \catcode125=2 % }
  \expandafter\ifx\csname ProvidesPackage\endcsname\relax
    \def\x#1#2#3[#4]{\endgroup
      \immediate\write-1{Package: #3 #4}%
      \xdef#1{#4}%
    }%
  \else
    \def\x#1#2[#3]{\endgroup
      #2[{#3}]%
      \ifx#1\@undefined
        \xdef#1{#3}%
      \fi
      \ifx#1\relax
        \xdef#1{#3}%
      \fi
    }%
  \fi
\expandafter\x\csname ver@hologo.sty\endcsname
\ProvidesPackage{hologo}%
  [2016/05/12 v1.11 A logo collection with bookmark support (HO)]%
%    \end{macrocode}
%
%    \begin{macrocode}
\begingroup\catcode61\catcode48\catcode32=10\relax%
  \catcode13=5 % ^^M
  \endlinechar=13 %
  \catcode123=1 % {
  \catcode125=2 % }
  \catcode64=11 % @
  \def\x{\endgroup
    \expandafter\edef\csname HOLOGO@AtEnd\endcsname{%
      \endlinechar=\the\endlinechar\relax
      \catcode13=\the\catcode13\relax
      \catcode32=\the\catcode32\relax
      \catcode35=\the\catcode35\relax
      \catcode61=\the\catcode61\relax
      \catcode64=\the\catcode64\relax
      \catcode123=\the\catcode123\relax
      \catcode125=\the\catcode125\relax
    }%
  }%
\x\catcode61\catcode48\catcode32=10\relax%
\catcode13=5 % ^^M
\endlinechar=13 %
\catcode35=6 % #
\catcode64=11 % @
\catcode123=1 % {
\catcode125=2 % }
\def\TMP@EnsureCode#1#2{%
  \edef\HOLOGO@AtEnd{%
    \HOLOGO@AtEnd
    \catcode#1=\the\catcode#1\relax
  }%
  \catcode#1=#2\relax
}
\TMP@EnsureCode{10}{12}% ^^J
\TMP@EnsureCode{33}{12}% !
\TMP@EnsureCode{34}{12}% "
\TMP@EnsureCode{36}{3}% $
\TMP@EnsureCode{38}{4}% &
\TMP@EnsureCode{39}{12}% '
\TMP@EnsureCode{40}{12}% (
\TMP@EnsureCode{41}{12}% )
\TMP@EnsureCode{42}{12}% *
\TMP@EnsureCode{43}{12}% +
\TMP@EnsureCode{44}{12}% ,
\TMP@EnsureCode{45}{12}% -
\TMP@EnsureCode{46}{12}% .
\TMP@EnsureCode{47}{12}% /
\TMP@EnsureCode{58}{12}% :
\TMP@EnsureCode{59}{12}% ;
\TMP@EnsureCode{60}{12}% <
\TMP@EnsureCode{62}{12}% >
\TMP@EnsureCode{63}{12}% ?
\TMP@EnsureCode{91}{12}% [
\TMP@EnsureCode{93}{12}% ]
\TMP@EnsureCode{94}{7}% ^ (superscript)
\TMP@EnsureCode{95}{8}% _ (subscript)
\TMP@EnsureCode{96}{12}% `
\TMP@EnsureCode{124}{12}% |
\edef\HOLOGO@AtEnd{%
  \HOLOGO@AtEnd
  \escapechar\the\escapechar\relax
  \noexpand\endinput
}
\escapechar=92 %
%    \end{macrocode}
%
% \subsection{Logo list}
%
%    \begin{macro}{\hologoList}
%    \begin{macrocode}
\def\hologoList{%
  \hologoEntry{(La)TeX}{}{2011/10/01}%
  \hologoEntry{AmSLaTeX}{}{2010/04/16}%
  \hologoEntry{AmSTeX}{}{2010/04/16}%
  \hologoEntry{biber}{}{2011/10/01}%
  \hologoEntry{BibTeX}{}{2011/10/01}%
  \hologoEntry{BibTeX}{sf}{2011/10/01}%
  \hologoEntry{BibTeX}{sc}{2011/10/01}%
  \hologoEntry{BibTeX8}{}{2011/11/22}%
  \hologoEntry{ConTeXt}{}{2011/03/25}%
  \hologoEntry{ConTeXt}{narrow}{2011/03/25}%
  \hologoEntry{ConTeXt}{simple}{2011/03/25}%
  \hologoEntry{emTeX}{}{2010/04/26}%
  \hologoEntry{eTeX}{}{2010/04/08}%
  \hologoEntry{ExTeX}{}{2011/10/01}%
  \hologoEntry{HanTheThanh}{}{2011/11/29}%
  \hologoEntry{iniTeX}{}{2011/10/01}%
  \hologoEntry{KOMAScript}{}{2011/10/01}%
  \hologoEntry{La}{}{2010/05/08}%
  \hologoEntry{LaTeX}{}{2010/04/08}%
  \hologoEntry{LaTeX2e}{}{2010/04/08}%
  \hologoEntry{LaTeX3}{}{2010/04/24}%
  \hologoEntry{LaTeXe}{}{2010/04/08}%
  \hologoEntry{LaTeXML}{}{2011/11/22}%
  \hologoEntry{LaTeXTeX}{}{2011/10/01}%
  \hologoEntry{LuaLaTeX}{}{2010/04/08}%
  \hologoEntry{LuaTeX}{}{2010/04/08}%
  \hologoEntry{LyX}{}{2011/10/01}%
  \hologoEntry{METAFONT}{}{2011/10/01}%
  \hologoEntry{MetaFun}{}{2011/10/01}%
  \hologoEntry{METAPOST}{}{2011/10/01}%
  \hologoEntry{MetaPost}{}{2011/10/01}%
  \hologoEntry{MiKTeX}{}{2011/10/01}%
  \hologoEntry{NTS}{}{2011/10/01}%
  \hologoEntry{OzMF}{}{2011/10/01}%
  \hologoEntry{OzMP}{}{2011/10/01}%
  \hologoEntry{OzTeX}{}{2011/10/01}%
  \hologoEntry{OzTtH}{}{2011/10/01}%
  \hologoEntry{PCTeX}{}{2011/10/01}%
  \hologoEntry{pdfTeX}{}{2011/10/01}%
  \hologoEntry{pdfLaTeX}{}{2011/10/01}%
  \hologoEntry{PiC}{}{2011/10/01}%
  \hologoEntry{PiCTeX}{}{2011/10/01}%
  \hologoEntry{plainTeX}{}{2010/04/08}%
  \hologoEntry{plainTeX}{space}{2010/04/16}%
  \hologoEntry{plainTeX}{hyphen}{2010/04/16}%
  \hologoEntry{plainTeX}{runtogether}{2010/04/16}%
  \hologoEntry{SageTeX}{}{2011/11/22}%
  \hologoEntry{SLiTeX}{}{2011/10/01}%
  \hologoEntry{SLiTeX}{lift}{2011/10/01}%
  \hologoEntry{SLiTeX}{narrow}{2011/10/01}%
  \hologoEntry{SLiTeX}{simple}{2011/10/01}%
  \hologoEntry{SliTeX}{}{2011/10/01}%
  \hologoEntry{SliTeX}{narrow}{2011/10/01}%
  \hologoEntry{SliTeX}{simple}{2011/10/01}%
  \hologoEntry{SliTeX}{lift}{2011/10/01}%
  \hologoEntry{teTeX}{}{2011/10/01}%
  \hologoEntry{TeX}{}{2010/04/08}%
  \hologoEntry{TeX4ht}{}{2011/11/22}%
  \hologoEntry{TTH}{}{2011/11/22}%
  \hologoEntry{virTeX}{}{2011/10/01}%
  \hologoEntry{VTeX}{}{2010/04/24}%
  \hologoEntry{Xe}{}{2010/04/08}%
  \hologoEntry{XeLaTeX}{}{2010/04/08}%
  \hologoEntry{XeTeX}{}{2010/04/08}%
}
%    \end{macrocode}
%    \end{macro}
%
% \subsection{Load resources}
%
%    \begin{macrocode}
\begingroup\expandafter\expandafter\expandafter\endgroup
\expandafter\ifx\csname RequirePackage\endcsname\relax
  \def\TMP@RequirePackage#1[#2]{%
    \begingroup\expandafter\expandafter\expandafter\endgroup
    \expandafter\ifx\csname ver@#1.sty\endcsname\relax
      \input #1.sty\relax
    \fi
  }%
  \TMP@RequirePackage{ltxcmds}[2011/02/04]%
  \TMP@RequirePackage{infwarerr}[2010/04/08]%
  \TMP@RequirePackage{kvsetkeys}[2010/03/01]%
  \TMP@RequirePackage{kvdefinekeys}[2010/03/01]%
  \TMP@RequirePackage{pdftexcmds}[2010/04/01]%
  \TMP@RequirePackage{ifpdf}[2010/01/28]%
  \TMP@RequirePackage{ifluatex}[2010/03/01]%
  \ltx@IfUndefined{newif}{%
    \expandafter\let\csname newif\endcsname\ltx@newif
  }{}%
  \TMP@RequirePackage{ifxetex}[2009/01/23]%
  \TMP@RequirePackage{ifvtex}[2010/03/01]%
\else
  \RequirePackage{ltxcmds}[2011/02/04]%
  \RequirePackage{infwarerr}[2010/04/08]%
  \RequirePackage{kvsetkeys}[2010/03/01]%
  \RequirePackage{kvdefinekeys}[2010/03/01]%
  \RequirePackage{pdftexcmds}[2010/04/01]%
  \RequirePackage{ifpdf}[2010/01/28]%
  \RequirePackage{ifluatex}[2010/03/01]%
  \RequirePackage{ifxetex}[2009/01/23]%
  \RequirePackage{ifvtex}[2010/03/01]%
\fi
%    \end{macrocode}
%
%    \begin{macro}{\HOLOGO@IfDefined}
%    \begin{macrocode}
\def\HOLOGO@IfExists#1{%
  \ifx\@undefined#1%
    \expandafter\ltx@secondoftwo
  \else
    \ifx\relax#1%
      \expandafter\ltx@secondoftwo
    \else
      \expandafter\expandafter\expandafter\ltx@firstoftwo
    \fi
  \fi
}
%    \end{macrocode}
%    \end{macro}
%
% \subsection{Setup macros}
%
%    \begin{macro}{\hologoSetup}
%    \begin{macrocode}
\def\hologoSetup{%
  \let\HOLOGO@name\relax
  \HOLOGO@Setup
}
%    \end{macrocode}
%    \end{macro}
%
%    \begin{macro}{\hologoLogoSetup}
%    \begin{macrocode}
\def\hologoLogoSetup#1{%
  \edef\HOLOGO@name{#1}%
  \ltx@IfUndefined{HoLogo@\HOLOGO@name}{%
    \@PackageError{hologo}{%
      Unknown logo `\HOLOGO@name'%
    }\@ehc
    \ltx@gobble
  }{%
    \HOLOGO@Setup
  }%
}
%    \end{macrocode}
%    \end{macro}
%
%    \begin{macro}{\HOLOGO@Setup}
%    \begin{macrocode}
\def\HOLOGO@Setup{%
  \kvsetkeys{HoLogo}%
}
%    \end{macrocode}
%    \end{macro}
%
% \subsection{Options}
%
%    \begin{macro}{\HOLOGO@DeclareBoolOption}
%    \begin{macrocode}
\def\HOLOGO@DeclareBoolOption#1{%
  \expandafter\chardef\csname HOLOGOOPT@#1\endcsname\ltx@zero
  \kv@define@key{HoLogo}{#1}[true]{%
    \def\HOLOGO@temp{##1}%
    \ifx\HOLOGO@temp\HOLOGO@true
      \ifx\HOLOGO@name\relax
        \expandafter\chardef\csname HOLOGOOPT@#1\endcsname=\ltx@one
      \else
        \expandafter\chardef\csname
        HoLogoOpt@#1@\HOLOGO@name\endcsname\ltx@one
      \fi
      \HOLOGO@SetBreakAll{#1}%
    \else
      \ifx\HOLOGO@temp\HOLOGO@false
        \ifx\HOLOGO@name\relax
          \expandafter\chardef\csname HOLOGOOPT@#1\endcsname=\ltx@zero
        \else
          \expandafter\chardef\csname
          HoLogoOpt@#1@\HOLOGO@name\endcsname=\ltx@zero
        \fi
        \HOLOGO@SetBreakAll{#1}%
      \else
        \@PackageError{hologo}{%
          Unknown value `##1' for boolean option `#1'.\MessageBreak
          Known values are `true' and `false'%
        }\@ehc
      \fi
    \fi
  }%
}
%    \end{macrocode}
%    \end{macro}
%
%    \begin{macro}{\HOLOGO@SetBreakAll}
%    \begin{macrocode}
\def\HOLOGO@SetBreakAll#1{%
  \def\HOLOGO@temp{#1}%
  \ifx\HOLOGO@temp\HOLOGO@break
    \ifx\HOLOGO@name\relax
      \chardef\HOLOGOOPT@hyphenbreak=\HOLOGOOPT@break
      \chardef\HOLOGOOPT@spacebreak=\HOLOGOOPT@break
      \chardef\HOLOGOOPT@discretionarybreak=\HOLOGOOPT@break
    \else
      \expandafter\chardef
         \csname HoLogoOpt@hyphenbreak@\HOLOGO@name\endcsname=%
         \csname HoLogoOpt@break@\HOLOGO@name\endcsname
      \expandafter\chardef
         \csname HoLogoOpt@spacebreak@\HOLOGO@name\endcsname=%
         \csname HoLogoOpt@break@\HOLOGO@name\endcsname
      \expandafter\chardef
         \csname HoLogoOpt@discretionarybreak@\HOLOGO@name
             \endcsname=%
         \csname HoLogoOpt@break@\HOLOGO@name\endcsname
    \fi
  \fi
}
%    \end{macrocode}
%    \end{macro}
%
%    \begin{macro}{\HOLOGO@true}
%    \begin{macrocode}
\def\HOLOGO@true{true}
%    \end{macrocode}
%    \end{macro}
%    \begin{macro}{\HOLOGO@false}
%    \begin{macrocode}
\def\HOLOGO@false{false}
%    \end{macrocode}
%    \end{macro}
%    \begin{macro}{\HOLOGO@break}
%    \begin{macrocode}
\def\HOLOGO@break{break}
%    \end{macrocode}
%    \end{macro}
%
%    \begin{macrocode}
\HOLOGO@DeclareBoolOption{break}
\HOLOGO@DeclareBoolOption{hyphenbreak}
\HOLOGO@DeclareBoolOption{spacebreak}
\HOLOGO@DeclareBoolOption{discretionarybreak}
%    \end{macrocode}
%
%    \begin{macrocode}
\kv@define@key{HoLogo}{variant}{%
  \ifx\HOLOGO@name\relax
    \@PackageError{hologo}{%
      Option `variant' is not available in \string\hologoSetup,%
      \MessageBreak
      Use \string\hologoLogoSetup\space instead%
    }\@ehc
  \else
    \edef\HOLOGO@temp{#1}%
    \ifx\HOLOGO@temp\ltx@empty
      \expandafter
      \let\csname HoLogoOpt@variant@\HOLOGO@name\endcsname\@undefined
    \else
      \ltx@IfUndefined{HoLogo@\HOLOGO@name @\HOLOGO@temp}{%
        \@PackageError{hologo}{%
          Unknown variant `\HOLOGO@temp' of logo `\HOLOGO@name'%
        }\@ehc
      }{%
        \expandafter
        \let\csname HoLogoOpt@variant@\HOLOGO@name\endcsname
            \HOLOGO@temp
      }%
    \fi
  \fi
}
%    \end{macrocode}
%
%    \begin{macro}{\HOLOGO@Variant}
%    \begin{macrocode}
\def\HOLOGO@Variant#1{%
  #1%
  \ltx@ifundefined{HoLogoOpt@variant@#1}{%
  }{%
    @\csname HoLogoOpt@variant@#1\endcsname
  }%
}
%    \end{macrocode}
%    \end{macro}
%
% \subsection{Break/no-break support}
%
%    \begin{macro}{\HOLOGO@space}
%    \begin{macrocode}
\def\HOLOGO@space{%
  \ltx@ifundefined{HoLogoOpt@spacebreak@\HOLOGO@name}{%
    \ltx@ifundefined{HoLogoOpt@break@\HOLOGO@name}{%
      \chardef\HOLOGO@temp=\HOLOGOOPT@spacebreak
    }{%
      \chardef\HOLOGO@temp=%
        \csname HoLogoOpt@break@\HOLOGO@name\endcsname
    }%
  }{%
    \chardef\HOLOGO@temp=%
      \csname HoLogoOpt@spacebreak@\HOLOGO@name\endcsname
  }%
  \ifcase\HOLOGO@temp
    \penalty10000 %
  \fi
  \ltx@space
}
%    \end{macrocode}
%    \end{macro}
%
%    \begin{macro}{\HOLOGO@hyphen}
%    \begin{macrocode}
\def\HOLOGO@hyphen{%
  \ltx@ifundefined{HoLogoOpt@hyphenbreak@\HOLOGO@name}{%
    \ltx@ifundefined{HoLogoOpt@break@\HOLOGO@name}{%
      \chardef\HOLOGO@temp=\HOLOGOOPT@hyphenbreak
    }{%
      \chardef\HOLOGO@temp=%
        \csname HoLogoOpt@break@\HOLOGO@name\endcsname
    }%
  }{%
    \chardef\HOLOGO@temp=%
      \csname HoLogoOpt@hyphenbreak@\HOLOGO@name\endcsname
  }%
  \ifcase\HOLOGO@temp
    \ltx@mbox{-}%
  \else
    -%
  \fi
}
%    \end{macrocode}
%    \end{macro}
%
%    \begin{macro}{\HOLOGO@discretionary}
%    \begin{macrocode}
\def\HOLOGO@discretionary{%
  \ltx@ifundefined{HoLogoOpt@discretionarybreak@\HOLOGO@name}{%
    \ltx@ifundefined{HoLogoOpt@break@\HOLOGO@name}{%
      \chardef\HOLOGO@temp=\HOLOGOOPT@discretionarybreak
    }{%
      \chardef\HOLOGO@temp=%
        \csname HoLogoOpt@break@\HOLOGO@name\endcsname
    }%
  }{%
    \chardef\HOLOGO@temp=%
      \csname HoLogoOpt@discretionarybreak@\HOLOGO@name\endcsname
  }%
  \ifcase\HOLOGO@temp
  \else
    \-%
  \fi
}
%    \end{macrocode}
%    \end{macro}
%
%    \begin{macro}{\HOLOGO@mbox}
%    \begin{macrocode}
\def\HOLOGO@mbox#1{%
  \ltx@ifundefined{HoLogoOpt@break@\HOLOGO@name}{%
    \chardef\HOLOGO@temp=\HOLOGOOPT@hyphenbreak
  }{%
    \chardef\HOLOGO@temp=%
      \csname HoLogoOpt@break@\HOLOGO@name\endcsname
  }%
  \ifcase\HOLOGO@temp
    \ltx@mbox{#1}%
  \else
    #1%
  \fi
}
%    \end{macrocode}
%    \end{macro}
%
% \subsection{Font support}
%
%    \begin{macro}{\HoLogoFont@font}
%    \begin{tabular}{@{}ll@{}}
%    |#1|:& logo name\\
%    |#2|:& font short name\\
%    |#3|:& text
%    \end{tabular}
%    \begin{macrocode}
\def\HoLogoFont@font#1#2#3{%
  \begingroup
    \ltx@IfUndefined{HoLogoFont@logo@#1.#2}{%
      \ltx@IfUndefined{HoLogoFont@font@#2}{%
        \@PackageWarning{hologo}{%
          Missing font `#2' for logo `#1'%
        }%
        #3%
      }{%
        \csname HoLogoFont@font@#2\endcsname{#3}%
      }%
    }{%
      \csname HoLogoFont@logo@#1.#2\endcsname{#3}%
    }%
  \endgroup
}
%    \end{macrocode}
%    \end{macro}
%
%    \begin{macro}{\HoLogoFont@Def}
%    \begin{macrocode}
\def\HoLogoFont@Def#1{%
  \expandafter\def\csname HoLogoFont@font@#1\endcsname
}
%    \end{macrocode}
%    \end{macro}
%    \begin{macro}{\HoLogoFont@LogoDef}
%    \begin{macrocode}
\def\HoLogoFont@LogoDef#1#2{%
  \expandafter\def\csname HoLogoFont@logo@#1.#2\endcsname
}
%    \end{macrocode}
%    \end{macro}
%
% \subsubsection{Font defaults}
%
%    \begin{macro}{\HoLogoFont@font@general}
%    \begin{macrocode}
\HoLogoFont@Def{general}{}%
%    \end{macrocode}
%    \end{macro}
%
%    \begin{macro}{\HoLogoFont@font@rm}
%    \begin{macrocode}
\ltx@IfUndefined{rmfamily}{%
  \ltx@IfUndefined{rm}{%
  }{%
    \HoLogoFont@Def{rm}{\rm}%
  }%
}{%
  \HoLogoFont@Def{rm}{\rmfamily}%
}
%    \end{macrocode}
%    \end{macro}
%
%    \begin{macro}{\HoLogoFont@font@sf}
%    \begin{macrocode}
\ltx@IfUndefined{sffamily}{%
  \ltx@IfUndefined{sf}{%
  }{%
    \HoLogoFont@Def{sf}{\sf}%
  }%
}{%
  \HoLogoFont@Def{sf}{\sffamily}%
}
%    \end{macrocode}
%    \end{macro}
%
%    \begin{macro}{\HoLogoFont@font@bibsf}
%    In case of \hologo{plainTeX} the original small caps
%    variant is used as default. In \hologo{LaTeX}
%    the definition of package \xpackage{dtklogos} \cite{dtklogos}
%    is used.
%\begin{quote}
%\begin{verbatim}
%\DeclareRobustCommand{\BibTeX}{%
%  B%
%  \kern-.05em%
%  \hbox{%
%    $\m@th$% %% force math size calculations
%    \csname S@\f@size\endcsname
%    \fontsize\sf@size\z@
%    \math@fontsfalse
%    \selectfont
%    I%
%    \kern-.025em%
%    B
%  }%
%  \kern-.08em%
%  \-%
%  \TeX
%}
%\end{verbatim}
%\end{quote}
%    \begin{macrocode}
\ltx@IfUndefined{selectfont}{%
  \ltx@IfUndefined{tensc}{%
    \font\tensc=cmcsc10\relax
  }{}%
  \HoLogoFont@Def{bibsf}{\tensc}%
}{%
  \HoLogoFont@Def{bibsf}{%
    $\mathsurround=0pt$%
    \csname S@\f@size\endcsname
    \fontsize\sf@size{0pt}%
    \math@fontsfalse
    \selectfont
  }%
}
%    \end{macrocode}
%    \end{macro}
%
%    \begin{macro}{\HoLogoFont@font@sc}
%    \begin{macrocode}
\ltx@IfUndefined{scshape}{%
  \ltx@IfUndefined{tensc}{%
    \font\tensc=cmcsc10\relax
  }{}%
  \HoLogoFont@Def{sc}{\tensc}%
}{%
  \HoLogoFont@Def{sc}{\scshape}%
}
%    \end{macrocode}
%    \end{macro}
%
%    \begin{macro}{\HoLogoFont@font@sy}
%    \begin{macrocode}
\ltx@IfUndefined{usefont}{%
  \ltx@IfUndefined{tensy}{%
  }{%
    \HoLogoFont@Def{sy}{\tensy}%
  }%
}{%
  \HoLogoFont@Def{sy}{%
    \usefont{OMS}{cmsy}{m}{n}%
  }%
}
%    \end{macrocode}
%    \end{macro}
%
%    \begin{macro}{\HoLogoFont@font@logo}
%    \begin{macrocode}
\begingroup
  \def\x{LaTeX2e}%
\expandafter\endgroup
\ifx\fmtname\x
  \ltx@IfUndefined{logofamily}{%
    \DeclareRobustCommand\logofamily{%
      \not@math@alphabet\logofamily\relax
      \fontencoding{U}%
      \fontfamily{logo}%
      \selectfont
    }%
  }{}%
  \ltx@IfUndefined{logofamily}{%
  }{%
    \HoLogoFont@Def{logo}{\logofamily}%
  }%
\else
  \ltx@IfUndefined{tenlogo}{%
    \font\tenlogo=logo10\relax
  }{}%
  \HoLogoFont@Def{logo}{\tenlogo}%
\fi
%    \end{macrocode}
%    \end{macro}
%
% \subsubsection{Font setup}
%
%    \begin{macro}{\hologoFontSetup}
%    \begin{macrocode}
\def\hologoFontSetup{%
  \let\HOLOGO@name\relax
  \HOLOGO@FontSetup
}
%    \end{macrocode}
%    \end{macro}
%
%    \begin{macro}{\hologoLogoFontSetup}
%    \begin{macrocode}
\def\hologoLogoFontSetup#1{%
  \edef\HOLOGO@name{#1}%
  \ltx@IfUndefined{HoLogo@\HOLOGO@name}{%
    \@PackageError{hologo}{%
      Unknown logo `\HOLOGO@name'%
    }\@ehc
    \ltx@gobble
  }{%
    \HOLOGO@FontSetup
  }%
}
%    \end{macrocode}
%    \end{macro}
%
%    \begin{macro}{\HOLOGO@FontSetup}
%    \begin{macrocode}
\def\HOLOGO@FontSetup{%
  \kvsetkeys{HoLogoFont}%
}
%    \end{macrocode}
%    \end{macro}
%
%    \begin{macrocode}
\def\HOLOGO@temp#1{%
  \kv@define@key{HoLogoFont}{#1}{%
    \ifx\HOLOGO@name\relax
      \HoLogoFont@Def{#1}{##1}%
    \else
      \HoLogoFont@LogoDef\HOLOGO@name{#1}{##1}%
    \fi
  }%
}
\HOLOGO@temp{general}
\HOLOGO@temp{sf}
%    \end{macrocode}
%
% \subsection{Generic logo commands}
%
%    \begin{macrocode}
\HOLOGO@IfExists\hologo{%
  \@PackageError{hologo}{%
    \string\hologo\ltx@space is already defined.\MessageBreak
    Package loading is aborted%
  }\@ehc
  \HOLOGO@AtEnd
}%
\HOLOGO@IfExists\hologoRobust{%
  \@PackageError{hologo}{%
    \string\hologoRobust\ltx@space is already defined.\MessageBreak
    Package loading is aborted%
  }\@ehc
  \HOLOGO@AtEnd
}%
%    \end{macrocode}
%
% \subsubsection{\cs{hologo} and friends}
%
%    \begin{macrocode}
\ifluatex
  \expandafter\ltx@firstofone
\else
  \expandafter\ltx@gobble
\fi
{%
  \ltx@IfUndefined{ifincsname}{%
    \ifnum\luatexversion<36 %
      \expandafter\ltx@gobble
    \else
      \expandafter\ltx@firstofone
    \fi
    {%
      \begingroup
        \ifcase0%
            \directlua{%
              if tex.enableprimitives then %
                tex.enableprimitives('HOLOGO@', {'ifincsname'})%
              else %
                tex.print('1')%
              end%
            }%
            \ifx\HOLOGO@ifincsname\@undefined 1\fi%
            \relax
          \expandafter\ltx@firstofone
        \else
          \endgroup
          \expandafter\ltx@gobble
        \fi
        {%
          \global\let\ifincsname\HOLOGO@ifincsname
        }%
      \HOLOGO@temp
    }%
  }{}%
}
%    \end{macrocode}
%    \begin{macrocode}
\ltx@IfUndefined{ifincsname}{%
  \catcode`$=14 %
}{%
  \catcode`$=9 %
}
%    \end{macrocode}
%
%    \begin{macro}{\hologo}
%    \begin{macrocode}
\def\hologo#1{%
$ \ifincsname
$   \ltx@ifundefined{HoLogoCs@\HOLOGO@Variant{#1}}{%
$     #1%
$   }{%
$     \csname HoLogoCs@\HOLOGO@Variant{#1}\endcsname\ltx@firstoftwo
$   }%
$ \else
    \HOLOGO@IfExists\texorpdfstring\texorpdfstring\ltx@firstoftwo
    {%
      \hologoRobust{#1}%
    }{%
      \ltx@ifundefined{HoLogoBkm@\HOLOGO@Variant{#1}}{%
        \ltx@ifundefined{HoLogo@#1}{?#1?}{#1}%
      }{%
        \csname HoLogoBkm@\HOLOGO@Variant{#1}\endcsname
        \ltx@firstoftwo
      }%
    }%
$ \fi
}
%    \end{macrocode}
%    \end{macro}
%    \begin{macro}{\Hologo}
%    \begin{macrocode}
\def\Hologo#1{%
$ \ifincsname
$   \ltx@ifundefined{HoLogoCs@\HOLOGO@Variant{#1}}{%
$     #1%
$   }{%
$     \csname HoLogoCs@\HOLOGO@Variant{#1}\endcsname\ltx@secondoftwo
$   }%
$ \else
    \HOLOGO@IfExists\texorpdfstring\texorpdfstring\ltx@firstoftwo
    {%
      \HologoRobust{#1}%
    }{%
      \ltx@ifundefined{HoLogoBkm@\HOLOGO@Variant{#1}}{%
        \ltx@ifundefined{HoLogo@#1}{?#1?}{#1}%
      }{%
        \csname HoLogoBkm@\HOLOGO@Variant{#1}\endcsname
        \ltx@secondoftwo
      }%
    }%
$ \fi
}
%    \end{macrocode}
%    \end{macro}
%
%    \begin{macro}{\hologoVariant}
%    \begin{macrocode}
\def\hologoVariant#1#2{%
  \ifx\relax#2\relax
    \hologo{#1}%
  \else
$   \ifincsname
$     \ltx@ifundefined{HoLogoCs@#1@#2}{%
$       #1%
$     }{%
$       \csname HoLogoCs@#1@#2\endcsname\ltx@firstoftwo
$     }%
$   \else
      \HOLOGO@IfExists\texorpdfstring\texorpdfstring\ltx@firstoftwo
      {%
        \hologoVariantRobust{#1}{#2}%
      }{%
        \ltx@ifundefined{HoLogoBkm@#1@#2}{%
          \ltx@ifundefined{HoLogo@#1}{?#1?}{#1}%
        }{%
          \csname HoLogoBkm@#1@#2\endcsname
          \ltx@firstoftwo
        }%
      }%
$   \fi
  \fi
}
%    \end{macrocode}
%    \end{macro}
%    \begin{macro}{\HologoVariant}
%    \begin{macrocode}
\def\HologoVariant#1#2{%
  \ifx\relax#2\relax
    \Hologo{#1}%
  \else
$   \ifincsname
$     \ltx@ifundefined{HoLogoCs@#1@#2}{%
$       #1%
$     }{%
$       \csname HoLogoCs@#1@#2\endcsname\ltx@secondoftwo
$     }%
$   \else
      \HOLOGO@IfExists\texorpdfstring\texorpdfstring\ltx@firstoftwo
      {%
        \HologoVariantRobust{#1}{#2}%
      }{%
        \ltx@ifundefined{HoLogoBkm@#1@#2}{%
          \ltx@ifundefined{HoLogo@#1}{?#1?}{#1}%
        }{%
          \csname HoLogoBkm@#1@#2\endcsname
          \ltx@secondoftwo
        }%
      }%
$   \fi
  \fi
}
%    \end{macrocode}
%    \end{macro}
%
%    \begin{macrocode}
\catcode`\$=3 %
%    \end{macrocode}
%
% \subsubsection{\cs{hologoRobust} and friends}
%
%    \begin{macro}{\hologoRobust}
%    \begin{macrocode}
\ltx@IfUndefined{protected}{%
  \ltx@IfUndefined{DeclareRobustCommand}{%
    \def\hologoRobust#1%
  }{%
    \DeclareRobustCommand*\hologoRobust[1]%
  }%
}{%
  \protected\def\hologoRobust#1%
}%
{%
  \edef\HOLOGO@name{#1}%
  \ltx@IfUndefined{HoLogo@\HOLOGO@Variant\HOLOGO@name}{%
    \@PackageError{hologo}{%
      Unknown logo `\HOLOGO@name'%
    }\@ehc
    ?\HOLOGO@name?%
  }{%
    \ltx@IfUndefined{ver@tex4ht.sty}{%
      \HoLogoFont@font\HOLOGO@name{general}{%
        \csname HoLogo@\HOLOGO@Variant\HOLOGO@name\endcsname
        \ltx@firstoftwo
      }%
    }{%
      \ltx@IfUndefined{HoLogoHtml@\HOLOGO@Variant\HOLOGO@name}{%
        \HOLOGO@name
      }{%
        \csname HoLogoHtml@\HOLOGO@Variant\HOLOGO@name\endcsname
        \ltx@firstoftwo
      }%
    }%
  }%
}
%    \end{macrocode}
%    \end{macro}
%    \begin{macro}{\HologoRobust}
%    \begin{macrocode}
\ltx@IfUndefined{protected}{%
  \ltx@IfUndefined{DeclareRobustCommand}{%
    \def\HologoRobust#1%
  }{%
    \DeclareRobustCommand*\HologoRobust[1]%
  }%
}{%
  \protected\def\HologoRobust#1%
}%
{%
  \edef\HOLOGO@name{#1}%
  \ltx@IfUndefined{HoLogo@\HOLOGO@Variant\HOLOGO@name}{%
    \@PackageError{hologo}{%
      Unknown logo `\HOLOGO@name'%
    }\@ehc
    ?\HOLOGO@name?%
  }{%
    \ltx@IfUndefined{ver@tex4ht.sty}{%
      \HoLogoFont@font\HOLOGO@name{general}{%
        \csname HoLogo@\HOLOGO@Variant\HOLOGO@name\endcsname
        \ltx@secondoftwo
      }%
    }{%
      \ltx@IfUndefined{HoLogoHtml@\HOLOGO@Variant\HOLOGO@name}{%
        \expandafter\HOLOGO@Uppercase\HOLOGO@name
      }{%
        \csname HoLogoHtml@\HOLOGO@Variant\HOLOGO@name\endcsname
        \ltx@secondoftwo
      }%
    }%
  }%
}
%    \end{macrocode}
%    \end{macro}
%    \begin{macro}{\hologoVariantRobust}
%    \begin{macrocode}
\ltx@IfUndefined{protected}{%
  \ltx@IfUndefined{DeclareRobustCommand}{%
    \def\hologoVariantRobust#1#2%
  }{%
    \DeclareRobustCommand*\hologoVariantRobust[2]%
  }%
}{%
  \protected\def\hologoVariantRobust#1#2%
}%
{%
  \begingroup
    \hologoLogoSetup{#1}{variant={#2}}%
    \hologoRobust{#1}%
  \endgroup
}
%    \end{macrocode}
%    \end{macro}
%    \begin{macro}{\HologoVariantRobust}
%    \begin{macrocode}
\ltx@IfUndefined{protected}{%
  \ltx@IfUndefined{DeclareRobustCommand}{%
    \def\HologoVariantRobust#1#2%
  }{%
    \DeclareRobustCommand*\HologoVariantRobust[2]%
  }%
}{%
  \protected\def\HologoVariantRobust#1#2%
}%
{%
  \begingroup
    \hologoLogoSetup{#1}{variant={#2}}%
    \HologoRobust{#1}%
  \endgroup
}
%    \end{macrocode}
%    \end{macro}
%
%    \begin{macro}{\hologorobust}
%    Macro \cs{hologorobust} is only defined for compatibility.
%    Its use is deprecated.
%    \begin{macrocode}
\def\hologorobust{\hologoRobust}
%    \end{macrocode}
%    \end{macro}
%
% \subsection{Helpers}
%
%    \begin{macro}{\HOLOGO@Uppercase}
%    Macro \cs{HOLOGO@Uppercase} is restricted to \cs{uppercase},
%    because \hologo{plainTeX} or \hologo{iniTeX} do not provide
%    \cs{MakeUppercase}.
%    \begin{macrocode}
\def\HOLOGO@Uppercase#1{\uppercase{#1}}
%    \end{macrocode}
%    \end{macro}
%
%    \begin{macro}{\HOLOGO@PdfdocUnicode}
%    \begin{macrocode}
\def\HOLOGO@PdfdocUnicode{%
  \ifx\ifHy@unicode\iftrue
    \expandafter\ltx@secondoftwo
  \else
    \expandafter\ltx@firstoftwo
  \fi
}
%    \end{macrocode}
%    \end{macro}
%
%    \begin{macro}{\HOLOGO@Math}
%    \begin{macrocode}
\def\HOLOGO@MathSetup{%
  \mathsurround0pt\relax
  \HOLOGO@IfExists\f@series{%
    \if b\expandafter\ltx@car\f@series x\@nil
      \csname boldmath\endcsname
   \fi
  }{}%
}
%    \end{macrocode}
%    \end{macro}
%
%    \begin{macro}{\HOLOGO@TempDimen}
%    \begin{macrocode}
\dimendef\HOLOGO@TempDimen=\ltx@zero
%    \end{macrocode}
%    \end{macro}
%    \begin{macro}{\HOLOGO@NegativeKerning}
%    \begin{macrocode}
\def\HOLOGO@NegativeKerning#1{%
  \begingroup
    \HOLOGO@TempDimen=0pt\relax
    \comma@parse@normalized{#1}{%
      \ifdim\HOLOGO@TempDimen=0pt %
        \expandafter\HOLOGO@@NegativeKerning\comma@entry
      \fi
      \ltx@gobble
    }%
    \ifdim\HOLOGO@TempDimen<0pt %
      \kern\HOLOGO@TempDimen
    \fi
  \endgroup
}
%    \end{macrocode}
%    \end{macro}
%    \begin{macro}{\HOLOGO@@NegativeKerning}
%    \begin{macrocode}
\def\HOLOGO@@NegativeKerning#1#2{%
  \setbox\ltx@zero\hbox{#1#2}%
  \HOLOGO@TempDimen=\wd\ltx@zero
  \setbox\ltx@zero\hbox{#1\kern0pt#2}%
  \advance\HOLOGO@TempDimen by -\wd\ltx@zero
}
%    \end{macrocode}
%    \end{macro}
%
%    \begin{macro}{\HOLOGO@SpaceFactor}
%    \begin{macrocode}
\def\HOLOGO@SpaceFactor{%
  \spacefactor1000 %
}
%    \end{macrocode}
%    \end{macro}
%
%    \begin{macro}{\HOLOGO@Span}
%    \begin{macrocode}
\def\HOLOGO@Span#1#2{%
  \HCode{<span class="HoLogo-#1">}%
  #2%
  \HCode{</span>}%
}
%    \end{macrocode}
%    \end{macro}
%
% \subsubsection{Text subscript}
%
%    \begin{macro}{\HOLOGO@SubScript}%
%    \begin{macrocode}
\def\HOLOGO@SubScript#1{%
  \ltx@IfUndefined{textsubscript}{%
    \ltx@IfUndefined{text}{%
      \ltx@mbox{%
        \mathsurround=0pt\relax
        $%
          _{%
            \ltx@IfUndefined{sf@size}{%
              \mathrm{#1}%
            }{%
              \mbox{%
                \fontsize\sf@size{0pt}\selectfont
                #1%
              }%
            }%
          }%
        $%
      }%
    }{%
      \ltx@mbox{%
        \mathsurround=0pt\relax
        $_{\text{#1}}$%
      }%
    }%
  }{%
    \textsubscript{#1}%
  }%
}
%    \end{macrocode}
%    \end{macro}
%
% \subsection{\hologo{TeX} and friends}
%
% \subsubsection{\hologo{TeX}}
%
%    \begin{macro}{\HoLogo@TeX}
%    Source: \hologo{LaTeX} kernel.
%    \begin{macrocode}
\def\HoLogo@TeX#1{%
  T\kern-.1667em\lower.5ex\hbox{E}\kern-.125emX\HOLOGO@SpaceFactor
}
%    \end{macrocode}
%    \end{macro}
%    \begin{macro}{\HoLogoHtml@TeX}
%    \begin{macrocode}
\def\HoLogoHtml@TeX#1{%
  \HoLogoCss@TeX
  \HOLOGO@Span{TeX}{%
    T%
    \HOLOGO@Span{e}{%
      E%
    }%
    X%
  }%
}
%    \end{macrocode}
%    \end{macro}
%    \begin{macro}{\HoLogoCss@TeX}
%    \begin{macrocode}
\def\HoLogoCss@TeX{%
  \Css{%
    span.HoLogo-TeX span.HoLogo-e{%
      position:relative;%
      top:.5ex;%
      margin-left:-.1667em;%
      margin-right:-.125em;%
    }%
  }%
  \Css{%
    a span.HoLogo-TeX span.HoLogo-e{%
      text-decoration:none;%
    }%
  }%
  \global\let\HoLogoCss@TeX\relax
}
%    \end{macrocode}
%    \end{macro}
%
% \subsubsection{\hologo{plainTeX}}
%
%    \begin{macro}{\HoLogo@plainTeX@space}
%    Source: ``The \hologo{TeX}book''
%    \begin{macrocode}
\def\HoLogo@plainTeX@space#1{%
  \HOLOGO@mbox{#1{p}{P}lain}\HOLOGO@space\hologo{TeX}%
}
%    \end{macrocode}
%    \end{macro}
%    \begin{macro}{\HoLogoCs@plainTeX@space}
%    \begin{macrocode}
\def\HoLogoCs@plainTeX@space#1{#1{p}{P}lain TeX}%
%    \end{macrocode}
%    \end{macro}
%    \begin{macro}{\HoLogoBkm@plainTeX@space}
%    \begin{macrocode}
\def\HoLogoBkm@plainTeX@space#1{%
  #1{p}{P}lain \hologo{TeX}%
}
%    \end{macrocode}
%    \end{macro}
%    \begin{macro}{\HoLogoHtml@plainTeX@space}
%    \begin{macrocode}
\def\HoLogoHtml@plainTeX@space#1{%
  #1{p}{P}lain \hologo{TeX}%
}
%    \end{macrocode}
%    \end{macro}
%
%    \begin{macro}{\HoLogo@plainTeX@hyphen}
%    \begin{macrocode}
\def\HoLogo@plainTeX@hyphen#1{%
  \HOLOGO@mbox{#1{p}{P}lain}\HOLOGO@hyphen\hologo{TeX}%
}
%    \end{macrocode}
%    \end{macro}
%    \begin{macro}{\HoLogoCs@plainTeX@hyphen}
%    \begin{macrocode}
\def\HoLogoCs@plainTeX@hyphen#1{#1{p}{P}lain-TeX}
%    \end{macrocode}
%    \end{macro}
%    \begin{macro}{\HoLogoBkm@plainTeX@hyphen}
%    \begin{macrocode}
\def\HoLogoBkm@plainTeX@hyphen#1{%
  #1{p}{P}lain-\hologo{TeX}%
}
%    \end{macrocode}
%    \end{macro}
%    \begin{macro}{\HoLogoHtml@plainTeX@hyphen}
%    \begin{macrocode}
\def\HoLogoHtml@plainTeX@hyphen#1{%
  #1{p}{P}lain-\hologo{TeX}%
}
%    \end{macrocode}
%    \end{macro}
%
%    \begin{macro}{\HoLogo@plainTeX@runtogether}
%    \begin{macrocode}
\def\HoLogo@plainTeX@runtogether#1{%
  \HOLOGO@mbox{#1{p}{P}lain\hologo{TeX}}%
}
%    \end{macrocode}
%    \end{macro}
%    \begin{macro}{\HoLogoCs@plainTeX@runtogether}
%    \begin{macrocode}
\def\HoLogoCs@plainTeX@runtogether#1{#1{p}{P}lainTeX}
%    \end{macrocode}
%    \end{macro}
%    \begin{macro}{\HoLogoBkm@plainTeX@runtogether}
%    \begin{macrocode}
\def\HoLogoBkm@plainTeX@runtogether#1{%
  #1{p}{P}lain\hologo{TeX}%
}
%    \end{macrocode}
%    \end{macro}
%    \begin{macro}{\HoLogoHtml@plainTeX@runtogether}
%    \begin{macrocode}
\def\HoLogoHtml@plainTeX@runtogether#1{%
  #1{p}{P}lain\hologo{TeX}%
}
%    \end{macrocode}
%    \end{macro}
%
%    \begin{macro}{\HoLogo@plainTeX}
%    \begin{macrocode}
\def\HoLogo@plainTeX{\HoLogo@plainTeX@space}
%    \end{macrocode}
%    \end{macro}
%    \begin{macro}{\HoLogoCs@plainTeX}
%    \begin{macrocode}
\def\HoLogoCs@plainTeX{\HoLogoCs@plainTeX@space}
%    \end{macrocode}
%    \end{macro}
%    \begin{macro}{\HoLogoBkm@plainTeX}
%    \begin{macrocode}
\def\HoLogoBkm@plainTeX{\HoLogoBkm@plainTeX@space}
%    \end{macrocode}
%    \end{macro}
%    \begin{macro}{\HoLogoHtml@plainTeX}
%    \begin{macrocode}
\def\HoLogoHtml@plainTeX{\HoLogoHtml@plainTeX@space}
%    \end{macrocode}
%    \end{macro}
%
% \subsubsection{\hologo{LaTeX}}
%
%    Source: \hologo{LaTeX} kernel.
%\begin{quote}
%\begin{verbatim}
%\DeclareRobustCommand{\LaTeX}{%
%  L%
%  \kern-.36em%
%  {%
%    \sbox\z@ T%
%    \vbox to\ht\z@{%
%      \hbox{%
%        \check@mathfonts
%        \fontsize\sf@size\z@
%        \math@fontsfalse
%        \selectfont
%        A%
%      }%
%      \vss
%    }%
%  }%
%  \kern-.15em%
%  \TeX
%}
%\end{verbatim}
%\end{quote}
%
%    \begin{macro}{\HoLogo@La}
%    \begin{macrocode}
\def\HoLogo@La#1{%
  L%
  \kern-.36em%
  \begingroup
    \setbox\ltx@zero\hbox{T}%
    \vbox to\ht\ltx@zero{%
      \hbox{%
        \ltx@ifundefined{check@mathfonts}{%
          \csname sevenrm\endcsname
        }{%
          \check@mathfonts
          \fontsize\sf@size{0pt}%
          \math@fontsfalse\selectfont
        }%
        A%
      }%
      \vss
    }%
  \endgroup
}
%    \end{macrocode}
%    \end{macro}
%
%    \begin{macro}{\HoLogo@LaTeX}
%    Source: \hologo{LaTeX} kernel.
%    \begin{macrocode}
\def\HoLogo@LaTeX#1{%
  \hologo{La}%
  \kern-.15em%
  \hologo{TeX}%
}
%    \end{macrocode}
%    \end{macro}
%    \begin{macro}{\HoLogoHtml@LaTeX}
%    \begin{macrocode}
\def\HoLogoHtml@LaTeX#1{%
  \HoLogoCss@LaTeX
  \HOLOGO@Span{LaTeX}{%
    L%
    \HOLOGO@Span{a}{%
      A%
    }%
    \hologo{TeX}%
  }%
}
%    \end{macrocode}
%    \end{macro}
%    \begin{macro}{\HoLogoCss@LaTeX}
%    \begin{macrocode}
\def\HoLogoCss@LaTeX{%
  \Css{%
    span.HoLogo-LaTeX span.HoLogo-a{%
      position:relative;%
      top:-.5ex;%
      margin-left:-.36em;%
      margin-right:-.15em;%
      font-size:85\%;%
    }%
  }%
  \global\let\HoLogoCss@LaTeX\relax
}
%    \end{macrocode}
%    \end{macro}
%
% \subsubsection{\hologo{(La)TeX}}
%
%    \begin{macro}{\HoLogo@LaTeXTeX}
%    The kerning around the parentheses is taken
%    from package \xpackage{dtklogos} \cite{dtklogos}.
%\begin{quote}
%\begin{verbatim}
%\DeclareRobustCommand{\LaTeXTeX}{%
%  (%
%  \kern-.15em%
%  L%
%  \kern-.36em%
%  {%
%    \sbox\z@ T%
%    \vbox to\ht0{%
%      \hbox{%
%        $\m@th$%
%        \csname S@\f@size\endcsname
%        \fontsize\sf@size\z@
%        \math@fontsfalse
%        \selectfont
%        A%
%      }%
%      \vss
%    }%
%  }%
%  \kern-.2em%
%  )%
%  \kern-.15em%
%  \TeX
%}
%\end{verbatim}
%\end{quote}
%    \begin{macrocode}
\def\HoLogo@LaTeXTeX#1{%
  (%
  \kern-.15em%
  \hologo{La}%
  \kern-.2em%
  )%
  \kern-.15em%
  \hologo{TeX}%
}
%    \end{macrocode}
%    \end{macro}
%    \begin{macro}{\HoLogoBkm@LaTeXTeX}
%    \begin{macrocode}
\def\HoLogoBkm@LaTeXTeX#1{(La)TeX}
%    \end{macrocode}
%    \end{macro}
%
%    \begin{macro}{\HoLogo@(La)TeX}
%    \begin{macrocode}
\expandafter
\let\csname HoLogo@(La)TeX\endcsname\HoLogo@LaTeXTeX
%    \end{macrocode}
%    \end{macro}
%    \begin{macro}{\HoLogoBkm@(La)TeX}
%    \begin{macrocode}
\expandafter
\let\csname HoLogoBkm@(La)TeX\endcsname\HoLogoBkm@LaTeXTeX
%    \end{macrocode}
%    \end{macro}
%    \begin{macro}{\HoLogoHtml@LaTeXTeX}
%    \begin{macrocode}
\def\HoLogoHtml@LaTeXTeX#1{%
  \HoLogoCss@LaTeXTeX
  \HOLOGO@Span{LaTeXTeX}{%
    (%
    \HOLOGO@Span{L}{L}%
    \HOLOGO@Span{a}{A}%
    \HOLOGO@Span{ParenRight}{)}%
    \hologo{TeX}%
  }%
}
%    \end{macrocode}
%    \end{macro}
%    \begin{macro}{\HoLogoHtml@(La)TeX}
%    Kerning after opening parentheses and before closing parentheses
%    is $-0.1$\,em. The original values $-0.15$\,em
%    looked too ugly for a serif font.
%    \begin{macrocode}
\expandafter
\let\csname HoLogoHtml@(La)TeX\endcsname\HoLogoHtml@LaTeXTeX
%    \end{macrocode}
%    \end{macro}
%    \begin{macro}{\HoLogoCss@LaTeXTeX}
%    \begin{macrocode}
\def\HoLogoCss@LaTeXTeX{%
  \Css{%
    span.HoLogo-LaTeXTeX span.HoLogo-L{%
      margin-left:-.1em;%
    }%
  }%
  \Css{%
    span.HoLogo-LaTeXTeX span.HoLogo-a{%
      position:relative;%
      top:-.5ex;%
      margin-left:-.36em;%
      margin-right:-.1em;%
      font-size:85\%;%
    }%
  }%
  \Css{%
    span.HoLogo-LaTeXTeX span.HoLogo-ParenRight{%
      margin-right:-.15em;%
    }%
  }%
  \global\let\HoLogoCss@LaTeXTeX\relax
}
%    \end{macrocode}
%    \end{macro}
%
% \subsubsection{\hologo{LaTeXe}}
%
%    \begin{macro}{\HoLogo@LaTeXe}
%    Source: \hologo{LaTeX} kernel
%    \begin{macrocode}
\def\HoLogo@LaTeXe#1{%
  \hologo{LaTeX}%
  \kern.15em%
  \hbox{%
    \HOLOGO@MathSetup
    2%
    $_{\textstyle\varepsilon}$%
  }%
}
%    \end{macrocode}
%    \end{macro}
%
%    \begin{macro}{\HoLogoCs@LaTeXe}
%    \begin{macrocode}
\ifnum64=`\^^^^0040\relax % test for big chars of LuaTeX/XeTeX
  \catcode`\$=9 %
  \catcode`\&=14 %
\else
  \catcode`\$=14 %
  \catcode`\&=9 %
\fi
\def\HoLogoCs@LaTeXe#1{%
  LaTeX2%
$ \string ^^^^0395%
& e%
}%
\catcode`\$=3 %
\catcode`\&=4 %
%    \end{macrocode}
%    \end{macro}
%
%    \begin{macro}{\HoLogoBkm@LaTeXe}
%    \begin{macrocode}
\def\HoLogoBkm@LaTeXe#1{%
  \hologo{LaTeX}%
  2%
  \HOLOGO@PdfdocUnicode{e}{\textepsilon}%
}
%    \end{macrocode}
%    \end{macro}
%
%    \begin{macro}{\HoLogoHtml@LaTeXe}
%    \begin{macrocode}
\def\HoLogoHtml@LaTeXe#1{%
  \HoLogoCss@LaTeXe
  \HOLOGO@Span{LaTeX2e}{%
    \hologo{LaTeX}%
    \HOLOGO@Span{2}{2}%
    \HOLOGO@Span{e}{%
      \HOLOGO@MathSetup
      \ensuremath{\textstyle\varepsilon}%
    }%
  }%
}
%    \end{macrocode}
%    \end{macro}
%    \begin{macro}{\HoLogoCss@LaTeXe}
%    \begin{macrocode}
\def\HoLogoCss@LaTeXe{%
  \Css{%
    span.HoLogo-LaTeX2e span.HoLogo-2{%
      padding-left:.15em;%
    }%
  }%
  \Css{%
    span.HoLogo-LaTeX2e span.HoLogo-e{%
      position:relative;%
      top:.35ex;%
      text-decoration:none;%
    }%
  }%
  \global\let\HoLogoCss@LaTeXe\relax
}
%    \end{macrocode}
%    \end{macro}
%
%    \begin{macro}{\HoLogo@LaTeX2e}
%    \begin{macrocode}
\expandafter
\let\csname HoLogo@LaTeX2e\endcsname\HoLogo@LaTeXe
%    \end{macrocode}
%    \end{macro}
%    \begin{macro}{\HoLogoCs@LaTeX2e}
%    \begin{macrocode}
\expandafter
\let\csname HoLogoCs@LaTeX2e\endcsname\HoLogoCs@LaTeXe
%    \end{macrocode}
%    \end{macro}
%    \begin{macro}{\HoLogoBkm@LaTeX2e}
%    \begin{macrocode}
\expandafter
\let\csname HoLogoBkm@LaTeX2e\endcsname\HoLogoBkm@LaTeXe
%    \end{macrocode}
%    \end{macro}
%    \begin{macro}{\HoLogoHtml@LaTeX2e}
%    \begin{macrocode}
\expandafter
\let\csname HoLogoHtml@LaTeX2e\endcsname\HoLogoHtml@LaTeXe
%    \end{macrocode}
%    \end{macro}
%
% \subsubsection{\hologo{LaTeX3}}
%
%    \begin{macro}{\HoLogo@LaTeX3}
%    Source: \hologo{LaTeX} kernel
%    \begin{macrocode}
\expandafter\def\csname HoLogo@LaTeX3\endcsname#1{%
  \hologo{LaTeX}%
  3%
}
%    \end{macrocode}
%    \end{macro}
%
%    \begin{macro}{\HoLogoBkm@LaTeX3}
%    \begin{macrocode}
\expandafter\def\csname HoLogoBkm@LaTeX3\endcsname#1{%
  \hologo{LaTeX}%
  3%
}
%    \end{macrocode}
%    \end{macro}
%    \begin{macro}{\HoLogoHtml@LaTeX3}
%    \begin{macrocode}
\expandafter
\let\csname HoLogoHtml@LaTeX3\expandafter\endcsname
\csname HoLogo@LaTeX3\endcsname
%    \end{macrocode}
%    \end{macro}
%
% \subsubsection{\hologo{LaTeXML}}
%
%    \begin{macro}{\HoLogo@LaTeXML}
%    \begin{macrocode}
\def\HoLogo@LaTeXML#1{%
  \HOLOGO@mbox{%
    \hologo{La}%
    \kern-.15em%
    T%
    \kern-.1667em%
    \lower.5ex\hbox{E}%
    \kern-.125em%
    \HoLogoFont@font{LaTeXML}{sc}{xml}%
  }%
}
%    \end{macrocode}
%    \end{macro}
%    \begin{macro}{\HoLogoHtml@pdfLaTeX}
%    \begin{macrocode}
\def\HoLogoHtml@LaTeXML#1{%
  \HOLOGO@Span{LaTeXML}{%
    \HoLogoCss@LaTeX
    \HoLogoCss@TeX
    \HOLOGO@Span{LaTeX}{%
      L%
      \HOLOGO@Span{a}{%
        A%
      }%
    }%
    \HOLOGO@Span{TeX}{%
      T%
      \HOLOGO@Span{e}{%
        E%
      }%
    }%
    \HCode{<span style="font-variant: small-caps;">}%
    xml%
    \HCode{</span>}%
  }%
}
%    \end{macrocode}
%    \end{macro}
%
% \subsubsection{\hologo{eTeX}}
%
%    \begin{macro}{\HoLogo@eTeX}
%    Source: package \xpackage{etex}
%    \begin{macrocode}
\def\HoLogo@eTeX#1{%
  \ltx@mbox{%
    \HOLOGO@MathSetup
    $\varepsilon$%
    -%
    \HOLOGO@NegativeKerning{-T,T-,To}%
    \hologo{TeX}%
  }%
}
%    \end{macrocode}
%    \end{macro}
%    \begin{macro}{\HoLogoCs@eTeX}
%    \begin{macrocode}
\ifnum64=`\^^^^0040\relax % test for big chars of LuaTeX/XeTeX
  \catcode`\$=9 %
  \catcode`\&=14 %
\else
  \catcode`\$=14 %
  \catcode`\&=9 %
\fi
\def\HoLogoCs@eTeX#1{%
$ #1{\string ^^^^0395}{\string ^^^^03b5}%
& #1{e}{E}%
  TeX%
}%
\catcode`\$=3 %
\catcode`\&=4 %
%    \end{macrocode}
%    \end{macro}
%    \begin{macro}{\HoLogoBkm@eTeX}
%    \begin{macrocode}
\def\HoLogoBkm@eTeX#1{%
  \HOLOGO@PdfdocUnicode{#1{e}{E}}{\textepsilon}%
  -%
  \hologo{TeX}%
}
%    \end{macrocode}
%    \end{macro}
%    \begin{macro}{\HoLogoHtml@eTeX}
%    \begin{macrocode}
\def\HoLogoHtml@eTeX#1{%
  \ltx@mbox{%
    \HOLOGO@MathSetup
    $\varepsilon$%
    -%
    \hologo{TeX}%
  }%
}
%    \end{macrocode}
%    \end{macro}
%
% \subsubsection{\hologo{iniTeX}}
%
%    \begin{macro}{\HoLogo@iniTeX}
%    \begin{macrocode}
\def\HoLogo@iniTeX#1{%
  \HOLOGO@mbox{%
    #1{i}{I}ni\hologo{TeX}%
  }%
}
%    \end{macrocode}
%    \end{macro}
%    \begin{macro}{\HoLogoCs@iniTeX}
%    \begin{macrocode}
\def\HoLogoCs@iniTeX#1{#1{i}{I}niTeX}
%    \end{macrocode}
%    \end{macro}
%    \begin{macro}{\HoLogoBkm@iniTeX}
%    \begin{macrocode}
\def\HoLogoBkm@iniTeX#1{%
  #1{i}{I}ni\hologo{TeX}%
}
%    \end{macrocode}
%    \end{macro}
%    \begin{macro}{\HoLogoHtml@iniTeX}
%    \begin{macrocode}
\let\HoLogoHtml@iniTeX\HoLogo@iniTeX
%    \end{macrocode}
%    \end{macro}
%
% \subsubsection{\hologo{virTeX}}
%
%    \begin{macro}{\HoLogo@virTeX}
%    \begin{macrocode}
\def\HoLogo@virTeX#1{%
  \HOLOGO@mbox{%
    #1{v}{V}ir\hologo{TeX}%
  }%
}
%    \end{macrocode}
%    \end{macro}
%    \begin{macro}{\HoLogoCs@virTeX}
%    \begin{macrocode}
\def\HoLogoCs@virTeX#1{#1{v}{V}irTeX}
%    \end{macrocode}
%    \end{macro}
%    \begin{macro}{\HoLogoBkm@virTeX}
%    \begin{macrocode}
\def\HoLogoBkm@virTeX#1{%
  #1{v}{V}ir\hologo{TeX}%
}
%    \end{macrocode}
%    \end{macro}
%    \begin{macro}{\HoLogoHtml@virTeX}
%    \begin{macrocode}
\let\HoLogoHtml@virTeX\HoLogo@virTeX
%    \end{macrocode}
%    \end{macro}
%
% \subsubsection{\hologo{SliTeX}}
%
% \paragraph{Definitions of the three variants.}
%
%    \begin{macro}{\HoLogo@SLiTeX@lift}
%    \begin{macrocode}
\def\HoLogo@SLiTeX@lift#1{%
  \HoLogoFont@font{SliTeX}{rm}{%
    S%
    \kern-.06em%
    L%
    \kern-.18em%
    \raise.32ex\hbox{\HoLogoFont@font{SliTeX}{sc}{i}}%
    \HOLOGO@discretionary
    \kern-.06em%
    \hologo{TeX}%
  }%
}
%    \end{macrocode}
%    \end{macro}
%    \begin{macro}{\HoLogoBkm@SLiTeX@lift}
%    \begin{macrocode}
\def\HoLogoBkm@SLiTeX@lift#1{SLiTeX}
%    \end{macrocode}
%    \end{macro}
%    \begin{macro}{\HoLogoHtml@SLiTeX@lift}
%    \begin{macrocode}
\def\HoLogoHtml@SLiTeX@lift#1{%
  \HoLogoCss@SLiTeX@lift
  \HOLOGO@Span{SLiTeX-lift}{%
    \HoLogoFont@font{SliTeX}{rm}{%
      S%
      \HOLOGO@Span{L}{L}%
      \HOLOGO@Span{i}{i}%
      \hologo{TeX}%
    }%
  }%
}
%    \end{macrocode}
%    \end{macro}
%    \begin{macro}{\HoLogoCss@SLiTeX@lift}
%    \begin{macrocode}
\def\HoLogoCss@SLiTeX@lift{%
  \Css{%
    span.HoLogo-SLiTeX-lift span.HoLogo-L{%
      margin-left:-.06em;%
      margin-right:-.18em;%
    }%
  }%
  \Css{%
    span.HoLogo-SLiTeX-lift span.HoLogo-i{%
      position:relative;%
      top:-.32ex;%
      margin-right:-.06em;%
      font-variant:small-caps;%
    }%
  }%
  \global\let\HoLogoCss@SLiTeX@lift\relax
}
%    \end{macrocode}
%    \end{macro}
%
%    \begin{macro}{\HoLogo@SliTeX@simple}
%    \begin{macrocode}
\def\HoLogo@SliTeX@simple#1{%
  \HoLogoFont@font{SliTeX}{rm}{%
    \ltx@mbox{%
      \HoLogoFont@font{SliTeX}{sc}{Sli}%
    }%
    \HOLOGO@discretionary
    \hologo{TeX}%
  }%
}
%    \end{macrocode}
%    \end{macro}
%    \begin{macro}{\HoLogoBkm@SliTeX@simple}
%    \begin{macrocode}
\def\HoLogoBkm@SliTeX@simple#1{SliTeX}
%    \end{macrocode}
%    \end{macro}
%    \begin{macro}{\HoLogoHtml@SliTeX@simple}
%    \begin{macrocode}
\let\HoLogoHtml@SliTeX@simple\HoLogo@SliTeX@simple
%    \end{macrocode}
%    \end{macro}
%
%    \begin{macro}{\HoLogo@SliTeX@narrow}
%    \begin{macrocode}
\def\HoLogo@SliTeX@narrow#1{%
  \HoLogoFont@font{SliTeX}{rm}{%
    \ltx@mbox{%
      S%
      \kern-.06em%
      \HoLogoFont@font{SliTeX}{sc}{%
        l%
        \kern-.035em%
        i%
      }%
    }%
    \HOLOGO@discretionary
    \kern-.06em%
    \hologo{TeX}%
  }%
}
%    \end{macrocode}
%    \end{macro}
%    \begin{macro}{\HoLogoBkm@SliTeX@narrow}
%    \begin{macrocode}
\def\HoLogoBkm@SliTeX@narrow#1{SliTeX}
%    \end{macrocode}
%    \end{macro}
%    \begin{macro}{\HoLogoHtml@SliTeX@narrow}
%    \begin{macrocode}
\def\HoLogoHtml@SliTeX@narrow#1{%
  \HoLogoCss@SliTeX@narrow
  \HOLOGO@Span{SliTeX-narrow}{%
    \HoLogoFont@font{SliTeX}{rm}{%
      S%
        \HOLOGO@Span{l}{l}%
        \HOLOGO@Span{i}{i}%
      \hologo{TeX}%
    }%
  }%
}
%    \end{macrocode}
%    \end{macro}
%    \begin{macro}{\HoLogoCss@SliTeX@narrow}
%    \begin{macrocode}
\def\HoLogoCss@SliTeX@narrow{%
  \Css{%
    span.HoLogo-SliTeX-narrow span.HoLogo-l{%
      margin-left:-.06em;%
      margin-right:-.035em;%
      font-variant:small-caps;%
    }%
  }%
  \Css{%
    span.HoLogo-SliTeX-narrow span.HoLogo-i{%
      margin-right:-.06em;%
      font-variant:small-caps;%
    }%
  }%
  \global\let\HoLogoCss@SliTeX@narrow\relax
}
%    \end{macrocode}
%    \end{macro}
%
% \paragraph{Macro set completion.}
%
%    \begin{macro}{\HoLogo@SLiTeX@simple}
%    \begin{macrocode}
\def\HoLogo@SLiTeX@simple{\HoLogo@SliTeX@simple}
%    \end{macrocode}
%    \end{macro}
%    \begin{macro}{\HoLogoBkm@SLiTeX@simple}
%    \begin{macrocode}
\def\HoLogoBkm@SLiTeX@simple{\HoLogoBkm@SliTeX@simple}
%    \end{macrocode}
%    \end{macro}
%    \begin{macro}{\HoLogoHtml@SLiTeX@simple}
%    \begin{macrocode}
\def\HoLogoHtml@SLiTeX@simple{\HoLogoHtml@SliTeX@simple}
%    \end{macrocode}
%    \end{macro}
%
%    \begin{macro}{\HoLogo@SLiTeX@narrow}
%    \begin{macrocode}
\def\HoLogo@SLiTeX@narrow{\HoLogo@SliTeX@narrow}
%    \end{macrocode}
%    \end{macro}
%    \begin{macro}{\HoLogoBkm@SLiTeX@narrow}
%    \begin{macrocode}
\def\HoLogoBkm@SLiTeX@narrow{\HoLogoBkm@SliTeX@narrow}
%    \end{macrocode}
%    \end{macro}
%    \begin{macro}{\HoLogoHtml@SLiTeX@narrow}
%    \begin{macrocode}
\def\HoLogoHtml@SLiTeX@narrow{\HoLogoHtml@SliTeX@narrow}
%    \end{macrocode}
%    \end{macro}
%
%    \begin{macro}{\HoLogo@SliTeX@lift}
%    \begin{macrocode}
\def\HoLogo@SliTeX@lift{\HoLogo@SLiTeX@lift}
%    \end{macrocode}
%    \end{macro}
%    \begin{macro}{\HoLogoBkm@SliTeX@lift}
%    \begin{macrocode}
\def\HoLogoBkm@SliTeX@lift{\HoLogoBkm@SLiTeX@lift}
%    \end{macrocode}
%    \end{macro}
%    \begin{macro}{\HoLogoHtml@SliTeX@lift}
%    \begin{macrocode}
\def\HoLogoHtml@SliTeX@lift{\HoLogoHtml@SLiTeX@lift}
%    \end{macrocode}
%    \end{macro}
%
% \paragraph{Defaults.}
%
%    \begin{macro}{\HoLogo@SLiTeX}
%    \begin{macrocode}
\def\HoLogo@SLiTeX{\HoLogo@SLiTeX@lift}
%    \end{macrocode}
%    \end{macro}
%    \begin{macro}{\HoLogoBkm@SLiTeX}
%    \begin{macrocode}
\def\HoLogoBkm@SLiTeX{\HoLogoBkm@SLiTeX@lift}
%    \end{macrocode}
%    \end{macro}
%    \begin{macro}{\HoLogoHtml@SLiTeX}
%    \begin{macrocode}
\def\HoLogoHtml@SLiTeX{\HoLogoHtml@SLiTeX@lift}
%    \end{macrocode}
%    \end{macro}
%
%    \begin{macro}{\HoLogo@SliTeX}
%    \begin{macrocode}
\def\HoLogo@SliTeX{\HoLogo@SliTeX@narrow}
%    \end{macrocode}
%    \end{macro}
%    \begin{macro}{\HoLogoBkm@SliTeX}
%    \begin{macrocode}
\def\HoLogoBkm@SliTeX{\HoLogoBkm@SliTeX@narrow}
%    \end{macrocode}
%    \end{macro}
%    \begin{macro}{\HoLogoHtml@SliTeX}
%    \begin{macrocode}
\def\HoLogoHtml@SliTeX{\HoLogoHtml@SliTeX@narrow}
%    \end{macrocode}
%    \end{macro}
%
% \subsubsection{\hologo{LuaTeX}}
%
%    \begin{macro}{\HoLogo@LuaTeX}
%    The kerning is an idea of Hans Hagen, see mailing list
%    `luatex at tug dot org' in March 2010.
%    \begin{macrocode}
\def\HoLogo@LuaTeX#1{%
  \HOLOGO@mbox{%
    Lua%
    \HOLOGO@NegativeKerning{aT,oT,To}%
    \hologo{TeX}%
  }%
}
%    \end{macrocode}
%    \end{macro}
%    \begin{macro}{\HoLogoHtml@LuaTeX}
%    \begin{macrocode}
\let\HoLogoHtml@LuaTeX\HoLogo@LuaTeX
%    \end{macrocode}
%    \end{macro}
%
% \subsubsection{\hologo{LuaLaTeX}}
%
%    \begin{macro}{\HoLogo@LuaLaTeX}
%    \begin{macrocode}
\def\HoLogo@LuaLaTeX#1{%
  \HOLOGO@mbox{%
    Lua%
    \hologo{LaTeX}%
  }%
}
%    \end{macrocode}
%    \end{macro}
%    \begin{macro}{\HoLogoHtml@LuaLaTeX}
%    \begin{macrocode}
\let\HoLogoHtml@LuaLaTeX\HoLogo@LuaLaTeX
%    \end{macrocode}
%    \end{macro}
%
% \subsubsection{\hologo{XeTeX}, \hologo{XeLaTeX}}
%
%    \begin{macro}{\HOLOGO@IfCharExists}
%    \begin{macrocode}
\ifluatex
  \ifnum\luatexversion<36 %
  \else
    \def\HOLOGO@IfCharExists#1{%
      \ifnum
        \directlua{%
           if luaotfload and luaotfload.aux then
             if luaotfload.aux.font_has_glyph(%
                    font.current(), \number#1) then % 	 
	       tex.print("1") % 	 
	     end % 	 
	   elseif font and font.fonts and font.current then %
            local f = font.fonts[font.current()]%
            if f.characters and f.characters[\number#1] then %
              tex.print("1")%
            end %
          end%
        }0=\ltx@zero
        \expandafter\ltx@secondoftwo
      \else
        \expandafter\ltx@firstoftwo
      \fi
    }%
  \fi
\fi
\ltx@IfUndefined{HOLOGO@IfCharExists}{%
  \def\HOLOGO@@IfCharExists#1{%
    \begingroup
      \tracinglostchars=\ltx@zero
      \setbox\ltx@zero=\hbox{%
        \kern7sp\char#1\relax
        \ifnum\lastkern>\ltx@zero
          \expandafter\aftergroup\csname iffalse\endcsname
        \else
          \expandafter\aftergroup\csname iftrue\endcsname
        \fi
      }%
      % \if{true|false} from \aftergroup
      \endgroup
      \expandafter\ltx@firstoftwo
    \else
      \endgroup
      \expandafter\ltx@secondoftwo
    \fi
  }%
  \ifxetex
    \ltx@IfUndefined{XeTeXfonttype}{}{%
      \ltx@IfUndefined{XeTeXcharglyph}{}{%
        \def\HOLOGO@IfCharExists#1{%
          \ifnum\XeTeXfonttype\font>\ltx@zero
            \expandafter\ltx@firstofthree
          \else
            \expandafter\ltx@gobble
          \fi
          {%
            \ifnum\XeTeXcharglyph#1>\ltx@zero
              \expandafter\ltx@firstoftwo
            \else
              \expandafter\ltx@secondoftwo
            \fi
          }%
          \HOLOGO@@IfCharExists{#1}%
        }%
      }%
    }%
  \fi
}{}
\ltx@ifundefined{HOLOGO@IfCharExists}{%
  \ifnum64=`\^^^^0040\relax % test for big chars of LuaTeX/XeTeX
    \let\HOLOGO@IfCharExists\HOLOGO@@IfCharExists
  \else
    \def\HOLOGO@IfCharExists#1{%
      \ifnum#1>255 %
        \expandafter\ltx@fourthoffour
      \fi
      \HOLOGO@@IfCharExists{#1}%
    }%
  \fi
}{}
%    \end{macrocode}
%    \end{macro}
%
%    \begin{macro}{\HoLogo@Xe}
%    Source: package \xpackage{dtklogos}
%    \begin{macrocode}
\def\HoLogo@Xe#1{%
  X%
  \kern-.1em\relax
  \HOLOGO@IfCharExists{"018E}{%
    \lower.5ex\hbox{\char"018E}%
  }{%
    \chardef\HOLOGO@choice=\ltx@zero
    \ifdim\fontdimen\ltx@one\font>0pt %
      \ltx@IfUndefined{rotatebox}{%
        \ltx@IfUndefined{pgftext}{%
          \ltx@IfUndefined{psscalebox}{%
            \ltx@IfUndefined{HOLOGO@ScaleBox@\hologoDriver}{%
            }{%
              \chardef\HOLOGO@choice=4 %
            }%
          }{%
            \chardef\HOLOGO@choice=3 %
          }%
        }{%
          \chardef\HOLOGO@choice=2 %
        }%
      }{%
        \chardef\HOLOGO@choice=1 %
      }%
      \ifcase\HOLOGO@choice
        \HOLOGO@WarningUnsupportedDriver{Xe}%
        e%
      \or % 1: \rotatebox
        \begingroup
          \setbox\ltx@zero\hbox{\rotatebox{180}{E}}%
          \ltx@LocDimenA=\dp\ltx@zero
          \advance\ltx@LocDimenA by -.5ex\relax
          \raise\ltx@LocDimenA\box\ltx@zero
        \endgroup
      \or % 2: \pgftext
        \lower.5ex\hbox{%
          \pgfpicture
            \pgftext[rotate=180]{E}%
          \endpgfpicture
        }%
      \or % 3: \psscalebox
        \begingroup
          \setbox\ltx@zero\hbox{\psscalebox{-1 -1}{E}}%
          \ltx@LocDimenA=\dp\ltx@zero
          \advance\ltx@LocDimenA by -.5ex\relax
          \raise\ltx@LocDimenA\box\ltx@zero
        \endgroup
      \or % 4: \HOLOGO@PointReflectBox
        \lower.5ex\hbox{\HOLOGO@PointReflectBox{E}}%
      \else
        \@PackageError{hologo}{Internal error (choice/it}\@ehc
      \fi
    \else
      \ltx@IfUndefined{reflectbox}{%
        \ltx@IfUndefined{pgftext}{%
          \ltx@IfUndefined{psscalebox}{%
            \ltx@IfUndefined{HOLOGO@ScaleBox@\hologoDriver}{%
            }{%
              \chardef\HOLOGO@choice=4 %
            }%
          }{%
            \chardef\HOLOGO@choice=3 %
          }%
        }{%
          \chardef\HOLOGO@choice=2 %
        }%
      }{%
        \chardef\HOLOGO@choice=1 %
      }%
      \ifcase\HOLOGO@choice
        \HOLOGO@WarningUnsupportedDriver{Xe}%
        e%
      \or % 1: reflectbox
        \lower.5ex\hbox{%
          \reflectbox{E}%
        }%
      \or % 2: \pgftext
        \lower.5ex\hbox{%
          \pgfpicture
            \pgftransformxscale{-1}%
            \pgftext{E}%
          \endpgfpicture
        }%
      \or % 3: \psscalebox
        \lower.5ex\hbox{%
          \psscalebox{-1 1}{E}%
        }%
      \or % 4: \HOLOGO@Reflectbox
        \lower.5ex\hbox{%
          \HOLOGO@ReflectBox{E}%
        }%
      \else
        \@PackageError{hologo}{Internal error (choice/up)}\@ehc
      \fi
    \fi
  }%
}
%    \end{macrocode}
%    \end{macro}
%    \begin{macro}{\HoLogoHtml@Xe}
%    \begin{macrocode}
\def\HoLogoHtml@Xe#1{%
  \HoLogoCss@Xe
  \HOLOGO@Span{Xe}{%
    X%
    \HOLOGO@Span{e}{%
      \HCode{&\ltx@hashchar x018e;}%
    }%
  }%
}
%    \end{macrocode}
%    \end{macro}
%    \begin{macro}{\HoLogoCss@Xe}
%    \begin{macrocode}
\def\HoLogoCss@Xe{%
  \Css{%
    span.HoLogo-Xe span.HoLogo-e{%
      position:relative;%
      top:.5ex;%
      left-margin:-.1em;%
    }%
  }%
  \global\let\HoLogoCss@Xe\relax
}
%    \end{macrocode}
%    \end{macro}
%
%    \begin{macro}{\HoLogo@XeTeX}
%    \begin{macrocode}
\def\HoLogo@XeTeX#1{%
  \hologo{Xe}%
  \kern-.15em\relax
  \hologo{TeX}%
}
%    \end{macrocode}
%    \end{macro}
%
%    \begin{macro}{\HoLogoHtml@XeTeX}
%    \begin{macrocode}
\def\HoLogoHtml@XeTeX#1{%
  \HoLogoCss@XeTeX
  \HOLOGO@Span{XeTeX}{%
    \hologo{Xe}%
    \hologo{TeX}%
  }%
}
%    \end{macrocode}
%    \end{macro}
%    \begin{macro}{\HoLogoCss@XeTeX}
%    \begin{macrocode}
\def\HoLogoCss@XeTeX{%
  \Css{%
    span.HoLogo-XeTeX span.HoLogo-TeX{%
      margin-left:-.15em;%
    }%
  }%
  \global\let\HoLogoCss@XeTeX\relax
}
%    \end{macrocode}
%    \end{macro}
%
%    \begin{macro}{\HoLogo@XeLaTeX}
%    \begin{macrocode}
\def\HoLogo@XeLaTeX#1{%
  \hologo{Xe}%
  \kern-.13em%
  \hologo{LaTeX}%
}
%    \end{macrocode}
%    \end{macro}
%    \begin{macro}{\HoLogoHtml@XeLaTeX}
%    \begin{macrocode}
\def\HoLogoHtml@XeLaTeX#1{%
  \HoLogoCss@XeLaTeX
  \HOLOGO@Span{XeLaTeX}{%
    \hologo{Xe}%
    \hologo{LaTeX}%
  }%
}
%    \end{macrocode}
%    \end{macro}
%    \begin{macro}{\HoLogoCss@XeLaTeX}
%    \begin{macrocode}
\def\HoLogoCss@XeLaTeX{%
  \Css{%
    span.HoLogo-XeLaTeX span.HoLogo-Xe{%
      margin-right:-.13em;%
    }%
  }%
  \global\let\HoLogoCss@XeLaTeX\relax
}
%    \end{macrocode}
%    \end{macro}
%
% \subsubsection{\hologo{pdfTeX}, \hologo{pdfLaTeX}}
%
%    \begin{macro}{\HoLogo@pdfTeX}
%    \begin{macrocode}
\def\HoLogo@pdfTeX#1{%
  \HOLOGO@mbox{%
    #1{p}{P}df\hologo{TeX}%
  }%
}
%    \end{macrocode}
%    \end{macro}
%    \begin{macro}{\HoLogoCs@pdfTeX}
%    \begin{macrocode}
\def\HoLogoCs@pdfTeX#1{#1{p}{P}dfTeX}
%    \end{macrocode}
%    \end{macro}
%    \begin{macro}{\HoLogoBkm@pdfTeX}
%    \begin{macrocode}
\def\HoLogoBkm@pdfTeX#1{%
  #1{p}{P}df\hologo{TeX}%
}
%    \end{macrocode}
%    \end{macro}
%    \begin{macro}{\HoLogoHtml@pdfTeX}
%    \begin{macrocode}
\let\HoLogoHtml@pdfTeX\HoLogo@pdfTeX
%    \end{macrocode}
%    \end{macro}
%
%    \begin{macro}{\HoLogo@pdfLaTeX}
%    \begin{macrocode}
\def\HoLogo@pdfLaTeX#1{%
  \HOLOGO@mbox{%
    #1{p}{P}df\hologo{LaTeX}%
  }%
}
%    \end{macrocode}
%    \end{macro}
%    \begin{macro}{\HoLogoCs@pdfLaTeX}
%    \begin{macrocode}
\def\HoLogoCs@pdfLaTeX#1{#1{p}{P}dfLaTeX}
%    \end{macrocode}
%    \end{macro}
%    \begin{macro}{\HoLogoBkm@pdfLaTeX}
%    \begin{macrocode}
\def\HoLogoBkm@pdfLaTeX#1{%
  #1{p}{P}df\hologo{LaTeX}%
}
%    \end{macrocode}
%    \end{macro}
%    \begin{macro}{\HoLogoHtml@pdfLaTeX}
%    \begin{macrocode}
\let\HoLogoHtml@pdfLaTeX\HoLogo@pdfLaTeX
%    \end{macrocode}
%    \end{macro}
%
% \subsubsection{\hologo{VTeX}}
%
%    \begin{macro}{\HoLogo@VTeX}
%    \begin{macrocode}
\def\HoLogo@VTeX#1{%
  \HOLOGO@mbox{%
    V\hologo{TeX}%
  }%
}
%    \end{macrocode}
%    \end{macro}
%    \begin{macro}{\HoLogoHtml@VTeX}
%    \begin{macrocode}
\let\HoLogoHtml@VTeX\HoLogo@VTeX
%    \end{macrocode}
%    \end{macro}
%
% \subsubsection{\hologo{AmS}, \dots}
%
%    Source: class \xclass{amsdtx}
%
%    \begin{macro}{\HoLogo@AmS}
%    \begin{macrocode}
\def\HoLogo@AmS#1{%
  \HoLogoFont@font{AmS}{sy}{%
    A%
    \kern-.1667em%
    \lower.5ex\hbox{M}%
    \kern-.125em%
    S%
  }%
}
%    \end{macrocode}
%    \end{macro}
%    \begin{macro}{\HoLogoBkm@AmS}
%    \begin{macrocode}
\def\HoLogoBkm@AmS#1{AmS}
%    \end{macrocode}
%    \end{macro}
%    \begin{macro}{\HoLogoHtml@AmS}
%    \begin{macrocode}
\def\HoLogoHtml@AmS#1{%
  \HoLogoCss@AmS
%  \HoLogoFont@font{AmS}{sy}{%
    \HOLOGO@Span{AmS}{%
      A%
      \HOLOGO@Span{M}{M}%
      S%
    }%
%   }%
}
%    \end{macrocode}
%    \end{macro}
%    \begin{macro}{\HoLogoCss@AmS}
%    \begin{macrocode}
\def\HoLogoCss@AmS{%
  \Css{%
    span.HoLogo-AmS span.HoLogo-M{%
      position:relative;%
      top:.5ex;%
      margin-left:-.1667em;%
      margin-right:-.125em;%
      text-decoration:none;%
    }%
  }%
  \global\let\HoLogoCss@AmS\relax
}
%    \end{macrocode}
%    \end{macro}
%
%    \begin{macro}{\HoLogo@AmSTeX}
%    \begin{macrocode}
\def\HoLogo@AmSTeX#1{%
  \hologo{AmS}%
  \HOLOGO@hyphen
  \hologo{TeX}%
}
%    \end{macrocode}
%    \end{macro}
%    \begin{macro}{\HoLogoBkm@AmSTeX}
%    \begin{macrocode}
\def\HoLogoBkm@AmSTeX#1{AmS-TeX}%
%    \end{macrocode}
%    \end{macro}
%    \begin{macro}{\HoLogoHtml@AmSTeX}
%    \begin{macrocode}
\let\HoLogoHtml@AmSTeX\HoLogo@AmSTeX
%    \end{macrocode}
%    \end{macro}
%
%    \begin{macro}{\HoLogo@AmSLaTeX}
%    \begin{macrocode}
\def\HoLogo@AmSLaTeX#1{%
  \hologo{AmS}%
  \HOLOGO@hyphen
  \hologo{LaTeX}%
}
%    \end{macrocode}
%    \end{macro}
%    \begin{macro}{\HoLogoBkm@AmSLaTeX}
%    \begin{macrocode}
\def\HoLogoBkm@AmSLaTeX#1{AmS-LaTeX}%
%    \end{macrocode}
%    \end{macro}
%    \begin{macro}{\HoLogoHtml@AmSLaTeX}
%    \begin{macrocode}
\let\HoLogoHtml@AmSLaTeX\HoLogo@AmSLaTeX
%    \end{macrocode}
%    \end{macro}
%
% \subsubsection{\hologo{BibTeX}}
%
%    \begin{macro}{\HoLogo@BibTeX@sc}
%    A definition of \hologo{BibTeX} is provided in
%    the documentation source for the manual of \hologo{BibTeX}
%    \cite{btxdoc}.
%\begin{quote}
%\begin{verbatim}
%\def\BibTeX{%
%  {%
%    \rm
%    B%
%    \kern-.05em%
%    {%
%      \sc
%      i%
%      \kern-.025em %
%      b%
%    }%
%    \kern-.08em
%    T%
%    \kern-.1667em%
%    \lower.7ex\hbox{E}%
%    \kern-.125em%
%    X%
%  }%
%}
%\end{verbatim}
%\end{quote}
%    \begin{macrocode}
\def\HoLogo@BibTeX@sc#1{%
  B%
  \kern-.05em%
  \HoLogoFont@font{BibTeX}{sc}{%
    i%
    \kern-.025em%
    b%
  }%
  \HOLOGO@discretionary
  \kern-.08em%
  \hologo{TeX}%
}
%    \end{macrocode}
%    \end{macro}
%    \begin{macro}{\HoLogoHtml@BibTeX@sc}
%    \begin{macrocode}
\def\HoLogoHtml@BibTeX@sc#1{%
  \HoLogoCss@BibTeX@sc
  \HOLOGO@Span{BibTeX-sc}{%
    B%
    \HOLOGO@Span{i}{i}%
    \HOLOGO@Span{b}{b}%
    \hologo{TeX}%
  }%
}
%    \end{macrocode}
%    \end{macro}
%    \begin{macro}{\HoLogoCss@BibTeX@sc}
%    \begin{macrocode}
\def\HoLogoCss@BibTeX@sc{%
  \Css{%
    span.HoLogo-BibTeX-sc span.HoLogo-i{%
      margin-left:-.05em;%
      margin-right:-.025em;%
      font-variant:small-caps;%
    }%
  }%
  \Css{%
    span.HoLogo-BibTeX-sc span.HoLogo-b{%
      margin-right:-.08em;%
      font-variant:small-caps;%
    }%
  }%
  \global\let\HoLogoCss@BibTeX@sc\relax
}
%    \end{macrocode}
%    \end{macro}
%
%    \begin{macro}{\HoLogo@BibTeX@sf}
%    Variant \xoption{sf} avoids trouble with unavailable
%    small caps fonts (e.g., bold versions of Computer Modern or
%    Latin Modern). The definition is taken from
%    package \xpackage{dtklogos} \cite{dtklogos}.
%\begin{quote}
%\begin{verbatim}
%\DeclareRobustCommand{\BibTeX}{%
%  B%
%  \kern-.05em%
%  \hbox{%
%    $\m@th$% %% force math size calculations
%    \csname S@\f@size\endcsname
%    \fontsize\sf@size\z@
%    \math@fontsfalse
%    \selectfont
%    I%
%    \kern-.025em%
%    B
%  }%
%  \kern-.08em%
%  \-%
%  \TeX
%}
%\end{verbatim}
%\end{quote}
%    \begin{macrocode}
\def\HoLogo@BibTeX@sf#1{%
  B%
  \kern-.05em%
  \HoLogoFont@font{BibTeX}{bibsf}{%
    I%
    \kern-.025em%
    B%
  }%
  \HOLOGO@discretionary
  \kern-.08em%
  \hologo{TeX}%
}
%    \end{macrocode}
%    \end{macro}
%    \begin{macro}{\HoLogoHtml@BibTeX@sf}
%    \begin{macrocode}
\def\HoLogoHtml@BibTeX@sf#1{%
  \HoLogoCss@BibTeX@sf
  \HOLOGO@Span{BibTeX-sf}{%
    B%
    \HoLogoFont@font{BibTeX}{bibsf}{%
      \HOLOGO@Span{i}{I}%
      B%
    }%
    \hologo{TeX}%
  }%
}
%    \end{macrocode}
%    \end{macro}
%    \begin{macro}{\HoLogoCss@BibTeX@sf}
%    \begin{macrocode}
\def\HoLogoCss@BibTeX@sf{%
  \Css{%
    span.HoLogo-BibTeX-sf span.HoLogo-i{%
      margin-left:-.05em;%
      margin-right:-.025em;%
    }%
  }%
  \Css{%
    span.HoLogo-BibTeX-sf span.HoLogo-TeX{%
      margin-left:-.08em;%
    }%
  }%
  \global\let\HoLogoCss@BibTeX@sf\relax
}
%    \end{macrocode}
%    \end{macro}
%
%    \begin{macro}{\HoLogo@BibTeX}
%    \begin{macrocode}
\def\HoLogo@BibTeX{\HoLogo@BibTeX@sf}
%    \end{macrocode}
%    \end{macro}
%    \begin{macro}{\HoLogoHtml@BibTeX}
%    \begin{macrocode}
\def\HoLogoHtml@BibTeX{\HoLogoHtml@BibTeX@sf}
%    \end{macrocode}
%    \end{macro}
%
% \subsubsection{\hologo{BibTeX8}}
%
%    \begin{macro}{\HoLogo@BibTeX8}
%    \begin{macrocode}
\expandafter\def\csname HoLogo@BibTeX8\endcsname#1{%
  \hologo{BibTeX}%
  8%
}
%    \end{macrocode}
%    \end{macro}
%
%    \begin{macro}{\HoLogoBkm@BibTeX8}
%    \begin{macrocode}
\expandafter\def\csname HoLogoBkm@BibTeX8\endcsname#1{%
  \hologo{BibTeX}%
  8%
}
%    \end{macrocode}
%    \end{macro}
%    \begin{macro}{\HoLogoHtml@BibTeX8}
%    \begin{macrocode}
\expandafter
\let\csname HoLogoHtml@BibTeX8\expandafter\endcsname
\csname HoLogo@BibTeX8\endcsname
%    \end{macrocode}
%    \end{macro}
%
% \subsubsection{\hologo{ConTeXt}}
%
%    \begin{macro}{\HoLogo@ConTeXt@simple}
%    \begin{macrocode}
\def\HoLogo@ConTeXt@simple#1{%
  \HOLOGO@mbox{Con}%
  \HOLOGO@discretionary
  \HOLOGO@mbox{\hologo{TeX}t}%
}
%    \end{macrocode}
%    \end{macro}
%    \begin{macro}{\HoLogoHtml@ConTeXt@simple}
%    \begin{macrocode}
\let\HoLogoHtml@ConTeXt@simple\HoLogo@ConTeXt@simple
%    \end{macrocode}
%    \end{macro}
%
%    \begin{macro}{\HoLogo@ConTeXt@narrow}
%    This definition of logo \hologo{ConTeXt} with variant \xoption{narrow}
%    comes from TUGboat's class \xclass{ltugboat} (version 2010/11/15 v2.8).
%    \begin{macrocode}
\def\HoLogo@ConTeXt@narrow#1{%
  \HOLOGO@mbox{C\kern-.0333emon}%
  \HOLOGO@discretionary
  \kern-.0667em%
  \HOLOGO@mbox{\hologo{TeX}\kern-.0333emt}%
}
%    \end{macrocode}
%    \end{macro}
%    \begin{macro}{\HoLogoHtml@ConTeXt@narrow}
%    \begin{macrocode}
\def\HoLogoHtml@ConTeXt@narrow#1{%
  \HoLogoCss@ConTeXt@narrow
  \HOLOGO@Span{ConTeXt-narrow}{%
    \HOLOGO@Span{C}{C}%
    on%
    \hologo{TeX}%
    t%
  }%
}
%    \end{macrocode}
%    \end{macro}
%    \begin{macro}{\HoLogoCss@ConTeXt@narrow}
%    \begin{macrocode}
\def\HoLogoCss@ConTeXt@narrow{%
  \Css{%
    span.HoLogo-ConTeXt-narrow span.HoLogo-C{%
      margin-left:-.0333em;%
    }%
  }%
  \Css{%
    span.HoLogo-ConTeXt-narrow span.HoLogo-TeX{%
      margin-left:-.0667em;%
      margin-right:-.0333em;%
    }%
  }%
  \global\let\HoLogoCss@ConTeXt@narrow\relax
}
%    \end{macrocode}
%    \end{macro}
%
%    \begin{macro}{\HoLogo@ConTeXt}
%    \begin{macrocode}
\def\HoLogo@ConTeXt{\HoLogo@ConTeXt@narrow}
%    \end{macrocode}
%    \end{macro}
%    \begin{macro}{\HoLogoHtml@ConTeXt}
%    \begin{macrocode}
\def\HoLogoHtml@ConTeXt{\HoLogoHtml@ConTeXt@narrow}
%    \end{macrocode}
%    \end{macro}
%
% \subsubsection{\hologo{emTeX}}
%
%    \begin{macro}{\HoLogo@emTeX}
%    \begin{macrocode}
\def\HoLogo@emTeX#1{%
  \HOLOGO@mbox{#1{e}{E}m}%
  \HOLOGO@discretionary
  \hologo{TeX}%
}
%    \end{macrocode}
%    \end{macro}
%    \begin{macro}{\HoLogoCs@emTeX}
%    \begin{macrocode}
\def\HoLogoCs@emTeX#1{#1{e}{E}mTeX}%
%    \end{macrocode}
%    \end{macro}
%    \begin{macro}{\HoLogoBkm@emTeX}
%    \begin{macrocode}
\def\HoLogoBkm@emTeX#1{%
  #1{e}{E}m\hologo{TeX}%
}
%    \end{macrocode}
%    \end{macro}
%    \begin{macro}{\HoLogoHtml@emTeX}
%    \begin{macrocode}
\let\HoLogoHtml@emTeX\HoLogo@emTeX
%    \end{macrocode}
%    \end{macro}
%
% \subsubsection{\hologo{ExTeX}}
%
%    \begin{macro}{\HoLogo@ExTeX}
%    The definition is taken from the FAQ of the
%    project \hologo{ExTeX}
%    \cite{ExTeX-FAQ}.
%\begin{quote}
%\begin{verbatim}
%\def\ExTeX{%
%  \textrm{% Logo always with serifs
%    \ensuremath{%
%      \textstyle
%      \varepsilon_{%
%        \kern-0.15em%
%        \mathcal{X}%
%      }%
%    }%
%    \kern-.15em%
%    \TeX
%  }%
%}
%\end{verbatim}
%\end{quote}
%    \begin{macrocode}
\def\HoLogo@ExTeX#1{%
  \HoLogoFont@font{ExTeX}{rm}{%
    \ltx@mbox{%
      \HOLOGO@MathSetup
      $%
        \textstyle
        \varepsilon_{%
          \kern-0.15em%
          \HoLogoFont@font{ExTeX}{sy}{X}%
        }%
      $%
    }%
    \HOLOGO@discretionary
    \kern-.15em%
    \hologo{TeX}%
  }%
}
%    \end{macrocode}
%    \end{macro}
%    \begin{macro}{\HoLogoHtml@ExTeX}
%    \begin{macrocode}
\def\HoLogoHtml@ExTeX#1{%
  \HoLogoCss@ExTeX
  \HoLogoFont@font{ExTeX}{rm}{%
    \HOLOGO@Span{ExTeX}{%
      \ltx@mbox{%
        \HOLOGO@MathSetup
        $\textstyle\varepsilon$%
        \HOLOGO@Span{X}{$\textstyle\chi$}%
        \hologo{TeX}%
      }%
    }%
  }%
}
%    \end{macrocode}
%    \end{macro}
%    \begin{macro}{\HoLogoBkm@ExTeX}
%    \begin{macrocode}
\def\HoLogoBkm@ExTeX#1{%
  \HOLOGO@PdfdocUnicode{#1{e}{E}x}{\textepsilon\textchi}%
  \hologo{TeX}%
}
%    \end{macrocode}
%    \end{macro}
%    \begin{macro}{\HoLogoCss@ExTeX}
%    \begin{macrocode}
\def\HoLogoCss@ExTeX{%
  \Css{%
    span.HoLogo-ExTeX{%
      font-family:serif;%
    }%
  }%
  \Css{%
    span.HoLogo-ExTeX span.HoLogo-TeX{%
      margin-left:-.15em;%
    }%
  }%
  \global\let\HoLogoCss@ExTeX\relax
}
%    \end{macrocode}
%    \end{macro}
%
% \subsubsection{\hologo{MiKTeX}}
%
%    \begin{macro}{\HoLogo@MiKTeX}
%    \begin{macrocode}
\def\HoLogo@MiKTeX#1{%
  \HOLOGO@mbox{MiK}%
  \HOLOGO@discretionary
  \hologo{TeX}%
}
%    \end{macrocode}
%    \end{macro}
%    \begin{macro}{\HoLogoHtml@MiKTeX}
%    \begin{macrocode}
\let\HoLogoHtml@MiKTeX\HoLogo@MiKTeX
%    \end{macrocode}
%    \end{macro}
%
% \subsubsection{\hologo{OzTeX} and friends}
%
%    Source: \hologo{OzTeX} FAQ \cite{OzTeX}:
%    \begin{quote}
%      |\def\OzTeX{O\kern-.03em z\kern-.15em\TeX}|\\
%      (There is no kerning in OzMF, OzMP and OzTtH.)
%    \end{quote}
%
%    \begin{macro}{\HoLogo@OzTeX}
%    \begin{macrocode}
\def\HoLogo@OzTeX#1{%
  O%
  \kern-.03em %
  z%
  \kern-.15em %
  \hologo{TeX}%
}
%    \end{macrocode}
%    \end{macro}
%    \begin{macro}{\HoLogoHtml@OzTeX}
%    \begin{macrocode}
\def\HoLogoHtml@OzTeX#1{%
  \HoLogoCss@OzTeX
  \HOLOGO@Span{OzTeX}{%
    O%
    \HOLOGO@Span{z}{z}%
    \hologo{TeX}%
  }%
}
%    \end{macrocode}
%    \end{macro}
%    \begin{macro}{\HoLogoCss@OzTeX}
%    \begin{macrocode}
\def\HoLogoCss@OzTeX{%
  \Css{%
    span.HoLogo-OzTeX span.HoLogo-z{%
      margin-left:-.03em;%
      margin-right:-.15em;%
    }%
  }%
  \global\let\HoLogoCss@OzTeX\relax
}
%    \end{macrocode}
%    \end{macro}
%
%    \begin{macro}{\HoLogo@OzMF}
%    \begin{macrocode}
\def\HoLogo@OzMF#1{%
  \HOLOGO@mbox{OzMF}%
}
%    \end{macrocode}
%    \end{macro}
%    \begin{macro}{\HoLogo@OzMP}
%    \begin{macrocode}
\def\HoLogo@OzMP#1{%
  \HOLOGO@mbox{OzMP}%
}
%    \end{macrocode}
%    \end{macro}
%    \begin{macro}{\HoLogo@OzTtH}
%    \begin{macrocode}
\def\HoLogo@OzTtH#1{%
  \HOLOGO@mbox{OzTtH}%
}
%    \end{macrocode}
%    \end{macro}
%
% \subsubsection{\hologo{PCTeX}}
%
%    \begin{macro}{\HoLogo@PCTeX}
%    \begin{macrocode}
\def\HoLogo@PCTeX#1{%
  \HOLOGO@mbox{PC}%
  \hologo{TeX}%
}
%    \end{macrocode}
%    \end{macro}
%    \begin{macro}{\HoLogoHtml@PCTeX}
%    \begin{macrocode}
\let\HoLogoHtml@PCTeX\HoLogo@PCTeX
%    \end{macrocode}
%    \end{macro}
%
% \subsubsection{\hologo{PiCTeX}}
%
%    The original definitions from \xfile{pictex.tex} \cite{PiCTeX}:
%\begin{quote}
%\begin{verbatim}
%\def\PiC{%
%  P%
%  \kern-.12em%
%  \lower.5ex\hbox{I}%
%  \kern-.075em%
%  C%
%}
%\def\PiCTeX{%
%  \PiC
%  \kern-.11em%
%  \TeX
%}
%\end{verbatim}
%\end{quote}
%
%    \begin{macro}{\HoLogo@PiC}
%    \begin{macrocode}
\def\HoLogo@PiC#1{%
  P%
  \kern-.12em%
  \lower.5ex\hbox{I}%
  \kern-.075em%
  C%
  \HOLOGO@SpaceFactor
}
%    \end{macrocode}
%    \end{macro}
%    \begin{macro}{\HoLogoHtml@PiC}
%    \begin{macrocode}
\def\HoLogoHtml@PiC#1{%
  \HoLogoCss@PiC
  \HOLOGO@Span{PiC}{%
    P%
    \HOLOGO@Span{i}{I}%
    C%
  }%
}
%    \end{macrocode}
%    \end{macro}
%    \begin{macro}{\HoLogoCss@PiC}
%    \begin{macrocode}
\def\HoLogoCss@PiC{%
  \Css{%
    span.HoLogo-PiC span.HoLogo-i{%
      position:relative;%
      top:.5ex;%
      margin-left:-.12em;%
      margin-right:-.075em;%
      text-decoration:none;%
    }%
  }%
  \global\let\HoLogoCss@PiC\relax
}
%    \end{macrocode}
%    \end{macro}
%
%    \begin{macro}{\HoLogo@PiCTeX}
%    \begin{macrocode}
\def\HoLogo@PiCTeX#1{%
  \hologo{PiC}%
  \HOLOGO@discretionary
  \kern-.11em%
  \hologo{TeX}%
}
%    \end{macrocode}
%    \end{macro}
%    \begin{macro}{\HoLogoHtml@PiCTeX}
%    \begin{macrocode}
\def\HoLogoHtml@PiCTeX#1{%
  \HoLogoCss@PiCTeX
  \HOLOGO@Span{PiCTeX}{%
    \hologo{PiC}%
    \hologo{TeX}%
  }%
}
%    \end{macrocode}
%    \end{macro}
%    \begin{macro}{\HoLogoCss@PiCTeX}
%    \begin{macrocode}
\def\HoLogoCss@PiCTeX{%
  \Css{%
    span.HoLogo-PiCTeX span.HoLogo-PiC{%
      margin-right:-.11em;%
    }%
  }%
  \global\let\HoLogoCss@PiCTeX\relax
}
%    \end{macrocode}
%    \end{macro}
%
% \subsubsection{\hologo{teTeX}}
%
%    \begin{macro}{\HoLogo@teTeX}
%    \begin{macrocode}
\def\HoLogo@teTeX#1{%
  \HOLOGO@mbox{#1{t}{T}e}%
  \HOLOGO@discretionary
  \hologo{TeX}%
}
%    \end{macrocode}
%    \end{macro}
%    \begin{macro}{\HoLogoCs@teTeX}
%    \begin{macrocode}
\def\HoLogoCs@teTeX#1{#1{t}{T}dfTeX}
%    \end{macrocode}
%    \end{macro}
%    \begin{macro}{\HoLogoBkm@teTeX}
%    \begin{macrocode}
\def\HoLogoBkm@teTeX#1{%
  #1{t}{T}e\hologo{TeX}%
}
%    \end{macrocode}
%    \end{macro}
%    \begin{macro}{\HoLogoHtml@teTeX}
%    \begin{macrocode}
\let\HoLogoHtml@teTeX\HoLogo@teTeX
%    \end{macrocode}
%    \end{macro}
%
% \subsubsection{\hologo{TeX4ht}}
%
%    \begin{macro}{\HoLogo@TeX4ht}
%    \begin{macrocode}
\expandafter\def\csname HoLogo@TeX4ht\endcsname#1{%
  \HOLOGO@mbox{\hologo{TeX}4ht}%
}
%    \end{macrocode}
%    \end{macro}
%    \begin{macro}{\HoLogoHtml@TeX4ht}
%    \begin{macrocode}
\expandafter
\let\csname HoLogoHtml@TeX4ht\expandafter\endcsname
\csname HoLogo@TeX4ht\endcsname
%    \end{macrocode}
%    \end{macro}
%
%
% \subsubsection{\hologo{SageTeX}}
%
%    \begin{macro}{\HoLogo@SageTeX}
%    \begin{macrocode}
\def\HoLogo@SageTeX#1{%
  \HOLOGO@mbox{Sage}%
  \HOLOGO@discretionary
  \HOLOGO@NegativeKerning{eT,oT,To}%
  \hologo{TeX}%
}
%    \end{macrocode}
%    \end{macro}
%    \begin{macro}{\HoLogoHtml@SageTeX}
%    \begin{macrocode}
\let\HoLogoHtml@SageTeX\HoLogo@SageTeX
%    \end{macrocode}
%    \end{macro}
%
% \subsection{\hologo{METAFONT} and friends}
%
%    \begin{macro}{\HoLogo@METAFONT}
%    \begin{macrocode}
\def\HoLogo@METAFONT#1{%
  \HoLogoFont@font{METAFONT}{logo}{%
    \HOLOGO@mbox{META}%
    \HOLOGO@discretionary
    \HOLOGO@mbox{FONT}%
  }%
}
%    \end{macrocode}
%    \end{macro}
%
%    \begin{macro}{\HoLogo@METAPOST}
%    \begin{macrocode}
\def\HoLogo@METAPOST#1{%
  \HoLogoFont@font{METAPOST}{logo}{%
    \HOLOGO@mbox{META}%
    \HOLOGO@discretionary
    \HOLOGO@mbox{POST}%
  }%
}
%    \end{macrocode}
%    \end{macro}
%
%    \begin{macro}{\HoLogo@MetaFun}
%    \begin{macrocode}
\def\HoLogo@MetaFun#1{%
  \HOLOGO@mbox{Meta}%
  \HOLOGO@discretionary
  \HOLOGO@mbox{Fun}%
}
%    \end{macrocode}
%    \end{macro}
%
%    \begin{macro}{\HoLogo@MetaPost}
%    \begin{macrocode}
\def\HoLogo@MetaPost#1{%
  \HOLOGO@mbox{Meta}%
  \HOLOGO@discretionary
  \HOLOGO@mbox{Post}%
}
%    \end{macrocode}
%    \end{macro}
%
% \subsection{Others}
%
% \subsubsection{\hologo{biber}}
%
%    \begin{macro}{\HoLogo@biber}
%    \begin{macrocode}
\def\HoLogo@biber#1{%
  \HOLOGO@mbox{#1{b}{B}i}%
  \HOLOGO@discretionary
  \HOLOGO@mbox{ber}%
}
%    \end{macrocode}
%    \end{macro}
%    \begin{macro}{\HoLogoCs@biber}
%    \begin{macrocode}
\def\HoLogoCs@biber#1{#1{b}{B}iber}
%    \end{macrocode}
%    \end{macro}
%    \begin{macro}{\HoLogoBkm@biber}
%    \begin{macrocode}
\def\HoLogoBkm@biber#1{%
  #1{b}{B}iber%
}
%    \end{macrocode}
%    \end{macro}
%    \begin{macro}{\HoLogoHtml@biber}
%    \begin{macrocode}
\let\HoLogoHtml@biber\HoLogo@biber
%    \end{macrocode}
%    \end{macro}
%
% \subsubsection{\hologo{KOMAScript}}
%
%    \begin{macro}{\HoLogo@KOMAScript}
%    The definition for \hologo{KOMAScript} is taken
%    from \hologo{KOMAScript} (\xfile{scrlogo.dtx}, reformatted) \cite{scrlogo}:
%\begin{quote}
%\begin{verbatim}
%\@ifundefined{KOMAScript}{%
%  \DeclareRobustCommand{\KOMAScript}{%
%    \textsf{%
%      K\kern.05em O\kern.05emM\kern.05em A%
%      \kern.1em-\kern.1em %
%      Script%
%    }%
%  }%
%}{}
%\end{verbatim}
%\end{quote}
%    \begin{macrocode}
\def\HoLogo@KOMAScript#1{%
  \HoLogoFont@font{KOMAScript}{sf}{%
    \HOLOGO@mbox{%
      K\kern.05em%
      O\kern.05em%
      M\kern.05em%
      A%
    }%
    \kern.1em%
    \HOLOGO@hyphen
    \kern.1em%
    \HOLOGO@mbox{Script}%
  }%
}
%    \end{macrocode}
%    \end{macro}
%    \begin{macro}{\HoLogoBkm@KOMAScript}
%    \begin{macrocode}
\def\HoLogoBkm@KOMAScript#1{%
  KOMA-Script%
}
%    \end{macrocode}
%    \end{macro}
%    \begin{macro}{\HoLogoHtml@KOMAScript}
%    \begin{macrocode}
\def\HoLogoHtml@KOMAScript#1{%
  \HoLogoCss@KOMAScript
  \HoLogoFont@font{KOMAScript}{sf}{%
    \HOLOGO@Span{KOMAScript}{%
      K%
      \HOLOGO@Span{O}{O}%
      M%
      \HOLOGO@Span{A}{A}%
      \HOLOGO@Span{hyphen}{-}%
      Script%
    }%
  }%
}
%    \end{macrocode}
%    \end{macro}
%    \begin{macro}{\HoLogoCss@KOMAScript}
%    \begin{macrocode}
\def\HoLogoCss@KOMAScript{%
  \Css{%
    span.HoLogo-KOMAScript{%
      font-family:sans-serif;%
    }%
  }%
  \Css{%
    span.HoLogo-KOMAScript span.HoLogo-O{%
      padding-left:.05em;%
      padding-right:.05em;%
    }%
  }%
  \Css{%
    span.HoLogo-KOMAScript span.HoLogo-A{%
      padding-left:.05em;%
    }%
  }%
  \Css{%
    span.HoLogo-KOMAScript span.HoLogo-hyphen{%
      padding-left:.1em;%
      padding-right:.1em;%
    }%
  }%
  \global\let\HoLogoCss@KOMAScript\relax
}
%    \end{macrocode}
%    \end{macro}
%
% \subsubsection{\hologo{LyX}}
%
%    \begin{macro}{\HoLogo@LyX}
%    The definition is taken from the documentation source files
%    of \hologo{LyX}, \xfile{Intro.lyx} \cite{LyX}:
%\begin{quote}
%\begin{verbatim}
%\def\LyX{%
%  \texorpdfstring{%
%    L\kern-.1667em\lower.25em\hbox{Y}\kern-.125emX\@%
%  }{%
%    LyX%
%  }%
%}
%\end{verbatim}
%\end{quote}
%    \begin{macrocode}
\def\HoLogo@LyX#1{%
  L%
  \kern-.1667em%
  \lower.25em\hbox{Y}%
  \kern-.125em%
  X%
  \HOLOGO@SpaceFactor
}
%    \end{macrocode}
%    \end{macro}
%    \begin{macro}{\HoLogoHtml@LyX}
%    \begin{macrocode}
\def\HoLogoHtml@LyX#1{%
  \HoLogoCss@LyX
  \HOLOGO@Span{LyX}{%
    L%
    \HOLOGO@Span{y}{Y}%
    X%
  }%
}
%    \end{macrocode}
%    \end{macro}
%    \begin{macro}{\HoLogoCss@LyX}
%    \begin{macrocode}
\def\HoLogoCss@LyX{%
  \Css{%
    span.HoLogo-LyX span.HoLogo-y{%
      position:relative;%
      top:.25em;%
      margin-left:-.1667em;%
      margin-right:-.125em;%
      text-decoration:none;%
    }%
  }%
  \global\let\HoLogoCss@LyX\relax
}
%    \end{macrocode}
%    \end{macro}
%
% \subsubsection{\hologo{NTS}}
%
%    \begin{macro}{\HoLogo@NTS}
%    Definition for \hologo{NTS} can be found in
%    package \xpackage{etex\textunderscore man} for the \hologo{eTeX} manual \cite{etexman}
%    and in package \xpackage{dtklogos} \cite{dtklogos}:
%\begin{quote}
%\begin{verbatim}
%\def\NTS{%
%  \leavevmode
%  \hbox{%
%    $%
%      \cal N%
%      \kern-0.35em%
%      \lower0.5ex\hbox{$\cal T$}%
%      \kern-0.2em%
%      S%
%    $%
%  }%
%}
%\end{verbatim}
%\end{quote}
%    \begin{macrocode}
\def\HoLogo@NTS#1{%
  \HoLogoFont@font{NTS}{sy}{%
    N\/%
    \kern-.35em%
    \lower.5ex\hbox{T\/}%
    \kern-.2em%
    S\/%
  }%
  \HOLOGO@SpaceFactor
}
%    \end{macrocode}
%    \end{macro}
%
% \subsubsection{\Hologo{TTH} (\hologo{TeX} to HTML translator)}
%
%    Source: \url{http://hutchinson.belmont.ma.us/tth/}
%    In the HTML source the second `T' is printed as subscript.
%\begin{quote}
%\begin{verbatim}
%T<sub>T</sub>H
%\end{verbatim}
%\end{quote}
%    \begin{macro}{\HoLogo@TTH}
%    \begin{macrocode}
\def\HoLogo@TTH#1{%
  \ltx@mbox{%
    T\HOLOGO@SubScript{T}H%
  }%
  \HOLOGO@SpaceFactor
}
%    \end{macrocode}
%    \end{macro}
%
%    \begin{macro}{\HoLogoHtml@TTH}
%    \begin{macrocode}
\def\HoLogoHtml@TTH#1{%
  T\HCode{<sub>}T\HCode{</sub>}H%
}
%    \end{macrocode}
%    \end{macro}
%
% \subsubsection{\Hologo{HanTheThanh}}
%
%    Partial source: Package \xpackage{dtklogos}.
%    The double accent is U+1EBF (latin small letter e with circumflex
%    and acute).
%    \begin{macro}{\HoLogo@HanTheThanh}
%    \begin{macrocode}
\def\HoLogo@HanTheThanh#1{%
  \ltx@mbox{H\`an}%
  \HOLOGO@space
  \ltx@mbox{%
    Th%
    \HOLOGO@IfCharExists{"1EBF}{%
      \char"1EBF\relax
    }{%
      \^e\hbox to 0pt{\hss\raise .5ex\hbox{\'{}}}%
    }%
  }%
  \HOLOGO@space
  \ltx@mbox{Th\`anh}%
}
%    \end{macrocode}
%    \end{macro}
%    \begin{macro}{\HoLogoBkm@HanTheThanh}
%    \begin{macrocode}
\def\HoLogoBkm@HanTheThanh#1{%
  H\`an %
  Th\HOLOGO@PdfdocUnicode{\^e}{\9036\277} %
  Th\`anh%
}
%    \end{macrocode}
%    \end{macro}
%    \begin{macro}{\HoLogoHtml@HanTheThanh}
%    \begin{macrocode}
\def\HoLogoHtml@HanTheThanh#1{%
  H\`an %
  Th\HCode{&\ltx@hashchar x1ebf;} %
  Th\`anh%
}
%    \end{macrocode}
%    \end{macro}
%
% \subsection{Driver detection}
%
%    \begin{macrocode}
\HOLOGO@IfExists\InputIfFileExists{%
  \InputIfFileExists{hologo.cfg}{}{}%
}{%
  \ltx@IfUndefined{pdf@filesize}{%
    \def\HOLOGO@InputIfExists{%
      \openin\HOLOGO@temp=hologo.cfg\relax
      \ifeof\HOLOGO@temp
        \closein\HOLOGO@temp
      \else
        \closein\HOLOGO@temp
        \begingroup
          \def\x{LaTeX2e}%
        \expandafter\endgroup
        \ifx\fmtname\x
          \input{hologo.cfg}%
        \else
          \input hologo.cfg\relax
        \fi
      \fi
    }%
    \ltx@IfUndefined{newread}{%
      \chardef\HOLOGO@temp=15 %
      \def\HOLOGO@CheckRead{%
        \ifeof\HOLOGO@temp
          \HOLOGO@InputIfExists
        \else
          \ifcase\HOLOGO@temp
            \@PackageWarningNoLine{hologo}{%
              Configuration file ignored, because\MessageBreak
              a free read register could not be found%
            }%
          \else
            \begingroup
              \count\ltx@cclv=\HOLOGO@temp
              \advance\ltx@cclv by \ltx@minusone
              \edef\x{\endgroup
                \chardef\noexpand\HOLOGO@temp=\the\count\ltx@cclv
                \relax
              }%
            \x
          \fi
        \fi
      }%
    }{%
      \csname newread\endcsname\HOLOGO@temp
      \HOLOGO@InputIfExists
    }%
  }{%
    \edef\HOLOGO@temp{\pdf@filesize{hologo.cfg}}%
    \ifx\HOLOGO@temp\ltx@empty
    \else
      \ifnum\HOLOGO@temp>0 %
        \begingroup
          \def\x{LaTeX2e}%
        \expandafter\endgroup
        \ifx\fmtname\x
          \input{hologo.cfg}%
        \else
          \input hologo.cfg\relax
        \fi
      \else
        \@PackageInfoNoLine{hologo}{%
          Empty configuration file `hologo.cfg' ignored%
        }%
      \fi
    \fi
  }%
}
%    \end{macrocode}
%
%    \begin{macrocode}
\def\HOLOGO@temp#1#2{%
  \kv@define@key{HoLogoDriver}{#1}[]{%
    \begingroup
      \def\HOLOGO@temp{##1}%
      \ltx@onelevel@sanitize\HOLOGO@temp
      \ifx\HOLOGO@temp\ltx@empty
      \else
        \@PackageError{hologo}{%
          Value (\HOLOGO@temp) not permitted for option `#1'%
        }%
        \@ehc
      \fi
    \endgroup
    \def\hologoDriver{#2}%
  }%
}%
\def\HOLOGO@@temp#1#2{%
  \ifx\kv@value\relax
    \HOLOGO@temp{#1}{#1}%
  \else
    \HOLOGO@temp{#1}{#2}%
  \fi
}%
\kv@parse@normalized{%
  pdftex,%
  luatex=pdftex,%
  dvipdfm,%
  dvipdfmx=dvipdfm,%
  dvips,%
  dvipsone=dvips,%
  xdvi=dvips,%
  xetex,%
  vtex,%
}\HOLOGO@@temp
%    \end{macrocode}
%
%    \begin{macrocode}
\kv@define@key{HoLogoDriver}{driverfallback}{%
  \def\HOLOGO@DriverFallback{#1}%
}
%    \end{macrocode}
%
%    \begin{macro}{\HOLOGO@DriverFallback}
%    \begin{macrocode}
\def\HOLOGO@DriverFallback{dvips}
%    \end{macrocode}
%    \end{macro}
%
%    \begin{macro}{\hologoDriverSetup}
%    \begin{macrocode}
\def\hologoDriverSetup{%
  \let\hologoDriver\ltx@undefined
  \HOLOGO@DriverSetup
}
%    \end{macrocode}
%    \end{macro}
%
%    \begin{macro}{\HOLOGO@DriverSetup}
%    \begin{macrocode}
\def\HOLOGO@DriverSetup#1{%
  \kvsetkeys{HoLogoDriver}{#1}%
  \HOLOGO@CheckDriver
  \ltx@ifundefined{hologoDriver}{%
    \begingroup
    \edef\x{\endgroup
      \noexpand\kvsetkeys{HoLogoDriver}{\HOLOGO@DriverFallback}%
    }\x
  }{}%
  \@PackageInfoNoLine{hologo}{Using driver `\hologoDriver'}%
}
%    \end{macrocode}
%    \end{macro}
%
%    \begin{macro}{\HOLOGO@CheckDriver}
%    \begin{macrocode}
\def\HOLOGO@CheckDriver{%
  \ifpdf
    \def\hologoDriver{pdftex}%
    \let\HOLOGO@pdfliteral\pdfliteral
    \ifluatex
      \ifx\pdfextension\@undefined\else
        \protected\def\pdfliteral{\pdfextension literal}%
        \let\HOLOGO@pdfliteral\pdfliteral
      \fi
      \ltx@IfUndefined{HOLOGO@pdfliteral}{%
        \ifnum\luatexversion<36 %
        \else
          \begingroup
            \let\HOLOGO@temp\endgroup
            \ifcase0%
                \directlua{%
                  if tex.enableprimitives then %
                    tex.enableprimitives('HOLOGO@', {'pdfliteral'})%
                  else %
                    tex.print('1')%
                  end%
                }%
                \ifx\HOLOGO@pdfliteral\@undefined 1\fi%
                \relax%
              \endgroup
              \let\HOLOGO@temp\relax
              \global\let\HOLOGO@pdfliteral\HOLOGO@pdfliteral
            \fi%
          \HOLOGO@temp
        \fi
      }{}%
    \fi
    \ltx@IfUndefined{HOLOGO@pdfliteral}{%
      \@PackageWarningNoLine{hologo}{%
        Cannot find \string\pdfliteral
      }%
    }{}%
  \else
    \ifxetex
      \def\hologoDriver{xetex}%
    \else
      \ifvtex
        \def\hologoDriver{vtex}%
      \fi
    \fi
  \fi
}
%    \end{macrocode}
%    \end{macro}
%
%    \begin{macro}{\HOLOGO@WarningUnsupportedDriver}
%    \begin{macrocode}
\def\HOLOGO@WarningUnsupportedDriver#1{%
  \@PackageWarningNoLine{hologo}{%
    Logo `#1' needs driver specific macros,\MessageBreak
    but driver `\hologoDriver' is not supported.\MessageBreak
    Use a different driver or\MessageBreak
    load package `graphics' or `pgf'%
  }%
}
%    \end{macrocode}
%    \end{macro}
%
% \subsubsection{Reflect box macros}
%
%    Skip driver part if not needed.
%    \begin{macrocode}
\ltx@IfUndefined{reflectbox}{}{%
  \ltx@IfUndefined{rotatebox}{}{%
    \HOLOGO@AtEnd
  }%
}
\ltx@IfUndefined{pgftext}{}{%
  \HOLOGO@AtEnd
}
\ltx@IfUndefined{psscalebox}{}{%
  \HOLOGO@AtEnd
}
%    \end{macrocode}
%
%    \begin{macrocode}
\def\HOLOGO@temp{LaTeX2e}
\ifx\fmtname\HOLOGO@temp
  \RequirePackage{kvoptions}[2011/06/30]%
  \ProcessKeyvalOptions{HoLogoDriver}%
\fi
\HOLOGO@DriverSetup{}
%    \end{macrocode}
%
%    \begin{macro}{\HOLOGO@ReflectBox}
%    \begin{macrocode}
\def\HOLOGO@ReflectBox#1{%
  \begingroup
    \setbox\ltx@zero\hbox{\begingroup#1\endgroup}%
    \setbox\ltx@two\hbox{%
      \kern\wd\ltx@zero
      \csname HOLOGO@ScaleBox@\hologoDriver\endcsname{-1}{1}{%
        \hbox to 0pt{\copy\ltx@zero\hss}%
      }%
    }%
    \wd\ltx@two=\wd\ltx@zero
    \box\ltx@two
  \endgroup
}
%    \end{macrocode}
%    \end{macro}
%
%    \begin{macro}{\HOLOGO@PointReflectBox}
%    \begin{macrocode}
\def\HOLOGO@PointReflectBox#1{%
  \begingroup
    \setbox\ltx@zero\hbox{\begingroup#1\endgroup}%
    \setbox\ltx@two\hbox{%
      \kern\wd\ltx@zero
      \raise\ht\ltx@zero\hbox{%
        \csname HOLOGO@ScaleBox@\hologoDriver\endcsname{-1}{-1}{%
          \hbox to 0pt{\copy\ltx@zero\hss}%
        }%
      }%
    }%
    \wd\ltx@two=\wd\ltx@zero
    \box\ltx@two
  \endgroup
}
%    \end{macrocode}
%    \end{macro}
%
%    We must define all variants because of dynamic driver setup.
%    \begin{macrocode}
\def\HOLOGO@temp#1#2{#2}
%    \end{macrocode}
%
%    \begin{macro}{\HOLOGO@ScaleBox@pdftex}
%    \begin{macrocode}
\HOLOGO@temp{pdftex}{%
  \def\HOLOGO@ScaleBox@pdftex#1#2#3{%
    \HOLOGO@pdfliteral{%
      q #1 0 0 #2 0 0 cm%
    }%
    #3%
    \HOLOGO@pdfliteral{%
      Q%
    }%
  }%
}
%    \end{macrocode}
%    \end{macro}
%    \begin{macro}{\HOLOGO@ScaleBox@dvips}
%    \begin{macrocode}
\HOLOGO@temp{dvips}{%
  \def\HOLOGO@ScaleBox@dvips#1#2#3{%
    \special{ps:%
      gsave %
      currentpoint %
      currentpoint translate %
      #1 #2 scale %
      neg exch neg exch translate%
    }%
    #3%
    \special{ps:%
      currentpoint %
      grestore %
      moveto%
    }%
  }%
}
%    \end{macrocode}
%    \end{macro}
%    \begin{macro}{\HOLOGO@ScaleBox@dvipdfm}
%    \begin{macrocode}
\HOLOGO@temp{dvipdfm}{%
  \let\HOLOGO@ScaleBox@dvipdfm\HOLOGO@ScaleBox@dvips
}
%    \end{macrocode}
%    \end{macro}
%    Since \hologo{XeTeX} v0.6.
%    \begin{macro}{\HOLOGO@ScaleBox@xetex}
%    \begin{macrocode}
\HOLOGO@temp{xetex}{%
  \def\HOLOGO@ScaleBox@xetex#1#2#3{%
    \special{x:gsave}%
    \special{x:scale #1 #2}%
    #3%
    \special{x:grestore}%
  }%
}
%    \end{macrocode}
%    \end{macro}
%    \begin{macro}{\HOLOGO@ScaleBox@vtex}
%    \begin{macrocode}
\HOLOGO@temp{vtex}{%
  \def\HOLOGO@ScaleBox@vtex#1#2#3{%
    \special{r(#1,0,0,#2,0,0}%
    #3%
    \special{r)}%
  }%
}
%    \end{macrocode}
%    \end{macro}
%
%    \begin{macrocode}
\HOLOGO@AtEnd%
%</package>
%    \end{macrocode}
%
% \section{Test}
%
% \subsection{Catcode checks for loading}
%
%    \begin{macrocode}
%<*test1>
%    \end{macrocode}
%    \begin{macrocode}
\catcode`\{=1 %
\catcode`\}=2 %
\catcode`\#=6 %
\catcode`\@=11 %
\expandafter\ifx\csname count@\endcsname\relax
  \countdef\count@=255 %
\fi
\expandafter\ifx\csname @gobble\endcsname\relax
  \long\def\@gobble#1{}%
\fi
\expandafter\ifx\csname @firstofone\endcsname\relax
  \long\def\@firstofone#1{#1}%
\fi
\expandafter\ifx\csname loop\endcsname\relax
  \expandafter\@firstofone
\else
  \expandafter\@gobble
\fi
{%
  \def\loop#1\repeat{%
    \def\body{#1}%
    \iterate
  }%
  \def\iterate{%
    \body
      \let\next\iterate
    \else
      \let\next\relax
    \fi
    \next
  }%
  \let\repeat=\fi
}%
\def\RestoreCatcodes{}
\count@=0 %
\loop
  \edef\RestoreCatcodes{%
    \RestoreCatcodes
    \catcode\the\count@=\the\catcode\count@\relax
  }%
\ifnum\count@<255 %
  \advance\count@ 1 %
\repeat

\def\RangeCatcodeInvalid#1#2{%
  \count@=#1\relax
  \loop
    \catcode\count@=15 %
  \ifnum\count@<#2\relax
    \advance\count@ 1 %
  \repeat
}
\def\RangeCatcodeCheck#1#2#3{%
  \count@=#1\relax
  \loop
    \ifnum#3=\catcode\count@
    \else
      \errmessage{%
        Character \the\count@\space
        with wrong catcode \the\catcode\count@\space
        instead of \number#3%
      }%
    \fi
  \ifnum\count@<#2\relax
    \advance\count@ 1 %
  \repeat
}
\def\space{ }
\expandafter\ifx\csname LoadCommand\endcsname\relax
  \def\LoadCommand{\input hologo.sty\relax}%
\fi
\def\Test{%
  \RangeCatcodeInvalid{0}{47}%
  \RangeCatcodeInvalid{58}{64}%
  \RangeCatcodeInvalid{91}{96}%
  \RangeCatcodeInvalid{123}{255}%
  \catcode`\@=12 %
  \catcode`\\=0 %
  \catcode`\%=14 %
  \LoadCommand
  \RangeCatcodeCheck{0}{36}{15}%
  \RangeCatcodeCheck{37}{37}{14}%
  \RangeCatcodeCheck{38}{47}{15}%
  \RangeCatcodeCheck{48}{57}{12}%
  \RangeCatcodeCheck{58}{63}{15}%
  \RangeCatcodeCheck{64}{64}{12}%
  \RangeCatcodeCheck{65}{90}{11}%
  \RangeCatcodeCheck{91}{91}{15}%
  \RangeCatcodeCheck{92}{92}{0}%
  \RangeCatcodeCheck{93}{96}{15}%
  \RangeCatcodeCheck{97}{122}{11}%
  \RangeCatcodeCheck{123}{255}{15}%
  \RestoreCatcodes
}
\Test
\csname @@end\endcsname
\end
%    \end{macrocode}
%    \begin{macrocode}
%</test1>
%    \end{macrocode}
%
% \subsection{Spacefactor}
%
%    The space factor must be 1000 after a logo. If it is greater 1000
%    then the following space is a space after a sentence closing point.
%    If the space factor is smaller 1000 then an immediate following
%    dot is interpreted as abbreviation, not sentence closing point.
%
%    \begin{macrocode}
%<*test-spacefactor>
\NeedsTeXFormat{LaTeX2e}
\documentclass{article}
\usepackage{hologo}[2016/05/12]
\usepackage{kvsetkeys}
\usepackage{qstest}
\IncludeTests{*}
\LogTests{log}{*}{*}
\begin{document}
\begin{qstest}{spacefactor}{spacefactor}
\newcommand*{\Test}[1]{%
  \sbox0{%
    \hologo{#1}%
    \Expect*{1000 (#1)}*{\the\spacefactor\space(#1)}%
  }%
}%
\makeatletter
\def\TestList{}
\def\hologoEntry#1#2#3{%
  \edef\TestList{%
    \ifx\TestList\@empty
    \else
      \TestList,%
    \fi
    #1%
    \ifx\\#2\\%
    \else
      ={variant=#2}%
    \fi
  }%
}
\hologoList
\expandafter\kv@parse@normalized\expandafter{%
  \TestList
}{%
  \begingroup
    \let\@logo=\kv@key
    \ifx\kv@value\relax
    \else
      \expandafter\hologoLogoSetup\expandafter\@logo\expandafter{%
        \kv@value
      }%
    \fi
    \Test\@logo
  \endgroup
  \@gobbletwo
}
\end{qstest}
\end{document}
%</test-spacefactor>
%    \end{macrocode}
%
% \subsection{Complete list}
%
%    \begin{macrocode}
%<*test-list>
\NeedsTeXFormat{LaTeX2e}
\documentclass[12pt,a4paper]{article}
\usepackage{hologo}[2016/05/12]
\usepackage[T1]{fontenc}
\usepackage{lmodern}
\usepackage{parskip}
\usepackage[unicode]{hyperref}[2011/09/28]
\usepackage{bookmark}[2011/09/19]
\bookmarksetup{%
  numbered,%
  open,%
  openlevel=2,%
}
\renewcommand*{\contentsname}{List of logos}
\begin{document}
\tableofcontents
\def\TestFont#1#2#3#4#5#6{%
  \begingroup
    \usefont{#3}{#4}{#5}{#6}%
    \HologoVariant{#1}{#2}/\hologoVariant{#1}{#2}%
    \quad
    \begingroup\scriptsize\hologoVariant{#1}{#2}\endgroup
    \quad
  \endgroup
  (#3/#4/#5/#6)%
  \par
}
\makeatletter
\def\hologoEntry#1#2#3{%
  \section{%
    \HologoVariant{#1}{#2}/\hologoVariant{#1}{#2} %
    {[#1\ifx\\#2\\\else\space(#2)\fi]}% hash-ok
  }% braces around [] because of bug in tex4ht
  \begingroup
    \hypersetup{unicode=false}%
    \bookmark[%
      dest=\@currentHref,%
      rellevel=1,%
      keeplevel,%
    ]{%
      \HologoVariant{#1}{#2}/\hologoVariant{#1}{#2} %
      (PDFDocEncoding)%
    }%
  \endgroup
  \TestFont{#1}{#2}{OT1}{cmr}{m}{n}%
  \TestFont{#1}{#2}{OT1}{cmss}{m}{n}%
  \TestFont{#1}{#2}{OT1}{cmr}{b}{n}%
  \TestFont{#1}{#2}{OT1}{cmr}{m}{it}%
  \TestFont{#1}{#2}{OT1}{cmtt}{m}{n}%
  \TestFont{#1}{#2}{T1}{lmr}{m}{n}%
  \TestFont{#1}{#2}{T1}{lmss}{m}{n}%
  \TestFont{#1}{#2}{T1}{lmr}{b}{n}%
  \TestFont{#1}{#2}{T1}{lmr}{m}{it}%
  \TestFont{#1}{#2}{T1}{lmtt}{m}{n}%
  \TestFont{#1}{#2}{T1}{lmvtt}{m}{n}%
  \TestFont{#1}{#2}{T1}{qtm}{m}{n}%
  \TestFont{#1}{#2}{T1}{qhv}{m}{n}%
  \TestFont{#1}{#2}{T1}{qtm}{b}{n}%
  \TestFont{#1}{#2}{T1}{qtm}{m}{it}%
  \TestFont{#1}{#2}{T1}{qcr}{m}{n}%
  \newpage
}
\makeatother
\hologoList
\end{document}
%</test-list>
%    \end{macrocode}
%
% \section{Installation}
%
% \subsection{Download}
%
% \paragraph{Package.} This package is available on
% CTAN\footnote{\url{ftp://ftp.ctan.org/tex-archive/}}:
% \begin{description}
% \item[\CTAN{macros/latex/contrib/oberdiek/hologo.dtx}] The source file.
% \item[\CTAN{macros/latex/contrib/oberdiek/hologo.pdf}] Documentation.
% \end{description}
%
%
% \paragraph{Bundle.} All the packages of the bundle `oberdiek'
% are also available in a TDS compliant ZIP archive. There
% the packages are already unpacked and the documentation files
% are generated. The files and directories obey the TDS standard.
% \begin{description}
% \item[\CTAN{install/macros/latex/contrib/oberdiek.tds.zip}]
% \end{description}
% \emph{TDS} refers to the standard ``A Directory Structure
% for \TeX\ Files'' (\CTAN{tds/tds.pdf}). Directories
% with \xfile{texmf} in their name are usually organized this way.
%
% \subsection{Bundle installation}
%
% \paragraph{Unpacking.} Unpack the \xfile{oberdiek.tds.zip} in the
% TDS tree (also known as \xfile{texmf} tree) of your choice.
% Example (linux):
% \begin{quote}
%   |unzip oberdiek.tds.zip -d ~/texmf|
% \end{quote}
%
% \paragraph{Script installation.}
% Check the directory \xfile{TDS:scripts/oberdiek/} for
% scripts that need further installation steps.
% Package \xpackage{attachfile2} comes with the Perl script
% \xfile{pdfatfi.pl} that should be installed in such a way
% that it can be called as \texttt{pdfatfi}.
% Example (linux):
% \begin{quote}
%   |chmod +x scripts/oberdiek/pdfatfi.pl|\\
%   |cp scripts/oberdiek/pdfatfi.pl /usr/local/bin/|
% \end{quote}
%
% \subsection{Package installation}
%
% \paragraph{Unpacking.} The \xfile{.dtx} file is a self-extracting
% \docstrip\ archive. The files are extracted by running the
% \xfile{.dtx} through \plainTeX:
% \begin{quote}
%   \verb|tex hologo.dtx|
% \end{quote}
%
% \paragraph{TDS.} Now the different files must be moved into
% the different directories in your installation TDS tree
% (also known as \xfile{texmf} tree):
% \begin{quote}
% \def\t{^^A
% \begin{tabular}{@{}>{\ttfamily}l@{ $\rightarrow$ }>{\ttfamily}l@{}}
%   hologo.sty & tex/generic/oberdiek/hologo.sty\\
%   hologo.pdf & doc/latex/oberdiek/hologo.pdf\\
%   example/hologo-example.tex & doc/latex/oberdiek/example/hologo-example.tex\\
%   test/hologo-test1.tex & doc/latex/oberdiek/test/hologo-test1.tex\\
%   test/hologo-test-spacefactor.tex & doc/latex/oberdiek/test/hologo-test-spacefactor.tex\\
%   test/hologo-test-list.tex & doc/latex/oberdiek/test/hologo-test-list.tex\\
%   hologo.dtx & source/latex/oberdiek/hologo.dtx\\
% \end{tabular}^^A
% }^^A
% \sbox0{\t}^^A
% \ifdim\wd0>\linewidth
%   \begingroup
%     \advance\linewidth by\leftmargin
%     \advance\linewidth by\rightmargin
%   \edef\x{\endgroup
%     \def\noexpand\lw{\the\linewidth}^^A
%   }\x
%   \def\lwbox{^^A
%     \leavevmode
%     \hbox to \linewidth{^^A
%       \kern-\leftmargin\relax
%       \hss
%       \usebox0
%       \hss
%       \kern-\rightmargin\relax
%     }^^A
%   }^^A
%   \ifdim\wd0>\lw
%     \sbox0{\small\t}^^A
%     \ifdim\wd0>\linewidth
%       \ifdim\wd0>\lw
%         \sbox0{\footnotesize\t}^^A
%         \ifdim\wd0>\linewidth
%           \ifdim\wd0>\lw
%             \sbox0{\scriptsize\t}^^A
%             \ifdim\wd0>\linewidth
%               \ifdim\wd0>\lw
%                 \sbox0{\tiny\t}^^A
%                 \ifdim\wd0>\linewidth
%                   \lwbox
%                 \else
%                   \usebox0
%                 \fi
%               \else
%                 \lwbox
%               \fi
%             \else
%               \usebox0
%             \fi
%           \else
%             \lwbox
%           \fi
%         \else
%           \usebox0
%         \fi
%       \else
%         \lwbox
%       \fi
%     \else
%       \usebox0
%     \fi
%   \else
%     \lwbox
%   \fi
% \else
%   \usebox0
% \fi
% \end{quote}
% If you have a \xfile{docstrip.cfg} that configures and enables \docstrip's
% TDS installing feature, then some files can already be in the right
% place, see the documentation of \docstrip.
%
% \subsection{Refresh file name databases}
%
% If your \TeX~distribution
% (\teTeX, \mikTeX, \dots) relies on file name databases, you must refresh
% these. For example, \teTeX\ users run \verb|texhash| or
% \verb|mktexlsr|.
%
% \subsection{Some details for the interested}
%
% \paragraph{Attached source.}
%
% The PDF documentation on CTAN also includes the
% \xfile{.dtx} source file. It can be extracted by
% AcrobatReader 6 or higher. Another option is \textsf{pdftk},
% e.g. unpack the file into the current directory:
% \begin{quote}
%   \verb|pdftk hologo.pdf unpack_files output .|
% \end{quote}
%
% \paragraph{Unpacking with \LaTeX.}
% The \xfile{.dtx} chooses its action depending on the format:
% \begin{description}
% \item[\plainTeX:] Run \docstrip\ and extract the files.
% \item[\LaTeX:] Generate the documentation.
% \end{description}
% If you insist on using \LaTeX\ for \docstrip\ (really,
% \docstrip\ does not need \LaTeX), then inform the autodetect routine
% about your intention:
% \begin{quote}
%   \verb|latex \let\install=y\input{hologo.dtx}|
% \end{quote}
% Do not forget to quote the argument according to the demands
% of your shell.
%
% \paragraph{Generating the documentation.}
% You can use both the \xfile{.dtx} or the \xfile{.drv} to generate
% the documentation. The process can be configured by the
% configuration file \xfile{ltxdoc.cfg}. For instance, put this
% line into this file, if you want to have A4 as paper format:
% \begin{quote}
%   \verb|\PassOptionsToClass{a4paper}{article}|
% \end{quote}
% An example follows how to generate the
% documentation with pdf\LaTeX:
% \begin{quote}
%\begin{verbatim}
%pdflatex hologo.dtx
%makeindex -s gind.ist hologo.idx
%pdflatex hologo.dtx
%makeindex -s gind.ist hologo.idx
%pdflatex hologo.dtx
%\end{verbatim}
% \end{quote}
%
% \section{Catalogue}
%
% The following XML file can be used as source for the
% \href{http://mirror.ctan.org/help/Catalogue/catalogue.html}{\TeX\ Catalogue}.
% The elements \texttt{caption} and \texttt{description} are imported
% from the original XML file from the Catalogue.
% The name of the XML file in the Catalogue is \xfile{hologo.xml}.
%    \begin{macrocode}
%<*catalogue>
<?xml version='1.0' encoding='us-ascii'?>
<!DOCTYPE entry SYSTEM 'catalogue.dtd'>
<entry datestamp='$Date$' modifier='$Author$' id='hologo'>
  <name>hologo</name>
  <caption>A collection of logos with bookmark support.</caption>
  <authorref id='auth:oberdiek'/>
  <copyright owner='Heiko Oberdiek' year='2010-2012'/>
  <license type='lppl1.3'/>
  <version number='1.10'/>
  <description>
    The package defines a single command <tt>\hologo</tt>, whose
    argument is the usual case-confused ASCII version of the logo.
    The command is bookmark-enabled, so that every logo becomes
    available in bookmarks without further work.
    <p/>
    The package is part of the <xref refid='oberdiek'>oberdiek</xref>
    bundle.
  </description>
  <documentation details='Package documentation'
      href='ctan:/macros/latex/contrib/oberdiek/hologo.pdf'/>
  <ctan file='true' path='/macros/latex/contrib/oberdiek/hologo.dtx'/>
  <miktex location='oberdiek'/>
  <texlive location='oberdiek'/>
  <install path='/macros/latex/contrib/oberdiek/oberdiek.tds.zip'/>
</entry>
%</catalogue>
%    \end{macrocode}
%
% \begin{thebibliography}{9}
% \raggedright
%
% \bibitem{btxdoc}
% Oren Patashnik,
% \textit{\hologo{BibTeX}ing},
% 1988-02-08.\\
% \CTAN{biblio/bibtex/base/}
%
% \bibitem{dtklogos}
% Gerd Neugebauer, DANTE,
% \textit{Package \xpackage{dtklogos}},
% 2011-04-25.\\
% \CTAN{usergrps/dante/dtk/dtklogos.sty}
%
% \bibitem{etexman}
% The \hologo{NTS} Team,
% \textit{The \hologo{eTeX} manual},
% 1998-02.\\
% \CTAN{systems/e-tex/v2/doc/}
%
% \bibitem{ExTeX-FAQ}
% The \hologo{ExTeX} group,
% \textit{\hologo{ExTeX}: FAQ -- How is \hologo{ExTeX} typeset?},
% 2007-04-14.\\
% \url{http://www.extex.org/documentation/faq.html}
%
% \bibitem{LyX}
% %@MISC{ LyX,
% %  title = {{LyX 2.0.0 -- The Document Processor [Computer software and manual]}},
% %  author = {{The LyX Team}},
% %  howpublished = {Internet: http://www.lyx.org},
% %  year = {2011-05-08},
% %  note = {Retrieved May 10, 2011, from http://www.lyx.org},
% %  url = {http://www.lyx.org/}
% %}
% The \hologo{LyX} Team,
% \textit{\hologo{LyX} -- The Document Processor},
% 2011-05-08.\\
% \url{http://www.lyx.org/}
%
% \bibitem{OzTeX}
% Andrew Trevorrow,
% \hologo{OzTeX} FAQ: What is the correct way to typeset ``\hologo{OzTeX}''?,
% 2011-09-15 (visited).
% \url{http://www.trevorrow.com/oztex/ozfaq.html#oztex-logo}
%
% \bibitem{PiCTeX}
% Michael Wichura,
% \textit{The \hologo{PiCTeX} macro package},
% 1987-09-21.
% \CTAN{graphics/pictex/}
%
% \bibitem{scrlogo}
% Markus Kohm,
% \textit{\hologo{KOMAScript} Datei \xfile{scrlogo.dtx}},
% 2009-01-30.\\
% \CTAN{install/macros/latex/contrib/komascript.tds.zip}
%
% \end{thebibliography}
%
% \begin{History}
%   \begin{Version}{2010/04/08 v1.0}
%   \item
%     The first version.
%   \end{Version}
%   \begin{Version}{2010/04/16 v1.1}
%   \item
%     \cs{Hologo} added for support of logos at start of a sentence.
%   \item
%     \cs{hologoSetup} and \cs{hologoLogoSetup} added.
%   \item
%     Options \xoption{break}, \xoption{hyphenbreak}, \xoption{spacebreak}
%     added.
%   \item
%     Variant support added by option \xoption{variant}.
%   \end{Version}
%   \begin{Version}{2010/04/24 v1.2}
%   \item
%     \hologo{LaTeX3} added.
%   \item
%     \hologo{VTeX} added.
%   \end{Version}
%   \begin{Version}{2010/11/21 v1.3}
%   \item
%     \hologo{iniTeX}, \hologo{virTeX} added.
%   \end{Version}
%   \begin{Version}{2011/03/25 v1.4}
%   \item
%     \hologo{ConTeXt} with variants added.
%   \item
%     Option \xoption{discretionarybreak} added as refinement for
%     option \xoption{break}.
%   \end{Version}
%   \begin{Version}{2011/04/21 v1.5}
%   \item
%     Wrong TDS directory for test files fixed.
%   \end{Version}
%   \begin{Version}{2011/10/01 v1.6}
%   \item
%     Support for package \xpackage{tex4ht} added.
%   \item
%     Support for \cs{csname} added if \cs{ifincsname} is available.
%   \item
%     New logos:
%     \hologo{(La)TeX},
%     \hologo{biber},
%     \hologo{BibTeX} (\xoption{sc}, \xoption{sf}),
%     \hologo{emTeX},
%     \hologo{ExTeX},
%     \hologo{KOMAScript},
%     \hologo{La},
%     \hologo{LyX},
%     \hologo{MiKTeX},
%     \hologo{NTS},
%     \hologo{OzMF},
%     \hologo{OzMP},
%     \hologo{OzTeX},
%     \hologo{OzTtH},
%     \hologo{PCTeX},
%     \hologo{PiC},
%     \hologo{PiCTeX},
%     \hologo{METAFONT},
%     \hologo{MetaFun},
%     \hologo{METAPOST},
%     \hologo{MetaPost},
%     \hologo{SLiTeX} (\xoption{lift}, \xoption{narrow}, \xoption{simple}),
%     \hologo{SliTeX} (\xoption{narrow}, \xoption{simple}, \xoption{lift}),
%     \hologo{teTeX}.
%   \item
%     Fixes:
%     \hologo{iniTeX},
%     \hologo{pdfLaTeX},
%     \hologo{pdfTeX},
%     \hologo{virTeX}.
%   \item
%     \cs{hologoFontSetup} and \cs{hologoLogoFontSetup} added.
%   \item
%     \cs{hologoVariant} and \cs{HologoVariant} added.
%   \end{Version}
%   \begin{Version}{2011/11/22 v1.7}
%   \item
%     New logos:
%     \hologo{BibTeX8},
%     \hologo{LaTeXML},
%     \hologo{SageTeX},
%     \hologo{TeX4ht},
%     \hologo{TTH}.
%   \item
%     \hologo{Xe} and friends: Driver stuff fixed.
%   \item
%     \hologo{Xe} and friends: Support for italic added.
%   \item
%     \hologo{Xe} and friends: Package support for \xpackage{pgf}
%     and \xpackage{pstricks} added.
%   \end{Version}
%   \begin{Version}{2011/11/29 v1.8}
%   \item
%     New logos:
%     \hologo{HanTheThanh}.
%   \end{Version}
%   \begin{Version}{2011/12/21 v1.9}
%   \item
%     Patch for package \xpackage{ifxetex} added for the case that
%     \cs{newif} is undefined in \hologo{iniTeX}.
%   \item
%     Some fixes for \hologo{iniTeX}.
%   \end{Version}
%   \begin{Version}{2012/04/26 v1.10}
%   \item
%     Fix in bookmark version of logo ``\hologo{HanTheThanh}''.
%   \end{Version}
%   \begin{Version}{2016/05/12 v1.11}
%   \item
%     Update HOLOGO@IfCharExists (previously in texlive)
%   \item define pdfliteral in current luatex.
%   \end{Version}
% \end{History}
%
% \PrintIndex
%
% \Finale
\endinput
|
% \end{quote}
% Do not forget to quote the argument according to the demands
% of your shell.
%
% \paragraph{Generating the documentation.}
% You can use both the \xfile{.dtx} or the \xfile{.drv} to generate
% the documentation. The process can be configured by the
% configuration file \xfile{ltxdoc.cfg}. For instance, put this
% line into this file, if you want to have A4 as paper format:
% \begin{quote}
%   \verb|\PassOptionsToClass{a4paper}{article}|
% \end{quote}
% An example follows how to generate the
% documentation with pdf\LaTeX:
% \begin{quote}
%\begin{verbatim}
%pdflatex hologo.dtx
%makeindex -s gind.ist hologo.idx
%pdflatex hologo.dtx
%makeindex -s gind.ist hologo.idx
%pdflatex hologo.dtx
%\end{verbatim}
% \end{quote}
%
% \section{Catalogue}
%
% The following XML file can be used as source for the
% \href{http://mirror.ctan.org/help/Catalogue/catalogue.html}{\TeX\ Catalogue}.
% The elements \texttt{caption} and \texttt{description} are imported
% from the original XML file from the Catalogue.
% The name of the XML file in the Catalogue is \xfile{hologo.xml}.
%    \begin{macrocode}
%<*catalogue>
<?xml version='1.0' encoding='us-ascii'?>
<!DOCTYPE entry SYSTEM 'catalogue.dtd'>
<entry datestamp='$Date$' modifier='$Author$' id='hologo'>
  <name>hologo</name>
  <caption>A collection of logos with bookmark support.</caption>
  <authorref id='auth:oberdiek'/>
  <copyright owner='Heiko Oberdiek' year='2010-2012'/>
  <license type='lppl1.3'/>
  <version number='1.10'/>
  <description>
    The package defines a single command <tt>\hologo</tt>, whose
    argument is the usual case-confused ASCII version of the logo.
    The command is bookmark-enabled, so that every logo becomes
    available in bookmarks without further work.
    <p/>
    The package is part of the <xref refid='oberdiek'>oberdiek</xref>
    bundle.
  </description>
  <documentation details='Package documentation'
      href='ctan:/macros/latex/contrib/oberdiek/hologo.pdf'/>
  <ctan file='true' path='/macros/latex/contrib/oberdiek/hologo.dtx'/>
  <miktex location='oberdiek'/>
  <texlive location='oberdiek'/>
  <install path='/macros/latex/contrib/oberdiek/oberdiek.tds.zip'/>
</entry>
%</catalogue>
%    \end{macrocode}
%
% \begin{thebibliography}{9}
% \raggedright
%
% \bibitem{btxdoc}
% Oren Patashnik,
% \textit{\hologo{BibTeX}ing},
% 1988-02-08.\\
% \CTAN{biblio/bibtex/base/}
%
% \bibitem{dtklogos}
% Gerd Neugebauer, DANTE,
% \textit{Package \xpackage{dtklogos}},
% 2011-04-25.\\
% \CTAN{usergrps/dante/dtk/dtklogos.sty}
%
% \bibitem{etexman}
% The \hologo{NTS} Team,
% \textit{The \hologo{eTeX} manual},
% 1998-02.\\
% \CTAN{systems/e-tex/v2/doc/}
%
% \bibitem{ExTeX-FAQ}
% The \hologo{ExTeX} group,
% \textit{\hologo{ExTeX}: FAQ -- How is \hologo{ExTeX} typeset?},
% 2007-04-14.\\
% \url{http://www.extex.org/documentation/faq.html}
%
% \bibitem{LyX}
% %@MISC{ LyX,
% %  title = {{LyX 2.0.0 -- The Document Processor [Computer software and manual]}},
% %  author = {{The LyX Team}},
% %  howpublished = {Internet: http://www.lyx.org},
% %  year = {2011-05-08},
% %  note = {Retrieved May 10, 2011, from http://www.lyx.org},
% %  url = {http://www.lyx.org/}
% %}
% The \hologo{LyX} Team,
% \textit{\hologo{LyX} -- The Document Processor},
% 2011-05-08.\\
% \url{http://www.lyx.org/}
%
% \bibitem{OzTeX}
% Andrew Trevorrow,
% \hologo{OzTeX} FAQ: What is the correct way to typeset ``\hologo{OzTeX}''?,
% 2011-09-15 (visited).
% \url{http://www.trevorrow.com/oztex/ozfaq.html#oztex-logo}
%
% \bibitem{PiCTeX}
% Michael Wichura,
% \textit{The \hologo{PiCTeX} macro package},
% 1987-09-21.
% \CTAN{graphics/pictex/}
%
% \bibitem{scrlogo}
% Markus Kohm,
% \textit{\hologo{KOMAScript} Datei \xfile{scrlogo.dtx}},
% 2009-01-30.\\
% \CTAN{install/macros/latex/contrib/komascript.tds.zip}
%
% \end{thebibliography}
%
% \begin{History}
%   \begin{Version}{2010/04/08 v1.0}
%   \item
%     The first version.
%   \end{Version}
%   \begin{Version}{2010/04/16 v1.1}
%   \item
%     \cs{Hologo} added for support of logos at start of a sentence.
%   \item
%     \cs{hologoSetup} and \cs{hologoLogoSetup} added.
%   \item
%     Options \xoption{break}, \xoption{hyphenbreak}, \xoption{spacebreak}
%     added.
%   \item
%     Variant support added by option \xoption{variant}.
%   \end{Version}
%   \begin{Version}{2010/04/24 v1.2}
%   \item
%     \hologo{LaTeX3} added.
%   \item
%     \hologo{VTeX} added.
%   \end{Version}
%   \begin{Version}{2010/11/21 v1.3}
%   \item
%     \hologo{iniTeX}, \hologo{virTeX} added.
%   \end{Version}
%   \begin{Version}{2011/03/25 v1.4}
%   \item
%     \hologo{ConTeXt} with variants added.
%   \item
%     Option \xoption{discretionarybreak} added as refinement for
%     option \xoption{break}.
%   \end{Version}
%   \begin{Version}{2011/04/21 v1.5}
%   \item
%     Wrong TDS directory for test files fixed.
%   \end{Version}
%   \begin{Version}{2011/10/01 v1.6}
%   \item
%     Support for package \xpackage{tex4ht} added.
%   \item
%     Support for \cs{csname} added if \cs{ifincsname} is available.
%   \item
%     New logos:
%     \hologo{(La)TeX},
%     \hologo{biber},
%     \hologo{BibTeX} (\xoption{sc}, \xoption{sf}),
%     \hologo{emTeX},
%     \hologo{ExTeX},
%     \hologo{KOMAScript},
%     \hologo{La},
%     \hologo{LyX},
%     \hologo{MiKTeX},
%     \hologo{NTS},
%     \hologo{OzMF},
%     \hologo{OzMP},
%     \hologo{OzTeX},
%     \hologo{OzTtH},
%     \hologo{PCTeX},
%     \hologo{PiC},
%     \hologo{PiCTeX},
%     \hologo{METAFONT},
%     \hologo{MetaFun},
%     \hologo{METAPOST},
%     \hologo{MetaPost},
%     \hologo{SLiTeX} (\xoption{lift}, \xoption{narrow}, \xoption{simple}),
%     \hologo{SliTeX} (\xoption{narrow}, \xoption{simple}, \xoption{lift}),
%     \hologo{teTeX}.
%   \item
%     Fixes:
%     \hologo{iniTeX},
%     \hologo{pdfLaTeX},
%     \hologo{pdfTeX},
%     \hologo{virTeX}.
%   \item
%     \cs{hologoFontSetup} and \cs{hologoLogoFontSetup} added.
%   \item
%     \cs{hologoVariant} and \cs{HologoVariant} added.
%   \end{Version}
%   \begin{Version}{2011/11/22 v1.7}
%   \item
%     New logos:
%     \hologo{BibTeX8},
%     \hologo{LaTeXML},
%     \hologo{SageTeX},
%     \hologo{TeX4ht},
%     \hologo{TTH}.
%   \item
%     \hologo{Xe} and friends: Driver stuff fixed.
%   \item
%     \hologo{Xe} and friends: Support for italic added.
%   \item
%     \hologo{Xe} and friends: Package support for \xpackage{pgf}
%     and \xpackage{pstricks} added.
%   \end{Version}
%   \begin{Version}{2011/11/29 v1.8}
%   \item
%     New logos:
%     \hologo{HanTheThanh}.
%   \end{Version}
%   \begin{Version}{2011/12/21 v1.9}
%   \item
%     Patch for package \xpackage{ifxetex} added for the case that
%     \cs{newif} is undefined in \hologo{iniTeX}.
%   \item
%     Some fixes for \hologo{iniTeX}.
%   \end{Version}
%   \begin{Version}{2012/04/26 v1.10}
%   \item
%     Fix in bookmark version of logo ``\hologo{HanTheThanh}''.
%   \end{Version}
%   \begin{Version}{2016/05/12 v1.11}
%   \item
%     Update HOLOGO@IfCharExists (previously in texlive)
%   \item define pdfliteral in current luatex.
%   \end{Version}
% \end{History}
%
% \PrintIndex
%
% \Finale
\endinput
|
% \end{quote}
% Do not forget to quote the argument according to the demands
% of your shell.
%
% \paragraph{Generating the documentation.}
% You can use both the \xfile{.dtx} or the \xfile{.drv} to generate
% the documentation. The process can be configured by the
% configuration file \xfile{ltxdoc.cfg}. For instance, put this
% line into this file, if you want to have A4 as paper format:
% \begin{quote}
%   \verb|\PassOptionsToClass{a4paper}{article}|
% \end{quote}
% An example follows how to generate the
% documentation with pdf\LaTeX:
% \begin{quote}
%\begin{verbatim}
%pdflatex hologo.dtx
%makeindex -s gind.ist hologo.idx
%pdflatex hologo.dtx
%makeindex -s gind.ist hologo.idx
%pdflatex hologo.dtx
%\end{verbatim}
% \end{quote}
%
% \section{Catalogue}
%
% The following XML file can be used as source for the
% \href{http://mirror.ctan.org/help/Catalogue/catalogue.html}{\TeX\ Catalogue}.
% The elements \texttt{caption} and \texttt{description} are imported
% from the original XML file from the Catalogue.
% The name of the XML file in the Catalogue is \xfile{hologo.xml}.
%    \begin{macrocode}
%<*catalogue>
<?xml version='1.0' encoding='us-ascii'?>
<!DOCTYPE entry SYSTEM 'catalogue.dtd'>
<entry datestamp='$Date$' modifier='$Author$' id='hologo'>
  <name>hologo</name>
  <caption>A collection of logos with bookmark support.</caption>
  <authorref id='auth:oberdiek'/>
  <copyright owner='Heiko Oberdiek' year='2010-2012'/>
  <license type='lppl1.3'/>
  <version number='1.10'/>
  <description>
    The package defines a single command <tt>\hologo</tt>, whose
    argument is the usual case-confused ASCII version of the logo.
    The command is bookmark-enabled, so that every logo becomes
    available in bookmarks without further work.
    <p/>
    The package is part of the <xref refid='oberdiek'>oberdiek</xref>
    bundle.
  </description>
  <documentation details='Package documentation'
      href='ctan:/macros/latex/contrib/oberdiek/hologo.pdf'/>
  <ctan file='true' path='/macros/latex/contrib/oberdiek/hologo.dtx'/>
  <miktex location='oberdiek'/>
  <texlive location='oberdiek'/>
  <install path='/macros/latex/contrib/oberdiek/oberdiek.tds.zip'/>
</entry>
%</catalogue>
%    \end{macrocode}
%
% \begin{thebibliography}{9}
% \raggedright
%
% \bibitem{btxdoc}
% Oren Patashnik,
% \textit{\hologo{BibTeX}ing},
% 1988-02-08.\\
% \CTAN{biblio/bibtex/base/}
%
% \bibitem{dtklogos}
% Gerd Neugebauer, DANTE,
% \textit{Package \xpackage{dtklogos}},
% 2011-04-25.\\
% \CTAN{usergrps/dante/dtk/dtklogos.sty}
%
% \bibitem{etexman}
% The \hologo{NTS} Team,
% \textit{The \hologo{eTeX} manual},
% 1998-02.\\
% \CTAN{systems/e-tex/v2/doc/}
%
% \bibitem{ExTeX-FAQ}
% The \hologo{ExTeX} group,
% \textit{\hologo{ExTeX}: FAQ -- How is \hologo{ExTeX} typeset?},
% 2007-04-14.\\
% \url{http://www.extex.org/documentation/faq.html}
%
% \bibitem{LyX}
% %@MISC{ LyX,
% %  title = {{LyX 2.0.0 -- The Document Processor [Computer software and manual]}},
% %  author = {{The LyX Team}},
% %  howpublished = {Internet: http://www.lyx.org},
% %  year = {2011-05-08},
% %  note = {Retrieved May 10, 2011, from http://www.lyx.org},
% %  url = {http://www.lyx.org/}
% %}
% The \hologo{LyX} Team,
% \textit{\hologo{LyX} -- The Document Processor},
% 2011-05-08.\\
% \url{http://www.lyx.org/}
%
% \bibitem{OzTeX}
% Andrew Trevorrow,
% \hologo{OzTeX} FAQ: What is the correct way to typeset ``\hologo{OzTeX}''?,
% 2011-09-15 (visited).
% \url{http://www.trevorrow.com/oztex/ozfaq.html#oztex-logo}
%
% \bibitem{PiCTeX}
% Michael Wichura,
% \textit{The \hologo{PiCTeX} macro package},
% 1987-09-21.
% \CTAN{graphics/pictex/}
%
% \bibitem{scrlogo}
% Markus Kohm,
% \textit{\hologo{KOMAScript} Datei \xfile{scrlogo.dtx}},
% 2009-01-30.\\
% \CTAN{install/macros/latex/contrib/komascript.tds.zip}
%
% \end{thebibliography}
%
% \begin{History}
%   \begin{Version}{2010/04/08 v1.0}
%   \item
%     The first version.
%   \end{Version}
%   \begin{Version}{2010/04/16 v1.1}
%   \item
%     \cs{Hologo} added for support of logos at start of a sentence.
%   \item
%     \cs{hologoSetup} and \cs{hologoLogoSetup} added.
%   \item
%     Options \xoption{break}, \xoption{hyphenbreak}, \xoption{spacebreak}
%     added.
%   \item
%     Variant support added by option \xoption{variant}.
%   \end{Version}
%   \begin{Version}{2010/04/24 v1.2}
%   \item
%     \hologo{LaTeX3} added.
%   \item
%     \hologo{VTeX} added.
%   \end{Version}
%   \begin{Version}{2010/11/21 v1.3}
%   \item
%     \hologo{iniTeX}, \hologo{virTeX} added.
%   \end{Version}
%   \begin{Version}{2011/03/25 v1.4}
%   \item
%     \hologo{ConTeXt} with variants added.
%   \item
%     Option \xoption{discretionarybreak} added as refinement for
%     option \xoption{break}.
%   \end{Version}
%   \begin{Version}{2011/04/21 v1.5}
%   \item
%     Wrong TDS directory for test files fixed.
%   \end{Version}
%   \begin{Version}{2011/10/01 v1.6}
%   \item
%     Support for package \xpackage{tex4ht} added.
%   \item
%     Support for \cs{csname} added if \cs{ifincsname} is available.
%   \item
%     New logos:
%     \hologo{(La)TeX},
%     \hologo{biber},
%     \hologo{BibTeX} (\xoption{sc}, \xoption{sf}),
%     \hologo{emTeX},
%     \hologo{ExTeX},
%     \hologo{KOMAScript},
%     \hologo{La},
%     \hologo{LyX},
%     \hologo{MiKTeX},
%     \hologo{NTS},
%     \hologo{OzMF},
%     \hologo{OzMP},
%     \hologo{OzTeX},
%     \hologo{OzTtH},
%     \hologo{PCTeX},
%     \hologo{PiC},
%     \hologo{PiCTeX},
%     \hologo{METAFONT},
%     \hologo{MetaFun},
%     \hologo{METAPOST},
%     \hologo{MetaPost},
%     \hologo{SLiTeX} (\xoption{lift}, \xoption{narrow}, \xoption{simple}),
%     \hologo{SliTeX} (\xoption{narrow}, \xoption{simple}, \xoption{lift}),
%     \hologo{teTeX}.
%   \item
%     Fixes:
%     \hologo{iniTeX},
%     \hologo{pdfLaTeX},
%     \hologo{pdfTeX},
%     \hologo{virTeX}.
%   \item
%     \cs{hologoFontSetup} and \cs{hologoLogoFontSetup} added.
%   \item
%     \cs{hologoVariant} and \cs{HologoVariant} added.
%   \end{Version}
%   \begin{Version}{2011/11/22 v1.7}
%   \item
%     New logos:
%     \hologo{BibTeX8},
%     \hologo{LaTeXML},
%     \hologo{SageTeX},
%     \hologo{TeX4ht},
%     \hologo{TTH}.
%   \item
%     \hologo{Xe} and friends: Driver stuff fixed.
%   \item
%     \hologo{Xe} and friends: Support for italic added.
%   \item
%     \hologo{Xe} and friends: Package support for \xpackage{pgf}
%     and \xpackage{pstricks} added.
%   \end{Version}
%   \begin{Version}{2011/11/29 v1.8}
%   \item
%     New logos:
%     \hologo{HanTheThanh}.
%   \end{Version}
%   \begin{Version}{2011/12/21 v1.9}
%   \item
%     Patch for package \xpackage{ifxetex} added for the case that
%     \cs{newif} is undefined in \hologo{iniTeX}.
%   \item
%     Some fixes for \hologo{iniTeX}.
%   \end{Version}
%   \begin{Version}{2012/04/26 v1.10}
%   \item
%     Fix in bookmark version of logo ``\hologo{HanTheThanh}''.
%   \end{Version}
%   \begin{Version}{2016/05/12 v1.11}
%   \item
%     Update HOLOGO@IfCharExists (previously in texlive)
%   \item define pdfliteral in current luatex.
%   \end{Version}
% \end{History}
%
% \PrintIndex
%
% \Finale
\endinput
|
% \end{quote}
% Do not forget to quote the argument according to the demands
% of your shell.
%
% \paragraph{Generating the documentation.}
% You can use both the \xfile{.dtx} or the \xfile{.drv} to generate
% the documentation. The process can be configured by the
% configuration file \xfile{ltxdoc.cfg}. For instance, put this
% line into this file, if you want to have A4 as paper format:
% \begin{quote}
%   \verb|\PassOptionsToClass{a4paper}{article}|
% \end{quote}
% An example follows how to generate the
% documentation with pdf\LaTeX:
% \begin{quote}
%\begin{verbatim}
%pdflatex hologo.dtx
%makeindex -s gind.ist hologo.idx
%pdflatex hologo.dtx
%makeindex -s gind.ist hologo.idx
%pdflatex hologo.dtx
%\end{verbatim}
% \end{quote}
%
% \section{Catalogue}
%
% The following XML file can be used as source for the
% \href{http://mirror.ctan.org/help/Catalogue/catalogue.html}{\TeX\ Catalogue}.
% The elements \texttt{caption} and \texttt{description} are imported
% from the original XML file from the Catalogue.
% The name of the XML file in the Catalogue is \xfile{hologo.xml}.
%    \begin{macrocode}
%<*catalogue>
<?xml version='1.0' encoding='us-ascii'?>
<!DOCTYPE entry SYSTEM 'catalogue.dtd'>
<entry datestamp='$Date$' modifier='$Author$' id='hologo'>
  <name>hologo</name>
  <caption>A collection of logos with bookmark support.</caption>
  <authorref id='auth:oberdiek'/>
  <copyright owner='Heiko Oberdiek' year='2010-2012'/>
  <license type='lppl1.3'/>
  <version number='1.10'/>
  <description>
    The package defines a single command <tt>\hologo</tt>, whose
    argument is the usual case-confused ASCII version of the logo.
    The command is bookmark-enabled, so that every logo becomes
    available in bookmarks without further work.
    <p/>
    The package is part of the <xref refid='oberdiek'>oberdiek</xref>
    bundle.
  </description>
  <documentation details='Package documentation'
      href='ctan:/macros/latex/contrib/oberdiek/hologo.pdf'/>
  <ctan file='true' path='/macros/latex/contrib/oberdiek/hologo.dtx'/>
  <miktex location='oberdiek'/>
  <texlive location='oberdiek'/>
  <install path='/macros/latex/contrib/oberdiek/oberdiek.tds.zip'/>
</entry>
%</catalogue>
%    \end{macrocode}
%
% \begin{thebibliography}{9}
% \raggedright
%
% \bibitem{btxdoc}
% Oren Patashnik,
% \textit{\hologo{BibTeX}ing},
% 1988-02-08.\\
% \CTAN{biblio/bibtex/base/}
%
% \bibitem{dtklogos}
% Gerd Neugebauer, DANTE,
% \textit{Package \xpackage{dtklogos}},
% 2011-04-25.\\
% \CTAN{usergrps/dante/dtk/dtklogos.sty}
%
% \bibitem{etexman}
% The \hologo{NTS} Team,
% \textit{The \hologo{eTeX} manual},
% 1998-02.\\
% \CTAN{systems/e-tex/v2/doc/}
%
% \bibitem{ExTeX-FAQ}
% The \hologo{ExTeX} group,
% \textit{\hologo{ExTeX}: FAQ -- How is \hologo{ExTeX} typeset?},
% 2007-04-14.\\
% \url{http://www.extex.org/documentation/faq.html}
%
% \bibitem{LyX}
% %@MISC{ LyX,
% %  title = {{LyX 2.0.0 -- The Document Processor [Computer software and manual]}},
% %  author = {{The LyX Team}},
% %  howpublished = {Internet: http://www.lyx.org},
% %  year = {2011-05-08},
% %  note = {Retrieved May 10, 2011, from http://www.lyx.org},
% %  url = {http://www.lyx.org/}
% %}
% The \hologo{LyX} Team,
% \textit{\hologo{LyX} -- The Document Processor},
% 2011-05-08.\\
% \url{http://www.lyx.org/}
%
% \bibitem{OzTeX}
% Andrew Trevorrow,
% \hologo{OzTeX} FAQ: What is the correct way to typeset ``\hologo{OzTeX}''?,
% 2011-09-15 (visited).
% \url{http://www.trevorrow.com/oztex/ozfaq.html#oztex-logo}
%
% \bibitem{PiCTeX}
% Michael Wichura,
% \textit{The \hologo{PiCTeX} macro package},
% 1987-09-21.
% \CTAN{graphics/pictex/}
%
% \bibitem{scrlogo}
% Markus Kohm,
% \textit{\hologo{KOMAScript} Datei \xfile{scrlogo.dtx}},
% 2009-01-30.\\
% \CTAN{install/macros/latex/contrib/komascript.tds.zip}
%
% \end{thebibliography}
%
% \begin{History}
%   \begin{Version}{2010/04/08 v1.0}
%   \item
%     The first version.
%   \end{Version}
%   \begin{Version}{2010/04/16 v1.1}
%   \item
%     \cs{Hologo} added for support of logos at start of a sentence.
%   \item
%     \cs{hologoSetup} and \cs{hologoLogoSetup} added.
%   \item
%     Options \xoption{break}, \xoption{hyphenbreak}, \xoption{spacebreak}
%     added.
%   \item
%     Variant support added by option \xoption{variant}.
%   \end{Version}
%   \begin{Version}{2010/04/24 v1.2}
%   \item
%     \hologo{LaTeX3} added.
%   \item
%     \hologo{VTeX} added.
%   \end{Version}
%   \begin{Version}{2010/11/21 v1.3}
%   \item
%     \hologo{iniTeX}, \hologo{virTeX} added.
%   \end{Version}
%   \begin{Version}{2011/03/25 v1.4}
%   \item
%     \hologo{ConTeXt} with variants added.
%   \item
%     Option \xoption{discretionarybreak} added as refinement for
%     option \xoption{break}.
%   \end{Version}
%   \begin{Version}{2011/04/21 v1.5}
%   \item
%     Wrong TDS directory for test files fixed.
%   \end{Version}
%   \begin{Version}{2011/10/01 v1.6}
%   \item
%     Support for package \xpackage{tex4ht} added.
%   \item
%     Support for \cs{csname} added if \cs{ifincsname} is available.
%   \item
%     New logos:
%     \hologo{(La)TeX},
%     \hologo{biber},
%     \hologo{BibTeX} (\xoption{sc}, \xoption{sf}),
%     \hologo{emTeX},
%     \hologo{ExTeX},
%     \hologo{KOMAScript},
%     \hologo{La},
%     \hologo{LyX},
%     \hologo{MiKTeX},
%     \hologo{NTS},
%     \hologo{OzMF},
%     \hologo{OzMP},
%     \hologo{OzTeX},
%     \hologo{OzTtH},
%     \hologo{PCTeX},
%     \hologo{PiC},
%     \hologo{PiCTeX},
%     \hologo{METAFONT},
%     \hologo{MetaFun},
%     \hologo{METAPOST},
%     \hologo{MetaPost},
%     \hologo{SLiTeX} (\xoption{lift}, \xoption{narrow}, \xoption{simple}),
%     \hologo{SliTeX} (\xoption{narrow}, \xoption{simple}, \xoption{lift}),
%     \hologo{teTeX}.
%   \item
%     Fixes:
%     \hologo{iniTeX},
%     \hologo{pdfLaTeX},
%     \hologo{pdfTeX},
%     \hologo{virTeX}.
%   \item
%     \cs{hologoFontSetup} and \cs{hologoLogoFontSetup} added.
%   \item
%     \cs{hologoVariant} and \cs{HologoVariant} added.
%   \end{Version}
%   \begin{Version}{2011/11/22 v1.7}
%   \item
%     New logos:
%     \hologo{BibTeX8},
%     \hologo{LaTeXML},
%     \hologo{SageTeX},
%     \hologo{TeX4ht},
%     \hologo{TTH}.
%   \item
%     \hologo{Xe} and friends: Driver stuff fixed.
%   \item
%     \hologo{Xe} and friends: Support for italic added.
%   \item
%     \hologo{Xe} and friends: Package support for \xpackage{pgf}
%     and \xpackage{pstricks} added.
%   \end{Version}
%   \begin{Version}{2011/11/29 v1.8}
%   \item
%     New logos:
%     \hologo{HanTheThanh}.
%   \end{Version}
%   \begin{Version}{2011/12/21 v1.9}
%   \item
%     Patch for package \xpackage{ifxetex} added for the case that
%     \cs{newif} is undefined in \hologo{iniTeX}.
%   \item
%     Some fixes for \hologo{iniTeX}.
%   \end{Version}
%   \begin{Version}{2012/04/26 v1.10}
%   \item
%     Fix in bookmark version of logo ``\hologo{HanTheThanh}''.
%   \end{Version}
%   \begin{Version}{2016/05/12 v1.11}
%   \item
%     Update HOLOGO@IfCharExists (previously in texlive)
%   \item define pdfliteral in current luatex.
%   \end{Version}
% \end{History}
%
% \PrintIndex
%
% \Finale
\endinput
